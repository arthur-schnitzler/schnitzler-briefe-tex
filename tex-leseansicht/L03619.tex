%% latex-leseansicht-vorspann.tex
%% Vorspann für die Leseansicht.
%% Lädt die gemeinsame Datei latex-vorspann.tex mit nicht gesetztem Schalter.

\newif\ifkorrekturansicht
\korrekturansichtfalse

\input{../tex-inputs/latex-vorspann}


\section[Karl Emil Franzos an Arthur Schnitzler, {[}3. 5. 1888–11. 5. 1888?{]}]{L03619 Karl Emil Franzos an Arthur Schnitzler, [3. 5. 1888–11. 5. 1888?]}
\nopagebreak\mylabel{L03619v}
\rehead{ }\normalsize\beginnumbering\briefempfaengerindex{Schnitzler, Arthur@\textsc{Schnitzler, Arthur}!zzzFranzos, Karl Emil@\emph{von Karl Emil Franzos}!1888-05-112@{[3. 5. 1888–11. 5. 1888?]}|(be}
\toendnotes[C]{\smallbreak\pagebreak[2]}
\correspDesc{Versand  durch Karl Emil Franzos am [3. 5. 1888 – 11. 5. 1888?] in Berlin
\newline{}Erhalt  durch Arthur Schnitzler im Zeitraum [11. 5. 1888 –
                  12. 5. 1888?] in Berlin}\toendnotes[C]{\smallbreak}
\Standort{DLA, A:Schnitzler, HS.1985.1.3025.}
\physDesc{Brief, 1 Blatt, 3 Seiten, 2073 Zeichen
\newline{}Handschrift Ottilie Franzos: schwarze Tinte, deutsche Kurrent
\newline{}Handschrift Karl Emil Franzos: schwarze Tinte, deutsche Kurrent (\noindent{}zwei Einfügungen, Unterschrift und Nachschrift)
\newline{}Schnitzler: 1) mit rotem Buntstift eine Unterstreichung  2) mit Bleistift »\textsc{Franzos}«}\toendnotes[C]{\smallbreak}
\pstart
           \centering{}{\pb}\textcolor{gray}{\textbf{Redaction der »Deutſchen
                        Dichtung\orgindex{Deutsche Dichtung@Deutsche Dichtung|pw}«.}}\pend
           
\pstart
           \textcolor{gray}{\textbf{Herausgeber:}}\hfill \textcolor{gray}{\textbf{Verlag:}}\pend
           
\pstart
           \textcolor{gray}{\textbf{Karl Emil Franzos}}\pwindex{Franzos, Karl Emil 25.\,10.\,1848 Tschortkiw – 28.\,1.\,1904 Berlin@\textsc{Franzos, Karl Emil} (25.\,10.\,1848 Tschortkiw – 28.\,1.\,1904 Berlin), \emph{Schriftsteller, Journalist}|pw}\hfill \textcolor{gray}{\textbf{Adolf Bonz {\kaufmannsund} Comp.}}\orgindex{Adolf Bonz und Comp.@Adolf Bonz {\kaufmannsund}  Comp.|pw}\pend
           
\pstart
           \textcolor{gray}{\textbf{Berlin\oindex{Berlin@\textbf{Berlin}, \emph{Hauptstadt}|pw}.}}\hfill \textcolor{gray}{\textbf{Stuttgart\oindex{Stuttgart@\textbf{Stuttgart}|pw}.}}\pend
           
\pstart
           \raggedleft{}\textcolor{gray}{\textbf{Berlin\oindex{Berlin@\textbf{Berlin}, \emph{Hauptstadt}|pw},}} den 3. Mai \textcolor{gray}{\textbf{188}}8.\pend
           
\pstart
           \raggedleft{}\textcolor{gray}{\textbf{W. Kaiserin Auguſtaſtraße 71\oindex{Kaiserin-Augusta-Straße 71@\textbf{Kaiserin-Augusta-Straße 71}, \emph{Wohngebäude}|pw}.}}\pend
           
\pstart\center{}Geehrter Herr Doctor!\pend\vspace{0.5em}
\pstart
           Ein an{ }ſich nicht gerade erfreulicher Umſtand, ein Unwohlſein nämlich, welches mich
               für einige Tage an’s Bett bannte und mir eine unfreiwillige Muße auferlegte, hat mir
               andrerſeits ermöglicht, Ihrem Wunſche, Ihnen meine Anſicht über Ihre beiden \label{K_L03619-1v}\edtext{Novellen\pwindex{Schnitzler, Arthur 15.\,5.\,1862 Wien – 21.\,10.\,1931 ebd.@\textsc{Schnitzler, Arthur} (15.\,5.\,1862 Wien – 21.\,10.\,1931 ebd.), \emph{Schriftsteller, Mediziner}!Amerika@\strich\emph{Amerika}|pwv}\pwindex{Schnitzler, Arthur 15.\,5.\,1862 Wien – 21.\,10.\,1931 ebd.@\textsc{Schnitzler, Arthur} (15.\,5.\,1862 Wien – 21.\,10.\,1931 ebd.), \emph{Schriftsteller, Mediziner}!Mein Freund Ypsilon. Aus den Papieren eines Arztes@\strich\emph{Mein Freund Ypsilon. Aus den Papieren eines Arztes}|pwv}\pwindex{Schnitzler, Arthur 15.\,5.\,1862 Wien – 21.\,10.\,1931 ebd.@\textsc{Schnitzler, Arthur} (15.\,5.\,1862 Wien – 21.\,10.\,1931 ebd.), \emph{Schriftsteller, Mediziner}!Erbschaft@\strich\emph{Erbschaft}|pwv}}{\lemma{\textnormal{\emph{Novellen}}}\Cendnote{\textnormal{Vgl. XXXX Auszeichnungsfehler: Dokument L03618 nicht gefunden.
               }}}\label{K_L03619-1} zu{ }ſagen,{ }ſchon jetzt entſprechen zu können, mehr aber als eben eine{ }ſubjektive Anſchauung beanſpruche ich gewiß nicht zu bieten. Beide Arbeiten\pwindex{Schnitzler, Arthur 15.\,5.\,1862 Wien – 21.\,10.\,1931 ebd.@\textsc{Schnitzler, Arthur} (15.\,5.\,1862 Wien – 21.\,10.\,1931 ebd.), \emph{Schriftsteller, Mediziner}!Amerika@\strich\emph{Amerika}|pwv}\pwindex{Schnitzler, Arthur 15.\,5.\,1862 Wien – 21.\,10.\,1931 ebd.@\textsc{Schnitzler, Arthur} (15.\,5.\,1862 Wien – 21.\,10.\,1931 ebd.), \emph{Schriftsteller, Mediziner}!Mein Freund Ypsilon. Aus den Papieren eines Arztes@\strich\emph{Mein Freund Ypsilon. Aus den Papieren eines Arztes}|pwv}\pwindex{Schnitzler, Arthur 15.\,5.\,1862 Wien – 21.\,10.\,1931 ebd.@\textsc{Schnitzler, Arthur} (15.\,5.\,1862 Wien – 21.\,10.\,1931 ebd.), \emph{Schriftsteller, Mediziner}!Erbschaft@\strich\emph{Erbschaft}|pwv} waren mir
               insbeſondere ihrer Entſtehung \introOben{}{[}hs.:{]} nach\introOben{} pſychologiſch intereſſant,{ }ſie{ }ſind{ }ſichtlich die Erzeugniſſe eines jungen Arztes, welcher den realen Thatſachen{ }ſeines
               Berufs dadurch eine Art idealiſirenden Gegengewichts zu geben verſucht. {\pb}Daraus erklärt{ }ſich das eigenthümliche
               Gegenüberſtehen der beiden Momente, welche{ }ſich in den Novellen\pwindex{Schnitzler, Arthur 15.\,5.\,1862 Wien – 21.\,10.\,1931 ebd.@\textsc{Schnitzler, Arthur} (15.\,5.\,1862 Wien – 21.\,10.\,1931 ebd.), \emph{Schriftsteller, Mediziner}!Amerika@\strich\emph{Amerika}|pwv}\pwindex{Schnitzler, Arthur 15.\,5.\,1862 Wien – 21.\,10.\,1931 ebd.@\textsc{Schnitzler, Arthur} (15.\,5.\,1862 Wien – 21.\,10.\,1931 ebd.), \emph{Schriftsteller, Mediziner}!Mein Freund Ypsilon. Aus den Papieren eines Arztes@\strich\emph{Mein Freund Ypsilon. Aus den Papieren eines Arztes}|pwv}\pwindex{Schnitzler, Arthur 15.\,5.\,1862 Wien – 21.\,10.\,1931 ebd.@\textsc{Schnitzler, Arthur} (15.\,5.\,1862 Wien – 21.\,10.\,1931 ebd.), \emph{Schriftsteller, Mediziner}!Erbschaft@\strich\emph{Erbschaft}|pwv} gleich{ }ſcharf
               vertreten finden, der romantiſchen Erfindung und der realiſtischen Wahl des
               Grundproblems, welches ja in beiden ein rein pathologisches iſt. Es iſt aber eben
               auch nur ein Nebeneinanderſtehen und keine harmoniſche Miſchung, was wohl darin{ }ſeine
               Erklärung findet, daß beide Elemente in ihrer extremſten Ausprägung hier vertreten
               erſcheinen. Einerſeits wird die Romantik in beiden Novellen\pwindex{Schnitzler, Arthur 15.\,5.\,1862 Wien – 21.\,10.\,1931 ebd.@\textsc{Schnitzler, Arthur} (15.\,5.\,1862 Wien – 21.\,10.\,1931 ebd.), \emph{Schriftsteller, Mediziner}!Amerika@\strich\emph{Amerika}|pwv}\pwindex{Schnitzler, Arthur 15.\,5.\,1862 Wien – 21.\,10.\,1931 ebd.@\textsc{Schnitzler, Arthur} (15.\,5.\,1862 Wien – 21.\,10.\,1931 ebd.), \emph{Schriftsteller, Mediziner}!Mein Freund Ypsilon. Aus den Papieren eines Arztes@\strich\emph{Mein Freund Ypsilon. Aus den Papieren eines Arztes}|pwv}\pwindex{Schnitzler, Arthur 15.\,5.\,1862 Wien – 21.\,10.\,1931 ebd.@\textsc{Schnitzler, Arthur} (15.\,5.\,1862 Wien – 21.\,10.\,1931 ebd.), \emph{Schriftsteller, Mediziner}!Erbschaft@\strich\emph{Erbschaft}|pwv} zur
               Hyperromantik \introOben{}{[}hs.:{]} getrieben\introOben{}, andrerſeits wird das pathologiſche
               Problem{ }ſehr hart und{ }ſtreng betont. Dies iſt meines bescheidenen Ermeſſens jene
               Klippe, welche Sie künftig zu umſchiffen haben werden, denn obwohl beide Novellen\pwindex{Schnitzler, Arthur 15.\,5.\,1862 Wien – 21.\,10.\,1931 ebd.@\textsc{Schnitzler, Arthur} (15.\,5.\,1862 Wien – 21.\,10.\,1931 ebd.), \emph{Schriftsteller, Mediziner}!Amerika@\strich\emph{Amerika}|pwv}\pwindex{Schnitzler, Arthur 15.\,5.\,1862 Wien – 21.\,10.\,1931 ebd.@\textsc{Schnitzler, Arthur} (15.\,5.\,1862 Wien – 21.\,10.\,1931 ebd.), \emph{Schriftsteller, Mediziner}!Mein Freund Ypsilon. Aus den Papieren eines Arztes@\strich\emph{Mein Freund Ypsilon. Aus den Papieren eines Arztes}|pwv}\pwindex{Schnitzler, Arthur 15.\,5.\,1862 Wien – 21.\,10.\,1931 ebd.@\textsc{Schnitzler, Arthur} (15.\,5.\,1862 Wien – 21.\,10.\,1931 ebd.), \emph{Schriftsteller, Mediziner}!Erbschaft@\strich\emph{Erbschaft}|pwv}
               meines Erachtens nicht{ }ſo druckreif{ }ſind, als daß ich einem ernſthaft{ }ſtrebenden
               Manne damit vor die Öffentlichkeit zu treten anrathen könnte,{ }ſo wäre es doch
               zunächſt für Sie und {\pb}wenn Sie die Arbeit ernſthaft
               anfaſſen, wohl nicht für Sie allein Schade, wenn Sie es dabei bewenden laſſen
               wollten.\pend
           
\pstart
           Mit beſten Empfehlungen{\\[\baselineskip]} Ihr ergebenſter {\\[\baselineskip]}\spacefill\mbox{{[}hs. Franzos:{]} Franzos}\pend
           \leftskip=0em{}
\pstart
           \noindent{}{[}hs. Franzos:{]} Herrn \textsc{Dr. A.
                  Schnitzler}.\pend
           \selectlanguage{ngerman}\vspace{1em}{\vspace{1\baselineskip}}
\pstart
           {[}hs. Franzos:{]} Geehrter Herr Dr! Der vorſtehende Brief iſt leider
               durch ein \label{K_L03619-3v}\edtext{Überſehen meiner Gattin\pwindex{Franzos, Ottilie 24.\,9.\,1856 Wien – 5.\,3.\,1932 ebd.@\textsc{Franzos, Ottilie} (24.\,9.\,1856 Wien – 5.\,3.\,1932 ebd.), \emph{Schriftstellerin}|pwv}}{\lemma{\textnormal{\emph{Übersehen meiner Gattin}}}\Cendnote{\textnormal{Die Involvierung von Ottilie Franzos\pwindex{Franzos, Ottilie 24.\,9.\,1856 Wien – 5.\,3.\,1932 ebd.@\textsc{Franzos, Ottilie} (24.\,9.\,1856 Wien – 5.\,3.\,1932 ebd.), \emph{Schriftstellerin}|pwk} in der Begründung lässt sich als Hinweis
                  lesen, dass sie den vorliegenden Brief auch für ihren Mann geschrieben hat.}}}\label{K_L03619-3}
               bis \label{K_L03619-2v}\edtext{heute}{\lemma{\textnormal{\emph{heute}}}\Cendnote{\textnormal{Die Nachschrift ist undatiert und folglich lässt sich nicht
                  mit Sicherheit feststellen, ob Schnitzler
                  das Korrespondenzstück noch vor seiner (vorgezogenen) Abreise aus Berlin\oindex{Berlin@\textbf{Berlin}, \emph{Hauptstadt}|pwk} am 12. 5. 1888 erhalten hat – oder es ihm nach Wien\oindex{Wien@\textbf{Wien}, \emph{Verwaltungsgebiet}|pwk} nachgesandt wurde. Es ist vorstellbar, dass
                     Franzos\pwindex{Franzos, Karl Emil 25.\,10.\,1848 Tschortkiw – 28.\,1.\,1904 Berlin@\textsc{Franzos, Karl Emil} (25.\,10.\,1848 Tschortkiw – 28.\,1.\,1904 Berlin), \emph{Schriftsteller, Journalist}|pwk} selbst bemerkte, dass seine
                  Antwort liegen geblieben war. Naheliegend ist aber, dass Schnitzlers Brief vom XXXX Auszeichnungsfehler: Dokument L03616 nicht gefunden{ }Franzos\pwindex{Franzos, Karl Emil 25.\,10.\,1848 Tschortkiw – 28.\,1.\,1904 Berlin@\textsc{Franzos, Karl Emil} (25.\,10.\,1848 Tschortkiw – 28.\,1.\,1904 Berlin), \emph{Schriftsteller, Journalist}|pwk} an sein nicht abgesandtes Schreiben
                  erinnerte und er die Nachschrift verfasste und schnell spedierte, um sie Schnitzler noch vor der Abreise zukommen zu
                  lassen.}}}\label{K_L03619-2} unbeſtellt geblieben. Ich{ }ſende Ihnen denſelben nun und unſere
               beſten Abſchiedsgrüße dazu. Vergeſſen Sie uns nicht, we{\geminationn}
               Sie Ihr Weg wieder hierher führt und{ }ſagen Sie Ihrem Herrn Vater\pwindex{Schnitzler, Johann 10.\,4.\,1835 Nagykanizsa – 2.\,5.\,1893 Wien@\textsc{Schnitzler, Johann} (10.\,4.\,1835 Nagykanizsa – 2.\,5.\,1893 Wien), \emph{Laryngologe}|pwv} unſere beſten Empfehlungen. Herzlich grüßend\pend
           \pstart Ihr \spacefill\mbox{Fr.}\pend{}\selectlanguage{ngerman}\endnumbering\briefempfaengerindex{Schnitzler, Arthur@\textsc{Schnitzler, Arthur}!zzzFranzos, Karl Emil@\emph{von Karl Emil Franzos}!1888-05-112@{[3. 5. 1888–11. 5. 1888?]}|)be}\mylabel{L03619h}  \newcommand{\dateiname}{L03619}\newcommand{\titel}{Karl Emil Franzos an Arthur Schnitzler, [3. 5. 1888 – 11. 5. 1888?]}\newcommand{\editorInnen}{Selma Jahnke und Martin Anton Müller}%% latex-leseansicht-abspann.tex
%% Abspann für die Leseansicht.
%% Der Schalter \ifkorrekturansicht ist bereits durch den Vorspann gesetzt.

%% latex-abspann.tex
%% Gemeinsamer Abspann für Korrekturansicht und Leseansicht.
%% Setzt den Schalter \ifkorrekturansicht voraus (gesetzt in den
%% einbindenden Dateien latex-korrekturansicht-abspann.tex bzw.
%% latex-leseansicht-abspann.tex).
%% ---------------------------------------------------------------

\normalsize

% Das esempio-Environment wird nur in der Leseansicht benötigt
\ifkorrekturansicht\else
\newenvironment{esempio}[3]%
{
    \vspace{1.5ex}
    \rlap{\underline{#1}}
    \par
    \setlength{\parindent}{0cm}
    \nopagebreak
    \leftskip=#2cm
    \rightskip=#3cm
}
{
    \par
}
\fi

\doendnotes{C}
\bigskip
\vfill

\clearpage

\footnotesize

\ifkorrekturansicht
  \lohead{\textsc{register}}
\fi

% theindex-Environment neu definieren ohne reledmac
\makeatletter
\renewenvironment{theindex}{%
  \ifkorrekturansicht
    \section*{\indexname}%
  \else
    \subsubsection*{Index der erwähnten Entitäten}%
  \fi
  \setlength{\parindent}{0pt}%
  \setlength{\parskip}{0pt plus 0.3pt}%
  \let\item\@idxitem
}{%
  \ifkorrekturansicht\clearpage\fi
}
\makeatother

\IfFileExists{\jobname-pw.ind}{\input{\jobname-pw.ind}}{}

% Quellenangabe nur in der Leseansicht
\ifkorrekturansicht\else
% Fallback-Definitionen, falls die .tex-Datei \titel etc. nicht gesetzt hat
\providecommand{\titel}{}
\providecommand{\editorInnen}{}
\providecommand{\dateiname}{\jobname}

\vspace{3cm}

\vfill

\footnotesize
\textsc{Quelle}: \titel. Herausgegeben von {\editorInnen}. In: \emph{Arthur Schnitzler: Briefwechsel mit Autorinnen und Autoren}.
 Digitale Edition, https://schnitzler-briefe.acdh.oeaw.ac.at/{\dateiname}.html (Stand \today)
\fi

\end{document}


