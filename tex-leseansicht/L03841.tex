%% latex-leseansicht-vorspann.tex
%% Vorspann für die Leseansicht.
%% Lädt die gemeinsame Datei latex-vorspann.tex mit nicht gesetztem Schalter.

\newif\ifkorrekturansicht
\korrekturansichtfalse

\input{../tex-inputs/latex-vorspann}


\section[Theodor Herzl an Arthur Schnitzler, 29. 12. 1894]{L03841 Theodor Herzl an Arthur Schnitzler, 29. 12. 1894}
\nopagebreak\mylabel{L03841v}
\rehead{ }\normalsize\beginnumbering\briefempfaengerindex{Schnitzler, Arthur@\textsc{Schnitzler, Arthur}!zzzHerzl, Theodor@\emph{von Theodor Herzl}!1894-12-291@{29. 12. 1894}|(be}
\toendnotes[C]{\smallbreak\pagebreak[2]}
\correspDesc{Versand  durch Theodor Herzl am 29. 12. 1894 in Paris
\newline{}Erhalt  durch Arthur Schnitzler im Zeitraum [30. 12. 1894 – 3. 1. 1895?] in Wien}\toendnotes[C]{\smallbreak}
\Standort{CUL, Schnitzler, B 39.}
\physDesc{Brief, 1 Blatt, 1 Seite, 573 Zeichen
\newline{}Handschrift: schwarze Tinte, lateinische Kurrent
\newline{}Ordnung: mit Bleistift von unbekannter Hand nummeriert: »20« }
\buchAbdrucke{\weitereDrucke{Theodor Herzl: \emph{Briefe und
                        autobiographische Notizen 1866–1895}. Bearbeitet von Johannes Wachten in Zusammenarbeit mit Chaya Harel, Daisy Tycho und Manfred Winkler. Berlin, Frankfurt am Main, Wien: \emph{Propyläen} 1983, S. 565 (Briefe und Tagebücher. Herausgegeben von Alex Bein, Hermann Greive, Moshe Schaerf, Julius H. Schoeps und Johannes Wachten, 1).} }\toendnotes[C]{\smallbreak}
\pstart
           {\pb}\textcolor{gray}{\textbf{NOUVELLE PRESSE LIBRE}}\orgindex{Neue Freie Presse@Neue Freie Presse|pw}\hfill \textcolor{gray}{\textbf{8, RUE DE MONCEAU }}\oindex{8, rue de Monceau@\textbf{8, rue de Monceau}, \emph{Wohngebäude}|pw}\pend
           
\pstart
           \textcolor{gray}{\textbf{D\textsuperscript{r}{ }TH. HERZL}}\hfill 29. XII. 94\pend
           
\pstart{}Mein lieber Freund!\pend\vspace{0.5em}
\pstart
           Hier \label{K_L03841-1v}\edtext{III.}{\lemma{\textnormal{\emph{III.}}}\Cendnote{\textnormal{Das Manuskript des dritten Aktes des Schauspiels \emph{Das neue Ghetto}\pwindex{Herzl, Theodor 2.\,5.\,1860 Budapest – 3.\,7.\,1904 Edlach@\textsc{Herzl, Theodor} (2.\,5.\,1860 Budapest – 3.\,7.\,1904 Edlach), \emph{Schriftsteller, Journalist}!neue Ghetto. Schauspiel in vier Acten@\strich\emph{Das neue Ghetto. Schauspiel in vier Acten}|pwk}, das als Beilage den Brief
                  begleitete, ist nicht überliefert.}}}\label{K_L03841-1}\pend
           
\pstart
           Morgen wahrscheinlich IV und Titelblatt mit Personenverzeichniss,
               vielleicht auch schon der Begleitbrief.\pend
           
\pstart
           Dann werde ich auch Ihren lieben \label{K_L03841-2v}\edtext{vorgestrigen Brief}{\lemma{\textnormal{\emph{vorgestrigen Brief}}}\Cendnote{\textnormal{XXXX26.12.1894}}}\label{K_L03841-2} beantworten.\pend
           
\pstart
           Herzlich Ihr aufrichtiger {\\[\baselineskip]}\spacefill\mbox{Th Herzl}\pend
           \leftskip=0em{}
\pstart
           \noindent{}Seite 9 \label{K_L03841-3v}\edtext{des heutigen Mscpts}{\lemma{\textnormal{\emph{des heutigen Mscpts}}}\Cendnote{\textnormal{dritter Akt des Schauspiels \emph{Das neue Ghetto}\pwindex{Herzl, Theodor 2.\,5.\,1860 Budapest – 3.\,7.\,1904 Edlach@\textsc{Herzl, Theodor} (2.\,5.\,1860 Budapest – 3.\,7.\,1904 Edlach), \emph{Schriftsteller, Journalist}!neue Ghetto. Schauspiel in vier Acten@\strich\emph{Das neue Ghetto. Schauspiel in vier Acten}|pwk}}}}\label{K_L03841-3} (ich fing die
                  Nummerirung von vorn an weil ich die frühere Zahl nicht mehr wusste) Seite 9 in
                  der \label{K_L03841-4v}\edtext{Erzählung des Rabbiners}{\lemma{\textnormal{\emph{Erzählung des Rabbiners}}}\Cendnote{\textnormal{Die Figur des Rabbiners
                     Dr. Friedheimer referiert in der sechsten Szene des dritten Aktes eine
                     Anekdote aus einer alten Chronik, die in der Druckfassung auf den Monat »Ab des Jahres 5143« des jüdischen Kalenders datiert wird, s. Theodor Herzl\pwindex{Herzl, Theodor 2.\,5.\,1860 Budapest – 3.\,7.\,1904 Edlach@\textsc{Herzl, Theodor} (2.\,5.\,1860 Budapest – 3.\,7.\,1904 Edlach), \emph{Schriftsteller, Journalist}|pwk}: \emph{Das neue Ghetto. Schauspiel in 4 Acten}\pwindex{Herzl, Theodor 2.\,5.\,1860 Budapest – 3.\,7.\,1904 Edlach@\textsc{Herzl, Theodor} (2.\,5.\,1860 Budapest – 3.\,7.\,1904 Edlach), \emph{Schriftsteller, Journalist}!neue Ghetto. Schauspiel in vier Acten@\strich\emph{Das neue Ghetto. Schauspiel in vier Acten}|pwk},
                        Wien: \emph{Buchdruckerei »Industrie« – Selbstverlag}{ }1903, S. 74. Das entspricht Juli oder August des Jahres 1383 nach christlicher Zeitrechnung.}}}\label{K_L03841-4}
                  ist der jüdische Name des Sommermonats (\label{K_L03841-5v}\edtext{\hebraeisch{Ab}}{\lemma{\textnormal{\emph{\hebraeisch{Ab}}}}\Cendnote{\textnormal{hebräisch \hebraeisch{אָב}, av: elfter Monat des jüdischen Kalenders}}}\label{K_L03841-5}? \label{K_L03841-6v}\edtext{\hebraeisch{Nischan}}{\lemma{\textnormal{\emph{\hebraeisch{Nischan}}}}\Cendnote{\textnormal{hebräisch \hebraeisch{ניסן}, nisan: siebter Monat des jüdischen Kalenders}}}\label{K_L03841-6}? ich weiss im Augenblick
                  nicht) und die geeignete Jahreszahl nachzutragen. Ich werde sie Ihnen
                     morgen schicken mit der Bitte sie einzuflicken\pend
           \selectlanguage{ngerman}\endnumbering\briefempfaengerindex{Schnitzler, Arthur@\textsc{Schnitzler, Arthur}!zzzHerzl, Theodor@\emph{von Theodor Herzl}!1894-12-291@{29. 12. 1894}|)be}\mylabel{L03841h}
\begin{anhang}
\end{anhang}\newcommand{\dateiname}{L03841}\newcommand{\titel}{Theodor Herzl an Arthur Schnitzler, 29. 12. 1894}\newcommand{\editorInnen}{Selma Jahnke und Martin Anton Müller}%% latex-leseansicht-abspann.tex
%% Abspann für die Leseansicht.
%% Der Schalter \ifkorrekturansicht ist bereits durch den Vorspann gesetzt.

%% latex-abspann.tex
%% Gemeinsamer Abspann für Korrekturansicht und Leseansicht.
%% Setzt den Schalter \ifkorrekturansicht voraus (gesetzt in den
%% einbindenden Dateien latex-korrekturansicht-abspann.tex bzw.
%% latex-leseansicht-abspann.tex).
%% ---------------------------------------------------------------

\normalsize

% Das esempio-Environment wird nur in der Leseansicht benötigt
\ifkorrekturansicht\else
\newenvironment{esempio}[3]%
{
    \vspace{1.5ex}
    \rlap{\underline{#1}}
    \par
    \setlength{\parindent}{0cm}
    \nopagebreak
    \leftskip=#2cm
    \rightskip=#3cm
}
{
    \par
}
\fi

\doendnotes{C}
\bigskip
\vfill

\clearpage

\footnotesize

\ifkorrekturansicht
  \lohead{\textsc{register}}
\fi

% theindex-Environment neu definieren ohne reledmac
\makeatletter
\renewenvironment{theindex}{%
  \ifkorrekturansicht
    \section*{\indexname}%
  \else
    \subsubsection*{Index der erwähnten Entitäten}%
  \fi
  \setlength{\parindent}{0pt}%
  \setlength{\parskip}{0pt plus 0.3pt}%
  \let\item\@idxitem
}{%
  \ifkorrekturansicht\clearpage\fi
}
\makeatother

\IfFileExists{\jobname-pw.ind}{\input{\jobname-pw.ind}}{}

% Quellenangabe nur in der Leseansicht
\ifkorrekturansicht\else
% Fallback-Definitionen, falls die .tex-Datei \titel etc. nicht gesetzt hat
\providecommand{\titel}{}
\providecommand{\editorInnen}{}
\providecommand{\dateiname}{\jobname}

\vspace{3cm}

\vfill

\footnotesize
\textsc{Quelle}: \titel. Herausgegeben von {\editorInnen}. In: \emph{Arthur Schnitzler: Briefwechsel mit Autorinnen und Autoren}.
 Digitale Edition, https://schnitzler-briefe.acdh.oeaw.ac.at/{\dateiname}.html (Stand \today)
\fi

\end{document}


