%% latex-leseansicht-vorspann.tex
%% Vorspann für die Leseansicht.
%% Lädt die gemeinsame Datei latex-vorspann.tex mit nicht gesetztem Schalter.

\newif\ifkorrekturansicht
\korrekturansichtfalse

\input{../tex-inputs/latex-vorspann}


         
         \renewcommand{\erwaehntePersonen}{Personen: Paul Goldmann, Wilhelm von Hartel, Robert Pattai, Olga Schnitzler}
         \renewcommand{\erwaehnteInstitutionen}{Institutionen: Bauernfeld-Preis, Reichsrat}
         \renewcommand{\erwaehnteOrte}{Orte: Berlin, Dessauer Straße, Wien}
         \renewcommand{\erwaehnteWerke}{Werke: Lebendige Stunden. Vier Einakter}
               \section[ Paul Goldmann an Arthur Schnitzler, 1. 4. {[}1903{]}]{ Paul Goldmann an Arthur Schnitzler, 1. 4. {[}1903{]}}\nopagebreak\mylabel{v}\rehead{ }\begin{ledgroupsized}[t]{13cm}\normalsize\beginnumbering \toendnotes[C]{\smallbreak\pagebreak[2]} \Standort{DLA, A:Schnitzler, HS.NZ85.1.3173.}
\physDesc{Brief, 1 Blatt, 1 Seite, 258 Zeichen
\newline{}Handschrift: blaue Tinte, deutsche Kurrent
\newline{}Schnitzler: 1) mit Bleistift das Jahr »903.« vermerkt  2) mit rotem Buntstift eine Unterstreichung}\toendnotes[C]{\smallbreak}\pstart
           \noindent{}\raggedleft{}{\pb}\textcolor{gray}{\textbf{DESSAUERSTRASSE 19\oindex{Dessauer Strasse@\textbf{Dessauer Straße}|pw}}}\pend
           \pstart
           Berlin\oindex{Berlin@\textbf{Berlin}|pw}, 1. April.\pend
           \pstart\center{}Mein lieber Freund,\pend\pstart
           Die \label{K_L03371-1v}\edtext{Interpellations{[}-{]}Beantwortung}{\lemma{\textnormal{\emph{Interpellations-Beantwortung}}}\Cendnote{\textnormal{Der antisemitische Abgeordnete Robert Pattai\pwindex{Pattai, Robert 09.08.1864 – 30.09.1920@\textsc{Pattai, Robert} (09.08.1864 – 30.09.1920), \emph{Politiker}|pwk} hatte am 18. 3. 1903 im Abgeordnetenhaus\orgindex{Reichsrat@Reichsrat|pwkv} die
                  Zuerkennung des \emph{Bauernfeld-Preis}\orgindex{Bauernfeld-Preis@Bauernfeld-Preis|pwk}es an den
                     »jüdischen Autor« Schnitzler\pwindex{Schnitzler, Arthur 15.05.1862 – 21.10.1931@\textsc{Schnitzler, Arthur} (15.05.1862 – 21.10.1931), \emph{Schriftsteller, Mediziner}|pwk} kritisiert, zumal dessen ausgezeichnetes Werk \emph{Lebendige Stunden}\pwindex{Schnitzler, Arthur 15.05.1862 – 21.10.1931@\textsc{Schnitzler, Arthur} (15.05.1862 – 21.10.1931), \emph{Schriftsteller, Mediziner}!Lebendige Stunden. Vier Einakter1901-12-23@\strich\emph{Lebendige Stunden. Vier Einakter} {[}1901-12-23{]}|pwk} von niederer Qualität sei (vgl. A. S.: \emph{»Das Zeitlose ist von kürzester Dauer«}, [Felix Salten]: Der Bauernfeld-Preis. Eine Interpellation, 19. 3. 1903). In der Sitzung des
                     Abgeordnetenhaus\orgindex{Reichsrat@Reichsrat|pwkv}es am
                     31. 3. 1903 hatte der Unterrichtsminister Wilhelm von Hartel\pwindex{Hartel, Wilhelm von 28.05.1839 – 14.01.1907@\textsc{Hartel, Wilhelm von} (28.05.1839 – 14.01.1907), \emph{Politiker, Philologe, Unterrichtsminister}|pwk} darauf geantwortet.
               }}}\label{K_L03371-1h} des Unterrichtsminiſter\pwindex{Hartel, Wilhelm von 28.05.1839 – 14.01.1907@\textsc{Hartel, Wilhelm von} (28.05.1839 – 14.01.1907), \emph{Politiker, Philologe, Unterrichtsminister}|pw}s iſt ſehr
               anſtändig und für Dich auch recht ehrenvoll. Ich habe mich darüber ſehr gefreut.\pend
           \pstart
           Warum ſchreibſt Du mir nicht?\pend
           \pstart Viele herzliche Grüße Dir und \textsc{Olga\pwindex{Schnitzler, Olga 17.01.1882 – 13.01.1970@\textsc{Schnitzler, Olga} (17.01.1882 – 13.01.1970), \emph{Schauspielerin, Sängerin}|pw}}! Dein \spacefill\mbox{Paul Goldmn}\pend{}
         
         \endnumbering\mylabel{h}\end{ledgroupsized}  \newcommand{\dateiname}{L03371}\newcommand{\titel}{Paul Goldmann an Arthur Schnitzler, 1. 4. [1903]}\newcommand{\editorInnen}{Martin Anton Müller und Laura Untner}%% latex-leseansicht-abspann.tex
%% Abspann für die Leseansicht.
%% Der Schalter \ifkorrekturansicht ist bereits durch den Vorspann gesetzt.

%% latex-abspann.tex
%% Gemeinsamer Abspann für Korrekturansicht und Leseansicht.
%% Setzt den Schalter \ifkorrekturansicht voraus (gesetzt in den
%% einbindenden Dateien latex-korrekturansicht-abspann.tex bzw.
%% latex-leseansicht-abspann.tex).
%% ---------------------------------------------------------------

\normalsize

% Das esempio-Environment wird nur in der Leseansicht benötigt
\ifkorrekturansicht\else
\newenvironment{esempio}[3]%
{
    \vspace{1.5ex}
    \rlap{\underline{#1}}
    \par
    \setlength{\parindent}{0cm}
    \nopagebreak
    \leftskip=#2cm
    \rightskip=#3cm
}
{
    \par
}
\fi

\doendnotes{C}
\bigskip
\vfill

\clearpage

\footnotesize

\ifkorrekturansicht
  \lohead{\textsc{register}}
\fi

% theindex-Environment neu definieren ohne reledmac
\makeatletter
\renewenvironment{theindex}{%
  \ifkorrekturansicht
    \section*{\indexname}%
  \else
    \subsubsection*{Index der erwähnten Entitäten}%
  \fi
  \setlength{\parindent}{0pt}%
  \setlength{\parskip}{0pt plus 0.3pt}%
  \let\item\@idxitem
}{%
  \ifkorrekturansicht\clearpage\fi
}
\makeatother

\IfFileExists{\jobname-pw.ind}{\input{\jobname-pw.ind}}{}

% Quellenangabe nur in der Leseansicht
\ifkorrekturansicht\else
% Fallback-Definitionen, falls die .tex-Datei \titel etc. nicht gesetzt hat
\providecommand{\titel}{}
\providecommand{\editorInnen}{}
\providecommand{\dateiname}{\jobname}

\vspace{3cm}

\vfill

\footnotesize
\textsc{Quelle}: \titel. Herausgegeben von {\editorInnen}. In: \emph{Arthur Schnitzler: Briefwechsel mit Autorinnen und Autoren}.
 Digitale Edition, https://schnitzler-briefe.acdh.oeaw.ac.at/{\dateiname}.html (Stand \today)
\fi

\end{document}


      