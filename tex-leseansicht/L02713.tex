%% latex-korrekturansicht-vorspann.tex
%% Vorspann für die Korrekturansicht.
%% Lädt die gemeinsame Datei latex-vorspann.tex mit gesetztem Schalter.

\newif\ifkorrekturansicht
\korrekturansichttrue

\input{../tex-inputs/latex-vorspann}


\section[Paul Goldmann an Arthur Schnitzler, 23. 8. {[}1893{]}]{L02713 Paul Goldmann an Arthur Schnitzler, 23. 8. {[}1893{]}}
\nopagebreak\mylabel{L02713v}
\rehead{ }\normalsize\beginnumbering\briefempfaengerindex{Schnitzler, Arthur@\textsc{Schnitzler, Arthur}!zzzGoldmann, Paul@\emph{von Paul Goldmann}!1893-08-231@{23. 8. {[}1893{]}}|(be}
\toendnotes[C]{\smallbreak\pagebreak[2]}\Standort{DLA, A:Schnitzler, HS.NZ85.1.3163.}
\physDesc{Brief, 1 Blatt, 4 Seiten, 1644 Zeichen
\newline{}Handschrift: schwarze Tinte, deutsche Kurrent
\newline{}Schnitzler: 1) mit Bleistift das Jahr »93« vermerkt  2) mit rotem Buntstift eine Unterstreichung}\toendnotes[C]{\smallbreak}
\pstart
           {\pb}\textcolor{gray}{\textbf{\textbf{Frankfurter Zeitung\orgindex{Frankfurter Zeitung@Frankfurter Zeitung|pw}.}}}\pend
           
\pstart
           \textcolor{gray}{\textbf{\textbf{(\begin{otherlanguage}{french}Gazette de Francfort\end{otherlanguage}\orgindex{Frankfurter Zeitung@Frankfurter Zeitung|pw}.)}}}\pend
           
\pstart
           \textcolor{gray}{\textbf{\begin{otherlanguage}{french}Directeur \end{otherlanguage}{ }\textbf{M. L. Sonnemann\pwindex{Sonnemann, Leopold 1831-10-29 – 1909-10-30@\textsc{Sonnemann, Leopold} (1831-10-29 – 1909-10-30), \emph{Journalist/Journalistin, Herausgeber/Herausgeberin}|pw}.}}}\hfill \textsc{Paris\oindex{Paris@\textbf{Paris}, \emph{P.PPLC}|pw}}, 23. August.\pend
           
\pstart
           \begin{otherlanguage}{french}\textcolor{gray}{\textbf{Journal politique, financier,}}\end{otherlanguage}\pend
           
\pstart
           \begin{otherlanguage}{french}\textcolor{gray}{\textbf{commercial et litteraire.}}\end{otherlanguage}\pend
           
\pstart
           \begin{otherlanguage}{french}\textcolor{gray}{\textbf{\textbf{Paraissant trois fois par jour}}}\end{otherlanguage}\pend
           
\pstart
           \begin{otherlanguage}{french}\textcolor{gray}{\textbf{\textbf{Bureaux à Paris\oindex{Paris@\textbf{Paris}, \emph{P.PPLC}|pw}:}}}\end{otherlanguage}\pend
           
\pstart
           \begin{otherlanguage}{french}\textcolor{gray}{\textbf{\textbf{rue Richelieu 75\oindex{rue Richelieu@\textbf{rue Richelieu}, \emph{Straße (K.STR)}|pw}.}}}\end{otherlanguage}\pend
           
\pstart\center{}Mein lieber Arthur!\pend\vspace{0.5em}
\pstart
           Ich könnte eigentlich jetzt ſchon fort. Aber eine unbezwingliche Geldverlegenheit
               hält mich noch zurück. Ich muß ſehen, irgendwo noch ein paar hundert \textsc{Frcs} aufzutreiben. Wenn mir das gelingt, will ich Montag fortgehen\textcolor{gray}{.} Aus verſchiedenen
               Gründen will und muß ich auf ein paar Tage zunächſt in die Schweiz\oindex{Schweiz@\textbf{Schweiz}, \emph{A.PCLI}|pw}. Du biſt im \label{K_L02713-1v}\edtext{\textsc{Pusterthal\oindex{Pustertal@\textbf{Pustertal}, \emph{T.VAL}|pw}}}{\lemma{\textnormal{\emph{Pusterthal}}}\Cendnote{\textnormal{Zu einem gemeinsamen Aufenthalt in der
                     Schweiz\oindex{Schweiz@\textbf{Schweiz}, \emph{A.PCLI}|pwk} kam es nicht. Schnitzler und Goldmann\pwindex{Goldmann, Paul 31.01.1865 – 25.09.1935@\textsc{Goldmann, Paul} (31.01.1865 – 25.09.1935), \emph{Schriftsteller/Schriftstellerin, Journalist/Journalistin}|pwk} sahen sich erst am 17. 9. 1893 und 18. 9. 1893 in Salzburg\oindex{Salzburg@\textbf{Salzburg}, \emph{A.ADM2}|pwk} wieder.}}}\label{K_L02713-1}, alſo nicht allzuweit davon. Könnten wir nicht
               die nächſte Woche mitſammen {\pb}in der Schweiz\oindex{Schweiz@\textbf{Schweiz}, \emph{A.PCLI}|pw} verbringen? Wir träfen uns z. B. an einem der Tage
               der nächſten Woche irgendwo da unten, und ich reiſte am Ende mit Dir nach Salzburg\oindex{Salzburg@\textbf{Salzburg}, \emph{A.ADM2}|pw} in der Richtung \textsc{Wien\oindex{Wien@\textbf{Wien}, \emph{A.ADM2}|pw}} zurück. Hältſt Du dieſen Plan für durchführbar, ſo ſei ſo gut mir \uline{telegraphiſch} eine Nachricht nach \textsc{Paris\oindex{Paris@\textbf{Paris}, \emph{P.PPLC}|pw}} zu geben (Adreſſe: \textsc{Goldmann}, \textsc{Paris, 75. Richelieu}\oindex{rue Richelieu@\textbf{rue Richelieu}, \emph{Straße (K.STR)}|pw}). Theile mir eine telegraphiſche Antwortadreſſe mit, und vielleicht wird auf
               dieſe Weiſe der kühne Plan zur Wahrheit. Ich warte jedenfalls auf Dein\strikeout{\textcolor{gray}{e}} Telegramm noch Dienſtag und Mittwoch\substVorne{}\textsuperscript{.}\substDazwischen{},\substHinten{} da ich nicht {\pb}weiß, ob Du meinen Brief
               rechtzeitig erhältſt. In einem Tage können alle Verabredungen getroffen ſein.\pend
           
\pstart
           Folgendes iſt ein Gerücht, für das ich nicht die mindeſte Bürgſchaft übernehme, da
               mein \label{K_L02713-2v}\edtext{Gewährsmann}{\lemma{\textnormal{\emph{Gewährsmann}}}\Cendnote{\textnormal{nicht identifiziert}}}\label{K_L02713-2} ebenſogut
               gelogen haben kann, um mir ein Vergnügen zu machen. Anderſeits möchte ich es Dir doch
               nicht vorenthalten: Ein von Berlin\oindex{Berlin@\textbf{Berlin}, \emph{P.PPLC}|pw}
               zurückkommender College ſagte auf meine Frage, er habe dort gehört, \label{K_L02713-3v}\edtext{\textsc{Blumenthal\pwindex{Blumenthal, Oskar 13.03.1852 – 24.04.1917@\textsc{Blumenthal, Oskar} (13.03.1852 – 24.04.1917), \emph{Schriftsteller/Schriftstellerin, Journalist/Journalistin, Theaterleiter/Theaterleiterin}|pw}} wolle das \textsc{Schnitzler}’ſche Stück\pwindex{Maerchen. Schauspiel in drei Aufzuegen@\emph{Das Märchen. Schauspiel in drei Aufzügen}|pwv} im Herbſt gleich nach dem von \label{K_L02713-4v}\edtext{\textsc{Skowronek\pwindex{Skowronnek, Richard 1862-03-12 – 1932-10-17@\textsc{Skowronnek, Richard} (1862-03-12 – 1932-10-17), \emph{Schriftsteller/Schriftstellerin, Journalist/Journalistin, Dramatiker/Dramatikerin}|pw}\pwindex{Eine Palastrevolution@\emph{Eine Palastrevolution}|pwv}}}{\lemma{\textnormal{\emph{Skowronek}}}\Cendnote{\textnormal{Richard Skowronneks\pwindex{Skowronnek, Richard 1862-03-12 – 1932-10-17@\textsc{Skowronnek, Richard} (1862-03-12 – 1932-10-17), \emph{Schriftsteller/Schriftstellerin, Journalist/Journalistin, Dramatiker/Dramatikerin}|pwk} vieraktiges Lustspiel
                     \emph{Der erste seines Stammes}\pwindex{erste seines Stammes. Lustspiel in vier Akten@\emph{Der erste seines Stammes. Lustspiel in vier Akten}|pwk} feierte am Berlin\oindex{Berlin@\textbf{Berlin}, \emph{P.PPLC}|pwk}er \emph{Lessing-Theater}\orgindex{Lessing-Theater@Lessing-Theater|pwk} am 11. 11. 1893 seine Uraufführung.}}}\label{K_L02713-4}
                  aufführen}{\lemma{\textnormal{\emph{Blumenthal … aufführen}}}\Cendnote{\textnormal{Oskar Blumenthal\pwindex{Blumenthal, Oskar 13.03.1852 – 24.04.1917@\textsc{Blumenthal, Oskar} (13.03.1852 – 24.04.1917), \emph{Schriftsteller/Schriftstellerin, Journalist/Journalistin, Theaterleiter/Theaterleiterin}|pwk}, Leiter\pwindex{Blumenthal, Oskar 13.03.1852 – 24.04.1917@\textsc{Blumenthal, Oskar} (13.03.1852 – 24.04.1917), \emph{Schriftsteller/Schriftstellerin, Journalist/Journalistin, Theaterleiter/Theaterleiterin}|pwkv} des \emph{Lessing-Theaters}\orgindex{Lessing-Theater@Lessing-Theater|pwk} in Berlin\oindex{Berlin@\textbf{Berlin}, \emph{P.PPLC}|pwk}, hatte Schnitzler am 12. 8. 1893 bereits
                  brieflich mitgeteilt, dass das Gerücht nicht wahr sei (vgl. Oscar Blumenthal an Arthur Schnitzler, 12. 8. 1893).}}}\label{K_L02713-3}. Nochmals: ohne \uline{jede}{ }{\pb}Garantie. Nur ein Möglichkeits-Spahn, um ihn mit
               Urlaubshoffnungen zu umſpinnen{\dotsfour}\pend
           
\pstart
           Wird aus der Reiſe nichts, ſo erhältſt Du nach 1. September Nachricht von mir in Wien\oindex{Wien@\textbf{Wien}, \emph{A.ADM2}|pw}.\pend
           
\pstart
           Viele treue Grüße! {\\[\baselineskip]}Dein {\\[\baselineskip]}\spacefill\mbox{Paul Goldmann.}\pend
           \leftskip=0em{}\selectlanguage{ngerman}\endnumbering\briefempfaengerindex{Schnitzler, Arthur@\textsc{Schnitzler, Arthur}!zzzGoldmann, Paul@\emph{von Paul Goldmann}!1893-08-231@{23. 8. {[}1893{]}}|)be}\mylabel{L02713h}  \normalsize

\doendnotes{C}
\bigskip
\vfill

\clearpage

\footnotesize

\lohead{\textsc{register}}

% Definiere theindex-Environment komplett neu ohne reledmac
\makeatletter
\renewenvironment{theindex}{%
  \section*{\indexname}%
  \setlength{\parindent}{0pt}%
  \setlength{\parskip}{0pt plus 0.3pt}%
  \let\item\@idxitem
}{%
  \clearpage
}
\makeatother

\IfFileExists{\jobname-pw.ind}{\input{\jobname-pw.ind}}{}

\end{document}

      