%% latex-leseansicht-vorspann.tex
%% Vorspann für die Leseansicht.
%% Lädt die gemeinsame Datei latex-vorspann.tex mit nicht gesetztem Schalter.

\newif\ifkorrekturansicht
\korrekturansichtfalse

\input{../tex-inputs/latex-vorspann}

\begin{center}
            \textcolor{red}{ENTWURF. ENTZIFFERUNG NOCH NICHT KORREKTURGELESEN}
                      \end{center}
            
               \section[Paul Goldmann an Arthur Schnitzler, 23. 8. {[}1893{]}]{ Paul Goldmann an Arthur Schnitzler, 23. 8. {[}1893{]}}\nopagebreak\mylabel{v}\rehead{ }\begin{ledgroupsized}[t]{13cm}\normalsize\beginnumbering\briefempfaengerindex{Schnitzler, Arthur@\textsc{Schnitzler, Arthur}!zzzGoldmann, Paul@\emph{von Paul Goldmann}!1893-08-231@{23. 8. {[}1893{]}}|(be} \toendnotes[C]{\smallbreak\pagebreak[2]} \Standort{DLA, A:Schnitzler, HS.NZ85.1.3163.}
\physDesc{Brief, 1 Blatt, 4 Seiten
\newline{}Handschrift: schwarze Tinte, deutsche Kurrent
\newline{}Schnitzler: 1) mit Bleistift das Jahr »93« vermerkt 2) mit rotem Buntstift eine Unterstreichung}\toendnotes[C]{\smallbreak}\pstart
           \noindent{}{\pb}\textcolor{gray}{\textbf{\textbf{Frankfurter Zeitung\orgindex{Frankfurter Zeitung@Frankfurter Zeitung|pw}.}}}\pend
           \pstart
           \textcolor{gray}{\textbf{\textbf{(\begin{otherlanguage}{french}Gazette de Francfort\end{otherlanguage}\orgindex{Frankfurter Zeitung@Frankfurter Zeitung|pw}.)}}}\pend
           \pstart
           \textcolor{gray}{\textbf{\begin{otherlanguage}{french}Directeur\pwindex{Sonnemann, Leopold 1831-10-29 – 1909-10-30@\textsc{Sonnemann, Leopold} (1831-10-29 – 1909-10-30), \emph{Journalist, Herausgeber}|pwv}\end{otherlanguage}{ }\textbf{M. L. Sonnemann\pwindex{Sonnemann, Leopold 1831-10-29 – 1909-10-30@\textsc{Sonnemann, Leopold} (1831-10-29 – 1909-10-30), \emph{Journalist, Herausgeber}|pw}.}}}\hfill \textsc{Paris\oindex{Paris@\textbf{Paris}|pw}}, 23. August.\pend
           \pstart
           \begin{otherlanguage}{french}\textcolor{gray}{\textbf{Journal\pwindex{Frankfurter Zeitung1856 – 1943@\emph{Frankfurter Zeitung}|pw} politique, financier,}}\end{otherlanguage}\pend
           \pstart
           \begin{otherlanguage}{french}\textcolor{gray}{\textbf{commercial et litteraire.}}\end{otherlanguage}\pend
           \pstart
           \begin{otherlanguage}{french}\textcolor{gray}{\textbf{\textbf{Paraissant trois fois par jour}}}\end{otherlanguage}\pend
           \pstart
           \begin{otherlanguage}{french}\textcolor{gray}{\textbf{\textbf{Bureaux à Paris\oindex{Paris@\textbf{Paris}|pw}:}}}\end{otherlanguage}\pend
           \pstart
           \begin{otherlanguage}{french}\textcolor{gray}{\textbf{\textbf{rue Richelieu 75\oindex{rue Richelieu@\textbf{rue Richelieu}|pw}.}}}\end{otherlanguage}\pend
           \pstart
           Mein lieber Arthur!\pend
           \pstart
           Ich könnte eigentlich jetzt ſchon fort. Aber eine unbezwingliche Geldverlegenheit
               hält mich noch zurück. Ich muß ſehen, irgendwo noch ein Paar hundert \textsc{Frcs} aufzutreiben. Wenn mir das gelingt, will ich Montag ſortgehen. Aus verſchiedenen Gründen will und muß
               ich auf ein Paar Tage unnächſt in die Schweiz\oindex{Schweiz@\textbf{Schweiz}|pw}.
               Du biſt im \label{K_L02713-1v}\edtext{\textsc{Pusterthal\oindex{Pustertal@\textbf{Pustertal}|pw}}}{\lemma{\textnormal{\emph{Pusterthal}}}\Cendnote{\textnormal{Zu einem gemeinsamen Aufenthalt in der
                     Schweiz\oindex{Schweiz@\textbf{Schweiz}|pwk} kam es nicht. Schnitzler\pwindex{Schnitzler, Arthur 15.05.1862 – 21.10.1931@\textsc{Schnitzler, Arthur} (15.05.1862 – 21.10.1931), \emph{Schriftsteller, Mediziner}|pwk} und Goldmann\pwindex{Goldmann, Paul 31.01.1865 – 25.09.1935@\textsc{Goldmann, Paul} (31.01.1865 – 25.09.1935), \emph{Schriftsteller, Journalist}|pwk} sahen sich erst am 17. 9. 1893 und 18. 9. 1893 in Salzburg\oindex{Salzburg@\textbf{Salzburg}|pwk} wieder.}}}\label{K_L02713-1h}, alſo nicht allzuweit davon. Könnten wir nicht
               die nächſte Woche mitſammen {\pb}in der Schweiz\oindex{Schweiz@\textbf{Schweiz}|pw} verbringen? Wir träfen uns z. B. an einem der Tage
               der nächſten Woche irgendwo da unten, und ich reiſte am Ende mit Dir nach Salzburg\oindex{Salzburg@\textbf{Salzburg}|pw} in der Richtung \textsc{Wien\oindex{Wien@\textbf{Wien}|pw}} zurück. Hältſt Du dieſen Plan für durchführbar, ſo ſei ſo gut mir \uline{telegraphisch} eine Nachricht nach \textsc{Paris\oindex{Paris@\textbf{Paris}|pw}} zu geben. (Adreſſe: \textsc{Goldmann}, \textsc{Paris}\oindex{Paris@\textbf{Paris}|pw}, 75. Richelieu\oindex{rue Richelieu@\textbf{rue Richelieu}|pw}). Theile mir eine
               telegraphiſche Antwortadreſſe mit und vielleicht wird auf dieſe Weiſe der kühne Plan
               zur Wahrheit. Ich warte jedenfalls auf Dein\strikeout{\textcolor{gray}{e}} Telegramm noch Dienſtag und Mittwoch, da ich nicht {\pb}weiß, ob Du meinen Brief rechtzeitig erhältſt. In einem Tage können alle
               Verabredungen getroffen ſein.\pend
           \pstart
           Folgendes iſt ein Gerücht, für das ich nicht die mindeſte Bürgſchaft übernehme, da
               mein \label{K_L02713-2v}\edtext{Gewährsmann}{\lemma{\textnormal{\emph{Gewährsmann}}}\Cendnote{\textnormal{nicht identifiziert}}}\label{K_L02713-2h} ebenſogut
               gelogen haben kann, um mir ein Vergnügen zu machen. Anderſeits möchte ich es Dir doch
               nicht vorenthalten: Ein von Berlin\oindex{Berlin@\textbf{Berlin}|pw}
               zurückkommender College ſagte auf meine Frage, er habe dort gehört, \label{K_L02713-3v}\edtext{\textsc{Blumenthal\pwindex{Blumenthal, Oskar 13.03.1852 – 24.04.1917@\textsc{Blumenthal, Oskar} (13.03.1852 – 24.04.1917), \emph{Schriftsteller, Journalist, Theaterleiter}|pw}} wolle das \textsc{Schnitzler}’ſche Stück\pwindex{Schnitzler, Arthur 15.05.1862 – 21.10.1931@\textsc{Schnitzler, Arthur} (15.05.1862 – 21.10.1931), \emph{Schriftsteller, Mediziner}!Maerchen. Schauspiel in drei Aufzuegen1891 – 1891@\strich\emph{Das Märchen. Schauspiel in drei Aufzügen} {[}1891 – 1891{]}|pwv} im Herbſt gleich nach dem von \label{K_L02713-13v}\edtext{\textsc{Skowronek\pwindex{Skowronnek, Richard 1862-03-12 – 1932-10-17@\textsc{Skowronnek, Richard} (1862-03-12 – 1932-10-17), \emph{Schriftsteller, Journalist, Schriftsteller}|pw}\pwindex{Skowronnek, Richard 1862-03-12 – 1932-10-17@\textsc{Skowronnek, Richard} (1862-03-12 – 1932-10-17), \emph{Schriftsteller, Journalist, Schriftsteller}!Eine Palastrevolution1893@\strich\emph{Eine Palastrevolution} {[}1893{]}|pwv}}}{\lemma{\textnormal{\emph{Skowronek}}}\Cendnote{\textnormal{Richard Skowronnek\pwindex{Skowronnek, Richard 1862-03-12 – 1932-10-17@\textsc{Skowronnek, Richard} (1862-03-12 – 1932-10-17), \emph{Schriftsteller, Journalist, Schriftsteller}|pwk}s Drama\pwindex{Skowronnek, Richard 1862-03-12 – 1932-10-17@\textsc{Skowronnek, Richard} (1862-03-12 – 1932-10-17), \emph{Schriftsteller, Journalist, Schriftsteller}!Eine Palastrevolution1893@\strich\emph{Eine Palastrevolution} {[}1893{]}|pwkv}{ }\emph{Eine Palastrevolution}\pwindex{Skowronnek, Richard 1862-03-12 – 1932-10-17@\textsc{Skowronnek, Richard} (1862-03-12 – 1932-10-17), \emph{Schriftsteller, Journalist, Schriftsteller}!Eine Palastrevolution1893@\strich\emph{Eine Palastrevolution} {[}1893{]}|pwk} feierte am Berlin\oindex{Berlin@\textbf{Berlin}|pwk}er \emph{Lessing-Theater}\orgindex{Lessing-Theater@Lessing-Theater|pwk} am 2. 2. 1893 Premiere. In
                     Wien\oindex{Paris@\textbf{Paris}|pwk} wurde das Stück\textcolor{red}{\textsuperscript{XXXX indx}} – in Anwesenheit Schnitzler\pwindex{Schnitzler, Arthur 15.05.1862 – 21.10.1931@\textsc{Schnitzler, Arthur} (15.05.1862 – 21.10.1931), \emph{Schriftsteller, Mediziner}|pwk}s – erstmals am 14. 10. 1893 gespielt.}}}\label{K_L02713-13h} aufführen}{\lemma{\textnormal{\emph{Blumenthal … aufführen}}}\Cendnote{\textnormal{Oskar Blumenthal\pwindex{Blumenthal, Oskar 13.03.1852 – 24.04.1917@\textsc{Blumenthal, Oskar} (13.03.1852 – 24.04.1917), \emph{Schriftsteller, Journalist, Theaterleiter}|pwk}, Leiter\pwindex{Blumenthal, Oskar 13.03.1852 – 24.04.1917@\textsc{Blumenthal, Oskar} (13.03.1852 – 24.04.1917), \emph{Schriftsteller, Journalist, Theaterleiter}|pwkv} des \emph{Lessing-Theater}\orgindex{Lessing-Theater@Lessing-Theater|pwk}s in Berlin\oindex{Berlin@\textbf{Berlin}|pwk}, revidierte dieses Gerücht in einem Brief an Schnitzler\pwindex{Schnitzler, Arthur 15.05.1862 – 21.10.1931@\textsc{Schnitzler, Arthur} (15.05.1862 – 21.10.1931), \emph{Schriftsteller, Mediziner}|pwk} vom 12. 8. 1893. Blumenthal\pwindex{Blumenthal, Oskar 13.03.1852 – 24.04.1917@\textsc{Blumenthal, Oskar} (13.03.1852 – 24.04.1917), \emph{Schriftsteller, Journalist, Theaterleiter}|pwk} wollte
                  zwar das \emph{Märchen}\pwindex{Schnitzler, Arthur 15.05.1862 – 21.10.1931@\textsc{Schnitzler, Arthur} (15.05.1862 – 21.10.1931), \emph{Schriftsteller, Mediziner}!Maerchen. Schauspiel in drei Aufzuegen1891 – 1891@\strich\emph{Das Märchen. Schauspiel in drei Aufzügen} {[}1891 – 1891{]}|pwk} aufgrund dessen Thematik,
                  die »ermüdete Hörer treffen würde«, nicht in naher Zukunft spielen,
                  habe das Stück\pwindex{Schnitzler, Arthur 15.05.1862 – 21.10.1931@\textsc{Schnitzler, Arthur} (15.05.1862 – 21.10.1931), \emph{Schriftsteller, Mediziner}!Maerchen. Schauspiel in drei Aufzuegen1891 – 1891@\strich\emph{Das Märchen. Schauspiel in drei Aufzügen} {[}1891 – 1891{]}|pwkv} aber
                  weiterhin im Kopf behalten. Über die Annahme des Stück\pwindex{Schnitzler, Arthur 15.05.1862 – 21.10.1931@\textsc{Schnitzler, Arthur} (15.05.1862 – 21.10.1931), \emph{Schriftsteller, Mediziner}!Maerchen. Schauspiel in drei Aufzuegen1891 – 1891@\strich\emph{Das Märchen. Schauspiel in drei Aufzügen} {[}1891 – 1891{]}|pwkv}s informierte Blumenthal\pwindex{Blumenthal, Oskar 13.03.1852 – 24.04.1917@\textsc{Blumenthal, Oskar} (13.03.1852 – 24.04.1917), \emph{Schriftsteller, Journalist, Theaterleiter}|pwk}{ }Schnitzler\pwindex{Schnitzler, Arthur 15.05.1862 – 21.10.1931@\textsc{Schnitzler, Arthur} (15.05.1862 – 21.10.1931), \emph{Schriftsteller, Mediziner}|pwk} bereits am 15. 12. 1891. Eine Aufführung am \emph{Lessing-Theater}\orgindex{Lessing-Theater@Lessing-Theater|pwk} blieb jedoch aus.}}}\label{K_L02713-3h}. Nochmals: ohne
                  \uline{jede}{\pb}Garantie. Nur ein Möglichkeits-Spahn, um ihn mit
               Urlaubshoffnungen zu umſpinnen{\dotsfour}\pend
           \pstart
           Wird aus der Reiſe nichts, ſo erhältſt Du nach 1. September Nachricht von mir in Wien\oindex{Wien@\textbf{Wien}|pw}.\pend
           \pstart
           Viele treue Grüße! {\\[\baselineskip]}Dein {\\[\baselineskip]}\spacefill\mbox{Paul Goldm.}\pend
           \leftskip=0em{}\endnumbering\briefempfaengerindex{Schnitzler, Arthur@\textsc{Schnitzler, Arthur}!zzzGoldmann, Paul@\emph{von Paul Goldmann}!1893-08-231@{23. 8. {[}1893{]}}|)be}\mylabel{h}\end{ledgroupsized}\begin{anhang}\end{anhang}\newcommand{\dateiname}{L02713}\newcommand{\titel}{Paul Goldmann an Arthur Schnitzler, 23. 8. [1893]}\newcommand{\editorInnen}{Martin Anton Müller und Laura Untner}%% latex-leseansicht-abspann.tex
%% Abspann für die Leseansicht.
%% Der Schalter \ifkorrekturansicht ist bereits durch den Vorspann gesetzt.

%% latex-abspann.tex
%% Gemeinsamer Abspann für Korrekturansicht und Leseansicht.
%% Setzt den Schalter \ifkorrekturansicht voraus (gesetzt in den
%% einbindenden Dateien latex-korrekturansicht-abspann.tex bzw.
%% latex-leseansicht-abspann.tex).
%% ---------------------------------------------------------------

\normalsize

% Das esempio-Environment wird nur in der Leseansicht benötigt
\ifkorrekturansicht\else
\newenvironment{esempio}[3]%
{
    \vspace{1.5ex}
    \rlap{\underline{#1}}
    \par
    \setlength{\parindent}{0cm}
    \nopagebreak
    \leftskip=#2cm
    \rightskip=#3cm
}
{
    \par
}
\fi

\doendnotes{C}
\bigskip
\vfill

\clearpage

\footnotesize

\ifkorrekturansicht
  \lohead{\textsc{register}}
\fi

% theindex-Environment neu definieren ohne reledmac
\makeatletter
\renewenvironment{theindex}{%
  \ifkorrekturansicht
    \section*{\indexname}%
  \else
    \subsubsection*{Index der erwähnten Entitäten}%
  \fi
  \setlength{\parindent}{0pt}%
  \setlength{\parskip}{0pt plus 0.3pt}%
  \let\item\@idxitem
}{%
  \ifkorrekturansicht\clearpage\fi
}
\makeatother

\IfFileExists{\jobname-pw.ind}{\input{\jobname-pw.ind}}{}

% Quellenangabe nur in der Leseansicht
\ifkorrekturansicht\else
% Fallback-Definitionen, falls die .tex-Datei \titel etc. nicht gesetzt hat
\providecommand{\titel}{}
\providecommand{\editorInnen}{}
\providecommand{\dateiname}{\jobname}

\vspace{3cm}

\vfill

\footnotesize
\textsc{Quelle}: \titel. Herausgegeben von {\editorInnen}. In: \emph{Arthur Schnitzler: Briefwechsel mit Autorinnen und Autoren}.
 Digitale Edition, https://schnitzler-briefe.acdh.oeaw.ac.at/{\dateiname}.html (Stand \today)
\fi

\end{document}


      