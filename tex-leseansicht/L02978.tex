%% latex-leseansicht-vorspann.tex
%% Vorspann für die Leseansicht.
%% Lädt die gemeinsame Datei latex-vorspann.tex mit nicht gesetztem Schalter.

\newif\ifkorrekturansicht
\korrekturansichtfalse

\input{../tex-inputs/latex-vorspann}


\section[ Arthur Schnitzler an Felix Salten, 30. 9. 1902]{L02978 Arthur Schnitzler an Felix Salten,  30. 9. 1902}
\nopagebreak\mylabel{L02978v}
\rehead{ }\normalsize\beginnumbering\briefempfaengerindex{Salten, Felix@\textsc{Salten, Felix}!zzzSchnitzler, Arthur@\emph{von Arthur Schnitzler}!1902-09-301@{30. 9. 1902}|(be}
\toendnotes[C]{\smallbreak\pagebreak[2]}
\correspDesc{Versand  durch Arthur Schnitzler am 30. 9. 1902 in Wien
\newline{}Erhalt  durch Felix Salten im Zeitraum [30. 9. 1902
                  – 3. 10. 1902?] in Wien}\toendnotes[C]{\smallbreak}
\Standort{Wienbibliothek im Rathaus, ZPH 1681, 2.1.516.}
\physDesc{Brief, 1 Blatt, 2 Seiten, 421 Zeichen
\newline{}Handschrift: schwarze Tinte, deutsche Kurrent
\newline{}Ordnung: mit Bleistift von unbekannter Hand nummeriert: »66« }\toendnotes[C]{\smallbreak}
\pstart
           \raggedleft{}{\pb}30. 9. 902\pend
           
\pstart{}lieber Freund,\pend\vspace{0.5em}
\pstart
           ich konnte leider geſtern nicht länger auf Sie \label{K_L02978-1v}\edtext{warten}{\lemma{\textnormal{\emph{warten}}}\Cendnote{\textnormal{Mutmaßlich im
               Kaffeehaus, nachdem Schnitzler im Raimundtheater\oindex{Wien@\textbf{Wien}!VI., Mariahilf@\textbf{VI., Mariahilf}!Raimund-Theater@\textbf{Raimund-Theater}, \emph{Theater}|pwk}{ }\emph{Abschiedssouper}\pwindex{Schnitzler, Arthur 15.\,5.\,1862 Wien – 21.\,10.\,1931 ebd.@\textsc{Schnitzler, Arthur} (15.\,5.\,1862 Wien – 21.\,10.\,1931 ebd.), \emph{Schriftsteller, Mediziner}!Abschiedssouper@\strich\emph{Abschiedssouper}|pwk}
                 gesehen hatte, vgl. A. S.: \emph{Tagebuch}, 29. 9. 1902.
               }}}\label{K_L02978-1}. Hatte arge Kopfſchmerzen.\pend
           
\pstart
           Ihr \label{K_L02978-2v}\edtext{Zola\pwindex{Zola, Émile 2.\,4.\,1840 Paris – 29.\,9.\,1902 ebd.@\textsc{Zola, Émile} (2.\,4.\,1840 Paris – 29.\,9.\,1902 ebd.), \emph{Schriftsteller, Journalist}|pw} Feu{[}i{]}lleton\pwindex{Salten, Felix 6.\,9.\,1869 Budapest – 8.\,10.\,1945 Zürich@\textsc{Salten, Felix} (6.\,9.\,1869 Budapest – 8.\,10.\,1945 Zürich), \emph{Schriftsteller, Journalist, Chefredakteur}!Zola’s Lebenswerk@\strich\emph{Zola’s Lebenswerk}|pwv}}{\lemma{\textnormal{\emph{Zola Feuilleton}}}\Cendnote{\textnormal{Felix Salten\pwindex{Salten, Felix 6.\,9.\,1869 Budapest – 8.\,10.\,1945 Zürich@\textsc{Salten, Felix} (6.\,9.\,1869 Budapest – 8.\,10.\,1945 Zürich), \emph{Schriftsteller, Journalist, Chefredakteur}|pwk}: \emph{Zola’s Lebenswerk}\pwindex{Salten, Felix 6.\,9.\,1869 Budapest – 8.\,10.\,1945 Zürich@\textsc{Salten, Felix} (6.\,9.\,1869 Budapest – 8.\,10.\,1945 Zürich), \emph{Schriftsteller, Journalist, Chefredakteur}!Zola’s Lebenswerk@\strich\emph{Zola’s Lebenswerk}|pwk}. In: \emph{Die Zeit}\pwindex{Zeit@\emph{Die Zeit}|pwk}, Jg. 1, Nr. 4, 30. 9. 1902,
                     Morgenblatt, S. 1–2.}}}\label{K_L02978-2} iſt glänzend – insbeſondre freu {\pb}ich mich, daſs Sie \textsc{oeuvre\pwindex{Zola, Émile 2.\,4.\,1840 Paris – 29.\,9.\,1902 ebd.@\textsc{Zola, Émile} (2.\,4.\,1840 Paris – 29.\,9.\,1902 ebd.), \emph{Schriftsteller, Journalist}!œuvre@\strich\emph{L’œuvre}|pw}} und \textsc{joie de vivre}\pwindex{Zola, Émile 2.\,4.\,1840 Paris – 29.\,9.\,1902 ebd.@\textsc{Zola, Émile} (2.\,4.\,1840 Paris – 29.\,9.\,1902 ebd.), \emph{Schriftsteller, Journalist}!joie de vivre@\strich\emph{La joie de vivre}|pw} als die ewigen unter{ }ſeinen Werken herausgegriffen haben. Und das ganze hat{ }ſo
               einen Schmiſs.\pend
           
\pstart
           – Hoffentlich \label{K_L02978-3v}\edtext{ſeh ich Sie heut{ }Abend im Café}{\lemma{\textnormal{\emph{seh … Café}}}\Cendnote{\textnormal{Ein Treffen an diesem Abend ist nicht
                  nachgewiesen.}}}\label{K_L02978-3} und Sie bringen die kleine \textsc{Veronika}\pwindex{Salten, Felix 6.\,9.\,1869 Budapest – 8.\,10.\,1945 Zürich@\textsc{Salten, Felix} (6.\,9.\,1869 Budapest – 8.\,10.\,1945 Zürich), \emph{Schriftsteller, Journalist, Chefredakteur}!kleine Veronika@\strich\emph{Die kleine Veronika}|pw} mit we{\geminationn}{ }ſie{ }ſchon ins Kaffehaus gehen darf.\pend
           
\pstart
           Herzlichſt Ihr {\\[\baselineskip]}\spacefill\mbox{Arth Sch\textcolor{gray}{.}}\pend
           \leftskip=0em{}\selectlanguage{ngerman}\endnumbering\briefempfaengerindex{Salten, Felix@\textsc{Salten, Felix}!zzzSchnitzler, Arthur@\emph{von Arthur Schnitzler}!1902-09-301@{30. 9. 1902}|)be}\mylabel{L02978h}  \newcommand{\dateiname}{L02978}\newcommand{\titel}{Arthur Schnitzler an Felix Salten, 30. 9. 1902}\newcommand{\editorInnen}{Martin Anton Müller und Laura Untner}%% latex-leseansicht-abspann.tex
%% Abspann für die Leseansicht.
%% Der Schalter \ifkorrekturansicht ist bereits durch den Vorspann gesetzt.

%% latex-abspann.tex
%% Gemeinsamer Abspann für Korrekturansicht und Leseansicht.
%% Setzt den Schalter \ifkorrekturansicht voraus (gesetzt in den
%% einbindenden Dateien latex-korrekturansicht-abspann.tex bzw.
%% latex-leseansicht-abspann.tex).
%% ---------------------------------------------------------------

\normalsize

% Das esempio-Environment wird nur in der Leseansicht benötigt
\ifkorrekturansicht\else
\newenvironment{esempio}[3]%
{
    \vspace{1.5ex}
    \rlap{\underline{#1}}
    \par
    \setlength{\parindent}{0cm}
    \nopagebreak
    \leftskip=#2cm
    \rightskip=#3cm
}
{
    \par
}
\fi

\doendnotes{C}
\bigskip
\vfill

\clearpage

\footnotesize

\ifkorrekturansicht
  \lohead{\textsc{register}}
\fi

% theindex-Environment neu definieren ohne reledmac
\makeatletter
\renewenvironment{theindex}{%
  \ifkorrekturansicht
    \section*{\indexname}%
  \else
    \subsubsection*{Index der erwähnten Entitäten}%
  \fi
  \setlength{\parindent}{0pt}%
  \setlength{\parskip}{0pt plus 0.3pt}%
  \let\item\@idxitem
}{%
  \ifkorrekturansicht\clearpage\fi
}
\makeatother

\IfFileExists{\jobname-pw.ind}{\input{\jobname-pw.ind}}{}

% Quellenangabe nur in der Leseansicht
\ifkorrekturansicht\else
% Fallback-Definitionen, falls die .tex-Datei \titel etc. nicht gesetzt hat
\providecommand{\titel}{}
\providecommand{\editorInnen}{}
\providecommand{\dateiname}{\jobname}

\vspace{3cm}

\vfill

\footnotesize
\textsc{Quelle}: \titel. Herausgegeben von {\editorInnen}. In: \emph{Arthur Schnitzler: Briefwechsel mit Autorinnen und Autoren}.
 Digitale Edition, https://schnitzler-briefe.acdh.oeaw.ac.at/{\dateiname}.html (Stand \today)
\fi

\end{document}


