%% latex-korrekturansicht-vorspann.tex
%% Vorspann für die Korrekturansicht.
%% Lädt die gemeinsame Datei latex-vorspann.tex mit gesetztem Schalter.

\newif\ifkorrekturansicht
\korrekturansichttrue

\input{../tex-inputs/latex-vorspann}


\section[ Arthur Schnitzler an Felix Salten, 30. 9. 1902]{L02978 Arthur Schnitzler an Felix Salten, 30. 9. 1902}
\nopagebreak\mylabel{L02978v}
\rehead{ }\normalsize\beginnumbering\briefempfaengerindex{Salten, Felix@\textsc{Salten, Felix}!zzzSchnitzler, Arthur@\emph{von Arthur Schnitzler}!1902-09-301@{30. 9. 1902}|(be}
\toendnotes[C]{\smallbreak\pagebreak[2]}\Standort{Wienbibliothek im Rathaus, ZPH 1681, 2.1.516.}
\physDesc{Brief, 1 Blatt, 2 Seiten, 421 Zeichen
\newline{}Handschrift: schwarze Tinte, deutsche Kurrent
\newline{}Ordnung: mit Bleistift von unbekannter Hand nummeriert: »66« }\toendnotes[C]{\smallbreak}
\pstart
           \raggedleft{}{\pb}30. 9. 902\pend
           
\pstart{}lieber Freund,\pend\vspace{0.5em}
\pstart
           ich konnte leider geſtern nicht länger auf Sie \label{K_L02978-1v}\edtext{warten}{\lemma{\textnormal{\emph{warten}}}\Cendnote{\textnormal{Mutmaßlich im
               Kaffeehaus, nachdem Schnitzler im Raimundtheater\oindex{Raimund-Theater@\textbf{Raimund-Theater}, \emph{Theater (K.THE)}|pwk}{ }\emph{Abschiedssouper}\pwindex{Abschiedssouper@\emph{Abschiedssouper}|pwk}
                 gesehen hatte, vgl. A. S.: \emph{Tagebuch}, 29. 9. 1902.
               }}}\label{K_L02978-1}. Hatte arge Kopfſchmerzen.\pend
           
\pstart
           Ihr \label{K_L02978-2v}\edtext{Zola\pwindex{Zola, Emile 02.04.1840 – 29.09.1902@\textsc{Zola, Émile} (02.04.1840 – 29.09.1902), \emph{Schriftsteller/Schriftstellerin, Journalist/Journalistin}|pw} Feu{[}i{]}lleton\pwindex{Zola s Lebenswerk@\emph{Zola’s Lebenswerk}|pwv}}{\lemma{\textnormal{\emph{Zola Feuilleton}}}\Cendnote{\textnormal{Felix Salten\pwindex{Salten, Felix 06.09.1869 – 08.10.1945@\textsc{Salten, Felix} (06.09.1869 – 08.10.1945), \emph{Schriftsteller/Schriftstellerin, Journalist/Journalistin, Chefredakteur/Chefredakteurin}|pwk}: \emph{Zola’s Lebenswerk}\pwindex{Zola s Lebenswerk@\emph{Zola’s Lebenswerk}|pwk}. In: \emph{Die Zeit}\pwindex{Zeit@\emph{Die Zeit}|pwk}, Jg. 1, Nr. 4, 30. 9. 1902,
                     Morgenblatt, S. 1–2.}}}\label{K_L02978-2} iſt glänzend – insbeſondre freu {\pb}ich mich, daſs Sie \textsc{oeuvre\pwindex{œuvre@\emph{L’œuvre}|pw}} und \textsc{joie de vivre}\pwindex{joie de vivre@\emph{La joie de vivre}|pw} als die ewigen unter ſeinen Werken herausgegriffen haben. Und das ganze hat ſo
               einen Schmiſs.\pend
           
\pstart
           – Hoffentlich \label{K_L02978-3v}\edtext{ſeh ich Sie heut{ }Abend im Café}{\lemma{\textnormal{\emph{ſeh … Café}}}\Cendnote{\textnormal{Ein Treffen an diesem Abend ist nicht
                  nachgewiesen.}}}\label{K_L02978-3} und Sie bringen die kleine \textsc{Veronika}\pwindex{kleine Veronika@\emph{Die kleine Veronika}|pw} mit we{\geminationn} ſie ſchon ins Kaffehaus gehen darf.\pend
           
\pstart
           Herzlichſt Ihr {\\[\baselineskip]}\spacefill\mbox{Arth Sch\textcolor{gray}{.}}\pend
           \leftskip=0em{}\selectlanguage{ngerman}\endnumbering\briefempfaengerindex{Salten, Felix@\textsc{Salten, Felix}!zzzSchnitzler, Arthur@\emph{von Arthur Schnitzler}!1902-09-301@{30. 9. 1902}|)be}\mylabel{L02978h}  \normalsize

\doendnotes{C}
\bigskip
\vfill

\clearpage

\footnotesize

\lohead{\textsc{register}}

% Definiere theindex-Environment komplett neu ohne reledmac
\makeatletter
\renewenvironment{theindex}{%
  \section*{\indexname}%
  \setlength{\parindent}{0pt}%
  \setlength{\parskip}{0pt plus 0.3pt}%
  \let\item\@idxitem
}{%
  \clearpage
}
\makeatother

\IfFileExists{\jobname-pw.ind}{\input{\jobname-pw.ind}}{}

\end{document}

      