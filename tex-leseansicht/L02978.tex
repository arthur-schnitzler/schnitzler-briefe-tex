%% latex-leseansicht-vorspann.tex
%% Vorspann für die Leseansicht.
%% Lädt die gemeinsame Datei latex-vorspann.tex mit nicht gesetztem Schalter.

\newif\ifkorrekturansicht
\korrekturansichtfalse

\input{../tex-inputs/latex-vorspann}

\begin{center}
            \textcolor{red}{ENTWURF, NICHT FERTIG KORRIGIERT}
                      \end{center}
            
         
         \renewcommand{\erwaehntePersonen}{Personen: Felix Salten, Émile Zola}
         \renewcommand{\erwaehnteOrte}{Orte: Wien}
         \renewcommand{\erwaehnteWerke}{Werke: Die Zeit, Die kleine Veronika. Novelle, La joie de vivre, L’œuvre, Zola’s Lebenswerk}
               \section[Arthur Schnitzler an Felix Salten, 30. 9. 1902]{ Arthur Schnitzler an Felix Salten, 30. 9. 1902}\nopagebreak\mylabel{v}\rehead{ }\begin{ledgroupsized}[t]{13cm}\normalsize\beginnumbering \toendnotes[C]{\smallbreak\pagebreak[2]} \Standort{Wienbibliothek im Rathaus, ZPH 1681, 2.1.516.}
\physDesc{Brief, 1 Blatt, 2 Seiten, 422 Zeichen
\newline{}Handschrift: schwarze Tinte, deutsche Kurrent
\newline{}Ordnung: mit Bleistift von unbekannter Hand Nummerierung der Blätter des
                                 Konvoluts: »66« }\toendnotes[C]{\smallbreak}\pstart
           \raggedleft{}{\pb}30. 9. 902\pend
           \pstart{}lieber Freund,\pend\pstart
           ich konnte leider geſtern nicht länger auf Sie warten, hatte arge Kopfſchmerzen. Ihr
                  \label{K_L02978-1v}\edtext{Zola\pwindex{Zola, Emile 02.04.1840 – 29.09.1902@\textsc{Zola, Émile} (02.04.1840 – 29.09.1902), \emph{Schriftsteller, Journalist}|pw} Feu{[}i{]}lleton\pwindex{Salten, Felix 06.09.1869 – 08.10.1945@\textsc{Salten, Felix} (06.09.1869 – 08.10.1945), \emph{Schriftsteller, Journalist}!Zola s Lebenswerk1902-09-30@\strich\emph{Zola’s Lebenswerk} {[}1902-09-30{]}|pwv}}{\lemma{\textnormal{\emph{Zola Feuilleton}}}\Cendnote{\textnormal{Felix Salten\pwindex{Salten, Felix 06.09.1869 – 08.10.1945@\textsc{Salten, Felix} (06.09.1869 – 08.10.1945), \emph{Schriftsteller, Journalist}|pwk}: \emph{Zola’s Lebenswerk}\pwindex{Salten, Felix 06.09.1869 – 08.10.1945@\textsc{Salten, Felix} (06.09.1869 – 08.10.1945), \emph{Schriftsteller, Journalist}!Zola s Lebenswerk1902-09-30@\strich\emph{Zola’s Lebenswerk} {[}1902-09-30{]}|pwk}. In: \emph{Die Zeit}\pwindex{Zeit1902-09-27 – 1919@\emph{Die Zeit} {[}1902-09-27 – 1919{]}|pwk}, Jg. 1, Nr. 4, 30. 9. 1902, Morgenblatt,
                     S. 1–2.}}}\label{K_L02978-1h} iſt glänzend – insbeſondre freu {\pb}ich mich, daſs Sie \textsc{oeuvre\pwindex{Zola, Emile 02.04.1840 – 29.09.1902@\textsc{Zola, Émile} (02.04.1840 – 29.09.1902), \emph{Schriftsteller, Journalist}!œuvre1886@\strich\emph{L’œuvre} {[}1886{]}|pw}} und \textsc{joie de vivre}\pwindex{Zola, Emile 02.04.1840 – 29.09.1902@\textsc{Zola, Émile} (02.04.1840 – 29.09.1902), \emph{Schriftsteller, Journalist}!joie de vivre1884@\strich\emph{La joie de vivre} {[}1884{]}|pw} als die ewigen ſeinen Werken herausgegriffen haben. Und das ganze hat ſo einen
               Schmiſs. \pend
           \pstart
           – Hoffentlich ſeh ich Sie heut Abend im Café und Sie bringen die kleine \textsc{Veronika}\pwindex{Salten, Felix 06.09.1869 – 08.10.1945@\textsc{Salten, Felix} (06.09.1869 – 08.10.1945), \emph{Schriftsteller, Journalist}!kleine Veronika. Novelle1902-12-01@\strich\emph{Die kleine Veronika. Novelle} {[}1902-12-01{]}|pw} mit we{\geminationn} ſie ſchon ins Kaffehaus gehen darf.
               Herzlichſt Ihr \pend
           \pstart \spacefill\mbox{ArthSch}\pend{}
         
         \endnumbering\mylabel{h}\end{ledgroupsized}\begin{anhang}\end{anhang}\newcommand{\dateiname}{L02978}\newcommand{\titel}{Arthur Schnitzler an Felix Salten, 30. 9. 1902}\newcommand{\editorInnen}{Martin Anton Müller und Laura Untner}%% latex-leseansicht-abspann.tex
%% Abspann für die Leseansicht.
%% Der Schalter \ifkorrekturansicht ist bereits durch den Vorspann gesetzt.

%% latex-abspann.tex
%% Gemeinsamer Abspann für Korrekturansicht und Leseansicht.
%% Setzt den Schalter \ifkorrekturansicht voraus (gesetzt in den
%% einbindenden Dateien latex-korrekturansicht-abspann.tex bzw.
%% latex-leseansicht-abspann.tex).
%% ---------------------------------------------------------------

\normalsize

% Das esempio-Environment wird nur in der Leseansicht benötigt
\ifkorrekturansicht\else
\newenvironment{esempio}[3]%
{
    \vspace{1.5ex}
    \rlap{\underline{#1}}
    \par
    \setlength{\parindent}{0cm}
    \nopagebreak
    \leftskip=#2cm
    \rightskip=#3cm
}
{
    \par
}
\fi

\doendnotes{C}
\bigskip
\vfill

\clearpage

\footnotesize

\ifkorrekturansicht
  \lohead{\textsc{register}}
\fi

% theindex-Environment neu definieren ohne reledmac
\makeatletter
\renewenvironment{theindex}{%
  \ifkorrekturansicht
    \section*{\indexname}%
  \else
    \subsubsection*{Index der erwähnten Entitäten}%
  \fi
  \setlength{\parindent}{0pt}%
  \setlength{\parskip}{0pt plus 0.3pt}%
  \let\item\@idxitem
}{%
  \ifkorrekturansicht\clearpage\fi
}
\makeatother

\IfFileExists{\jobname-pw.ind}{\input{\jobname-pw.ind}}{}

% Quellenangabe nur in der Leseansicht
\ifkorrekturansicht\else
% Fallback-Definitionen, falls die .tex-Datei \titel etc. nicht gesetzt hat
\providecommand{\titel}{}
\providecommand{\editorInnen}{}
\providecommand{\dateiname}{\jobname}

\vspace{3cm}

\vfill

\footnotesize
\textsc{Quelle}: \titel. Herausgegeben von {\editorInnen}. In: \emph{Arthur Schnitzler: Briefwechsel mit Autorinnen und Autoren}.
 Digitale Edition, https://schnitzler-briefe.acdh.oeaw.ac.at/{\dateiname}.html (Stand \today)
\fi

\end{document}


      