%% latex-leseansicht-vorspann.tex
%% Vorspann für die Leseansicht.
%% Lädt die gemeinsame Datei latex-vorspann.tex mit nicht gesetztem Schalter.

\newif\ifkorrekturansicht
\korrekturansichtfalse

\input{../tex-inputs/latex-vorspann}


\section[ Paul Goldmann an Arthur Schnitzler, 7. 1. [1902]]{L03191 Paul Goldmann an Arthur Schnitzler,  7. 1. [1902]}
\nopagebreak\mylabel{L03191v}
\rehead{ }\normalsize\beginnumbering\briefempfaengerindex{Schnitzler, Arthur@\textsc{Schnitzler, Arthur}!zzzGoldmann, Paul@\emph{von Paul Goldmann}!1902-01-074@{7. 1. [1902]}|(be}
\toendnotes[C]{\smallbreak\pagebreak[2]}
\correspDesc{Versand  durch Paul Goldmann am 7. 1. [1902] in Berlin
\newline{}Erhalt  durch Arthur Schnitzler im Zeitraum [8. 1. 1902
                  – 12. 1. 1902?] in Wien}\toendnotes[C]{\smallbreak}
\Standort{DLA, A:Schnitzler, HS.NZ85.1.3172.}
\physDesc{Brief, 1 Blatt, 1 Seite, 314 Zeichen
\newline{}Handschrift: blaue Tinte, deutsche Kurrent
\newline{}Schnitzler: mit Bleistift das Jahr »902« vermerkt }\toendnotes[C]{\smallbreak}
\pstart
           \raggedleft{}{\pb}\textcolor{gray}{\textbf{DESSAUERSTRASSE 19}}\oindex{Dessauer Straße@\textbf{Dessauer Straße}, \emph{Straße}|pw}\pend
           
\pstart
           Berlin\oindex{Berlin@\textbf{Berlin}, \emph{Hauptstadt}|pw}, 7. Januar.\pend
           
\pstart\center{}Mein lieber Freund,\pend\vspace{0.5em}
\pstart
           \damage{B}itte, beantworte dieſe \label{K_L03191-1v}\edtext{Karte}{\lemma{\textnormal{\emph{Karte}}}\Cendnote{\textnormal{Beilage nicht erhalten}}}\label{K_L03191-1}
               direkt. \damage{(}Herrn \textsc{Paul Block\pwindex{Block, Paul 30.\,5.\,1862 Klaipėda – 15.\,8.\,1934 Bad Harzburg@\textsc{Block, Paul} (30.\,5.\,1862 Klaipėda – 15.\,8.\,1934 Bad Harzburg), \emph{Schriftsteller, Journalist}|pw}}, Redaktion des Berliner Tageblatt\orgindex{Berliner Tageblatt@Berliner Tageblatt|pw}). \strikeout{Der} Dieſer Herr \textsc{Block\pwindex{Block, Paul 30.\,5.\,1862 Klaipėda – 15.\,8.\,1934 Bad Harzburg@\textsc{Block, Paul} (30.\,5.\,1862 Klaipėda – 15.\,8.\,1934 Bad Harzburg), \emph{Schriftsteller, Journalist}|pw}} iſt ein{ }ſehr liebenswürdiger, für Deine Werke{ }ſehr begeiſteter Mann. Vielleicht
               kannſt Du ihm und dem Berliner Tageblatt\orgindex{Berliner Tageblatt@Berliner Tageblatt|pw}{ }\label{K_L03191-2v}\edtext{gefällig{ }ſein}{\lemma{\textnormal{\emph{gefällig sein}}}\Cendnote{\textnormal{Bezug unklar}}}\label{K_L03191-2}. Viele herzliche Grüße! Dein \spacefill\mbox{Paul
                  Goldmn}\pend
           \selectlanguage{ngerman}\endnumbering\briefempfaengerindex{Schnitzler, Arthur@\textsc{Schnitzler, Arthur}!zzzGoldmann, Paul@\emph{von Paul Goldmann}!1902-01-074@{7. 1. [1902]}|)be}\mylabel{L03191h}  \newcommand{\dateiname}{L03191}\newcommand{\titel}{Paul Goldmann an Arthur Schnitzler, 7. 1. [1902]}\newcommand{\editorInnen}{Martin Anton Müller und Laura Untner}%% latex-leseansicht-abspann.tex
%% Abspann für die Leseansicht.
%% Der Schalter \ifkorrekturansicht ist bereits durch den Vorspann gesetzt.

%% latex-abspann.tex
%% Gemeinsamer Abspann für Korrekturansicht und Leseansicht.
%% Setzt den Schalter \ifkorrekturansicht voraus (gesetzt in den
%% einbindenden Dateien latex-korrekturansicht-abspann.tex bzw.
%% latex-leseansicht-abspann.tex).
%% ---------------------------------------------------------------

\normalsize

% Das esempio-Environment wird nur in der Leseansicht benötigt
\ifkorrekturansicht\else
\newenvironment{esempio}[3]%
{
    \vspace{1.5ex}
    \rlap{\underline{#1}}
    \par
    \setlength{\parindent}{0cm}
    \nopagebreak
    \leftskip=#2cm
    \rightskip=#3cm
}
{
    \par
}
\fi

\doendnotes{C}
\bigskip
\vfill

\clearpage

\footnotesize

\ifkorrekturansicht
  \lohead{\textsc{register}}
\fi

% theindex-Environment neu definieren ohne reledmac
\makeatletter
\renewenvironment{theindex}{%
  \ifkorrekturansicht
    \section*{\indexname}%
  \else
    \subsubsection*{Index der erwähnten Entitäten}%
  \fi
  \setlength{\parindent}{0pt}%
  \setlength{\parskip}{0pt plus 0.3pt}%
  \let\item\@idxitem
}{%
  \ifkorrekturansicht\clearpage\fi
}
\makeatother

\IfFileExists{\jobname-pw.ind}{\input{\jobname-pw.ind}}{}

% Quellenangabe nur in der Leseansicht
\ifkorrekturansicht\else
% Fallback-Definitionen, falls die .tex-Datei \titel etc. nicht gesetzt hat
\providecommand{\titel}{}
\providecommand{\editorInnen}{}
\providecommand{\dateiname}{\jobname}

\vspace{3cm}

\vfill

\footnotesize
\textsc{Quelle}: \titel. Herausgegeben von {\editorInnen}. In: \emph{Arthur Schnitzler: Briefwechsel mit Autorinnen und Autoren}.
 Digitale Edition, https://schnitzler-briefe.acdh.oeaw.ac.at/{\dateiname}.html (Stand \today)
\fi

\end{document}


