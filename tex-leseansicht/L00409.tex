%% latex-leseansicht-vorspann.tex
%% Vorspann für die Leseansicht.
%% Lädt die gemeinsame Datei latex-vorspann.tex mit nicht gesetztem Schalter.

\newif\ifkorrekturansicht
\korrekturansichtfalse

\input{../tex-inputs/latex-vorspann}


\section[Arthur Schnitzler an Fedor Mamroth, 7. 12. 1894]{L00409 Arthur Schnitzler an Fedor Mamroth, 7. 12. 1894}
\nopagebreak\mylabel{L00409v}
\rehead{ }\normalsize\beginnumbering\briefempfaengerindex{Mamroth, Fedor@\textsc{Mamroth, Fedor}!zzzSchnitzler, Arthur@\emph{von Arthur Schnitzler}!1894-12-071@{7. 12. 1894}|(be}
\toendnotes[C]{\smallbreak\pagebreak[2]}
\correspDesc{Versand  durch Arthur Schnitzler am 7. 12. 1894 in Wien
\newline{}Erhalt  durch Fedor Mamroth im Zeitraum [7. 12. 1894
                  – 11. 12. 1894?] \textbf{Ort fehlend} }\toendnotes[C]{\smallbreak}
\Standort{YCGL, MSS 31.}
\physDesc{Brief, 1 Blatt, 2 Seiten, 494 Zeichen
\newline{}Handschrift: schwarze Tinte, deutsche Kurrent
\newline{}Ordnung: 1) mit blauer Tinte von unbekannter Hand wurde die Unterschrift
                                 ›entziffert‹: »Schnitzler«  2) mit Bleistift von unbekannter Hand wurde bei der Entzifferung
                                 des Nachnamens der Vorname »Arthur« ergänzt. 3) mit Bleistift von unbekannter Hand
                                    nummeriert: »421« und
                                 Vermerk: »1K«
\newline{}Zusatz: Als Empfänger ist Fedor Mamroth anzunehmen, den Schnitzler
                                 bereits vor dessen Engagement für die Frankfurter Zeitung\orgindex{Frankfurter Zeitung@Frankfurter Zeitung|pw} kennengelernt hatte. Der
                                 Brief wird unter jenen Schnitzlers an Richard Beer-Hofmann\pwindex{Beer-Hofmann, Richard 11.\,7.\,1866 Wien – 26.\,9.\,1945 New York City@\textsc{Beer-Hofmann, Richard} (11.\,7.\,1866 Wien – 26.\,9.\,1945 New York City), \emph{Schriftsteller}|pw} aufbewahrt. Erklären
                                 ließe sich dies etwa damit, dass es sich um einen Briefentwurf und
                                 nicht den tatsächlich gesandten Brief handeln könnte, oder dass Beer-Hofmann\pwindex{Beer-Hofmann, Richard 11.\,7.\,1866 Wien – 26.\,9.\,1945 New York City@\textsc{Beer-Hofmann, Richard} (11.\,7.\,1866 Wien – 26.\,9.\,1945 New York City), \emph{Schriftsteller}|pw} für die
                                 Übermittlung zuständig war und hier etwas schief lief. }\toendnotes[C]{\smallbreak}
\pstart{}{\pb}Verehrteſter Herr Doktor,\pend\vspace{0.5em}
\pstart
           es iſt mir ein Bedürfnis Ihnen für die liebenswürdige Raſchheit, mit welcher Sie die
                  \label{K_L00409-1v}\edtext{Beſprechung\pwindex{Schwarz, J. @\textsc{Schwarz, J.}, \emph{Journalist/Journalistin}!Belletristische Rundschau@\strich\emph{Belletristische Rundschau}|pwv}}{\lemma{\textnormal{\emph{Besprechung}}}\Cendnote{\textnormal{J. Schwarz\pwindex{Schwarz, J. @\textsc{Schwarz, J.}, \emph{Journalist/Journalistin}|pwk}: \emph{Belletristische Rundschau}\pwindex{Schwarz, J. @\textsc{Schwarz, J.}, \emph{Journalist/Journalistin}!Belletristische Rundschau@\strich\emph{Belletristische Rundschau}|pwk}. In: \emph{Frankfurter Zeitung}\pwindex{Frankfurter Zeitung@\emph{Frankfurter Zeitung}|pwk}, Nr. 336, 4. 12. 1894,
                     S. 1–3.}}}\label{K_L00409-1} meines letzten Buches\pwindex{Schnitzler, Arthur 15.\,5.\,1862 Wien – 21.\,10.\,1931 ebd.@\textsc{Schnitzler, Arthur} (15.\,5.\,1862 Wien – 21.\,10.\,1931 ebd.), \emph{Schriftsteller, Mediziner}!Sterben. Novelle@\strich\emph{Sterben. Novelle}|pwv} in der Frkf.
                  Ztg.\pwindex{Frankfurter Zeitung@\emph{Frankfurter Zeitung}|pw} erſcheinen ließen, aufs wärmſte zu danken. Darf ich Sie auch bitten, dem
                  Autor\pwindex{Schwarz, J. @\textsc{Schwarz, J.}, \emph{Journalist/Journalistin}|pwv} des Feuilletons\pwindex{Schwarz, J. @\textsc{Schwarz, J.}, \emph{Journalist/Journalistin}!Belletristische Rundschau@\strich\emph{Belletristische Rundschau}|pwv} gütigſt mitzutheilen, wie{ }ſehr
               mich {\pb}die{ }ſo erſtaunlich tiefen und warmen Worte
               gefreut haben, die er dem Buch\pwindex{Schnitzler, Arthur 15.\,5.\,1862 Wien – 21.\,10.\,1931 ebd.@\textsc{Schnitzler, Arthur} (15.\,5.\,1862 Wien – 21.\,10.\,1931 ebd.), \emph{Schriftsteller, Mediziner}!Sterben. Novelle@\strich\emph{Sterben. Novelle}|pwv}
               gewidmet hat? –\pend
           
\pstart
           Seien Sie, verehrteſter Herr Doktor, meiner herzlichen Ergebenheit jederzeit
               verſichert!\pend
           \pstart Ihr \spacefill\mbox{DrArthur Schnitzler}\pend{}
\pstart
           Wien\oindex{Wien@\textbf{Wien}, \emph{Verwaltungsgebiet}|pw}, 7. 12. 94.\pend
           \selectlanguage{ngerman}\endnumbering\briefempfaengerindex{Mamroth, Fedor@\textsc{Mamroth, Fedor}!zzzSchnitzler, Arthur@\emph{von Arthur Schnitzler}!1894-12-071@{7. 12. 1894}|)be}\mylabel{L00409h}  \newcommand{\dateiname}{L00409}\newcommand{\titel}{Arthur Schnitzler an Fedor Mamroth, 7. 12. 1894}\newcommand{\editorInnen}{Martin Anton Müller und Gerd-Hermann Susen}%% latex-leseansicht-abspann.tex
%% Abspann für die Leseansicht.
%% Der Schalter \ifkorrekturansicht ist bereits durch den Vorspann gesetzt.

%% latex-abspann.tex
%% Gemeinsamer Abspann für Korrekturansicht und Leseansicht.
%% Setzt den Schalter \ifkorrekturansicht voraus (gesetzt in den
%% einbindenden Dateien latex-korrekturansicht-abspann.tex bzw.
%% latex-leseansicht-abspann.tex).
%% ---------------------------------------------------------------

\normalsize

% Das esempio-Environment wird nur in der Leseansicht benötigt
\ifkorrekturansicht\else
\newenvironment{esempio}[3]%
{
    \vspace{1.5ex}
    \rlap{\underline{#1}}
    \par
    \setlength{\parindent}{0cm}
    \nopagebreak
    \leftskip=#2cm
    \rightskip=#3cm
}
{
    \par
}
\fi

\doendnotes{C}
\bigskip
\vfill

\clearpage

\footnotesize

\ifkorrekturansicht
  \lohead{\textsc{register}}
\fi

% theindex-Environment neu definieren ohne reledmac
\makeatletter
\renewenvironment{theindex}{%
  \ifkorrekturansicht
    \section*{\indexname}%
  \else
    \subsubsection*{Index der erwähnten Entitäten}%
  \fi
  \setlength{\parindent}{0pt}%
  \setlength{\parskip}{0pt plus 0.3pt}%
  \let\item\@idxitem
}{%
  \ifkorrekturansicht\clearpage\fi
}
\makeatother

\IfFileExists{\jobname-pw.ind}{\input{\jobname-pw.ind}}{}

% Quellenangabe nur in der Leseansicht
\ifkorrekturansicht\else
% Fallback-Definitionen, falls die .tex-Datei \titel etc. nicht gesetzt hat
\providecommand{\titel}{}
\providecommand{\editorInnen}{}
\providecommand{\dateiname}{\jobname}

\vspace{3cm}

\vfill

\footnotesize
\textsc{Quelle}: \titel. Herausgegeben von {\editorInnen}. In: \emph{Arthur Schnitzler: Briefwechsel mit Autorinnen und Autoren}.
 Digitale Edition, https://schnitzler-briefe.acdh.oeaw.ac.at/{\dateiname}.html (Stand \today)
\fi

\end{document}


