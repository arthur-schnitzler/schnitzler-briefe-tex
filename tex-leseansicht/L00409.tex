\input{../tex-inputs/latex-pdf-vorspann}
\begin{center}
            \textcolor{red}{ENTWURF. ENTZIFFERUNG NOCH NICHT KORREKTURGELESEN}
                      \end{center}
            
               \section[Arthur Schnitzler an Fedor Mamroth, 7. 12. 1894]{ Arthur Schnitzler an Fedor Mamroth, 7. 12. 1894}\nopagebreak\mylabel{v}\rehead{ }\begin{ledgroupsized}[t]{13cm}\normalsize\beginnumbering\briefempfaengerindex{Mamroth, Fedor@\textsc{Mamroth, Fedor}!zzzSchnitzler, Arthur@\emph{von Arthur Schnitzler}!1894-12-071@{7. 12. 1894}|(be} \toendnotes[C]{\smallbreak\pagebreak[2]} \Standort{YCGL, MSS 31.}
\physDesc{Brief, 1 Blatt, 2 Seiten
\newline{}Handschrift: schwarze Tinte, deutsche Kurrent\newline{}Ordnung: 1) mit blauer Tinte von unbekannter Hand wurde die Unterschrift
                                 ›entziffert‹: »Schnitzler« 2) mit Bleistift von unbekannter Hand wurde bei der Entzifferung
                                 des Nachnamens der Vorname »Arthur« ergänzt.3) mit Bleistift von unbekannter Hand
                                    nummeriert: »421« und
                                 Vermerk: »1K«\newline{}Zusatz: Als Empfänger ist Fedor Mamroth anzunehmen, den Schnitzler
                                 bereits vor dessen Engagement für die Frankfurter Zeitung\orgindex{Frankfurter Zeitung@Frankfurter Zeitung|pw} kennengelernt hatte. Der
                                 Brief wird unter jenen Schnitzlers an Richard Beer-Hofmann\pwindex{Beer-Hofmann, Richard 11.07.1866 – 26.09.1945@\textsc{Beer-Hofmann, Richard} (11.07.1866 – 26.09.1945), \emph{Schriftsteller}|pw} aufbewahrt. Erklären
                                 ließe sich dies etwa damit, dass es sich um einen Briefentwurf und
                                 nicht den tatsächlich gesandten Brief handeln könnte, oder dass Beer-Hofmann\pwindex{Beer-Hofmann, Richard 11.07.1866 – 26.09.1945@\textsc{Beer-Hofmann, Richard} (11.07.1866 – 26.09.1945), \emph{Schriftsteller}|pw} für die
                                 Übermittlung zuständig war und hier etwas schief lief. }\toendnotes[C]{\smallbreak}\pstart{}{\pb}Verehrteſter Herr Doktor,\pend\pstart
           es iſt mir ein Bedürfnis Ihnen für die liebenswürdige Raſchheit, mit welcher Sie die
                  \label{K_L00409-1v}\edtext{Beſprechung\pwindex{Schwarz, J. @\textsc{Schwarz, J.}, \emph{Journalist/Journalistin}!Belletristische Rundschau4.12.1894 – 4.12.1894@\strich\emph{Belletristische Rundschau} {[}4.12.1894 – 4.12.1894{]}|pwv}}{\lemma{\textnormal{\emph{Beſprechung}}}\Cendnote{\textnormal{J. Schwarz\pwindex{Schwarz, J. @\textsc{Schwarz, J.}, \emph{Journalist/Journalistin}|pwk}: \emph{Belletristische Rundschau}\pwindex{Schwarz, J. @\textsc{Schwarz, J.}, \emph{Journalist/Journalistin}!Belletristische Rundschau4.12.1894 – 4.12.1894@\strich\emph{Belletristische Rundschau} {[}4.12.1894 – 4.12.1894{]}|pwk}. In: \emph{Frankfurter Zeitung}\pwindex{Frankfurter Zeitung1856 – 1943@\emph{Frankfurter Zeitung}|pwk}, Nr. 336, 4. 12. 1894,
                     S. 1–3.}}}\label{K_L00409-1h} meines letzten Buches\pwindex{Schnitzler, Arthur 15.05.1862 – 21.10.1931@\textsc{Schnitzler, Arthur} (15.05.1862 – 21.10.1931), \emph{Schriftsteller, Mediziner}!Sterben. Novelle1.10.1894 – 1.12.1894@\strich\emph{Sterben. Novelle} {[}1.10.1894 – 1.12.1894{]}|pwv} in der Frkf.
                  Ztg.\pwindex{Frankfurter Zeitung1856 – 1943@\emph{Frankfurter Zeitung}|pw} erſcheinen ließen, aufs wärmſte zu danken. Darf ich Sie auch bitten, dem
                  Autor\pwindex{Schwarz, J. @\textsc{Schwarz, J.}, \emph{Journalist/Journalistin}|pwv} des Feuilletons\pwindex{Schwarz, J. @\textsc{Schwarz, J.}, \emph{Journalist/Journalistin}!Belletristische Rundschau4.12.1894 – 4.12.1894@\strich\emph{Belletristische Rundschau} {[}4.12.1894 – 4.12.1894{]}|pwv} gütigſt mitzutheilen, wie ſehr
               mich {\pb}die ſo erſtaunlich tiefen und warmen Worte
               gefreut haben, die er dem Buch\pwindex{Schnitzler, Arthur 15.05.1862 – 21.10.1931@\textsc{Schnitzler, Arthur} (15.05.1862 – 21.10.1931), \emph{Schriftsteller, Mediziner}!Sterben. Novelle1.10.1894 – 1.12.1894@\strich\emph{Sterben. Novelle} {[}1.10.1894 – 1.12.1894{]}|pwv}
               gewidmet hat? –\pend
           \pstart
           Seien Sie, verehrteſter Herr Doktor, meiner herzlichen Ergebenheit jederzeit
               verſichert!\pend
           \pstart Ihr \spacefill\mbox{DrArthur Schnitzler}\pend{}\pstart
           Wien\oindex{Wien@\textbf{Wien}|pw}, 7. 12. 94.\pend
           \endnumbering\briefempfaengerindex{Mamroth, Fedor@\textsc{Mamroth, Fedor}!zzzSchnitzler, Arthur@\emph{von Arthur Schnitzler}!1894-12-071@{7. 12. 1894}|)be}\mylabel{h}\end{ledgroupsized}  \newcommand{\dateiname}{L00409}\newcommand{\titel}{Arthur Schnitzler an Fedor Mamroth, 7. 12. 1894}\newcommand{\editorInnen}{Martin Anton Müller und Gerd-Hermann Susen}\input{../tex-inputs/latex-pdf-abspann}
      