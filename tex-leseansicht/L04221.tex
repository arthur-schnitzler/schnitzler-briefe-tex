%% latex-leseansicht-vorspann.tex
%% Vorspann für die Leseansicht.
%% Lädt die gemeinsame Datei latex-vorspann.tex mit nicht gesetztem Schalter.

\newif\ifkorrekturansicht
\korrekturansichtfalse

\input{../tex-inputs/latex-vorspann}


\section[Marie Holzer an Arthur Schnitzler, 17. 1. 1908]{L04221 Marie Holzer an Arthur Schnitzler, 17. 1. 1908}
\nopagebreak\mylabel{L04221v}
\rehead{ }\normalsize\beginnumbering\briefempfaengerindex{Schnitzler, Arthur@\textsc{Schnitzler, Arthur}!zzzHolzer, Marie@\emph{von Marie Holzer}!1908-01-173@{17. 1. 1908}|(be}
\toendnotes[C]{\smallbreak\pagebreak[2]}
\correspDesc{Versand  durch Marie Holzer am 17. 1. 1908 in Prag
\newline{}Erhalt  durch Arthur Schnitzler im Zeitraum [18. 1. 1908 – 22. 1. 1908?] in Wien}\toendnotes[C]{\smallbreak}
\Standort{CUL, Schnitzler, B 45.}
\physDesc{Brief, 1 Blatt, 2 Seiten, 793 Zeichen
\newline{}Handschrift: schwarze Tinte, deutsche Kurrent}\toendnotes[C]{\smallbreak}
\pstart
           \raggedleft{}{\pb}\textsc{Prag-Weinberge\oindex{Vinohrady@\textbf{Vinohrady}, \emph{Teil eines besiedelten Ortes}|pw}} den 17. 1. 1908.\pend
           
\pstart
           \raggedleft{}\textsc{Kroneng. 78\oindex{Korunní 1153/78@\textbf{Korunní 1153/78}, \emph{Wohngebäude}|pw}}.\pend
           
\pstart{}Sehr geehrter Herr Doktor!\pend\vspace{0.5em}
\pstart
           Ich hätte Ihnen gerne dies und jenesmal geschrieben – wenn ich eines
      Ihrer Bücher las, die mit ihren
      warmen Worten die ſtummen
      Seiten unſerer Seele zum klingen bringen und Gedanken aufſtöbern. Die irgendwie in einer verlore{\pb}nen Ecke stu{\geminationm} und ſcheu in uns
      leben. Aber Briefe von Bewunderern
      und Verehrerinnen haben Sie recht
      oft und oft beko{\geminationm}en – und da
      ko{\geminationm}t es auf einen mehr oder
      weniger nicht an. Aber heute anläßlich der Grillparzerpreiſes\orgindex{Franz-Grillparzer-Preis@Franz-Grillparzer-Preis|pw}
      möchte ich Ihnen doch ſagen, daß
      all Ihre Worte – all Ihre Gedanken
      ein wunderſames Schein in meiner
      Brust gefunden. Ich erlaube mir Ihnen
      einen kleinen \label{K_L04221-1v}\edtext{Aufſatz\pwindex{Holzer, Marie 11.\,1.\,1874 Czernowitz – 5.\,6.\,1924 Innsbruck@\textsc{Holzer, Marie} (11.\,1.\,1874 Czernowitz – 5.\,6.\,1924 Innsbruck)!Arthur Schnitzler@\strich\emph{Arthur Schnitzler}|pwv}}{\lemma{\textnormal{\emph{Aufsatz}}}\Cendnote{\textnormal{Marie Holzer\pwindex{Holzer, Marie 11.\,1.\,1874 Czernowitz – 5.\,6.\,1924 Innsbruck@\textsc{Holzer, Marie} (11.\,1.\,1874 Czernowitz – 5.\,6.\,1924 Innsbruck)|pwk}: \emph{Arthur Schnitzler}\pwindex{Holzer, Marie 11.\,1.\,1874 Czernowitz – 5.\,6.\,1924 Innsbruck@\textsc{Holzer, Marie} (11.\,1.\,1874 Czernowitz – 5.\,6.\,1924 Innsbruck)!Arthur Schnitzler@\strich\emph{Arthur Schnitzler}|pwk}. In:
            \emph{Prager Tagblatt}\pwindex{Prager Tagblatt@\emph{Prager Tagblatt}|pwk}, Jg. 32, Nr. 16, 17. 1. 1908, Morgen-Ausgabe, S. 8.
        
      }}}\label{K_L04221-1} zu ſenden, den
      ich zwiſchen heute und morgen im
               Prager Tagblatt\pwindex{Prager Tagblatt@\emph{Prager Tagblatt}|pw} veröffentlichte.\pend
           
\pstart
           Mit{\\[\baselineskip]}freundlichem Gruß,{\\[\baselineskip]}\spacefill\mbox{Marie Holzer}\pend
           \leftskip=0em{}
\pstart
           \noindent{}\raggedleft{}Hauptmannsgattin\pwindex{Holzer, Johann 15.\,5.\,1866 – 5.\,6.\,1924 Innsbruck@\textsc{Holzer, Johann} (15.\,5.\,1866 – 5.\,6.\,1924 Innsbruck), \emph{Militär, Mörder}|pwv}.\pend
           \selectlanguage{ngerman}\endnumbering\briefempfaengerindex{Schnitzler, Arthur@\textsc{Schnitzler, Arthur}!zzzHolzer, Marie@\emph{von Marie Holzer}!1908-01-173@{17. 1. 1908}|)be}\mylabel{L04221h}
\begin{anhang}
\end{anhang}\newcommand{\dateiname}{L04221}\newcommand{\titel}{Marie Holzer an Arthur Schnitzler, 17. 1. 1908}\newcommand{\editorInnen}{Herausgegeben von Jahnke, SelmaMüller, Martin Anton}%% latex-leseansicht-abspann.tex
%% Abspann für die Leseansicht.
%% Der Schalter \ifkorrekturansicht ist bereits durch den Vorspann gesetzt.

%% latex-abspann.tex
%% Gemeinsamer Abspann für Korrekturansicht und Leseansicht.
%% Setzt den Schalter \ifkorrekturansicht voraus (gesetzt in den
%% einbindenden Dateien latex-korrekturansicht-abspann.tex bzw.
%% latex-leseansicht-abspann.tex).
%% ---------------------------------------------------------------

\normalsize

% Das esempio-Environment wird nur in der Leseansicht benötigt
\ifkorrekturansicht\else
\newenvironment{esempio}[3]%
{
    \vspace{1.5ex}
    \rlap{\underline{#1}}
    \par
    \setlength{\parindent}{0cm}
    \nopagebreak
    \leftskip=#2cm
    \rightskip=#3cm
}
{
    \par
}
\fi

\doendnotes{C}
\bigskip
\vfill

\clearpage

\footnotesize

\ifkorrekturansicht
  \lohead{\textsc{register}}
\fi

% theindex-Environment neu definieren ohne reledmac
\makeatletter
\renewenvironment{theindex}{%
  \ifkorrekturansicht
    \section*{\indexname}%
  \else
    \subsubsection*{Index der erwähnten Entitäten}%
  \fi
  \setlength{\parindent}{0pt}%
  \setlength{\parskip}{0pt plus 0.3pt}%
  \let\item\@idxitem
}{%
  \ifkorrekturansicht\clearpage\fi
}
\makeatother

\IfFileExists{\jobname-pw.ind}{\input{\jobname-pw.ind}}{}

% Quellenangabe nur in der Leseansicht
\ifkorrekturansicht\else
% Fallback-Definitionen, falls die .tex-Datei \titel etc. nicht gesetzt hat
\providecommand{\titel}{}
\providecommand{\editorInnen}{}
\providecommand{\dateiname}{\jobname}

\vspace{3cm}

\vfill

\footnotesize
\textsc{Quelle}: \titel. Herausgegeben von {\editorInnen}. In: \emph{Arthur Schnitzler: Briefwechsel mit Autorinnen und Autoren}.
 Digitale Edition, https://schnitzler-briefe.acdh.oeaw.ac.at/{\dateiname}.html (Stand \today)
\fi

\end{document}


