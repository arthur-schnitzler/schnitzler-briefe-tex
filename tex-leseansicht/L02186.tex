%% latex-korrekturansicht-vorspann.tex
%% Vorspann für die Korrekturansicht.
%% Lädt die gemeinsame Datei latex-vorspann.tex mit gesetztem Schalter.

\newif\ifkorrekturansicht
\korrekturansichttrue

\input{../tex-inputs/latex-vorspann}


\section[Arthur und Olga Schnitzler u. a. an Richard Beer-Hofmann, 15. 7. 1914]{L02186 Arthur und Olga Schnitzler u. a. an Richard Beer-Hofmann,
               15. 7. 1914}
\nopagebreak\mylabel{L02186v}
\rehead{ }\normalsize\beginnumbering\briefempfaengerindex{Beer-Hofmann, Richard@\textsc{Beer-Hofmann, Richard}!zzzSchmidl, Paula@\emph{von Paula Schmidl}!1914-07-151@{15. 7. 1914}|(be}\briefempfaengerindex{Beer-Hofmann, Richard@\textsc{Beer-Hofmann, Richard}!zzzSchmidl, Hugo@\emph{von Hugo Schmidl}!1914-07-151@{15. 7. 1914}|(be}\briefempfaengerindex{Beer-Hofmann, Richard@\textsc{Beer-Hofmann, Richard}!zzzSchnitzler, Olga@\emph{von Olga Schnitzler}!1914-07-151@{15. 7. 1914}|(be}\briefempfaengerindex{Beer-Hofmann, Richard@\textsc{Beer-Hofmann, Richard}!zzzSchnitzler, Arthur@\emph{von Arthur Schnitzler}!1914-07-151@{15. 7. 1914}|(be}
\toendnotes[C]{\smallbreak\pagebreak[2]}\Standort{YCGL, MSS 31.}
\physDesc{Bildpostkarte, 215 Zeichen
\newline{}Handschrift Arthur Schnitzler: Bleistift, deutsche Kurrent
\newline{}Handschrift Paula Schmidl: Bleistift
\newline{}Handschrift Olga Schnitzler: Bleistift, lateinische Kurrent
\newline{}Handschrift Hugo Schmidl: Bleistift, lateinische Kurrent
\newline{}Versand: Stempel: »\nobreak{}1\textcolor{gray}{5}. VII. 14\nobreak{}«.  
\newline{}Beer-Hofmann: mit blauem Buntstift Vermerk: »E« (für
                                 »Erhalt«?) }\toendnotes[C]{\smallbreak}\pstart{}{\pb}Hrn \textsc{Dr. Richard
                     Beer-Hofmann}\pend{}\pstart{}\textsc{Weissenbach\oindex{Weissenbach am Attersee@\textbf{Weißenbach am Attersee}, \emph{A.ADM3}|pw}}\pend{}\pstart{}\textsc{Am Attersee\oindex{Attersee@\textbf{Attersee}, \emph{H.LK}|pw}}\pend{}\pstart{}\textsc{Ob.Oe.}\oindex{Oberoesterreich@\textbf{Oberösterreich}, \emph{A.ADM1}|pw}\pend{}{\bigskip}
\pstart
           \noindent{}\centering{}{\pb}\textcolor{gray}{\textbf{Mariazell\oindex{Mariazell@\textbf{Mariazell}, \emph{P.PPLA3}|pw}, 862 m Seehöhe, Steiermark\oindex{Steiermark@\textbf{Steiermark}, \emph{A.ADM1}|pw} gegen Hochschwab\oindex{Hochschwab@\textbf{Hochschwab}, \emph{T.MTS}|pw}.}}\pend
           \vspace{1em}
\pstart
           \noindent{}{\pb}Herzliche Grüße Ihnen Allen und vielen Dank für den
                  \label{K_L02186-1v}\edtext{Schachtelkäſe}{\lemma{\textnormal{\emph{Schachtelkäſe}}}\Cendnote{\textnormal{hier wohl in seiner weiter gefassten
                  Bedeutung eines in einer Schachtel verkauften Käses}}}\label{K_L02186-1}.\pend
           
\pstart
           Von einer Autopartie \textsc{Lunz}\oindex{Lunz am See@\textbf{Lunz am See}, \emph{P.PPLA3}|pw}–\textsc{Mariazell}\oindex{Mariazell@\textbf{Mariazell}, \emph{P.PPLA3}|pw}, 15/7 914.\pend
           \pstart \spacefill\mbox{Arthur}\pend{}\selectlanguage{ngerman}\vspace{1em}
\pstart
           \noindent{}{[}hs. :{]} Herzlichst\pend
           \pstart \spacefill\mbox{Olga.}\pend{}\selectlanguage{ngerman}\vspace{1em}\pstart \spacefill\mbox{{[}hs. :{]} Paula Schmidl}\pend{}\selectlanguage{ngerman}\vspace{1em}
\pstart
           \noindent{}{[}hs. :{]} Gruss!\pend
           \pstart \spacefill\mbox{Hugo Schmidl}\pend{}\selectlanguage{ngerman}\endnumbering\briefempfaengerindex{Beer-Hofmann, Richard@\textsc{Beer-Hofmann, Richard}!zzzSchmidl, Paula@\emph{von Paula Schmidl}!1914-07-151@{15. 7. 1914}|)be}\briefempfaengerindex{Beer-Hofmann, Richard@\textsc{Beer-Hofmann, Richard}!zzzSchmidl, Hugo@\emph{von Hugo Schmidl}!1914-07-151@{15. 7. 1914}|)be}\briefempfaengerindex{Beer-Hofmann, Richard@\textsc{Beer-Hofmann, Richard}!zzzSchnitzler, Olga@\emph{von Olga Schnitzler}!1914-07-151@{15. 7. 1914}|)be}\briefempfaengerindex{Beer-Hofmann, Richard@\textsc{Beer-Hofmann, Richard}!zzzSchnitzler, Arthur@\emph{von Arthur Schnitzler}!1914-07-151@{15. 7. 1914}|)be}\mylabel{L02186h}  \normalsize

\doendnotes{C}
\bigskip
\vfill

\clearpage

\footnotesize

\lohead{\textsc{register}}

% Definiere theindex-Environment komplett neu ohne reledmac
\makeatletter
\renewenvironment{theindex}{%
  \section*{\indexname}%
  \setlength{\parindent}{0pt}%
  \setlength{\parskip}{0pt plus 0.3pt}%
  \let\item\@idxitem
}{%
  \clearpage
}
\makeatother

\IfFileExists{\jobname-pw.ind}{\input{\jobname-pw.ind}}{}

\end{document}

      