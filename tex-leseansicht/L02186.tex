%% latex-leseansicht-vorspann.tex
%% Vorspann für die Leseansicht.
%% Lädt die gemeinsame Datei latex-vorspann.tex mit nicht gesetztem Schalter.

\newif\ifkorrekturansicht
\korrekturansichtfalse

\input{../tex-inputs/latex-vorspann}


               \section[Arthur und Olga Schnitzler u. a. an Richard Beer-Hofmann, 15. 7. 1914]{ Arthur und Olga Schnitzler u. a. an Richard Beer-Hofmann,
               15. 7. 1914}\nopagebreak\mylabel{v}\rehead{ }\begin{ledgroupsized}[t]{13cm}\normalsize\beginnumbering\briefempfaengerindex{Beer-Hofmann, Richard@\textsc{Beer-Hofmann, Richard}!zzzSchmidl, Paula@\emph{von Paula Schmidl}!1914-07-151@{15. 7. 1914}|(be}\briefempfaengerindex{Beer-Hofmann, Richard@\textsc{Beer-Hofmann, Richard}!zzzSchmidl, Hugo@\emph{von Hugo Schmidl}!1914-07-151@{15. 7. 1914}|(be}\briefempfaengerindex{Beer-Hofmann, Richard@\textsc{Beer-Hofmann, Richard}!zzzSchnitzler, Olga@\emph{von Olga Schnitzler}!1914-07-151@{15. 7. 1914}|(be}\briefempfaengerindex{Beer-Hofmann, Richard@\textsc{Beer-Hofmann, Richard}!zzzSchnitzler, Arthur@\emph{von Arthur Schnitzler}!1914-07-151@{15. 7. 1914}|(be} \toendnotes[C]{\smallbreak\pagebreak[2]} \Standort{YCGL, MSS 31.}
\physDesc{Bildpostkarte
\newline{}Handschrift Arthur Schnitzler: Bleistift, deutsche Kurrent\newline{}Handschrift Paula Schmidl: Bleistift\newline{}Handschrift Olga Schnitzler: Bleistift, lateinische Kurrent\newline{}Handschrift Hugo Schmidl: Bleistift, lateinische Kurrent\newline{}Versand: Stempel: »\nobreak{}1\textcolor{gray}{5}. VII. 14\nobreak{}«.  
\newline{}Beer-Hofmann: mit blauem Buntstift Vermerk: »E« (für
                                 »Erhalt«?) }\toendnotes[C]{\smallbreak}\pstart{}{\pb}Hrn \textsc{Dr. Richard Beer-Hofmann}\pend{}\pstart{}\textsc{Weissenbach\oindex{Weissenbach am Attersee@\textbf{Weißenbach am Attersee}|pw}}\pend{}\pstart{}\textsc{Am Attersee\oindex{Attersee@\textbf{Attersee}|pw}}\pend{}\pstart{}\textsc{Ob.Oe.}\oindex{Oberoesterreich@\textbf{Oberösterreich}|pw}\pend{}{\bigskip}\pstart
           \noindent{}\centering{}{\pb}\textcolor{gray}{\textbf{Mariazell\oindex{Mariazell@\textbf{Mariazell}|pw}, 862 m Seehöhe, Steiermark\oindex{Steiermark@\textbf{Steiermark}|pw} gegen Hochschwab\oindex{Hochschwab@\textbf{Hochschwab}|pw}.}}\pend
           \pstart
           {\pb}Herzliche Grüße Ihnen Allen und vielen Dank für den
                  \label{K_L02186-1v}\edtext{Schachtelkäſe}{\lemma{\textnormal{\emph{Schachtelkäſe}}}\Cendnote{\textnormal{hier wohl in seiner weiter gefassten
                  Bedeutung eines in einer Schachtel verkauften Käses}}}\label{K_L02186-1h}.\pend
           \pstart
           Von einer Autopartie \textsc{Lunz}\oindex{Lunz am See@\textbf{Lunz am See}|pw}–\textsc{Mariazell}\oindex{Mariazell@\textbf{Mariazell}|pw}, 15/7 914.\pend
           \pstart \spacefill\mbox{Arthur}\pend{}\pstart
           \noindent{}{[}hs. O. Schnitzler:{]} Herzlichst\pend
           \pstart \spacefill\mbox{Olga.}\pend{}\pstart \spacefill\mbox{{[}hs. Schmidl:{]} Paula Schmidl}\pend{}\pstart
           \noindent{}{[}hs. Schmidl:{]} Gruss!\pend
           \pstart \spacefill\mbox{Hugo Schmidl}\pend{}          \endnumbering\briefempfaengerindex{Beer-Hofmann, Richard@\textsc{Beer-Hofmann, Richard}!zzzSchmidl, Paula@\emph{von Paula Schmidl}!1914-07-151@{15. 7. 1914}|)be}\briefempfaengerindex{Beer-Hofmann, Richard@\textsc{Beer-Hofmann, Richard}!zzzSchmidl, Hugo@\emph{von Hugo Schmidl}!1914-07-151@{15. 7. 1914}|)be}\briefempfaengerindex{Beer-Hofmann, Richard@\textsc{Beer-Hofmann, Richard}!zzzSchnitzler, Olga@\emph{von Olga Schnitzler}!1914-07-151@{15. 7. 1914}|)be}\briefempfaengerindex{Beer-Hofmann, Richard@\textsc{Beer-Hofmann, Richard}!zzzSchnitzler, Arthur@\emph{von Arthur Schnitzler}!1914-07-151@{15. 7. 1914}|)be}\mylabel{h}\end{ledgroupsized}  \newcommand{\dateiname}{L02186}\newcommand{\titel}{Arthur und Olga Schnitzler u. a. an Richard Beer-Hofmann, 15. 7. 1914}\newcommand{\editorInnen}{Martin Anton Müller und Gerd-Hermann Susen}
            \footnotesize
\begin{ledgroupsized}[t]{11.5cm}
\doendnotes{C}
\end{ledgroupsized}
         %% latex-leseansicht-abspann.tex
%% Abspann für die Leseansicht.
%% Der Schalter \ifkorrekturansicht ist bereits durch den Vorspann gesetzt.

%% latex-abspann.tex
%% Gemeinsamer Abspann für Korrekturansicht und Leseansicht.
%% Setzt den Schalter \ifkorrekturansicht voraus (gesetzt in den
%% einbindenden Dateien latex-korrekturansicht-abspann.tex bzw.
%% latex-leseansicht-abspann.tex).
%% ---------------------------------------------------------------

\normalsize

% Das esempio-Environment wird nur in der Leseansicht benötigt
\ifkorrekturansicht\else
\newenvironment{esempio}[3]%
{
    \vspace{1.5ex}
    \rlap{\underline{#1}}
    \par
    \setlength{\parindent}{0cm}
    \nopagebreak
    \leftskip=#2cm
    \rightskip=#3cm
}
{
    \par
}
\fi

\doendnotes{C}
\bigskip
\vfill

\clearpage

\footnotesize

\ifkorrekturansicht
  \lohead{\textsc{register}}
\fi

% theindex-Environment neu definieren ohne reledmac
\makeatletter
\renewenvironment{theindex}{%
  \ifkorrekturansicht
    \section*{\indexname}%
  \else
    \subsubsection*{Index der erwähnten Entitäten}%
  \fi
  \setlength{\parindent}{0pt}%
  \setlength{\parskip}{0pt plus 0.3pt}%
  \let\item\@idxitem
}{%
  \ifkorrekturansicht\clearpage\fi
}
\makeatother

\IfFileExists{\jobname-pw.ind}{\input{\jobname-pw.ind}}{}

% Quellenangabe nur in der Leseansicht
\ifkorrekturansicht\else
% Fallback-Definitionen, falls die .tex-Datei \titel etc. nicht gesetzt hat
\providecommand{\titel}{}
\providecommand{\editorInnen}{}
\providecommand{\dateiname}{\jobname}

\vspace{3cm}

\vfill

\footnotesize
\textsc{Quelle}: \titel. Herausgegeben von {\editorInnen}. In: \emph{Arthur Schnitzler: Briefwechsel mit Autorinnen und Autoren}.
 Digitale Edition, https://schnitzler-briefe.acdh.oeaw.ac.at/{\dateiname}.html (Stand \today)
\fi

\end{document}


      