%% latex-korrekturansicht-vorspann.tex
%% Vorspann für die Korrekturansicht.
%% Lädt die gemeinsame Datei latex-vorspann.tex mit gesetztem Schalter.

\newif\ifkorrekturansicht
\korrekturansichttrue

\input{../tex-inputs/latex-vorspann}


\section[Richard Beer-Hofmann an Arthur Schnitzler, {[}23. 3. 1897{]}]{L00660 Richard Beer-Hofmann an Arthur Schnitzler, {[}23. 3. 1897{]}}
\nopagebreak\mylabel{L00660v}
\rehead{ }\normalsize\beginnumbering\briefempfaengerindex{Schnitzler, Arthur@\textsc{Schnitzler, Arthur}!zzzBeer-Hofmann, Richard@\emph{von Richard Beer-Hofmann}!1897-03-233@{23. 3. 1897}|(be}
\toendnotes[C]{\smallbreak\pagebreak[2]}\Standort{CUL, Schnitzler, B 8.}
\physDesc{Brief, 1 Blatt, 1 Seite, 172 Zeichen
\newline{}Handschrift: Bleistift, lateinische Kurrent
\newline{}Schnitzler: mit Bleistift datiert: »23/III 97« 
\newline{}Ordnung: mit Bleistift von unbekannter Hand nummeriert:
                                    »102« }
\pstart
           \noindent{}{\pb}Lieber Arthur! Ein Frl. Wengeroff\pwindex{Vengerova, Isabella 01.03.1877 – 07.02.1956@\textsc{Vengerova, Isabella} (01.03.1877 – 07.02.1956), \emph{Musikpädagoge/Musikpädagogin, Pianist/Pianistin}|pw} (Russin\oindex{Russland@\textbf{Russland}, \emph{A.PCLI}|pw}) möchte Sie und Hugo\pwindex{Hofmannsthal, Hugo von 1874-02-01 – 1929-07-15@\textsc{Hofmannsthal, Hugo von} (1874-02-01 – 1929-07-15), \emph{Schriftsteller/Schriftstellerin}|pw} heut nach 10 im Caffee sehn.
               Wenn Sie können ko{\geminationm}en Sie doch. Führer = Herr A. Brauner\pwindex{Brauner, Alexander 1871-10-21 – 1937@\textsc{Brauner, Alexander} (1871-10-21 – 1937), \emph{Techniker/Technikerin, Fremdenführer/Fremdenführerin}|pw}. Gefasst sein!\pend
           
\pstart
           Herzlichst{\\[\baselineskip]}\spacefill\mbox{Richard}\pend
           \leftskip=0em{}\selectlanguage{ngerman}\endnumbering\briefempfaengerindex{Schnitzler, Arthur@\textsc{Schnitzler, Arthur}!zzzBeer-Hofmann, Richard@\emph{von Richard Beer-Hofmann}!1897-03-233@{23. 3. 1897}|)be}\mylabel{L00660h}  \normalsize

\doendnotes{C}
\bigskip
\vfill

\clearpage

\footnotesize

\lohead{\textsc{register}}

% Definiere theindex-Environment komplett neu ohne reledmac
\makeatletter
\renewenvironment{theindex}{%
  \section*{\indexname}%
  \setlength{\parindent}{0pt}%
  \setlength{\parskip}{0pt plus 0.3pt}%
  \let\item\@idxitem
}{%
  \clearpage
}
\makeatother

\IfFileExists{\jobname-pw.ind}{\input{\jobname-pw.ind}}{}

\end{document}

      