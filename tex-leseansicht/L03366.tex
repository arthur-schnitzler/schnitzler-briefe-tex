%% latex-korrekturansicht-vorspann.tex
%% Vorspann für die Korrekturansicht.
%% Lädt die gemeinsame Datei latex-vorspann.tex mit gesetztem Schalter.

\newif\ifkorrekturansicht
\korrekturansichttrue

\input{../tex-inputs/latex-vorspann}


\section[ Paul Goldmann an Arthur Schnitzler, 6. 3. 1903]{L03366 Paul Goldmann an Arthur Schnitzler, 6. 3. 1903}
\nopagebreak\mylabel{L03366v}
\rehead{ }\normalsize\beginnumbering\briefempfaengerindex{Schnitzler, Arthur@\textsc{Schnitzler, Arthur}!zzzGoldmann, Paul@\emph{von Paul Goldmann}!1903-03-061@{6. 3. 1903}|(be}
\toendnotes[C]{\smallbreak\pagebreak[2]}\Standort{DLA, A:Schnitzler, HS.NZ85.1.3173.}
\physDesc{Postkarte, 306 Zeichen
\newline{}Handschrift: 1) blaue Tinte, deutsche Kurrent\hspace{1em}2) blaue Tinte, lateinische Kurrent (\noindent{}Adresse)\hspace{1em}
\newline{}Versand: Stempel: »\nobreak{}\oindex{Berlin@\textbf{Berlin}, \emph{P.PPLC}|pwk}Berlin, S. W. 11 a, 6. 3. 03, 7–8 N.\nobreak{}«. Stempel: »\nobreak{}\oindex{Berlin@\textbf{Berlin}, \emph{P.PPLC}|pwk}{[}Berli{]}\textcolor{gray}{n}, 7/3. 03, Beste\textcolor{gray}{llt} vom Postamte
                                          \textcolor{gray}{9}\nobreak{}«.  
\newline{}Schnitzler: mit Bleistift das Jahr »903« vermerkt }\toendnotes[C]{\smallbreak}\pstart{}{\pb}Herrn\pend{}\pstart{}Dr. Arthur Schnitzler\pend{}\pstart{}\strikeout{Wien\oindex{Wien@\textbf{Wien}, \emph{A.ADM2}|pw}}{ }Berlin W.\oindex{Berlin@\textbf{Berlin}, \emph{P.PPLC}|pw}\pend{}\pstart{}Palasthôtel\oindex{Palasthotel Berlin@\textbf{Palasthotel Berlin}, \emph{Hotel (K.HTL)}|pw}\pend{}{\bigskip}\vspace{1em}
\pstart
           {\pb}Berlin\oindex{Berlin@\textbf{Berlin}, \emph{P.PPLC}|pw}, 6. März.\pend
           \vspace{0.5em}
\pstart
           Liebſter Freund, Es thut mir unendlich leid, Deinen
               lieben Beſuch verfehlt zu haben. Heut{ }Abend habe ich mit einer großen Zuckerſteuerdebatte im Reichstag\orgindex{Reichstag@Reichstag|pw} mindeſtens bis zehn Uhr zu thun.
                  Morgen um ½ 2 komme ich ins \label{K_L03366-1v}\edtext{Palaſthotel\oindex{Palasthotel Berlin@\textbf{Palasthotel Berlin}, \emph{Hotel (K.HTL)}|pw}}{\lemma{\textnormal{\emph{Palaſthotel}}}\Cendnote{\textnormal{Während seines Berlin\oindex{Berlin@\textbf{Berlin}, \emph{P.PPLC}|pwk}-Aufenthalts zwischen 22. 2. 1903 und 9. 3. 1903 wohnte Schnitzler im Palasthotel\oindex{Palasthotel Berlin@\textbf{Palasthotel Berlin}, \emph{Hotel (K.HTL)}|pwk}. Dem \emph{Tagebuch}\pwindex{Tagebuch@\emph{Tagebuch}|pwk} ist nicht
                  zu entnehmen, ob Goldmann\pwindex{Goldmann, Paul 31.01.1865 – 25.09.1935@\textsc{Goldmann, Paul} (31.01.1865 – 25.09.1935), \emph{Schriftsteller/Schriftstellerin, Journalist/Journalistin}|pwk} ihn dort am 7. 3. 1903, noch vor
                  der Premiere von \emph{Der Schleier der Beatrice}\pwindex{Schleier der Beatrice. Schauspiel in fuenf Akten@\emph{Der Schleier der Beatrice. Schauspiel in fünf Akten}|pwk} am
                     Deutschen Theater Berlin\oindex{Deutsches Theater Berlin@\textbf{Deutsches Theater Berlin}, \emph{Theater (K.THE)}|pwk},
               besuchte.}}}\label{K_L03366-1}.\pend
           
\pstart
           Herzlichſt {\\[\baselineskip]}Dein \spacefill\mbox{Paul Goldm}\pend
           \leftskip=0em{}\selectlanguage{ngerman}\endnumbering\briefempfaengerindex{Schnitzler, Arthur@\textsc{Schnitzler, Arthur}!zzzGoldmann, Paul@\emph{von Paul Goldmann}!1903-03-061@{6. 3. 1903}|)be}\mylabel{L03366h}  \normalsize

\doendnotes{C}
\bigskip
\vfill

\clearpage

\footnotesize

\lohead{\textsc{register}}

% Definiere theindex-Environment komplett neu ohne reledmac
\makeatletter
\renewenvironment{theindex}{%
  \section*{\indexname}%
  \setlength{\parindent}{0pt}%
  \setlength{\parskip}{0pt plus 0.3pt}%
  \let\item\@idxitem
}{%
  \clearpage
}
\makeatother

\IfFileExists{\jobname-pw.ind}{\input{\jobname-pw.ind}}{}

\end{document}

      