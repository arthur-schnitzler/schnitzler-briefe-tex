%% latex-leseansicht-vorspann.tex
%% Vorspann für die Leseansicht.
%% Lädt die gemeinsame Datei latex-vorspann.tex mit nicht gesetztem Schalter.

\newif\ifkorrekturansicht
\korrekturansichtfalse

\input{../tex-inputs/latex-vorspann}


               \section[Max Burckhard an Arthur Schnitzler, 15. 9. 1895]{ Max Burckhard an Arthur Schnitzler, 15. 9. 1895}\nopagebreak\mylabel{v}\rehead{ }\begin{ledgroupsized}[t]{13cm}\normalsize\beginnumbering\briefempfaengerindex{Schnitzler, Arthur@\textsc{Schnitzler, Arthur}!zzzBurckhard, Max Eugen@\emph{von Max Eugen Burckhard}!1895-09-152@{15. 9. 1895}|(be} \toendnotes[C]{\smallbreak\pagebreak[2]} \Standort{CUL, Schnitzler, B 20.}
\physDesc{Brief, 1 Blatt, 2 Seiten
\newline{}Handschrift: schwarze Tinte, deutsche Kurrent\newline{}Ordnung: mit rotem Buntstift von unbekannter Hand nummeriert:
                                        »6.«, mutmaßlich von anderer Hand mit
                                    Bleistift durchgestrichen und nummeriert:
                                    »7« }\toendnotes[C]{\smallbreak}\pstart
           \noindent{}{\pb}\textcolor{gray}{\textbf{\label{T_L00484-1v}\edtext{k. k. Hofburgtheater Direction}{\lemma{\textnormal{\emph{k. k. … Direction}}}\Cendnote{\textnormal{Wappen in Prägedruck}}}\label{T_L00484-1h}\orgindex{Burgtheater@Burgtheater|pw}}}\hfill Wien\oindex{Wien@\textbf{Wien}|pw}{ }15. 9. 95\pend
           \pstart{}Sehr verehrter Herr Doctor!\pend\pstart
           Ich bin ſo frei Sie herzlichſt zur Leſeprobe\pwindex{Schnitzler, Arthur 15.05.1862 – 21.10.1931@\textsc{Schnitzler, Arthur} (15.05.1862 – 21.10.1931), \emph{Schriftsteller, Mediziner}!Liebelei. Schauspiel in drei Akten9. 10. 1895@\strich\emph{Liebelei. Schauspiel in drei Akten} {[}9. 10. 1895{]}|pwv} für Mittwoch 18 d. M. einzuladen. \uline{Es iſt Alles in Ordnung}. Ich bin leider an dem
                    Tage in Sprottau\oindex{Sprottau@\textbf{Sprottau}|pw}, Hr So{\geminationn}enthal\pwindex{Sonnenthal, Adolf von 21.12.1834 – 04.04.1909@\textsc{Sonnenthal, Adolf von} (21.12.1834 – 04.04.1909), \emph{Schauspieler}|pw} wird die Leſeprobe
                    leiten. Wenn etwas mit dem Dialect nicht zuſa{\geminationm}engeht, machen Sie ſich nichts draus, bei den Proben werde ich das ſchon
                    ausgleichen. Eine Rolle habe ich doch anders beſetzt – die Katharina\pwindex{Schnitzler, Arthur 15.05.1862 – 21.10.1931@\textsc{Schnitzler, Arthur} (15.05.1862 – 21.10.1931), \emph{Schriftsteller, Mediziner}!Liebelei. Schauspiel in drei Akten9. 10. 1895@\strich\emph{Liebelei. Schauspiel in drei Akten} {[}9. 10. 1895{]}|pwv} mit der Walbeck\pwindex{Walbeck, Fanny 11.10.1852 – 15.08.1919@\textsc{Walbeck, Fanny} (11.10.1852 – 15.08.1919), \emph{Schauspieler/Schauspielerin}|pw}: die Bauer\pwindex{Bauer, Anna 21.11.1853 – 07.12.1898@\textsc{Bauer, Anna} (21.11.1853 – 07.12.1898), \emph{Schauspieler/Schauspielerin}|pw}
                    iſt zu fein; ich werde die Walbeck\pwindex{Walbeck, Fanny 11.10.1852 – 15.08.1919@\textsc{Walbeck, Fanny} (11.10.1852 – 15.08.1919), \emph{Schauspieler/Schauspielerin}|pw}{ }ſchon »zurückhalten«.\pend
           \pstart
           {\pb}Ich habe jetzt auch einen Einakter
                    dazu, der würdig iſt und doch nicht im Styl widerſtreitet: \textsc{Giacosa}\pwindex{Giacosa, Giuseppe 21.10.1847 – 02.09.1906@\textsc{Giacosa, Giuseppe} (21.10.1847 – 02.09.1906), \emph{Schriftsteller}|pw}’s Rechte der Seele\pwindex{Giacosa, Giuseppe 21.10.1847 – 02.09.1906@\textsc{Giacosa, Giuseppe} (21.10.1847 – 02.09.1906), \emph{Schriftsteller}!Rechte der Seele1894@\strich\emph{Rechte der Seele} {[}1894{]}|pw}.\pend
           \pstart
           Anfangs Oktober hoffe ich ſind wir heraußen.\pend
           \pstart
           Herzlichſt Ihr ergebener{\\[\baselineskip]}\spacefill\mbox{D\textsuperscript{r}Burckhard}\pend
           \leftskip=0em{}\endnumbering\briefempfaengerindex{Schnitzler, Arthur@\textsc{Schnitzler, Arthur}!zzzBurckhard, Max Eugen@\emph{von Max Eugen Burckhard}!1895-09-152@{15. 9. 1895}|)be}\mylabel{h}\end{ledgroupsized}  \newcommand{\dateiname}{L00484}\newcommand{\titel}{Max Burckhard an Arthur Schnitzler, 15. 9. 1895}\newcommand{\editorInnen}{Martin Anton Müller und Gerd-Hermann Susen}%% latex-leseansicht-abspann.tex
%% Abspann für die Leseansicht.
%% Der Schalter \ifkorrekturansicht ist bereits durch den Vorspann gesetzt.

%% latex-abspann.tex
%% Gemeinsamer Abspann für Korrekturansicht und Leseansicht.
%% Setzt den Schalter \ifkorrekturansicht voraus (gesetzt in den
%% einbindenden Dateien latex-korrekturansicht-abspann.tex bzw.
%% latex-leseansicht-abspann.tex).
%% ---------------------------------------------------------------

\normalsize

% Das esempio-Environment wird nur in der Leseansicht benötigt
\ifkorrekturansicht\else
\newenvironment{esempio}[3]%
{
    \vspace{1.5ex}
    \rlap{\underline{#1}}
    \par
    \setlength{\parindent}{0cm}
    \nopagebreak
    \leftskip=#2cm
    \rightskip=#3cm
}
{
    \par
}
\fi

\doendnotes{C}
\bigskip
\vfill

\clearpage

\footnotesize

\ifkorrekturansicht
  \lohead{\textsc{register}}
\fi

% theindex-Environment neu definieren ohne reledmac
\makeatletter
\renewenvironment{theindex}{%
  \ifkorrekturansicht
    \section*{\indexname}%
  \else
    \subsubsection*{Index der erwähnten Entitäten}%
  \fi
  \setlength{\parindent}{0pt}%
  \setlength{\parskip}{0pt plus 0.3pt}%
  \let\item\@idxitem
}{%
  \ifkorrekturansicht\clearpage\fi
}
\makeatother

\IfFileExists{\jobname-pw.ind}{\input{\jobname-pw.ind}}{}

% Quellenangabe nur in der Leseansicht
\ifkorrekturansicht\else
% Fallback-Definitionen, falls die .tex-Datei \titel etc. nicht gesetzt hat
\providecommand{\titel}{}
\providecommand{\editorInnen}{}
\providecommand{\dateiname}{\jobname}

\vspace{3cm}

\vfill

\footnotesize
\textsc{Quelle}: \titel. Herausgegeben von {\editorInnen}. In: \emph{Arthur Schnitzler: Briefwechsel mit Autorinnen und Autoren}.
 Digitale Edition, https://schnitzler-briefe.acdh.oeaw.ac.at/{\dateiname}.html (Stand \today)
\fi

\end{document}


      