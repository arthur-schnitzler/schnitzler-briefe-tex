%% latex-korrekturansicht-vorspann.tex
%% Vorspann für die Korrekturansicht.
%% Lädt die gemeinsame Datei latex-vorspann.tex mit gesetztem Schalter.

\newif\ifkorrekturansicht
\korrekturansichttrue

\input{../tex-inputs/latex-vorspann}


\section[Max Burckhard an Arthur Schnitzler, 15. 9. 1895]{L00484 Max Burckhard an Arthur Schnitzler, 15. 9. 1895}
\nopagebreak\mylabel{L00484v}
\rehead{ }\normalsize\beginnumbering\briefempfaengerindex{Schnitzler, Arthur@\textsc{Schnitzler, Arthur}!zzzBurckhard, Max Eugen@\emph{von Max Eugen Burckhard}!1895-09-152@{15. 9. 1895}|(be}
\toendnotes[C]{\smallbreak\pagebreak[2]}\Standort{CUL, Schnitzler, B 20.}
\physDesc{Brief, 1 Blatt, 2 Seiten, 671 Zeichen
\newline{}Handschrift: schwarze Tinte, deutsche Kurrent
\newline{}Ordnung: mit rotem Buntstift von unbekannter Hand nummeriert:
                                    »6.«, mutmaßlich von anderer Hand mit Bleistift
                                 durchgestrichen und nummeriert: »7« }\toendnotes[C]{\smallbreak}
\pstart
           {\pb}\textcolor{gray}{\textbf{\label{T_L00484-1v}\edtext{k. k. Hofburgtheater
                              Direction}{\lemma{\textnormal{\emph{k. k. … Direction}}}\Cendnote{\textnormal{Wappen in
                              Prägedruck}}}\label{T_L00484-1}\orgindex{Burgtheater@Burgtheater|pw}}}\hfill Wien\oindex{Wien@\textbf{Wien}, \emph{A.ADM2}|pw}{ }15. 9. 95\pend
           
\pstart{}Sehr verehrter Herr Doctor!\pend\vspace{0.5em}
\pstart
           Ich bin ſo frei Sie herzlichſt zur Leſeprobe\pwindex{Liebelei. Schauspiel in drei Akten@\emph{Liebelei. Schauspiel in drei Akten}|pwv} für Mittwoch 18 d. M. einzuladen. \uline{Es iſt Alles in Ordnung}. Ich bin leider an dem Tage in
                  Sprottau\oindex{Sprottau@\textbf{Sprottau}, \emph{P.PPL}|pw}, Hr So{\geminationn}enthal\pwindex{Sonnenthal, Adolf von 1834-12-21 – 1909-04-04@\textsc{Sonnenthal, Adolf von} (1834-12-21 – 1909-04-04), \emph{Schauspieler/Schauspielerin}|pw} wird die Leſeprobe
               leiten. Wenn etwas mit dem Dialect nicht zuſa{\geminationm}engeht,
               machen Sie ſich nichts draus, bei den Proben werde ich das ſchon ausgleichen. Eine
               Rolle habe ich doch anders beſetzt – die Katharina\pwindex{Liebelei. Schauspiel in drei Akten@\emph{Liebelei. Schauspiel in drei Akten}|pwv} mit der Walbeck\pwindex{Walbeck, Fanny 1850-10-11 – 1919-08-15@\textsc{Walbeck, Fanny} (1850-10-11 – 1919-08-15), \emph{Schauspieler/Schauspielerin}|pw}: die Bauer\pwindex{Bauer, Anna 21.11.1853 – 07.12.1898@\textsc{Bauer, Anna} (21.11.1853 – 07.12.1898), \emph{Schauspieler/Schauspielerin}|pw} iſt zu fein; ich
               werde die Walbeck\pwindex{Walbeck, Fanny 1850-10-11 – 1919-08-15@\textsc{Walbeck, Fanny} (1850-10-11 – 1919-08-15), \emph{Schauspieler/Schauspielerin}|pw}{ }ſchon »zurückhalten«.\pend
           
\pstart
           {\pb}Ich habe jetzt auch einen Einakter dazu,
               der würdig iſt und doch nicht im Styl widerſtreitet: \textsc{Giacosa}\pwindex{Giacosa, Giuseppe 21.10.1847 – 02.09.1906@\textsc{Giacosa, Giuseppe} (21.10.1847 – 02.09.1906), \emph{Schriftsteller/Schriftstellerin}|pw}’s Rechte der Seele\pwindex{Rechte der Seele. Schauspiel in einem Act@\emph{Rechte der Seele. Schauspiel in einem Act}|pw}.\pend
           
\pstart
           Anfangs Oktober hoffe ich ſind wir heraußen.\pend
           
\pstart
           Herzlichſt Ihr ergebener{\\[\baselineskip]}\spacefill\mbox{D\textsuperscript{r}Burckhard}\pend
           \leftskip=0em{}\selectlanguage{ngerman}\endnumbering\briefempfaengerindex{Schnitzler, Arthur@\textsc{Schnitzler, Arthur}!zzzBurckhard, Max Eugen@\emph{von Max Eugen Burckhard}!1895-09-152@{15. 9. 1895}|)be}\mylabel{L00484h}  \normalsize

\doendnotes{C}
\bigskip
\vfill

\clearpage

\footnotesize

\lohead{\textsc{register}}

% Definiere theindex-Environment komplett neu ohne reledmac
\makeatletter
\renewenvironment{theindex}{%
  \section*{\indexname}%
  \setlength{\parindent}{0pt}%
  \setlength{\parskip}{0pt plus 0.3pt}%
  \let\item\@idxitem
}{%
  \clearpage
}
\makeatother

\IfFileExists{\jobname-pw.ind}{\input{\jobname-pw.ind}}{}

\end{document}

      