%% latex-korrekturansicht-vorspann.tex
%% Vorspann für die Korrekturansicht.
%% Lädt die gemeinsame Datei latex-vorspann.tex mit gesetztem Schalter.

\newif\ifkorrekturansicht
\korrekturansichttrue

\input{../tex-inputs/latex-vorspann}


\section[Michael Georg Conrad an Arthur Schnitzler, 22. 1. 1904]{L01362 Michael Georg Conrad an Arthur Schnitzler, 22. 1. 1904}
\nopagebreak\mylabel{L01362v}
\rehead{ }\normalsize\beginnumbering\briefempfaengerindex{Schnitzler, Arthur@\textsc{Schnitzler, Arthur}!zzzConrad, Michael Georg@\emph{von Michael Georg Conrad}!1904-01-221@{22. 1. 1904}|(be}
\toendnotes[C]{\smallbreak\pagebreak[2]}\Standort{CUL, Schnitzler, B 22.}
\physDesc{Postkarte, 737 Zeichen
\newline{}Handschrift: schwarze Tinte, deutsche Kurrent
\newline{}Versand: 1) Stempel: »\nobreak{}\oindex{Muenchen@\textbf{München}, \emph{P.PPLA}|pwk}München 26, 22 Jan 04, 6–7 N\nobreak{}«.   2) Stempel: »\nobreak{}\oindex{IX., Alsergrund@\textbf{IX., Alsergrund}, \emph{A.ADM3}|pwk}Wien 9/3 73, 23. 1. 04, 11. V\nobreak{}«.  3) Stempel: »\nobreak{}\oindex{I., Innere Stadt@\textbf{I., Innere Stadt}, \emph{A.ADM3}|pwk}Wien 110, 23. 1. 04, 3. N\nobreak{}«.  4) nachgesandt nach: Spöttelg 7\oindex{Edmund-Weiss-Gasse 7@\textbf{Edmund-Weiß-Gasse 7}, \emph{Wohngebäude (K.WHS)}|pw}\hspace*{1.5em}XVIII/I}\toendnotes[C]{\smallbreak}\pstart{}{\pb}Hochwohlgeboren\pend{}\pstart{}Herrn \textsc{D\textsuperscript{r}} Arthur Schnitzler\pend{}\pstart{}Dichter\pend{}\pstart{}\textsc{Wien XII\oindex{XII., Meidling@\textbf{XII., Meidling}, \emph{A.ADM3}|pw}}.\pend{}\pstart{}\textsc{Frankgasse 1\oindex{Frankgasse 1@\textbf{Frankgasse 1}, \emph{Wohngebäude (K.WHS)}|pw}.}\pend{}{\bigskip}\vspace{1em}
\pstart
           {\pb}München\oindex{Muenchen@\textbf{München}, \emph{P.PPLA}|pw}, Steinsdorfſtr. 7\oindex{Steinsdorfstrasse@\textbf{Steinsdorfstraße}, \emph{Straße (K.STR)}|pw}\pend
           
\pstart
           \raggedleft{}22. 1. 04.\pend
           \vspace{0.5em}
\pstart
           Lieber Herr Doktor, ein mediumiſtiſches Schreibweibchen, Frau Marie Knorr-Schmidt\pwindex{Knorr-Schmidt, Marie 1861-08-20 – 1942?@\textsc{Knorr-Schmidt, Marie} (1861-08-20 – 1942?), \emph{Schriftsteller/Schriftstellerin}|pw} aus Meerane\oindex{Meerane@\textbf{Meerane}, \emph{P.PPL}|pw} in Sachſen\oindex{Sachsen@\textbf{Sachsen}, \emph{A.ADM1}|pw}, Bismarckſtr. 3\oindex{Innere Crimmitschauer Strasse@\textbf{Innere Crimmitschauer Straße}, \emph{Straße (K.STR)}|pw}, will Sie ein wenig anöden mit
               Dichteleien aus der vierten Dimenſion. Das Buch\pwindex{Evoe Ein Schritt zur Lichtung des Seelenlebens@\emph{Evoë{\rufezeichen} Ein Schritt zur Lichtung des Seelenlebens}|pwv} geht Ihnen heute zu. Bitte, werfen Sie einen Blick
               hinein. Ich habe nämlich der Dame – um endlich Ruhe zu kriegen – verſprochen, Sie
               durch inſtändiges Bitten dahin zu bringen, daß Sie einen Blick hineinwerfen. Dann
               nehmen Sie eine Postkarte und beſtätigen mir: Ich habe einen Blick hineingeworfen.
               Das genügt. \textsc{Voilà tout}. Der Geiſter-Dichter aus der vierten
               Dimenſion wird beſchwichtigt und wir können uns wieder wichtigen Dingen widmen. Gruß!
                  \spacefill\mbox{C.}\pend
           \selectlanguage{ngerman}\endnumbering\briefempfaengerindex{Schnitzler, Arthur@\textsc{Schnitzler, Arthur}!zzzConrad, Michael Georg@\emph{von Michael Georg Conrad}!1904-01-221@{22. 1. 1904}|)be}\mylabel{L01362h}  \normalsize

\doendnotes{C}
\bigskip
\vfill

\clearpage

\footnotesize

\lohead{\textsc{register}}

% Definiere theindex-Environment komplett neu ohne reledmac
\makeatletter
\renewenvironment{theindex}{%
  \section*{\indexname}%
  \setlength{\parindent}{0pt}%
  \setlength{\parskip}{0pt plus 0.3pt}%
  \let\item\@idxitem
}{%
  \clearpage
}
\makeatother

\IfFileExists{\jobname-pw.ind}{\input{\jobname-pw.ind}}{}

\end{document}

      