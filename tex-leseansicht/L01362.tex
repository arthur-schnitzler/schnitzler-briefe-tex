%% latex-leseansicht-vorspann.tex
%% Vorspann für die Leseansicht.
%% Lädt die gemeinsame Datei latex-vorspann.tex mit nicht gesetztem Schalter.

\newif\ifkorrekturansicht
\korrekturansichtfalse

\input{../tex-inputs/latex-vorspann}


         
         \newcommand{\erwaehntePersonen}{Personen: Marie Knorr-Schmidt}
         \newcommand{\erwaehnteInstitutionen}{}
         \newcommand{\erwaehnteOrte}{Orte: Edmund-Weiß-Gasse, Frankgasse, I., Innere Stadt, IX., Alsergrund, Innere Crimmitschauer Straße, Meerane, München, Sachsen, Steinsdorfstraße, XII., Meidling}
         \newcommand{\erwaehnteWerke}{Werke: Evoë! Ein Schritt zur Lichtung des Seelenlebens}
               \section[Michael Georg Conrad an Arthur Schnitzler, 22. 1. 1904]{ Michael Georg Conrad an Arthur Schnitzler, 22. 1. 1904}\nopagebreak\mylabel{v}\rehead{ }\begin{ledgroupsized}[t]{13cm}\normalsize\beginnumbering \toendnotes[C]{\smallbreak\pagebreak[2]} \Standort{CUL, Schnitzler, B 22.}
\physDesc{Postkarte
\newline{}Handschrift: schwarze Tinte, deutsche Kurrent\newline{}Versand: 1) Stempel: »\nobreak{}\oindex{Muenchen@\textbf{München}|pwk}München 26, 22 Jan 04, 6–7 N\nobreak{}«.   2) Stempel: »\nobreak{}\oindex{IX., Alsergrund@\textbf{IX., Alsergrund}|pwk}Wien 9/3 73, 23. 1. 04, 11. V\nobreak{}«.  3) Stempel: »\nobreak{}\oindex{I., Innere Stadt@\textbf{I., Innere Stadt}|pwk}Wien 110, 23. 1. 04, 3. N\nobreak{}«.  4) nachgesandt nach: Spöttelg 7\oindex{Edmund-Weiss-Gasse@\textbf{Edmund-Weiß-Gasse}|pw}\hspace*{1.5em}XVIII/I}\toendnotes[C]{\smallbreak}\pstart{}{\pb}Hochwohlgeboren\pend{}\pstart{}Herrn \textsc{D\textsuperscript{r}} Arthur Schnitzler\pend{}\pstart{}Dichter\pend{}\pstart{}\textsc{Wien XII\oindex{XII., Meidling@\textbf{XII., Meidling}|pw}}.\pend{}\pstart{}\textsc{Frankgasse 1\oindex{Frankgasse@\textbf{Frankgasse}|pw}.}\pend{}{\bigskip}\pstart
           \noindent{}{\pb}München\oindex{Muenchen@\textbf{München}|pw}, Steinsdorfſtr. 7\oindex{Steinsdorfstrasse@\textbf{Steinsdorfstraße}|pw}\pend
           \pstart
           \raggedleft{}22. 1. 04.\pend
           \pstart
           Lieber Herr Doktor, ein mediumiſtiſches Schreibweibchen, Frau Marie Knorr-Schmidt\pwindex{Knorr-Schmidt, Marie 1861-08-20 – 1942?@\textsc{Knorr-Schmidt, Marie} (1861-08-20 – 1942?), \emph{Schriftstellerin}|pw} aus Meerane\oindex{Meerane@\textbf{Meerane}|pw} in Sachſen\oindex{Sachsen@\textbf{Sachsen}|pw}, Bismarckſtr. 3\oindex{Innere Crimmitschauer Strasse@\textbf{Innere Crimmitschauer Straße}|pw}, will Sie ein wenig anöden mit
                    Dichteleien aus der vierten Dimenſion. Das Buch\pwindex{Knorr-Schmidt, Marie 1861-08-20 – 1942?@\textsc{Knorr-Schmidt, Marie} (1861-08-20 – 1942?), \emph{Schriftstellerin}!Evoe Ein Schritt zur Lichtung des Seelenlebens1903@\strich\emph{Evoë{\rufezeichen} Ein Schritt zur Lichtung des Seelenlebens} {[}1903{]}|pwv} geht Ihnen heute zu. Bitte, werfen Sie einen
                    Blick hinein. Ich habe nämlich der Dame – um endlich Ruhe zu kriegen –
                    verſprochen, Sie durch inſtändiges Bitten dahin zu bringen, daß Sie einen Blick
                    hineinwerfen. Dann nehmen Sie eine Postkarte und beſtätigen mir: Ich habe einen
                    Blick hineingeworfen. Das genügt. \textsc{Voilà tout}. Der
                    Geiſter-Dichter aus der vierten Dimenſion wird beſchwichtigt und wir können uns
                    wieder wichtigen Dingen widmen. Gruß! \spacefill\mbox{C.}\pend
           
         
         \endnumbering\mylabel{h}\end{ledgroupsized}  \newcommand{\dateiname}{L01362}\newcommand{\titel}{Michael Georg Conrad an Arthur Schnitzler, 22. 1. 1904}\newcommand{\editorInnen}{Martin Anton Müller und Gerd-Hermann Susen}%% latex-leseansicht-abspann.tex
%% Abspann für die Leseansicht.
%% Der Schalter \ifkorrekturansicht ist bereits durch den Vorspann gesetzt.

%% latex-abspann.tex
%% Gemeinsamer Abspann für Korrekturansicht und Leseansicht.
%% Setzt den Schalter \ifkorrekturansicht voraus (gesetzt in den
%% einbindenden Dateien latex-korrekturansicht-abspann.tex bzw.
%% latex-leseansicht-abspann.tex).
%% ---------------------------------------------------------------

\normalsize

% Das esempio-Environment wird nur in der Leseansicht benötigt
\ifkorrekturansicht\else
\newenvironment{esempio}[3]%
{
    \vspace{1.5ex}
    \rlap{\underline{#1}}
    \par
    \setlength{\parindent}{0cm}
    \nopagebreak
    \leftskip=#2cm
    \rightskip=#3cm
}
{
    \par
}
\fi

\doendnotes{C}
\bigskip
\vfill

\clearpage

\footnotesize

\ifkorrekturansicht
  \lohead{\textsc{register}}
\fi

% theindex-Environment neu definieren ohne reledmac
\makeatletter
\renewenvironment{theindex}{%
  \ifkorrekturansicht
    \section*{\indexname}%
  \else
    \subsubsection*{Index der erwähnten Entitäten}%
  \fi
  \setlength{\parindent}{0pt}%
  \setlength{\parskip}{0pt plus 0.3pt}%
  \let\item\@idxitem
}{%
  \ifkorrekturansicht\clearpage\fi
}
\makeatother

\IfFileExists{\jobname-pw.ind}{\input{\jobname-pw.ind}}{}

% Quellenangabe nur in der Leseansicht
\ifkorrekturansicht\else
% Fallback-Definitionen, falls die .tex-Datei \titel etc. nicht gesetzt hat
\providecommand{\titel}{}
\providecommand{\editorInnen}{}
\providecommand{\dateiname}{\jobname}

\vspace{3cm}

\vfill

\footnotesize
\textsc{Quelle}: \titel. Herausgegeben von {\editorInnen}. In: \emph{Arthur Schnitzler: Briefwechsel mit Autorinnen und Autoren}.
 Digitale Edition, https://schnitzler-briefe.acdh.oeaw.ac.at/{\dateiname}.html (Stand \today)
\fi

\end{document}


      