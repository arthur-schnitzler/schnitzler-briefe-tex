%% latex-leseansicht-vorspann.tex
%% Vorspann für die Leseansicht.
%% Lädt die gemeinsame Datei latex-vorspann.tex mit nicht gesetztem Schalter.

\newif\ifkorrekturansicht
\korrekturansichtfalse

\input{../tex-inputs/latex-vorspann}


\section[Arthur Schnitzler an Richard Beer-Hofmann, 19. 4. 1897]{L00666 Arthur Schnitzler an Richard Beer-Hofmann, 19. 4. 1897}
\nopagebreak\mylabel{L00666v}
\rehead{ }\normalsize\beginnumbering\briefempfaengerindex{Beer-Hofmann, Richard@\textsc{Beer-Hofmann, Richard}!zzzSchnitzler, Arthur@\emph{von Arthur Schnitzler}!1897-04-191@{19. 4. 1897}|(be}
\toendnotes[C]{\smallbreak\pagebreak[2]}
\correspDesc{Versand  durch Arthur Schnitzler am 19. 4. 1897 in Paris
\newline{}Erhalt  durch Richard Beer-Hofmann am 21. 4. 1897 in Wien}\toendnotes[C]{\smallbreak}
\Standort{YCGL, MSS 31.}
\physDesc{Brief, 1 Blatt, 4 Seiten, Kuvert, 1502 Zeichen
\newline{}Handschrift: schwarze Tinte, deutsche Kurrent
\newline{}Versand: 1) Stempel: »\nobreak{}\oindex{rue La Fayette@\textbf{rue La Fayette}, \emph{Straße}|pwk}Paris 51 R. Lafayette, 19 Avril 97, 5\textsuperscript{E}\nobreak{}«.   2) Stempel: »\nobreak{}\oindex{I., Innere Stadt@\textbf{I., Innere Stadt}, \emph{Verwaltungsgebiet}|pwk}Wien 1/1, 21 4. 97, 6–8½V., Bestellt\nobreak{}«. }
\buchAbdrucke{\weitereDrucke{Arthur Schnitzler, Richard Beer-Hofmann: \emph{Briefwechsel 1891–1931}. Herausgegeben von Konstanze Fliedl. Wien, Zürich: \emph{Europaverlag} 1992, S. 101.} }\toendnotes[C]{\smallbreak}\pstart{}{\pb}Herrn \textsc{Dr. Rich.
                     Beer-Hofmann}\pend{}\pstart{}Wien\oindex{Wien@\textbf{Wien}, \emph{Verwaltungsgebiet}|pw}\pend{}\pstart{}\textsc{I. Wollzeile 15}\oindex{Wien@\textbf{Wien}!I., Innere Stadt@\textbf{I., Innere Stadt}!Wollzeile 15 (»Berthahof«)@\textbf{Wollzeile 15 (»Berthahof«)}, \emph{Wohngebäude}|pw}.\pend{}\pstart{}\textsc{Autriche\oindex{Österreich@\textbf{Österreich}|pw}}\pend{}{\bigskip}\vspace{1em}
\pstart
           \raggedleft{}{\pb}Oſtermontag, 19. 4. 97.\pend
           \vspace{0.5em}
\pstart
           Lieber Richard, ich weiſss ja doch nicht, wa{\geminationn} ich endlich Luſt zu einem wirklichen Brief beko{\geminationm}en werde;{ }ſo{ }ſchreib ich Ihnen lieber dieſe paar Worte,
               um Ihnen zu{ }ſagen, daſs ich an Wien\oindex{Wien@\textbf{Wien}, \emph{Verwaltungsgebiet}|pw} mit heftigem
               Widerwillen, aber an \substVorne{}\textsuperscript{p}\substDazwischen{}e\substHinten{}in paar Menſchen, die ich nicht zu ne{\geminationn}en
               brauche, mit einer Art \introOben{}von\introOben{} nicht beſonders {\pb}ſchmerzlicher Sehnſucht denke. Es geht mir ganz gut;
               aber es iſt eine verwickelte Art von Wohlbefinden,{ }ſo daſs ich durchaus nicht
               verwundert bin, mich zu Zeiten{ }ſehr miſerabel zu befinden. Ich bin natürlich nicht
               allein und doch viel allein; bin im weſentlichen frei und doch zuweilen gebunden;
               freue mich{ }ſehr hier zu{ }ſein, weiſs aber nicht wieviel auf Rechnung der {\pb}Freude ko{\geminationm}t, nicht in Wien\oindex{Wien@\textbf{Wien}, \emph{Verwaltungsgebiet}|pw} zu{ }ſein. Viel hier intereſſirt mich – und doch
               hab ich bei den allgemeinern Eindrücken nicht das Gefühl, neues zu erfahren; es
               beſtätigt{ }ſich nur das meiſte. Ich glaube daſs ich gerne hier leben würde; man
               verſchwindet und iſt durchaus nicht beleidigt. Daſs Verkehr etwas{ }ſehr großes
               bedeuten kann,{ }ſpürt man hier; nicht {\pb}durch
               Multiplicationen ka{\geminationn} man das mit Wien\oindex{Wien@\textbf{Wien}, \emph{Verwaltungsgebiet}|pw} vergleichen; es iſt was andres; brutaler,{ }ſchöner und
               gemeiner. –\pend
           
\pstart
           Paul\pwindex{Goldmann, Paul 31.\,1.\,1865 Breslau – 25.\,9.\,1935 Wien@\textsc{Goldmann, Paul} (31.\,1.\,1865 Breslau – 25.\,9.\,1935 Wien), \emph{Schriftsteller, Journalist}|pw} iſt auf ein paar Tage nach Frankfurt\oindex{Frankfurt am Main@\textbf{Frankfurt am Main}, \emph{Hauptstadt}|pw}. Mir schreiben Sie nur weiter (nur
               weiter iſt gut) an die Adreſſe Pauls\pwindex{Goldmann, Paul 31.\,1.\,1865 Breslau – 25.\,9.\,1935 Wien@\textsc{Goldmann, Paul} (31.\,1.\,1865 Breslau – 25.\,9.\,1935 Wien), \emph{Schriftsteller, Journalist}|pw}, die ist
               jetzt \textsc{10 rue de la Bourse\oindex{rue de la Bourse@\textbf{rue de la Bourse}, \emph{Straße}|pw}}. – Ich wohne woanders, angenehm. Schreiben Sie mir was es Neues gibt. Aber{ }ſicher, bitte. Grüßen Sie Hugo\pwindex{Hofmannsthal, Hugo von 1.\,2.\,1874 Wien – 15.\,7.\,1929 Rodaun@\textsc{Hofmannsthal, Hugo von} (1.\,2.\,1874 Wien – 15.\,7.\,1929 Rodaun), \emph{Schriftsteller}|pw}, Leo\pwindex{Van-Jung, Leo 15.\,10.\,1866 Odessa – 2.\,7.\,1939 Riga@\textsc{Van-Jung, Leo} (15.\,10.\,1866 Odessa – 2.\,7.\,1939 Riga), \emph{Gesangspädagoge, Mathematiker}|pw}, Salten\pwindex{Salten, Felix 6.\,9.\,1869 Budapest – 8.\,10.\,1945 Zürich@\textsc{Salten, Felix} (6.\,9.\,1869 Budapest – 8.\,10.\,1945 Zürich), \emph{Schriftsteller, Journalist, Chefredakteur}|pw}, Schwarzk\pwindex{Schwarzkopf, Gustav 7.\,11.\,1853 Wien – 13.\,11.\,1939 ebd.@\textsc{Schwarzkopf, Gustav} (7.\,11.\,1853 Wien – 13.\,11.\,1939 ebd.), \emph{Schriftsteller}|pw}, Paula\pwindex{Beer-Hofmann, Paula 25.\,2.\,1879 Wien – 30.\,10.\,1939 Zürich@\textsc{Beer-Hofmann, Paula} (25.\,2.\,1879 Wien – 30.\,10.\,1939 Zürich)|pw} und \label{T_L00666-1v}\edtext{andere
                  \textsc{a discrétion}. Ihr \spacefill\mbox{Arthur.}}{\lemma{\textnormal{\emph{andere … Arthur.}}}\Cendnote{\textnormal{auf der ersten Seite unter dem
                  Text.}}}\label{T_L00666-1}\pend
           \selectlanguage{ngerman}\endnumbering\briefempfaengerindex{Beer-Hofmann, Richard@\textsc{Beer-Hofmann, Richard}!zzzSchnitzler, Arthur@\emph{von Arthur Schnitzler}!1897-04-191@{19. 4. 1897}|)be}\mylabel{L00666h}  \newcommand{\dateiname}{L00666}\newcommand{\titel}{Arthur Schnitzler an Richard Beer-Hofmann, 19. 4. 1897}\newcommand{\editorInnen}{Martin Anton Müller und Gerd-Hermann Susen}%% latex-leseansicht-abspann.tex
%% Abspann für die Leseansicht.
%% Der Schalter \ifkorrekturansicht ist bereits durch den Vorspann gesetzt.

%% latex-abspann.tex
%% Gemeinsamer Abspann für Korrekturansicht und Leseansicht.
%% Setzt den Schalter \ifkorrekturansicht voraus (gesetzt in den
%% einbindenden Dateien latex-korrekturansicht-abspann.tex bzw.
%% latex-leseansicht-abspann.tex).
%% ---------------------------------------------------------------

\normalsize

% Das esempio-Environment wird nur in der Leseansicht benötigt
\ifkorrekturansicht\else
\newenvironment{esempio}[3]%
{
    \vspace{1.5ex}
    \rlap{\underline{#1}}
    \par
    \setlength{\parindent}{0cm}
    \nopagebreak
    \leftskip=#2cm
    \rightskip=#3cm
}
{
    \par
}
\fi

\doendnotes{C}
\bigskip
\vfill

\clearpage

\footnotesize

\ifkorrekturansicht
  \lohead{\textsc{register}}
\fi

% theindex-Environment neu definieren ohne reledmac
\makeatletter
\renewenvironment{theindex}{%
  \ifkorrekturansicht
    \section*{\indexname}%
  \else
    \subsubsection*{Index der erwähnten Entitäten}%
  \fi
  \setlength{\parindent}{0pt}%
  \setlength{\parskip}{0pt plus 0.3pt}%
  \let\item\@idxitem
}{%
  \ifkorrekturansicht\clearpage\fi
}
\makeatother

\IfFileExists{\jobname-pw.ind}{\input{\jobname-pw.ind}}{}

% Quellenangabe nur in der Leseansicht
\ifkorrekturansicht\else
% Fallback-Definitionen, falls die .tex-Datei \titel etc. nicht gesetzt hat
\providecommand{\titel}{}
\providecommand{\editorInnen}{}
\providecommand{\dateiname}{\jobname}

\vspace{3cm}

\vfill

\footnotesize
\textsc{Quelle}: \titel. Herausgegeben von {\editorInnen}. In: \emph{Arthur Schnitzler: Briefwechsel mit Autorinnen und Autoren}.
 Digitale Edition, https://schnitzler-briefe.acdh.oeaw.ac.at/{\dateiname}.html (Stand \today)
\fi

\end{document}


