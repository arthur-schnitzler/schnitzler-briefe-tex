%% latex-leseansicht-vorspann.tex
%% Vorspann für die Leseansicht.
%% Lädt die gemeinsame Datei latex-vorspann.tex mit nicht gesetztem Schalter.

\newif\ifkorrekturansicht
\korrekturansichtfalse

\input{../tex-inputs/latex-vorspann}


\section[Robert Adam an Arthur Schnitzler, 26. 9. 1915]{L02219 Robert Adam an Arthur Schnitzler, 26. 9. 1915}
\nopagebreak\mylabel{L02219v}
\rehead{ }\normalsize\beginnumbering\briefempfaengerindex{Schnitzler, Arthur@\textsc{Schnitzler, Arthur}!zzzAdam, Robert@\emph{von Robert Adam}!1915-09-261@{26. 9. 1915}|(be}
\toendnotes[C]{\smallbreak\pagebreak[2]}
\correspDesc{Versand  durch Robert Adam am 26. 9. 1915 in Wien
\newline{}Erhalt  durch Arthur Schnitzler im Zeitraum [26. 9. 1915
                  – 30. 9. 1915?] in Wien}\toendnotes[C]{\smallbreak}
\Standort{DLA, A:Schnitzler, HS.NZ85.1.4230,11.}
\physDesc{Brief, 1 Blatt, 3 Seiten, 1641 Zeichen
\newline{}Handschrift: schwarze Tinte, deutsche Kurrent
\newline{}Schnitzler: 1) mit Bleistift beschriftet: »\textsc{Adam}«  2) mit rotem Buntstift drei Unterstreichungen}\Standort{Wien, Österreichische Nationalbibliothek, Cod.ser. 52.267, 109–110.}
\physDesc{Briefentwurf, maschinenschriftliche Abschrift, 2 Blätter, 2 Seiten, 1641 Zeichen
\newline{}Schreibmaschine}
\pstart
           \raggedleft{}{\pb}Wien\oindex{Wien@\textbf{Wien}, \emph{Verwaltungsgebiet}|pw}, am 26. September 1915\pend
           
\pstart{}Hochverehrter Herr Doktor!\pend\vspace{0.5em}
\pstart
           Es hat mir außerordentlich leid getan, Sie bei meinem Beſuche nicht anzutreffen. Ich
               wollte Ihnen die für mich{ }ſehr{ }ſchmerzliche Mitteilung machen, daß der Fiſcher’ſche Verlag\orgindex{S. Fischer Verlag@S. Fischer Verlag|pw} »weder einen inneren noch
               einen äußeren Anlaß« gefunden hat, die »Fremdenſzenen\pwindex{Adam, Robert 20.\,4.\,1877 Wien – 16.\,10.\,1961 Baden bei Wien@\textsc{Adam, Robert} (20.\,4.\,1877 Wien – 16.\,10.\,1961 Baden bei Wien), \emph{Schriftsteller, Richter}!Fremde@\strich\emph{Der Fremde}|pw}« zu übernehmen, und ich benütze jetzt den erſten Moment der
               Ruhe, den mir Amtsgeſchäft und die endloſen Mühen der Überſiedlung nach Wien\oindex{Wien@\textbf{Wien}, \emph{Verwaltungsgebiet}|pw} freilaſſen, Ihnen dieſe Nachricht, die Ihnen
               wohl{ }ſchon direkt zugekommen{ }ſein mag, zu übermitteln.\pend
           
\pstart
           Daß ich Ihnen für Ihre gütige Vermittlung außerordentlich dankbar bin und daß mich
               das {\pb}Intereſſe, das Sie als Einziger meinen Arbeiten
               entgegenbrachten, innerlich{ }ſtärkt und tröſtet, habe ich Ihnen{ }ſchon geſagt und ich
               werde nicht müde, Ihnen meinen Dank zu wiederholen.\pend
           
\pstart
           Ich bin seit einiger Zeit von Ziſtersdorf\oindex{Zistersdorf@\textbf{Zistersdorf}, \emph{Verwaltungsgebiet}|pw} nach
                  Wien\oindex{Wien@\textbf{Wien}, \emph{Verwaltungsgebiet}|pw} verſetzt, hier proviſoriſch dem Bezirksgericht Floridsdorf\orgindex{Bezirksgericht Wien Floridsdorf@Bezirksgericht Wien Floridsdorf|pw} zugeteilt und verbringe
               meine Tage auf der Elektriſchen (der Weg von Meidling\oindex{XII., Meidling@\textbf{XII., Meidling}, \emph{Verwaltungsgebiet}|pw} nach Floridsdorf\oindex{XXI., Floridsdorf@\textbf{XXI., Floridsdorf}, \emph{Verwaltungsgebiet}|pw} iſt{ }ſchrecklich
               weit!) und mit der Aburteilung größtenteils recht unintereſſanter Straffälle.\pend
           
\pstart
           Meine unglückſelige Arbeit verſchließe ich, indem ich dieſe Enttäuſchung, wie{ }ſo
               viele früher, geduldig trage, zu den andern nicht glücklicheren Arbeiten in die
               Schreibtiſchlade und warte auf beſſere Zeiten, um mit einer neuen Arbeit den Kampf um
               Geltung in einer Literatur wiederaufzunehmen, die von mir halt abſolut nichts wiſſen
               will. Daß ich die{\pb}ſen Kampf noch nicht aufgegeben
               habe, iſt mir einigermaßen{ }ſelbſt rätſelhaft. –\pend
           
\pstart
           Mit der Verſicherung meiner Dankbarkeit und Hochachtung Ihr{ }ſehr ergebener\pend
           \pstart \spacefill\mbox{D\textsuperscript{r}RAdam}\pend{}
\pstart
           \noindent{}Wien 12/\textsubscript{1} Meidlinger
                     Hauptſtraße 58\oindex{Wien@\textbf{Wien}!XII., Meidling@\textbf{XII., Meidling}!Meidlinger Hauptstraße@\textbf{Meidlinger Hauptstraße}, \emph{Straße}|pw}\pend
           \selectlanguage{ngerman}\endnumbering\briefempfaengerindex{Schnitzler, Arthur@\textsc{Schnitzler, Arthur}!zzzAdam, Robert@\emph{von Robert Adam}!1915-09-261@{26. 9. 1915}|)be}\mylabel{L02219h}  \newcommand{\dateiname}{L02219}\newcommand{\titel}{Robert Adam an Arthur Schnitzler, 26. 9. 1915}\newcommand{\editorInnen}{Martin Anton Müller und Gerd-Hermann Susen}%% latex-leseansicht-abspann.tex
%% Abspann für die Leseansicht.
%% Der Schalter \ifkorrekturansicht ist bereits durch den Vorspann gesetzt.

%% latex-abspann.tex
%% Gemeinsamer Abspann für Korrekturansicht und Leseansicht.
%% Setzt den Schalter \ifkorrekturansicht voraus (gesetzt in den
%% einbindenden Dateien latex-korrekturansicht-abspann.tex bzw.
%% latex-leseansicht-abspann.tex).
%% ---------------------------------------------------------------

\normalsize

% Das esempio-Environment wird nur in der Leseansicht benötigt
\ifkorrekturansicht\else
\newenvironment{esempio}[3]%
{
    \vspace{1.5ex}
    \rlap{\underline{#1}}
    \par
    \setlength{\parindent}{0cm}
    \nopagebreak
    \leftskip=#2cm
    \rightskip=#3cm
}
{
    \par
}
\fi

\doendnotes{C}
\bigskip
\vfill

\clearpage

\footnotesize

\ifkorrekturansicht
  \lohead{\textsc{register}}
\fi

% theindex-Environment neu definieren ohne reledmac
\makeatletter
\renewenvironment{theindex}{%
  \ifkorrekturansicht
    \section*{\indexname}%
  \else
    \subsubsection*{Index der erwähnten Entitäten}%
  \fi
  \setlength{\parindent}{0pt}%
  \setlength{\parskip}{0pt plus 0.3pt}%
  \let\item\@idxitem
}{%
  \ifkorrekturansicht\clearpage\fi
}
\makeatother

\IfFileExists{\jobname-pw.ind}{\input{\jobname-pw.ind}}{}

% Quellenangabe nur in der Leseansicht
\ifkorrekturansicht\else
% Fallback-Definitionen, falls die .tex-Datei \titel etc. nicht gesetzt hat
\providecommand{\titel}{}
\providecommand{\editorInnen}{}
\providecommand{\dateiname}{\jobname}

\vspace{3cm}

\vfill

\footnotesize
\textsc{Quelle}: \titel. Herausgegeben von {\editorInnen}. In: \emph{Arthur Schnitzler: Briefwechsel mit Autorinnen und Autoren}.
 Digitale Edition, https://schnitzler-briefe.acdh.oeaw.ac.at/{\dateiname}.html (Stand \today)
\fi

\end{document}


