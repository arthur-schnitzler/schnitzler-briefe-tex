%% latex-korrekturansicht-vorspann.tex
%% Vorspann für die Korrekturansicht.
%% Lädt die gemeinsame Datei latex-vorspann.tex mit gesetztem Schalter.

\newif\ifkorrekturansicht
\korrekturansichttrue

\input{../tex-inputs/latex-vorspann}


\section[Robert Adam an Arthur Schnitzler, 26. 9. 1915]{L02219 Robert Adam an Arthur Schnitzler, 26. 9. 1915}
\nopagebreak\mylabel{L02219v}
\rehead{ }\normalsize\beginnumbering\briefempfaengerindex{Schnitzler, Arthur@\textsc{Schnitzler, Arthur}!zzzAdam, Robert@\emph{von Robert Adam}!1915-09-261@{26. 9. 1915}|(be}
\toendnotes[C]{\smallbreak\pagebreak[2]}\Standort{DLA, A:Schnitzler, HS.NZ85.1.4230,11.}
\physDesc{Brief, 1 Blatt, 3 Seiten, 1641 Zeichen
\newline{}Handschrift: schwarze Tinte, deutsche Kurrent
\newline{}Schnitzler: 1) mit Bleistift beschriftet: »\textsc{Adam}«  2) mit rotem Buntstift drei Unterstreichungen}\Standort{Wien, Österreichische Nationalbibliothek, Cod.ser. 52.267, 109–110.}
\physDesc{Briefentwurf, maschinenschriftliche Abschrift2 Blätter, 2 Seiten, 1641 Zeichen
\newline{}Schreibmaschine}
\pstart
           \raggedleft{}{\pb}Wien\oindex{Wien@\textbf{Wien}, \emph{A.ADM2}|pw}, am 26. September 1915\pend
           
\pstart{}Hochverehrter Herr Doktor!\pend\vspace{0.5em}
\pstart
           Es hat mir außerordentlich leid getan, Sie bei meinem Beſuche nicht anzutreffen. Ich
               wollte Ihnen die für mich ſehr ſchmerzliche Mitteilung machen, daß der Fiſcher’ſche Verlag\orgindex{S. Fischer Verlag@S. Fischer Verlag|pw} »weder einen inneren noch
               einen äußeren Anlaß« gefunden hat, die »Fremdenſzenen\pwindex{Fremde@\emph{Der Fremde}|pw}« zu übernehmen, und ich benütze jetzt den erſten Moment der
               Ruhe, den mir Amtsgeſchäft und die endloſen Mühen der Überſiedlung nach Wien\oindex{Wien@\textbf{Wien}, \emph{A.ADM2}|pw} freilaſſen, Ihnen dieſe Nachricht, die Ihnen
               wohl ſchon direkt zugekommen ſein mag, zu übermitteln.\pend
           
\pstart
           Daß ich Ihnen für Ihre gütige Vermittlung außerordentlich dankbar bin und daß mich
               das {\pb}Intereſſe, das Sie als Einziger meinen Arbeiten
               entgegenbrachten, innerlich ſtärkt und tröſtet, habe ich Ihnen ſchon geſagt und ich
               werde nicht müde, Ihnen meinen Dank zu wiederholen.\pend
           
\pstart
           Ich bin seit einiger Zeit von Ziſtersdorf\oindex{Zistersdorf@\textbf{Zistersdorf}, \emph{A.ADM3}|pw} nach
                  Wien\oindex{Wien@\textbf{Wien}, \emph{A.ADM2}|pw} verſetzt, hier proviſoriſch dem Bezirksgericht Floridsdorf\orgindex{Bezirksgericht Wien Floridsdorf@Bezirksgericht Wien Floridsdorf|pw} zugeteilt und verbringe
               meine Tage auf der Elektriſchen (der Weg von Meidling\oindex{XII., Meidling@\textbf{XII., Meidling}, \emph{A.ADM3}|pw} nach Floridsdorf\oindex{XXI., Floridsdorf@\textbf{XXI., Floridsdorf}, \emph{A.ADM3}|pw} iſt ſchrecklich
               weit!) und mit der Aburteilung größtenteils recht unintereſſanter Straffälle.\pend
           
\pstart
           Meine unglückſelige Arbeit verſchließe ich, indem ich dieſe Enttäuſchung, wie ſo
               viele früher, geduldig trage, zu den andern nicht glücklicheren Arbeiten in die
               Schreibtiſchlade und warte auf beſſere Zeiten, um mit einer neuen Arbeit den Kampf um
               Geltung in einer Literatur wiederaufzunehmen, die von mir halt abſolut nichts wiſſen
               will. Daß ich die{\pb}ſen Kampf noch nicht aufgegeben
               habe, iſt mir einigermaßen ſelbſt rätſelhaft. –\pend
           
\pstart
           Mit der Verſicherung meiner Dankbarkeit und Hochachtung Ihr ſehr ergebener\pend
           \pstart \spacefill\mbox{D\textsuperscript{r}RAdam}\pend{}
\pstart
           \noindent{}Wien 12/\textsubscript{1} Meidlinger
                     Hauptſtraße 58\oindex{Meidlinger Hauptstrasse@\textbf{Meidlinger Hauptstraße}, \emph{Straße (K.STR)}|pw}\pend
           \selectlanguage{ngerman}\endnumbering\briefempfaengerindex{Schnitzler, Arthur@\textsc{Schnitzler, Arthur}!zzzAdam, Robert@\emph{von Robert Adam}!1915-09-261@{26. 9. 1915}|)be}\mylabel{L02219h}  \normalsize

\doendnotes{C}
\bigskip
\vfill

\clearpage

\footnotesize

\lohead{\textsc{register}}

% Definiere theindex-Environment komplett neu ohne reledmac
\makeatletter
\renewenvironment{theindex}{%
  \section*{\indexname}%
  \setlength{\parindent}{0pt}%
  \setlength{\parskip}{0pt plus 0.3pt}%
  \let\item\@idxitem
}{%
  \clearpage
}
\makeatother

\IfFileExists{\jobname-pw.ind}{\input{\jobname-pw.ind}}{}

\end{document}

      