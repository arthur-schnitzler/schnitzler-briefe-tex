%% latex-leseansicht-vorspann.tex
%% Vorspann für die Leseansicht.
%% Lädt die gemeinsame Datei latex-vorspann.tex mit nicht gesetztem Schalter.

\newif\ifkorrekturansicht
\korrekturansichtfalse

\input{../tex-inputs/latex-vorspann}


\section[Arthur Schnitzler an Berta Zuckerkandl, 24. 11. 1927]{L03966 Arthur Schnitzler an Berta Zuckerkandl, 24. 11. 1927}
\nopagebreak\mylabel{L03966v}
\rehead{ }\normalsize\beginnumbering\briefempfaengerindex{Zuckerkandl, Berta@\textsc{Zuckerkandl, Berta}!zzzSchnitzler, Arthur@\emph{von Arthur Schnitzler}!1927-11-241@{24. 11. 1927}|(be}
\toendnotes[C]{\smallbreak\pagebreak[2]}
\correspDesc{Versand  durch Arthur Schnitzler am 24. 11. 1927 in Wien
\newline{}Erhalt  durch Berta Zuckerkandl im Zeitraum [25. 11. 1927 – 29. 11. 1927?] in Paris}\toendnotes[C]{\smallbreak}
\Standort{DLA, HS.1985.1.2282.}
\physDesc{Brief, Durchschlag, 1 Blatt, 1 Seite, 320 Zeichen
\newline{}Schreibmaschine
\newline{}Handschrift: roter Buntstift, lateinische Kurrent (\noindent{}beschriftet: »\uline{Zuckerkandl}« und »\uline{Paris}«, fünf Unterstreichungen)}\toendnotes[C]{\smallbreak}
\pstart
           \raggedleft{}{\pb}24. 11. 1927.\pend
           
\pstart{}Liebe und verehrte Frau Hofrätin.\pend\vspace{0.5em}
\pstart
           Zu Ihrer Information lege ich eine \label{K_L03966-1v}\edtext{Abschrift
               des Briefes}{\lemma{\textnormal{\emph{Abschrift
               des Briefes}}}\Cendnote{\textnormal{Arthur Schnitzler an Robert Blum\pwindex{Blum, Robert 17.\,4.\,1881 Wien – 3.\,7.\,1952 Paris@\textsc{Blum, Robert} (17.\,4.\,1881 Wien – 3.\,7.\,1952 Paris), \emph{Schriftsteller, Journalist, Theaterleiter}|pwk}, 24. 11. 1927, \emph{Deutsches Literaturarchiv Marbach},
                     HS.1985.1.413.
               }}}\label{K_L03966-1} bei, den ich eben durch Fräulein Reichelt\pwindex{Reichelt @\textsc{Reichelt}, \emph{Sekretärin}|pw}
               an seine\pwindex{Blum, Robert 17.\,4.\,1881 Wien – 3.\,7.\,1952 Paris@\textsc{Blum, Robert} (17.\,4.\,1881 Wien – 3.\,7.\,1952 Paris), \emph{Schriftsteller, Journalist, Theaterleiter}|pwv} Adresse befördern lasse. Hoffentlich höre
      ich bald von Ihnen. Montag fahre ich auf zehn Tage
               \label{K_L03966-2v}\edtext{nach Berlin\oindex{Berlin@\textbf{Berlin}, \emph{Hauptstadt}|pw}}{\lemma{\textnormal{\emph{nach Berlin}}}\Cendnote{\textnormal{Schnitzler hielt sich vom 29. 11. 1927 bis zum
                  9. 12. 1927 in
                  Berlin\oindex{Berlin@\textbf{Berlin}, \emph{Hauptstadt}|pwk} auf.}}}\label{K_L03966-2}.\pend
           
\pstart
           Herzlichst grüssend{\\[\baselineskip]} Ihr\pend
           \leftskip=0em{}{\vspace{1\baselineskip}}
\pstart
           \noindent{}1 \label{K_L03966-3v}\edtext{Beilage}{\lemma{\textnormal{\emph{Beilage}}}\Cendnote{\textnormal{Abschrift eines Briefes an Robert Blum\pwindex{Blum, Robert 17.\,4.\,1881 Wien – 3.\,7.\,1952 Paris@\textsc{Blum, Robert} (17.\,4.\,1881 Wien – 3.\,7.\,1952 Paris), \emph{Schriftsteller, Journalist, Theaterleiter}|pwk}, siehe oben.}}}\label{K_L03966-3}.\pend
           
\pstart
           Frau Hofrätin Bertha Zuckerkandl,{\\}Paris\oindex{Paris@\textbf{Paris}, \emph{Hauptstadt}|pw}.\pend
           \selectlanguage{ngerman}\endnumbering\briefempfaengerindex{Zuckerkandl, Berta@\textsc{Zuckerkandl, Berta}!zzzSchnitzler, Arthur@\emph{von Arthur Schnitzler}!1927-11-241@{24. 11. 1927}|)be}\mylabel{L03966h}
\begin{anhang}
\end{anhang}\newcommand{\dateiname}{L03966}\newcommand{\titel}{Arthur Schnitzler an Berta Zuckerkandl, 24. 11. 1927}\newcommand{\editorInnen}{Herausgegeben von Jahnke, SelmaMüller, Martin Anton}%% latex-leseansicht-abspann.tex
%% Abspann für die Leseansicht.
%% Der Schalter \ifkorrekturansicht ist bereits durch den Vorspann gesetzt.

%% latex-abspann.tex
%% Gemeinsamer Abspann für Korrekturansicht und Leseansicht.
%% Setzt den Schalter \ifkorrekturansicht voraus (gesetzt in den
%% einbindenden Dateien latex-korrekturansicht-abspann.tex bzw.
%% latex-leseansicht-abspann.tex).
%% ---------------------------------------------------------------

\normalsize

% Das esempio-Environment wird nur in der Leseansicht benötigt
\ifkorrekturansicht\else
\newenvironment{esempio}[3]%
{
    \vspace{1.5ex}
    \rlap{\underline{#1}}
    \par
    \setlength{\parindent}{0cm}
    \nopagebreak
    \leftskip=#2cm
    \rightskip=#3cm
}
{
    \par
}
\fi

\doendnotes{C}
\bigskip
\vfill

\clearpage

\footnotesize

\ifkorrekturansicht
  \lohead{\textsc{register}}
\fi

% theindex-Environment neu definieren ohne reledmac
\makeatletter
\renewenvironment{theindex}{%
  \ifkorrekturansicht
    \section*{\indexname}%
  \else
    \subsubsection*{Index der erwähnten Entitäten}%
  \fi
  \setlength{\parindent}{0pt}%
  \setlength{\parskip}{0pt plus 0.3pt}%
  \let\item\@idxitem
}{%
  \ifkorrekturansicht\clearpage\fi
}
\makeatother

\IfFileExists{\jobname-pw.ind}{\input{\jobname-pw.ind}}{}

% Quellenangabe nur in der Leseansicht
\ifkorrekturansicht\else
% Fallback-Definitionen, falls die .tex-Datei \titel etc. nicht gesetzt hat
\providecommand{\titel}{}
\providecommand{\editorInnen}{}
\providecommand{\dateiname}{\jobname}

\vspace{3cm}

\vfill

\footnotesize
\textsc{Quelle}: \titel. Herausgegeben von {\editorInnen}. In: \emph{Arthur Schnitzler: Briefwechsel mit Autorinnen und Autoren}.
 Digitale Edition, https://schnitzler-briefe.acdh.oeaw.ac.at/{\dateiname}.html (Stand \today)
\fi

\end{document}


