%% latex-leseansicht-vorspann.tex
%% Vorspann für die Leseansicht.
%% Lädt die gemeinsame Datei latex-vorspann.tex mit nicht gesetztem Schalter.

\newif\ifkorrekturansicht
\korrekturansichtfalse

\input{../tex-inputs/latex-vorspann}


\section[Richard Beer-Hofmann an Arthur Schnitzler, 26. 3.  1903]{L01279 Richard Beer-Hofmann an Arthur Schnitzler, 26. 3.  1903}
\nopagebreak\mylabel{L01279v}
\rehead{ }\normalsize\beginnumbering\briefempfaengerindex{Schnitzler, Arthur@\textsc{Schnitzler, Arthur}!zzzBeer-Hofmann, Richard@\emph{von Richard Beer-Hofmann}!1903-03-261@{26. 3. 1903}|(be}
\toendnotes[C]{\smallbreak\pagebreak[2]}
\correspDesc{Versand  durch Richard Beer-Hofmann am 26. 3. 1903 in Perchtoldsdorf
\newline{}Erhalt  durch Arthur Schnitzler am 26. 3. 1903 in Wien}\toendnotes[C]{\smallbreak}
\Standort{CUL, Schnitzler, B 8.}
\physDesc{Postkarte, 291 Zeichen
\newline{}Handschrift: schwarze Tinte, lateinische Kurrent
\newline{}Versand: 1) Stempel: »\nobreak{}\oindex{Perchtoldsdorf@\textbf{Perchtoldsdorf}, \emph{Verwaltungsgebiet}|pwk}Perchtoldsdorf, 26. 3. 03, 10-12V\nobreak{}«.   2) Stempel: »\nobreak{}\oindex{IX., Alsergrund@\textbf{IX., Alsergrund}, \emph{Verwaltungsgebiet}|pwk}9/3 Wien 72, 26. 3. 03, 7.N, Bestellt\nobreak{}«.  3) die Adresse von Hofmannsthal\pwindex{Hofmannsthal, Hugo von 1.\,2.\,1874 Wien – 15.\,7.\,1929 Rodaun@\textsc{Hofmannsthal, Hugo von} (1.\,2.\,1874 Wien – 15.\,7.\,1929 Rodaun), \emph{Schriftsteller}|pw} geschrieben
\newline{}Ordnung: mit Bleistift von unbekannter Hand nummeriert:
                                    »178« }
\buchAbdrucke{\weitereDrucke{Arthur Schnitzler, Richard Beer-Hofmann: \emph{Briefwechsel 1891–1931}. Herausgegeben von Konstanze Fliedl. Wien, Zürich: \emph{Europaverlag} 1992, S. 162.} }\toendnotes[C]{\smallbreak}\pstart{}{\pb}Herrn Dr. Arthur
                  Schnitzler\pend{}\pstart{}Schleiermacher\pwindex{Schleiermacher, Friedrich 21.\,11.\,1768 Breslau – 12.\,2.\,1834 Berlin@\textsc{Schleiermacher, Friedrich} (21.\,11.\,1768 Breslau – 12.\,2.\,1834 Berlin), \emph{Philosoph}|pwv}\pwindex{Schnitzler, Arthur 15.\,5.\,1862 Wien – 21.\,10.\,1931 ebd.@\textsc{Schnitzler, Arthur} (15.\,5.\,1862 Wien – 21.\,10.\,1931 ebd.), \emph{Schriftsteller, Mediziner}!Schleier der Beatrice. Schauspiel in fünf Akten@\strich\emph{Der Schleier der Beatrice. Schauspiel in fünf Akten}|pwv}\pend{}\pstart{}Wien\oindex{Wien@\textbf{Wien}, \emph{Verwaltungsgebiet}|pw}\pend{}\pstart{}IX. Franckgasse 1\oindex{Wien@\textbf{Wien}!IX., Alsergrund@\textbf{IX., Alsergrund}!Frankgasse 1@\textbf{Frankgasse 1}, \emph{Wohngebäude}|pw}.\pend{}{\bigskip}\vspace{1em}
\pstart
           \noindent{}{\pb}Lieber Arthur! Nr\textsuperscript{o} 12 rechts, II. Stock
                  Burg\oindex{Wien@\textbf{Wien}!I., Innere Stadt@\textbf{I., Innere Stadt}!Burgtheater@\textbf{Burgtheater}, \emph{Theater}|pw}, \uline{Lear}\pwindex{\textcolor{red}{\textsuperscript{XXXX indx1}}!König Lear. Trauerspiel in fünf Aufzügen@\strich\emph{König Lear. Trauerspiel in fünf Aufzügen}|pw}, sind \uline{wir} (\label{K_L01279-1v}\edtext{Vers}{\lemma{\textnormal{\emph{Vers}}}\Cendnote{\textnormal{Gemeint ist, dass sich
                   »Lear« und »wir« reimen.}}}\label{K_L01279-1}) Hugo\pwindex{Hofmannsthal, Hugo von 1.\,2.\,1874 Wien – 15.\,7.\,1929 Rodaun@\textsc{Hofmannsthal, Hugo von} (1.\,2.\,1874 Wien – 15.\,7.\,1929 Rodaun), \emph{Schriftsteller}|pw}{ }\introOben{}mit Frau\pwindex{Hofmannsthal, Gertrude von 16.\,3.\,1880 Wien – 9.\,11.\,1959 Paddington@\textsc{Hofmannsthal, Gertrude von} (16.\,3.\,1880 Wien – 9.\,11.\,1959 Paddington)|pwv}\introOben{} u. ich \introOben{}mit Frau\pwindex{Beer-Hofmann, Paula 25.\,2.\,1879 Wien – 30.\,10.\,1939 Zürich@\textsc{Beer-Hofmann, Paula} (25.\,2.\,1879 Wien – 30.\,10.\,1939 Zürich)|pwv}\introOben{} am Samstag. Vielleicht können Sie – als Gast – mitko{\geminationm}en?\pend
           
\pstart
           Man sieht und spricht Sie ohnehin vielzuwenig. (Für mich – und Sie.).\pend
           
\pstart
           Herzlich{\\[\baselineskip]}Ihr{\\[\baselineskip]}\spacefill\mbox{Richard}\pend
           \leftskip=0em{}\selectlanguage{ngerman}\endnumbering\briefempfaengerindex{Schnitzler, Arthur@\textsc{Schnitzler, Arthur}!zzzBeer-Hofmann, Richard@\emph{von Richard Beer-Hofmann}!1903-03-261@{26. 3. 1903}|)be}\mylabel{L01279h}  \newcommand{\dateiname}{L01279}\newcommand{\titel}{Richard Beer-Hofmann an Arthur Schnitzler, 26. 3.  1903}\newcommand{\editorInnen}{Martin Anton Müller und Gerd-Hermann Susen}%% latex-leseansicht-abspann.tex
%% Abspann für die Leseansicht.
%% Der Schalter \ifkorrekturansicht ist bereits durch den Vorspann gesetzt.

%% latex-abspann.tex
%% Gemeinsamer Abspann für Korrekturansicht und Leseansicht.
%% Setzt den Schalter \ifkorrekturansicht voraus (gesetzt in den
%% einbindenden Dateien latex-korrekturansicht-abspann.tex bzw.
%% latex-leseansicht-abspann.tex).
%% ---------------------------------------------------------------

\normalsize

% Das esempio-Environment wird nur in der Leseansicht benötigt
\ifkorrekturansicht\else
\newenvironment{esempio}[3]%
{
    \vspace{1.5ex}
    \rlap{\underline{#1}}
    \par
    \setlength{\parindent}{0cm}
    \nopagebreak
    \leftskip=#2cm
    \rightskip=#3cm
}
{
    \par
}
\fi

\doendnotes{C}
\bigskip
\vfill

\clearpage

\footnotesize

\ifkorrekturansicht
  \lohead{\textsc{register}}
\fi

% theindex-Environment neu definieren ohne reledmac
\makeatletter
\renewenvironment{theindex}{%
  \ifkorrekturansicht
    \section*{\indexname}%
  \else
    \subsubsection*{Index der erwähnten Entitäten}%
  \fi
  \setlength{\parindent}{0pt}%
  \setlength{\parskip}{0pt plus 0.3pt}%
  \let\item\@idxitem
}{%
  \ifkorrekturansicht\clearpage\fi
}
\makeatother

\IfFileExists{\jobname-pw.ind}{\input{\jobname-pw.ind}}{}

% Quellenangabe nur in der Leseansicht
\ifkorrekturansicht\else
% Fallback-Definitionen, falls die .tex-Datei \titel etc. nicht gesetzt hat
\providecommand{\titel}{}
\providecommand{\editorInnen}{}
\providecommand{\dateiname}{\jobname}

\vspace{3cm}

\vfill

\footnotesize
\textsc{Quelle}: \titel. Herausgegeben von {\editorInnen}. In: \emph{Arthur Schnitzler: Briefwechsel mit Autorinnen und Autoren}.
 Digitale Edition, https://schnitzler-briefe.acdh.oeaw.ac.at/{\dateiname}.html (Stand \today)
\fi

\end{document}


