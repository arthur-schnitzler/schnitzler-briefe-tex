%% latex-korrekturansicht-vorspann.tex
%% Vorspann für die Korrekturansicht.
%% Lädt die gemeinsame Datei latex-vorspann.tex mit gesetztem Schalter.

\newif\ifkorrekturansicht
\korrekturansichttrue

\input{../tex-inputs/latex-vorspann}


\section[Richard Beer-Hofmann an Arthur Schnitzler, 26. 3.  1903]{L01279 Richard Beer-Hofmann an Arthur Schnitzler, 26. 3.  1903}
\nopagebreak\mylabel{L01279v}
\rehead{ }\normalsize\beginnumbering\briefempfaengerindex{Schnitzler, Arthur@\textsc{Schnitzler, Arthur}!zzzBeer-Hofmann, Richard@\emph{von Richard Beer-Hofmann}!1903-03-261@{26. 3. 1903}|(be}
\toendnotes[C]{\smallbreak\pagebreak[2]}\Standort{CUL, Schnitzler, B 8.}
\physDesc{Postkarte, 291 Zeichen
\newline{}Handschrift: schwarze Tinte, lateinische Kurrent
\newline{}Versand: 1) Stempel: »\nobreak{}\oindex{Perchtoldsdorf@\textbf{Perchtoldsdorf}, \emph{A.ADM3}|pwk}Perchtoldsdorf, 26. 3. 03, 10-12V\nobreak{}«.   2) Stempel: »\nobreak{}\oindex{IX., Alsergrund@\textbf{IX., Alsergrund}, \emph{A.ADM3}|pwk}9/3 Wien 72, 26. 3. 03, 7.N, Bestellt\nobreak{}«.  3) die Adresse von Hofmannsthal\pwindex{Hofmannsthal, Hugo von 1874-02-01 – 1929-07-15@\textsc{Hofmannsthal, Hugo von} (1874-02-01 – 1929-07-15), \emph{Schriftsteller/Schriftstellerin}|pw} geschrieben
\newline{}Ordnung: mit Bleistift von unbekannter Hand nummeriert:
                                    »178« }
\buchAbdrucke{\weitereDrucke{Arthur Schnitzler, Richard Beer-Hofmann: \emph{Briefwechsel 1891–1931}. Wien, Zürich: \emph{Europaverlag} 1992, S. 162.} }\toendnotes[C]{\smallbreak}\pstart{}{\pb}Herrn Dr. Arthur
                  Schnitzler\pend{}\pstart{}Schleiermacher\pwindex{Schleiermacher, Friedrich 21.11.1768 – 12.02.1834@\textsc{Schleiermacher, Friedrich} (21.11.1768 – 12.02.1834), \emph{Philosoph/Philosophin}|pwv}\pwindex{Schleier der Beatrice. Schauspiel in fuenf Akten@\emph{Der Schleier der Beatrice. Schauspiel in fünf Akten}|pwv}\pend{}\pstart{}Wien\oindex{Wien@\textbf{Wien}, \emph{A.ADM2}|pw}\pend{}\pstart{}IX. Franckgasse 1\oindex{Frankgasse 1@\textbf{Frankgasse 1}, \emph{Wohngebäude (K.WHS)}|pw}.\pend{}{\bigskip}\vspace{1em}
\pstart
           \noindent{}{\pb}Lieber Arthur! Nr\textsuperscript{o} 12 rechts, II. Stock
                  Burg\oindex{Burgtheater@\textbf{Burgtheater}, \emph{S.THTR}|pw}, \uline{Lear}\pwindex{Koenig Lear. Trauerspiel in fuenf Aufzuegen@\emph{König Lear. Trauerspiel in fünf Aufzügen}|pw}, sind \uline{wir} (\label{K_L01279-1v}\edtext{Vers}{\lemma{\textnormal{\emph{Vers}}}\Cendnote{\textnormal{Gemeint ist, dass sich
                   »Lear« und »wir« reimen.}}}\label{K_L01279-1}) Hugo\pwindex{Hofmannsthal, Hugo von 1874-02-01 – 1929-07-15@\textsc{Hofmannsthal, Hugo von} (1874-02-01 – 1929-07-15), \emph{Schriftsteller/Schriftstellerin}|pw}{ }\introOben{}mit Frau\pwindex{Hofmannsthal, Gertrude von 16.03.1880 – 09.11.1959@\textsc{Hofmannsthal, Gertrude von} (16.03.1880 – 09.11.1959)|pwv}\introOben{} u. ich \introOben{}mit Frau\pwindex{Beer-Hofmann, Paula 25.02.1879 – 30.10.1939@\textsc{Beer-Hofmann, Paula} (25.02.1879 – 30.10.1939)|pwv}\introOben{} am Samstag. Vielleicht können Sie – als Gast – mitko{\geminationm}en?\pend
           
\pstart
           Man sieht und spricht Sie ohnehin vielzuwenig. (Für mich – und Sie.).\pend
           
\pstart
           Herzlich{\\[\baselineskip]}Ihr{\\[\baselineskip]}\spacefill\mbox{Richard}\pend
           \leftskip=0em{}\selectlanguage{ngerman}\endnumbering\briefempfaengerindex{Schnitzler, Arthur@\textsc{Schnitzler, Arthur}!zzzBeer-Hofmann, Richard@\emph{von Richard Beer-Hofmann}!1903-03-261@{26. 3. 1903}|)be}\mylabel{L01279h}  \normalsize

\doendnotes{C}
\bigskip
\vfill

\clearpage

\footnotesize

\lohead{\textsc{register}}

% Definiere theindex-Environment komplett neu ohne reledmac
\makeatletter
\renewenvironment{theindex}{%
  \section*{\indexname}%
  \setlength{\parindent}{0pt}%
  \setlength{\parskip}{0pt plus 0.3pt}%
  \let\item\@idxitem
}{%
  \clearpage
}
\makeatother

\IfFileExists{\jobname-pw.ind}{\input{\jobname-pw.ind}}{}

\end{document}

      