%% latex-korrekturansicht-vorspann.tex
%% Vorspann für die Korrekturansicht.
%% Lädt die gemeinsame Datei latex-vorspann.tex mit gesetztem Schalter.

\newif\ifkorrekturansicht
\korrekturansichttrue

\input{../tex-inputs/latex-vorspann}


\section[Richard Beer-Hofmann an Arthur Schnitzler, 5. 10. 1904]{L01452 Richard Beer-Hofmann an Arthur Schnitzler, 5. 10. 1904}
\nopagebreak\mylabel{L01452v}
\rehead{ }\normalsize\beginnumbering\briefempfaengerindex{Schnitzler, Arthur@\textsc{Schnitzler, Arthur}!zzzBeer-Hofmann, Richard@\emph{von Richard Beer-Hofmann}!1904-10-051@{5. 10. 1904}|(be}
\toendnotes[C]{\smallbreak\pagebreak[2]}\Standort{CUL, Schnitzler, B 8.}
\physDesc{Bildpostkarte, 92 Zeichen
\newline{}Handschrift: schwarze Tinte, lateinische Kurrent
\newline{}Versand: Stempel: »\nobreak{}\oindex{Amsterdam@\textbf{Amsterdam}, \emph{P.PPLC}|pwk}Ams{[}terdam{]}, 6 Oct 04, 12–6\nobreak{}«.  
\newline{}Ordnung: mit Bleistift von unbekannter Hand nummeriert:
                                    »192« }\pstart{}{\pb}Herrn\pend{}\pstart{}Arthur Schnitzler\pend{}\pstart{}Wien\oindex{Wien@\textbf{Wien}, \emph{A.ADM2}|pw}\pend{}\pstart{}XVIII. Spöttelgasse 7\oindex{Edmund-Weiss-Gasse 7@\textbf{Edmund-Weiß-Gasse 7}, \emph{Wohngebäude (K.WHS)}|pw}\pend{}\pstart{}Austria\oindex{Oesterreich@\textbf{Österreich}, \emph{A.PCLI}|pw}\pend{}{\bigskip}
\pstart
           \noindent{}\centering{}{\pb}\textcolor{gray}{\textbf{Rembrandt\pwindex{Rembrandt van Rijn 15.07.1606 – 04.10.1669@\textsc{Rembrandt van Rijn} (15.07.1606 – 04.10.1669), \emph{Maler/Malerin}|pw}.}}\pend
           
\pstart
           \centering{}\textcolor{gray}{\textbf{De Staalmesters\pwindex{De Staalmesters@\emph{De Staalmesters}|pw}.}}\pend
           \vspace{1em}
\pstart
           {\pb}Amsterdam\oindex{Amsterdam@\textbf{Amsterdam}, \emph{P.PPLC}|pw}{ }5/X 04\pend
           \vspace{0.5em}
\pstart
           Herzliche Grüße \spacefill\mbox{Richard}\pend
           \selectlanguage{ngerman}\endnumbering\briefempfaengerindex{Schnitzler, Arthur@\textsc{Schnitzler, Arthur}!zzzBeer-Hofmann, Richard@\emph{von Richard Beer-Hofmann}!1904-10-051@{5. 10. 1904}|)be}\mylabel{L01452h}  \normalsize

\doendnotes{C}
\bigskip
\vfill

\clearpage

\footnotesize

\lohead{\textsc{register}}

% Definiere theindex-Environment komplett neu ohne reledmac
\makeatletter
\renewenvironment{theindex}{%
  \section*{\indexname}%
  \setlength{\parindent}{0pt}%
  \setlength{\parskip}{0pt plus 0.3pt}%
  \let\item\@idxitem
}{%
  \clearpage
}
\makeatother

\IfFileExists{\jobname-pw.ind}{\input{\jobname-pw.ind}}{}

\end{document}

      