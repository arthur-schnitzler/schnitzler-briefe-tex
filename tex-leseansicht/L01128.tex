%% latex-korrekturansicht-vorspann.tex
%% Vorspann für die Korrekturansicht.
%% Lädt die gemeinsame Datei latex-vorspann.tex mit gesetztem Schalter.

\newif\ifkorrekturansicht
\korrekturansichttrue

\input{../tex-inputs/latex-vorspann}


\section[Hugo von Hofmannsthal an Arthur Schnitzler, 14. 6. 1901]{L01128 Hugo von Hofmannsthal an Arthur Schnitzler, 14. 6. 1901}
\nopagebreak\mylabel{L01128v}
\rehead{ }\normalsize\beginnumbering\briefempfaengerindex{Schnitzler, Arthur@\textsc{Schnitzler, Arthur}!zzzHofmannsthal, Hugo von@\emph{von Hugo von Hofmannsthal}!1901-06-141@{14. 6. 1901}|(be}
\toendnotes[C]{\smallbreak\pagebreak[2]}\Standort{CUL, Schnitzler, B 43.}
\physDesc{Bildpostkarte, 216 Zeichen
\newline{}Handschrift: schwarze Tinte, deutsche Kurrent
\newline{}Versand: 1) Stempel: »\nobreak{}\oindex{Lido@\textbf{Lido}, \emph{P.PPL}|pwk}Grand Hôtel des Bains Dépendance et
                                       Châlets Lido – Venise F. Schlössing directeur\nobreak{}«.   2) Stempel: »\nobreak{}\oindex{Lido@\textbf{Lido}, \emph{P.PPL}|pwk}Elisabetta di Lido
                                       (Venezia), 14 6 01\nobreak{}«.  3) Stempel: »\nobreak{}16. 6. 01, 9.V\nobreak{}«. 
\newline{}Schnitzler: mit Bleistift datiert: »14/6 901« 
\newline{}Ordnung: mit Bleistift von unbekannter Hand nummeriert:
                                    »174« }
\buchAbdrucke{\weitereDrucke{Hugo von Hofmannsthal, Arthur Schnitzler: \emph{Briefwechsel}. Frankfurt am Main: \emph{S. Fischer} 1964, S. 147.} }\toendnotes[C]{\smallbreak}\pstart{}{\pb}\textsc{Herrn D\textsuperscript{r} Arthur Schnitzler}\pend{}\pstart{}\textsc{Wien}\oindex{Wien@\textbf{Wien}, \emph{A.ADM2}|pw}\pend{}\pstart{}\textsc{IX. Frankgasse 1}\oindex{Frankgasse 1@\textbf{Frankgasse 1}, \emph{Wohngebäude (K.WHS)}|pw}. \pend{}\pstart{}\textsc{Austria\oindex{Oesterreich@\textbf{Österreich}, \emph{A.PCLI}|pw}}\pend{}{\bigskip}
\pstart
           \noindent{}\centering{}{\pb}\textcolor{gray}{\textbf{Bagni di Lido\oindex{Lido@\textbf{Lido}, \emph{P.PPL}|pw}}}\pend
           
\pstart
           \centering{}\textcolor{gray}{\textbf{Venezia\oindex{Venedig@\textbf{Venedig}, \emph{P.PPLA}|pw}}}\pend
           \vspace{1em}
\pstart
           \noindent{}{\pb}Wir thun hier\oindex{Lido@\textbf{Lido}, \emph{P.PPL}|pwv}{ }See-baden und ich leſe dazu die natürliche Tochter\pwindex{natuerliche Tochter@\emph{Die natürliche Tochter}|pw}. Hoffentlich liegt in Rodaun\oindex{Rodaun@\textbf{Rodaun}, \emph{A.ADM4}|pw} in 8 Tagen eine Zeile von Ihnen. Viele Grüße von Gerty\pwindex{Hofmannsthal, Gertrude von 16.03.1880 – 09.11.1959@\textsc{Hofmannsthal, Gertrude von} (16.03.1880 – 09.11.1959)|pw}. Von Herzen Ihr\pend
           \pstart \spacefill\mbox{Hugo}\pend{}\selectlanguage{ngerman}\endnumbering\briefempfaengerindex{Schnitzler, Arthur@\textsc{Schnitzler, Arthur}!zzzHofmannsthal, Hugo von@\emph{von Hugo von Hofmannsthal}!1901-06-141@{14. 6. 1901}|)be}\mylabel{L01128h}  \normalsize

\doendnotes{C}
\bigskip
\vfill

\clearpage

\footnotesize

\lohead{\textsc{register}}

% Definiere theindex-Environment komplett neu ohne reledmac
\makeatletter
\renewenvironment{theindex}{%
  \section*{\indexname}%
  \setlength{\parindent}{0pt}%
  \setlength{\parskip}{0pt plus 0.3pt}%
  \let\item\@idxitem
}{%
  \clearpage
}
\makeatother

\IfFileExists{\jobname-pw.ind}{\input{\jobname-pw.ind}}{}

\end{document}

      