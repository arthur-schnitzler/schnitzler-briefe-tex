%% latex-leseansicht-vorspann.tex
%% Vorspann für die Leseansicht.
%% Lädt die gemeinsame Datei latex-vorspann.tex mit nicht gesetztem Schalter.

\newif\ifkorrekturansicht
\korrekturansichtfalse

\input{../tex-inputs/latex-vorspann}


         \renewcommand{\erwaehnteInstitutionen}{Institutionen: Burgtheater}
         \renewcommand{\erwaehnteOrte}{Orte: Himmelpfortgasse, Wien}
         \renewcommand{\erwaehnteWerke}{Werke: Fuchs. Schauspiel in einem Act}
               \section[Hugo von Hofmannsthal an Arthur Schnitzler, {[}13. 3. 1907{]}]{ Hugo von Hofmannsthal an Arthur Schnitzler, {[}13. 3. 1907{]}}\nopagebreak\mylabel{v}\rehead{ }\begin{ledgroupsized}[t]{13cm}\normalsize\beginnumbering \toendnotes[C]{\smallbreak\pagebreak[2]} \Standort{CUL, Schnitzler, B 43.}
\physDesc{Brief, 1 Blatt, 2 Seiten, 260 Zeichen
\newline{}Handschrift: schwarze Tinte, deutsche Kurrent
\newline{}Schnitzler: mit Bleistift datiert: »13/3 907« 
\newline{}Ordnung: 1) mit Bleistift von unbekannter Hand nummeriert: »\strikeout{268}«  2) mit Bleistift von unbekannter Hand nummeriert:
                                    »272«}\buchAbdrucke{\weitereDrucke{Hugo von Hofmannsthal, Arthur Schnitzler: \emph{Briefwechsel}. Hg. Therese Nickl und Heinrich Schnitzler. Frankfurt am Main: \emph{S. Fischer} 1964, S. 227.} }\toendnotes[C]{\smallbreak}\pstart{}{\pb}mein lieber Arthur\pend\pstart
           bitte ſchreiben Sie mir \label{T_L01663-1v}\edtext{\uuline{gleich}}{\lemma{\textnormal{\emph{gleich}}}\Cendnote{\textnormal{nachträglich mit Bleistift
                  eingefügt}}}\label{T_L01663-1h}{ }\uline{pneumatiſch} nach \textsc{Wien\oindex{Wien@\textbf{Wien}|pw}{ }I Hi{\geminationm}elpfortgasse 17\oindex{Himmelpfortgasse@\textbf{Himmelpfortgasse}|pw}} ob das Burgtheater\orgindex{Burgtheater@Burgtheater|pw} auf das Stück »Fuchs\pwindex{\textcolor{red}{\textsuperscript{XXXX1 indx}}!Fuchs. Schauspiel in einem Act1900@\strich\emph{Fuchs. Schauspiel in einem Act} {[}1900{]}|pw}« von mir, das es ſeit 5 Jahren nicht
               geſpielt hat, {\pb}noch Rechte haben
               kann. Dieſes Recht kann doch nicht auf ewige Zeiten gelten.\pend
           \pstart
           Ihr{\\[\baselineskip]}\spacefill\mbox{Hugo}\pend
           \leftskip=0em{}
         
         \endnumbering\mylabel{h}\end{ledgroupsized}  \newcommand{\dateiname}{L01663}\newcommand{\titel}{Hugo von Hofmannsthal an Arthur Schnitzler, [13. 3. 1907]}\newcommand{\editorInnen}{Martin Anton Müller und Gerd-Hermann Susen}%% latex-leseansicht-abspann.tex
%% Abspann für die Leseansicht.
%% Der Schalter \ifkorrekturansicht ist bereits durch den Vorspann gesetzt.

%% latex-abspann.tex
%% Gemeinsamer Abspann für Korrekturansicht und Leseansicht.
%% Setzt den Schalter \ifkorrekturansicht voraus (gesetzt in den
%% einbindenden Dateien latex-korrekturansicht-abspann.tex bzw.
%% latex-leseansicht-abspann.tex).
%% ---------------------------------------------------------------

\normalsize

% Das esempio-Environment wird nur in der Leseansicht benötigt
\ifkorrekturansicht\else
\newenvironment{esempio}[3]%
{
    \vspace{1.5ex}
    \rlap{\underline{#1}}
    \par
    \setlength{\parindent}{0cm}
    \nopagebreak
    \leftskip=#2cm
    \rightskip=#3cm
}
{
    \par
}
\fi

\doendnotes{C}
\bigskip
\vfill

\clearpage

\footnotesize

\ifkorrekturansicht
  \lohead{\textsc{register}}
\fi

% theindex-Environment neu definieren ohne reledmac
\makeatletter
\renewenvironment{theindex}{%
  \ifkorrekturansicht
    \section*{\indexname}%
  \else
    \subsubsection*{Index der erwähnten Entitäten}%
  \fi
  \setlength{\parindent}{0pt}%
  \setlength{\parskip}{0pt plus 0.3pt}%
  \let\item\@idxitem
}{%
  \ifkorrekturansicht\clearpage\fi
}
\makeatother

\IfFileExists{\jobname-pw.ind}{\input{\jobname-pw.ind}}{}

% Quellenangabe nur in der Leseansicht
\ifkorrekturansicht\else
% Fallback-Definitionen, falls die .tex-Datei \titel etc. nicht gesetzt hat
\providecommand{\titel}{}
\providecommand{\editorInnen}{}
\providecommand{\dateiname}{\jobname}

\vspace{3cm}

\vfill

\footnotesize
\textsc{Quelle}: \titel. Herausgegeben von {\editorInnen}. In: \emph{Arthur Schnitzler: Briefwechsel mit Autorinnen und Autoren}.
 Digitale Edition, https://schnitzler-briefe.acdh.oeaw.ac.at/{\dateiname}.html (Stand \today)
\fi

\end{document}


      