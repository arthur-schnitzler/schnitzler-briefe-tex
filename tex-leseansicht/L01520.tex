%% latex-leseansicht-vorspann.tex
%% Vorspann für die Leseansicht.
%% Lädt die gemeinsame Datei latex-vorspann.tex mit nicht gesetztem Schalter.

\newif\ifkorrekturansicht
\korrekturansichtfalse

\input{../tex-inputs/latex-vorspann}


\section[Arthur Schnitzler an Richard Beer-Hofmann, 26. 5. 1905]{L01520 Arthur Schnitzler an Richard Beer-Hofmann, 26. 5. 1905}
\nopagebreak\mylabel{L01520v}
\rehead{ }\normalsize\beginnumbering\briefempfaengerindex{Beer-Hofmann, Richard@\textsc{Beer-Hofmann, Richard}!zzzSchnitzler, Arthur@\emph{von Arthur Schnitzler}!1905-05-262@{26. 5. 1905}|(be}
\toendnotes[C]{\smallbreak\pagebreak[2]}
\correspDesc{Versand  durch Arthur Schnitzler am 26. 5. 1905 in Wien
\newline{}Erhalt  durch Richard Beer-Hofmann am 27. 5. 1905 in Rodaun}\toendnotes[C]{\smallbreak}
\Standort{YCGL, MSS 31.}
\physDesc{Brief, 2 Blätter, 5 Seiten, Kuvert, 2010 Zeichen
\newline{}Handschrift: Bleistift, deutsche Kurrent
\newline{}Versand: 1) Stempel: »\nobreak{}\oindex{XVIII., Währing@\textbf{XVIII., Währing}, \emph{Verwaltungsgebiet}|pwk}Wien 18/1 110, 26. 5. 05, 8–9N\nobreak{}«.   2) Stempel: »\nobreak{}\oindex{Wien@\textbf{Wien}!XXIII., Liesing@\textbf{XXIII., Liesing}!Rodaun@\textbf{Rodaun}, \emph{Region}|pwk}{\pb}Rodaun, 27 {[}5. 1905{]}, 8–9V\nobreak{}«. }
\buchAbdrucke{\weitereDrucke{Arthur Schnitzler, Richard Beer-Hofmann: \emph{Briefwechsel 1891–1931}. Herausgegeben von Konstanze Fliedl. Wien, Zürich: \emph{Europaverlag} 1992, S. 173–174.} }\toendnotes[C]{\smallbreak}\pstart{}{\pb}\textcolor{gray}{\textbf{Dr. Arthur Schnitzler}}\pend{}\pstart{}\textcolor{gray}{\textbf{Wien XVIII. Spoettelgasse 7\oindex{Wien@\textbf{Wien}!XVIII., Währing@\textbf{XVIII., Währing}!Edmund-Weiß-Gasse 7@\textbf{Edmund-Weiß-Gasse 7}, \emph{Wohngebäude}|pw}.}}\pend{}{\bigskip}\pstart{}{\pb}\textsc{Herrn Dr Richard Beer-Hofmann}\pend{}\pstart{}Rodaun\oindex{Wien@\textbf{Wien}!XXIII., Liesing@\textbf{XXIII., Liesing}!Rodaun@\textbf{Rodaun}, \emph{Region}|pw}\pend{}\pstart{}\textsc{Liesingerstraße 2}\oindex{Liesingerstraße@\textbf{Liesingerstraße}, \emph{Straße}|pw}.\pend{}\pstart{}\textsc{bei Wien\oindex{Wien@\textbf{Wien}, \emph{Verwaltungsgebiet}|pw}.}\pend{}{\bigskip}\vspace{1em}
\pstart
           \raggedleft{}{\pb}Wien\oindex{Wien@\textbf{Wien}, \emph{Verwaltungsgebiet}|pw}{ }26. 5. 905\pend
           \vspace{0.5em}
\pstart
           lieber Richard, eigentlich hab ich mir gedacht, daſs das viele
               unverſtändige u perfide, das Sie nun leſen mußten (mußten?), Sie kühler gelaſſen
               hätte – aber es{ }ſcheint wirklich: auf etwas gefaſſt{ }ſein hilft uns i{\geminationm}er nur{ }ſo lange als es nicht da iſt. Mir war am
               zuwiderſten Polgar\pwindex{Polgar, Alfred 17.\,10.\,1873 Wien – 24.\,4.\,1955 Zürich@\textsc{Polgar, Alfred} (17.\,10.\,1873 Wien – 24.\,4.\,1955 Zürich), \emph{Schriftsteller, Journalist, Kritiker}|pw}, der mir nebſtbei Talent zu
               haben{ }ſcheint und gut{ }ſchreibt, – und der{ }ſich zum Schluſs, in{ }ſeiner Sehnſucht {\pb}nach dem gemeinen Kerl,{ }ſo anmutig verräth. Er hat
               doch bisher{ }ſo{ }ſelten vergeblich gelechzt; – man dürfte ihm{ }ſagen: Warum in die Ferne{ }ſchweifen? Ach das gemeine liegt{ }ſo nah. Auch er gehört übrigens zu denjenigen, denen
               man doch einmal Zeit gö{\geminationn}en{ }ſollte – meinetwegen 12
               Jahre, damit{ }ſie ungeſtört ihren Grafen von \textsc{Charolais}\pwindex{Beer-Hofmann, Richard 11.\,7.\,1866 Wien – 26.\,9.\,1945 New York City@\textsc{Beer-Hofmann, Richard} (11.\,7.\,1866 Wien – 26.\,9.\,1945 New York City), \emph{Schriftsteller}!Graf von Charolais. Ein Trauerspiel@\strich\emph{Der Graf von Charolais. Ein Trauerspiel}|pw} oder auch nur die 10{ }ſchönen Verſe dichten können – da{\geminationn} würde man doch {\pb}ſehen,
               was herausko{\geminationm}t {\dots} mit Bildung
               und Fleiß und Willen {\dotsfour}\pend
           
\pstart
           – Was mich nicht hindert, mich dem Wunſche mancher andrer anzuſchließen, daſs Sie
               bald was neues anfangen –; wohl aus andern Motiven wünſch ich das, als die manchen
               andern; aber ich wünſch es{ }ſehr. Vor allem darum weil Sie da{\geminationn} die Empfindung hätten, daſs die Leute, die über den
               Dichter des \textsc{Charolais}\pwindex{Beer-Hofmann, Richard 11.\,7.\,1866 Wien – 26.\,9.\,1945 New York City@\textsc{Beer-Hofmann, Richard} (11.\,7.\,1866 Wien – 26.\,9.\,1945 New York City), \emph{Schriftsteller}!Graf von Charolais. Ein Trauerspiel@\strich\emph{Der Graf von Charolais. Ein Trauerspiel}|pw}{ }ſchreiben, eigentlich nicht mehr über Sie,{ }ſondern über {\pb}einen andern{ }ſchreiben, und \substVorne{}\textsuperscript{das}\substDazwischen{}es\substHinten{} iſt Einem, ich verſichre Sie, \strikeout{da{\geminationn}} ziemlich gleichgiltig, – was die Leute über einen andern{ }ſchreiben.\pend
           
\pstart
           – Heute erſt hab ich wieder Ihren Grund bewundert. Frl. \textsc{Erl}\pwindex{Erl, Dora @\textsc{Erl, Dora}, \emph{Schauspielerin, Gesangspädagogin}|pw}, die mit uns war,{ }ſagte: Wieſo ist er ihm noch nicht weg gekauft worden? –\pend
           
\pstart
           Ko{\geminationm}en Sie bald, vielleicht zu Tiſch? Ich dictire jetzt
               manchmal Nachmittag alſo wärs mir lieb, we{\geminationn} ich früher
               von Ihrem Ko{\geminationm}en unterrichtet {\pb}wäre. – Vormittag{ }ſpielen wir 3mal \textsc{Tennis}, was mir enorm viel Vergnügen macht. Müſſen Sie auch,{ }ſobald Sie Währinger\oindex{XVIII., Währing@\textbf{XVIII., Währing}, \emph{Verwaltungsgebiet}|pw} geworden{ }ſind.\pend
           
\pstart
           Wir grüßen Sie beide\pwindex{Beer-Hofmann, Paula 25.\,2.\,1879 Wien – 30.\,10.\,1939 Zürich@\textsc{Beer-Hofmann, Paula} (25.\,2.\,1879 Wien – 30.\,10.\,1939 Zürich)|pwv} und die
                  Kinder\pwindex{Beer-Hofmann, Naëmah 20.\,12.\,1898 Wien – 10.\,11.\,1971 New York City@\textsc{Beer-Hofmann, Naëmah} (20.\,12.\,1898 Wien – 10.\,11.\,1971 New York City)|pwv}\pwindex{Beer-Hofmann, Gabriel 9.\,1.\,1901 Wien – 24.\,3.\,1971 St Albans@\textsc{Beer-Hofmann, Gabriel} (9.\,1.\,1901 Wien – 24.\,3.\,1971 St Albans), \emph{Schriftsteller, Filmagent}|pwv}\pwindex{Beer-Hofmann, Mirjam 4.\,9.\,1897 Wien – 24.\,12.\,1984 New York City@\textsc{Beer-Hofmann, Mirjam} (4.\,9.\,1897 Wien – 24.\,12.\,1984 New York City)|pwv}.
                  Olga\pwindex{Schnitzler, Olga 17.\,1.\,1882 Wien – 13.\,1.\,1970 Lugano@\textsc{Schnitzler, Olga} (17.\,1.\,1882 Wien – 13.\,1.\,1970 Lugano), \emph{Schauspielerin, Sängerin}|pw} war von Ihrem Brief{ }ſo ergriffen, daſs{ }ſie eine Thräne im Augenwinkel hatte. Ich{ }ſage nichts als: dos is e Dichter. Aber ich
               hab mich{ }ſehr gefreut. Warum »aber«?\pend
           
\pstart
           Herzlichſt{\\}Ihr{\\}\spacefill\mbox{A.}\pend
           \selectlanguage{ngerman}\endnumbering\briefempfaengerindex{Beer-Hofmann, Richard@\textsc{Beer-Hofmann, Richard}!zzzSchnitzler, Arthur@\emph{von Arthur Schnitzler}!1905-05-262@{26. 5. 1905}|)be}\mylabel{L01520h}  \newcommand{\dateiname}{L01520}\newcommand{\titel}{Arthur Schnitzler an Richard Beer-Hofmann, 26. 5. 1905}\newcommand{\editorInnen}{Martin Anton Müller und Gerd-Hermann Susen}%% latex-leseansicht-abspann.tex
%% Abspann für die Leseansicht.
%% Der Schalter \ifkorrekturansicht ist bereits durch den Vorspann gesetzt.

%% latex-abspann.tex
%% Gemeinsamer Abspann für Korrekturansicht und Leseansicht.
%% Setzt den Schalter \ifkorrekturansicht voraus (gesetzt in den
%% einbindenden Dateien latex-korrekturansicht-abspann.tex bzw.
%% latex-leseansicht-abspann.tex).
%% ---------------------------------------------------------------

\normalsize

% Das esempio-Environment wird nur in der Leseansicht benötigt
\ifkorrekturansicht\else
\newenvironment{esempio}[3]%
{
    \vspace{1.5ex}
    \rlap{\underline{#1}}
    \par
    \setlength{\parindent}{0cm}
    \nopagebreak
    \leftskip=#2cm
    \rightskip=#3cm
}
{
    \par
}
\fi

\doendnotes{C}
\bigskip
\vfill

\clearpage

\footnotesize

\ifkorrekturansicht
  \lohead{\textsc{register}}
\fi

% theindex-Environment neu definieren ohne reledmac
\makeatletter
\renewenvironment{theindex}{%
  \ifkorrekturansicht
    \section*{\indexname}%
  \else
    \subsubsection*{Index der erwähnten Entitäten}%
  \fi
  \setlength{\parindent}{0pt}%
  \setlength{\parskip}{0pt plus 0.3pt}%
  \let\item\@idxitem
}{%
  \ifkorrekturansicht\clearpage\fi
}
\makeatother

\IfFileExists{\jobname-pw.ind}{\input{\jobname-pw.ind}}{}

% Quellenangabe nur in der Leseansicht
\ifkorrekturansicht\else
% Fallback-Definitionen, falls die .tex-Datei \titel etc. nicht gesetzt hat
\providecommand{\titel}{}
\providecommand{\editorInnen}{}
\providecommand{\dateiname}{\jobname}

\vspace{3cm}

\vfill

\footnotesize
\textsc{Quelle}: \titel. Herausgegeben von {\editorInnen}. In: \emph{Arthur Schnitzler: Briefwechsel mit Autorinnen und Autoren}.
 Digitale Edition, https://schnitzler-briefe.acdh.oeaw.ac.at/{\dateiname}.html (Stand \today)
\fi

\end{document}


