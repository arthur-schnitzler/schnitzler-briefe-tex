%% latex-korrekturansicht-vorspann.tex
%% Vorspann für die Korrekturansicht.
%% Lädt die gemeinsame Datei latex-vorspann.tex mit gesetztem Schalter.

\newif\ifkorrekturansicht
\korrekturansichttrue

\input{../tex-inputs/latex-vorspann}


\section[Arthur Schnitzler an Richard Beer-Hofmann, 26. 5. 1905]{L01520 Arthur Schnitzler an Richard Beer-Hofmann, 26. 5. 1905}
\nopagebreak\mylabel{L01520v}
\rehead{ }\normalsize\beginnumbering\briefempfaengerindex{Beer-Hofmann, Richard@\textsc{Beer-Hofmann, Richard}!zzzSchnitzler, Arthur@\emph{von Arthur Schnitzler}!1905-05-262@{26. 5. 1905}|(be}
\toendnotes[C]{\smallbreak\pagebreak[2]}\Standort{YCGL, MSS 31.}
\physDesc{Brief, 2 Blätter, 5 Seiten, Umschlag, 2010 Zeichen
\newline{}Handschrift: Bleistift, deutsche Kurrent
\newline{}Versand: 1) Stempel: »\nobreak{}\oindex{XVIII., Waehring@\textbf{XVIII., Währing}, \emph{A.ADM3}|pwk}Wien 18/1 110, 26. 5. 05, 8–9N\nobreak{}«.   2) Stempel: »\nobreak{}\oindex{Rodaun@\textbf{Rodaun}, \emph{A.ADM4}|pwk}{\pb}Rodaun, 27 {[}5. 1905{]}, 8–9V\nobreak{}«. }
\buchAbdrucke{\weitereDrucke{Arthur Schnitzler, Richard Beer-Hofmann: \emph{Briefwechsel 1891–1931}. Wien, Zürich: \emph{Europaverlag} 1992, S. 173–174.} }\toendnotes[C]{\smallbreak}\pstart{}{\pb}\textcolor{gray}{\textbf{Dr. Arthur Schnitzler}}\pend{}\pstart{}\textcolor{gray}{\textbf{Wien XVIII. Spoettelgasse 7\oindex{Edmund-Weiss-Gasse 7@\textbf{Edmund-Weiß-Gasse 7}, \emph{Wohngebäude (K.WHS)}|pw}.}}\pend{}{\bigskip}\pstart{}{\pb}\textsc{Herrn Dr Richard Beer-Hofmann}\pend{}\pstart{}Rodaun\oindex{Rodaun@\textbf{Rodaun}, \emph{A.ADM4}|pw}\pend{}\pstart{}\textsc{Liesingerstraße 2}\oindex{Liesingerstrasse@\textbf{Liesingerstraße}, \emph{Straße (K.STR)}|pw}.\pend{}\pstart{}\textsc{bei Wien\oindex{Wien@\textbf{Wien}, \emph{A.ADM2}|pw}.}\pend{}{\bigskip}\vspace{1em}
\pstart
           \raggedleft{}{\pb}Wien\oindex{Wien@\textbf{Wien}, \emph{A.ADM2}|pw}{ }26. 5. 905\pend
           \vspace{0.5em}
\pstart
           lieber Richard, eigentlich hab ich mir gedacht, daſs das viele
               unverſtändige u perfide, das Sie nun leſen mußten (mußten?), Sie kühler gelaſſen
               hätte – aber es ſcheint wirklich: auf etwas gefaſſt ſein hilft uns i{\geminationm}er nur ſo lange als es nicht da iſt. Mir war am
               zuwiderſten Polgar\pwindex{Polgar, Alfred 17.10.1873 – 24.04.1955@\textsc{Polgar, Alfred} (17.10.1873 – 24.04.1955), \emph{Schriftsteller/Schriftstellerin, Journalist/Journalistin, Kritiker/Kritikerin}|pw}, der mir nebſtbei Talent zu
               haben ſcheint und gut ſchreibt, – und der ſich zum Schluſs, in ſeiner Sehnſucht {\pb}nach dem gemeinen Kerl, ſo anmutig verräth. Er hat
               doch bisher ſo ſelten vergeblich gelechzt; – man dürfte ihm ſagen: Warum in die Ferne
               ſchweifen? Ach das gemeine liegt ſo nah. Auch er gehört übrigens zu denjenigen, denen
               man doch einmal Zeit gö{\geminationn}en ſollte – meinetwegen 12
               Jahre, damit ſie ungeſtört ihren Grafen von \textsc{Charolais}\pwindex{Graf von Charolais. Ein Trauerspiel@\emph{Der Graf von Charolais. Ein Trauerspiel}|pw} oder auch nur die 10 ſchönen Verſe dichten können – da{\geminationn} würde man doch {\pb}ſehen,
               was herausko{\geminationm}t {\dots} mit Bildung
               und Fleiß und Willen {\dotsfour}\pend
           
\pstart
           – Was mich nicht hindert, mich dem Wunſche mancher andrer anzuſchließen, daſs Sie
               bald was neues anfangen –; wohl aus andern Motiven wünſch ich das, als die manchen
               andern; aber ich wünſch es ſehr. Vor allem darum weil Sie da{\geminationn} die Empfindung hätten, daſs die Leute, die über den
               Dichter des \textsc{Charolais}\pwindex{Graf von Charolais. Ein Trauerspiel@\emph{Der Graf von Charolais. Ein Trauerspiel}|pw} ſchreiben, eigentlich nicht mehr über Sie, ſondern über {\pb}einen andern ſchreiben, und \substVorne{}\textsuperscript{das}\substDazwischen{}es\substHinten{} iſt Einem, ich verſichre Sie, \strikeout{da{\geminationn}} ziemlich gleichgiltig, – was die Leute über einen andern ſchreiben.\pend
           
\pstart
           – Heute erſt hab ich wieder Ihren Grund bewundert. Frl. \textsc{Erl}\pwindex{Erl, Dora @\textsc{Erl, Dora}, \emph{Schauspieler/Schauspielerin, Gesangspädagoge/Gesangspädagogin}|pw}, die mit uns war, ſagte: Wieſo ist er ihm noch nicht weg gekauft worden? –\pend
           
\pstart
           Ko{\geminationm}en Sie bald, vielleicht zu Tiſch? Ich dictire jetzt
               manchmal Nachmittag alſo wärs mir lieb, we{\geminationn} ich früher
               von Ihrem Ko{\geminationm}en unterrichtet {\pb}wäre. – Vormittag ſpielen wir 3mal \textsc{Tennis}, was mir enorm viel Vergnügen macht. Müſſen Sie auch,
               ſobald Sie Währinger\oindex{XVIII., Waehring@\textbf{XVIII., Währing}, \emph{A.ADM3}|pw} geworden ſind.\pend
           
\pstart
           Wir grüßen Sie beide\pwindex{Beer-Hofmann, Paula 25.02.1879 – 30.10.1939@\textsc{Beer-Hofmann, Paula} (25.02.1879 – 30.10.1939)|pwv} und die
                  Kinder\pwindex{Beer-Hofmann, Naemah 20.12.1898 – 10.11.1971@\textsc{Beer-Hofmann, Naëmah} (20.12.1898 – 10.11.1971)|pwv}\pwindex{Beer-Hofmann, Gabriel 09.01.1901 – 24.03.1971@\textsc{Beer-Hofmann, Gabriel} (09.01.1901 – 24.03.1971), \emph{Schriftsteller/Schriftstellerin, Filmagent/Filmagentin}|pwv}\pwindex{Beer-Hofmann, Mirjam 04.09.1897 – 24.12.1984@\textsc{Beer-Hofmann, Mirjam} (04.09.1897 – 24.12.1984)|pwv}.
                  Olga\pwindex{Schnitzler, Olga 17.01.1882 – 13.01.1970@\textsc{Schnitzler, Olga} (17.01.1882 – 13.01.1970), \emph{Schauspieler/Schauspielerin, Sänger/Sängerin}|pw} war von Ihrem Brief ſo ergriffen, daſs
               ſie eine Thräne im Augenwinkel hatte. Ich ſage nichts als: dos is e Dichter. Aber ich
               hab mich ſehr gefreut. Warum »aber«?\pend
           
\pstart
           Herzlichſt{\\}Ihr{\\}\spacefill\mbox{A.}\pend
           \selectlanguage{ngerman}\endnumbering\briefempfaengerindex{Beer-Hofmann, Richard@\textsc{Beer-Hofmann, Richard}!zzzSchnitzler, Arthur@\emph{von Arthur Schnitzler}!1905-05-262@{26. 5. 1905}|)be}\mylabel{L01520h}  \normalsize

\doendnotes{C}
\bigskip
\vfill

\clearpage

\footnotesize

\lohead{\textsc{register}}

% Definiere theindex-Environment komplett neu ohne reledmac
\makeatletter
\renewenvironment{theindex}{%
  \section*{\indexname}%
  \setlength{\parindent}{0pt}%
  \setlength{\parskip}{0pt plus 0.3pt}%
  \let\item\@idxitem
}{%
  \clearpage
}
\makeatother

\IfFileExists{\jobname-pw.ind}{\input{\jobname-pw.ind}}{}

\end{document}

      