%% latex-korrekturansicht-vorspann.tex
%% Vorspann für die Korrekturansicht.
%% Lädt die gemeinsame Datei latex-vorspann.tex mit gesetztem Schalter.

\newif\ifkorrekturansicht
\korrekturansichttrue

\input{../tex-inputs/latex-vorspann}


\section[Friedrich M. Fels an Arthur Schnitzler, 26. 10. 1894]{L00393 Friedrich M. Fels an Arthur Schnitzler, 26. 10. 1894}
\nopagebreak\mylabel{L00393v}
\rehead{ }\normalsize\beginnumbering\briefempfaengerindex{Schnitzler, Arthur@\textsc{Schnitzler, Arthur}!zzzFels, Friedrich Michael@\emph{von Friedrich Michael Fels}!1894-10-262@{26. 10. 1894}|(be}
\toendnotes[C]{\smallbreak\pagebreak[2]}\Standort{DLA, A:Schnitzler, HS.NZ85.1.2956.}
\physDesc{Brief, 1 Blatt, 2 Seiten, 1387 Zeichen
\newline{}Handschrift: schwarze Tinte, lateinische Kurrent
\newline{}Schnitzler: mit Bleistift nummeriert: »17« }\toendnotes[C]{\smallbreak}
\pstart
           \raggedleft{}{\pb}Wien\oindex{Wien@\textbf{Wien}, \emph{A.ADM2}|pw}{ }26. Okt. 94\pend
           
\pstart{}Lieber Dr Schnitzler!\pend\vspace{0.5em}
\pstart
           Danke für Ihre frdl. Bemühungen wegen Extrapost\orgindex{Extrapost@Extrapost|pw};
               sie sind gegenstandslos geworden. Ich soeben, mit Empfehlung von Dr. Brüll-Neuda\pwindex{Bruell-Neuda, Wilhelm 29.7.1852 – 21.10.1931@\textsc{Brüll-Neuda, Wilhelm} (29.7.1852 – 21.10.1931), \emph{Rechtsanwalt/Rechtsanwältin}|pw}, bei dem Besitzer, Konsul Thalberg\pwindex{Thalberg, Josef 24.6.1838 – 8.12.1902@\textsc{Thalberg, Josef} (24.6.1838 – 8.12.1902), \emph{Herausgeber/Herausgeberin, Geschäftsmann/Geschäftsfrau, Bankier/Bankierin}|pw}, der mir sagte, mit Theater- und
               Kunstreferat sei er versorgt, dagegen möge ich ihm Feuilletons geben: er habe gestern
               den \label{K_L00393-1v}\edtext{Nietzsche\pwindex{Nietzsche, Friedrich 15.10.1844 – 25.08.1900@\textsc{Nietzsche, Friedrich} (15.10.1844 – 25.08.1900), \emph{Schriftsteller/Schriftstellerin, Philosoph/Philosophin}|pw}artikel\pwindex{Friedrich Nietzsche@\emph{Friedrich Nietzsche}|pwv}}{\lemma{\textnormal{\emph{Nietzscheartikel}}}\Cendnote{\textnormal{Friedr. M. Fels\pwindex{Fels, Friedrich Michael *~1864@\textsc{Fels, Friedrich Michael} (*~1864), \emph{Journalist/Journalistin}|pwk}: \emph{Friedrich Nietzsche}\pwindex{Friedrich Nietzsche@\emph{Friedrich Nietzsche}|pwk}. In: \emph{Wiener Allgemeine Zeitung}\pwindex{Wiener Allgemeine Zeitung@\emph{Wiener Allgemeine Zeitung}|pwk}, Nr. 4988,
                        26. 10. 1894, S. 2–3.}}}\label{K_L00393-1} in der Allg.\pwindex{Wiener Allgemeine Zeitung@\emph{Wiener Allgemeine Zeitung}|pw} gelesen.\pend
           
\pstart
           Das Folgende bitte ich geheim zu halten: Dr. Ludassy\pwindex{Gans-Ludassy, Julius von 13.04.1858 – 30.09.1922@\textsc{Gans-Ludassy, Julius von} (13.04.1858 – 30.09.1922), \emph{Schriftsteller/Schriftstellerin, Journalist/Journalistin, Herausgeber/Herausgeberin}|pw} hat vor ein paar Tagen den Kraus\pwindex{Kraus, Karl 28.04.1874 – 12.06.1936@\textsc{Kraus, Karl} (28.04.1874 – 12.06.1936), \emph{Schriftsteller/Schriftstellerin, Publizist/Publizistin, Schriftsteller/Schriftstellerin}|pw} ko{\geminationm}en laſsen; er möge versuchen,
               Theaterreferate zu schreiben; er, Ludassy\pwindex{Gans-Ludassy, Julius von 13.04.1858 – 30.09.1922@\textsc{Gans-Ludassy, Julius von} (13.04.1858 – 30.09.1922), \emph{Schriftsteller/Schriftstellerin, Journalist/Journalistin, Herausgeber/Herausgeberin}|pw},
               werde suchen, sie unterzubringen, nachdem er mit Glücksma{\geminationn}s\pwindex{Gluecksmann, Heinrich 08.07.1864 – 01.03.1943@\textsc{Glücksmann, Heinrich} (08.07.1864 – 01.03.1943), \emph{Schriftsteller/Schriftstellerin, Journalist/Journalistin, Dramaturg/Dramaturgin}|pw} Berichten nicht zufrieden sei. So steht also die Sache diesmal so: ich bin
               nicht etwa, wie schon mehrmals zu spät geko{\geminationm}en, sondern
               einfach übergangen worden wegen – Kraus\pwindex{Kraus, Karl 28.04.1874 – 12.06.1936@\textsc{Kraus, Karl} (28.04.1874 – 12.06.1936), \emph{Schriftsteller/Schriftstellerin, Publizist/Publizistin, Schriftsteller/Schriftstellerin}|pw}, den
               Sie zwar schätzen, der aber nichts weiſs und nichts ka{\geminationn}.\pend
           
\pstart
           {\pb}An sich geht mir die Sache nicht nahe; dazu schätze
               ich mich viel zu sehr und weiſs, daſs, wer Kraus\pwindex{Kraus, Karl 28.04.1874 – 12.06.1936@\textsc{Kraus, Karl} (28.04.1874 – 12.06.1936), \emph{Schriftsteller/Schriftstellerin, Publizist/Publizistin, Schriftsteller/Schriftstellerin}|pw} mir vorzieht, um seinen Geschmack nicht zu beneiden ist; auch Neuma{\geminationn}-Hofer\pwindex{Neumann-Hofer, Gilbert Otto 04.02.1857 – 14.04.1941@\textsc{Neumann-Hofer, Gilbert Otto} (04.02.1857 – 14.04.1941), \emph{Kritiker/Kritikerin, Theaterleiter/Theaterleiterin}|pw} hat
               den \introOben{}Kraus\pwindex{Kraus, Karl 28.04.1874 – 12.06.1936@\textsc{Kraus, Karl} (28.04.1874 – 12.06.1936), \emph{Schriftsteller/Schriftstellerin, Publizist/Publizistin, Schriftsteller/Schriftstellerin}|pw}\introOben{} ja wegen »Unwiſsenheit, die durch einen schneidigen Ton allein nicht gut zu
               machen sei«, hinausgeschmiſsen. Aber daſs ich wieder einmal kein ständiges Referat
                  beko{\geminationm}en habe, das schmerzt mich, we{\geminationn} ich bedenke, daſs nun wieder mehr Aussicht für mich
               vorhanden ist, das nicht zu erreichen, was ich anstrebe. Mögen also die Dinge ihren
               Lauf nehmen: ich hadere mit niemanden.\pend
           
\pstart
           Herzlichen Gruſs{\\[\baselineskip]}von Ihrem \spacefill\mbox{Fels}\pend
           \leftskip=0em{}\selectlanguage{ngerman}\endnumbering\briefempfaengerindex{Schnitzler, Arthur@\textsc{Schnitzler, Arthur}!zzzFels, Friedrich Michael@\emph{von Friedrich Michael Fels}!1894-10-262@{26. 10. 1894}|)be}\mylabel{L00393h}  \normalsize

\doendnotes{C}
\bigskip
\vfill

\clearpage

\footnotesize

\lohead{\textsc{register}}

% Definiere theindex-Environment komplett neu ohne reledmac
\makeatletter
\renewenvironment{theindex}{%
  \section*{\indexname}%
  \setlength{\parindent}{0pt}%
  \setlength{\parskip}{0pt plus 0.3pt}%
  \let\item\@idxitem
}{%
  \clearpage
}
\makeatother

\IfFileExists{\jobname-pw.ind}{\input{\jobname-pw.ind}}{}

\end{document}

      