%% latex-leseansicht-vorspann.tex
%% Vorspann für die Leseansicht.
%% Lädt die gemeinsame Datei latex-vorspann.tex mit nicht gesetztem Schalter.

\newif\ifkorrekturansicht
\korrekturansichtfalse

\input{../tex-inputs/latex-vorspann}


               \section[Friedrich M. Fels an Arthur Schnitzler, 26. 10. 1894]{ Friedrich M. Fels an Arthur Schnitzler, 26. 10. 1894}\nopagebreak\mylabel{v}\rehead{ }\begin{ledgroupsized}[t]{13cm}\normalsize\beginnumbering\briefempfaengerindex{Schnitzler, Arthur@\textsc{Schnitzler, Arthur}!zzzFels, Friedrich Michael@\emph{von Friedrich Michael Fels}!1894-10-262@{26. 10. 1894}|(be} \toendnotes[C]{\smallbreak\pagebreak[2]} \Standort{DLA, A:Schnitzler, HS.NZ85.1.2956.}
\physDesc{Brief, 1 Blatt, 2 Seiten
\newline{}Handschrift: schwarze Tinte, lateinische Kurrent
\newline{}Schnitzler: mit Bleistift nummeriert: »17« }\toendnotes[C]{\smallbreak}\pstart
           \raggedleft{}{\pb}Wien\oindex{Wien@\textbf{Wien}|pw}{ }26. Okt. 94\pend
           \pstart{}Lieber Dr Schnitzler!\pend\pstart
           Danke für Ihre frdl. Bemühungen wegen Extrapost\orgindex{Extrapost@Extrapost|pw};
                    sie sind gegenstandslos geworden. Ich soeben, mit Empfehlung von Dr. Brüll-Neuda\pwindex{Bruell-Neuda, Wilhelm 29.7.1852 – 21.10.1931@\textsc{Brüll-Neuda, Wilhelm} (29.7.1852 – 21.10.1931), \emph{Rechtsanwalt}|pw}, bei dem Besitzer, Konsul Thalberg\pwindex{Thalberg, Josef 24.6.1838 – 8.12.1902@\textsc{Thalberg, Josef} (24.6.1838 – 8.12.1902), \emph{Herausgeber, Geschäftsmann, Bankier}|pw}, der mir sagte, mit Theater- und
                    Kunstreferat sei er versorgt, dagegen möge ich ihm Feuilletons geben: er habe
                    gestern den \label{K_L00393_1v}\edtext{Nietzsche\pwindex{Nietzsche, Friedrich 15.10.1844 – 25.08.1900@\textsc{Nietzsche, Friedrich} (15.10.1844 – 25.08.1900), \emph{Schriftsteller, Philosoph}|pw}artikel\pwindex{Fels, Friedrich Michael *~1864@\textsc{Fels, Friedrich Michael} (*~1864), \emph{Journalist}!Friedrich Nietzsche26.10.1894 – 26.10.1894@\strich\emph{Friedrich Nietzsche} {[}26.10.1894 – 26.10.1894{]}|pwv}}{\lemma{\textnormal{\emph{Nietzscheartikel}}}\Cendnote{\textnormal{Friedr. M. Fels\pwindex{Fels, Friedrich Michael *~1864@\textsc{Fels, Friedrich Michael} (*~1864), \emph{Journalist}|pwk}: \emph{Friedrich Nietzsche}\pwindex{Fels, Friedrich Michael *~1864@\textsc{Fels, Friedrich Michael} (*~1864), \emph{Journalist}!Friedrich Nietzsche26.10.1894 – 26.10.1894@\strich\emph{Friedrich Nietzsche} {[}26.10.1894 – 26.10.1894{]}|pwk}. In: \emph{Wiener Allgemeine Zeitung}\pwindex{Wiener Allgemeine Zeitung1.3.1880 – 11.2.1934@\emph{Wiener Allgemeine Zeitung}|pwk}, Nr. 4988,
                                26. 10. 1894, S. 2–3.}}}\label{K_L00393_1h} in der Allg.\pwindex{Wiener Allgemeine Zeitung1.3.1880 – 11.2.1934@\emph{Wiener Allgemeine Zeitung}|pw} gelesen.\pend
           \pstart
           Das Folgende bitte ich geheim zu halten: Dr. Ludassy\pwindex{Gans-Ludassy, Julius von 13.04.1858 – 30.09.1922@\textsc{Gans-Ludassy, Julius von} (13.04.1858 – 30.09.1922), \emph{Schriftsteller, Journalist}|pw} hat vor ein paar Tagen den Kraus\pwindex{Kraus, Karl 28.04.1874 – 12.06.1936@\textsc{Kraus, Karl} (28.04.1874 – 12.06.1936), \emph{Schriftsteller, Publizist}|pw} ko{\geminationm}en laſsen; er möge versuchen,
                    Theaterreferate zu schreiben; er, Ludassy\pwindex{Gans-Ludassy, Julius von 13.04.1858 – 30.09.1922@\textsc{Gans-Ludassy, Julius von} (13.04.1858 – 30.09.1922), \emph{Schriftsteller, Journalist}|pw},
                    werde suchen, sie unterzubringen, nachdem er mit Glücksma{\geminationn}\pwindex{Gluecksmann, Heinrich 07.07.1864 – 01.03.1943@\textsc{Glücksmann, Heinrich} (07.07.1864 – 01.03.1943), \emph{Schriftsteller, Journalist, Dramaturg}|pw}s Berichten nicht zufrieden sei. So steht also die Sache diesmal so: ich
                    bin nicht etwa, wie schon mehrmals zu spät geko{\geminationm}en,
                    sondern einfach übergangen worden wegen – Kraus\pwindex{Kraus, Karl 28.04.1874 – 12.06.1936@\textsc{Kraus, Karl} (28.04.1874 – 12.06.1936), \emph{Schriftsteller, Publizist}|pw}, den Sie zwar schätzen, der aber nichts weiſs und nichts ka{\geminationn}.\pend
           \pstart
           {\pb}An sich geht mir die Sache nicht nahe;
                    dazu schätze ich mich viel zu sehr und weiſs, daſs, wer Kraus\pwindex{Kraus, Karl 28.04.1874 – 12.06.1936@\textsc{Kraus, Karl} (28.04.1874 – 12.06.1936), \emph{Schriftsteller, Publizist}|pw} mir vorzieht, um seinen Geschmack nicht zu beneiden
                    ist; auch Neuma{\geminationn}-Hofer\pwindex{Neumann-Hofer, Gilbert Otto 04.02.1857 – 14.04.1941@\textsc{Neumann-Hofer, Gilbert Otto} (04.02.1857 – 14.04.1941), \emph{Kritiker, Theaterleiter}|pw} hat den \introOben{}Kraus\pwindex{Kraus, Karl 28.04.1874 – 12.06.1936@\textsc{Kraus, Karl} (28.04.1874 – 12.06.1936), \emph{Schriftsteller, Publizist}|pw}\introOben{} ja wegen »Unwiſsenheit, die durch einen schneidigen Ton allein nicht gut
                    zu machen sei«, hinausgeschmiſsen. Aber daſs ich wieder einmal kein ständiges
                    Referat beko{\geminationm}en habe, das schmerzt mich, we{\geminationn} ich bedenke, daſs nun wieder mehr Aussicht für
                    mich vorhanden ist, das nicht zu erreichen, was ich anstrebe. Mögen also die
                    Dinge ihren Lauf nehmen: ich hadere mit niemanden.\pend
           \pstart
           Herzlichen Gruſs{\\[\baselineskip]}von Ihrem \spacefill\mbox{Fels}\pend
           \leftskip=0em{}          \endnumbering\briefempfaengerindex{Schnitzler, Arthur@\textsc{Schnitzler, Arthur}!zzzFels, Friedrich Michael@\emph{von Friedrich Michael Fels}!1894-10-262@{26. 10. 1894}|)be}\mylabel{h}\end{ledgroupsized}  \newcommand{\dateiname}{L00393}\newcommand{\titel}{Friedrich M. Fels an Arthur Schnitzler, 26. 10. 1894}\newcommand{\editorInnen}{Martin Anton Müller und Gerd-Hermann Susen}
            \footnotesize
\begin{ledgroupsized}[t]{11.5cm}
\doendnotes{C}
\end{ledgroupsized}
         %% latex-leseansicht-abspann.tex
%% Abspann für die Leseansicht.
%% Der Schalter \ifkorrekturansicht ist bereits durch den Vorspann gesetzt.

%% latex-abspann.tex
%% Gemeinsamer Abspann für Korrekturansicht und Leseansicht.
%% Setzt den Schalter \ifkorrekturansicht voraus (gesetzt in den
%% einbindenden Dateien latex-korrekturansicht-abspann.tex bzw.
%% latex-leseansicht-abspann.tex).
%% ---------------------------------------------------------------

\normalsize

% Das esempio-Environment wird nur in der Leseansicht benötigt
\ifkorrekturansicht\else
\newenvironment{esempio}[3]%
{
    \vspace{1.5ex}
    \rlap{\underline{#1}}
    \par
    \setlength{\parindent}{0cm}
    \nopagebreak
    \leftskip=#2cm
    \rightskip=#3cm
}
{
    \par
}
\fi

\doendnotes{C}
\bigskip
\vfill

\clearpage

\footnotesize

\ifkorrekturansicht
  \lohead{\textsc{register}}
\fi

% theindex-Environment neu definieren ohne reledmac
\makeatletter
\renewenvironment{theindex}{%
  \ifkorrekturansicht
    \section*{\indexname}%
  \else
    \subsubsection*{Index der erwähnten Entitäten}%
  \fi
  \setlength{\parindent}{0pt}%
  \setlength{\parskip}{0pt plus 0.3pt}%
  \let\item\@idxitem
}{%
  \ifkorrekturansicht\clearpage\fi
}
\makeatother

\IfFileExists{\jobname-pw.ind}{\input{\jobname-pw.ind}}{}

% Quellenangabe nur in der Leseansicht
\ifkorrekturansicht\else
% Fallback-Definitionen, falls die .tex-Datei \titel etc. nicht gesetzt hat
\providecommand{\titel}{}
\providecommand{\editorInnen}{}
\providecommand{\dateiname}{\jobname}

\vspace{3cm}

\vfill

\footnotesize
\textsc{Quelle}: \titel. Herausgegeben von {\editorInnen}. In: \emph{Arthur Schnitzler: Briefwechsel mit Autorinnen und Autoren}.
 Digitale Edition, https://schnitzler-briefe.acdh.oeaw.ac.at/{\dateiname}.html (Stand \today)
\fi

\end{document}


      