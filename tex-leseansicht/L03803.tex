%% latex-korrekturansicht-vorspann.tex
%% Vorspann für die Korrekturansicht.
%% Lädt die gemeinsame Datei latex-vorspann.tex mit gesetztem Schalter.

\newif\ifkorrekturansicht
\korrekturansichttrue

\input{../tex-inputs/latex-vorspann}


\section[Arthur Schnitzler an Stefan Zweig, 2. 6. 1909]{L03803 Arthur Schnitzler an Stefan Zweig, 2. 6. 1909}
\nopagebreak\mylabel{L03803v}
\rehead{ }\normalsize\beginnumbering\briefempfaengerindex{Zweig, Stefan@\textsc{Zweig, Stefan}!zzzSchnitzler, Arthur@\emph{von Arthur Schnitzler}!1909-06-021@{2. 6. 1909}|(be}
\toendnotes[C]{\smallbreak\pagebreak[2]}\Standort{Jerusalem, National Library of Israel, ARC. Ms. Var. 305 1 58 Stefan Zweig Collection.}
\physDesc{Postkarte, 1 Blatt, 2 Seiten, 276 Zeichen
\newline{}Handschrift: schwarze Tinte, lateinische Kurrent
\newline{}Versand: 1) Stempel: »\nobreak{}\oindex{XVIII., Waehring@\textbf{XVIII., Währing}, \emph{A.ADM3}|pwk}Wien 18, 2 VI 0\textcolor{gray}{9}, 3\textsuperscript{10}N\nobreak{}«.   2) Stempel: »\nobreak{}\oindex{VIII., Josefstadt@\textbf{VIII., Josefstadt}, \emph{A.ADM3}|pwk}8/\textsubscript{1} Wien
                                       64, 2. VI. 09, 3\textsuperscript{5\textcolor{gray}{0}}\nobreak{}«. 
\newline{}Ordnung: mit Bleistift Vermerk mit Klärung der Jahresangabe in Schnitzlers Datierung: »1909«  }\toendnotes[C]{\smallbreak}\pstart{}{\pb}\textcolor{gray}{\textbf{Dr Arthur Schnitzler}}\pend{}\pstart{}\textcolor{gray}{\textbf{Wien XVIII. Spoettelgasse 7\oindex{Edmund-Weiss-Gasse@\textbf{Edmund-Weiß-Gasse}, \emph{R.ST}|pw}.}}\pend{}{\bigskip}\pstart{}Dr. Stefan Zweig\pend{}\pstart{}Wien VIII\oindex{VIII., Josefstadt@\textbf{VIII., Josefstadt}, \emph{A.ADM3}|pw}\pend{}\pstart{}Kochgasse 8\oindex{Kochgasse 8@\textbf{Kochgasse 8}, \emph{Wohngebäude (K.WHS)}|pw}.\pend{}{\bigskip}\vspace{1em}
\pstart
           \raggedleft{}{\pb}2. 6. 0\textcolor{gray}{9}\pend
           
\pstart{}lieber Herr Doctor\pend\vspace{0.5em}
\pstart
           wir nachtmahlen heute im Türkenschanzpark\oindex{Restauration Tuerkenschanz-Park@\textbf{Restauration Türkenschanz-Park}, \emph{Lokal (K.LKL)}|pw}, etwa
                  ½ 9 – 9\pend
           
\pstart
           Wollen Sie \label{K_L03803-1v}\edtext{hinko{\geminationm}en}{\lemma{\textnormal{\emph{hinkommen}}}\Cendnote{\textnormal{Vgl. A. S.: \emph{Tagebuch}, 2. 6. 1909.}}}\label{K_L03803-1}? Oder uns
               abholen? (Sind bis gegen ½ 9 zu Hause, von 8 an bereit.)
                  Wassermanns\pwindex{Wassermann, Jakob 10.03.1873 – 01.01.1934@\textsc{Wassermann, Jakob} (10.03.1873 – 01.01.1934), \emph{Schriftsteller/Schriftstellerin}|pw}\pwindex{Wassermann, Julie 05.12.1876 – April 1963@\textsc{Wassermann, Julie} (05.12.1876 – April 1963), \emph{Schriftsteller/Schriftstellerin}|pw} sind wahrscheinlich
               auch im T.p.\oindex{Restauration Tuerkenschanz-Park@\textbf{Restauration Türkenschanz-Park}, \emph{Lokal (K.LKL)}|pw} Antwort \uline{nicht nöthig}.\pend
           \pstart Herzlichst Ihr \spacefill\mbox{A. S.}\pend{}\selectlanguage{ngerman}\endnumbering\briefempfaengerindex{Zweig, Stefan@\textsc{Zweig, Stefan}!zzzSchnitzler, Arthur@\emph{von Arthur Schnitzler}!1909-06-021@{2. 6. 1909}|)be}\mylabel{L03803h}  \normalsize

\doendnotes{C}
\bigskip
\vfill

\clearpage

\footnotesize

\lohead{\textsc{register}}

% Definiere theindex-Environment komplett neu ohne reledmac
\makeatletter
\renewenvironment{theindex}{%
  \section*{\indexname}%
  \setlength{\parindent}{0pt}%
  \setlength{\parskip}{0pt plus 0.3pt}%
  \let\item\@idxitem
}{%
  \clearpage
}
\makeatother

\IfFileExists{\jobname-pw.ind}{\input{\jobname-pw.ind}}{}

\end{document}

      