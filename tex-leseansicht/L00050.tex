%% latex-korrekturansicht-vorspann.tex
%% Vorspann für die Korrekturansicht.
%% Lädt die gemeinsame Datei latex-vorspann.tex mit gesetztem Schalter.

\newif\ifkorrekturansicht
\korrekturansichttrue

\input{../tex-inputs/latex-vorspann}


\section[Hugo August von Hofmannsthal an Arthur Schnitzler, 7. 12. 1891]{L00050 Hugo August von Hofmannsthal an Arthur Schnitzler, 7. 12. 1891}
\nopagebreak\mylabel{L00050v}
\rehead{ }\normalsize\beginnumbering\briefempfaengerindex{Schnitzler, Arthur@\textsc{Schnitzler, Arthur}!zzzHofmannsthal, Hugo August von@\emph{von Hugo August von Hofmannsthal}!1891-12-071@{7. 12. 1891}|(be}
\toendnotes[C]{\smallbreak\pagebreak[2]}\Standort{DLA, A:Schnitzler, HS.NZ85.1.3483.}
\physDesc{Briefkarte, 603 Zeichen (aufgeprägtes Wappen )
\newline{}Handschrift: schwarze Tinte, deutsche Kurrent}\toendnotes[C]{\smallbreak}
\pstart
           \raggedleft{}{\pb}Wien\oindex{Wien@\textbf{Wien}, \emph{A.ADM2}|pw}{ }7/12 91.\pend
           \vspace{0.5em}
\pstart
           Draußen Nebel u Influenza. Drinnen im Zi{\geminationm}er alles was
               dasſelbe behaglich macht, Licht, Wärme, ein guter \textsc{Fauteuil},
               ein auf drei Acte berechneter »\textsc{Pfosten}« u \textsc{A. Schnitzler Mährchen\pwindex{Maerchen. Schauspiel in drei Aufzuegen@\emph{Das Märchen. Schauspiel in drei Aufzügen}|pw}}! Dſs ich den beſagten \textsc{Pfosten} im zweiten Act
               erbarmungslos ausgehen ließ mag Ihnen beweiſen, dſs Ihr Stück auch auf den
               mindergebildeten von \label{K_L00050-1v}\edtext{Wandel\pwindex{Maerchen. Schauspiel in drei Aufzuegen@\emph{Das Märchen. Schauspiel in drei Aufzügen}|pwv}ſchen \textsc{veilletäten}}{\lemma{\textnormal{\emph{Wandelſchen veilletäten}}}\Cendnote{\textnormal{Adalbert Wandel\pwindex{Maerchen. Schauspiel in drei Aufzuegen@\emph{Das Märchen. Schauspiel in drei Aufzügen}|pwkv} ist eine
                  Figur aus dem \emph{Märchen}\pwindex{Maerchen. Schauspiel in drei Aufzuegen@\emph{Das Märchen. Schauspiel in drei Aufzügen}|pwk}. Eine »Velleität« ist
                  ein Vorsatz, der nicht umgesetzt wird.}}}\label{K_L00050-1} angehauchten \textsc{Philister} ſeine {\pb}Wirkung nicht verleugnet.
                  \textsc{Charakterisirung}, \textsc{Motivirung},
                  \textsc{Dialog}, Alles glänzend u intereſsant!\pend
           
\pstart
           Nehmen Sie alſo meinen herzlichen Dank für die Überſendg.\pend
           
\pstart
           Mit den beſten Wünſchen für durchſchlagenden Erfolg Ihr{\\[\baselineskip]}ergebenſter{\\[\baselineskip]}\spacefill\mbox{D\textsuperscript{r} Hofmannsthal.}\pend
           \leftskip=0em{}\selectlanguage{ngerman}\endnumbering\briefempfaengerindex{Schnitzler, Arthur@\textsc{Schnitzler, Arthur}!zzzHofmannsthal, Hugo August von@\emph{von Hugo August von Hofmannsthal}!1891-12-071@{7. 12. 1891}|)be}\mylabel{L00050h}  \normalsize

\doendnotes{C}
\bigskip
\vfill

\clearpage

\footnotesize

\lohead{\textsc{register}}

% Definiere theindex-Environment komplett neu ohne reledmac
\makeatletter
\renewenvironment{theindex}{%
  \section*{\indexname}%
  \setlength{\parindent}{0pt}%
  \setlength{\parskip}{0pt plus 0.3pt}%
  \let\item\@idxitem
}{%
  \clearpage
}
\makeatother

\IfFileExists{\jobname-pw.ind}{\input{\jobname-pw.ind}}{}

\end{document}

      