%% latex-korrekturansicht-vorspann.tex
%% Vorspann für die Korrekturansicht.
%% Lädt die gemeinsame Datei latex-vorspann.tex mit gesetztem Schalter.

\newif\ifkorrekturansicht
\korrekturansichttrue

\input{../tex-inputs/latex-vorspann}


\section[ Felix Salten an Arthur Schnitzler, 26. 1. 1908]{L03491 Felix Salten an Arthur Schnitzler, 26. 1. 1908}
\nopagebreak\mylabel{L03491v}
\rehead{ }\normalsize\beginnumbering\briefempfaengerindex{Schnitzler, Arthur@\textsc{Schnitzler, Arthur}!zzzSalten, Felix@\emph{von Felix Salten}!1908-01-261@{26. 1. 1908}|(be}
\toendnotes[C]{\smallbreak\pagebreak[2]}\Standort{CUL, Schnitzler, B 89, B 1.}
\physDesc{Brief, 1 Blatt, 2 Seiten, 2083 Zeichen
\newline{}Handschrift: Bleistift, lateinische Kurrent
\newline{}Schnitzler: mit Bleistift Vermerk »\textsc{Salt\textcolor{gray}{en}}« 
\newline{}Ordnung: mit Bleistift von unbekannter Hand nummeriert: »241« }\toendnotes[C]{\smallbreak}
\pstart
           {\pb}\textcolor{gray}{\textbf{Südbahn-Hôtel\oindex{Suedbahnhotel [Semmering]@\textbf{Südbahnhotel [Semmering]}, \emph{Hotel (K.HTL)}|pw}}}\pend
           
\pstart
           \textcolor{gray}{\textbf{Semmering\oindex{Semmering@\textbf{Semmering}, \emph{A.ADM3}|pw}}}\hfill 26./1. 08\pend
           
\pstart
           \textcolor{gray}{\textbf{Austria\oindex{Oesterreich@\textbf{Österreich}, \emph{A.PCLI}|pw}.}}\pend
           
\pstart
           \textcolor{gray}{\textbf{\textbf{TELEGRAMME:}}}\pend
           
\pstart
           \textcolor{gray}{\textbf{\textbf{SÜDBAHNHÔTEL SEMMERING\oindex{Suedbahnhotel [Semmering]@\textbf{Südbahnhotel [Semmering]}, \emph{Hotel (K.HTL)}|pw}.}}}\pend
           
\pstart
           \textcolor{gray}{\textbf{TELEPHON:}}\pend
           
\pstart
           \textcolor{gray}{\textbf{HÔTEL {\dotsfour} NR. 5.}}\pend
           
\pstart
           \textcolor{gray}{\textbf{DEPENDANCE NR. 6.}}\pend
           
\pstart{}Lieber,\pend\vspace{0.5em}
\pstart
           danke sehr für Ihren ausführlichen \label{K_L03491-1v}\edtext{Brief}{\lemma{\textnormal{\emph{Brief}}}\Cendnote{\textnormal{Arthur Schnitzler an Felix Salten, 25. 1. 1908.
               }}}\label{K_L03491-1}, der mich sehr gefreut hat. Den letzten Satz, da wo Sie sagen, dass Sie sich
               wieder »\label{K_L03491-2v}\edtext{keck mitten ins Leben}{\lemma{\textnormal{\emph{keck mitten ins Leben}}}\Cendnote{\textnormal{Schnitzler schrieb »frech wieder
                     mitten ins Leben hinein«.}}}\label{K_L03491-2}« u. s. w. habe ich, wie ich Ihnen
               gestehen muss, mit einer plötzlich aufsteigenden, sehr starken Ergriffenheit gelesen.
               Denn aus ihm sah ich erst ganz deutlich, \uline{wo} Sie in
               dieser letzten Zeit mit Ihren Gedanken und Sorgen gewesen sind, und was Sie
               durchgemacht haben. Nun aber dürfen Sie sich wol freuen und Ihre Freunde mit Ihnen.
               Wundervoll ist es ja, wie diese \label{K_L03491-3v}\edtext{Gefahr}{\lemma{\textnormal{\emph{Gefahr}}}\Cendnote{\textnormal{Siehe Felix Salten an Arthur Schnitzler, [10. 12. 1907].
               }}}\label{K_L03491-3} an Ihnen u. Ihrer Frau\pwindex{Schnitzler, Olga 17.01.1882 – 13.01.1970@\textsc{Schnitzler, Olga} (17.01.1882 – 13.01.1970), \emph{Schauspieler/Schauspielerin, Sänger/Sängerin}|pwv} vorbeigeschwebt ist, und wie dann mit dem Grillparzer Preis\orgindex{Franz-Grillparzer-Preis@Franz-Grillparzer-Preis|pw} etwas zu Ihnen kam, was schließlich doch im
               Tiefsten so etwas wie einen Schimmer von Glück bedeutet. Wir gehen dem Frühling
               entgegen, und Ihre Frau\pwindex{Schnitzler, Olga 17.01.1882 – 13.01.1970@\textsc{Schnitzler, Olga} (17.01.1882 – 13.01.1970), \emph{Schauspieler/Schauspielerin, Sänger/Sängerin}|pwv} wird
               sich hoffentlich rasch erholen. Man sagt ja, dass nach dem Scharlach die Gesundheit
               intensiver wird, und so wird Frau Olga\pwindex{Schnitzler, Olga 17.01.1882 – 13.01.1970@\textsc{Schnitzler, Olga} (17.01.1882 – 13.01.1970), \emph{Schauspieler/Schauspielerin, Sänger/Sängerin}|pw} jetzt in
               ein schönes Genesen und Glühen kommen, und mit der Jahreszeit gehen. Besseres läßt
               sich kaum denken. Ihren \label{K_L03491-4v}\edtext{Roman\pwindex{Weg ins Freie. Roman@\emph{Der Weg ins Freie. Roman}|pwv} las ich nun doch in den
               ersten zwei Fortsetzungen}{\lemma{\textnormal{\emph{Roman … Fortsetzungen}}}\Cendnote{\textnormal{Vgl. Felix Salten an Arthur Schnitzler, 16. 1. 1908. Saltens\pwindex{Salten, Felix 06.09.1869 – 08.10.1945@\textsc{Salten, Felix} (06.09.1869 – 08.10.1945), \emph{Schriftsteller/Schriftstellerin, Journalist/Journalistin, Chefredakteur/Chefredakteurin}|pwk} Lektüre der ersten Fortsetzung bedeutet, dass das
                  Monatsheft des Februar bereits vorzeitig ausgeliefert wurde.}}}\label{K_L03491-4}. Sie werden
               meine Neugierde begreifen u. entschuldigen. \label{K_L03491-5v}\edtext{Sagen kann ich jetzt natürlich noch nichts}{\lemma{\textnormal{\emph{Sagen … nichts}}}\Cendnote{\textnormal{Nachdem sie sich wenige Tage später, am
                     4. 2. 1908, auf
                  dem Weg zum Semmering\oindex{Semmering@\textbf{Semmering}, \emph{A.ADM3}|pwk} getroffen hatten,
                  notierte Schnitzler in seinem \emph{Tagebuch}\pwindex{Tagebuch@\emph{Tagebuch}|pwk}: »Er [ = Salten\pwindex{Salten, Felix 06.09.1869 – 08.10.1945@\textsc{Salten, Felix} (06.09.1869 – 08.10.1945), \emph{Schriftsteller/Schriftstellerin, Journalist/Journalistin, Chefredakteur/Chefredakteurin}|pwk}] sagt über einen Roman\pwindex{Weg ins Freie. Roman@\emph{Der Weg ins Freie. Roman}|pwkv}, dessen 2 erste Theile (Jänner-, Feberheft\pwindex{neue Rundschau@\emph{Die neue Rundschau}|pwkv}) er gelesen: Sehr
                  lebendige Gestalten. Dann (zögernd) {\dots} ›Aber es hat mir erst recht leid gethan,
                  dass ich’s nicht im Manuscript gelesen {\dots} es sind stilistische (Fehler?) Mängel,
                  Härten (erinner mich des Worts nicht) – wie sie natürlich bei einem so großen Werk\pwindex{Weg ins Freie. Roman@\emph{Der Weg ins Freie. Roman}|pwkv} nicht zu vermeiden
                  sind. –‹ Es ärgerte, ja empörte mich beinahe – obwohl, oder weil ich darauf
                  vorbereitet war. – ›Er wird nicht wollen‹ sagte ich neulich. – Wer wird wollen –?‹«
                  Diese Kritik Saltens\pwindex{Salten, Felix 06.09.1869 – 08.10.1945@\textsc{Salten, Felix} (06.09.1869 – 08.10.1945), \emph{Schriftsteller/Schriftstellerin, Journalist/Journalistin, Chefredakteur/Chefredakteurin}|pwk} sollte Schnitzler noch lange beschäftigen,  vgl. A. S.: \emph{Tagebuch}, 28. 4. 1908.}}}\label{K_L03491-5}, ahne auch nur
               von weitem, wohin der Weg ins
                  Freie\pwindex{Weg ins Freie. Roman@\emph{Der Weg ins Freie. Roman}|pwv} führt. Aber eine Menge Menschen wird mir jetzt schon sehr lebendig und
               das Abreißen der Fortsetzung mir freilich je mehr zur Qual, je näher einem diese
               Menschen kommen.\pend
           
\pstart
           {\pb}Ich bin seit \label{K_L03491-6v}\edtext{Donnerstag voriger Woche}{\lemma{\textnormal{\emph{Donnerstag voriger Woche}}}\Cendnote{\textnormal{Er dürfte vom 23. 1. 1908
                  sprechen und sich also seit vier Tagen am Semmering\oindex{Semmering@\textbf{Semmering}, \emph{A.ADM3}|pwk} 
                  aufhalten. Vgl. Felix Salten an Arthur Schnitzler, 16. 1. 1908.}}}\label{K_L03491-6}{ }hier oben\oindex{Semmering@\textbf{Semmering}, \emph{A.ADM3}|pwv}; traf hier Frau Kainz\pwindex{Kainz, Margarethe 13.12.1858 – 12.02.1950@\textsc{Kainz, Margarethe} (13.12.1858 – 12.02.1950), \emph{Schauspieler/Schauspielerin}|pw} mit Frau Schlenther\pwindex{Schlenther, Paula 27.02.1860 – 09.08.1938@\textsc{Schlenther, Paula} (27.02.1860 – 09.08.1938), \emph{Schauspieler/Schauspielerin}|pw}, mit der ich komischerweise sehr sympathisirte. (Nett hat sich
                  \label{K_L03491-7v}\edtext{Schlenther\pwindex{Schlenther, Paul 20.08.1854 – 30.04.1916@\textsc{Schlenther, Paul} (20.08.1854 – 30.04.1916), \emph{Schriftsteller/Schriftstellerin, Kritiker/Kritikerin, Theaterleiter/Theaterleiterin}|pw} in der Preis\orgindex{Franz-Grillparzer-Preis@Franz-Grillparzer-Preis|pw}-Angelegenheit}{\lemma{\textnormal{\emph{Schlenther … Preis-Angelegenheit}}}\Cendnote{\textnormal{Siehe Felix Salten an Arthur Schnitzler, 15. 1. 1908.
               }}}\label{K_L03491-7} benommen) Samstag kam Otti\pwindex{Salten, Ottilie 07.03.1868 – 22.06.1942@\textsc{Salten, Ottilie} (07.03.1868 – 22.06.1942), \emph{Schauspieler/Schauspielerin}|pw} mit den Kindern\pwindex{Salten, Paul 11.08.1903 – 08.05.1937@\textsc{Salten, Paul} (11.08.1903 – 08.05.1937), \emph{Filmcutter/Filmcutterin}|pwv}\pwindex{Rehmann, Anna Katharina 18.08.1904 – 27.03.1977@\textsc{Rehmann, Anna Katharina} (18.08.1904 – 27.03.1977), \emph{Schauspieler/Schauspielerin, Übersetzer/Übersetzerin}|pwv}, Sonntag kamen Fischers\pwindex{Fischer, Samuel 24.12.1859 – 15.10.1934@\textsc{Fischer, Samuel} (24.12.1859 – 15.10.1934), \emph{Verleger/Verlegerin}|pw}\pwindex{Fischer, Hedwig 08.09.1871 – 11.04.1952@\textsc{Fischer, Hedwig} (08.09.1871 – 11.04.1952)|pw}, gestern u. heute ist der Kainz\pwindex{Kainz, Josef 02.01.1858 – 20.09.1910@\textsc{Kainz, Josef} (02.01.1858 – 20.09.1910), \emph{Schauspieler/Schauspielerin}|pw} dagewesen, und Herr Fred\pwindex{W. Fred 29.06.1879 – 23.10.1922@\textsc{W. Fred} (29.06.1879 – 23.10.1922), \emph{Schriftsteller/Schriftstellerin, Journalist/Journalistin}|pw} ist immer da. Ich arbeite ein bischen und spüre noch immer meine
               Darmzustände. – Hoffentlich sehen wir uns hier oben\oindex{Semmering@\textbf{Semmering}, \emph{A.ADM3}|pwv} oder in Wien\oindex{Wien@\textbf{Wien}, \emph{A.ADM2}|pw}.
               Ängstlich bin ich ja, das gebe ich zu. Sie wißen doch, dass ich wegen meiner Kinder\pwindex{Salten, Paul 11.08.1903 – 08.05.1937@\textsc{Salten, Paul} (11.08.1903 – 08.05.1937), \emph{Filmcutter/Filmcutterin}|pwv}\pwindex{Rehmann, Anna Katharina 18.08.1904 – 27.03.1977@\textsc{Rehmann, Anna Katharina} (18.08.1904 – 27.03.1977), \emph{Schauspieler/Schauspielerin, Übersetzer/Übersetzerin}|pwv} beständig in
               einer halbtollen Furcht lebe. Aber ich denke, wenn Sie Heini\pwindex{Schnitzler, Heinrich 09.08.1902 – 12.07.1982@\textsc{Schnitzler, Heinrich} (09.08.1902 – 12.07.1982), \emph{Regisseur/Regisseurin, Schauspieler/Schauspielerin}|pw} bei sich haben, ist wol nichts mehr zu besorgen.\pend
           
\pstart
           Also vieles Gute und Herzliche von uns\pwindex{Salten, Ottilie 07.03.1868 – 22.06.1942@\textsc{Salten, Ottilie} (07.03.1868 – 22.06.1942), \emph{Schauspieler/Schauspielerin}|pwv} zu Ihnen. Otti\pwindex{Salten, Ottilie 07.03.1868 – 22.06.1942@\textsc{Salten, Ottilie} (07.03.1868 – 22.06.1942), \emph{Schauspieler/Schauspielerin}|pw} u.
               ich laßen Frau Olga\pwindex{Schnitzler, Olga 17.01.1882 – 13.01.1970@\textsc{Schnitzler, Olga} (17.01.1882 – 13.01.1970), \emph{Schauspieler/Schauspielerin, Sänger/Sängerin}|pw} besonders grüßen.\pend
           
\pstart
           Ihr {\\[\baselineskip]}\spacefill\mbox{Salten}\pend
           \leftskip=0em{}\selectlanguage{ngerman}\endnumbering\briefempfaengerindex{Schnitzler, Arthur@\textsc{Schnitzler, Arthur}!zzzSalten, Felix@\emph{von Felix Salten}!1908-01-261@{26. 1. 1908}|)be}\mylabel{L03491h}  \normalsize

\doendnotes{C}
\bigskip
\vfill

\clearpage

\footnotesize

\lohead{\textsc{register}}

% Definiere theindex-Environment komplett neu ohne reledmac
\makeatletter
\renewenvironment{theindex}{%
  \section*{\indexname}%
  \setlength{\parindent}{0pt}%
  \setlength{\parskip}{0pt plus 0.3pt}%
  \let\item\@idxitem
}{%
  \clearpage
}
\makeatother

\IfFileExists{\jobname-pw.ind}{\input{\jobname-pw.ind}}{}

\end{document}

      