%% latex-leseansicht-vorspann.tex
%% Vorspann für die Leseansicht.
%% Lädt die gemeinsame Datei latex-vorspann.tex mit nicht gesetztem Schalter.

\newif\ifkorrekturansicht
\korrekturansichtfalse

\input{../tex-inputs/latex-vorspann}

\begin{center}
            \textcolor{red}{ENTWURF, NICHT FERTIG KORRIGIERT}
                      \end{center}
            
         
         \renewcommand{\erwaehntePersonen}{Personen: Samuel Fischer, Hedwig Fischer, Margarethe Kainz, Josef Kainz, Anna Katharina Rehmann, Felix Salten, Ottilie Salten, Paul Salten, Paula Schlenther, Paul Schlenther, Olga Schnitzler, Heinrich Schnitzler,  W. Fred}
         \renewcommand{\erwaehnteInstitutionen}{Institutionen: Franz-Grillparzer-Preis}
         \renewcommand{\erwaehnteOrte}{Orte: Semmering, Südbahnhotel, Wien, Österreich}
         \renewcommand{\erwaehnteWerke}{Werke: Der Weg ins Freie. Roman, Die neue Rundschau, Tagebuch}
               \section[ Felix Salten an Arthur Schnitzler, 26. 1. 1908]{ Felix Salten an Arthur Schnitzler, 26. 1. 1908}\nopagebreak\mylabel{v}\rehead{ }\begin{ledgroupsized}[t]{13cm}\normalsize\beginnumbering \toendnotes[C]{\smallbreak\pagebreak[2]} \Standort{CUL, Schnitzler, B 89, B 1.}
\physDesc{Brief, 1 Blatt, 2 Seiten, 2088 Zeichen
\newline{}Handschrift: Bleistift, lateinische Kurrent
\newline{}Schnitzler: mit Bleistift Vermerk »\textsc{Salt{[}en{]}}« 
\newline{}Ordnung: mit Bleistift von unbekannter Hand nummeriert: »241« }\toendnotes[C]{\smallbreak}\pstart
           \noindent{}{\pb}\textcolor{gray}{\textbf{Südbahn-Hôtel\oindex{Suedbahnhotel@\textbf{Südbahnhotel}|pw}}}\pend
           \pstart
           \textcolor{gray}{\textbf{Semmering\oindex{Semmering@\textbf{Semmering}|pw}}}\hfill 26./1. 08\pend
           \pstart
           \textcolor{gray}{\textbf{Austria\oindex{Oesterreich@\textbf{Österreich}|pw}.}}\pend
           \pstart
           \textcolor{gray}{\textbf{\textsc{\textbf{Telegramme:}}}}\pend
           \pstart
           \textcolor{gray}{\textbf{\textsc{\textbf{Südbahnhôtel Semmering\oindex{Suedbahnhotel@\textbf{Südbahnhotel}|pw}.}}}}\pend
           \pstart
           \textcolor{gray}{\textbf{\textsc{Telephon:}}}\pend
           \pstart
           \textcolor{gray}{\textbf{\textsc{Hôtel {\dotsfour} Nr. 5.}}}\pend
           \pstart
           \textcolor{gray}{\textbf{\textsc{Dependance Nr. 6.}}}\pend
           \pstart{}Lieber,\pend\pstart
           danke sehr für Ihren ausführlichen \label{K_L03491-1v}\edtext{Brief}{\lemma{\textnormal{\emph{Brief}}}\Cendnote{\textnormal{Arthur Schnitzler an Felix Salten, 25. 1. 1908}}}\label{K_L03491-1h}, der mich sehr gefreut hat. Den letzten Satz, da wo Sie sagen, dass Sie sich
               wieder »keck mitten ins Leben« u. s. w. habe ich, wie ich Ihnen gestehen muss, mit
               einer plötzlich aufsteigenden, sehr starken Ergriffenheit gelesen. Denn aus ihm sah
               ich erst ganz deutlich, \uline{wo} Sie in dieser letzten Zeit
               mit Ihren Gedanken und Sorgen gewesen sind, und was Sie durchgemacht haben. Nun aber
               dürfen Sie sich wol freuen und Ihre Freunde mit Ihnen. Wundervoll ist es ja, wie
               diese \label{K_L03491-2v}\edtext{Gefahr}{\lemma{\textnormal{\emph{Gefahr}}}\Cendnote{\textnormal{siehe Felix Salten an Arthur Schnitzler, [10. 12. 1907]}}}\label{K_L03491-2h} an Ihnen u. Ihrer Frau\pwindex{Schnitzler, Olga 17.01.1882 – 13.01.1970@\textsc{Schnitzler, Olga} (17.01.1882 – 13.01.1970), \emph{Schauspielerin, Sängerin}|pwv} vorbeigeschwebt ist, und wie dann mit dem Grillparzer Preis\orgindex{Franz-Grillparzer-Preis@Franz-Grillparzer-Preis|pw} etwas zu Ihnen kam, was schließlich doch im
               Tiefsten so etwas wie einen Schimmer von Glück bedeutet. Wir gehen dem Frühling
               entgegen, und Ihre Frau\pwindex{Schnitzler, Olga 17.01.1882 – 13.01.1970@\textsc{Schnitzler, Olga} (17.01.1882 – 13.01.1970), \emph{Schauspielerin, Sängerin}|pwv} wird
               sich hoffentlich rasch erholen. Man sagt ja, dass nach dem Scharlach die Gesundheit
               intensiver wird, und so wird Frau Olga\pwindex{Schnitzler, Olga 17.01.1882 – 13.01.1970@\textsc{Schnitzler, Olga} (17.01.1882 – 13.01.1970), \emph{Schauspielerin, Sängerin}|pw} jetzt in
               ein schönes Genesen und Glühen kommen, und mit der Jahreszeit gehen. Besseres läßt
               sich kaum denken. Ihren \label{K_L03491-3v}\edtext{Roman\pwindex{Schnitzler, Arthur 15.05.1862 – 21.10.1931@\textsc{Schnitzler, Arthur} (15.05.1862 – 21.10.1931), \emph{Schriftsteller, Mediziner}!Weg ins Freie. Roman1.1.1908 – 1.6.1908@\strich\emph{Der Weg ins Freie. Roman} {[}1.1.1908 – 1.6.1908{]}|pwv} las ich nun doch in den
               ersten zwei Fortsetzungen}{\lemma{\textnormal{\emph{Roman … Fortsetzungen}}}\Cendnote{\textnormal{siehe Felix Salten an Arthur Schnitzler, 16. 1. 1908}}}\label{K_L03491-3h}. Sie werden meine Neugierde begreifen u. entschuldigen. \label{K_L03491-4v}\edtext{Sagen kann ich jetzt natürlich noch
                  nichts}{\lemma{\textnormal{\emph{Sagen … nichts}}}\Cendnote{\textnormal{Nachdem sie sich wenige Tage
                  später, am 4. 2. 1908, auf dem Weg zum Semmering\oindex{Semmering@\textbf{Semmering}|pwk} getroffen
                  hatten, notierte Schnitzler\pwindex{Schnitzler, Arthur 15.05.1862 – 21.10.1931@\textsc{Schnitzler, Arthur} (15.05.1862 – 21.10.1931), \emph{Schriftsteller, Mediziner}|pwk} in seinem \emph{Tagebuch}\pwindex{\textcolor{red}{\textsuperscript{XXXX1 indx}}!Tagebuch1981 – 2000@\strich\emph{Tagebuch} {[}Hrsg., 1981 – 2000{]}|pwk}: »Er [ = Salten\pwindex{Salten, Felix 06.09.1869 – 08.10.1945@\textsc{Salten, Felix} (06.09.1869 – 08.10.1945), \emph{Schriftsteller, Journalist}|pwk}] sagt über einen Roman\pwindex{Schnitzler, Arthur 15.05.1862 – 21.10.1931@\textsc{Schnitzler, Arthur} (15.05.1862 – 21.10.1931), \emph{Schriftsteller, Mediziner}!Weg ins Freie. Roman1.1.1908 – 1.6.1908@\strich\emph{Der Weg ins Freie. Roman} {[}1.1.1908 – 1.6.1908{]}|pwkv}, dessen 2 erste Theile (Jänner-, Feberheft\pwindex{?? Werk@Nicht ermittelte Verfasserinnen und Verfasser!neue Rundschau1904@\emph{Die neue Rundschau} {[}1904{]}|pwkv}) er gelesen: Sehr
                  lebendige Gestalten. Dann (zögernd) … ›Aber es hat mir erst recht leid gethan,
                  dass ich’s nicht im Manuscript gelesen … es sind stilistische (Fehler?) Mängel,
                  Härten (erinner mich des Worts nicht) – wie sie natürlich bei einem so großen Werk\pwindex{Schnitzler, Arthur 15.05.1862 – 21.10.1931@\textsc{Schnitzler, Arthur} (15.05.1862 – 21.10.1931), \emph{Schriftsteller, Mediziner}!Weg ins Freie. Roman1.1.1908 – 1.6.1908@\strich\emph{Der Weg ins Freie. Roman} {[}1.1.1908 – 1.6.1908{]}|pwkv} nicht zu vermeiden
                  sind.–‹ Es ärgerte, ja empörte mich beinahe – obwohl, oder weil ich darauf
                  vorbereitet war.– ›Er wird nicht wollen‹ sagte ich neulich.– Wer wird wollen –?‹«
                  Diese Kritik Salten\pwindex{Salten, Felix 06.09.1869 – 08.10.1945@\textsc{Salten, Felix} (06.09.1869 – 08.10.1945), \emph{Schriftsteller, Journalist}|pwk}s sollte Schnitzler\pwindex{Schnitzler, Arthur 15.05.1862 – 21.10.1931@\textsc{Schnitzler, Arthur} (15.05.1862 – 21.10.1931), \emph{Schriftsteller, Mediziner}|pwk} noch lange beschäftigen. Siehe
                  etwa A. S.: \emph{Tagebuch}, 28. 4. 1908.}}}\label{K_L03491-4h}, ahne auch nur
               von weitem, wohin der Weg ins
                  Freie\pwindex{Schnitzler, Arthur 15.05.1862 – 21.10.1931@\textsc{Schnitzler, Arthur} (15.05.1862 – 21.10.1931), \emph{Schriftsteller, Mediziner}!Weg ins Freie. Roman1.1.1908 – 1.6.1908@\strich\emph{Der Weg ins Freie. Roman} {[}1.1.1908 – 1.6.1908{]}|pwv} führt. Aber eine Menge Menschen wird mir jetzt schon sehr lebendig und
               das Abreißen der Fortsetzung mir freilich je mehr zur Qual, je näher einem diese
               Menschen kommen.\pend
           \pstart
           {\pb}Ich bin seit \label{K_L03491-5v}\edtext{Donnerstag voriger Woche}{\lemma{\textnormal{\emph{Donnerstag voriger Woche}}}\Cendnote{\textnormal{Salten\pwindex{Salten, Felix 06.09.1869 – 08.10.1945@\textsc{Salten, Felix} (06.09.1869 – 08.10.1945), \emph{Schriftsteller, Journalist}|pwk} dürfte seine Pläne kurzfristig
                  geändert haben, hatte er doch am 16. 1. 1908 noch geschrieben, dass er erst »voraussichtlich Sonntag oder Montag auf
                  den Semmering\oindex{Semmering@\textbf{Semmering}|pwk}« fahren wolle. Der 23. 1. 1908 kann durch die folgenden Ausführungen
                  ausgeschlossen werden.}}}\label{K_L03491-5h}{ }hier oben\oindex{Semmering@\textbf{Semmering}|pwv}; traf hier Frau Kainz\pwindex{Kainz, Margarethe 13.12.1858 – 12.02.1950@\textsc{Kainz, Margarethe} (13.12.1858 – 12.02.1950), \emph{Schauspielerin}|pw} mit Frau Schlenther\pwindex{Schlenther, Paula 27.02.1860 – 09.08.1938@\textsc{Schlenther, Paula} (27.02.1860 – 09.08.1938), \emph{Schauspielerin}|pw}, mit der ich komischerweise sehr sympathisirte. (Nett hat sich
                  \label{K_L03491-6v}\edtext{Schlenther\pwindex{Schlenther, Paul 20.08.1854 – 30.04.1916@\textsc{Schlenther, Paul} (20.08.1854 – 30.04.1916), \emph{Schriftsteller, Kritiker, Theaterleiter}|pw} in der Preis\orgindex{Franz-Grillparzer-Preis@Franz-Grillparzer-Preis|pw}-Angelegenheit}{\lemma{\textnormal{\emph{Schlenther … Preis-Angelegenheit}}}\Cendnote{\textnormal{siehe Felix Salten an Arthur Schnitzler, 15. 1. 1908}}}\label{K_L03491-6h} benommen) Samstag kam Otti\pwindex{Salten, Ottilie 07.03.1868 – 22.06.1942@\textsc{Salten, Ottilie} (07.03.1868 – 22.06.1942), \emph{Schauspielerin}|pw} mit den Kindern\pwindex{Salten, Paul 11.08.1903 – 08.05.1937@\textsc{Salten, Paul} (11.08.1903 – 08.05.1937), \emph{Filmcutter}|pwv}\pwindex{Rehmann, Anna Katharina 18.08.1904 – 27.03.1977@\textsc{Rehmann, Anna Katharina} (18.08.1904 – 27.03.1977), \emph{Schauspielerin, Übersetzerin}|pwv}, Sonntag kamen Fischers\pwindex{Fischer, Samuel 24.12.1859 – 15.10.1934@\textsc{Fischer, Samuel} (24.12.1859 – 15.10.1934), \emph{Verleger}|pw}\pwindex{Fischer, Hedwig 08.09.1871 – 11.04.1952@\textsc{Fischer, Hedwig} (08.09.1871 – 11.04.1952)|pw}, gestern u. heute ist der Kainz\pwindex{Kainz, Josef 02.01.1858 – 20.09.1910@\textsc{Kainz, Josef} (02.01.1858 – 20.09.1910), \emph{Schauspieler}|pw} dagewesen, und Herr Fred\pwindex{W. Fred 29.06.1879 – 23.10.1922@\textsc{W. Fred} (29.06.1879 – 23.10.1922), \emph{Schriftsteller, Journalist}|pw} ist immer da. Ich arbeite ein bischen und spüre noch immer meine
               Darmzustände. – Hoffentlich sehen wir uns hier oben\oindex{Semmering@\textbf{Semmering}|pwv} oder in Wien\oindex{Wien@\textbf{Wien}|pw}.
               Ängstlich bin ich ja, das gebe ich zu. Sie wißen doch, dass ich wegen meiner Kinder\pwindex{Salten, Paul 11.08.1903 – 08.05.1937@\textsc{Salten, Paul} (11.08.1903 – 08.05.1937), \emph{Filmcutter}|pwv}\pwindex{Rehmann, Anna Katharina 18.08.1904 – 27.03.1977@\textsc{Rehmann, Anna Katharina} (18.08.1904 – 27.03.1977), \emph{Schauspielerin, Übersetzerin}|pwv} beständig in
               einer halbtollen Furcht lebe. Aber ich denke, wenn Sie Heini\pwindex{Schnitzler, Heinrich 09.08.1902 – 12.07.1982@\textsc{Schnitzler, Heinrich} (09.08.1902 – 12.07.1982), \emph{Regisseur, Schauspieler}|pw} bei sich haben, ist wol nichts mehr zu
                  be\textcolor{gray}{sor}gen.\pend
           \pstart
           Also vieles Gute und Herzliche von uns\pwindex{Salten, Ottilie 07.03.1868 – 22.06.1942@\textsc{Salten, Ottilie} (07.03.1868 – 22.06.1942), \emph{Schauspielerin}|pwv} zu Ihnen. Otti\pwindex{Salten, Ottilie 07.03.1868 – 22.06.1942@\textsc{Salten, Ottilie} (07.03.1868 – 22.06.1942), \emph{Schauspielerin}|pw} u.
               ich laßen Frau Olga\pwindex{Schnitzler, Olga 17.01.1882 – 13.01.1970@\textsc{Schnitzler, Olga} (17.01.1882 – 13.01.1970), \emph{Schauspielerin, Sängerin}|pw} besonders grüßen.\pend
           \pstart
           Ihr {\\[\baselineskip]}\spacefill\mbox{Salten}\pend
           \leftskip=0em{}
         
         \endnumbering\mylabel{h}\end{ledgroupsized}  \newcommand{\dateiname}{L03491}\newcommand{\titel}{Felix Salten an Arthur Schnitzler, 26. 1. 1908}\newcommand{\editorInnen}{Martin Anton Müller und Laura Untner}%% latex-leseansicht-abspann.tex
%% Abspann für die Leseansicht.
%% Der Schalter \ifkorrekturansicht ist bereits durch den Vorspann gesetzt.

%% latex-abspann.tex
%% Gemeinsamer Abspann für Korrekturansicht und Leseansicht.
%% Setzt den Schalter \ifkorrekturansicht voraus (gesetzt in den
%% einbindenden Dateien latex-korrekturansicht-abspann.tex bzw.
%% latex-leseansicht-abspann.tex).
%% ---------------------------------------------------------------

\normalsize

% Das esempio-Environment wird nur in der Leseansicht benötigt
\ifkorrekturansicht\else
\newenvironment{esempio}[3]%
{
    \vspace{1.5ex}
    \rlap{\underline{#1}}
    \par
    \setlength{\parindent}{0cm}
    \nopagebreak
    \leftskip=#2cm
    \rightskip=#3cm
}
{
    \par
}
\fi

\doendnotes{C}
\bigskip
\vfill

\clearpage

\footnotesize

\ifkorrekturansicht
  \lohead{\textsc{register}}
\fi

% theindex-Environment neu definieren ohne reledmac
\makeatletter
\renewenvironment{theindex}{%
  \ifkorrekturansicht
    \section*{\indexname}%
  \else
    \subsubsection*{Index der erwähnten Entitäten}%
  \fi
  \setlength{\parindent}{0pt}%
  \setlength{\parskip}{0pt plus 0.3pt}%
  \let\item\@idxitem
}{%
  \ifkorrekturansicht\clearpage\fi
}
\makeatother

\IfFileExists{\jobname-pw.ind}{\input{\jobname-pw.ind}}{}

% Quellenangabe nur in der Leseansicht
\ifkorrekturansicht\else
% Fallback-Definitionen, falls die .tex-Datei \titel etc. nicht gesetzt hat
\providecommand{\titel}{}
\providecommand{\editorInnen}{}
\providecommand{\dateiname}{\jobname}

\vspace{3cm}

\vfill

\footnotesize
\textsc{Quelle}: \titel. Herausgegeben von {\editorInnen}. In: \emph{Arthur Schnitzler: Briefwechsel mit Autorinnen und Autoren}.
 Digitale Edition, https://schnitzler-briefe.acdh.oeaw.ac.at/{\dateiname}.html (Stand \today)
\fi

\end{document}


      