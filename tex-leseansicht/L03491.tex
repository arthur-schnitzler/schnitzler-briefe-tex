%% latex-leseansicht-vorspann.tex
%% Vorspann für die Leseansicht.
%% Lädt die gemeinsame Datei latex-vorspann.tex mit nicht gesetztem Schalter.

\newif\ifkorrekturansicht
\korrekturansichtfalse

\input{../tex-inputs/latex-vorspann}


\section[ Felix Salten an Arthur Schnitzler, 26. 1. 1908]{L03491 Felix Salten an Arthur Schnitzler,  26. 1. 1908}
\nopagebreak\mylabel{L03491v}
\rehead{ }\normalsize\beginnumbering\briefempfaengerindex{Schnitzler, Arthur@\textsc{Schnitzler, Arthur}!zzzSalten, Felix@\emph{von Felix Salten}!1908-01-261@{26. 1. 1908}|(be}
\toendnotes[C]{\smallbreak\pagebreak[2]}
\correspDesc{Versand  durch Felix Salten am 26. 1. 1908 in Semmering
\newline{}Erhalt  durch Arthur Schnitzler im Zeitraum [27. 1. 1908
                  – 31. 1. 1908?] in Wien}\toendnotes[C]{\smallbreak}
\Standort{CUL, Schnitzler, B 89, B 1.}
\physDesc{Brief, 1 Blatt, 2 Seiten, 2083 Zeichen
\newline{}Handschrift: Bleistift, lateinische Kurrent
\newline{}Schnitzler: mit Bleistift Vermerk »\textsc{Salt\textcolor{gray}{en}}« 
\newline{}Ordnung: mit Bleistift von unbekannter Hand nummeriert: »241« }\toendnotes[C]{\smallbreak}
\pstart
           {\pb}\textcolor{gray}{\textbf{Südbahn-Hôtel\oindex{Südbahnhotel [Semmering]@\textbf{Südbahnhotel [Semmering]}, \emph{Hotel}|pw}}}\pend
           
\pstart
           \textcolor{gray}{\textbf{Semmering\oindex{Semmering@\textbf{Semmering}, \emph{Verwaltungsgebiet}|pw}}}\hfill 26./1. 08\pend
           
\pstart
           \textcolor{gray}{\textbf{Austria\oindex{Österreich@\textbf{Österreich}|pw}.}}\pend
           
\pstart
           \textcolor{gray}{\textbf{\textbf{TELEGRAMME:}}}\pend
           
\pstart
           \textcolor{gray}{\textbf{\textbf{SÜDBAHNHÔTEL SEMMERING\oindex{Südbahnhotel [Semmering]@\textbf{Südbahnhotel [Semmering]}, \emph{Hotel}|pw}.}}}\pend
           
\pstart
           \textcolor{gray}{\textbf{TELEPHON:}}\pend
           
\pstart
           \textcolor{gray}{\textbf{HÔTEL {\dotsfour} NR. 5.}}\pend
           
\pstart
           \textcolor{gray}{\textbf{DEPENDANCE NR. 6.}}\pend
           
\pstart{}Lieber,\pend\vspace{0.5em}
\pstart
           danke sehr für Ihren ausführlichen \label{K_L03491-1v}\edtext{Brief}{\lemma{\textnormal{\emph{Brief}}}\Cendnote{\textnormal{XXXX Auszeichnungsfehler: Dokument L03011 nicht gefunden.
               }}}\label{K_L03491-1}, der mich sehr gefreut hat. Den letzten Satz, da wo Sie sagen, dass Sie sich
               wieder »\label{K_L03491-2v}\edtext{keck mitten ins Leben}{\lemma{\textnormal{\emph{keck mitten ins Leben}}}\Cendnote{\textnormal{Schnitzler schrieb »frech wieder
                     mitten ins Leben hinein«.}}}\label{K_L03491-2}« u. s. w. habe ich, wie ich Ihnen
               gestehen muss, mit einer plötzlich aufsteigenden, sehr starken Ergriffenheit gelesen.
               Denn aus ihm sah ich erst ganz deutlich, \uline{wo} Sie in
               dieser letzten Zeit mit Ihren Gedanken und Sorgen gewesen sind, und was Sie
               durchgemacht haben. Nun aber dürfen Sie sich wol freuen und Ihre Freunde mit Ihnen.
               Wundervoll ist es ja, wie diese \label{K_L03491-3v}\edtext{Gefahr}{\lemma{\textnormal{\emph{Gefahr}}}\Cendnote{\textnormal{Siehe XXXX Auszeichnungsfehler: Dokument L03494 nicht gefunden.
               }}}\label{K_L03491-3} an Ihnen u. Ihrer Frau\pwindex{Schnitzler, Olga 17.\,1.\,1882 Wien – 13.\,1.\,1970 Lugano@\textsc{Schnitzler, Olga} (17.\,1.\,1882 Wien – 13.\,1.\,1970 Lugano), \emph{Schauspielerin, Sängerin}|pwv} vorbeigeschwebt ist, und wie dann mit dem Grillparzer Preis\orgindex{Franz-Grillparzer-Preis@Franz-Grillparzer-Preis|pw} etwas zu Ihnen kam, was schließlich doch im
               Tiefsten so etwas wie einen Schimmer von Glück bedeutet. Wir gehen dem Frühling
               entgegen, und Ihre Frau\pwindex{Schnitzler, Olga 17.\,1.\,1882 Wien – 13.\,1.\,1970 Lugano@\textsc{Schnitzler, Olga} (17.\,1.\,1882 Wien – 13.\,1.\,1970 Lugano), \emph{Schauspielerin, Sängerin}|pwv} wird
               sich hoffentlich rasch erholen. Man sagt ja, dass nach dem Scharlach die Gesundheit
               intensiver wird, und so wird Frau Olga\pwindex{Schnitzler, Olga 17.\,1.\,1882 Wien – 13.\,1.\,1970 Lugano@\textsc{Schnitzler, Olga} (17.\,1.\,1882 Wien – 13.\,1.\,1970 Lugano), \emph{Schauspielerin, Sängerin}|pw} jetzt in
               ein schönes Genesen und Glühen kommen, und mit der Jahreszeit gehen. Besseres läßt
               sich kaum denken. Ihren \label{K_L03491-4v}\edtext{Roman\pwindex{Schnitzler, Arthur 15.\,5.\,1862 Wien – 21.\,10.\,1931 ebd.@\textsc{Schnitzler, Arthur} (15.\,5.\,1862 Wien – 21.\,10.\,1931 ebd.), \emph{Schriftsteller, Mediziner}!Weg ins Freie. Roman@\strich\emph{Der Weg ins Freie. Roman}|pwv} las ich nun doch in den
               ersten zwei Fortsetzungen}{\lemma{\textnormal{\emph{Roman … Fortsetzungen}}}\Cendnote{\textnormal{Vgl. XXXX Auszeichnungsfehler: Dokument L03509 nicht gefunden. Saltens\pwindex{Salten, Felix 6.\,9.\,1869 Budapest – 8.\,10.\,1945 Zürich@\textsc{Salten, Felix} (6.\,9.\,1869 Budapest – 8.\,10.\,1945 Zürich), \emph{Schriftsteller, Journalist, Chefredakteur}|pwk} Lektüre der ersten Fortsetzung bedeutet, dass das
                  Monatsheft des Februar bereits vorzeitig ausgeliefert wurde.}}}\label{K_L03491-4}. Sie werden
               meine Neugierde begreifen u. entschuldigen. \label{K_L03491-5v}\edtext{Sagen kann ich jetzt natürlich noch nichts}{\lemma{\textnormal{\emph{Sagen … nichts}}}\Cendnote{\textnormal{Nachdem sie sich wenige Tage später, am
                     4. 2. 1908, auf
                  dem Weg zum Semmering\oindex{Semmering@\textbf{Semmering}, \emph{Verwaltungsgebiet}|pwk} getroffen hatten,
                  notierte Schnitzler in seinem \emph{Tagebuch}\pwindex{Schnitzler, Arthur 15.\,5.\,1862 Wien – 21.\,10.\,1931 ebd.@\textsc{Schnitzler, Arthur} (15.\,5.\,1862 Wien – 21.\,10.\,1931 ebd.), \emph{Schriftsteller, Mediziner}!Tagebuch@\strich\emph{Tagebuch}|pwk}: »Er [ = Salten\pwindex{Salten, Felix 6.\,9.\,1869 Budapest – 8.\,10.\,1945 Zürich@\textsc{Salten, Felix} (6.\,9.\,1869 Budapest – 8.\,10.\,1945 Zürich), \emph{Schriftsteller, Journalist, Chefredakteur}|pwk}] sagt über einen Roman\pwindex{Schnitzler, Arthur 15.\,5.\,1862 Wien – 21.\,10.\,1931 ebd.@\textsc{Schnitzler, Arthur} (15.\,5.\,1862 Wien – 21.\,10.\,1931 ebd.), \emph{Schriftsteller, Mediziner}!Weg ins Freie. Roman@\strich\emph{Der Weg ins Freie. Roman}|pwkv}, dessen 2 erste Theile (Jänner-, Feberheft\pwindex{neue Rundschau@\emph{Die neue Rundschau}|pwkv}) er gelesen: Sehr
                  lebendige Gestalten. Dann (zögernd) {\dots} ›Aber es hat mir erst recht leid gethan,
                  dass ich’s nicht im Manuscript gelesen {\dots} es sind stilistische (Fehler?) Mängel,
                  Härten (erinner mich des Worts nicht) – wie sie natürlich bei einem so großen Werk\pwindex{Schnitzler, Arthur 15.\,5.\,1862 Wien – 21.\,10.\,1931 ebd.@\textsc{Schnitzler, Arthur} (15.\,5.\,1862 Wien – 21.\,10.\,1931 ebd.), \emph{Schriftsteller, Mediziner}!Weg ins Freie. Roman@\strich\emph{Der Weg ins Freie. Roman}|pwkv} nicht zu vermeiden
                  sind. –‹ Es ärgerte, ja empörte mich beinahe – obwohl, oder weil ich darauf
                  vorbereitet war. – ›Er wird nicht wollen‹ sagte ich neulich. – Wer wird wollen –?‹«
                  Diese Kritik Saltens\pwindex{Salten, Felix 6.\,9.\,1869 Budapest – 8.\,10.\,1945 Zürich@\textsc{Salten, Felix} (6.\,9.\,1869 Budapest – 8.\,10.\,1945 Zürich), \emph{Schriftsteller, Journalist, Chefredakteur}|pwk} sollte Schnitzler noch lange beschäftigen,  vgl. A. S.: \emph{Tagebuch}, 28. 4. 1908.}}}\label{K_L03491-5}, ahne auch nur
               von weitem, wohin der Weg ins
                  Freie\pwindex{Schnitzler, Arthur 15.\,5.\,1862 Wien – 21.\,10.\,1931 ebd.@\textsc{Schnitzler, Arthur} (15.\,5.\,1862 Wien – 21.\,10.\,1931 ebd.), \emph{Schriftsteller, Mediziner}!Weg ins Freie. Roman@\strich\emph{Der Weg ins Freie. Roman}|pwv} führt. Aber eine Menge Menschen wird mir jetzt schon sehr lebendig und
               das Abreißen der Fortsetzung mir freilich je mehr zur Qual, je näher einem diese
               Menschen kommen.\pend
           
\pstart
           {\pb}Ich bin seit \label{K_L03491-6v}\edtext{Donnerstag voriger Woche}{\lemma{\textnormal{\emph{Donnerstag voriger Woche}}}\Cendnote{\textnormal{Er dürfte vom 23. 1. 1908
                  sprechen und sich also seit vier Tagen am Semmering\oindex{Semmering@\textbf{Semmering}, \emph{Verwaltungsgebiet}|pwk} 
                  aufhalten. Vgl. XXXX Auszeichnungsfehler: Dokument L03509 nicht gefunden.}}}\label{K_L03491-6}{ }hier oben\oindex{Semmering@\textbf{Semmering}, \emph{Verwaltungsgebiet}|pwv}; traf hier Frau Kainz\pwindex{Kainz, Margarethe 31.\,12.\,1885 Berlin – 12.\,2.\,1950 Wien@\textsc{Kainz, Margarethe} (31.\,12.\,1885 Berlin – 12.\,2.\,1950 Wien), \emph{Schauspielerin}|pw} mit Frau Schlenther\pwindex{Schlenther, Paula 27.\,2.\,1860 Wien – 9.\,8.\,1938 Berlin@\textsc{Schlenther, Paula} (27.\,2.\,1860 Wien – 9.\,8.\,1938 Berlin), \emph{Schauspielerin}|pw}, mit der ich komischerweise sehr sympathisirte. (Nett hat sich
                  \label{K_L03491-7v}\edtext{Schlenther\pwindex{Schlenther, Paul 20.\,8.\,1854 Chernyakhovsk – 30.\,4.\,1916 Berlin@\textsc{Schlenther, Paul} (20.\,8.\,1854 Chernyakhovsk – 30.\,4.\,1916 Berlin), \emph{Schriftsteller, Kritiker, Theaterleiter}|pw} in der Preis\orgindex{Franz-Grillparzer-Preis@Franz-Grillparzer-Preis|pw}-Angelegenheit}{\lemma{\textnormal{\emph{Schlenther … Preis-Angelegenheit}}}\Cendnote{\textnormal{Siehe XXXX Auszeichnungsfehler: Dokument L03490 nicht gefunden.
               }}}\label{K_L03491-7} benommen) Samstag kam Otti\pwindex{Salten, Ottilie 7.\,3.\,1868 Prag – 22.\,6.\,1942 Zürich@\textsc{Salten, Ottilie} (7.\,3.\,1868 Prag – 22.\,6.\,1942 Zürich), \emph{Schauspielerin}|pw} mit den Kindern\pwindex{Salten, Paul 11.\,8.\,1903 Wien – 8.\,5.\,1937 ebd.@\textsc{Salten, Paul} (11.\,8.\,1903 Wien – 8.\,5.\,1937 ebd.), \emph{Filmcutter}|pwv}\pwindex{Rehmann, Anna Katharina 18.\,8.\,1904 Wien – 27.\,3.\,1977 Zürich@\textsc{Rehmann, Anna Katharina} (18.\,8.\,1904 Wien – 27.\,3.\,1977 Zürich), \emph{Schauspielerin, Übersetzerin}|pwv}, Sonntag kamen Fischers\pwindex{Fischer, Samuel 24.\,12.\,1859 Liptovský Mikuláš – 15.\,10.\,1934 Berlin@\textsc{Fischer, Samuel} (24.\,12.\,1859 Liptovský Mikuláš – 15.\,10.\,1934 Berlin), \emph{Verleger}|pw}\pwindex{Fischer, Hedwig 8.\,9.\,1871 Szczecin – 11.\,4.\,1952 Königstein im Taunus@\textsc{Fischer, Hedwig} (8.\,9.\,1871 Szczecin – 11.\,4.\,1952 Königstein im Taunus)|pw}, gestern u. heute ist der Kainz\pwindex{Kainz, Josef 2.\,1.\,1858 Mosonmagyaróvár – 20.\,9.\,1910 Wien@\textsc{Kainz, Josef} (2.\,1.\,1858 Mosonmagyaróvár – 20.\,9.\,1910 Wien), \emph{Schauspieler}|pw} dagewesen, und Herr Fred\pwindex{W. Fred 29.\,6.\,1879 Wien – 23.\,10.\,1922 Berlin@\textsc{W. Fred} (29.\,6.\,1879 Wien – 23.\,10.\,1922 Berlin), \emph{Schriftsteller, Journalist}|pw} ist immer da. Ich arbeite ein bischen und spüre noch immer meine
               Darmzustände. – Hoffentlich sehen wir uns hier oben\oindex{Semmering@\textbf{Semmering}, \emph{Verwaltungsgebiet}|pwv} oder in Wien\oindex{Wien@\textbf{Wien}, \emph{Verwaltungsgebiet}|pw}.
               Ängstlich bin ich ja, das gebe ich zu. Sie wißen doch, dass ich wegen meiner Kinder\pwindex{Salten, Paul 11.\,8.\,1903 Wien – 8.\,5.\,1937 ebd.@\textsc{Salten, Paul} (11.\,8.\,1903 Wien – 8.\,5.\,1937 ebd.), \emph{Filmcutter}|pwv}\pwindex{Rehmann, Anna Katharina 18.\,8.\,1904 Wien – 27.\,3.\,1977 Zürich@\textsc{Rehmann, Anna Katharina} (18.\,8.\,1904 Wien – 27.\,3.\,1977 Zürich), \emph{Schauspielerin, Übersetzerin}|pwv} beständig in
               einer halbtollen Furcht lebe. Aber ich denke, wenn Sie Heini\pwindex{Schnitzler, Heinrich 9.\,8.\,1902 Hinterbrühl – 12.\,7.\,1982 Wien@\textsc{Schnitzler, Heinrich} (9.\,8.\,1902 Hinterbrühl – 12.\,7.\,1982 Wien), \emph{Regisseur, Schauspieler}|pw} bei sich haben, ist wol nichts mehr zu besorgen.\pend
           
\pstart
           Also vieles Gute und Herzliche von uns\pwindex{Salten, Ottilie 7.\,3.\,1868 Prag – 22.\,6.\,1942 Zürich@\textsc{Salten, Ottilie} (7.\,3.\,1868 Prag – 22.\,6.\,1942 Zürich), \emph{Schauspielerin}|pwv} zu Ihnen. Otti\pwindex{Salten, Ottilie 7.\,3.\,1868 Prag – 22.\,6.\,1942 Zürich@\textsc{Salten, Ottilie} (7.\,3.\,1868 Prag – 22.\,6.\,1942 Zürich), \emph{Schauspielerin}|pw} u.
               ich laßen Frau Olga\pwindex{Schnitzler, Olga 17.\,1.\,1882 Wien – 13.\,1.\,1970 Lugano@\textsc{Schnitzler, Olga} (17.\,1.\,1882 Wien – 13.\,1.\,1970 Lugano), \emph{Schauspielerin, Sängerin}|pw} besonders grüßen.\pend
           
\pstart
           Ihr {\\[\baselineskip]}\spacefill\mbox{Salten}\pend
           \leftskip=0em{}\selectlanguage{ngerman}\endnumbering\briefempfaengerindex{Schnitzler, Arthur@\textsc{Schnitzler, Arthur}!zzzSalten, Felix@\emph{von Felix Salten}!1908-01-261@{26. 1. 1908}|)be}\mylabel{L03491h}  \newcommand{\dateiname}{L03491}\newcommand{\titel}{Felix Salten an Arthur Schnitzler, 26. 1. 1908}\newcommand{\editorInnen}{Martin Anton Müller und Laura Untner}%% latex-leseansicht-abspann.tex
%% Abspann für die Leseansicht.
%% Der Schalter \ifkorrekturansicht ist bereits durch den Vorspann gesetzt.

%% latex-abspann.tex
%% Gemeinsamer Abspann für Korrekturansicht und Leseansicht.
%% Setzt den Schalter \ifkorrekturansicht voraus (gesetzt in den
%% einbindenden Dateien latex-korrekturansicht-abspann.tex bzw.
%% latex-leseansicht-abspann.tex).
%% ---------------------------------------------------------------

\normalsize

% Das esempio-Environment wird nur in der Leseansicht benötigt
\ifkorrekturansicht\else
\newenvironment{esempio}[3]%
{
    \vspace{1.5ex}
    \rlap{\underline{#1}}
    \par
    \setlength{\parindent}{0cm}
    \nopagebreak
    \leftskip=#2cm
    \rightskip=#3cm
}
{
    \par
}
\fi

\doendnotes{C}
\bigskip
\vfill

\clearpage

\footnotesize

\ifkorrekturansicht
  \lohead{\textsc{register}}
\fi

% theindex-Environment neu definieren ohne reledmac
\makeatletter
\renewenvironment{theindex}{%
  \ifkorrekturansicht
    \section*{\indexname}%
  \else
    \subsubsection*{Index der erwähnten Entitäten}%
  \fi
  \setlength{\parindent}{0pt}%
  \setlength{\parskip}{0pt plus 0.3pt}%
  \let\item\@idxitem
}{%
  \ifkorrekturansicht\clearpage\fi
}
\makeatother

\IfFileExists{\jobname-pw.ind}{\input{\jobname-pw.ind}}{}

% Quellenangabe nur in der Leseansicht
\ifkorrekturansicht\else
% Fallback-Definitionen, falls die .tex-Datei \titel etc. nicht gesetzt hat
\providecommand{\titel}{}
\providecommand{\editorInnen}{}
\providecommand{\dateiname}{\jobname}

\vspace{3cm}

\vfill

\footnotesize
\textsc{Quelle}: \titel. Herausgegeben von {\editorInnen}. In: \emph{Arthur Schnitzler: Briefwechsel mit Autorinnen und Autoren}.
 Digitale Edition, https://schnitzler-briefe.acdh.oeaw.ac.at/{\dateiname}.html (Stand \today)
\fi

\end{document}


