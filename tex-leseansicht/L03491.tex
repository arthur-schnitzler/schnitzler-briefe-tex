%% latex-leseansicht-vorspann.tex
%% Vorspann für die Leseansicht.
%% Lädt die gemeinsame Datei latex-vorspann.tex mit nicht gesetztem Schalter.

\newif\ifkorrekturansicht
\korrekturansichtfalse

\input{../tex-inputs/latex-vorspann}

\begin{center}
            \textcolor{red}{ENTWURF, NICHT FERTIG KORRIGIERT}
                      \end{center}
            
         
         \renewcommand{\erwaehntePersonen}{Personen: Samuel Fischer, Hedwig Fischer, Margarethe Kainz, Josef Kainz, Anna Katharina Rehmann, Ottilie Salten, Paul Salten, Paula Schlenther, Paul Schlenther, Olga Schnitzler, Heinrich Schnitzler,  W. Fred}
         \renewcommand{\erwaehnteInstitutionen}{Institutionen: Franz-Grillparzer-Preis}
         \renewcommand{\erwaehnteOrte}{Orte: Semmering, Südbahnhotel, Wien, Österreich}
         \renewcommand{\erwaehnteWerke}{Werke: Der Weg ins Freie. Roman}
               \section[Felix Salten an Arthur Schnitzler, 26. 1. 1908]{ Felix Salten an Arthur Schnitzler, 26. 1. 1908}\nopagebreak\mylabel{v}\rehead{ }\begin{ledgroupsized}[t]{13cm}\normalsize\beginnumbering \toendnotes[C]{\smallbreak\pagebreak[2]} \Standort{CUL, Schnitzler, B 89, B 1.}
\physDesc{Brief, 1 Blatt, 2 Seiten, 2108 Zeichen
\newline{}Handschrift: Bleistift, lateinische Kurrent
\newline{}Schnitzler: mit Bleistift Vermerk »\textsc{Salt{[}en{]}}« 
\newline{}Ordnung: mit Bleistift von unbekannter Hand nummeriert:
                                    »241« }\toendnotes[C]{\smallbreak}\pstart
           \noindent{}{\pb}\textcolor{gray}{\textbf{Südbahn-Hôtel\oindex{Suedbahnhotel@\textbf{Südbahnhotel}|pw}}}\pend
           \pstart
           \textcolor{gray}{\textbf{Semmering\oindex{Semmering@\textbf{Semmering}|pw}}}\hfill 26./1. 08\pend
           \pstart
           \textcolor{gray}{\textbf{Austria\oindex{Oesterreich@\textbf{Österreich}|pw}}}\pend
           \pstart
           \textcolor{gray}{\textbf{\textsc{Telegramme:}}}\pend
           \pstart
           \textcolor{gray}{\textbf{\textsc{Südbahnhôtel Semmering\oindex{Suedbahnhotel@\textbf{Südbahnhotel}|pw}.}}}\pend
           \pstart
           \textcolor{gray}{\textbf{\textsc{Telephon:}}}\pend
           \pstart
           \textcolor{gray}{\textbf{\textsc{Hôtel {\dots} Nr. 5.}}}\pend
           \pstart
           \textcolor{gray}{\textbf{\textsc{Dependance Nr. 6.}}}\pend
           \pstart{}Lieber,\pend\pstart
           danke sehr für Ihren ausführlichen Brief, der mich sehr gefreut hat. Den letzten
               Satz, da wo Sie sagen, dass Sie sich wieder »keck mitten ins Leben« u. s. w. habe
               ich, wie ich Ihnen gestehen muss, mit einer plötzlich austeigenden, sehr starken
               Ergriffenheit gelesen. Denn aus ihm sah ich erst ganz deutlich, \uline{wo} Sie in dieser letzten Zeit mit Ihren Gedanken und Sorgen gewesen
               sind, und was Sie durchgemacht haben. Nun aber dürfen Sie sich wol freuen und Ihre
               Freunde mit Ihnen. Wundervoll ist es ja, wie diese Gefahr an Ihnen u. Ihrer Frau\pwindex{Schnitzler, Olga 17.01.1882 – 13.01.1970@\textsc{Schnitzler, Olga} (17.01.1882 – 13.01.1970), \emph{Schauspielerin, Sängerin}|pwv} vorbeigeschwebt ist, und
               wie dann mit dem Grillparzer Preis\orgindex{Franz-Grillparzer-Preis@Franz-Grillparzer-Preis|pw} etwas zu Ihnen
               kam, was schließlich doch im Tiefsten so etwas wie einen Schimmer von Glück bedeutet.
               Wir gehen dem Frühling entgegen, und Ihre Frau\pwindex{Schnitzler, Olga 17.01.1882 – 13.01.1970@\textsc{Schnitzler, Olga} (17.01.1882 – 13.01.1970), \emph{Schauspielerin, Sängerin}|pwv} wird sich hoffentlich rasch erholen. Man sagt ja, dass
               nach dan Scharlach die Gesundheit intensiver wird, und so wird Frau Olga\pwindex{Schnitzler, Olga 17.01.1882 – 13.01.1970@\textsc{Schnitzler, Olga} (17.01.1882 – 13.01.1970), \emph{Schauspielerin, Sängerin}|pw} jetzt in in ein schönes Genesen und Glühen kommen, und
               mit der Jahreszeit gehen. Besseres läſst sich kaum denken. Ihren Roman\pwindex{Schnitzler, Arthur 15.05.1862 – 21.10.1931@\textsc{Schnitzler, Arthur} (15.05.1862 – 21.10.1931), \emph{Schriftsteller, Mediziner}!Weg ins Freie. Roman1.1.1908 – 1.6.1908@\strich\emph{Der Weg ins Freie. Roman} {[}1.1.1908 – 1.6.1908{]}|pwv} las ich nun doch in den ersten zwei
               Fortsetzungen. Sie werden meine Neugierde begreifen u. entschuldigen. Sagen kann ich
               jetzt natürlich noch nichts, ahne auch nur von weitem, wohin derWeg ins Freie\pwindex{Schnitzler, Arthur 15.05.1862 – 21.10.1931@\textsc{Schnitzler, Arthur} (15.05.1862 – 21.10.1931), \emph{Schriftsteller, Mediziner}!Weg ins Freie. Roman1.1.1908 – 1.6.1908@\strich\emph{Der Weg ins Freie. Roman} {[}1.1.1908 – 1.6.1908{]}|pwv} führt. Aber eine Menge
               Menschen wird mir jetzt schon sehr lebendig und das Abreißen der Fortsetzung mir
               freilich je mehr zur Qual, je näher einem diese Menschen kommen.\pend
           \pstart
           {\pb}Ich bin seit Donnerstag voriger
               Woche hier oben; traf hier Frau Kainz\pwindex{Kainz, Margarethe 13.12.1858 – 12.02.1950@\textsc{Kainz, Margarethe} (13.12.1858 – 12.02.1950), \emph{Schauspielerin}|pw} mit Frau
                  Schlenther\pwindex{Schlenther, Paula 27.02.1860 – 09.08.1938@\textsc{Schlenther, Paula} (27.02.1860 – 09.08.1938), \emph{Schauspielerin}|pw}, mit der ich komischerweise sehr
               sympathisirte. (Nett hat sich Schlenther\pwindex{Schlenther, Paul 20.08.1854 – 30.04.1916@\textsc{Schlenther, Paul} (20.08.1854 – 30.04.1916), \emph{Schriftsteller, Kritiker, Theaterleiter}|pw} in
               der Preis-Angelegenheit benommen) Samstag kam Otti\pwindex{Salten, Ottilie 07.03.1868 – 22.06.1942@\textsc{Salten, Ottilie} (07.03.1868 – 22.06.1942), \emph{Schauspielerin}|pw} mit den Kindern\pwindex{Salten, Paul 11.08.1903 – 08.05.1937@\textsc{Salten, Paul} (11.08.1903 – 08.05.1937), \emph{Filmcutter}|pwv}\pwindex{Rehmann, Anna Katharina 18.08.1904 – 27.03.1977@\textsc{Rehmann, Anna Katharina} (18.08.1904 – 27.03.1977), \emph{Schauspielerin}|pwv}, Sonntag kamen Fischers\pwindex{Fischer, Samuel 24.12.1859 – 15.10.1934@\textsc{Fischer, Samuel} (24.12.1859 – 15.10.1934), \emph{Verleger}|pw}\pwindex{Fischer, Hedwig 08.09.1871 – 11.04.1952@\textsc{Fischer, Hedwig} (08.09.1871 – 11.04.1952)|pw}, gestern u. heute ist der Kainz\pwindex{Kainz, Josef 02.01.1858 – 20.09.1910@\textsc{Kainz, Josef} (02.01.1858 – 20.09.1910), \emph{Schauspieler}|pw} dagewesen, und Herr Fred\pwindex{W. Fred 29.06.1879 – 23.10.1922@\textsc{W. Fred} (29.06.1879 – 23.10.1922), \emph{Schriftsteller, Journalist}|pw} ist
               immer da. Ich arbeite ein bisschen und spüre noch immer meine Darmzustände. –
               Hoffentlich sehen wir uns hier oben oder in Wien\oindex{Wien@\textbf{Wien}|pw}.
               Ängstlich bin ich ja, das gebe ich zu. Sie wiſsen doch, dass ich wegen meiner Kinder\pwindex{Salten, Paul 11.08.1903 – 08.05.1937@\textsc{Salten, Paul} (11.08.1903 – 08.05.1937), \emph{Filmcutter}|pwv}\pwindex{Rehmann, Anna Katharina 18.08.1904 – 27.03.1977@\textsc{Rehmann, Anna Katharina} (18.08.1904 – 27.03.1977), \emph{Schauspielerin}|pwv} beständig in
               einer halbtollen Furcht lebe. Aber ich denke, wenn Sie Heini\pwindex{Schnitzler, Heinrich 09.08.1902 – 12.07.1982@\textsc{Schnitzler, Heinrich} (09.08.1902 – 12.07.1982), \emph{Regisseur, Schauspieler}|pw} bei sich haben, ist wol nichts mehr zu
                  be\textcolor{gray}{sor}gen. \pend
           \pstart
           Also vieles Gute und Herzliche von uns zu Ihnen. Otti\pwindex{Salten, Ottilie 07.03.1868 – 22.06.1942@\textsc{Salten, Ottilie} (07.03.1868 – 22.06.1942), \emph{Schauspielerin}|pw} u. ich laſsen Frau Olga\pwindex{Schnitzler, Olga 17.01.1882 – 13.01.1970@\textsc{Schnitzler, Olga} (17.01.1882 – 13.01.1970), \emph{Schauspielerin, Sängerin}|pw} besonders
               grüßen.\pend
           \pstart
           Ihr {\\[\baselineskip]}\spacefill\mbox{Salten}\pend
           \leftskip=0em{}
         
         \endnumbering\mylabel{h}\end{ledgroupsized}\begin{anhang}\end{anhang}\newcommand{\dateiname}{L03491}\newcommand{\titel}{Felix Salten an Arthur Schnitzler, 26. 1. 1908}\newcommand{\editorInnen}{Martin Anton Müller und Laura Untner}%% latex-leseansicht-abspann.tex
%% Abspann für die Leseansicht.
%% Der Schalter \ifkorrekturansicht ist bereits durch den Vorspann gesetzt.

%% latex-abspann.tex
%% Gemeinsamer Abspann für Korrekturansicht und Leseansicht.
%% Setzt den Schalter \ifkorrekturansicht voraus (gesetzt in den
%% einbindenden Dateien latex-korrekturansicht-abspann.tex bzw.
%% latex-leseansicht-abspann.tex).
%% ---------------------------------------------------------------

\normalsize

% Das esempio-Environment wird nur in der Leseansicht benötigt
\ifkorrekturansicht\else
\newenvironment{esempio}[3]%
{
    \vspace{1.5ex}
    \rlap{\underline{#1}}
    \par
    \setlength{\parindent}{0cm}
    \nopagebreak
    \leftskip=#2cm
    \rightskip=#3cm
}
{
    \par
}
\fi

\doendnotes{C}
\bigskip
\vfill

\clearpage

\footnotesize

\ifkorrekturansicht
  \lohead{\textsc{register}}
\fi

% theindex-Environment neu definieren ohne reledmac
\makeatletter
\renewenvironment{theindex}{%
  \ifkorrekturansicht
    \section*{\indexname}%
  \else
    \subsubsection*{Index der erwähnten Entitäten}%
  \fi
  \setlength{\parindent}{0pt}%
  \setlength{\parskip}{0pt plus 0.3pt}%
  \let\item\@idxitem
}{%
  \ifkorrekturansicht\clearpage\fi
}
\makeatother

\IfFileExists{\jobname-pw.ind}{\input{\jobname-pw.ind}}{}

% Quellenangabe nur in der Leseansicht
\ifkorrekturansicht\else
% Fallback-Definitionen, falls die .tex-Datei \titel etc. nicht gesetzt hat
\providecommand{\titel}{}
\providecommand{\editorInnen}{}
\providecommand{\dateiname}{\jobname}

\vspace{3cm}

\vfill

\footnotesize
\textsc{Quelle}: \titel. Herausgegeben von {\editorInnen}. In: \emph{Arthur Schnitzler: Briefwechsel mit Autorinnen und Autoren}.
 Digitale Edition, https://schnitzler-briefe.acdh.oeaw.ac.at/{\dateiname}.html (Stand \today)
\fi

\end{document}


      