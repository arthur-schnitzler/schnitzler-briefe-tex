%% latex-leseansicht-vorspann.tex
%% Vorspann für die Leseansicht.
%% Lädt die gemeinsame Datei latex-vorspann.tex mit nicht gesetztem Schalter.

\newif\ifkorrekturansicht
\korrekturansichtfalse

\input{../tex-inputs/latex-vorspann}


\section[Stefan Zweig an Arthur Schnitzler, {[}4.{]} 4. 1911]{L03631 Stefan Zweig an Arthur Schnitzler, [4.] 4. 1911}
\nopagebreak\mylabel{L03631v}
\rehead{ }\normalsize\beginnumbering\briefempfaengerindex{Schnitzler, Arthur@\textsc{Schnitzler, Arthur}!zzzZweig, Stefan@\emph{von Stefan Zweig}!1911-04-041@{[4.] 4. 1911}|(be}
\toendnotes[C]{\smallbreak\pagebreak[2]}
\correspDesc{Versand  durch Stefan Zweig am [4.] 4. 1911 in Havanna
\newline{}Erhalt  durch Arthur Schnitzler im Zeitraum [10. 4. 1911
                  – 31. 5. 1911?] in Wien}\toendnotes[C]{\smallbreak}
\Standort{CUL, Schnitzler, B 118.}
\physDesc{Bildpostkarte, 244 Zeichen
\newline{}Handschrift: blaue Tinte, lateinische Kurrent
\newline{}Versand: Stempel: »\nobreak{}\oindex{Havanna@\textbf{Havanna}, \emph{Hauptstadt}|pwk}Habana Cuba, Apr \textcolor{gray}{4} 1911, 4 pm\nobreak{}«.  }
\buchAbdrucke{\weitereDrucke{Stefan Zweig: \emph{Briefwechsel mit Hermann Bahr, Sigmund Freud, Rainer Maria
                        Rilke und Arthur Schnitzler}. Herausgegeben von Jeffrey B. Berlin, Hans-Ulrich Lindken und Donald A. Prater. Frankfurt am Main: \emph{S. Fischer} 1987, S. 363.} }\toendnotes[C]{\smallbreak}\pstart{}{\pb}Austria (Europe)\oindex{Österreich@\textbf{Österreich}|pw}\pend{}\pstart{}D\textsuperscript{r} Artur Schnitzler\pend{}\pstart{}Vienna (Austria)\oindex{Wien@\textbf{Wien}, \emph{Verwaltungsgebiet}|pw}\pend{}\pstart{}XVIII \label{K_L03631-1v}\edtext{Sternwartestrasse 72}{\lemma{\textnormal{\emph{Sternwartestrasse 72}}}\Cendnote{\textnormal{Zweig\pwindex{Zweig, Stefan 28.\,11.\,1881 Wien – 23.\,2.\,1942 Petrópolis@\textsc{Zweig, Stefan} (28.\,11.\,1881 Wien – 23.\,2.\,1942 Petrópolis), \emph{Schriftsteller}|pwk} wechselt bei der Adressierung
                        seiner Schreiben an Schnitzler immer
                        wieder zwischen der falschen Hausnummer »72« und der
                        richtigen »71«.}}}\label{K_L03631-1}\oindex{Wien@\textbf{Wien}!XVIII., Währing@\textbf{XVIII., Währing}!Sternwartestraße 71@\textbf{Sternwartestraße 71}, \emph{Wohngebäude}|pw}\pend{}{\bigskip}
\pstart
           \noindent{}\centering{}{\pb}\textcolor{gray}{\textbf{Watch Tower, La Fuerza, Havana, Cuba\oindex{Castillo de la Real Fuerza@\textbf{Castillo de la Real Fuerza}, \emph{Burg}|pw}.}}\pend
           {\vspace{1\baselineskip}}\vspace{1em}
\pstart{}{\pb}Lieber verehrter Herr Doktor,\pend\vspace{0.5em}
\pstart
           \label{K_L03631-2v}\edtext{meine Reise}{\lemma{\textnormal{\emph{meine Reise}}}\Cendnote{\textnormal{Vom 22. 2. bis zum 21. 4. 1911 unternahm Stefan Zweig\pwindex{Zweig, Stefan 28.\,11.\,1881 Wien – 23.\,2.\,1942 Petrópolis@\textsc{Zweig, Stefan} (28.\,11.\,1881 Wien – 23.\,2.\,1942 Petrópolis), \emph{Schriftsteller}|pwk} eine amerikanische\oindex{Amerika@\textbf{Amerika}|pwk} Reise, beginnend in New York\oindex{New York City@\textbf{New York City}|pwk}. Von dort reiste er in mehrere Städte an der
                     nordamerikanischen\oindex{Nordamerika@\textbf{Nordamerika}, \emph{Region}|pwk} Ostküste, dann nach Chicago\oindex{Chicago@\textbf{Chicago}, \emph{Hauptstadt}|pwk} und Kanada\oindex{Kanada@\textbf{Kanada}|pwk}, um über Bermuda\oindex{Bermuda@\textbf{Bermuda}, \emph{Exterritoriales Gebiet}|pwk} und Kuba\oindex{Kuba@\textbf{Kuba}|pwk} bis nach Südamerika\oindex{Südamerika@\textbf{Südamerika}|pwk} zu gelangen.}}}\label{K_L03631-2} geht langsam zu Ende und ich freue mich
               schon sehr, Sie und Ihre verehrte Frau Gemahlin\pwindex{Schnitzler, Olga 17.\,1.\,1882 Wien – 13.\,1.\,1970 Lugano@\textsc{Schnitzler, Olga} (17.\,1.\,1882 Wien – 13.\,1.\,1970 Lugano), \emph{Schauspielerin, Sängerin}|pwv} bald wiederzusehn.\pend
           
\pstart
           Ihr getreuer{\\[\baselineskip]}\spacefill\mbox{\label{T_L03631-1v}\edtext{Stefan Zweig}{\lemma{\textnormal{\emph{Stefan Zweig}}}\Cendnote{\textnormal{seitlich, entlang des linken Rands}}}\label{T_L03631-1}}\pend
           \leftskip=0em{}\selectlanguage{ngerman}\endnumbering\briefempfaengerindex{Schnitzler, Arthur@\textsc{Schnitzler, Arthur}!zzzZweig, Stefan@\emph{von Stefan Zweig}!1911-04-041@{[4.] 4. 1911}|)be}\mylabel{L03631h}  \newcommand{\dateiname}{L03631}\newcommand{\titel}{Stefan Zweig an Arthur Schnitzler, [4.] 4. 1911}\newcommand{\editorInnen}{Selma Jahnke und Martin Anton Müller}%% latex-leseansicht-abspann.tex
%% Abspann für die Leseansicht.
%% Der Schalter \ifkorrekturansicht ist bereits durch den Vorspann gesetzt.

%% latex-abspann.tex
%% Gemeinsamer Abspann für Korrekturansicht und Leseansicht.
%% Setzt den Schalter \ifkorrekturansicht voraus (gesetzt in den
%% einbindenden Dateien latex-korrekturansicht-abspann.tex bzw.
%% latex-leseansicht-abspann.tex).
%% ---------------------------------------------------------------

\normalsize

% Das esempio-Environment wird nur in der Leseansicht benötigt
\ifkorrekturansicht\else
\newenvironment{esempio}[3]%
{
    \vspace{1.5ex}
    \rlap{\underline{#1}}
    \par
    \setlength{\parindent}{0cm}
    \nopagebreak
    \leftskip=#2cm
    \rightskip=#3cm
}
{
    \par
}
\fi

\doendnotes{C}
\bigskip
\vfill

\clearpage

\footnotesize

\ifkorrekturansicht
  \lohead{\textsc{register}}
\fi

% theindex-Environment neu definieren ohne reledmac
\makeatletter
\renewenvironment{theindex}{%
  \ifkorrekturansicht
    \section*{\indexname}%
  \else
    \subsubsection*{Index der erwähnten Entitäten}%
  \fi
  \setlength{\parindent}{0pt}%
  \setlength{\parskip}{0pt plus 0.3pt}%
  \let\item\@idxitem
}{%
  \ifkorrekturansicht\clearpage\fi
}
\makeatother

\IfFileExists{\jobname-pw.ind}{\input{\jobname-pw.ind}}{}

% Quellenangabe nur in der Leseansicht
\ifkorrekturansicht\else
% Fallback-Definitionen, falls die .tex-Datei \titel etc. nicht gesetzt hat
\providecommand{\titel}{}
\providecommand{\editorInnen}{}
\providecommand{\dateiname}{\jobname}

\vspace{3cm}

\vfill

\footnotesize
\textsc{Quelle}: \titel. Herausgegeben von {\editorInnen}. In: \emph{Arthur Schnitzler: Briefwechsel mit Autorinnen und Autoren}.
 Digitale Edition, https://schnitzler-briefe.acdh.oeaw.ac.at/{\dateiname}.html (Stand \today)
\fi

\end{document}


