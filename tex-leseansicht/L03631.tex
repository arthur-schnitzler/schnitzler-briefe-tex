%% latex-korrekturansicht-vorspann.tex
%% Vorspann für die Korrekturansicht.
%% Lädt die gemeinsame Datei latex-vorspann.tex mit gesetztem Schalter.

\newif\ifkorrekturansicht
\korrekturansichttrue

\input{../tex-inputs/latex-vorspann}


\section[Stefan Zweig an Arthur Schnitzler, {[}4.{]} 4. 1911]{L03631 Stefan Zweig an Arthur Schnitzler, {[}4.{]} 4. 1911}
\nopagebreak\mylabel{L03631v}
\rehead{ }\normalsize\beginnumbering\briefempfaengerindex{Schnitzler, Arthur@\textsc{Schnitzler, Arthur}!zzzZweig, Stefan@\emph{von Stefan Zweig}!1911-04-041@{{[}4.{]} 4. 1911}|(be}
\toendnotes[C]{\smallbreak\pagebreak[2]}\Standort{CUL, Schnitzler, B 118.}
\physDesc{Bildpostkarte, 244 Zeichen
\newline{}Handschrift: blaue Tinte, lateinische Kurrent
\newline{}Versand: Stempel: »\nobreak{}\oindex{Havanna@\textbf{Havanna}, \emph{P.PPLC}|pwk}Habana Cuba, Apr \textcolor{gray}{4} 1911, 4 pm\nobreak{}«.  }
\buchAbdrucke{\weitereDrucke{Stefan Zweig: \emph{Briefwechsel mit Hermann Bahr, Sigmund Freud, Rainer Maria
                        Rilke und Arthur Schnitzler}. Frankfurt am Main: \emph{S. Fischer} 1987, S. 363.} }\toendnotes[C]{\smallbreak}\pstart{}{\pb}Austria (Europe)\oindex{Oesterreich@\textbf{Österreich}, \emph{A.PCLI}|pw}\pend{}\pstart{}D\textsuperscript{r} Artur Schnitzler\pend{}\pstart{}Vienna (Austria)\oindex{Wien@\textbf{Wien}, \emph{A.ADM2}|pw}\pend{}\pstart{}XVIII \label{K_L03631-1v}\edtext{Sternwartestrasse 72}{\lemma{\textnormal{\emph{Sternwartestrasse 72}}}\Cendnote{\textnormal{Zweig\pwindex{Zweig, Stefan 28.11.1881 – 23.02.1942@\textsc{Zweig, Stefan} (28.11.1881 – 23.02.1942), \emph{Schriftsteller/Schriftstellerin}|pwk} wechselt bei der Adressierung
                        seiner Schreiben an Schnitzler immer
                        wieder zwischen der falschen Hausnummer »72« und der
                        richtigen »71«.}}}\label{K_L03631-1}\oindex{Sternwartestrasse 71@\textbf{Sternwartestraße 71}, \emph{Wohngebäude (K.WHS)}|pw}\pend{}{\bigskip}
\pstart
           \noindent{}\centering{}{\pb}\textcolor{gray}{\textbf{Watch Tower, La Fuerza, Havana, Cuba\oindex{Castillo de la Real Fuerza@\textbf{Castillo de la Real Fuerza}, \emph{S.CSTL}|pw}.}}\pend
           {\vspace{1\baselineskip}}\vspace{1em}
\pstart{}{\pb}Lieber verehrter Herr Doktor,\pend\vspace{0.5em}
\pstart
           \label{K_L03631-2v}\edtext{meine Reise}{\lemma{\textnormal{\emph{meine Reise}}}\Cendnote{\textnormal{Vom 22. 2. bis zum 21. 4. 1911 unternahm Stefan Zweig\pwindex{Zweig, Stefan 28.11.1881 – 23.02.1942@\textsc{Zweig, Stefan} (28.11.1881 – 23.02.1942), \emph{Schriftsteller/Schriftstellerin}|pwk} eine amerikanische\oindex{Amerika@\textbf{Amerika}, \emph{kein passender Code gefunden}|pwk} Reise, beginnend in New York\oindex{New York City@\textbf{New York City}, \emph{P.PPL}|pwk}. Von dort reiste er in mehrere Städte an der
                     nordamerikanischen\oindex{Nordamerika@\textbf{Nordamerika}, \emph{L.RGN}|pwk} Ostküste, dann nach Chicago\oindex{Chicago@\textbf{Chicago}, \emph{P.PPLA2}|pwk} und Kanada\oindex{Kanada@\textbf{Kanada}, \emph{A.PCLI}|pwk}, um über Bermuda\oindex{Bermuda@\textbf{Bermuda}, \emph{A.PCLD}|pwk} und Kuba\oindex{Cuba@\textbf{Cuba}, \emph{A.PCLI}|pwk} bis nach Südamerika\oindex{Suedamerika@\textbf{Südamerika}, \emph{Kontinent (A.KNT)}|pwk} zu gelangen.}}}\label{K_L03631-2} geht langsam zu Ende und ich freue mich
               schon sehr, Sie und Ihre verehrte Frau Gemahlin\pwindex{Schnitzler, Olga 17.01.1882 – 13.01.1970@\textsc{Schnitzler, Olga} (17.01.1882 – 13.01.1970), \emph{Schauspieler/Schauspielerin, Sänger/Sängerin}|pwv} bald wiederzusehn.\pend
           
\pstart
           Ihr getreuer{\\[\baselineskip]}\spacefill\mbox{\label{T_L03631-1v}\edtext{Stefan Zweig}{\lemma{\textnormal{\emph{Stefan Zweig}}}\Cendnote{\textnormal{seitlich, entlang des linken Rands}}}\label{T_L03631-1}}\pend
           \leftskip=0em{}\selectlanguage{ngerman}\endnumbering\briefempfaengerindex{Schnitzler, Arthur@\textsc{Schnitzler, Arthur}!zzzZweig, Stefan@\emph{von Stefan Zweig}!1911-04-041@{{[}4.{]} 4. 1911}|)be}\mylabel{L03631h}  \normalsize

\doendnotes{C}
\bigskip
\vfill

\clearpage

\footnotesize

\lohead{\textsc{register}}

% Definiere theindex-Environment komplett neu ohne reledmac
\makeatletter
\renewenvironment{theindex}{%
  \section*{\indexname}%
  \setlength{\parindent}{0pt}%
  \setlength{\parskip}{0pt plus 0.3pt}%
  \let\item\@idxitem
}{%
  \clearpage
}
\makeatother

\IfFileExists{\jobname-pw.ind}{\input{\jobname-pw.ind}}{}

\end{document}

      