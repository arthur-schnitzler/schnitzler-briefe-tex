%% latex-korrekturansicht-vorspann.tex
%% Vorspann für die Korrekturansicht.
%% Lädt die gemeinsame Datei latex-vorspann.tex mit gesetztem Schalter.

\newif\ifkorrekturansicht
\korrekturansichttrue

\input{../tex-inputs/latex-vorspann}


\section[Arthur Schnitzler an Richard Beer-Hofmann, 19. 9. 1895]{L00487 Arthur Schnitzler an Richard Beer-Hofmann, 19. 9. 1895}
\nopagebreak\mylabel{L00487v}
\rehead{ }\normalsize\beginnumbering\briefempfaengerindex{Beer-Hofmann, Richard@\textsc{Beer-Hofmann, Richard}!zzzSchnitzler, Arthur@\emph{von Arthur Schnitzler}!1895-09-191@{19. 9. 1895}|(be}
\toendnotes[C]{\smallbreak\pagebreak[2]}\Standort{YCGL, MSS 31.}
\physDesc{Postkarte, 636 Zeichen
\newline{}Handschrift: 1) schwarze Tinte, deutsche Kurrent\hspace{1em}2) schwarze Tinte, lateinische Kurrent (\noindent{}Adresse)\hspace{1em}
\newline{}Versand: 1) Stempel: »\nobreak{}\oindex{IX., Alsergrund@\textbf{IX., Alsergrund}, \emph{A.ADM3}|pwk}Wien 9/3, 19. 9. 95, 3–4N\nobreak{}«.   2) Stempel: »\nobreak{}\oindex{Riva del Garda@\textbf{Riva del Garda}, \emph{P.PPLA3}|pwk}Riva, 26. 9. 95, Nachm.\nobreak{}«. }
\buchAbdrucke{\weitereDrucke{Arthur Schnitzler, Richard Beer-Hofmann: \emph{Briefwechsel 1891–1931}. Wien, Zürich: \emph{Europaverlag} 1992, S. 82–83.} }\pstart{}{\pb}Dr. Richard Beer-Hofmann\pend{}\pstart{}Riva am Gardasee\oindex{Riva del Garda@\textbf{Riva del Garda}, \emph{P.PPLA3}|pw}.\pend{}\pstart{}post restante\pend{}{\bigskip}\vspace{1em}
\pstart
           \raggedleft{}{\pb}Do{\geminationn}erſtg\pend
           \vspace{0.5em}
\pstart
           Lieber Freund, die Briefe gehen unerhört lang hin u. her. Ich
               ſchreibe Ihnen der Sicherheit wegen nach Riva\oindex{Riva del Garda@\textbf{Riva del Garda}, \emph{P.PPLA3}|pw}; in
                  Schberg\oindex{Schoenberg im Stubaital@\textbf{Schönberg im Stubaital}, \emph{P.PPLA3}|pw} würden Sie dieſe Zeilen nicht mehr
               erreichen. Und da Sie nur \uline{einen} Tag in Riva\oindex{Riva del Garda@\textbf{Riva del Garda}, \emph{P.PPLA3}|pw} bleiben, ich alſo genau heute einen Brief an
               Sie ſchreiben müßte, ka{\geminationn} ich natürlich nicht. Geben Sie
               mir mehr Spielraum. – Das weſentliche: L.\pwindex{Liebelei. Schauspiel in drei Akten@\emph{Liebelei. Schauspiel in drei Akten}|pw} ko{\geminationm}t wohl zwiſchen 6. u. 10. October zur
               Aufführg. – Geſtern war Leſeprobe, die recht gut ausfiel. – Meine Sti{\geminationm}ung aus ma{\geminationn}igfachen
               Gründen im Abſinken. Ich beneide Sie. Wegreiſen möcht ich am liebsten. Schreiben Sie
               gleich. Herzlich der Ihre \spacefill\mbox{Arth}\pend
           \selectlanguage{ngerman}\endnumbering\briefempfaengerindex{Beer-Hofmann, Richard@\textsc{Beer-Hofmann, Richard}!zzzSchnitzler, Arthur@\emph{von Arthur Schnitzler}!1895-09-191@{19. 9. 1895}|)be}\mylabel{L00487h}  \normalsize

\doendnotes{C}
\bigskip
\vfill

\clearpage

\footnotesize

\lohead{\textsc{register}}

% Definiere theindex-Environment komplett neu ohne reledmac
\makeatletter
\renewenvironment{theindex}{%
  \section*{\indexname}%
  \setlength{\parindent}{0pt}%
  \setlength{\parskip}{0pt plus 0.3pt}%
  \let\item\@idxitem
}{%
  \clearpage
}
\makeatother

\IfFileExists{\jobname-pw.ind}{\input{\jobname-pw.ind}}{}

\end{document}

      