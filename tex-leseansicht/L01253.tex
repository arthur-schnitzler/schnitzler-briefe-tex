%% latex-leseansicht-vorspann.tex
%% Vorspann für die Leseansicht.
%% Lädt die gemeinsame Datei latex-vorspann.tex mit nicht gesetztem Schalter.

\newif\ifkorrekturansicht
\korrekturansichtfalse

\input{../tex-inputs/latex-vorspann}


         
         \renewcommand{\erwaehntePersonen}{Personen: Gerhart Hauptmann}
         \renewcommand{\erwaehnteOrte}{Orte: Hotel Sacher, Wien}
         \renewcommand{\erwaehnteWerke}{}
               \section[Gerhart Hauptmann an Arthur Schnitzler, {[}29. 11. 1902{]}]{ Gerhart Hauptmann an Arthur Schnitzler, {[}29. 11. 1902{]}}\nopagebreak\mylabel{v}\rehead{ }\begin{ledgroupsized}[t]{13cm}\normalsize\beginnumbering \toendnotes[C]{\smallbreak\pagebreak[2]} \Standort{Staatsbibliothek Berlin – Preußischer Kulturbesitz, Autogr. I/4486.}
\physDesc{Brief, 1 Blatt, 1 Seite, 92 Zeichen
\newline{}Handschrift: Bleistift, lateinische Kurrent
\newline{}Schnitzler: mit Bleistift datiert: »29. 11. 902« }\Standort{DLA, A:Schnitzler, HS.NZ85.1.3362.}
\physDesc{Fotokopie, 1 Blatt, 1 Seite, 92 Zeichen
\newline{}Handschrift: Bleistift, lateinische Kurrent}\toendnotes[C]{\smallbreak}\pstart
           \noindent{}{\pb}\textcolor{gray}{\textbf{\label{T_L01253-1v}\edtext{HOTEL SACHER}{\lemma{\textnormal{\emph{Hotel Sacher}}}\Cendnote{\textnormal{Briefkopf mit gedrucktem
                              Wappen}}}\label{T_L01253-1h}\oindex{Hotel Sacher@\textbf{Hotel Sacher}|pw}}}\hfill \textcolor{gray}{\textbf{WIEN\oindex{Wien@\textbf{Wien}|pw}, .......... 19{\dots}}}\pend
           \pstart
           Herzlichen Gruss, lieber Herr Schnitzler, ich freue mich, dass Sie
               doch noch kommen.\pend
           \pstart
           Ihr{\\[\baselineskip]}\spacefill\mbox{G. Hptm}\pend
           \leftskip=0em{}
         
         \endnumbering\mylabel{h}\end{ledgroupsized}  \newcommand{\dateiname}{L01253}\newcommand{\titel}{Gerhart Hauptmann an Arthur Schnitzler, [29. 11. 1902]}\newcommand{\editorInnen}{Martin Anton Müller und Gerd-Hermann Susen}%% latex-leseansicht-abspann.tex
%% Abspann für die Leseansicht.
%% Der Schalter \ifkorrekturansicht ist bereits durch den Vorspann gesetzt.

%% latex-abspann.tex
%% Gemeinsamer Abspann für Korrekturansicht und Leseansicht.
%% Setzt den Schalter \ifkorrekturansicht voraus (gesetzt in den
%% einbindenden Dateien latex-korrekturansicht-abspann.tex bzw.
%% latex-leseansicht-abspann.tex).
%% ---------------------------------------------------------------

\normalsize

% Das esempio-Environment wird nur in der Leseansicht benötigt
\ifkorrekturansicht\else
\newenvironment{esempio}[3]%
{
    \vspace{1.5ex}
    \rlap{\underline{#1}}
    \par
    \setlength{\parindent}{0cm}
    \nopagebreak
    \leftskip=#2cm
    \rightskip=#3cm
}
{
    \par
}
\fi

\doendnotes{C}
\bigskip
\vfill

\clearpage

\footnotesize

\ifkorrekturansicht
  \lohead{\textsc{register}}
\fi

% theindex-Environment neu definieren ohne reledmac
\makeatletter
\renewenvironment{theindex}{%
  \ifkorrekturansicht
    \section*{\indexname}%
  \else
    \subsubsection*{Index der erwähnten Entitäten}%
  \fi
  \setlength{\parindent}{0pt}%
  \setlength{\parskip}{0pt plus 0.3pt}%
  \let\item\@idxitem
}{%
  \ifkorrekturansicht\clearpage\fi
}
\makeatother

\IfFileExists{\jobname-pw.ind}{\input{\jobname-pw.ind}}{}

% Quellenangabe nur in der Leseansicht
\ifkorrekturansicht\else
% Fallback-Definitionen, falls die .tex-Datei \titel etc. nicht gesetzt hat
\providecommand{\titel}{}
\providecommand{\editorInnen}{}
\providecommand{\dateiname}{\jobname}

\vspace{3cm}

\vfill

\footnotesize
\textsc{Quelle}: \titel. Herausgegeben von {\editorInnen}. In: \emph{Arthur Schnitzler: Briefwechsel mit Autorinnen und Autoren}.
 Digitale Edition, https://schnitzler-briefe.acdh.oeaw.ac.at/{\dateiname}.html (Stand \today)
\fi

\end{document}


      