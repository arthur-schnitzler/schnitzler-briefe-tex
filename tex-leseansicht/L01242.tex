%% latex-leseansicht-vorspann.tex
%% Vorspann für die Leseansicht.
%% Lädt die gemeinsame Datei latex-vorspann.tex mit nicht gesetztem Schalter.

\newif\ifkorrekturansicht
\korrekturansichtfalse

\input{../tex-inputs/latex-vorspann}


\section[Hermann Bahr an Arthur Schnitzler, 15. 10. 1902]{L01242 Hermann Bahr an Arthur Schnitzler, 15. 10. 1902}
\nopagebreak\mylabel{L01242v}
\rehead{ }\normalsize\beginnumbering\briefempfaengerindex{Schnitzler, Arthur@\textsc{Schnitzler, Arthur}!zzzBahr, Hermann@\emph{von Hermann Bahr}!1902-10-152@{15. 10. 1902}|(be}
\toendnotes[C]{\smallbreak\pagebreak[2]}
\correspDesc{Versand  durch Hermann Bahr am 15. 10. 1902 in Berlin
\newline{}Erhalt  durch Arthur Schnitzler im Zeitraum [15. 10. 1902 – 19. 10. 1902?] in Berlin}\toendnotes[C]{\smallbreak}
\Standort{CUL, Schnitzler, B 5b.}
\physDesc{Brief, 1 Blatt, 1 Seite, 367 Zeichen
\newline{}Handschrift: schwarze Tinte, deutsche Kurrent
\newline{}Ordnung: mit Bleistift von unbekannter Hand nummeriert:
                                    »91« }
\buchAbdrucke{\weitereDrucke{Hermann Bahr, Arthur Schnitzler: \emph{Briefwechsel, Aufzeichnungen, Dokumente (1891–1931)}. Herausgegeben von Kurt Ifkovits und Martin Anton Müller. Göttingen: \emph{Wallstein} 2018, S. 244.} }\toendnotes[C]{\smallbreak}
\pstart
           {\pb}\textcolor{gray}{\textbf{GRAND HÔTEL\oindex{Hotel de Rome@\textbf{Hotel de Rome}, \emph{Hotel}|pw}}}\hfill \textcolor{gray}{\textbf{Berlin N. W.\oindex{Berlin@\textbf{Berlin}, \emph{Hauptstadt}|pw}, den .......... 190}}\pend
           
\pstart
           \textcolor{gray}{\textbf{DE ROME U. DU NORD}}\hfill \textcolor{gray}{\textbf{Unter den Linden 39.}}\oindex{Unter den Linden@\textbf{Unter den Linden}, \emph{Ehemaliger Ort}|pw}\pend
           
\pstart
           \textcolor{gray}{\textbf{A. MÜHLING}}\pwindex{Mühling, Adolph 1819 – 1910@\textsc{Mühling, Adolph} (1819 – 1910), \emph{Hotelbesitzer}|pw}\hfill 15. 10\pend
           
\pstart
           \textcolor{gray}{\textbf{Kgl. Hoflieferant}}\pend
           
\pstart
           \textcolor{gray}{\textbf{BERLIN}}\oindex{Berlin@\textbf{Berlin}, \emph{Hauptstadt}|pw}\pend
           
\pstart
           \textcolor{gray}{\textbf{Fernsprecher: Amt I, No. 4438.}}\pend
           
\pstart\center{}Lieber Arthur!\pend\vspace{0.5em}
\pstart
           Herzlichſten Dank! In einer Zeitung las ich: Halm\pwindex{Halm, Alfred 9.\,12.\,1861 Wien – 5.\,2.\,1951 Berlin@\textsc{Halm, Alfred} (9.\,12.\,1861 Wien – 5.\,2.\,1951 Berlin), \emph{Schauspieler}|pw} hätte als D\textsuperscript{r}{ }\label{K_L01242-1v}\edtext{Mohn\pwindex{Bahr, Hermann 19.\,7.\,1863 Linz – 15.\,1.\,1934 München@\textsc{Bahr, Hermann} (19.\,7.\,1863 Linz – 15.\,1.\,1934 München), \emph{Schriftsteller, Kritiker}!Wienerinnen. Lustspiel in drei Akten@\strich\emph{Wienerinnen. Lustspiel in drei Akten}|pwv}}{\lemma{\textnormal{\emph{Mohn}}}\Cendnote{\textnormal{Figur aus \emph{Wienerinnen}\pwindex{Bahr, Hermann 19.\,7.\,1863 Linz – 15.\,1.\,1934 München@\textsc{Bahr, Hermann} (19.\,7.\,1863 Linz – 15.\,1.\,1934 München), \emph{Schriftsteller, Kritiker}!Wienerinnen. Lustspiel in drei Akten@\strich\emph{Wienerinnen. Lustspiel in drei Akten}|pwk}}}}\label{K_L01242-1}{ }\label{K_L01242-2v}\edtext{Deine Maske gehabt}{\lemma{\textnormal{\emph{Deine Maske gehabt}}}\Cendnote{\textnormal{nicht nachgewiesen; vielleicht eine
                  Fehlleistung Bahrs\pwindex{Bahr, Hermann 19.\,7.\,1863 Linz – 15.\,1.\,1934 München@\textsc{Bahr, Hermann} (19.\,7.\,1863 Linz – 15.\,1.\,1934 München), \emph{Schriftsteller, Kritiker}|pwk} zur Rezension\pwindex{Strecker, Karl 8.\,4.\,1862 Tąpadły – 19.\,2.\,1933 Garmisch-Partenkirchen@\textsc{Strecker, Karl} (8.\,4.\,1862 Tąpadły – 19.\,2.\,1933 Garmisch-Partenkirchen), \emph{Theaterkritiker}!Berliner Theater. Hermann Bahr: »Wienerinnen«. (Eine nicht einwandfreie Kritik)@\strich\emph{Berliner Theater. Hermann Bahr: »Wienerinnen«. (Eine nicht einwandfreie Kritik)}|pwkv} von Karl Strecker\pwindex{Strecker, Karl 8.\,4.\,1862 Tąpadły – 19.\,2.\,1933 Garmisch-Partenkirchen@\textsc{Strecker, Karl} (8.\,4.\,1862 Tąpadły – 19.\,2.\,1933 Garmisch-Partenkirchen), \emph{Theaterkritiker}|pwk}: »Herr Halm\pwindex{Halm, Alfred 9.\,12.\,1861 Wien – 5.\,2.\,1951 Berlin@\textsc{Halm, Alfred} (9.\,12.\,1861 Wien – 5.\,2.\,1951 Berlin), \emph{Schauspieler}|pw}, der auch die Regie führte, gab einen modernen Ästheten mit
                     gedrehter Stirnlocke, einen eitlen Faiseur, seltsamerweise aber in der Maske
                     von Hermann Bahr\pwindex{Bahr, Hermann 19.\,7.\,1863 Linz – 15.\,1.\,1934 München@\textsc{Bahr, Hermann} (19.\,7.\,1863 Linz – 15.\,1.\,1934 München), \emph{Schriftsteller, Kritiker}|pw}.« (\emph{Berliner Theater. Hermann Bahr: »Wienerinnen«.
                        (Eine nicht einwandfreie Kritik)}\pwindex{Strecker, Karl 8.\,4.\,1862 Tąpadły – 19.\,2.\,1933 Garmisch-Partenkirchen@\textsc{Strecker, Karl} (8.\,4.\,1862 Tąpadły – 19.\,2.\,1933 Garmisch-Partenkirchen), \emph{Theaterkritiker}!Berliner Theater. Hermann Bahr: »Wienerinnen«. (Eine nicht einwandfreie Kritik)@\strich\emph{Berliner Theater. Hermann Bahr: »Wienerinnen«. (Eine nicht einwandfreie Kritik)}|pwk}. In: \emph{Tägliche Rundschau}\pwindex{Tägliche Rundschau@\emph{Tägliche Rundschau}|pwk}, Jg. 20, Nr. 483, Morgenblatt, 1. Ausgabe,
                        15. 10. 1902, S. [2]). Vgl. A. S.: \emph{Tagebuch}, 18. 10. 1894.}}}\label{K_L01242-2}. Wahr iſt, daß er einen blonden Vollbart trug, aus lauter Angſt, in die Maske
                  Sudermanns\pwindex{Sudermann, Hermann 30.\,9.\,1857 Macikai – 21.\,11.\,1928 Berlin@\textsc{Sudermann, Hermann} (30.\,9.\,1857 Macikai – 21.\,11.\,1928 Berlin), \emph{Schriftsteller}|pw} zu gerathen. Daß es ganz albern
               wäre, einem{ }ſpöttelnden Salon-Kritiker Deine Züge zu geben, brauche ich Dir ja nicht
               erſt zu{ }ſagen. Die Leut{ }ſind{ }ſo blöd!\pend
           
\pstart
           Herzlichſt{\\[\baselineskip]}Dein{\\[\baselineskip]}\spacefill\mbox{Hermann}\pend
           \leftskip=0em{}\selectlanguage{ngerman}\endnumbering\briefempfaengerindex{Schnitzler, Arthur@\textsc{Schnitzler, Arthur}!zzzBahr, Hermann@\emph{von Hermann Bahr}!1902-10-152@{15. 10. 1902}|)be}\mylabel{L01242h}  \newcommand{\dateiname}{L01242}\newcommand{\titel}{Hermann Bahr an Arthur Schnitzler, 15. 10. 1902}\newcommand{\editorInnen}{Herausgegeben von Martin Anton Müller}%% latex-leseansicht-abspann.tex
%% Abspann für die Leseansicht.
%% Der Schalter \ifkorrekturansicht ist bereits durch den Vorspann gesetzt.

%% latex-abspann.tex
%% Gemeinsamer Abspann für Korrekturansicht und Leseansicht.
%% Setzt den Schalter \ifkorrekturansicht voraus (gesetzt in den
%% einbindenden Dateien latex-korrekturansicht-abspann.tex bzw.
%% latex-leseansicht-abspann.tex).
%% ---------------------------------------------------------------

\normalsize

% Das esempio-Environment wird nur in der Leseansicht benötigt
\ifkorrekturansicht\else
\newenvironment{esempio}[3]%
{
    \vspace{1.5ex}
    \rlap{\underline{#1}}
    \par
    \setlength{\parindent}{0cm}
    \nopagebreak
    \leftskip=#2cm
    \rightskip=#3cm
}
{
    \par
}
\fi

\doendnotes{C}
\bigskip
\vfill

\clearpage

\footnotesize

\ifkorrekturansicht
  \lohead{\textsc{register}}
\fi

% theindex-Environment neu definieren ohne reledmac
\makeatletter
\renewenvironment{theindex}{%
  \ifkorrekturansicht
    \section*{\indexname}%
  \else
    \subsubsection*{Index der erwähnten Entitäten}%
  \fi
  \setlength{\parindent}{0pt}%
  \setlength{\parskip}{0pt plus 0.3pt}%
  \let\item\@idxitem
}{%
  \ifkorrekturansicht\clearpage\fi
}
\makeatother

\IfFileExists{\jobname-pw.ind}{\input{\jobname-pw.ind}}{}

% Quellenangabe nur in der Leseansicht
\ifkorrekturansicht\else
% Fallback-Definitionen, falls die .tex-Datei \titel etc. nicht gesetzt hat
\providecommand{\titel}{}
\providecommand{\editorInnen}{}
\providecommand{\dateiname}{\jobname}

\vspace{3cm}

\vfill

\footnotesize
\textsc{Quelle}: \titel. Herausgegeben von {\editorInnen}. In: \emph{Arthur Schnitzler: Briefwechsel mit Autorinnen und Autoren}.
 Digitale Edition, https://schnitzler-briefe.acdh.oeaw.ac.at/{\dateiname}.html (Stand \today)
\fi

\end{document}


