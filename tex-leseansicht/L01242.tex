%% latex-leseansicht-vorspann.tex
%% Vorspann für die Leseansicht.
%% Lädt die gemeinsame Datei latex-vorspann.tex mit nicht gesetztem Schalter.

\newif\ifkorrekturansicht
\korrekturansichtfalse

\input{../tex-inputs/latex-vorspann}


         
         \newcommand{\erwaehntePersonen}{Personen: Alfred Halm, Adolph Mühling, Karl Strecker, Hermann Sudermann}
         \newcommand{\erwaehnteOrte}{Orte: Berlin, Hotel de Rome, Unter den Linden}
         \newcommand{\erwaehnteWerke}{Werke: Berliner Theater. Hermann Bahr: »Wienerinnen«. (Eine nicht einwandfreie Kritik), Tägliche Rundschau, Wienerinnen. Lustspiel in drei Akten}
               \section[Hermann Bahr an Arthur Schnitzler, 15. 10. 1902]{ Hermann Bahr an Arthur Schnitzler, 15. 10. 1902}\nopagebreak\mylabel{v}\rehead{ }\begin{ledgroupsized}[t]{13cm}\normalsize\beginnumbering \toendnotes[C]{\smallbreak\pagebreak[2]} \Standort{CUL, Schnitzler, B 5b.}
\physDesc{Brief, 1 Blatt, 1 Seite
\newline{}Handschrift: schwarze Tinte, deutsche Kurrent\newline{}Ordnung: mit Bleistift von unbekannter Hand nummeriert:
                                    »91« }\buchAbdrucke{\weitereDrucke{Hermann Bahr, Arthur Schnitzler: \emph{Briefwechsel, Aufzeichnungen, Dokumente (1891–1931)}. Hg. Kurt Ifkovits und Martin Anton Müller. Göttingen: \emph{Wallstein} 2018, S. 244.} }\toendnotes[C]{\smallbreak}\pstart
           \noindent{}{\pb}\textcolor{gray}{\textbf{GRAND HÔTEL\oindex{Hotel de Rome@\textbf{Hotel de Rome}|pw}}}\hfill \textcolor{gray}{\textbf{Berlin N. W.\oindex{Berlin@\textbf{Berlin}|pw}, den .......... 190}}\pend
           \pstart
           \textcolor{gray}{\textbf{DE ROME U. DU NORD}}\hfill \textcolor{gray}{\textbf{Unter den Linden 39.}}\oindex{Unter den Linden@\textbf{Unter den Linden}|pw}\pend
           \pstart
           \textcolor{gray}{\textbf{A. MÜHLING}}\pwindex{Muehling, Adolph 1819 – 1910@\textsc{Mühling, Adolph} (1819 – 1910), \emph{Hotelbesitzer}|pw}\hfill 15. 10\pend
           \pstart
           \textcolor{gray}{\textbf{Kgl. Hoflieferant}}\pend
           \pstart
           \textcolor{gray}{\textbf{BERLIN}}\oindex{Berlin@\textbf{Berlin}|pw}\pend
           \pstart
           \textcolor{gray}{\textbf{Fernsprecher: Amt I, No. 4438.}}\pend
           \pstart\center{}Lieber Arthur!\pend\pstart
           Herzlichſten Dank! In einer Zeitung las ich: Halm\pwindex{Halm, Alfred 1861-12-09 – 1951-02-05@\textsc{Halm, Alfred} (1861-12-09 – 1951-02-05), \emph{Schauspieler}|pw}
               hätte als D\textsuperscript{r}{ }\label{K_L01242_1v}\edtext{Mohn\pwindex{Bahr, Hermann 19.07.1863 – 15.01.1934@\textsc{Bahr, Hermann} (19.07.1863 – 15.01.1934), \emph{Schriftsteller, Kritiker}!Wienerinnen. Lustspiel in drei Akten1900@\strich\emph{Wienerinnen. Lustspiel in drei Akten} {[}1900{]}|pwv}}{\lemma{\textnormal{\emph{Mohn}}}\Cendnote{\textnormal{Figur aus \emph{Wienerinnen}\pwindex{Bahr, Hermann 19.07.1863 – 15.01.1934@\textsc{Bahr, Hermann} (19.07.1863 – 15.01.1934), \emph{Schriftsteller, Kritiker}!Wienerinnen. Lustspiel in drei Akten1900@\strich\emph{Wienerinnen. Lustspiel in drei Akten} {[}1900{]}|pwk}}}}\label{K_L01242_1h}{ }\label{K_L01242_2v}\edtext{Deine Maske gehabt}{\lemma{\textnormal{\emph{Deine Maske gehabt}}}\Cendnote{\textnormal{nicht nachgewiesen; vielleicht eine
                  Fehlleistung Bahr\pwindex{Bahr, Hermann 19.07.1863 – 15.01.1934@\textsc{Bahr, Hermann} (19.07.1863 – 15.01.1934), \emph{Schriftsteller, Kritiker}|pwk}s zur Rezension\pwindex{Strecker, Karl 1862-04-08 – 1933-02-19@\textsc{Strecker, Karl} (1862-04-08 – 1933-02-19), \emph{Theaterkritiker}!Berliner Theater. Hermann Bahr: »Wienerinnen«. (Eine nicht einwandfreie Kritik)15. 10. 1902@\strich\emph{Berliner Theater. Hermann Bahr: »Wienerinnen«. (Eine nicht einwandfreie Kritik)} {[}15. 10. 1902{]}|pwkv} von Karl Strecker\pwindex{Strecker, Karl 1862-04-08 – 1933-02-19@\textsc{Strecker, Karl} (1862-04-08 – 1933-02-19), \emph{Theaterkritiker}|pwk}: »Herr Halm\pwindex{Halm, Alfred 1861-12-09 – 1951-02-05@\textsc{Halm, Alfred} (1861-12-09 – 1951-02-05), \emph{Schauspieler}|pw}, der
                     auch die Regie führte, gab einen modernen Ästheten mit gedrehter Stirnlocke,
                     einen eitlen Faiseur, seltsamerweise aber in der Maske von Hermann Bahr\pwindex{Bahr, Hermann 19.07.1863 – 15.01.1934@\textsc{Bahr, Hermann} (19.07.1863 – 15.01.1934), \emph{Schriftsteller, Kritiker}|pw}.« (\emph{Berliner Theater. Hermann Bahr: »Wienerinnen«.
                        (Eine nicht einwandfreie Kritik)}\pwindex{Strecker, Karl 1862-04-08 – 1933-02-19@\textsc{Strecker, Karl} (1862-04-08 – 1933-02-19), \emph{Theaterkritiker}!Berliner Theater. Hermann Bahr: »Wienerinnen«. (Eine nicht einwandfreie Kritik)15. 10. 1902@\strich\emph{Berliner Theater. Hermann Bahr: »Wienerinnen«. (Eine nicht einwandfreie Kritik)} {[}15. 10. 1902{]}|pwk}. In: \emph{Tägliche Rundschau}\pwindex{?? Werk@Nicht ermittelte Verfasserinnen und Verfasser!Taegliche Rundschau1881 – 1933@\emph{Tägliche Rundschau} {[}1881 – 1933{]}|pwk}, Jg. 20, Nr. 483, Morgenblatt, 1. Ausgabe,
                        15. 10. 1902, S. [2]). Vgl. A. S.: \emph{Tagebuch}, 18. 10. 1894}}}\label{K_L01242_2h}. Wahr iſt, daß er einen blonden Vollbart trug, aus lauter Angſt, in die Maske
                  Sudermanns\pwindex{Sudermann, Hermann 30.09.1857 – 21.11.1928@\textsc{Sudermann, Hermann} (30.09.1857 – 21.11.1928), \emph{Schriftsteller}|pw} zu gerathen. Daß es ganz albern
               wäre, einem ſpöttelnden Salon-Kritiker Deine Züge zu geben, brauche ich Dir ja nicht
               erſt zu ſagen. Die Leut ſind ſo blöd!\pend
           \pstart
           Herzlichſt{\\[\baselineskip]}Dein{\\[\baselineskip]}\spacefill\mbox{Hermann}\pend
           \leftskip=0em{}
         
         \endnumbering\mylabel{h}\end{ledgroupsized}  \newcommand{\dateiname}{L01242}\newcommand{\titel}{Hermann Bahr an Arthur Schnitzler, 15. 10. 1902}\newcommand{\editorInnen}{ Kurt Ifkovits,  Martin Anton Müller}%% latex-leseansicht-abspann.tex
%% Abspann für die Leseansicht.
%% Der Schalter \ifkorrekturansicht ist bereits durch den Vorspann gesetzt.

%% latex-abspann.tex
%% Gemeinsamer Abspann für Korrekturansicht und Leseansicht.
%% Setzt den Schalter \ifkorrekturansicht voraus (gesetzt in den
%% einbindenden Dateien latex-korrekturansicht-abspann.tex bzw.
%% latex-leseansicht-abspann.tex).
%% ---------------------------------------------------------------

\normalsize

% Das esempio-Environment wird nur in der Leseansicht benötigt
\ifkorrekturansicht\else
\newenvironment{esempio}[3]%
{
    \vspace{1.5ex}
    \rlap{\underline{#1}}
    \par
    \setlength{\parindent}{0cm}
    \nopagebreak
    \leftskip=#2cm
    \rightskip=#3cm
}
{
    \par
}
\fi

\doendnotes{C}
\bigskip
\vfill

\clearpage

\footnotesize

\ifkorrekturansicht
  \lohead{\textsc{register}}
\fi

% theindex-Environment neu definieren ohne reledmac
\makeatletter
\renewenvironment{theindex}{%
  \ifkorrekturansicht
    \section*{\indexname}%
  \else
    \subsubsection*{Index der erwähnten Entitäten}%
  \fi
  \setlength{\parindent}{0pt}%
  \setlength{\parskip}{0pt plus 0.3pt}%
  \let\item\@idxitem
}{%
  \ifkorrekturansicht\clearpage\fi
}
\makeatother

\IfFileExists{\jobname-pw.ind}{\input{\jobname-pw.ind}}{}

% Quellenangabe nur in der Leseansicht
\ifkorrekturansicht\else
% Fallback-Definitionen, falls die .tex-Datei \titel etc. nicht gesetzt hat
\providecommand{\titel}{}
\providecommand{\editorInnen}{}
\providecommand{\dateiname}{\jobname}

\vspace{3cm}

\vfill

\footnotesize
\textsc{Quelle}: \titel. Herausgegeben von {\editorInnen}. In: \emph{Arthur Schnitzler: Briefwechsel mit Autorinnen und Autoren}.
 Digitale Edition, https://schnitzler-briefe.acdh.oeaw.ac.at/{\dateiname}.html (Stand \today)
\fi

\end{document}


      