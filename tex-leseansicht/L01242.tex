%% latex-korrekturansicht-vorspann.tex
%% Vorspann für die Korrekturansicht.
%% Lädt die gemeinsame Datei latex-vorspann.tex mit gesetztem Schalter.

\newif\ifkorrekturansicht
\korrekturansichttrue

\input{../tex-inputs/latex-vorspann}


\section[Hermann Bahr an Arthur Schnitzler, 15. 10. 1902]{L01242 Hermann Bahr an Arthur Schnitzler, 15. 10. 1902}
\nopagebreak\mylabel{L01242v}
\rehead{ }\normalsize\beginnumbering\briefempfaengerindex{Schnitzler, Arthur@\textsc{Schnitzler, Arthur}!zzzBahr, Hermann@\emph{von Hermann Bahr}!1902-10-152@{15. 10. 1902}|(be}
\toendnotes[C]{\smallbreak\pagebreak[2]}\Standort{CUL, Schnitzler, B 5b.}
\physDesc{Brief, 1 Blatt, 1 Seite, 367 Zeichen
\newline{}Handschrift: schwarze Tinte, deutsche Kurrent
\newline{}Ordnung: mit Bleistift von unbekannter Hand nummeriert:
                                    »91« }
\buchAbdrucke{\weitereDrucke{Hermann Bahr, Arthur Schnitzler: \emph{Briefwechsel, Aufzeichnungen, Dokumente (1891–1931)}. Göttingen: \emph{Wallstein} 2018, S. 244.} }\toendnotes[C]{\smallbreak}
\pstart
           {\pb}\textcolor{gray}{\textbf{GRAND HÔTEL\oindex{Hotel de Rome@\textbf{Hotel de Rome}, \emph{Hotel (K.HTL)}|pw}}}\hfill \textcolor{gray}{\textbf{Berlin N. W.\oindex{Berlin@\textbf{Berlin}, \emph{P.PPLC}|pw}, den .......... 190}}\pend
           
\pstart
           \textcolor{gray}{\textbf{DE ROME U. DU NORD}}\hfill \textcolor{gray}{\textbf{Unter den Linden 39.}}\oindex{Unter den Linden@\textbf{Unter den Linden}, \emph{P.PPLX}|pw}\pend
           
\pstart
           \textcolor{gray}{\textbf{A. MÜHLING}}\pwindex{Muehling, Adolph 1819 – 1910@\textsc{Mühling, Adolph} (1819 – 1910), \emph{Hotelbesitzer/Hotelbesitzerin}|pw}\hfill 15. 10\pend
           
\pstart
           \textcolor{gray}{\textbf{Kgl. Hoflieferant}}\pend
           
\pstart
           \textcolor{gray}{\textbf{BERLIN}}\oindex{Berlin@\textbf{Berlin}, \emph{P.PPLC}|pw}\pend
           
\pstart
           \textcolor{gray}{\textbf{Fernsprecher: Amt I, No. 4438.}}\pend
           
\pstart\center{}Lieber Arthur!\pend\vspace{0.5em}
\pstart
           Herzlichſten Dank! In einer Zeitung las ich: Halm\pwindex{Halm, Alfred 1861-12-09 – 1951-02-05@\textsc{Halm, Alfred} (1861-12-09 – 1951-02-05), \emph{Schauspieler/Schauspielerin}|pw} hätte als D\textsuperscript{r}{ }\label{K_L01242-1v}\edtext{Mohn\pwindex{Wienerinnen. Lustspiel in drei Akten@\emph{Wienerinnen. Lustspiel in drei Akten}|pwv}}{\lemma{\textnormal{\emph{Mohn}}}\Cendnote{\textnormal{Figur aus \emph{Wienerinnen}\pwindex{Wienerinnen. Lustspiel in drei Akten@\emph{Wienerinnen. Lustspiel in drei Akten}|pwk}}}}\label{K_L01242-1}{ }\label{K_L01242-2v}\edtext{Deine Maske gehabt}{\lemma{\textnormal{\emph{Deine Maske gehabt}}}\Cendnote{\textnormal{nicht nachgewiesen; vielleicht eine
                  Fehlleistung Bahrs\pwindex{Bahr, Hermann 19.07.1863 – 15.01.1934@\textsc{Bahr, Hermann} (19.07.1863 – 15.01.1934), \emph{Schriftsteller/Schriftstellerin, Kritiker/Kritikerin}|pwk} zur Rezension\pwindex{Berliner Theater. Hermann Bahr: »Wienerinnen«. (Eine nicht einwandfreie Kritik)@\emph{Berliner Theater. Hermann Bahr: »Wienerinnen«. (Eine nicht einwandfreie Kritik)}|pwkv} von Karl Strecker\pwindex{Strecker, Karl 1862-04-08 – 1933-02-19@\textsc{Strecker, Karl} (1862-04-08 – 1933-02-19), \emph{Theaterkritiker/Theaterkritikerin}|pwk}: »Herr Halm\pwindex{Halm, Alfred 1861-12-09 – 1951-02-05@\textsc{Halm, Alfred} (1861-12-09 – 1951-02-05), \emph{Schauspieler/Schauspielerin}|pw}, der auch die Regie führte, gab einen modernen Ästheten mit
                     gedrehter Stirnlocke, einen eitlen Faiseur, seltsamerweise aber in der Maske
                     von Hermann Bahr\pwindex{Bahr, Hermann 19.07.1863 – 15.01.1934@\textsc{Bahr, Hermann} (19.07.1863 – 15.01.1934), \emph{Schriftsteller/Schriftstellerin, Kritiker/Kritikerin}|pw}.« (\emph{Berliner Theater. Hermann Bahr: »Wienerinnen«.
                        (Eine nicht einwandfreie Kritik)}\pwindex{Berliner Theater. Hermann Bahr: »Wienerinnen«. (Eine nicht einwandfreie Kritik)@\emph{Berliner Theater. Hermann Bahr: »Wienerinnen«. (Eine nicht einwandfreie Kritik)}|pwk}. In: \emph{Tägliche Rundschau}\pwindex{Taegliche Rundschau@\emph{Tägliche Rundschau}|pwk}, Jg. 20, Nr. 483, Morgenblatt, 1. Ausgabe,
                        15. 10. 1902, S. [2]). Vgl. A. S.: \emph{Tagebuch}, 18. 10. 1894.}}}\label{K_L01242-2}. Wahr iſt, daß er einen blonden Vollbart trug, aus lauter Angſt, in die Maske
                  Sudermanns\pwindex{Sudermann, Hermann 30.09.1857 – 21.11.1928@\textsc{Sudermann, Hermann} (30.09.1857 – 21.11.1928), \emph{Schriftsteller/Schriftstellerin}|pw} zu gerathen. Daß es ganz albern
               wäre, einem ſpöttelnden Salon-Kritiker Deine Züge zu geben, brauche ich Dir ja nicht
               erſt zu ſagen. Die Leut ſind ſo blöd!\pend
           
\pstart
           Herzlichſt{\\[\baselineskip]}Dein{\\[\baselineskip]}\spacefill\mbox{Hermann}\pend
           \leftskip=0em{}\selectlanguage{ngerman}\endnumbering\briefempfaengerindex{Schnitzler, Arthur@\textsc{Schnitzler, Arthur}!zzzBahr, Hermann@\emph{von Hermann Bahr}!1902-10-152@{15. 10. 1902}|)be}\mylabel{L01242h}  \normalsize

\doendnotes{C}
\bigskip
\vfill

\clearpage

\footnotesize

\lohead{\textsc{register}}

% Definiere theindex-Environment komplett neu ohne reledmac
\makeatletter
\renewenvironment{theindex}{%
  \section*{\indexname}%
  \setlength{\parindent}{0pt}%
  \setlength{\parskip}{0pt plus 0.3pt}%
  \let\item\@idxitem
}{%
  \clearpage
}
\makeatother

\IfFileExists{\jobname-pw.ind}{\input{\jobname-pw.ind}}{}

\end{document}

      