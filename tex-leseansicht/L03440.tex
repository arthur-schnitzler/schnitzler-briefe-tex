%% latex-leseansicht-vorspann.tex
%% Vorspann für die Leseansicht.
%% Lädt die gemeinsame Datei latex-vorspann.tex mit nicht gesetztem Schalter.

\newif\ifkorrekturansicht
\korrekturansichtfalse

\input{../tex-inputs/latex-vorspann}


\section[ Paul Goldmann an Arthur Schnitzler, 14. 3. {[}1904{]}]{L03440 Paul Goldmann an Arthur Schnitzler,  14. 3. [1904]}
\nopagebreak\mylabel{L03440v}
\rehead{ }\normalsize\beginnumbering\briefempfaengerindex{Schnitzler, Arthur@\textsc{Schnitzler, Arthur}!zzzGoldmann, Paul@\emph{von Paul Goldmann}!1904-03-142@{14. 3. [1904]}|(be}
\toendnotes[C]{\smallbreak\pagebreak[2]}
\correspDesc{Versand  durch Paul Goldmann am 14. 3. [1904] in Berlin
\newline{}Erhalt  durch Arthur Schnitzler im Zeitraum [15. 3. 1904
                  – 19. 3. 1904?] in Wien}\toendnotes[C]{\smallbreak}
\Standort{DLA, A:Schnitzler, HS.NZ85.1.3174.}
\physDesc{Brief, 2 Blätter, 5 Seiten, 3192 Zeichen
\newline{}Handschrift: blaue Tinte, deutsche Kurrent
\newline{}Schnitzler: 1) mit Bleistift das Jahr »904« vermerkt  2) mit rotem Buntstift eine Unterstreichung}\toendnotes[C]{\smallbreak}
\pstart
           \raggedleft{}{\pb}\textcolor{gray}{\textbf{DESSAUERSTRASSE 19\oindex{Dessauer Straße@\textbf{Dessauer Straße}, \emph{Straße}|pw}}}\pend
           
\pstart
           Berlin\oindex{Berlin@\textbf{Berlin}, \emph{Hauptstadt}|pw}, 14. März.\pend
           
\pstart{}Mein lieber Freund,\pend\vspace{0.5em}
\pstart
           Dein lieber Brief, der mich, wenigſtens durch{ }ſeinen Schlußabſatz,{ }ſehr erfreut hat,
               traf mich inmitten einer{ }ſtürmiſch \strikeout{be} bewegten Zeit.
               Meine Freundin\pwindex{Rottenberg, Theodore 7.\,9.\,1875 – 5.\,4.\,1945 Limburg an der Lahn@\textsc{Rottenberg, Theodore} (7.\,9.\,1875 – 5.\,4.\,1945 Limburg an der Lahn)|pwv} war – aus
               Gründen, die Du Dir denken kannſt – erkrankt, \strikeout{ſie}{ }ſie
               hat längere Zeit hier\oindex{Berlin@\textbf{Berlin}, \emph{Hauptstadt}|pwv} auf einer
               Klinik gelegen, auch jetzt iſt{ }ſie noch recht leidend und immer noch hier. Ich habe
               viel Aufregungen und Sorgen durchgemacht, und{ }ſo kommt es, daß ich \introOben{}für\introOben{} Deinen Brief, den ich, wenn ich meinem Wunſche hätte \strikeout{\textcolor{gray}{erf}} folgen können,{ }ſofort beantwortet hätte, Dir erſt heute danken kann.\pend
           
\pstart
           Ich unterlaſſe es, auf das Einzelne {\pb}einzugehen.
                  \label{K_L03440-1v}\edtext{Äußerungen}{\lemma{\textnormal{\emph{Äußerungen}}}\Cendnote{\textnormal{Schnitzlers nicht überlieferter Brief dürfte
                  eine Abrechnung mit Goldmanns\pwindex{Goldmann, Paul 31.\,1.\,1865 Breslau – 25.\,9.\,1935 Wien@\textsc{Goldmann, Paul} (31.\,1.\,1865 Breslau – 25.\,9.\,1935 Wien), \emph{Schriftsteller, Journalist}|pwk}{ }Rezension\pwindex{Goldmann, Paul 31.\,1.\,1865 Breslau – 25.\,9.\,1935 Wien@\textsc{Goldmann, Paul} (31.\,1.\,1865 Breslau – 25.\,9.\,1935 Wien), \emph{Schriftsteller, Journalist}!Berliner Theater. »Der einsame Weg«. Von Arthur Schnitzler@\strich\emph{Berliner Theater. »Der einsame Weg«. Von Arthur Schnitzler}|pwkv} zur Uraufführung
                  von \emph{Der einsame Weg}\pwindex{Schnitzler, Arthur 15.\,5.\,1862 Wien – 21.\,10.\,1931 ebd.@\textsc{Schnitzler, Arthur} (15.\,5.\,1862 Wien – 21.\,10.\,1931 ebd.), \emph{Schriftsteller, Mediziner}!einsame Weg. Schauspiel in fünf Akten@\strich\emph{Der einsame Weg. Schauspiel in fünf Akten}|pwk} (13. 2. 1904, Deutsches Theater Berlin\oindex{Deutsches Theater Berlin@\textbf{Deutsches Theater Berlin}, \emph{Theater}|pwk}) enthalten haben.
                     Vgl. Paul Goldmann\pwindex{Goldmann, Paul 31.\,1.\,1865 Breslau – 25.\,9.\,1935 Wien@\textsc{Goldmann, Paul} (31.\,1.\,1865 Breslau – 25.\,9.\,1935 Wien), \emph{Schriftsteller, Journalist}|pwk}: \emph{Berliner Theater. »Der einsame Weg«. Von Arthur
                        Schnitzler}\pwindex{Goldmann, Paul 31.\,1.\,1865 Breslau – 25.\,9.\,1935 Wien@\textsc{Goldmann, Paul} (31.\,1.\,1865 Breslau – 25.\,9.\,1935 Wien), \emph{Schriftsteller, Journalist}!Berliner Theater. »Der einsame Weg«. Von Arthur Schnitzler@\strich\emph{Berliner Theater. »Der einsame Weg«. Von Arthur Schnitzler}|pwk}. In: \emph{Neue Freie Presse}\pwindex{Neue Freie Presse@\emph{Neue Freie Presse}|pwk},
                     Nr. 14.187, 23. 2. 1904, Morgenblatt,
                     S. 1–3.}}}\label{K_L03440-1} in Deinem Briefe wie »Dein kritiſches Gebahren«, – die
               Meinung, ich hätte Dir zugemuthet, das Stück\pwindex{Schnitzler, Arthur 15.\,5.\,1862 Wien – 21.\,10.\,1931 ebd.@\textsc{Schnitzler, Arthur} (15.\,5.\,1862 Wien – 21.\,10.\,1931 ebd.), \emph{Schriftsteller, Mediziner}!einsame Weg. Schauspiel in fünf Akten@\strich\emph{Der einsame Weg. Schauspiel in fünf Akten}|pwv}{ }ſtatt als Trauerſpiel als Luſtſpiel zu{ }ſchreiben – die
               Aufforderung »ich{ }ſollte \strikeout{D} mir den Inhalt des Ganzen
               einmal überlegen«, – die Anſicht, ich wiſſe nicht immer »mit{ }ſoviel Klugheit und
               Würde zu wägen« \textsc{etc.} – das alles zeigt mir nur von Neuem,
               wie unrichtig Du \strikeout{me} meine kritiſche Thätigkeit
               beurtheilſt und \strikeout{\textcolor{gray}{mit}} wie{ }ſehr es Dir (wenn Du auch mir ein offenes Wort erlaubſt) an Verſtändniß
               für den Ernſt und die Höhe meines Strebens fehlt. Darüber läßt{ }ſich, meiner Anſicht,
               nicht diskutiren, und Diskuſſionen{ }ſchaffen nur \strikeout{eine}
               unnütze Verbitterung in einem Fall, wo, wie in dem {\pb}unſerigen, nicht eine Verſchiedenheit der Anſichten,{ }ſondern eine Verſchiedenheit
               der Standpunkte vorliegt, die ihren Grund wohl darin haben, daß \substVorne{}\textsuperscript{ſich d\textcolor{gray}{×}\-\textcolor{gray}{×}}\substDazwischen{}unſere\substHinten{} Lebenswege{ }ſich{ }ſeit Langem getrennt und in verſchiedenen Richtungen bewegt
               haben.\pend
           
\pstart
           Eines nur bitte ich Dich, mir zu glauben: Es gehört zu den peinlichſten Aufgaben
               meiner Stellung, ein Stück von Dir \strikeout{\textcolor{gray}{×}} kritiſiren zu müſſen, wenn ich nicht ganz damit einverſtanden bin; und ich
               habe den{ }ſehnlichen Wunſch, Dein nächſtes Stück möge{ }ſo{ }ſchön{ }ſein, daß ich mit
               rückhaltsloſer Anerkennung darüber berichten kann, oder es \strikeout{\textcolor{gray}{×}} möge mir überhaupt erſpart bleiben, darüber zu berichten{\dotsfive}\pend
           
\pstart
           Von ganzem Herzen \strikeout{abe\textcolor{gray}{r}} aber{ }ſtimme ich dem Schluß Deines Briefes zu, und ich danke Dir für dieſe
               lieben {\pb}und{ }ſchönen Worte. Du haſt ganz recht, wenn
               Du{ }ſagſt, daß da\textcolor{gray}{s} Beſte gelebt und nicht geſchrieben wird.
               Vielleicht wird es gut{ }ſein, wenn wir fürs Erſte überhaupt vermeiden, über Literatur
               zu{ }ſprechen. Aber im großen Leben bildet die Literatur ja nur ein ganz kleines
               Gebiet, und es bleibt noch Raum genug für eine Freundſchaft die auf dieſem
               literariſchen Gebiete nicht mehr zuſammengehen kann. Was mich anlangt,{ }ſo hoffe ich
               Dir dieſe Freundſchaft noch oft beweiſen zu können; und \strikeout{wenn} wenn Du mir Deine Hände reichſt,{ }ſo wirſt Du die meinen immer bereit
               finden,{ }ſie \strikeout{\textcolor{gray}{×}\-\textcolor{gray}{×}\-\textcolor{gray}{×}} in alter Treue und Herzlichkeit zu drücken.\pend
           
\pstart
           \strikeout{D\textcolor{gray}{×}\-\textcolor{gray}{×}} Ich merke aber, daß ich ein wenig in die großen Worte hineingerathen bin. Das
               iſt überflüſſig, und ich denke, wir Zwei verſtehen uns {\pb}auch ohne dieſe{ }ſehr gut und werden uns – im Weſentlichen – immer verſtehen{\dotsfour}\pend
           
\pstart
           Ich hoffe, daß dieſer Brief Dich bereits inmitten der Vorbereitungen zur \label{K_L03440-2v}\edtext{ſicili\oindex{Sizilien@\textbf{Sizilien}, \emph{Land}|pwv}aniſchen Reiſe}{\lemma{\textnormal{\emph{sicilianischen Reise}}}\Cendnote{\textnormal{Zwischen 1. 5. 1904 und 29. 5. 1904 reisten Arthur und Olga
                     Schnitzler\pwindex{Schnitzler, Olga 17.\,1.\,1882 Wien – 13.\,1.\,1970 Lugano@\textsc{Schnitzler, Olga} (17.\,1.\,1882 Wien – 13.\,1.\,1970 Lugano), \emph{Schauspielerin, Sängerin}|pwk} nach Italien\oindex{Italien@\textbf{Italien}|pwk}. Die
                  Hauptstationen bildeten Rom\oindex{Rom@\textbf{Rom}, \emph{Hauptstadt}|pwk}, Neapel\oindex{Neapel@\textbf{Neapel}|pwk}, Pompeji\oindex{Pompeji@\textbf{Pompeji}, \emph{Ausgrabung}|pwk}, Palermo\oindex{Palermo@\textbf{Palermo}|pwk} und Taormina\oindex{Taormina@\textbf{Taormina}, \emph{Hauptstadt}|pwk}.}}}\label{K_L03440-2} trifft. Zu meiner Freude höre ich, daß der »Einſame Weg\pwindex{Schnitzler, Arthur 15.\,5.\,1862 Wien – 21.\,10.\,1931 ebd.@\textsc{Schnitzler, Arthur} (15.\,5.\,1862 Wien – 21.\,10.\,1931 ebd.), \emph{Schriftsteller, Mediziner}!einsame Weg. Schauspiel in fünf Akten@\strich\emph{Der einsame Weg. Schauspiel in fünf Akten}|pw}« dem Berlin\oindex{Berlin@\textbf{Berlin}, \emph{Hauptstadt}|pw}er Publikum gefällt und daß das Theater\oindex{Deutsches Theater Berlin@\textbf{Deutsches Theater Berlin}, \emph{Theater}|pwv} immer voll iſt. Laß’ mich wiſſen, wie es Dir und Deiner
               kleinen Familie\pwindex{Schnitzler, Olga 17.\,1.\,1882 Wien – 13.\,1.\,1970 Lugano@\textsc{Schnitzler, Olga} (17.\,1.\,1882 Wien – 13.\,1.\,1970 Lugano), \emph{Schauspielerin, Sängerin}|pwv}\pwindex{Schnitzler, Heinrich 9.\,8.\,1902 Hinterbrühl – 12.\,7.\,1982 Wien@\textsc{Schnitzler, Heinrich} (9.\,8.\,1902 Hinterbrühl – 12.\,7.\,1982 Wien), \emph{Regisseur, Schauspieler}|pwv}
               geht, und{ }ſei herzlichſt gegrüßt von Deinem getreuen \spacefill\mbox{Paul Goldmann}\pend
           
\pstart
           \noindent{}Meine Freundin\pwindex{Rottenberg, Theodore 7.\,9.\,1875 – 5.\,4.\,1945 Limburg an der Lahn@\textsc{Rottenberg, Theodore} (7.\,9.\,1875 – 5.\,4.\,1945 Limburg an der Lahn)|pwv} bittet
                  mich, Dich zu grüßen.\pend
           \selectlanguage{ngerman}\endnumbering\briefempfaengerindex{Schnitzler, Arthur@\textsc{Schnitzler, Arthur}!zzzGoldmann, Paul@\emph{von Paul Goldmann}!1904-03-142@{14. 3. [1904]}|)be}\mylabel{L03440h}  \newcommand{\dateiname}{L03440}\newcommand{\titel}{Paul Goldmann an Arthur Schnitzler, 14. 3. [1904]}\newcommand{\editorInnen}{Martin Anton Müller und Laura Untner}%% latex-leseansicht-abspann.tex
%% Abspann für die Leseansicht.
%% Der Schalter \ifkorrekturansicht ist bereits durch den Vorspann gesetzt.

%% latex-abspann.tex
%% Gemeinsamer Abspann für Korrekturansicht und Leseansicht.
%% Setzt den Schalter \ifkorrekturansicht voraus (gesetzt in den
%% einbindenden Dateien latex-korrekturansicht-abspann.tex bzw.
%% latex-leseansicht-abspann.tex).
%% ---------------------------------------------------------------

\normalsize

% Das esempio-Environment wird nur in der Leseansicht benötigt
\ifkorrekturansicht\else
\newenvironment{esempio}[3]%
{
    \vspace{1.5ex}
    \rlap{\underline{#1}}
    \par
    \setlength{\parindent}{0cm}
    \nopagebreak
    \leftskip=#2cm
    \rightskip=#3cm
}
{
    \par
}
\fi

\doendnotes{C}
\bigskip
\vfill

\clearpage

\footnotesize

\ifkorrekturansicht
  \lohead{\textsc{register}}
\fi

% theindex-Environment neu definieren ohne reledmac
\makeatletter
\renewenvironment{theindex}{%
  \ifkorrekturansicht
    \section*{\indexname}%
  \else
    \subsubsection*{Index der erwähnten Entitäten}%
  \fi
  \setlength{\parindent}{0pt}%
  \setlength{\parskip}{0pt plus 0.3pt}%
  \let\item\@idxitem
}{%
  \ifkorrekturansicht\clearpage\fi
}
\makeatother

\IfFileExists{\jobname-pw.ind}{\input{\jobname-pw.ind}}{}

% Quellenangabe nur in der Leseansicht
\ifkorrekturansicht\else
% Fallback-Definitionen, falls die .tex-Datei \titel etc. nicht gesetzt hat
\providecommand{\titel}{}
\providecommand{\editorInnen}{}
\providecommand{\dateiname}{\jobname}

\vspace{3cm}

\vfill

\footnotesize
\textsc{Quelle}: \titel. Herausgegeben von {\editorInnen}. In: \emph{Arthur Schnitzler: Briefwechsel mit Autorinnen und Autoren}.
 Digitale Edition, https://schnitzler-briefe.acdh.oeaw.ac.at/{\dateiname}.html (Stand \today)
\fi

\end{document}


