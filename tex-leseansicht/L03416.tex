%% latex-leseansicht-vorspann.tex
%% Vorspann für die Leseansicht.
%% Lädt die gemeinsame Datei latex-vorspann.tex mit nicht gesetztem Schalter.

\newif\ifkorrekturansicht
\korrekturansichtfalse

\input{../tex-inputs/latex-vorspann}


\section[ Felix Salten an Arthur Schnitzler, 28. 3. 1906]{L03416 Felix Salten an Arthur Schnitzler,  28. 3. 1906}
\nopagebreak\mylabel{L03416v}
\rehead{ }\normalsize\beginnumbering\briefempfaengerindex{Schnitzler, Arthur@\textsc{Schnitzler, Arthur}!zzzSalten, Felix@\emph{von Felix Salten}!1906-03-281@{28. 3. 1906}|(be}
\toendnotes[C]{\smallbreak\pagebreak[2]}
\correspDesc{Versand  durch Felix Salten am 28. 3. 1906 in Berlin
\newline{}Erhalt  durch Arthur Schnitzler im Zeitraum [29. 3. 1906
                  – 2. 4. 1906?] in Wien}\toendnotes[C]{\smallbreak}
\Standort{CUL, Schnitzler, B 89, B 1.}
\physDesc{Brief, 1 Blatt, 2 Seiten, 3211 Zeichen
\newline{}Handschrift: schwarze Tinte, lateinische Kurrent
\newline{}Ordnung: mit Bleistift von unbekannter Hand nummeriert: »207« }
\buchAbdrucke{\weitereDrucke{Hermann Bahr, Arthur Schnitzler: \emph{Briefwechsel, Aufzeichnungen, Dokumente (1891–1931)}. Herausgegeben von Kurt Ifkovits und Martin Anton Müller. Göttingen: \emph{Wallstein} 2018, S. 376–377.} }\toendnotes[C]{\smallbreak}
\pstart
           {\pb}\textcolor{gray}{\textbf{\emph{B. Z. am Mittag}}}\orgindex{B.Z. am Mittag@B.Z. am Mittag|pw}\hfill \textcolor{gray}{\textbf{\emph{BERLIN SW\oindex{Berlin@\textbf{Berlin}, \emph{Hauptstadt}|pw}},}}{ }28. III. 06\pend
           
\pstart
           \textcolor{gray}{\textbf{\emph{Chefredaktion}}}\hfill \textcolor{gray}{\textbf{\emph{Kochstr. 23–25}}}\oindex{Kochstraße@\textbf{Kochstraße}, \emph{Straße}|pw}\pend
           \vspace{0.5em}
\pstart
           Lieber, dass wir eine \label{K_L03416-1v}\edtext{Radtour machen}{\lemma{\textnormal{\emph{Radtour machen}}}\Cendnote{\textnormal{Diese fand nicht statt,
                     siehe XXXX Auszeichnungsfehler: Dokument L03422 nicht gefunden. Vgl. XXXX Auszeichnungsfehler: Dokument L03427 nicht gefunden. }}}\label{K_L03416-1} könnten, ist
               mir heute wie ein absolutes Muß! Es wäre so schön 6–8 Tage irgendwo durch die Welt zu
               gleiten, wo sie schön ist, und wo man wieder einmal so viel Behagen empfinden könnte,
               wie »einst im Mai«{[}.{]} Denken Sie etwas Gutes aus, und ziehen Sie
               dabei in Betracht, ob wir nicht eine Gegend wählen wollen, die wir noch nicht kennen.
               Deutsches Gebirge, Thüringen\oindex{Thüringen@\textbf{Thüringen}, \emph{Land}|pw}, Rhein\oindex{Rheinland@\textbf{Rheinland}|pw}, u. s. w. Ich bin aber auch mit Tirol\oindex{Tirol@\textbf{Tirol}, \emph{Land}|pw}\oindex{Südtirol@\textbf{Südtirol}, \emph{Verwaltungsgebiet}|pw} oder Schweiz\oindex{Schweiz@\textbf{Schweiz}|pw} (Lugano\oindex{Lugano@\textbf{Lugano}, \emph{Hauptstadt}|pw} oder Genfer See\oindex{Genfer See@\textbf{Genfer See}, \emph{See}|pw}) einverstanden. Ihr Brief kam heute aber auch \label{K_L03416-2v}\edtext{\begin{otherlanguage}{italian}a tempo\end{otherlanguage}}{\lemma{\textnormal{\emph{a tempo}}}\Cendnote{\textnormal{italienisch: zur rechten Zeit}}}\label{K_L03416-2}: es
               ist \substVorne{}\textsuperscript{\textcolor{gray}{jetzt}}\substDazwischen{}nach\substHinten{} langem Winter wieder die erste Frühlingswärme, die erste Sonne wieder da,
               und alle Reisepläne, alles Reiseverlangen – »Wanderlust« – regt sich. An solchen
               Tagen hat auch Berlin\oindex{Berlin@\textbf{Berlin}, \emph{Hauptstadt}|pw} seine Schönheit. An solchen
               Tagen würde übrigens auch Magdeburg\oindex{Magdeburg@\textbf{Magdeburg}|pw} oder Genthinen\oindex{Genthin@\textbf{Genthin}|pw} nicht ohne Reiz sein. Ich überlege mir
               heute zum 20\textsuperscript{ten} Mal, wie man es macht, sich ein ganz ein
               kleines Automobil zu kaufen. Geht aber leider im Moment nicht. Wenn ich die große
               Zeitung gegründet habe, Neue freie Presse\orgindex{Neue Freie Presse@Neue Freie Presse|pw} in Berlin\oindex{Berlin@\textbf{Berlin}, \emph{Hauptstadt}|pw}, eine Wochenschrift im Zukunft\orgindex{Zukunft@Die Zukunft|pw}-Stil und dann vier Blätter regiere, statt zwei\orgindex{Berliner Morgenpost@Berliner Morgenpost|pwv}\orgindex{B.Z. am Mittag@B.Z. am Mittag|pwv} (was ich armselig
                  finde){[},{]} dann werde ich gewiss auch das langerflehte Auto
               haben. Inzwischen freu ich mich, wenn nur eine Radtour zustande kommt, und die
               übrigen Dinge, die ich für den Sommer vorhabe (Holland\oindex{Niederlande@\textbf{Niederlande}|pw}, zu Wasser nach Kiel\oindex{Kiel@\textbf{Kiel}|pw}){[}.{]} Die Radtour könnte auch durch einige deutsch\oindex{Deutschland@\textbf{Deutschland}|pwv}e Städte gemacht werden,
               – Rothenburg ob. d. Tauber\oindex{Rothenburg ob der Tauber@\textbf{Rothenburg ob der Tauber}|pw} – Bayreuth\oindex{Bayreuth@\textbf{Bayreuth}, \emph{Hauptstadt}|pw}, wozu man freilich jetzt schon die Sitze bestellen
               müsste. Das \label{K_L03416-3v}\edtext{dänische Seebad\oindex{Marienlyst@\textbf{Marienlyst}, \emph{Gut}|pw}}{\lemma{\textnormal{\emph{dänische Seebad}}}\Cendnote{\textnormal{Schnitzler war zwischen 28. 6. 1906 und 9. 8. 1906 in Marienlyst\oindex{Marienlyst@\textbf{Marienlyst}, \emph{Gut}|pwk}. Felix\pwindex{Salten, Felix 6.\,9.\,1869 Budapest – 8.\,10.\,1945 Zürich@\textsc{Salten, Felix} (6.\,9.\,1869 Budapest – 8.\,10.\,1945 Zürich), \emph{Schriftsteller, Journalist, Chefredakteur}|pwk} und Ottilie Salten\pwindex{Salten, Ottilie 7.\,3.\,1868 Prag – 22.\,6.\,1942 Zürich@\textsc{Salten, Ottilie} (7.\,3.\,1868 Prag – 22.\,6.\,1942 Zürich), \emph{Schauspielerin}|pwk} besuchten
                  ihn dort am 2. 8. 1906.}}}\label{K_L03416-3}, das Sie vorhaben, verdrießt mich – wenn ich
               aufrichtig sein darf – immer. Weil ich {\dotstwo} aus
               wirthschaftlichen Gründen {\dotstwo} nicht hinkann, wenn ich schon
               einmal an der Ostsee\oindex{Ostsee@\textbf{Ostsee}|pw} sitze, und weil ich mir
               denke, wenn uns ein mehrwöchiges Beisammensein schon beschieden sein könnte, dann
               ließe sich vielleicht doch auf Dänemark\oindex{Dänemark@\textbf{Dänemark}|pw}
               verzichten. Der Unterschied ist nicht so groß, und Wälder gibt’s auch am diesseitigen
               Strand der Ostsee\oindex{Ostsee@\textbf{Ostsee}|pw}.\pend
           
\pstart
           Augenblicklich ist Wien\oindex{Wien@\textbf{Wien}, \emph{Verwaltungsgebiet}|pw} durch M\textsuperscript{r}{ }\label{K_L03416-4v}\edtext{Triebeitsch\pwindex{Trebitsch, Siegfried 22.\,12.\,1868 Wien – 3.\,6.\,1956 Zürich@\textsc{Trebitsch, Siegfried} (22.\,12.\,1868 Wien – 3.\,6.\,1956 Zürich), \emph{Schriftsteller, Übersetzer}|pw}}{\lemma{\textnormal{\emph{Triebeitsch}}}\Cendnote{\textnormal{Hier findet das Naserümpfen über Trebitsch\pwindex{Trebitsch, Siegfried 22.\,12.\,1868 Wien – 3.\,6.\,1956 Zürich@\textsc{Trebitsch, Siegfried} (22.\,12.\,1868 Wien – 3.\,6.\,1956 Zürich), \emph{Schriftsteller, Übersetzer}|pwk} eine Form, in der die Herabsetzung
                  durch die Imitation einer englischen Aussprache seines Namens erfolgt.}}}\label{K_L03416-4}
               vertreten, der in seinem \label{K_L03416-5v}\edtext{Premiere\pwindex{Shaw, George Bernard 26.\,7.\,1856 Dublin – 2.\,11.\,1950 Ayot Saint Lawrence@\textsc{Shaw, George Bernard} (26.\,7.\,1856 Dublin – 2.\,11.\,1950 Ayot Saint Lawrence), \emph{Schriftsteller}!Cäsar und Cleopatra. Eine historische Komödie@\strich\emph{Cäsar und Cleopatra. Eine historische Komödie}|pwv}nfieber}{\lemma{\textnormal{\emph{Premierenfieber}}}\Cendnote{\textnormal{Am 31. 3. 1906 fand am \emph{Neuen Theater}\orgindex{Neues Theater@Neues Theater|pwk}
                  die deutschsprachige Uraufführung von \emph{Caesar und
                     Cleopatra}\pwindex{Shaw, George Bernard 26.\,7.\,1856 Dublin – 2.\,11.\,1950 Ayot Saint Lawrence@\textsc{Shaw, George Bernard} (26.\,7.\,1856 Dublin – 2.\,11.\,1950 Ayot Saint Lawrence), \emph{Schriftsteller}!Cäsar und Cleopatra. Eine historische Komödie@\strich\emph{Cäsar und Cleopatra. Eine historische Komödie}|pwk} von George Bernard Shaw\pwindex{Shaw, George Bernard 26.\,7.\,1856 Dublin – 2.\,11.\,1950 Ayot Saint Lawrence@\textsc{Shaw, George Bernard} (26.\,7.\,1856 Dublin – 2.\,11.\,1950 Ayot Saint Lawrence), \emph{Schriftsteller}|pwk} in
                  der Übersetzung von Siegfried Trebitsch\pwindex{Trebitsch, Siegfried 22.\,12.\,1868 Wien – 3.\,6.\,1956 Zürich@\textsc{Trebitsch, Siegfried} (22.\,12.\,1868 Wien – 3.\,6.\,1956 Zürich), \emph{Schriftsteller, Übersetzer}|pwk}
                  statt.}}}\label{K_L03416-5} wegen Shaw\pwindex{Shaw, George Bernard 26.\,7.\,1856 Dublin – 2.\,11.\,1950 Ayot Saint Lawrence@\textsc{Shaw, George Bernard} (26.\,7.\,1856 Dublin – 2.\,11.\,1950 Ayot Saint Lawrence), \emph{Schriftsteller}|pw} das Maß des
               Lächerlichen erreicht. Seine erste Frage, als er hier\oindex{Berlin@\textbf{Berlin}, \emph{Hauptstadt}|pwv} eintraf, war (natürlich per Telefon) was ich von seinem
                  \label{K_L03416-6v}\edtext{Vorschlag\pwindex{Trebitsch, Siegfried 22.\,12.\,1868 Wien – 3.\,6.\,1956 Zürich@\textsc{Trebitsch, Siegfried} (22.\,12.\,1868 Wien – 3.\,6.\,1956 Zürich), \emph{Schriftsteller, Übersetzer}!Bühnenvertrieb@\strich\emph{Bühnenvertrieb}|pwv}}{\lemma{\textnormal{\emph{Vorschlag}}}\Cendnote{\textnormal{Siegfried Trebitsch\pwindex{Trebitsch, Siegfried 22.\,12.\,1868 Wien – 3.\,6.\,1956 Zürich@\textsc{Trebitsch, Siegfried} (22.\,12.\,1868 Wien – 3.\,6.\,1956 Zürich), \emph{Schriftsteller, Übersetzer}|pwk}: \emph{Bühnenvertrieb}\pwindex{Trebitsch, Siegfried 22.\,12.\,1868 Wien – 3.\,6.\,1956 Zürich@\textsc{Trebitsch, Siegfried} (22.\,12.\,1868 Wien – 3.\,6.\,1956 Zürich), \emph{Schriftsteller, Übersetzer}!Bühnenvertrieb@\strich\emph{Bühnenvertrieb}|pwk}. In: \emph{Die
                        Schaubühne}\pwindex{Schaubühne@\emph{Die Schaubühne}|pwk}, Jg. 2, Nr. 12, 22. 3. 1906,
                     S. 348–350. Darin forderte Trebitsch\pwindex{Trebitsch, Siegfried 22.\,12.\,1868 Wien – 3.\,6.\,1956 Zürich@\textsc{Trebitsch, Siegfried} (22.\,12.\,1868 Wien – 3.\,6.\,1956 Zürich), \emph{Schriftsteller, Übersetzer}|pwk} die Einrichtung einer Bühnengenossenschaft zur Vertretung von
                  Autoren- und Autorinnenrechten. Das motivierte den Herausgeber der Zeitschrift\pwindex{Schaubühne@\emph{Die Schaubühne}|pwkv}, Siegfried Jacobsohn\pwindex{Jacobsohn, Siegfried 28.\,1.\,1881 Berlin – 3.\,12.\,1926 ebd.@\textsc{Jacobsohn, Siegfried} (28.\,1.\,1881 Berlin – 3.\,12.\,1926 ebd.), \emph{Journalist, Kritiker, Publizist}|pwk}, zu einer mehrteiligen Debatte, die
                  sich über Monate erstreckte. In der zweiten Fortsetzung\pwindex{Bund der Bühnendichter. II@\emph{Bund der Bühnendichter. II}|pwkv} findet sich ein Beitrag Schnitzlers. Siehe A. S.: \emph{»Das Zeitlose ist von kürzester Dauer«}, Bund der Bühnendichter, 12. 4. 1906.}}}\label{K_L03416-6} in der »Schaubühne\pwindex{Schaubühne@\emph{Die Schaubühne}|pw}« halte. Ich sagte, dass ich dagegen
               sei. Er ließ seinen erstaunten Klagelaut vernehmen, und meinte dann, {\pb}\uline{Sie} hätten ihm\pwindex{Trebitsch, Siegfried 22.\,12.\,1868 Wien – 3.\,6.\,1956 Zürich@\textsc{Trebitsch, Siegfried} (22.\,12.\,1868 Wien – 3.\,6.\,1956 Zürich), \emph{Schriftsteller, Übersetzer}|pwv} einen »begeisterten« Brief geschrieben. Ich bin wirklich
               nicht sehr für diesen Vorschlag, der nur aus der \label{K_L03416-7v}\edtext{Seidenbranche}{\lemma{\textnormal{\emph{Seidenbranche}}}\Cendnote{\textnormal{Anspielung auf Trebitschs\pwindex{Trebitsch, Siegfried 22.\,12.\,1868 Wien – 3.\,6.\,1956 Zürich@\textsc{Trebitsch, Siegfried} (22.\,12.\,1868 Wien – 3.\,6.\,1956 Zürich), \emph{Schriftsteller, Übersetzer}|pwk}
                  großindustriellen Hintergrund}}}\label{K_L03416-7} kommt; glaube an Ihre »Begeisterung«
               natürlich nicht, und halte die ganze Sache für unwichtig. Auch die Dienstboten
               betrügen uns, und man denkt nicht daran, sie abzuschaffen. Es fragt sich immer nur,
               um wie viel die Agenten die Autoren übervorteilen. Und das ist im Ganzen nicht gar so
               erheblich.\pend
           
\pstart
           Heute schrieb mir Bahr\pwindex{Bahr, Hermann 19.\,7.\,1863 Linz – 15.\,1.\,1934 München@\textsc{Bahr, Hermann} (19.\,7.\,1863 Linz – 15.\,1.\,1934 München), \emph{Schriftsteller, Kritiker}|pw}, dass er Samstag{ }Abend auf zwei Tage her\oindex{Berlin@\textbf{Berlin}, \emph{Hauptstadt}|pwv}kommt. Das ist mir weitaus angenehmer. Sonst bin ich ziemlich allein;
               kann mir zu Harden\pwindex{Harden, Maximilian 20.\,10.\,1861 Berlin – 30.\,10.\,1927 Montana@\textsc{Harden, Maximilian} (20.\,10.\,1861 Berlin – 30.\,10.\,1927 Montana), \emph{Schriftsteller, Publizist}|pw} kein Herz faßen seit jenem
                  \label{K_L03416-8v}\edtext{Artikel\pwindex{Harden, Maximilian 20.\,10.\,1861 Berlin – 30.\,10.\,1927 Montana@\textsc{Harden, Maximilian} (20.\,10.\,1861 Berlin – 30.\,10.\,1927 Montana), \emph{Schriftsteller, Publizist}!Theater@\strich\emph{Theater}|pwv}}{\lemma{\textnormal{\emph{Artikel}}}\Cendnote{\textnormal{Siehe XXXX Auszeichnungsfehler: Dokument L03415 nicht gefunden. }}}\label{K_L03416-8} und hab’
               ihn seither auch nicht gesehen noch gesucht. Heute –
               es ist überhaupt ein lebhafter Tag – telefonirte mir Ihre Schwägerin\pwindex{Steinrück, Elisabeth 19.\,11.\,1885 – 7.\,4.\,1920 Partenkirchen@\textsc{Steinrück, Elisabeth} (19.\,11.\,1885 – 7.\,4.\,1920 Partenkirchen)|pwv} wegen einer \label{K_L03416-9v}\edtext{Schiffskarte}{\lemma{\textnormal{\emph{Schiffskarte}}}\Cendnote{\textnormal{Elisabeth Gussmann\pwindex{Steinrück, Elisabeth 19.\,11.\,1885 – 7.\,4.\,1920 Partenkirchen@\textsc{Steinrück, Elisabeth} (19.\,11.\,1885 – 7.\,4.\,1920 Partenkirchen)|pwkv} hatte
                  momentan kein Engagement und war gesundheitlich angeschlagen. Letzteres hoffte sie
                  durch eine Seereise zu kurieren. Aus dem Reiseplan wurde nichts, eventuell zog sie
                  für ein paar Tage in der Umgebung von Berlin\oindex{Berlin@\textbf{Berlin}, \emph{Hauptstadt}|pwk}
                  auf’s Land. Den Sommer verbrachte sie mit ihrem nachmaligen Ehemann Albert Steinrück\pwindex{Steinrück, Albert 20.\,5.\,1872 Wetterburg – 11.\,2.\,1929 Berlin@\textsc{Steinrück, Albert} (20.\,5.\,1872 Wetterburg – 11.\,2.\,1929 Berlin), \emph{Schauspieler}|pwk} in Gilleleje\oindex{Gilleleje@\textbf{Gilleleje}|pwk}, vgl. XXXX Auszeichnungsfehler: Dokument L01625 nicht gefunden.}}}\label{K_L03416-9}. Ich bat sie, dieser Tage zu uns\pwindex{Salten, Ottilie 7.\,3.\,1868 Prag – 22.\,6.\,1942 Zürich@\textsc{Salten, Ottilie} (7.\,3.\,1868 Prag – 22.\,6.\,1942 Zürich), \emph{Schauspielerin}|pwv} zu kommen, damit wir alles genauer
               besprechen.\pend
           
\pstart
           Hier lege ich Ihnen das zweite \label{K_L03416-10v}\edtext{Russenfeuilleton\pwindex{Salten, Felix 6.\,9.\,1869 Budapest – 8.\,10.\,1945 Zürich@\textsc{Salten, Felix} (6.\,9.\,1869 Budapest – 8.\,10.\,1945 Zürich), \emph{Schriftsteller, Journalist, Chefredakteur}!Russisches Theater. II@\strich\emph{Russisches Theater. II}|pwv}}{\lemma{\textnormal{\emph{Russenfeuilleton}}}\Cendnote{\textnormal{Felix Salten\pwindex{Salten, Felix 6.\,9.\,1869 Budapest – 8.\,10.\,1945 Zürich@\textsc{Salten, Felix} (6.\,9.\,1869 Budapest – 8.\,10.\,1945 Zürich), \emph{Schriftsteller, Journalist, Chefredakteur}|pwk}: \emph{Russisches Theater. II}\pwindex{Salten, Felix 6.\,9.\,1869 Budapest – 8.\,10.\,1945 Zürich@\textsc{Salten, Felix} (6.\,9.\,1869 Budapest – 8.\,10.\,1945 Zürich), \emph{Schriftsteller, Journalist, Chefredakteur}!Russisches Theater. II@\strich\emph{Russisches Theater. II}|pwk}. In: \emph{B. Z. am Mittag}\pwindex{B.Z. am Mittag@\emph{B.Z. am Mittag}|pwk}, Jg. 30, Nr. 70, 23. 3. 1906, S. 2–3.}}}\label{K_L03416-10} bei, und das
               über \label{K_L03416-11v}\edtext{Kater Lampe\pwindex{\textcolor{red}{\textsuperscript{XXXX indx1}}!Kater Lampe@\strich\emph{Kater Lampe}|pw}}{\lemma{\textnormal{\emph{Kater Lampe}}}\Cendnote{\textnormal{Felix Salten\pwindex{Salten, Felix 6.\,9.\,1869 Budapest – 8.\,10.\,1945 Zürich@\textsc{Salten, Felix} (6.\,9.\,1869 Budapest – 8.\,10.\,1945 Zürich), \emph{Schriftsteller, Journalist, Chefredakteur}|pwk}: \emph{»Kater Lampe«}\pwindex{Salten, Felix 6.\,9.\,1869 Budapest – 8.\,10.\,1945 Zürich@\textsc{Salten, Felix} (6.\,9.\,1869 Budapest – 8.\,10.\,1945 Zürich), \emph{Schriftsteller, Journalist, Chefredakteur}!Kater Lampe«@\strich\emph{»Kater Lampe«}|pwk}. In: \emph{B. Z. am Mittag}\pwindex{B.Z. am Mittag@\emph{B.Z. am Mittag}|pwk}, Jg. 30, Nr. 72, 26. 3. 1906, S. 2.}}}\label{K_L03416-11}. Herzliche Grüße von uns\pwindex{Salten, Ottilie 7.\,3.\,1868 Prag – 22.\,6.\,1942 Zürich@\textsc{Salten, Ottilie} (7.\,3.\,1868 Prag – 22.\,6.\,1942 Zürich), \emph{Schauspielerin}|pwv} zu Ihnen.{\\}Ihr
                  \spacefill\mbox{Salten}\pend
           \selectlanguage{ngerman}\endnumbering\briefempfaengerindex{Schnitzler, Arthur@\textsc{Schnitzler, Arthur}!zzzSalten, Felix@\emph{von Felix Salten}!1906-03-281@{28. 3. 1906}|)be}\mylabel{L03416h}  \newcommand{\dateiname}{L03416}\newcommand{\titel}{Felix Salten an Arthur Schnitzler, 28. 3. 1906}\newcommand{\editorInnen}{Martin Anton Müller und Laura Untner}%% latex-leseansicht-abspann.tex
%% Abspann für die Leseansicht.
%% Der Schalter \ifkorrekturansicht ist bereits durch den Vorspann gesetzt.

%% latex-abspann.tex
%% Gemeinsamer Abspann für Korrekturansicht und Leseansicht.
%% Setzt den Schalter \ifkorrekturansicht voraus (gesetzt in den
%% einbindenden Dateien latex-korrekturansicht-abspann.tex bzw.
%% latex-leseansicht-abspann.tex).
%% ---------------------------------------------------------------

\normalsize

% Das esempio-Environment wird nur in der Leseansicht benötigt
\ifkorrekturansicht\else
\newenvironment{esempio}[3]%
{
    \vspace{1.5ex}
    \rlap{\underline{#1}}
    \par
    \setlength{\parindent}{0cm}
    \nopagebreak
    \leftskip=#2cm
    \rightskip=#3cm
}
{
    \par
}
\fi

\doendnotes{C}
\bigskip
\vfill

\clearpage

\footnotesize

\ifkorrekturansicht
  \lohead{\textsc{register}}
\fi

% theindex-Environment neu definieren ohne reledmac
\makeatletter
\renewenvironment{theindex}{%
  \ifkorrekturansicht
    \section*{\indexname}%
  \else
    \subsubsection*{Index der erwähnten Entitäten}%
  \fi
  \setlength{\parindent}{0pt}%
  \setlength{\parskip}{0pt plus 0.3pt}%
  \let\item\@idxitem
}{%
  \ifkorrekturansicht\clearpage\fi
}
\makeatother

\IfFileExists{\jobname-pw.ind}{\input{\jobname-pw.ind}}{}

% Quellenangabe nur in der Leseansicht
\ifkorrekturansicht\else
% Fallback-Definitionen, falls die .tex-Datei \titel etc. nicht gesetzt hat
\providecommand{\titel}{}
\providecommand{\editorInnen}{}
\providecommand{\dateiname}{\jobname}

\vspace{3cm}

\vfill

\footnotesize
\textsc{Quelle}: \titel. Herausgegeben von {\editorInnen}. In: \emph{Arthur Schnitzler: Briefwechsel mit Autorinnen und Autoren}.
 Digitale Edition, https://schnitzler-briefe.acdh.oeaw.ac.at/{\dateiname}.html (Stand \today)
\fi

\end{document}


