%% latex-korrekturansicht-vorspann.tex
%% Vorspann für die Korrekturansicht.
%% Lädt die gemeinsame Datei latex-vorspann.tex mit gesetztem Schalter.

\newif\ifkorrekturansicht
\korrekturansichttrue

\input{../tex-inputs/latex-vorspann}


\section[ Felix Salten an Arthur Schnitzler, 28. 3. 1906]{L03416 Felix Salten an Arthur Schnitzler, 28. 3. 1906}
\nopagebreak\mylabel{L03416v}
\rehead{ }\normalsize\beginnumbering\briefempfaengerindex{Schnitzler, Arthur@\textsc{Schnitzler, Arthur}!zzzSalten, Felix@\emph{von Felix Salten}!1906-03-281@{28. 3. 1906}|(be}
\toendnotes[C]{\smallbreak\pagebreak[2]}\Standort{CUL, Schnitzler, B 89, B 1.}
\physDesc{Brief, 1 Blatt, 2 Seiten, 3211 Zeichen
\newline{}Handschrift: schwarze Tinte, lateinische Kurrent
\newline{}Ordnung: mit Bleistift von unbekannter Hand nummeriert: »207« }
\buchAbdrucke{\weitereDrucke{Hermann Bahr, Arthur Schnitzler: \emph{Briefwechsel, Aufzeichnungen, Dokumente (1891–1931)}. Göttingen: \emph{Wallstein} 2018, S. 376–377.} }\toendnotes[C]{\smallbreak}
\pstart
           {\pb}\textcolor{gray}{\textbf{\emph{B. Z. am Mittag}}}\orgindex{B.Z. am Mittag@B.Z. am Mittag|pw}\hfill \textcolor{gray}{\textbf{\emph{BERLIN SW\oindex{Berlin@\textbf{Berlin}, \emph{P.PPLC}|pw}},}}{ }28. III. 06\pend
           
\pstart
           \textcolor{gray}{\textbf{\emph{Chefredaktion}}}\hfill \textcolor{gray}{\textbf{\emph{Kochstr. 23–25}}}\oindex{Kochstrasse@\textbf{Kochstraße}, \emph{Straße (K.STR)}|pw}\pend
           \vspace{0.5em}
\pstart
           Lieber, dass wir eine \label{K_L03416-1v}\edtext{Radtour machen}{\lemma{\textnormal{\emph{Radtour machen}}}\Cendnote{\textnormal{Diese fand nicht statt,
                     siehe Felix Salten an Arthur Schnitzler, 1. 5. 1906. Vgl. Felix Salten an Arthur Schnitzler, 19. 6. 1906. }}}\label{K_L03416-1} könnten, ist
               mir heute wie ein absolutes Muß! Es wäre so schön 6–8 Tage irgendwo durch die Welt zu
               gleiten, wo sie schön ist, und wo man wieder einmal so viel Behagen empfinden könnte,
               wie »einst im Mai«{[}.{]} Denken Sie etwas Gutes aus, und ziehen Sie
               dabei in Betracht, ob wir nicht eine Gegend wählen wollen, die wir noch nicht kennen.
               Deutsches Gebirge, Thüringen\oindex{Thueringen@\textbf{Thüringen}, \emph{A.ADM1}|pw}, Rhein\oindex{Rheinland@\textbf{Rheinland}, \emph{Teil eines Landes (A.LNDX)}|pw}, u. s. w. Ich bin aber auch mit Tirol\oindex{Tirol@\textbf{Tirol}, \emph{A.ADM1}|pw}\oindex{Suedtirol@\textbf{Südtirol}, \emph{A.ADM2}|pw} oder Schweiz\oindex{Schweiz@\textbf{Schweiz}, \emph{A.PCLI}|pw} (Lugano\oindex{Lugano@\textbf{Lugano}, \emph{P.PPLA2}|pw} oder Genfer See\oindex{Genfer See@\textbf{Genfer See}, \emph{H.LK}|pw}) einverstanden. Ihr Brief kam heute aber auch \label{K_L03416-2v}\edtext{\begin{otherlanguage}{italian}a tempo\end{otherlanguage}}{\lemma{\textnormal{\emph{a tempo}}}\Cendnote{\textnormal{italienisch: zur rechten Zeit}}}\label{K_L03416-2}: es
               ist \substVorne{}\textsuperscript{\textcolor{gray}{jetzt}}\substDazwischen{}nach\substHinten{} langem Winter wieder die erste Frühlingswärme, die erste Sonne wieder da,
               und alle Reisepläne, alles Reiseverlangen – »Wanderlust« – regt sich. An solchen
               Tagen hat auch Berlin\oindex{Berlin@\textbf{Berlin}, \emph{P.PPLC}|pw} seine Schönheit. An solchen
               Tagen würde übrigens auch Magdeburg\oindex{Magdeburg@\textbf{Magdeburg}, \emph{P.PPLA}|pw} oder Genthinen\oindex{Genthin@\textbf{Genthin}, \emph{P.PPL}|pw} nicht ohne Reiz sein. Ich überlege mir
               heute zum 20\textsuperscript{ten} Mal, wie man es macht, sich ein ganz ein
               kleines Automobil zu kaufen. Geht aber leider im Moment nicht. Wenn ich die große
               Zeitung gegründet habe, Neue freie Presse\orgindex{Neue Freie Presse@Neue Freie Presse|pw} in Berlin\oindex{Berlin@\textbf{Berlin}, \emph{P.PPLC}|pw}, eine Wochenschrift im Zukunft\orgindex{Zukunft@Die Zukunft|pw}-Stil und dann vier Blätter regiere, statt zwei\orgindex{Berliner Morgenpost@Berliner Morgenpost|pwv}\orgindex{B.Z. am Mittag@B.Z. am Mittag|pwv} (was ich armselig
                  finde){[},{]} dann werde ich gewiss auch das langerflehte Auto
               haben. Inzwischen freu ich mich, wenn nur eine Radtour zustande kommt, und die
               übrigen Dinge, die ich für den Sommer vorhabe (Holland\oindex{Niederlande@\textbf{Niederlande}, \emph{A.PCLI}|pw}, zu Wasser nach Kiel\oindex{Kiel@\textbf{Kiel}, \emph{P.PPLA}|pw}){[}.{]} Die Radtour könnte auch durch einige deutsch\oindex{Deutschland@\textbf{Deutschland}, \emph{A.PCLI}|pwv}e Städte gemacht werden,
               – Rothenburg ob. d. Tauber\oindex{Rothenburg ob der Tauber@\textbf{Rothenburg ob der Tauber}, \emph{P.PPL}|pw} – Bayreuth\oindex{Bayreuth@\textbf{Bayreuth}, \emph{P.PPLA2}|pw}, wozu man freilich jetzt schon die Sitze bestellen
               müsste. Das \label{K_L03416-3v}\edtext{dänische Seebad\oindex{Marienlyst@\textbf{Marienlyst}, \emph{S.EST}|pw}}{\lemma{\textnormal{\emph{dänische Seebad}}}\Cendnote{\textnormal{Schnitzler war zwischen 28. 6. 1906 und 9. 8. 1906 in Marienlyst\oindex{Marienlyst@\textbf{Marienlyst}, \emph{S.EST}|pwk}. Felix\pwindex{Salten, Felix 06.09.1869 – 08.10.1945@\textsc{Salten, Felix} (06.09.1869 – 08.10.1945), \emph{Schriftsteller/Schriftstellerin, Journalist/Journalistin, Chefredakteur/Chefredakteurin}|pwk} und Ottilie Salten\pwindex{Salten, Ottilie 07.03.1868 – 22.06.1942@\textsc{Salten, Ottilie} (07.03.1868 – 22.06.1942), \emph{Schauspieler/Schauspielerin}|pwk} besuchten
                  ihn dort am 2. 8. 1906.}}}\label{K_L03416-3}, das Sie vorhaben, verdrießt mich – wenn ich
               aufrichtig sein darf – immer. Weil ich {\dotstwo} aus
               wirthschaftlichen Gründen {\dotstwo} nicht hinkann, wenn ich schon
               einmal an der Ostsee\oindex{Ostsee@\textbf{Ostsee}, \emph{Meer (N.MER)}|pw} sitze, und weil ich mir
               denke, wenn uns ein mehrwöchiges Beisammensein schon beschieden sein könnte, dann
               ließe sich vielleicht doch auf Dänemark\oindex{Daenemark@\textbf{Dänemark}, \emph{A.PCLI}|pw}
               verzichten. Der Unterschied ist nicht so groß, und Wälder gibt's auch am diesseitigen
               Strand der Ostsee\oindex{Ostsee@\textbf{Ostsee}, \emph{Meer (N.MER)}|pw}.\pend
           
\pstart
           Augenblicklich ist Wien\oindex{Wien@\textbf{Wien}, \emph{A.ADM2}|pw} durch M\textsuperscript{r}{ }\label{K_L03416-4v}\edtext{Triebeitsch\pwindex{Trebitsch, Siegfried 22.12.1868 – 03.06.1956@\textsc{Trebitsch, Siegfried} (22.12.1868 – 03.06.1956), \emph{Schriftsteller/Schriftstellerin, Übersetzer/Übersetzerin}|pw}}{\lemma{\textnormal{\emph{Triebeitsch}}}\Cendnote{\textnormal{Hier findet das Naserümpfen über Trebitsch\pwindex{Trebitsch, Siegfried 22.12.1868 – 03.06.1956@\textsc{Trebitsch, Siegfried} (22.12.1868 – 03.06.1956), \emph{Schriftsteller/Schriftstellerin, Übersetzer/Übersetzerin}|pwk} eine Form, in der die Herabsetzung
                  durch die Imitation einer englischen Aussprache seines Namens erfolgt.}}}\label{K_L03416-4}
               vertreten, der in seinem \label{K_L03416-5v}\edtext{Premiere\pwindex{Caesar und Cleopatra. Eine historische Komoedie@\emph{Cäsar und Cleopatra. Eine historische Komödie}|pwv}nfieber}{\lemma{\textnormal{\emph{Premierenfieber}}}\Cendnote{\textnormal{Am 31. 3. 1906 fand am \emph{Neuen Theater}\orgindex{Neues Theater@Neues Theater|pwk}
                  die deutschsprachige Uraufführung von \emph{Caesar und
                     Cleopatra}\pwindex{Caesar und Cleopatra. Eine historische Komoedie@\emph{Cäsar und Cleopatra. Eine historische Komödie}|pwk} von George Bernard Shaw\pwindex{Shaw, George Bernard 26.07.1856 – 02.11.1950@\textsc{Shaw, George Bernard} (26.07.1856 – 02.11.1950), \emph{Schriftsteller/Schriftstellerin}|pwk} in
                  der Übersetzung von Siegfried Trebitsch\pwindex{Trebitsch, Siegfried 22.12.1868 – 03.06.1956@\textsc{Trebitsch, Siegfried} (22.12.1868 – 03.06.1956), \emph{Schriftsteller/Schriftstellerin, Übersetzer/Übersetzerin}|pwk}
                  statt.}}}\label{K_L03416-5} wegen Shaw\pwindex{Shaw, George Bernard 26.07.1856 – 02.11.1950@\textsc{Shaw, George Bernard} (26.07.1856 – 02.11.1950), \emph{Schriftsteller/Schriftstellerin}|pw} das Maß des
               Lächerlichen erreicht. Seine erste Frage, als er hier\oindex{Berlin@\textbf{Berlin}, \emph{P.PPLC}|pwv} eintraf, war (natürlich per Telefon) was ich von seinem
                  \label{K_L03416-6v}\edtext{Vorschlag\pwindex{Buehnenvertrieb@\emph{Bühnenvertrieb}|pwv}}{\lemma{\textnormal{\emph{Vorschlag}}}\Cendnote{\textnormal{Siegfried Trebitsch\pwindex{Trebitsch, Siegfried 22.12.1868 – 03.06.1956@\textsc{Trebitsch, Siegfried} (22.12.1868 – 03.06.1956), \emph{Schriftsteller/Schriftstellerin, Übersetzer/Übersetzerin}|pwk}: \emph{Bühnenvertrieb}\pwindex{Buehnenvertrieb@\emph{Bühnenvertrieb}|pwk}. In: \emph{Die
                        Schaubühne}\pwindex{Schaubuehne@\emph{Die Schaubühne}|pwk}, Jg. 2, Nr. 12, 22. 3. 1906,
                     S. 348–350. Darin forderte Trebitsch\pwindex{Trebitsch, Siegfried 22.12.1868 – 03.06.1956@\textsc{Trebitsch, Siegfried} (22.12.1868 – 03.06.1956), \emph{Schriftsteller/Schriftstellerin, Übersetzer/Übersetzerin}|pwk} die Einrichtung einer Bühnengenossenschaft zur Vertretung von
                  Autoren- und Autorinnenrechten. Das motivierte den Herausgeber der Zeitschrift\pwindex{Schaubuehne@\emph{Die Schaubühne}|pwkv}, Siegfried Jacobsohn\pwindex{Jacobsohn, Siegfried 28.01.1881 – 03.12.1926@\textsc{Jacobsohn, Siegfried} (28.01.1881 – 03.12.1926), \emph{Journalist/Journalistin, Kritiker/Kritikerin, Publizist/Publizistin}|pwk}, zu einer mehrteiligen Debatte, die
                  sich über Monate erstreckte. In der zweiten Fortsetzung\pwindex{Bund der Buehnendichter. II@\emph{Bund der Bühnendichter. II}|pwkv} findet sich ein Beitrag Schnitzlers. Siehe A. S.: \emph{»Das Zeitlose ist von kürzester Dauer«}, Bund der Bühnendichter, 12. 4. 1906.}}}\label{K_L03416-6} in der »Schaubühne\pwindex{Schaubuehne@\emph{Die Schaubühne}|pw}« halte. Ich sagte, dass ich dagegen
               sei. Er ließ seinen erstaunten Klagelaut vernehmen, und meinte dann, {\pb}\uline{Sie} hätten ihm\pwindex{Trebitsch, Siegfried 22.12.1868 – 03.06.1956@\textsc{Trebitsch, Siegfried} (22.12.1868 – 03.06.1956), \emph{Schriftsteller/Schriftstellerin, Übersetzer/Übersetzerin}|pwv} einen »begeisterten« Brief geschrieben. Ich bin wirklich
               nicht sehr für diesen Vorschlag, der nur aus der \label{K_L03416-7v}\edtext{Seidenbranche}{\lemma{\textnormal{\emph{Seidenbranche}}}\Cendnote{\textnormal{Anspielung auf Trebitschs\pwindex{Trebitsch, Siegfried 22.12.1868 – 03.06.1956@\textsc{Trebitsch, Siegfried} (22.12.1868 – 03.06.1956), \emph{Schriftsteller/Schriftstellerin, Übersetzer/Übersetzerin}|pwk}
                  großindustriellen Hintergrund}}}\label{K_L03416-7} kommt; glaube an Ihre »Begeisterung«
               natürlich nicht, und halte die ganze Sache für unwichtig. Auch die Dienstboten
               betrügen uns, und man denkt nicht daran, sie abzuschaffen. Es fragt sich immer nur,
               um wie viel die Agenten die Autoren übervorteilen. Und das ist im Ganzen nicht gar so
               erheblich.\pend
           
\pstart
           Heute schrieb mir Bahr\pwindex{Bahr, Hermann 19.07.1863 – 15.01.1934@\textsc{Bahr, Hermann} (19.07.1863 – 15.01.1934), \emph{Schriftsteller/Schriftstellerin, Kritiker/Kritikerin}|pw}, dass er Samstag{ }Abend auf zwei Tage her\oindex{Berlin@\textbf{Berlin}, \emph{P.PPLC}|pwv}kommt. Das ist mir weitaus angenehmer. Sonst bin ich ziemlich allein;
               kann mir zu Harden\pwindex{Harden, Maximilian 20.10.1861 – 30.10.1927@\textsc{Harden, Maximilian} (20.10.1861 – 30.10.1927), \emph{Schriftsteller/Schriftstellerin, Publizist/Publizistin}|pw} kein Herz faßen seit jenem
                  \label{K_L03416-8v}\edtext{Artikel\pwindex{Theater@\emph{Theater}|pwv}}{\lemma{\textnormal{\emph{Artikel}}}\Cendnote{\textnormal{Siehe Felix Salten an Arthur Schnitzler, 9. 3. 1906. }}}\label{K_L03416-8} und hab’
               ihn seither auch nicht gesehen noch gesucht. Heute –
               es ist überhaupt ein lebhafter Tag – telefonirte mir Ihre Schwägerin\pwindex{Steinrueck, Elisabeth 19.11.1885 – 07.04.1920@\textsc{Steinrück, Elisabeth} (19.11.1885 – 07.04.1920)|pwv} wegen einer \label{K_L03416-9v}\edtext{Schiffskarte}{\lemma{\textnormal{\emph{Schiffskarte}}}\Cendnote{\textnormal{Elisabeth Gussmann\pwindex{Steinrueck, Elisabeth 19.11.1885 – 07.04.1920@\textsc{Steinrück, Elisabeth} (19.11.1885 – 07.04.1920)|pwkv} hatte
                  momentan kein Engagement und war gesundheitlich angeschlagen. Letzteres hoffte sie
                  durch eine Seereise zu kurieren. Aus dem Reiseplan wurde nichts, eventuell zog sie
                  für ein paar Tage in der Umgebung von Berlin\oindex{Berlin@\textbf{Berlin}, \emph{P.PPLC}|pwk}
                  auf’s Land. Den Sommer verbrachte sie mit ihrem nachmaligen Ehemann Albert Steinrück\pwindex{Steinrueck, Albert 20.05.1872 – 11.02.1929@\textsc{Steinrück, Albert} (20.05.1872 – 11.02.1929), \emph{Schauspieler/Schauspielerin}|pwk} in Gilleleje\oindex{Gilleleje@\textbf{Gilleleje}, \emph{P.PPL}|pwk}, vgl. Arthur Schnitzler an Hugo von Hofmannsthal, 8. 9. 1906.}}}\label{K_L03416-9}. Ich bat sie, dieser Tage zu uns\pwindex{Salten, Ottilie 07.03.1868 – 22.06.1942@\textsc{Salten, Ottilie} (07.03.1868 – 22.06.1942), \emph{Schauspieler/Schauspielerin}|pwv} zu kommen, damit wir alles genauer
               besprechen.\pend
           
\pstart
           Hier lege ich Ihnen das zweite \label{K_L03416-10v}\edtext{Russenfeuilleton\pwindex{Russisches Theater. II@\emph{Russisches Theater. II}|pwv}}{\lemma{\textnormal{\emph{Russenfeuilleton}}}\Cendnote{\textnormal{Felix Salten\pwindex{Salten, Felix 06.09.1869 – 08.10.1945@\textsc{Salten, Felix} (06.09.1869 – 08.10.1945), \emph{Schriftsteller/Schriftstellerin, Journalist/Journalistin, Chefredakteur/Chefredakteurin}|pwk}: \emph{Russisches Theater. II}\pwindex{Russisches Theater. II@\emph{Russisches Theater. II}|pwk}. In: \emph{B. Z. am Mittag}\pwindex{B.Z. am Mittag@\emph{B.Z. am Mittag}|pwk}, Jg. 30, Nr. 70, 23. 3. 1906, S. 2–3.}}}\label{K_L03416-10} bei, und das
               über \label{K_L03416-11v}\edtext{Kater Lampe\pwindex{Kater Lampe@\emph{Kater Lampe}|pw}}{\lemma{\textnormal{\emph{Kater Lampe}}}\Cendnote{\textnormal{Felix Salten\pwindex{Salten, Felix 06.09.1869 – 08.10.1945@\textsc{Salten, Felix} (06.09.1869 – 08.10.1945), \emph{Schriftsteller/Schriftstellerin, Journalist/Journalistin, Chefredakteur/Chefredakteurin}|pwk}: \emph{»Kater Lampe«}\pwindex{Kater Lampe«@\emph{»Kater Lampe«}|pwk}. In: \emph{B. Z. am Mittag}\pwindex{B.Z. am Mittag@\emph{B.Z. am Mittag}|pwk}, Jg. 30, Nr. 72, 26. 3. 1906, S. 2.}}}\label{K_L03416-11}. Herzliche Grüße von uns\pwindex{Salten, Ottilie 07.03.1868 – 22.06.1942@\textsc{Salten, Ottilie} (07.03.1868 – 22.06.1942), \emph{Schauspieler/Schauspielerin}|pwv} zu Ihnen.{\\}Ihr
                  \spacefill\mbox{Salten}\pend
           \selectlanguage{ngerman}\endnumbering\briefempfaengerindex{Schnitzler, Arthur@\textsc{Schnitzler, Arthur}!zzzSalten, Felix@\emph{von Felix Salten}!1906-03-281@{28. 3. 1906}|)be}\mylabel{L03416h}  \normalsize

\doendnotes{C}
\bigskip
\vfill

\clearpage

\footnotesize

\lohead{\textsc{register}}

% Definiere theindex-Environment komplett neu ohne reledmac
\makeatletter
\renewenvironment{theindex}{%
  \section*{\indexname}%
  \setlength{\parindent}{0pt}%
  \setlength{\parskip}{0pt plus 0.3pt}%
  \let\item\@idxitem
}{%
  \clearpage
}
\makeatother

\IfFileExists{\jobname-pw.ind}{\input{\jobname-pw.ind}}{}

\end{document}

      