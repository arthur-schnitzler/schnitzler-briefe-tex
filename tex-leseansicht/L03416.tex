%% latex-leseansicht-vorspann.tex
%% Vorspann für die Leseansicht.
%% Lädt die gemeinsame Datei latex-vorspann.tex mit nicht gesetztem Schalter.

\newif\ifkorrekturansicht
\korrekturansichtfalse

\input{../tex-inputs/latex-vorspann}

\begin{center}
            \textcolor{red}{ENTWURF, NICHT FERTIG KORRIGIERT}
                      \end{center}
            
         
         \renewcommand{\erwaehntePersonen}{Personen: Hermann Bahr, Maximilian Harden, Siegfried Jacobsohn, Emil Rosenow, George Bernard Shaw, Elisabeth Steinrück, Siegfried Trebitsch}
         \renewcommand{\erwaehnteInstitutionen}{Institutionen: B.Z. am Mittag, Die Zukunft, Neue Freie Presse, Neues Theater}
         \renewcommand{\erwaehnteOrte}{Orte: Bayreuth, Berlin, Dänemark, Genfer See, Genthin, Kiel, Kochstraße, Lugano, Magdeburg, Marienlyst, Niederlande, Ostsee, Rheinland, Rothenburg ob der Tauber, Schweiz, Südtirol, Thüringen, Tirol, Wien}
         \renewcommand{\erwaehnteWerke}{Werke: B.Z. am Mittag, Bühnenvertrieb, Cäsar und Cleopatra. Eine historische Komödie, Die Schaubühne, Die Zukunft, Kater Lampe, Russisches Theater. II, Theater, »Kater Lampe«}
               \section[Felix Salten an Arthur Schnitzler, 28. 3. 1906]{ Felix Salten an Arthur Schnitzler, 28. 3. 1906}\nopagebreak\mylabel{v}\rehead{ }\begin{ledgroupsized}[t]{13cm}\normalsize\beginnumbering \toendnotes[C]{\smallbreak\pagebreak[2]} \Standort{CUL, Schnitzler, B 89, B 1.}
\physDesc{Brief, 1 Blatt, 2 Seiten
\newline{}Handschrift: schwarze Tinte, lateinische Kurrent\newline{}Ordnung: mit Bleistift von unbekannter Hand nummeriert:
                                    »207« }\toendnotes[C]{\smallbreak}\pstart
           \noindent{}{\pb}\textcolor{gray}{\textbf{B. Z. am Mittag}}\orgindex{B.Z. am Mittag@B.Z. am Mittag|pw}\hfill \textcolor{gray}{\textbf{BERLIN SW\oindex{Berlin@\textbf{Berlin}|pw},}}{ }28. III. 06\pend
           \pstart
           \textcolor{gray}{\textbf{Chefredaktion}}\hfill \textcolor{gray}{\textbf{Kochstr. 23–25}}\oindex{Kochstrasse@\textbf{Kochstraße}|pw}\pend
           \pstart
           Lieber, dass wir eine Radtour machen könnten, ist mir heute wie ein
               absoluter Muß! Es wäre so schön 6–8 Tage irgendwo – durch die Welt zu gleiten, wo sie
               schön ist, und wo man wieder einmal so viel Behagen empfinden könnte, wie »einst im
                  Mai«{[}.{]} Denken Sie etwas Gutes aus, und ziehen Sie dabei in
               Betracht, ob wir nicht eine Gegend wählen wollen, die wir noch nicht kennen.
               Deutsches Gebirge, Thüringen\oindex{Thueringen@\textbf{Thüringen}|pw}, Rhein\oindex{Rheinland@\textbf{Rheinland}|pw}, u. s. w. Ich bin aber auch mit Tirol\oindex{Tirol@\textbf{Tirol}|pw}\oindex{Suedtirol@\textbf{Südtirol}|pw} oder Schweiz\oindex{Schweiz@\textbf{Schweiz}|pw} (Lugano\oindex{Lugano@\textbf{Lugano}|pw} oder Genfer See\oindex{Genfer See@\textbf{Genfer See}|pw}) einverstanden. Ihr Brief kam heute aber auch a
               tempo: es ist \substVorne{}\textsuperscript{\textcolor{gray}{seit}}\substDazwischen{}nach\substHinten{} langem Winter wieder die erste Frühlingswärme, die erste Sonne wieder da,
               und alle Reisepläne, alles Reiseverlangen – »Wanderlust« – regt sich. An solchen
               Tagen hat auch Berlin\oindex{Berlin@\textbf{Berlin}|pw} seine Schöhheit. An solchen
               Tagen würde übrigens auch Magdeburg\oindex{Magdeburg@\textbf{Magdeburg}|pw} oder Genthinen\oindex{Genthin@\textbf{Genthin}|pw} nicht ohne Reiz sein. Ich überlege mir
               heute zum 20\textsuperscript{ten} Mal, wie man es macht, sich ein ganz ein
               kleines Automobil zu kaufen. Geht aber leider im Moment nicht. Wenn ich die große
               Zeitung gegründet habe, Neue freie Presse\orgindex{Neue Freie Presse@Neue Freie Presse|pw} in Berlin\oindex{Berlin@\textbf{Berlin}|pw}, eine Wochenschrift im Zukunft\orgindex{Zukunft@Die Zukunft|pw}-Stil und dann vier Blätter regiere, statt zwei (was
               ich armselig finde){[},{]} dann werde ich gewiss auch das langerflehte
               Auto haben. Inzwischen freu ich mich, wenn nur eine Radtour zustande kommt, und die
               übrigen Dinge, die ich für den Sommer vorhabe (Holland\oindex{Niederlande@\textbf{Niederlande}|pw}, zu Wasser nach Kiel\oindex{Kiel@\textbf{Kiel}|pw}){[},{]} die Radtour könnte auch durch einige deutsche
               Städte gemacht werden, – Rothenburg ob. d. Tauber\oindex{Rothenburg ob der Tauber@\textbf{Rothenburg ob der Tauber}|pw}
               – Bayreuth\oindex{Bayreuth@\textbf{Bayreuth}|pw}, wozu man freilich jetzt schon die
               Sitze bestellen müsste. Das dänische Seebad\oindex{Marienlyst@\textbf{Marienlyst}|pw}, das
               Sie vorhaben, verdrießt mich – wenn ich aufrichtig sein darf – immer. Weil ich {\dotstwo} aus wirthschaftlichen Gründen {\dotstwo} nicht hinkann, wenn ich schon einmal an der Ostsee\oindex{Ostsee@\textbf{Ostsee}|pw} sitze, und weil ich mir denke, wenn uns ein mehrwöchiges
               Beisammensein schon beschieden sein könnte, dann ließe sich vielleicht doch auf Dänemark\oindex{Daenemark@\textbf{Dänemark}|pw} verzichten. Der Unterschied ist nicht
               so groß, und Wälder gibt's auch am diesseitigen Strand der Ostsee\oindex{Ostsee@\textbf{Ostsee}|pw}. \pend
           \pstart
           Augenblicklich ist Wien\oindex{Wien@\textbf{Wien}|pw} durch M\textsuperscript{r}{ }Triebeitsch\pwindex{Trebitsch, Siegfried 22.12.1868 – 03.06.1956@\textsc{Trebitsch, Siegfried} (22.12.1868 – 03.06.1956), \emph{Schriftsteller, Übersetzer}|pw} vertreten, der in seinem \label{K_L03416-11v}\edtext{Premiere\pwindex{Shaw, George Bernard 26.07.1856 – 02.11.1950@\textsc{Shaw, George Bernard} (26.07.1856 – 02.11.1950), \emph{Schriftsteller}!Caesar und Cleopatra. Eine historische Komoedie1906-03-31@\strich\emph{Cäsar und Cleopatra. Eine historische Komödie} {[}1906-03-31{]}|pwv}nfieber}{\lemma{\textnormal{\emph{Premierenfieber}}}\Cendnote{\textnormal{Am 31. 3. 1906 fand am \emph{Neuen Theater}\orgindex{Neues Theater@Neues Theater|pwk}
                  die deutschsprachige Uraufführung von \emph{Caesar und
                     Cleopatra}\pwindex{Shaw, George Bernard 26.07.1856 – 02.11.1950@\textsc{Shaw, George Bernard} (26.07.1856 – 02.11.1950), \emph{Schriftsteller}!Caesar und Cleopatra. Eine historische Komoedie1906-03-31@\strich\emph{Cäsar und Cleopatra. Eine historische Komödie} {[}1906-03-31{]}|pwk} von George Bernard Shaw\pwindex{Shaw, George Bernard 26.07.1856 – 02.11.1950@\textsc{Shaw, George Bernard} (26.07.1856 – 02.11.1950), \emph{Schriftsteller}|pwk}
                  statt, das von Trebitsch\pwindex{Trebitsch, Siegfried 22.12.1868 – 03.06.1956@\textsc{Trebitsch, Siegfried} (22.12.1868 – 03.06.1956), \emph{Schriftsteller, Übersetzer}|pwk} übersetzt
                  war.}}}\label{K_L03416-11h} wegen Shaw\pwindex{Shaw, George Bernard 26.07.1856 – 02.11.1950@\textsc{Shaw, George Bernard} (26.07.1856 – 02.11.1950), \emph{Schriftsteller}|pw} das Maß des
               lächerlichen erreicht. Seine erste Frage, als er hier eintraf, war (natürlich per
               Telefon) was ich von seinem \label{K_L03416-1v}\edtext{Vorschlag\pwindex{Trebitsch, Siegfried 22.12.1868 – 03.06.1956@\textsc{Trebitsch, Siegfried} (22.12.1868 – 03.06.1956), \emph{Schriftsteller, Übersetzer}!Buehnenvertrieb22. 03. 1906@\strich\emph{Bühnenvertrieb} {[}22. 03. 1906{]}|pwv}}{\lemma{\textnormal{\emph{Vorschlag}}}\Cendnote{\textnormal{Siegfried Trebitsch\pwindex{Trebitsch, Siegfried 22.12.1868 – 03.06.1956@\textsc{Trebitsch, Siegfried} (22.12.1868 – 03.06.1956), \emph{Schriftsteller, Übersetzer}|pwk}: \emph{Bühnenvertrieb}\pwindex{Trebitsch, Siegfried 22.12.1868 – 03.06.1956@\textsc{Trebitsch, Siegfried} (22.12.1868 – 03.06.1956), \emph{Schriftsteller, Übersetzer}!Buehnenvertrieb22. 03. 1906@\strich\emph{Bühnenvertrieb} {[}22. 03. 1906{]}|pwk}, Jg. 2, Nr. 12,
                     22. 3. 1906, S. 348–350. Darin forderte Trebitsch\pwindex{Trebitsch, Siegfried 22.12.1868 – 03.06.1956@\textsc{Trebitsch, Siegfried} (22.12.1868 – 03.06.1956), \emph{Schriftsteller, Übersetzer}|pwk} die Einrichtung einer Bühnengenossenschaft zur
                  Vertretung der Autorenrechte. Das motivierte den Herausgeber der Zeitschrift, Siegfried Jacobsohn\pwindex{Jacobsohn, Siegfried 28.01.1881 – 03.12.1926@\textsc{Jacobsohn, Siegfried} (28.01.1881 – 03.12.1926), \emph{Journalist, Kritiker, Publizist}|pwk}, zu einer mehrteiligen
                  Debatte, die sich über Monate streckte. In der zweiten Fortsetzung findet sich ein
                  Beitrag Schnitzler\pwindex{Schnitzler, Arthur 15.05.1862 – 21.10.1931@\textsc{Schnitzler, Arthur} (15.05.1862 – 21.10.1931), \emph{Schriftsteller, Mediziner}|pwk}s. vgl. A. S.: \emph{»Das Zeitlose ist von kürzester Dauer«}, Bund der Bühnendichter, 12. 4. 1906.}}}\label{K_L03416-1h} in der »Schaubühne\pwindex{Schaubuehne7.9.1905 – 1993@\emph{Die Schaubühne} {[}7.9.1905 – 1993{]}|pw}« halte. Ich sagte, dass ich dagegen
               sei. Er ließ seinen erstaunten Klagelaut vernehmen, und meinte dann, {\pb}\uline{Sie} hätten ihm einen »begeisterten« Brief
               geschrieben. Ich bin wirklich nicht sehr für diesen Vorschlag, der nur aus der
               Seidenbranche kommt; glaube an Ihre »Begeisterung« natürlich nicht, und halte die
               ganze Sache für unwichtig. Auch die Dienstboten betrügen uns, und man denkt nicht
               daran, sie abzuschaffen. Es fragt sich immer nur, um wie viel die Agenten die Autoren
               übervorteilen. Und das ist im Ganzen nicht gar so erheblich. \pend
           \pstart
           Heute schrieb mir Bahr\pwindex{Bahr, Hermann 19.07.1863 – 15.01.1934@\textsc{Bahr, Hermann} (19.07.1863 – 15.01.1934), \emph{Schriftsteller, Kritiker}|pw}, dass er
                  Sonntag Abend auf zwei Tage herkommt. Das ist mir weitaus angenehmer.
               Sonst bin ich ziemlich allein; kann mir zu Harden\pwindex{Harden, Maximilian 20.10.1861 – 30.10.1927@\textsc{Harden, Maximilian} (20.10.1861 – 30.10.1927), \emph{Schriftsteller, Publizist}|pw} kein Herz faßen seit jenem \label{K_L03416-45v}\edtext{Artikel\pwindex{Theater03. 03. 1906@\emph{Theater} {[}03. 03. 1906{]}|pwv}}{\lemma{\textnormal{\emph{Artikel}}}\Cendnote{\textnormal{M. H.\pwindex{Harden, Maximilian 20.10.1861 – 30.10.1927@\textsc{Harden, Maximilian} (20.10.1861 – 30.10.1927), \emph{Schriftsteller, Publizist}|pwk} [ = Maximilian Harden\pwindex{Harden, Maximilian 20.10.1861 – 30.10.1927@\textsc{Harden, Maximilian} (20.10.1861 – 30.10.1927), \emph{Schriftsteller, Publizist}|pwk}]: \emph{Theater}\pwindex{Theater03. 03. 1906@\emph{Theater} {[}03. 03. 1906{]}|pwk}. In: \emph{Die Zukunft}\pwindex{Zukunft1892 – 1922@\emph{Die Zukunft} {[}1892 – 1922{]}|pwk}, Bd. 54,
                     H. 9, 3. 3. 1906, S. 346–356.}}}\label{K_L03416-45h} und hab ihn seither
               auch nicht gesehen noch gesucht. Heute – es ist überhaupt ein lebhafter Tag –
               telefonirte mir Ihre Schwägerin\pwindex{Steinrueck, Elisabeth 19.11.1885 – 07.04.1920@\textsc{Steinrück, Elisabeth} (19.11.1885 – 07.04.1920)|pwv} wegen einer Schiffskarte. Ich bat sie, dieser Tage zu uns zu
               kommen, damit wir alles genauer besprechen. \pend
           \pstart
           Hier lege ich Ihnen das zweite \label{K_L03416-456v}\edtext{Russenfeuilleton\pwindex{Salten, Felix 06.09.1869 – 08.10.1945@\textsc{Salten, Felix} (06.09.1869 – 08.10.1945), \emph{Schriftsteller, Journalist}!Russisches Theater. II23. 03. 1906@\strich\emph{Russisches Theater. II} {[}23. 03. 1906{]}|pwv}}{\lemma{\textnormal{\emph{Russenfeuilleton}}}\Cendnote{\textnormal{Felix Salten\pwindex{Salten, Felix 06.09.1869 – 08.10.1945@\textsc{Salten, Felix} (06.09.1869 – 08.10.1945), \emph{Schriftsteller, Journalist}|pwk}: \emph{Russisches Theater. II}\pwindex{Salten, Felix 06.09.1869 – 08.10.1945@\textsc{Salten, Felix} (06.09.1869 – 08.10.1945), \emph{Schriftsteller, Journalist}!Russisches Theater. II23. 03. 1906@\strich\emph{Russisches Theater. II} {[}23. 03. 1906{]}|pwk}. In: \emph{B. Z. am Mittag}\pwindex{?? Werk@Nicht ermittelte Verfasserinnen und Verfasser!B.Z. am Mittag1904-10-22 – 1943@\emph{B.Z. am Mittag} {[}1904-10-22 – 1943{]}|pwk}, Jg. 30, Nr. 70,
                        23. 3. 1906, S. 2–3.}}}\label{K_L03416-456h} bei, und das über \label{KL373-5v}\edtext{Kater Lampe\pwindex{Rosenow, Emil 09.03.1871 – 07.02.1904@\textsc{Rosenow, Emil} (09.03.1871 – 07.02.1904), \emph{Schriftsteller, Journalist}!Kater Lampe1902@\strich\emph{Kater Lampe} {[}1902{]}|pw}}{\lemma{\textnormal{\emph{Kater Lampe}}}\Cendnote{\textnormal{Das Stück\pwindex{Rosenow, Emil 09.03.1871 – 07.02.1904@\textsc{Rosenow, Emil} (09.03.1871 – 07.02.1904), \emph{Schriftsteller, Journalist}!Kater Lampe1902@\strich\emph{Kater Lampe} {[}1902{]}|pwkv} von Rosenow\pwindex{Rosenow, Emil 09.03.1871 – 07.02.1904@\textsc{Rosenow, Emil} (09.03.1871 – 07.02.1904), \emph{Schriftsteller, Journalist}|pwk}
                  besprochen in: Felix Salten\pwindex{Salten, Felix 06.09.1869 – 08.10.1945@\textsc{Salten, Felix} (06.09.1869 – 08.10.1945), \emph{Schriftsteller, Journalist}|pwk}: \emph{»Kater Lampe«}\pwindex{Salten, Felix 06.09.1869 – 08.10.1945@\textsc{Salten, Felix} (06.09.1869 – 08.10.1945), \emph{Schriftsteller, Journalist}!Kater Lampe«26. 03. 1906@\strich\emph{»Kater Lampe«} {[}26. 03. 1906{]}|pwk}. In: \emph{B.
                        Z. am Mittag}\pwindex{?? Werk@Nicht ermittelte Verfasserinnen und Verfasser!B.Z. am Mittag1904-10-22 – 1943@\emph{B.Z. am Mittag} {[}1904-10-22 – 1943{]}|pwk}, Jg. 30, Nr. 72, 26. 3. 1906,
                  S. 2.}}}\label{KL373-5h}.\pend
           \pstart
           Herzliche Grüße von uns zu Ihnen.{\\[\baselineskip]}Ihr \spacefill\mbox{Salten}\pend
           \leftskip=0em{}
         
         \endnumbering\mylabel{h}\end{ledgroupsized}\begin{anhang}\end{anhang}\newcommand{\dateiname}{L03416}\newcommand{\titel}{Felix Salten an Arthur Schnitzler, 28. 3. 1906}\newcommand{\editorInnen}{Martin Anton Müller und Laura Untner}%% latex-leseansicht-abspann.tex
%% Abspann für die Leseansicht.
%% Der Schalter \ifkorrekturansicht ist bereits durch den Vorspann gesetzt.

%% latex-abspann.tex
%% Gemeinsamer Abspann für Korrekturansicht und Leseansicht.
%% Setzt den Schalter \ifkorrekturansicht voraus (gesetzt in den
%% einbindenden Dateien latex-korrekturansicht-abspann.tex bzw.
%% latex-leseansicht-abspann.tex).
%% ---------------------------------------------------------------

\normalsize

% Das esempio-Environment wird nur in der Leseansicht benötigt
\ifkorrekturansicht\else
\newenvironment{esempio}[3]%
{
    \vspace{1.5ex}
    \rlap{\underline{#1}}
    \par
    \setlength{\parindent}{0cm}
    \nopagebreak
    \leftskip=#2cm
    \rightskip=#3cm
}
{
    \par
}
\fi

\doendnotes{C}
\bigskip
\vfill

\clearpage

\footnotesize

\ifkorrekturansicht
  \lohead{\textsc{register}}
\fi

% theindex-Environment neu definieren ohne reledmac
\makeatletter
\renewenvironment{theindex}{%
  \ifkorrekturansicht
    \section*{\indexname}%
  \else
    \subsubsection*{Index der erwähnten Entitäten}%
  \fi
  \setlength{\parindent}{0pt}%
  \setlength{\parskip}{0pt plus 0.3pt}%
  \let\item\@idxitem
}{%
  \ifkorrekturansicht\clearpage\fi
}
\makeatother

\IfFileExists{\jobname-pw.ind}{\input{\jobname-pw.ind}}{}

% Quellenangabe nur in der Leseansicht
\ifkorrekturansicht\else
% Fallback-Definitionen, falls die .tex-Datei \titel etc. nicht gesetzt hat
\providecommand{\titel}{}
\providecommand{\editorInnen}{}
\providecommand{\dateiname}{\jobname}

\vspace{3cm}

\vfill

\footnotesize
\textsc{Quelle}: \titel. Herausgegeben von {\editorInnen}. In: \emph{Arthur Schnitzler: Briefwechsel mit Autorinnen und Autoren}.
 Digitale Edition, https://schnitzler-briefe.acdh.oeaw.ac.at/{\dateiname}.html (Stand \today)
\fi

\end{document}


      