%% latex-korrekturansicht-vorspann.tex
%% Vorspann für die Korrekturansicht.
%% Lädt die gemeinsame Datei latex-vorspann.tex mit gesetztem Schalter.

\newif\ifkorrekturansicht
\korrekturansichttrue

\input{../tex-inputs/latex-vorspann}


\section[Hugo von Hofmannsthal an Arthur Schnitzler, {[}29. 11. 1912{]}]{L02105 Hugo von Hofmannsthal an Arthur Schnitzler, {[}29. 11. 1912{]}}
\nopagebreak\mylabel{L02105v}
\rehead{ }\normalsize\beginnumbering\briefempfaengerindex{Schnitzler, Arthur@\textsc{Schnitzler, Arthur}!zzzHofmannsthal, Hugo von@\emph{von Hugo von Hofmannsthal}!1912-11-292@{{[}29. 11. 1912{]}}|(be}
\toendnotes[C]{\smallbreak\pagebreak[2]}\Standort{CUL, Schnitzler, B 43.}
\physDesc{Telegramm, 115 Zeichen
\newline{}maschinell
\newline{}Versand: mit schwarzer Tinte auf der Rückseite der postalische Vermerk
                                 des Telegrammboten: »\noindent{}{\pb}Adr.
                                          wohn{[}t nicht{]}{ }\textsc{Esplanade\oindex{Hotel Esplanade [Berlin]@\textbf{Hotel Esplanade [Berlin]}, \emph{Hotel (K.HTL)}|pw}}, nach Aussage des Poſt-Chefs ſoll Adr. im \textsc{Hotel Adlon\oindex{Hotel Adlon@\textbf{Hotel Adlon}, \emph{Hotel (K.HTL)}|pw}} wohnen?{ / }Geier\pwindex{Geier, Gustav @\textsc{Geier, Gustav}, \emph{Briefträger/Briefträgerin}|pw}{ }11/9.« 
\newline{}Schnitzler: mit Bleistift datiert: »29/11 912« 
\newline{}Ordnung: 1) beschnitten  2) mit Bleistift von unbekannter Hand nummeriert:
                                    »241«}
\buchAbdrucke{\weitereDrucke{Hugo von Hofmannsthal, Arthur Schnitzler: \emph{Briefwechsel}. Frankfurt am Main: \emph{S. Fischer} 1964, S. 270.} }
\pstart
           \noindent{}{\pb}tieftraurig um guten lieben nie
               wieder zufindenden brahm\pwindex{Brahm, Otto 05.02.1856 – 28.11.1912@\textsc{Brahm, Otto} (05.02.1856 – 28.11.1912), \emph{Theaterleiter/Theaterleiterin, Regisseur/Regisseurin}|pw} bitte ihm auch fuer
               mich blumen bringen von herzen ihr \spacefill\mbox{hugo +}\pend
           \selectlanguage{ngerman}\endnumbering\briefempfaengerindex{Schnitzler, Arthur@\textsc{Schnitzler, Arthur}!zzzHofmannsthal, Hugo von@\emph{von Hugo von Hofmannsthal}!1912-11-292@{{[}29. 11. 1912{]}}|)be}\mylabel{L02105h}  \normalsize

\doendnotes{C}
\bigskip
\vfill

\clearpage

\footnotesize

\lohead{\textsc{register}}

% Definiere theindex-Environment komplett neu ohne reledmac
\makeatletter
\renewenvironment{theindex}{%
  \section*{\indexname}%
  \setlength{\parindent}{0pt}%
  \setlength{\parskip}{0pt plus 0.3pt}%
  \let\item\@idxitem
}{%
  \clearpage
}
\makeatother

\IfFileExists{\jobname-pw.ind}{\input{\jobname-pw.ind}}{}

\end{document}

      