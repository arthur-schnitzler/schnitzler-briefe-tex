%% latex-leseansicht-vorspann.tex
%% Vorspann für die Leseansicht.
%% Lädt die gemeinsame Datei latex-vorspann.tex mit nicht gesetztem Schalter.

\newif\ifkorrekturansicht
\korrekturansichtfalse

\input{../tex-inputs/latex-vorspann}

\begin{center}
            \textcolor{red}{ENTWURF, NICHT FERTIG KORRIGIERT}
                      \end{center}
            
               \section[Paul Goldmann an Arthur Schnitzler, 8. 12. {[}1893{]}]{ Paul Goldmann an Arthur Schnitzler, 8. 12. {[}1893{]}}\nopagebreak\mylabel{v}\rehead{ }\begin{ledgroupsized}[t]{13cm}\normalsize\beginnumbering\briefempfaengerindex{Schnitzler, Arthur@\textsc{Schnitzler, Arthur}!zzzGoldmann, Paul@\emph{von Paul Goldmann}!1893-12-081@{8. 12. {[}1893{]}}|(be} \toendnotes[C]{\smallbreak\pagebreak[2]} \Standort{DLA, A:Schnitzler, HS.NZ85.1.3163.}
\physDesc{Brief, 1 Blatt, 4 Seiten
\newline{}Handschrift: schwarze Tinte, deutsche Kurrent
\newline{}Schnitzler: 1) mit Bleistift das Jahr »93« vermerkt 2) mit rotem Buntstift drei Unterstreichungen sowie ein nach unten
                                 gerichteter Pfeil am linken Rand der vierten Seite}\toendnotes[C]{\smallbreak}\pstart
           \noindent{}{\pb}\textcolor{gray}{\textbf{\textbf{Frankfurter Zeitung\orgindex{Frankfurter Zeitung@Frankfurter Zeitung|pw}.}}}\pend
           \pstart
           \textcolor{gray}{\textbf{\textbf{(\begin{otherlanguage}{french}Gazette de Francfort\end{otherlanguage}\orgindex{Frankfurter Zeitung@Frankfurter Zeitung|pw}.)}}}\pend
           \pstart
           \textcolor{gray}{\textbf{\begin{otherlanguage}{french}Directeur\pwindex{Sonnemann, Leopold 1831-10-29 – 1909-10-30@\textsc{Sonnemann, Leopold} (1831-10-29 – 1909-10-30), \emph{Journalist, Herausgeber}|pwv}\end{otherlanguage}{ }\textbf{M. L. Sonnemann\pwindex{Sonnemann, Leopold 1831-10-29 – 1909-10-30@\textsc{Sonnemann, Leopold} (1831-10-29 – 1909-10-30), \emph{Journalist, Herausgeber}|pw}.}}}\hfill \textsc{Paris\oindex{Paris@\textbf{Paris}|pw}}, 8. December\pend
           \pstart
           \begin{otherlanguage}{french}\textcolor{gray}{\textbf{Journal\pwindex{Frankfurter Zeitung1856 – 1943@\emph{Frankfurter Zeitung}|pw} politique, financier,}}\end{otherlanguage}\pend
           \pstart
           \begin{otherlanguage}{french}\textcolor{gray}{\textbf{commercial et litteraire.}}\end{otherlanguage}\pend
           \pstart
           \begin{otherlanguage}{french}\textcolor{gray}{\textbf{\textbf{Paraissant trois fois par jour}}}\end{otherlanguage}\pend
           \pstart
           \begin{otherlanguage}{french}\textcolor{gray}{\textbf{\textbf{Bureaux à Paris\oindex{Paris@\textbf{Paris}|pw}:}}}\end{otherlanguage}\pend
           \pstart
           \begin{otherlanguage}{french}\textcolor{gray}{\textbf{\textbf{rue Richelieu 75\oindex{rue Richelieu@\textbf{rue Richelieu}|pw}.}}}\end{otherlanguage}\pend
           \pstart
           Mein lieber Freund!\pend
           \pstart
           Dank für die \label{K_L02723-1v}\edtext{Kritiken\pwindex{Bahr, Hermann 19.07.1863 – 15.01.1934@\textsc{Bahr, Hermann} (19.07.1863 – 15.01.1934), \emph{Schriftsteller, Kritiker}!Maerchen (Schauspiel in drei Aufzuegen von Arthur Schnitzler)1893-12-02 – 1893-12-02@\strich\emph{Das Märchen (Schauspiel in drei Aufzügen von Arthur Schnitzler)} {[}1893-12-02 – 1893-12-02{]}|pwuv}\pwindex{Theater- und Kunstnachrichten [Maerchen-Urauffuehrung]1893-12-02 – 1893-12-02@\emph{Theater- und Kunstnachrichten [Märchen-Uraufführung]} {[}1893-12-02 – 1893-12-02{]}|pwuv}\pwindex{Notizen eines Theater-Habitues. (Raimund-Theater. – Das Maerchen.)1893-12-03 – 1893-12-03@\emph{Notizen eines Theater-Habitués. (Raimund-Theater. – Das Märchen.)} {[}1893-12-03 – 1893-12-03{]}|pwuv}}{\lemma{\textnormal{\emph{Kritiken}}}\Cendnote{\textnormal{Höchstwahrscheinlich handelte es sich um
                  folgende Kritiken\pwindex{Bahr, Hermann 19.07.1863 – 15.01.1934@\textsc{Bahr, Hermann} (19.07.1863 – 15.01.1934), \emph{Schriftsteller, Kritiker}!Maerchen (Schauspiel in drei Aufzuegen von Arthur Schnitzler)1893-12-02 – 1893-12-02@\strich\emph{Das Märchen (Schauspiel in drei Aufzügen von Arthur Schnitzler)} {[}1893-12-02 – 1893-12-02{]}|pwkv}\pwindex{Theater- und Kunstnachrichten [Maerchen-Urauffuehrung]1893-12-02 – 1893-12-02@\emph{Theater- und Kunstnachrichten [Märchen-Uraufführung]} {[}1893-12-02 – 1893-12-02{]}|pwkv}\textcolor{red}{\textsuperscript{XXXX indx}}: Hermann Bahr\pwindex{Bahr, Hermann 19.07.1863 – 15.01.1934@\textsc{Bahr, Hermann} (19.07.1863 – 15.01.1934), \emph{Schriftsteller, Kritiker}|pwk}: \emph{Das Märchen (Schauspiel in drei Aufzügen von Arthur
                        Schnitzler. Zum ersten Male aufgeführt am Deutschen Volkstheater den 1.
                        December)}\pwindex{Bahr, Hermann 19.07.1863 – 15.01.1934@\textsc{Bahr, Hermann} (19.07.1863 – 15.01.1934), \emph{Schriftsteller, Kritiker}!Maerchen (Schauspiel in drei Aufzuegen von Arthur Schnitzler)1893-12-02 – 1893-12-02@\strich\emph{Das Märchen (Schauspiel in drei Aufzügen von Arthur Schnitzler)} {[}1893-12-02 – 1893-12-02{]}|pwk}. In: \emph{Deutsche Zeitung}\pwindex{Deutsche Zeitung1871 – 1907@\emph{Deutsche Zeitung}|pwk},
                     Jg. 23, Nr.  7879, 2. 12. 1893, Morgen-Ausgabe\pwindex{Deutsche Zeitung1871 – 1907@\emph{Deutsche Zeitung}|pwkv}, S. 1–3, g.\pwindex{Granichstaedten, Emil 08.07.1847 – 02.07.1904@\textsc{Granichstaedten, Emil} (08.07.1847 – 02.07.1904), \emph{Journalist, Wissenschaftler}|pwkv} [=Emil Granichstaedten\pwindex{Granichstaedten, Emil 08.07.1847 – 02.07.1904@\textsc{Granichstaedten, Emil} (08.07.1847 – 02.07.1904), \emph{Journalist, Wissenschaftler}|pwk}]: \emph{Theater- und Kunstnachrichten}\pwindex{Theater- und Kunstnachrichten [Maerchen-Urauffuehrung]1893-12-02 – 1893-12-02@\emph{Theater- und Kunstnachrichten [Märchen-Uraufführung]} {[}1893-12-02 – 1893-12-02{]}|pwk}. In: \emph{Die
                        Presse}\pwindex{Presse3. 7. 1848@\emph{Die Presse}|pwk}, Jg. 46, Nr. 333, 2. 12. 1893,
                     S. 11, A. M. W.\pwindex{Willner, Alfred Maria 11.07.1859 – 27.10.1929@\textsc{Willner, Alfred Maria} (11.07.1859 – 27.10.1929), \emph{Schriftsteller, Journalist}|pwkv} [=Alfred Maria Willner\pwindex{Willner, Alfred Maria 11.07.1859 – 27.10.1929@\textsc{Willner, Alfred Maria} (11.07.1859 – 27.10.1929), \emph{Schriftsteller, Journalist}|pwk}]: \emph{Notizen eines Theater-Habitués. (Raimund-Theater. – Das
                        Märchen.)}\pwindex{Notizen eines Theater-Habitues. (Raimund-Theater. – Das Maerchen.)1893-12-03 – 1893-12-03@\emph{Notizen eines Theater-Habitués. (Raimund-Theater. – Das Märchen.)} {[}1893-12-03 – 1893-12-03{]}|pwk}. In: \emph{Wiener Salonblatt}\pwindex{Wiener Salonblatt1870 – 1938@\emph{Wiener Salonblatt}|pwk},
                     Jg. 24, Nr. 49, 3. 12. 1893, S. 8–9. Bei
                  der Kritik von Granichstaedten\pwindex{Granichstaedten, Emil 08.07.1847 – 02.07.1904@\textsc{Granichstaedten, Emil} (08.07.1847 – 02.07.1904), \emph{Journalist, Wissenschaftler}|pwk} könnte es
                  sich auch um folgende handeln: Emil Granichstaedten\pwindex{Granichstaedten, Emil 08.07.1847 – 02.07.1904@\textsc{Granichstaedten, Emil} (08.07.1847 – 02.07.1904), \emph{Journalist, Wissenschaftler}|pwk}: \emph{Feuilleton. Deutsches Volkstheater}\pwindex{Granichstaedten, Emil 08.07.1847 – 02.07.1904@\textsc{Granichstaedten, Emil} (08.07.1847 – 02.07.1904), \emph{Journalist, Wissenschaftler}!Feuilleton. Deutsches Volkstheater [Maerchen]1893-12-03 – 1893-12-03@\strich\emph{Feuilleton. Deutsches Volkstheater [Märchen]} {[}1893-12-03 – 1893-12-03{]}|pwk}. In: \emph{Die Presse}\pwindex{Presse3. 7. 1848@\emph{Die Presse}|pwk}, Jg. 46, Nr. 334, 3. 12. 1893, S.  1–2. Wahrscheinlicher ist jedoch, dass die
                     Kritik\pwindex{Theater- und Kunstnachrichten [Maerchen-Urauffuehrung]1893-12-02 – 1893-12-02@\emph{Theater- und Kunstnachrichten [Märchen-Uraufführung]} {[}1893-12-02 – 1893-12-02{]}|pwkv} vom 2. 12. 1893 gemeint war, insofern Schnitzler\pwindex{Schnitzler, Arthur 15.05.1862 – 21.10.1931@\textsc{Schnitzler, Arthur} (15.05.1862 – 21.10.1931), \emph{Schriftsteller, Mediziner}|pwk} diese auch im \emph{Tagebuch}\pwindex{Schnitzler, Arthur 15.05.1862 – 21.10.1931@\textsc{Schnitzler, Arthur} (15.05.1862 – 21.10.1931), \emph{Schriftsteller, Mediziner}!Tagebuch1981 – 2000@\strich\emph{Tagebuch} {[}1981 – 2000{]}|pwk} erwähnte (s. u.).}}}\label{K_L02723-1h}; ich kannte ſie größtentheils ſchon.
               Drei oder vier verſtehen Dich oder geben ſich wenigſtens ehrliche Mühe, Dich zu
               verſtehen. Der kleine \textsc{Salonblatt\orgindex{Wiener Salonblatt@Wiener Salonblatt|pw}}-Mann\pwindex{Willner, Alfred Maria 11.07.1859 – 27.10.1929@\textsc{Willner, Alfred Maria} (11.07.1859 – 27.10.1929), \emph{Schriftsteller, Journalist}|pwv}, der \strikeout{Dich} Dir zum Luſtſpiel räth,
               iſt auch auf der richtigen Fährte. Du brauchteſt unbedingt ein Paar Monate Pariſ\oindex{Paris@\textbf{Paris}|pw}er Theater; Du würdeſt die unermüdliche
               Anſtrengung des \label{K_L02723-2v}\edtext{jungen Stücks}{\lemma{\textnormal{\emph{jungen Stücks}}}\Cendnote{\textnormal{Siehe dazu etwa Sally Debra Charnow:
                        \emph{Theatre, Politics, and Markets in Fin-de-Siècle Paris.
                        Staging Modernity}. Basingstoke:
                        \emph{Palgrave Macmillan}{ }2005.}}}\label{K_L02723-2h} ſehen, objectiv, kurz, natürlich, luſtig zu werden. Das iſt der
               Weg, {\pb}der geradeaus in die Zukunft geht. Das iſt
               auch der Weg Deines Talents. Ein Luſtſpiel, theuerſter Freund, – oder ein Schauſpiel,
               aber ohne Herzensergüſſe. Könnteſt Du Dich nur mit meinen Augen ſehen – Du würdeſt
               keinen Augenblick mehr zögern, und in einem Jahre wäre die Vollendung da, in
               Production wie Erfolg. Bitte ſchreib’ mir ein Wort über Deine Pläne.\pend
           \pstart
           \textsc{Bahr}\pwindex{Bahr, Hermann 19.07.1863 – 15.01.1934@\textsc{Bahr, Hermann} (19.07.1863 – 15.01.1934), \emph{Schriftsteller, Kritiker}|pw} – der kränkt Dich ſo? Er iſt frech, größenwahnſinnig, unausſtehlich doctrinär.
                  \strikeout{E} Der Verweis\pwindex{Bahr, Hermann 19.07.1863 – 15.01.1934@\textsc{Bahr, Hermann} (19.07.1863 – 15.01.1934), \emph{Schriftsteller, Kritiker}!Maerchen (Schauspiel in drei Aufzuegen von Arthur Schnitzler)1893-12-02 – 1893-12-02@\strich\emph{Das Märchen (Schauspiel in drei Aufzügen von Arthur Schnitzler)} {[}1893-12-02 – 1893-12-02{]}|pwv} auf ſeine »Neuen
                  Menſchen\pwindex{Bahr, Hermann 19.07.1863 – 15.01.1934@\textsc{Bahr, Hermann} (19.07.1863 – 15.01.1934), \emph{Schriftsteller, Kritiker}!neuen Menschen. Ein Schauspiel1887@\strich\emph{Die neuen Menschen. Ein Schauspiel} {[}1887{]}|pw}« iſt eine glatte Gemeinheit. Und doch finde ich ihn nicht
               reſpectlos; und doch finde ich, daß {\pb}er manches
               Richtige ſagt. Vielleicht aber fehlt mir auch das richtige Urtheil; ich bin ſo außer
               Zuſammenhang mit den Wien\oindex{Wien@\textbf{Wien}|pw}er Verhältniſſen. Heiter
               iſt nur, wie der Burſch\pwindex{Bahr, Hermann 19.07.1863 – 15.01.1934@\textsc{Bahr, Hermann} (19.07.1863 – 15.01.1934), \emph{Schriftsteller, Kritiker}|pwv}
               franzöſiſche Dinge citirt: \label{K_L02723-3v}\edtext{»\begin{otherlanguage}{french}Le grappin\end{otherlanguage}\pwindex{Salandri, Gaston 1856 – 1917@\textsc{Salandri, Gaston} (1856 – 1917), \emph{Schriftsteller/Schriftstellerin}!Le Grappin1892@\strich\emph{Le Grappin} {[}1892{]}|pw}«}{\lemma{\textnormal{\emph{»Le grappin«}}}\Cendnote{\textnormal{\emph{Le Grappin}\pwindex{Salandri, Gaston 1856 – 1917@\textsc{Salandri, Gaston} (1856 – 1917), \emph{Schriftsteller/Schriftstellerin}!Le Grappin1892@\strich\emph{Le Grappin} {[}1892{]}|pwk} ist eine dreiaktige Komödie\pwindex{Salandri, Gaston 1856 – 1917@\textsc{Salandri, Gaston} (1856 – 1917), \emph{Schriftsteller/Schriftstellerin}!Le Grappin1892@\strich\emph{Le Grappin} {[}1892{]}|pwkv} von Gaston Salandri\pwindex{Salandri, Gaston 1856 – 1917@\textsc{Salandri, Gaston} (1856 – 1917), \emph{Schriftsteller/Schriftstellerin}|pwk} aus dem Jahr 1892. Der entsprechende Absatz in Bahr\pwindex{Bahr, Hermann 19.07.1863 – 15.01.1934@\textsc{Bahr, Hermann} (19.07.1863 – 15.01.1934), \emph{Schriftsteller, Kritiker}|pwk}s Kritik\pwindex{Bahr, Hermann 19.07.1863 – 15.01.1934@\textsc{Bahr, Hermann} (19.07.1863 – 15.01.1934), \emph{Schriftsteller, Kritiker}!Maerchen (Schauspiel in drei Aufzuegen von Arthur Schnitzler)1893-12-02 – 1893-12-02@\strich\emph{Das Märchen (Schauspiel in drei Aufzügen von Arthur Schnitzler)} {[}1893-12-02 – 1893-12-02{]}|pwkv} lautete wie folgt: »Er konnte die Eiferſucht der
                     Vergangenheit am Werke\pwindex{Schnitzler, Arthur 15.05.1862 – 21.10.1931@\textsc{Schnitzler, Arthur} (15.05.1862 – 21.10.1931), \emph{Schriftsteller, Mediziner}!Maerchen. Schauspiel in drei Aufzuegen1891 – 1891@\strich\emph{Das Märchen. Schauspiel in drei Aufzügen} {[}1891 – 1891{]}|pwv}
                     zeigen; wie etwa Othello\pwindex{\textcolor{red}{\textsuperscript{XXXX1 indx}}!Othello1604@\strich\emph{Othello} {[}1604{]}|pw} die Eiferſucht in
                     der Gegenwart zeigt: er nahm dann eine Liebe und ließ ſie an der Vergangenheit
                     des Mädchens verderben, die allmälig ſei es geſtanden, ſei es verrathen wird;
                     der Schmerz des Mannes zwiſchen Leidenſchaft und Ehre und die Buße der
                     Gefallenen waren da die Kräfte, die die Handlung trieben. Oder er konnte einen
                     Spötter gegen dieſe Eiferſucht zeigen, der ſich über ſie heben will, aber
                     leidend von ihrem Rechte gezwungen wird; er ſchrieb dann das Stück\pwindex{Salandri, Gaston 1856 – 1917@\textsc{Salandri, Gaston} (1856 – 1917), \emph{Schriftsteller/Schriftstellerin}!Le Grappin1892@\strich\emph{Le Grappin} {[}1892{]}|pwv}, das Gaſton Salandri\pwindex{Salandri, Gaston 1856 – 1917@\textsc{Salandri, Gaston} (1856 – 1917), \emph{Schriftsteller/Schriftstellerin}|pw} als »Le
                        Grappin\pwindex{Salandri, Gaston 1856 – 1917@\textsc{Salandri, Gaston} (1856 – 1917), \emph{Schriftsteller/Schriftstellerin}!Le Grappin1892@\strich\emph{Le Grappin} {[}1892{]}|pw}« geschrieben und die Pariſ\oindex{Paris@\textbf{Paris}|pw}er Freie Bühne\orgindex{Theâtre Libre@Théâtre Libre|pwv}
                     geſpielt hat, die Geſchichte des Herrn Jacques Privat\pwindex{Salandri, Gaston 1856 – 1917@\textsc{Salandri, Gaston} (1856 – 1917), \emph{Schriftsteller/Schriftstellerin}!Le Grappin1892@\strich\emph{Le Grappin} {[}1892{]}|pwv}, der das Vorurtheil verachtet und ſich
                     mit ſeiner Geliebten vermählt, obwohl er weiß, daß ſie vor ihm Anderen gehörte
                     und liederlich lebte; da wird gezeigt, daß alle Liebe die Vergangenheit nicht
                     tilgen, nicht verwiſchen kann, ja, durch die tauſend Stiche der Nerven, des
                     Gemüthes und die Kränkungen der Ehre ſich in Zorn, Ekel, Haß verwandeln muß.
                     Mit dem erſten Stücke\pwindex{\textcolor{red}{\textsuperscript{XXXX1 indx}}!Othello1604@\strich\emph{Othello} {[}1604{]}|pwv}
                     geht der Hörer, auch wenn er dieſe Eiferſucht nicht hat, weil er ſich doch aus
                     Anderen in ſie denken kann. Mit dem zweiten\pwindex{Salandri, Gaston 1856 – 1917@\textsc{Salandri, Gaston} (1856 – 1917), \emph{Schriftsteller/Schriftstellerin}!Le Grappin1892@\strich\emph{Le Grappin} {[}1892{]}|pwv} kann er gegen das Vorurtheil, das ja von dem
                     Helden beſtritten, und er kann für das Vorurtheil mit ihm gehen, das doch
                     ſchließlich beſtätigt wird. Er iſt Beiden\pwindex{\textcolor{red}{\textsuperscript{XXXX1 indx}}!Othello1604@\strich\emph{Othello} {[}1604{]}|pwv}\pwindex{Salandri, Gaston 1856 – 1917@\textsc{Salandri, Gaston} (1856 – 1917), \emph{Schriftsteller/Schriftstellerin}!Le Grappin1892@\strich\emph{Le Grappin} {[}1892{]}|pwv} empfänglich.« (S. 1) Das Stück
                     \emph{Die neuen Menschen}\pwindex{Bahr, Hermann 19.07.1863 – 15.01.1934@\textsc{Bahr, Hermann} (19.07.1863 – 15.01.1934), \emph{Schriftsteller, Kritiker}!neuen Menschen. Ein Schauspiel1887@\strich\emph{Die neuen Menschen. Ein Schauspiel} {[}1887{]}|pwk} erwähnte Bahr\pwindex{Bahr, Hermann 19.07.1863 – 15.01.1934@\textsc{Bahr, Hermann} (19.07.1863 – 15.01.1934), \emph{Schriftsteller, Kritiker}|pwk} als weiteres Beispiel für ein Drama\pwindex{Bahr, Hermann 19.07.1863 – 15.01.1934@\textsc{Bahr, Hermann} (19.07.1863 – 15.01.1934), \emph{Schriftsteller, Kritiker}!neuen Menschen. Ein Schauspiel1887@\strich\emph{Die neuen Menschen. Ein Schauspiel} {[}1887{]}|pwkv}, in dem es um
                     »jene[n] Zwiſt[] von Verſtand und Gefühl« (S. 2) gehe.}}}\label{K_L02723-3h},
               das Theâtre-Libre\orgindex{Theâtre Libre@Théâtre Libre|pw}-Sück\pwindex{Salandri, Gaston 1856 – 1917@\textsc{Salandri, Gaston} (1856 – 1917), \emph{Schriftsteller/Schriftstellerin}!Le Grappin1892@\strich\emph{Le Grappin} {[}1892{]}|pwv}, von dem er ſpricht,
               behandelt etwas abſolut Anderes als das, was er behauptet. Ein frecher Schwindel, um
               ſich in allen \introOben{}franzöſiſchen\introOben{} Sätteln moderner Literatur
               gerecht zu zeigen.\pend
           \pstart
           \textsc{Granichstaedten\pwindex{Granichstaedten, Emil 08.07.1847 – 02.07.1904@\textsc{Granichstaedten, Emil} (08.07.1847 – 02.07.1904), \emph{Journalist, Wissenschaftler}|pw}} hätte ich an Deiner Stelle geohrfeigt. Das iſt keine Kritik\pwindex{Theater- und Kunstnachrichten [Maerchen-Urauffuehrung]1893-12-02 – 1893-12-02@\emph{Theater- und Kunstnachrichten [Märchen-Uraufführung]} {[}1893-12-02 – 1893-12-02{]}|pwuv}\pwindex{Granichstaedten, Emil 08.07.1847 – 02.07.1904@\textsc{Granichstaedten, Emil} (08.07.1847 – 02.07.1904), \emph{Journalist, Wissenschaftler}!Feuilleton. Deutsches Volkstheater [Maerchen]1893-12-03 – 1893-12-03@\strich\emph{Feuilleton. Deutsches Volkstheater [Märchen]} {[}1893-12-03 – 1893-12-03{]}|pwuv}, ſondern ein
                  \label{K_L02723-5v}\edtext{Gaſſenbube\pwindex{Granichstaedten, Emil 08.07.1847 – 02.07.1904@\textsc{Granichstaedten, Emil} (08.07.1847 – 02.07.1904), \emph{Journalist, Wissenschaftler}|pwv}nſtreich}{\lemma{\textnormal{\emph{Gaſſenbubenſtreich}}}\Cendnote{\textnormal{Auch Schnitzler\pwindex{Schnitzler, Arthur 15.05.1862 – 21.10.1931@\textsc{Schnitzler, Arthur} (15.05.1862 – 21.10.1931), \emph{Schriftsteller, Mediziner}|pwk} scheint über die Kritik\pwindex{Theater- und Kunstnachrichten [Maerchen-Urauffuehrung]1893-12-02 – 1893-12-02@\emph{Theater- und Kunstnachrichten [Märchen-Uraufführung]} {[}1893-12-02 – 1893-12-02{]}|pwkv} von Emil
                     Granichstaedten\pwindex{Granichstaedten, Emil 08.07.1847 – 02.07.1904@\textsc{Granichstaedten, Emil} (08.07.1847 – 02.07.1904), \emph{Journalist, Wissenschaftler}|pwk} verärgert gewesen zu sein, bezeichnete er sie doch am 2. 12. 1893 im \emph{Tagebuch}\pwindex{Schnitzler, Arthur 15.05.1862 – 21.10.1931@\textsc{Schnitzler, Arthur} (15.05.1862 – 21.10.1931), \emph{Schriftsteller, Mediziner}!Tagebuch1981 – 2000@\strich\emph{Tagebuch} {[}1981 – 2000{]}|pwk} als »[p]erfid dumm«.
                  In der Kritik\pwindex{Theater- und Kunstnachrichten [Maerchen-Urauffuehrung]1893-12-02 – 1893-12-02@\emph{Theater- und Kunstnachrichten [Märchen-Uraufführung]} {[}1893-12-02 – 1893-12-02{]}|pwkv} vom 2. 12. 1893 ging es Granichstaedten\pwindex{Granichstaedten, Emil 08.07.1847 – 02.07.1904@\textsc{Granichstaedten, Emil} (08.07.1847 – 02.07.1904), \emph{Journalist, Wissenschaftler}|pwk}, der insbesondere die Schauspielkunst Adele Sandrock\pwindex{Sandrock, Adele 19.08.1863 – 30.08.1937@\textsc{Sandrock, Adele} (19.08.1863 – 30.08.1937), \emph{Schauspielerin}|pwk}s lobte, vor allem um sexuelle Aspekte im
                     \emph{Märchen}\pwindex{Schnitzler, Arthur 15.05.1862 – 21.10.1931@\textsc{Schnitzler, Arthur} (15.05.1862 – 21.10.1931), \emph{Schriftsteller, Mediziner}!Maerchen. Schauspiel in drei Aufzuegen1891 – 1891@\strich\emph{Das Märchen. Schauspiel in drei Aufzügen} {[}1891 – 1891{]}|pwk}, auf die er recht abschätzig
                  anspielte. Im Zuge dessen kritisierte er zwischen den Zeilen auch die Handlung\pwindex{Schnitzler, Arthur 15.05.1862 – 21.10.1931@\textsc{Schnitzler, Arthur} (15.05.1862 – 21.10.1931), \emph{Schriftsteller, Mediziner}!Maerchen. Schauspiel in drei Aufzuegen1891 – 1891@\strich\emph{Das Märchen. Schauspiel in drei Aufzügen} {[}1891 – 1891{]}|pwkv} an sich und die
                  Figuren des Fedor\pwindex{Schnitzler, Arthur 15.05.1862 – 21.10.1931@\textsc{Schnitzler, Arthur} (15.05.1862 – 21.10.1931), \emph{Schriftsteller, Mediziner}!Maerchen. Schauspiel in drei Aufzuegen1891 – 1891@\strich\emph{Das Märchen. Schauspiel in drei Aufzügen} {[}1891 – 1891{]}|pwkv} und der
                     Fanny\pwindex{Schnitzler, Arthur 15.05.1862 – 21.10.1931@\textsc{Schnitzler, Arthur} (15.05.1862 – 21.10.1931), \emph{Schriftsteller, Mediziner}!Maerchen. Schauspiel in drei Aufzuegen1891 – 1891@\strich\emph{Das Märchen. Schauspiel in drei Aufzügen} {[}1891 – 1891{]}|pwkv}. Am 3. 12. 1893 holte Granichstaedten\pwindex{Granichstaedten, Emil 08.07.1847 – 02.07.1904@\textsc{Granichstaedten, Emil} (08.07.1847 – 02.07.1904), \emph{Journalist, Wissenschaftler}|pwk} weiter aus. Angefangen beim »Peſſimismus unſerer
                        ›Wien\oindex{Wien@\textbf{Wien}|pw}er Modernen‹« (S. 1)
                  kritisierte der sich auf der Seite des Naturalismus Positionierende\pwindex{Granichstaedten, Emil 08.07.1847 – 02.07.1904@\textsc{Granichstaedten, Emil} (08.07.1847 – 02.07.1904), \emph{Journalist, Wissenschaftler}|pwkv} auf äußerst abwertende
                  Weise ganz grundsätzlich das noch junge Werk Schnitzler\pwindex{Schnitzler, Arthur 15.05.1862 – 21.10.1931@\textsc{Schnitzler, Arthur} (15.05.1862 – 21.10.1931), \emph{Schriftsteller, Mediziner}|pwk}s, bezog er sich doch auch auf den \emph{Anatol-Zyklus}\pwindex{Schnitzler, Arthur 15.05.1862 – 21.10.1931@\textsc{Schnitzler, Arthur} (15.05.1862 – 21.10.1931), \emph{Schriftsteller, Mediziner}!Anatol1892-10-29 – 1892-10-29@\strich\emph{Anatol} {[}1892-10-29 – 1892-10-29{]}|pwk}. Der Autor\pwindex{Schnitzler, Arthur 15.05.1862 – 21.10.1931@\textsc{Schnitzler, Arthur} (15.05.1862 – 21.10.1931), \emph{Schriftsteller, Mediziner}|pwkv} orientiere sich zu stark an »modernen«, französischen
                  Strömungen, was ihm jedoch nicht gelinge: »Für dieſen Fedor\pwindex{Schnitzler, Arthur 15.05.1862 – 21.10.1931@\textsc{Schnitzler, Arthur} (15.05.1862 – 21.10.1931), \emph{Schriftsteller, Mediziner}!Maerchen. Schauspiel in drei Aufzuegen1891 – 1891@\strich\emph{Das Märchen. Schauspiel in drei Aufzügen} {[}1891 – 1891{]}|pwv} und dieſe Fanny\pwindex{Schnitzler, Arthur 15.05.1862 – 21.10.1931@\textsc{Schnitzler, Arthur} (15.05.1862 – 21.10.1931), \emph{Schriftsteller, Mediziner}!Maerchen. Schauspiel in drei Aufzuegen1891 – 1891@\strich\emph{Das Märchen. Schauspiel in drei Aufzügen} {[}1891 – 1891{]}|pwv} kann kein Publikum der Welt
                     ſich intereſſieren.« (S. 2), \emph{Das
                     Märchen}\pwindex{Schnitzler, Arthur 15.05.1862 – 21.10.1931@\textsc{Schnitzler, Arthur} (15.05.1862 – 21.10.1931), \emph{Schriftsteller, Mediziner}!Maerchen. Schauspiel in drei Aufzuegen1891 – 1891@\strich\emph{Das Märchen. Schauspiel in drei Aufzügen} {[}1891 – 1891{]}|pwk} sei »nicht tugendhaft« und »[u]m
                     Reinlichkeit wird gebeten« (S. 2).}}}\label{K_L02723-5h}.\pend
           \pstart
           Freut mich, daß Du nicht {\pb}verbittert biſt. Das
               gehört ſich auch ſo. Ich meine, Du kannſt mit Deinem Debüt\pwindex{Schnitzler, Arthur 15.05.1862 – 21.10.1931@\textsc{Schnitzler, Arthur} (15.05.1862 – 21.10.1931), \emph{Schriftsteller, Mediziner}!Maerchen. Schauspiel in drei Aufzuegen1891 – 1891@\strich\emph{Das Märchen. Schauspiel in drei Aufzügen} {[}1891 – 1891{]}|pwv} ſehr zufrieden ſein. Man gibt Dir
               Credit, und das iſt enorm für einen Jungen.\pend
           \pstart
           Haſt Du \label{K_L02723-4v}\edtext{\textsc{Loris\pwindex{Hofmannsthal, Hugo von 01.02.1874 – 15.07.1929@\textsc{Hofmannsthal, Hugo von} (01.02.1874 – 15.07.1929), \emph{Schriftsteller}|pw}} über \textsc{Bauernfeld\pwindex{Bauernfeld, Eduard von 13.01.1802 – 04.08.1890@\textsc{Bauernfeld, Eduard von} (13.01.1802 – 04.08.1890)|pw}}\pwindex{Eduard von BauernfelDs dramatischer Nachlass1893-12-06 – 1893-12-06@\emph{Eduard von Bauernfeld’s dramatischer Nachlaß} {[}1893-12-06 – 1893-12-06{]}|pwv}}{\lemma{\textnormal{\emph{Loris über Bauernfeld}}}\Cendnote{\textnormal{Loris\pwindex{Hofmannsthal, Hugo von 01.02.1874 – 15.07.1929@\textsc{Hofmannsthal, Hugo von} (01.02.1874 – 15.07.1929), \emph{Schriftsteller}|pwk}: \emph{Eduard von Bauernfeld’s dramatischer Nachlaß}\pwindex{Eduard von BauernfelDs dramatischer Nachlass1893-12-06 – 1893-12-06@\emph{Eduard von Bauernfeld’s dramatischer Nachlaß} {[}1893-12-06 – 1893-12-06{]}|pwk}. In: \emph{Frankfurter Zeitung}\pwindex{Frankfurter Zeitung1856 – 1943@\emph{Frankfurter Zeitung}|pwk}, Jg. 38, Nr. 338, 6. 12. 1893, erstes Morgenblatt,
                  S. 1.}}}\label{K_L02723-4h} geleſen? Wie aus dieſem gottbegnadeten Menſchen\pwindex{Hofmannsthal, Hugo von 01.02.1874 – 15.07.1929@\textsc{Hofmannsthal, Hugo von} (01.02.1874 – 15.07.1929), \emph{Schriftsteller}|pwv} die entzückenden Dinge
               herausquellen, ſo leicht und ſprudelnd. Ein Dichter\pwindex{Hofmannsthal, Hugo von 01.02.1874 – 15.07.1929@\textsc{Hofmannsthal, Hugo von} (01.02.1874 – 15.07.1929), \emph{Schriftsteller}|pwv}! Derjenige vielleicht, den man ſeit fünfzig Jahren
               erwartet!\pend
           \pstart
           Grüß’ ihn von mir, denn ich habe keine directe Verbindung mehr mit ihm; Grüße auch
                  \textsc{Richard\pwindex{Beer-Hofmann, Richard 11.07.1866 – 26.09.1945@\textsc{Beer-Hofmann, Richard} (11.07.1866 – 26.09.1945), \emph{Schriftsteller}|pw}} aus ſelbigem Grunde; ſei ſelbſt herzlichſt gegrüßt und ſchreibe bald!\pend
           \pstart Dein \spacefill\mbox{Paul Goldm}\pend{}\endnumbering\briefempfaengerindex{Schnitzler, Arthur@\textsc{Schnitzler, Arthur}!zzzGoldmann, Paul@\emph{von Paul Goldmann}!1893-12-081@{8. 12. {[}1893{]}}|)be}\mylabel{h}\end{ledgroupsized}\begin{anhang}\end{anhang}\newcommand{\dateiname}{L02723}\newcommand{\titel}{Paul Goldmann an Arthur Schnitzler, 8. 12. [1893]}\newcommand{\editorInnen}{Martin Anton Müller und Laura Untner}
            \footnotesize
\begin{ledgroupsized}[t]{11.5cm}
\doendnotes{C}
\end{ledgroupsized}
         %% latex-leseansicht-abspann.tex
%% Abspann für die Leseansicht.
%% Der Schalter \ifkorrekturansicht ist bereits durch den Vorspann gesetzt.

%% latex-abspann.tex
%% Gemeinsamer Abspann für Korrekturansicht und Leseansicht.
%% Setzt den Schalter \ifkorrekturansicht voraus (gesetzt in den
%% einbindenden Dateien latex-korrekturansicht-abspann.tex bzw.
%% latex-leseansicht-abspann.tex).
%% ---------------------------------------------------------------

\normalsize

% Das esempio-Environment wird nur in der Leseansicht benötigt
\ifkorrekturansicht\else
\newenvironment{esempio}[3]%
{
    \vspace{1.5ex}
    \rlap{\underline{#1}}
    \par
    \setlength{\parindent}{0cm}
    \nopagebreak
    \leftskip=#2cm
    \rightskip=#3cm
}
{
    \par
}
\fi

\doendnotes{C}
\bigskip
\vfill

\clearpage

\footnotesize

\ifkorrekturansicht
  \lohead{\textsc{register}}
\fi

% theindex-Environment neu definieren ohne reledmac
\makeatletter
\renewenvironment{theindex}{%
  \ifkorrekturansicht
    \section*{\indexname}%
  \else
    \subsubsection*{Index der erwähnten Entitäten}%
  \fi
  \setlength{\parindent}{0pt}%
  \setlength{\parskip}{0pt plus 0.3pt}%
  \let\item\@idxitem
}{%
  \ifkorrekturansicht\clearpage\fi
}
\makeatother

\IfFileExists{\jobname-pw.ind}{\input{\jobname-pw.ind}}{}

% Quellenangabe nur in der Leseansicht
\ifkorrekturansicht\else
% Fallback-Definitionen, falls die .tex-Datei \titel etc. nicht gesetzt hat
\providecommand{\titel}{}
\providecommand{\editorInnen}{}
\providecommand{\dateiname}{\jobname}

\vspace{3cm}

\vfill

\footnotesize
\textsc{Quelle}: \titel. Herausgegeben von {\editorInnen}. In: \emph{Arthur Schnitzler: Briefwechsel mit Autorinnen und Autoren}.
 Digitale Edition, https://schnitzler-briefe.acdh.oeaw.ac.at/{\dateiname}.html (Stand \today)
\fi

\end{document}


      