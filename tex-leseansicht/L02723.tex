%% latex-leseansicht-vorspann.tex
%% Vorspann für die Leseansicht.
%% Lädt die gemeinsame Datei latex-vorspann.tex mit nicht gesetztem Schalter.

\newif\ifkorrekturansicht
\korrekturansichtfalse

\input{../tex-inputs/latex-vorspann}


\section[Paul Goldmann an Arthur Schnitzler, 8. 12. [1893]]{L02723 Paul Goldmann an Arthur Schnitzler, 8. 12. [1893]}
\nopagebreak\mylabel{L02723v}
\rehead{ }\normalsize\beginnumbering\briefempfaengerindex{Schnitzler, Arthur@\textsc{Schnitzler, Arthur}!zzzGoldmann, Paul@\emph{von Paul Goldmann}!1893-12-081@{8. 12. [1893]}|(be}
\toendnotes[C]{\smallbreak\pagebreak[2]}
\correspDesc{Versand  durch Paul Goldmann am 8. 12. [1893] in Paris
\newline{}Erhalt  durch Arthur Schnitzler im Zeitraum [9. 12. 1893
                  – 13. 12. 1893?] in Wien}\toendnotes[C]{\smallbreak}
\Standort{DLA, A:Schnitzler, HS.NZ85.1.3163.}
\physDesc{Brief, 1 Blatt, 4 Seiten, 2096 Zeichen
\newline{}Handschrift: schwarze Tinte, deutsche Kurrent
\newline{}Schnitzler: 1) mit Bleistift das Jahr »93« vermerkt  2) mit rotem Buntstift drei Unterstreichungen sowie ein Pfeil, der
                                 den ganzen Absatz zu Hofmannsthal\pwindex{Hofmannsthal, Hugo von 1.\,2.\,1874 Wien – 15.\,7.\,1929 Rodaun@\textsc{Hofmannsthal, Hugo von} (1.\,2.\,1874 Wien – 15.\,7.\,1929 Rodaun), \emph{Schriftsteller}|pw} markieren soll}\toendnotes[C]{\smallbreak}
\pstart
           {\pb}\textcolor{gray}{\textbf{\textbf{Frankfurter Zeitung\orgindex{Frankfurter Zeitung@Frankfurter Zeitung|pw}.}}}\pend
           
\pstart
           \textcolor{gray}{\textbf{\textbf{(\begin{otherlanguage}{french}Gazette de Francfort\end{otherlanguage}\orgindex{Frankfurter Zeitung@Frankfurter Zeitung|pw}.)}}}\pend
           
\pstart
           \textcolor{gray}{\textbf{\begin{otherlanguage}{french}Directeur\end{otherlanguage}{ }\textbf{M. L. Sonnemann\pwindex{Sonnemann, Leopold 29.\,10.\,1831 Höchberg – 30.\,10.\,1909 Frankfurt am Main@\textsc{Sonnemann, Leopold} (29.\,10.\,1831 Höchberg – 30.\,10.\,1909 Frankfurt am Main), \emph{Journalist, Herausgeber}|pw}.}}}\hfill \textsc{Paris\oindex{Paris@\textbf{Paris}, \emph{Hauptstadt}|pw}}, 8. December.\pend
           
\pstart
           \begin{otherlanguage}{french}\textcolor{gray}{\textbf{Journal politique, financier,}}\end{otherlanguage}\pend
           
\pstart
           \begin{otherlanguage}{french}\textcolor{gray}{\textbf{commercial et litteraire.}}\end{otherlanguage}\pend
           
\pstart
           \begin{otherlanguage}{french}\textcolor{gray}{\textbf{\textbf{Paraissant trois fois par jour}}}\end{otherlanguage}\pend
           
\pstart
           \begin{otherlanguage}{french}\textcolor{gray}{\textbf{\textbf{Bureaux à Paris\oindex{Paris@\textbf{Paris}, \emph{Hauptstadt}|pw}:}}}\end{otherlanguage}\pend
           
\pstart
           \begin{otherlanguage}{french}\textcolor{gray}{\textbf{\textbf{rue Richelieu 75\oindex{rue Richelieu@\textbf{rue Richelieu}, \emph{Straße}|pw}.}}}\end{otherlanguage}\pend
           
\pstart\center{}Mein lieber Freund!\pend\vspace{0.5em}
\pstart
           Dank für die \label{K_L02723-1v}\edtext{Kritiken}{\lemma{\textnormal{\emph{Kritiken}}}\Cendnote{\textnormal{Zu den ihm bekannten Kritiken vgl. XXXX Auszeichnungsfehler: Dokument L02721 nicht gefunden.
               }}}\label{K_L02723-1}; ich kannte{ }ſie größtentheils{ }ſchon. Drei oder vier verſtehen Dich oder geben{ }ſich wenigſtens ehrliche Mühe, Dich zu verſtehen. Der \label{K_L02723-2v}\edtext{kleine \textsc{Salonblatt\orgindex{Wiener Salonblatt@Wiener Salonblatt|pw}}-Mann\pwindex{Willner, Alfred Maria 11.\,7.\,1859 Wien – 27.\,10.\,1929 ebd.@\textsc{Willner, Alfred Maria} (11.\,7.\,1859 Wien – 27.\,10.\,1929 ebd.), \emph{Schriftsteller, Journalist}|pwv}, der \strikeout{Dich} Dir zum Luſtſpiel}{\lemma{\textnormal{\emph{kleine … Lustspiel}}}\Cendnote{\textnormal{A. M. W.\pwindex{Willner, Alfred Maria 11.\,7.\,1859 Wien – 27.\,10.\,1929 ebd.@\textsc{Willner, Alfred Maria} (11.\,7.\,1859 Wien – 27.\,10.\,1929 ebd.), \emph{Schriftsteller, Journalist}|pwkv} [ = Alfred Maria Willner\pwindex{Willner, Alfred Maria 11.\,7.\,1859 Wien – 27.\,10.\,1929 ebd.@\textsc{Willner, Alfred Maria} (11.\,7.\,1859 Wien – 27.\,10.\,1929 ebd.), \emph{Schriftsteller, Journalist}|pwk}]: \emph{Notizen eines Theater-Habitués. (Raimund-Theater. – Das
                        Märchen.)}\pwindex{Willner, Alfred Maria 11.\,7.\,1859 Wien – 27.\,10.\,1929 ebd.@\textsc{Willner, Alfred Maria} (11.\,7.\,1859 Wien – 27.\,10.\,1929 ebd.), \emph{Schriftsteller, Journalist}!Notizen eines Theater-Habitués. (Raimund-Theater. – Das Märchen.)@\strich\emph{Notizen eines Theater-Habitués. (Raimund-Theater. – Das Märchen.)}|pwk} In: \emph{Wiener Salonblatt}\pwindex{Wiener Salonblatt@\emph{Wiener Salonblatt}|pwk},
                     Jg. 24, Nr. 49, 3. 12. 1893, S. 8–9. Das
                  Thema »Lustspiel« blieb für Schnitzler
                  zeitlebens eine Herausforderung, die er immer wieder erwog, an der er aber auch
                  scheiterte.}}}\label{K_L02723-2} räth, iſt auch auf der richtigen Fährte. Du brauchteſt
               unbedingt ein paar Monate Pariſ\oindex{Paris@\textbf{Paris}, \emph{Hauptstadt}|pw}er Theater; Du
               würdeſt die unermüdliche Anſtrengung des \label{K_L02723-3v}\edtext{jungen Stücks}{\lemma{\textnormal{\emph{jungen Stücks}}}\Cendnote{\textnormal{Siehe dazu etwa Sally Debra Charnow: \emph{Theatre, Politics,
                        and Markets in Fin-de-Siècle Paris. Staging Modernity}.
                     Basingstoke: \emph{Palgrave Macmillan}{ }2005.}}}\label{K_L02723-3}{ }ſehen, objectiv, kurz, natürlich, luſtig zu werden. Das iſt der
               Weg, {\pb}der geradeaus in die Zukunft geht. Das iſt
               auch der Weg Deines Talents. Ein Luſtſpiel, theuerſter Freund, – oder ein Schauſpiel,
               aber ohne Herzensergüſſe! Könnteſt Du Dich nur mit meinen Augen{ }ſehen – Du würdeſt
               keinen Augenblick mehr zögern, und in einem Jahre wäre die Vollendung da, in
               Production wie Erfolg. Bitte{ }ſchreib’ mir ein Wort über Deine Pläne.\pend
           
\pstart
           \textsc{Bahr}\pwindex{Bahr, Hermann 19.\,7.\,1863 Linz – 15.\,1.\,1934 München@\textsc{Bahr, Hermann} (19.\,7.\,1863 Linz – 15.\,1.\,1934 München), \emph{Schriftsteller, Kritiker}|pw} – der kränkt Dich{ }ſo? Er iſt frech, größenwahnſinnig, unausſtehlich doctrinär.
                  \strikeout{E} Der \label{K_L02723-4v}\edtext{Verweis\pwindex{Bahr, Hermann 19.\,7.\,1863 Linz – 15.\,1.\,1934 München@\textsc{Bahr, Hermann} (19.\,7.\,1863 Linz – 15.\,1.\,1934 München), \emph{Schriftsteller, Kritiker}!Märchen (Schauspiel in drei Aufzügen von Arthur Schnitzler)@\strich\emph{Das Märchen (Schauspiel in drei Aufzügen von Arthur Schnitzler)}|pwv} auf{ }ſeine »Neuen Menſchen\pwindex{Bahr, Hermann 19.\,7.\,1863 Linz – 15.\,1.\,1934 München@\textsc{Bahr, Hermann} (19.\,7.\,1863 Linz – 15.\,1.\,1934 München), \emph{Schriftsteller, Kritiker}!neuen Menschen. Ein Schauspiel@\strich\emph{Die neuen Menschen. Ein Schauspiel}|pw}« iſt eine glatte Gemeinheit}{\lemma{\textnormal{\emph{Verweis … Gemeinheit}}}\Cendnote{\textnormal{Hermann Bahr\pwindex{Bahr, Hermann 19.\,7.\,1863 Linz – 15.\,1.\,1934 München@\textsc{Bahr, Hermann} (19.\,7.\,1863 Linz – 15.\,1.\,1934 München), \emph{Schriftsteller, Kritiker}|pwk}: \emph{Die neuen Menschen. Ein Schauspiel}\pwindex{Bahr, Hermann 19.\,7.\,1863 Linz – 15.\,1.\,1934 München@\textsc{Bahr, Hermann} (19.\,7.\,1863 Linz – 15.\,1.\,1934 München), \emph{Schriftsteller, Kritiker}!neuen Menschen. Ein Schauspiel@\strich\emph{Die neuen Menschen. Ein Schauspiel}|pwk}.
                     Zürich: \emph{Verlags-Magazin (J.
                        Schabelitz)}\orgindex{Verlags-Magazin (J. Schabelitz)@Verlags-Magazin (J. Schabelitz)|pwk}{ }1887. In seiner Rezension\pwindex{Bahr, Hermann 19.\,7.\,1863 Linz – 15.\,1.\,1934 München@\textsc{Bahr, Hermann} (19.\,7.\,1863 Linz – 15.\,1.\,1934 München), \emph{Schriftsteller, Kritiker}!Märchen (Schauspiel in drei Aufzügen von Arthur Schnitzler)@\strich\emph{Das Märchen (Schauspiel in drei Aufzügen von Arthur Schnitzler)}|pwkv} kommt Bahr\pwindex{Bahr, Hermann 19.\,7.\,1863 Linz – 15.\,1.\,1934 München@\textsc{Bahr, Hermann} (19.\,7.\,1863 Linz – 15.\,1.\,1934 München), \emph{Schriftsteller, Kritiker}|pwk} auf die
                  vielen Stücke zu sprechen, die im \emph{Märchen}\pwindex{Schnitzler, Arthur 15.\,5.\,1862 Wien – 21.\,10.\,1931 ebd.@\textsc{Schnitzler, Arthur} (15.\,5.\,1862 Wien – 21.\,10.\,1931 ebd.), \emph{Schriftsteller, Mediziner}!Märchen. Schauspiel in drei Aufzügen@\strich\emph{Das Märchen. Schauspiel in drei Aufzügen}|pwk}
                  anklingen, darunter sein eigenes: »das Stück\pwindex{Schnitzler, Arthur 15.\,5.\,1862 Wien – 21.\,10.\,1931 ebd.@\textsc{Schnitzler, Arthur} (15.\,5.\,1862 Wien – 21.\,10.\,1931 ebd.), \emph{Schriftsteller, Mediziner}!Märchen. Schauspiel in drei Aufzügen@\strich\emph{Das Märchen. Schauspiel in drei Aufzügen}|pwv} jenes Zwistes von Verstand und Gefühl, das auch
                     ich einmal, im Sturme der ersten Jugend, mit meinen ›neuen Menschen\pwindex{Bahr, Hermann 19.\,7.\,1863 Linz – 15.\,1.\,1934 München@\textsc{Bahr, Hermann} (19.\,7.\,1863 Linz – 15.\,1.\,1934 München), \emph{Schriftsteller, Kritiker}!neuen Menschen. Ein Schauspiel@\strich\emph{Die neuen Menschen. Ein Schauspiel}|pw}‹ versuchte.« (Hermann Bahr\pwindex{Bahr, Hermann 19.\,7.\,1863 Linz – 15.\,1.\,1934 München@\textsc{Bahr, Hermann} (19.\,7.\,1863 Linz – 15.\,1.\,1934 München), \emph{Schriftsteller, Kritiker}|pwk}: \emph{Das Märchen (Schauspiel in drei Aufzügen von Arthur
                        Schnitzler. Zum ersten Male aufgeführt am Deutschen Volkstheater den 1.
                        December)}\pwindex{Bahr, Hermann 19.\,7.\,1863 Linz – 15.\,1.\,1934 München@\textsc{Bahr, Hermann} (19.\,7.\,1863 Linz – 15.\,1.\,1934 München), \emph{Schriftsteller, Kritiker}!Märchen (Schauspiel in drei Aufzügen von Arthur Schnitzler)@\strich\emph{Das Märchen (Schauspiel in drei Aufzügen von Arthur Schnitzler)}|pwk}. In: \emph{Deutsche Zeitung}\pwindex{Deutsche Zeitung@\emph{Deutsche Zeitung}|pwk},
                     Jg. 23, Nr. 7879, 2. 12. 1893, Morgen-Ausgabe,
                     S. 1–3, hier S. 2.)}}}\label{K_L02723-4}. Und doch finde ich ihn nicht reſpectlos; und
               doch finde ich, daß {\pb}er manches Richtige{ }ſagt.
               Vielleicht aber fehlt mir auch das richtige Urtheil; ich bin{ }ſo außer Zuſammenhang
               mit den Wien\oindex{Wien@\textbf{Wien}, \emph{Verwaltungsgebiet}|pw}er Verhältniſſen. Heiter iſt nur, wie
               der Burſch\pwindex{Bahr, Hermann 19.\,7.\,1863 Linz – 15.\,1.\,1934 München@\textsc{Bahr, Hermann} (19.\,7.\,1863 Linz – 15.\,1.\,1934 München), \emph{Schriftsteller, Kritiker}|pwv}{ }fran\oindex{Frankreich@\textbf{Frankreich}|pwv}zöſiſche Dinge
                  citirt\textcolor{gray}{.}\label{K_L02723-5v}\edtext{»\begin{otherlanguage}{french}Le grappin\end{otherlanguage}\pwindex{Salandri, Gaston 1856 – 1917@\textsc{Salandri, Gaston} (1856 – 1917), \emph{Schriftsteller}!Le Grappin. Comédie en trois actes@\strich\emph{Le Grappin. Comédie en trois actes}|pw}«}{\lemma{\textnormal{\emph{»Le grappin«}}}\Cendnote{\textnormal{ Der entsprechende Absatz in Bahrs\pwindex{Bahr, Hermann 19.\,7.\,1863 Linz – 15.\,1.\,1934 München@\textsc{Bahr, Hermann} (19.\,7.\,1863 Linz – 15.\,1.\,1934 München), \emph{Schriftsteller, Kritiker}|pwk}{ }Kritik\pwindex{Bahr, Hermann 19.\,7.\,1863 Linz – 15.\,1.\,1934 München@\textsc{Bahr, Hermann} (19.\,7.\,1863 Linz – 15.\,1.\,1934 München), \emph{Schriftsteller, Kritiker}!Märchen (Schauspiel in drei Aufzügen von Arthur Schnitzler)@\strich\emph{Das Märchen (Schauspiel in drei Aufzügen von Arthur Schnitzler)}|pwkv} lautet: »Er konnte die Eifersucht der
                     Vergangenheit am Werke\pwindex{Schnitzler, Arthur 15.\,5.\,1862 Wien – 21.\,10.\,1931 ebd.@\textsc{Schnitzler, Arthur} (15.\,5.\,1862 Wien – 21.\,10.\,1931 ebd.), \emph{Schriftsteller, Mediziner}!Märchen. Schauspiel in drei Aufzügen@\strich\emph{Das Märchen. Schauspiel in drei Aufzügen}|pwv}
                     zeigen; wie etwa Othello\pwindex{\textcolor{red}{\textsuperscript{XXXX indx1}}!Tragedy of Othello, the Moor of Venice@\strich\emph{The Tragedy of Othello, the Moor of Venice}|pw} die Eifersucht in
                     der Gegenwart zeigt: er nahm dann eine Liebe und ließ sie an der Vergangenheit
                     des Mädchens verderben, die allmälig sei es gestanden, sei es verrathen wird;
                     der Schmerz des Mannes zwischen Leidenschaft und Ehre und die Buße der
                     Gefallenen waren da die Kräfte, die die Handlung trieben. Oder er konnte einen
                     Spötter gegen diese Eifersucht zeigen, der sich über sie heben will, aber
                     leidend von ihrem Rechte gezwungen wird; er schrieb dann das Stück\pwindex{Salandri, Gaston 1856 – 1917@\textsc{Salandri, Gaston} (1856 – 1917), \emph{Schriftsteller}!Le Grappin. Comédie en trois actes@\strich\emph{Le Grappin. Comédie en trois actes}|pwv}, das Gaston Salandri\pwindex{Salandri, Gaston 1856 – 1917@\textsc{Salandri, Gaston} (1856 – 1917), \emph{Schriftsteller}|pw} als ›Le
                        Grappin\pwindex{Salandri, Gaston 1856 – 1917@\textsc{Salandri, Gaston} (1856 – 1917), \emph{Schriftsteller}!Le Grappin. Comédie en trois actes@\strich\emph{Le Grappin. Comédie en trois actes}|pw}‹ geschrieben und die Paris\oindex{Paris@\textbf{Paris}, \emph{Hauptstadt}|pw}er Freie Bühne\orgindex{Théâtre Libre@Théâtre Libre|pwv}
                     gespielt hat, die Geschichte des Herrn Jacques Privat\pwindex{Salandri, Gaston 1856 – 1917@\textsc{Salandri, Gaston} (1856 – 1917), \emph{Schriftsteller}!Le Grappin. Comédie en trois actes@\strich\emph{Le Grappin. Comédie en trois actes}|pwv}, der das Vorurtheil verachtet und sich
                     mit seiner Geliebten vermählt, obwohl er weiß, daß sie vor ihm Anderen gehörte
                     und liederlich lebte; da wird gezeigt, daß alle Liebe die Vergangenheit nicht
                     tilgen, nicht verwischen kann, ja, durch die tausend Stiche der Nerven, des
                     Gemüthes und die Kränkungen der Ehre sich in Zorn, Ekel, Haß verwandeln muß.
                     Mit dem ersten Stücke\pwindex{\textcolor{red}{\textsuperscript{XXXX indx1}}!Tragedy of Othello, the Moor of Venice@\strich\emph{The Tragedy of Othello, the Moor of Venice}|pwv}
                     geht der Hörer, auch wenn er diese Eifersucht nicht hat, weil er sich doch aus
                     Anderen in sie denken kann. Mit dem zweiten\pwindex{Salandri, Gaston 1856 – 1917@\textsc{Salandri, Gaston} (1856 – 1917), \emph{Schriftsteller}!Le Grappin. Comédie en trois actes@\strich\emph{Le Grappin. Comédie en trois actes}|pwv} kann er gegen das Vorurtheil, das ja von dem
                     Helden bestritten, und er kann für das Vorurtheil mit ihm gehen, das doch
                     schließlich bestätigt wird. Er ist Beiden\pwindex{\textcolor{red}{\textsuperscript{XXXX indx1}}!Tragedy of Othello, the Moor of Venice@\strich\emph{The Tragedy of Othello, the Moor of Venice}|pwv}\pwindex{Salandri, Gaston 1856 – 1917@\textsc{Salandri, Gaston} (1856 – 1917), \emph{Schriftsteller}!Le Grappin. Comédie en trois actes@\strich\emph{Le Grappin. Comédie en trois actes}|pwv} empfänglich.«
                     (S. 1.)}}}\label{K_L02723-5}, das Théâtre-Libre\orgindex{Théâtre Libre@Théâtre Libre|pw}-Stück\pwindex{Salandri, Gaston 1856 – 1917@\textsc{Salandri, Gaston} (1856 – 1917), \emph{Schriftsteller}!Le Grappin. Comédie en trois actes@\strich\emph{Le Grappin. Comédie en trois actes}|pwv}, von dem er{ }ſpricht,
               behandelt etwas abſolut Anderes als das, was er behauptet. Ein frecher Schwindel, um{ }ſich in allen Sätteln moderner \introOben{}fran\oindex{Frankreich@\textbf{Frankreich}|pwv}zöſiſcher\introOben{}
               Literatur gerecht zu zeigen.\pend
           
\pstart
           \label{K_L02723-6v}\edtext{\textsc{Granichstaedten\pwindex{Granichstaedten, Emil 8.\,7.\,1847 Wien – 2.\,7.\,1904 Berlin@\textsc{Granichstaedten, Emil} (8.\,7.\,1847 Wien – 2.\,7.\,1904 Berlin), \emph{Journalist, Rechtswissenschaftler}|pw}} hätte ich an Deiner Stelle geohrfeigt. Das iſt keine Kritik\pwindex{Granichstaedten, Emil 8.\,7.\,1847 Wien – 2.\,7.\,1904 Berlin@\textsc{Granichstaedten, Emil} (8.\,7.\,1847 Wien – 2.\,7.\,1904 Berlin), \emph{Journalist, Rechtswissenschaftler}!Theater- und Kunstnachrichten [Uraufführung Das Märchen]@\strich\emph{Theater- und Kunstnachrichten [Uraufführung Das Märchen]}|pwuv}\pwindex{Granichstaedten, Emil 8.\,7.\,1847 Wien – 2.\,7.\,1904 Berlin@\textsc{Granichstaedten, Emil} (8.\,7.\,1847 Wien – 2.\,7.\,1904 Berlin), \emph{Journalist, Rechtswissenschaftler}!Feuilleton. Deutsches Volkstheater [Märchen]@\strich\emph{Feuilleton. Deutsches Volkstheater [Märchen]}|pwuv}}{\lemma{\textnormal{\emph{Granichstaedten … Kritik}}}\Cendnote{\textnormal{Emil Granichstaedten\pwindex{Granichstaedten, Emil 8.\,7.\,1847 Wien – 2.\,7.\,1904 Berlin@\textsc{Granichstaedten, Emil} (8.\,7.\,1847 Wien – 2.\,7.\,1904 Berlin), \emph{Journalist, Rechtswissenschaftler}|pwk} verfasste eine Nachtkritik\pwindex{Granichstaedten, Emil 8.\,7.\,1847 Wien – 2.\,7.\,1904 Berlin@\textsc{Granichstaedten, Emil} (8.\,7.\,1847 Wien – 2.\,7.\,1904 Berlin), \emph{Journalist, Rechtswissenschaftler}!Theater- und Kunstnachrichten [Uraufführung Das Märchen]@\strich\emph{Theater- und Kunstnachrichten [Uraufführung Das Märchen]}|pwkv} (g.\pwindex{Granichstaedten, Emil 8.\,7.\,1847 Wien – 2.\,7.\,1904 Berlin@\textsc{Granichstaedten, Emil} (8.\,7.\,1847 Wien – 2.\,7.\,1904 Berlin), \emph{Journalist, Rechtswissenschaftler}|pwkv}: \emph{Theater- und Kunstnachrichten}\pwindex{Granichstaedten, Emil 8.\,7.\,1847 Wien – 2.\,7.\,1904 Berlin@\textsc{Granichstaedten, Emil} (8.\,7.\,1847 Wien – 2.\,7.\,1904 Berlin), \emph{Journalist, Rechtswissenschaftler}!Theater- und Kunstnachrichten [Uraufführung Das Märchen]@\strich\emph{Theater- und Kunstnachrichten [Uraufführung Das Märchen]}|pwk}. In: \emph{Die Presse}\pwindex{Presse@\emph{Die Presse}|pwk}, Jg. 46, Nr. 333, 2. 12. 1893, S. 11) und am Folgetag ein Feuilleton\pwindex{Granichstaedten, Emil 8.\,7.\,1847 Wien – 2.\,7.\,1904 Berlin@\textsc{Granichstaedten, Emil} (8.\,7.\,1847 Wien – 2.\,7.\,1904 Berlin), \emph{Journalist, Rechtswissenschaftler}!Feuilleton. Deutsches Volkstheater [Märchen]@\strich\emph{Feuilleton. Deutsches Volkstheater [Märchen]}|pwkv} (Emil Granichstaedten\pwindex{Granichstaedten, Emil 8.\,7.\,1847 Wien – 2.\,7.\,1904 Berlin@\textsc{Granichstaedten, Emil} (8.\,7.\,1847 Wien – 2.\,7.\,1904 Berlin), \emph{Journalist, Rechtswissenschaftler}|pwk}: \emph{Feuilleton. Deutsches Volkstheater}\pwindex{Granichstaedten, Emil 8.\,7.\,1847 Wien – 2.\,7.\,1904 Berlin@\textsc{Granichstaedten, Emil} (8.\,7.\,1847 Wien – 2.\,7.\,1904 Berlin), \emph{Journalist, Rechtswissenschaftler}!Feuilleton. Deutsches Volkstheater [Märchen]@\strich\emph{Feuilleton. Deutsches Volkstheater [Märchen]}|pwk}. In: \emph{Die Presse}\pwindex{Presse@\emph{Die Presse}|pwk}, Jg. 46, Nr. 334, 3. 12. 1893, S. 1–2). Auch Schnitzler war über die Nachtkritik\pwindex{Granichstaedten, Emil 8.\,7.\,1847 Wien – 2.\,7.\,1904 Berlin@\textsc{Granichstaedten, Emil} (8.\,7.\,1847 Wien – 2.\,7.\,1904 Berlin), \emph{Journalist, Rechtswissenschaftler}!Theater- und Kunstnachrichten [Uraufführung Das Märchen]@\strich\emph{Theater- und Kunstnachrichten [Uraufführung Das Märchen]}|pwkv} verärgert und bezeichnete sie im \emph{Tagebuch}\pwindex{Schnitzler, Arthur 15.\,5.\,1862 Wien – 21.\,10.\,1931 ebd.@\textsc{Schnitzler, Arthur} (15.\,5.\,1862 Wien – 21.\,10.\,1931 ebd.), \emph{Schriftsteller, Mediziner}!Tagebuch@\strich\emph{Tagebuch}|pwk} als »[p]erfid dumm« (2. 12. 1893). Granichstaedten\pwindex{Granichstaedten, Emil 8.\,7.\,1847 Wien – 2.\,7.\,1904 Berlin@\textsc{Granichstaedten, Emil} (8.\,7.\,1847 Wien – 2.\,7.\,1904 Berlin), \emph{Journalist, Rechtswissenschaftler}|pwk} lobte die Schauspielkunst Adele Sandrocks\pwindex{Sandrock, Adele 19.\,8.\,1863 Rotterdam – 30.\,8.\,1937 Berlin@\textsc{Sandrock, Adele} (19.\,8.\,1863 Rotterdam – 30.\,8.\,1937 Berlin), \emph{Schauspielerin}|pwk}, spielte aber auf sexuelle
                  Aspekte im \emph{Märchen}\pwindex{Schnitzler, Arthur 15.\,5.\,1862 Wien – 21.\,10.\,1931 ebd.@\textsc{Schnitzler, Arthur} (15.\,5.\,1862 Wien – 21.\,10.\,1931 ebd.), \emph{Schriftsteller, Mediziner}!Märchen. Schauspiel in drei Aufzügen@\strich\emph{Das Märchen. Schauspiel in drei Aufzügen}|pwk} recht abschätzig an.
                  Zwischen den Zeilen kritisierte er die Handlung\pwindex{Schnitzler, Arthur 15.\,5.\,1862 Wien – 21.\,10.\,1931 ebd.@\textsc{Schnitzler, Arthur} (15.\,5.\,1862 Wien – 21.\,10.\,1931 ebd.), \emph{Schriftsteller, Mediziner}!Märchen. Schauspiel in drei Aufzügen@\strich\emph{Das Märchen. Schauspiel in drei Aufzügen}|pwkv} an sich und die Figuren des Fedor\pwindex{Schnitzler, Arthur 15.\,5.\,1862 Wien – 21.\,10.\,1931 ebd.@\textsc{Schnitzler, Arthur} (15.\,5.\,1862 Wien – 21.\,10.\,1931 ebd.), \emph{Schriftsteller, Mediziner}!Märchen. Schauspiel in drei Aufzügen@\strich\emph{Das Märchen. Schauspiel in drei Aufzügen}|pwkv} und der Fanny\pwindex{Schnitzler, Arthur 15.\,5.\,1862 Wien – 21.\,10.\,1931 ebd.@\textsc{Schnitzler, Arthur} (15.\,5.\,1862 Wien – 21.\,10.\,1931 ebd.), \emph{Schriftsteller, Mediziner}!Märchen. Schauspiel in drei Aufzügen@\strich\emph{Das Märchen. Schauspiel in drei Aufzügen}|pwkv}. Am 3. 12. 1893 positionierte Granichstaedten\pwindex{Granichstaedten, Emil 8.\,7.\,1847 Wien – 2.\,7.\,1904 Berlin@\textsc{Granichstaedten, Emil} (8.\,7.\,1847 Wien – 2.\,7.\,1904 Berlin), \emph{Journalist, Rechtswissenschaftler}|pwk} sich auf der Seite des Naturalismus und holte weiter aus.
                  Angefangen beim »Pessimismus unserer ›Wien\oindex{Wien@\textbf{Wien}, \emph{Verwaltungsgebiet}|pw}er Modernen‹« (S. 1) kritisierte er auf
                  abwertende Weise ganz grundsätzlich das junge Werk Schnitzlers und bezog sich auch auf den \emph{Anatol}\pwindex{Schnitzler, Arthur 15.\,5.\,1862 Wien – 21.\,10.\,1931 ebd.@\textsc{Schnitzler, Arthur} (15.\,5.\,1862 Wien – 21.\,10.\,1931 ebd.), \emph{Schriftsteller, Mediziner}!Anatol@\strich\emph{Anatol}|pwk}-Zyklus. Der Autor orientiere sich zu stark an »modernen«, französischen
                  Strömungen, was ihm jedoch nicht gelinge: »Für dieſen Fedor\pwindex{Schnitzler, Arthur 15.\,5.\,1862 Wien – 21.\,10.\,1931 ebd.@\textsc{Schnitzler, Arthur} (15.\,5.\,1862 Wien – 21.\,10.\,1931 ebd.), \emph{Schriftsteller, Mediziner}!Märchen. Schauspiel in drei Aufzügen@\strich\emph{Das Märchen. Schauspiel in drei Aufzügen}|pwv} und dieſe Fanny\pwindex{Schnitzler, Arthur 15.\,5.\,1862 Wien – 21.\,10.\,1931 ebd.@\textsc{Schnitzler, Arthur} (15.\,5.\,1862 Wien – 21.\,10.\,1931 ebd.), \emph{Schriftsteller, Mediziner}!Märchen. Schauspiel in drei Aufzügen@\strich\emph{Das Märchen. Schauspiel in drei Aufzügen}|pwv} kann kein Publikum der Welt
                     sich interessieren.« (S. 2) \emph{Das Märchen}\pwindex{Schnitzler, Arthur 15.\,5.\,1862 Wien – 21.\,10.\,1931 ebd.@\textsc{Schnitzler, Arthur} (15.\,5.\,1862 Wien – 21.\,10.\,1931 ebd.), \emph{Schriftsteller, Mediziner}!Märchen. Schauspiel in drei Aufzügen@\strich\emph{Das Märchen. Schauspiel in drei Aufzügen}|pwk} sei »nicht tugendhaft« und
                     »[u]m Reinlichkeit wird gebeten« (S. 2).}}}\label{K_L02723-6},{ }ſondern ein Gaſſenbubenſtreich.\pend
           
\pstart
           Freut mich, daß Du nicht {\pb}verbittert biſt. Das
               gehört{ }ſich auch{ }ſo. Ich meine, Du kannſt mit Deinem Debüt\pwindex{Schnitzler, Arthur 15.\,5.\,1862 Wien – 21.\,10.\,1931 ebd.@\textsc{Schnitzler, Arthur} (15.\,5.\,1862 Wien – 21.\,10.\,1931 ebd.), \emph{Schriftsteller, Mediziner}!Märchen. Schauspiel in drei Aufzügen@\strich\emph{Das Märchen. Schauspiel in drei Aufzügen}|pwv}{ }ſehr zufrieden{ }ſein. Man gibt Dir
               Credit, und das iſt enorm für einen Jungen.\pend
           
\pstart
           Haſt Du \label{K_L02723-7v}\edtext{\textsc{Loris\pwindex{Hofmannsthal, Hugo von 1.\,2.\,1874 Wien – 15.\,7.\,1929 Rodaun@\textsc{Hofmannsthal, Hugo von} (1.\,2.\,1874 Wien – 15.\,7.\,1929 Rodaun), \emph{Schriftsteller}|pw}} über \textsc{Bauernfeld\pwindex{Bauernfeld, Eduard von 13.\,1.\,1802 Wien – 4.\,8.\,1890 ebd.@\textsc{Bauernfeld, Eduard von} (13.\,1.\,1802 Wien – 4.\,8.\,1890 ebd.)|pw}}\pwindex{Hofmannsthal, Hugo von 1.\,2.\,1874 Wien – 15.\,7.\,1929 Rodaun@\textsc{Hofmannsthal, Hugo von} (1.\,2.\,1874 Wien – 15.\,7.\,1929 Rodaun), \emph{Schriftsteller}!Eduard von Bauernfeld’s dramatischer Nachlaß@\strich\emph{Eduard von Bauernfeld’s dramatischer Nachlaß}|pwv}}{\lemma{\textnormal{\emph{Loris über Bauernfeld}}}\Cendnote{\textnormal{Loris\pwindex{Hofmannsthal, Hugo von 1.\,2.\,1874 Wien – 15.\,7.\,1929 Rodaun@\textsc{Hofmannsthal, Hugo von} (1.\,2.\,1874 Wien – 15.\,7.\,1929 Rodaun), \emph{Schriftsteller}|pwk}: \emph{Eduard von Bauernfeld’s dramatischer Nachlaß}\pwindex{Hofmannsthal, Hugo von 1.\,2.\,1874 Wien – 15.\,7.\,1929 Rodaun@\textsc{Hofmannsthal, Hugo von} (1.\,2.\,1874 Wien – 15.\,7.\,1929 Rodaun), \emph{Schriftsteller}!Eduard von Bauernfeld’s dramatischer Nachlaß@\strich\emph{Eduard von Bauernfeld’s dramatischer Nachlaß}|pwk}. In: \emph{Frankfurter Zeitung}\pwindex{Frankfurter Zeitung@\emph{Frankfurter Zeitung}|pwk}, Jg. 38, Nr. 338, 6. 12. 1893, Erstes Morgenblatt,
                  S. 1.}}}\label{K_L02723-7} geleſen? Wie aus dieſem gottbegnadeten Menſchen\pwindex{Hofmannsthal, Hugo von 1.\,2.\,1874 Wien – 15.\,7.\,1929 Rodaun@\textsc{Hofmannsthal, Hugo von} (1.\,2.\,1874 Wien – 15.\,7.\,1929 Rodaun), \emph{Schriftsteller}|pwv} die entzückenden Dinge
               herausquellen,{ }ſo leicht und{ }ſprudelnd. Ein Dichter\pwindex{Hofmannsthal, Hugo von 1.\,2.\,1874 Wien – 15.\,7.\,1929 Rodaun@\textsc{Hofmannsthal, Hugo von} (1.\,2.\,1874 Wien – 15.\,7.\,1929 Rodaun), \emph{Schriftsteller}|pwv}! Derjenige vielleicht, den man{ }ſeit fünfzig Jahren
               erwartet!\pend
           
\pstart
           Grüß’ ihn von mir, denn ich habe \label{K_L02723-8v}\edtext{keine directe Verbindung mehr}{\lemma{\textnormal{\emph{keine … mehr}}}\Cendnote{\textnormal{Im
                  Nachlass Hofmannsthals\pwindex{Hofmannsthal, Hugo von 1.\,2.\,1874 Wien – 15.\,7.\,1929 Rodaun@\textsc{Hofmannsthal, Hugo von} (1.\,2.\,1874 Wien – 15.\,7.\,1929 Rodaun), \emph{Schriftsteller}|pwk} sind keine
                  Korrespondenzstücke Goldmanns\pwindex{Goldmann, Paul 31.\,1.\,1865 Breslau – 25.\,9.\,1935 Wien@\textsc{Goldmann, Paul} (31.\,1.\,1865 Breslau – 25.\,9.\,1935 Wien), \emph{Schriftsteller, Journalist}|pwk} überliefert.
                  Unter den Briefen Beer-Hofmanns\pwindex{Beer-Hofmann, Richard 11.\,7.\,1866 Wien – 26.\,9.\,1945 New York City@\textsc{Beer-Hofmann, Richard} (11.\,7.\,1866 Wien – 26.\,9.\,1945 New York City), \emph{Schriftsteller}|pwk} in der 
                     \emph{Houghton Library}\orgindex{Houghton Library@Houghton Library|pwk} dürften keine
                     Korrespondenzstücke aus dem Zeitraum Sommer 1893–1895
                     erhalten sein, wobei viele Briefe ohne Jahresangabe sind und eine genauere
                     Zuordnung notwendig wäre, um die Behauptung mit letzter Sicherheit treffen zu
                     können.
               }}}\label{K_L02723-8} mit ihm; grüße auch \textsc{Richard\pwindex{Beer-Hofmann, Richard 11.\,7.\,1866 Wien – 26.\,9.\,1945 New York City@\textsc{Beer-Hofmann, Richard} (11.\,7.\,1866 Wien – 26.\,9.\,1945 New York City), \emph{Schriftsteller}|pw}} aus{ }ſelbigem Grunde;{ }ſei{ }ſelbſt herzlichſt gegrüßt und{ }ſchreibe bald!\pend
           
\pstart
           Dein{\\[\baselineskip]}\spacefill\mbox{Paul Goldmn}\pend
           \leftskip=0em{}\selectlanguage{ngerman}\endnumbering\briefempfaengerindex{Schnitzler, Arthur@\textsc{Schnitzler, Arthur}!zzzGoldmann, Paul@\emph{von Paul Goldmann}!1893-12-081@{8. 12. [1893]}|)be}\mylabel{L02723h}  \newcommand{\dateiname}{L02723}\newcommand{\titel}{Paul Goldmann an Arthur Schnitzler, 8. 12. [1893]}\newcommand{\editorInnen}{Martin Anton Müller und Laura Untner}%% latex-leseansicht-abspann.tex
%% Abspann für die Leseansicht.
%% Der Schalter \ifkorrekturansicht ist bereits durch den Vorspann gesetzt.

%% latex-abspann.tex
%% Gemeinsamer Abspann für Korrekturansicht und Leseansicht.
%% Setzt den Schalter \ifkorrekturansicht voraus (gesetzt in den
%% einbindenden Dateien latex-korrekturansicht-abspann.tex bzw.
%% latex-leseansicht-abspann.tex).
%% ---------------------------------------------------------------

\normalsize

% Das esempio-Environment wird nur in der Leseansicht benötigt
\ifkorrekturansicht\else
\newenvironment{esempio}[3]%
{
    \vspace{1.5ex}
    \rlap{\underline{#1}}
    \par
    \setlength{\parindent}{0cm}
    \nopagebreak
    \leftskip=#2cm
    \rightskip=#3cm
}
{
    \par
}
\fi

\doendnotes{C}
\bigskip
\vfill

\clearpage

\footnotesize

\ifkorrekturansicht
  \lohead{\textsc{register}}
\fi

% theindex-Environment neu definieren ohne reledmac
\makeatletter
\renewenvironment{theindex}{%
  \ifkorrekturansicht
    \section*{\indexname}%
  \else
    \subsubsection*{Index der erwähnten Entitäten}%
  \fi
  \setlength{\parindent}{0pt}%
  \setlength{\parskip}{0pt plus 0.3pt}%
  \let\item\@idxitem
}{%
  \ifkorrekturansicht\clearpage\fi
}
\makeatother

\IfFileExists{\jobname-pw.ind}{\input{\jobname-pw.ind}}{}

% Quellenangabe nur in der Leseansicht
\ifkorrekturansicht\else
% Fallback-Definitionen, falls die .tex-Datei \titel etc. nicht gesetzt hat
\providecommand{\titel}{}
\providecommand{\editorInnen}{}
\providecommand{\dateiname}{\jobname}

\vspace{3cm}

\vfill

\footnotesize
\textsc{Quelle}: \titel. Herausgegeben von {\editorInnen}. In: \emph{Arthur Schnitzler: Briefwechsel mit Autorinnen und Autoren}.
 Digitale Edition, https://schnitzler-briefe.acdh.oeaw.ac.at/{\dateiname}.html (Stand \today)
\fi

\end{document}


