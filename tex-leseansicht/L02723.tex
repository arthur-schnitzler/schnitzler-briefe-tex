%% latex-korrekturansicht-vorspann.tex
%% Vorspann für die Korrekturansicht.
%% Lädt die gemeinsame Datei latex-vorspann.tex mit gesetztem Schalter.

\newif\ifkorrekturansicht
\korrekturansichttrue

\input{../tex-inputs/latex-vorspann}


\section[Paul Goldmann an Arthur Schnitzler, 8. 12. {[}1893{]}]{L02723 Paul Goldmann an Arthur Schnitzler, 8. 12. {[}1893{]}}
\nopagebreak\mylabel{L02723v}
\rehead{ }\normalsize\beginnumbering\briefempfaengerindex{Schnitzler, Arthur@\textsc{Schnitzler, Arthur}!zzzGoldmann, Paul@\emph{von Paul Goldmann}!1893-12-081@{8. 12. {[}1893{]}}|(be}
\toendnotes[C]{\smallbreak\pagebreak[2]}\Standort{DLA, A:Schnitzler, HS.NZ85.1.3163.}
\physDesc{Brief, 1 Blatt, 4 Seiten, 2096 Zeichen
\newline{}Handschrift: schwarze Tinte, deutsche Kurrent
\newline{}Schnitzler: 1) mit Bleistift das Jahr »93« vermerkt  2) mit rotem Buntstift drei Unterstreichungen sowie ein Pfeil, der
                                 den ganzen Absatz zu Hofmannsthal\pwindex{Hofmannsthal, Hugo von 1874-02-01 – 1929-07-15@\textsc{Hofmannsthal, Hugo von} (1874-02-01 – 1929-07-15), \emph{Schriftsteller/Schriftstellerin}|pw} markieren soll}\toendnotes[C]{\smallbreak}
\pstart
           {\pb}\textcolor{gray}{\textbf{\textbf{Frankfurter Zeitung\orgindex{Frankfurter Zeitung@Frankfurter Zeitung|pw}.}}}\pend
           
\pstart
           \textcolor{gray}{\textbf{\textbf{(\begin{otherlanguage}{french}Gazette de Francfort\end{otherlanguage}\orgindex{Frankfurter Zeitung@Frankfurter Zeitung|pw}.)}}}\pend
           
\pstart
           \textcolor{gray}{\textbf{\begin{otherlanguage}{french}Directeur\end{otherlanguage}{ }\textbf{M. L. Sonnemann\pwindex{Sonnemann, Leopold 1831-10-29 – 1909-10-30@\textsc{Sonnemann, Leopold} (1831-10-29 – 1909-10-30), \emph{Journalist/Journalistin, Herausgeber/Herausgeberin}|pw}.}}}\hfill \textsc{Paris\oindex{Paris@\textbf{Paris}, \emph{P.PPLC}|pw}}, 8. December.\pend
           
\pstart
           \begin{otherlanguage}{french}\textcolor{gray}{\textbf{Journal politique, financier,}}\end{otherlanguage}\pend
           
\pstart
           \begin{otherlanguage}{french}\textcolor{gray}{\textbf{commercial et litteraire.}}\end{otherlanguage}\pend
           
\pstart
           \begin{otherlanguage}{french}\textcolor{gray}{\textbf{\textbf{Paraissant trois fois par jour}}}\end{otherlanguage}\pend
           
\pstart
           \begin{otherlanguage}{french}\textcolor{gray}{\textbf{\textbf{Bureaux à Paris\oindex{Paris@\textbf{Paris}, \emph{P.PPLC}|pw}:}}}\end{otherlanguage}\pend
           
\pstart
           \begin{otherlanguage}{french}\textcolor{gray}{\textbf{\textbf{rue Richelieu 75\oindex{rue Richelieu@\textbf{rue Richelieu}, \emph{Straße (K.STR)}|pw}.}}}\end{otherlanguage}\pend
           
\pstart\center{}Mein lieber Freund!\pend\vspace{0.5em}
\pstart
           Dank für die \label{K_L02723-1v}\edtext{Kritiken}{\lemma{\textnormal{\emph{Kritiken}}}\Cendnote{\textnormal{Zu den ihm bekannten Kritiken vgl. Paul Goldmann an Arthur Schnitzler, 5. 12. [1893].
               }}}\label{K_L02723-1}; ich kannte ſie größtentheils ſchon. Drei oder vier verſtehen Dich oder geben
               ſich wenigſtens ehrliche Mühe, Dich zu verſtehen. Der \label{K_L02723-2v}\edtext{kleine \textsc{Salonblatt\orgindex{Wiener Salonblatt@Wiener Salonblatt|pw}}-Mann\pwindex{Willner, Alfred Maria 11.07.1859 – 27.10.1929@\textsc{Willner, Alfred Maria} (11.07.1859 – 27.10.1929), \emph{Schriftsteller/Schriftstellerin, Journalist/Journalistin}|pwv}, der \strikeout{Dich} Dir zum Luſtſpiel}{\lemma{\textnormal{\emph{kleine … Luſtſpiel}}}\Cendnote{\textnormal{A. M. W.\pwindex{Willner, Alfred Maria 11.07.1859 – 27.10.1929@\textsc{Willner, Alfred Maria} (11.07.1859 – 27.10.1929), \emph{Schriftsteller/Schriftstellerin, Journalist/Journalistin}|pwkv} [ = Alfred Maria Willner\pwindex{Willner, Alfred Maria 11.07.1859 – 27.10.1929@\textsc{Willner, Alfred Maria} (11.07.1859 – 27.10.1929), \emph{Schriftsteller/Schriftstellerin, Journalist/Journalistin}|pwk}]: \emph{Notizen eines Theater-Habitués. (Raimund-Theater. – Das
                        Märchen.)}\pwindex{Notizen eines Theater-Habitues. (Raimund-Theater. – Das Maerchen.)@\emph{Notizen eines Theater-Habitués. (Raimund-Theater. – Das Märchen.)}|pwk} In: \emph{Wiener Salonblatt}\pwindex{Wiener Salonblatt@\emph{Wiener Salonblatt}|pwk},
                     Jg. 24, Nr. 49, 3. 12. 1893, S. 8–9. Das
                  Thema »Lustspiel« blieb für Schnitzler
                  zeitlebens eine Herausforderung, die er immer wieder erwog, an der er aber auch
                  scheiterte.}}}\label{K_L02723-2} räth, iſt auch auf der richtigen Fährte. Du brauchteſt
               unbedingt ein paar Monate Pariſ\oindex{Paris@\textbf{Paris}, \emph{P.PPLC}|pw}er Theater; Du
               würdeſt die unermüdliche Anſtrengung des \label{K_L02723-3v}\edtext{jungen Stücks}{\lemma{\textnormal{\emph{jungen Stücks}}}\Cendnote{\textnormal{Siehe dazu etwa Sally Debra Charnow: \emph{Theatre, Politics,
                        and Markets in Fin-de-Siècle Paris. Staging Modernity}.
                     Basingstoke: \emph{Palgrave Macmillan}{ }2005.}}}\label{K_L02723-3} ſehen, objectiv, kurz, natürlich, luſtig zu werden. Das iſt der
               Weg, {\pb}der geradeaus in die Zukunft geht. Das iſt
               auch der Weg Deines Talents. Ein Luſtſpiel, theuerſter Freund, – oder ein Schauſpiel,
               aber ohne Herzensergüſſe! Könnteſt Du Dich nur mit meinen Augen ſehen – Du würdeſt
               keinen Augenblick mehr zögern, und in einem Jahre wäre die Vollendung da, in
               Production wie Erfolg. Bitte ſchreib’ mir ein Wort über Deine Pläne.\pend
           
\pstart
           \textsc{Bahr}\pwindex{Bahr, Hermann 19.07.1863 – 15.01.1934@\textsc{Bahr, Hermann} (19.07.1863 – 15.01.1934), \emph{Schriftsteller/Schriftstellerin, Kritiker/Kritikerin}|pw} – der kränkt Dich ſo? Er iſt frech, größenwahnſinnig, unausſtehlich doctrinär.
                  \strikeout{E} Der \label{K_L02723-4v}\edtext{Verweis\pwindex{Maerchen (Schauspiel in drei Aufzuegen von Arthur Schnitzler)@\emph{Das Märchen (Schauspiel in drei Aufzügen von Arthur Schnitzler)}|pwv} auf ſeine »Neuen Menſchen\pwindex{neuen Menschen. Ein Schauspiel@\emph{Die neuen Menschen. Ein Schauspiel}|pw}« iſt eine glatte Gemeinheit}{\lemma{\textnormal{\emph{Verweis … Gemeinheit}}}\Cendnote{\textnormal{Hermann Bahr\pwindex{Bahr, Hermann 19.07.1863 – 15.01.1934@\textsc{Bahr, Hermann} (19.07.1863 – 15.01.1934), \emph{Schriftsteller/Schriftstellerin, Kritiker/Kritikerin}|pwk}: \emph{Die neuen Menschen. Ein Schauspiel}\pwindex{neuen Menschen. Ein Schauspiel@\emph{Die neuen Menschen. Ein Schauspiel}|pwk}.
                     Zürich: \emph{Verlags-Magazin (J.
                        Schabelitz)}\orgindex{Verlags-Magazin (J. Schabelitz)@Verlags-Magazin (J. Schabelitz)|pwk}{ }1887. In seiner Rezension\pwindex{Maerchen (Schauspiel in drei Aufzuegen von Arthur Schnitzler)@\emph{Das Märchen (Schauspiel in drei Aufzügen von Arthur Schnitzler)}|pwkv} kommt Bahr\pwindex{Bahr, Hermann 19.07.1863 – 15.01.1934@\textsc{Bahr, Hermann} (19.07.1863 – 15.01.1934), \emph{Schriftsteller/Schriftstellerin, Kritiker/Kritikerin}|pwk} auf die
                  vielen Stücke zu sprechen, die im \emph{Märchen}\pwindex{Maerchen. Schauspiel in drei Aufzuegen@\emph{Das Märchen. Schauspiel in drei Aufzügen}|pwk}
                  anklingen, darunter sein eigenes: »das Stück\pwindex{Maerchen. Schauspiel in drei Aufzuegen@\emph{Das Märchen. Schauspiel in drei Aufzügen}|pwv} jenes Zwistes von Verstand und Gefühl, das auch
                     ich einmal, im Sturme der ersten Jugend, mit meinen ›neuen Menschen\pwindex{neuen Menschen. Ein Schauspiel@\emph{Die neuen Menschen. Ein Schauspiel}|pw}‹ versuchte.« (Hermann Bahr\pwindex{Bahr, Hermann 19.07.1863 – 15.01.1934@\textsc{Bahr, Hermann} (19.07.1863 – 15.01.1934), \emph{Schriftsteller/Schriftstellerin, Kritiker/Kritikerin}|pwk}: \emph{Das Märchen (Schauspiel in drei Aufzügen von Arthur
                        Schnitzler. Zum ersten Male aufgeführt am Deutschen Volkstheater den 1.
                        December)}\pwindex{Maerchen (Schauspiel in drei Aufzuegen von Arthur Schnitzler)@\emph{Das Märchen (Schauspiel in drei Aufzügen von Arthur Schnitzler)}|pwk}. In: \emph{Deutsche Zeitung}\pwindex{Deutsche Zeitung@\emph{Deutsche Zeitung}|pwk},
                     Jg. 23, Nr. 7879, 2. 12. 1893, Morgen-Ausgabe,
                     S. 1–3, hier S. 2.)}}}\label{K_L02723-4}. Und doch finde ich ihn nicht reſpectlos; und
               doch finde ich, daß {\pb}er manches Richtige ſagt.
               Vielleicht aber fehlt mir auch das richtige Urtheil; ich bin ſo außer Zuſammenhang
               mit den Wien\oindex{Wien@\textbf{Wien}, \emph{A.ADM2}|pw}er Verhältniſſen. Heiter iſt nur, wie
               der Burſch\pwindex{Bahr, Hermann 19.07.1863 – 15.01.1934@\textsc{Bahr, Hermann} (19.07.1863 – 15.01.1934), \emph{Schriftsteller/Schriftstellerin, Kritiker/Kritikerin}|pwv}{ }fran\oindex{Frankreich@\textbf{Frankreich}, \emph{A.PCLI}|pwv}zöſiſche Dinge
                  citirt\textcolor{gray}{.}\label{K_L02723-5v}\edtext{»\begin{otherlanguage}{french}Le grappin\end{otherlanguage}\pwindex{Le Grappin. Comedie en trois actes@\emph{Le Grappin. Comédie en trois actes}|pw}«}{\lemma{\textnormal{\emph{»Le grappin«}}}\Cendnote{\textnormal{ Der entsprechende Absatz in Bahrs\pwindex{Bahr, Hermann 19.07.1863 – 15.01.1934@\textsc{Bahr, Hermann} (19.07.1863 – 15.01.1934), \emph{Schriftsteller/Schriftstellerin, Kritiker/Kritikerin}|pwk}{ }Kritik\pwindex{Maerchen (Schauspiel in drei Aufzuegen von Arthur Schnitzler)@\emph{Das Märchen (Schauspiel in drei Aufzügen von Arthur Schnitzler)}|pwkv} lautet: »Er konnte die Eifersucht der
                     Vergangenheit am Werke\pwindex{Maerchen. Schauspiel in drei Aufzuegen@\emph{Das Märchen. Schauspiel in drei Aufzügen}|pwv}
                     zeigen; wie etwa Othello\pwindex{Tragedy of Othello, the Moor of Venice@\emph{The Tragedy of Othello, the Moor of Venice}|pw} die Eifersucht in
                     der Gegenwart zeigt: er nahm dann eine Liebe und ließ sie an der Vergangenheit
                     des Mädchens verderben, die allmälig sei es gestanden, sei es verrathen wird;
                     der Schmerz des Mannes zwischen Leidenschaft und Ehre und die Buße der
                     Gefallenen waren da die Kräfte, die die Handlung trieben. Oder er konnte einen
                     Spötter gegen diese Eifersucht zeigen, der sich über sie heben will, aber
                     leidend von ihrem Rechte gezwungen wird; er schrieb dann das Stück\pwindex{Le Grappin. Comedie en trois actes@\emph{Le Grappin. Comédie en trois actes}|pwv}, das Gaston Salandri\pwindex{Salandri, Gaston 1856 – 1917@\textsc{Salandri, Gaston} (1856 – 1917), \emph{Schriftsteller/Schriftstellerin}|pw} als ›Le
                        Grappin\pwindex{Le Grappin. Comedie en trois actes@\emph{Le Grappin. Comédie en trois actes}|pw}‹ geschrieben und die Paris\oindex{Paris@\textbf{Paris}, \emph{P.PPLC}|pw}er Freie Bühne\orgindex{Theâtre Libre@Théâtre Libre|pwv}
                     gespielt hat, die Geschichte des Herrn Jacques Privat\pwindex{Le Grappin. Comedie en trois actes@\emph{Le Grappin. Comédie en trois actes}|pwv}, der das Vorurtheil verachtet und sich
                     mit seiner Geliebten vermählt, obwohl er weiß, daß sie vor ihm Anderen gehörte
                     und liederlich lebte; da wird gezeigt, daß alle Liebe die Vergangenheit nicht
                     tilgen, nicht verwischen kann, ja, durch die tausend Stiche der Nerven, des
                     Gemüthes und die Kränkungen der Ehre sich in Zorn, Ekel, Haß verwandeln muß.
                     Mit dem ersten Stücke\pwindex{Tragedy of Othello, the Moor of Venice@\emph{The Tragedy of Othello, the Moor of Venice}|pwv}
                     geht der Hörer, auch wenn er diese Eifersucht nicht hat, weil er sich doch aus
                     Anderen in sie denken kann. Mit dem zweiten\pwindex{Le Grappin. Comedie en trois actes@\emph{Le Grappin. Comédie en trois actes}|pwv} kann er gegen das Vorurtheil, das ja von dem
                     Helden bestritten, und er kann für das Vorurtheil mit ihm gehen, das doch
                     schließlich bestätigt wird. Er ist Beiden\pwindex{Tragedy of Othello, the Moor of Venice@\emph{The Tragedy of Othello, the Moor of Venice}|pwv}\pwindex{Le Grappin. Comedie en trois actes@\emph{Le Grappin. Comédie en trois actes}|pwv} empfänglich.«
                     (S. 1.)}}}\label{K_L02723-5}, das Théâtre-Libre\orgindex{Theâtre Libre@Théâtre Libre|pw}-Stück\pwindex{Le Grappin. Comedie en trois actes@\emph{Le Grappin. Comédie en trois actes}|pwv}, von dem er ſpricht,
               behandelt etwas abſolut Anderes als das, was er behauptet. Ein frecher Schwindel, um
               ſich in allen Sätteln moderner \introOben{}fran\oindex{Frankreich@\textbf{Frankreich}, \emph{A.PCLI}|pwv}zöſiſcher\introOben{}
               Literatur gerecht zu zeigen.\pend
           
\pstart
           \label{K_L02723-6v}\edtext{\textsc{Granichstaedten\pwindex{Granichstaedten, Emil 1847-07-08 – 1904-07-02@\textsc{Granichstaedten, Emil} (1847-07-08 – 1904-07-02), \emph{Journalist/Journalistin, Rechtswissenschaftler/Rechtswissenschaftlerin}|pw}} hätte ich an Deiner Stelle geohrfeigt. Das iſt keine Kritik\pwindex{Theater- und Kunstnachrichten [Urauffuehrung Das Maerchen]@\emph{Theater- und Kunstnachrichten [Uraufführung Das Märchen]}|pwuv}\pwindex{Feuilleton. Deutsches Volkstheater [Maerchen]@\emph{Feuilleton. Deutsches Volkstheater [Märchen]}|pwuv}}{\lemma{\textnormal{\emph{Granichstaedten … Kritik}}}\Cendnote{\textnormal{Emil Granichstaedten\pwindex{Granichstaedten, Emil 1847-07-08 – 1904-07-02@\textsc{Granichstaedten, Emil} (1847-07-08 – 1904-07-02), \emph{Journalist/Journalistin, Rechtswissenschaftler/Rechtswissenschaftlerin}|pwk} verfasste eine Nachtkritik\pwindex{Theater- und Kunstnachrichten [Urauffuehrung Das Maerchen]@\emph{Theater- und Kunstnachrichten [Uraufführung Das Märchen]}|pwkv} (g.\pwindex{Granichstaedten, Emil 1847-07-08 – 1904-07-02@\textsc{Granichstaedten, Emil} (1847-07-08 – 1904-07-02), \emph{Journalist/Journalistin, Rechtswissenschaftler/Rechtswissenschaftlerin}|pwkv}: \emph{Theater- und Kunstnachrichten}\pwindex{Theater- und Kunstnachrichten [Urauffuehrung Das Maerchen]@\emph{Theater- und Kunstnachrichten [Uraufführung Das Märchen]}|pwk}. In: \emph{Die Presse}\pwindex{Presse@\emph{Die Presse}|pwk}, Jg. 46, Nr. 333, 2. 12. 1893, S. 11) und am Folgetag ein Feuilleton\pwindex{Feuilleton. Deutsches Volkstheater [Maerchen]@\emph{Feuilleton. Deutsches Volkstheater [Märchen]}|pwkv} (Emil Granichstaedten\pwindex{Granichstaedten, Emil 1847-07-08 – 1904-07-02@\textsc{Granichstaedten, Emil} (1847-07-08 – 1904-07-02), \emph{Journalist/Journalistin, Rechtswissenschaftler/Rechtswissenschaftlerin}|pwk}: \emph{Feuilleton. Deutsches Volkstheater}\pwindex{Feuilleton. Deutsches Volkstheater [Maerchen]@\emph{Feuilleton. Deutsches Volkstheater [Märchen]}|pwk}. In: \emph{Die Presse}\pwindex{Presse@\emph{Die Presse}|pwk}, Jg. 46, Nr. 334, 3. 12. 1893, S. 1–2). Auch Schnitzler war über die Nachtkritik\pwindex{Theater- und Kunstnachrichten [Urauffuehrung Das Maerchen]@\emph{Theater- und Kunstnachrichten [Uraufführung Das Märchen]}|pwkv} verärgert und bezeichnete sie im \emph{Tagebuch}\pwindex{Tagebuch@\emph{Tagebuch}|pwk} als »[p]erfid dumm« (2. 12. 1893). Granichstaedten\pwindex{Granichstaedten, Emil 1847-07-08 – 1904-07-02@\textsc{Granichstaedten, Emil} (1847-07-08 – 1904-07-02), \emph{Journalist/Journalistin, Rechtswissenschaftler/Rechtswissenschaftlerin}|pwk} lobte die Schauspielkunst Adele Sandrocks\pwindex{Sandrock, Adele 1863-08-19 – 1937-08-30@\textsc{Sandrock, Adele} (1863-08-19 – 1937-08-30), \emph{Schauspieler/Schauspielerin}|pwk}, spielte aber auf sexuelle
                  Aspekte im \emph{Märchen}\pwindex{Maerchen. Schauspiel in drei Aufzuegen@\emph{Das Märchen. Schauspiel in drei Aufzügen}|pwk} recht abschätzig an.
                  Zwischen den Zeilen kritisierte er die Handlung\pwindex{Maerchen. Schauspiel in drei Aufzuegen@\emph{Das Märchen. Schauspiel in drei Aufzügen}|pwkv} an sich und die Figuren des Fedor\pwindex{Maerchen. Schauspiel in drei Aufzuegen@\emph{Das Märchen. Schauspiel in drei Aufzügen}|pwkv} und der Fanny\pwindex{Maerchen. Schauspiel in drei Aufzuegen@\emph{Das Märchen. Schauspiel in drei Aufzügen}|pwkv}. Am 3. 12. 1893 positionierte Granichstaedten\pwindex{Granichstaedten, Emil 1847-07-08 – 1904-07-02@\textsc{Granichstaedten, Emil} (1847-07-08 – 1904-07-02), \emph{Journalist/Journalistin, Rechtswissenschaftler/Rechtswissenschaftlerin}|pwk} sich auf der Seite des Naturalismus und holte weiter aus.
                  Angefangen beim »Pessimismus unserer ›Wien\oindex{Wien@\textbf{Wien}, \emph{A.ADM2}|pw}er Modernen‹« (S. 1) kritisierte er auf
                  abwertende Weise ganz grundsätzlich das junge Werk Schnitzlers und bezog sich auch auf den \emph{Anatol}\pwindex{Anatol@\emph{Anatol}|pwk}-Zyklus. Der Autor orientiere sich zu stark an »modernen«, französischen
                  Strömungen, was ihm jedoch nicht gelinge: »Für dieſen Fedor\pwindex{Maerchen. Schauspiel in drei Aufzuegen@\emph{Das Märchen. Schauspiel in drei Aufzügen}|pwv} und dieſe Fanny\pwindex{Maerchen. Schauspiel in drei Aufzuegen@\emph{Das Märchen. Schauspiel in drei Aufzügen}|pwv} kann kein Publikum der Welt
                     sich interessieren.« (S. 2) \emph{Das Märchen}\pwindex{Maerchen. Schauspiel in drei Aufzuegen@\emph{Das Märchen. Schauspiel in drei Aufzügen}|pwk} sei »nicht tugendhaft« und
                     »[u]m Reinlichkeit wird gebeten« (S. 2).}}}\label{K_L02723-6},
               ſondern ein Gaſſenbubenſtreich.\pend
           
\pstart
           Freut mich, daß Du nicht {\pb}verbittert biſt. Das
               gehört ſich auch ſo. Ich meine, Du kannſt mit Deinem Debüt\pwindex{Maerchen. Schauspiel in drei Aufzuegen@\emph{Das Märchen. Schauspiel in drei Aufzügen}|pwv} ſehr zufrieden ſein. Man gibt Dir
               Credit, und das iſt enorm für einen Jungen.\pend
           
\pstart
           Haſt Du \label{K_L02723-7v}\edtext{\textsc{Loris\pwindex{Hofmannsthal, Hugo von 1874-02-01 – 1929-07-15@\textsc{Hofmannsthal, Hugo von} (1874-02-01 – 1929-07-15), \emph{Schriftsteller/Schriftstellerin}|pw}} über \textsc{Bauernfeld\pwindex{Bauernfeld, Eduard von 13.01.1802 – 04.08.1890@\textsc{Bauernfeld, Eduard von} (13.01.1802 – 04.08.1890)|pw}}\pwindex{Eduard von BauernfelDs dramatischer Nachlass@\emph{Eduard von Bauernfeld’s dramatischer Nachlaß}|pwv}}{\lemma{\textnormal{\emph{Loris über Bauernfeld}}}\Cendnote{\textnormal{Loris\pwindex{Hofmannsthal, Hugo von 1874-02-01 – 1929-07-15@\textsc{Hofmannsthal, Hugo von} (1874-02-01 – 1929-07-15), \emph{Schriftsteller/Schriftstellerin}|pwk}: \emph{Eduard von Bauernfeld’s dramatischer Nachlaß}\pwindex{Eduard von BauernfelDs dramatischer Nachlass@\emph{Eduard von Bauernfeld’s dramatischer Nachlaß}|pwk}. In: \emph{Frankfurter Zeitung}\pwindex{Frankfurter Zeitung@\emph{Frankfurter Zeitung}|pwk}, Jg. 38, Nr. 338, 6. 12. 1893, Erstes Morgenblatt,
                  S. 1.}}}\label{K_L02723-7} geleſen? Wie aus dieſem gottbegnadeten Menſchen\pwindex{Hofmannsthal, Hugo von 1874-02-01 – 1929-07-15@\textsc{Hofmannsthal, Hugo von} (1874-02-01 – 1929-07-15), \emph{Schriftsteller/Schriftstellerin}|pwv} die entzückenden Dinge
               herausquellen, ſo leicht und ſprudelnd. Ein Dichter\pwindex{Hofmannsthal, Hugo von 1874-02-01 – 1929-07-15@\textsc{Hofmannsthal, Hugo von} (1874-02-01 – 1929-07-15), \emph{Schriftsteller/Schriftstellerin}|pwv}! Derjenige vielleicht, den man ſeit fünfzig Jahren
               erwartet!\pend
           
\pstart
           Grüß’ ihn von mir, denn ich habe \label{K_L02723-8v}\edtext{keine directe Verbindung mehr}{\lemma{\textnormal{\emph{keine … mehr}}}\Cendnote{\textnormal{Im
                  Nachlass Hofmannsthals\pwindex{Hofmannsthal, Hugo von 1874-02-01 – 1929-07-15@\textsc{Hofmannsthal, Hugo von} (1874-02-01 – 1929-07-15), \emph{Schriftsteller/Schriftstellerin}|pwk} sind keine
                  Korrespondenzstücke Goldmanns\pwindex{Goldmann, Paul 31.01.1865 – 25.09.1935@\textsc{Goldmann, Paul} (31.01.1865 – 25.09.1935), \emph{Schriftsteller/Schriftstellerin, Journalist/Journalistin}|pwk} überliefert.
                  Unter den Briefen Beer-Hofmanns\pwindex{Beer-Hofmann, Richard 1866-07-11 – 1945-09-26@\textsc{Beer-Hofmann, Richard} (1866-07-11 – 1945-09-26), \emph{Schriftsteller/Schriftstellerin}|pwk} in der 
                     \emph{Houghton Library}\orgindex{Houghton Library@Houghton Library|pwk} dürften keine
                     Korrespondenzstücke aus dem Zeitraum Sommer 1893–1895
                     erhalten sein, wobei viele Briefe ohne Jahresangabe sind und eine genauere
                     Zuordnung notwendig wäre, um die Behauptung mit letzter Sicherheit treffen zu
                     können.
               }}}\label{K_L02723-8} mit ihm; grüße auch \textsc{Richard\pwindex{Beer-Hofmann, Richard 1866-07-11 – 1945-09-26@\textsc{Beer-Hofmann, Richard} (1866-07-11 – 1945-09-26), \emph{Schriftsteller/Schriftstellerin}|pw}} aus ſelbigem Grunde; ſei ſelbſt herzlichſt gegrüßt und ſchreibe bald!\pend
           
\pstart
           Dein{\\[\baselineskip]}\spacefill\mbox{Paul Goldmn}\pend
           \leftskip=0em{}\selectlanguage{ngerman}\endnumbering\briefempfaengerindex{Schnitzler, Arthur@\textsc{Schnitzler, Arthur}!zzzGoldmann, Paul@\emph{von Paul Goldmann}!1893-12-081@{8. 12. {[}1893{]}}|)be}\mylabel{L02723h}  \normalsize

\doendnotes{C}
\bigskip
\vfill

\clearpage

\footnotesize

\lohead{\textsc{register}}

% Definiere theindex-Environment komplett neu ohne reledmac
\makeatletter
\renewenvironment{theindex}{%
  \section*{\indexname}%
  \setlength{\parindent}{0pt}%
  \setlength{\parskip}{0pt plus 0.3pt}%
  \let\item\@idxitem
}{%
  \clearpage
}
\makeatother

\IfFileExists{\jobname-pw.ind}{\input{\jobname-pw.ind}}{}

\end{document}

      