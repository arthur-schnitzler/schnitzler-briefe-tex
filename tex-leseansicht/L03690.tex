%% latex-korrekturansicht-vorspann.tex
%% Vorspann für die Korrekturansicht.
%% Lädt die gemeinsame Datei latex-vorspann.tex mit gesetztem Schalter.

\newif\ifkorrekturansicht
\korrekturansichttrue

\input{../tex-inputs/latex-vorspann}


\section[Stefan Zweig an Arthur Schnitzler, 6. 7. 1929]{L03690 Stefan Zweig an Arthur Schnitzler, 6. 7. 1929}
\nopagebreak\mylabel{L03690v}
\rehead{ }\normalsize\beginnumbering\briefempfaengerindex{Schnitzler, Arthur@\textsc{Schnitzler, Arthur}!zzzZweig, Stefan@\emph{von Stefan Zweig}!1929-07-061@{6. 7. 1929}|(be}
\toendnotes[C]{\smallbreak\pagebreak[2]}\Standort{CUL, Schnitzler, B 118.}
\physDesc{Brief, 1 Blatt, 2 Seiten, 3020 Zeichen
\newline{}Schreibmaschine
\newline{}Handschrift: blaue Tinte, lateinische Kurrent (\noindent{}Korrekturen, Unterschrift)
\newline{}Schnitzler: 1) mit rotem Buntstift beschriftet: »\textsc{Zweig}«  2) mit rotem Buntstift zehn Unterstreichungen}
\buchAbdrucke{\weitereDrucke{Stefan Zweig: \emph{Briefwechsel mit Hermann Bahr, Sigmund Freud, Rainer Maria
                        Rilke und Arthur Schnitzler}. Frankfurt am Main: \emph{S. Fischer} 1987, S. 443–444.} }\toendnotes[C]{\smallbreak}
\pstart
           {\pb}\textcolor{gray}{\textbf{SZ}}\hfill \textcolor{gray}{\textbf{SALZBURG\oindex{Salzburg@\textbf{Salzburg}, \emph{A.ADM2}|pw}}}\pend
           
\pstart
           \raggedleft{}\textcolor{gray}{\textbf{KAPUZINERBERG 5\oindex{Paschinger Schloessl@\textbf{Paschinger Schlössl}, \emph{Wohngebäude (K.WHS)}|pw}}}\pend
           
\pstart
           \raggedleft{}6. Juli 1929.\pend
           
\pstart{}Hochverehrter, lieber Herr Doktor!\pend\vspace{0.5em}
\pstart
           Ich bin immer so erfreut, Ihre Schrift und Unterschrift zu sehen. Eigentlich hatte
               ich im Stillen gehofft, Ihnen bei der Eröffnung des Pen-Klubs\orgindex{Oesterreichischer PEN-Club@Österreichischer PEN-Club|pw} in Wien\oindex{Wien@\textbf{Wien}, \emph{A.ADM2}|pw} zu begegnen, aber ich
               war nur zwei Tage dort. Eine Hemmung, die ich selbst nicht zu erklären vermag, hält
               mich seit Jahren von Wien\oindex{Wien@\textbf{Wien}, \emph{A.ADM2}|pw} und Berlin\oindex{Berlin@\textbf{Berlin}, \emph{P.PPLC}|pw} weg. Nur so ist es zu erklären, dass ich es nicht wagte,
               mich bei Ihnen zu melden.\pend
           
\pstart
           Roda Roda\pwindex{Roda Roda, Alexander 13.04.1872 – 20.08.1945@\textsc{Roda Roda, Alexander} (13.04.1872 – 20.08.1945), \emph{Schriftsteller/Schriftstellerin}|pw} hat einen sehr richtigen Gedanken,
               wenn er die Manuskripte entweder als ganze Sammlung oder in Serien versteigern lassen
               will, denn es entstünden sonst leicht aus der Verschiedenheit der erzielten Preise
               Rivalitäten und die überflüssigsten Beleidigungen. Legt er immer drei oder vier
               Manuskripte zusammen, so kann jeder einzelne von den dreien oder vieren sich
               freundlich einbilden, er sei das Haupt- und Kapitalstück gewesen. Während in allen
               anderen Ländern die Manuskripte der lebenden Schriftsteller rasend teuer sind (ein
                  Shaw\pwindex{Shaw, George Bernard 26.07.1856 – 02.11.1950@\textsc{Shaw, George Bernard} (26.07.1856 – 02.11.1950), \emph{Schriftsteller/Schriftstellerin}|pw} oder Galsworthy\pwindex{Galsworthy, John 14.08.1867 – 31.01.1933@\textsc{Galsworthy, John} (14.08.1867 – 31.01.1933), \emph{Schriftsteller/Schriftstellerin}|pw} Manuskript von einigem Umfang würde nicht unter hunderttausend
               Mark zu haben sein), pflegt Deutschland\oindex{Deutschland@\textbf{Deutschland}, \emph{A.PCLI}|pw} als Land
               der Historie nur die Vergangenheit und hat für gegenwärtige Autoren kaum nennenswerte
               Preise. Immerhin schätze ich doch, dass ein solches, der deutschen\oindex{Deutschland@\textbf{Deutschland}, \emph{A.PCLI}|pw} Literatur dauernd angehöriges Werk wie der »Grüne Kakadu\pwindex{gruene Kakadu. Groteske in einem Akt@\emph{Der grüne Kakadu. Groteske in einem Akt}|pw}« mindestens \strikeout{selbst} tausend Mark einbrächte, umso mehr, als ich mich
               nicht erinnere, \substVorne{}\textsuperscript{nie}\substDazwischen{}je\substHinten{}mals andere Arbeiten von Ihnen als Gedichte und Briefe im »Handel« gesehen zu
               haben. Es muss aber auf jeden Fall – und da hat Roda
                  Roda\pwindex{Roda Roda, Alexander 13.04.1872 – 20.08.1945@\textsc{Roda Roda, Alexander} (13.04.1872 – 20.08.1945), \emph{Schriftsteller/Schriftstellerin}|pw} ganz recht – eine Form gefunden werden,{\pb}welche die Spender wenigstens davor
               schützt, dass sie eine gewisse öffentliche Unfreundlichkeit erleiden, indem ihre
               Manuskripte keinen Käufer finden, was natürlich auf einem puren Zufall beruhen kann.
               Aber Sie kennen ja die Zeitungen, die sich's nicht verkneifen werden, eine derartige
               Zufälligkeit literarisch zu kommentieren. Darum bin ich unbedingt gegen jede
               Vereinzelnung.\pend
           
\pstart
           Sommerpläne habe ich noch gar keine, alles hängt davon ab, ob ein Stück\pwindex{Lamm des Armen. Tragikomoedie in drei Akten@\emph{Das Lamm des Armen. Tragikomödie in drei Akten}|pwv}, das ich schrieb und mit dem ich
               fertig zu sein glaubte, wirklich fertig wird, aber ich bin mit dem letzten Akt nicht
               zufrieden und gebe es nicht aus der Hand, ehe nicht der \strikeout{bekannte} Engel eines neuen Einfalls erschienen ist, mit dem ich ringen
               kann. Gerade weil mir das Stück\pwindex{Lamm des Armen. Tragikomoedie in drei Akten@\emph{Das Lamm des Armen. Tragikomödie in drei Akten}|pwv} wichtig ist, wäre es mir ein besonderes Glück gewesen, Ihren Rat zu
               erbitten, und ein Gespräch hätte mich unermesslich gefördert. \strikeout{Gerade} \strikeout{i}\introOben{}I\introOben{}n solchen Verlegenheiten zündet oft ein einziges
               schöpferisches Wort. Vielleicht also winkt mir die besondere Freude, Ihnen irgendwo
               zu begegnen. Ich will wohl auch für einige Tage in die Schweiz\oindex{Schweiz@\textbf{Schweiz}, \emph{A.PCLI}|pw} oder wenn nicht dorthin, nach Ferleiten\oindex{Ferleiten@\textbf{Ferleiten}, \emph{P.PPL}|pw} oder einen anderen versteckteren Ort.\pend
           
\pstart
           Bitte, wagen Sie eine Postkarte, ehe Sie abreisen, die mir sagt, wohin Sie wandern;
               es wäre mir ein sehr tiefes und innerliches Bedürfnis, mit Ihnen beisammen zu sein.
               So selten ich Sie sehe, habe ich doch das Gefühl, Sie von Jahr zu Jahr besser zu
               verstehen und die Gewissheit, Sie noch immer inniger zu lieben.\pend
           
\pstart
           Ihr getreu ergebener{\\[\baselineskip]}\spacefill\mbox{{[}hs.:{]} Stefan Zweig}\pend
           \leftskip=0em{}\selectlanguage{ngerman}\endnumbering\briefempfaengerindex{Schnitzler, Arthur@\textsc{Schnitzler, Arthur}!zzzZweig, Stefan@\emph{von Stefan Zweig}!1929-07-061@{6. 7. 1929}|)be}\mylabel{L03690h}
\begin{anhang}
\end{anhang}\normalsize

\doendnotes{C}
\bigskip
\vfill

\clearpage

\footnotesize

\lohead{\textsc{register}}

% Definiere theindex-Environment komplett neu ohne reledmac
\makeatletter
\renewenvironment{theindex}{%
  \section*{\indexname}%
  \setlength{\parindent}{0pt}%
  \setlength{\parskip}{0pt plus 0.3pt}%
  \let\item\@idxitem
}{%
  \clearpage
}
\makeatother

\IfFileExists{\jobname-pw.ind}{\input{\jobname-pw.ind}}{}

\end{document}

      