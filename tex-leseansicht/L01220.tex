%% latex-korrekturansicht-vorspann.tex
%% Vorspann für die Korrekturansicht.
%% Lädt die gemeinsame Datei latex-vorspann.tex mit gesetztem Schalter.

\newif\ifkorrekturansicht
\korrekturansichttrue

\input{../tex-inputs/latex-vorspann}


\section[Arthur Schnitzler an Hermann Bahr, 16. 5. 1902]{L01220 Arthur Schnitzler an Hermann Bahr, 16. 5. 1902}
\nopagebreak\mylabel{L01220v}
\rehead{ }\normalsize\beginnumbering\briefempfaengerindex{Bahr, Hermann@\textsc{Bahr, Hermann}!zzzSchnitzler, Arthur@\emph{von Arthur Schnitzler}!1902-05-161@{16. 5. 1902}|(be}
\toendnotes[C]{\smallbreak\pagebreak[2]}\Standort{TMW, HS AM 23351 Ba.}
\physDesc{Brief, 1 Blatt, 3 Seiten, 841 Zeichen
\newline{}Handschrift: schwarze Tinte, deutsche Kurrent
\newline{}Ordnung: Lochung }
\buchAbdrucke{\weitereDrucke{1) Arthur Schnitzler: \emph{The Letters of Arthur Schnitzler to Hermann Bahr}. Chapel Hill: \emph{The University of North Carolina Press} 1978, S. 75.} \weitereDrucke{2) Hermann Bahr, Arthur Schnitzler: \emph{Briefwechsel, Aufzeichnungen, Dokumente (1891–1931)}. Göttingen: \emph{Wallstein} 2018, S. 238.} }\toendnotes[C]{\smallbreak}
\pstart{}{\pb}mein lieber
                  Hermann,\pend\vspace{0.5em}
\pstart
           bevor ich zu dir hinausko{\geminationm}e, dir für deinen guten
               ſchönen Brief zu danken, wollte ich dir heute ſchon ſagen, wie herzlich er mich
               gefreut hat – und daſs die Blumen, d\damage{ie} du mir \introOben{}ge\introOben{}ſchickt haſt, mindeſtens ebenſo wohl u
               herrlich duften als wenn ſie von einem weiblichen Weſen kämen – und je{\pb}denfalls zu den
               freundlichſten Enttäuſchungen gehören, die mir geworden ſind – Noch mehreres wollte
               ich dir ſchreiben, was aber zu leſen dir heute die Sti{\geminationm}ung fehlen wird, denn eben leſe ich daſs deine \label{K_L01220-1v}\edtext{Mutter\pwindex{Bahr, Wilhelmine 06.06.1835 – 16.05.1902@\textsc{Bahr, Wilhelmine} (06.06.1835 – 16.05.1902)|pwv} geſtorben}{\lemma{\textnormal{\emph{Mutter geſtorben}}}\Cendnote{\textnormal{Mina Bahr\pwindex{Bahr, Wilhelmine 06.06.1835 – 16.05.1902@\textsc{Bahr, Wilhelmine} (06.06.1835 – 16.05.1902)|pwk} war am 15. 5. 1902
                  in Salzburg\oindex{Salzburg@\textbf{Salzburg}, \emph{A.ADM2}|pwk} gestorben. Eine Meldung brachte etwa die \emph{Neue Freie Presse}\orgindex{Neue Freie Presse@Neue Freie Presse|pwk}, Nr. 13.551,
                        16. 5. 1902, Abendblatt, S. 2.}}}\label{K_L01220-1} iſt, und ſo ka{\geminationn} ich für heute nichts anderes mehr ſagen, als daſs ich
               dich bitte, an die innigſte {\pb}Theilnahme eines
               Menſchen zu glauben, der dein Freund \uline{geworden} iſt.
               Und was man ſo allmälig wurde, bleibt man – beſonders in unſeren Jahren. Nicht mehr
               für heute. Ich hoffe dich bald zu ſehen.\pend
           
\pstart
           In Treue dein{\\[\baselineskip]}\spacefill\mbox{Arthur}\pend
           \leftskip=0em{}
\pstart
           Wien\oindex{Wien@\textbf{Wien}, \emph{A.ADM2}|pw}{ }16. 5. 902\pend
           \selectlanguage{ngerman}\endnumbering\briefempfaengerindex{Bahr, Hermann@\textsc{Bahr, Hermann}!zzzSchnitzler, Arthur@\emph{von Arthur Schnitzler}!1902-05-161@{16. 5. 1902}|)be}\mylabel{L01220h}  \normalsize

\doendnotes{C}
\bigskip
\vfill

\clearpage

\footnotesize

\lohead{\textsc{register}}

% Definiere theindex-Environment komplett neu ohne reledmac
\makeatletter
\renewenvironment{theindex}{%
  \section*{\indexname}%
  \setlength{\parindent}{0pt}%
  \setlength{\parskip}{0pt plus 0.3pt}%
  \let\item\@idxitem
}{%
  \clearpage
}
\makeatother

\IfFileExists{\jobname-pw.ind}{\input{\jobname-pw.ind}}{}

\end{document}

      