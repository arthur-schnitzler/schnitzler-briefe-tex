%% latex-korrekturansicht-vorspann.tex
%% Vorspann für die Korrekturansicht.
%% Lädt die gemeinsame Datei latex-vorspann.tex mit gesetztem Schalter.

\newif\ifkorrekturansicht
\korrekturansichttrue

\input{../tex-inputs/latex-vorspann}


\section[Arthur Schnitzler an Richard Beer-Hofmann, 6. 7. 1898]{L00814 Arthur Schnitzler an Richard Beer-Hofmann, 6. 7. 1898}
\nopagebreak\mylabel{L00814v}
\rehead{ }\normalsize\beginnumbering\briefempfaengerindex{Beer-Hofmann, Richard@\textsc{Beer-Hofmann, Richard}!zzzSchnitzler, Arthur@\emph{von Arthur Schnitzler}!1898-07-061@{6. 7. 1898}|(be}
\toendnotes[C]{\smallbreak\pagebreak[2]}\Standort{YCGL, MSS 31.}
\physDesc{Brief, 1 Blatt, 4 Seiten, Umschlag, 1422 Zeichen
\newline{}Handschrift: Bleistift, deutsche Kurrent
\newline{}Versand: 1) Stempel: »\nobreak{}\oindex{IX., Alsergrund@\textbf{IX., Alsergrund}, \emph{A.ADM3}|pwk}Wien 9/3 72, 6. 7. 98, 4–5N\nobreak{}«.   2) Stempel: »\nobreak{}\oindex{Steindorf am Ossiacher See@\textbf{Steindorf am Ossiacher See}, \emph{A.ADM3}|pwk}{\pb}Steindorf am Ossiacher
                                       See, 7 7 98\nobreak{}«. }
\buchAbdrucke{\weitereDrucke{Arthur Schnitzler, Richard Beer-Hofmann: \emph{Briefwechsel 1891–1931}. Wien, Zürich: \emph{Europaverlag} 1992, S. 122.} }\toendnotes[C]{\smallbreak}\pstart{}{\pb}Herrn \textsc{Dr. Rich.
                     Beer-Hofmann}\pend{}\pstart{}\textsc{Steindorf\oindex{Steindorf am Ossiacher See@\textbf{Steindorf am Ossiacher See}, \emph{A.ADM3}|pw}}\pend{}\pstart{}\textsc{am}{ }\textsc{Ossiacher}ſee\oindex{Ossiacher See@\textbf{Ossiacher See}, \emph{See (N.SEE)}|pw}\pend{}\pstart{}\textsc{Kärnthen}\oindex{Kaernten@\textbf{Kärnten}, \emph{A.ADM1}|pw}\pend{}{\bigskip}\vspace{1em}
\pstart
           \raggedleft{}{\pb}6/7. 98\pend
           \vspace{0.5em}
\pstart
           Mein lieber Richard, das iſt aber wirklich Verfolgungswahn. Man ka{\geminationn} unmöglich ernſthaft darüber reden. Ich habe nach Ihrem
               Telegr das lautete Nr. 16, 1. Juni, ſowohl \strikeout{mir} Nr 16, als 1. Juni{ }ſchicken laſſen – was mir umſo leichter war als \textsc{Eisenstein}\orgindex{J. Eisenstein und Co.@J. Eisenstein {\kaufmannsund}  Co.|pw} beide Nrn gleich auf Ihre Rechnung ſchrieb. – \pend
           
\pstart
           {\pb}– Sie ſcheinen im ganzen nervöſer zu ſein, als ich
               gern hören möchte; vielleicht haben Sie doch Luſt, mich ſo zwiſchen 20.
               u 26. Juli irgendwo im Salzburg\oindex{Salzburg [Land]@\textbf{Salzburg [Land]}, \emph{A.ADM1}|pw}iſchen zu treffen? Der Auguſt iſt mir noch verſchwo{\geminationm}en. Hugo\pwindex{Hofmannsthal, Hugo von 1874-02-01 – 1929-07-15@\textsc{Hofmannsthal, Hugo von} (1874-02-01 – 1929-07-15), \emph{Schriftsteller/Schriftstellerin}|pw} hat
               erſt vom 9. Auguſt an Zeit – wir möchten gern in die Schweiz\oindex{Schweiz@\textbf{Schweiz}, \emph{A.PCLI}|pw}; überlegen Sie ſich das. –\pend
           
\pstart
           {\pb}– Die 3 Einakter\pwindex{gruene Kakadu – Paracelsus – Die Gefaehrtin. Drei Einakter@\emph{Der grüne Kakadu – Paracelsus – Die Gefährtin. Drei Einakter}|pwv} heißen: Paracelſus\pwindex{Paracelsus. Versspiel in einem Akt@\emph{Paracelsus. Versspiel in einem Akt}|pw}, Die Gefährtin\pwindex{Gefaehrtin. Schauspiel in einem Akt@\emph{Die Gefährtin. Schauspiel in einem Akt}|pw}, Der grüne Kakadu\pwindex{gruene Kakadu. Groteske in einem Akt@\emph{Der grüne Kakadu. Groteske in einem Akt}|pw}. Die beiden erſten (P.\pwindex{Paracelsus. Versspiel in einem Akt@\emph{Paracelsus. Versspiel in einem Akt}|pw} in Verſen) hab ich Hugo\pwindex{Hofmannsthal, Hugo von 1874-02-01 – 1929-07-15@\textsc{Hofmannsthal, Hugo von} (1874-02-01 – 1929-07-15), \emph{Schriftsteller/Schriftstellerin}|pw} Nachts vor ſeiner Abreiſe nach Czortkow\oindex{Tschortkiw@\textbf{Tschortkiw}, \emph{P.PPLA2}|pw}{ }\label{K_L00814-1v}\edtext{vorgeleſen}{\lemma{\textnormal{\emph{vorgeleſen}}}\Cendnote{\textnormal{Vgl. A. S.: \emph{Tagebuch}, 28. 6. 1898.
               }}}\label{K_L00814-1}; ſie ſcheinen – nein, nein, ſie haben ihm ſehr gut gefallen – insbeſondre im
                  P.\pwindex{Paracelsus. Versspiel in einem Akt@\emph{Paracelsus. Versspiel in einem Akt}|pw} findet er auch nicht eine Zeile zu
               ändern.\pend
           
\pstart
           – Mein neues Stück\pwindex{Schleier der Beatrice. Schauspiel in fuenf Akten@\emph{Der Schleier der Beatrice. Schauspiel in fünf Akten}|pwv} hat
               unterdeſſen ſonderbare Wandlungen durchgemacht – {\pb}es
               ſpielt wo anders u zu einer andren Zeit, als ich anfangs vermuthete; – jetzt iſt es
               aber dort, wo es ſein ſoll. (5 Akte.) Ich möchte es im Sommer ſchreiben, auf der
               Reiſe, freue mich ſehr darauf.\pend
           
\pstart
           – Die Arbeit bedeutet alles mögliche für mich – nicht \uuline{die}, sondern die \uuline{\edtext{Arbeit}{\Cendnote{dreifach unterstrichen}}}.\pend
           
\pstart
           – Einen Traum von \label{K_L00814-2v}\edtext{Flirt}{\lemma{\textnormal{\emph{Flirt}}}\Cendnote{\textnormal{Beer-Hofmanns\pwindex{Beer-Hofmann, Richard 1866-07-11 – 1945-09-26@\textsc{Beer-Hofmann, Richard} (1866-07-11 – 1945-09-26), \emph{Schriftsteller/Schriftstellerin}|pwk} Hund}}}\label{K_L00814-2} will ich Ihnen
               nicht erzählen; ſchreiben Sie mir bald, dſs es Ihnen und dem Götterliebling\pwindex{Tod Georgs@\emph{Der Tod Georgs}|pw} und den Ihren gut geht. Von Herzen Ihr
                  \spacefill\mbox{Arthur.}\pend
           \selectlanguage{ngerman}\endnumbering\briefempfaengerindex{Beer-Hofmann, Richard@\textsc{Beer-Hofmann, Richard}!zzzSchnitzler, Arthur@\emph{von Arthur Schnitzler}!1898-07-061@{6. 7. 1898}|)be}\mylabel{L00814h}  \normalsize

\doendnotes{C}
\bigskip
\vfill

\clearpage

\footnotesize

\lohead{\textsc{register}}

% Definiere theindex-Environment komplett neu ohne reledmac
\makeatletter
\renewenvironment{theindex}{%
  \section*{\indexname}%
  \setlength{\parindent}{0pt}%
  \setlength{\parskip}{0pt plus 0.3pt}%
  \let\item\@idxitem
}{%
  \clearpage
}
\makeatother

\IfFileExists{\jobname-pw.ind}{\input{\jobname-pw.ind}}{}

\end{document}

      