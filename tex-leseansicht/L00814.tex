%% latex-leseansicht-vorspann.tex
%% Vorspann für die Leseansicht.
%% Lädt die gemeinsame Datei latex-vorspann.tex mit nicht gesetztem Schalter.

\newif\ifkorrekturansicht
\korrekturansichtfalse

\input{../tex-inputs/latex-vorspann}


\section[Arthur Schnitzler an Richard Beer-Hofmann, 6. 7. 1898]{L00814 Arthur Schnitzler an Richard Beer-Hofmann, 6. 7. 1898}
\nopagebreak\mylabel{L00814v}
\rehead{ }\normalsize\beginnumbering\briefempfaengerindex{Beer-Hofmann, Richard@\textsc{Beer-Hofmann, Richard}!zzzSchnitzler, Arthur@\emph{von Arthur Schnitzler}!1898-07-061@{6. 7. 1898}|(be}
\toendnotes[C]{\smallbreak\pagebreak[2]}
\correspDesc{Versand  durch Arthur Schnitzler am 6. 7. 1898 in Wien
\newline{}Erhalt  durch Richard Beer-Hofmann am 7. 7. 1898 in Steindorf am Ossiacher See}\toendnotes[C]{\smallbreak}
\Standort{YCGL, MSS 31.}
\physDesc{Brief, 1 Blatt, 4 Seiten, Kuvert, 1422 Zeichen
\newline{}Handschrift: Bleistift, deutsche Kurrent
\newline{}Versand: 1) Stempel: »\nobreak{}\oindex{IX., Alsergrund@\textbf{IX., Alsergrund}, \emph{Verwaltungsgebiet}|pwk}Wien 9/3 72, 6. 7. 98, 4–5N\nobreak{}«.   2) Stempel: »\nobreak{}\oindex{Steindorf am Ossiacher See@\textbf{Steindorf am Ossiacher See}, \emph{Verwaltungsgebiet}|pwk}{\pb}Steindorf am Ossiacher
                                       See, 7 7 98\nobreak{}«. }
\buchAbdrucke{\weitereDrucke{Arthur Schnitzler, Richard Beer-Hofmann: \emph{Briefwechsel 1891–1931}. Herausgegeben von Konstanze Fliedl. Wien, Zürich: \emph{Europaverlag} 1992, S. 122.} }\toendnotes[C]{\smallbreak}\pstart{}{\pb}Herrn \textsc{Dr. Rich.
                     Beer-Hofmann}\pend{}\pstart{}\textsc{Steindorf\oindex{Steindorf am Ossiacher See@\textbf{Steindorf am Ossiacher See}, \emph{Verwaltungsgebiet}|pw}}\pend{}\pstart{}\textsc{am}{ }\textsc{Ossiacher}ſee\oindex{Ossiacher See@\textbf{Ossiacher See}, \emph{See}|pw}\pend{}\pstart{}\textsc{Kärnthen}\oindex{Kärnten@\textbf{Kärnten}, \emph{Land}|pw}\pend{}{\bigskip}\vspace{1em}
\pstart
           \raggedleft{}{\pb}6/7. 98\pend
           \vspace{0.5em}
\pstart
           Mein lieber Richard, das iſt aber wirklich Verfolgungswahn. Man ka{\geminationn} unmöglich ernſthaft darüber reden. Ich habe nach Ihrem
               Telegr das lautete Nr. 16, 1. Juni,{ }ſowohl \strikeout{mir} Nr 16, als 1. Juni{ }ſchicken laſſen – was mir umſo leichter war als \textsc{Eisenstein}\orgindex{J. Eisenstein und Co.@J. Eisenstein {\kaufmannsund}  Co.|pw} beide Nrn gleich auf Ihre Rechnung{ }ſchrieb. –\pend
           
\pstart
           {\pb}– Sie{ }ſcheinen im ganzen nervöſer zu{ }ſein, als ich
               gern hören möchte; vielleicht haben Sie doch Luſt, mich{ }ſo zwiſchen 20.
               u 26. Juli irgendwo im Salzburg\oindex{Salzburg [Land]@\textbf{Salzburg [Land]}, \emph{Land}|pw}iſchen zu treffen? Der Auguſt iſt mir noch verſchwo{\geminationm}en. Hugo\pwindex{Hofmannsthal, Hugo von 1.\,2.\,1874 Wien – 15.\,7.\,1929 Rodaun@\textsc{Hofmannsthal, Hugo von} (1.\,2.\,1874 Wien – 15.\,7.\,1929 Rodaun), \emph{Schriftsteller}|pw} hat
               erſt vom 9. Auguſt an Zeit – wir möchten gern in die Schweiz\oindex{Schweiz@\textbf{Schweiz}|pw}; überlegen Sie{ }ſich das. –\pend
           
\pstart
           {\pb}– Die 3 Einakter\pwindex{Schnitzler, Arthur 15.\,5.\,1862 Wien – 21.\,10.\,1931 ebd.@\textsc{Schnitzler, Arthur} (15.\,5.\,1862 Wien – 21.\,10.\,1931 ebd.), \emph{Schriftsteller, Mediziner}!grüne Kakadu – Paracelsus – Die Gefährtin. Drei Einakter@\strich\emph{Der grüne Kakadu – Paracelsus – Die Gefährtin. Drei Einakter}|pwv} heißen: Paracelſus\pwindex{Schnitzler, Arthur 15.\,5.\,1862 Wien – 21.\,10.\,1931 ebd.@\textsc{Schnitzler, Arthur} (15.\,5.\,1862 Wien – 21.\,10.\,1931 ebd.), \emph{Schriftsteller, Mediziner}!Paracelsus. Versspiel in einem Akt@\strich\emph{Paracelsus. Versspiel in einem Akt}|pw}, Die Gefährtin\pwindex{Schnitzler, Arthur 15.\,5.\,1862 Wien – 21.\,10.\,1931 ebd.@\textsc{Schnitzler, Arthur} (15.\,5.\,1862 Wien – 21.\,10.\,1931 ebd.), \emph{Schriftsteller, Mediziner}!Gefährtin. Schauspiel in einem Akt@\strich\emph{Die Gefährtin. Schauspiel in einem Akt}|pw}, Der grüne Kakadu\pwindex{Schnitzler, Arthur 15.\,5.\,1862 Wien – 21.\,10.\,1931 ebd.@\textsc{Schnitzler, Arthur} (15.\,5.\,1862 Wien – 21.\,10.\,1931 ebd.), \emph{Schriftsteller, Mediziner}!grüne Kakadu. Groteske in einem Akt@\strich\emph{Der grüne Kakadu. Groteske in einem Akt}|pw}. Die beiden erſten (P.\pwindex{Schnitzler, Arthur 15.\,5.\,1862 Wien – 21.\,10.\,1931 ebd.@\textsc{Schnitzler, Arthur} (15.\,5.\,1862 Wien – 21.\,10.\,1931 ebd.), \emph{Schriftsteller, Mediziner}!Paracelsus. Versspiel in einem Akt@\strich\emph{Paracelsus. Versspiel in einem Akt}|pw} in Verſen) hab ich Hugo\pwindex{Hofmannsthal, Hugo von 1.\,2.\,1874 Wien – 15.\,7.\,1929 Rodaun@\textsc{Hofmannsthal, Hugo von} (1.\,2.\,1874 Wien – 15.\,7.\,1929 Rodaun), \emph{Schriftsteller}|pw} Nachts vor{ }ſeiner Abreiſe nach Czortkow\oindex{Tschortkiw@\textbf{Tschortkiw}, \emph{Hauptstadt}|pw}{ }\label{K_L00814-1v}\edtext{vorgeleſen}{\lemma{\textnormal{\emph{vorgelesen}}}\Cendnote{\textnormal{Vgl. A. S.: \emph{Tagebuch}, 28. 6. 1898.
               }}}\label{K_L00814-1};{ }ſie{ }ſcheinen – nein, nein,{ }ſie haben ihm{ }ſehr gut gefallen – insbeſondre im
                  P.\pwindex{Schnitzler, Arthur 15.\,5.\,1862 Wien – 21.\,10.\,1931 ebd.@\textsc{Schnitzler, Arthur} (15.\,5.\,1862 Wien – 21.\,10.\,1931 ebd.), \emph{Schriftsteller, Mediziner}!Paracelsus. Versspiel in einem Akt@\strich\emph{Paracelsus. Versspiel in einem Akt}|pw} findet er auch nicht eine Zeile zu
               ändern.\pend
           
\pstart
           – Mein neues Stück\pwindex{Schnitzler, Arthur 15.\,5.\,1862 Wien – 21.\,10.\,1931 ebd.@\textsc{Schnitzler, Arthur} (15.\,5.\,1862 Wien – 21.\,10.\,1931 ebd.), \emph{Schriftsteller, Mediziner}!Schleier der Beatrice. Schauspiel in fünf Akten@\strich\emph{Der Schleier der Beatrice. Schauspiel in fünf Akten}|pwv} hat
               unterdeſſen{ }ſonderbare Wandlungen durchgemacht – {\pb}es{ }ſpielt wo anders u zu einer andren Zeit, als ich anfangs vermuthete; – jetzt iſt es
               aber dort, wo es{ }ſein{ }ſoll. (5 Akte.) Ich möchte es im Sommer{ }ſchreiben, auf der
               Reiſe, freue mich{ }ſehr darauf.\pend
           
\pstart
           – Die Arbeit bedeutet alles mögliche für mich – nicht \uuline{die}, sondern die \uuline{\edtext{Arbeit}{\Cendnote{dreifach unterstrichen}}}.\pend
           
\pstart
           – Einen Traum von \label{K_L00814-2v}\edtext{Flirt}{\lemma{\textnormal{\emph{Flirt}}}\Cendnote{\textnormal{Beer-Hofmanns\pwindex{Beer-Hofmann, Richard 11.\,7.\,1866 Wien – 26.\,9.\,1945 New York City@\textsc{Beer-Hofmann, Richard} (11.\,7.\,1866 Wien – 26.\,9.\,1945 New York City), \emph{Schriftsteller}|pwk} Hund}}}\label{K_L00814-2} will ich Ihnen
               nicht erzählen;{ }ſchreiben Sie mir bald, dſs es Ihnen und dem Götterliebling\pwindex{Beer-Hofmann, Richard 11.\,7.\,1866 Wien – 26.\,9.\,1945 New York City@\textsc{Beer-Hofmann, Richard} (11.\,7.\,1866 Wien – 26.\,9.\,1945 New York City), \emph{Schriftsteller}!Tod Georgs@\strich\emph{Der Tod Georgs}|pw} und den Ihren gut geht. Von Herzen Ihr
                  \spacefill\mbox{Arthur.}\pend
           \selectlanguage{ngerman}\endnumbering\briefempfaengerindex{Beer-Hofmann, Richard@\textsc{Beer-Hofmann, Richard}!zzzSchnitzler, Arthur@\emph{von Arthur Schnitzler}!1898-07-061@{6. 7. 1898}|)be}\mylabel{L00814h}  \newcommand{\dateiname}{L00814}\newcommand{\titel}{Arthur Schnitzler an Richard Beer-Hofmann, 6. 7. 1898}\newcommand{\editorInnen}{Martin Anton Müller und Gerd-Hermann Susen}%% latex-leseansicht-abspann.tex
%% Abspann für die Leseansicht.
%% Der Schalter \ifkorrekturansicht ist bereits durch den Vorspann gesetzt.

%% latex-abspann.tex
%% Gemeinsamer Abspann für Korrekturansicht und Leseansicht.
%% Setzt den Schalter \ifkorrekturansicht voraus (gesetzt in den
%% einbindenden Dateien latex-korrekturansicht-abspann.tex bzw.
%% latex-leseansicht-abspann.tex).
%% ---------------------------------------------------------------

\normalsize

% Das esempio-Environment wird nur in der Leseansicht benötigt
\ifkorrekturansicht\else
\newenvironment{esempio}[3]%
{
    \vspace{1.5ex}
    \rlap{\underline{#1}}
    \par
    \setlength{\parindent}{0cm}
    \nopagebreak
    \leftskip=#2cm
    \rightskip=#3cm
}
{
    \par
}
\fi

\doendnotes{C}
\bigskip
\vfill

\clearpage

\footnotesize

\ifkorrekturansicht
  \lohead{\textsc{register}}
\fi

% theindex-Environment neu definieren ohne reledmac
\makeatletter
\renewenvironment{theindex}{%
  \ifkorrekturansicht
    \section*{\indexname}%
  \else
    \subsubsection*{Index der erwähnten Entitäten}%
  \fi
  \setlength{\parindent}{0pt}%
  \setlength{\parskip}{0pt plus 0.3pt}%
  \let\item\@idxitem
}{%
  \ifkorrekturansicht\clearpage\fi
}
\makeatother

\IfFileExists{\jobname-pw.ind}{\input{\jobname-pw.ind}}{}

% Quellenangabe nur in der Leseansicht
\ifkorrekturansicht\else
% Fallback-Definitionen, falls die .tex-Datei \titel etc. nicht gesetzt hat
\providecommand{\titel}{}
\providecommand{\editorInnen}{}
\providecommand{\dateiname}{\jobname}

\vspace{3cm}

\vfill

\footnotesize
\textsc{Quelle}: \titel. Herausgegeben von {\editorInnen}. In: \emph{Arthur Schnitzler: Briefwechsel mit Autorinnen und Autoren}.
 Digitale Edition, https://schnitzler-briefe.acdh.oeaw.ac.at/{\dateiname}.html (Stand \today)
\fi

\end{document}


