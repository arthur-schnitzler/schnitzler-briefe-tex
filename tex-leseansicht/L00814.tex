%% latex-leseansicht-vorspann.tex
%% Vorspann für die Leseansicht.
%% Lädt die gemeinsame Datei latex-vorspann.tex mit nicht gesetztem Schalter.

\newif\ifkorrekturansicht
\korrekturansichtfalse

\input{../tex-inputs/latex-vorspann}


         
         \newcommand{\erwaehntePersonen}{Personen: }
         \newcommand{\erwaehnteInstitutionen}{}
         \newcommand{\erwaehnteOrte}{}
         \newcommand{\erwaehnteWerke}{
               \section[Arthur Schnitzler an Richard Beer-Hofmann, 6. 7. 1898]{ Arthur Schnitzler an Richard Beer-Hofmann, 6. 7. 1898}\nopagebreak\mylabel{v}\rehead{ }\begin{ledgroupsized}[t]{13cm}\normalsize\beginnumbering \toendnotes[C]{\smallbreak\pagebreak[2]} \Standort{YCGL, MSS 31.}
\physDesc{Brief, 1 Blatt, 4 Seiten, Umschlag
\newline{}Handschrift: Bleistift, deutsche Kurrent\newline{}Versand: 1) Stempel: »\nobreak{}\oindex{XXXX Ortsangabe fehlt|pwk}Wien 9/3 72, 6. 7. 98, 4–5N\nobreak{}«.   2) Stempel: »\nobreak{}\oindex{XXXX Ortsangabe fehlt|pwk}{\pb}Steindorf am
                              Ossiacher See, 7 7 98\nobreak{}«. }\buchAbdrucke{\weitereDrucke{Arthur Schnitzler, Richard Beer-Hofmann: \emph{Briefwechsel 1891–1931}. Hg. Konstanze Fliedl. Wien, Zürich: \emph{Europaverlag} 1992, S. 122.} }\toendnotes[C]{\smallbreak}\pstart{}{\pb}Herrn \textsc{Dr. Rich.
                     Beer-Hofmann}\pend{}\pstart{}\textsc{Steindorf\oindex{XXXX Ortsangabe fehlt|pw}}\pend{}\pstart{}\textsc{am }\textsc{Ossiacher}ſee\oindex{XXXX Ortsangabe fehlt|pw}\pend{}\pstart{}\textsc{Kärnthen}\oindex{XXXX Ortsangabe fehlt|pw}\pend{}{\bigskip}\pstart
           \raggedleft{}{\pb}6/7. 98\pend
           \pstart
           Mein lieber Richard, das iſt aber wirklich Verfolgungswahn. Man ka{\geminationn} unmöglich ernſthaft darüber reden. Ich habe nach Ihrem
               Telegr das lautete Nr. 16, 1. Juni, ſowohl \strikeout{mir} Nr 16, als 1. Juni{ }ſchicken laſſen – was mir umſo leichter war als \textsc{Eisenstein}XXXX ORGangabe fehlt beide Nrn gleich auf Ihre Rechnung ſchrieb. – \pend
           \pstart
           {\pb}– Sie ſcheinen im ganzen nervöſer zu ſein, als ich
               gern hören möchte; vielleicht haben Sie doch Luſt, mich ſo zwiſchen 20.
               u 26. Juli irgendwo im Salzburg\oindex{XXXX Ortsangabe fehlt|pw}iſchen
               zu treffen? Der Auguſt iſt mir noch verſchwo{\geminationm}en. Hugo\pwindex{\textcolor{red}{\textsuperscript{XXXX1 indx}}|pw} hat erſt vom 9. Auguſt an
               Zeit – wir möchten gern in die Schweiz\oindex{XXXX Ortsangabe fehlt|pw}; überlegen
               Sie ſich das. –\pend
           \pstart
           {\pb}– Die 3 Einakter\textcolor{red}{\textsuperscript{XXXX indx}} heißen: Paracelſus\textcolor{red}{\textsuperscript{XXXX indx}}, Die Gefährtin\textcolor{red}{\textsuperscript{XXXX indx}}, Der
                  grüne Kakadu\textcolor{red}{\textsuperscript{XXXX indx}}. Die beiden erſten (P.\textcolor{red}{\textsuperscript{XXXX indx}} in
               Verſen) hab ich Hugo\pwindex{\textcolor{red}{\textsuperscript{XXXX1 indx}}|pw} Nachts vor ſeiner Abreiſe
               nach Czortkow\oindex{XXXX Ortsangabe fehlt|pw}{ }\label{K_L00814_1v}\edtext{vorgeleſen}{\lemma{\textnormal{\emph{vorgeleſen}}}\Cendnote{\textnormal{vgl. A. S.: \emph{Tagebuch}, 28. 6. 1898}}}\label{K_L00814_1h}; ſie
               ſcheinen – nein, nein, ſie haben ihm ſehr gut gefallen – insbeſondre im P.\textcolor{red}{\textsuperscript{XXXX indx}} findet er auch nicht eine Zeile zu ändern.\pend
           \pstart
           – Mein neues Stück\textcolor{red}{\textsuperscript{XXXX indx}} hat
               unterdeſſen ſonderbare Wandlungen durchgemacht – {\pb}es
               ſpielt wo anders u zu einer andren Zeit, als ich anfangs vermuthete; – jetzt iſt es
               aber dort, wo es ſein ſoll. (5 Akte.) Ich möchte es im Sommer ſchreiben, auf der
               Reiſe, freue mich ſehr darauf.\pend
           \pstart
           – Die Arbeit bedeutet alles mögliche für mich – nicht \uuline{die}, sondern die \uuline{\edtext{Arbeit}{\Cendnote{dreifach unterstrichen}}}.\pend
           \pstart
           – Einen Traum von \label{K_L00814_2v}\edtext{Flirt}{\lemma{\textnormal{\emph{Flirt}}}\Cendnote{\textnormal{Beer-Hofmann\pwindex{\textcolor{red}{\textsuperscript{XXXX1 indx}}|pwk}s Hund}}}\label{K_L00814_2h} will ich Ihnen nicht erzählen; ſchreiben Sie
               mir bald, dſs es Ihnen und dem Götterliebling\textcolor{red}{\textsuperscript{XXXX indx}} und
               den Ihren gut geht. Von Herzen Ihr \spacefill\mbox{Arthur.}\pend
           
         
         \endnumbering\mylabel{h}\end{ledgroupsized}  \newcommand{\dateiname}{L00814}\newcommand{\titel}{Arthur Schnitzler an Richard Beer-Hofmann, 6. 7. 1898}\newcommand{\editorInnen}{Martin Anton Müller und Gerd-Hermann Susen}%% latex-leseansicht-abspann.tex
%% Abspann für die Leseansicht.
%% Der Schalter \ifkorrekturansicht ist bereits durch den Vorspann gesetzt.

%% latex-abspann.tex
%% Gemeinsamer Abspann für Korrekturansicht und Leseansicht.
%% Setzt den Schalter \ifkorrekturansicht voraus (gesetzt in den
%% einbindenden Dateien latex-korrekturansicht-abspann.tex bzw.
%% latex-leseansicht-abspann.tex).
%% ---------------------------------------------------------------

\normalsize

% Das esempio-Environment wird nur in der Leseansicht benötigt
\ifkorrekturansicht\else
\newenvironment{esempio}[3]%
{
    \vspace{1.5ex}
    \rlap{\underline{#1}}
    \par
    \setlength{\parindent}{0cm}
    \nopagebreak
    \leftskip=#2cm
    \rightskip=#3cm
}
{
    \par
}
\fi

\doendnotes{C}
\bigskip
\vfill

\clearpage

\footnotesize

\ifkorrekturansicht
  \lohead{\textsc{register}}
\fi

% theindex-Environment neu definieren ohne reledmac
\makeatletter
\renewenvironment{theindex}{%
  \ifkorrekturansicht
    \section*{\indexname}%
  \else
    \subsubsection*{Index der erwähnten Entitäten}%
  \fi
  \setlength{\parindent}{0pt}%
  \setlength{\parskip}{0pt plus 0.3pt}%
  \let\item\@idxitem
}{%
  \ifkorrekturansicht\clearpage\fi
}
\makeatother

\IfFileExists{\jobname-pw.ind}{\input{\jobname-pw.ind}}{}

% Quellenangabe nur in der Leseansicht
\ifkorrekturansicht\else
% Fallback-Definitionen, falls die .tex-Datei \titel etc. nicht gesetzt hat
\providecommand{\titel}{}
\providecommand{\editorInnen}{}
\providecommand{\dateiname}{\jobname}

\vspace{3cm}

\vfill

\footnotesize
\textsc{Quelle}: \titel. Herausgegeben von {\editorInnen}. In: \emph{Arthur Schnitzler: Briefwechsel mit Autorinnen und Autoren}.
 Digitale Edition, https://schnitzler-briefe.acdh.oeaw.ac.at/{\dateiname}.html (Stand \today)
\fi

\end{document}


      