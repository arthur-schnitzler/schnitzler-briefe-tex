%% latex-korrekturansicht-vorspann.tex
%% Vorspann für die Korrekturansicht.
%% Lädt die gemeinsame Datei latex-vorspann.tex mit gesetztem Schalter.

\newif\ifkorrekturansicht
\korrekturansichttrue

\input{../tex-inputs/latex-vorspann}


\section[Hermann Bahr an Arthur Schnitzler, 11. 1. 1907]{L01650 Hermann Bahr an Arthur Schnitzler, 11. 1. 1907}
\nopagebreak\mylabel{L01650v}
\rehead{ }\normalsize\beginnumbering\briefempfaengerindex{Schnitzler, Arthur@\textsc{Schnitzler, Arthur}!zzzBahr, Hermann@\emph{von Hermann Bahr}!1907-01-111@{11. 1. 1907}|(be}
\toendnotes[C]{\smallbreak\pagebreak[2]}\Standort{CUL, Schnitzler, B 5b.}
\physDesc{Brief, 1 Blatt, 3 Seiten, 1090 Zeichen
\newline{}Handschrift Lisa Clarus: blaue Tinte, lateinische Kurrent
\newline{}Handschrift Hermann Bahr: blaue Tinte (\noindent{}Unterschrift)
\newline{}Ordnung: mit Bleistift von unbekannter Hand nummeriert:
                                    »143« }
\buchAbdrucke{\weitereDrucke{Hermann Bahr, Arthur Schnitzler: \emph{Briefwechsel, Aufzeichnungen, Dokumente (1891–1931)}. Göttingen: \emph{Wallstein} 2018, S. 387.} }\toendnotes[C]{\smallbreak}
\pstart
           \raggedleft{}{\pb}Wien XIII/\textsubscript{7}\oindex{Ober Sankt Veit@\textbf{Ober Sankt Veit}, \emph{P.PPLX}|pw} den 11. 1. 07.\pend
           
\pstart\center{}Lieber Arthur!\pend\vspace{0.5em}
\pstart
           Ich war \label{K_L01650-1v}\edtext{vierzehn Tage}{\lemma{\textnormal{\emph{vierzehn Tage}}}\Cendnote{\textnormal{Nachweislich war Bahr\pwindex{Bahr, Hermann 19.07.1863 – 15.01.1934@\textsc{Bahr, Hermann} (19.07.1863 – 15.01.1934), \emph{Schriftsteller/Schriftstellerin, Kritiker/Kritikerin}|pwk} am 2. und 4. 1. 1907 auf dem
                     Semmering\oindex{Semmering@\textbf{Semmering}, \emph{A.ADM3}|pwk}.}}}\label{K_L01650-1} auf dem Semmering\oindex{Semmering@\textbf{Semmering}, \emph{A.ADM3}|pw} und bin nun seit Dienstag hier, für etwa zwölf Tage,
               mit dem Vorsatze:\pend
           
\pstart
           1. Das \label{K_L01650-2v}\edtext{Regiebuch von Hedda Gabler\pwindex{Hedda Gabler@\emph{Hedda Gabler}|pw}}{\lemma{\textnormal{\emph{Regiebuch … Gabler}}}\Cendnote{\textnormal{Dieses findet sich in Bahrs\pwindex{Bahr, Hermann 19.07.1863 – 15.01.1934@\textsc{Bahr, Hermann} (19.07.1863 – 15.01.1934), \emph{Schriftsteller/Schriftstellerin, Kritiker/Kritikerin}|pwk} Nachlass (\emph{Theatermuseum Wien}, H. VM 3683 Ba). Die Premiere
                  von \emph{Hedda Gabler}\pwindex{Hedda Gabler@\emph{Hedda Gabler}|pwk} von Henrik Ibsen\pwindex{Ibsen, Henrik 20.03.1828 – 23.05.1906@\textsc{Ibsen, Henrik} (20.03.1828 – 23.05.1906), \emph{Schriftsteller/Schriftstellerin}|pwk}
                  fand am 11. 3. 1907 an den \emph{Kammerspielen}\orgindex{Kammerspiele Berlin@Kammerspiele Berlin|pwk} in Berlin\oindex{Berlin@\textbf{Berlin}, \emph{P.PPLC}|pwk} statt.}}}\label{K_L01650-2} zu machen, deren Proben am 24. d. beginnen sollen.\pend
           
\pstart
           \introOben{}2.\introOben{}{ }\substVorne{}\textsuperscript{z}\substDazwischen{}Z\substHinten{}u versuchen, ob mein neues Stück\pwindex{gelbe Nachtigall@\emph{Die gelbe Nachtigall}|pwv} schon so weit ist, dass sich mir ungefähr ein Szenarium ergibt,
               welches dann im Sommer ausgearbeitet werden soll, und 3. ein{\pb}mal mit Dir, Richard\pwindex{Beer-Hofmann, Richard 1866-07-11 – 1945-09-26@\textsc{Beer-Hofmann, Richard} (1866-07-11 – 1945-09-26), \emph{Schriftsteller/Schriftstellerin}|pw} und Salten\pwindex{Salten, Felix 06.09.1869 – 08.10.1945@\textsc{Salten, Felix} (06.09.1869 – 08.10.1945), \emph{Schriftsteller/Schriftstellerin, Journalist/Journalistin, Chefredakteur/Chefredakteurin}|pw} zusammen zu sein,
               einmal mit Kainz\pwindex{Kainz, Margarethe 13.12.1858 – 12.02.1950@\textsc{Kainz, Margarethe} (13.12.1858 – 12.02.1950), \emph{Schauspieler/Schauspielerin}|pw}, gelegentlich auch Fred\pwindex{W. Fred 29.06.1879 – 23.10.1922@\textsc{W. Fred} (29.06.1879 – 23.10.1922), \emph{Schriftsteller/Schriftstellerin, Journalist/Journalistin}|pw} und Handl\pwindex{Handl, Willi 12.2.1872 – 26.5.1920@\textsc{Handl, Willi} (12.2.1872 – 26.5.1920), \emph{Schriftsteller/Schriftstellerin, Journalist/Journalistin}|pw} zu sehen, sonst aber mich zu verstecken. Dies ist es was ich
               »incognito« nenne. Meine Absicht war, Dir vorzuschlagen, ob ich nicht nächste Woche
               einmal von \substVorne{}\textsuperscript{e}\substDazwischen{}E\substHinten{}ilf bis Drei bei Dir sein und dort vielleicht auch gleich Salten\pwindex{Salten, Felix 06.09.1869 – 08.10.1945@\textsc{Salten, Felix} (06.09.1869 – 08.10.1945), \emph{Schriftsteller/Schriftstellerin, Journalist/Journalistin, Chefredakteur/Chefredakteurin}|pw} und Richard\pwindex{Beer-Hofmann, Richard 1866-07-11 – 1945-09-26@\textsc{Beer-Hofmann, Richard} (1866-07-11 – 1945-09-26), \emph{Schriftsteller/Schriftstellerin}|pw}
               treffen könnte. Dass Du nun aber Sonntag Vormittag zu mir kommen willst, ist mir sehr
               erwünscht, stört mich gar nicht, freut mich riesig (ich kann Dir nur nichts zu {\pb}essen geben, weil ich keine Köchin habe) und wir
               können dann alles Mögliche besprechen.\pend
           
\pstart
           Mit den herzlichsten Grüssen an Deine liebe Frau\pwindex{Schnitzler, Olga 17.01.1882 – 13.01.1970@\textsc{Schnitzler, Olga} (17.01.1882 – 13.01.1970), \emph{Schauspieler/Schauspielerin, Sänger/Sängerin}|pwv}{\\[\baselineskip]}Dein alter{\\[\baselineskip]}\spacefill\mbox{{[}hs. :{]} Hermann}\pend
           \leftskip=0em{}
\pstart
           \noindent{}{[}hs. :{]} PS.{\\}»Ringelspiel\pwindex{Brief an Arthur Schnitzler@\emph{Brief an Arthur Schnitzler}|pw}« und »Grotesken\pwindex{Grotesken. Der Klub der Erloeser (Schauspiel in einem Akt) – Der  Faun – Die tiefe Natur (Ein Akt)@\emph{Grotesken. Der Klub der Erlöser (Schauspiel in einem Akt) – Der Faun – Die tiefe Natur (Ein Akt)}|pw}« hast Du
                  hoffentlich richtig bekommen?\pend
           \selectlanguage{ngerman}\endnumbering\briefempfaengerindex{Schnitzler, Arthur@\textsc{Schnitzler, Arthur}!zzzBahr, Hermann@\emph{von Hermann Bahr}!1907-01-111@{11. 1. 1907}|)be}\mylabel{L01650h}  \normalsize

\doendnotes{C}
\bigskip
\vfill

\clearpage

\footnotesize

\lohead{\textsc{register}}

% Definiere theindex-Environment komplett neu ohne reledmac
\makeatletter
\renewenvironment{theindex}{%
  \section*{\indexname}%
  \setlength{\parindent}{0pt}%
  \setlength{\parskip}{0pt plus 0.3pt}%
  \let\item\@idxitem
}{%
  \clearpage
}
\makeatother

\IfFileExists{\jobname-pw.ind}{\input{\jobname-pw.ind}}{}

\end{document}

      