%% latex-leseansicht-vorspann.tex
%% Vorspann für die Leseansicht.
%% Lädt die gemeinsame Datei latex-vorspann.tex mit nicht gesetztem Schalter.

\newif\ifkorrekturansicht
\korrekturansichtfalse

\input{../tex-inputs/latex-vorspann}


\section[Hermann Bahr an Arthur Schnitzler, 11. 1. 1907]{L01650 Hermann Bahr an Arthur Schnitzler, 11. 1. 1907}
\nopagebreak\mylabel{L01650v}
\rehead{ }\normalsize\beginnumbering\briefempfaengerindex{Schnitzler, Arthur@\textsc{Schnitzler, Arthur}!zzzBahr, Hermann@\emph{von Hermann Bahr}!1907-01-111@{11. 1. 1907}|(be}
\toendnotes[C]{\smallbreak\pagebreak[2]}
\correspDesc{Versand  durch Hermann Bahr am 11. 1. 1907 in Wien
\newline{}Erhalt  durch Arthur Schnitzler im Zeitraum [11. 1. 1907
                  – 15. 1. 1907?] in Wien}\toendnotes[C]{\smallbreak}
\Standort{CUL, Schnitzler, B 5b.}
\physDesc{Brief, 1 Blatt, 3 Seiten, 1090 Zeichen
\newline{}Handschrift Lisa Clarus: blaue Tinte, lateinische Kurrent
\newline{}Handschrift Hermann Bahr: blaue Tinte (\noindent{}Unterschrift)
\newline{}Ordnung: mit Bleistift von unbekannter Hand nummeriert:
                                    »143« }
\buchAbdrucke{\weitereDrucke{Hermann Bahr, Arthur Schnitzler: \emph{Briefwechsel, Aufzeichnungen, Dokumente (1891–1931)}. Herausgegeben von Kurt Ifkovits und Martin Anton Müller. Göttingen: \emph{Wallstein} 2018, S. 387.} }\toendnotes[C]{\smallbreak}
\pstart
           \raggedleft{}{\pb}Wien XIII/\textsubscript{7}\oindex{Wien@\textbf{Wien}!XIII., Hietzing@\textbf{XIII., Hietzing}!Ober Sankt Veit@\textbf{Ober Sankt Veit}, \emph{Ehemaliger Ort}|pw} den 11. 1. 07.\pend
           
\pstart\center{}Lieber Arthur!\pend\vspace{0.5em}
\pstart
           Ich war \label{K_L01650-1v}\edtext{vierzehn Tage}{\lemma{\textnormal{\emph{vierzehn Tage}}}\Cendnote{\textnormal{Nachweislich war Bahr\pwindex{Bahr, Hermann 19.\,7.\,1863 Linz – 15.\,1.\,1934 München@\textsc{Bahr, Hermann} (19.\,7.\,1863 Linz – 15.\,1.\,1934 München), \emph{Schriftsteller, Kritiker}|pwk} am 2. und 4. 1. 1907 auf dem
                     Semmering\oindex{Semmering@\textbf{Semmering}, \emph{Verwaltungsgebiet}|pwk}.}}}\label{K_L01650-1} auf dem Semmering\oindex{Semmering@\textbf{Semmering}, \emph{Verwaltungsgebiet}|pw} und bin nun seit Dienstag hier, für etwa zwölf Tage,
               mit dem Vorsatze:\pend
           
\pstart
           1. Das \label{K_L01650-2v}\edtext{Regiebuch von Hedda Gabler\pwindex{Ibsen, Henrik 20.\,3.\,1828 Skien – 23.\,5.\,1906 Oslo@\textsc{Ibsen, Henrik} (20.\,3.\,1828 Skien – 23.\,5.\,1906 Oslo), \emph{Schriftsteller}!Hedda Gabler@\strich\emph{Hedda Gabler}|pw}}{\lemma{\textnormal{\emph{Regiebuch … Gabler}}}\Cendnote{\textnormal{Dieses findet sich in Bahrs\pwindex{Bahr, Hermann 19.\,7.\,1863 Linz – 15.\,1.\,1934 München@\textsc{Bahr, Hermann} (19.\,7.\,1863 Linz – 15.\,1.\,1934 München), \emph{Schriftsteller, Kritiker}|pwk} Nachlass (\emph{Theatermuseum Wien}, H. VM 3683 Ba). Die Premiere
                  von \emph{Hedda Gabler}\pwindex{Ibsen, Henrik 20.\,3.\,1828 Skien – 23.\,5.\,1906 Oslo@\textsc{Ibsen, Henrik} (20.\,3.\,1828 Skien – 23.\,5.\,1906 Oslo), \emph{Schriftsteller}!Hedda Gabler@\strich\emph{Hedda Gabler}|pwk} von Henrik Ibsen\pwindex{Ibsen, Henrik 20.\,3.\,1828 Skien – 23.\,5.\,1906 Oslo@\textsc{Ibsen, Henrik} (20.\,3.\,1828 Skien – 23.\,5.\,1906 Oslo), \emph{Schriftsteller}|pwk}
                  fand am 11. 3. 1907 an den \emph{Kammerspielen}\orgindex{Kammerspiele Berlin@Kammerspiele Berlin|pwk} in Berlin\oindex{Berlin@\textbf{Berlin}, \emph{Hauptstadt}|pwk} statt.}}}\label{K_L01650-2} zu machen, deren Proben am 24. d. beginnen sollen.\pend
           
\pstart
           \introOben{}2.\introOben{}{ }\substVorne{}\textsuperscript{z}\substDazwischen{}Z\substHinten{}u versuchen, ob mein neues Stück\pwindex{Bahr, Hermann 19.\,7.\,1863 Linz – 15.\,1.\,1934 München@\textsc{Bahr, Hermann} (19.\,7.\,1863 Linz – 15.\,1.\,1934 München), \emph{Schriftsteller, Kritiker}!gelbe Nachtigall@\strich\emph{Die gelbe Nachtigall}|pwv} schon so weit ist, dass sich mir ungefähr ein Szenarium ergibt,
               welches dann im Sommer ausgearbeitet werden soll, und 3. ein{\pb}mal mit Dir, Richard\pwindex{Beer-Hofmann, Richard 11.\,7.\,1866 Wien – 26.\,9.\,1945 New York City@\textsc{Beer-Hofmann, Richard} (11.\,7.\,1866 Wien – 26.\,9.\,1945 New York City), \emph{Schriftsteller}|pw} und Salten\pwindex{Salten, Felix 6.\,9.\,1869 Budapest – 8.\,10.\,1945 Zürich@\textsc{Salten, Felix} (6.\,9.\,1869 Budapest – 8.\,10.\,1945 Zürich), \emph{Schriftsteller, Journalist, Chefredakteur}|pw} zusammen zu sein,
               einmal mit Kainz\pwindex{Kainz, Margarethe 31.\,12.\,1885 Berlin – 12.\,2.\,1950 Wien@\textsc{Kainz, Margarethe} (31.\,12.\,1885 Berlin – 12.\,2.\,1950 Wien), \emph{Schauspielerin}|pw}, gelegentlich auch Fred\pwindex{W. Fred 29.\,6.\,1879 Wien – 23.\,10.\,1922 Berlin@\textsc{W. Fred} (29.\,6.\,1879 Wien – 23.\,10.\,1922 Berlin), \emph{Schriftsteller, Journalist}|pw} und Handl\pwindex{Handl, Willi 12.\,2.\,1872 Wien – 26.\,5.\,1920 Berlin@\textsc{Handl, Willi} (12.\,2.\,1872 Wien – 26.\,5.\,1920 Berlin), \emph{Schriftsteller, Journalist}|pw} zu sehen, sonst aber mich zu verstecken. Dies ist es was ich
               »incognito« nenne. Meine Absicht war, Dir vorzuschlagen, ob ich nicht nächste Woche
               einmal von \substVorne{}\textsuperscript{e}\substDazwischen{}E\substHinten{}ilf bis Drei bei Dir sein und dort vielleicht auch gleich Salten\pwindex{Salten, Felix 6.\,9.\,1869 Budapest – 8.\,10.\,1945 Zürich@\textsc{Salten, Felix} (6.\,9.\,1869 Budapest – 8.\,10.\,1945 Zürich), \emph{Schriftsteller, Journalist, Chefredakteur}|pw} und Richard\pwindex{Beer-Hofmann, Richard 11.\,7.\,1866 Wien – 26.\,9.\,1945 New York City@\textsc{Beer-Hofmann, Richard} (11.\,7.\,1866 Wien – 26.\,9.\,1945 New York City), \emph{Schriftsteller}|pw}
               treffen könnte. Dass Du nun aber Sonntag Vormittag zu mir kommen willst, ist mir sehr
               erwünscht, stört mich gar nicht, freut mich riesig (ich kann Dir nur nichts zu {\pb}essen geben, weil ich keine Köchin habe) und wir
               können dann alles Mögliche besprechen.\pend
           
\pstart
           Mit den herzlichsten Grüssen an Deine liebe Frau\pwindex{Schnitzler, Olga 17.\,1.\,1882 Wien – 13.\,1.\,1970 Lugano@\textsc{Schnitzler, Olga} (17.\,1.\,1882 Wien – 13.\,1.\,1970 Lugano), \emph{Schauspielerin, Sängerin}|pwv}{\\[\baselineskip]}Dein alter{\\[\baselineskip]}\spacefill\mbox{{[}hs. Bahr:{]} Hermann}\pend
           \leftskip=0em{}
\pstart
           \noindent{}{[}hs. Clarus:{]} PS.{\\}»Ringelspiel\pwindex{Bahr, Hermann 19.\,7.\,1863 Linz – 15.\,1.\,1934 München@\textsc{Bahr, Hermann} (19.\,7.\,1863 Linz – 15.\,1.\,1934 München), \emph{Schriftsteller, Kritiker}!Brief an Arthur Schnitzler@\strich\emph{Brief an Arthur Schnitzler}|pw}« und »Grotesken\pwindex{Bahr, Hermann 19.\,7.\,1863 Linz – 15.\,1.\,1934 München@\textsc{Bahr, Hermann} (19.\,7.\,1863 Linz – 15.\,1.\,1934 München), \emph{Schriftsteller, Kritiker}!Grotesken. Der Klub der Erlöser (Schauspiel in einem Akt) – Der  Faun – Die tiefe Natur (Ein Akt)@\strich\emph{Grotesken. Der Klub der Erlöser (Schauspiel in einem Akt) – Der Faun – Die tiefe Natur (Ein Akt)}|pw}« hast Du
                  hoffentlich richtig bekommen?\pend
           \selectlanguage{ngerman}\endnumbering\briefempfaengerindex{Schnitzler, Arthur@\textsc{Schnitzler, Arthur}!zzzBahr, Hermann@\emph{von Hermann Bahr}!1907-01-111@{11. 1. 1907}|)be}\mylabel{L01650h}  \newcommand{\dateiname}{L01650}\newcommand{\titel}{Hermann Bahr an Arthur Schnitzler, 11. 1. 1907}\newcommand{\editorInnen}{Herausgegeben von Martin Anton Müller}%% latex-leseansicht-abspann.tex
%% Abspann für die Leseansicht.
%% Der Schalter \ifkorrekturansicht ist bereits durch den Vorspann gesetzt.

%% latex-abspann.tex
%% Gemeinsamer Abspann für Korrekturansicht und Leseansicht.
%% Setzt den Schalter \ifkorrekturansicht voraus (gesetzt in den
%% einbindenden Dateien latex-korrekturansicht-abspann.tex bzw.
%% latex-leseansicht-abspann.tex).
%% ---------------------------------------------------------------

\normalsize

% Das esempio-Environment wird nur in der Leseansicht benötigt
\ifkorrekturansicht\else
\newenvironment{esempio}[3]%
{
    \vspace{1.5ex}
    \rlap{\underline{#1}}
    \par
    \setlength{\parindent}{0cm}
    \nopagebreak
    \leftskip=#2cm
    \rightskip=#3cm
}
{
    \par
}
\fi

\doendnotes{C}
\bigskip
\vfill

\clearpage

\footnotesize

\ifkorrekturansicht
  \lohead{\textsc{register}}
\fi

% theindex-Environment neu definieren ohne reledmac
\makeatletter
\renewenvironment{theindex}{%
  \ifkorrekturansicht
    \section*{\indexname}%
  \else
    \subsubsection*{Index der erwähnten Entitäten}%
  \fi
  \setlength{\parindent}{0pt}%
  \setlength{\parskip}{0pt plus 0.3pt}%
  \let\item\@idxitem
}{%
  \ifkorrekturansicht\clearpage\fi
}
\makeatother

\IfFileExists{\jobname-pw.ind}{\input{\jobname-pw.ind}}{}

% Quellenangabe nur in der Leseansicht
\ifkorrekturansicht\else
% Fallback-Definitionen, falls die .tex-Datei \titel etc. nicht gesetzt hat
\providecommand{\titel}{}
\providecommand{\editorInnen}{}
\providecommand{\dateiname}{\jobname}

\vspace{3cm}

\vfill

\footnotesize
\textsc{Quelle}: \titel. Herausgegeben von {\editorInnen}. In: \emph{Arthur Schnitzler: Briefwechsel mit Autorinnen und Autoren}.
 Digitale Edition, https://schnitzler-briefe.acdh.oeaw.ac.at/{\dateiname}.html (Stand \today)
\fi

\end{document}


