%% latex-leseansicht-vorspann.tex
%% Vorspann für die Leseansicht.
%% Lädt die gemeinsame Datei latex-vorspann.tex mit nicht gesetztem Schalter.

\newif\ifkorrekturansicht
\korrekturansichtfalse

\input{../tex-inputs/latex-vorspann}


         
         \renewcommand{\erwaehntePersonen}{Personen: Richard Beer-Hofmann, Lisa Clarus, Willi Handl, Henrik Ibsen, Margarethe Kainz, Felix Salten, Olga Schnitzler,  W. Fred}
         \renewcommand{\erwaehnteOrte}{Orte: Ober Sankt Veit, Semmering, Wien}
         \renewcommand{\erwaehnteWerke}{Werke: Brief an Arthur Schnitzler, Die gelbe Nachtigall, Grotesken. Der Klub der Erlöser (Schauspiel in einem Akt) – Der  Faun – Die tiefe Natur (Ein Akt), Hedda Gabler}
               \section[Hermann Bahr an Arthur Schnitzler, 11. 1. 1907]{ Hermann Bahr an Arthur Schnitzler, 11. 1. 1907}\nopagebreak\mylabel{v}\rehead{ }\begin{ledgroupsized}[t]{13cm}\normalsize\beginnumbering \toendnotes[C]{\smallbreak\pagebreak[2]} \Standort{CUL, Schnitzler, B 5b.}
\physDesc{Brief, 1 Blatt, 3 Seiten
\newline{}Handschrift Lisa Clarus: blaue Tinte, lateinische Kurrent\newline{}Handschrift Hermann Bahr: blaue Tinte (\noindent{}Unterschrift)\newline{}Ordnung: mit Bleistift von unbekannter Hand nummeriert:
                                    »143« }\buchAbdrucke{\weitereDrucke{Hermann Bahr, Arthur Schnitzler: \emph{Briefwechsel, Aufzeichnungen, Dokumente (1891–1931)}. Hg. Kurt Ifkovits und Martin Anton Müller. Göttingen: \emph{Wallstein} 2018, S. 387.} }\toendnotes[C]{\smallbreak}\pstart
           \raggedleft{}{\pb}Wien XIII/\textsubscript{7}\oindex{Ober Sankt Veit@\textbf{Ober Sankt Veit}|pw} den 11. 1. 07.\pend
           \pstart\center{}Lieber Arthur!\pend\pstart
           Ich war \label{K_L01650_1v}\edtext{vierzehn Tage}{\lemma{\textnormal{\emph{vierzehn Tage}}}\Cendnote{\textnormal{Nachweisbar war Bahr\pwindex{Bahr, Hermann 19.07.1863 – 15.01.1934@\textsc{Bahr, Hermann} (19.07.1863 – 15.01.1934), \emph{Schriftsteller, Kritiker}|pwk} am 2. und 4. 1. 1907 auf dem
                     Semmering\oindex{Semmering@\textbf{Semmering}|pwk}.}}}\label{K_L01650_1h} auf dem Semmering\oindex{Semmering@\textbf{Semmering}|pw} und bin nun seit Dienstag hier, für etwa zwölf Tage,
               mit dem Vorsatze:\pend
           \pstart
           1. Das \label{K_L01650_2v}\edtext{Regiebuch von Hedda Gabler\pwindex{Ibsen, Henrik 20.03.1828 – 23.05.1906@\textsc{Ibsen, Henrik} (20.03.1828 – 23.05.1906), \emph{Schriftsteller}!Hedda Gabler1891@\strich\emph{Hedda Gabler} {[}1891{]}|pw}}{\lemma{\textnormal{\emph{Regiebuch … Gabler}}}\Cendnote{\textnormal{in Bahrs\pwindex{Bahr, Hermann 19.07.1863 – 15.01.1934@\textsc{Bahr, Hermann} (19.07.1863 – 15.01.1934), \emph{Schriftsteller, Kritiker}|pwk} Nachlass (\emph{Theatermuseum Wien}, HS VM 3683 Ba), die Premiere
                  von Ibsen\pwindex{Ibsen, Henrik 20.03.1828 – 23.05.1906@\textsc{Ibsen, Henrik} (20.03.1828 – 23.05.1906), \emph{Schriftsteller}|pwk}s Stück\pwindex{Ibsen, Henrik 20.03.1828 – 23.05.1906@\textsc{Ibsen, Henrik} (20.03.1828 – 23.05.1906), \emph{Schriftsteller}!Hedda Gabler1891@\strich\emph{Hedda Gabler} {[}1891{]}|pwkv} am 11. 3. 1907}}}\label{K_L01650_2h} zu machen,
               deren Proben am 24. d. beginnen sollen.\pend
           \pstart
           \introOben{}2.\introOben{}{ }\substVorne{}\textsuperscript{z}\substDazwischen{}Z\substHinten{}u versuchen, ob mein neues Stück\pwindex{Bahr, Hermann 19.07.1863 – 15.01.1934@\textsc{Bahr, Hermann} (19.07.1863 – 15.01.1934), \emph{Schriftsteller, Kritiker}!gelbe Nachtigall1907@\strich\emph{Die gelbe Nachtigall} {[}1907{]}|pwv} schon so weit ist, dass sich mir ungefähr ein Szenarium ergibt,
               welches dann im Sommer ausgearbeitet werden soll, und 3. ein{\pb}mal mit Dir, Richard\pwindex{Beer-Hofmann, Richard 1866-07-11 – 1945-09-26@\textsc{Beer-Hofmann, Richard} (1866-07-11 – 1945-09-26), \emph{Schriftsteller}|pw} und Salten\pwindex{Salten, Felix 06.09.1869 – 08.10.1945@\textsc{Salten, Felix} (06.09.1869 – 08.10.1945), \emph{Schriftsteller, Journalist}|pw} zusammen zu sein,
               einmal mit Kainz\pwindex{Kainz, Margarethe 13.12.1858 – 12.02.1950@\textsc{Kainz, Margarethe} (13.12.1858 – 12.02.1950), \emph{Schauspielerin}|pw}, gelegentlich auch Fred\pwindex{W. Fred 29.06.1879 – 23.10.1922@\textsc{W. Fred} (29.06.1879 – 23.10.1922), \emph{Schriftsteller, Journalist}|pw} und Handl\pwindex{Handl, Willi 12.2.1872 – 26.5.1920@\textsc{Handl, Willi} (12.2.1872 – 26.5.1920), \emph{Schriftsteller, Journalist}|pw}
               zu sehen, sonst aber mich zu verstecken. Dies ist es was ich »incognito« nenne. Meine
               Absicht war, Dir vorzuschlagen, ob ich nicht nächste Woche einmal von \substVorne{}\textsuperscript{e}\substDazwischen{}E\substHinten{}ilf bis Drei bei Dir sein und dort vielleicht auch gleich Salten\pwindex{Salten, Felix 06.09.1869 – 08.10.1945@\textsc{Salten, Felix} (06.09.1869 – 08.10.1945), \emph{Schriftsteller, Journalist}|pw} und Richard\pwindex{Beer-Hofmann, Richard 1866-07-11 – 1945-09-26@\textsc{Beer-Hofmann, Richard} (1866-07-11 – 1945-09-26), \emph{Schriftsteller}|pw} treffen
               könnte. Dass Du nun aber Sonntag Vormittag zu mir kommen willst, ist mir sehr
               erwünscht, stört mich gar nicht, freut mich riesig (ich kann Dir nur nichts zu {\pb}essen geben, weil ich keine Köchin habe) und wir
               können dann alles Mögliche besprechen.\pend
           \pstart
           Mit den herzlichsten Grüssen an Deine liebe Frau\pwindex{Schnitzler, Olga 17.01.1882 – 13.01.1970@\textsc{Schnitzler, Olga} (17.01.1882 – 13.01.1970), \emph{Schauspielerin, Sängerin}|pwv}{\\[\baselineskip]}Dein alter{\\[\baselineskip]}\spacefill\mbox{{[}hs. Bahr:{]} Hermann}\pend
           \leftskip=0em{}\pstart
           \noindent{}{[}hs. Clarus:{]} PS.{\\}»Ringelspiel\pwindex{Bahr, Hermann 19.07.1863 – 15.01.1934@\textsc{Bahr, Hermann} (19.07.1863 – 15.01.1934), \emph{Schriftsteller, Kritiker}!Brief an Arthur SchnitzlerMai 1922@\strich\emph{Brief an Arthur Schnitzler} {[}Mai 1922{]}|pw}« und »Grotesken\pwindex{Bahr, Hermann 19.07.1863 – 15.01.1934@\textsc{Bahr, Hermann} (19.07.1863 – 15.01.1934), \emph{Schriftsteller, Kritiker}!Grotesken. Der Klub der Erloeser (Schauspiel in einem Akt) – Der  Faun – Die tiefe Natur (Ein Akt)1906@\strich\emph{Grotesken. Der Klub der Erlöser (Schauspiel in einem Akt) – Der Faun – Die tiefe Natur (Ein Akt)} {[}1906{]}|pw}« hast Du
                  hoffentlich richtig bekommen?\pend
           
         
         \endnumbering\mylabel{h}\end{ledgroupsized}  \newcommand{\dateiname}{L01650}\newcommand{\titel}{Hermann Bahr an Arthur Schnitzler, 11. 1. 1907}\newcommand{\editorInnen}{ Kurt Ifkovits,  Martin Anton Müller}%% latex-leseansicht-abspann.tex
%% Abspann für die Leseansicht.
%% Der Schalter \ifkorrekturansicht ist bereits durch den Vorspann gesetzt.

%% latex-abspann.tex
%% Gemeinsamer Abspann für Korrekturansicht und Leseansicht.
%% Setzt den Schalter \ifkorrekturansicht voraus (gesetzt in den
%% einbindenden Dateien latex-korrekturansicht-abspann.tex bzw.
%% latex-leseansicht-abspann.tex).
%% ---------------------------------------------------------------

\normalsize

% Das esempio-Environment wird nur in der Leseansicht benötigt
\ifkorrekturansicht\else
\newenvironment{esempio}[3]%
{
    \vspace{1.5ex}
    \rlap{\underline{#1}}
    \par
    \setlength{\parindent}{0cm}
    \nopagebreak
    \leftskip=#2cm
    \rightskip=#3cm
}
{
    \par
}
\fi

\doendnotes{C}
\bigskip
\vfill

\clearpage

\footnotesize

\ifkorrekturansicht
  \lohead{\textsc{register}}
\fi

% theindex-Environment neu definieren ohne reledmac
\makeatletter
\renewenvironment{theindex}{%
  \ifkorrekturansicht
    \section*{\indexname}%
  \else
    \subsubsection*{Index der erwähnten Entitäten}%
  \fi
  \setlength{\parindent}{0pt}%
  \setlength{\parskip}{0pt plus 0.3pt}%
  \let\item\@idxitem
}{%
  \ifkorrekturansicht\clearpage\fi
}
\makeatother

\IfFileExists{\jobname-pw.ind}{\input{\jobname-pw.ind}}{}

% Quellenangabe nur in der Leseansicht
\ifkorrekturansicht\else
% Fallback-Definitionen, falls die .tex-Datei \titel etc. nicht gesetzt hat
\providecommand{\titel}{}
\providecommand{\editorInnen}{}
\providecommand{\dateiname}{\jobname}

\vspace{3cm}

\vfill

\footnotesize
\textsc{Quelle}: \titel. Herausgegeben von {\editorInnen}. In: \emph{Arthur Schnitzler: Briefwechsel mit Autorinnen und Autoren}.
 Digitale Edition, https://schnitzler-briefe.acdh.oeaw.ac.at/{\dateiname}.html (Stand \today)
\fi

\end{document}


      