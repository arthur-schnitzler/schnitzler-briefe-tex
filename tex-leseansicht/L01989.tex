%% latex-leseansicht-vorspann.tex
%% Vorspann für die Leseansicht.
%% Lädt die gemeinsame Datei latex-vorspann.tex mit nicht gesetztem Schalter.

\newif\ifkorrekturansicht
\korrekturansichtfalse

\input{../tex-inputs/latex-vorspann}


         
         \renewcommand{\erwaehntePersonen}{Personen: Paul Goldmann, Hugo von Hofmannsthal}
         \renewcommand{\erwaehnteOrte}{Orte: München, Rodaun, Semmering, Wien}
         \renewcommand{\erwaehnteWerke}{Werke: Berliner Theater. »König Oedipus« im Zirkus Schumann, Der junge Medardus. Dramatische Historie in einem Vorspiel und fünf Aufzügen, Neue Freie Presse}
               \section[Hugo von Hofmannsthal an Arthur Schnitzler, 2. 12. 1910]{ Hugo von Hofmannsthal an Arthur Schnitzler, 2. 12. 1910}\nopagebreak\mylabel{v}\rehead{ }\begin{ledgroupsized}[t]{13cm}\normalsize\beginnumbering \toendnotes[C]{\smallbreak\pagebreak[2]} \Standort{CUL, Schnitzler, B 43.}
\physDesc{Briefkarte, 528 Zeichen
\newline{}Handschrift: schwarze Tinte, deutsche Kurrent
\newline{}Schnitzler: mit Bleistift beschriftet: »Hugo« 
\newline{}Ordnung: 1) mit Bleistift von unbekannter Hand nummeriert: »\strikeout{309}«  2) mit Bleistift von unbekannter Hand nummeriert:
                                    »327«}\buchAbdrucke{\weitereDrucke{Hugo von Hofmannsthal, Arthur Schnitzler: \emph{Briefwechsel}. Hg. Therese Nickl und Heinrich Schnitzler. Frankfurt am Main: \emph{S. Fischer} 1964, S. 260.} }\toendnotes[C]{\smallbreak}\pstart
           \raggedleft{}{\pb}Rodaun\oindex{Rodaun@\textbf{Rodaun}|pw}{ }2 XII. 10.\pend
           \pstart{}mein lieber Arthur\pend\pstart
           verzeihen Sie die elende Schlamperei, Ihnen bei \label{K_L01989-1v}\edtext{2 Begegnungen}{\lemma{\textnormal{\emph{2 Begegnungen}}}\Cendnote{\textnormal{siehe A. S.: \emph{Tagebuch}, 29. 11. 1910, 1. 12. 1910}}}\label{K_L01989-1h} das ausgelegte Geld für die 2 Plätze\pwindex{Schnitzler, Arthur 15.05.1862 – 21.10.1931@\textsc{Schnitzler, Arthur} (15.05.1862 – 21.10.1931), \emph{Schriftsteller, Mediziner}!junge Medardus. Dramatische Historie in einem Vorspiel und fuenf
                  Aufzuegen1910-10-26@\strich\emph{Der junge Medardus. Dramatische Historie in einem Vorspiel und fünf Aufzügen} {[}1910-10-26{]}|pwv} nicht rückerſtattet zu haben.\hspace*{1.5em}–
               Haben Sie gute Tage \label{K_L01989-2v}\edtext{in München\oindex{Muenchen@\textbf{München}|pw}}{\lemma{\textnormal{\emph{in München}}}\Cendnote{\textnormal{Von 8. 12. 1910 an war er für eine Vorlesung
                  eigener Stücke sowie einer Premiere mehrerer Einakter in München\oindex{Muenchen@\textbf{München}|pwk}.}}}\label{K_L01989-2h}.\hspace*{1.5em}Vielleicht
               verbringen wir {\pb}doch noch vor
               Weihnachten ein paar Tage auf dem Semmering\oindex{Semmering@\textbf{Semmering}|pw}, das
               wäre ſehr ſchön.\hspace*{1.5em}Daſs Sie in der \label{K_L01989-3v}\edtext{Goldmann\pwindex{Goldmann, Paul 31.01.1865 – 25.09.1935@\textsc{Goldmann, Paul} (31.01.1865 – 25.09.1935), \emph{Schriftsteller, Journalist}|pw}ſache}{\lemma{\textnormal{\emph{Goldmannſache}}}\Cendnote{\textnormal{Er ärgerte sich über das Feuilleton \emph{Berliner Theater. ›König Oedipus‹ im Zirkus Schumann}\pwindex{Goldmann, Paul 31.01.1865 – 25.09.1935@\textsc{Goldmann, Paul} (31.01.1865 – 25.09.1935), \emph{Schriftsteller, Journalist}!Berliner Theater. »Koenig Oedipus« im Zirkus Schumann1910-11-26@\strich\emph{Berliner Theater. »König Oedipus« im Zirkus Schumann} {[}1910-11-26{]}|pwk} (\emph{Neue Freie Presse}\pwindex{Neue Freie Presse1864 – 1939@\emph{Neue Freie Presse} {[}1864 – 1939{]}|pwk}, Nr. 16618,
                        26. 11. 1910, Morgenblatt, S. 1–3), vgl. A. S.: \emph{Tagebuch}, 1. 12. 1910}}}\label{K_L01989-3h} eine Unannehmlichkeit die hauptſächlich mich trifft, ſo ſtark fühlen, iſt mir
               unendlich woltuend, und für mich das einzig Reale an der läſtigen, aber eigentlich
                  \label{T_L01989-1v}\edtext{weſenloſen Angelegenheit}{\lemma{\textnormal{\emph{weſenloſen Angelegenheit}}}\Cendnote{\textnormal{ab hier weiter quer am linken
               Rand}}}\label{T_L01989-1h}.\pend
           \pstart Von Herzen Ihr\spacefill\mbox{Hugo.}\pend{}
         
         \endnumbering\mylabel{h}\end{ledgroupsized}  \newcommand{\dateiname}{L01989}\newcommand{\titel}{Hugo von Hofmannsthal an Arthur Schnitzler, 2. 12. 1910}\newcommand{\editorInnen}{Martin Anton Müller und Gerd-Hermann Susen}%% latex-leseansicht-abspann.tex
%% Abspann für die Leseansicht.
%% Der Schalter \ifkorrekturansicht ist bereits durch den Vorspann gesetzt.

%% latex-abspann.tex
%% Gemeinsamer Abspann für Korrekturansicht und Leseansicht.
%% Setzt den Schalter \ifkorrekturansicht voraus (gesetzt in den
%% einbindenden Dateien latex-korrekturansicht-abspann.tex bzw.
%% latex-leseansicht-abspann.tex).
%% ---------------------------------------------------------------

\normalsize

% Das esempio-Environment wird nur in der Leseansicht benötigt
\ifkorrekturansicht\else
\newenvironment{esempio}[3]%
{
    \vspace{1.5ex}
    \rlap{\underline{#1}}
    \par
    \setlength{\parindent}{0cm}
    \nopagebreak
    \leftskip=#2cm
    \rightskip=#3cm
}
{
    \par
}
\fi

\doendnotes{C}
\bigskip
\vfill

\clearpage

\footnotesize

\ifkorrekturansicht
  \lohead{\textsc{register}}
\fi

% theindex-Environment neu definieren ohne reledmac
\makeatletter
\renewenvironment{theindex}{%
  \ifkorrekturansicht
    \section*{\indexname}%
  \else
    \subsubsection*{Index der erwähnten Entitäten}%
  \fi
  \setlength{\parindent}{0pt}%
  \setlength{\parskip}{0pt plus 0.3pt}%
  \let\item\@idxitem
}{%
  \ifkorrekturansicht\clearpage\fi
}
\makeatother

\IfFileExists{\jobname-pw.ind}{\input{\jobname-pw.ind}}{}

% Quellenangabe nur in der Leseansicht
\ifkorrekturansicht\else
% Fallback-Definitionen, falls die .tex-Datei \titel etc. nicht gesetzt hat
\providecommand{\titel}{}
\providecommand{\editorInnen}{}
\providecommand{\dateiname}{\jobname}

\vspace{3cm}

\vfill

\footnotesize
\textsc{Quelle}: \titel. Herausgegeben von {\editorInnen}. In: \emph{Arthur Schnitzler: Briefwechsel mit Autorinnen und Autoren}.
 Digitale Edition, https://schnitzler-briefe.acdh.oeaw.ac.at/{\dateiname}.html (Stand \today)
\fi

\end{document}


      