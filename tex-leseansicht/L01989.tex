%% latex-korrekturansicht-vorspann.tex
%% Vorspann für die Korrekturansicht.
%% Lädt die gemeinsame Datei latex-vorspann.tex mit gesetztem Schalter.

\newif\ifkorrekturansicht
\korrekturansichttrue

\input{../tex-inputs/latex-vorspann}


\section[Hugo von Hofmannsthal an Arthur Schnitzler, 2. 12. 1910]{L01989 Hugo von Hofmannsthal an Arthur Schnitzler, 2. 12. 1910}
\nopagebreak\mylabel{L01989v}
\rehead{ }\normalsize\beginnumbering\briefempfaengerindex{Schnitzler, Arthur@\textsc{Schnitzler, Arthur}!zzzHofmannsthal, Hugo von@\emph{von Hugo von Hofmannsthal}!1910-12-021@{2. 12. 1910}|(be}
\toendnotes[C]{\smallbreak\pagebreak[2]}\Standort{CUL, Schnitzler, B 43.}
\physDesc{Briefkarte, 528 Zeichen
\newline{}Handschrift: schwarze Tinte, deutsche Kurrent
\newline{}Schnitzler: mit Bleistift beschriftet: »Hugo« 
\newline{}Ordnung: 1) mit Bleistift von unbekannter Hand nummeriert: »\strikeout{309}«  2) mit Bleistift von unbekannter Hand nummeriert:
                                    »327«}
\buchAbdrucke{\weitereDrucke{Hugo von Hofmannsthal, Arthur Schnitzler: \emph{Briefwechsel}. Frankfurt am Main: \emph{S. Fischer} 1964, S. 260.} }\toendnotes[C]{\smallbreak}
\pstart
           \raggedleft{}{\pb}Rodaun\oindex{Rodaun@\textbf{Rodaun}, \emph{A.ADM4}|pw}{ }2 XII. 10.\pend
           
\pstart{}mein lieber Arthur\pend\vspace{0.5em}
\pstart
           verzeihen Sie die elende Schlamperei, Ihnen bei \label{K_L01989-1v}\edtext{2 Begegnungen}{\lemma{\textnormal{\emph{2 Begegnungen}}}\Cendnote{\textnormal{Siehe A. S.: \emph{Tagebuch}, 29. 11. 1910 und 1. 12. 1910.
               }}}\label{K_L01989-1} das ausgelegte Geld für die 2 Plätze\pwindex{junge Medardus. Dramatische Historie in einem Vorspiel und fuenf Aufzuegen@\emph{Der junge Medardus. Dramatische Historie in einem Vorspiel und fünf Aufzügen}|pwv} nicht rückerſtattet zu haben.\hspace*{1.5em}–
               Haben Sie gute Tage \label{K_L01989-2v}\edtext{in München\oindex{Muenchen@\textbf{München}, \emph{P.PPLA}|pw}}{\lemma{\textnormal{\emph{in München}}}\Cendnote{\textnormal{Vom 8. 12. 1910 an war er für eine Vorlesung
                  eigener Stücke sowie einer Premiere mehrerer Einakter in München\oindex{Muenchen@\textbf{München}, \emph{P.PPLA}|pwk}.}}}\label{K_L01989-2}.\hspace*{1.5em}Vielleicht
               verbringen wir {\pb}doch noch vor
               Weihnachten ein paar Tage auf dem Semmering\oindex{Semmering@\textbf{Semmering}, \emph{A.ADM3}|pw}, das
               wäre ſehr ſchön.\hspace*{1.5em}Daſs Sie in der \label{K_L01989-3v}\edtext{Goldmann\pwindex{Goldmann, Paul 31.01.1865 – 25.09.1935@\textsc{Goldmann, Paul} (31.01.1865 – 25.09.1935), \emph{Schriftsteller/Schriftstellerin, Journalist/Journalistin}|pw}ſache}{\lemma{\textnormal{\emph{Goldmannſache}}}\Cendnote{\textnormal{Er ärgerte sich über das Feuilleton \emph{Berliner Theater. ›König Oedipus‹ im Zirkus Schumann}\pwindex{Berliner Theater. »Koenig Oedipus« im Zirkus Schumann@\emph{Berliner Theater. »König Oedipus« im Zirkus Schumann}|pwk} (\emph{Neue Freie Presse}\pwindex{Neue Freie Presse@\emph{Neue Freie Presse}|pwk}, Nr. 16.618,
                        26. 11. 1910, Morgenblatt, S. 1–3), vgl. A. S.: \emph{Tagebuch}, 1. 12. 1910.}}}\label{K_L01989-3} eine Unannehmlichkeit die hauptſächlich mich trifft, ſo ſtark fühlen, iſt mir
               unendlich woltuend, und für mich das einzig Reale an der läſtigen, aber eigentlich
                  \label{T_L01989-1v}\edtext{weſenloſen Angelegenheit}{\lemma{\textnormal{\emph{weſenloſen Angelegenheit}}}\Cendnote{\textnormal{ab hier weiter quer am linken
               Rand}}}\label{T_L01989-1}.\pend
           \pstart Von Herzen Ihr\spacefill\mbox{Hugo.}\pend{}\selectlanguage{ngerman}\endnumbering\briefempfaengerindex{Schnitzler, Arthur@\textsc{Schnitzler, Arthur}!zzzHofmannsthal, Hugo von@\emph{von Hugo von Hofmannsthal}!1910-12-021@{2. 12. 1910}|)be}\mylabel{L01989h}  \normalsize

\doendnotes{C}
\bigskip
\vfill

\clearpage

\footnotesize

\lohead{\textsc{register}}

% Definiere theindex-Environment komplett neu ohne reledmac
\makeatletter
\renewenvironment{theindex}{%
  \section*{\indexname}%
  \setlength{\parindent}{0pt}%
  \setlength{\parskip}{0pt plus 0.3pt}%
  \let\item\@idxitem
}{%
  \clearpage
}
\makeatother

\IfFileExists{\jobname-pw.ind}{\input{\jobname-pw.ind}}{}

\end{document}

      