%% latex-korrekturansicht-vorspann.tex
%% Vorspann für die Korrekturansicht.
%% Lädt die gemeinsame Datei latex-vorspann.tex mit gesetztem Schalter.

\newif\ifkorrekturansicht
\korrekturansichttrue

\input{../tex-inputs/latex-vorspann}


\section[Arthur Schnitzler an Felix Salten, 21. 3. 1892]{L02955 Arthur Schnitzler an Felix Salten, 21. 3. 1892}
\nopagebreak\mylabel{L02955v}
\rehead{ }\normalsize\beginnumbering\briefempfaengerindex{Salten, Felix@\textsc{Salten, Felix}!zzzSchnitzler, Arthur@\emph{von Arthur Schnitzler}!1892-03-211@{21. 3. 1892}|(be}
\toendnotes[C]{\smallbreak\pagebreak[2]}\Standort{Wienbibliothek im Rathaus, ZPH 1681, 2.1.516.}
\physDesc{Brief, 1 Blatt, 4 Seiten, 623 Zeichen
\newline{}Handschrift: Bleistift, deutsche Kurrent
\newline{}Ordnung: mit Bleistift von unbekannter Hand Nummerierung der Doppelseiten des
                                 Konvoluts: »84«–»85« }
\buchAbdrucke{\weitereDrucke{Arthur Schnitzler: \emph{Briefe 1875–1912}. Frankfurt am Main: \emph{S. Fischer} 1981, S. 123.} }\toendnotes[C]{\smallbreak}
\pstart
           \raggedleft{}{\pb}21/3 92{\\}Wien\oindex{Wien@\textbf{Wien}, \emph{A.ADM2}|pw}.\pend
           
\pstart{}Lieber Freund,\pend\vspace{0.5em}
\pstart
           \label{K_L02955-1v}\edtext{\textsc{Loris\pwindex{Hofmannsthal, Hugo von 1874-02-01 – 1929-07-15@\textsc{Hofmannsthal, Hugo von} (1874-02-01 – 1929-07-15), \emph{Schriftsteller/Schriftstellerin}|pw}} war Nachmittg bei mir}{\lemma{\textnormal{\emph{Loris … mir}}}\Cendnote{\textnormal{Siehe A. S.: \emph{Tagebuch}, 21. 3. 1892.
               }}}\label{K_L02955-1}. Hat beiliegenden \label{K_L02955-2v}\edtext{Brief}{\lemma{\textnormal{\emph{Brief}}}\Cendnote{\textnormal{Beilage nicht erhalten}}}\label{K_L02955-2} erhalten,
               welchen er Sie zu erledigen bittet. – Zugleich erſucht er Sie um ſeine \textsc{\label{K_L02955-3v}\edtext{Distichen}{\lemma{\textnormal{\emph{Distichen}}}\Cendnote{\textnormal{Ende Juli 1891 hatte Hofmannsthal\pwindex{Hofmannsthal, Hugo von 1874-02-01 – 1929-07-15@\textsc{Hofmannsthal, Hugo von} (1874-02-01 – 1929-07-15), \emph{Schriftsteller/Schriftstellerin}|pwk} an Salten\pwindex{Salten, Felix 06.09.1869 – 08.10.1945@\textsc{Salten, Felix} (06.09.1869 – 08.10.1945), \emph{Schriftsteller/Schriftstellerin, Journalist/Journalistin, Chefredakteur/Chefredakteurin}|pwk}{ }\emph{Vielfarbige Distichen V}\pwindex{Vielfarbige Distichen V.@\emph{Vielfarbige Distichen V.}|pwk} gesandt.
                           (Hugo von Hofmannsthal: \emph{Brief-Chronik.
                              Regest-Ausgabe}. Herausgegeben von Martin E. Schmid. Band 1: 1874–1911.
                           Heidelberg: \emph{Winter}{ }2003, S. 21.)}}}\label{K_L02955-3}\pwindex{Vielfarbige Distichen V.@\emph{Vielfarbige Distichen V.}|pw}}, von denen er kein \textsc{Duplium} beſitzt. Dann, we{\geminationn} Sie’s {\pb}nicht
               etwa ſelber verliehen haben, die \label{K_L02955-4v}\edtext{\textsc{Bilanz der Ehe\pwindex{Bilanz der Ehe. Novellistische Studien. 2 Bde.@\emph{Die Bilanz der Ehe. Novellistische Studien. 2 Bde.}|pw}}}{\lemma{\textnormal{\emph{Bilanz der Ehe}}}\Cendnote{\textnormal{Gustav Schwarzkopf\pwindex{Schwarzkopf, Gustav 07.11.1853 – 13.11.1939@\textsc{Schwarzkopf, Gustav} (07.11.1853 – 13.11.1939), \emph{Schriftsteller/Schriftstellerin}|pwk}: \emph{Bilanz der Ehe. Novellistische Studien}\pwindex{Bilanz der Ehe. Novellistische Studien. 2 Bde.@\emph{Die Bilanz der Ehe. Novellistische Studien. 2 Bde.}|pwk}. 2 Bde. Dresden\oindex{Dresden@\textbf{Dresden}, \emph{P.PPLA}|pwk}/Leipzig\oindex{Leipzig@\textbf{Leipzig}, \emph{P.PPLA3}|pwk}: \emph{Heinrich Minden}\orgindex{Heinrich Minden@Heinrich Minden|pwk}{ }1885.}}}\label{K_L02955-4}. –\pend
           
\pstart
           Er ſchickt mit größter Eile den Tod des Tizian\pwindex{Tod des Tizian. Ein Bruchstueck@\emph{Der Tod des Tizian. Ein Bruchstück}|pw}
               als Fragment an die neue \label{K_L02955-5v}\edtext{\textsc{Henze\pwindex{Henze, Max 26.01.1871 – 25.11.1903@\textsc{Henze, Max} (26.01.1871 – 25.11.1903), \emph{Journalist/Journalistin, Schauspieler/Schauspielerin}|pw}}’ſche Zeitung\orgindex{Allgemeine Theater-Revue fuer Buehne und Welt. Illustrierte Halbmonatsschrift@Allgemeine Theater-Revue für Bühne und Welt. Illustrierte Halbmonatsschrift|pwv}}{\lemma{\textnormal{\emph{Henze’ſche Zeitung}}}\Cendnote{\textnormal{Das Dramenfragment\pwindex{Tod des Tizian. Ein Bruchstueck@\emph{Der Tod des Tizian. Ein Bruchstück}|pwkv} erschien schließlich in Stefan Georges\pwindex{George, Stefan 17.07.1868 – 04.12.1933@\textsc{George, Stefan} (17.07.1868 – 04.12.1933), \emph{Schriftsteller/Schriftstellerin, Übersetzer/Übersetzerin}|pwk}{ }\emph{Blätter für
                     die Kunst}\pwindex{Blaetter fuer die Kunst@\emph{Blätter für die Kunst}|pwk}: Hugo von Hofmannsthal\pwindex{Hofmannsthal, Hugo von 1874-02-01 – 1929-07-15@\textsc{Hofmannsthal, Hugo von} (1874-02-01 – 1929-07-15), \emph{Schriftsteller/Schriftstellerin}|pwk}: \emph{Der Tod des Tizian. Ein Bruchstück}\pwindex{Tod des Tizian. Ein Bruchstueck@\emph{Der Tod des Tizian. Ein Bruchstück}|pwk}. In: \emph{Blätter für die Kunst}\pwindex{Blaetter fuer die Kunst@\emph{Blätter für die Kunst}|pwk}, Jg. 1, H. 1, Oktober 1892, S. 12–24.}}}\label{K_L02955-5}{ }\textsc{Berlin\oindex{Berlin@\textbf{Berlin}, \emph{P.PPLC}|pw}}, las ihn mir heute{ }Nachmittag vor. – Schön – ! Na, wir {\pb}reden bald drüber, hoffentlich beko{\geminationm}en Sie’s bald zu leſen; ſchade daſs Sie’s heut nicht gehört haben.\pend
           
\pstart
           – Ich ko{\geminationm}e, we{\geminationn} nicht früher, \label{K_L02955-66v}\edtext{\substVorne{}\textsuperscript{Fre}\substDazwischen{}\textsc{Do{\geminationn}}\substHinten{}\textsc{erstag}{ }Abend ins \textsc{Central\oindex{Cafe Central@\textbf{Café Central}, \emph{Kaffeehaus (K.KAF)}|pw}}}{\lemma{\textnormal{\emph{Donnerstag … Central}}}\Cendnote{\textnormal{Nicht im \emph{Tagebuch}\pwindex{Tagebuch@\emph{Tagebuch}|pwk}. Zumindest ein
                   Indiz gibt diese Stelle, dass Schnitzler seine Kaffeehausbesuche in der
                  Nacht nur dann ansetzte, wenn er am Folgetag keine Ordination hielt.}}}\label{K_L02955-66}
                (Freitg iſt nämlich \label{K_L02955-6v}\edtext{Feiertag}{\lemma{\textnormal{\emph{Feiertag}}}\Cendnote{\textnormal{Mariä
                  Verkündigung / Verkündigung des Herrn}}}\label{K_L02955-6}.) \pend
           
\pstart
           Herzlichſt {\pb}der Ihre {\\[\baselineskip]}\spacefill\mbox{ArthSch}\pend
           \leftskip=0em{}\selectlanguage{ngerman}\endnumbering\briefempfaengerindex{Salten, Felix@\textsc{Salten, Felix}!zzzSchnitzler, Arthur@\emph{von Arthur Schnitzler}!1892-03-211@{21. 3. 1892}|)be}\mylabel{L02955h}  \normalsize

\doendnotes{C}
\bigskip
\vfill

\clearpage

\footnotesize

\lohead{\textsc{register}}

% Definiere theindex-Environment komplett neu ohne reledmac
\makeatletter
\renewenvironment{theindex}{%
  \section*{\indexname}%
  \setlength{\parindent}{0pt}%
  \setlength{\parskip}{0pt plus 0.3pt}%
  \let\item\@idxitem
}{%
  \clearpage
}
\makeatother

\IfFileExists{\jobname-pw.ind}{\input{\jobname-pw.ind}}{}

\end{document}

      