%% latex-leseansicht-vorspann.tex
%% Vorspann für die Leseansicht.
%% Lädt die gemeinsame Datei latex-vorspann.tex mit nicht gesetztem Schalter.

\newif\ifkorrekturansicht
\korrekturansichtfalse

\input{../tex-inputs/latex-vorspann}


         
         \renewcommand{\erwaehntePersonen}{Personen: Stefan George, Max Henze, Hugo von Hofmannsthal,  Jesus, Felix Salten, Gustav Schwarzkopf}
         \renewcommand{\erwaehnteInstitutionen}{Institutionen: Allgemeine Theater-Revue für Bühne und Welt. Illustrierte Halbmonatsschrift, Heinrich Minden}
         \renewcommand{\erwaehnteOrte}{Orte: Berlin, Café Central, Dresden, Leipzig, Wien}
         \renewcommand{\erwaehnteWerke}{Werke: Blätter für die Kunst, Der Tod des Tizian. Ein Bruchstück, Die Bilanz der Ehe. Novellistische Studien. 2 Bde., Vielfarbige Distichen V.}
               \section[Arthur Schnitzler an Felix Salten, 21. 3. 1892]{ Arthur Schnitzler an Felix Salten, 21. 3. 1892}\nopagebreak\mylabel{v}\rehead{ }\begin{ledgroupsized}[t]{13cm}\normalsize\beginnumbering\briefempfaengerindex{Salten, Felix@\textsc{Salten, Felix}!zzzSchnitzler, Arthur@\emph{von Arthur Schnitzler}!1892-03-211@{21. 3. 1892}|(be} \toendnotes[C]{\smallbreak\pagebreak[2]} \Standort{Wienbibliothek im Rathaus, ZPH 1681, 2.1.516.}
\physDesc{Brief, 1 Blatt, 4 Seiten, 623 Zeichen
\newline{}Handschrift: Bleistift, deutsche Kurrent
\newline{}Ordnung: mit Bleistift von unbekannter Hand Nummerierung der Doppelseiten des
                                 Konvoluts: »84«–»85« }\buchAbdrucke{\weitereDrucke{Arthur Schnitzler: \emph{Briefe 1875–1912}. Hg. Therese Nickl und Heinrich Schnitzler. Frankfurt am Main: \emph{S. Fischer} 1981, S. 123.} }\toendnotes[C]{\smallbreak}\pstart
           \raggedleft{}{\pb}21/3 92{\\}Wien\oindex{Wien@\textbf{Wien}|pw}.\pend
           \pstart{}Lieber Freund,\pend\pstart
           \label{K_L02955-1v}\edtext{\textsc{Loris\pwindex{Hofmannsthal, Hugo von 1874-02-01 – 1929-07-15@\textsc{Hofmannsthal, Hugo von} (1874-02-01 – 1929-07-15), \emph{Schriftsteller}|pw}} war Nachmittg bei mir}{\lemma{\textnormal{\emph{Loris … mir}}}\Cendnote{\textnormal{Siehe A. S.: \emph{Tagebuch}, 21. 3. 1892.
               }}}\label{K_L02955-1h}. Hat beiliegenden \label{K_L02955-2v}\edtext{Brief}{\lemma{\textnormal{\emph{Brief}}}\Cendnote{\textnormal{Beilage nicht erhalten}}}\label{K_L02955-2h} erhalten,
               welchen er Sie zu erledigen bittet. – Zugleich erſucht er Sie um ſeine \textsc{\label{K_L02955-3v}\edtext{Distichen}{\lemma{\textnormal{\emph{Distichen}}}\Cendnote{\textnormal{Ende Juli 1891 hatte Hofmannsthal\pwindex{Hofmannsthal, Hugo von 1874-02-01 – 1929-07-15@\textsc{Hofmannsthal, Hugo von} (1874-02-01 – 1929-07-15), \emph{Schriftsteller}|pwk} an Salten\pwindex{Salten, Felix 06.09.1869 – 08.10.1945@\textsc{Salten, Felix} (06.09.1869 – 08.10.1945), \emph{Schriftsteller, Journalist}|pwk}{ }\emph{Vielfarbige Distichen V}\pwindex{Hofmannsthal, Hugo von 1874-02-01 – 1929-07-15@\textsc{Hofmannsthal, Hugo von} (1874-02-01 – 1929-07-15), \emph{Schriftsteller}!Vielfarbige Distichen V.1940@\strich\emph{Vielfarbige Distichen V.} {[}1940{]}|pwk} gesandt.
                           (Hugo von Hofmannsthal: \emph{Brief-Chronik.
                              Regest-Ausgabe}. Herausgegeben von Martin E. Schmid. Band 1: 1874–1911.
                           Heidelberg: \emph{Winter}{ }2003, S. 21.)}}}\label{K_L02955-3h}\pwindex{Hofmannsthal, Hugo von 1874-02-01 – 1929-07-15@\textsc{Hofmannsthal, Hugo von} (1874-02-01 – 1929-07-15), \emph{Schriftsteller}!Vielfarbige Distichen V.1940@\strich\emph{Vielfarbige Distichen V.} {[}1940{]}|pw}}, von denen er kein \textsc{Duplium} beſitzt. Dann, we{\geminationn} Sie’s {\pb}nicht
               etwa ſelber verliehen haben, die \label{K_L02955-4v}\edtext{\textsc{Bilanz der Ehe\pwindex{Schwarzkopf, Gustav 07.11.1853 – 13.11.1939@\textsc{Schwarzkopf, Gustav} (07.11.1853 – 13.11.1939), \emph{Schriftsteller}!Bilanz der Ehe. Novellistische Studien. 2 Bde.1885-05-16 – 1885-12-14@\strich\emph{Die Bilanz der Ehe. Novellistische Studien. 2 Bde.} {[}1885-05-16 – 1885-12-14{]}|pw}}}{\lemma{\textnormal{\emph{Bilanz der Ehe}}}\Cendnote{\textnormal{Gustav Schwarzkopf\pwindex{Schwarzkopf, Gustav 07.11.1853 – 13.11.1939@\textsc{Schwarzkopf, Gustav} (07.11.1853 – 13.11.1939), \emph{Schriftsteller}|pwk}: \emph{Bilanz der Ehe. Novellistische Studien}\pwindex{Schwarzkopf, Gustav 07.11.1853 – 13.11.1939@\textsc{Schwarzkopf, Gustav} (07.11.1853 – 13.11.1939), \emph{Schriftsteller}!Bilanz der Ehe. Novellistische Studien. 2 Bde.1885-05-16 – 1885-12-14@\strich\emph{Die Bilanz der Ehe. Novellistische Studien. 2 Bde.} {[}1885-05-16 – 1885-12-14{]}|pwk}. 2 Bde. Dresden\oindex{Dresden@\textbf{Dresden}|pwk}/Leipzig\oindex{Leipzig@\textbf{Leipzig}|pwk}: \emph{Heinrich Minden}\orgindex{Heinrich Minden@Heinrich Minden|pwk}{ }1885.}}}\label{K_L02955-4h}. –\pend
           \pstart
           Er ſchickt mit größter Eile den Tod des Tizian\pwindex{Hofmannsthal, Hugo von 1874-02-01 – 1929-07-15@\textsc{Hofmannsthal, Hugo von} (1874-02-01 – 1929-07-15), \emph{Schriftsteller}!Tod des Tizian. Ein BruchstueckOktober 1892@\strich\emph{Der Tod des Tizian. Ein Bruchstück} {[}Oktober 1892{]}|pw}
               als Fragment an die neue \label{K_L02955-5v}\edtext{\textsc{Henze\pwindex{Henze, Max 26.01.1871 – 25.11.1903@\textsc{Henze, Max} (26.01.1871 – 25.11.1903), \emph{Journalist, Schauspieler}|pw}}’ſche Zeitung\orgindex{Allgemeine Theater-Revue fuer Buehne und Welt. Illustrierte Halbmonatsschrift@Allgemeine Theater-Revue für Bühne und Welt. Illustrierte Halbmonatsschrift|pwv}}{\lemma{\textnormal{\emph{Henze’ſche Zeitung}}}\Cendnote{\textnormal{Das Dramenfragment\pwindex{Hofmannsthal, Hugo von 1874-02-01 – 1929-07-15@\textsc{Hofmannsthal, Hugo von} (1874-02-01 – 1929-07-15), \emph{Schriftsteller}!Tod des Tizian. Ein BruchstueckOktober 1892@\strich\emph{Der Tod des Tizian. Ein Bruchstück} {[}Oktober 1892{]}|pwkv} erschien schließlich in Stefan Georges\pwindex{George, Stefan 17.07.1868 – 04.12.1933@\textsc{George, Stefan} (17.07.1868 – 04.12.1933), \emph{Schriftsteller, Übersetzer}|pwk}{ }\emph{Blätter für
                     die Kunst}\pwindex{?? Werk@Nicht ermittelte Verfasserinnen und Verfasser!Blaetter fuer die Kunst1892 – 1919@\emph{Blätter für die Kunst} {[}1892 – 1919{]}|pwk}: Hugo von Hofmannsthal\pwindex{Hofmannsthal, Hugo von 1874-02-01 – 1929-07-15@\textsc{Hofmannsthal, Hugo von} (1874-02-01 – 1929-07-15), \emph{Schriftsteller}|pwk}: \emph{Der Tod des Tizian. Ein Bruchstück}\pwindex{Hofmannsthal, Hugo von 1874-02-01 – 1929-07-15@\textsc{Hofmannsthal, Hugo von} (1874-02-01 – 1929-07-15), \emph{Schriftsteller}!Tod des Tizian. Ein BruchstueckOktober 1892@\strich\emph{Der Tod des Tizian. Ein Bruchstück} {[}Oktober 1892{]}|pwk}. In: \emph{Blätter für die Kunst}\pwindex{?? Werk@Nicht ermittelte Verfasserinnen und Verfasser!Blaetter fuer die Kunst1892 – 1919@\emph{Blätter für die Kunst} {[}1892 – 1919{]}|pwk}, Jg. 1, H. 1, Oktober 1892, S. 12–24.}}}\label{K_L02955-5h}{ }\textsc{Berlin\oindex{Berlin@\textbf{Berlin}|pw}}, las ihn mir heute{ }Nachmittag vor. – Schön – ! Na, wir {\pb}reden bald drüber, hoffentlich beko{\geminationm}en Sie’s bald zu leſen; ſchade daſs Sie’s heut nicht gehört haben.\pend
           \pstart
           – Ich ko{\geminationm}e, we{\geminationn} nicht früher, \label{K_L02955-66v}\edtext{\substVorne{}\textsuperscript{Fre}\substDazwischen{}\textsc{Do{\geminationn}}\substHinten{}\textsc{erstag}{ }Abend ins \textsc{Central\oindex{Cafe Central@\textbf{Café Central}|pw}}}{\lemma{\textnormal{\emph{Donnerstag … Central}}}\Cendnote{\textnormal{Nicht im \emph{Tagebuch}\textcolor{red}{\textsuperscript{XXXX indx}}. Zumindest ein
                   Indiz gibt diese Stelle, dass Schnitzler\pwindex{Schnitzler, Arthur 15.05.1862 – 21.10.1931@\textsc{Schnitzler, Arthur} (15.05.1862 – 21.10.1931), \emph{Schriftsteller, Mediziner}|pwk} seine Kaffeehausbesuche in der
                  Nacht nur dann ansetzte, wenn er am Folgetag keine Ordination hielt.}}}\label{K_L02955-66h}
                (Freitg iſt nämlich \label{K_L02955-6v}\edtext{Feiertag}{\lemma{\textnormal{\emph{Feiertag}}}\Cendnote{\textnormal{Mariä
                  Verkündigung / Verkündigung des Herrn}}}\label{K_L02955-6h}.) \pend
           \pstart
           Herzlichſt {\pb}der Ihre {\\[\baselineskip]}\spacefill\mbox{ArthSch}\pend
           \leftskip=0em{}
         
         \endnumbering\mylabel{h}\end{ledgroupsized}  \newcommand{\dateiname}{L02955}\newcommand{\titel}{Arthur Schnitzler an Felix Salten, 21. 3. 1892}\newcommand{\editorInnen}{Martin Anton Müller und Laura Untner}%% latex-leseansicht-abspann.tex
%% Abspann für die Leseansicht.
%% Der Schalter \ifkorrekturansicht ist bereits durch den Vorspann gesetzt.

%% latex-abspann.tex
%% Gemeinsamer Abspann für Korrekturansicht und Leseansicht.
%% Setzt den Schalter \ifkorrekturansicht voraus (gesetzt in den
%% einbindenden Dateien latex-korrekturansicht-abspann.tex bzw.
%% latex-leseansicht-abspann.tex).
%% ---------------------------------------------------------------

\normalsize

% Das esempio-Environment wird nur in der Leseansicht benötigt
\ifkorrekturansicht\else
\newenvironment{esempio}[3]%
{
    \vspace{1.5ex}
    \rlap{\underline{#1}}
    \par
    \setlength{\parindent}{0cm}
    \nopagebreak
    \leftskip=#2cm
    \rightskip=#3cm
}
{
    \par
}
\fi

\doendnotes{C}
\bigskip
\vfill

\clearpage

\footnotesize

\ifkorrekturansicht
  \lohead{\textsc{register}}
\fi

% theindex-Environment neu definieren ohne reledmac
\makeatletter
\renewenvironment{theindex}{%
  \ifkorrekturansicht
    \section*{\indexname}%
  \else
    \subsubsection*{Index der erwähnten Entitäten}%
  \fi
  \setlength{\parindent}{0pt}%
  \setlength{\parskip}{0pt plus 0.3pt}%
  \let\item\@idxitem
}{%
  \ifkorrekturansicht\clearpage\fi
}
\makeatother

\IfFileExists{\jobname-pw.ind}{\input{\jobname-pw.ind}}{}

% Quellenangabe nur in der Leseansicht
\ifkorrekturansicht\else
% Fallback-Definitionen, falls die .tex-Datei \titel etc. nicht gesetzt hat
\providecommand{\titel}{}
\providecommand{\editorInnen}{}
\providecommand{\dateiname}{\jobname}

\vspace{3cm}

\vfill

\footnotesize
\textsc{Quelle}: \titel. Herausgegeben von {\editorInnen}. In: \emph{Arthur Schnitzler: Briefwechsel mit Autorinnen und Autoren}.
 Digitale Edition, https://schnitzler-briefe.acdh.oeaw.ac.at/{\dateiname}.html (Stand \today)
\fi

\end{document}


      