%% latex-leseansicht-vorspann.tex
%% Vorspann für die Leseansicht.
%% Lädt die gemeinsame Datei latex-vorspann.tex mit nicht gesetztem Schalter.

\newif\ifkorrekturansicht
\korrekturansichtfalse

\input{../tex-inputs/latex-vorspann}

\begin{center}
            \textcolor{red}{ENTWURF, NICHT FERTIG KORRIGIERT}
                      \end{center}
            
         
         \renewcommand{\erwaehntePersonen}{Personen: Max Henze, Hugo von Hofmannsthal, Felix Salten}
         \renewcommand{\erwaehnteInstitutionen}{Institutionen: Allgemeine Theater-Revue für Bühne und Welt. Illustrierte Halbmonatsschrift}
         \renewcommand{\erwaehnteOrte}{Orte: Berlin, Café Central, Wien}
         \renewcommand{\erwaehnteWerke}{Werke: Der Tod des Tizian}
               \section[Arthur Schnitzler an Felix Salten, 21. 3. 1892]{ Arthur Schnitzler an Felix Salten, 21. 3. 1892}\nopagebreak\mylabel{v}\rehead{ }\begin{ledgroupsized}[t]{13cm}\normalsize\beginnumbering \toendnotes[C]{\smallbreak\pagebreak[2]} \Standort{Wienbibliothek im Rathaus, ZPH 1681, 2.1.516.}
\physDesc{
\newline{}Handschrift: , deutsche Kurrent}\toendnotes[C]{\smallbreak}\pstart
           \raggedleft{}{\pb}21/3 92{\\}Wien\oindex{Wien@\textbf{Wien}|pw}. \pend
           \pstart{}Lieber Freund,\pend\pstart
           \textsc{Loris\pwindex{Hofmannsthal, Hugo von 1874-02-01 – 1929-07-15@\textsc{Hofmannsthal, Hugo von} (1874-02-01 – 1929-07-15), \emph{Schriftsteller}|pw}} war Nachmittg bei mir. Hat beiliegenden \label{K_L02955-1v}\edtext{Brief}{\lemma{\textnormal{\emph{Brief}}}\Cendnote{\textnormal{Beilage
                  nicht erhalten}}}\label{K_L02955-1h} erhalten, welchen er Sie zu erledigen bittet.– Zugleich
               erſucht er Sie um ſeine \textsc{Distichen\textcolor{red}{\textsuperscript{\textbf{KEY}}}}, von denen er kein \textsc{Duplium} beſitzt. Drum, we{\geminationn} Sie’s {\pb}nicht etwa ſelber verliehen haben,
               die \textsc{Bilanz der Ehe\textcolor{red}{\textsuperscript{\textbf{KEY}}}}.– \pend
           \pstart
           Er ſchickt mit größter Eile den Tod des Tizian\pwindex{Hofmannsthal, Hugo von 1874-02-01 – 1929-07-15@\textsc{Hofmannsthal, Hugo von} (1874-02-01 – 1929-07-15), \emph{Schriftsteller}!Tod des TizianOktober 1892@\strich\emph{Der Tod des Tizian} {[}Oktober 1892{]}|pw}
               als Fragment an die neue \textsc{Henze\pwindex{Henze, Max 26.01.1871 – 25.11.1903@\textsc{Henze, Max} (26.01.1871 – 25.11.1903), \emph{Journalist, Schauspieler}|pw}}’ſche Zeitung\orgindex{Allgemeine Theater-Revue fuer Buehne und Welt. Illustrierte Halbmonatsschrift@Allgemeine Theater-Revue für Bühne und Welt. Illustrierte Halbmonatsschrift|pwv}\textsc{Berlin\oindex{Berlin@\textbf{Berlin}|pw}}, las ihn mir heute Nachmittag vor. – Schön – ! Na, wir {\pb}reden bald drüber, hoffentlich
                  beko{\geminationm}en Sie’s bald zu leſen; ſchade daſs Sie’s heut
               nicht gehört haben. – Ich ko{\geminationm}e, wēn nicht früher, \textsc{Do{\geminationn}erstag} Abend ins \textsc{Central\oindex{Cafe Central@\textbf{Café Central}|pw}} (Freitg iſt nämlich \label{K_L02955-12v}\edtext{Feiertag}{\lemma{\textnormal{\emph{Feiertag}}}\Cendnote{\textnormal{XXXX}}}\label{K_L02955-12h}.) \pend
           \pstart
           Herzlichſt {\pb}der Ihre {\\[\baselineskip]}\spacefill\mbox{ArthSch}\pend
           \leftskip=0em{}
         
         \endnumbering\mylabel{h}\end{ledgroupsized}\begin{anhang}\end{anhang}\newcommand{\dateiname}{L02955}\newcommand{\titel}{Arthur Schnitzler an Felix Salten, 21. 3. 1892}\newcommand{\editorInnen}{Martin Anton Müller und Laura Untner}%% latex-leseansicht-abspann.tex
%% Abspann für die Leseansicht.
%% Der Schalter \ifkorrekturansicht ist bereits durch den Vorspann gesetzt.

%% latex-abspann.tex
%% Gemeinsamer Abspann für Korrekturansicht und Leseansicht.
%% Setzt den Schalter \ifkorrekturansicht voraus (gesetzt in den
%% einbindenden Dateien latex-korrekturansicht-abspann.tex bzw.
%% latex-leseansicht-abspann.tex).
%% ---------------------------------------------------------------

\normalsize

% Das esempio-Environment wird nur in der Leseansicht benötigt
\ifkorrekturansicht\else
\newenvironment{esempio}[3]%
{
    \vspace{1.5ex}
    \rlap{\underline{#1}}
    \par
    \setlength{\parindent}{0cm}
    \nopagebreak
    \leftskip=#2cm
    \rightskip=#3cm
}
{
    \par
}
\fi

\doendnotes{C}
\bigskip
\vfill

\clearpage

\footnotesize

\ifkorrekturansicht
  \lohead{\textsc{register}}
\fi

% theindex-Environment neu definieren ohne reledmac
\makeatletter
\renewenvironment{theindex}{%
  \ifkorrekturansicht
    \section*{\indexname}%
  \else
    \subsubsection*{Index der erwähnten Entitäten}%
  \fi
  \setlength{\parindent}{0pt}%
  \setlength{\parskip}{0pt plus 0.3pt}%
  \let\item\@idxitem
}{%
  \ifkorrekturansicht\clearpage\fi
}
\makeatother

\IfFileExists{\jobname-pw.ind}{\input{\jobname-pw.ind}}{}

% Quellenangabe nur in der Leseansicht
\ifkorrekturansicht\else
% Fallback-Definitionen, falls die .tex-Datei \titel etc. nicht gesetzt hat
\providecommand{\titel}{}
\providecommand{\editorInnen}{}
\providecommand{\dateiname}{\jobname}

\vspace{3cm}

\vfill

\footnotesize
\textsc{Quelle}: \titel. Herausgegeben von {\editorInnen}. In: \emph{Arthur Schnitzler: Briefwechsel mit Autorinnen und Autoren}.
 Digitale Edition, https://schnitzler-briefe.acdh.oeaw.ac.at/{\dateiname}.html (Stand \today)
\fi

\end{document}


      