%% latex-leseansicht-vorspann.tex
%% Vorspann für die Leseansicht.
%% Lädt die gemeinsame Datei latex-vorspann.tex mit nicht gesetztem Schalter.

\newif\ifkorrekturansicht
\korrekturansichtfalse

\input{../tex-inputs/latex-vorspann}


         
         \newcommand{\erwaehntePersonen}{Personen: Paula Beer-Hofmann, Mirjam Beer-Hofmann, Paul Goldmann, Hugo von Hofmannsthal, Gustav Schwarzkopf}
         \newcommand{\erwaehnteOrte}{Orte: Bad Ischl, Bayreuth, Traun, Wien}
         \newcommand{\erwaehnteWerke}{Werke: Der grüne Kakadu. Groteske in einem Akt, Faust. Eine Tragödie}
               \section[Richard Beer-Hofmann an Arthur Schnitzler, 13. 6. 1897]{ Richard Beer-Hofmann an Arthur Schnitzler,
               13. 6. 1897}\nopagebreak\mylabel{v}\rehead{ }\begin{ledgroupsized}[t]{13cm}\normalsize\beginnumbering \toendnotes[C]{\smallbreak\pagebreak[2]} \Standort{CUL, Schnitzler, B 8.}
\physDesc{Brief, 3 Blätter, 9 Seiten
\newline{}Handschrift: blauer Buntstift, lateinische Kurrent\newline{}Ordnung: mit Bleistift von unbekannter Hand nummeriert: »99« }\buchAbdrucke{\weitereDrucke{Arthur Schnitzler, Richard Beer-Hofmann: \emph{Briefwechsel 1891–1931}. Hg. Konstanze Fliedl. Wien, Zürich: \emph{Europaverlag} 1992, S. 109–110.} }\toendnotes[C]{\smallbreak}\pstart
           \centering{}{\pb}Ischl\oindex{Bad Ischl@\textbf{Bad Ischl}|pw}\hspace*{1.5em}13/VI 97\pend
           \pstart
           Lieber Arthur, ich weiß noch gar nichts wegen Bayreuth\oindex{Bayreuth@\textbf{Bayreuth}|pw}, und will mich nicht entschließen.\pend
           \pstart
           Ihr Brief ist wieder so unleserlich! An \uline{was} arbeiten
               Sie? An einem Stück – da Sie von Scenen sprechen aber soll das \strikeout{»}Unleser{\pb}liche »Revolutionsstück\pwindex{Schnitzler, Arthur 15.05.1862 – 21.10.1931@\textsc{Schnitzler, Arthur} (15.05.1862 – 21.10.1931), \emph{Schriftsteller, Mediziner}!gruene Kakadu. Groteske in einem Akt1. 3. 1899@\strich\emph{Der grüne Kakadu. Groteske in einem Akt} {[}1. 3. 1899{]}|pwuv}« heißen?\pend
           \pstart
           Ob mich’s mit »ahnungsvoller Gegenwart
                  ängstigt\pwindex{\textcolor{red}{\textsuperscript{XXXX1 indx}}!Faust. Eine Tragoedie1808@\strich\emph{Faust. Eine Tragödie} {[}1808{]}|pwv}«? fragen Sie? In mir wird so Vieles jetzt Anders als es bis her war
               daß ich nicht weiß wie viel auf Rech{\pb}nung »\uline{davon}« zu setzen ist. Manchmal hab ich die
               Empfindung als würde ich im Herbst nicht »Vater« sondern »Großvater« wenn ich sehe
               wie kindisch und jung noch Paula\pwindex{Beer-Hofmann, Paula 25.02.1879 – 30.10.1939@\textsc{Beer-Hofmann, Paula} (25.02.1879 – 30.10.1939)|pw} ist, und dann
               muß ich wieder {\pb}über mich lachen
               mit meiner Neigung die Dinge zu leicht oder zu schwer zu nehmen. Augenblicklich
               sitzen wir – das ist Paula\pwindex{Beer-Hofmann, Paula 25.02.1879 – 30.10.1939@\textsc{Beer-Hofmann, Paula} (25.02.1879 – 30.10.1939)|pw}, und ich, und die ko{\geminationm}ende
                  Generation\pwindex{Beer-Hofmann, Mirjam 04.09.1897 – 24.12.1984@\textsc{Beer-Hofmann, Mirjam} (04.09.1897 – 24.12.1984)|pwv} und Flirt der bald sechs Jahre {\pb}alt wird – es gibt Hunde die
               achtzehn werden – in einem kleinen Lusthaus das man eigens für uns zurechtgezi{\geminationm}ert hat. Unter uns sehen wir die Strasse, und dann die
               Bahn, und dann die Traun\oindex{Traun@\textbf{Traun}|pw} und drüben wieder die
               Straße.\pend
           \pstart
           Ich scheine recht nervös {\pb}zu sein,
               oder sonst was, so sehr impressioniren mich jetzt gleichgiltige Dinge. Ich glaube
               manchmal daß ganz alte gute Leute, die bald sterben müßen diese leichte Rührung und
               Zärtlichkeit bei todten Dingen – wie Bäumen und {\pb}Straßen, und Flüßen haben; wie ich
               dazu ko{\geminationm}e weiß ich nicht. Oder ist am Ende doch daran
               schuld daß ich weiß, daß jetzt das im Werden ist was uns – oder mich – überleben und
               begraben soll. {\pb}Am Ende fängt mit
               jedem Kinderhaben doch ein unbewußtes Abdanken und Resigniren an; oder spüren wir daß
               wir nun überflüssig sind nachdem etwas von uns in {\pb}Anderem weiter lebt.\pend
           \pstart
           Wann müßen Sie eigentlich wieder nach Wien\oindex{Wien@\textbf{Wien}|pw} zurück?
               Ich muß wol zwischen 15{ }{\kaufmannsund}{ }20 Aug. auf einige Tage nach Wien\oindex{Wien@\textbf{Wien}|pw} »deswegen«. Wo werden Sie um diese Zeit sein? Wann ko{\geminationm}t voraussichtlich Paul\pwindex{Goldmann, Paul 31.01.1865 – 25.09.1935@\textsc{Goldmann, Paul} (31.01.1865 – 25.09.1935), \emph{Schriftsteller, Journalist}|pw} hieher? Grüßen {\pb}Sie Schwarzkopf\pwindex{Schwarzkopf, Gustav 07.11.1853 – 13.11.1939@\textsc{Schwarzkopf, Gustav} (07.11.1853 – 13.11.1939), \emph{Schriftsteller}|pw} und Hugo\pwindex{Hofmannsthal, Hugo von 1874-02-01 – 1929-07-15@\textsc{Hofmannsthal, Hugo von} (1874-02-01 – 1929-07-15), \emph{Schriftsteller}|pw} von mir und
               schreiben Sie mir bald.\pend
           \pstart
           Ihr{\\[\baselineskip]}\spacefill\mbox{Richard}\pend
           \leftskip=0em{}
         
         \endnumbering\mylabel{h}\end{ledgroupsized}  \newcommand{\dateiname}{L00686}\newcommand{\titel}{Richard Beer-Hofmann an Arthur Schnitzler, 13. 6. 1897}\newcommand{\editorInnen}{Martin Anton Müller und Gerd-Hermann Susen}%% latex-leseansicht-abspann.tex
%% Abspann für die Leseansicht.
%% Der Schalter \ifkorrekturansicht ist bereits durch den Vorspann gesetzt.

%% latex-abspann.tex
%% Gemeinsamer Abspann für Korrekturansicht und Leseansicht.
%% Setzt den Schalter \ifkorrekturansicht voraus (gesetzt in den
%% einbindenden Dateien latex-korrekturansicht-abspann.tex bzw.
%% latex-leseansicht-abspann.tex).
%% ---------------------------------------------------------------

\normalsize

% Das esempio-Environment wird nur in der Leseansicht benötigt
\ifkorrekturansicht\else
\newenvironment{esempio}[3]%
{
    \vspace{1.5ex}
    \rlap{\underline{#1}}
    \par
    \setlength{\parindent}{0cm}
    \nopagebreak
    \leftskip=#2cm
    \rightskip=#3cm
}
{
    \par
}
\fi

\doendnotes{C}
\bigskip
\vfill

\clearpage

\footnotesize

\ifkorrekturansicht
  \lohead{\textsc{register}}
\fi

% theindex-Environment neu definieren ohne reledmac
\makeatletter
\renewenvironment{theindex}{%
  \ifkorrekturansicht
    \section*{\indexname}%
  \else
    \subsubsection*{Index der erwähnten Entitäten}%
  \fi
  \setlength{\parindent}{0pt}%
  \setlength{\parskip}{0pt plus 0.3pt}%
  \let\item\@idxitem
}{%
  \ifkorrekturansicht\clearpage\fi
}
\makeatother

\IfFileExists{\jobname-pw.ind}{\input{\jobname-pw.ind}}{}

% Quellenangabe nur in der Leseansicht
\ifkorrekturansicht\else
% Fallback-Definitionen, falls die .tex-Datei \titel etc. nicht gesetzt hat
\providecommand{\titel}{}
\providecommand{\editorInnen}{}
\providecommand{\dateiname}{\jobname}

\vspace{3cm}

\vfill

\footnotesize
\textsc{Quelle}: \titel. Herausgegeben von {\editorInnen}. In: \emph{Arthur Schnitzler: Briefwechsel mit Autorinnen und Autoren}.
 Digitale Edition, https://schnitzler-briefe.acdh.oeaw.ac.at/{\dateiname}.html (Stand \today)
\fi

\end{document}


      