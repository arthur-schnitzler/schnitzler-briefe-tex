%% latex-korrekturansicht-vorspann.tex
%% Vorspann für die Korrekturansicht.
%% Lädt die gemeinsame Datei latex-vorspann.tex mit gesetztem Schalter.

\newif\ifkorrekturansicht
\korrekturansichttrue

\input{../tex-inputs/latex-vorspann}


\section[Richard Beer-Hofmann an Arthur Schnitzler, 13. 6. 1897]{L00686 Richard Beer-Hofmann an Arthur Schnitzler, 13. 6. 1897}
\nopagebreak\mylabel{L00686v}
\rehead{ }\normalsize\beginnumbering\briefempfaengerindex{Schnitzler, Arthur@\textsc{Schnitzler, Arthur}!zzzBeer-Hofmann, Richard@\emph{von Richard Beer-Hofmann}!1897-06-131@{13. 6. 1897}|(be}
\toendnotes[C]{\smallbreak\pagebreak[2]}\Standort{CUL, Schnitzler, B 8.}
\physDesc{Brief, 3 Blätter, 9 Seiten, 1867 Zeichen
\newline{}Handschrift: blauer Buntstift, lateinische Kurrent
\newline{}Ordnung: mit Bleistift von unbekannter Hand nummeriert:
                                    »99« }
\buchAbdrucke{\weitereDrucke{Arthur Schnitzler, Richard Beer-Hofmann: \emph{Briefwechsel 1891–1931}. Wien, Zürich: \emph{Europaverlag} 1992, S. 109–110.} }\toendnotes[C]{\smallbreak}
\pstart
           \centering{}{\pb}Ischl\oindex{Bad Ischl@\textbf{Bad Ischl}, \emph{P.PPL}|pw}\hspace*{1.5em}13/VI 97\pend
           \vspace{0.5em}
\pstart
           Lieber Arthur, ich weiß noch gar nichts wegen Bayreuth\oindex{Bayreuth@\textbf{Bayreuth}, \emph{P.PPLA2}|pw}, und will mich nicht entschließen.\pend
           
\pstart
           Ihr Brief ist wieder so unleserlich! An \uline{was} arbeiten
               Sie? An einem Stück – da Sie von Scenen sprechen aber soll das \strikeout{»}Unleser{\pb}liche »Revolutionsstück\pwindex{gruene Kakadu. Groteske in einem Akt@\emph{Der grüne Kakadu. Groteske in einem Akt}|pwuv}« heißen?\pend
           
\pstart
           Ob mich’s mit »ahnungsvoller
                  Gegenwart ängstigt\pwindex{Faust. Eine Tragoedie@\emph{Faust. Eine Tragödie}|pwv}«? fragen Sie? In mir wird so Vieles jetzt Anders als es
               bis her war daß ich nicht weiß wie viel auf Rech{\pb}nung »\uline{davon}« zu setzen ist. Manchmal hab ich die Empfindung als würde ich im
               Herbst nicht »Vater« sondern »Großvater« wenn ich sehe wie kindisch und jung noch Paula\pwindex{Beer-Hofmann, Paula 25.02.1879 – 30.10.1939@\textsc{Beer-Hofmann, Paula} (25.02.1879 – 30.10.1939)|pw} ist, und dann muß ich wieder {\pb}über mich lachen mit meiner Neigung
               die Dinge zu leicht oder zu schwer zu nehmen. Augenblicklich sitzen wir – das ist Paula\pwindex{Beer-Hofmann, Paula 25.02.1879 – 30.10.1939@\textsc{Beer-Hofmann, Paula} (25.02.1879 – 30.10.1939)|pw}, und ich, und die ko{\geminationm}ende
                  Generation\pwindex{Beer-Hofmann, Mirjam 04.09.1897 – 24.12.1984@\textsc{Beer-Hofmann, Mirjam} (04.09.1897 – 24.12.1984)|pwv} und Flirt der bald sechs Jahre {\pb}alt wird – es gibt Hunde die
               achtzehn werden – in einem kleinen Lusthaus das man eigens für uns zurechtgezi{\geminationm}ert hat. Unter uns sehen wir die Strasse, und dann die
               Bahn, und dann die Traun\oindex{Traun@\textbf{Traun}, \emph{Fluss (N.FLS)}|pw} und drüben wieder die
               Straße.\pend
           
\pstart
           Ich scheine recht nervös {\pb}zu sein,
               oder sonst was, so sehr impressioniren mich jetzt gleichgiltige Dinge. Ich glaube
               manchmal daß ganz alte gute Leute, die bald sterben müßen diese leichte Rührung und
               Zärtlichkeit bei todten Dingen – wie Bäumen und {\pb}Straßen, und Flüßen haben; wie ich
               dazu ko{\geminationm}e weiß ich nicht. Oder ist am Ende doch daran
               schuld daß ich weiß, daß jetzt das im Werden ist was uns – oder mich – überleben und
               begraben soll. {\pb}Am Ende fängt mit
               jedem Kinderhaben doch ein unbewußtes Abdanken und Resigniren an; oder spüren wir daß
               wir nun überflüssig sind nachdem etwas von uns in {\pb}Anderem weiter lebt.\pend
           
\pstart
           Wann müßen Sie eigentlich wieder nach Wien\oindex{Wien@\textbf{Wien}, \emph{A.ADM2}|pw} zurück?
               Ich muß wol zwischen 15{ }{\kaufmannsund}{ }20 Aug. auf einige Tage nach Wien\oindex{Wien@\textbf{Wien}, \emph{A.ADM2}|pw}
               »deswegen«. Wo werden Sie um diese Zeit sein? Wann ko{\geminationm}t
               voraussichtlich Paul\pwindex{Goldmann, Paul 31.01.1865 – 25.09.1935@\textsc{Goldmann, Paul} (31.01.1865 – 25.09.1935), \emph{Schriftsteller/Schriftstellerin, Journalist/Journalistin}|pw} hieher? Grüßen {\pb}Sie Schwarzkopf\pwindex{Schwarzkopf, Gustav 07.11.1853 – 13.11.1939@\textsc{Schwarzkopf, Gustav} (07.11.1853 – 13.11.1939), \emph{Schriftsteller/Schriftstellerin}|pw} und Hugo\pwindex{Hofmannsthal, Hugo von 1874-02-01 – 1929-07-15@\textsc{Hofmannsthal, Hugo von} (1874-02-01 – 1929-07-15), \emph{Schriftsteller/Schriftstellerin}|pw} von mir und schreiben Sie mir bald.\pend
           
\pstart
           Ihr{\\[\baselineskip]}\spacefill\mbox{Richard}\pend
           \leftskip=0em{}\selectlanguage{ngerman}\endnumbering\briefempfaengerindex{Schnitzler, Arthur@\textsc{Schnitzler, Arthur}!zzzBeer-Hofmann, Richard@\emph{von Richard Beer-Hofmann}!1897-06-131@{13. 6. 1897}|)be}\mylabel{L00686h}  \normalsize

\doendnotes{C}
\bigskip
\vfill

\clearpage

\footnotesize

\lohead{\textsc{register}}

% Definiere theindex-Environment komplett neu ohne reledmac
\makeatletter
\renewenvironment{theindex}{%
  \section*{\indexname}%
  \setlength{\parindent}{0pt}%
  \setlength{\parskip}{0pt plus 0.3pt}%
  \let\item\@idxitem
}{%
  \clearpage
}
\makeatother

\IfFileExists{\jobname-pw.ind}{\input{\jobname-pw.ind}}{}

\end{document}

      