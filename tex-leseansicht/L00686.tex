%% latex-leseansicht-vorspann.tex
%% Vorspann für die Leseansicht.
%% Lädt die gemeinsame Datei latex-vorspann.tex mit nicht gesetztem Schalter.

\newif\ifkorrekturansicht
\korrekturansichtfalse

\input{../tex-inputs/latex-vorspann}


\section[Richard Beer-Hofmann an Arthur Schnitzler, 13. 6. 1897]{L00686 Richard Beer-Hofmann an Arthur Schnitzler, 13. 6. 1897}
\nopagebreak\mylabel{L00686v}
\rehead{ }\normalsize\beginnumbering\briefempfaengerindex{Schnitzler, Arthur@\textsc{Schnitzler, Arthur}!zzzBeer-Hofmann, Richard@\emph{von Richard Beer-Hofmann}!1897-06-131@{13. 6. 1897}|(be}
\toendnotes[C]{\smallbreak\pagebreak[2]}
\correspDesc{Versand  durch Richard Beer-Hofmann am 13. 6. 1897 in Bad Ischl
\newline{}Erhalt  durch Arthur Schnitzler im Zeitraum [14. 6. 1897
                  – 18. 6. 1897?] in Wien}\toendnotes[C]{\smallbreak}
\Standort{CUL, Schnitzler, B 8.}
\physDesc{Brief, 3 Blätter, 9 Seiten, 1867 Zeichen
\newline{}Handschrift: blauer Buntstift, lateinische Kurrent
\newline{}Ordnung: mit Bleistift von unbekannter Hand nummeriert:
                                    »99« }
\buchAbdrucke{\weitereDrucke{Arthur Schnitzler, Richard Beer-Hofmann: \emph{Briefwechsel 1891–1931}. Herausgegeben von Konstanze Fliedl. Wien, Zürich: \emph{Europaverlag} 1992, S. 109–110.} }\toendnotes[C]{\smallbreak}
\pstart
           \centering{}{\pb}Ischl\oindex{Bad Ischl@\textbf{Bad Ischl}|pw}\hspace*{1.5em}13/VI 97\pend
           \vspace{0.5em}
\pstart
           Lieber Arthur, ich weiß noch gar nichts wegen Bayreuth\oindex{Bayreuth@\textbf{Bayreuth}, \emph{Hauptstadt}|pw}, und will mich nicht entschließen.\pend
           
\pstart
           Ihr Brief ist wieder so unleserlich! An \uline{was} arbeiten
               Sie? An einem Stück – da Sie von Scenen sprechen aber soll das \strikeout{»}Unleser{\pb}liche »Revolutionsstück\pwindex{Schnitzler, Arthur 15.\,5.\,1862 Wien – 21.\,10.\,1931 ebd.@\textsc{Schnitzler, Arthur} (15.\,5.\,1862 Wien – 21.\,10.\,1931 ebd.), \emph{Schriftsteller, Mediziner}!grüne Kakadu. Groteske in einem Akt@\strich\emph{Der grüne Kakadu. Groteske in einem Akt}|pwuv}« heißen?\pend
           
\pstart
           Ob mich’s mit »ahnungsvoller
                  Gegenwart ängstigt\pwindex{\textcolor{red}{\textsuperscript{XXXX indx1}}!Faust. Eine Tragödie@\strich\emph{Faust. Eine Tragödie}|pwv}«? fragen Sie? In mir wird so Vieles jetzt Anders als es
               bis her war daß ich nicht weiß wie viel auf Rech{\pb}nung »\uline{davon}« zu setzen ist. Manchmal hab ich die Empfindung als würde ich im
               Herbst nicht »Vater« sondern »Großvater« wenn ich sehe wie kindisch und jung noch Paula\pwindex{Beer-Hofmann, Paula 25.\,2.\,1879 Wien – 30.\,10.\,1939 Zürich@\textsc{Beer-Hofmann, Paula} (25.\,2.\,1879 Wien – 30.\,10.\,1939 Zürich)|pw} ist, und dann muß ich wieder {\pb}über mich lachen mit meiner Neigung
               die Dinge zu leicht oder zu schwer zu nehmen. Augenblicklich sitzen wir – das ist Paula\pwindex{Beer-Hofmann, Paula 25.\,2.\,1879 Wien – 30.\,10.\,1939 Zürich@\textsc{Beer-Hofmann, Paula} (25.\,2.\,1879 Wien – 30.\,10.\,1939 Zürich)|pw}, und ich, und die ko{\geminationm}ende
                  Generation\pwindex{Beer-Hofmann, Mirjam 4.\,9.\,1897 Wien – 24.\,12.\,1984 New York City@\textsc{Beer-Hofmann, Mirjam} (4.\,9.\,1897 Wien – 24.\,12.\,1984 New York City)|pwv} und Flirt der bald sechs Jahre {\pb}alt wird – es gibt Hunde die
               achtzehn werden – in einem kleinen Lusthaus das man eigens für uns zurechtgezi{\geminationm}ert hat. Unter uns sehen wir die Strasse, und dann die
               Bahn, und dann die Traun\oindex{Traun@\textbf{Traun}, \emph{Fluss}|pw} und drüben wieder die
               Straße.\pend
           
\pstart
           Ich scheine recht nervös {\pb}zu sein,
               oder sonst was, so sehr impressioniren mich jetzt gleichgiltige Dinge. Ich glaube
               manchmal daß ganz alte gute Leute, die bald sterben müßen diese leichte Rührung und
               Zärtlichkeit bei todten Dingen – wie Bäumen und {\pb}Straßen, und Flüßen haben; wie ich
               dazu ko{\geminationm}e weiß ich nicht. Oder ist am Ende doch daran
               schuld daß ich weiß, daß jetzt das im Werden ist was uns – oder mich – überleben und
               begraben soll. {\pb}Am Ende fängt mit
               jedem Kinderhaben doch ein unbewußtes Abdanken und Resigniren an; oder spüren wir daß
               wir nun überflüssig sind nachdem etwas von uns in {\pb}Anderem weiter lebt.\pend
           
\pstart
           Wann müßen Sie eigentlich wieder nach Wien\oindex{Wien@\textbf{Wien}, \emph{Verwaltungsgebiet}|pw} zurück?
               Ich muß wol zwischen 15{ }{\kaufmannsund}{ }20 Aug. auf einige Tage nach Wien\oindex{Wien@\textbf{Wien}, \emph{Verwaltungsgebiet}|pw}
               »deswegen«. Wo werden Sie um diese Zeit sein? Wann ko{\geminationm}t
               voraussichtlich Paul\pwindex{Goldmann, Paul 31.\,1.\,1865 Breslau – 25.\,9.\,1935 Wien@\textsc{Goldmann, Paul} (31.\,1.\,1865 Breslau – 25.\,9.\,1935 Wien), \emph{Schriftsteller, Journalist}|pw} hieher? Grüßen {\pb}Sie Schwarzkopf\pwindex{Schwarzkopf, Gustav 7.\,11.\,1853 Wien – 13.\,11.\,1939 ebd.@\textsc{Schwarzkopf, Gustav} (7.\,11.\,1853 Wien – 13.\,11.\,1939 ebd.), \emph{Schriftsteller}|pw} und Hugo\pwindex{Hofmannsthal, Hugo von 1.\,2.\,1874 Wien – 15.\,7.\,1929 Rodaun@\textsc{Hofmannsthal, Hugo von} (1.\,2.\,1874 Wien – 15.\,7.\,1929 Rodaun), \emph{Schriftsteller}|pw} von mir und schreiben Sie mir bald.\pend
           
\pstart
           Ihr{\\[\baselineskip]}\spacefill\mbox{Richard}\pend
           \leftskip=0em{}\selectlanguage{ngerman}\endnumbering\briefempfaengerindex{Schnitzler, Arthur@\textsc{Schnitzler, Arthur}!zzzBeer-Hofmann, Richard@\emph{von Richard Beer-Hofmann}!1897-06-131@{13. 6. 1897}|)be}\mylabel{L00686h}  \newcommand{\dateiname}{L00686}\newcommand{\titel}{Richard Beer-Hofmann an Arthur Schnitzler, 13. 6. 1897}\newcommand{\editorInnen}{Martin Anton Müller und Gerd-Hermann Susen}%% latex-leseansicht-abspann.tex
%% Abspann für die Leseansicht.
%% Der Schalter \ifkorrekturansicht ist bereits durch den Vorspann gesetzt.

%% latex-abspann.tex
%% Gemeinsamer Abspann für Korrekturansicht und Leseansicht.
%% Setzt den Schalter \ifkorrekturansicht voraus (gesetzt in den
%% einbindenden Dateien latex-korrekturansicht-abspann.tex bzw.
%% latex-leseansicht-abspann.tex).
%% ---------------------------------------------------------------

\normalsize

% Das esempio-Environment wird nur in der Leseansicht benötigt
\ifkorrekturansicht\else
\newenvironment{esempio}[3]%
{
    \vspace{1.5ex}
    \rlap{\underline{#1}}
    \par
    \setlength{\parindent}{0cm}
    \nopagebreak
    \leftskip=#2cm
    \rightskip=#3cm
}
{
    \par
}
\fi

\doendnotes{C}
\bigskip
\vfill

\clearpage

\footnotesize

\ifkorrekturansicht
  \lohead{\textsc{register}}
\fi

% theindex-Environment neu definieren ohne reledmac
\makeatletter
\renewenvironment{theindex}{%
  \ifkorrekturansicht
    \section*{\indexname}%
  \else
    \subsubsection*{Index der erwähnten Entitäten}%
  \fi
  \setlength{\parindent}{0pt}%
  \setlength{\parskip}{0pt plus 0.3pt}%
  \let\item\@idxitem
}{%
  \ifkorrekturansicht\clearpage\fi
}
\makeatother

\IfFileExists{\jobname-pw.ind}{\input{\jobname-pw.ind}}{}

% Quellenangabe nur in der Leseansicht
\ifkorrekturansicht\else
% Fallback-Definitionen, falls die .tex-Datei \titel etc. nicht gesetzt hat
\providecommand{\titel}{}
\providecommand{\editorInnen}{}
\providecommand{\dateiname}{\jobname}

\vspace{3cm}

\vfill

\footnotesize
\textsc{Quelle}: \titel. Herausgegeben von {\editorInnen}. In: \emph{Arthur Schnitzler: Briefwechsel mit Autorinnen und Autoren}.
 Digitale Edition, https://schnitzler-briefe.acdh.oeaw.ac.at/{\dateiname}.html (Stand \today)
\fi

\end{document}


