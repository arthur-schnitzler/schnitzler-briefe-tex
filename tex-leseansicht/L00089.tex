%% latex-leseansicht-vorspann.tex
%% Vorspann für die Leseansicht.
%% Lädt die gemeinsame Datei latex-vorspann.tex mit nicht gesetztem Schalter.

\newif\ifkorrekturansicht
\korrekturansichtfalse

\input{../tex-inputs/latex-vorspann}


         
         \renewcommand{\erwaehntePersonen}{Personen: Wilhelm Bölsche}
         \renewcommand{\erwaehnteOrte}{Orte: Berlin, Ordination Dr. Arthur Schnitzler Giselastraße 11, Wien}
         \renewcommand{\erwaehnteWerke}{Werke: Das Himmelbett, Die drei Elixire, Freie Bühne für den Entwickelungskampf der Zeit, Freie Bühne für modernes Leben}
               \section[Arthur Schnitzler an Wilhelm Bölsche, 27. 3. 1892]{ Arthur Schnitzler an Wilhelm Bölsche, 27. 3. 1892}\nopagebreak\mylabel{v}\rehead{ }\begin{ledgroupsized}[t]{13cm}\normalsize\beginnumbering\briefempfaengerindex{Boelsche, Wilhelm@\textsc{Bölsche, Wilhelm}!zzzSchnitzler, Arthur@\emph{von Arthur Schnitzler}!1892-03-271@{27. 3. 1892}|(be} \toendnotes[C]{\smallbreak\pagebreak[2]} \Standort{Wrocław, Biblioteka Uniwersytecka, Böl.Pis 1763.}
\physDesc{Brief, 1 Blatt, 2 Seiten, 861 Zeichen
\newline{}Handschrift: schwarze Tinte, deutsche Kurrent
\newline{}Bölsche: mit schwarzer Tinte als »Erl{[}edigt{]}« gezeichnet }\buchAbdrucke{\weitereDrucke{1) Alois Woldan: \emph{Arthur Schnitzler – Briefe an Wilhelm Bölsche.} In: \emph{Germanica Wratislaviensia} (1987) Nr. 77, S. 460.} \weitereDrucke{2) Wilhelm Bölsche: \emph{Briefwechsel. Mit Autoren der Freien Bühne}. Hg. Gerd-Hermann Susen. Berlin: \emph{Weidler} 2010, S. 678–679 (Werke und Briefe. Wissenschaftliche Ausgabe, Briefe I).} }\toendnotes[C]{\smallbreak}\pstart
           {\pb}\textsc{Wien I Giselastraße 11}\oindex{Ordination Dr. Arthur Schnitzler Giselastrasse 11@\textbf{Ordination Dr. Arthur Schnitzler Giselastraße 11}|pw}.\hfill 27. 3. 92.\pend
           \pstart{}Sehr geehrter Herr,\pend\pstart
           beſten Dank für Ihre freundliche Antwort. Und nun wieder eine Frage, die aber ohne
               jede Mühe in Kürze mit einem Ja oder Nein zu beantworten iſt. Ich möchte Ihnen gerne
               eine kleine Geſchichte\pwindex{Schnitzler, Arthur 15.05.1862 – 21.10.1931@\textsc{Schnitzler, Arthur} (15.05.1862 – 21.10.1931), \emph{Schriftsteller, Mediziner}!Himmelbett1977@\strich\emph{Das Himmelbett} {[}1977{]}|pwv}{ }ſtatt der Elixire\pwindex{Schnitzler, Arthur 15.05.1862 – 21.10.1931@\textsc{Schnitzler, Arthur} (15.05.1862 – 21.10.1931), \emph{Schriftsteller, Mediziner}!drei Elixire1893@\strich\emph{Die drei Elixire} {[}1893{]}|pw}{ }ſchicken, die Ihnen nicht zu gefallen ſcheinen,
                  \introOben{}eine Geſchichte\pwindex{Schnitzler, Arthur 15.05.1862 – 21.10.1931@\textsc{Schnitzler, Arthur} (15.05.1862 – 21.10.1931), \emph{Schriftsteller, Mediziner}!Himmelbett1977@\strich\emph{Das Himmelbett} {[}1977{]}|pwv}\introOben{}, die wohl auch beſſer in den Rahmen Ihres Blattes\pwindex{Freie Buehne fuer den Entwickelungskampf der Zeit1892-01-01 – 1893-12-31@\emph{Freie Bühne für den Entwickelungskampf der Zeit} {[}1892-01-01 – 1893-12-31{]}|pwv} paſſen dürfte. Nur läge mir aber ſehr viel daran,
               daß ſie ſchon im \uline{Maiheft} der Freien Bühne\pwindex{Freie Buehne fuer modernes Leben1890 – 1891@\emph{Freie Bühne für modernes Leben} {[}1890 – 1891{]}|pw}{ }{\pb}erſchiene. (Sie faſſt im ganzen 3–4 Seiten.) Wäre dies – im
               Fall natürlich, daß Ihnen die kleine Arbeit\pwindex{Schnitzler, Arthur 15.05.1862 – 21.10.1931@\textsc{Schnitzler, Arthur} (15.05.1862 – 21.10.1931), \emph{Schriftsteller, Mediziner}!Himmelbett1977@\strich\emph{Das Himmelbett} {[}1977{]}|pwv}{ }ſonſt convenirt – möglich, ſo theilen Sie mir das
               freundlichſt durch ein \uline{Ja} mit. 2 Tage drauf ſind Sie
               im Beſitz des Manuscriptes\pwindex{Schnitzler, Arthur 15.05.1862 – 21.10.1931@\textsc{Schnitzler, Arthur} (15.05.1862 – 21.10.1931), \emph{Schriftsteller, Mediziner}!Himmelbett1977@\strich\emph{Das Himmelbett} {[}1977{]}|pwv},
               das ja in einer viertel Stunde geleſen iſt.\pend
           \pstart
           Für die Erfüllung meines Erſuchens wäre ich Ihnen herzlichſt verbunden.\pend
           \pstart
           Mit aufrichtiger Hochachtung{\\[\baselineskip]}Ihr ergebner\spacefill\mbox{DrArthurSchnitzler}\pend
           \leftskip=0em{}
         
         \endnumbering\mylabel{h}\end{ledgroupsized}  \newcommand{\dateiname}{L00089}\newcommand{\titel}{Arthur Schnitzler an Wilhelm Bölsche, 27. 3. 1892}\newcommand{\editorInnen}{Martin Anton Müller und Gerd-Hermann Susen}%% latex-leseansicht-abspann.tex
%% Abspann für die Leseansicht.
%% Der Schalter \ifkorrekturansicht ist bereits durch den Vorspann gesetzt.

%% latex-abspann.tex
%% Gemeinsamer Abspann für Korrekturansicht und Leseansicht.
%% Setzt den Schalter \ifkorrekturansicht voraus (gesetzt in den
%% einbindenden Dateien latex-korrekturansicht-abspann.tex bzw.
%% latex-leseansicht-abspann.tex).
%% ---------------------------------------------------------------

\normalsize

% Das esempio-Environment wird nur in der Leseansicht benötigt
\ifkorrekturansicht\else
\newenvironment{esempio}[3]%
{
    \vspace{1.5ex}
    \rlap{\underline{#1}}
    \par
    \setlength{\parindent}{0cm}
    \nopagebreak
    \leftskip=#2cm
    \rightskip=#3cm
}
{
    \par
}
\fi

\doendnotes{C}
\bigskip
\vfill

\clearpage

\footnotesize

\ifkorrekturansicht
  \lohead{\textsc{register}}
\fi

% theindex-Environment neu definieren ohne reledmac
\makeatletter
\renewenvironment{theindex}{%
  \ifkorrekturansicht
    \section*{\indexname}%
  \else
    \subsubsection*{Index der erwähnten Entitäten}%
  \fi
  \setlength{\parindent}{0pt}%
  \setlength{\parskip}{0pt plus 0.3pt}%
  \let\item\@idxitem
}{%
  \ifkorrekturansicht\clearpage\fi
}
\makeatother

\IfFileExists{\jobname-pw.ind}{\input{\jobname-pw.ind}}{}

% Quellenangabe nur in der Leseansicht
\ifkorrekturansicht\else
% Fallback-Definitionen, falls die .tex-Datei \titel etc. nicht gesetzt hat
\providecommand{\titel}{}
\providecommand{\editorInnen}{}
\providecommand{\dateiname}{\jobname}

\vspace{3cm}

\vfill

\footnotesize
\textsc{Quelle}: \titel. Herausgegeben von {\editorInnen}. In: \emph{Arthur Schnitzler: Briefwechsel mit Autorinnen und Autoren}.
 Digitale Edition, https://schnitzler-briefe.acdh.oeaw.ac.at/{\dateiname}.html (Stand \today)
\fi

\end{document}


      