%% latex-korrekturansicht-vorspann.tex
%% Vorspann für die Korrekturansicht.
%% Lädt die gemeinsame Datei latex-vorspann.tex mit gesetztem Schalter.

\newif\ifkorrekturansicht
\korrekturansichttrue

\input{../tex-inputs/latex-vorspann}


\section[Arthur Schnitzler an Wilhelm Bölsche, 27. 3. 1892]{L00089 Arthur Schnitzler an Wilhelm Bölsche, 27. 3. 1892}
\nopagebreak\mylabel{L00089v}
\rehead{ }\normalsize\beginnumbering\briefempfaengerindex{Boelsche, Wilhelm@\textsc{Bölsche, Wilhelm}!zzzSchnitzler, Arthur@\emph{von Arthur Schnitzler}!1892-03-271@{27. 3. 1892}|(be}
\toendnotes[C]{\smallbreak\pagebreak[2]}\Standort{Wrocław, Biblioteka Uniwersytecka, Böl.Pis 1763.}
\physDesc{Brief, 1 Blatt, 2 Seiten, 861 Zeichen
\newline{}Handschrift: schwarze Tinte, deutsche Kurrent
\newline{}Bölsche: mit schwarzer Tinte als »Erl{[}edigt{]}« gezeichnet }
\buchAbdrucke{\weitereDrucke{1) \emph{Germanica Wratislaviensia} (1987) Nr. 77, S. 460.} \weitereDrucke{2) Wilhelm Bölsche: \emph{Briefwechsel. Mit Autoren der Freien Bühne}. Berlin: \emph{Weidler} 2010, S. 678–679.} }\toendnotes[C]{\smallbreak}
\pstart
           
\pstart
           {\pb}\textsc{Wien I Giselastraße 11}\oindex{Ordination Arthur Schnitzler [Boesendorferstrasse 11]@\textbf{Ordination Arthur Schnitzler [Bösendorferstraße 11]}, \emph{Ordination}|pw}.\pend
           
\pstart
           \raggedleft{}27. 3. 92.\pend
           \pend
           
\pstart{}Sehr geehrter Herr,\pend\vspace{0.5em}
\pstart
           beſten Dank für Ihre freundliche Antwort. Und nun wieder eine Frage, die aber ohne
               jede Mühe in Kürze mit einem Ja oder Nein zu beantworten iſt. Ich möchte Ihnen gerne
               eine kleine Geſchichte\pwindex{Himmelbett@\emph{Das Himmelbett}|pwv}{ }ſtatt der Elixire\pwindex{drei Elixire@\emph{Die drei Elixire}|pw}{ }ſchicken, die Ihnen nicht zu gefallen ſcheinen,
                  \introOben{}eine Geſchichte\pwindex{Himmelbett@\emph{Das Himmelbett}|pwv}\introOben{}, die wohl auch beſſer in den Rahmen Ihres Blattes\pwindex{Freie Buehne fuer den Entwickelungskampf der Zeit@\emph{Freie Bühne für den Entwickelungskampf der Zeit}|pwv} paſſen dürfte. Nur läge mir aber ſehr viel daran,
               daß ſie ſchon im \uline{Maiheft} der Freien Bühne\pwindex{Freie Buehne fuer modernes Leben@\emph{Freie Bühne für modernes Leben}|pw}{ }{\pb}erſchiene. (Sie faſſt im ganzen 3–4 Seiten.) Wäre dies – im
               Fall natürlich, daß Ihnen die kleine Arbeit\pwindex{Himmelbett@\emph{Das Himmelbett}|pwv}{ }ſonſt convenirt – möglich, ſo theilen Sie mir das
               freundlichſt durch ein \uline{Ja} mit. 2 Tage drauf ſind Sie
               im Beſitz des Manuscriptes\pwindex{Himmelbett@\emph{Das Himmelbett}|pwv},
               das ja in einer viertel Stunde geleſen iſt.\pend
           
\pstart
           Für die Erfüllung meines Erſuchens wäre ich Ihnen herzlichſt verbunden.\pend
           
\pstart
           Mit aufrichtiger Hochachtung{\\[\baselineskip]}Ihr ergebner\spacefill\mbox{DrArthurSchnitzler}\pend
           \leftskip=0em{}\selectlanguage{ngerman}\endnumbering\briefempfaengerindex{Boelsche, Wilhelm@\textsc{Bölsche, Wilhelm}!zzzSchnitzler, Arthur@\emph{von Arthur Schnitzler}!1892-03-271@{27. 3. 1892}|)be}\mylabel{L00089h}  \normalsize

\doendnotes{C}
\bigskip
\vfill

\clearpage

\footnotesize

\lohead{\textsc{register}}

% Definiere theindex-Environment komplett neu ohne reledmac
\makeatletter
\renewenvironment{theindex}{%
  \section*{\indexname}%
  \setlength{\parindent}{0pt}%
  \setlength{\parskip}{0pt plus 0.3pt}%
  \let\item\@idxitem
}{%
  \clearpage
}
\makeatother

\IfFileExists{\jobname-pw.ind}{\input{\jobname-pw.ind}}{}

\end{document}

      