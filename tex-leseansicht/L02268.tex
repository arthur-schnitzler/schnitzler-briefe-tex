%% latex-leseansicht-vorspann.tex
%% Vorspann für die Leseansicht.
%% Lädt die gemeinsame Datei latex-vorspann.tex mit nicht gesetztem Schalter.

\newif\ifkorrekturansicht
\korrekturansichtfalse

\input{../tex-inputs/latex-vorspann}


         
         \renewcommand{\erwaehntePersonen}{Personen: Robert Adam}
         \renewcommand{\erwaehnteOrte}{Orte: Wien}
         \renewcommand{\erwaehnteWerke}{Werke: Das Ende des Judas, Doktor Gräsler, Badearzt, Don Quijote}
               \section[Robert Adam an Arthur Schnitzler, 23. 8. 1917]{ Robert Adam an Arthur Schnitzler, 23. 8. 1917}\nopagebreak\mylabel{v}\rehead{ }\begin{ledgroupsized}[t]{13cm}\normalsize\beginnumbering\briefempfaengerindex{Schnitzler, Arthur@\textsc{Schnitzler, Arthur}!zzzAdam, Robert@\emph{von Robert Adam}!1917-08-231@{23. 8. 1917}|(be} \toendnotes[C]{\smallbreak\pagebreak[2]} \Standort{DLA, A:Schnitzler, HS.NZ85.1.4230,20.}
\physDesc{Brief, 1 Blatt, 4 Seiten, 2889 Zeichen
\newline{}Handschrift: schwarze Tinte, deutsche Kurrent
\newline{}Schnitzler: 1) mit Bleistift beschriftet: »\textsc{Adam}«  2) mit rotem Buntstift mehrere Unterstreichungen}\Standort{Wien, Österreichische Nationalbibliothek, Cod.ser. 52.263, 200.}
\physDesc{Brief, maschinenschriftliche Abschrift, 1 Blatt, 1 Seite, 2889 Zeichen
\newline{}Schreibmaschine}\toendnotes[C]{\smallbreak}\pstart
           \raggedleft{}{\pb}Wien\oindex{Wien@\textbf{Wien}|pw}, am 23. August 1917\pend
           \pstart{}Hochverehrter Herr Doktor!\pend\pstart
           Ich habe heute früh zu meiner freudigen Überraſchung Ihren \textsc{D\textsuperscript{r}{ }Gräsler}\pwindex{Schnitzler, Arthur 15.05.1862 – 21.10.1931@\textsc{Schnitzler, Arthur} (15.05.1862 – 21.10.1931), \emph{Schriftsteller, Mediziner}!Doktor Graesler, Badearzt1917-02-10 – 1917-03-18@\strich\emph{Doktor Gräsler, Badearzt} {[}1917-02-10 – 1917-03-18{]}|pw} zugeſtellt erhalten und beeile mich, Ihnen, obwohl ich nur erſt wenige Seiten
               leſen konnte, herzlichſt für Ihre liebenswürdige Sendung und Widmung zu danken.\pend
           \pstart
           Ich wollte in den nächſten Tagen bei Ihnen anfragen, ob Ihnen ein Beſuch angelegen
               käme (die Anfrage verſchob ich aus einem einigermaßen kindiſchen Grunde: vorerſt
               ſollte nämlich eine lange Komödie\pwindex{Adam, Robert 20.04.1877 – 16.10.1961@\textsc{Adam, Robert} (20.04.1877 – 16.10.1961), \emph{Schriftsteller, Richter}!Ende des Judas@\strich\emph{Das Ende des Judas}|pwv} – wenn man’s ſo \strikeout{d} nennen darf – im
               erſten Entwurf fertiggeſtellt ſein, aber die letzte Szene, die allerdings ein
               ſchwieriges Unge{\pb}heuer iſt, dehnt ſich und ſtreckt
               ſich und will nicht zum Schluß kommen). Nun aber frage ich doch an, ob ich wieder
               einmal bei Ihnen erſcheinen darf? Bevor ich auf Urlaub ging, ſprach ich einmal bei
               Ihnen vor, traf Sie aber leider nicht an.\pend
           \pstart
           Über meine jetzige amtliche Tätigkeit läßt der \damage{Gerich}tsſaalberichterſtatter manchmal etwas \damage{ve}rlauten: ich kämpfe tagaus tagein mit der Preistreiberei, von Arbeit
               überhäuft, mit gutem Willen, aber in dem vollkommenen Gefühle, ich mag nicht ſagen,
               der Don Quixoterie\pwindex{\textcolor{red}{\textsuperscript{XXXX1 indx}}!Don Quijote1605@\strich\emph{Don Quijote} {[}1605{]}|pwv}, \strikeout{aber} (denn es handelt ſich weder um Windmühlen noch um
               harmloſe Barbiere) aber doch lächerlicher Ohnmacht. An Bildern, die Art dieſes
               Kampfes darzuſtellen, kann’s ja nicht fehlen: Peitſchen des Meeres, Salzbeſtreuen des
               Schwanzes, Hüten von Ameiſen. Die Preiſe ſteigen mit unheimlicher Konſequenz und
               unſereins wandelt ihnen mißbilligend nach {\pb}und
               verſichert ihnen immer wieder, ſie hätten nicht gut daran getan zu ſteigen und ſie
               ſollten es wenigſtens jetzt unterlaſſen. Man ſpielt die lahme Gouvernante wilder
               Kinder, die den Trieben der Natur folgen. Wenn es nur wenigſtens irgend einen Weiſen
               gäbe, der Herr des großen Geheimniſſes wäre: was denn eigentlich Preistreiberei ſei?
               an welchem ſicheren Kainszeichen man die »offenbar übermäßigen« Preiſe erkennen und
               von den unſchuldigen nicht übermäßigen, ſondern bloß exorbitanten Preiſen
               unterſcheiden können? Aber: »Gefühl iſt alles« –\pend
           \pstart
           Dauert dieſer Kriegszuſtand der Jurisprudenz noch lange an, ſo könnte neben dem
               Lächeln der Auguren jenes andere verzweiflungsvolle Lächeln berühmt werden, mit dem
               während einer Preistreibereiverhandlung der Angeklagte den Verteidiger, der
               Verteidiger den Staatsanwalt, dieſer den Richter und der Richter den Angeklagten
               anſieht: »Vielleicht biſt du {\pb}klüger als ich – oder am
               Ende auch nicht?« Man möchte vermuten, daß wenigſtens die Preistreiber \introOben{}\strikeout{ſi}\introOben{}{ }ſelbst \introOben{}ſich\introOben{} darüber klar
               ſein müßten, ob ſie Preistreiber ſeien: aber auch dieſe Vermutung iſt \strikeout{nicht} kaum zutreffend. –\pend
           \pstart
           Verzeihen Sie, daß ich Sie mit Berufsklagen langweile; aber in dieſer Zeit, da ich
               von allen Seiten nur Lebensmittelklagen höre, ſcheinen mir jene noch die
               erfreulichſte Art zu ſein. Und über’s Jammern kommt man jetzt ja doch nicht
               hinaus. –\pend
           \pstart
           Nochmals, hochverehrter Herr Doktor, meinen herzlichſten Dank und die ergebenſten
               Grüße!\pend
           \pstart
           Ihr{\\[\baselineskip]}\spacefill\mbox{Robert Adam}\pend
           \leftskip=0em{}
         
         \endnumbering\mylabel{h}\end{ledgroupsized}  \newcommand{\dateiname}{L02268}\newcommand{\titel}{Robert Adam an Arthur Schnitzler, 23. 8. 1917}\newcommand{\editorInnen}{Martin Anton Müller und Gerd-Hermann Susen}%% latex-leseansicht-abspann.tex
%% Abspann für die Leseansicht.
%% Der Schalter \ifkorrekturansicht ist bereits durch den Vorspann gesetzt.

%% latex-abspann.tex
%% Gemeinsamer Abspann für Korrekturansicht und Leseansicht.
%% Setzt den Schalter \ifkorrekturansicht voraus (gesetzt in den
%% einbindenden Dateien latex-korrekturansicht-abspann.tex bzw.
%% latex-leseansicht-abspann.tex).
%% ---------------------------------------------------------------

\normalsize

% Das esempio-Environment wird nur in der Leseansicht benötigt
\ifkorrekturansicht\else
\newenvironment{esempio}[3]%
{
    \vspace{1.5ex}
    \rlap{\underline{#1}}
    \par
    \setlength{\parindent}{0cm}
    \nopagebreak
    \leftskip=#2cm
    \rightskip=#3cm
}
{
    \par
}
\fi

\doendnotes{C}
\bigskip
\vfill

\clearpage

\footnotesize

\ifkorrekturansicht
  \lohead{\textsc{register}}
\fi

% theindex-Environment neu definieren ohne reledmac
\makeatletter
\renewenvironment{theindex}{%
  \ifkorrekturansicht
    \section*{\indexname}%
  \else
    \subsubsection*{Index der erwähnten Entitäten}%
  \fi
  \setlength{\parindent}{0pt}%
  \setlength{\parskip}{0pt plus 0.3pt}%
  \let\item\@idxitem
}{%
  \ifkorrekturansicht\clearpage\fi
}
\makeatother

\IfFileExists{\jobname-pw.ind}{\input{\jobname-pw.ind}}{}

% Quellenangabe nur in der Leseansicht
\ifkorrekturansicht\else
% Fallback-Definitionen, falls die .tex-Datei \titel etc. nicht gesetzt hat
\providecommand{\titel}{}
\providecommand{\editorInnen}{}
\providecommand{\dateiname}{\jobname}

\vspace{3cm}

\vfill

\footnotesize
\textsc{Quelle}: \titel. Herausgegeben von {\editorInnen}. In: \emph{Arthur Schnitzler: Briefwechsel mit Autorinnen und Autoren}.
 Digitale Edition, https://schnitzler-briefe.acdh.oeaw.ac.at/{\dateiname}.html (Stand \today)
\fi

\end{document}


      