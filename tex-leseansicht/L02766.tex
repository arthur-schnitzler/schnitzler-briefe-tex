%% latex-korrekturansicht-vorspann.tex
%% Vorspann für die Korrekturansicht.
%% Lädt die gemeinsame Datei latex-vorspann.tex mit gesetztem Schalter.

\newif\ifkorrekturansicht
\korrekturansichttrue

\input{../tex-inputs/latex-vorspann}


\section[ Paul Goldmann an Arthur Schnitzler, 1. 2. {[}1896{]}]{L02766 Paul Goldmann an Arthur Schnitzler, 1. 2. {[}1896{]}}
\nopagebreak\mylabel{L02766v}
\rehead{ }\normalsize\beginnumbering\briefempfaengerindex{Schnitzler, Arthur@\textsc{Schnitzler, Arthur}!zzzGoldmann, Paul@\emph{von Paul Goldmann}!1896-02-011@{1. 2. {[}1896{]}}|(be}
\toendnotes[C]{\smallbreak\pagebreak[2]}\Standort{DLA, A:Schnitzler, HS.NZ85.1.3166.}
\physDesc{Brief, 4 Blätter, 14 Seiten, 4746 Zeichen
\newline{}Handschrift: blaue Tinte, deutsche Kurrent
\newline{}Beilage: handschriftlicher Brief: 1 stark beschnittener Ausschnitt aus
                                 einem Brief von Wally
                                    Rosengart\pwindex{Rosengart, Vally 1866-12-29 – nach 1926@\textsc{Rosengart, Vally} (1866-12-29 – nach 1926)|pw} an Goldmann\pwindex{Goldmann, Paul 31.01.1865 – 25.09.1935@\textsc{Goldmann, Paul} (31.01.1865 – 25.09.1935), \emph{Schriftsteller/Schriftstellerin, Journalist/Journalistin}|pw}, blaue Tinte, deutsche Kurrentschrift. Auf der
                                 Rückseite des Schnipsels steht: »{\pb}Mein lieber Paul\pwindex{Goldmann, Paul 31.01.1865 – 25.09.1935@\textsc{Goldmann, Paul} (31.01.1865 – 25.09.1935), \emph{Schriftsteller/Schriftstellerin, Journalist/Journalistin}|pw} – es fehlt \damage{uns} leider alles, um d\textcolor{gray}{en}« 
\newline{}Schnitzler: 1) mit Bleistift das Jahr »96« vermerkt  2) mit rotem Buntstift zwei Unterstreichungen}\toendnotes[C]{\smallbreak}
\pstart
           {\pb}\textcolor{gray}{\textbf{\textbf{Frankfurter Zeitung\orgindex{Frankfurter Zeitung@Frankfurter Zeitung|pw}}}}\pend
           
\pstart
           \textcolor{gray}{\textbf{(\begin{otherlanguage}{french}Gazette de Francfort\end{otherlanguage}\orgindex{Frankfurter Zeitung@Frankfurter Zeitung|pw}).}}\pend
           
\pstart
           \textcolor{gray}{\textbf{\textbf{\begin{otherlanguage}{french}Fondateur M.\end{otherlanguage}{ }L. Sonnemann\pwindex{Sonnemann, Leopold 1831-10-29 – 1909-10-30@\textsc{Sonnemann, Leopold} (1831-10-29 – 1909-10-30), \emph{Journalist/Journalistin, Herausgeber/Herausgeberin}|pw}.}}}\pend
           
\pstart
           \begin{otherlanguage}{french}\textcolor{gray}{\textbf{Journal\pwindex{Frankfurter Zeitung@\emph{Frankfurter Zeitung}|pwv} politique,
                        financier,}}\end{otherlanguage}\pend
           
\pstart
           \begin{otherlanguage}{french}\textcolor{gray}{\textbf{commercial et littéraire.}}\end{otherlanguage}\pend
           
\pstart
           \begin{otherlanguage}{french}\textcolor{gray}{\textbf{\textbf{Paraissant trois fois par jour.}}}\end{otherlanguage}\hfill \textsc{Paris\oindex{Paris@\textbf{Paris}, \emph{P.PPLC}|pw}}, 1. Februar.\pend
           
\pstart
           \begin{otherlanguage}{french}\textcolor{gray}{\textbf{\textbf{Bureau à Paris\oindex{Paris@\textbf{Paris}, \emph{P.PPLC}|pw}:}}}\end{otherlanguage}\pend
           
\pstart
           \begin{otherlanguage}{french}\textcolor{gray}{\textbf{\textbf{24. Rue Feydeau\oindex{rue Feydeau@\textbf{rue Feydeau}, \emph{Straße (K.STR)}|pw}.}}}\end{otherlanguage}\pend
           
\pstart\center{}Mein lieber Freund,\pend\vspace{0.5em}
\pstart
           Herzlich willkommen in \label{K_L02766-1v}\edtext{Berlin\oindex{Berlin@\textbf{Berlin}, \emph{P.PPLC}|pw}}{\lemma{\textnormal{\emph{Berlin}}}\Cendnote{\textnormal{Für die Premiere von \emph{Liebelei}\pwindex{Liebelei. Schauspiel in drei Akten@\emph{Liebelei. Schauspiel in drei Akten}|pwk} am \emph{Deutschen
                     Theater}\orgindex{Deutsches Theater Berlin@Deutsches Theater Berlin|pwk} (4. 2. 1896) war Schnitzler
                  zwischen 30. 1. 1896
                  und 10. 2. 1896 in
                     Berlin\oindex{Berlin@\textbf{Berlin}, \emph{P.PPLC}|pwk}.}}}\label{K_L02766-1}! Möge Dir neues Gute dort
               beſchieden ſein!\pend
           
\pstart
           Ich hörte dieſer Tage, »Sterben\pwindex{Mourir. Roman@\emph{Mourir. Roman}|pwv}\pwindex{Sterben. Novelle@\emph{Sterben. Novelle}|pw}« werde demnächſt hier bei \textsc{Perrin\orgindex{Editions Perrin@Éditions Perrin|pw}} erſcheinen u. \textsc{Ed. Rod\pwindex{Rod, Edouard 1857-03-31 – 1910@\textsc{Rod, Édouard} (1857-03-31 – 1910), \emph{Schriftsteller/Schriftstellerin}|pw}} intereſſire ſich ganz beſonders dafür. Das wird Dir hoffentlich einen großen
                  \label{K_L02766-2v}\edtext{Artikel in den »\textsc{Débats\pwindex{Journal des debats. Politiques et litteraires@\emph{Journal des débats. Politiques et littéraires}|pw}}}{\lemma{\textnormal{\emph{Artikel in den »Débats}}}\Cendnote{\textnormal{Dazu kam es nicht.}}}\label{K_L02766-2}« eintragen, zu deſſen Literatur-Referenten \textsc{Rod\pwindex{Rod, Edouard 1857-03-31 – 1910@\textsc{Rod, Édouard} (1857-03-31 – 1910), \emph{Schriftsteller/Schriftstellerin}|pw}} gehört.\pend
           
\pstart
           Von der Überſetzungs\pwindex{Amourette. Piece en trois actes. Adaptee de Arthur Schnitzler@\emph{Amourette. Pièce en trois actes. Adaptée de Arthur Schnitzler}|pwv}-Angelegenheit betreffend die {\pb}»Liebelei\pwindex{Liebelei. Schauspiel in drei Akten@\emph{Liebelei. Schauspiel in drei Akten}|pw}« habe ich einſtweilen wenig
               Erfreuliches zu melden. Ich hatte dieſer Tage Rendezvous mit \textsc{Thorel\pwindex{Thorel, Jean 1859-09-11 – 1916-08-20@\textsc{Thorel, Jean} (1859-09-11 – 1916-08-20), \emph{Übersetzer/Übersetzerin, Dramatiker/Dramatikerin}|pw}}. Er hat Schritte bei \textsc{Carré\pwindex{Carre, Albert 22.06.1852 – 11.12.1938@\textsc{Carré, Albert} (22.06.1852 – 11.12.1938), \emph{Schriftsteller/Schriftstellerin, Theaterleiter/Theaterleiterin, Schauspieler/Schauspielerin}|pw}}, dem Director\pwindex{Carre, Albert 22.06.1852 – 11.12.1938@\textsc{Carré, Albert} (22.06.1852 – 11.12.1938), \emph{Schriftsteller/Schriftstellerin, Theaterleiter/Theaterleiterin, Schauspieler/Schauspielerin}|pwv} des »\textsc{Vaudeville\orgindex{Theâtre du Vaudeville@Théâtre du Vaudeville|pw}}« gethan; aber \textsc{Carré\pwindex{Carre, Albert 22.06.1852 – 11.12.1938@\textsc{Carré, Albert} (22.06.1852 – 11.12.1938), \emph{Schriftsteller/Schriftstellerin, Theaterleiter/Theaterleiterin, Schauspieler/Schauspielerin}|pw}} hat geantwortet: das Pariſ\oindex{Paris@\textbf{Paris}, \emph{P.PPLC}|pw}er Publicum
               intereſſire ſich nicht mehr für fremde Stücke (was wahr iſt), intereſſire ſich nicht
               für \textsc{\label{K_L02766-3v}\edtext{\begin{otherlanguage}{french}moeurs Vienn\oindex{Wien@\textbf{Wien}, \emph{A.ADM2}|pwv}oises\end{otherlanguage}}{\lemma{\textnormal{\emph{moeurs Viennoises}}}\Cendnote{\textnormal{französisch: Wien\oindex{Wien@\textbf{Wien}, \emph{A.ADM2}|pwk}er Sitten}}}\label{K_L02766-3} etc}. Immerhin, wenn \textsc{Thorel\pwindex{Thorel, Jean 1859-09-11 – 1916-08-20@\textsc{Thorel, Jean} (1859-09-11 – 1916-08-20), \emph{Übersetzer/Übersetzerin, Dramatiker/Dramatikerin}|pw}}{ }\strikeout{es} das Stück\pwindex{Liebelei. Schauspiel in drei Akten@\emph{Liebelei. Schauspiel in drei Akten}|pwv} überſetzen wolle, werde er es gern leſen. Das iſt kein
               abſolutes Nein, aber es iſt nicht viel Hoffnung {\pb}in
               der Antwort. Ich denke daran, die Überſetzung\pwindex{Amourette. Piece en trois actes. Adaptee de Arthur Schnitzler@\emph{Amourette. Pièce en trois actes. Adaptée de Arthur Schnitzler}|pwv} eventuell der \strikeout{\textsc{Réjane\pwindex{Rejane 1856-06-05 – 1920-06-14@\textsc{Réjane} (1856-06-05 – 1920-06-14), \emph{Schauspieler/Schauspielerin}|pw}}}{ }\textsc{Réjane\pwindex{Rejane 1856-06-05 – 1920-06-14@\textsc{Réjane} (1856-06-05 – 1920-06-14), \emph{Schauspieler/Schauspielerin}|pw}} zu ſenden. Wenn dieſe das Stück\pwindex{Liebelei. Schauspiel in drei Akten@\emph{Liebelei. Schauspiel in drei Akten}|pwv} ſpielen will, iſt die Sache gemacht, trotz der Anſichten \textsc{Carrés\pwindex{Carre, Albert 22.06.1852 – 11.12.1938@\textsc{Carré, Albert} (22.06.1852 – 11.12.1938), \emph{Schriftsteller/Schriftstellerin, Theaterleiter/Theaterleiterin, Schauspieler/Schauspielerin}|pw}} über die \textsc{\begin{otherlanguage}{french}moeurs Vienn\oindex{Wien@\textbf{Wien}, \emph{A.ADM2}|pwv}oises\end{otherlanguage}}. Aber dazu muß es erſt überſetzt ſein. Das einzige \introOben{}große\introOben{} Theater, das außer dem \textsc{Vaudeville\orgindex{Theâtre du Vaudeville@Théâtre du Vaudeville|pw}}{ }\strikeout{ſ} noch in Betracht käme, wäre \textsc{Sarah Bernhardts\pwindex{Bernhardt, Sarah 22.10.1844 – 26.03.1923@\textsc{Bernhardt, Sarah} (22.10.1844 – 26.03.1923), \emph{Schauspieler/Schauspielerin}|pw}{ }Renaissance\orgindex{Theâtre de la Renaissance@Théâtre de la Renaissance|pw}}, die \textsc{Sudermanns\pwindex{Sudermann, Hermann 30.09.1857 – 21.11.1928@\textsc{Sudermann, Hermann} (30.09.1857 – 21.11.1928), \emph{Schriftsteller/Schriftstellerin}|pw}} »Heimath\pwindex{Heimat@\emph{Heimat}|pw}« geſpielt hat. Aber ich glaube, da iſt erſt recht
               keine Ausſicht, denn \textsc{Sarah\pwindex{Bernhardt, Sarah 22.10.1844 – 26.03.1923@\textsc{Bernhardt, Sarah} (22.10.1844 – 26.03.1923), \emph{Schauspieler/Schauspielerin}|pw}} wird kaum ein {\pb}ausländiſches Stück
                  ſpielen\textcolor{gray}{,} das keine Rolle für ſie enthält. Bleiben die freien
                  Bühnen\orgindex{Theâtre de l Œuvre@Théâtre de l’Œuvre|pwv}\orgindex{Theâtre Libre@Théâtre Libre|pwv}\orgindex{Theâtre des Escholiers@Théâtre des Escholiers|pwv}:
                  \textsc{Œuvre\orgindex{Theâtre de l Œuvre@Théâtre de l’Œuvre|pw}, Théâtre
                     Libre\orgindex{Theâtre Libre@Théâtre Libre|pw}, Escholiers\orgindex{Theâtre des Escholiers@Théâtre des Escholiers|pw} etc}. \strikeout{Hi} Hier ſetzen wir ſo gut wie ſicher eine Aufführung
               durch. Aber wie wird man da Dein ſchönes Stück\pwindex{Liebelei. Schauspiel in drei Akten@\emph{Liebelei. Schauspiel in drei Akten}|pwv} ſpielen!\pend
           
\pstart
           Für alle weiteren Schritte iſt es \strikeout{a} jedenfalls
               nothwendig, daß wir eine Überſetzung zur Hand haben. Dieſe iſt aber nur zu bekommen,
               wenn man zahlt. \textsc{Thorel\pwindex{Thorel, Jean 1859-09-11 – 1916-08-20@\textsc{Thorel, Jean} (1859-09-11 – 1916-08-20), \emph{Übersetzer/Übersetzerin, Dramatiker/Dramatikerin}|pw}} iſt ein armer \strikeout{T\textcolor{gray}{e}} Teufel, {\pb}der von ſeiner Feder lebt. Er kann
               ſich nicht an eine größere Arbeit machen, ohne daß man ſie ihm ſofort honorirt. \strikeout{Wer} Der \label{K_L02766-4v}\edtext{Herr\pwindex{Riaz, Henri de 1871 – 1951@\textsc{Riaz, Henri de} (1871 – 1951), \emph{Dichter/Dichterin}|pwv}\strikeout{\textcolor{gray}{n}} in \textsc{Lyon\oindex{Lyon@\textbf{Lyon}, \emph{P.PPLA}|pw}}}{\lemma{\textnormal{\emph{Herr in Lyon}}}\Cendnote{\textnormal{Siehe Paul Goldmann an Arthur Schnitzler, 11. 1. [1896].
               }}}\label{K_L02766-4} würde die Sache vielleicht umſonſt machen, aber nochmals: es wäre barer
               Unſinn, aus \textsc{Lyon\oindex{Lyon@\textbf{Lyon}, \emph{P.PPLA}|pw}} ſich eine Überſetzung kommen zu laſſen. \strikeout{Die} Was
               aus der Provinz kommt, gilt hier für ſchlecht. Mein Rath iſt einſtweilen der: Warten
               wir die Berlin\oindex{Berlin@\textbf{Berlin}, \emph{P.PPLC}|pw}er Aufführung {\pb}ab. Ich werde ſuchen, \label{K_L02766-5v}\edtext{etwas\pwindex{Courrier des Theâtres [Liebelei-Premiere Berlin]@\emph{Courrier des Théâtres [Liebelei-Premiere Berlin]}|pwv}}{\lemma{\textnormal{\emph{etwas}}}\Cendnote{\textnormal{Siehe Paul Goldmann an Arthur Schnitzler, 6. 2. [1896].
               }}}\label{K_L02766-5} darüber in die hieſigen Blätter zu bringen. (Wenn es Dir nicht zuviel Mühe
               macht, ſchickſt Du mir wohl ein kleines Telegramm am nächſten Morgen). Dann wollen
               wir ſehen. Vielleicht bekommſt Du neue Anerbietungen von ernſten Leuten, welche die
               Sache umſonſt machen wollen. Wenn nicht, ſo geht auch mein Rath dahin, zu zahlen,
               umſomehr als kein anderer Weg da iſt. Entweder findeſt Du einen \label{K_L02766-6v}\edtext{Verleger}{\lemma{\textnormal{\emph{Verleger}}}\Cendnote{\textnormal{Jean Thorels\pwindex{Thorel, Jean 1859-09-11 – 1916-08-20@\textsc{Thorel, Jean} (1859-09-11 – 1916-08-20), \emph{Übersetzer/Übersetzerin, Dramatiker/Dramatikerin}|pwk} Übersetzung von \emph{Liebelei}\pwindex{Liebelei. Schauspiel in drei Akten@\emph{Liebelei. Schauspiel in drei Akten}|pwk}, \emph{Amourette. Pièce
                     en trois actes. Adaptée de Arthur Schnitzler}\pwindex{Amourette. Piece en trois actes. Adaptee de Arthur Schnitzler@\emph{Amourette. Pièce en trois actes. Adaptée de Arthur Schnitzler}|pwk}, wurde nur als
                  Bühnenmanuskript veröffentlicht.}}}\label{K_L02766-6}, {\pb}der
               die Koſten übernimmt, oder aber Du verwendeſt ſelbſt einen kleinen Theil der
               Einnahmen, die das Stück\pwindex{Liebelei. Schauspiel in drei Akten@\emph{Liebelei. Schauspiel in drei Akten}|pwv} Dir
               in Deutſchland\oindex{Deutschland@\textbf{Deutschland}, \emph{A.PCLI}|pw} bringt, darauf, eine franzöſiſche
               Überſetzung herſtellen zu laſſen, um eine Aufführung in \textsc{Paris\oindex{Paris@\textbf{Paris}, \emph{P.PPLC}|pw}} zu ermöglichen. Freilich mußt Du Dir denken, daß Du das Geld \label{K_L02766-7v}\edtext{\textsc{à fonds perdus}}{\lemma{\textnormal{\emph{à fonds perdus}}}\Cendnote{\textnormal{französisch: verlorenes Kapital, ohne
                  Aussicht auf Rückgewinn}}}\label{K_L02766-7} ausgibſt; denn eine \uline{abſolute} Garantie für die Aufführung kann man nicht gewähren. \textsc{Thorel\pwindex{Thorel, Jean 1859-09-11 – 1916-08-20@\textsc{Thorel, Jean} (1859-09-11 – 1916-08-20), \emph{Übersetzer/Übersetzerin, Dramatiker/Dramatikerin}|pw}} würde Dir die Überſetzung\pwindex{Amourette. Piece en trois actes. Adaptee de Arthur Schnitzler@\emph{Amourette. Pièce en trois actes. Adaptée de Arthur Schnitzler}|pwv} wohl für \label{K_L02766-8v}\edtext{500 \textsc{Francs}{ }}{\lemma{\textnormal{\emph{500 Francs }}}\Cendnote{\textnormal{Siehe Paul Goldmann an Arthur Schnitzler, 26. 9. [1896].
               }}}\label{K_L02766-8}{\pb}herſtellen. Er ſprach zwar von 200 pro Akt\pwindex{Liebelei. Schauspiel in drei Akten@\emph{Liebelei. Schauspiel in drei Akten}|pwv}, aber ich handle ſchon
               noch 100 herunter. Warten wir alſo einſtweilen noch ein paar Wochen\strikeout{\textcolor{gray}{n}} und reden wir dann weiter über die Sache.\pend
           
\pstart
           Ich hoffe, Du ſchreibſt mir ein paar Zeilen über Deine Berlin\oindex{Berlin@\textbf{Berlin}, \emph{P.PPLC}|pw}er Eindrücke und Erlebniſſe, die gewiß gut und froh ſein werden. In
                  Berlin\oindex{Berlin@\textbf{Berlin}, \emph{P.PPLC}|pw} habe ich einen Onkel\pwindex{Mamroth, Hermann @\textsc{Mamroth, Hermann}|pwv}, den Bruder\pwindex{Mamroth, Hermann @\textsc{Mamroth, Hermann}|pwv} meiner Mutter\pwindex{Goldmann, Clementine 1842-05-15 – 1924-02-24@\textsc{Goldmann, Clementine} (1842-05-15 – 1924-02-24)|pwv}, einen braven, einfachen und \strikeout{ſeelens} ſeelensguten {\pb}Mann\strikeout{\textcolor{gray}{e}}, der mich erzogen hat. Er heißt \textsc{Hermann Mamroth\pwindex{Mamroth, Hermann @\textsc{Mamroth, Hermann}|pw}} und wohnt \textsc{Bruecken-Allee} 8\oindex{Bartningallee@\textbf{Bartningallee}, \emph{Straße (K.STR)}|pw}. Wenn es Dir möglich wäre, ihm ein
               Billet zu einer Deiner Aufführungen zu ſchicken oder gar ihn zu \label{K_L02766-9v}\edtext{beſuchen}{\lemma{\textnormal{\emph{beſuchen}}}\Cendnote{\textnormal{Ein Besuch lässt sich nicht belegen.}}}\label{K_L02766-9}, ſo würdeſt \strikeout{Du} Du ihm und mir eine große Freude machen. Wenn es Dir
               aber auch nur die mindeſten Umſtände macht, ſo laß’ \strikeout{es} es gehen {\pb}und betrachte dieſe Zeilen
               als nicht geſchrieben{\dotsfive}\pend
           
\pstart
           Dein Bericht über die \label{K_L02766-10v}\edtext{Unterredung mit
                  \textsc{Bahr\pwindex{Bahr, Hermann 19.07.1863 – 15.01.1934@\textsc{Bahr, Hermann} (19.07.1863 – 15.01.1934), \emph{Schriftsteller/Schriftstellerin, Kritiker/Kritikerin}|pw}}}{\lemma{\textnormal{\emph{Unterredung mit
                  Bahr}}}\Cendnote{\textnormal{Siehe A. S.: \emph{Tagebuch}, 21. 1. 1896.
               }}}\label{K_L02766-10} hat mich ungemein intereſſirt. Aber geh’ mir doch mit all’ der complicirten
               Pſychologie. Setzen wir die einfachen Worte, die das Herz erleichtern: \textsc{Bahr\pwindex{Bahr, Hermann 19.07.1863 – 15.01.1934@\textsc{Bahr, Hermann} (19.07.1863 – 15.01.1934), \emph{Schriftsteller/Schriftstellerin, Kritiker/Kritikerin}|pw}} iſt ſo zu Dir, \strikeout{weil} weil er ein Schurke iſt,
               und er haßt Dich, weil er neidiſch auf Dich iſt. Das iſt der Kern der Sache. Dem
               kleinen {\pb}\textsc{Hugo\pwindex{Hofmannsthal, Hugo von 1874-02-01 – 1929-07-15@\textsc{Hofmannsthal, Hugo von} (1874-02-01 – 1929-07-15), \emph{Schriftsteller/Schriftstellerin}|pw}} bin ich ſehr böſe. Man kann ſich wohl über Deine \strikeout{Lau} Launen ärgern, aber man ſchwankt nicht über die \label{K_L02766-11v}\edtext{Stellung zu Dir}{\lemma{\textnormal{\emph{Stellung zu Dir}}}\Cendnote{\textnormal{Siehe A. S.: \emph{Tagebuch}, 21. 12. 1895.
               }}}\label{K_L02766-11}. Leute, die nicht klar ſehen, wer und was Du biſt, haben ſelber einen Defect.
               Ich erwarte mir längſt allerlei Enttäuſchungen \strikeout{über}
               von dem kleinen \textsc{Hugo\pwindex{Hofmannsthal, Hugo von 1874-02-01 – 1929-07-15@\textsc{Hofmannsthal, Hugo von} (1874-02-01 – 1929-07-15), \emph{Schriftsteller/Schriftstellerin}|pw}} – vor allen \strikeout{Di} Dingen auf der Character-Seite.
               Er iſt viel zu eitel für ſeine jungen Jahre. Der Schurke \textsc{Bahr\pwindex{Bahr, Hermann 19.07.1863 – 15.01.1934@\textsc{Bahr, Hermann} (19.07.1863 – 15.01.1934), \emph{Schriftsteller/Schriftstellerin, Kritiker/Kritikerin}|pw}} trägt {\pb}die Hauptſchuld daran, aber auch Ihr
               habt Schuld, denn Ihr habt ihn verziehen helfen{\dotsfive}\pend
           
\pstart
           Wenn Du alſo irgend etwas in Berlin\oindex{Berlin@\textbf{Berlin}, \emph{P.PPLC}|pw} brauchſt, ſo
               telegraphire. Du haſt Recht, auf alle Empfehlungen zu verzichten. Die beſte
               Empfehlung iſt Dein Stück\pwindex{Liebelei. Schauspiel in drei Akten@\emph{Liebelei. Schauspiel in drei Akten}|pwv}.\pend
           
\pstart
           Und nun von Herzen Glück für Dienſtag!\pend
           
\pstart
           In Treue{\\[\baselineskip]}Dein {\\[\baselineskip]}\spacefill\mbox{Paul Goldmnn}\pend
           \leftskip=0em{}
\pstart
           \noindent{}\label{T_L02766-1v}\edtext{Autograph meiner Schweſter\pwindex{Rosengart, Vally 1866-12-29 – nach 1926@\textsc{Rosengart, Vally} (1866-12-29 – nach 1926)|pwv}, das eben eintrifft:}{\lemma{\textnormal{\emph{Autograph … eintrifft:}}}\Cendnote{\textnormal{Klebespuren legen nahe, dass die Beilage
                     ursprünglich auf die letzte Seite geklebt war.}}}\label{T_L02766-1}\pend
           
\pstart
           {\pb}{[}hs. :{]} \textsc{Schnitzler} iſt ein lieber, reizender Mensch\pend
           \selectlanguage{ngerman}\endnumbering\briefempfaengerindex{Schnitzler, Arthur@\textsc{Schnitzler, Arthur}!zzzGoldmann, Paul@\emph{von Paul Goldmann}!1896-02-011@{1. 2. {[}1896{]}}|)be}\mylabel{L02766h}  \normalsize

\doendnotes{C}
\bigskip
\vfill

\clearpage

\footnotesize

\lohead{\textsc{register}}

% Definiere theindex-Environment komplett neu ohne reledmac
\makeatletter
\renewenvironment{theindex}{%
  \section*{\indexname}%
  \setlength{\parindent}{0pt}%
  \setlength{\parskip}{0pt plus 0.3pt}%
  \let\item\@idxitem
}{%
  \clearpage
}
\makeatother

\IfFileExists{\jobname-pw.ind}{\input{\jobname-pw.ind}}{}

\end{document}

      