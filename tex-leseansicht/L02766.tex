%% latex-leseansicht-vorspann.tex
%% Vorspann für die Leseansicht.
%% Lädt die gemeinsame Datei latex-vorspann.tex mit nicht gesetztem Schalter.

\newif\ifkorrekturansicht
\korrekturansichtfalse

\input{../tex-inputs/latex-vorspann}


\section[ Paul Goldmann an Arthur Schnitzler, 1. 2. [1896]]{L02766 Paul Goldmann an Arthur Schnitzler,  1. 2. [1896]}
\nopagebreak\mylabel{L02766v}
\rehead{ }\normalsize\beginnumbering\briefempfaengerindex{Schnitzler, Arthur@\textsc{Schnitzler, Arthur}!zzzGoldmann, Paul@\emph{von Paul Goldmann}!1896-02-011@{1. 2. [1896]}|(be}
\toendnotes[C]{\smallbreak\pagebreak[2]}
\correspDesc{Versand  durch Paul Goldmann am 1. 2. [1896] in Paris
\newline{}Erhalt  durch Arthur Schnitzler im Zeitraum [2. 2. 1896
                  – 6. 2. 1896?] in Wien}\toendnotes[C]{\smallbreak}
\Standort{DLA, A:Schnitzler, HS.NZ85.1.3166.}
\physDesc{Brief, 4 Blätter, 14 Seiten, 4746 Zeichen
\newline{}Handschrift: blaue Tinte, deutsche Kurrent
\newline{}Beilage: handschriftlicher Brief: 1 stark beschnittener Ausschnitt aus
                                 einem Brief von Wally
                                    Rosengart\pwindex{Rosengart, Vally 29.\,12.\,1866 Breslau – nach 1926@\textsc{Rosengart, Vally} (29.\,12.\,1866 Breslau – nach 1926)|pw} an Goldmann\pwindex{Goldmann, Paul 31.\,1.\,1865 Breslau – 25.\,9.\,1935 Wien@\textsc{Goldmann, Paul} (31.\,1.\,1865 Breslau – 25.\,9.\,1935 Wien), \emph{Schriftsteller, Journalist}|pw}, blaue Tinte, deutsche Kurrentschrift. Auf der
                                 Rückseite des Schnipsels steht: »{\pb}Mein lieber Paul\pwindex{Goldmann, Paul 31.\,1.\,1865 Breslau – 25.\,9.\,1935 Wien@\textsc{Goldmann, Paul} (31.\,1.\,1865 Breslau – 25.\,9.\,1935 Wien), \emph{Schriftsteller, Journalist}|pw} – es fehlt \damage{uns} leider alles, um d\textcolor{gray}{en}« 
\newline{}Schnitzler: 1) mit Bleistift das Jahr »96« vermerkt  2) mit rotem Buntstift zwei Unterstreichungen}\toendnotes[C]{\smallbreak}
\pstart
           {\pb}\textcolor{gray}{\textbf{\textbf{Frankfurter Zeitung\orgindex{Frankfurter Zeitung@Frankfurter Zeitung|pw}}}}\pend
           
\pstart
           \textcolor{gray}{\textbf{(\begin{otherlanguage}{french}Gazette de Francfort\end{otherlanguage}\orgindex{Frankfurter Zeitung@Frankfurter Zeitung|pw}).}}\pend
           
\pstart
           \textcolor{gray}{\textbf{\textbf{\begin{otherlanguage}{french}Fondateur M.\end{otherlanguage}{ }L. Sonnemann\pwindex{Sonnemann, Leopold 29.\,10.\,1831 Höchberg – 30.\,10.\,1909 Frankfurt am Main@\textsc{Sonnemann, Leopold} (29.\,10.\,1831 Höchberg – 30.\,10.\,1909 Frankfurt am Main), \emph{Journalist, Herausgeber}|pw}.}}}\pend
           
\pstart
           \begin{otherlanguage}{french}\textcolor{gray}{\textbf{Journal\pwindex{Frankfurter Zeitung@\emph{Frankfurter Zeitung}|pwv} politique,
                        financier,}}\end{otherlanguage}\pend
           
\pstart
           \begin{otherlanguage}{french}\textcolor{gray}{\textbf{commercial et littéraire.}}\end{otherlanguage}\pend
           
\pstart
           \begin{otherlanguage}{french}\textcolor{gray}{\textbf{\textbf{Paraissant trois fois par jour.}}}\end{otherlanguage}\hfill \textsc{Paris\oindex{Paris@\textbf{Paris}, \emph{Hauptstadt}|pw}}, 1. Februar.\pend
           
\pstart
           \begin{otherlanguage}{french}\textcolor{gray}{\textbf{\textbf{Bureau à Paris\oindex{Paris@\textbf{Paris}, \emph{Hauptstadt}|pw}:}}}\end{otherlanguage}\pend
           
\pstart
           \begin{otherlanguage}{french}\textcolor{gray}{\textbf{\textbf{24. Rue Feydeau\oindex{rue Feydeau@\textbf{rue Feydeau}, \emph{Straße}|pw}.}}}\end{otherlanguage}\pend
           
\pstart\center{}Mein lieber Freund,\pend\vspace{0.5em}
\pstart
           Herzlich willkommen in \label{K_L02766-1v}\edtext{Berlin\oindex{Berlin@\textbf{Berlin}, \emph{Hauptstadt}|pw}}{\lemma{\textnormal{\emph{Berlin}}}\Cendnote{\textnormal{Für die Premiere von \emph{Liebelei}\pwindex{Schnitzler, Arthur 15.\,5.\,1862 Wien – 21.\,10.\,1931 ebd.@\textsc{Schnitzler, Arthur} (15.\,5.\,1862 Wien – 21.\,10.\,1931 ebd.), \emph{Schriftsteller, Mediziner}!Liebelei. Schauspiel in drei Akten@\strich\emph{Liebelei. Schauspiel in drei Akten}|pwk} am \emph{Deutschen
                     Theater}\orgindex{Deutsches Theater Berlin@Deutsches Theater Berlin|pwk} (4. 2. 1896) war Schnitzler
                  zwischen 30. 1. 1896
                  und 10. 2. 1896 in
                     Berlin\oindex{Berlin@\textbf{Berlin}, \emph{Hauptstadt}|pwk}.}}}\label{K_L02766-1}! Möge Dir neues Gute dort
               beſchieden{ }ſein!\pend
           
\pstart
           Ich hörte dieſer Tage, »Sterben\pwindex{Schnitzler, Arthur 15.\,5.\,1862 Wien – 21.\,10.\,1931 ebd.@\textsc{Schnitzler, Arthur} (15.\,5.\,1862 Wien – 21.\,10.\,1931 ebd.), \emph{Schriftsteller, Mediziner}!Mourir. Roman@\strich\emph{Mourir. Roman}|pwv}\pwindex{Schnitzler, Arthur 15.\,5.\,1862 Wien – 21.\,10.\,1931 ebd.@\textsc{Schnitzler, Arthur} (15.\,5.\,1862 Wien – 21.\,10.\,1931 ebd.), \emph{Schriftsteller, Mediziner}!Sterben. Novelle@\strich\emph{Sterben. Novelle}|pw}« werde demnächſt hier bei \textsc{Perrin\orgindex{Éditions Perrin@Éditions Perrin|pw}} erſcheinen u. \textsc{Ed. Rod\pwindex{Rod, Édouard 31.\,3.\,1857 Nyon – 1910 Grasse@\textsc{Rod, Édouard} (31.\,3.\,1857 Nyon – 1910 Grasse), \emph{Schriftsteller}|pw}} intereſſire{ }ſich ganz beſonders dafür. Das wird Dir hoffentlich einen großen
                  \label{K_L02766-2v}\edtext{Artikel in den »\textsc{Débats\pwindex{Journal des débats. Politiques et littéraires@\emph{Journal des débats. Politiques et littéraires}|pw}}}{\lemma{\textnormal{\emph{Artikel in den »Débats}}}\Cendnote{\textnormal{Dazu kam es nicht.}}}\label{K_L02766-2}« eintragen, zu deſſen Literatur-Referenten \textsc{Rod\pwindex{Rod, Édouard 31.\,3.\,1857 Nyon – 1910 Grasse@\textsc{Rod, Édouard} (31.\,3.\,1857 Nyon – 1910 Grasse), \emph{Schriftsteller}|pw}} gehört.\pend
           
\pstart
           Von der Überſetzungs\pwindex{Schnitzler, Arthur 15.\,5.\,1862 Wien – 21.\,10.\,1931 ebd.@\textsc{Schnitzler, Arthur} (15.\,5.\,1862 Wien – 21.\,10.\,1931 ebd.), \emph{Schriftsteller, Mediziner}!Amourette. Pièce en trois actes. Adaptée de Arthur Schnitzler@\strich\emph{Amourette. Pièce en trois actes. Adaptée de Arthur Schnitzler}|pwv}-Angelegenheit betreffend die {\pb}»Liebelei\pwindex{Schnitzler, Arthur 15.\,5.\,1862 Wien – 21.\,10.\,1931 ebd.@\textsc{Schnitzler, Arthur} (15.\,5.\,1862 Wien – 21.\,10.\,1931 ebd.), \emph{Schriftsteller, Mediziner}!Liebelei. Schauspiel in drei Akten@\strich\emph{Liebelei. Schauspiel in drei Akten}|pw}« habe ich einſtweilen wenig
               Erfreuliches zu melden. Ich hatte dieſer Tage Rendezvous mit \textsc{Thorel\pwindex{Thorel, Jean 11.\,9.\,1859 Éragny – 20.\,8.\,1916 Enghien-les-Bains@\textsc{Thorel, Jean} (11.\,9.\,1859 Éragny – 20.\,8.\,1916 Enghien-les-Bains), \emph{Übersetzer, Dramatiker}|pw}}. Er hat Schritte bei \textsc{Carré\pwindex{Carré, Albert 22.\,6.\,1852 Straßburg – 11.\,12.\,1938 Paris@\textsc{Carré, Albert} (22.\,6.\,1852 Straßburg – 11.\,12.\,1938 Paris), \emph{Schriftsteller, Theaterleiter, Schauspieler}|pw}}, dem Director\pwindex{Carré, Albert 22.\,6.\,1852 Straßburg – 11.\,12.\,1938 Paris@\textsc{Carré, Albert} (22.\,6.\,1852 Straßburg – 11.\,12.\,1938 Paris), \emph{Schriftsteller, Theaterleiter, Schauspieler}|pwv} des »\textsc{Vaudeville\orgindex{Théâtre du Vaudeville@Théâtre du Vaudeville|pw}}« gethan; aber \textsc{Carré\pwindex{Carré, Albert 22.\,6.\,1852 Straßburg – 11.\,12.\,1938 Paris@\textsc{Carré, Albert} (22.\,6.\,1852 Straßburg – 11.\,12.\,1938 Paris), \emph{Schriftsteller, Theaterleiter, Schauspieler}|pw}} hat geantwortet: das Pariſ\oindex{Paris@\textbf{Paris}, \emph{Hauptstadt}|pw}er Publicum
               intereſſire{ }ſich nicht mehr für fremde Stücke (was wahr iſt), intereſſire{ }ſich nicht
               für \textsc{\label{K_L02766-3v}\edtext{\begin{otherlanguage}{french}moeurs Vienn\oindex{Wien@\textbf{Wien}, \emph{Verwaltungsgebiet}|pwv}oises\end{otherlanguage}}{\lemma{\textnormal{\emph{moeurs Viennoises}}}\Cendnote{\textnormal{französisch: Wien\oindex{Wien@\textbf{Wien}, \emph{Verwaltungsgebiet}|pwk}er Sitten}}}\label{K_L02766-3} etc}. Immerhin, wenn \textsc{Thorel\pwindex{Thorel, Jean 11.\,9.\,1859 Éragny – 20.\,8.\,1916 Enghien-les-Bains@\textsc{Thorel, Jean} (11.\,9.\,1859 Éragny – 20.\,8.\,1916 Enghien-les-Bains), \emph{Übersetzer, Dramatiker}|pw}}{ }\strikeout{es} das Stück\pwindex{Schnitzler, Arthur 15.\,5.\,1862 Wien – 21.\,10.\,1931 ebd.@\textsc{Schnitzler, Arthur} (15.\,5.\,1862 Wien – 21.\,10.\,1931 ebd.), \emph{Schriftsteller, Mediziner}!Liebelei. Schauspiel in drei Akten@\strich\emph{Liebelei. Schauspiel in drei Akten}|pwv} überſetzen wolle, werde er es gern leſen. Das iſt kein
               abſolutes Nein, aber es iſt nicht viel Hoffnung {\pb}in
               der Antwort. Ich denke daran, die Überſetzung\pwindex{Schnitzler, Arthur 15.\,5.\,1862 Wien – 21.\,10.\,1931 ebd.@\textsc{Schnitzler, Arthur} (15.\,5.\,1862 Wien – 21.\,10.\,1931 ebd.), \emph{Schriftsteller, Mediziner}!Amourette. Pièce en trois actes. Adaptée de Arthur Schnitzler@\strich\emph{Amourette. Pièce en trois actes. Adaptée de Arthur Schnitzler}|pwv} eventuell der \strikeout{\textsc{Réjane\pwindex{Réjane 5.\,6.\,1856 Paris – 14.\,6.\,1920 Asnières-sur-Seine@\textsc{Réjane} (5.\,6.\,1856 Paris – 14.\,6.\,1920 Asnières-sur-Seine), \emph{Schauspielerin}|pw}}}{ }\textsc{Réjane\pwindex{Réjane 5.\,6.\,1856 Paris – 14.\,6.\,1920 Asnières-sur-Seine@\textsc{Réjane} (5.\,6.\,1856 Paris – 14.\,6.\,1920 Asnières-sur-Seine), \emph{Schauspielerin}|pw}} zu{ }ſenden. Wenn dieſe das Stück\pwindex{Schnitzler, Arthur 15.\,5.\,1862 Wien – 21.\,10.\,1931 ebd.@\textsc{Schnitzler, Arthur} (15.\,5.\,1862 Wien – 21.\,10.\,1931 ebd.), \emph{Schriftsteller, Mediziner}!Liebelei. Schauspiel in drei Akten@\strich\emph{Liebelei. Schauspiel in drei Akten}|pwv}{ }ſpielen will, iſt die Sache gemacht, trotz der Anſichten \textsc{Carrés\pwindex{Carré, Albert 22.\,6.\,1852 Straßburg – 11.\,12.\,1938 Paris@\textsc{Carré, Albert} (22.\,6.\,1852 Straßburg – 11.\,12.\,1938 Paris), \emph{Schriftsteller, Theaterleiter, Schauspieler}|pw}} über die \textsc{\begin{otherlanguage}{french}moeurs Vienn\oindex{Wien@\textbf{Wien}, \emph{Verwaltungsgebiet}|pwv}oises\end{otherlanguage}}. Aber dazu muß es erſt überſetzt{ }ſein. Das einzige \introOben{}große\introOben{} Theater, das außer dem \textsc{Vaudeville\orgindex{Théâtre du Vaudeville@Théâtre du Vaudeville|pw}}{ }\strikeout{ſ} noch in Betracht käme, wäre \textsc{Sarah Bernhardts\pwindex{Bernhardt, Sarah 22.\,10.\,1844 Paris – 26.\,3.\,1923 ebd.@\textsc{Bernhardt, Sarah} (22.\,10.\,1844 Paris – 26.\,3.\,1923 ebd.), \emph{Schauspielerin}|pw}{ }Renaissance\orgindex{Théâtre de la Renaissance@Théâtre de la Renaissance|pw}}, die \textsc{Sudermanns\pwindex{Sudermann, Hermann 30.\,9.\,1857 Macikai – 21.\,11.\,1928 Berlin@\textsc{Sudermann, Hermann} (30.\,9.\,1857 Macikai – 21.\,11.\,1928 Berlin), \emph{Schriftsteller}|pw}} »Heimath\pwindex{Sudermann, Hermann 30.\,9.\,1857 Macikai – 21.\,11.\,1928 Berlin@\textsc{Sudermann, Hermann} (30.\,9.\,1857 Macikai – 21.\,11.\,1928 Berlin), \emph{Schriftsteller}!Heimat. Schauspiel in vier Akten@\strich\emph{Heimat. Schauspiel in vier Akten}|pw}« geſpielt hat. Aber ich glaube, da iſt erſt recht
               keine Ausſicht, denn \textsc{Sarah\pwindex{Bernhardt, Sarah 22.\,10.\,1844 Paris – 26.\,3.\,1923 ebd.@\textsc{Bernhardt, Sarah} (22.\,10.\,1844 Paris – 26.\,3.\,1923 ebd.), \emph{Schauspielerin}|pw}} wird kaum ein {\pb}ausländiſches Stück{ }ſpielen\textcolor{gray}{,} das keine Rolle für{ }ſie enthält. Bleiben die freien
                  Bühnen\orgindex{Théâtre de l’Œuvre@Théâtre de l’Œuvre|pwv}\orgindex{Théâtre Libre@Théâtre Libre|pwv}\orgindex{Théâtre des Escholiers@Théâtre des Escholiers|pwv}:
                  \textsc{Œuvre\orgindex{Théâtre de l’Œuvre@Théâtre de l’Œuvre|pw}, Théâtre
                     Libre\orgindex{Théâtre Libre@Théâtre Libre|pw}, Escholiers\orgindex{Théâtre des Escholiers@Théâtre des Escholiers|pw} etc}. \strikeout{Hi} Hier{ }ſetzen wir{ }ſo gut wie{ }ſicher eine Aufführung
               durch. Aber wie wird man da Dein{ }ſchönes Stück\pwindex{Schnitzler, Arthur 15.\,5.\,1862 Wien – 21.\,10.\,1931 ebd.@\textsc{Schnitzler, Arthur} (15.\,5.\,1862 Wien – 21.\,10.\,1931 ebd.), \emph{Schriftsteller, Mediziner}!Liebelei. Schauspiel in drei Akten@\strich\emph{Liebelei. Schauspiel in drei Akten}|pwv}{ }ſpielen!\pend
           
\pstart
           Für alle weiteren Schritte iſt es \strikeout{a} jedenfalls
               nothwendig, daß wir eine Überſetzung zur Hand haben. Dieſe iſt aber nur zu bekommen,
               wenn man zahlt. \textsc{Thorel\pwindex{Thorel, Jean 11.\,9.\,1859 Éragny – 20.\,8.\,1916 Enghien-les-Bains@\textsc{Thorel, Jean} (11.\,9.\,1859 Éragny – 20.\,8.\,1916 Enghien-les-Bains), \emph{Übersetzer, Dramatiker}|pw}} iſt ein armer \strikeout{T\textcolor{gray}{e}} Teufel, {\pb}der von{ }ſeiner Feder lebt. Er kann{ }ſich nicht an eine größere Arbeit machen, ohne daß man{ }ſie ihm{ }ſofort honorirt. \strikeout{Wer} Der \label{K_L02766-4v}\edtext{Herr\pwindex{Riaz, Henri de 1871 Lyon – 1951 Lausanne@\textsc{Riaz, Henri de} (1871 Lyon – 1951 Lausanne), \emph{Dichter}|pwv}\strikeout{\textcolor{gray}{n}} in \textsc{Lyon\oindex{Lyon@\textbf{Lyon}|pw}}}{\lemma{\textnormal{\emph{Herr in Lyon}}}\Cendnote{\textnormal{Siehe XXXX Auszeichnungsfehler: Dokument L02762 nicht gefunden.
               }}}\label{K_L02766-4} würde die Sache vielleicht umſonſt machen, aber nochmals: es wäre barer
               Unſinn, aus \textsc{Lyon\oindex{Lyon@\textbf{Lyon}|pw}}{ }ſich eine Überſetzung kommen zu laſſen. \strikeout{Die} Was
               aus der Provinz kommt, gilt hier für{ }ſchlecht. Mein Rath iſt einſtweilen der: Warten
               wir die Berlin\oindex{Berlin@\textbf{Berlin}, \emph{Hauptstadt}|pw}er Aufführung {\pb}ab. Ich werde{ }ſuchen, \label{K_L02766-5v}\edtext{etwas\pwindex{Courrier des Théâtres [Liebelei-Premiere Berlin]@\emph{Courrier des Théâtres [Liebelei-Premiere Berlin]}|pwv}}{\lemma{\textnormal{\emph{etwas}}}\Cendnote{\textnormal{Siehe XXXX Auszeichnungsfehler: Dokument L02767 nicht gefunden.
               }}}\label{K_L02766-5} darüber in die hieſigen Blätter zu bringen. (Wenn es Dir nicht zuviel Mühe
               macht,{ }ſchickſt Du mir wohl ein kleines Telegramm am nächſten Morgen). Dann wollen
               wir{ }ſehen. Vielleicht bekommſt Du neue Anerbietungen von ernſten Leuten, welche die
               Sache umſonſt machen wollen. Wenn nicht,{ }ſo geht auch mein Rath dahin, zu zahlen,
               umſomehr als kein anderer Weg da iſt. Entweder findeſt Du einen \label{K_L02766-6v}\edtext{Verleger}{\lemma{\textnormal{\emph{Verleger}}}\Cendnote{\textnormal{Jean Thorels\pwindex{Thorel, Jean 11.\,9.\,1859 Éragny – 20.\,8.\,1916 Enghien-les-Bains@\textsc{Thorel, Jean} (11.\,9.\,1859 Éragny – 20.\,8.\,1916 Enghien-les-Bains), \emph{Übersetzer, Dramatiker}|pwk} Übersetzung von \emph{Liebelei}\pwindex{Schnitzler, Arthur 15.\,5.\,1862 Wien – 21.\,10.\,1931 ebd.@\textsc{Schnitzler, Arthur} (15.\,5.\,1862 Wien – 21.\,10.\,1931 ebd.), \emph{Schriftsteller, Mediziner}!Liebelei. Schauspiel in drei Akten@\strich\emph{Liebelei. Schauspiel in drei Akten}|pwk}, \emph{Amourette. Pièce
                     en trois actes. Adaptée de Arthur Schnitzler}\pwindex{Schnitzler, Arthur 15.\,5.\,1862 Wien – 21.\,10.\,1931 ebd.@\textsc{Schnitzler, Arthur} (15.\,5.\,1862 Wien – 21.\,10.\,1931 ebd.), \emph{Schriftsteller, Mediziner}!Amourette. Pièce en trois actes. Adaptée de Arthur Schnitzler@\strich\emph{Amourette. Pièce en trois actes. Adaptée de Arthur Schnitzler}|pwk}, wurde nur als
                  Bühnenmanuskript veröffentlicht.}}}\label{K_L02766-6}, {\pb}der
               die Koſten übernimmt, oder aber Du verwendeſt{ }ſelbſt einen kleinen Theil der
               Einnahmen, die das Stück\pwindex{Schnitzler, Arthur 15.\,5.\,1862 Wien – 21.\,10.\,1931 ebd.@\textsc{Schnitzler, Arthur} (15.\,5.\,1862 Wien – 21.\,10.\,1931 ebd.), \emph{Schriftsteller, Mediziner}!Liebelei. Schauspiel in drei Akten@\strich\emph{Liebelei. Schauspiel in drei Akten}|pwv} Dir
               in Deutſchland\oindex{Deutschland@\textbf{Deutschland}|pw} bringt, darauf, eine franzöſiſche
               Überſetzung herſtellen zu laſſen, um eine Aufführung in \textsc{Paris\oindex{Paris@\textbf{Paris}, \emph{Hauptstadt}|pw}} zu ermöglichen. Freilich mußt Du Dir denken, daß Du das Geld \label{K_L02766-7v}\edtext{\textsc{à fonds perdus}}{\lemma{\textnormal{\emph{à fonds perdus}}}\Cendnote{\textnormal{französisch: verlorenes Kapital, ohne
                  Aussicht auf Rückgewinn}}}\label{K_L02766-7} ausgibſt; denn eine \uline{abſolute} Garantie für die Aufführung kann man nicht gewähren. \textsc{Thorel\pwindex{Thorel, Jean 11.\,9.\,1859 Éragny – 20.\,8.\,1916 Enghien-les-Bains@\textsc{Thorel, Jean} (11.\,9.\,1859 Éragny – 20.\,8.\,1916 Enghien-les-Bains), \emph{Übersetzer, Dramatiker}|pw}} würde Dir die Überſetzung\pwindex{Schnitzler, Arthur 15.\,5.\,1862 Wien – 21.\,10.\,1931 ebd.@\textsc{Schnitzler, Arthur} (15.\,5.\,1862 Wien – 21.\,10.\,1931 ebd.), \emph{Schriftsteller, Mediziner}!Amourette. Pièce en trois actes. Adaptée de Arthur Schnitzler@\strich\emph{Amourette. Pièce en trois actes. Adaptée de Arthur Schnitzler}|pwv} wohl für \label{K_L02766-8v}\edtext{500 \textsc{Francs}{ }}{\lemma{\textnormal{\emph{500 Francs }}}\Cendnote{\textnormal{Siehe XXXX Auszeichnungsfehler: Dokument L02786 nicht gefunden.
               }}}\label{K_L02766-8}{\pb}herſtellen. Er{ }ſprach zwar von 200 pro Akt\pwindex{Schnitzler, Arthur 15.\,5.\,1862 Wien – 21.\,10.\,1931 ebd.@\textsc{Schnitzler, Arthur} (15.\,5.\,1862 Wien – 21.\,10.\,1931 ebd.), \emph{Schriftsteller, Mediziner}!Liebelei. Schauspiel in drei Akten@\strich\emph{Liebelei. Schauspiel in drei Akten}|pwv}, aber ich handle{ }ſchon
               noch 100 herunter. Warten wir alſo einſtweilen noch ein paar Wochen\strikeout{\textcolor{gray}{n}} und reden wir dann weiter über die Sache.\pend
           
\pstart
           Ich hoffe, Du{ }ſchreibſt mir ein paar Zeilen über Deine Berlin\oindex{Berlin@\textbf{Berlin}, \emph{Hauptstadt}|pw}er Eindrücke und Erlebniſſe, die gewiß gut und froh{ }ſein werden. In
                  Berlin\oindex{Berlin@\textbf{Berlin}, \emph{Hauptstadt}|pw} habe ich einen Onkel\pwindex{Mamroth, Hermann @\textsc{Mamroth, Hermann}|pwv}, den Bruder\pwindex{Mamroth, Hermann @\textsc{Mamroth, Hermann}|pwv} meiner Mutter\pwindex{Goldmann, Clementine 15.\,5.\,1842 Breslau – 24.\,2.\,1924 Frankfurt am Main@\textsc{Goldmann, Clementine} (15.\,5.\,1842 Breslau – 24.\,2.\,1924 Frankfurt am Main)|pwv}, einen braven, einfachen und \strikeout{ſeelens}{ }ſeelensguten {\pb}Mann\strikeout{\textcolor{gray}{e}}, der mich erzogen hat. Er heißt \textsc{Hermann Mamroth\pwindex{Mamroth, Hermann @\textsc{Mamroth, Hermann}|pw}} und wohnt \textsc{Bruecken-Allee} 8\oindex{Bartningallee@\textbf{Bartningallee}, \emph{Straße}|pw}. Wenn es Dir möglich wäre, ihm ein
               Billet zu einer Deiner Aufführungen zu{ }ſchicken oder gar ihn zu \label{K_L02766-9v}\edtext{beſuchen}{\lemma{\textnormal{\emph{besuchen}}}\Cendnote{\textnormal{Ein Besuch lässt sich nicht belegen.}}}\label{K_L02766-9},{ }ſo würdeſt \strikeout{Du} Du ihm und mir eine große Freude machen. Wenn es Dir
               aber auch nur die mindeſten Umſtände macht,{ }ſo laß’ \strikeout{es} es gehen {\pb}und betrachte dieſe Zeilen
               als nicht geſchrieben{\dotsfive}\pend
           
\pstart
           Dein Bericht über die \label{K_L02766-10v}\edtext{Unterredung mit
                  \textsc{Bahr\pwindex{Bahr, Hermann 19.\,7.\,1863 Linz – 15.\,1.\,1934 München@\textsc{Bahr, Hermann} (19.\,7.\,1863 Linz – 15.\,1.\,1934 München), \emph{Schriftsteller, Kritiker}|pw}}}{\lemma{\textnormal{\emph{Unterredung mit
                  Bahr}}}\Cendnote{\textnormal{Siehe A. S.: \emph{Tagebuch}, 21. 1. 1896.
               }}}\label{K_L02766-10} hat mich ungemein intereſſirt. Aber geh’ mir doch mit all’ der complicirten
               Pſychologie. Setzen wir die einfachen Worte, die das Herz erleichtern: \textsc{Bahr\pwindex{Bahr, Hermann 19.\,7.\,1863 Linz – 15.\,1.\,1934 München@\textsc{Bahr, Hermann} (19.\,7.\,1863 Linz – 15.\,1.\,1934 München), \emph{Schriftsteller, Kritiker}|pw}} iſt{ }ſo zu Dir, \strikeout{weil} weil er ein Schurke iſt,
               und er haßt Dich, weil er neidiſch auf Dich iſt. Das iſt der Kern der Sache. Dem
               kleinen {\pb}\textsc{Hugo\pwindex{Hofmannsthal, Hugo von 1.\,2.\,1874 Wien – 15.\,7.\,1929 Rodaun@\textsc{Hofmannsthal, Hugo von} (1.\,2.\,1874 Wien – 15.\,7.\,1929 Rodaun), \emph{Schriftsteller}|pw}} bin ich{ }ſehr böſe. Man kann{ }ſich wohl über Deine \strikeout{Lau} Launen ärgern, aber man{ }ſchwankt nicht über die \label{K_L02766-11v}\edtext{Stellung zu Dir}{\lemma{\textnormal{\emph{Stellung zu Dir}}}\Cendnote{\textnormal{Siehe A. S.: \emph{Tagebuch}, 21. 12. 1895.
               }}}\label{K_L02766-11}. Leute, die nicht klar{ }ſehen, wer und was Du biſt, haben{ }ſelber einen Defect.
               Ich erwarte mir längſt allerlei Enttäuſchungen \strikeout{über}
               von dem kleinen \textsc{Hugo\pwindex{Hofmannsthal, Hugo von 1.\,2.\,1874 Wien – 15.\,7.\,1929 Rodaun@\textsc{Hofmannsthal, Hugo von} (1.\,2.\,1874 Wien – 15.\,7.\,1929 Rodaun), \emph{Schriftsteller}|pw}} – vor allen \strikeout{Di} Dingen auf der Character-Seite.
               Er iſt viel zu eitel für{ }ſeine jungen Jahre. Der Schurke \textsc{Bahr\pwindex{Bahr, Hermann 19.\,7.\,1863 Linz – 15.\,1.\,1934 München@\textsc{Bahr, Hermann} (19.\,7.\,1863 Linz – 15.\,1.\,1934 München), \emph{Schriftsteller, Kritiker}|pw}} trägt {\pb}die Hauptſchuld daran, aber auch Ihr
               habt Schuld, denn Ihr habt ihn verziehen helfen{\dotsfive}\pend
           
\pstart
           Wenn Du alſo irgend etwas in Berlin\oindex{Berlin@\textbf{Berlin}, \emph{Hauptstadt}|pw} brauchſt,{ }ſo
               telegraphire. Du haſt Recht, auf alle Empfehlungen zu verzichten. Die beſte
               Empfehlung iſt Dein Stück\pwindex{Schnitzler, Arthur 15.\,5.\,1862 Wien – 21.\,10.\,1931 ebd.@\textsc{Schnitzler, Arthur} (15.\,5.\,1862 Wien – 21.\,10.\,1931 ebd.), \emph{Schriftsteller, Mediziner}!Liebelei. Schauspiel in drei Akten@\strich\emph{Liebelei. Schauspiel in drei Akten}|pwv}.\pend
           
\pstart
           Und nun von Herzen Glück für Dienſtag!\pend
           
\pstart
           In Treue{\\[\baselineskip]}Dein {\\[\baselineskip]}\spacefill\mbox{Paul Goldmnn}\pend
           \leftskip=0em{}
\pstart
           \noindent{}\label{T_L02766-1v}\edtext{Autograph meiner Schweſter\pwindex{Rosengart, Vally 29.\,12.\,1866 Breslau – nach 1926@\textsc{Rosengart, Vally} (29.\,12.\,1866 Breslau – nach 1926)|pwv}, das eben eintrifft:}{\lemma{\textnormal{\emph{Autograph … eintrifft:}}}\Cendnote{\textnormal{Klebespuren legen nahe, dass die Beilage
                     ursprünglich auf die letzte Seite geklebt war.}}}\label{T_L02766-1}\pend
           
\pstart
           {\pb}{[}hs. Rosengart:{]} \textsc{Schnitzler} iſt ein lieber, reizender Mensch\pend
           \selectlanguage{ngerman}\endnumbering\briefempfaengerindex{Schnitzler, Arthur@\textsc{Schnitzler, Arthur}!zzzGoldmann, Paul@\emph{von Paul Goldmann}!1896-02-011@{1. 2. [1896]}|)be}\mylabel{L02766h}  \newcommand{\dateiname}{L02766}\newcommand{\titel}{Paul Goldmann an Arthur Schnitzler, 1. 2. [1896]}\newcommand{\editorInnen}{Martin Anton Müller und Laura Untner}%% latex-leseansicht-abspann.tex
%% Abspann für die Leseansicht.
%% Der Schalter \ifkorrekturansicht ist bereits durch den Vorspann gesetzt.

%% latex-abspann.tex
%% Gemeinsamer Abspann für Korrekturansicht und Leseansicht.
%% Setzt den Schalter \ifkorrekturansicht voraus (gesetzt in den
%% einbindenden Dateien latex-korrekturansicht-abspann.tex bzw.
%% latex-leseansicht-abspann.tex).
%% ---------------------------------------------------------------

\normalsize

% Das esempio-Environment wird nur in der Leseansicht benötigt
\ifkorrekturansicht\else
\newenvironment{esempio}[3]%
{
    \vspace{1.5ex}
    \rlap{\underline{#1}}
    \par
    \setlength{\parindent}{0cm}
    \nopagebreak
    \leftskip=#2cm
    \rightskip=#3cm
}
{
    \par
}
\fi

\doendnotes{C}
\bigskip
\vfill

\clearpage

\footnotesize

\ifkorrekturansicht
  \lohead{\textsc{register}}
\fi

% theindex-Environment neu definieren ohne reledmac
\makeatletter
\renewenvironment{theindex}{%
  \ifkorrekturansicht
    \section*{\indexname}%
  \else
    \subsubsection*{Index der erwähnten Entitäten}%
  \fi
  \setlength{\parindent}{0pt}%
  \setlength{\parskip}{0pt plus 0.3pt}%
  \let\item\@idxitem
}{%
  \ifkorrekturansicht\clearpage\fi
}
\makeatother

\IfFileExists{\jobname-pw.ind}{\input{\jobname-pw.ind}}{}

% Quellenangabe nur in der Leseansicht
\ifkorrekturansicht\else
% Fallback-Definitionen, falls die .tex-Datei \titel etc. nicht gesetzt hat
\providecommand{\titel}{}
\providecommand{\editorInnen}{}
\providecommand{\dateiname}{\jobname}

\vspace{3cm}

\vfill

\footnotesize
\textsc{Quelle}: \titel. Herausgegeben von {\editorInnen}. In: \emph{Arthur Schnitzler: Briefwechsel mit Autorinnen und Autoren}.
 Digitale Edition, https://schnitzler-briefe.acdh.oeaw.ac.at/{\dateiname}.html (Stand \today)
\fi

\end{document}


