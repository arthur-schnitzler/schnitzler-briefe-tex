%% latex-leseansicht-vorspann.tex
%% Vorspann für die Leseansicht.
%% Lädt die gemeinsame Datei latex-vorspann.tex mit nicht gesetztem Schalter.

\newif\ifkorrekturansicht
\korrekturansichtfalse

\input{../tex-inputs/latex-vorspann}

\begin{center}
            \textcolor{red}{ENTWURF, NICHT FERTIG KORRIGIERT}
                      \end{center}
            
         
         \renewcommand{\erwaehntePersonen}{Personen: Bertha Hausner, Felix Salten, Adele Sandrock}
         \renewcommand{\erwaehnteInstitutionen}{Institutionen: Volkstheater}
         \renewcommand{\erwaehnteOrte}{Orte: Café Central, Wien}
         \renewcommand{\erwaehnteWerke}{Werke: Francillon. Schauspiel in 3 Aufzügen}
               \section[Arthur Schnitzler an Felix Salten, {[}14. 4. 1894?{]}]{ Arthur Schnitzler an Felix Salten, {[}14. 4. 1894?{]}}\nopagebreak\mylabel{v}\rehead{ }\begin{ledgroupsized}[t]{13cm}\normalsize\beginnumbering \toendnotes[C]{\smallbreak\pagebreak[2]} \Standort{Wienbibliothek im Rathaus, ZPH 1681, 2.1.516.}
\physDesc{
\newline{}Handschrift: , deutsche Kurrent}\toendnotes[C]{\smallbreak}\pstart
           \noindent{}{\pb}Lieber Freund, \label{K_L03028-1v}\edtext{\textsc{Francillon\pwindex{\textcolor{red}{\textsuperscript{XXXX1 indx}}!Francillon. Schauspiel in 3 Aufzuegen1888@\strich\emph{Francillon. Schauspiel in 3 Aufzügen} {[}1888{]}|pw}} iſt \uline{abgeſagt}}{\lemma{\textnormal{\emph{Francillon iſt abgeſagt}}}\Cendnote{\textnormal{Für den 15. 4. 1894 war am \emph{Deutschen
                     Volkstheater}\orgindex{Volkstheater@Volkstheater|pwk}{ }\emph{Francillon}\pwindex{\textcolor{red}{\textsuperscript{XXXX1 indx}}!Francillon. Schauspiel in 3 Aufzuegen1888@\strich\emph{Francillon. Schauspiel in 3 Aufzügen} {[}1888{]}|pwk} mit Adele
                     Sandrock\pwindex{Sandrock, Adele 1863-08-19 – 1937-08-30@\textsc{Sandrock, Adele} (1863-08-19 – 1937-08-30), \emph{Schauspielerin}|pwk} als »Francine\pwindex{\textcolor{red}{\textsuperscript{XXXX1 indx}}!Francillon. Schauspiel in 3 Aufzuegen1888@\strich\emph{Francillon. Schauspiel in 3 Aufzügen} {[}1888{]}|pwkv}« angesetzt, musste aber wegen Erkrankung von Bertha Hausner\pwindex{Hausner, Bertha 1869-03-21 – 11.03.1932@\textsc{Hausner, Bertha} (1869-03-21 – 11.03.1932), \emph{Schauspielerin}|pwk} (»Anette\pwindex{\textcolor{red}{\textsuperscript{XXXX1 indx}}!Francillon. Schauspiel in 3 Aufzuegen1888@\strich\emph{Francillon. Schauspiel in 3 Aufzügen} {[}1888{]}|pwkv}«) kurzfristig abgesagt werden. Schnitzler\pwindex{Schnitzler, Arthur 15.05.1862 – 21.10.1931@\textsc{Schnitzler, Arthur} (15.05.1862 – 21.10.1931), \emph{Schriftsteller, Mediziner}|pwk} verbrachte den Abend bei Sandrock\pwindex{Sandrock, Adele 1863-08-19 – 1937-08-30@\textsc{Sandrock, Adele} (1863-08-19 – 1937-08-30), \emph{Schauspielerin}|pwk}, jedoch ohne Salten\pwindex{Salten, Felix 06.09.1869 – 08.10.1945@\textsc{Salten, Felix} (06.09.1869 – 08.10.1945), \emph{Schriftsteller, Journalist}|pwk}. Das erlaubt die Datierung des Korrespondenzstücks, wenngleich
                  nicht ausgeschlossen werden kann, dass auch ein anderer Abend im Zeitraum
                     1893/1894 in Frage kommt.}}}\label{K_L03028-1h}.– Wir gehen um
                  9 zu Frl. S.\pwindex{Sandrock, Adele 1863-08-19 – 1937-08-30@\textsc{Sandrock, Adele} (1863-08-19 – 1937-08-30), \emph{Schauspielerin}|pw}– Ich hole Sie um
                  ½ 9\textsc{Café Centra{[}l{]}\oindex{Cafe Central@\textbf{Café Central}|pw}} ab.– Herzlichen Gruß. \spacefill\mbox{Arth}\pend
           
         
         \endnumbering\mylabel{h}\end{ledgroupsized}\begin{anhang}\end{anhang}\newcommand{\dateiname}{L03028}\newcommand{\titel}{Arthur Schnitzler an Felix Salten, [14. 4. 1894?]}\newcommand{\editorInnen}{Martin Anton Müller und Laura Untner}%% latex-leseansicht-abspann.tex
%% Abspann für die Leseansicht.
%% Der Schalter \ifkorrekturansicht ist bereits durch den Vorspann gesetzt.

%% latex-abspann.tex
%% Gemeinsamer Abspann für Korrekturansicht und Leseansicht.
%% Setzt den Schalter \ifkorrekturansicht voraus (gesetzt in den
%% einbindenden Dateien latex-korrekturansicht-abspann.tex bzw.
%% latex-leseansicht-abspann.tex).
%% ---------------------------------------------------------------

\normalsize

% Das esempio-Environment wird nur in der Leseansicht benötigt
\ifkorrekturansicht\else
\newenvironment{esempio}[3]%
{
    \vspace{1.5ex}
    \rlap{\underline{#1}}
    \par
    \setlength{\parindent}{0cm}
    \nopagebreak
    \leftskip=#2cm
    \rightskip=#3cm
}
{
    \par
}
\fi

\doendnotes{C}
\bigskip
\vfill

\clearpage

\footnotesize

\ifkorrekturansicht
  \lohead{\textsc{register}}
\fi

% theindex-Environment neu definieren ohne reledmac
\makeatletter
\renewenvironment{theindex}{%
  \ifkorrekturansicht
    \section*{\indexname}%
  \else
    \subsubsection*{Index der erwähnten Entitäten}%
  \fi
  \setlength{\parindent}{0pt}%
  \setlength{\parskip}{0pt plus 0.3pt}%
  \let\item\@idxitem
}{%
  \ifkorrekturansicht\clearpage\fi
}
\makeatother

\IfFileExists{\jobname-pw.ind}{\input{\jobname-pw.ind}}{}

% Quellenangabe nur in der Leseansicht
\ifkorrekturansicht\else
% Fallback-Definitionen, falls die .tex-Datei \titel etc. nicht gesetzt hat
\providecommand{\titel}{}
\providecommand{\editorInnen}{}
\providecommand{\dateiname}{\jobname}

\vspace{3cm}

\vfill

\footnotesize
\textsc{Quelle}: \titel. Herausgegeben von {\editorInnen}. In: \emph{Arthur Schnitzler: Briefwechsel mit Autorinnen und Autoren}.
 Digitale Edition, https://schnitzler-briefe.acdh.oeaw.ac.at/{\dateiname}.html (Stand \today)
\fi

\end{document}


      