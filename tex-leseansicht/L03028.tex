%% latex-leseansicht-vorspann.tex
%% Vorspann für die Leseansicht.
%% Lädt die gemeinsame Datei latex-vorspann.tex mit nicht gesetztem Schalter.

\newif\ifkorrekturansicht
\korrekturansichtfalse

\input{../tex-inputs/latex-vorspann}


\section[Arthur Schnitzler an Felix Salten, {{[}}14. 4. 1894?{{]}}]{L03028 Arthur Schnitzler an Felix Salten, {[}14. 4. 1894?{]}}
\nopagebreak\mylabel{L03028v}
\rehead{ }\normalsize\beginnumbering\briefempfaengerindex{Salten, Felix@\textsc{Salten, Felix}!zzzSchnitzler, Arthur@\emph{von Arthur Schnitzler}!1894-04-141@{{[}14. 4. 1894?{]}}|(be}
\toendnotes[C]{\smallbreak\pagebreak[2]}
\correspDesc{Versand  durch Arthur Schnitzler am [14. 4. 1894?] in Wien
\newline{}Erhalt  durch Felix Salten am [14. 4. 1894?] in Wien}\toendnotes[C]{\smallbreak}
\Standort{Wienbibliothek im Rathaus, ZPH 1681, 2.1.516.}
\physDesc{Briefkarte, 191 Zeichen (Briefkarte mit Trauerrand)
\newline{}Handschrift: Bleistift, deutsche Kurrent
\newline{}Ordnung: mit Bleistift von unbekannter Hand nummeriert: »6« }\toendnotes[C]{\smallbreak}
\pstart
           \noindent{}{\pb}Lieber Freund,{ }\label{K_L03028-1v}\edtext{\textsc{Francillon\pwindex{\textcolor{red}{\textsuperscript{XXXX indx1}}!Francillon. Schauspiel in 3 Aufzügen@\strich\emph{Francillon. Schauspiel in 3 Aufzügen}|pw}} iſt \uline{abgeſagt}}{\lemma{\textnormal{\emph{Francillon ist abgesagt}}}\Cendnote{\textnormal{Für den 15. 4. 1894 war am \emph{Deutschen Volkstheater}\orgindex{Volkstheater@Volkstheater|pwk}{ }\emph{Francillon}\pwindex{\textcolor{red}{\textsuperscript{XXXX indx1}}!Francillon. Schauspiel in 3 Aufzügen@\strich\emph{Francillon. Schauspiel in 3 Aufzügen}|pwk} mit Adele Sandrock\pwindex{Sandrock, Adele 19.\,8.\,1863 Rotterdam – 30.\,8.\,1937 Berlin@\textsc{Sandrock, Adele} (19.\,8.\,1863 Rotterdam – 30.\,8.\,1937 Berlin), \emph{Schauspielerin}|pwk} als »Francine\pwindex{\textcolor{red}{\textsuperscript{XXXX indx1}}!Francillon. Schauspiel in 3 Aufzügen@\strich\emph{Francillon. Schauspiel in 3 Aufzügen}|pwkv}« angesetzt, musste aber wegen Erkrankung von
                     Bertha Hausner\pwindex{Hausner, Bertha 21.\,3.\,1869 Olomouc – 11.\,3.\,1932 Berlin@\textsc{Hausner, Bertha} (21.\,3.\,1869 Olomouc – 11.\,3.\,1932 Berlin), \emph{Schauspielerin}|pwk} (»Anette\pwindex{\textcolor{red}{\textsuperscript{XXXX indx1}}!Francillon. Schauspiel in 3 Aufzügen@\strich\emph{Francillon. Schauspiel in 3 Aufzügen}|pwkv}«) kurzfristig abgesagt werden.
                     Schnitzler verbrachte den Abend bei Sandrock\pwindex{Sandrock, Adele 19.\,8.\,1863 Rotterdam – 30.\,8.\,1937 Berlin@\textsc{Sandrock, Adele} (19.\,8.\,1863 Rotterdam – 30.\,8.\,1937 Berlin), \emph{Schauspielerin}|pwk}, jedoch ohne Salten\pwindex{Salten, Felix 6.\,9.\,1869 Budapest – 8.\,10.\,1945 Zürich@\textsc{Salten, Felix} (6.\,9.\,1869 Budapest – 8.\,10.\,1945 Zürich), \emph{Schriftsteller, Journalist, Chefredakteur}|pwk}. Das erlaubt eine mögliche Datierung des Korrespondenzstücks.}}}\label{K_L03028-1}. – Wir gehen um 9 zu Frl. S.\pwindex{Sandrock, Adele 19.\,8.\,1863 Rotterdam – 30.\,8.\,1937 Berlin@\textsc{Sandrock, Adele} (19.\,8.\,1863 Rotterdam – 30.\,8.\,1937 Berlin), \emph{Schauspielerin}|pw} – Ich hole Sie um ½ 9{ }\textsc{Café Centra{[}l{]}\oindex{Wien@\textbf{Wien}!I., Innere Stadt@\textbf{I., Innere Stadt}!Café Central@\textbf{Café Central}, \emph{Kaffeehaus}|pw}} ab. – Herzlichen Gruß. \spacefill\mbox{Arth}\pend
           
\pstart
           \noindent{}{\pb}Vielleicht{ }ſchaun Sie gleich nach Tiſch
                  auf 5 Minuten zu mir herüber?\pend
           \selectlanguage{ngerman}\endnumbering\briefempfaengerindex{Salten, Felix@\textsc{Salten, Felix}!zzzSchnitzler, Arthur@\emph{von Arthur Schnitzler}!1894-04-141@{{[}14. 4. 1894?{]}}|)be}\mylabel{L03028h}  \newcommand{\dateiname}{L03028}\newcommand{\titel}{Arthur Schnitzler an Felix Salten, [14. 4. 1894?]}\newcommand{\editorInnen}{Martin Anton Müller und Laura Untner}%% latex-leseansicht-abspann.tex
%% Abspann für die Leseansicht.
%% Der Schalter \ifkorrekturansicht ist bereits durch den Vorspann gesetzt.

%% latex-abspann.tex
%% Gemeinsamer Abspann für Korrekturansicht und Leseansicht.
%% Setzt den Schalter \ifkorrekturansicht voraus (gesetzt in den
%% einbindenden Dateien latex-korrekturansicht-abspann.tex bzw.
%% latex-leseansicht-abspann.tex).
%% ---------------------------------------------------------------

\normalsize

% Das esempio-Environment wird nur in der Leseansicht benötigt
\ifkorrekturansicht\else
\newenvironment{esempio}[3]%
{
    \vspace{1.5ex}
    \rlap{\underline{#1}}
    \par
    \setlength{\parindent}{0cm}
    \nopagebreak
    \leftskip=#2cm
    \rightskip=#3cm
}
{
    \par
}
\fi

\doendnotes{C}
\bigskip
\vfill

\clearpage

\footnotesize

\ifkorrekturansicht
  \lohead{\textsc{register}}
\fi

% theindex-Environment neu definieren ohne reledmac
\makeatletter
\renewenvironment{theindex}{%
  \ifkorrekturansicht
    \section*{\indexname}%
  \else
    \subsubsection*{Index der erwähnten Entitäten}%
  \fi
  \setlength{\parindent}{0pt}%
  \setlength{\parskip}{0pt plus 0.3pt}%
  \let\item\@idxitem
}{%
  \ifkorrekturansicht\clearpage\fi
}
\makeatother

\IfFileExists{\jobname-pw.ind}{\input{\jobname-pw.ind}}{}

% Quellenangabe nur in der Leseansicht
\ifkorrekturansicht\else
% Fallback-Definitionen, falls die .tex-Datei \titel etc. nicht gesetzt hat
\providecommand{\titel}{}
\providecommand{\editorInnen}{}
\providecommand{\dateiname}{\jobname}

\vspace{3cm}

\vfill

\footnotesize
\textsc{Quelle}: \titel. Herausgegeben von {\editorInnen}. In: \emph{Arthur Schnitzler: Briefwechsel mit Autorinnen und Autoren}.
 Digitale Edition, https://schnitzler-briefe.acdh.oeaw.ac.at/{\dateiname}.html (Stand \today)
\fi

\end{document}


