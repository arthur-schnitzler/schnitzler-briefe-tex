%% latex-korrekturansicht-vorspann.tex
%% Vorspann für die Korrekturansicht.
%% Lädt die gemeinsame Datei latex-vorspann.tex mit gesetztem Schalter.

\newif\ifkorrekturansicht
\korrekturansichttrue

\input{../tex-inputs/latex-vorspann}


\section[Arthur Schnitzler an Felix Salten, {[}14. 4. 1894?{]}]{L03028 Arthur Schnitzler an Felix Salten, {[}14. 4. 1894?{]}}
\nopagebreak\mylabel{L03028v}
\rehead{ }\normalsize\beginnumbering\briefempfaengerindex{Salten, Felix@\textsc{Salten, Felix}!zzzSchnitzler, Arthur@\emph{von Arthur Schnitzler}!1894-04-141@{{[}14. 4. 1894?{]}}|(be}
\toendnotes[C]{\smallbreak\pagebreak[2]}\Standort{Wienbibliothek im Rathaus, ZPH 1681, 2.1.516.}
\physDesc{Briefkarte, 191 Zeichen (Briefkarte mit Trauerrand)
\newline{}Handschrift: Bleistift, deutsche Kurrent
\newline{}Ordnung: mit Bleistift von unbekannter Hand nummeriert: »6« }\toendnotes[C]{\smallbreak}
\pstart
           \noindent{}{\pb}Lieber Freund,{ }\label{K_L03028-1v}\edtext{\textsc{Francillon\pwindex{Francillon. Schauspiel in 3 Aufzuegen@\emph{Francillon. Schauspiel in 3 Aufzügen}|pw}} iſt \uline{abgeſagt}}{\lemma{\textnormal{\emph{Francillon iſt abgeſagt}}}\Cendnote{\textnormal{Für den 15. 4. 1894 war am \emph{Deutschen Volkstheater}\orgindex{Volkstheater@Volkstheater|pwk}{ }\emph{Francillon}\pwindex{Francillon. Schauspiel in 3 Aufzuegen@\emph{Francillon. Schauspiel in 3 Aufzügen}|pwk} mit Adele Sandrock\pwindex{Sandrock, Adele 1863-08-19 – 1937-08-30@\textsc{Sandrock, Adele} (1863-08-19 – 1937-08-30), \emph{Schauspieler/Schauspielerin}|pwk} als »Francine\pwindex{Francillon. Schauspiel in 3 Aufzuegen@\emph{Francillon. Schauspiel in 3 Aufzügen}|pwkv}« angesetzt, musste aber wegen Erkrankung von
                     Bertha Hausner\pwindex{Hausner, Bertha 1869-03-21 – 11.03.1932@\textsc{Hausner, Bertha} (1869-03-21 – 11.03.1932), \emph{Schauspieler/Schauspielerin}|pwk} (»Anette\pwindex{Francillon. Schauspiel in 3 Aufzuegen@\emph{Francillon. Schauspiel in 3 Aufzügen}|pwkv}«) kurzfristig abgesagt werden.
                     Schnitzler verbrachte den Abend bei Sandrock\pwindex{Sandrock, Adele 1863-08-19 – 1937-08-30@\textsc{Sandrock, Adele} (1863-08-19 – 1937-08-30), \emph{Schauspieler/Schauspielerin}|pwk}, jedoch ohne Salten\pwindex{Salten, Felix 06.09.1869 – 08.10.1945@\textsc{Salten, Felix} (06.09.1869 – 08.10.1945), \emph{Schriftsteller/Schriftstellerin, Journalist/Journalistin, Chefredakteur/Chefredakteurin}|pwk}. Das erlaubt eine mögliche Datierung des Korrespondenzstücks.}}}\label{K_L03028-1}. – Wir gehen um 9 zu Frl. S.\pwindex{Sandrock, Adele 1863-08-19 – 1937-08-30@\textsc{Sandrock, Adele} (1863-08-19 – 1937-08-30), \emph{Schauspieler/Schauspielerin}|pw} – Ich hole Sie um ½ 9{ }\textsc{Café Centra{[}l{]}\oindex{Cafe Central@\textbf{Café Central}, \emph{Kaffeehaus (K.KAF)}|pw}} ab. – Herzlichen Gruß. \spacefill\mbox{Arth}\pend
           
\pstart
           \noindent{}{\pb}Vielleicht ſchaun Sie gleich nach Tiſch
                  auf 5 Minuten zu mir herüber?\pend
           \selectlanguage{ngerman}\endnumbering\briefempfaengerindex{Salten, Felix@\textsc{Salten, Felix}!zzzSchnitzler, Arthur@\emph{von Arthur Schnitzler}!1894-04-141@{{[}14. 4. 1894?{]}}|)be}\mylabel{L03028h}  \normalsize

\doendnotes{C}
\bigskip
\vfill

\clearpage

\footnotesize

\lohead{\textsc{register}}

% Definiere theindex-Environment komplett neu ohne reledmac
\makeatletter
\renewenvironment{theindex}{%
  \section*{\indexname}%
  \setlength{\parindent}{0pt}%
  \setlength{\parskip}{0pt plus 0.3pt}%
  \let\item\@idxitem
}{%
  \clearpage
}
\makeatother

\IfFileExists{\jobname-pw.ind}{\input{\jobname-pw.ind}}{}

\end{document}

      