%% latex-leseansicht-vorspann.tex
%% Vorspann für die Leseansicht.
%% Lädt die gemeinsame Datei latex-vorspann.tex mit nicht gesetztem Schalter.

\newif\ifkorrekturansicht
\korrekturansichtfalse

\input{../tex-inputs/latex-vorspann}


         
         \newcommand{\erwaehntePersonen}{Personen: Gustav Seidler, Alfred von Winterstein}
         \newcommand{\erwaehnteInstitutionen}{}
         \newcommand{\erwaehnteOrte}{Orte: Rodaun, Wien}
         \newcommand{\erwaehnteWerke}{Werke: Der Weg ins Freie. Roman, [Gedichte]}
               \section[Hugo von Hofmannsthal an Arthur Schnitzler, {[}23. 11. 1908{]}]{ Hugo von Hofmannsthal an Arthur Schnitzler, {[}23. 11. 1908{]}}\nopagebreak\mylabel{v}\rehead{ }\begin{ledgroupsized}[t]{13cm}\normalsize\beginnumbering \toendnotes[C]{\smallbreak\pagebreak[2]} \Standort{CUL, Schnitzler, B 43.}
\physDesc{Brief, 1 Blatt, 4 Seiten
\newline{}Handschrift: schwarze Tinte, deutsche Kurrent
\newline{}Schnitzler: mit Bleistift datiert: »Früh 909« und beschriftet: »Hugo« \newline{}Ordnung: 1) mit Bleistift von unbekannter Hand nummeriert: »\strikeout{298}«  2) mit Bleistift von unbekannter Hand nummeriert: »306«}\buchAbdrucke{\weitereDrucke{Hugo von Hofmannsthal, Arthur Schnitzler: \emph{Briefwechsel}. Hg. Therese Nickl und Heinrich Schnitzler. Frankfurt am Main: \emph{S. Fischer} 1964, S. 242–243.} }\toendnotes[C]{\smallbreak}\pstart
           \raggedleft{}{\pb}R.\oindex{Rodaun@\textbf{Rodaun}|pw}{\\}Montag.\pend
           \pstart{}mein lieber Arthur\pend\pstart
           ſo nett und gemütlich es \label{K_L01808_1v}\edtext{neulich}{\lemma{\textnormal{\emph{neulich}}}\Cendnote{\textnormal{am
                     26. 10. 1908 und am 15. 11. 1908}}}\label{K_L01808_1h} abends bei Euch war, ſo ſehr wünſche ich mir nach der ungewohnten
               Zufälligkeit, daſs wir \introOben{}2mal\introOben{} Fremde bei Euch trafen, wieder
               die Freude, Sie allein zu ſehen.\hspace*{1.5em}Es gibt Zeiten, in
               welchen man beſonders deutlich fühlt, welche Menſchen {\pb}auf der Welt man ſehr gern hat,
               und für mich ist dieſe jetzige Zeit eine ſolche.\pend
           \pstart
           Vielleicht, da Ihr viel vorhabt, telegrafiert ihr einmal, 1–2 Tage voraus, einen
               Abend wo wir kommen dürfen.\pend
           \pstart
           Die Gedichte\pwindex{Winterstein, Alfred von 25.09.1885 – 28.04.1958@\textsc{Winterstein, Alfred von} (25.09.1885 – 28.04.1958), \emph{Schriftsteller, Psychoanalytiker, Beamter}!Gedichte]None@\strich\emph{[Gedichte]} {[}None{]}|pwv} von Winterſtein\pwindex{Winterstein, Alfred von 25.09.1885 – 28.04.1958@\textsc{Winterstein, Alfred von} (25.09.1885 – 28.04.1958), \emph{Schriftsteller, Psychoanalytiker, Beamter}|pw} gefallen mir \uline{sehr}
               gut. Was würde ihm wünſchens{\pb}wert
               ſein daſs man dafür thäte? \pend
           \pstart
           Ich ſage mir manchmal, daſs vermutlich die Anfänge dieſer Erkrankung meiner Nerven
               weit zurück liegen und daſs meine \label{K_L01808_2v}\edtext{Verſtörtheit}{\lemma{\textnormal{\emph{Verſtörtheit}}}\Cendnote{\textnormal{siehe Hugo von Hofmannsthal an Arthur Schnitzler, 24. 7. [1908], 
                  vgl. A. S.: \emph{Tagebuch}, 24. 11. 1908}}}\label{K_L01808_2h}
               über gewiſſe Dinge in Ihrem Roman\pwindex{Schnitzler, Arthur 15.05.1862 – 21.10.1931@\textsc{Schnitzler, Arthur} (15.05.1862 – 21.10.1931), \emph{Schriftsteller, Mediziner}!Weg ins Freie. Roman1.1.1908 – 1.6.1908@\strich\emph{Der Weg ins Freie. Roman} {[}1.1.1908 – 1.6.1908{]}|pwv}
               (menſchliche viel mehr als künſtleriſche, aber \uline{nicht}
               im Bereich des Judenproblems) {\pb}vielleicht ſchon nichts normales mehr war.\pend
           \pstart
           Auf Wiederſehen, mein lieber Arthur.\pend
           \pstart
           Ihr alter{\\[\baselineskip]}\spacefill\mbox{Hugo.}\pend
           \leftskip=0em{}\pstart
           \noindent{}Dem Profeſſor Seidler\pwindex{Seidler, Gustav 03.05.1858 – 27.03.1933@\textsc{Seidler, Gustav} (03.05.1858 – 27.03.1933), \emph{Rechtswissenschaftler}|pw} hab ich gedankt.\pend
           
         
         \endnumbering\mylabel{h}\end{ledgroupsized}  \newcommand{\dateiname}{L01808}\newcommand{\titel}{Hugo von Hofmannsthal an Arthur Schnitzler, [23. 11. 1908]}\newcommand{\editorInnen}{Martin Anton Müller und Gerd-Hermann Susen}%% latex-leseansicht-abspann.tex
%% Abspann für die Leseansicht.
%% Der Schalter \ifkorrekturansicht ist bereits durch den Vorspann gesetzt.

%% latex-abspann.tex
%% Gemeinsamer Abspann für Korrekturansicht und Leseansicht.
%% Setzt den Schalter \ifkorrekturansicht voraus (gesetzt in den
%% einbindenden Dateien latex-korrekturansicht-abspann.tex bzw.
%% latex-leseansicht-abspann.tex).
%% ---------------------------------------------------------------

\normalsize

% Das esempio-Environment wird nur in der Leseansicht benötigt
\ifkorrekturansicht\else
\newenvironment{esempio}[3]%
{
    \vspace{1.5ex}
    \rlap{\underline{#1}}
    \par
    \setlength{\parindent}{0cm}
    \nopagebreak
    \leftskip=#2cm
    \rightskip=#3cm
}
{
    \par
}
\fi

\doendnotes{C}
\bigskip
\vfill

\clearpage

\footnotesize

\ifkorrekturansicht
  \lohead{\textsc{register}}
\fi

% theindex-Environment neu definieren ohne reledmac
\makeatletter
\renewenvironment{theindex}{%
  \ifkorrekturansicht
    \section*{\indexname}%
  \else
    \subsubsection*{Index der erwähnten Entitäten}%
  \fi
  \setlength{\parindent}{0pt}%
  \setlength{\parskip}{0pt plus 0.3pt}%
  \let\item\@idxitem
}{%
  \ifkorrekturansicht\clearpage\fi
}
\makeatother

\IfFileExists{\jobname-pw.ind}{\input{\jobname-pw.ind}}{}

% Quellenangabe nur in der Leseansicht
\ifkorrekturansicht\else
% Fallback-Definitionen, falls die .tex-Datei \titel etc. nicht gesetzt hat
\providecommand{\titel}{}
\providecommand{\editorInnen}{}
\providecommand{\dateiname}{\jobname}

\vspace{3cm}

\vfill

\footnotesize
\textsc{Quelle}: \titel. Herausgegeben von {\editorInnen}. In: \emph{Arthur Schnitzler: Briefwechsel mit Autorinnen und Autoren}.
 Digitale Edition, https://schnitzler-briefe.acdh.oeaw.ac.at/{\dateiname}.html (Stand \today)
\fi

\end{document}


      