%% latex-korrekturansicht-vorspann.tex
%% Vorspann für die Korrekturansicht.
%% Lädt die gemeinsame Datei latex-vorspann.tex mit gesetztem Schalter.

\newif\ifkorrekturansicht
\korrekturansichttrue

\input{../tex-inputs/latex-vorspann}


\section[ Paul Goldmann an Arthur Schnitzler, 16. 11. 1908]{L03466 Paul Goldmann an Arthur Schnitzler, 16. 11. 1908}
\nopagebreak\mylabel{L03466v}
\rehead{ }\normalsize\beginnumbering\briefempfaengerindex{Schnitzler, Arthur@\textsc{Schnitzler, Arthur}!zzzGoldmann, Paul@\emph{von Paul Goldmann}!1908-11-161@{16. 11. 1908}|(be}
\toendnotes[C]{\smallbreak\pagebreak[2]}\Standort{DLA, A:Schnitzler, HS.NZ85.1.3175.}
\physDesc{Brief, 1 Blatt, 1 Seite, 181 Zeichen
\newline{}Handschrift: blaue Tinte, deutsche Kurrent}\toendnotes[C]{\smallbreak}
\pstart
           {\pb}16. 11. 08.\pend
           
\pstart\center{}Lieber Freund,\pend\vspace{0.5em}
\pstart
           Im Namen meiner Frau\pwindex{Goldmann, Eva Marie 27.10.1877 – 02.11.1937@\textsc{Goldmann, Eva Marie} (27.10.1877 – 02.11.1937)|pwv} u. in
               meinem eigenen ſage ich Dir u. Deiner Frau\pwindex{Schnitzler, Olga 17.01.1882 – 13.01.1970@\textsc{Schnitzler, Olga} (17.01.1882 – 13.01.1970), \emph{Schauspieler/Schauspielerin, Sänger/Sängerin}|pwv} unſeren herzlichen Dank für den Ausdruck Eurer \label{K_L03466-1v}\edtext{Teilnahme}{\lemma{\textnormal{\emph{Teilnahme}}}\Cendnote{\textnormal{Goldmanns\pwindex{Goldmann, Paul 31.01.1865 – 25.09.1935@\textsc{Goldmann, Paul} (31.01.1865 – 25.09.1935), \emph{Schriftsteller/Schriftstellerin, Journalist/Journalistin}|pwk} Schwiegervater, Alfred Fränkel\pwindex{Fraenkel, Alfred 1843-01-06 – 08.11.1908@\textsc{Fränkel, Alfred} (1843-01-06 – 08.11.1908), \emph{Fabrikant/Fabrikantin}|pwk}, war am 8. 11. 1908 in Berlin\oindex{Berlin@\textbf{Berlin}, \emph{P.PPLC}|pwk} verstorben.
                     Schnitzler hatte ihn persönlich
                  gekannt.}}}\label{K_L03466-1}.\pend
           
\pstart
           Mit vielen Grüßen {\\[\baselineskip]}Dein {\\[\baselineskip]}\spacefill\mbox{Paul Goldmann}\pend
           \leftskip=0em{}\selectlanguage{ngerman}\endnumbering\briefempfaengerindex{Schnitzler, Arthur@\textsc{Schnitzler, Arthur}!zzzGoldmann, Paul@\emph{von Paul Goldmann}!1908-11-161@{16. 11. 1908}|)be}\mylabel{L03466h}  \normalsize

\doendnotes{C}
\bigskip
\vfill

\clearpage

\footnotesize

\lohead{\textsc{register}}

% Definiere theindex-Environment komplett neu ohne reledmac
\makeatletter
\renewenvironment{theindex}{%
  \section*{\indexname}%
  \setlength{\parindent}{0pt}%
  \setlength{\parskip}{0pt plus 0.3pt}%
  \let\item\@idxitem
}{%
  \clearpage
}
\makeatother

\IfFileExists{\jobname-pw.ind}{\input{\jobname-pw.ind}}{}

\end{document}

      