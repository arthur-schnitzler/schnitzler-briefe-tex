%% latex-leseansicht-vorspann.tex
%% Vorspann für die Leseansicht.
%% Lädt die gemeinsame Datei latex-vorspann.tex mit nicht gesetztem Schalter.

\newif\ifkorrekturansicht
\korrekturansichtfalse

\input{../tex-inputs/latex-vorspann}


\section[ Felix Salten an Arthur Schnitzler, 9. 12. 1903]{L03354 Felix Salten an Arthur Schnitzler,  9. 12. 1903}
\nopagebreak\mylabel{L03354v}
\rehead{ }\normalsize\beginnumbering\briefempfaengerindex{Schnitzler, Arthur@\textsc{Schnitzler, Arthur}!zzzSalten, Felix@\emph{von Felix Salten}!1903-12-091@{9. 12. 1903}|(be}
\toendnotes[C]{\smallbreak\pagebreak[2]}
\correspDesc{Versand  durch Felix Salten am 9. 12. 1903 in Wien
\newline{}Erhalt  durch Arthur Schnitzler im Zeitraum [9. 12. 1903
                  – 12. 12. 1903?] in Wien}\toendnotes[C]{\smallbreak}
\Standort{CUL, Schnitzler, B 89, A 2.}
\physDesc{Brief, 1 Blatt, 1 Seite, 514 Zeichen
\newline{}Schreibmaschine
\newline{}Handschrift: schwarze Tinte, lateinische Kurrent (\noindent{}Korrektur und Unterschrift)
\newline{}Ordnung: mit Bleistift von unbekannter Hand nummeriert: »180« }\toendnotes[C]{\smallbreak}
\pstart
           {\pb}\textcolor{gray}{\textbf{DIE}}\pend
           
\pstart
           \textcolor{gray}{\textbf{ZEIT\orgindex{Zeit@Die Zeit|pw}}}\hfill \textcolor{gray}{\textbf{\emph{WIEN,\oindex{Wien@\textbf{Wien}, \emph{Verwaltungsgebiet}|pw}}}}{ }9. Dezember 1903.\pend
           
\pstart
           \textcolor{gray}{\textbf{WIEN\oindex{Wien@\textbf{Wien}, \emph{Verwaltungsgebiet}|pw}ER TAGESZEITUNG}}\hfill \textcolor{gray}{\textbf{\emph{I. Wipplingerstrasse 38\oindex{Wien@\textbf{Wien}!I., Innere Stadt@\textbf{I., Innere Stadt}!Wipplingerstraße@\textbf{Wipplingerstraße}, \emph{Straße}|pw}.}}}\pend
           
\pstart
           \textcolor{gray}{\textbf{Herausgeber:}}\pend
           
\pstart
           \textcolor{gray}{\textbf{Prof. Dr. I. Singer\pwindex{Singer, Isidor 16.\,1.\,1857 Budapest – 8.\,12.\,1927 Wien@\textsc{Singer, Isidor} (16.\,1.\,1857 Budapest – 8.\,12.\,1927 Wien), \emph{Journalist, Herausgeber, Soziologe}|pw}}}\pend
           
\pstart
           \textcolor{gray}{\textbf{Dr. Heinrich Kanner\pwindex{Kanner, Heinrich 9.\,11.\,1864 Galați – 15.\,2.\,1930 Wien@\textsc{Kanner, Heinrich} (9.\,11.\,1864 Galați – 15.\,2.\,1930 Wien), \emph{Herausgeber, Publizist}|pw}}}\pend
           
\pstart
           \textcolor{gray}{\textbf{\textbf{Redaction}}}\pend
           
\pstart
           \textcolor{gray}{\textbf{Telegramm-Adresse: \so{Zeit}\orgindex{Zeit@Die Zeit|pw}\so{,}{ }\so{Wien}\oindex{Wien@\textbf{Wien}, \emph{Verwaltungsgebiet}|pw}}}\pend
           
\pstart
           \textcolor{gray}{\textbf{\textbf{Telephone:}}}\pend
           
\pstart
           \textcolor{gray}{\textbf{Interurbanes Telephon Nr. 15.988}}\pend
           
\pstart
           \textcolor{gray}{\textbf{= Telephone Nr. 17.040, 17.041 =}}\pend
           
\pstart
           \textcolor{gray}{\textbf{Depeschensaal Nr. 4548.}}\pend
           
\pstart
           Sa/H\pend
           
\pstart\center{}Lieber Freund!\pend\vspace{0.5em}
\pstart
           Da unsere Weihnachtsnummer\pwindex{Zeit@\emph{Die Zeit}|pwv}
               jetzt fertig gestellt werden muss, frage ich Sie, ob Sie \label{K_L03354-1v}\edtext{etwas für mich haben}{\lemma{\textnormal{\emph{etwas für mich haben}}}\Cendnote{\textnormal{Von Schnitzler erschien
                  nichts in der Weihnachtsbeilage der \emph{Zeit}\pwindex{Zeit@\emph{Die Zeit}|pwk}.}}}\label{K_L03354-1}. Es muss nichts Grosses sein aber aus mancherlei Gründen wäre
               es mir lieb, wenn Sie mir irgend etwas schicken können. Die \label{K_L03354-2v}\edtext{Schlenther\pwindex{Schlenther, Paul 20.\,8.\,1854 Chernyakhovsk – 30.\,4.\,1916 Berlin@\textsc{Schlenther, Paul} (20.\,8.\,1854 Chernyakhovsk – 30.\,4.\,1916 Berlin), \emph{Schriftsteller, Kritiker, Theaterleiter}|pw}-Briefe}{\lemma{\textnormal{\emph{Schlenther-Briefe}}}\Cendnote{\textnormal{Eventuell handelte es sich noch um die Briefe, die Schlenther\pwindex{Schlenther, Paul 20.\,8.\,1854 Chernyakhovsk – 30.\,4.\,1916 Berlin@\textsc{Schlenther, Paul} (20.\,8.\,1854 Chernyakhovsk – 30.\,4.\,1916 Berlin), \emph{Schriftsteller, Kritiker, Theaterleiter}|pwk}{ }Schnitzler zur geplanten Annahme und
                  späteren Ablehnung von \emph{Der Schleier der
                     Beatrice}\pwindex{Schnitzler, Arthur 15.\,5.\,1862 Wien – 21.\,10.\,1931 ebd.@\textsc{Schnitzler, Arthur} (15.\,5.\,1862 Wien – 21.\,10.\,1931 ebd.), \emph{Schriftsteller, Mediziner}!Schleier der Beatrice. Schauspiel in fünf Akten@\strich\emph{Der Schleier der Beatrice. Schauspiel in fünf Akten}|pwk} geschickt hatte. Salten\pwindex{Salten, Felix 6.\,9.\,1869 Budapest – 8.\,10.\,1945 Zürich@\textsc{Salten, Felix} (6.\,9.\,1869 Budapest – 8.\,10.\,1945 Zürich), \emph{Schriftsteller, Journalist, Chefredakteur}|pwk}
                  hatte damals den Protest organisiert, der zur \emph{Erklärung}\pwindex{\textcolor{red}{\textsuperscript{XXXX indx1}}!Erklärung [Schleier der Beatrice]@\strich\emph{Erklärung [Schleier der Beatrice]}|pwk}\pwindex{Salten, Felix 6.\,9.\,1869 Budapest – 8.\,10.\,1945 Zürich@\textsc{Salten, Felix} (6.\,9.\,1869 Budapest – 8.\,10.\,1945 Zürich), \emph{Schriftsteller, Journalist, Chefredakteur}!Erklärung [Schleier der Beatrice]@\strich\emph{Erklärung [Schleier der Beatrice]}|pwk}\pwindex{\textcolor{red}{\textsuperscript{XXXX indx1}}!Erklärung [Schleier der Beatrice]@\strich\emph{Erklärung [Schleier der Beatrice]}|pwk}\pwindex{\textcolor{red}{\textsuperscript{XXXX indx1}}!Erklärung [Schleier der Beatrice]@\strich\emph{Erklärung [Schleier der Beatrice]}|pwk}\pwindex{\textcolor{red}{\textsuperscript{XXXX indx1}}!Erklärung [Schleier der Beatrice]@\strich\emph{Erklärung [Schleier der Beatrice]}|pwk}\pwindex{\textcolor{red}{\textsuperscript{XXXX indx1}}!Erklärung [Schleier der Beatrice]@\strich\emph{Erklärung [Schleier der Beatrice]}|pwk} von sechs Autoren in den Tageszeitungen geführt hatte. Siehe XXXX Auszeichnungsfehler: Dokument L01073 nicht gefunden.}}}\label{K_L03354-2} habe ich Ihnen
               gleich am Montag rekommandiert zurückgeschickt.
               Hoffentlich bin ich in der nächsten Woche mit dem \label{K_L03354-3v}\edtext{Preisausschreiben}{\lemma{\textnormal{\emph{Preisausschreiben}}}\Cendnote{\textnormal{Vgl. XXXX Auszeichnungsfehler: Dokument L03344 nicht gefunden.
               }}}\label{K_L03354-3} so weit fertig, um \label{K_L03354-4v}\edtext{einmal
                  nachmittags zu Ihnen kommen}{\lemma{\textnormal{\emph{einmal … kommen}}}\Cendnote{\textnormal{Vgl. A. S.: \emph{Tagebuch}, 16. 12. 1903.
               }}}\label{K_L03354-4} zu können.\pend
           
\pstart
           Herzlichst {\\[\baselineskip]}Ihr {\\[\baselineskip]}\spacefill\mbox{Salten}\pend
           \leftskip=0em{}
\pstart
           \noindent{}Herrn Dr. Arthur Schnitzler\pend
           
\pstart
           \so{Wien.}\oindex{Wien@\textbf{Wien}, \emph{Verwaltungsgebiet}|pw}\pend
           
\pstart
           \textcolor{gray}{\textbf{\emph{Alle für »Die Zeit\orgindex{Zeit@Die Zeit|pw}«
                        bestimmten Zuschriften und Sendungen sind an die Redaction »Die Zeit\orgindex{Zeit@Die Zeit|pw}« und \textbf{nicht} an die
                        Person eines der Herausgeber oder Mitarbeiter zu richten.}}}\pend
           \selectlanguage{ngerman}\endnumbering\briefempfaengerindex{Schnitzler, Arthur@\textsc{Schnitzler, Arthur}!zzzSalten, Felix@\emph{von Felix Salten}!1903-12-091@{9. 12. 1903}|)be}\mylabel{L03354h}  \newcommand{\dateiname}{L03354}\newcommand{\titel}{Felix Salten an Arthur Schnitzler, 9. 12. 1903}\newcommand{\editorInnen}{Martin Anton Müller und Laura Untner}%% latex-leseansicht-abspann.tex
%% Abspann für die Leseansicht.
%% Der Schalter \ifkorrekturansicht ist bereits durch den Vorspann gesetzt.

%% latex-abspann.tex
%% Gemeinsamer Abspann für Korrekturansicht und Leseansicht.
%% Setzt den Schalter \ifkorrekturansicht voraus (gesetzt in den
%% einbindenden Dateien latex-korrekturansicht-abspann.tex bzw.
%% latex-leseansicht-abspann.tex).
%% ---------------------------------------------------------------

\normalsize

% Das esempio-Environment wird nur in der Leseansicht benötigt
\ifkorrekturansicht\else
\newenvironment{esempio}[3]%
{
    \vspace{1.5ex}
    \rlap{\underline{#1}}
    \par
    \setlength{\parindent}{0cm}
    \nopagebreak
    \leftskip=#2cm
    \rightskip=#3cm
}
{
    \par
}
\fi

\doendnotes{C}
\bigskip
\vfill

\clearpage

\footnotesize

\ifkorrekturansicht
  \lohead{\textsc{register}}
\fi

% theindex-Environment neu definieren ohne reledmac
\makeatletter
\renewenvironment{theindex}{%
  \ifkorrekturansicht
    \section*{\indexname}%
  \else
    \subsubsection*{Index der erwähnten Entitäten}%
  \fi
  \setlength{\parindent}{0pt}%
  \setlength{\parskip}{0pt plus 0.3pt}%
  \let\item\@idxitem
}{%
  \ifkorrekturansicht\clearpage\fi
}
\makeatother

\IfFileExists{\jobname-pw.ind}{\input{\jobname-pw.ind}}{}

% Quellenangabe nur in der Leseansicht
\ifkorrekturansicht\else
% Fallback-Definitionen, falls die .tex-Datei \titel etc. nicht gesetzt hat
\providecommand{\titel}{}
\providecommand{\editorInnen}{}
\providecommand{\dateiname}{\jobname}

\vspace{3cm}

\vfill

\footnotesize
\textsc{Quelle}: \titel. Herausgegeben von {\editorInnen}. In: \emph{Arthur Schnitzler: Briefwechsel mit Autorinnen und Autoren}.
 Digitale Edition, https://schnitzler-briefe.acdh.oeaw.ac.at/{\dateiname}.html (Stand \today)
\fi

\end{document}


