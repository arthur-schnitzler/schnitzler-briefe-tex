%% latex-korrekturansicht-vorspann.tex
%% Vorspann für die Korrekturansicht.
%% Lädt die gemeinsame Datei latex-vorspann.tex mit gesetztem Schalter.

\newif\ifkorrekturansicht
\korrekturansichttrue

\input{../tex-inputs/latex-vorspann}


\section[ Felix Salten an Arthur Schnitzler, 9. 12. 1903]{L03354 Felix Salten an Arthur Schnitzler, 9. 12. 1903}
\nopagebreak\mylabel{L03354v}
\rehead{ }\normalsize\beginnumbering\briefempfaengerindex{Schnitzler, Arthur@\textsc{Schnitzler, Arthur}!zzzSalten, Felix@\emph{von Felix Salten}!1903-12-091@{9. 12. 1903}|(be}
\toendnotes[C]{\smallbreak\pagebreak[2]}\Standort{CUL, Schnitzler, B 89, A 2.}
\physDesc{Brief, 1 Blatt, 1 Seite, 514 Zeichen
\newline{}Schreibmaschine
\newline{}Handschrift: schwarze Tinte, lateinische Kurrent (\noindent{}Korrektur und Unterschrift)
\newline{}Ordnung: mit Bleistift von unbekannter Hand nummeriert: »180« }\toendnotes[C]{\smallbreak}
\pstart
           {\pb}\textcolor{gray}{\textbf{DIE}}\pend
           
\pstart
           \textcolor{gray}{\textbf{ZEIT\orgindex{Zeit@Die Zeit|pw}}}\hfill \textcolor{gray}{\textbf{\emph{WIEN,\oindex{Wien@\textbf{Wien}, \emph{A.ADM2}|pw}}}}{ }9. Dezember 1903.\pend
           
\pstart
           \textcolor{gray}{\textbf{WIEN\oindex{Wien@\textbf{Wien}, \emph{A.ADM2}|pw}ER TAGESZEITUNG}}\hfill \textcolor{gray}{\textbf{\emph{I. Wipplingerstrasse 38\oindex{Wipplingerstrasse@\textbf{Wipplingerstraße}, \emph{Straße (K.STR)}|pw}.}}}\pend
           
\pstart
           \textcolor{gray}{\textbf{Herausgeber:}}\pend
           
\pstart
           \textcolor{gray}{\textbf{Prof. Dr. I. Singer\pwindex{Singer, Isidor 16.01.1857 – 08.12.1927@\textsc{Singer, Isidor} (16.01.1857 – 08.12.1927), \emph{Journalist/Journalistin, Herausgeber/Herausgeberin, Soziologe/Soziologin}|pw}}}\pend
           
\pstart
           \textcolor{gray}{\textbf{Dr. Heinrich Kanner\pwindex{Kanner, Heinrich 09.11.1864 – 15.02.1930@\textsc{Kanner, Heinrich} (09.11.1864 – 15.02.1930), \emph{Herausgeber/Herausgeberin, Publizist/Publizistin}|pw}}}\pend
           
\pstart
           \textcolor{gray}{\textbf{\textbf{Redaction}}}\pend
           
\pstart
           \textcolor{gray}{\textbf{Telegramm-Adresse: \so{Zeit}\orgindex{Zeit@Die Zeit|pw}\so{,}{ }\so{Wien}\oindex{Wien@\textbf{Wien}, \emph{A.ADM2}|pw}}}\pend
           
\pstart
           \textcolor{gray}{\textbf{\textbf{Telephone:}}}\pend
           
\pstart
           \textcolor{gray}{\textbf{Interurbanes Telephon Nr. 15.988}}\pend
           
\pstart
           \textcolor{gray}{\textbf{= Telephone Nr. 17.040, 17.041 =}}\pend
           
\pstart
           \textcolor{gray}{\textbf{Depeschensaal Nr. 4548.}}\pend
           
\pstart
           Sa/H\pend
           
\pstart\center{}Lieber Freund!\pend\vspace{0.5em}
\pstart
           Da unsere Weihnachtsnummer\pwindex{Zeit@\emph{Die Zeit}|pwv}
               jetzt fertig gestellt werden muss, frage ich Sie, ob Sie \label{K_L03354-1v}\edtext{etwas für mich haben}{\lemma{\textnormal{\emph{etwas für mich haben}}}\Cendnote{\textnormal{Von Schnitzler erschien
                  nichts in der Weihnachtsbeilage der \emph{Zeit}\pwindex{Zeit@\emph{Die Zeit}|pwk}.}}}\label{K_L03354-1}. Es muss nichts Grosses sein aber aus mancherlei Gründen wäre
               es mir lieb, wenn Sie mir irgend etwas schicken können. Die \label{K_L03354-2v}\edtext{Schlenther\pwindex{Schlenther, Paul 20.08.1854 – 30.04.1916@\textsc{Schlenther, Paul} (20.08.1854 – 30.04.1916), \emph{Schriftsteller/Schriftstellerin, Kritiker/Kritikerin, Theaterleiter/Theaterleiterin}|pw}-Briefe}{\lemma{\textnormal{\emph{Schlenther-Briefe}}}\Cendnote{\textnormal{Eventuell handelte es sich noch um die Briefe, die Schlenther\pwindex{Schlenther, Paul 20.08.1854 – 30.04.1916@\textsc{Schlenther, Paul} (20.08.1854 – 30.04.1916), \emph{Schriftsteller/Schriftstellerin, Kritiker/Kritikerin, Theaterleiter/Theaterleiterin}|pwk}{ }Schnitzler zur geplanten Annahme und
                  späteren Ablehnung von \emph{Der Schleier der
                     Beatrice}\pwindex{Schleier der Beatrice. Schauspiel in fuenf Akten@\emph{Der Schleier der Beatrice. Schauspiel in fünf Akten}|pwk} geschickt hatte. Salten\pwindex{Salten, Felix 06.09.1869 – 08.10.1945@\textsc{Salten, Felix} (06.09.1869 – 08.10.1945), \emph{Schriftsteller/Schriftstellerin, Journalist/Journalistin, Chefredakteur/Chefredakteurin}|pwk}
                  hatte damals den Protest organisiert, der zur \emph{Erklärung}\pwindex{Erklaerung [Schleier der Beatrice]@\emph{Erklärung [Schleier der Beatrice]}|pwk} von sechs Autoren in den Tageszeitungen geführt hatte. Siehe Richard Beer-Hofmann an Arthur Schnitzler, 14. 9. 1900.}}}\label{K_L03354-2} habe ich Ihnen
               gleich am Montag rekommandiert zurückgeschickt.
               Hoffentlich bin ich in der nächsten Woche mit dem \label{K_L03354-3v}\edtext{Preisausschreiben}{\lemma{\textnormal{\emph{Preisausschreiben}}}\Cendnote{\textnormal{Vgl. Felix Salten an Arthur Schnitzler, 19. 9. [1903].
               }}}\label{K_L03354-3} so weit fertig, um \label{K_L03354-4v}\edtext{einmal
                  nachmittags zu Ihnen kommen}{\lemma{\textnormal{\emph{einmal … kommen}}}\Cendnote{\textnormal{Vgl. A. S.: \emph{Tagebuch}, 16. 12. 1903.
               }}}\label{K_L03354-4} zu können.\pend
           
\pstart
           Herzlichst {\\[\baselineskip]}Ihr {\\[\baselineskip]}\spacefill\mbox{Salten}\pend
           \leftskip=0em{}
\pstart
           \noindent{}Herrn Dr. Arthur Schnitzler\pend
           
\pstart
           \so{Wien.}\oindex{Wien@\textbf{Wien}, \emph{A.ADM2}|pw}\pend
           
\pstart
           \textcolor{gray}{\textbf{\emph{Alle für »Die Zeit\orgindex{Zeit@Die Zeit|pw}«
                        bestimmten Zuschriften und Sendungen sind an die Redaction »Die Zeit\orgindex{Zeit@Die Zeit|pw}« und \textbf{nicht} an die
                        Person eines der Herausgeber oder Mitarbeiter zu richten.}}}\pend
           \selectlanguage{ngerman}\endnumbering\briefempfaengerindex{Schnitzler, Arthur@\textsc{Schnitzler, Arthur}!zzzSalten, Felix@\emph{von Felix Salten}!1903-12-091@{9. 12. 1903}|)be}\mylabel{L03354h}  \normalsize

\doendnotes{C}
\bigskip
\vfill

\clearpage

\footnotesize

\lohead{\textsc{register}}

% Definiere theindex-Environment komplett neu ohne reledmac
\makeatletter
\renewenvironment{theindex}{%
  \section*{\indexname}%
  \setlength{\parindent}{0pt}%
  \setlength{\parskip}{0pt plus 0.3pt}%
  \let\item\@idxitem
}{%
  \clearpage
}
\makeatother

\IfFileExists{\jobname-pw.ind}{\input{\jobname-pw.ind}}{}

\end{document}

      