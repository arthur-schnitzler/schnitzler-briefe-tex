%% latex-leseansicht-vorspann.tex
%% Vorspann für die Leseansicht.
%% Lädt die gemeinsame Datei latex-vorspann.tex mit nicht gesetztem Schalter.

\newif\ifkorrekturansicht
\korrekturansichtfalse

\input{../tex-inputs/latex-vorspann}

\begin{center}
            \textcolor{red}{ENTWURF, NICHT FERTIG KORRIGIERT}
                      \end{center}
            
         
         \renewcommand{\erwaehntePersonen}{Personen: Heinrich Kanner, Paul Schlenther, Isidor Singer}
         \renewcommand{\erwaehnteInstitutionen}{Institutionen: Die Zeit}
         \renewcommand{\erwaehnteOrte}{Orte: Wien, Wipplingerstraße}
         \renewcommand{\erwaehnteWerke}{Werke: Der Schleier der Beatrice. Schauspiel in fünf Akten, Die Zeit, Erklärung [Schleier der Beatrice]}
               \section[Felix Salten an Arthur Schnitzler, 9. 12. 1903]{ Felix Salten an Arthur Schnitzler, 9. 12. 1903}\nopagebreak\mylabel{v}\rehead{ }\begin{ledgroupsized}[t]{13cm}\normalsize\beginnumbering \toendnotes[C]{\smallbreak\pagebreak[2]} \Standort{CUL, Schnitzler, B 89, A 2.}
\physDesc{Brief, 1 Blatt, 1 Seite, 524 Zeichen
\newline{}Schreibmaschine
\newline{}Handschrift: schwarze Tinte, lateinische Kurrent (\noindent{}Korrektur und Unterschrift)
\newline{}Ordnung: mit Bleistift von unbekannter Hand nummeriert:
                                    »180« }\toendnotes[C]{\smallbreak}\pstart
           \noindent{}{\pb}\textcolor{gray}{\textbf{DIE}}\pend
           \pstart
           \textcolor{gray}{\textbf{ZEIT\orgindex{Zeit@Die Zeit|pw}}}\pend
           \pstart
           \textcolor{gray}{\textbf{Wien\oindex{Wien@\textbf{Wien}|pw}er Tageszeitung}}\hfill \textcolor{gray}{\textbf{WIEN\oindex{Wien@\textbf{Wien}|pw}}}{ }9. Dezember 1903. \pend
           \pstart
           \textcolor{gray}{\textbf{Herausgeber: }}\hfill \textcolor{gray}{\textbf{I. Wipplingerstrasse 38\oindex{Wipplingerstrasse@\textbf{Wipplingerstraße}|pw}}}\pend
           \pstart
           \textcolor{gray}{\textbf{Prof. Dr. I. Singer\pwindex{Singer, Isidor 16.01.1857 – 08.12.1927@\textsc{Singer, Isidor} (16.01.1857 – 08.12.1927), \emph{Journalist, Herausgeber, Soziologe}|pw}}}\pend
           \pstart
           \textcolor{gray}{\textbf{Dr. Heinrich Kanner\pwindex{Kanner, Heinrich 09.11.1864 – 15.02.1930@\textsc{Kanner, Heinrich} (09.11.1864 – 15.02.1930), \emph{Herausgeber, Publizist}|pw}}}\pend
           \pstart
           \textcolor{gray}{\textbf{\textbf{Redaction}}}\pend
           \pstart
           \textcolor{gray}{\textbf{Telegramm-Adresse: \so{Zeit}\orgindex{Zeit@Die Zeit|pw}\so{,{ }}\so{Wien}\oindex{Wien@\textbf{Wien}|pw}}}\pend
           \pstart
           \textcolor{gray}{\textbf{\textbf{Telephone:}}}\pend
           \pstart
           \textcolor{gray}{\textbf{Interurbanes Telephon Nr. 15.988}}\pend
           \pstart
           \textcolor{gray}{\textbf{= Telephone Nr. 17.040, 17.041 =}}\pend
           \pstart
           \textcolor{gray}{\textbf{Depeschensaal Nr. 4548.}}\pend
           \pstart
           Sa/H\pend
           \pstart\center{}Lieber Freund!\pend\pstart
           Da unsere Weihnachtsnummer jetzt fertig gestellt werden muss, frage ich Sie, ob Sie
                  \label{K_L03354-54v}\edtext{etwas für mich haben}{\lemma{\textnormal{\emph{etwas für mich haben}}}\Cendnote{\textnormal{Von Schnitzler\pwindex{Schnitzler, Arthur 15.05.1862 – 21.10.1931@\textsc{Schnitzler, Arthur} (15.05.1862 – 21.10.1931), \emph{Schriftsteller, Mediziner}|pwk} erschien nichts in der Weihnachtsbeilage der \emph{Zeit}\pwindex{Zeit1902-09-27 – 1919@\emph{Die Zeit} {[}1902-09-27 – 1919{]}|pwk}.}}}\label{K_L03354-54h}. Es muss nichts Grosses sein, aber aus
               mancherlei Gründen wäre es mir lieb, wenn Sie mir irgend etwas schicken können. Die
                  \label{K_L03354-1v}\edtext{Schlenther\pwindex{Schlenther, Paul 20.08.1854 – 30.04.1916@\textsc{Schlenther, Paul} (20.08.1854 – 30.04.1916), \emph{Schriftsteller, Kritiker, Theaterleiter}|pw}-Briefe}{\lemma{\textnormal{\emph{Schlenther-Briefe}}}\Cendnote{\textnormal{Handelt es sich noch um die Briefe, die Schlenther\pwindex{Schlenther, Paul 20.08.1854 – 30.04.1916@\textsc{Schlenther, Paul} (20.08.1854 – 30.04.1916), \emph{Schriftsteller, Kritiker, Theaterleiter}|pwk}{ }Schnitzler\pwindex{Schnitzler, Arthur 15.05.1862 – 21.10.1931@\textsc{Schnitzler, Arthur} (15.05.1862 – 21.10.1931), \emph{Schriftsteller, Mediziner}|pwk} zur geplanten Annahme und
                  späteren Ablehnung von \emph{Der Schleier der
                     Beatrice}\pwindex{Schnitzler, Arthur 15.05.1862 – 21.10.1931@\textsc{Schnitzler, Arthur} (15.05.1862 – 21.10.1931), \emph{Schriftsteller, Mediziner}!Schleier der Beatrice. Schauspiel in fuenf Akten1900-12-01@\strich\emph{Der Schleier der Beatrice. Schauspiel in fünf Akten} {[}1900-12-01{]}|pwk} geschickt hatte? Salten\pwindex{Salten, Felix 06.09.1869 – 08.10.1945@\textsc{Salten, Felix} (06.09.1869 – 08.10.1945), \emph{Schriftsteller, Journalist}|pwk}
                  organisierte damals den Protest, der zur \emph{Erklärung}\pwindex{\textcolor{red}{\textsuperscript{XXXX1 indx}}!Erklaerung [Schleier der Beatrice]1900-09-14@\strich\emph{Erklärung [Schleier der Beatrice]} {[}1900-09-14{]}|pwk}\pwindex{Salten, Felix 06.09.1869 – 08.10.1945@\textsc{Salten, Felix} (06.09.1869 – 08.10.1945), \emph{Schriftsteller, Journalist}!Erklaerung [Schleier der Beatrice]1900-09-14@\strich\emph{Erklärung [Schleier der Beatrice]} {[}1900-09-14{]}|pwk}\pwindex{\textcolor{red}{\textsuperscript{XXXX1 indx}}!Erklaerung [Schleier der Beatrice]1900-09-14@\strich\emph{Erklärung [Schleier der Beatrice]} {[}1900-09-14{]}|pwk}\pwindex{\textcolor{red}{\textsuperscript{XXXX1 indx}}!Erklaerung [Schleier der Beatrice]1900-09-14@\strich\emph{Erklärung [Schleier der Beatrice]} {[}1900-09-14{]}|pwk}\pwindex{\textcolor{red}{\textsuperscript{XXXX1 indx}}!Erklaerung [Schleier der Beatrice]1900-09-14@\strich\emph{Erklärung [Schleier der Beatrice]} {[}1900-09-14{]}|pwk}\pwindex{\textcolor{red}{\textsuperscript{XXXX1 indx}}!Erklaerung [Schleier der Beatrice]1900-09-14@\strich\emph{Erklärung [Schleier der Beatrice]} {[}1900-09-14{]}|pwk} von 6 Autoren in den Tageszeitungen geführt hatte. Vgl. Richard Beer-Hofmann an Arthur Schnitzler, 14. 9. 1900.}}}\label{K_L03354-1h} habe ich Ihnen
               gleich am Montag rekommandiert zurückgeschickt. Hoffentlich bin ich in der nächsten
               Woche mit dem Preisausschreiben so weit fertig, um einmal nachmittags zu Ihnen kommen
               zu können. \pend
           \pstart
           Herzlichst {\\[\baselineskip]}Ihr {\\[\baselineskip]}\spacefill\mbox{Salten}\pend
           \leftskip=0em{}\pstart
           \noindent{}Herrn Dr. Arthur Schnitzler\pend
           \pstart
           \so{Wien}\oindex{Wien@\textbf{Wien}|pw}\pend
           \pstart
           \textcolor{gray}{\textbf{Alle für »Die Zeit\orgindex{Zeit@Die Zeit|pw}«
                     bestimmten Zuschriften und Sendungen sind an die Redaction »Die Zeit\orgindex{Zeit@Die Zeit|pw}« und \textbf{nicht} an die Person
                     eines der Herausgeber oder Mitarbeiter zu richten.}}\pend
           
         
         \endnumbering\mylabel{h}\end{ledgroupsized}\begin{anhang}\end{anhang}\newcommand{\dateiname}{L03354}\newcommand{\titel}{Felix Salten an Arthur Schnitzler, 9. 12. 1903}\newcommand{\editorInnen}{Martin Anton Müller und Laura Untner}%% latex-leseansicht-abspann.tex
%% Abspann für die Leseansicht.
%% Der Schalter \ifkorrekturansicht ist bereits durch den Vorspann gesetzt.

%% latex-abspann.tex
%% Gemeinsamer Abspann für Korrekturansicht und Leseansicht.
%% Setzt den Schalter \ifkorrekturansicht voraus (gesetzt in den
%% einbindenden Dateien latex-korrekturansicht-abspann.tex bzw.
%% latex-leseansicht-abspann.tex).
%% ---------------------------------------------------------------

\normalsize

% Das esempio-Environment wird nur in der Leseansicht benötigt
\ifkorrekturansicht\else
\newenvironment{esempio}[3]%
{
    \vspace{1.5ex}
    \rlap{\underline{#1}}
    \par
    \setlength{\parindent}{0cm}
    \nopagebreak
    \leftskip=#2cm
    \rightskip=#3cm
}
{
    \par
}
\fi

\doendnotes{C}
\bigskip
\vfill

\clearpage

\footnotesize

\ifkorrekturansicht
  \lohead{\textsc{register}}
\fi

% theindex-Environment neu definieren ohne reledmac
\makeatletter
\renewenvironment{theindex}{%
  \ifkorrekturansicht
    \section*{\indexname}%
  \else
    \subsubsection*{Index der erwähnten Entitäten}%
  \fi
  \setlength{\parindent}{0pt}%
  \setlength{\parskip}{0pt plus 0.3pt}%
  \let\item\@idxitem
}{%
  \ifkorrekturansicht\clearpage\fi
}
\makeatother

\IfFileExists{\jobname-pw.ind}{\input{\jobname-pw.ind}}{}

% Quellenangabe nur in der Leseansicht
\ifkorrekturansicht\else
% Fallback-Definitionen, falls die .tex-Datei \titel etc. nicht gesetzt hat
\providecommand{\titel}{}
\providecommand{\editorInnen}{}
\providecommand{\dateiname}{\jobname}

\vspace{3cm}

\vfill

\footnotesize
\textsc{Quelle}: \titel. Herausgegeben von {\editorInnen}. In: \emph{Arthur Schnitzler: Briefwechsel mit Autorinnen und Autoren}.
 Digitale Edition, https://schnitzler-briefe.acdh.oeaw.ac.at/{\dateiname}.html (Stand \today)
\fi

\end{document}


      