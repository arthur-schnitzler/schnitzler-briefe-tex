%% latex-leseansicht-vorspann.tex
%% Vorspann für die Leseansicht.
%% Lädt die gemeinsame Datei latex-vorspann.tex mit nicht gesetztem Schalter.

\newif\ifkorrekturansicht
\korrekturansichtfalse

\input{../tex-inputs/latex-vorspann}

\begin{center}
            \textcolor{red}{ENTWURF, NICHT FERTIG KORRIGIERT}
                      \end{center}
            
         
         \renewcommand{\erwaehntePersonen}{Personen: Eduard Bacher, Moriz Benedikt, Hugo Ganz}
         \renewcommand{\erwaehnteInstitutionen}{Institutionen: Die Zeit. Wiener Wochenschrift, Neue Freie Presse}
         \renewcommand{\erwaehnteOrte}{Orte: Berlin, Brühl, Dessauer Straße, Wien}
         \renewcommand{\erwaehnteWerke}{Werke: Neue Freie Presse}
               \section[ Paul Goldmann an Arthur Schnitzler, 5. 5. {[}1902{]}]{ Paul Goldmann an Arthur Schnitzler, 5. 5. {[}1902{]}}\nopagebreak\mylabel{v}\rehead{ }\begin{ledgroupsized}[t]{13cm}\normalsize\beginnumbering \toendnotes[C]{\smallbreak\pagebreak[2]} \Standort{DLA, A:Schnitzler, HS.NZ85.1.3172.}
\physDesc{Brief, 1 Blatt, 1 Seite
\newline{}Handschrift: blaue Tinte, deutsche Kurrent
\newline{}Schnitzler: 1) mit Bleistift das Jahr »1902« und »\textcolor{gray}{I}« vermerkt  2) mit rotem Buntstift eine Unterstreichung}\toendnotes[C]{\smallbreak}\pstart
           \noindent{}\raggedleft{}{\pb}\textcolor{gray}{\textbf{DESSAUERSTRASSE 19}}\oindex{Dessauer Strasse@\textbf{Dessauer Straße}|pw}\pend
           \pstart
           Berlin\oindex{Berlin@\textbf{Berlin}|pw}, 5. Mai.\pend
           \pstart\center{}Mein lieber Freund,\pend\pstart
           Ich möchte \introOben{}zu Pfingſten\introOben{} auf ein paar Tage \label{K_L03207-1v}\edtext{nach Wien\oindex{Wien@\textbf{Wien}|pw} kommen}{\lemma{\textnormal{\emph{nach Wien kommen}}}\Cendnote{\textnormal{siehe Paul Goldmann an Arthur Schnitzler, 2. 5. [1902]}}}\label{K_L03207-1h}, um mit den Herausgeber\pwindex{Bacher, Eduard 07.03.1846 – 16.01.1908@\textsc{Bacher, Eduard} (07.03.1846 – 16.01.1908), \emph{Journalist, Herausgeber}|pwv}\pwindex{Benedikt, Moriz 27.05.1849 – 18.03.1920@\textsc{Benedikt, Moriz} (27.05.1849 – 18.03.1920), \emph{Journalist, Herausgeber}|pwv}n der N. Fr. Pr.\pwindex{Neue Freie Presse1864 – 1939@\emph{Neue Freie Presse} {[}1864 – 1939{]}|pw} Einiges zu
               beſprechen. Schon deshalb kann ich nicht in der Brühl\oindex{Bruehl@\textbf{Brühl}|pw} wohnen. Wohnſt Du denn auch in der \label{K_L03207-2v}\edtext{Brühl\oindex{Bruehl@\textbf{Brühl}|pw}}{\lemma{\textnormal{\emph{Brühl}}}\Cendnote{\textnormal{Schnitzler\pwindex{Schnitzler, Arthur 15.05.1862 – 21.10.1931@\textsc{Schnitzler, Arthur} (15.05.1862 – 21.10.1931), \emph{Schriftsteller, Mediziner}|pwk} war zu dieser Zeit sehr häufig in
                  der Brühl\oindex{Bruehl@\textbf{Brühl}|pwk}. Als Goldmann\pwindex{Goldmann, Paul 31.01.1865 – 25.09.1935@\textsc{Goldmann, Paul} (31.01.1865 – 25.09.1935), \emph{Schriftsteller, Journalist}|pwk} ihn zu Pfingsten
                  in Wien\oindex{Wien@\textbf{Wien}|pwk} besuchte, waren sie auch gemeinsam
                  dort, jedenfalls am 19. 5. 1902 und am 25. 5. 1902, eventuell auch am 20. 5. 1902.}}}\label{K_L03207-2h}?\pend
           \pstart
           \label{K_L03207-3v}\edtext{\textsc{Ganz\pwindex{Ganz, Hugo 24.04.1862 – 02.01.1922@\textsc{Ganz, Hugo} (24.04.1862 – 02.01.1922), \emph{Schriftsteller, Journalist}|pw}} geht zur »Zeit\orgindex{Zeit. Wiener Wochenschrift@Die Zeit. Wiener Wochenschrift|pw}«}{\lemma{\textnormal{\emph{Ganz geht zur »Zeit«}}}\Cendnote{\textnormal{Hugo Ganz\pwindex{Ganz, Hugo 24.04.1862 – 02.01.1922@\textsc{Ganz, Hugo} (24.04.1862 – 02.01.1922), \emph{Schriftsteller, Journalist}|pwk}, der zuvor für die \emph{Neue Freie Presse}\orgindex{Neue Freie Presse@Neue Freie Presse|pwk} arbeitete, hatte am 25. 4. 1902 einen fünfjährigen Vertrag als
                  Leitartikler, politischer Redakteur und Chefredakteurstellvertreter mit der \emph{Zeit}\orgindex{Zeit. Wiener Wochenschrift@Die Zeit. Wiener Wochenschrift|pwk} unterzeichnet. Vgl. \emph{Das Recht. Volkstümliche Zeitschrift für österreichisches Rechtsleben.
                        Bde. 1–3}. Wien\oindex{Wien@\textbf{Wien}|pwk}{ }1902, S. 84.}}}\label{K_L03207-3h}.\pend
           \pstart
           Viele treue Grüße! {\\[\baselineskip]}Dein {\\[\baselineskip]}\spacefill\mbox{Paul Goldm}\pend
           \leftskip=0em{}
         
         \endnumbering\mylabel{h}\end{ledgroupsized}\begin{anhang}\end{anhang}\newcommand{\dateiname}{L03207}\newcommand{\titel}{Paul Goldmann an Arthur Schnitzler, 5. 5. [1902]}\newcommand{\editorInnen}{Martin Anton Müller und Laura Untner}%% latex-leseansicht-abspann.tex
%% Abspann für die Leseansicht.
%% Der Schalter \ifkorrekturansicht ist bereits durch den Vorspann gesetzt.

%% latex-abspann.tex
%% Gemeinsamer Abspann für Korrekturansicht und Leseansicht.
%% Setzt den Schalter \ifkorrekturansicht voraus (gesetzt in den
%% einbindenden Dateien latex-korrekturansicht-abspann.tex bzw.
%% latex-leseansicht-abspann.tex).
%% ---------------------------------------------------------------

\normalsize

% Das esempio-Environment wird nur in der Leseansicht benötigt
\ifkorrekturansicht\else
\newenvironment{esempio}[3]%
{
    \vspace{1.5ex}
    \rlap{\underline{#1}}
    \par
    \setlength{\parindent}{0cm}
    \nopagebreak
    \leftskip=#2cm
    \rightskip=#3cm
}
{
    \par
}
\fi

\doendnotes{C}
\bigskip
\vfill

\clearpage

\footnotesize

\ifkorrekturansicht
  \lohead{\textsc{register}}
\fi

% theindex-Environment neu definieren ohne reledmac
\makeatletter
\renewenvironment{theindex}{%
  \ifkorrekturansicht
    \section*{\indexname}%
  \else
    \subsubsection*{Index der erwähnten Entitäten}%
  \fi
  \setlength{\parindent}{0pt}%
  \setlength{\parskip}{0pt plus 0.3pt}%
  \let\item\@idxitem
}{%
  \ifkorrekturansicht\clearpage\fi
}
\makeatother

\IfFileExists{\jobname-pw.ind}{\input{\jobname-pw.ind}}{}

% Quellenangabe nur in der Leseansicht
\ifkorrekturansicht\else
% Fallback-Definitionen, falls die .tex-Datei \titel etc. nicht gesetzt hat
\providecommand{\titel}{}
\providecommand{\editorInnen}{}
\providecommand{\dateiname}{\jobname}

\vspace{3cm}

\vfill

\footnotesize
\textsc{Quelle}: \titel. Herausgegeben von {\editorInnen}. In: \emph{Arthur Schnitzler: Briefwechsel mit Autorinnen und Autoren}.
 Digitale Edition, https://schnitzler-briefe.acdh.oeaw.ac.at/{\dateiname}.html (Stand \today)
\fi

\end{document}


      