%% latex-leseansicht-vorspann.tex
%% Vorspann für die Leseansicht.
%% Lädt die gemeinsame Datei latex-vorspann.tex mit nicht gesetztem Schalter.

\newif\ifkorrekturansicht
\korrekturansichtfalse

\input{../tex-inputs/latex-vorspann}


\section[ Paul Goldmann an Arthur Schnitzler, 5. 5. {[}1902{]}]{L03207 Paul Goldmann an Arthur Schnitzler,  5. 5. [1902]}
\nopagebreak\mylabel{L03207v}
\rehead{ }\normalsize\beginnumbering\briefempfaengerindex{Schnitzler, Arthur@\textsc{Schnitzler, Arthur}!zzzGoldmann, Paul@\emph{von Paul Goldmann}!1902-05-051@{5. 5. [1902]}|(be}
\toendnotes[C]{\smallbreak\pagebreak[2]}
\correspDesc{Versand  durch Paul Goldmann am 5. 5. [1902] in Berlin
\newline{}Erhalt  durch Arthur Schnitzler im Zeitraum [6. 5. 1902
                  – 8. 5. 1902?] in Wien}\toendnotes[C]{\smallbreak}
\Standort{DLA, A:Schnitzler, HS.NZ85.1.3172.}
\physDesc{Brief, 1 Blatt, 1 Seite, 278 Zeichen
\newline{}Handschrift: blaue Tinte, deutsche Kurrent
\newline{}Schnitzler: 1) mit Bleistift das Jahr »1902« und »1« vermerkt  2) mit rotem Buntstift eine Unterstreichung}\toendnotes[C]{\smallbreak}
\pstart
           \raggedleft{}{\pb}\textcolor{gray}{\textbf{DESSAUERSTRASSE 19}}\oindex{Dessauer Straße@\textbf{Dessauer Straße}, \emph{Straße}|pw}\pend
           
\pstart
           Berlin\oindex{Berlin@\textbf{Berlin}, \emph{Hauptstadt}|pw}, 5. Mai.\pend
           
\pstart\center{}Mein lieber Freund,\pend\vspace{0.5em}
\pstart
           Ich möchte \introOben{}zu \label{K_L03207-1v}\edtext{Pfingſten}{\lemma{\textnormal{\emph{Pfingsten}}}\Cendnote{\textnormal{Pfingstsonntag war der 18. 5. 1902.}}}\label{K_L03207-1}\introOben{} auf ein paar Tage \label{K_L03207-2v}\edtext{nach Wien\oindex{Wien@\textbf{Wien}, \emph{Verwaltungsgebiet}|pw} kommen}{\lemma{\textnormal{\emph{nach Wien kommen}}}\Cendnote{\textnormal{Siehe XXXX Auszeichnungsfehler: Dokument L03206 nicht gefunden.
               }}}\label{K_L03207-2}, um mit den Herausgeber\pwindex{Bacher, Eduard 7.\,3.\,1846 Postoloprty – 16.\,1.\,1908 Wien@\textsc{Bacher, Eduard} (7.\,3.\,1846 Postoloprty – 16.\,1.\,1908 Wien), \emph{Journalist, Herausgeber}|pwv}\pwindex{Benedikt, Moriz 27.\,5.\,1849 Kvačice – 18.\,3.\,1920 Wien@\textsc{Benedikt, Moriz} (27.\,5.\,1849 Kvačice – 18.\,3.\,1920 Wien), \emph{Journalist, Herausgeber}|pwv}n der N. Fr. Pr.\pwindex{Neue Freie Presse@\emph{Neue Freie Presse}|pw} Einiges zu
               beſprechen. Schon deshalb kann ich nicht in der Brühl\oindex{Brühl@\textbf{Brühl}, \emph{Tal}|pw} wohnen. Wohnſt Du denn auch in der \label{K_L03207-3v}\edtext{Brühl\oindex{Brühl@\textbf{Brühl}, \emph{Tal}|pw}}{\lemma{\textnormal{\emph{Brühl}}}\Cendnote{\textnormal{Olga Gussmann\pwindex{Schnitzler, Olga 17.\,1.\,1882 Wien – 13.\,1.\,1970 Lugano@\textsc{Schnitzler, Olga} (17.\,1.\,1882 Wien – 13.\,1.\,1970 Lugano), \emph{Schauspielerin, Sängerin}|pwk}, die mit dem gemeinsamen Sohn\pwindex{Schnitzler, Heinrich 9.\,8.\,1902 Hinterbrühl – 12.\,7.\,1982 Wien@\textsc{Schnitzler, Heinrich} (9.\,8.\,1902 Hinterbrühl – 12.\,7.\,1982 Wien), \emph{Regisseur, Schauspieler}|pwkv}
                      schwanger war, wohnte zu dieser
                  Zeit in einer Villa\oindex{Hauptstraße 56@\textbf{Hauptstraße 56}, \emph{Wohngebäude}|pwkv} in
                     Hinterbrühl\oindex{Hinterbrühl@\textbf{Hinterbrühl}, \emph{Hauptstadt}|pwk}. Schnitzler besuchte sie häufig, auch über Nacht. Als Goldmann\pwindex{Goldmann, Paul 31.\,1.\,1865 Breslau – 25.\,9.\,1935 Wien@\textsc{Goldmann, Paul} (31.\,1.\,1865 Breslau – 25.\,9.\,1935 Wien), \emph{Schriftsteller, Journalist}|pwk} ihn zu Pfingsten in Wien\oindex{Wien@\textbf{Wien}, \emph{Verwaltungsgebiet}|pwk} besuchte, waren
                  sie auch gemeinsam dort, jedenfalls am 19. 5. 1902 und am 25. 5. 1902, eventuell
                  auch am 20. 5. 1902.}}}\label{K_L03207-3}?\pend
           
\pstart
           \label{K_L03207-4v}\edtext{\textsc{Ganz\pwindex{Ganz, Hugo 24.\,4.\,1862 Mainz – 2.\,1.\,1922 Wien@\textsc{Ganz, Hugo} (24.\,4.\,1862 Mainz – 2.\,1.\,1922 Wien), \emph{Schriftsteller, Journalist}|pw}} geht zur »Zeit\orgindex{Zeit@Die Zeit|pw}«}{\lemma{\textnormal{\emph{Ganz geht zur »Zeit«}}}\Cendnote{\textnormal{Hugo Ganz\pwindex{Ganz, Hugo 24.\,4.\,1862 Mainz – 2.\,1.\,1922 Wien@\textsc{Ganz, Hugo} (24.\,4.\,1862 Mainz – 2.\,1.\,1922 Wien), \emph{Schriftsteller, Journalist}|pwk}, der zuvor für die \emph{Neue Freie Presse}\orgindex{Neue Freie Presse@Neue Freie Presse|pwk} gearbeitet hatte, hatte am 25. 4. 1902 einen Fünfjahresvertrag als
                  Leitartikler, politischer Redakteur und Chefredakteurstellvertreter mit der \emph{Zeit}\orgindex{Zeit@Die Zeit|pwk} unterzeichnet. Vgl. \emph{Das Recht. Volkstümliche Zeitschrift für österreichisches Rechtsleben.
                        Bde. 1–3}. Wien\oindex{Wien@\textbf{Wien}, \emph{Verwaltungsgebiet}|pwk}{ }1902, S. 84.}}}\label{K_L03207-4}.\pend
           
\pstart
           Viele treue Grüße! {\\[\baselineskip]}Dein {\\[\baselineskip]}\spacefill\mbox{Paul Goldm}\pend
           \leftskip=0em{}\selectlanguage{ngerman}\endnumbering\briefempfaengerindex{Schnitzler, Arthur@\textsc{Schnitzler, Arthur}!zzzGoldmann, Paul@\emph{von Paul Goldmann}!1902-05-051@{5. 5. [1902]}|)be}\mylabel{L03207h}  \newcommand{\dateiname}{L03207}\newcommand{\titel}{Paul Goldmann an Arthur Schnitzler, 5. 5. [1902]}\newcommand{\editorInnen}{Martin Anton Müller und Laura Untner}%% latex-leseansicht-abspann.tex
%% Abspann für die Leseansicht.
%% Der Schalter \ifkorrekturansicht ist bereits durch den Vorspann gesetzt.

%% latex-abspann.tex
%% Gemeinsamer Abspann für Korrekturansicht und Leseansicht.
%% Setzt den Schalter \ifkorrekturansicht voraus (gesetzt in den
%% einbindenden Dateien latex-korrekturansicht-abspann.tex bzw.
%% latex-leseansicht-abspann.tex).
%% ---------------------------------------------------------------

\normalsize

% Das esempio-Environment wird nur in der Leseansicht benötigt
\ifkorrekturansicht\else
\newenvironment{esempio}[3]%
{
    \vspace{1.5ex}
    \rlap{\underline{#1}}
    \par
    \setlength{\parindent}{0cm}
    \nopagebreak
    \leftskip=#2cm
    \rightskip=#3cm
}
{
    \par
}
\fi

\doendnotes{C}
\bigskip
\vfill

\clearpage

\footnotesize

\ifkorrekturansicht
  \lohead{\textsc{register}}
\fi

% theindex-Environment neu definieren ohne reledmac
\makeatletter
\renewenvironment{theindex}{%
  \ifkorrekturansicht
    \section*{\indexname}%
  \else
    \subsubsection*{Index der erwähnten Entitäten}%
  \fi
  \setlength{\parindent}{0pt}%
  \setlength{\parskip}{0pt plus 0.3pt}%
  \let\item\@idxitem
}{%
  \ifkorrekturansicht\clearpage\fi
}
\makeatother

\IfFileExists{\jobname-pw.ind}{\input{\jobname-pw.ind}}{}

% Quellenangabe nur in der Leseansicht
\ifkorrekturansicht\else
% Fallback-Definitionen, falls die .tex-Datei \titel etc. nicht gesetzt hat
\providecommand{\titel}{}
\providecommand{\editorInnen}{}
\providecommand{\dateiname}{\jobname}

\vspace{3cm}

\vfill

\footnotesize
\textsc{Quelle}: \titel. Herausgegeben von {\editorInnen}. In: \emph{Arthur Schnitzler: Briefwechsel mit Autorinnen und Autoren}.
 Digitale Edition, https://schnitzler-briefe.acdh.oeaw.ac.at/{\dateiname}.html (Stand \today)
\fi

\end{document}


