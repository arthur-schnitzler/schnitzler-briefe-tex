%% latex-korrekturansicht-vorspann.tex
%% Vorspann für die Korrekturansicht.
%% Lädt die gemeinsame Datei latex-vorspann.tex mit gesetztem Schalter.

\newif\ifkorrekturansicht
\korrekturansichttrue

\input{../tex-inputs/latex-vorspann}


\section[ Paul Goldmann an Arthur Schnitzler, 5. 5. {[}1902{]}]{L03207 Paul Goldmann an Arthur Schnitzler, 5. 5. {[}1902{]}}
\nopagebreak\mylabel{L03207v}
\rehead{ }\normalsize\beginnumbering\briefempfaengerindex{Schnitzler, Arthur@\textsc{Schnitzler, Arthur}!zzzGoldmann, Paul@\emph{von Paul Goldmann}!1902-05-051@{5. 5. {[}1902{]}}|(be}
\toendnotes[C]{\smallbreak\pagebreak[2]}\Standort{DLA, A:Schnitzler, HS.NZ85.1.3172.}
\physDesc{Brief, 1 Blatt, 1 Seite, 278 Zeichen
\newline{}Handschrift: blaue Tinte, deutsche Kurrent
\newline{}Schnitzler: 1) mit Bleistift das Jahr »1902« und »1« vermerkt  2) mit rotem Buntstift eine Unterstreichung}\toendnotes[C]{\smallbreak}
\pstart
           \raggedleft{}{\pb}\textcolor{gray}{\textbf{DESSAUERSTRASSE 19}}\oindex{Dessauer Strasse@\textbf{Dessauer Straße}, \emph{Straße (K.STR)}|pw}\pend
           
\pstart
           Berlin\oindex{Berlin@\textbf{Berlin}, \emph{P.PPLC}|pw}, 5. Mai.\pend
           
\pstart\center{}Mein lieber Freund,\pend\vspace{0.5em}
\pstart
           Ich möchte \introOben{}zu \label{K_L03207-1v}\edtext{Pfingſten}{\lemma{\textnormal{\emph{Pfingſten}}}\Cendnote{\textnormal{Pfingstsonntag war der 18. 5. 1902.}}}\label{K_L03207-1}\introOben{} auf ein paar Tage \label{K_L03207-2v}\edtext{nach Wien\oindex{Wien@\textbf{Wien}, \emph{A.ADM2}|pw} kommen}{\lemma{\textnormal{\emph{nach Wien kommen}}}\Cendnote{\textnormal{Siehe Paul Goldmann an Arthur Schnitzler, 2. 5. [1902].
               }}}\label{K_L03207-2}, um mit den Herausgeber\pwindex{Bacher, Eduard 07.03.1846 – 16.01.1908@\textsc{Bacher, Eduard} (07.03.1846 – 16.01.1908), \emph{Journalist/Journalistin, Herausgeber/Herausgeberin}|pwv}\pwindex{Benedikt, Moriz 27.05.1849 – 18.03.1920@\textsc{Benedikt, Moriz} (27.05.1849 – 18.03.1920), \emph{Journalist/Journalistin, Herausgeber/Herausgeberin}|pwv}n der N. Fr. Pr.\pwindex{Neue Freie Presse@\emph{Neue Freie Presse}|pw} Einiges zu
               beſprechen. Schon deshalb kann ich nicht in der Brühl\oindex{Bruehl@\textbf{Brühl}, \emph{Tal (N.TAL)}|pw} wohnen. Wohnſt Du denn auch in der \label{K_L03207-3v}\edtext{Brühl\oindex{Bruehl@\textbf{Brühl}, \emph{Tal (N.TAL)}|pw}}{\lemma{\textnormal{\emph{Brühl}}}\Cendnote{\textnormal{Olga Gussmann\pwindex{Schnitzler, Olga 17.01.1882 – 13.01.1970@\textsc{Schnitzler, Olga} (17.01.1882 – 13.01.1970), \emph{Schauspieler/Schauspielerin, Sänger/Sängerin}|pwk}, die mit dem gemeinsamen Sohn\pwindex{Schnitzler, Heinrich 09.08.1902 – 12.07.1982@\textsc{Schnitzler, Heinrich} (09.08.1902 – 12.07.1982), \emph{Regisseur/Regisseurin, Schauspieler/Schauspielerin}|pwkv}
                      schwanger war, wohnte zu dieser
                  Zeit in einer Villa\oindex{Hauptstrasse 56@\textbf{Hauptstraße 56}, \emph{Wohngebäude (K.WHS)}|pwkv} in
                     Hinterbrühl\oindex{Hinterbruehl@\textbf{Hinterbrühl}, \emph{P.PPLA3}|pwk}. Schnitzler besuchte sie häufig, auch über Nacht. Als Goldmann\pwindex{Goldmann, Paul 31.01.1865 – 25.09.1935@\textsc{Goldmann, Paul} (31.01.1865 – 25.09.1935), \emph{Schriftsteller/Schriftstellerin, Journalist/Journalistin}|pwk} ihn zu Pfingsten in Wien\oindex{Wien@\textbf{Wien}, \emph{A.ADM2}|pwk} besuchte, waren
                  sie auch gemeinsam dort, jedenfalls am 19. 5. 1902 und am 25. 5. 1902, eventuell
                  auch am 20. 5. 1902.}}}\label{K_L03207-3}?\pend
           
\pstart
           \label{K_L03207-4v}\edtext{\textsc{Ganz\pwindex{Ganz, Hugo 24.04.1862 – 02.01.1922@\textsc{Ganz, Hugo} (24.04.1862 – 02.01.1922), \emph{Schriftsteller/Schriftstellerin, Journalist/Journalistin}|pw}} geht zur »Zeit\orgindex{Zeit@Die Zeit|pw}«}{\lemma{\textnormal{\emph{Ganz geht zur »Zeit«}}}\Cendnote{\textnormal{Hugo Ganz\pwindex{Ganz, Hugo 24.04.1862 – 02.01.1922@\textsc{Ganz, Hugo} (24.04.1862 – 02.01.1922), \emph{Schriftsteller/Schriftstellerin, Journalist/Journalistin}|pwk}, der zuvor für die \emph{Neue Freie Presse}\orgindex{Neue Freie Presse@Neue Freie Presse|pwk} gearbeitet hatte, hatte am 25. 4. 1902 einen Fünfjahresvertrag als
                  Leitartikler, politischer Redakteur und Chefredakteurstellvertreter mit der \emph{Zeit}\orgindex{Zeit@Die Zeit|pwk} unterzeichnet. Vgl. \emph{Das Recht. Volkstümliche Zeitschrift für österreichisches Rechtsleben.
                        Bde. 1–3}. Wien\oindex{Wien@\textbf{Wien}, \emph{A.ADM2}|pwk}{ }1902, S. 84.}}}\label{K_L03207-4}.\pend
           
\pstart
           Viele treue Grüße! {\\[\baselineskip]}Dein {\\[\baselineskip]}\spacefill\mbox{Paul Goldm}\pend
           \leftskip=0em{}\selectlanguage{ngerman}\endnumbering\briefempfaengerindex{Schnitzler, Arthur@\textsc{Schnitzler, Arthur}!zzzGoldmann, Paul@\emph{von Paul Goldmann}!1902-05-051@{5. 5. {[}1902{]}}|)be}\mylabel{L03207h}  \normalsize

\doendnotes{C}
\bigskip
\vfill

\clearpage

\footnotesize

\lohead{\textsc{register}}

% Definiere theindex-Environment komplett neu ohne reledmac
\makeatletter
\renewenvironment{theindex}{%
  \section*{\indexname}%
  \setlength{\parindent}{0pt}%
  \setlength{\parskip}{0pt plus 0.3pt}%
  \let\item\@idxitem
}{%
  \clearpage
}
\makeatother

\IfFileExists{\jobname-pw.ind}{\input{\jobname-pw.ind}}{}

\end{document}

      