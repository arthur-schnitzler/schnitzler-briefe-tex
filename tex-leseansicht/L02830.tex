%% latex-leseansicht-vorspann.tex
%% Vorspann für die Leseansicht.
%% Lädt die gemeinsame Datei latex-vorspann.tex mit nicht gesetztem Schalter.

\newif\ifkorrekturansicht
\korrekturansichtfalse

\input{../tex-inputs/latex-vorspann}


\section[ Paul Goldmann an Arthur Schnitzler, 27. 10. [1897]]{L02830 Paul Goldmann an Arthur Schnitzler,  27. 10. [1897]}
\nopagebreak\mylabel{L02830v}
\rehead{ }\normalsize\beginnumbering\briefempfaengerindex{Schnitzler, Arthur@\textsc{Schnitzler, Arthur}!zzzGoldmann, Paul@\emph{von Paul Goldmann}!1897-10-273@{27. 10. [1897]}|(be}
\toendnotes[C]{\smallbreak\pagebreak[2]}
\correspDesc{Versand  durch Paul Goldmann am 27. 10. [1897] in Paris
\newline{}Erhalt  durch Arthur Schnitzler im Zeitraum [28. 10. 1897 – 1. 11. 1897?] in Wien}\toendnotes[C]{\smallbreak}
\Standort{DLA, A:Schnitzler, HS.NZ85.1.3167.}
\physDesc{Brief, 1 Blatt, 2 Seiten, 1054 Zeichen
\newline{}Handschrift: blaue Tinte, deutsche Kurrent
\newline{}Schnitzler: 1) mit Bleistift das Jahr »97« vermerkt  2) mit rotem Buntstift drei Unterstreichungen}\toendnotes[C]{\smallbreak}
\pstart
           {\pb}\textcolor{gray}{\textbf{\textbf{Frankfurter Zeitung\orgindex{Frankfurter Zeitung@Frankfurter Zeitung|pw}}}}\pend
           
\pstart
           \textcolor{gray}{\textbf{(\begin{otherlanguage}{french}Gazette de Francfort\end{otherlanguage}\orgindex{Frankfurter Zeitung@Frankfurter Zeitung|pw}).}}\pend
           
\pstart
           \textcolor{gray}{\textbf{\textbf{\begin{otherlanguage}{french}Fondateur M.\end{otherlanguage}{ }L. Sonnemann\pwindex{Sonnemann, Leopold 29.\,10.\,1831 Höchberg – 30.\,10.\,1909 Frankfurt am Main@\textsc{Sonnemann, Leopold} (29.\,10.\,1831 Höchberg – 30.\,10.\,1909 Frankfurt am Main), \emph{Journalist, Herausgeber}|pw}.}}}\pend
           
\pstart
           \begin{otherlanguage}{french}\textcolor{gray}{\textbf{Journal politique, financier,}}\end{otherlanguage}\pend
           
\pstart
           \begin{otherlanguage}{french}\textcolor{gray}{\textbf{commercial et littéraire.}}\end{otherlanguage}\pend
           
\pstart
           \begin{otherlanguage}{french}\textcolor{gray}{\textbf{\textbf{Paraissant trois fois par jour.}}}\end{otherlanguage}\pend
           
\pstart
           \begin{otherlanguage}{french}\textcolor{gray}{\textbf{\textbf{Bureau à Paris\oindex{Paris@\textbf{Paris}, \emph{Hauptstadt}|pw}}}}\end{otherlanguage}\hfill \textsc{Paris\oindex{Paris@\textbf{Paris}, \emph{Hauptstadt}|pw}}, 27. Oktober.\pend
           
\pstart
           \begin{otherlanguage}{french}\textcolor{gray}{\textbf{\textbf{10 \so{Rue de la Bourse}\oindex{rue de la Bourse@\textbf{rue de la Bourse}, \emph{Straße}|pw}.}}}\end{otherlanguage}\pend
           \vspace{0.5em}
\pstart
           Bitte, liebſter Freund, laß’ doch endlich wieder
               einmal etwas von Dir hören. Wie gehts Dir? Wie gehts »ihr\pwindex{Reinhard, Marie 13.\,3.\,1871 Wien – 18.\,3.\,1899 ebd.@\textsc{Reinhard, Marie} (13.\,3.\,1871 Wien – 18.\,3.\,1899 ebd.), \emph{Gesangspädagogin}|pwv}«? Wie gehts den Freunden?\pend
           
\pstart
           Alles{ }ſchweigt um mich herum, und ich bin ganz einſam.\pend
           
\pstart
           Ich{ }ſende Dir einen amüſanten \label{K_L02830-1v}\edtext{Artikel\pwindex{Rochefort, Henri de 31.\,1.\,1830 Paris – 30.\,6.\,1913 Aix-les-Bains@\textsc{Rochefort, Henri de} (31.\,1.\,1830 Paris – 30.\,6.\,1913 Aix-les-Bains), \emph{Schriftsteller, Politiker, Journalist}!?? [Artikel von Rochefort über Jesus]@\strich\emph{?? [Artikel von Rochefort über Jesus]}|pwv} von \textsc{Rochefort\pwindex{Rochefort, Henri de 31.\,1.\,1830 Paris – 30.\,6.\,1913 Aix-les-Bains@\textsc{Rochefort, Henri de} (31.\,1.\,1830 Paris – 30.\,6.\,1913 Aix-les-Bains), \emph{Schriftsteller, Politiker, Journalist}|pw}}}{\lemma{\textnormal{\emph{Artikel von Rochefort}}}\Cendnote{\textnormal{nicht ermittelt}}}\label{K_L02830-1}, welcher von
               unſerem Glaubensgenoſſen\pwindex{Jesus 7–4 v.\,u.\,Z. Nazareth – 30/31 Jerusalem@\textsc{Jesus} (7–4 v.\,u.\,Z. Nazareth – 30/31 Jerusalem), \emph{Wanderprediger}|pwv}
               handelt, der am Kreuz geſtorben iſt{\dots}\pend
           
\pstart
           \textsc{Thorel\pwindex{Thorel, Jean 11.\,9.\,1859 Éragny – 20.\,8.\,1916 Enghien-les-Bains@\textsc{Thorel, Jean} (11.\,9.\,1859 Éragny – 20.\,8.\,1916 Enghien-les-Bains), \emph{Übersetzer, Dramatiker}|pw}}{ }ſprach ich. Er müht{ }ſich, das Stück\pwindex{Schnitzler, Arthur 15.\,5.\,1862 Wien – 21.\,10.\,1931 ebd.@\textsc{Schnitzler, Arthur} (15.\,5.\,1862 Wien – 21.\,10.\,1931 ebd.), \emph{Schriftsteller, Mediziner}!Amourette. Pièce en trois actes. Adaptée de Arthur Schnitzler@\strich\emph{Amourette. Pièce en trois actes. Adaptée de Arthur Schnitzler}|pwv} anzubringen (aber vielleicht bemüht er{ }ſich nicht
                  genug?){[}.{]} Die Nachrichten{ }ſind wenig günſtig. \textsc{Antoine\pwindex{Antoine, André 31.\,1.\,1858 Limoges – 23.\,10.\,1943 Le Pouliguen@\textsc{Antoine, André} (31.\,1.\,1858 Limoges – 23.\,10.\,1943 Le Pouliguen), \emph{Theaterleiter, Schauspieler}|pw}} hat{ }ſich die Antwort vorbehalten,{ }ſcheint aber nicht{ }ſehr geneigt zur \label{K_L02830-2v}\edtext{Aufführung\pwindex{Schnitzler, Arthur 15.\,5.\,1862 Wien – 21.\,10.\,1931 ebd.@\textsc{Schnitzler, Arthur} (15.\,5.\,1862 Wien – 21.\,10.\,1931 ebd.), \emph{Schriftsteller, Mediziner}!Amourette. Pièce en trois actes. Adaptée de Arthur Schnitzler@\strich\emph{Amourette. Pièce en trois actes. Adaptée de Arthur Schnitzler}|pwv}}{\lemma{\textnormal{\emph{Aufführung}}}\Cendnote{\textnormal{Jean Thorel\pwindex{Thorel, Jean 11.\,9.\,1859 Éragny – 20.\,8.\,1916 Enghien-les-Bains@\textsc{Thorel, Jean} (11.\,9.\,1859 Éragny – 20.\,8.\,1916 Enghien-les-Bains), \emph{Übersetzer, Dramatiker}|pwk} versuchte (erfolglos) seine \emph{Liebelei}\pwindex{Schnitzler, Arthur 15.\,5.\,1862 Wien – 21.\,10.\,1931 ebd.@\textsc{Schnitzler, Arthur} (15.\,5.\,1862 Wien – 21.\,10.\,1931 ebd.), \emph{Schriftsteller, Mediziner}!Liebelei. Schauspiel in drei Akten@\strich\emph{Liebelei. Schauspiel in drei Akten}|pwk}-Übersetzung\pwindex{Schnitzler, Arthur 15.\,5.\,1862 Wien – 21.\,10.\,1931 ebd.@\textsc{Schnitzler, Arthur} (15.\,5.\,1862 Wien – 21.\,10.\,1931 ebd.), \emph{Schriftsteller, Mediziner}!Amourette. Pièce en trois actes. Adaptée de Arthur Schnitzler@\strich\emph{Amourette. Pièce en trois actes. Adaptée de Arthur Schnitzler}|pwkv} dem \emph{Théâtre Antoine}\orgindex{Théâtre Antoine@Théâtre Antoine|pwk} (von André Antoine\pwindex{Antoine, André 31.\,1.\,1858 Limoges – 23.\,10.\,1943 Le Pouliguen@\textsc{Antoine, André} (31.\,1.\,1858 Limoges – 23.\,10.\,1943 Le Pouliguen), \emph{Theaterleiter, Schauspieler}|pwk} geleitet) oder dem \emph{Odéon}\orgindex{Odéon@Odéon|pwk} zu vermitteln.}}}\label{K_L02830-2}.\pend
           
\pstart
           {\pb}Willſt Du Dich mit \textsc{Molière\pwindex{Molière 14.\,1.\,1622 Paris – 17.\,2.\,1673 ebd.@\textsc{Molière} (14.\,1.\,1622 Paris – 17.\,2.\,1673 ebd.), \emph{Schriftsteller, Theaterleiter, Schauspieler}|pw}} ganz, aber ganz befreunden? \label{K_L02830-3v}\edtext{Lies{ }ſeinen \textsc{Don Juan\pwindex{Molière 14.\,1.\,1622 Paris – 17.\,2.\,1673 ebd.@\textsc{Molière} (14.\,1.\,1622 Paris – 17.\,2.\,1673 ebd.), \emph{Schriftsteller, Theaterleiter, Schauspieler}!Dom Juan ou le Festin de pierre@\strich\emph{Dom Juan ou le Festin de pierre}|pwv}}}{\lemma{\textnormal{\emph{Lies seinen Don Juan}}}\Cendnote{\textnormal{Lektüre nicht nachweisbar, jedoch sah
                     Schnitzler in späteren Jahren mehrere
                  Inszenierungen von Molières\pwindex{Molière 14.\,1.\,1622 Paris – 17.\,2.\,1673 ebd.@\textsc{Molière} (14.\,1.\,1622 Paris – 17.\,2.\,1673 ebd.), \emph{Schriftsteller, Theaterleiter, Schauspieler}|pwk}{ }\emph{Don Juan}\pwindex{Molière 14.\,1.\,1622 Paris – 17.\,2.\,1673 ebd.@\textsc{Molière} (14.\,1.\,1622 Paris – 17.\,2.\,1673 ebd.), \emph{Schriftsteller, Theaterleiter, Schauspieler}!Dom Juan ou le Festin de pierre@\strich\emph{Dom Juan ou le Festin de pierre}|pwk} (vgl. A. S.: \emph{Tagebuch}, 21. 10. 1915, 2. 2. 1916 und 27. 9. 1919).}}}\label{K_L02830-3}, von ihm genannt »\textsc{Le festin de Pierre\pwindex{Molière 14.\,1.\,1622 Paris – 17.\,2.\,1673 ebd.@\textsc{Molière} (14.\,1.\,1622 Paris – 17.\,2.\,1673 ebd.), \emph{Schriftsteller, Theaterleiter, Schauspieler}!Dom Juan ou le Festin de pierre@\strich\emph{Dom Juan ou le Festin de pierre}|pw}}.«\pend
           
\pstart
           Ich weiß Dir nichts mehr zu{ }ſchreiben, als daß ich namenloſes Heimweh habe nach Wien\oindex{Wien@\textbf{Wien}, \emph{Verwaltungsgebiet}|pw}, nach Freundſchaſt, nach Heimlichkeit und
               Gemüthlichkeit. Von Liebe w\textcolor{gray}{i}ll ich nicht reden. So anſpruchsvoll
               bin ich{ }ſchon längſt nicht mehr. Aber nicht mehr fremd{ }ſein in der Fremde! {\dots}\pend
           
\pstart
           Grüß’ Dich Gott, liebſter Freund, und vergiß mich nicht gar{ }ſo{ }ſehr!\pend
           
\pstart
           Dein treuer {\\[\baselineskip]}\spacefill\mbox{Paul Goldmann}\pend
           \leftskip=0em{}
\pstart
           \noindent{}Deiner Freundin\pwindex{Reinhard, Marie 13.\,3.\,1871 Wien – 18.\,3.\,1899 ebd.@\textsc{Reinhard, Marie} (13.\,3.\,1871 Wien – 18.\,3.\,1899 ebd.), \emph{Gesangspädagogin}|pwv} viele
                  herzliche Grüße!\pend
           \selectlanguage{ngerman}\endnumbering\briefempfaengerindex{Schnitzler, Arthur@\textsc{Schnitzler, Arthur}!zzzGoldmann, Paul@\emph{von Paul Goldmann}!1897-10-273@{27. 10. [1897]}|)be}\mylabel{L02830h}  \newcommand{\dateiname}{L02830}\newcommand{\titel}{Paul Goldmann an Arthur Schnitzler, 27. 10. [1897]}\newcommand{\editorInnen}{Martin Anton Müller und Laura Untner}%% latex-leseansicht-abspann.tex
%% Abspann für die Leseansicht.
%% Der Schalter \ifkorrekturansicht ist bereits durch den Vorspann gesetzt.

%% latex-abspann.tex
%% Gemeinsamer Abspann für Korrekturansicht und Leseansicht.
%% Setzt den Schalter \ifkorrekturansicht voraus (gesetzt in den
%% einbindenden Dateien latex-korrekturansicht-abspann.tex bzw.
%% latex-leseansicht-abspann.tex).
%% ---------------------------------------------------------------

\normalsize

% Das esempio-Environment wird nur in der Leseansicht benötigt
\ifkorrekturansicht\else
\newenvironment{esempio}[3]%
{
    \vspace{1.5ex}
    \rlap{\underline{#1}}
    \par
    \setlength{\parindent}{0cm}
    \nopagebreak
    \leftskip=#2cm
    \rightskip=#3cm
}
{
    \par
}
\fi

\doendnotes{C}
\bigskip
\vfill

\clearpage

\footnotesize

\ifkorrekturansicht
  \lohead{\textsc{register}}
\fi

% theindex-Environment neu definieren ohne reledmac
\makeatletter
\renewenvironment{theindex}{%
  \ifkorrekturansicht
    \section*{\indexname}%
  \else
    \subsubsection*{Index der erwähnten Entitäten}%
  \fi
  \setlength{\parindent}{0pt}%
  \setlength{\parskip}{0pt plus 0.3pt}%
  \let\item\@idxitem
}{%
  \ifkorrekturansicht\clearpage\fi
}
\makeatother

\IfFileExists{\jobname-pw.ind}{\input{\jobname-pw.ind}}{}

% Quellenangabe nur in der Leseansicht
\ifkorrekturansicht\else
% Fallback-Definitionen, falls die .tex-Datei \titel etc. nicht gesetzt hat
\providecommand{\titel}{}
\providecommand{\editorInnen}{}
\providecommand{\dateiname}{\jobname}

\vspace{3cm}

\vfill

\footnotesize
\textsc{Quelle}: \titel. Herausgegeben von {\editorInnen}. In: \emph{Arthur Schnitzler: Briefwechsel mit Autorinnen und Autoren}.
 Digitale Edition, https://schnitzler-briefe.acdh.oeaw.ac.at/{\dateiname}.html (Stand \today)
\fi

\end{document}


