%% latex-korrekturansicht-vorspann.tex
%% Vorspann für die Korrekturansicht.
%% Lädt die gemeinsame Datei latex-vorspann.tex mit gesetztem Schalter.

\newif\ifkorrekturansicht
\korrekturansichttrue

\input{../tex-inputs/latex-vorspann}


\section[ Paul Goldmann an Arthur Schnitzler, 27. 10. {[}1897{]}]{L02830 Paul Goldmann an Arthur Schnitzler, 27. 10. {[}1897{]}}
\nopagebreak\mylabel{L02830v}
\rehead{ }\normalsize\beginnumbering\briefempfaengerindex{Schnitzler, Arthur@\textsc{Schnitzler, Arthur}!zzzGoldmann, Paul@\emph{von Paul Goldmann}!1897-10-273@{27. 10. {[}1897{]}}|(be}
\toendnotes[C]{\smallbreak\pagebreak[2]}\Standort{DLA, A:Schnitzler, HS.NZ85.1.3167.}
\physDesc{Brief, 1 Blatt, 2 Seiten, 1054 Zeichen
\newline{}Handschrift: blaue Tinte, deutsche Kurrent
\newline{}Schnitzler: 1) mit Bleistift das Jahr »97« vermerkt  2) mit rotem Buntstift drei Unterstreichungen}\toendnotes[C]{\smallbreak}
\pstart
           {\pb}\textcolor{gray}{\textbf{\textbf{Frankfurter Zeitung\orgindex{Frankfurter Zeitung@Frankfurter Zeitung|pw}}}}\pend
           
\pstart
           \textcolor{gray}{\textbf{(\begin{otherlanguage}{french}Gazette de Francfort\end{otherlanguage}\orgindex{Frankfurter Zeitung@Frankfurter Zeitung|pw}).}}\pend
           
\pstart
           \textcolor{gray}{\textbf{\textbf{\begin{otherlanguage}{french}Fondateur M.\end{otherlanguage}{ }L. Sonnemann\pwindex{Sonnemann, Leopold 1831-10-29 – 1909-10-30@\textsc{Sonnemann, Leopold} (1831-10-29 – 1909-10-30), \emph{Journalist/Journalistin, Herausgeber/Herausgeberin}|pw}.}}}\pend
           
\pstart
           \begin{otherlanguage}{french}\textcolor{gray}{\textbf{Journal politique, financier,}}\end{otherlanguage}\pend
           
\pstart
           \begin{otherlanguage}{french}\textcolor{gray}{\textbf{commercial et littéraire.}}\end{otherlanguage}\pend
           
\pstart
           \begin{otherlanguage}{french}\textcolor{gray}{\textbf{\textbf{Paraissant trois fois par jour.}}}\end{otherlanguage}\pend
           
\pstart
           \begin{otherlanguage}{french}\textcolor{gray}{\textbf{\textbf{Bureau à Paris\oindex{Paris@\textbf{Paris}, \emph{P.PPLC}|pw}}}}\end{otherlanguage}\hfill \textsc{Paris\oindex{Paris@\textbf{Paris}, \emph{P.PPLC}|pw}}, 27. Oktober.\pend
           
\pstart
           \begin{otherlanguage}{french}\textcolor{gray}{\textbf{\textbf{10 \so{Rue de la Bourse}\oindex{rue de la Bourse@\textbf{rue de la Bourse}, \emph{Straße (K.STR)}|pw}.}}}\end{otherlanguage}\pend
           \vspace{0.5em}
\pstart
           Bitte, liebſter Freund, laß’ doch endlich wieder
               einmal etwas von Dir hören. Wie gehts Dir? Wie gehts »ihr\pwindex{Reinhard, Marie 1871-03-13 – 1899-03-18@\textsc{Reinhard, Marie} (1871-03-13 – 1899-03-18), \emph{Gesangspädagoge/Gesangspädagogin}|pwv}«? Wie gehts den Freunden?\pend
           
\pstart
           Alles ſchweigt um mich herum, und ich bin ganz einſam.\pend
           
\pstart
           Ich ſende Dir einen amüſanten \label{K_L02830-1v}\edtext{Artikel\pwindex{?? [Artikel von Rochefort ueber Jesus]@\emph{?? [Artikel von Rochefort über Jesus]}|pwv} von \textsc{Rochefort\pwindex{Rochefort, Henri de 1830-01-31 – 1913-06-30@\textsc{Rochefort, Henri de} (1830-01-31 – 1913-06-30), \emph{Schriftsteller/Schriftstellerin, Politiker/Politikerin, Journalist/Journalistin}|pw}}}{\lemma{\textnormal{\emph{Artikel von Rochefort}}}\Cendnote{\textnormal{nicht ermittelt}}}\label{K_L02830-1}, welcher von
               unſerem Glaubensgenoſſen\pwindex{Jesus 7–4 v. u. Z. – 30/31@\textsc{Jesus} (7–4 v. u. Z. – 30/31), \emph{Wanderprediger/Wanderpredigerin}|pwv}
               handelt, der am Kreuz geſtorben iſt{\dots}\pend
           
\pstart
           \textsc{Thorel\pwindex{Thorel, Jean 1859-09-11 – 1916-08-20@\textsc{Thorel, Jean} (1859-09-11 – 1916-08-20), \emph{Übersetzer/Übersetzerin, Dramatiker/Dramatikerin}|pw}} ſprach ich. Er müht ſich, das Stück\pwindex{Amourette. Piece en trois actes. Adaptee de Arthur Schnitzler@\emph{Amourette. Pièce en trois actes. Adaptée de Arthur Schnitzler}|pwv} anzubringen (aber vielleicht bemüht er ſich nicht
                  genug?){[}.{]} Die Nachrichten ſind wenig günſtig. \textsc{Antoine\pwindex{Antoine, Andre 1858-01-31 – 1943-10-23@\textsc{Antoine, André} (1858-01-31 – 1943-10-23), \emph{Theaterleiter/Theaterleiterin, Schauspieler/Schauspielerin}|pw}} hat ſich die Antwort vorbehalten, ſcheint aber nicht ſehr geneigt zur \label{K_L02830-2v}\edtext{Aufführung\pwindex{Amourette. Piece en trois actes. Adaptee de Arthur Schnitzler@\emph{Amourette. Pièce en trois actes. Adaptée de Arthur Schnitzler}|pwv}}{\lemma{\textnormal{\emph{Aufführung}}}\Cendnote{\textnormal{Jean Thorel\pwindex{Thorel, Jean 1859-09-11 – 1916-08-20@\textsc{Thorel, Jean} (1859-09-11 – 1916-08-20), \emph{Übersetzer/Übersetzerin, Dramatiker/Dramatikerin}|pwk} versuchte (erfolglos) seine \emph{Liebelei}\pwindex{Liebelei. Schauspiel in drei Akten@\emph{Liebelei. Schauspiel in drei Akten}|pwk}-Übersetzung\pwindex{Amourette. Piece en trois actes. Adaptee de Arthur Schnitzler@\emph{Amourette. Pièce en trois actes. Adaptée de Arthur Schnitzler}|pwkv} dem \emph{Théâtre Antoine}\orgindex{Theâtre Antoine@Théâtre Antoine|pwk} (von André Antoine\pwindex{Antoine, Andre 1858-01-31 – 1943-10-23@\textsc{Antoine, André} (1858-01-31 – 1943-10-23), \emph{Theaterleiter/Theaterleiterin, Schauspieler/Schauspielerin}|pwk} geleitet) oder dem \emph{Odéon}\orgindex{Odeon@Odéon|pwk} zu vermitteln.}}}\label{K_L02830-2}.\pend
           
\pstart
           {\pb}Willſt Du Dich mit \textsc{Molière\pwindex{Moliere 14.01.1622 – 17.02.1673@\textsc{Molière} (14.01.1622 – 17.02.1673), \emph{Schriftsteller/Schriftstellerin, Theaterleiter/Theaterleiterin, Schauspieler/Schauspielerin}|pw}} ganz, aber ganz befreunden? \label{K_L02830-3v}\edtext{Lies
               ſeinen \textsc{Don Juan\pwindex{Dom Juan ou le Festin de pierre@\emph{Dom Juan ou le Festin de pierre}|pwv}}}{\lemma{\textnormal{\emph{Lies
               ſeinen Don Juan}}}\Cendnote{\textnormal{Lektüre nicht nachweisbar, jedoch sah
                     Schnitzler in späteren Jahren mehrere
                  Inszenierungen von Molières\pwindex{Moliere 14.01.1622 – 17.02.1673@\textsc{Molière} (14.01.1622 – 17.02.1673), \emph{Schriftsteller/Schriftstellerin, Theaterleiter/Theaterleiterin, Schauspieler/Schauspielerin}|pwk}{ }\emph{Don Juan}\pwindex{Dom Juan ou le Festin de pierre@\emph{Dom Juan ou le Festin de pierre}|pwk} (vgl. A. S.: \emph{Tagebuch}, 21. 10. 1915, 2. 2. 1916 und 27. 9. 1919).}}}\label{K_L02830-3}, von ihm genannt »\textsc{Le festin de Pierre\pwindex{Dom Juan ou le Festin de pierre@\emph{Dom Juan ou le Festin de pierre}|pw}}.«\pend
           
\pstart
           Ich weiß Dir nichts mehr zu ſchreiben, als daß ich namenloſes Heimweh habe nach Wien\oindex{Wien@\textbf{Wien}, \emph{A.ADM2}|pw}, nach Freundſchaſt, nach Heimlichkeit und
               Gemüthlichkeit. Von Liebe w\textcolor{gray}{i}ll ich nicht reden. So anſpruchsvoll
               bin ich ſchon längſt nicht mehr. Aber nicht mehr fremd ſein in der Fremde! {\dots}\pend
           
\pstart
           Grüß’ Dich Gott, liebſter Freund, und vergiß mich nicht gar ſo ſehr!\pend
           
\pstart
           Dein treuer {\\[\baselineskip]}\spacefill\mbox{Paul Goldmann}\pend
           \leftskip=0em{}
\pstart
           \noindent{}Deiner Freundin\pwindex{Reinhard, Marie 1871-03-13 – 1899-03-18@\textsc{Reinhard, Marie} (1871-03-13 – 1899-03-18), \emph{Gesangspädagoge/Gesangspädagogin}|pwv} viele
                  herzliche Grüße!\pend
           \selectlanguage{ngerman}\endnumbering\briefempfaengerindex{Schnitzler, Arthur@\textsc{Schnitzler, Arthur}!zzzGoldmann, Paul@\emph{von Paul Goldmann}!1897-10-273@{27. 10. {[}1897{]}}|)be}\mylabel{L02830h}  \normalsize

\doendnotes{C}
\bigskip
\vfill

\clearpage

\footnotesize

\lohead{\textsc{register}}

% Definiere theindex-Environment komplett neu ohne reledmac
\makeatletter
\renewenvironment{theindex}{%
  \section*{\indexname}%
  \setlength{\parindent}{0pt}%
  \setlength{\parskip}{0pt plus 0.3pt}%
  \let\item\@idxitem
}{%
  \clearpage
}
\makeatother

\IfFileExists{\jobname-pw.ind}{\input{\jobname-pw.ind}}{}

\end{document}

      