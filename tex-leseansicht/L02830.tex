%% latex-leseansicht-vorspann.tex
%% Vorspann für die Leseansicht.
%% Lädt die gemeinsame Datei latex-vorspann.tex mit nicht gesetztem Schalter.

\newif\ifkorrekturansicht
\korrekturansichtfalse

\input{../tex-inputs/latex-vorspann}


         
         \newcommand{\erwaehntePersonen}{Personen: André Antoine,  Jesus,  Molière, Marie Reinhard, Victor Henri de Rochefort, Leopold Sonnemann, Jean Thorel}
         \newcommand{\erwaehnteInstitutionen}{Institutionen: Frankfurter Zeitung, Odéon, Théâtre Antoine}
         \newcommand{\erwaehnteOrte}{Orte: Paris, Wien, rue de la Bourse}
         \newcommand{\erwaehnteWerke}{Werke: Amourette. Pièce en trois actes. Adaptée de Arthur Schnitzler, Dom Juan ou le Festin de pierre, Don Juan oder Der steinerne Gast, Liebelei. Schauspiel in drei Akten, [?? Artikel von Rochefort über Jesus]}
               \section[ Paul Goldmann an Arthur Schnitzler, 27. 10. {[}1897{]}]{ Paul Goldmann an Arthur Schnitzler, 27. 10. {[}1897{]}}\nopagebreak\mylabel{v}\rehead{ }\begin{ledgroupsized}[t]{13cm}\normalsize\beginnumbering \toendnotes[C]{\smallbreak\pagebreak[2]} \Standort{DLA, A:Schnitzler, HS.NZ85.1.3167.}
\physDesc{Brief, 1 Blatt, 2 Seiten
\newline{}Handschrift: blaue Tinte, deutsche Kurrent
\newline{}Schnitzler: 1) mit Bleistift das Jahr »97« vermerkt  2) mit rotem Buntstift drei Unterstreichungen}\toendnotes[C]{\smallbreak}\pstart
           \noindent{}{\pb}\textcolor{gray}{\textbf{\textbf{Frankfurter Zeitung\orgindex{Frankfurter Zeitung@Frankfurter Zeitung|pw}}}}\pend
           \pstart
           \textcolor{gray}{\textbf{(\begin{otherlanguage}{french}Gazette de Francfort\end{otherlanguage}\orgindex{Frankfurter Zeitung@Frankfurter Zeitung|pw}).}}\pend
           \pstart
           \textcolor{gray}{\textbf{\textbf{\begin{otherlanguage}{french}Fondateur M.\end{otherlanguage}{ }L. Sonnemann\pwindex{Sonnemann, Leopold 1831-10-29 – 1909-10-30@\textsc{Sonnemann, Leopold} (1831-10-29 – 1909-10-30), \emph{Journalist, Herausgeber}|pw}.}}}\pend
           \pstart
           \begin{otherlanguage}{french}\textcolor{gray}{\textbf{Journal politique, financier,}}\end{otherlanguage}\pend
           \pstart
           \begin{otherlanguage}{french}\textcolor{gray}{\textbf{commercial et littéraire.}}\end{otherlanguage}\pend
           \pstart
           \begin{otherlanguage}{french}\textcolor{gray}{\textbf{\textbf{Paraissant trois fois par jour.}}}\end{otherlanguage}\pend
           \pstart
           \begin{otherlanguage}{french}\textcolor{gray}{\textbf{\textbf{Bureau à Paris\oindex{Paris@\textbf{Paris}|pw}}}}\end{otherlanguage}\hfill \textsc{Paris\oindex{Paris@\textbf{Paris}|pw}}, 27. Oktober.\pend
           \pstart
           \begin{otherlanguage}{french}\textcolor{gray}{\textbf{\textbf{10 \so{Rue de la Bourse}\oindex{rue de la Bourse@\textbf{rue de la Bourse}|pw}.}}}\end{otherlanguage}\pend
           \pstart
           Bitte, liebſter Freund, laß’ doch endlich wieder
               einmal etwas von Dir hören. Wie gehts Dir? Wie gehts »ihr\pwindex{Reinhard, Marie 1871-03-13 – 1899-03-18@\textsc{Reinhard, Marie} (1871-03-13 – 1899-03-18), \emph{Gesangspädagogin}|pwv}«? Wie gehts den Freunden?\pend
           \pstart
           Alles ſchweigt um mich herum, und ich bin ganz einſam.\pend
           \pstart
           Ich ſende Dir einen amüſanten \label{K_L02830-1v}\edtext{Artikel\pwindex{Rochefort, Victor Henri de 1830-01-31 – 1913-06-30@\textsc{Rochefort, Victor Henri de} (1830-01-31 – 1913-06-30), \emph{Schriftsteller, Politiker, Politiker}!?? Artikel von Rochefort ueber Jesus]1897@\strich\emph{[?? Artikel von Rochefort über Jesus]} {[}1897{]}|pwv} von \textsc{Rochefort\pwindex{Rochefort, Victor Henri de 1830-01-31 – 1913-06-30@\textsc{Rochefort, Victor Henri de} (1830-01-31 – 1913-06-30), \emph{Schriftsteller, Politiker, Politiker}|pw}}}{\lemma{\textnormal{\emph{Artikel von Rochefort}}}\Cendnote{\textnormal{nicht ermittelt}}}\label{K_L02830-1h}, welcher von
               unſerem Glaubensgenoſſen\pwindex{Jesus 7–4 v. u. Z. – 30/31@\textsc{Jesus} (7–4 v. u. Z. – 30/31), \emph{Wanderprediger}|pwv}
               handelt, der am Kreuz geſtorben iſt{\dots}\pend
           \pstart
           \textsc{Thorel\pwindex{Thorel, Jean 1859-09-11 – 1916-08-20@\textsc{Thorel, Jean} (1859-09-11 – 1916-08-20), \emph{Übersetzer, Dramatiker}|pw}} ſprach ich. Er müht ſich, das Stück\pwindex{Thorel, Jean 1859-09-11 – 1916-08-20@\textsc{Thorel, Jean} (1859-09-11 – 1916-08-20), \emph{Übersetzer, Dramatiker}!Amourette. Piece en trois actes. Adaptee de Arthur Schnitzler1897@\strich\emph{Amourette. Pièce en trois actes. Adaptée de Arthur Schnitzler} {[}Übersetzung, 1897{]}|pwv} anzubringen (aber vielleicht bemüht er ſich nicht
                  genug?){[}.{]} Die Nachrichten ſind wenig günſtig. \textsc{Antoine\pwindex{Antoine, Andre 1858-01-31 – 1943-10-23@\textsc{Antoine, André} (1858-01-31 – 1943-10-23), \emph{Theaterleiter, Schauspieler}|pw}} hat ſich die Antwort vorbehalten, ſcheint aber nicht ſehr geneigt zur \label{K_L02830-2v}\edtext{Aufführung\pwindex{Thorel, Jean 1859-09-11 – 1916-08-20@\textsc{Thorel, Jean} (1859-09-11 – 1916-08-20), \emph{Übersetzer, Dramatiker}!Amourette. Piece en trois actes. Adaptee de Arthur Schnitzler1897@\strich\emph{Amourette. Pièce en trois actes. Adaptée de Arthur Schnitzler} {[}Übersetzung, 1897{]}|pwv}}{\lemma{\textnormal{\emph{Aufführung}}}\Cendnote{\textnormal{Jean Thorel\pwindex{Thorel, Jean 1859-09-11 – 1916-08-20@\textsc{Thorel, Jean} (1859-09-11 – 1916-08-20), \emph{Übersetzer, Dramatiker}|pwk} versuchte (erfolglos) seine \emph{Liebelei}\pwindex{Schnitzler, Arthur 15.05.1862 – 21.10.1931@\textsc{Schnitzler, Arthur} (15.05.1862 – 21.10.1931), \emph{Schriftsteller, Mediziner}!Liebelei. Schauspiel in drei Akten1895-10-09@\strich\emph{Liebelei. Schauspiel in drei Akten} {[}1895-10-09{]}|pwk}-Übersetzung\pwindex{Thorel, Jean 1859-09-11 – 1916-08-20@\textsc{Thorel, Jean} (1859-09-11 – 1916-08-20), \emph{Übersetzer, Dramatiker}!Amourette. Piece en trois actes. Adaptee de Arthur Schnitzler1897@\strich\emph{Amourette. Pièce en trois actes. Adaptée de Arthur Schnitzler} {[}Übersetzung, 1897{]}|pwkv} dem \emph{Théâtre Antoine}\orgindex{Theâtre Antoine@Théâtre Antoine|pwk} (von André Antoine\pwindex{Antoine, Andre 1858-01-31 – 1943-10-23@\textsc{Antoine, André} (1858-01-31 – 1943-10-23), \emph{Theaterleiter, Schauspieler}|pwk} geleitet) oder dem \emph{Odéon}\orgindex{Odeon@Odéon|pwk} zu vermitteln.}}}\label{K_L02830-2h}.\pend
           \pstart
           {\pb}Willſt Du Dich mit \textsc{Molière\pwindex{Moliere 14.01.1622 – 17.02.1673@\textsc{Molière} (14.01.1622 – 17.02.1673), \emph{Schriftsteller, Theaterleiter, Schauspieler}|pw}} ganz, aber ganz befreunden? \label{K_L02830-5v}\edtext{Lies
               ſeinen \textsc{Don Juan\pwindex{Moliere 14.01.1622 – 17.02.1673@\textsc{Molière} (14.01.1622 – 17.02.1673), \emph{Schriftsteller, Theaterleiter, Schauspieler}!Don Juan oder Der steinerne GastNone@\strich\emph{Don Juan oder Der steinerne Gast} {[}None{]}|pwv}}}{\lemma{\textnormal{\emph{Lies
               ſeinen Don Juan}}}\Cendnote{\textnormal{Lektüre nicht nachweisbar, jedoch sah
                     Schnitzler\pwindex{Schnitzler, Arthur 15.05.1862 – 21.10.1931@\textsc{Schnitzler, Arthur} (15.05.1862 – 21.10.1931), \emph{Schriftsteller, Mediziner}|pwk} in späteren Jahren mehrere
                  Inszenierungen von Molière\pwindex{Moliere 14.01.1622 – 17.02.1673@\textsc{Molière} (14.01.1622 – 17.02.1673), \emph{Schriftsteller, Theaterleiter, Schauspieler}|pwk}s \emph{Don Juan}\pwindex{Moliere 14.01.1622 – 17.02.1673@\textsc{Molière} (14.01.1622 – 17.02.1673), \emph{Schriftsteller, Theaterleiter, Schauspieler}!Don Juan oder Der steinerne GastNone@\strich\emph{Don Juan oder Der steinerne Gast} {[}None{]}|pwk} (vgl. A. S.: \emph{Tagebuch}, 21. 10. 1915, 2. 2. 1916 und 27. 9. 1919).}}}\label{K_L02830-5h}, von ihm genannt »\textsc{Le festin de Pierre\pwindex{Moliere 14.01.1622 – 17.02.1673@\textsc{Molière} (14.01.1622 – 17.02.1673), \emph{Schriftsteller, Theaterleiter, Schauspieler}!Dom Juan ou le Festin de pierre1682@\strich\emph{Dom Juan ou le Festin de pierre} {[}1682{]}|pw}}.«\pend
           \pstart
           Ich weiß Dir nichts mehr zu ſchreiben, als daß ich namenloſes Heimweh habe nach Wien\oindex{Wien@\textbf{Wien}|pw}, nach Freundſchaſt, nach Heimlichkeit und
               Gemüthlichkeit. Von Liebe w\textcolor{gray}{i}ll ich nicht reden. So anſpruchsvoll
               bin ich ſchon längſt nicht mehr. Aber nicht mehr fremd ſein in der Fremde! {\dots}\pend
           \pstart
           Grüß’ Dich Gott, liebſter Freund, und vergiß mich nicht gar ſo ſehr!\pend
           \pstart
           Dein treuer {\\[\baselineskip]}\spacefill\mbox{Paul Goldmann}\pend
           \leftskip=0em{}\pstart
           \noindent{}Deiner Freundin\pwindex{Reinhard, Marie 1871-03-13 – 1899-03-18@\textsc{Reinhard, Marie} (1871-03-13 – 1899-03-18), \emph{Gesangspädagogin}|pwv} viele
                  herzliche Grüße!\pend
           
         
         \endnumbering\mylabel{h}\end{ledgroupsized}  \newcommand{\dateiname}{L02830}\newcommand{\titel}{Paul Goldmann an Arthur Schnitzler, 27. 10. [1897]}\newcommand{\editorInnen}{Martin Anton Müller und Laura Untner}%% latex-leseansicht-abspann.tex
%% Abspann für die Leseansicht.
%% Der Schalter \ifkorrekturansicht ist bereits durch den Vorspann gesetzt.

%% latex-abspann.tex
%% Gemeinsamer Abspann für Korrekturansicht und Leseansicht.
%% Setzt den Schalter \ifkorrekturansicht voraus (gesetzt in den
%% einbindenden Dateien latex-korrekturansicht-abspann.tex bzw.
%% latex-leseansicht-abspann.tex).
%% ---------------------------------------------------------------

\normalsize

% Das esempio-Environment wird nur in der Leseansicht benötigt
\ifkorrekturansicht\else
\newenvironment{esempio}[3]%
{
    \vspace{1.5ex}
    \rlap{\underline{#1}}
    \par
    \setlength{\parindent}{0cm}
    \nopagebreak
    \leftskip=#2cm
    \rightskip=#3cm
}
{
    \par
}
\fi

\doendnotes{C}
\bigskip
\vfill

\clearpage

\footnotesize

\ifkorrekturansicht
  \lohead{\textsc{register}}
\fi

% theindex-Environment neu definieren ohne reledmac
\makeatletter
\renewenvironment{theindex}{%
  \ifkorrekturansicht
    \section*{\indexname}%
  \else
    \subsubsection*{Index der erwähnten Entitäten}%
  \fi
  \setlength{\parindent}{0pt}%
  \setlength{\parskip}{0pt plus 0.3pt}%
  \let\item\@idxitem
}{%
  \ifkorrekturansicht\clearpage\fi
}
\makeatother

\IfFileExists{\jobname-pw.ind}{\input{\jobname-pw.ind}}{}

% Quellenangabe nur in der Leseansicht
\ifkorrekturansicht\else
% Fallback-Definitionen, falls die .tex-Datei \titel etc. nicht gesetzt hat
\providecommand{\titel}{}
\providecommand{\editorInnen}{}
\providecommand{\dateiname}{\jobname}

\vspace{3cm}

\vfill

\footnotesize
\textsc{Quelle}: \titel. Herausgegeben von {\editorInnen}. In: \emph{Arthur Schnitzler: Briefwechsel mit Autorinnen und Autoren}.
 Digitale Edition, https://schnitzler-briefe.acdh.oeaw.ac.at/{\dateiname}.html (Stand \today)
\fi

\end{document}


      