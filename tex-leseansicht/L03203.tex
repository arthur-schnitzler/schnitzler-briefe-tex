%% latex-leseansicht-vorspann.tex
%% Vorspann für die Leseansicht.
%% Lädt die gemeinsame Datei latex-vorspann.tex mit nicht gesetztem Schalter.

\newif\ifkorrekturansicht
\korrekturansichtfalse

\input{../tex-inputs/latex-vorspann}


\section[ Paul Goldmann an Arthur Schnitzler, 1. 4. {[}1902{]}]{L03203 Paul Goldmann an Arthur Schnitzler,  1. 4. [1902]}
\nopagebreak\mylabel{L03203v}
\rehead{ }\normalsize\beginnumbering\briefempfaengerindex{Schnitzler, Arthur@\textsc{Schnitzler, Arthur}!zzzGoldmann, Paul@\emph{von Paul Goldmann}!1902-04-013@{1. 4. [1902]}|(be}
\toendnotes[C]{\smallbreak\pagebreak[2]}
\correspDesc{Versand  durch Paul Goldmann am 1. 4. [1902] in Prag
\newline{}Erhalt  durch Arthur Schnitzler im Zeitraum [2. 4. 1902
                  – 6. 4. 1902?] in Wien}\toendnotes[C]{\smallbreak}
\Standort{DLA, A:Schnitzler, HS.NZ85.1.3172.}
\physDesc{Brief, 1 Blatt, 2 Seiten, 1373 Zeichen
\newline{}Handschrift: schwarze Tinte, deutsche Kurrent
\newline{}Schnitzler: 1) mit Bleistift das Jahr »90\substVorne{}\textsuperscript{2}\substDazwischen{}2\substHinten{}« vermerkt  2) mit Bleistift auf der letzten Bogenseite weitgehend kryptisch
                                 bleibende Vermerke: »\noindent{}{\pb}\textcolor{gray}{L}\textcolor{gray}{×}\-\textcolor{gray}{×}\-\textcolor{gray}{×}\-\textcolor{gray}{×}a\textcolor{gray}{×}\-\textcolor{gray}{×}.{ / }\textcolor{gray}{Haus Hugo\pwindex{Hofmannsthal, Hugo von 1.\,2.\,1874 Wien – 15.\,7.\,1929 Rodaun@\textsc{Hofmannsthal, Hugo von} (1.\,2.\,1874 Wien – 15.\,7.\,1929 Rodaun), \emph{Schriftsteller}|pw}.}{ / }\textcolor{gray}{Ella\pwindex{Naschauer, Ella 6.\,11.\,1875 Wien – 17.\,12.\,1939 ebd.@\textsc{Naschauer, Ella} (6.\,11.\,1875 Wien – 17.\,12.\,1939 ebd.)|pwu}}{[}rn{]}.{ / }\textsc{Harden}\pwindex{Harden, Maximilian 20.\,10.\,1861 Berlin – 30.\,10.\,1927 Montana@\textsc{Harden, Maximilian} (20.\,10.\,1861 Berlin – 30.\,10.\,1927 Montana), \emph{Schriftsteller, Publizist}|pw}.{ / }\textsc{Feu\textcolor{gray}{i}ll}
                                          T\textcolor{gray}{itel}.« 3) mit rotem Buntstift vier Unterstreichungen}\toendnotes[C]{\smallbreak}
\pstart
           {\pb}\textcolor{gray}{\textbf{\textsc{HÔTEL BLAUER STERN\oindex{Hotel Blauer Stern@\textbf{Hotel Blauer Stern}, \emph{Hotel}|pw}}}}\pend
           
\pstart
           \textcolor{gray}{\textbf{\textsc{\textbf{CARL SELTMANN\pwindex{Seltmann, Carl @\textsc{Seltmann, Carl}, \emph{Hotelbesitzer}|pw}.}}}}\hfill \textcolor{gray}{\textbf{Prag\oindex{Prag@\textbf{Prag}, \emph{Land}|pw},}}{ }1. April.\pend
           
\pstart
           \textcolor{gray}{\textbf{TELEGRAMM-ADRESSE:}}\pend
           
\pstart
           \textcolor{gray}{\textbf{\textsc{Sternhôtel Prag\oindex{Hotel Blauer Stern@\textbf{Hotel Blauer Stern}, \emph{Hotel}|pw}.}}}\pend
           
\pstart{}Mein lieber Freund,\pend\vspace{0.5em}
\pstart
           Ich habe einige angenehme Tage verlebt in einer{ }ſchönen Stadt\oindex{Prag@\textbf{Prag}, \emph{Land}|pwv} mit lieben Menſchen. Morgen fahre ich wieder heim.\pend
           
\pstart
           Ich habe viel von Dir geſprochen. \textsc{Salus\pwindex{Salus, Hugo 3.\,8.\,1866 Česká Lípa – 4.\,2.\,1929 Prag@\textsc{Salus, Hugo} (3.\,8.\,1866 Česká Lípa – 4.\,2.\,1929 Prag), \emph{Schriftsteller, Mediziner}|pw}} (ein kluger und{ }ſympathiſcher Menſch unter einer Schicht von Affektirtheit)
               läßt Dich und \textsc{Richard\pwindex{Beer-Hofmann, Richard 11.\,7.\,1866 Wien – 26.\,9.\,1945 New York City@\textsc{Beer-Hofmann, Richard} (11.\,7.\,1866 Wien – 26.\,9.\,1945 New York City), \emph{Schriftsteller}|pw}} grüßen. Ebenſo \textsc{Teweles\pwindex{Teweles, Heinrich 13.\,11.\,1856 Prag – 9.\,8.\,1927 Prein an der Rax@\textsc{Teweles, Heinrich} (13.\,11.\,1856 Prag – 9.\,8.\,1927 Prein an der Rax), \emph{Schriftsteller, Journalist, Theaterleiter}|pw}} und \textsc{Bondy\pwindex{Bondy, Charlotte 25.\,3.\,1854 Bielsko-Biała – 7.\,3.\,1914 Prag@\textsc{Bondy, Charlotte} (25.\,3.\,1854 Bielsko-Biała – 7.\,3.\,1914 Prag), \emph{Schauspielerin}|pwv}\pwindex{Ziegler, Alice 5.\,1.\,1880 Prag – Dezember 1943 Konzentrationslager Auschwitz-Birkenau@\textsc{Ziegler, Alice} (5.\,1.\,1880 Prag – Dezember 1943 Konzentrationslager Auschwitz-Birkenau)|pwv}}, Mutter\pwindex{Bondy, Charlotte 25.\,3.\,1854 Bielsko-Biała – 7.\,3.\,1914 Prag@\textsc{Bondy, Charlotte} (25.\,3.\,1854 Bielsko-Biała – 7.\,3.\,1914 Prag), \emph{Schauspielerin}|pwv} und Tochter\pwindex{Ziegler, Alice 5.\,1.\,1880 Prag – Dezember 1943 Konzentrationslager Auschwitz-Birkenau@\textsc{Ziegler, Alice} (5.\,1.\,1880 Prag – Dezember 1943 Konzentrationslager Auschwitz-Birkenau)|pwv}.\pend
           
\pstart
           \textsc{Alice\pwindex{Ziegler, Alice 5.\,1.\,1880 Prag – Dezember 1943 Konzentrationslager Auschwitz-Birkenau@\textsc{Ziegler, Alice} (5.\,1.\,1880 Prag – Dezember 1943 Konzentrationslager Auschwitz-Birkenau)|pw}} iſt ein{ }ſchönes Mädchen geworden und auch geiſtig gereift. Ich war ein Thor
               ohnegleichen, daß ich{ }ſie \label{K_L03203-1v}\edtext{nicht
                  geheirathet}{\lemma{\textnormal{\emph{nicht
                  geheirathet}}}\Cendnote{\textnormal{Siehe XXXX Auszeichnungsfehler: Dokument L03193 nicht gefunden.
               }}}\label{K_L03203-1} habe. Sie wäre die Frau geweſen, wie ich{ }ſie mir immer ausgedacht habe. In
               der Kunſt, die Gelegenheiten zu verſäumen, iſt mir Keiner über. Sie hat{ }ſich als
               Bräutigam eine Art Kraftmenſch\pwindex{Ziegler, Arnost 6.\,12.\,1871 Polička – 2.\,1.\,1943 Terezín@\textsc{Ziegler, Arnost} (6.\,12.\,1871 Polička – 2.\,1.\,1943 Terezín), \emph{Bankdirektor}|pwv} ausgeſucht, der mir {\pb}ſehr
               unſympathiſch iſt. Aber es iſt ganz natürlich. \label{K_L03203-2v}\edtext{\begin{otherlanguage}{french}\textsc{Très-femelle}\end{otherlanguage}}{\lemma{\textnormal{\emph{Très-femelle}}}\Cendnote{\textnormal{französisch: sehr weiblich}}}\label{K_L03203-2}, wie{ }ſie iſt, hat ihr Inſtinkt \strikeout{ſ\textcolor{gray}{e}\textcolor{gray}{×}}{ }ſie zu dem Gegenpol\pwindex{Ziegler, Arnost 6.\,12.\,1871 Polička – 2.\,1.\,1943 Terezín@\textsc{Ziegler, Arnost} (6.\,12.\,1871 Polička – 2.\,1.\,1943 Terezín), \emph{Bankdirektor}|pwv}{ }\label{K_L03203-3v}\edtext{\begin{otherlanguage}{french}\textsc{très-mâle}\end{otherlanguage}}{\lemma{\textnormal{\emph{très-mâle}}}\Cendnote{\textnormal{französisch: sehr männlich}}}\label{K_L03203-3}
               geleitet.\pend
           
\pstart
           Von der »Neuen Freien Preſſe\pwindex{Neue Freie Presse@\emph{Neue Freie Presse}|pw}« höre ich hier{ }ſo
               viel Schlechtes und von der »Zeit\pwindex{Zeit@\emph{Die Zeit}|pw}«{ }ſo viel
               Gutes, daß ich in{ }ſchweren \label{K_L03203-4v}\edtext{Sorgen}{\lemma{\textnormal{\emph{Sorgen}}}\Cendnote{\textnormal{womöglich Bezug auf Goldmanns\pwindex{Goldmann, Paul 31.\,1.\,1865 Breslau – 25.\,9.\,1935 Wien@\textsc{Goldmann, Paul} (31.\,1.\,1865 Breslau – 25.\,9.\,1935 Wien), \emph{Schriftsteller, Journalist}|pwk} Angst, die \emph{Zeit}\pwindex{Zeit@\emph{Die Zeit}|pwk} könnte die \emph{Neue Freie Presse}\pwindex{Neue Freie Presse@\emph{Neue Freie Presse}|pwk}
                  ablösen, siehe XXXX Auszeichnungsfehler: Dokument L03072 nicht gefunden.}}}\label{K_L03203-4} heimfahre.\pend
           
\pstart
           Wie geht es Dir, mein lieber Freund? Es thut mir unendlich leid, daß ich Dir nicht
               habe die Hand drücken können. Die Leute{ }ſprechen hier\oindex{Prag@\textbf{Prag}, \emph{Land}|pwv} nicht nur mit Liebe von deinem Talent,{ }ſondern auch mit
               Reſpekt von Deinem (künſtleriſchen und moraliſchen) Charakter\strikeout{)}.\pend
           
\pstart
           Schreib’ mir nach Berlin\oindex{Berlin@\textbf{Berlin}, \emph{Hauptstadt}|pw}. Was macht \textsc{Olga\pwindex{Schnitzler, Olga 17.\,1.\,1882 Wien – 13.\,1.\,1970 Lugano@\textsc{Schnitzler, Olga} (17.\,1.\,1882 Wien – 13.\,1.\,1970 Lugano), \emph{Schauspielerin, Sängerin}|pw}}? Grüße{ }ſie vielmals.\pend
           
\pstart
           In den \label{K_L03203-5v}\edtext{Böhmerwald\oindex{Böhmerwald@\textbf{Böhmerwald}, \emph{Gebirge}|pw}}{\lemma{\textnormal{\emph{Böhmerwald}}}\Cendnote{\textnormal{Es dürfte sich um eine geplante Reise
                  gehandelt haben, die auch nach Berlin\oindex{Berlin@\textbf{Berlin}, \emph{Hauptstadt}|pwk} hätte
                  führen sollen, die aber nicht stattfand.}}}\label{K_L03203-5} werde ich mit Euch\pwindex{Schnitzler, Olga 17.\,1.\,1882 Wien – 13.\,1.\,1970 Lugano@\textsc{Schnitzler, Olga} (17.\,1.\,1882 Wien – 13.\,1.\,1970 Lugano), \emph{Schauspielerin, Sängerin}|pwv} leider nicht gehen
               können. Aber ich rechne{ }ſicher darauf, Euch\pwindex{Schnitzler, Olga 17.\,1.\,1882 Wien – 13.\,1.\,1970 Lugano@\textsc{Schnitzler, Olga} (17.\,1.\,1882 Wien – 13.\,1.\,1970 Lugano), \emph{Schauspielerin, Sängerin}|pwv}{ }\strikeout{\textcolor{gray}{an}} in \label{K_L03203-6v}\edtext{Berlin\oindex{Berlin@\textbf{Berlin}, \emph{Hauptstadt}|pw}}{\lemma{\textnormal{\emph{Berlin}}}\Cendnote{\textnormal{In Berlin\oindex{Berlin@\textbf{Berlin}, \emph{Hauptstadt}|pwk} sahen sich Goldmann\pwindex{Goldmann, Paul 31.\,1.\,1865 Breslau – 25.\,9.\,1935 Wien@\textsc{Goldmann, Paul} (31.\,1.\,1865 Breslau – 25.\,9.\,1935 Wien), \emph{Schriftsteller, Journalist}|pwk} und Schnitzler erst zwischen 13. 10. 1902 und 18. 10. 1902 wieder.
                  Davor begegneten sie sich bei Goldmanns\pwindex{Goldmann, Paul 31.\,1.\,1865 Breslau – 25.\,9.\,1935 Wien@\textsc{Goldmann, Paul} (31.\,1.\,1865 Breslau – 25.\,9.\,1935 Wien), \emph{Schriftsteller, Journalist}|pwk} Aufenthalt in Wien\oindex{Wien@\textbf{Wien}, \emph{Verwaltungsgebiet}|pwk} bzw. in der Brühl\oindex{Brühl@\textbf{Brühl}, \emph{Tal}|pwk} vom 18. 5. 1902 bis jedenfalls zum 25. 5. 1902.}}}\label{K_L03203-6} zu{ }ſehen.\pend
           
\pstart
           Viele treue Grüße! Dein {\\[\baselineskip]}\spacefill\mbox{Paul Goldmann}\pend
           \leftskip=0em{}\selectlanguage{ngerman}\endnumbering\briefempfaengerindex{Schnitzler, Arthur@\textsc{Schnitzler, Arthur}!zzzGoldmann, Paul@\emph{von Paul Goldmann}!1902-04-013@{1. 4. [1902]}|)be}\mylabel{L03203h}  \newcommand{\dateiname}{L03203}\newcommand{\titel}{Paul Goldmann an Arthur Schnitzler, 1. 4. [1902]}\newcommand{\editorInnen}{Martin Anton Müller und Laura Untner}%% latex-leseansicht-abspann.tex
%% Abspann für die Leseansicht.
%% Der Schalter \ifkorrekturansicht ist bereits durch den Vorspann gesetzt.

%% latex-abspann.tex
%% Gemeinsamer Abspann für Korrekturansicht und Leseansicht.
%% Setzt den Schalter \ifkorrekturansicht voraus (gesetzt in den
%% einbindenden Dateien latex-korrekturansicht-abspann.tex bzw.
%% latex-leseansicht-abspann.tex).
%% ---------------------------------------------------------------

\normalsize

% Das esempio-Environment wird nur in der Leseansicht benötigt
\ifkorrekturansicht\else
\newenvironment{esempio}[3]%
{
    \vspace{1.5ex}
    \rlap{\underline{#1}}
    \par
    \setlength{\parindent}{0cm}
    \nopagebreak
    \leftskip=#2cm
    \rightskip=#3cm
}
{
    \par
}
\fi

\doendnotes{C}
\bigskip
\vfill

\clearpage

\footnotesize

\ifkorrekturansicht
  \lohead{\textsc{register}}
\fi

% theindex-Environment neu definieren ohne reledmac
\makeatletter
\renewenvironment{theindex}{%
  \ifkorrekturansicht
    \section*{\indexname}%
  \else
    \subsubsection*{Index der erwähnten Entitäten}%
  \fi
  \setlength{\parindent}{0pt}%
  \setlength{\parskip}{0pt plus 0.3pt}%
  \let\item\@idxitem
}{%
  \ifkorrekturansicht\clearpage\fi
}
\makeatother

\IfFileExists{\jobname-pw.ind}{\input{\jobname-pw.ind}}{}

% Quellenangabe nur in der Leseansicht
\ifkorrekturansicht\else
% Fallback-Definitionen, falls die .tex-Datei \titel etc. nicht gesetzt hat
\providecommand{\titel}{}
\providecommand{\editorInnen}{}
\providecommand{\dateiname}{\jobname}

\vspace{3cm}

\vfill

\footnotesize
\textsc{Quelle}: \titel. Herausgegeben von {\editorInnen}. In: \emph{Arthur Schnitzler: Briefwechsel mit Autorinnen und Autoren}.
 Digitale Edition, https://schnitzler-briefe.acdh.oeaw.ac.at/{\dateiname}.html (Stand \today)
\fi

\end{document}


