%% latex-korrekturansicht-vorspann.tex
%% Vorspann für die Korrekturansicht.
%% Lädt die gemeinsame Datei latex-vorspann.tex mit gesetztem Schalter.

\newif\ifkorrekturansicht
\korrekturansichttrue

\input{../tex-inputs/latex-vorspann}


\section[ Paul Goldmann an Arthur Schnitzler, 1. 4. {[}1902{]}]{L03203 Paul Goldmann an Arthur Schnitzler, 1. 4. {[}1902{]}}
\nopagebreak\mylabel{L03203v}
\rehead{ }\normalsize\beginnumbering\briefempfaengerindex{Schnitzler, Arthur@\textsc{Schnitzler, Arthur}!zzzGoldmann, Paul@\emph{von Paul Goldmann}!1902-04-013@{1. 4. {[}1902{]}}|(be}
\toendnotes[C]{\smallbreak\pagebreak[2]}\Standort{DLA, A:Schnitzler, HS.NZ85.1.3172.}
\physDesc{Brief, 1 Blatt, 2 Seiten, 1373 Zeichen
\newline{}Handschrift: schwarze Tinte, deutsche Kurrent
\newline{}Schnitzler: 1) mit Bleistift das Jahr »90\substVorne{}\textsuperscript{2}\substDazwischen{}2\substHinten{}« vermerkt  2) mit Bleistift auf der letzten Bogenseite weitgehend kryptisch
                                 bleibende Vermerke: »\noindent{}{\pb}\textcolor{gray}{L}\textcolor{gray}{×}\-\textcolor{gray}{×}\-\textcolor{gray}{×}\-\textcolor{gray}{×}a\textcolor{gray}{×}\-\textcolor{gray}{×}.{ / }\textcolor{gray}{Haus Hugo\pwindex{Hofmannsthal, Hugo von 1874-02-01 – 1929-07-15@\textsc{Hofmannsthal, Hugo von} (1874-02-01 – 1929-07-15), \emph{Schriftsteller/Schriftstellerin}|pw}.}{ / }\textcolor{gray}{Ella\pwindex{Naschauer, Ella 06.11.1875 – 17.12.1939@\textsc{Naschauer, Ella} (06.11.1875 – 17.12.1939)|pwu}}{[}rn{]}.{ / }\textsc{Harden}\pwindex{Harden, Maximilian 20.10.1861 – 30.10.1927@\textsc{Harden, Maximilian} (20.10.1861 – 30.10.1927), \emph{Schriftsteller/Schriftstellerin, Publizist/Publizistin}|pw}.{ / }\textsc{Feu\textcolor{gray}{i}ll}
                                          T\textcolor{gray}{itel}.« 3) mit rotem Buntstift vier Unterstreichungen}\toendnotes[C]{\smallbreak}
\pstart
           {\pb}\textcolor{gray}{\textbf{\textsc{HÔTEL BLAUER STERN\oindex{Hotel Blauer Stern@\textbf{Hotel Blauer Stern}, \emph{Hotel (K.HTL)}|pw}}}}\pend
           
\pstart
           \textcolor{gray}{\textbf{\textsc{\textbf{CARL SELTMANN\pwindex{Seltmann, Carl @\textsc{Seltmann, Carl}, \emph{Hotelbesitzer/Hotelbesitzerin}|pw}.}}}}\hfill \textcolor{gray}{\textbf{Prag\oindex{Prag@\textbf{Prag}, \emph{A.ADM1}|pw},}}{ }1. April.\pend
           
\pstart
           \textcolor{gray}{\textbf{TELEGRAMM-ADRESSE:}}\pend
           
\pstart
           \textcolor{gray}{\textbf{\textsc{Sternhôtel Prag\oindex{Hotel Blauer Stern@\textbf{Hotel Blauer Stern}, \emph{Hotel (K.HTL)}|pw}.}}}\pend
           
\pstart{}Mein lieber Freund,\pend\vspace{0.5em}
\pstart
           Ich habe einige angenehme Tage verlebt in einer ſchönen Stadt\oindex{Prag@\textbf{Prag}, \emph{A.ADM1}|pwv} mit lieben Menſchen. Morgen fahre ich wieder heim.\pend
           
\pstart
           Ich habe viel von Dir geſprochen. \textsc{Salus\pwindex{Salus, Hugo 03.08.1866 – 04.02.1929@\textsc{Salus, Hugo} (03.08.1866 – 04.02.1929), \emph{Schriftsteller/Schriftstellerin, Mediziner/Medizinerin}|pw}} (ein kluger und ſympathiſcher Menſch unter einer Schicht von Affektirtheit)
               läßt Dich und \textsc{Richard\pwindex{Beer-Hofmann, Richard 1866-07-11 – 1945-09-26@\textsc{Beer-Hofmann, Richard} (1866-07-11 – 1945-09-26), \emph{Schriftsteller/Schriftstellerin}|pw}} grüßen. Ebenſo \textsc{Teweles\pwindex{Teweles, Heinrich 13.11.1856 – 09.08.1927@\textsc{Teweles, Heinrich} (13.11.1856 – 09.08.1927), \emph{Schriftsteller/Schriftstellerin, Journalist/Journalistin, Theaterleiter/Theaterleiterin}|pw}} und \textsc{Bondy\pwindex{Bondy, Charlotte 25.03.1854 – 1914-03-07@\textsc{Bondy, Charlotte} (25.03.1854 – 1914-03-07), \emph{Schauspieler/Schauspielerin}|pwv}\pwindex{Ziegler, Alice 1880-01-05 – Dezember 1943@\textsc{Ziegler, Alice} (1880-01-05 – Dezember 1943)|pwv}}, Mutter\pwindex{Bondy, Charlotte 25.03.1854 – 1914-03-07@\textsc{Bondy, Charlotte} (25.03.1854 – 1914-03-07), \emph{Schauspieler/Schauspielerin}|pwv} und Tochter\pwindex{Ziegler, Alice 1880-01-05 – Dezember 1943@\textsc{Ziegler, Alice} (1880-01-05 – Dezember 1943)|pwv}.\pend
           
\pstart
           \textsc{Alice\pwindex{Ziegler, Alice 1880-01-05 – Dezember 1943@\textsc{Ziegler, Alice} (1880-01-05 – Dezember 1943)|pw}} iſt ein ſchönes Mädchen geworden und auch geiſtig gereift. Ich war ein Thor
               ohnegleichen, daß ich ſie \label{K_L03203-1v}\edtext{nicht
                  geheirathet}{\lemma{\textnormal{\emph{nicht
                  geheirathet}}}\Cendnote{\textnormal{Siehe Paul Goldmann an Arthur Schnitzler, 16. 1. [1902].
               }}}\label{K_L03203-1} habe. Sie wäre die Frau geweſen, wie ich ſie mir immer ausgedacht habe. In
               der Kunſt, die Gelegenheiten zu verſäumen, iſt mir Keiner über. Sie hat ſich als
               Bräutigam eine Art Kraftmenſch\pwindex{Ziegler, Arnost 1871-12-06 – 1943-01-02@\textsc{Ziegler, Arnost} (1871-12-06 – 1943-01-02), \emph{Bankdirektor/Bankdirektorin}|pwv} ausgeſucht, der mir {\pb}ſehr
               unſympathiſch iſt. Aber es iſt ganz natürlich. \label{K_L03203-2v}\edtext{\begin{otherlanguage}{french}\textsc{Très-femelle}\end{otherlanguage}}{\lemma{\textnormal{\emph{Très-femelle}}}\Cendnote{\textnormal{französisch: sehr weiblich}}}\label{K_L03203-2}, wie
               ſie iſt, hat ihr Inſtinkt \strikeout{ſ\textcolor{gray}{e}\textcolor{gray}{×}} ſie zu dem Gegenpol\pwindex{Ziegler, Arnost 1871-12-06 – 1943-01-02@\textsc{Ziegler, Arnost} (1871-12-06 – 1943-01-02), \emph{Bankdirektor/Bankdirektorin}|pwv}{ }\label{K_L03203-3v}\edtext{\begin{otherlanguage}{french}\textsc{très-mâle}\end{otherlanguage}}{\lemma{\textnormal{\emph{très-mâle}}}\Cendnote{\textnormal{französisch: sehr männlich}}}\label{K_L03203-3}
               geleitet.\pend
           
\pstart
           Von der »Neuen Freien Preſſe\pwindex{Neue Freie Presse@\emph{Neue Freie Presse}|pw}« höre ich hier ſo
               viel Schlechtes und von der »Zeit\pwindex{Zeit@\emph{Die Zeit}|pw}« ſo viel
               Gutes, daß ich in ſchweren \label{K_L03203-4v}\edtext{Sorgen}{\lemma{\textnormal{\emph{Sorgen}}}\Cendnote{\textnormal{womöglich Bezug auf Goldmanns\pwindex{Goldmann, Paul 31.01.1865 – 25.09.1935@\textsc{Goldmann, Paul} (31.01.1865 – 25.09.1935), \emph{Schriftsteller/Schriftstellerin, Journalist/Journalistin}|pwk} Angst, die \emph{Zeit}\pwindex{Zeit@\emph{Die Zeit}|pwk} könnte die \emph{Neue Freie Presse}\pwindex{Neue Freie Presse@\emph{Neue Freie Presse}|pwk}
                  ablösen, siehe Paul Goldmann an Arthur Schnitzler und Olga
               Gussmann, 7. 7. [1901].}}}\label{K_L03203-4} heimfahre.\pend
           
\pstart
           Wie geht es Dir, mein lieber Freund? Es thut mir unendlich leid, daß ich Dir nicht
               habe die Hand drücken können. Die Leute ſprechen hier\oindex{Prag@\textbf{Prag}, \emph{A.ADM1}|pwv} nicht nur mit Liebe von deinem Talent, ſondern auch mit
               Reſpekt von Deinem (künſtleriſchen und moraliſchen) Charakter\strikeout{)}.\pend
           
\pstart
           Schreib’ mir nach Berlin\oindex{Berlin@\textbf{Berlin}, \emph{P.PPLC}|pw}. Was macht \textsc{Olga\pwindex{Schnitzler, Olga 17.01.1882 – 13.01.1970@\textsc{Schnitzler, Olga} (17.01.1882 – 13.01.1970), \emph{Schauspieler/Schauspielerin, Sänger/Sängerin}|pw}}? Grüße ſie vielmals.\pend
           
\pstart
           In den \label{K_L03203-5v}\edtext{Böhmerwald\oindex{Boehmerwald@\textbf{Böhmerwald}, \emph{T.MTS}|pw}}{\lemma{\textnormal{\emph{Böhmerwald}}}\Cendnote{\textnormal{Es dürfte sich um eine geplante Reise
                  gehandelt haben, die auch nach Berlin\oindex{Berlin@\textbf{Berlin}, \emph{P.PPLC}|pwk} hätte
                  führen sollen, die aber nicht stattfand.}}}\label{K_L03203-5} werde ich mit Euch\pwindex{Schnitzler, Olga 17.01.1882 – 13.01.1970@\textsc{Schnitzler, Olga} (17.01.1882 – 13.01.1970), \emph{Schauspieler/Schauspielerin, Sänger/Sängerin}|pwv} leider nicht gehen
               können. Aber ich rechne ſicher darauf, Euch\pwindex{Schnitzler, Olga 17.01.1882 – 13.01.1970@\textsc{Schnitzler, Olga} (17.01.1882 – 13.01.1970), \emph{Schauspieler/Schauspielerin, Sänger/Sängerin}|pwv}{ }\strikeout{\textcolor{gray}{an}} in \label{K_L03203-6v}\edtext{Berlin\oindex{Berlin@\textbf{Berlin}, \emph{P.PPLC}|pw}}{\lemma{\textnormal{\emph{Berlin}}}\Cendnote{\textnormal{In Berlin\oindex{Berlin@\textbf{Berlin}, \emph{P.PPLC}|pwk} sahen sich Goldmann\pwindex{Goldmann, Paul 31.01.1865 – 25.09.1935@\textsc{Goldmann, Paul} (31.01.1865 – 25.09.1935), \emph{Schriftsteller/Schriftstellerin, Journalist/Journalistin}|pwk} und Schnitzler erst zwischen 13. 10. 1902 und 18. 10. 1902 wieder.
                  Davor begegneten sie sich bei Goldmanns\pwindex{Goldmann, Paul 31.01.1865 – 25.09.1935@\textsc{Goldmann, Paul} (31.01.1865 – 25.09.1935), \emph{Schriftsteller/Schriftstellerin, Journalist/Journalistin}|pwk} Aufenthalt in Wien\oindex{Wien@\textbf{Wien}, \emph{A.ADM2}|pwk} bzw. in der Brühl\oindex{Bruehl@\textbf{Brühl}, \emph{Tal (N.TAL)}|pwk} vom 18. 5. 1902 bis jedenfalls zum 25. 5. 1902.}}}\label{K_L03203-6} zu
               ſehen.\pend
           
\pstart
           Viele treue Grüße! Dein {\\[\baselineskip]}\spacefill\mbox{Paul Goldmann}\pend
           \leftskip=0em{}\selectlanguage{ngerman}\endnumbering\briefempfaengerindex{Schnitzler, Arthur@\textsc{Schnitzler, Arthur}!zzzGoldmann, Paul@\emph{von Paul Goldmann}!1902-04-013@{1. 4. {[}1902{]}}|)be}\mylabel{L03203h}  \normalsize

\doendnotes{C}
\bigskip
\vfill

\clearpage

\footnotesize

\lohead{\textsc{register}}

% Definiere theindex-Environment komplett neu ohne reledmac
\makeatletter
\renewenvironment{theindex}{%
  \section*{\indexname}%
  \setlength{\parindent}{0pt}%
  \setlength{\parskip}{0pt plus 0.3pt}%
  \let\item\@idxitem
}{%
  \clearpage
}
\makeatother

\IfFileExists{\jobname-pw.ind}{\input{\jobname-pw.ind}}{}

\end{document}

      