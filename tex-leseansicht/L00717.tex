%% latex-leseansicht-vorspann.tex
%% Vorspann für die Leseansicht.
%% Lädt die gemeinsame Datei latex-vorspann.tex mit nicht gesetztem Schalter.

\newif\ifkorrekturansicht
\korrekturansichtfalse

\input{../tex-inputs/latex-vorspann}


         
         \newcommand{\erwaehntePersonen}{Personen: Richard Beer-Hofmann, Paul Goldmann, Leopold Petter}
         \newcommand{\erwaehnteOrte}{Orte: Bad Ischl, Eglmoosgasse, Salzburg, Wien, Österreichischer Hof}
         \newcommand{\erwaehnteWerke}{
               \section[Arthur Schnitzler an Richard Beer-Hofmann, 17. 8. 1897]{ Arthur Schnitzler an Richard Beer-Hofmann, 17. 8. 1897}\nopagebreak\mylabel{v}\rehead{ }\begin{ledgroupsized}[t]{13cm}\normalsize\beginnumbering \toendnotes[C]{\smallbreak\pagebreak[2]} \Standort{YCGL, MSS 31.}
\physDesc{Brief, 1 Blatt, 1 Seite, Umschlag
\newline{}Handschrift: Bleistift, deutsche Kurrent\newline{}Versand: 1) Stempel: »\nobreak{}Wien, 17 8 97, 11–12 N\nobreak{}«.   2) Stempel: »\nobreak{}\oindex{Salzburg@\textbf{Salzburg}|pwk}Salzburg Stadt, 18 8 97, 11–F\nobreak{}«.  3) Stempel: »\nobreak{}\oindex{Salzburg@\textbf{Salzburg}|pwk}Salzburg Stadt, 18 8 97, 1–F\nobreak{}«.  4) Stempel: »\nobreak{}\oindex{Bad Ischl@\textbf{Bad Ischl}|pwk}Ischl, 18. 8. 97, 7–8 N\nobreak{}«.  5) die drei Adresszeilen
                                    durchgestrichen und darunter von unbekannter Hand mit Bleistift: »\noindent{}Ischl, Eglmoos 22\oindex{Eglmoosgasse@\textbf{Eglmoosgasse}|pw}.«}\pstart{}{\pb}Herrn Dr. \textsc{Richard Beer-Hofmann}\pend{}\pstart{}\textsc{Salzburg}\oindex{Salzburg@\textbf{Salzburg}|pw}\pend{}\pstart{}\textsc{Hotel Oesterreichischer Hof\oindex{Oesterreichischer Hof@\textbf{Österreichischer Hof}|pw}.}\pend{}{\bigskip}\pstart
           \raggedleft{}{\pb}Dinſtag\pend
           \pstart
           Lieber Richard. Do{\geminationn}erſtag{ }Abend
                    oder Freitg{ }früh bin ich in Iſchl\oindex{Bad Ischl@\textbf{Bad Ischl}|pw}. Das Zi{\geminationm}er für Paul\pwindex{Goldmann, Paul 31.01.1865 – 25.09.1935@\textsc{Goldmann, Paul} (31.01.1865 – 25.09.1935), \emph{Schriftsteller, Journalist}|pw}
                    bei Petter\pwindex{Petter, Leopold 17.11.1850 – 03.07.1917@\textsc{Petter, Leopold} (17.11.1850 – 03.07.1917), \emph{Hotelier}|pw} beſtellt. Trifft Sie dieſer Brief
                    überhaupt noch in Salzburg\oindex{Salzburg@\textbf{Salzburg}|pw}? –\pend
           \pstart
           Grüßen Sie Paul\pwindex{Goldmann, Paul 31.01.1865 – 25.09.1935@\textsc{Goldmann, Paul} (31.01.1865 – 25.09.1935), \emph{Schriftsteller, Journalist}|pw} herzlich; auch ſich
                        ſelbſt.\pend
           \pstart Ihr \spacefill\mbox{Arthur}\pend{}
         
         \endnumbering\mylabel{h}\end{ledgroupsized}  \newcommand{\dateiname}{L00717}\newcommand{\titel}{Arthur Schnitzler an Richard Beer-Hofmann, 17. 8. 1897}\newcommand{\editorInnen}{ Martin Anton Müller und Gerd-Hermann Susen}%% latex-leseansicht-abspann.tex
%% Abspann für die Leseansicht.
%% Der Schalter \ifkorrekturansicht ist bereits durch den Vorspann gesetzt.

%% latex-abspann.tex
%% Gemeinsamer Abspann für Korrekturansicht und Leseansicht.
%% Setzt den Schalter \ifkorrekturansicht voraus (gesetzt in den
%% einbindenden Dateien latex-korrekturansicht-abspann.tex bzw.
%% latex-leseansicht-abspann.tex).
%% ---------------------------------------------------------------

\normalsize

% Das esempio-Environment wird nur in der Leseansicht benötigt
\ifkorrekturansicht\else
\newenvironment{esempio}[3]%
{
    \vspace{1.5ex}
    \rlap{\underline{#1}}
    \par
    \setlength{\parindent}{0cm}
    \nopagebreak
    \leftskip=#2cm
    \rightskip=#3cm
}
{
    \par
}
\fi

\doendnotes{C}
\bigskip
\vfill

\clearpage

\footnotesize

\ifkorrekturansicht
  \lohead{\textsc{register}}
\fi

% theindex-Environment neu definieren ohne reledmac
\makeatletter
\renewenvironment{theindex}{%
  \ifkorrekturansicht
    \section*{\indexname}%
  \else
    \subsubsection*{Index der erwähnten Entitäten}%
  \fi
  \setlength{\parindent}{0pt}%
  \setlength{\parskip}{0pt plus 0.3pt}%
  \let\item\@idxitem
}{%
  \ifkorrekturansicht\clearpage\fi
}
\makeatother

\IfFileExists{\jobname-pw.ind}{\input{\jobname-pw.ind}}{}

% Quellenangabe nur in der Leseansicht
\ifkorrekturansicht\else
% Fallback-Definitionen, falls die .tex-Datei \titel etc. nicht gesetzt hat
\providecommand{\titel}{}
\providecommand{\editorInnen}{}
\providecommand{\dateiname}{\jobname}

\vspace{3cm}

\vfill

\footnotesize
\textsc{Quelle}: \titel. Herausgegeben von {\editorInnen}. In: \emph{Arthur Schnitzler: Briefwechsel mit Autorinnen und Autoren}.
 Digitale Edition, https://schnitzler-briefe.acdh.oeaw.ac.at/{\dateiname}.html (Stand \today)
\fi

\end{document}


      