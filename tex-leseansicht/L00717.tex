%% latex-leseansicht-vorspann.tex
%% Vorspann für die Leseansicht.
%% Lädt die gemeinsame Datei latex-vorspann.tex mit nicht gesetztem Schalter.

\newif\ifkorrekturansicht
\korrekturansichtfalse

\input{../tex-inputs/latex-vorspann}


\section[Arthur Schnitzler an Richard Beer-Hofmann, 17. 8. 1897]{L00717 Arthur Schnitzler an Richard Beer-Hofmann, 17. 8. 1897}
\nopagebreak\mylabel{L00717v}
\rehead{ }\normalsize\beginnumbering\briefempfaengerindex{Beer-Hofmann, Richard@\textsc{Beer-Hofmann, Richard}!zzzSchnitzler, Arthur@\emph{von Arthur Schnitzler}!1897-08-171@{17. 8. 1897}|(be}
\toendnotes[C]{\smallbreak\pagebreak[2]}
\correspDesc{Versand  durch Arthur Schnitzler am 17. 8. 1897 in Wien
\newline{}Weiterleitung  am 18. 8. 1897 in Salzburg
\newline{}Erhalt  durch Richard Beer-Hofmann am 18. 8. 1897 in Bad Ischl}\toendnotes[C]{\smallbreak}
\Standort{YCGL, MSS 31.}
\physDesc{Brief, 1 Blatt, 1 Seite, Kuvert, 273 Zeichen
\newline{}Handschrift: Bleistift, deutsche Kurrent
\newline{}Versand: 1) Stempel: »\nobreak{}\oindex{Wien@\textbf{Wien}, \emph{Verwaltungsgebiet}|pwk}Wien, 17 8 97, 11–12 N\nobreak{}«.   2) Stempel: »\nobreak{}\oindex{Salzburg@\textbf{Salzburg}, \emph{Verwaltungsgebiet}|pwk}Salzburg Stadt, 18 8 97, 11–F\nobreak{}«.  3) Stempel: »\nobreak{}\oindex{Salzburg@\textbf{Salzburg}, \emph{Verwaltungsgebiet}|pwk}Salzburg Stadt, 18 8 97, 1–F\nobreak{}«.  4) Stempel: »\nobreak{}\oindex{Bad Ischl@\textbf{Bad Ischl}|pwk}Ischl, 18. 8. 97, 7–8 N\nobreak{}«.  5) die drei Adresszeilen durchgestrichen und darunter von
                                 unbekannter Hand mit Bleistift: »\noindent{}Ischl, Eglmoos 22\oindex{Eglmoosgasse@\textbf{Eglmoosgasse}, \emph{Bezirk}|pw}.«}\pstart{}{\pb}Herrn Dr. \textsc{Richard
                     Beer-Hofmann}\pend{}\pstart{}\textsc{Salzburg}\oindex{Salzburg@\textbf{Salzburg}, \emph{Verwaltungsgebiet}|pw}\pend{}\pstart{}\textsc{Hotel Oesterreichischer Hof\oindex{Österreichischer Hof@\textbf{Österreichischer Hof}, \emph{Hotel}|pw}.}\pend{}{\bigskip}\vspace{1em}
\pstart
           \raggedleft{}{\pb}Dinſtag\pend
           \vspace{0.5em}
\pstart
           Lieber Richard.{ }Do{\geminationn}erſtag{ }Abend oder Freitg{ }früh bin ich in Iſchl\oindex{Bad Ischl@\textbf{Bad Ischl}|pw}. Das Zi{\geminationm}er für Paul\pwindex{Goldmann, Paul 31.\,1.\,1865 Breslau – 25.\,9.\,1935 Wien@\textsc{Goldmann, Paul} (31.\,1.\,1865 Breslau – 25.\,9.\,1935 Wien), \emph{Schriftsteller, Journalist}|pw} bei
                  Petter\pwindex{Petter, Leopold 17.\,11.\,1850 Bad Ischl – 3.\,7.\,1917 ebd.@\textsc{Petter, Leopold} (17.\,11.\,1850 Bad Ischl – 3.\,7.\,1917 ebd.), \emph{Hotelier}|pw} beſtellt. Trifft Sie dieſer Brief
               überhaupt noch in Salzburg\oindex{Salzburg@\textbf{Salzburg}, \emph{Verwaltungsgebiet}|pw}? –\pend
           
\pstart
           Grüßen Sie Paul\pwindex{Goldmann, Paul 31.\,1.\,1865 Breslau – 25.\,9.\,1935 Wien@\textsc{Goldmann, Paul} (31.\,1.\,1865 Breslau – 25.\,9.\,1935 Wien), \emph{Schriftsteller, Journalist}|pw} herzlich; auch{ }ſich{ }ſelbſt.\pend
           \pstart Ihr \spacefill\mbox{Arthur}\pend{}\selectlanguage{ngerman}\endnumbering\briefempfaengerindex{Beer-Hofmann, Richard@\textsc{Beer-Hofmann, Richard}!zzzSchnitzler, Arthur@\emph{von Arthur Schnitzler}!1897-08-171@{17. 8. 1897}|)be}\mylabel{L00717h}  \newcommand{\dateiname}{L00717}\newcommand{\titel}{Arthur Schnitzler an Richard Beer-Hofmann, 17. 8. 1897}\newcommand{\editorInnen}{Herausgegeben von Martin Anton Müller}%% latex-leseansicht-abspann.tex
%% Abspann für die Leseansicht.
%% Der Schalter \ifkorrekturansicht ist bereits durch den Vorspann gesetzt.

%% latex-abspann.tex
%% Gemeinsamer Abspann für Korrekturansicht und Leseansicht.
%% Setzt den Schalter \ifkorrekturansicht voraus (gesetzt in den
%% einbindenden Dateien latex-korrekturansicht-abspann.tex bzw.
%% latex-leseansicht-abspann.tex).
%% ---------------------------------------------------------------

\normalsize

% Das esempio-Environment wird nur in der Leseansicht benötigt
\ifkorrekturansicht\else
\newenvironment{esempio}[3]%
{
    \vspace{1.5ex}
    \rlap{\underline{#1}}
    \par
    \setlength{\parindent}{0cm}
    \nopagebreak
    \leftskip=#2cm
    \rightskip=#3cm
}
{
    \par
}
\fi

\doendnotes{C}
\bigskip
\vfill

\clearpage

\footnotesize

\ifkorrekturansicht
  \lohead{\textsc{register}}
\fi

% theindex-Environment neu definieren ohne reledmac
\makeatletter
\renewenvironment{theindex}{%
  \ifkorrekturansicht
    \section*{\indexname}%
  \else
    \subsubsection*{Index der erwähnten Entitäten}%
  \fi
  \setlength{\parindent}{0pt}%
  \setlength{\parskip}{0pt plus 0.3pt}%
  \let\item\@idxitem
}{%
  \ifkorrekturansicht\clearpage\fi
}
\makeatother

\IfFileExists{\jobname-pw.ind}{\input{\jobname-pw.ind}}{}

% Quellenangabe nur in der Leseansicht
\ifkorrekturansicht\else
% Fallback-Definitionen, falls die .tex-Datei \titel etc. nicht gesetzt hat
\providecommand{\titel}{}
\providecommand{\editorInnen}{}
\providecommand{\dateiname}{\jobname}

\vspace{3cm}

\vfill

\footnotesize
\textsc{Quelle}: \titel. Herausgegeben von {\editorInnen}. In: \emph{Arthur Schnitzler: Briefwechsel mit Autorinnen und Autoren}.
 Digitale Edition, https://schnitzler-briefe.acdh.oeaw.ac.at/{\dateiname}.html (Stand \today)
\fi

\end{document}


