%% latex-korrekturansicht-vorspann.tex
%% Vorspann für die Korrekturansicht.
%% Lädt die gemeinsame Datei latex-vorspann.tex mit gesetztem Schalter.

\newif\ifkorrekturansicht
\korrekturansichttrue

\input{../tex-inputs/latex-vorspann}


\section[Hermann Bahr: Widmungsexemplar Theater. Roman für Arthur Schnitzler, {[}nach dem 20. 3. 1897{]}]{L00655 Hermann Bahr: Widmungsexemplar Theater. Roman für Arthur Schnitzler,
               {[}nach dem 20. 3. 1897{]}}
\nopagebreak\mylabel{L00655v}
\rehead{ }\normalsize\beginnumbering\briefempfaengerindex{Schnitzler, Arthur@\textsc{Schnitzler, Arthur}!zzzBahr, Hermann@\emph{von Hermann Bahr}!1897-03-311@{{[}nach dem
                  20. 3. 1897{]}}|(be}
\toendnotes[C]{\smallbreak\pagebreak[2]}\Standort{DLA, G:Schnitzler, Arthur (Sammlung Heinrich Schnitzler).}
\physDesc{Widmung am Schmutztitel, 44 Zeichen
\newline{}Handschrift: schwarze Tinte, deutsche Kurrent
\newline{}Ordnung: bei der Enteignung des Exemplars 1938 von
                                 unbekannter Hand mit Bleistift ergänzte Informationen: »\noindent{}1. Ausgabe 0{ / }handſchriftliche Widm. d. Verf.« }
\buchAbdrucke{\weitereDrucke{Hermann Bahr, Arthur Schnitzler: \emph{Briefwechsel, Aufzeichnungen, Dokumente (1891–1931)}. Göttingen: \emph{Wallstein} 2018, S. 139.} }
\pstart
           \noindent{}{\pb}Meinem lieben Arthur\pend
           \pstart \spacefill\mbox{Hermann Bahr}\pend{}
\pstart
           \noindent{}März 1897\pend
           {\vspace{1\baselineskip}}
\pstart
           \centering{}\textcolor{gray}{\textbf{Theater\pwindex{Theater. Ein Wiener Roman@\emph{Theater. Ein Wiener Roman}|pw}.}}\pend
           \selectlanguage{ngerman}\vspace{1em}{\vspace{1\baselineskip}}
\pstart
           \centering{}{\pb}\textcolor{gray}{\textbf{HERMANN BAHR.}}\pend
           
\pstart
           \centering{}\textcolor{gray}{\textbf{Theater\pwindex{Theater. Ein Wiener Roman@\emph{Theater. Ein Wiener Roman}|pw}.}}\pend
           
\pstart
           \centering{}\textcolor{gray}{\textbf{\textbf{Ein Wien\oindex{Wien@\textbf{Wien}, \emph{A.ADM2}|pw}er Roman.}}}\pend
           {\vspace{1\baselineskip}}
\pstart
           \centering{}\textcolor{gray}{\textbf{BERLIN\oindex{Berlin@\textbf{Berlin}, \emph{P.PPLC}|pw}}}\pend
           
\pstart
           \centering{}\textcolor{gray}{\textbf{\so{S. Fiſcher, Verlag}\orgindex{S. Fischer Verlag@S. Fischer Verlag|pw}}}\pend
           
\pstart
           \centering{}\textcolor{gray}{\textbf{1897.}}\pend
           \selectlanguage{ngerman}\endnumbering\briefempfaengerindex{Schnitzler, Arthur@\textsc{Schnitzler, Arthur}!zzzBahr, Hermann@\emph{von Hermann Bahr}!1897-03-201@{{[}nach dem
                  20. 3. 1897{]}}|)be}\mylabel{L00655h}  \normalsize

\doendnotes{C}
\bigskip
\vfill

\clearpage

\footnotesize

\lohead{\textsc{register}}

% Definiere theindex-Environment komplett neu ohne reledmac
\makeatletter
\renewenvironment{theindex}{%
  \section*{\indexname}%
  \setlength{\parindent}{0pt}%
  \setlength{\parskip}{0pt plus 0.3pt}%
  \let\item\@idxitem
}{%
  \clearpage
}
\makeatother

\IfFileExists{\jobname-pw.ind}{\input{\jobname-pw.ind}}{}

\end{document}

      