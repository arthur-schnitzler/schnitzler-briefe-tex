%% latex-leseansicht-vorspann.tex
%% Vorspann für die Leseansicht.
%% Lädt die gemeinsame Datei latex-vorspann.tex mit nicht gesetztem Schalter.

\newif\ifkorrekturansicht
\korrekturansichtfalse

\input{../tex-inputs/latex-vorspann}


         
         \renewcommand{\erwaehntePersonen}{Personen: Alfred Gerasch, Albert Heine, Hugo von Hofmannsthal, Josef Kainz, Georg Reimers}
         \renewcommand{\erwaehnteInstitutionen}{Institutionen: Burgtheater, Kaiserlich-Königliches Hof-Burgtheater}
         \renewcommand{\erwaehnteOrte}{Orte: Frankfurt am Main, Hotel Marienbad, München, Wien}
         \renewcommand{\erwaehnteWerke}{Werke: Das weite Land. Tragikomödie in fünf Akten, Der Thor und der Tod, Liebelei. Oper in drei Akten, Liebelei. Schauspiel in drei Akten}
               \section[Hugo von Hofmannsthal an Arthur Schnitzler, 30. 9. {[}1910{]}]{ Hugo von Hofmannsthal an Arthur Schnitzler, 30. 9. {[}1910{]}}\nopagebreak\mylabel{v}\rehead{ }\begin{ledgroupsized}[t]{13cm}\normalsize\beginnumbering\briefempfaengerindex{Schnitzler, Arthur@\textsc{Schnitzler, Arthur}!zzzHofmannsthal, Hugo von@\emph{von Hugo von Hofmannsthal}!1910-09-302@{30. 9. {[}1910{]}}|(be} \toendnotes[C]{\smallbreak\pagebreak[2]} \Standort{CUL, Schnitzler, B 43.}
\physDesc{Brief, 1 Blatt, 4 Seiten, 986 Zeichen
\newline{}Handschrift: schwarze Tinte, deutsche Kurrent
\newline{}Schnitzler: mit Bleistift die Jahreszahl ergänzt: »1910« und beschriftet: »\textsc{Hugo}« 
\newline{}Ordnung: 1) mit Bleistift von unbekannter Hand nummeriert:
                                    »315«  2) mit Bleistift von unbekannter Hand nummeriert:
                                    »322«}\buchAbdrucke{\weitereDrucke{Hugo von Hofmannsthal, Arthur Schnitzler: \emph{Briefwechsel}. Hg. Therese Nickl und Heinrich Schnitzler. Frankfurt am Main: \emph{S. Fischer} 1964, S. 253.} }\toendnotes[C]{\smallbreak}\pstart
           \noindent{}{\pb}30 IX.\hfill München, Hotel Marienbad\oindex{Hotel Marienbad@\textbf{Hotel Marienbad}|pw}\pend
           \pstart
           mein lieber, wenn Ihnen auch wie mir, inliegender \label{K_L01961-1v}\edtext{Beſetzungsvorſchlag}{\lemma{\textnormal{\emph{Beſetzungsvorſchlag}}}\Cendnote{\textnormal{Es handelt sich um die Trauerfeier für Kainz\pwindex{Kainz, Josef 02.01.1858 – 20.09.1910@\textsc{Kainz, Josef} (02.01.1858 – 20.09.1910), \emph{Schauspieler}|pwk}, die am 23. 10. 1910
                  stattfinden sollte und bei der – neben anderem – der \emph{Der Tor und der Tod}\pwindex{Hofmannsthal, Hugo von 1874-02-01 – 1929-07-15@\textsc{Hofmannsthal, Hugo von} (1874-02-01 – 1929-07-15), \emph{Schriftsteller}!Thor und der Tod1893@\strich\emph{Der Thor und der Tod} {[}1893{]}|pwk} gegeben werden sollte. Gerasch\pwindex{Gerasch, Alfred 17.08.1877 – 12.08.1955@\textsc{Gerasch, Alfred} (17.08.1877 – 12.08.1955), \emph{Schauspieler}|pwk} bekam die ihm hier zugedachte Rolle,
                  die Rolle des Tod\pwindex{Hofmannsthal, Hugo von 1874-02-01 – 1929-07-15@\textsc{Hofmannsthal, Hugo von} (1874-02-01 – 1929-07-15), \emph{Schriftsteller}!Thor und der Tod1893@\strich\emph{Der Thor und der Tod} {[}1893{]}|pwkv}s sollte
                     Albert Heine\pwindex{Heine, Albert 16.11.1867 – 13.4.1949@\textsc{Heine, Albert} (16.11.1867 – 13.4.1949), \emph{Theaterleiter, Schauspieler}|pwk} spielen.}}}\label{K_L01961-1h} absurd
               erſcheint und die Beſetzung \textsc{Claudio\pwindex{Hofmannsthal, Hugo von 1874-02-01 – 1929-07-15@\textsc{Hofmannsthal, Hugo von} (1874-02-01 – 1929-07-15), \emph{Schriftsteller}!Thor und der Tod1893@\strich\emph{Der Thor und der Tod} {[}1893{]}|pwv} – Gerasch\pwindex{Gerasch, Alfred 17.08.1877 – 12.08.1955@\textsc{Gerasch, Alfred} (17.08.1877 – 12.08.1955), \emph{Schauspieler}|pw}} / \textsc{Tod\pwindex{Hofmannsthal, Hugo von 1874-02-01 – 1929-07-15@\textsc{Hofmannsthal, Hugo von} (1874-02-01 – 1929-07-15), \emph{Schriftsteller}!Thor und der Tod1893@\strich\emph{Der Thor und der Tod} {[}1893{]}|pwv} – Reimers\pwindex{Reimers, Georg 04.04.1860 – 15.04.1936@\textsc{Reimers, Georg} (04.04.1860 – 15.04.1936), \emph{Schauspieler}|pw}} als die richtigere, ſo tun Sie mir den großen Gefallen und bringen dieſe meine
               und Ihre Auffassung bei \textsc{Berger}{ }{\pb}\textsc{telephonisch} in meinem Namen unter Berufung auf dieſen
               Brief vor.\pend
           \pstart
           Ich finde den Gedanken, \textsc{Tressler} eine geiſtige Geſtalt
               agieren zu ſehen, ſcheußlich und möchte das Ganze faſt lieber inhibieren, ſcheue aber
               dann wieder den {\pb}überflüſſigen
               Rummel. O ekelhaftes Wien\oindex{Wien@\textbf{Wien}|pw}! ekelhafteres Burgtheater\orgindex{Burgtheater@Burgtheater|pw}! ekelhaft wenn es einen nicht ſpielt
               und noch fühlbar ekelhafter, wenn es Miene macht, einen zu ſpielen! (Gilt für mich,
               und nicht für Sie). Bitte depeſchieren Sie mir {\pb}hieher was Sie getan oder nicht
               getan haben.\pend
           \pstart
           Freute mich ſehr über den ſo \label{K_L01961-2v}\edtext{ſtarken
                  Erfolg}{\lemma{\textnormal{\emph{ſtarken
                  Erfolg}}}\Cendnote{\textnormal{Diese war am
                     15. 9. 1910 im \emph{Burgtheater}\orgindex{Kaiserlich-Koenigliches Hof-Burgtheater@Kaiserlich-Königliches Hof-Burgtheater|pwk}
                  neuerlich inszeniert worden. Schnitzler\pwindex{Schnitzler, Arthur 15.05.1862 – 21.10.1931@\textsc{Schnitzler, Arthur} (15.05.1862 – 21.10.1931), \emph{Schriftsteller, Mediziner}|pwk}
                  weilte zu der Zeit in Frankfurt am Main\oindex{Frankfurt am Main@\textbf{Frankfurt am Main}|pwk}, um
                  der Uraufführung der Opernfassung\pwindex{Schnitzler, Arthur 15.05.1862 – 21.10.1931@\textsc{Schnitzler, Arthur} (15.05.1862 – 21.10.1931), \emph{Schriftsteller, Mediziner}!Liebelei. Oper in drei Akten1909-09-18@\strich\emph{Liebelei. Oper in drei Akten} {[}1909-09-18{]}|pwkv} am 18. 9. 1910 beizuwohnen.}}}\label{K_L01961-2h} der braven
               alten »Liebelei\pwindex{Schnitzler, Arthur 15.05.1862 – 21.10.1931@\textsc{Schnitzler, Arthur} (15.05.1862 – 21.10.1931), \emph{Schriftsteller, Mediziner}!Liebelei. Schauspiel in drei Akten1895-10-09@\strich\emph{Liebelei. Schauspiel in drei Akten} {[}1895-10-09{]}|pw}«.\hspace*{1.5em}Wenn Sie ein überflüſſiges Exemplar vom »Weiten Land\pwindex{Schnitzler, Arthur 15.05.1862 – 21.10.1931@\textsc{Schnitzler, Arthur} (15.05.1862 – 21.10.1931), \emph{Schriftsteller, Mediziner}!weite Land. Tragikomoedie in fuenf Akten1910-10-20@\strich\emph{Das weite Land. Tragikomödie in fünf Akten} {[}1910-10-20{]}|pw}« haben, ſo trifft es mich von
                  Dienſtag an auf \textsc{Schloss Neubeuern am Inn}
               und macht mir große Freude.\pend
           \pstart Ihr \spacefill\mbox{Hugo.}\pend{}
         
         \endnumbering\mylabel{h}\end{ledgroupsized}  \newcommand{\dateiname}{L01961}\newcommand{\titel}{Hugo von Hofmannsthal an Arthur Schnitzler, 30. 9. [1910]}\newcommand{\editorInnen}{Martin Anton Müller und Gerd-Hermann Susen}%% latex-leseansicht-abspann.tex
%% Abspann für die Leseansicht.
%% Der Schalter \ifkorrekturansicht ist bereits durch den Vorspann gesetzt.

%% latex-abspann.tex
%% Gemeinsamer Abspann für Korrekturansicht und Leseansicht.
%% Setzt den Schalter \ifkorrekturansicht voraus (gesetzt in den
%% einbindenden Dateien latex-korrekturansicht-abspann.tex bzw.
%% latex-leseansicht-abspann.tex).
%% ---------------------------------------------------------------

\normalsize

% Das esempio-Environment wird nur in der Leseansicht benötigt
\ifkorrekturansicht\else
\newenvironment{esempio}[3]%
{
    \vspace{1.5ex}
    \rlap{\underline{#1}}
    \par
    \setlength{\parindent}{0cm}
    \nopagebreak
    \leftskip=#2cm
    \rightskip=#3cm
}
{
    \par
}
\fi

\doendnotes{C}
\bigskip
\vfill

\clearpage

\footnotesize

\ifkorrekturansicht
  \lohead{\textsc{register}}
\fi

% theindex-Environment neu definieren ohne reledmac
\makeatletter
\renewenvironment{theindex}{%
  \ifkorrekturansicht
    \section*{\indexname}%
  \else
    \subsubsection*{Index der erwähnten Entitäten}%
  \fi
  \setlength{\parindent}{0pt}%
  \setlength{\parskip}{0pt plus 0.3pt}%
  \let\item\@idxitem
}{%
  \ifkorrekturansicht\clearpage\fi
}
\makeatother

\IfFileExists{\jobname-pw.ind}{\input{\jobname-pw.ind}}{}

% Quellenangabe nur in der Leseansicht
\ifkorrekturansicht\else
% Fallback-Definitionen, falls die .tex-Datei \titel etc. nicht gesetzt hat
\providecommand{\titel}{}
\providecommand{\editorInnen}{}
\providecommand{\dateiname}{\jobname}

\vspace{3cm}

\vfill

\footnotesize
\textsc{Quelle}: \titel. Herausgegeben von {\editorInnen}. In: \emph{Arthur Schnitzler: Briefwechsel mit Autorinnen und Autoren}.
 Digitale Edition, https://schnitzler-briefe.acdh.oeaw.ac.at/{\dateiname}.html (Stand \today)
\fi

\end{document}


      