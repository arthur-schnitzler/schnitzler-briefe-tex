%% latex-leseansicht-vorspann.tex
%% Vorspann für die Leseansicht.
%% Lädt die gemeinsame Datei latex-vorspann.tex mit nicht gesetztem Schalter.

\newif\ifkorrekturansicht
\korrekturansichtfalse

\input{../tex-inputs/latex-vorspann}


\section[Richard Beer-Hofmann an Arthur Schnitzler, Mitte August 1905]{L01542 Richard Beer-Hofmann an Arthur Schnitzler, Mitte August 1905}
\nopagebreak\mylabel{L01542v}
\rehead{ }\normalsize\beginnumbering\briefempfaengerindex{Schnitzler, Arthur@\textsc{Schnitzler, Arthur}!zzzBeer-Hofmann, Richard@\emph{von Richard Beer-Hofmann}!1905-08-151@{Mitte August 1905}|(be}
\toendnotes[C]{\smallbreak\pagebreak[2]}
\correspDesc{Versand  durch Richard Beer-Hofmann am Mitte August 1905 in Fulpmes
\newline{}Erhalt  durch Arthur Schnitzler im Zeitraum [16. 8. 1905
                  – 20. 8. 1905?] in Wien}\toendnotes[C]{\smallbreak}
\Standort{CUL, Schnitzler, B 8.}
\physDesc{Brief, 1 Blatt, 4 Seiten, 1554 Zeichen
\newline{}Handschrift: schwarze Tinte, lateinische Kurrent
\newline{}Schnitzler: mit Bleistift datiert: »Mitte Auguſt 905« 
\newline{}Ordnung: mit Bleistift von unbekannter Hand nummeriert:
                                    »204« }
\buchAbdrucke{\weitereDrucke{Arthur Schnitzler, Richard Beer-Hofmann: \emph{Briefwechsel 1891–1931}. Herausgegeben von Konstanze Fliedl. Wien, Zürich: \emph{Europaverlag} 1992, S. 174–175.} }\toendnotes[C]{\smallbreak}
\pstart
           \noindent{}\centering{}{\pb}\textcolor{gray}{\textbf{GRAND HOTEL STUBAI}}\oindex{Grand Hotel Stubai@\textbf{Grand Hotel Stubai}, \emph{Hotel}|pw}\pend
           
\pstart
           \centering{}\textcolor{gray}{\textbf{FULPMES\oindex{Fulpmes@\textbf{Fulpmes}, \emph{Hauptstadt}|pw}}}{ }\textsc{\textcolor{gray}{\textbf{bei innsbruck\oindex{Innsbruck@\textbf{Innsbruck}, \emph{Verwaltungsgebiet}|pw} (tirol\oindex{Tirol@\textbf{Tirol}, \emph{Land}|pw})}}}\pend
           
\pstart
           \textcolor{gray}{\textbf{TELEGRAMM-ADRESSE:}}\pend
           
\pstart
           \textcolor{gray}{\textbf{STUBAIHOTEL FULPMES-INNSBRUCK\oindex{Grand Hotel Stubai@\textbf{Grand Hotel Stubai}, \emph{Hotel}|pw}}}\pend
           
\pstart
           \textcolor{gray}{\textbf{ENDSTATION DER ELEKTRISCHEN BERGBAHN}}\pend
           
\pstart
           \textcolor{gray}{\textbf{INNSBRUCK-FULPMES}}\pend
           
\pstart
           Lieber Arthur! Wir sind da \label{T_L01542-1v}\edtext{oben}{\lemma{\textnormal{\emph{oben}}}\Cendnote{\textnormal{Ein Pfeil weist
                  auf die Hoteladresse.}}}\label{T_L01542-1}. In Kärnten\oindex{Kärnten@\textbf{Kärnten}, \emph{Land}|pw} fand
               ich keine Unterkunft. Dort – wie im Pusterthal\oindex{Pustertal@\textbf{Pustertal}, \emph{Tal}|pw}
               alles furchtbar überfüllt, so dass ich froh war hier unterzukommen. Das Hôtel ist
                  \strikeout{es} erst voriges Jahr eröffnet worden, noch nicht
               sehr bekannt und daher halbleer.\pend
           
\pstart
           Wassermanns\pwindex{Wassermann, Jakob 10.\,3.\,1873 Fürth – 1.\,1.\,1934 Altaussee@\textsc{Wassermann, Jakob} (10.\,3.\,1873 Fürth – 1.\,1.\,1934 Altaussee), \emph{Schriftsteller}|pw}\pwindex{Wassermann, Julie 5.\,12.\,1876 Wien – April 1963 Zürich@\textsc{Wassermann, Julie} (5.\,12.\,1876 Wien – April 1963 Zürich), \emph{Schriftstellerin}|pw}, S. Fischer\pwindex{Fischer, Samuel 24.\,12.\,1859 Liptovský Mikuláš – 15.\,10.\,1934 Berlin@\textsc{Fischer, Samuel} (24.\,12.\,1859 Liptovský Mikuláš – 15.\,10.\,1934 Berlin), \emph{Verleger}|pw}, Bella
                  Wengerow\pwindex{Vengerova, Isabella 1.\,3.\,1877 Minsk – 7.\,2.\,1956 New York City@\textsc{Vengerova, Isabella} (1.\,3.\,1877 Minsk – 7.\,2.\,1956 New York City), \emph{Musikpädagogin, Pianistin}|pw}, Schwester\pwindex{Vengerova, Zinaida A. 18.\,4.\,1867 Suomenlinna – 29.\,6.\,1941 New York City@\textsc{Vengerova, Zinaida A.} (18.\,4.\,1867 Suomenlinna – 29.\,6.\,1941 New York City), \emph{Kritikerin, Übersetzerin}|pwv},
                  Mutter\pwindex{Wengeroff, Pauline 1.\,1.\,1833 Babruysk – 1916 Minsk@\textsc{Wengeroff, Pauline} (1.\,1.\,1833 Babruysk – 1916 Minsk), \emph{Schriftstellerin}|pwv}, u. D\textsuperscript{r}{ }Kaufmann\pwindex{Kaufmann, Arthur 4.\,4.\,1872 Iași – 25.\,7.\,1938 Wien@\textsc{Kaufmann, Arthur} (4.\,4.\,1872 Iași – 25.\,7.\,1938 Wien), \emph{Rechtswissenschaftler, Privatgelehrte, Privatier}|pw}{ }{\pb}sind hier. Paula\pwindex{Beer-Hofmann, Paula 25.\,2.\,1879 Wien – 30.\,10.\,1939 Zürich@\textsc{Beer-Hofmann, Paula} (25.\,2.\,1879 Wien – 30.\,10.\,1939 Zürich)|pw} hat kein Behagen an den kühlen Abenden und auch sonst an
               der Gegend – der Wald ist für sie – augenblicklich – zu weit vom Hôtel. Ich will also
               am 21 oder 22 von hier weg, und über Bozen\oindex{Bozen@\textbf{Bozen}, \emph{Hauptstadt}|pw}, eventuell Gardasee\oindex{Lago di Garda@\textbf{Lago di Garda}, \emph{See}|pw},
               an den Lido\oindex{Lido@\textbf{Lido}|pw}. Hoffentlich tut ihr der Aufenthalt
               dort gut. Sie ist \uline{sehr} blutleer, und hat recht
               miserable Nerven. Das Stück\pwindex{Bahr, Hermann 19.\,7.\,1863 Linz – 15.\,1.\,1934 München@\textsc{Bahr, Hermann} (19.\,7.\,1863 Linz – 15.\,1.\,1934 München), \emph{Schriftsteller, Kritiker}!Andere@\strich\emph{Die Andere}|pwv}
               von Bahr\pwindex{Bahr, Hermann 19.\,7.\,1863 Linz – 15.\,1.\,1934 München@\textsc{Bahr, Hermann} (19.\,7.\,1863 Linz – 15.\,1.\,1934 München), \emph{Schriftsteller, Kritiker}|pw} blieb in Rodaun\oindex{Wien@\textbf{Wien}!XXIII., Liesing@\textbf{XXIII., Liesing}!Rodaun@\textbf{Rodaun}, \emph{Region}|pw} liegen, weil in folge der Aufschrift {\pb}»Eisenstein\orgindex{J. Eisenstein und Co.@J. Eisenstein {\kaufmannsund}  Co.|pw}« nur Bücher darin vermutet wurden, mit denen es nicht eilig sei;
               ich lasse es mir heute nachschicken.\pend
           
\pstart
           Bitte sind Sie so gut und fügen Sie auf beiliegendem Brief die Adresse hinzu. Wer »A«
               sagt – –!\pend
           
\pstart
           Hier hat sich das Gerücht verbreitet, Sie hätten dem Hugo\pwindex{Hofmannsthal, Hugo von 1.\,2.\,1874 Wien – 15.\,7.\,1929 Rodaun@\textsc{Hofmannsthal, Hugo von} (1.\,2.\,1874 Wien – 15.\,7.\,1929 Rodaun), \emph{Schriftsteller}|pw} zwei wunderschöne Stücke\pwindex{Schnitzler, Arthur 15.\,5.\,1862 Wien – 21.\,10.\,1931 ebd.@\textsc{Schnitzler, Arthur} (15.\,5.\,1862 Wien – 21.\,10.\,1931 ebd.), \emph{Schriftsteller, Mediziner}!Zwischenspiel. Komödie in drei Akten@\strich\emph{Zwischenspiel. Komödie in drei Akten}|pwv}\pwindex{Schnitzler, Arthur 15.\,5.\,1862 Wien – 21.\,10.\,1931 ebd.@\textsc{Schnitzler, Arthur} (15.\,5.\,1862 Wien – 21.\,10.\,1931 ebd.), \emph{Schriftsteller, Mediziner}!Ruf des Lebens. Schauspiel in drei Akten@\strich\emph{Der Ruf des Lebens. Schauspiel in drei Akten}|pwv}{ }\label{K_L01542-1v}\edtext{vorgelesen}{\lemma{\textnormal{\emph{vorgelesen}}}\Cendnote{\textnormal{Vgl. A. S.: \emph{Tagebuch}, 12. 8. 1905.
               }}}\label{K_L01542-1}. Ich freue mich sehr im Oktober mehr davon zu erfahren.\pend
           
\pstart
           Von mir will ich nichts schreiben, ich ziehe es vor Ihnen mündlich vorzujammern –
               obgleich {\pb}Sie mir dann bei
               physischen Dingen versichern werden Sie hätten dies Alles seit Jahren.\pend
           
\pstart
           Schreiben Sie mir, bitte, i{\geminationm}er wo Sie sind – ich will es
               auch tun. Die Möglichkeit soll uns doch bleiben, uns etwas zu sagen.\pend
           
\pstart
           Viele Grüsse an Ihre Frau\pwindex{Schnitzler, Olga 17.\,1.\,1882 Wien – 13.\,1.\,1970 Lugano@\textsc{Schnitzler, Olga} (17.\,1.\,1882 Wien – 13.\,1.\,1970 Lugano), \emph{Schauspielerin, Sängerin}|pwv} von
               mir und Paula\pwindex{Beer-Hofmann, Paula 25.\,2.\,1879 Wien – 30.\,10.\,1939 Zürich@\textsc{Beer-Hofmann, Paula} (25.\,2.\,1879 Wien – 30.\,10.\,1939 Zürich)|pw}.\pend
           
\pstart
           Von Herzen Ihr{\\[\baselineskip]}\spacefill\mbox{Richard}\pend
           \leftskip=0em{}
\pstart
           \noindent{}Bitte entschuldigen Sie mich gelegentlich bei Ihrer Schwägerin\pwindex{Steinrück, Elisabeth 19.\,11.\,1885 – 7.\,4.\,1920 Partenkirchen@\textsc{Steinrück, Elisabeth} (19.\,11.\,1885 – 7.\,4.\,1920 Partenkirchen)|pwv}, u. Steinrück\pwindex{Steinrück, Albert 20.\,5.\,1872 Wetterburg – 11.\,2.\,1929 Berlin@\textsc{Steinrück, Albert} (20.\,5.\,1872 Wetterburg – 11.\,2.\,1929 Berlin), \emph{Schauspieler}|pw}. Ich hatte vor der Abreise zuviel zu
                     besorgen.\hspace*{1.5em}R.\pend
           \selectlanguage{ngerman}\endnumbering\briefempfaengerindex{Schnitzler, Arthur@\textsc{Schnitzler, Arthur}!zzzBeer-Hofmann, Richard@\emph{von Richard Beer-Hofmann}!1905-08-151@{Mitte August 1905}|)be}\mylabel{L01542h}  \newcommand{\dateiname}{L01542}\newcommand{\titel}{Richard Beer-Hofmann an Arthur Schnitzler, Mitte August 1905}\newcommand{\editorInnen}{Herausgegeben von Martin Anton Müller}%% latex-leseansicht-abspann.tex
%% Abspann für die Leseansicht.
%% Der Schalter \ifkorrekturansicht ist bereits durch den Vorspann gesetzt.

%% latex-abspann.tex
%% Gemeinsamer Abspann für Korrekturansicht und Leseansicht.
%% Setzt den Schalter \ifkorrekturansicht voraus (gesetzt in den
%% einbindenden Dateien latex-korrekturansicht-abspann.tex bzw.
%% latex-leseansicht-abspann.tex).
%% ---------------------------------------------------------------

\normalsize

% Das esempio-Environment wird nur in der Leseansicht benötigt
\ifkorrekturansicht\else
\newenvironment{esempio}[3]%
{
    \vspace{1.5ex}
    \rlap{\underline{#1}}
    \par
    \setlength{\parindent}{0cm}
    \nopagebreak
    \leftskip=#2cm
    \rightskip=#3cm
}
{
    \par
}
\fi

\doendnotes{C}
\bigskip
\vfill

\clearpage

\footnotesize

\ifkorrekturansicht
  \lohead{\textsc{register}}
\fi

% theindex-Environment neu definieren ohne reledmac
\makeatletter
\renewenvironment{theindex}{%
  \ifkorrekturansicht
    \section*{\indexname}%
  \else
    \subsubsection*{Index der erwähnten Entitäten}%
  \fi
  \setlength{\parindent}{0pt}%
  \setlength{\parskip}{0pt plus 0.3pt}%
  \let\item\@idxitem
}{%
  \ifkorrekturansicht\clearpage\fi
}
\makeatother

\IfFileExists{\jobname-pw.ind}{\input{\jobname-pw.ind}}{}

% Quellenangabe nur in der Leseansicht
\ifkorrekturansicht\else
% Fallback-Definitionen, falls die .tex-Datei \titel etc. nicht gesetzt hat
\providecommand{\titel}{}
\providecommand{\editorInnen}{}
\providecommand{\dateiname}{\jobname}

\vspace{3cm}

\vfill

\footnotesize
\textsc{Quelle}: \titel. Herausgegeben von {\editorInnen}. In: \emph{Arthur Schnitzler: Briefwechsel mit Autorinnen und Autoren}.
 Digitale Edition, https://schnitzler-briefe.acdh.oeaw.ac.at/{\dateiname}.html (Stand \today)
\fi

\end{document}


