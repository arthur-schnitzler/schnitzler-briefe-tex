%% latex-leseansicht-vorspann.tex
%% Vorspann für die Leseansicht.
%% Lädt die gemeinsame Datei latex-vorspann.tex mit nicht gesetztem Schalter.

\newif\ifkorrekturansicht
\korrekturansichtfalse

\input{../tex-inputs/latex-vorspann}


               \section[Richard Beer-Hofmann an Arthur Schnitzler, Mitte August 1905]{ Richard Beer-Hofmann an Arthur Schnitzler, Mitte August 1905}\nopagebreak\mylabel{v}\rehead{ }\begin{ledgroupsized}[t]{13cm}\normalsize\beginnumbering\briefempfaengerindex{Schnitzler, Arthur@\textsc{Schnitzler, Arthur}!zzzBeer-Hofmann, Richard@\emph{von Richard Beer-Hofmann}!1905-08-151@{Mitte August 1905}|(be} \toendnotes[C]{\smallbreak\pagebreak[2]} \Standort{CUL, Schnitzler, B 8.}
\physDesc{Brief, 1 Blatt, 4 Seiten
\newline{}Handschrift: schwarze Tinte, lateinische Kurrent
\newline{}Schnitzler: mit Bleistift datiert: »Mitte Auguſt 905« \newline{}Ordnung: mit Bleistift von unbekannter Hand nummeriert:
                                    »204« }\buchAbdrucke{\weitereDrucke{Arthur Schnitzler, Richard Beer-Hofmann: \emph{Briefwechsel 1891–1931}. Hg. Konstanze Fliedl. Wien, Zürich: \emph{Europaverlag} 1992, S. 174–175.} }\toendnotes[C]{\smallbreak}\pstart
           \noindent{}\centering{}{\pb}\textcolor{gray}{\textbf{GRAND HOTEL STUBAI}}\oindex{Grand Hotel Stubai@\textbf{Grand Hotel Stubai}|pw}\pend
           \pstart
           \noindent{}\centering{}\textcolor{gray}{\textbf{FULPMES\oindex{Fulpmes@\textbf{Fulpmes}|pw}}}{ }\textsc{\textcolor{gray}{\textbf{bei innsbruck\oindex{Innsbruck@\textbf{Innsbruck}|pw} (tirol\oindex{Tirol@\textbf{Tirol}|pw})}}}\pend
           \pstart
           \noindent{}\textcolor{gray}{\textbf{TELEGRAMM-ADRESSE:}}\pend
           \pstart
           \textcolor{gray}{\textbf{STUBAIHOTEL FULPMES-INNSBRUCK\oindex{Grand Hotel Stubai@\textbf{Grand Hotel Stubai}|pw}}}\pend
           \pstart
           \textcolor{gray}{\textbf{ENDSTATION DER ELEKTRISCHEN BERGBAHN}}\pend
           \pstart
           \textcolor{gray}{\textbf{INNSBRUCK-FULPMES}}\pend
           \pstart
           Lieber Arthur! Wir sind da \label{T_L01542_1v}\edtext{oben}{\lemma{\textnormal{\emph{oben}}}\Cendnote{\textnormal{Ein Pfeil weist
                  auf die Hoteladresse.}}}\label{T_L01542_1h}. In Kärnten\oindex{Kaernten@\textbf{Kärnten}|pw} fand ich
               keine Unterkunft. Dort – wie im Pusterthal\oindex{Pustertal@\textbf{Pustertal}|pw} alles
               furchtbar überfüllt, so dass ich froh war hier unterzukommen. Das Hôtel ist \strikeout{es} erst voriges Jahr eröffnet worden, noch nicht sehr
               bekannt und daher halbleer.\pend
           \pstart
           Wassermanns\pwindex{Wassermann, Jakob 10.03.1873 – 01.01.1934@\textsc{Wassermann, Jakob} (10.03.1873 – 01.01.1934), \emph{Schriftsteller}|pw}\pwindex{Wassermann, Julie 05.12.1876 – April 1963@\textsc{Wassermann, Julie} (05.12.1876 – April 1963), \emph{Schriftstellerin}|pw}, S. Fischer\pwindex{Fischer, Samuel 24.12.1859 – 15.10.1934@\textsc{Fischer, Samuel} (24.12.1859 – 15.10.1934), \emph{Verleger}|pw}, Bella Wengerow\pwindex{Vengerova, Isabella 01.03.1877 – 07.02.1956@\textsc{Vengerova, Isabella} (01.03.1877 – 07.02.1956), \emph{Musikpädagogin, Pianistin}|pw},
                  Schwester\pwindex{Vengerova, Zinaida A. 18.04.1867 – 29.06.1941@\textsc{Vengerova, Zinaida A.} (18.04.1867 – 29.06.1941), \emph{Kritikerin, Übersetzerin}|pwv}, Mutter\pwindex{Wengeroff, Pauline 1833 – 1916@\textsc{Wengeroff, Pauline} (1833 – 1916), \emph{Schriftstellerin}|pwv}, u. D\textsuperscript{r}{ }Kaufmann\pwindex{Kaufmann, Arthur 04.04.1872 – 25.07.1938@\textsc{Kaufmann, Arthur} (04.04.1872 – 25.07.1938), \emph{Rechtswissenschaftler, Privatgelehrte, Privatier}|pw}{ }{\pb}sind hier. Paula\pwindex{Beer-Hofmann, Paula 25.02.1879 – 30.10.1939@\textsc{Beer-Hofmann, Paula} (25.02.1879 – 30.10.1939)|pw} hat kein Behagen an den kühlen Abenden und auch sonst an
               der Gegend – der Wald ist für sie – augenblicklich – zu weit vom Hôtel. Ich will also
               am 21 oder 22 von hier weg, und über Bozen\oindex{Bozen@\textbf{Bozen}|pw}, eventuell Gardasee\oindex{Lago di Garda@\textbf{Lago di Garda}|pw}, an
               den Lido\oindex{Lido@\textbf{Lido}|pw}. Hoffentlich tut ihr der Aufenthalt dort
               gut. Sie ist \uline{sehr} blutleer, und hat recht miserable
               Nerven. Das Stück\pwindex{Bahr, Hermann 19.07.1863 – 15.01.1934@\textsc{Bahr, Hermann} (19.07.1863 – 15.01.1934), \emph{Schriftsteller, Kritiker}!Andere1905@\strich\emph{Die Andere} {[}1905{]}|pwv} von Bahr\pwindex{Bahr, Hermann 19.07.1863 – 15.01.1934@\textsc{Bahr, Hermann} (19.07.1863 – 15.01.1934), \emph{Schriftsteller, Kritiker}|pw} blieb in Rodaun\oindex{Rodaun@\textbf{Rodaun}|pw} liegen, weil in folge der Aufschrift {\pb}»Eisenstein\orgindex{J. Eisenstein und Co.@J. Eisenstein {\kaufmannsund}  Co.|pw}« nur Bücher darin vermutet wurden, mit denen es nicht eilig sei;
               ich lasse es mir heute nachschicken.\pend
           \pstart
           Bitte sind Sie so gut und fügen Sie auf beiliegendem Brief die Adresse hinzu. Wer »A«
               sagt – –!\pend
           \pstart
           Hier hat sich das Gerücht verbreitet, Sie hätten dem Hugo\pwindex{Hofmannsthal, Hugo von 01.02.1874 – 15.07.1929@\textsc{Hofmannsthal, Hugo von} (01.02.1874 – 15.07.1929), \emph{Schriftsteller}|pw} zwei wunderschöne Stücke\pwindex{Schnitzler, Arthur 15.05.1862 – 21.10.1931@\textsc{Schnitzler, Arthur} (15.05.1862 – 21.10.1931), \emph{Schriftsteller, Mediziner}!Zwischenspiel. Komoedie in drei Akten1905-10-12 – 1905-10-12@\strich\emph{Zwischenspiel. Komödie in drei Akten} {[}1905-10-12 – 1905-10-12{]}|pwv}\pwindex{Schnitzler, Arthur 15.05.1862 – 21.10.1931@\textsc{Schnitzler, Arthur} (15.05.1862 – 21.10.1931), \emph{Schriftsteller, Mediziner}!Ruf des Lebens. Schauspiel in drei Akten1906-02-20@\strich\emph{Der Ruf des Lebens. Schauspiel in drei Akten} {[}1906-02-20{]}|pwv}{ }\label{K_L01542_2v}\edtext{vorgelesen}{\lemma{\textnormal{\emph{vorgelesen}}}\Cendnote{\textnormal{vgl. A. S.: \emph{Tagebuch}, 12. 8. 1905}}}\label{K_L01542_2h}. Ich freue mich sehr im
                  Oktober mehr davon zu erfahren.\pend
           \pstart
           Von mir will ich nichts schreiben, ich ziehe es vor Ihnen mündlich vorzujammern –
               obgleich {\pb}Sie mir dann bei
               physischen Dingen versichern werden Sie hätten dies Alles seit Jahren.\pend
           \pstart
           Schreiben Sie mir, bitte, i{\geminationm}er wo Sie sind – ich will es
               auch tun. Die Möglichkeit soll uns doch bleiben, uns etwas zu sagen.\pend
           \pstart
           Viele Grüsse an Ihre Frau\pwindex{Schnitzler, Olga 17.01.1882 – 13.01.1970@\textsc{Schnitzler, Olga} (17.01.1882 – 13.01.1970), \emph{Schauspielerin, Sängerin}|pwv} von
               mir und Paula\pwindex{Beer-Hofmann, Paula 25.02.1879 – 30.10.1939@\textsc{Beer-Hofmann, Paula} (25.02.1879 – 30.10.1939)|pw}.\pend
           \pstart
           Von Herzen Ihr{\\[\baselineskip]}\spacefill\mbox{Richard}\pend
           \leftskip=0em{}\pstart
           \noindent{}Bitte entschuldigen Sie mich gelegentlich bei Ihrer Schwägerin\pwindex{Steinrueck, Elisabeth 19.11.1885 – 07.04.1920@\textsc{Steinrück, Elisabeth} (19.11.1885 – 07.04.1920)|pwv}, u. Steinrück\pwindex{Steinrueck, Albert 20.05.1872 – 11.02.1929@\textsc{Steinrück, Albert} (20.05.1872 – 11.02.1929), \emph{Schauspieler}|pw}. Ich hatte vor der Abreise zuviel zu besorgen.\hspace*{1.5em}R.\pend
                     \endnumbering\briefempfaengerindex{Schnitzler, Arthur@\textsc{Schnitzler, Arthur}!zzzBeer-Hofmann, Richard@\emph{von Richard Beer-Hofmann}!1905-08-151@{Mitte August 1905}|)be}\mylabel{h}\end{ledgroupsized}  \newcommand{\dateiname}{L01542}\newcommand{\titel}{Richard Beer-Hofmann an Arthur Schnitzler, Mitte August 1905}\newcommand{\editorInnen}{ Martin Anton Müller und Gerd-Hermann Susen}
            \footnotesize
\begin{ledgroupsized}[t]{11.5cm}
\doendnotes{C}
\end{ledgroupsized}
         %% latex-leseansicht-abspann.tex
%% Abspann für die Leseansicht.
%% Der Schalter \ifkorrekturansicht ist bereits durch den Vorspann gesetzt.

%% latex-abspann.tex
%% Gemeinsamer Abspann für Korrekturansicht und Leseansicht.
%% Setzt den Schalter \ifkorrekturansicht voraus (gesetzt in den
%% einbindenden Dateien latex-korrekturansicht-abspann.tex bzw.
%% latex-leseansicht-abspann.tex).
%% ---------------------------------------------------------------

\normalsize

% Das esempio-Environment wird nur in der Leseansicht benötigt
\ifkorrekturansicht\else
\newenvironment{esempio}[3]%
{
    \vspace{1.5ex}
    \rlap{\underline{#1}}
    \par
    \setlength{\parindent}{0cm}
    \nopagebreak
    \leftskip=#2cm
    \rightskip=#3cm
}
{
    \par
}
\fi

\doendnotes{C}
\bigskip
\vfill

\clearpage

\footnotesize

\ifkorrekturansicht
  \lohead{\textsc{register}}
\fi

% theindex-Environment neu definieren ohne reledmac
\makeatletter
\renewenvironment{theindex}{%
  \ifkorrekturansicht
    \section*{\indexname}%
  \else
    \subsubsection*{Index der erwähnten Entitäten}%
  \fi
  \setlength{\parindent}{0pt}%
  \setlength{\parskip}{0pt plus 0.3pt}%
  \let\item\@idxitem
}{%
  \ifkorrekturansicht\clearpage\fi
}
\makeatother

\IfFileExists{\jobname-pw.ind}{\input{\jobname-pw.ind}}{}

% Quellenangabe nur in der Leseansicht
\ifkorrekturansicht\else
% Fallback-Definitionen, falls die .tex-Datei \titel etc. nicht gesetzt hat
\providecommand{\titel}{}
\providecommand{\editorInnen}{}
\providecommand{\dateiname}{\jobname}

\vspace{3cm}

\vfill

\footnotesize
\textsc{Quelle}: \titel. Herausgegeben von {\editorInnen}. In: \emph{Arthur Schnitzler: Briefwechsel mit Autorinnen und Autoren}.
 Digitale Edition, https://schnitzler-briefe.acdh.oeaw.ac.at/{\dateiname}.html (Stand \today)
\fi

\end{document}


      