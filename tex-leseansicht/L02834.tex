%% latex-leseansicht-vorspann.tex
%% Vorspann für die Leseansicht.
%% Lädt die gemeinsame Datei latex-vorspann.tex mit nicht gesetztem Schalter.

\newif\ifkorrekturansicht
\korrekturansichtfalse

\input{../tex-inputs/latex-vorspann}


         
         \renewcommand{\erwaehntePersonen}{Personen: Hermann Bahr, Richard Beer-Hofmann, Mirjam Beer-Hofmann, Alfred Heinrich Bulthaupt, Max Eugen Burckhard, Emil Claar, Max Devrient, Paul Goldmann, Robert Michel, Friedrich Gustav Piffl, Marie Reinhard, Jocza Savits, Paul Schlenther, Leopold Sonnemann}
         \renewcommand{\erwaehnteInstitutionen}{Institutionen: Burgtheater, Frankfurter Zeitung}
         \renewcommand{\erwaehnteOrte}{Orte: Berlin, Burgtheater, Frankreich, Paris, Wien, rue de la Bourse}
         \renewcommand{\erwaehnteWerke}{Werke: Der Tod Georgs, Die Schwestern oder Casanova in Spa. Lustspiel in Versen, Tagebuch}
               \section[ Paul Goldmann an Arthur Schnitzler, 23. 12. {[}1897{]}]{ Paul Goldmann an Arthur Schnitzler, 23. 12. {[}1897{]}}\nopagebreak\mylabel{v}\rehead{ }\begin{ledgroupsized}[t]{13cm}\normalsize\beginnumbering \toendnotes[C]{\smallbreak\pagebreak[2]} \Standort{DLA, A:Schnitzler, HS.NZ85.1.3167.}
\physDesc{Brief, 3 Blätter, 9 Seiten, 4684 Zeichen
\newline{}Handschrift: blaue Tinte, deutsche Kurrent
\newline{}Schnitzler: 1) mit Bleistift das Jahr »97« vermerkt  2) mit rotem Buntstift fünf Unterstreichungen}\toendnotes[C]{\smallbreak}\pstart
           \noindent{}{\pb}\textcolor{gray}{\textbf{\textbf{Frankfurter Zeitung\orgindex{Frankfurter Zeitung@Frankfurter Zeitung|pw}}}}\pend
           \pstart
           \textcolor{gray}{\textbf{(\begin{otherlanguage}{french}Gazette de Francfort\end{otherlanguage}\orgindex{Frankfurter Zeitung@Frankfurter Zeitung|pw}).}}\pend
           \pstart
           \textcolor{gray}{\textbf{\textbf{\begin{otherlanguage}{french}Fondateur M.\end{otherlanguage}{ }L. Sonnemann\pwindex{Sonnemann, Leopold 1831-10-29 – 1909-10-30@\textsc{Sonnemann, Leopold} (1831-10-29 – 1909-10-30), \emph{Journalist, Herausgeber}|pw}.}}}\pend
           \pstart
           \begin{otherlanguage}{french}\textcolor{gray}{\textbf{Journal politique, financier,}}\end{otherlanguage}\pend
           \pstart
           \begin{otherlanguage}{french}\textcolor{gray}{\textbf{commercial et littéraire.}}\end{otherlanguage}\pend
           \pstart
           \begin{otherlanguage}{french}\textcolor{gray}{\textbf{\textbf{Paraissant trois fois par jour.}}}\end{otherlanguage}\pend
           \pstart
           \begin{otherlanguage}{french}\textcolor{gray}{\textbf{\textbf{Bureau à Paris\oindex{Paris@\textbf{Paris}|pw}}}}\end{otherlanguage}\hfill \textsc{Paris\oindex{Paris@\textbf{Paris}|pw}}, 23. December.\pend
           \pstart
           \begin{otherlanguage}{french}\textcolor{gray}{\textbf{\textbf{10 \so{Rue de la Bourse}\oindex{rue de la Bourse@\textbf{rue de la Bourse}|pw}.}}}\end{otherlanguage}\pend
           \pstart{}Frohe Weihnachten, liebſter Freund!\pend\pstart
           Mit Deinem \label{K_L02834-1v}\edtext{Auge}{\lemma{\textnormal{\emph{Auge}}}\Cendnote{\textnormal{Schnitzler\pwindex{Schnitzler, Arthur 15.05.1862 – 21.10.1931@\textsc{Schnitzler, Arthur} (15.05.1862 – 21.10.1931), \emph{Schriftsteller, Mediziner}|pwk} litt an einem
               Gerstenkorn, siehe Arthur Schnitzler an Richard Beer-Hofmann, 11. 12. 1897.}}}\label{K_L02834-1h}
               geht es wohl beſſer? Dein letzter lieber Brief war recht verſtimmt. Freilich, mit
               einem Abſceß im Augenlid ſieht ſich das Leben nicht ſchön an.\pend
           \pstart
           Und doch hat mich Dein letzter Brief nachdenklich gemacht. Du darfſt mir nicht
               hypochondriſch werden! Und wenn es Dir ſchon im Ohre klingt! Muß man denn ganz geſund
               ſein?! Wer von uns iſt geſund? Man lebt und leidet eben. Iſt das nicht eine alte
               Geſchichte? Und lebt man deshalb weniger, weil man leidet? Eher mehr.\pend
           \pstart
           {\pb}Bei alledem glaube ich Dir Deine Krankheit gar
               nicht. Du haſt das, weil Dir, Gott ſei Dank, nichts Ernſtes fehlt. Du haſt viel Gutes
               und Herrliches ſchon genoſſen, Du biſt ein wenig abgeſtumpft geworden gegen all’ die
               ſchönen Dinge in Deinem Leben, das Errungene bildet darum kein rechtes Gegengewicht
               mehr gegen die Melancholie, die von Natur aus in dir wohnt, und ich glaube faſt, daß
               die Hypochondrie bei Dir eine Form der Blaſirtheit iſt.\pend
           \pstart
           Aufgeſchüttelt werden müßteſt Du, heraus müßteſt Du aus Deinem behaglichen Wien\oindex{Wien@\textbf{Wien}|pw}er Neſt, heraus in die Kälte, in die Fremde! Es
               iſt ganz natürlich, daß Du ſo, im gleichmäßigen {\pb}Weiterſchreiten, das Bewußtſein der Kräfte verlierſt, die in Dir wohnen.\pend
           \pstart
           Wie darfſt Du ſagen, daß Du nicht an Deine Zukunft glaubſt?! Wer hat Zukunft, wenn
               nicht Du?! Nur muß die Zukunft von ſelbſt erwachſen, als natürliche Frucht einer
               kräftigen Gegenwart. Ruhig leben, ſeine Kraft ſtärken, ausreifen laſſen, was reifen
               ſoll, und keine Ungeduld! Wenn man natürlich ſich jeden Tag hinſetzt und ſeine
               Zukunft machen will, ſo geht es nicht. Auch hier gibt es \strikeout{e\textcolor{gray}{r}} eine pſychiſche Impotenz. Nein, ſei ruhig und Deiner ſelbſt ſicher (weiß Gott,
               Du kannſt es!), {\pb}wenn es mit \strikeout{de} dem Produciren nicht geht, ſo leg’ es ein wenig
               beiſeite, ſchaffe Dir ſchöne Tage, und laß’ aus Tagen und Tagen ganz unmerklich die
               Zukunft werden! {\dotsfour}\pend
           \pstart
           Übrigens, was rede ich? Wenn Du dieſen Brief bekommſt, biſt Du ſicherlich bereits in
               ganz anderer Stimmung, wie damals, wo Du mir \strikeout{de} den
               Brief ſchriebſt, der vor mir liegt.\pend
           \pstart
           Keiner von Deinen Briefen aus de\substVorne{}\textsuperscript{r}\substDazwischen{}n\substHinten{} letzten Monaten iſt mir \label{K_L02834-2v}\edtext{geſtohlen}{\lemma{\textnormal{\emph{geſtohlen}}}\Cendnote{\textnormal{siehe Paul Goldmann an Arthur Schnitzler, 10. 12. [1897]}}}\label{K_L02834-2h} worden. Sei ganz beruhigt! Es handelt ſich um einige wenige Briefe früheren
               Datums, in denen ſicher nichts Wichtiges oder beſonders Vertrauliches ſteht.\pend
           \pstart
           {\pb}Was iſt mit dem \label{K_L02834-3v}\edtext{Burgtheater\orgindex{Burgtheater@Burgtheater|pw}}{\lemma{\textnormal{\emph{Burgtheater}}}\Cendnote{\textnormal{Max Burckhard\pwindex{Burckhard, Max Eugen 14.07.1854 – 16.03.1912@\textsc{Burckhard, Max Eugen} (14.07.1854 – 16.03.1912), \emph{Schriftsteller, Wissenschaftler, Theaterleiter}|pwk} trat als Direktor des \emph{Burgtheater}\orgindex{Burgtheater@Burgtheater|pwk}s zurück – seine Position war
                  unhaltbar geworden, nachdem er als Dramatiker an einem anderen Theater in
                  Erscheinung trat. Unter den potenziellen Nachfolgern fanden sich Heinrich Bulthaupt\pwindex{Bulthaupt, Alfred Heinrich 1849-10-26 – 1905-08-20@\textsc{Bulthaupt, Alfred Heinrich} (1849-10-26 – 1905-08-20), \emph{Theaterkritiker, Bibliothekar, Dichter}|pwk}, Emil
                     Claar\pwindex{Claar, Emil 1842-10-07 – 1930-07-25@\textsc{Claar, Emil} (1842-10-07 – 1930-07-25), \emph{Theaterleiter, Schauspieler}|pwk}, Jocza Savits\pwindex{Savits, Jocza 1847-05-10 – 1915-05-07@\textsc{Savits, Jocza} (1847-05-10 – 1915-05-07), \emph{Regisseur, Schauspieler, Übersetzer}|pwk} und Paul Schlenther\pwindex{Schlenther, Paul 20.08.1854 – 30.04.1916@\textsc{Schlenther, Paul} (20.08.1854 – 30.04.1916), \emph{Schriftsteller, Kritiker, Theaterleiter}|pwk}. Letztlich wurde Schlenther\pwindex{Schlenther, Paul 20.08.1854 – 30.04.1916@\textsc{Schlenther, Paul} (20.08.1854 – 30.04.1916), \emph{Schriftsteller, Kritiker, Theaterleiter}|pwk} am 25. 1. 1898 zum neuen Direktor bestimmt.}}}\label{K_L02834-3h}? Alſo hat es den \textsc{Burckhardt\pwindex{Burckhard, Max Eugen 14.07.1854 – 16.03.1912@\textsc{Burckhard, Max Eugen} (14.07.1854 – 16.03.1912), \emph{Schriftsteller, Wissenschaftler, Theaterleiter}|pw}} doch \strikeout{er} ereilt? Ich wundere mich nur, daß ich
               nicht den \textsc{Bahr\pwindex{Bahr, Hermann 19.07.1863 – 15.01.1934@\textsc{Bahr, Hermann} (19.07.1863 – 15.01.1934), \emph{Schriftsteller, Kritiker}|pw}} unter den Directions\orgindex{Burgtheater@Burgtheater|pwv}-Candidaten\pwindex{Bulthaupt, Alfred Heinrich 1849-10-26 – 1905-08-20@\textsc{Bulthaupt, Alfred Heinrich} (1849-10-26 – 1905-08-20), \emph{Theaterkritiker, Bibliothekar, Dichter}|pwv}\pwindex{Claar, Emil 1842-10-07 – 1930-07-25@\textsc{Claar, Emil} (1842-10-07 – 1930-07-25), \emph{Theaterleiter, Schauspieler}|pwv}\pwindex{Savits, Jocza 1847-05-10 – 1915-05-07@\textsc{Savits, Jocza} (1847-05-10 – 1915-05-07), \emph{Regisseur, Schauspieler, Übersetzer}|pwv}\pwindex{Schlenther, Paul 20.08.1854 – 30.04.1916@\textsc{Schlenther, Paul} (20.08.1854 – 30.04.1916), \emph{Schriftsteller, Kritiker, Theaterleiter}|pwv}
               leſe. Der Kerl\pwindex{Bahr, Hermann 19.07.1863 – 15.01.1934@\textsc{Bahr, Hermann} (19.07.1863 – 15.01.1934), \emph{Schriftsteller, Kritiker}|pwv} hat in Wien\oindex{Wien@\textbf{Wien}|pw}{ }\strikeout{den} den ſchlechten und faulen Boden gefunden, in dem
               allein er gedeihen konnte, und er gedeiht. Er wird großer \textsc{Pontifex} werden, und ich denke, \label{K_L02834-4v}\edtext{in ein paar Jahren}{\lemma{\textnormal{\emph{in ein paar Jahren}}}\Cendnote{\textnormal{Das war gewissermaßen prophetisch. Hermann Bahr\pwindex{Bahr, Hermann 19.07.1863 – 15.01.1934@\textsc{Bahr, Hermann} (19.07.1863 – 15.01.1934), \emph{Schriftsteller, Kritiker}|pwk} wurde im September 1918 als Teil des Dreierkollegiums (gemeinsam mit Max Devrient\pwindex{Devrient, Max 12.12.1857 – 13.06.1929@\textsc{Devrient, Max} (12.12.1857 – 13.06.1929), \emph{Regisseur, Schauspieler}|pwk} und Robert Michel\pwindex{Michel, Robert 24.02.1876 – 12.02.1957@\textsc{Michel, Robert} (24.02.1876 – 12.02.1957), \emph{Schriftsteller}|pwk}) erster Dramaturg des \emph{Burgtheater}\orgindex{Burgtheater@Burgtheater|pwk}s. Vgl. A. S.: \emph{Tagebuch}, 20. 9. 1918: »Wer ihm’s prophezeit hätte – vor
                     25 Jahren – daß seine erste Amtshandlung im B.
                        Th.\oindex{Burgtheater@\textbf{Burgtheater}|pw} sein würde, des ›Kampfgenossen aus Jugendjahren‹ Stück\pwindex{Schnitzler, Arthur 15.05.1862 – 21.10.1931@\textsc{Schnitzler, Arthur} (15.05.1862 – 21.10.1931), \emph{Schriftsteller, Mediziner}!Schwestern oder Casanova in Spa. Lustspiel in Versen01. 10. 1919@\strich\emph{Die Schwestern oder Casanova in Spa. Lustspiel in Versen} {[}01. 10. 1919{]}|pwv} – zu refusiren – weil dem Cardinal\pwindex{Piffl, Friedrich Gustav 15.10.1864 – 21.04.1932@\textsc{Piffl, Friedrich Gustav} (15.10.1864 – 21.04.1932), \emph{Kardinal, Erzbischof}|pwv} die Aufführung
                     peinlich sein könnte!–«}}}\label{K_L02834-4h} wird man ihm auch das Burgtheater\orgindex{Burgtheater@Burgtheater|pw} anbieten. Eines Tages werden dann vielleicht auch
               andere Leute entdecken, daß er ein unehrlicher und unverſtändiger Menſch iſt, aber
               dann wird es zu ſpät ſein.\pend
           \pstart
           {\pb}Dir ſollten ſie das Burgtheater\orgindex{Burgtheater@Burgtheater|pw} geben. Ich wüßte in der Welt keinen beſſeren Director. \textsc{Schlenther\pwindex{Schlenther, Paul 20.08.1854 – 30.04.1916@\textsc{Schlenther, Paul} (20.08.1854 – 30.04.1916), \emph{Schriftsteller, Kritiker, Theaterleiter}|pw}}? Wäre das der \strikeout{\textcolor{gray}{×}} Richtige? Dieſer Berlin\oindex{Berlin@\textbf{Berlin}|pw}er\pwindex{Schlenther, Paul 20.08.1854 – 30.04.1916@\textsc{Schlenther, Paul} (20.08.1854 – 30.04.1916), \emph{Schriftsteller, Kritiker, Theaterleiter}|pwv} und Proteſtant\pwindex{Schlenther, Paul 20.08.1854 – 30.04.1916@\textsc{Schlenther, Paul} (20.08.1854 – 30.04.1916), \emph{Schriftsteller, Kritiker, Theaterleiter}|pwv}, der wahrſcheinlich ein kluger
               Mann, aber ſicherlich ein kalter und \strikeout{åunk\textcolor{gray}{ü}nſ} unkünſtleriſcher Mann iſt?\pend
           \pstart
           Bitte, grüß’ mir Deine Freundin\pwindex{Reinhard, Marie 1871-03-13 – 1899-03-18@\textsc{Reinhard, Marie} (1871-03-13 – 1899-03-18), \emph{Gesangspädagogin}|pwv} recht herzlich. Ich bringe es nicht fertig, ihr irgend etwas von
               meinen Arbeiten zu ſchicken. Ich weiß, daß das, was ich ſchreibe, der Vergeſſenheit
               verfallen iſt, und dieſes Bewußtſein lähmt mich ſo, daß ich nicht \strikeout{es} e\textcolor{gray}{i}nmal die Kraft habe, einen
               Artikel {\pb}herauszuſuchen und ihn auf die Poſt zu
               geben. Ich bin eben ein Journaliſt und nichts Anderes. Frage nur den Herrn \textsc{Bahr\pwindex{Bahr, Hermann 19.07.1863 – 15.01.1934@\textsc{Bahr, Hermann} (19.07.1863 – 15.01.1934), \emph{Schriftsteller, Kritiker}|pw}} und ſeine Bande, ſie werden es Dir ſchon ſagen.\pend
           \pstart
           Was macht \textsc{Richard\pwindex{Beer-Hofmann, Richard 1866-07-11 – 1945-09-26@\textsc{Beer-Hofmann, Richard} (1866-07-11 – 1945-09-26), \emph{Schriftsteller}|pw}}? Iſt ſeine \label{K_L02834-5v}\edtext{Novelle\pwindex{Beer-Hofmann, Richard 1866-07-11 – 1945-09-26@\textsc{Beer-Hofmann, Richard} (1866-07-11 – 1945-09-26), \emph{Schriftsteller}!Tod Georgs1900@\strich\emph{Der Tod Georgs} {[}1900{]}|pwv} beendet}{\lemma{\textnormal{\emph{Novelle beendet}}}\Cendnote{\textnormal{Richard Beer-Hofmann\pwindex{Beer-Hofmann, Richard 1866-07-11 – 1945-09-26@\textsc{Beer-Hofmann, Richard} (1866-07-11 – 1945-09-26), \emph{Schriftsteller}|pwk} stellte \emph{Der Tod Georgs}\pwindex{Beer-Hofmann, Richard 1866-07-11 – 1945-09-26@\textsc{Beer-Hofmann, Richard} (1866-07-11 – 1945-09-26), \emph{Schriftsteller}!Tod Georgs1900@\strich\emph{Der Tod Georgs} {[}1900{]}|pwk} erst Ende Juli 1899 fertig (vgl. Richard Beer-Hofmann an Arthur Schnitzler, 31. 7. 1899).}}}\label{K_L02834-5h}? Ich fürchte ſehr, daß es dem Helden einfallen könnte,
               zum Schluß noch von einem anderen Tempel zu träumen, und das würde dann wieder ein
               bis zwei Jahre dauern. Und \label{K_L02834-6v}\edtext{\textsc{Mirjam\pwindex{Beer-Hofmann, Mirjam 04.09.1897 – 24.12.1984@\textsc{Beer-Hofmann, Mirjam} (04.09.1897 – 24.12.1984)|pw}}}{\lemma{\textnormal{\emph{Mirjam}}}\Cendnote{\textnormal{Beer-Hofmann\pwindex{Beer-Hofmann, Richard 1866-07-11 – 1945-09-26@\textsc{Beer-Hofmann, Richard} (1866-07-11 – 1945-09-26), \emph{Schriftsteller}|pwkv}s dreieinhalb
                  Monate alte Tochter\pwindex{Beer-Hofmann, Mirjam 04.09.1897 – 24.12.1984@\textsc{Beer-Hofmann, Mirjam} (04.09.1897 – 24.12.1984)|pwkv}}}}\label{K_L02834-6h}? {\dotsfour}\pend
           \pstart
           Ich habe arge Wochen durchgemacht und fürchterlich gelitten. Es iſt ſchlimm, \label{K_L02834-7v}\edtext{Beſchimpfungen}{\lemma{\textnormal{\emph{Beſchimpfungen}}}\Cendnote{\textnormal{siehe Paul Goldmann an Arthur Schnitzler, 10. 12. [1897]}}}\label{K_L02834-7h} ertragen zu müſſen, {\pb}ohne ſich wehren zu
               können, und zu fühlen, wie rings um Einen das Mißtrauen ſchleicht. Und dabei ganz
               allein, im fremden \strikeout{Lan}{ }Lande\oindex{Frankreich@\textbf{Frankreich}|pwv}, ohne Freund, ohne
               ermuthigenden Zuſpruch! Und nichts thun können, als einfach ruhig bei ſeiner
               Überzeugung bleiben. Man muß \strikeout{ſtillſ} ſtillſtehen und
               ſeine Pflicht thun, und in dieſer harten Pflichterfüllung iſt keinerlei Ruhe\strikeout{\textcolor{gray}{n}} zu holen. Nichts als Schläge, und bitterer Zweifel im Innern! Und doch, ich
               kann mich nicht entſchließen, jede Hoffnung aufzugeben. Auf der einen Seite die
               Wahrheit, auf der anderen Seite ein ganzes Volk. Es iſt nicht geſagt, {\pb}daß das Volk der ſtärkere Theil ſein
               muß.\pend
           \pstart
           Ich habe \textsc{Paris\oindex{Paris@\textbf{Paris}|pw}} ſatt über alle Maßen. Ich möchte ſo gerne fort, aber meine Zeitung\orgindex{Frankfurter Zeitung@Frankfurter Zeitung|pwv} will \strikeout{\textcolor{gray}{m}} es bisher nicht zugeben. Es iſt ihnen ſo bequem, mich als \strikeout{Ar} Arbeitsthier hier zu haben.\pend
           \pstart
           Nicht wahr, liebſter Freund, Du ſchreibſt mir bald?\pend
           \pstart
           Und nochmals von Herzen fröhliche Feiertage!\pend
           \pstart
           In Treue {\\[\baselineskip]}Dein {\\[\baselineskip]}\spacefill\mbox{Paul Goldmann}\pend
           \leftskip=0em{}
         
         \endnumbering\mylabel{h}\end{ledgroupsized}  \newcommand{\dateiname}{L02834}\newcommand{\titel}{Paul Goldmann an Arthur Schnitzler, 23. 12. [1897]}\newcommand{\editorInnen}{Martin Anton Müller und Laura Untner}%% latex-leseansicht-abspann.tex
%% Abspann für die Leseansicht.
%% Der Schalter \ifkorrekturansicht ist bereits durch den Vorspann gesetzt.

%% latex-abspann.tex
%% Gemeinsamer Abspann für Korrekturansicht und Leseansicht.
%% Setzt den Schalter \ifkorrekturansicht voraus (gesetzt in den
%% einbindenden Dateien latex-korrekturansicht-abspann.tex bzw.
%% latex-leseansicht-abspann.tex).
%% ---------------------------------------------------------------

\normalsize

% Das esempio-Environment wird nur in der Leseansicht benötigt
\ifkorrekturansicht\else
\newenvironment{esempio}[3]%
{
    \vspace{1.5ex}
    \rlap{\underline{#1}}
    \par
    \setlength{\parindent}{0cm}
    \nopagebreak
    \leftskip=#2cm
    \rightskip=#3cm
}
{
    \par
}
\fi

\doendnotes{C}
\bigskip
\vfill

\clearpage

\footnotesize

\ifkorrekturansicht
  \lohead{\textsc{register}}
\fi

% theindex-Environment neu definieren ohne reledmac
\makeatletter
\renewenvironment{theindex}{%
  \ifkorrekturansicht
    \section*{\indexname}%
  \else
    \subsubsection*{Index der erwähnten Entitäten}%
  \fi
  \setlength{\parindent}{0pt}%
  \setlength{\parskip}{0pt plus 0.3pt}%
  \let\item\@idxitem
}{%
  \ifkorrekturansicht\clearpage\fi
}
\makeatother

\IfFileExists{\jobname-pw.ind}{\input{\jobname-pw.ind}}{}

% Quellenangabe nur in der Leseansicht
\ifkorrekturansicht\else
% Fallback-Definitionen, falls die .tex-Datei \titel etc. nicht gesetzt hat
\providecommand{\titel}{}
\providecommand{\editorInnen}{}
\providecommand{\dateiname}{\jobname}

\vspace{3cm}

\vfill

\footnotesize
\textsc{Quelle}: \titel. Herausgegeben von {\editorInnen}. In: \emph{Arthur Schnitzler: Briefwechsel mit Autorinnen und Autoren}.
 Digitale Edition, https://schnitzler-briefe.acdh.oeaw.ac.at/{\dateiname}.html (Stand \today)
\fi

\end{document}


      