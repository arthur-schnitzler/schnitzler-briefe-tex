%% latex-leseansicht-vorspann.tex
%% Vorspann für die Leseansicht.
%% Lädt die gemeinsame Datei latex-vorspann.tex mit nicht gesetztem Schalter.

\newif\ifkorrekturansicht
\korrekturansichtfalse

\input{../tex-inputs/latex-vorspann}


\section[ Felix Salten an Arthur Schnitzler, 14. {[}10. 1903{]}]{L03358 Felix Salten an Arthur Schnitzler,  14. [10. 1903]}
\nopagebreak\mylabel{L03358v}
\rehead{ }\normalsize\beginnumbering\briefempfaengerindex{Schnitzler, Arthur@\textsc{Schnitzler, Arthur}!zzzSalten, Felix@\emph{von Felix Salten}!1903-10-141@{14. [10. 1903]}|(be}
\toendnotes[C]{\smallbreak\pagebreak[2]}
\correspDesc{Versand  durch Felix Salten am 14. [10. 1903] in Wien
\newline{}Erhalt  durch Arthur Schnitzler am [14. 10. 1903] in Wien}\toendnotes[C]{\smallbreak}
\Standort{CUL, Schnitzler, B 89, A 2.}
\physDesc{Karte, 541 Zeichen
\newline{}Handschrift: Bleistift, lateinische Kurrent
\newline{}Schnitzler: mit Bleistift die Monatsangabe verdeutlicht und die Jahreszahl ergänzt: »X 90\textcolor{gray}{3}« 
\newline{}Ordnung: mit Bleistift von unbekannter Hand nummeriert: »{\pb}173« }\toendnotes[C]{\smallbreak}
\pstart
           \raggedleft{}14. \textcolor{gray}{X}.\pend
           \vspace{0.5em}
\pstart
           {\pb}Lieber, ich muß leider auch für Freitag absagen. Ich bin diese Woche zu sehr in Anspruch genommen. Aber
                  \label{K_L03358-1v}\edtext{Mittwoch}{\lemma{\textnormal{\emph{Mittwoch}}}\Cendnote{\textnormal{Siehe A. S.: \emph{Tagebuch}, 21. 10. 1903.
               }}}\label{K_L03358-1} ganz \uline{bestimmt}. Hoffentlich passt Ihnen dieser
               Tag. Wenn \label{K_L03358-2v}\edtext{Sonntag}{\lemma{\textnormal{\emph{Sonntag}}}\Cendnote{\textnormal{Siehe A. S.: \emph{Tagebuch}, 18. 10. 1903.
               }}}\label{K_L03358-2} schönes Wetter ist, fahren wir Vormittag schon irgendwo hinaus,
               um im Freien zu essen. Am liebsten nach Hietzing\oindex{XIII., Hietzing@\textbf{XIII., Hietzing}, \emph{Verwaltungsgebiet}|pw},
               weil ich meinem \label{K_L03358-3v}\edtext{Mäderl\pwindex{Kotter, Caroline 7.\,7.\,1893 Wien – 1.\,7.\,1964 ebd.@\textsc{Kotter, Caroline} (7.\,7.\,1893 Wien – 1.\,7.\,1964 ebd.)|pwv}}{\lemma{\textnormal{\emph{Mäderl}}}\Cendnote{\textnormal{Caroline Kotter\pwindex{Kotter, Caroline 7.\,7.\,1893 Wien – 1.\,7.\,1964 ebd.@\textsc{Kotter, Caroline} (7.\,7.\,1893 Wien – 1.\,7.\,1964 ebd.)|pwk}, Saltens\pwindex{Salten, Felix 6.\,9.\,1869 Budapest – 8.\,10.\,1945 Zürich@\textsc{Salten, Felix} (6.\,9.\,1869 Budapest – 8.\,10.\,1945 Zürich), \emph{Schriftsteller, Journalist, Chefredakteur}|pwk} Tochter mit Elisabeth Kotter\pwindex{Kotter, Elisabeth *~1873 Groß-Enzersdorf@\textsc{Kotter, Elisabeth} (*~1873 Groß-Enzersdorf), \emph{Haushaltshilfe}|pwk}, die er kürzlich bei sich aufgenommen hatte}}}\label{K_L03358-3}{ }Schönbrunn\oindex{Wien@\textbf{Wien}!XIII., Hietzing@\textbf{XIII., Hietzing}!Schlosspark Schönbrunn@\textbf{Schlosspark Schönbrunn}, \emph{Park}|pw} zeigen möchte. Wir würden uns sehr
               freuen, wenn Sie mir uns beisammen sein könnten.\pend
           \pstart herzlichst Ihr \spacefill\mbox{S.}\pend{}
\pstart
           \noindent{}Wir nehmen \textcolor{gray}{auch} den Paul\pwindex{Salten, Paul 11.\,8.\,1903 Wien – 8.\,5.\,1937 ebd.@\textsc{Salten, Paul} (11.\,8.\,1903 Wien – 8.\,5.\,1937 ebd.), \emph{Filmcutter}|pw}
                  mit, und hätten mit Heinrich\pwindex{Schnitzler, Heinrich 9.\,8.\,1902 Hinterbrühl – 12.\,7.\,1982 Wien@\textsc{Schnitzler, Heinrich} (9.\,8.\,1902 Hinterbrühl – 12.\,7.\,1982 Wien), \emph{Regisseur, Schauspieler}|pw} eine Freude.
                  Wagen? Die Omnibus C\textsuperscript{o} stellt vis a vis Wagen.
                     Gummi{[},{]} sehr billig!\pend
           \selectlanguage{ngerman}\endnumbering\briefempfaengerindex{Schnitzler, Arthur@\textsc{Schnitzler, Arthur}!zzzSalten, Felix@\emph{von Felix Salten}!1903-10-141@{14. [10. 1903]}|)be}\mylabel{L03358h}  \newcommand{\dateiname}{L03358}\newcommand{\titel}{Felix Salten an Arthur Schnitzler, 14. [10. 1903]}\newcommand{\editorInnen}{Martin Anton Müller und Laura Untner}%% latex-leseansicht-abspann.tex
%% Abspann für die Leseansicht.
%% Der Schalter \ifkorrekturansicht ist bereits durch den Vorspann gesetzt.

%% latex-abspann.tex
%% Gemeinsamer Abspann für Korrekturansicht und Leseansicht.
%% Setzt den Schalter \ifkorrekturansicht voraus (gesetzt in den
%% einbindenden Dateien latex-korrekturansicht-abspann.tex bzw.
%% latex-leseansicht-abspann.tex).
%% ---------------------------------------------------------------

\normalsize

% Das esempio-Environment wird nur in der Leseansicht benötigt
\ifkorrekturansicht\else
\newenvironment{esempio}[3]%
{
    \vspace{1.5ex}
    \rlap{\underline{#1}}
    \par
    \setlength{\parindent}{0cm}
    \nopagebreak
    \leftskip=#2cm
    \rightskip=#3cm
}
{
    \par
}
\fi

\doendnotes{C}
\bigskip
\vfill

\clearpage

\footnotesize

\ifkorrekturansicht
  \lohead{\textsc{register}}
\fi

% theindex-Environment neu definieren ohne reledmac
\makeatletter
\renewenvironment{theindex}{%
  \ifkorrekturansicht
    \section*{\indexname}%
  \else
    \subsubsection*{Index der erwähnten Entitäten}%
  \fi
  \setlength{\parindent}{0pt}%
  \setlength{\parskip}{0pt plus 0.3pt}%
  \let\item\@idxitem
}{%
  \ifkorrekturansicht\clearpage\fi
}
\makeatother

\IfFileExists{\jobname-pw.ind}{\input{\jobname-pw.ind}}{}

% Quellenangabe nur in der Leseansicht
\ifkorrekturansicht\else
% Fallback-Definitionen, falls die .tex-Datei \titel etc. nicht gesetzt hat
\providecommand{\titel}{}
\providecommand{\editorInnen}{}
\providecommand{\dateiname}{\jobname}

\vspace{3cm}

\vfill

\footnotesize
\textsc{Quelle}: \titel. Herausgegeben von {\editorInnen}. In: \emph{Arthur Schnitzler: Briefwechsel mit Autorinnen und Autoren}.
 Digitale Edition, https://schnitzler-briefe.acdh.oeaw.ac.at/{\dateiname}.html (Stand \today)
\fi

\end{document}


