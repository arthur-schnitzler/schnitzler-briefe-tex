%% latex-korrekturansicht-vorspann.tex
%% Vorspann für die Korrekturansicht.
%% Lädt die gemeinsame Datei latex-vorspann.tex mit gesetztem Schalter.

\newif\ifkorrekturansicht
\korrekturansichttrue

\input{../tex-inputs/latex-vorspann}


\section[ Felix Salten an Arthur Schnitzler, 14. {[}10. 1903{]}]{L03358 Felix Salten an Arthur Schnitzler, 14. {[}10. 1903{]}}
\nopagebreak\mylabel{L03358v}
\rehead{ }\normalsize\beginnumbering\briefempfaengerindex{Schnitzler, Arthur@\textsc{Schnitzler, Arthur}!zzzSalten, Felix@\emph{von Felix Salten}!1903-10-141@{14. {[}10. 1903{]}}|(be}
\toendnotes[C]{\smallbreak\pagebreak[2]}\Standort{CUL, Schnitzler, B 89, A 2.}
\physDesc{Karte, 541 Zeichen
\newline{}Handschrift: Bleistift, lateinische Kurrent
\newline{}Schnitzler: mit Bleistift die Monatsangabe verdeutlicht und die Jahreszahl ergänzt: »X 90\textcolor{gray}{3}« 
\newline{}Ordnung: mit Bleistift von unbekannter Hand nummeriert: »{\pb}173« }\toendnotes[C]{\smallbreak}
\pstart
           \raggedleft{}14. \textcolor{gray}{X}.\pend
           \vspace{0.5em}
\pstart
           {\pb}Lieber, ich muß leider auch für Freitag absagen. Ich bin diese Woche zu sehr in Anspruch genommen. Aber
                  \label{K_L03358-1v}\edtext{Mittwoch}{\lemma{\textnormal{\emph{Mittwoch}}}\Cendnote{\textnormal{Siehe A. S.: \emph{Tagebuch}, 21. 10. 1903.
               }}}\label{K_L03358-1} ganz \uline{bestimmt}. Hoffentlich passt Ihnen dieser
               Tag. Wenn \label{K_L03358-2v}\edtext{Sonntag}{\lemma{\textnormal{\emph{Sonntag}}}\Cendnote{\textnormal{Siehe A. S.: \emph{Tagebuch}, 18. 10. 1903.
               }}}\label{K_L03358-2} schönes Wetter ist, fahren wir Vormittag schon irgendwo hinaus,
               um im Freien zu essen. Am liebsten nach Hietzing\oindex{XIII., Hietzing@\textbf{XIII., Hietzing}, \emph{A.ADM3}|pw},
               weil ich meinem \label{K_L03358-3v}\edtext{Mäderl\pwindex{Kotter, Caroline 1893-07-07 – 1964-07-01@\textsc{Kotter, Caroline} (1893-07-07 – 1964-07-01)|pwv}}{\lemma{\textnormal{\emph{Mäderl}}}\Cendnote{\textnormal{Caroline Kotter\pwindex{Kotter, Caroline 1893-07-07 – 1964-07-01@\textsc{Kotter, Caroline} (1893-07-07 – 1964-07-01)|pwk}, Saltens\pwindex{Salten, Felix 06.09.1869 – 08.10.1945@\textsc{Salten, Felix} (06.09.1869 – 08.10.1945), \emph{Schriftsteller/Schriftstellerin, Journalist/Journalistin, Chefredakteur/Chefredakteurin}|pwk} Tochter mit Elisabeth Kotter\pwindex{Kotter, Elisabeth *~1873@\textsc{Kotter, Elisabeth} (*~1873), \emph{Haushaltshilfe/Haushaltshilfe}|pwk}, die er kürzlich bei sich aufgenommen hatte}}}\label{K_L03358-3}{ }Schönbrunn\oindex{Schlosspark Schoenbrunn@\textbf{Schlosspark Schönbrunn}, \emph{Park (K.PRK)}|pw} zeigen möchte. Wir würden uns sehr
               freuen, wenn Sie mir uns beisammen sein könnten.\pend
           \pstart herzlichst Ihr \spacefill\mbox{S.}\pend{}
\pstart
           \noindent{}Wir nehmen \textcolor{gray}{auch} den Paul\pwindex{Salten, Paul 11.08.1903 – 08.05.1937@\textsc{Salten, Paul} (11.08.1903 – 08.05.1937), \emph{Filmcutter/Filmcutterin}|pw}
                  mit, und hätten mit Heinrich\pwindex{Schnitzler, Heinrich 09.08.1902 – 12.07.1982@\textsc{Schnitzler, Heinrich} (09.08.1902 – 12.07.1982), \emph{Regisseur/Regisseurin, Schauspieler/Schauspielerin}|pw} eine Freude.
                  Wagen? Die Omnibus C\textsuperscript{o} stellt vis a vis Wagen.
                     Gummi{[},{]} sehr billig!\pend
           \selectlanguage{ngerman}\endnumbering\briefempfaengerindex{Schnitzler, Arthur@\textsc{Schnitzler, Arthur}!zzzSalten, Felix@\emph{von Felix Salten}!1903-10-141@{14. {[}10. 1903{]}}|)be}\mylabel{L03358h}  \normalsize

\doendnotes{C}
\bigskip
\vfill

\clearpage

\footnotesize

\lohead{\textsc{register}}

% Definiere theindex-Environment komplett neu ohne reledmac
\makeatletter
\renewenvironment{theindex}{%
  \section*{\indexname}%
  \setlength{\parindent}{0pt}%
  \setlength{\parskip}{0pt plus 0.3pt}%
  \let\item\@idxitem
}{%
  \clearpage
}
\makeatother

\IfFileExists{\jobname-pw.ind}{\input{\jobname-pw.ind}}{}

\end{document}

      