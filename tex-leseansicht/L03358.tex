%% latex-leseansicht-vorspann.tex
%% Vorspann für die Leseansicht.
%% Lädt die gemeinsame Datei latex-vorspann.tex mit nicht gesetztem Schalter.

\newif\ifkorrekturansicht
\korrekturansichtfalse

\input{../tex-inputs/latex-vorspann}

\begin{center}
            \textcolor{red}{ENTWURF, NICHT FERTIG KORRIGIERT}
                      \end{center}
            
         
         \renewcommand{\erwaehntePersonen}{Personen: Caroline Kotter, Paul Salten, Heinrich Schnitzler}
         \renewcommand{\erwaehnteOrte}{Orte: Schlosspark Schönbrunn, Wien, XIII., Hietzing}
         \renewcommand{\erwaehnteWerke}{}
               \section[Felix Salten an Arthur Schnitzler, 14. {[}10. 1903{]}]{ Felix Salten an Arthur Schnitzler, 14. {[}10. 1903{]}}\nopagebreak\mylabel{v}\rehead{ }\begin{ledgroupsized}[t]{13cm}\normalsize\beginnumbering \toendnotes[C]{\smallbreak\pagebreak[2]} \Standort{CUL, Schnitzler, B 89, A 2.}
\physDesc{Karte
\newline{}Handschrift: Bleistift, lateinische Kurrent
\newline{}Schnitzler: mit Bleistift Monatsangabe verdeutlicht und Jahreszahl ergänzt: »X 90\textcolor{gray}{3}« }\toendnotes[C]{\smallbreak}\pstart
           \raggedleft{}14. \textcolor{gray}{X}.\pend
           \pstart
           {\pb}Lieber, ich muß leider auch für Freitag absagen. Ich
               bin diese Woche zu sehr in Anspruch genommen. Aber \label{K_L03358-v}\edtext{Mittwoch}{\lemma{\textnormal{\emph{Mittwoch}}}\Cendnote{\textnormal{}}}\label{K_L03358-h} ganz \uline{bestimmt}. Hoffentlich passt Ihnen dieser Tag. Wenn Sonntag
               schönes Wetter ist, fahren wir Vormittag schon irgendwo hinaus, um im Freien zu
               essen. Am liebsten nach Hietzing\oindex{XIII., Hietzing@\textbf{XIII., Hietzing}|pw}, weil ich meinem
                  Mäderl\pwindex{Kotter, Caroline 1893-07-07 – 1964-07-01@\textsc{Kotter, Caroline} (1893-07-07 – 1964-07-01)|pwv}{ }Schönbrunn\oindex{Schlosspark Schoenbrunn@\textbf{Schlosspark Schönbrunn}|pw} zeigen möchte. Wir würden uns sehr
               freuen, wenn Sie mir uns beisammen sein könnten. \pend
           \pstart Herzlichst Ihr \spacefill\mbox{S.}\pend{}\pstart
           \noindent{}Wir nehmen auch den Paul\pwindex{Salten, Paul 11.08.1903 – 08.05.1937@\textsc{Salten, Paul} (11.08.1903 – 08.05.1937), \emph{Filmcutter}|pw} mit, und hätten mit
                     Heinrich\pwindex{Schnitzler, Heinrich 09.08.1902 – 12.07.1982@\textsc{Schnitzler, Heinrich} (09.08.1902 – 12.07.1982), \emph{Regisseur, Schauspieler}|pw} eine Freude. Wagen? Die Omnibus
                     C\textsuperscript{o} stellt vis a vis Wagen.
                     Gummi\textcolor{gray}{,} sehr billig! \pend
           
         
         \endnumbering\mylabel{h}\end{ledgroupsized}\begin{anhang}\end{anhang}\newcommand{\dateiname}{L03358}\newcommand{\titel}{Felix Salten an Arthur Schnitzler, 14. [10. 1903]}\newcommand{\editorInnen}{Martin Anton Müller und Laura Untner}%% latex-leseansicht-abspann.tex
%% Abspann für die Leseansicht.
%% Der Schalter \ifkorrekturansicht ist bereits durch den Vorspann gesetzt.

%% latex-abspann.tex
%% Gemeinsamer Abspann für Korrekturansicht und Leseansicht.
%% Setzt den Schalter \ifkorrekturansicht voraus (gesetzt in den
%% einbindenden Dateien latex-korrekturansicht-abspann.tex bzw.
%% latex-leseansicht-abspann.tex).
%% ---------------------------------------------------------------

\normalsize

% Das esempio-Environment wird nur in der Leseansicht benötigt
\ifkorrekturansicht\else
\newenvironment{esempio}[3]%
{
    \vspace{1.5ex}
    \rlap{\underline{#1}}
    \par
    \setlength{\parindent}{0cm}
    \nopagebreak
    \leftskip=#2cm
    \rightskip=#3cm
}
{
    \par
}
\fi

\doendnotes{C}
\bigskip
\vfill

\clearpage

\footnotesize

\ifkorrekturansicht
  \lohead{\textsc{register}}
\fi

% theindex-Environment neu definieren ohne reledmac
\makeatletter
\renewenvironment{theindex}{%
  \ifkorrekturansicht
    \section*{\indexname}%
  \else
    \subsubsection*{Index der erwähnten Entitäten}%
  \fi
  \setlength{\parindent}{0pt}%
  \setlength{\parskip}{0pt plus 0.3pt}%
  \let\item\@idxitem
}{%
  \ifkorrekturansicht\clearpage\fi
}
\makeatother

\IfFileExists{\jobname-pw.ind}{\input{\jobname-pw.ind}}{}

% Quellenangabe nur in der Leseansicht
\ifkorrekturansicht\else
% Fallback-Definitionen, falls die .tex-Datei \titel etc. nicht gesetzt hat
\providecommand{\titel}{}
\providecommand{\editorInnen}{}
\providecommand{\dateiname}{\jobname}

\vspace{3cm}

\vfill

\footnotesize
\textsc{Quelle}: \titel. Herausgegeben von {\editorInnen}. In: \emph{Arthur Schnitzler: Briefwechsel mit Autorinnen und Autoren}.
 Digitale Edition, https://schnitzler-briefe.acdh.oeaw.ac.at/{\dateiname}.html (Stand \today)
\fi

\end{document}


      