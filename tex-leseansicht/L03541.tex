%% latex-leseansicht-vorspann.tex
%% Vorspann für die Leseansicht.
%% Lädt die gemeinsame Datei latex-vorspann.tex mit nicht gesetztem Schalter.

\newif\ifkorrekturansicht
\korrekturansichtfalse

\input{../tex-inputs/latex-vorspann}


\section[ Eva Marie Goldmann an Arthur Schnitzler, 1. 10. 1911]{L03541 Eva Marie Goldmann an Arthur Schnitzler,  1. 10. 1911}
\nopagebreak\mylabel{L03541v}
\rehead{ }\normalsize\beginnumbering\briefempfaengerindex{Schnitzler, Arthur@\textsc{Schnitzler, Arthur}!zzzGoldmann, Eva Marie@\emph{von Eva Marie Goldmann}!1911-10-011@{1. 10. 1911}|(be}
\toendnotes[C]{\smallbreak\pagebreak[2]}
\correspDesc{Versand  durch Eva Marie Goldmann am 1. 10. 1911 in Berlin
\newline{}Erhalt  durch Arthur Schnitzler im Zeitraum [2. 10. 1911
                  – 5. 10. 1911] in Wien}\toendnotes[C]{\smallbreak}
\Standort{DLA, A:Schnitzler, HS.NZ85.1.3160.}
\physDesc{Brief, 1 Blatt, 2 Seiten, 489 Zeichen
\newline{}Handschrift: lila Tinte, lateinische Kurrent
\newline{}Schnitzler: mit Bleistift Unterstreichung des »G« im vorgedruckten
                                 Briefkopf }\toendnotes[C]{\smallbreak}
\pstart
           \raggedleft{}{\pb}Berlin\oindex{Berlin@\textbf{Berlin}, \emph{Hauptstadt}|pw}, d. 1. X. \uline{1911}\uline{.}\pend
           
\pstart
           \textcolor{gray}{\textbf{EG}}\hfill \textcolor{gray}{\textbf{W. SCHÖNEBERGER-UFER 34\oindex{Schöneberger Ufer@\textbf{Schöneberger Ufer}, \emph{Straße}|pw}.}}\pend
           
\pstart{}Verehrter Herr Doctor,\pend\vspace{0.5em}
\pstart
           ich will Ihnen nur rasch den Empfang Ihres liebenswürdigen Briefes bestätigen, u.
               Ihnen für Ihre freundlichen Zeilen herzlichst danken. Beantworten kann ich sie heute nicht – aus irdischem Jammer. {\pb}Ich stecke nämlich mitten in den \label{K_L03541-1v}\edtext{Umzugsvorbereitungen.}{\lemma{\textnormal{\emph{Umzugsvorbereitungen.}}}\Cendnote{\textnormal{Sie zogen in die Bendlerstraße 36\oindex{Stauffenbergstraße@\textbf{Stauffenbergstraße}, \emph{Straße}|pwk}.}}}\label{K_L03541-1} Und was das bedeutet, kann \uline{nur} eine Frau ermessen!\pend
           
\pstart
           Hoffentlich wird Sie in absehbarer Zeit \label{K_L03541-2v}\edtext{»Das \substVorne{}\textsuperscript{W}\substDazwischen{}w\substHinten{}eite Land\pwindex{Schnitzler, Arthur 15.\,5.\,1862 Wien – 21.\,10.\,1931 ebd.@\textsc{Schnitzler, Arthur} (15.\,5.\,1862 Wien – 21.\,10.\,1931 ebd.), \emph{Schriftsteller, Mediziner}!weite Land. Tragikomödie in fünf Akten@\strich\emph{Das weite Land. Tragikomödie in fünf Akten}|pw}« persönlich nach Berlin\oindex{Berlin@\textbf{Berlin}, \emph{Hauptstadt}|pw}
               führen}{\lemma{\textnormal{\emph{»Das … führen}}}\Cendnote{\textnormal{Am 14. 10. 1911 fanden die
                  parallelen Uraufführungen von \emph{Das weite Land}\pwindex{Schnitzler, Arthur 15.\,5.\,1862 Wien – 21.\,10.\,1931 ebd.@\textsc{Schnitzler, Arthur} (15.\,5.\,1862 Wien – 21.\,10.\,1931 ebd.), \emph{Schriftsteller, Mediziner}!weite Land. Tragikomödie in fünf Akten@\strich\emph{Das weite Land. Tragikomödie in fünf Akten}|pwk}
                  in neun Städten statt, darunter am \emph{Burgtheater}\orgindex{Burgtheater@Burgtheater|pwk} und am \emph{Lessing-Theater}\orgindex{Lessing-Theater@Lessing-Theater|pwk} in Berlin\oindex{Berlin@\textbf{Berlin}, \emph{Hauptstadt}|pwk}. An letzterem Ort sah Schnitzler das Stück\pwindex{Schnitzler, Arthur 15.\,5.\,1862 Wien – 21.\,10.\,1931 ebd.@\textsc{Schnitzler, Arthur} (15.\,5.\,1862 Wien – 21.\,10.\,1931 ebd.), \emph{Schriftsteller, Mediziner}!weite Land. Tragikomödie in fünf Akten@\strich\emph{Das weite Land. Tragikomödie in fünf Akten}|pwkv} am 2. 11. 1911. Dem \emph{Tagebuch}\pwindex{Schnitzler, Arthur 15.\,5.\,1862 Wien – 21.\,10.\,1931 ebd.@\textsc{Schnitzler, Arthur} (15.\,5.\,1862 Wien – 21.\,10.\,1931 ebd.), \emph{Schriftsteller, Mediziner}!Tagebuch@\strich\emph{Tagebuch}|pwk} ist für diesen Tag keine Begegnung
                  mit Paul\pwindex{Goldmann, Paul 31.\,1.\,1865 Breslau – 25.\,9.\,1935 Wien@\textsc{Goldmann, Paul} (31.\,1.\,1865 Breslau – 25.\,9.\,1935 Wien), \emph{Schriftsteller, Journalist}|pwk} und Eva Goldmann\pwindex{Goldmann, Eva Marie 27.\,10.\,1877 Wien – 2.\,11.\,1937 ebd.@\textsc{Goldmann, Eva Marie} (27.\,10.\,1877 Wien – 2.\,11.\,1937 ebd.)|pwk} zu entnehmen. Das nächste belegte
                  Zusammentreffen fand an einem Bahnsteig am 28. 4. 1912 statt. Während Eva Goldmann\pwindex{Goldmann, Eva Marie 27.\,10.\,1877 Wien – 2.\,11.\,1937 ebd.@\textsc{Goldmann, Eva Marie} (27.\,10.\,1877 Wien – 2.\,11.\,1937 ebd.)|pwk}{ }Schnitzler begrüßte, vermied Paul Goldmann\pwindex{Goldmann, Paul 31.\,1.\,1865 Breslau – 25.\,9.\,1935 Wien@\textsc{Goldmann, Paul} (31.\,1.\,1865 Breslau – 25.\,9.\,1935 Wien), \emph{Schriftsteller, Journalist}|pwk} eine Begegnung.}}}\label{K_L03541-2}.\pend
           
\pstart
           Mit den besten Grüssen für Frau Olga\pwindex{Schnitzler, Olga 17.\,1.\,1882 Wien – 13.\,1.\,1970 Lugano@\textsc{Schnitzler, Olga} (17.\,1.\,1882 Wien – 13.\,1.\,1970 Lugano), \emph{Schauspielerin, Sängerin}|pw} u.
               Sie {\\[\baselineskip]}Ihre ergebene {\\[\baselineskip]}\spacefill\mbox{EvaMGoldmann.}\pend
           \leftskip=0em{}\selectlanguage{ngerman}\endnumbering\briefempfaengerindex{Schnitzler, Arthur@\textsc{Schnitzler, Arthur}!zzzGoldmann, Eva Marie@\emph{von Eva Marie Goldmann}!1911-10-011@{1. 10. 1911}|)be}\mylabel{L03541h}  \newcommand{\dateiname}{L03541}\newcommand{\titel}{Eva Marie Goldmann an Arthur Schnitzler, 1. 10. 1911}\newcommand{\editorInnen}{Martin Anton Müller und Laura Untner}%% latex-leseansicht-abspann.tex
%% Abspann für die Leseansicht.
%% Der Schalter \ifkorrekturansicht ist bereits durch den Vorspann gesetzt.

%% latex-abspann.tex
%% Gemeinsamer Abspann für Korrekturansicht und Leseansicht.
%% Setzt den Schalter \ifkorrekturansicht voraus (gesetzt in den
%% einbindenden Dateien latex-korrekturansicht-abspann.tex bzw.
%% latex-leseansicht-abspann.tex).
%% ---------------------------------------------------------------

\normalsize

% Das esempio-Environment wird nur in der Leseansicht benötigt
\ifkorrekturansicht\else
\newenvironment{esempio}[3]%
{
    \vspace{1.5ex}
    \rlap{\underline{#1}}
    \par
    \setlength{\parindent}{0cm}
    \nopagebreak
    \leftskip=#2cm
    \rightskip=#3cm
}
{
    \par
}
\fi

\doendnotes{C}
\bigskip
\vfill

\clearpage

\footnotesize

\ifkorrekturansicht
  \lohead{\textsc{register}}
\fi

% theindex-Environment neu definieren ohne reledmac
\makeatletter
\renewenvironment{theindex}{%
  \ifkorrekturansicht
    \section*{\indexname}%
  \else
    \subsubsection*{Index der erwähnten Entitäten}%
  \fi
  \setlength{\parindent}{0pt}%
  \setlength{\parskip}{0pt plus 0.3pt}%
  \let\item\@idxitem
}{%
  \ifkorrekturansicht\clearpage\fi
}
\makeatother

\IfFileExists{\jobname-pw.ind}{\input{\jobname-pw.ind}}{}

% Quellenangabe nur in der Leseansicht
\ifkorrekturansicht\else
% Fallback-Definitionen, falls die .tex-Datei \titel etc. nicht gesetzt hat
\providecommand{\titel}{}
\providecommand{\editorInnen}{}
\providecommand{\dateiname}{\jobname}

\vspace{3cm}

\vfill

\footnotesize
\textsc{Quelle}: \titel. Herausgegeben von {\editorInnen}. In: \emph{Arthur Schnitzler: Briefwechsel mit Autorinnen und Autoren}.
 Digitale Edition, https://schnitzler-briefe.acdh.oeaw.ac.at/{\dateiname}.html (Stand \today)
\fi

\end{document}


