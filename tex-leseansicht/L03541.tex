%% latex-korrekturansicht-vorspann.tex
%% Vorspann für die Korrekturansicht.
%% Lädt die gemeinsame Datei latex-vorspann.tex mit gesetztem Schalter.

\newif\ifkorrekturansicht
\korrekturansichttrue

\input{../tex-inputs/latex-vorspann}


\section[ Eva Marie Goldmann an Arthur Schnitzler, 1. 10. 1911]{L03541 Eva Marie Goldmann an Arthur Schnitzler, 1. 10. 1911}
\nopagebreak\mylabel{L03541v}
\rehead{ }\normalsize\beginnumbering\briefempfaengerindex{Schnitzler, Arthur@\textsc{Schnitzler, Arthur}!zzzGoldmann, Eva Marie@\emph{von Eva Marie Goldmann}!1911-10-011@{1. 10. 1911}|(be}
\toendnotes[C]{\smallbreak\pagebreak[2]}\Standort{DLA, A:Schnitzler, HS.NZ85.1.3160.}
\physDesc{Brief, 1 Blatt, 2 Seiten, 489 Zeichen
\newline{}Handschrift: lila Tinte, lateinische Kurrent
\newline{}Schnitzler: mit Bleistift Unterstreichung des »G« im vorgedruckten
                                 Briefkopf }\toendnotes[C]{\smallbreak}
\pstart
           \raggedleft{}{\pb}Berlin\oindex{Berlin@\textbf{Berlin}, \emph{P.PPLC}|pw}, d. 1. X. \uline{1911}\uline{.}\pend
           
\pstart
           \textcolor{gray}{\textbf{EG}}\hfill \textcolor{gray}{\textbf{W. SCHÖNEBERGER-UFER 34\oindex{Schoeneberger Ufer@\textbf{Schöneberger Ufer}, \emph{Straße (K.STR)}|pw}.}}\pend
           
\pstart{}Verehrter Herr Doctor,\pend\vspace{0.5em}
\pstart
           ich will Ihnen nur rasch den Empfang Ihres liebenswürdigen Briefes bestätigen, u.
               Ihnen für Ihre freundlichen Zeilen herzlichst danken. Beantworten kann ich sie heute nicht – aus irdischem Jammer. {\pb}Ich stecke nämlich mitten in den \label{K_L03541-1v}\edtext{Umzugsvorbereitungen.}{\lemma{\textnormal{\emph{Umzugsvorbereitungen.}}}\Cendnote{\textnormal{Sie zogen in die Bendlerstraße 36\oindex{Stauffenbergstrasse@\textbf{Stauffenbergstraße}, \emph{Straße (K.STR)}|pwk}.}}}\label{K_L03541-1} Und was das bedeutet, kann \uline{nur} eine Frau ermessen!\pend
           
\pstart
           Hoffentlich wird Sie in absehbarer Zeit \label{K_L03541-2v}\edtext{»Das \substVorne{}\textsuperscript{W}\substDazwischen{}w\substHinten{}eite Land\pwindex{weite Land. Tragikomoedie in fuenf Akten@\emph{Das weite Land. Tragikomödie in fünf Akten}|pw}« persönlich nach Berlin\oindex{Berlin@\textbf{Berlin}, \emph{P.PPLC}|pw}
               führen}{\lemma{\textnormal{\emph{»Das … führen}}}\Cendnote{\textnormal{Am 14. 10. 1911 fanden die
                  parallelen Uraufführungen von \emph{Das weite Land}\pwindex{weite Land. Tragikomoedie in fuenf Akten@\emph{Das weite Land. Tragikomödie in fünf Akten}|pwk}
                  in neun Städten statt, darunter am \emph{Burgtheater}\orgindex{Burgtheater@Burgtheater|pwk} und am \emph{Lessing-Theater}\orgindex{Lessing-Theater@Lessing-Theater|pwk} in Berlin\oindex{Berlin@\textbf{Berlin}, \emph{P.PPLC}|pwk}. An letzterem Ort sah Schnitzler das Stück\pwindex{weite Land. Tragikomoedie in fuenf Akten@\emph{Das weite Land. Tragikomödie in fünf Akten}|pwkv} am 2. 11. 1911. Dem \emph{Tagebuch}\pwindex{Tagebuch@\emph{Tagebuch}|pwk} ist für diesen Tag keine Begegnung
                  mit Paul\pwindex{Goldmann, Paul 31.01.1865 – 25.09.1935@\textsc{Goldmann, Paul} (31.01.1865 – 25.09.1935), \emph{Schriftsteller/Schriftstellerin, Journalist/Journalistin}|pwk} und Eva Goldmann\pwindex{Goldmann, Eva Marie 27.10.1877 – 02.11.1937@\textsc{Goldmann, Eva Marie} (27.10.1877 – 02.11.1937)|pwk} zu entnehmen. Das nächste belegte
                  Zusammentreffen fand an einem Bahnsteig am 28. 4. 1912 statt. Während Eva Goldmann\pwindex{Goldmann, Eva Marie 27.10.1877 – 02.11.1937@\textsc{Goldmann, Eva Marie} (27.10.1877 – 02.11.1937)|pwk}{ }Schnitzler begrüßte, vermied Paul Goldmann\pwindex{Goldmann, Paul 31.01.1865 – 25.09.1935@\textsc{Goldmann, Paul} (31.01.1865 – 25.09.1935), \emph{Schriftsteller/Schriftstellerin, Journalist/Journalistin}|pwk} eine Begegnung.}}}\label{K_L03541-2}.\pend
           
\pstart
           Mit den besten Grüssen für Frau Olga\pwindex{Schnitzler, Olga 17.01.1882 – 13.01.1970@\textsc{Schnitzler, Olga} (17.01.1882 – 13.01.1970), \emph{Schauspieler/Schauspielerin, Sänger/Sängerin}|pw} u.
               Sie {\\[\baselineskip]}Ihre ergebene {\\[\baselineskip]}\spacefill\mbox{EvaMGoldmann.}\pend
           \leftskip=0em{}\selectlanguage{ngerman}\endnumbering\briefempfaengerindex{Schnitzler, Arthur@\textsc{Schnitzler, Arthur}!zzzGoldmann, Eva Marie@\emph{von Eva Marie Goldmann}!1911-10-011@{1. 10. 1911}|)be}\mylabel{L03541h}  \normalsize

\doendnotes{C}
\bigskip
\vfill

\clearpage

\footnotesize

\lohead{\textsc{register}}

% Definiere theindex-Environment komplett neu ohne reledmac
\makeatletter
\renewenvironment{theindex}{%
  \section*{\indexname}%
  \setlength{\parindent}{0pt}%
  \setlength{\parskip}{0pt plus 0.3pt}%
  \let\item\@idxitem
}{%
  \clearpage
}
\makeatother

\IfFileExists{\jobname-pw.ind}{\input{\jobname-pw.ind}}{}

\end{document}

      