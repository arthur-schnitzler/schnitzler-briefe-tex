%% latex-leseansicht-vorspann.tex
%% Vorspann für die Leseansicht.
%% Lädt die gemeinsame Datei latex-vorspann.tex mit nicht gesetztem Schalter.

\newif\ifkorrekturansicht
\korrekturansichtfalse

\input{../tex-inputs/latex-vorspann}

\begin{center}
            \textcolor{red}{ENTWURF, NICHT FERTIG KORRIGIERT}
                      \end{center}
            
         
         \renewcommand{\erwaehntePersonen}{Personen: Olga Schnitzler, Elisabeth Steinrück, Max Sternfeld}
         \renewcommand{\erwaehnteInstitutionen}{Institutionen: Berliner Tageblatt, Goethe-Bund}
         \renewcommand{\erwaehnteOrte}{Orte: Alte Philharmonie (Berlin), Berlin, Deutschland, Palasthotel Berlin, Redaktion des Berliner Tageblatts, Wallnertheaterstraße, Wien}
         \renewcommand{\erwaehnteWerke}{Werke: Der Goethebund gegen die Theatercensur. (Telegramm der »Neuen Freien Presse«.), Neue Freie Presse, Tagebuch}
               \section[ Paul Goldmann an Arthur Schnitzler, 8. 3. 1903]{ Paul Goldmann an Arthur Schnitzler, 8. 3. 1903}\nopagebreak\mylabel{v}\rehead{ }\begin{ledgroupsized}[t]{13cm}\normalsize\beginnumbering \toendnotes[C]{\smallbreak\pagebreak[2]} \Standort{DLA, A:Schnitzler, HS.NZ85.1.3173.}
\physDesc{Postkarte
\newline{}Handschrift: 1) blaue Tinte, deutsche Kurrent\hspace{1em}2) blaue Tinte, lateinische Kurrent (\noindent{}Adresse)\hspace{1em}\newline{}Versand: Stempel: »\nobreak{}\oindex{Berlin@\textbf{Berlin}|pwk}Berlin, W. 9, 8. 3. 03, 5—N.\nobreak{}«. Stempel: »\nobreak{}\oindex{Berlin@\textbf{Berlin}|pwk}Berlin\textcolor{gray}{,} O.
                                       P27 (R15), 8 III 03, 5\textsuperscript{30} N\textcolor{gray}{.}\nobreak{}«.  }\toendnotes[C]{\smallbreak}\pstart{}{\pb}Fräulein Elisabeth Gussmann\pend{}\pstart{}für Herrn Dr. Schnitzler\pend{}\pstart{}Wallnertheaterstraße 40\oindex{Wallnertheaterstrasse@\textbf{Wallnertheaterstraße}|pw}\pend{}\pstart{}II. bei Sternfeld\pwindex{Sternfeld, Max @\textsc{Sternfeld, Max}, \emph{Kaufmann}|pw}.\pend{}{\bigskip}\pstart
           \noindent{}{\pb}Berlin\oindex{Berlin@\textbf{Berlin}|pw}, 8. März.\hfill Mein lieber Freund, \pend
           \pstart
           Ich habe Dich zwei Mal im \textsc{Hotel}\oindex{Palasthotel Berlin@\textbf{Palasthotel Berlin}|pwv} geſucht, um Dir zu ſagen, daß ich heut{ }{ }Abend\strikeout{\textcolor{gray}{×}\-\textcolor{gray}{×}\-\textcolor{gray}{×}} leider \label{K_L03368-1v}\edtext{nicht kommen}{\lemma{\textnormal{\emph{nicht kommen}}}\Cendnote{\textnormal{vermutlich zu Elisabeth Gussmann\pwindex{Steinrueck, Elisabeth 19.11.1885 – 07.04.1920@\textsc{Steinrück, Elisabeth} (19.11.1885 – 07.04.1920)|pwk} – dafür spricht der \emph{Tagebuch}\pwindex{Schnitzler, Arthur 15.05.1862 – 21.10.1931@\textsc{Schnitzler, Arthur} (15.05.1862 – 21.10.1931), \emph{Schriftsteller, Mediziner}!Tagebuch1981 – 2000@\strich\emph{Tagebuch} {[}1981 – 2000{]}|pwk}-Eintrag zum 8. 3. 1903 und die Adressierung der Postkarte an
                  sie}}}\label{K_L03368-1h} kann. Ich erhielt heut{ }Morgen telegraphiſchen Auftrag aus Wien\oindex{Wien@\textbf{Wien}|pw}, den \label{K_L03368-2v}\edtext{Bericht\pwindex{Goethebund gegen die Theatercensur. (Telegramm der »Neuen Freien
                  Presse«.)1903-03-09@\emph{Der Goethebund gegen die Theatercensur. (Telegramm der »Neuen Freien Presse«.)} {[}1903-03-09{]}|pwv} über die Goethebund\orgindex{Goethe-Bund@Goethe-Bund|pw}-Verſammlung}{\lemma{\textnormal{\emph{Bericht … Goethebund-Verſammlung}}}\Cendnote{\textnormal{Der deutsch\oindex{Deutschland@\textbf{Deutschland}|pwkv}e \emph{Goethe-Bund}\orgindex{Goethe-Bund@Goethe-Bund|pwk} tagte am 8. 3. 1903 in der Alten Berliner Philharmonie\oindex{Alte Philharmonie (Berlin)@\textbf{Alte Philharmonie (Berlin)}|pwk}. 
                  [Paul Goldmann\pwindex{Goldmann, Paul 31.01.1865 – 25.09.1935@\textsc{Goldmann, Paul} (31.01.1865 – 25.09.1935), \emph{Schriftsteller, Journalist}|pwk}:] \emph{Der Goethebund gegen die Theatercensur. (Telegramm der
                        »Neuen Freien Presse«.)}\pwindex{Goethebund gegen die Theatercensur. (Telegramm der »Neuen Freien
                  Presse«.)1903-03-09@\emph{Der Goethebund gegen die Theatercensur. (Telegramm der »Neuen Freien Presse«.)} {[}1903-03-09{]}|pwk}. In: \emph{Neue Freie
                        Presse}\pwindex{Neue Freie Presse1864 – 1939@\emph{Neue Freie Presse} {[}1864 – 1939{]}|pwk}, Nr. 13.841, 9. 3. 1903,
                     Abendblatt, S. 3–4. }}}\label{K_L03368-2h} noch heut zu ſchicken, muß ihn mir alſo heut{ }Abend auf der Redaktion\oindex{Redaktion des Berliner Tageblatts@\textbf{Redaktion des Berliner Tageblatts}|pw} des Berl. Tagebl.\orgindex{Berliner Tageblatt@Berliner Tageblatt|pw} beſorgen und von dort abſenden. Das
               dauert mindeſtens bis 10. Wo u. wann kann ich Dich \label{K_L03368-3v}\edtext{morgen}{\lemma{\textnormal{\emph{morgen}}}\Cendnote{\textnormal{Am 9. 3. 1903 holte Goldmann\pwindex{Goldmann, Paul 31.01.1865 – 25.09.1935@\textsc{Goldmann, Paul} (31.01.1865 – 25.09.1935), \emph{Schriftsteller, Journalist}|pwk}{ }Schnitzler\pwindex{Schnitzler, Arthur 15.05.1862 – 21.10.1931@\textsc{Schnitzler, Arthur} (15.05.1862 – 21.10.1931), \emph{Schriftsteller, Mediziner}|pwk} und Olga Gussmann\pwindex{Schnitzler, Olga 17.01.1882 – 13.01.1970@\textsc{Schnitzler, Olga} (17.01.1882 – 13.01.1970), \emph{Schauspielerin, Sängerin}|pwk} im Palasthotel\oindex{Palasthotel Berlin@\textbf{Palasthotel Berlin}|pwk} ab und begleitete sie zum Zug Richtung Wien\oindex{Wien@\textbf{Wien}|pwk}.}}}\label{K_L03368-3h} ſehen? Viele herzliche Grüße an Dich und die
               Anderen, namentlich an \textsc{Olga\pwindex{Schnitzler, Olga 17.01.1882 – 13.01.1970@\textsc{Schnitzler, Olga} (17.01.1882 – 13.01.1970), \emph{Schauspielerin, Sängerin}|pw}}. Dein \spacefill\mbox{P. G.}\pend
           
         
         \endnumbering\mylabel{h}\end{ledgroupsized}  \newcommand{\dateiname}{L03368}\newcommand{\titel}{Paul Goldmann an Arthur Schnitzler, 8. 3. 1903}\newcommand{\editorInnen}{Martin Anton Müller und Laura Untner}%% latex-leseansicht-abspann.tex
%% Abspann für die Leseansicht.
%% Der Schalter \ifkorrekturansicht ist bereits durch den Vorspann gesetzt.

%% latex-abspann.tex
%% Gemeinsamer Abspann für Korrekturansicht und Leseansicht.
%% Setzt den Schalter \ifkorrekturansicht voraus (gesetzt in den
%% einbindenden Dateien latex-korrekturansicht-abspann.tex bzw.
%% latex-leseansicht-abspann.tex).
%% ---------------------------------------------------------------

\normalsize

% Das esempio-Environment wird nur in der Leseansicht benötigt
\ifkorrekturansicht\else
\newenvironment{esempio}[3]%
{
    \vspace{1.5ex}
    \rlap{\underline{#1}}
    \par
    \setlength{\parindent}{0cm}
    \nopagebreak
    \leftskip=#2cm
    \rightskip=#3cm
}
{
    \par
}
\fi

\doendnotes{C}
\bigskip
\vfill

\clearpage

\footnotesize

\ifkorrekturansicht
  \lohead{\textsc{register}}
\fi

% theindex-Environment neu definieren ohne reledmac
\makeatletter
\renewenvironment{theindex}{%
  \ifkorrekturansicht
    \section*{\indexname}%
  \else
    \subsubsection*{Index der erwähnten Entitäten}%
  \fi
  \setlength{\parindent}{0pt}%
  \setlength{\parskip}{0pt plus 0.3pt}%
  \let\item\@idxitem
}{%
  \ifkorrekturansicht\clearpage\fi
}
\makeatother

\IfFileExists{\jobname-pw.ind}{\input{\jobname-pw.ind}}{}

% Quellenangabe nur in der Leseansicht
\ifkorrekturansicht\else
% Fallback-Definitionen, falls die .tex-Datei \titel etc. nicht gesetzt hat
\providecommand{\titel}{}
\providecommand{\editorInnen}{}
\providecommand{\dateiname}{\jobname}

\vspace{3cm}

\vfill

\footnotesize
\textsc{Quelle}: \titel. Herausgegeben von {\editorInnen}. In: \emph{Arthur Schnitzler: Briefwechsel mit Autorinnen und Autoren}.
 Digitale Edition, https://schnitzler-briefe.acdh.oeaw.ac.at/{\dateiname}.html (Stand \today)
\fi

\end{document}


      