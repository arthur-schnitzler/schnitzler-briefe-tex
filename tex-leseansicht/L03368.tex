%% latex-korrekturansicht-vorspann.tex
%% Vorspann für die Korrekturansicht.
%% Lädt die gemeinsame Datei latex-vorspann.tex mit gesetztem Schalter.

\newif\ifkorrekturansicht
\korrekturansichttrue

\input{../tex-inputs/latex-vorspann}


\section[ Paul Goldmann an Arthur Schnitzler, 8. 3. 1903]{L03368 Paul Goldmann an Arthur Schnitzler, 8. 3. 1903}
\nopagebreak\mylabel{L03368v}
\rehead{ }\normalsize\beginnumbering\briefempfaengerindex{Schnitzler, Arthur@\textsc{Schnitzler, Arthur}!zzzGoldmann, Paul@\emph{von Paul Goldmann}!1903-03-082@{8. 3. 1903}|(be}
\toendnotes[C]{\smallbreak\pagebreak[2]}\Standort{DLA, A:Schnitzler, HS.NZ85.1.3173.}
\physDesc{Postkarte, 570 Zeichen
\newline{}Handschrift: 1) blaue Tinte, deutsche Kurrent\hspace{1em}2) blaue Tinte, lateinische Kurrent (\noindent{}Adresse)\hspace{1em}
\newline{}Versand: Stempel: »\nobreak{}\oindex{Berlin@\textbf{Berlin}, \emph{P.PPLC}|pwk}Berlin, W. 9, 8. 3. 03., 5–N.\nobreak{}«. Stempel: »\nobreak{}\oindex{Berlin@\textbf{Berlin}, \emph{P.PPLC}|pwk}Berlin\textcolor{gray}{,} O.
                                       P27 (R15), 8 III 03, 5\textsuperscript{30} N\textcolor{gray}{.}\nobreak{}«.  }\toendnotes[C]{\smallbreak}\pstart{}{\pb}Fräulein Elisabeth Gussmann\pend{}\pstart{}für Herrn Dr. Schnitzler\pend{}\pstart{}Wallnertheate\textcolor{gray}{r}straße 40\oindex{Wallnertheaterstrasse@\textbf{Wallnertheaterstraße}, \emph{Straße (K.STR)}|pw}\pend{}\pstart{}II. bei Sternfeld\pwindex{Sternfeld, Max @\textsc{Sternfeld, Max}, \emph{Kaufmann/Kauffrau}|pw}.\pend{}{\bigskip}\vspace{1em}
\pstart
           {\pb}Berlin\oindex{Berlin@\textbf{Berlin}, \emph{P.PPLC}|pw}, 8. März.\hfill Mein lieber Freund, \pend
           \vspace{0.5em}
\pstart
           Ich habe Dich zwei Mal im \textsc{Hotel}\oindex{Palasthotel Berlin@\textbf{Palasthotel Berlin}, \emph{Hotel (K.HTL)}|pwv} geſucht, um Dir zu ſagen, daß ich heut{ }Abend{ }\strikeout{\textcolor{gray}{×}\-\textcolor{gray}{×}\-\textcolor{gray}{×}} leider \label{K_L03368-1v}\edtext{nicht kommen}{\lemma{\textnormal{\emph{nicht kommen}}}\Cendnote{\textnormal{vermutlich zu Elisabeth Gussmann\pwindex{Steinrueck, Elisabeth 19.11.1885 – 07.04.1920@\textsc{Steinrück, Elisabeth} (19.11.1885 – 07.04.1920)|pwk} – dafür spricht der \emph{Tagebuch}\pwindex{Tagebuch@\emph{Tagebuch}|pwk}-Eintrag zum 8. 3. 1903 und die Adressierung der Postkarte an
                  sie}}}\label{K_L03368-1} kann. Ich erhielt heut{ }Morgen telegraphiſchen Auftrag aus Wien\oindex{Wien@\textbf{Wien}, \emph{A.ADM2}|pw}, den \label{K_L03368-2v}\edtext{Bericht\pwindex{Goethebund gegen die Theatercensur. (Telegramm der »Neuen Freien Presse«.)@\emph{Der Goethebund gegen die Theatercensur. (Telegramm der »Neuen Freien Presse«.)}|pwv} über die Goethebund\orgindex{Goethe-Bund@Goethe-Bund|pw}-Verſammlung}{\lemma{\textnormal{\emph{Bericht … Goethebund-Verſammlung}}}\Cendnote{\textnormal{Der deutsch\oindex{Deutschland@\textbf{Deutschland}, \emph{A.PCLI}|pwkv}e \emph{Goethe-Bund}\orgindex{Goethe-Bund@Goethe-Bund|pwk} tagte am 8. 3. 1903 in der Alten Berliner Philharmonie\oindex{Alte Philharmonie [Berlin]@\textbf{Alte Philharmonie [Berlin]}, \emph{Konzertsaal (K.KNZ)}|pwk}. [Paul Goldmann\pwindex{Goldmann, Paul 31.01.1865 – 25.09.1935@\textsc{Goldmann, Paul} (31.01.1865 – 25.09.1935), \emph{Schriftsteller/Schriftstellerin, Journalist/Journalistin}|pwk}]: \emph{Der Goethebund gegen
                        die Theatercensur. (Telegramm der »Neuen Freien Presse«)}\pwindex{Goethebund gegen die Theatercensur. (Telegramm der »Neuen Freien Presse«.)@\emph{Der Goethebund gegen die Theatercensur. (Telegramm der »Neuen Freien Presse«.)}|pwk}. In: \emph{Neue Freie Presse}\pwindex{Neue Freie Presse@\emph{Neue Freie Presse}|pwk}, Nr. 13.841, 9. 3. 1903, Abendblatt, S. 3–4. }}}\label{K_L03368-2} noch
                  heut zu ſchicken, muß ihn mir alſo heut{ }Abend auf der Redaktion\oindex{Redaktion des Berliner Tageblatts@\textbf{Redaktion des Berliner Tageblatts}, \emph{Redaktionsgebäude (K.RDK)}|pw} des Berl. Tagebl.\orgindex{Berliner Tageblatt@Berliner Tageblatt|pw} beſorgen und von dort abſenden. Das
               dauert mindeſtens bis 10. Wo u. wann kann ich Dich \label{K_L03368-3v}\edtext{morgen}{\lemma{\textnormal{\emph{morgen}}}\Cendnote{\textnormal{Am 9. 3. 1903 holte Goldmann\pwindex{Goldmann, Paul 31.01.1865 – 25.09.1935@\textsc{Goldmann, Paul} (31.01.1865 – 25.09.1935), \emph{Schriftsteller/Schriftstellerin, Journalist/Journalistin}|pwk}{ }Schnitzler und Olga Gussmann\pwindex{Schnitzler, Olga 17.01.1882 – 13.01.1970@\textsc{Schnitzler, Olga} (17.01.1882 – 13.01.1970), \emph{Schauspieler/Schauspielerin, Sänger/Sängerin}|pwk} im Palasthotel\oindex{Palasthotel Berlin@\textbf{Palasthotel Berlin}, \emph{Hotel (K.HTL)}|pwk} ab und begleitete sie zum Zug Richtung Wien\oindex{Wien@\textbf{Wien}, \emph{A.ADM2}|pwk}.}}}\label{K_L03368-3} ſehen? Viele herzliche Grüße an Dich und die
               Anderen, namentlich an \textsc{Olga\pwindex{Schnitzler, Olga 17.01.1882 – 13.01.1970@\textsc{Schnitzler, Olga} (17.01.1882 – 13.01.1970), \emph{Schauspieler/Schauspielerin, Sänger/Sängerin}|pw}}. Dein \spacefill\mbox{P. G.}\pend
           \selectlanguage{ngerman}\endnumbering\briefempfaengerindex{Schnitzler, Arthur@\textsc{Schnitzler, Arthur}!zzzGoldmann, Paul@\emph{von Paul Goldmann}!1903-03-082@{8. 3. 1903}|)be}\mylabel{L03368h}  \normalsize

\doendnotes{C}
\bigskip
\vfill

\clearpage

\footnotesize

\lohead{\textsc{register}}

% Definiere theindex-Environment komplett neu ohne reledmac
\makeatletter
\renewenvironment{theindex}{%
  \section*{\indexname}%
  \setlength{\parindent}{0pt}%
  \setlength{\parskip}{0pt plus 0.3pt}%
  \let\item\@idxitem
}{%
  \clearpage
}
\makeatother

\IfFileExists{\jobname-pw.ind}{\input{\jobname-pw.ind}}{}

\end{document}

      