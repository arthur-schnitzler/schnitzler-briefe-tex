%% latex-leseansicht-vorspann.tex
%% Vorspann für die Leseansicht.
%% Lädt die gemeinsame Datei latex-vorspann.tex mit nicht gesetztem Schalter.

\newif\ifkorrekturansicht
\korrekturansichtfalse

\input{../tex-inputs/latex-vorspann}


\section[ Paul Goldmann an Arthur Schnitzler, 8. 3. 1903]{L03368 Paul Goldmann an Arthur Schnitzler,  8. 3. 1903}
\nopagebreak\mylabel{L03368v}
\rehead{ }\normalsize\beginnumbering\briefempfaengerindex{Schnitzler, Arthur@\textsc{Schnitzler, Arthur}!zzzGoldmann, Paul@\emph{von Paul Goldmann}!1903-03-082@{8. 3. 1903}|(be}
\toendnotes[C]{\smallbreak\pagebreak[2]}
\correspDesc{Versand  durch Paul Goldmann am 8. 3. 1903 in Berlin
\newline{}Erhalt  durch Arthur Schnitzler am 8. 3. 1903 in Berlin}\toendnotes[C]{\smallbreak}
\Standort{DLA, A:Schnitzler, HS.NZ85.1.3173.}
\physDesc{Postkarte, 570 Zeichen
\newline{}Handschrift: blaue Tinte, deutsche Kurrent
\newline{}Versand: Stempel: »\nobreak{}\oindex{Berlin@\textbf{Berlin}, \emph{Hauptstadt}|pwk}Berlin, W. 9, 8. 3. 03., 5–N.\nobreak{}«. Stempel: »\nobreak{}\oindex{Berlin@\textbf{Berlin}, \emph{Hauptstadt}|pwk}Berlin\textcolor{gray}{,} O.
                                       P27 (R15), 8 III 03, 5\textsuperscript{30} N\textcolor{gray}{.}\nobreak{}«.  }\toendnotes[C]{\smallbreak}\pstart{}\textsc{{\pb}Fräulein Elisabeth Gussmann}\pend{}\pstart{}\textsc{für Herrn Dr. Schnitzler}\pend{}\pstart{}\textsc{Wallnertheate\textcolor{gray}{r}straße 40\oindex{Wallnertheaterstraße@\textbf{Wallnertheaterstraße}, \emph{Straße}|pw}}\pend{}\pstart{}\textsc{II. bei Sternfeld\pwindex{Sternfeld, Max @\textsc{Sternfeld, Max}, \emph{Kaufmann}|pw}.}\pend{}{\bigskip}\vspace{1em}
\pstart
           {\pb}Berlin\oindex{Berlin@\textbf{Berlin}, \emph{Hauptstadt}|pw}, 8. März.\hfill Mein lieber Freund,\pend
           \vspace{0.5em}
\pstart
           Ich habe Dich zwei Mal im \textsc{Hotel}\oindex{Palasthotel Berlin@\textbf{Palasthotel Berlin}, \emph{Hotel}|pwv} geſucht, um Dir zu{ }ſagen, daß ich heut{ }Abend{ }\strikeout{\textcolor{gray}{×}\-\textcolor{gray}{×}\-\textcolor{gray}{×}} leider \label{K_L03368-1v}\edtext{nicht kommen}{\lemma{\textnormal{\emph{nicht kommen}}}\Cendnote{\textnormal{vermutlich zu Elisabeth Gussmann\pwindex{Steinrück, Elisabeth 19.\,11.\,1885 – 7.\,4.\,1920 Partenkirchen@\textsc{Steinrück, Elisabeth} (19.\,11.\,1885 – 7.\,4.\,1920 Partenkirchen)|pwk} – dafür spricht der \emph{Tagebuch}\pwindex{Schnitzler, Arthur 15.\,5.\,1862 Wien – 21.\,10.\,1931 ebd.@\textsc{Schnitzler, Arthur} (15.\,5.\,1862 Wien – 21.\,10.\,1931 ebd.), \emph{Schriftsteller, Mediziner}!Tagebuch@\strich\emph{Tagebuch}|pwk}-Eintrag zum 8. 3. 1903 und die Adressierung der Postkarte an
                  sie}}}\label{K_L03368-1} kann. Ich erhielt heut{ }Morgen telegraphiſchen Auftrag aus Wien\oindex{Wien@\textbf{Wien}, \emph{Verwaltungsgebiet}|pw}, den \label{K_L03368-2v}\edtext{Bericht\pwindex{Goethebund gegen die Theatercensur. (Telegramm der »Neuen Freien Presse«.)@\emph{Der Goethebund gegen die Theatercensur. (Telegramm der »Neuen Freien Presse«.)}|pwv} über die Goethebund\orgindex{Goethe-Bund@Goethe-Bund|pw}-Verſammlung}{\lemma{\textnormal{\emph{Bericht … Goethebund-Versammlung}}}\Cendnote{\textnormal{Der deutsch\oindex{Deutschland@\textbf{Deutschland}|pwkv}e \emph{Goethe-Bund}\orgindex{Goethe-Bund@Goethe-Bund|pwk} tagte am 8. 3. 1903 in der Alten Berliner Philharmonie\oindex{Alte Philharmonie [Berlin]@\textbf{Alte Philharmonie [Berlin]}, \emph{Konzertsaal}|pwk}. [Paul Goldmann\pwindex{Goldmann, Paul 31.\,1.\,1865 Breslau – 25.\,9.\,1935 Wien@\textsc{Goldmann, Paul} (31.\,1.\,1865 Breslau – 25.\,9.\,1935 Wien), \emph{Schriftsteller, Journalist}|pwk}]: \emph{Der Goethebund gegen
                        die Theatercensur. (Telegramm der »Neuen Freien Presse«)}\pwindex{Goethebund gegen die Theatercensur. (Telegramm der »Neuen Freien Presse«.)@\emph{Der Goethebund gegen die Theatercensur. (Telegramm der »Neuen Freien Presse«.)}|pwk}. In: \emph{Neue Freie Presse}\pwindex{Neue Freie Presse@\emph{Neue Freie Presse}|pwk}, Nr. 13.841, 9. 3. 1903, Abendblatt, S. 3–4. }}}\label{K_L03368-2} noch
                  heut zu{ }ſchicken, muß ihn mir alſo heut{ }Abend auf der Redaktion\oindex{Redaktion des Berliner Tageblatts@\textbf{Redaktion des Berliner Tageblatts}, \emph{Redaktionsgebäude}|pw} des Berl. Tagebl.\orgindex{Berliner Tageblatt@Berliner Tageblatt|pw} beſorgen und von dort abſenden. Das
               dauert mindeſtens bis 10. Wo u. wann kann ich Dich \label{K_L03368-3v}\edtext{morgen}{\lemma{\textnormal{\emph{morgen}}}\Cendnote{\textnormal{Am 9. 3. 1903 holte Goldmann\pwindex{Goldmann, Paul 31.\,1.\,1865 Breslau – 25.\,9.\,1935 Wien@\textsc{Goldmann, Paul} (31.\,1.\,1865 Breslau – 25.\,9.\,1935 Wien), \emph{Schriftsteller, Journalist}|pwk}{ }Schnitzler und Olga Gussmann\pwindex{Schnitzler, Olga 17.\,1.\,1882 Wien – 13.\,1.\,1970 Lugano@\textsc{Schnitzler, Olga} (17.\,1.\,1882 Wien – 13.\,1.\,1970 Lugano), \emph{Schauspielerin, Sängerin}|pwk} im Palasthotel\oindex{Palasthotel Berlin@\textbf{Palasthotel Berlin}, \emph{Hotel}|pwk} ab und begleitete sie zum Zug Richtung Wien\oindex{Wien@\textbf{Wien}, \emph{Verwaltungsgebiet}|pwk}.}}}\label{K_L03368-3}{ }ſehen? Viele herzliche Grüße an Dich und die
               Anderen, namentlich an \textsc{Olga\pwindex{Schnitzler, Olga 17.\,1.\,1882 Wien – 13.\,1.\,1970 Lugano@\textsc{Schnitzler, Olga} (17.\,1.\,1882 Wien – 13.\,1.\,1970 Lugano), \emph{Schauspielerin, Sängerin}|pw}}. Dein \spacefill\mbox{P. G.}\pend
           \selectlanguage{ngerman}\endnumbering\briefempfaengerindex{Schnitzler, Arthur@\textsc{Schnitzler, Arthur}!zzzGoldmann, Paul@\emph{von Paul Goldmann}!1903-03-082@{8. 3. 1903}|)be}\mylabel{L03368h}  \newcommand{\dateiname}{L03368}\newcommand{\titel}{Paul Goldmann an Arthur Schnitzler, 8. 3. 1903}\newcommand{\editorInnen}{Martin Anton Müller und Laura Untner}%% latex-leseansicht-abspann.tex
%% Abspann für die Leseansicht.
%% Der Schalter \ifkorrekturansicht ist bereits durch den Vorspann gesetzt.

%% latex-abspann.tex
%% Gemeinsamer Abspann für Korrekturansicht und Leseansicht.
%% Setzt den Schalter \ifkorrekturansicht voraus (gesetzt in den
%% einbindenden Dateien latex-korrekturansicht-abspann.tex bzw.
%% latex-leseansicht-abspann.tex).
%% ---------------------------------------------------------------

\normalsize

% Das esempio-Environment wird nur in der Leseansicht benötigt
\ifkorrekturansicht\else
\newenvironment{esempio}[3]%
{
    \vspace{1.5ex}
    \rlap{\underline{#1}}
    \par
    \setlength{\parindent}{0cm}
    \nopagebreak
    \leftskip=#2cm
    \rightskip=#3cm
}
{
    \par
}
\fi

\doendnotes{C}
\bigskip
\vfill

\clearpage

\footnotesize

\ifkorrekturansicht
  \lohead{\textsc{register}}
\fi

% theindex-Environment neu definieren ohne reledmac
\makeatletter
\renewenvironment{theindex}{%
  \ifkorrekturansicht
    \section*{\indexname}%
  \else
    \subsubsection*{Index der erwähnten Entitäten}%
  \fi
  \setlength{\parindent}{0pt}%
  \setlength{\parskip}{0pt plus 0.3pt}%
  \let\item\@idxitem
}{%
  \ifkorrekturansicht\clearpage\fi
}
\makeatother

\IfFileExists{\jobname-pw.ind}{\input{\jobname-pw.ind}}{}

% Quellenangabe nur in der Leseansicht
\ifkorrekturansicht\else
% Fallback-Definitionen, falls die .tex-Datei \titel etc. nicht gesetzt hat
\providecommand{\titel}{}
\providecommand{\editorInnen}{}
\providecommand{\dateiname}{\jobname}

\vspace{3cm}

\vfill

\footnotesize
\textsc{Quelle}: \titel. Herausgegeben von {\editorInnen}. In: \emph{Arthur Schnitzler: Briefwechsel mit Autorinnen und Autoren}.
 Digitale Edition, https://schnitzler-briefe.acdh.oeaw.ac.at/{\dateiname}.html (Stand \today)
\fi

\end{document}


