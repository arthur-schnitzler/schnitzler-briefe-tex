%% latex-leseansicht-vorspann.tex
%% Vorspann für die Leseansicht.
%% Lädt die gemeinsame Datei latex-vorspann.tex mit nicht gesetztem Schalter.

\newif\ifkorrekturansicht
\korrekturansichtfalse

\input{../tex-inputs/latex-vorspann}


         
         \renewcommand{\erwaehntePersonen}{Personen: Hugo von Hofmannsthal, Gertrude von Hofmannsthal, Christiane Zimmer}
         \renewcommand{\erwaehnteOrte}{Orte: Dubrovnik, Maria Santissima Assunta, Mittelmeer, Monte Pellegrino, Palermo, Sternwartestraße 71, Wien, Österreich}
         \renewcommand{\erwaehnteWerke}{}
               \section[Hugo, Gerty und Christiane Hofmannsthal an Arthur Schnitzler, 8. 5. 1924]{ Hugo, Gerty und Christiane Hofmannsthal an Arthur Schnitzler,
               8. 5. 1924}\nopagebreak\mylabel{v}\rehead{ }\begin{ledgroupsized}[t]{13cm}\normalsize\beginnumbering\briefempfaengerindex{Schnitzler, Arthur@\textsc{Schnitzler, Arthur}!zzzZimmer, Christiane@\emph{von Christiane Zimmer}!1924-05-081@{8. 5. 1924}|(be}\briefempfaengerindex{Schnitzler, Arthur@\textsc{Schnitzler, Arthur}!zzzHofmannsthal, Gertrude von@\emph{von Gertrude von Hofmannsthal}!1924-05-081@{8. 5. 1924}|(be}\briefempfaengerindex{Schnitzler, Arthur@\textsc{Schnitzler, Arthur}!zzzHofmannsthal, Hugo von@\emph{von Hugo von Hofmannsthal}!1924-05-081@{8. 5. 1924}|(be} \toendnotes[C]{\smallbreak\pagebreak[2]} \Standort{CUL, Schnitzler, B 43.}
\physDesc{Bildpostkarte, 175 Zeichen
\newline{}Handschrift Hugo von Hofmannsthal: Bleistift, lateinische Kurrent\newline{}Handschrift Gertrude von Hofmannsthal: schwarze Tinte, lateinische Kurrent\newline{}Handschrift Christiane Zimmer: schwarze Tinte
\newline{}Versand: Stempel: »\nobreak{}\oindex{Palermo@\textbf{Palermo}|pwk}Palermo Porto, 8. 5. 24, 17\nobreak{}«.  
\newline{}Ordnung: 1) mit Bleistift von unbekannter Hand die Jahreszahl ergänzt: »\strikeout{23.} 1924«  2) mit Bleistift von unbekannter Hand nummeriert: »\strikeout{390}« 3) mit Bleistift von unbekannter Hand nummeriert: »\strikeout{370}« 4) mit Bleistift von unbekannter Hand nummeriert:
                                    »374«}\buchAbdrucke{\weitereDrucke{Hugo von Hofmannsthal, Arthur Schnitzler: \emph{Briefwechsel}. Hg. Therese Nickl und Heinrich Schnitzler. Frankfurt am Main: \emph{S. Fischer} 1964, S. 299.} }\toendnotes[C]{\smallbreak}\pstart{}{\pb}D\textsuperscript{r}
                  Arthur Schnitzler\pend{}\pstart{}Wien\oindex{Wien@\textbf{Wien}|pw}\pend{}\pstart{}XVIII Sternwartestrasse 71\oindex{Sternwartestrasse 71@\textbf{Sternwartestraße 71}|pw}\pend{}\pstart{}Austria\oindex{Oesterreich@\textbf{Österreich}|pw}\pend{}{\bigskip}\pstart
           \noindent{}\centering{}{\pb}\textcolor{gray}{\textbf{PALERMO\oindex{Palermo@\textbf{Palermo}|pw} – Panorama colla Cattedral\oindex{Maria Santissima Assunta@\textbf{Maria Santissima Assunta}|pw}e e Monte Pellegrino\oindex{Monte Pellegrino@\textbf{Monte Pellegrino}|pw}}}\pend
           \pstart
           \centering{}{\pb}8 V\pend
           \pstart
           Einen Gruss zur Eri{\geminationn}erung an Ihre Mittelmeer\oindex{Mittelmeer@\textbf{Mittelmeer}|pw}fahrt auf der Sie mich in \label{K_L02412-1v}\edtext{Ragusa\oindex{Dubrovnik@\textbf{Dubrovnik}|pw}}{\lemma{\textnormal{\emph{Ragusa}}}\Cendnote{\textnormal{Vgl. A. S.: \emph{Tagebuch}, 14. 3. 1905.
               }}}\label{K_L02412-1h} besuchten! \pend
           \pstart
           Ihr \spacefill\mbox{Hugo}{\\[\baselineskip]}\spacefill\mbox{{[}hs. Zimmer:{]} Christiane}{\\[\baselineskip]}{[}hs. Gertrude von Hofmannsthal:{]} Herzlichst \spacefill\mbox{Gerty}\pend
           \leftskip=0em{}
         
         \endnumbering\mylabel{h}\end{ledgroupsized}  \newcommand{\dateiname}{L02412}\newcommand{\titel}{Hugo, Gerty und Christiane Hofmannsthal an Arthur Schnitzler, 8. 5. 1924}\newcommand{\editorInnen}{Martin Anton Müller und Gerd-Hermann Susen}%% latex-leseansicht-abspann.tex
%% Abspann für die Leseansicht.
%% Der Schalter \ifkorrekturansicht ist bereits durch den Vorspann gesetzt.

%% latex-abspann.tex
%% Gemeinsamer Abspann für Korrekturansicht und Leseansicht.
%% Setzt den Schalter \ifkorrekturansicht voraus (gesetzt in den
%% einbindenden Dateien latex-korrekturansicht-abspann.tex bzw.
%% latex-leseansicht-abspann.tex).
%% ---------------------------------------------------------------

\normalsize

% Das esempio-Environment wird nur in der Leseansicht benötigt
\ifkorrekturansicht\else
\newenvironment{esempio}[3]%
{
    \vspace{1.5ex}
    \rlap{\underline{#1}}
    \par
    \setlength{\parindent}{0cm}
    \nopagebreak
    \leftskip=#2cm
    \rightskip=#3cm
}
{
    \par
}
\fi

\doendnotes{C}
\bigskip
\vfill

\clearpage

\footnotesize

\ifkorrekturansicht
  \lohead{\textsc{register}}
\fi

% theindex-Environment neu definieren ohne reledmac
\makeatletter
\renewenvironment{theindex}{%
  \ifkorrekturansicht
    \section*{\indexname}%
  \else
    \subsubsection*{Index der erwähnten Entitäten}%
  \fi
  \setlength{\parindent}{0pt}%
  \setlength{\parskip}{0pt plus 0.3pt}%
  \let\item\@idxitem
}{%
  \ifkorrekturansicht\clearpage\fi
}
\makeatother

\IfFileExists{\jobname-pw.ind}{\input{\jobname-pw.ind}}{}

% Quellenangabe nur in der Leseansicht
\ifkorrekturansicht\else
% Fallback-Definitionen, falls die .tex-Datei \titel etc. nicht gesetzt hat
\providecommand{\titel}{}
\providecommand{\editorInnen}{}
\providecommand{\dateiname}{\jobname}

\vspace{3cm}

\vfill

\footnotesize
\textsc{Quelle}: \titel. Herausgegeben von {\editorInnen}. In: \emph{Arthur Schnitzler: Briefwechsel mit Autorinnen und Autoren}.
 Digitale Edition, https://schnitzler-briefe.acdh.oeaw.ac.at/{\dateiname}.html (Stand \today)
\fi

\end{document}


      