%% latex-leseansicht-vorspann.tex
%% Vorspann für die Leseansicht.
%% Lädt die gemeinsame Datei latex-vorspann.tex mit nicht gesetztem Schalter.

\newif\ifkorrekturansicht
\korrekturansichtfalse

\input{../tex-inputs/latex-vorspann}


\section[Stefan Zweig an Arthur Schnitzler, 23. 5. 1913]{L03641 Stefan Zweig an Arthur Schnitzler, 23. 5. 1913}
\nopagebreak\mylabel{L03641v}
\rehead{ }\normalsize\beginnumbering\briefempfaengerindex{Schnitzler, Arthur@\textsc{Schnitzler, Arthur}!zzzZweig, Stefan@\emph{von Stefan Zweig}!1913-05-231@{23. 5. 1913}|(be}
\toendnotes[C]{\smallbreak\pagebreak[2]}
\correspDesc{Versand  durch Stefan Zweig am 23. 5. 1913 in Wien
\newline{}Erhalt  durch Arthur Schnitzler im Zeitraum [23. 5. 1913
                  – 26. 5. 1913?] in Wien}\toendnotes[C]{\smallbreak}
\Standort{CUL, Schnitzler, B 118.}
\physDesc{Brief, 2 Blätter, 6 Seiten, 3918 Zeichen
\newline{}Handschrift: lila Tinte, lateinische Kurrent
\newline{}Schnitzler: 1) mit Bleistift »\textsc{Zweig}«  2) mit rotem Buntstift eine Unterstreichung}
\buchAbdrucke{\weitereDrucke{1) Stefan Zweig: \emph{Briefwechsel mit Hermann Bahr, Sigmund Freud, Rainer Maria
                        Rilke und Arthur Schnitzler}. Herausgegeben von Jeffrey B. Berlin, Hans-Ulrich Lindken und Donald A. Prater. Frankfurt am Main: \emph{S. Fischer} 1987, S. 375–377.} \weitereDrucke{2) Stefan Zweig: \emph{Briefe. Bd. I: 1897–1914}. Herausgegeben von Knut Beck, Jeffrey B. Berlin und Natascha Weschenbach-Feggeler. Frankfurt am Main: \emph{S. Fischer} 1995, S. 273–275.} \weitereDrucke{3) Hermann Bahr, Arthur Schnitzler: \emph{Briefwechsel, Aufzeichnungen, Dokumente (1891–1931)}. Herausgegeben von Kurt Ifkovits und Martin Anton Müller. Göttingen: \emph{Wallstein} 2018, S. 486–487.} }\toendnotes[C]{\smallbreak}
\pstart
           {\pb}\textcolor{gray}{\textbf{SZ}}\hfill \textcolor{gray}{\textbf{VIII. KOCHGASSE\oindex{Wien@\textbf{Wien}!VIII., Josefstadt@\textbf{VIII., Josefstadt}!Kochgasse 8@\textbf{Kochgasse 8}, \emph{Wohngebäude}|pw}}}\pend
           
\pstart
           \raggedleft{}\textcolor{gray}{\textbf{WIEN\oindex{Wien@\textbf{Wien}, \emph{Verwaltungsgebiet}|pw},{ }23. V. 13}}\pend
           \vspace{0.5em}
\pstart
           Verehrter Herr Doktor, ich habe soeben Ihre neue Novelle\pwindex{Schnitzler, Arthur 15.\,5.\,1862 Wien – 21.\,10.\,1931 ebd.@\textsc{Schnitzler, Arthur} (15.\,5.\,1862 Wien – 21.\,10.\,1931 ebd.), \emph{Schriftsteller, Mediziner}!Frau Beate und ihr Sohn. Novelle@\strich\emph{Frau Beate und ihr Sohn. Novelle}|pwv} empfangen und keine eigene Arbeit
               ist mir so wichtig, um nicht sofort für so liebe Lectüre unterbrochen zu werden. In
               einem Zug von Anfang bis zu Ende, hatte ich doch nachher das Gefühl einer grossen
               Fülle, das einen immer überkommt, wenn man Existenzen nicht an einem zufälligen
               Punkte ihres Schicksals anstreift sondern durchlebt bis zu jenem innersten Kern, in
               dem die ganze Summe ihres Lebens in stärkstem Extract eingepresst ist. Nichts ist
               darin eigentlich episodisch, sondern Alles so zum {\pb}Notwendigen herangedrängt, dass – wie in
               jedem vollendeten epischen Werk – es gar keine Hauptfigur mehr gibt, sondern jeder
               von seiner Seite das Geschehen beherrscht und ihr Gegeneinanderspiel zu einem
               harmonischen Kampf der Kräfte wird. Reute es mich im ersten und mittleren Teile, das
               Geschehnis nicht dramatisch gestaltet zu sehen – ich habe das Empfinden, als hätte
               sich der Stoff ihnen zuerst dramatisch dargestellt, so stark ist das \strikeout{Entg} plastische Entgegentreten der Figuren – der
               Schluss überzeugte mich durch seine Harmonie, dass hier die Form zu wählen war, die
               Novelle das einzig mögliche, weil nur sie die erhabene Beschwichtigung so erregter
               Gefühle duldet. Ich muss für sie befürchten, dass Sie auf manche Gegnerschaft gera{\pb}de diesmal stossen werden, weil Sie in
               so grosser Wahrhaftigkeit dem primitiv Sexuellen entgegengetreten sind, indess die
               meisten Menschen aus einer merkwürdigen innern Verlogenheit jede ihrer rein sexuellen
               Empfindungen mit dem Begriff Liebe verbrämen und \strikeout{sie}
               selbst im Kunstwerk das reine na{[}c{]}kte Blutgefühl nicht dulden wollen: sie verwandeln
               dann gern ein falsches Schamgefühl in moralische oder ästhetische Abneigung, indess
               ich gerade jene Intensität des Körperlichen in Verbindung mit der atmosphärischen
               Elektricität dieser (wundervoll hingemalten) Sommertage als stärkste Wahrheit dieses
                  Werkes\pwindex{Schnitzler, Arthur 15.\,5.\,1862 Wien – 21.\,10.\,1931 ebd.@\textsc{Schnitzler, Arthur} (15.\,5.\,1862 Wien – 21.\,10.\,1931 ebd.), \emph{Schriftsteller, Mediziner}!Frau Beate und ihr Sohn. Novelle@\strich\emph{Frau Beate und ihr Sohn. Novelle}|pwv} empfinde. Der
               Schluss hat freilich auch mich im ersten Lesen befremdet, doch bin ich meiner hier
               weniger sicher als Ihrer und zweifle nicht, dass eine zweite und {\pb}nun weniger von der Spannung nach
               vorwärts gejagte Lectüre mir die Notwendigkeit fühlbarer machen wird. Das Motiv der
               Entladung aufgestauter erotischer Kräfte, das die Garlan\pwindex{Schnitzler, Arthur 15.\,5.\,1862 Wien – 21.\,10.\,1931 ebd.@\textsc{Schnitzler, Arthur} (15.\,5.\,1862 Wien – 21.\,10.\,1931 ebd.), \emph{Schriftsteller, Mediziner}!Frau Bertha Garlan. Roman@\strich\emph{Frau Bertha Garlan. Roman}|pw} und Das weite Land\pwindex{Schnitzler, Arthur 15.\,5.\,1862 Wien – 21.\,10.\,1931 ebd.@\textsc{Schnitzler, Arthur} (15.\,5.\,1862 Wien – 21.\,10.\,1931 ebd.), \emph{Schriftsteller, Mediziner}!weite Land. Tragikomödie in fünf Akten@\strich\emph{Das weite Land. Tragikomödie in fünf Akten}|pw} schon so
               prachtvoll ausbildeten, ist hier zu wundervoller Vehemenz geworden und ich freue
               mich, dass wir an Ihnen gerade in jenen Jahren, wo die Dichter sonst gemessen und
               vorsichtig werden, ebenso wie im \label{K_L03641-1v}\edtext{Bernhardi\pwindex{Schnitzler, Arthur 15.\,5.\,1862 Wien – 21.\,10.\,1931 ebd.@\textsc{Schnitzler, Arthur} (15.\,5.\,1862 Wien – 21.\,10.\,1931 ebd.), \emph{Schriftsteller, Mediziner}!Professor Bernhardi. Komödie in fünf Akten@\strich\emph{Professor Bernhardi. Komödie in fünf Akten}|pw}}{\lemma{\textnormal{\emph{Bernhardi}}}\Cendnote{\textnormal{In der erhaltenen Korrespondenz von Schnitzler und Zweig\pwindex{Zweig, Stefan 28.\,11.\,1881 Wien – 23.\,2.\,1942 Petrópolis@\textsc{Zweig, Stefan} (28.\,11.\,1881 Wien – 23.\,2.\,1942 Petrópolis), \emph{Schriftsteller}|pwk} ist \emph{Professor
                        Bernhardi}\pwindex{Schnitzler, Arthur 15.\,5.\,1862 Wien – 21.\,10.\,1931 ebd.@\textsc{Schnitzler, Arthur} (15.\,5.\,1862 Wien – 21.\,10.\,1931 ebd.), \emph{Schriftsteller, Mediziner}!Professor Bernhardi. Komödie in fünf Akten@\strich\emph{Professor Bernhardi. Komödie in fünf Akten}|pwk} kaum thematisiert (XXXX Auszeichnungsfehler: Dokument L03784 nicht gefunden, XXXX Auszeichnungsfehler: Dokument L03639 nicht gefunden). Im Nachlass
                     Zweigs\pwindex{Zweig, Stefan 28.\,11.\,1881 Wien – 23.\,2.\,1942 Petrópolis@\textsc{Zweig, Stefan} (28.\,11.\,1881 Wien – 23.\,2.\,1942 Petrópolis), \emph{Schriftsteller}|pwk}, in dem sich die
                  Korrespondenzstücke Schnitzlers befinden,
                  sind auch zwei Briefe zu einer möglichen Lesung von \emph{Professor Bernhardi}\pwindex{Schnitzler, Arthur 15.\,5.\,1862 Wien – 21.\,10.\,1931 ebd.@\textsc{Schnitzler, Arthur} (15.\,5.\,1862 Wien – 21.\,10.\,1931 ebd.), \emph{Schriftsteller, Mediziner}!Professor Bernhardi. Komödie in fünf Akten@\strich\emph{Professor Bernhardi. Komödie in fünf Akten}|pwk} enthalten. Ob Zweig\pwindex{Zweig, Stefan 28.\,11.\,1881 Wien – 23.\,2.\,1942 Petrópolis@\textsc{Zweig, Stefan} (28.\,11.\,1881 Wien – 23.\,2.\,1942 Petrópolis), \emph{Schriftsteller}|pwk} diese von Schnitzler zu einem nicht genauer bestimmbaren Zeitpunkt erhalten, oder
                  ob sie Zweig\pwindex{Zweig, Stefan 28.\,11.\,1881 Wien – 23.\,2.\,1942 Petrópolis@\textsc{Zweig, Stefan} (28.\,11.\,1881 Wien – 23.\,2.\,1942 Petrópolis), \emph{Schriftsteller}|pwk} anderwärtig erworben hat,
                  bleibt ungeklärt. Der erste ist an Hugo
                     Heller\pwindex{Heller, Hugo 8.\,5.\,1870 Székesfehérvár – 29.\,11.\,1923 Wien@\textsc{Heller, Hugo} (8.\,5.\,1870 Székesfehérvár – 29.\,11.\,1923 Wien), \emph{Verleger, Buchhändler}|pwk} gerichtet (1 Blatt, 2 Seiten, Maschinschrift mit eigenhändiger
                  Unterschrift und minimalen Korrekturen in schwarzer Tinte): »\textcolor{gray}{\textbf{Dr. Arthur Schnitzler}}{ / }\textcolor{gray}{\textbf{Wien XVIII. Sternwartestrasse 71\oindex{Wien@\textbf{Wien}!XVIII., Währing@\textbf{XVIII., Währing}!Sternwartestraße 71@\textbf{Sternwartestraße 71}, \emph{Wohngebäude}|pw}}}{ / }29. 4. 1913.{ / }Sehr geehrter Herr Heller\pwindex{Heller, Hugo 8.\,5.\,1870 Székesfehérvár – 29.\,11.\,1923 Wien@\textsc{Heller, Hugo} (8.\,5.\,1870 Székesfehérvár – 29.\,11.\,1923 Wien), \emph{Verleger, Buchhändler}|pw}.{ / }Ich sende Ihnen hier das Schreiben Ihrer Bezirkshauptmannschaft\orgindex{Bezirkshauptmannschaft Leitmeritz@Bezirkshauptmannschaft Leitmeritz|pwv} mit bestem Dank zurück und
                        lege zugleich eine Abschrift der am 31. Jänner d. J. an die
                        Direktion des Deutschen Volkstheaters\orgindex{Volkstheater@Volkstheater|pw}
                        gelangten Rekursbeantwortung bei. Sie werden gewiss nicht ohne Heiterkeit
                        bemerken, dass Ihrer Bezirkshauptmannschaft\orgindex{Bezirkshauptmannschaft Leitmeritz@Bezirkshauptmannschaft Leitmeritz|pwv} anlässlich des ›Bernhardi\pwindex{Schnitzler, Arthur 15.\,5.\,1862 Wien – 21.\,10.\,1931 ebd.@\textsc{Schnitzler, Arthur} (15.\,5.\,1862 Wien – 21.\,10.\,1931 ebd.), \emph{Schriftsteller, Mediziner}!Professor Bernhardi. Komödie in fünf Akten@\strich\emph{Professor Bernhardi. Komödie in fünf Akten}|pw}‹ nahezu wörtlich dasselbe eingefallen ist
                        wie dem k. k. Ministerium des Innern\orgindex{Ministerium für Inneres@Ministerium für Inneres|pw},
                        und zweifeln natürlich so wenig wie ich daran, dass der Gleichlaut der
                        beiden behördlichen Zuschriften nicht etwa auf eine von höherem Ort
                        ergangene Weisung, sondern nur auf die rührende Seelen- und
                        Geistesverwandtschaft zwischen den Leitmeritzer\oindex{Litoměřice@\textbf{Litoměřice}|pw} und Wiener\oindex{Wien@\textbf{Wien}, \emph{Verwaltungsgebiet}|pw} Beamten,
                        zwischen einer gut österreichischen\oindex{Österreich@\textbf{Österreich}|pw}{ }Bezirkshauptmannschaft\orgindex{Bezirkshauptmannschaft Leitmeritz@Bezirkshauptmannschaft Leitmeritz|pwv} und einem ebenso gut österreichischen\oindex{Österreich@\textbf{Österreich}|pw}{ }Ministerium\orgindex{Ministerium für Inneres@Ministerium für Inneres|pwv}
                        zurückzuführen sein dürfte.{ / }Ob bei dieser Verfassung unserer hohen
                        Behörden Ihre weiteren Schritte zur Aufhebung des Vorleseverbotes einen
                        Erfolg versprechen möchte ich dahingestellt sein lassen. Darüber aber, dass
                        diese Schritte in jedem Fall getan werden sollten, wissen Sie mich schon
                        durch mein Telegramm einer Ansicht mit Ihnen. Ihren weiteren Nachrichten
                        sehe ich mit Interesse entgegen. –{ / }Mit verbindlichen Grüssen Ihr sehr ergebener{ / }{[}hs.:{]} Arthur Schnitzler{ / }{[}ms.:{]} Zu bemerken wäre noch, dass die Vorlesungen
                        in Wien\oindex{Wien@\textbf{Wien}, \emph{Verwaltungsgebiet}|pw} gleich nach dem Aufführungsverbot
                        nicht nur anstandslos gestattet wurden, sondern dass die Polizeidirektion\orgindex{Polizeidirektion Wien@Polizeidirektion Wien|pw} von dem Veranstalter Buchhändler Heller\orgindex{Hugo Heller@Hugo Heller|pw} nicht einmal die Vorlage des Buches\pwindex{Schnitzler, Arthur 15.\,5.\,1862 Wien – 21.\,10.\,1931 ebd.@\textsc{Schnitzler, Arthur} (15.\,5.\,1862 Wien – 21.\,10.\,1931 ebd.), \emph{Schriftsteller, Mediziner}!Professor Bernhardi. Komödie in fünf Akten@\strich\emph{Professor Bernhardi. Komödie in fünf Akten}|pwv} verlangte. Was
                        sich seither geändert hat ist schwer zu sagen. Mein Stück\pwindex{Schnitzler, Arthur 15.\,5.\,1862 Wien – 21.\,10.\,1931 ebd.@\textsc{Schnitzler, Arthur} (15.\,5.\,1862 Wien – 21.\,10.\,1931 ebd.), \emph{Schriftsteller, Mediziner}!Professor Bernhardi. Komödie in fünf Akten@\strich\emph{Professor Bernhardi. Komödie in fünf Akten}|pwv} ist jedenfalls genau
                        dasselbe geblieben.« Der zweite Brief stammt von der \emph{K. K.
                     Polizei-Direktion}\orgindex{Polizeidirektion Wien@Polizeidirektion Wien|pwk} und ist an den Direktor Adolf Weisse\pwindex{Weisse, Adolf 4.\,4.\,1855 Tauţ – 17.\,7.\,1933 Wien@\textsc{Weisse, Adolf} (4.\,4.\,1855 Tauţ – 17.\,7.\,1933 Wien), \emph{Theaterleiter, Schauspieler}|pwk} vom \emph{Deutschen
                     Volkstheaters}\orgindex{Volkstheater@Volkstheater|pwk} gerichtet (1 Blatt, 2 Seiten, Maschinschrift, eine Ergänzung
                  mit Bleistift) »K. K. Polizei-Direktion\orgindex{Polizeidirektion Wien@Polizeidirektion Wien|pw} in Wien\oindex{Wien@\textbf{Wien}, \emph{Verwaltungsgebiet}|pw}.{ / }Wien\oindex{Wien@\textbf{Wien}, \emph{Verwaltungsgebiet}|pw}, am 31. Jänner 1913.{ / }P.B. 290.{ / }Bühnenwerk: ›Professor Bernhardi\pwindex{Schnitzler, Arthur 15.\,5.\,1862 Wien – 21.\,10.\,1931 ebd.@\textsc{Schnitzler, Arthur} (15.\,5.\,1862 Wien – 21.\,10.\,1931 ebd.), \emph{Schriftsteller, Mediziner}!Professor Bernhardi. Komödie in fünf Akten@\strich\emph{Professor Bernhardi. Komödie in fünf Akten}|pw}‹,
                        Aufführungsverbot.{ / }An{ / }Seine Wolgeboren Herrn Adolf Weisse\pwindex{Weisse, Adolf 4.\,4.\,1855 Tauţ – 17.\,7.\,1933 Wien@\textsc{Weisse, Adolf} (4.\,4.\,1855 Tauţ – 17.\,7.\,1933 Wien), \emph{Theaterleiter, Schauspieler}|pw},
                        Direktor des Deutschen Volkstheaters\orgindex{Volkstheater@Volkstheater|pw},
                           Wien VII.\oindex{VII., Neubau@\textbf{VII., Neubau}, \emph{Verwaltungsgebiet}|pw}{ / }Mit dem Erlasse der k. k. n. ö.
                           Statthalterei\orgindex{Niederösterreichische Statthalterei@Niederösterreichische Statthalterei|pw} vom 25. Oktober 1912, Pr.Z. 2910/1, wurde
                        auf Grund des eingeholten Gutachtens des Zensurbeirates\orgindex{Niederösterreichischer Zensurbeirat@Niederösterreichischer Zensurbeirat|pw} die von der Direktion des ›Deutschen Volkstheaters\orgindex{Volkstheater@Volkstheater|pw}‹ in Wien\oindex{Wien@\textbf{Wien}, \emph{Verwaltungsgebiet}|pw} angesuchte Bewilligung zur Aufführung des
                        Bühnenwerkes ›Professor Bernhardi\pwindex{Schnitzler, Arthur 15.\,5.\,1862 Wien – 21.\,10.\,1931 ebd.@\textsc{Schnitzler, Arthur} (15.\,5.\,1862 Wien – 21.\,10.\,1931 ebd.), \emph{Schriftsteller, Mediziner}!Professor Bernhardi. Komödie in fünf Akten@\strich\emph{Professor Bernhardi. Komödie in fünf Akten}|pw}‹,
                        Komödie in fünf Akten von Arthur Schnitzler, verweigert.{ / }Dem dagegen eingebrachten Rekurse der genannten Direktion\orgindex{Volkstheater@Volkstheater|pwv} hat das k.k. Ministerium des Innern\orgindex{Ministerium für Inneres@Ministerium für Inneres|pw} mit dem Erlasse vom
                           25. Jänner 1913, Z. 1962, in nachstehender Erwägung keine
                        Folge gegeben.{ / }Wenn auch die Bedenken, die gegen die Aufführung
                        des Werkes\pwindex{Schnitzler, Arthur 15.\,5.\,1862 Wien – 21.\,10.\,1931 ebd.@\textsc{Schnitzler, Arthur} (15.\,5.\,1862 Wien – 21.\,10.\,1931 ebd.), \emph{Schriftsteller, Mediziner}!Professor Bernhardi. Komödie in fünf Akten@\strich\emph{Professor Bernhardi. Komödie in fünf Akten}|pwv} vom
                        Standpunkte der Wahrung religiöser Gefühle der Bevölkerung vorliegen, durch
                        Striche oder durch Aenderung einiger Textstellen immerhin beseitigt werden
                        könnten, so stellt doch das Bühnenwerk\pwindex{Schnitzler, Arthur 15.\,5.\,1862 Wien – 21.\,10.\,1931 ebd.@\textsc{Schnitzler, Arthur} (15.\,5.\,1862 Wien – 21.\,10.\,1931 ebd.), \emph{Schriftsteller, Mediziner}!Professor Bernhardi. Komödie in fünf Akten@\strich\emph{Professor Bernhardi. Komödie in fünf Akten}|pwv} schon in seinem gesamten Aufbau durch
                        das Zusammenwirken der zur Beleuchtung unseres öffentlichen Lebens
                        gebrachten Episoden österreichische\oindex{Österreich@\textbf{Österreich}|pw}
                        staatliche Einrichtungen unter vielfacher Entstellung hierländischer
                        Zustände in einer so herabsetzenden Weise dar, dass seine Aufführung auf
                        einer inländischen Bühne wegen der zu wahrenden öffentlichen Interessen
                        nicht zugelassen werden kann. Dem gegenüber kann für die Frage der
                        Aufführung des Bühnenwerkes\pwindex{Schnitzler, Arthur 15.\,5.\,1862 Wien – 21.\,10.\,1931 ebd.@\textsc{Schnitzler, Arthur} (15.\,5.\,1862 Wien – 21.\,10.\,1931 ebd.), \emph{Schriftsteller, Mediziner}!Professor Bernhardi. Komödie in fünf Akten@\strich\emph{Professor Bernhardi. Komödie in fünf Akten}|pw} dessen
                        literarische Bedeutung nicht als entscheidend ins Gewicht fallen.– Hievon
                        werden Euer Wolgeboren in Gemässheit des Erlasses der k. k. niederösterr. Statthalterei\orgindex{Niederösterreichische Statthalterei@Niederösterreichische Statthalterei|pw} v.
                           30. Jänner 1913, Pr.Z. 462/6 verständigt.{ / }J. V.\pwindex{V., J. @\textsc{V., J.}, \emph{Ministerialbeamter}|pw}{ / }eine Unterschrift{ / }(unlserl.)«}}}\label{K_L03641-1} eine männliche geradeausblickende Kühnheit so sehr bewundern
               dürfen. Für mich werden ihre Werke immer selbstbewusster{[},{]} immer näher der Wahrheit,
               immer weiter vom Illusionären, das doch irgendwie immer mit Jugend und Träumerei
               zusammenhängt. Haben Sie innigen Dank auch für dieses Werk\pwindex{Schnitzler, Arthur 15.\,5.\,1862 Wien – 21.\,10.\,1931 ebd.@\textsc{Schnitzler, Arthur} (15.\,5.\,1862 Wien – 21.\,10.\,1931 ebd.), \emph{Schriftsteller, Mediziner}!Frau Beate und ihr Sohn. Novelle@\strich\emph{Frau Beate und ihr Sohn. Novelle}|pwv} wie für all die an{\pb}dern, (mit denen ich öfter dank der
               schönen \label{T_L03641-1v}\edtext{Gesamtausgabe}{\lemma{\textnormal{\emph{Gesamtausgabe}}}\Cendnote{\textnormal{Er schreibt: »Gesammausgabe«.}}}\label{T_L03641-1}\pwindex{Schnitzler, Arthur 15.\,5.\,1862 Wien – 21.\,10.\,1931 ebd.@\textsc{Schnitzler, Arthur} (15.\,5.\,1862 Wien – 21.\,10.\,1931 ebd.), \emph{Schriftsteller, Mediziner}!Gesammelte Werke@\strich\emph{Gesammelte Werke}|pwv} jetzt Zwiesprache tausche, als Sie es vermuten möchten)\pend
           
\pstart
           Vielleicht komme ich noch irgendwo zurecht, mich über das Werk\pwindex{Schnitzler, Arthur 15.\,5.\,1862 Wien – 21.\,10.\,1931 ebd.@\textsc{Schnitzler, Arthur} (15.\,5.\,1862 Wien – 21.\,10.\,1931 ebd.), \emph{Schriftsteller, Mediziner}!Frau Beate und ihr Sohn. Novelle@\strich\emph{Frau Beate und ihr Sohn. Novelle}|pwv} öffentlich auseinanderzusetzen: es
               reizen mich so viele ineinandergefaltete Probleme hier einzeln vor den Blick zu
               stellen. Leider sind Ihre Bücher bei den Blättern fast immer schon am Tage des
               Erscheinens vergeben\strikeout{s} und man käme \label{K_L03641-2v}\edtext{post festum}{\lemma{\textnormal{\emph{post festum}}}\Cendnote{\textnormal{lateinisch: nach dem Fest, zu
                  spät}}}\label{K_L03641-2}.\pend
           
\pstart
           Nun noch Eines: ich spreche \uline{Montag} um \uline{½ 8 Uhr im kleinen Festsaal} der Universität\oindex{Wien@\textbf{Wien}!I., Innere Stadt@\textbf{I., Innere Stadt}!Universität Wien@\textbf{Universität Wien}, \emph{Universität}|pw} zur \label{K_L03641-3v}\edtext{Bahr\pwindex{Bahr, Hermann 19.\,7.\,1863 Linz – 15.\,1.\,1934 München@\textsc{Bahr, Hermann} (19.\,7.\,1863 Linz – 15.\,1.\,1934 München), \emph{Schriftsteller, Kritiker}|pw}-Feier\eventindex{Elektrotechnisches Institut der Technischen Universität@\textbf{Elektrotechnisches Institut der Technischen Universität}!Hermann-Bahr-Feier, 26.5.1913@Hermann-Bahr-Feier, 26.5.1913|pw}}{\lemma{\textnormal{\emph{Bahr-Feier}}}\Cendnote{\textnormal{Am 26. 5. 1913 veranstaltete der \emph{Akademische Verein für Kunst und Literatur}\orgindex{Akademischer Verein für Kunst und Literatur@Akademischer Verein für Kunst und Literatur|pwk} eine Feier\eventindex{Elektrotechnisches Institut der Technischen Universität@\textbf{Elektrotechnisches Institut der Technischen Universität}!Hermann-Bahr-Feier, 26.5.1913@Hermann-Bahr-Feier, 26.5.1913|pwkv}
                  anlässlich des bevorstehenden 50. Geburtstages von Hermann Bahr\pwindex{Bahr, Hermann 19.\,7.\,1863 Linz – 15.\,1.\,1934 München@\textsc{Bahr, Hermann} (19.\,7.\,1863 Linz – 15.\,1.\,1934 München), \emph{Schriftsteller, Kritiker}|pwk}. Der Veranstaltungsort änderte sich
                  kurzfristig zum Hörsaal III des Elektrotechnischen
                     Instituts der Technischen Universität\oindex{Wien@\textbf{Wien}!IV., Wieden@\textbf{IV., Wieden}!Elektrotechnisches Institut der Technischen Universität@\textbf{Elektrotechnisches Institut der Technischen Universität}, \emph{Universität}|pwk}.}}}\label{K_L03641-3} und sage es offen, dass ich
               Sie \uline{sehr gerne} unter den Anwesenden \introOben{}sähe\introOben{}. Nicht um {\pb}meinetwillen (der vielfach Widerspruch wecken dürfte, denn Bahr\pwindex{Bahr, Hermann 19.\,7.\,1863 Linz – 15.\,1.\,1934 München@\textsc{Bahr, Hermann} (19.\,7.\,1863 Linz – 15.\,1.\,1934 München), \emph{Schriftsteller, Kritiker}|pw} ist eine so provocant agressive Persönlichkeit, dass er
                  \strikeout{\textcolor{gray}{zum}} sogar noch als Thema erbittert) sondern um Bahr’s\pwindex{Bahr, Hermann 19.\,7.\,1863 Linz – 15.\,1.\,1934 München@\textsc{Bahr, Hermann} (19.\,7.\,1863 Linz – 15.\,1.\,1934 München), \emph{Schriftsteller, Kritiker}|pw} willen, von dem vielfach vermeint wird, er sei von seiner ganzen
               Generation heute irgendwie verlassen oder ihr entfremdet. Es ist ja zum Teil leider
               wahr, nicht aber, wie ich doch weiss, bei Ihnen: deshalb hätte ich, ohne zudringlich
               sein zu wollen, \uline{für ihn} gerne Ihre Gegenwart
               erbeten.\pend
           
\pstart
           Empfangen Sie verehrter Herr Doktor nochmals den Dank alter und immer wieder
               erneuter Liebe und Verehrung von Ihrem getreu ergebenen{\\[\baselineskip]}\spacefill\mbox{Stefan Zweig}\pend
           \leftskip=0em{}\selectlanguage{ngerman}\endnumbering\briefempfaengerindex{Schnitzler, Arthur@\textsc{Schnitzler, Arthur}!zzzZweig, Stefan@\emph{von Stefan Zweig}!1913-05-231@{23. 5. 1913}|)be}\mylabel{L03641h}  \newcommand{\dateiname}{L03641}\newcommand{\titel}{Stefan Zweig an Arthur Schnitzler, 23. 5. 1913}\newcommand{\editorInnen}{Selma Jahnke und Martin Anton Müller}%% latex-leseansicht-abspann.tex
%% Abspann für die Leseansicht.
%% Der Schalter \ifkorrekturansicht ist bereits durch den Vorspann gesetzt.

%% latex-abspann.tex
%% Gemeinsamer Abspann für Korrekturansicht und Leseansicht.
%% Setzt den Schalter \ifkorrekturansicht voraus (gesetzt in den
%% einbindenden Dateien latex-korrekturansicht-abspann.tex bzw.
%% latex-leseansicht-abspann.tex).
%% ---------------------------------------------------------------

\normalsize

% Das esempio-Environment wird nur in der Leseansicht benötigt
\ifkorrekturansicht\else
\newenvironment{esempio}[3]%
{
    \vspace{1.5ex}
    \rlap{\underline{#1}}
    \par
    \setlength{\parindent}{0cm}
    \nopagebreak
    \leftskip=#2cm
    \rightskip=#3cm
}
{
    \par
}
\fi

\doendnotes{C}
\bigskip
\vfill

\clearpage

\footnotesize

\ifkorrekturansicht
  \lohead{\textsc{register}}
\fi

% theindex-Environment neu definieren ohne reledmac
\makeatletter
\renewenvironment{theindex}{%
  \ifkorrekturansicht
    \section*{\indexname}%
  \else
    \subsubsection*{Index der erwähnten Entitäten}%
  \fi
  \setlength{\parindent}{0pt}%
  \setlength{\parskip}{0pt plus 0.3pt}%
  \let\item\@idxitem
}{%
  \ifkorrekturansicht\clearpage\fi
}
\makeatother

\IfFileExists{\jobname-pw.ind}{\input{\jobname-pw.ind}}{}

% Quellenangabe nur in der Leseansicht
\ifkorrekturansicht\else
% Fallback-Definitionen, falls die .tex-Datei \titel etc. nicht gesetzt hat
\providecommand{\titel}{}
\providecommand{\editorInnen}{}
\providecommand{\dateiname}{\jobname}

\vspace{3cm}

\vfill

\footnotesize
\textsc{Quelle}: \titel. Herausgegeben von {\editorInnen}. In: \emph{Arthur Schnitzler: Briefwechsel mit Autorinnen und Autoren}.
 Digitale Edition, https://schnitzler-briefe.acdh.oeaw.ac.at/{\dateiname}.html (Stand \today)
\fi

\end{document}


