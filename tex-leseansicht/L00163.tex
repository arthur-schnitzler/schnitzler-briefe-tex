%% latex-korrekturansicht-vorspann.tex
%% Vorspann für die Korrekturansicht.
%% Lädt die gemeinsame Datei latex-vorspann.tex mit gesetztem Schalter.

\newif\ifkorrekturansicht
\korrekturansichttrue

\input{../tex-inputs/latex-vorspann}


\section[Hugo von Hofmannsthal an Arthur Schnitzler, 25. 1. {[}1893{]}]{L00163 Hugo von Hofmannsthal an Arthur Schnitzler, 25. 1. {[}1893{]}}
\nopagebreak\mylabel{L00163v}
\rehead{ }\normalsize\beginnumbering\briefempfaengerindex{Schnitzler, Arthur@\textsc{Schnitzler, Arthur}!zzzHofmannsthal, Hugo von@\emph{von Hugo von Hofmannsthal}!1893-01-251@{25. 1. {[}1893{]}}|(be}
\toendnotes[C]{\smallbreak\pagebreak[2]}\Standort{CUL, Schnitzler, B 43.}
\physDesc{Briefkarte, 416 Zeichen (aufgeprägtes Wappen )
\newline{}Handschrift: schwarze Tinte, deutsche Kurrent
\newline{}Schnitzler: mit Bleistift die Jahreszahl ergänzt: »93« 
\newline{}Ordnung: mit Bleistift von unbekannter Hand nummeriert:
                                    »36« }
\buchAbdrucke{\weitereDrucke{Hugo von Hofmannsthal, Arthur Schnitzler: \emph{Briefwechsel}. Frankfurt am Main: \emph{S. Fischer} 1964, S. 35.} }\toendnotes[C]{\smallbreak}
\pstart
           \raggedleft{}{\pb}25. I.\pend
           
\pstart{}mein lieber Arthur.\pend\vspace{0.5em}
\pstart
           \label{K_L00163-1v}\edtext{L. Marholm\pwindex{Marholm, Laura 19.04.1854 – 06.10.1928@\textsc{Marholm, Laura} (19.04.1854 – 06.10.1928), \emph{Schriftsteller/Schriftstellerin}|pw}, Friedrichshagen \introOben{}bei Berlin\introOben{}\oindex{Friedrichshagen@\textbf{Friedrichshagen}, \emph{P.PPLX}|pw}}{\lemma{\textnormal{\emph{L. … Berlin}}}\Cendnote{\textnormal{Hofmannsthal\pwindex{Hofmannsthal, Hugo von 1874-02-01 – 1929-07-15@\textsc{Hofmannsthal, Hugo von} (1874-02-01 – 1929-07-15), \emph{Schriftsteller/Schriftstellerin}|pwk} hatte sich am
                     19. 1. 1893 bei Marie
                     Herzfeld\pwindex{Herzfeld, Marie 20.03.1855 – 22.09.1940@\textsc{Herzfeld, Marie} (20.03.1855 – 22.09.1940), \emph{Schriftsteller/Schriftstellerin, Übersetzer/Übersetzerin}|pwk} wegen der Adresse erkundigt. (Hugo von Hofmannsthal\pwindex{Hofmannsthal, Hugo von 1874-02-01 – 1929-07-15@\textsc{Hofmannsthal, Hugo von} (1874-02-01 – 1929-07-15), \emph{Schriftsteller/Schriftstellerin}|pwk}: \emph{Briefe an Marie Herzfeld}. Herausgegeben von Horst Weber. Heidelberg:
                        \emph{Lothar Stiehm}{ }1967, S. 36.)}}}\label{K_L00163-1}, genügt.\pend
           
\pstart
           Sie würden, glaub’ ich, nicht unpractiſch handeln, wenn Sie der »akademiſchen Vereinigung\orgindex{Wiener Akademische Vereinigung@Wiener Akademische Vereinigung|pw}« ein Exemplar von Anatol\pwindex{Anatol@\emph{Anatol}|pw} (etwa mit der Widmung »als Gaſtgeſchenk«) zukommen ließen.
               Das ſind 30 ſichere Leſer, die in ſonſt verſchloſſenen Geſellſchaftsgruppen wieder
               ſympathiſche Kreiſe ziehen. Übrigens nur ein Vorſchlag! Auf Wiederſehen!\pend
           
\pstart
           Herzlichſt Ihr{\\[\baselineskip]}\spacefill\mbox{Loris}\pend
           \leftskip=0em{}\selectlanguage{ngerman}\endnumbering\briefempfaengerindex{Schnitzler, Arthur@\textsc{Schnitzler, Arthur}!zzzHofmannsthal, Hugo von@\emph{von Hugo von Hofmannsthal}!1893-01-251@{25. 1. {[}1893{]}}|)be}\mylabel{L00163h}  \normalsize

\doendnotes{C}
\bigskip
\vfill

\clearpage

\footnotesize

\lohead{\textsc{register}}

% Definiere theindex-Environment komplett neu ohne reledmac
\makeatletter
\renewenvironment{theindex}{%
  \section*{\indexname}%
  \setlength{\parindent}{0pt}%
  \setlength{\parskip}{0pt plus 0.3pt}%
  \let\item\@idxitem
}{%
  \clearpage
}
\makeatother

\IfFileExists{\jobname-pw.ind}{\input{\jobname-pw.ind}}{}

\end{document}

      