%% latex-leseansicht-vorspann.tex
%% Vorspann für die Leseansicht.
%% Lädt die gemeinsame Datei latex-vorspann.tex mit nicht gesetztem Schalter.

\newif\ifkorrekturansicht
\korrekturansichtfalse

\input{../tex-inputs/latex-vorspann}


\section[ Arthur Schnitzler an Felix Salten, 27. 4. 1906]{L03004 Arthur Schnitzler an Felix Salten,  27. 4. 1906}
\nopagebreak\mylabel{L03004v}
\rehead{ }\normalsize\beginnumbering\briefempfaengerindex{Salten, Felix@\textsc{Salten, Felix}!zzzSchnitzler, Arthur@\emph{von Arthur Schnitzler}!1906-04-271@{27. 4. 1906}|(be}
\toendnotes[C]{\smallbreak\pagebreak[2]}
\correspDesc{Versand  durch Arthur Schnitzler am 27. 4. 1906 in Wien
\newline{}Erhalt  durch Felix Salten im Zeitraum [28. 4. 1906
                  – 1. 5. 1906?] in Berlin}\toendnotes[C]{\smallbreak}
\Standort{Wienbibliothek im Rathaus, ZPH 1681, 2.1.516.}
\physDesc{Brief, 2 Blätter, 7 Seiten, 3652 Zeichen
\newline{}Handschrift: schwarze Tinte, deutsche Kurrent
\newline{}Ordnung: mit Bleistift von unbekannter Hand Nummerierung der Doppelseiten des
                                 Konvoluts: »16«–»19«  }
\buchAbdrucke{\weitereDrucke{Arthur Schnitzler: \emph{Briefe 1875–1912}. Herausgegeben von Therese Nickl und Heinrich Schnitzler. Frankfurt am Main: \emph{S. Fischer} 1981, S. 529–531.} }\toendnotes[C]{\smallbreak}
\pstart
           \textcolor{gray}{\textbf{Dr. Arthur Schnitzler}}\hfill {\pb}27. 4. 906\pend
           
\pstart
           \textcolor{gray}{\textbf{Wien, XVIII. Spoettelgasse 7\oindex{Wien@\textbf{Wien}!XVIII., Währing@\textbf{XVIII., Währing}!Edmund-Weiß-Gasse 7@\textbf{Edmund-Weiß-Gasse 7}, \emph{Wohngebäude}|pw}.}}\pend
           \vspace{0.5em}
\pstart
           lieber, Sie haben natürlich ganz recht. Unmöglich konnten Sie{ }ſich
                  Brahm\pwindex{Brahm, Otto 5.\,2.\,1856 Hamburg – 28.\,11.\,1912 Berlin@\textsc{Brahm, Otto} (5.\,2.\,1856 Hamburg – 28.\,11.\,1912 Berlin), \emph{Theaterleiter, Regisseur}|pw} gegenüber als ungebetener Rathgeber
               aufſpielen, und als ich mein Telegra{\geminationm} an Sie abſandte,
               hatt ich begreiflicherweiſe nicht an irgend einen \textsc{adhoc}-Beſuch od dergl bei Brahm\pwindex{Brahm, Otto 5.\,2.\,1856 Hamburg – 28.\,11.\,1912 Berlin@\textsc{Brahm, Otto} (5.\,2.\,1856 Hamburg – 28.\,11.\,1912 Berlin), \emph{Theaterleiter, Regisseur}|pw} gedacht,{ }ſondern an etwas beiläufigeres, ohne mir über das »wie« weitere Gedanken zu machen.
               (Damit dſs Brahm\pwindex{Brahm, Otto 5.\,2.\,1856 Hamburg – 28.\,11.\,1912 Berlin@\textsc{Brahm, Otto} (5.\,2.\,1856 Hamburg – 28.\,11.\,1912 Berlin), \emph{Theaterleiter, Regisseur}|pw} auf Ihr Urtheil nichts geben
               könnte,{ }ſind Sie{ }ſehr im Irrtum.) – Nun hab ich die Sache indeſs auf andre, directe
               Weise zu ordnen geſucht. {\pb}(\uline{Dies vollko{\geminationm}en unter
                  uns.}) Nach Ihrem Brief, in dem Sie mir Ihr Geſpräch mit R.\pwindex{Rittner, Rudolf 30.\,6.\,1869 Bílý Potok – 4.\,2.\,1943 ebd.@\textsc{Rittner, Rudolf} (30.\,6.\,1869 Bílý Potok – 4.\,2.\,1943 ebd.), \emph{Theaterleiter, Schauspieler}|pw} erzählten u einen \label{K_L03004-1v}\edtext{Brief Jacobsohns\pwindex{Jacobsohn, Siegfried 28.\,1.\,1881 Berlin – 3.\,12.\,1926 ebd.@\textsc{Jacobsohn, Siegfried} (28.\,1.\,1881 Berlin – 3.\,12.\,1926 ebd.), \emph{Journalist, Kritiker, Publizist}|pw}}{\lemma{\textnormal{\emph{Brief Jacobsohns}}}\Cendnote{\textnormal{Im Brief Jacobsohns\pwindex{Jacobsohn, Siegfried 28.\,1.\,1881 Berlin – 3.\,12.\,1926 ebd.@\textsc{Jacobsohn, Siegfried} (28.\,1.\,1881 Berlin – 3.\,12.\,1926 ebd.), \emph{Journalist, Kritiker, Publizist}|pwk} vom 20. 4. 1906.
                   heißt es: »Verhindern Sie, wenns irgend geht, dß Reicher\pwindex{Reicher, Emanuel 18.\,6.\,1849 Bochnia – 15.\,5.\,1924 Berlin@\textsc{Reicher, Emanuel} (18.\,6.\,1849 Bochnia – 15.\,5.\,1924 Berlin), \emph{Schauspieler}|pw} in Wien\oindex{Wien@\textbf{Wien}, \emph{Verwaltungsgebiet}|pw}
                     Ihren Julian Fichtner\pwindex{Schnitzler, Arthur 15.\,5.\,1862 Wien – 21.\,10.\,1931 ebd.@\textsc{Schnitzler, Arthur} (15.\,5.\,1862 Wien – 21.\,10.\,1931 ebd.), \emph{Schriftsteller, Mediziner}!einsame Weg. Schauspiel in fünf Akten@\strich\emph{Der einsame Weg. Schauspiel in fünf Akten}|pwv}
                      spielt. Es war eine Schmach, was sich gestern im Lessing-Theater\orgindex{Lessing-Theater@Lessing-Theater|pw} abspielte. Der
                     Mann kann kein Wort von der Rolle. Die Souffleuse schrie sich heiser.«
                        (\emph{CUL}, B 46). Vgl. A. S.: \emph{Tagebuch}, 21. 4. 1906.}}}\label{K_L03004-1}, der auch
               telephoniſch eine Art Bereitwilligkeit R.s\pwindex{Rittner, Rudolf 30.\,6.\,1869 Bílý Potok – 4.\,2.\,1943 ebd.@\textsc{Rittner, Rudolf} (30.\,6.\,1869 Bílý Potok – 4.\,2.\,1943 ebd.), \emph{Theaterleiter, Schauspieler}|pw}
               erfahren haben wollte, telegr ich an Brahm\pwindex{Brahm, Otto 5.\,2.\,1856 Hamburg – 28.\,11.\,1912 Berlin@\textsc{Brahm, Otto} (5.\,2.\,1856 Hamburg – 28.\,11.\,1912 Berlin), \emph{Theaterleiter, Regisseur}|pw}, ob
               er mir überlaſſen wolle \textsc{Rittner}\pwindex{Rittner, Rudolf 30.\,6.\,1869 Bílý Potok – 4.\,2.\,1943 ebd.@\textsc{Rittner, Rudolf} (30.\,6.\,1869 Bílý Potok – 4.\,2.\,1943 ebd.), \emph{Theaterleiter, Schauspieler}|pw} zur Übernahme zu bewegen. Er konnte nichts dagegen haben, warnte mich für alle
               Fälle, wuſch{ }ſeine Hände in Unſchuld \textsc{etc.} Ich telegr. nun
               an \textsc{Rittner}\pwindex{Rittner, Rudolf 30.\,6.\,1869 Bílý Potok – 4.\,2.\,1943 ebd.@\textsc{Rittner, Rudolf} (30.\,6.\,1869 Bílý Potok – 4.\,2.\,1943 ebd.), \emph{Theaterleiter, Schauspieler}|pw}, der mir in einem{ }ſehr liebenswürdigen Telegra{\geminationm}
               nein{ }ſagte. Ich hatte es natürlich nicht anders erwartet – die Gegengründe lagen für
                  Rittner\pwindex{Rittner, Rudolf 30.\,6.\,1869 Bílý Potok – 4.\,2.\,1943 ebd.@\textsc{Rittner, Rudolf} (30.\,6.\,1869 Bílý Potok – 4.\,2.\,1943 ebd.), \emph{Theaterleiter, Schauspieler}|pw} zu nah, als daſs er nicht von ihnen
               hätte {\pb}Gebrauch machen{ }ſollen. Aber ich
               wollte mir keine Vorwürfe zu machen haben – und da mir \textsc{Rittner}\pwindex{Rittner, Rudolf 30.\,6.\,1869 Bílý Potok – 4.\,2.\,1943 ebd.@\textsc{Rittner, Rudolf} (30.\,6.\,1869 Bílý Potok – 4.\,2.\,1943 ebd.), \emph{Theaterleiter, Schauspieler}|pw}{ }ſtrengſte Discretion zugeſagt hat, hoffe ich daſs nicht am End noch eine für
               die Wien\oindex{Wien@\textbf{Wien}, \emph{Verwaltungsgebiet}|pw}er Aufführg (auf die ich{ }ſchließlich doch
               nicht verzichten möchte) gefährliche Couliſſenklatſcherei heraus ko{\geminationm}t. Sonderbar iſt, daſs vor 2 Jahren, nach Rittners\pwindex{Rittner, Rudolf 30.\,6.\,1869 Bílý Potok – 4.\,2.\,1943 ebd.@\textsc{Rittner, Rudolf} (30.\,6.\,1869 Bílý Potok – 4.\,2.\,1943 ebd.), \emph{Theaterleiter, Schauspieler}|pw} Verſagen (aus Unluſt) an der Rolle
               alle, auch Brahm\pwindex{Brahm, Otto 5.\,2.\,1856 Hamburg – 28.\,11.\,1912 Berlin@\textsc{Brahm, Otto} (5.\,2.\,1856 Hamburg – 28.\,11.\,1912 Berlin), \emph{Theaterleiter, Regisseur}|pw} und ich dachten, Reicher\pwindex{Reicher, Emanuel 18.\,6.\,1849 Bochnia – 15.\,5.\,1924 Berlin@\textsc{Reicher, Emanuel} (18.\,6.\,1849 Bochnia – 15.\,5.\,1924 Berlin), \emph{Schauspieler}|pw} wäre der richtige Darſteller für die
               Rolle. Nach der erſchütternden Charakteriſtik, die Sie von{ }ſeiner Auffaſſung geben,
                  ka{\geminationn} ich mir nun wohl vorſtellen, was mir {\pb}bevorſteht. Übrigens gibt es meiner Empfindg
               nach nur einen Darſteller für den \textsc{Julian\pwindex{Schnitzler, Arthur 15.\,5.\,1862 Wien – 21.\,10.\,1931 ebd.@\textsc{Schnitzler, Arthur} (15.\,5.\,1862 Wien – 21.\,10.\,1931 ebd.), \emph{Schriftsteller, Mediziner}!einsame Weg. Schauspiel in fünf Akten@\strich\emph{Der einsame Weg. Schauspiel in fünf Akten}|pwv}}: \textsc{Wischnevski\pwindex{Wischnewski, Alexander Leonidowitsch 20.\,1.\,1861 Taganrog – 27.\,2.\,1943 Tashkent@\textsc{Wischnewski, Alexander Leonidowitsch} (20.\,1.\,1861 Taganrog – 27.\,2.\,1943 Tashkent), \emph{Schauspieler}|pw}}. Sie haben ihn ja als Onkel \textsc{Wanja}\pwindex{\textcolor{red}{\textsuperscript{XXXX indx1}}!Onkel Wanja. Szenen aus dem Landleben in vier Akten@\strich\emph{Onkel Wanja. Szenen aus dem Landleben in vier Akten}|pw} geſehen. Und \textsc{Stanislawski\pwindex{Stanislavskij, Konstantin S. 17.\,1.\,1863 Moskau – 7.\,8.\,1938 ebd.@\textsc{Stanislavskij, Konstantin S.} (17.\,1.\,1863 Moskau – 7.\,8.\,1938 ebd.), \emph{Theaterleiter, Regisseur, Schauspieler}|pw}} als \textsc{Sala\pwindex{Schnitzler, Arthur 15.\,5.\,1862 Wien – 21.\,10.\,1931 ebd.@\textsc{Schnitzler, Arthur} (15.\,5.\,1862 Wien – 21.\,10.\,1931 ebd.), \emph{Schriftsteller, Mediziner}!einsame Weg. Schauspiel in fünf Akten@\strich\emph{Der einsame Weg. Schauspiel in fünf Akten}|pwv}} wär auch nicht übel. Wir haben dieſe beiden, auch \textsc{Ljuschin\pwindex{Južin, Aleksandr I. 16.\,9.\,1857 Tula Oblast – 27.\,9.\,1927 Juan-les-Pins@\textsc{Južin, Aleksandr I.} (16.\,9.\,1857 Tula Oblast – 27.\,9.\,1927 Juan-les-Pins), \emph{Theaterleiter, Schauspieler}|pw}} (Profeſſor\pwindex{\textcolor{red}{\textsuperscript{XXXX indx1}}!Onkel Wanja. Szenen aus dem Landleben in vier Akten@\strich\emph{Onkel Wanja. Szenen aus dem Landleben in vier Akten}|pwv} in \textsc{Wanja\pwindex{\textcolor{red}{\textsuperscript{XXXX indx1}}!Onkel Wanja. Szenen aus dem Landleben in vier Akten@\strich\emph{Onkel Wanja. Szenen aus dem Landleben in vier Akten}|pw}}), \textsc{Leonidow\pwindex{Leonidow, Leonid M. 22.\,3.\,1873 Odessa – 6.\,8.\,1941 Moskau@\textsc{Leonidow, Leonid M.} (22.\,3.\,1873 Odessa – 6.\,8.\,1941 Moskau), \emph{Regisseur, Schauspieler}|pw}}, Frau Tſchechow\pwindex{Čechowa, Olga L. 21.\,9.\,1868 – 22.\,3.\,1959 Moskau@\textsc{Čechowa, Olga L.} (21.\,9.\,1868 – 22.\,3.\,1959 Moskau), \emph{Schauspielerin}|pw}{ }\label{K_L03004-2v}\edtext{bei Rotenſtern’s\pwindex{Rotenstern, Peter 10.\,1.\,1868 Odessa – 1944@\textsc{Rotenstern, Peter} (10.\,1.\,1868 Odessa – 1944), \emph{Journalist, Übersetzer}|pw}\pwindex{Rotenstern-Tesi, Anna *~11.\,1.\,1871 Odessa@\textsc{Rotenstern-Tesi, Anna} (*~11.\,1.\,1871 Odessa), \emph{Übersetzerin}|pw} kennengelernt}{\lemma{\textnormal{\emph{bei … kennengelernt}}}\Cendnote{\textnormal{Vgl. A. S.: \emph{Tagebuch}, 17. 4. 1906.
               }}}\label{K_L03004-2}; auch im Theater \label{K_L03004-3v}\edtext{hinter den Couliſſen ein paar mal geſprochen}{\lemma{\textnormal{\emph{hinter … gesprochen}}}\Cendnote{\textnormal{Vgl. A. S.: \emph{Tagebuch}, 18. 4. 1906.
               }}}\label{K_L03004-3}. Es hat mich{ }ſehr gefreut, daſs ihnen
               viel daran zu liegen{ }ſchien, ein Stück von mir für ihr Theater zu beko{\geminationm}en. Jedenfalls gibt es keins, an dem ich lieber
               aufgeführt werden möchte. Sieht man{ }ſolche {\pb}um alles dramatiſche unbekü{\geminationm}erte Geſtalten- und
               Lebensſtücke wie den Onkel \textsc{Wanja}\pwindex{\textcolor{red}{\textsuperscript{XXXX indx1}}!Onkel Wanja. Szenen aus dem Landleben in vier Akten@\strich\emph{Onkel Wanja. Szenen aus dem Landleben in vier Akten}|pw},{ }ſo ist einem, als braucht man{ }ſich nur hinzuſetzen, um ein viertel Dutzend im
               Jahr zu{ }ſchreiben. Und doch{\dots} Allerdings fiele man auch
               durch. –\pend
           
\pstart
           Tennis{ }ſpielen wir{ }ſchon ziemlich regelmäßig – d. h. meiſtens ich, Dr \textsc{Kaufmann\pwindex{Kaufmann, Arthur 4.\,4.\,1872 Iași – 25.\,7.\,1938 Wien@\textsc{Kaufmann, Arthur} (4.\,4.\,1872 Iași – 25.\,7.\,1938 Wien), \emph{Rechtswissenschaftler, Privatgelehrte, Privatier}|pw}}, Frl \textsc{Erl\pwindex{Erl, Dora @\textsc{Erl, Dora}, \emph{Schauspielerin, Gesangspädagogin}|pw}}, Olga\pwindex{Schnitzler, Olga 17.\,1.\,1882 Wien – 13.\,1.\,1970 Lugano@\textsc{Schnitzler, Olga} (17.\,1.\,1882 Wien – 13.\,1.\,1970 Lugano), \emph{Schauspielerin, Sängerin}|pw}{ }ſeltener. Zuweilen geh ich im Pötzleinsdorferwalde\oindex{Wien@\textbf{Wien}!XVIII., Währing@\textbf{XVIII., Währing}!Pötzleinsdorf@\textbf{Pötzleinsdorf}, \emph{Ehemaliger Ort}|pw}{ }ſpaziren. Es iſt{ }ſchon beinah{ }ſo{\geminationm}erlich, um mindeſten\textcolor{gray}{s} vierzehn
               Tage weiter vor, als voriges Jahr. \label{K_L03004-4v}\edtext{Neulich war \textsc{Fred\pwindex{W. Fred 29.\,6.\,1879 Wien – 23.\,10.\,1922 Berlin@\textsc{W. Fred} (29.\,6.\,1879 Wien – 23.\,10.\,1922 Berlin), \emph{Schriftsteller, Journalist}|pw}} bei uns}{\lemma{\textnormal{\emph{Neulich war Fred bei uns}}}\Cendnote{\textnormal{Siehe A. S.: \emph{Tagebuch}, 23. 4. 1906.
               }}}\label{K_L03004-4}, der{ }ſich im Lauf der Jahre höchſt vorteilhaft verändert hat. (Dieſer {\pb}Tage wird er (wahrſcheinlich von meinem Bruder\pwindex{Schnitzler, Julius 13.\,7.\,1865 Wien – 29.\,6.\,1939 ebd.@\textsc{Schnitzler, Julius} (13.\,7.\,1865 Wien – 29.\,6.\,1939 ebd.), \emph{Chirurg}|pwv}) an Gallenſteinen
               operirt.) –\pend
           
\pstart
           Über Ihre So{\geminationm}erpläne möcht ich recht bald näheres
               wiſſen. Meine Karte, Frau \textsc{v Lützow\pwindex{Lützow, Linda von 5.\,9.\,1832 Heidelberg – 4.\,7.\,1922@\textsc{Lützow, Linda von} (5.\,9.\,1832 Heidelberg – 4.\,7.\,1922), \emph{Übersetzerin}|pw}} betreffend, haben Sie wohl erhalten? Neulich war hier das Gerücht verbreitet,
               daſs Sie auf ein paar Tage nach Wien\oindex{Wien@\textbf{Wien}, \emph{Verwaltungsgebiet}|pw} kämen. Wie{ }ſteht die \label{K_L03004-5v}\edtext{Proceſsangelegenheit}{\lemma{\textnormal{\emph{Processangelegenheit}}}\Cendnote{\textnormal{Siehe XXXX Auszeichnungsfehler: Dokument L03415 nicht gefunden.
               }}}\label{K_L03004-5}? Ich{ }ſtelle mir Ludaſſy\pwindex{Gans-Ludassy, Julius von 13.\,4.\,1858 Wien – 30.\,9.\,1922 ebd.@\textsc{Gans-Ludassy, Julius von} (13.\,4.\,1858 Wien – 30.\,9.\,1922 ebd.), \emph{Schriftsteller, Journalist, Herausgeber}|pw} verda{\geminationm}t wenig dazu gelaunt vor. –\pend
           
\pstart
           \label{K_L03004-6v}\edtext{Neulich, mit dem reparirten Rad}{\lemma{\textnormal{\emph{Neulich, … Rad}}}\Cendnote{\textnormal{Vgl. A. S.: \emph{Tagebuch}, 17. 4. 1906.
               }}}\label{K_L03004-6} (alles mögliche, 55 Kronen!) erſter
               Verſuch, in Neuwaldegg\oindex{Wien@\textbf{Wien}!XVII., Hernals@\textbf{XVII., Hernals}!Neuwaldegg@\textbf{Neuwaldegg}, \emph{Ehemaliger Ort}|pw} brach die Axe. Trotzdem
               bleibt die Sehnſucht nach den gemeinſchaftlichen Partien beſtehen. Haben Sie{ }ſich
               nicht die Sache wegen \label{K_L03004-7v}\edtext{Daenemark\oindex{Dänemark@\textbf{Dänemark}|pw}}{\lemma{\textnormal{\emph{Daenemark}}}\Cendnote{\textnormal{Siehe XXXX Auszeichnungsfehler: Dokument L03416 nicht gefunden.
               }}}\label{K_L03004-7}{ }{\pb}überlegt? –\pend
           
\pstart
           Ich arbeite (am Roman\pwindex{Schnitzler, Arthur 15.\,5.\,1862 Wien – 21.\,10.\,1931 ebd.@\textsc{Schnitzler, Arthur} (15.\,5.\,1862 Wien – 21.\,10.\,1931 ebd.), \emph{Schriftsteller, Mediziner}!Weg ins Freie. Roman@\strich\emph{Der Weg ins Freie. Roman}|pwv})
               ziemlich regelmäßig aber ohne die nöthige Intenſität. Mir thut es{ }ſo leid, daſs ich
               Sie in der B. Z.\pwindex{B.Z. am Mittag@\emph{B.Z. am Mittag}|pw} beinah niemals finde. Was
               machen Sie{ }ſonſt? Ich nehme an, daſs Sie mit adminiſtrativen und organiſatoriſchen
               Arbeiten überhäuft{ }ſind. –\pend
           
\pstart
           Seien Sie herzlich gegrüßt, ebenſo Otti\pwindex{Salten, Ottilie 7.\,3.\,1868 Prag – 22.\,6.\,1942 Zürich@\textsc{Salten, Ottilie} (7.\,3.\,1868 Prag – 22.\,6.\,1942 Zürich), \emph{Schauspielerin}|pw} u
               die Kinder\pwindex{Salten, Paul 11.\,8.\,1903 Wien – 8.\,5.\,1937 ebd.@\textsc{Salten, Paul} (11.\,8.\,1903 Wien – 8.\,5.\,1937 ebd.), \emph{Filmcutter}|pwv}\pwindex{Rehmann, Anna Katharina 18.\,8.\,1904 Wien – 27.\,3.\,1977 Zürich@\textsc{Rehmann, Anna Katharina} (18.\,8.\,1904 Wien – 27.\,3.\,1977 Zürich), \emph{Schauspielerin, Übersetzerin}|pwv}, von uns\pwindex{Schnitzler, Olga 17.\,1.\,1882 Wien – 13.\,1.\,1970 Lugano@\textsc{Schnitzler, Olga} (17.\,1.\,1882 Wien – 13.\,1.\,1970 Lugano), \emph{Schauspielerin, Sängerin}|pwv}\pwindex{Schnitzler, Heinrich 9.\,8.\,1902 Hinterbrühl – 12.\,7.\,1982 Wien@\textsc{Schnitzler, Heinrich} (9.\,8.\,1902 Hinterbrühl – 12.\,7.\,1982 Wien), \emph{Regisseur, Schauspieler}|pwv} allen. {\\[\baselineskip]}Ihr {\\[\baselineskip]}\spacefill\mbox{A.}\pend
           \leftskip=0em{}\selectlanguage{ngerman}\endnumbering\briefempfaengerindex{Salten, Felix@\textsc{Salten, Felix}!zzzSchnitzler, Arthur@\emph{von Arthur Schnitzler}!1906-04-271@{27. 4. 1906}|)be}\mylabel{L03004h}  \newcommand{\dateiname}{L03004}\newcommand{\titel}{Arthur Schnitzler an Felix Salten, 27. 4. 1906}\newcommand{\editorInnen}{Martin Anton Müller und Laura Untner}%% latex-leseansicht-abspann.tex
%% Abspann für die Leseansicht.
%% Der Schalter \ifkorrekturansicht ist bereits durch den Vorspann gesetzt.

%% latex-abspann.tex
%% Gemeinsamer Abspann für Korrekturansicht und Leseansicht.
%% Setzt den Schalter \ifkorrekturansicht voraus (gesetzt in den
%% einbindenden Dateien latex-korrekturansicht-abspann.tex bzw.
%% latex-leseansicht-abspann.tex).
%% ---------------------------------------------------------------

\normalsize

% Das esempio-Environment wird nur in der Leseansicht benötigt
\ifkorrekturansicht\else
\newenvironment{esempio}[3]%
{
    \vspace{1.5ex}
    \rlap{\underline{#1}}
    \par
    \setlength{\parindent}{0cm}
    \nopagebreak
    \leftskip=#2cm
    \rightskip=#3cm
}
{
    \par
}
\fi

\doendnotes{C}
\bigskip
\vfill

\clearpage

\footnotesize

\ifkorrekturansicht
  \lohead{\textsc{register}}
\fi

% theindex-Environment neu definieren ohne reledmac
\makeatletter
\renewenvironment{theindex}{%
  \ifkorrekturansicht
    \section*{\indexname}%
  \else
    \subsubsection*{Index der erwähnten Entitäten}%
  \fi
  \setlength{\parindent}{0pt}%
  \setlength{\parskip}{0pt plus 0.3pt}%
  \let\item\@idxitem
}{%
  \ifkorrekturansicht\clearpage\fi
}
\makeatother

\IfFileExists{\jobname-pw.ind}{\input{\jobname-pw.ind}}{}

% Quellenangabe nur in der Leseansicht
\ifkorrekturansicht\else
% Fallback-Definitionen, falls die .tex-Datei \titel etc. nicht gesetzt hat
\providecommand{\titel}{}
\providecommand{\editorInnen}{}
\providecommand{\dateiname}{\jobname}

\vspace{3cm}

\vfill

\footnotesize
\textsc{Quelle}: \titel. Herausgegeben von {\editorInnen}. In: \emph{Arthur Schnitzler: Briefwechsel mit Autorinnen und Autoren}.
 Digitale Edition, https://schnitzler-briefe.acdh.oeaw.ac.at/{\dateiname}.html (Stand \today)
\fi

\end{document}


