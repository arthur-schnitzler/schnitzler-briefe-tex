%% latex-leseansicht-vorspann.tex
%% Vorspann für die Leseansicht.
%% Lädt die gemeinsame Datei latex-vorspann.tex mit nicht gesetztem Schalter.

\newif\ifkorrekturansicht
\korrekturansichtfalse

\input{../tex-inputs/latex-vorspann}

\begin{center}
            \textcolor{red}{ENTWURF, NICHT FERTIG KORRIGIERT}
                      \end{center}
            
         
         \renewcommand{\erwaehntePersonen}{Personen: Otto Brahm, Dora Erl, Julius von Gans-Ludassy, Siegfried Jacobsohn, Aleksandr I. Južin, Arthur Kaufmann, Leonid M. Leonidow, Linda von Lützow, Anna Katharina Rehmann, Emanuel Reicher, Rudolf Rittner, Peter Rotenstern, Anna Rotenstern-Tesi, Felix Salten, Ottilie Salten, Paul Salten, Olga Schnitzler, Julius Schnitzler, Heinrich Schnitzler, Konstantin S. Stanislavskij,  W. Fred, Alexander Leonidowitsch Wischnewski, Olga L. Čechowa}
         \renewcommand{\erwaehnteOrte}{Orte: Berlin, Dänemark, Edmund-Weiß-Gasse 7, Neuwaldegg, Pötzleinsdorf, Wien}
         \renewcommand{\erwaehnteWerke}{Werke: B.Z. am Mittag, Der Weg ins Freie. Roman, Der einsame Weg. Schauspiel in fünf Akten, Onkel Wanja. Szenen aus dem Landleben in vier Akten}
               \section[ Arthur Schnitzler an Felix Salten, 27. 4. 1906]{ Arthur Schnitzler an Felix Salten, 27. 4. 1906}\nopagebreak\mylabel{v}\rehead{ }\begin{ledgroupsized}[t]{13cm}\normalsize\beginnumbering \toendnotes[C]{\smallbreak\pagebreak[2]} \Standort{Wienbibliothek im Rathaus, ZPH 1681, 2.1.516.}
\physDesc{Brief, 2 Blätter, 7 Seiten, 3654 Zeichen
\newline{}Handschrift: schwarze Tinte, deutsche Kurrent
\newline{}Ordnung: mit Bleistift von unbekannter Hand Nummerierung der Doppelseiten des Konvoluts:
                                    »16«–»19«  }\buchAbdrucke{\weitereDrucke{Arthur Schnitzler: \emph{Briefe 1875–1912}. Hg. Therese Nickl und Heinrich Schnitzler. Frankfurt am Main: \emph{S. Fischer} 1981, S. 529–531.} }\toendnotes[C]{\smallbreak}\pstart
           \noindent{}\textcolor{gray}{\textbf{Dr. Arthur Schnitzler}}\hfill {\pb}27. 4. 906\pend
           \pstart
           \textcolor{gray}{\textbf{Wien, XVIII. Spoettelgasse 7\oindex{Edmund-Weiss-Gasse 7@\textbf{Edmund-Weiß-Gasse 7}|pw}.}}\pend
           \pstart
           lieber, Sie haben natürlich ganz recht. Unmöglich konnten Sie ſich
                  Brahm\pwindex{Brahm, Otto 05.02.1856 – 28.11.1912@\textsc{Brahm, Otto} (05.02.1856 – 28.11.1912), \emph{Theaterleiter, Regisseur}|pw} gegenüber als ungebetener Rathgeber
               aufſpielen, und als ich mein Telegra{\geminationm} an Sie abſandte,
               hatt ich begreiflicherweiſe nicht an irgend einen \textsc{adhoc}-Beſuch od dergl bei Brahm\pwindex{Brahm, Otto 05.02.1856 – 28.11.1912@\textsc{Brahm, Otto} (05.02.1856 – 28.11.1912), \emph{Theaterleiter, Regisseur}|pw} gedacht,
               ſondern an etwas beiläufigeres, ohne mir über das »wie« weitere Gedanken zu machen.
               (Damit dſs Brahm\pwindex{Brahm, Otto 05.02.1856 – 28.11.1912@\textsc{Brahm, Otto} (05.02.1856 – 28.11.1912), \emph{Theaterleiter, Regisseur}|pw} auf Ihr Urtheil nichts geben
               könnte, ſind Sie ſehr im Irrtum.) – Nun hab ich die Sache indeſs auf andre, directe
               Weise zu ordnen geſucht. {\pb}(\uline{Dies vollko{\geminationm}en unter
               uns.}) Nach Ihrem Brief, in dem Sie mir Ihr Geſpräch mit R.\pwindex{Rittner, Rudolf 30.06.1869 – 04.02.1943@\textsc{Rittner, Rudolf} (30.06.1869 – 04.02.1943), \emph{Theaterleiter, Schauspieler}|pw} erzählten u einen Brief Jacobsohn\pwindex{Jacobsohn, Siegfried 28.01.1881 – 03.12.1926@\textsc{Jacobsohn, Siegfried} (28.01.1881 – 03.12.1926), \emph{Journalist, Kritiker, Publizist}|pw}s, der auch telephoniſch eine Art Bereitwilligkeit R.s\pwindex{Rittner, Rudolf 30.06.1869 – 04.02.1943@\textsc{Rittner, Rudolf} (30.06.1869 – 04.02.1943), \emph{Theaterleiter, Schauspieler}|pw} erfahren haben wollte, telegr ich an Brahm\pwindex{Brahm, Otto 05.02.1856 – 28.11.1912@\textsc{Brahm, Otto} (05.02.1856 – 28.11.1912), \emph{Theaterleiter, Regisseur}|pw}, ob er mir überlaſſen wolle Rittner\pwindex{Rittner, Rudolf 30.06.1869 – 04.02.1943@\textsc{Rittner, Rudolf} (30.06.1869 – 04.02.1943), \emph{Theaterleiter, Schauspieler}|pw} zur Übernahme zu bewegen. Er konnte
               nichts dagegen haben, warnte mich für alle Fälle, wuſch ſeine Hände in Unſchuld \textsc{etc.} Ich telegr. nun an Rittner\pwindex{Rittner, Rudolf 30.06.1869 – 04.02.1943@\textsc{Rittner, Rudolf} (30.06.1869 – 04.02.1943), \emph{Theaterleiter, Schauspieler}|pw}, der mir in einem ſehr liebenswürdigen Telegra{\geminationm} nein ſagte. Ich hatte es natürlich nicht anders
               erwartet – die Gegengründe lagen für Rittner\pwindex{Rittner, Rudolf 30.06.1869 – 04.02.1943@\textsc{Rittner, Rudolf} (30.06.1869 – 04.02.1943), \emph{Theaterleiter, Schauspieler}|pw}
               zu nah, als daſs er nicht von ihnen hätte {\pb}Gebrauch machen ſollen. Aber ich wollte mir keine Vorwürfe zu machen haben – und da
               mir Rittner\pwindex{Rittner, Rudolf 30.06.1869 – 04.02.1943@\textsc{Rittner, Rudolf} (30.06.1869 – 04.02.1943), \emph{Theaterleiter, Schauspieler}|pw} ſtrengſte Discretion zugeſagt hat,
               hoffe ich daſs nicht am End noch eine für die Wien\oindex{Wien@\textbf{Wien}|pw}er Aufführg (auf die ich ſchließlich doch nicht verzichten möchte),
               gefährliche Couliſſenklatſcherei heraus ko{\geminationm}t. Sonderbar
               iſt, daſs vor 2 Jahren, nach Rittner\pwindex{Rittner, Rudolf 30.06.1869 – 04.02.1943@\textsc{Rittner, Rudolf} (30.06.1869 – 04.02.1943), \emph{Theaterleiter, Schauspieler}|pw}s Verſagen
               (aus Unluſt) an der Rolle alle, auch Brahm\pwindex{Brahm, Otto 05.02.1856 – 28.11.1912@\textsc{Brahm, Otto} (05.02.1856 – 28.11.1912), \emph{Theaterleiter, Regisseur}|pw} und
               ich dachten, Reicher\pwindex{Reicher, Emanuel 18.06.1849 – 15.05.1924@\textsc{Reicher, Emanuel} (18.06.1849 – 15.05.1924), \emph{Schauspieler}|pw} wäre der richtige
               Darſteller für die Rolle. Nach der erſchütternden Charakteriſtik, die Sie von ſeiner
               Auffaſſung geben, ka{\geminationn} ich mir nun wohl vorſtellen, was
               mir {\pb}bevorſteht. Übrigens gibt es meiner
               Empfindg nach nur einen Darſteller für den \textsc{Julian\pwindex{Schnitzler, Arthur 15.05.1862 – 21.10.1931@\textsc{Schnitzler, Arthur} (15.05.1862 – 21.10.1931), \emph{Schriftsteller, Mediziner}!einsame Weg. Schauspiel in fuenf Akten1904@\strich\emph{Der einsame Weg. Schauspiel in fünf Akten} {[}1904{]}|pwv}}: \textsc{Wischnevski\pwindex{Wischnewski, Alexander Leonidowitsch 20.01.1861 – 27.02.1943@\textsc{Wischnewski, Alexander Leonidowitsch} (20.01.1861 – 27.02.1943), \emph{Schauspieler}|pw}}. Sie haben ihn ja als \textsc{Onkel Wanja\pwindex{\textcolor{red}{\textsuperscript{XXXX1 indx}}!Onkel Wanja. Szenen aus dem Landleben in vier Akten1896@\strich\emph{Onkel Wanja. Szenen aus dem Landleben in vier Akten} {[}1896{]}|pw}} geſehen. Und \textsc{Stanislawski\pwindex{Stanislavskij, Konstantin S. 17.01.1863 – 07.08.1938@\textsc{Stanislavskij, Konstantin S.} (17.01.1863 – 07.08.1938), \emph{Theaterleiter, Regisseur, Schauspieler}|pw}} als \textsc{Sala\pwindex{Schnitzler, Arthur 15.05.1862 – 21.10.1931@\textsc{Schnitzler, Arthur} (15.05.1862 – 21.10.1931), \emph{Schriftsteller, Mediziner}!einsame Weg. Schauspiel in fuenf Akten1904@\strich\emph{Der einsame Weg. Schauspiel in fünf Akten} {[}1904{]}|pwv}} wär auch nicht übel. Wir haben dieſe beiden, auch \textsc{Ljuschin\pwindex{Južin, Aleksandr I. 1857-09-16 – 1927-09-27@\textsc{Južin, Aleksandr I.} (1857-09-16 – 1927-09-27), \emph{Theaterleiter, Schauspieler}|pw}} (Profeſſor\pwindex{\textcolor{red}{\textsuperscript{XXXX1 indx}}!Onkel Wanja. Szenen aus dem Landleben in vier Akten1896@\strich\emph{Onkel Wanja. Szenen aus dem Landleben in vier Akten} {[}1896{]}|pwv} in \textsc{Wanja\pwindex{\textcolor{red}{\textsuperscript{XXXX1 indx}}!Onkel Wanja. Szenen aus dem Landleben in vier Akten1896@\strich\emph{Onkel Wanja. Szenen aus dem Landleben in vier Akten} {[}1896{]}|pw}}), \textsc{Leonidow\pwindex{Leonidow, Leonid M. 22.03.1873 – 06.08.1941@\textsc{Leonidow, Leonid M.} (22.03.1873 – 06.08.1941), \emph{Regisseur, Schauspieler}|pw}}, Frau Tſchechow\pwindex{Cechowa, Olga L. 21.9.1868 – 22.03.1959@\textsc{Čechowa, Olga L.} (21.9.1868 – 22.03.1959), \emph{Schauspielerin}|pw} bei Rotenſtern’s\pwindex{Rotenstern, Peter 10.01.1868 – 1944@\textsc{Rotenstern, Peter} (10.01.1868 – 1944), \emph{Journalist, Übersetzer}|pw}\pwindex{Rotenstern-Tesi, Anna *~1871-01-11@\textsc{Rotenstern-Tesi, Anna} (*~1871-01-11), \emph{Übersetzerin}|pw} kennengelernt; auch im Theater hinter
               den Couliſſen ein paar mal geſprochen. Es hat mich ſehr gefreut, daſs ihnen viel
               daran zu liegen ſchien, ein Stück von mir für ihr Theater zu beko{\geminationm}en. Jedenfalls gibt es keins, an dem ich lieber
               aufgeführt werden möchte. Sieht man ſolche {\pb}um alles dramatiſche unbekü{\geminationm}erte Geſtalten – und
               Lebensſtücke wie den \textsc{Onkel Wanja\pwindex{\textcolor{red}{\textsuperscript{XXXX1 indx}}!Onkel Wanja. Szenen aus dem Landleben in vier Akten1896@\strich\emph{Onkel Wanja. Szenen aus dem Landleben in vier Akten} {[}1896{]}|pw}}, ſo ist einem, als braucht man ſich nur hinzuſetzen, um ein viertel Dutzend im
               Jahr zu ſchreiben. Und doch{\dots} Allerdings fiele man auch
               durch.–\pend
           \pstart
           Tennis ſpielen wir ſchon ziemlich regelmäßig – d. h. meiſtens ich, Dr \textsc{Kaufmann\pwindex{Kaufmann, Arthur 04.04.1872 – 25.07.1938@\textsc{Kaufmann, Arthur} (04.04.1872 – 25.07.1938), \emph{Rechtswissenschaftler, Privatgelehrte, Privatier}|pw}}, Frl \textsc{Erl\pwindex{Erl, Dora @\textsc{Erl, Dora}, \emph{Schauspielerin, Gesangspädagogin}|pw}}, Olga\pwindex{Schnitzler, Olga 17.01.1882 – 13.01.1970@\textsc{Schnitzler, Olga} (17.01.1882 – 13.01.1970), \emph{Schauspielerin, Sängerin}|pw} ſeltener. Zuweilen geh ich im Pötzleinsdorferwalde\oindex{Poetzleinsdorf@\textbf{Pötzleinsdorf}|pw} ſpaziren. Es iſt ſchon beinah
                  ſo{\geminationm}erlich, um mindeſten{[}s{]}
               vierzehn Tage weiter vor, als voriges Jahr. \label{K_L03004-1v}\edtext{Neulich war \textsc{Fred\pwindex{W. Fred 29.06.1879 – 23.10.1922@\textsc{W. Fred} (29.06.1879 – 23.10.1922), \emph{Schriftsteller, Journalist}|pw}} bei uns}{\lemma{\textnormal{\emph{Neulich war Fred bei uns}}}\Cendnote{\textnormal{siehe A. S.: \emph{Tagebuch}, 23. 4. 1906}}}\label{K_L03004-1h}, der ſich im Lauf der Jahre höchſt vorteilhaft verändert hat. (Dieſer {\pb}Tage wird er (wahrſcheinlich von meinem Bruder\pwindex{Schnitzler, Julius 13.07.1865 – 29.06.1939@\textsc{Schnitzler, Julius} (13.07.1865 – 29.06.1939), \emph{Chirurg}|pwv}) an Gallenſteinen
               operirt.) –\pend
           \pstart
           Über Ihre So{\geminationm}erpläne möcht ich recht bald näheres
               wiſſen. Meine Karte, Frau \textsc{v Lützow\pwindex{Luetzow, Linda von 05.09.1832 – 1922-07-04@\textsc{Lützow, Linda von} (05.09.1832 – 1922-07-04), \emph{Übersetzerin}|pw}} betreffend, haben Sie wohl erhalten? Neulich war hier das Gerücht verbreitet,
               daſs Sie auf ein paar Tage nach Wien\oindex{Wien@\textbf{Wien}|pw} kämen. Wie
               ſteht die \label{K_L03004-2v}\edtext{Proceſsangelegenheit}{\lemma{\textnormal{\emph{Proceſsangelegenheit}}}\Cendnote{\textnormal{siehe Felix Salten an Arthur Schnitzler, 9. 3. 1906}}}\label{K_L03004-2h}? Ich ſtelle mir Ludaſſy\pwindex{Gans-Ludassy, Julius von 13.04.1858 – 30.09.1922@\textsc{Gans-Ludassy, Julius von} (13.04.1858 – 30.09.1922), \emph{Schriftsteller, Journalist, Herausgeber}|pw} verda{\geminationm}t wenig dazu gelaunt vor.–\pend
           \pstart
           Neulich, mit dem reparirten Rad (alles mögliche, 55 Kronen!) erſter Verſuch, in Neuwaldegg\oindex{Neuwaldegg@\textbf{Neuwaldegg}|pw} brach die Axe. Trotzdem bleibt die
               Sehnſucht nach den gemeinſchaftlichen Partien beſtehen. Haben Sie ſich nicht die
               Sache wegen \label{K_L03004-3v}\edtext{Daenemark\oindex{Daenemark@\textbf{Dänemark}|pw}}{\lemma{\textnormal{\emph{Daenemark}}}\Cendnote{\textnormal{siehe Felix Salten an Arthur Schnitzler, 28. 3. 1906}}}\label{K_L03004-3h}{ }{\pb}überlegt?\pend
           \pstart
           Ich arbeite (am \label{K_L03004-4v}\edtext{Roman\pwindex{Schnitzler, Arthur 15.05.1862 – 21.10.1931@\textsc{Schnitzler, Arthur} (15.05.1862 – 21.10.1931), \emph{Schriftsteller, Mediziner}!Weg ins Freie. Roman1.1.1908 – 1.6.1908@\strich\emph{Der Weg ins Freie. Roman} {[}1.1.1908 – 1.6.1908{]}|pwv}}{\lemma{\textnormal{\emph{Roman}}}\Cendnote{\textnormal{\emph{Der Weg ins Freie}\pwindex{Schnitzler, Arthur 15.05.1862 – 21.10.1931@\textsc{Schnitzler, Arthur} (15.05.1862 – 21.10.1931), \emph{Schriftsteller, Mediziner}!Weg ins Freie. Roman1.1.1908 – 1.6.1908@\strich\emph{Der Weg ins Freie. Roman} {[}1.1.1908 – 1.6.1908{]}|pwk}}}}\label{K_L03004-4h}) ziemlich regelmäßig aber ohne die nöthige Intenſität. Mir thut es ſo leid,
               daſs ich Sie in der B. Z.\pwindex{?? Werk@Nicht ermittelte Verfasserinnen und Verfasser!B.Z. am Mittag1904-10-22 – 1943@\emph{B.Z. am Mittag} {[}1904-10-22 – 1943{]}|pw} beinah niemals finde.
               Was machen Sie ſonſt? Ich nehme an, daſs Sie mit adminiſtrativen und
               organiſatoriſchen Arbeiten überhäuft ſind.–\pend
           \pstart
           Seien Sie herzlich gegrüßt, ebenſo Otti\pwindex{Salten, Ottilie 07.03.1868 – 22.06.1942@\textsc{Salten, Ottilie} (07.03.1868 – 22.06.1942), \emph{Schauspielerin}|pw} u
               die Kinder\pwindex{Salten, Paul 11.08.1903 – 08.05.1937@\textsc{Salten, Paul} (11.08.1903 – 08.05.1937), \emph{Filmcutter}|pwv}\pwindex{Rehmann, Anna Katharina 18.08.1904 – 27.03.1977@\textsc{Rehmann, Anna Katharina} (18.08.1904 – 27.03.1977), \emph{Schauspielerin, Übersetzerin}|pwv}, von uns\pwindex{Schnitzler, Olga 17.01.1882 – 13.01.1970@\textsc{Schnitzler, Olga} (17.01.1882 – 13.01.1970), \emph{Schauspielerin, Sängerin}|pwv}\pwindex{Schnitzler, Heinrich 09.08.1902 – 12.07.1982@\textsc{Schnitzler, Heinrich} (09.08.1902 – 12.07.1982), \emph{Regisseur, Schauspieler}|pwv} allen. {\\[\baselineskip]}Ihr {\\[\baselineskip]}\spacefill\mbox{A.}\pend
           \leftskip=0em{}
         
         \endnumbering\mylabel{h}\end{ledgroupsized}  \newcommand{\dateiname}{L03004}\newcommand{\titel}{Arthur Schnitzler an Felix Salten, 27. 4. 1906}\newcommand{\editorInnen}{Martin Anton Müller und Laura Untner}%% latex-leseansicht-abspann.tex
%% Abspann für die Leseansicht.
%% Der Schalter \ifkorrekturansicht ist bereits durch den Vorspann gesetzt.

%% latex-abspann.tex
%% Gemeinsamer Abspann für Korrekturansicht und Leseansicht.
%% Setzt den Schalter \ifkorrekturansicht voraus (gesetzt in den
%% einbindenden Dateien latex-korrekturansicht-abspann.tex bzw.
%% latex-leseansicht-abspann.tex).
%% ---------------------------------------------------------------

\normalsize

% Das esempio-Environment wird nur in der Leseansicht benötigt
\ifkorrekturansicht\else
\newenvironment{esempio}[3]%
{
    \vspace{1.5ex}
    \rlap{\underline{#1}}
    \par
    \setlength{\parindent}{0cm}
    \nopagebreak
    \leftskip=#2cm
    \rightskip=#3cm
}
{
    \par
}
\fi

\doendnotes{C}
\bigskip
\vfill

\clearpage

\footnotesize

\ifkorrekturansicht
  \lohead{\textsc{register}}
\fi

% theindex-Environment neu definieren ohne reledmac
\makeatletter
\renewenvironment{theindex}{%
  \ifkorrekturansicht
    \section*{\indexname}%
  \else
    \subsubsection*{Index der erwähnten Entitäten}%
  \fi
  \setlength{\parindent}{0pt}%
  \setlength{\parskip}{0pt plus 0.3pt}%
  \let\item\@idxitem
}{%
  \ifkorrekturansicht\clearpage\fi
}
\makeatother

\IfFileExists{\jobname-pw.ind}{\input{\jobname-pw.ind}}{}

% Quellenangabe nur in der Leseansicht
\ifkorrekturansicht\else
% Fallback-Definitionen, falls die .tex-Datei \titel etc. nicht gesetzt hat
\providecommand{\titel}{}
\providecommand{\editorInnen}{}
\providecommand{\dateiname}{\jobname}

\vspace{3cm}

\vfill

\footnotesize
\textsc{Quelle}: \titel. Herausgegeben von {\editorInnen}. In: \emph{Arthur Schnitzler: Briefwechsel mit Autorinnen und Autoren}.
 Digitale Edition, https://schnitzler-briefe.acdh.oeaw.ac.at/{\dateiname}.html (Stand \today)
\fi

\end{document}


      