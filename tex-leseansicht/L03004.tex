%% latex-korrekturansicht-vorspann.tex
%% Vorspann für die Korrekturansicht.
%% Lädt die gemeinsame Datei latex-vorspann.tex mit gesetztem Schalter.

\newif\ifkorrekturansicht
\korrekturansichttrue

\input{../tex-inputs/latex-vorspann}


\section[ Arthur Schnitzler an Felix Salten, 27. 4. 1906]{L03004 Arthur Schnitzler an Felix Salten, 27. 4. 1906}
\nopagebreak\mylabel{L03004v}
\rehead{ }\normalsize\beginnumbering\briefempfaengerindex{Salten, Felix@\textsc{Salten, Felix}!zzzSchnitzler, Arthur@\emph{von Arthur Schnitzler}!1906-04-271@{27. 4. 1906}|(be}
\toendnotes[C]{\smallbreak\pagebreak[2]}\Standort{Wienbibliothek im Rathaus, ZPH 1681, 2.1.516.}
\physDesc{Brief, 2 Blätter, 7 Seiten, 3652 Zeichen
\newline{}Handschrift: schwarze Tinte, deutsche Kurrent
\newline{}Ordnung: mit Bleistift von unbekannter Hand Nummerierung der Doppelseiten des
                                 Konvoluts: »16«–»19«  }
\buchAbdrucke{\weitereDrucke{Arthur Schnitzler: \emph{Briefe 1875–1912}. Frankfurt am Main: \emph{S. Fischer} 1981, S. 529–531.} }\toendnotes[C]{\smallbreak}
\pstart
           \textcolor{gray}{\textbf{Dr. Arthur Schnitzler}}\hfill {\pb}27. 4. 906\pend
           
\pstart
           \textcolor{gray}{\textbf{Wien, XVIII. Spoettelgasse 7\oindex{Edmund-Weiss-Gasse 7@\textbf{Edmund-Weiß-Gasse 7}, \emph{Wohngebäude (K.WHS)}|pw}.}}\pend
           \vspace{0.5em}
\pstart
           lieber, Sie haben natürlich ganz recht. Unmöglich konnten Sie ſich
                  Brahm\pwindex{Brahm, Otto 05.02.1856 – 28.11.1912@\textsc{Brahm, Otto} (05.02.1856 – 28.11.1912), \emph{Theaterleiter/Theaterleiterin, Regisseur/Regisseurin}|pw} gegenüber als ungebetener Rathgeber
               aufſpielen, und als ich mein Telegra{\geminationm} an Sie abſandte,
               hatt ich begreiflicherweiſe nicht an irgend einen \textsc{adhoc}-Beſuch od dergl bei Brahm\pwindex{Brahm, Otto 05.02.1856 – 28.11.1912@\textsc{Brahm, Otto} (05.02.1856 – 28.11.1912), \emph{Theaterleiter/Theaterleiterin, Regisseur/Regisseurin}|pw} gedacht,
               ſondern an etwas beiläufigeres, ohne mir über das »wie« weitere Gedanken zu machen.
               (Damit dſs Brahm\pwindex{Brahm, Otto 05.02.1856 – 28.11.1912@\textsc{Brahm, Otto} (05.02.1856 – 28.11.1912), \emph{Theaterleiter/Theaterleiterin, Regisseur/Regisseurin}|pw} auf Ihr Urtheil nichts geben
               könnte, ſind Sie ſehr im Irrtum.) – Nun hab ich die Sache indeſs auf andre, directe
               Weise zu ordnen geſucht. {\pb}(\uline{Dies vollko{\geminationm}en unter
                  uns.}) Nach Ihrem Brief, in dem Sie mir Ihr Geſpräch mit R.\pwindex{Rittner, Rudolf 30.06.1869 – 04.02.1943@\textsc{Rittner, Rudolf} (30.06.1869 – 04.02.1943), \emph{Theaterleiter/Theaterleiterin, Schauspieler/Schauspielerin}|pw} erzählten u einen \label{K_L03004-1v}\edtext{Brief Jacobsohns\pwindex{Jacobsohn, Siegfried 28.01.1881 – 03.12.1926@\textsc{Jacobsohn, Siegfried} (28.01.1881 – 03.12.1926), \emph{Journalist/Journalistin, Kritiker/Kritikerin, Publizist/Publizistin}|pw}}{\lemma{\textnormal{\emph{Brief Jacobsohns}}}\Cendnote{\textnormal{Im Brief Jacobsohns\pwindex{Jacobsohn, Siegfried 28.01.1881 – 03.12.1926@\textsc{Jacobsohn, Siegfried} (28.01.1881 – 03.12.1926), \emph{Journalist/Journalistin, Kritiker/Kritikerin, Publizist/Publizistin}|pwk} vom 20. 4. 1906.
                   heißt es: »Verhindern Sie, wenns irgend geht, dß Reicher\pwindex{Reicher, Emanuel 18.06.1849 – 15.05.1924@\textsc{Reicher, Emanuel} (18.06.1849 – 15.05.1924), \emph{Schauspieler/Schauspielerin}|pw} in Wien\oindex{Wien@\textbf{Wien}, \emph{A.ADM2}|pw}
                     Ihren Julian Fichtner\pwindex{einsame Weg. Schauspiel in fuenf Akten@\emph{Der einsame Weg. Schauspiel in fünf Akten}|pwv}
                      spielt. Es war eine Schmach, was sich gestern im Lessing-Theater\orgindex{Lessing-Theater@Lessing-Theater|pw} abspielte. Der
                     Mann kann kein Wort von der Rolle. Die Souffleuse schrie sich heiser.«
                        (\emph{CUL}, B 46). Vgl. A. S.: \emph{Tagebuch}, 21. 4. 1906.}}}\label{K_L03004-1}, der auch
               telephoniſch eine Art Bereitwilligkeit R.s\pwindex{Rittner, Rudolf 30.06.1869 – 04.02.1943@\textsc{Rittner, Rudolf} (30.06.1869 – 04.02.1943), \emph{Theaterleiter/Theaterleiterin, Schauspieler/Schauspielerin}|pw}
               erfahren haben wollte, telegr ich an Brahm\pwindex{Brahm, Otto 05.02.1856 – 28.11.1912@\textsc{Brahm, Otto} (05.02.1856 – 28.11.1912), \emph{Theaterleiter/Theaterleiterin, Regisseur/Regisseurin}|pw}, ob
               er mir überlaſſen wolle \textsc{Rittner}\pwindex{Rittner, Rudolf 30.06.1869 – 04.02.1943@\textsc{Rittner, Rudolf} (30.06.1869 – 04.02.1943), \emph{Theaterleiter/Theaterleiterin, Schauspieler/Schauspielerin}|pw} zur Übernahme zu bewegen. Er konnte nichts dagegen haben, warnte mich für alle
               Fälle, wuſch ſeine Hände in Unſchuld \textsc{etc.} Ich telegr. nun
               an \textsc{Rittner}\pwindex{Rittner, Rudolf 30.06.1869 – 04.02.1943@\textsc{Rittner, Rudolf} (30.06.1869 – 04.02.1943), \emph{Theaterleiter/Theaterleiterin, Schauspieler/Schauspielerin}|pw}, der mir in einem ſehr liebenswürdigen Telegra{\geminationm}
               nein ſagte. Ich hatte es natürlich nicht anders erwartet – die Gegengründe lagen für
                  Rittner\pwindex{Rittner, Rudolf 30.06.1869 – 04.02.1943@\textsc{Rittner, Rudolf} (30.06.1869 – 04.02.1943), \emph{Theaterleiter/Theaterleiterin, Schauspieler/Schauspielerin}|pw} zu nah, als daſs er nicht von ihnen
               hätte {\pb}Gebrauch machen ſollen. Aber ich
               wollte mir keine Vorwürfe zu machen haben – und da mir \textsc{Rittner}\pwindex{Rittner, Rudolf 30.06.1869 – 04.02.1943@\textsc{Rittner, Rudolf} (30.06.1869 – 04.02.1943), \emph{Theaterleiter/Theaterleiterin, Schauspieler/Schauspielerin}|pw} ſtrengſte Discretion zugeſagt hat, hoffe ich daſs nicht am End noch eine für
               die Wien\oindex{Wien@\textbf{Wien}, \emph{A.ADM2}|pw}er Aufführg (auf die ich ſchließlich doch
               nicht verzichten möchte) gefährliche Couliſſenklatſcherei heraus ko{\geminationm}t. Sonderbar iſt, daſs vor 2 Jahren, nach Rittners\pwindex{Rittner, Rudolf 30.06.1869 – 04.02.1943@\textsc{Rittner, Rudolf} (30.06.1869 – 04.02.1943), \emph{Theaterleiter/Theaterleiterin, Schauspieler/Schauspielerin}|pw} Verſagen (aus Unluſt) an der Rolle
               alle, auch Brahm\pwindex{Brahm, Otto 05.02.1856 – 28.11.1912@\textsc{Brahm, Otto} (05.02.1856 – 28.11.1912), \emph{Theaterleiter/Theaterleiterin, Regisseur/Regisseurin}|pw} und ich dachten, Reicher\pwindex{Reicher, Emanuel 18.06.1849 – 15.05.1924@\textsc{Reicher, Emanuel} (18.06.1849 – 15.05.1924), \emph{Schauspieler/Schauspielerin}|pw} wäre der richtige Darſteller für die
               Rolle. Nach der erſchütternden Charakteriſtik, die Sie von ſeiner Auffaſſung geben,
                  ka{\geminationn} ich mir nun wohl vorſtellen, was mir {\pb}bevorſteht. Übrigens gibt es meiner Empfindg
               nach nur einen Darſteller für den \textsc{Julian\pwindex{einsame Weg. Schauspiel in fuenf Akten@\emph{Der einsame Weg. Schauspiel in fünf Akten}|pwv}}: \textsc{Wischnevski\pwindex{Wischnewski, Alexander Leonidowitsch 20.01.1861 – 27.02.1943@\textsc{Wischnewski, Alexander Leonidowitsch} (20.01.1861 – 27.02.1943), \emph{Schauspieler/Schauspielerin}|pw}}. Sie haben ihn ja als Onkel \textsc{Wanja}\pwindex{Onkel Wanja. Szenen aus dem Landleben in vier Akten@\emph{Onkel Wanja. Szenen aus dem Landleben in vier Akten}|pw} geſehen. Und \textsc{Stanislawski\pwindex{Stanislavskij, Konstantin S. 17.01.1863 – 07.08.1938@\textsc{Stanislavskij, Konstantin S.} (17.01.1863 – 07.08.1938), \emph{Theaterleiter/Theaterleiterin, Regisseur/Regisseurin, Schauspieler/Schauspielerin}|pw}} als \textsc{Sala\pwindex{einsame Weg. Schauspiel in fuenf Akten@\emph{Der einsame Weg. Schauspiel in fünf Akten}|pwv}} wär auch nicht übel. Wir haben dieſe beiden, auch \textsc{Ljuschin\pwindex{Južin, Aleksandr I. 1857-09-16 – 1927-09-27@\textsc{Južin, Aleksandr I.} (1857-09-16 – 1927-09-27), \emph{Theaterleiter/Theaterleiterin, Schauspieler/Schauspielerin}|pw}} (Profeſſor\pwindex{Onkel Wanja. Szenen aus dem Landleben in vier Akten@\emph{Onkel Wanja. Szenen aus dem Landleben in vier Akten}|pwv} in \textsc{Wanja\pwindex{Onkel Wanja. Szenen aus dem Landleben in vier Akten@\emph{Onkel Wanja. Szenen aus dem Landleben in vier Akten}|pw}}), \textsc{Leonidow\pwindex{Leonidow, Leonid M. 22.03.1873 – 06.08.1941@\textsc{Leonidow, Leonid M.} (22.03.1873 – 06.08.1941), \emph{Regisseur/Regisseurin, Schauspieler/Schauspielerin}|pw}}, Frau Tſchechow\pwindex{Cechowa, Olga L. 21.9.1868 – 22.03.1959@\textsc{Čechowa, Olga L.} (21.9.1868 – 22.03.1959), \emph{Schauspieler/Schauspielerin}|pw}{ }\label{K_L03004-2v}\edtext{bei Rotenſtern’s\pwindex{Rotenstern, Peter 10.01.1868 – 1944@\textsc{Rotenstern, Peter} (10.01.1868 – 1944), \emph{Journalist/Journalistin, Übersetzer/Übersetzerin}|pw}\pwindex{Rotenstern-Tesi, Anna *~1871-01-11@\textsc{Rotenstern-Tesi, Anna} (*~1871-01-11), \emph{Übersetzer/Übersetzerin}|pw} kennengelernt}{\lemma{\textnormal{\emph{bei … kennengelernt}}}\Cendnote{\textnormal{Vgl. A. S.: \emph{Tagebuch}, 17. 4. 1906.
               }}}\label{K_L03004-2}; auch im Theater \label{K_L03004-3v}\edtext{hinter den Couliſſen ein paar mal geſprochen}{\lemma{\textnormal{\emph{hinter … geſprochen}}}\Cendnote{\textnormal{Vgl. A. S.: \emph{Tagebuch}, 18. 4. 1906.
               }}}\label{K_L03004-3}. Es hat mich ſehr gefreut, daſs ihnen
               viel daran zu liegen ſchien, ein Stück von mir für ihr Theater zu beko{\geminationm}en. Jedenfalls gibt es keins, an dem ich lieber
               aufgeführt werden möchte. Sieht man ſolche {\pb}um alles dramatiſche unbekü{\geminationm}erte Geſtalten- und
               Lebensſtücke wie den Onkel \textsc{Wanja}\pwindex{Onkel Wanja. Szenen aus dem Landleben in vier Akten@\emph{Onkel Wanja. Szenen aus dem Landleben in vier Akten}|pw}, ſo ist einem, als braucht man ſich nur hinzuſetzen, um ein viertel Dutzend im
               Jahr zu ſchreiben. Und doch{\dots} Allerdings fiele man auch
               durch. –\pend
           
\pstart
           Tennis ſpielen wir ſchon ziemlich regelmäßig – d. h. meiſtens ich, Dr \textsc{Kaufmann\pwindex{Kaufmann, Arthur 04.04.1872 – 25.07.1938@\textsc{Kaufmann, Arthur} (04.04.1872 – 25.07.1938), \emph{Rechtswissenschaftler/Rechtswissenschaftlerin, Privatgelehrte/Privatgelehrte, Privatier/Privatière}|pw}}, Frl \textsc{Erl\pwindex{Erl, Dora @\textsc{Erl, Dora}, \emph{Schauspieler/Schauspielerin, Gesangspädagoge/Gesangspädagogin}|pw}}, Olga\pwindex{Schnitzler, Olga 17.01.1882 – 13.01.1970@\textsc{Schnitzler, Olga} (17.01.1882 – 13.01.1970), \emph{Schauspieler/Schauspielerin, Sänger/Sängerin}|pw} ſeltener. Zuweilen geh ich im Pötzleinsdorferwalde\oindex{Poetzleinsdorf@\textbf{Pötzleinsdorf}, \emph{P.PPLX}|pw} ſpaziren. Es iſt ſchon beinah
                  ſo{\geminationm}erlich, um mindeſten\textcolor{gray}{s} vierzehn
               Tage weiter vor, als voriges Jahr. \label{K_L03004-4v}\edtext{Neulich war \textsc{Fred\pwindex{W. Fred 29.06.1879 – 23.10.1922@\textsc{W. Fred} (29.06.1879 – 23.10.1922), \emph{Schriftsteller/Schriftstellerin, Journalist/Journalistin}|pw}} bei uns}{\lemma{\textnormal{\emph{Neulich war Fred bei uns}}}\Cendnote{\textnormal{Siehe A. S.: \emph{Tagebuch}, 23. 4. 1906.
               }}}\label{K_L03004-4}, der ſich im Lauf der Jahre höchſt vorteilhaft verändert hat. (Dieſer {\pb}Tage wird er (wahrſcheinlich von meinem Bruder\pwindex{Schnitzler, Julius 13.07.1865 – 29.06.1939@\textsc{Schnitzler, Julius} (13.07.1865 – 29.06.1939), \emph{Chirurg/Chirurgin}|pwv}) an Gallenſteinen
               operirt.) –\pend
           
\pstart
           Über Ihre So{\geminationm}erpläne möcht ich recht bald näheres
               wiſſen. Meine Karte, Frau \textsc{v Lützow\pwindex{Luetzow, Linda von 05.09.1832 – 1922-07-04@\textsc{Lützow, Linda von} (05.09.1832 – 1922-07-04), \emph{Übersetzer/Übersetzerin}|pw}} betreffend, haben Sie wohl erhalten? Neulich war hier das Gerücht verbreitet,
               daſs Sie auf ein paar Tage nach Wien\oindex{Wien@\textbf{Wien}, \emph{A.ADM2}|pw} kämen. Wie
               ſteht die \label{K_L03004-5v}\edtext{Proceſsangelegenheit}{\lemma{\textnormal{\emph{Proceſsangelegenheit}}}\Cendnote{\textnormal{Siehe Felix Salten an Arthur Schnitzler, 9. 3. 1906.
               }}}\label{K_L03004-5}? Ich ſtelle mir Ludaſſy\pwindex{Gans-Ludassy, Julius von 13.04.1858 – 30.09.1922@\textsc{Gans-Ludassy, Julius von} (13.04.1858 – 30.09.1922), \emph{Schriftsteller/Schriftstellerin, Journalist/Journalistin, Herausgeber/Herausgeberin}|pw} verda{\geminationm}t wenig dazu gelaunt vor. –\pend
           
\pstart
           \label{K_L03004-6v}\edtext{Neulich, mit dem reparirten Rad}{\lemma{\textnormal{\emph{Neulich, … Rad}}}\Cendnote{\textnormal{Vgl. A. S.: \emph{Tagebuch}, 17. 4. 1906.
               }}}\label{K_L03004-6} (alles mögliche, 55 Kronen!) erſter
               Verſuch, in Neuwaldegg\oindex{Neuwaldegg@\textbf{Neuwaldegg}, \emph{P.PPLX}|pw} brach die Axe. Trotzdem
               bleibt die Sehnſucht nach den gemeinſchaftlichen Partien beſtehen. Haben Sie ſich
               nicht die Sache wegen \label{K_L03004-7v}\edtext{Daenemark\oindex{Daenemark@\textbf{Dänemark}, \emph{A.PCLI}|pw}}{\lemma{\textnormal{\emph{Daenemark}}}\Cendnote{\textnormal{Siehe Felix Salten an Arthur Schnitzler, 28. 3. 1906.
               }}}\label{K_L03004-7}{ }{\pb}überlegt? –\pend
           
\pstart
           Ich arbeite (am Roman\pwindex{Weg ins Freie. Roman@\emph{Der Weg ins Freie. Roman}|pwv})
               ziemlich regelmäßig aber ohne die nöthige Intenſität. Mir thut es ſo leid, daſs ich
               Sie in der B. Z.\pwindex{B.Z. am Mittag@\emph{B.Z. am Mittag}|pw} beinah niemals finde. Was
               machen Sie ſonſt? Ich nehme an, daſs Sie mit adminiſtrativen und organiſatoriſchen
               Arbeiten überhäuft ſind. –\pend
           
\pstart
           Seien Sie herzlich gegrüßt, ebenſo Otti\pwindex{Salten, Ottilie 07.03.1868 – 22.06.1942@\textsc{Salten, Ottilie} (07.03.1868 – 22.06.1942), \emph{Schauspieler/Schauspielerin}|pw} u
               die Kinder\pwindex{Salten, Paul 11.08.1903 – 08.05.1937@\textsc{Salten, Paul} (11.08.1903 – 08.05.1937), \emph{Filmcutter/Filmcutterin}|pwv}\pwindex{Rehmann, Anna Katharina 18.08.1904 – 27.03.1977@\textsc{Rehmann, Anna Katharina} (18.08.1904 – 27.03.1977), \emph{Schauspieler/Schauspielerin, Übersetzer/Übersetzerin}|pwv}, von uns\pwindex{Schnitzler, Olga 17.01.1882 – 13.01.1970@\textsc{Schnitzler, Olga} (17.01.1882 – 13.01.1970), \emph{Schauspieler/Schauspielerin, Sänger/Sängerin}|pwv}\pwindex{Schnitzler, Heinrich 09.08.1902 – 12.07.1982@\textsc{Schnitzler, Heinrich} (09.08.1902 – 12.07.1982), \emph{Regisseur/Regisseurin, Schauspieler/Schauspielerin}|pwv} allen. {\\[\baselineskip]}Ihr {\\[\baselineskip]}\spacefill\mbox{A.}\pend
           \leftskip=0em{}\selectlanguage{ngerman}\endnumbering\briefempfaengerindex{Salten, Felix@\textsc{Salten, Felix}!zzzSchnitzler, Arthur@\emph{von Arthur Schnitzler}!1906-04-271@{27. 4. 1906}|)be}\mylabel{L03004h}  \normalsize

\doendnotes{C}
\bigskip
\vfill

\clearpage

\footnotesize

\lohead{\textsc{register}}

% Definiere theindex-Environment komplett neu ohne reledmac
\makeatletter
\renewenvironment{theindex}{%
  \section*{\indexname}%
  \setlength{\parindent}{0pt}%
  \setlength{\parskip}{0pt plus 0.3pt}%
  \let\item\@idxitem
}{%
  \clearpage
}
\makeatother

\IfFileExists{\jobname-pw.ind}{\input{\jobname-pw.ind}}{}

\end{document}

      