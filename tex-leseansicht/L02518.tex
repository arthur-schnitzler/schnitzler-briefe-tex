%% latex-korrekturansicht-vorspann.tex
%% Vorspann für die Korrekturansicht.
%% Lädt die gemeinsame Datei latex-vorspann.tex mit gesetztem Schalter.

\newif\ifkorrekturansicht
\korrekturansichttrue

\input{../tex-inputs/latex-vorspann}


\section[Arthur Schnitzler an Gerty von Hofmannsthal, 12. 8. 1929]{L02518 Arthur Schnitzler an Gerty von Hofmannsthal, 12. 8. 1929}
\nopagebreak\mylabel{L02518v}
\rehead{ }\normalsize\beginnumbering\briefempfaengerindex{Hofmannsthal, Gertrude von@\textsc{Hofmannsthal, Gertrude von}!zzzSchnitzler, Arthur@\emph{von Arthur Schnitzler}!1929-08-121@{12. 8. 1929}|(be}
\toendnotes[C]{\smallbreak\pagebreak[2]}\Standort{FDH, Hs-31346,3.}
\physDesc{Brief, 1 Blatt, 1 Seite, 495 Zeichen (Briefpapier mit Trauerrand)
\newline{}Handschrift: schwarze Tinte, lateinische Kurrent}
\pstart
           \raggedleft{}{\pb}Wien\oindex{Wien@\textbf{Wien}, \emph{A.ADM2}|pw}, 12/8 929\pend
           \vspace{0.5em}
\pstart
           liebe Gerty, also Fräulein Pollak\pwindex{Pollak, Frieda 08.12.1881 – 13.07.1937@\textsc{Pollak, Frieda} (08.12.1881 – 13.07.1937), \emph{Sekretär/Sekretärin}|pw} besorgt Ihnen die Briefe, wie ich eben an Christiane\pwindex{Zimmer, Christiane 14.05.1902 – 05.01.1987@\textsc{Zimmer, Christiane} (14.05.1902 – 05.01.1987)|pw} schrieb. We{\geminationn} Sie eine
               Beihilfe zum Abschreiben der Briefe benötigen – es ist nicht viel, – bringe ich \uline{Magda Pollaczek}\pwindex{Pollaczek, Magda 1909-06-20 – 2006-09-20@\textsc{Pollaczek, Magda} (1909-06-20 – 2006-09-20), \emph{Schreiber/Schreiberin}|pw} in Vorschlag;– sie hat in der letzten Zeit manches, auch schwer leserliches für
               mich abgeschrieben und macht das ausgezeichnet.\pend
           
\pstart
           Ich bin noch immer hier, morgen ko{\geminationm}t Heini\pwindex{Schnitzler, Heinrich 09.08.1902 – 12.07.1982@\textsc{Schnitzler, Heinrich} (09.08.1902 – 12.07.1982), \emph{Regisseur/Regisseurin, Schauspieler/Schauspielerin}|pw}, etwa in 8 Tagen dürfte ich abreisen, – vermutlich in
               die französ. Schweiz\oindex{Schweiz@\textbf{Schweiz}, \emph{A.PCLI}|pw}. \pend
           
\pstart
           Alles herzliche, auf Wiedersehen,{\\[\baselineskip]}Ihr{\\[\baselineskip]}\spacefill\mbox{Arthur}\pend
           \leftskip=0em{}\selectlanguage{ngerman}\endnumbering\briefempfaengerindex{Hofmannsthal, Gertrude von@\textsc{Hofmannsthal, Gertrude von}!zzzSchnitzler, Arthur@\emph{von Arthur Schnitzler}!1929-08-121@{12. 8. 1929}|)be}\mylabel{L02518h}  \normalsize

\doendnotes{C}
\bigskip
\vfill

\clearpage

\footnotesize

\lohead{\textsc{register}}

% Definiere theindex-Environment komplett neu ohne reledmac
\makeatletter
\renewenvironment{theindex}{%
  \section*{\indexname}%
  \setlength{\parindent}{0pt}%
  \setlength{\parskip}{0pt plus 0.3pt}%
  \let\item\@idxitem
}{%
  \clearpage
}
\makeatother

\IfFileExists{\jobname-pw.ind}{\input{\jobname-pw.ind}}{}

\end{document}

      