\input{../tex-inputs/latex-pdf-vorspann}
\begin{center}
            \textcolor{red}{ENTWURF. ENTZIFFERUNG NOCH NICHT KORREKTURGELESEN}
                      \end{center}
            
               \section[Arthur Schnitzler an Gerty von Hofmannsthal, 12. 8. 1929]{ Arthur Schnitzler an Gerty von Hofmannsthal, 12. 8. 1929}\nopagebreak\mylabel{v}\rehead{ }\begin{ledgroupsized}[t]{13cm}\normalsize\beginnumbering\briefempfaengerindex{Hofmannsthal, Gertrude von@\textsc{Hofmannsthal, Gertrude von}!zzzSchnitzler, Arthur@\emph{von Arthur Schnitzler}!1929-08-121@{12. 8. 1929}|(be} \toendnotes[C]{\smallbreak\pagebreak[2]} \Standort{FDH, Hs-31346,3.}
\physDesc{Brief, 1 Blatt (Briefpapier mit Trauerrand), 1 Seite
\newline{}Handschrift: schwarze Tinte, lateinische Kurrent}\pstart
           \raggedleft{}{\pb}Wien\oindex{Wien@\textbf{Wien}|pw}, 12/8 929\pend
           \pstart
           liebe Gerty, also Fräulein Pollak\pwindex{Pollak, Frieda 08.12.1881 – 13.07.1937@\textsc{Pollak, Frieda} (08.12.1881 – 13.07.1937), \emph{Sekretärin}|pw} besorgt Ihnen die Briefe, wie ich eben an Christiane\pwindex{Hofmannsthal, Christiane von 14.05.1902 – 05.01.1987@\textsc{Hofmannsthal, Christiane von} (14.05.1902 – 05.01.1987)|pw} schrieb. We{\geminationn} Sie eine
               Beihilfe zum Abschreiben der Briefe benötigen – es ist nicht viel, – bringe ich \uline{Magda Pollaczek}\pwindex{Pollaczek, Magda 1909-06-20 – 2006-09-20@\textsc{Pollaczek, Magda} (1909-06-20 – 2006-09-20), \emph{Schreiberin}|pw} in Vorschlag;– sie hat in der letzten Zeit manches, auch schwer leserliches für
               mich abgeschrieben und macht das ausgezeichnet.\pend
           \pstart
           Ich bin noch immer hier, morgen ko{\geminationm}t Heini\pwindex{Schnitzler, Heinrich 09.08.1902 – 12.07.1982@\textsc{Schnitzler, Heinrich} (09.08.1902 – 12.07.1982), \emph{Regisseur, Schauspieler}|pw}, etwa in 8 Tagen dürfte ich abreisen, – vermutlich in die
                  französ. Schweiz\oindex{Schweiz@\textbf{Schweiz}|pw}. \pend
           \pstart
           Alles herzliche, auf Wiedersehen,{\\[\baselineskip]}Ihr{\\[\baselineskip]}\spacefill\mbox{Arthur}\pend
           \leftskip=0em{}\endnumbering\briefempfaengerindex{Hofmannsthal, Gertrude von@\textsc{Hofmannsthal, Gertrude von}!zzzSchnitzler, Arthur@\emph{von Arthur Schnitzler}!1929-08-121@{12. 8. 1929}|)be}\mylabel{h}\end{ledgroupsized}  \newcommand{\dateiname}{L02518}\newcommand{\titel}{Arthur Schnitzler an Gerty von Hofmannsthal, 12. 8. 1929}\newcommand{\editorInnen}{Martin Anton Müller und Gerd-Hermann Susen}\input{../tex-inputs/latex-pdf-abspann}
      