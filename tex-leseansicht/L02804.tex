%% latex-leseansicht-vorspann.tex
%% Vorspann für die Leseansicht.
%% Lädt die gemeinsame Datei latex-vorspann.tex mit nicht gesetztem Schalter.

\newif\ifkorrekturansicht
\korrekturansichtfalse

\input{../tex-inputs/latex-vorspann}


\section[ Paul Goldmann an Arthur Schnitzler, 24. 2. [1897]]{L02804 Paul Goldmann an Arthur Schnitzler,  24. 2. [1897]}
\nopagebreak\mylabel{L02804v}
\rehead{ }\normalsize\beginnumbering\briefempfaengerindex{Schnitzler, Arthur@\textsc{Schnitzler, Arthur}!zzzGoldmann, Paul@\emph{von Paul Goldmann}!1897-02-241@{24. 2. [1897]}|(be}
\toendnotes[C]{\smallbreak\pagebreak[2]}
\correspDesc{Versand  durch Paul Goldmann am 24. 2. [1897] in Paris
\newline{}Erhalt  durch Arthur Schnitzler im Zeitraum [25. 2. 1897
                  – 1. 3. 1897?] in Wien}\toendnotes[C]{\smallbreak}
\Standort{DLA, A:Schnitzler, HS.NZ85.1.3167.}
\physDesc{Brief, 1 Blatt, 4 Seiten, 1941 Zeichen
\newline{}Handschrift: blaue Tinte, deutsche Kurrent
\newline{}Schnitzler: mit Bleistift das Jahr »97« vermerkt }\toendnotes[C]{\smallbreak}
\pstart
           {\pb}\textcolor{gray}{\textbf{\textbf{Frankfurter Zeitung\orgindex{Frankfurter Zeitung@Frankfurter Zeitung|pw}}}}\pend
           
\pstart
           \textcolor{gray}{\textbf{(\begin{otherlanguage}{french}Gazette de Francfort\end{otherlanguage}\orgindex{Frankfurter Zeitung@Frankfurter Zeitung|pw}).}}\pend
           
\pstart
           \textcolor{gray}{\textbf{\textbf{\begin{otherlanguage}{french}Fondateur M.\end{otherlanguage}{ }L. Sonnemann\pwindex{Sonnemann, Leopold 29.\,10.\,1831 Höchberg – 30.\,10.\,1909 Frankfurt am Main@\textsc{Sonnemann, Leopold} (29.\,10.\,1831 Höchberg – 30.\,10.\,1909 Frankfurt am Main), \emph{Journalist, Herausgeber}|pw}.}}}\pend
           
\pstart
           \begin{otherlanguage}{french}\textcolor{gray}{\textbf{Journal politique, financier,}}\end{otherlanguage}\hfill \textsc{Paris\oindex{Paris@\textbf{Paris}, \emph{Hauptstadt}|pw}}, 24. Februar.\pend
           
\pstart
           \begin{otherlanguage}{french}\textcolor{gray}{\textbf{commercial et littéraire.}}\end{otherlanguage}\pend
           
\pstart
           \begin{otherlanguage}{french}\textcolor{gray}{\textbf{\textbf{Paraissant trois fois par jour.}}}\end{otherlanguage}\pend
           
\pstart
           \begin{otherlanguage}{french}\textcolor{gray}{\textbf{\textbf{Bureau à Paris\oindex{Paris@\textbf{Paris}, \emph{Hauptstadt}|pw}}}}\end{otherlanguage}\pend
           
\pstart
           \begin{otherlanguage}{french}\textcolor{gray}{\textbf{\textbf{24. Rue Feydeau\oindex{rue Feydeau@\textbf{rue Feydeau}, \emph{Straße}|pw}.}}}\end{otherlanguage}\pend
           
\pstart\center{}Mein lieber Freund,\pend\vspace{0.5em}
\pstart
           Du{ }ſchreibſt mir wohl umgehend ein kurzes Wort über die Art, wie der Vater\pwindex{Reinhard, Karl 2.\,3.\,1834 Prag – 28.\,4.\,1905 Wien@\textsc{Reinhard, Karl} (2.\,3.\,1834 Prag – 28.\,4.\,1905 Wien), \emph{Geschäftsführer}|pwv}{ }\strikeout{die} die \label{K_L02804-1v}\edtext{Sache}{\lemma{\textnormal{\emph{Sache}}}\Cendnote{\textnormal{Carl Reinhard\pwindex{Reinhard, Karl 2.\,3.\,1834 Prag – 28.\,4.\,1905 Wien@\textsc{Reinhard, Karl} (2.\,3.\,1834 Prag – 28.\,4.\,1905 Wien), \emph{Geschäftsführer}|pwk} wurde am 23. 2. 1897 über Marie Reinhards\pwindex{Reinhard, Marie 13.\,3.\,1871 Wien – 18.\,3.\,1899 ebd.@\textsc{Reinhard, Marie} (13.\,3.\,1871 Wien – 18.\,3.\,1899 ebd.), \emph{Gesangspädagogin}|pwk} Schwangerschaft informiert.
                  Laut Schnitzlers{ }\emph{Tagebuch}\pwindex{Schnitzler, Arthur 15.\,5.\,1862 Wien – 21.\,10.\,1931 ebd.@\textsc{Schnitzler, Arthur} (15.\,5.\,1862 Wien – 21.\,10.\,1931 ebd.), \emph{Schriftsteller, Mediziner}!Tagebuch@\strich\emph{Tagebuch}|pwk} sei er »entsetzt« gewesen. Schnitzler habe jedoch versprochen, Marie Reinhard\pwindex{Reinhard, Marie 13.\,3.\,1871 Wien – 18.\,3.\,1899 ebd.@\textsc{Reinhard, Marie} (13.\,3.\,1871 Wien – 18.\,3.\,1899 ebd.), \emph{Gesangspädagogin}|pwk} so bald wie möglich zu
                  heiraten.}}}\label{K_L02804-1} aufgenommen hat. Hoffentlich bleibts bei der Pariſ\oindex{Paris@\textbf{Paris}, \emph{Hauptstadt}|pw}er Reiſe. Ich habe mich mit dem Gedanken, Dich einige
               Wochen hier zu haben, bereits{ }ſo vertraut gemacht, daß es mir recht{ }ſchmerzlich wäre,
               darauf zu verzichten. Daß das Mädel\pwindex{Reinhard, Marie 13.\,3.\,1871 Wien – 18.\,3.\,1899 ebd.@\textsc{Reinhard, Marie} (13.\,3.\,1871 Wien – 18.\,3.\,1899 ebd.), \emph{Gesangspädagogin}|pwv}{ }ſich{ }ſo brav benimmt, freut mich{ }ſehr; übrigens überraſcht mich nichts
               Günſtiges, \strikeout{d} was ich von einer jungen Dame\pwindex{Reinhard, Marie 13.\,3.\,1871 Wien – 18.\,3.\,1899 ebd.@\textsc{Reinhard, Marie} (13.\,3.\,1871 Wien – 18.\,3.\,1899 ebd.), \emph{Gesangspädagogin}|pwv} höre, welche zwei Jahre
               lang Dich geliebt hat und von Dir geliebt worden iſt. {\pb}Ich wünſchte nur, Du wäreſt aus allen dieſen
               Aufregungen{ }ſchon heraus.\pend
           
\pstart
           Ein comfortables und ruhiges \textsc{Hotel} wird natürlich hier \strikeout{raſch} raſch gefunden{ }ſein. Du brauchſt mir nur die
               ungefähre \strikeout{Pres} Preislage mitzutheilen und anzugeben,
               ob Du im Centrum der Stadt\oindex{Paris@\textbf{Paris}, \emph{Hauptstadt}|pwv}
               wohnen willſt. Jedenfalls möchte ich, daß Du den \textsc{Hotel}-Aufenthalt möglichſt abkürzeſt; die Pariſ\oindex{Paris@\textbf{Paris}, \emph{Hauptstadt}|pw}er Hotels{ }ſind ungemüthlich, und{ }ſelbſt die comfortablen mangeln des
               Comforts. Die Art, wie Du wohnen willſt, mußt Du Dir aber dann hier{ }ſelbſt ausſuchen.
               Ich werde Dir einige Vorſchläge machen, wage aber nicht, für Dich {\pb}eine Wohnung aufzunehmen. Die Idee der Penſion bei
               einer gut bürgerlichen Familie iſt undurchführbar. Die gut bürgerlichen fran\oindex{Frankreich@\textbf{Frankreich}|pwv}zöſiſchen Familien geben
               keine Penſion. Die Fremden gehen hier in die \textsc{Hotels} mit
               Penſion, die im Style der engl\oindex{England@\textbf{England}, \emph{Land}|pw}iſchen{ }\begin{otherlanguage}{english}\textsc{boarding-houses}\end{otherlanguage}{ }ſind. Das möchte ich aber auch nicht rathen, wegen
               des Schlangenfraßes. Das Beſte wäre, daß Du{ }ſowohl wie Deine Freundin\pwindex{Reinhard, Marie 13.\,3.\,1871 Wien – 18.\,3.\,1899 ebd.@\textsc{Reinhard, Marie} (13.\,3.\,1871 Wien – 18.\,3.\,1899 ebd.), \emph{Gesangspädagogin}|pwv} je eine kleine möblirte Wohnung in
               einer der{ }ſtillen Seitenſtraßen der \textsc{Champs Élysées\oindex{avenue des Champs-Élysées@\textbf{avenue des Champs-Élysées}, \emph{Straße}|pw}} nähmet. Eſſen im Reſtaurant. \strikeout{\textcolor{gray}{×}} Mittag vielleicht zu Haufe. {\pb}So{ }ſeid Ihr
               ungeſtört. Die junge Dame\pwindex{Reinhard, Marie 13.\,3.\,1871 Wien – 18.\,3.\,1899 ebd.@\textsc{Reinhard, Marie} (13.\,3.\,1871 Wien – 18.\,3.\,1899 ebd.), \emph{Gesangspädagogin}|pwv}
               wird allerdings{ }ſehr allein{ }ſein, aber das liegt vielleicht in ihren Wünſchen. Preis
               einer{ }ſolchen Wohnung: 150 bis 200 \textsc{Francs} monatlich.\pend
           
\pstart
           \strikeout{Anf\textcolor{gray}{a}} Ende März bin ich jedenfalls hier. Es iſt noch
               ganz unbeſtimmt, ob ich überhaupt fortgehe.\pend
           
\pstart
           Schreib’ mir bald und{ }ſei von Herzen gegrüßt!\pend
           
\pstart
           Dein treuer {\\[\baselineskip]}\spacefill\mbox{Paul Goldmnn}\pend
           \leftskip=0em{}\selectlanguage{ngerman}\endnumbering\briefempfaengerindex{Schnitzler, Arthur@\textsc{Schnitzler, Arthur}!zzzGoldmann, Paul@\emph{von Paul Goldmann}!1897-02-241@{24. 2. [1897]}|)be}\mylabel{L02804h}  \newcommand{\dateiname}{L02804}\newcommand{\titel}{Paul Goldmann an Arthur Schnitzler, 24. 2. [1897]}\newcommand{\editorInnen}{Martin Anton Müller und Laura Untner}%% latex-leseansicht-abspann.tex
%% Abspann für die Leseansicht.
%% Der Schalter \ifkorrekturansicht ist bereits durch den Vorspann gesetzt.

%% latex-abspann.tex
%% Gemeinsamer Abspann für Korrekturansicht und Leseansicht.
%% Setzt den Schalter \ifkorrekturansicht voraus (gesetzt in den
%% einbindenden Dateien latex-korrekturansicht-abspann.tex bzw.
%% latex-leseansicht-abspann.tex).
%% ---------------------------------------------------------------

\normalsize

% Das esempio-Environment wird nur in der Leseansicht benötigt
\ifkorrekturansicht\else
\newenvironment{esempio}[3]%
{
    \vspace{1.5ex}
    \rlap{\underline{#1}}
    \par
    \setlength{\parindent}{0cm}
    \nopagebreak
    \leftskip=#2cm
    \rightskip=#3cm
}
{
    \par
}
\fi

\doendnotes{C}
\bigskip
\vfill

\clearpage

\footnotesize

\ifkorrekturansicht
  \lohead{\textsc{register}}
\fi

% theindex-Environment neu definieren ohne reledmac
\makeatletter
\renewenvironment{theindex}{%
  \ifkorrekturansicht
    \section*{\indexname}%
  \else
    \subsubsection*{Index der erwähnten Entitäten}%
  \fi
  \setlength{\parindent}{0pt}%
  \setlength{\parskip}{0pt plus 0.3pt}%
  \let\item\@idxitem
}{%
  \ifkorrekturansicht\clearpage\fi
}
\makeatother

\IfFileExists{\jobname-pw.ind}{\input{\jobname-pw.ind}}{}

% Quellenangabe nur in der Leseansicht
\ifkorrekturansicht\else
% Fallback-Definitionen, falls die .tex-Datei \titel etc. nicht gesetzt hat
\providecommand{\titel}{}
\providecommand{\editorInnen}{}
\providecommand{\dateiname}{\jobname}

\vspace{3cm}

\vfill

\footnotesize
\textsc{Quelle}: \titel. Herausgegeben von {\editorInnen}. In: \emph{Arthur Schnitzler: Briefwechsel mit Autorinnen und Autoren}.
 Digitale Edition, https://schnitzler-briefe.acdh.oeaw.ac.at/{\dateiname}.html (Stand \today)
\fi

\end{document}


