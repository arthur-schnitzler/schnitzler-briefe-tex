%% latex-leseansicht-vorspann.tex
%% Vorspann für die Leseansicht.
%% Lädt die gemeinsame Datei latex-vorspann.tex mit nicht gesetztem Schalter.

\newif\ifkorrekturansicht
\korrekturansichtfalse

\input{../tex-inputs/latex-vorspann}


         
         \renewcommand{\erwaehntePersonen}{Personen: Paul Goldmann, Carl Reinhard, Marie Reinhard, Leopold Sonnemann}
         \renewcommand{\erwaehnteInstitutionen}{Institutionen: Frankfurter Zeitung}
         \renewcommand{\erwaehnteOrte}{Orte: Champs-Élysées, England, Frankreich, Paris, Wien, rue Feydeau}
         \renewcommand{\erwaehnteWerke}{Werke: Tagebuch}
               \section[ Paul Goldmann an Arthur Schnitzler, 24. 2. {[}1897{]}]{ Paul Goldmann an Arthur Schnitzler, 24. 2. {[}1897{]}}\nopagebreak\mylabel{v}\rehead{ }\begin{ledgroupsized}[t]{13cm}\normalsize\beginnumbering\briefempfaengerindex{Schnitzler, Arthur@\textsc{Schnitzler, Arthur}!zzzGoldmann, Paul@\emph{von Paul Goldmann}!1897-02-241@{24. 2. {[}1897{]}}|(be} \toendnotes[C]{\smallbreak\pagebreak[2]} \Standort{DLA, A:Schnitzler, HS.NZ85.1.3167.}
\physDesc{Brief, 1 Blatt, 4 Seiten, 1941 Zeichen
\newline{}Handschrift: blaue Tinte, deutsche Kurrent
\newline{}Schnitzler: mit Bleistift das Jahr »97« vermerkt }\toendnotes[C]{\smallbreak}\pstart
           \noindent{}{\pb}\textcolor{gray}{\textbf{\textbf{Frankfurter Zeitung\orgindex{Frankfurter Zeitung@Frankfurter Zeitung|pw}}}}\pend
           \pstart
           \textcolor{gray}{\textbf{(\begin{otherlanguage}{french}Gazette de Francfort\end{otherlanguage}\orgindex{Frankfurter Zeitung@Frankfurter Zeitung|pw}).}}\pend
           \pstart
           \textcolor{gray}{\textbf{\textbf{\begin{otherlanguage}{french}Fondateur M.\end{otherlanguage}{ }L. Sonnemann\pwindex{Sonnemann, Leopold 1831-10-29 – 1909-10-30@\textsc{Sonnemann, Leopold} (1831-10-29 – 1909-10-30), \emph{Journalist, Herausgeber}|pw}.}}}\pend
           \pstart
           \begin{otherlanguage}{french}\textcolor{gray}{\textbf{Journal politique, financier,}}\end{otherlanguage}\hfill \textsc{Paris\oindex{Paris@\textbf{Paris}|pw}}, 24. Februar.\pend
           \pstart
           \begin{otherlanguage}{french}\textcolor{gray}{\textbf{commercial et littéraire.}}\end{otherlanguage}\pend
           \pstart
           \begin{otherlanguage}{french}\textcolor{gray}{\textbf{\textbf{Paraissant trois fois par jour.}}}\end{otherlanguage}\pend
           \pstart
           \begin{otherlanguage}{french}\textcolor{gray}{\textbf{\textbf{Bureau à Paris\oindex{Paris@\textbf{Paris}|pw}}}}\end{otherlanguage}\pend
           \pstart
           \begin{otherlanguage}{french}\textcolor{gray}{\textbf{\textbf{24. Rue Feydeau\oindex{rue Feydeau@\textbf{rue Feydeau}|pw}.}}}\end{otherlanguage}\pend
           \pstart\center{}Mein lieber Freund,\pend\pstart
           Du ſchreibſt mir wohl umgehend ein kurzes Wort über die Art, wie der Vater\pwindex{Reinhard, Carl 02.03.1834 – 28.04.1905@\textsc{Reinhard, Carl} (02.03.1834 – 28.04.1905), \emph{Geschäftsführer}|pwv}{ }\strikeout{die} die \label{K_L02804-1v}\edtext{Sache}{\lemma{\textnormal{\emph{Sache}}}\Cendnote{\textnormal{Carl Reinhard\pwindex{Reinhard, Carl 02.03.1834 – 28.04.1905@\textsc{Reinhard, Carl} (02.03.1834 – 28.04.1905), \emph{Geschäftsführer}|pwk} wurde am 23. 2. 1897 über Marie Reinhard\pwindex{Reinhard, Marie 1871-03-13 – 1899-03-18@\textsc{Reinhard, Marie} (1871-03-13 – 1899-03-18), \emph{Gesangspädagogin}|pwk}s Schwangerschaft informiert.
                  Laut Schnitzler\pwindex{Schnitzler, Arthur 15.05.1862 – 21.10.1931@\textsc{Schnitzler, Arthur} (15.05.1862 – 21.10.1931), \emph{Schriftsteller, Mediziner}|pwk}s \emph{Tagebuch}\pwindex{Schnitzler, Arthur 15.05.1862 – 21.10.1931@\textsc{Schnitzler, Arthur} (15.05.1862 – 21.10.1931), \emph{Schriftsteller, Mediziner}!Tagebuch1981 – 2000@\strich\emph{Tagebuch} {[}1981 – 2000{]}|pwk} sei er »entsetzt« gewesen. Schnitzler\pwindex{Schnitzler, Arthur 15.05.1862 – 21.10.1931@\textsc{Schnitzler, Arthur} (15.05.1862 – 21.10.1931), \emph{Schriftsteller, Mediziner}|pwk} habe jedoch versprochen, Marie Reinhard\pwindex{Reinhard, Marie 1871-03-13 – 1899-03-18@\textsc{Reinhard, Marie} (1871-03-13 – 1899-03-18), \emph{Gesangspädagogin}|pwk} so bald wie möglich zu
                  heiraten.}}}\label{K_L02804-1h} aufgenommen hat. Hoffentlich bleibts bei der Pariſ\oindex{Paris@\textbf{Paris}|pw}er Reiſe. Ich habe mich mit dem Gedanken, Dich einige
               Wochen hier zu haben, bereits ſo vertraut gemacht, daß es mir recht ſchmerzlich wäre,
               darauf zu verzichten. Daß das Mädel\pwindex{Reinhard, Marie 1871-03-13 – 1899-03-18@\textsc{Reinhard, Marie} (1871-03-13 – 1899-03-18), \emph{Gesangspädagogin}|pwv} ſich ſo brav benimmt, freut mich ſehr; übrigens überraſcht mich nichts
               Günſtiges, \strikeout{d} was ich von einer jungen Dame\pwindex{Reinhard, Marie 1871-03-13 – 1899-03-18@\textsc{Reinhard, Marie} (1871-03-13 – 1899-03-18), \emph{Gesangspädagogin}|pwv} höre, welche zwei Jahre
               lang Dich geliebt hat und von Dir geliebt worden iſt. {\pb}Ich wünſchte nur, Du wäreſt aus allen dieſen
               Aufregungen ſchon heraus.\pend
           \pstart
           Ein comfortables und ruhiges \textsc{Hotel} wird natürlich hier \strikeout{raſch} raſch gefunden ſein. Du brauchſt mir nur die
               ungefähre \strikeout{Pres} Preislage mitzutheilen und anzugeben,
               ob Du im Centrum der Stadt\oindex{Paris@\textbf{Paris}|pwv}
               wohnen willſt. Jedenfalls möchte ich, daß Du den \textsc{Hotel}-Aufenthalt möglichſt abkürzeſt; die Pariſ\oindex{Paris@\textbf{Paris}|pw}er Hotels ſind ungemüthlich, und ſelbſt die comfortablen mangeln des
               Comforts. Die Art, wie Du wohnen willſt, mußt Du Dir aber dann hier ſelbſt ausſuchen.
               Ich werde Dir einige Vorſchläge machen, wage aber nicht, für Dich {\pb}eine Wohnung aufzunehmen. Die Idee der Penſion bei
               einer gut bürgerlichen Familie iſt undurchführbar. Die gut bürgerlichen fran\oindex{Frankreich@\textbf{Frankreich}|pwv}zöſiſchen Familien geben
               keine Penſion. Die Fremden gehen hier in die \textsc{Hotels} mit
               Penſion, die im Style der engl\oindex{England@\textbf{England}|pw}iſchen{ }\begin{otherlanguage}{english}\textsc{boarding-houses}\end{otherlanguage}{ }ſind. Das möchte ich aber auch nicht rathen, wegen
               des Schlangenfraßes. Das Beſte wäre, daß Du ſowohl wie Deine Freundin\pwindex{Reinhard, Marie 1871-03-13 – 1899-03-18@\textsc{Reinhard, Marie} (1871-03-13 – 1899-03-18), \emph{Gesangspädagogin}|pwv} je eine kleine möblirte Wohnung in
               einer der ſtillen Seitenſtraßen der \textsc{Champs Élysées\oindex{Champs-Elysees@\textbf{Champs-Élysées}|pw}} nähmet. Eſſen im Reſtaurant. \strikeout{\textcolor{gray}{×}} Mittag vielleicht zu Haufe. {\pb}So ſeid Ihr
               ungeſtört. Die junge Dame\pwindex{Reinhard, Marie 1871-03-13 – 1899-03-18@\textsc{Reinhard, Marie} (1871-03-13 – 1899-03-18), \emph{Gesangspädagogin}|pwv}
               wird allerdings ſehr allein ſein, aber das liegt vielleicht in ihren Wünſchen. Preis
               einer ſolchen Wohnung: 150 bis 200 \textsc{Francs} monatlich.\pend
           \pstart
           \strikeout{Anf\textcolor{gray}{a}} Ende März bin ich jedenfalls hier. Es iſt noch
               ganz unbeſtimmt, ob ich überhaupt fortgehe.\pend
           \pstart
           Schreib’ mir bald und ſei von Herzen gegrüßt!\pend
           \pstart
           Dein treuer {\\[\baselineskip]}\spacefill\mbox{Paul Goldmnn}\pend
           \leftskip=0em{}
         
         \endnumbering\mylabel{h}\end{ledgroupsized}  \newcommand{\dateiname}{L02804}\newcommand{\titel}{Paul Goldmann an Arthur Schnitzler, 24. 2. [1897]}\newcommand{\editorInnen}{Martin Anton Müller und Laura Untner}%% latex-leseansicht-abspann.tex
%% Abspann für die Leseansicht.
%% Der Schalter \ifkorrekturansicht ist bereits durch den Vorspann gesetzt.

%% latex-abspann.tex
%% Gemeinsamer Abspann für Korrekturansicht und Leseansicht.
%% Setzt den Schalter \ifkorrekturansicht voraus (gesetzt in den
%% einbindenden Dateien latex-korrekturansicht-abspann.tex bzw.
%% latex-leseansicht-abspann.tex).
%% ---------------------------------------------------------------

\normalsize

% Das esempio-Environment wird nur in der Leseansicht benötigt
\ifkorrekturansicht\else
\newenvironment{esempio}[3]%
{
    \vspace{1.5ex}
    \rlap{\underline{#1}}
    \par
    \setlength{\parindent}{0cm}
    \nopagebreak
    \leftskip=#2cm
    \rightskip=#3cm
}
{
    \par
}
\fi

\doendnotes{C}
\bigskip
\vfill

\clearpage

\footnotesize

\ifkorrekturansicht
  \lohead{\textsc{register}}
\fi

% theindex-Environment neu definieren ohne reledmac
\makeatletter
\renewenvironment{theindex}{%
  \ifkorrekturansicht
    \section*{\indexname}%
  \else
    \subsubsection*{Index der erwähnten Entitäten}%
  \fi
  \setlength{\parindent}{0pt}%
  \setlength{\parskip}{0pt plus 0.3pt}%
  \let\item\@idxitem
}{%
  \ifkorrekturansicht\clearpage\fi
}
\makeatother

\IfFileExists{\jobname-pw.ind}{\input{\jobname-pw.ind}}{}

% Quellenangabe nur in der Leseansicht
\ifkorrekturansicht\else
% Fallback-Definitionen, falls die .tex-Datei \titel etc. nicht gesetzt hat
\providecommand{\titel}{}
\providecommand{\editorInnen}{}
\providecommand{\dateiname}{\jobname}

\vspace{3cm}

\vfill

\footnotesize
\textsc{Quelle}: \titel. Herausgegeben von {\editorInnen}. In: \emph{Arthur Schnitzler: Briefwechsel mit Autorinnen und Autoren}.
 Digitale Edition, https://schnitzler-briefe.acdh.oeaw.ac.at/{\dateiname}.html (Stand \today)
\fi

\end{document}


      