%% latex-leseansicht-vorspann.tex
%% Vorspann für die Leseansicht.
%% Lädt die gemeinsame Datei latex-vorspann.tex mit nicht gesetztem Schalter.

\newif\ifkorrekturansicht
\korrekturansichtfalse

\input{../tex-inputs/latex-vorspann}


\section[Hugo von Hofmannsthal an Arthur Schnitzler, {[}3. 8. 1905?{]}]{L01536 Hugo von Hofmannsthal an Arthur Schnitzler, {[}3. 8. 1905?{]}}
\nopagebreak\mylabel{L01536v}
\rehead{ }\normalsize\beginnumbering\briefempfaengerindex{Schnitzler, Arthur@\textsc{Schnitzler, Arthur}!zzzHofmannsthal, Hugo von@\emph{von Hugo von Hofmannsthal}!1905-08-032@{{[}3. 8. 1905?{]}}|(be}
\toendnotes[C]{\smallbreak\pagebreak[2]}
\correspDesc{Versand  durch Hugo von Hofmannsthal am [3. 8. 1905?] in Rodaun
\newline{}Erhalt  durch Arthur Schnitzler am [3. 8. 1905?] in Wien}\toendnotes[C]{\smallbreak}
\Standort{CUL, Schnitzler, B 43.}
\physDesc{Telegramm, 89 Zeichen
\newline{}HandschriftX2 einer Schreibkraft: blaue Tinte, lateinische Kurrent
\newline{}Versand: Übermittlungszeile: »\noindent{}\textcolor{gray}{\textbf{Aufgabe-Nr.}} 27 \textcolor{gray}{\textbf{mit}} 19 \textcolor{gray}{\textbf{Taxworten (.......... Worten ..........Chiffern}}« und der Empfangszeit: »\textcolor{gray}{\textbf{um}}{ }10 \textcolor{gray}{\textbf{uhr}} 15 \textcolor{gray}{\textbf{Min.}} V\textcolor{gray}{\textbf{Mittag}}« 
\newline{}Schnitzler: mit Bleistift datiert »August 905« 
\newline{}Ordnung: 1) mit Bleistift von unbekannter Hand auf der Vorder- und Rückseite nummeriert:
                                    »218« respektive »258«  2) beschnitten}
\buchAbdrucke{\weitereDrucke{Hugo von Hofmannsthal, Arthur Schnitzler: \emph{Briefwechsel}. Herausgegeben von Therese Nickl und Heinrich Schnitzler. Frankfurt am Main: \emph{S. Fischer} 1964, S. 211.} }\toendnotes[C]{\smallbreak}
\pstart
           \noindent{}{\pb}Sind \label{K_L01536-1v}\edtext{zurück}{\lemma{\textnormal{\emph{zurück}}}\Cendnote{\textnormal{Am
                     3. 8. 1905 kehrte er von einer Waffenübung zurück, die vom
                     6. 7. 1905 bis zum 31. 7. 1905 gedauert
                  hatte. Das Telegramm ist undatiert und wird hier unter der Annahme eingeordnet,
                  dass es die Enttäuschung Hofmannsthals\pwindex{Hofmannsthal, Hugo von 1.\,2.\,1874 Wien – 15.\,7.\,1929 Rodaun@\textsc{Hofmannsthal, Hugo von} (1.\,2.\,1874 Wien – 15.\,7.\,1929 Rodaun), \emph{Schriftsteller}|pwk}
                  vorbereitet, nachdem Schnitzler eine nicht
                  erhaltene, verzögernde Antwort auf dieses Telegramm gegeben hat.}}}\label{K_L01536-1} bin sehr
               verlangend Sie sehen bitten euch für baldigsten Abend hier ansagen\pend
           \pstart \spacefill\mbox{Hugo}\pend{}\selectlanguage{ngerman}\endnumbering\briefempfaengerindex{Schnitzler, Arthur@\textsc{Schnitzler, Arthur}!zzzHofmannsthal, Hugo von@\emph{von Hugo von Hofmannsthal}!1905-08-032@{{[}3. 8. 1905?{]}}|)be}\mylabel{L01536h}  \newcommand{\dateiname}{L01536}\newcommand{\titel}{Hugo von Hofmannsthal an Arthur Schnitzler, [3. 8. 1905?]}\newcommand{\editorInnen}{Martin Anton Müller und Gerd-Hermann Susen}%% latex-leseansicht-abspann.tex
%% Abspann für die Leseansicht.
%% Der Schalter \ifkorrekturansicht ist bereits durch den Vorspann gesetzt.

%% latex-abspann.tex
%% Gemeinsamer Abspann für Korrekturansicht und Leseansicht.
%% Setzt den Schalter \ifkorrekturansicht voraus (gesetzt in den
%% einbindenden Dateien latex-korrekturansicht-abspann.tex bzw.
%% latex-leseansicht-abspann.tex).
%% ---------------------------------------------------------------

\normalsize

% Das esempio-Environment wird nur in der Leseansicht benötigt
\ifkorrekturansicht\else
\newenvironment{esempio}[3]%
{
    \vspace{1.5ex}
    \rlap{\underline{#1}}
    \par
    \setlength{\parindent}{0cm}
    \nopagebreak
    \leftskip=#2cm
    \rightskip=#3cm
}
{
    \par
}
\fi

\doendnotes{C}
\bigskip
\vfill

\clearpage

\footnotesize

\ifkorrekturansicht
  \lohead{\textsc{register}}
\fi

% theindex-Environment neu definieren ohne reledmac
\makeatletter
\renewenvironment{theindex}{%
  \ifkorrekturansicht
    \section*{\indexname}%
  \else
    \subsubsection*{Index der erwähnten Entitäten}%
  \fi
  \setlength{\parindent}{0pt}%
  \setlength{\parskip}{0pt plus 0.3pt}%
  \let\item\@idxitem
}{%
  \ifkorrekturansicht\clearpage\fi
}
\makeatother

\IfFileExists{\jobname-pw.ind}{\input{\jobname-pw.ind}}{}

% Quellenangabe nur in der Leseansicht
\ifkorrekturansicht\else
% Fallback-Definitionen, falls die .tex-Datei \titel etc. nicht gesetzt hat
\providecommand{\titel}{}
\providecommand{\editorInnen}{}
\providecommand{\dateiname}{\jobname}

\vspace{3cm}

\vfill

\footnotesize
\textsc{Quelle}: \titel. Herausgegeben von {\editorInnen}. In: \emph{Arthur Schnitzler: Briefwechsel mit Autorinnen und Autoren}.
 Digitale Edition, https://schnitzler-briefe.acdh.oeaw.ac.at/{\dateiname}.html (Stand \today)
\fi

\end{document}


