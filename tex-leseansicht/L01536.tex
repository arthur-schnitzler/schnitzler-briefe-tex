%% latex-korrekturansicht-vorspann.tex
%% Vorspann für die Korrekturansicht.
%% Lädt die gemeinsame Datei latex-vorspann.tex mit gesetztem Schalter.

\newif\ifkorrekturansicht
\korrekturansichttrue

\input{../tex-inputs/latex-vorspann}


\section[Hugo von Hofmannsthal an Arthur Schnitzler, {[}3. 8. 1905?{]}]{L01536 Hugo von Hofmannsthal an Arthur Schnitzler, {[}3. 8. 1905?{]}}
\nopagebreak\mylabel{L01536v}
\rehead{ }\normalsize\beginnumbering\briefempfaengerindex{Schnitzler, Arthur@\textsc{Schnitzler, Arthur}!zzzHofmannsthal, Hugo von@\emph{von Hugo von Hofmannsthal}!1905-08-032@{{[}3. 8. 1905?{]}}|(be}
\toendnotes[C]{\smallbreak\pagebreak[2]}\Standort{CUL, Schnitzler, B 43.}
\physDesc{Telegramm, 89 Zeichen
\newline{}Handschrift einer Schreibkraft: blaue Tinte, lateinische Kurrent
\newline{}Versand: Übermittlungszeile: »\noindent{}\textcolor{gray}{\textbf{Aufgabe-Nr.}} 27 \textcolor{gray}{\textbf{mit}} 19 \textcolor{gray}{\textbf{Taxworten (.......... Worten ..........Chiffern}}« und der Empfangszeit: »\textcolor{gray}{\textbf{um}}{ }10 \textcolor{gray}{\textbf{uhr}} 15 \textcolor{gray}{\textbf{Min.}} V\textcolor{gray}{\textbf{Mittag}}« 
\newline{}Schnitzler: mit Bleistift datiert »August 905« 
\newline{}Ordnung: 1) mit Bleistift von unbekannter Hand auf der Vorder- und Rückseite nummeriert:
                                    »218« respektive »258«  2) beschnitten}
\buchAbdrucke{\weitereDrucke{Hugo von Hofmannsthal, Arthur Schnitzler: \emph{Briefwechsel}. Frankfurt am Main: \emph{S. Fischer} 1964, S. 211.} }\toendnotes[C]{\smallbreak}
\pstart
           \noindent{}{\pb}Sind \label{K_L01536-1v}\edtext{zurück}{\lemma{\textnormal{\emph{zurück}}}\Cendnote{\textnormal{Am
                     3. 8. 1905 kehrte er von einer Waffenübung zurück, die vom
                     6. 7. 1905 bis zum 31. 7. 1905 gedauert
                  hatte. Das Telegramm ist undatiert und wird hier unter der Annahme eingeordnet,
                  dass es die Enttäuschung Hofmannsthals\pwindex{Hofmannsthal, Hugo von 1874-02-01 – 1929-07-15@\textsc{Hofmannsthal, Hugo von} (1874-02-01 – 1929-07-15), \emph{Schriftsteller/Schriftstellerin}|pwk}
                  vorbereitet, nachdem Schnitzler eine nicht
                  erhaltene, verzögernde Antwort auf dieses Telegramm gegeben hat.}}}\label{K_L01536-1} bin sehr
               verlangend Sie sehen bitten euch für baldigsten Abend hier ansagen\pend
           \pstart \spacefill\mbox{Hugo}\pend{}\selectlanguage{ngerman}\endnumbering\briefempfaengerindex{Schnitzler, Arthur@\textsc{Schnitzler, Arthur}!zzzHofmannsthal, Hugo von@\emph{von Hugo von Hofmannsthal}!1905-08-032@{{[}3. 8. 1905?{]}}|)be}\mylabel{L01536h}  \normalsize

\doendnotes{C}
\bigskip
\vfill

\clearpage

\footnotesize

\lohead{\textsc{register}}

% Definiere theindex-Environment komplett neu ohne reledmac
\makeatletter
\renewenvironment{theindex}{%
  \section*{\indexname}%
  \setlength{\parindent}{0pt}%
  \setlength{\parskip}{0pt plus 0.3pt}%
  \let\item\@idxitem
}{%
  \clearpage
}
\makeatother

\IfFileExists{\jobname-pw.ind}{\input{\jobname-pw.ind}}{}

\end{document}

      