%% latex-leseansicht-vorspann.tex
%% Vorspann für die Leseansicht.
%% Lädt die gemeinsame Datei latex-vorspann.tex mit nicht gesetztem Schalter.

\newif\ifkorrekturansicht
\korrekturansichtfalse

\input{../tex-inputs/latex-vorspann}


         
         \renewcommand{\erwaehntePersonen}{Personen: Robert Adam, Lili Schnitzler}
         \renewcommand{\erwaehnteOrte}{Orte: IX., Alsergrund, Meidlinger Hauptstraße, Sanatorium Loew, Sternwartestraße, Wien, XII., Meidling}
         \renewcommand{\erwaehnteWerke}{}
               \section[Arthur Schnitzler an Robert Adam, 30. 3. 1921]{ Arthur Schnitzler an Robert Adam, 30. 3. 1921}\nopagebreak\mylabel{v}\rehead{ }\begin{ledgroupsized}[t]{13cm}\normalsize\beginnumbering\briefempfaengerindex{Adam, Robert@\textsc{Adam, Robert}!zzzSchnitzler, Arthur@\emph{von Arthur Schnitzler}!1921-03-301@{30. 3. 1921}|(be} \toendnotes[C]{\smallbreak\pagebreak[2]} \Standort{DLA, 96.34.2/25.}
\physDesc{Postkarte, 739 Zeichen
\newline{}Handschrift: schwarze Tinte, lateinische Kurrent
\newline{}Versand: Stempel: »\nobreak{}\oindex{IX., Alsergrund@\textbf{IX., Alsergrund}|pwk}9/4 Wien 68, 30. III. 21, \textcolor{gray}{8}\nobreak{}«.  }\toendnotes[C]{\smallbreak}\pstart{}{\pb}XVIII. Sternwartestr 71\oindex{XXXX Ortsangabe fehlt|pw}\pend{}{\bigskip}\pstart{}Herrn Ob. Landesger. Rath\pend{}\pstart{}Dr. Robert Adam Pollak\pend{}\pstart{}Wien XII/\textsubscript{1}\oindex{XII., Meidling@\textbf{XII., Meidling}|pw}\pend{}\pstart{}Meidlinger Hptstr. 58\oindex{Meidlinger Hauptstrasse@\textbf{Meidlinger Hauptstraße}|pw}\pend{}{\bigskip}\pstart
           \raggedleft{}{\pb}30. 3. 1921\pend
           \pstart{}Verehrtester Herr Doctor\pend\pstart
           entschuldigen Sie, dſs ich Ihre liebe Karte so lange nicht beantwortet habe, – und
               daß ich Ihnen auch heute noch keinen besti{\geminationm}ten Tag
               nenne, an dem ich endlich wieder das Vergnügen zu haben hoffe Sie zu sehen; – diese
               letzten Wochen waren wie verhext, und für die nächsten Tage will ich mich noch nicht
               verpflichten, weil an meiner kleinen Tochter\pwindex{Schnitzler, Lili 13.09.1909 – 26.07.1928@\textsc{Schnitzler, Lili} (13.09.1909 – 26.07.1928)|pwv} eine kleine Operation (Rachenmandel) vorgeno{\geminationm}en werden soll, und ich im Sanatorium\oindex{Sanatorium Loew@\textbf{Sanatorium Loew}|pwv} bei ihr sein {\pb}werde. Ich denke daß ich Ende der \introOben{}nächsten\introOben{} Woche Ihnen zur Verfügung stehen kann. Bis dahin seien Sie aufs
               herzlichste gegrüßt von Ihrem sehr ergebnen\pend
           \pstart \spacefill\mbox{ArthurSchnitzler}\pend{}
         
         \endnumbering\mylabel{h}\end{ledgroupsized}  \newcommand{\dateiname}{L02365}\newcommand{\titel}{Arthur Schnitzler an Robert Adam, 30. 3. 1921}\newcommand{\editorInnen}{Martin Anton Müller und Gerd-Hermann Susen}%% latex-leseansicht-abspann.tex
%% Abspann für die Leseansicht.
%% Der Schalter \ifkorrekturansicht ist bereits durch den Vorspann gesetzt.

%% latex-abspann.tex
%% Gemeinsamer Abspann für Korrekturansicht und Leseansicht.
%% Setzt den Schalter \ifkorrekturansicht voraus (gesetzt in den
%% einbindenden Dateien latex-korrekturansicht-abspann.tex bzw.
%% latex-leseansicht-abspann.tex).
%% ---------------------------------------------------------------

\normalsize

% Das esempio-Environment wird nur in der Leseansicht benötigt
\ifkorrekturansicht\else
\newenvironment{esempio}[3]%
{
    \vspace{1.5ex}
    \rlap{\underline{#1}}
    \par
    \setlength{\parindent}{0cm}
    \nopagebreak
    \leftskip=#2cm
    \rightskip=#3cm
}
{
    \par
}
\fi

\doendnotes{C}
\bigskip
\vfill

\clearpage

\footnotesize

\ifkorrekturansicht
  \lohead{\textsc{register}}
\fi

% theindex-Environment neu definieren ohne reledmac
\makeatletter
\renewenvironment{theindex}{%
  \ifkorrekturansicht
    \section*{\indexname}%
  \else
    \subsubsection*{Index der erwähnten Entitäten}%
  \fi
  \setlength{\parindent}{0pt}%
  \setlength{\parskip}{0pt plus 0.3pt}%
  \let\item\@idxitem
}{%
  \ifkorrekturansicht\clearpage\fi
}
\makeatother

\IfFileExists{\jobname-pw.ind}{\input{\jobname-pw.ind}}{}

% Quellenangabe nur in der Leseansicht
\ifkorrekturansicht\else
% Fallback-Definitionen, falls die .tex-Datei \titel etc. nicht gesetzt hat
\providecommand{\titel}{}
\providecommand{\editorInnen}{}
\providecommand{\dateiname}{\jobname}

\vspace{3cm}

\vfill

\footnotesize
\textsc{Quelle}: \titel. Herausgegeben von {\editorInnen}. In: \emph{Arthur Schnitzler: Briefwechsel mit Autorinnen und Autoren}.
 Digitale Edition, https://schnitzler-briefe.acdh.oeaw.ac.at/{\dateiname}.html (Stand \today)
\fi

\end{document}


      