%% latex-korrekturansicht-vorspann.tex
%% Vorspann für die Korrekturansicht.
%% Lädt die gemeinsame Datei latex-vorspann.tex mit gesetztem Schalter.

\newif\ifkorrekturansicht
\korrekturansichttrue

\input{../tex-inputs/latex-vorspann}


\section[Arthur Schnitzler an Stefan Zweig, 19. 2. 1931]{L03738 Arthur Schnitzler an Stefan Zweig, 19. 2. 1931}
\nopagebreak\mylabel{L03738v}
\rehead{ }\normalsize\beginnumbering\briefempfaengerindex{Zweig, Stefan@\textsc{Zweig, Stefan}!zzzSchnitzler, Arthur@\emph{von Arthur Schnitzler}!1931-02-191@{19. 2. 1931}|(be}
\toendnotes[C]{\smallbreak\pagebreak[2]}\Standort{Jerusalem, National Library of Israel, ARC. Ms. Var. 305 1 58 Stefan Zweig Collection.}
\physDesc{Postkarte, 1 Blatt, 2 Seiten, 344 Zeichen
\newline{}Handschrift: schwarze Tinte, lateinische Kurrent
\newline{}Versand: Stempel: »\nobreak{}\oindex{I., Innere Stadt@\textbf{I., Innere Stadt}, \emph{A.ADM3}|pwk}1 Wien 1, 20. II. 1931, 12–13\nobreak{}«.  
\newline{}Ordnung: mit Bleistift Beschriftung: »\textsc{Arthur Schnitzler}« }\toendnotes[C]{\smallbreak}\pstart{}{\pb}\label{T_L03738-1v}\edtext{\textcolor{gray}{\textbf{A. S.}}}{\lemma{\textnormal{\emph{A. S.}}}\Cendnote{\textnormal{ovaler Absenderkleber}}}\label{T_L03738-1}\pend{}\pstart{}\textcolor{gray}{\textbf{WIEN, XVIII.}}\oindex{XVIII., Waehring@\textbf{XVIII., Währing}, \emph{A.ADM3}|pw}\pend{}\pstart{}\textcolor{gray}{\textbf{STERNWARTESTR. 71}}\oindex{Sternwartestrasse 71@\textbf{Sternwartestraße 71}, \emph{Wohngebäude (K.WHS)}|pw}\pend{}{\bigskip}\pstart{}France. Riviera\oindex{Riviera@\textbf{Riviera}, \emph{Strand (N.STR)}|pw}.\pend{}\pstart{}Hr Dr Stefan Zweig\pend{}\pstart{}Antibes\oindex{Antibes@\textbf{Antibes}, \emph{P.PPL}|pw}. \pend{}\pstart{}Hotel du Cap\oindex{Hotel du Cap@\textbf{Hotel du Cap}, \emph{Hotel (K.HTL)}|pw}\pend{}{\bigskip}\vspace{1em}
\pstart
           \raggedleft{}{\pb}Wien\oindex{Wien@\textbf{Wien}, \emph{A.ADM2}|pw}, 19. 1. 31\pend
           \vspace{0.5em}
\pstart
           lieber  Doctor Zweig, ich danke Ihren und Ihrer verehrten Gattin\pwindex{Zweig, Friderike Maria 1882-12-04 – 1971-01-18@\textsc{Zweig, Friderike Maria} (1882-12-04 – 1971-01-18), \emph{Schriftsteller/Schriftstellerin}|pwv} aufs herzlichste.
               Lassen Sie mich die Gelegenheit benutzen, um Ihnen auch für Ihr schönes neues Buch\pwindex{Heilung durch den Geist@\emph{Die Heilung durch den Geist}|pwv} (das ich natürlich
                  \label{K_L03738-1v}\edtext{partienweise}{\lemma{\textnormal{\emph{partienweise}}}\Cendnote{\textnormal{Von den drei \emph{Essays}\pwindex{Franz Anton Mesmer. Bildnis eines Vorausgaengers@\emph{Franz Anton Mesmer. Bildnis eines Vorausgängers}|pwk}\pwindex{Mary Baker-Eddy@\emph{Mary Baker-Eddy}|pwk}\pwindex{Sigmund Freud@\emph{Sigmund Freud}|pwk} erschienen die folgenden
                  Vorabdrucke: Stefan Zweig\pwindex{Zweig, Stefan 28.11.1881 – 23.02.1942@\textsc{Zweig, Stefan} (28.11.1881 – 23.02.1942), \emph{Schriftsteller/Schriftstellerin}|pwk}: \emph{Franz Anton Mesmer. Bildnis eines Vorausgängers}\pwindex{Franz Anton Mesmer. Bildnis eines Vorausgaengers@\emph{Franz Anton Mesmer. Bildnis eines Vorausgängers}|pwk}. In:
                        \emph{Neue Freie Presse}\pwindex{Neue Freie Presse@\emph{Neue Freie Presse}|pwk}, Nr. 23.646,
                        13. 7. 1930, Morgenblatt, Beilage, S. 25; Nr. 23.648,
                        15. 7. 1930, Morgenblatt, S. 1–3; Nr. 23.650,
                        17. 7. 1930, Morgenblatt, S. 1–3; Nr. 23.653,
                        20. 7. 1930, Morgenblatt, Beilage, S. 25–26; Nr. 23.660,
                        27. 7. 1930, Morgenblatt, Beilage, S. 25–26; Nr. 23.664,
                        31. 7. 1930, Morgenblatt, S. 1–3; Nr. 23.667,
                        3. 8. 1930, Morgenblatt, Beilage, S. 25–26; Nr. 23.671,
                        7. 8. 1930, Morgenblatt, S. 1–4; Nr. 23.674,
                        10. 8. 1930, Morgenblatt, Beilage, S. 25–27; Nr. 23.681,
                        17. 8. 1930, Morgenblatt, Beilage, S. 21–22. Stefan Zweig\pwindex{Zweig, Stefan 28.11.1881 – 23.02.1942@\textsc{Zweig, Stefan} (28.11.1881 – 23.02.1942), \emph{Schriftsteller/Schriftstellerin}|pwk}: \emph{Franz Anton Mesmer}\pwindex{Franz Anton Mesmer. Bildnis eines Vorausgaengers@\emph{Franz Anton Mesmer. Bildnis eines Vorausgängers}|pwk}. In: \emph{Insel-Almanach auf das Jahr 1931}\pwindex{Insel-Almanach auf das Jahr 1931@\emph{Insel-Almanach auf das Jahr 1931}|pwk}, S. 11–18.
                        Stefan Zweig\pwindex{Zweig, Stefan 28.11.1881 – 23.02.1942@\textsc{Zweig, Stefan} (28.11.1881 – 23.02.1942), \emph{Schriftsteller/Schriftstellerin}|pwk}: \emph{Das Leben und die Lehre der Mary Baker-Eddy}\pwindex{Leben und die Lehre der Mary Baker-Eddy@\emph{Das Leben und die Lehre der Mary Baker-Eddy}|pwk}. In: \emph{Pester Lloyd}\pwindex{Pester Lloyd@\emph{Pester Lloyd}|pwk}, Jg. 77, Nr. 157,
                        13. 7. 1930, Morgenblatt, S. 1–4.
                        Stefan Zweig\pwindex{Zweig, Stefan 28.11.1881 – 23.02.1942@\textsc{Zweig, Stefan} (28.11.1881 – 23.02.1942), \emph{Schriftsteller/Schriftstellerin}|pwk}: \emph{Bildnis Sigmund Freuds}\pwindex{Bildnis Sigmund Freuds@\emph{Bildnis Sigmund Freuds}|pwk}. In: \emph{Almanach der Psychoanalyse 1931}\pwindex{Almanach der Psychoanalyse 1931@\emph{Almanach der Psychoanalyse 1931}|pwk}. Hg. A. J. Storfer\pwindex{Storfer, Adolf J. 1888-01-11 – 1944-12-02@\textsc{Storfer, Adolf J.} (1888-01-11 – 1944-12-02), \emph{Schriftsteller/Schriftstellerin, Journalist/Journalistin, Verleger/Verlegerin}|pwk}. Wien: \emph{Internationaler Psychoanalytischer
                        Verlag}\orgindex{Internationaler Psychoanalytischer Verlag@Internationaler Psychoanalytischer Verlag|pwk}{ }1931,
                     S. 9–15.}}}\label{K_L03738-1} schon kannnte, meinen Dank zu sagen. Auf Wiedersehen
               und viele Grüße\pend
           
\pstart
           Ihr{\\[\baselineskip]}\spacefill\mbox{ArthurSchnitzl}\pend
           \leftskip=0em{}\selectlanguage{ngerman}\endnumbering\briefempfaengerindex{Zweig, Stefan@\textsc{Zweig, Stefan}!zzzSchnitzler, Arthur@\emph{von Arthur Schnitzler}!1931-02-191@{19. 2. 1931}|)be}\mylabel{L03738h}
\begin{anhang}
\end{anhang}\normalsize

\doendnotes{C}
\bigskip
\vfill

\clearpage

\footnotesize

\lohead{\textsc{register}}

% Definiere theindex-Environment komplett neu ohne reledmac
\makeatletter
\renewenvironment{theindex}{%
  \section*{\indexname}%
  \setlength{\parindent}{0pt}%
  \setlength{\parskip}{0pt plus 0.3pt}%
  \let\item\@idxitem
}{%
  \clearpage
}
\makeatother

\IfFileExists{\jobname-pw.ind}{\input{\jobname-pw.ind}}{}

\end{document}

      