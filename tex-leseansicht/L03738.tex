%% latex-leseansicht-vorspann.tex
%% Vorspann für die Leseansicht.
%% Lädt die gemeinsame Datei latex-vorspann.tex mit nicht gesetztem Schalter.

\newif\ifkorrekturansicht
\korrekturansichtfalse

\input{../tex-inputs/latex-vorspann}


\section[Arthur Schnitzler an Stefan Zweig, 19. 2. 1931]{L03738 Arthur Schnitzler an Stefan Zweig, 19. 2. 1931}
\nopagebreak\mylabel{L03738v}
\rehead{ }\normalsize\beginnumbering\briefempfaengerindex{Zweig, Stefan@\textsc{Zweig, Stefan}!zzzSchnitzler, Arthur@\emph{von Arthur Schnitzler}!1931-02-191@{19. 2. 1931}|(be}
\toendnotes[C]{\smallbreak\pagebreak[2]}
\correspDesc{Versand  durch Arthur Schnitzler am 19. 2. 1931 in Wien
\newline{}Übermittlung  am 20. 2. 1931 in Wien
\newline{}Erhalt  durch Stefan Zweig im Zeitraum [22. 2. 1931
                  – 28. 2. 1931?] in Antibes}\toendnotes[C]{\smallbreak}
\Standort{Jerusalem, National Library of Israel, ARC. Ms. Var. 305 1 58 Stefan Zweig Collection.}
\physDesc{Postkarte, 344 Zeichen
\newline{}Handschrift: schwarze Tinte, lateinische Kurrent
\newline{}Versand: Stempel: »\nobreak{}\oindex{I., Innere Stadt@\textbf{I., Innere Stadt}, \emph{Verwaltungsgebiet}|pwk}1 Wien 1, 20. II. 1931, 12–13\nobreak{}«.  
\newline{}Ordnung: mit Bleistift Beschriftung: »\textsc{Arthur Schnitzler}« }\toendnotes[C]{\smallbreak}\pstart{}{\pb}\label{T_L03738-1v}\edtext{\textcolor{gray}{\textbf{A. S.}}}{\lemma{\textnormal{\emph{A. S.}}}\Cendnote{\textnormal{ovaler Absenderkleber}}}\label{T_L03738-1}\pend{}\pstart{}\textcolor{gray}{\textbf{WIEN, XVIII.}}\oindex{XVIII., Währing@\textbf{XVIII., Währing}, \emph{Verwaltungsgebiet}|pw}\pend{}\pstart{}\textcolor{gray}{\textbf{STERNWARTESTR. 71}}\oindex{Wien@\textbf{Wien}!XVIII., Währing@\textbf{XVIII., Währing}!Sternwartestraße 71@\textbf{Sternwartestraße 71}, \emph{Wohngebäude}|pw}\pend{}{\bigskip}\pstart{}France, Riviera\oindex{Riviera@\textbf{Riviera}|pw}.\pend{}\pstart{}Hr Dr Stefan Zweig\pend{}\pstart{}Antibes\oindex{Antibes@\textbf{Antibes}|pw}. \pend{}\pstart{}Hotel du Cap\oindex{Hotel du Cap@\textbf{Hotel du Cap}, \emph{Hotel}|pw}\pend{}{\bigskip}\vspace{1em}
\pstart
           \raggedleft{}{\pb}Wien\oindex{Wien@\textbf{Wien}, \emph{Verwaltungsgebiet}|pw}, 19. 2. 31\pend
           \vspace{0.5em}
\pstart
           lieber  Doctor Zweig, ich danke Ihren und Ihrer verehrten Gattin\pwindex{Zweig, Friderike Maria 4.\,12.\,1882 Wien – 18.\,1.\,1971 Stamford@\textsc{Zweig, Friderike Maria} (4.\,12.\,1882 Wien – 18.\,1.\,1971 Stamford), \emph{Schriftstellerin}|pwv} aufs herzlichste.
               Lassen Sie mich die Gelegenheit benutzen, um Ihnen auch für Ihr schönes neues Buch\pwindex{Zweig, Stefan 28.\,11.\,1881 Wien – 23.\,2.\,1942 Petrópolis@\textsc{Zweig, Stefan} (28.\,11.\,1881 Wien – 23.\,2.\,1942 Petrópolis), \emph{Schriftsteller}!Heilung durch den Geist@\strich\emph{Die Heilung durch den Geist}|pwv} (das ich natürlich
                  \label{K_L03738-1v}\edtext{partienweise}{\lemma{\textnormal{\emph{partienweise}}}\Cendnote{\textnormal{Von den drei \emph{Essays}\pwindex{Zweig, Stefan 28.\,11.\,1881 Wien – 23.\,2.\,1942 Petrópolis@\textsc{Zweig, Stefan} (28.\,11.\,1881 Wien – 23.\,2.\,1942 Petrópolis), \emph{Schriftsteller}!Franz Anton Mesmer. Bildnis eines Vorausgängers@\strich\emph{Franz Anton Mesmer. Bildnis eines Vorausgängers}|pwk}\pwindex{Zweig, Stefan 28.\,11.\,1881 Wien – 23.\,2.\,1942 Petrópolis@\textsc{Zweig, Stefan} (28.\,11.\,1881 Wien – 23.\,2.\,1942 Petrópolis), \emph{Schriftsteller}!Mary Baker-Eddy@\strich\emph{Mary Baker-Eddy}|pwk}\pwindex{Zweig, Stefan 28.\,11.\,1881 Wien – 23.\,2.\,1942 Petrópolis@\textsc{Zweig, Stefan} (28.\,11.\,1881 Wien – 23.\,2.\,1942 Petrópolis), \emph{Schriftsteller}!Sigmund Freud@\strich\emph{Sigmund Freud}|pwk} erschienen die folgenden
                  Vorabdrucke: Stefan Zweig\pwindex{Zweig, Stefan 28.\,11.\,1881 Wien – 23.\,2.\,1942 Petrópolis@\textsc{Zweig, Stefan} (28.\,11.\,1881 Wien – 23.\,2.\,1942 Petrópolis), \emph{Schriftsteller}|pwk}: \emph{Franz Anton Mesmer. Bildnis eines Vorausgängers}\pwindex{Zweig, Stefan 28.\,11.\,1881 Wien – 23.\,2.\,1942 Petrópolis@\textsc{Zweig, Stefan} (28.\,11.\,1881 Wien – 23.\,2.\,1942 Petrópolis), \emph{Schriftsteller}!Franz Anton Mesmer. Bildnis eines Vorausgängers@\strich\emph{Franz Anton Mesmer. Bildnis eines Vorausgängers}|pwk}. In:
                        \emph{Neue Freie Presse}\pwindex{Neue Freie Presse@\emph{Neue Freie Presse}|pwk}, Nr. 23.646,
                        13. 7. 1930, Morgenblatt, Beilage, S. 25; Nr. 23.648,
                        15. 7. 1930, Morgenblatt, S. 1–3; Nr. 23.650,
                        17. 7. 1930, Morgenblatt, S. 1–3; Nr. 23.653,
                        20. 7. 1930, Morgenblatt, Beilage, S. 25–26; Nr. 23.660,
                        27. 7. 1930, Morgenblatt, Beilage, S. 25–26; Nr. 23.664,
                        31. 7. 1930, Morgenblatt, S. 1–3; Nr. 23.667,
                        3. 8. 1930, Morgenblatt, Beilage, S. 25–26; Nr. 23.671,
                        7. 8. 1930, Morgenblatt, S. 1–4; Nr. 23.674,
                        10. 8. 1930, Morgenblatt, Beilage, S. 25–27; Nr. 23.681,
                        17. 8. 1930, Morgenblatt, Beilage, S. 21–22. Stefan Zweig\pwindex{Zweig, Stefan 28.\,11.\,1881 Wien – 23.\,2.\,1942 Petrópolis@\textsc{Zweig, Stefan} (28.\,11.\,1881 Wien – 23.\,2.\,1942 Petrópolis), \emph{Schriftsteller}|pwk}: \emph{Franz Anton Mesmer}\pwindex{Zweig, Stefan 28.\,11.\,1881 Wien – 23.\,2.\,1942 Petrópolis@\textsc{Zweig, Stefan} (28.\,11.\,1881 Wien – 23.\,2.\,1942 Petrópolis), \emph{Schriftsteller}!Franz Anton Mesmer. Bildnis eines Vorausgängers@\strich\emph{Franz Anton Mesmer. Bildnis eines Vorausgängers}|pwk}. In: \emph{Insel-Almanach auf das Jahr 1931}\pwindex{Insel-Almanach auf das Jahr 1931@\emph{Insel-Almanach auf das Jahr 1931}|pwk}, S. 11–18. Stefan Zweig\pwindex{Zweig, Stefan 28.\,11.\,1881 Wien – 23.\,2.\,1942 Petrópolis@\textsc{Zweig, Stefan} (28.\,11.\,1881 Wien – 23.\,2.\,1942 Petrópolis), \emph{Schriftsteller}|pwk}: \emph{Das Leben und die Lehre der Mary Baker-Eddy}\pwindex{Zweig, Stefan 28.\,11.\,1881 Wien – 23.\,2.\,1942 Petrópolis@\textsc{Zweig, Stefan} (28.\,11.\,1881 Wien – 23.\,2.\,1942 Petrópolis), \emph{Schriftsteller}!Leben und die Lehre der Mary Baker-Eddy@\strich\emph{Das Leben und die Lehre der Mary Baker-Eddy}|pwk}. In: \emph{Pester Lloyd}\pwindex{Pester Lloyd@\emph{Pester Lloyd}|pwk}, Jg. 77, Nr. 157,
                        13. 7. 1930, Morgenblatt, S. 1–4. Stefan Zweig\pwindex{Zweig, Stefan 28.\,11.\,1881 Wien – 23.\,2.\,1942 Petrópolis@\textsc{Zweig, Stefan} (28.\,11.\,1881 Wien – 23.\,2.\,1942 Petrópolis), \emph{Schriftsteller}|pwk}: \emph{Bildnis Sigmund Freuds}\pwindex{Zweig, Stefan 28.\,11.\,1881 Wien – 23.\,2.\,1942 Petrópolis@\textsc{Zweig, Stefan} (28.\,11.\,1881 Wien – 23.\,2.\,1942 Petrópolis), \emph{Schriftsteller}!Bildnis Sigmund Freuds@\strich\emph{Bildnis Sigmund Freuds}|pwk}. In: \emph{Almanach der Psychoanalyse 1931}\pwindex{Almanach der Psychoanalyse 1931@\emph{Almanach der Psychoanalyse 1931}|pwk}. Hg. A. J. Storfer\pwindex{Storfer, Adolf J. 11.\,1.\,1888 Botoșani – 2.\,12.\,1944@\textsc{Storfer, Adolf J.} (11.\,1.\,1888 Botoșani – 2.\,12.\,1944), \emph{Schriftsteller, Journalist, Verleger}|pwk}. Wien: \emph{Internationaler Psychoanalytischer Verlag}\orgindex{Internationaler Psychoanalytischer Verlag@Internationaler Psychoanalytischer Verlag|pwk}{ }1931, S. 9–15.}}}\label{K_L03738-1} schon kannnte, meinen Dank zu sagen.
               Auf \label{K_L03738-5v}\edtext{Wiedersehen}{\lemma{\textnormal{\emph{Wiedersehen}}}\Cendnote{\textnormal{Das \emph{Tagebuch}\pwindex{Schnitzler, Arthur 15.\,5.\,1862 Wien – 21.\,10.\,1931 ebd.@\textsc{Schnitzler, Arthur} (15.\,5.\,1862 Wien – 21.\,10.\,1931 ebd.), \emph{Schriftsteller, Mediziner}!Tagebuch@\strich\emph{Tagebuch}|pwk} nennt keine
               weiteren persönliche Treffen.}}}\label{K_L03738-5} und viele Grüße\pend
           
\pstart
           Ihr{\\[\baselineskip]}\spacefill\mbox{ArthurSchnitzler}\pend
           \leftskip=0em{}\selectlanguage{ngerman}\endnumbering\briefempfaengerindex{Zweig, Stefan@\textsc{Zweig, Stefan}!zzzSchnitzler, Arthur@\emph{von Arthur Schnitzler}!1931-02-191@{19. 2. 1931}|)be}\mylabel{L03738h}  \newcommand{\dateiname}{L03738}\newcommand{\titel}{Arthur Schnitzler an Stefan Zweig, 19. 2. 1931}\newcommand{\editorInnen}{Selma Jahnke und Martin Anton Müller}%% latex-leseansicht-abspann.tex
%% Abspann für die Leseansicht.
%% Der Schalter \ifkorrekturansicht ist bereits durch den Vorspann gesetzt.

%% latex-abspann.tex
%% Gemeinsamer Abspann für Korrekturansicht und Leseansicht.
%% Setzt den Schalter \ifkorrekturansicht voraus (gesetzt in den
%% einbindenden Dateien latex-korrekturansicht-abspann.tex bzw.
%% latex-leseansicht-abspann.tex).
%% ---------------------------------------------------------------

\normalsize

% Das esempio-Environment wird nur in der Leseansicht benötigt
\ifkorrekturansicht\else
\newenvironment{esempio}[3]%
{
    \vspace{1.5ex}
    \rlap{\underline{#1}}
    \par
    \setlength{\parindent}{0cm}
    \nopagebreak
    \leftskip=#2cm
    \rightskip=#3cm
}
{
    \par
}
\fi

\doendnotes{C}
\bigskip
\vfill

\clearpage

\footnotesize

\ifkorrekturansicht
  \lohead{\textsc{register}}
\fi

% theindex-Environment neu definieren ohne reledmac
\makeatletter
\renewenvironment{theindex}{%
  \ifkorrekturansicht
    \section*{\indexname}%
  \else
    \subsubsection*{Index der erwähnten Entitäten}%
  \fi
  \setlength{\parindent}{0pt}%
  \setlength{\parskip}{0pt plus 0.3pt}%
  \let\item\@idxitem
}{%
  \ifkorrekturansicht\clearpage\fi
}
\makeatother

\IfFileExists{\jobname-pw.ind}{\input{\jobname-pw.ind}}{}

% Quellenangabe nur in der Leseansicht
\ifkorrekturansicht\else
% Fallback-Definitionen, falls die .tex-Datei \titel etc. nicht gesetzt hat
\providecommand{\titel}{}
\providecommand{\editorInnen}{}
\providecommand{\dateiname}{\jobname}

\vspace{3cm}

\vfill

\footnotesize
\textsc{Quelle}: \titel. Herausgegeben von {\editorInnen}. In: \emph{Arthur Schnitzler: Briefwechsel mit Autorinnen und Autoren}.
 Digitale Edition, https://schnitzler-briefe.acdh.oeaw.ac.at/{\dateiname}.html (Stand \today)
\fi

\end{document}


