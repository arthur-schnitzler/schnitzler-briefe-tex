%% latex-leseansicht-vorspann.tex
%% Vorspann für die Leseansicht.
%% Lädt die gemeinsame Datei latex-vorspann.tex mit nicht gesetztem Schalter.

\newif\ifkorrekturansicht
\korrekturansichtfalse

\input{../tex-inputs/latex-vorspann}

\begin{center}
            \textcolor{red}{ENTWURF, NICHT FERTIG KORRIGIERT}
                      \end{center}
            
         
         \newcommand{\erwaehntePersonen}{Personen: Richard Beer-Hofmann, Rudolf Lothar, Marie Reinhard}
         \newcommand{\erwaehnteInstitutionen}{}
         \newcommand{\erwaehnteOrte}{Orte: Asien, Astor House Hotel [Tianjin], Berlin, China, Deutsches Theater Berlin, Salzburg, Tianjin, Wien}
         \newcommand{\erwaehnteWerke}{Werke: Beim Tao-tai Tsai von Shanghai, Briefe an eine Dame, Chinesische Zeitungen, Das Vermächtnis. Schauspiel in drei Akten, Der General Tscheng-Ki-tong, Der Tod Georgs, Der Tod Georgs. Fragment, Der französisch-chinesische Zwischenfall in Shanghai, Der französisch-chinesische Zwischenfall in Shanghai, Die Pest in Hongkong, Die Wage. Eine Wiener Wochenschrift, Die Zeit. Wiener Wochenschrift, Die deutschen Militär-Instruktoren in China, Ein Kapitel über chinesische Eisenbahnen. Chinesische Eisenbahnen und deutsche Versäumnisse, Ein Kapitel über chinesische Eisenbahnen. Die Bahn in Shanghai nach Wu-sung, Eine Unterredung mit dem Tao-tai Wang, dem Sekretär des Vicekönigs von Canton, Eine Unterredung mit dem Tao-tai Wang, dem Sekretär des Vicekönigs von Canton [zweiter Teil], Frankfurter Zeitung, In Kiautschou. I, In Kiautschou. II, In Kiautschou. III, In Kiautschou. IV [letzter Teil], In Ostasien. Reiseskizzen. Auf dem Perlfluß nach Canton-Shameen, In Ostasien. Reiseskizzen. Auf dem Perlfluß nach Canton-Shameen [zweiter Teil], In Ostasien. Reiseskizzen. Auf dem Yang-tse-Kiang, In Ostasien. Reiseskizzen. Auf dem Yang-tse-Kiang [zweiter Teil], In Ostasien. Reiseskizzen. Canton!, In Ostasien. Reiseskizzen. Canton! [zweiter Teil], In Ostasien. Reiseskizzen. Chinesisches Nachtleben, In Ostasien. Reiseskizzen. Chinesisches Nachtleben [zweiter Teil], In Ostasien. Reiseskizzen. Hankow, In Ostasien. Reiseskizzen. Hongkong, In Ostasien. Reiseskizzen. Hongkong [zweiter Teil], In Ostasien. Reiseskizzen. Im Golf von Pe-tschi-li, In Ostasien. Reiseskizzen. Im Golf von Pe-tschi-li [zweiter Teil], In Ostasien. Reiseskizzen. Shanghai, In Ostasien. Reiseskizzen. Shanghai [zweiter Teil], In Ostasien. Reiseskizzen. Singapore, In Ostasien. Reiseskizzen. Von Hongkong nach Shanghai, In Ostasien. Reiseskizzen. Von Hongkong nach Shanghai [zweiter Teil], In Ostasien. Reiseskizzen. Von Tschifu nach Tientsin, In Ostasien. Reiseskizzen. Von Tschifu nach Tientsin [zweiter Teil], In Ostasien. Reiseskizzen. Wu-tschang, In Ostasien. Reiseskizzen. Wu-tschang [zweiter Teil], Kiautschou-Eindrücke. I. Wie man ankommt, Kiautschou-Eindrücke. I. Wie man ankommt [zweiter Teil], Kiautschou-Eindrücke. II. Tsintau, Kiautschou-Eindrücke. II. Tsintau [zweiter Teil], Nach Ostasien. Reiseskizzen, Nach Ostasien. Reiseskizzen, Nach Ostasien. Reiseskizzen, Nach Ostasien. Reiseskizzen. Eine Nacht und ein Morgen in Colombo}
               \section[ Paul Goldmann an Arthur Schnitzler, 25. 9. 1898]{ Paul Goldmann an Arthur Schnitzler, 25. 9. 1898}\nopagebreak\mylabel{v}\rehead{ }\begin{ledgroupsized}[t]{13cm}\normalsize\beginnumbering \toendnotes[C]{\smallbreak\pagebreak[2]} \Standort{DLA, A:Schnitzler, HS.NZ85.1.3168.}
\physDesc{Brief, 2 Blätter, 7 Seiten
\newline{}Handschrift: blaue Tinte, deutsche Kurrent
\newline{}Schnitzler: 1) mit Bleistift das Jahr »98« vermerkt  2) mit rotem Buntstift zwei Unterstreichungen}\toendnotes[C]{\smallbreak}\pstart
           \noindent{}\centering{}{\pb}\textsc{\textcolor{gray}{\textbf{Astor House Hôtel, L\textsuperscript{td}.}}}\oindex{Astor House Hotel [Tianjin]@\textbf{Astor House Hotel [Tianjin]}|pw}\pend
           \pstart
           \noindent{}\raggedleft{}\textcolor{gray}{\textbf{Tientsin\oindex{Tianjin@\textbf{Tianjin}|pw},}}{ }25. September \textcolor{gray}{\textbf{189}}8\pend
           \pstart\center{}Mein lieber Freund,\pend\pstart
           Ich bin jetzt ſehr außerhalb der Poſt-Verbindungen u. habe daher erſt dieſer Tage
               Deinen lieben Brief aus \label{K_L02858-1v}\edtext{\textsc{Salzburg\oindex{Salzburg@\textbf{Salzburg}|pw}}}{\lemma{\textnormal{\emph{Salzburg}}}\Cendnote{\textnormal{siehe A. S.: \emph{Tagebuch}, 28. 7. 1898}}}\label{K_L02858-1h} vom 28. Juli erhalten. Inzwiſchen biſt Du ja
               längſt glücklich heimgekehrt; und wenn Du meinen Brief erhältſt, iſt wohl auch ſchon
               die \label{K_L02858-2v}\edtext{\begin{otherlanguage}{french}\textsc{Première}\end{otherlanguage}{ }Deines neuen Stück\pwindex{Schnitzler, Arthur 15.05.1862 – 21.10.1931@\textsc{Schnitzler, Arthur} (15.05.1862 – 21.10.1931), \emph{Schriftsteller, Mediziner}!Vermaechtnis. Schauspiel in drei Akten1898@\strich\emph{Das Vermächtnis. Schauspiel in drei Akten} {[}1898{]}|pwv}es}{\lemma{\textnormal{\emph{Première … Stückes}}}\Cendnote{\textnormal{\emph{Das Vermächtnis}\pwindex{Schnitzler, Arthur 15.05.1862 – 21.10.1931@\textsc{Schnitzler, Arthur} (15.05.1862 – 21.10.1931), \emph{Schriftsteller, Mediziner}!Vermaechtnis. Schauspiel in drei Akten1898@\strich\emph{Das Vermächtnis. Schauspiel in drei Akten} {[}1898{]}|pwk} wurde am 8. 10. 1898 am Deutschen Theater\oindex{Deutsches Theater Berlin@\textbf{Deutsches Theater Berlin}|pwk} in Berlin\oindex{Berlin@\textbf{Berlin}|pwk} uraufgeführt.}}}\label{K_L02858-2h} vorüber und Du biſt um einen
               neuen Erfolg reicher.\pend
           \pstart
           {\pb}Es iſt heut wieder
               ein Tag, wo ich unſägliches Heimweh habe. Manchmal erwache ich wie aus einen Traume
               und frage\strikeout{,} mich, was ich denn eigentlich hier in
               dieſem Lande\oindex{China@\textbf{China}|pwv} mache? Noch dazu
               bin ich ſeit einigen Wochen recht elend. Die \label{K_L02858-3v}\edtext{\textsc{Dysenterie}}{\lemma{\textnormal{\emph{Dysenterie}}}\Cendnote{\textnormal{Darmentzündung}}}\label{K_L02858-3h} iſt mir in den
               Leib gefahren\substVorne{}\textsuperscript{\textcolor{gray}{,}}\substDazwischen{}und\substHinten{} geht natürlich nicht wieder weg. Das iſt eine ſchlimme Geſchichte. Allein im
               fremden Lande\oindex{China@\textbf{China}|pwv} und auch noch
               krank dazu und die Heimath ſo weit! {\dotsfive}\pend
           \pstart
           {\pb}Ich danke Dir von Herzen für die Aufmerkſamkeit,
               mit der Du meine \label{K_L02858-4v}\edtext{Arbeiten\pwindex{Goldmann, Paul 31.01.1865 – 25.09.1935@\textsc{Goldmann, Paul} (31.01.1865 – 25.09.1935), \emph{Schriftsteller, Journalist}!Nach Ostasien. Reiseskizzen1898-04-24@\strich\emph{Nach Ostasien. Reiseskizzen} {[}1898-04-24{]}|pwv}\pwindex{Goldmann, Paul 31.01.1865 – 25.09.1935@\textsc{Goldmann, Paul} (31.01.1865 – 25.09.1935), \emph{Schriftsteller, Journalist}!Nach Ostasien. Reiseskizzen1898-05-01@\strich\emph{Nach Ostasien. Reiseskizzen} {[}1898-05-01{]}|pwv}\pwindex{Goldmann, Paul 31.01.1865 – 25.09.1935@\textsc{Goldmann, Paul} (31.01.1865 – 25.09.1935), \emph{Schriftsteller, Journalist}!Nach Ostasien. Reiseskizzen1898-05-19@\strich\emph{Nach Ostasien. Reiseskizzen} {[}1898-05-19{]}|pwv}\pwindex{Goldmann, Paul 31.01.1865 – 25.09.1935@\textsc{Goldmann, Paul} (31.01.1865 – 25.09.1935), \emph{Schriftsteller, Journalist}!Nach Ostasien. Reiseskizzen. Eine Nacht und ein Morgen in Colombo1898-05-22@\strich\emph{Nach Ostasien. Reiseskizzen. Eine Nacht und ein Morgen in Colombo} {[}1898-05-22{]}|pwv}\pwindex{Goldmann, Paul 31.01.1865 – 25.09.1935@\textsc{Goldmann, Paul} (31.01.1865 – 25.09.1935), \emph{Schriftsteller, Journalist}!In Ostasien. Reiseskizzen. Singapore1898-06-12@\strich\emph{In Ostasien. Reiseskizzen. Singapore} {[}1898-06-12{]}|pwv}\pwindex{Goldmann, Paul 31.01.1865 – 25.09.1935@\textsc{Goldmann, Paul} (31.01.1865 – 25.09.1935), \emph{Schriftsteller, Journalist}!In Ostasien. Reiseskizzen. Hongkong1898-06-16@\strich\emph{In Ostasien. Reiseskizzen. Hongkong} {[}1898-06-16{]}|pwv}\pwindex{Goldmann, Paul 31.01.1865 – 25.09.1935@\textsc{Goldmann, Paul} (31.01.1865 – 25.09.1935), \emph{Schriftsteller, Journalist}!In Ostasien. Reiseskizzen. Hongkong [zweiter Teil]1898-06-17@\strich\emph{In Ostasien. Reiseskizzen. Hongkong [zweiter Teil]} {[}1898-06-17{]}|pwv}\pwindex{Goldmann, Paul 31.01.1865 – 25.09.1935@\textsc{Goldmann, Paul} (31.01.1865 – 25.09.1935), \emph{Schriftsteller, Journalist}!In Ostasien. Reiseskizzen. Auf dem Perlfluss nach Canton-Shameen1898-06-23@\strich\emph{In Ostasien. Reiseskizzen. Auf dem Perlfluß nach Canton-Shameen} {[}1898-06-23{]}|pwv}\pwindex{Goldmann, Paul 31.01.1865 – 25.09.1935@\textsc{Goldmann, Paul} (31.01.1865 – 25.09.1935), \emph{Schriftsteller, Journalist}!In Ostasien. Reiseskizzen. Auf dem Perlfluss nach Canton-Shameen [zweiter Teil]1898-06-24@\strich\emph{In Ostasien. Reiseskizzen. Auf dem Perlfluß nach Canton-Shameen [zweiter Teil]} {[}1898-06-24{]}|pwv}\pwindex{Goldmann, Paul 31.01.1865 – 25.09.1935@\textsc{Goldmann, Paul} (31.01.1865 – 25.09.1935), \emph{Schriftsteller, Journalist}!In Ostasien. Reiseskizzen. Canton1898-06-29@\strich\emph{In Ostasien. Reiseskizzen. Canton{\rufezeichen}} {[}1898-06-29{]}|pwv}\pwindex{Goldmann, Paul 31.01.1865 – 25.09.1935@\textsc{Goldmann, Paul} (31.01.1865 – 25.09.1935), \emph{Schriftsteller, Journalist}!In Ostasien. Reiseskizzen. Canton [zweiter Teil]1898-06-30@\strich\emph{In Ostasien. Reiseskizzen. Canton{\rufezeichen} [zweiter Teil]} {[}1898-06-30{]}|pwv}\pwindex{Goldmann, Paul 31.01.1865 – 25.09.1935@\textsc{Goldmann, Paul} (31.01.1865 – 25.09.1935), \emph{Schriftsteller, Journalist}!In Ostasien. Reiseskizzen. Von Hongkong nach Shanghai1898-07-14@\strich\emph{In Ostasien. Reiseskizzen. Von Hongkong nach Shanghai} {[}1898-07-14{]}|pwv}\pwindex{Goldmann, Paul 31.01.1865 – 25.09.1935@\textsc{Goldmann, Paul} (31.01.1865 – 25.09.1935), \emph{Schriftsteller, Journalist}!In Ostasien. Reiseskizzen. Von Hongkong nach Shanghai [zweiter Teil]1898-07-15@\strich\emph{In Ostasien. Reiseskizzen. Von Hongkong nach Shanghai [zweiter Teil]} {[}1898-07-15{]}|pwv}\pwindex{Goldmann, Paul 31.01.1865 – 25.09.1935@\textsc{Goldmann, Paul} (31.01.1865 – 25.09.1935), \emph{Schriftsteller, Journalist}!In Ostasien. Reiseskizzen. Shanghai1898-07-24@\strich\emph{In Ostasien. Reiseskizzen. Shanghai} {[}1898-07-24{]}|pwv}\pwindex{Goldmann, Paul 31.01.1865 – 25.09.1935@\textsc{Goldmann, Paul} (31.01.1865 – 25.09.1935), \emph{Schriftsteller, Journalist}!In Ostasien. Reiseskizzen. Shanghai [zweiter Teil]1898-07-26@\strich\emph{In Ostasien. Reiseskizzen. Shanghai [zweiter Teil]} {[}1898-07-26{]}|pwv}\pwindex{Goldmann, Paul 31.01.1865 – 25.09.1935@\textsc{Goldmann, Paul} (31.01.1865 – 25.09.1935), \emph{Schriftsteller, Journalist}!In Ostasien. Reiseskizzen. Chinesisches Nachtleben1898-08-07@\strich\emph{In Ostasien. Reiseskizzen. Chinesisches Nachtleben} {[}1898-08-07{]}|pwv}\pwindex{Goldmann, Paul 31.01.1865 – 25.09.1935@\textsc{Goldmann, Paul} (31.01.1865 – 25.09.1935), \emph{Schriftsteller, Journalist}!In Ostasien. Reiseskizzen. Chinesisches Nachtleben [zweiter Teil]1898-08-09@\strich\emph{In Ostasien. Reiseskizzen. Chinesisches Nachtleben [zweiter Teil]} {[}1898-08-09{]}|pwv}\pwindex{Goldmann, Paul 31.01.1865 – 25.09.1935@\textsc{Goldmann, Paul} (31.01.1865 – 25.09.1935), \emph{Schriftsteller, Journalist}!In Ostasien. Reiseskizzen. Auf dem Yang-tse-Kiang1898-08-21@\strich\emph{In Ostasien. Reiseskizzen. Auf dem Yang-tse-Kiang} {[}1898-08-21{]}|pwv}\pwindex{Goldmann, Paul 31.01.1865 – 25.09.1935@\textsc{Goldmann, Paul} (31.01.1865 – 25.09.1935), \emph{Schriftsteller, Journalist}!In Ostasien. Reiseskizzen. Auf dem Yang-tse-Kiang [zweiter Teil]1898-08-22@\strich\emph{In Ostasien. Reiseskizzen. Auf dem Yang-tse-Kiang [zweiter Teil]} {[}1898-08-22{]}|pwv}\pwindex{Goldmann, Paul 31.01.1865 – 25.09.1935@\textsc{Goldmann, Paul} (31.01.1865 – 25.09.1935), \emph{Schriftsteller, Journalist}!In Ostasien. Reiseskizzen. Hankow1898-08-28@\strich\emph{In Ostasien. Reiseskizzen. Hankow} {[}1898-08-28{]}|pwv}\pwindex{Goldmann, Paul 31.01.1865 – 25.09.1935@\textsc{Goldmann, Paul} (31.01.1865 – 25.09.1935), \emph{Schriftsteller, Journalist}!In Ostasien. Reiseskizzen. Wu-tschang1898-08-30@\strich\emph{In Ostasien. Reiseskizzen. Wu-tschang} {[}1898-08-30{]}|pwv}\pwindex{Goldmann, Paul 31.01.1865 – 25.09.1935@\textsc{Goldmann, Paul} (31.01.1865 – 25.09.1935), \emph{Schriftsteller, Journalist}!In Ostasien. Reiseskizzen. Wu-tschang [zweiter Teil]1898-08-31@\strich\emph{In Ostasien. Reiseskizzen. Wu-tschang [zweiter Teil]} {[}1898-08-31{]}|pwv}\pwindex{Goldmann, Paul 31.01.1865 – 25.09.1935@\textsc{Goldmann, Paul} (31.01.1865 – 25.09.1935), \emph{Schriftsteller, Journalist}!Kiautschou-Eindruecke. I. Wie man ankommt1898-10-05@\strich\emph{Kiautschou-Eindrücke. I. Wie man ankommt} {[}1898-10-05{]}|pwv}\pwindex{Goldmann, Paul 31.01.1865 – 25.09.1935@\textsc{Goldmann, Paul} (31.01.1865 – 25.09.1935), \emph{Schriftsteller, Journalist}!Kiautschou-Eindruecke. I. Wie man ankommt [zweiter Teil]1898-10-06@\strich\emph{Kiautschou-Eindrücke. I. Wie man ankommt [zweiter Teil]} {[}1898-10-06{]}|pwv}\pwindex{Goldmann, Paul 31.01.1865 – 25.09.1935@\textsc{Goldmann, Paul} (31.01.1865 – 25.09.1935), \emph{Schriftsteller, Journalist}!Kiautschou-Eindruecke. II. Tsintau1898-10-08@\strich\emph{Kiautschou-Eindrücke. II. Tsintau} {[}1898-10-08{]}|pwv}\pwindex{Goldmann, Paul 31.01.1865 – 25.09.1935@\textsc{Goldmann, Paul} (31.01.1865 – 25.09.1935), \emph{Schriftsteller, Journalist}!Kiautschou-Eindruecke. II. Tsintau [zweiter Teil]1898-10-09@\strich\emph{Kiautschou-Eindrücke. II. Tsintau [zweiter Teil]} {[}1898-10-09{]}|pwv}\pwindex{Goldmann, Paul 31.01.1865 – 25.09.1935@\textsc{Goldmann, Paul} (31.01.1865 – 25.09.1935), \emph{Schriftsteller, Journalist}!In Ostasien. Reiseskizzen. Im Golf von Pe-tschi-li1898-10-16@\strich\emph{In Ostasien. Reiseskizzen. Im Golf von Pe-tschi-li} {[}1898-10-16{]}|pwv}\pwindex{Goldmann, Paul 31.01.1865 – 25.09.1935@\textsc{Goldmann, Paul} (31.01.1865 – 25.09.1935), \emph{Schriftsteller, Journalist}!In Ostasien. Reiseskizzen. Im Golf von Pe-tschi-li [zweiter Teil]1898-10-18@\strich\emph{In Ostasien. Reiseskizzen. Im Golf von Pe-tschi-li [zweiter Teil]} {[}1898-10-18{]}|pwv}\pwindex{Goldmann, Paul 31.01.1865 – 25.09.1935@\textsc{Goldmann, Paul} (31.01.1865 – 25.09.1935), \emph{Schriftsteller, Journalist}!In Ostasien. Reiseskizzen. Von Tschifu nach Tientsin1898-10-30@\strich\emph{In Ostasien. Reiseskizzen. Von Tschifu nach Tientsin} {[}1898-10-30{]}|pwv}\pwindex{Goldmann, Paul 31.01.1865 – 25.09.1935@\textsc{Goldmann, Paul} (31.01.1865 – 25.09.1935), \emph{Schriftsteller, Journalist}!In Ostasien. Reiseskizzen. Von Tschifu nach Tientsin [zweiter Teil]1898-10-31@\strich\emph{In Ostasien. Reiseskizzen. Von Tschifu nach Tientsin [zweiter Teil]} {[}1898-10-31{]}|pwv}\pwindex{Goldmann, Paul 31.01.1865 – 25.09.1935@\textsc{Goldmann, Paul} (31.01.1865 – 25.09.1935), \emph{Schriftsteller, Journalist}!Pest in Hongkong1898-06-08@\strich\emph{Die Pest in Hongkong} {[}1898-06-08{]}|pwv}\pwindex{Goldmann, Paul 31.01.1865 – 25.09.1935@\textsc{Goldmann, Paul} (31.01.1865 – 25.09.1935), \emph{Schriftsteller, Journalist}!Eine Unterredung mit dem Tao-tai Wang, dem Sekretaer des Vicekoenigs von Canton1898-06-23@\strich\emph{Eine Unterredung mit dem Tao-tai Wang, dem Sekretär des Vicekönigs von Canton} {[}1898-06-23{]}|pwv}\pwindex{Goldmann, Paul 31.01.1865 – 25.09.1935@\textsc{Goldmann, Paul} (31.01.1865 – 25.09.1935), \emph{Schriftsteller, Journalist}!Eine Unterredung mit dem Tao-tai Wang, dem Sekretaer des Vicekoenigs von Canton [zweiter Teil]1898-06-23@\strich\emph{Eine Unterredung mit dem Tao-tai Wang, dem Sekretär des Vicekönigs von Canton [zweiter Teil]} {[}1898-06-23{]}|pwv}\pwindex{Goldmann, Paul 31.01.1865 – 25.09.1935@\textsc{Goldmann, Paul} (31.01.1865 – 25.09.1935), \emph{Schriftsteller, Journalist}!Beim Tao-tai Tsai von Shanghai1898-07-21@\strich\emph{Beim Tao-tai Tsai von Shanghai} {[}1898-07-21{]}|pwv}\pwindex{Goldmann, Paul 31.01.1865 – 25.09.1935@\textsc{Goldmann, Paul} (31.01.1865 – 25.09.1935), \emph{Schriftsteller, Journalist}!deutschen Militaer-Instruktoren in China1898-07-23@\strich\emph{Die deutschen Militär-Instruktoren in China} {[}1898-07-23{]}|pwv}\pwindex{Goldmann, Paul 31.01.1865 – 25.09.1935@\textsc{Goldmann, Paul} (31.01.1865 – 25.09.1935), \emph{Schriftsteller, Journalist}!Kapitel ueber chinesische Eisenbahnen. Chinesische Eisenbahnen und deutsche Versaeumnisse1898-08-03@\strich\emph{Ein Kapitel über chinesische Eisenbahnen. Chinesische Eisenbahnen und deutsche Versäumnisse} {[}1898-08-03{]}|pwv}\pwindex{Goldmann, Paul 31.01.1865 – 25.09.1935@\textsc{Goldmann, Paul} (31.01.1865 – 25.09.1935), \emph{Schriftsteller, Journalist}!Kapitel ueber chinesische Eisenbahnen. Die Bahn in Shanghai nach Wu-sung1898-08-04@\strich\emph{Ein Kapitel über chinesische Eisenbahnen. Die Bahn in Shanghai nach Wu-sung} {[}1898-08-04{]}|pwv}\pwindex{Goldmann, Paul 31.01.1865 – 25.09.1935@\textsc{Goldmann, Paul} (31.01.1865 – 25.09.1935), \emph{Schriftsteller, Journalist}!Chinesische Zeitungen1898-08-17@\strich\emph{Chinesische Zeitungen} {[}1898-08-17{]}|pwv}\pwindex{Goldmann, Paul 31.01.1865 – 25.09.1935@\textsc{Goldmann, Paul} (31.01.1865 – 25.09.1935), \emph{Schriftsteller, Journalist}!franzoesisch-chinesische Zwischenfall in Shanghai1898-08-25@\strich\emph{Der französisch-chinesische Zwischenfall in Shanghai} {[}1898-08-25{]}|pwv}\pwindex{Goldmann, Paul 31.01.1865 – 25.09.1935@\textsc{Goldmann, Paul} (31.01.1865 – 25.09.1935), \emph{Schriftsteller, Journalist}!franzoesisch-chinesische Zwischenfall in Shanghai1898-09-09@\strich\emph{Der französisch-chinesische Zwischenfall in Shanghai} {[}1898-09-09{]}|pwv}\pwindex{Goldmann, Paul 31.01.1865 – 25.09.1935@\textsc{Goldmann, Paul} (31.01.1865 – 25.09.1935), \emph{Schriftsteller, Journalist}!In Kiautschou. I1898-09-23@\strich\emph{In Kiautschou. I} {[}1898-09-23{]}|pwv}\pwindex{Goldmann, Paul 31.01.1865 – 25.09.1935@\textsc{Goldmann, Paul} (31.01.1865 – 25.09.1935), \emph{Schriftsteller, Journalist}!In Kiautschou. II1898-09-24@\strich\emph{In Kiautschou. II} {[}1898-09-24{]}|pwv}\pwindex{Goldmann, Paul 31.01.1865 – 25.09.1935@\textsc{Goldmann, Paul} (31.01.1865 – 25.09.1935), \emph{Schriftsteller, Journalist}!In Kiautschou. III1898-09-25@\strich\emph{In Kiautschou. III} {[}1898-09-25{]}|pwv}\pwindex{Goldmann, Paul 31.01.1865 – 25.09.1935@\textsc{Goldmann, Paul} (31.01.1865 – 25.09.1935), \emph{Schriftsteller, Journalist}!In Kiautschou. IV [letzter Teil]1898-09-26@\strich\emph{In Kiautschou. IV [letzter Teil]} {[}1898-09-26{]}|pwv}\pwindex{Goldmann, Paul 31.01.1865 – 25.09.1935@\textsc{Goldmann, Paul} (31.01.1865 – 25.09.1935), \emph{Schriftsteller, Journalist}!General Tscheng-Ki-tong1898-10-25@\strich\emph{Der General Tscheng-Ki-tong} {[}1898-10-25{]}|pwv}}{\lemma{\textnormal{\emph{Arbeiten}}}\Cendnote{\textnormal{Schnitzler\pwindex{Schnitzler, Arthur 15.05.1862 – 21.10.1931@\textsc{Schnitzler, Arthur} (15.05.1862 – 21.10.1931), \emph{Schriftsteller, Mediziner}|pwk} dürfte regelmäßig die \emph{Frankfurter Zeitung}\pwindex{?? Werk@Nicht ermittelte Verfasserinnen und Verfasser!Frankfurter Zeitung1856 – 1943@\emph{Frankfurter Zeitung} {[}1856 – 1943{]}|pwk} gelesen haben, in der Goldmann\pwindex{Goldmann, Paul 31.01.1865 – 25.09.1935@\textsc{Goldmann, Paul} (31.01.1865 – 25.09.1935), \emph{Schriftsteller, Journalist}|pwk}s Reisefeuilletons\pwindex{Goldmann, Paul 31.01.1865 – 25.09.1935@\textsc{Goldmann, Paul} (31.01.1865 – 25.09.1935), \emph{Schriftsteller, Journalist}!Nach Ostasien. Reiseskizzen1898-04-24@\strich\emph{Nach Ostasien. Reiseskizzen} {[}1898-04-24{]}|pwkv}\pwindex{Goldmann, Paul 31.01.1865 – 25.09.1935@\textsc{Goldmann, Paul} (31.01.1865 – 25.09.1935), \emph{Schriftsteller, Journalist}!Nach Ostasien. Reiseskizzen1898-05-01@\strich\emph{Nach Ostasien. Reiseskizzen} {[}1898-05-01{]}|pwkv}\pwindex{Goldmann, Paul 31.01.1865 – 25.09.1935@\textsc{Goldmann, Paul} (31.01.1865 – 25.09.1935), \emph{Schriftsteller, Journalist}!Nach Ostasien. Reiseskizzen1898-05-19@\strich\emph{Nach Ostasien. Reiseskizzen} {[}1898-05-19{]}|pwkv}\pwindex{Goldmann, Paul 31.01.1865 – 25.09.1935@\textsc{Goldmann, Paul} (31.01.1865 – 25.09.1935), \emph{Schriftsteller, Journalist}!Nach Ostasien. Reiseskizzen. Eine Nacht und ein Morgen in Colombo1898-05-22@\strich\emph{Nach Ostasien. Reiseskizzen. Eine Nacht und ein Morgen in Colombo} {[}1898-05-22{]}|pwkv}\pwindex{Goldmann, Paul 31.01.1865 – 25.09.1935@\textsc{Goldmann, Paul} (31.01.1865 – 25.09.1935), \emph{Schriftsteller, Journalist}!In Ostasien. Reiseskizzen. Singapore1898-06-12@\strich\emph{In Ostasien. Reiseskizzen. Singapore} {[}1898-06-12{]}|pwkv}\pwindex{Goldmann, Paul 31.01.1865 – 25.09.1935@\textsc{Goldmann, Paul} (31.01.1865 – 25.09.1935), \emph{Schriftsteller, Journalist}!In Ostasien. Reiseskizzen. Hongkong1898-06-16@\strich\emph{In Ostasien. Reiseskizzen. Hongkong} {[}1898-06-16{]}|pwkv}\pwindex{Goldmann, Paul 31.01.1865 – 25.09.1935@\textsc{Goldmann, Paul} (31.01.1865 – 25.09.1935), \emph{Schriftsteller, Journalist}!In Ostasien. Reiseskizzen. Hongkong [zweiter Teil]1898-06-17@\strich\emph{In Ostasien. Reiseskizzen. Hongkong [zweiter Teil]} {[}1898-06-17{]}|pwkv}\pwindex{Goldmann, Paul 31.01.1865 – 25.09.1935@\textsc{Goldmann, Paul} (31.01.1865 – 25.09.1935), \emph{Schriftsteller, Journalist}!In Ostasien. Reiseskizzen. Auf dem Perlfluss nach Canton-Shameen1898-06-23@\strich\emph{In Ostasien. Reiseskizzen. Auf dem Perlfluß nach Canton-Shameen} {[}1898-06-23{]}|pwkv}\pwindex{Goldmann, Paul 31.01.1865 – 25.09.1935@\textsc{Goldmann, Paul} (31.01.1865 – 25.09.1935), \emph{Schriftsteller, Journalist}!In Ostasien. Reiseskizzen. Auf dem Perlfluss nach Canton-Shameen [zweiter Teil]1898-06-24@\strich\emph{In Ostasien. Reiseskizzen. Auf dem Perlfluß nach Canton-Shameen [zweiter Teil]} {[}1898-06-24{]}|pwkv}\pwindex{Goldmann, Paul 31.01.1865 – 25.09.1935@\textsc{Goldmann, Paul} (31.01.1865 – 25.09.1935), \emph{Schriftsteller, Journalist}!In Ostasien. Reiseskizzen. Canton1898-06-29@\strich\emph{In Ostasien. Reiseskizzen. Canton{\rufezeichen}} {[}1898-06-29{]}|pwkv}\pwindex{Goldmann, Paul 31.01.1865 – 25.09.1935@\textsc{Goldmann, Paul} (31.01.1865 – 25.09.1935), \emph{Schriftsteller, Journalist}!In Ostasien. Reiseskizzen. Canton [zweiter Teil]1898-06-30@\strich\emph{In Ostasien. Reiseskizzen. Canton{\rufezeichen} [zweiter Teil]} {[}1898-06-30{]}|pwkv}\pwindex{Goldmann, Paul 31.01.1865 – 25.09.1935@\textsc{Goldmann, Paul} (31.01.1865 – 25.09.1935), \emph{Schriftsteller, Journalist}!In Ostasien. Reiseskizzen. Von Hongkong nach Shanghai1898-07-14@\strich\emph{In Ostasien. Reiseskizzen. Von Hongkong nach Shanghai} {[}1898-07-14{]}|pwkv}\pwindex{Goldmann, Paul 31.01.1865 – 25.09.1935@\textsc{Goldmann, Paul} (31.01.1865 – 25.09.1935), \emph{Schriftsteller, Journalist}!In Ostasien. Reiseskizzen. Von Hongkong nach Shanghai [zweiter Teil]1898-07-15@\strich\emph{In Ostasien. Reiseskizzen. Von Hongkong nach Shanghai [zweiter Teil]} {[}1898-07-15{]}|pwkv}\pwindex{Goldmann, Paul 31.01.1865 – 25.09.1935@\textsc{Goldmann, Paul} (31.01.1865 – 25.09.1935), \emph{Schriftsteller, Journalist}!In Ostasien. Reiseskizzen. Shanghai1898-07-24@\strich\emph{In Ostasien. Reiseskizzen. Shanghai} {[}1898-07-24{]}|pwkv}\pwindex{Goldmann, Paul 31.01.1865 – 25.09.1935@\textsc{Goldmann, Paul} (31.01.1865 – 25.09.1935), \emph{Schriftsteller, Journalist}!In Ostasien. Reiseskizzen. Shanghai [zweiter Teil]1898-07-26@\strich\emph{In Ostasien. Reiseskizzen. Shanghai [zweiter Teil]} {[}1898-07-26{]}|pwkv}\pwindex{Goldmann, Paul 31.01.1865 – 25.09.1935@\textsc{Goldmann, Paul} (31.01.1865 – 25.09.1935), \emph{Schriftsteller, Journalist}!In Ostasien. Reiseskizzen. Chinesisches Nachtleben1898-08-07@\strich\emph{In Ostasien. Reiseskizzen. Chinesisches Nachtleben} {[}1898-08-07{]}|pwkv}\pwindex{Goldmann, Paul 31.01.1865 – 25.09.1935@\textsc{Goldmann, Paul} (31.01.1865 – 25.09.1935), \emph{Schriftsteller, Journalist}!In Ostasien. Reiseskizzen. Chinesisches Nachtleben [zweiter Teil]1898-08-09@\strich\emph{In Ostasien. Reiseskizzen. Chinesisches Nachtleben [zweiter Teil]} {[}1898-08-09{]}|pwkv}\pwindex{Goldmann, Paul 31.01.1865 – 25.09.1935@\textsc{Goldmann, Paul} (31.01.1865 – 25.09.1935), \emph{Schriftsteller, Journalist}!In Ostasien. Reiseskizzen. Auf dem Yang-tse-Kiang1898-08-21@\strich\emph{In Ostasien. Reiseskizzen. Auf dem Yang-tse-Kiang} {[}1898-08-21{]}|pwkv}\pwindex{Goldmann, Paul 31.01.1865 – 25.09.1935@\textsc{Goldmann, Paul} (31.01.1865 – 25.09.1935), \emph{Schriftsteller, Journalist}!In Ostasien. Reiseskizzen. Auf dem Yang-tse-Kiang [zweiter Teil]1898-08-22@\strich\emph{In Ostasien. Reiseskizzen. Auf dem Yang-tse-Kiang [zweiter Teil]} {[}1898-08-22{]}|pwkv}\pwindex{Goldmann, Paul 31.01.1865 – 25.09.1935@\textsc{Goldmann, Paul} (31.01.1865 – 25.09.1935), \emph{Schriftsteller, Journalist}!In Ostasien. Reiseskizzen. Hankow1898-08-28@\strich\emph{In Ostasien. Reiseskizzen. Hankow} {[}1898-08-28{]}|pwkv}\pwindex{Goldmann, Paul 31.01.1865 – 25.09.1935@\textsc{Goldmann, Paul} (31.01.1865 – 25.09.1935), \emph{Schriftsteller, Journalist}!In Ostasien. Reiseskizzen. Wu-tschang1898-08-30@\strich\emph{In Ostasien. Reiseskizzen. Wu-tschang} {[}1898-08-30{]}|pwkv}\pwindex{Goldmann, Paul 31.01.1865 – 25.09.1935@\textsc{Goldmann, Paul} (31.01.1865 – 25.09.1935), \emph{Schriftsteller, Journalist}!In Ostasien. Reiseskizzen. Wu-tschang [zweiter Teil]1898-08-31@\strich\emph{In Ostasien. Reiseskizzen. Wu-tschang [zweiter Teil]} {[}1898-08-31{]}|pwkv}\pwindex{Goldmann, Paul 31.01.1865 – 25.09.1935@\textsc{Goldmann, Paul} (31.01.1865 – 25.09.1935), \emph{Schriftsteller, Journalist}!Kiautschou-Eindruecke. I. Wie man ankommt1898-10-05@\strich\emph{Kiautschou-Eindrücke. I. Wie man ankommt} {[}1898-10-05{]}|pwkv}\pwindex{Goldmann, Paul 31.01.1865 – 25.09.1935@\textsc{Goldmann, Paul} (31.01.1865 – 25.09.1935), \emph{Schriftsteller, Journalist}!Kiautschou-Eindruecke. I. Wie man ankommt [zweiter Teil]1898-10-06@\strich\emph{Kiautschou-Eindrücke. I. Wie man ankommt [zweiter Teil]} {[}1898-10-06{]}|pwkv}\pwindex{Goldmann, Paul 31.01.1865 – 25.09.1935@\textsc{Goldmann, Paul} (31.01.1865 – 25.09.1935), \emph{Schriftsteller, Journalist}!Kiautschou-Eindruecke. II. Tsintau1898-10-08@\strich\emph{Kiautschou-Eindrücke. II. Tsintau} {[}1898-10-08{]}|pwkv}\pwindex{Goldmann, Paul 31.01.1865 – 25.09.1935@\textsc{Goldmann, Paul} (31.01.1865 – 25.09.1935), \emph{Schriftsteller, Journalist}!Kiautschou-Eindruecke. II. Tsintau [zweiter Teil]1898-10-09@\strich\emph{Kiautschou-Eindrücke. II. Tsintau [zweiter Teil]} {[}1898-10-09{]}|pwkv}\pwindex{Goldmann, Paul 31.01.1865 – 25.09.1935@\textsc{Goldmann, Paul} (31.01.1865 – 25.09.1935), \emph{Schriftsteller, Journalist}!In Ostasien. Reiseskizzen. Im Golf von Pe-tschi-li1898-10-16@\strich\emph{In Ostasien. Reiseskizzen. Im Golf von Pe-tschi-li} {[}1898-10-16{]}|pwkv}\pwindex{Goldmann, Paul 31.01.1865 – 25.09.1935@\textsc{Goldmann, Paul} (31.01.1865 – 25.09.1935), \emph{Schriftsteller, Journalist}!In Ostasien. Reiseskizzen. Im Golf von Pe-tschi-li [zweiter Teil]1898-10-18@\strich\emph{In Ostasien. Reiseskizzen. Im Golf von Pe-tschi-li [zweiter Teil]} {[}1898-10-18{]}|pwkv}\pwindex{Goldmann, Paul 31.01.1865 – 25.09.1935@\textsc{Goldmann, Paul} (31.01.1865 – 25.09.1935), \emph{Schriftsteller, Journalist}!In Ostasien. Reiseskizzen. Von Tschifu nach Tientsin1898-10-30@\strich\emph{In Ostasien. Reiseskizzen. Von Tschifu nach Tientsin} {[}1898-10-30{]}|pwkv}\pwindex{Goldmann, Paul 31.01.1865 – 25.09.1935@\textsc{Goldmann, Paul} (31.01.1865 – 25.09.1935), \emph{Schriftsteller, Journalist}!In Ostasien. Reiseskizzen. Von Tschifu nach Tientsin [zweiter Teil]1898-10-31@\strich\emph{In Ostasien. Reiseskizzen. Von Tschifu nach Tientsin [zweiter Teil]} {[}1898-10-31{]}|pwkv} (unter Angabe des vollen Namens) und Berichterstattungen\pwindex{Goldmann, Paul 31.01.1865 – 25.09.1935@\textsc{Goldmann, Paul} (31.01.1865 – 25.09.1935), \emph{Schriftsteller, Journalist}!Pest in Hongkong1898-06-08@\strich\emph{Die Pest in Hongkong} {[}1898-06-08{]}|pwkv}\pwindex{Goldmann, Paul 31.01.1865 – 25.09.1935@\textsc{Goldmann, Paul} (31.01.1865 – 25.09.1935), \emph{Schriftsteller, Journalist}!Eine Unterredung mit dem Tao-tai Wang, dem Sekretaer des Vicekoenigs von Canton1898-06-23@\strich\emph{Eine Unterredung mit dem Tao-tai Wang, dem Sekretär des Vicekönigs von Canton} {[}1898-06-23{]}|pwkv}\pwindex{Goldmann, Paul 31.01.1865 – 25.09.1935@\textsc{Goldmann, Paul} (31.01.1865 – 25.09.1935), \emph{Schriftsteller, Journalist}!Eine Unterredung mit dem Tao-tai Wang, dem Sekretaer des Vicekoenigs von Canton [zweiter Teil]1898-06-23@\strich\emph{Eine Unterredung mit dem Tao-tai Wang, dem Sekretär des Vicekönigs von Canton [zweiter Teil]} {[}1898-06-23{]}|pwkv}\pwindex{Goldmann, Paul 31.01.1865 – 25.09.1935@\textsc{Goldmann, Paul} (31.01.1865 – 25.09.1935), \emph{Schriftsteller, Journalist}!Beim Tao-tai Tsai von Shanghai1898-07-21@\strich\emph{Beim Tao-tai Tsai von Shanghai} {[}1898-07-21{]}|pwkv}\pwindex{Goldmann, Paul 31.01.1865 – 25.09.1935@\textsc{Goldmann, Paul} (31.01.1865 – 25.09.1935), \emph{Schriftsteller, Journalist}!deutschen Militaer-Instruktoren in China1898-07-23@\strich\emph{Die deutschen Militär-Instruktoren in China} {[}1898-07-23{]}|pwkv}\pwindex{Goldmann, Paul 31.01.1865 – 25.09.1935@\textsc{Goldmann, Paul} (31.01.1865 – 25.09.1935), \emph{Schriftsteller, Journalist}!Kapitel ueber chinesische Eisenbahnen. Chinesische Eisenbahnen und deutsche Versaeumnisse1898-08-03@\strich\emph{Ein Kapitel über chinesische Eisenbahnen. Chinesische Eisenbahnen und deutsche Versäumnisse} {[}1898-08-03{]}|pwkv}\pwindex{Goldmann, Paul 31.01.1865 – 25.09.1935@\textsc{Goldmann, Paul} (31.01.1865 – 25.09.1935), \emph{Schriftsteller, Journalist}!Kapitel ueber chinesische Eisenbahnen. Die Bahn in Shanghai nach Wu-sung1898-08-04@\strich\emph{Ein Kapitel über chinesische Eisenbahnen. Die Bahn in Shanghai nach Wu-sung} {[}1898-08-04{]}|pwkv}\pwindex{Goldmann, Paul 31.01.1865 – 25.09.1935@\textsc{Goldmann, Paul} (31.01.1865 – 25.09.1935), \emph{Schriftsteller, Journalist}!Chinesische Zeitungen1898-08-17@\strich\emph{Chinesische Zeitungen} {[}1898-08-17{]}|pwkv}\pwindex{Goldmann, Paul 31.01.1865 – 25.09.1935@\textsc{Goldmann, Paul} (31.01.1865 – 25.09.1935), \emph{Schriftsteller, Journalist}!franzoesisch-chinesische Zwischenfall in Shanghai1898-08-25@\strich\emph{Der französisch-chinesische Zwischenfall in Shanghai} {[}1898-08-25{]}|pwkv}\pwindex{Goldmann, Paul 31.01.1865 – 25.09.1935@\textsc{Goldmann, Paul} (31.01.1865 – 25.09.1935), \emph{Schriftsteller, Journalist}!franzoesisch-chinesische Zwischenfall in Shanghai1898-09-09@\strich\emph{Der französisch-chinesische Zwischenfall in Shanghai} {[}1898-09-09{]}|pwkv}\pwindex{Goldmann, Paul 31.01.1865 – 25.09.1935@\textsc{Goldmann, Paul} (31.01.1865 – 25.09.1935), \emph{Schriftsteller, Journalist}!In Kiautschou. I1898-09-23@\strich\emph{In Kiautschou. I} {[}1898-09-23{]}|pwkv}\pwindex{Goldmann, Paul 31.01.1865 – 25.09.1935@\textsc{Goldmann, Paul} (31.01.1865 – 25.09.1935), \emph{Schriftsteller, Journalist}!In Kiautschou. II1898-09-24@\strich\emph{In Kiautschou. II} {[}1898-09-24{]}|pwkv}\pwindex{Goldmann, Paul 31.01.1865 – 25.09.1935@\textsc{Goldmann, Paul} (31.01.1865 – 25.09.1935), \emph{Schriftsteller, Journalist}!In Kiautschou. III1898-09-25@\strich\emph{In Kiautschou. III} {[}1898-09-25{]}|pwkv}\pwindex{Goldmann, Paul 31.01.1865 – 25.09.1935@\textsc{Goldmann, Paul} (31.01.1865 – 25.09.1935), \emph{Schriftsteller, Journalist}!In Kiautschou. IV [letzter Teil]1898-09-26@\strich\emph{In Kiautschou. IV [letzter Teil]} {[}1898-09-26{]}|pwkv}\pwindex{Goldmann, Paul 31.01.1865 – 25.09.1935@\textsc{Goldmann, Paul} (31.01.1865 – 25.09.1935), \emph{Schriftsteller, Journalist}!General Tscheng-Ki-tong1898-10-25@\strich\emph{Der General Tscheng-Ki-tong} {[}1898-10-25{]}|pwkv} (unter Angabe des Kürzels »G\pwindex{Goldmann, Paul 31.01.1865 – 25.09.1935@\textsc{Goldmann, Paul} (31.01.1865 – 25.09.1935), \emph{Schriftsteller, Journalist}|pwkv}«) aus Ostasien\oindex{Asien@\textbf{Asien}|pwk} erschienen. Reisefeuilletons
                  erschienen am 24. 4.\pwindex{Goldmann, Paul 31.01.1865 – 25.09.1935@\textsc{Goldmann, Paul} (31.01.1865 – 25.09.1935), \emph{Schriftsteller, Journalist}!Nach Ostasien. Reiseskizzen1898-04-24@\strich\emph{Nach Ostasien. Reiseskizzen} {[}1898-04-24{]}|pwkv}, 1. 5.\pwindex{Goldmann, Paul 31.01.1865 – 25.09.1935@\textsc{Goldmann, Paul} (31.01.1865 – 25.09.1935), \emph{Schriftsteller, Journalist}!Nach Ostasien. Reiseskizzen1898-05-01@\strich\emph{Nach Ostasien. Reiseskizzen} {[}1898-05-01{]}|pwkv}, 19. 5.\pwindex{Goldmann, Paul 31.01.1865 – 25.09.1935@\textsc{Goldmann, Paul} (31.01.1865 – 25.09.1935), \emph{Schriftsteller, Journalist}!Nach Ostasien. Reiseskizzen1898-05-19@\strich\emph{Nach Ostasien. Reiseskizzen} {[}1898-05-19{]}|pwkv}, 22. 5.\pwindex{Goldmann, Paul 31.01.1865 – 25.09.1935@\textsc{Goldmann, Paul} (31.01.1865 – 25.09.1935), \emph{Schriftsteller, Journalist}!Nach Ostasien. Reiseskizzen. Eine Nacht und ein Morgen in Colombo1898-05-22@\strich\emph{Nach Ostasien. Reiseskizzen. Eine Nacht und ein Morgen in Colombo} {[}1898-05-22{]}|pwkv}, 12. 6.\pwindex{Goldmann, Paul 31.01.1865 – 25.09.1935@\textsc{Goldmann, Paul} (31.01.1865 – 25.09.1935), \emph{Schriftsteller, Journalist}!In Ostasien. Reiseskizzen. Singapore1898-06-12@\strich\emph{In Ostasien. Reiseskizzen. Singapore} {[}1898-06-12{]}|pwkv}, 16. 6.\pwindex{Goldmann, Paul 31.01.1865 – 25.09.1935@\textsc{Goldmann, Paul} (31.01.1865 – 25.09.1935), \emph{Schriftsteller, Journalist}!In Ostasien. Reiseskizzen. Hongkong1898-06-16@\strich\emph{In Ostasien. Reiseskizzen. Hongkong} {[}1898-06-16{]}|pwkv}, 17. 6.\pwindex{Goldmann, Paul 31.01.1865 – 25.09.1935@\textsc{Goldmann, Paul} (31.01.1865 – 25.09.1935), \emph{Schriftsteller, Journalist}!In Ostasien. Reiseskizzen. Hongkong [zweiter Teil]1898-06-17@\strich\emph{In Ostasien. Reiseskizzen. Hongkong [zweiter Teil]} {[}1898-06-17{]}|pwkv}, 23. 6.\pwindex{Goldmann, Paul 31.01.1865 – 25.09.1935@\textsc{Goldmann, Paul} (31.01.1865 – 25.09.1935), \emph{Schriftsteller, Journalist}!In Ostasien. Reiseskizzen. Auf dem Perlfluss nach Canton-Shameen1898-06-23@\strich\emph{In Ostasien. Reiseskizzen. Auf dem Perlfluß nach Canton-Shameen} {[}1898-06-23{]}|pwkv}, 24. 6.\pwindex{Goldmann, Paul 31.01.1865 – 25.09.1935@\textsc{Goldmann, Paul} (31.01.1865 – 25.09.1935), \emph{Schriftsteller, Journalist}!In Ostasien. Reiseskizzen. Auf dem Perlfluss nach Canton-Shameen [zweiter Teil]1898-06-24@\strich\emph{In Ostasien. Reiseskizzen. Auf dem Perlfluß nach Canton-Shameen [zweiter Teil]} {[}1898-06-24{]}|pwkv}, 29. 6.\pwindex{Goldmann, Paul 31.01.1865 – 25.09.1935@\textsc{Goldmann, Paul} (31.01.1865 – 25.09.1935), \emph{Schriftsteller, Journalist}!In Ostasien. Reiseskizzen. Canton1898-06-29@\strich\emph{In Ostasien. Reiseskizzen. Canton{\rufezeichen}} {[}1898-06-29{]}|pwkv}, 30. 6.\pwindex{Goldmann, Paul 31.01.1865 – 25.09.1935@\textsc{Goldmann, Paul} (31.01.1865 – 25.09.1935), \emph{Schriftsteller, Journalist}!In Ostasien. Reiseskizzen. Canton [zweiter Teil]1898-06-30@\strich\emph{In Ostasien. Reiseskizzen. Canton{\rufezeichen} [zweiter Teil]} {[}1898-06-30{]}|pwkv}, 14. 7.\pwindex{Goldmann, Paul 31.01.1865 – 25.09.1935@\textsc{Goldmann, Paul} (31.01.1865 – 25.09.1935), \emph{Schriftsteller, Journalist}!In Ostasien. Reiseskizzen. Von Hongkong nach Shanghai1898-07-14@\strich\emph{In Ostasien. Reiseskizzen. Von Hongkong nach Shanghai} {[}1898-07-14{]}|pwkv}, 15. 7.\pwindex{Goldmann, Paul 31.01.1865 – 25.09.1935@\textsc{Goldmann, Paul} (31.01.1865 – 25.09.1935), \emph{Schriftsteller, Journalist}!In Ostasien. Reiseskizzen. Von Hongkong nach Shanghai [zweiter Teil]1898-07-15@\strich\emph{In Ostasien. Reiseskizzen. Von Hongkong nach Shanghai [zweiter Teil]} {[}1898-07-15{]}|pwkv}, 24. 7.\pwindex{Goldmann, Paul 31.01.1865 – 25.09.1935@\textsc{Goldmann, Paul} (31.01.1865 – 25.09.1935), \emph{Schriftsteller, Journalist}!In Ostasien. Reiseskizzen. Shanghai1898-07-24@\strich\emph{In Ostasien. Reiseskizzen. Shanghai} {[}1898-07-24{]}|pwkv}, 26. 7.\pwindex{Goldmann, Paul 31.01.1865 – 25.09.1935@\textsc{Goldmann, Paul} (31.01.1865 – 25.09.1935), \emph{Schriftsteller, Journalist}!In Ostasien. Reiseskizzen. Shanghai [zweiter Teil]1898-07-26@\strich\emph{In Ostasien. Reiseskizzen. Shanghai [zweiter Teil]} {[}1898-07-26{]}|pwkv}, 7. 8.\pwindex{Goldmann, Paul 31.01.1865 – 25.09.1935@\textsc{Goldmann, Paul} (31.01.1865 – 25.09.1935), \emph{Schriftsteller, Journalist}!In Ostasien. Reiseskizzen. Chinesisches Nachtleben1898-08-07@\strich\emph{In Ostasien. Reiseskizzen. Chinesisches Nachtleben} {[}1898-08-07{]}|pwkv}, 9. 8.\pwindex{Goldmann, Paul 31.01.1865 – 25.09.1935@\textsc{Goldmann, Paul} (31.01.1865 – 25.09.1935), \emph{Schriftsteller, Journalist}!In Ostasien. Reiseskizzen. Chinesisches Nachtleben [zweiter Teil]1898-08-09@\strich\emph{In Ostasien. Reiseskizzen. Chinesisches Nachtleben [zweiter Teil]} {[}1898-08-09{]}|pwkv}, 21. 8.\pwindex{Goldmann, Paul 31.01.1865 – 25.09.1935@\textsc{Goldmann, Paul} (31.01.1865 – 25.09.1935), \emph{Schriftsteller, Journalist}!In Ostasien. Reiseskizzen. Auf dem Yang-tse-Kiang1898-08-21@\strich\emph{In Ostasien. Reiseskizzen. Auf dem Yang-tse-Kiang} {[}1898-08-21{]}|pwkv}, 22. 8.\pwindex{Goldmann, Paul 31.01.1865 – 25.09.1935@\textsc{Goldmann, Paul} (31.01.1865 – 25.09.1935), \emph{Schriftsteller, Journalist}!In Ostasien. Reiseskizzen. Auf dem Yang-tse-Kiang [zweiter Teil]1898-08-22@\strich\emph{In Ostasien. Reiseskizzen. Auf dem Yang-tse-Kiang [zweiter Teil]} {[}1898-08-22{]}|pwkv}, 28. 8.\pwindex{Goldmann, Paul 31.01.1865 – 25.09.1935@\textsc{Goldmann, Paul} (31.01.1865 – 25.09.1935), \emph{Schriftsteller, Journalist}!In Ostasien. Reiseskizzen. Hankow1898-08-28@\strich\emph{In Ostasien. Reiseskizzen. Hankow} {[}1898-08-28{]}|pwkv}, 30. 8.\pwindex{Goldmann, Paul 31.01.1865 – 25.09.1935@\textsc{Goldmann, Paul} (31.01.1865 – 25.09.1935), \emph{Schriftsteller, Journalist}!In Ostasien. Reiseskizzen. Wu-tschang1898-08-30@\strich\emph{In Ostasien. Reiseskizzen. Wu-tschang} {[}1898-08-30{]}|pwkv}, 31. 8.\pwindex{Goldmann, Paul 31.01.1865 – 25.09.1935@\textsc{Goldmann, Paul} (31.01.1865 – 25.09.1935), \emph{Schriftsteller, Journalist}!In Ostasien. Reiseskizzen. Wu-tschang [zweiter Teil]1898-08-31@\strich\emph{In Ostasien. Reiseskizzen. Wu-tschang [zweiter Teil]} {[}1898-08-31{]}|pwkv}, 5. 10.\pwindex{Goldmann, Paul 31.01.1865 – 25.09.1935@\textsc{Goldmann, Paul} (31.01.1865 – 25.09.1935), \emph{Schriftsteller, Journalist}!Kiautschou-Eindruecke. I. Wie man ankommt1898-10-05@\strich\emph{Kiautschou-Eindrücke. I. Wie man ankommt} {[}1898-10-05{]}|pwkv}, 6. 10.\pwindex{Goldmann, Paul 31.01.1865 – 25.09.1935@\textsc{Goldmann, Paul} (31.01.1865 – 25.09.1935), \emph{Schriftsteller, Journalist}!Kiautschou-Eindruecke. I. Wie man ankommt [zweiter Teil]1898-10-06@\strich\emph{Kiautschou-Eindrücke. I. Wie man ankommt [zweiter Teil]} {[}1898-10-06{]}|pwkv}, 8. 10.\pwindex{Goldmann, Paul 31.01.1865 – 25.09.1935@\textsc{Goldmann, Paul} (31.01.1865 – 25.09.1935), \emph{Schriftsteller, Journalist}!Kiautschou-Eindruecke. II. Tsintau1898-10-08@\strich\emph{Kiautschou-Eindrücke. II. Tsintau} {[}1898-10-08{]}|pwkv}, 9. 10.\pwindex{Goldmann, Paul 31.01.1865 – 25.09.1935@\textsc{Goldmann, Paul} (31.01.1865 – 25.09.1935), \emph{Schriftsteller, Journalist}!Kiautschou-Eindruecke. II. Tsintau [zweiter Teil]1898-10-09@\strich\emph{Kiautschou-Eindrücke. II. Tsintau [zweiter Teil]} {[}1898-10-09{]}|pwkv}, 16. 10.\pwindex{Goldmann, Paul 31.01.1865 – 25.09.1935@\textsc{Goldmann, Paul} (31.01.1865 – 25.09.1935), \emph{Schriftsteller, Journalist}!In Ostasien. Reiseskizzen. Im Golf von Pe-tschi-li1898-10-16@\strich\emph{In Ostasien. Reiseskizzen. Im Golf von Pe-tschi-li} {[}1898-10-16{]}|pwkv}, 18. 10.\pwindex{Goldmann, Paul 31.01.1865 – 25.09.1935@\textsc{Goldmann, Paul} (31.01.1865 – 25.09.1935), \emph{Schriftsteller, Journalist}!In Ostasien. Reiseskizzen. Im Golf von Pe-tschi-li [zweiter Teil]1898-10-18@\strich\emph{In Ostasien. Reiseskizzen. Im Golf von Pe-tschi-li [zweiter Teil]} {[}1898-10-18{]}|pwkv}, 30. 10.\pwindex{Goldmann, Paul 31.01.1865 – 25.09.1935@\textsc{Goldmann, Paul} (31.01.1865 – 25.09.1935), \emph{Schriftsteller, Journalist}!In Ostasien. Reiseskizzen. Von Tschifu nach Tientsin1898-10-30@\strich\emph{In Ostasien. Reiseskizzen. Von Tschifu nach Tientsin} {[}1898-10-30{]}|pwkv} und 31. 10. 1898\pwindex{Goldmann, Paul 31.01.1865 – 25.09.1935@\textsc{Goldmann, Paul} (31.01.1865 – 25.09.1935), \emph{Schriftsteller, Journalist}!In Ostasien. Reiseskizzen. Von Tschifu nach Tientsin [zweiter Teil]1898-10-31@\strich\emph{In Ostasien. Reiseskizzen. Von Tschifu nach Tientsin [zweiter Teil]} {[}1898-10-31{]}|pwkv}. Berichterstattungen gab es am 8. 6.\pwindex{Goldmann, Paul 31.01.1865 – 25.09.1935@\textsc{Goldmann, Paul} (31.01.1865 – 25.09.1935), \emph{Schriftsteller, Journalist}!Pest in Hongkong1898-06-08@\strich\emph{Die Pest in Hongkong} {[}1898-06-08{]}|pwkv}, 23. 6.\pwindex{Goldmann, Paul 31.01.1865 – 25.09.1935@\textsc{Goldmann, Paul} (31.01.1865 – 25.09.1935), \emph{Schriftsteller, Journalist}!Eine Unterredung mit dem Tao-tai Wang, dem Sekretaer des Vicekoenigs von Canton1898-06-23@\strich\emph{Eine Unterredung mit dem Tao-tai Wang, dem Sekretär des Vicekönigs von Canton} {[}1898-06-23{]}|pwkv}\pwindex{Goldmann, Paul 31.01.1865 – 25.09.1935@\textsc{Goldmann, Paul} (31.01.1865 – 25.09.1935), \emph{Schriftsteller, Journalist}!Eine Unterredung mit dem Tao-tai Wang, dem Sekretaer des Vicekoenigs von Canton [zweiter Teil]1898-06-23@\strich\emph{Eine Unterredung mit dem Tao-tai Wang, dem Sekretär des Vicekönigs von Canton [zweiter Teil]} {[}1898-06-23{]}|pwkv}, 21. 7.\pwindex{Goldmann, Paul 31.01.1865 – 25.09.1935@\textsc{Goldmann, Paul} (31.01.1865 – 25.09.1935), \emph{Schriftsteller, Journalist}!Beim Tao-tai Tsai von Shanghai1898-07-21@\strich\emph{Beim Tao-tai Tsai von Shanghai} {[}1898-07-21{]}|pwkv}, 23. 7.\pwindex{Goldmann, Paul 31.01.1865 – 25.09.1935@\textsc{Goldmann, Paul} (31.01.1865 – 25.09.1935), \emph{Schriftsteller, Journalist}!deutschen Militaer-Instruktoren in China1898-07-23@\strich\emph{Die deutschen Militär-Instruktoren in China} {[}1898-07-23{]}|pwkv}, 3. 8.\pwindex{Goldmann, Paul 31.01.1865 – 25.09.1935@\textsc{Goldmann, Paul} (31.01.1865 – 25.09.1935), \emph{Schriftsteller, Journalist}!Kapitel ueber chinesische Eisenbahnen. Chinesische Eisenbahnen und deutsche Versaeumnisse1898-08-03@\strich\emph{Ein Kapitel über chinesische Eisenbahnen. Chinesische Eisenbahnen und deutsche Versäumnisse} {[}1898-08-03{]}|pwkv}, 4. 8.\pwindex{Goldmann, Paul 31.01.1865 – 25.09.1935@\textsc{Goldmann, Paul} (31.01.1865 – 25.09.1935), \emph{Schriftsteller, Journalist}!Kapitel ueber chinesische Eisenbahnen. Die Bahn in Shanghai nach Wu-sung1898-08-04@\strich\emph{Ein Kapitel über chinesische Eisenbahnen. Die Bahn in Shanghai nach Wu-sung} {[}1898-08-04{]}|pwkv}, 17. 8.\pwindex{Goldmann, Paul 31.01.1865 – 25.09.1935@\textsc{Goldmann, Paul} (31.01.1865 – 25.09.1935), \emph{Schriftsteller, Journalist}!Chinesische Zeitungen1898-08-17@\strich\emph{Chinesische Zeitungen} {[}1898-08-17{]}|pwkv}, 25. 8.\pwindex{Goldmann, Paul 31.01.1865 – 25.09.1935@\textsc{Goldmann, Paul} (31.01.1865 – 25.09.1935), \emph{Schriftsteller, Journalist}!franzoesisch-chinesische Zwischenfall in Shanghai1898-08-25@\strich\emph{Der französisch-chinesische Zwischenfall in Shanghai} {[}1898-08-25{]}|pwkv}, 9. 9.\pwindex{Goldmann, Paul 31.01.1865 – 25.09.1935@\textsc{Goldmann, Paul} (31.01.1865 – 25.09.1935), \emph{Schriftsteller, Journalist}!franzoesisch-chinesische Zwischenfall in Shanghai1898-09-09@\strich\emph{Der französisch-chinesische Zwischenfall in Shanghai} {[}1898-09-09{]}|pwkv}, 23. 9.\pwindex{Goldmann, Paul 31.01.1865 – 25.09.1935@\textsc{Goldmann, Paul} (31.01.1865 – 25.09.1935), \emph{Schriftsteller, Journalist}!In Kiautschou. I1898-09-23@\strich\emph{In Kiautschou. I} {[}1898-09-23{]}|pwkv}, 24. 9.\pwindex{Goldmann, Paul 31.01.1865 – 25.09.1935@\textsc{Goldmann, Paul} (31.01.1865 – 25.09.1935), \emph{Schriftsteller, Journalist}!In Kiautschou. II1898-09-24@\strich\emph{In Kiautschou. II} {[}1898-09-24{]}|pwkv}, 25. 9.\pwindex{Goldmann, Paul 31.01.1865 – 25.09.1935@\textsc{Goldmann, Paul} (31.01.1865 – 25.09.1935), \emph{Schriftsteller, Journalist}!In Kiautschou. III1898-09-25@\strich\emph{In Kiautschou. III} {[}1898-09-25{]}|pwkv}, 26. 9.\pwindex{Goldmann, Paul 31.01.1865 – 25.09.1935@\textsc{Goldmann, Paul} (31.01.1865 – 25.09.1935), \emph{Schriftsteller, Journalist}!In Kiautschou. IV [letzter Teil]1898-09-26@\strich\emph{In Kiautschou. IV [letzter Teil]} {[}1898-09-26{]}|pwkv} und 25. 10. 1898\pwindex{Goldmann, Paul 31.01.1865 – 25.09.1935@\textsc{Goldmann, Paul} (31.01.1865 – 25.09.1935), \emph{Schriftsteller, Journalist}!General Tscheng-Ki-tong1898-10-25@\strich\emph{Der General Tscheng-Ki-tong} {[}1898-10-25{]}|pwkv}. XXXX (Nachträge aus November und Dezember 1899 sobald gesichtet, dann auch
                  in ref-Werte nachtragen: Feuilletons bei »Reisefeuilletons«, Berichterstattungen
                  bei »Berichterstattungen«, beide bei »Arbeiten«; in PMB Zugehörigkeit zur
                  Feiulletonreihe nicht vergessen)}}}\label{K_L02858-4h} verfolgſt. Du nennſt ſie »intereſſant« und
               ahnſt gewiß nicht, daß das ihre Verurtheilung iſt. Intereſſant iſt die Rubrik
               »Vermiſchtes« in den Zeitungen, die von einem wunderbaren Walfiſch-Fang berichtet
               oder vom tätowirten Indianer. Die unbeſchreibliche künſtleriſche Anſtrengung, die ich
               auf meine Arbeiten verwende, das Beſtreben, einfach, klar und doch maleriſch
               darzuſtellen, {\pb}kommt alſo nicht zum Ausdruck. Wenn
               ſelbſt Du es nicht ſiehſt, ſo beweiſt das, daß meine Arbeiten verfehlt ſind, was ich
               von Anfang an \strikeout{\textcolor{gray}{×}\-\textcolor{gray}{×}\-\textcolor{gray}{×}\-\textcolor{gray}{×}\-\textcolor{gray}{×}\-\textcolor{gray}{×}\-\textcolor{gray}{×}} geahnt habe. Es iſt ſehr bitter, liebſter Freund, intereſſant zu
               ſchreiben.\pend
           \pstart
           Mein Brief findet Dich hoffentlich in guter, froher Arbeit und in heller Stimmung.
               Denke Dir nur, welch’ ein \label{K_L02858-6v}\edtext{\textsc{Schemen}}{\lemma{\textnormal{\emph{Schemen}}}\Cendnote{\textnormal{Trugbild}}}\label{K_L02858-6h}{ }\strikeout{alle} alle Deine Leiden ſein müſſen, {\pb}wenn eine einzige \label{K_L02858-8v}\edtext{Reiſe}{\lemma{\textnormal{\emph{Reiſe}}}\Cendnote{\textnormal{siehe Paul Goldmann an Arthur Schnitzler, 16. 5. 1898}}}\label{K_L02858-8h} von Wien\oindex{Wien@\textbf{Wien}|pw} nach Salzburg\oindex{Salzburg@\textbf{Salzburg}|pw} ſie verblaſſen macht. Quäle Dich nicht und mache Dir
               einen frohen Winter!\pend
           \pstart
           Grüß’ mir den \textsc{Richard\pwindex{Beer-Hofmann, Richard 1866-07-11 – 1945-09-26@\textsc{Beer-Hofmann, Richard} (1866-07-11 – 1945-09-26), \emph{Schriftsteller}|pw}}! Ich \strikeout{h\textcolor{gray}{×}\-\textcolor{gray}{×}\-\textcolor{gray}{×}} freue mich, daß er das \label{K_L02858-11v}\edtext{dritte Capitel des »Götterliebling\pwindex{Beer-Hofmann, Richard 1866-07-11 – 1945-09-26@\textsc{Beer-Hofmann, Richard} (1866-07-11 – 1945-09-26), \emph{Schriftsteller}!Tod Georgs1900@\strich\emph{Der Tod Georgs} {[}1900{]}|pwv}«}{\lemma{\textnormal{\emph{dritte … »Götterliebling«}}}\Cendnote{\textnormal{Als Schnitzler\pwindex{Schnitzler, Arthur 15.05.1862 – 21.10.1931@\textsc{Schnitzler, Arthur} (15.05.1862 – 21.10.1931), \emph{Schriftsteller, Mediziner}|pwk} am 28. 7. 1898 in Salzburg\oindex{Salzburg@\textbf{Salzburg}|pwk} war, las ihm Beer-Hofmann\pwindex{Beer-Hofmann, Richard 1866-07-11 – 1945-09-26@\textsc{Beer-Hofmann, Richard} (1866-07-11 – 1945-09-26), \emph{Schriftsteller}|pwk} das dritte Kapitel des Götterliebling\pwindex{Beer-Hofmann, Richard 1866-07-11 – 1945-09-26@\textsc{Beer-Hofmann, Richard} (1866-07-11 – 1945-09-26), \emph{Schriftsteller}!Tod Georgs1900@\strich\emph{Der Tod Georgs} {[}1900{]}|pwkv}s vor. Die Erzählung\pwindex{Beer-Hofmann, Richard 1866-07-11 – 1945-09-26@\textsc{Beer-Hofmann, Richard} (1866-07-11 – 1945-09-26), \emph{Schriftsteller}!Tod Georgs1900@\strich\emph{Der Tod Georgs} {[}1900{]}|pwkv} erschien zuerst zwischen 4. 11. 1899 und 25. 11. 1899 als Fragment unter dem Titel \emph{Der Tod Georgs}\pwindex{Beer-Hofmann, Richard 1866-07-11 – 1945-09-26@\textsc{Beer-Hofmann, Richard} (1866-07-11 – 1945-09-26), \emph{Schriftsteller}!Tod Georgs. Fragment4.11.1899 – 25.11.1899@\strich\emph{Der Tod Georgs. Fragment} {[}4.11.1899 – 25.11.1899{]}|pwk} in der \emph{Zeit}\pwindex{Zeit. Wiener Wochenschrift1894 – 1904@\emph{Die Zeit. Wiener Wochenschrift} {[}1894 – 1904{]}|pwk}.}}}\label{K_L02858-11h} beendet hat. Nur fürchte ich, im vierten Capitel\pwindex{Beer-Hofmann, Richard 1866-07-11 – 1945-09-26@\textsc{Beer-Hofmann, Richard} (1866-07-11 – 1945-09-26), \emph{Schriftsteller}!Tod Georgs1900@\strich\emph{Der Tod Georgs} {[}1900{]}|pwv} wird der Held wieder einſchlafen
                  {\pb}und einige Jahrhundert Weltgeſchichte \strikeout{t\textcolor{gray}{r}} träumen, und das wird \substVorne{}\textsuperscript{\textcolor{gray}{wie}d\textcolor{gray}{er}}{\allowbreak}\substDazwischen{}noch\substHinten{} recht lang werden.\pend
           \pstart
           Man ſandte mir hierher einen \label{K_L02858-17v}\edtext{Artikel\pwindex{Lothar, Rudolf 23.2.1865 – 2.10.1943@\textsc{Lothar, Rudolf} (23.2.1865 – 2.10.1943), \emph{Schriftsteller, Journalist, Theaterdirektor}!Briefe an eine Dame1898-06@\strich\emph{Briefe an eine Dame} {[}1898-06{]}|pwv}}{\lemma{\textnormal{\emph{Artikel}}}\Cendnote{\textnormal{Rudolf Lothar\pwindex{Lothar, Rudolf 23.2.1865 – 2.10.1943@\textsc{Lothar, Rudolf} (23.2.1865 – 2.10.1943), \emph{Schriftsteller, Journalist, Theaterdirektor}|pwk}: \emph{Briefe an eine Dame}\pwindex{Lothar, Rudolf 23.2.1865 – 2.10.1943@\textsc{Lothar, Rudolf} (23.2.1865 – 2.10.1943), \emph{Schriftsteller, Journalist, Theaterdirektor}!Briefe an eine Dame1898-06@\strich\emph{Briefe an eine Dame} {[}1898-06{]}|pwk}. In: \emph{Die Wage. Eine Wiener Wochenschrift}\pwindex{Wage. Eine Wiener Wochenschrift1898-01-01 – 1925@\emph{Die Wage. Eine Wiener Wochenschrift} {[}1898-01-01 – 1925{]}|pwk}, Jg. 1, Nr. 26,
                        25. 6. 1898, S. 439–440.}}}\label{K_L02858-17h} von
                  \textsc{Rudolf Lothar\pwindex{Lothar, Rudolf 23.2.1865 – 2.10.1943@\textsc{Lothar, Rudolf} (23.2.1865 – 2.10.1943), \emph{Schriftsteller, Journalist, Theaterdirektor}|pw}} über Dich in der »Wage\pwindex{Wage. Eine Wiener Wochenschrift1898-01-01 – 1925@\emph{Die Wage. Eine Wiener Wochenschrift} {[}1898-01-01 – 1925{]}|pw}«. Wenn Du den Autor\pwindex{Lothar, Rudolf 23.2.1865 – 2.10.1943@\textsc{Lothar, Rudolf} (23.2.1865 – 2.10.1943), \emph{Schriftsteller, Journalist, Theaterdirektor}|pwv} ſiehſt, ſo grüße ihn von
               mir und ſage ihm, meines Wiſſens ſei noch nie über Dich ein ähnlicher Blödſinn
               geſchrieben worden. Auch erfahre ich daraus, daß \strikeout{D} Du
                  {\pb}durch \textsc{Rudolf Lothar\pwindex{Lothar, Rudolf 23.2.1865 – 2.10.1943@\textsc{Lothar, Rudolf} (23.2.1865 – 2.10.1943), \emph{Schriftsteller, Journalist, Theaterdirektor}|pw}} zum Schreiben ermuntert worden biſt. Jetzt weiß ich, warum Du ein Dichter
               biſt!\pend
           \pstart
           Grüß’ Dich Gott, liebſter Freund!\pend
           \pstart
           Dein treuer {\\[\baselineskip]}\spacefill\mbox{Paul Goldmann}\pend
           \leftskip=0em{}\pstart
           \noindent{}Viele Grüße an Deine Freundin\pwindex{Reinhard, Marie 1871-03-13 – 1899-03-18@\textsc{Reinhard, Marie} (1871-03-13 – 1899-03-18), \emph{Gesangspädagogin}|pwv}!\pend
           
         
         \endnumbering\mylabel{h}\end{ledgroupsized}  \newcommand{\dateiname}{L02858}\newcommand{\titel}{Paul Goldmann an Arthur Schnitzler, 25. 9. 1898}\newcommand{\editorInnen}{Martin Anton Müller und Laura Untner}%% latex-leseansicht-abspann.tex
%% Abspann für die Leseansicht.
%% Der Schalter \ifkorrekturansicht ist bereits durch den Vorspann gesetzt.

%% latex-abspann.tex
%% Gemeinsamer Abspann für Korrekturansicht und Leseansicht.
%% Setzt den Schalter \ifkorrekturansicht voraus (gesetzt in den
%% einbindenden Dateien latex-korrekturansicht-abspann.tex bzw.
%% latex-leseansicht-abspann.tex).
%% ---------------------------------------------------------------

\normalsize

% Das esempio-Environment wird nur in der Leseansicht benötigt
\ifkorrekturansicht\else
\newenvironment{esempio}[3]%
{
    \vspace{1.5ex}
    \rlap{\underline{#1}}
    \par
    \setlength{\parindent}{0cm}
    \nopagebreak
    \leftskip=#2cm
    \rightskip=#3cm
}
{
    \par
}
\fi

\doendnotes{C}
\bigskip
\vfill

\clearpage

\footnotesize

\ifkorrekturansicht
  \lohead{\textsc{register}}
\fi

% theindex-Environment neu definieren ohne reledmac
\makeatletter
\renewenvironment{theindex}{%
  \ifkorrekturansicht
    \section*{\indexname}%
  \else
    \subsubsection*{Index der erwähnten Entitäten}%
  \fi
  \setlength{\parindent}{0pt}%
  \setlength{\parskip}{0pt plus 0.3pt}%
  \let\item\@idxitem
}{%
  \ifkorrekturansicht\clearpage\fi
}
\makeatother

\IfFileExists{\jobname-pw.ind}{\input{\jobname-pw.ind}}{}

% Quellenangabe nur in der Leseansicht
\ifkorrekturansicht\else
% Fallback-Definitionen, falls die .tex-Datei \titel etc. nicht gesetzt hat
\providecommand{\titel}{}
\providecommand{\editorInnen}{}
\providecommand{\dateiname}{\jobname}

\vspace{3cm}

\vfill

\footnotesize
\textsc{Quelle}: \titel. Herausgegeben von {\editorInnen}. In: \emph{Arthur Schnitzler: Briefwechsel mit Autorinnen und Autoren}.
 Digitale Edition, https://schnitzler-briefe.acdh.oeaw.ac.at/{\dateiname}.html (Stand \today)
\fi

\end{document}


      