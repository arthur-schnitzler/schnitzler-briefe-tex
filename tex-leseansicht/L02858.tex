%% latex-leseansicht-vorspann.tex
%% Vorspann für die Leseansicht.
%% Lädt die gemeinsame Datei latex-vorspann.tex mit nicht gesetztem Schalter.

\newif\ifkorrekturansicht
\korrekturansichtfalse

\input{../tex-inputs/latex-vorspann}


\section[ Paul Goldmann an Arthur Schnitzler, 25. 9. 1898]{L02858 Paul Goldmann an Arthur Schnitzler,  25. 9. 1898}
\nopagebreak\mylabel{L02858v}
\rehead{ }\normalsize\beginnumbering\briefempfaengerindex{Schnitzler, Arthur@\textsc{Schnitzler, Arthur}!zzzGoldmann, Paul@\emph{von Paul Goldmann}!1898-09-251@{25. 9. 1898}|(be}
\toendnotes[C]{\smallbreak\pagebreak[2]}
\correspDesc{Versand  durch Paul Goldmann am 25. 9. 1898 in Tianjin
\newline{}Erhalt  durch Arthur Schnitzler im Zeitraum [1. 11. 1898
                  – 15. 12. 1898?] in Wien}\toendnotes[C]{\smallbreak}
\Standort{DLA, A:Schnitzler, HS.NZ85.1.3168.}
\physDesc{Brief, 2 Blätter, 7 Seiten, 2368 Zeichen
\newline{}Handschrift: blaue Tinte, deutsche Kurrent
\newline{}Schnitzler: 1) mit Bleistift das Jahr »98« vermerkt  2) mit rotem Buntstift zwei Unterstreichungen}\toendnotes[C]{\smallbreak}
\pstart
           \centering{}{\pb}\textsc{\textcolor{gray}{\textbf{Astor House Hôtel, L\textsuperscript{td}.}}}\oindex{Astor House Hotel [Tianjin]@\textbf{Astor House Hotel [Tianjin]}, \emph{Hotel}|pw}\pend
           
\pstart
           \raggedleft{}\textcolor{gray}{\textbf{Tientsin\oindex{Tianjin@\textbf{Tianjin}|pw},}}{ }25. September \textcolor{gray}{\textbf{189}}8\pend
           
\pstart\center{}Mein lieber Freund,\pend\vspace{0.5em}
\pstart
           Ich bin jetzt{ }ſehr außerhalb der Poſt-Verbindungen u. habe daher erſt dieſer Tage
               Deinen lieben Brief aus \label{K_L02858-1v}\edtext{\textsc{Salzburg\oindex{Salzburg@\textbf{Salzburg}, \emph{Verwaltungsgebiet}|pw}}}{\lemma{\textnormal{\emph{Salzburg}}}\Cendnote{\textnormal{Siehe A. S.: \emph{Tagebuch}, 28. 7. 1898.
               }}}\label{K_L02858-1} vom 28. Juli erhalten. Inzwiſchen biſt Du ja
               längſt glücklich heimgekehrt; und wenn Du meinen Brief erhältſt, iſt wohl auch{ }ſchon
               die \label{K_L02858-2v}\edtext{\begin{otherlanguage}{french}\textsc{Première}\end{otherlanguage}{ }Deines neuen Stück\pwindex{Schnitzler, Arthur 15.\,5.\,1862 Wien – 21.\,10.\,1931 ebd.@\textsc{Schnitzler, Arthur} (15.\,5.\,1862 Wien – 21.\,10.\,1931 ebd.), \emph{Schriftsteller, Mediziner}!Vermächtnis. Schauspiel in drei Akten@\strich\emph{Das Vermächtnis. Schauspiel in drei Akten}|pwv}es}{\lemma{\textnormal{\emph{Première … Stückes}}}\Cendnote{\textnormal{\emph{Das Vermächtnis}\pwindex{Schnitzler, Arthur 15.\,5.\,1862 Wien – 21.\,10.\,1931 ebd.@\textsc{Schnitzler, Arthur} (15.\,5.\,1862 Wien – 21.\,10.\,1931 ebd.), \emph{Schriftsteller, Mediziner}!Vermächtnis. Schauspiel in drei Akten@\strich\emph{Das Vermächtnis. Schauspiel in drei Akten}|pwk} war am 8. 10. 1898 am Deutschen Theater\oindex{Deutsches Theater Berlin@\textbf{Deutsches Theater Berlin}, \emph{Theater}|pwk} in Berlin\oindex{Berlin@\textbf{Berlin}, \emph{Hauptstadt}|pwk} uraufgeführt worden.}}}\label{K_L02858-2} vorüber und Du biſt um einen
               neuen Erfolg reicher.\pend
           
\pstart
           {\pb}Es iſt heut wieder
               ein Tag, wo ich unſägliches Heimweh habe. Manchmal erwache ich wie aus einen Traume
               und frage\strikeout{,} mich, was ich denn eigentlich hier in
               dieſem Lande\oindex{China@\textbf{China}|pwv} mache? Noch dazu
               bin ich{ }ſeit einigen Wochen recht elend. Die \label{K_L02858-3v}\edtext{\textsc{Dysenterie}}{\lemma{\textnormal{\emph{Dysenterie}}}\Cendnote{\textnormal{Darmentzündung}}}\label{K_L02858-3} iſt mir in den
               Leib gefahren\substVorne{}\textsuperscript{\textcolor{gray}{,}}\substDazwischen{}{ }und\substHinten{} geht natürlich nicht wieder weg. Das iſt eine{ }ſchlimme Geſchichte. Allein im
               fremden Lande\oindex{China@\textbf{China}|pwv} und auch noch
               krank dazu und die Heimath{ }ſo weit! {\dotsfive}\pend
           
\pstart
           {\pb}Ich danke Dir von Herzen für die Aufmerkſamkeit,
               mit der Du meine \label{K_L02858-4v}\edtext{Arbeiten\pwindex{Goldmann, Paul 31.\,1.\,1865 Breslau – 25.\,9.\,1935 Wien@\textsc{Goldmann, Paul} (31.\,1.\,1865 Breslau – 25.\,9.\,1935 Wien), \emph{Schriftsteller, Journalist}!In Ostasien. Reiseskizzen@\strich\emph{In Ostasien. Reiseskizzen}|pwv}}{\lemma{\textnormal{\emph{Arbeiten}}}\Cendnote{\textnormal{Schnitzler dürfte regelmäßig die \emph{Frankfurter Zeitung}\pwindex{Frankfurter Zeitung@\emph{Frankfurter Zeitung}|pwk} gelesen haben, in der Goldmanns\pwindex{Goldmann, Paul 31.\,1.\,1865 Breslau – 25.\,9.\,1935 Wien@\textsc{Goldmann, Paul} (31.\,1.\,1865 Breslau – 25.\,9.\,1935 Wien), \emph{Schriftsteller, Journalist}|pwk}{ }Feuilletons\pwindex{Goldmann, Paul 31.\,1.\,1865 Breslau – 25.\,9.\,1935 Wien@\textsc{Goldmann, Paul} (31.\,1.\,1865 Breslau – 25.\,9.\,1935 Wien), \emph{Schriftsteller, Journalist}!In Ostasien. Reiseskizzen@\strich\emph{In Ostasien. Reiseskizzen}|pwkv} (unter Angabe des vollen Namens) mit dem
                  Titel \emph{In Ostasien. Reiseskizzen}\pwindex{Goldmann, Paul 31.\,1.\,1865 Breslau – 25.\,9.\,1935 Wien@\textsc{Goldmann, Paul} (31.\,1.\,1865 Breslau – 25.\,9.\,1935 Wien), \emph{Schriftsteller, Journalist}!In Ostasien. Reiseskizzen@\strich\emph{In Ostasien. Reiseskizzen}|pwk} erschienen.
                  Die Texte erschienen an den folgenden Tagen: 24. 4. 1898\pwindex{Goldmann, Paul 31.\,1.\,1865 Breslau – 25.\,9.\,1935 Wien@\textsc{Goldmann, Paul} (31.\,1.\,1865 Breslau – 25.\,9.\,1935 Wien), \emph{Schriftsteller, Journalist}!Nach Ostasien. Reiseskizzen@\strich\emph{Nach Ostasien. Reiseskizzen}|pwkv}, 1. 5. 1898\pwindex{Goldmann, Paul 31.\,1.\,1865 Breslau – 25.\,9.\,1935 Wien@\textsc{Goldmann, Paul} (31.\,1.\,1865 Breslau – 25.\,9.\,1935 Wien), \emph{Schriftsteller, Journalist}!Nach Ostasien. Reiseskizzen@\strich\emph{Nach Ostasien. Reiseskizzen}|pwkv}, 19. 5. 1898\pwindex{Goldmann, Paul 31.\,1.\,1865 Breslau – 25.\,9.\,1935 Wien@\textsc{Goldmann, Paul} (31.\,1.\,1865 Breslau – 25.\,9.\,1935 Wien), \emph{Schriftsteller, Journalist}!Nach Ostasien. Reiseskizzen@\strich\emph{Nach Ostasien. Reiseskizzen}|pwkv}, 22. 5. 1898\pwindex{Goldmann, Paul 31.\,1.\,1865 Breslau – 25.\,9.\,1935 Wien@\textsc{Goldmann, Paul} (31.\,1.\,1865 Breslau – 25.\,9.\,1935 Wien), \emph{Schriftsteller, Journalist}!Nach Ostasien. Reiseskizzen. Eine Nacht und ein Morgen in Colombo@\strich\emph{Nach Ostasien. Reiseskizzen. Eine Nacht und ein Morgen in Colombo}|pwkv}, 12. 6. 1898\pwindex{Goldmann, Paul 31.\,1.\,1865 Breslau – 25.\,9.\,1935 Wien@\textsc{Goldmann, Paul} (31.\,1.\,1865 Breslau – 25.\,9.\,1935 Wien), \emph{Schriftsteller, Journalist}!In Ostasien. Reiseskizzen. Singapore@\strich\emph{In Ostasien. Reiseskizzen. Singapore}|pwkv}, 16. 6. 1898\pwindex{Goldmann, Paul 31.\,1.\,1865 Breslau – 25.\,9.\,1935 Wien@\textsc{Goldmann, Paul} (31.\,1.\,1865 Breslau – 25.\,9.\,1935 Wien), \emph{Schriftsteller, Journalist}!In Ostasien. Reiseskizzen. Hongkong@\strich\emph{In Ostasien. Reiseskizzen. Hongkong}|pwkv}, 17. 6. 1898\pwindex{Goldmann, Paul 31.\,1.\,1865 Breslau – 25.\,9.\,1935 Wien@\textsc{Goldmann, Paul} (31.\,1.\,1865 Breslau – 25.\,9.\,1935 Wien), \emph{Schriftsteller, Journalist}!In Ostasien. Reiseskizzen. Hongkong [zweiter Teil]@\strich\emph{In Ostasien. Reiseskizzen. Hongkong [zweiter Teil]}|pwkv}, 23. 6. 1898\pwindex{Goldmann, Paul 31.\,1.\,1865 Breslau – 25.\,9.\,1935 Wien@\textsc{Goldmann, Paul} (31.\,1.\,1865 Breslau – 25.\,9.\,1935 Wien), \emph{Schriftsteller, Journalist}!In Ostasien. Reiseskizzen. Auf dem Perlfluß nach Canton-Shameen@\strich\emph{In Ostasien. Reiseskizzen. Auf dem Perlfluß nach Canton-Shameen}|pwkv}, 24. 6. 1898\pwindex{Goldmann, Paul 31.\,1.\,1865 Breslau – 25.\,9.\,1935 Wien@\textsc{Goldmann, Paul} (31.\,1.\,1865 Breslau – 25.\,9.\,1935 Wien), \emph{Schriftsteller, Journalist}!In Ostasien. Reiseskizzen. Auf dem Perlfluß nach Canton-Shameen [zweiter Teil]@\strich\emph{In Ostasien. Reiseskizzen. Auf dem Perlfluß nach Canton-Shameen [zweiter Teil]}|pwkv}, 29. 6. 1898\pwindex{Goldmann, Paul 31.\,1.\,1865 Breslau – 25.\,9.\,1935 Wien@\textsc{Goldmann, Paul} (31.\,1.\,1865 Breslau – 25.\,9.\,1935 Wien), \emph{Schriftsteller, Journalist}!In Ostasien. Reiseskizzen. Canton@\strich\emph{In Ostasien. Reiseskizzen. Canton{\rufezeichen}}|pwkv}, 30. 6. 1898\pwindex{Goldmann, Paul 31.\,1.\,1865 Breslau – 25.\,9.\,1935 Wien@\textsc{Goldmann, Paul} (31.\,1.\,1865 Breslau – 25.\,9.\,1935 Wien), \emph{Schriftsteller, Journalist}!In Ostasien. Reiseskizzen. Canton [zweiter Teil]@\strich\emph{In Ostasien. Reiseskizzen. Canton{\rufezeichen} [zweiter Teil]}|pwkv}, 14. 7. 1898\pwindex{Goldmann, Paul 31.\,1.\,1865 Breslau – 25.\,9.\,1935 Wien@\textsc{Goldmann, Paul} (31.\,1.\,1865 Breslau – 25.\,9.\,1935 Wien), \emph{Schriftsteller, Journalist}!In Ostasien. Reiseskizzen. Von Hongkong nach Shanghai@\strich\emph{In Ostasien. Reiseskizzen. Von Hongkong nach Shanghai}|pwkv}, 15. 7. 1898\pwindex{Goldmann, Paul 31.\,1.\,1865 Breslau – 25.\,9.\,1935 Wien@\textsc{Goldmann, Paul} (31.\,1.\,1865 Breslau – 25.\,9.\,1935 Wien), \emph{Schriftsteller, Journalist}!In Ostasien. Reiseskizzen. Von Hongkong nach Shanghai [zweiter Teil]@\strich\emph{In Ostasien. Reiseskizzen. Von Hongkong nach Shanghai [zweiter Teil]}|pwkv}, 24. 7. 1898\pwindex{Goldmann, Paul 31.\,1.\,1865 Breslau – 25.\,9.\,1935 Wien@\textsc{Goldmann, Paul} (31.\,1.\,1865 Breslau – 25.\,9.\,1935 Wien), \emph{Schriftsteller, Journalist}!In Ostasien. Reiseskizzen. Shanghai@\strich\emph{In Ostasien. Reiseskizzen. Shanghai}|pwkv}, 26. 7. 1898\pwindex{Goldmann, Paul 31.\,1.\,1865 Breslau – 25.\,9.\,1935 Wien@\textsc{Goldmann, Paul} (31.\,1.\,1865 Breslau – 25.\,9.\,1935 Wien), \emph{Schriftsteller, Journalist}!In Ostasien. Reiseskizzen. Shanghai [zweiter Teil]@\strich\emph{In Ostasien. Reiseskizzen. Shanghai [zweiter Teil]}|pwkv}, 7. 8. 1898\pwindex{Goldmann, Paul 31.\,1.\,1865 Breslau – 25.\,9.\,1935 Wien@\textsc{Goldmann, Paul} (31.\,1.\,1865 Breslau – 25.\,9.\,1935 Wien), \emph{Schriftsteller, Journalist}!In Ostasien. Reiseskizzen. Chinesisches Nachtleben@\strich\emph{In Ostasien. Reiseskizzen. Chinesisches Nachtleben}|pwkv}, 9. 8. 1898\pwindex{Goldmann, Paul 31.\,1.\,1865 Breslau – 25.\,9.\,1935 Wien@\textsc{Goldmann, Paul} (31.\,1.\,1865 Breslau – 25.\,9.\,1935 Wien), \emph{Schriftsteller, Journalist}!In Ostasien. Reiseskizzen. Chinesisches Nachtleben [zweiter Teil]@\strich\emph{In Ostasien. Reiseskizzen. Chinesisches Nachtleben [zweiter Teil]}|pwkv}, 21. 8. 1898\pwindex{Goldmann, Paul 31.\,1.\,1865 Breslau – 25.\,9.\,1935 Wien@\textsc{Goldmann, Paul} (31.\,1.\,1865 Breslau – 25.\,9.\,1935 Wien), \emph{Schriftsteller, Journalist}!In Ostasien. Reiseskizzen. Auf dem Yang-tse-Kiang@\strich\emph{In Ostasien. Reiseskizzen. Auf dem Yang-tse-Kiang}|pwkv}, 22. 8. 1898\pwindex{Goldmann, Paul 31.\,1.\,1865 Breslau – 25.\,9.\,1935 Wien@\textsc{Goldmann, Paul} (31.\,1.\,1865 Breslau – 25.\,9.\,1935 Wien), \emph{Schriftsteller, Journalist}!In Ostasien. Reiseskizzen. Auf dem Yang-tse-Kiang [zweiter Teil]@\strich\emph{In Ostasien. Reiseskizzen. Auf dem Yang-tse-Kiang [zweiter Teil]}|pwkv}, 28. 8. 1898\pwindex{Goldmann, Paul 31.\,1.\,1865 Breslau – 25.\,9.\,1935 Wien@\textsc{Goldmann, Paul} (31.\,1.\,1865 Breslau – 25.\,9.\,1935 Wien), \emph{Schriftsteller, Journalist}!In Ostasien. Reiseskizzen. Hankow@\strich\emph{In Ostasien. Reiseskizzen. Hankow}|pwkv}, 30. 8. 1898\pwindex{Goldmann, Paul 31.\,1.\,1865 Breslau – 25.\,9.\,1935 Wien@\textsc{Goldmann, Paul} (31.\,1.\,1865 Breslau – 25.\,9.\,1935 Wien), \emph{Schriftsteller, Journalist}!In Ostasien. Reiseskizzen. Wu-tschang@\strich\emph{In Ostasien. Reiseskizzen. Wu-tschang}|pwkv}, 31. 8. 1898\pwindex{Goldmann, Paul 31.\,1.\,1865 Breslau – 25.\,9.\,1935 Wien@\textsc{Goldmann, Paul} (31.\,1.\,1865 Breslau – 25.\,9.\,1935 Wien), \emph{Schriftsteller, Journalist}!In Ostasien. Reiseskizzen. Wu-tschang [zweiter Teil]@\strich\emph{In Ostasien. Reiseskizzen. Wu-tschang [zweiter Teil]}|pwkv}, 5. 10. 1898\pwindex{Goldmann, Paul 31.\,1.\,1865 Breslau – 25.\,9.\,1935 Wien@\textsc{Goldmann, Paul} (31.\,1.\,1865 Breslau – 25.\,9.\,1935 Wien), \emph{Schriftsteller, Journalist}!Kiautschou-Eindrücke. I. Wie man ankommt@\strich\emph{Kiautschou-Eindrücke. I. Wie man ankommt}|pwkv}, 6. 10. 1898\pwindex{Goldmann, Paul 31.\,1.\,1865 Breslau – 25.\,9.\,1935 Wien@\textsc{Goldmann, Paul} (31.\,1.\,1865 Breslau – 25.\,9.\,1935 Wien), \emph{Schriftsteller, Journalist}!Kiautschou-Eindrücke. I. Wie man ankommt [zweiter Teil]@\strich\emph{Kiautschou-Eindrücke. I. Wie man ankommt [zweiter Teil]}|pwkv}, 8. 10. 1898\pwindex{Goldmann, Paul 31.\,1.\,1865 Breslau – 25.\,9.\,1935 Wien@\textsc{Goldmann, Paul} (31.\,1.\,1865 Breslau – 25.\,9.\,1935 Wien), \emph{Schriftsteller, Journalist}!Kiautschou-Eindrücke. II. Tsintau@\strich\emph{Kiautschou-Eindrücke. II. Tsintau}|pwkv}, 9. 10. 1898\pwindex{Goldmann, Paul 31.\,1.\,1865 Breslau – 25.\,9.\,1935 Wien@\textsc{Goldmann, Paul} (31.\,1.\,1865 Breslau – 25.\,9.\,1935 Wien), \emph{Schriftsteller, Journalist}!Kiautschou-Eindrücke. II. Tsintau [zweiter Teil]@\strich\emph{Kiautschou-Eindrücke. II. Tsintau [zweiter Teil]}|pwkv}, 16. 10. 1898\pwindex{Goldmann, Paul 31.\,1.\,1865 Breslau – 25.\,9.\,1935 Wien@\textsc{Goldmann, Paul} (31.\,1.\,1865 Breslau – 25.\,9.\,1935 Wien), \emph{Schriftsteller, Journalist}!In Ostasien. Reiseskizzen. Im Golf von Pe-tschi-li@\strich\emph{In Ostasien. Reiseskizzen. Im Golf von Pe-tschi-li}|pwkv}, 18. 10. 1898\pwindex{Goldmann, Paul 31.\,1.\,1865 Breslau – 25.\,9.\,1935 Wien@\textsc{Goldmann, Paul} (31.\,1.\,1865 Breslau – 25.\,9.\,1935 Wien), \emph{Schriftsteller, Journalist}!In Ostasien. Reiseskizzen. Im Golf von Pe-tschi-li [zweiter Teil]@\strich\emph{In Ostasien. Reiseskizzen. Im Golf von Pe-tschi-li [zweiter Teil]}|pwkv}, 30. 10. 1898\pwindex{Goldmann, Paul 31.\,1.\,1865 Breslau – 25.\,9.\,1935 Wien@\textsc{Goldmann, Paul} (31.\,1.\,1865 Breslau – 25.\,9.\,1935 Wien), \emph{Schriftsteller, Journalist}!In Ostasien. Reiseskizzen. Von Tschifu nach Tientsin@\strich\emph{In Ostasien. Reiseskizzen. Von Tschifu nach Tientsin}|pwkv}, 31. 10. 1898\pwindex{Goldmann, Paul 31.\,1.\,1865 Breslau – 25.\,9.\,1935 Wien@\textsc{Goldmann, Paul} (31.\,1.\,1865 Breslau – 25.\,9.\,1935 Wien), \emph{Schriftsteller, Journalist}!In Ostasien. Reiseskizzen. Von Tschifu nach Tientsin [zweiter Teil]@\strich\emph{In Ostasien. Reiseskizzen. Von Tschifu nach Tientsin [zweiter Teil]}|pwkv}, 13. 11. 1898\pwindex{Goldmann, Paul 31.\,1.\,1865 Breslau – 25.\,9.\,1935 Wien@\textsc{Goldmann, Paul} (31.\,1.\,1865 Breslau – 25.\,9.\,1935 Wien), \emph{Schriftsteller, Journalist}!In Ostasien. Reiseskizzen. In Tsientsin@\strich\emph{In Ostasien. Reiseskizzen. In Tsientsin}|pwkv}, 14. 11. 1898\pwindex{Goldmann, Paul 31.\,1.\,1865 Breslau – 25.\,9.\,1935 Wien@\textsc{Goldmann, Paul} (31.\,1.\,1865 Breslau – 25.\,9.\,1935 Wien), \emph{Schriftsteller, Journalist}!In Ostasien. Reiseskizzen. Tsientsin (Fortsetzung)@\strich\emph{In Ostasien. Reiseskizzen. Tsientsin (Fortsetzung)}|pwkv}, 15. 11. 1898\pwindex{Goldmann, Paul 31.\,1.\,1865 Breslau – 25.\,9.\,1935 Wien@\textsc{Goldmann, Paul} (31.\,1.\,1865 Breslau – 25.\,9.\,1935 Wien), \emph{Schriftsteller, Journalist}!In Ostasien. Reiseskizzen. Tsientsin (Schluß)@\strich\emph{In Ostasien. Reiseskizzen. Tsientsin (Schluß)}|pwkv}, 18. 12. 1898\pwindex{Goldmann, Paul 31.\,1.\,1865 Breslau – 25.\,9.\,1935 Wien@\textsc{Goldmann, Paul} (31.\,1.\,1865 Breslau – 25.\,9.\,1935 Wien), \emph{Schriftsteller, Journalist}!In Ostasien. Reiseskizzen. Anlage der Stadt Peking@\strich\emph{In Ostasien. Reiseskizzen. Anlage der Stadt Peking}|pwkv}, 20. 12. 1898\pwindex{Goldmann, Paul 31.\,1.\,1865 Breslau – 25.\,9.\,1935 Wien@\textsc{Goldmann, Paul} (31.\,1.\,1865 Breslau – 25.\,9.\,1935 Wien), \emph{Schriftsteller, Journalist}!In Ostasien. Reiseskizzen. Anlage der Stadt Peking (Schluß)@\strich\emph{In Ostasien. Reiseskizzen. Anlage der Stadt Peking (Schluß)}|pwkv}, 25. 12. 1898\pwindex{Goldmann, Paul 31.\,1.\,1865 Breslau – 25.\,9.\,1935 Wien@\textsc{Goldmann, Paul} (31.\,1.\,1865 Breslau – 25.\,9.\,1935 Wien), \emph{Schriftsteller, Journalist}!In Ostasien. Reiseskizzen. Straßenleben in Peking@\strich\emph{In Ostasien. Reiseskizzen. Straßenleben in Peking}|pwkv} und am 28. 12. 1898\pwindex{Goldmann, Paul 31.\,1.\,1865 Breslau – 25.\,9.\,1935 Wien@\textsc{Goldmann, Paul} (31.\,1.\,1865 Breslau – 25.\,9.\,1935 Wien), \emph{Schriftsteller, Journalist}!In Ostasien. Reiseskizzen. Straßenleben in Peking (Schluß)@\strich\emph{In Ostasien. Reiseskizzen. Straßenleben in Peking (Schluß)}|pwkv}. Am 30. 4. 1899\pwindex{Goldmann, Paul 31.\,1.\,1865 Breslau – 25.\,9.\,1935 Wien@\textsc{Goldmann, Paul} (31.\,1.\,1865 Breslau – 25.\,9.\,1935 Wien), \emph{Schriftsteller, Journalist}!Heimkehr@\strich\emph{Heimkehr}|pwkv} erschien mit \emph{Heimkehr}\pwindex{Goldmann, Paul 31.\,1.\,1865 Breslau – 25.\,9.\,1935 Wien@\textsc{Goldmann, Paul} (31.\,1.\,1865 Breslau – 25.\,9.\,1935 Wien), \emph{Schriftsteller, Journalist}!Heimkehr@\strich\emph{Heimkehr}|pwk} noch ein
                  Schlussartikel, der womöglich bereits für die Buchausgabe der Feuilletons – \emph{Ein Sommer in China}\pwindex{Goldmann, Paul 31.\,1.\,1865 Breslau – 25.\,9.\,1935 Wien@\textsc{Goldmann, Paul} (31.\,1.\,1865 Breslau – 25.\,9.\,1935 Wien), \emph{Schriftsteller, Journalist}!Sommer in China. Reisebilder@\strich\emph{Ein Sommer in China. Reisebilder}|pwk} – verfasst war. Zusätzlich zu den
                  allgemeineren Reiseschilderungen
                  erschienen tagesaktuelle Berichterstattungen\pwindex{Goldmann, Paul 31.\,1.\,1865 Breslau – 25.\,9.\,1935 Wien@\textsc{Goldmann, Paul} (31.\,1.\,1865 Breslau – 25.\,9.\,1935 Wien), \emph{Schriftsteller, Journalist}!Pest in Hongkong@\strich\emph{Die Pest in Hongkong}|pwkv}\pwindex{Goldmann, Paul 31.\,1.\,1865 Breslau – 25.\,9.\,1935 Wien@\textsc{Goldmann, Paul} (31.\,1.\,1865 Breslau – 25.\,9.\,1935 Wien), \emph{Schriftsteller, Journalist}!Eine Unterredung mit dem Tao-tai Wang, dem Sekretär des Vicekönigs von Canton@\strich\emph{Eine Unterredung mit dem Tao-tai Wang, dem Sekretär des Vicekönigs von Canton}|pwkv}\pwindex{Goldmann, Paul 31.\,1.\,1865 Breslau – 25.\,9.\,1935 Wien@\textsc{Goldmann, Paul} (31.\,1.\,1865 Breslau – 25.\,9.\,1935 Wien), \emph{Schriftsteller, Journalist}!Eine Unterredung mit dem Tao-tai Wang, dem Sekretär des Vicekönigs von Canton [zweiter Teil]@\strich\emph{Eine Unterredung mit dem Tao-tai Wang, dem Sekretär des Vicekönigs von Canton [zweiter Teil]}|pwkv}\pwindex{Goldmann, Paul 31.\,1.\,1865 Breslau – 25.\,9.\,1935 Wien@\textsc{Goldmann, Paul} (31.\,1.\,1865 Breslau – 25.\,9.\,1935 Wien), \emph{Schriftsteller, Journalist}!Beim Tao-tai Tsai von Shanghai@\strich\emph{Beim Tao-tai Tsai von Shanghai}|pwkv}\pwindex{Goldmann, Paul 31.\,1.\,1865 Breslau – 25.\,9.\,1935 Wien@\textsc{Goldmann, Paul} (31.\,1.\,1865 Breslau – 25.\,9.\,1935 Wien), \emph{Schriftsteller, Journalist}!deutschen Militär-Instruktoren in China@\strich\emph{Die deutschen Militär-Instruktoren in China}|pwkv}\pwindex{Goldmann, Paul 31.\,1.\,1865 Breslau – 25.\,9.\,1935 Wien@\textsc{Goldmann, Paul} (31.\,1.\,1865 Breslau – 25.\,9.\,1935 Wien), \emph{Schriftsteller, Journalist}!Kapitel über chinesische Eisenbahnen. Chinesische Eisenbahnen und deutsche Versäumnisse@\strich\emph{Ein Kapitel über chinesische Eisenbahnen. Chinesische Eisenbahnen und deutsche Versäumnisse}|pwkv}\pwindex{Goldmann, Paul 31.\,1.\,1865 Breslau – 25.\,9.\,1935 Wien@\textsc{Goldmann, Paul} (31.\,1.\,1865 Breslau – 25.\,9.\,1935 Wien), \emph{Schriftsteller, Journalist}!Kapitel über chinesische Eisenbahnen. Die Bahn in Shanghai nach Wu-sung@\strich\emph{Ein Kapitel über chinesische Eisenbahnen. Die Bahn in Shanghai nach Wu-sung}|pwkv}\pwindex{Goldmann, Paul 31.\,1.\,1865 Breslau – 25.\,9.\,1935 Wien@\textsc{Goldmann, Paul} (31.\,1.\,1865 Breslau – 25.\,9.\,1935 Wien), \emph{Schriftsteller, Journalist}!Chinesische Zeitungen@\strich\emph{Chinesische Zeitungen}|pwkv}\pwindex{Goldmann, Paul 31.\,1.\,1865 Breslau – 25.\,9.\,1935 Wien@\textsc{Goldmann, Paul} (31.\,1.\,1865 Breslau – 25.\,9.\,1935 Wien), \emph{Schriftsteller, Journalist}!französisch-chinesische Zwischenfall in Shanghai@\strich\emph{Der französisch-chinesische Zwischenfall in Shanghai}|pwkv}\pwindex{Goldmann, Paul 31.\,1.\,1865 Breslau – 25.\,9.\,1935 Wien@\textsc{Goldmann, Paul} (31.\,1.\,1865 Breslau – 25.\,9.\,1935 Wien), \emph{Schriftsteller, Journalist}!französisch-chinesische Zwischenfall in Shanghai@\strich\emph{Der französisch-chinesische Zwischenfall in Shanghai}|pwkv}\pwindex{Goldmann, Paul 31.\,1.\,1865 Breslau – 25.\,9.\,1935 Wien@\textsc{Goldmann, Paul} (31.\,1.\,1865 Breslau – 25.\,9.\,1935 Wien), \emph{Schriftsteller, Journalist}!In Kiautschou. I@\strich\emph{In Kiautschou. I}|pwkv}\pwindex{Goldmann, Paul 31.\,1.\,1865 Breslau – 25.\,9.\,1935 Wien@\textsc{Goldmann, Paul} (31.\,1.\,1865 Breslau – 25.\,9.\,1935 Wien), \emph{Schriftsteller, Journalist}!In Kiautschou. II@\strich\emph{In Kiautschou. II}|pwkv}\pwindex{Goldmann, Paul 31.\,1.\,1865 Breslau – 25.\,9.\,1935 Wien@\textsc{Goldmann, Paul} (31.\,1.\,1865 Breslau – 25.\,9.\,1935 Wien), \emph{Schriftsteller, Journalist}!In Kiautschou. III@\strich\emph{In Kiautschou. III}|pwkv}\pwindex{Goldmann, Paul 31.\,1.\,1865 Breslau – 25.\,9.\,1935 Wien@\textsc{Goldmann, Paul} (31.\,1.\,1865 Breslau – 25.\,9.\,1935 Wien), \emph{Schriftsteller, Journalist}!In Kiautschou. IV [letzter Teil]@\strich\emph{In Kiautschou. IV [letzter Teil]}|pwkv}\pwindex{Goldmann, Paul 31.\,1.\,1865 Breslau – 25.\,9.\,1935 Wien@\textsc{Goldmann, Paul} (31.\,1.\,1865 Breslau – 25.\,9.\,1935 Wien), \emph{Schriftsteller, Journalist}!General Tscheng-Ki-tong@\strich\emph{Der General Tscheng-Ki-tong}|pwkv} unter dem Kürzel »G\pwindex{Goldmann, Paul 31.\,1.\,1865 Breslau – 25.\,9.\,1935 Wien@\textsc{Goldmann, Paul} (31.\,1.\,1865 Breslau – 25.\,9.\,1935 Wien), \emph{Schriftsteller, Journalist}|pwkv}«
                   am 8. 6. 1898\pwindex{Goldmann, Paul 31.\,1.\,1865 Breslau – 25.\,9.\,1935 Wien@\textsc{Goldmann, Paul} (31.\,1.\,1865 Breslau – 25.\,9.\,1935 Wien), \emph{Schriftsteller, Journalist}!Pest in Hongkong@\strich\emph{Die Pest in Hongkong}|pwkv}, 23. 6. 1898\pwindex{Goldmann, Paul 31.\,1.\,1865 Breslau – 25.\,9.\,1935 Wien@\textsc{Goldmann, Paul} (31.\,1.\,1865 Breslau – 25.\,9.\,1935 Wien), \emph{Schriftsteller, Journalist}!Eine Unterredung mit dem Tao-tai Wang, dem Sekretär des Vicekönigs von Canton@\strich\emph{Eine Unterredung mit dem Tao-tai Wang, dem Sekretär des Vicekönigs von Canton}|pwkv}\pwindex{Goldmann, Paul 31.\,1.\,1865 Breslau – 25.\,9.\,1935 Wien@\textsc{Goldmann, Paul} (31.\,1.\,1865 Breslau – 25.\,9.\,1935 Wien), \emph{Schriftsteller, Journalist}!Eine Unterredung mit dem Tao-tai Wang, dem Sekretär des Vicekönigs von Canton [zweiter Teil]@\strich\emph{Eine Unterredung mit dem Tao-tai Wang, dem Sekretär des Vicekönigs von Canton [zweiter Teil]}|pwkv}, 21. 7. 1898\pwindex{Goldmann, Paul 31.\,1.\,1865 Breslau – 25.\,9.\,1935 Wien@\textsc{Goldmann, Paul} (31.\,1.\,1865 Breslau – 25.\,9.\,1935 Wien), \emph{Schriftsteller, Journalist}!Beim Tao-tai Tsai von Shanghai@\strich\emph{Beim Tao-tai Tsai von Shanghai}|pwkv}, 23. 7. 1898\pwindex{Goldmann, Paul 31.\,1.\,1865 Breslau – 25.\,9.\,1935 Wien@\textsc{Goldmann, Paul} (31.\,1.\,1865 Breslau – 25.\,9.\,1935 Wien), \emph{Schriftsteller, Journalist}!deutschen Militär-Instruktoren in China@\strich\emph{Die deutschen Militär-Instruktoren in China}|pwkv}, 3. 8. 1898\pwindex{Goldmann, Paul 31.\,1.\,1865 Breslau – 25.\,9.\,1935 Wien@\textsc{Goldmann, Paul} (31.\,1.\,1865 Breslau – 25.\,9.\,1935 Wien), \emph{Schriftsteller, Journalist}!Kapitel über chinesische Eisenbahnen. Chinesische Eisenbahnen und deutsche Versäumnisse@\strich\emph{Ein Kapitel über chinesische Eisenbahnen. Chinesische Eisenbahnen und deutsche Versäumnisse}|pwkv}, 4. 8. 1898\pwindex{Goldmann, Paul 31.\,1.\,1865 Breslau – 25.\,9.\,1935 Wien@\textsc{Goldmann, Paul} (31.\,1.\,1865 Breslau – 25.\,9.\,1935 Wien), \emph{Schriftsteller, Journalist}!Kapitel über chinesische Eisenbahnen. Die Bahn in Shanghai nach Wu-sung@\strich\emph{Ein Kapitel über chinesische Eisenbahnen. Die Bahn in Shanghai nach Wu-sung}|pwkv}, 17. 8. 1898\pwindex{Goldmann, Paul 31.\,1.\,1865 Breslau – 25.\,9.\,1935 Wien@\textsc{Goldmann, Paul} (31.\,1.\,1865 Breslau – 25.\,9.\,1935 Wien), \emph{Schriftsteller, Journalist}!Chinesische Zeitungen@\strich\emph{Chinesische Zeitungen}|pwkv}, 25. 8. 1898\pwindex{Goldmann, Paul 31.\,1.\,1865 Breslau – 25.\,9.\,1935 Wien@\textsc{Goldmann, Paul} (31.\,1.\,1865 Breslau – 25.\,9.\,1935 Wien), \emph{Schriftsteller, Journalist}!französisch-chinesische Zwischenfall in Shanghai@\strich\emph{Der französisch-chinesische Zwischenfall in Shanghai}|pwkv}, 9. 9. 1898\pwindex{Goldmann, Paul 31.\,1.\,1865 Breslau – 25.\,9.\,1935 Wien@\textsc{Goldmann, Paul} (31.\,1.\,1865 Breslau – 25.\,9.\,1935 Wien), \emph{Schriftsteller, Journalist}!französisch-chinesische Zwischenfall in Shanghai@\strich\emph{Der französisch-chinesische Zwischenfall in Shanghai}|pwkv}, 23. 9. 1898\pwindex{Goldmann, Paul 31.\,1.\,1865 Breslau – 25.\,9.\,1935 Wien@\textsc{Goldmann, Paul} (31.\,1.\,1865 Breslau – 25.\,9.\,1935 Wien), \emph{Schriftsteller, Journalist}!In Kiautschou. I@\strich\emph{In Kiautschou. I}|pwkv}, 24. 9. 1898\pwindex{Goldmann, Paul 31.\,1.\,1865 Breslau – 25.\,9.\,1935 Wien@\textsc{Goldmann, Paul} (31.\,1.\,1865 Breslau – 25.\,9.\,1935 Wien), \emph{Schriftsteller, Journalist}!In Kiautschou. II@\strich\emph{In Kiautschou. II}|pwkv}, 25. 9. 1898\pwindex{Goldmann, Paul 31.\,1.\,1865 Breslau – 25.\,9.\,1935 Wien@\textsc{Goldmann, Paul} (31.\,1.\,1865 Breslau – 25.\,9.\,1935 Wien), \emph{Schriftsteller, Journalist}!In Kiautschou. III@\strich\emph{In Kiautschou. III}|pwkv}, 26. 9. 1898\pwindex{Goldmann, Paul 31.\,1.\,1865 Breslau – 25.\,9.\,1935 Wien@\textsc{Goldmann, Paul} (31.\,1.\,1865 Breslau – 25.\,9.\,1935 Wien), \emph{Schriftsteller, Journalist}!In Kiautschou. IV [letzter Teil]@\strich\emph{In Kiautschou. IV [letzter Teil]}|pwkv}, 25. 10. 1898\pwindex{Goldmann, Paul 31.\,1.\,1865 Breslau – 25.\,9.\,1935 Wien@\textsc{Goldmann, Paul} (31.\,1.\,1865 Breslau – 25.\,9.\,1935 Wien), \emph{Schriftsteller, Journalist}!General Tscheng-Ki-tong@\strich\emph{Der General Tscheng-Ki-tong}|pwkv} und darüber hinaus.}}}\label{K_L02858-4} verfolgſt. Du nennſt{ }ſie »intereſſant« und ahnſt
               gewiß nicht, daß das ihre Verurtheilung iſt. Intereſſant iſt die Rubrik »Vermiſchtes«
               in den Zeitungen, die von einem wunderbaren Walfiſch-Fang berichtet oder vom
               tätowirten Indianer. Die unbeſchreibliche künſtleriſche Anſtrengung, die ich auf
               meine Arbeiten verwende, das Beſtreben, einfach, klar und doch maleriſch
               darzuſtellen, {\pb}kommt alſo nicht zum Ausdruck. Wenn{ }ſelbſt Du es nicht{ }ſiehſt,{ }ſo beweiſt das, daß meine Arbeiten verfehlt{ }ſind, was ich
               von Anfang an \strikeout{\textcolor{gray}{×}\-\textcolor{gray}{×}\-\textcolor{gray}{×}\-\textcolor{gray}{×}\-\textcolor{gray}{×}\-\textcolor{gray}{×}\-\textcolor{gray}{×}} geahnt habe. Es iſt{ }ſehr bitter, liebſter Freund, intereſſant zu{ }ſchreiben.\pend
           
\pstart
           Mein Brief findet Dich hoffentlich in guter, froher Arbeit und in heller Stimmung.
               Denke Dir nur, welch’ ein \label{K_L02858-5v}\edtext{\textsc{Schemen}}{\lemma{\textnormal{\emph{Schemen}}}\Cendnote{\textnormal{Trugbild}}}\label{K_L02858-5}{ }\strikeout{alle} alle Deine Leiden{ }ſein müſſen, {\pb}wenn eine einzige \label{K_L02858-6v}\edtext{Reiſe}{\lemma{\textnormal{\emph{Reise}}}\Cendnote{\textnormal{Siehe XXXX Auszeichnungsfehler: Dokument L02845 nicht gefunden.
               }}}\label{K_L02858-6} von Wien\oindex{Wien@\textbf{Wien}, \emph{Verwaltungsgebiet}|pw} nach Salzburg\oindex{Salzburg@\textbf{Salzburg}, \emph{Verwaltungsgebiet}|pw}{ }ſie verblaſſen macht. Quäle Dich nicht und mache Dir
               einen frohen Winter!\pend
           
\pstart
           Grüß’ mir den \textsc{Richard\pwindex{Beer-Hofmann, Richard 11.\,7.\,1866 Wien – 26.\,9.\,1945 New York City@\textsc{Beer-Hofmann, Richard} (11.\,7.\,1866 Wien – 26.\,9.\,1945 New York City), \emph{Schriftsteller}|pw}}! Ich \strikeout{h\textcolor{gray}{×}\-\textcolor{gray}{×}\-\textcolor{gray}{×}} freue mich, daß er das \label{K_L02858-7v}\edtext{dritte
               Capitel des »Götterliebling\pwindex{Beer-Hofmann, Richard 11.\,7.\,1866 Wien – 26.\,9.\,1945 New York City@\textsc{Beer-Hofmann, Richard} (11.\,7.\,1866 Wien – 26.\,9.\,1945 New York City), \emph{Schriftsteller}!Tod Georgs@\strich\emph{Der Tod Georgs}|pwv}«}{\lemma{\textnormal{\emph{dritte … »Götterliebling«}}}\Cendnote{\textnormal{Als Schnitzler am 28. 7. 1898 in Salzburg\oindex{Salzburg@\textbf{Salzburg}, \emph{Verwaltungsgebiet}|pwk} gewesen war, hatte ihm Beer-Hofmann\pwindex{Beer-Hofmann, Richard 11.\,7.\,1866 Wien – 26.\,9.\,1945 New York City@\textsc{Beer-Hofmann, Richard} (11.\,7.\,1866 Wien – 26.\,9.\,1945 New York City), \emph{Schriftsteller}|pwk}
                  das dritte Kapitel des Götterlieblings\pwindex{Beer-Hofmann, Richard 11.\,7.\,1866 Wien – 26.\,9.\,1945 New York City@\textsc{Beer-Hofmann, Richard} (11.\,7.\,1866 Wien – 26.\,9.\,1945 New York City), \emph{Schriftsteller}!Tod Georgs@\strich\emph{Der Tod Georgs}|pwkv} vorgelesen. Die Erzählung\pwindex{Beer-Hofmann, Richard 11.\,7.\,1866 Wien – 26.\,9.\,1945 New York City@\textsc{Beer-Hofmann, Richard} (11.\,7.\,1866 Wien – 26.\,9.\,1945 New York City), \emph{Schriftsteller}!Tod Georgs@\strich\emph{Der Tod Georgs}|pwkv} erschien zuerst zwischen 4. 11. 1899 und 25. 11. 1899 als
                  Fragment unter dem Titel \emph{Der Tod Georgs}\pwindex{Beer-Hofmann, Richard 11.\,7.\,1866 Wien – 26.\,9.\,1945 New York City@\textsc{Beer-Hofmann, Richard} (11.\,7.\,1866 Wien – 26.\,9.\,1945 New York City), \emph{Schriftsteller}!Tod Georgs. Fragment@\strich\emph{Der Tod Georgs. Fragment}|pwk} in
                  der \emph{Zeit}\pwindex{Zeit. Wiener Wochenschrift@\emph{Die Zeit. Wiener Wochenschrift}|pwk}.}}}\label{K_L02858-7} beendet hat. Nur fürchte
               ich, im vierten Capitel\pwindex{Beer-Hofmann, Richard 11.\,7.\,1866 Wien – 26.\,9.\,1945 New York City@\textsc{Beer-Hofmann, Richard} (11.\,7.\,1866 Wien – 26.\,9.\,1945 New York City), \emph{Schriftsteller}!Tod Georgs@\strich\emph{Der Tod Georgs}|pwv} wird
               der Held wieder einſchlafen {\pb}und einige Jahrhundert
               Weltgeſchichte \strikeout{t\textcolor{gray}{r}} träumen, und das wird \substVorne{}\textsuperscript{\textcolor{gray}{wie}d\textcolor{gray}{er}}\substDazwischen{}noch\substHinten{} recht lang werden.\pend
           
\pstart
           Man{ }ſandte mir hierher einen \label{K_L02858-8v}\edtext{Artikel\pwindex{Lothar, Rudolf 23.\,2.\,1865 Budapest – 2.\,10.\,1943 ebd.@\textsc{Lothar, Rudolf} (23.\,2.\,1865 Budapest – 2.\,10.\,1943 ebd.), \emph{Schriftsteller, Journalist, Theaterdirektor}!Briefe an eine Dame@\strich\emph{Briefe an eine Dame}|pwv}}{\lemma{\textnormal{\emph{Artikel}}}\Cendnote{\textnormal{Rudolf Lothar\pwindex{Lothar, Rudolf 23.\,2.\,1865 Budapest – 2.\,10.\,1943 ebd.@\textsc{Lothar, Rudolf} (23.\,2.\,1865 Budapest – 2.\,10.\,1943 ebd.), \emph{Schriftsteller, Journalist, Theaterdirektor}|pwk}: \emph{Briefe an eine Dame}\pwindex{Lothar, Rudolf 23.\,2.\,1865 Budapest – 2.\,10.\,1943 ebd.@\textsc{Lothar, Rudolf} (23.\,2.\,1865 Budapest – 2.\,10.\,1943 ebd.), \emph{Schriftsteller, Journalist, Theaterdirektor}!Briefe an eine Dame@\strich\emph{Briefe an eine Dame}|pwk}. In: \emph{Die Wage. Eine Wiener Wochenschrift}\pwindex{Wage. Eine Wiener Wochenschrift@\emph{Die Wage. Eine Wiener Wochenschrift}|pwk}, Jg. 1, Nr. 26,
                        25. 6. 1898, S. 439–440.}}}\label{K_L02858-8} von
                  \textsc{Rudolf Lothar\pwindex{Lothar, Rudolf 23.\,2.\,1865 Budapest – 2.\,10.\,1943 ebd.@\textsc{Lothar, Rudolf} (23.\,2.\,1865 Budapest – 2.\,10.\,1943 ebd.), \emph{Schriftsteller, Journalist, Theaterdirektor}|pw}} über Dich in der »Wage\pwindex{Wage. Eine Wiener Wochenschrift@\emph{Die Wage. Eine Wiener Wochenschrift}|pw}«. Wenn Du den Autor\pwindex{Lothar, Rudolf 23.\,2.\,1865 Budapest – 2.\,10.\,1943 ebd.@\textsc{Lothar, Rudolf} (23.\,2.\,1865 Budapest – 2.\,10.\,1943 ebd.), \emph{Schriftsteller, Journalist, Theaterdirektor}|pwv}{ }ſiehſt,{ }ſo grüße ihn von
               mir und{ }ſage ihm, meines Wiſſens{ }ſei noch nie über Dich ein ähnlicher Blödſinn
               geſchrieben worden. Auch erfahre ich daraus, daß \strikeout{D} Du
                  {\pb}durch \textsc{Rudolf Lothar\pwindex{Lothar, Rudolf 23.\,2.\,1865 Budapest – 2.\,10.\,1943 ebd.@\textsc{Lothar, Rudolf} (23.\,2.\,1865 Budapest – 2.\,10.\,1943 ebd.), \emph{Schriftsteller, Journalist, Theaterdirektor}|pw}} zum Schreiben ermuntert worden biſt. Jetzt weiß ich, warum Du ein Dichter
               biſt!\pend
           
\pstart
           Grüß’ Dich Gott, liebſter Freund!\pend
           
\pstart
           Dein treuer {\\[\baselineskip]}\spacefill\mbox{Paul Goldmann}\pend
           \leftskip=0em{}
\pstart
           \noindent{}Viele Grüße an Deine Freundin\pwindex{Reinhard, Marie 13.\,3.\,1871 Wien – 18.\,3.\,1899 ebd.@\textsc{Reinhard, Marie} (13.\,3.\,1871 Wien – 18.\,3.\,1899 ebd.), \emph{Gesangspädagogin}|pwv}!\pend
           \selectlanguage{ngerman}\endnumbering\briefempfaengerindex{Schnitzler, Arthur@\textsc{Schnitzler, Arthur}!zzzGoldmann, Paul@\emph{von Paul Goldmann}!1898-09-251@{25. 9. 1898}|)be}\mylabel{L02858h}  \newcommand{\dateiname}{L02858}\newcommand{\titel}{Paul Goldmann an Arthur Schnitzler, 25. 9. 1898}\newcommand{\editorInnen}{Martin Anton Müller und Laura Untner}%% latex-leseansicht-abspann.tex
%% Abspann für die Leseansicht.
%% Der Schalter \ifkorrekturansicht ist bereits durch den Vorspann gesetzt.

%% latex-abspann.tex
%% Gemeinsamer Abspann für Korrekturansicht und Leseansicht.
%% Setzt den Schalter \ifkorrekturansicht voraus (gesetzt in den
%% einbindenden Dateien latex-korrekturansicht-abspann.tex bzw.
%% latex-leseansicht-abspann.tex).
%% ---------------------------------------------------------------

\normalsize

% Das esempio-Environment wird nur in der Leseansicht benötigt
\ifkorrekturansicht\else
\newenvironment{esempio}[3]%
{
    \vspace{1.5ex}
    \rlap{\underline{#1}}
    \par
    \setlength{\parindent}{0cm}
    \nopagebreak
    \leftskip=#2cm
    \rightskip=#3cm
}
{
    \par
}
\fi

\doendnotes{C}
\bigskip
\vfill

\clearpage

\footnotesize

\ifkorrekturansicht
  \lohead{\textsc{register}}
\fi

% theindex-Environment neu definieren ohne reledmac
\makeatletter
\renewenvironment{theindex}{%
  \ifkorrekturansicht
    \section*{\indexname}%
  \else
    \subsubsection*{Index der erwähnten Entitäten}%
  \fi
  \setlength{\parindent}{0pt}%
  \setlength{\parskip}{0pt plus 0.3pt}%
  \let\item\@idxitem
}{%
  \ifkorrekturansicht\clearpage\fi
}
\makeatother

\IfFileExists{\jobname-pw.ind}{\input{\jobname-pw.ind}}{}

% Quellenangabe nur in der Leseansicht
\ifkorrekturansicht\else
% Fallback-Definitionen, falls die .tex-Datei \titel etc. nicht gesetzt hat
\providecommand{\titel}{}
\providecommand{\editorInnen}{}
\providecommand{\dateiname}{\jobname}

\vspace{3cm}

\vfill

\footnotesize
\textsc{Quelle}: \titel. Herausgegeben von {\editorInnen}. In: \emph{Arthur Schnitzler: Briefwechsel mit Autorinnen und Autoren}.
 Digitale Edition, https://schnitzler-briefe.acdh.oeaw.ac.at/{\dateiname}.html (Stand \today)
\fi

\end{document}


