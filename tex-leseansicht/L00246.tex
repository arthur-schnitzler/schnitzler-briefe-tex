%% latex-leseansicht-vorspann.tex
%% Vorspann für die Leseansicht.
%% Lädt die gemeinsame Datei latex-vorspann.tex mit nicht gesetztem Schalter.

\newif\ifkorrekturansicht
\korrekturansichtfalse

\input{../tex-inputs/latex-vorspann}


         
         \newcommand{\erwaehntePersonen}{Personen:  ?? [Schreibkraft für Arthur Schnitzler], Richard Beer-Hofmann, Hugo von Hofmannsthal, Karl Kraus, Emil Löbl, Adele von Suppé, Moriz Szeps}
         \newcommand{\erwaehnteOrte}{Orte: Bad Ischl, Lessing-Theater, Schulgasse, Wien}
         \newcommand{\erwaehnteWerke}{Werke: Abschiedssouper, Berliner Börsen-Courier, Das Kind, Das Märchen. Schauspiel in drei Aufzügen, Der Tod Georgs, Die Saison in Ischl, Die kleine Komödie, Magazin für die Literatur des Auslandes, Wiener Tagblatt, [Am Lessingtheater … Ohne Geläut … Das Märchen … zur Aufführung], [Man schreibt uns aus Ischl]}
               \section[Arthur Schnitzler an Richard Beer-Hofmann, 29. 7. 1893]{ Arthur Schnitzler an Richard Beer-Hofmann, 29. 7. 1893}\nopagebreak\mylabel{v}\rehead{ }\begin{ledgroupsized}[t]{13cm}\normalsize\beginnumbering \toendnotes[C]{\smallbreak\pagebreak[2]} \Standort{YCGL, MSS 31.}
\physDesc{Briefkarte mit Trauerrand, Umschlag mit Trauerrand
\newline{}Handschrift: Bleistift, deutsche Kurrent\newline{}Versand: 1) Stempel: »\nobreak{}Wien 1/1, 29. 7. 93, 2–3 N\nobreak{}«.   2) Stempel: »\nobreak{}\oindex{Bad Ischl@\textbf{Bad Ischl}|pwk}Ischl, 30 7 93, 7–F\nobreak{}«. \newline{}Ordnung: mit Rotstift von unbekannter Hand oberhalb des Textes mit einem
                                    »X« versehen }\buchAbdrucke{\weitereDrucke{Arthur Schnitzler, Richard Beer-Hofmann: \emph{Briefwechsel 1891–1931}. Hg. Konstanze Fliedl. Wien, Zürich: \emph{Europaverlag} 1992, S. 49.} }\toendnotes[C]{\smallbreak}\pstart{}{\pb}\textsc{Herrn Doctor}\pend{}\pstart{}\textsc{Richard Beer-Hof\damage{mann}}\pend{}\pstart{}\textsc{Isch\oindex{Bad Ischl@\textbf{Bad Ischl}|pw}\damage{l}}\pend{}\pstart{}\textsc{Schulgasse }\oindex{Schulgasse@\textbf{Schulgasse}|pw}\damage{8}\pend{}{\bigskip}\pstart
           \noindent{}{\pb}Lieber Richard! – Der Abſchreiber\pwindex{?? [Schreibkraft fuer Arthur Schnitzler] @\textsc{?? [Schreibkraft für Arthur Schnitzler]}|pwv} bringt die Novelle\pwindex{Beer-Hofmann, Richard 1866-07-11 – 1945-09-26@\textsc{Beer-Hofmann, Richard} (1866-07-11 – 1945-09-26), \emph{Schriftsteller}!Kind1893@\strich\emph{Das Kind} {[}1893{]}|pwv}{ }Montag; – Dinſtag haben Sie ſie. – Neulich ſtand\pwindex{?? Werk@Nicht ermittelte Verfasserinnen und Verfasser!Am Lessingtheater … Ohne Gelaeut … Das Maerchen … zur Auffuehrung]22. 7. 1893@\emph{[Am Lessingtheater … Ohne Geläut … Das Märchen … zur Aufführung]} {[}22. 7. 1893{]}|pwv} im Magazin\pwindex{?? Werk@Nicht ermittelte Verfasserinnen und Verfasser!Magazin fuer die Literatur des Auslandes1832 – 1915@\emph{Magazin für die Literatur des Auslandes} {[}1832 – 1915{]}|pw} (Kraus\pwindex{Kraus, Karl 28.04.1874 – 12.06.1936@\textsc{Kraus, Karl} (28.04.1874 – 12.06.1936), \emph{Schriftsteller, Publizist}|pw}{ }ſchickt es mir) dſs noch dieſen So{\geminationm}er im Leſſ.th.\oindex{Lessing-Theater@\textbf{Lessing-Theater}|pw} das Märchen\pwindex{Schnitzler, Arthur 15.05.1862 – 21.10.1931@\textsc{Schnitzler, Arthur} (15.05.1862 – 21.10.1931), \emph{Schriftsteller, Mediziner}!Maerchen. Schauspiel in drei Aufzuegen1893-12-01@\strich\emph{Das Märchen. Schauspiel in drei Aufzügen} {[}1893-12-01{]}|pw} dranko{\geminationm}t. – Die »lustige« Novelle\pwindex{Schnitzler, Arthur 15.05.1862 – 21.10.1931@\textsc{Schnitzler, Arthur} (15.05.1862 – 21.10.1931), \emph{Schriftsteller, Mediziner}!kleine Komoedie1895-08-01@\strich\emph{Die kleine Komödie} {[}1895-08-01{]}|pwv} beendet. – Aerztlich beſchäftigt, eine Cousine\pwindex{Suppe, Adele von 05.12.1879 – 04.08.1893@\textsc{Suppé, Adele von} (05.12.1879 – 04.08.1893)|pwv}, 14j. Mädel, ſchwerer
               Typhus. – Habe noch keine {\pb}Einberufung. – Notiz\pwindex{?? Werk@Nicht ermittelte Verfasserinnen und Verfasser!Man schreibt uns aus Ischl]25. 07. 1893@\emph{[Man schreibt uns aus Ischl]} {[}25. 07. 1893{]}|pwv} im B. B.\pwindex{?? Werk@Nicht ermittelte Verfasserinnen und Verfasser!Berliner Boersen-Courier1868 – 1933@\emph{Berliner Börsen-Courier} {[}1868 – 1933{]}|pw} geleſen; ſehr gut – aber natürlich »naturaliſtiſcher Dichter\pwindex{?? Werk@Nicht ermittelte Verfasserinnen und Verfasser!Berliner Boersen-Courier1868 – 1933@\emph{Berliner Börsen-Courier} {[}1868 – 1933{]}|pwv}«. – Geſtern war ich
               angeblich im \textsc{Szeps}\pwindex{Wiener Tagblatt1886 – 1901@\emph{Wiener Tagblatt} {[}1886 – 1901{]}|pwv}\pwindex{Szeps, Moriz 04.11.1834 – 09.08.1902@\textsc{Szeps, Moriz} (04.11.1834 – 09.08.1902), \emph{Journalist}|pw}{ }\label{K_L00246_1v}\edtext{verschimpfirt\pwindex{?? Werk@Nicht ermittelte Verfasserinnen und Verfasser!Saison in Ischl28. 7. 1893@\emph{Die Saison in Ischl} {[}28. 7. 1893{]}|pwv}}{\lemma{\textnormal{\emph{verschimpfirt}}}\Cendnote{\textnormal{In dem Bericht ohne Autornennung heißt
                  es: »Das Theaterleben ist ein sehr bewegtes, Tag für Tag Vorstellung,
                     berühmte und unberühmte Gäste, ja sogar Novitäten, sogenannte Sommer-Einakter,
                     die freilich oft nur aus Courtoisie aufgeführt werden. Ein realistisches
                     Stückchen ›Das Abschieds-Souper\pwindex{Schnitzler, Arthur 15.05.1862 – 21.10.1931@\textsc{Schnitzler, Arthur} (15.05.1862 – 21.10.1931), \emph{Schriftsteller, Mediziner}!Abschiedssouper1892@\strich\emph{Abschiedssouper} {[}1892{]}|pw}‹, aus der
                     Feder eines jungen Wien\oindex{Wien@\textbf{Wien}|pw}er Realisten\pwindex{Schnitzler, Arthur 15.05.1862 – 21.10.1931@\textsc{Schnitzler, Arthur} (15.05.1862 – 21.10.1931), \emph{Schriftsteller, Mediziner}|pwv} hat wenig Erfolg gehabt, um
                     nicht zu sagen, gar keinen«. (\emph{Die Saison in Ischl}\pwindex{?? Werk@Nicht ermittelte Verfasserinnen und Verfasser!Saison in Ischl28. 7. 1893@\emph{Die Saison in Ischl} {[}28. 7. 1893{]}|pwk}. In: \emph{Wiener Tagblatt}\pwindex{Wiener Tagblatt1886 – 1901@\emph{Wiener Tagblatt} {[}1886 – 1901{]}|pwk}, Jg. 43, Nr. 206, 28. 7. 1893,
                     S. 4.)}}}\label{K_L00246_1h} (las es nicht) – nachdem ich vor 3 Tagen \label{K_L00246_2v}\edtext{gelobt}{\lemma{\textnormal{\emph{gelobt}}}\Cendnote{\textnormal{nicht nachweisbar}}}\label{K_L00246_2h} war. Gute Redaction! – Was macht der
                  Götterliebling\pwindex{Beer-Hofmann, Richard 1866-07-11 – 1945-09-26@\textsc{Beer-Hofmann, Richard} (1866-07-11 – 1945-09-26), \emph{Schriftsteller}!Tod Georgs1900@\strich\emph{Der Tod Georgs} {[}1900{]}|pw}? – Ist Löbl\pwindex{Loebl, Emil 05.02.1863 – 26.08.1942@\textsc{Löbl, Emil} (05.02.1863 – 26.08.1942), \emph{Journalist, Journalist, Schriftsteller}|pw} noch in Iſchl\oindex{Bad Ischl@\textbf{Bad Ischl}|pw}? Wohin
               ſchreibt man ihm? Las übrigens die Nu{\geminationm}er noch gar
               nicht. – Schreibt Loris\pwindex{Hofmannsthal, Hugo von 1874-02-01 – 1929-07-15@\textsc{Hofmannsthal, Hugo von} (1874-02-01 – 1929-07-15), \emph{Schriftsteller}|pw}? – Grüßen Sie alles!
                  \label{T_L00246_1v}\edtext{Ich würde mehr ſchreiben, we{\geminationn} ich nicht auf}{\lemma{\textnormal{\emph{Ich … auf}}}\Cendnote{\textnormal{quer am rechten Rand weiter}}}\label{T_L00246_1h}{ }\label{T_L00246_2v}\edtext{dieſem blöden Karterl angefangen
                  hätte.}{\lemma{\textnormal{\emph{dieſem … hätte.}}}\Cendnote{\textnormal{am linken Rand der
                  Vorderseite}}}\label{T_L00246_2h}\pend
           
         
         \endnumbering\mylabel{h}\end{ledgroupsized}  \newcommand{\dateiname}{L00246}\newcommand{\titel}{Arthur Schnitzler an Richard Beer-Hofmann, 29. 7. 1893}\newcommand{\editorInnen}{Martin Anton Müller und Gerd-Hermann Susen}%% latex-leseansicht-abspann.tex
%% Abspann für die Leseansicht.
%% Der Schalter \ifkorrekturansicht ist bereits durch den Vorspann gesetzt.

%% latex-abspann.tex
%% Gemeinsamer Abspann für Korrekturansicht und Leseansicht.
%% Setzt den Schalter \ifkorrekturansicht voraus (gesetzt in den
%% einbindenden Dateien latex-korrekturansicht-abspann.tex bzw.
%% latex-leseansicht-abspann.tex).
%% ---------------------------------------------------------------

\normalsize

% Das esempio-Environment wird nur in der Leseansicht benötigt
\ifkorrekturansicht\else
\newenvironment{esempio}[3]%
{
    \vspace{1.5ex}
    \rlap{\underline{#1}}
    \par
    \setlength{\parindent}{0cm}
    \nopagebreak
    \leftskip=#2cm
    \rightskip=#3cm
}
{
    \par
}
\fi

\doendnotes{C}
\bigskip
\vfill

\clearpage

\footnotesize

\ifkorrekturansicht
  \lohead{\textsc{register}}
\fi

% theindex-Environment neu definieren ohne reledmac
\makeatletter
\renewenvironment{theindex}{%
  \ifkorrekturansicht
    \section*{\indexname}%
  \else
    \subsubsection*{Index der erwähnten Entitäten}%
  \fi
  \setlength{\parindent}{0pt}%
  \setlength{\parskip}{0pt plus 0.3pt}%
  \let\item\@idxitem
}{%
  \ifkorrekturansicht\clearpage\fi
}
\makeatother

\IfFileExists{\jobname-pw.ind}{\input{\jobname-pw.ind}}{}

% Quellenangabe nur in der Leseansicht
\ifkorrekturansicht\else
% Fallback-Definitionen, falls die .tex-Datei \titel etc. nicht gesetzt hat
\providecommand{\titel}{}
\providecommand{\editorInnen}{}
\providecommand{\dateiname}{\jobname}

\vspace{3cm}

\vfill

\footnotesize
\textsc{Quelle}: \titel. Herausgegeben von {\editorInnen}. In: \emph{Arthur Schnitzler: Briefwechsel mit Autorinnen und Autoren}.
 Digitale Edition, https://schnitzler-briefe.acdh.oeaw.ac.at/{\dateiname}.html (Stand \today)
\fi

\end{document}


      