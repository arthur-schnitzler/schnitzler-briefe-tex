%% latex-korrekturansicht-vorspann.tex
%% Vorspann für die Korrekturansicht.
%% Lädt die gemeinsame Datei latex-vorspann.tex mit gesetztem Schalter.

\newif\ifkorrekturansicht
\korrekturansichttrue

\input{../tex-inputs/latex-vorspann}


\section[Arthur Schnitzler an Richard Beer-Hofmann, 29. 7. 1893]{L00246 Arthur Schnitzler an Richard Beer-Hofmann, 29. 7. 1893}
\nopagebreak\mylabel{L00246v}
\rehead{ }\normalsize\beginnumbering\briefempfaengerindex{Beer-Hofmann, Richard@\textsc{Beer-Hofmann, Richard}!zzzSchnitzler, Arthur@\emph{von Arthur Schnitzler}!1893-07-291@{29. 7. 1893}|(be}
\toendnotes[C]{\smallbreak\pagebreak[2]}\Standort{YCGL, MSS 31.}
\physDesc{Briefkarte, , Umschlag, 780 Zeichen (Karte und Umschlag mit Trauerrand )
\newline{}Handschrift: Bleistift, deutsche Kurrent
\newline{}Versand: 1) Stempel: »\nobreak{}Wien 1/1, 29. 7. 93, 2–3 N\nobreak{}«.   2) Stempel: »\nobreak{}\oindex{Bad Ischl@\textbf{Bad Ischl}, \emph{P.PPL}|pwk}Ischl, 30 7 93, 7–F\nobreak{}«. 
\newline{}Ordnung: mit rotem Buntstift von unbekannter Hand oberhalb des Textes mit einem
                                    »X« versehen }
\buchAbdrucke{\weitereDrucke{Arthur Schnitzler, Richard Beer-Hofmann: \emph{Briefwechsel 1891–1931}. Wien, Zürich: \emph{Europaverlag} 1992, S. 49.} }\toendnotes[C]{\smallbreak}\pstart{}{\pb}\textsc{Herrn Doctor}\pend{}\pstart{}\textsc{Richard Beer-Hof\damage{mann}}\pend{}\pstart{}\textsc{Isch\oindex{Bad Ischl@\textbf{Bad Ischl}, \emph{P.PPL}|pw}\damage{l}}\pend{}\pstart{}\textsc{Schulgasse }\oindex{Schulgasse@\textbf{Schulgasse}, \emph{Straße (K.STR)}|pw}\damage{8}\pend{}{\bigskip}\vspace{1em}
\pstart
           \noindent{}{\pb}Lieber Richard! – Der Abſchreiber\pwindex{?? [Schreibkraft fuer Arthur Schnitzler] @\textsc{?? [Schreibkraft für Arthur Schnitzler]}|pwv} bringt die Novelle\pwindex{Kind@\emph{Das Kind}|pwv}{ }Montag; – Dinſtag haben Sie ſie. – Neulich ſtand\pwindex{Am Lessingtheater … Ohne Gelaeut … Das Maerchen … zur Auffuehrung]@\emph{[Am Lessingtheater … Ohne Geläut … Das Märchen … zur Aufführung]}|pwv} im Magazin\pwindex{Magazin fuer die Literatur des Auslandes@\emph{Magazin für die Literatur des Auslandes}|pw} (Kraus\pwindex{Kraus, Karl 28.04.1874 – 12.06.1936@\textsc{Kraus, Karl} (28.04.1874 – 12.06.1936), \emph{Schriftsteller/Schriftstellerin, Publizist/Publizistin, Schriftsteller/Schriftstellerin}|pw}{ }ſchickt es mir) dſs noch dieſen So{\geminationm}er im Leſſ.th.\oindex{Lessing-Theater@\textbf{Lessing-Theater}, \emph{Theater (K.THE)}|pw} das Märchen\pwindex{Maerchen. Schauspiel in drei Aufzuegen@\emph{Das Märchen. Schauspiel in drei Aufzügen}|pw} dranko{\geminationm}t. – Die »lustige« Novelle\pwindex{kleine Komoedie@\emph{Die kleine Komödie}|pwv} beendet. – Aerztlich beſchäftigt, eine Cousine\pwindex{Suppe, Adele von 05.12.1879 – 04.08.1893@\textsc{Suppé, Adele von} (05.12.1879 – 04.08.1893)|pwv}, 14j. Mädel, ſchwerer
               Typhus. – Habe noch keine {\pb}Einberufung. – Notiz\pwindex{Man schreibt uns aus Ischl]@\emph{[Man schreibt uns aus Ischl]}|pwv} im B. B.\pwindex{Berliner Boersen-Courier@\emph{Berliner Börsen-Courier}|pw} geleſen; ſehr gut – aber natürlich »naturaliſtiſcher Dichter\pwindex{Berliner Boersen-Courier@\emph{Berliner Börsen-Courier}|pwv}«. – Geſtern war
               ich angeblich im \textsc{Szeps}\pwindex{Wiener Tagblatt@\emph{Wiener Tagblatt}|pwv}\pwindex{Szeps, Moriz 04.11.1834 – 09.08.1902@\textsc{Szeps, Moriz} (04.11.1834 – 09.08.1902), \emph{Journalist/Journalistin}|pw}{ }\label{K_L00246-1v}\edtext{verschimpfirt\pwindex{Saison in Ischl@\emph{Die Saison in Ischl}|pwv}}{\lemma{\textnormal{\emph{verschimpfirt}}}\Cendnote{\textnormal{In dem Bericht ohne Autornennung heißt
                  es: »Das Theaterleben ist ein sehr bewegtes, Tag für Tag Vorstellung,
                     berühmte und unberühmte Gäste, ja sogar Novitäten, sogenannte Sommer-Einakter,
                     die freilich oft nur aus Courtoisie aufgeführt werden. Ein realistisches
                     Stückchen ›Das Abschieds-Souper\pwindex{Abschiedssouper@\emph{Abschiedssouper}|pw}‹, aus der
                     Feder eines jungen Wien\oindex{Wien@\textbf{Wien}, \emph{A.ADM2}|pw}er Realisten hat wenig Erfolg gehabt, um
                     nicht zu sagen, gar keinen«. (\emph{Die Saison in Ischl}\pwindex{Saison in Ischl@\emph{Die Saison in Ischl}|pwk}. In: \emph{Wiener Tagblatt}\pwindex{Wiener Tagblatt@\emph{Wiener Tagblatt}|pwk}, Jg. 43, Nr. 206,
                        28. 7. 1893, S. 4.)}}}\label{K_L00246-1} (las es nicht) – nachdem ich
               vor 3 Tagen \label{K_L00246-2v}\edtext{gelobt}{\lemma{\textnormal{\emph{gelobt}}}\Cendnote{\textnormal{nicht nachweisbar}}}\label{K_L00246-2} war. Gute
               Redaction! – Was macht der Götterliebling\pwindex{Tod Georgs@\emph{Der Tod Georgs}|pw}? – Ist
                  Löbl\pwindex{Loebl, Emil 05.02.1863 – 26.08.1942@\textsc{Löbl, Emil} (05.02.1863 – 26.08.1942), \emph{Schriftsteller/Schriftstellerin, Journalist/Journalistin}|pw} noch in Iſchl\oindex{Bad Ischl@\textbf{Bad Ischl}, \emph{P.PPL}|pw}? Wohin ſchreibt man ihm? Las übrigens die Nu{\geminationm}er noch gar nicht. – Schreibt Loris\pwindex{Hofmannsthal, Hugo von 1874-02-01 – 1929-07-15@\textsc{Hofmannsthal, Hugo von} (1874-02-01 – 1929-07-15), \emph{Schriftsteller/Schriftstellerin}|pw}? – Grüßen Sie alles! \label{T_L00246-1v}\edtext{Ich würde mehr ſchreiben, we{\geminationn} ich
               nicht auf}{\lemma{\textnormal{\emph{Ich … auf}}}\Cendnote{\textnormal{quer am rechten Rand
                  weiter}}}\label{T_L00246-1}{ }\label{T_L00246-2v}\edtext{dieſem blöden Karterl angefangen
                  hätte.}{\lemma{\textnormal{\emph{dieſem … hätte.}}}\Cendnote{\textnormal{am linken Rand der
                  Vorderseite}}}\label{T_L00246-2}\pend
           \selectlanguage{ngerman}\endnumbering\briefempfaengerindex{Beer-Hofmann, Richard@\textsc{Beer-Hofmann, Richard}!zzzSchnitzler, Arthur@\emph{von Arthur Schnitzler}!1893-07-291@{29. 7. 1893}|)be}\mylabel{L00246h}  \normalsize

\doendnotes{C}
\bigskip
\vfill

\clearpage

\footnotesize

\lohead{\textsc{register}}

% Definiere theindex-Environment komplett neu ohne reledmac
\makeatletter
\renewenvironment{theindex}{%
  \section*{\indexname}%
  \setlength{\parindent}{0pt}%
  \setlength{\parskip}{0pt plus 0.3pt}%
  \let\item\@idxitem
}{%
  \clearpage
}
\makeatother

\IfFileExists{\jobname-pw.ind}{\input{\jobname-pw.ind}}{}

\end{document}

      