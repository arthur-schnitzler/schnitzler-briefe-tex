%% latex-leseansicht-vorspann.tex
%% Vorspann für die Leseansicht.
%% Lädt die gemeinsame Datei latex-vorspann.tex mit nicht gesetztem Schalter.

\newif\ifkorrekturansicht
\korrekturansichtfalse

\input{../tex-inputs/latex-vorspann}


\section[Arthur Schnitzler an Thomas Mann, 18. 11. 1929]{L02523 Arthur Schnitzler an Thomas Mann, 18. 11. 1929}
\nopagebreak\mylabel{L02523v}
\rehead{ }\normalsize\beginnumbering\briefempfaengerindex{Mann, Thomas@\textsc{Mann, Thomas}!zzzSchnitzler, Arthur@\emph{von Arthur Schnitzler}!1929-11-181@{18. 11. 1929}|(be}
\toendnotes[C]{\smallbreak\pagebreak[2]}
\correspDesc{Versand  durch Arthur Schnitzler am 18. 11. 1929 in Wien
\newline{}Erhalt  durch Thomas Mann im Zeitraum [19. 11. 1929 – 23. 11. 1929?] in München}\toendnotes[C]{\smallbreak}
\Standort{Zürich, Thomas-Mann-Archiv, B-II-SCHNM-4.}
\physDesc{Brief, 1 Blatt, 2 Seiten, Kuvert, 741 Zeichen (Briefpaper und Umschlag mit Trauerrand )
\newline{}Handschrift: schwarze Tinte, lateinische Kurrent
\newline{}Versand: Stempel: »\nobreak{}\oindex{XVIII., Währing@\textbf{XVIII., Währing}, \emph{Verwaltungsgebiet}|pwk}18/1 Wien 110, 18. XI. 29, 17\nobreak{}«.  }\pstart{}{\pb}\textcolor{gray}{\textbf{A. S.}}\pend{}\pstart{}\textcolor{gray}{\textbf{WIEN, XVIII.}}\oindex{XVIII., Währing@\textbf{XVIII., Währing}, \emph{Verwaltungsgebiet}|pw}\pend{}\pstart{}\textcolor{gray}{\textbf{STERNWARTESTR. 71}}\oindex{Wien@\textbf{Wien}!XVIII., Währing@\textbf{XVIII., Währing}!Sternwartestraße 71@\textbf{Sternwartestraße 71}, \emph{Wohngebäude}|pw}\pend{}{\bigskip}\pstart{}{\pb}Herrn Thomas Mann\pend{}\pstart{}München\oindex{München@\textbf{München}|pw}\pend{}\pstart{}Puschingerstr. 1\oindex{Poschingerstraße@\textbf{Poschingerstraße}, \emph{Straße}|pw}.\pend{}{\bigskip}\vspace{1em}
\pstart
           \raggedleft{}{\pb}Wien\oindex{Wien@\textbf{Wien}, \emph{Verwaltungsgebiet}|pw}, 18. 11. 924\pend
           
\pstart{}Mein lieber und verehrter Thomas Mann,\pend\vspace{0.5em}
\pstart
           Sie und der Nobelpreis\orgindex{Nobelpreis@Nobelpreis|pw} Sie gehören schon lang
               zusammen – womit ich keineswegs die Bedeutung von Preisen überhaupt überschätzen
               möchte. Trotzdem freut es Einen – und ich hoffe, auch Sie haben sich gefreut.\pend
           
\pstart
           Im übrigen glaub ich, dſs ich Ihnen weiter nicht viel sagen muſs. Sie wissen was Sie
               der Welt, – Sie wissen auch was {\pb}mir sind. Ich liebe Ihre Haltung, Ihr Werk, ich liebe Sie. Von meiner Bewunderung
               spreche ich nicht, – ich finde, hier ist beides, Bewunderung und Liebe eins.\pend
           
\pstart
           Bleiben Sie der Sie sind, und lange; damit ist auch etwas ausgedrückt, daſs Sie immer
               mehr werden.\pend
           
\pstart
           Glückwunsch und Gruß, und auf Wiedersehen, hoffentlich.\pend
           
\pstart
           Ihr{\\[\baselineskip]}\spacefill\mbox{ArthSchnitzler}\pend
           \leftskip=0em{}\selectlanguage{ngerman}\endnumbering\briefempfaengerindex{Mann, Thomas@\textsc{Mann, Thomas}!zzzSchnitzler, Arthur@\emph{von Arthur Schnitzler}!1929-11-181@{18. 11. 1929}|)be}\mylabel{L02523h}  \newcommand{\dateiname}{L02523}\newcommand{\titel}{Arthur Schnitzler an Thomas Mann, 18. 11. 1929}\newcommand{\editorInnen}{Martin Anton Müller und Gerd-Hermann Susen}%% latex-leseansicht-abspann.tex
%% Abspann für die Leseansicht.
%% Der Schalter \ifkorrekturansicht ist bereits durch den Vorspann gesetzt.

%% latex-abspann.tex
%% Gemeinsamer Abspann für Korrekturansicht und Leseansicht.
%% Setzt den Schalter \ifkorrekturansicht voraus (gesetzt in den
%% einbindenden Dateien latex-korrekturansicht-abspann.tex bzw.
%% latex-leseansicht-abspann.tex).
%% ---------------------------------------------------------------

\normalsize

% Das esempio-Environment wird nur in der Leseansicht benötigt
\ifkorrekturansicht\else
\newenvironment{esempio}[3]%
{
    \vspace{1.5ex}
    \rlap{\underline{#1}}
    \par
    \setlength{\parindent}{0cm}
    \nopagebreak
    \leftskip=#2cm
    \rightskip=#3cm
}
{
    \par
}
\fi

\doendnotes{C}
\bigskip
\vfill

\clearpage

\footnotesize

\ifkorrekturansicht
  \lohead{\textsc{register}}
\fi

% theindex-Environment neu definieren ohne reledmac
\makeatletter
\renewenvironment{theindex}{%
  \ifkorrekturansicht
    \section*{\indexname}%
  \else
    \subsubsection*{Index der erwähnten Entitäten}%
  \fi
  \setlength{\parindent}{0pt}%
  \setlength{\parskip}{0pt plus 0.3pt}%
  \let\item\@idxitem
}{%
  \ifkorrekturansicht\clearpage\fi
}
\makeatother

\IfFileExists{\jobname-pw.ind}{\input{\jobname-pw.ind}}{}

% Quellenangabe nur in der Leseansicht
\ifkorrekturansicht\else
% Fallback-Definitionen, falls die .tex-Datei \titel etc. nicht gesetzt hat
\providecommand{\titel}{}
\providecommand{\editorInnen}{}
\providecommand{\dateiname}{\jobname}

\vspace{3cm}

\vfill

\footnotesize
\textsc{Quelle}: \titel. Herausgegeben von {\editorInnen}. In: \emph{Arthur Schnitzler: Briefwechsel mit Autorinnen und Autoren}.
 Digitale Edition, https://schnitzler-briefe.acdh.oeaw.ac.at/{\dateiname}.html (Stand \today)
\fi

\end{document}


