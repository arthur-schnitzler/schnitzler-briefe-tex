%% latex-korrekturansicht-vorspann.tex
%% Vorspann für die Korrekturansicht.
%% Lädt die gemeinsame Datei latex-vorspann.tex mit gesetztem Schalter.

\newif\ifkorrekturansicht
\korrekturansichttrue

\input{../tex-inputs/latex-vorspann}


\section[Robert Adam an Arthur Schnitzler, 12. 6. 1918]{L02287 Robert Adam an Arthur Schnitzler, 12. 6. 1918}
\nopagebreak\mylabel{L02287v}
\rehead{ }\normalsize\beginnumbering\briefempfaengerindex{Schnitzler, Arthur@\textsc{Schnitzler, Arthur}!zzzAdam, Robert@\emph{von Robert Adam}!1918-06-121@{12. 6. 1918}|(be}
\toendnotes[C]{\smallbreak\pagebreak[2]}\Standort{CUL, Schnitzler, B 1.}
\physDesc{Brief, 1 Blatt, 2 Seiten, 1016 Zeichen
\newline{}Handschrift: schwarze Tinte, deutsche Kurrent
\newline{}Schnitzler: mit Bleistift beschriftet: »\textsc{Adam}« 
\newline{}Ordnung: von unbekannter Hand nummeriert: »3« }\Standort{Wien, Österreichische Nationalbibliothek, Cod.ser. 52.263, 214 recto.}
\physDesc{Brief, maschinenschriftliche Abschrift1 Blatt, 1 Seite, 1016 Zeichen
\newline{}Schreibmaschine}\toendnotes[C]{\smallbreak}
\pstart
           \raggedleft{}{\pb}Wien\oindex{Wien@\textbf{Wien}, \emph{A.ADM2}|pw}, am 12. Juni 1918\pend
           
\pstart\center{}Hochverehrter Herr Doktor!\pend\vspace{0.5em}
\pstart
           Ihre liebenswürdigen Zeilen haben mich außerordentlich erfreut (um nicht zu ſagen:
               gerührt). Ich hätte ſchon längſt wieder bei Ihnen vorgeſprochen, wüßte ich nicht aus
               Erfahrung, daß ein Beſuch ohne vorhergehende Anmeldung ein ausſichtsloſes Unternehmen
               ſei; und es ſchien mir anderſeits, als wäre eine ſolche Anmeldung, ohne daß ich Ihnen
               etwas Beſonderes mitzuteilen hätte, Arroganz und Beläſtigung. So hoffte ich, daß ich
               Sie entweder zufällig irgendwo träfe oder daß ſich mir ein Anlaß böte, Ihnen zu
               ſchreiben: beides iſt nicht eingetreten.\pend
           
\pstart
           Ich lebe monoton, verärgert und {\pb}deprimiert dahin. Gearbeitet habe ich gar nichts (wenn man von
               rechtsphiloſophiſchen und orientaliſchen Dingen abſieht).\pend
           
\pstart
           Darf ich alſo wieder einmal bei Ihnen erſcheinen? Ich möchte Sie gerne der Mühe des
               Schreibens entheben: wenn es Ihnen lieb iſt, könnten Sie mir den beſtimmten Tag
               telephoniſch (82202) mitteilen. (Telephon meiner Eltern\pwindex{Pollak, Emil 03.06.1841 – 21.01.1919@\textsc{Pollak, Emil} (03.06.1841 – 21.01.1919), \emph{Möbelfabrikant/Möbelfabrikantin}|pwv}\pwindex{Pollak, Sidonie 05.04.1851 – 07.04.1928@\textsc{Pollak, Sidonie} (05.04.1851 – 07.04.1928)|pwv}).\pend
           
\pstart
           Mit beſtem Dank und ergebenſten Grüßen Ihr\pend
           \pstart \spacefill\mbox{Robert Adam}\pend{}\selectlanguage{ngerman}\endnumbering\briefempfaengerindex{Schnitzler, Arthur@\textsc{Schnitzler, Arthur}!zzzAdam, Robert@\emph{von Robert Adam}!1918-06-121@{12. 6. 1918}|)be}\mylabel{L02287h}  \normalsize

\doendnotes{C}
\bigskip
\vfill

\clearpage

\footnotesize

\lohead{\textsc{register}}

% Definiere theindex-Environment komplett neu ohne reledmac
\makeatletter
\renewenvironment{theindex}{%
  \section*{\indexname}%
  \setlength{\parindent}{0pt}%
  \setlength{\parskip}{0pt plus 0.3pt}%
  \let\item\@idxitem
}{%
  \clearpage
}
\makeatother

\IfFileExists{\jobname-pw.ind}{\input{\jobname-pw.ind}}{}

\end{document}

      