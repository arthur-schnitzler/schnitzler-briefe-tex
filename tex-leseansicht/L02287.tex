%% latex-leseansicht-vorspann.tex
%% Vorspann für die Leseansicht.
%% Lädt die gemeinsame Datei latex-vorspann.tex mit nicht gesetztem Schalter.

\newif\ifkorrekturansicht
\korrekturansichtfalse

\input{../tex-inputs/latex-vorspann}


\section[Robert Adam an Arthur Schnitzler, 12. 6. 1918]{L02287 Robert Adam an Arthur Schnitzler, 12. 6. 1918}
\nopagebreak\mylabel{L02287v}
\rehead{ }\normalsize\beginnumbering\briefempfaengerindex{Schnitzler, Arthur@\textsc{Schnitzler, Arthur}!zzzAdam, Robert@\emph{von Robert Adam}!1918-06-121@{12. 6. 1918}|(be}
\toendnotes[C]{\smallbreak\pagebreak[2]}
\correspDesc{Versand  durch Robert Adam am 12. 6. 1918 in Wien
\newline{}Erhalt  durch Arthur Schnitzler im Zeitraum [12. 6. 1918
                  – 16. 6. 1918?] in Wien}\toendnotes[C]{\smallbreak}
\Standort{CUL, Schnitzler, B 1.}
\physDesc{Brief, 1 Blatt, 2 Seiten, 1016 Zeichen
\newline{}Handschrift: schwarze Tinte, deutsche Kurrent
\newline{}Schnitzler: mit Bleistift beschriftet: »\textsc{Adam}« 
\newline{}Ordnung: von unbekannter Hand nummeriert: »3« }\Standort{Wien, Österreichische Nationalbibliothek, Cod.ser. 52.263, 214 recto.}
\physDesc{Brief, maschinenschriftliche Abschrift, 1 Blatt, 1 Seite, 1016 Zeichen
\newline{}Schreibmaschine}\toendnotes[C]{\smallbreak}
\pstart
           \raggedleft{}{\pb}Wien\oindex{Wien@\textbf{Wien}, \emph{Verwaltungsgebiet}|pw}, am 12. Juni 1918\pend
           
\pstart\center{}Hochverehrter Herr Doktor!\pend\vspace{0.5em}
\pstart
           Ihre liebenswürdigen Zeilen haben mich außerordentlich erfreut (um nicht zu{ }ſagen:
               gerührt). Ich hätte{ }ſchon längſt wieder bei Ihnen vorgeſprochen, wüßte ich nicht aus
               Erfahrung, daß ein Beſuch ohne vorhergehende Anmeldung ein ausſichtsloſes Unternehmen{ }ſei; und es{ }ſchien mir anderſeits, als wäre eine{ }ſolche Anmeldung, ohne daß ich Ihnen
               etwas Beſonderes mitzuteilen hätte, Arroganz und Beläſtigung. So hoffte ich, daß ich
               Sie entweder zufällig irgendwo träfe oder daß{ }ſich mir ein Anlaß böte, Ihnen zu{ }ſchreiben: beides iſt nicht eingetreten.\pend
           
\pstart
           Ich lebe monoton, verärgert und {\pb}deprimiert dahin. Gearbeitet habe ich gar nichts (wenn man von
               rechtsphiloſophiſchen und orientaliſchen Dingen abſieht).\pend
           
\pstart
           Darf ich alſo wieder einmal bei Ihnen erſcheinen? Ich möchte Sie gerne der Mühe des
               Schreibens entheben: wenn es Ihnen lieb iſt, könnten Sie mir den beſtimmten Tag
               telephoniſch (82202) mitteilen. (Telephon meiner Eltern\pwindex{Pollak, Emil 3.\,6.\,1841 Prag – 21.\,1.\,1919 Wien@\textsc{Pollak, Emil} (3.\,6.\,1841 Prag – 21.\,1.\,1919 Wien), \emph{Möbelfabrikant}|pwv}\pwindex{Pollak, Sidonie 5.\,4.\,1851 Budapest – 7.\,4.\,1928 Wien@\textsc{Pollak, Sidonie} (5.\,4.\,1851 Budapest – 7.\,4.\,1928 Wien)|pwv}).\pend
           
\pstart
           Mit beſtem Dank und ergebenſten Grüßen Ihr\pend
           \pstart \spacefill\mbox{Robert Adam}\pend{}\selectlanguage{ngerman}\endnumbering\briefempfaengerindex{Schnitzler, Arthur@\textsc{Schnitzler, Arthur}!zzzAdam, Robert@\emph{von Robert Adam}!1918-06-121@{12. 6. 1918}|)be}\mylabel{L02287h}  \newcommand{\dateiname}{L02287}\newcommand{\titel}{Robert Adam an Arthur Schnitzler, 12. 6. 1918}\newcommand{\editorInnen}{Martin Anton Müller und Gerd-Hermann Susen}%% latex-leseansicht-abspann.tex
%% Abspann für die Leseansicht.
%% Der Schalter \ifkorrekturansicht ist bereits durch den Vorspann gesetzt.

%% latex-abspann.tex
%% Gemeinsamer Abspann für Korrekturansicht und Leseansicht.
%% Setzt den Schalter \ifkorrekturansicht voraus (gesetzt in den
%% einbindenden Dateien latex-korrekturansicht-abspann.tex bzw.
%% latex-leseansicht-abspann.tex).
%% ---------------------------------------------------------------

\normalsize

% Das esempio-Environment wird nur in der Leseansicht benötigt
\ifkorrekturansicht\else
\newenvironment{esempio}[3]%
{
    \vspace{1.5ex}
    \rlap{\underline{#1}}
    \par
    \setlength{\parindent}{0cm}
    \nopagebreak
    \leftskip=#2cm
    \rightskip=#3cm
}
{
    \par
}
\fi

\doendnotes{C}
\bigskip
\vfill

\clearpage

\footnotesize

\ifkorrekturansicht
  \lohead{\textsc{register}}
\fi

% theindex-Environment neu definieren ohne reledmac
\makeatletter
\renewenvironment{theindex}{%
  \ifkorrekturansicht
    \section*{\indexname}%
  \else
    \subsubsection*{Index der erwähnten Entitäten}%
  \fi
  \setlength{\parindent}{0pt}%
  \setlength{\parskip}{0pt plus 0.3pt}%
  \let\item\@idxitem
}{%
  \ifkorrekturansicht\clearpage\fi
}
\makeatother

\IfFileExists{\jobname-pw.ind}{\input{\jobname-pw.ind}}{}

% Quellenangabe nur in der Leseansicht
\ifkorrekturansicht\else
% Fallback-Definitionen, falls die .tex-Datei \titel etc. nicht gesetzt hat
\providecommand{\titel}{}
\providecommand{\editorInnen}{}
\providecommand{\dateiname}{\jobname}

\vspace{3cm}

\vfill

\footnotesize
\textsc{Quelle}: \titel. Herausgegeben von {\editorInnen}. In: \emph{Arthur Schnitzler: Briefwechsel mit Autorinnen und Autoren}.
 Digitale Edition, https://schnitzler-briefe.acdh.oeaw.ac.at/{\dateiname}.html (Stand \today)
\fi

\end{document}


