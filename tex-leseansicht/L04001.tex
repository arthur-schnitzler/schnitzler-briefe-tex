%% latex-leseansicht-vorspann.tex
%% Vorspann für die Leseansicht.
%% Lädt die gemeinsame Datei latex-vorspann.tex mit nicht gesetztem Schalter.

\newif\ifkorrekturansicht
\korrekturansichtfalse

\input{../tex-inputs/latex-vorspann}


\section[Berta Zuckerkandl an Arthur Schnitzler, 27. 2. 192[3?]]{L04001 Berta Zuckerkandl an Arthur Schnitzler, 27. 2. 192[3?]}
\nopagebreak\mylabel{L04001v}
\rehead{ }\normalsize\beginnumbering\briefempfaengerindex{Schnitzler, Arthur@\textsc{Schnitzler, Arthur}!zzzZuckerkandl, Berta@\emph{von Berta Zuckerkandl}!1923-02-271@{27. 2. 192[3?]}|(be}
\toendnotes[C]{\smallbreak\pagebreak[2]}
\correspDesc{Versand  durch Berta Zuckerkandl am 27. 2. 192[3?] in Paris
\newline{}Erhalt  durch Arthur Schnitzler im Zeitraum [28. 2. 1923
                  – 4. 3. 1923?] in Wien}\toendnotes[C]{\smallbreak}
\Standort{CUL, Schnitzler, B 200.}
\physDesc{Kartenbrief, 564 Zeichen
\newline{}Handschrift: schwarze Tinte, lateinische Kurrent
\newline{}Versand: 1) Stempel: »\nobreak{}\oindex{place Chopin@\textbf{place Chopin}, \emph{Platz}|pwk}{[}Place{]} Chopin , {[}27{]}.{ }II
                                          {[}1{]}\textcolor{gray}{9}23, \nobreak{}«.   2) Stempel: »\nobreak{}\textcolor{gray}{co}ller le timbre en haut {\kaufmannsund} à droite de l’enveloppe
                                    \nobreak{}«. }\toendnotes[C]{\smallbreak}\pstart{}{\pb}Berthe Zuckerkandl-Szeps.\pend{}\pstart{}12. Avenue d’Eylau\oindex{12, Avenue d’Eylau@\textbf{12, Avenue d’Eylau}, \emph{Wohngebäude}|pw}.\pend{}\pstart{}Paris\oindex{Paris@\textbf{Paris}, \emph{Hauptstadt}|pw}. \pend{}{\bigskip}\pstart{}Monsieur Artur Schnitzler\pend{}\pstart{}XVIII. Sternwartestrasse 71\oindex{Wien@\textbf{Wien}!XVIII., Währing@\textbf{XVIII., Währing}!Sternwartestraße 71@\textbf{Sternwartestraße 71}, \emph{Wohngebäude}|pw}\pend{}\pstart{}Vienne\oindex{Wien@\textbf{Wien}, \emph{Verwaltungsgebiet}|pw}\pend{}\pstart{}Autriche\oindex{Österreich-Ungarn@\textbf{Österreich-Ungarn}|pw}.\pend{}{\bigskip}\vspace{1em}
\pstart
           {\pb}\label{K_L04001-1v}\edtext{27. F. 1922}{\lemma{\textnormal{\emph{27. F. 1922}}}\Cendnote{\textnormal{Berta Zuckerkandl\pwindex{Zuckerkandl, Berta 13.\,4.\,1864 Wien – 16.\,10.\,1945 Paris@\textsc{Zuckerkandl, Berta} (13.\,4.\,1864 Wien – 16.\,10.\,1945 Paris), \emph{Schriftstellerin, Journalistin, Übersetzerin}|pwk} verschrieb sich bei der
                     Jahreszahl: Der Brief wurde am 27. 2. 1923 verfasst. Dafür spricht
                     neben den erkennbaren Resten des Poststempels, dass Schnitzler im Brief vom XXXX Auszeichnungsfehler: Dokument L03946 nicht gefunden antwortete: »Für Ihre
                        bisherigen Bemühungen, verehrteste Frau Hofrätin, mit Mme. Cabir\pwindex{Cabire, Emma @\textsc{Cabire, Emma}, \emph{Übersetzerin, Redakteurin, Literaturagentin}|pw} und M. Hella\pwindex{Hella, Alzir 30.\,12.\,1881 Vieux Condé – 14.\,7.\,1953 Paris@\textsc{Hella, Alzir} (30.\,12.\,1881 Vieux Condé – 14.\,7.\,1953 Paris), \emph{Übersetzer}|pw} sage ich Ihnen vielen Dank«, und dabei Bezug nahm auf
                     die beiden in diesem Brief erwähnten Übersetzer\pwindex{Cabire, Emma @\textsc{Cabire, Emma}, \emph{Übersetzerin, Redakteurin, Literaturagentin}|pwkv}\pwindex{Hella, Alzir 30.\,12.\,1881 Vieux Condé – 14.\,7.\,1953 Paris@\textsc{Hella, Alzir} (30.\,12.\,1881 Vieux Condé – 14.\,7.\,1953 Paris), \emph{Übersetzer}|pwkv}. Auch sind im
                        \emph{Tagebuch}\pwindex{Schnitzler, Arthur 15. 5. 1862 Wien – 21. 10. 1931 ebd.@\textsc{Schnitzler, Arthur} (15. 5. 1862 Wien – 21. 10. 1931 ebd.), \emph{Schriftsteller, Mediziner}!Tagebuch@\strich\emph{Tagebuch}|pwk} Treffen vor (14. 2. 1923) und
                     nach (23. 4. 1923) Zuckerkandls\pwindex{Zuckerkandl, Berta 13.\,4.\,1864 Wien – 16.\,10.\,1945 Paris@\textsc{Zuckerkandl, Berta} (13.\,4.\,1864 Wien – 16.\,10.\,1945 Paris), \emph{Schriftstellerin, Journalistin, Übersetzerin}|pwk}{ }Paris\oindex{Paris@\textbf{Paris}, \emph{Hauptstadt}|pwk}reise dokumentiert, bei denen die französischen\oindex{Frankreich@\textbf{Frankreich}|pwk} Angelegenheiten besprochen
                     wurden, während im Frühjahr 1922 kein Austausch über das Thema
                     überliefert ist.}}}\label{K_L04001-1}.\pend
           
\pstart
           \raggedleft{}\textcolor{gray}{\textbf{12 AVENUE D' EYLAU. XVI\textsuperscript{E}}}\oindex{12, Avenue d’Eylau@\textbf{12, Avenue d’Eylau}, \emph{Wohngebäude}|pw}\pend
           \vspace{0.5em}
\pstart
           Verehrter Freund! Da Sie mir aufgetragen hatten zu warten bis sich
                  Hella\pwindex{Hella, Alzir 30.\,12.\,1881 Vieux Condé – 14.\,7.\,1953 Paris@\textsc{Hella, Alzir} (30.\,12.\,1881 Vieux Condé – 14.\,7.\,1953 Paris), \emph{Übersetzer}|pw} und Madame Cabire\pwindex{Cabire, Emma @\textsc{Cabire, Emma}, \emph{Übersetzerin, Redakteurin, Literaturagentin}|pw}{ }\label{K_L04001-2v}\edtext{melden würden}{\lemma{\textnormal{\emph{melden würden}}}\Cendnote{\textnormal{Im Nachlass Schnitzlers
                  befindet sich der Durchschlag eines Briefes an den Übersetzer Alzir Hella\pwindex{Hella, Alzir 30.\,12.\,1881 Vieux Condé – 14.\,7.\,1953 Paris@\textsc{Hella, Alzir} (30.\,12.\,1881 Vieux Condé – 14.\,7.\,1953 Paris), \emph{Übersetzer}|pwk}: »19. 2. 1923{ / }Sehr geehrter Herr Hella\pwindex{Hella, Alzir 30.\,12.\,1881 Vieux Condé – 14.\,7.\,1953 Paris@\textsc{Hella, Alzir} (30.\,12.\,1881 Vieux Condé – 14.\,7.\,1953 Paris), \emph{Übersetzer}|pw}.{ / }In den nächsten Tagen kommt Frau Hofrätin Bertha Zuckerkandl\pwindex{Zuckerkandl, Berta 13.\,4.\,1864 Wien – 16.\,10.\,1945 Paris@\textsc{Zuckerkandl, Berta} (13.\,4.\,1864 Wien – 16.\,10.\,1945 Paris), \emph{Schriftstellerin, Journalistin, Übersetzerin}|pw} nach Paris\oindex{Paris@\textbf{Paris}, \emph{Hauptstadt}|pw}
                        und wird dort bei ihrer Schwester, MMe.
                           Paul Clemenceau\pwindex{Clemenceau, Sophie 25.\,5.\,1862 – 24.\,9.\,1937@\textsc{Clemenceau, Sophie} (25.\,5.\,1862 – 24.\,9.\,1937)|pw}, 12, Avenue
                           d’Eylau\oindex{12, Avenue d’Eylau@\textbf{12, Avenue d’Eylau}, \emph{Wohngebäude}|pw} wohnen. Darf ich Sie bitten sich mit ihr in Verbindung zu
                           setzen{[},{]} ich habe ihr von Ihrem freundlichen Antrag
                        Mitteilung gemacht und sie ermächtigt mit Ihnen weiter darüber zu
                        unterhandeln. Es wäre mir natürlich sehr willkommen, wenn eine meiner
                        Novellen in ›Monde Nouveau\pwindex{Monde nouveau@\emph{Monde nouveau}|pw}‹ zum
                        Abdruck käme. ›Casanovas Heimfahrt\pwindex{Schnitzler, Arthur 15. 5. 1862 Wien – 21. 10. 1931 ebd.@\textsc{Schnitzler, Arthur} (15. 5. 1862 Wien – 21. 10. 1931 ebd.), \emph{Schriftsteller, Mediziner}!Casanovas Heimfahrt@\strich\emph{Casanovas Heimfahrt}|pw}‹ ist
                        nicht frei, aber vielleicht erlange ich mein Rechte auch auf diese Novelle
                        wieder zurück, da der Bewerber\pwindex{?? [Französischer Übersetzer, der Casanovas Heimfahrt übersetzen will, 1293] @\textsc{?? [Französischer Übersetzer, der Casanovas Heimfahrt übersetzen will, 1293]}|pwv} bisher meines Wissens die Uebersetzung nicht in Angriff
                        genommen hat. Ueber die Honorarbedingungen wird Frau Hofrätin Zuckerkandl\pwindex{Zuckerkandl, Berta 13.\,4.\,1864 Wien – 16.\,10.\,1945 Paris@\textsc{Zuckerkandl, Berta} (13.\,4.\,1864 Wien – 16.\,10.\,1945 Paris), \emph{Schriftstellerin, Journalistin, Übersetzerin}|pw} mit Ihnen reden.{ / }Mit verbindlichem Dank für Ihr freundliches Interesse und Ihre
                        liebenswürdigen Worte{ / }Ihr sehr ergebener{ / }{[}Raum für die Unterschrift{]}{ / }Herrn Alzir Hella\pwindex{Hella, Alzir 30.\,12.\,1881 Vieux Condé – 14.\,7.\,1953 Paris@\textsc{Hella, Alzir} (30.\,12.\,1881 Vieux Condé – 14.\,7.\,1953 Paris), \emph{Übersetzer}|pw}, Paris, 18, rue de l’Odéon\oindex{18, rue de l’Odéon@\textbf{18, rue de l’Odéon}, \emph{Wohngebäude}|pw}« (Brief von Schnitzler an Alzir Hella\pwindex{Hella, Alzir 30.\,12.\,1881 Vieux Condé – 14.\,7.\,1953 Paris@\textsc{Hella, Alzir} (30.\,12.\,1881 Vieux Condé – 14.\,7.\,1953 Paris), \emph{Übersetzer}|pwk}, 19. 2. 1923,
                        \emph{DLA}, HS.1985.1.969). Wie aus einem Brief Fischers\pwindex{Fischer, Samuel 24.\,12.\,1859 Liptovský Mikuláš – 15.\,10.\,1934 Berlin@\textsc{Fischer, Samuel} (24.\,12.\,1859 Liptovský Mikuláš – 15.\,10.\,1934 Berlin), \emph{Verleger}|pwk} an Schnitzler hervorgeht, hatte sich Hella\pwindex{Hella, Alzir 30.\,12.\,1881 Vieux Condé – 14.\,7.\,1953 Paris@\textsc{Hella, Alzir} (30.\,12.\,1881 Vieux Condé – 14.\,7.\,1953 Paris), \emph{Übersetzer}|pwk} zu Beginn des Jahres zunächst an den \emph{Fischerverlag}\orgindex{S. Fischer Verlag@S. Fischer Verlag|pwk} gewand (vgl. Arthur Schnitzler: \emph{Mikrofilme}, \url{https://schnitzler\_mikrofilme.acdh.oeaw.ac.at/1416743\_0576}), wenig später kontaktierte er Schnitzler direkt (vgl. Arthur Schnitzler: \emph{Mikrofilme}, \url{https://schnitzler\_mikrofilme.acdh.oeaw.ac.at/1416739\_0263}). Er
                  übersetzte schließlich gemeinsam mit Olivier
                     Bournac\pwindex{Bournac, Olivier 13.\,8.\,1885 Saint-Amans-du-Pech – Anfang Januar 1931 Toulon@\textsc{Bournac, Olivier} (13.\,8.\,1885 Saint-Amans-du-Pech – Anfang Januar 1931 Toulon), \emph{Schriftsteller, Übersetzer}|pwk} drei Texte von Schnitzler.
                  Als erstes erschien 1925 mit \emph{Mourir}\pwindex{Schnitzler, Arthur 15. 5. 1862 Wien – 21. 10. 1931 ebd.@\textsc{Schnitzler, Arthur} (15. 5. 1862 Wien – 21. 10. 1931 ebd.), \emph{Schriftsteller, Mediziner}!Mourir. Roman [1925]@\strich\emph{Mourir. Roman [1925]}|pwk} eine Neuübersetzung von \emph{Sterben}\pwindex{Schnitzler, Arthur 15. 5. 1862 Wien – 21. 10. 1931 ebd.@\textsc{Schnitzler, Arthur} (15. 5. 1862 Wien – 21. 10. 1931 ebd.), \emph{Schriftsteller, Mediziner}!Sterben. Novelle@\strich\emph{Sterben. Novelle}|pwk}, danach kamen noch \emph{Madame Beate
                     et son fils}\pwindex{Schnitzler, Arthur 15. 5. 1862 Wien – 21. 10. 1931 ebd.@\textsc{Schnitzler, Arthur} (15. 5. 1862 Wien – 21. 10. 1931 ebd.), \emph{Schriftsteller, Mediziner}!Madame Beate et son fils@\strich\emph{Madame Beate et son fils}|pwk} (Oktober–November 1928) und \emph{Le Célibataire}\pwindex{Schnitzler, Arthur 15. 5. 1862 Wien – 21. 10. 1931 ebd.@\textsc{Schnitzler, Arthur} (15. 5. 1862 Wien – 21. 10. 1931 ebd.), \emph{Schriftsteller, Mediziner}!Le Célibataire@\strich\emph{Le Célibataire}|pwk} (\emph{Der Tod
                     des Junggesellen}\pwindex{Schnitzler, Arthur 15. 5. 1862 Wien – 21. 10. 1931 ebd.@\textsc{Schnitzler, Arthur} (15. 5. 1862 Wien – 21. 10. 1931 ebd.), \emph{Schriftsteller, Mediziner}!Tod des Junggesellen. Novelle@\strich\emph{Der Tod des Junggesellen. Novelle}|pwk}, März 1929). Emma Cabire\pwindex{Cabire, Emma @\textsc{Cabire, Emma}, \emph{Übersetzerin, Redakteurin, Literaturagentin}|pwk} wurde mit der Übersetzung\pwindex{Schnitzler, Arthur 15. 5. 1862 Wien – 21. 10. 1931 ebd.@\textsc{Schnitzler, Arthur} (15. 5. 1862 Wien – 21. 10. 1931 ebd.), \emph{Schriftsteller, Mediziner}!Le Pays de l’âme. Drame en 5 actes@\strich\emph{Le Pays de l’âme. Drame en 5 actes}|pwkv} von \emph{Das
                     weite Land}\pwindex{Schnitzler, Arthur 15. 5. 1862 Wien – 21. 10. 1931 ebd.@\textsc{Schnitzler, Arthur} (15. 5. 1862 Wien – 21. 10. 1931 ebd.), \emph{Schriftsteller, Mediziner}!weite Land. Tragikomödie in fünf Akten@\strich\emph{Das weite Land. Tragikomödie in fünf Akten}|pwk} betraut.}}}\label{K_L04001-2}, so – wartete ich. Nun aber – schreibe ich
               beiden umgehend u bestellte sie zu mir – da die Zeit hier so rasch vergeht dass man
               sie wirklich nicht verlieren darf. – Hier allgemeine Misstim̅ung, Angst – Unruhe. Und
                  \label{K_L04001-3v}\edtext{Viele – die – missbilligen}{\lemma{\textnormal{\emph{Viele … missbilligen}}}\Cendnote{\textnormal{Seit dem 11. 1. 1923 hatten
                     französische\oindex{Frankreich@\textbf{Frankreich}|pwk} Truppen im Streit um die durch
                  den Versailler\oindex{Versailles@\textbf{Versailles}, \emph{Hauptstadt}|pwk} Vertrag festgesetzten deutschen\oindex{Deutschland@\textbf{Deutschland}|pwk} Reparationszahlungen das Ruhrgebiet\oindex{Ruhrgebiet@\textbf{Ruhrgebiet}, \emph{Region}|pwk} besetzt, was auch in Frankreich\oindex{Frankreich@\textbf{Frankreich}|pwk} kontrovers diskutiert wurde.}}}\label{K_L04001-3}.
               Aber schweigen weil es noch zu \uline{früh} wäre – zu
               sprechen.\pend
           \pstart Alles Herzlichste von Ihrer getreuen \spacefill\mbox{B. Z.}\pend{}\selectlanguage{ngerman}\endnumbering\briefempfaengerindex{Schnitzler, Arthur@\textsc{Schnitzler, Arthur}!zzzZuckerkandl, Berta@\emph{von Berta Zuckerkandl}!1923-02-271@{27. 2. 192[3?]}|)be}\mylabel{L04001h}
\begin{anhang}
\end{anhang}\newcommand{\dateiname}{L04001}\newcommand{\titel}{Berta Zuckerkandl an Arthur Schnitzler, 27. 2. 192[3?]}\newcommand{\editorInnen}{Herausgegeben von Jahnke, SelmaMüller, Martin Anton}%% latex-leseansicht-abspann.tex
%% Abspann für die Leseansicht.
%% Der Schalter \ifkorrekturansicht ist bereits durch den Vorspann gesetzt.

%% latex-abspann.tex
%% Gemeinsamer Abspann für Korrekturansicht und Leseansicht.
%% Setzt den Schalter \ifkorrekturansicht voraus (gesetzt in den
%% einbindenden Dateien latex-korrekturansicht-abspann.tex bzw.
%% latex-leseansicht-abspann.tex).
%% ---------------------------------------------------------------

\normalsize

% Das esempio-Environment wird nur in der Leseansicht benötigt
\ifkorrekturansicht\else
\newenvironment{esempio}[3]%
{
    \vspace{1.5ex}
    \rlap{\underline{#1}}
    \par
    \setlength{\parindent}{0cm}
    \nopagebreak
    \leftskip=#2cm
    \rightskip=#3cm
}
{
    \par
}
\fi

\doendnotes{C}
\bigskip
\vfill

\clearpage

\footnotesize

\ifkorrekturansicht
  \lohead{\textsc{register}}
\fi

% theindex-Environment neu definieren ohne reledmac
\makeatletter
\renewenvironment{theindex}{%
  \ifkorrekturansicht
    \section*{\indexname}%
  \else
    \subsubsection*{Index der erwähnten Entitäten}%
  \fi
  \setlength{\parindent}{0pt}%
  \setlength{\parskip}{0pt plus 0.3pt}%
  \let\item\@idxitem
}{%
  \ifkorrekturansicht\clearpage\fi
}
\makeatother

\IfFileExists{\jobname-pw.ind}{\input{\jobname-pw.ind}}{}

% Quellenangabe nur in der Leseansicht
\ifkorrekturansicht\else
% Fallback-Definitionen, falls die .tex-Datei \titel etc. nicht gesetzt hat
\providecommand{\titel}{}
\providecommand{\editorInnen}{}
\providecommand{\dateiname}{\jobname}

\vspace{3cm}

\vfill

\footnotesize
\textsc{Quelle}: \titel. Herausgegeben von {\editorInnen}. In: \emph{Arthur Schnitzler: Briefwechsel mit Autorinnen und Autoren}.
 Digitale Edition, https://schnitzler-briefe.acdh.oeaw.ac.at/{\dateiname}.html (Stand \today)
\fi

\end{document}


