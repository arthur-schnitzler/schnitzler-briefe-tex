%% latex-leseansicht-vorspann.tex
%% Vorspann für die Leseansicht.
%% Lädt die gemeinsame Datei latex-vorspann.tex mit nicht gesetztem Schalter.

\newif\ifkorrekturansicht
\korrekturansichtfalse

\input{../tex-inputs/latex-vorspann}


\section[ Paul Goldmann an Arthur Schnitzler, 11. 12. {[}1900{]}]{L02945 Paul Goldmann an Arthur Schnitzler,  11. 12. [1900]}
\nopagebreak\mylabel{L02945v}
\rehead{ }\normalsize\beginnumbering\briefempfaengerindex{Schnitzler, Arthur@\textsc{Schnitzler, Arthur}!zzzGoldmann, Paul@\emph{von Paul Goldmann}!1900-12-111@{11. 12. [1900]}|(be}
\toendnotes[C]{\smallbreak\pagebreak[2]}
\correspDesc{Versand  durch Paul Goldmann am 11. 12. [1900] in Berlin
\newline{}Erhalt  durch Arthur Schnitzler im Zeitraum [12. 12. 1900 – 16. 12. 1900?] in Wien}\toendnotes[C]{\smallbreak}
\Standort{DLA, A:Schnitzler, HS.NZ85.1.3170.}
\physDesc{Brief, 1 Blatt, 3 Seiten, 658 Zeichen
\newline{}Handschrift: blaue Tinte, deutsche Kurrent
\newline{}Schnitzler: 1) mit Bleistift das Jahr »900.« vermerkt  2) mit rotem Buntstift eine Unterstreichung}\toendnotes[C]{\smallbreak}
\pstart
           \raggedleft{}{\pb}\textcolor{gray}{\textbf{DESSAUERSTRASSE 19}}\oindex{Dessauer Straße@\textbf{Dessauer Straße}, \emph{Straße}|pw}\pend
           
\pstart
           Berlin\oindex{Berlin@\textbf{Berlin}, \emph{Hauptstadt}|pw}, 11. December.\pend
           
\pstart\center{}Mein lieber Freund,\pend\vspace{0.5em}
\pstart
           Gewiß, die N. Fr. Pr.\orgindex{Neue Freie Presse@Neue Freie Presse|pw} hat{ }ſich \label{K_L02945-1v}\edtext{niederträchtig benommen}{\lemma{\textnormal{\emph{niederträchtig benommen}}}\Cendnote{\textnormal{Bezug auf die Berichterstattung\pwindex{Man telegraphirt uns aus Breslau…]@\emph{[Man telegraphirt uns aus Breslau…]}|pwkv} der \emph{Neuen Freien Presse}\orgindex{Neue Freie Presse@Neue Freie Presse|pwk} über die Uraufführung von \emph{Der Schleier der Beatrice}\pwindex{Schnitzler, Arthur 15.\,5.\,1862 Wien – 21.\,10.\,1931 ebd.@\textsc{Schnitzler, Arthur} (15.\,5.\,1862 Wien – 21.\,10.\,1931 ebd.), \emph{Schriftsteller, Mediziner}!Schleier der Beatrice. Schauspiel in fünf Akten@\strich\emph{Der Schleier der Beatrice. Schauspiel in fünf Akten}|pwk}\eventindex{Lobe-Theater@\textbf{Lobe-Theater}!Uraufführung von Der Schleier der Beatrice, 1.12.1900@Uraufführung von Der Schleier der Beatrice, 1.12.1900|pwk}, siehe XXXX Auszeichnungsfehler: Dokument L02943 nicht gefunden und XXXX Auszeichnungsfehler: Dokument L02944 nicht gefunden.
               }}}\label{K_L02945-1}. Ob man dagegen nichts thun kann? \strikeout{Jawoh}
               Jawohl. Beiſpielsweiſe: Schreib’ an das Blatt\orgindex{Neue Freie Presse@Neue Freie Presse|pwv} einen Brief, worin Du mittheilſt, daß Du wegen der Dir
               gegenüber bewieſenen niederträchtigen Parteilichkeit die für die {\pb}Weihnachtsnummer\pwindex{Neue Freie Presse@\emph{Neue Freie Presse}|pwv} beſtimmte Novelle\pwindex{Schnitzler, Arthur 15.\,5.\,1862 Wien – 21.\,10.\,1931 ebd.@\textsc{Schnitzler, Arthur} (15.\,5.\,1862 Wien – 21.\,10.\,1931 ebd.), \emph{Schriftsteller, Mediziner}!Lieutenant Gustl. Novelle@\strich\emph{Lieutenant Gustl. Novelle}|pwv}{ }\label{K_L02945-2v}\edtext{zurückziehſt}{\lemma{\textnormal{\emph{zurückziehst}}}\Cendnote{\textnormal{In der Korrespondenz mit Theodor Herzl\pwindex{Herzl, Theodor 2.\,5.\,1860 Budapest – 3.\,7.\,1904 Edlach@\textsc{Herzl, Theodor} (2.\,5.\,1860 Budapest – 3.\,7.\,1904 Edlach), \emph{Schriftsteller, Journalist}|pwk}, mit dem Schnitzler
                  die Aufnahme von \emph{Lieutenant Gustl}\pwindex{Schnitzler, Arthur 15.\,5.\,1862 Wien – 21.\,10.\,1931 ebd.@\textsc{Schnitzler, Arthur} (15.\,5.\,1862 Wien – 21.\,10.\,1931 ebd.), \emph{Schriftsteller, Mediziner}!Lieutenant Gustl. Novelle@\strich\emph{Lieutenant Gustl. Novelle}|pwk} in der \emph{Neuen Freien Presse}\pwindex{Neue Freie Presse@\emph{Neue Freie Presse}|pwk} verhandelte, ist von der
                     Berichterstattung\pwindex{Man telegraphirt uns aus Breslau…]@\emph{[Man telegraphirt uns aus Breslau…]}|pwkv} über
                  die Uraufführung von \emph{Der Schleier der
                     Beatrice}\pwindex{Schnitzler, Arthur 15.\,5.\,1862 Wien – 21.\,10.\,1931 ebd.@\textsc{Schnitzler, Arthur} (15.\,5.\,1862 Wien – 21.\,10.\,1931 ebd.), \emph{Schriftsteller, Mediziner}!Schleier der Beatrice. Schauspiel in fünf Akten@\strich\emph{Der Schleier der Beatrice. Schauspiel in fünf Akten}|pwk}\eventindex{Lobe-Theater@\textbf{Lobe-Theater}!Uraufführung von Der Schleier der Beatrice, 1.12.1900@Uraufführung von Der Schleier der Beatrice, 1.12.1900|pwk} nicht die Rede.}}}\label{K_L02945-2}. Das wäre eine Lektion. Aber wenn Ihr
               Unabhängigen nichts gegen das Blatt\orgindex{Neue Freie Presse@Neue Freie Presse|pwv} thun wollt, was{ }ſollen dann wir Abhängigen thun?\pend
           
\pstart
           Die Streichung in dem Telegramm\pwindex{Man telegraphirt uns aus Breslau…]@\emph{[Man telegraphirt uns aus Breslau…]}|pwv} iſt offenbar erfolgt, weil man dem Herrn \label{K_L02945-3v}\edtext{\textsc{Loewe}\pwindex{Loewe, Theodor 1.\,1.\,1855 Wien – 1935 Breslau@\textsc{Loewe, Theodor} (1.\,1.\,1855 Wien – 1935 Breslau), \emph{Theaterleiter}|pw} nicht wehthun}{\lemma{\textnormal{\emph{Loewe nicht wehthun}}}\Cendnote{\textnormal{Siehe XXXX Auszeichnungsfehler: Dokument L02936 nicht gefunden. }}}\label{K_L02945-3} wollte. Da
               hat man lieber {\pb}den Sachverhalt gefälſcht und den
               Autor geſchädigt.\pend
           
\pstart
           Viele treue Grüße! {\\[\baselineskip]}Dein {\\[\baselineskip]}\spacefill\mbox{Paul Goldmann.}\pend
           \leftskip=0em{}\selectlanguage{ngerman}\endnumbering\briefempfaengerindex{Schnitzler, Arthur@\textsc{Schnitzler, Arthur}!zzzGoldmann, Paul@\emph{von Paul Goldmann}!1900-12-111@{11. 12. [1900]}|)be}\mylabel{L02945h}  \newcommand{\dateiname}{L02945}\newcommand{\titel}{Paul Goldmann an Arthur Schnitzler, 11. 12. [1900]}\newcommand{\editorInnen}{Martin Anton Müller und Laura Untner}%% latex-leseansicht-abspann.tex
%% Abspann für die Leseansicht.
%% Der Schalter \ifkorrekturansicht ist bereits durch den Vorspann gesetzt.

%% latex-abspann.tex
%% Gemeinsamer Abspann für Korrekturansicht und Leseansicht.
%% Setzt den Schalter \ifkorrekturansicht voraus (gesetzt in den
%% einbindenden Dateien latex-korrekturansicht-abspann.tex bzw.
%% latex-leseansicht-abspann.tex).
%% ---------------------------------------------------------------

\normalsize

% Das esempio-Environment wird nur in der Leseansicht benötigt
\ifkorrekturansicht\else
\newenvironment{esempio}[3]%
{
    \vspace{1.5ex}
    \rlap{\underline{#1}}
    \par
    \setlength{\parindent}{0cm}
    \nopagebreak
    \leftskip=#2cm
    \rightskip=#3cm
}
{
    \par
}
\fi

\doendnotes{C}
\bigskip
\vfill

\clearpage

\footnotesize

\ifkorrekturansicht
  \lohead{\textsc{register}}
\fi

% theindex-Environment neu definieren ohne reledmac
\makeatletter
\renewenvironment{theindex}{%
  \ifkorrekturansicht
    \section*{\indexname}%
  \else
    \subsubsection*{Index der erwähnten Entitäten}%
  \fi
  \setlength{\parindent}{0pt}%
  \setlength{\parskip}{0pt plus 0.3pt}%
  \let\item\@idxitem
}{%
  \ifkorrekturansicht\clearpage\fi
}
\makeatother

\IfFileExists{\jobname-pw.ind}{\input{\jobname-pw.ind}}{}

% Quellenangabe nur in der Leseansicht
\ifkorrekturansicht\else
% Fallback-Definitionen, falls die .tex-Datei \titel etc. nicht gesetzt hat
\providecommand{\titel}{}
\providecommand{\editorInnen}{}
\providecommand{\dateiname}{\jobname}

\vspace{3cm}

\vfill

\footnotesize
\textsc{Quelle}: \titel. Herausgegeben von {\editorInnen}. In: \emph{Arthur Schnitzler: Briefwechsel mit Autorinnen und Autoren}.
 Digitale Edition, https://schnitzler-briefe.acdh.oeaw.ac.at/{\dateiname}.html (Stand \today)
\fi

\end{document}


