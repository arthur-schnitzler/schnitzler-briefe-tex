%% latex-korrekturansicht-vorspann.tex
%% Vorspann für die Korrekturansicht.
%% Lädt die gemeinsame Datei latex-vorspann.tex mit gesetztem Schalter.

\newif\ifkorrekturansicht
\korrekturansichttrue

\input{../tex-inputs/latex-vorspann}


\section[ Paul Goldmann an Arthur Schnitzler, 11. 12. {[}1900{]}]{L02945 Paul Goldmann an Arthur Schnitzler, 11. 12. {[}1900{]}}
\nopagebreak\mylabel{L02945v}
\rehead{ }\normalsize\beginnumbering\briefempfaengerindex{Schnitzler, Arthur@\textsc{Schnitzler, Arthur}!zzzGoldmann, Paul@\emph{von Paul Goldmann}!1900-12-111@{11. 12. {[}1900{]}}|(be}
\toendnotes[C]{\smallbreak\pagebreak[2]}\Standort{DLA, A:Schnitzler, HS.NZ85.1.3170.}
\physDesc{Brief, 1 Blatt, 3 Seiten, 658 Zeichen
\newline{}Handschrift: blaue Tinte, deutsche Kurrent
\newline{}Schnitzler: 1) mit Bleistift das Jahr »900.« vermerkt  2) mit rotem Buntstift eine Unterstreichung}\toendnotes[C]{\smallbreak}
\pstart
           \raggedleft{}{\pb}\textcolor{gray}{\textbf{DESSAUERSTRASSE 19}}\oindex{Dessauer Strasse@\textbf{Dessauer Straße}, \emph{Straße (K.STR)}|pw}\pend
           
\pstart
           Berlin\oindex{Berlin@\textbf{Berlin}, \emph{P.PPLC}|pw}, 11. December.\pend
           
\pstart\center{}Mein lieber Freund,\pend\vspace{0.5em}
\pstart
           Gewiß, die N. Fr. Pr.\orgindex{Neue Freie Presse@Neue Freie Presse|pw} hat ſich \label{K_L02945-1v}\edtext{niederträchtig benommen}{\lemma{\textnormal{\emph{niederträchtig benommen}}}\Cendnote{\textnormal{Bezug auf die Berichterstattung\pwindex{Man telegraphirt uns aus Breslau…]@\emph{[Man telegraphirt uns aus Breslau…]}|pwkv} der \emph{Neuen Freien Presse}\orgindex{Neue Freie Presse@Neue Freie Presse|pwk} über die Uraufführung von \emph{Der Schleier der Beatrice}\pwindex{Schleier der Beatrice. Schauspiel in fuenf Akten@\emph{Der Schleier der Beatrice. Schauspiel in fünf Akten}|pwk}, siehe Paul Goldmann an Arthur Schnitzler, 3. 12. [1900] und 9. 12. [1900].
               }}}\label{K_L02945-1}. Ob man dagegen nichts thun kann? \strikeout{Jawoh}
               Jawohl. Beiſpielsweiſe: Schreib’ an das Blatt\orgindex{Neue Freie Presse@Neue Freie Presse|pwv} einen Brief, worin Du mittheilſt, daß Du wegen der Dir
               gegenüber bewieſenen niederträchtigen Parteilichkeit die für die {\pb}Weihnachtsnummer\pwindex{Neue Freie Presse@\emph{Neue Freie Presse}|pwv} beſtimmte Novelle\pwindex{Lieutenant Gustl. Novelle@\emph{Lieutenant Gustl. Novelle}|pwv}{ }\label{K_L02945-2v}\edtext{zurückziehſt}{\lemma{\textnormal{\emph{zurückziehſt}}}\Cendnote{\textnormal{In der Korrespondenz mit Theodor Herzl\pwindex{Herzl, Theodor 1860-05-02 – 1904-07-03@\textsc{Herzl, Theodor} (1860-05-02 – 1904-07-03), \emph{Schriftsteller/Schriftstellerin, Journalist/Journalistin}|pwk}, mit dem Schnitzler
                  die Aufnahme von \emph{Lieutenant Gustl}\pwindex{Lieutenant Gustl. Novelle@\emph{Lieutenant Gustl. Novelle}|pwk} in der \emph{Neuen Freien Presse}\pwindex{Neue Freie Presse@\emph{Neue Freie Presse}|pwk} verhandelte, ist von der
                     Berichterstattung\pwindex{Man telegraphirt uns aus Breslau…]@\emph{[Man telegraphirt uns aus Breslau…]}|pwkv} über
                  die Uraufführung von \emph{Der Schleier der
                     Beatrice}\pwindex{Schleier der Beatrice. Schauspiel in fuenf Akten@\emph{Der Schleier der Beatrice. Schauspiel in fünf Akten}|pwk} nicht die Rede.}}}\label{K_L02945-2}. Das wäre eine Lektion. Aber wenn Ihr
               Unabhängigen nichts gegen das Blatt\orgindex{Neue Freie Presse@Neue Freie Presse|pwv} thun wollt, was ſollen dann wir Abhängigen thun?\pend
           
\pstart
           Die Streichung in dem Telegramm\pwindex{Man telegraphirt uns aus Breslau…]@\emph{[Man telegraphirt uns aus Breslau…]}|pwv} iſt offenbar erfolgt, weil man dem Herrn \label{K_L02945-3v}\edtext{\textsc{Loewe}\pwindex{Loewe, Theodor 1855-01-01 – 1935@\textsc{Loewe, Theodor} (1855-01-01 – 1935), \emph{Theaterleiter/Theaterleiterin}|pw} nicht wehthun}{\lemma{\textnormal{\emph{Loewe nicht wehthun}}}\Cendnote{\textnormal{Siehe Paul Goldmann an Arthur Schnitzler, 14. 10. [1900]. }}}\label{K_L02945-3} wollte. Da
               hat man lieber {\pb}den Sachverhalt gefälſcht und den
               Autor geſchädigt.\pend
           
\pstart
           Viele treue Grüße! {\\[\baselineskip]}Dein {\\[\baselineskip]}\spacefill\mbox{Paul Goldmann.}\pend
           \leftskip=0em{}\selectlanguage{ngerman}\endnumbering\briefempfaengerindex{Schnitzler, Arthur@\textsc{Schnitzler, Arthur}!zzzGoldmann, Paul@\emph{von Paul Goldmann}!1900-12-111@{11. 12. {[}1900{]}}|)be}\mylabel{L02945h}  \normalsize

\doendnotes{C}
\bigskip
\vfill

\clearpage

\footnotesize

\lohead{\textsc{register}}

% Definiere theindex-Environment komplett neu ohne reledmac
\makeatletter
\renewenvironment{theindex}{%
  \section*{\indexname}%
  \setlength{\parindent}{0pt}%
  \setlength{\parskip}{0pt plus 0.3pt}%
  \let\item\@idxitem
}{%
  \clearpage
}
\makeatother

\IfFileExists{\jobname-pw.ind}{\input{\jobname-pw.ind}}{}

\end{document}

      