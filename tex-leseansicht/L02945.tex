%% latex-leseansicht-vorspann.tex
%% Vorspann für die Leseansicht.
%% Lädt die gemeinsame Datei latex-vorspann.tex mit nicht gesetztem Schalter.

\newif\ifkorrekturansicht
\korrekturansichtfalse

\input{../tex-inputs/latex-vorspann}

\begin{center}
            \textcolor{red}{ENTWURF, NICHT FERTIG KORRIGIERT}
                      \end{center}
            
         
         \newcommand{\erwaehntePersonen}{Personen: Theodor Loewe}
         \newcommand{\erwaehnteInstitutionen}{Institutionen: Neue Freie Presse}
         \newcommand{\erwaehnteOrte}{Orte: Berlin, Dessauer Straße, Wien}
         \newcommand{\erwaehnteWerke}{Werke: Der Schleier der Beatrice. Schauspiel in fünf Akten, Lieutenant Gustl. Novelle, Neue Freie Presse, Theater- und Kunstnachrichten [Uraufführung von Schleier der Beatrice]}
               \section[ Paul Goldmann an Arthur Schnitzler, 11. 12. {[}1900{]}]{ Paul Goldmann an Arthur Schnitzler, 11. 12. {[}1900{]}}\nopagebreak\mylabel{v}\rehead{ }\begin{ledgroupsized}[t]{13cm}\normalsize\beginnumbering \toendnotes[C]{\smallbreak\pagebreak[2]} \Standort{DLA, A:Schnitzler, HS.NZ85.1.3170.}
\physDesc{Brief, 1 Blatt, 3 Seiten
\newline{}Handschrift: blaue Tinte, deutsche Kurrent
\newline{}Schnitzler: 1) mit Bleistift das Jahr »{[}1{]}900« vermerkt  2) mit rotem Buntstift eine Unterstreichung}\toendnotes[C]{\smallbreak}\pstart{}{\pb}\textcolor{gray}{\textbf{DESSAUERSTRASSE 19}}\oindex{Dessauer Strasse@\textbf{Dessauer Straße}|pw}\pend{}{\bigskip}\pstart
           Berlin\oindex{Berlin@\textbf{Berlin}|pw}, 11. December.\pend
           \pstart\center{}Mein lieber Freund,\pend\pstart
           Gewiß, die N. Fr. Pr.\orgindex{Neue Freie Presse@Neue Freie Presse|pw} hat ſich \label{K_L02945-1v}\edtext{niederträchtig benommen}{\lemma{\textnormal{\emph{niederträchtig benommen}}}\Cendnote{\textnormal{Bezug auf die Berichterstattung\pwindex{Theater- und Kunstnachrichten [Urauffuehrung von Schleier der Beatrice]1900-12-02@\emph{Theater- und Kunstnachrichten [Uraufführung von Schleier der Beatrice]} {[}1900-12-02{]}|pwkv} der \emph{Neuen Freien Presse}\orgindex{Neue Freie Presse@Neue Freie Presse|pwk} über die Uraufführung von \emph{Der Schleier der Beatrice}\pwindex{Schnitzler, Arthur 15.05.1862 – 21.10.1931@\textsc{Schnitzler, Arthur} (15.05.1862 – 21.10.1931), \emph{Schriftsteller, Mediziner}!Schleier der Beatrice. Schauspiel in fuenf Akten1900-12-01@\strich\emph{Der Schleier der Beatrice. Schauspiel in fünf Akten} {[}1900-12-01{]}|pwk}, siehe Paul Goldmann an Arthur Schnitzler, 3. 12. [1900] und 3. 12. [1900]}}}\label{K_L02945-1h}. Ob man dagegen nichts thun kann? \strikeout{Jawoh}
               Jawohl. Beiſpielsweiſe: Schreib’ an das Blatt\orgindex{Neue Freie Presse@Neue Freie Presse|pwv} einen Brief, worin Du mittheilſt, daß Du wegen der Dir
               gegenüber bewieſenen niederträchtigen Parteilichkeit die für die {\pb}Weihnachtsnummer\pwindex{Neue Freie Presse1864 – 1939@\emph{Neue Freie Presse} {[}1864 – 1939{]}|pwv} beſtimmte Novelle\pwindex{Schnitzler, Arthur 15.05.1862 – 21.10.1931@\textsc{Schnitzler, Arthur} (15.05.1862 – 21.10.1931), \emph{Schriftsteller, Mediziner}!Lieutenant Gustl. Novelle1900-12-25@\strich\emph{Lieutenant Gustl. Novelle} {[}1900-12-25{]}|pwv}{ }\label{K_L02945-2v}\edtext{zurückziehſt}{\lemma{\textnormal{\emph{zurückziehſt}}}\Cendnote{\textnormal{nicht geschehen}}}\label{K_L02945-2h}. Das wäre eine Lektion. Aber wenn Ihr
               Unabhängigen nichts gegen das Blatt\orgindex{Neue Freie Presse@Neue Freie Presse|pwv} thun wollt, was ſollen dann wir Abhängigen thun?\pend
           \pstart
           Die Streichung in dem Telegramm iſt offenbar erfolgt, weil man dem Herrn \label{K_L02945-12v}\edtext{Loewe\pwindex{Loewe, Theodor 1855-01-01 – 1935@\textsc{Loewe, Theodor} (1855-01-01 – 1935), \emph{Theaterleiter}|pw} nicht wehthun}{\lemma{\textnormal{\emph{Loewe nicht wehthun}}}\Cendnote{\textnormal{siehe Paul Goldmann an Arthur Schnitzler, 14. 10. [1900]}}}\label{K_L02945-12h} wollte. Da hat man lieber {\pb}den
                  Sach\textcolor{gray}{v}erhalt gefälſcht und den Autor geſchädigt.\pend
           \pstart
           Viele treue Grüße! {\\[\baselineskip]}Dein {\\[\baselineskip]}\spacefill\mbox{Paul Goldmann.}\pend
           \leftskip=0em{}
         
         \endnumbering\mylabel{h}\end{ledgroupsized}\begin{anhang}\end{anhang}\newcommand{\dateiname}{L02945}\newcommand{\titel}{Paul Goldmann an Arthur Schnitzler, 11. 12. [1900]}\newcommand{\editorInnen}{Martin Anton Müller und Laura Untner}%% latex-leseansicht-abspann.tex
%% Abspann für die Leseansicht.
%% Der Schalter \ifkorrekturansicht ist bereits durch den Vorspann gesetzt.

%% latex-abspann.tex
%% Gemeinsamer Abspann für Korrekturansicht und Leseansicht.
%% Setzt den Schalter \ifkorrekturansicht voraus (gesetzt in den
%% einbindenden Dateien latex-korrekturansicht-abspann.tex bzw.
%% latex-leseansicht-abspann.tex).
%% ---------------------------------------------------------------

\normalsize

% Das esempio-Environment wird nur in der Leseansicht benötigt
\ifkorrekturansicht\else
\newenvironment{esempio}[3]%
{
    \vspace{1.5ex}
    \rlap{\underline{#1}}
    \par
    \setlength{\parindent}{0cm}
    \nopagebreak
    \leftskip=#2cm
    \rightskip=#3cm
}
{
    \par
}
\fi

\doendnotes{C}
\bigskip
\vfill

\clearpage

\footnotesize

\ifkorrekturansicht
  \lohead{\textsc{register}}
\fi

% theindex-Environment neu definieren ohne reledmac
\makeatletter
\renewenvironment{theindex}{%
  \ifkorrekturansicht
    \section*{\indexname}%
  \else
    \subsubsection*{Index der erwähnten Entitäten}%
  \fi
  \setlength{\parindent}{0pt}%
  \setlength{\parskip}{0pt plus 0.3pt}%
  \let\item\@idxitem
}{%
  \ifkorrekturansicht\clearpage\fi
}
\makeatother

\IfFileExists{\jobname-pw.ind}{\input{\jobname-pw.ind}}{}

% Quellenangabe nur in der Leseansicht
\ifkorrekturansicht\else
% Fallback-Definitionen, falls die .tex-Datei \titel etc. nicht gesetzt hat
\providecommand{\titel}{}
\providecommand{\editorInnen}{}
\providecommand{\dateiname}{\jobname}

\vspace{3cm}

\vfill

\footnotesize
\textsc{Quelle}: \titel. Herausgegeben von {\editorInnen}. In: \emph{Arthur Schnitzler: Briefwechsel mit Autorinnen und Autoren}.
 Digitale Edition, https://schnitzler-briefe.acdh.oeaw.ac.at/{\dateiname}.html (Stand \today)
\fi

\end{document}


      