%% latex-korrekturansicht-vorspann.tex
%% Vorspann für die Korrekturansicht.
%% Lädt die gemeinsame Datei latex-vorspann.tex mit gesetztem Schalter.

\newif\ifkorrekturansicht
\korrekturansichttrue

\input{../tex-inputs/latex-vorspann}


\section[Georg Brandes an Arthur Schnitzler, 10. 5. {[}1901{]}]{L01116 Georg Brandes an Arthur Schnitzler, 10. 5. {[}1901{]}}
\nopagebreak\mylabel{L01116v}
\rehead{ }\normalsize\beginnumbering\briefempfaengerindex{Schnitzler, Arthur@\textsc{Schnitzler, Arthur}!zzzBrandes, Georg@\emph{von Georg Brandes}!1901-05-101@{10. 5. {[}1901{]}}|(be}
\toendnotes[C]{\smallbreak\pagebreak[2]}\Standort{CUL, Schnitzler, B 17.}
\physDesc{Brief, 1 Blatt, 2 Seiten, 931 Zeichen
\newline{}Handschrift: schwarze Tinte, lateinische Kurrent
\newline{}Schnitzler: mit Bleistift die Jahreszahl ergänzt: »901« 
\newline{}Ordnung: mit Bleistift von unbekannter Hand nummeriert:
                                    »21« }
\buchAbdrucke{\weitereDrucke{Georg Brandes, Arthur Schnitzler: \emph{Ein Briefwechsel}. Bern: \emph{Francke} 1956, S. 85.} }\toendnotes[C]{\smallbreak}
\pstart
           \raggedleft{}{\pb}Schloss Strzebowitz\oindex{Schloss Strzebowitz@\textbf{Schloss Strzebowitz}, \emph{Schloss (K.SLS)}|pw}{\\}Schlesien. Oesterreich\oindex{Schlesien@\textbf{Schlesien}, \emph{L.RGN}|pw}{\\}10 Mai\pend
           \vspace{0.5em}
\pstart
           Liebster! Ich habe Ihren Brief und ich habe den Roman\pwindex{Frau Bertha Garlan. Roman@\emph{Frau Bertha Garlan. Roman}|pwv} mit der grössten Freude gelesen. Er
               ist so wahr und tief. Ein ganz klein wenig zu roh haben Sie doch vielleicht den
               Virtuosen gemacht. Man hat den Eindruck, er habe eine sinnliche Enttäuschung
               erfahren, die Dame hat ja freilich nicht vor der Umarmung Toilette machen können. Wie
               es bei der Marni\pwindex{Marni, Jeanne 1854-01-31 – 1910-01-06@\textsc{Marni, Jeanne} (1854-01-31 – 1910-01-06), \emph{Schriftsteller/Schriftstellerin}|pw} heisst \label{K_L01116-1v}\edtext{tub be or not tub be, that is the
                  question\pwindex{Hamlet@\emph{Hamlet}|pwv}}{\lemma{\textnormal{\emph{tub … question}}}\Cendnote{\textnormal{nicht nachgewiesen}}}\label{K_L01116-1}. Oder er hat
               vielleicht, wie es geht, so viele Frauen an den Hals, dass er nicht mehr verträgt.
               Jedenfalls {\pb}das Buch\pwindex{Frau Bertha Garlan. Roman@\emph{Frau Bertha Garlan. Roman}|pwv} ist gut. Die Nebenhandlung, die
               Geschichte der schönen Frau, sehr fein geführt.\pend
           
\pstart
           Ich glaube dass ich am 16\textsuperscript{sten} von hier über Wien\oindex{Wien@\textbf{Wien}, \emph{A.ADM2}|pw} nach Abbazia\oindex{Opatija@\textbf{Opatija}, \emph{P.PPLA2}|pw} reise.\hspace*{1.5em}Wenn Sie in Wien\oindex{Wien@\textbf{Wien}, \emph{A.ADM2}|pw} dann sind und ein Paar Stunden
               für mich übrig haben, möchte ich schon Mittags um 3,48 nach Wien\oindex{Wien@\textbf{Wien}, \emph{A.ADM2}|pw} kommen und bis 8 Uhr Abends
                  bleiben.\hspace*{1.5em}Sonst reise ich durch.\pend
           
\pstart
           Bitte, liebster Freund und Poet, um eine Zeile Antwort.\pend
           
\pstart
           Ihr{\\[\baselineskip]}\spacefill\mbox{Georg Brandes}\pend
           \leftskip=0em{}\selectlanguage{ngerman}\endnumbering\briefempfaengerindex{Schnitzler, Arthur@\textsc{Schnitzler, Arthur}!zzzBrandes, Georg@\emph{von Georg Brandes}!1901-05-101@{10. 5. {[}1901{]}}|)be}\mylabel{L01116h}  \normalsize

\doendnotes{C}
\bigskip
\vfill

\clearpage

\footnotesize

\lohead{\textsc{register}}

% Definiere theindex-Environment komplett neu ohne reledmac
\makeatletter
\renewenvironment{theindex}{%
  \section*{\indexname}%
  \setlength{\parindent}{0pt}%
  \setlength{\parskip}{0pt plus 0.3pt}%
  \let\item\@idxitem
}{%
  \clearpage
}
\makeatother

\IfFileExists{\jobname-pw.ind}{\input{\jobname-pw.ind}}{}

\end{document}

      