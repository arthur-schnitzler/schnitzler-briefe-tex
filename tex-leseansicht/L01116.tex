%% latex-leseansicht-vorspann.tex
%% Vorspann für die Leseansicht.
%% Lädt die gemeinsame Datei latex-vorspann.tex mit nicht gesetztem Schalter.

\newif\ifkorrekturansicht
\korrekturansichtfalse

\input{../tex-inputs/latex-vorspann}


         
         \renewcommand{\erwaehntePersonen}{Personen: Jeanne Marni}
         \renewcommand{\erwaehnteOrte}{Orte: Opatija, Ostrava, Schlesien, Schloss Strzebowitz, Wien}
         \renewcommand{\erwaehnteWerke}{Werke: Frau Bertha Garlan. Roman, Hamlet}
               \section[Georg Brandes an Arthur Schnitzler, 10. 5. {[}1901{]}]{ Georg Brandes an Arthur Schnitzler, 10. 5. {[}1901{]}}\nopagebreak\mylabel{v}\rehead{ }\begin{ledgroupsized}[t]{13cm}\normalsize\beginnumbering \toendnotes[C]{\smallbreak\pagebreak[2]} \Standort{CUL, Schnitzler, B 17.}
\physDesc{Brief, 1 Blatt, 2 Seiten, 931 Zeichen
\newline{}Handschrift: schwarze Tinte, lateinische Kurrent
\newline{}Schnitzler: mit Bleistift die Jahreszahl ergänzt: »901« 
\newline{}Ordnung: mit Bleistift von unbekannter Hand nummeriert:
                                    »21« }\buchAbdrucke{\weitereDrucke{Georg Brandes, Arthur Schnitzler: \emph{Ein Briefwechsel}. Hg. Kurt Bergel. Bern: \emph{Francke} 1956, S. 85.} }\toendnotes[C]{\smallbreak}\pstart
           \raggedleft{}{\pb}Schloss Strzebowitz\oindex{Schloss Strzebowitz@\textbf{Schloss Strzebowitz}|pw}{\\}Schlesien. Oesterreich\oindex{Schlesien@\textbf{Schlesien}|pw}{\\}10 Mai\pend
           \pstart
           Liebster! Ich habe Ihren Brief und ich habe den Roman\pwindex{Schnitzler, Arthur 15.05.1862 – 21.10.1931@\textsc{Schnitzler, Arthur} (15.05.1862 – 21.10.1931), \emph{Schriftsteller, Mediziner}!Frau Bertha Garlan. Roman15.1.1901 – 15.3.1901@\strich\emph{Frau Bertha Garlan. Roman} {[}15.1.1901 – 15.3.1901{]}|pwv} mit der grössten Freude gelesen. Er
               ist so wahr und tief. Ein ganz klein wenig zu roh haben Sie doch vielleicht den
               Virtuosen gemacht. Man hat den Eindruck, er habe eine sinnliche Enttäuschung
               erfahren, die Dame hat ja freilich nicht vor der Umarmung Toilette machen können. Wie
               es bei der Marni\pwindex{Marni, Jeanne 1854-01-31 – 1910-01-06@\textsc{Marni, Jeanne} (1854-01-31 – 1910-01-06), \emph{Schriftstellerin}|pw} heisst \label{K_L01116_1v}\edtext{tub be or not tub be, that is the
                  question\pwindex{\textcolor{red}{\textsuperscript{XXXX1 indx}}!Hamlet1600@\strich\emph{Hamlet} {[}1600{]}|pwv}}{\lemma{\textnormal{\emph{tub … question}}}\Cendnote{\textnormal{nicht nachgewiesen}}}\label{K_L01116_1h}. Oder er hat
               vielleicht, wie es geht, so viele Frauen an den Hals, dass er nicht mehr verträgt.
               Jedenfalls {\pb}das Buch\pwindex{Schnitzler, Arthur 15.05.1862 – 21.10.1931@\textsc{Schnitzler, Arthur} (15.05.1862 – 21.10.1931), \emph{Schriftsteller, Mediziner}!Frau Bertha Garlan. Roman15.1.1901 – 15.3.1901@\strich\emph{Frau Bertha Garlan. Roman} {[}15.1.1901 – 15.3.1901{]}|pwv} ist gut. Die Nebenhandlung, die
               Geschichte der schönen Frau, sehr fein geführt.\pend
           \pstart
           Ich glaube dass ich am 16\textsuperscript{sten} von hier über Wien\oindex{Wien@\textbf{Wien}|pw} nach Abbazia\oindex{Opatija@\textbf{Opatija}|pw} reise.\hspace*{1.5em}Wenn Sie in Wien\oindex{Wien@\textbf{Wien}|pw} dann sind und ein Paar Stunden
               für mich übrig haben, möchte ich schon Mittags um 3,48 nach Wien\oindex{Wien@\textbf{Wien}|pw} kommen und bis 8 Uhr Abends
                  bleiben.\hspace*{1.5em}Sonst reise ich durch.\pend
           \pstart
           Bitte, liebster Freund und Poet, um eine Zeile Antwort.\pend
           \pstart
           Ihr{\\[\baselineskip]}\spacefill\mbox{Georg Brandes}\pend
           \leftskip=0em{}
         
         \endnumbering\mylabel{h}\end{ledgroupsized}  \newcommand{\dateiname}{L01116}\newcommand{\titel}{Georg Brandes an Arthur Schnitzler, 10. 5. [1901]}\newcommand{\editorInnen}{Martin Anton Müller und Gerd-Hermann Susen}%% latex-leseansicht-abspann.tex
%% Abspann für die Leseansicht.
%% Der Schalter \ifkorrekturansicht ist bereits durch den Vorspann gesetzt.

%% latex-abspann.tex
%% Gemeinsamer Abspann für Korrekturansicht und Leseansicht.
%% Setzt den Schalter \ifkorrekturansicht voraus (gesetzt in den
%% einbindenden Dateien latex-korrekturansicht-abspann.tex bzw.
%% latex-leseansicht-abspann.tex).
%% ---------------------------------------------------------------

\normalsize

% Das esempio-Environment wird nur in der Leseansicht benötigt
\ifkorrekturansicht\else
\newenvironment{esempio}[3]%
{
    \vspace{1.5ex}
    \rlap{\underline{#1}}
    \par
    \setlength{\parindent}{0cm}
    \nopagebreak
    \leftskip=#2cm
    \rightskip=#3cm
}
{
    \par
}
\fi

\doendnotes{C}
\bigskip
\vfill

\clearpage

\footnotesize

\ifkorrekturansicht
  \lohead{\textsc{register}}
\fi

% theindex-Environment neu definieren ohne reledmac
\makeatletter
\renewenvironment{theindex}{%
  \ifkorrekturansicht
    \section*{\indexname}%
  \else
    \subsubsection*{Index der erwähnten Entitäten}%
  \fi
  \setlength{\parindent}{0pt}%
  \setlength{\parskip}{0pt plus 0.3pt}%
  \let\item\@idxitem
}{%
  \ifkorrekturansicht\clearpage\fi
}
\makeatother

\IfFileExists{\jobname-pw.ind}{\input{\jobname-pw.ind}}{}

% Quellenangabe nur in der Leseansicht
\ifkorrekturansicht\else
% Fallback-Definitionen, falls die .tex-Datei \titel etc. nicht gesetzt hat
\providecommand{\titel}{}
\providecommand{\editorInnen}{}
\providecommand{\dateiname}{\jobname}

\vspace{3cm}

\vfill

\footnotesize
\textsc{Quelle}: \titel. Herausgegeben von {\editorInnen}. In: \emph{Arthur Schnitzler: Briefwechsel mit Autorinnen und Autoren}.
 Digitale Edition, https://schnitzler-briefe.acdh.oeaw.ac.at/{\dateiname}.html (Stand \today)
\fi

\end{document}


      