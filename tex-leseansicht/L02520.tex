%% latex-korrekturansicht-vorspann.tex
%% Vorspann für die Korrekturansicht.
%% Lädt die gemeinsame Datei latex-vorspann.tex mit gesetztem Schalter.

\newif\ifkorrekturansicht
\korrekturansichttrue

\input{../tex-inputs/latex-vorspann}


\section[Arthur Schnitzler an Robert Adam, 14. 8. 1929]{L02520 Arthur Schnitzler an Robert Adam, 14. 8. 1929}
\nopagebreak\mylabel{L02520v}
\rehead{ }\normalsize\beginnumbering\briefempfaengerindex{Adam, Robert@\textsc{Adam, Robert}!zzzSchnitzler, Arthur@\emph{von Arthur Schnitzler}!1929-08-141@{14. 8. 1929}|(be}
\toendnotes[C]{\smallbreak\pagebreak[2]}\Standort{DLA, 96.34.2/32.}
\physDesc{Postkarte, 291 Zeichen
\newline{}Handschrift: schwarze Tinte, lateinische Kurrent
\newline{}Versand: 1) nachgesandt nach \textsc{Bad-Aussee}\oindex{Bad Aussee@\textbf{Bad Aussee}, \emph{P.PPLA3}|pw}, \textsc{Meraner Haus}\oindex{Meran Haus@\textbf{Meran Haus}, \emph{Gebäude (K.GBD)}|pw}«  2) Stempel: »\nobreak{}\oindex{IX., Alsergrund@\textbf{IX., Alsergrund}, \emph{A.ADM3}|pwk}9/\textsubscript{1} Wien
                                       38, 15. VIII. 29, 18\nobreak{}«.  3) Stempel: »\nobreak{}\oindex{XII., Meidling@\textbf{XII., Meidling}, \emph{A.ADM3}|pwk}12/\textsubscript{1} Wien
                                       82, 16. VIII. 29, 19\nobreak{}«. }\toendnotes[C]{\smallbreak}\pstart{}{\pb}\label{T_L02520-1v}\edtext{\textcolor{gray}{\textbf{A. S.}}}{\lemma{\textnormal{\emph{A. S.}}}\Cendnote{\textnormal{ovaler Absenderkleber}}}\label{T_L02520-1}\pend{}\pstart{}\textcolor{gray}{\textbf{WIEN, XVIII.}}\oindex{XVIII., Waehring@\textbf{XVIII., Währing}, \emph{A.ADM3}|pw}\pend{}\pstart{}\textcolor{gray}{\textbf{STERNWARTESTR. 71}}\oindex{Sternwartestrasse 71@\textbf{Sternwartestraße 71}, \emph{Wohngebäude (K.WHS)}|pw}\pend{}{\bigskip}\pstart{}{\pb}Hrn Ober L.\textcolor{gray}{g}r. Rath\pend{}\pstart{}Dr. Robert Adam Pollak\pend{}\pstart{}Wien XIII\oindex{XIII., Hietzing@\textbf{XIII., Hietzing}, \emph{A.ADM3}|pw}\pend{}\pstart{}Meidlinger Hptstr 58\oindex{Meidlinger Hauptstrasse@\textbf{Meidlinger Hauptstraße}, \emph{Straße (K.STR)}|pw}.\pend{}{\bigskip}\vspace{1em}
\pstart
           \raggedleft{}{\pb}Wien\oindex{Wien@\textbf{Wien}, \emph{A.ADM2}|pw}, 14/8 929\pend
           
\pstart{}verehrter Herr Doctor, \pend\vspace{0.5em}
\pstart
           Ihren Aufsatz\pwindex{Zur Frage des Laienrichtertums beim Handelsgericht@\emph{Zur Frage des Laienrichtertums beim Handelsgericht}|pwv}, so präcise und
               so klar hab ich mit aufrichtigem Vergnügen gelesen. Ich danke Ihnen sehr, auch für
               den lieben Brief und grüße Sie herzlichst.\pend
           
\pstart
           {\pb}Ihr sehr ergebner{\\[\baselineskip]}\spacefill\mbox{ArthSchnitzler}\pend
           \leftskip=0em{}\selectlanguage{ngerman}\endnumbering\briefempfaengerindex{Adam, Robert@\textsc{Adam, Robert}!zzzSchnitzler, Arthur@\emph{von Arthur Schnitzler}!1929-08-141@{14. 8. 1929}|)be}\mylabel{L02520h}  \normalsize

\doendnotes{C}
\bigskip
\vfill

\clearpage

\footnotesize

\lohead{\textsc{register}}

% Definiere theindex-Environment komplett neu ohne reledmac
\makeatletter
\renewenvironment{theindex}{%
  \section*{\indexname}%
  \setlength{\parindent}{0pt}%
  \setlength{\parskip}{0pt plus 0.3pt}%
  \let\item\@idxitem
}{%
  \clearpage
}
\makeatother

\IfFileExists{\jobname-pw.ind}{\input{\jobname-pw.ind}}{}

\end{document}

      