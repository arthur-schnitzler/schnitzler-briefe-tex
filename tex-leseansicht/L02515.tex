%% latex-korrekturansicht-vorspann.tex
%% Vorspann für die Korrekturansicht.
%% Lädt die gemeinsame Datei latex-vorspann.tex mit gesetztem Schalter.

\newif\ifkorrekturansicht
\korrekturansichttrue

\input{../tex-inputs/latex-vorspann}


\section[Gerty Hofmannsthal an Arthur Schnitzler, {[}29. 7. 1929{]}]{L02515 Gerty Hofmannsthal an Arthur Schnitzler, {[}29. 7. 1929{]}}
\nopagebreak\mylabel{L02515v}
\rehead{ }\normalsize\beginnumbering\briefempfaengerindex{Schnitzler, Arthur@\textsc{Schnitzler, Arthur}!zzzHofmannsthal, Gertrude von@\emph{von Gertrude von Hofmannsthal}!1929-07-292@{{[}29. 7. 1929{]}}|(be}
\toendnotes[C]{\smallbreak\pagebreak[2]}\Standort{CUL, Schnitzler, B 43.}
\physDesc{Brief, 1 Blatt, 1 Seite, 362 Zeichen (Briefpapier mit Trauerrand)
\newline{}Handschrift: schwarze Tinte, lateinische Kurrent
\newline{}Schnitzler: 1) mit rotem Buntstift beschriftet: »\textsc{Gerty}« und datiert: »30/7 929«  2) mit rotem Buntstift zwei Unterstreichungen}
\pstart
           \raggedleft{}{\pb}Dienstag.\pend
           
\pstart{}Lieber Artur\pend\vspace{0.5em}
\pstart
           Ich schicke Ihnen eingeschrieben ein Packerl Briefe – sie sind allerdings wie ich
               drauf notiert sahe nur die frühen, während die andern in den grossen Packeten sind,
               die jährlich datiert sind und alphabetisch geordnet!\pend
           
\pstart
           Nun kann ich dies leider in der Eile nicht mehr sortieren. Werde es aber im Herbst
               dann gerne tun. \pend
           
\pstart
           Allerherzlichst Ihre{\\[\baselineskip]}\spacefill\mbox{Gerty}\pend
           \leftskip=0em{}\selectlanguage{ngerman}\endnumbering\briefempfaengerindex{Schnitzler, Arthur@\textsc{Schnitzler, Arthur}!zzzHofmannsthal, Gertrude von@\emph{von Gertrude von Hofmannsthal}!1929-07-292@{{[}29. 7. 1929{]}}|)be}\mylabel{L02515h}  \normalsize

\doendnotes{C}
\bigskip
\vfill

\clearpage

\footnotesize

\lohead{\textsc{register}}

% Definiere theindex-Environment komplett neu ohne reledmac
\makeatletter
\renewenvironment{theindex}{%
  \section*{\indexname}%
  \setlength{\parindent}{0pt}%
  \setlength{\parskip}{0pt plus 0.3pt}%
  \let\item\@idxitem
}{%
  \clearpage
}
\makeatother

\IfFileExists{\jobname-pw.ind}{\input{\jobname-pw.ind}}{}

\end{document}

      