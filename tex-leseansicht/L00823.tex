\input{../tex-inputs/latex-pdf-vorspann}
\begin{center}
            \textcolor{red}{ENTWURF. ENTZIFFERUNG NOCH NICHT KORREKTURGELESEN}
                      \end{center}
            
               \section[Arthur Schnitzler an Hugo von Hofmannsthal, 18. 7. 1898]{ Arthur Schnitzler an Hugo von Hofmannsthal, 18. 7. 1898}\nopagebreak\mylabel{v}\rehead{ }\begin{ledgroupsized}[t]{13cm}\normalsize\beginnumbering\briefempfaengerindex{Hofmannsthal, Hugo von@\textsc{Hofmannsthal, Hugo von}!zzzSchnitzler, Arthur@\emph{von Arthur Schnitzler}!1898-07-182@{18. 7. 1898}|(be} \toendnotes[C]{\smallbreak\pagebreak[2]} \Standort{FDH, Hs-30885,71.}
\physDesc{Bildpostkarte
\newline{}Handschrift: Bleistift, deutsche Kurrent\newline{}Versand: Stempel: »\nobreak{}\oindex{Tschortkiw@\textbf{Tschortkiw}|pwk}Czortków, 20 7. 98, X\nobreak{}«.  \newline{}Ordnung: von Schnitzler mit Bleistift mutmaßlich bei der
                                            Durchsicht der Briefe 1929 zweimal mit dem
                                            Datum des Stempels datiert: »20/7 98« }\buchAbdrucke{\weitereDrucke{Hugo von Hofmannsthal, Arthur Schnitzler: \emph{Briefwechsel}. Hg. Therese Nickl und Heinrich Schnitzler. Frankfurt am Main: \emph{S. Fischer} 1964, S. 107.} }\toendnotes[C]{\smallbreak}\pstart{}{\pb}Herrn Hugo von Hofmannsthal\pend{}\pstart{}\textsc{kuk Ltnd i. d. R. des kuk VIII. Ulan-Rgmts}\pend{}\pstart{}\textsc{Czortków\oindex{Tschortkiw@\textbf{Tschortkiw}|pw}}\pend{}\pstart{}\textsc{Galizien\oindex{Galizien@\textbf{Galizien}|pw}}\pend{}{\bigskip}\pstart
           \noindent{}\centering{}\textcolor{gray}{\textbf{{\pb}Liechtensteinklamm\oindex{Liechtensteinklamm@\textbf{Liechtensteinklamm}|pw}}}\pend
           \pstart
           \centering{}\textcolor{gray}{\textbf{Gruss aus der Liechtensteinklamm\oindex{Liechtensteinklamm@\textbf{Liechtensteinklamm}|pw}}}\pend
           \pstart
           \noindent{}Schöne Radtour: geſtern \introOben{}Nachm\introOben{}{ }Steinach\oindex{Steinach@\textbf{Steinach}|pw} bis Schladming\oindex{Schladming@\textbf{Schladming}|pw}; \label{K_L00823_1v}\edtext{heute}{\lemma{\textnormal{\emph{heute}}}\Cendnote{\textnormal{Die beschriebenen Ausflüge und das
                        Treffen mit den Eltern\pwindex{Hofmannsthal, Hugo August von 21.12.1841 – 08.12.1915@\textsc{Hofmannsthal, Hugo August von} (21.12.1841 – 08.12.1915), \emph{Bankdirektor}|pwkv}\pwindex{Hofmannsthal, Anna von 27.01.1849 – 22.03.1904@\textsc{Hofmannsthal, Anna von} (27.01.1849 – 22.03.1904)|pwkv}{ }Hofmannsthal\pwindex{Beer-Hofmann, Richard 11.07.1866 – 26.09.1945@\textsc{Beer-Hofmann, Richard} (11.07.1866 – 26.09.1945), \emph{Schriftsteller}|pwk}s erlauben die Datierung auf
                        den 18. 7. 1898.}}}\label{K_L00823_1h}{ }Vormitt{ }Schladming\oindex{Schladming@\textbf{Schladming}|pw}, bis zur Liechtenſteinkla{\geminationm}\oindex{Liechtensteinklamm@\textbf{Liechtensteinklamm}|pw}; heut abends werd ich wohl in der Fuſch\oindex{Bad Fusch@\textbf{Bad Fusch}|pw}
                    Ihre Eltern\pwindex{Hofmannsthal, Hugo August von 21.12.1841 – 08.12.1915@\textsc{Hofmannsthal, Hugo August von} (21.12.1841 – 08.12.1915), \emph{Bankdirektor}|pwv}\pwindex{Hofmannsthal, Anna von 27.01.1849 – 22.03.1904@\textsc{Hofmannsthal, Anna von} (27.01.1849 – 22.03.1904)|pwv}{ }ſehn. Seien Sie herzlich gegrüßt. Ihr
                        \spacefill\mbox{Arth.}\pend
           \endnumbering\briefempfaengerindex{Hofmannsthal, Hugo von@\textsc{Hofmannsthal, Hugo von}!zzzSchnitzler, Arthur@\emph{von Arthur Schnitzler}!1898-07-182@{18. 7. 1898}|)be}\mylabel{h}\end{ledgroupsized}  \newcommand{\dateiname}{L00823}\newcommand{\titel}{Arthur Schnitzler an Hugo von Hofmannsthal, 18. 7. 1898}\newcommand{\editorInnen}{Martin Anton Müller und Gerd-Hermann Susen}\input{../tex-inputs/latex-pdf-abspann}
      