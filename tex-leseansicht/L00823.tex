%% latex-korrekturansicht-vorspann.tex
%% Vorspann für die Korrekturansicht.
%% Lädt die gemeinsame Datei latex-vorspann.tex mit gesetztem Schalter.

\newif\ifkorrekturansicht
\korrekturansichttrue

\input{../tex-inputs/latex-vorspann}


\section[Arthur Schnitzler an Hugo von Hofmannsthal, 18. 7. 1898]{L00823 Arthur Schnitzler an Hugo von Hofmannsthal, 18. 7. 1898}
\nopagebreak\mylabel{L00823v}
\rehead{ }\normalsize\beginnumbering\briefempfaengerindex{Hofmannsthal, Hugo von@\textsc{Hofmannsthal, Hugo von}!zzzSchnitzler, Arthur@\emph{von Arthur Schnitzler}!1898-07-182@{18. 7. 1898}|(be}
\toendnotes[C]{\smallbreak\pagebreak[2]}\Standort{FDH, Hs-30885,71.}
\physDesc{Bildpostkarte, 278 Zeichen
\newline{}Handschrift: Bleistift, deutsche Kurrent
\newline{}Versand: Stempel: »\nobreak{}\oindex{Tschortkiw@\textbf{Tschortkiw}, \emph{P.PPLA2}|pwk}Czortków, 20 7. 98, X\nobreak{}«.  
\newline{}Ordnung: mit Bleistift von Schnitzler mutmaßlich bei der Durchsicht der
                                 Briefe 1929 zweimal mit dem Datum des Stempels
                                 datiert: »20/7 98« }
\buchAbdrucke{\weitereDrucke{Hugo von Hofmannsthal, Arthur Schnitzler: \emph{Briefwechsel}. Frankfurt am Main: \emph{S. Fischer} 1964, S. 107.} }\toendnotes[C]{\smallbreak}\pstart{}{\pb}Herrn Hugo von Hofmannsthal\pend{}\pstart{}\textsc{kuk Ltnd i. d. R. des kuk VIII. Ulan-Rgmts}\pend{}\pstart{}\textsc{Czortków\oindex{Tschortkiw@\textbf{Tschortkiw}, \emph{P.PPLA2}|pw}}\pend{}\pstart{}\textsc{Galizien\oindex{Galizien@\textbf{Galizien}, \emph{Region}|pw}}\pend{}{\bigskip}
\pstart
           \noindent{}\centering{}{\pb}\textcolor{gray}{\textbf{Liechtensteinklamm\oindex{Liechtensteinklamm@\textbf{Liechtensteinklamm}, \emph{Schlucht}|pw}}}\pend
           \vspace{1em}
\pstart
           \noindent{}{\pb}Schöne Radtour: geſtern \introOben{}Nachm\introOben{}{ }Steinach\oindex{Steinach@\textbf{Steinach}, \emph{P.PPL}|pw} bis Schladming\oindex{Schladming@\textbf{Schladming}, \emph{A.ADM3}|pw}; \label{K_L00823-1v}\edtext{heute}{\lemma{\textnormal{\emph{heute}}}\Cendnote{\textnormal{Die beschriebenen Ausflüge und das Treffen
                  mit den Eltern\pwindex{Hofmannsthal, Hugo August von 21.12.1841 – 08.12.1915@\textsc{Hofmannsthal, Hugo August von} (21.12.1841 – 08.12.1915), \emph{Bankdirektor/Bankdirektorin}|pwkv}\pwindex{Hofmannsthal, Anna von 27.01.1849 – 22.03.1904@\textsc{Hofmannsthal, Anna von} (27.01.1849 – 22.03.1904)|pwkv}{ }Hofmannsthals\pwindex{Beer-Hofmann, Richard 1866-07-11 – 1945-09-26@\textsc{Beer-Hofmann, Richard} (1866-07-11 – 1945-09-26), \emph{Schriftsteller/Schriftstellerin}|pwk} erlauben die Datierung auf
                  den 18. 7. 1898.}}}\label{K_L00823-1}{ }Vormitt{ }Schladming\oindex{Schladming@\textbf{Schladming}, \emph{A.ADM3}|pw}, bis zur Liechtenſteinkla{\geminationm}\oindex{Liechtensteinklamm@\textbf{Liechtensteinklamm}, \emph{Schlucht}|pw}; heut abends werd ich wohl in der Fuſch\oindex{Bad Fusch@\textbf{Bad Fusch}, \emph{A.ADM3}|pw}
               Ihre Eltern\pwindex{Hofmannsthal, Hugo August von 21.12.1841 – 08.12.1915@\textsc{Hofmannsthal, Hugo August von} (21.12.1841 – 08.12.1915), \emph{Bankdirektor/Bankdirektorin}|pwv}\pwindex{Hofmannsthal, Anna von 27.01.1849 – 22.03.1904@\textsc{Hofmannsthal, Anna von} (27.01.1849 – 22.03.1904)|pwv}{ }ſehn. Seien Sie herzlich gegrüßt. Ihr
                  \spacefill\mbox{Arth.}\pend
           \selectlanguage{ngerman}\endnumbering\briefempfaengerindex{Hofmannsthal, Hugo von@\textsc{Hofmannsthal, Hugo von}!zzzSchnitzler, Arthur@\emph{von Arthur Schnitzler}!1898-07-182@{18. 7. 1898}|)be}\mylabel{L00823h}  \normalsize

\doendnotes{C}
\bigskip
\vfill

\clearpage

\footnotesize

\lohead{\textsc{register}}

% Definiere theindex-Environment komplett neu ohne reledmac
\makeatletter
\renewenvironment{theindex}{%
  \section*{\indexname}%
  \setlength{\parindent}{0pt}%
  \setlength{\parskip}{0pt plus 0.3pt}%
  \let\item\@idxitem
}{%
  \clearpage
}
\makeatother

\IfFileExists{\jobname-pw.ind}{\input{\jobname-pw.ind}}{}

\end{document}

      