%% latex-leseansicht-vorspann.tex
%% Vorspann für die Leseansicht.
%% Lädt die gemeinsame Datei latex-vorspann.tex mit nicht gesetztem Schalter.

\newif\ifkorrekturansicht
\korrekturansichtfalse

\input{../tex-inputs/latex-vorspann}


         
         \renewcommand{\erwaehntePersonen}{Personen: Richard Beer-Hofmann, Hugo von Hofmannsthal, Hugo August von Hofmannsthal, Anna von Hofmannsthal}
         \renewcommand{\erwaehnteOrte}{Orte: Bad Fusch, Galizien, Liechtensteinklamm, Schladming, St. Johann im Pongau, Steinach, Tschortkiw}
         \renewcommand{\erwaehnteWerke}{}
               \section[Arthur Schnitzler an Hugo von Hofmannsthal, 18. 7. 1898]{ Arthur Schnitzler an Hugo von Hofmannsthal, 18. 7. 1898}\nopagebreak\mylabel{v}\rehead{ }\begin{ledgroupsized}[t]{13cm}\normalsize\beginnumbering \toendnotes[C]{\smallbreak\pagebreak[2]} \Standort{FDH, Hs-30885,71.}
\physDesc{Bildpostkarte, 278 Zeichen
\newline{}Handschrift: Bleistift, deutsche Kurrent
\newline{}Versand: Stempel: »\nobreak{}\oindex{Tschortkiw@\textbf{Tschortkiw}|pwk}Czortków, 20 7. 98, X\nobreak{}«.  
\newline{}Ordnung: mit Bleistift von Schnitzler mutmaßlich bei der Durchsicht der
                                 Briefe 1929 zweimal mit dem Datum des Stempels
                                 datiert: »20/7 98« }\buchAbdrucke{\weitereDrucke{Hugo von Hofmannsthal, Arthur Schnitzler: \emph{Briefwechsel}. Hg. Therese Nickl und Heinrich Schnitzler. Frankfurt am Main: \emph{S. Fischer} 1964, S. 107.} }\toendnotes[C]{\smallbreak}\pstart{}{\pb}Herrn Hugo von Hofmannsthal\pend{}\pstart{}\textsc{kuk Ltnd i. d. R. des kuk VIII. Ulan-Rgmts}\pend{}\pstart{}\textsc{Czortków\oindex{Tschortkiw@\textbf{Tschortkiw}|pw}}\pend{}\pstart{}\textsc{Galizien\oindex{Galizien@\textbf{Galizien}|pw}}\pend{}{\bigskip}\pstart
           \noindent{}\centering{}\textcolor{gray}{\textbf{{\pb}Liechtensteinklamm\oindex{Liechtensteinklamm@\textbf{Liechtensteinklamm}|pw}}}\pend
           \pstart
           \centering{}\textcolor{gray}{\textbf{Gruss aus der Liechtensteinklamm\oindex{Liechtensteinklamm@\textbf{Liechtensteinklamm}|pw}}}\pend
           \pstart
           \noindent{}Schöne Radtour: geſtern \introOben{}Nachm\introOben{}{ }Steinach\oindex{Steinach@\textbf{Steinach}|pw} bis Schladming\oindex{Schladming@\textbf{Schladming}|pw}; \label{K_L00823-1v}\edtext{heute}{\lemma{\textnormal{\emph{heute}}}\Cendnote{\textnormal{Die beschriebenen Ausflüge und das Treffen
                  mit den Eltern\pwindex{Hofmannsthal, Hugo August von 21.12.1841 – 08.12.1915@\textsc{Hofmannsthal, Hugo August von} (21.12.1841 – 08.12.1915), \emph{Bankdirektor}|pwkv}\pwindex{Hofmannsthal, Anna von 27.01.1849 – 22.03.1904@\textsc{Hofmannsthal, Anna von} (27.01.1849 – 22.03.1904)|pwkv}{ }Hofmannsthal\pwindex{Beer-Hofmann, Richard 1866-07-11 – 1945-09-26@\textsc{Beer-Hofmann, Richard} (1866-07-11 – 1945-09-26), \emph{Schriftsteller}|pwk}s erlauben die Datierung auf
                  den 18. 7. 1898.}}}\label{K_L00823-1h}{ }Vormitt{ }Schladming\oindex{Schladming@\textbf{Schladming}|pw}, bis zur Liechtenſteinkla{\geminationm}\oindex{Liechtensteinklamm@\textbf{Liechtensteinklamm}|pw}; heut abends werd ich wohl in der Fuſch\oindex{Bad Fusch@\textbf{Bad Fusch}|pw}
               Ihre Eltern\pwindex{Hofmannsthal, Hugo August von 21.12.1841 – 08.12.1915@\textsc{Hofmannsthal, Hugo August von} (21.12.1841 – 08.12.1915), \emph{Bankdirektor}|pwv}\pwindex{Hofmannsthal, Anna von 27.01.1849 – 22.03.1904@\textsc{Hofmannsthal, Anna von} (27.01.1849 – 22.03.1904)|pwv}{ }ſehn. Seien Sie herzlich gegrüßt. Ihr
                  \spacefill\mbox{Arth.}\pend
           
         
         \endnumbering\mylabel{h}\end{ledgroupsized}  \newcommand{\dateiname}{L00823}\newcommand{\titel}{Arthur Schnitzler an Hugo von Hofmannsthal, 18. 7. 1898}\newcommand{\editorInnen}{Martin Anton Müller und Gerd-Hermann Susen}%% latex-leseansicht-abspann.tex
%% Abspann für die Leseansicht.
%% Der Schalter \ifkorrekturansicht ist bereits durch den Vorspann gesetzt.

%% latex-abspann.tex
%% Gemeinsamer Abspann für Korrekturansicht und Leseansicht.
%% Setzt den Schalter \ifkorrekturansicht voraus (gesetzt in den
%% einbindenden Dateien latex-korrekturansicht-abspann.tex bzw.
%% latex-leseansicht-abspann.tex).
%% ---------------------------------------------------------------

\normalsize

% Das esempio-Environment wird nur in der Leseansicht benötigt
\ifkorrekturansicht\else
\newenvironment{esempio}[3]%
{
    \vspace{1.5ex}
    \rlap{\underline{#1}}
    \par
    \setlength{\parindent}{0cm}
    \nopagebreak
    \leftskip=#2cm
    \rightskip=#3cm
}
{
    \par
}
\fi

\doendnotes{C}
\bigskip
\vfill

\clearpage

\footnotesize

\ifkorrekturansicht
  \lohead{\textsc{register}}
\fi

% theindex-Environment neu definieren ohne reledmac
\makeatletter
\renewenvironment{theindex}{%
  \ifkorrekturansicht
    \section*{\indexname}%
  \else
    \subsubsection*{Index der erwähnten Entitäten}%
  \fi
  \setlength{\parindent}{0pt}%
  \setlength{\parskip}{0pt plus 0.3pt}%
  \let\item\@idxitem
}{%
  \ifkorrekturansicht\clearpage\fi
}
\makeatother

\IfFileExists{\jobname-pw.ind}{\input{\jobname-pw.ind}}{}

% Quellenangabe nur in der Leseansicht
\ifkorrekturansicht\else
% Fallback-Definitionen, falls die .tex-Datei \titel etc. nicht gesetzt hat
\providecommand{\titel}{}
\providecommand{\editorInnen}{}
\providecommand{\dateiname}{\jobname}

\vspace{3cm}

\vfill

\footnotesize
\textsc{Quelle}: \titel. Herausgegeben von {\editorInnen}. In: \emph{Arthur Schnitzler: Briefwechsel mit Autorinnen und Autoren}.
 Digitale Edition, https://schnitzler-briefe.acdh.oeaw.ac.at/{\dateiname}.html (Stand \today)
\fi

\end{document}


      