%% latex-leseansicht-vorspann.tex
%% Vorspann für die Leseansicht.
%% Lädt die gemeinsame Datei latex-vorspann.tex mit nicht gesetztem Schalter.

\newif\ifkorrekturansicht
\korrekturansichtfalse

\input{../tex-inputs/latex-vorspann}


               \section[Arthur Schnitzler an Richard Beer-Hofmann, 12. 8. 1918]{ Arthur Schnitzler an Richard Beer-Hofmann, 12. 8. 1918}\nopagebreak\mylabel{v}\rehead{ }\begin{ledgroupsized}[t]{13cm}\normalsize\beginnumbering\briefempfaengerindex{Beer-Hofmann, Richard@\textsc{Beer-Hofmann, Richard}!zzzSchnitzler, Arthur@\emph{von Arthur Schnitzler}!1918-08-121@{12. 8. 1918}|(be} \toendnotes[C]{\smallbreak\pagebreak[2]} \Standort{YCGL, MSS 31.}
\physDesc{Postkarte
\newline{}Handschrift: Bleistift, lateinische Kurrent\newline{}Versand: Stempel: »\nobreak{}\oindex{I., Innere Stadt@\textbf{I., Innere Stadt}|pwk}1/1 Wien 8, 13. VIII. 18, 1\nobreak{}«.  
\newline{}Beer-Hofmann: mit blauem Buntstift Erhalt und Beantwortung vermerkt: »\noindent{}E.{ / }B. 13/VIII 18« }\buchAbdrucke{\weitereDrucke{Arthur Schnitzler, Richard Beer-Hofmann: \emph{Briefwechsel 1891–1931}. Hg. Konstanze Fliedl. Wien, Zürich: \emph{Europaverlag} 1992, S. 225–226.} }\toendnotes[C]{\smallbreak}\pstart{}{\pb}\textcolor{gray}{\textbf{D\textsuperscript{R} ARTHUR SCHNITZLER}}\pend{}\pstart{}\textcolor{gray}{\textbf{WIEN, XVIII. STERNWARTESTRASSE 71\oindex{Sternwartestrasse@\textbf{Sternwartestraße}|pw}.}}\pend{}{\bigskip}\pstart{}Hrn Dr Richard Beer Hofmann\pend{}\pstart{}Bad Ischl\oindex{Bad Ischl@\textbf{Bad Ischl}|pw}\pend{}\pstart{}Grazerstraße 56\oindex{Grazer Strasse@\textbf{Grazer Straße}|pw}.\pend{}{\bigskip}\pstart
           \raggedleft{}{\pb}12. 8. 18\pend
           \pstart
           lieber Richard, es wäre nicht undenkbar, daß ich mich auf der Reise
               nach Bayern\oindex{Bayern@\textbf{Bayern}|pw} in Salzburg\oindex{Salzburg@\textbf{Salzburg}|pw} aufhielte. Bitte schreiben Sie mir ein Wort, ob Sie in der nächsten
               Woche (etwa um 22., 23., 24.) dort sind – da
               Sie doch, we{\geminationn} ich gut unterrichtet bin, um des Leopoldskron\oindex{Salzburg-Leopoldskron@\textbf{Salzburg-Leopoldskron}|pw}er Schloßherrn\pwindex{Reinhardt, Max 09.09.1873 – 30.10.1943@\textsc{Reinhardt, Max} (09.09.1873 – 30.10.1943), \emph{Theaterleiter, Regisseur, Schauspieler}|pwv} willen
               hinzufahren gedenken. Ich hoffe, Sie fühlen sich, nach der unfreiwilligen \label{KLL02295_Beer-Hofmann-1v}\edtext{Unterbrechung}{\lemma{\textnormal{\emph{Unterbrechung}}}\Cendnote{\textnormal{In seinem Haus war
                  eingebrochen worden. Aus diesem Zweck war er für kurze Zeit in Wien\oindex{Wien@\textbf{Wien}|pwk} gewesen.}}}\label{KLL02295_Beer-Hofmann-1h}, wohler als vorher, – auch das Wetter
               scheint sich ja besinnen zu wollen. Alles übrige sieht freilich nicht {\pb}nach Besserwerden aus. Grüßen Sie die Ihrigen\pwindex{Beer-Hofmann, Naemah 20.12.1898 – 10.11.1971@\textsc{Beer-Hofmann, Naëmah} (20.12.1898 – 10.11.1971)|pwv}\pwindex{Beer-Hofmann, Mirjam 04.09.1897 – 24.12.1984@\textsc{Beer-Hofmann, Mirjam} (04.09.1897 – 24.12.1984)|pwv}\pwindex{Beer-Hofmann, Paula 25.02.1879 – 30.10.1939@\textsc{Beer-Hofmann, Paula} (25.02.1879 – 30.10.1939)|pwv}\pwindex{Beer-Hofmann, Gabriel 09.01.1901 – 24.03.1971@\textsc{Beer-Hofmann, Gabriel} (09.01.1901 – 24.03.1971), \emph{Schriftsteller, Filmagent}|pwv}. Von
               Herzen Ihr\pend
           \pstart \spacefill\mbox{Arthur}\pend{}          \endnumbering\briefempfaengerindex{Beer-Hofmann, Richard@\textsc{Beer-Hofmann, Richard}!zzzSchnitzler, Arthur@\emph{von Arthur Schnitzler}!1918-08-121@{12. 8. 1918}|)be}\mylabel{h}\end{ledgroupsized}  \newcommand{\dateiname}{L02295}\newcommand{\titel}{Arthur Schnitzler an Richard Beer-Hofmann, 12. 8. 1918}\newcommand{\editorInnen}{Martin Anton Müller und Gerd-Hermann Susen}
            \footnotesize
\begin{ledgroupsized}[t]{11.5cm}
\doendnotes{C}
\end{ledgroupsized}
         %% latex-leseansicht-abspann.tex
%% Abspann für die Leseansicht.
%% Der Schalter \ifkorrekturansicht ist bereits durch den Vorspann gesetzt.

%% latex-abspann.tex
%% Gemeinsamer Abspann für Korrekturansicht und Leseansicht.
%% Setzt den Schalter \ifkorrekturansicht voraus (gesetzt in den
%% einbindenden Dateien latex-korrekturansicht-abspann.tex bzw.
%% latex-leseansicht-abspann.tex).
%% ---------------------------------------------------------------

\normalsize

% Das esempio-Environment wird nur in der Leseansicht benötigt
\ifkorrekturansicht\else
\newenvironment{esempio}[3]%
{
    \vspace{1.5ex}
    \rlap{\underline{#1}}
    \par
    \setlength{\parindent}{0cm}
    \nopagebreak
    \leftskip=#2cm
    \rightskip=#3cm
}
{
    \par
}
\fi

\doendnotes{C}
\bigskip
\vfill

\clearpage

\footnotesize

\ifkorrekturansicht
  \lohead{\textsc{register}}
\fi

% theindex-Environment neu definieren ohne reledmac
\makeatletter
\renewenvironment{theindex}{%
  \ifkorrekturansicht
    \section*{\indexname}%
  \else
    \subsubsection*{Index der erwähnten Entitäten}%
  \fi
  \setlength{\parindent}{0pt}%
  \setlength{\parskip}{0pt plus 0.3pt}%
  \let\item\@idxitem
}{%
  \ifkorrekturansicht\clearpage\fi
}
\makeatother

\IfFileExists{\jobname-pw.ind}{\input{\jobname-pw.ind}}{}

% Quellenangabe nur in der Leseansicht
\ifkorrekturansicht\else
% Fallback-Definitionen, falls die .tex-Datei \titel etc. nicht gesetzt hat
\providecommand{\titel}{}
\providecommand{\editorInnen}{}
\providecommand{\dateiname}{\jobname}

\vspace{3cm}

\vfill

\footnotesize
\textsc{Quelle}: \titel. Herausgegeben von {\editorInnen}. In: \emph{Arthur Schnitzler: Briefwechsel mit Autorinnen und Autoren}.
 Digitale Edition, https://schnitzler-briefe.acdh.oeaw.ac.at/{\dateiname}.html (Stand \today)
\fi

\end{document}


      