%% latex-korrekturansicht-vorspann.tex
%% Vorspann für die Korrekturansicht.
%% Lädt die gemeinsame Datei latex-vorspann.tex mit gesetztem Schalter.

\newif\ifkorrekturansicht
\korrekturansichttrue

\input{../tex-inputs/latex-vorspann}


\section[Arthur Schnitzler an Richard Beer-Hofmann, 28. 6. 1905]{L01529 Arthur Schnitzler an Richard Beer-Hofmann, 28. 6. 1905}
\nopagebreak\mylabel{L01529v}
\rehead{ }\normalsize\beginnumbering\briefempfaengerindex{Beer-Hofmann, Richard@\textsc{Beer-Hofmann, Richard}!zzzSchnitzler, Arthur@\emph{von Arthur Schnitzler}!1905-06-281@{28. 6. 1905}|(be}
\toendnotes[C]{\smallbreak\pagebreak[2]}\Standort{YCGL, MSS 31.}
\physDesc{Brief, 1 Blatt, 2 Seiten, Umschlag, 216 Zeichen
\newline{}Handschrift: Bleistift, deutsche Kurrent
\newline{}Versand: 1) Stempel: »\nobreak{}\oindex{IX., Alsergrund@\textbf{IX., Alsergrund}, \emph{A.ADM3}|pwk}9/4 Wien, 2\textcolor{gray}{8}. VI. \textcolor{gray}{05}, 1\textcolor{gray}{7}\nobreak{}«.   2) Stempel: »\nobreak{}\oindex{Rodaun@\textbf{Rodaun}, \emph{A.ADM4}|pwk}Rodaun, 29 {[}6. 1905{]}\nobreak{}«. }\pstart{}{\pb}\textcolor{gray}{\textbf{Dr. Arthur Schnitzler}}\pend{}\pstart{}\textcolor{gray}{\textbf{Wien, XVIII. Spoettelgasse 7\oindex{Edmund-Weiss-Gasse 7@\textbf{Edmund-Weiß-Gasse 7}, \emph{Wohngebäude (K.WHS)}|pw}.}}\pend{}{\bigskip}\pstart{}{\pb}\textsc{Herrn Doctor}\pend{}\pstart{}\textsc{Richard Beer-Hofmann}\pend{}\pstart{}\textsc{Rodaun bei Liesing}\oindex{Rodaun@\textbf{Rodaun}, \emph{A.ADM4}|pw}\pend{}\pstart{}\textsc{Liesingerstraße 1}\oindex{Liesingerstrasse@\textbf{Liesingerstraße}, \emph{Straße (K.STR)}|pw}.\pend{}{\bigskip}\vspace{1em}
\pstart
           \raggedleft{}{\pb}28/6 905\pend
           
\pstart{}lieber Richard\pend\vspace{0.5em}
\pstart
           unſre Briefe haben ſich gekreuzt. Es bleibt alſo bei Samſtag.\pend
           
\pstart
           Die »Juden\pwindex{Juden. Schauspiel in drei Aufzuegen@\emph{Die Juden. Schauspiel in drei Aufzügen}|pw}« müſſen Sie ſich anſehen.\pend
           
\pstart
           {\pb}Auf Wiederſehen{\\}Herzlichſt{\\}Ihr{\\}\spacefill\mbox{A.}\pend
           \selectlanguage{ngerman}\endnumbering\briefempfaengerindex{Beer-Hofmann, Richard@\textsc{Beer-Hofmann, Richard}!zzzSchnitzler, Arthur@\emph{von Arthur Schnitzler}!1905-06-281@{28. 6. 1905}|)be}\mylabel{L01529h}  \normalsize

\doendnotes{C}
\bigskip
\vfill

\clearpage

\footnotesize

\lohead{\textsc{register}}

% Definiere theindex-Environment komplett neu ohne reledmac
\makeatletter
\renewenvironment{theindex}{%
  \section*{\indexname}%
  \setlength{\parindent}{0pt}%
  \setlength{\parskip}{0pt plus 0.3pt}%
  \let\item\@idxitem
}{%
  \clearpage
}
\makeatother

\IfFileExists{\jobname-pw.ind}{\input{\jobname-pw.ind}}{}

\end{document}

      