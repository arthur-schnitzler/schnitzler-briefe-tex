%% latex-korrekturansicht-vorspann.tex
%% Vorspann für die Korrekturansicht.
%% Lädt die gemeinsame Datei latex-vorspann.tex mit gesetztem Schalter.

\newif\ifkorrekturansicht
\korrekturansichttrue

\input{../tex-inputs/latex-vorspann}


\section[Arthur Schnitzler an Hugo von Hofmannsthal, 6. 3. 1906]{L01587 Arthur Schnitzler an Hugo von Hofmannsthal, 6. 3. 1906}
\nopagebreak\mylabel{L01587v}
\rehead{ }\normalsize\beginnumbering\briefempfaengerindex{Hofmannsthal, Hugo von@\textsc{Hofmannsthal, Hugo von}!zzzSchnitzler, Arthur@\emph{von Arthur Schnitzler}!1906-03-061@{6. 3. 1906}|(be}
\toendnotes[C]{\smallbreak\pagebreak[2]}\Standort{FDH, Hs-30885,124.}
\physDesc{Brief, 2 Blätter, 8 Seiten, 3107 Zeichen
\newline{}Handschrift: schwarze Tinte, deutsche Kurrent
\newline{}Ordnung: mit Bleistift von Schnitzler mutmaßlich bei der Durchsicht der
                                 Korrespondenz 1929 das zweite Blatt datiert: »6/3 906« und nummeriert: »II.« }
\buchAbdrucke{\weitereDrucke{Hugo von Hofmannsthal, Arthur Schnitzler: \emph{Briefwechsel}. Frankfurt am Main: \emph{S. Fischer} 1964, S. 218.} }\toendnotes[C]{\smallbreak}
\pstart
           
\pstart
           {\pb}\textcolor{gray}{\textbf{Dr. Arthur Schnitzler}}\pend
           
\pstart
           \raggedleft{}6. 3. 906\pend
           \pend
           
\pstart
           \textcolor{gray}{\textbf{Wien XVIII. Spoettelgasse 7\oindex{Edmund-Weiss-Gasse 7@\textbf{Edmund-Weiß-Gasse 7}, \emph{Wohngebäude (K.WHS)}|pw}.}}\pend
           
\pstart{}mein lieber Hugo, \pend\vspace{0.5em}
\pstart
           aus verſchiedenen Gründen ſind wir erſt Samſtag Abend frei u Ihnen zur
               Verfügung und fragen Sie, ob Sie lieber bei uns nachtmahlen \strikeout{wollen} oder ob wir einander in Hietzing\oindex{XIII., Hietzing@\textbf{XIII., Hietzing}, \emph{A.ADM3}|pw}
               treffen wollen? Es wäre ſehr nett von Ihnen beiden\pwindex{Hofmannsthal, Gertrude von 16.03.1880 – 09.11.1959@\textsc{Hofmannsthal, Gertrude von} (16.03.1880 – 09.11.1959)|pwv}, wenn Sie die Reiſe in die Spöttelgaſſe\oindex{Edmund-Weiss-Gasse 7@\textbf{Edmund-Weiß-Gasse 7}, \emph{Wohngebäude (K.WHS)}|pw} nicht ſcheuten. –\pend
           
\pstart
           {\pb}Harden\pwindex{Harden, Maximilian 20.10.1861 – 30.10.1927@\textsc{Harden, Maximilian} (20.10.1861 – 30.10.1927), \emph{Schriftsteller/Schriftstellerin, Publizist/Publizistin}|pw} hat mich nur mäßig irritirt. Erſtens
               weil ich auf alles mögliche gefaſſt war, da man mir ja gleich (Theaterberlin\oindex{Berlin@\textbf{Berlin}, \emph{P.PPLC}|pw} iſt ja ein Tratſchneſt) von ſeinem albern
               taktloſen Benehmen im Theater bei der \textsc{Première}\pwindex{Ruf des Lebens. Schauspiel in drei Akten@\emph{Der Ruf des Lebens. Schauspiel in drei Akten}|pwv} erzählt hatte. Ferner iſt mir ſeine Erſcheinung als die eines Politikers, eines
               großen u amuſanten Politikers in allen Dingen dieſer Welt alſo auch in der Kunſt (und
               ſogar in der Politik) ſeit lange ſo feſtſtehend, {\pb}daſs
               mir alle ſeine Emanationen auch nur in dieſem Sinne wirklich intereſſant ſind. Daſs
               er trotzdem manchmal höchſt vorzügliches \substVorne{}\textsuperscript{mit}\substDazwischen{}und\substHinten{}{ }\strikeout{über} ſogar treffendes über Menſchen, Künſtler,
               Bücher, Stücke ſagt – insbeſondere wenn er vom »politiſchen« abſehen kann, und noch
               öfter, wenn ſein Geſchmack und ſeine Parteiſtellung in einer ihm ſelbſt unbewußten
               Weiſe ineinanderfließen – würd ich nicht leugnen, auch we{\geminationn} er noch lächerlicher über mich geſchrie{\pb}ben\pwindex{Theater@\emph{Theater}|pwv} hätte. Im übrigen hab ich nicht
               einmal die Empfindung, daſs er mich hat treffen wollen, und käme der Fall vor
               Gericht, ſo würd ich ihn vielleicht wegen momentaner Si{\geminationn}esverwirrung freiſprechen. Ja we{\geminationn} ich alle die
               vielfältigen Elemente meines heutigen Verhältniſſes zu ihm unterſuche, ſo möcht ich
               faſt glauben, dſs auch irgend ein Hauch von Mitleid dabei iſt.\pend
           
\pstart
           Nun was das Stück\pwindex{Ruf des Lebens. Schauspiel in drei Akten@\emph{Der Ruf des Lebens. Schauspiel in drei Akten}|pwv}{ }ſelbſt anbelangt ſo iſt ja beim beſten Willen nicht
               zu überſehen, daſs im 3. Akt\pwindex{Ruf des Lebens. Schauspiel in drei Akten@\emph{Der Ruf des Lebens. Schauspiel in drei Akten}|pwv}
               ein {\pb}tiefer Fehler ſteckt – der damit nicht geringer
               erklärt wird, daſs man ihn \introOben{}im\introOben{} architektoniſchen am
               deutlichſten entdeckt. Auf einem Spaziergang heute, an dieſem ſchönen Frühlingstag,
               durch den Dornbacherpark\oindex{Schwarzenbergpark@\textbf{Schwarzenbergpark}, \emph{Park (K.PRK)}|pw}, hab ich mir den »Ruf\pwindex{Ruf des Lebens. Schauspiel in drei Akten@\emph{Der Ruf des Lebens. Schauspiel in drei Akten}|pw}« neu entworfen (ſchreiben werd ich ihn wohl
               nie) in fünf Akten und glaube an den Wurzeln geweſen zu ſein. So klug wie meine
               klügſten Kritiker bin ich lange noch: ich müßte {\pb}nur
               noch um einiges mehr Dichter ſein und die Welt \substVorne{}\textsuperscript{könnte}\substDazwischen{}dürfte\substHinten{} Dramen von mir erwarten, die weder durch die Talentloſigkeit des Fräulein
                  Schiff\pwindex{Bassermann, Else 14.01.1878 – 30.05.1961@\textsc{Bassermann, Else} (14.01.1878 – 30.05.1961), \emph{Schauspieler/Schauspielerin}|pw} noch durch die Boſheit des Herrn Rittner\pwindex{Rittner, Rudolf 30.06.1869 – 04.02.1943@\textsc{Rittner, Rudolf} (30.06.1869 – 04.02.1943), \emph{Theaterleiter/Theaterleiterin, Schauspieler/Schauspielerin}|pw} umzubringen wären.\pend
           
\pstart
           Im Oedipus\pwindex{Oedipus und die Sphinx. Tragoedie in drei Aufzuegen@\emph{Oedipus und die Sphinx. Tragödie in drei Aufzügen}|pw} haben die \textsc{Sandrock}\pwindex{Sandrock, Adele 1863-08-19 – 1937-08-30@\textsc{Sandrock, Adele} (1863-08-19 – 1937-08-30), \emph{Schauspieler/Schauspielerin}|pw} und \textsc{Moissi}\pwindex{Moissi, Alexander 02.04.1879 – 22.03.1935@\textsc{Moissi, Alexander} (02.04.1879 – 22.03.1935), \emph{Schauspieler/Schauspielerin}|pw} am ſtärkſten auf mich gewirkt (\label{K_L01587-1v}\edtext{Dinſtag den 24. Feber}{\lemma{\textnormal{\emph{Dinſtag den 24. Feber}}}\Cendnote{\textnormal{Er war am 26. 2. 1906, einem Montag, in der
                  Vorführung.}}}\label{K_L01587-1}), die \textsc{Sorma}\pwindex{Sorma, Agnes 17.05.1862 – 10.02.1927@\textsc{Sorma, Agnes} (17.05.1862 – 10.02.1927), \emph{Schauspieler/Schauspielerin}|pw} bei aller edeln Süßigkeit ſchien mir nicht ohne Manier. Was mit dem Chor \introOben{}(von Reinhardt\pwindex{Reinhardt, Max 09.09.1873 – 30.10.1943@\textsc{Reinhardt, Max} (09.09.1873 – 30.10.1943), \emph{Theaterleiter/Theaterleiterin, Regisseur/Regisseurin, Schauspieler/Schauspielerin}|pw})\introOben{}
               intendirt war, hat mich mächtig ergriffen, in der Ausführung ſtörte mich zuweilen
               bildlich {\pb}geſprochen die überdeutliche Arbeit der
               Maſchinerie. Was mich aus dem dritten Akt des Werkes\pwindex{Oedipus und die Sphinx. Tragoedie in drei Aufzuegen@\emph{Oedipus und die Sphinx. Tragödie in drei Aufzügen}|pwv}, das ich bewundere, etwas kühl angeweht hat, weiſs
               ich mir ſelbſt noch nicht recht zu deuten – vielleicht war es nichts andres, als daſs
               ich nach Hauſe geſchickt wurde, während ich, in höherm Sinn, nur in einen Zwiſchenakt
               entlaſſen werden durfte. Um was ich Sie diesmal beſonders beneide, iſt, daſs Sie mit
               einem Regiſſeur\pwindex{Reinhardt, Max 09.09.1873 – 30.10.1943@\textsc{Reinhardt, Max} (09.09.1873 – 30.10.1943), \emph{Theaterleiter/Theaterleiterin, Regisseur/Regisseurin, Schauspieler/Schauspielerin}|pwv} arbeiten
               konnten, der an Ihr Werk glaubte. Die \substVorne{}\textsuperscript{Mischung}\substDazwischen{}Atmosphäre\substHinten{} von Pflichttreue und künſtleriſcher Feindſeligkeit, in der \strikeout{mich} mein Werk\pwindex{Ruf des Lebens. Schauspiel in drei Akten@\emph{Der Ruf des Lebens. Schauspiel in drei Akten}|pwv} zum Bühnenleben erwuchs, hatte {\pb}etwas niederdrückendes.\pend
           
\pstart
           Herzlichſt{\\[\baselineskip]}Ihr{\\[\baselineskip]}\spacefill\mbox{A.}\pend
           \leftskip=0em{}\selectlanguage{ngerman}\endnumbering\briefempfaengerindex{Hofmannsthal, Hugo von@\textsc{Hofmannsthal, Hugo von}!zzzSchnitzler, Arthur@\emph{von Arthur Schnitzler}!1906-03-061@{6. 3. 1906}|)be}\mylabel{L01587h}  \normalsize

\doendnotes{C}
\bigskip
\vfill

\clearpage

\footnotesize

\lohead{\textsc{register}}

% Definiere theindex-Environment komplett neu ohne reledmac
\makeatletter
\renewenvironment{theindex}{%
  \section*{\indexname}%
  \setlength{\parindent}{0pt}%
  \setlength{\parskip}{0pt plus 0.3pt}%
  \let\item\@idxitem
}{%
  \clearpage
}
\makeatother

\IfFileExists{\jobname-pw.ind}{\input{\jobname-pw.ind}}{}

\end{document}

      