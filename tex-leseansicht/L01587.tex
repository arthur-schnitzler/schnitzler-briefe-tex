%% latex-leseansicht-vorspann.tex
%% Vorspann für die Leseansicht.
%% Lädt die gemeinsame Datei latex-vorspann.tex mit nicht gesetztem Schalter.

\newif\ifkorrekturansicht
\korrekturansichtfalse

\input{../tex-inputs/latex-vorspann}


         
         \renewcommand{\erwaehntePersonen}{Personen: Else Bassermann, Maximilian Harden, Hugo von Hofmannsthal, Gertrude von Hofmannsthal, Alexander Moissi, Max Reinhardt, Rudolf Rittner, Adele Sandrock, Agnes Sorma}
         \renewcommand{\erwaehnteOrte}{Orte: Berlin, Dornbacher Park, Edmund-Weiß-Gasse 7, Wien, XIII., Hietzing}
         \renewcommand{\erwaehnteWerke}{Werke: Der Ruf des Lebens. Schauspiel in drei Akten, Oedipus und die Sphinx. Tragödie in drei Aufzügen, Theater}
               \section[Arthur Schnitzler an Hugo von Hofmannsthal, 6. 3. 1906]{ Arthur Schnitzler an Hugo von Hofmannsthal, 6. 3. 1906}\nopagebreak\mylabel{v}\rehead{ }\begin{ledgroupsized}[t]{13cm}\normalsize\beginnumbering \toendnotes[C]{\smallbreak\pagebreak[2]} \Standort{FDH, Hs-30885,124.}
\physDesc{Brief, 2 Blätter, 8 Seiten, 3107 Zeichen
\newline{}Handschrift: schwarze Tinte, deutsche Kurrent
\newline{}Ordnung: mit Bleistift von Schnitzler mutmaßlich bei der Durchsicht der
                                 Korrespondenz 1929 das zweite Blatt datiert: »6/3 906« und nummeriert: »II.« }\buchAbdrucke{\weitereDrucke{Hugo von Hofmannsthal, Arthur Schnitzler: \emph{Briefwechsel}. Hg. Therese Nickl und Heinrich Schnitzler. Frankfurt am Main: \emph{S. Fischer} 1964, S. 218.} }\toendnotes[C]{\smallbreak}\pstart
           {\pb}\textcolor{gray}{\textbf{Dr. Arthur Schnitzler}}\hfill 6. 3. 906\pend
           \pstart
           \textcolor{gray}{\textbf{Wien XVIII. Spoettelgasse 7\oindex{Edmund-Weiss-Gasse 7@\textbf{Edmund-Weiß-Gasse 7}|pw}.}}\pend
           \pstart{}mein lieber Hugo, \pend\pstart
           aus verſchiedenen Gründen ſind wir erſt Samſtag Abend frei u Ihnen zur
               Verfügung und fragen Sie, ob Sie lieber bei uns nachtmahlen \strikeout{wollen} oder ob wir einander in Hietzing\oindex{XIII., Hietzing@\textbf{XIII., Hietzing}|pw}
               treffen wollen? Es wäre ſehr nett von Ihnen beiden\pwindex{Hofmannsthal, Gertrude von 16.03.1880 – 09.11.1959@\textsc{Hofmannsthal, Gertrude von} (16.03.1880 – 09.11.1959)|pwv}, wenn Sie die Reiſe in die Spöttelgaſſe\oindex{Edmund-Weiss-Gasse 7@\textbf{Edmund-Weiß-Gasse 7}|pw} nicht ſcheuten. –\pend
           \pstart
           {\pb}Harden\pwindex{Harden, Maximilian 20.10.1861 – 30.10.1927@\textsc{Harden, Maximilian} (20.10.1861 – 30.10.1927), \emph{Schriftsteller, Publizist}|pw} hat mich nur mäßig irritirt. Erſtens
               weil ich auf alles mögliche gefaſſt war, da man mir ja gleich (Theaterberlin\oindex{Berlin@\textbf{Berlin}|pw} iſt ja ein Tratſchneſt) von ſeinem albern
               taktloſen Benehmen im Theater bei der \textsc{Première}\pwindex{Schnitzler, Arthur 15.05.1862 – 21.10.1931@\textsc{Schnitzler, Arthur} (15.05.1862 – 21.10.1931), \emph{Schriftsteller, Mediziner}!Ruf des Lebens. Schauspiel in drei Akten1906-02-20@\strich\emph{Der Ruf des Lebens. Schauspiel in drei Akten} {[}1906-02-20{]}|pwv} erzählt hatte. Ferner iſt mir ſeine Erſcheinung als die eines Politikers, eines
               großen u amuſanten Politikers in allen Dingen dieſer Welt alſo auch in der Kunſt (und
               ſogar in der Politik) ſeit lange ſo feſtſtehend, {\pb}daſs
               mir alle ſeine Emanationen auch nur in dieſem Sinne wirklich intereſſant ſind. Daſs
               er trotzdem manchmal höchſt vorzügliches \substVorne{}\textsuperscript{mit}\substDazwischen{}und\substHinten{}{ }\strikeout{über} ſogar treffendes über Menſchen, Künſtler,
               Bücher, Stücke ſagt – insbeſondere wenn er vom »politiſchen« abſehen kann, und noch
               öfter, wenn ſein Geſchmack und ſeine Parteiſtellung in einer ihm ſelbſt unbewußten
               Weiſe ineinanderfließen – würd ich nicht leugnen, auch we{\geminationn} er noch lächerlicher über mich geſchrie{\pb}ben\pwindex{Theater03. 03. 1906@\emph{Theater} {[}03. 03. 1906{]}|pwv} hätte. Im übrigen hab ich nicht
               einmal die Empfindung, daſs er mich hat treffen wollen, und käme der Fall vor
               Gericht, ſo würd ich ihn vielleicht wegen momentaner Si{\geminationn}esverwirrung freiſprechen. Ja we{\geminationn} ich alle die
               vielfältigen Elemente meines heutigen Verhältniſſes zu ihm unterſuche, ſo möcht ich
               faſt glauben, dſs auch irgend ein Hauch von Mitleid dabei iſt.\pend
           \pstart
           Nun was das Stück\pwindex{Schnitzler, Arthur 15.05.1862 – 21.10.1931@\textsc{Schnitzler, Arthur} (15.05.1862 – 21.10.1931), \emph{Schriftsteller, Mediziner}!Ruf des Lebens. Schauspiel in drei Akten1906-02-20@\strich\emph{Der Ruf des Lebens. Schauspiel in drei Akten} {[}1906-02-20{]}|pwv}{ }ſelbſt anbelangt ſo iſt ja beim beſten Willen nicht
               zu überſehen, daſs im 3. Akt\pwindex{Schnitzler, Arthur 15.05.1862 – 21.10.1931@\textsc{Schnitzler, Arthur} (15.05.1862 – 21.10.1931), \emph{Schriftsteller, Mediziner}!Ruf des Lebens. Schauspiel in drei Akten1906-02-20@\strich\emph{Der Ruf des Lebens. Schauspiel in drei Akten} {[}1906-02-20{]}|pwv}
               ein {\pb}tiefer Fehler ſteckt – der damit nicht geringer
               erklärt wird, daſs man ihn \introOben{}im\introOben{} architektoniſchen am
               deutlichſten entdeckt. Auf einem Spaziergang heute, an dieſem ſchönen Frühlingstag,
               durch den Dornbacherpark\oindex{Dornbacher Park@\textbf{Dornbacher Park}|pw}, hab ich mir den »Ruf\pwindex{Schnitzler, Arthur 15.05.1862 – 21.10.1931@\textsc{Schnitzler, Arthur} (15.05.1862 – 21.10.1931), \emph{Schriftsteller, Mediziner}!Ruf des Lebens. Schauspiel in drei Akten1906-02-20@\strich\emph{Der Ruf des Lebens. Schauspiel in drei Akten} {[}1906-02-20{]}|pw}« neu entworfen (ſchreiben werd ich ihn wohl
               nie) in fünf Akten und glaube an den Wurzeln geweſen zu ſein. So klug wie meine
               klügſten Kritiker bin ich lange noch: ich müßte {\pb}nur
               noch um einiges mehr Dichter ſein und die Welt \substVorne{}\textsuperscript{könnte}{\allowbreak}\substDazwischen{}dürfte\substHinten{} Dramen von mir erwarten, die weder durch die Talentloſigkeit des Fräulein
                  Schiff\pwindex{Bassermann, Else 14.01.1878 – 30.05.1961@\textsc{Bassermann, Else} (14.01.1878 – 30.05.1961), \emph{Schauspielerin}|pw} noch durch die Boſheit des Herrn Rittner\pwindex{Rittner, Rudolf 30.06.1869 – 04.02.1943@\textsc{Rittner, Rudolf} (30.06.1869 – 04.02.1943), \emph{Theaterleiter, Schauspieler}|pw} umzubringen wären.\pend
           \pstart
           Im Oedipus\pwindex{Hofmannsthal, Hugo von 1874-02-01 – 1929-07-15@\textsc{Hofmannsthal, Hugo von} (1874-02-01 – 1929-07-15), \emph{Schriftsteller}!Oedipus und die Sphinx. Tragoedie in drei Aufzuegen1906@\strich\emph{Oedipus und die Sphinx. Tragödie in drei Aufzügen} {[}1906{]}|pw} haben die \textsc{Sandrock}\pwindex{Sandrock, Adele 1863-08-19 – 1937-08-30@\textsc{Sandrock, Adele} (1863-08-19 – 1937-08-30), \emph{Schauspielerin}|pw} und \textsc{Moissi}\pwindex{Moissi, Alexander 02.04.1879 – 22.03.1935@\textsc{Moissi, Alexander} (02.04.1879 – 22.03.1935), \emph{Schauspieler}|pw} am ſtärkſten auf mich gewirkt (\label{K_L01587-1v}\edtext{Dinſtag den 24. Feber}{\lemma{\textnormal{\emph{Dinſtag den 24. Feber}}}\Cendnote{\textnormal{Er war am 26. 2. 1906, einem Montag, in der
                  Vorführung.}}}\label{K_L01587-1h}), die \textsc{Sorma}\pwindex{Sorma, Agnes 17.05.1862 – 10.02.1927@\textsc{Sorma, Agnes} (17.05.1862 – 10.02.1927), \emph{Schauspielerin}|pw} bei aller edeln Süßigkeit ſchien mir nicht ohne Manier. Was mit dem Chor \introOben{}(von Reinhardt\pwindex{Reinhardt, Max 09.09.1873 – 30.10.1943@\textsc{Reinhardt, Max} (09.09.1873 – 30.10.1943), \emph{Theaterleiter, Regisseur, Schauspieler}|pw})\introOben{}
               intendirt war, hat mich mächtig ergriffen, in der Ausführung ſtörte mich zuweilen
               bildlich {\pb}geſprochen die überdeutliche Arbeit der
               Maſchinerie. Was mich aus dem dritten Akt des Werkes\pwindex{Hofmannsthal, Hugo von 1874-02-01 – 1929-07-15@\textsc{Hofmannsthal, Hugo von} (1874-02-01 – 1929-07-15), \emph{Schriftsteller}!Oedipus und die Sphinx. Tragoedie in drei Aufzuegen1906@\strich\emph{Oedipus und die Sphinx. Tragödie in drei Aufzügen} {[}1906{]}|pwv}, das ich bewundere, etwas kühl angeweht hat, weiſs
               ich mir ſelbſt noch nicht recht zu deuten – vielleicht war es nichts andres, als daſs
               ich nach Hauſe geſchickt wurde, während ich, in höherm Sinn, nur in einen Zwiſchenakt
               entlaſſen werden durfte. Um was ich Sie diesmal beſonders beneide, iſt, daſs Sie mit
               einem Regiſſeur\pwindex{Reinhardt, Max 09.09.1873 – 30.10.1943@\textsc{Reinhardt, Max} (09.09.1873 – 30.10.1943), \emph{Theaterleiter, Regisseur, Schauspieler}|pwv} arbeiten
               konnten, der an Ihr Werk glaubte. Die \substVorne{}\textsuperscript{Mischung}{\allowbreak}\substDazwischen{}Atmosphäre\substHinten{} von Pflichttreue und künſtleriſcher Feindſeligkeit, in der \strikeout{mich} mein Werk\pwindex{Schnitzler, Arthur 15.05.1862 – 21.10.1931@\textsc{Schnitzler, Arthur} (15.05.1862 – 21.10.1931), \emph{Schriftsteller, Mediziner}!Ruf des Lebens. Schauspiel in drei Akten1906-02-20@\strich\emph{Der Ruf des Lebens. Schauspiel in drei Akten} {[}1906-02-20{]}|pwv} zum Bühnenleben erwuchs, hatte {\pb}etwas niederdrückendes.\pend
           \pstart
           Herzlichſt{\\[\baselineskip]}Ihr{\\[\baselineskip]}\spacefill\mbox{A.}\pend
           \leftskip=0em{}
         
         \endnumbering\mylabel{h}\end{ledgroupsized}  \newcommand{\dateiname}{L01587}\newcommand{\titel}{Arthur Schnitzler an Hugo von Hofmannsthal, 6. 3. 1906}\newcommand{\editorInnen}{Martin Anton Müller und Gerd-Hermann Susen}%% latex-leseansicht-abspann.tex
%% Abspann für die Leseansicht.
%% Der Schalter \ifkorrekturansicht ist bereits durch den Vorspann gesetzt.

%% latex-abspann.tex
%% Gemeinsamer Abspann für Korrekturansicht und Leseansicht.
%% Setzt den Schalter \ifkorrekturansicht voraus (gesetzt in den
%% einbindenden Dateien latex-korrekturansicht-abspann.tex bzw.
%% latex-leseansicht-abspann.tex).
%% ---------------------------------------------------------------

\normalsize

% Das esempio-Environment wird nur in der Leseansicht benötigt
\ifkorrekturansicht\else
\newenvironment{esempio}[3]%
{
    \vspace{1.5ex}
    \rlap{\underline{#1}}
    \par
    \setlength{\parindent}{0cm}
    \nopagebreak
    \leftskip=#2cm
    \rightskip=#3cm
}
{
    \par
}
\fi

\doendnotes{C}
\bigskip
\vfill

\clearpage

\footnotesize

\ifkorrekturansicht
  \lohead{\textsc{register}}
\fi

% theindex-Environment neu definieren ohne reledmac
\makeatletter
\renewenvironment{theindex}{%
  \ifkorrekturansicht
    \section*{\indexname}%
  \else
    \subsubsection*{Index der erwähnten Entitäten}%
  \fi
  \setlength{\parindent}{0pt}%
  \setlength{\parskip}{0pt plus 0.3pt}%
  \let\item\@idxitem
}{%
  \ifkorrekturansicht\clearpage\fi
}
\makeatother

\IfFileExists{\jobname-pw.ind}{\input{\jobname-pw.ind}}{}

% Quellenangabe nur in der Leseansicht
\ifkorrekturansicht\else
% Fallback-Definitionen, falls die .tex-Datei \titel etc. nicht gesetzt hat
\providecommand{\titel}{}
\providecommand{\editorInnen}{}
\providecommand{\dateiname}{\jobname}

\vspace{3cm}

\vfill

\footnotesize
\textsc{Quelle}: \titel. Herausgegeben von {\editorInnen}. In: \emph{Arthur Schnitzler: Briefwechsel mit Autorinnen und Autoren}.
 Digitale Edition, https://schnitzler-briefe.acdh.oeaw.ac.at/{\dateiname}.html (Stand \today)
\fi

\end{document}


      