%% latex-leseansicht-vorspann.tex
%% Vorspann für die Leseansicht.
%% Lädt die gemeinsame Datei latex-vorspann.tex mit nicht gesetztem Schalter.

\newif\ifkorrekturansicht
\korrekturansichtfalse

\input{../tex-inputs/latex-vorspann}


\section[ Arthur Schnitzler an Felix Salten, 14. 6. 1902]{L02975 Arthur Schnitzler an Felix Salten,  14. 6. 1902}
\nopagebreak\mylabel{L02975v}
\rehead{ }\normalsize\beginnumbering\briefempfaengerindex{Salten, Felix@\textsc{Salten, Felix}!zzzSchnitzler, Arthur@\emph{von Arthur Schnitzler}!1902-06-141@{14. 6. 1902}|(be}
\toendnotes[C]{\smallbreak\pagebreak[2]}
\correspDesc{Versand  durch Arthur Schnitzler am 14. 6. 1902 in Wien
\newline{}Erhalt  durch Felix Salten im Zeitraum [14. 6. 1902
                  – 17. 6. 1902?] in Kaltenleutgeben}\toendnotes[C]{\smallbreak}
\Standort{Wienbibliothek im Rathaus, ZPH 1681, 2.1.516.}
\physDesc{Brief, 1 Blatt, 2 Seiten, 308 Zeichen
\newline{}Handschrift: Bleistift, deutsche Kurrent
\newline{}Ordnung: mit Bleistift von unbekannter Hand nummeriert: »61« }\toendnotes[C]{\smallbreak}
\pstart
           \raggedleft{}{\pb}14. 6. 902.\pend
           \vspace{0.5em}
\pstart
           lieber, wie ein Herr Dr \textsc{Winterstein\pwindex{Winterstein, Richard 26.\,11.\,1862 Wien – 11.\,12.\,1931 ebd.@\textsc{Winterstein, Richard} (26.\,11.\,1862 Wien – 11.\,12.\,1931 ebd.), \emph{Rechtsanwalt}|pw}} dem Dr. \textsc{Schwarzkopf\pwindex{Schwarzkopf, Gustav 7.\,11.\,1853 Wien – 13.\,11.\,1939 ebd.@\textsc{Schwarzkopf, Gustav} (7.\,11.\,1853 Wien – 13.\,11.\,1939 ebd.), \emph{Schriftsteller}|pw}} erzählte, war \textsc{Karl Kraus\pwindex{Kraus, Karl 28.\,4.\,1874 Jičín – 12.\,6.\,1936 Wien@\textsc{Kraus, Karl} (28.\,4.\,1874 Jičín – 12.\,6.\,1936 Wien), \emph{Schriftsteller, Publizist, Schriftsteller}|pw}} von \label{K_L02975-1v}\edtext{\textsc{Martin Finder}}{\lemma{\textnormal{\emph{Martin Finder}}}\Cendnote{\textnormal{Unter diesem Pseudonym Saltens\pwindex{Salten, Felix 6.\,9.\,1869 Budapest – 8.\,10.\,1945 Zürich@\textsc{Salten, Felix} (6.\,9.\,1869 Budapest – 8.\,10.\,1945 Zürich), \emph{Schriftsteller, Journalist, Chefredakteur}|pwk} (siehe XXXX Auszeichnungsfehler: Dokument L02974 nicht gefunden) erschienen – mit der Unsicherheit der mit Kürzel publizierten
                  Veröffentlichung – fünf Texte in Band 31 von \emph{Die
                     Zeit. Wiener Wochenschrift}\pwindex{Zeit. Wiener Wochenschrift@\emph{Die Zeit. Wiener Wochenschrift}|pwk}: Martin Finder\pwindex{Salten, Felix 6.\,9.\,1869 Budapest – 8.\,10.\,1945 Zürich@\textsc{Salten, Felix} (6.\,9.\,1869 Budapest – 8.\,10.\,1945 Zürich), \emph{Schriftsteller, Journalist, Chefredakteur}|pwk}: \emph{Der Fall Baumberg}\pwindex{Salten, Felix 6.\,9.\,1869 Budapest – 8.\,10.\,1945 Zürich@\textsc{Salten, Felix} (6.\,9.\,1869 Budapest – 8.\,10.\,1945 Zürich), \emph{Schriftsteller, Journalist, Chefredakteur}!Fall Baumberg@\strich\emph{Der Fall Baumberg}|pwk}. In: Nr. 394, 19. 4. 1902, S. 42–43; Martin Finder\pwindex{Salten, Felix 6.\,9.\,1869 Budapest – 8.\,10.\,1945 Zürich@\textsc{Salten, Felix} (6.\,9.\,1869 Budapest – 8.\,10.\,1945 Zürich), \emph{Schriftsteller, Journalist, Chefredakteur}|pwk}: \emph{»Sonnwendtag.« (Drama in fünf Aufzügen von Karl Schönherr.
                        – Erste Aufführung im Burgtheater den 19. April 1902)}\pwindex{Salten, Felix 6.\,9.\,1869 Budapest – 8.\,10.\,1945 Zürich@\textsc{Salten, Felix} (6.\,9.\,1869 Budapest – 8.\,10.\,1945 Zürich), \emph{Schriftsteller, Journalist, Chefredakteur}!Sonnwendtag.« (Drama in fünf Aufzügen von Karl Schönherr. – Erste Aufführung im Burgtheater den 19. April 1902)@\strich\emph{»Sonnwendtag.« (Drama in fünf Aufzügen von Karl Schönherr. – Erste Aufführung im Burgtheater den 19. April 1902)}|pwk}. In: Nr. 395,
                        26. 4. 1902, S. 58–59; M. F.\pwindex{Salten, Felix 6.\,9.\,1869 Budapest – 8.\,10.\,1945 Zürich@\textsc{Salten, Felix} (6.\,9.\,1869 Budapest – 8.\,10.\,1945 Zürich), \emph{Schriftsteller, Journalist, Chefredakteur}|pwk}: \emph{[In dieser Woche wird man wieder einmal Bernhard Baumeister feiern]}\pwindex{Salten, Felix 6.\,9.\,1869 Budapest – 8.\,10.\,1945 Zürich@\textsc{Salten, Felix} (6.\,9.\,1869 Budapest – 8.\,10.\,1945 Zürich), \emph{Schriftsteller, Journalist, Chefredakteur}!In dieser Woche wird man wieder einmal Bernhard Baumeister feiern]@\strich\emph{[In dieser Woche wird man wieder einmal Bernhard Baumeister feiern]}|pwk}.
                     In: Nr. 396, 3. 5. 1902, S. 75; Martin Finder\pwindex{Salten, Felix 6.\,9.\,1869 Budapest – 8.\,10.\,1945 Zürich@\textsc{Salten, Felix} (6.\,9.\,1869 Budapest – 8.\,10.\,1945 Zürich), \emph{Schriftsteller, Journalist, Chefredakteur}|pwk}: \emph{Arthur Moeller-Bruck: Die moderne Literatur in Gruppen- und
                        Einzeldarstellungen. Band X. Das junge Wien. Verlag von Schuster und
                        Löffler, Berlin und Leipzig}\pwindex{Salten, Felix 6.\,9.\,1869 Budapest – 8.\,10.\,1945 Zürich@\textsc{Salten, Felix} (6.\,9.\,1869 Budapest – 8.\,10.\,1945 Zürich), \emph{Schriftsteller, Journalist, Chefredakteur}!Arthur Moeller-Bruck: Die moderne Literatur in Gruppen- und Einzeldarstellungen. Band X. Das junge Wien. Verlag von Schuster und Löffler, Berlin und Leipzig@\strich\emph{Arthur Moeller-Bruck: Die moderne Literatur in Gruppen- und Einzeldarstellungen. Band X. Das junge Wien. Verlag von Schuster und Löffler, Berlin und Leipzig}|pwk}. In: Nr. 399, 24. 5. 1902, S. 127; Martin Finder\pwindex{Salten, Felix 6.\,9.\,1869 Budapest – 8.\,10.\,1945 Zürich@\textsc{Salten, Felix} (6.\,9.\,1869 Budapest – 8.\,10.\,1945 Zürich), \emph{Schriftsteller, Journalist, Chefredakteur}|pwk}: \emph{Eine Variété-Komödie}\pwindex{Salten, Felix 6.\,9.\,1869 Budapest – 8.\,10.\,1945 Zürich@\textsc{Salten, Felix} (6.\,9.\,1869 Budapest – 8.\,10.\,1945 Zürich), \emph{Schriftsteller, Journalist, Chefredakteur}!Eine Variété-Komödie@\strich\emph{Eine Variété-Komödie}|pwk} In: Nr. 400, 31. 5. 1902, S. 138–139. Jahre später
                  verwendete Salten\pwindex{Salten, Felix 6.\,9.\,1869 Budapest – 8.\,10.\,1945 Zürich@\textsc{Salten, Felix} (6.\,9.\,1869 Budapest – 8.\,10.\,1945 Zürich), \emph{Schriftsteller, Journalist, Chefredakteur}|pwk} das Pseudonym gelegentlich
                  immer noch.}}}\label{K_L02975-1}{ }ſehr entzückt, den er offenbar wegen der \label{K_L02975-2v}\edtext{bekannten Stelle\pwindex{Salten, Felix 6.\,9.\,1869 Budapest – 8.\,10.\,1945 Zürich@\textsc{Salten, Felix} (6.\,9.\,1869 Budapest – 8.\,10.\,1945 Zürich), \emph{Schriftsteller, Journalist, Chefredakteur}!Fall Baumberg@\strich\emph{Der Fall Baumberg}|pwuv} für einen Chriſten, oder gar für einen
                  Antiſemiten}{\lemma{\textnormal{\emph{bekannten … Antisemiten}}}\Cendnote{\textnormal{Am naheliegendsten ist,
                  dass Karl Kraus\pwindex{Kraus, Karl 28.\,4.\,1874 Jičín – 12.\,6.\,1936 Wien@\textsc{Kraus, Karl} (28.\,4.\,1874 Jičín – 12.\,6.\,1936 Wien), \emph{Schriftsteller, Publizist, Schriftsteller}|pwk} die Besprechung \emph{Arthur Moeller-Bruck: Die moderne Literatur in
                     Gruppen- und Einzeldarstellungen}\pwindex{Salten, Felix 6.\,9.\,1869 Budapest – 8.\,10.\,1945 Zürich@\textsc{Salten, Felix} (6.\,9.\,1869 Budapest – 8.\,10.\,1945 Zürich), \emph{Schriftsteller, Journalist, Chefredakteur}!Arthur Moeller-Bruck: Die moderne Literatur in Gruppen- und Einzeldarstellungen. Band X. Das junge Wien. Verlag von Schuster und Löffler, Berlin und Leipzig@\strich\emph{Arthur Moeller-Bruck: Die moderne Literatur in Gruppen- und Einzeldarstellungen. Band X. Das junge Wien. Verlag von Schuster und Löffler, Berlin und Leipzig}|pwk} gefiel, da hier, vergleichbar mit seinen
                  Kritiken, der schlechten Sprache des besprochenen Texts\pwindex{\textcolor{red}{\textsuperscript{XXXX indx1}}!moderne Literatur in Gruppen- und Einzeldarstellungen. Band X. Das junge Wien@\strich\emph{Die moderne Literatur in Gruppen- und Einzeldarstellungen. Band X. Das junge Wien}|pwkv} viel Aufmerksamkeit geschenkt
                  wird. Die »bekannte[] Stelle« kann hingegen nicht mit Sicherheit
                  ermittelt werden. Eventuell bezog sich Schnitzler gleich auf den ersten Text (\emph{Der Fall Baumberg}\pwindex{Salten, Felix 6.\,9.\,1869 Budapest – 8.\,10.\,1945 Zürich@\textsc{Salten, Felix} (6.\,9.\,1869 Budapest – 8.\,10.\,1945 Zürich), \emph{Schriftsteller, Journalist, Chefredakteur}!Fall Baumberg@\strich\emph{Der Fall Baumberg}|pwk}) bzw. folgende Passage: »Von allen Erwerbsarten iſt
                        das Theater heute noch die beſte. Beſſer{ }ſogar als die Börſe, weil man ja
                        nur gewinnen, aber nichts verlieren kann, weshalb wir denn auch{ }ſo manchen
                        unter den Bühnendichtern{ }ſehen, der{ }ſonſt gewiſs nur als Börſeaner{ }ſich
                        fortgebracht hätte.\pwindex{Salten, Felix 6.\,9.\,1869 Budapest – 8.\,10.\,1945 Zürich@\textsc{Salten, Felix} (6.\,9.\,1869 Budapest – 8.\,10.\,1945 Zürich), \emph{Schriftsteller, Journalist, Chefredakteur}!Fall Baumberg@\strich\emph{Der Fall Baumberg}|pwv}« Schnitzler fand das Lob des
                  unwissenden Kraus\pwindex{Kraus, Karl 28.\,4.\,1874 Jičín – 12.\,6.\,1936 Wien@\textsc{Kraus, Karl} (28.\,4.\,1874 Jičín – 12.\,6.\,1936 Wien), \emph{Schriftsteller, Publizist, Schriftsteller}|pwk}’ wohl deshalb so
                     »amuſant«, weil Salten\pwindex{Salten, Felix 6.\,9.\,1869 Budapest – 8.\,10.\,1945 Zürich@\textsc{Salten, Felix} (6.\,9.\,1869 Budapest – 8.\,10.\,1945 Zürich), \emph{Schriftsteller, Journalist, Chefredakteur}|pwk} und
                     Kraus\pwindex{Kraus, Karl 28.\,4.\,1874 Jičín – 12.\,6.\,1936 Wien@\textsc{Kraus, Karl} (28.\,4.\,1874 Jičín – 12.\,6.\,1936 Wien), \emph{Schriftsteller, Publizist, Schriftsteller}|pwk} zerstritten waren und Salten\pwindex{Salten, Felix 6.\,9.\,1869 Budapest – 8.\,10.\,1945 Zürich@\textsc{Salten, Felix} (6.\,9.\,1869 Budapest – 8.\,10.\,1945 Zürich), \emph{Schriftsteller, Journalist, Chefredakteur}|pwk} in der \emph{Fackel}\pwindex{Fackel@\emph{Die Fackel}|pwk} häufig kritisiert wurde.}}}\label{K_L02975-2}{ }{\pb}hielt.\pend
           
\pstart
           Ich finde dieſe Sachlichkeit wider Willen amuſant genug, um{ }ſie Ihnen
               mitzutheilen {\\[\baselineskip]}Herzlich {\\[\baselineskip]}Ihr {\\[\baselineskip]}\spacefill\mbox{A.}\pend
           \leftskip=0em{}\selectlanguage{ngerman}\endnumbering\briefempfaengerindex{Salten, Felix@\textsc{Salten, Felix}!zzzSchnitzler, Arthur@\emph{von Arthur Schnitzler}!1902-06-141@{14. 6. 1902}|)be}\mylabel{L02975h}  \newcommand{\dateiname}{L02975}\newcommand{\titel}{Arthur Schnitzler an Felix Salten, 14. 6. 1902}\newcommand{\editorInnen}{Martin Anton Müller und Laura Untner}%% latex-leseansicht-abspann.tex
%% Abspann für die Leseansicht.
%% Der Schalter \ifkorrekturansicht ist bereits durch den Vorspann gesetzt.

%% latex-abspann.tex
%% Gemeinsamer Abspann für Korrekturansicht und Leseansicht.
%% Setzt den Schalter \ifkorrekturansicht voraus (gesetzt in den
%% einbindenden Dateien latex-korrekturansicht-abspann.tex bzw.
%% latex-leseansicht-abspann.tex).
%% ---------------------------------------------------------------

\normalsize

% Das esempio-Environment wird nur in der Leseansicht benötigt
\ifkorrekturansicht\else
\newenvironment{esempio}[3]%
{
    \vspace{1.5ex}
    \rlap{\underline{#1}}
    \par
    \setlength{\parindent}{0cm}
    \nopagebreak
    \leftskip=#2cm
    \rightskip=#3cm
}
{
    \par
}
\fi

\doendnotes{C}
\bigskip
\vfill

\clearpage

\footnotesize

\ifkorrekturansicht
  \lohead{\textsc{register}}
\fi

% theindex-Environment neu definieren ohne reledmac
\makeatletter
\renewenvironment{theindex}{%
  \ifkorrekturansicht
    \section*{\indexname}%
  \else
    \subsubsection*{Index der erwähnten Entitäten}%
  \fi
  \setlength{\parindent}{0pt}%
  \setlength{\parskip}{0pt plus 0.3pt}%
  \let\item\@idxitem
}{%
  \ifkorrekturansicht\clearpage\fi
}
\makeatother

\IfFileExists{\jobname-pw.ind}{\input{\jobname-pw.ind}}{}

% Quellenangabe nur in der Leseansicht
\ifkorrekturansicht\else
% Fallback-Definitionen, falls die .tex-Datei \titel etc. nicht gesetzt hat
\providecommand{\titel}{}
\providecommand{\editorInnen}{}
\providecommand{\dateiname}{\jobname}

\vspace{3cm}

\vfill

\footnotesize
\textsc{Quelle}: \titel. Herausgegeben von {\editorInnen}. In: \emph{Arthur Schnitzler: Briefwechsel mit Autorinnen und Autoren}.
 Digitale Edition, https://schnitzler-briefe.acdh.oeaw.ac.at/{\dateiname}.html (Stand \today)
\fi

\end{document}


