%% latex-korrekturansicht-vorspann.tex
%% Vorspann für die Korrekturansicht.
%% Lädt die gemeinsame Datei latex-vorspann.tex mit gesetztem Schalter.

\newif\ifkorrekturansicht
\korrekturansichttrue

\input{../tex-inputs/latex-vorspann}


\section[ Arthur Schnitzler an Felix Salten, 14. 6. 1902]{L02975 Arthur Schnitzler an Felix Salten, 14. 6. 1902}
\nopagebreak\mylabel{L02975v}
\rehead{ }\normalsize\beginnumbering\briefempfaengerindex{Salten, Felix@\textsc{Salten, Felix}!zzzSchnitzler, Arthur@\emph{von Arthur Schnitzler}!1902-06-141@{14. 6. 1902}|(be}
\toendnotes[C]{\smallbreak\pagebreak[2]}\Standort{Wienbibliothek im Rathaus, ZPH 1681, 2.1.516.}
\physDesc{Brief, 1 Blatt, 2 Seiten, 308 Zeichen
\newline{}Handschrift: Bleistift, deutsche Kurrent
\newline{}Ordnung: mit Bleistift von unbekannter Hand nummeriert: »61« }\toendnotes[C]{\smallbreak}
\pstart
           \raggedleft{}{\pb}14. 6. 902.\pend
           \vspace{0.5em}
\pstart
           lieber, wie ein Herr Dr \textsc{Winterstein\pwindex{Winterstein, Richard 26.11.1862 – 11.12.1931@\textsc{Winterstein, Richard} (26.11.1862 – 11.12.1931), \emph{Rechtsanwalt/Rechtsanwältin}|pw}} dem Dr. \textsc{Schwarzkopf\pwindex{Schwarzkopf, Gustav 07.11.1853 – 13.11.1939@\textsc{Schwarzkopf, Gustav} (07.11.1853 – 13.11.1939), \emph{Schriftsteller/Schriftstellerin}|pw}} erzählte, war \textsc{Karl Kraus\pwindex{Kraus, Karl 28.04.1874 – 12.06.1936@\textsc{Kraus, Karl} (28.04.1874 – 12.06.1936), \emph{Schriftsteller/Schriftstellerin, Publizist/Publizistin, Schriftsteller/Schriftstellerin}|pw}} von \label{K_L02975-1v}\edtext{\textsc{Martin Finder}}{\lemma{\textnormal{\emph{Martin Finder}}}\Cendnote{\textnormal{Unter diesem Pseudonym Saltens\pwindex{Salten, Felix 06.09.1869 – 08.10.1945@\textsc{Salten, Felix} (06.09.1869 – 08.10.1945), \emph{Schriftsteller/Schriftstellerin, Journalist/Journalistin, Chefredakteur/Chefredakteurin}|pwk} (siehe Arthur Schnitzler an Felix Salten, 27. 5. 1902) erschienen – mit der Unsicherheit der mit Kürzel publizierten
                  Veröffentlichung – fünf Texte in Band 31 von \emph{Die
                     Zeit. Wiener Wochenschrift}\pwindex{Zeit. Wiener Wochenschrift@\emph{Die Zeit. Wiener Wochenschrift}|pwk}: Martin Finder\pwindex{Salten, Felix 06.09.1869 – 08.10.1945@\textsc{Salten, Felix} (06.09.1869 – 08.10.1945), \emph{Schriftsteller/Schriftstellerin, Journalist/Journalistin, Chefredakteur/Chefredakteurin}|pwk}: \emph{Der Fall Baumberg}\pwindex{Fall Baumberg@\emph{Der Fall Baumberg}|pwk}. In: Nr. 394, 19. 4. 1902, S. 42–43; Martin Finder\pwindex{Salten, Felix 06.09.1869 – 08.10.1945@\textsc{Salten, Felix} (06.09.1869 – 08.10.1945), \emph{Schriftsteller/Schriftstellerin, Journalist/Journalistin, Chefredakteur/Chefredakteurin}|pwk}: \emph{»Sonnwendtag.« (Drama in fünf Aufzügen von Karl Schönherr.
                        – Erste Aufführung im Burgtheater den 19. April 1902)}\pwindex{Sonnwendtag.« (Drama in fuenf Aufzuegen von Karl Schoenherr. – Erste Auffuehrung im Burgtheater den 19. April 1902)@\emph{»Sonnwendtag.« (Drama in fünf Aufzügen von Karl Schönherr. – Erste Aufführung im Burgtheater den 19. April 1902)}|pwk}. In: Nr. 395,
                        26. 4. 1902, S. 58–59; M. F.\pwindex{Salten, Felix 06.09.1869 – 08.10.1945@\textsc{Salten, Felix} (06.09.1869 – 08.10.1945), \emph{Schriftsteller/Schriftstellerin, Journalist/Journalistin, Chefredakteur/Chefredakteurin}|pwk}: \emph{[In dieser Woche wird man wieder einmal Bernhard Baumeister feiern]}\pwindex{In dieser Woche wird man wieder einmal Bernhard Baumeister feiern]@\emph{[In dieser Woche wird man wieder einmal Bernhard Baumeister feiern]}|pwk}.
                     In: Nr. 396, 3. 5. 1902, S. 75; Martin Finder\pwindex{Salten, Felix 06.09.1869 – 08.10.1945@\textsc{Salten, Felix} (06.09.1869 – 08.10.1945), \emph{Schriftsteller/Schriftstellerin, Journalist/Journalistin, Chefredakteur/Chefredakteurin}|pwk}: \emph{Arthur Moeller-Bruck: Die moderne Literatur in Gruppen- und
                        Einzeldarstellungen. Band X. Das junge Wien. Verlag von Schuster und
                        Löffler, Berlin und Leipzig}\pwindex{Arthur Moeller-Bruck: Die moderne Literatur in Gruppen- und Einzeldarstellungen. Band X. Das junge Wien. Verlag von Schuster und Loeffler, Berlin und Leipzig@\emph{Arthur Moeller-Bruck: Die moderne Literatur in Gruppen- und Einzeldarstellungen. Band X. Das junge Wien. Verlag von Schuster und Löffler, Berlin und Leipzig}|pwk}. In: Nr. 399, 24. 5. 1902, S. 127; Martin Finder\pwindex{Salten, Felix 06.09.1869 – 08.10.1945@\textsc{Salten, Felix} (06.09.1869 – 08.10.1945), \emph{Schriftsteller/Schriftstellerin, Journalist/Journalistin, Chefredakteur/Chefredakteurin}|pwk}: \emph{Eine Variété-Komödie}\pwindex{Eine Variete-Komoedie@\emph{Eine Variété-Komödie}|pwk} In: Nr. 400, 31. 5. 1902, S. 138–139. Jahre später
                  verwendete Salten\pwindex{Salten, Felix 06.09.1869 – 08.10.1945@\textsc{Salten, Felix} (06.09.1869 – 08.10.1945), \emph{Schriftsteller/Schriftstellerin, Journalist/Journalistin, Chefredakteur/Chefredakteurin}|pwk} das Pseudonym gelegentlich
                  immer noch.}}}\label{K_L02975-1} ſehr entzückt, den er offenbar wegen der \label{K_L02975-2v}\edtext{bekannten Stelle\pwindex{Fall Baumberg@\emph{Der Fall Baumberg}|pwuv} für einen Chriſten, oder gar für einen
                  Antiſemiten}{\lemma{\textnormal{\emph{bekannten … Antiſemiten}}}\Cendnote{\textnormal{Am naheliegendsten ist,
                  dass Karl Kraus\pwindex{Kraus, Karl 28.04.1874 – 12.06.1936@\textsc{Kraus, Karl} (28.04.1874 – 12.06.1936), \emph{Schriftsteller/Schriftstellerin, Publizist/Publizistin, Schriftsteller/Schriftstellerin}|pwk} die Besprechung \emph{Arthur Moeller-Bruck: Die moderne Literatur in
                     Gruppen- und Einzeldarstellungen}\pwindex{Arthur Moeller-Bruck: Die moderne Literatur in Gruppen- und Einzeldarstellungen. Band X. Das junge Wien. Verlag von Schuster und Loeffler, Berlin und Leipzig@\emph{Arthur Moeller-Bruck: Die moderne Literatur in Gruppen- und Einzeldarstellungen. Band X. Das junge Wien. Verlag von Schuster und Löffler, Berlin und Leipzig}|pwk} gefiel, da hier, vergleichbar mit seinen
                  Kritiken, der schlechten Sprache des besprochenen Texts\pwindex{moderne Literatur in Gruppen- und Einzeldarstellungen. Band X. Das junge Wien@\emph{Die moderne Literatur in Gruppen- und Einzeldarstellungen. Band X. Das junge Wien}|pwkv} viel Aufmerksamkeit geschenkt
                  wird. Die »bekannte[] Stelle« kann hingegen nicht mit Sicherheit
                  ermittelt werden. Eventuell bezog sich Schnitzler gleich auf den ersten Text (\emph{Der Fall Baumberg}\pwindex{Fall Baumberg@\emph{Der Fall Baumberg}|pwk}) bzw. folgende Passage: »Von allen Erwerbsarten iſt
                        das Theater heute noch die beſte. Beſſer ſogar als die Börſe, weil man ja
                        nur gewinnen, aber nichts verlieren kann, weshalb wir denn auch ſo manchen
                        unter den Bühnendichtern ſehen, der ſonſt gewiſs nur als Börſeaner ſich
                        fortgebracht hätte.\pwindex{Fall Baumberg@\emph{Der Fall Baumberg}|pwv}« Schnitzler fand das Lob des
                  unwissenden Kraus\pwindex{Kraus, Karl 28.04.1874 – 12.06.1936@\textsc{Kraus, Karl} (28.04.1874 – 12.06.1936), \emph{Schriftsteller/Schriftstellerin, Publizist/Publizistin, Schriftsteller/Schriftstellerin}|pwk}’ wohl deshalb so
                     »amuſant«, weil Salten\pwindex{Salten, Felix 06.09.1869 – 08.10.1945@\textsc{Salten, Felix} (06.09.1869 – 08.10.1945), \emph{Schriftsteller/Schriftstellerin, Journalist/Journalistin, Chefredakteur/Chefredakteurin}|pwk} und
                     Kraus\pwindex{Kraus, Karl 28.04.1874 – 12.06.1936@\textsc{Kraus, Karl} (28.04.1874 – 12.06.1936), \emph{Schriftsteller/Schriftstellerin, Publizist/Publizistin, Schriftsteller/Schriftstellerin}|pwk} zerstritten waren und Salten\pwindex{Salten, Felix 06.09.1869 – 08.10.1945@\textsc{Salten, Felix} (06.09.1869 – 08.10.1945), \emph{Schriftsteller/Schriftstellerin, Journalist/Journalistin, Chefredakteur/Chefredakteurin}|pwk} in der \emph{Fackel}\pwindex{Fackel@\emph{Die Fackel}|pwk} häufig kritisiert wurde.}}}\label{K_L02975-2}{ }{\pb}hielt.\pend
           
\pstart
           Ich finde dieſe Sachlichkeit wider Willen amuſant genug, um ſie Ihnen
               mitzutheilen {\\[\baselineskip]}Herzlich {\\[\baselineskip]}Ihr {\\[\baselineskip]}\spacefill\mbox{A.}\pend
           \leftskip=0em{}\selectlanguage{ngerman}\endnumbering\briefempfaengerindex{Salten, Felix@\textsc{Salten, Felix}!zzzSchnitzler, Arthur@\emph{von Arthur Schnitzler}!1902-06-141@{14. 6. 1902}|)be}\mylabel{L02975h}  \normalsize

\doendnotes{C}
\bigskip
\vfill

\clearpage

\footnotesize

\lohead{\textsc{register}}

% Definiere theindex-Environment komplett neu ohne reledmac
\makeatletter
\renewenvironment{theindex}{%
  \section*{\indexname}%
  \setlength{\parindent}{0pt}%
  \setlength{\parskip}{0pt plus 0.3pt}%
  \let\item\@idxitem
}{%
  \clearpage
}
\makeatother

\IfFileExists{\jobname-pw.ind}{\input{\jobname-pw.ind}}{}

\end{document}

      