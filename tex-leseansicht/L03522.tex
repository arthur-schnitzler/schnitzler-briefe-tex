%% latex-leseansicht-vorspann.tex
%% Vorspann für die Leseansicht.
%% Lädt die gemeinsame Datei latex-vorspann.tex mit nicht gesetztem Schalter.

\newif\ifkorrekturansicht
\korrekturansichtfalse

\input{../tex-inputs/latex-vorspann}


\section[ Felix Salten an Arthur Schnitzler, 15. 2. 1909]{L03522 Felix Salten an Arthur Schnitzler,  15. 2. 1909}
\nopagebreak\mylabel{L03522v}
\rehead{ }\normalsize\beginnumbering\briefempfaengerindex{Schnitzler, Arthur@\textsc{Schnitzler, Arthur}!zzzSalten, Felix@\emph{von Felix Salten}!1909-02-152@{15. 2. 1909}|(be}
\toendnotes[C]{\smallbreak\pagebreak[2]}
\correspDesc{Versand  durch Felix Salten am 15. 2. 1909 in Semmering
\newline{}Erhalt  durch Arthur Schnitzler im Zeitraum [16. 2. 1909
                  – 20. 2. 1909?] in Wien}\toendnotes[C]{\smallbreak}
\Standort{CUL, Schnitzler, B 89, B 1.}
\physDesc{Brief, 1 Blatt, 1 Seite, 461 Zeichen
\newline{}Handschrift: schwarze Tinte, lateinische Kurrent
\newline{}Schnitzler: mit Bleistift Vermerk: »\textsc{Salten}« 
\newline{}Ordnung: mit Bleistift von unbekannter Hand nummeriert: »248« }\toendnotes[C]{\smallbreak}
\pstart
           {\pb}\textcolor{gray}{\textbf{Südbahn-Hôtel\oindex{Südbahnhotel [Semmering]@\textbf{Südbahnhotel [Semmering]}, \emph{Hotel}|pw}}}\pend
           
\pstart
           \textcolor{gray}{\textbf{Semmering\oindex{Semmering@\textbf{Semmering}, \emph{Verwaltungsgebiet}|pw}}}\pend
           
\pstart
           \textcolor{gray}{\textbf{Austria\oindex{Österreich@\textbf{Österreich}|pw}}}\pend
           
\pstart
           \textcolor{gray}{\textbf{\textsc{TELEGRAMME:}}}\pend
           
\pstart
           \textcolor{gray}{\textbf{\textsc{SÜDBAHNHÔTEL SEMMERING\oindex{Südbahnhotel [Semmering]@\textbf{Südbahnhotel [Semmering]}, \emph{Hotel}|pw}}}}\pend
           
\pstart
           \textcolor{gray}{\textbf{\textsc{TELEPHON:}}}\pend
           
\pstart
           \textcolor{gray}{\textbf{\textsc{HÔTEL {\dotsfour} NR. 5.}}}\pend
           
\pstart
           \textcolor{gray}{\textbf{\textsc{DEPENDANCE . NR. 6.}}}\hfill 15. II. 09\pend
           \vspace{0.5em}
\pstart
           Lieber, wir wollen noch etwa acht bis zehn Tage bleiben, falls das
               Wetter weiter so herrlich ist und sonst nichts dazwischen kommt. Wenn ich Samstag ins Theater muß, fahre ich Sonntag früh wieder herauf. Wir wünschen sehr, dass Frau
                  \label{K_L03522-1v}\edtext{Olga\pwindex{Schnitzler, Olga 17.\,1.\,1882 Wien – 13.\,1.\,1970 Lugano@\textsc{Schnitzler, Olga} (17.\,1.\,1882 Wien – 13.\,1.\,1970 Lugano), \emph{Schauspielerin, Sängerin}|pw} recht bald wieder wol}{\lemma{\textnormal{\emph{Olga … wol}}}\Cendnote{\textnormal{Siehe A. S.: \emph{Tagebuch}, 14. 2. 1909.
               }}}\label{K_L03522-1} ist, und dass Sie Beide\pwindex{Schnitzler, Olga 17.\,1.\,1882 Wien – 13.\,1.\,1970 Lugano@\textsc{Schnitzler, Olga} (17.\,1.\,1882 Wien – 13.\,1.\,1970 Lugano), \emph{Schauspielerin, Sängerin}|pwv} noch \label{K_L03522-2v}\edtext{\uline{vor} dem Sonntag hier
                  sein}{\lemma{\textnormal{\emph{vor … sein}}}\Cendnote{\textnormal{Dazu kam es nicht.}}}\label{K_L03522-2} können.
                  Gestern waren noch Sportspiele da (übrigens sehr
                  schön){[},{]} dafür wird’s jetzt still. Alles Gute Ihrer Frau\pwindex{Schnitzler, Olga 17.\,1.\,1882 Wien – 13.\,1.\,1970 Lugano@\textsc{Schnitzler, Olga} (17.\,1.\,1882 Wien – 13.\,1.\,1970 Lugano), \emph{Schauspielerin, Sängerin}|pwv} und herzliche Grüße von
                  uns\pwindex{Salten, Ottilie 7.\,3.\,1868 Prag – 22.\,6.\,1942 Zürich@\textsc{Salten, Ottilie} (7.\,3.\,1868 Prag – 22.\,6.\,1942 Zürich), \emph{Schauspielerin}|pwv} zu Ihnen\pend
           \pstart Ihr \spacefill\mbox{Salten}\pend{}\selectlanguage{ngerman}\endnumbering\briefempfaengerindex{Schnitzler, Arthur@\textsc{Schnitzler, Arthur}!zzzSalten, Felix@\emph{von Felix Salten}!1909-02-152@{15. 2. 1909}|)be}\mylabel{L03522h}  \newcommand{\dateiname}{L03522}\newcommand{\titel}{Felix Salten an Arthur Schnitzler, 15. 2. 1909}\newcommand{\editorInnen}{Martin Anton Müller und Laura Untner}%% latex-leseansicht-abspann.tex
%% Abspann für die Leseansicht.
%% Der Schalter \ifkorrekturansicht ist bereits durch den Vorspann gesetzt.

%% latex-abspann.tex
%% Gemeinsamer Abspann für Korrekturansicht und Leseansicht.
%% Setzt den Schalter \ifkorrekturansicht voraus (gesetzt in den
%% einbindenden Dateien latex-korrekturansicht-abspann.tex bzw.
%% latex-leseansicht-abspann.tex).
%% ---------------------------------------------------------------

\normalsize

% Das esempio-Environment wird nur in der Leseansicht benötigt
\ifkorrekturansicht\else
\newenvironment{esempio}[3]%
{
    \vspace{1.5ex}
    \rlap{\underline{#1}}
    \par
    \setlength{\parindent}{0cm}
    \nopagebreak
    \leftskip=#2cm
    \rightskip=#3cm
}
{
    \par
}
\fi

\doendnotes{C}
\bigskip
\vfill

\clearpage

\footnotesize

\ifkorrekturansicht
  \lohead{\textsc{register}}
\fi

% theindex-Environment neu definieren ohne reledmac
\makeatletter
\renewenvironment{theindex}{%
  \ifkorrekturansicht
    \section*{\indexname}%
  \else
    \subsubsection*{Index der erwähnten Entitäten}%
  \fi
  \setlength{\parindent}{0pt}%
  \setlength{\parskip}{0pt plus 0.3pt}%
  \let\item\@idxitem
}{%
  \ifkorrekturansicht\clearpage\fi
}
\makeatother

\IfFileExists{\jobname-pw.ind}{\input{\jobname-pw.ind}}{}

% Quellenangabe nur in der Leseansicht
\ifkorrekturansicht\else
% Fallback-Definitionen, falls die .tex-Datei \titel etc. nicht gesetzt hat
\providecommand{\titel}{}
\providecommand{\editorInnen}{}
\providecommand{\dateiname}{\jobname}

\vspace{3cm}

\vfill

\footnotesize
\textsc{Quelle}: \titel. Herausgegeben von {\editorInnen}. In: \emph{Arthur Schnitzler: Briefwechsel mit Autorinnen und Autoren}.
 Digitale Edition, https://schnitzler-briefe.acdh.oeaw.ac.at/{\dateiname}.html (Stand \today)
\fi

\end{document}


