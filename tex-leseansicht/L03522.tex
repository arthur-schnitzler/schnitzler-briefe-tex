%% latex-korrekturansicht-vorspann.tex
%% Vorspann für die Korrekturansicht.
%% Lädt die gemeinsame Datei latex-vorspann.tex mit gesetztem Schalter.

\newif\ifkorrekturansicht
\korrekturansichttrue

\input{../tex-inputs/latex-vorspann}


\section[ Felix Salten an Arthur Schnitzler, 15. 2. 1909]{L03522 Felix Salten an Arthur Schnitzler, 15. 2. 1909}
\nopagebreak\mylabel{L03522v}
\rehead{ }\normalsize\beginnumbering\briefempfaengerindex{Schnitzler, Arthur@\textsc{Schnitzler, Arthur}!zzzSalten, Felix@\emph{von Felix Salten}!1909-02-151@{15. 2. 1909}|(be}
\toendnotes[C]{\smallbreak\pagebreak[2]}\Standort{CUL, Schnitzler, B 89, B 1.}
\physDesc{Brief, 1 Blatt, 1 Seite, 461 Zeichen
\newline{}Handschrift: schwarze Tinte, lateinische Kurrent
\newline{}Schnitzler: mit Bleistift Vermerk: »\textsc{Salten}« 
\newline{}Ordnung: mit Bleistift von unbekannter Hand nummeriert: »248« }\toendnotes[C]{\smallbreak}
\pstart
           {\pb}\textcolor{gray}{\textbf{Südbahn-Hôtel\oindex{Suedbahnhotel [Semmering]@\textbf{Südbahnhotel [Semmering]}, \emph{Hotel (K.HTL)}|pw}}}\pend
           
\pstart
           \textcolor{gray}{\textbf{Semmering\oindex{Semmering@\textbf{Semmering}, \emph{A.ADM3}|pw}}}\pend
           
\pstart
           \textcolor{gray}{\textbf{Austria\oindex{Oesterreich@\textbf{Österreich}, \emph{A.PCLI}|pw}}}\pend
           
\pstart
           \textcolor{gray}{\textbf{\textsc{TELEGRAMME:}}}\pend
           
\pstart
           \textcolor{gray}{\textbf{\textsc{SÜDBAHNHÔTEL SEMMERING\oindex{Suedbahnhotel [Semmering]@\textbf{Südbahnhotel [Semmering]}, \emph{Hotel (K.HTL)}|pw}}}}\pend
           
\pstart
           \textcolor{gray}{\textbf{\textsc{TELEPHON:}}}\pend
           
\pstart
           \textcolor{gray}{\textbf{\textsc{HÔTEL {\dotsfour} NR. 5.}}}\pend
           
\pstart
           \textcolor{gray}{\textbf{\textsc{DEPENDANCE . NR. 6.}}}\hfill 15. II. 09\pend
           \vspace{0.5em}
\pstart
           Lieber, wir wollen noch etwa acht bis zehn Tage bleiben, falls das
               Wetter weiter so herrlich ist und sonst nichts dazwischen kommt. Wenn ich Samstag ins Theater muß, fahre ich Sonntag früh wieder herauf. Wir wünschen sehr, dass Frau
                  \label{K_L03522-1v}\edtext{Olga\pwindex{Schnitzler, Olga 17.01.1882 – 13.01.1970@\textsc{Schnitzler, Olga} (17.01.1882 – 13.01.1970), \emph{Schauspieler/Schauspielerin, Sänger/Sängerin}|pw} recht bald wieder wol}{\lemma{\textnormal{\emph{Olga … wol}}}\Cendnote{\textnormal{Siehe A. S.: \emph{Tagebuch}, 14. 2. 1909.
               }}}\label{K_L03522-1} ist, und dass Sie Beide\pwindex{Schnitzler, Olga 17.01.1882 – 13.01.1970@\textsc{Schnitzler, Olga} (17.01.1882 – 13.01.1970), \emph{Schauspieler/Schauspielerin, Sänger/Sängerin}|pwv} noch \label{K_L03522-2v}\edtext{\uline{vor} dem Sonntag hier
                  sein}{\lemma{\textnormal{\emph{vor … sein}}}\Cendnote{\textnormal{Dazu kam es nicht.}}}\label{K_L03522-2} können.
                  Gestern waren noch Sportspiele da (übrigens sehr
                  schön){[},{]} dafür wird’s jetzt still. Alles Gute Ihrer Frau\pwindex{Schnitzler, Olga 17.01.1882 – 13.01.1970@\textsc{Schnitzler, Olga} (17.01.1882 – 13.01.1970), \emph{Schauspieler/Schauspielerin, Sänger/Sängerin}|pwv} und herzliche Grüße von
                  uns\pwindex{Salten, Ottilie 07.03.1868 – 22.06.1942@\textsc{Salten, Ottilie} (07.03.1868 – 22.06.1942), \emph{Schauspieler/Schauspielerin}|pwv} zu Ihnen\pend
           \pstart Ihr \spacefill\mbox{Salten}\pend{}\selectlanguage{ngerman}\endnumbering\briefempfaengerindex{Schnitzler, Arthur@\textsc{Schnitzler, Arthur}!zzzSalten, Felix@\emph{von Felix Salten}!1909-02-151@{15. 2. 1909}|)be}\mylabel{L03522h}  \normalsize

\doendnotes{C}
\bigskip
\vfill

\clearpage

\footnotesize

\lohead{\textsc{register}}

% Definiere theindex-Environment komplett neu ohne reledmac
\makeatletter
\renewenvironment{theindex}{%
  \section*{\indexname}%
  \setlength{\parindent}{0pt}%
  \setlength{\parskip}{0pt plus 0.3pt}%
  \let\item\@idxitem
}{%
  \clearpage
}
\makeatother

\IfFileExists{\jobname-pw.ind}{\input{\jobname-pw.ind}}{}

\end{document}

      