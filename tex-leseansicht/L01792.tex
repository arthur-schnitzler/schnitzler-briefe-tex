%% latex-leseansicht-vorspann.tex
%% Vorspann für die Leseansicht.
%% Lädt die gemeinsame Datei latex-vorspann.tex mit nicht gesetztem Schalter.

\newif\ifkorrekturansicht
\korrekturansichtfalse

\input{../tex-inputs/latex-vorspann}


         \renewcommand{\erwaehnteOrte}{Orte: Ottakringerstraße, Wien}
         \renewcommand{\erwaehnteWerke}{Werke: Seltene Gäste, Tubutsch}
               \section[Albert Ehrenstein an Arthur Schnitzler, 10. 10. 1908]{ Albert Ehrenstein an Arthur Schnitzler, 10. 10. 1908}\nopagebreak\mylabel{v}\rehead{ }\begin{ledgroupsized}[t]{13cm}\normalsize\beginnumbering \toendnotes[C]{\smallbreak\pagebreak[2]} \Standort{CUL, Schnitzler, B 30.}
\physDesc{Brief, 1 Blatt, 4 Seiten
\newline{}Handschrift: schwarze Tinte, deutsche Kurrent
\newline{}Schnitzler: mit Bleistift Vermerk: »\textsc{Ehrenstein}« }\Standort{Jerusalem, The National Library of Israel, ARC. Ms. Var. 306 1 117.}
\physDesc{Briefentwurf, 1 Blatt, 4 Seiten
\newline{}Handschrift: schwarze Tinte, deutsche Kurrent}\Standort{Jerusalem, The National Library of Israel, ARC. Ms. Var. 306 1 117.}
\physDesc{Briefentwurf, 1 Blatt, 4 Seiten
\newline{}Handschrift: schwarze Tinte, deutsche Kurrent}\Standort{Jerusalem, The National Library of Israel, ARC. Ms. Var. 306 1 117.}
\physDesc{Briefentwurf, 1 Blatt, 2 Seiten
\newline{}Handschrift: schwarze Tinte, deutsche Kurrent}\Standort{Jerusalem, The National Library of Israel, ARC. Ms. Var. 306 1 117.}
\physDesc{Briefentwurf, 1 Blatt, 2 Seiten
\newline{}Handschrift: schwarze Tinte, deutsche Kurrent}\buchAbdrucke{\weitereDrucke{Albert Ehrenstein: \emph{Briefe}. Hg. Hanni Mittelmann. München: \emph{Boer} 1989, S. 22–23 (Werke, 1).} }\toendnotes[C]{\smallbreak}\pstart
           {\pb}Wien, XVI. \textsc{Ottakringerstr 114}\oindex{Ottakringerstrasse@\textbf{Ottakringerstraße}|pw}.\hfill \textsc{10. Oktober 1908}.\pend
           \pstart\center{}\textsc{Sehr geehrter Herr Doktor!}\pend\pstart
           Verhindert durch \label{K_L01792_1v}\edtext{Handarbeiten}{\lemma{\textnormal{\emph{Handarbeiten}}}\Cendnote{\textnormal{Ehrenstein\pwindex{Ehrenstein, Albert 23.12.1886 – 08.04.1950@\textsc{Ehrenstein, Albert} (23.12.1886 – 08.04.1950), \emph{Schriftsteller}|pwk} hatte 1905 ein
                        Universitätsstudium der Geschichte, Kunstgeschichte und Geographie
                        aufgenommen.}}}\label{K_L01792_1h} geographiſch-geſchichtlichen Charakters, noch mehr aber
                    durch das Nochnichtvorhandenſein eigener Artefakte, die mir als halbwegs
                    annehmbare Legitimation für eine abermalige Beläſtigung hätten dienen können,
                    kam ich im Januar nicht Ihrer mich erfreuenden Aufforderung nach,
                    bei Ihnen ſehr geehrter Herr Doktor, einmal vorzuſprechen. Die Behelligung durch
                    Studien hat nicht aufgehört, Zeitmangel alſo könnte manche der in den
                    beiliegenden Skizzen zutagetretenden Flüchtigkeiten, das Fehlen intimerer
                    Feilung erklären, {\pb}abgeſehen von
                    meinem Widerwillen dagegen, Kleinigkeiten ſelber an das gedulderſchöpfende,
                    zeitraubende Überſchreiben vielleicht ausſichtsloſer Erzeugniſſe zu schreiten.
                    Leider ſind die genannten Unterlaſſungen das Wenigste. Kein der Produktion
                    gewidmeter Tag iſt ohne hunderterlei teils ungewollte, teils mehr als
                    beabſichtigte Störungen häuslicherſeits dahingegangen. Der ruhige Fluß der
                    Darſtellungen, mit dem endlich beſchenkt worden zu ſein ich mich ſchon freute,
                    bald gehemmt, unterbrochen machte einer mehr stoßweiſen, abgeriſſenen Art der
                    der Erzählung Platz. Notwendig ſind die vorliegenden Darbietungen,
                    ſobald Schwung {\pb}und Stimmung von
                    außen verſcheucht worden, in einem dem Laſter ſozuſagen jeden Augenblick
                    freigebendem Stil geſchrieben, was beſonders bei der letzten Novellette\pwindex{Ehrenstein, Albert 23.12.1886 – 08.04.1950@\textsc{Ehrenstein, Albert} (23.12.1886 – 08.04.1950), \emph{Schriftsteller}!Tubutsch1911@\strich\emph{Tubutsch} {[}1911{]}|pwv} ermüden muß, welche an ſich
                    Langeweile und Enttäuſchung, einen an den Auslagen der Geſchäfte und Leute
                    entlang lebenden Menſchen zu ſchildern unternimmt. Wenn ich mich trotz alledem
                    erkühne, an Sie, ſehr geehrter Herr Doktor, mit dem wenig gerechtfertigten
                    Anſinnen heranzutreten, die übrigens teilweiſe untereinander in Konnex und
                    Abfolge ſtehenden Werkchen (einzeln) zu beurteilen die Güte zu haben, {\pb}die möglicherweiſe wertvolle \label{K_L01792_2v}\edtext{Titelnovelle\pwindex{Ehrenstein, Albert 23.12.1886 – 08.04.1950@\textsc{Ehrenstein, Albert} (23.12.1886 – 08.04.1950), \emph{Schriftsteller}!Seltene Gaeste1991@\strich\emph{Seltene Gäste} {[}1991{]}|pwuv}}{\lemma{\textnormal{\emph{Titelnovelle}}}\Cendnote{\textnormal{Es dürfte sich, was durch den
                        Hinweis auf den Umfang angedeutet wird, um das 87 Seiten umfassende
                        Manuskript von \emph{Seltene Gäste}\pwindex{Ehrenstein, Albert 23.12.1886 – 08.04.1950@\textsc{Ehrenstein, Albert} (23.12.1886 – 08.04.1950), \emph{Schriftsteller}!Seltene Gaeste1991@\strich\emph{Seltene Gäste} {[}1991{]}|pwk} handeln, das
                        in dieser Form erst 1991 veröffentlicht wurde.}}}\label{K_L01792_2h}, falls es
                    irgend angeht, auf einmal leſen zu wollen – ſo bitte ich dieſe nicht
                    anſpruchsvollen Zumutungen nicht zu mißdeuten. Nichts liegt mir ferner als
                    Prätention, nichts wünſche ich ſoſehr als Rat und Hilfe. In der Hoffnung,
                    diesmal, wenn verdient, realerer Erfolge teilhaftig zu werden, verbleibe
                        ich\hspace*{1.5em}Hochachtungsvoll ergebenſt{\\}Ihr Sie,
                    ſehr geehrter Herr Doktor, verehrender\pend
           \pstart \spacefill\mbox{Albert Ehrenstein.}\pend{}
         
         \endnumbering\mylabel{h}\end{ledgroupsized}  \newcommand{\dateiname}{L01792}\newcommand{\titel}{Albert Ehrenstein an Arthur Schnitzler, 10. 10. 1908}\newcommand{\editorInnen}{Martin Anton Müller und Gerd-Hermann Susen}%% latex-leseansicht-abspann.tex
%% Abspann für die Leseansicht.
%% Der Schalter \ifkorrekturansicht ist bereits durch den Vorspann gesetzt.

%% latex-abspann.tex
%% Gemeinsamer Abspann für Korrekturansicht und Leseansicht.
%% Setzt den Schalter \ifkorrekturansicht voraus (gesetzt in den
%% einbindenden Dateien latex-korrekturansicht-abspann.tex bzw.
%% latex-leseansicht-abspann.tex).
%% ---------------------------------------------------------------

\normalsize

% Das esempio-Environment wird nur in der Leseansicht benötigt
\ifkorrekturansicht\else
\newenvironment{esempio}[3]%
{
    \vspace{1.5ex}
    \rlap{\underline{#1}}
    \par
    \setlength{\parindent}{0cm}
    \nopagebreak
    \leftskip=#2cm
    \rightskip=#3cm
}
{
    \par
}
\fi

\doendnotes{C}
\bigskip
\vfill

\clearpage

\footnotesize

\ifkorrekturansicht
  \lohead{\textsc{register}}
\fi

% theindex-Environment neu definieren ohne reledmac
\makeatletter
\renewenvironment{theindex}{%
  \ifkorrekturansicht
    \section*{\indexname}%
  \else
    \subsubsection*{Index der erwähnten Entitäten}%
  \fi
  \setlength{\parindent}{0pt}%
  \setlength{\parskip}{0pt plus 0.3pt}%
  \let\item\@idxitem
}{%
  \ifkorrekturansicht\clearpage\fi
}
\makeatother

\IfFileExists{\jobname-pw.ind}{\input{\jobname-pw.ind}}{}

% Quellenangabe nur in der Leseansicht
\ifkorrekturansicht\else
% Fallback-Definitionen, falls die .tex-Datei \titel etc. nicht gesetzt hat
\providecommand{\titel}{}
\providecommand{\editorInnen}{}
\providecommand{\dateiname}{\jobname}

\vspace{3cm}

\vfill

\footnotesize
\textsc{Quelle}: \titel. Herausgegeben von {\editorInnen}. In: \emph{Arthur Schnitzler: Briefwechsel mit Autorinnen und Autoren}.
 Digitale Edition, https://schnitzler-briefe.acdh.oeaw.ac.at/{\dateiname}.html (Stand \today)
\fi

\end{document}


      