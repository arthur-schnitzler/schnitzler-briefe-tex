%% latex-korrekturansicht-vorspann.tex
%% Vorspann für die Korrekturansicht.
%% Lädt die gemeinsame Datei latex-vorspann.tex mit gesetztem Schalter.

\newif\ifkorrekturansicht
\korrekturansichttrue

\input{../tex-inputs/latex-vorspann}


\section[Albert Ehrenstein an Arthur Schnitzler, 10. 10. 1908]{L01792 Albert Ehrenstein an Arthur Schnitzler, 10. 10. 1908}
\nopagebreak\mylabel{L01792v}
\rehead{ }\normalsize\beginnumbering\briefempfaengerindex{Schnitzler, Arthur@\textsc{Schnitzler, Arthur}!zzzEhrenstein, Albert@\emph{von Albert Ehrenstein}!1908-10-101@{10. 10. 1908}|(be}
\toendnotes[C]{\smallbreak\pagebreak[2]}\Standort{CUL, Schnitzler, B 30.}
\physDesc{Brief, 1 Blatt, 4 Seiten, 2237 Zeichen
\newline{}Handschrift: schwarze Tinte, deutsche Kurrent
\newline{}Schnitzler: mit Bleistift Vermerk: »\textsc{Ehrenstein}« }\Standort{Jerusalem, The National Library of Israel, ARC. Ms. Var. 306 1 117.}
\physDesc{Briefentwurf, 1 Blatt, 4 Seiten, 2237 Zeichen
\newline{}Handschrift: schwarze Tinte, deutsche Kurrent}\Standort{Jerusalem, The National Library of Israel, ARC. Ms. Var. 306 1 117.}
\physDesc{Briefentwurf, 1 Blatt, 4 Seiten, 2237 Zeichen
\newline{}Handschrift: schwarze Tinte, deutsche Kurrent}\Standort{Jerusalem, The National Library of Israel, ARC. Ms. Var. 306 1 117.}
\physDesc{Briefentwurf, 1 Blatt, 2 Seiten, 2237 Zeichen
\newline{}Handschrift: schwarze Tinte, deutsche Kurrent}\Standort{Jerusalem, The National Library of Israel, ARC. Ms. Var. 306 1 117.}
\physDesc{Briefentwurf, 1 Blatt, 2 Seiten, 2237 Zeichen
\newline{}Handschrift: schwarze Tinte, deutsche Kurrent}
\buchAbdrucke{\weitereDrucke{Albert Ehrenstein: \emph{Briefe}. München: \emph{Boer} 1989, S. 22–23.} }\toendnotes[C]{\smallbreak}
\pstart
           
\pstart
           {\pb}Wien, XVI. \textsc{Ottakringerstr
                           114}\oindex{Ottakringer Strasse@\textbf{Ottakringer Straße}, \emph{Straße (K.STR)}|pw}.\pend
           
\pstart
           \raggedleft{}\textsc{10. Oktober 1908}.\pend
           \pend
           
\pstart\center{}\textsc{Sehr geehrter Herr Doktor!}\pend\vspace{0.5em}
\pstart
           Verhindert durch \label{K_L01792-1v}\edtext{Handarbeiten}{\lemma{\textnormal{\emph{Handarbeiten}}}\Cendnote{\textnormal{Ehrenstein\pwindex{Ehrenstein, Albert 23.12.1886 – 08.04.1950@\textsc{Ehrenstein, Albert} (23.12.1886 – 08.04.1950), \emph{Schriftsteller/Schriftstellerin}|pwk} hatte 1905 ein
                  Universitätsstudium der Geschichte, Kunstgeschichte und Geographie
                  aufgenommen.}}}\label{K_L01792-1} geographiſch-geſchichtlichen Charakters, noch mehr aber durch
               das Nochnichtvorhandenſein eigener Artefakte, die mir als halbwegs annehmbare
               Legitimation für eine abermalige Beläſtigung hätten dienen können, kam ich im
                  Januar nicht Ihrer mich erfreuenden Aufforderung nach, bei Ihnen ſehr
               geehrter Herr Doktor, einmal vorzuſprechen. Die Behelligung durch Studien hat nicht
               aufgehört, Zeitmangel alſo könnte manche der in den beiliegenden Skizzen
               zutagetretenden Flüchtigkeiten, das Fehlen intimerer Feilung erklären, {\pb}abgeſehen von meinem Widerwillen dagegen,
               Kleinigkeiten ſelber an das gedulderſchöpfende, zeitraubende Überſchreiben vielleicht
               ausſichtsloſer Erzeugniſſe zu schreiten. Leider ſind die genannten Unterlaſſungen das
               Wenigste. Kein der Produktion gewidmeter Tag iſt ohne hunderterlei teils ungewollte,
               teils mehr als beabſichtigte Störungen häuslicherſeits dahingegangen. Der ruhige Fluß
               der Darſtellungen, mit dem endlich beſchenkt worden zu ſein ich mich ſchon freute,
               bald gehemmt, unterbrochen machte einer mehr stoßweiſen, abgeriſſenen Art der
               der Erzählung Platz. Notwendig ſind die vorliegenden Darbietungen, ſobald
               Schwung {\pb}und Stimmung von außen
               verſcheucht worden, in einem dem Laſter ſozuſagen jeden Augenblick freigebendem Stil
               geſchrieben, was beſonders bei der letzten Novellette\pwindex{Tubutsch@\emph{Tubutsch}|pwv} ermüden muß, welche an ſich Langeweile und
               Enttäuſchung, einen an den Auslagen der Geſchäfte und Leute entlang lebenden Menſchen
               zu ſchildern unternimmt. Wenn ich mich trotz alledem erkühne, an Sie, ſehr geehrter
               Herr Doktor, mit dem wenig gerechtfertigten Anſinnen heranzutreten, die übrigens
               teilweiſe untereinander in Konnex und Abfolge ſtehenden Werkchen (einzeln) zu
               beurteilen die Güte zu haben, {\pb}die
               möglicherweiſe wertvolle \label{K_L01792-2v}\edtext{Titelnovelle\pwindex{Seltene Gaeste@\emph{Seltene Gäste}|pwuv}}{\lemma{\textnormal{\emph{Titelnovelle}}}\Cendnote{\textnormal{Es dürfte sich, was durch den Hinweis
                  auf den Umfang angedeutet wird, um das 87 Seiten umfassende Manuskript von \emph{Seltene Gäste}\pwindex{Seltene Gaeste@\emph{Seltene Gäste}|pwk} handeln, das in dieser Form erst
                     1991 veröffentlicht wurde.}}}\label{K_L01792-2}, falls es irgend angeht, auf
               einmal leſen zu wollen – ſo bitte ich dieſe nicht anſpruchsvollen Zumutungen nicht zu
               mißdeuten. Nichts liegt mir ferner als Prätention, nichts wünſche ich ſoſehr als Rat
               und Hilfe. In der Hoffnung, diesmal, wenn verdient, realerer Erfolge teilhaftig zu
               werden, verbleibe ich\hspace*{1.5em}Hochachtungsvoll
               ergebenſt{\\}Ihr Sie, ſehr geehrter Herr Doktor, verehrender\pend
           \pstart \spacefill\mbox{Albert Ehrenstein.}\pend{}\selectlanguage{ngerman}\endnumbering\briefempfaengerindex{Schnitzler, Arthur@\textsc{Schnitzler, Arthur}!zzzEhrenstein, Albert@\emph{von Albert Ehrenstein}!1908-10-101@{10. 10. 1908}|)be}\mylabel{L01792h}  \normalsize

\doendnotes{C}
\bigskip
\vfill

\clearpage

\footnotesize

\lohead{\textsc{register}}

% Definiere theindex-Environment komplett neu ohne reledmac
\makeatletter
\renewenvironment{theindex}{%
  \section*{\indexname}%
  \setlength{\parindent}{0pt}%
  \setlength{\parskip}{0pt plus 0.3pt}%
  \let\item\@idxitem
}{%
  \clearpage
}
\makeatother

\IfFileExists{\jobname-pw.ind}{\input{\jobname-pw.ind}}{}

\end{document}

      