%% latex-leseansicht-vorspann.tex
%% Vorspann für die Leseansicht.
%% Lädt die gemeinsame Datei latex-vorspann.tex mit nicht gesetztem Schalter.

\newif\ifkorrekturansicht
\korrekturansichtfalse

\input{../tex-inputs/latex-vorspann}

\begin{center}
            \textcolor{red}{ENTWURF. ENTZIFFERUNG NOCH NICHT KORREKTURGELESEN}
                      \end{center}
            
               \section[Arthur Schnitzler an Hugo von Hofmannsthal, 15. 10. 1904]{ Arthur Schnitzler an Hugo von Hofmannsthal, 15. 10. 1904}\nopagebreak\mylabel{v}\rehead{ }\begin{ledgroupsized}[t]{13cm}\normalsize\beginnumbering\briefempfaengerindex{Hofmannsthal, Hugo von@\textsc{Hofmannsthal, Hugo von}!zzzSchnitzler, Arthur@\emph{von Arthur Schnitzler}!1904-10-151@{15. 10. 1904}|(be} \toendnotes[C]{\smallbreak\pagebreak[2]} \Standort{FDH, Hs-30885,117.}
\physDesc{Brief, 5 Blätter, 5 Seiten
\newline{}Schreibmaschine
\newline{}Handschrift: schwarze Tinte, deutsche Kurrent (\noindent{}Korrekturen, Paginierung und Schluss)\newline{}Ordnung: mit Bleistift von Schnitzler ab dem zweiten Blatt mutmaßlich bei
                                 der Durchsicht der Briefe 1929 jeweils datiert: »15/10 904« }\buchAbdrucke{\weitereDrucke{Hugo von Hofmannsthal, Arthur Schnitzler: \emph{Briefwechsel}. Hg. Therese Nickl und Heinrich Schnitzler. Frankfurt am Main: \emph{S. Fischer} 1964, S. 206.} }\toendnotes[C]{\smallbreak}\pstart
           \raggedleft{}{\pb}Wien\oindex{Wien@\textbf{Wien}|pw}, 15. Oktober 1904.\pend
           \pstart{}Lieber Hugo!\pend\pstart
           Dass Sie Lindemann\pwindex{Lindemann, Gustav 24.08.1872 – 06.05.1960@\textsc{Lindemann, Gustav} (24.08.1872 – 06.05.1960), \emph{Theaterleiter, Schauspieler}|pw} Ihre Stücke verweigerten,
               wundert mich, denn dazu liegt meiner Empfindung nach keine Ursache vor. Fischer\pwindex{Fischer, Samuel 24.12.1859 – 15.10.1934@\textsc{Fischer, Samuel} (24.12.1859 – 15.10.1934), \emph{Verleger}|pw} schrieb mir vor \label{K_L01455_1v}\edtext{Monaten}{\lemma{\textnormal{\emph{Monaten}}}\Cendnote{\textnormal{Im Januar 1904 hatten Verhandlungen über eine
                  Tournee Lindemann\pwindex{Lindemann, Gustav 24.08.1872 – 06.05.1960@\textsc{Lindemann, Gustav} (24.08.1872 – 06.05.1960), \emph{Theaterleiter, Schauspieler}|pwk}s, bei der \emph{Der einsame Weg}\pwindex{Schnitzler, Arthur 15.05.1862 – 21.10.1931@\textsc{Schnitzler, Arthur} (15.05.1862 – 21.10.1931), \emph{Schriftsteller, Mediziner}!einsame Weg. Schauspiel in fuenf Akten1904@\strich\emph{Der einsame Weg. Schauspiel in fünf Akten} {[}1904{]}|pwk} gegeben werden sollte, stattgefunden.
                  Ab Mai beherrscht die Frage, ob es eine eine Vorauszahlung hätte
                  geben sollen, die Korrespondenz Schnitzler\pwindex{Schnitzler, Arthur 15.05.1862 – 21.10.1931@\textsc{Schnitzler, Arthur} (15.05.1862 – 21.10.1931), \emph{Schriftsteller, Mediziner}|pwk}s mit
                     Fischer\pwindex{Fischer, Samuel 24.12.1859 – 15.10.1934@\textsc{Fischer, Samuel} (24.12.1859 – 15.10.1934), \emph{Verleger}|pwk}. Am 29. 8. 1904
                  schreibt Fischer\pwindex{Fischer, Samuel 24.12.1859 – 15.10.1934@\textsc{Fischer, Samuel} (24.12.1859 – 15.10.1934), \emph{Verleger}|pwk} besagte Aufforderung, dass
                  die Verlagsautoren gemeinsam agieren sollen.}}}\label{K_L01455_1h}, er wolle seinen Autoren das
               Ansinnen stellen, aus Ursache des bewussten Streitfalles zwischen ihm und L.\pwindex{Lindemann, Gustav 24.08.1872 – 06.05.1960@\textsc{Lindemann, Gustav} (24.08.1872 – 06.05.1960), \emph{Theaterleiter, Schauspieler}|pw}, resp. zwischen mir und L.\pwindex{Lindemann, Gustav 24.08.1872 – 06.05.1960@\textsc{Lindemann, Gustav} (24.08.1872 – 06.05.1960), \emph{Theaterleiter, Schauspieler}|pw} in Betreff des »Einsamen
                  Wegs\pwindex{Schnitzler, Arthur 15.05.1862 – 21.10.1931@\textsc{Schnitzler, Arthur} (15.05.1862 – 21.10.1931), \emph{Schriftsteller, Mediziner}!einsame Weg. Schauspiel in fuenf Akten1904@\strich\emph{Der einsame Weg. Schauspiel in fünf Akten} {[}1904{]}|pw}«, \introOben{}dem L.\pwindex{Lindemann, Gustav 24.08.1872 – 06.05.1960@\textsc{Lindemann, Gustav} (24.08.1872 – 06.05.1960), \emph{Theaterleiter, Schauspieler}|pw}ſchen
                  Unternehmen\introOben{} ihre dramatischen Arbeiten bis auf Weiteres zu verweigern. Ich
               sprach mich mit Entschiedenheit dagegen aus, da mir jede Art von Solidarität ziemlich
               zuwider ist und ich besonders in dem vorliegenden Fall es auch von jedem andern Autor
               unrichtig gefunden hätte, aus einer rein \introOben{}privat-\introOben{}prozessualen Sache eine \introOben{}öffentliche\introOben{} Affäre zu
               machen und damit vielleicht \strikeout{noch} andere Leute, die
               die ganze Geschichte nicht interessiert, materiell zu schädigen. Damit erledigt sich
               Ihre Frage von selbst, und ich bit{\pb}te Sie nur, ohne jede
               Rücksicht auf mich, auch bei Lindemann\pwindex{Lindemann, Gustav 24.08.1872 – 06.05.1960@\textsc{Lindemann, Gustav} (24.08.1872 – 06.05.1960), \emph{Theaterleiter, Schauspieler}|pw} Ihre
               Stücke ganz nach Gutdünken zu placieren.\pend
           \pstart
           Aber sonst steht die Sache nicht so einfach, und Lindemann\pwindex{Lindemann, Gustav 24.08.1872 – 06.05.1960@\textsc{Lindemann, Gustav} (24.08.1872 – 06.05.1960), \emph{Theaterleiter, Schauspieler}|pw} ist gewiss nicht so frei von Schuld, als es im Brief des Fräulein
                  Dumont\pwindex{Dumont, Louise 22.02.1862 – 16.05.1932@\textsc{Dumont, Louise} (22.02.1862 – 16.05.1932), \emph{Theaterleiterin, Schauspielerin}|pw} an Sie in allerbestem Glauben
               dargestellt wird.\pend
           \pstart
           Insbesondere handelt es sich ja darum, dass L.\pwindex{Lindemann, Gustav 24.08.1872 – 06.05.1960@\textsc{Lindemann, Gustav} (24.08.1872 – 06.05.1960), \emph{Theaterleiter, Schauspieler}|pw}
               nach der matten Aufnahme des Stück\pwindex{Schnitzler, Arthur 15.05.1862 – 21.10.1931@\textsc{Schnitzler, Arthur} (15.05.1862 – 21.10.1931), \emph{Schriftsteller, Mediziner}!einsame Weg. Schauspiel in fuenf Akten1904@\strich\emph{Der einsame Weg. Schauspiel in fünf Akten} {[}1904{]}|pwv}s durch das Berlin\oindex{Berlin@\textbf{Berlin}|pw}er Publikum weder
               von einer vorher, noch von einer nachher zu zahlenden Garantiesumme etwas wissen
               wollte, trotzdem vor der Aufführung – ich glaube, am Tage der Aufführung – ein
               Telegramm \introOben{}von ihm\introOben{} eingelaufen war, das sich mit den letzten
               Bedingungen Fischers\pwindex{Fischer, Samuel 24.12.1859 – 15.10.1934@\textsc{Fischer, Samuel} (24.12.1859 – 15.10.1934), \emph{Verleger}|pw} einverstanden erklärte, –
               womit nicht nur nach allgemeinem Usus, sondern auch nach dem Urteil juridischer
               Sachverständiger, ein rechtsgiltiger Vertrag zustande gekommen war; \introOben{}–\introOben{} und dass sich Fischer\pwindex{Fischer, Samuel 24.12.1859 – 15.10.1934@\textsc{Fischer, Samuel} (24.12.1859 – 15.10.1934), \emph{Verleger}|pw}
               durchaus nicht hütet, die Angelegenheit auf dem Klageweg zu erledi{\pb}gen, /wie Frl. Dumont\pwindex{Dumont, Louise 22.02.1862 – 16.05.1932@\textsc{Dumont, Louise} (22.02.1862 – 16.05.1932), \emph{Theaterleiterin, Schauspielerin}|pw} in ihrem Brief sagt/ ersehen Sie am besten aus den zwei Briefen, die
               ich Ihnen hier beilege und um deren Rücksendung ich Sie bitte, und aus denen sie
               erstens ersehen, dass Justizrat Jonas\pwindex{Jonas, Paul 28.09.1850 – 11.02.1916@\textsc{Jonas, Paul} (28.09.1850 – 11.02.1916), \emph{Rechtsanwalt}|pw} die
               Forderung der sofortigen Zahlung der 5000 M. für begründet hält, und zweitens dass
                  Fischer\pwindex{Fischer, Samuel 24.12.1859 – 15.10.1934@\textsc{Fischer, Samuel} (24.12.1859 – 15.10.1934), \emph{Verleger}|pw} nur meine Einwilligung abwartet, um
               den Prozess gegen Lindemann\pwindex{Lindemann, Gustav 24.08.1872 – 06.05.1960@\textsc{Lindemann, Gustav} (24.08.1872 – 06.05.1960), \emph{Theaterleiter, Schauspieler}|pw} einzuleiten. Diese
               Einwilligung werde ich ihm natürlich nicht versagen.\pend
           \pstart
           Worin ich Fischer\pwindex{Fischer, Samuel 24.12.1859 – 15.10.1934@\textsc{Fischer, Samuel} (24.12.1859 – 15.10.1934), \emph{Verleger}|pw} Unrecht gebe, ist eigentlich
               nur, dass er nicht gleich zu Beginn der Verhandlungen – lange vor Aufführung des Stücks\pwindex{Schnitzler, Arthur 15.05.1862 – 21.10.1931@\textsc{Schnitzler, Arthur} (15.05.1862 – 21.10.1931), \emph{Schriftsteller, Mediziner}!einsame Weg. Schauspiel in fuenf Akten1904@\strich\emph{Der einsame Weg. Schauspiel in fünf Akten} {[}1904{]}|pwv} in Berlin\oindex{Berlin@\textbf{Berlin}|pw} – den Lindemann\pwindex{Lindemann, Gustav 24.08.1872 – 06.05.1960@\textsc{Lindemann, Gustav} (24.08.1872 – 06.05.1960), \emph{Theaterleiter, Schauspieler}|pw}’schen
               Antrag in seinem ganzen Umfang /5000 M. Garantie und Aufführung des Stücks\pwindex{Schnitzler, Arthur 15.05.1862 – 21.10.1931@\textsc{Schnitzler, Arthur} (15.05.1862 – 21.10.1931), \emph{Schriftsteller, Mediziner}!einsame Weg. Schauspiel in fuenf Akten1904@\strich\emph{Der einsame Weg. Schauspiel in fünf Akten} {[}1904{]}|pwv} in allen von L.\pwindex{Lindemann, Gustav 24.08.1872 – 06.05.1960@\textsc{Lindemann, Gustav} (24.08.1872 – 06.05.1960), \emph{Theaterleiter, Schauspieler}|pw} angegebenen Städten/ angenommen hat, obwol ich ihm
               telegraphisch meine entschiedene Zustimmung kundgab, sondern dass er sich dann erst
               in Verhandlungen über einzelne Städte einliess, die von der Tournée ausgeschlossen
               sein sollte\introOben{}n\introOben{}. Aber \introOben{}»\introOben{}unvornehm\introOben{}«\introOben{} kann ich das auch nicht finden.\pend
           \pstart
           {\pb}Was aber nun eine \uline{vorherige} Zahlung der Garantiesumme anlangt, so würde ich zu dieser
               Forderung in einem ähnlichen Fall meinen Vertreter neuerdings autorisieren; denn
               gerade die in dem Brief des Frl. Dumont\pwindex{Dumont, Louise 22.02.1862 – 16.05.1932@\textsc{Dumont, Louise} (22.02.1862 – 16.05.1932), \emph{Theaterleiterin, Schauspielerin}|pw}
               angeführten Daten beweisen, wie gering die finanzielle Sicherheit ist, die in einem
               Unternehmen in der Art des Lindemann\pwindex{Lindemann, Gustav 24.08.1872 – 06.05.1960@\textsc{Lindemann, Gustav} (24.08.1872 – 06.05.1960), \emph{Theaterleiter, Schauspieler}|pw}’schen,
               selbst bei den besten Absichten und den reinsten künstlerischen Intentionen, den
               Autoren geboten werden kann.\pend
           \pstart
           Uebrigens hätte ja Lindemann\pwindex{Lindemann, Gustav 24.08.1872 – 06.05.1960@\textsc{Lindemann, Gustav} (24.08.1872 – 06.05.1960), \emph{Theaterleiter, Schauspieler}|pw} sich mindestens zu
               einer teilweisen vorherigen Zahlung verstehen können; aber, ganz im Gegenteil, – und
               dies ist wol das Wichtigste bei der Betrachtung des ganzen Streitfalls –, nach der
                  Berlin\oindex{Berlin@\textbf{Berlin}|pw}er Première wollte er, trotz des vor der
               Première eingelangten vertragsgleichen Telegramms, weder von einer vorher, noch von
               einer nachher zu zahlenden Garantie, noch überhaupt von einer Aufführung des Stück\pwindex{Schnitzler, Arthur 15.05.1862 – 21.10.1931@\textsc{Schnitzler, Arthur} (15.05.1862 – 21.10.1931), \emph{Schriftsteller, Mediziner}!einsame Weg. Schauspiel in fuenf Akten1904@\strich\emph{Der einsame Weg. Schauspiel in fünf Akten} {[}1904{]}|pwv}es im Verlauf seiner Tournée
               etwas wissen.\pend
           \pstart
           {\pb}Bitte, lieber Hugo, grüssen Sie Frl. Dumont\pwindex{Dumont, Louise 22.02.1862 – 16.05.1932@\textsc{Dumont, Louise} (22.02.1862 – 16.05.1932), \emph{Theaterleiterin, Schauspielerin}|pw} herzlich und teilen Sie ihr doch in Ihrer Antwort auch
               mit, was ich Ihnen gleich im Beginn dieses Briefs \substVorne{}\textsuperscript{erzählt}{\allowbreak}\substDazwischen{}gesagt\substHinten{} habe: dass es durchaus meinen Intentionen widersprach und widerspricht, wenn
                  Fischer\pwindex{Fischer, Samuel 24.12.1859 – 15.10.1934@\textsc{Fischer, Samuel} (24.12.1859 – 15.10.1934), \emph{Verleger}|pw} aus Anlass des bekannten Streitfalls
               dem neuen Unternehmen auch Stücke seiner anderen Autoren verweigert, dass mir im
               übrigen aber das Vorgehen Fischers\pwindex{Fischer, Samuel 24.12.1859 – 15.10.1934@\textsc{Fischer, Samuel} (24.12.1859 – 15.10.1934), \emph{Verleger}|pw} in meiner
               Sache einwandfrei erscheint.\pend
           \pstart
           {[}hs.:{]} Herzliche Grüße und auf baldigs Wiederſehen\damage{.} Mit meinem »\textsc{burlesken} Abend« bei Rhardt\pwindex{Reinhardt, Max 09.09.1873 – 30.10.1943@\textsc{Reinhardt, Max} (09.09.1873 – 30.10.1943), \emph{Theaterleiter, Regisseur, Schauspieler}|pw} iſt’s nichts. Er will die Familienſcene\pwindex{Schnitzler, Arthur 15.05.1862 – 21.10.1931@\textsc{Schnitzler, Arthur} (15.05.1862 – 21.10.1931), \emph{Schriftsteller, Mediziner}!Haus Delorme. Eine Familienszene1977@\strich\emph{Das Haus Delorme. Eine Familienszene} {[}1977{]}|pwv} allein, die ich aber lieber für beſſere
               Gelegenheit zurückbehalte. Über Kakadu\pwindex{Schnitzler, Arthur 15.05.1862 – 21.10.1931@\textsc{Schnitzler, Arthur} (15.05.1862 – 21.10.1931), \emph{Schriftsteller, Mediziner}!gruene Kakadu. Groteske in einem Akt1.3.1899 – 1.3.1899@\strich\emph{Der grüne Kakadu. Groteske in einem Akt} {[}1.3.1899 – 1.3.1899{]}|pw}–Abenteurer\pwindex{Hofmannsthal, Hugo von 01.02.1874 – 15.07.1929@\textsc{Hofmannsthal, Hugo von} (01.02.1874 – 15.07.1929), \emph{Schriftsteller}!Abenteurer und die Saengerin oder Die Geschenke des Lebens18. 3. 1899@\strich\emph{Der Abenteurer und die Sängerin oder Die Geschenke des Lebens} {[}18. 3. 1899{]}|pw} iſt noch kein Telegramm eingelangt.\pend
           \pstart Ihr \spacefill\mbox{A.}\pend{}\endnumbering\briefempfaengerindex{Hofmannsthal, Hugo von@\textsc{Hofmannsthal, Hugo von}!zzzSchnitzler, Arthur@\emph{von Arthur Schnitzler}!1904-10-151@{15. 10. 1904}|)be}\mylabel{h}\end{ledgroupsized}  \newcommand{\dateiname}{L01455}\newcommand{\titel}{Arthur Schnitzler an Hugo von Hofmannsthal, 15. 10. 1904}\newcommand{\editorInnen}{Martin Anton Müller und Gerd-Hermann Susen}%% latex-leseansicht-abspann.tex
%% Abspann für die Leseansicht.
%% Der Schalter \ifkorrekturansicht ist bereits durch den Vorspann gesetzt.

%% latex-abspann.tex
%% Gemeinsamer Abspann für Korrekturansicht und Leseansicht.
%% Setzt den Schalter \ifkorrekturansicht voraus (gesetzt in den
%% einbindenden Dateien latex-korrekturansicht-abspann.tex bzw.
%% latex-leseansicht-abspann.tex).
%% ---------------------------------------------------------------

\normalsize

% Das esempio-Environment wird nur in der Leseansicht benötigt
\ifkorrekturansicht\else
\newenvironment{esempio}[3]%
{
    \vspace{1.5ex}
    \rlap{\underline{#1}}
    \par
    \setlength{\parindent}{0cm}
    \nopagebreak
    \leftskip=#2cm
    \rightskip=#3cm
}
{
    \par
}
\fi

\doendnotes{C}
\bigskip
\vfill

\clearpage

\footnotesize

\ifkorrekturansicht
  \lohead{\textsc{register}}
\fi

% theindex-Environment neu definieren ohne reledmac
\makeatletter
\renewenvironment{theindex}{%
  \ifkorrekturansicht
    \section*{\indexname}%
  \else
    \subsubsection*{Index der erwähnten Entitäten}%
  \fi
  \setlength{\parindent}{0pt}%
  \setlength{\parskip}{0pt plus 0.3pt}%
  \let\item\@idxitem
}{%
  \ifkorrekturansicht\clearpage\fi
}
\makeatother

\IfFileExists{\jobname-pw.ind}{\input{\jobname-pw.ind}}{}

% Quellenangabe nur in der Leseansicht
\ifkorrekturansicht\else
% Fallback-Definitionen, falls die .tex-Datei \titel etc. nicht gesetzt hat
\providecommand{\titel}{}
\providecommand{\editorInnen}{}
\providecommand{\dateiname}{\jobname}

\vspace{3cm}

\vfill

\footnotesize
\textsc{Quelle}: \titel. Herausgegeben von {\editorInnen}. In: \emph{Arthur Schnitzler: Briefwechsel mit Autorinnen und Autoren}.
 Digitale Edition, https://schnitzler-briefe.acdh.oeaw.ac.at/{\dateiname}.html (Stand \today)
\fi

\end{document}


      