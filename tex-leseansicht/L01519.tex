%% latex-leseansicht-vorspann.tex
%% Vorspann für die Leseansicht.
%% Lädt die gemeinsame Datei latex-vorspann.tex mit nicht gesetztem Schalter.

\newif\ifkorrekturansicht
\korrekturansichtfalse

\input{../tex-inputs/latex-vorspann}


\section[Richard Beer-Hofmann an Arthur Schnitzler, 25. 5. 1905]{L01519 Richard Beer-Hofmann an Arthur Schnitzler, 25. 5. 1905}
\nopagebreak\mylabel{L01519v}
\rehead{ }\normalsize\beginnumbering\briefempfaengerindex{Schnitzler, Arthur@\textsc{Schnitzler, Arthur}!zzzBeer-Hofmann, Richard@\emph{von Richard Beer-Hofmann}!1905-05-251@{25. 5. 1905}|(be}
\toendnotes[C]{\smallbreak\pagebreak[2]}
\correspDesc{Versand  durch Richard Beer-Hofmann am 25. 5. 1905 in Rodaun
\newline{}Erhalt  durch Arthur Schnitzler im Zeitraum [26. 5. 1905
                  – 30. 5. 1905?] in Wien}\toendnotes[C]{\smallbreak}
\Standort{CUL, Schnitzler, B 8.}
\physDesc{Brief, 1 Blatt, 2 Seiten, 945 Zeichen
\newline{}Handschrift: Bleistift, lateinische Kurrent
\newline{}Ordnung: mit Bleistift von unbekannter Hand nummeriert:
                                    »199« }
\buchAbdrucke{\weitereDrucke{Arthur Schnitzler, Richard Beer-Hofmann: \emph{Briefwechsel 1891–1931}. Herausgegeben von Konstanze Fliedl. Wien, Zürich: \emph{Europaverlag} 1992, S. 173.} }\toendnotes[C]{\smallbreak}
\pstart
           \raggedleft{}{\pb}Rodaun\oindex{Wien@\textbf{Wien}!XXIII., Liesing@\textbf{XXIII., Liesing}!Rodaun@\textbf{Rodaun}, \emph{Region}|pw}{ }25/V 05\pend
           \vspace{0.5em}
\pstart
           Lieber! Es bedeutet den Anfang. Sie haben es errathen. Ich \uline{glaube}, Ende dieser Woche wird der Grund gekauft
               werden. So{\geminationm}erpläne? Ich habe keine, ausser – so hoffe
               ich – Lido\oindex{Lido@\textbf{Lido}|pw} im September.
               Augenblicklich viel Unruhe – wir haben die arme alte Tante\pwindex{Beer, Agnes 12.\,2.\,1833 – 27.\,7.\,1909 Wien@\textsc{Beer, Agnes} (12.\,2.\,1833 – 27.\,7.\,1909 Wien)|pwv} zu uns herausgeno{\geminationm}en.\pend
           
\pstart
           Ihre lieben Worte habe ich gut brauchen können, nach all dem Widerlichen und
               Lügenhaften das ich zu hören bekam. I{\geminationm}er wieder die
               Legende von meiner »zwölf{\pb}jährigen«
               Arbeit, und i{\geminationm}er wieder bei Allen »Schule« »Schüler«!
               Und \uline{dies}e Aufführung\pwindex{Beer-Hofmann, Richard 11.\,7.\,1866 Wien – 26.\,9.\,1945 New York City@\textsc{Beer-Hofmann, Richard} (11.\,7.\,1866 Wien – 26.\,9.\,1945 New York City), \emph{Schriftsteller}!Graf von Charolais. Ein Trauerspiel@\strich\emph{Der Graf von Charolais. Ein Trauerspiel}|pwv}. Ihre Berechtigung war die Exaktheit der vier
               ersten Akte, der letzte wurde i{\geminationm}er verschleppt. Und
               durch Neubesetzungen gieng die Exaktheit verloren, auch durch Schlamperei. Blieben
               die Einzelleistungen!\pend
           
\pstart
           Ich ko{\geminationm}e sehr bald zu Ihnen. Wir sehen dann zusa{\geminationm}en den Grund an, sobald er mir gehört. Sehen Sie ihn
               dann mit dem selben gütigen Blick an, mit dem Sie seit so vielen Jahren alles ansahen
                  \label{T_L01519-1v}\edtext{was in jedem Sinne mein Eigen war.
                  Ihr}{\lemma{\textnormal{\emph{was … Ihr}}}\Cendnote{\textnormal{weiter am rechten Rand}}}\label{T_L01519-1}\spacefill\mbox{Richard}\pend
           \selectlanguage{ngerman}\endnumbering\briefempfaengerindex{Schnitzler, Arthur@\textsc{Schnitzler, Arthur}!zzzBeer-Hofmann, Richard@\emph{von Richard Beer-Hofmann}!1905-05-251@{25. 5. 1905}|)be}\mylabel{L01519h}  \newcommand{\dateiname}{L01519}\newcommand{\titel}{Richard Beer-Hofmann an Arthur Schnitzler, 25. 5. 1905}\newcommand{\editorInnen}{Martin Anton Müller und Gerd-Hermann Susen}%% latex-leseansicht-abspann.tex
%% Abspann für die Leseansicht.
%% Der Schalter \ifkorrekturansicht ist bereits durch den Vorspann gesetzt.

%% latex-abspann.tex
%% Gemeinsamer Abspann für Korrekturansicht und Leseansicht.
%% Setzt den Schalter \ifkorrekturansicht voraus (gesetzt in den
%% einbindenden Dateien latex-korrekturansicht-abspann.tex bzw.
%% latex-leseansicht-abspann.tex).
%% ---------------------------------------------------------------

\normalsize

% Das esempio-Environment wird nur in der Leseansicht benötigt
\ifkorrekturansicht\else
\newenvironment{esempio}[3]%
{
    \vspace{1.5ex}
    \rlap{\underline{#1}}
    \par
    \setlength{\parindent}{0cm}
    \nopagebreak
    \leftskip=#2cm
    \rightskip=#3cm
}
{
    \par
}
\fi

\doendnotes{C}
\bigskip
\vfill

\clearpage

\footnotesize

\ifkorrekturansicht
  \lohead{\textsc{register}}
\fi

% theindex-Environment neu definieren ohne reledmac
\makeatletter
\renewenvironment{theindex}{%
  \ifkorrekturansicht
    \section*{\indexname}%
  \else
    \subsubsection*{Index der erwähnten Entitäten}%
  \fi
  \setlength{\parindent}{0pt}%
  \setlength{\parskip}{0pt plus 0.3pt}%
  \let\item\@idxitem
}{%
  \ifkorrekturansicht\clearpage\fi
}
\makeatother

\IfFileExists{\jobname-pw.ind}{\input{\jobname-pw.ind}}{}

% Quellenangabe nur in der Leseansicht
\ifkorrekturansicht\else
% Fallback-Definitionen, falls die .tex-Datei \titel etc. nicht gesetzt hat
\providecommand{\titel}{}
\providecommand{\editorInnen}{}
\providecommand{\dateiname}{\jobname}

\vspace{3cm}

\vfill

\footnotesize
\textsc{Quelle}: \titel. Herausgegeben von {\editorInnen}. In: \emph{Arthur Schnitzler: Briefwechsel mit Autorinnen und Autoren}.
 Digitale Edition, https://schnitzler-briefe.acdh.oeaw.ac.at/{\dateiname}.html (Stand \today)
\fi

\end{document}


