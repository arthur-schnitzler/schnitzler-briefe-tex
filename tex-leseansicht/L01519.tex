%% latex-korrekturansicht-vorspann.tex
%% Vorspann für die Korrekturansicht.
%% Lädt die gemeinsame Datei latex-vorspann.tex mit gesetztem Schalter.

\newif\ifkorrekturansicht
\korrekturansichttrue

\input{../tex-inputs/latex-vorspann}


\section[Richard Beer-Hofmann an Arthur Schnitzler, 25. 5. 1905]{L01519 Richard Beer-Hofmann an Arthur Schnitzler, 25. 5. 1905}
\nopagebreak\mylabel{L01519v}
\rehead{ }\normalsize\beginnumbering\briefempfaengerindex{Schnitzler, Arthur@\textsc{Schnitzler, Arthur}!zzzBeer-Hofmann, Richard@\emph{von Richard Beer-Hofmann}!1905-05-251@{25. 5. 1905}|(be}
\toendnotes[C]{\smallbreak\pagebreak[2]}\Standort{CUL, Schnitzler, B 8.}
\physDesc{Brief, 1 Blatt, 2 Seiten, 945 Zeichen
\newline{}Handschrift: Bleistift, lateinische Kurrent
\newline{}Ordnung: mit Bleistift von unbekannter Hand nummeriert:
                                    »199« }
\buchAbdrucke{\weitereDrucke{Arthur Schnitzler, Richard Beer-Hofmann: \emph{Briefwechsel 1891–1931}. Wien, Zürich: \emph{Europaverlag} 1992, S. 173.} }\toendnotes[C]{\smallbreak}
\pstart
           \raggedleft{}{\pb}Rodaun\oindex{Rodaun@\textbf{Rodaun}, \emph{A.ADM4}|pw}{ }25/V 05\pend
           \vspace{0.5em}
\pstart
           Lieber! Es bedeutet den Anfang. Sie haben es errathen. Ich \uline{glaube}, Ende dieser Woche wird der Grund gekauft
               werden. So{\geminationm}erpläne? Ich habe keine, ausser – so hoffe
               ich – Lido\oindex{Lido@\textbf{Lido}, \emph{P.PPL}|pw} im September.
               Augenblicklich viel Unruhe – wir haben die arme alte Tante\pwindex{Beer, Agnes 1833-02-12 – 27.7.1909@\textsc{Beer, Agnes} (1833-02-12 – 27.7.1909)|pwv} zu uns herausgeno{\geminationm}en.\pend
           
\pstart
           Ihre lieben Worte habe ich gut brauchen können, nach all dem Widerlichen und
               Lügenhaften das ich zu hören bekam. I{\geminationm}er wieder die
               Legende von meiner »zwölf{\pb}jährigen«
               Arbeit, und i{\geminationm}er wieder bei Allen »Schule« »Schüler«!
               Und \uline{dies}e Aufführung\pwindex{Graf von Charolais. Ein Trauerspiel@\emph{Der Graf von Charolais. Ein Trauerspiel}|pwv}. Ihre Berechtigung war die Exaktheit der vier
               ersten Akte, der letzte wurde i{\geminationm}er verschleppt. Und
               durch Neubesetzungen gieng die Exaktheit verloren, auch durch Schlamperei. Blieben
               die Einzelleistungen!\pend
           
\pstart
           Ich ko{\geminationm}e sehr bald zu Ihnen. Wir sehen dann zusa{\geminationm}en den Grund an, sobald er mir gehört. Sehen Sie ihn
               dann mit dem selben gütigen Blick an, mit dem Sie seit so vielen Jahren alles ansahen
                  \label{T_L01519-1v}\edtext{was in jedem Sinne mein Eigen war.
                  Ihr}{\lemma{\textnormal{\emph{was … Ihr}}}\Cendnote{\textnormal{weiter am rechten Rand}}}\label{T_L01519-1}\spacefill\mbox{Richard}\pend
           \selectlanguage{ngerman}\endnumbering\briefempfaengerindex{Schnitzler, Arthur@\textsc{Schnitzler, Arthur}!zzzBeer-Hofmann, Richard@\emph{von Richard Beer-Hofmann}!1905-05-251@{25. 5. 1905}|)be}\mylabel{L01519h}  \normalsize

\doendnotes{C}
\bigskip
\vfill

\clearpage

\footnotesize

\lohead{\textsc{register}}

% Definiere theindex-Environment komplett neu ohne reledmac
\makeatletter
\renewenvironment{theindex}{%
  \section*{\indexname}%
  \setlength{\parindent}{0pt}%
  \setlength{\parskip}{0pt plus 0.3pt}%
  \let\item\@idxitem
}{%
  \clearpage
}
\makeatother

\IfFileExists{\jobname-pw.ind}{\input{\jobname-pw.ind}}{}

\end{document}

      