%% latex-leseansicht-vorspann.tex
%% Vorspann für die Leseansicht.
%% Lädt die gemeinsame Datei latex-vorspann.tex mit nicht gesetztem Schalter.

\newif\ifkorrekturansicht
\korrekturansichtfalse

\input{../tex-inputs/latex-vorspann}


         
         \renewcommand{\erwaehntePersonen}{Personen: Stefanie Bachrach, Felix Salten, Olga Schnitzler, Oscar Wettstein}
         \renewcommand{\erwaehnteOrte}{Orte: Cottagegasse, Salzburg, Wien, XVIII., Währing}
         \renewcommand{\erwaehnteWerke}{}
               \section[ Arthur Schnitzler an Felix Salten, 17. 5. 1917]{ Arthur Schnitzler an Felix Salten, 17. 5. 1917}\nopagebreak\mylabel{v}\rehead{ }\begin{ledgroupsized}[t]{13cm}\normalsize\beginnumbering\briefempfaengerindex{Salten, Felix@\textsc{Salten, Felix}!zzzSchnitzler, Arthur@\emph{von Arthur Schnitzler}!1917-05-171@{17. 5. 1917}|(be} \toendnotes[C]{\smallbreak\pagebreak[2]} \Standort{Wienbibliothek im Rathaus, ZPH 1681, 2.1.516.}
\physDesc{Kartenbrief, 502 Zeichen
\newline{}Handschrift: 1) schwarze Tinte, lateinische Kurrent\hspace{1em}2) schwarze Tinte, deutsche Kurrent (\noindent{}Adresse)\hspace{1em}
\newline{}Ordnung: mit Bleistift von unbekannter Hand nummeriert: »8« }\toendnotes[C]{\smallbreak}\pstart{}{\pb}Herrn \textsc{Felix
                     Salten}.\pend{}\pstart{}Wien XVIII\oindex{XVIII., Waehring@\textbf{XVIII., Währing}|pw}\pend{}\pstart{}\textsc{Cottagegasse 37}\oindex{Cottagegasse@\textbf{Cottagegasse}|pw}.\pend{}{\bigskip}\pstart
           \raggedleft{}{\pb}17/5 17\textcolor{gray}{.}\pend
           \pstart
           \raggedleft{}früh.\pend
           \pstart
           lieber, von einer \label{K_L03019-1v}\edtext{Reise (Salzburg\oindex{Salzburg@\textbf{Salzburg}|pw})}{\lemma{\textnormal{\emph{Reise (Salzburg)}}}\Cendnote{\textnormal{Arthur\pwindex{Schnitzler, Arthur 15.05.1862 – 21.10.1931@\textsc{Schnitzler, Arthur} (15.05.1862 – 21.10.1931), \emph{Schriftsteller, Mediziner}|pwk} und Olga Schnitzler\pwindex{Schnitzler, Olga 17.01.1882 – 13.01.1970@\textsc{Schnitzler, Olga} (17.01.1882 – 13.01.1970), \emph{Schauspielerin, Sängerin}|pwk} waren am 14. 5. 1917 nach Salzburg\oindex{Salzburg@\textbf{Salzburg}|pwk} abgereist. Am 16. 5. 1917 fuhren sie zurück nach Wien\oindex{Wien@\textbf{Wien}|pwk}.}}}\label{K_L03019-1h} heimgekehrt, die durch die
               Nachricht vom \label{K_L03019-2v}\edtext{Tode unsrer Freundin
                  Stephi Bachrach\pwindex{Bachrach, Stefanie 22.05.1887 – 16.05.1917@\textsc{Bachrach, Stefanie} (22.05.1887 – 16.05.1917), \emph{Krankenpflegerin}|pw}}{\lemma{\textnormal{\emph{Tode … Bachrach}}}\Cendnote{\textnormal{Stefanie Bachrach\pwindex{Bachrach, Stefanie 22.05.1887 – 16.05.1917@\textsc{Bachrach, Stefanie} (22.05.1887 – 16.05.1917), \emph{Krankenpflegerin}|pwk} war am 16. 5. 1917
                  verstorben.}}}\label{K_L03019-2h} jäh unterbrochen wurde, finde ich Ihre freundliche Einladg zu
               dem Wettstein\pwindex{Wettstein, Oscar 1866-03-26 – 1952-02-16@\textsc{Wettstein, Oscar} (1866-03-26 – 1952-02-16), \emph{Politiker, Journalist, Jurist}|pw} Souper und
                  bitte Sie zugleich mein Fernbleiben mit Rücksicht auf diesen
               Trauerfall zu entschuldigen, der mich sehr tief bewegt.\pend
           \pstart
           Die Einladg zu dem Vortrag, auf die Sie sich beziehen, ist übrigens nicht an mich
               gelangt.\pend
           \pstart
           Auf Wiedersehen und herzlichen Dank. {\\[\baselineskip]}Ihr {\\[\baselineskip]}\spacefill\mbox{Arth Sch}\pend
           \leftskip=0em{}
         
         \endnumbering\mylabel{h}\end{ledgroupsized}  \newcommand{\dateiname}{L03019}\newcommand{\titel}{Arthur Schnitzler an Felix Salten, 17. 5. 1917}\newcommand{\editorInnen}{Martin Anton Müller und Laura Untner}%% latex-leseansicht-abspann.tex
%% Abspann für die Leseansicht.
%% Der Schalter \ifkorrekturansicht ist bereits durch den Vorspann gesetzt.

%% latex-abspann.tex
%% Gemeinsamer Abspann für Korrekturansicht und Leseansicht.
%% Setzt den Schalter \ifkorrekturansicht voraus (gesetzt in den
%% einbindenden Dateien latex-korrekturansicht-abspann.tex bzw.
%% latex-leseansicht-abspann.tex).
%% ---------------------------------------------------------------

\normalsize

% Das esempio-Environment wird nur in der Leseansicht benötigt
\ifkorrekturansicht\else
\newenvironment{esempio}[3]%
{
    \vspace{1.5ex}
    \rlap{\underline{#1}}
    \par
    \setlength{\parindent}{0cm}
    \nopagebreak
    \leftskip=#2cm
    \rightskip=#3cm
}
{
    \par
}
\fi

\doendnotes{C}
\bigskip
\vfill

\clearpage

\footnotesize

\ifkorrekturansicht
  \lohead{\textsc{register}}
\fi

% theindex-Environment neu definieren ohne reledmac
\makeatletter
\renewenvironment{theindex}{%
  \ifkorrekturansicht
    \section*{\indexname}%
  \else
    \subsubsection*{Index der erwähnten Entitäten}%
  \fi
  \setlength{\parindent}{0pt}%
  \setlength{\parskip}{0pt plus 0.3pt}%
  \let\item\@idxitem
}{%
  \ifkorrekturansicht\clearpage\fi
}
\makeatother

\IfFileExists{\jobname-pw.ind}{\input{\jobname-pw.ind}}{}

% Quellenangabe nur in der Leseansicht
\ifkorrekturansicht\else
% Fallback-Definitionen, falls die .tex-Datei \titel etc. nicht gesetzt hat
\providecommand{\titel}{}
\providecommand{\editorInnen}{}
\providecommand{\dateiname}{\jobname}

\vspace{3cm}

\vfill

\footnotesize
\textsc{Quelle}: \titel. Herausgegeben von {\editorInnen}. In: \emph{Arthur Schnitzler: Briefwechsel mit Autorinnen und Autoren}.
 Digitale Edition, https://schnitzler-briefe.acdh.oeaw.ac.at/{\dateiname}.html (Stand \today)
\fi

\end{document}


      