%% latex-leseansicht-vorspann.tex
%% Vorspann für die Leseansicht.
%% Lädt die gemeinsame Datei latex-vorspann.tex mit nicht gesetztem Schalter.

\newif\ifkorrekturansicht
\korrekturansichtfalse

\input{../tex-inputs/latex-vorspann}


               \section[Arthur Schnitzler an Hermann Bahr, 11. 11. 1903]{ Arthur Schnitzler an Hermann Bahr, 11. 11. 1903}\nopagebreak\mylabel{v}\rehead{ }\begin{ledgroupsized}[t]{13cm}\normalsize\beginnumbering\briefempfaengerindex{Bahr, Hermann@\textsc{Bahr, Hermann}!zzzSchnitzler, Arthur@\emph{von Arthur Schnitzler}!1903-11-111@{11. 11. 1903}|(be} \toendnotes[C]{\smallbreak\pagebreak[2]} \Standort{TMW, HS AM 23360 Ba.}
\physDesc{Kartenbrief
\newline{}Handschrift: schwarze Tinte, deutsche Kurrent\newline{}Versand: 1) Stempel: »\nobreak{}\oindex{XVIII., Waehring@\textbf{XVIII., Währing}|pwk}18/1 Wien, 11. 11. 03, 11–12 V\nobreak{}«.  2) Stempel: »\nobreak{}\oindex{XIII., Hietzing@\textbf{XIII., Hietzing}|pwk}Wien 13/7, 11. 11. 03\nobreak{}«. }\buchAbdrucke{\weitereDrucke{1) \emph{11. 11. 1903.} In: Arthur Schnitzler: \emph{The Letters of Arthur Schnitzler to Hermann Bahr}. Edited, annotated, and with an introduction, by Donald G.
                        Daviau. Chapel Hill: \emph{The University of North Carolina Press} 1978, S. 82 (University of North Carolina studies in the Germanic languages
                        and literatures, 89).} \weitereDrucke{2) Hermann Bahr, Arthur Schnitzler: \emph{Briefwechsel, Aufzeichnungen, Dokumente (1891–1931)}. Hg. Kurt Ifkovits und Martin Anton Müller. Göttingen: \emph{Wallstein} 2018, S. 280.} }\toendnotes[C]{\smallbreak}\pstart{}{\pb}Herrn Hermann Bahr\pend{}\pstart{}Wien Ob St Veit\oindex{Ober Sankt Veit@\textbf{Ober Sankt Veit}|pw}\pend{}\pstart{}Veitliſſngaſſe.\oindex{Veitlissengasse@\textbf{Veitlissengasse}|pw}\pend{}{\bigskip}\pstart
           {\pb}11. 11. 903.\pend
           \pstart
           lieber Hermann, ich habe mich gleich an Julius\pwindex{Schnitzler, Julius 13.07.1865 – 29.06.1939@\textsc{Schnitzler, Julius} (13.07.1865 – 29.06.1939), \emph{Chirurg}|pw} gewandt, da mir diese Titelſache ſelbſt nicht
               erinnerlich iſt; er wird dir wohl gleich direct antworten.\pend
           \pstart
           In einem \label{K_L01341_1v}\edtext{Brief von Brahm\pwindex{Brahm, Otto 05.02.1856 – 28.11.1912@\textsc{Brahm, Otto} (05.02.1856 – 28.11.1912), \emph{Theaterleiter, Regisseur}|pw}}{\lemma{\textnormal{\emph{Brief von Brahm}}}\Cendnote{\textnormal{Brief vom 8. 11. 1903
                  (\emph{Briefwechsel}
                     Schnitzler/Brahm
                     152–153).}}}\label{K_L01341_1h}, der vorgeſtern anlangte, ist von einem \uline{Termin} meines Stückes\pwindex{Schnitzler, Arthur 15.05.1862 – 21.10.1931@\textsc{Schnitzler, Arthur} (15.05.1862 – 21.10.1931), \emph{Schriftsteller, Mediziner}!einsame Weg. Schauspiel in fuenf Akten1904@\strich\emph{Der einsame Weg. Schauspiel in fünf Akten} {[}1904{]}|pwv} noch nicht die Rede; er ſchreibt mir nur die
               Beſetzung und will alles nähere nächſte Woche \uline{mündlich} mit mir beſprechen\introOben{}·\introOben{}) Er kommt, (was
               vielleicht noch niemand wiſſen ſoll?) zum \label{K_L01341_2v}\edtext{Fulda\pwindex{\textcolor{red}{\textsuperscript{XXXX1 indx}}!Novella DAndrea1903@\strich\emph{Novella d’Andrea} {[}1903{]}|pwv}}{\lemma{\textnormal{\emph{Fulda}}}\Cendnote{\textnormal{Uraufführung von \emph{Novella d’Andrea}\pwindex{\textcolor{red}{\textsuperscript{XXXX1 indx}}!Novella DAndrea1903@\strich\emph{Novella d’Andrea} {[}1903{]}|pwk} am 21. 11. 1903 im Burgtheater\oindex{Burgtheater@\textbf{Burgtheater}|pwk}}}}\label{K_L01341_2h} her. Nach dem Telegr\damage{a{\geminationm}} an dich zu ſchließen, dürfteſt du wohl \uline{vor
                  mir}, etwa Anfang Dezember, dranko{\geminationm}en\pwindex{Bahr, Hermann 19.07.1863 – 15.01.1934@\textsc{Bahr, Hermann} (19.07.1863 – 15.01.1934), \emph{Schriftsteller, Kritiker}!Meister1903@\strich\emph{Der Meister} {[}1903{]}|pwv}?\pend
           \pstart Herzlichen Gruſs. Dein \spacefill\mbox{A.}\pend{}\pstart
           \noindent{}·) Auch einige (nicht beträchtliche) Aenderungen ſchlägt er vor.\pend
                     \endnumbering\briefempfaengerindex{Bahr, Hermann@\textsc{Bahr, Hermann}!zzzSchnitzler, Arthur@\emph{von Arthur Schnitzler}!1903-11-111@{11. 11. 1903}|)be}\mylabel{h}\end{ledgroupsized}  \newcommand{\dateiname}{L01341}\newcommand{\titel}{Arthur Schnitzler an Hermann Bahr, 11. 11. 1903}\newcommand{\editorInnen}{ Kurt Ifkovits,  Martin Anton Müller}
            \footnotesize
\begin{ledgroupsized}[t]{11.5cm}
\doendnotes{C}
\end{ledgroupsized}
         %% latex-leseansicht-abspann.tex
%% Abspann für die Leseansicht.
%% Der Schalter \ifkorrekturansicht ist bereits durch den Vorspann gesetzt.

%% latex-abspann.tex
%% Gemeinsamer Abspann für Korrekturansicht und Leseansicht.
%% Setzt den Schalter \ifkorrekturansicht voraus (gesetzt in den
%% einbindenden Dateien latex-korrekturansicht-abspann.tex bzw.
%% latex-leseansicht-abspann.tex).
%% ---------------------------------------------------------------

\normalsize

% Das esempio-Environment wird nur in der Leseansicht benötigt
\ifkorrekturansicht\else
\newenvironment{esempio}[3]%
{
    \vspace{1.5ex}
    \rlap{\underline{#1}}
    \par
    \setlength{\parindent}{0cm}
    \nopagebreak
    \leftskip=#2cm
    \rightskip=#3cm
}
{
    \par
}
\fi

\doendnotes{C}
\bigskip
\vfill

\clearpage

\footnotesize

\ifkorrekturansicht
  \lohead{\textsc{register}}
\fi

% theindex-Environment neu definieren ohne reledmac
\makeatletter
\renewenvironment{theindex}{%
  \ifkorrekturansicht
    \section*{\indexname}%
  \else
    \subsubsection*{Index der erwähnten Entitäten}%
  \fi
  \setlength{\parindent}{0pt}%
  \setlength{\parskip}{0pt plus 0.3pt}%
  \let\item\@idxitem
}{%
  \ifkorrekturansicht\clearpage\fi
}
\makeatother

\IfFileExists{\jobname-pw.ind}{\input{\jobname-pw.ind}}{}

% Quellenangabe nur in der Leseansicht
\ifkorrekturansicht\else
% Fallback-Definitionen, falls die .tex-Datei \titel etc. nicht gesetzt hat
\providecommand{\titel}{}
\providecommand{\editorInnen}{}
\providecommand{\dateiname}{\jobname}

\vspace{3cm}

\vfill

\footnotesize
\textsc{Quelle}: \titel. Herausgegeben von {\editorInnen}. In: \emph{Arthur Schnitzler: Briefwechsel mit Autorinnen und Autoren}.
 Digitale Edition, https://schnitzler-briefe.acdh.oeaw.ac.at/{\dateiname}.html (Stand \today)
\fi

\end{document}


      