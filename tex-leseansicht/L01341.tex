%% latex-leseansicht-vorspann.tex
%% Vorspann für die Leseansicht.
%% Lädt die gemeinsame Datei latex-vorspann.tex mit nicht gesetztem Schalter.

\newif\ifkorrekturansicht
\korrekturansichtfalse

\input{../tex-inputs/latex-vorspann}


\section[Arthur Schnitzler an Hermann Bahr, 11. 11. 1903]{L01341 Arthur Schnitzler an Hermann Bahr, 11. 11. 1903}
\nopagebreak\mylabel{L01341v}
\rehead{ }\normalsize\beginnumbering\briefempfaengerindex{Bahr, Hermann@\textsc{Bahr, Hermann}!zzzSchnitzler, Arthur@\emph{von Arthur Schnitzler}!1903-11-111@{11. 11. 1903}|(be}
\toendnotes[C]{\smallbreak\pagebreak[2]}
\correspDesc{Versand  durch Arthur Schnitzler am 11. 11. 1903 in Wien
\newline{}Erhalt  durch Hermann Bahr im Zeitraum [11. 11. 1903 – 15. 11. 1903?] in Wien}\toendnotes[C]{\smallbreak}
\Standort{TMW, HS AM 23360 Ba.}
\physDesc{Kartenbrief, 590 Zeichen
\newline{}Handschrift: schwarze Tinte, deutsche Kurrent
\newline{}Versand: 1) Stempel: »\nobreak{}\oindex{XVIII., Währing@\textbf{XVIII., Währing}, \emph{Verwaltungsgebiet}|pwk}18/1 Wien, 11. 11. 03, 11–12 V\nobreak{}«.   2) Stempel: »\nobreak{}\oindex{XIII., Hietzing@\textbf{XIII., Hietzing}, \emph{Verwaltungsgebiet}|pwk}Wien 13/7, 11. 11. 03\nobreak{}«. }
\buchAbdrucke{\weitereDrucke{1) \emph{11. 11. 1903.} In: Arthur Schnitzler: \emph{The Letters of Arthur Schnitzler to Hermann Bahr}. Edited, annotated, and with an introduction, by Donald G. Daviau. Chapel Hill: \emph{The University of North Carolina Press} 1978, S. 82 (University of North Carolina studies in the Germanic languages
                        and literatures, 89).} \weitereDrucke{2) Hermann Bahr, Arthur Schnitzler: \emph{Briefwechsel, Aufzeichnungen, Dokumente (1891–1931)}. Herausgegeben von Kurt Ifkovits und Martin Anton Müller. Göttingen: \emph{Wallstein} 2018, S. 280.} }\toendnotes[C]{\smallbreak}\pstart{}{\pb}Herrn Hermann
                  Bahr\pend{}\pstart{}Wien Ob St Veit\oindex{Wien@\textbf{Wien}!XIII., Hietzing@\textbf{XIII., Hietzing}!Ober Sankt Veit@\textbf{Ober Sankt Veit}, \emph{Ehemaliger Ort}|pw}\pend{}\pstart{}Veitliſſengaſſe.\oindex{Wien@\textbf{Wien}!XIII., Hietzing@\textbf{XIII., Hietzing}!Veitlissengasse@\textbf{Veitlissengasse}, \emph{Straße}|pw}\pend{}{\bigskip}\vspace{1em}
\pstart
           {\pb}11. 11. 903.\pend
           \vspace{0.5em}
\pstart
           lieber Hermann, ich habe mich gleich an Julius\pwindex{Schnitzler, Julius 13.\,7.\,1865 Wien – 29.\,6.\,1939 ebd.@\textsc{Schnitzler, Julius} (13.\,7.\,1865 Wien – 29.\,6.\,1939 ebd.), \emph{Chirurg}|pw} gewandt, da mir diese Titelſache{ }ſelbſt nicht
               erinnerlich iſt; er wird dir wohl gleich direct antworten.\pend
           
\pstart
           In einem \label{K_L01341-1v}\edtext{Brief von Brahm\pwindex{Brahm, Otto 5.\,2.\,1856 Hamburg – 28.\,11.\,1912 Berlin@\textsc{Brahm, Otto} (5.\,2.\,1856 Hamburg – 28.\,11.\,1912 Berlin), \emph{Theaterleiter, Regisseur}|pw}}{\lemma{\textnormal{\emph{Brief von Brahm}}}\Cendnote{\textnormal{Brief vom 8. 11. 1903 (\emph{Der Briefwechsel Arthur Schnitzler – Otto Brahm}.
                  Vollständige Ausgabe. Herausgegeben, eingeleitet und erläutert von Oskar
                  Seidlin. Tübingen: \emph{Niemeyer}{ }1975, S. 152–153).}}}\label{K_L01341-1},
               der vorgeſtern anlangte, ist von einem \uline{Termin} meines
                  Stückes\pwindex{Schnitzler, Arthur 15.\,5.\,1862 Wien – 21.\,10.\,1931 ebd.@\textsc{Schnitzler, Arthur} (15.\,5.\,1862 Wien – 21.\,10.\,1931 ebd.), \emph{Schriftsteller, Mediziner}!einsame Weg. Schauspiel in fünf Akten@\strich\emph{Der einsame Weg. Schauspiel in fünf Akten}|pwv} noch nicht die
               Rede; er{ }ſchreibt mir nur die Beſetzung und will alles nähere nächſte Woche \uline{mündlich} mit mir beſprechen\footnote{\noindent{}Auch einige (nicht beträchtliche) Aenderungen{ }ſchlägt er vor.} Er
               kommt, (was vielleicht noch niemand wiſſen{ }ſoll?) zum \label{K_L01341-2v}\edtext{Fulda\eventindex{Burgtheater@\textbf{Burgtheater}!Uraufführung von Novella d’Andrea, 21.11.1903@Uraufführung von Novella d’Andrea, 21.11.1903|pwv}}{\lemma{\textnormal{\emph{Fulda}}}\Cendnote{\textnormal{Uraufführung\eventindex{Burgtheater@\textbf{Burgtheater}!Uraufführung von Novella d’Andrea, 21.11.1903@Uraufführung von Novella d’Andrea, 21.11.1903|pwkv} von \emph{Novella d’Andrea}\pwindex{\textcolor{red}{\textsuperscript{XXXX indx1}}!Novella d’Andrea@\strich\emph{Novella d’Andrea}|pwk} am 21. 11. 1903 im Burgtheater\oindex{Wien@\textbf{Wien}!I., Innere Stadt@\textbf{I., Innere Stadt}!Burgtheater@\textbf{Burgtheater}, \emph{Theater}|pwk}}}}\label{K_L01341-2} her. Nach dem Telegr\damage{a{\geminationm}} an dich zu{ }ſchließen, dürfteſt du wohl \uline{vor
                  mir}, etwa Anfang Dezember, dranko{\geminationm}en\pwindex{Bahr, Hermann 19.\,7.\,1863 Linz – 15.\,1.\,1934 München@\textsc{Bahr, Hermann} (19.\,7.\,1863 Linz – 15.\,1.\,1934 München), \emph{Schriftsteller, Kritiker}!Meister. Komödie in drei Akten@\strich\emph{Der Meister. Komödie in drei Akten}|pwv}\pwindex{Bahr, Hermann 19.\,7.\,1863 Linz – 15.\,1.\,1934 München@\textsc{Bahr, Hermann} (19.\,7.\,1863 Linz – 15.\,1.\,1934 München), \emph{Schriftsteller, Kritiker}!Meister. Komödie in drei Akten@\strich\emph{Der Meister. Komödie in drei Akten}|pwv}?\pend
           \pstart Herzlichen Gruſs. Dein \spacefill\mbox{A.}\pend{}\selectlanguage{ngerman}\endnumbering\briefempfaengerindex{Bahr, Hermann@\textsc{Bahr, Hermann}!zzzSchnitzler, Arthur@\emph{von Arthur Schnitzler}!1903-11-111@{11. 11. 1903}|)be}\mylabel{L01341h}  \newcommand{\dateiname}{L01341}\newcommand{\titel}{Arthur Schnitzler an Hermann Bahr, 11. 11. 1903}\newcommand{\editorInnen}{Herausgegeben von Martin Anton Müller}%% latex-leseansicht-abspann.tex
%% Abspann für die Leseansicht.
%% Der Schalter \ifkorrekturansicht ist bereits durch den Vorspann gesetzt.

%% latex-abspann.tex
%% Gemeinsamer Abspann für Korrekturansicht und Leseansicht.
%% Setzt den Schalter \ifkorrekturansicht voraus (gesetzt in den
%% einbindenden Dateien latex-korrekturansicht-abspann.tex bzw.
%% latex-leseansicht-abspann.tex).
%% ---------------------------------------------------------------

\normalsize

% Das esempio-Environment wird nur in der Leseansicht benötigt
\ifkorrekturansicht\else
\newenvironment{esempio}[3]%
{
    \vspace{1.5ex}
    \rlap{\underline{#1}}
    \par
    \setlength{\parindent}{0cm}
    \nopagebreak
    \leftskip=#2cm
    \rightskip=#3cm
}
{
    \par
}
\fi

\doendnotes{C}
\bigskip
\vfill

\clearpage

\footnotesize

\ifkorrekturansicht
  \lohead{\textsc{register}}
\fi

% theindex-Environment neu definieren ohne reledmac
\makeatletter
\renewenvironment{theindex}{%
  \ifkorrekturansicht
    \section*{\indexname}%
  \else
    \subsubsection*{Index der erwähnten Entitäten}%
  \fi
  \setlength{\parindent}{0pt}%
  \setlength{\parskip}{0pt plus 0.3pt}%
  \let\item\@idxitem
}{%
  \ifkorrekturansicht\clearpage\fi
}
\makeatother

\IfFileExists{\jobname-pw.ind}{\input{\jobname-pw.ind}}{}

% Quellenangabe nur in der Leseansicht
\ifkorrekturansicht\else
% Fallback-Definitionen, falls die .tex-Datei \titel etc. nicht gesetzt hat
\providecommand{\titel}{}
\providecommand{\editorInnen}{}
\providecommand{\dateiname}{\jobname}

\vspace{3cm}

\vfill

\footnotesize
\textsc{Quelle}: \titel. Herausgegeben von {\editorInnen}. In: \emph{Arthur Schnitzler: Briefwechsel mit Autorinnen und Autoren}.
 Digitale Edition, https://schnitzler-briefe.acdh.oeaw.ac.at/{\dateiname}.html (Stand \today)
\fi

\end{document}


