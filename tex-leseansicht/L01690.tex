%% latex-korrekturansicht-vorspann.tex
%% Vorspann für die Korrekturansicht.
%% Lädt die gemeinsame Datei latex-vorspann.tex mit gesetztem Schalter.

\newif\ifkorrekturansicht
\korrekturansichttrue

\input{../tex-inputs/latex-vorspann}


\section[Hugo von Hofmannsthal an Arthur Schnitzler, 11. 7. 1907]{L01690 Hugo von Hofmannsthal an Arthur Schnitzler, 11. 7. 1907}
\nopagebreak\mylabel{L01690v}
\rehead{ }\normalsize\beginnumbering\briefempfaengerindex{Schnitzler, Arthur@\textsc{Schnitzler, Arthur}!zzzHofmannsthal, Hugo von@\emph{von Hugo von Hofmannsthal}!1907-07-112@{11. 7. 1907}|(be}
\toendnotes[C]{\smallbreak\pagebreak[2]}\Standort{CUL, Schnitzler, B 43.}
\physDesc{Postkarte, 247 Zeichen
\newline{}Handschrift: 1) schwarze Tinte, deutsche Kurrent\hspace{1em}2) schwarze Tinte, lateinische Kurrent (\noindent{}Adresse)\hspace{1em}
\newline{}Versand: 1) Stempel: »\nobreak{}\oindex{Cortina DAmpezzo@\textbf{Cortina d’Ampezzo}, \emph{P.PPLA3}|pwk}Cortina, 11. VII. 07\nobreak{}«.   2) Stempel: »\nobreak{}\oindex{Welsberg-Taisten@\textbf{Welsberg-Taisten}, \emph{A.ADM3}|pwk}Welsbe{[}rg{]}, \textcolor{gray}{12.}{[} 7. 1907{]}\nobreak{}«. 
\newline{}Schnitzler: mit Bleistift datiert: »11/7 90\textcolor{gray}{7}« 
\newline{}Ordnung: 1) mit Bleistift von unbekannter Hand nummeriert: »\strikeout{281}«  2) mit Bleistift von unbekannter Hand nummeriert:
                                    »283«}
\buchAbdrucke{\weitereDrucke{Hugo von Hofmannsthal, Arthur Schnitzler: \emph{Briefwechsel}. Frankfurt am Main: \emph{S. Fischer} 1964, S. 230.} }\toendnotes[C]{\smallbreak}\pstart{}{\pb}Herrn D\textsuperscript{r} Arthur Schnitzler\pend{}\pstart{}Wildbad Waldbrunn\oindex{Wildbad Waldbrunn@\textbf{Wildbad Waldbrunn}, \emph{S.SPA}|pw}\pend{}\pstart{}Welsberg\oindex{Welsberg-Taisten@\textbf{Welsberg-Taisten}, \emph{A.ADM3}|pw}\pend{}\pstart{}Pusterthal\oindex{Pustertal@\textbf{Pustertal}, \emph{T.VAL}|pw}\pend{}{\bigskip}\vspace{1em}
\pstart
           
\pstart
           {\pb}\textsc{Cort.}\oindex{Cortina DAmpezzo@\textbf{Cortina d’Ampezzo}, \emph{P.PPLA3}|pw}\pend
           
\pstart
           \raggedleft{}Donnerstag\pend
           \pend
           \vspace{0.5em}
\pstart
           Sie arbeiten von 2–5? Gut. Ich werde von \label{K_L01690-1v}\edtext{¼ 3 bis ¾ 5}{\lemma{\textnormal{\emph{¼ 3 bis ¾ 5}}}\Cendnote{\textnormal{von 14:15 Uhr bis 16:45 Uhr}}}\label{K_L01690-1}{ }arbeiten\pwindex{Silvia im »Stern«@\emph{Silvia im »Stern«}|pwv} und dafür das
               doppelte Honorar verlangen.\pend
           
\pstart
           Wir sind \uline{Sonntag}{ }1\textsuperscript{h}10 nachmittags bei Ihnen. Freuen uns
               ſehr.\pend
           
\pstart
           Von Herzen{\\[\baselineskip]}\spacefill\mbox{Hugo.}\pend
           \leftskip=0em{}\selectlanguage{ngerman}\endnumbering\briefempfaengerindex{Schnitzler, Arthur@\textsc{Schnitzler, Arthur}!zzzHofmannsthal, Hugo von@\emph{von Hugo von Hofmannsthal}!1907-07-112@{11. 7. 1907}|)be}\mylabel{L01690h}  \normalsize

\doendnotes{C}
\bigskip
\vfill

\clearpage

\footnotesize

\lohead{\textsc{register}}

% Definiere theindex-Environment komplett neu ohne reledmac
\makeatletter
\renewenvironment{theindex}{%
  \section*{\indexname}%
  \setlength{\parindent}{0pt}%
  \setlength{\parskip}{0pt plus 0.3pt}%
  \let\item\@idxitem
}{%
  \clearpage
}
\makeatother

\IfFileExists{\jobname-pw.ind}{\input{\jobname-pw.ind}}{}

\end{document}

      