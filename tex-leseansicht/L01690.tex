%% latex-leseansicht-vorspann.tex
%% Vorspann für die Leseansicht.
%% Lädt die gemeinsame Datei latex-vorspann.tex mit nicht gesetztem Schalter.

\newif\ifkorrekturansicht
\korrekturansichtfalse

\input{../tex-inputs/latex-vorspann}


               \section[Hugo von Hofmannsthal an Arthur Schnitzler, 11. 7. 1907]{ Hugo von Hofmannsthal an Arthur Schnitzler, 11. 7. 1907}\nopagebreak\mylabel{v}\rehead{ }\begin{ledgroupsized}[t]{13cm}\normalsize\beginnumbering\briefempfaengerindex{Schnitzler, Arthur@\textsc{Schnitzler, Arthur}!zzzHofmannsthal, Hugo von@\emph{von Hugo von Hofmannsthal}!1907-07-112@{11. 7. 1907}|(be} \toendnotes[C]{\smallbreak\pagebreak[2]} \Standort{CUL, Schnitzler, B 43.}
\physDesc{Postkarte
\newline{}Handschrift: schwarze Tinte, deutsche Kurrent\newline{}Versand: 1) Stempel: »\nobreak{}\oindex{Cortina d'Ampezzo@\textbf{Cortina d'Ampezzo}|pwk}Cortina, 11. VII. 07\nobreak{}«.  2) Stempel: »\nobreak{}\oindex{Welsberg-Taisten@\textbf{Welsberg-Taisten}|pwk}Welsbe{[}rg{]}, \textcolor{gray}{12.}{[} 7. 1907{]}\nobreak{}«. 
\newline{}Schnitzler: mit Bleistift datiert: »11/7 90\textcolor{gray}{7}« \newline{}Ordnung: 1) mit Bleistift von unbekannter Hand nummeriert: »\strikeout{281}« 2) mit Bleistift von unbekannter Hand nummeriert:
                                    »283«}\buchAbdrucke{\weitereDrucke{Hugo von Hofmannsthal, Arthur Schnitzler: \emph{Briefwechsel}. Hg. Therese Nickl und Heinrich Schnitzler. Frankfurt am Main: \emph{S. Fischer} 1964, S. 230.} }\toendnotes[C]{\smallbreak}\pstart{}{\pb}\textsc{Herrn D\textsuperscript{r} Arthur Schnitzler}\pend{}\pstart{}\textsc{Wildbad Waldbrunn\oindex{Wildbad Waldbrunn@\textbf{Wildbad Waldbrunn}|pw}}\pend{}\pstart{}\textsc{Welsberg\oindex{Welsberg-Taisten@\textbf{Welsberg-Taisten}|pw}}\pend{}\pstart{}\textsc{Pusterthal\oindex{Pustertal@\textbf{Pustertal}|pw}}\pend{}{\bigskip}\pstart
           {\pb}\textsc{Cort.}\oindex{Cortina d'Ampezzo@\textbf{Cortina d'Ampezzo}|pw}\hfill Donnerstag\pend
           \pstart
           Sie arbeiten von 2–5? Gut. Ich werde von \label{K_L01690_1v}\edtext{¼ 3 bis ¾ 5}{\lemma{\textnormal{\emph{¼ 3 bis ¾ 5}}}\Cendnote{\textnormal{von 14:15 Uhr bis 16:45 Uhr}}}\label{K_L01690_1h}{ }arbeiten\pwindex{Hofmannsthal, Hugo von 01.02.1874 – 15.07.1929@\textsc{Hofmannsthal, Hugo von} (01.02.1874 – 15.07.1929), \emph{Schriftsteller}!Silvia im »Stern«1909@\strich\emph{Silvia im »Stern«} {[}1909{]}|pwv} und dafür das doppelte
               Honorar verlangen.\pend
           \pstart
           Wir sind \uline{Sonntag}{ }1\textsuperscript{h}10 nachmittags bei Ihnen. Freuen uns
               ſehr.\pend
           \pstart
           Von Herzen{\\[\baselineskip]}\spacefill\mbox{Hugo.}\pend
           \leftskip=0em{}\endnumbering\briefempfaengerindex{Schnitzler, Arthur@\textsc{Schnitzler, Arthur}!zzzHofmannsthal, Hugo von@\emph{von Hugo von Hofmannsthal}!1907-07-112@{11. 7. 1907}|)be}\mylabel{h}\end{ledgroupsized}  \newcommand{\dateiname}{L01690}\newcommand{\titel}{Hugo von Hofmannsthal an Arthur Schnitzler, 11. 7. 1907}\newcommand{\editorInnen}{Martin Anton Müller und Gerd-Hermann Susen}
            \footnotesize
\begin{ledgroupsized}[t]{11.5cm}
\doendnotes{C}
\end{ledgroupsized}
         %% latex-leseansicht-abspann.tex
%% Abspann für die Leseansicht.
%% Der Schalter \ifkorrekturansicht ist bereits durch den Vorspann gesetzt.

%% latex-abspann.tex
%% Gemeinsamer Abspann für Korrekturansicht und Leseansicht.
%% Setzt den Schalter \ifkorrekturansicht voraus (gesetzt in den
%% einbindenden Dateien latex-korrekturansicht-abspann.tex bzw.
%% latex-leseansicht-abspann.tex).
%% ---------------------------------------------------------------

\normalsize

% Das esempio-Environment wird nur in der Leseansicht benötigt
\ifkorrekturansicht\else
\newenvironment{esempio}[3]%
{
    \vspace{1.5ex}
    \rlap{\underline{#1}}
    \par
    \setlength{\parindent}{0cm}
    \nopagebreak
    \leftskip=#2cm
    \rightskip=#3cm
}
{
    \par
}
\fi

\doendnotes{C}
\bigskip
\vfill

\clearpage

\footnotesize

\ifkorrekturansicht
  \lohead{\textsc{register}}
\fi

% theindex-Environment neu definieren ohne reledmac
\makeatletter
\renewenvironment{theindex}{%
  \ifkorrekturansicht
    \section*{\indexname}%
  \else
    \subsubsection*{Index der erwähnten Entitäten}%
  \fi
  \setlength{\parindent}{0pt}%
  \setlength{\parskip}{0pt plus 0.3pt}%
  \let\item\@idxitem
}{%
  \ifkorrekturansicht\clearpage\fi
}
\makeatother

\IfFileExists{\jobname-pw.ind}{\input{\jobname-pw.ind}}{}

% Quellenangabe nur in der Leseansicht
\ifkorrekturansicht\else
% Fallback-Definitionen, falls die .tex-Datei \titel etc. nicht gesetzt hat
\providecommand{\titel}{}
\providecommand{\editorInnen}{}
\providecommand{\dateiname}{\jobname}

\vspace{3cm}

\vfill

\footnotesize
\textsc{Quelle}: \titel. Herausgegeben von {\editorInnen}. In: \emph{Arthur Schnitzler: Briefwechsel mit Autorinnen und Autoren}.
 Digitale Edition, https://schnitzler-briefe.acdh.oeaw.ac.at/{\dateiname}.html (Stand \today)
\fi

\end{document}


      