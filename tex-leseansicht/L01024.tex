%% latex-leseansicht-vorspann.tex
%% Vorspann für die Leseansicht.
%% Lädt die gemeinsame Datei latex-vorspann.tex mit nicht gesetztem Schalter.

\newif\ifkorrekturansicht
\korrekturansichtfalse

\input{../tex-inputs/latex-vorspann}


\section[Arthur Schnitzler an Hugo von Hofmannsthal, 23. 3. 1900]{L01024 Arthur Schnitzler an Hugo von Hofmannsthal, 23. 3. 1900}
\nopagebreak\mylabel{L01024v}
\rehead{ }\normalsize\beginnumbering\briefempfaengerindex{Hofmannsthal, Hugo von@\textsc{Hofmannsthal, Hugo von}!zzzSchnitzler, Arthur@\emph{von Arthur Schnitzler}!1900-03-231@{23. 3. 1900}|(be}
\toendnotes[C]{\smallbreak\pagebreak[2]}
\correspDesc{Versand  durch Arthur Schnitzler am 23. 3. 1900 in Wien
\newline{}Erhalt  durch Hugo von Hofmannsthal im Zeitraum [24. 3. 1900
                  – 28. 3. 1900?] in Paris}\toendnotes[C]{\smallbreak}
\Standort{FDH, Hs-30885,90.}
\physDesc{Brief, 2 Blätter, 8 Seiten, 2415 Zeichen
\newline{}Handschrift: schwarze Tinte, deutsche Kurrent
\newline{}Ordnung: mit Bleistift von Schnitzler mutmaßlich bei der
                                 Durchsicht der Korrespondenz 1929 auch das zweite Blatt mit »23. 3. 900.« datiert }
\buchAbdrucke{\weitereDrucke{Hugo von Hofmannsthal, Arthur Schnitzler: \emph{Briefwechsel}. Herausgegeben von Therese Nickl und Heinrich Schnitzler. Frankfurt am Main: \emph{S. Fischer} 1964, S. 135–136.} }\toendnotes[C]{\smallbreak}
\pstart
           \raggedleft{}{\pb}23. 3. 900.\pend
           \vspace{0.5em}
\pstart
           mein lieber Hugo, Sie haben mich recht lang warten laſſen, aber was
               Sie mir{ }ſchreiben iſt alles erfreulich und{ }ſchön, und so hab ich es erwartet. Der
               kleine Ort heißt \textsc{Vilennes}\oindex{Villennes-sur-Seine@\textbf{Villennes-sur-Seine}|pw} oder \textsc{Vilaines}\oindex{Villennes-sur-Seine@\textbf{Villennes-sur-Seine}|pw} – bei \textsc{Poissy}\oindex{Poissy@\textbf{Poissy}|pw}, we{\geminationn} mich nicht die Erinnerg trügt, an der
               \textsc{Marne}\oindex{Marne@\textbf{Marne}, \emph{Fluss}|pw}. Ich ka{\geminationn} nie an jene Stunde zurückdenken,
               ohne daſs{ }ſich mein ganzes Weſen mit einem unbegreiflichen Schauer füllt,{ }ſo als we{\geminationn} ich dort es eigentlich{ }ſchon hätte wiſſen müssen – {\pb}oder gar – es gewußt hätte – (»dort – wo wir an lichten Tagen nicht
                  hineinſchaun!\pwindex{Schnitzler, Arthur 15.\,5.\,1862 Wien – 21.\,10.\,1931 ebd.@\textsc{Schnitzler, Arthur} (15.\,5.\,1862 Wien – 21.\,10.\,1931 ebd.), \emph{Schriftsteller, Mediziner}!Schleier der Beatrice. Schauspiel in fünf Akten@\strich\emph{Der Schleier der Beatrice. Schauspiel in fünf Akten}|pwv}«) – Ihr Brief kam grad am Morgen des \label{K_L01024-1v}\edtext{18. März}{\lemma{\textnormal{\emph{18. März}}}\Cendnote{\textnormal{Maria Reinhards\pwindex{Reinhard, Marie 13.\,3.\,1871 Wien – 18.\,3.\,1899 ebd.@\textsc{Reinhard, Marie} (13.\,3.\,1871 Wien – 18.\,3.\,1899 ebd.), \emph{Gesangspädagogin}|pwk} erster
               Todestag.}}}\label{K_L01024-1}. –\pend
           
\pstart
           Ihr kleines Vorſpiel\pwindex{Hofmannsthal, Hugo von 1.\,2.\,1874 Wien – 15.\,7.\,1929 Rodaun@\textsc{Hofmannsthal, Hugo von} (1.\,2.\,1874 Wien – 15.\,7.\,1929 Rodaun), \emph{Schriftsteller}!Vorspiel zur Antigone des Sophokles@\strich\emph{Vorspiel zur Antigone des Sophokles}|pwv}, das ich{ }ſehr einfach und{ }ſchön finde, hab ich gleich an Paul Goldma{\geminationn}\pwindex{Goldmann, Paul 31.\,1.\,1865 Breslau – 25.\,9.\,1935 Wien@\textsc{Goldmann, Paul} (31.\,1.\,1865 Breslau – 25.\,9.\,1935 Wien), \emph{Schriftsteller, Journalist}|pw} (\textsc{Berlin}\oindex{Berlin@\textbf{Berlin}, \emph{Hauptstadt}|pw}, \textsc{Dessauer}ſtraße 19\oindex{Dessauer Straße@\textbf{Dessauer Straße}, \emph{Straße}|pw}) geſchickt, vielleicht{ }ſchreiben
               Sie ihm auch ein Wort?\pend
           
\pstart
           – Wir leben hier noch im ewigen {\pb}Winter. Schnee heut
               Nacht! – Und Wind, Regen, Koth. Es ist abſcheulich. Ich will in den nächſten Tagen
               ein bischen in den Süden fahren, bis Raguſa\oindex{Dubrovnik@\textbf{Dubrovnik}|pw}.
               Nicht mit rechter Freude. Aber ich hab auch i{\geminationm}er
               Katarrhe, jetzt noch dazu dumme Geſchichten mit plombirten Zähnen, dazu alles andre,
               kurz, ich ka{\geminationn}{ }{\pb}mich kaum je eine viertel Stunde wohl fühlen.
                  Anfang März war ich ein paar Tage in Edlach\oindex{Edlach@\textbf{Edlach}|pw}; habe dort den Frühling finden wollen, aber Eis und 10 Grad Kälte,{ }ſowie \textsc{Dora Speyer}\pwindex{Michaelis, Dora 23.\,5.\,1881 Wien – 22.\,1.\,1946 New York City@\textsc{Michaelis, Dora} (23.\,5.\,1881 Wien – 22.\,1.\,1946 New York City)|pw} gefunden, die übrigens lieb iſt.\pend
           
\pstart
           – Jetzt iſt \textsc{Brandes}\pwindex{Brandes, Georg 4.\,2.\,1842 Kopenhagen – 19.\,2.\,1927 ebd.@\textsc{Brandes, Georg} (4.\,2.\,1842 Kopenhagen – 19.\,2.\,1927 ebd.)|pw} hier, erzählt{ }ſehr amüſant, und iſt gewiſs was{ }ſehr beſondres. Und {\pb}doch (warum »und doch«?) hab ich eher ein Gefühl der
               Entfremdung diesmal ihm gegenüber. Liegt wohl an meiner Sti{\geminationm}ung. –\pend
           
\pstart
           Ich arbeite an nichts als an der langen Novelle\pwindex{Schnitzler, Arthur 15.\,5.\,1862 Wien – 21.\,10.\,1931 ebd.@\textsc{Schnitzler, Arthur} (15.\,5.\,1862 Wien – 21.\,10.\,1931 ebd.), \emph{Schriftsteller, Mediziner}!Frau Bertha Garlan. Roman@\strich\emph{Frau Bertha Garlan. Roman}|pwv}, die wohl (ſtofflich){ }ſo eine Art Seitenstück zur
                  \textsc{Femme de 30 ans}\pwindex{\textcolor{red}{\textsuperscript{XXXX indx1}}!Eine Frau von dreißig Jahren@\strich\emph{Eine Frau von dreißig Jahren}|pw} wird, eine \label{K_L01024-2v}\edtext{\textsc{veuve de 30 ans}}{\lemma{\textnormal{\emph{veuve de 30 ans}}}\Cendnote{\textnormal{französisch: Witwe von 30 Jahren}}}\label{K_L01024-2}
               – viel{\pb}leicht{ }ſchließ ich{ }ſie auf der dalmatiniſchen\oindex{Dalmatien@\textbf{Dalmatien}, \emph{Ehemalige Region}|pw} Küſtenfahrt ab. –\pend
           
\pstart
           Eben telephonirt mir Richard\pwindex{Beer-Hofmann, Richard 11.\,7.\,1866 Wien – 26.\,9.\,1945 New York City@\textsc{Beer-Hofmann, Richard} (11.\,7.\,1866 Wien – 26.\,9.\,1945 New York City), \emph{Schriftsteller}|pw} ich möge in den
                  Schachclub\orgindex{Wiener Schachclub@Wiener Schachclub|pw} ko{\geminationm}en
               – Iſt das nicht ganz unwahrſcheinlich in Paris\oindex{Paris@\textbf{Paris}, \emph{Hauptstadt}|pw} zu
               hören, daſs hier weiter telephonirt wird – in den Schachclub\orgindex{Wiener Schachclub@Wiener Schachclub|pw} gegangen –? So iſt es mir gewiſſermaßen räthſelhaft, daſs gewiſs
               das Haus {\pb}in der \textsc{rue Maubeuge Nr. 5}\oindex{5, rue de Maubeuge@\textbf{5, rue de Maubeuge}, \emph{Wohngebäude}|pw}{ }ſteht – ja daſs noch die Zi{\geminationm}er exiſtiren, die Fenſter – die Waſchtiſche – –\pend
           
\pstart
           Ich ka{\geminationn} Ihnen gar nicht{ }ſagen wie mir iſt, während ich
               dieſen Brief ende. Als hätt ich’s noch i{\geminationm}er nicht ganz
                  verſtanden\pwindex{Reinhard, Marie 13.\,3.\,1871 Wien – 18.\,3.\,1899 ebd.@\textsc{Reinhard, Marie} (13.\,3.\,1871 Wien – 18.\,3.\,1899 ebd.), \emph{Gesangspädagogin}|pwv} – denn in
               dieſem Augenblick{ }ſind mir Dinge eingefallen, an die ich{ }ſeitdem nicht gedacht.\pend
           
\pstart
           {\pb}leben Sie wohl. Wann kommen Sie wieder? Werden wir
                  zuſa{\geminationm}en radeln? Ich bin neugierig auf das, was Sie
               mir von den Namenloſen erzählen werden.\pend
           
\pstart
           Von Herzen{\\[\baselineskip]}Ihr{\\[\baselineskip]}\spacefill\mbox{Arthur.}\pend
           \leftskip=0em{}
\pstart
           Grüßen Sie Hans Schleſinger\pwindex{Schlesinger, Hans Bernhard 20.\,7.\,1875 Wien – 13.\,3.\,1932 Salzburg@\textsc{Schlesinger, Hans Bernhard} (20.\,7.\,1875 Wien – 13.\,3.\,1932 Salzburg), \emph{Maler}|pw} u. Bubi Franckenſtein\pwindex{Franckenstein, Georg von 18.\,3.\,1878 Dresden – 14.\,10.\,1953 Kelsterbach@\textsc{Franckenstein, Georg von} (18.\,3.\,1878 Dresden – 14.\,10.\,1953 Kelsterbach), \emph{Diplomat}|pw}.\pend
           \selectlanguage{ngerman}\endnumbering\briefempfaengerindex{Hofmannsthal, Hugo von@\textsc{Hofmannsthal, Hugo von}!zzzSchnitzler, Arthur@\emph{von Arthur Schnitzler}!1900-03-231@{23. 3. 1900}|)be}\mylabel{L01024h}  \newcommand{\dateiname}{L01024}\newcommand{\titel}{Arthur Schnitzler an Hugo von Hofmannsthal, 23. 3. 1900}\newcommand{\editorInnen}{Martin Anton Müller und Gerd-Hermann Susen}%% latex-leseansicht-abspann.tex
%% Abspann für die Leseansicht.
%% Der Schalter \ifkorrekturansicht ist bereits durch den Vorspann gesetzt.

%% latex-abspann.tex
%% Gemeinsamer Abspann für Korrekturansicht und Leseansicht.
%% Setzt den Schalter \ifkorrekturansicht voraus (gesetzt in den
%% einbindenden Dateien latex-korrekturansicht-abspann.tex bzw.
%% latex-leseansicht-abspann.tex).
%% ---------------------------------------------------------------

\normalsize

% Das esempio-Environment wird nur in der Leseansicht benötigt
\ifkorrekturansicht\else
\newenvironment{esempio}[3]%
{
    \vspace{1.5ex}
    \rlap{\underline{#1}}
    \par
    \setlength{\parindent}{0cm}
    \nopagebreak
    \leftskip=#2cm
    \rightskip=#3cm
}
{
    \par
}
\fi

\doendnotes{C}
\bigskip
\vfill

\clearpage

\footnotesize

\ifkorrekturansicht
  \lohead{\textsc{register}}
\fi

% theindex-Environment neu definieren ohne reledmac
\makeatletter
\renewenvironment{theindex}{%
  \ifkorrekturansicht
    \section*{\indexname}%
  \else
    \subsubsection*{Index der erwähnten Entitäten}%
  \fi
  \setlength{\parindent}{0pt}%
  \setlength{\parskip}{0pt plus 0.3pt}%
  \let\item\@idxitem
}{%
  \ifkorrekturansicht\clearpage\fi
}
\makeatother

\IfFileExists{\jobname-pw.ind}{\input{\jobname-pw.ind}}{}

% Quellenangabe nur in der Leseansicht
\ifkorrekturansicht\else
% Fallback-Definitionen, falls die .tex-Datei \titel etc. nicht gesetzt hat
\providecommand{\titel}{}
\providecommand{\editorInnen}{}
\providecommand{\dateiname}{\jobname}

\vspace{3cm}

\vfill

\footnotesize
\textsc{Quelle}: \titel. Herausgegeben von {\editorInnen}. In: \emph{Arthur Schnitzler: Briefwechsel mit Autorinnen und Autoren}.
 Digitale Edition, https://schnitzler-briefe.acdh.oeaw.ac.at/{\dateiname}.html (Stand \today)
\fi

\end{document}


