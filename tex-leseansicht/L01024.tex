%% latex-korrekturansicht-vorspann.tex
%% Vorspann für die Korrekturansicht.
%% Lädt die gemeinsame Datei latex-vorspann.tex mit gesetztem Schalter.

\newif\ifkorrekturansicht
\korrekturansichttrue

\input{../tex-inputs/latex-vorspann}


\section[Arthur Schnitzler an Hugo von Hofmannsthal, 23. 3. 1900]{L01024 Arthur Schnitzler an Hugo von Hofmannsthal, 23. 3. 1900}
\nopagebreak\mylabel{L01024v}
\rehead{ }\normalsize\beginnumbering\briefempfaengerindex{Hofmannsthal, Hugo von@\textsc{Hofmannsthal, Hugo von}!zzzSchnitzler, Arthur@\emph{von Arthur Schnitzler}!1900-03-231@{23. 3. 1900}|(be}
\toendnotes[C]{\smallbreak\pagebreak[2]}\Standort{FDH, Hs-30885,90.}
\physDesc{Brief, 2 Blätter, 8 Seiten, 2415 Zeichen
\newline{}Handschrift: schwarze Tinte, deutsche Kurrent
\newline{}Ordnung: mit Bleistift von Schnitzler mutmaßlich bei der
                                 Durchsicht der Korrespondenz 1929 auch das zweite Blatt mit »23. 3. 900.« datiert }
\buchAbdrucke{\weitereDrucke{Hugo von Hofmannsthal, Arthur Schnitzler: \emph{Briefwechsel}. Frankfurt am Main: \emph{S. Fischer} 1964, S. 135–136.} }\toendnotes[C]{\smallbreak}
\pstart
           \raggedleft{}{\pb}23. 3. 900.\pend
           \vspace{0.5em}
\pstart
           mein lieber Hugo, Sie haben mich recht lang warten laſſen, aber was
               Sie mir ſchreiben iſt alles erfreulich und ſchön, und so hab ich es erwartet. Der
               kleine Ort heißt \textsc{Vilennes}\oindex{Villennes-sur-Seine@\textbf{Villennes-sur-Seine}, \emph{P.PPL}|pw} oder \textsc{Vilaines}\oindex{Villennes-sur-Seine@\textbf{Villennes-sur-Seine}, \emph{P.PPL}|pw} – bei \textsc{Poissy}\oindex{Poissy@\textbf{Poissy}, \emph{P.PPL}|pw}, we{\geminationn} mich nicht die Erinnerg trügt, an der
               \textsc{Marne}\oindex{Marne@\textbf{Marne}, \emph{H.STM}|pw}. Ich ka{\geminationn} nie an jene Stunde zurückdenken,
               ohne daſs ſich mein ganzes Weſen mit einem unbegreiflichen Schauer füllt, ſo als we{\geminationn} ich dort es eigentlich ſchon hätte wiſſen müssen – {\pb}oder gar – es gewußt hätte – (»dort – wo wir an lichten Tagen nicht
                  hineinſchaun!\pwindex{Schleier der Beatrice. Schauspiel in fuenf Akten@\emph{Der Schleier der Beatrice. Schauspiel in fünf Akten}|pwv}«) – Ihr Brief kam grad am Morgen des \label{K_L01024-1v}\edtext{18. März}{\lemma{\textnormal{\emph{18. März}}}\Cendnote{\textnormal{Maria Reinhards\pwindex{Reinhard, Marie 1871-03-13 – 1899-03-18@\textsc{Reinhard, Marie} (1871-03-13 – 1899-03-18), \emph{Gesangspädagoge/Gesangspädagogin}|pwk} erster
               Todestag.}}}\label{K_L01024-1}. –\pend
           
\pstart
           Ihr kleines Vorſpiel\pwindex{Vorspiel zur Antigone des Sophokles@\emph{Vorspiel zur Antigone des Sophokles}|pwv}, das ich
               ſehr einfach und ſchön finde, hab ich gleich an Paul Goldma{\geminationn}\pwindex{Goldmann, Paul 31.01.1865 – 25.09.1935@\textsc{Goldmann, Paul} (31.01.1865 – 25.09.1935), \emph{Schriftsteller/Schriftstellerin, Journalist/Journalistin}|pw} (\textsc{Berlin}\oindex{Berlin@\textbf{Berlin}, \emph{P.PPLC}|pw}, \textsc{Dessauer}ſtraße 19\oindex{Dessauer Strasse@\textbf{Dessauer Straße}, \emph{Straße (K.STR)}|pw}) geſchickt, vielleicht ſchreiben
               Sie ihm auch ein Wort? \pend
           
\pstart
           – Wir leben hier noch im ewigen {\pb}Winter. Schnee heut
               Nacht! – Und Wind, Regen, Koth. Es ist abſcheulich. Ich will in den nächſten Tagen
               ein bischen in den Süden fahren, bis Raguſa\oindex{Dubrovnik@\textbf{Dubrovnik}, \emph{P.PPLA}|pw}.
               Nicht mit rechter Freude. Aber ich hab auch i{\geminationm}er
               Katarrhe, jetzt noch dazu dumme Geſchichten mit plombirten Zähnen, dazu alles andre,
               kurz, ich ka{\geminationn}{ }{\pb}mich kaum je eine viertel Stunde wohl fühlen.
                  Anfang März war ich ein paar Tage in Edlach\oindex{Edlach@\textbf{Edlach}, \emph{P.PPL}|pw}; habe dort den Frühling finden wollen, aber Eis und 10 Grad Kälte,
               ſowie \textsc{Dora Speyer}\pwindex{Michaelis, Dora 23.05.1881 – 22.01.1946@\textsc{Michaelis, Dora} (23.05.1881 – 22.01.1946)|pw} gefunden, die übrigens lieb iſt.\pend
           
\pstart
           – Jetzt iſt \textsc{Brandes}\pwindex{Brandes, Georg 04.02.1842 – 19.02.1927@\textsc{Brandes, Georg} (04.02.1842 – 19.02.1927)|pw} hier, erzählt ſehr amüſant, und iſt gewiſs was ſehr beſondres. Und {\pb}doch (warum »und doch«?) hab ich eher ein Gefühl der
               Entfremdung diesmal ihm gegenüber. Liegt wohl an meiner Sti{\geminationm}ung. –\pend
           
\pstart
           Ich arbeite an nichts als an der langen Novelle\pwindex{Frau Bertha Garlan. Roman@\emph{Frau Bertha Garlan. Roman}|pwv}, die wohl (ſtofflich) ſo eine Art Seitenstück zur
                  \textsc{Femme de 30 ans}\pwindex{Eine Frau von dreissig Jahren@\emph{Eine Frau von dreißig Jahren}|pw} wird, eine \label{K_L01024-2v}\edtext{\textsc{veuve de 30 ans}}{\lemma{\textnormal{\emph{veuve de 30 ans}}}\Cendnote{\textnormal{französisch: Witwe von 30 Jahren}}}\label{K_L01024-2}
               – viel{\pb}leicht ſchließ ich ſie auf der dalmatiniſchen\oindex{Dalmatien@\textbf{Dalmatien}, \emph{L.RGNH}|pw} Küſtenfahrt ab. –\pend
           
\pstart
           Eben telephonirt mir Richard\pwindex{Beer-Hofmann, Richard 1866-07-11 – 1945-09-26@\textsc{Beer-Hofmann, Richard} (1866-07-11 – 1945-09-26), \emph{Schriftsteller/Schriftstellerin}|pw} ich möge in den
                  Schachclub\orgindex{Wiener Schachclub@Wiener Schachclub|pw} ko{\geminationm}en
               – Iſt das nicht ganz unwahrſcheinlich in Paris\oindex{Paris@\textbf{Paris}, \emph{P.PPLC}|pw} zu
               hören, daſs hier weiter telephonirt wird – in den Schachclub\orgindex{Wiener Schachclub@Wiener Schachclub|pw} gegangen –? So iſt es mir gewiſſermaßen räthſelhaft, daſs gewiſs
               das Haus {\pb}in der \textsc{rue Maubeuge Nr. 5}\oindex{rue de Maubeuge@\textbf{rue de Maubeuge}, \emph{Straße (K.STR)}|pw}{ }ſteht – ja daſs noch die Zi{\geminationm}er exiſtiren, die Fenſter – die Waſchtiſche – –\pend
           
\pstart
           Ich ka{\geminationn} Ihnen gar nicht ſagen wie mir iſt, während ich
               dieſen Brief ende. Als hätt ich’s noch i{\geminationm}er nicht ganz
                  verſtanden\pwindex{Reinhard, Marie 1871-03-13 – 1899-03-18@\textsc{Reinhard, Marie} (1871-03-13 – 1899-03-18), \emph{Gesangspädagoge/Gesangspädagogin}|pwv} – denn in
               dieſem Augenblick ſind mir Dinge eingefallen, an die ich ſeitdem nicht gedacht.\pend
           
\pstart
           {\pb}leben Sie wohl. Wann kommen Sie wieder? Werden wir
                  zuſa{\geminationm}en radeln? Ich bin neugierig auf das, was Sie
               mir von den Namenloſen erzählen werden.\pend
           
\pstart
           Von Herzen{\\[\baselineskip]}Ihr{\\[\baselineskip]}\spacefill\mbox{Arthur.}\pend
           \leftskip=0em{}
\pstart
           Grüßen Sie Hans Schleſinger\pwindex{Schlesinger, Hans Bernhard 20.07.1875 – 13.3.1932@\textsc{Schlesinger, Hans Bernhard} (20.07.1875 – 13.3.1932), \emph{Maler/Malerin}|pw} u. Bubi Franckenſtein\pwindex{Franckenstein, Georg von 18.03.1878 – 14.10.1953@\textsc{Franckenstein, Georg von} (18.03.1878 – 14.10.1953), \emph{Diplomat/Diplomatin}|pw}.\pend
           \selectlanguage{ngerman}\endnumbering\briefempfaengerindex{Hofmannsthal, Hugo von@\textsc{Hofmannsthal, Hugo von}!zzzSchnitzler, Arthur@\emph{von Arthur Schnitzler}!1900-03-231@{23. 3. 1900}|)be}\mylabel{L01024h}  \normalsize

\doendnotes{C}
\bigskip
\vfill

\clearpage

\footnotesize

\lohead{\textsc{register}}

% Definiere theindex-Environment komplett neu ohne reledmac
\makeatletter
\renewenvironment{theindex}{%
  \section*{\indexname}%
  \setlength{\parindent}{0pt}%
  \setlength{\parskip}{0pt plus 0.3pt}%
  \let\item\@idxitem
}{%
  \clearpage
}
\makeatother

\IfFileExists{\jobname-pw.ind}{\input{\jobname-pw.ind}}{}

\end{document}

      