%% latex-leseansicht-vorspann.tex
%% Vorspann für die Leseansicht.
%% Lädt die gemeinsame Datei latex-vorspann.tex mit nicht gesetztem Schalter.

\newif\ifkorrekturansicht
\korrekturansichtfalse

\input{../tex-inputs/latex-vorspann}


\section[Theodor Herzl an Arthur Schnitzler, 3. 4. 1895]{L03856 Theodor Herzl an Arthur Schnitzler, 3. 4. 1895}
\nopagebreak\mylabel{L03856v}
\rehead{ }\normalsize\beginnumbering\briefempfaengerindex{Schnitzler, Arthur@\textsc{Schnitzler, Arthur}!zzzHerzl, Theodor@\emph{von Theodor Herzl}!1895-04-032@{3. 4. 1895}|(be}
\toendnotes[C]{\smallbreak\pagebreak[2]}
\correspDesc{Versand  durch Theodor Herzl am 3. 4. 1895 in Paris
\newline{}Erhalt  durch Arthur Schnitzler im Zeitraum [4. 4. 1895
                  – 8. 4. 1895?] in Wien}\toendnotes[C]{\smallbreak}
\Standort{CUL, Schnitzler, B 39.}
\physDesc{Brief, 2 Blätter, 5 Seiten, 3973 Zeichen (Nummerierung des 2. Bogens: »2
                                 Blatt«)
\newline{}Handschrift: schwarze Tinte, lateinische Kurrent
\newline{}Ordnung: mit Bleistift von unbekannter Hand nummeriert: »35« }
\buchAbdrucke{\weitereDrucke{Theodor Herzl: \emph{Briefe und autobiographische Notizen 1866–1895}. Bearbeitet von Johannes Wachten in Zusammenarbeit mit Chaya Harel, Daisy Tycho und Manfred Winkler. Berlin, Frankfurt am Main, Wien: \emph{Propyläen} 1983, S. 580–582 (Briefe und Tagebücher. Herausgegeben von Alex Bein, Hermann Greive, Moshe Schaerf, Julius H. Schoeps und Johannes Wachten, 1).} }\toendnotes[C]{\smallbreak}
\pstart
           \raggedleft{}{\pb}Paris\oindex{Paris@\textbf{Paris}, \emph{Hauptstadt}|pw}{ }3. IV. 95\pend
           
\pstart{}Lieber Freund!\pend\vspace{0.5em}
\pstart
           Folgendes ist meiner Weisheit letzter Schluss: ich will noch einmal zu Blumenthal\pwindex{Blumenthal, Oskar 13.\,3.\,1852 Berlin – 24.\,4.\,1917 ebd.@\textsc{Blumenthal, Oskar} (13.\,3.\,1852 Berlin – 24.\,4.\,1917 ebd.), \emph{Schriftsteller, Journalist, Theaterleiter}|pw} gehen und wenn der nicht drauf
               eingeht mein Stück\pwindex{Herzl, Theodor 2.\,5.\,1860 Budapest – 3.\,7.\,1904 Edlach@\textsc{Herzl, Theodor} (2.\,5.\,1860 Budapest – 3.\,7.\,1904 Edlach), \emph{Schriftsteller, Journalist}!neue Ghetto. Schauspiel in vier Acten@\strich\emph{Das neue Ghetto. Schauspiel in vier Acten}|pwv} bei S. Fischer\pwindex{Fischer, Samuel 24.\,12.\,1859 Liptovský Mikuláš – 15.\,10.\,1934 Berlin@\textsc{Fischer, Samuel} (24.\,12.\,1859 Liptovský Mikuláš – 15.\,10.\,1934 Berlin), \emph{Verleger}|pw} dem \label{K_L03856-1v}\edtext{Jungen-Verleger}{\lemma{\textnormal{\emph{Jungen-Verleger}}}\Cendnote{\textnormal{Der 1886 gegründete \emph{Fischerverlag}\orgindex{S. Fischer Verlag@S. Fischer Verlag|pwk} publizierte Literatur von noch wenig bekannten Autoren.
                  Einige der Autoren die der Gruppierung \emph{Jung
                     Wien}\orgindex{Jung Wien@Jung Wien|pwk} zugeordnet werden, gehörten zum Verlagsprogramm des ersten Jahrzehnts
                     wie Hermann Bahr\pwindex{Bahr, Hermann 19.\,7.\,1863 Linz – 15.\,1.\,1934 München@\textsc{Bahr, Hermann} (19.\,7.\,1863 Linz – 15.\,1.\,1934 München), \emph{Schriftsteller, Kritiker}|pwk} (\emph{Die gute Schule}\pwindex{Bahr, Hermann 19.\,7.\,1863 Linz – 15.\,1.\,1934 München@\textsc{Bahr, Hermann} (19.\,7.\,1863 Linz – 15.\,1.\,1934 München), \emph{Schriftsteller, Kritiker}!gute Schule. Seelenstände@\strich\emph{Die gute Schule. Seelenstände}|pwk}, \emph{Die häusliche Frau}\pwindex{Bahr, Hermann 19.\,7.\,1863 Linz – 15.\,1.\,1934 München@\textsc{Bahr, Hermann} (19.\,7.\,1863 Linz – 15.\,1.\,1934 München), \emph{Schriftsteller, Kritiker}!häusliche Frau. Ein Lustspiel@\strich\emph{Die häusliche Frau. Ein Lustspiel}|pwk}), Peter Altenberg\pwindex{Altenberg, Peter 9.\,3.\,1859 Wien – 8.\,1.\,1919 ebd.@\textsc{Altenberg, Peter} (9.\,3.\,1859 Wien – 8.\,1.\,1919 ebd.), \emph{Schriftsteller}|pwk} (\emph{Wie ich es
                     sehe}\pwindex{Altenberg, Peter 9.\,3.\,1859 Wien – 8.\,1.\,1919 ebd.@\textsc{Altenberg, Peter} (9.\,3.\,1859 Wien – 8.\,1.\,1919 ebd.), \emph{Schriftsteller}!Wie ich es sehe@\strich\emph{Wie ich es sehe}|pwk}) und Leopold von
                     Andrian-Werbung\pwindex{Andrian-Werburg, Leopold von 9.\,5.\,1875 Berlin – 19.\,11.\,1951 Fribourg@\textsc{Andrian-Werburg, Leopold von} (9.\,5.\,1875 Berlin – 19.\,11.\,1951 Fribourg), \emph{Schriftsteller, Diplomat}|pwk} (\emph{Der Garten der
                     Erkenntnis}\pwindex{Andrian-Werburg, Leopold von 9.\,5.\,1875 Berlin – 19.\,11.\,1951 Fribourg@\textsc{Andrian-Werburg, Leopold von} (9.\,5.\,1875 Berlin – 19.\,11.\,1951 Fribourg), \emph{Schriftsteller, Diplomat}!Garten der Erkenntnis@\strich\emph{Der Garten der Erkenntnis}|pwk}).}}}\label{K_L03856-1} in Berlin\oindex{Berlin@\textbf{Berlin}, \emph{Hauptstadt}|pw}
               herausgeben, eventuell die Druckkosten selbst bezahlen.\pend
           
\pstart
           Ich bitte Sie also die nächstehenden drei Briefe zu schreiben:\pend
           
\pstart
           I Herrn Müller Guttenbrunn\pwindex{Müller-Guttenbrunn, Adam 22.\,10.\,1852 Zăbrani – 5.\,1.\,1923 Wien@\textsc{Müller-Guttenbrunn, Adam} (22.\,10.\,1852 Zăbrani – 5.\,1.\,1923 Wien), \emph{Schriftsteller, Theaterleiter, Beamter}|pw} (\label{K_L03856-2v}\edtext{recommandirt}{\lemma{\textnormal{\emph{recommandirt}}}\Cendnote{\textnormal{per Einschreiben}}}\label{K_L03856-2})\pend
           
\pstart
           Geehrter Herr! Herr \label{K_L03856-3v}\edtext{A. Schnabel}{\lemma{\textnormal{\emph{A. Schnabel}}}\Cendnote{\textnormal{Herzls\pwindex{Herzl, Theodor 2.\,5.\,1860 Budapest – 3.\,7.\,1904 Edlach@\textsc{Herzl, Theodor} (2.\,5.\,1860 Budapest – 3.\,7.\,1904 Edlach), \emph{Schriftsteller, Journalist}|pwk} Pseudonym bei der Verfasserschaft
                  von \emph{Das Neue Ghetto}\pwindex{Herzl, Theodor 2.\,5.\,1860 Budapest – 3.\,7.\,1904 Edlach@\textsc{Herzl, Theodor} (2.\,5.\,1860 Budapest – 3.\,7.\,1904 Edlach), \emph{Schriftsteller, Journalist}!neue Ghetto. Schauspiel in vier Acten@\strich\emph{Das neue Ghetto. Schauspiel in vier Acten}|pwk}}}}\label{K_L03856-3} dankt Ihnen für die rasche Erledigung. Er begreift die Opportunitätsgründe
               Ihrer Ablehnung. Endlich bittet er mich, Ihnen in seinem Namen mitzutheilen, dass er
               Sie Ihres \label{K_L03856-4v}\edtext{Ehrenwortes über seine
                  Verfasserschaft}{\lemma{\textnormal{\emph{Ehrenwortes … Verfasserschaft}}}\Cendnote{\textnormal{Im Unterschied zu den
                     Berliner\oindex{Berlin@\textbf{Berlin}, \emph{Hauptstadt}|pwk}{ }Theaterdirektoren\pwindex{Brahm, Otto 5.\,2.\,1856 Hamburg – 28.\,11.\,1912 Berlin@\textsc{Brahm, Otto} (5.\,2.\,1856 Hamburg – 28.\,11.\,1912 Berlin), \emph{Theaterleiter, Regisseur}|pwkv}\pwindex{Blumenthal, Oskar 13.\,3.\,1852 Berlin – 24.\,4.\,1917 ebd.@\textsc{Blumenthal, Oskar} (13.\,3.\,1852 Berlin – 24.\,4.\,1917 ebd.), \emph{Schriftsteller, Journalist, Theaterleiter}|pwkv}, hatte Herzl\pwindex{Herzl, Theodor 2.\,5.\,1860 Budapest – 3.\,7.\,1904 Edlach@\textsc{Herzl, Theodor} (2.\,5.\,1860 Budapest – 3.\,7.\,1904 Edlach), \emph{Schriftsteller, Journalist}|pwk} im
                  Fall von Müller-Guttenbrunn\pwindex{Müller-Guttenbrunn, Adam 22.\,10.\,1852 Zăbrani – 5.\,1.\,1923 Wien@\textsc{Müller-Guttenbrunn, Adam} (22.\,10.\,1852 Zăbrani – 5.\,1.\,1923 Wien), \emph{Schriftsteller, Theaterleiter, Beamter}|pwk} zugestimmt,
                  dass Schnitzler seinen richtigen Namen
                  preisgab, siehe XXXX Auszeichnungsfehler: Dokument L03850 nicht gefunden.}}}\label{K_L03856-4}
               absolutes Stillschweigen zu beobachten selbst dann nicht entbindet, wenn er
               gezwungen sein sollte, sich einem anderen Director zu nennen. Ich bitte Sie mir
               mitzutheilen, ob Sie nichts dagegen haben, dass Ihr \label{K_L03856-5v}\edtext{Ablehnungsbrief}{\lemma{\textnormal{\emph{Ablehnungsbrief}}}\Cendnote{\textnormal{Anders als der vorangegangene und der nachfolgende Brief Müller-Guttenbrunns\pwindex{Müller-Guttenbrunn, Adam 22.\,10.\,1852 Zăbrani – 5.\,1.\,1923 Wien@\textsc{Müller-Guttenbrunn, Adam} (22.\,10.\,1852 Zăbrani – 5.\,1.\,1923 Wien), \emph{Schriftsteller, Theaterleiter, Beamter}|pwk} in dieser Angelegenheit (Beilage zu
                  refXXXX27.3.1895 und Beilage zu refXXXX1.5.1895) ist der Absagebrief nicht
                  überliefert.}}}\label{K_L03856-5} ganz oder theilweise in {\pb}die Vorrede der Buchausgabe dieses Stückes\pwindex{Herzl, Theodor 2.\,5.\,1860 Budapest – 3.\,7.\,1904 Edlach@\textsc{Herzl, Theodor} (2.\,5.\,1860 Budapest – 3.\,7.\,1904 Edlach), \emph{Schriftsteller, Journalist}!neue Ghetto. Schauspiel in vier Acten@\strich\emph{Das neue Ghetto. Schauspiel in vier Acten}|pwv} aufgenommen werde.\pend
           
\pstart
           Hochacht{ } Dr A Schnitzler\pend
           
\pstart
           Sollte Ihnen dieser Wortlaut zu schroff sein, so bitte ich den Brief der genau diesen
               Inhalt haben muss, in der ersten Person des A. Schnabel von Schick\pwindex{Schik, Friedrich *~6.\,9.\,1857 Wien@\textsc{Schik, Friedrich} (*~6.\,9.\,1857 Wien), \emph{Notar, Journalist, Dramaturg}|pw} schreiben zu lassen.\pend
           
\pstart
           \strikeout{II Herrn S. Fischer Berlin}\pend
           
\pstart
           \strikeout{Diesen Brief bitte ich Sie jeden}\pend
           
\pstart
           II Herrn O. Blumenthal\pwindex{Blumenthal, Oskar 13.\,3.\,1852 Berlin – 24.\,4.\,1917 ebd.@\textsc{Blumenthal, Oskar} (13.\,3.\,1852 Berlin – 24.\,4.\,1917 ebd.), \emph{Schriftsteller, Journalist, Theaterleiter}|pw}{ }Berlin\oindex{Berlin@\textbf{Berlin}, \emph{Hauptstadt}|pw}\pend
           
\pstart
           Geehrter Herr! Sie haben mein Stück D. G...\pwindex{Herzl, Theodor 2.\,5.\,1860 Budapest – 3.\,7.\,1904 Edlach@\textsc{Herzl, Theodor} (2.\,5.\,1860 Budapest – 3.\,7.\,1904 Edlach), \emph{Schriftsteller, Journalist}!neue Ghetto. Schauspiel in vier Acten@\strich\emph{Das neue Ghetto. Schauspiel in vier Acten}|pwv} kurzweg abgelehnt ohne es zu lesen. Ich nehme Ihnen das nicht weiter
               übel. Unbekannte Autoren müssen sich von Theaterdirectoren Einiges gefallen lassen.
               Aber es scheint, dass Sie Unrecht hatten, das Buch\pwindex{Herzl, Theodor 2.\,5.\,1860 Budapest – 3.\,7.\,1904 Edlach@\textsc{Herzl, Theodor} (2.\,5.\,1860 Budapest – 3.\,7.\,1904 Edlach), \emph{Schriftsteller, Journalist}!neue Ghetto. Schauspiel in vier Acten@\strich\emph{Das neue Ghetto. Schauspiel in vier Acten}|pwv} nicht einmal zu öffnen. Herr Müller Guttenbrunn\pwindex{Müller-Guttenbrunn, Adam 22.\,10.\,1852 Zăbrani – 5.\,1.\,1923 Wien@\textsc{Müller-Guttenbrunn, Adam} (22.\,10.\,1852 Zăbrani – 5.\,1.\,1923 Wien), \emph{Schriftsteller, Theaterleiter, Beamter}|pw} hat das Stück\pwindex{Herzl, Theodor 2.\,5.\,1860 Budapest – 3.\,7.\,1904 Edlach@\textsc{Herzl, Theodor} (2.\,5.\,1860 Budapest – 3.\,7.\,1904 Edlach), \emph{Schriftsteller, Journalist}!neue Ghetto. Schauspiel in vier Acten@\strich\emph{Das neue Ghetto. Schauspiel in vier Acten}|pwv} gelesen und will es nur aus Rücksicht auf die Juden
               nicht spielen. Sie sind Jude, wie der Verfasser selbst, und können die Unternehmung
               wagen. Es gibt keinen vernünftigen jüdischen Rabbiner, der anders spräche, als dieses
                  Stück\pwindex{Herzl, Theodor 2.\,5.\,1860 Budapest – 3.\,7.\,1904 Edlach@\textsc{Herzl, Theodor} (2.\,5.\,1860 Budapest – 3.\,7.\,1904 Edlach), \emph{Schriftsteller, Journalist}!neue Ghetto. Schauspiel in vier Acten@\strich\emph{Das neue Ghetto. Schauspiel in vier Acten}|pwv} spricht. Es ist eine
                  {\pb}ehrliche Judenpredigt. Die Juden
               werden an den heimlichen Judenzeichen erkennen, dass ein wohlwollender Bruder zu
               ihnen aus dem Stücke\pwindex{Herzl, Theodor 2.\,5.\,1860 Budapest – 3.\,7.\,1904 Edlach@\textsc{Herzl, Theodor} (2.\,5.\,1860 Budapest – 3.\,7.\,1904 Edlach), \emph{Schriftsteller, Journalist}!neue Ghetto. Schauspiel in vier Acten@\strich\emph{Das neue Ghetto. Schauspiel in vier Acten}|pwv} heraus
               redet.\pend
           
\pstart
           Müller Guttenbrunn\pwindex{Müller-Guttenbrunn, Adam 22.\,10.\,1852 Zăbrani – 5.\,1.\,1923 Wien@\textsc{Müller-Guttenbrunn, Adam} (22.\,10.\,1852 Zăbrani – 5.\,1.\,1923 Wien), \emph{Schriftsteller, Theaterleiter, Beamter}|pw} schreibt: Ich habe das
               Schauspiel »D. Ghetto\pwindex{Herzl, Theodor 2.\,5.\,1860 Budapest – 3.\,7.\,1904 Edlach@\textsc{Herzl, Theodor} (2.\,5.\,1860 Budapest – 3.\,7.\,1904 Edlach), \emph{Schriftsteller, Journalist}!neue Ghetto. Schauspiel in vier Acten@\strich\emph{Das neue Ghetto. Schauspiel in vier Acten}|pw}« von zwei gebildeten
               Männern lesen lassen, von einem Juden u. einem Christen, u. Beide verwarfen das Stück\pwindex{Herzl, Theodor 2.\,5.\,1860 Budapest – 3.\,7.\,1904 Edlach@\textsc{Herzl, Theodor} (2.\,5.\,1860 Budapest – 3.\,7.\,1904 Edlach), \emph{Schriftsteller, Journalist}!neue Ghetto. Schauspiel in vier Acten@\strich\emph{Das neue Ghetto. Schauspiel in vier Acten}|pwv}, beide aus
               Opportunitätsgründen, aber sie {\dots} (etc aus dem \label{K_L03856-6v}\edtext{Brief}{\lemma{\textnormal{\emph{Brief}}}\Cendnote{\textnormal{nicht
                  überliefert}}}\label{K_L03856-6} bis inclusive »verzichte« abzuschreiben)\pend
           
\pstart
           Dieser Brief u. das Manuscript\pwindex{Herzl, Theodor 2.\,5.\,1860 Budapest – 3.\,7.\,1904 Edlach@\textsc{Herzl, Theodor} (2.\,5.\,1860 Budapest – 3.\,7.\,1904 Edlach), \emph{Schriftsteller, Journalist}!neue Ghetto. Schauspiel in vier Acten@\strich\emph{Das neue Ghetto. Schauspiel in vier Acten}|pwv}
               erliegen beim Verleger S. Fischer\pwindex{Fischer, Samuel 24.\,12.\,1859 Liptovský Mikuláš – 15.\,10.\,1934 Berlin@\textsc{Fischer, Samuel} (24.\,12.\,1859 Liptovský Mikuláš – 15.\,10.\,1934 Berlin), \emph{Verleger}|pw} in Berlin\oindex{Berlin@\textbf{Berlin}, \emph{Hauptstadt}|pw}. Wenn Sie es nun lesen wollen, so bitte ich
               das Manuscript\pwindex{Herzl, Theodor 2.\,5.\,1860 Budapest – 3.\,7.\,1904 Edlach@\textsc{Herzl, Theodor} (2.\,5.\,1860 Budapest – 3.\,7.\,1904 Edlach), \emph{Schriftsteller, Journalist}!neue Ghetto. Schauspiel in vier Acten@\strich\emph{Das neue Ghetto. Schauspiel in vier Acten}|pwv} von Herrn Fischer\pwindex{Fischer, Samuel 24.\,12.\,1859 Liptovský Mikuláš – 15.\,10.\,1934 Berlin@\textsc{Fischer, Samuel} (24.\,12.\,1859 Liptovský Mikuláš – 15.\,10.\,1934 Berlin), \emph{Verleger}|pw} abholen zu lassen. Er wird es ihnen auf
               acht Tage leihen.\pend
           
\pstart
           Lassen Sie es nicht holen oder entscheiden Sie sich nicht innerhalb einer Woche, so
               erscheint das Stück\pwindex{Herzl, Theodor 2.\,5.\,1860 Budapest – 3.\,7.\,1904 Edlach@\textsc{Herzl, Theodor} (2.\,5.\,1860 Budapest – 3.\,7.\,1904 Edlach), \emph{Schriftsteller, Journalist}!neue Ghetto. Schauspiel in vier Acten@\strich\emph{Das neue Ghetto. Schauspiel in vier Acten}|pwv} im
               Druck.\pend
           
\pstart
           \centering{}Hochachtungsvoll\pend
           
\pstart
           \raggedleft{}Albert Schnabel\pend
           
\pstart
           (Brief II soll Schick\pwindex{Schik, Friedrich *~6.\,9.\,1857 Wien@\textsc{Schik, Friedrich} (*~6.\,9.\,1857 Wien), \emph{Notar, Journalist, Dramaturg}|pw} oder sonst wer
               abschreiben)\pend
           
\pstart
           III Herrn S. Fischer\pwindex{Fischer, Samuel 24.\,12.\,1859 Liptovský Mikuláš – 15.\,10.\,1934 Berlin@\textsc{Fischer, Samuel} (24.\,12.\,1859 Liptovský Mikuláš – 15.\,10.\,1934 Berlin), \emph{Verleger}|pw}{ }Berlin\oindex{Berlin@\textbf{Berlin}, \emph{Hauptstadt}|pw}\pend
           
\pstart
           Geehrter Herr. Ich sende Ihnen heute das Manuscript eines 4 actigen {\pb}Schauspiels D. G.\pwindex{Herzl, Theodor 2.\,5.\,1860 Budapest – 3.\,7.\,1904 Edlach@\textsc{Herzl, Theodor} (2.\,5.\,1860 Budapest – 3.\,7.\,1904 Edlach), \emph{Schriftsteller, Journalist}!neue Ghetto. Schauspiel in vier Acten@\strich\emph{Das neue Ghetto. Schauspiel in vier Acten}|pwv} von A. Schnabel. Lesen Sie es sofort.
               Der \strikeout{Ber} Brief des Directors Müller\pwindex{Müller-Guttenbrunn, Adam 22.\,10.\,1852 Zăbrani – 5.\,1.\,1923 Wien@\textsc{Müller-Guttenbrunn, Adam} (22.\,10.\,1852 Zăbrani – 5.\,1.\,1923 Wien), \emph{Schriftsteller, Theaterleiter, Beamter}|pw}, den ich Ihnen hier beilege u. \uline{sorgsam} aufzubewahren bitte, sagt Ihnen welcher Art das Wagniss ist.
               Schreiben Sie mir gefälligst, ob u. unter welchen Bedingungen Sie geneigt sind, das
                  Stück\pwindex{Herzl, Theodor 2.\,5.\,1860 Budapest – 3.\,7.\,1904 Edlach@\textsc{Herzl, Theodor} (2.\,5.\,1860 Budapest – 3.\,7.\,1904 Edlach), \emph{Schriftsteller, Journalist}!neue Ghetto. Schauspiel in vier Acten@\strich\emph{Das neue Ghetto. Schauspiel in vier Acten}|pwv} zu verlegen.\pend
           
\pstart
           Sollte Director Blumenthal\pwindex{Blumenthal, Oskar 13.\,3.\,1852 Berlin – 24.\,4.\,1917 ebd.@\textsc{Blumenthal, Oskar} (13.\,3.\,1852 Berlin – 24.\,4.\,1917 ebd.), \emph{Schriftsteller, Journalist, Theaterleiter}|pw} das Manuscript\pwindex{Herzl, Theodor 2.\,5.\,1860 Budapest – 3.\,7.\,1904 Edlach@\textsc{Herzl, Theodor} (2.\,5.\,1860 Budapest – 3.\,7.\,1904 Edlach), \emph{Schriftsteller, Journalist}!neue Ghetto. Schauspiel in vier Acten@\strich\emph{Das neue Ghetto. Schauspiel in vier Acten}|pwv} von Ihnen verlangen, so bitte
               ich es ihm auf acht Tage zu leihen. Nicht für länger.\pend
           
\pstart
           Ich bitte Sie, meine Intervention Jedermann, besonders Blumenthal\pwindex{Blumenthal, Oskar 13.\,3.\,1852 Berlin – 24.\,4.\,1917 ebd.@\textsc{Blumenthal, Oskar} (13.\,3.\,1852 Berlin – 24.\,4.\,1917 ebd.), \emph{Schriftsteller, Journalist, Theaterleiter}|pw} gegenüber streng geheim zu halten. Dr. Schnabel
               bat mich nur, ihn bei Ihnen einzuführen, im Uebrigen wünscht er nicht, unter irgend
               Jemandes Patronanz zu stehen.\pend
           
\pstart
           Bestätigen Sie nur freundlichst, dass Sie diese Bedingung genau einhalten werden.\pend
           
\pstart
           \centering{}Hochachtungsvoll\pend
           
\pstart
           \raggedleft{}Dr. A. Schnitzler.\pend
           
\pstart
           Diese 3 Briefe enthalten alle Vorschriften für Sie, mein gefälliger lieber Freund.
               Ich habe der Kürze {\pb}halber den Briefen
               gleich die definitive Form gegeben.\pend
           
\pstart
           Wenn Fischer\pwindex{Fischer, Samuel 24.\,12.\,1859 Liptovský Mikuláš – 15.\,10.\,1934 Berlin@\textsc{Fischer, Samuel} (24.\,12.\,1859 Liptovský Mikuláš – 15.\,10.\,1934 Berlin), \emph{Verleger}|pw} antwortet, dass er es nicht auf
                  \strikeout{eig} seine Kosten drucken lassen will, so
               bewilligen Sie ihm die Druckkosten, die ich Ihnen sofort einschicken werde, sobald
               ich den Betrag kenne. Bedingung: sofortiges Erscheinen.\pend
           
\pstart
           Ihre Antwort erbitte ich wieder wie früher \label{K_L03856-7v}\edtext{\begin{otherlanguage}{french}poste restante\end{otherlanguage}}{\lemma{\textnormal{\emph{poste restante}}}\Cendnote{\textnormal{französisch: postlagernd}}}\label{K_L03856-7} mit
               gleichzeitiger \label{K_L03856-8v}\edtext{Verständigung an
                  Albert}{\lemma{\textnormal{\emph{Verständigung an
                  Albert}}}\Cendnote{\textnormal{Zu Herzls\pwindex{Herzl, Theodor 2.\,5.\,1860 Budapest – 3.\,7.\,1904 Edlach@\textsc{Herzl, Theodor} (2.\,5.\,1860 Budapest – 3.\,7.\,1904 Edlach), \emph{Schriftsteller, Journalist}|pwk} Vorgaben für die klandestine Kommunikation über
                  sein Stück\pwindex{Herzl, Theodor 2.\,5.\,1860 Budapest – 3.\,7.\,1904 Edlach@\textsc{Herzl, Theodor} (2.\,5.\,1860 Budapest – 3.\,7.\,1904 Edlach), \emph{Schriftsteller, Journalist}!neue Ghetto. Schauspiel in vier Acten@\strich\emph{Das neue Ghetto. Schauspiel in vier Acten}|pwkv}, { }vgl. XXXX Auszeichnungsfehler: Dokument L03836 nicht gefunden. }}}\label{K_L03856-8}
               geschrieben.\pend
           
\pstart
           Tausend Dank im voraus. Ich hoffe, dass Ihre Mühe jetzt zu Ende ist.\pend
           \pstart Mit herzlichen Grüssen Ihr aufrichtig ergebener \spacefill\mbox{Th Herzl}\pend{}\selectlanguage{ngerman}\endnumbering\briefempfaengerindex{Schnitzler, Arthur@\textsc{Schnitzler, Arthur}!zzzHerzl, Theodor@\emph{von Theodor Herzl}!1895-04-032@{3. 4. 1895}|)be}\mylabel{L03856h}
\begin{anhang}
\end{anhang}\newcommand{\dateiname}{L03856}\newcommand{\titel}{Theodor Herzl an Arthur Schnitzler, 3. 4. 1895}\newcommand{\editorInnen}{Selma Jahnke und Martin Anton Müller}%% latex-leseansicht-abspann.tex
%% Abspann für die Leseansicht.
%% Der Schalter \ifkorrekturansicht ist bereits durch den Vorspann gesetzt.

%% latex-abspann.tex
%% Gemeinsamer Abspann für Korrekturansicht und Leseansicht.
%% Setzt den Schalter \ifkorrekturansicht voraus (gesetzt in den
%% einbindenden Dateien latex-korrekturansicht-abspann.tex bzw.
%% latex-leseansicht-abspann.tex).
%% ---------------------------------------------------------------

\normalsize

% Das esempio-Environment wird nur in der Leseansicht benötigt
\ifkorrekturansicht\else
\newenvironment{esempio}[3]%
{
    \vspace{1.5ex}
    \rlap{\underline{#1}}
    \par
    \setlength{\parindent}{0cm}
    \nopagebreak
    \leftskip=#2cm
    \rightskip=#3cm
}
{
    \par
}
\fi

\doendnotes{C}
\bigskip
\vfill

\clearpage

\footnotesize

\ifkorrekturansicht
  \lohead{\textsc{register}}
\fi

% theindex-Environment neu definieren ohne reledmac
\makeatletter
\renewenvironment{theindex}{%
  \ifkorrekturansicht
    \section*{\indexname}%
  \else
    \subsubsection*{Index der erwähnten Entitäten}%
  \fi
  \setlength{\parindent}{0pt}%
  \setlength{\parskip}{0pt plus 0.3pt}%
  \let\item\@idxitem
}{%
  \ifkorrekturansicht\clearpage\fi
}
\makeatother

\IfFileExists{\jobname-pw.ind}{\input{\jobname-pw.ind}}{}

% Quellenangabe nur in der Leseansicht
\ifkorrekturansicht\else
% Fallback-Definitionen, falls die .tex-Datei \titel etc. nicht gesetzt hat
\providecommand{\titel}{}
\providecommand{\editorInnen}{}
\providecommand{\dateiname}{\jobname}

\vspace{3cm}

\vfill

\footnotesize
\textsc{Quelle}: \titel. Herausgegeben von {\editorInnen}. In: \emph{Arthur Schnitzler: Briefwechsel mit Autorinnen und Autoren}.
 Digitale Edition, https://schnitzler-briefe.acdh.oeaw.ac.at/{\dateiname}.html (Stand \today)
\fi

\end{document}


