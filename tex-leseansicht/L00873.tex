%% latex-leseansicht-vorspann.tex
%% Vorspann für die Leseansicht.
%% Lädt die gemeinsame Datei latex-vorspann.tex mit nicht gesetztem Schalter.

\newif\ifkorrekturansicht
\korrekturansichtfalse

\input{../tex-inputs/latex-vorspann}


         
         \renewcommand{\erwaehntePersonen}{Personen:  ?? [Gesprächspartner von Hofmannsthal 1],  ?? [Gesprächspartner von Hofmannsthal 2], Richard Beer-Hofmann, Hugo von Hofmannsthal, Gertrude von Hofmannsthal, Hugo August von Hofmannsthal, Anna von Hofmannsthal, Paul Schlenther, Franziska Schlesinger}
         \renewcommand{\erwaehnteOrte}{Orte: Baden bei Wien, Deutschland, Erbprinz, Hörselberge, Ilmenau, Julienhof, Weimar, Wien}
         \renewcommand{\erwaehnteWerke}{Werke: Der grüne Kakadu. Groteske in einem Akt, Die Hochzeit der Sobeide}
               \section[Hugo von Hofmannsthal an Arthur Schnitzler, {[}1. 1. 1899{]}]{ Hugo von Hofmannsthal an Arthur Schnitzler, {[}1. 1. 1899{]}}\nopagebreak\mylabel{v}\rehead{ }\begin{ledgroupsized}[t]{13cm}\normalsize\beginnumbering\briefempfaengerindex{Schnitzler, Arthur@\textsc{Schnitzler, Arthur}!zzzHofmannsthal, Hugo von@\emph{von Hugo von Hofmannsthal}!1899-01-011@{{[}1. 1. 1899{]}}|(be} \toendnotes[C]{\smallbreak\pagebreak[2]} \Standort{CUL, Schnitzler, B 43.}
\physDesc{Brief, 1 Blatt, 4 Seiten, 905 Zeichen
\newline{}Handschrift: schwarze Tinte, deutsche Kurrent
\newline{}Schnitzler: mit Bleistift datiert: »Jänner? 99« 
\newline{}Ordnung: 1) mit Bleistift von unbekannter Hand nummeriert: »\strikeout{138}«  2) mit Bleistift von unbekannter Hand nummeriert:
                                    »130«}\buchAbdrucke{\weitereDrucke{Hugo von Hofmannsthal, Arthur Schnitzler: \emph{Briefwechsel}. Hg. Therese Nickl und Heinrich Schnitzler. Frankfurt am Main: \emph{S. Fischer} 1964, S. 115–116.} }\toendnotes[C]{\smallbreak}\pstart
           \raggedleft{}{\pb}\label{K_L00873-1v}\edtext{Baden, Julienhof\oindex{Julienhof@\textbf{Julienhof}|pw}}{\lemma{\textnormal{\emph{Baden, Julienhof}}}\Cendnote{\textnormal{Hofmannsthal\pwindex{Hofmannsthal, Hugo von 1874-02-01 – 1929-07-15@\textsc{Hofmannsthal, Hugo von} (1874-02-01 – 1929-07-15), \emph{Schriftsteller}|pwk} hielt sich vom
                        28. 12. 1898 bis zum 9. 1. 1899 in der Pension Julienhof\oindex{Julienhof@\textbf{Julienhof}|pwk} in Baden\oindex{Baden bei Wien@\textbf{Baden bei Wien}|pwk} auf.}}}\label{K_L00873-1h}\pend
           \pstart
           lieber Arthur, mir gehts hier gut und ich hab am
                  Silveſterabend in der ſchönſten Stille die neue 2\textsuperscript{te}{ }Verwandlung\pwindex{Hofmannsthal, Hugo von 1874-02-01 – 1929-07-15@\textsc{Hofmannsthal, Hugo von} (1874-02-01 – 1929-07-15), \emph{Schriftsteller}!Hochzeit der Sobeide1899-03-18@\strich\emph{Die Hochzeit der Sobeide} {[}1899-03-18{]}|pwv} vollendet. \label{K_L00873-2v}\edtext{Heut}{\lemma{\textnormal{\emph{Heut}}}\Cendnote{\textnormal{Die genauere Datierung des Korrespondenzstücks gelingt durch den Brief an
                     Franziska Schlesinger\pwindex{Schlesinger, Franziska 17.08.1851 – 11.08.1932@\textsc{Schlesinger, Franziska} (17.08.1851 – 11.08.1932)|pwk} vom
                  4. 1. 1899, worin Hofmannsthal\pwindex{Hofmannsthal, Hugo von 1874-02-01 – 1929-07-15@\textsc{Hofmannsthal, Hugo von} (1874-02-01 – 1929-07-15), \emph{Schriftsteller}|pwk} berichtet, am ersten Tag des Jahres kurz in
                     Wien\oindex{Wien@\textbf{Wien}|pwk} gewesen zu sein und dort ihren Brief
                  vorgefunden zu haben.}}}\label{K_L00873-2h} war ich wenige Stunden in der Stadt\oindex{Wien@\textbf{Wien}|pw}, habs dem Richard\pwindex{Beer-Hofmann, Richard 1866-07-11 – 1945-09-26@\textsc{Beer-Hofmann, Richard} (1866-07-11 – 1945-09-26), \emph{Schriftsteller}|pw}
               vorgeleſen der es nun in Ordnung findet, ſo daſs ich’s nicht mehr zu Ihnen ſondern
               zum {\pb}Typieren getragen habe.\pend
           \pstart
           Habe auch Schlenther\pwindex{Schlenther, Paul 20.08.1854 – 30.04.1916@\textsc{Schlenther, Paul} (20.08.1854 – 30.04.1916), \emph{Schriftsteller, Kritiker, Theaterleiter}|pw} geſprochen. Haben Sie
               Nachrichten über den »Kakadu\pwindex{Schnitzler, Arthur 15.05.1862 – 21.10.1931@\textsc{Schnitzler, Arthur} (15.05.1862 – 21.10.1931), \emph{Schriftsteller, Mediziner}!gruene Kakadu. Groteske in einem Akt1. 3. 1899@\strich\emph{Der grüne Kakadu. Groteske in einem Akt} {[}1. 3. 1899{]}|pw}«?\hspace*{2em}Neulich hab ich mir von 2 geſcheiten Leuten\pwindex{?? [Gespraechspartner von Hofmannsthal 1] Ende 1898 – Ende 1898@\textsc{?? [Gesprächspartner von Hofmannsthal 1]} (Ende 1898 – Ende 1898)|pwv}\pwindex{?? [Gespraechspartner von Hofmannsthal 2] Ende 1898 – Ende 1898@\textsc{?? [Gesprächspartner von Hofmannsthal 2]} (Ende 1898 – Ende 1898)|pwv} unſre ſchöne
               Juniradpartie durch Mitteldeutſchland\oindex{Deutschland@\textbf{Deutschland}|pw}
               aufſchreiben laſſen. Wir kommen am Hörſelberg\oindex{Hoerselberge@\textbf{Hörselberge}|pw}
               und vielen ſchönen Sachen vorbei, {\pb}fahren über Ilmenau\oindex{Ilmenau@\textbf{Ilmenau}|pw} in Weimar\oindex{Weimar@\textbf{Weimar}|pw} ein, wohnen 4 Tage im »Erbprinzen\oindex{Erbprinz@\textbf{Erbprinz}|pw}« und ſind – hoffentlich – brav und luſtig.\pend
           \pstart
           Ich hab heut in Wien\oindex{Wien@\textbf{Wien}|pw} mit \label{K_L00873-3v}\edtext{jemand\pwindex{Hofmannsthal, Gertrude von 16.03.1880 – 09.11.1959@\textsc{Hofmannsthal, Gertrude von} (16.03.1880 – 09.11.1959)|pwuv}}{\lemma{\textnormal{\emph{jemand}}}\Cendnote{\textnormal{Wenngleich nicht mit Sicherheit zu
                  belegen, liegt es nahe, dass er seinen Eltern\pwindex{Hofmannsthal, Hugo August von 21.12.1841 – 08.12.1915@\textsc{Hofmannsthal, Hugo August von} (21.12.1841 – 08.12.1915), \emph{Bankdirektor}|pwkv}\pwindex{Hofmannsthal, Anna von 27.01.1849 – 22.03.1904@\textsc{Hofmannsthal, Anna von} (27.01.1849 – 22.03.1904)|pwkv} ein Treffen mit seiner späteren Frau Gerty\pwindex{Hofmannsthal, Gertrude von 16.03.1880 – 09.11.1959@\textsc{Hofmannsthal, Gertrude von} (16.03.1880 – 09.11.1959)|pwk} verheimlicht hat.}}}\label{K_L00873-3h} gegeſſen und dann
               zuhaus geſagt, ich hab bei Ihnen gegeſſen. Da ich ſolche Lügen ſehr ungern hab {\pb}und auch dieſe nur halb in
               Zerſtreutheit geſagt habe, bitte dementieren Sie nicht, falls Sie zufällig meine Eltern\pwindex{Hofmannsthal, Hugo August von 21.12.1841 – 08.12.1915@\textsc{Hofmannsthal, Hugo August von} (21.12.1841 – 08.12.1915), \emph{Bankdirektor}|pwv}\pwindex{Hofmannsthal, Anna von 27.01.1849 – 22.03.1904@\textsc{Hofmannsthal, Anna von} (27.01.1849 – 22.03.1904)|pwv}{ }ſehen.\pend
           \pstart
           Von Herzen Ihr{\\[\baselineskip]}\spacefill\mbox{Hugo.}\pend
           \leftskip=0em{}
         
         \endnumbering\mylabel{h}\end{ledgroupsized}  \newcommand{\dateiname}{L00873}\newcommand{\titel}{Hugo von Hofmannsthal an Arthur Schnitzler, [1. 1. 1899]}\newcommand{\editorInnen}{Martin Anton Müller und Gerd-Hermann Susen}%% latex-leseansicht-abspann.tex
%% Abspann für die Leseansicht.
%% Der Schalter \ifkorrekturansicht ist bereits durch den Vorspann gesetzt.

%% latex-abspann.tex
%% Gemeinsamer Abspann für Korrekturansicht und Leseansicht.
%% Setzt den Schalter \ifkorrekturansicht voraus (gesetzt in den
%% einbindenden Dateien latex-korrekturansicht-abspann.tex bzw.
%% latex-leseansicht-abspann.tex).
%% ---------------------------------------------------------------

\normalsize

% Das esempio-Environment wird nur in der Leseansicht benötigt
\ifkorrekturansicht\else
\newenvironment{esempio}[3]%
{
    \vspace{1.5ex}
    \rlap{\underline{#1}}
    \par
    \setlength{\parindent}{0cm}
    \nopagebreak
    \leftskip=#2cm
    \rightskip=#3cm
}
{
    \par
}
\fi

\doendnotes{C}
\bigskip
\vfill

\clearpage

\footnotesize

\ifkorrekturansicht
  \lohead{\textsc{register}}
\fi

% theindex-Environment neu definieren ohne reledmac
\makeatletter
\renewenvironment{theindex}{%
  \ifkorrekturansicht
    \section*{\indexname}%
  \else
    \subsubsection*{Index der erwähnten Entitäten}%
  \fi
  \setlength{\parindent}{0pt}%
  \setlength{\parskip}{0pt plus 0.3pt}%
  \let\item\@idxitem
}{%
  \ifkorrekturansicht\clearpage\fi
}
\makeatother

\IfFileExists{\jobname-pw.ind}{\input{\jobname-pw.ind}}{}

% Quellenangabe nur in der Leseansicht
\ifkorrekturansicht\else
% Fallback-Definitionen, falls die .tex-Datei \titel etc. nicht gesetzt hat
\providecommand{\titel}{}
\providecommand{\editorInnen}{}
\providecommand{\dateiname}{\jobname}

\vspace{3cm}

\vfill

\footnotesize
\textsc{Quelle}: \titel. Herausgegeben von {\editorInnen}. In: \emph{Arthur Schnitzler: Briefwechsel mit Autorinnen und Autoren}.
 Digitale Edition, https://schnitzler-briefe.acdh.oeaw.ac.at/{\dateiname}.html (Stand \today)
\fi

\end{document}


      