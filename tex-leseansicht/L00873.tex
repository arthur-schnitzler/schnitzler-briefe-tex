%% latex-korrekturansicht-vorspann.tex
%% Vorspann für die Korrekturansicht.
%% Lädt die gemeinsame Datei latex-vorspann.tex mit gesetztem Schalter.

\newif\ifkorrekturansicht
\korrekturansichttrue

\input{../tex-inputs/latex-vorspann}


\section[Hugo von Hofmannsthal an Arthur Schnitzler, {[}1. 1. 1899{]}]{L00873 Hugo von Hofmannsthal an Arthur Schnitzler, {[}1. 1. 1899{]}}
\nopagebreak\mylabel{L00873v}
\rehead{ }\normalsize\beginnumbering\briefempfaengerindex{Schnitzler, Arthur@\textsc{Schnitzler, Arthur}!zzzHofmannsthal, Hugo von@\emph{von Hugo von Hofmannsthal}!1899-01-011@{{[}1. 1. 1899{]}}|(be}
\toendnotes[C]{\smallbreak\pagebreak[2]}\Standort{CUL, Schnitzler, B 43.}
\physDesc{Brief, 1 Blatt, 4 Seiten, 905 Zeichen
\newline{}Handschrift: schwarze Tinte, deutsche Kurrent
\newline{}Schnitzler: mit Bleistift datiert: »Jänner? 99« 
\newline{}Ordnung: 1) mit Bleistift von unbekannter Hand nummeriert: »\strikeout{138}«  2) mit Bleistift von unbekannter Hand nummeriert:
                                    »130«}
\buchAbdrucke{\weitereDrucke{Hugo von Hofmannsthal, Arthur Schnitzler: \emph{Briefwechsel}. Frankfurt am Main: \emph{S. Fischer} 1964, S. 115–116.} }\toendnotes[C]{\smallbreak}
\pstart
           \raggedleft{}{\pb}\label{K_L00873-1v}\edtext{Baden, Julienhof\oindex{Julienhof@\textbf{Julienhof}, \emph{Hotel (K.HTL)}|pw}}{\lemma{\textnormal{\emph{Baden, Julienhof}}}\Cendnote{\textnormal{Hofmannsthal\pwindex{Hofmannsthal, Hugo von 1874-02-01 – 1929-07-15@\textsc{Hofmannsthal, Hugo von} (1874-02-01 – 1929-07-15), \emph{Schriftsteller/Schriftstellerin}|pwk} hielt sich vom
                        28. 12. 1898 bis zum 9. 1. 1899 in der Pension Julienhof\oindex{Julienhof@\textbf{Julienhof}, \emph{Hotel (K.HTL)}|pwk} in Baden\oindex{Baden bei Wien@\textbf{Baden bei Wien}, \emph{P.PPLA3}|pwk} auf.}}}\label{K_L00873-1}\pend
           \vspace{0.5em}
\pstart
           lieber Arthur, mir gehts hier gut und ich hab am
                  Silveſterabend in der ſchönſten Stille die neue 2\textsuperscript{te}{ }Verwandlung\pwindex{Hochzeit der Sobeide@\emph{Die Hochzeit der Sobeide}|pwv} vollendet. \label{K_L00873-2v}\edtext{Heut}{\lemma{\textnormal{\emph{Heut}}}\Cendnote{\textnormal{Die genauere Datierung des Korrespondenzstücks gelingt durch den Brief an
                     Franziska Schlesinger\pwindex{Schlesinger, Franziska 17.08.1851 – 11.08.1932@\textsc{Schlesinger, Franziska} (17.08.1851 – 11.08.1932)|pwk} vom
                  4. 1. 1899, worin Hofmannsthal\pwindex{Hofmannsthal, Hugo von 1874-02-01 – 1929-07-15@\textsc{Hofmannsthal, Hugo von} (1874-02-01 – 1929-07-15), \emph{Schriftsteller/Schriftstellerin}|pwk} berichtet, am ersten Tag des Jahres kurz in
                     Wien\oindex{Wien@\textbf{Wien}, \emph{A.ADM2}|pwk} gewesen zu sein und dort ihren Brief
                  vorgefunden zu haben.}}}\label{K_L00873-2} war ich wenige Stunden in der Stadt\oindex{Wien@\textbf{Wien}, \emph{A.ADM2}|pw}, habs dem Richard\pwindex{Beer-Hofmann, Richard 1866-07-11 – 1945-09-26@\textsc{Beer-Hofmann, Richard} (1866-07-11 – 1945-09-26), \emph{Schriftsteller/Schriftstellerin}|pw}
               vorgeleſen der es nun in Ordnung findet, ſo daſs ich’s nicht mehr zu Ihnen ſondern
               zum {\pb}Typieren getragen habe.\pend
           
\pstart
           Habe auch Schlenther\pwindex{Schlenther, Paul 20.08.1854 – 30.04.1916@\textsc{Schlenther, Paul} (20.08.1854 – 30.04.1916), \emph{Schriftsteller/Schriftstellerin, Kritiker/Kritikerin, Theaterleiter/Theaterleiterin}|pw} geſprochen. Haben Sie
               Nachrichten über den »Kakadu\pwindex{gruene Kakadu. Groteske in einem Akt@\emph{Der grüne Kakadu. Groteske in einem Akt}|pw}«?\hspace*{2em}Neulich hab ich mir von 2 geſcheiten Leuten\pwindex{?? [Gespraechspartner von Hofmannsthal 1] Ende 1898 – Ende 1898@\textsc{?? [Gesprächspartner von Hofmannsthal 1]} (Ende 1898 – Ende 1898)|pwv}\pwindex{?? [Gespraechspartner von Hofmannsthal 2] Ende 1898 – Ende 1898@\textsc{?? [Gesprächspartner von Hofmannsthal 2]} (Ende 1898 – Ende 1898)|pwv} unſre ſchöne
               Juniradpartie durch Mitteldeutſchland\oindex{Deutschland@\textbf{Deutschland}, \emph{A.PCLI}|pw}
               aufſchreiben laſſen. Wir kommen am Hörſelberg\oindex{Hoerselberge@\textbf{Hörselberge}, \emph{Berg (N.BRG)}|pw}
               und vielen ſchönen Sachen vorbei, {\pb}fahren über Ilmenau\oindex{Ilmenau@\textbf{Ilmenau}, \emph{P.PPL}|pw} in Weimar\oindex{Weimar@\textbf{Weimar}, \emph{A.ADM3}|pw} ein, wohnen 4 Tage im »Erbprinzen\oindex{Hotel Erbprinz@\textbf{Hotel Erbprinz}, \emph{Hotel (K.HTL)}|pw}« und ſind – hoffentlich – brav und luſtig.\pend
           
\pstart
           Ich hab heut in Wien\oindex{Wien@\textbf{Wien}, \emph{A.ADM2}|pw} mit \label{K_L00873-3v}\edtext{jemand\pwindex{Hofmannsthal, Gertrude von 16.03.1880 – 09.11.1959@\textsc{Hofmannsthal, Gertrude von} (16.03.1880 – 09.11.1959)|pwuv}}{\lemma{\textnormal{\emph{jemand}}}\Cendnote{\textnormal{Wenngleich nicht mit Sicherheit zu
                  belegen, liegt es nahe, dass er seinen Eltern\pwindex{Hofmannsthal, Hugo August von 21.12.1841 – 08.12.1915@\textsc{Hofmannsthal, Hugo August von} (21.12.1841 – 08.12.1915), \emph{Bankdirektor/Bankdirektorin}|pwkv}\pwindex{Hofmannsthal, Anna von 27.01.1849 – 22.03.1904@\textsc{Hofmannsthal, Anna von} (27.01.1849 – 22.03.1904)|pwkv} ein Treffen mit seiner späteren Frau Gerty\pwindex{Hofmannsthal, Gertrude von 16.03.1880 – 09.11.1959@\textsc{Hofmannsthal, Gertrude von} (16.03.1880 – 09.11.1959)|pwk} verheimlicht hat.}}}\label{K_L00873-3} gegeſſen und dann
               zuhaus geſagt, ich hab bei Ihnen gegeſſen. Da ich ſolche Lügen ſehr ungern hab {\pb}und auch dieſe nur halb in
               Zerſtreutheit geſagt habe, bitte dementieren Sie nicht, falls Sie zufällig meine Eltern\pwindex{Hofmannsthal, Hugo August von 21.12.1841 – 08.12.1915@\textsc{Hofmannsthal, Hugo August von} (21.12.1841 – 08.12.1915), \emph{Bankdirektor/Bankdirektorin}|pwv}\pwindex{Hofmannsthal, Anna von 27.01.1849 – 22.03.1904@\textsc{Hofmannsthal, Anna von} (27.01.1849 – 22.03.1904)|pwv}{ }ſehen.\pend
           
\pstart
           Von Herzen Ihr{\\[\baselineskip]}\spacefill\mbox{Hugo.}\pend
           \leftskip=0em{}\selectlanguage{ngerman}\endnumbering\briefempfaengerindex{Schnitzler, Arthur@\textsc{Schnitzler, Arthur}!zzzHofmannsthal, Hugo von@\emph{von Hugo von Hofmannsthal}!1899-01-011@{{[}1. 1. 1899{]}}|)be}\mylabel{L00873h}  \normalsize

\doendnotes{C}
\bigskip
\vfill

\clearpage

\footnotesize

\lohead{\textsc{register}}

% Definiere theindex-Environment komplett neu ohne reledmac
\makeatletter
\renewenvironment{theindex}{%
  \section*{\indexname}%
  \setlength{\parindent}{0pt}%
  \setlength{\parskip}{0pt plus 0.3pt}%
  \let\item\@idxitem
}{%
  \clearpage
}
\makeatother

\IfFileExists{\jobname-pw.ind}{\input{\jobname-pw.ind}}{}

\end{document}

      