%% latex-korrekturansicht-vorspann.tex
%% Vorspann für die Korrekturansicht.
%% Lädt die gemeinsame Datei latex-vorspann.tex mit gesetztem Schalter.

\newif\ifkorrekturansicht
\korrekturansichttrue

\input{../tex-inputs/latex-vorspann}


\section[Arthur Schnitzler an Robert Adam, 19. 8. 1918]{L02298 Arthur Schnitzler an Robert Adam, 19. 8. 1918}
\nopagebreak\mylabel{L02298v}
\rehead{ }\normalsize\beginnumbering\briefempfaengerindex{Adam, Robert@\textsc{Adam, Robert}!zzzSchnitzler, Arthur@\emph{von Arthur Schnitzler}!1918-08-191@{19. 8. 1918}|(be}
\toendnotes[C]{\smallbreak\pagebreak[2]}\Standort{DLA, 96.34.2/12.}
\physDesc{Brief, 1 Blatt, 2 Seiten, Umschlag, 1495 Zeichen
\newline{}Schreibmaschine
\newline{}Handschrift: schwarze Tinte, deutsche Kurrent (\noindent{}Korrektur und Nachschrift)
\newline{}Versand: Stempel: »\nobreak{}Wien, 19. VIII. 18, 3\nobreak{}«.  }\Standort{DLA, A:Schnitzler, 85.1.1621.}
\physDesc{Brief, Durchschlag2 Blätter, 2 Seiten, Umschlag, 1495 Zeichen
\newline{}Schreibmaschine
\newline{}Handschrift: Bleistift, lateinische Kurrent (\noindent{}Beschriftung »Adam« und
                                       »Kr{[}itik{]}«)}\toendnotes[C]{\smallbreak}\pstart{}{\pb}\textcolor{gray}{\textbf{Dr. Arthur Schnitzler}}\pend{}\pstart{}\textcolor{gray}{\textbf{Wien, XVIII. Sternwartestrasse 71}}\oindex{Sternwartestrasse 71@\textbf{Sternwartestraße 71}, \emph{Wohngebäude (K.WHS)}|pw}\pend{}{\bigskip}\pstart{}{\pb}Herrn Robert Adam Pollak\pend{}\pstart{}\so{Wien XII}\oindex{XII., Meidling@\textbf{XII., Meidling}, \emph{A.ADM3}|pw}.\pend{}\pstart{}Meidlinger Hauptstrasse 58\oindex{Meidlinger Hauptstrasse@\textbf{Meidlinger Hauptstraße}, \emph{Straße (K.STR)}|pw}.\pend{}{\bigskip}\vspace{1em}
\pstart
           
\pstart
           {\pb}\textcolor{gray}{\textbf{Dr. Arthur Schnitzler}}\pend
           
\pstart
           \raggedleft{}19. 8. 1918.\pend
           \pend
           
\pstart
           \textcolor{gray}{\textbf{Wien XVIII. Sternwartestrasse 71\oindex{Sternwartestrasse 71@\textbf{Sternwartestraße 71}, \emph{Wohngebäude (K.WHS)}|pw}}}\pend
           
\pstart\center{}Verehrtester Herr Doktor.\pend\vspace{0.5em}
\pstart
           Bei der Lektüre Ihres »Yppl\pwindex{Yppl. Idylle in fuenf Akten@\emph{Yppl. Idylle in fünf Akten}|pw}« habe ich mich recht
               wohlbehagt. Die Milieuschilderung ist hübsch gelungen, vielleicht etwas zu sehr
               biedermeierisch geraten, wenn auch nicht ganz ohne moderne Durchleuchtung. Die
               Charakteristik ist fein, nur der Held kommt, wie das ja so häufig der Fall ist, etwas
               blässlich heraus. Die Chargen sind am besten, besonders der Almeseder\pwindex{Yppl. Idylle in fuenf Akten@\emph{Yppl. Idylle in fünf Akten}|pwv}, auch der Hans Sachs\pwindex{Sachs, Hans 05.11.1494 – 19.01.1576@\textsc{Sachs, Hans} (05.11.1494 – 19.01.1576), \emph{Schriftsteller/Schriftstellerin}|pw}\substVorne{}\textsuperscript{sche}\substDazwischen{}hafte\substHinten{} Präsident hat mir ganz wohl gefallen.\pend
           
\pstart
           Ob sich die Idylle auf dem Theater würde behaupten können, ist schwer vorher zu
               sagen. Dazu hat sie vielleicht doch nicht Eigenart und Kraft genug. Auch bin ich
               zweifelhaft, ob die Wiederholung der Situation des 2. Aktes im 4. (Probe) glückliche
               Wirkung tun möchte. Immerhin sollten Sie einen Versuch mit dem {\pb}Stück\pwindex{Yppl. Idylle in fuenf Akten@\emph{Yppl. Idylle in fünf Akten}|pwv} machen und vielleicht
               könnte man eine kleine Bühne – ich meine eine räumlich kleine wie etwa die Kammerspiele\oindex{Kammerspiele Wien@\textbf{Kammerspiele Wien}, \emph{Theater (K.THE)}|pw} – dafür interessieren. Wenn es Ihnen
               Recht ist, will ich gerne den Regisseur Dr. Rosenthal\pwindex{Rosenthal, Friedrich 20.07.1885 – 31.08.1942@\textsc{Rosenthal, Friedrich} (20.07.1885 – 31.08.1942), \emph{Regisseur/Regisseurin, Dramaturg/Dramaturgin}|pw} auf Ihr Stück\pwindex{Yppl. Idylle in fuenf Akten@\emph{Yppl. Idylle in fünf Akten}|pwv} aufmerksam machen, das ich Ihnen hiemit mit bestem Danke zurückstelle.
               Wir reden wohl noch ausführlicher darüber. Von Mitte September an stehe
               ich gerne zur Verfügung.\pend
           
\pstart
           Herzlichst grüssend{\\[\baselineskip]}Ihr \pend
           \leftskip=0em{}
\pstart
           \noindent{}Das Stück\pwindex{Yppl. Idylle in fuenf Akten@\emph{Yppl. Idylle in fünf Akten}|pwv} liegt Ihrem
                  Wunsch gemäss zum Abholen bei mir bereit.\pend
           \selectlanguage{ngerman}\vspace{1em}
\pstart
           \noindent{}{[}hs.:{]} Vielen Dank für das Verzeichnis. Wie viel Mühe haben Sie ſich gemacht –
               ich bin ganz gerührt. Einige der Bücher würden mich ſehr intereſſieren, – beſonders
                  \label{K_L02298-1v}\edtext{\textsc{Mönckenmüller}\pwindex{Moenkemoeller, Otto 05.05.1867 – 10.05.1930@\textsc{Mönkemöller, Otto} (05.05.1867 – 10.05.1930), \emph{Mediziner/Medizinerin, Psychiater/Psychiaterin}|pw}}{\lemma{\textnormal{\emph{Mönckenmüller}}}\Cendnote{\textnormal{Vermutlich: \emph{Geistesstörung und Verbrechen im
                        Kindesalter}\pwindex{Geistesstoerung und Verbrechen im Kindesalter@\emph{Geistesstörung und Verbrechen im Kindesalter}|pwk} von Dr. Mönkemöller\pwindex{Moenkemoeller, Otto 05.05.1867 – 10.05.1930@\textsc{Mönkemöller, Otto} (05.05.1867 – 10.05.1930), \emph{Mediziner/Medizinerin, Psychiater/Psychiaterin}|pwk},
                     Oberarzt an der Provinzial-Heil- und
                        Pflegeanstalt Osnabrück\oindex{Provinzial-Heil- und Pflegeanstalt@\textbf{Provinzial-Heil- und Pflegeanstalt}, \emph{Krankenhaus (K.KKH)}|pwk}. Berlin: \emph{Verlag
                        von Reuther {\kaufmannsund} Reichard}\orgindex{Reuther und Reichard@Reuther {\kaufmannsund}  Reichard|pwk}{ }1903.}}}\label{K_L02298-1} u \label{K_L02298-2v}\edtext{\textsc{Ferrioni}\pwindex{Ferriani, Lino 6.11.1852 – 3.6.1921@\textsc{Ferriani, Lino} (6.11.1852 – 3.6.1921), \emph{Rechtswissenschaftler/Rechtswissenschaftlerin}|pw}}{\lemma{\textnormal{\emph{Ferrioni}}}\Cendnote{\textnormal{Vermutlich: \emph{Minderjährige Verbrecher. (Versuch einer
                        strafgerichtlichen Psychologie) mit Original-Gutachten von Berenini – Brusa
                        – Colajanni – Negri – Nordau – Pierantoni}\pwindex{Minderjaehrige Verbrecher. (Versuch einer strafgerichtlichen Psychologie) mit Original-Gutachten von Berenini – Brusa – Colajanni – Negri – Nordau – Pierantoni@\emph{Minderjährige Verbrecher. (Versuch einer strafgerichtlichen Psychologie) mit Original-Gutachten von Berenini – Brusa – Colajanni – Negri – Nordau – Pierantoni}|pwk}. Von Cav. Lino Ferriani\pwindex{Ferriani, Lino 6.11.1852 – 3.6.1921@\textsc{Ferriani, Lino} (6.11.1852 – 3.6.1921), \emph{Rechtswissenschaftler/Rechtswissenschaftlerin}|pwk}, Staatsanwalt in Como. Deutsch von Alfred Ruhemann\pwindex{Ruhemann, Alfred *~1856@\textsc{Ruhemann, Alfred} (*~1856), \emph{Übersetzer/Übersetzerin, Literaturwissenschaftler/Literaturwissenschaftlerin}|pwk}. Autorisierte Ausgabe.
                     Berlin: \emph{Siegfried Cronbach}\orgindex{Siegfried Cronbach@Siegfried Cronbach|pwk}{ }1896.}}}\label{K_L02298-2} – dazu nächſtens. \spacefill\mbox{A. S.}\pend
           \selectlanguage{ngerman}\endnumbering\briefempfaengerindex{Adam, Robert@\textsc{Adam, Robert}!zzzSchnitzler, Arthur@\emph{von Arthur Schnitzler}!1918-08-191@{19. 8. 1918}|)be}\mylabel{L02298h}  \normalsize

\doendnotes{C}
\bigskip
\vfill

\clearpage

\footnotesize

\lohead{\textsc{register}}

% Definiere theindex-Environment komplett neu ohne reledmac
\makeatletter
\renewenvironment{theindex}{%
  \section*{\indexname}%
  \setlength{\parindent}{0pt}%
  \setlength{\parskip}{0pt plus 0.3pt}%
  \let\item\@idxitem
}{%
  \clearpage
}
\makeatother

\IfFileExists{\jobname-pw.ind}{\input{\jobname-pw.ind}}{}

\end{document}

      