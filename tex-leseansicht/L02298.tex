%% latex-leseansicht-vorspann.tex
%% Vorspann für die Leseansicht.
%% Lädt die gemeinsame Datei latex-vorspann.tex mit nicht gesetztem Schalter.

\newif\ifkorrekturansicht
\korrekturansichtfalse

\input{../tex-inputs/latex-vorspann}


         
         \renewcommand{\erwaehntePersonen}{Personen: Robert Adam, Lino Ferriani, Otto Mönkemöller, Friedrich Rosenthal, Alfred Ruhemann, Hans Sachs}
         \renewcommand{\erwaehnteInstitutionen}{Institutionen: Reuther & Reichard, Siegfried Cronbach}
         \renewcommand{\erwaehnteOrte}{Orte: Kammerspiele Wien, Meidlinger Hauptstraße, Provinzial-Heil- und Pflegeanstalt, Sternwartestraße, Wien, XII., Meidling}
         \renewcommand{\erwaehnteWerke}{Werke: Geistesstörung und Verbrechen im Kindesalter, Minderjährige Verbrecher. (Versuch einer strafgerichtlichen Psychologie) mit Original-Gutachten von Berenini – Brusa – Colajanni – Negri – Nordau – Pierantoni, Yppl. Idylle in fünf Akten}
               \section[Arthur Schnitzler an Robert Adam, 19. 8. 1918]{ Arthur Schnitzler an Robert Adam, 19. 8. 1918}\nopagebreak\mylabel{v}\rehead{ }\begin{ledgroupsized}[t]{13cm}\normalsize\beginnumbering \toendnotes[C]{\smallbreak\pagebreak[2]} \Standort{DLA, 96.34.2/12.}
\physDesc{Brief, 1 Blatt, 2 Seiten, Umschlag
\newline{}Schreibmaschine
\newline{}Handschrift: schwarze Tinte, deutsche Kurrent (\noindent{}Korrektur und Nachschrift)\newline{}Versand: Stempel: »\nobreak{}Wien, 19. VIII. 18, 3\nobreak{}«.  }\Standort{DLA, A:Schnitzler, 85.1.1621.}
\physDesc{Brief, 2 Blätter, 2 Seiten, Umschlag, maschineller Durchschlag
\newline{}Schreibmaschine
\newline{}Handschrift: Bleistift, lateinische Kurrent (\noindent{}Beschriftung »Adam« und »Kr{[}itik{]}«)}\toendnotes[C]{\smallbreak}\pstart{}{\pb}\textcolor{gray}{\textbf{Dr. Arthur Schnitzler}}\pend{}\pstart{}\textcolor{gray}{\textbf{Wien, XVIII. Sternwartestrasse 71}}\oindex{Sternwartestrasse@\textbf{Sternwartestraße}|pw}\pend{}{\bigskip}\pstart{}{\pb}Herrn Robert Adam Pollak\pend{}\pstart{}\so{Wien XII}\oindex{XII., Meidling@\textbf{XII., Meidling}|pw}.\pend{}\pstart{}Meidlinger Hauptstrasse 58\oindex{Meidlinger Hauptstrasse@\textbf{Meidlinger Hauptstraße}|pw}.\pend{}{\bigskip}\pstart
           {\pb}\textcolor{gray}{\textbf{Dr. Arthur Schnitzler}}\hfill 19. 8. 1918.\pend
           \pstart
           \textcolor{gray}{\textbf{Wien XVIII. Sternwartestrasse 71\oindex{Sternwartestrasse@\textbf{Sternwartestraße}|pw}}}\pend
           \pstart\center{}Verehrtester Herr Doktor.\pend\pstart
           Bei der Lektüre Ihres »Yppl\pwindex{Adam, Robert 20.04.1877 – 16.10.1961@\textsc{Adam, Robert} (20.04.1877 – 16.10.1961), \emph{Schriftsteller, Richter}!Yppl. Idylle in fuenf AktenNone@\strich\emph{Yppl. Idylle in fünf Akten} {[}None{]}|pw}« habe ich mich recht
               wohlbehagt. Die Milieuschilderung ist hübsch gelungen, vielleicht etwas zu sehr
               biedermeierisch geraten, wenn auch nicht ganz ohne moderne Durchleuchtung. Die
               Charakteristik ist fein, nur der Held kommt, wie das ja so häufig der Fall ist, etwas
               blässlich heraus. Die Chargen sind am besten, besonders der Almeseder\pwindex{Adam, Robert 20.04.1877 – 16.10.1961@\textsc{Adam, Robert} (20.04.1877 – 16.10.1961), \emph{Schriftsteller, Richter}!Yppl. Idylle in fuenf AktenNone@\strich\emph{Yppl. Idylle in fünf Akten} {[}None{]}|pwv}, auch der Hans Sachs\pwindex{Sachs, Hans 05.11.1494 – 19.01.1576@\textsc{Sachs, Hans} (05.11.1494 – 19.01.1576), \emph{Schriftsteller}|pw}\substVorne{}\textsuperscript{sche}\substDazwischen{}hafte\substHinten{} Präsident hat mir ganz wohl gefallen.\pend
           \pstart
           Ob sich die Idylle auf dem Theater würde behaupten können, ist schwer vorher zu
               sagen. Dazu hat sie vielleicht doch nicht Eigenart und Kraft genug. Auch bin ich
               zweifelhaft, ob die Wiederholung der Situation des 2. Aktes im 4. (Probe) glückliche
               Wirkung tun möchte. Immerhin sollten Sie einen Versuch mit dem {\pb}Stück\pwindex{Adam, Robert 20.04.1877 – 16.10.1961@\textsc{Adam, Robert} (20.04.1877 – 16.10.1961), \emph{Schriftsteller, Richter}!Yppl. Idylle in fuenf AktenNone@\strich\emph{Yppl. Idylle in fünf Akten} {[}None{]}|pwv} machen und vielleicht
               könnte man eine kleine Bühne – ich meine eine räumlich kleine wie etwa die Kammerspiele\oindex{Kammerspiele Wien@\textbf{Kammerspiele Wien}|pw} – dafür interessieren. Wenn es Ihnen
               Recht ist, will ich gerne den Regisseur Dr. Rosenthal\pwindex{Rosenthal, Friedrich 20.07.1885 – 31.08.1942@\textsc{Rosenthal, Friedrich} (20.07.1885 – 31.08.1942), \emph{Regisseur, Dramaturg}|pw} auf Ihr Stück\pwindex{Adam, Robert 20.04.1877 – 16.10.1961@\textsc{Adam, Robert} (20.04.1877 – 16.10.1961), \emph{Schriftsteller, Richter}!Yppl. Idylle in fuenf AktenNone@\strich\emph{Yppl. Idylle in fünf Akten} {[}None{]}|pwv}
               aufmerksam machen, das ich Ihnen hiemit mit bestem Danke zurückstelle. Wir reden wohl
               noch ausführlicher darüber. Von Mitte September an stehe ich gerne zur
               Verfügung.\pend
           \pstart
           Herzlichst grüssend{\\[\baselineskip]}Ihr \pend
           \leftskip=0em{}\pstart
           \noindent{}Das Stück\pwindex{Adam, Robert 20.04.1877 – 16.10.1961@\textsc{Adam, Robert} (20.04.1877 – 16.10.1961), \emph{Schriftsteller, Richter}!Yppl. Idylle in fuenf AktenNone@\strich\emph{Yppl. Idylle in fünf Akten} {[}None{]}|pwv} liegt Ihrem Wunsch
                  gemäss zum Abholen bei mir bereit.\pend
           \pstart
           \noindent{}{[}hs.:{]} Vielen Dank für das Verzeichnis. Wie viel Mühe haben Sie ſich gemacht –
               ich bin ganz gerührt. Einige der Bücher würden mich ſehr intereſſieren, – beſonders
                  \label{K_L02298_1v}\edtext{\textsc{Mönckenmüller}\pwindex{Moenkemoeller, Otto 05.05.1867 – 10.05.1930@\textsc{Mönkemöller, Otto} (05.05.1867 – 10.05.1930), \emph{Mediziner, Psychiater}|pw}}{\lemma{\textnormal{\emph{Mönckenmüller}}}\Cendnote{\textnormal{Vermutlich: \emph{Geistesstörung und Verbrechen im Kindesalter}\pwindex{Geistesstoerung und Verbrechen im Kindesalter1903@\emph{Geistesstörung und Verbrechen im Kindesalter} {[}1903{]}|pwk}
                     von Dr. Mönkemöller\pwindex{Moenkemoeller, Otto 05.05.1867 – 10.05.1930@\textsc{Mönkemöller, Otto} (05.05.1867 – 10.05.1930), \emph{Mediziner, Psychiater}|pwk}, Oberarzt an der Provinzial-Heil- und Pflegeanstalt Osnabrück\oindex{Provinzial-Heil- und Pflegeanstalt@\textbf{Provinzial-Heil- und Pflegeanstalt}|pwk}.
                     Berlin: \emph{Verlag von Reuther {\kaufmannsund} Reichard}\orgindex{Reuther und Reichard@Reuther {\kaufmannsund}  Reichard|pwk}{ }1903.}}}\label{K_L02298_1h} u \label{K_L02298_2v}\edtext{\textsc{Ferrioni}\pwindex{Ferriani, Lino 6.11.1852 – 3.6.1921@\textsc{Ferriani, Lino} (6.11.1852 – 3.6.1921), \emph{Rechtswissenschaftler}|pw}}{\lemma{\textnormal{\emph{Ferrioni}}}\Cendnote{\textnormal{Vermutlich: \emph{Minderjährige Verbrecher. (Versuch einer
                        strafgerichtlichen Psychologie) mit Original-Gutachten von Berenini – Brusa
                        – Colajanni – Negri – Nordau – Pierantoni}\pwindex{Ferriani, Lino 6.11.1852 – 3.6.1921@\textsc{Ferriani, Lino} (6.11.1852 – 3.6.1921), \emph{Rechtswissenschaftler}!Minderjaehrige Verbrecher. (Versuch einer strafgerichtlichen Psychologie) mit Original-Gutachten von Berenini – Brusa – Colajanni – Negri – Nordau – Pierantoni1896@\strich\emph{Minderjährige Verbrecher. (Versuch einer strafgerichtlichen Psychologie) mit Original-Gutachten von Berenini – Brusa – Colajanni – Negri – Nordau – Pierantoni} {[}1896{]}|pwk}. Von Cav. Lino Ferriani\pwindex{Ferriani, Lino 6.11.1852 – 3.6.1921@\textsc{Ferriani, Lino} (6.11.1852 – 3.6.1921), \emph{Rechtswissenschaftler}|pwk}, Staatsanwalt in Como. Deutsch von Alfred Ruhemann\pwindex{Ruhemann, Alfred *~1856@\textsc{Ruhemann, Alfred} (*~1856), \emph{Übersetzer, Literaturwissenschaftler}|pwk}. Autorisierte Ausgabe.
                     Berlin: \emph{Siegfried Cronbach}\orgindex{Siegfried Cronbach@Siegfried Cronbach|pwk}{ }1896.}}}\label{K_L02298_2h} – dazu nächſtens. \spacefill\mbox{A. S.}\pend
           
         
         \endnumbering\mylabel{h}\end{ledgroupsized}  \newcommand{\dateiname}{L02298}\newcommand{\titel}{Arthur Schnitzler an Robert Adam, 19. 8. 1918}\newcommand{\editorInnen}{Martin Anton Müller und Gerd-Hermann Susen}%% latex-leseansicht-abspann.tex
%% Abspann für die Leseansicht.
%% Der Schalter \ifkorrekturansicht ist bereits durch den Vorspann gesetzt.

%% latex-abspann.tex
%% Gemeinsamer Abspann für Korrekturansicht und Leseansicht.
%% Setzt den Schalter \ifkorrekturansicht voraus (gesetzt in den
%% einbindenden Dateien latex-korrekturansicht-abspann.tex bzw.
%% latex-leseansicht-abspann.tex).
%% ---------------------------------------------------------------

\normalsize

% Das esempio-Environment wird nur in der Leseansicht benötigt
\ifkorrekturansicht\else
\newenvironment{esempio}[3]%
{
    \vspace{1.5ex}
    \rlap{\underline{#1}}
    \par
    \setlength{\parindent}{0cm}
    \nopagebreak
    \leftskip=#2cm
    \rightskip=#3cm
}
{
    \par
}
\fi

\doendnotes{C}
\bigskip
\vfill

\clearpage

\footnotesize

\ifkorrekturansicht
  \lohead{\textsc{register}}
\fi

% theindex-Environment neu definieren ohne reledmac
\makeatletter
\renewenvironment{theindex}{%
  \ifkorrekturansicht
    \section*{\indexname}%
  \else
    \subsubsection*{Index der erwähnten Entitäten}%
  \fi
  \setlength{\parindent}{0pt}%
  \setlength{\parskip}{0pt plus 0.3pt}%
  \let\item\@idxitem
}{%
  \ifkorrekturansicht\clearpage\fi
}
\makeatother

\IfFileExists{\jobname-pw.ind}{\input{\jobname-pw.ind}}{}

% Quellenangabe nur in der Leseansicht
\ifkorrekturansicht\else
% Fallback-Definitionen, falls die .tex-Datei \titel etc. nicht gesetzt hat
\providecommand{\titel}{}
\providecommand{\editorInnen}{}
\providecommand{\dateiname}{\jobname}

\vspace{3cm}

\vfill

\footnotesize
\textsc{Quelle}: \titel. Herausgegeben von {\editorInnen}. In: \emph{Arthur Schnitzler: Briefwechsel mit Autorinnen und Autoren}.
 Digitale Edition, https://schnitzler-briefe.acdh.oeaw.ac.at/{\dateiname}.html (Stand \today)
\fi

\end{document}


      