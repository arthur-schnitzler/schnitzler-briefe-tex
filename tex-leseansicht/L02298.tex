%% latex-leseansicht-vorspann.tex
%% Vorspann für die Leseansicht.
%% Lädt die gemeinsame Datei latex-vorspann.tex mit nicht gesetztem Schalter.

\newif\ifkorrekturansicht
\korrekturansichtfalse

\input{../tex-inputs/latex-vorspann}


\section[Arthur Schnitzler an Robert Adam, 19. 8. 1918]{L02298 Arthur Schnitzler an Robert Adam, 19. 8. 1918}
\nopagebreak\mylabel{L02298v}
\rehead{ }\normalsize\beginnumbering\briefempfaengerindex{Adam, Robert@\textsc{Adam, Robert}!zzzSchnitzler, Arthur@\emph{von Arthur Schnitzler}!1918-08-191@{19. 8. 1918}|(be}
\toendnotes[C]{\smallbreak\pagebreak[2]}
\correspDesc{Versand  durch Arthur Schnitzler am 19. 8. 1918 in Wien
\newline{}Erhalt  durch Robert Adam im Zeitraum [19. 8. 1918
                  – 23. 8. 1918?] in Wien}\toendnotes[C]{\smallbreak}
\Standort{DLA, 96.34.2/12.}
\physDesc{Brief, 1 Blatt, 2 Seiten, Kuvert, 1495 Zeichen
\newline{}Schreibmaschine
\newline{}Handschrift: schwarze Tinte, deutsche Kurrent (\noindent{}Korrektur und Nachschrift)
\newline{}Versand: Stempel: »\nobreak{}\oindex{Wien@\textbf{Wien}, \emph{Verwaltungsgebiet}|pwk}Wien, 19. VIII. 18, 3\nobreak{}«.  }\Standort{DLA, A:Schnitzler, 85.1.1621.}
\physDesc{Brief, Durchschlag, 2 Blätter, 2 Seiten, Kuvert, 1495 Zeichen
\newline{}Schreibmaschine
\newline{}Handschrift: Bleistift, lateinische Kurrent (\noindent{}Beschriftung »Adam« und
                                       »Kr{[}itik{]}«)}\toendnotes[C]{\smallbreak}\pstart{}{\pb}\textcolor{gray}{\textbf{Dr. Arthur Schnitzler}}\pend{}\pstart{}\textcolor{gray}{\textbf{Wien, XVIII. Sternwartestrasse 71}}\oindex{Wien@\textbf{Wien}!XVIII., Währing@\textbf{XVIII., Währing}!Sternwartestraße 71@\textbf{Sternwartestraße 71}, \emph{Wohngebäude}|pw}\pend{}{\bigskip}\pstart{}{\pb}Herrn Robert Adam Pollak\pend{}\pstart{}\so{Wien XII}\oindex{XII., Meidling@\textbf{XII., Meidling}, \emph{Verwaltungsgebiet}|pw}.\pend{}\pstart{}Meidlinger Hauptstrasse 58\oindex{Wien@\textbf{Wien}!XII., Meidling@\textbf{XII., Meidling}!Meidlinger Hauptstraße@\textbf{Meidlinger Hauptstraße}, \emph{Straße}|pw}.\pend{}{\bigskip}\vspace{1em}
\pstart
           
\pstart
           {\pb}\textcolor{gray}{\textbf{Dr. Arthur Schnitzler}}\pend
           
\pstart
           \raggedleft{}19. 8. 1918.\pend
           \pend
           
\pstart
           \textcolor{gray}{\textbf{Wien XVIII. Sternwartestrasse 71\oindex{Wien@\textbf{Wien}!XVIII., Währing@\textbf{XVIII., Währing}!Sternwartestraße 71@\textbf{Sternwartestraße 71}, \emph{Wohngebäude}|pw}}}\pend
           
\pstart\center{}Verehrtester Herr Doktor.\pend\vspace{0.5em}
\pstart
           Bei der Lektüre Ihres »Yppl\pwindex{Adam, Robert 20.\,4.\,1877 Wien – 16.\,10.\,1961 Baden bei Wien@\textsc{Adam, Robert} (20.\,4.\,1877 Wien – 16.\,10.\,1961 Baden bei Wien), \emph{Schriftsteller, Richter}!Yppl. Idylle in fünf Akten@\strich\emph{Yppl. Idylle in fünf Akten}|pw}« habe ich mich recht
               wohlbehagt. Die Milieuschilderung ist hübsch gelungen, vielleicht etwas zu sehr
               biedermeierisch geraten, wenn auch nicht ganz ohne moderne Durchleuchtung. Die
               Charakteristik ist fein, nur der Held kommt, wie das ja so häufig der Fall ist, etwas
               blässlich heraus. Die Chargen sind am besten, besonders der Almeseder\pwindex{Adam, Robert 20.\,4.\,1877 Wien – 16.\,10.\,1961 Baden bei Wien@\textsc{Adam, Robert} (20.\,4.\,1877 Wien – 16.\,10.\,1961 Baden bei Wien), \emph{Schriftsteller, Richter}!Yppl. Idylle in fünf Akten@\strich\emph{Yppl. Idylle in fünf Akten}|pwv}, auch der Hans Sachs\pwindex{Sachs, Hans 5.\,11.\,1494 Nürnberg – 19.\,1.\,1576 ebd.@\textsc{Sachs, Hans} (5.\,11.\,1494 Nürnberg – 19.\,1.\,1576 ebd.), \emph{Schriftsteller}|pw}\substVorne{}\textsuperscript{sche}\substDazwischen{}hafte\substHinten{} Präsident hat mir ganz wohl gefallen.\pend
           
\pstart
           Ob sich die Idylle auf dem Theater würde behaupten können, ist schwer vorher zu
               sagen. Dazu hat sie vielleicht doch nicht Eigenart und Kraft genug. Auch bin ich
               zweifelhaft, ob die Wiederholung der Situation des 2. Aktes im 4. (Probe) glückliche
               Wirkung tun möchte. Immerhin sollten Sie einen Versuch mit dem {\pb}Stück\pwindex{Adam, Robert 20.\,4.\,1877 Wien – 16.\,10.\,1961 Baden bei Wien@\textsc{Adam, Robert} (20.\,4.\,1877 Wien – 16.\,10.\,1961 Baden bei Wien), \emph{Schriftsteller, Richter}!Yppl. Idylle in fünf Akten@\strich\emph{Yppl. Idylle in fünf Akten}|pwv} machen und vielleicht
               könnte man eine kleine Bühne – ich meine eine räumlich kleine wie etwa die Kammerspiele\oindex{Wien@\textbf{Wien}!I., Innere Stadt@\textbf{I., Innere Stadt}!Kammerspiele Wien@\textbf{Kammerspiele Wien}, \emph{Theater}|pw} – dafür interessieren. Wenn es Ihnen
               Recht ist, will ich gerne den Regisseur Dr. Rosenthal\pwindex{Rosenthal, Friedrich 20.\,7.\,1885 Wien – 31.\,8.\,1942 Konzentrationslager Auschwitz-Birkenau@\textsc{Rosenthal, Friedrich} (20.\,7.\,1885 Wien – 31.\,8.\,1942 Konzentrationslager Auschwitz-Birkenau), \emph{Regisseur, Dramaturg}|pw} auf Ihr Stück\pwindex{Adam, Robert 20.\,4.\,1877 Wien – 16.\,10.\,1961 Baden bei Wien@\textsc{Adam, Robert} (20.\,4.\,1877 Wien – 16.\,10.\,1961 Baden bei Wien), \emph{Schriftsteller, Richter}!Yppl. Idylle in fünf Akten@\strich\emph{Yppl. Idylle in fünf Akten}|pwv} aufmerksam machen, das ich Ihnen hiemit mit bestem Danke zurückstelle.
               Wir reden wohl noch ausführlicher darüber. Von Mitte September an stehe
               ich gerne zur Verfügung.\pend
           
\pstart
           Herzlichst grüssend{\\[\baselineskip]}Ihr\pend
           \leftskip=0em{}
\pstart
           \noindent{}Das Stück\pwindex{Adam, Robert 20.\,4.\,1877 Wien – 16.\,10.\,1961 Baden bei Wien@\textsc{Adam, Robert} (20.\,4.\,1877 Wien – 16.\,10.\,1961 Baden bei Wien), \emph{Schriftsteller, Richter}!Yppl. Idylle in fünf Akten@\strich\emph{Yppl. Idylle in fünf Akten}|pwv} liegt Ihrem
                  Wunsch gemäss zum Abholen bei mir bereit.\pend
           \selectlanguage{ngerman}\vspace{1em}
\pstart
           \noindent{}{[}hs.:{]} Vielen Dank für das Verzeichnis. Wie viel Mühe haben Sie{ }ſich gemacht –
               ich bin ganz gerührt. Einige der Bücher würden mich{ }ſehr intereſſieren, – beſonders
                  \label{K_L02298-1v}\edtext{\textsc{Mönckenmüller}\pwindex{Mönkemöller, Otto 5.\,5.\,1867 Bonn – 10.\,5.\,1930 Hildesheim@\textsc{Mönkemöller, Otto} (5.\,5.\,1867 Bonn – 10.\,5.\,1930 Hildesheim), \emph{Mediziner, Psychiater}|pw}}{\lemma{\textnormal{\emph{Mönckenmüller}}}\Cendnote{\textnormal{Vermutlich: \emph{Geistesstörung und Verbrechen im
                        Kindesalter}\pwindex{Geistesstörung und Verbrechen im Kindesalter@\emph{Geistesstörung und Verbrechen im Kindesalter}|pwk} von Dr. Mönkemöller\pwindex{Mönkemöller, Otto 5.\,5.\,1867 Bonn – 10.\,5.\,1930 Hildesheim@\textsc{Mönkemöller, Otto} (5.\,5.\,1867 Bonn – 10.\,5.\,1930 Hildesheim), \emph{Mediziner, Psychiater}|pwk},
                     Oberarzt an der Provinzial-Heil- und
                        Pflegeanstalt Osnabrück\oindex{Provinzial-Heil- und Pflegeanstalt@\textbf{Provinzial-Heil- und Pflegeanstalt}, \emph{Krankenhaus}|pwk}. Berlin: \emph{Verlag
                        von Reuther {\kaufmannsund} Reichard}\orgindex{Reuther und Reichard@Reuther {\kaufmannsund}  Reichard|pwk}{ }1903.}}}\label{K_L02298-1} u \label{K_L02298-2v}\edtext{\textsc{Ferrioni}\pwindex{Ferriani, Lino 6.\,11.\,1852 Ferrara – 3.\,6.\,1921 Como@\textsc{Ferriani, Lino} (6.\,11.\,1852 Ferrara – 3.\,6.\,1921 Como), \emph{Rechtswissenschaftler}|pw}}{\lemma{\textnormal{\emph{Ferrioni}}}\Cendnote{\textnormal{Vermutlich: \emph{Minderjährige Verbrecher. (Versuch einer
                        strafgerichtlichen Psychologie) mit Original-Gutachten von Berenini – Brusa
                        – Colajanni – Negri – Nordau – Pierantoni}\pwindex{Ferriani, Lino 6.\,11.\,1852 Ferrara – 3.\,6.\,1921 Como@\textsc{Ferriani, Lino} (6.\,11.\,1852 Ferrara – 3.\,6.\,1921 Como), \emph{Rechtswissenschaftler}!Minderjährige Verbrecher. (Versuch einer strafgerichtlichen Psychologie) mit Original-Gutachten von Berenini – Brusa – Colajanni – Negri – Nordau – Pierantoni@\strich\emph{Minderjährige Verbrecher. (Versuch einer strafgerichtlichen Psychologie) mit Original-Gutachten von Berenini – Brusa – Colajanni – Negri – Nordau – Pierantoni}|pwk}. Von Cav. Lino Ferriani\pwindex{Ferriani, Lino 6.\,11.\,1852 Ferrara – 3.\,6.\,1921 Como@\textsc{Ferriani, Lino} (6.\,11.\,1852 Ferrara – 3.\,6.\,1921 Como), \emph{Rechtswissenschaftler}|pwk}, Staatsanwalt in Como. Deutsch von Alfred Ruhemann\pwindex{Ruhemann, Alfred *~1856@\textsc{Ruhemann, Alfred} (*~1856), \emph{Übersetzer, Literaturwissenschaftler}|pwk}. Autorisierte Ausgabe.
                     Berlin: \emph{Siegfried Cronbach}\orgindex{Siegfried Cronbach@Siegfried Cronbach|pwk}{ }1896.}}}\label{K_L02298-2} – dazu nächſtens. \spacefill\mbox{A. S.}\pend
           \selectlanguage{ngerman}\endnumbering\briefempfaengerindex{Adam, Robert@\textsc{Adam, Robert}!zzzSchnitzler, Arthur@\emph{von Arthur Schnitzler}!1918-08-191@{19. 8. 1918}|)be}\mylabel{L02298h}  \newcommand{\dateiname}{L02298}\newcommand{\titel}{Arthur Schnitzler an Robert Adam, 19. 8. 1918}\newcommand{\editorInnen}{Martin Anton Müller und Gerd-Hermann Susen}%% latex-leseansicht-abspann.tex
%% Abspann für die Leseansicht.
%% Der Schalter \ifkorrekturansicht ist bereits durch den Vorspann gesetzt.

%% latex-abspann.tex
%% Gemeinsamer Abspann für Korrekturansicht und Leseansicht.
%% Setzt den Schalter \ifkorrekturansicht voraus (gesetzt in den
%% einbindenden Dateien latex-korrekturansicht-abspann.tex bzw.
%% latex-leseansicht-abspann.tex).
%% ---------------------------------------------------------------

\normalsize

% Das esempio-Environment wird nur in der Leseansicht benötigt
\ifkorrekturansicht\else
\newenvironment{esempio}[3]%
{
    \vspace{1.5ex}
    \rlap{\underline{#1}}
    \par
    \setlength{\parindent}{0cm}
    \nopagebreak
    \leftskip=#2cm
    \rightskip=#3cm
}
{
    \par
}
\fi

\doendnotes{C}
\bigskip
\vfill

\clearpage

\footnotesize

\ifkorrekturansicht
  \lohead{\textsc{register}}
\fi

% theindex-Environment neu definieren ohne reledmac
\makeatletter
\renewenvironment{theindex}{%
  \ifkorrekturansicht
    \section*{\indexname}%
  \else
    \subsubsection*{Index der erwähnten Entitäten}%
  \fi
  \setlength{\parindent}{0pt}%
  \setlength{\parskip}{0pt plus 0.3pt}%
  \let\item\@idxitem
}{%
  \ifkorrekturansicht\clearpage\fi
}
\makeatother

\IfFileExists{\jobname-pw.ind}{\input{\jobname-pw.ind}}{}

% Quellenangabe nur in der Leseansicht
\ifkorrekturansicht\else
% Fallback-Definitionen, falls die .tex-Datei \titel etc. nicht gesetzt hat
\providecommand{\titel}{}
\providecommand{\editorInnen}{}
\providecommand{\dateiname}{\jobname}

\vspace{3cm}

\vfill

\footnotesize
\textsc{Quelle}: \titel. Herausgegeben von {\editorInnen}. In: \emph{Arthur Schnitzler: Briefwechsel mit Autorinnen und Autoren}.
 Digitale Edition, https://schnitzler-briefe.acdh.oeaw.ac.at/{\dateiname}.html (Stand \today)
\fi

\end{document}


