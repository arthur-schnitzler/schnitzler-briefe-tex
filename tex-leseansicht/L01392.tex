%% latex-korrekturansicht-vorspann.tex
%% Vorspann für die Korrekturansicht.
%% Lädt die gemeinsame Datei latex-vorspann.tex mit gesetztem Schalter.

\newif\ifkorrekturansicht
\korrekturansichttrue

\input{../tex-inputs/latex-vorspann}


\section[Hermann Bahr an Arthur Schnitzler, 23. 4. {[}1904{]}]{L01392 Hermann Bahr an Arthur Schnitzler, 23. 4. {[}1904{]}}
\nopagebreak\mylabel{L01392v}
\rehead{ }\normalsize\beginnumbering\briefempfaengerindex{Schnitzler, Arthur@\textsc{Schnitzler, Arthur}!zzzBahr, Hermann@\emph{von Hermann Bahr}!1904-04-231@{23. 4. {[}1904{]}}|(be}
\toendnotes[C]{\smallbreak\pagebreak[2]}\Standort{CUL, Schnitzler, B 5b.}
\physDesc{Brief, 1 Blatt, 1 Seite, 328 Zeichen
\newline{}Handschrift: schwarze Tinte, deutsche Kurrent
\newline{}Schnitzler: mit Bleistift Jahreszahl ergänzt: »904« 
\newline{}Ordnung: mit Bleistift von unbekannter Hand nummeriert: »115« }
\buchAbdrucke{\weitereDrucke{Hermann Bahr, Arthur Schnitzler: \emph{Briefwechsel, Aufzeichnungen, Dokumente (1891–1931)}. Göttingen: \emph{Wallstein} 2018, S. 306.} }\toendnotes[C]{\smallbreak}
\pstart
           \raggedleft{}{\pb}23. 4.\pend
           
\pstart\center{}Lieber Arthur!\pend\vspace{0.5em}
\pstart
           Ich bin zurück, möchte Dich bald ſehen, höre leider, daß man nicht zu Dir darf, hoffe
               den Jüngling\pwindex{Schnitzler, Heinrich 09.08.1902 – 12.07.1982@\textsc{Schnitzler, Heinrich} (09.08.1902 – 12.07.1982), \emph{Regisseur/Regisseurin, Schauspieler/Schauspielerin}|pwv} jedoch bald
                  \label{K_L01392-1v}\edtext{geneſen}{\lemma{\textnormal{\emph{geneſen}}}\Cendnote{\textnormal{Heinrich\pwindex{Schnitzler, Heinrich 09.08.1902 – 12.07.1982@\textsc{Schnitzler, Heinrich} (09.08.1902 – 12.07.1982), \emph{Regisseur/Regisseurin, Schauspieler/Schauspielerin}|pwk} hatte die Masern, vgl. A. S.: \emph{Tagebuch}, 12. 4. 1904.}}}\label{K_L01392-1} und bitte
               Dich dann um ein Wort, wann ich Dich treffe.\pend
           
\pstart
           Mit vielen Grüßen an Deine Frau\pwindex{Schnitzler, Olga 17.01.1882 – 13.01.1970@\textsc{Schnitzler, Olga} (17.01.1882 – 13.01.1970), \emph{Schauspieler/Schauspielerin, Sänger/Sängerin}|pwv}{\\[\baselineskip]}herzlichſt{\\[\baselineskip]}Dein{\\[\baselineskip]}\spacefill\mbox{Hermann}\pend
           \leftskip=0em{}
\pstart
           \noindent{}Über Deinen Pariſ\oindex{Paris@\textbf{Paris}, \emph{P.PPLC}|pw}er \label{K_L01392-2v}\edtext{Rieſenerfolg\pwindex{Abschiedssouper@\emph{Abschiedssouper}|pwv}}{\lemma{\textnormal{\emph{Rieſenerfolg}}}\Cendnote{\textnormal{Vgl. Hermann Bahr, Arthur Schnitzler: \emph{Briefwechsel, Aufzeichnungen, Dokumente (1891–1931)}, Stephan Epstein an Hermann Bahr, 15. 2. 1904.
                  }}}\label{K_L01392-2},
                  von dem D\textsuperscript{r}{ }Epſtein\pwindex{Epstein, Stephan 12.11.1866 – 1941@\textsc{Epstein, Stephan} (12.11.1866 – 1941), \emph{Schriftsteller/Schriftstellerin, Dramaturg/Dramaturgin, Übersetzer/Übersetzerin}|pw} erzälte, hab ich mich ſo ſehr
                  gefreut.\pend
           \selectlanguage{ngerman}\endnumbering\briefempfaengerindex{Schnitzler, Arthur@\textsc{Schnitzler, Arthur}!zzzBahr, Hermann@\emph{von Hermann Bahr}!1904-04-231@{23. 4. {[}1904{]}}|)be}\mylabel{L01392h}  \normalsize

\doendnotes{C}
\bigskip
\vfill

\clearpage

\footnotesize

\lohead{\textsc{register}}

% Definiere theindex-Environment komplett neu ohne reledmac
\makeatletter
\renewenvironment{theindex}{%
  \section*{\indexname}%
  \setlength{\parindent}{0pt}%
  \setlength{\parskip}{0pt plus 0.3pt}%
  \let\item\@idxitem
}{%
  \clearpage
}
\makeatother

\IfFileExists{\jobname-pw.ind}{\input{\jobname-pw.ind}}{}

\end{document}

      