%% latex-korrekturansicht-vorspann.tex
%% Vorspann für die Korrekturansicht.
%% Lädt die gemeinsame Datei latex-vorspann.tex mit gesetztem Schalter.

\newif\ifkorrekturansicht
\korrekturansichttrue

\input{../tex-inputs/latex-vorspann}


\section[ Felix Salten an Arthur Schnitzler, {[}23. 6. 1896{]}]{L03173 Felix Salten an Arthur Schnitzler, {[}23. 6. 1896{]}}
\nopagebreak\mylabel{L03173v}
\rehead{ }\normalsize\beginnumbering\briefempfaengerindex{Schnitzler, Arthur@\textsc{Schnitzler, Arthur}!zzzSalten, Felix@\emph{von Felix Salten}!1896-06-231@{{[}23. 6. 1896{]}}|(be}
\toendnotes[C]{\smallbreak\pagebreak[2]}\Standort{CUL, Schnitzler, B 89, A 1.}
\physDesc{Brief, 1 Blatt, 1 Seite, 95 Zeichen
\newline{}Handschrift: Bleistift, lateinische Kurrent
\newline{}Schnitzler: mit Bleistift datiert: »23/6 96« 
\newline{}Ordnung: mit Bleistift von unbekannter Hand nummeriert: »72« }\toendnotes[C]{\smallbreak}
\pstart{}{\pb}lieber Arthur\pend\vspace{0.5em}
\pstart
           Wir\pwindex{Salten, Ottilie 07.03.1868 – 22.06.1942@\textsc{Salten, Ottilie} (07.03.1868 – 22.06.1942), \emph{Schauspieler/Schauspielerin}|pwv} sind beim Domayer\oindex{Cafe Dommayer@\textbf{Café Dommayer}, \emph{Kaffeehaus (K.KAF)}|pw} in Hietzing\oindex{XIII., Hietzing@\textbf{XIII., Hietzing}, \emph{A.ADM3}|pw} und ich bitte Sie \uline{besti{\geminationm}t} dahin zu \label{K_L03173-1v}\edtext{kommen}{\lemma{\textnormal{\emph{kommen}}}\Cendnote{\textnormal{Siehe A. S.: \emph{Tagebuch}, 23. 6. 1896.
               }}}\label{K_L03173-1}.\pend
           
\pstart
           Herzlich {\\[\baselineskip]}\spacefill\mbox{Salten}\pend
           \leftskip=0em{}\selectlanguage{ngerman}\endnumbering\briefempfaengerindex{Schnitzler, Arthur@\textsc{Schnitzler, Arthur}!zzzSalten, Felix@\emph{von Felix Salten}!1896-06-231@{{[}23. 6. 1896{]}}|)be}\mylabel{L03173h}  \normalsize

\doendnotes{C}
\bigskip
\vfill

\clearpage

\footnotesize

\lohead{\textsc{register}}

% Definiere theindex-Environment komplett neu ohne reledmac
\makeatletter
\renewenvironment{theindex}{%
  \section*{\indexname}%
  \setlength{\parindent}{0pt}%
  \setlength{\parskip}{0pt plus 0.3pt}%
  \let\item\@idxitem
}{%
  \clearpage
}
\makeatother

\IfFileExists{\jobname-pw.ind}{\input{\jobname-pw.ind}}{}

\end{document}

      