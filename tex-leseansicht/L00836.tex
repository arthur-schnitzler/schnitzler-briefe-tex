%% latex-leseansicht-vorspann.tex
%% Vorspann für die Leseansicht.
%% Lädt die gemeinsame Datei latex-vorspann.tex mit nicht gesetztem Schalter.

\newif\ifkorrekturansicht
\korrekturansichtfalse

\input{../tex-inputs/latex-vorspann}


\section[Arthur Schnitzler an Richard Beer-Hofmann, 22. 8. 1898]{L00836 Arthur Schnitzler an Richard Beer-Hofmann, 22. 8. 1898}
\nopagebreak\mylabel{L00836v}
\rehead{ }\normalsize\beginnumbering\briefempfaengerindex{Beer-Hofmann, Richard@\textsc{Beer-Hofmann, Richard}!zzzSchnitzler, Arthur@\emph{von Arthur Schnitzler}!1898-08-221@{22. 8. 1898}|(be}
\toendnotes[C]{\smallbreak\pagebreak[2]}
\correspDesc{Versand  durch Arthur Schnitzler am 22. 8. 1898 in Luzern
\newline{}Erhalt  durch Richard Beer-Hofmann am 24. 8. 1898 in Steindorf am Ossiacher See}\toendnotes[C]{\smallbreak}
\Standort{YCGL, MSS 31.}
\physDesc{Postkarte, 456 Zeichen
\newline{}Handschrift: Bleistift, deutsche Kurrent
\newline{}Versand: 1) Stempel: »\nobreak{}\oindex{Luzern@\textbf{Luzern}|pwk}Luzern, 22 VIII 98, 4\nobreak{}«.   2) Stempel: »\nobreak{}\oindex{Steindorf am Ossiacher See@\textbf{Steindorf am Ossiacher See}, \emph{Verwaltungsgebiet}|pwk}\textcolor{gray}{Steindorf} am Ossiacher See, 24 \textcolor{gray}{8 98}\nobreak{}«. }\pstart{}{\pb}Herrn \textsc{Dr. Richard
                     Beer-Hofmann}\pend{}\pstart{}\textsc{Steindorf am Ossiacher}ſee\oindex{Steindorf am Ossiacher See@\textbf{Steindorf am Ossiacher See}, \emph{Verwaltungsgebiet}|pw}.\pend{}\pstart{}\textsc{Kärnthen}\oindex{Kärnten@\textbf{Kärnten}, \emph{Land}|pw}.\pend{}{\bigskip}\vspace{1em}
\pstart
           \raggedleft{}{\pb}\textsc{Luzern}\oindex{Luzern@\textbf{Luzern}|pw}, 22. 8. 98\pend
           \vspace{0.5em}
\pstart
           Nach einer{ }ſehr{ }ſchönen Tour bis Genf\oindex{Genf@\textbf{Genf}|pw} hat{ }ſich
                  Hugo\pwindex{Hofmannsthal, Hugo von 1.\,2.\,1874 Wien – 15.\,7.\,1929 Rodaun@\textsc{Hofmannsthal, Hugo von} (1.\,2.\,1874 Wien – 15.\,7.\,1929 Rodaun), \emph{Schriftsteller}|pw} nach Lugano\oindex{Lugano@\textbf{Lugano}, \emph{Hauptstadt}|pw} und ich, in prachtvollen Fahrten durchs \textsc{Berner} Oberland\oindex{Berner Oberland@\textbf{Berner Oberland}|pw}, hieher gewandt, wo ich vielleicht
               acht Tage bleibe, um da{\geminationn}, möglicherweiſe streckenweiſe
               per Rad zurück nach Wien\oindex{Wien@\textbf{Wien}, \emph{Verwaltungsgebiet}|pw} zurück zu reiſen. Es geht
               mir gut; nach Arbeit\introOben{}en\introOben{}{ }ſehne ich mich ein bischen; gerne hätt ich eine
               Nachricht von Ihnen; hieher \textsc{post rest}. Von Herzen Ihr
                  \spacefill\mbox{A.}\pend
           \selectlanguage{ngerman}\endnumbering\briefempfaengerindex{Beer-Hofmann, Richard@\textsc{Beer-Hofmann, Richard}!zzzSchnitzler, Arthur@\emph{von Arthur Schnitzler}!1898-08-221@{22. 8. 1898}|)be}\mylabel{L00836h}  \newcommand{\dateiname}{L00836}\newcommand{\titel}{Arthur Schnitzler an Richard Beer-Hofmann, 22. 8. 1898}\newcommand{\editorInnen}{Martin Anton Müller und Gerd-Hermann Susen}%% latex-leseansicht-abspann.tex
%% Abspann für die Leseansicht.
%% Der Schalter \ifkorrekturansicht ist bereits durch den Vorspann gesetzt.

%% latex-abspann.tex
%% Gemeinsamer Abspann für Korrekturansicht und Leseansicht.
%% Setzt den Schalter \ifkorrekturansicht voraus (gesetzt in den
%% einbindenden Dateien latex-korrekturansicht-abspann.tex bzw.
%% latex-leseansicht-abspann.tex).
%% ---------------------------------------------------------------

\normalsize

% Das esempio-Environment wird nur in der Leseansicht benötigt
\ifkorrekturansicht\else
\newenvironment{esempio}[3]%
{
    \vspace{1.5ex}
    \rlap{\underline{#1}}
    \par
    \setlength{\parindent}{0cm}
    \nopagebreak
    \leftskip=#2cm
    \rightskip=#3cm
}
{
    \par
}
\fi

\doendnotes{C}
\bigskip
\vfill

\clearpage

\footnotesize

\ifkorrekturansicht
  \lohead{\textsc{register}}
\fi

% theindex-Environment neu definieren ohne reledmac
\makeatletter
\renewenvironment{theindex}{%
  \ifkorrekturansicht
    \section*{\indexname}%
  \else
    \subsubsection*{Index der erwähnten Entitäten}%
  \fi
  \setlength{\parindent}{0pt}%
  \setlength{\parskip}{0pt plus 0.3pt}%
  \let\item\@idxitem
}{%
  \ifkorrekturansicht\clearpage\fi
}
\makeatother

\IfFileExists{\jobname-pw.ind}{\input{\jobname-pw.ind}}{}

% Quellenangabe nur in der Leseansicht
\ifkorrekturansicht\else
% Fallback-Definitionen, falls die .tex-Datei \titel etc. nicht gesetzt hat
\providecommand{\titel}{}
\providecommand{\editorInnen}{}
\providecommand{\dateiname}{\jobname}

\vspace{3cm}

\vfill

\footnotesize
\textsc{Quelle}: \titel. Herausgegeben von {\editorInnen}. In: \emph{Arthur Schnitzler: Briefwechsel mit Autorinnen und Autoren}.
 Digitale Edition, https://schnitzler-briefe.acdh.oeaw.ac.at/{\dateiname}.html (Stand \today)
\fi

\end{document}


