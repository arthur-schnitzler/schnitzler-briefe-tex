%% latex-korrekturansicht-vorspann.tex
%% Vorspann für die Korrekturansicht.
%% Lädt die gemeinsame Datei latex-vorspann.tex mit gesetztem Schalter.

\newif\ifkorrekturansicht
\korrekturansichttrue

\input{../tex-inputs/latex-vorspann}


\section[Arthur Schnitzler an Richard Beer-Hofmann, 22. 8. 1898]{L00836 Arthur Schnitzler an Richard Beer-Hofmann, 22. 8. 1898}
\nopagebreak\mylabel{L00836v}
\rehead{ }\normalsize\beginnumbering\briefempfaengerindex{Beer-Hofmann, Richard@\textsc{Beer-Hofmann, Richard}!zzzSchnitzler, Arthur@\emph{von Arthur Schnitzler}!1898-08-221@{22. 8. 1898}|(be}
\toendnotes[C]{\smallbreak\pagebreak[2]}\Standort{YCGL, MSS 31.}
\physDesc{Postkarte, 456 Zeichen
\newline{}Handschrift: Bleistift, deutsche Kurrent
\newline{}Versand: 1) Stempel: »\nobreak{}\oindex{Luzern@\textbf{Luzern}, \emph{P.PPLA}|pwk}Luzern, 22 VIII 98, 4\nobreak{}«.   2) Stempel: »\nobreak{}\oindex{Steindorf am Ossiacher See@\textbf{Steindorf am Ossiacher See}, \emph{A.ADM3}|pwk}\textcolor{gray}{Steindorf} am Ossiacher See, 24 \textcolor{gray}{8 98}\nobreak{}«. }\pstart{}{\pb}Herrn \textsc{Dr. Richard
                     Beer-Hofmann}\pend{}\pstart{}\textsc{Steindorf am Ossiacher}ſee\oindex{Steindorf am Ossiacher See@\textbf{Steindorf am Ossiacher See}, \emph{A.ADM3}|pw}.\pend{}\pstart{}\textsc{Kärnthen}\oindex{Kaernten@\textbf{Kärnten}, \emph{A.ADM1}|pw}.\pend{}{\bigskip}\vspace{1em}
\pstart
           \raggedleft{}{\pb}\textsc{Luzern}\oindex{Luzern@\textbf{Luzern}, \emph{P.PPLA}|pw}, 22. 8. 98\pend
           \vspace{0.5em}
\pstart
           Nach einer ſehr ſchönen Tour bis Genf\oindex{Genf@\textbf{Genf}, \emph{P.PPLA}|pw} hat ſich
                  Hugo\pwindex{Hofmannsthal, Hugo von 1874-02-01 – 1929-07-15@\textsc{Hofmannsthal, Hugo von} (1874-02-01 – 1929-07-15), \emph{Schriftsteller/Schriftstellerin}|pw} nach Lugano\oindex{Lugano@\textbf{Lugano}, \emph{P.PPLA2}|pw} und ich, in prachtvollen Fahrten durchs \textsc{Berner} Oberland\oindex{Berner Oberland@\textbf{Berner Oberland}, \emph{Teil eines Landes (A.LNDX)}|pw}, hieher gewandt, wo ich vielleicht
               acht Tage bleibe, um da{\geminationn}, möglicherweiſe streckenweiſe
               per Rad zurück nach Wien\oindex{Wien@\textbf{Wien}, \emph{A.ADM2}|pw} zurück zu reiſen. Es geht
               mir gut; nach Arbeit\introOben{}en\introOben{}{ }ſehne ich mich ein bischen; gerne hätt ich eine
               Nachricht von Ihnen; hieher \textsc{post rest}. Von Herzen Ihr
                  \spacefill\mbox{A.}\pend
           \selectlanguage{ngerman}\endnumbering\briefempfaengerindex{Beer-Hofmann, Richard@\textsc{Beer-Hofmann, Richard}!zzzSchnitzler, Arthur@\emph{von Arthur Schnitzler}!1898-08-221@{22. 8. 1898}|)be}\mylabel{L00836h}  \normalsize

\doendnotes{C}
\bigskip
\vfill

\clearpage

\footnotesize

\lohead{\textsc{register}}

% Definiere theindex-Environment komplett neu ohne reledmac
\makeatletter
\renewenvironment{theindex}{%
  \section*{\indexname}%
  \setlength{\parindent}{0pt}%
  \setlength{\parskip}{0pt plus 0.3pt}%
  \let\item\@idxitem
}{%
  \clearpage
}
\makeatother

\IfFileExists{\jobname-pw.ind}{\input{\jobname-pw.ind}}{}

\end{document}

      