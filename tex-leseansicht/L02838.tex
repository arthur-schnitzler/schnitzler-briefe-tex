%% latex-leseansicht-vorspann.tex
%% Vorspann für die Leseansicht.
%% Lädt die gemeinsame Datei latex-vorspann.tex mit nicht gesetztem Schalter.

\newif\ifkorrekturansicht
\korrekturansichtfalse

\input{../tex-inputs/latex-vorspann}


         
         \renewcommand{\erwaehntePersonen}{Personen: Georg Brandes, Alfred Dreyfus, Marie Glümer, Paul Goldmann, Leopold Sonnemann, Émile Zola}
         \renewcommand{\erwaehnteInstitutionen}{Institutionen: Frankfurter Zeitung, Illustrirtes Wiener Extrablatt}
         \renewcommand{\erwaehnteOrte}{Orte: Carl-Theater, Paris, Salzburg, Wien, rue de la Bourse}
         \renewcommand{\erwaehnteWerke}{Werke: Freiwild. Schauspiel in 3 Akten, Illustrirtes Wiener Extrablatt, J’accuse…{\rufezeichen}, Theaterzeitung. Carltheater [Freiwild]}
               \section[ Paul Goldmann an Arthur Schnitzler, 6. 2. {[}1898{]}]{ Paul Goldmann an Arthur Schnitzler, 6. 2. {[}1898{]}}\nopagebreak\mylabel{v}\rehead{ }\begin{ledgroupsized}[t]{13cm}\normalsize\beginnumbering \toendnotes[C]{\smallbreak\pagebreak[2]} \Standort{DLA, A:Schnitzler, HS.NZ85.1.3168.}
\physDesc{Brief, 1 Blatt, 2 Seiten, 758 Zeichen
\newline{}Handschrift: blaue Tinte, deutsche Kurrent
\newline{}Schnitzler: 1) mit schwarzer Tinte das Jahr »98« vermerkt  2) mit rotem Buntstift zwei Unterstreichungen}\toendnotes[C]{\smallbreak}\pstart
           \noindent{}{\pb}\textcolor{gray}{\textbf{\textbf{Frankfurter Zeitung\orgindex{Frankfurter Zeitung@Frankfurter Zeitung|pw}}}}\pend
           \pstart
           \textcolor{gray}{\textbf{(\begin{otherlanguage}{french}Gazette de Francfort\end{otherlanguage}\orgindex{Frankfurter Zeitung@Frankfurter Zeitung|pw}).}}\pend
           \pstart
           \textcolor{gray}{\textbf{\textbf{\begin{otherlanguage}{french}Fondateur M.\end{otherlanguage}{ }L. Sonnemann\pwindex{Sonnemann, Leopold 1831-10-29 – 1909-10-30@\textsc{Sonnemann, Leopold} (1831-10-29 – 1909-10-30), \emph{Journalist, Herausgeber}|pw}.}}}\pend
           \pstart
           \begin{otherlanguage}{french}\textcolor{gray}{\textbf{Journal politique, financier,}}\end{otherlanguage}\pend
           \pstart
           \begin{otherlanguage}{french}\textcolor{gray}{\textbf{commercial et littéraire.}}\end{otherlanguage}\pend
           \pstart
           \begin{otherlanguage}{french}\textcolor{gray}{\textbf{\textbf{Paraissant trois fois par jour.}}}\end{otherlanguage}\hfill \textsc{Paris\oindex{Paris@\textbf{Paris}|pw}}, 6. Februar.\pend
           \pstart
           \begin{otherlanguage}{french}\textcolor{gray}{\textbf{\textbf{Bureau à Paris\oindex{Paris@\textbf{Paris}|pw}}}}\end{otherlanguage}\pend
           \pstart
           \begin{otherlanguage}{french}\textcolor{gray}{\textbf{\textbf{10 \so{Rue de la Bourse}\oindex{rue de la Bourse@\textbf{rue de la Bourse}|pw}.}}}\end{otherlanguage}\pend
           \pstart\center{}Mein lieber Freund,\pend\pstart
           Ich bin in tollſter Arbeit. Morgen beginnt der \label{K_L02838-1v}\edtext{Prozeß \textsc{Zola\pwindex{Zola, Emile 02.04.1840 – 29.09.1902@\textsc{Zola, Émile} (02.04.1840 – 29.09.1902), \emph{Schriftsteller, Journalist}|pw}}}{\lemma{\textnormal{\emph{Prozeß Zola}}}\Cendnote{\textnormal{Am 13. 1. 1898 hatte Émile Zola\pwindex{Zola, Emile 02.04.1840 – 29.09.1902@\textsc{Zola, Émile} (02.04.1840 – 29.09.1902), \emph{Schriftsteller, Journalist}|pwk}
                  seinen offenen Brief \emph{J’accuse…!}\pwindex{Zola, Emile 02.04.1840 – 29.09.1902@\textsc{Zola, Émile} (02.04.1840 – 29.09.1902), \emph{Schriftsteller, Journalist}!J accuse…1898-01-13@\strich\emph{J’accuse…{\rufezeichen}} {[}1898-01-13{]}|pwk}
                  veröffentlicht, in dem er offen für Dreyfus\pwindex{Dreyfus, Alfred 1859-10-09 – 1935-07-12@\textsc{Dreyfus, Alfred} (1859-10-09 – 1935-07-12), \emph{Militär}|pwk}
                  Partei ergriff. Nach einem Verleumdungsprozess, der zwischen 7. 2. 1898 und 23. 2. 1898 abgehalten wurde, wurde Zola\pwindex{Zola, Emile 02.04.1840 – 29.09.1902@\textsc{Zola, Émile} (02.04.1840 – 29.09.1902), \emph{Schriftsteller, Journalist}|pwk} zu einer einjährigen Haftstrafe verurteilt. Der Verhaftung entkam er
                  durch eine Flucht ins Exil, wo er bis zu seiner Begnadigung nach zwei Jahren
                  blieb. Für Dreyfus\pwindex{Dreyfus, Alfred 1859-10-09 – 1935-07-12@\textsc{Dreyfus, Alfred} (1859-10-09 – 1935-07-12), \emph{Militär}|pwk} brachte die öffentliche
                  Anprangerung des Unrechts, das ihm angetan wurde, den Wendepunkt, der letztlich zu
                  seiner Entlassung aus der Gefangenschaft führte und klärte, dass er das Opfer
                  eines Justizversagens geworden war.}}}\label{K_L02838-1h}. Ich habe nur eine Minute, um Dich zu
               dem neuen ſchönen \label{K_L02838-2v}\edtext{Erfolge\pwindex{Schnitzler, Arthur 15.05.1862 – 21.10.1931@\textsc{Schnitzler, Arthur} (15.05.1862 – 21.10.1931), \emph{Schriftsteller, Mediziner}!Freiwild. Schauspiel in 3 Akten1896@\strich\emph{Freiwild. Schauspiel in 3 Akten} {[}1896{]}|pwv}}{\lemma{\textnormal{\emph{Erfolge}}}\Cendnote{\textnormal{Am 4. 2. 1898 hatte die \emph{Freiwild}\pwindex{Schnitzler, Arthur 15.05.1862 – 21.10.1931@\textsc{Schnitzler, Arthur} (15.05.1862 – 21.10.1931), \emph{Schriftsteller, Mediziner}!Freiwild. Schauspiel in 3 Akten1896@\strich\emph{Freiwild. Schauspiel in 3 Akten} {[}1896{]}|pwk}-Premiere im Wien\oindex{Wien@\textbf{Wien}|pwk}er Carl-Theater\oindex{Carl-Theater@\textbf{Carl-Theater}|pwk} stattgefunden.
                     Marie Glümer\pwindex{Gluemer, Marie 03.07.1867 – 16.11.1925@\textsc{Glümer, Marie} (03.07.1867 – 16.11.1925), \emph{Schauspielerin}|pwk} spielte die Rolle der Pepi Fischer\pwindex{Schnitzler, Arthur 15.05.1862 – 21.10.1931@\textsc{Schnitzler, Arthur} (15.05.1862 – 21.10.1931), \emph{Schriftsteller, Mediziner}!Freiwild. Schauspiel in 3 Akten1896@\strich\emph{Freiwild. Schauspiel in 3 Akten} {[}1896{]}|pwkv}. }}}\label{K_L02838-2h} in Wien\oindex{Wien@\textbf{Wien}|pw} zu beglückwünſchen. Ich ſchöpfe meine Kenntniß
               des Erfolg\pwindex{Schnitzler, Arthur 15.05.1862 – 21.10.1931@\textsc{Schnitzler, Arthur} (15.05.1862 – 21.10.1931), \emph{Schriftsteller, Mediziner}!Freiwild. Schauspiel in 3 Akten1896@\strich\emph{Freiwild. Schauspiel in 3 Akten} {[}1896{]}|pwv}es nur aus der
                  \label{K_L02838-3v}\edtext{Kritik\pwindex{?? Werk@Nicht ermittelte Verfasserinnen und Verfasser!Theaterzeitung. Carltheater [Freiwild]1898-02-05@\emph{Theaterzeitung. Carltheater [Freiwild]} {[}1898-02-05{]}|pwv} des Extrablatt\pwindex{Illustrirtes Wiener Extrablatt1872 – 1928@\emph{Illustrirtes Wiener Extrablatt} {[}1872 – 1928{]}|pw}}{\lemma{\textnormal{\emph{Kritik des Extrablatt}}}\Cendnote{\textnormal{[O. V.]: \emph{Theaterzeitung. Carltheater}\pwindex{?? Werk@Nicht ermittelte Verfasserinnen und Verfasser!Theaterzeitung. Carltheater [Freiwild]1898-02-05@\emph{Theaterzeitung. Carltheater [Freiwild]} {[}1898-02-05{]}|pwk}.
                     In: \emph{Illustrirtes Wiener Extrablatt}\pwindex{Illustrirtes Wiener Extrablatt1872 – 1928@\emph{Illustrirtes Wiener Extrablatt} {[}1872 – 1928{]}|pwk},
                     Jg. 27, Nr. 35, 5. 2. 1898, S. 5.}}}\label{K_L02838-3h}.
               Aber ich denke mir, wenn ſchon dieſes dumme Blatt\orgindex{Illustrirtes Wiener Extrablatt@Illustrirtes Wiener Extrablatt|pwv} ſo freundlich iſt, wie ruhmreich muß da in Wirklichkeit
               der \textsc{Premièren}-Abend\pwindex{Schnitzler, Arthur 15.05.1862 – 21.10.1931@\textsc{Schnitzler, Arthur} (15.05.1862 – 21.10.1931), \emph{Schriftsteller, Mediziner}!Freiwild. Schauspiel in 3 Akten1896@\strich\emph{Freiwild. Schauspiel in 3 Akten} {[}1896{]}|pwv} geweſen ſein! Ich freue mich von
                  \strikeout{H\textcolor{gray}{erz}} ganzem Herzen, {\pb}daß ich Dich ſo ſtolz und
               ſicher \strikeout{\textcolor{gray}{r}} weiterſchreiten ſehe.\pend
           \pstart
           Ich danke Dir für Deinen letzten lieben Brief. Bitte, ſchreib’ mir bald! Schreib’
               mir, wie die Première\pwindex{Schnitzler, Arthur 15.05.1862 – 21.10.1931@\textsc{Schnitzler, Arthur} (15.05.1862 – 21.10.1931), \emph{Schriftsteller, Mediziner}!Freiwild. Schauspiel in 3 Akten1896@\strich\emph{Freiwild. Schauspiel in 3 Akten} {[}1896{]}|pwv} war, wie
                  \textsc{Frl. G.\pwindex{Gluemer, Marie 03.07.1867 – 16.11.1925@\textsc{Glümer, Marie} (03.07.1867 – 16.11.1925), \emph{Schauspielerin}|pw}} geſpielt hat und was es ſonſt dabei gab.\pend
           \pstart
           Iſt \label{K_L02838-4v}\edtext{\textsc{Dr. Brandes\pwindex{Brandes, Georg 04.02.1842 – 19.02.1927@\textsc{Brandes, Georg} (04.02.1842 – 19.02.1927)|pw}} ſehr böſe}{\lemma{\textnormal{\emph{Dr. Brandes ſehr böſe}}}\Cendnote{\textnormal{Georg Brandes\pwindex{Brandes, Georg 04.02.1842 – 19.02.1927@\textsc{Brandes, Georg} (04.02.1842 – 19.02.1927)|pwk} hielt sich zwischen 25. 1. 1898 und 2. 2. 1898 in Wien\oindex{Wien@\textbf{Wien}|pwk} auf und traf sich mehrfach mit Schnitzler\pwindex{Schnitzler, Arthur 15.05.1862 – 21.10.1931@\textsc{Schnitzler, Arthur} (15.05.1862 – 21.10.1931), \emph{Schriftsteller, Mediziner}|pwk}.}}}\label{K_L02838-4h} auf mich, weil ich ihm
               nicht geſchrieben habe?\pend
           \pstart
           Ich begrüße Dich von Herzen und in Treue {\\[\baselineskip]}Dein {\\[\baselineskip]}\spacefill\mbox{Paul Goldmann.}\pend
           \leftskip=0em{}
         
         \endnumbering\mylabel{h}\end{ledgroupsized}  \newcommand{\dateiname}{L02838}\newcommand{\titel}{Paul Goldmann an Arthur Schnitzler, 6. 2. [1898]}\newcommand{\editorInnen}{Martin Anton Müller und Laura Untner}%% latex-leseansicht-abspann.tex
%% Abspann für die Leseansicht.
%% Der Schalter \ifkorrekturansicht ist bereits durch den Vorspann gesetzt.

%% latex-abspann.tex
%% Gemeinsamer Abspann für Korrekturansicht und Leseansicht.
%% Setzt den Schalter \ifkorrekturansicht voraus (gesetzt in den
%% einbindenden Dateien latex-korrekturansicht-abspann.tex bzw.
%% latex-leseansicht-abspann.tex).
%% ---------------------------------------------------------------

\normalsize

% Das esempio-Environment wird nur in der Leseansicht benötigt
\ifkorrekturansicht\else
\newenvironment{esempio}[3]%
{
    \vspace{1.5ex}
    \rlap{\underline{#1}}
    \par
    \setlength{\parindent}{0cm}
    \nopagebreak
    \leftskip=#2cm
    \rightskip=#3cm
}
{
    \par
}
\fi

\doendnotes{C}
\bigskip
\vfill

\clearpage

\footnotesize

\ifkorrekturansicht
  \lohead{\textsc{register}}
\fi

% theindex-Environment neu definieren ohne reledmac
\makeatletter
\renewenvironment{theindex}{%
  \ifkorrekturansicht
    \section*{\indexname}%
  \else
    \subsubsection*{Index der erwähnten Entitäten}%
  \fi
  \setlength{\parindent}{0pt}%
  \setlength{\parskip}{0pt plus 0.3pt}%
  \let\item\@idxitem
}{%
  \ifkorrekturansicht\clearpage\fi
}
\makeatother

\IfFileExists{\jobname-pw.ind}{\input{\jobname-pw.ind}}{}

% Quellenangabe nur in der Leseansicht
\ifkorrekturansicht\else
% Fallback-Definitionen, falls die .tex-Datei \titel etc. nicht gesetzt hat
\providecommand{\titel}{}
\providecommand{\editorInnen}{}
\providecommand{\dateiname}{\jobname}

\vspace{3cm}

\vfill

\footnotesize
\textsc{Quelle}: \titel. Herausgegeben von {\editorInnen}. In: \emph{Arthur Schnitzler: Briefwechsel mit Autorinnen und Autoren}.
 Digitale Edition, https://schnitzler-briefe.acdh.oeaw.ac.at/{\dateiname}.html (Stand \today)
\fi

\end{document}


      