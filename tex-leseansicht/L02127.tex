%% latex-leseansicht-vorspann.tex
%% Vorspann für die Leseansicht.
%% Lädt die gemeinsame Datei latex-vorspann.tex mit nicht gesetztem Schalter.

\newif\ifkorrekturansicht
\korrekturansichtfalse

\input{../tex-inputs/latex-vorspann}


\section[Peter Altenberg an Arthur Schnitzler, {{[}}20.? 4. 1913{{]}}]{L02127 Peter Altenberg an Arthur Schnitzler, {[}20.? 4. 1913{]}}
\nopagebreak\mylabel{L02127v}
\rehead{ }\normalsize\beginnumbering\briefempfaengerindex{Schnitzler, Arthur@\textsc{Schnitzler, Arthur}!zzzAltenberg, Peter@\emph{von Peter Altenberg}!1913-04-203@{{[}20.? 4. 1913{]}}|(be}
\toendnotes[C]{\smallbreak\pagebreak[2]}
\correspDesc{Versand  durch Peter Altenberg am [20.? 4. 1913] in Wien
\newline{}Erhalt  durch Arthur Schnitzler im Zeitraum [20. 4. 1913
                  – 24. 4. 1913?] in Wien}\toendnotes[C]{\smallbreak}
\Standort{CUL, Schnitzler, B 2.}
\physDesc{Brief, 1 Blatt, 2 Seiten, 321 Zeichen
\newline{}Handschrift: schwarze Tinte, deutsche Kurrent
\newline{}Schnitzler: mit Bleistift datiert: »19/4 913« 
\newline{}Ordnung: von unbekannter Hand nummeriert: »12« }\toendnotes[C]{\smallbreak}
\pstart
           \noindent{}{\pb}Ich habe auf Sie vertraut, daſs Sie mir
               helfen werden, daſs dieſe \uline{entſetzliche unertragbare}
               Leidenszeit auf ein \uline{Minnimum} von \uline{einigen} Tagen beſchränkt werde!?!?\pend
           
\pstart
           \label{K_L02127-1v}\edtext{Sie hätten}{\lemma{\textnormal{\emph{Sie hätten}}}\Cendnote{\textnormal{Da Schnitzler am
                     20. 4. 1913 bei Altenberg\pwindex{Altenberg, Peter 9.\,3.\,1859 Wien – 8.\,1.\,1919 ebd.@\textsc{Altenberg, Peter} (9.\,3.\,1859 Wien – 8.\,1.\,1919 ebd.), \emph{Schriftsteller}|pwk}
                  war, dürfte die handschriftliche Datierung auf »19« nicht stimmen, sondern das Korrespondenzstück als unmittelbare Reaktion
                  auf den Besuch aufzufassen sein. Vgl. XXXX Auszeichnungsfehler: Dokument L02124 nicht gefunden.}}}\label{K_L02127-1} den Primarius\pwindex{Richter, Karl 9.\,3.\,1862 Bruntál – 25.\,6.\,1937 Wien@\textsc{Richter, Karl} (9.\,3.\,1862 Bruntál – 25.\,6.\,1937 Wien), \emph{Mediziner, Sanatoriumsleiter}|pwv}{ }\uline{beſtimmen}{ }ſollen, mich \uuline{ſogleich} frei zu geben!\pend
           
\pstart
           {\pb}Helfen Sie, um Gotteswillen!!!\pend
           
\pstart
           Ich \uuline{\edtext{muß}{\Cendnote{dreifach unterstrichen}}} meine Freiheit haben!\pend
           
\pstart
           Bitte um \uline{Antwort.}\pend
           
\pstart
           Ihr dankbarer{\\[\baselineskip]}\spacefill\mbox{Peter Altenberg}\pend
           \leftskip=0em{}\selectlanguage{ngerman}\endnumbering\briefempfaengerindex{Schnitzler, Arthur@\textsc{Schnitzler, Arthur}!zzzAltenberg, Peter@\emph{von Peter Altenberg}!1913-04-203@{{[}20.? 4. 1913{]}}|)be}\mylabel{L02127h}  \newcommand{\dateiname}{L02127}\newcommand{\titel}{Peter Altenberg an Arthur Schnitzler, [20.? 4. 1913]}\newcommand{\editorInnen}{Martin Anton Müller und Gerd-Hermann Susen}%% latex-leseansicht-abspann.tex
%% Abspann für die Leseansicht.
%% Der Schalter \ifkorrekturansicht ist bereits durch den Vorspann gesetzt.

%% latex-abspann.tex
%% Gemeinsamer Abspann für Korrekturansicht und Leseansicht.
%% Setzt den Schalter \ifkorrekturansicht voraus (gesetzt in den
%% einbindenden Dateien latex-korrekturansicht-abspann.tex bzw.
%% latex-leseansicht-abspann.tex).
%% ---------------------------------------------------------------

\normalsize

% Das esempio-Environment wird nur in der Leseansicht benötigt
\ifkorrekturansicht\else
\newenvironment{esempio}[3]%
{
    \vspace{1.5ex}
    \rlap{\underline{#1}}
    \par
    \setlength{\parindent}{0cm}
    \nopagebreak
    \leftskip=#2cm
    \rightskip=#3cm
}
{
    \par
}
\fi

\doendnotes{C}
\bigskip
\vfill

\clearpage

\footnotesize

\ifkorrekturansicht
  \lohead{\textsc{register}}
\fi

% theindex-Environment neu definieren ohne reledmac
\makeatletter
\renewenvironment{theindex}{%
  \ifkorrekturansicht
    \section*{\indexname}%
  \else
    \subsubsection*{Index der erwähnten Entitäten}%
  \fi
  \setlength{\parindent}{0pt}%
  \setlength{\parskip}{0pt plus 0.3pt}%
  \let\item\@idxitem
}{%
  \ifkorrekturansicht\clearpage\fi
}
\makeatother

\IfFileExists{\jobname-pw.ind}{\input{\jobname-pw.ind}}{}

% Quellenangabe nur in der Leseansicht
\ifkorrekturansicht\else
% Fallback-Definitionen, falls die .tex-Datei \titel etc. nicht gesetzt hat
\providecommand{\titel}{}
\providecommand{\editorInnen}{}
\providecommand{\dateiname}{\jobname}

\vspace{3cm}

\vfill

\footnotesize
\textsc{Quelle}: \titel. Herausgegeben von {\editorInnen}. In: \emph{Arthur Schnitzler: Briefwechsel mit Autorinnen und Autoren}.
 Digitale Edition, https://schnitzler-briefe.acdh.oeaw.ac.at/{\dateiname}.html (Stand \today)
\fi

\end{document}


