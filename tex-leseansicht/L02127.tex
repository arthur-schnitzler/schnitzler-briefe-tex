%% latex-korrekturansicht-vorspann.tex
%% Vorspann für die Korrekturansicht.
%% Lädt die gemeinsame Datei latex-vorspann.tex mit gesetztem Schalter.

\newif\ifkorrekturansicht
\korrekturansichttrue

\input{../tex-inputs/latex-vorspann}


\section[Peter Altenberg an Arthur Schnitzler, {[}20.? 4. 1913{]}]{L02127 Peter Altenberg an Arthur Schnitzler, {[}20.? 4. 1913{]}}
\nopagebreak\mylabel{L02127v}
\rehead{ }\normalsize\beginnumbering\briefempfaengerindex{Schnitzler, Arthur@\textsc{Schnitzler, Arthur}!zzzAltenberg, Peter@\emph{von Peter Altenberg}!1913-04-203@{{[}20.? 4. 1913{]}}|(be}
\toendnotes[C]{\smallbreak\pagebreak[2]}\Standort{CUL, Schnitzler, B 2.}
\physDesc{Brief, 1 Blatt, 2 Seiten, 321 Zeichen
\newline{}Handschrift: schwarze Tinte, deutsche Kurrent
\newline{}Schnitzler: mit Bleistift datiert: »19/4 913« 
\newline{}Ordnung: von unbekannter Hand nummeriert: »12« }\toendnotes[C]{\smallbreak}
\pstart
           \noindent{}{\pb}Ich habe auf Sie vertraut, daſs Sie mir
               helfen werden, daſs dieſe \uline{entſetzliche unertragbare}
               Leidenszeit auf ein \uline{Minnimum} von \uline{einigen} Tagen beſchränkt werde!?!?\pend
           
\pstart
           \label{K_L02127-1v}\edtext{Sie hätten}{\lemma{\textnormal{\emph{Sie hätten}}}\Cendnote{\textnormal{Da Schnitzler am
                     20. 4. 1913 bei Altenberg\pwindex{Altenberg, Peter 09.03.1859 – 08.01.1919@\textsc{Altenberg, Peter} (09.03.1859 – 08.01.1919), \emph{Schriftsteller/Schriftstellerin}|pwk}
                  war, dürfte die handschriftliche Datierung auf »19« nicht stimmen, sondern das Korrespondenzstück als unmittelbare Reaktion
                  auf den Besuch aufzufassen sein. Vgl. Peter Altenberg an Arthur Schnitzler, [19.? 4. 1913].}}}\label{K_L02127-1} den Primarius\pwindex{Richter, Karl 09.03.1862 – 25.06.1937@\textsc{Richter, Karl} (09.03.1862 – 25.06.1937), \emph{Mediziner/Medizinerin, Sanatoriumsleiter/Sanatoriumsleiterin}|pwv}{ }\uline{beſtimmen}{ }ſollen, mich \uuline{ſogleich} frei zu geben!\pend
           
\pstart
           {\pb}Helfen Sie, um Gotteswillen!!!\pend
           
\pstart
           Ich \uuline{\edtext{muß}{\Cendnote{dreifach unterstrichen}}} meine Freiheit haben!\pend
           
\pstart
           Bitte um \uline{Antwort.}\pend
           
\pstart
           Ihr dankbarer{\\[\baselineskip]}\spacefill\mbox{Peter Altenberg}\pend
           \leftskip=0em{}\selectlanguage{ngerman}\endnumbering\briefempfaengerindex{Schnitzler, Arthur@\textsc{Schnitzler, Arthur}!zzzAltenberg, Peter@\emph{von Peter Altenberg}!1913-04-203@{{[}20.? 4. 1913{]}}|)be}\mylabel{L02127h}  \normalsize

\doendnotes{C}
\bigskip
\vfill

\clearpage

\footnotesize

\lohead{\textsc{register}}

% Definiere theindex-Environment komplett neu ohne reledmac
\makeatletter
\renewenvironment{theindex}{%
  \section*{\indexname}%
  \setlength{\parindent}{0pt}%
  \setlength{\parskip}{0pt plus 0.3pt}%
  \let\item\@idxitem
}{%
  \clearpage
}
\makeatother

\IfFileExists{\jobname-pw.ind}{\input{\jobname-pw.ind}}{}

\end{document}

      