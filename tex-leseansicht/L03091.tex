%% latex-korrekturansicht-vorspann.tex
%% Vorspann für die Korrekturansicht.
%% Lädt die gemeinsame Datei latex-vorspann.tex mit gesetztem Schalter.

\newif\ifkorrekturansicht
\korrekturansichttrue

\input{../tex-inputs/latex-vorspann}


\section[ Paul Goldmann an Arthur Schnitzler, 23. 11. {[}1901{]}]{L03091 Paul Goldmann an Arthur Schnitzler, 23. 11. {[}1901{]}}
\nopagebreak\mylabel{L03091v}
\rehead{ }\normalsize\beginnumbering\briefempfaengerindex{Schnitzler, Arthur@\textsc{Schnitzler, Arthur}!zzzGoldmann, Paul@\emph{von Paul Goldmann}!1901-11-231@{23. 11. {[}1901{]}}|(be}
\toendnotes[C]{\smallbreak\pagebreak[2]}\Standort{DLA, A:Schnitzler, HS.NZ85.1.3171.}
\physDesc{Brief, 2 Blätter, 8 Seiten, 3249 Zeichen
\newline{}Handschrift: blaue Tinte, deutsche Kurrent
\newline{}Schnitzler: 1) mit Bleistift das Jahr »1901« vermerkt  2) mit rotem Buntstift sieben Unterstreichungen}\toendnotes[C]{\smallbreak}
\pstart
           \raggedleft{}{\pb}\textcolor{gray}{\textbf{DESSAUERSTRASSE 19}}\oindex{Dessauer Strasse@\textbf{Dessauer Straße}, \emph{Straße (K.STR)}|pw}\pend
           
\pstart
           Berlin\oindex{Berlin@\textbf{Berlin}, \emph{P.PPLC}|pw}, 23. November.\pend
           
\pstart\center{}Mein lieber Freund,\pend\vspace{0.5em}
\pstart
           Tauſend Dank für Deine lieben Worte! Es war wirklich nicht nöthig, mir deshalb einen
               großen Brief zu ſchreiben, und ich bitte Dich, auch \textsc{Olga\pwindex{Schnitzler, Olga 17.01.1882 – 13.01.1970@\textsc{Schnitzler, Olga} (17.01.1882 – 13.01.1970), \emph{Schauspieler/Schauspielerin, Sänger/Sängerin}|pw}} zu veranlaſſen, daß ſie mir über die \label{K_L03091-1v}\edtext{Affaire}{\lemma{\textnormal{\emph{Affaire}}}\Cendnote{\textnormal{Bezug auf
                  den Konflikt rund um Goldmanns\pwindex{Goldmann, Paul 31.01.1865 – 25.09.1935@\textsc{Goldmann, Paul} (31.01.1865 – 25.09.1935), \emph{Schriftsteller/Schriftstellerin, Journalist/Journalistin}|pwk} Kritik an
                     Gerhart Hauptmann\pwindex{Hauptmann, Gerhart 15.11.1862 – 06.06.1946@\textsc{Hauptmann, Gerhart} (15.11.1862 – 06.06.1946), \emph{Schriftsteller/Schriftstellerin}|pwk}, siehe Paul Goldmann an Arthur Schnitzler, 9. 11. [1901]. }}}\label{K_L03091-1} nicht mehr
               ſchreibt. Die Sache iſt abgethan; und ich bedaure lebhaft, daß ich dem Unwillen, den
               ich über den zurechtweiſenden Ton von \textsc{Olgas\pwindex{Schnitzler, Olga 17.01.1882 – 13.01.1970@\textsc{Schnitzler, Olga} (17.01.1882 – 13.01.1970), \emph{Schauspieler/Schauspielerin, Sänger/Sängerin}|pw}} Brief empfunden, überhaupt
               Ausdruck gegeben habe. Im Übrigen nimmſt Du nach wie vor in der Frage einen
               erſtaunlich \label{K_L03091-2v}\edtext{einſeitigen
                  Standpunkt}{\lemma{\textnormal{\emph{einſeitigen
                  Standpunkt}}}\Cendnote{\textnormal{Auch Schnitzler schätzte Goldmanns\pwindex{Goldmann, Paul 31.01.1865 – 25.09.1935@\textsc{Goldmann, Paul} (31.01.1865 – 25.09.1935), \emph{Schriftsteller/Schriftstellerin, Journalist/Journalistin}|pwk} Standpunkt als einseitig ein, vgl. A. S.: \emph{Tagebuch}, 27. 11. 1901.}}}\label{K_L03091-2} ein. Ich kann Dir verſichern, daß
                  {\pb}nicht nur \label{K_L03091-3v}\edtext{widerliche Kerle}{\lemma{\textnormal{\emph{widerliche Kerle}}}\Cendnote{\textnormal{womöglich Anspielung auf Leo Ebermann\pwindex{Ebermann, Leo 16.07.1863 – 09.10.1914@\textsc{Ebermann, Leo} (16.07.1863 – 09.10.1914), \emph{Schriftsteller/Schriftstellerin, Journalist/Journalistin, Rechtswissenschaftler/Rechtswissenschaftlerin}|pwk}, vgl. Paul Goldmann an Arthur Schnitzler, 23. 11. [1901].}}}\label{K_L03091-3} ſich über meine Kritiken freuen, ſondern auch ſehr anſtändige Leute. Und was
               habe ich mich um die Wirkungen zu bekümmern, die meine Kritiken { } auf widerliche Kerle \substVorne{}\textsuperscript{\textcolor{gray}{×}\-\textcolor{gray}{×}\-\textcolor{gray}{×}\-\textcolor{gray}{×}\-\textcolor{gray}{×}}\substDazwischen{}ausüben\substHinten{}? Was habe ich mich überhaupt um die Wirkungen meiner Arbeiten zu bekümmern?
               Das iſt \strikeout{\textcolor{gray}{doch}} ein ganz unkünſtleriſches Verlangen, das Du da an mich ſtellſt. Die einzige
               Frage kann doch nur die ſein, ob meine Kritiken meine Überzeugung und meine Stimmung
               ausdrücken. Und da meine Überzeugung die iſt, daß \textsc{Gerhart Hauptmann\pwindex{Hauptmann, Gerhart 15.11.1862 – 06.06.1946@\textsc{Hauptmann, Gerhart} (15.11.1862 – 06.06.1946), \emph{Schriftsteller/Schriftstellerin}|pw}} ein minderwerthiger {\pb}und verworrener Geiſt
               iſt, und da ich Erbitterung darüber empfinde, dieſen minderwerthigen Geiſt\pwindex{Hauptmann, Gerhart 15.11.1862 – 06.06.1946@\textsc{Hauptmann, Gerhart} (15.11.1862 – 06.06.1946), \emph{Schriftsteller/Schriftstellerin}|pwv} als großen Dichter
               geprieſen zu ſehen, ſo \strikeout{\textcolor{gray}{ſ}} können meine Kritiken abſolut nicht anders lauten und können auch in keinem
               anderen Tone geſchrieben ſein.\pend
           
\pstart
           Du irrſt Dich auch, wenn Du glaubſt, daß Du mir immer ſchreibſt, wenn Du über eine
               meiner Arbeiten »entzückt« biſt. Ich bin überzeugt, daß Du in Wien\oindex{Wien@\textbf{Wien}, \emph{A.ADM2}|pw} dieſem »Entzücken« Worte verleihſt, Du vergißt es nur in
               der {\pb}Regel, mir mitzutheilen. Ich habe oft genug, wenn
               ich das Bewußtſein hatte, eine Arbeit von Werth vollendet zu haben, mich nach einem
               Wort der Zuſtimmung von Deiner Seite geſehnt, und oft genug iſt dieſes Wort der
               Zuſtimmung ausgeblieben. Pünktlich und ausführlich ſchreibſt Du mir nur, wenn Du an
               meinen Arbeiten etwas zu tadeln haſt.\pend
           
\pstart
           So, und nun genug!\pend
           
\pstart
           Ich habe mich von Herzen gefreut, endlich wieder einmal etwas von Dir zu hören, und
               habe mich insbeſondere gefreut, {\pb}daß Du und \textsc{Olga\pwindex{Schnitzler, Olga 17.01.1882 – 13.01.1970@\textsc{Schnitzler, Olga} (17.01.1882 – 13.01.1970), \emph{Schauspieler/Schauspielerin, Sänger/Sängerin}|pw}} (wie ich aus \textsc{Olgas\pwindex{Schnitzler, Olga 17.01.1882 – 13.01.1970@\textsc{Schnitzler, Olga} (17.01.1882 – 13.01.1970), \emph{Schauspieler/Schauspielerin, Sänger/Sängerin}|pw}} Brief erſehen) in \label{K_L03091-4v}\edtext{\textsc{Reichenau\oindex{Reichenau an der Rax@\textbf{Reichenau an der Rax}, \emph{A.ADM3}|pw}}}{\lemma{\textnormal{\emph{Reichenau}}}\Cendnote{\textnormal{Schnitzler und Olga Gussmann\pwindex{Schnitzler, Olga 17.01.1882 – 13.01.1970@\textsc{Schnitzler, Olga} (17.01.1882 – 13.01.1970), \emph{Schauspieler/Schauspielerin, Sänger/Sängerin}|pwk} waren zwischen 11. 11. 1901 und 13. 11. 1901 in Reichenau\oindex{Reichenau an der Rax@\textbf{Reichenau an der Rax}, \emph{A.ADM3}|pwk} gewesen.}}}\label{K_L03091-4} ſo ſchöne Tage
               verlebt habt.\pend
           
\pstart
           Die Aufführung Deiner Einakter\pwindex{Lebendige Stunden. Vier Einakter@\emph{Lebendige Stunden. Vier Einakter}|pwv}
               am 4. Jänner ſollteſt Du zu \label{K_L03091-5v}\edtext{verhindern}{\lemma{\textnormal{\emph{verhindern}}}\Cendnote{\textnormal{Dazu kam es nicht.}}}\label{K_L03091-5} ſuchen. So wenige Tage nach Neujahr iſt eine recht ungünſtige Theaterzeit. Hat \textsc{Brahm}\pwindex{Brahm, Otto 05.02.1856 – 28.11.1912@\textsc{Brahm, Otto} (05.02.1856 – 28.11.1912), \emph{Theaterleiter/Theaterleiterin, Regisseur/Regisseurin}|pw} ſolange gewartet, ſo kann er auch noch eine Woche länger warten. Ich ſelbſt
               werde \label{K_L03091-6v}\edtext{am 4. Jänner kaum in Berlin\oindex{Berlin@\textbf{Berlin}, \emph{P.PPLC}|pw} ſein}{\lemma{\textnormal{\emph{am … ſein}}}\Cendnote{\textnormal{Goldmann\pwindex{Goldmann, Paul 31.01.1865 – 25.09.1935@\textsc{Goldmann, Paul} (31.01.1865 – 25.09.1935), \emph{Schriftsteller/Schriftstellerin, Journalist/Journalistin}|pwk} war zur Uraufführung von \emph{Lebendige Stunden}\pwindex{Lebendige Stunden. Vier Einakter@\emph{Lebendige Stunden. Vier Einakter}|pwk} wieder in Berlin\oindex{Berlin@\textbf{Berlin}, \emph{P.PPLC}|pwk}.}}}\label{K_L03091-6}, da ich, wie alljährlich, {\pb}die Weihnachts- und Neujahrstage bei meiner \strikeout{Sch\textcolor{gray}{w}}{ }Familie\pwindex{Goldmann, Clementine 1842-05-15 – 1924-02-24@\textsc{Goldmann, Clementine} (1842-05-15 – 1924-02-24)|pwv}\pwindex{Rosengart, Josef 1860-02-08 – 1927-08-04@\textsc{Rosengart, Josef} (1860-02-08 – 1927-08-04), \emph{Arzt/Ärztin}|pwv}\pwindex{Rosengart, Vally 1866-12-29 – nach 1926@\textsc{Rosengart, Vally} (1866-12-29 – nach 1926)|pwv}
               in Frankfurt\oindex{Frankfurt am Main@\textbf{Frankfurt am Main}, \emph{P.PPLA3}|pw} zu verbringen hoffe.\pend
           
\pstart
           Geſtern ſahen wir hier ein ſtellenweiſe ſehr hübſches
                  \label{K_L03091-7v}\edtext{Stück\pwindex{Alt-Heidelberg. Schauspiel in 5 Aufzuegen@\emph{Alt-Heidelberg. Schauspiel in 5 Aufzügen}|pwv} von \textsc{Meyer-Förster\pwindex{Meyer-Foerster, Wilhelm 12.06.1862 – 17.03.1934@\textsc{Meyer-Förster, Wilhelm} (12.06.1862 – 17.03.1934), \emph{Schriftsteller/Schriftstellerin}|pw}}}{\lemma{\textnormal{\emph{Stück von Meyer-Förster}}}\Cendnote{\textnormal{Die Uraufführung von Wilhelm
                  Meyer-Försters\pwindex{Meyer-Foerster, Wilhelm 12.06.1862 – 17.03.1934@\textsc{Meyer-Förster, Wilhelm} (12.06.1862 – 17.03.1934), \emph{Schriftsteller/Schriftstellerin}|pwk}{ }\emph{Alt-Heidelberg. Schauspiel
                     in 5 Aufzügen}\pwindex{Alt-Heidelberg. Schauspiel in 5 Aufzuegen@\emph{Alt-Heidelberg. Schauspiel in 5 Aufzügen}|pwk} fand am 22. 11. 1901 am \emph{Berliner
                     Theater}\orgindex{Berliner Theater@Berliner Theater|pwk} statt.}}}\label{K_L03091-7}. Ich werde leider kaum Zeit finden, darüber zu ſchreiben,
               da nächſte Woche der Reichſtag\orgindex{Reichstag@Reichstag|pw} zuſammentritt.
               Auch muß ich in meinem nächſtem \label{K_L03091-8v}\edtext{Feuilleton\pwindex{Berliner Theater. »Der Rothe Hahn.«@\emph{Berliner Theater. »Der Rothe Hahn.«}|pwv} den »Rothen Hahn\pwindex{rothe Hahn. Tragikomoedie in vier Akten@\emph{Der rothe Hahn. Tragikomödie in vier Akten}|pw}«}{\lemma{\textnormal{\emph{Feuilleton … Hahn«}}}\Cendnote{\textnormal{Paul Goldmann\pwindex{Goldmann, Paul 31.01.1865 – 25.09.1935@\textsc{Goldmann, Paul} (31.01.1865 – 25.09.1935), \emph{Schriftsteller/Schriftstellerin, Journalist/Journalistin}|pwk}: \emph{Berliner Theater. »Der Rothe Hahn«}\pwindex{Berliner Theater. »Der Rothe Hahn.«@\emph{Berliner Theater. »Der Rothe Hahn.«}|pwk}. In: \emph{Neue Freie Presse}\pwindex{Neue Freie Presse@\emph{Neue Freie Presse}|pwk}, Nr. 13.391, 4. 12. 1901, Morgenblatt, S. 1–3. Siehe auch
                     Paul Goldmann an Arthur Schnitzler, 29. 11. [1901] und 6. 12. [1901]. }}}\label{K_L03091-8}
                  behandeln\textcolor{gray}{.}\pend
           
\pstart
           {\pb}Was Du über \label{K_L03091-9v}\edtext{die Haltung\pwindex{Theater- und Kunstnachrichten. Jung-Wiener-Theater »Zum lieben Augustin«@\emph{Theater- und Kunstnachrichten. Jung-Wiener-Theater »Zum lieben Augustin«}|pwv} der N. Fr. Pr.\orgindex{Neue Freie Presse@Neue Freie Presse|pw} gegenüber dem
                  »Jungwiener Theater\orgindex{Jung-Wiener Theater zum Lieben Augustin@Jung-Wiener Theater zum Lieben Augustin|pw}«}{\lemma{\textnormal{\emph{die … Theater«}}}\Cendnote{\textnormal{–da\pwindex{Neuda, Moriz 1842-11-22 – 1917-02-04@\textsc{Neuda, Moriz} (1842-11-22 – 1917-02-04), \emph{Journalist/Journalistin}|pwkv} [ = Moriz Neuda\pwindex{Neuda, Moriz 1842-11-22 – 1917-02-04@\textsc{Neuda, Moriz} (1842-11-22 – 1917-02-04), \emph{Journalist/Journalistin}|pwk}]: \emph{Theater- und Kunstnachrichten. Jung-Wiener-Theater »Zum
                        lieben Augustin«}\pwindex{Theater- und Kunstnachrichten. Jung-Wiener-Theater »Zum lieben Augustin«@\emph{Theater- und Kunstnachrichten. Jung-Wiener-Theater »Zum lieben Augustin«}|pwk}. In: \emph{Neue Freie
                        Presse}\pwindex{Neue Freie Presse@\emph{Neue Freie Presse}|pwk}, Nr. 13.374, 17. 11. 1901,
                     Morgenblatt, S. 8–9.}}}\label{K_L03091-9} ſchreibſt, iſt durchaus berechtigt. Aber \textsc{Salten\pwindex{Salten, Felix 06.09.1869 – 08.10.1945@\textsc{Salten, Felix} (06.09.1869 – 08.10.1945), \emph{Schriftsteller/Schriftstellerin, Journalist/Journalistin, Chefredakteur/Chefredakteurin}|pw}} trägt doch wohl die Hauptſchuld. Er machte \strikeout{h\textcolor{gray}{i}} mir hier in Berlin\oindex{Berlin@\textbf{Berlin}, \emph{P.PPLC}|pw} den Eindruck eines
               Mannes, der abſolut keine Ahnung hat, was er will. Und wie kann man ſich zu einem
                  \label{K_L03091-10v}\edtext{künſtleriſchen Unternehmen mit \textsc{Siegfried Löwy\pwindex{Loewy, Siegfried 01.11.1857 – 08.05.1931@\textsc{Loewy, Siegfried} (01.11.1857 – 08.05.1931), \emph{Schriftsteller/Schriftstellerin, Journalist/Journalistin}|pw}}}{\lemma{\textnormal{\emph{künſtleriſchen … Löwy}}}\Cendnote{\textnormal{Felix Salten\pwindex{Salten, Felix 06.09.1869 – 08.10.1945@\textsc{Salten, Felix} (06.09.1869 – 08.10.1945), \emph{Schriftsteller/Schriftstellerin, Journalist/Journalistin, Chefredakteur/Chefredakteurin}|pwk} hatte das \emph{Jung-Wiener Theater zum lieben Augustin}\orgindex{Jung-Wiener Theater zum Lieben Augustin@Jung-Wiener Theater zum Lieben Augustin|pwk} gemeinsam mit Siegfried Loewy\pwindex{Loewy, Siegfried 01.11.1857 – 08.05.1931@\textsc{Loewy, Siegfried} (01.11.1857 – 08.05.1931), \emph{Schriftsteller/Schriftstellerin, Journalist/Journalistin}|pwk} gegründet und am 16. 11. 1901 eröffnet.
                  Die Resonanz war schlecht. Bereits nach sechs Aufführungen wurde das Theater\orgindex{Jung-Wiener Theater zum Lieben Augustin@Jung-Wiener Theater zum Lieben Augustin|pwkv} wieder
                  eingestellt.}}}\label{K_L03091-10} aſſociiren?\pend
           
\pstart
           Mit Deinem neuen Stück\pwindex{einsame Weg. Schauspiel in fuenf Akten@\emph{Der einsame Weg. Schauspiel in fünf Akten}|pwv} wirſt
               Du Dich ſchon wieder {\pb}\label{K_L03091-11v}\edtext{zurechtfinden}{\lemma{\textnormal{\emph{zurechtfinden}}}\Cendnote{\textnormal{Siehe A. S.: \emph{Tagebuch}, 20. 11. 1901.
               }}}\label{K_L03091-11}. Je mehr Du daran arbeiteſt, umſo tiefer wird es werden. Quäle Dich alſo nur
               ein wenig. Es ſchadet gar nichts.\pend
           
\pstart
           Grüße mir die Mädeln\pwindex{Schnitzler, Olga 17.01.1882 – 13.01.1970@\textsc{Schnitzler, Olga} (17.01.1882 – 13.01.1970), \emph{Schauspieler/Schauspielerin, Sänger/Sängerin}|pwv}\pwindex{Steinrueck, Elisabeth 19.11.1885 – 07.04.1920@\textsc{Steinrück, Elisabeth} (19.11.1885 – 07.04.1920)|pwv} und ſei Du ſelbſt vielmals und von Herzen gegrüßt! {\\[\baselineskip]}Dein {\\[\baselineskip]}\spacefill\mbox{Paul Goldmann}\pend
           \leftskip=0em{}\selectlanguage{ngerman}\endnumbering\briefempfaengerindex{Schnitzler, Arthur@\textsc{Schnitzler, Arthur}!zzzGoldmann, Paul@\emph{von Paul Goldmann}!1901-11-231@{23. 11. {[}1901{]}}|)be}\mylabel{L03091h}  \normalsize

\doendnotes{C}
\bigskip
\vfill

\clearpage

\footnotesize

\lohead{\textsc{register}}

% Definiere theindex-Environment komplett neu ohne reledmac
\makeatletter
\renewenvironment{theindex}{%
  \section*{\indexname}%
  \setlength{\parindent}{0pt}%
  \setlength{\parskip}{0pt plus 0.3pt}%
  \let\item\@idxitem
}{%
  \clearpage
}
\makeatother

\IfFileExists{\jobname-pw.ind}{\input{\jobname-pw.ind}}{}

\end{document}

      