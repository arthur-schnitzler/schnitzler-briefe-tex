%% latex-leseansicht-vorspann.tex
%% Vorspann für die Leseansicht.
%% Lädt die gemeinsame Datei latex-vorspann.tex mit nicht gesetztem Schalter.

\newif\ifkorrekturansicht
\korrekturansichtfalse

\input{../tex-inputs/latex-vorspann}


\section[ Paul Goldmann an Arthur Schnitzler, 23. 11. {[}1901{]}]{L03091 Paul Goldmann an Arthur Schnitzler,  23. 11. [1901]}
\nopagebreak\mylabel{L03091v}
\rehead{ }\normalsize\beginnumbering\briefempfaengerindex{Schnitzler, Arthur@\textsc{Schnitzler, Arthur}!zzzGoldmann, Paul@\emph{von Paul Goldmann}!1901-11-231@{23. 11. [1901]}|(be}
\toendnotes[C]{\smallbreak\pagebreak[2]}
\correspDesc{Versand  durch Paul Goldmann am 23. 11. [1901] in Berlin
\newline{}Erhalt  durch Arthur Schnitzler im Zeitraum [24. 11. 1901 – 28. 11. 1901?] in Wien}\toendnotes[C]{\smallbreak}
\Standort{DLA, A:Schnitzler, HS.NZ85.1.3171.}
\physDesc{Brief, 2 Blätter, 8 Seiten, 3249 Zeichen
\newline{}Handschrift: blaue Tinte, deutsche Kurrent
\newline{}Schnitzler: 1) mit Bleistift das Jahr »1901« vermerkt  2) mit rotem Buntstift sieben Unterstreichungen}\toendnotes[C]{\smallbreak}
\pstart
           \raggedleft{}{\pb}\textcolor{gray}{\textbf{DESSAUERSTRASSE 19}}\oindex{Dessauer Straße@\textbf{Dessauer Straße}, \emph{Straße}|pw}\pend
           
\pstart
           Berlin\oindex{Berlin@\textbf{Berlin}, \emph{Hauptstadt}|pw}, 23. November.\pend
           
\pstart\center{}Mein lieber Freund,\pend\vspace{0.5em}
\pstart
           Tauſend Dank für Deine lieben Worte! Es war wirklich nicht nöthig, mir deshalb einen
               großen Brief zu{ }ſchreiben, und ich bitte Dich, auch \textsc{Olga\pwindex{Schnitzler, Olga 17.\,1.\,1882 Wien – 13.\,1.\,1970 Lugano@\textsc{Schnitzler, Olga} (17.\,1.\,1882 Wien – 13.\,1.\,1970 Lugano), \emph{Schauspielerin, Sängerin}|pw}} zu veranlaſſen, daß{ }ſie mir über die \label{K_L03091-1v}\edtext{Affaire}{\lemma{\textnormal{\emph{Affaire}}}\Cendnote{\textnormal{Bezug auf
                  den Konflikt rund um Goldmanns\pwindex{Goldmann, Paul 31.\,1.\,1865 Breslau – 25.\,9.\,1935 Wien@\textsc{Goldmann, Paul} (31.\,1.\,1865 Breslau – 25.\,9.\,1935 Wien), \emph{Schriftsteller, Journalist}|pwk} Kritik an
                     Gerhart Hauptmann\pwindex{Hauptmann, Gerhart 15.\,11.\,1862 Szczawno-Zdrój – 6.\,6.\,1946 Jagniątków@\textsc{Hauptmann, Gerhart} (15.\,11.\,1862 Szczawno-Zdrój – 6.\,6.\,1946 Jagniątków), \emph{Schriftsteller}|pwk}, siehe XXXX Auszeichnungsfehler: Dokument L03090 nicht gefunden. }}}\label{K_L03091-1} nicht mehr{ }ſchreibt. Die Sache iſt abgethan; und ich bedaure lebhaft, daß ich dem Unwillen, den
               ich über den zurechtweiſenden Ton von \textsc{Olgas\pwindex{Schnitzler, Olga 17.\,1.\,1882 Wien – 13.\,1.\,1970 Lugano@\textsc{Schnitzler, Olga} (17.\,1.\,1882 Wien – 13.\,1.\,1970 Lugano), \emph{Schauspielerin, Sängerin}|pw}} Brief empfunden, überhaupt
               Ausdruck gegeben habe. Im Übrigen nimmſt Du nach wie vor in der Frage einen
               erſtaunlich \label{K_L03091-2v}\edtext{einſeitigen
                  Standpunkt}{\lemma{\textnormal{\emph{einseitigen
                  Standpunkt}}}\Cendnote{\textnormal{Auch Schnitzler schätzte Goldmanns\pwindex{Goldmann, Paul 31.\,1.\,1865 Breslau – 25.\,9.\,1935 Wien@\textsc{Goldmann, Paul} (31.\,1.\,1865 Breslau – 25.\,9.\,1935 Wien), \emph{Schriftsteller, Journalist}|pwk} Standpunkt als einseitig ein, vgl. A. S.: \emph{Tagebuch}, 27. 11. 1901.}}}\label{K_L03091-2} ein. Ich kann Dir verſichern, daß
                  {\pb}nicht nur \label{K_L03091-3v}\edtext{widerliche Kerle}{\lemma{\textnormal{\emph{widerliche Kerle}}}\Cendnote{\textnormal{womöglich Anspielung auf Leo Ebermann\pwindex{Ebermann, Leo 16.\,7.\,1863 Draganovka – 9.\,10.\,1914 Wien@\textsc{Ebermann, Leo} (16.\,7.\,1863 Draganovka – 9.\,10.\,1914 Wien), \emph{Schriftsteller, Journalist, Rechtswissenschaftler}|pwk}, vgl. XXXX Auszeichnungsfehler: Dokument L03091 nicht gefunden.}}}\label{K_L03091-3}{ }ſich über meine Kritiken freuen,{ }ſondern auch{ }ſehr anſtändige Leute. Und was
               habe ich mich um die Wirkungen zu bekümmern, die meine Kritiken auf widerliche Kerle \substVorne{}\textsuperscript{\textcolor{gray}{×}\-\textcolor{gray}{×}\-\textcolor{gray}{×}\-\textcolor{gray}{×}\-\textcolor{gray}{×}}\substDazwischen{}ausüben\substHinten{}? Was habe ich mich überhaupt um die Wirkungen meiner Arbeiten zu bekümmern?
               Das iſt \strikeout{\textcolor{gray}{doch}} ein ganz unkünſtleriſches Verlangen, das Du da an mich{ }ſtellſt. Die einzige
               Frage kann doch nur die{ }ſein, ob meine Kritiken meine Überzeugung und meine Stimmung
               ausdrücken. Und da meine Überzeugung die iſt, daß \textsc{Gerhart Hauptmann\pwindex{Hauptmann, Gerhart 15.\,11.\,1862 Szczawno-Zdrój – 6.\,6.\,1946 Jagniątków@\textsc{Hauptmann, Gerhart} (15.\,11.\,1862 Szczawno-Zdrój – 6.\,6.\,1946 Jagniątków), \emph{Schriftsteller}|pw}} ein minderwerthiger {\pb}und verworrener Geiſt
               iſt, und da ich Erbitterung darüber empfinde, dieſen minderwerthigen Geiſt\pwindex{Hauptmann, Gerhart 15.\,11.\,1862 Szczawno-Zdrój – 6.\,6.\,1946 Jagniątków@\textsc{Hauptmann, Gerhart} (15.\,11.\,1862 Szczawno-Zdrój – 6.\,6.\,1946 Jagniątków), \emph{Schriftsteller}|pwv} als großen Dichter
               geprieſen zu{ }ſehen,{ }ſo \strikeout{\textcolor{gray}{ſ}} können meine Kritiken abſolut nicht anders lauten und können auch in keinem
               anderen Tone geſchrieben{ }ſein.\pend
           
\pstart
           Du irrſt Dich auch, wenn Du glaubſt, daß Du mir immer{ }ſchreibſt, wenn Du über eine
               meiner Arbeiten »entzückt« biſt. Ich bin überzeugt, daß Du in Wien\oindex{Wien@\textbf{Wien}, \emph{Verwaltungsgebiet}|pw} dieſem »Entzücken« Worte verleihſt, Du vergißt es nur in
               der {\pb}Regel, mir mitzutheilen. Ich habe oft genug, wenn
               ich das Bewußtſein hatte, eine Arbeit von Werth vollendet zu haben, mich nach einem
               Wort der Zuſtimmung von Deiner Seite geſehnt, und oft genug iſt dieſes Wort der
               Zuſtimmung ausgeblieben. Pünktlich und ausführlich{ }ſchreibſt Du mir nur, wenn Du an
               meinen Arbeiten etwas zu tadeln haſt.\pend
           
\pstart
           So, und nun genug!\pend
           
\pstart
           Ich habe mich von Herzen gefreut, endlich wieder einmal etwas von Dir zu hören, und
               habe mich insbeſondere gefreut, {\pb}daß Du und \textsc{Olga\pwindex{Schnitzler, Olga 17.\,1.\,1882 Wien – 13.\,1.\,1970 Lugano@\textsc{Schnitzler, Olga} (17.\,1.\,1882 Wien – 13.\,1.\,1970 Lugano), \emph{Schauspielerin, Sängerin}|pw}} (wie ich aus \textsc{Olgas\pwindex{Schnitzler, Olga 17.\,1.\,1882 Wien – 13.\,1.\,1970 Lugano@\textsc{Schnitzler, Olga} (17.\,1.\,1882 Wien – 13.\,1.\,1970 Lugano), \emph{Schauspielerin, Sängerin}|pw}} Brief erſehen) in \label{K_L03091-4v}\edtext{\textsc{Reichenau\oindex{Reichenau an der Rax@\textbf{Reichenau an der Rax}, \emph{Verwaltungsgebiet}|pw}}}{\lemma{\textnormal{\emph{Reichenau}}}\Cendnote{\textnormal{Schnitzler und Olga Gussmann\pwindex{Schnitzler, Olga 17.\,1.\,1882 Wien – 13.\,1.\,1970 Lugano@\textsc{Schnitzler, Olga} (17.\,1.\,1882 Wien – 13.\,1.\,1970 Lugano), \emph{Schauspielerin, Sängerin}|pwk} waren zwischen 11. 11. 1901 und 13. 11. 1901 in Reichenau\oindex{Reichenau an der Rax@\textbf{Reichenau an der Rax}, \emph{Verwaltungsgebiet}|pwk} gewesen.}}}\label{K_L03091-4}{ }ſo{ }ſchöne Tage
               verlebt habt.\pend
           
\pstart
           Die Aufführung\eventindex{Deutsches Theater Berlin@\textbf{Deutsches Theater Berlin}!Uraufführung von Lebendige Stunden, 4.1.1902@Uraufführung von Lebendige Stunden, 4.1.1902|pwv} Deiner Einakter\pwindex{Schnitzler, Arthur 15.\,5.\,1862 Wien – 21.\,10.\,1931 ebd.@\textsc{Schnitzler, Arthur} (15.\,5.\,1862 Wien – 21.\,10.\,1931 ebd.), \emph{Schriftsteller, Mediziner}!Lebendige Stunden. Vier Einakter@\strich\emph{Lebendige Stunden. Vier Einakter}|pwv}
               am 4. Jänner{ }ſollteſt Du zu \label{K_L03091-5v}\edtext{verhindern}{\lemma{\textnormal{\emph{verhindern}}}\Cendnote{\textnormal{Dazu kam es nicht.}}}\label{K_L03091-5}{ }ſuchen. So wenige Tage nach Neujahr iſt eine recht ungünſtige Theaterzeit. Hat \textsc{Brahm}\pwindex{Brahm, Otto 5.\,2.\,1856 Hamburg – 28.\,11.\,1912 Berlin@\textsc{Brahm, Otto} (5.\,2.\,1856 Hamburg – 28.\,11.\,1912 Berlin), \emph{Theaterleiter, Regisseur}|pw}{ }ſolange gewartet,{ }ſo kann er auch noch eine Woche länger warten. Ich{ }ſelbſt
               werde \label{K_L03091-6v}\edtext{am 4. Jänner kaum in Berlin\oindex{Berlin@\textbf{Berlin}, \emph{Hauptstadt}|pw}{ }ſein}{\lemma{\textnormal{\emph{am … sein}}}\Cendnote{\textnormal{Goldmann\pwindex{Goldmann, Paul 31.\,1.\,1865 Breslau – 25.\,9.\,1935 Wien@\textsc{Goldmann, Paul} (31.\,1.\,1865 Breslau – 25.\,9.\,1935 Wien), \emph{Schriftsteller, Journalist}|pwk} war zur Uraufführung von \emph{Lebendige Stunden}\pwindex{Schnitzler, Arthur 15.\,5.\,1862 Wien – 21.\,10.\,1931 ebd.@\textsc{Schnitzler, Arthur} (15.\,5.\,1862 Wien – 21.\,10.\,1931 ebd.), \emph{Schriftsteller, Mediziner}!Lebendige Stunden. Vier Einakter@\strich\emph{Lebendige Stunden. Vier Einakter}|pwk}\eventindex{Deutsches Theater Berlin@\textbf{Deutsches Theater Berlin}!Uraufführung von Lebendige Stunden, 4.1.1902@Uraufführung von Lebendige Stunden, 4.1.1902|pwk} wieder in Berlin\oindex{Berlin@\textbf{Berlin}, \emph{Hauptstadt}|pwk}.}}}\label{K_L03091-6}, da ich, wie alljährlich, {\pb}die Weihnachts- und Neujahrstage bei meiner \strikeout{Sch\textcolor{gray}{w}}{ }Familie\pwindex{Goldmann, Clementine 15.\,5.\,1842 Breslau – 24.\,2.\,1924 Frankfurt am Main@\textsc{Goldmann, Clementine} (15.\,5.\,1842 Breslau – 24.\,2.\,1924 Frankfurt am Main)|pwv}\pwindex{Rosengart, Josef 8.\,2.\,1860 Laupheim – 4.\,8.\,1927 Frankfurt am Main@\textsc{Rosengart, Josef} (8.\,2.\,1860 Laupheim – 4.\,8.\,1927 Frankfurt am Main), \emph{Arzt}|pwv}\pwindex{Rosengart, Vally 29.\,12.\,1866 Breslau – nach 1926@\textsc{Rosengart, Vally} (29.\,12.\,1866 Breslau – nach 1926)|pwv}
               in Frankfurt\oindex{Frankfurt am Main@\textbf{Frankfurt am Main}, \emph{Hauptstadt}|pw} zu verbringen hoffe.\pend
           
\pstart
           Geſtern{ }ſahen wir hier ein{ }ſtellenweiſe{ }ſehr hübſches
                  \label{K_L03091-7v}\edtext{Stück\eventindex{Berliner Theater@\textbf{Berliner Theater}!Uraufführung von Alt-Heidelberg, 22.11.1901@Uraufführung von Alt-Heidelberg, 22.11.1901|pwv}\pwindex{Meyer-Förster, Wilhelm 12.\,6.\,1862 Hannover – 17.\,3.\,1934 Heringsdorf@\textsc{Meyer-Förster, Wilhelm} (12.\,6.\,1862 Hannover – 17.\,3.\,1934 Heringsdorf), \emph{Schriftsteller}!Alt-Heidelberg. Schauspiel in 5 Aufzügen@\strich\emph{Alt-Heidelberg. Schauspiel in 5 Aufzügen}|pwv} von \textsc{Meyer-Förster\pwindex{Meyer-Förster, Wilhelm 12.\,6.\,1862 Hannover – 17.\,3.\,1934 Heringsdorf@\textsc{Meyer-Förster, Wilhelm} (12.\,6.\,1862 Hannover – 17.\,3.\,1934 Heringsdorf), \emph{Schriftsteller}|pw}}}{\lemma{\textnormal{\emph{Stück von Meyer-Förster}}}\Cendnote{\textnormal{Die Uraufführung\eventindex{Berliner Theater@\textbf{Berliner Theater}!Uraufführung von Alt-Heidelberg, 22.11.1901@Uraufführung von Alt-Heidelberg, 22.11.1901|pwkv} von Wilhelm
                  Meyer-Försters\pwindex{Meyer-Förster, Wilhelm 12.\,6.\,1862 Hannover – 17.\,3.\,1934 Heringsdorf@\textsc{Meyer-Förster, Wilhelm} (12.\,6.\,1862 Hannover – 17.\,3.\,1934 Heringsdorf), \emph{Schriftsteller}|pwk}{ }\emph{Alt-Heidelberg. Schauspiel
                     in 5 Aufzügen}\pwindex{Meyer-Förster, Wilhelm 12.\,6.\,1862 Hannover – 17.\,3.\,1934 Heringsdorf@\textsc{Meyer-Förster, Wilhelm} (12.\,6.\,1862 Hannover – 17.\,3.\,1934 Heringsdorf), \emph{Schriftsteller}!Alt-Heidelberg. Schauspiel in 5 Aufzügen@\strich\emph{Alt-Heidelberg. Schauspiel in 5 Aufzügen}|pwk} fand am 22. 11. 1901 am \emph{Berliner
                     Theater}\orgindex{Berliner Theater@Berliner Theater|pwk} statt.}}}\label{K_L03091-7}. Ich werde leider kaum Zeit finden, darüber zu{ }ſchreiben,
               da nächſte Woche der Reichſtag\orgindex{Reichstag@Reichstag|pw} zuſammentritt.
               Auch muß ich in meinem nächſtem \label{K_L03091-8v}\edtext{Feuilleton\pwindex{Goldmann, Paul 31.\,1.\,1865 Breslau – 25.\,9.\,1935 Wien@\textsc{Goldmann, Paul} (31.\,1.\,1865 Breslau – 25.\,9.\,1935 Wien), \emph{Schriftsteller, Journalist}!Berliner Theater. »Der Rothe Hahn.«@\strich\emph{Berliner Theater. »Der Rothe Hahn.«}|pwv} den »Rothen Hahn\pwindex{Hauptmann, Gerhart 15.\,11.\,1862 Szczawno-Zdrój – 6.\,6.\,1946 Jagniątków@\textsc{Hauptmann, Gerhart} (15.\,11.\,1862 Szczawno-Zdrój – 6.\,6.\,1946 Jagniątków), \emph{Schriftsteller}!rothe Hahn. Tragikomödie in vier Akten@\strich\emph{Der rothe Hahn. Tragikomödie in vier Akten}|pw}«}{\lemma{\textnormal{\emph{Feuilleton … Hahn«}}}\Cendnote{\textnormal{Paul Goldmann\pwindex{Goldmann, Paul 31.\,1.\,1865 Breslau – 25.\,9.\,1935 Wien@\textsc{Goldmann, Paul} (31.\,1.\,1865 Breslau – 25.\,9.\,1935 Wien), \emph{Schriftsteller, Journalist}|pwk}: \emph{Berliner Theater. »Der Rothe Hahn«}\pwindex{Goldmann, Paul 31.\,1.\,1865 Breslau – 25.\,9.\,1935 Wien@\textsc{Goldmann, Paul} (31.\,1.\,1865 Breslau – 25.\,9.\,1935 Wien), \emph{Schriftsteller, Journalist}!Berliner Theater. »Der Rothe Hahn.«@\strich\emph{Berliner Theater. »Der Rothe Hahn.«}|pwk}. In: \emph{Neue Freie Presse}\pwindex{Neue Freie Presse@\emph{Neue Freie Presse}|pwk}, Nr. 13.391, 4. 12. 1901, Morgenblatt, S. 1–3. Siehe auch
                     XXXX Auszeichnungsfehler: Dokument L03092 nicht gefunden und XXXX Auszeichnungsfehler: Dokument L03094 nicht gefunden. }}}\label{K_L03091-8}
                  behandeln\textcolor{gray}{.}\pend
           
\pstart
           {\pb}Was Du über \label{K_L03091-9v}\edtext{die Haltung\pwindex{Neuda, Moriz 22.\,11.\,1842 Loštice – 4.\,2.\,1917 Wien@\textsc{Neuda, Moriz} (22.\,11.\,1842 Loštice – 4.\,2.\,1917 Wien), \emph{Journalist}!Theater- und Kunstnachrichten. Jung-Wiener-Theater »Zum lieben Augustin«@\strich\emph{Theater- und Kunstnachrichten. Jung-Wiener-Theater »Zum lieben Augustin«}|pwv} der N. Fr. Pr.\orgindex{Neue Freie Presse@Neue Freie Presse|pw} gegenüber dem
                  »Jungwiener Theater\orgindex{Jung-Wiener Theater zum Lieben Augustin@Jung-Wiener Theater zum Lieben Augustin|pw}«}{\lemma{\textnormal{\emph{die … Theater«}}}\Cendnote{\textnormal{–da\pwindex{Neuda, Moriz 22.\,11.\,1842 Loštice – 4.\,2.\,1917 Wien@\textsc{Neuda, Moriz} (22.\,11.\,1842 Loštice – 4.\,2.\,1917 Wien), \emph{Journalist}|pwkv} [ = Moriz Neuda\pwindex{Neuda, Moriz 22.\,11.\,1842 Loštice – 4.\,2.\,1917 Wien@\textsc{Neuda, Moriz} (22.\,11.\,1842 Loštice – 4.\,2.\,1917 Wien), \emph{Journalist}|pwk}]: \emph{Theater- und Kunstnachrichten. Jung-Wiener-Theater »Zum
                        lieben Augustin«}\pwindex{Neuda, Moriz 22.\,11.\,1842 Loštice – 4.\,2.\,1917 Wien@\textsc{Neuda, Moriz} (22.\,11.\,1842 Loštice – 4.\,2.\,1917 Wien), \emph{Journalist}!Theater- und Kunstnachrichten. Jung-Wiener-Theater »Zum lieben Augustin«@\strich\emph{Theater- und Kunstnachrichten. Jung-Wiener-Theater »Zum lieben Augustin«}|pwk}. In: \emph{Neue Freie
                        Presse}\pwindex{Neue Freie Presse@\emph{Neue Freie Presse}|pwk}, Nr. 13.374, 17. 11. 1901,
                     Morgenblatt, S. 8–9.}}}\label{K_L03091-9}{ }ſchreibſt, iſt durchaus berechtigt. Aber \textsc{Salten\pwindex{Salten, Felix 6.\,9.\,1869 Budapest – 8.\,10.\,1945 Zürich@\textsc{Salten, Felix} (6.\,9.\,1869 Budapest – 8.\,10.\,1945 Zürich), \emph{Schriftsteller, Journalist, Chefredakteur}|pw}} trägt doch wohl die Hauptſchuld. Er machte \strikeout{h\textcolor{gray}{i}} mir hier in Berlin\oindex{Berlin@\textbf{Berlin}, \emph{Hauptstadt}|pw} den Eindruck eines
               Mannes, der abſolut keine Ahnung hat, was er will. Und wie kann man{ }ſich zu einem
                  \label{K_L03091-10v}\edtext{künſtleriſchen Unternehmen mit \textsc{Siegfried Löwy\pwindex{Loewy, Siegfried 1.\,11.\,1857 Wien – 8.\,5.\,1931 ebd.@\textsc{Loewy, Siegfried} (1.\,11.\,1857 Wien – 8.\,5.\,1931 ebd.), \emph{Schriftsteller, Journalist}|pw}}}{\lemma{\textnormal{\emph{künstlerischen … Löwy}}}\Cendnote{\textnormal{Felix Salten\pwindex{Salten, Felix 6.\,9.\,1869 Budapest – 8.\,10.\,1945 Zürich@\textsc{Salten, Felix} (6.\,9.\,1869 Budapest – 8.\,10.\,1945 Zürich), \emph{Schriftsteller, Journalist, Chefredakteur}|pwk} hatte das \emph{Jung-Wiener Theater zum lieben Augustin}\orgindex{Jung-Wiener Theater zum Lieben Augustin@Jung-Wiener Theater zum Lieben Augustin|pwk} gemeinsam mit Siegfried Loewy\pwindex{Loewy, Siegfried 1.\,11.\,1857 Wien – 8.\,5.\,1931 ebd.@\textsc{Loewy, Siegfried} (1.\,11.\,1857 Wien – 8.\,5.\,1931 ebd.), \emph{Schriftsteller, Journalist}|pwk} gegründet und am 16. 11. 1901 eröffnet.
                  Die Resonanz war schlecht. Bereits nach sechs Aufführungen wurde das Theater\orgindex{Jung-Wiener Theater zum Lieben Augustin@Jung-Wiener Theater zum Lieben Augustin|pwkv} wieder
                  eingestellt.}}}\label{K_L03091-10} aſſociiren?\pend
           
\pstart
           Mit Deinem neuen Stück\pwindex{Schnitzler, Arthur 15.\,5.\,1862 Wien – 21.\,10.\,1931 ebd.@\textsc{Schnitzler, Arthur} (15.\,5.\,1862 Wien – 21.\,10.\,1931 ebd.), \emph{Schriftsteller, Mediziner}!einsame Weg. Schauspiel in fünf Akten@\strich\emph{Der einsame Weg. Schauspiel in fünf Akten}|pwv} wirſt
               Du Dich{ }ſchon wieder {\pb}\label{K_L03091-11v}\edtext{zurechtfinden}{\lemma{\textnormal{\emph{zurechtfinden}}}\Cendnote{\textnormal{Siehe A. S.: \emph{Tagebuch}, 20. 11. 1901.
               }}}\label{K_L03091-11}. Je mehr Du daran arbeiteſt, umſo tiefer wird es werden. Quäle Dich alſo nur
               ein wenig. Es{ }ſchadet gar nichts.\pend
           
\pstart
           Grüße mir die Mädeln\pwindex{Schnitzler, Olga 17.\,1.\,1882 Wien – 13.\,1.\,1970 Lugano@\textsc{Schnitzler, Olga} (17.\,1.\,1882 Wien – 13.\,1.\,1970 Lugano), \emph{Schauspielerin, Sängerin}|pwv}\pwindex{Steinrück, Elisabeth 19.\,11.\,1885 – 7.\,4.\,1920 Partenkirchen@\textsc{Steinrück, Elisabeth} (19.\,11.\,1885 – 7.\,4.\,1920 Partenkirchen)|pwv} und{ }ſei Du{ }ſelbſt vielmals und von Herzen gegrüßt! {\\[\baselineskip]}Dein {\\[\baselineskip]}\spacefill\mbox{Paul Goldmann}\pend
           \leftskip=0em{}\selectlanguage{ngerman}\endnumbering\briefempfaengerindex{Schnitzler, Arthur@\textsc{Schnitzler, Arthur}!zzzGoldmann, Paul@\emph{von Paul Goldmann}!1901-11-231@{23. 11. [1901]}|)be}\mylabel{L03091h}  \newcommand{\dateiname}{L03091}\newcommand{\titel}{Paul Goldmann an Arthur Schnitzler, 23. 11. [1901]}\newcommand{\editorInnen}{Martin Anton Müller und Laura Untner}%% latex-leseansicht-abspann.tex
%% Abspann für die Leseansicht.
%% Der Schalter \ifkorrekturansicht ist bereits durch den Vorspann gesetzt.

%% latex-abspann.tex
%% Gemeinsamer Abspann für Korrekturansicht und Leseansicht.
%% Setzt den Schalter \ifkorrekturansicht voraus (gesetzt in den
%% einbindenden Dateien latex-korrekturansicht-abspann.tex bzw.
%% latex-leseansicht-abspann.tex).
%% ---------------------------------------------------------------

\normalsize

% Das esempio-Environment wird nur in der Leseansicht benötigt
\ifkorrekturansicht\else
\newenvironment{esempio}[3]%
{
    \vspace{1.5ex}
    \rlap{\underline{#1}}
    \par
    \setlength{\parindent}{0cm}
    \nopagebreak
    \leftskip=#2cm
    \rightskip=#3cm
}
{
    \par
}
\fi

\doendnotes{C}
\bigskip
\vfill

\clearpage

\footnotesize

\ifkorrekturansicht
  \lohead{\textsc{register}}
\fi

% theindex-Environment neu definieren ohne reledmac
\makeatletter
\renewenvironment{theindex}{%
  \ifkorrekturansicht
    \section*{\indexname}%
  \else
    \subsubsection*{Index der erwähnten Entitäten}%
  \fi
  \setlength{\parindent}{0pt}%
  \setlength{\parskip}{0pt plus 0.3pt}%
  \let\item\@idxitem
}{%
  \ifkorrekturansicht\clearpage\fi
}
\makeatother

\IfFileExists{\jobname-pw.ind}{\input{\jobname-pw.ind}}{}

% Quellenangabe nur in der Leseansicht
\ifkorrekturansicht\else
% Fallback-Definitionen, falls die .tex-Datei \titel etc. nicht gesetzt hat
\providecommand{\titel}{}
\providecommand{\editorInnen}{}
\providecommand{\dateiname}{\jobname}

\vspace{3cm}

\vfill

\footnotesize
\textsc{Quelle}: \titel. Herausgegeben von {\editorInnen}. In: \emph{Arthur Schnitzler: Briefwechsel mit Autorinnen und Autoren}.
 Digitale Edition, https://schnitzler-briefe.acdh.oeaw.ac.at/{\dateiname}.html (Stand \today)
\fi

\end{document}


