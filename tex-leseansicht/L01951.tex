%% latex-leseansicht-vorspann.tex
%% Vorspann für die Leseansicht.
%% Lädt die gemeinsame Datei latex-vorspann.tex mit nicht gesetztem Schalter.

\newif\ifkorrekturansicht
\korrekturansichtfalse

\input{../tex-inputs/latex-vorspann}


\section[Hugo von Hofmannsthal u. a. an Arthur Schnitzler, 23. 7. 1910]{L01951 Hugo von Hofmannsthal u. a. an Arthur Schnitzler, 23. 7. 1910}
\nopagebreak\mylabel{L01951v}
\rehead{ }\normalsize\beginnumbering\briefempfaengerindex{Schnitzler, Arthur@\textsc{Schnitzler, Arthur}!zzzFriedmann, Rose@\emph{von Rose Friedmann}!1910-07-232@{23. 7. 1910}|(be}\briefempfaengerindex{Schnitzler, Arthur@\textsc{Schnitzler, Arthur}!zzzHofmannsthal, Gertrude von@\emph{von Gertrude von Hofmannsthal}!1910-07-232@{23. 7. 1910}|(be}\briefempfaengerindex{Schnitzler, Arthur@\textsc{Schnitzler, Arthur}!zzzFriedmann, Louis Philipp@\emph{von Louis Philipp Friedmann}!1910-07-232@{23. 7. 1910}|(be}\briefempfaengerindex{Schnitzler, Arthur@\textsc{Schnitzler, Arthur}!zzzHofmannsthal, Hugo von@\emph{von Hugo von Hofmannsthal}!1910-07-232@{23. 7. 1910}|(be}
\toendnotes[C]{\smallbreak\pagebreak[2]}
\correspDesc{Versand  durch Hugo von Hofmannsthal, Louis Friedmann, Gerty von Hofmannsthal, Rose Friedmann am 23. 7. 1910 in München
\newline{}Erhalt  durch Arthur Schnitzler im Zeitraum [24. 7. 1910
                  – 28. 7. 1910?] in Wien}\toendnotes[C]{\smallbreak}
\Standort{CUL, Schnitzler, B 43.}
\physDesc{Bildpostkarte, 325 Zeichen
\newline{}Handschrift Hugo von Hofmannsthal: schwarze Tinte, deutsche Kurrent
\newline{}Handschrift Louis Philipp Friedmann: Bleistift
\newline{}Handschrift Gertrude von Hofmannsthal: schwarze Tinte
\newline{}Handschrift Rose Friedmann: schwarze Tinte
\newline{}Versand: Stempel: »\nobreak{}\oindex{München@\textbf{München}|pwk}München, 23. 7. 10, 4–5N\nobreak{}«.  
\newline{}Schnitzler: mit Bleistift beschriftet: »\textsc{Hugo}« 
\newline{}Ordnung: 1) mit Bleistift von unbekannter Hand nummeriert:
                                    »321«  2) mit Bleistift von unbekannter Hand nummeriert:
                                    »376«}
\buchAbdrucke{\weitereDrucke{Hugo von Hofmannsthal, Arthur Schnitzler: \emph{Briefwechsel}. Herausgegeben von Therese Nickl und Heinrich Schnitzler. Frankfurt am Main: \emph{S. Fischer} 1964, S. 251.} }\pstart{}\textsc{{\pb}Herrn D\textsuperscript{r}}\pend{}\pstart{}\textsc{Arthur Schnitzler}\pend{}\pstart{}\textsc{Wien\oindex{Wien@\textbf{Wien}, \emph{Verwaltungsgebiet}|pw}}\pend{}\pstart{}\textsc{XVIII. Spöttelgasse 7\oindex{Wien@\textbf{Wien}!XVIII., Währing@\textbf{XVIII., Währing}!Edmund-Weiß-Gasse 7@\textbf{Edmund-Weiß-Gasse 7}, \emph{Wohngebäude}|pw}.}\pend{}{\bigskip}
\pstart
           \noindent{}\centering{}{\pb}\textcolor{gray}{\textbf{M. von Schwind\pwindex{Schwind, Moritz von 21.\,1.\,1804 Wien – 8.\,2.\,1871 München@\textsc{Schwind, Moritz von} (21.\,1.\,1804 Wien – 8.\,2.\,1871 München), \emph{Maler}|pw}}}\pend
           
\pstart
           \centering{}\textcolor{gray}{\textbf{Die Landpartie – La partie de campagne – The
                     rural excursion\pwindex{Schwind, Moritz von 21.\,1.\,1804 Wien – 8.\,2.\,1871 München@\textsc{Schwind, Moritz von} (21.\,1.\,1804 Wien – 8.\,2.\,1871 München), \emph{Maler}!Landpartie. Schwind und Bauernfeld auf einem Leiterwagen@\strich\emph{Die Landpartie. Schwind und Bauernfeld auf einem Leiterwagen}|pw} (Schwind\pwindex{Schwind, Moritz von 21.\,1.\,1804 Wien – 8.\,2.\,1871 München@\textsc{Schwind, Moritz von} (21.\,1.\,1804 Wien – 8.\,2.\,1871 München), \emph{Maler}|pw} u. Bauernfeld\pwindex{Bauernfeld, Eduard von 13.\,1.\,1802 Wien – 4.\,8.\,1890 ebd.@\textsc{Bauernfeld, Eduard von} (13.\,1.\,1802 Wien – 4.\,8.\,1890 ebd.)|pw}).}}\pend
           \vspace{1em}
\pstart
           \centering{}{\pb}München\oindex{München@\textbf{München}|pw}. 23 VII.\pend
           \vspace{0.5em}
\pstart
           Ganz ähnlich{ }ſind wir geſtern u. vorgeſtern über Sa\substVorne{}\textsuperscript{z}\substDazwischen{}lz\substHinten{}burg\oindex{Salzburg@\textbf{Salzburg}, \emph{Verwaltungsgebiet}|pw} hierher gefahren und doch auch wieder unähnlich. Es iſt eine
               Preisaufgabe für \textsc{Heini}\pwindex{Schnitzler, Heinrich 9.\,8.\,1902 Hinterbrühl – 12.\,7.\,1982 Wien@\textsc{Schnitzler, Heinrich} (9.\,8.\,1902 Hinterbrühl – 12.\,7.\,1982 Wien), \emph{Regisseur, Schauspieler}|pw}, a. die Ähnlichkeit – b.) die Unterſchiede herauszufinden.\hspace*{1.5em}Viele Grüße Euch beiden.\pend
           
\pstart
           \spacefill\mbox{Hugo.}{\\[\baselineskip]}\spacefill\mbox{{[}hs. Friedmann:{]} L. Friedmann}{\\[\baselineskip]}\spacefill\mbox{{[}hs. Hofmannsthal:{]} Gerty}{\\[\baselineskip]}\spacefill\mbox{{[}hs. Friedmann:{]} Rose Friedmann}\pend
           \leftskip=0em{}\selectlanguage{ngerman}\endnumbering\briefempfaengerindex{Schnitzler, Arthur@\textsc{Schnitzler, Arthur}!zzzFriedmann, Rose@\emph{von Rose Friedmann}!1910-07-232@{23. 7. 1910}|)be}\briefempfaengerindex{Schnitzler, Arthur@\textsc{Schnitzler, Arthur}!zzzHofmannsthal, Gertrude von@\emph{von Gertrude von Hofmannsthal}!1910-07-232@{23. 7. 1910}|)be}\briefempfaengerindex{Schnitzler, Arthur@\textsc{Schnitzler, Arthur}!zzzFriedmann, Louis Philipp@\emph{von Louis Philipp Friedmann}!1910-07-232@{23. 7. 1910}|)be}\briefempfaengerindex{Schnitzler, Arthur@\textsc{Schnitzler, Arthur}!zzzHofmannsthal, Hugo von@\emph{von Hugo von Hofmannsthal}!1910-07-232@{23. 7. 1910}|)be}\mylabel{L01951h}  \newcommand{\dateiname}{L01951}\newcommand{\titel}{Hugo von Hofmannsthal u. a. an Arthur Schnitzler, 23. 7. 1910}\newcommand{\editorInnen}{Martin Anton Müller und Gerd-Hermann Susen}%% latex-leseansicht-abspann.tex
%% Abspann für die Leseansicht.
%% Der Schalter \ifkorrekturansicht ist bereits durch den Vorspann gesetzt.

%% latex-abspann.tex
%% Gemeinsamer Abspann für Korrekturansicht und Leseansicht.
%% Setzt den Schalter \ifkorrekturansicht voraus (gesetzt in den
%% einbindenden Dateien latex-korrekturansicht-abspann.tex bzw.
%% latex-leseansicht-abspann.tex).
%% ---------------------------------------------------------------

\normalsize

% Das esempio-Environment wird nur in der Leseansicht benötigt
\ifkorrekturansicht\else
\newenvironment{esempio}[3]%
{
    \vspace{1.5ex}
    \rlap{\underline{#1}}
    \par
    \setlength{\parindent}{0cm}
    \nopagebreak
    \leftskip=#2cm
    \rightskip=#3cm
}
{
    \par
}
\fi

\doendnotes{C}
\bigskip
\vfill

\clearpage

\footnotesize

\ifkorrekturansicht
  \lohead{\textsc{register}}
\fi

% theindex-Environment neu definieren ohne reledmac
\makeatletter
\renewenvironment{theindex}{%
  \ifkorrekturansicht
    \section*{\indexname}%
  \else
    \subsubsection*{Index der erwähnten Entitäten}%
  \fi
  \setlength{\parindent}{0pt}%
  \setlength{\parskip}{0pt plus 0.3pt}%
  \let\item\@idxitem
}{%
  \ifkorrekturansicht\clearpage\fi
}
\makeatother

\IfFileExists{\jobname-pw.ind}{\input{\jobname-pw.ind}}{}

% Quellenangabe nur in der Leseansicht
\ifkorrekturansicht\else
% Fallback-Definitionen, falls die .tex-Datei \titel etc. nicht gesetzt hat
\providecommand{\titel}{}
\providecommand{\editorInnen}{}
\providecommand{\dateiname}{\jobname}

\vspace{3cm}

\vfill

\footnotesize
\textsc{Quelle}: \titel. Herausgegeben von {\editorInnen}. In: \emph{Arthur Schnitzler: Briefwechsel mit Autorinnen und Autoren}.
 Digitale Edition, https://schnitzler-briefe.acdh.oeaw.ac.at/{\dateiname}.html (Stand \today)
\fi

\end{document}


