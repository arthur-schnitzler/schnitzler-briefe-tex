%% latex-leseansicht-vorspann.tex
%% Vorspann für die Leseansicht.
%% Lädt die gemeinsame Datei latex-vorspann.tex mit nicht gesetztem Schalter.

\newif\ifkorrekturansicht
\korrekturansichtfalse

\input{../tex-inputs/latex-vorspann}


         
         \renewcommand{\erwaehntePersonen}{Personen: Eduard von Bauernfeld, Heinrich Schnitzler, Moritz von Schwind}
         \renewcommand{\erwaehnteOrte}{Orte: Edmund-Weiß-Gasse, München, Salzburg, Wien}
         \renewcommand{\erwaehnteWerke}{Werke: Die Landpartie. Schwind und Bauernfeld auf einem Leiterwagen}
               \section[Hugo von Hofmannsthal u. a. an Arthur Schnitzler, 23. 7. 1910]{ Hugo von Hofmannsthal u. a. an Arthur Schnitzler, 23. 7. 1910}\nopagebreak\mylabel{v}\rehead{ }\begin{ledgroupsized}[t]{13cm}\normalsize\beginnumbering \toendnotes[C]{\smallbreak\pagebreak[2]} \Standort{CUL, Schnitzler, B 43.}
\physDesc{Bildpostkarte, 325 Zeichen
\newline{}Handschrift Hugo von Hofmannsthal: 1) schwarze Tinte, deutsche Kurrent\hspace{1em}2) schwarze Tinte, lateinische Kurrent (\noindent{}Adresse)\hspace{1em}\newline{}Handschrift Louis Philipp Friedmann: Bleistift\newline{}Handschrift Gertrude von Hofmannsthal: schwarze Tinte\newline{}Handschrift Rose Friedmann: schwarze Tinte
\newline{}Versand: Stempel: »\nobreak{}\oindex{Muenchen@\textbf{München}|pwk}München, 23. 7. 10, 4–5N\nobreak{}«.  
\newline{}Schnitzler: mit Bleistift beschriftet: »\textsc{Hugo}« 
\newline{}Ordnung: 1) mit Bleistift von unbekannter Hand nummeriert:
                                    »321«  2) mit Bleistift von unbekannter Hand nummeriert:
                                    »376«}\buchAbdrucke{\weitereDrucke{Hugo von Hofmannsthal, Arthur Schnitzler: \emph{Briefwechsel}. Hg. Therese Nickl und Heinrich Schnitzler. Frankfurt am Main: \emph{S. Fischer} 1964, S. 251.} }\pstart{}{\pb}Herrn D\textsuperscript{r}\pend{}\pstart{}Arthur Schnitzler\pend{}\pstart{}Wien\oindex{Wien@\textbf{Wien}|pw}\pend{}\pstart{}XVIII. Spöttelgasse 7\oindex{Edmund-Weiss-Gasse@\textbf{Edmund-Weiß-Gasse}|pw}.\pend{}{\bigskip}\pstart
           \noindent{}\centering{}\textcolor{gray}{\textbf{{\pb}M. von Schwind\pwindex{Schwind, Moritz von 21.01.1804 – 08.02.1871@\textsc{Schwind, Moritz von} (21.01.1804 – 08.02.1871), \emph{Bildender Künstler}|pw}}}\pend
           \pstart
           \noindent{}\centering{}\textcolor{gray}{\textbf{Die Landpartie – La partie de campagne – The
                        rural excursion\pwindex{Schwind, Moritz von 21.01.1804 – 08.02.1871@\textsc{Schwind, Moritz von} (21.01.1804 – 08.02.1871), \emph{Bildender Künstler}!Landpartie. Schwind und Bauernfeld auf einem Leiterwagenvor 1871@\strich\emph{Die Landpartie. Schwind und Bauernfeld auf einem Leiterwagen} {[}vor 1871{]}|pw} (Schwind\pwindex{Schwind, Moritz von 21.01.1804 – 08.02.1871@\textsc{Schwind, Moritz von} (21.01.1804 – 08.02.1871), \emph{Bildender Künstler}|pw} u. Bauernfeld\pwindex{Bauernfeld, Eduard von 13.01.1802 – 04.08.1890@\textsc{Bauernfeld, Eduard von} (13.01.1802 – 04.08.1890)|pw}).}}\pend
           \pstart
           \centering{}{\pb}München\oindex{Muenchen@\textbf{München}|pw}. 23 VII.\pend
           \pstart
           Ganz ähnlich ſind wir geſtern u. vorgeſtern über Sa\substVorne{}\textsuperscript{z}\substDazwischen{}lz\substHinten{}burg\oindex{Salzburg@\textbf{Salzburg}|pw} hierher gefahren und doch auch wieder unähnlich. Es iſt eine
               Preisaufgabe für \textsc{Heini}\pwindex{Schnitzler, Heinrich 09.08.1902 – 12.07.1982@\textsc{Schnitzler, Heinrich} (09.08.1902 – 12.07.1982), \emph{Regisseur, Schauspieler}|pw}, a. die Ähnlichkeit – b.) die Unterſchiede herauszufinden.\hspace*{1.5em}Viele Grüße Euch beiden.\pend
           \pstart
           \spacefill\mbox{Hugo.}{\\[\baselineskip]}\spacefill\mbox{{[}hs. Louis Philipp Friedmann:{]} L. Friedmann}{\\[\baselineskip]}\spacefill\mbox{{[}hs. Gertrude von Hofmannsthal:{]} Gerty}{\\[\baselineskip]}\spacefill\mbox{{[}hs. Rose Friedmann:{]} Rose Friedmann}\pend
           \leftskip=0em{}
         
         \endnumbering\mylabel{h}\end{ledgroupsized}  \newcommand{\dateiname}{L01951}\newcommand{\titel}{Hugo von Hofmannsthal u. a. an Arthur Schnitzler, 23. 7. 1910}\newcommand{\editorInnen}{Martin Anton Müller und Gerd-Hermann Susen}%% latex-leseansicht-abspann.tex
%% Abspann für die Leseansicht.
%% Der Schalter \ifkorrekturansicht ist bereits durch den Vorspann gesetzt.

%% latex-abspann.tex
%% Gemeinsamer Abspann für Korrekturansicht und Leseansicht.
%% Setzt den Schalter \ifkorrekturansicht voraus (gesetzt in den
%% einbindenden Dateien latex-korrekturansicht-abspann.tex bzw.
%% latex-leseansicht-abspann.tex).
%% ---------------------------------------------------------------

\normalsize

% Das esempio-Environment wird nur in der Leseansicht benötigt
\ifkorrekturansicht\else
\newenvironment{esempio}[3]%
{
    \vspace{1.5ex}
    \rlap{\underline{#1}}
    \par
    \setlength{\parindent}{0cm}
    \nopagebreak
    \leftskip=#2cm
    \rightskip=#3cm
}
{
    \par
}
\fi

\doendnotes{C}
\bigskip
\vfill

\clearpage

\footnotesize

\ifkorrekturansicht
  \lohead{\textsc{register}}
\fi

% theindex-Environment neu definieren ohne reledmac
\makeatletter
\renewenvironment{theindex}{%
  \ifkorrekturansicht
    \section*{\indexname}%
  \else
    \subsubsection*{Index der erwähnten Entitäten}%
  \fi
  \setlength{\parindent}{0pt}%
  \setlength{\parskip}{0pt plus 0.3pt}%
  \let\item\@idxitem
}{%
  \ifkorrekturansicht\clearpage\fi
}
\makeatother

\IfFileExists{\jobname-pw.ind}{\input{\jobname-pw.ind}}{}

% Quellenangabe nur in der Leseansicht
\ifkorrekturansicht\else
% Fallback-Definitionen, falls die .tex-Datei \titel etc. nicht gesetzt hat
\providecommand{\titel}{}
\providecommand{\editorInnen}{}
\providecommand{\dateiname}{\jobname}

\vspace{3cm}

\vfill

\footnotesize
\textsc{Quelle}: \titel. Herausgegeben von {\editorInnen}. In: \emph{Arthur Schnitzler: Briefwechsel mit Autorinnen und Autoren}.
 Digitale Edition, https://schnitzler-briefe.acdh.oeaw.ac.at/{\dateiname}.html (Stand \today)
\fi

\end{document}


      