%% latex-korrekturansicht-vorspann.tex
%% Vorspann für die Korrekturansicht.
%% Lädt die gemeinsame Datei latex-vorspann.tex mit gesetztem Schalter.

\newif\ifkorrekturansicht
\korrekturansichttrue

\input{../tex-inputs/latex-vorspann}


\section[Hugo von Hofmannsthal u. a. an Arthur Schnitzler, 23. 7. 1910]{L01951 Hugo von Hofmannsthal u. a. an Arthur Schnitzler, 23. 7. 1910}
\nopagebreak\mylabel{L01951v}
\rehead{ }\normalsize\beginnumbering\briefempfaengerindex{Schnitzler, Arthur@\textsc{Schnitzler, Arthur}!zzzFriedmann, Rose@\emph{von Rose Friedmann}!1910-07-232@{23. 7. 1910}|(be}\briefempfaengerindex{Schnitzler, Arthur@\textsc{Schnitzler, Arthur}!zzzHofmannsthal, Gertrude von@\emph{von Gertrude von Hofmannsthal}!1910-07-232@{23. 7. 1910}|(be}\briefempfaengerindex{Schnitzler, Arthur@\textsc{Schnitzler, Arthur}!zzzFriedmann, Louis Philipp@\emph{von Louis Philipp Friedmann}!1910-07-232@{23. 7. 1910}|(be}\briefempfaengerindex{Schnitzler, Arthur@\textsc{Schnitzler, Arthur}!zzzHofmannsthal, Hugo von@\emph{von Hugo von Hofmannsthal}!1910-07-232@{23. 7. 1910}|(be}
\toendnotes[C]{\smallbreak\pagebreak[2]}\Standort{CUL, Schnitzler, B 43.}
\physDesc{Bildpostkarte, 325 Zeichen
\newline{}Handschrift Hugo von Hofmannsthal: 1) schwarze Tinte, deutsche Kurrent\hspace{1em}2) schwarze Tinte, lateinische Kurrent (\noindent{}Adresse)\hspace{1em}
\newline{}Handschrift Louis Philipp Friedmann: Bleistift
\newline{}Handschrift Gertrude von Hofmannsthal: schwarze Tinte
\newline{}Handschrift Rose Friedmann: schwarze Tinte
\newline{}Versand: Stempel: »\nobreak{}\oindex{Muenchen@\textbf{München}, \emph{P.PPLA}|pwk}München, 23. 7. 10, 4–5N\nobreak{}«.  
\newline{}Schnitzler: mit Bleistift beschriftet: »\textsc{Hugo}« 
\newline{}Ordnung: 1) mit Bleistift von unbekannter Hand nummeriert:
                                    »321«  2) mit Bleistift von unbekannter Hand nummeriert:
                                    »376«}
\buchAbdrucke{\weitereDrucke{Hugo von Hofmannsthal, Arthur Schnitzler: \emph{Briefwechsel}. Frankfurt am Main: \emph{S. Fischer} 1964, S. 251.} }\pstart{}{\pb}Herrn D\textsuperscript{r}\pend{}\pstart{}Arthur Schnitzler\pend{}\pstart{}Wien\oindex{Wien@\textbf{Wien}, \emph{A.ADM2}|pw}\pend{}\pstart{}XVIII. Spöttelgasse 7\oindex{Edmund-Weiss-Gasse 7@\textbf{Edmund-Weiß-Gasse 7}, \emph{Wohngebäude (K.WHS)}|pw}.\pend{}{\bigskip}
\pstart
           \noindent{}\centering{}{\pb}\textcolor{gray}{\textbf{M. von Schwind\pwindex{Schwind, Moritz von 21.01.1804 – 08.02.1871@\textsc{Schwind, Moritz von} (21.01.1804 – 08.02.1871), \emph{Maler/Malerin}|pw}}}\pend
           
\pstart
           \centering{}\textcolor{gray}{\textbf{Die Landpartie – La partie de campagne – The
                     rural excursion\pwindex{Landpartie. Schwind und Bauernfeld auf einem Leiterwagen@\emph{Die Landpartie. Schwind und Bauernfeld auf einem Leiterwagen}|pw} (Schwind\pwindex{Schwind, Moritz von 21.01.1804 – 08.02.1871@\textsc{Schwind, Moritz von} (21.01.1804 – 08.02.1871), \emph{Maler/Malerin}|pw} u. Bauernfeld\pwindex{Bauernfeld, Eduard von 13.01.1802 – 04.08.1890@\textsc{Bauernfeld, Eduard von} (13.01.1802 – 04.08.1890)|pw}).}}\pend
           \vspace{1em}
\pstart
           \centering{}{\pb}München\oindex{Muenchen@\textbf{München}, \emph{P.PPLA}|pw}. 23 VII.\pend
           \vspace{0.5em}
\pstart
           Ganz ähnlich ſind wir geſtern u. vorgeſtern über Sa\substVorne{}\textsuperscript{z}\substDazwischen{}lz\substHinten{}burg\oindex{Salzburg@\textbf{Salzburg}, \emph{A.ADM2}|pw} hierher gefahren und doch auch wieder unähnlich. Es iſt eine
               Preisaufgabe für \textsc{Heini}\pwindex{Schnitzler, Heinrich 09.08.1902 – 12.07.1982@\textsc{Schnitzler, Heinrich} (09.08.1902 – 12.07.1982), \emph{Regisseur/Regisseurin, Schauspieler/Schauspielerin}|pw}, a. die Ähnlichkeit – b.) die Unterſchiede herauszufinden.\hspace*{1.5em}Viele Grüße Euch beiden.\pend
           
\pstart
           \spacefill\mbox{Hugo.}{\\[\baselineskip]}\spacefill\mbox{{[}hs. :{]} L. Friedmann}{\\[\baselineskip]}\spacefill\mbox{{[}hs. :{]} Gerty}{\\[\baselineskip]}\spacefill\mbox{{[}hs. :{]} Rose Friedmann}\pend
           \leftskip=0em{}\selectlanguage{ngerman}\endnumbering\briefempfaengerindex{Schnitzler, Arthur@\textsc{Schnitzler, Arthur}!zzzFriedmann, Rose@\emph{von Rose Friedmann}!1910-07-232@{23. 7. 1910}|)be}\briefempfaengerindex{Schnitzler, Arthur@\textsc{Schnitzler, Arthur}!zzzHofmannsthal, Gertrude von@\emph{von Gertrude von Hofmannsthal}!1910-07-232@{23. 7. 1910}|)be}\briefempfaengerindex{Schnitzler, Arthur@\textsc{Schnitzler, Arthur}!zzzFriedmann, Louis Philipp@\emph{von Louis Philipp Friedmann}!1910-07-232@{23. 7. 1910}|)be}\briefempfaengerindex{Schnitzler, Arthur@\textsc{Schnitzler, Arthur}!zzzHofmannsthal, Hugo von@\emph{von Hugo von Hofmannsthal}!1910-07-232@{23. 7. 1910}|)be}\mylabel{L01951h}  \normalsize

\doendnotes{C}
\bigskip
\vfill

\clearpage

\footnotesize

\lohead{\textsc{register}}

% Definiere theindex-Environment komplett neu ohne reledmac
\makeatletter
\renewenvironment{theindex}{%
  \section*{\indexname}%
  \setlength{\parindent}{0pt}%
  \setlength{\parskip}{0pt plus 0.3pt}%
  \let\item\@idxitem
}{%
  \clearpage
}
\makeatother

\IfFileExists{\jobname-pw.ind}{\input{\jobname-pw.ind}}{}

\end{document}

      