\input{../tex-inputs/latex-pdf-vorspann}
\begin{center}
            \textcolor{red}{ENTWURF. ENTZIFFERUNG NOCH NICHT KORREKTURGELESEN}
                      \end{center}
            
               \section[Hermann Bahr an Arthur Schnitzler, 14. 6. 1914]{ Hermann Bahr an Arthur Schnitzler, 14. 6. 1914}\nopagebreak\mylabel{v}\rehead{ }\begin{ledgroupsized}[t]{13cm}\normalsize\beginnumbering\briefempfaengerindex{Schnitzler, Arthur@\textsc{Schnitzler, Arthur}!zzzBahr, Hermann@\emph{von Hermann Bahr}!1914-06-141@{14. 6. 1914}|(be} \toendnotes[C]{\smallbreak\pagebreak[2]} \Standort{CUL, Schnitzler, B 5b.}
\physDesc{Brief, 1 Blatt, 1 Seite
\newline{}Handschrift: schwarze Tinte, deutsche Kurrent
\newline{}Schnitzler: 1) mit Bleistift ergänzt »Bahr« 2) mit rotem Buntstift eine Unterstreichung\newline{}Ordnung: mit Bleistift von unbekannter Hand
                           nummeriert: »180« }\buchAbdrucke{\weitereDrucke{Hermann Bahr, Arthur Schnitzler: \emph{Briefwechsel, Aufzeichnungen, Dokumente (1891–1931)}. Hg. Kurt Ifkovits und Martin Anton Müller. Göttingen: \emph{Wallstein} 2018, S. 494.} }\toendnotes[C]{\smallbreak}\pstart
           \raggedleft{}{\pb}Venedig{ }Lido\hspace*{1.5em}\textsc{Villa Trieste}\oindex{Villa Trieste@\textbf{Villa Trieste}|pw}{\\}14. 6. 14\pend
           \pstart\center{}Lieber Arthur!\pend\pstart
           An den Rekurs\pwindex{Schnitzler, Arthur 15.05.1862 – 21.10.1931@\textsc{Schnitzler, Arthur} (15.05.1862 – 21.10.1931), \emph{Schriftsteller, Mediziner}!Reigen. Zehn Dialoge1900@\strich\emph{Reigen. Zehn Dialoge} {[}1900{]}|pwv}{ }Burckhards\pwindex{Burckhard, Max Eugen 14.07.1854 – 16.03.1912@\textsc{Burckhard, Max Eugen} (14.07.1854 – 16.03.1912), \emph{Schriftsteller, Rechtswissenschaftler, Theaterleiter}|pw} erinnere ich mich, weiß aber gar
               nicht, ob ich ihn noch habe, ob er nicht vielleicht noch irgendwo bei Gericht liegt.
               Nun ist das Ungeſchickte nur, daß ich erſt Ende \uline{Auguſt} wieder nach Salzburg\oindex{Salzburg@\textbf{Salzburg}|pw} komme, meine
               Laden u. Kaſten alle verſperrt ſind und ich keinen Menschen in der Wohnung habe, der
               ſuchen könnte. Wenn ich Anfang September wieder daheim bin, will ich gleich einmal
               ſuchen. Hoffentlich hats ſo lang Zeit!\pend
           \pstart
           Dir und Deiner lieben Frau\pwindex{Schnitzler, Olga 17.01.1882 – 13.01.1970@\textsc{Schnitzler, Olga} (17.01.1882 – 13.01.1970), \emph{Schauspielerin, Sängerin}|pwv} von
                  uns Beiden\pwindex{Bahr-Mildenburg, Anna 29.11.1872 – 27.01.1947@\textsc{Bahr-Mildenburg, Anna} (29.11.1872 – 27.01.1947), \emph{Sängerin}|pwv} alles Schönſte
               und Beſte!\pend
           \pstart
           Dein alter{\\[\baselineskip]}\spacefill\mbox{Hermann}\pend
           \leftskip=0em{}\endnumbering\briefempfaengerindex{Schnitzler, Arthur@\textsc{Schnitzler, Arthur}!zzzBahr, Hermann@\emph{von Hermann Bahr}!1914-06-141@{14. 6. 1914}|)be}\mylabel{h}\end{ledgroupsized}  \newcommand{\dateiname}{L02183}\newcommand{\titel}{Hermann Bahr an Arthur Schnitzler, 14. 6. 1914}\newcommand{\editorInnen}{ Kurt Ifkovits,  Martin Anton Müller}\input{../tex-inputs/latex-pdf-abspann}
      