%% latex-leseansicht-vorspann.tex
%% Vorspann für die Leseansicht.
%% Lädt die gemeinsame Datei latex-vorspann.tex mit nicht gesetztem Schalter.

\newif\ifkorrekturansicht
\korrekturansichtfalse

\input{../tex-inputs/latex-vorspann}


\section[Arthur Schnitzler an Hugo von Hofmannsthal, 8. 9. 1906]{L01625 Arthur Schnitzler an Hugo von Hofmannsthal, 8. 9. 1906}
\nopagebreak\mylabel{L01625v}
\rehead{ }\normalsize\beginnumbering\briefempfaengerindex{Hofmannsthal, Hugo von@\textsc{Hofmannsthal, Hugo von}!zzzSchnitzler, Arthur@\emph{von Arthur Schnitzler}!1906-09-081@{8. 9. 1906}|(be}
\toendnotes[C]{\smallbreak\pagebreak[2]}
\correspDesc{Versand  durch Arthur Schnitzler am 8. 9. 1906 in Wien
\newline{}Erhalt  durch Hugo von Hofmannsthal im Zeitraum [9. 9. 1906
                  – 13. 9. 1906?] in St. Gilgen}\toendnotes[C]{\smallbreak}
\Standort{FDH, Hs-30885,125.}
\physDesc{Brief, 1 Blatt, 4 Seiten, 2460 Zeichen
\newline{}Handschrift: schwarze Tinte, deutsche Kurrent}
\buchAbdrucke{\weitereDrucke{Hugo von Hofmannsthal, Arthur Schnitzler: \emph{Briefwechsel}. Herausgegeben von Therese Nickl und Heinrich Schnitzler. Frankfurt am Main: \emph{S. Fischer} 1964, S. 221–222.} }\toendnotes[C]{\smallbreak}
\pstart
           \raggedleft{}{\pb}Wien\oindex{Wien@\textbf{Wien}, \emph{Verwaltungsgebiet}|pw}, 8. 9. 906\pend
           
\pstart{}mein lieber Hugo,\pend\vspace{0.5em}
\pstart
           auch unſer Sommer war gut. In \textsc{Marienlyst}\oindex{Marienlyst@\textbf{Marienlyst}, \emph{Gut}|pw} waren wir volle{ }ſechs Wochen. Schöne Seebäder, höchſt anmuthige
               Waldſpaziergänge, ein angenehmes Hotel. Schrieb ein fünfactiges Stück\pwindex{Schnitzler, Arthur 15.\,5.\,1862 Wien – 21.\,10.\,1931 ebd.@\textsc{Schnitzler, Arthur} (15.\,5.\,1862 Wien – 21.\,10.\,1931 ebd.), \emph{Schriftsteller, Mediziner}!Wort. Tragikomödie in fünf Akten@\strich\emph{Das Wort. Tragikomödie in fünf Akten}|pwv}, das natürlich vorläufig nicht zu
               brauchen iſt und von dem ich noch nicht weiſs, wa{\geminationn} ich
               es vollende. Auch einen Einakter\pwindex{Schnitzler, Arthur 15.\,5.\,1862 Wien – 21.\,10.\,1931 ebd.@\textsc{Schnitzler, Arthur} (15.\,5.\,1862 Wien – 21.\,10.\,1931 ebd.), \emph{Schriftsteller, Mediziner}!Komtesse Mizzi oder: Der Familientag@\strich\emph{Komtesse Mizzi oder: Der Familientag}|pwv} hab ich ausführlich{ }ſkizzirt. Salten\pwindex{Salten, Felix 6.\,9.\,1869 Budapest – 8.\,10.\,1945 Zürich@\textsc{Salten, Felix} (6.\,9.\,1869 Budapest – 8.\,10.\,1945 Zürich), \emph{Schriftsteller, Journalist, Chefredakteur}|pw} und Frau\pwindex{Salten, Ottilie 7.\,3.\,1868 Prag – 22.\,6.\,1942 Zürich@\textsc{Salten, Ottilie} (7.\,3.\,1868 Prag – 22.\,6.\,1942 Zürich), \emph{Schauspielerin}|pwv} war
                  \label{K_L01625-1v}\edtext{einen Nachmittag}{\lemma{\textnormal{\emph{einen Nachmittag}}}\Cendnote{\textnormal{Siehe A. S.: \emph{Tagebuch}, 2. 8. 1906.
               }}}\label{K_L01625-1} bei uns, mit Verwandten\pwindex{Metzl, Richard 20.\,4.\,1870 Prag – 31.\,10.\,1941 Paris@\textsc{Metzl, Richard} (20.\,4.\,1870 Prag – 31.\,10.\,1941 Paris), \emph{Regisseur, Schauspieler, Theatersekretär}|pwv}\pwindex{Metzl, Wladimir 1882 Moskau – 1950 Uckfield@\textsc{Metzl, Wladimir} (1882 Moskau – 1950 Uckfield), \emph{Komponist}|pwv}\pwindex{Metzl, Jelisaweta 1880 oder 1885 – 1960@\textsc{Metzl, Jelisaweta} (1880 oder 1885 – 1960), \emph{Geigerin}|pwv}\pwindex{Salzmann, Michael Emil 19.\,1.\,1858 Szigetvár – 26.\,6.\,1908 Wien@\textsc{Salzmann, Michael Emil} (19.\,1.\,1858 Szigetvár – 26.\,6.\,1908 Wien), \emph{Versicherungsbeamter}|pwv}. Schon nach
               Erledigung der \label{K_L01625-2v}\edtext{Umzugsfrage}{\lemma{\textnormal{\emph{Umzugsfrage}}}\Cendnote{\textnormal{Vgl. XXXX Auszeichnungsfehler: Dokument L03433 nicht gefunden.
               }}}\label{K_L01625-2}\strikeout{.} und daher in guter Sti{\geminationm}ung. Ich freu mich{ }ſehr, daſs er wieder zu uns kommt. Frau Fulda\pwindex{d’Albert, Ida 5.\,12.\,1869 Wien – 6.\,10.\,1926 Berlin@\textsc{d’Albert, Ida} (5.\,12.\,1869 Wien – 6.\,10.\,1926 Berlin), \emph{Schauspielerin}|pw} war ein paar Wochen in \textsc{Marienlyst}\oindex{Marienlyst@\textbf{Marienlyst}, \emph{Gut}|pw} und blieb noch nach unſrer Abreiſe. {\pb}Meine \label{K_L01625-3v}\edtext{Schwägerin\pwindex{Steinrück, Elisabeth 19.\,11.\,1885 – 7.\,4.\,1920 Partenkirchen@\textsc{Steinrück, Elisabeth} (19.\,11.\,1885 – 7.\,4.\,1920 Partenkirchen)|pwv} war in \textsc{Gilleleje}\oindex{Gilleleje@\textbf{Gilleleje}|pw}}{\lemma{\textnormal{\emph{Schwägerin … Gilleleje}}}\Cendnote{\textnormal{Elisabeth Steinrück\pwindex{Steinrück, Elisabeth 19.\,11.\,1885 – 7.\,4.\,1920 Partenkirchen@\textsc{Steinrück, Elisabeth} (19.\,11.\,1885 – 7.\,4.\,1920 Partenkirchen)|pwk} war gesundheitlich seit längerer Zeit angeschlagen, vgl. XXXX Auszeichnungsfehler: Dokument L03416 nicht gefunden.}}}\label{K_L01625-3}, nördlich von \textsc{Marienlyst}\oindex{Marienlyst@\textbf{Marienlyst}, \emph{Gut}|pw}, am offnen Meer, kam dann auf ein paar Tage, mit Steinrück\pwindex{Steinrück, Albert 20.\,5.\,1872 Wetterburg – 11.\,2.\,1929 Berlin@\textsc{Steinrück, Albert} (20.\,5.\,1872 Wetterburg – 11.\,2.\,1929 Berlin), \emph{Schauspieler}|pw} zu uns, wir fuhren gemeinſchaftlich nach \textsc{Kopenhagen\oindex{Kopenhagen@\textbf{Kopenhagen}, \emph{Hauptstadt}|pw}}. Sie iſt jetzt in \textsc{Görbersdorf}\oindex{Görbersdorf@\textbf{Görbersdorf}|pw}, es geht ihr recht gut. Von \textsc{Kopenhagen}\oindex{Kopenhagen@\textbf{Kopenhagen}, \emph{Hauptstadt}|pw} aus wurde Heini\pwindex{Schnitzler, Heinrich 9.\,8.\,1902 Hinterbrühl – 12.\,7.\,1982 Wien@\textsc{Schnitzler, Heinrich} (9.\,8.\,1902 Hinterbrühl – 12.\,7.\,1982 Wien), \emph{Regisseur, Schauspieler}|pw}, dem das Meer{ }ſehr
               imponirt hat und der jetzt wo er kann, mit{ }ſeinen Reiſeerlebniſſen protzt, mit dem
                  Fräulein\pwindex{Loew, Anna *~11.\,4.\,1888 Ješín@\textsc{Loew, Anna} (*~11.\,4.\,1888 Ješín), \emph{Kinderbetreuerin, Dienstbotin}|pwv} nach Wien\oindex{Wien@\textbf{Wien}, \emph{Verwaltungsgebiet}|pw}{ }ſpedirt. Wir zwei fuhren nach Weimar\oindex{Weimar@\textbf{Weimar}, \emph{Verwaltungsgebiet}|pw}, das uns aufs tiefſte ergriff. Fred\pwindex{W. Fred 29.\,6.\,1879 Wien – 23.\,10.\,1922 Berlin@\textsc{W. Fred} (29.\,6.\,1879 Wien – 23.\,10.\,1922 Berlin), \emph{Schriftsteller, Journalist}|pw}, äußerſt{ }ſympathiſch, aber recht leidend, war ein paar
               Tage mit uns zuſa{\geminationm}en. Von Weimar\oindex{Weimar@\textbf{Weimar}, \emph{Verwaltungsgebiet}|pw} nach Ilmenau\oindex{Ilmenau@\textbf{Ilmenau}|pw}, auf den \textsc{Kickelhahn}\oindex{Kickelhahn@\textbf{Kickelhahn}, \emph{Berg}|pw}; von \textsc{Ilmenau}\oindex{Ilmenau@\textbf{Ilmenau}|pw} zu Wagen, {\pb}durch den reizvollen Thüringerwald\oindex{Thüringer Wald@\textbf{Thüringer Wald}, \emph{Gebirge}|pw}, über die Schmücke\oindex{Schmücke@\textbf{Schmücke}, \emph{Gebirge}|pw}, nach Oberhof\oindex{Oberhof [Thüringen]@\textbf{Oberhof [Thüringen]}|pw}, das{ }ſich ganz
               alpenhaft geberdet, gleich weiter nach Eiſenach\oindex{Eisenach@\textbf{Eisenach}|pw}, nach Nürnberg\oindex{Nürnberg@\textbf{Nürnberg}|pw}, wo wir das
               hübſche Marionettentheater von Brann\pwindex{Brann, Paul 5.\,1.\,1873 Oleśnica – 2.\,9.\,1955 Oxford@\textsc{Brann, Paul} (5.\,1.\,1873 Oleśnica – 2.\,9.\,1955 Oxford), \emph{Theaterleiter}|pw}{ }ſahen, und von da nach Wien\oindex{Wien@\textbf{Wien}, \emph{Verwaltungsgebiet}|pw}. Hier{ }ſind wir{ }ſeit beinah drei Wochen. Olga\pwindex{Schnitzler, Olga 17.\,1.\,1882 Wien – 13.\,1.\,1970 Lugano@\textsc{Schnitzler, Olga} (17.\,1.\,1882 Wien – 13.\,1.\,1970 Lugano), \emph{Schauspielerin, Sängerin}|pw} ließ{ }ſich von Julius\pwindex{Schnitzler, Julius 13.\,7.\,1865 Wien – 29.\,6.\,1939 ebd.@\textsc{Schnitzler, Julius} (13.\,7.\,1865 Wien – 29.\,6.\,1939 ebd.), \emph{Chirurg}|pw} eine Kleinigkeit an den Füßen \label{T_L01625-1v}\edtext{operiren}{\lemma{\textnormal{\emph{operiren}}}\Cendnote{\textnormal{geschrieben:
                     »operirte«}}}\label{T_L01625-1},{ }ſo dſs{ }ſie noch nicht Tennis{ }ſpielen kann.
               Ich hingegen{ }ſehr fleißig, beinah täglich. Mit \textsc{Wassermann\pwindex{Wassermann, Jakob 10.\,3.\,1873 Fürth – 1.\,1.\,1934 Altaussee@\textsc{Wassermann, Jakob} (10.\,3.\,1873 Fürth – 1.\,1.\,1934 Altaussee), \emph{Schriftsteller}|pw}, Agnes Speyer\pwindex{Ulmann, Agnes 23.\,12.\,1875 Wien – 1.\,4.\,1942 New York City@\textsc{Ulmann, Agnes} (23.\,12.\,1875 Wien – 1.\,4.\,1942 New York City), \emph{Malerin, Bildhauerin}|pw}, Speidel\pwindex{Speidel, Felix 2.\,7.\,1875 Stuttgart – 3.\,10.\,1952 Unterach am Attersee@\textsc{Speidel, Felix} (2.\,7.\,1875 Stuttgart – 3.\,10.\,1952 Unterach am Attersee), \emph{Schriftsteller, Verleger}|pw}} u Frau\pwindex{Speidel-Haeberle, Else 11.\,7.\,1877 Stuttgart – 21.\,7.\,1937 Augustenfeld@\textsc{Speidel-Haeberle, Else} (11.\,7.\,1877 Stuttgart – 21.\,7.\,1937 Augustenfeld), \emph{Schauspielerin}|pwv}. Arbeite wenig.
               Beſchäftigt mit einem Stück\pwindex{Schnitzler, Arthur 15.\,5.\,1862 Wien – 21.\,10.\,1931 ebd.@\textsc{Schnitzler, Arthur} (15.\,5.\,1862 Wien – 21.\,10.\,1931 ebd.), \emph{Schriftsteller, Mediziner}!Fink und Fliederbusch. Komödie in drei Akten@\strich\emph{Fink und Fliederbusch. Komödie in drei Akten}|pwuv}, das ich{ }ſchon vor 3 Jahren begonnen habe (modern.) – Morgen fahren
               wir alle auf den Semmering\oindex{Semmering@\textbf{Semmering}, \emph{Verwaltungsgebiet}|pw}, für etwa {\pb}acht Tage. Es wäre nicht unmöglich, dſs ich für meinen
               Theil von dort aus noch weiterwandere oder radle, vielleicht mit Waſſermann\pwindex{Wassermann, Jakob 10.\,3.\,1873 Fürth – 1.\,1.\,1934 Altaussee@\textsc{Wassermann, Jakob} (10.\,3.\,1873 Fürth – 1.\,1.\,1934 Altaussee), \emph{Schriftsteller}|pw}, ins Salzka{\geminationm}ergut\oindex{Salzkammergut@\textbf{Salzkammergut}, \emph{Region}|pw}. Laſſen Sie mich jedenfalls wiſſen (Südbahnhotel\oindex{Südbahnhotel [Semmering]@\textbf{Südbahnhotel [Semmering]}, \emph{Hotel}|pw}) wie lange Sie noch in Lueg\oindex{Lueg@\textbf{Lueg}, \emph{Teil eines besiedelten Ortes}|pw} bleiben. Hiemit wäre das äußerliche der
               letzten Monate und der nächſten Zukunft in Kürze mitgetheilt; es gab im übrigen recht
               viele gute Stunden aber mehr hypochondriſche als mit Ruhe zu tragen wären.
               Künſtleriſche Intenſitäten wurden \substVorne{}\textsuperscript{mehr}\substDazwischen{}häufiger\substHinten{} auf Spaziergängen durchlebt als am Schreibtiſch, und die neueſten Geſtalten
               laſſen{ }ſich wohl bis ins tiefſte erkennen aber nicht bis ins letzte regieren. Ich
               freue mich auf unſer nächſtes Zuſa{\geminationm}enſein und erhoffe es
               bald.\pend
           
\pstart
           Herzlichst Ihr{\\[\baselineskip]}\spacefill\mbox{A.}\pend
           \leftskip=0em{}\selectlanguage{ngerman}\endnumbering\briefempfaengerindex{Hofmannsthal, Hugo von@\textsc{Hofmannsthal, Hugo von}!zzzSchnitzler, Arthur@\emph{von Arthur Schnitzler}!1906-09-081@{8. 9. 1906}|)be}\mylabel{L01625h}  \newcommand{\dateiname}{L01625}\newcommand{\titel}{Arthur Schnitzler an Hugo von Hofmannsthal, 8. 9. 1906}\newcommand{\editorInnen}{Martin Anton Müller und Gerd-Hermann Susen}%% latex-leseansicht-abspann.tex
%% Abspann für die Leseansicht.
%% Der Schalter \ifkorrekturansicht ist bereits durch den Vorspann gesetzt.

%% latex-abspann.tex
%% Gemeinsamer Abspann für Korrekturansicht und Leseansicht.
%% Setzt den Schalter \ifkorrekturansicht voraus (gesetzt in den
%% einbindenden Dateien latex-korrekturansicht-abspann.tex bzw.
%% latex-leseansicht-abspann.tex).
%% ---------------------------------------------------------------

\normalsize

% Das esempio-Environment wird nur in der Leseansicht benötigt
\ifkorrekturansicht\else
\newenvironment{esempio}[3]%
{
    \vspace{1.5ex}
    \rlap{\underline{#1}}
    \par
    \setlength{\parindent}{0cm}
    \nopagebreak
    \leftskip=#2cm
    \rightskip=#3cm
}
{
    \par
}
\fi

\doendnotes{C}
\bigskip
\vfill

\clearpage

\footnotesize

\ifkorrekturansicht
  \lohead{\textsc{register}}
\fi

% theindex-Environment neu definieren ohne reledmac
\makeatletter
\renewenvironment{theindex}{%
  \ifkorrekturansicht
    \section*{\indexname}%
  \else
    \subsubsection*{Index der erwähnten Entitäten}%
  \fi
  \setlength{\parindent}{0pt}%
  \setlength{\parskip}{0pt plus 0.3pt}%
  \let\item\@idxitem
}{%
  \ifkorrekturansicht\clearpage\fi
}
\makeatother

\IfFileExists{\jobname-pw.ind}{\input{\jobname-pw.ind}}{}

% Quellenangabe nur in der Leseansicht
\ifkorrekturansicht\else
% Fallback-Definitionen, falls die .tex-Datei \titel etc. nicht gesetzt hat
\providecommand{\titel}{}
\providecommand{\editorInnen}{}
\providecommand{\dateiname}{\jobname}

\vspace{3cm}

\vfill

\footnotesize
\textsc{Quelle}: \titel. Herausgegeben von {\editorInnen}. In: \emph{Arthur Schnitzler: Briefwechsel mit Autorinnen und Autoren}.
 Digitale Edition, https://schnitzler-briefe.acdh.oeaw.ac.at/{\dateiname}.html (Stand \today)
\fi

\end{document}


