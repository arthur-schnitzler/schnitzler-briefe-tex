%% latex-korrekturansicht-vorspann.tex
%% Vorspann für die Korrekturansicht.
%% Lädt die gemeinsame Datei latex-vorspann.tex mit gesetztem Schalter.

\newif\ifkorrekturansicht
\korrekturansichttrue

\input{../tex-inputs/latex-vorspann}


\section[Arthur Schnitzler an Hugo von Hofmannsthal, 8. 9. 1906]{L01625 Arthur Schnitzler an Hugo von Hofmannsthal, 8. 9. 1906}
\nopagebreak\mylabel{L01625v}
\rehead{ }\normalsize\beginnumbering\briefempfaengerindex{Hofmannsthal, Hugo von@\textsc{Hofmannsthal, Hugo von}!zzzSchnitzler, Arthur@\emph{von Arthur Schnitzler}!1906-09-081@{8. 9. 1906}|(be}
\toendnotes[C]{\smallbreak\pagebreak[2]}\Standort{FDH, Hs-30885,125.}
\physDesc{Brief, 1 Blatt, 4 Seiten, 2460 Zeichen
\newline{}Handschrift: schwarze Tinte, deutsche Kurrent}
\buchAbdrucke{\weitereDrucke{Hugo von Hofmannsthal, Arthur Schnitzler: \emph{Briefwechsel}. Frankfurt am Main: \emph{S. Fischer} 1964, S. 221–222.} }\toendnotes[C]{\smallbreak}
\pstart
           \raggedleft{}{\pb}Wien\oindex{Wien@\textbf{Wien}, \emph{A.ADM2}|pw}, 8. 9. 906\pend
           
\pstart{}mein lieber Hugo, \pend\vspace{0.5em}
\pstart
           auch unſer Sommer war gut. In \textsc{Marienlyst}\oindex{Marienlyst@\textbf{Marienlyst}, \emph{S.EST}|pw} waren wir volle ſechs Wochen. Schöne Seebäder, höchſt anmuthige
               Waldſpaziergänge, ein angenehmes Hotel. Schrieb ein fünfactiges Stück\pwindex{Wort. Tragikomoedie in fuenf Akten@\emph{Das Wort. Tragikomödie in fünf Akten}|pwv}, das natürlich vorläufig nicht zu
               brauchen iſt und von dem ich noch nicht weiſs, wa{\geminationn} ich
               es vollende. Auch einen Einakter\pwindex{Komtesse Mizzi oder: Der Familientag@\emph{Komtesse Mizzi oder: Der Familientag}|pwv} hab ich ausführlich ſkizzirt. Salten\pwindex{Salten, Felix 06.09.1869 – 08.10.1945@\textsc{Salten, Felix} (06.09.1869 – 08.10.1945), \emph{Schriftsteller/Schriftstellerin, Journalist/Journalistin, Chefredakteur/Chefredakteurin}|pw} und Frau\pwindex{Salten, Ottilie 07.03.1868 – 22.06.1942@\textsc{Salten, Ottilie} (07.03.1868 – 22.06.1942), \emph{Schauspieler/Schauspielerin}|pwv} war
                  \label{K_L01625-1v}\edtext{einen Nachmittag}{\lemma{\textnormal{\emph{einen Nachmittag}}}\Cendnote{\textnormal{Siehe A. S.: \emph{Tagebuch}, 2. 8. 1906.
               }}}\label{K_L01625-1} bei uns, mit Verwandten\pwindex{Metzl, Richard 20.04.1870 – 31.10.1941@\textsc{Metzl, Richard} (20.04.1870 – 31.10.1941), \emph{Regisseur/Regisseurin, Schauspieler/Schauspielerin, Theatersekretär/Theatersekretärin}|pwv}\pwindex{Metzl, Wladimir 1882 – 1950@\textsc{Metzl, Wladimir} (1882 – 1950), \emph{Komponist/Komponistin}|pwv}\pwindex{Metzl, Jelisaweta 1880 oder 1885 – 1960@\textsc{Metzl, Jelisaweta} (1880 oder 1885 – 1960), \emph{Geiger/Geigerin}|pwv}\pwindex{Salzmann, Michael Emil 1858-01-19 – 1908-06-26@\textsc{Salzmann, Michael Emil} (1858-01-19 – 1908-06-26), \emph{Versicherungsbeamter/Versicherungsbeamtin}|pwv}. Schon nach
               Erledigung der \label{K_L01625-2v}\edtext{Umzugsfrage}{\lemma{\textnormal{\emph{Umzugsfrage}}}\Cendnote{\textnormal{Vgl. Felix Salten an Arthur Schnitzler, 23. 8. 1906.
               }}}\label{K_L01625-2}\strikeout{.} und daher in guter Sti{\geminationm}ung. Ich freu mich ſehr, daſs er wieder zu uns kommt. Frau Fulda\pwindex{DAlbert, Ida 05.12.1869 – 1926-10-06@\textsc{d’Albert, Ida} (05.12.1869 – 1926-10-06), \emph{Schauspieler/Schauspielerin}|pw} war ein paar Wochen in \textsc{Marienlyst}\oindex{Marienlyst@\textbf{Marienlyst}, \emph{S.EST}|pw} und blieb noch nach unſrer Abreiſe. {\pb}Meine \label{K_L01625-3v}\edtext{Schwägerin\pwindex{Steinrueck, Elisabeth 19.11.1885 – 07.04.1920@\textsc{Steinrück, Elisabeth} (19.11.1885 – 07.04.1920)|pwv} war in \textsc{Gilleleje}\oindex{Gilleleje@\textbf{Gilleleje}, \emph{P.PPL}|pw}}{\lemma{\textnormal{\emph{Schwägerin … Gilleleje}}}\Cendnote{\textnormal{Elisabeth Steinrück\pwindex{Steinrueck, Elisabeth 19.11.1885 – 07.04.1920@\textsc{Steinrück, Elisabeth} (19.11.1885 – 07.04.1920)|pwk} war gesundheitlich seit längerer Zeit angeschlagen, vgl. Felix Salten an Arthur Schnitzler, 28. 3. 1906.}}}\label{K_L01625-3}, nördlich von \textsc{Marienlyst}\oindex{Marienlyst@\textbf{Marienlyst}, \emph{S.EST}|pw}, am offnen Meer, kam dann auf ein paar Tage, mit Steinrück\pwindex{Steinrueck, Albert 20.05.1872 – 11.02.1929@\textsc{Steinrück, Albert} (20.05.1872 – 11.02.1929), \emph{Schauspieler/Schauspielerin}|pw} zu uns, wir fuhren gemeinſchaftlich nach \textsc{Kopenhagen\oindex{Kopenhagen@\textbf{Kopenhagen}, \emph{P.PPLC}|pw}}. Sie iſt jetzt in \textsc{Görbersdorf}\oindex{Goerbersdorf@\textbf{Görbersdorf}, \emph{P.PPL}|pw}, es geht ihr recht gut. Von \textsc{Kopenhagen}\oindex{Kopenhagen@\textbf{Kopenhagen}, \emph{P.PPLC}|pw} aus wurde Heini\pwindex{Schnitzler, Heinrich 09.08.1902 – 12.07.1982@\textsc{Schnitzler, Heinrich} (09.08.1902 – 12.07.1982), \emph{Regisseur/Regisseurin, Schauspieler/Schauspielerin}|pw}, dem das Meer ſehr
               imponirt hat und der jetzt wo er kann, mit ſeinen Reiſeerlebniſſen protzt, mit dem
                  Fräulein\pwindex{Loew, Anna *~11.04.1888@\textsc{Loew, Anna} (*~11.04.1888), \emph{Kinderbetreuer/Kinderbetreuerin, Dienstbote/Dienstbotin}|pwv} nach Wien\oindex{Wien@\textbf{Wien}, \emph{A.ADM2}|pw}{ }ſpedirt. Wir zwei fuhren nach Weimar\oindex{Weimar@\textbf{Weimar}, \emph{A.ADM3}|pw}, das uns aufs tiefſte ergriff. Fred\pwindex{W. Fred 29.06.1879 – 23.10.1922@\textsc{W. Fred} (29.06.1879 – 23.10.1922), \emph{Schriftsteller/Schriftstellerin, Journalist/Journalistin}|pw}, äußerſt ſympathiſch, aber recht leidend, war ein paar
               Tage mit uns zuſa{\geminationm}en. Von Weimar\oindex{Weimar@\textbf{Weimar}, \emph{A.ADM3}|pw} nach Ilmenau\oindex{Ilmenau@\textbf{Ilmenau}, \emph{P.PPL}|pw}, auf den \textsc{Kickelhahn}\oindex{Kickelhahn@\textbf{Kickelhahn}, \emph{Berg (N.BRG)}|pw}; von \textsc{Ilmenau}\oindex{Ilmenau@\textbf{Ilmenau}, \emph{P.PPL}|pw} zu Wagen, {\pb}durch den reizvollen Thüringerwald\oindex{Thueringer Wald@\textbf{Thüringer Wald}, \emph{Gebirge (N.GBR)}|pw}, über die Schmücke\oindex{Schmuecke@\textbf{Schmücke}, \emph{Gebirge (N.GBR)}|pw}, nach Oberhof\oindex{Oberhof [Thueringen]@\textbf{Oberhof [Thüringen]}, \emph{Besiedelter Ort (A.BSO)}|pw}, das ſich ganz
               alpenhaft geberdet, gleich weiter nach Eiſenach\oindex{Eisenach@\textbf{Eisenach}, \emph{P.PPL}|pw}, nach Nürnberg\oindex{Nuernberg@\textbf{Nürnberg}, \emph{P.PPL}|pw}, wo wir das
               hübſche Marionettentheater von Brann\pwindex{Brann, Paul 05.01.1873 – 02.09.1955@\textsc{Brann, Paul} (05.01.1873 – 02.09.1955), \emph{Theaterleiter/Theaterleiterin}|pw}{ }ſahen, und von da nach Wien\oindex{Wien@\textbf{Wien}, \emph{A.ADM2}|pw}. Hier ſind wir ſeit beinah drei Wochen. Olga\pwindex{Schnitzler, Olga 17.01.1882 – 13.01.1970@\textsc{Schnitzler, Olga} (17.01.1882 – 13.01.1970), \emph{Schauspieler/Schauspielerin, Sänger/Sängerin}|pw} ließ ſich von Julius\pwindex{Schnitzler, Julius 13.07.1865 – 29.06.1939@\textsc{Schnitzler, Julius} (13.07.1865 – 29.06.1939), \emph{Chirurg/Chirurgin}|pw} eine Kleinigkeit an den Füßen \label{T_L01625-1v}\edtext{operiren}{\lemma{\textnormal{\emph{operiren}}}\Cendnote{\textnormal{geschrieben:
                     »operirte«}}}\label{T_L01625-1}, ſo dſs ſie noch nicht Tennis ſpielen kann.
               Ich hingegen ſehr fleißig, beinah täglich. Mit \textsc{Wassermann\pwindex{Wassermann, Jakob 10.03.1873 – 01.01.1934@\textsc{Wassermann, Jakob} (10.03.1873 – 01.01.1934), \emph{Schriftsteller/Schriftstellerin}|pw}, Agnes Speyer\pwindex{Ulmann, Agnes 23. 12. 1875 – 1. 4. 1942@\textsc{Ulmann, Agnes} (23. 12. 1875 – 1. 4. 1942), \emph{Maler/Malerin, Bildhauer/Bildhauerin}|pw}, Speidel\pwindex{Speidel, Felix 02.07.1875 – 1952-10-03@\textsc{Speidel, Felix} (02.07.1875 – 1952-10-03), \emph{Schriftsteller/Schriftstellerin, Verleger/Verlegerin}|pw}} u Frau\pwindex{Speidel-Haeberle, Else 11.07.1877 – 21.07.1937@\textsc{Speidel-Haeberle, Else} (11.07.1877 – 21.07.1937), \emph{Schauspieler/Schauspielerin}|pwv}. Arbeite wenig.
               Beſchäftigt mit einem Stück\pwindex{Fink und Fliederbusch. Komoedie in drei Akten@\emph{Fink und Fliederbusch. Komödie in drei Akten}|pwuv}, das ich ſchon vor 3 Jahren begonnen habe (modern.) – Morgen fahren
               wir alle auf den Semmering\oindex{Semmering@\textbf{Semmering}, \emph{A.ADM3}|pw}, für etwa {\pb}acht Tage. Es wäre nicht unmöglich, dſs ich für meinen
               Theil von dort aus noch weiterwandere oder radle, vielleicht mit Waſſermann\pwindex{Wassermann, Jakob 10.03.1873 – 01.01.1934@\textsc{Wassermann, Jakob} (10.03.1873 – 01.01.1934), \emph{Schriftsteller/Schriftstellerin}|pw}, ins Salzka{\geminationm}ergut\oindex{Salzkammergut@\textbf{Salzkammergut}, \emph{L.RGN}|pw}. Laſſen Sie mich jedenfalls wiſſen (Südbahnhotel\oindex{Suedbahnhotel [Semmering]@\textbf{Südbahnhotel [Semmering]}, \emph{Hotel (K.HTL)}|pw}) wie lange Sie noch in Lueg\oindex{Lueg@\textbf{Lueg}, \emph{Teil eines besiedelten Ortes (A.BSOX)}|pw} bleiben. Hiemit wäre das äußerliche der
               letzten Monate und der nächſten Zukunft in Kürze mitgetheilt; es gab im übrigen recht
               viele gute Stunden aber mehr hypochondriſche als mit Ruhe zu tragen wären.
               Künſtleriſche Intenſitäten wurden \substVorne{}\textsuperscript{mehr}\substDazwischen{}häufiger\substHinten{} auf Spaziergängen durchlebt als am Schreibtiſch, und die neueſten Geſtalten
               laſſen ſich wohl bis ins tiefſte erkennen aber nicht bis ins letzte regieren. Ich
               freue mich auf unſer nächſtes Zuſa{\geminationm}enſein und erhoffe es
               bald.\pend
           
\pstart
           Herzlichst Ihr{\\[\baselineskip]}\spacefill\mbox{A.}\pend
           \leftskip=0em{}\selectlanguage{ngerman}\endnumbering\briefempfaengerindex{Hofmannsthal, Hugo von@\textsc{Hofmannsthal, Hugo von}!zzzSchnitzler, Arthur@\emph{von Arthur Schnitzler}!1906-09-081@{8. 9. 1906}|)be}\mylabel{L01625h}  \normalsize

\doendnotes{C}
\bigskip
\vfill

\clearpage

\footnotesize

\lohead{\textsc{register}}

% Definiere theindex-Environment komplett neu ohne reledmac
\makeatletter
\renewenvironment{theindex}{%
  \section*{\indexname}%
  \setlength{\parindent}{0pt}%
  \setlength{\parskip}{0pt plus 0.3pt}%
  \let\item\@idxitem
}{%
  \clearpage
}
\makeatother

\IfFileExists{\jobname-pw.ind}{\input{\jobname-pw.ind}}{}

\end{document}

      