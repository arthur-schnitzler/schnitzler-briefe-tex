%% latex-leseansicht-vorspann.tex
%% Vorspann für die Leseansicht.
%% Lädt die gemeinsame Datei latex-vorspann.tex mit nicht gesetztem Schalter.

\newif\ifkorrekturansicht
\korrekturansichtfalse

\input{../tex-inputs/latex-vorspann}


               \section[Arthur Schnitzler an Hugo von Hofmannsthal, 8. 9. 1906]{ Arthur Schnitzler an Hugo von Hofmannsthal, 8. 9. 1906}\nopagebreak\mylabel{v}\rehead{ }\begin{ledgroupsized}[t]{13cm}\normalsize\beginnumbering\briefempfaengerindex{Hofmannsthal, Hugo von@\textsc{Hofmannsthal, Hugo von}!zzzSchnitzler, Arthur@\emph{von Arthur Schnitzler}!1906-09-081@{8. 9. 1906}|(be} \toendnotes[C]{\smallbreak\pagebreak[2]} \Standort{FDH, Hs-30885,125.}
\physDesc{Brief, 1 Blatt, 4 Seiten
\newline{}Handschrift: schwarze Tinte, deutsche Kurrent}\buchAbdrucke{\weitereDrucke{Hugo von Hofmannsthal, Arthur Schnitzler: \emph{Briefwechsel}. Hg. Therese Nickl und Heinrich Schnitzler. Frankfurt am Main: \emph{S. Fischer} 1964, S. 221–222.} }\toendnotes[C]{\smallbreak}\pstart
           \raggedleft{}{\pb}Wien\oindex{Wien@\textbf{Wien}|pw}, 8. 9. 906\pend
           \pstart{}mein lieber Hugo, \pend\pstart
           auch unſer Sommer war gut. In \textsc{Marienlyst}\oindex{Marienlyst@\textbf{Marienlyst}|pw} waren wir volle ſechs Wochen. Schöne Seebäder, höchſt
               anmuthige Waldſpaziergänge, ein angenehmes Hotel. Schrieb ein fünfactiges Stück\pwindex{Schnitzler, Arthur 15.05.1862 – 21.10.1931@\textsc{Schnitzler, Arthur} (15.05.1862 – 21.10.1931), \emph{Schriftsteller, Mediziner}!Wort. Tragikomoedie in fuenf Akten1966@\strich\emph{Das Wort. Tragikomödie in fünf Akten} {[}1966{]}|pwv}, das natürlich vorläufig
               nicht zu brauchen iſt und von dem ich noch nicht weiſs, wa{\geminationn} ich es vollende. Auch einen Einakter\pwindex{Schnitzler, Arthur 15.05.1862 – 21.10.1931@\textsc{Schnitzler, Arthur} (15.05.1862 – 21.10.1931), \emph{Schriftsteller, Mediziner}!Komtesse Mizzi oder Der Familientag1908-04-19@\strich\emph{Komtesse Mizzi oder Der Familientag} {[}1908-04-19{]}|pwv} hab ich ausführlich ſkizzirt. Salten\pwindex{Salten, Felix 06.09.1869 – 08.10.1945@\textsc{Salten, Felix} (06.09.1869 – 08.10.1945), \emph{Schriftsteller, Journalist}|pw} und Frau\pwindex{Salten, Ottilie 07.03.1868 – 22.06.1942@\textsc{Salten, Ottilie} (07.03.1868 – 22.06.1942), \emph{Schauspielerin}|pwv} war \label{K_L01625_1v}\edtext{einen Nachmittag}{\lemma{\textnormal{\emph{einen Nachmittag}}}\Cendnote{\textnormal{siehe A. S.: \emph{Tagebuch}, 2. 8. 1906}}}\label{K_L01625_1h} bei uns, mit
                  Verwandten\pwindex{Metzl, Richard 20.04.1870 – 31.10.1941@\textsc{Metzl, Richard} (20.04.1870 – 31.10.1941), \emph{Regisseur, Schauspieler, Theatersekretär}|pwv}\pwindex{Metzl, Wladimir 1882 – 1950@\textsc{Metzl, Wladimir} (1882 – 1950), \emph{Komponist}|pwv}\pwindex{Metzl @\textsc{Metzl}|pwv}\pwindex{Salzmann, Emil 1858-01-19 – 1908-06-26@\textsc{Salzmann, Emil} (1858-01-19 – 1908-06-26)|pwv}. Schon nach Erledigung der \label{K_L01625_2v}\edtext{Umzugsfrage}{\lemma{\textnormal{\emph{Umzugsfrage}}}\Cendnote{\textnormal{Sie übersiedelten zum
                     15. 9. 1906 aus Berlin\oindex{Berlin@\textbf{Berlin}|pwk}
                  nach Wien\oindex{Wien@\textbf{Wien}|pwk}.}}}\label{K_L01625_2h}\strikeout{.}
               und daher in guter Sti{\geminationm}ung. Ich freu mich ſehr, daſs er
               wieder zu uns kommt. Frau Fulda\pwindex{DAlbert, Ida 05.12.1869 – 1926-10-06@\textsc{d’Albert, Ida} (05.12.1869 – 1926-10-06)|pw} war ein paar
               Wochen in \textsc{Marienlyst}\oindex{Marienlyst@\textbf{Marienlyst}|pw} und
               blieb noch nach unſrer Abreiſe. {\pb}Meine Schwägerin\pwindex{Steinrueck, Elisabeth 19.11.1885 – 07.04.1920@\textsc{Steinrück, Elisabeth} (19.11.1885 – 07.04.1920)|pwv} war in \textsc{Gilleleje}\oindex{Gilleleje@\textbf{Gilleleje}|pw}, nördlich von \textsc{Marienlyst}\oindex{Marienlyst@\textbf{Marienlyst}|pw}, am offnen Meer, kam dann
               auf ein paar Tage, mit Steinrück\pwindex{Steinrueck, Albert 20.05.1872 – 11.02.1929@\textsc{Steinrück, Albert} (20.05.1872 – 11.02.1929), \emph{Schauspieler}|pw} zu uns, wir
               fuhren gemeinſchaftlich nach \textsc{Kopenhagen\oindex{Kopenhagen@\textbf{Kopenhagen}|pw}}. Sie iſt jetzt in \textsc{Görbersdorf}\oindex{Goerbersdorf@\textbf{Görbersdorf}|pw}, es geht ihr recht gut. Von \textsc{Kopenhagen}\oindex{Kopenhagen@\textbf{Kopenhagen}|pw} aus wurde Heini\pwindex{Schnitzler, Heinrich 09.08.1902 – 12.07.1982@\textsc{Schnitzler, Heinrich} (09.08.1902 – 12.07.1982), \emph{Regisseur, Schauspieler}|pw}, dem das Meer ſehr imponirt hat und der
               jetzt wo er kann, mit ſeinen Reiſeerlebniſſen protzt, mit dem Fräulein\pwindex{Loew, Anna *~11.04.1888@\textsc{Loew, Anna} (*~11.04.1888), \emph{Kinderbetreuerin, Dienstbotin}|pwv} nach Wien\oindex{Wien@\textbf{Wien}|pw}{ }ſpedirt. Wir zwei fuhren nach Weimar\oindex{Weimar@\textbf{Weimar}|pw}, das uns aufs tiefſte ergriff. Fred\pwindex{W. Fred 29.06.1879 – 23.10.1922@\textsc{W. Fred} (29.06.1879 – 23.10.1922), \emph{Schriftsteller, Journalist}|pw}, äußerſt ſympathiſch, aber recht leidend, war ein paar
               Tage mit uns zuſa{\geminationm}en. Von Weimar\oindex{Weimar@\textbf{Weimar}|pw} nach Ilmenau\oindex{Ilmenau@\textbf{Ilmenau}|pw}, auf den \textsc{Kickelhahn}\oindex{Kickelhahn@\textbf{Kickelhahn}|pw}; von \textsc{Ilmenau}\oindex{Ilmenau@\textbf{Ilmenau}|pw} zu Wagen, {\pb}durch den reizvollen Thüringerwald\oindex{Thueringer Wald@\textbf{Thüringer Wald}|pw}, über die Schmücke\oindex{Schmuecke@\textbf{Schmücke}|pw}, nach
                  Oberhof\oindex{Oberhof@\textbf{Oberhof}|pw}, das ſich ganz alpenhaft geberdet,
               gleich weiter nach Eiſenach\oindex{Eisenach@\textbf{Eisenach}|pw}, nach Nürnberg\oindex{Nuernberg@\textbf{Nürnberg}|pw}, wo wir das hübſche Marionettentheater von Brann\pwindex{Brann, Paul 05.01.1873 – 02.09.1955@\textsc{Brann, Paul} (05.01.1873 – 02.09.1955), \emph{Theaterleiter}|pw}{ }ſahen, und von da nach Wien\oindex{Wien@\textbf{Wien}|pw}. Hier ſind wir ſeit beinah drei Wochen. Olga\pwindex{Schnitzler, Olga 17.01.1882 – 13.01.1970@\textsc{Schnitzler, Olga} (17.01.1882 – 13.01.1970), \emph{Schauspielerin, Sängerin}|pw} ließ ſich von Julius\pwindex{Schnitzler, Julius 13.07.1865 – 29.06.1939@\textsc{Schnitzler, Julius} (13.07.1865 – 29.06.1939), \emph{Chirurg}|pw} eine
               Kleinigkeit an den Füßen \label{T_L01625_1v}\edtext{operiren}{\lemma{\textnormal{\emph{operiren}}}\Cendnote{\textnormal{geschrieben:
                  »operirte«}}}\label{T_L01625_1h}, ſo dſs ſie noch nicht Tennis ſpielen kann. Ich
               hingegen ſehr fleißig, beinah täglich. Mit \textsc{Wassermann\pwindex{Wassermann, Jakob 10.03.1873 – 01.01.1934@\textsc{Wassermann, Jakob} (10.03.1873 – 01.01.1934), \emph{Schriftsteller}|pw}, Agnes
                  Speyer\pwindex{Ulmann, Agnes 23. 12. 1875 – 1. 4. 1942@\textsc{Ulmann, Agnes} (23. 12. 1875 – 1. 4. 1942), \emph{Malerin, Bildhauerin}|pw}, Speidel\pwindex{Speidel, Felix 02.07.1875 – 1952-10-03@\textsc{Speidel, Felix} (02.07.1875 – 1952-10-03), \emph{Schriftsteller, Verleger}|pw}} u Frau\pwindex{Speidel-Haeberle, Else 11.07.1877 – 21.07.1937@\textsc{Speidel-Haeberle, Else} (11.07.1877 – 21.07.1937), \emph{Schauspielerin}|pwv}. Arbeite wenig. Beſchäftigt mit einem Stück\pwindex{Schnitzler, Arthur 15.05.1862 – 21.10.1931@\textsc{Schnitzler, Arthur} (15.05.1862 – 21.10.1931), \emph{Schriftsteller, Mediziner}!Fink und Fliederbusch. Komoedie in drei Akten1917@\strich\emph{Fink und Fliederbusch. Komödie in drei Akten} {[}1917{]}|pwuv}, das ich ſchon
               vor 3 Jahren begonnen habe (modern.) – Morgen fahren wir alle auf den Semmering\oindex{Semmering@\textbf{Semmering}|pw}, für etwa {\pb}acht Tage. Es
               wäre nicht unmöglich, dſs ich für meinen Theil von dort aus noch weiterwandere oder
               radle, vielleicht mit Waſſermann\pwindex{Wassermann, Jakob 10.03.1873 – 01.01.1934@\textsc{Wassermann, Jakob} (10.03.1873 – 01.01.1934), \emph{Schriftsteller}|pw}, ins Salzka{\geminationm}ergut\oindex{Salzkammergut@\textbf{Salzkammergut}|pw}. Laſſen Sie
               mich jedenfalls wiſſen (Südbahnhotel\oindex{Suedbahnhotel@\textbf{Südbahnhotel}|pw}) wie lange Sie
               noch in Lueg\oindex{Lueg am Wolfgangsee@\textbf{Lueg am Wolfgangsee}|pw} bleiben. Hiemit wäre das äußerliche der
               letzten Monate und der nächſten Zukunft in Kürze mitgetheilt; es gab im übrigen recht
               viele gute Stunden aber mehr hypochondriſche als mit Ruhe zu tragen wären.
               Künſtleriſche Intenſitäten wurden \substVorne{}\textsuperscript{mehr}\substDazwischen{}häufiger\substHinten{} auf Spaziergängen durchlebt als am Schreibtiſch, und die neueſten Geſtalten
               laſſen ſich wohl bis ins tiefſte erkennen aber nicht bis ins letzte regieren. Ich
               freue mich auf unſer nächſtes Zuſa{\geminationm}enſein und erhoffe es
               bald.\pend
           \pstart
           Herzlichst Ihr{\\[\baselineskip]}\spacefill\mbox{A.}\pend
           \leftskip=0em{}          \endnumbering\briefempfaengerindex{Hofmannsthal, Hugo von@\textsc{Hofmannsthal, Hugo von}!zzzSchnitzler, Arthur@\emph{von Arthur Schnitzler}!1906-09-081@{8. 9. 1906}|)be}\mylabel{h}\end{ledgroupsized}  \newcommand{\dateiname}{L01625}\newcommand{\titel}{Arthur Schnitzler an Hugo von Hofmannsthal, 8. 9. 1906}\newcommand{\editorInnen}{Martin Anton Müller und Gerd-Hermann Susen}
            \footnotesize
\begin{ledgroupsized}[t]{11.5cm}
\doendnotes{C}
\end{ledgroupsized}
         %% latex-leseansicht-abspann.tex
%% Abspann für die Leseansicht.
%% Der Schalter \ifkorrekturansicht ist bereits durch den Vorspann gesetzt.

%% latex-abspann.tex
%% Gemeinsamer Abspann für Korrekturansicht und Leseansicht.
%% Setzt den Schalter \ifkorrekturansicht voraus (gesetzt in den
%% einbindenden Dateien latex-korrekturansicht-abspann.tex bzw.
%% latex-leseansicht-abspann.tex).
%% ---------------------------------------------------------------

\normalsize

% Das esempio-Environment wird nur in der Leseansicht benötigt
\ifkorrekturansicht\else
\newenvironment{esempio}[3]%
{
    \vspace{1.5ex}
    \rlap{\underline{#1}}
    \par
    \setlength{\parindent}{0cm}
    \nopagebreak
    \leftskip=#2cm
    \rightskip=#3cm
}
{
    \par
}
\fi

\doendnotes{C}
\bigskip
\vfill

\clearpage

\footnotesize

\ifkorrekturansicht
  \lohead{\textsc{register}}
\fi

% theindex-Environment neu definieren ohne reledmac
\makeatletter
\renewenvironment{theindex}{%
  \ifkorrekturansicht
    \section*{\indexname}%
  \else
    \subsubsection*{Index der erwähnten Entitäten}%
  \fi
  \setlength{\parindent}{0pt}%
  \setlength{\parskip}{0pt plus 0.3pt}%
  \let\item\@idxitem
}{%
  \ifkorrekturansicht\clearpage\fi
}
\makeatother

\IfFileExists{\jobname-pw.ind}{\input{\jobname-pw.ind}}{}

% Quellenangabe nur in der Leseansicht
\ifkorrekturansicht\else
% Fallback-Definitionen, falls die .tex-Datei \titel etc. nicht gesetzt hat
\providecommand{\titel}{}
\providecommand{\editorInnen}{}
\providecommand{\dateiname}{\jobname}

\vspace{3cm}

\vfill

\footnotesize
\textsc{Quelle}: \titel. Herausgegeben von {\editorInnen}. In: \emph{Arthur Schnitzler: Briefwechsel mit Autorinnen und Autoren}.
 Digitale Edition, https://schnitzler-briefe.acdh.oeaw.ac.at/{\dateiname}.html (Stand \today)
\fi

\end{document}


      