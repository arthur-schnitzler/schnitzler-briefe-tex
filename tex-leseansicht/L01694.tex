%% latex-leseansicht-vorspann.tex
%% Vorspann für die Leseansicht.
%% Lädt die gemeinsame Datei latex-vorspann.tex mit nicht gesetztem Schalter.

\newif\ifkorrekturansicht
\korrekturansichtfalse

\input{../tex-inputs/latex-vorspann}


\section[Richard Beer-Hofmann an Arthur Schnitzler, 26. 7. 1907]{L01694 Richard Beer-Hofmann an Arthur Schnitzler, 26. 7. 1907}
\nopagebreak\mylabel{L01694v}
\rehead{ }\normalsize\beginnumbering\briefempfaengerindex{Schnitzler, Arthur@\textsc{Schnitzler, Arthur}!zzzBeer-Hofmann, Richard@\emph{von Richard Beer-Hofmann}!1907-07-261@{26. 7. 1907}|(be}
\toendnotes[C]{\smallbreak\pagebreak[2]}
\correspDesc{Versand  durch Richard Beer-Hofmann am 26. 7. 1907 in Maria Schutz
\newline{}Erhalt  durch Arthur Schnitzler im Zeitraum [27. 7. 1907
                  – 31. 7. 1907?] in Welsberg-Taisten}\toendnotes[C]{\smallbreak}
\Standort{CUL, Schnitzler, B 8.}
\physDesc{Brief, 1 Blatt, 3 Seiten, 1055 Zeichen
\newline{}Handschrift: Bleistift, lateinische Kurrent
\newline{}Ordnung: mit Bleistift von unbekannter Hand nummeriert:
                                    »209« }
\buchAbdrucke{\weitereDrucke{Arthur Schnitzler, Richard Beer-Hofmann: \emph{Briefwechsel 1891–1931}. Herausgegeben von Konstanze Fliedl. Wien, Zürich: \emph{Europaverlag} 1992, S. 181.} }
\pstart
           \raggedleft{}{\pb}Maria-Schutz\oindex{Maria Schutz@\textbf{Maria Schutz}|pw}{ }26./VII.  07.\pend
           \vspace{0.5em}
\pstart
           Lieber Arthur! Ihren lieben Brief vom 14. habe ich
               erhalten. Am 4. Abends sind wir hier angekommen, am 6. bin
               ich nach Wien\oindex{Wien@\textbf{Wien}, \emph{Verwaltungsgebiet}|pw} zurück, am 7. wieder
               hieher um Paula\pwindex{Beer-Hofmann, Paula 25.\,2.\,1879 Wien – 30.\,10.\,1939 Zürich@\textsc{Beer-Hofmann, Paula} (25.\,2.\,1879 Wien – 30.\,10.\,1939 Zürich)|pw} zu holen, und von
                  8. an bis zum 11. waren {[}wir{]} wieder
               in Wien\oindex{Wien@\textbf{Wien}, \emph{Verwaltungsgebiet}|pw}, die letzten zwei Tage davon in Purkersdorf\oindex{Purkersdorf@\textbf{Purkersdorf}, \emph{Verwaltungsgebiet}|pw}. Ich habe für lange, ich glaube für
               sehr lange, einen recht bittern Geschmack im Munde.\pend
           
\pstart
           Wir bleiben bis 3./VIII. hier, gehen dann nach Wien\oindex{Wien@\textbf{Wien}, \emph{Verwaltungsgebiet}|pw}. Zwischen 14. und 19. August
               wollen wir {\pb}nach Villach\oindex{Villach@\textbf{Villach}, \emph{Verwaltungsgebiet}|pw}, um an irgendeinem Kärntner\oindex{Kärnten@\textbf{Kärnten}, \emph{Land}|pw}see für 8 Tage unterzuko{\geminationm}en. Dann Südtirol\oindex{Südtirol@\textbf{Südtirol}, \emph{Verwaltungsgebiet}|pw}, womöglich Gardasee\oindex{Lago di Garda@\textbf{Lago di Garda}, \emph{See}|pw}. Waren Sie im Lido-Hôtel\oindex{Palast Hotel Lido@\textbf{Palast Hotel Lido}, \emph{Hotel}|pw} in Riva\oindex{Riva del Garda@\textbf{Riva del Garda}, \emph{Hauptstadt}|pw}
               zufrieden? Und hat das Hôtel eine wirkliche Bade\uline{anstalt}? Mit Schwimmmeister? Ist vielleicht Hôtel du Lac (Witzmann)\oindex{Hotel du Lac@\textbf{Hotel du Lac}, \emph{Hotel}|pw} zu empfehlen? oder Torbole\oindex{Torbole sul Garda@\textbf{Torbole sul Garda}, \emph{Hauptstadt}|pw}? Da Sie Ende August in Bozens\oindex{Bozen@\textbf{Bozen}, \emph{Hauptstadt}|pw} Umgebung sein wollen, so rechne ich damit Sie um diese Zeit irgendwo
               sehen zu können. Ich würde {\pb}mich
               sehr freuen. Ich glaube, es wird ganz leicht gehen, wenn Sie mich rechtzeitig
               verständigen. Von Kärnten\oindex{Kärnten@\textbf{Kärnten}, \emph{Land}|pw} nach Bozen\oindex{Bozen@\textbf{Bozen}, \emph{Hauptstadt}|pw} möchte ich über die neue Dolomitenstrasse\oindex{Große Dolomitenstraße@\textbf{Große Dolomitenstraße}, \emph{Straße}|pw}.\pend
           
\pstart
           Ich grüsse Sie, Olga\pwindex{Schnitzler, Olga 17.\,1.\,1882 Wien – 13.\,1.\,1970 Lugano@\textsc{Schnitzler, Olga} (17.\,1.\,1882 Wien – 13.\,1.\,1970 Lugano), \emph{Schauspielerin, Sängerin}|pw} und Heini\pwindex{Schnitzler, Heinrich 9.\,8.\,1902 Hinterbrühl – 12.\,7.\,1982 Wien@\textsc{Schnitzler, Heinrich} (9.\,8.\,1902 Hinterbrühl – 12.\,7.\,1982 Wien), \emph{Regisseur, Schauspieler}|pw} herzlichst. Paula\pwindex{Beer-Hofmann, Paula 25.\,2.\,1879 Wien – 30.\,10.\,1939 Zürich@\textsc{Beer-Hofmann, Paula} (25.\,2.\,1879 Wien – 30.\,10.\,1939 Zürich)|pw}
               tut dasselbe.\pend
           
\pstart
           Ihr{\\[\baselineskip]}\spacefill\mbox{Richard}\pend
           \leftskip=0em{}\selectlanguage{ngerman}\endnumbering\briefempfaengerindex{Schnitzler, Arthur@\textsc{Schnitzler, Arthur}!zzzBeer-Hofmann, Richard@\emph{von Richard Beer-Hofmann}!1907-07-261@{26. 7. 1907}|)be}\mylabel{L01694h}  \newcommand{\dateiname}{L01694}\newcommand{\titel}{Richard Beer-Hofmann an Arthur Schnitzler, 26. 7. 1907}\newcommand{\editorInnen}{Martin Anton Müller und Gerd-Hermann Susen}%% latex-leseansicht-abspann.tex
%% Abspann für die Leseansicht.
%% Der Schalter \ifkorrekturansicht ist bereits durch den Vorspann gesetzt.

%% latex-abspann.tex
%% Gemeinsamer Abspann für Korrekturansicht und Leseansicht.
%% Setzt den Schalter \ifkorrekturansicht voraus (gesetzt in den
%% einbindenden Dateien latex-korrekturansicht-abspann.tex bzw.
%% latex-leseansicht-abspann.tex).
%% ---------------------------------------------------------------

\normalsize

% Das esempio-Environment wird nur in der Leseansicht benötigt
\ifkorrekturansicht\else
\newenvironment{esempio}[3]%
{
    \vspace{1.5ex}
    \rlap{\underline{#1}}
    \par
    \setlength{\parindent}{0cm}
    \nopagebreak
    \leftskip=#2cm
    \rightskip=#3cm
}
{
    \par
}
\fi

\doendnotes{C}
\bigskip
\vfill

\clearpage

\footnotesize

\ifkorrekturansicht
  \lohead{\textsc{register}}
\fi

% theindex-Environment neu definieren ohne reledmac
\makeatletter
\renewenvironment{theindex}{%
  \ifkorrekturansicht
    \section*{\indexname}%
  \else
    \subsubsection*{Index der erwähnten Entitäten}%
  \fi
  \setlength{\parindent}{0pt}%
  \setlength{\parskip}{0pt plus 0.3pt}%
  \let\item\@idxitem
}{%
  \ifkorrekturansicht\clearpage\fi
}
\makeatother

\IfFileExists{\jobname-pw.ind}{\input{\jobname-pw.ind}}{}

% Quellenangabe nur in der Leseansicht
\ifkorrekturansicht\else
% Fallback-Definitionen, falls die .tex-Datei \titel etc. nicht gesetzt hat
\providecommand{\titel}{}
\providecommand{\editorInnen}{}
\providecommand{\dateiname}{\jobname}

\vspace{3cm}

\vfill

\footnotesize
\textsc{Quelle}: \titel. Herausgegeben von {\editorInnen}. In: \emph{Arthur Schnitzler: Briefwechsel mit Autorinnen und Autoren}.
 Digitale Edition, https://schnitzler-briefe.acdh.oeaw.ac.at/{\dateiname}.html (Stand \today)
\fi

\end{document}


