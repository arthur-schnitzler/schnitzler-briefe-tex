%% latex-korrekturansicht-vorspann.tex
%% Vorspann für die Korrekturansicht.
%% Lädt die gemeinsame Datei latex-vorspann.tex mit gesetztem Schalter.

\newif\ifkorrekturansicht
\korrekturansichttrue

\input{../tex-inputs/latex-vorspann}


\section[ Felix Salten an Arthur Schnitzler, 10. 3. 1907]{L03436 Felix Salten an Arthur Schnitzler, 10. 3. 1907}
\nopagebreak\mylabel{L03436v}
\rehead{ }\normalsize\beginnumbering\briefempfaengerindex{Schnitzler, Arthur@\textsc{Schnitzler, Arthur}!zzzSalten, Felix@\emph{von Felix Salten}!1907-03-101@{10. 3. 1907}|(be}
\toendnotes[C]{\smallbreak\pagebreak[2]}\Standort{CUL, Schnitzler, B 89, B 1.}
\physDesc{Brief, 1 Blatt, 1 Seite, 617 Zeichen
\newline{}Handschrift: schwarze Tinte, lateinische Kurrent
\newline{}Ordnung: mit Bleistift von unbekannter Hand nummeriert: »227« }\toendnotes[C]{\smallbreak}
\pstart
           \raggedleft{}{\pb}Heiligenstadt\oindex{Heiligenstadt@\textbf{Heiligenstadt}, \emph{P.PPL}|pw}. 10. III. 07\pend
           \vspace{0.5em}
\pstart
           Lieber, danke schön für Ihr neues \label{K_L03436-1v}\edtext{Buch\pwindex{Daemmerseelen. Novellen@\emph{Dämmerseelen. Novellen}|pwv}}{\lemma{\textnormal{\emph{Buch}}}\Cendnote{\textnormal{Siehe Arthur Schnitzler: Widmungsexemplar Dämmerseelen für Felix
               Salten, 2. 3. 1907.
               }}}\label{K_L03436-1}. Es kam heute{ }früh, ich hab es vormittag gleich gelesen und es hat wieder
               sehr auf mich gewirkt. Am meisten der Leisenbohg\pwindex{Schicksal des Freiherrn von Leisenbohg. Novellette@\emph{Das Schicksal des Freiherrn von Leisenbohg. Novellette}|pw}
               und der Thameyer\pwindex{Andreas Thameyers letzter Brief@\emph{Andreas Thameyers letzter Brief}|pw}. Dann noch »die Fremde\pwindex{Daemmerseele@\emph{Dämmerseele}|pw}«. Gegen das »neue
                  Lied\pwindex{neue Lied@\emph{Das neue Lied}|pw}« hätte ich einiges zu sagen. Zunächst scheint mir das Anekdotische darin
               nicht ganz überwunden. Ein Roman, dessen Art aus dem Leisenbohg\pwindex{Schicksal des Freiherrn von Leisenbohg. Novellette@\emph{Das Schicksal des Freiherrn von Leisenbohg. Novellette}|pw}, der Fremden\pwindex{Daemmerseele@\emph{Dämmerseele}|pw}, und Thameyer\pwindex{Andreas Thameyers letzter Brief@\emph{Andreas Thameyers letzter Brief}|pw} sich zusammensetzte, der diese Farben
               und Schatten brächte, müßte etwas ganz Unvergleichliches sein. Hoffentlich \label{K_L03436-2v}\edtext{sehen wir uns bald}{\lemma{\textnormal{\emph{sehen wir uns bald}}}\Cendnote{\textnormal{Nachweisbar sahen sich die beiden am 16. 3. 1907
                  wieder.}}}\label{K_L03436-2}. Es ist noch manches über das Buch\pwindex{Daemmerseelen. Novellen@\emph{Dämmerseelen. Novellen}|pwv} zu sagen.\pend
           
\pstart
           Viele Grüße von uns\pwindex{Salten, Ottilie 07.03.1868 – 22.06.1942@\textsc{Salten, Ottilie} (07.03.1868 – 22.06.1942), \emph{Schauspieler/Schauspielerin}|pwv} zu
               Ihnen. {\\[\baselineskip]}Ihr {\\[\baselineskip]}\spacefill\mbox{Felix Salten}\pend
           \leftskip=0em{}\selectlanguage{ngerman}\endnumbering\briefempfaengerindex{Schnitzler, Arthur@\textsc{Schnitzler, Arthur}!zzzSalten, Felix@\emph{von Felix Salten}!1907-03-101@{10. 3. 1907}|)be}\mylabel{L03436h}  \normalsize

\doendnotes{C}
\bigskip
\vfill

\clearpage

\footnotesize

\lohead{\textsc{register}}

% Definiere theindex-Environment komplett neu ohne reledmac
\makeatletter
\renewenvironment{theindex}{%
  \section*{\indexname}%
  \setlength{\parindent}{0pt}%
  \setlength{\parskip}{0pt plus 0.3pt}%
  \let\item\@idxitem
}{%
  \clearpage
}
\makeatother

\IfFileExists{\jobname-pw.ind}{\input{\jobname-pw.ind}}{}

\end{document}

      