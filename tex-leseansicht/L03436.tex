%% latex-leseansicht-vorspann.tex
%% Vorspann für die Leseansicht.
%% Lädt die gemeinsame Datei latex-vorspann.tex mit nicht gesetztem Schalter.

\newif\ifkorrekturansicht
\korrekturansichtfalse

\input{../tex-inputs/latex-vorspann}

\begin{center}
            \textcolor{red}{ENTWURF, NICHT FERTIG KORRIGIERT}
                      \end{center}
            
         
         \renewcommand{\erwaehntePersonen}{Personen: Ottilie Salten}
         \renewcommand{\erwaehnteOrte}{Orte: Heiligenstadt, Wien}
         \renewcommand{\erwaehnteWerke}{Werke: Andreas Thameyers letzter Brief, Das Schicksal des Freiherrn von Leisenbohg. Novellette, Das neue Lied, Dämmerseele, Dämmerseelen. Novellen}
               \section[ Felix Salten an Arthur Schnitzler, 10. 3. 1907]{ Felix Salten an Arthur Schnitzler, 10. 3. 1907}\nopagebreak\mylabel{v}\rehead{ }\begin{ledgroupsized}[t]{13cm}\normalsize\beginnumbering \toendnotes[C]{\smallbreak\pagebreak[2]} \Standort{CUL, Schnitzler, B 89, B 1.}
\physDesc{Brief, 1 Blatt, 1 Seite, 618 Zeichen
\newline{}Handschrift: schwarze Tinte, lateinische Kurrent
\newline{}Ordnung: mit Bleistift von unbekannter Hand nummeriert: »227« }\toendnotes[C]{\smallbreak}\pstart
           \raggedleft{}{\pb}Heiligenstadt\oindex{Heiligenstadt@\textbf{Heiligenstadt}|pw}. 10. III. 07\pend
           \pstart
           Lieber, danke schön für Ihr neues \label{K_L03436-1v}\edtext{Buch\pwindex{Schnitzler, Arthur 15.05.1862 – 21.10.1931@\textsc{Schnitzler, Arthur} (15.05.1862 – 21.10.1931), \emph{Schriftsteller, Mediziner}!Daemmerseelen. Novellen1907-03-07@\strich\emph{Dämmerseelen. Novellen} {[}1907-03-07{]}|pwv}}{\lemma{\textnormal{\emph{Buch}}}\Cendnote{\textnormal{siehe Arthur Schnitzler: Widmungsexemplar Dämmerseelen für Felix
               Salten, 2. 3. 1907}}}\label{K_L03436-1h}. Es kam heute{ }früh, ich hab es vormittag gleich gelesen und es hat wieder
               sehr auf mich gewirkt. Am meisten der Leisenbohg\pwindex{Schnitzler, Arthur 15.05.1862 – 21.10.1931@\textsc{Schnitzler, Arthur} (15.05.1862 – 21.10.1931), \emph{Schriftsteller, Mediziner}!Schicksal des Freiherrn von Leisenbohg. Novellette01. 07. 1904@\strich\emph{Das Schicksal des Freiherrn von Leisenbohg. Novellette} {[}01. 07. 1904{]}|pw}
               und der Thameyer\pwindex{Schnitzler, Arthur 15.05.1862 – 21.10.1931@\textsc{Schnitzler, Arthur} (15.05.1862 – 21.10.1931), \emph{Schriftsteller, Mediziner}!Andreas Thameyers letzter Brief1902-07-26@\strich\emph{Andreas Thameyers letzter Brief} {[}1902-07-26{]}|pw}. Dann noch »die Fremde\pwindex{Schnitzler, Arthur 15.05.1862 – 21.10.1931@\textsc{Schnitzler, Arthur} (15.05.1862 – 21.10.1931), \emph{Schriftsteller, Mediziner}!Daemmerseele18. 5. 1902@\strich\emph{Dämmerseele} {[}18. 5. 1902{]}|pw}«. Gegen das »neue
                  Lied\pwindex{Schnitzler, Arthur 15.05.1862 – 21.10.1931@\textsc{Schnitzler, Arthur} (15.05.1862 – 21.10.1931), \emph{Schriftsteller, Mediziner}!neue Lied23. 04. 1905@\strich\emph{Das neue Lied} {[}23. 04. 1905{]}|pw}« hätte ich einiges zu sagen. Zunächst scheint mir das Anekdotische darin
               nicht ganz überwunden. Ein Roman, dessen Art aus dem Leisenbohg\pwindex{Schnitzler, Arthur 15.05.1862 – 21.10.1931@\textsc{Schnitzler, Arthur} (15.05.1862 – 21.10.1931), \emph{Schriftsteller, Mediziner}!Schicksal des Freiherrn von Leisenbohg. Novellette01. 07. 1904@\strich\emph{Das Schicksal des Freiherrn von Leisenbohg. Novellette} {[}01. 07. 1904{]}|pw}, der Fremden\pwindex{Schnitzler, Arthur 15.05.1862 – 21.10.1931@\textsc{Schnitzler, Arthur} (15.05.1862 – 21.10.1931), \emph{Schriftsteller, Mediziner}!Daemmerseele18. 5. 1902@\strich\emph{Dämmerseele} {[}18. 5. 1902{]}|pw}, und Thameyer\pwindex{Schnitzler, Arthur 15.05.1862 – 21.10.1931@\textsc{Schnitzler, Arthur} (15.05.1862 – 21.10.1931), \emph{Schriftsteller, Mediziner}!Andreas Thameyers letzter Brief1902-07-26@\strich\emph{Andreas Thameyers letzter Brief} {[}1902-07-26{]}|pw} sich zusammensetzte, der diese Farben
               und Schatten brächte, müßte etwas ganz Unvergleichliches sein. Hoffentlich \label{K_L03436-2v}\edtext{sehen wir uns bald}{\lemma{\textnormal{\emph{sehen wir uns bald}}}\Cendnote{\textnormal{Nachweisbar sahen sich die beiden am 16. 3. 1907
                  wieder.}}}\label{K_L03436-2h}. Es ist noch manches über das Buch\pwindex{Schnitzler, Arthur 15.05.1862 – 21.10.1931@\textsc{Schnitzler, Arthur} (15.05.1862 – 21.10.1931), \emph{Schriftsteller, Mediziner}!Daemmerseelen. Novellen1907-03-07@\strich\emph{Dämmerseelen. Novellen} {[}1907-03-07{]}|pwv} zu sagen.\pend
           \pstart
           Viele Grüße von uns\pwindex{Salten, Ottilie 07.03.1868 – 22.06.1942@\textsc{Salten, Ottilie} (07.03.1868 – 22.06.1942), \emph{Schauspielerin}|pwv} zu
               Ihnen. {\\[\baselineskip]}Ihr {\\[\baselineskip]}\spacefill\mbox{Felix Salten}\pend
           \leftskip=0em{}
         
         \endnumbering\mylabel{h}\end{ledgroupsized}  \newcommand{\dateiname}{L03436}\newcommand{\titel}{Felix Salten an Arthur Schnitzler, 10. 3. 1907}\newcommand{\editorInnen}{Martin Anton Müller und Laura Untner}%% latex-leseansicht-abspann.tex
%% Abspann für die Leseansicht.
%% Der Schalter \ifkorrekturansicht ist bereits durch den Vorspann gesetzt.

%% latex-abspann.tex
%% Gemeinsamer Abspann für Korrekturansicht und Leseansicht.
%% Setzt den Schalter \ifkorrekturansicht voraus (gesetzt in den
%% einbindenden Dateien latex-korrekturansicht-abspann.tex bzw.
%% latex-leseansicht-abspann.tex).
%% ---------------------------------------------------------------

\normalsize

% Das esempio-Environment wird nur in der Leseansicht benötigt
\ifkorrekturansicht\else
\newenvironment{esempio}[3]%
{
    \vspace{1.5ex}
    \rlap{\underline{#1}}
    \par
    \setlength{\parindent}{0cm}
    \nopagebreak
    \leftskip=#2cm
    \rightskip=#3cm
}
{
    \par
}
\fi

\doendnotes{C}
\bigskip
\vfill

\clearpage

\footnotesize

\ifkorrekturansicht
  \lohead{\textsc{register}}
\fi

% theindex-Environment neu definieren ohne reledmac
\makeatletter
\renewenvironment{theindex}{%
  \ifkorrekturansicht
    \section*{\indexname}%
  \else
    \subsubsection*{Index der erwähnten Entitäten}%
  \fi
  \setlength{\parindent}{0pt}%
  \setlength{\parskip}{0pt plus 0.3pt}%
  \let\item\@idxitem
}{%
  \ifkorrekturansicht\clearpage\fi
}
\makeatother

\IfFileExists{\jobname-pw.ind}{\input{\jobname-pw.ind}}{}

% Quellenangabe nur in der Leseansicht
\ifkorrekturansicht\else
% Fallback-Definitionen, falls die .tex-Datei \titel etc. nicht gesetzt hat
\providecommand{\titel}{}
\providecommand{\editorInnen}{}
\providecommand{\dateiname}{\jobname}

\vspace{3cm}

\vfill

\footnotesize
\textsc{Quelle}: \titel. Herausgegeben von {\editorInnen}. In: \emph{Arthur Schnitzler: Briefwechsel mit Autorinnen und Autoren}.
 Digitale Edition, https://schnitzler-briefe.acdh.oeaw.ac.at/{\dateiname}.html (Stand \today)
\fi

\end{document}


      