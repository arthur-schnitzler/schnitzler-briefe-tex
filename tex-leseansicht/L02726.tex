%% latex-leseansicht-vorspann.tex
%% Vorspann für die Leseansicht.
%% Lädt die gemeinsame Datei latex-vorspann.tex mit nicht gesetztem Schalter.

\newif\ifkorrekturansicht
\korrekturansichtfalse

\input{../tex-inputs/latex-vorspann}


         
         \renewcommand{\erwaehntePersonen}{Personen: Josef von Bezecný, Paul Goldmann, Julius Schnitzler, Helene Schnitzler, Leopold Sonnemann, Carl von Torresani-Lanzenfeld}
         \renewcommand{\erwaehnteInstitutionen}{Institutionen: Artaria {\kaufmannsund}  Co., Burgtheater, Frankfurter Zeitung}
         \renewcommand{\erwaehnteOrte}{Orte: Bad Ischl, Burgtheater, Frankreich, Paris, Ringstraße, Wien, rue Feydeau}
         \renewcommand{\erwaehnteWerke}{Werke: Lehmann’s Allgemeiner Wohnungs-Anzeiger, Liebelei. Schauspiel in drei Akten}
               \section[Paul Goldmann an Arthur Schnitzler, 5. 1. {[}1895{]}]{ Paul Goldmann an Arthur Schnitzler, 5. 1. {[}1895{]}}\nopagebreak\mylabel{v}\rehead{ }\begin{ledgroupsized}[t]{13cm}\normalsize\beginnumbering\briefempfaengerindex{Schnitzler, Arthur@\textsc{Schnitzler, Arthur}!zzzGoldmann, Paul@\emph{von Paul Goldmann}!1895-01-051@{5. 1. {[}1895{]}}|(be} \toendnotes[C]{\smallbreak\pagebreak[2]} \Standort{DLA, A:Schnitzler, HS.NZ85.1.3165.}
\physDesc{Brief, 3 Blätter, 12 Seiten, 5616 Zeichen
\newline{}Handschrift: schwarze Tinte, deutsche Kurrent
\newline{}Schnitzler: 1) mit rotem Buntstift »\textsc{Liebelei\pwindex{Schnitzler, Arthur 15.05.1862 – 21.10.1931@\textsc{Schnitzler, Arthur} (15.05.1862 – 21.10.1931), \emph{Schriftsteller, Mediziner}!Liebelei. Schauspiel in drei Akten1895-10-09@\strich\emph{Liebelei. Schauspiel in drei Akten} {[}1895-10-09{]}|pw}}« vermerkt  2) mit Bleistift das Jahr »1895« vermerkt}\toendnotes[C]{\smallbreak}\pstart
           \noindent{}{\pb}\textcolor{gray}{\textbf{\textbf{Frankfurter Zeitung\orgindex{Frankfurter Zeitung@Frankfurter Zeitung|pw}}}}\pend
           \pstart
           \textcolor{gray}{\textbf{(\begin{otherlanguage}{french}Gazette de Francfort\end{otherlanguage}\orgindex{Frankfurter Zeitung@Frankfurter Zeitung|pw}).}}\hfill \textsc{Paris}\oindex{Paris@\textbf{Paris}|pw}, 5. Januar.\pend
           \pstart
           \textcolor{gray}{\textbf{\textbf{\begin{otherlanguage}{french}Fondateur M. L.
                              Sonnemann\pwindex{Sonnemann, Leopold 1831-10-29 – 1909-10-30@\textsc{Sonnemann, Leopold} (1831-10-29 – 1909-10-30), \emph{Journalist, Herausgeber}|pw}\end{otherlanguage}.}}}\pend
           \pstart
           \begin{otherlanguage}{french}\textcolor{gray}{\textbf{Journal politique, financier,}}\end{otherlanguage}\pend
           \pstart
           \begin{otherlanguage}{french}\textcolor{gray}{\textbf{commercial et littéraire.}}\end{otherlanguage}\pend
           \pstart
           \begin{otherlanguage}{french}\textcolor{gray}{\textbf{\textbf{Paraissant trois fois par jour.}}}\end{otherlanguage}\pend
           \pstart
           \begin{otherlanguage}{french}\textcolor{gray}{\textbf{\textbf{Bureau à Paris\oindex{Paris@\textbf{Paris}|pw}:}}}\end{otherlanguage}\pend
           \pstart
           \begin{otherlanguage}{french}\textcolor{gray}{\textbf{\textbf{24. Rue Feydeau\oindex{rue Feydeau@\textbf{rue Feydeau}|pw}.}}}\end{otherlanguage}\pend
           \pstart\center{}Mein lieber Freund,\pend\pstart
           Ich danke Dir von Herzen, daß Du meine Bitte ſo raſch erfüllt haſt. Entſchuldige nur
               die \label{K_L02726-1v}\edtext{großen Koſten}{\lemma{\textnormal{\emph{großen Koſten}}}\Cendnote{\textnormal{Schnitzler\pwindex{Schnitzler, Arthur 15.05.1862 – 21.10.1931@\textsc{Schnitzler, Arthur} (15.05.1862 – 21.10.1931), \emph{Schriftsteller, Mediziner}|pwk} hatte am 1. 1. 1895 eine
                  Abschrift von \emph{Liebelei}\pwindex{Schnitzler, Arthur 15.05.1862 – 21.10.1931@\textsc{Schnitzler, Arthur} (15.05.1862 – 21.10.1931), \emph{Schriftsteller, Mediziner}!Liebelei. Schauspiel in drei Akten1895-10-09@\strich\emph{Liebelei. Schauspiel in drei Akten} {[}1895-10-09{]}|pwk} geschickt.}}}\label{K_L02726-1h}, die
               ich Dir verurſacht; aber Du haſt mir eine große Freude gemacht. Mittags
               bekam ich es\pwindex{Schnitzler, Arthur 15.05.1862 – 21.10.1931@\textsc{Schnitzler, Arthur} (15.05.1862 – 21.10.1931), \emph{Schriftsteller, Mediziner}!Liebelei. Schauspiel in drei Akten1895-10-09@\strich\emph{Liebelei. Schauspiel in drei Akten} {[}1895-10-09{]}|pwv}, in einer Stunde
               war es geleſen, und am ſelben Tage ſende ich es Dir
               noch zurück.\pend
           \pstart
           Da ich ſofort ſchreiben muß, bin ich meiner Eindrücke noch nicht ganz ſicher. Der
               erſte Akt\pwindex{Schnitzler, Arthur 15.05.1862 – 21.10.1931@\textsc{Schnitzler, Arthur} (15.05.1862 – 21.10.1931), \emph{Schriftsteller, Mediziner}!Liebelei. Schauspiel in drei Akten1895-10-09@\strich\emph{Liebelei. Schauspiel in drei Akten} {[}1895-10-09{]}|pwv} iſt voll Anmuth,
               voll Bewegung, er endet aufs {\pb}Packendſte. Ich
               glaube, er wird ſehr gut geſpielt werden müſſen. Die zwangloſe, natürliche
               Fröhlichkeit ſtellt den Komödianten keine leichte Aufgabe. Auch möchte ich gleich
               hier ſagen, daß ich beſonders dieſe einfache Sprache überall bewundert habe. \strikeout{Das} Die Leute ſprechen im Stück\pwindex{Schnitzler, Arthur 15.05.1862 – 21.10.1931@\textsc{Schnitzler, Arthur} (15.05.1862 – 21.10.1931), \emph{Schriftsteller, Mediziner}!Liebelei. Schauspiel in drei Akten1895-10-09@\strich\emph{Liebelei. Schauspiel in drei Akten} {[}1895-10-09{]}|pwv}, wie im Leben. Welch’ eine Kunſt da
               drinſteckt! Im zweiten Akt\pwindex{Schnitzler, Arthur 15.05.1862 – 21.10.1931@\textsc{Schnitzler, Arthur} (15.05.1862 – 21.10.1931), \emph{Schriftsteller, Mediziner}!Liebelei. Schauspiel in drei Akten1895-10-09@\strich\emph{Liebelei. Schauspiel in drei Akten} {[}1895-10-09{]}|pwv} –
               und auch ſonſt – hätte ich gern, daß der alte \textsc{Weiring\pwindex{Schnitzler, Arthur 15.05.1862 – 21.10.1931@\textsc{Schnitzler, Arthur} (15.05.1862 – 21.10.1931), \emph{Schriftsteller, Mediziner}!Liebelei. Schauspiel in drei Akten1895-10-09@\strich\emph{Liebelei. Schauspiel in drei Akten} {[}1895-10-09{]}|pwv}} etwas mehr \strikeout{he\textcolor{gray}{r}} hervorträte, als blos mit ein wenig Profil. Ich hätte ihn etwas ausführlicher
               gewünſcht, \strikeout{umſomehr als ich} eine kleine Scene
               rührender Vaterliebe zwiſchen ihm und dem Mädel hätte das Ende\pwindex{Schnitzler, Arthur 15.05.1862 – 21.10.1931@\textsc{Schnitzler, Arthur} (15.05.1862 – 21.10.1931), \emph{Schriftsteller, Mediziner}!Liebelei. Schauspiel in drei Akten1895-10-09@\strich\emph{Liebelei. Schauspiel in drei Akten} {[}1895-10-09{]}|pwv}{ }{\pb}noch um eine \textsc{Nuance}
               tragiſcher gemacht. »Ich alter Mann\strikeout{,} habe nur noch
               Dich.« Es gibt nichts mehr zum Weinen, als hilfloſes, verlaſſenes Alter. Zudem bin
               ich überzeugt, daß der \label{K_L02726-2v}\edtext{Herr\pwindex{Bezecný, Josef von 05.02.1829 – 17.06.1904@\textsc{Bezecný, Josef von} (05.02.1829 – 17.06.1904), \emph{Theaterintendant}|pwv}, der von
                  Cenſur-Schwierigkeiten}{\lemma{\textnormal{\emph{Herr, … Cenſur-Schwierigkeiten}}}\Cendnote{\textnormal{Siehe A. S.: \emph{Tagebuch}, 26. 12. 1894.}}}\label{K_L02726-2h}
               ſprach, gerade die Reden \textsc{Weirings\pwindex{Schnitzler, Arthur 15.05.1862 – 21.10.1931@\textsc{Schnitzler, Arthur} (15.05.1862 – 21.10.1931), \emph{Schriftsteller, Mediziner}!Liebelei. Schauspiel in drei Akten1895-10-09@\strich\emph{Liebelei. Schauspiel in drei Akten} {[}1895-10-09{]}|pwv}{ }} über Tugend
               und Behütung von Glück gemeint hat. Das iſt zwar eine Hauptſache, ein Grundgedanke
               des Stück\pwindex{Schnitzler, Arthur 15.05.1862 – 21.10.1931@\textsc{Schnitzler, Arthur} (15.05.1862 – 21.10.1931), \emph{Schriftsteller, Mediziner}!Liebelei. Schauspiel in drei Akten1895-10-09@\strich\emph{Liebelei. Schauspiel in drei Akten} {[}1895-10-09{]}|pwv}es. Das liegt aber
               den Trotteln wenig auf. Niemals wird man im kaiſerlichen \strikeout{Hofteh}{ }Hoftheater\orgindex{Burgtheater@Burgtheater|pwv} ſo etwas ſagen
               laſſen. Sonſt iſt die Scene\pwindex{Schnitzler, Arthur 15.05.1862 – 21.10.1931@\textsc{Schnitzler, Arthur} (15.05.1862 – 21.10.1931), \emph{Schriftsteller, Mediziner}!Liebelei. Schauspiel in drei Akten1895-10-09@\strich\emph{Liebelei. Schauspiel in drei Akten} {[}1895-10-09{]}|pwv}
               ergreifend. Die Abſchiedsſcene\pwindex{Schnitzler, Arthur 15.05.1862 – 21.10.1931@\textsc{Schnitzler, Arthur} (15.05.1862 – 21.10.1931), \emph{Schriftsteller, Mediziner}!Liebelei. Schauspiel in drei Akten1895-10-09@\strich\emph{Liebelei. Schauspiel in drei Akten} {[}1895-10-09{]}|pwv}
               hätte ich auch {\pb}noch um einen Grad kräftiger
               gewünſcht, mit etwas mehr Betonung darauf, daß es der \uline{Abſchied} ist. \introOben{}Auch ſollte er einmal vom Sterben ſprechen
                  und Angſt zeigen.\introOben{} Sonſt iſt ſie entzückend. Der Schluß\pwindex{Schnitzler, Arthur 15.05.1862 – 21.10.1931@\textsc{Schnitzler, Arthur} (15.05.1862 – 21.10.1931), \emph{Schriftsteller, Mediziner}!Liebelei. Schauspiel in drei Akten1895-10-09@\strich\emph{Liebelei. Schauspiel in drei Akten} {[}1895-10-09{]}|pwv} mit der letzten Umarmung \strikeout{\textcolor{gray}{m}} wird ungeheuer wirken. Einfach, aber ſo ſchön! Der dritte Akt\pwindex{Schnitzler, Arthur 15.05.1862 – 21.10.1931@\textsc{Schnitzler, Arthur} (15.05.1862 – 21.10.1931), \emph{Schriftsteller, Mediziner}!Liebelei. Schauspiel in drei Akten1895-10-09@\strich\emph{Liebelei. Schauspiel in drei Akten} {[}1895-10-09{]}|pwv} iſt der Höhepunkt; überhaupt iſt das
                  Stück\pwindex{Schnitzler, Arthur 15.05.1862 – 21.10.1931@\textsc{Schnitzler, Arthur} (15.05.1862 – 21.10.1931), \emph{Schriftsteller, Mediziner}!Liebelei. Schauspiel in drei Akten1895-10-09@\strich\emph{Liebelei. Schauspiel in drei Akten} {[}1895-10-09{]}|pwv} vorzüglich gebaut, es
               wächſt ſo allmälig ins große Dramatiſche hinein. Bewundert habe ich nebenbei die
               Kunſt, mit der Du all’ die techniſchen Schwierigkeiten für den dritten Akt\pwindex{Schnitzler, Arthur 15.05.1862 – 21.10.1931@\textsc{Schnitzler, Arthur} (15.05.1862 – 21.10.1931), \emph{Schriftsteller, Mediziner}!Liebelei. Schauspiel in drei Akten1895-10-09@\strich\emph{Liebelei. Schauspiel in drei Akten} {[}1895-10-09{]}|pwv} bewältigt haſt, von denen
               Du in \label{K_L02726-3v}\edtext{\textsc{Ischl\oindex{Bad Ischl@\textbf{Bad Ischl}|pw}}}{\lemma{\textnormal{\emph{Ischl}}}\Cendnote{\textnormal{Zwischen 23. 8. 1894 und 3. 9. 1894 verbrachten Schnitzler\pwindex{Schnitzler, Arthur 15.05.1862 – 21.10.1931@\textsc{Schnitzler, Arthur} (15.05.1862 – 21.10.1931), \emph{Schriftsteller, Mediziner}|pwk} und Goldmann\pwindex{Goldmann, Paul 31.01.1865 – 25.09.1935@\textsc{Goldmann, Paul} (31.01.1865 – 25.09.1935), \emph{Schriftsteller, Journalist}|pwk} einige Zeit gemeinsam in Ischl\oindex{Bad Ischl@\textbf{Bad Ischl}|pwk}. Am 30. 8. 1894 sowie am 1. 9. 1894 diskutierten sie
                     »fruchtbar« über \emph{Liebelei}\pwindex{Schnitzler, Arthur 15.05.1862 – 21.10.1931@\textsc{Schnitzler, Arthur} (15.05.1862 – 21.10.1931), \emph{Schriftsteller, Mediziner}!Liebelei. Schauspiel in drei Akten1895-10-09@\strich\emph{Liebelei. Schauspiel in drei Akten} {[}1895-10-09{]}|pwk},
                  damals noch unter dem Titel \emph{Armes Mädl}\pwindex{Schnitzler, Arthur 15.05.1862 – 21.10.1931@\textsc{Schnitzler, Arthur} (15.05.1862 – 21.10.1931), \emph{Schriftsteller, Mediziner}!Liebelei. Schauspiel in drei Akten1895-10-09@\strich\emph{Liebelei. Schauspiel in drei Akten} {[}1895-10-09{]}|pwk} in
                  Arbeit.}}}\label{K_L02726-3h} ſprachſt. Ma\substVorne{}\textsuperscript{\textcolor{gray}{m}}\substDazwischen{}n\substHinten{} kann ſich keinen zwangloſeren und natürlicheren {\pb}Vorgang denken. Beſonders daß die Sache\pwindex{Schnitzler, Arthur 15.05.1862 – 21.10.1931@\textsc{Schnitzler, Arthur} (15.05.1862 – 21.10.1931), \emph{Schriftsteller, Mediziner}!Liebelei. Schauspiel in drei Akten1895-10-09@\strich\emph{Liebelei. Schauspiel in drei Akten} {[}1895-10-09{]}|pwv} »übermorgen« ſpielt, iſt zugleich
               techniſch fein und dramatiſch wirkſam. Nun möchte ich auf eine kleine Gefahr
               aufmerkſam machen: daß man nämlich den \textsc{Theodor\pwindex{Schnitzler, Arthur 15.05.1862 – 21.10.1931@\textsc{Schnitzler, Arthur} (15.05.1862 – 21.10.1931), \emph{Schriftsteller, Mediziner}!Liebelei. Schauspiel in drei Akten1895-10-09@\strich\emph{Liebelei. Schauspiel in drei Akten} {[}1895-10-09{]}|pwv}}, wenn er nicht \strikeout{vortrefflich} ſehr geſchickt
               geſpielt wird, im Publikum zuerſt komiſch nehmen kann. Er iſt auch gar zu ſehr
                  »\label{K_L02726-4v}\edtext{\textsc{\begin{otherlanguage}{french}mufle\end{otherlanguage}}}{\lemma{\textnormal{\emph{mufle}}}\Cendnote{\textnormal{französisch: Rüpel}}}\label{K_L02726-4h}«. Insbeſondere
               möchte ich, daß er das von dem Fallen im Duell nicht gar zu trocken herausſagt. Ich
               weiß wohl, was Du damit willſt: mit {\pb}dem Mädel macht
               man eben keine Umſtände. Aber ſo ein roher Kerl iſt der \textsc{Theodor\pwindex{Schnitzler, Arthur 15.05.1862 – 21.10.1931@\textsc{Schnitzler, Arthur} (15.05.1862 – 21.10.1931), \emph{Schriftsteller, Mediziner}!Liebelei. Schauspiel in drei Akten1895-10-09@\strich\emph{Liebelei. Schauspiel in drei Akten} {[}1895-10-09{]}|pwv}} doch nicht. Er ſollte wenigſtens verlegen ſein, zu umſchreiben verſuchen:
               Unfall {\dotsfour} ſchwer verwundet {\dotsfour} und
                  \strikeout{lan} dann erſt das Duell herausbringen. Die Tragik,
               die dann mit elementarer Gewalt lospraſſelt, – die Reden des Mädels – das iſt ein Meiſterſtück\pwindex{Schnitzler, Arthur 15.05.1862 – 21.10.1931@\textsc{Schnitzler, Arthur} (15.05.1862 – 21.10.1931), \emph{Schriftsteller, Mediziner}!Liebelei. Schauspiel in drei Akten1895-10-09@\strich\emph{Liebelei. Schauspiel in drei Akten} {[}1895-10-09{]}|pwv}. Mich hats bereits
               beim Leſen in der Kehle gewürgt. Auf dem Theater kann dem kein Menſch wiederſtehen.
               Herrlich und tief ergreifend! Der Schluß\pwindex{Schnitzler, Arthur 15.05.1862 – 21.10.1931@\textsc{Schnitzler, Arthur} (15.05.1862 – 21.10.1931), \emph{Schriftsteller, Mediziner}!Liebelei. Schauspiel in drei Akten1895-10-09@\strich\emph{Liebelei. Schauspiel in drei Akten} {[}1895-10-09{]}|pwv} gefällt mir nicht. Ich möchte nicht, daß ſie ſich umbringt. Das iſt
                  {\pb}gar nicht nöthig. Laß’ dem dummen Publikum
               wenigſtens den kleinen Troſt, daß ſie leben bleibt. Es kann viel erſchütternder
               enden. Sinkt dem Vater weinend an die Bruſt und der hebt ſchluchzend ſeinen
               zitternden Arm und ſchreit zu \textsc{Theodor\pwindex{Schnitzler, Arthur 15.05.1862 – 21.10.1931@\textsc{Schnitzler, Arthur} (15.05.1862 – 21.10.1931), \emph{Schriftsteller, Mediziner}!Liebelei. Schauspiel in drei Akten1895-10-09@\strich\emph{Liebelei. Schauspiel in drei Akten} {[}1895-10-09{]}|pwv}}, dem Repräſentanten der »Welt draußen«: »Ihr habt mir mein Mädel umgebracht.«
               Oder ſo was. Aber kein Weglaufen. Man verhindert \strikeout{ſo\textcolor{gray}{d}} ſie auch, ans Grab zu gehen, damit baſta! Die Fenſter-Hinausſchreierei iſt
               verfehlt. Die Hauptperſon\pwindex{Schnitzler, Arthur 15.05.1862 – 21.10.1931@\textsc{Schnitzler, Arthur} (15.05.1862 – 21.10.1931), \emph{Schriftsteller, Mediziner}!Liebelei. Schauspiel in drei Akten1895-10-09@\strich\emph{Liebelei. Schauspiel in drei Akten} {[}1895-10-09{]}|pwv} muß
               auf der Bühne bleiben. Und dann ſo unwahrſcheinlich. {\pb}Er holt ſie ja doch ein\strikeout{;} bis zum Kirchhof, braucht
               ſich nur einen Fiaker zu nehmen, um ihr zuvorzukommen. Oder die \textsc{Mizzi\pwindex{Schnitzler, Arthur 15.05.1862 – 21.10.1931@\textsc{Schnitzler, Arthur} (15.05.1862 – 21.10.1931), \emph{Schriftsteller, Mediziner}!Liebelei. Schauspiel in drei Akten1895-10-09@\strich\emph{Liebelei. Schauspiel in drei Akten} {[}1895-10-09{]}|pwv}} ſchreit aus dem Fenſter den Paſſanten zu: »Haltets auf!« Das \uline{mußt} Du ändern. Es iſt ein Fehler, das Ende hinter die
               Couliſſen zu verlegen.\pend
           \pstart
           Im Ganzen: ein edles und reifes Werk\pwindex{Schnitzler, Arthur 15.05.1862 – 21.10.1931@\textsc{Schnitzler, Arthur} (15.05.1862 – 21.10.1931), \emph{Schriftsteller, Mediziner}!Liebelei. Schauspiel in drei Akten1895-10-09@\strich\emph{Liebelei. Schauspiel in drei Akten} {[}1895-10-09{]}|pwv}. Ich beglückwünſche Dich dazu von ganzem Herzen. Ich kenne zur Zeit
               Niemanden, der ſo etwas ſchreiben könnte, auch hier in Frankreich\oindex{Frankreich@\textbf{Frankreich}|pw} nicht. Es iſt die Krönung Deines bisherigen Lebens und Schaffens,
                  {\pb}und wird es erſt einmal aufgeführt, ſo wird die
               Welt mit Erſtaunen ſehen, daß Du ein Dichter biſt{\dots}\pend
           \pstart
           \label{K_L02726-5v}\edtext{Gräulich iſt, nochmals, der Titel\pwindex{Schnitzler, Arthur 15.05.1862 – 21.10.1931@\textsc{Schnitzler, Arthur} (15.05.1862 – 21.10.1931), \emph{Schriftsteller, Mediziner}!Liebelei. Schauspiel in drei Akten1895-10-09@\strich\emph{Liebelei. Schauspiel in drei Akten} {[}1895-10-09{]}|pwv}}{\lemma{\textnormal{\emph{Gräulich … Titel}}}\Cendnote{\textnormal{Siehe Paul Goldmann an Arthur Schnitzler, 31. 12. [1894]. }}}\label{K_L02726-5h}. Wenn Du
               einen hätteſt wählen wollen, der alle ſchlimmen Vorurtheile gegen das Stück\pwindex{Schnitzler, Arthur 15.05.1862 – 21.10.1931@\textsc{Schnitzler, Arthur} (15.05.1862 – 21.10.1931), \emph{Schriftsteller, Mediziner}!Liebelei. Schauspiel in drei Akten1895-10-09@\strich\emph{Liebelei. Schauspiel in drei Akten} {[}1895-10-09{]}|pwv} erwecken ſollte, ſo hätteſt Du keinen
               beſſern finden können. Du mußt es umtaufen. Kannſt und willſt Du es nicht »Eine Liebſchaft\pwindex{Schnitzler, Arthur 15.05.1862 – 21.10.1931@\textsc{Schnitzler, Arthur} (15.05.1862 – 21.10.1931), \emph{Schriftsteller, Mediziner}!Liebelei. Schauspiel in drei Akten1895-10-09@\strich\emph{Liebelei. Schauspiel in drei Akten} {[}1895-10-09{]}|pwv}« nennen – das
               wäre das weitaus Beſte – ſo {\pb}möchte ich Dir
               vorſchlagen: »Arme Liebe\pwindex{Schnitzler, Arthur 15.05.1862 – 21.10.1931@\textsc{Schnitzler, Arthur} (15.05.1862 – 21.10.1931), \emph{Schriftsteller, Mediziner}!Liebelei. Schauspiel in drei Akten1895-10-09@\strich\emph{Liebelei. Schauspiel in drei Akten} {[}1895-10-09{]}|pwv}«.
               Leicht \strikeout{kan} kannſt Du der Chriſtine\pwindex{Schnitzler, Arthur 15.05.1862 – 21.10.1931@\textsc{Schnitzler, Arthur} (15.05.1862 – 21.10.1931), \emph{Schriftsteller, Mediziner}!Liebelei. Schauspiel in drei Akten1895-10-09@\strich\emph{Liebelei. Schauspiel in drei Akten} {[}1895-10-09{]}|pwv} im dritten Akt\pwindex{Schnitzler, Arthur 15.05.1862 – 21.10.1931@\textsc{Schnitzler, Arthur} (15.05.1862 – 21.10.1931), \emph{Schriftsteller, Mediziner}!Liebelei. Schauspiel in drei Akten1895-10-09@\strich\emph{Liebelei. Schauspiel in drei Akten} {[}1895-10-09{]}|pwv} noch zehn Worte in den Mund legen, die
               dieſen Titel erklären\substVorne{}\textsuperscript{:}\substDazwischen{};\substHinten{} oder noch beſſer der Vater ſoll es zum Schluß\pwindex{Schnitzler, Arthur 15.05.1862 – 21.10.1931@\textsc{Schnitzler, Arthur} (15.05.1862 – 21.10.1931), \emph{Schriftsteller, Mediziner}!Liebelei. Schauspiel in drei Akten1895-10-09@\strich\emph{Liebelei. Schauspiel in drei Akten} {[}1895-10-09{]}|pwv} ſagen: »Wein’ Dich aus, \strikeout{armes} Kind. Wenn arme Leute lieben, ſo dürfen ſie nichts beanſpruchen, als
               Thränen.« \strikeout{D} In der Größe ſeines Schmerzes wird der
               Alte aphoriſtiſch \substVorne{}\textsuperscript{.}\substDazwischen{}–\substHinten{} ein einziges Mal. Das wäre umſo wirkſamer. Und denk’ Dir nur, was \strikeout{ſich} für eine {\pb}große
               allgemeine Perſpektive ſich am Schluß\pwindex{Schnitzler, Arthur 15.05.1862 – 21.10.1931@\textsc{Schnitzler, Arthur} (15.05.1862 – 21.10.1931), \emph{Schriftsteller, Mediziner}!Liebelei. Schauspiel in drei Akten1895-10-09@\strich\emph{Liebelei. Schauspiel in drei Akten} {[}1895-10-09{]}|pwv} durch dieſe Worte noch öffnen würde. Das wäre doch beſſer, als die
                  Fenſter-Geſchichten {\dotsfive}\pend
           \pstart
           Vielen, vielen Dank, mein lieber Freund, für den großen Genuß, den Du mir verſchafft
               haſt. Wie ſtehts nun mit der \label{K_L02726-6v}\edtext{Aufführung\pwindex{Schnitzler, Arthur 15.05.1862 – 21.10.1931@\textsc{Schnitzler, Arthur} (15.05.1862 – 21.10.1931), \emph{Schriftsteller, Mediziner}!Liebelei. Schauspiel in drei Akten1895-10-09@\strich\emph{Liebelei. Schauspiel in drei Akten} {[}1895-10-09{]}|pwv}}{\lemma{\textnormal{\emph{Aufführung}}}\Cendnote{\textnormal{\emph{Liebelei}\pwindex{Schnitzler, Arthur 15.05.1862 – 21.10.1931@\textsc{Schnitzler, Arthur} (15.05.1862 – 21.10.1931), \emph{Schriftsteller, Mediziner}!Liebelei. Schauspiel in drei Akten1895-10-09@\strich\emph{Liebelei. Schauspiel in drei Akten} {[}1895-10-09{]}|pwk} wurde am 9. 10. 1895 am Wien\oindex{Wien@\textbf{Wien}|pwk}er Burgtheater\oindex{Burgtheater@\textbf{Burgtheater}|pwk} uraufgeführt.}}}\label{K_L02726-6h}? Schreib’ mir
               bald und ausführlich.\pend
           \pstart
           Zwei Bitten: Erſtens. Ich habe zum Neujahr ein ſchönes
                  \label{K_L02726-7v}\edtext{Alt-Wien\oindex{Wien@\textbf{Wien}|pw}er Bild}{\lemma{\textnormal{\emph{Alt-Wiener Bild}}}\Cendnote{\textnormal{Nicht ermittelt. Mit
                        ›Alt-Wien\oindex{Wien@\textbf{Wien}|pw}‹ wird ein Motiv
                  oder eine Darstellung aus der Zeit vor der Schleifung der Basteien und dem Ringstraße\oindex{Ringstrasse@\textbf{Ringstraße}|pwk}nbau bezeichnet.}}}\label{K_L02726-7h} erhalten, von
                  \textsc{Artaria\orgindex{Artaria und Co.@Artaria {\kaufmannsund}  Co.|pw}}\substVorne{}\textsuperscript{.}\substDazwischen{},\substHinten{} mit dem ich mich unbändig gefreut habe. Aber ohne {\pb}Begleitbrief. Ein ſo zartſinniges, von Herzen zu
               Herzen gehendes Geſchenk kann nur von \label{K_L02726-8v}\edtext{Jemandem\pwindex{Schnitzler, Julius 13.07.1865 – 29.06.1939@\textsc{Schnitzler, Julius} (13.07.1865 – 29.06.1939), \emph{Mediziner}|pwv}\pwindex{Schnitzler, Helene 16.07.1871 – September 1941@\textsc{Schnitzler, Helene} (16.07.1871 – September 1941)|pwv} aus
               Deinem Kreiſe}{\lemma{\textnormal{\emph{Jemandem … Kreiſe}}}\Cendnote{\textnormal{Es kam von Schnitzlers\pwindex{Schnitzler, Arthur 15.05.1862 – 21.10.1931@\textsc{Schnitzler, Arthur} (15.05.1862 – 21.10.1931), \emph{Schriftsteller, Mediziner}|pwk} Bruder Julius\pwindex{Schnitzler, Julius 13.07.1865 – 29.06.1939@\textsc{Schnitzler, Julius} (13.07.1865 – 29.06.1939), \emph{Mediziner}|pwk} und dessen Frau Helene\pwindex{Schnitzler, Helene 16.07.1871 – September 1941@\textsc{Schnitzler, Helene} (16.07.1871 – September 1941)|pwk}, vgl. Paul Goldmann an Arthur Schnitzler, 2. 3. [1895].
               }}}\label{K_L02726-8h} herkommen. Sag’ mir, wer der Spender\pwindex{Schnitzler, Julius 13.07.1865 – 29.06.1939@\textsc{Schnitzler, Julius} (13.07.1865 – 29.06.1939), \emph{Mediziner}|pwv}\pwindex{Schnitzler, Helene 16.07.1871 – September 1941@\textsc{Schnitzler, Helene} (16.07.1871 – September 1941)|pwv} iſt.\pend
           \pstart
           Zweitens. Schreib’ mir \label{K_L02726-9v}\edtext{Torresanis\pwindex{Torresani-Lanzenfeld, Carl von 19.04.1846 – 16.04.1907@\textsc{Torresani-Lanzenfeld, Carl von} (19.04.1846 – 16.04.1907), \emph{Schriftsteller}|pw} Adreſſe}{\lemma{\textnormal{\emph{Torresanis Adreſſe}}}\Cendnote{\textnormal{Torresani\pwindex{Torresani-Lanzenfeld, Carl von 19.04.1846 – 16.04.1907@\textsc{Torresani-Lanzenfeld, Carl von} (19.04.1846 – 16.04.1907), \emph{Schriftsteller}|pwk} scheint im Adressbuch \emph{Lehmann}\pwindex{?? Werk@Nicht ermittelte Verfasserinnen und Verfasser!Lehmann s Allgemeiner Wohnungs-Anzeiger1859 – 1942@\emph{Lehmann’s Allgemeiner Wohnungs-Anzeiger} {[}1859 – 1942{]}|pwk} für das Jahr 1891 zum
                  letzten Mal als wohnhaft in Wien\oindex{Wien@\textbf{Wien}|pwk} auf. Danach
                  reiste er jahrelang.}}}\label{K_L02726-9h}.\pend
           \pstart
           Viele treue Grüße! {\\[\baselineskip]}Dein {\\[\baselineskip]}\spacefill\mbox{Paul Goldmann.}\pend
           \leftskip=0em{}
         
         \endnumbering\mylabel{h}\end{ledgroupsized}  \newcommand{\dateiname}{L02726}\newcommand{\titel}{Paul Goldmann an Arthur Schnitzler, 5. 1. [1895]}\newcommand{\editorInnen}{Martin Anton Müller und Laura Untner}%% latex-leseansicht-abspann.tex
%% Abspann für die Leseansicht.
%% Der Schalter \ifkorrekturansicht ist bereits durch den Vorspann gesetzt.

%% latex-abspann.tex
%% Gemeinsamer Abspann für Korrekturansicht und Leseansicht.
%% Setzt den Schalter \ifkorrekturansicht voraus (gesetzt in den
%% einbindenden Dateien latex-korrekturansicht-abspann.tex bzw.
%% latex-leseansicht-abspann.tex).
%% ---------------------------------------------------------------

\normalsize

% Das esempio-Environment wird nur in der Leseansicht benötigt
\ifkorrekturansicht\else
\newenvironment{esempio}[3]%
{
    \vspace{1.5ex}
    \rlap{\underline{#1}}
    \par
    \setlength{\parindent}{0cm}
    \nopagebreak
    \leftskip=#2cm
    \rightskip=#3cm
}
{
    \par
}
\fi

\doendnotes{C}
\bigskip
\vfill

\clearpage

\footnotesize

\ifkorrekturansicht
  \lohead{\textsc{register}}
\fi

% theindex-Environment neu definieren ohne reledmac
\makeatletter
\renewenvironment{theindex}{%
  \ifkorrekturansicht
    \section*{\indexname}%
  \else
    \subsubsection*{Index der erwähnten Entitäten}%
  \fi
  \setlength{\parindent}{0pt}%
  \setlength{\parskip}{0pt plus 0.3pt}%
  \let\item\@idxitem
}{%
  \ifkorrekturansicht\clearpage\fi
}
\makeatother

\IfFileExists{\jobname-pw.ind}{\input{\jobname-pw.ind}}{}

% Quellenangabe nur in der Leseansicht
\ifkorrekturansicht\else
% Fallback-Definitionen, falls die .tex-Datei \titel etc. nicht gesetzt hat
\providecommand{\titel}{}
\providecommand{\editorInnen}{}
\providecommand{\dateiname}{\jobname}

\vspace{3cm}

\vfill

\footnotesize
\textsc{Quelle}: \titel. Herausgegeben von {\editorInnen}. In: \emph{Arthur Schnitzler: Briefwechsel mit Autorinnen und Autoren}.
 Digitale Edition, https://schnitzler-briefe.acdh.oeaw.ac.at/{\dateiname}.html (Stand \today)
\fi

\end{document}


      