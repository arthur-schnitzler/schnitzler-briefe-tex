%% latex-leseansicht-vorspann.tex
%% Vorspann für die Leseansicht.
%% Lädt die gemeinsame Datei latex-vorspann.tex mit nicht gesetztem Schalter.

\newif\ifkorrekturansicht
\korrekturansichtfalse

\input{../tex-inputs/latex-vorspann}


         
         \renewcommand{\erwaehntePersonen}{Personen: Honoré de Balzac, Richard Beer-Hofmann, Arthur Klein, Richard Klein, Peter Nansen, Marie Reinhard, Carl Reinhard, Leopold Sonnemann}
         \renewcommand{\erwaehnteInstitutionen}{Institutionen: Christlichsoziale Partei, Frankfurter Zeitung, Reichsrat}
         \renewcommand{\erwaehnteOrte}{Orte: 16. Arrondissement (Passy), Europa, Frankreich, Griechenland, Im Reichsrat vertretene Königreiche und Länder, Kreta, München, Nizza, Paris, Schweiz, Türkei, Wien, Zürich, rue Feydeau, Österreich-Ungarn}
         \renewcommand{\erwaehnteWerke}{Werke: Arthur Schnitzler. »Elskovsleg«s Forfatter, Die deutsche Orientpolitik und das Ausland, Frankfurter Zeitung, Frankreich. [Zum Tod des Persers Djemal-ed-din], Illusions perdues, Politiken, Tagebuch}
               \section[ Paul Goldmann an Arthur Schnitzler, 11. 3. {[}1897{]}]{ Paul Goldmann an Arthur Schnitzler, 11. 3. {[}1897{]}}\nopagebreak\mylabel{v}\rehead{ }\begin{ledgroupsized}[t]{13cm}\normalsize\beginnumbering \toendnotes[C]{\smallbreak\pagebreak[2]} \Standort{DLA, A:Schnitzler, HS.NZ85.1.3167.}
\physDesc{Brief, 2 Blätter, 7 Seiten
\newline{}Handschrift: blaue Tinte, deutsche Kurrent
\newline{}Schnitzler: mit Bleistift das Jahr »97« vermerkt }\toendnotes[C]{\smallbreak}\pstart
           \noindent{}{\pb}\textcolor{gray}{\textbf{\textbf{Frankfurter Zeitung\orgindex{Frankfurter Zeitung@Frankfurter Zeitung|pw}}}}\pend
           \pstart
           \textcolor{gray}{\textbf{(\begin{otherlanguage}{french}Gazette de Francfort\end{otherlanguage}\orgindex{Frankfurter Zeitung@Frankfurter Zeitung|pw}).}}\pend
           \pstart
           \textcolor{gray}{\textbf{\textbf{\begin{otherlanguage}{french}Fondateur M.\end{otherlanguage}{ }L. Sonnemann\pwindex{Sonnemann, Leopold 1831-10-29 – 1909-10-30@\textsc{Sonnemann, Leopold} (1831-10-29 – 1909-10-30), \emph{Journalist, Herausgeber}|pw}.}}}\hfill \textsc{Paris\oindex{Paris@\textbf{Paris}|pw}}, 1\substVorne{}\textsuperscript{0}\substDazwischen{}1\substHinten{}. März.\pend
           \pstart
           \begin{otherlanguage}{french}\textcolor{gray}{\textbf{Journal politique, financier,}}\end{otherlanguage}\pend
           \pstart
           \begin{otherlanguage}{french}\textcolor{gray}{\textbf{commercial et littéraire.}}\end{otherlanguage}\pend
           \pstart
           \begin{otherlanguage}{french}\textcolor{gray}{\textbf{\textbf{Paraissant trois fois par jour.}}}\end{otherlanguage}\pend
           \pstart
           \begin{otherlanguage}{french}\textcolor{gray}{\textbf{\textbf{Bureau à Paris\oindex{Paris@\textbf{Paris}|pw}}}}\end{otherlanguage}\pend
           \pstart
           \begin{otherlanguage}{french}\textcolor{gray}{\textbf{\textbf{24. Rue Feydeau\oindex{rue Feydeau@\textbf{rue Feydeau}|pw}.}}}\end{otherlanguage}\pend
           \pstart{}Mein lieber Freund,\pend\pstart
           Ich habe mit der verfluchten \label{K_L02805-1v}\edtext{Orient-Geſchichte}{\lemma{\textnormal{\emph{Orient-Geſchichte}}}\Cendnote{\textnormal{Höchstwahrscheinlich Bezug auf den sich zunehmend zum (Türk\oindex{Tuerkei@\textbf{Türkei}|pwkv}isch-Griech\oindex{Griechenland@\textbf{Griechenland}|pwkv}ischen) Krieg aufschaukelnden
                  Konflikt auf Kreta\oindex{Kreta@\textbf{Kreta}|pwk}, über den Goldmann\pwindex{Goldmann, Paul 31.01.1865 – 25.09.1935@\textsc{Goldmann, Paul} (31.01.1865 – 25.09.1935), \emph{Schriftsteller, Journalist}|pwk} intensiv berichtete (vgl. Paul Goldmann an Arthur Schnitzler, 16. 2. [1897]). Daneben könnte sich
                     Goldmann\pwindex{Goldmann, Paul 31.01.1865 – 25.09.1935@\textsc{Goldmann, Paul} (31.01.1865 – 25.09.1935), \emph{Schriftsteller, Journalist}|pwk} auch auf folgende Berichte\pwindex{Goldmann, Paul 31.01.1865 – 25.09.1935@\textsc{Goldmann, Paul} (31.01.1865 – 25.09.1935), \emph{Schriftsteller, Journalist}!deutsche Orientpolitik und das Ausland1897-03-05@\strich\emph{Die deutsche Orientpolitik und das Ausland} {[}1897-03-05{]}|pwkv}\pwindex{Goldmann, Paul 31.01.1865 – 25.09.1935@\textsc{Goldmann, Paul} (31.01.1865 – 25.09.1935), \emph{Schriftsteller, Journalist}!Frankreich. [Zum Tod des Persers Djemal-ed-din]1897-03-13@\strich\emph{Frankreich. [Zum Tod des Persers Djemal-ed-din]} {[}1897-03-13{]}|pwkv} bezogen
                  haben: G\pwindex{Goldmann, Paul 31.01.1865 – 25.09.1935@\textsc{Goldmann, Paul} (31.01.1865 – 25.09.1935), \emph{Schriftsteller, Journalist}|pwkv} [=Paul Goldmann\pwindex{Goldmann, Paul 31.01.1865 – 25.09.1935@\textsc{Goldmann, Paul} (31.01.1865 – 25.09.1935), \emph{Schriftsteller, Journalist}|pwk}]: \emph{Die
                        deutsche Orientpolitik und das Ausland}\pwindex{Goldmann, Paul 31.01.1865 – 25.09.1935@\textsc{Goldmann, Paul} (31.01.1865 – 25.09.1935), \emph{Schriftsteller, Journalist}!deutsche Orientpolitik und das Ausland1897-03-05@\strich\emph{Die deutsche Orientpolitik und das Ausland} {[}1897-03-05{]}|pwk}. In: \emph{Frankfurter Zeitung}\pwindex{?? Werk@Nicht ermittelte Verfasserinnen und Verfasser!Frankfurter Zeitung1856 – 1943@\emph{Frankfurter Zeitung} {[}1856 – 1943{]}|pwk}, Jg. 41, Nr. 64, 5. 3. 1897, S. 1; G\pwindex{Goldmann, Paul 31.01.1865 – 25.09.1935@\textsc{Goldmann, Paul} (31.01.1865 – 25.09.1935), \emph{Schriftsteller, Journalist}|pwkv} [=Paul Goldmann\pwindex{Goldmann, Paul 31.01.1865 – 25.09.1935@\textsc{Goldmann, Paul} (31.01.1865 – 25.09.1935), \emph{Schriftsteller, Journalist}|pwk}]: \emph{Frankreich. [Zum Tod des Persers Djemal-ed-din]}\pwindex{Goldmann, Paul 31.01.1865 – 25.09.1935@\textsc{Goldmann, Paul} (31.01.1865 – 25.09.1935), \emph{Schriftsteller, Journalist}!Frankreich. [Zum Tod des Persers Djemal-ed-din]1897-03-13@\strich\emph{Frankreich. [Zum Tod des Persers Djemal-ed-din]} {[}1897-03-13{]}|pwk}. In: \emph{Frankfurter Zeitung}\pwindex{?? Werk@Nicht ermittelte Verfasserinnen und Verfasser!Frankfurter Zeitung1856 – 1943@\emph{Frankfurter Zeitung} {[}1856 – 1943{]}|pwk}, Jg. 41, Nr. 72, 13. 3. 1897, Erstes Morgenblatt,
                  S. 1.}}}\label{K_L02805-1h} unbändig zu thun. Auch \strikeout{\textcolor{gray}{er}} thut mir mein Auge \strikeout{\textcolor{gray}{f}} unerträglich weh. So kommt es, daß ich Deinen lieben Brief erſt heut beantworte.\pend
           \pstart
           Ich danke Dir von ganzem Herzen für den Beiſtand, den Du mir in der Angelegenheit mit
                  \textsc{Klein\pwindex{Klein, Arthur 27.11.1868 – 28.07.1943@\textsc{Klein, Arthur} (27.11.1868 – 28.07.1943)|pw}s}{ }Bruder\pwindex{Klein, Richard *~07.08.1873@\textsc{Klein, Richard} (*~07.08.1873), \emph{Maler}|pwv} geliehen. Ich bin
               ſelbſt wohl auch nicht ohne Schuld an dieſen Unannehmlichkeiten. Ich laſſe mir Leute
               dieſer Art zu nahe kommen, in einer gewiſſen ſchlamperten Liebenswürdigkeit. Auch
               habe ich mich von meiner Heftigkeit zu ſehr hinreißen laſſen. \textsc{Arthur Klein\pwindex{Klein, Arthur 27.11.1868 – 28.07.1943@\textsc{Klein, Arthur} (27.11.1868 – 28.07.1943)|pw}} hat ſich prachtvoll benommen. {\pb}Wenn Du ihn\pwindex{Klein, Arthur 27.11.1868 – 28.07.1943@\textsc{Klein, Arthur} (27.11.1868 – 28.07.1943)|pwv} ſiehſt, ſo danke ihm noch
               beſonders, bitte\substVorne{}\textsuperscript{,}\substDazwischen{}!\substHinten{} Freilich hat es weiterhin noch einige Klatſchereien gegeben, und die
               Unannehmlichkeiten ſind noch nicht zu Ende. \strikeout{Abe\textcolor{gray}{r}} Aber ich mache mir heut große Vorwürfe, Dich
               mit der ganzen Sache behelligt zu haben{\dotsfive}\pend
           \pstart
           Soeben erhalte ich für \strikeout{Euch} Dich und \textsc{Richard\pwindex{Beer-Hofmann, Richard 1866-07-11 – 1945-09-26@\textsc{Beer-Hofmann, Richard} (1866-07-11 – 1945-09-26), \emph{Schriftsteller}|pw}} zwei Nummern von »\textsc{Politiken\pwindex{?? Werk@Nicht ermittelte Verfasserinnen und Verfasser!Politiken1. 1. 1884@\emph{Politiken} {[}1. 1. 1884{]}|pw}}«, wo \textsc{Peter Nansen\pwindex{Nansen, Peter 20.01.1861 – 31.07.1918@\textsc{Nansen, Peter} (20.01.1861 – 31.07.1918), \emph{Schriftsteller, Journalist, Verleger}|pw}} über Dich und zugleich über uns \label{K_L02805-3v}\edtext{geſchrieben\pwindex{Arthur Schnitzler. »Elskovsleg«s Forfatter1897-03-09@\emph{Arthur Schnitzler. »Elskovsleg«s Forfatter} {[}1897-03-09{]}|pwv}}{\lemma{\textnormal{\emph{geſchrieben}}}\Cendnote{\textnormal{–n–\pwindex{Nansen, Peter 20.01.1861 – 31.07.1918@\textsc{Nansen, Peter} (20.01.1861 – 31.07.1918), \emph{Schriftsteller, Journalist, Verleger}|pwkv} [=Peter Nansen\pwindex{Nansen, Peter 20.01.1861 – 31.07.1918@\textsc{Nansen, Peter} (20.01.1861 – 31.07.1918), \emph{Schriftsteller, Journalist, Verleger}|pwk}]: \emph{Arthur
                        Schnitzler. »Elskovsleg«s Forfatter}\pwindex{Arthur Schnitzler. »Elskovsleg«s Forfatter1897-03-09@\emph{Arthur Schnitzler. »Elskovsleg«s Forfatter} {[}1897-03-09{]}|pwk}. In: \emph{Politiken}\pwindex{?? Werk@Nicht ermittelte Verfasserinnen und Verfasser!Politiken1. 1. 1884@\emph{Politiken} {[}1. 1. 1884{]}|pwk}, Nr. 68, 9. 3. 1897, S. 1.}}}\label{K_L02805-3h} hat. Ich verſtehe kein Wort davon,
               aber es ſcheint prächtig zu ſein. \strikeout{Du} Ich ſende beide
                  Nummern\pwindex{?? Werk@Nicht ermittelte Verfasserinnen und Verfasser!Politiken1. 1. 1884@\emph{Politiken} {[}1. 1. 1884{]}|pwv} an Dich.\pend
           \pstart
           Meine \label{K_L02805-5v}\edtext{Reiſe nach \textsc{Nizza\oindex{Nizza@\textbf{Nizza}|pw}}}{\lemma{\textnormal{\emph{Reiſe nach Nizza}}}\Cendnote{\textnormal{siehe Paul Goldmann an Arthur Schnitzler, 16. 2. [1897]}}}\label{K_L02805-5h} iſt infolge der Orient-Ereigniſſe auf nächſte Woche verſchoben.\pend
           \pstart
           {\pb}Ich kann Dir gar nicht ſagen, wie ich mich auf Dein
               Kommen freue! Ein vorheriges Zuſammentreffen in der \label{K_L02805-9v}\edtext{Schweiz\oindex{Schweiz@\textbf{Schweiz}|pw}}{\lemma{\textnormal{\emph{Schweiz}}}\Cendnote{\textnormal{Schnitzler\pwindex{Schnitzler, Arthur 15.05.1862 – 21.10.1931@\textsc{Schnitzler, Arthur} (15.05.1862 – 21.10.1931), \emph{Schriftsteller, Mediziner}|pwk} war von 10. 4. 1897 bis 11. 4. 1897 in Zürich\oindex{Zuerich@\textbf{Zürich}|pwk}. Er kam gerade aus München\oindex{Muenchen@\textbf{München}|pwk} und reiste nach Paris\oindex{Paris@\textbf{Paris}|pwk} weiter.}}}\label{K_L02805-9h} iſt leider unmöglich. Ich darf mich nicht vom Flecke
               rühren; hoffentlich habe ich nur hier während Deiner Anweſenheit wenig zu thun, damit
               ich Dich ordentlich genießen kann. Die Wohnungsfrage wird freilich nicht leicht zu
               erledigen ſein. \strikeout{D} Ich habe nochmals energiſcheſte
               Nachforſchungen angeſtellt. Das Reſultat iſt das, was ich gewußt hatte: Anſtändige
                  fran\oindex{Frankreich@\textbf{Frankreich}|pwv}zöſiſche Familien geben
               keine \textsc{Pension}, und diejenigen Familien, welche \textsc{Pension} geben, ſind nicht anſtändig. Ausnahmen gibt {\pb}es wohl, aber eine ſolche zu finden, iſt reine
               Zufallsſache. Im Übrigen denke auch ich, daß Du irgendwo zwiſchen Stadt und Land
               wohnen ſollſt, am Beſten in \textsc{Passy\oindex{16. Arrondissement (Passy)@\textbf{16. Arrondissement (Passy)}|pw}}, das beſonders anmuthig und zugleich bequem iſt. Was ich Dir ſage, ſind keine
               definitiven Reſultate. Ich habe einige fran\oindex{Frankreich@\textbf{Frankreich}|pwv}zöſiſche Bekannte mit Umfragen beauftragt, und die
               Nachforſchungen dauern fort. Ein \textsc{Hotel}, wie Du es
               wünſcheſt, wird raſch gefunden ſein, ſobald Du mir das Datum \strikeout{meiner} Deiner Ankunft mittheilſt. Allzuviel \textsc{Comfort} wirſt Du freilich nicht finden. Das Pariſ\oindex{Paris@\textbf{Paris}|pw}er {\pb}Hotelweſen iſt ſehr
               zurück. Das hat ſchon \label{K_L02805-888v}\edtext{\textsc{Balzac\pwindex{Balzac, Honore de 20.05.1799 – 18.08.1850@\textsc{Balzac, Honoré de} (20.05.1799 – 18.08.1850), \emph{Schriftsteller}|pw}} conſtatirt}{\lemma{\textnormal{\emph{Balzac conſtatirt}}}\Cendnote{\textnormal{Balzac\pwindex{Balzac, Honore de 20.05.1799 – 18.08.1850@\textsc{Balzac, Honoré de} (20.05.1799 – 18.08.1850), \emph{Schriftsteller}|pwk} thematisierte die
                  Beherbergungsindustrie in Paris\oindex{Paris@\textbf{Paris}|pwk} in mehreren
                  seiner Bücher. Er beschrieb die Hotels als überfüllt, schmutzig und überteuert,
                  mit schlechtem Service und wenig Privatsphäre. Kritisiert wurden von ihm auch die
                  Eigentümerinnen und Eigentümer dieser Hotels, die die Bedürfnisse der Reisenden
                  ausnutzten und überhöhte Preise für minderwertige Unterkünfte verlangten:
                     »il n’existe pas encore un seul hôtel où tout voyageur riche puisse
                     retrouver son chez soi« (»es gibt bislang kein einziges Hotel, in dem
                  selbst ein reicher Reisende sich zuhause fühlen kann«; \emph{Illusions Perdues}\pwindex{Balzac, Honore de 20.05.1799 – 18.08.1850@\textsc{Balzac, Honoré de} (20.05.1799 – 18.08.1850), \emph{Schriftsteller}!Illusions perdues1837@\strich\emph{Illusions perdues} {[}1837{]}|pwk}, 2. Teil)}}}\label{K_L02805-888h},
               und ſeit \textsc{Balzac\pwindex{Balzac, Honore de 20.05.1799 – 18.08.1850@\textsc{Balzac, Honoré de} (20.05.1799 – 18.08.1850), \emph{Schriftsteller}|pw}} hat ſich wenig geändert{\dotsseven}\pend
           \pstart
           Was Du mir über Deine Freundin\pwindex{Reinhard, Marie 1871-03-13 – 1899-03-18@\textsc{Reinhard, Marie} (1871-03-13 – 1899-03-18), \emph{Gesangspädagogin}|pwv} ſchreibſt, iſt ſehr ſchön. Ich habe nie daran gezweifelt, daß ſie
               »auf unſerem \textsc{Niveau}« iſt, ſchon weil ſie Deine Freundin\pwindex{Reinhard, Marie 1871-03-13 – 1899-03-18@\textsc{Reinhard, Marie} (1871-03-13 – 1899-03-18), \emph{Gesangspädagogin}|pwv} iſt. Du kannſt Dir
               denken, wie ich mich darauf freue, ſie kennen zu lernen. Darf ich Dich einſtweilen
               bitten, mich ihr zu empfehlen? {\dotsfour}\pend
           \pstart
           Nach der ſo gut verlaufenen \label{K_L02805-11v}\edtext{Unterredung mit dem {\pb}Vater\pwindex{Reinhard, Carl 02.03.1834 – 28.04.1905@\textsc{Reinhard, Carl} (02.03.1834 – 28.04.1905), \emph{Geschäftsführer}|pwv}}{\lemma{\textnormal{\emph{Unterredung … Vater}}}\Cendnote{\textnormal{siehe A. S.: \emph{Tagebuch}, 23. 2. 1897 und Paul Goldmann an Arthur Schnitzler, 24. 2. [1897]}}}\label{K_L02805-11h} ſind wohl die ſchlimmſten Unannehmlichkeiten vorüber. Ich halte es für ein
               großes Glück, daß ein äußerer Zwang Dich auf einige Zeit von Wien\oindex{Wien@\textbf{Wien}|pw} wegtreibt. Ich verſpreche mir viel von der Wirkung, die \textsc{Paris\oindex{Paris@\textbf{Paris}|pw}} auf Dich haben
               wird. Es wird Dich elektriſiren, und Dich mit Schaffenskraft und Schaffensluſt
               erfüllen. Auch wirſt Du den Pariſ\oindex{Paris@\textbf{Paris}|pw}er Frühling
               ſehen, welcher eine der Gnaden Gottes iſt.\pend
           \pstart
           Freilich könnte es ſich auch ereignen, daß Dir hier Alles ſehr zuwider iſt.\pend
           \pstart
           {\pb}Wir wollen den Himmel bitten, daß es gut
               ausgeht.\pend
           \pstart
           Bald höre ich wohl Näheres?\pend
           \pstart
           Ich begrüße Dich von Herzen!\pend
           \pstart
           Dein {\\[\baselineskip]}\spacefill\mbox{Paul Goldmn}\pend
           \leftskip=0em{}\pstart
           \noindent{}Schön habt Ihr wieder in \textsc{Wien\oindex{Wien@\textbf{Wien}|pw}}{ }\label{K_L02805-34v}\edtext{gewählt}{\lemma{\textnormal{\emph{gewählt}}}\Cendnote{\textnormal{Am 4. 3. 1897 begannen in
                        Cisleithanien\oindex{Im Reichsrat vertretene Koenigreiche und Laender@\textbf{Im Reichsrat vertretene Königreiche und Länder}|pwkv}, dem
                     nördlichen und westlichen Teils Österreich-Ungarn\oindex{Oesterreich-Ungarn@\textbf{Österreich-Ungarn}|pwk}s, die \emph{Reichsrat}\orgindex{Reichsrat@Reichsrat|pwk}s-, also Parlament\orgindex{Reichsrat@Reichsrat|pwkv}swahlen. In Wien\oindex{Wien@\textbf{Wien}|pwk} feierte
                     insbesondere die \emph{Christlichsoziale Partei}\orgindex{Christlichsoziale Partei@Christlichsoziale Partei|pwk}
                     Erfolge. Schnitzler\pwindex{Schnitzler, Arthur 15.05.1862 – 21.10.1931@\textsc{Schnitzler, Arthur} (15.05.1862 – 21.10.1931), \emph{Schriftsteller, Mediziner}|pwk} notierte dazu am 12. 3. 1897 im \emph{Tagebuch}\pwindex{Schnitzler, Arthur 15.05.1862 – 21.10.1931@\textsc{Schnitzler, Arthur} (15.05.1862 – 21.10.1931), \emph{Schriftsteller, Mediziner}!Tagebuch1981 – 2000@\strich\emph{Tagebuch} {[}1981 – 2000{]}|pwk}: »Sehr verstimmt, auch
                        durch den Antisem.–«}}}\label{K_L02805-34h}. Ihr ſeid eine rechte Bagage. Schämt Ihr
                  Euch gar nicht vor Europa\oindex{Europa@\textbf{Europa}|pw}?\pend
           
         
         \endnumbering\mylabel{h}\end{ledgroupsized}  \newcommand{\dateiname}{L02805}\newcommand{\titel}{Paul Goldmann an Arthur Schnitzler, 11. 3. [1897]}\newcommand{\editorInnen}{Martin Anton Müller und Laura Untner}%% latex-leseansicht-abspann.tex
%% Abspann für die Leseansicht.
%% Der Schalter \ifkorrekturansicht ist bereits durch den Vorspann gesetzt.

%% latex-abspann.tex
%% Gemeinsamer Abspann für Korrekturansicht und Leseansicht.
%% Setzt den Schalter \ifkorrekturansicht voraus (gesetzt in den
%% einbindenden Dateien latex-korrekturansicht-abspann.tex bzw.
%% latex-leseansicht-abspann.tex).
%% ---------------------------------------------------------------

\normalsize

% Das esempio-Environment wird nur in der Leseansicht benötigt
\ifkorrekturansicht\else
\newenvironment{esempio}[3]%
{
    \vspace{1.5ex}
    \rlap{\underline{#1}}
    \par
    \setlength{\parindent}{0cm}
    \nopagebreak
    \leftskip=#2cm
    \rightskip=#3cm
}
{
    \par
}
\fi

\doendnotes{C}
\bigskip
\vfill

\clearpage

\footnotesize

\ifkorrekturansicht
  \lohead{\textsc{register}}
\fi

% theindex-Environment neu definieren ohne reledmac
\makeatletter
\renewenvironment{theindex}{%
  \ifkorrekturansicht
    \section*{\indexname}%
  \else
    \subsubsection*{Index der erwähnten Entitäten}%
  \fi
  \setlength{\parindent}{0pt}%
  \setlength{\parskip}{0pt plus 0.3pt}%
  \let\item\@idxitem
}{%
  \ifkorrekturansicht\clearpage\fi
}
\makeatother

\IfFileExists{\jobname-pw.ind}{\input{\jobname-pw.ind}}{}

% Quellenangabe nur in der Leseansicht
\ifkorrekturansicht\else
% Fallback-Definitionen, falls die .tex-Datei \titel etc. nicht gesetzt hat
\providecommand{\titel}{}
\providecommand{\editorInnen}{}
\providecommand{\dateiname}{\jobname}

\vspace{3cm}

\vfill

\footnotesize
\textsc{Quelle}: \titel. Herausgegeben von {\editorInnen}. In: \emph{Arthur Schnitzler: Briefwechsel mit Autorinnen und Autoren}.
 Digitale Edition, https://schnitzler-briefe.acdh.oeaw.ac.at/{\dateiname}.html (Stand \today)
\fi

\end{document}


      