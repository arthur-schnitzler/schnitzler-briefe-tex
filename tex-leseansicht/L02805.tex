%% latex-leseansicht-vorspann.tex
%% Vorspann für die Leseansicht.
%% Lädt die gemeinsame Datei latex-vorspann.tex mit nicht gesetztem Schalter.

\newif\ifkorrekturansicht
\korrekturansichtfalse

\input{../tex-inputs/latex-vorspann}


\section[ Paul Goldmann an Arthur Schnitzler, 11. 3. [1897]]{L02805 Paul Goldmann an Arthur Schnitzler,  11. 3. [1897]}
\nopagebreak\mylabel{L02805v}
\rehead{ }\normalsize\beginnumbering\briefempfaengerindex{Schnitzler, Arthur@\textsc{Schnitzler, Arthur}!zzzGoldmann, Paul@\emph{von Paul Goldmann}!1897-03-111@{11. 3. [1897]}|(be}
\toendnotes[C]{\smallbreak\pagebreak[2]}
\correspDesc{Versand  durch Paul Goldmann am 11. 3. [1897] in Paris
\newline{}Erhalt  durch Arthur Schnitzler am [15. 3. 1897] in Wien}\toendnotes[C]{\smallbreak}
\Standort{DLA, A:Schnitzler, HS.NZ85.1.3167.}
\physDesc{Brief, 2 Blätter, 7 Seiten, 3385 Zeichen
\newline{}Handschrift: blaue Tinte, deutsche Kurrent
\newline{}Schnitzler: mit Bleistift das Jahr »97« vermerkt }\toendnotes[C]{\smallbreak}
\pstart
           {\pb}\textcolor{gray}{\textbf{\textbf{Frankfurter Zeitung\orgindex{Frankfurter Zeitung@Frankfurter Zeitung|pw}}}}\pend
           
\pstart
           \textcolor{gray}{\textbf{(\begin{otherlanguage}{french}Gazette de Francfort\end{otherlanguage}\orgindex{Frankfurter Zeitung@Frankfurter Zeitung|pw}).}}\pend
           
\pstart
           \textcolor{gray}{\textbf{\textbf{\begin{otherlanguage}{french}Fondateur M.\end{otherlanguage}{ }L. Sonnemann\pwindex{Sonnemann, Leopold 29.\,10.\,1831 Höchberg – 30.\,10.\,1909 Frankfurt am Main@\textsc{Sonnemann, Leopold} (29.\,10.\,1831 Höchberg – 30.\,10.\,1909 Frankfurt am Main), \emph{Journalist, Herausgeber}|pw}.}}}\hfill \textsc{Paris\oindex{Paris@\textbf{Paris}, \emph{Hauptstadt}|pw}}, 1\substVorne{}\textsuperscript{0}\substDazwischen{}1\substHinten{}. März.\pend
           
\pstart
           \begin{otherlanguage}{french}\textcolor{gray}{\textbf{Journal politique, financier,}}\end{otherlanguage}\pend
           
\pstart
           \begin{otherlanguage}{french}\textcolor{gray}{\textbf{commercial et littéraire.}}\end{otherlanguage}\pend
           
\pstart
           \begin{otherlanguage}{french}\textcolor{gray}{\textbf{\textbf{Paraissant trois fois par jour.}}}\end{otherlanguage}\pend
           
\pstart
           \begin{otherlanguage}{french}\textcolor{gray}{\textbf{\textbf{Bureau à Paris\oindex{Paris@\textbf{Paris}, \emph{Hauptstadt}|pw}}}}\end{otherlanguage}\pend
           
\pstart
           \begin{otherlanguage}{french}\textcolor{gray}{\textbf{\textbf{24. Rue Feydeau\oindex{rue Feydeau@\textbf{rue Feydeau}, \emph{Straße}|pw}.}}}\end{otherlanguage}\pend
           
\pstart{}Mein lieber Freund,\pend\vspace{0.5em}
\pstart
           Ich habe mit der verfluchten \label{K_L02805-1v}\edtext{Orient-Geſchichte}{\lemma{\textnormal{\emph{Orient-Geschichte}}}\Cendnote{\textnormal{Höchstwahrscheinlich Bezug auf den sich zunehmend zum (Türk\oindex{Türkei@\textbf{Türkei}|pwkv}isch-Griech\oindex{Griechenland@\textbf{Griechenland}|pwkv}ischen) Krieg aufschaukelnden
                  Konflikt auf Kreta\oindex{Kreta@\textbf{Kreta}, \emph{Insel}|pwk}, über den Goldmann\pwindex{Goldmann, Paul 31.\,1.\,1865 Breslau – 25.\,9.\,1935 Wien@\textsc{Goldmann, Paul} (31.\,1.\,1865 Breslau – 25.\,9.\,1935 Wien), \emph{Schriftsteller, Journalist}|pwk} intensiv berichtet hat (vgl. XXXX Auszeichnungsfehler: Dokument L02803 nicht gefunden). Daneben könnte sich
                     Goldmann\pwindex{Goldmann, Paul 31.\,1.\,1865 Breslau – 25.\,9.\,1935 Wien@\textsc{Goldmann, Paul} (31.\,1.\,1865 Breslau – 25.\,9.\,1935 Wien), \emph{Schriftsteller, Journalist}|pwk} auch auf folgende Berichte\pwindex{Goldmann, Paul 31.\,1.\,1865 Breslau – 25.\,9.\,1935 Wien@\textsc{Goldmann, Paul} (31.\,1.\,1865 Breslau – 25.\,9.\,1935 Wien), \emph{Schriftsteller, Journalist}!deutsche Orientpolitik und das Ausland@\strich\emph{Die deutsche Orientpolitik und das Ausland}|pwkv}\pwindex{Goldmann, Paul 31.\,1.\,1865 Breslau – 25.\,9.\,1935 Wien@\textsc{Goldmann, Paul} (31.\,1.\,1865 Breslau – 25.\,9.\,1935 Wien), \emph{Schriftsteller, Journalist}!Frankreich. [Zum Tod des Persers Djemal-ed-din]@\strich\emph{Frankreich. [Zum Tod des Persers Djemal-ed-din]}|pwkv} beziehen: G\pwindex{Goldmann, Paul 31.\,1.\,1865 Breslau – 25.\,9.\,1935 Wien@\textsc{Goldmann, Paul} (31.\,1.\,1865 Breslau – 25.\,9.\,1935 Wien), \emph{Schriftsteller, Journalist}|pwkv} [ = Paul Goldmann\pwindex{Goldmann, Paul 31.\,1.\,1865 Breslau – 25.\,9.\,1935 Wien@\textsc{Goldmann, Paul} (31.\,1.\,1865 Breslau – 25.\,9.\,1935 Wien), \emph{Schriftsteller, Journalist}|pwk}]: \emph{Die
                        deutsche Orientpolitik und das Ausland}\pwindex{Goldmann, Paul 31.\,1.\,1865 Breslau – 25.\,9.\,1935 Wien@\textsc{Goldmann, Paul} (31.\,1.\,1865 Breslau – 25.\,9.\,1935 Wien), \emph{Schriftsteller, Journalist}!deutsche Orientpolitik und das Ausland@\strich\emph{Die deutsche Orientpolitik und das Ausland}|pwk}. In: \emph{Frankfurter Zeitung}\pwindex{Frankfurter Zeitung@\emph{Frankfurter Zeitung}|pwk}, Jg. 41, Nr. 64, 5. 3. 1897, S. 1; G\pwindex{Goldmann, Paul 31.\,1.\,1865 Breslau – 25.\,9.\,1935 Wien@\textsc{Goldmann, Paul} (31.\,1.\,1865 Breslau – 25.\,9.\,1935 Wien), \emph{Schriftsteller, Journalist}|pwkv} [ = Paul Goldmann\pwindex{Goldmann, Paul 31.\,1.\,1865 Breslau – 25.\,9.\,1935 Wien@\textsc{Goldmann, Paul} (31.\,1.\,1865 Breslau – 25.\,9.\,1935 Wien), \emph{Schriftsteller, Journalist}|pwk}]: \emph{Frankreich. [Zum Tod des Persers Djemal-ed-din]}\pwindex{Goldmann, Paul 31.\,1.\,1865 Breslau – 25.\,9.\,1935 Wien@\textsc{Goldmann, Paul} (31.\,1.\,1865 Breslau – 25.\,9.\,1935 Wien), \emph{Schriftsteller, Journalist}!Frankreich. [Zum Tod des Persers Djemal-ed-din]@\strich\emph{Frankreich. [Zum Tod des Persers Djemal-ed-din]}|pwk}. In: \emph{Frankfurter Zeitung}\pwindex{Frankfurter Zeitung@\emph{Frankfurter Zeitung}|pwk}, Jg. 41, Nr. 72, 13. 3. 1897, Erstes Morgenblatt,
                  S. 1.}}}\label{K_L02805-1} unbändig zu thun. Auch \strikeout{\textcolor{gray}{er}} thut mir mein Auge \strikeout{\textcolor{gray}{f}} unerträglich weh. So kommt es, daß ich Deinen lieben Brief erſt heut beantworte.\pend
           
\pstart
           Ich danke Dir von ganzem Herzen für den Beiſtand, den Du mir in der Angelegenheit mit
                  \textsc{Kleins\pwindex{Klein, Arthur 27.\,11.\,1868 Wien – 28.\,7.\,1943@\textsc{Klein, Arthur} (27.\,11.\,1868 Wien – 28.\,7.\,1943)|pw}}{ }Bruder\pwindex{Klein, Richard *~7.\,8.\,1873 Baden bei Wien@\textsc{Klein, Richard} (*~7.\,8.\,1873 Baden bei Wien), \emph{Maler}|pwv} geliehen. Ich bin{ }ſelbſt wohl auch nicht ohne Schuld an dieſen Unannehmlichkeiten. Ich laſſe mir Leute
               dieſer Art zu nahe kommen, in einer gewiſſen{ }ſchlamperten Liebenswürdigkeit. Auch
               habe ich mich von meiner Heftigkeit zu{ }ſehr hinreißen laſſen. \textsc{Arthur Klein\pwindex{Klein, Arthur 27.\,11.\,1868 Wien – 28.\,7.\,1943@\textsc{Klein, Arthur} (27.\,11.\,1868 Wien – 28.\,7.\,1943)|pw}} hat{ }ſich prachtvoll benommen. {\pb}Wenn Du ihn\pwindex{Klein, Arthur 27.\,11.\,1868 Wien – 28.\,7.\,1943@\textsc{Klein, Arthur} (27.\,11.\,1868 Wien – 28.\,7.\,1943)|pwv}{ }ſiehſt,{ }ſo danke ihm noch
               beſonders, bitte\substVorne{}\textsuperscript{,}\substDazwischen{}!\substHinten{} Freilich hat es weiterhin noch einige Klatſchereien gegeben, und die
               Unannehmlichkeiten{ }ſind noch nicht zu Ende. \strikeout{Abe\textcolor{gray}{r}} Aber ich mache mir heut große Vorwürfe, Dich
               mit der ganzen Sache behelligt zu haben{\dotsfive}\pend
           
\pstart
           Soeben erhalte ich für \strikeout{Euch} Dich und \textsc{Richard\pwindex{Beer-Hofmann, Richard 11.\,7.\,1866 Wien – 26.\,9.\,1945 New York City@\textsc{Beer-Hofmann, Richard} (11.\,7.\,1866 Wien – 26.\,9.\,1945 New York City), \emph{Schriftsteller}|pw}} zwei Nummern von »\textsc{Politiken\pwindex{Politiken@\emph{Politiken}|pw}}«, wo \textsc{Peter Nansen\pwindex{Nansen, Peter 20.\,1.\,1861 Kopenhagen – 31.\,7.\,1918 Mariager@\textsc{Nansen, Peter} (20.\,1.\,1861 Kopenhagen – 31.\,7.\,1918 Mariager), \emph{Schriftsteller, Journalist, Verleger}|pw}} über Dich und zugleich über uns \label{K_L02805-2v}\edtext{geſchrieben\pwindex{Nansen, Peter 20.\,1.\,1861 Kopenhagen – 31.\,7.\,1918 Mariager@\textsc{Nansen, Peter} (20.\,1.\,1861 Kopenhagen – 31.\,7.\,1918 Mariager), \emph{Schriftsteller, Journalist, Verleger}!Arthur Schnitzler. »Elskovsleg«s Forfatter@\strich\emph{Arthur Schnitzler. »Elskovsleg«s Forfatter}|pwv}}{\lemma{\textnormal{\emph{geschrieben}}}\Cendnote{\textnormal{–n–\pwindex{Nansen, Peter 20.\,1.\,1861 Kopenhagen – 31.\,7.\,1918 Mariager@\textsc{Nansen, Peter} (20.\,1.\,1861 Kopenhagen – 31.\,7.\,1918 Mariager), \emph{Schriftsteller, Journalist, Verleger}|pwkv} [ = Peter Nansen\pwindex{Nansen, Peter 20.\,1.\,1861 Kopenhagen – 31.\,7.\,1918 Mariager@\textsc{Nansen, Peter} (20.\,1.\,1861 Kopenhagen – 31.\,7.\,1918 Mariager), \emph{Schriftsteller, Journalist, Verleger}|pwk}]: \emph{Arthur
                        Schnitzler. »Elskovsleg«s Forfatter}\pwindex{Nansen, Peter 20.\,1.\,1861 Kopenhagen – 31.\,7.\,1918 Mariager@\textsc{Nansen, Peter} (20.\,1.\,1861 Kopenhagen – 31.\,7.\,1918 Mariager), \emph{Schriftsteller, Journalist, Verleger}!Arthur Schnitzler. »Elskovsleg«s Forfatter@\strich\emph{Arthur Schnitzler. »Elskovsleg«s Forfatter}|pwk}. In: \emph{Politiken}\pwindex{Politiken@\emph{Politiken}|pwk}, Nr. 68, 9. 3. 1897, S. 1.}}}\label{K_L02805-2} hat. Ich verſtehe kein Wort davon,
               aber es{ }ſcheint prächtig zu{ }ſein. \strikeout{Du} Ich{ }ſende beide
                  Nummern\pwindex{Politiken@\emph{Politiken}|pwv} an Dich.\pend
           
\pstart
           Meine \label{K_L02805-3v}\edtext{Reiſe nach \textsc{Nizza\oindex{Nizza@\textbf{Nizza}, \emph{Hauptstadt}|pw}}}{\lemma{\textnormal{\emph{Reise nach Nizza}}}\Cendnote{\textnormal{Siehe XXXX Auszeichnungsfehler: Dokument L02803 nicht gefunden.
               }}}\label{K_L02805-3} iſt infolge der Orient-Ereigniſſe auf nächſte Woche verſchoben.\pend
           
\pstart
           {\pb}Ich kann Dir gar nicht{ }ſagen, wie ich mich auf Dein
               Kommen freue! Ein vorheriges Zuſammentreffen in der \label{K_L02805-4v}\edtext{Schweiz\oindex{Schweiz@\textbf{Schweiz}|pw}}{\lemma{\textnormal{\emph{Schweiz}}}\Cendnote{\textnormal{Schnitzler war vom 10. 4. 1897 bis zum 11. 4. 1897 in Zürich\oindex{Zürich@\textbf{Zürich}|pwk}. Er kam gerade aus München\oindex{München@\textbf{München}|pwk} und reiste nach Paris\oindex{Paris@\textbf{Paris}, \emph{Hauptstadt}|pwk} weiter.}}}\label{K_L02805-4} iſt leider unmöglich. Ich darf mich nicht vom Flecke
               rühren; hoffentlich habe ich nur hier während Deiner Anweſenheit wenig zu thun, damit
               ich Dich ordentlich genießen kann. Die Wohnungsfrage wird freilich nicht leicht zu
               erledigen{ }ſein. \strikeout{D} Ich habe nochmals energiſcheſte
               Nachforſchungen angeſtellt. Das Reſultat iſt das, was ich gewußt hatte: Anſtändige
                  fran\oindex{Frankreich@\textbf{Frankreich}|pwv}zöſiſche Familien geben
               keine \textsc{Pension}, und diejenigen Familien, welche \textsc{Pension} geben,{ }ſind nicht anſtändig. Ausnahmen gibt {\pb}es wohl, aber eine{ }ſolche zu finden, iſt reine
               Zufallsſache. Im Übrigen denke auch ich, daß Du irgendwo zwiſchen Stadt und Land
               wohnen{ }ſollſt, am Beſten in \textsc{Passy\oindex{16. arrondissement [Paris]@\textbf{16. arrondissement [Paris]}|pw}}, das beſonders anmuthig und zugleich bequem iſt. Was ich Dir{ }ſage,{ }ſind keine
               definitiven Reſultate. Ich habe einige fran\oindex{Frankreich@\textbf{Frankreich}|pwv}zöſiſche Bekannte mit Umfragen beauftragt, und die
               Nachforſchungen dauern fort. Ein \textsc{Hotel}, wie Du es
               wünſcheſt, wird raſch gefunden{ }ſein,{ }ſobald Du mir das Datum \strikeout{meiner} Deiner Ankunft mittheilſt. Allzuviel \textsc{Comfort} wirſt Du freilich nicht finden. Das Pariſ\oindex{Paris@\textbf{Paris}, \emph{Hauptstadt}|pw}er {\pb}Hotelweſen iſt{ }ſehr
               zurück. Das hat{ }ſchon \label{K_L02805-5v}\edtext{\textsc{Balzac\pwindex{Balzac, Honoré de 20.\,5.\,1799 Tours – 18.\,8.\,1850 Paris@\textsc{Balzac, Honoré de} (20.\,5.\,1799 Tours – 18.\,8.\,1850 Paris), \emph{Schriftsteller}|pw}} conſtatirt}{\lemma{\textnormal{\emph{Balzac constatirt}}}\Cendnote{\textnormal{Balzac\pwindex{Balzac, Honoré de 20.\,5.\,1799 Tours – 18.\,8.\,1850 Paris@\textsc{Balzac, Honoré de} (20.\,5.\,1799 Tours – 18.\,8.\,1850 Paris), \emph{Schriftsteller}|pwk} thematisierte die
                  Beherbergungsindustrie in Paris\oindex{Paris@\textbf{Paris}, \emph{Hauptstadt}|pwk} in mehreren
                  seiner Bücher. Er beschrieb die Hotels als überfüllt, schmutzig und überteuert,
                  mit schlechtem Service und wenig Privatsphäre. Kritisiert wurden von ihm auch die
                  Eigentümerinnen und Eigentümer dieser Hotels, die die Bedürfnisse der Reisenden
                  ausnutzten und überhöhte Preise für minderwertige Unterkünfte verlangten:
                     »il n’existe pas encore un seul hôtel où tout voyageur riche puisse
                     retrouver son chez soi« (»es gibt bislang kein einziges Hotel, in dem
                  selbst ein reicher Reisender sich zu Hause fühlen kann«; \emph{Illusions Perdues}\pwindex{Balzac, Honoré de 20.\,5.\,1799 Tours – 18.\,8.\,1850 Paris@\textsc{Balzac, Honoré de} (20.\,5.\,1799 Tours – 18.\,8.\,1850 Paris), \emph{Schriftsteller}!Illusions perdues@\strich\emph{Illusions perdues}|pwk}, 2. Teil.)}}}\label{K_L02805-5},
               und{ }ſeit \textsc{Balzac\pwindex{Balzac, Honoré de 20.\,5.\,1799 Tours – 18.\,8.\,1850 Paris@\textsc{Balzac, Honoré de} (20.\,5.\,1799 Tours – 18.\,8.\,1850 Paris), \emph{Schriftsteller}|pw}} hat{ }ſich wenig geändert{\dotsseven}\pend
           
\pstart
           Was Du mir über Deine Freundin\pwindex{Reinhard, Marie 13.\,3.\,1871 Wien – 18.\,3.\,1899 ebd.@\textsc{Reinhard, Marie} (13.\,3.\,1871 Wien – 18.\,3.\,1899 ebd.), \emph{Gesangspädagogin}|pwv}{ }ſchreibſt, iſt{ }ſehr{ }ſchön. Ich habe nie daran gezweifelt, daß{ }ſie
               »auf unſerem \textsc{Niveau}« iſt,{ }ſchon weil{ }ſie Deine Freundin\pwindex{Reinhard, Marie 13.\,3.\,1871 Wien – 18.\,3.\,1899 ebd.@\textsc{Reinhard, Marie} (13.\,3.\,1871 Wien – 18.\,3.\,1899 ebd.), \emph{Gesangspädagogin}|pwv} iſt. Du kannſt Dir
               denken, wie ich mich darauf freue,{ }ſie kennen zu lernen. Darf ich Dich einſtweilen
               bitten, mich ihr zu empfehlen? {\dotsfour}\pend
           
\pstart
           Nach der{ }ſo gut verlaufenen \label{K_L02805-6v}\edtext{Unterredung mit dem {\pb}Vater\pwindex{Reinhard, Karl 2.\,3.\,1834 Prag – 28.\,4.\,1905 Wien@\textsc{Reinhard, Karl} (2.\,3.\,1834 Prag – 28.\,4.\,1905 Wien), \emph{Geschäftsführer}|pwv}}{\lemma{\textnormal{\emph{Unterredung … Vater}}}\Cendnote{\textnormal{Siehe A. S.: \emph{Tagebuch}, 23. 2. 1897 und XXXX Auszeichnungsfehler: Dokument L02804 nicht gefunden.
               }}}\label{K_L02805-6}{ }ſind wohl die{ }ſchlimmſten Unannehmlichkeiten vorüber. Ich halte es für ein
               großes Glück, daß ein äußerer Zwang Dich auf einige Zeit von Wien\oindex{Wien@\textbf{Wien}, \emph{Verwaltungsgebiet}|pw} wegtreibt. Ich verſpreche mir viel von der Wirkung, die \textsc{Paris\oindex{Paris@\textbf{Paris}, \emph{Hauptstadt}|pw}} auf Dich haben wird. Es wird Dich elektriſiren, und Dich mit Schaffenskraft und
               Schaffensluſt erfüllen. Auch wirſt Du den Pariſ\oindex{Paris@\textbf{Paris}, \emph{Hauptstadt}|pw}er
               Frühling{ }ſehen, welcher eine der Gnaden Gottes iſt.\pend
           
\pstart
           Freilich könnte es{ }ſich auch ereignen, daß Dir hier Alles{ }ſehr zuwider iſt.\pend
           
\pstart
           {\pb}Wir wollen den Himmel bitten, daß es gut
               ausgeht.\pend
           
\pstart
           Bald höre ich wohl Näheres?\pend
           
\pstart
           Ich begrüße Dich von Herzen!\pend
           
\pstart
           Dein {\\[\baselineskip]}\spacefill\mbox{Paul Goldmn}\pend
           \leftskip=0em{}
\pstart
           \noindent{}Schön habt Ihr wieder in \textsc{Wien\oindex{Wien@\textbf{Wien}, \emph{Verwaltungsgebiet}|pw}}{ }\label{K_L02805-7v}\edtext{gewählt}{\lemma{\textnormal{\emph{gewählt}}}\Cendnote{\textnormal{Am 4. 3. 1897 begannen in
                        Cisleithanien\oindex{Cisleithanien@\textbf{Cisleithanien}|pwkv}, dem
                     nördlichen und westlichen Teils Österreich-Ungarns\oindex{Österreich-Ungarn@\textbf{Österreich-Ungarn}|pwk}, die Reichsrats-\orgindex{Reichsrat@Reichsrat|pwkv}, also Parlamentswahlen\orgindex{Reichsrat@Reichsrat|pwkv}. In Wien\oindex{Wien@\textbf{Wien}, \emph{Verwaltungsgebiet}|pwk} feierte
                     insbesondere die \emph{Christlichsoziale Partei}\orgindex{Christlichsoziale Partei@Christlichsoziale Partei|pwk}
                     Erfolge. Schnitzler notierte dazu am 12. 3. 1897 im \emph{Tagebuch}\pwindex{Schnitzler, Arthur 15.\,5.\,1862 Wien – 21.\,10.\,1931 ebd.@\textsc{Schnitzler, Arthur} (15.\,5.\,1862 Wien – 21.\,10.\,1931 ebd.), \emph{Schriftsteller, Mediziner}!Tagebuch@\strich\emph{Tagebuch}|pwk}: »Sehr verstimmt, auch
                        durch den Antisem. –«}}}\label{K_L02805-7}. Ihr{ }ſeid eine rechte Bagage. Schämt Ihr
                  Euch gar nicht vor Europa\oindex{Europa@\textbf{Europa}|pw}?\pend
           \selectlanguage{ngerman}\endnumbering\briefempfaengerindex{Schnitzler, Arthur@\textsc{Schnitzler, Arthur}!zzzGoldmann, Paul@\emph{von Paul Goldmann}!1897-03-111@{11. 3. [1897]}|)be}\mylabel{L02805h}  \newcommand{\dateiname}{L02805}\newcommand{\titel}{Paul Goldmann an Arthur Schnitzler, 11. 3. [1897]}\newcommand{\editorInnen}{Martin Anton Müller und Laura Untner}%% latex-leseansicht-abspann.tex
%% Abspann für die Leseansicht.
%% Der Schalter \ifkorrekturansicht ist bereits durch den Vorspann gesetzt.

%% latex-abspann.tex
%% Gemeinsamer Abspann für Korrekturansicht und Leseansicht.
%% Setzt den Schalter \ifkorrekturansicht voraus (gesetzt in den
%% einbindenden Dateien latex-korrekturansicht-abspann.tex bzw.
%% latex-leseansicht-abspann.tex).
%% ---------------------------------------------------------------

\normalsize

% Das esempio-Environment wird nur in der Leseansicht benötigt
\ifkorrekturansicht\else
\newenvironment{esempio}[3]%
{
    \vspace{1.5ex}
    \rlap{\underline{#1}}
    \par
    \setlength{\parindent}{0cm}
    \nopagebreak
    \leftskip=#2cm
    \rightskip=#3cm
}
{
    \par
}
\fi

\doendnotes{C}
\bigskip
\vfill

\clearpage

\footnotesize

\ifkorrekturansicht
  \lohead{\textsc{register}}
\fi

% theindex-Environment neu definieren ohne reledmac
\makeatletter
\renewenvironment{theindex}{%
  \ifkorrekturansicht
    \section*{\indexname}%
  \else
    \subsubsection*{Index der erwähnten Entitäten}%
  \fi
  \setlength{\parindent}{0pt}%
  \setlength{\parskip}{0pt plus 0.3pt}%
  \let\item\@idxitem
}{%
  \ifkorrekturansicht\clearpage\fi
}
\makeatother

\IfFileExists{\jobname-pw.ind}{\input{\jobname-pw.ind}}{}

% Quellenangabe nur in der Leseansicht
\ifkorrekturansicht\else
% Fallback-Definitionen, falls die .tex-Datei \titel etc. nicht gesetzt hat
\providecommand{\titel}{}
\providecommand{\editorInnen}{}
\providecommand{\dateiname}{\jobname}

\vspace{3cm}

\vfill

\footnotesize
\textsc{Quelle}: \titel. Herausgegeben von {\editorInnen}. In: \emph{Arthur Schnitzler: Briefwechsel mit Autorinnen und Autoren}.
 Digitale Edition, https://schnitzler-briefe.acdh.oeaw.ac.at/{\dateiname}.html (Stand \today)
\fi

\end{document}


