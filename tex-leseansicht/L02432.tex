%% latex-korrekturansicht-vorspann.tex
%% Vorspann für die Korrekturansicht.
%% Lädt die gemeinsame Datei latex-vorspann.tex mit gesetztem Schalter.

\newif\ifkorrekturansicht
\korrekturansichttrue

\input{../tex-inputs/latex-vorspann}


\section[Arthur Schnitzler an Georg Brandes, 11. 2. 1925]{L02432 Arthur Schnitzler an Georg Brandes, 11. 2. 1925}
\nopagebreak\mylabel{L02432v}
\rehead{ }\normalsize\beginnumbering\briefempfaengerindex{Brandes, Georg@\textsc{Brandes, Georg}!zzzSchnitzler, Arthur@\emph{von Arthur Schnitzler}!1925-02-111@{11. 2. 1925}|(be}
\toendnotes[C]{\smallbreak\pagebreak[2]}\Standort{Kopenhagen, Det Kongelige Bibliotek, Georg Brandes Arkiv, box 125.}
\physDesc{Brief, 1 Blatt, 2 Seiten, 856 Zeichen
\newline{}Handschrift: schwarze Tinte, lateinische Kurrent
\newline{}Ordnung: 1) mit Bleistift von unbekannter Hand nummeriert: »\strikeout{49}«  2) mit Bleistift von unbekannter Hand nummeriert:
                                    »50.«}
\buchAbdrucke{\weitereDrucke{Georg Brandes, Arthur Schnitzler: \emph{Ein Briefwechsel}. Bern: \emph{Francke} 1956, S. 143–144.} }\toendnotes[C]{\smallbreak}
\pstart
           \raggedleft{}{\pb}Wien\oindex{Wien@\textbf{Wien}, \emph{A.ADM2}|pw}, 11. 2. 1925\pend
           
\pstart{}lieber und verehrter Freund,\pend\vspace{0.5em}
\pstart
           ich lese, und mein Sohn\pwindex{Schnitzler, Heinrich 09.08.1902 – 12.07.1982@\textsc{Schnitzler, Heinrich} (09.08.1902 – 12.07.1982), \emph{Regisseur/Regisseurin, Schauspieler/Schauspielerin}|pwv}
               schreibt mir, dſs Sie im Laufe des März nach Berlin\oindex{Berlin@\textbf{Berlin}, \emph{P.PPLC}|pw} kommen wollen. Ich hatte die gleiche Absicht; und wäre
               nun sehr froh, we{\geminationn} ich Ihnen dort begegnen dürfte. Sind
               Sie sich über den Termin Ihrer Reise schon klar? Wollten Sie mir darüber so \uline{bald als möglich} ein Wort schreiben, wär ich Ihnen von
               Herzen dankbar.\pend
           
\pstart
           In der Schweiz\oindex{Schweiz@\textbf{Schweiz}, \emph{A.PCLI}|pw} (Vortragsreise und nachheriger
               Aufenthalt im Engadin\oindex{Engadin@\textbf{Engadin}, \emph{T.VAL}|pw}) {\pb}hatte ich einen kurzen Bericht über Sie durch Dr.
                  Zbinden\pwindex{Zbinden, Hans 26.08.1893 – 09.05.1971@\textsc{Zbinden, Hans} (26.08.1893 – 09.05.1971), \emph{Schriftsteller/Schriftstellerin}|pw}, der Sie damals in Kopenhagen\oindex{Kopenhagen@\textbf{Kopenhagen}, \emph{P.PPLC}|pw} etwas leidend angetroffen hatte. Nun
               gehts Ihnen hoffentlich wieder ganz gut. Mir auch ganz leidlich. Manche \strikeout{g} schöne Abendstunde verbring ich mit Ihren Büchern,
               den neuen und den alten. Jetzt bin ich wieder einmal in der »Romantik« der Hauptströmungen\pwindex{Hauptstroemungen der Literatur des neunzehnten Jahrhunderts@\emph{Hauptströmungen der Literatur des neunzehnten Jahrhunderts}|pw}.\pend
           
\pstart
           Also bitte, schreiben Sie mir gleich ein Wort.\pend
           
\pstart
           Sie von Herzen grüßend{\\[\baselineskip]}Ihr{\\[\baselineskip]}\spacefill\mbox{Arthur Schnitzler}\pend
           \leftskip=0em{}\selectlanguage{ngerman}\endnumbering\briefempfaengerindex{Brandes, Georg@\textsc{Brandes, Georg}!zzzSchnitzler, Arthur@\emph{von Arthur Schnitzler}!1925-02-111@{11. 2. 1925}|)be}\mylabel{L02432h}  \normalsize

\doendnotes{C}
\bigskip
\vfill

\clearpage

\footnotesize

\lohead{\textsc{register}}

% Definiere theindex-Environment komplett neu ohne reledmac
\makeatletter
\renewenvironment{theindex}{%
  \section*{\indexname}%
  \setlength{\parindent}{0pt}%
  \setlength{\parskip}{0pt plus 0.3pt}%
  \let\item\@idxitem
}{%
  \clearpage
}
\makeatother

\IfFileExists{\jobname-pw.ind}{\input{\jobname-pw.ind}}{}

\end{document}

      