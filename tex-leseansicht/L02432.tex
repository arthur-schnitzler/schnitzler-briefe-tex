%% latex-leseansicht-vorspann.tex
%% Vorspann für die Leseansicht.
%% Lädt die gemeinsame Datei latex-vorspann.tex mit nicht gesetztem Schalter.

\newif\ifkorrekturansicht
\korrekturansichtfalse

\input{../tex-inputs/latex-vorspann}


               \section[Arthur Schnitzler an Georg Brandes, 11. 2. 1925]{ Arthur Schnitzler an Georg Brandes, 11. 2. 1925}\nopagebreak\mylabel{v}\rehead{ }\begin{ledgroupsized}[t]{13cm}\normalsize\beginnumbering\briefempfaengerindex{Brandes, Georg@\textsc{Brandes, Georg}!zzzSchnitzler, Arthur@\emph{von Arthur Schnitzler}!1925-02-111@{11. 2. 1925}|(be} \toendnotes[C]{\smallbreak\pagebreak[2]} \Standort{Kopenhagen, Det Kongelige Bibliotek, Georg Brandes Arkiv, box 125.}
\physDesc{Brief, 1 Blatt, 2 Seiten
\newline{}Handschrift: schwarze Tinte, lateinische Kurrent\newline{}Ordnung: 1) mit Bleistift von unbekannter Hand nummeriert: »\strikeout{49}« 2) mit Bleistift von unbekannter Hand nummeriert: »50.«}\buchAbdrucke{\weitereDrucke{Georg Brandes, Arthur Schnitzler: \emph{Ein Briefwechsel}. Hg. Kurt Bergel. Bern: \emph{Francke} 1956, S. 143–144.} }\toendnotes[C]{\smallbreak}\pstart
           \raggedleft{}{\pb}Wien\oindex{Wien@\textbf{Wien}|pw}, 11. 2. 1925\pend
           \pstart{}lieber und verehrter Freund,\pend\pstart
           ich lese, und mein Sohn\pwindex{Schnitzler, Heinrich 09.08.1902 – 12.07.1982@\textsc{Schnitzler, Heinrich} (09.08.1902 – 12.07.1982), \emph{Regisseur, Schauspieler}|pwv}
                    schreibt mir, dſs Sie im Laufe des März nach Berlin\oindex{Berlin@\textbf{Berlin}|pw} kommen wollen. Ich hatte die gleiche Absicht; und
                    wäre nun sehr froh, we{\geminationn} ich Ihnen dort begegnen
                    dürfte. Sind Sie sich über den Termin Ihrer Reise schon klar? Wollten Sie mir
                    darüber so \uline{bald als möglich} ein Wort schreiben,
                    wär ich Ihnen von Herzen dankbar.\pend
           \pstart
           In der Schweiz\oindex{Schweiz@\textbf{Schweiz}|pw} (Vortragsreise und nachheriger
                    Aufenthalt im Engadin\oindex{Engadin@\textbf{Engadin}|pw}) {\pb}hatte ich einen kurzen Bericht über Sie
                    durch Dr. Zbinden\pwindex{Zbinden, Hans 26.08.1893 – 09.05.1971@\textsc{Zbinden, Hans} (26.08.1893 – 09.05.1971), \emph{Schriftsteller}|pw}, der Sie damals in Kopenhagen\oindex{Kopenhagen@\textbf{Kopenhagen}|pw} etwas leidend angetroffen hatte.
                    Nun gehts Ihnen hoffentlich wieder ganz gut. Mir auch ganz leidlich. Manche \strikeout{g} schöne Abendstunde verbring ich mit Ihren
                    Büchern, den neuen und den alten. Jetzt bin ich wieder einmal in der »Romantik«
                    der Hauptströmungen\pwindex{Brandes, Georg 04.02.1842 – 19.02.1927@\textsc{Brandes, Georg} (04.02.1842 – 19.02.1927)!Hauptstroemungen der Literatur des neunzehnten Jahrhunderts1872@\strich\emph{Hauptströmungen der Literatur des neunzehnten Jahrhunderts} {[}1872{]}|pw}.\pend
           \pstart
           Also bitte, schreiben Sie mir gleich ein Wort.\pend
           \pstart
           Sie von Herzen grüßend{\\[\baselineskip]}Ihr{\\[\baselineskip]}\spacefill\mbox{Arthur Schnitzler}\pend
           \leftskip=0em{}\endnumbering\briefempfaengerindex{Brandes, Georg@\textsc{Brandes, Georg}!zzzSchnitzler, Arthur@\emph{von Arthur Schnitzler}!1925-02-111@{11. 2. 1925}|)be}\mylabel{h}\end{ledgroupsized}  \newcommand{\dateiname}{L02432}\newcommand{\titel}{Arthur Schnitzler an Georg Brandes, 11. 2. 1925}\newcommand{\editorInnen}{Martin Anton Müller und Gerd-Hermann Susen}%% latex-leseansicht-abspann.tex
%% Abspann für die Leseansicht.
%% Der Schalter \ifkorrekturansicht ist bereits durch den Vorspann gesetzt.

%% latex-abspann.tex
%% Gemeinsamer Abspann für Korrekturansicht und Leseansicht.
%% Setzt den Schalter \ifkorrekturansicht voraus (gesetzt in den
%% einbindenden Dateien latex-korrekturansicht-abspann.tex bzw.
%% latex-leseansicht-abspann.tex).
%% ---------------------------------------------------------------

\normalsize

% Das esempio-Environment wird nur in der Leseansicht benötigt
\ifkorrekturansicht\else
\newenvironment{esempio}[3]%
{
    \vspace{1.5ex}
    \rlap{\underline{#1}}
    \par
    \setlength{\parindent}{0cm}
    \nopagebreak
    \leftskip=#2cm
    \rightskip=#3cm
}
{
    \par
}
\fi

\doendnotes{C}
\bigskip
\vfill

\clearpage

\footnotesize

\ifkorrekturansicht
  \lohead{\textsc{register}}
\fi

% theindex-Environment neu definieren ohne reledmac
\makeatletter
\renewenvironment{theindex}{%
  \ifkorrekturansicht
    \section*{\indexname}%
  \else
    \subsubsection*{Index der erwähnten Entitäten}%
  \fi
  \setlength{\parindent}{0pt}%
  \setlength{\parskip}{0pt plus 0.3pt}%
  \let\item\@idxitem
}{%
  \ifkorrekturansicht\clearpage\fi
}
\makeatother

\IfFileExists{\jobname-pw.ind}{\input{\jobname-pw.ind}}{}

% Quellenangabe nur in der Leseansicht
\ifkorrekturansicht\else
% Fallback-Definitionen, falls die .tex-Datei \titel etc. nicht gesetzt hat
\providecommand{\titel}{}
\providecommand{\editorInnen}{}
\providecommand{\dateiname}{\jobname}

\vspace{3cm}

\vfill

\footnotesize
\textsc{Quelle}: \titel. Herausgegeben von {\editorInnen}. In: \emph{Arthur Schnitzler: Briefwechsel mit Autorinnen und Autoren}.
 Digitale Edition, https://schnitzler-briefe.acdh.oeaw.ac.at/{\dateiname}.html (Stand \today)
\fi

\end{document}


      