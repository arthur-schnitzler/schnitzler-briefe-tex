%% latex-leseansicht-vorspann.tex
%% Vorspann für die Leseansicht.
%% Lädt die gemeinsame Datei latex-vorspann.tex mit nicht gesetztem Schalter.

\newif\ifkorrekturansicht
\korrekturansichtfalse

\input{../tex-inputs/latex-vorspann}


\section[Richard Beer-Hofmann an Arthur Schnitzler, {[}29. 5. 1895{]}]{L00446 Richard Beer-Hofmann an Arthur Schnitzler, {[}29. 5. 1895{]}}
\nopagebreak\mylabel{L00446v}
\rehead{ }\normalsize\beginnumbering\briefempfaengerindex{Schnitzler, Arthur@\textsc{Schnitzler, Arthur}!zzzBeer-Hofmann, Richard@\emph{von Richard Beer-Hofmann}!1895-05-291@{{[}29. 5. 1895{]}}|(be}
\toendnotes[C]{\smallbreak\pagebreak[2]}
\correspDesc{Versand  durch Richard Beer-Hofmann am [29. 5. 1895] in Wien
\newline{}Erhalt  durch Arthur Schnitzler im Zeitraum [29. 5. 1895
                  – 2. 6. 1895?] in Wien}\toendnotes[C]{\smallbreak}
\Standort{CUL, Schnitzler, B 8.}
\physDesc{Brief, 1 Blatt, 3 Seiten, 461 Zeichen
\newline{}Handschrift: Bleistift, lateinische Kurrent
\newline{}Schnitzler: mit Bleistift nummeriert: »57« und datiert »29/5 95« }
\buchAbdrucke{\weitereDrucke{Arthur Schnitzler, Richard Beer-Hofmann: \emph{Briefwechsel 1891–1931}. Herausgegeben von Konstanze Fliedl. Wien, Zürich: \emph{Europaverlag} 1992, S. 72.} }
\pstart
           \noindent{}{\pb}Lieber Arthur! Dr. G. N.\pwindex{Nobl, Gabor 12.\,10.\,1864 Szombathely – 14.\,3.\,1938 Wien@\textsc{Nobl, Gabor} (12.\,10.\,1864 Szombathely – 14.\,3.\,1938 Wien), \emph{Mediziner, Dermatologe}|pw} hätte
               gestern zu mir ko{\geminationm}en sollen; er war aber weder gestern
               noch heute bei mir: Haben Sie die Güte ihm beiliegende 20 fl zu übermitteln. Er gibt
               Ihnen wol auch Auskunft über den wirklichen Tatbestand, den er ja inzwischen erhoben
                  {\pb}haben dürfte. Meine Adresse
               ist\pend
           
\pstart
           \uline{n. a. Lieut. im k-k. Landw. Inf. Rgmt. Caslau\oindex{Čáslav@\textbf{Čáslav}|pw} – N\textsuperscript{o} 12}. Bitte
               schreiben Sie mir. Grüßen Sie bitte Salten\pwindex{Salten, Felix 6.\,9.\,1869 Budapest – 8.\,10.\,1945 Zürich@\textsc{Salten, Felix} (6.\,9.\,1869 Budapest – 8.\,10.\,1945 Zürich), \emph{Schriftsteller, Journalist, Chefredakteur}|pw},
               auch D\textsuperscript{r.}{ }G. N.\pwindex{Nobl, Gabor 12.\,10.\,1864 Szombathely – 14.\,3.\,1938 Wien@\textsc{Nobl, Gabor} (12.\,10.\,1864 Szombathely – 14.\,3.\,1938 Wien), \emph{Mediziner, Dermatologe}|pw} Empfehlung und besten Dank.\pend
           
\pstart
           {\pb}Mir ist mis.\pend
           
\pstart
           Herzlichst Ihr{\\[\baselineskip]}\spacefill\mbox{Richard.}\pend
           \leftskip=0em{}\selectlanguage{ngerman}\endnumbering\briefempfaengerindex{Schnitzler, Arthur@\textsc{Schnitzler, Arthur}!zzzBeer-Hofmann, Richard@\emph{von Richard Beer-Hofmann}!1895-05-291@{{[}29. 5. 1895{]}}|)be}\mylabel{L00446h}  \newcommand{\dateiname}{L00446}\newcommand{\titel}{Richard Beer-Hofmann an Arthur Schnitzler, [29. 5. 1895]}\newcommand{\editorInnen}{Martin Anton Müller und Gerd-Hermann Susen}%% latex-leseansicht-abspann.tex
%% Abspann für die Leseansicht.
%% Der Schalter \ifkorrekturansicht ist bereits durch den Vorspann gesetzt.

%% latex-abspann.tex
%% Gemeinsamer Abspann für Korrekturansicht und Leseansicht.
%% Setzt den Schalter \ifkorrekturansicht voraus (gesetzt in den
%% einbindenden Dateien latex-korrekturansicht-abspann.tex bzw.
%% latex-leseansicht-abspann.tex).
%% ---------------------------------------------------------------

\normalsize

% Das esempio-Environment wird nur in der Leseansicht benötigt
\ifkorrekturansicht\else
\newenvironment{esempio}[3]%
{
    \vspace{1.5ex}
    \rlap{\underline{#1}}
    \par
    \setlength{\parindent}{0cm}
    \nopagebreak
    \leftskip=#2cm
    \rightskip=#3cm
}
{
    \par
}
\fi

\doendnotes{C}
\bigskip
\vfill

\clearpage

\footnotesize

\ifkorrekturansicht
  \lohead{\textsc{register}}
\fi

% theindex-Environment neu definieren ohne reledmac
\makeatletter
\renewenvironment{theindex}{%
  \ifkorrekturansicht
    \section*{\indexname}%
  \else
    \subsubsection*{Index der erwähnten Entitäten}%
  \fi
  \setlength{\parindent}{0pt}%
  \setlength{\parskip}{0pt plus 0.3pt}%
  \let\item\@idxitem
}{%
  \ifkorrekturansicht\clearpage\fi
}
\makeatother

\IfFileExists{\jobname-pw.ind}{\input{\jobname-pw.ind}}{}

% Quellenangabe nur in der Leseansicht
\ifkorrekturansicht\else
% Fallback-Definitionen, falls die .tex-Datei \titel etc. nicht gesetzt hat
\providecommand{\titel}{}
\providecommand{\editorInnen}{}
\providecommand{\dateiname}{\jobname}

\vspace{3cm}

\vfill

\footnotesize
\textsc{Quelle}: \titel. Herausgegeben von {\editorInnen}. In: \emph{Arthur Schnitzler: Briefwechsel mit Autorinnen und Autoren}.
 Digitale Edition, https://schnitzler-briefe.acdh.oeaw.ac.at/{\dateiname}.html (Stand \today)
\fi

\end{document}


