%% latex-korrekturansicht-vorspann.tex
%% Vorspann für die Korrekturansicht.
%% Lädt die gemeinsame Datei latex-vorspann.tex mit gesetztem Schalter.

\newif\ifkorrekturansicht
\korrekturansichttrue

\input{../tex-inputs/latex-vorspann}


\section[Theodor Herzl an Arthur Schnitzler, 29. 5. 1885]{L03820 Theodor Herzl an Arthur Schnitzler, 29. 5. 1885}
\nopagebreak\mylabel{L03820v}
\rehead{ }\normalsize\beginnumbering\briefempfaengerindex{Schnitzler, Arthur@\textsc{Schnitzler, Arthur}!zzzHerzl, Theodor@\emph{von Theodor Herzl}!1885-05-291@{29. 5. 1885}|(be}
\toendnotes[C]{\smallbreak\pagebreak[2]}\Standort{CUL, Schnitzler, B 39.}
\physDesc{Visitenkarte, 205 Zeichen
\newline{}Handschrift: Bleistift, lateinische Kurrent
\newline{}Ordnung: mit Bleistift von unbekannter Hand nummeriert: »1« }\toendnotes[C]{\smallbreak}
\pstart
           \centering{}{\pb}\textcolor{gray}{\textbf{D\textsuperscript{r}
                Theodor Herzl}}\pend
           
\pstart
           \raggedleft{}{\pb}29/5 85\pend
           
\pstart{}Lieber Herr Doctor!\pend\vspace{0.5em}
\pstart
           Herzlich habe ich mich gefreut und bedaure nur,
      durch verfluchte Amtspflicht
      verhindert zu sein, Ihrer
      \label{K_L03820-1v}\edtext{Promotion}{\lemma{\textnormal{\emph{Promotion}}}\Cendnote{\textnormal{Am 30. 5. 1885 wurde Schnitzler zum Dr. med. promoviert, vgl. A. S.: \emph{Tagebuch}, 5. 6. 1885.}}}\label{K_L03820-1} beizuwohnen.\pend
           
\pstart
           Glückwunsch, Gruss u. Handschlag von Ihrem ergebener{\\[\baselineskip]}\spacefill\mbox{Herzl}\pend
           \leftskip=0em{}\selectlanguage{ngerman}\endnumbering\briefempfaengerindex{Schnitzler, Arthur@\textsc{Schnitzler, Arthur}!zzzHerzl, Theodor@\emph{von Theodor Herzl}!1885-05-291@{29. 5. 1885}|)be}\mylabel{L03820h}
\begin{anhang}
\end{anhang}\normalsize

\doendnotes{C}
\bigskip
\vfill

\clearpage

\footnotesize

\lohead{\textsc{register}}

% Definiere theindex-Environment komplett neu ohne reledmac
\makeatletter
\renewenvironment{theindex}{%
  \section*{\indexname}%
  \setlength{\parindent}{0pt}%
  \setlength{\parskip}{0pt plus 0.3pt}%
  \let\item\@idxitem
}{%
  \clearpage
}
\makeatother

\IfFileExists{\jobname-pw.ind}{\input{\jobname-pw.ind}}{}

\end{document}

      