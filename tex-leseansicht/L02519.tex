%% latex-leseansicht-vorspann.tex
%% Vorspann für die Leseansicht.
%% Lädt die gemeinsame Datei latex-vorspann.tex mit nicht gesetztem Schalter.

\newif\ifkorrekturansicht
\korrekturansichtfalse

\input{../tex-inputs/latex-vorspann}


\section[Christiane Hofmannsthal an Arthur Schnitzler, 13. 8. 1929]{L02519 Christiane Hofmannsthal an Arthur Schnitzler, 13. 8. 1929}
\nopagebreak\mylabel{L02519v}
\rehead{ }\normalsize\beginnumbering\briefempfaengerindex{Schnitzler, Arthur@\textsc{Schnitzler, Arthur}!zzzZimmer, Christiane@\emph{von Christiane Zimmer}!1929-08-131@{13. 8. 1929}|(be}
\toendnotes[C]{\smallbreak\pagebreak[2]}
\correspDesc{Versand  durch Christiane Hofmannsthal am 13. 8. 1929 in Bad Aussee
\newline{}Erhalt  durch Arthur Schnitzler im Zeitraum [14. 8. 1929
                  – 18. 8. 1929?] in Wien}\toendnotes[C]{\smallbreak}
\Standort{CUL, Schnitzler, B 43.}
\physDesc{Brief, 1 Blatt, 1 Seite, 925 Zeichen (Briefpapier mit Trauerrand)
\newline{}Schreibmaschine
\newline{}Handschrift: 1) schwarze Tinte (\noindent{}Unterschrift)\hspace{1em}2) Bleistift, lateinische Kurrent (\noindent{}Fußnote, Fußnotenzeichen)\hspace{1em}
\newline{}Schnitzler: mit rotem Buntstift fünf Unterstreichungen }\toendnotes[C]{\smallbreak}
\pstart
           \raggedleft{}{\pb}Bad Aussee\oindex{Bad Aussee@\textbf{Bad Aussee}, \emph{Hauptstadt}|pw}, am 13. August 1929\pend
           
\pstart{}Lieber Arthur,\pend\vspace{0.5em}
\pstart
           Danke für Deinen lieben Brief, ich erwarte also die Briefe von Frl. Pollack\pwindex{Pollak, Frieda 8.\,12.\,1881 Wien – 13.\,7.\,1937 ebd.@\textsc{Pollak, Frieda} (8.\,12.\,1881 Wien – 13.\,7.\,1937 ebd.), \emph{Sekretärin}|pw} zu bekommen.\pend
           
\pstart
           Wenn wir einen oder den anderen für das neue \label{K_L02519-1v}\edtext{Rundschau\pwindex{neue Rundschau@\emph{Die neue Rundschau}|pw}heft}{\lemma{\textnormal{\emph{Rundschauheft}}}\Cendnote{\textnormal{Im November erschienen erstmals Texte aus dem
                  Nachlass in der \emph{Neuen Deutschen Rundschau}\pwindex{neue Rundschau@\emph{Die neue Rundschau}|pwk} (\emph{Aus dem Nachlass}\pwindex{Hofmannsthal, Hugo von 1.\,2.\,1874 Wien – 15.\,7.\,1929 Rodaun@\textsc{Hofmannsthal, Hugo von} (1.\,2.\,1874 Wien – 15.\,7.\,1929 Rodaun), \emph{Schriftsteller}!Aus dem Nachlass@\strich\emph{Aus dem Nachlass}|pwk}, Jg. 40, H. 11,
                     S. 613–625), aber keine Briefe. Diese folgten erst im April 1930 (\emph{Aus dem Nachlass}\pwindex{Hofmannsthal, Hugo von 1.\,2.\,1874 Wien – 15.\,7.\,1929 Rodaun@\textsc{Hofmannsthal, Hugo von} (1.\,2.\,1874 Wien – 15.\,7.\,1929 Rodaun), \emph{Schriftsteller}!Aus dem Nachlass@\strich\emph{Aus dem Nachlass}|pwk}, Jg. 41, H. 4,
                     S. 497–519).}}}\label{K_L02519-1} für geeignet halten, werden wir ihn Dir vorher zur
               Einsicht übersenden.\pend
           
\pstart
           Bezüglich des Franzosen\pwindex{?? [Franzose, der sich für Hofmannsthal interessiert 1929] @\textsc{?? [Franzose, der sich für Hofmannsthal interessiert 1929]}|pwv}
               weiss ich nicht recht, was da zu empfehlen wäre, als Papas\pwindex{Hofmannsthal, Hugo von 1.\,2.\,1874 Wien – 15.\,7.\,1929 Rodaun@\textsc{Hofmannsthal, Hugo von} (1.\,2.\,1874 Wien – 15.\,7.\,1929 Rodaun), \emph{Schriftsteller}|pwv} Werke selber? Es gibt eine ganz brave
                  französische\oindex{Frankreich@\textbf{Frankreich}|pw}{ }\label{K_L02519-2v}\edtext{Thèse de Doctorat\pwindex{Bianquis, Geneviève 19.\,9.\,1886 Rouen – 24.\,3.\,1972 Antony@\textsc{Bianquis, Geneviève} (19.\,9.\,1886 Rouen – 24.\,3.\,1972 Antony), \emph{Übersetzerin, Literaturhistorikerin}!poésie autrichienne de Hofmannsthal à Rilke@\strich\emph{La poésie autrichienne de Hofmannsthal à Rilke}|pwv}}{\lemma{\textnormal{\emph{Thèse de Doctorat}}}\Cendnote{\textnormal{Geneviève Bianquis\pwindex{Bianquis, Geneviève 19.\,9.\,1886 Rouen – 24.\,3.\,1972 Antony@\textsc{Bianquis, Geneviève} (19.\,9.\,1886 Rouen – 24.\,3.\,1972 Antony), \emph{Übersetzerin, Literaturhistorikerin}|pwk}: \emph{La poésie autrichienne de Hofmannsthal à Rilke}\pwindex{Bianquis, Geneviève 19.\,9.\,1886 Rouen – 24.\,3.\,1972 Antony@\textsc{Bianquis, Geneviève} (19.\,9.\,1886 Rouen – 24.\,3.\,1972 Antony), \emph{Übersetzerin, Literaturhistorikerin}!poésie autrichienne de Hofmannsthal à Rilke@\strich\emph{La poésie autrichienne de Hofmannsthal à Rilke}|pwk}. Paris:
                        \emph{Presses universitaires de France}\orgindex{Presses Universitaires de France@Presses Universitaires de France|pwk}{ }1926.
               }}}\label{K_L02519-2} von einer Mlle. Genevieve
                  Bianquis\pwindex{Bianquis, Geneviève 19.\,9.\,1886 Rouen – 24.\,3.\,1972 Antony@\textsc{Bianquis, Geneviève} (19.\,9.\,1886 Rouen – 24.\,3.\,1972 Antony), \emph{Übersetzerin, Literaturhistorikerin}|pw},\footnote{\noindent{}{[}hs.:{]} auch in Buchform erschienen.} wo alles sehr gewissenhaft, aber weiter nicht hervorragend\introOben{}es\introOben{} drinsteht, und dann gibts wohl nur einzelne Aufsätze von Leuten über
               spezielle Sachen, aber da weiss ich auch nicht, was ich da empfehlen soll. Vielleicht
               fällt Dir noch was Gescheites ein.\pend
           
\pstart
           Hier ist es hässlich und regnerisch wie immer und eher traurig und zuviel bekannte
               Menschen.\pend
           
\pstart
           Sonst geht es soweit ganz gut.\pend
           
\pstart
           Ich freue mich sehr, Dich im Herbst wiederzusehen und Deine Ratschläge
               bekommen zu können.\pend
           
\pstart
           Alles Liebe Deine{\\[\baselineskip]}\spacefill\mbox{{[}hs.:{]} Christiane}\pend
           \leftskip=0em{}\selectlanguage{ngerman}\endnumbering\briefempfaengerindex{Schnitzler, Arthur@\textsc{Schnitzler, Arthur}!zzzZimmer, Christiane@\emph{von Christiane Zimmer}!1929-08-131@{13. 8. 1929}|)be}\mylabel{L02519h}  \newcommand{\dateiname}{L02519}\newcommand{\titel}{Christiane Hofmannsthal an Arthur Schnitzler, 13. 8. 1929}\newcommand{\editorInnen}{Martin Anton Müller und Gerd-Hermann Susen}%% latex-leseansicht-abspann.tex
%% Abspann für die Leseansicht.
%% Der Schalter \ifkorrekturansicht ist bereits durch den Vorspann gesetzt.

%% latex-abspann.tex
%% Gemeinsamer Abspann für Korrekturansicht und Leseansicht.
%% Setzt den Schalter \ifkorrekturansicht voraus (gesetzt in den
%% einbindenden Dateien latex-korrekturansicht-abspann.tex bzw.
%% latex-leseansicht-abspann.tex).
%% ---------------------------------------------------------------

\normalsize

% Das esempio-Environment wird nur in der Leseansicht benötigt
\ifkorrekturansicht\else
\newenvironment{esempio}[3]%
{
    \vspace{1.5ex}
    \rlap{\underline{#1}}
    \par
    \setlength{\parindent}{0cm}
    \nopagebreak
    \leftskip=#2cm
    \rightskip=#3cm
}
{
    \par
}
\fi

\doendnotes{C}
\bigskip
\vfill

\clearpage

\footnotesize

\ifkorrekturansicht
  \lohead{\textsc{register}}
\fi

% theindex-Environment neu definieren ohne reledmac
\makeatletter
\renewenvironment{theindex}{%
  \ifkorrekturansicht
    \section*{\indexname}%
  \else
    \subsubsection*{Index der erwähnten Entitäten}%
  \fi
  \setlength{\parindent}{0pt}%
  \setlength{\parskip}{0pt plus 0.3pt}%
  \let\item\@idxitem
}{%
  \ifkorrekturansicht\clearpage\fi
}
\makeatother

\IfFileExists{\jobname-pw.ind}{\input{\jobname-pw.ind}}{}

% Quellenangabe nur in der Leseansicht
\ifkorrekturansicht\else
% Fallback-Definitionen, falls die .tex-Datei \titel etc. nicht gesetzt hat
\providecommand{\titel}{}
\providecommand{\editorInnen}{}
\providecommand{\dateiname}{\jobname}

\vspace{3cm}

\vfill

\footnotesize
\textsc{Quelle}: \titel. Herausgegeben von {\editorInnen}. In: \emph{Arthur Schnitzler: Briefwechsel mit Autorinnen und Autoren}.
 Digitale Edition, https://schnitzler-briefe.acdh.oeaw.ac.at/{\dateiname}.html (Stand \today)
\fi

\end{document}


