%% latex-leseansicht-vorspann.tex
%% Vorspann für die Leseansicht.
%% Lädt die gemeinsame Datei latex-vorspann.tex mit nicht gesetztem Schalter.

\newif\ifkorrekturansicht
\korrekturansichtfalse

\input{../tex-inputs/latex-vorspann}


\section[ Paul Goldmann an Arthur Schnitzler, 29. 6. [1896]]{L02779 Paul Goldmann an Arthur Schnitzler,  29. 6. [1896]}
\nopagebreak\mylabel{L02779v}
\rehead{ }\normalsize\beginnumbering\briefempfaengerindex{Schnitzler, Arthur@\textsc{Schnitzler, Arthur}!zzzGoldmann, Paul@\emph{von Paul Goldmann}!1896-06-292@{29. 6. [1896]}|(be}
\toendnotes[C]{\smallbreak\pagebreak[2]}
\correspDesc{Versand  durch Paul Goldmann am 29. 6. [1896] in Paris
\newline{}Erhalt  durch Arthur Schnitzler im Zeitraum [30. 6. 1896
                  – 4. 7. 1896?] in Wien}\toendnotes[C]{\smallbreak}
\Standort{DLA, A:Schnitzler, HS.NZ85.1.3166.}
\physDesc{Brief, 1 Blatt, 4 Seiten, 1817 Zeichen
\newline{}Handschrift: blaue Tinte, deutsche Kurrent
\newline{}Schnitzler: 1) mit Bleistift das Jahr »96« vermerkt  2) mit rotem Buntstift eine Unterstreichung}\toendnotes[C]{\smallbreak}
\pstart
           {\pb}\textcolor{gray}{\textbf{\textbf{Frankfurter Zeitung\orgindex{Frankfurter Zeitung@Frankfurter Zeitung|pw}}}}\pend
           
\pstart
           \textcolor{gray}{\textbf{(\begin{otherlanguage}{french}Gazette de Francfort\end{otherlanguage}\orgindex{Frankfurter Zeitung@Frankfurter Zeitung|pw}).}}\pend
           
\pstart
           \textcolor{gray}{\textbf{\textbf{\begin{otherlanguage}{french}Fondateur M.\end{otherlanguage}{ }L. Sonnemann\pwindex{Sonnemann, Leopold 29.\,10.\,1831 Höchberg – 30.\,10.\,1909 Frankfurt am Main@\textsc{Sonnemann, Leopold} (29.\,10.\,1831 Höchberg – 30.\,10.\,1909 Frankfurt am Main), \emph{Journalist, Herausgeber}|pw}.}}}\pend
           
\pstart
           \begin{otherlanguage}{french}\textcolor{gray}{\textbf{Journal\pwindex{Frankfurter Zeitung@\emph{Frankfurter Zeitung}|pwv} politique,
                        financier,}}\end{otherlanguage}\pend
           
\pstart
           \begin{otherlanguage}{french}\textcolor{gray}{\textbf{commercial et littéraire.}}\end{otherlanguage}\pend
           
\pstart
           \begin{otherlanguage}{french}\textcolor{gray}{\textbf{\textbf{Paraissant trois fois par jour.}}}\end{otherlanguage}\hfill \textsc{Paris\oindex{Paris@\textbf{Paris}, \emph{Hauptstadt}|pw}}, 29. Juni.\pend
           
\pstart
           \begin{otherlanguage}{french}\textcolor{gray}{\textbf{\textbf{Bureau à Paris\oindex{Paris@\textbf{Paris}, \emph{Hauptstadt}|pw}}}}\end{otherlanguage}\pend
           
\pstart
           \begin{otherlanguage}{french}\textcolor{gray}{\textbf{\textbf{24. Rue Feydeau\oindex{rue Feydeau@\textbf{rue Feydeau}, \emph{Straße}|pw}.}}}\end{otherlanguage}\pend
           
\pstart\center{}Mein lieber Freund,\pend\vspace{0.5em}
\pstart
           Was{ }ſoll man gegen ein viermal unterſtrichenes »durchaus« machen? Gar{ }ſo »durchaus« \substVorne{}\textsuperscript{\textcolor{gray}{×}\-\textcolor{gray}{×}\-\textcolor{gray}{×}\-\textcolor{gray}{×}}\substDazwischen{}bin\substHinten{} ich ja nicht gegen Dänemark\oindex{Dänemark@\textbf{Dänemark}|pw}
               eingenommen. Ich habe nur nicht die Mittel, um hinzufahren, und nicht die mindeſte
               Luſt, dortzubleiben. Da Du aber meinſt, daß dies{ }ſchwächliche Gründe{ }ſind,{ }ſo haſt Du
               jedenfalls Recht und ich werde \label{K_L02779-1v}\edtext{hinkommen}{\lemma{\textnormal{\emph{hinkommen}}}\Cendnote{\textnormal{Siehe XXXX Auszeichnungsfehler: Dokument L02772 nicht gefunden.
               }}}\label{K_L02779-1}. Alſo, wenn ich bis Anfang {\pb}Auguſt nicht ganz bankrott bin (was möglich iſt) und wenn
               nichts Anderes Wichtiges dazwiſchen kommt,{ }ſo treffen wir uns zwiſchen dem 5. u. 10. Auguſt in \textsc{Scottsborg\oindex{Skodsborg@\textbf{Skodsborg}|pw}}, welcher Ort\oindex{Skodsborg@\textbf{Skodsborg}|pwv} nach Deinen
               Schilderungen{ }ſo billig iſt, daß man ihn{ }ſchon wegen{ }ſeiner Billigkeit aufſuchen
               müßte. Ich kehre{ }ſicher mit großen Erſparniſſen heim. Andere Leute gehen auf die
               Goldfelder von \textsc{Transvaal\oindex{Transvaal@\textbf{Transvaal}|pw}}, ich werde nach \textsc{Scottsborg\oindex{Skodsborg@\textbf{Skodsborg}|pw}} gehen. Gott allein weiß, wer Euch\pwindex{Beer-Hofmann, Richard 11.\,7.\,1866 Wien – 26.\,9.\,1945 New York City@\textsc{Beer-Hofmann, Richard} (11.\,7.\,1866 Wien – 26.\,9.\,1945 New York City), \emph{Schriftsteller}|pwv} dieſe dän\oindex{Dänemark@\textbf{Dänemark}|pwv}iſche Idee in den Kopf geſetzt hat! {\pb}Europa\oindex{Europa@\textbf{Europa}|pw} iſt{ }ſo{ }ſchön und es gibt{ }ſoviel Herrliches
               zu{ }ſehen. Muß man alſo gerade in ein Land\oindex{Dänemark@\textbf{Dänemark}|pwv} gehen, in dem es \strikeout{a\textcolor{gray}{b}ſ\textcolor{gray}{o}} abſolut nichts gibt: weder Gebirge, noch Kunſt, noch Vergangenheit, –
               höchſtens Meer, aber auch das wird vielleicht ein Schwindel{ }ſein und ich werde es
               erſt glauben, wenn ich es geſehen habe.\pend
           
\pstart
           \label{K_L02779-2v}\edtext{\begin{otherlanguage}{french}\textsc{Enfin}\end{otherlanguage}}{\lemma{\textnormal{\emph{Enfin}}}\Cendnote{\textnormal{französisch: kurzum}}}\label{K_L02779-2}, ich komme
               nach Dänemark\oindex{Dänemark@\textbf{Dänemark}|pw}. Ihr\pwindex{Beer-Hofmann, Richard 11.\,7.\,1866 Wien – 26.\,9.\,1945 New York City@\textsc{Beer-Hofmann, Richard} (11.\,7.\,1866 Wien – 26.\,9.\,1945 New York City), \emph{Schriftsteller}|pwv} werdet mich hoffentlich über Eure
               Unterwegs-Adreſſen auf dem Laufenden halten. \textsc{Richard\pwindex{Beer-Hofmann, Richard 11.\,7.\,1866 Wien – 26.\,9.\,1945 New York City@\textsc{Beer-Hofmann, Richard} (11.\,7.\,1866 Wien – 26.\,9.\,1945 New York City), \emph{Schriftsteller}|pw}} wird{ }ſich auch zu einer Correſpondenzkarte einmal entſchließen {\pb}müſſen; aber ich glaube, die dän\oindex{Dänemark@\textbf{Dänemark}|pwv}iſchen Poſtkarten{ }ſind kleiner als die
                  öſterreich\oindex{Österreich@\textbf{Österreich}|pwv}iſchen, was
               wieder ein Vortheil dieſes{ }ſchönen Land\oindex{Dänemark@\textbf{Dänemark}|pwv}es iſt.\pend
           
\pstart
           Du aber, mein lieber Freund, reiſe glücklich. Ich wünſche Dir von Herzen alles Gute
               auf den Weg.\pend
           
\pstart
           Die Zeitungen, die Du auf dem Zettel angegeben, kann ich Dir erſt morgen{ }ſchicken, da weite Wege zu ihrer Beſorgung zu
               machen{ }ſind. Gib alſo \label{K_L02779-3v}\edtext{\begin{otherlanguage}{french}Ordre\end{otherlanguage}}{\lemma{\textnormal{\emph{Ordre}}}\Cendnote{\textnormal{französisch: Anordnung}}}\label{K_L02779-3}, daß{ }ſie
               Dir nachgeſandt werden.\pend
           
\pstart
           Von Herzen und in Treue{\\[\baselineskip]}Dein{\\[\baselineskip]}\spacefill\mbox{Paul Goldmann.}\pend
           \leftskip=0em{}
\pstart
           \noindent{}Schick’ mir, bitte, das Buch\pwindex{Altenberg, Peter 9.\,3.\,1859 Wien – 8.\,1.\,1919 ebd.@\textsc{Altenberg, Peter} (9.\,3.\,1859 Wien – 8.\,1.\,1919 ebd.), \emph{Schriftsteller}!Wie ich es sehe@\strich\emph{Wie ich es sehe}|pwv} von \textsc{Altenberg\pwindex{Altenberg, Peter 9.\,3.\,1859 Wien – 8.\,1.\,1919 ebd.@\textsc{Altenberg, Peter} (9.\,3.\,1859 Wien – 8.\,1.\,1919 ebd.), \emph{Schriftsteller}|pw}}.\pend
           \selectlanguage{ngerman}\endnumbering\briefempfaengerindex{Schnitzler, Arthur@\textsc{Schnitzler, Arthur}!zzzGoldmann, Paul@\emph{von Paul Goldmann}!1896-06-292@{29. 6. [1896]}|)be}\mylabel{L02779h}  \newcommand{\dateiname}{L02779}\newcommand{\titel}{Paul Goldmann an Arthur Schnitzler, 29. 6. [1896]}\newcommand{\editorInnen}{Martin Anton Müller und Laura Untner}%% latex-leseansicht-abspann.tex
%% Abspann für die Leseansicht.
%% Der Schalter \ifkorrekturansicht ist bereits durch den Vorspann gesetzt.

%% latex-abspann.tex
%% Gemeinsamer Abspann für Korrekturansicht und Leseansicht.
%% Setzt den Schalter \ifkorrekturansicht voraus (gesetzt in den
%% einbindenden Dateien latex-korrekturansicht-abspann.tex bzw.
%% latex-leseansicht-abspann.tex).
%% ---------------------------------------------------------------

\normalsize

% Das esempio-Environment wird nur in der Leseansicht benötigt
\ifkorrekturansicht\else
\newenvironment{esempio}[3]%
{
    \vspace{1.5ex}
    \rlap{\underline{#1}}
    \par
    \setlength{\parindent}{0cm}
    \nopagebreak
    \leftskip=#2cm
    \rightskip=#3cm
}
{
    \par
}
\fi

\doendnotes{C}
\bigskip
\vfill

\clearpage

\footnotesize

\ifkorrekturansicht
  \lohead{\textsc{register}}
\fi

% theindex-Environment neu definieren ohne reledmac
\makeatletter
\renewenvironment{theindex}{%
  \ifkorrekturansicht
    \section*{\indexname}%
  \else
    \subsubsection*{Index der erwähnten Entitäten}%
  \fi
  \setlength{\parindent}{0pt}%
  \setlength{\parskip}{0pt plus 0.3pt}%
  \let\item\@idxitem
}{%
  \ifkorrekturansicht\clearpage\fi
}
\makeatother

\IfFileExists{\jobname-pw.ind}{\input{\jobname-pw.ind}}{}

% Quellenangabe nur in der Leseansicht
\ifkorrekturansicht\else
% Fallback-Definitionen, falls die .tex-Datei \titel etc. nicht gesetzt hat
\providecommand{\titel}{}
\providecommand{\editorInnen}{}
\providecommand{\dateiname}{\jobname}

\vspace{3cm}

\vfill

\footnotesize
\textsc{Quelle}: \titel. Herausgegeben von {\editorInnen}. In: \emph{Arthur Schnitzler: Briefwechsel mit Autorinnen und Autoren}.
 Digitale Edition, https://schnitzler-briefe.acdh.oeaw.ac.at/{\dateiname}.html (Stand \today)
\fi

\end{document}


