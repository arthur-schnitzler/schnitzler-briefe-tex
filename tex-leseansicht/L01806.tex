%% latex-korrekturansicht-vorspann.tex
%% Vorspann für die Korrekturansicht.
%% Lädt die gemeinsame Datei latex-vorspann.tex mit gesetztem Schalter.

\newif\ifkorrekturansicht
\korrekturansichttrue

\input{../tex-inputs/latex-vorspann}


\section[Arthur Schnitzler an Richard Beer-Hofmann, {[}17.? 11. 1908{]}]{L01806 Arthur Schnitzler an Richard Beer-Hofmann, {[}17.? 11. 1908{]}}
\nopagebreak\mylabel{L01806v}
\rehead{ }\normalsize\beginnumbering\briefempfaengerindex{Beer-Hofmann, Richard@\textsc{Beer-Hofmann, Richard}!zzzSchnitzler, Arthur@\emph{von Arthur Schnitzler}!1908-11-172@{{[}17.? 11. 1908{]}}|(be}
\toendnotes[C]{\smallbreak\pagebreak[2]}\Standort{YCGL, MSS 31.}
\physDesc{Briefkarte, , Umschlag, 260 Zeichen
\newline{}Handschrift: Bleistift, deutsche Kurrent
\newline{}Versand: ohne postalischen Übermittlungsvermerk 
\newline{}Beer-Hofmann: auf der Rückseite des Umschlags mit blauem Buntstift datiert: »19/XI 08«, wobei es sich um den Empfang oder eine (nicht
                                 überlieferte) Beantwortung handeln könnte }
\buchAbdrucke{\weitereDrucke{Arthur Schnitzler, Richard Beer-Hofmann: \emph{Briefwechsel 1891–1931}. Wien, Zürich: \emph{Europaverlag} 1992, S. 191.} }\toendnotes[C]{\smallbreak}\pstart{}{\pb}\textcolor{gray}{\textbf{Dr. Arthur Schnitzler}}\pend{}\pstart{}\textcolor{gray}{\textbf{Wien XVIII. Spoettelgasse 7\oindex{Edmund-Weiss-Gasse 7@\textbf{Edmund-Weiß-Gasse 7}, \emph{Wohngebäude (K.WHS)}|pw}.}}\pend{}{\bigskip}\pstart{}{\pb}\textsc{Dr. Richard Beer Hofmann}\pend{}\pstart{}\textsc{Wien XVIII}\oindex{XVIII., Waehring@\textbf{XVIII., Währing}, \emph{A.ADM3}|pw}\pend{}\pstart{}\textsc{Hasenauerstr} 59\oindex{Hasenauerstrasse 59@\textbf{Hasenauerstraße 59}, \emph{Wohngebäude (K.WHS)}|pw}\pend{}{\bigskip}\vspace{1em}
\pstart
           \noindent{}{\pb}\textcolor{gray}{\textbf{Dr. Arthur Schnitzler}}{\\}\textcolor{gray}{\textbf{Wien XVIII. Spoettelgasse 7\oindex{Edmund-Weiss-Gasse 7@\textbf{Edmund-Weiß-Gasse 7}, \emph{Wohngebäude (K.WHS)}|pw}.}}\pend
           
\pstart
           lieber Richard, hier der \textsc{Tantris}\pwindex{Tantris der Narr. Drama in fuenf Aufzuegen@\emph{Tantris der Narr. Drama in fünf Aufzügen}|pw}. Bringen Sie ihn bitte \label{K_L01806-1v}\edtext{morgen}{\lemma{\textnormal{\emph{morgen}}}\Cendnote{\textnormal{Das deutet darauf, dass das
                  Korrespondenzstück zwei Tage vor dem Datumsvermerk von Beer-Hofmann\pwindex{Beer-Hofmann, Richard 1866-07-11 – 1945-09-26@\textsc{Beer-Hofmann, Richard} (1866-07-11 – 1945-09-26), \emph{Schriftsteller/Schriftstellerin}|pwk} anzusiedeln ist, da am 18. 11. 1908 die
                  Generalprobe von \emph{Tantris}\pwindex{Tantris der Narr. Drama in fuenf Aufzuegen@\emph{Tantris der Narr. Drama in fünf Aufzügen}|pwk} stattfand. Als
                  weiteres Indiz antwortet die Korrespondenzkarte auf ein mündliches Gespräch vom
                  selben Tag.}}}\label{K_L01806-1} gleich mit, auf dſs er eventuell {\pb}zur Hand wäre.\pend
           
\pstart
           Mir fiel noch als \label{K_L01806-2v}\edtext{Ma{\geminationn} der Wiſſenſchaft}{\lemma{\textnormal{\emph{Mann der Wiſſenſchaft}}}\Cendnote{\textnormal{Beer-Hofmann\pwindex{Beer-Hofmann, Richard 1866-07-11 – 1945-09-26@\textsc{Beer-Hofmann, Richard} (1866-07-11 – 1945-09-26), \emph{Schriftsteller/Schriftstellerin}|pwk} sammelte Unterstützer für
                  einen Aufruf für ein jüdisches Studentenheim.}}}\label{K_L01806-2} Hofrat Prof \textsc{Oser}\pwindex{Oser, Leopold 24. 7. 1839 – 22. 8. 1910@\textsc{Oser, Leopold} (24. 7. 1839 – 22. 8. 1910), \emph{Mediziner/Medizinerin}|pw} ein; als Großinduſtrieller \textsc{Gutma{\geminationn} v Gelse}\pwindex{Gutmann-Gelse, Edmund von 03.03.1841 – 17.01.1918@\textsc{Gutmann-Gelse, Edmund von} (03.03.1841 – 17.01.1918), \emph{Industrieller/Industrielle}|pw}!\pend
           \pstart Herzlichſt Ihr \spacefill\mbox{A.}\pend{}\selectlanguage{ngerman}\endnumbering\briefempfaengerindex{Beer-Hofmann, Richard@\textsc{Beer-Hofmann, Richard}!zzzSchnitzler, Arthur@\emph{von Arthur Schnitzler}!1908-11-172@{{[}17.? 11. 1908{]}}|)be}\mylabel{L01806h}  \normalsize

\doendnotes{C}
\bigskip
\vfill

\clearpage

\footnotesize

\lohead{\textsc{register}}

% Definiere theindex-Environment komplett neu ohne reledmac
\makeatletter
\renewenvironment{theindex}{%
  \section*{\indexname}%
  \setlength{\parindent}{0pt}%
  \setlength{\parskip}{0pt plus 0.3pt}%
  \let\item\@idxitem
}{%
  \clearpage
}
\makeatother

\IfFileExists{\jobname-pw.ind}{\input{\jobname-pw.ind}}{}

\end{document}

      