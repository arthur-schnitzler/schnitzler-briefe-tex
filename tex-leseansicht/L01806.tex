\input{../tex-inputs/latex-pdf-vorspann}
\begin{center}
            \textcolor{red}{ENTWURF. ENTZIFFERUNG NOCH NICHT KORREKTURGELESEN}
                      \end{center}
            
               \section[Arthur Schnitzler an Richard Beer-Hofmann, {[}17.? 11. 1908{]}]{ Arthur Schnitzler an Richard Beer-Hofmann, {[}17.? 11. 1908{]}}\nopagebreak\mylabel{v}\rehead{ }\begin{ledgroupsized}[t]{13cm}\normalsize\beginnumbering\briefempfaengerindex{Beer-Hofmann, Richard@\textsc{Beer-Hofmann, Richard}!zzzSchnitzler, Arthur@\emph{von Arthur Schnitzler}!1908-11-172@{{[}17.? 11. 1908{]}}|(be} \toendnotes[C]{\smallbreak\pagebreak[2]} \Standort{YCGL, MSS 31.}
\physDesc{Briefkarte, Umschlag
\newline{}Handschrift: Bleistift, deutsche Kurrent\newline{}Versand: ohne postalischen Übermittlungsvermerk 
\newline{}Beer-Hofmann: auf der Rückseite des Umschlags mit blauem Buntstift datiert: »19/XI 08«, wobei es sich um den Empfang oder eine (nicht überlieferte) Beantwortung
            handeln könnte }\buchAbdrucke{\weitereDrucke{Arthur Schnitzler, Richard Beer-Hofmann: \emph{Briefwechsel 1891–1931}. Hg. Konstanze Fliedl. Wien, Zürich: \emph{Europaverlag} 1992, S. 191.} }\toendnotes[C]{\smallbreak}\pstart{}{\pb}\textcolor{gray}{\textbf{Dr. Arthur Schnitzler}}\pend{}\pstart{}\textcolor{gray}{\textbf{Wien XVIII. Spoettelgasse 7\oindex{Edmund-Weiss-Gasse@\textbf{Edmund-Weiß-Gasse}|pw}.}}\pend{}{\bigskip}\pstart{}{\pb}\textsc{Dr. Richard Beer Hofmann}\pend{}\pstart{}\textsc{Wien XVIII}\oindex{XVIII., Waehring@\textbf{XVIII., Währing}|pw}\pend{}\pstart{}\textsc{Hasenauerstr} 59\oindex{Hasenauerstrasse@\textbf{Hasenauerstraße}|pw}\pend{}{\bigskip}\pstart
           \noindent{}{\pb}\textcolor{gray}{\textbf{Dr. Arthur Schnitzler}}{\\}\textcolor{gray}{\textbf{Wien XVIII. Spoettelgasse 7\oindex{Edmund-Weiss-Gasse@\textbf{Edmund-Weiß-Gasse}|pw}.}}\pend
           \pstart
           lieber Richard, hier der \textsc{Tantris}\pwindex{\textcolor{red}{\textsuperscript{XXXX1 indx}}!Tantris der Narr. Drama in fuenf Aufzuegen1907@\strich\emph{Tantris der Narr. Drama in fünf Aufzügen} {[}1907{]}|pw}. Bringen Sie ihn bitte \label{K_L01806-1v}\edtext{morgen}{\lemma{\textnormal{\emph{morgen}}}\Cendnote{\textnormal{Das deutet darauf, dass das
                  Korrespondenzstück zwei Tage vor dem Datumsvermerk von Beer-Hofmann\pwindex{Beer-Hofmann, Richard 11.07.1866 – 26.09.1945@\textsc{Beer-Hofmann, Richard} (11.07.1866 – 26.09.1945), \emph{Schriftsteller}|pwk} anzusiedeln ist, da am 18. 11. 1908 die Generalprobe von \emph{Tantris}\pwindex{\textcolor{red}{\textsuperscript{XXXX1 indx}}!Tantris der Narr. Drama in fuenf Aufzuegen1907@\strich\emph{Tantris der Narr. Drama in fünf Aufzügen} {[}1907{]}|pwk} stattfand. Als weiteres Indiz antwortet
                  die Korrespondenzkarte auf ein mündliches Gespräch vom selben Tag.}}}\label{K_L01806-1h} gleich
               mit, auf dſs er eventuell {\pb}zur Hand wäre.\pend
           \pstart
           Mir fiel noch als \label{K_L01806_1v}\edtext{Ma{\geminationn} der Wiſſenſchaft}{\lemma{\textnormal{\emph{Ma der Wiſſenſchaft}}}\Cendnote{\textnormal{Beer-Hofmann\pwindex{Beer-Hofmann, Richard 11.07.1866 – 26.09.1945@\textsc{Beer-Hofmann, Richard} (11.07.1866 – 26.09.1945), \emph{Schriftsteller}|pwk} sammelte Unterstützer für einen
                  Aufruf für ein jüdisches Studentenheim.}}}\label{K_L01806_1h} Hofrat Prof \textsc{Oser}\pwindex{Oser, Leopold 24. 7. 1839 – 22. 8. 1910@\textsc{Oser, Leopold} (24. 7. 1839 – 22. 8. 1910), \emph{Mediziner}|pw} ein; als Großinduſtrieller \textsc{Gutma{\geminationn} v Gelse}\pwindex{Gutmann-Gelse, Edmund von 03.03.1841 – 17.01.1918@\textsc{Gutmann-Gelse, Edmund von} (03.03.1841 – 17.01.1918), \emph{Industrieller}|pw}!\pend
           \pstart Herzlichſt Ihr \spacefill\mbox{A.}\pend{}\endnumbering\briefempfaengerindex{Beer-Hofmann, Richard@\textsc{Beer-Hofmann, Richard}!zzzSchnitzler, Arthur@\emph{von Arthur Schnitzler}!1908-11-172@{{[}17.? 11. 1908{]}}|)be}\mylabel{h}\end{ledgroupsized}  \newcommand{\dateiname}{L01806}\newcommand{\titel}{Arthur Schnitzler an Richard Beer-Hofmann, [17.? 11. 1908]}\newcommand{\editorInnen}{Martin Anton Müller und Gerd-Hermann Susen}\input{../tex-inputs/latex-pdf-abspann}
      