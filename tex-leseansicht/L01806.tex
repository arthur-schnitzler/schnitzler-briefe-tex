%% latex-leseansicht-vorspann.tex
%% Vorspann für die Leseansicht.
%% Lädt die gemeinsame Datei latex-vorspann.tex mit nicht gesetztem Schalter.

\newif\ifkorrekturansicht
\korrekturansichtfalse

\input{../tex-inputs/latex-vorspann}


\section[Arthur Schnitzler an Richard Beer-Hofmann, {[}17.? 11. 1908{]}]{L01806 Arthur Schnitzler an Richard Beer-Hofmann, {[}17.? 11. 1908{]}}
\nopagebreak\mylabel{L01806v}
\rehead{ }\normalsize\beginnumbering\briefempfaengerindex{Beer-Hofmann, Richard@\textsc{Beer-Hofmann, Richard}!zzzSchnitzler, Arthur@\emph{von Arthur Schnitzler}!1908-11-172@{{[}17.? 11. 1908{]}}|(be}
\toendnotes[C]{\smallbreak\pagebreak[2]}
\correspDesc{Versand  durch Arthur Schnitzler am [17.? 11. 1908] in Wien
\newline{}Erhalt  durch Richard Beer-Hofmann im Zeitraum [17. 11. 1908 – 21. 11. 1908?] in Wien}\toendnotes[C]{\smallbreak}
\Standort{YCGL, MSS 31.}
\physDesc{Briefkarte, , Kuvert, 260 Zeichen
\newline{}Handschrift: Bleistift, deutsche Kurrent
\newline{}Versand: ohne postalischen Übermittlungsvermerk 
\newline{}Beer-Hofmann: auf der Rückseite des Umschlags mit blauem Buntstift datiert: »19/XI 08«, wobei es sich um den Empfang oder eine (nicht
                                 überlieferte) Beantwortung handeln könnte }
\buchAbdrucke{\weitereDrucke{Arthur Schnitzler, Richard Beer-Hofmann: \emph{Briefwechsel 1891–1931}. Herausgegeben von Konstanze Fliedl. Wien, Zürich: \emph{Europaverlag} 1992, S. 191.} }\toendnotes[C]{\smallbreak}\pstart{}{\pb}\textcolor{gray}{\textbf{Dr. Arthur Schnitzler}}\pend{}\pstart{}\textcolor{gray}{\textbf{Wien XVIII. Spoettelgasse 7\oindex{Wien@\textbf{Wien}!XVIII., Währing@\textbf{XVIII., Währing}!Edmund-Weiß-Gasse 7@\textbf{Edmund-Weiß-Gasse 7}, \emph{Wohngebäude}|pw}.}}\pend{}{\bigskip}\pstart{}{\pb}\textsc{Dr. Richard Beer Hofmann}\pend{}\pstart{}\textsc{Wien XVIII}\oindex{XVIII., Währing@\textbf{XVIII., Währing}, \emph{Verwaltungsgebiet}|pw}\pend{}\pstart{}\textsc{Hasenauerstr} 59\oindex{Wien@\textbf{Wien}!XVIII., Währing@\textbf{XVIII., Währing}!Hasenauerstraße 59@\textbf{Hasenauerstraße 59}, \emph{Wohngebäude}|pw}\pend{}{\bigskip}\vspace{1em}
\pstart
           \noindent{}{\pb}\textcolor{gray}{\textbf{Dr. Arthur Schnitzler}}{\\}\textcolor{gray}{\textbf{Wien XVIII. Spoettelgasse 7\oindex{Wien@\textbf{Wien}!XVIII., Währing@\textbf{XVIII., Währing}!Edmund-Weiß-Gasse 7@\textbf{Edmund-Weiß-Gasse 7}, \emph{Wohngebäude}|pw}.}}\pend
           
\pstart
           lieber Richard, hier der \textsc{Tantris}\pwindex{\textcolor{red}{\textsuperscript{XXXX indx1}}!Tantris der Narr. Drama in fünf Aufzügen@\strich\emph{Tantris der Narr. Drama in fünf Aufzügen}|pw}. Bringen Sie ihn bitte \label{K_L01806-1v}\edtext{morgen}{\lemma{\textnormal{\emph{morgen}}}\Cendnote{\textnormal{Das deutet darauf, dass das
                  Korrespondenzstück zwei Tage vor dem Datumsvermerk von Beer-Hofmann\pwindex{Beer-Hofmann, Richard 11.\,7.\,1866 Wien – 26.\,9.\,1945 New York City@\textsc{Beer-Hofmann, Richard} (11.\,7.\,1866 Wien – 26.\,9.\,1945 New York City), \emph{Schriftsteller}|pwk} anzusiedeln ist, da am 18. 11. 1908 die
                  Generalprobe von \emph{Tantris}\pwindex{\textcolor{red}{\textsuperscript{XXXX indx1}}!Tantris der Narr. Drama in fünf Aufzügen@\strich\emph{Tantris der Narr. Drama in fünf Aufzügen}|pwk} stattfand. Als
                  weiteres Indiz antwortet die Korrespondenzkarte auf ein mündliches Gespräch vom
                  selben Tag.}}}\label{K_L01806-1} gleich mit, auf dſs er eventuell {\pb}zur Hand wäre.\pend
           
\pstart
           Mir fiel noch als \label{K_L01806-2v}\edtext{Ma{\geminationn} der Wiſſenſchaft}{\lemma{\textnormal{\emph{Mann der Wissenschaft}}}\Cendnote{\textnormal{Beer-Hofmann\pwindex{Beer-Hofmann, Richard 11.\,7.\,1866 Wien – 26.\,9.\,1945 New York City@\textsc{Beer-Hofmann, Richard} (11.\,7.\,1866 Wien – 26.\,9.\,1945 New York City), \emph{Schriftsteller}|pwk} sammelte Unterstützer für
                  einen Aufruf für ein jüdisches Studentenheim.}}}\label{K_L01806-2} Hofrat Prof \textsc{Oser}\pwindex{Oser, Leopold 24.\,7.\,1839 Mikulov – 22.\,8.\,1910 Gainfarn@\textsc{Oser, Leopold} (24.\,7.\,1839 Mikulov – 22.\,8.\,1910 Gainfarn), \emph{Mediziner}|pw} ein; als Großinduſtrieller \textsc{Gutma{\geminationn} v Gelse}\pwindex{Gutmann-Gelse, Edmund von 3.\,3.\,1841 Nagykanizsa – 17.\,1.\,1918 Belišće@\textsc{Gutmann-Gelse, Edmund von} (3.\,3.\,1841 Nagykanizsa – 17.\,1.\,1918 Belišće), \emph{Industrieller}|pw}!\pend
           \pstart Herzlichſt Ihr \spacefill\mbox{A.}\pend{}\selectlanguage{ngerman}\endnumbering\briefempfaengerindex{Beer-Hofmann, Richard@\textsc{Beer-Hofmann, Richard}!zzzSchnitzler, Arthur@\emph{von Arthur Schnitzler}!1908-11-172@{{[}17.? 11. 1908{]}}|)be}\mylabel{L01806h}  \newcommand{\dateiname}{L01806}\newcommand{\titel}{Arthur Schnitzler an Richard Beer-Hofmann, [17.? 11. 1908]}\newcommand{\editorInnen}{Martin Anton Müller und Gerd-Hermann Susen}%% latex-leseansicht-abspann.tex
%% Abspann für die Leseansicht.
%% Der Schalter \ifkorrekturansicht ist bereits durch den Vorspann gesetzt.

%% latex-abspann.tex
%% Gemeinsamer Abspann für Korrekturansicht und Leseansicht.
%% Setzt den Schalter \ifkorrekturansicht voraus (gesetzt in den
%% einbindenden Dateien latex-korrekturansicht-abspann.tex bzw.
%% latex-leseansicht-abspann.tex).
%% ---------------------------------------------------------------

\normalsize

% Das esempio-Environment wird nur in der Leseansicht benötigt
\ifkorrekturansicht\else
\newenvironment{esempio}[3]%
{
    \vspace{1.5ex}
    \rlap{\underline{#1}}
    \par
    \setlength{\parindent}{0cm}
    \nopagebreak
    \leftskip=#2cm
    \rightskip=#3cm
}
{
    \par
}
\fi

\doendnotes{C}
\bigskip
\vfill

\clearpage

\footnotesize

\ifkorrekturansicht
  \lohead{\textsc{register}}
\fi

% theindex-Environment neu definieren ohne reledmac
\makeatletter
\renewenvironment{theindex}{%
  \ifkorrekturansicht
    \section*{\indexname}%
  \else
    \subsubsection*{Index der erwähnten Entitäten}%
  \fi
  \setlength{\parindent}{0pt}%
  \setlength{\parskip}{0pt plus 0.3pt}%
  \let\item\@idxitem
}{%
  \ifkorrekturansicht\clearpage\fi
}
\makeatother

\IfFileExists{\jobname-pw.ind}{\input{\jobname-pw.ind}}{}

% Quellenangabe nur in der Leseansicht
\ifkorrekturansicht\else
% Fallback-Definitionen, falls die .tex-Datei \titel etc. nicht gesetzt hat
\providecommand{\titel}{}
\providecommand{\editorInnen}{}
\providecommand{\dateiname}{\jobname}

\vspace{3cm}

\vfill

\footnotesize
\textsc{Quelle}: \titel. Herausgegeben von {\editorInnen}. In: \emph{Arthur Schnitzler: Briefwechsel mit Autorinnen und Autoren}.
 Digitale Edition, https://schnitzler-briefe.acdh.oeaw.ac.at/{\dateiname}.html (Stand \today)
\fi

\end{document}


