\input{../tex-inputs/latex-pdf-vorspann}
\begin{center}
            \textcolor{red}{ENTWURF. ENTZIFFERUNG NOCH NICHT KORREKTURGELESEN}
                      \end{center}
            
               \section[Paul Goldmann an Arthur Schnitzler, 22. 11. {[}1891{]}]{ Paul Goldmann an Arthur Schnitzler, 22. 11. {[}1891{]}}\nopagebreak\mylabel{v}\rehead{ }\begin{ledgroupsized}[t]{13cm}\normalsize\beginnumbering\briefempfaengerindex{Schnitzler, Arthur@\textsc{Schnitzler, Arthur}!zzzGoldmann, Paul@\emph{von Paul Goldmann}!1891-11-221@{22. 11. {[}1891{]}}|(be} \toendnotes[C]{\smallbreak\pagebreak[2]} \Standort{DLA, A:Schnitzler, HS.NZ85.1.3162.}
\physDesc{Brief, 1 Blatt, 3 Seiten
\newline{}Handschrift: blaue Tinte, deutsche Kurrent
\newline{}Schnitzler: 1) mit rotem Buntstift eine Unterstreichung 2) mit Bleistift datiert: »9\textcolor{gray}{1}«}\toendnotes[C]{\smallbreak}\pstart
           \noindent{}\centering{}{\pb}\textcolor{gray}{\textbf{Dr. jur. Paul Goldmann}}\pend
           \pstart
           \noindent{}\centering{}\textcolor{gray}{\textbf{\begin{otherlanguage}{french}Correspondant de la »Gazette de Francfort\orgindex{Frankfurter Zeitung@Frankfurter Zeitung|pw}«\end{otherlanguage}}}\pend
           \pstart
           \noindent{}\centering{}\textcolor{gray}{\textbf{\begin{otherlanguage}{french}Bruxelles, 21, rue des Plantes\end{otherlanguage}\oindex{rue des Plantes@\textbf{rue des Plantes}|pw}.}}\pend
           \pstart
           \raggedleft{}Brüſſel\oindex{Bruessel@\textbf{Brüssel}|pw}, 22. November.\pend
           \pstart\center{}Mein lieber Arthur!\pend\pstart
           Im Fluge: vielen, vielen, vielen Dank für den lieben Brief und die heutige Sendung.
               Ich ſchleppe das \label{K_L02671-4v}\edtext{Büchlein\pwindex{Schnitzler, Arthur 15.05.1862 – 21.10.1931@\textsc{Schnitzler, Arthur} (15.05.1862 – 21.10.1931), \emph{Schriftsteller, Mediziner}!Maerchen. Schauspiel in drei Aufzuegen1891 – 1891@\strich\emph{Das Märchen. Schauspiel in drei Aufzügen} {[}1891 – 1891{]}|pwv}}{\lemma{\textnormal{\emph{Büchlein}}}\Cendnote{\textnormal{Es dürfte sich noch nicht um das
                  Bühnenmanuskript von \emph{Das Märchen}\pwindex{Schnitzler, Arthur 15.05.1862 – 21.10.1931@\textsc{Schnitzler, Arthur} (15.05.1862 – 21.10.1931), \emph{Schriftsteller, Mediziner}!Maerchen. Schauspiel in drei Aufzuegen1891 – 1891@\strich\emph{Das Märchen. Schauspiel in drei Aufzügen} {[}1891 – 1891{]}|pwk} handeln, das
                     Schnitzler\pwindex{Schnitzler, Arthur 15.05.1862 – 21.10.1931@\textsc{Schnitzler, Arthur} (15.05.1862 – 21.10.1931), \emph{Schriftsteller, Mediziner}|pwk} erst am 5. 12. 1891 geliefert
                  bekam. Wahrscheinlich hatte er eine Abschrift geschickt, die dadurch verfügbar
                  wurde, dass sich das Manuskript in Druck befand.}}}\label{K_L02671-4h} den ganzen Tag mit mir
               herum, getraue mich aber nicht hineinzublicken, weil \strikeout{h\textcolor{gray}{eut}}{ }heut wieder einmal die Wien\oindex{Wien@\textbf{Wien}|pw}-Wunde offen iſt und mir jede Beſchäftigung mit dem, was mir dort lieb
               und theuer iſt, wüthendes Herz- und Heimweh verurſacht. Nächſtens hoffentlich eine
               ausführliche Antwort. Das heutige nur als Thatbeſtandaufnahme meiner Freude und
               meines Dankes{\dotsfour}\pend
           \pstart
           Die Fäden! Die Fäden! In Paris\oindex{Paris@\textbf{Paris}|pw} hat die Frkf. Ztg.\orgindex{Frankfurter Zeitung@Frankfurter Zeitung|pw} auch {\pb}einen neuen Correſpondenten\pwindex{Spitzer, Leopold 1865 – 1913-06-28@\textsc{Spitzer, Leopold} (1865 – 1913-06-28), \emph{Journalist}|pwv} für den finanziellen Theil ernannt, der mein engerer College\pwindex{Spitzer, Leopold 1865 – 1913-06-28@\textsc{Spitzer, Leopold} (1865 – 1913-06-28), \emph{Journalist}|pwv}{ }\strikeout{\textcolor{gray}{w}} und zugleich ein wenig mein Mitarbeiter\pwindex{Spitzer, Leopold 1865 – 1913-06-28@\textsc{Spitzer, Leopold} (1865 – 1913-06-28), \emph{Journalist}|pwv} werden ſoll. Weißt Du wer? Dein Freund \textsc{Spitzer\pwindex{Spitzer, Leopold 1865 – 1913-06-28@\textsc{Spitzer, Leopold} (1865 – 1913-06-28), \emph{Journalist}|pw}}, von dem Du mir erſt kürzlich ſchriebſt, daß er Dich in Wien\oindex{Wien@\textbf{Wien}|pw}{ }\label{K_L02671-1v}\edtext{ beſucht }{\lemma{\textnormal{\emph{ beſucht }}}\Cendnote{\textnormal{nicht bekannt}}}\label{K_L02671-1h}{ }\textsc{etc}. Wir werden eine \textsc{Schnitzler}-Gemeinde in \strikeout{\textcolor{gray}{Wi}}{ }Paris\oindex{Paris@\textbf{Paris}|pw}
               begründen. Und von nun an werden die zwei Pariſ\oindex{Paris@\textbf{Paris}|pw}er
                  Correſpondenten\pwindex{Spitzer, Leopold 1865 – 1913-06-28@\textsc{Spitzer, Leopold} (1865 – 1913-06-28), \emph{Journalist}|pwv} eines der
               größten deutschen Blätter\orgindex{Frankfurter Zeitung@Frankfurter Zeitung|pwv}{ }\strikeout{v\textcolor{gray}{on}} mit vereinten Kräften »an Dich glauben«, was gewiß ein ganzes Publicum
               aufwiegt. Kind, das Du biſt, mit Deinen Zweifeln, die doch übrigens für den
               Eingeweihten eine ſo deutliche Beſtätigung Deines Talentes bilden{\dotsfour}\pend
           \pstart
           {\pb}Dein nächſtjähriger \label{K_L02671-2v}\edtext{Reiſeplan}{\lemma{\textnormal{\emph{Reiſeplan}}}\Cendnote{\textnormal{Schnitzler kam das nächste Mal erst am 12. 4. 1897 nach Paris\oindex{Paris@\textbf{Paris}|pwk}.}}}\label{K_L02671-2h} enthält doch Paris\oindex{Paris@\textbf{Paris}|pw}? Ich
               halte das übrigens für ſo ſelbſtverſtändlich, daß ich gar nicht danach frage. Ich
               ſehe nur eine Schwierigkeit: nämlich daß ich bis zu Deiner Ankunft nicht etwa bereits
               wieder entlaſſen bin.\pend
           \pstart
           Das gehört übrigens Alles bereits in den nächſten großen Brief. Gott grüße Dich, mein
               lieber Alter!\pend
           \pstart
           Dein {\\[\baselineskip]}treuer {\\[\baselineskip]}\spacefill\mbox{Paul.}\pend
           \leftskip=0em{}\pstart
           \noindent{}Grüße an {\dots}{ } Du weißt ſchon{\dots}\pend
           \endnumbering\briefempfaengerindex{Schnitzler, Arthur@\textsc{Schnitzler, Arthur}!zzzGoldmann, Paul@\emph{von Paul Goldmann}!1891-11-221@{22. 11. {[}1891{]}}|)be}\mylabel{h}\end{ledgroupsized}  \newcommand{\dateiname}{L02671}\newcommand{\titel}{Paul Goldmann an Arthur Schnitzler, 22. 11. [1891]}\newcommand{\editorInnen}{Martin Anton Müller und Laura Untner}\input{../tex-inputs/latex-pdf-abspann}
      