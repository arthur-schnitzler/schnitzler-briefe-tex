%% latex-leseansicht-vorspann.tex
%% Vorspann für die Leseansicht.
%% Lädt die gemeinsame Datei latex-vorspann.tex mit nicht gesetztem Schalter.

\newif\ifkorrekturansicht
\korrekturansichtfalse

\input{../tex-inputs/latex-vorspann}


\section[Paul Goldmann an Arthur Schnitzler, 22. 11. {[}1891{]}]{L02671 Paul Goldmann an Arthur Schnitzler, 22. 11. [1891]}
\nopagebreak\mylabel{L02671v}
\rehead{ }\normalsize\beginnumbering\briefempfaengerindex{Schnitzler, Arthur@\textsc{Schnitzler, Arthur}!zzzGoldmann, Paul@\emph{von Paul Goldmann}!1891-11-221@{22. 11. [1891]}|(be}
\toendnotes[C]{\smallbreak\pagebreak[2]}
\correspDesc{Versand  durch Paul Goldmann am 22. 11. [1891] in Brüssel
\newline{}Erhalt  durch Arthur Schnitzler im Zeitraum [23. 11. 1891 – 27. 11. 1891?] in Wien}\toendnotes[C]{\smallbreak}
\Standort{DLA, A:Schnitzler, HS.NZ85.1.3162.}
\physDesc{Brief, 1 Blatt, 3 Seiten, 1543 Zeichen
\newline{}Handschrift: blaue Tinte, deutsche Kurrent
\newline{}Schnitzler: 1) mit rotem Buntstift eine Unterstreichung  2) mit Bleistift datiert: »9\textcolor{gray}{1}«}\toendnotes[C]{\smallbreak}
\pstart
           \centering{}{\pb}\textcolor{gray}{\textbf{Dr. jur. Paul Goldmann}}\pend
           
\pstart
           \centering{}\textcolor{gray}{\textbf{\begin{otherlanguage}{french}Correspondant de la »Gazette de Francfort\orgindex{Frankfurter Zeitung@Frankfurter Zeitung|pw}«\end{otherlanguage}}}\pend
           
\pstart
           \centering{}\textcolor{gray}{\textbf{\begin{otherlanguage}{french}Bruxelles, 21, rue des Plantes\end{otherlanguage}\oindex{rue des Plantes@\textbf{rue des Plantes}, \emph{Straße}|pw}.}}\pend
           
\pstart
           \raggedleft{}Brüſſel\oindex{Brüssel@\textbf{Brüssel}, \emph{Hauptstadt}|pw}, 22. November.\pend
           
\pstart\center{}Mein lieber Arthur!\pend\vspace{0.5em}
\pstart
           Im Fluge: vielen, vielen, vielen Dank für den lieben Brief und die heutige Sendung.
               Ich{ }ſchleppe das \label{K_L02671-1v}\edtext{Büchlein\pwindex{Schnitzler, Arthur 15.\,5.\,1862 Wien – 21.\,10.\,1931 ebd.@\textsc{Schnitzler, Arthur} (15.\,5.\,1862 Wien – 21.\,10.\,1931 ebd.), \emph{Schriftsteller, Mediziner}!Märchen. Schauspiel in drei Aufzügen@\strich\emph{Das Märchen. Schauspiel in drei Aufzügen}|pwv}}{\lemma{\textnormal{\emph{Büchlein}}}\Cendnote{\textnormal{Es dürfte sich noch nicht um das
                  Bühnenmanuskript von \emph{Das Märchen}\pwindex{Schnitzler, Arthur 15.\,5.\,1862 Wien – 21.\,10.\,1931 ebd.@\textsc{Schnitzler, Arthur} (15.\,5.\,1862 Wien – 21.\,10.\,1931 ebd.), \emph{Schriftsteller, Mediziner}!Märchen. Schauspiel in drei Aufzügen@\strich\emph{Das Märchen. Schauspiel in drei Aufzügen}|pwk} handeln, das
                     Schnitzler erst am 5. 12. 1891 geliefert
                  bekam. Wahrscheinlich hatte er eine Abschrift geschickt, die dadurch verfügbar
                  wurde, dass sich das Manuskript in Druck befand.}}}\label{K_L02671-1} den ganzen Tag mit mir
               herum, getraue mich aber nicht hineinzublicken, weil \strikeout{h\textcolor{gray}{eut}}{ }heut wieder einmal die Wien\oindex{Wien@\textbf{Wien}, \emph{Verwaltungsgebiet}|pw}-Wunde offen iſt und mir jede Beſchäftigung mit dem, was mir dort lieb
               und theuer iſt, wüthendes Herz- und Heimweh verurſacht. Nächſtens hoffentlich eine
               ausführliche Antwort. Das heutige nur als Thatbeſtandaufnahme meiner Freude und
               meines Dankes{\dotsfour}\pend
           
\pstart
           Die Fäden! Die Fäden! In Paris\oindex{Paris@\textbf{Paris}, \emph{Hauptstadt}|pw} hat die Frkf. Ztg.\orgindex{Frankfurter Zeitung@Frankfurter Zeitung|pw} auch {\pb}einen neuen Correſpondenten\pwindex{Spitzer, Leopold 1865 – 28.\,6.\,1913 Wien@\textsc{Spitzer, Leopold} (1865 – 28.\,6.\,1913 Wien), \emph{Journalist}|pwv} für den finanziellen Theil ernannt, der mein engerer College\pwindex{Spitzer, Leopold 1865 – 28.\,6.\,1913 Wien@\textsc{Spitzer, Leopold} (1865 – 28.\,6.\,1913 Wien), \emph{Journalist}|pwv}{ }\strikeout{\textcolor{gray}{w}} und zugleich ein wenig mein Mitarbeiter\pwindex{Spitzer, Leopold 1865 – 28.\,6.\,1913 Wien@\textsc{Spitzer, Leopold} (1865 – 28.\,6.\,1913 Wien), \emph{Journalist}|pwv} werden{ }ſoll. Weißt Du wer? Dein Freund \textsc{Spitzer\pwindex{Spitzer, Leopold 1865 – 28.\,6.\,1913 Wien@\textsc{Spitzer, Leopold} (1865 – 28.\,6.\,1913 Wien), \emph{Journalist}|pw}}, von dem Du mir erſt kürzlich{ }ſchriebſt, daß er Dich in Wien\oindex{Wien@\textbf{Wien}, \emph{Verwaltungsgebiet}|pw}{ }\label{K_L02671-2v}\edtext{beſucht}{\lemma{\textnormal{\emph{besucht}}}\Cendnote{\textnormal{nicht bekannt}}}\label{K_L02671-2}{ }\textsc{etc}. Wir werden eine \textsc{Schnitzler}-Gemeinde in \strikeout{\textcolor{gray}{Wi}}{ }Paris\oindex{Paris@\textbf{Paris}, \emph{Hauptstadt}|pw} begründen. Und von nun an werden die zwei
                  Pariſ\oindex{Paris@\textbf{Paris}, \emph{Hauptstadt}|pw}er Correſpondenten\pwindex{Spitzer, Leopold 1865 – 28.\,6.\,1913 Wien@\textsc{Spitzer, Leopold} (1865 – 28.\,6.\,1913 Wien), \emph{Journalist}|pwv} eines der größten deutschen Blätter\orgindex{Frankfurter Zeitung@Frankfurter Zeitung|pwv}{ }\strikeout{v\textcolor{gray}{on}} mit vereinten Kräften »an Dich glauben«, was gewiß ein ganzes Publicum
               aufwiegt. Kind, das Du biſt, mit Deinen Zweifeln, die doch übrigens für den
               Eingeweihten eine{ }ſo deutliche Beſtätigung Deines Talentes bilden{\dotsfour}\pend
           
\pstart
           {\pb}Dein nächſtjähriger \label{K_L02671-3v}\edtext{Reiſeplan}{\lemma{\textnormal{\emph{Reiseplan}}}\Cendnote{\textnormal{Schnitzler kam das nächste Mal erst am 12. 4. 1897 nach Paris\oindex{Paris@\textbf{Paris}, \emph{Hauptstadt}|pwk}.}}}\label{K_L02671-3} enthält doch Paris\oindex{Paris@\textbf{Paris}, \emph{Hauptstadt}|pw}? Ich
               halte das übrigens für{ }ſo{ }ſelbſtverſtändlich, daß ich gar nicht danach frage. Ich{ }ſehe nur eine Schwierigkeit: nämlich daß ich bis zu Deiner Ankunft nicht etwa bereits
               wieder entlaſſen bin.\pend
           
\pstart
           Das gehört übrigens Alles bereits in den nächſten großen Brief. Gott grüße Dich, mein
               lieber Alter!\pend
           
\pstart
           Dein {\\[\baselineskip]}treuer {\\[\baselineskip]}\spacefill\mbox{Paul.}\pend
           \leftskip=0em{}
\pstart
           \noindent{}Grüße an {\dots}{ } Du weißt{ }ſchon{\dots}\pend
           \selectlanguage{ngerman}\endnumbering\briefempfaengerindex{Schnitzler, Arthur@\textsc{Schnitzler, Arthur}!zzzGoldmann, Paul@\emph{von Paul Goldmann}!1891-11-221@{22. 11. [1891]}|)be}\mylabel{L02671h}  \newcommand{\dateiname}{L02671}\newcommand{\titel}{Paul Goldmann an Arthur Schnitzler, 22. 11. [1891]}\newcommand{\editorInnen}{Martin Anton Müller und Laura Untner}%% latex-leseansicht-abspann.tex
%% Abspann für die Leseansicht.
%% Der Schalter \ifkorrekturansicht ist bereits durch den Vorspann gesetzt.

%% latex-abspann.tex
%% Gemeinsamer Abspann für Korrekturansicht und Leseansicht.
%% Setzt den Schalter \ifkorrekturansicht voraus (gesetzt in den
%% einbindenden Dateien latex-korrekturansicht-abspann.tex bzw.
%% latex-leseansicht-abspann.tex).
%% ---------------------------------------------------------------

\normalsize

% Das esempio-Environment wird nur in der Leseansicht benötigt
\ifkorrekturansicht\else
\newenvironment{esempio}[3]%
{
    \vspace{1.5ex}
    \rlap{\underline{#1}}
    \par
    \setlength{\parindent}{0cm}
    \nopagebreak
    \leftskip=#2cm
    \rightskip=#3cm
}
{
    \par
}
\fi

\doendnotes{C}
\bigskip
\vfill

\clearpage

\footnotesize

\ifkorrekturansicht
  \lohead{\textsc{register}}
\fi

% theindex-Environment neu definieren ohne reledmac
\makeatletter
\renewenvironment{theindex}{%
  \ifkorrekturansicht
    \section*{\indexname}%
  \else
    \subsubsection*{Index der erwähnten Entitäten}%
  \fi
  \setlength{\parindent}{0pt}%
  \setlength{\parskip}{0pt plus 0.3pt}%
  \let\item\@idxitem
}{%
  \ifkorrekturansicht\clearpage\fi
}
\makeatother

\IfFileExists{\jobname-pw.ind}{\input{\jobname-pw.ind}}{}

% Quellenangabe nur in der Leseansicht
\ifkorrekturansicht\else
% Fallback-Definitionen, falls die .tex-Datei \titel etc. nicht gesetzt hat
\providecommand{\titel}{}
\providecommand{\editorInnen}{}
\providecommand{\dateiname}{\jobname}

\vspace{3cm}

\vfill

\footnotesize
\textsc{Quelle}: \titel. Herausgegeben von {\editorInnen}. In: \emph{Arthur Schnitzler: Briefwechsel mit Autorinnen und Autoren}.
 Digitale Edition, https://schnitzler-briefe.acdh.oeaw.ac.at/{\dateiname}.html (Stand \today)
\fi

\end{document}


