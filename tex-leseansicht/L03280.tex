%% latex-korrekturansicht-vorspann.tex
%% Vorspann für die Korrekturansicht.
%% Lädt die gemeinsame Datei latex-vorspann.tex mit gesetztem Schalter.

\newif\ifkorrekturansicht
\korrekturansichttrue

\input{../tex-inputs/latex-vorspann}


\section[ Felix Salten an Arthur Schnitzler, 30. 7. 1898]{L03280 Felix Salten an Arthur Schnitzler, 30. 7. 1898}
\nopagebreak\mylabel{L03280v}
\rehead{ }\normalsize\beginnumbering\briefempfaengerindex{Schnitzler, Arthur@\textsc{Schnitzler, Arthur}!zzzSalten, Felix@\emph{von Felix Salten}!1898-07-301@{30. 7. 1898}|(be}
\toendnotes[C]{\smallbreak\pagebreak[2]}\Standort{CUL, Schnitzler, B 89, A 2.}
\physDesc{Brief, 1 Blatt, 2 Seiten, 542 Zeichen
\newline{}Handschrift: schwarze Tinte, lateinische Kurrent
\newline{}Ordnung: mit Bleistift von unbekannter Hand nummeriert: »104« }\toendnotes[C]{\smallbreak}
\pstart
           \raggedleft{}{\pb}Wien\oindex{Wien@\textbf{Wien}, \emph{A.ADM2}|pw}, 30. Juli 98.\pend
           \vspace{0.5em}
\pstart
           Lieber Arthur, bis heute war ich
               nicht in Wien\oindex{Wien@\textbf{Wien}, \emph{A.ADM2}|pw}. Meine Arbeit habe ich in Pressbaum\oindex{Pressbaum@\textbf{Pressbaum}, \emph{P.PPLA3}|pw} fertig gemacht, dann bin ich in Karlsbad\oindex{Karlsbad@\textbf{Karlsbad}, \emph{P.PPLA}|pw} gewesen, und jetzt war ich wieder in Pressbaum\oindex{Pressbaum@\textbf{Pressbaum}, \emph{P.PPLA3}|pw}. Ich gehe am 8\textsuperscript{ten} nach \label{K_L03280-1v}\edtext{Reichenhall\oindex{Bad Reichenhall@\textbf{Bad Reichenhall}, \emph{A.ADM4}|pw}, wo ich bis 1. September bleibe. Vielleicht kommen Sie einmal vorbei}{\lemma{\textnormal{\emph{Reichenhall, … vorbei}}}\Cendnote{\textnormal{Schnitzler kam nicht vorbei.}}}\label{K_L03280-1}.
               Dort schreibe ich das \label{K_L03280-2v}\edtext{österr.\oindex{Oesterreich@\textbf{Österreich}, \emph{A.PCLI}|pw} Theater}{\lemma{\textnormal{\emph{österr. Theater}}}\Cendnote{\textnormal{Ein größerer Essay über die österreichische Theaterszene\oindex{Oesterreich@\textbf{Österreich}, \emph{A.PCLI}|pwk}
                  konnte nicht nachgewiesen werden.}}}\label{K_L03280-2}. Stimmung und Befinden nicht
               hervorragend. In Karlsbad\oindex{Karlsbad@\textbf{Karlsbad}, \emph{P.PPLA}|pw} ein hübsches Erlebnis.
               Ab 1. August wohne ich Hietzing, Wattmanngasse 11\oindex{Wattmanngasse@\textbf{Wattmanngasse}, \emph{Straße (K.STR)}|pw}, doch bitte ich mir Briefe nur \label{K_L03280-3v}\edtext{hieher}{\lemma{\textnormal{\emph{hieher}}}\Cendnote{\textnormal{in die Sensengasse 5\oindex{Sensengasse@\textbf{Sensengasse}, \emph{Straße (K.STR)}|pwk}, vgl. Felix Salten an Arthur Schnitzler, [10. 7. 1898]?}}}\label{K_L03280-3}, damit sie mir nachgeschickt werden.\pend
           
\pstart
           {\pb}Viele Grüße {\\[\baselineskip]}herzlichst Ihr {\\[\baselineskip]}\spacefill\mbox{Salten}\pend
           \leftskip=0em{}\selectlanguage{ngerman}\endnumbering\briefempfaengerindex{Schnitzler, Arthur@\textsc{Schnitzler, Arthur}!zzzSalten, Felix@\emph{von Felix Salten}!1898-07-301@{30. 7. 1898}|)be}\mylabel{L03280h}  \normalsize

\doendnotes{C}
\bigskip
\vfill

\clearpage

\footnotesize

\lohead{\textsc{register}}

% Definiere theindex-Environment komplett neu ohne reledmac
\makeatletter
\renewenvironment{theindex}{%
  \section*{\indexname}%
  \setlength{\parindent}{0pt}%
  \setlength{\parskip}{0pt plus 0.3pt}%
  \let\item\@idxitem
}{%
  \clearpage
}
\makeatother

\IfFileExists{\jobname-pw.ind}{\input{\jobname-pw.ind}}{}

\end{document}

      