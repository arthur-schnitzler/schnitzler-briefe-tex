%% latex-leseansicht-vorspann.tex
%% Vorspann für die Leseansicht.
%% Lädt die gemeinsame Datei latex-vorspann.tex mit nicht gesetztem Schalter.

\newif\ifkorrekturansicht
\korrekturansichtfalse

\input{../tex-inputs/latex-vorspann}

\begin{center}
            \textcolor{red}{ENTWURF, NICHT FERTIG KORRIGIERT}
                      \end{center}
            
         \renewcommand{\erwaehnteOrte}{Orte: Bad Reichenhall, Karlsbad, Pressbaum, Wattmanngasse, Wien, Österreich}
         \renewcommand{\erwaehnteWerke}{}
               \section[Felix Salten an Arthur Schnitzler, 30. 7. 1898]{ Felix Salten an Arthur Schnitzler, 30. 7. 1898}\nopagebreak\mylabel{v}\rehead{ }\begin{ledgroupsized}[t]{13cm}\normalsize\beginnumbering \toendnotes[C]{\smallbreak\pagebreak[2]} \Standort{CUL, Schnitzler, B 89, A 2.}
\physDesc{Brief, 1 Blatt, 2 Seiten, 545 Zeichen
\newline{}Handschrift: schwarze Tinte, lateinische Kurrent
\newline{}Ordnung: mit Bleistift von unbekannter Hand nummeriert:
                                    »104« }\toendnotes[C]{\smallbreak}\pstart
           {\pb}Wien\oindex{Wien@\textbf{Wien}|pw}, 30. Juli 98. \pend
           \pstart
           Lieber Arthur, bis heute war ich nicht in Wien\oindex{Wien@\textbf{Wien}|pw}. Meine Arbeit habe ich in Pressbaum\oindex{Pressbaum@\textbf{Pressbaum}|pw} fertig gemacht, dann bin ich in Karlsbad\oindex{Karlsbad@\textbf{Karlsbad}|pw} gewesen, und jetzt war ich wieder in Pressbaum\oindex{Pressbaum@\textbf{Pressbaum}|pw}. Ich gehe am 8\textsuperscript{ten} nach Reichenhall\oindex{Bad Reichenhall@\textbf{Bad Reichenhall}|pw}, wo ich bis 1.
                  September bleibe. Vielleicht kommen Sie einmal vorbei. Dort schreibe ich
               das \label{K_L03280-1v}\edtext{österr.\oindex{Oesterreich@\textbf{Österreich}|pw} Theater}{\lemma{\textnormal{\emph{österr. Theater}}}\Cendnote{\textnormal{XXXX}}}\label{K_L03280-1h}. Stimmung und Befinden nicht hervorragend. In Karlsbad\oindex{Karlsbad@\textbf{Karlsbad}|pw} ein hübsches Erlebnis. Ab 1.
                  August wohne ich Hietzing, Wattmanngasse
                  11\oindex{Wattmanngasse@\textbf{Wattmanngasse}|pw}, doch bitte ich mir Briefe nur hieher, damit sie mir nachgeschickt
               werden.\pend
           \pstart
           {\pb}Viele Grüße {\\[\baselineskip]}herzlichst
               {\\[\baselineskip]}Ihr {\\[\baselineskip]}\spacefill\mbox{Salten}\pend
           \leftskip=0em{}
         
         \endnumbering\mylabel{h}\end{ledgroupsized}\begin{anhang}\end{anhang}\newcommand{\dateiname}{L03280}\newcommand{\titel}{Felix Salten an Arthur Schnitzler, 30. 7. 1898}\newcommand{\editorInnen}{Martin Anton Müller und Laura Untner}%% latex-leseansicht-abspann.tex
%% Abspann für die Leseansicht.
%% Der Schalter \ifkorrekturansicht ist bereits durch den Vorspann gesetzt.

%% latex-abspann.tex
%% Gemeinsamer Abspann für Korrekturansicht und Leseansicht.
%% Setzt den Schalter \ifkorrekturansicht voraus (gesetzt in den
%% einbindenden Dateien latex-korrekturansicht-abspann.tex bzw.
%% latex-leseansicht-abspann.tex).
%% ---------------------------------------------------------------

\normalsize

% Das esempio-Environment wird nur in der Leseansicht benötigt
\ifkorrekturansicht\else
\newenvironment{esempio}[3]%
{
    \vspace{1.5ex}
    \rlap{\underline{#1}}
    \par
    \setlength{\parindent}{0cm}
    \nopagebreak
    \leftskip=#2cm
    \rightskip=#3cm
}
{
    \par
}
\fi

\doendnotes{C}
\bigskip
\vfill

\clearpage

\footnotesize

\ifkorrekturansicht
  \lohead{\textsc{register}}
\fi

% theindex-Environment neu definieren ohne reledmac
\makeatletter
\renewenvironment{theindex}{%
  \ifkorrekturansicht
    \section*{\indexname}%
  \else
    \subsubsection*{Index der erwähnten Entitäten}%
  \fi
  \setlength{\parindent}{0pt}%
  \setlength{\parskip}{0pt plus 0.3pt}%
  \let\item\@idxitem
}{%
  \ifkorrekturansicht\clearpage\fi
}
\makeatother

\IfFileExists{\jobname-pw.ind}{\input{\jobname-pw.ind}}{}

% Quellenangabe nur in der Leseansicht
\ifkorrekturansicht\else
% Fallback-Definitionen, falls die .tex-Datei \titel etc. nicht gesetzt hat
\providecommand{\titel}{}
\providecommand{\editorInnen}{}
\providecommand{\dateiname}{\jobname}

\vspace{3cm}

\vfill

\footnotesize
\textsc{Quelle}: \titel. Herausgegeben von {\editorInnen}. In: \emph{Arthur Schnitzler: Briefwechsel mit Autorinnen und Autoren}.
 Digitale Edition, https://schnitzler-briefe.acdh.oeaw.ac.at/{\dateiname}.html (Stand \today)
\fi

\end{document}


      