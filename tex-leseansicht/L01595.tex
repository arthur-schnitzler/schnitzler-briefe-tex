%% latex-leseansicht-vorspann.tex
%% Vorspann für die Leseansicht.
%% Lädt die gemeinsame Datei latex-vorspann.tex mit nicht gesetztem Schalter.

\newif\ifkorrekturansicht
\korrekturansichtfalse

\input{../tex-inputs/latex-vorspann}


         
         \renewcommand{\erwaehntePersonen}{Personen: Theodor Antropp, Gustav Harpner, Robert Hirschfeld, Karl Höger, Engelbert Pernerstorfer, Alfred Polgar, Richard Vallentin, Leopold Winarsky}
         \renewcommand{\erwaehnteInstitutionen}{Institutionen: Arbeiter-Zeitung, Volksbühne Berlin, Wiener Freie Volksbühne}
         \renewcommand{\erwaehnteOrte}{Orte: Graben, Mariahilferstraße, Wien}
         \renewcommand{\erwaehnteWerke}{
               \section[Stefan Großmann an Arthur Schnitzler, 4. 5. 1906]{ Stefan Großmann an Arthur Schnitzler, 4. 5. 1906}\nopagebreak\mylabel{v}\rehead{ }\begin{ledgroupsized}[t]{13cm}\normalsize\beginnumbering \toendnotes[C]{\smallbreak\pagebreak[2]} \Standort{CUL, Schnitzler, B 34.}
\physDesc{Brief, 1 Blatt (Briefpapier mit Trauerrand), 3 Seiten
\newline{}Handschrift: schwarze Tinte, deutsche Kurrent
\newline{}Schnitzler: mit Bleistift beschriftet: »Großman« \newline{}Ordnung: mit Bleistift von unbekannter Hand
                           nummeriert: »4« }\pstart
           \noindent{}\centering{}{\pb}\textcolor{gray}{\textbf{Arbeiter-Zeitung\orgindex{Arbeiter-Zeitung@Arbeiter-Zeitung|pw}.}}\pend
           \pstart
           \noindent{}\centering{}\textcolor{gray}{\textbf{Zentral-Organ der österreichischen Sozialdemokratie.}}\pend
           \pstart
           \noindent{}\textcolor{gray}{\textbf{Redaktion:}}\hfill \textcolor{gray}{\textbf{Administration und Inseraten-Aufnahme:}}\pend
           \pstart
           \textcolor{gray}{\textbf{Wien\oindex{Wien@\textbf{Wien}|pw} VI/\textsubscript{1}. Mariahilferstrasse Nr. 89.\oindex{Mariahilferstrasse@\textbf{Mariahilferstraße}|pw}}}\hfill \textcolor{gray}{\textbf{Wien\oindex{Wien@\textbf{Wien}|pw}}}\pend
           \pstart
           \textcolor{gray}{\textbf{Telephon Nr. 880.}}\hfill \textcolor{gray}{\textbf{VI/\textsubscript{1}. Mariahilferstrasse
                           Nr. 89.\oindex{Mariahilferstrasse@\textbf{Mariahilferstraße}|pw}}}\pend
           \pstart
           \textcolor{gray}{\textbf{Postsparkassen-Scheck-Konto Nr. 819.210.}}\hfill \textcolor{gray}{\textbf{Telephon Nr. 900.}}\pend
           \pstart
           \raggedleft{}\textcolor{gray}{\textbf{Wien\oindex{Wien@\textbf{Wien}|pw}, am}}{ }4. Mai \textcolor{gray}{\textbf{190}}6\pend
           \pstart{}Sehr geehrter Herr.\pend\pstart
           Aus den Kreiſen der Wiener Arbeiterſchaft ſoll nun endlich, nach dem Vorbild der Berliner\orgindex{Volksbuehne Berlin@Volksbühne Berlin|pw}, ein Verein \textsc{\uline{Freie Volksbühne}}\orgindex{Wiener Freie Volksbuehne@Wiener Freie Volksbühne|pw} gebildet werden, der mit einem aus allen Wien\oindex{Wien@\textbf{Wien}|pw}er
               Theatern zuſammengeſtellten Enſemble Vorſtellungen zu mäßigen Preiſen veranſtalten
               will, die an anderen Bühnen nicht gebracht werden.\pend
           \pstart
           Es hat ſich ein \uline{Komité} gebildet, dem bisher
               angehören:\pend
           \settowidth{\longeste}{Arbeiterführer}\settowidth{\longestz}{Vallentin (der sich zur Leitung des Unternehmens bereit erklärt hat)}\settowidth{\longestd}{}\settowidth{\longestv}{}\settowidth{\longestf}{}\addtolength\longeste{1em}
        \addtolength\longestz{1em}
      \pstart\noindent\makebox[\the\longeste][l]{Reichsrathabg.}\makebox[\the\longestz][l]{\uline{\textsc{Pernerstorfer\pwindex{Pernerstorfer, Engelbert 27.04.1850 – 06.01.1918@\textsc{Pernerstorfer, Engelbert} (27.04.1850 – 06.01.1918), \emph{Politiker, Journalist}|pw}}}}
                  \pend\pstart\noindent\makebox[\the\longeste][l]{Regiſſeur}\makebox[\the\longestz][l]{\textsc{\uline{Vallentin\pwindex{Vallentin, Richard 03.02.1874 – 14.01.1908@\textsc{Vallentin, Richard} (03.02.1874 – 14.01.1908), \emph{Regisseur, Schauspieler}|pw}}} (der sich zur Leitung des Unternehmens bereit erklärt hat)}
                  \pend\pstart\noindent\makebox[\the\longeste][l]{Schriftſteller}\makebox[\the\longestz][l]{D\textsuperscript{r}{ }\textsc{Robert \uline{Hirschfeld}}\pwindex{Hirschfeld, Robert 17.09.1857 – 02.04.1914@\textsc{Hirschfeld, Robert} (17.09.1857 – 02.04.1914), \emph{Journalist, Musikkritiker}|pw}}
                  \pend\pstart\noindent\makebox[\the\longeste][l]{\hspace*{2em}„}\makebox[\the\longestz][l]{\textsc{Alfred \uline{Polgar}}\pwindex{Polgar, Alfred 17.10.1873 – 24.04.1955@\textsc{Polgar, Alfred} (17.10.1873 – 24.04.1955), \emph{Schriftsteller, Journalist, Kritiker}|pw}}
                  \pend\pstart\noindent\makebox[\the\longeste][l]{\hspace*{2em}„}\makebox[\the\longestz][l]{\textsc{Theodor \uline{Antropp}}\pwindex{Antropp, Theodor 29.10.1864 – 18.11.1923@\textsc{Antropp, Theodor} (29.10.1864 – 18.11.1923), \emph{Schriftsteller, Journalist, Theaterleiter}|pw}}
                  \pend\pstart\noindent\makebox[\the\longeste][l]{\hspace*{2em}„}\makebox[\the\longestz][l]{Stefan \uline{\textsc{Großmann}}\pwindex{Grossmann, Stefan 19.05.1875 – 03.01.1935@\textsc{Großmann, Stefan} (19.05.1875 – 03.01.1935), \emph{Schriftsteller, Journalist}|pw}}
                  \pend\pstart\noindent\makebox[\the\longeste][l]{Arbeiterführer}\makebox[\the\longestz][l]{\textsc{Leopold \uline{Winarsky}}\pwindex{Winarsky, Leopold 1873-04-20 – 1915-11-22@\textsc{Winarsky, Leopold} (1873-04-20 – 1915-11-22), \emph{Politiker}|pw}}
                  \pend\pstart\noindent\makebox[\the\longeste][l]{Buchdrucker}\makebox[\the\longestz][l]{\textsc{K. \uline{Höger}}\pwindex{Hoeger, Karl 1847-10-03 – 1913-10-17@\textsc{Höger, Karl} (1847-10-03 – 1913-10-17), \emph{Buchdrucker, Sozialdemokrat}|pw}}
                  \pend\pstart
           {\pb}Die Statuten des Vereines hat D\textsuperscript{r}{ }\textsc{Harpner}\pwindex{Harpner, Gustav 25.03.1864 – 10.07.1924@\textsc{Harpner, Gustav} (25.03.1864 – 10.07.1924), \emph{Rechtsanwalt}|pw} bereits ausgearbeitet\pend
           \pstart
           Dem Comité läge nun \uline{ſehr} viel daran, wenn Sie, ſehr
               geehrter Herr, dem Ausſchuſſe beitreten wollten. Wir glauben, daß unſer Unternehmen,
               an deſſen Beſtand und Wirkſamkeit (vom Herbſt an) nicht mehr zu rütteln
               ſein wird, auch Ihren Wünſchen und Hoffnungen für das Theaterwesen\strikeout{s}{ }Wien\oindex{Wien@\textbf{Wien}|pw}s entſprechen wird und würden es als Ehre und
               auch als große Freude empfinden, wenn Sie unſerem ſchönen Beginnen Ihre freundliche
               Mithilfe widmen wollten.\pend
           \pstart
           Eine conſtituierende Verſammlung des Ausſchuſſes ſoll \uline{\textsc{Dienstag}} abends (gegen 10\textsuperscript{h}) ſtattfinden. Wenn Sie daran
               theilnehmen wollten, würden Sie uns zu großem Dank verpflichten. Auch iſt der
               Unterzeichnete gern bereit, Ihnen {\pb}– wenn Sie
               es wünſchen – die nöthigen Aufklärungen über das Detail des Werkes mitzutheilen.
               Soviel ſei betont, daſs wir \uline{Mustervorſtellungen} zu
               machen gedenken und daſs uns vor Allem eine \uline{Erweiterung
                  des Spielplans} der W\textsuperscript{r}\oindex{Wien@\textbf{Wien}|pw} Bühnen, die ja faſt durchwegs im Familienſtück zugrundegehen, unerläſslich
               erſcheint.\pend
           \pstart
           Die freie Volksbühne\orgindex{Wiener Freie Volksbuehne@Wiener Freie Volksbühne|pw} würde es ſich zur Ehre rechnen,
               Ihren Namen unter den Begründern dieſes \strikeout{bühne}
               Unternehmens nennen zu dürfen.\pend
           \pstart
           Ihrer freundlichen Antwort gewärtig, {\\[\baselineskip]}mit aller Hochſchätzung:{\\[\baselineskip]}i. A.{\\[\baselineskip]}\spacefill\mbox{Stefan Großmann}\pend
           \leftskip=0em{}\pstart
           \noindent{}Wien I. Graben 29\textsuperscript{a}\oindex{Graben@\textbf{Graben}|pw}\pend
           
         
         \endnumbering\mylabel{h}\end{ledgroupsized}  \newcommand{\dateiname}{L01595}\newcommand{\titel}{Stefan Großmann an Arthur Schnitzler, 4. 5. 1906}\newcommand{\editorInnen}{ Martin Anton Müller und Gerd-Hermann Susen}%% latex-leseansicht-abspann.tex
%% Abspann für die Leseansicht.
%% Der Schalter \ifkorrekturansicht ist bereits durch den Vorspann gesetzt.

%% latex-abspann.tex
%% Gemeinsamer Abspann für Korrekturansicht und Leseansicht.
%% Setzt den Schalter \ifkorrekturansicht voraus (gesetzt in den
%% einbindenden Dateien latex-korrekturansicht-abspann.tex bzw.
%% latex-leseansicht-abspann.tex).
%% ---------------------------------------------------------------

\normalsize

% Das esempio-Environment wird nur in der Leseansicht benötigt
\ifkorrekturansicht\else
\newenvironment{esempio}[3]%
{
    \vspace{1.5ex}
    \rlap{\underline{#1}}
    \par
    \setlength{\parindent}{0cm}
    \nopagebreak
    \leftskip=#2cm
    \rightskip=#3cm
}
{
    \par
}
\fi

\doendnotes{C}
\bigskip
\vfill

\clearpage

\footnotesize

\ifkorrekturansicht
  \lohead{\textsc{register}}
\fi

% theindex-Environment neu definieren ohne reledmac
\makeatletter
\renewenvironment{theindex}{%
  \ifkorrekturansicht
    \section*{\indexname}%
  \else
    \subsubsection*{Index der erwähnten Entitäten}%
  \fi
  \setlength{\parindent}{0pt}%
  \setlength{\parskip}{0pt plus 0.3pt}%
  \let\item\@idxitem
}{%
  \ifkorrekturansicht\clearpage\fi
}
\makeatother

\IfFileExists{\jobname-pw.ind}{\input{\jobname-pw.ind}}{}

% Quellenangabe nur in der Leseansicht
\ifkorrekturansicht\else
% Fallback-Definitionen, falls die .tex-Datei \titel etc. nicht gesetzt hat
\providecommand{\titel}{}
\providecommand{\editorInnen}{}
\providecommand{\dateiname}{\jobname}

\vspace{3cm}

\vfill

\footnotesize
\textsc{Quelle}: \titel. Herausgegeben von {\editorInnen}. In: \emph{Arthur Schnitzler: Briefwechsel mit Autorinnen und Autoren}.
 Digitale Edition, https://schnitzler-briefe.acdh.oeaw.ac.at/{\dateiname}.html (Stand \today)
\fi

\end{document}


      