%% latex-leseansicht-vorspann.tex
%% Vorspann für die Leseansicht.
%% Lädt die gemeinsame Datei latex-vorspann.tex mit nicht gesetztem Schalter.

\newif\ifkorrekturansicht
\korrekturansichtfalse

\input{../tex-inputs/latex-vorspann}


\section[Stefan Großmann an Arthur Schnitzler, 4. 5. 1906]{L01595 Stefan Großmann an Arthur Schnitzler, 4. 5. 1906}
\nopagebreak\mylabel{L01595v}
\rehead{ }\normalsize\beginnumbering\briefempfaengerindex{Schnitzler, Arthur@\textsc{Schnitzler, Arthur}!zzzGroßmann, Stefan@\emph{von Stefan Großmann}!1906-05-041@{4. 5. 1906}|(be}
\toendnotes[C]{\smallbreak\pagebreak[2]}
\correspDesc{Versand  durch Stefan Großmann am 4. 5. 1906 in Wien
\newline{}Erhalt  durch Arthur Schnitzler im Zeitraum [4. 5. 1906
                  – 8. 5. 1906?] in Wien}\toendnotes[C]{\smallbreak}
\Standort{CUL, Schnitzler, B 34.}
\physDesc{Brief, 1 Blatt, 3 Seiten, 1838 Zeichen (Briefpapier mit Trauerrand)
\newline{}Handschrift: schwarze Tinte, deutsche Kurrent
\newline{}Schnitzler: mit Bleistift beschriftet: »Großman« 
\newline{}Ordnung: mit Bleistift von unbekannter Hand nummeriert:
                                 »4« }
\pstart
           \centering{}{\pb}\textcolor{gray}{\textbf{Arbeiter-Zeitung\orgindex{Arbeiter-Zeitung@Arbeiter-Zeitung|pw}.}}\pend
           
\pstart
           \centering{}\textcolor{gray}{\textbf{Zentral-Organ der österreichischen Sozialdemokratie.}}\pend
           
\pstart
           \textcolor{gray}{\textbf{Redaktion:}}\hfill \textcolor{gray}{\textbf{Administration und Inseraten-Aufnahme:}}\pend
           
\pstart
           \textcolor{gray}{\textbf{Wien\oindex{Wien@\textbf{Wien}, \emph{Verwaltungsgebiet}|pw} VI/\textsubscript{1}. Mariahilferstrasse Nr. 89.\oindex{Wien@\textbf{Wien}!VI., Mariahilf@\textbf{VI., Mariahilf}!Mariahilfer Straße@\textbf{Mariahilfer Straße}, \emph{Straße}|pw}}}\hfill \textcolor{gray}{\textbf{Wien\oindex{Wien@\textbf{Wien}, \emph{Verwaltungsgebiet}|pw}}}\pend
           
\pstart
           \textcolor{gray}{\textbf{Telephon Nr. 880.}}\hfill \textcolor{gray}{\textbf{VI/\textsubscript{1}. Mariahilferstrasse Nr. 89.\oindex{Wien@\textbf{Wien}!VI., Mariahilf@\textbf{VI., Mariahilf}!Mariahilfer Straße@\textbf{Mariahilfer Straße}, \emph{Straße}|pw}}}\pend
           
\pstart
           \textcolor{gray}{\textbf{Postsparkassen-Scheck-Konto Nr. 819.210.}}\hfill \textcolor{gray}{\textbf{Telephon Nr. 900.}}\pend
           
\pstart
           \raggedleft{}\textcolor{gray}{\textbf{Wien\oindex{Wien@\textbf{Wien}, \emph{Verwaltungsgebiet}|pw}, am}}{ }4. Mai \textcolor{gray}{\textbf{190}}6\pend
           
\pstart{}Sehr geehrter Herr.\pend\vspace{0.5em}
\pstart
           Aus den Kreiſen der Wiener Arbeiterſchaft{ }ſoll nun endlich, nach dem Vorbild der Berliner\orgindex{Volksbühne Berlin@Volksbühne Berlin|pw}, ein Verein \textsc{\uline{Freie Volksbühne}}\orgindex{Wiener Freie Volksbühne@Wiener Freie Volksbühne|pw} gebildet werden, der mit einem aus allen Wien\oindex{Wien@\textbf{Wien}, \emph{Verwaltungsgebiet}|pw}er Theatern zuſammengeſtellten Enſemble Vorſtellungen zu mäßigen Preiſen
               veranſtalten will, die an anderen Bühnen nicht gebracht werden.\pend
           
\pstart
           Es hat{ }ſich ein \uline{Komité} gebildet, dem bisher
               angehören:\pend
           \settowidth{\longeste}{Schriftstellerm}\settowidth{\longestz}{Vallentin(der sich zur Leitung des Unternehmens bereit erklärt hat)}\settowidth{\longestd}{}\settowidth{\longestv}{}\settowidth{\longestf}{}\addtolength\longeste{1em}
        \addtolength\longestz{1em}
      \pstart\noindent\makebox[\the\longeste][l]{Reichsrathabg.}\makebox[\the\longestz][l]{\uline{\textsc{Pernerstorfer\pwindex{Pernerstorfer, Engelbert 27.\,4.\,1850 Wien – 6.\,1.\,1918 ebd.@\textsc{Pernerstorfer, Engelbert} (27.\,4.\,1850 Wien – 6.\,1.\,1918 ebd.), \emph{Politiker, Journalist}|pw}}}}
                  \pend\pstart\noindent\makebox[\the\longeste][l]{Regiſſeur}\makebox[\the\longestz][l]{\textsc{\uline{Vallentin\pwindex{Vallentin, Richard 3.\,2.\,1874 Luzern – 14.\,1.\,1908 Berlin@\textsc{Vallentin, Richard} (3.\,2.\,1874 Luzern – 14.\,1.\,1908 Berlin), \emph{Regisseur, Schauspieler}|pw}}} (der sich zur Leitung des Unternehmens bereit erklärt hat)}
                  \pend\pstart\noindent\makebox[\the\longeste][l]{Schriftſteller}\makebox[\the\longestz][l]{D\textsuperscript{r}{ }\textsc{Robert \uline{Hirschfeld}}\pwindex{Hirschfeld, Robert 17.\,9.\,1857 Žďár nad Sázavou – 2.\,4.\,1914 Salzburg@\textsc{Hirschfeld, Robert} (17.\,9.\,1857 Žďár nad Sázavou – 2.\,4.\,1914 Salzburg), \emph{Journalist, Musikkritiker}|pw}}
                  \pend\pstart\noindent\makebox[\the\longeste][l]{\hspace*{2em}„}\makebox[\the\longestz][l]{\textsc{Alfred \uline{Polgar}}\pwindex{Polgar, Alfred 17.\,10.\,1873 Wien – 24.\,4.\,1955 Zürich@\textsc{Polgar, Alfred} (17.\,10.\,1873 Wien – 24.\,4.\,1955 Zürich), \emph{Schriftsteller, Journalist, Kritiker}|pw}}
                  \pend\pstart\noindent\makebox[\the\longeste][l]{\hspace*{2em}„}\makebox[\the\longestz][l]{\textsc{Theodor \uline{Antropp}}\pwindex{Antropp, Theodor 29.\,10.\,1864 Wien – 18.\,11.\,1923 ebd.@\textsc{Antropp, Theodor} (29.\,10.\,1864 Wien – 18.\,11.\,1923 ebd.), \emph{Schriftsteller, Journalist, Theaterleiter}|pw}}
                  \pend\pstart\noindent\makebox[\the\longeste][l]{\hspace*{2em}„}\makebox[\the\longestz][l]{Stefan \uline{\textsc{Großmann}}\pwindex{Großmann, Stefan 19.\,5.\,1875 Wien – 3.\,1.\,1935 ebd.@\textsc{Großmann, Stefan} (19.\,5.\,1875 Wien – 3.\,1.\,1935 ebd.), \emph{Schriftsteller, Journalist}|pw}}
                  \pend\pstart\noindent\makebox[\the\longeste][l]{Arbeiterführer}\makebox[\the\longestz][l]{\textsc{Leopold \uline{Winarsky}}\pwindex{Winarsky, Leopold 20.\,4.\,1873 Brünn – 22.\,11.\,1915 Wien@\textsc{Winarsky, Leopold} (20.\,4.\,1873 Brünn – 22.\,11.\,1915 Wien), \emph{Politiker}|pw}}
                  \pend\pstart\noindent\makebox[\the\longeste][l]{Buchdrucker}\makebox[\the\longestz][l]{\textsc{K. \uline{Höger}}\pwindex{Höger, Karl 3.\,10.\,1847 Wien – 17.\,10.\,1913 ebd.@\textsc{Höger, Karl} (3.\,10.\,1847 Wien – 17.\,10.\,1913 ebd.), \emph{Buchdrucker, Sozialdemokrat}|pw}}
                  \pend
\pstart
           {\pb}Die Statuten des Vereines hat D\textsuperscript{r}{ }\textsc{Harpner}\pwindex{Harpner, Gustav 25.\,3.\,1864 Brünn – 10.\,7.\,1924 Wien@\textsc{Harpner, Gustav} (25.\,3.\,1864 Brünn – 10.\,7.\,1924 Wien), \emph{Rechtsanwalt}|pw} bereits ausgearbeitet\pend
           
\pstart
           Dem Comité läge nun \uline{ſehr} viel daran, wenn Sie,{ }ſehr
               geehrter Herr, dem Ausſchuſſe beitreten wollten. Wir glauben, daß unſer Unternehmen,
               an deſſen Beſtand und Wirkſamkeit (vom Herbſt an) nicht mehr zu rütteln{ }ſein wird, auch Ihren Wünſchen und Hoffnungen für das Theaterwesen\strikeout{s}{ }Wiens\oindex{Wien@\textbf{Wien}, \emph{Verwaltungsgebiet}|pw} entſprechen wird und würden es als Ehre und
               auch als große Freude empfinden, wenn Sie unſerem{ }ſchönen Beginnen Ihre freundliche
               Mithilfe widmen wollten.\pend
           
\pstart
           Eine conſtituierende Verſammlung des Ausſchuſſes{ }ſoll \uline{\textsc{Dienstag}} abends (gegen 10\textsuperscript{h}){ }ſtattfinden. Wenn Sie daran
               theilnehmen wollten, würden Sie uns zu großem Dank verpflichten. Auch iſt der
               Unterzeichnete gern bereit, Ihnen {\pb}– wenn Sie
               es wünſchen – die nöthigen Aufklärungen über das Detail des Werkes mitzutheilen.
               Soviel{ }ſei betont, daſs wir \uline{Mustervorſtellungen} zu
               machen gedenken und daſs uns vor Allem eine \uline{Erweiterung
                  des Spielplans} der W\textsuperscript{r}\oindex{Wien@\textbf{Wien}, \emph{Verwaltungsgebiet}|pw} Bühnen, die ja faſt durchwegs im Familienſtück zugrundegehen, unerläſslich
               erſcheint.\pend
           
\pstart
           Die freie Volksbühne\orgindex{Wiener Freie Volksbühne@Wiener Freie Volksbühne|pw} würde es{ }ſich zur Ehre
               rechnen, Ihren Namen unter den Begründern dieſes \strikeout{bühne} Unternehmens nennen zu dürfen.\pend
           
\pstart
           Ihrer freundlichen Antwort gewärtig, {\\[\baselineskip]}mit aller Hochſchätzung:{\\[\baselineskip]}i. A.{\\[\baselineskip]}\spacefill\mbox{Stefan Großmann}\pend
           \leftskip=0em{}
\pstart
           \noindent{}Wien I. Graben 29\textsuperscript{a}\oindex{Wien@\textbf{Wien}!I., Innere Stadt@\textbf{I., Innere Stadt}!Graben@\textbf{Graben}, \emph{Straße}|pw}\pend
           \selectlanguage{ngerman}\endnumbering\briefempfaengerindex{Schnitzler, Arthur@\textsc{Schnitzler, Arthur}!zzzGroßmann, Stefan@\emph{von Stefan Großmann}!1906-05-041@{4. 5. 1906}|)be}\mylabel{L01595h}  \newcommand{\dateiname}{L01595}\newcommand{\titel}{Stefan Großmann an Arthur Schnitzler, 4. 5. 1906}\newcommand{\editorInnen}{Herausgegeben von Martin Anton Müller}%% latex-leseansicht-abspann.tex
%% Abspann für die Leseansicht.
%% Der Schalter \ifkorrekturansicht ist bereits durch den Vorspann gesetzt.

%% latex-abspann.tex
%% Gemeinsamer Abspann für Korrekturansicht und Leseansicht.
%% Setzt den Schalter \ifkorrekturansicht voraus (gesetzt in den
%% einbindenden Dateien latex-korrekturansicht-abspann.tex bzw.
%% latex-leseansicht-abspann.tex).
%% ---------------------------------------------------------------

\normalsize

% Das esempio-Environment wird nur in der Leseansicht benötigt
\ifkorrekturansicht\else
\newenvironment{esempio}[3]%
{
    \vspace{1.5ex}
    \rlap{\underline{#1}}
    \par
    \setlength{\parindent}{0cm}
    \nopagebreak
    \leftskip=#2cm
    \rightskip=#3cm
}
{
    \par
}
\fi

\doendnotes{C}
\bigskip
\vfill

\clearpage

\footnotesize

\ifkorrekturansicht
  \lohead{\textsc{register}}
\fi

% theindex-Environment neu definieren ohne reledmac
\makeatletter
\renewenvironment{theindex}{%
  \ifkorrekturansicht
    \section*{\indexname}%
  \else
    \subsubsection*{Index der erwähnten Entitäten}%
  \fi
  \setlength{\parindent}{0pt}%
  \setlength{\parskip}{0pt plus 0.3pt}%
  \let\item\@idxitem
}{%
  \ifkorrekturansicht\clearpage\fi
}
\makeatother

\IfFileExists{\jobname-pw.ind}{\input{\jobname-pw.ind}}{}

% Quellenangabe nur in der Leseansicht
\ifkorrekturansicht\else
% Fallback-Definitionen, falls die .tex-Datei \titel etc. nicht gesetzt hat
\providecommand{\titel}{}
\providecommand{\editorInnen}{}
\providecommand{\dateiname}{\jobname}

\vspace{3cm}

\vfill

\footnotesize
\textsc{Quelle}: \titel. Herausgegeben von {\editorInnen}. In: \emph{Arthur Schnitzler: Briefwechsel mit Autorinnen und Autoren}.
 Digitale Edition, https://schnitzler-briefe.acdh.oeaw.ac.at/{\dateiname}.html (Stand \today)
\fi

\end{document}


