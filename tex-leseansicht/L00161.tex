%% latex-korrekturansicht-vorspann.tex
%% Vorspann für die Korrekturansicht.
%% Lädt die gemeinsame Datei latex-vorspann.tex mit gesetztem Schalter.

\newif\ifkorrekturansicht
\korrekturansichttrue

\input{../tex-inputs/latex-vorspann}


\section[Karl Kraus an Arthur Schnitzler, 22. 1. 1893]{L00161 Karl Kraus an Arthur Schnitzler, 22. 1. 1893}
\nopagebreak\mylabel{L00161v}
\rehead{ }\normalsize\beginnumbering\briefempfaengerindex{Schnitzler, Arthur@\textsc{Schnitzler, Arthur}!zzzKraus, Karl@\emph{von Karl Kraus}!1893-01-221@{22. 1. 1893}|(be}
\toendnotes[C]{\smallbreak\pagebreak[2]}\Standort{CUL, Schnitzler, B 55.}
\physDesc{Briefkarte, 891 Zeichen
\newline{}Handschrift: schwarze Tinte, deutsche Kurrent}
\buchAbdrucke{\weitereDrucke{1) \emph{Literatur und Kritik}, Bd. 49, Oktober 1970, S. 514–515.} \weitereDrucke{2) Hermann Bahr, Arthur Schnitzler: \emph{Briefwechsel, Aufzeichnungen, Dokumente (1891–1931)}. Göttingen: \emph{Wallstein} 2018, S. 32.} }\toendnotes[C]{\smallbreak}
\pstart
           {\pb}Wien\oindex{Wien@\textbf{Wien}, \emph{A.ADM2}|pw}, 22/\textsubscript{I} 93.\pend
           \vspace{0.5em}
\pstart
           Lieber Herr Doctor! Bin grade in einer Hochzeit drin; beeile mich
               aber trotzdem Ihren lieben Brief, den ich eben erhielt, zu beantworten; ich hatte
               nämlich gleich nachm. für Sie Kritikauschnitt vorbereitet u. dazu ein Briefchen
               geſchrieben, welches ich nun freilich nicht benutzen kann.\pend
           
\pstart
           Alſo ich bin in der angenehmen Lage, Ihnen einen Ausschnitt bereits heute verschaffen
               zu können. Anbei ist er.\pend
           
\pstart
           {\pb}Haben Sie zufällig Fr. Bühne\pwindex{Freie Buehne fuer den Entwickelungskampf der Zeit@\emph{Freie Bühne für den Entwickelungskampf der Zeit}|pw} Januarheft in die Hand bekommen?\pend
           
\pstart
           Leſen Sie den \label{K_L00161-1v}\edtext{Artikel\pwindex{Von Hermann Bahr und seiner Buecherei@\emph{Von Hermann Bahr und seiner Bücherei}|pwv}}{\lemma{\textnormal{\emph{Artikel}}}\Cendnote{\textnormal{Felix Hollaender\pwindex{Hollaender, Felix 01.11.1867 – 29.05.1931@\textsc{Hollaender, Felix} (01.11.1867 – 29.05.1931), \emph{Schriftsteller/Schriftstellerin, Theaterleiter/Theaterleiterin, Regisseur/Regisseurin}|pwk}: \emph{Von Hermann Bahr und seiner Bücherei}\pwindex{Von Hermann Bahr und seiner Buecherei@\emph{Von Hermann Bahr und seiner Bücherei}|pwk}. In: \emph{Freie Bühne}\pwindex{Freie Buehne fuer den Entwickelungskampf der Zeit@\emph{Freie Bühne für den Entwickelungskampf der Zeit}|pwk}, Jg. 4, Nr. 1,
                        1. 1. 1893, S. 82–89.}}}\label{K_L00161-1} von \introOben{}F.\introOben{}{ }Holländer\pwindex{Hollaender, Felix 01.11.1867 – 29.05.1931@\textsc{Hollaender, Felix} (01.11.1867 – 29.05.1931), \emph{Schriftsteller/Schriftstellerin, Theaterleiter/Theaterleiterin, Regisseur/Regisseurin}|pw} über \uline{Hermann Bahr}\pwindex{Bahr, Hermann 19.07.1863 – 15.01.1934@\textsc{Bahr, Hermann} (19.07.1863 – 15.01.1934), \emph{Schriftsteller/Schriftstellerin, Kritiker/Kritikerin}|pw}, den er in geradezu dummer Weiſe in den Himmel hebt. Dort finden Sie bei der
               Stelle über Bahr\pwindex{Bahr, Hermann 19.07.1863 – 15.01.1934@\textsc{Bahr, Hermann} (19.07.1863 – 15.01.1934), \emph{Schriftsteller/Schriftstellerin, Kritiker/Kritikerin}|pw}’s \label{K_L00161-2v}\edtext{Dora\pwindex{Dora@\emph{Dora}|pw}-Schmarren}{\lemma{\textnormal{\emph{Dora-Schmarren}}}\Cendnote{\textnormal{Hermann Bahr\pwindex{Bahr, Hermann 19.07.1863 – 15.01.1934@\textsc{Bahr, Hermann} (19.07.1863 – 15.01.1934), \emph{Schriftsteller/Schriftstellerin, Kritiker/Kritikerin}|pwk}: \emph{Dora}\pwindex{Dora@\emph{Dora}|pwk}. Berlin: \emph{S. Fischer}\orgindex{S. Fischer Verlag@S. Fischer Verlag|pwk}{ }1893 (erschienen November 1892). Schmarren,
                  hier: Unsinn.}}}\label{K_L00161-2}, den Holl.\pwindex{Hollaender, Felix 01.11.1867 – 29.05.1931@\textsc{Hollaender, Felix} (01.11.1867 – 29.05.1931), \emph{Schriftsteller/Schriftstellerin, Theaterleiter/Theaterleiterin, Regisseur/Regisseurin}|pw} für das
               größte psycholog. Kunſtwerk hält (!!!!), eine \uline{ſehr,
                  ſehr}{ }\label{K_L00161-3v}\edtext{ſchmeichelhafte Bemerkung}{\lemma{\textnormal{\emph{ſchmeichelhafte Bemerkung}}}\Cendnote{\textnormal{S. 88: »Ich weiß bei uns Niemanden,
                     der nach diesem Büchlein\pwindex{Dora@\emph{Dora}|pwv}
                     sich mit Bahr\pwindex{Bahr, Hermann 19.07.1863 – 15.01.1934@\textsc{Bahr, Hermann} (19.07.1863 – 15.01.1934), \emph{Schriftsteller/Schriftstellerin, Kritiker/Kritikerin}|pw} messen könnte; in Oesterreich\oindex{Oesterreich@\textbf{Österreich}, \emph{A.PCLI}|pw} käme nur noch Arthur Schnitzler in
                  Betracht.«}}}\label{K_L00161-3} über einen gewiſſen Arthur Schnitzler. Verzeihen Sie
               mir, Liebster, den \label{K_L00161-4v}\edtext{Franz Moor\pwindex{Raeuber. Ein Schauspiel@\emph{Die Räuber. Ein Schauspiel}|pwv}}{\lemma{\textnormal{\emph{Franz Moor}}}\Cendnote{\textnormal{Vgl. A. S.: \emph{Tagebuch}, 14. 1. 1893.
               }}}\label{K_L00161-4}. Soll gewiss nimmer vorkommen! bitte, bitte! Viele Grüße\hspace*{3.5em}Ihr ſehr ergeb. \spacefill\mbox{Karl Kraus.}\pend
           \selectlanguage{ngerman}\endnumbering\briefempfaengerindex{Schnitzler, Arthur@\textsc{Schnitzler, Arthur}!zzzKraus, Karl@\emph{von Karl Kraus}!1893-01-221@{22. 1. 1893}|)be}\mylabel{L00161h}  \normalsize

\doendnotes{C}
\bigskip
\vfill

\clearpage

\footnotesize

\lohead{\textsc{register}}

% Definiere theindex-Environment komplett neu ohne reledmac
\makeatletter
\renewenvironment{theindex}{%
  \section*{\indexname}%
  \setlength{\parindent}{0pt}%
  \setlength{\parskip}{0pt plus 0.3pt}%
  \let\item\@idxitem
}{%
  \clearpage
}
\makeatother

\IfFileExists{\jobname-pw.ind}{\input{\jobname-pw.ind}}{}

\end{document}

      