%% latex-leseansicht-vorspann.tex
%% Vorspann für die Leseansicht.
%% Lädt die gemeinsame Datei latex-vorspann.tex mit nicht gesetztem Schalter.

\newif\ifkorrekturansicht
\korrekturansichtfalse

\input{../tex-inputs/latex-vorspann}


         
         \renewcommand{\erwaehntePersonen}{Personen: Hermann Bahr, Felix Hollaender}
         \renewcommand{\erwaehnteInstitutionen}{Institutionen: S. Fischer Verlag}
         \renewcommand{\erwaehnteOrte}{Orte: Wien, Österreich}
         \renewcommand{\erwaehnteWerke}{Werke: Die Räuber, Dora, Freie Bühne für den Entwickelungskampf der Zeit, Von Hermann Bahr und seiner Bücherei}
               \section[Karl Kraus an Arthur Schnitzler, 22. 1. 1893]{ Karl Kraus an Arthur Schnitzler, 22. 1. 1893}\nopagebreak\mylabel{v}\rehead{ }\begin{ledgroupsized}[t]{13cm}\normalsize\beginnumbering \toendnotes[C]{\smallbreak\pagebreak[2]} \Standort{CUL, Schnitzler, B 55.}
\physDesc{Briefkarte
\newline{}Handschrift: schwarze Tinte, deutsche Kurrent}\buchAbdrucke{\weitereDrucke{1) \emph{Karl Kraus und Arthur Schnitzler. Eine Dokumentation.} Hg. Reinhard Urbach. In: \emph{Literatur und Kritik}, Bd. 49, Oktober 1970, S. 514–515.} \weitereDrucke{2) Hermann Bahr, Arthur Schnitzler: \emph{Briefwechsel, Aufzeichnungen, Dokumente
                                (1891–1931)}. Hg. Kurt Ifkovits und Martin Anton Müller. Göttingen: \emph{Wallstein} 2018, S. 32.} }\toendnotes[C]{\smallbreak}\pstart
           {\pb}Wien\oindex{Wien@\textbf{Wien}|pw}, 22/\textsubscript{I} 93.\pend
           \pstart
           Lieber Herr Doctor! Bin grade in einer Hochzeit drin; beeile
                    mich aber trotzdem Ihren lieben Brief, den ich eben erhielt, zu beantworten; ich
                    hatte nämlich gleich nachm. für Sie Kritikauschnitt vorbereitet u. dazu ein
                    Briefchen geſchrieben, welches ich nun freilich nicht benutzen kann.\pend
           \pstart
           Alſo ich bin in der angenehmen Lage, Ihnen einen Ausschnitt bereits heute
                    verschaffen zu können. Anbei ist er.\pend
           \pstart
           {\pb}Haben Sie zufällig Fr. Bühne\pwindex{Freie Buehne fuer den Entwickelungskampf der Zeit1892 – 1893@\emph{Freie Bühne für den Entwickelungskampf der Zeit} {[}1892 – 1893{]}|pw} Januarheft in die Hand bekommen?\pend
           \pstart
           Leſen Sie den \label{K_L00161_1v}\edtext{Artikel\pwindex{Hollaender, Felix 01.11.1867 – 29.05.1931@\textsc{Hollaender, Felix} (01.11.1867 – 29.05.1931), \emph{Schriftsteller, Theaterleiter, Regisseur}!Von Hermann Bahr und seiner Buecherei01. 01. 1893@\strich\emph{Von Hermann Bahr und seiner Bücherei} {[}01. 01. 1893{]}|pwv}}{\lemma{\textnormal{\emph{Artikel}}}\Cendnote{\textnormal{Felix Hollaender\pwindex{Hollaender, Felix 01.11.1867 – 29.05.1931@\textsc{Hollaender, Felix} (01.11.1867 – 29.05.1931), \emph{Schriftsteller, Theaterleiter, Regisseur}|pwk}: \emph{Von Hermann Bahr und seiner Bücherei}\pwindex{Hollaender, Felix 01.11.1867 – 29.05.1931@\textsc{Hollaender, Felix} (01.11.1867 – 29.05.1931), \emph{Schriftsteller, Theaterleiter, Regisseur}!Von Hermann Bahr und seiner Buecherei01. 01. 1893@\strich\emph{Von Hermann Bahr und seiner Bücherei} {[}01. 01. 1893{]}|pwk}. In: \emph{Freie Bühne}\pwindex{Freie Buehne fuer den Entwickelungskampf der Zeit1892 – 1893@\emph{Freie Bühne für den Entwickelungskampf der Zeit} {[}1892 – 1893{]}|pwk}, Jg. 4, Nr. 1,
                                1. 1. 1893, S. 82–89.}}}\label{K_L00161_1h} von \introOben{}F.\introOben{}{ }Holländer\pwindex{Hollaender, Felix 01.11.1867 – 29.05.1931@\textsc{Hollaender, Felix} (01.11.1867 – 29.05.1931), \emph{Schriftsteller, Theaterleiter, Regisseur}|pw} über \uline{Hermann Bahr}\pwindex{Bahr, Hermann 19.07.1863 – 15.01.1934@\textsc{Bahr, Hermann} (19.07.1863 – 15.01.1934), \emph{Schriftsteller, Kritiker}|pw}, den er in geradezu dummer Weiſe in den Himmel hebt. Dort finden Sie bei
                    der Stelle über Bahr\pwindex{Bahr, Hermann 19.07.1863 – 15.01.1934@\textsc{Bahr, Hermann} (19.07.1863 – 15.01.1934), \emph{Schriftsteller, Kritiker}|pw}’s \label{K_L00161_2v}\edtext{Dora\pwindex{Bahr, Hermann 19.07.1863 – 15.01.1934@\textsc{Bahr, Hermann} (19.07.1863 – 15.01.1934), \emph{Schriftsteller, Kritiker}!Dora1892@\strich\emph{Dora} {[}1892{]}|pw}-Schmarren}{\lemma{\textnormal{\emph{Dora-Schmarren}}}\Cendnote{\textnormal{Hermann Bahr\pwindex{Bahr, Hermann 19.07.1863 – 15.01.1934@\textsc{Bahr, Hermann} (19.07.1863 – 15.01.1934), \emph{Schriftsteller, Kritiker}|pwk}: \emph{Dora}\pwindex{Bahr, Hermann 19.07.1863 – 15.01.1934@\textsc{Bahr, Hermann} (19.07.1863 – 15.01.1934), \emph{Schriftsteller, Kritiker}!Dora1892@\strich\emph{Dora} {[}1892{]}|pwk}. Berlin: \emph{S. Fischer}\orgindex{S. Fischer Verlag@S. Fischer Verlag|pwk}{ }1893 (erschienen November 1892).
                        Schmarren, hier: Unsinn.}}}\label{K_L00161_2h}, den Holl.\pwindex{Hollaender, Felix 01.11.1867 – 29.05.1931@\textsc{Hollaender, Felix} (01.11.1867 – 29.05.1931), \emph{Schriftsteller, Theaterleiter, Regisseur}|pw} für das größte psycholog. Kunſtwerk hält (!!!!), eine \uline{ſehr, ſehr}{ }\label{K_L00161_3v}\edtext{ſchmeichelhafte Bemerkung}{\lemma{\textnormal{\emph{ſchmeichelhafte Bemerkung}}}\Cendnote{\textnormal{S. 88: »Ich weiß bei uns
                            Niemanden, der nach diesem Büchlein\pwindex{Bahr, Hermann 19.07.1863 – 15.01.1934@\textsc{Bahr, Hermann} (19.07.1863 – 15.01.1934), \emph{Schriftsteller, Kritiker}!Dora1892@\strich\emph{Dora} {[}1892{]}|pwv}
                     sich mit Bahr\pwindex{Bahr, Hermann 19.07.1863 – 15.01.1934@\textsc{Bahr, Hermann} (19.07.1863 – 15.01.1934), \emph{Schriftsteller, Kritiker}|pw} messen könnte; in Oesterreich\oindex{Oesterreich@\textbf{Österreich}|pw} käme nur noch Arthur Schnitzler\pwindex{Schnitzler, Arthur 15.05.1862 – 21.10.1931@\textsc{Schnitzler, Arthur} (15.05.1862 – 21.10.1931), \emph{Schriftsteller, Mediziner}|pw} in Betracht.«}}}\label{K_L00161_3h}
                    über einen gewiſſen Arthur Schnitzler. Verzeihen Sie mir, Liebster, den \label{K_L00161_4v}\edtext{Franz Moor\pwindex{\textcolor{red}{\textsuperscript{XXXX1 indx}}!Raeuber1781@\strich\emph{Die Räuber} {[}1781{]}|pwv}}{\lemma{\textnormal{\emph{Franz Moor}}}\Cendnote{\textnormal{vgl. A. S.: \emph{Tagebuch}, 14. 1. 1893}}}\label{K_L00161_4h}. Soll gewiss nimmer
                    vorkommen! bitte, bitte! Viele Grüße\hspace*{3.5em}Ihr ſehr
                    ergeb. \spacefill\mbox{Karl Kraus}.\pend
           
         
         \endnumbering\mylabel{h}\end{ledgroupsized}  \newcommand{\dateiname}{L00161}\newcommand{\titel}{Karl Kraus an Arthur Schnitzler, 22. 1. 1893}\newcommand{\editorInnen}{ Martin Anton Müller und Gerd-Hermann Susen}%% latex-leseansicht-abspann.tex
%% Abspann für die Leseansicht.
%% Der Schalter \ifkorrekturansicht ist bereits durch den Vorspann gesetzt.

%% latex-abspann.tex
%% Gemeinsamer Abspann für Korrekturansicht und Leseansicht.
%% Setzt den Schalter \ifkorrekturansicht voraus (gesetzt in den
%% einbindenden Dateien latex-korrekturansicht-abspann.tex bzw.
%% latex-leseansicht-abspann.tex).
%% ---------------------------------------------------------------

\normalsize

% Das esempio-Environment wird nur in der Leseansicht benötigt
\ifkorrekturansicht\else
\newenvironment{esempio}[3]%
{
    \vspace{1.5ex}
    \rlap{\underline{#1}}
    \par
    \setlength{\parindent}{0cm}
    \nopagebreak
    \leftskip=#2cm
    \rightskip=#3cm
}
{
    \par
}
\fi

\doendnotes{C}
\bigskip
\vfill

\clearpage

\footnotesize

\ifkorrekturansicht
  \lohead{\textsc{register}}
\fi

% theindex-Environment neu definieren ohne reledmac
\makeatletter
\renewenvironment{theindex}{%
  \ifkorrekturansicht
    \section*{\indexname}%
  \else
    \subsubsection*{Index der erwähnten Entitäten}%
  \fi
  \setlength{\parindent}{0pt}%
  \setlength{\parskip}{0pt plus 0.3pt}%
  \let\item\@idxitem
}{%
  \ifkorrekturansicht\clearpage\fi
}
\makeatother

\IfFileExists{\jobname-pw.ind}{\input{\jobname-pw.ind}}{}

% Quellenangabe nur in der Leseansicht
\ifkorrekturansicht\else
% Fallback-Definitionen, falls die .tex-Datei \titel etc. nicht gesetzt hat
\providecommand{\titel}{}
\providecommand{\editorInnen}{}
\providecommand{\dateiname}{\jobname}

\vspace{3cm}

\vfill

\footnotesize
\textsc{Quelle}: \titel. Herausgegeben von {\editorInnen}. In: \emph{Arthur Schnitzler: Briefwechsel mit Autorinnen und Autoren}.
 Digitale Edition, https://schnitzler-briefe.acdh.oeaw.ac.at/{\dateiname}.html (Stand \today)
\fi

\end{document}


      