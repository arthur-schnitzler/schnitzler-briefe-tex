%% latex-leseansicht-vorspann.tex
%% Vorspann für die Leseansicht.
%% Lädt die gemeinsame Datei latex-vorspann.tex mit nicht gesetztem Schalter.

\newif\ifkorrekturansicht
\korrekturansichtfalse

\input{../tex-inputs/latex-vorspann}


\section[Karl Kraus an Arthur Schnitzler, 22. 1. 1893]{L00161 Karl Kraus an Arthur Schnitzler, 22. 1. 1893}
\nopagebreak\mylabel{L00161v}
\rehead{ }\normalsize\beginnumbering\briefempfaengerindex{Schnitzler, Arthur@\textsc{Schnitzler, Arthur}!zzzKraus, Karl@\emph{von Karl Kraus}!1893-01-221@{22. 1. 1893}|(be}
\toendnotes[C]{\smallbreak\pagebreak[2]}
\correspDesc{Versand  durch Karl Kraus am 22. 1. 1893 in Wien
\newline{}Erhalt  durch Arthur Schnitzler im Zeitraum [22. 1. 1893
                  – 26. 1. 1893?] in Wien}\toendnotes[C]{\smallbreak}
\Standort{CUL, Schnitzler, B 55.}
\physDesc{Briefkarte, 891 Zeichen
\newline{}Handschrift: schwarze Tinte, deutsche Kurrent}
\buchAbdrucke{\weitereDrucke{1) \emph{Karl Kraus und Arthur Schnitzler. Eine Dokumentation.}Herausgegeben von Reinhard Urbach In: \emph{Literatur und Kritik}, Bd. 49, Oktober 1970, S. 514–515.} \weitereDrucke{2) Hermann Bahr, Arthur Schnitzler: \emph{Briefwechsel, Aufzeichnungen, Dokumente (1891–1931)}. Herausgegeben von Kurt Ifkovits und Martin Anton Müller. Göttingen: \emph{Wallstein} 2018, S. 32.} }\toendnotes[C]{\smallbreak}
\pstart
           {\pb}Wien\oindex{Wien@\textbf{Wien}, \emph{Verwaltungsgebiet}|pw}, 22/\textsubscript{I} 93.\pend
           \vspace{0.5em}
\pstart
           Lieber Herr Doctor! Bin grade in einer Hochzeit drin; beeile mich
               aber trotzdem Ihren lieben Brief, den ich eben erhielt, zu beantworten; ich hatte
               nämlich gleich nachm. für Sie Kritikauschnitt vorbereitet u. dazu ein Briefchen
               geſchrieben, welches ich nun freilich nicht benutzen kann.\pend
           
\pstart
           Alſo ich bin in der angenehmen Lage, Ihnen einen Ausschnitt bereits heute verschaffen
               zu können. Anbei ist er.\pend
           
\pstart
           {\pb}Haben Sie zufällig Fr. Bühne\pwindex{Freie Bühne für den Entwickelungskampf der Zeit@\emph{Freie Bühne für den Entwickelungskampf der Zeit}|pw} Januarheft in die Hand bekommen?\pend
           
\pstart
           Leſen Sie den \label{K_L00161-1v}\edtext{Artikel\pwindex{Hollaender, Felix 1.\,11.\,1867 Głubczyce – 29.\,5.\,1931 Berlin@\textsc{Hollaender, Felix} (1.\,11.\,1867 Głubczyce – 29.\,5.\,1931 Berlin), \emph{Schriftsteller, Theaterleiter, Regisseur}!Von Hermann Bahr und seiner Bücherei@\strich\emph{Von Hermann Bahr und seiner Bücherei}|pwv}}{\lemma{\textnormal{\emph{Artikel}}}\Cendnote{\textnormal{Felix Hollaender\pwindex{Hollaender, Felix 1.\,11.\,1867 Głubczyce – 29.\,5.\,1931 Berlin@\textsc{Hollaender, Felix} (1.\,11.\,1867 Głubczyce – 29.\,5.\,1931 Berlin), \emph{Schriftsteller, Theaterleiter, Regisseur}|pwk}: \emph{Von Hermann Bahr und seiner Bücherei}\pwindex{Hollaender, Felix 1.\,11.\,1867 Głubczyce – 29.\,5.\,1931 Berlin@\textsc{Hollaender, Felix} (1.\,11.\,1867 Głubczyce – 29.\,5.\,1931 Berlin), \emph{Schriftsteller, Theaterleiter, Regisseur}!Von Hermann Bahr und seiner Bücherei@\strich\emph{Von Hermann Bahr und seiner Bücherei}|pwk}. In: \emph{Freie Bühne}\pwindex{Freie Bühne für den Entwickelungskampf der Zeit@\emph{Freie Bühne für den Entwickelungskampf der Zeit}|pwk}, Jg. 4, Nr. 1,
                        1. 1. 1893, S. 82–89.}}}\label{K_L00161-1} von \introOben{}F.\introOben{}{ }Holländer\pwindex{Hollaender, Felix 1.\,11.\,1867 Głubczyce – 29.\,5.\,1931 Berlin@\textsc{Hollaender, Felix} (1.\,11.\,1867 Głubczyce – 29.\,5.\,1931 Berlin), \emph{Schriftsteller, Theaterleiter, Regisseur}|pw} über \uline{Hermann Bahr}\pwindex{Bahr, Hermann 19.\,7.\,1863 Linz – 15.\,1.\,1934 München@\textsc{Bahr, Hermann} (19.\,7.\,1863 Linz – 15.\,1.\,1934 München), \emph{Schriftsteller, Kritiker}|pw}, den er in geradezu dummer Weiſe in den Himmel hebt. Dort finden Sie bei der
               Stelle über Bahr\pwindex{Bahr, Hermann 19.\,7.\,1863 Linz – 15.\,1.\,1934 München@\textsc{Bahr, Hermann} (19.\,7.\,1863 Linz – 15.\,1.\,1934 München), \emph{Schriftsteller, Kritiker}|pw}’s \label{K_L00161-2v}\edtext{Dora\pwindex{Bahr, Hermann 19.\,7.\,1863 Linz – 15.\,1.\,1934 München@\textsc{Bahr, Hermann} (19.\,7.\,1863 Linz – 15.\,1.\,1934 München), \emph{Schriftsteller, Kritiker}!Dora@\strich\emph{Dora}|pw}-Schmarren}{\lemma{\textnormal{\emph{Dora-Schmarren}}}\Cendnote{\textnormal{Hermann Bahr\pwindex{Bahr, Hermann 19.\,7.\,1863 Linz – 15.\,1.\,1934 München@\textsc{Bahr, Hermann} (19.\,7.\,1863 Linz – 15.\,1.\,1934 München), \emph{Schriftsteller, Kritiker}|pwk}: \emph{Dora}\pwindex{Bahr, Hermann 19.\,7.\,1863 Linz – 15.\,1.\,1934 München@\textsc{Bahr, Hermann} (19.\,7.\,1863 Linz – 15.\,1.\,1934 München), \emph{Schriftsteller, Kritiker}!Dora@\strich\emph{Dora}|pwk}. Berlin: \emph{S. Fischer}\orgindex{S. Fischer Verlag@S. Fischer Verlag|pwk}{ }1893 (erschienen November 1892). Schmarren,
                  hier: Unsinn.}}}\label{K_L00161-2}, den Holl.\pwindex{Hollaender, Felix 1.\,11.\,1867 Głubczyce – 29.\,5.\,1931 Berlin@\textsc{Hollaender, Felix} (1.\,11.\,1867 Głubczyce – 29.\,5.\,1931 Berlin), \emph{Schriftsteller, Theaterleiter, Regisseur}|pw} für das
               größte psycholog. Kunſtwerk hält (!!!!), eine \uline{ſehr,{ }ſehr}{ }\label{K_L00161-3v}\edtext{ſchmeichelhafte Bemerkung}{\lemma{\textnormal{\emph{schmeichelhafte Bemerkung}}}\Cendnote{\textnormal{S. 88: »Ich weiß bei uns Niemanden,
                     der nach diesem Büchlein\pwindex{Bahr, Hermann 19.\,7.\,1863 Linz – 15.\,1.\,1934 München@\textsc{Bahr, Hermann} (19.\,7.\,1863 Linz – 15.\,1.\,1934 München), \emph{Schriftsteller, Kritiker}!Dora@\strich\emph{Dora}|pwv}
                     sich mit Bahr\pwindex{Bahr, Hermann 19.\,7.\,1863 Linz – 15.\,1.\,1934 München@\textsc{Bahr, Hermann} (19.\,7.\,1863 Linz – 15.\,1.\,1934 München), \emph{Schriftsteller, Kritiker}|pw} messen könnte; in Oesterreich\oindex{Österreich@\textbf{Österreich}|pw} käme nur noch Arthur Schnitzler in
                  Betracht.«}}}\label{K_L00161-3} über einen gewiſſen Arthur Schnitzler. Verzeihen Sie
               mir, Liebster, den \label{K_L00161-4v}\edtext{Franz Moor\pwindex{\textcolor{red}{\textsuperscript{XXXX indx1}}!Räuber. Ein Schauspiel@\strich\emph{Die Räuber. Ein Schauspiel}|pwv}}{\lemma{\textnormal{\emph{Franz Moor}}}\Cendnote{\textnormal{Vgl. A. S.: \emph{Tagebuch}, 14. 1. 1893.
               }}}\label{K_L00161-4}. Soll gewiss nimmer vorkommen! bitte, bitte! Viele Grüße\hspace*{3.5em}Ihr{ }ſehr ergeb. \spacefill\mbox{Karl Kraus.}\pend
           \selectlanguage{ngerman}\endnumbering\briefempfaengerindex{Schnitzler, Arthur@\textsc{Schnitzler, Arthur}!zzzKraus, Karl@\emph{von Karl Kraus}!1893-01-221@{22. 1. 1893}|)be}\mylabel{L00161h}  \newcommand{\dateiname}{L00161}\newcommand{\titel}{Karl Kraus an Arthur Schnitzler, 22. 1. 1893}\newcommand{\editorInnen}{Herausgegeben von Martin Anton Müller}%% latex-leseansicht-abspann.tex
%% Abspann für die Leseansicht.
%% Der Schalter \ifkorrekturansicht ist bereits durch den Vorspann gesetzt.

%% latex-abspann.tex
%% Gemeinsamer Abspann für Korrekturansicht und Leseansicht.
%% Setzt den Schalter \ifkorrekturansicht voraus (gesetzt in den
%% einbindenden Dateien latex-korrekturansicht-abspann.tex bzw.
%% latex-leseansicht-abspann.tex).
%% ---------------------------------------------------------------

\normalsize

% Das esempio-Environment wird nur in der Leseansicht benötigt
\ifkorrekturansicht\else
\newenvironment{esempio}[3]%
{
    \vspace{1.5ex}
    \rlap{\underline{#1}}
    \par
    \setlength{\parindent}{0cm}
    \nopagebreak
    \leftskip=#2cm
    \rightskip=#3cm
}
{
    \par
}
\fi

\doendnotes{C}
\bigskip
\vfill

\clearpage

\footnotesize

\ifkorrekturansicht
  \lohead{\textsc{register}}
\fi

% theindex-Environment neu definieren ohne reledmac
\makeatletter
\renewenvironment{theindex}{%
  \ifkorrekturansicht
    \section*{\indexname}%
  \else
    \subsubsection*{Index der erwähnten Entitäten}%
  \fi
  \setlength{\parindent}{0pt}%
  \setlength{\parskip}{0pt plus 0.3pt}%
  \let\item\@idxitem
}{%
  \ifkorrekturansicht\clearpage\fi
}
\makeatother

\IfFileExists{\jobname-pw.ind}{\input{\jobname-pw.ind}}{}

% Quellenangabe nur in der Leseansicht
\ifkorrekturansicht\else
% Fallback-Definitionen, falls die .tex-Datei \titel etc. nicht gesetzt hat
\providecommand{\titel}{}
\providecommand{\editorInnen}{}
\providecommand{\dateiname}{\jobname}

\vspace{3cm}

\vfill

\footnotesize
\textsc{Quelle}: \titel. Herausgegeben von {\editorInnen}. In: \emph{Arthur Schnitzler: Briefwechsel mit Autorinnen und Autoren}.
 Digitale Edition, https://schnitzler-briefe.acdh.oeaw.ac.at/{\dateiname}.html (Stand \today)
\fi

\end{document}


