%% latex-leseansicht-vorspann.tex
%% Vorspann für die Leseansicht.
%% Lädt die gemeinsame Datei latex-vorspann.tex mit nicht gesetztem Schalter.

\newif\ifkorrekturansicht
\korrekturansichtfalse

\input{../tex-inputs/latex-vorspann}

\begin{center}
            \textcolor{red}{ENTWURF. ENTZIFFERUNG NOCH NICHT KORREKTURGELESEN}
                      \end{center}
            
               \section[Georg Brandes an Arthur Schnitzler, 5. 1. 1922]{ Georg Brandes an Arthur Schnitzler, 5. 1. 1922}\nopagebreak\mylabel{v}\rehead{ }\begin{ledgroupsized}[t]{13cm}\normalsize\beginnumbering\briefempfaengerindex{Schnitzler, Arthur@\textsc{Schnitzler, Arthur}!zzzBrandes, Georg@\emph{von Georg Brandes}!1922-01-051@{5. 1. 1922}|(be} \toendnotes[C]{\smallbreak\pagebreak[2]} \Standort{CUL, Schnitzler, B 17.}
\physDesc{Brief, 1 Blatt, 4 Seiten
\newline{}Handschrift: schwarze Tinte, lateinische Kurrent
\newline{}Schnitzler: mit rotem Buntstift vereinzelte Unterstreichungen \newline{}Ordnung: mit Bleistift von unbekannter Hand nummeriert:
                                    »52« }\buchAbdrucke{\weitereDrucke{Georg Brandes, Arthur Schnitzler: \emph{Ein Briefwechsel}. Hg. Kurt Bergel. Bern: \emph{Francke} 1956, S. 132–133.} }\toendnotes[C]{\smallbreak}\pstart
           \raggedleft{}{\pb}Kopenhagen\oindex{Kopenhagen@\textbf{Kopenhagen}|pw}{ }5 Januar 22\pend
           \pstart{}Verehrter lieber Freund\pend\pstart
           Es war mir eine Freude, von Ihnen zu hören, eine noch grössere, dass Sie jenes schon
               alte Buch\pwindex{Brandes, Georg 04.02.1842 – 19.02.1927@\textsc{Brandes, Georg} (04.02.1842 – 19.02.1927)!Wolfgang Goethe1915@\strich\emph{Wolfgang Goethe} {[}1915{]}|pwv}, das ich seit
                  1915 nie wieder angesehen habe, mit Befriedigung gelesen. Welcher
               Fluch für mich, eine Sprache zu schreiben, die Niemand versteht. Ich möchte Ihnen so
               gern die späteren Bücher, Voltaire\pwindex{Voltaire 21.11.1694 – 30.05.1778@\textsc{Voltaire} (21.11.1694 – 30.05.1778), \emph{Schriftsteller, Philosoph}|pw}\pwindex{Brandes, Georg 04.02.1842 – 19.02.1927@\textsc{Brandes, Georg} (04.02.1842 – 19.02.1927)!Voltaire und sein Jahrhundert1916 – 1917@\strich\emph{Voltaire und sein Jahrhundert} {[}1916 – 1917{]}|pwv}, Cäsar\pwindex{Caesar, Gaius Iulius 13.7.100? v. Chr. – 15.3.44 v. Chr.@\textsc{Caesar, Gaius Iulius} (13.7.100? v. Chr. – 15.3.44 v. Chr.), \emph{Politiker, Kaiser, Heerführer}|pw}\pwindex{Brandes, Georg 04.02.1842 – 19.02.1927@\textsc{Brandes, Georg} (04.02.1842 – 19.02.1927)!Gaius Julius Cæsar1918@\strich\emph{Gaius Julius Cæsar} {[}1918{]}|pwv}, Michelangelo\pwindex{Michelangelo, Buonarroti 6.03.1475 – 18.02.1564@\textsc{Michelangelo, Buonarroti} (6.03.1475 – 18.02.1564), \emph{Maler, Bildhauer, Architekt}|pw}\pwindex{Brandes, Georg 04.02.1842 – 19.02.1927@\textsc{Brandes, Georg} (04.02.1842 – 19.02.1927)!Michelangelo Buonarotti1921@\strich\emph{Michelangelo Buonarotti} {[}1921{]}|pwv} zugeschickt haben. Auch was ich in der letzten Zeit \label{K_L02373_1v}\edtext{über \uline{Homer}\pwindex{Homer @\textsc{Homer}, \emph{Schriftsteller}|pw}\pwindex{Brandes, Georg 04.02.1842 – 19.02.1927@\textsc{Brandes, Georg} (04.02.1842 – 19.02.1927)!Homer1921@\strich\emph{Homer} {[}1921{]}|pwv} geschrieben}{\lemma{\textnormal{\emph{über Homer geschrieben}}}\Cendnote{\textnormal{Die Rede, die er bei
                  der in Folge erwähnten großen Feier an der Universität\orgindex{Københavns Universitet@Københavns Universitet|pwkv} am 3. 11. 1921 über Homer\pwindex{Homer @\textsc{Homer}, \emph{Schriftsteller}|pwk} hielt, erschien zuerst als eigener Druck (\emph{Homer}\pwindex{Brandes, Georg 04.02.1842 – 19.02.1927@\textsc{Brandes, Georg} (04.02.1842 – 19.02.1927)!Homer1921@\strich\emph{Homer} {[}1921{]}|pwk}. Hg. \emph{Studentersamfundet}\orgindex{Studentersamfundet@Studentersamfundet|pwk} (in Kommission bei
                        \emph{Gyldendalske}) 1921) und wurde in
                  Folge in sein Buch \emph{Hellas}\pwindex{Brandes, Georg 04.02.1842 – 19.02.1927@\textsc{Brandes, Georg} (04.02.1842 – 19.02.1927)!Hellas1925@\strich\emph{Hellas} {[}1925{]}|pwk} (Kopenhagen:
                        \emph{Gyldendal, Nordisk forlag}{ }1925) integriert. Auf deutsch erschien der Aufsatz über Homer\pwindex{Homer @\textsc{Homer}, \emph{Schriftsteller}|pwk}{ }1927 im \emph{Philipp
                  Reclam}\orgindex{Philipp Reclam jun.@Philipp Reclam jun.|pwk}-Verlag.}}}\label{K_L02373_1h}.\pend
           \pstart
           Ich weiss nicht, ob Ihre Zeitungen davon gesprochen, dass (weil es am
                  3. November 50 Jahre her war, dass ich meine ersten Vorträge an der
                  Kopenhagener Universität\orgindex{Københavns Universitet@Københavns Universitet|pw} hielt) hier grosse Feier
               waren, Fackelzug der Studenten {\pb}und anderes. Es würde mich vor 40 Jahren sehr erfreut haben.\pend
           \pstart
           Am 15. Januar soll ich vor der Aufführung von Tartufe\pwindex{Moliere 14.01.1622 – 17.02.1673@\textsc{Molière} (14.01.1622 – 17.02.1673), \emph{Schriftsteller, Theaterleiter, Schauspieler}!Tartuffe1664@\strich\emph{Tartuffe} {[}1664{]}|pw} von der Bühne des Dagmar-Teaters\oindex{Dagmar Teatret@\textbf{Dagmar Teatret}|pw} über Molière\pwindex{Moliere 14.01.1622 – 17.02.1673@\textsc{Molière} (14.01.1622 – 17.02.1673), \emph{Schriftsteller, Theaterleiter, Schauspieler}|pw} reden. Am
                  19 wieder an die russischen\oindex{Russland@\textbf{Russland}|pw}
               Schauspieler französisch\oindex{Frankreich@\textbf{Frankreich}|pw} reden.\pend
           \pstart
           Dann verschwinde ich Ende dieses Monats für einige Zeit. Ich will mich wahrlich nicht
               zu meinem 80 Geburtstag Glück wünschen lassen. Die Lächerlichkeit wäre zu gross.\pend
           \pstart
           Ich las hier einmal im Herbst in einer Zeitung ein
               Interview\pwindex{?? [daenischer Journalist, der Schnitzler interviewt] @\textsc{?? [dänischer Journalist, der Schnitzler interviewt]}!?? [nicht ermitteltes daenisches Interview]Herbst 1921@\strich\emph{?? [nicht ermitteltes dänisches Interview]} {[}Herbst 1921{]}|pwv} eines mir
               unbekannten { }dänischen Journalisten\pwindex{?? [daenischer Journalist, der Schnitzler interviewt] @\textsc{?? [dänischer Journalist, der Schnitzler interviewt]}|pwv} mit
               Ihnen, worin Sie sehr \label{K_L02373-1v}\edtext{freundliche
               Worte}{\lemma{\textnormal{\emph{freundliche
               Worte}}}\Cendnote{\textnormal{nicht nachgewiesen}}}\label{K_L02373-1h} über mich
               sagten, ich glaube die freundlichsten, die in jenem Blatte je über mich gestanden
               haben.\pend
           \pstart
           Ich bleibe Ihnen immer verpflichtet und {\pb}verbunden. Der Genuss, den ich
               durch das Lesen Ihrer Werke gehabt habe, ist hundert Mal grösser als das mögliche
               Vergnügen, das Sie durch meine nur belehrenden Bücher gehabt haben können.\pend
           \pstart
           Ich sah durch dies Interview\pwindex{\textcolor{red}{\textsuperscript{XXXX1 indx}}!Arthur Schnitzler. En Samtale med en beremt Wiener [1921]10. 04. 1921@\strich\emph{Arthur Schnitzler. En Samtale med en beremt Wiener [1921]} {[}10. 04. 1921{]}|pwv}, wie
               viel Unannehmlichkeiten Sie durch das alte, nur scherzhafte und witzige, \uline{Reigen}\pwindex{Schnitzler, Arthur 15.05.1862 – 21.10.1931@\textsc{Schnitzler, Arthur} (15.05.1862 – 21.10.1931), \emph{Schriftsteller, Mediziner}!Reigen. Zehn Dialoge1900@\strich\emph{Reigen. Zehn Dialoge} {[}1900{]}|pw} gehabt haben. Der jetzt überall glühende Antisemitismus und die Tugendbolderei
                  \introOben{}geben\introOben{} im Verein \strikeout{geben}
               solche Resultate. Als ob die Menschen durch die Umstände dieser Zeit nicht genug
               litten, gehen sie mit zehnfachem Eifer darauf los, sich gegenseitig das Leben noch
               saurer zu machen.\pend
           \pstart
           Ich habe immer Wien\oindex{Wien@\textbf{Wien}|pw} in meinen Gedanken, immer mit
               Mitleid, Trauer und Dankbarkeit. Können Sie verstehen, das unser Freund Beer-Hofmann\pwindex{Beer-Hofmann, Richard 11.07.1866 – 26.09.1945@\textsc{Beer-Hofmann, Richard} (11.07.1866 – 26.09.1945), \emph{Schriftsteller}|pw} sich {\pb}mit solcher Leidenschaft an das
               Judenthum krampft. Es hat mich im Grunde nie interessiert; nur wenn die Juden
               verfolgt wurden, und wenn sie es werden, habe ich für sie heisses Mitgefühl, wie für
               alle ungerecht unterdrückten. Ich kenne nicht einen einzigen hebräischen
               Buchstaben. – Es scheint mir auch von ihm so \uline{gewollt}.\pend
           \pstart
           Ich denke mir, Sie haben sich in den späteren Jahren mit Casanova\pwindex{Casanova, Giacomo Girolamo 02.04.1725 – 04.06.1798@\textsc{Casanova, Giacomo Girolamo} (02.04.1725 – 04.06.1798), \emph{Schriftsteller, Abenteurer}|pw} beschäftigt, am meisten um sich nicht mit dem
               Gegenwärtigen herumzuschlagen. Wenn der Einzelne seine Ohnmacht fühlt, nützt es ja
               nichts mitzureden. Deshalb schweige ich selbst, wo ich viel zu sagen hätte. Ich habe
               nicht Frithiof Nansens\pwindex{Nansen, Fridtjof 10.10.1861 – 30.05.1930@\textsc{Nansen, Fridtjof} (10.10.1861 – 30.05.1930), \emph{Forscher}|pw} praktische Begabung so
               wenig wie sein Ansehen. Er ist durch den Krieg sehr gewachsen.\pend
           \pstart
           Ich bitte Sie Ihrer Frau Gemahlin\pwindex{Schnitzler, Olga 17.01.1882 – 13.01.1970@\textsc{Schnitzler, Olga} (17.01.1882 – 13.01.1970), \emph{Schauspielerin, Sängerin}|pwv} meine Huldigung, Ihren Kindern\pwindex{Schnitzler, Heinrich 09.08.1902 – 12.07.1982@\textsc{Schnitzler, Heinrich} (09.08.1902 – 12.07.1982), \emph{Regisseur, Schauspieler}|pwv}\pwindex{Schnitzler, Lili 13.09.1909 – 26.07.1928@\textsc{Schnitzler, Lili} (13.09.1909 – 26.07.1928)|pwv} meine Sympathie zu überbringen.\pend
           \pstart
           Ihr Freund{\\[\baselineskip]}\spacefill\mbox{Georg Brandes}\pend
           \leftskip=0em{}\endnumbering\briefempfaengerindex{Schnitzler, Arthur@\textsc{Schnitzler, Arthur}!zzzBrandes, Georg@\emph{von Georg Brandes}!1922-01-051@{5. 1. 1922}|)be}\mylabel{h}\end{ledgroupsized}  \newcommand{\dateiname}{L02373}\newcommand{\titel}{Georg Brandes an Arthur Schnitzler, 5. 1. 1922}\newcommand{\editorInnen}{Martin Anton Müller und Gerd-Hermann Susen}%% latex-leseansicht-abspann.tex
%% Abspann für die Leseansicht.
%% Der Schalter \ifkorrekturansicht ist bereits durch den Vorspann gesetzt.

%% latex-abspann.tex
%% Gemeinsamer Abspann für Korrekturansicht und Leseansicht.
%% Setzt den Schalter \ifkorrekturansicht voraus (gesetzt in den
%% einbindenden Dateien latex-korrekturansicht-abspann.tex bzw.
%% latex-leseansicht-abspann.tex).
%% ---------------------------------------------------------------

\normalsize

% Das esempio-Environment wird nur in der Leseansicht benötigt
\ifkorrekturansicht\else
\newenvironment{esempio}[3]%
{
    \vspace{1.5ex}
    \rlap{\underline{#1}}
    \par
    \setlength{\parindent}{0cm}
    \nopagebreak
    \leftskip=#2cm
    \rightskip=#3cm
}
{
    \par
}
\fi

\doendnotes{C}
\bigskip
\vfill

\clearpage

\footnotesize

\ifkorrekturansicht
  \lohead{\textsc{register}}
\fi

% theindex-Environment neu definieren ohne reledmac
\makeatletter
\renewenvironment{theindex}{%
  \ifkorrekturansicht
    \section*{\indexname}%
  \else
    \subsubsection*{Index der erwähnten Entitäten}%
  \fi
  \setlength{\parindent}{0pt}%
  \setlength{\parskip}{0pt plus 0.3pt}%
  \let\item\@idxitem
}{%
  \ifkorrekturansicht\clearpage\fi
}
\makeatother

\IfFileExists{\jobname-pw.ind}{\input{\jobname-pw.ind}}{}

% Quellenangabe nur in der Leseansicht
\ifkorrekturansicht\else
% Fallback-Definitionen, falls die .tex-Datei \titel etc. nicht gesetzt hat
\providecommand{\titel}{}
\providecommand{\editorInnen}{}
\providecommand{\dateiname}{\jobname}

\vspace{3cm}

\vfill

\footnotesize
\textsc{Quelle}: \titel. Herausgegeben von {\editorInnen}. In: \emph{Arthur Schnitzler: Briefwechsel mit Autorinnen und Autoren}.
 Digitale Edition, https://schnitzler-briefe.acdh.oeaw.ac.at/{\dateiname}.html (Stand \today)
\fi

\end{document}


      