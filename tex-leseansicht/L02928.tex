%% latex-leseansicht-vorspann.tex
%% Vorspann für die Leseansicht.
%% Lädt die gemeinsame Datei latex-vorspann.tex mit nicht gesetztem Schalter.

\newif\ifkorrekturansicht
\korrekturansichtfalse

\input{../tex-inputs/latex-vorspann}

\begin{center}
            \textcolor{red}{ENTWURF, NICHT FERTIG KORRIGIERT}
                      \end{center}
            
         
         \renewcommand{\erwaehntePersonen}{Personen: Richard Beer-Hofmann}
         \renewcommand{\erwaehnteOrte}{Orte: Hotel Trafoi, Tirol, Trafoi, Wien}
         \renewcommand{\erwaehnteWerke}{Werke: Der blinde Geronimo und sein Bruder, Der blinde Musikant}
               \section[ Paul Goldmann an Arthur Schnitzler, 28. 8. {[}1900{]}]{ Paul Goldmann an Arthur Schnitzler, 28. 8. {[}1900{]}}\nopagebreak\mylabel{v}\rehead{ }\begin{ledgroupsized}[t]{13cm}\normalsize\beginnumbering \toendnotes[C]{\smallbreak\pagebreak[2]} \Standort{DLA, A:Schnitzler, HS.NZ85.1.3170.}
\physDesc{Brief, 1 Blatt, 2 Seiten
\newline{}Handschrift: schwarze Tinte, deutsche Kurrent
\newline{}Schnitzler: mit Bleistift das Jahr »{[}1{]}900.« vermerkt }\toendnotes[C]{\smallbreak}\pstart
           \raggedleft{}{\pb}\textcolor{gray}{\textbf{HOTEL TRAFOI\oindex{Hotel Trafoi@\textbf{Hotel Trafoi}|pwv}{ }TIROL\oindex{Tirol@\textbf{Tirol}|pw}.}}{ }28. Auguſt.\pend
           \stanza{}\label{K_L02928-1v}\edtext{\uline{Der blinde Muſikant.}}{\lemma{\textnormal{\emph{Der blinde Muſikant.}}}\Cendnote{\textnormal{Es ist davon auszugehen, dass eine
                        wahre Begegnung mit einem (blinden?) Sänger dieses Gedicht\pwindex{Goldmann, Paul 31.01.1865 – 25.09.1935@\textsc{Goldmann, Paul} (31.01.1865 – 25.09.1935), \emph{Schriftsteller, Journalist}!blinde Musikant1900-08-28@\strich\emph{Der blinde Musikant} {[}1900-08-28{]}|pwkv} inspiriert hatte. Schnitzler\pwindex{Schnitzler, Arthur 15.05.1862 – 21.10.1931@\textsc{Schnitzler, Arthur} (15.05.1862 – 21.10.1931), \emph{Schriftsteller, Mediziner}|pwk} und Goldmann\pwindex{Goldmann, Paul 31.01.1865 – 25.09.1935@\textsc{Goldmann, Paul} (31.01.1865 – 25.09.1935), \emph{Schriftsteller, Journalist}|pwk} hatten von einem »Tirol\oindex{Tirol@\textbf{Tirol}|pw}er Sänger« bereits zwei
                        Tage zuvor an Richard Beer-Hofmann\pwindex{Beer-Hofmann, Richard 1866-07-11 – 1945-09-26@\textsc{Beer-Hofmann, Richard} (1866-07-11 – 1945-09-26), \emph{Schriftsteller}|pwk}
                        geschrieben (Arthur Schnitzler und Paul Goldmann an Richard Beer-Hofmann,
                    26. 8. 1900). Da in diesem
                           Gedicht\pwindex{Goldmann, Paul 31.01.1865 – 25.09.1935@\textsc{Goldmann, Paul} (31.01.1865 – 25.09.1935), \emph{Schriftsteller, Journalist}!blinde Musikant1900-08-28@\strich\emph{Der blinde Musikant} {[}1900-08-28{]}|pwkv} explizit
                        von einem blinden Sänger die Rede ist, kann noch einmal mehr vermutet
                        werden, dass die Begegnung mit dem »Tirol\oindex{Tirol@\textbf{Tirol}|pwk}er Sänger« die Novelle \emph{Der
                           blinde Geronimo}\pwindex{Schnitzler, Arthur 15.05.1862 – 21.10.1931@\textsc{Schnitzler, Arthur} (15.05.1862 – 21.10.1931), \emph{Schriftsteller, Mediziner}!blinde Geronimo und sein Bruder22.12.1900 – 12.1.1901@\strich\emph{Der blinde Geronimo und sein Bruder} {[}22.12.1900 – 12.1.1901{]}|pwk} inspirierte. XXXX sobald 1901 angelegt: auf S. 21 in
                        Transkribus bzw. den entsprechenden Brief verweisen, dort ist der ›Beweis‹
                        zur Vorlage}}}\label{K_L02928-1h}\stanzaend{}\stanza{}Ein altes Haus auf \textcolor{gray}{Pa}ſſes Höh’n\newverse{}Beſchloß die erſte Strecke;\newverse{}Da klang Harmonika-Getön\newverse{}Hervor aus dunkler Ecke.\stanzaend{}\stanza{}Gelehnt an regenfeuchte Wand,\newverse{}Von Kälte ſtarr die Glieder,\newverse{}Stand dort ein blinder Muſikant\newverse{}Und ſpielte ſeine Lieder.\stanzaend{}\stanza{}Er ſpielte und ſein Auge war\newverse{}Gerichtet ſtarr nach oben\newverse{}Und wurde doch kein Licht gewahr,\newverse{}So hoch es auch erhoben.\stanzaend{}\stanza{}{\pb}Er ſpielte luſt’ge Melodie’n\newverse{}Und ſang dazu ganz ſachte;\newverse{}Das Singen faſt ein Weinen ſchien,\newverse{}Nur daß er dazu lachte.\stanzaend{}\stanza{}Wie thut mir Deine bitt’re Noth,\newverse{}Du armer Mann, ſo wehe!\newverse{}Du mit den Augen leer und todt,\newverse{}Verzeih’ mir, daß ich ſehe!\stanzaend{}\stanza{}Bin ich gleich ſehend, ſeh’ ich \strikeout{ih\textcolor{gray}{n}} nicht,\newverse{}Du kannſt mir leicht vergeben.\newverse{}Das Licht, das heißgeliebte Licht,\newverse{}Ich ſuch’s im dunklen Leben.\stanzaend{}\stanza{}Und ſuch’ es heut und immerzu\newverse{}Und ſeh’ es nimmer gleißen.\newverse{}Oh armer blinder Bettler Du,\newverse{}Du ſollſt mich Bruder heißen! {\dotssix}\stanzaend{}\stanza{}Der Wagen rollet aus dem Thor,\newverse{}Klimmt dann auf ſteilem Pfade,\newverse{}Und lange klingt mir noch im Ohr\newverse{}Die Jammer-Serenade.\stanzaend{}\pstart
           Gruß! {\\[\baselineskip]}\spacefill\mbox{P. G.}\pend
           \leftskip=0em{}
         
         \endnumbering\mylabel{h}\end{ledgroupsized}\begin{anhang}\end{anhang}\newcommand{\dateiname}{L02928}\newcommand{\titel}{Paul Goldmann an Arthur Schnitzler, 28. 8. [1900]}\newcommand{\editorInnen}{Martin Anton Müller und Laura Untner}%% latex-leseansicht-abspann.tex
%% Abspann für die Leseansicht.
%% Der Schalter \ifkorrekturansicht ist bereits durch den Vorspann gesetzt.

%% latex-abspann.tex
%% Gemeinsamer Abspann für Korrekturansicht und Leseansicht.
%% Setzt den Schalter \ifkorrekturansicht voraus (gesetzt in den
%% einbindenden Dateien latex-korrekturansicht-abspann.tex bzw.
%% latex-leseansicht-abspann.tex).
%% ---------------------------------------------------------------

\normalsize

% Das esempio-Environment wird nur in der Leseansicht benötigt
\ifkorrekturansicht\else
\newenvironment{esempio}[3]%
{
    \vspace{1.5ex}
    \rlap{\underline{#1}}
    \par
    \setlength{\parindent}{0cm}
    \nopagebreak
    \leftskip=#2cm
    \rightskip=#3cm
}
{
    \par
}
\fi

\doendnotes{C}
\bigskip
\vfill

\clearpage

\footnotesize

\ifkorrekturansicht
  \lohead{\textsc{register}}
\fi

% theindex-Environment neu definieren ohne reledmac
\makeatletter
\renewenvironment{theindex}{%
  \ifkorrekturansicht
    \section*{\indexname}%
  \else
    \subsubsection*{Index der erwähnten Entitäten}%
  \fi
  \setlength{\parindent}{0pt}%
  \setlength{\parskip}{0pt plus 0.3pt}%
  \let\item\@idxitem
}{%
  \ifkorrekturansicht\clearpage\fi
}
\makeatother

\IfFileExists{\jobname-pw.ind}{\input{\jobname-pw.ind}}{}

% Quellenangabe nur in der Leseansicht
\ifkorrekturansicht\else
% Fallback-Definitionen, falls die .tex-Datei \titel etc. nicht gesetzt hat
\providecommand{\titel}{}
\providecommand{\editorInnen}{}
\providecommand{\dateiname}{\jobname}

\vspace{3cm}

\vfill

\footnotesize
\textsc{Quelle}: \titel. Herausgegeben von {\editorInnen}. In: \emph{Arthur Schnitzler: Briefwechsel mit Autorinnen und Autoren}.
 Digitale Edition, https://schnitzler-briefe.acdh.oeaw.ac.at/{\dateiname}.html (Stand \today)
\fi

\end{document}


      