%% latex-leseansicht-vorspann.tex
%% Vorspann für die Leseansicht.
%% Lädt die gemeinsame Datei latex-vorspann.tex mit nicht gesetztem Schalter.

\newif\ifkorrekturansicht
\korrekturansichtfalse

\input{../tex-inputs/latex-vorspann}


\section[ Paul Goldmann an Arthur Schnitzler, 28. 8. [1900]]{L02928 Paul Goldmann an Arthur Schnitzler,  28. 8. [1900]}
\nopagebreak\mylabel{L02928v}
\rehead{ }\normalsize\beginnumbering\briefempfaengerindex{Schnitzler, Arthur@\textsc{Schnitzler, Arthur}!zzzGoldmann, Paul@\emph{von Paul Goldmann}!1900-08-281@{28. 8. [1900]}|(be}
\toendnotes[C]{\smallbreak\pagebreak[2]}
\correspDesc{Versand  durch Paul Goldmann am 28. 8. [1900] in Trafoi
\newline{}Erhalt  durch Arthur Schnitzler im Zeitraum [29. 8. 1900
                  – 30. 8. 1900?] in Meran?}\toendnotes[C]{\smallbreak}
\Standort{DLA, A:Schnitzler, HS.NZ85.1.3170.}
\physDesc{Brief, 1 Blatt, 2 Seiten, 960 Zeichen
\newline{}Handschrift: schwarze Tinte, deutsche Kurrent
\newline{}Schnitzler: mit Bleistift das Jahr »900.« vermerkt }\toendnotes[C]{\smallbreak}
\pstart
           \raggedleft{}{\pb}\textcolor{gray}{\textbf{HOTEL TRAFOI\oindex{Hotel Trafoi@\textbf{Hotel Trafoi}, \emph{Hotel}|pwv}}}\pend
           
\pstart
           \raggedleft{}\textcolor{gray}{\textbf{TIROL\oindex{Tirol@\textbf{Tirol}, \emph{Land}|pw}.}}\pend
           
\pstart
           \raggedleft{}28. Auguſt.\pend
           \vspace{0.5em}\stanza{}\label{K_L02928-1v}\edtext{\uline{Der blinde Muſikant\pwindex{Goldmann, Paul 31.\,1.\,1865 Breslau – 25.\,9.\,1935 Wien@\textsc{Goldmann, Paul} (31.\,1.\,1865 Breslau – 25.\,9.\,1935 Wien), \emph{Schriftsteller, Journalist}!blinde Musikant@\strich\emph{Der blinde Musikant}|pw}.}}{\lemma{\textnormal{\emph{Der blinde Musikant.}}}\Cendnote{\textnormal{Bereits zwei Tage zuvor schrieben
                           Schnitzler und Goldmann\pwindex{Goldmann, Paul 31.\,1.\,1865 Breslau – 25.\,9.\,1935 Wien@\textsc{Goldmann, Paul} (31.\,1.\,1865 Breslau – 25.\,9.\,1935 Wien), \emph{Schriftsteller, Journalist}|pwk} an Richard Beer-Hofmann\pwindex{Beer-Hofmann, Richard 11.\,7.\,1866 Wien – 26.\,9.\,1945 New York City@\textsc{Beer-Hofmann, Richard} (11.\,7.\,1866 Wien – 26.\,9.\,1945 New York City), \emph{Schriftsteller}|pwk} von einem »Tirol\oindex{Tirol@\textbf{Tirol}, \emph{Land}|pw}er Sänger«. (XXXX Auszeichnungsfehler: Dokument L01068 nicht gefunden.) Dass es sich bei der
                        Begegnung nicht nur um den Textimpuls für dieses Gedicht\pwindex{Goldmann, Paul 31.\,1.\,1865 Breslau – 25.\,9.\,1935 Wien@\textsc{Goldmann, Paul} (31.\,1.\,1865 Breslau – 25.\,9.\,1935 Wien), \emph{Schriftsteller, Journalist}!blinde Musikant@\strich\emph{Der blinde Musikant}|pwkv}, sondern auch für die
                        Novelle \emph{Der blinde Geronimo und sein
                           Bruder}\pwindex{Schnitzler, Arthur 15.\,5.\,1862 Wien – 21.\,10.\,1931 ebd.@\textsc{Schnitzler, Arthur} (15.\,5.\,1862 Wien – 21.\,10.\,1931 ebd.), \emph{Schriftsteller, Mediziner}!blinde Geronimo und sein Bruder@\strich\emph{Der blinde Geronimo und sein Bruder}|pwk} handelt, geht aus Goldmanns\pwindex{Goldmann, Paul 31.\,1.\,1865 Breslau – 25.\,9.\,1935 Wien@\textsc{Goldmann, Paul} (31.\,1.\,1865 Breslau – 25.\,9.\,1935 Wien), \emph{Schriftsteller, Journalist}|pwk} Brief vom XXXX Auszeichnungsfehler: Dokument L03059 nicht gefunden hervor, in dem Schnitzlers{ }Novelle\pwindex{Schnitzler, Arthur 15.\,5.\,1862 Wien – 21.\,10.\,1931 ebd.@\textsc{Schnitzler, Arthur} (15.\,5.\,1862 Wien – 21.\,10.\,1931 ebd.), \emph{Schriftsteller, Mediziner}!blinde Geronimo und sein Bruder@\strich\emph{Der blinde Geronimo und sein Bruder}|pwkv} als gegenüber der Vorlage fahl kritisiert wird. Siehe dazu
                        auch Paul Goldmann\pwindex{Goldmann, Paul 31.\,1.\,1865 Breslau – 25.\,9.\,1935 Wien@\textsc{Goldmann, Paul} (31.\,1.\,1865 Breslau – 25.\,9.\,1935 Wien), \emph{Schriftsteller, Journalist}|pwk}: \emph{Erinnerungen an Arthur Schnitzler}\pwindex{Goldmann, Paul 31.\,1.\,1865 Breslau – 25.\,9.\,1935 Wien@\textsc{Goldmann, Paul} (31.\,1.\,1865 Breslau – 25.\,9.\,1935 Wien), \emph{Schriftsteller, Journalist}!Erinnerungen an Arthur Schnitzler@\strich\emph{Erinnerungen an Arthur Schnitzler}|pwk}. In: \emph{Neue Freie Presse}\pwindex{Neue Freie Presse@\emph{Neue Freie Presse}|pwk}, Nr. 24.121, 8. 11. 1931, Morgenblatt, S. 25–26, hier:
                           S. 26.}}}\label{K_L02928-1}\stanzaend{}\stanza{}Ein altes Haus auf Paſſes Höh’n\newverse{}Beſchloß die erſte Strecke;\newverse{}Da klang Harmonika-Getön\newverse{}Hervor aus dunkler Ecke.\stanzaend{}\stanza{}Gelehnt an regenfeuchte Wand,\newverse{}Von Kälte{ }ſtarr die Glieder,\newverse{}Stand dort ein blinder Muſikant\newverse{}Und{ }ſpielte{ }ſeine Lieder.\stanzaend{}\stanza{}Er{ }ſpielte, und{ }ſein Auge war\newverse{}Gerichtet{ }ſtarr nach oben\newverse{}Und wurde doch kein Licht gewahr,\newverse{}So hoch es auch erhoben.\stanzaend{}\stanza{}{\pb}Er{ }ſpielte luſt’ge Melodie’n\newverse{}Und{ }ſang dazu ganz{ }ſachte;\newverse{}Das Singen faſt ein Weinen{ }ſchien,\newverse{}Nur daß er dazu lachte.\stanzaend{}\stanza{}Wie thut mir Deine bitt’re Noth,\newverse{}Du armer Mann,{ }ſo wehe!\newverse{}Du mit den Augen leer und todt,\newverse{}Verzeih’ mir, daß ich{ }ſehe!\stanzaend{}\stanza{}Bin ich gleich{ }ſehend,{ }ſeh’ ich \strikeout{\textcolor{gray}{nic}h} nicht,\newverse{}Du kannſt mir leicht vergeben.\newverse{}Das Licht, das heißgeliebte Licht,\newverse{}Ich{ }ſuch’s im dunklen Leben.\stanzaend{}\stanza{}Und{ }ſuch’ es heut und immerzu\newverse{}Und{ }ſeh’ es nimmer gleißen.\newverse{}Oh armer blinder Bettler Du,\newverse{}Du{ }ſollſt mich Bruder heißen! {\dotssix}\stanzaend{}\stanza{}Der Wagen rollet aus dem Thor,\newverse{}Klimmt dann auf{ }ſteilem Pfade,\newverse{}Und lange klingt mir noch im Ohr\newverse{}Die Jammer-Serenade.\stanzaend{}
\pstart
           Gruß! {\\[\baselineskip]}\spacefill\mbox{P. G.}\pend
           \leftskip=0em{}\selectlanguage{ngerman}\endnumbering\briefempfaengerindex{Schnitzler, Arthur@\textsc{Schnitzler, Arthur}!zzzGoldmann, Paul@\emph{von Paul Goldmann}!1900-08-281@{28. 8. [1900]}|)be}\mylabel{L02928h}  \newcommand{\dateiname}{L02928}\newcommand{\titel}{Paul Goldmann an Arthur Schnitzler, 28. 8. [1900]}\newcommand{\editorInnen}{Martin Anton Müller und Laura Untner}%% latex-leseansicht-abspann.tex
%% Abspann für die Leseansicht.
%% Der Schalter \ifkorrekturansicht ist bereits durch den Vorspann gesetzt.

%% latex-abspann.tex
%% Gemeinsamer Abspann für Korrekturansicht und Leseansicht.
%% Setzt den Schalter \ifkorrekturansicht voraus (gesetzt in den
%% einbindenden Dateien latex-korrekturansicht-abspann.tex bzw.
%% latex-leseansicht-abspann.tex).
%% ---------------------------------------------------------------

\normalsize

% Das esempio-Environment wird nur in der Leseansicht benötigt
\ifkorrekturansicht\else
\newenvironment{esempio}[3]%
{
    \vspace{1.5ex}
    \rlap{\underline{#1}}
    \par
    \setlength{\parindent}{0cm}
    \nopagebreak
    \leftskip=#2cm
    \rightskip=#3cm
}
{
    \par
}
\fi

\doendnotes{C}
\bigskip
\vfill

\clearpage

\footnotesize

\ifkorrekturansicht
  \lohead{\textsc{register}}
\fi

% theindex-Environment neu definieren ohne reledmac
\makeatletter
\renewenvironment{theindex}{%
  \ifkorrekturansicht
    \section*{\indexname}%
  \else
    \subsubsection*{Index der erwähnten Entitäten}%
  \fi
  \setlength{\parindent}{0pt}%
  \setlength{\parskip}{0pt plus 0.3pt}%
  \let\item\@idxitem
}{%
  \ifkorrekturansicht\clearpage\fi
}
\makeatother

\IfFileExists{\jobname-pw.ind}{\input{\jobname-pw.ind}}{}

% Quellenangabe nur in der Leseansicht
\ifkorrekturansicht\else
% Fallback-Definitionen, falls die .tex-Datei \titel etc. nicht gesetzt hat
\providecommand{\titel}{}
\providecommand{\editorInnen}{}
\providecommand{\dateiname}{\jobname}

\vspace{3cm}

\vfill

\footnotesize
\textsc{Quelle}: \titel. Herausgegeben von {\editorInnen}. In: \emph{Arthur Schnitzler: Briefwechsel mit Autorinnen und Autoren}.
 Digitale Edition, https://schnitzler-briefe.acdh.oeaw.ac.at/{\dateiname}.html (Stand \today)
\fi

\end{document}


