%% latex-korrekturansicht-vorspann.tex
%% Vorspann für die Korrekturansicht.
%% Lädt die gemeinsame Datei latex-vorspann.tex mit gesetztem Schalter.

\newif\ifkorrekturansicht
\korrekturansichttrue

\input{../tex-inputs/latex-vorspann}


\section[ Paul Goldmann an Arthur Schnitzler, 28. 8. {[}1900{]}]{L02928 Paul Goldmann an Arthur Schnitzler, 28. 8. {[}1900{]}}
\nopagebreak\mylabel{L02928v}
\rehead{ }\normalsize\beginnumbering\briefempfaengerindex{Schnitzler, Arthur@\textsc{Schnitzler, Arthur}!zzzGoldmann, Paul@\emph{von Paul Goldmann}!1900-08-281@{28. 8. {[}1900{]}}|(be}
\toendnotes[C]{\smallbreak\pagebreak[2]}\Standort{DLA, A:Schnitzler, HS.NZ85.1.3170.}
\physDesc{Brief, 1 Blatt, 2 Seiten, 960 Zeichen
\newline{}Handschrift: schwarze Tinte, deutsche Kurrent
\newline{}Schnitzler: mit Bleistift das Jahr »900.« vermerkt }\toendnotes[C]{\smallbreak}
\pstart
           \raggedleft{}{\pb}\textcolor{gray}{\textbf{HOTEL TRAFOI\oindex{Hotel Trafoi@\textbf{Hotel Trafoi}, \emph{Hotel (K.HTL)}|pwv}}}\pend
           
\pstart
           \raggedleft{}\textcolor{gray}{\textbf{TIROL\oindex{Tirol@\textbf{Tirol}, \emph{A.ADM1}|pw}.}}\pend
           
\pstart
           \raggedleft{}28. Auguſt.\pend
           \vspace{0.5em}\stanza{}\label{K_L02928-1v}\edtext{\uline{Der blinde Muſikant\pwindex{blinde Musikant@\emph{Der blinde Musikant}|pw}.}}{\lemma{\textnormal{\emph{Der blinde Muſikant.}}}\Cendnote{\textnormal{Bereits zwei Tage zuvor schrieben
                           Schnitzler und Goldmann\pwindex{Goldmann, Paul 31.01.1865 – 25.09.1935@\textsc{Goldmann, Paul} (31.01.1865 – 25.09.1935), \emph{Schriftsteller/Schriftstellerin, Journalist/Journalistin}|pwk} an Richard Beer-Hofmann\pwindex{Beer-Hofmann, Richard 1866-07-11 – 1945-09-26@\textsc{Beer-Hofmann, Richard} (1866-07-11 – 1945-09-26), \emph{Schriftsteller/Schriftstellerin}|pwk} von einem »Tirol\oindex{Tirol@\textbf{Tirol}, \emph{A.ADM1}|pw}er Sänger«. (Arthur Schnitzler und Paul Goldmann an Richard Beer-Hofmann,
               26. 8. 1900.) Dass es sich bei der
                        Begegnung nicht nur um den Textimpuls für dieses Gedicht\pwindex{blinde Musikant@\emph{Der blinde Musikant}|pwkv}, sondern auch für die
                        Novelle \emph{Der blinde Geronimo und sein
                           Bruder}\pwindex{blinde Geronimo und sein Bruder@\emph{Der blinde Geronimo und sein Bruder}|pwk} handelt, geht aus Goldmanns\pwindex{Goldmann, Paul 31.01.1865 – 25.09.1935@\textsc{Goldmann, Paul} (31.01.1865 – 25.09.1935), \emph{Schriftsteller/Schriftstellerin, Journalist/Journalistin}|pwk} Brief vom 18. 2. [1901] hervor, in dem Schnitzlers{ }Novelle\pwindex{blinde Geronimo und sein Bruder@\emph{Der blinde Geronimo und sein Bruder}|pwkv} als gegenüber der Vorlage fahl kritisiert wird. Siehe dazu
                        auch Paul Goldmann\pwindex{Goldmann, Paul 31.01.1865 – 25.09.1935@\textsc{Goldmann, Paul} (31.01.1865 – 25.09.1935), \emph{Schriftsteller/Schriftstellerin, Journalist/Journalistin}|pwk}: \emph{Erinnerungen an Arthur Schnitzler}\pwindex{Erinnerungen an Arthur Schnitzler@\emph{Erinnerungen an Arthur Schnitzler}|pwk}. In: \emph{Neue Freie Presse}\pwindex{Neue Freie Presse@\emph{Neue Freie Presse}|pwk}, Nr. 24.121, 8. 11. 1931, Morgenblatt, S. 25–26, hier:
                           S. 26.}}}\label{K_L02928-1}\stanzaend{}\stanza{}Ein altes Haus auf Paſſes Höh’nBeſchloß die erſte Strecke;Da klang Harmonika-GetönHervor aus dunkler Ecke.\stanzaend{}\stanza{}Gelehnt an regenfeuchte Wand,Von Kälte ſtarr die Glieder,Stand dort ein blinder MuſikantUnd ſpielte ſeine Lieder.\stanzaend{}\stanza{}Er ſpielte, und ſein Auge warGerichtet ſtarr nach obenUnd wurde doch kein Licht gewahr,So hoch es auch erhoben.\stanzaend{}\stanza{}{\pb}Er ſpielte luſt’ge Melodie’nUnd ſang dazu ganz ſachte;Das Singen faſt ein Weinen ſchien,Nur daß er dazu lachte.\stanzaend{}\stanza{}Wie thut mir Deine bitt’re Noth,Du armer Mann, ſo wehe!Du mit den Augen leer und todt,Verzeih’ mir, daß ich ſehe!\stanzaend{}\stanza{}Bin ich gleich ſehend, ſeh’ ich \strikeout{\textcolor{gray}{nic}h} nicht,Du kannſt mir leicht vergeben.Das Licht, das heißgeliebte Licht,Ich ſuch’s im dunklen Leben.\stanzaend{}\stanza{}Und ſuch’ es heut und immerzuUnd ſeh’ es nimmer gleißen.Oh armer blinder Bettler Du,Du ſollſt mich Bruder heißen! {\dotssix}\stanzaend{}\stanza{}Der Wagen rollet aus dem Thor,Klimmt dann auf ſteilem Pfade,Und lange klingt mir noch im OhrDie Jammer-Serenade.\stanzaend{}
\pstart
           Gruß! {\\[\baselineskip]}\spacefill\mbox{P. G.}\pend
           \leftskip=0em{}\selectlanguage{ngerman}\endnumbering\briefempfaengerindex{Schnitzler, Arthur@\textsc{Schnitzler, Arthur}!zzzGoldmann, Paul@\emph{von Paul Goldmann}!1900-08-281@{28. 8. {[}1900{]}}|)be}\mylabel{L02928h}  \normalsize

\doendnotes{C}
\bigskip
\vfill

\clearpage

\footnotesize

\lohead{\textsc{register}}

% Definiere theindex-Environment komplett neu ohne reledmac
\makeatletter
\renewenvironment{theindex}{%
  \section*{\indexname}%
  \setlength{\parindent}{0pt}%
  \setlength{\parskip}{0pt plus 0.3pt}%
  \let\item\@idxitem
}{%
  \clearpage
}
\makeatother

\IfFileExists{\jobname-pw.ind}{\input{\jobname-pw.ind}}{}

\end{document}

      