%% latex-leseansicht-vorspann.tex
%% Vorspann für die Leseansicht.
%% Lädt die gemeinsame Datei latex-vorspann.tex mit nicht gesetztem Schalter.

\newif\ifkorrekturansicht
\korrekturansichtfalse

\input{../tex-inputs/latex-vorspann}


         \renewcommand{\erwaehnteOrte}{Orte: Am Karlsbad, Berlin, Wien}
         \renewcommand{\erwaehnteWerke}{}
               \section[ Auguste Hauschner an Arthur Schnitzler, 2. 2. 1909]{ Auguste Hauschner an Arthur Schnitzler, 2. 2. 1909}\nopagebreak\mylabel{v}\rehead{ }\begin{ledgroupsized}[t]{13cm}\normalsize\beginnumbering \toendnotes[C]{\smallbreak\pagebreak[2]} \Standort{DLA, A:Schnitzler, HS1985.1.3363.}
\physDesc{Brief, 1 Blatt, 2 Seiten
\newline{}Handschrift: schwarze Tinte, lateinische Kurrent
\newline{}Schnitzler: mit Bleistift Vermerk »\textsc{Hauschner\pwindex{Hauschner, Auguste 12.02.1850 – 10.04.1924@\textsc{Hauschner, Auguste} (12.02.1850 – 10.04.1924), \emph{Schriftstellerin}|pw}}« und »Am Karlsbad 25\oindex{Am Karlsbad@\textbf{Am Karlsbad}|pw}«  }\toendnotes[C]{\smallbreak}\pstart
           \noindent{}{\pb}Das wäre mir freilich eine grosse Freude, geehrter Herr
               Doctor Schnitzler, wenn Sie mich in Berlin\oindex{Berlin@\textbf{Berlin}|pw}
               aufsuchen und etwas von Ihrem Schaffen mit mir sprechen würden. Und da es doch nicht
               zum Unmöglichen gehört, dass ich das erlebe, so will ich Ihnen sagen, dass ich,
               leider, leider, das Heim, in dem ich seit fast zwanzig Jahren lebe, im April verlassen muss, und dann Am {\pb}Karlsbad 25\oindex{Am Karlsbad@\textbf{Am Karlsbad}|pw} wohnen werde.\pend
           \pstart
           Es wäre schön, wenn mir diese Freude durch ein so glückliches geistiges Erlebniss
               heimischer gerecht würde, wie Ihre \label{K_L02588-1v}\edtext{persönliche Bekanntschaft}{\lemma{\textnormal{\emph{persönliche Bekanntschaft}}}\Cendnote{\textnormal{Es dürfte
                  weder zu einem solchen Besuch, noch zu einer persönlichen Bekanntschaft gekommen
                  sein. }}}\label{K_L02588-1h} es für mich wäre.\pend
           \pstart
           Mit verbindlichen Grüssen und vielen Dank für Ihren Brief{\\[\baselineskip]}\spacefill\mbox{Auguste Hauschner}\pend
           \leftskip=0em{}\pstart
           Berlin\oindex{Berlin@\textbf{Berlin}|pw}{ }2. 2. 09\pend
           
         
         \endnumbering\mylabel{h}\end{ledgroupsized}  \newcommand{\dateiname}{L02588}\newcommand{\titel}{Auguste Hauschner an Arthur Schnitzler, 2. 2. 1909}\newcommand{\editorInnen}{Martin Anton Müller und Laura Untner}%% latex-leseansicht-abspann.tex
%% Abspann für die Leseansicht.
%% Der Schalter \ifkorrekturansicht ist bereits durch den Vorspann gesetzt.

%% latex-abspann.tex
%% Gemeinsamer Abspann für Korrekturansicht und Leseansicht.
%% Setzt den Schalter \ifkorrekturansicht voraus (gesetzt in den
%% einbindenden Dateien latex-korrekturansicht-abspann.tex bzw.
%% latex-leseansicht-abspann.tex).
%% ---------------------------------------------------------------

\normalsize

% Das esempio-Environment wird nur in der Leseansicht benötigt
\ifkorrekturansicht\else
\newenvironment{esempio}[3]%
{
    \vspace{1.5ex}
    \rlap{\underline{#1}}
    \par
    \setlength{\parindent}{0cm}
    \nopagebreak
    \leftskip=#2cm
    \rightskip=#3cm
}
{
    \par
}
\fi

\doendnotes{C}
\bigskip
\vfill

\clearpage

\footnotesize

\ifkorrekturansicht
  \lohead{\textsc{register}}
\fi

% theindex-Environment neu definieren ohne reledmac
\makeatletter
\renewenvironment{theindex}{%
  \ifkorrekturansicht
    \section*{\indexname}%
  \else
    \subsubsection*{Index der erwähnten Entitäten}%
  \fi
  \setlength{\parindent}{0pt}%
  \setlength{\parskip}{0pt plus 0.3pt}%
  \let\item\@idxitem
}{%
  \ifkorrekturansicht\clearpage\fi
}
\makeatother

\IfFileExists{\jobname-pw.ind}{\input{\jobname-pw.ind}}{}

% Quellenangabe nur in der Leseansicht
\ifkorrekturansicht\else
% Fallback-Definitionen, falls die .tex-Datei \titel etc. nicht gesetzt hat
\providecommand{\titel}{}
\providecommand{\editorInnen}{}
\providecommand{\dateiname}{\jobname}

\vspace{3cm}

\vfill

\footnotesize
\textsc{Quelle}: \titel. Herausgegeben von {\editorInnen}. In: \emph{Arthur Schnitzler: Briefwechsel mit Autorinnen und Autoren}.
 Digitale Edition, https://schnitzler-briefe.acdh.oeaw.ac.at/{\dateiname}.html (Stand \today)
\fi

\end{document}


      