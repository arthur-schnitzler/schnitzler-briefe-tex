%% latex-korrekturansicht-vorspann.tex
%% Vorspann für die Korrekturansicht.
%% Lädt die gemeinsame Datei latex-vorspann.tex mit gesetztem Schalter.

\newif\ifkorrekturansicht
\korrekturansichttrue

\input{../tex-inputs/latex-vorspann}


\section[Richard Beer-Hofmann an Arthur Schnitzler, 20. 6. 1900]{L01046 Richard Beer-Hofmann an Arthur Schnitzler, 20. 6. 1900}
\nopagebreak\mylabel{L01046v}
\rehead{ }\normalsize\beginnumbering\briefempfaengerindex{Schnitzler, Arthur@\textsc{Schnitzler, Arthur}!zzzBeer-Hofmann, Richard@\emph{von Richard Beer-Hofmann}!1900-06-201@{20. 6. 1900}|(be}
\toendnotes[C]{\smallbreak\pagebreak[2]}\Standort{CUL, Schnitzler, B 8.}
\physDesc{Brief, 1 Blatt, 2 Seiten, 774 Zeichen
\newline{}Handschrift: schwarze Tinte, lateinische Kurrent
\newline{}Ordnung: mit Bleistift von unbekannter Hand nummeriert:
                                    »154« }
\buchAbdrucke{\weitereDrucke{Arthur Schnitzler, Richard Beer-Hofmann: \emph{Briefwechsel 1891–1931}. Wien, Zürich: \emph{Europaverlag} 1992, S. 145–146.} }\toendnotes[C]{\smallbreak}
\pstart
           \raggedleft{}{\pb}Alt-Aussee\oindex{Altaussee@\textbf{Altaussee}, \emph{A.ADM3}|pw}{ }20/VI 1900\pend
           \vspace{0.5em}
\pstart
           Lieber Arthur! Natürlich sollen Sie herko{\geminationm}en. Schreiben Sie mir für wann, und ob ich Zimmer
               (eins) für Sie bestellen soll. \uline{Seewirth}\oindex{Hotel am See@\textbf{Hotel am See}, \emph{Hotel (K.HTL)}|pw} oder Brunnthaler\oindex{Gasthaus Brunnthaler@\textbf{Gasthaus Brunnthaler}, \emph{Hotel (K.HTL)}|pw} (wo Hugo\pwindex{Hofmannsthal, Hugo von 1874-02-01 – 1929-07-15@\textsc{Hofmannsthal, Hugo von} (1874-02-01 – 1929-07-15), \emph{Schriftsteller/Schriftstellerin}|pw} wohnte). Übrigens ist es überflüssig da keine Überfülle
               von Fremden hier ist. Jedenfalls telegraphiren Sie. Ich arbeite erst seit 5 Tagen;
               mehr, wäre mehr. S.\pwindex{Schlenther, Paul 20.08.1854 – 30.04.1916@\textsc{Schlenther, Paul} (20.08.1854 – 30.04.1916), \emph{Schriftsteller/Schriftstellerin, Kritiker/Kritikerin, Theaterleiter/Theaterleiterin}|pw} richtet sich danach, daß
                  B.\pwindex{Brahm, Otto 05.02.1856 – 28.11.1912@\textsc{Brahm, Otto} (05.02.1856 – 28.11.1912), \emph{Theaterleiter/Theaterleiterin, Regisseur/Regisseurin}|pw} es nicht geno{\geminationm}en hat (S = Schlenther\pwindex{Schlenther, Paul 20.08.1854 – 30.04.1916@\textsc{Schlenther, Paul} (20.08.1854 – 30.04.1916), \emph{Schriftsteller/Schriftstellerin, Kritiker/Kritikerin, Theaterleiter/Theaterleiterin}|pw}, B = Brahm\pwindex{Brahm, Otto 05.02.1856 – 28.11.1912@\textsc{Brahm, Otto} (05.02.1856 – 28.11.1912), \emph{Theaterleiter/Theaterleiterin, Regisseur/Regisseurin}|pw}. Bemerk. des
               Herausgebers). Ich habe aber wirklich keinen Grund »Witze« zu machen. Ich halte meine
               Laune mit knapper Mühe auf arbeits{\pb}fähigem Niveau. Ende Juli könnte ich nicht mit. Je später im August,
               desto wahrscheinlicher; jedenfalls etwas Süden ins Programm nehmen. \label{K_L01046-1v}\edtext{Im Juli}{\lemma{\textnormal{\emph{Im Juli}}}\Cendnote{\textnormal{Beer-Hofmann\pwindex{Beer-Hofmann, Richard 1866-07-11 – 1945-09-26@\textsc{Beer-Hofmann, Richard} (1866-07-11 – 1945-09-26), \emph{Schriftsteller/Schriftstellerin}|pwk} ist am 11. 7. 1866
                  geboren.}}}\label{K_L01046-1} werde ich vierunddreißig, – um Ihnen zum Schluß noch etwas
               Angenehmes zu sagen.\pend
           \pstart Von Herzen Ihr \spacefill\mbox{Richard}\pend{}\selectlanguage{ngerman}\endnumbering\briefempfaengerindex{Schnitzler, Arthur@\textsc{Schnitzler, Arthur}!zzzBeer-Hofmann, Richard@\emph{von Richard Beer-Hofmann}!1900-06-201@{20. 6. 1900}|)be}\mylabel{L01046h}  \normalsize

\doendnotes{C}
\bigskip
\vfill

\clearpage

\footnotesize

\lohead{\textsc{register}}

% Definiere theindex-Environment komplett neu ohne reledmac
\makeatletter
\renewenvironment{theindex}{%
  \section*{\indexname}%
  \setlength{\parindent}{0pt}%
  \setlength{\parskip}{0pt plus 0.3pt}%
  \let\item\@idxitem
}{%
  \clearpage
}
\makeatother

\IfFileExists{\jobname-pw.ind}{\input{\jobname-pw.ind}}{}

\end{document}

      