%% latex-korrekturansicht-vorspann.tex
%% Vorspann für die Korrekturansicht.
%% Lädt die gemeinsame Datei latex-vorspann.tex mit gesetztem Schalter.

\newif\ifkorrekturansicht
\korrekturansichttrue

\input{../tex-inputs/latex-vorspann}


\section[Richard Beer-Hofmann an Arthur Schnitzler, 18. 6. 1898]{L00807 Richard Beer-Hofmann an Arthur Schnitzler, 18. 6. 1898}
\nopagebreak\mylabel{L00807v}
\rehead{ }\normalsize\beginnumbering\briefempfaengerindex{Schnitzler, Arthur@\textsc{Schnitzler, Arthur}!zzzBeer-Hofmann, Richard@\emph{von Richard Beer-Hofmann}!1898-06-181@{18. 6. 1898}|(be}
\toendnotes[C]{\smallbreak\pagebreak[2]}\Standort{CUL, Schnitzler, B 8.}
\physDesc{Brief, 1 Blatt, 4 Seiten, 1052 Zeichen
\newline{}Handschrift: Bleistift, lateinische Kurrent
\newline{}Ordnung: mit Bleistift von unbekannter Hand nummeriert:
                                    »117« }
\buchAbdrucke{\weitereDrucke{Arthur Schnitzler, Richard Beer-Hofmann: \emph{Briefwechsel 1891–1931}. Wien, Zürich: \emph{Europaverlag} 1992, S. 120.} }\toendnotes[C]{\smallbreak}
\pstart
           \raggedleft{}{\pb}Steindorf\oindex{Steindorf am Ossiacher See@\textbf{Steindorf am Ossiacher See}, \emph{A.ADM3}|pw}{ }18/VI 98\pend
           \vspace{0.5em}
\pstart
           Lieber Arthur, vielen Dank für Ihr »Interpunctationsgefühl\pwindex{Schlaflied fuer Mirjam@\emph{Schlaflied für Mirjam}|pwv}«. Auch mir waren die
                  \uline{–} anstatt \uline{,} zu
               ausdrucksvoll, zu überquellend von Empfindung – wollte nur nichts sagen, um Ihre
               Unbefangenheit nicht zu stören.\pend
           
\pstart
           Da es scheint daß Sie \strikeout{zwisch} nach
                  27 Juli nach Tegernsee\oindex{Tegernsee@\textbf{Tegernsee}, \emph{P.PPL}|pw} per Rad
               fahren, so dürfte wol unsere Zusa{\geminationm}enkunft {\pb}am besten in der I oder
                  II. Augustwoche um Salzburg\oindex{Salzburg@\textbf{Salzburg}, \emph{A.ADM2}|pw} herum
               stattfinden. Das würde auch für Hugo\pwindex{Hofmannsthal, Hugo von 1874-02-01 – 1929-07-15@\textsc{Hofmannsthal, Hugo von} (1874-02-01 – 1929-07-15), \emph{Schriftsteller/Schriftstellerin}|pw} nach
               seinem letzten Brief die beste Zeit sein.\pend
           
\pstart
           Vielleicht auch – wenn ich trainirt bin – im September im Ampezzo\oindex{Valle DAmpezzo@\textbf{Valle d’Ampezzo}, \emph{T.VAL}|pw}. 20–27 Juli
               ist unsicher da mein Papa\pwindex{Hofmann, Alois 30.3.1830 – 11.7.1907@\textsc{Hofmann, Alois} (30.3.1830 – 11.7.1907), \emph{Industrieller/Industrielle}|pwv}
               mich ungern abseits von Mirjam\pwindex{Beer-Hofmann, Mirjam 04.09.1897 – 24.12.1984@\textsc{Beer-Hofmann, Mirjam} (04.09.1897 – 24.12.1984)|pw} sieht. Ich
               arbeite {\pb}– nicht genug. Ich hoffe,
               es wird besser. Wetter ist scheusslich; heute regenlos, aber der Regen ko{\geminationm}t noch.\pend
           
\pstart
           Bitte schreiben Sie mir so oft als möglich; wenn man – wie der zudringliche Mime\pwindex{?? [Schauspieler] 3.7.1898 – 3.7.1898@\textsc{?? [Schauspieler]} (3.7.1898 – 3.7.1898)|pwv} das nennt, keine
               »Ansprache« hat!\pend
           
\pstart
           Grüßen Sie wie {\pb}gewöhnlich nach
               Gutdünken und nuancirt. Ich lese ein gutes Buch von \uline{Mach}\pwindex{Mach, Ernst 18.02.1838 – 19.02.1916@\textsc{Mach, Ernst} (18.02.1838 – 19.02.1916), \emph{Philosoph/Philosophin, Philosophiehistoriker/Philosophiehistorikerin, Physiker/Physikerin}|pw} (Populärwissensch. Vorles.\pwindex{Populaer-Wissenschaftliche Vorlesungen@\emph{Populär-Wissenschaftliche Vorlesungen}|pw}).\pend
           
\pstart
           Von Herzen{\\[\baselineskip]}Ihr{\\[\baselineskip]}\spacefill\mbox{Richard}\pend
           \leftskip=0em{}
\pstart
           \noindent{}Paula\pwindex{Beer-Hofmann, Paula 25.02.1879 – 30.10.1939@\textsc{Beer-Hofmann, Paula} (25.02.1879 – 30.10.1939)|pw} erwidert Ihren Gruß – Mirjam\pwindex{Beer-Hofmann, Mirjam 04.09.1897 – 24.12.1984@\textsc{Beer-Hofmann, Mirjam} (04.09.1897 – 24.12.1984)|pw} hab ich ihn mitgeteilt; sie hat mich
                  hierauf in den Finger gebissen.\pend
           \selectlanguage{ngerman}\endnumbering\briefempfaengerindex{Schnitzler, Arthur@\textsc{Schnitzler, Arthur}!zzzBeer-Hofmann, Richard@\emph{von Richard Beer-Hofmann}!1898-06-181@{18. 6. 1898}|)be}\mylabel{L00807h}  \normalsize

\doendnotes{C}
\bigskip
\vfill

\clearpage

\footnotesize

\lohead{\textsc{register}}

% Definiere theindex-Environment komplett neu ohne reledmac
\makeatletter
\renewenvironment{theindex}{%
  \section*{\indexname}%
  \setlength{\parindent}{0pt}%
  \setlength{\parskip}{0pt plus 0.3pt}%
  \let\item\@idxitem
}{%
  \clearpage
}
\makeatother

\IfFileExists{\jobname-pw.ind}{\input{\jobname-pw.ind}}{}

\end{document}

      