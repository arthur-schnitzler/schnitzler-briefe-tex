%% latex-leseansicht-vorspann.tex
%% Vorspann für die Leseansicht.
%% Lädt die gemeinsame Datei latex-vorspann.tex mit nicht gesetztem Schalter.

\newif\ifkorrekturansicht
\korrekturansichtfalse

\input{../tex-inputs/latex-vorspann}


\section[Richard Beer-Hofmann an Arthur Schnitzler, 18. 6. 1898]{L00807 Richard Beer-Hofmann an Arthur Schnitzler, 18. 6. 1898}
\nopagebreak\mylabel{L00807v}
\rehead{ }\normalsize\beginnumbering\briefempfaengerindex{Schnitzler, Arthur@\textsc{Schnitzler, Arthur}!zzzBeer-Hofmann, Richard@\emph{von Richard Beer-Hofmann}!1898-06-181@{18. 6. 1898}|(be}
\toendnotes[C]{\smallbreak\pagebreak[2]}
\correspDesc{Versand  durch Richard Beer-Hofmann am 18. 6. 1898 in Steindorf am Ossiacher See
\newline{}Erhalt  durch Arthur Schnitzler im Zeitraum [19. 6. 1898
                  – 23. 6. 1898?] in Wien}\toendnotes[C]{\smallbreak}
\Standort{CUL, Schnitzler, B 8.}
\physDesc{Brief, 1 Blatt, 4 Seiten, 1052 Zeichen
\newline{}Handschrift: Bleistift, lateinische Kurrent
\newline{}Ordnung: mit Bleistift von unbekannter Hand nummeriert:
                                    »117« }
\buchAbdrucke{\weitereDrucke{Arthur Schnitzler, Richard Beer-Hofmann: \emph{Briefwechsel 1891–1931}. Herausgegeben von Konstanze Fliedl. Wien, Zürich: \emph{Europaverlag} 1992, S. 120.} }\toendnotes[C]{\smallbreak}
\pstart
           \raggedleft{}{\pb}Steindorf\oindex{Steindorf am Ossiacher See@\textbf{Steindorf am Ossiacher See}, \emph{Verwaltungsgebiet}|pw}{ }18/VI 98\pend
           \vspace{0.5em}
\pstart
           Lieber Arthur, vielen Dank für Ihr »Interpunctationsgefühl\pwindex{Beer-Hofmann, Richard 11.\,7.\,1866 Wien – 26.\,9.\,1945 New York City@\textsc{Beer-Hofmann, Richard} (11.\,7.\,1866 Wien – 26.\,9.\,1945 New York City), \emph{Schriftsteller}!Schlaflied für Mirjam@\strich\emph{Schlaflied für Mirjam}|pwv}«. Auch mir waren die
                  \uline{–} anstatt \uline{,} zu
               ausdrucksvoll, zu überquellend von Empfindung – wollte nur nichts sagen, um Ihre
               Unbefangenheit nicht zu stören.\pend
           
\pstart
           Da es scheint daß Sie \strikeout{zwisch} nach
                  27 Juli nach Tegernsee\oindex{Tegernsee@\textbf{Tegernsee}|pw} per Rad
               fahren, so dürfte wol unsere Zusa{\geminationm}enkunft {\pb}am besten in der I oder
                  II. Augustwoche um Salzburg\oindex{Salzburg@\textbf{Salzburg}, \emph{Verwaltungsgebiet}|pw} herum
               stattfinden. Das würde auch für Hugo\pwindex{Hofmannsthal, Hugo von 1.\,2.\,1874 Wien – 15.\,7.\,1929 Rodaun@\textsc{Hofmannsthal, Hugo von} (1.\,2.\,1874 Wien – 15.\,7.\,1929 Rodaun), \emph{Schriftsteller}|pw} nach
               seinem letzten Brief die beste Zeit sein.\pend
           
\pstart
           Vielleicht auch – wenn ich trainirt bin – im September im Ampezzo\oindex{Valle d’Ampezzo@\textbf{Valle d’Ampezzo}, \emph{Tal}|pw}. 20–27 Juli
               ist unsicher da mein Papa\pwindex{Hofmann, Alois 30.\,3.\,1830 Lomnice – 11.\,7.\,1907 Wien@\textsc{Hofmann, Alois} (30.\,3.\,1830 Lomnice – 11.\,7.\,1907 Wien), \emph{Industrieller}|pwv}
               mich ungern abseits von Mirjam\pwindex{Beer-Hofmann, Mirjam 4.\,9.\,1897 Wien – 24.\,12.\,1984 New York City@\textsc{Beer-Hofmann, Mirjam} (4.\,9.\,1897 Wien – 24.\,12.\,1984 New York City)|pw} sieht. Ich
               arbeite {\pb}– nicht genug. Ich hoffe,
               es wird besser. Wetter ist scheusslich; heute regenlos, aber der Regen ko{\geminationm}t noch.\pend
           
\pstart
           Bitte schreiben Sie mir so oft als möglich; wenn man – wie der zudringliche Mime\pwindex{?? [Schauspieler] 3.\,7.\,1898 – 3.\,7.\,1898@\textsc{?? [Schauspieler]} (3.\,7.\,1898 – 3.\,7.\,1898)|pwv} das nennt, keine
               »Ansprache« hat!\pend
           
\pstart
           Grüßen Sie wie {\pb}gewöhnlich nach
               Gutdünken und nuancirt. Ich lese ein gutes Buch von \uline{Mach}\pwindex{Mach, Ernst 18.\,2.\,1838 Tuřany – 19.\,2.\,1916 Vaterstetten@\textsc{Mach, Ernst} (18.\,2.\,1838 Tuřany – 19.\,2.\,1916 Vaterstetten), \emph{Philosoph, Philosophiehistoriker, Physiker}|pw} (Populärwissensch. Vorles.\pwindex{Mach, Ernst 18.\,2.\,1838 Tuřany – 19.\,2.\,1916 Vaterstetten@\textsc{Mach, Ernst} (18.\,2.\,1838 Tuřany – 19.\,2.\,1916 Vaterstetten), \emph{Philosoph, Philosophiehistoriker, Physiker}!Populär-Wissenschaftliche Vorlesungen@\strich\emph{Populär-Wissenschaftliche Vorlesungen}|pw}).\pend
           
\pstart
           Von Herzen{\\[\baselineskip]}Ihr{\\[\baselineskip]}\spacefill\mbox{Richard}\pend
           \leftskip=0em{}
\pstart
           \noindent{}Paula\pwindex{Beer-Hofmann, Paula 25.\,2.\,1879 Wien – 30.\,10.\,1939 Zürich@\textsc{Beer-Hofmann, Paula} (25.\,2.\,1879 Wien – 30.\,10.\,1939 Zürich)|pw} erwidert Ihren Gruß – Mirjam\pwindex{Beer-Hofmann, Mirjam 4.\,9.\,1897 Wien – 24.\,12.\,1984 New York City@\textsc{Beer-Hofmann, Mirjam} (4.\,9.\,1897 Wien – 24.\,12.\,1984 New York City)|pw} hab ich ihn mitgeteilt; sie hat mich
                  hierauf in den Finger gebissen.\pend
           \selectlanguage{ngerman}\endnumbering\briefempfaengerindex{Schnitzler, Arthur@\textsc{Schnitzler, Arthur}!zzzBeer-Hofmann, Richard@\emph{von Richard Beer-Hofmann}!1898-06-181@{18. 6. 1898}|)be}\mylabel{L00807h}  \newcommand{\dateiname}{L00807}\newcommand{\titel}{Richard Beer-Hofmann an Arthur Schnitzler, 18. 6. 1898}\newcommand{\editorInnen}{Martin Anton Müller und Gerd-Hermann Susen}%% latex-leseansicht-abspann.tex
%% Abspann für die Leseansicht.
%% Der Schalter \ifkorrekturansicht ist bereits durch den Vorspann gesetzt.

%% latex-abspann.tex
%% Gemeinsamer Abspann für Korrekturansicht und Leseansicht.
%% Setzt den Schalter \ifkorrekturansicht voraus (gesetzt in den
%% einbindenden Dateien latex-korrekturansicht-abspann.tex bzw.
%% latex-leseansicht-abspann.tex).
%% ---------------------------------------------------------------

\normalsize

% Das esempio-Environment wird nur in der Leseansicht benötigt
\ifkorrekturansicht\else
\newenvironment{esempio}[3]%
{
    \vspace{1.5ex}
    \rlap{\underline{#1}}
    \par
    \setlength{\parindent}{0cm}
    \nopagebreak
    \leftskip=#2cm
    \rightskip=#3cm
}
{
    \par
}
\fi

\doendnotes{C}
\bigskip
\vfill

\clearpage

\footnotesize

\ifkorrekturansicht
  \lohead{\textsc{register}}
\fi

% theindex-Environment neu definieren ohne reledmac
\makeatletter
\renewenvironment{theindex}{%
  \ifkorrekturansicht
    \section*{\indexname}%
  \else
    \subsubsection*{Index der erwähnten Entitäten}%
  \fi
  \setlength{\parindent}{0pt}%
  \setlength{\parskip}{0pt plus 0.3pt}%
  \let\item\@idxitem
}{%
  \ifkorrekturansicht\clearpage\fi
}
\makeatother

\IfFileExists{\jobname-pw.ind}{\input{\jobname-pw.ind}}{}

% Quellenangabe nur in der Leseansicht
\ifkorrekturansicht\else
% Fallback-Definitionen, falls die .tex-Datei \titel etc. nicht gesetzt hat
\providecommand{\titel}{}
\providecommand{\editorInnen}{}
\providecommand{\dateiname}{\jobname}

\vspace{3cm}

\vfill

\footnotesize
\textsc{Quelle}: \titel. Herausgegeben von {\editorInnen}. In: \emph{Arthur Schnitzler: Briefwechsel mit Autorinnen und Autoren}.
 Digitale Edition, https://schnitzler-briefe.acdh.oeaw.ac.at/{\dateiname}.html (Stand \today)
\fi

\end{document}


