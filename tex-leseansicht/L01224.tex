\input{../tex-inputs/latex-pdf-vorspann}
\begin{center}
            \textcolor{red}{ENTWURF. ENTZIFFERUNG NOCH NICHT KORREKTURGELESEN}
                      \end{center}
            
               \section[Hugo von Hofmannsthal an Arthur Schnitzler, 25. 6. 1902]{ Hugo von Hofmannsthal an Arthur Schnitzler,
               25. 6. 1902}\nopagebreak\mylabel{v}\rehead{ }\begin{ledgroupsized}[t]{13cm}\normalsize\beginnumbering\briefempfaengerindex{Schnitzler, Arthur@\textsc{Schnitzler, Arthur}!zzzHofmannsthal, Hugo von@\emph{von Hugo von Hofmannsthal}!1902-06-251@{25. 6. 1902}|(be} \toendnotes[C]{\smallbreak\pagebreak[2]} \Standort{CUL, Schnitzler, B 43.}
\physDesc{Postkarte
\newline{}Handschrift: schwarze Tinte, deutsche Kurrent\newline{}Versand: 1) Stempel: »\nobreak{}\oindex{Gainfarn@\textbf{Gainfarn}|pwk}Gainfahrn b. Vöslau, 25 6 02\nobreak{}«.  2) Stempel: »\nobreak{}\oindex{IX., Alsergrund@\textbf{IX., Alsergrund}|pwk}Wien 9/3, 26. 6. 02, 8.V, Bestellt\nobreak{}«. 
\newline{}Schnitzler: mit Bleistift datiert: »25/6 902.« \newline{}Ordnung: 1) mit Bleistift von unbekannter Hand nummeriert: »\strikeout{197}« 2) mit Bleistift von unbekannter Hand nummeriert: »180«}\pstart{}{\pb}\textsc{Herrn D\textsuperscript{r} Arthur Schnitzler}\pend{}\pstart{}\textsc{Wien}\oindex{Wien@\textbf{Wien}|pw}\pend{}\pstart{}\textsc{IX. Franckgasse 1}\oindex{Frankgasse@\textbf{Frankgasse}|pw}.
               \pend{}{\bigskip}\pstart
           \noindent{}{\pb}lieber, falls aber in
                  \textsc{Salzburg\oindex{Salzburg@\textbf{Salzburg}|pw}} ſchönes, wirklich so{\geminationm}erliches Wetter, mit dem
               Charakter des Bleibenden ſein ſollte, ſo telegrafiren Sie mir \uline{ſogleich}.\pend
           \pstart
           Ihr{\\[\baselineskip]}\spacefill\mbox{Hugo.}\pend
           \leftskip=0em{}\endnumbering\briefempfaengerindex{Schnitzler, Arthur@\textsc{Schnitzler, Arthur}!zzzHofmannsthal, Hugo von@\emph{von Hugo von Hofmannsthal}!1902-06-251@{25. 6. 1902}|)be}\mylabel{h}\end{ledgroupsized}  \newcommand{\dateiname}{L01224}\newcommand{\titel}{Hugo von Hofmannsthal an Arthur Schnitzler, 25. 6. 1902}\newcommand{\editorInnen}{Martin Anton Müller und Gerd-Hermann Susen}\input{../tex-inputs/latex-pdf-abspann}
      