\input{../tex-inputs/latex-pdf-vorspann}
\begin{center}
            \textcolor{red}{ENTWURF. ENTZIFFERUNG NOCH NICHT KORREKTURGELESEN}
                      \end{center}
            
               \section[Arthur Schnitzler an Hermann Bahr, 28. 10. 1901]{ Arthur Schnitzler an Hermann Bahr, 28. 10. 1901}\nopagebreak\mylabel{v}\rehead{ }\begin{ledgroupsized}[t]{13cm}\normalsize\beginnumbering\briefempfaengerindex{Bahr, Hermann@\textsc{Bahr, Hermann}!zzzSchnitzler, Arthur@\emph{von Arthur Schnitzler}!1901-10-281@{28. 10. 1901}|(be} \toendnotes[C]{\smallbreak\pagebreak[2]} \Standort{TMW, HS AM 23347 Ba.}
\physDesc{Brief, 1 Blatt, 4 Seiten
\newline{}Handschrift: schwarze Tinte, deutsche Kurrent\newline{}Ordnung: 1) Lochung 2) mit Bleistift von unbekannter Hand datiert: »26 X. 01«}\buchAbdrucke{\weitereDrucke{1) \emph{28. 10. 1901.} In: Arthur Schnitzler: \emph{The Letters of Arthur Schnitzler to Hermann Bahr}. Edited, annotated, and with an introduction, by Donald G.
                        Daviau. Chapel Hill: \emph{The University of North Carolina Press} 1978, S. 72 (University of North Carolina studies in the Germanic languages
                        and literatures, 89).} \weitereDrucke{2) Hermann Bahr, Arthur Schnitzler: \emph{Briefwechsel, Aufzeichnungen, Dokumente (1891–1931)}. Hg. Kurt Ifkovits und Martin Anton Müller. Göttingen: \emph{Wallstein} 2018, S. 217.} }\toendnotes[C]{\smallbreak}\pstart{}{\pb}lieber Hermann,
               \pend\pstart
           aus deinem lieben Brief entnehme ich u. a. dſs Berger\pwindex{Berger, Alfred von 30.04.1853 – 24.08.1912@\textsc{Berger, Alfred von} (30.04.1853 – 24.08.1912), \emph{Schriftsteller, Journalist, Theaterleiter}|pw} hier war. \uline{Iſt} er noch in Wien\oindex{Wien@\textbf{Wien}|pw}? (Er schrieb mir eine \label{K_L01185_1v}\edtext{Karte}{\lemma{\textnormal{\emph{Karte}}}\Cendnote{\textnormal{»Hochgeehrter Herr Doctor!{ / }Nächste Woche spreche ich Sie in Wien\oindex{Wien@\textbf{Wien}|pw}. Ich
                        bin von den ›letzten Stunden\pwindex{Schnitzler, Arthur 15.05.1862 – 21.10.1931@\textsc{Schnitzler, Arthur} (15.05.1862 – 21.10.1931), \emph{Schriftsteller, Mediziner}!Lebendige Stunden. Vier Einakter1901-12-23@\strich\emph{Lebendige Stunden. Vier Einakter} {[}1901-12-23{]}|pw}‹ entzückt, so
                        entzückt, als die Hamburg\oindex{Hamburg@\textbf{Hamburg}|pw}er darüber empört
                        sein werden. Alles Nähere mündlich.\hspace*{1.5em}Herzlich grüßt{ / }Alfred v. Berger\pwindex{Berger, Alfred von 30.04.1853 – 24.08.1912@\textsc{Berger, Alfred von} (30.04.1853 – 24.08.1912), \emph{Schriftsteller, Journalist, Theaterleiter}|pw}{ / }18/10 1901« (gedruckter Kopf: »Deutsches Schauspielhaus in Hamburg\oindex{Deutsches Schauspielhaus@\textbf{Deutsches Schauspielhaus}|pw}«, \emph{Cambridge University Library}, Schnitzler, B 10).}}}\label{K_L01185_1h}{ }\introOben{}(aus Hamburg\oindex{Hamburg@\textbf{Hamburg}|pw})\introOben{}, dſs er mich
               perſönlich ſprechen wollte, in Angelegenheit der Stücke\pwindex{Schnitzler, Arthur 15.05.1862 – 21.10.1931@\textsc{Schnitzler, Arthur} (15.05.1862 – 21.10.1931), \emph{Schriftsteller, Mediziner}!letzten Masken1901@\strich\emph{Die letzten Masken} {[}1901{]}|pwv}\pwindex{Schnitzler, Arthur 15.05.1862 – 21.10.1931@\textsc{Schnitzler, Arthur} (15.05.1862 – 21.10.1931), \emph{Schriftsteller, Mediziner}!Literatur1901@\strich\emph{Literatur} {[}1901{]}|pwv}\pwindex{Schnitzler, Arthur 15.05.1862 – 21.10.1931@\textsc{Schnitzler, Arthur} (15.05.1862 – 21.10.1931), \emph{Schriftsteller, Mediziner}!Frau mit dem Dolche1901@\strich\emph{Die Frau mit dem Dolche} {[}1901{]}|pwv}.) –\pend
           \pstart
           Die Dolchdame\pwindex{Schnitzler, Arthur 15.05.1862 – 21.10.1931@\textsc{Schnitzler, Arthur} (15.05.1862 – 21.10.1931), \emph{Schriftsteller, Mediziner}!Frau mit dem Dolche1901@\strich\emph{Die Frau mit dem Dolche} {[}1901{]}|pw} iſt gewiſs ein ſchweres ſcenisches
                  {\pb}Ding; aber ſo weit
               ſind wir heute doch ſchon in dieſen Sachen, dſs es unbedingt gehen muſs. –\pend
           \pstart
           \textsc{Bukovics\pwindex{Bukovics, Emerich von 28.02.1844 – 04.07.1905@\textsc{Bukovics, Emerich von} (28.02.1844 – 04.07.1905), \emph{Journalist, Theaterleiter}|pw}} hat mich neulich mit der Ausſicht entlaſſen, dſs er über die Beſetz nachdenken
               werde. Du haſt ja recht; ich muſs energiſcher mit ihm ſein, aber mir fehlt die rechte
               Begeiſterung für die vorausſichtliche Volks{\pb}theateraufführg. Nun es
               bleibt mir ja nichts andres übrig. Ich werde nächſtens »ſtürmiſch« einen \label{K_L01185_2v}\edtext{Contract mit einer Million Poenale
               verlangen}{\lemma{\textnormal{\emph{Contract … verlangen}}}\Cendnote{\textnormal{Vgl. den Brief Schnitzler\pwindex{Schnitzler, Arthur 15.05.1862 – 21.10.1931@\textsc{Schnitzler, Arthur} (15.05.1862 – 21.10.1931), \emph{Schriftsteller, Mediziner}|pwk}s an Emerich von Bukovics\pwindex{Bukovics, Emerich von 28.02.1844 – 04.07.1905@\textsc{Bukovics, Emerich von} (28.02.1844 – 04.07.1905), \emph{Journalist, Theaterleiter}|pwk}, 11. 12. 1901, in \emph{Briefwechsel} Bahr/Schnitzler 219–220.}}}\label{K_L01185_2h}.\pend
           \pstart
           – Wie man die »Literatur\pwindex{Schnitzler, Arthur 15.05.1862 – 21.10.1931@\textsc{Schnitzler, Arthur} (15.05.1862 – 21.10.1931), \emph{Schriftsteller, Mediziner}!Literatur1901@\strich\emph{Literatur} {[}1901{]}|pw}« ſo beſonders gut finden
               kann, verſteh ich abſolut nicht; mein \textsc{faible}{ }ſind die »lebendigen
                  Stunden\pwindex{Schnitzler, Arthur 15.05.1862 – 21.10.1931@\textsc{Schnitzler, Arthur} (15.05.1862 – 21.10.1931), \emph{Schriftsteller, Mediziner}!Lebendige Stunden01. 12. 1901@\strich\emph{Lebendige Stunden} {[}01. 12. 1901{]}|pw}.«\pend
           \pstart
           Kainz\pwindex{Kainz, Josef 02.01.1858 – 20.09.1910@\textsc{Kainz, Josef} (02.01.1858 – 20.09.1910), \emph{Schauspieler}|pw} wollte am 5. den Gustl\pwindex{Schnitzler, Arthur 15.05.1862 – 21.10.1931@\textsc{Schnitzler, Arthur} (15.05.1862 – 21.10.1931), \emph{Schriftsteller, Mediziner}!Lieutenant Gustl. Novelle25. 12. 1900@\strich\emph{Lieutenant Gustl. Novelle} {[}25. 12. 1900{]}|pw}{ }{\pb}leſen; aber \introOben{}–\introOben{} Herr \label{K_L01185_3v}\edtext{Gutmann\pwindex{Gutmann, Albert 20.06.1851 – 07.03.1915@\textsc{Gutmann, Albert} (20.06.1851 – 07.03.1915), \emph{Veranstalter, Agent}|pw}}{\lemma{\textnormal{\emph{Gutmann}}}\Cendnote{\textnormal{Betreiber einer Konzertagentur, die im Bösendorfer-Saal\oindex{Boesendorfer-Saal@\textbf{Bösendorfer-Saal}|pwk} Veranstaltungen
                  organisierte.}}}\label{K_L01185_3h} hat Angſt gehabt. Ich werde anfangen, die militäriſche
               Verachtg gegen das Civil zu theilen.\pend
           \pstart
           Herzlichſt dein{\\[\baselineskip]}\spacefill\mbox{Arthur}\pend
           \leftskip=0em{}\pstart
           28. X. 901.\pend
           \endnumbering\briefempfaengerindex{Bahr, Hermann@\textsc{Bahr, Hermann}!zzzSchnitzler, Arthur@\emph{von Arthur Schnitzler}!1901-10-281@{28. 10. 1901}|)be}\mylabel{h}\end{ledgroupsized}  \newcommand{\dateiname}{L01185}\newcommand{\titel}{Arthur Schnitzler an Hermann Bahr, 28. 10. 1901}\newcommand{\editorInnen}{ Kurt Ifkovits,  Martin Anton Müller}\input{../tex-inputs/latex-pdf-abspann}
      