%% latex-korrekturansicht-vorspann.tex
%% Vorspann für die Korrekturansicht.
%% Lädt die gemeinsame Datei latex-vorspann.tex mit gesetztem Schalter.

\newif\ifkorrekturansicht
\korrekturansichttrue

\input{../tex-inputs/latex-vorspann}


\section[Arthur Schnitzler an Hermann Bahr, 28. 10. 1901]{L01185 Arthur Schnitzler an Hermann Bahr, 28. 10. 1901}
\nopagebreak\mylabel{L01185v}
\rehead{ }\normalsize\beginnumbering\briefempfaengerindex{Bahr, Hermann@\textsc{Bahr, Hermann}!zzzSchnitzler, Arthur@\emph{von Arthur Schnitzler}!1901-10-281@{28. 10. 1901}|(be}
\toendnotes[C]{\smallbreak\pagebreak[2]}\Standort{TMW, HS AM 23347 Ba.}
\physDesc{Brief, 1 Blatt, 4 Seiten, 996 Zeichen
\newline{}Handschrift: schwarze Tinte, deutsche Kurrent
\newline{}Ordnung: 1) Lochung  2) mit Bleistift von unbekannter Hand datiert: »26 X. 01«}
\buchAbdrucke{\weitereDrucke{1) Arthur Schnitzler: \emph{The Letters of Arthur Schnitzler to Hermann Bahr}. Chapel Hill: \emph{The University of North Carolina Press} 1978, S. 72.} \weitereDrucke{2) Hermann Bahr, Arthur Schnitzler: \emph{Briefwechsel, Aufzeichnungen, Dokumente (1891–1931)}. Göttingen: \emph{Wallstein} 2018, S. 217.} }\toendnotes[C]{\smallbreak}
\pstart{}{\pb}lieber Hermann,
               \pend\vspace{0.5em}
\pstart
           aus deinem lieben Brief entnehme ich u. a. dſs Berger\pwindex{Berger, Alfred von 30.04.1853 – 24.08.1912@\textsc{Berger, Alfred von} (30.04.1853 – 24.08.1912), \emph{Schriftsteller/Schriftstellerin, Journalist/Journalistin, Theaterleiter/Theaterleiterin}|pw} hier war. \uline{Iſt} er noch in Wien\oindex{Wien@\textbf{Wien}, \emph{A.ADM2}|pw}? (Er schrieb mir eine \label{K_L01185-1v}\edtext{Karte}{\lemma{\textnormal{\emph{Karte}}}\Cendnote{\textnormal{»Hochgeehrter Herr Doctor!{ / }Nächste Woche spreche ich Sie in Wien\oindex{Wien@\textbf{Wien}, \emph{A.ADM2}|pw}.
                        Ich bin von den ›letzten Stunden\pwindex{Lebendige Stunden. Vier Einakter@\emph{Lebendige Stunden. Vier Einakter}|pw}‹
                        entzückt, so entzückt, als die Hamburg\oindex{Hamburg@\textbf{Hamburg}, \emph{P.PPLA}|pw}er
                        darüber empört sein werden. Alles Nähere mündlich.\hspace*{1.5em}Herzlich grüßt{ / }Alfred v. Berger\pwindex{Berger, Alfred von 30.04.1853 – 24.08.1912@\textsc{Berger, Alfred von} (30.04.1853 – 24.08.1912), \emph{Schriftsteller/Schriftstellerin, Journalist/Journalistin, Theaterleiter/Theaterleiterin}|pw}{ / }18/10 1901« (gedruckter Kopf: »Deutsches Schauspielhaus in Hamburg\oindex{Deutsches Schauspielhaus in Hamburg@\textbf{Deutsches Schauspielhaus in Hamburg}, \emph{Theater (K.THE)}|pw}«, \emph{Cambridge University Library}, Schnitzler,
                  B 10).}}}\label{K_L01185-1}{ }\introOben{}(aus Hamburg\oindex{Hamburg@\textbf{Hamburg}, \emph{P.PPLA}|pw})\introOben{}, dſs er
               mich perſönlich ſprechen wollte, in Angelegenheit der Stücke\pwindex{letzten Masken@\emph{Die letzten Masken}|pwv}\pwindex{Literatur@\emph{Literatur}|pwv}\pwindex{Frau mit dem Dolche@\emph{Die Frau mit dem Dolche}|pwv}.) –\pend
           
\pstart
           Die Dolchdame\pwindex{Frau mit dem Dolche@\emph{Die Frau mit dem Dolche}|pw} iſt gewiſs ein ſchweres ſcenisches
                  {\pb}Ding; aber ſo weit
               ſind wir heute doch ſchon in dieſen Sachen, dſs es unbedingt gehen muſs. –\pend
           
\pstart
           \textsc{Bukovics\pwindex{Bukovics, Emerich von 28.02.1844 – 04.07.1905@\textsc{Bukovics, Emerich von} (28.02.1844 – 04.07.1905), \emph{Journalist/Journalistin, Theaterleiter/Theaterleiterin}|pw}} hat mich neulich mit der Ausſicht entlaſſen, dſs er über die Beſetz nachdenken
               werde. Du haſt ja recht; ich muſs energiſcher mit ihm ſein, aber mir fehlt die rechte
               Begeiſterung für die vorausſichtliche Volks{\pb}theateraufführg. Nun es
               bleibt mir ja nichts andres übrig. Ich werde nächſtens »ſtürmiſch« einen \label{K_L01185-2v}\edtext{Contract mit einer Million Poenale
                  verlangen}{\lemma{\textnormal{\emph{Contract … verlangen}}}\Cendnote{\textnormal{Vgl. den Brief Schnitzlers an Emerich von Bukovics\pwindex{Bukovics, Emerich von 28.02.1844 – 04.07.1905@\textsc{Bukovics, Emerich von} (28.02.1844 – 04.07.1905), \emph{Journalist/Journalistin, Theaterleiter/Theaterleiterin}|pwk},
                     11. 12. 1901, Hermann Bahr, Arthur Schnitzler: \emph{Briefwechsel, Aufzeichnungen, Dokumente (1891–1931)}, Arthur Schnitzler an Emerich von Bukovics, 11. 12. 1901.}}}\label{K_L01185-2}.\pend
           
\pstart
           – Wie man die »Literatur\pwindex{Literatur@\emph{Literatur}|pw}« ſo beſonders gut
               finden kann, verſteh ich abſolut nicht; mein \textsc{faible}{ }ſind die »lebendigen Stunden\pwindex{Lebendige Stunden@\emph{Lebendige Stunden}|pw}.«\pend
           
\pstart
           Kainz\pwindex{Kainz, Josef 02.01.1858 – 20.09.1910@\textsc{Kainz, Josef} (02.01.1858 – 20.09.1910), \emph{Schauspieler/Schauspielerin}|pw} wollte am 5. den Gustl\pwindex{Lieutenant Gustl. Novelle@\emph{Lieutenant Gustl. Novelle}|pw}{ }{\pb}leſen; aber \introOben{}–\introOben{} Herr \label{K_L01185-3v}\edtext{Gutmann\pwindex{Gutmann, Albert 20.06.1851 – 07.03.1915@\textsc{Gutmann, Albert} (20.06.1851 – 07.03.1915), \emph{Veranstalter/Veranstalterin, Agent/Agentin}|pw}}{\lemma{\textnormal{\emph{Gutmann}}}\Cendnote{\textnormal{Betreiber einer Konzertagentur, die im
                     Bösendorfer-Saal\oindex{Boesendorfer-Saal@\textbf{Bösendorfer-Saal}, \emph{Veranstaltungsgebäude (K.VSB)}|pwk} Veranstaltungen
                  organisierte.}}}\label{K_L01185-3} hat Angſt gehabt. Ich werde anfangen, die militäriſche
               Verachtg gegen das Civil zu theilen.\pend
           
\pstart
           Herzlichſt dein{\\[\baselineskip]}\spacefill\mbox{Arthur}\pend
           \leftskip=0em{}
\pstart
           28. X. 901.\pend
           \selectlanguage{ngerman}\endnumbering\briefempfaengerindex{Bahr, Hermann@\textsc{Bahr, Hermann}!zzzSchnitzler, Arthur@\emph{von Arthur Schnitzler}!1901-10-281@{28. 10. 1901}|)be}\mylabel{L01185h}  \normalsize

\doendnotes{C}
\bigskip
\vfill

\clearpage

\footnotesize

\lohead{\textsc{register}}

% Definiere theindex-Environment komplett neu ohne reledmac
\makeatletter
\renewenvironment{theindex}{%
  \section*{\indexname}%
  \setlength{\parindent}{0pt}%
  \setlength{\parskip}{0pt plus 0.3pt}%
  \let\item\@idxitem
}{%
  \clearpage
}
\makeatother

\IfFileExists{\jobname-pw.ind}{\input{\jobname-pw.ind}}{}

\end{document}

      