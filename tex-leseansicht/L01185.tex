%% latex-leseansicht-vorspann.tex
%% Vorspann für die Leseansicht.
%% Lädt die gemeinsame Datei latex-vorspann.tex mit nicht gesetztem Schalter.

\newif\ifkorrekturansicht
\korrekturansichtfalse

\input{../tex-inputs/latex-vorspann}


\section[Arthur Schnitzler an Hermann Bahr, 28. 10. 1901]{L01185 Arthur Schnitzler an Hermann Bahr, 28. 10. 1901}
\nopagebreak\mylabel{L01185v}
\rehead{ }\normalsize\beginnumbering\briefempfaengerindex{Bahr, Hermann@\textsc{Bahr, Hermann}!zzzSchnitzler, Arthur@\emph{von Arthur Schnitzler}!1901-10-281@{28. 10. 1901}|(be}
\toendnotes[C]{\smallbreak\pagebreak[2]}
\correspDesc{Versand  durch Arthur Schnitzler am 28. 10. 1901 in Wien
\newline{}Erhalt  durch Hermann Bahr im Zeitraum [28. 10. 1901 – 1. 11. 1901?] in Wien}\toendnotes[C]{\smallbreak}
\Standort{TMW, HS AM 23347 Ba.}
\physDesc{Brief, 1 Blatt, 4 Seiten, 996 Zeichen
\newline{}Handschrift: schwarze Tinte, deutsche Kurrent
\newline{}Ordnung: 1) Lochung  2) mit Bleistift von unbekannter Hand datiert: »26 X. 01«}
\buchAbdrucke{\weitereDrucke{1) \emph{28. 10. 1901.} In: Arthur Schnitzler: \emph{The Letters of Arthur Schnitzler to Hermann Bahr}. Edited, annotated, and with an introduction, by Donald G. Daviau. Chapel Hill: \emph{The University of North Carolina Press} 1978, S. 72 (University of North Carolina studies in the Germanic languages
                        and literatures, 89).} \weitereDrucke{2) Hermann Bahr, Arthur Schnitzler: \emph{Briefwechsel, Aufzeichnungen, Dokumente (1891–1931)}. Herausgegeben von Kurt Ifkovits und Martin Anton Müller. Göttingen: \emph{Wallstein} 2018, S. 217.} }\toendnotes[C]{\smallbreak}
\pstart{}{\pb}lieber Hermann,\pend\vspace{0.5em}
\pstart
           aus deinem lieben Brief entnehme ich u. a. dſs Berger\pwindex{Berger, Alfred von 30.\,4.\,1853 Wien – 24.\,8.\,1912 ebd.@\textsc{Berger, Alfred von} (30.\,4.\,1853 Wien – 24.\,8.\,1912 ebd.), \emph{Schriftsteller, Journalist, Theaterleiter}|pw} hier war. \uline{Iſt} er noch in Wien\oindex{Wien@\textbf{Wien}, \emph{Verwaltungsgebiet}|pw}? (Er schrieb mir eine \label{K_L01185-1v}\edtext{Karte}{\lemma{\textnormal{\emph{Karte}}}\Cendnote{\textnormal{»Hochgeehrter Herr Doctor!{ / }Nächste Woche spreche ich Sie in Wien\oindex{Wien@\textbf{Wien}, \emph{Verwaltungsgebiet}|pw}.
                        Ich bin von den ›letzten Stunden\pwindex{Schnitzler, Arthur 15.\,5.\,1862 Wien – 21.\,10.\,1931 ebd.@\textsc{Schnitzler, Arthur} (15.\,5.\,1862 Wien – 21.\,10.\,1931 ebd.), \emph{Schriftsteller, Mediziner}!Lebendige Stunden. Vier Einakter@\strich\emph{Lebendige Stunden. Vier Einakter}|pw}‹
                        entzückt, so entzückt, als die Hamburg\oindex{Hamburg@\textbf{Hamburg}|pw}er
                        darüber empört sein werden. Alles Nähere mündlich.\hspace*{1.5em}Herzlich grüßt{ / }Alfred v. Berger\pwindex{Berger, Alfred von 30.\,4.\,1853 Wien – 24.\,8.\,1912 ebd.@\textsc{Berger, Alfred von} (30.\,4.\,1853 Wien – 24.\,8.\,1912 ebd.), \emph{Schriftsteller, Journalist, Theaterleiter}|pw}{ / }18/10 1901« (gedruckter Kopf: »Deutsches Schauspielhaus in Hamburg\oindex{Deutsches Schauspielhaus in Hamburg@\textbf{Deutsches Schauspielhaus in Hamburg}, \emph{Theater}|pw}«, \emph{Cambridge University Library}, Schnitzler,
                  B 10).}}}\label{K_L01185-1}{ }\introOben{}(aus Hamburg\oindex{Hamburg@\textbf{Hamburg}|pw})\introOben{}, dſs er
               mich perſönlich{ }ſprechen wollte, in Angelegenheit der Stücke\pwindex{Schnitzler, Arthur 15.\,5.\,1862 Wien – 21.\,10.\,1931 ebd.@\textsc{Schnitzler, Arthur} (15.\,5.\,1862 Wien – 21.\,10.\,1931 ebd.), \emph{Schriftsteller, Mediziner}!letzten Masken@\strich\emph{Die letzten Masken}|pwv}\pwindex{Schnitzler, Arthur 15.\,5.\,1862 Wien – 21.\,10.\,1931 ebd.@\textsc{Schnitzler, Arthur} (15.\,5.\,1862 Wien – 21.\,10.\,1931 ebd.), \emph{Schriftsteller, Mediziner}!Literatur@\strich\emph{Literatur}|pwv}\pwindex{Schnitzler, Arthur 15.\,5.\,1862 Wien – 21.\,10.\,1931 ebd.@\textsc{Schnitzler, Arthur} (15.\,5.\,1862 Wien – 21.\,10.\,1931 ebd.), \emph{Schriftsteller, Mediziner}!Frau mit dem Dolche@\strich\emph{Die Frau mit dem Dolche}|pwv}.) –\pend
           
\pstart
           Die Dolchdame\pwindex{Schnitzler, Arthur 15.\,5.\,1862 Wien – 21.\,10.\,1931 ebd.@\textsc{Schnitzler, Arthur} (15.\,5.\,1862 Wien – 21.\,10.\,1931 ebd.), \emph{Schriftsteller, Mediziner}!Frau mit dem Dolche@\strich\emph{Die Frau mit dem Dolche}|pw} iſt gewiſs ein{ }ſchweres{ }ſcenisches
               {\pb}Ding; aber{ }ſo weit{ }ſind wir heute doch{ }ſchon in dieſen Sachen, dſs es unbedingt gehen muſs. –\pend
           
\pstart
           \textsc{Bukovics\pwindex{Bukovics, Emerich von 28.\,2.\,1844 Wien – 4.\,7.\,1905 ebd.@\textsc{Bukovics, Emerich von} (28.\,2.\,1844 Wien – 4.\,7.\,1905 ebd.), \emph{Journalist, Theaterleiter}|pw}} hat mich neulich mit der Ausſicht entlaſſen, dſs er über die Beſetz nachdenken
               werde. Du haſt ja recht; ich muſs energiſcher mit ihm{ }ſein, aber mir fehlt die rechte
               Begeiſterung für die vorausſichtliche Volks{\pb}theateraufführg. Nun es
               bleibt mir ja nichts andres übrig. Ich werde nächſtens »ſtürmiſch« einen \label{K_L01185-2v}\edtext{Contract mit einer Million Poenale
                  verlangen}{\lemma{\textnormal{\emph{Contract … verlangen}}}\Cendnote{\textnormal{Vgl. den Brief Schnitzlers an Emerich von Bukovics\pwindex{Bukovics, Emerich von 28.\,2.\,1844 Wien – 4.\,7.\,1905 ebd.@\textsc{Bukovics, Emerich von} (28.\,2.\,1844 Wien – 4.\,7.\,1905 ebd.), \emph{Journalist, Theaterleiter}|pwk},
                     11. 12. 1901, Hermann Bahr, Arthur Schnitzler: \emph{Briefwechsel, Aufzeichnungen, Dokumente (1891–1931)}, Arthur Schnitzler an Emerich von Bukovics, 11. 12. 1901.}}}\label{K_L01185-2}.\pend
           
\pstart
           – Wie man die »Literatur\pwindex{Schnitzler, Arthur 15.\,5.\,1862 Wien – 21.\,10.\,1931 ebd.@\textsc{Schnitzler, Arthur} (15.\,5.\,1862 Wien – 21.\,10.\,1931 ebd.), \emph{Schriftsteller, Mediziner}!Literatur@\strich\emph{Literatur}|pw}«{ }ſo beſonders gut
               finden kann, verſteh ich abſolut nicht; mein \textsc{faible}{ }ſind die »lebendigen Stunden\pwindex{Schnitzler, Arthur 15.\,5.\,1862 Wien – 21.\,10.\,1931 ebd.@\textsc{Schnitzler, Arthur} (15.\,5.\,1862 Wien – 21.\,10.\,1931 ebd.), \emph{Schriftsteller, Mediziner}!Lebendige Stunden@\strich\emph{Lebendige Stunden}|pw}.«\pend
           
\pstart
           Kainz\pwindex{Kainz, Josef 2.\,1.\,1858 Mosonmagyaróvár – 20.\,9.\,1910 Wien@\textsc{Kainz, Josef} (2.\,1.\,1858 Mosonmagyaróvár – 20.\,9.\,1910 Wien), \emph{Schauspieler}|pw} wollte am 5. den Gustl\pwindex{Schnitzler, Arthur 15.\,5.\,1862 Wien – 21.\,10.\,1931 ebd.@\textsc{Schnitzler, Arthur} (15.\,5.\,1862 Wien – 21.\,10.\,1931 ebd.), \emph{Schriftsteller, Mediziner}!Lieutenant Gustl. Novelle@\strich\emph{Lieutenant Gustl. Novelle}|pw}{ }{\pb}leſen; aber \introOben{}–\introOben{} Herr \label{K_L01185-3v}\edtext{Gutmann\pwindex{Gutmann, Albert 20.\,6.\,1851 Fürth – 7.\,3.\,1915 Wien@\textsc{Gutmann, Albert} (20.\,6.\,1851 Fürth – 7.\,3.\,1915 Wien), \emph{Veranstalter, Agent}|pw}}{\lemma{\textnormal{\emph{Gutmann}}}\Cendnote{\textnormal{Betreiber einer Konzertagentur, die im
                     Bösendorfer-Saal\oindex{Wien@\textbf{Wien}!I., Innere Stadt@\textbf{I., Innere Stadt}!Bösendorfer-Saal@\textbf{Bösendorfer-Saal}, \emph{Veranstaltungsgebäude}|pwk} Veranstaltungen
                  organisierte.}}}\label{K_L01185-3} hat Angſt gehabt. Ich werde anfangen, die militäriſche
               Verachtg gegen das Civil zu theilen.\pend
           
\pstart
           Herzlichſt dein{\\[\baselineskip]}\spacefill\mbox{Arthur}\pend
           \leftskip=0em{}
\pstart
           28. X. 901.\pend
           \selectlanguage{ngerman}\endnumbering\briefempfaengerindex{Bahr, Hermann@\textsc{Bahr, Hermann}!zzzSchnitzler, Arthur@\emph{von Arthur Schnitzler}!1901-10-281@{28. 10. 1901}|)be}\mylabel{L01185h}  \newcommand{\dateiname}{L01185}\newcommand{\titel}{Arthur Schnitzler an Hermann Bahr, 28. 10. 1901}\newcommand{\editorInnen}{Herausgegeben von Martin Anton Müller}%% latex-leseansicht-abspann.tex
%% Abspann für die Leseansicht.
%% Der Schalter \ifkorrekturansicht ist bereits durch den Vorspann gesetzt.

%% latex-abspann.tex
%% Gemeinsamer Abspann für Korrekturansicht und Leseansicht.
%% Setzt den Schalter \ifkorrekturansicht voraus (gesetzt in den
%% einbindenden Dateien latex-korrekturansicht-abspann.tex bzw.
%% latex-leseansicht-abspann.tex).
%% ---------------------------------------------------------------

\normalsize

% Das esempio-Environment wird nur in der Leseansicht benötigt
\ifkorrekturansicht\else
\newenvironment{esempio}[3]%
{
    \vspace{1.5ex}
    \rlap{\underline{#1}}
    \par
    \setlength{\parindent}{0cm}
    \nopagebreak
    \leftskip=#2cm
    \rightskip=#3cm
}
{
    \par
}
\fi

\doendnotes{C}
\bigskip
\vfill

\clearpage

\footnotesize

\ifkorrekturansicht
  \lohead{\textsc{register}}
\fi

% theindex-Environment neu definieren ohne reledmac
\makeatletter
\renewenvironment{theindex}{%
  \ifkorrekturansicht
    \section*{\indexname}%
  \else
    \subsubsection*{Index der erwähnten Entitäten}%
  \fi
  \setlength{\parindent}{0pt}%
  \setlength{\parskip}{0pt plus 0.3pt}%
  \let\item\@idxitem
}{%
  \ifkorrekturansicht\clearpage\fi
}
\makeatother

\IfFileExists{\jobname-pw.ind}{\input{\jobname-pw.ind}}{}

% Quellenangabe nur in der Leseansicht
\ifkorrekturansicht\else
% Fallback-Definitionen, falls die .tex-Datei \titel etc. nicht gesetzt hat
\providecommand{\titel}{}
\providecommand{\editorInnen}{}
\providecommand{\dateiname}{\jobname}

\vspace{3cm}

\vfill

\footnotesize
\textsc{Quelle}: \titel. Herausgegeben von {\editorInnen}. In: \emph{Arthur Schnitzler: Briefwechsel mit Autorinnen und Autoren}.
 Digitale Edition, https://schnitzler-briefe.acdh.oeaw.ac.at/{\dateiname}.html (Stand \today)
\fi

\end{document}


