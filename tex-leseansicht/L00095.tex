%% latex-leseansicht-vorspann.tex
%% Vorspann für die Leseansicht.
%% Lädt die gemeinsame Datei latex-vorspann.tex mit nicht gesetztem Schalter.

\newif\ifkorrekturansicht
\korrekturansichtfalse

\input{../tex-inputs/latex-vorspann}


         
         \renewcommand{\erwaehntePersonen}{Personen: Wilhelm Bölsche}
         \renewcommand{\erwaehnteOrte}{Orte: Berlin, Ordination Dr. Arthur Schnitzler Giselastraße 11, Wien}
         \renewcommand{\erwaehnteWerke}{Werke: Das Himmelbett}
               \section[Arthur Schnitzler an Wilhelm Bölsche, 20. 4. 1892]{ Arthur Schnitzler an Wilhelm Bölsche, 20. 4. 1892}\nopagebreak\mylabel{v}\rehead{ }\begin{ledgroupsized}[t]{13cm}\normalsize\beginnumbering\briefempfaengerindex{Boelsche, Wilhelm@\textsc{Bölsche, Wilhelm}!zzzSchnitzler, Arthur@\emph{von Arthur Schnitzler}!1892-04-201@{20. 4. 1892}|(be} \toendnotes[C]{\smallbreak\pagebreak[2]} \Standort{Wrocław, Biblioteka Uniwersytecka, Böl.Pis 1764.}
\physDesc{Brief, 1 Blatt, 2 Seiten, 619 Zeichen (Seite 3 quer zur üblichen Schreibrichtung)
\newline{}Handschrift: schwarze Tinte, deutsche Kurrent
\newline{}Bölsche: mit schwarzer Tinte als »Erl{[}edigt{]}« gezeichnet }\buchAbdrucke{\weitereDrucke{1) Alois Woldan: \emph{Arthur Schnitzler – Briefe an Wilhelm Bölsche.} In: \emph{Germanica Wratislaviensia} (1987) Nr. 77, S. 460.} \weitereDrucke{2) Wilhelm Bölsche: \emph{Briefwechsel. Mit Autoren der Freien Bühne}. Hg. Gerd-Hermann Susen. Berlin: \emph{Weidler} 2010, S. 680 (Werke und Briefe. Wissenschaftliche Ausgabe, Briefe I).} }\toendnotes[C]{\smallbreak}\pstart
           \raggedleft{}{\pb}Wien\oindex{Wien@\textbf{Wien}|pw}, 20. April 92\pend
           \pstart{}Verehrteſter Herr,\pend\pstart
           ich ſchicke Ihnen hier die Skizze\pwindex{Schnitzler, Arthur 15.05.1862 – 21.10.1931@\textsc{Schnitzler, Arthur} (15.05.1862 – 21.10.1931), \emph{Schriftsteller, Mediziner}!Himmelbett1977@\strich\emph{Das Himmelbett} {[}1977{]}|pwv} mit der beſondern Bitte, mir falls Sie ſie zu veröffentlichen
               gedenken, gütigſt eine \uuline{Correctur}{ }ſenden laſſen zu wollen; ſie ſoll beſti{\geminationm}t in 24 Stunden erledigt ſein. Sollten Sie das Manuscript\pwindex{Schnitzler, Arthur 15.05.1862 – 21.10.1931@\textsc{Schnitzler, Arthur} (15.05.1862 – 21.10.1931), \emph{Schriftsteller, Mediziner}!Himmelbett1977@\strich\emph{Das Himmelbett} {[}1977{]}|pwv}{ }{\pb}nicht brauchen können, was mir aufrichtig leid thäte, ſo
               haben Sie wohl die Liebenswürdigkeit, es mir recht bald zurückzuſenden.\pend
           \pstart
           Hochachtungsvoll{\\[\baselineskip]}Ihr ſehr ergebner{\\[\baselineskip]}\spacefill\mbox{Dr Arthur Schnitzler}\pend
           \leftskip=0em{}\pstart
           \noindent{}\textsc{I. Giselastraße 11}.\oindex{Ordination Dr. Arthur Schnitzler Giselastrasse 11@\textbf{Ordination Dr. Arthur Schnitzler Giselastraße 11}|pw}\pend
           \pstart
           {\pb}Scheint Ihnen etwa der Titel zu riskant, ſo könnte die
                     Skizze\pwindex{Schnitzler, Arthur 15.05.1862 – 21.10.1931@\textsc{Schnitzler, Arthur} (15.05.1862 – 21.10.1931), \emph{Schriftsteller, Mediziner}!Himmelbett1977@\strich\emph{Das Himmelbett} {[}1977{]}|pwv} auch »Verblaßende Farben\pwindex{Schnitzler, Arthur 15.05.1862 – 21.10.1931@\textsc{Schnitzler, Arthur} (15.05.1862 – 21.10.1931), \emph{Schriftsteller, Mediziner}!Himmelbett1977@\strich\emph{Das Himmelbett} {[}1977{]}|pw}« genannt werden; lieber iſt
                  mir allerdings der erſte »Das
                  Himmelbett\pwindex{Schnitzler, Arthur 15.05.1862 – 21.10.1931@\textsc{Schnitzler, Arthur} (15.05.1862 – 21.10.1931), \emph{Schriftsteller, Mediziner}!Himmelbett1977@\strich\emph{Das Himmelbett} {[}1977{]}|pw}.«\pend
           \pstart
           \raggedleft{}ArthSch\pend
           
         
         \endnumbering\mylabel{h}\end{ledgroupsized}  \newcommand{\dateiname}{L00095}\newcommand{\titel}{Arthur Schnitzler an Wilhelm Bölsche, 20. 4. 1892}\newcommand{\editorInnen}{Martin Anton Müller und Gerd-Hermann Susen}%% latex-leseansicht-abspann.tex
%% Abspann für die Leseansicht.
%% Der Schalter \ifkorrekturansicht ist bereits durch den Vorspann gesetzt.

%% latex-abspann.tex
%% Gemeinsamer Abspann für Korrekturansicht und Leseansicht.
%% Setzt den Schalter \ifkorrekturansicht voraus (gesetzt in den
%% einbindenden Dateien latex-korrekturansicht-abspann.tex bzw.
%% latex-leseansicht-abspann.tex).
%% ---------------------------------------------------------------

\normalsize

% Das esempio-Environment wird nur in der Leseansicht benötigt
\ifkorrekturansicht\else
\newenvironment{esempio}[3]%
{
    \vspace{1.5ex}
    \rlap{\underline{#1}}
    \par
    \setlength{\parindent}{0cm}
    \nopagebreak
    \leftskip=#2cm
    \rightskip=#3cm
}
{
    \par
}
\fi

\doendnotes{C}
\bigskip
\vfill

\clearpage

\footnotesize

\ifkorrekturansicht
  \lohead{\textsc{register}}
\fi

% theindex-Environment neu definieren ohne reledmac
\makeatletter
\renewenvironment{theindex}{%
  \ifkorrekturansicht
    \section*{\indexname}%
  \else
    \subsubsection*{Index der erwähnten Entitäten}%
  \fi
  \setlength{\parindent}{0pt}%
  \setlength{\parskip}{0pt plus 0.3pt}%
  \let\item\@idxitem
}{%
  \ifkorrekturansicht\clearpage\fi
}
\makeatother

\IfFileExists{\jobname-pw.ind}{\input{\jobname-pw.ind}}{}

% Quellenangabe nur in der Leseansicht
\ifkorrekturansicht\else
% Fallback-Definitionen, falls die .tex-Datei \titel etc. nicht gesetzt hat
\providecommand{\titel}{}
\providecommand{\editorInnen}{}
\providecommand{\dateiname}{\jobname}

\vspace{3cm}

\vfill

\footnotesize
\textsc{Quelle}: \titel. Herausgegeben von {\editorInnen}. In: \emph{Arthur Schnitzler: Briefwechsel mit Autorinnen und Autoren}.
 Digitale Edition, https://schnitzler-briefe.acdh.oeaw.ac.at/{\dateiname}.html (Stand \today)
\fi

\end{document}


      