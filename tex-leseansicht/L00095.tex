%% latex-korrekturansicht-vorspann.tex
%% Vorspann für die Korrekturansicht.
%% Lädt die gemeinsame Datei latex-vorspann.tex mit gesetztem Schalter.

\newif\ifkorrekturansicht
\korrekturansichttrue

\input{../tex-inputs/latex-vorspann}


\section[Arthur Schnitzler an Wilhelm Bölsche, 20. 4. 1892]{L00095 Arthur Schnitzler an Wilhelm Bölsche, 20. 4. 1892}
\nopagebreak\mylabel{L00095v}
\rehead{ }\normalsize\beginnumbering\briefempfaengerindex{Boelsche, Wilhelm@\textsc{Bölsche, Wilhelm}!zzzSchnitzler, Arthur@\emph{von Arthur Schnitzler}!1892-04-201@{20. 4. 1892}|(be}
\toendnotes[C]{\smallbreak\pagebreak[2]}\Standort{Wrocław, Biblioteka Uniwersytecka, Böl.Pis 1764.}
\physDesc{Brief, 1 Blatt, 2 Seiten, 619 Zeichen (Seite 3 quer zur üblichen Schreibrichtung)
\newline{}Handschrift: schwarze Tinte, deutsche Kurrent
\newline{}Bölsche: mit schwarzer Tinte als »Erl{[}edigt{]}« gezeichnet }
\buchAbdrucke{\weitereDrucke{1) \emph{Germanica Wratislaviensia} (1987) Nr. 77, S. 460.} \weitereDrucke{2) Wilhelm Bölsche: \emph{Briefwechsel. Mit Autoren der Freien Bühne}. Berlin: \emph{Weidler} 2010, S. 680.} }\toendnotes[C]{\smallbreak}
\pstart
           \raggedleft{}{\pb}Wien\oindex{Wien@\textbf{Wien}, \emph{A.ADM2}|pw}, 20. April 92\pend
           
\pstart{}Verehrteſter Herr,\pend\vspace{0.5em}
\pstart
           ich ſchicke Ihnen hier die Skizze\pwindex{Himmelbett@\emph{Das Himmelbett}|pwv} mit der beſondern Bitte, mir falls Sie ſie zu veröffentlichen
               gedenken, gütigſt eine \uuline{Correctur}{ }ſenden laſſen zu wollen; ſie ſoll beſti{\geminationm}t in 24 Stunden erledigt ſein. Sollten Sie das Manuscript\pwindex{Himmelbett@\emph{Das Himmelbett}|pwv}{ }{\pb}nicht brauchen können, was mir aufrichtig leid thäte, ſo
               haben Sie wohl die Liebenswürdigkeit, es mir recht bald zurückzuſenden.\pend
           
\pstart
           Hochachtungsvoll{\\[\baselineskip]}Ihr ſehr ergebner{\\[\baselineskip]}\spacefill\mbox{Dr Arthur Schnitzler}\pend
           \leftskip=0em{}
\pstart
           \noindent{}\textsc{I. Giselastraße 11}.\oindex{Ordination Arthur Schnitzler [Boesendorferstrasse 11]@\textbf{Ordination Arthur Schnitzler [Bösendorferstraße 11]}, \emph{Ordination}|pw}\pend
           
\pstart
           {\pb}Scheint Ihnen etwa der Titel zu riskant, ſo könnte die
                     Skizze\pwindex{Himmelbett@\emph{Das Himmelbett}|pwv} auch »Verblaßende Farben\pwindex{Himmelbett@\emph{Das Himmelbett}|pw}« genannt werden; lieber iſt
                  mir allerdings der erſte »Das
                  Himmelbett\pwindex{Himmelbett@\emph{Das Himmelbett}|pw}.«\pend
           
\pstart
           \raggedleft{}ArthSch\pend
           \selectlanguage{ngerman}\endnumbering\briefempfaengerindex{Boelsche, Wilhelm@\textsc{Bölsche, Wilhelm}!zzzSchnitzler, Arthur@\emph{von Arthur Schnitzler}!1892-04-201@{20. 4. 1892}|)be}\mylabel{L00095h}  \normalsize

\doendnotes{C}
\bigskip
\vfill

\clearpage

\footnotesize

\lohead{\textsc{register}}

% Definiere theindex-Environment komplett neu ohne reledmac
\makeatletter
\renewenvironment{theindex}{%
  \section*{\indexname}%
  \setlength{\parindent}{0pt}%
  \setlength{\parskip}{0pt plus 0.3pt}%
  \let\item\@idxitem
}{%
  \clearpage
}
\makeatother

\IfFileExists{\jobname-pw.ind}{\input{\jobname-pw.ind}}{}

\end{document}

      