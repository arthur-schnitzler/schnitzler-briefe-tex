%% latex-korrekturansicht-vorspann.tex
%% Vorspann für die Korrekturansicht.
%% Lädt die gemeinsame Datei latex-vorspann.tex mit gesetztem Schalter.

\newif\ifkorrekturansicht
\korrekturansichttrue

\input{../tex-inputs/latex-vorspann}


\section[Arthur Schnitzler an Hugo von Hofmannsthal, 4. 11. 1903]{L01335 Arthur Schnitzler an Hugo von Hofmannsthal, 4. 11. 1903}
\nopagebreak\mylabel{L01335v}
\rehead{ }\normalsize\beginnumbering\briefempfaengerindex{Hofmannsthal, Hugo von@\textsc{Hofmannsthal, Hugo von}!zzzSchnitzler, Arthur@\emph{von Arthur Schnitzler}!1903-11-041@{4. 11. 1903}|(be}
\toendnotes[C]{\smallbreak\pagebreak[2]}\Standort{FDH, Hs-30885,105.}
\physDesc{Brief, 1 Blatt, 4 Seiten, 827 Zeichen
\newline{}Handschrift: Bleistift, deutsche Kurrent}
\buchAbdrucke{\weitereDrucke{Hugo von Hofmannsthal, Arthur Schnitzler: \emph{Briefwechsel}. Frankfurt am Main: \emph{S. Fischer} 1964, S. 176.} }\toendnotes[C]{\smallbreak}
\pstart
           \raggedleft{}{\pb}\textsc{Spöttelgasse 7}\oindex{Edmund-Weiss-Gasse 7@\textbf{Edmund-Weiß-Gasse 7}, \emph{Wohngebäude (K.WHS)}|pw}. 4. 11. 903\pend
           
\pstart{}lieber Hugo, \pend\vspace{0.5em}
\pstart
           über Elektra\pwindex{Elektra. Tragoedie in einem Aufzug@\emph{Elektra. Tragödie in einem Aufzug}|pw} hab ich mich ſehr gefreut, und das
                  
                     Goldma{\geminationn}sche\pwindex{Goldmann, Paul 31.01.1865 – 25.09.1935@\textsc{Goldmann, Paul} (31.01.1865 – 25.09.1935), \emph{Schriftsteller/Schriftstellerin, Journalist/Journalistin}|pw}{ }Telegramm\pwindex{Aus Berlin [Elektra-Premiere]@\emph{Aus Berlin [Elektra-Premiere]}|pwv} gehört zu
               dem Übrigen. Denken Sie, daſs er \strikeout{mir}, ſeit er Wien\oindex{Wien@\textbf{Wien}, \emph{A.ADM2}|pw} verlaſſen \strikeout{hat},
                  Mitte September, \label{K_L01335-1v}\edtext{keine Zeile an mich geſchrieben}{\lemma{\textnormal{\emph{keine … geſchrieben}}}\Cendnote{\textnormal{Den
                  nächsten Brief von Goldmann\pwindex{Goldmann, Paul 31.01.1865 – 25.09.1935@\textsc{Goldmann, Paul} (31.01.1865 – 25.09.1935), \emph{Schriftsteller/Schriftstellerin, Journalist/Journalistin}|pwk} erhielt er am
                     14. 11. [1903].}}}\label{K_L01335-1}
               hat.\pend
           
\pstart
           – Das Stück\pwindex{einsame Weg. Schauspiel in fuenf Akten@\emph{Der einsame Weg. Schauspiel in fünf Akten}|pwv} iſt ſchon an Brahm\pwindex{Brahm, Otto 05.02.1856 – 28.11.1912@\textsc{Brahm, Otto} (05.02.1856 – 28.11.1912), \emph{Theaterleiter/Theaterleiterin, Regisseur/Regisseurin}|pw} abgegangen. Freitag gehn wir
                  {\pb}auf ein paar Tage auf den Semmering\oindex{Semmering@\textbf{Semmering}, \emph{A.ADM3}|pw}. Mitte nächſter Woche möchte ich \label{K_L01335-2v}\edtext{vorleſen\pwindex{einsame Weg. Schauspiel in fuenf Akten@\emph{Der einsame Weg. Schauspiel in fünf Akten}|pwv}}{\lemma{\textnormal{\emph{vorleſen}}}\Cendnote{\textnormal{Vgl. A. S.: \emph{Tagebuch}, 12. 11. 1903.
               }}}\label{K_L01335-2}. Sagen Sie mir bitte, ob Ihnen Dienſtag{ }Abend ½ 7 recht wäre. Fragen Sie auch gleich den Richard\pwindex{Beer-Hofmann, Richard 1866-07-11 – 1945-09-26@\textsc{Beer-Hofmann, Richard} (1866-07-11 – 1945-09-26), \emph{Schriftsteller/Schriftstellerin}|pw}.\pend
           
\pstart
           Dieſer Tage iſt die \label{K_L01335-3v}\edtext{\textsc{Kakadu}\pwindex{gruene Kakadu. Groteske in einem Akt@\emph{Der grüne Kakadu. Groteske in einem Akt}|pw}\pwindex{Au Perroquet Vert@\emph{Au Perroquet Vert}|pw}\textsc{première} in Paris\oindex{Paris@\textbf{Paris}, \emph{P.PPLC}|pw}}{\lemma{\textnormal{\emph{Kakadupremière in Paris}}}\Cendnote{\textnormal{am 7. 11. 1903}}}\label{K_L01335-3}; \textsc{Antoine}\pwindex{Antoine, Andre 1858-01-31 – 1943-10-23@\textsc{Antoine, André} (1858-01-31 – 1943-10-23), \emph{Theaterleiter/Theaterleiterin, Schauspieler/Schauspielerin}|pw}{ }ſcheint ſich nach einem Brief von ihm und von
               einigen andern, die Proben geſehen haben, viel {\pb}zu
               verſprechen.\pend
           
\pstart
           Grüßen Sie von uns beiden herzlich \textsc{Gerty}\pwindex{Hofmannsthal, Gertrude von 16.03.1880 – 09.11.1959@\textsc{Hofmannsthal, Gertrude von} (16.03.1880 – 09.11.1959)|pw} und Hofmannsthal den
                  Winzigen\pwindex{Hofmannsthal, Franz von 20.10.1903 – 13.07.1929@\textsc{Hofmannsthal, Franz von} (20.10.1903 – 13.07.1929)|pwv}. Sich ſelber desgleichen.\pend
           
\pstart
           – Hat ſich die Burg\orgindex{Burgtheater@Burgtheater|pw} um die ihrer Hoheit entkleidete Griechin\pwindex{Elektra. Tragoedie in einem Aufzug@\emph{Elektra. Tragödie in einem Aufzug}|pwv}
                  beworben?{\dotstwo} Aus dem alten \textsc{Sophokles}\pwindex{Elektra. Tragoedie@\emph{Elektra. Tragödie}|pwv}\pwindex{Sophokles 497/496? v. u. Z. – 406/405 v. u. Z.@\textsc{Sophokles} (497/496? v. u. Z. – 406/405 v. u. Z.), \emph{Schriftsteller/Schriftstellerin}|pw} ein Zugstück zu machen! Echt {\pb}jüdiſch.\pend
           
\pstart
           Ihr{\\[\baselineskip]}\spacefill\mbox{A.}\pend
           \leftskip=0em{}\selectlanguage{ngerman}\endnumbering\briefempfaengerindex{Hofmannsthal, Hugo von@\textsc{Hofmannsthal, Hugo von}!zzzSchnitzler, Arthur@\emph{von Arthur Schnitzler}!1903-11-041@{4. 11. 1903}|)be}\mylabel{L01335h}  \normalsize

\doendnotes{C}
\bigskip
\vfill

\clearpage

\footnotesize

\lohead{\textsc{register}}

% Definiere theindex-Environment komplett neu ohne reledmac
\makeatletter
\renewenvironment{theindex}{%
  \section*{\indexname}%
  \setlength{\parindent}{0pt}%
  \setlength{\parskip}{0pt plus 0.3pt}%
  \let\item\@idxitem
}{%
  \clearpage
}
\makeatother

\IfFileExists{\jobname-pw.ind}{\input{\jobname-pw.ind}}{}

\end{document}

      