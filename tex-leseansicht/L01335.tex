%% latex-leseansicht-vorspann.tex
%% Vorspann für die Leseansicht.
%% Lädt die gemeinsame Datei latex-vorspann.tex mit nicht gesetztem Schalter.

\newif\ifkorrekturansicht
\korrekturansichtfalse

\input{../tex-inputs/latex-vorspann}


\section[Arthur Schnitzler an Hugo von Hofmannsthal, 4. 11. 1903]{L01335 Arthur Schnitzler an Hugo von Hofmannsthal, 4. 11. 1903}
\nopagebreak\mylabel{L01335v}
\rehead{ }\normalsize\beginnumbering\briefempfaengerindex{Hofmannsthal, Hugo von@\textsc{Hofmannsthal, Hugo von}!zzzSchnitzler, Arthur@\emph{von Arthur Schnitzler}!1903-11-041@{4. 11. 1903}|(be}
\toendnotes[C]{\smallbreak\pagebreak[2]}
\correspDesc{Versand  durch Arthur Schnitzler am 4. 11. 1903 in Wien
\newline{}Erhalt  durch Hugo von Hofmannsthal im Zeitraum [4. 11. 1903
                  – 8. 11. 1903?] in Wien}\toendnotes[C]{\smallbreak}
\Standort{FDH, Hs-30885,105.}
\physDesc{Brief, 1 Blatt, 4 Seiten, 827 Zeichen
\newline{}Handschrift: Bleistift, deutsche Kurrent}
\buchAbdrucke{\weitereDrucke{Hugo von Hofmannsthal, Arthur Schnitzler: \emph{Briefwechsel}. Herausgegeben von Therese Nickl und Heinrich Schnitzler. Frankfurt am Main: \emph{S. Fischer} 1964, S. 176.} }\toendnotes[C]{\smallbreak}
\pstart
           \raggedleft{}{\pb}\textsc{Spöttelgasse 7}\oindex{Wien@\textbf{Wien}!XVIII., Währing@\textbf{XVIII., Währing}!Edmund-Weiß-Gasse 7@\textbf{Edmund-Weiß-Gasse 7}, \emph{Wohngebäude}|pw}. 4. 11. 903\pend
           
\pstart{}lieber Hugo,\pend\vspace{0.5em}
\pstart
           über Elektra\pwindex{Hofmannsthal, Hugo von 1.\,2.\,1874 Wien – 15.\,7.\,1929 Rodaun@\textsc{Hofmannsthal, Hugo von} (1.\,2.\,1874 Wien – 15.\,7.\,1929 Rodaun), \emph{Schriftsteller}!Elektra. Tragödie in einem Aufzug@\strich\emph{Elektra. Tragödie in einem Aufzug}|pw} hab ich mich{ }ſehr gefreut, und das
                  Goldma{\geminationn}sche\pwindex{Goldmann, Paul 31.\,1.\,1865 Breslau – 25.\,9.\,1935 Wien@\textsc{Goldmann, Paul} (31.\,1.\,1865 Breslau – 25.\,9.\,1935 Wien), \emph{Schriftsteller, Journalist}|pw}{ }Telegramm\pwindex{Aus Berlin [Elektra-Premiere]@\emph{Aus Berlin [Elektra-Premiere]}|pwv} gehört zu
               dem Übrigen. Denken Sie, daſs er \strikeout{mir},{ }ſeit er Wien\oindex{Wien@\textbf{Wien}, \emph{Verwaltungsgebiet}|pw} verlaſſen \strikeout{hat},
                  Mitte September, \label{K_L01335-1v}\edtext{keine Zeile an mich geſchrieben}{\lemma{\textnormal{\emph{keine … geschrieben}}}\Cendnote{\textnormal{Den
                  nächsten Brief von Goldmann\pwindex{Goldmann, Paul 31.\,1.\,1865 Breslau – 25.\,9.\,1935 Wien@\textsc{Goldmann, Paul} (31.\,1.\,1865 Breslau – 25.\,9.\,1935 Wien), \emph{Schriftsteller, Journalist}|pwk} erhielt er am
                     XXXX Auszeichnungsfehler: Dokument L03388 nicht gefunden.}}}\label{K_L01335-1}
               hat.\pend
           
\pstart
           – Das Stück\pwindex{Schnitzler, Arthur 15.\,5.\,1862 Wien – 21.\,10.\,1931 ebd.@\textsc{Schnitzler, Arthur} (15.\,5.\,1862 Wien – 21.\,10.\,1931 ebd.), \emph{Schriftsteller, Mediziner}!einsame Weg. Schauspiel in fünf Akten@\strich\emph{Der einsame Weg. Schauspiel in fünf Akten}|pwv} iſt{ }ſchon an Brahm\pwindex{Brahm, Otto 5.\,2.\,1856 Hamburg – 28.\,11.\,1912 Berlin@\textsc{Brahm, Otto} (5.\,2.\,1856 Hamburg – 28.\,11.\,1912 Berlin), \emph{Theaterleiter, Regisseur}|pw} abgegangen. Freitag gehn wir
                  {\pb}auf ein paar Tage auf den Semmering\oindex{Semmering@\textbf{Semmering}, \emph{Verwaltungsgebiet}|pw}. Mitte nächſter Woche möchte ich \label{K_L01335-2v}\edtext{vorleſen\pwindex{Schnitzler, Arthur 15.\,5.\,1862 Wien – 21.\,10.\,1931 ebd.@\textsc{Schnitzler, Arthur} (15.\,5.\,1862 Wien – 21.\,10.\,1931 ebd.), \emph{Schriftsteller, Mediziner}!einsame Weg. Schauspiel in fünf Akten@\strich\emph{Der einsame Weg. Schauspiel in fünf Akten}|pwv}}{\lemma{\textnormal{\emph{vorlesen}}}\Cendnote{\textnormal{Vgl. A. S.: \emph{Tagebuch}, 12. 11. 1903.
               }}}\label{K_L01335-2}. Sagen Sie mir bitte, ob Ihnen Dienſtag{ }Abend ½ 7 recht wäre. Fragen Sie auch gleich den Richard\pwindex{Beer-Hofmann, Richard 11.\,7.\,1866 Wien – 26.\,9.\,1945 New York City@\textsc{Beer-Hofmann, Richard} (11.\,7.\,1866 Wien – 26.\,9.\,1945 New York City), \emph{Schriftsteller}|pw}.\pend
           
\pstart
           Dieſer Tage iſt die \label{K_L01335-3v}\edtext{\textsc{Kakadu}\pwindex{Schnitzler, Arthur 15.\,5.\,1862 Wien – 21.\,10.\,1931 ebd.@\textsc{Schnitzler, Arthur} (15.\,5.\,1862 Wien – 21.\,10.\,1931 ebd.), \emph{Schriftsteller, Mediziner}!grüne Kakadu. Groteske in einem Akt@\strich\emph{Der grüne Kakadu. Groteske in einem Akt}|pw}\pwindex{Schnitzler, Arthur 15.\,5.\,1862 Wien – 21.\,10.\,1931 ebd.@\textsc{Schnitzler, Arthur} (15.\,5.\,1862 Wien – 21.\,10.\,1931 ebd.), \emph{Schriftsteller, Mediziner}!Au Perroquet Vert@\strich\emph{Au Perroquet Vert}|pw}\textsc{première} in Paris\oindex{Paris@\textbf{Paris}, \emph{Hauptstadt}|pw}}{\lemma{\textnormal{\emph{Kakadupremière in Paris}}}\Cendnote{\textnormal{am 7. 11. 1903}}}\label{K_L01335-3}; \textsc{Antoine}\pwindex{Antoine, André 31.\,1.\,1858 Limoges – 23.\,10.\,1943 Le Pouliguen@\textsc{Antoine, André} (31.\,1.\,1858 Limoges – 23.\,10.\,1943 Le Pouliguen), \emph{Theaterleiter, Schauspieler}|pw}{ }ſcheint{ }ſich nach einem Brief von ihm und von
               einigen andern, die Proben geſehen haben, viel {\pb}zu
               verſprechen.\pend
           
\pstart
           Grüßen Sie von uns beiden herzlich \textsc{Gerty}\pwindex{Hofmannsthal, Gertrude von 16.\,3.\,1880 Wien – 9.\,11.\,1959 Paddington@\textsc{Hofmannsthal, Gertrude von} (16.\,3.\,1880 Wien – 9.\,11.\,1959 Paddington)|pw} und Hofmannsthal den
                  Winzigen\pwindex{Hofmannsthal, Franz von 20.\,10.\,1903 Wien – 13.\,7.\,1929 ebd.@\textsc{Hofmannsthal, Franz von} (20.\,10.\,1903 Wien – 13.\,7.\,1929 ebd.)|pwv}. Sich{ }ſelber desgleichen.\pend
           
\pstart
           – Hat{ }ſich die Burg\orgindex{Burgtheater@Burgtheater|pw} um die ihrer Hoheit entkleidete Griechin\pwindex{Hofmannsthal, Hugo von 1.\,2.\,1874 Wien – 15.\,7.\,1929 Rodaun@\textsc{Hofmannsthal, Hugo von} (1.\,2.\,1874 Wien – 15.\,7.\,1929 Rodaun), \emph{Schriftsteller}!Elektra. Tragödie in einem Aufzug@\strich\emph{Elektra. Tragödie in einem Aufzug}|pwv}
                  beworben?{\dotstwo} Aus dem alten \textsc{Sophokles}\pwindex{Sophokles 497/496? v.\,u.\,Z. Kolonos – 406/405 v.\,u.\,Z. Athen@\textsc{Sophokles} (497/496? v.\,u.\,Z. Kolonos – 406/405 v.\,u.\,Z. Athen), \emph{Schriftsteller}!Elektra. Tragödie@\strich\emph{Elektra. Tragödie}|pwv}\pwindex{Sophokles 497/496? v.\,u.\,Z. Kolonos – 406/405 v.\,u.\,Z. Athen@\textsc{Sophokles} (497/496? v.\,u.\,Z. Kolonos – 406/405 v.\,u.\,Z. Athen), \emph{Schriftsteller}|pw} ein Zugstück zu machen! Echt {\pb}jüdiſch.\pend
           
\pstart
           Ihr{\\[\baselineskip]}\spacefill\mbox{A.}\pend
           \leftskip=0em{}\selectlanguage{ngerman}\endnumbering\briefempfaengerindex{Hofmannsthal, Hugo von@\textsc{Hofmannsthal, Hugo von}!zzzSchnitzler, Arthur@\emph{von Arthur Schnitzler}!1903-11-041@{4. 11. 1903}|)be}\mylabel{L01335h}  \newcommand{\dateiname}{L01335}\newcommand{\titel}{Arthur Schnitzler an Hugo von Hofmannsthal, 4. 11. 1903}\newcommand{\editorInnen}{Martin Anton Müller und Gerd-Hermann Susen}%% latex-leseansicht-abspann.tex
%% Abspann für die Leseansicht.
%% Der Schalter \ifkorrekturansicht ist bereits durch den Vorspann gesetzt.

%% latex-abspann.tex
%% Gemeinsamer Abspann für Korrekturansicht und Leseansicht.
%% Setzt den Schalter \ifkorrekturansicht voraus (gesetzt in den
%% einbindenden Dateien latex-korrekturansicht-abspann.tex bzw.
%% latex-leseansicht-abspann.tex).
%% ---------------------------------------------------------------

\normalsize

% Das esempio-Environment wird nur in der Leseansicht benötigt
\ifkorrekturansicht\else
\newenvironment{esempio}[3]%
{
    \vspace{1.5ex}
    \rlap{\underline{#1}}
    \par
    \setlength{\parindent}{0cm}
    \nopagebreak
    \leftskip=#2cm
    \rightskip=#3cm
}
{
    \par
}
\fi

\doendnotes{C}
\bigskip
\vfill

\clearpage

\footnotesize

\ifkorrekturansicht
  \lohead{\textsc{register}}
\fi

% theindex-Environment neu definieren ohne reledmac
\makeatletter
\renewenvironment{theindex}{%
  \ifkorrekturansicht
    \section*{\indexname}%
  \else
    \subsubsection*{Index der erwähnten Entitäten}%
  \fi
  \setlength{\parindent}{0pt}%
  \setlength{\parskip}{0pt plus 0.3pt}%
  \let\item\@idxitem
}{%
  \ifkorrekturansicht\clearpage\fi
}
\makeatother

\IfFileExists{\jobname-pw.ind}{\input{\jobname-pw.ind}}{}

% Quellenangabe nur in der Leseansicht
\ifkorrekturansicht\else
% Fallback-Definitionen, falls die .tex-Datei \titel etc. nicht gesetzt hat
\providecommand{\titel}{}
\providecommand{\editorInnen}{}
\providecommand{\dateiname}{\jobname}

\vspace{3cm}

\vfill

\footnotesize
\textsc{Quelle}: \titel. Herausgegeben von {\editorInnen}. In: \emph{Arthur Schnitzler: Briefwechsel mit Autorinnen und Autoren}.
 Digitale Edition, https://schnitzler-briefe.acdh.oeaw.ac.at/{\dateiname}.html (Stand \today)
\fi

\end{document}


