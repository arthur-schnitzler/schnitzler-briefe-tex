%% latex-leseansicht-vorspann.tex
%% Vorspann für die Leseansicht.
%% Lädt die gemeinsame Datei latex-vorspann.tex mit nicht gesetztem Schalter.

\newif\ifkorrekturansicht
\korrekturansichtfalse

\input{../tex-inputs/latex-vorspann}

\begin{center}
            \textcolor{red}{ENTWURF. ENTZIFFERUNG NOCH NICHT KORREKTURGELESEN}
                      \end{center}
            
               \section[Arthur Schnitzler an Hugo von Hofmannsthal, 4. 11. 1903]{ Arthur Schnitzler an Hugo von Hofmannsthal, 4. 11. 1903}\nopagebreak\mylabel{v}\rehead{ }\begin{ledgroupsized}[t]{13cm}\normalsize\beginnumbering\briefempfaengerindex{Hofmannsthal, Hugo von@\textsc{Hofmannsthal, Hugo von}!zzzSchnitzler, Arthur@\emph{von Arthur Schnitzler}!1903-11-041@{4. 11. 1903}|(be} \toendnotes[C]{\smallbreak\pagebreak[2]} \Standort{FDH, Hs-30885,105.}
\physDesc{Brief, 1 Blatt, 4 Seiten
\newline{}Handschrift: Bleistift, deutsche Kurrent}\buchAbdrucke{\weitereDrucke{Hugo von Hofmannsthal, Arthur Schnitzler: \emph{Briefwechsel}. Hg. Therese Nickl und Heinrich Schnitzler. Frankfurt am Main: \emph{S. Fischer} 1964, S. 176.} }\toendnotes[C]{\smallbreak}\pstart
           \raggedleft{}{\pb}\textsc{Spöttelgasse 7}\oindex{Edmund-Weiss-Gasse@\textbf{Edmund-Weiß-Gasse}|pw}.
                     4. 11. 903\pend
           \pstart{}lieber Hugo, \pend\pstart
           über Elektra\pwindex{Hofmannsthal, Hugo von 01.02.1874 – 15.07.1929@\textsc{Hofmannsthal, Hugo von} (01.02.1874 – 15.07.1929), \emph{Schriftsteller}!Elektra. Tragoedie in einem Aufzug1903@\strich\emph{Elektra. Tragödie in einem Aufzug} {[}1903{]}|pw} hab ich mich ſehr gefreut, und das Goldma{\geminationn}\pwindex{Goldmann, Paul 31.01.1865 – 25.09.1935@\textsc{Goldmann, Paul} (31.01.1865 – 25.09.1935), \emph{Schriftsteller, Journalist}|pw}sche Telegramm\pwindex{Aus Berlin [Elektra-Premiere]31. 10. 1903@\emph{Aus Berlin [Elektra-Premiere]} {[}31. 10. 1903{]}|pwv} gehört zu dem Übrigen. Denken
               Sie, daſs er \strikeout{mir}, ſeit er Wien\oindex{Wien@\textbf{Wien}|pw} verlaſſen \strikeout{hat}, Mitte
                  September, keine Zeile an mich geſchrieben hat.\pend
           \pstart
           – Das Stück\pwindex{Schnitzler, Arthur 15.05.1862 – 21.10.1931@\textsc{Schnitzler, Arthur} (15.05.1862 – 21.10.1931), \emph{Schriftsteller, Mediziner}!einsame Weg. Schauspiel in fuenf Akten1904@\strich\emph{Der einsame Weg. Schauspiel in fünf Akten} {[}1904{]}|pwv} iſt ſchon an Brahm\pwindex{Brahm, Otto 05.02.1856 – 28.11.1912@\textsc{Brahm, Otto} (05.02.1856 – 28.11.1912), \emph{Theaterleiter, Regisseur}|pw} abgegangen. Freitag gehn wir
                  {\pb}auf ein paar Tage auf den Semmering\oindex{Semmering@\textbf{Semmering}|pw}. Mitte nächſter Woche möchte ich \label{K_L01335_1v}\edtext{vorleſen\pwindex{Schnitzler, Arthur 15.05.1862 – 21.10.1931@\textsc{Schnitzler, Arthur} (15.05.1862 – 21.10.1931), \emph{Schriftsteller, Mediziner}!einsame Weg. Schauspiel in fuenf Akten1904@\strich\emph{Der einsame Weg. Schauspiel in fünf Akten} {[}1904{]}|pwv}}{\lemma{\textnormal{\emph{vorleſen}}}\Cendnote{\textnormal{vgl. A. S.: \emph{Tagebuch}, 12. 11. 1903}}}\label{K_L01335_1h}. Sagen Sie mir bitte, ob Ihnen
                  Dienſtag{ }Abend ½ 7 recht wäre. Fragen Sie auch gleich den Richard\pwindex{Beer-Hofmann, Richard 11.07.1866 – 26.09.1945@\textsc{Beer-Hofmann, Richard} (11.07.1866 – 26.09.1945), \emph{Schriftsteller}|pw}.\pend
           \pstart
           Dieſer Tage iſt die \label{K_L01335_2v}\edtext{\textsc{Kakadu}\pwindex{Schnitzler, Arthur 15.05.1862 – 21.10.1931@\textsc{Schnitzler, Arthur} (15.05.1862 – 21.10.1931), \emph{Schriftsteller, Mediziner}!gruene Kakadu. Groteske in einem Akt1.3.1899 – 1.3.1899@\strich\emph{Der grüne Kakadu. Groteske in einem Akt} {[}1.3.1899 – 1.3.1899{]}|pw}\textsc{première} in Paris\oindex{Paris@\textbf{Paris}|pw}}{\lemma{\textnormal{\emph{Kakadupremière in Paris}}}\Cendnote{\textnormal{am
                     7. 11. 1903}}}\label{K_L01335_2h}; \textsc{Antoine}\pwindex{Antoine, Andre 31.01.1858 – 23.10.1943@\textsc{Antoine, André} (31.01.1858 – 23.10.1943), \emph{Theaterleiter, Schauspieler}|pw}{ }ſcheint ſich nach einem Brief von ihm und von
               einigen andern, die Proben geſehen haben, viel {\pb}zu
               verſprechen.\pend
           \pstart
           Grüßen Sie von uns beiden herzlich \textsc{Gerty}\pwindex{Hofmannsthal, Gertrude von 16.03.1880 – 09.11.1959@\textsc{Hofmannsthal, Gertrude von} (16.03.1880 – 09.11.1959)|pw} und Hofmannsthal den
                  Winzigen\pwindex{Hofmannsthal, Franz von 20.10.1903 – 13.07.1929@\textsc{Hofmannsthal, Franz von} (20.10.1903 – 13.07.1929)|pwv}. Sich ſelber desgleichen.\pend
           \pstart
           – Hat ſich die Burg\orgindex{Burgtheater@Burgtheater|pw} um die ihrer Hoheit entkleidete Griechin\pwindex{Hofmannsthal, Hugo von 01.02.1874 – 15.07.1929@\textsc{Hofmannsthal, Hugo von} (01.02.1874 – 15.07.1929), \emph{Schriftsteller}!Elektra. Tragoedie in einem Aufzug1903@\strich\emph{Elektra. Tragödie in einem Aufzug} {[}1903{]}|pwv} beworben?{\dotstwo} Aus dem alten \textsc{Sophokles}\pwindex{Sophokles 497/496? v. u. Z. – 406/405 v. u. Z.@\textsc{Sophokles} (497/496? v. u. Z. – 406/405 v. u. Z.), \emph{Schriftsteller}!Elektra. Tragoedie413 v. Chr.@\strich\emph{Elektra. Tragödie} {[}413 v. Chr.{]}|pwv}\pwindex{Sophokles 497/496? v. u. Z. – 406/405 v. u. Z.@\textsc{Sophokles} (497/496? v. u. Z. – 406/405 v. u. Z.), \emph{Schriftsteller}|pw} ein Zugstück zu machen! Echt {\pb}jüdiſch.\pend
           \pstart
           Ihr{\\[\baselineskip]}\spacefill\mbox{A.}\pend
           \leftskip=0em{}\endnumbering\briefempfaengerindex{Hofmannsthal, Hugo von@\textsc{Hofmannsthal, Hugo von}!zzzSchnitzler, Arthur@\emph{von Arthur Schnitzler}!1903-11-041@{4. 11. 1903}|)be}\mylabel{h}\end{ledgroupsized}  \newcommand{\dateiname}{L01335}\newcommand{\titel}{Arthur Schnitzler an Hugo von Hofmannsthal, 4. 11. 1903}\newcommand{\editorInnen}{Martin Anton Müller und Gerd-Hermann Susen}%% latex-leseansicht-abspann.tex
%% Abspann für die Leseansicht.
%% Der Schalter \ifkorrekturansicht ist bereits durch den Vorspann gesetzt.

%% latex-abspann.tex
%% Gemeinsamer Abspann für Korrekturansicht und Leseansicht.
%% Setzt den Schalter \ifkorrekturansicht voraus (gesetzt in den
%% einbindenden Dateien latex-korrekturansicht-abspann.tex bzw.
%% latex-leseansicht-abspann.tex).
%% ---------------------------------------------------------------

\normalsize

% Das esempio-Environment wird nur in der Leseansicht benötigt
\ifkorrekturansicht\else
\newenvironment{esempio}[3]%
{
    \vspace{1.5ex}
    \rlap{\underline{#1}}
    \par
    \setlength{\parindent}{0cm}
    \nopagebreak
    \leftskip=#2cm
    \rightskip=#3cm
}
{
    \par
}
\fi

\doendnotes{C}
\bigskip
\vfill

\clearpage

\footnotesize

\ifkorrekturansicht
  \lohead{\textsc{register}}
\fi

% theindex-Environment neu definieren ohne reledmac
\makeatletter
\renewenvironment{theindex}{%
  \ifkorrekturansicht
    \section*{\indexname}%
  \else
    \subsubsection*{Index der erwähnten Entitäten}%
  \fi
  \setlength{\parindent}{0pt}%
  \setlength{\parskip}{0pt plus 0.3pt}%
  \let\item\@idxitem
}{%
  \ifkorrekturansicht\clearpage\fi
}
\makeatother

\IfFileExists{\jobname-pw.ind}{\input{\jobname-pw.ind}}{}

% Quellenangabe nur in der Leseansicht
\ifkorrekturansicht\else
% Fallback-Definitionen, falls die .tex-Datei \titel etc. nicht gesetzt hat
\providecommand{\titel}{}
\providecommand{\editorInnen}{}
\providecommand{\dateiname}{\jobname}

\vspace{3cm}

\vfill

\footnotesize
\textsc{Quelle}: \titel. Herausgegeben von {\editorInnen}. In: \emph{Arthur Schnitzler: Briefwechsel mit Autorinnen und Autoren}.
 Digitale Edition, https://schnitzler-briefe.acdh.oeaw.ac.at/{\dateiname}.html (Stand \today)
\fi

\end{document}


      