%% latex-leseansicht-vorspann.tex
%% Vorspann für die Leseansicht.
%% Lädt die gemeinsame Datei latex-vorspann.tex mit nicht gesetztem Schalter.

\newif\ifkorrekturansicht
\korrekturansichtfalse

\input{../tex-inputs/latex-vorspann}


\section[Theodor Herzl an Arthur Schnitzler, 19. 5. 1895]{L03862 Theodor Herzl an Arthur Schnitzler, 19. 5. 1895}
\nopagebreak\mylabel{L03862v}
\rehead{ }\normalsize\beginnumbering\briefempfaengerindex{Schnitzler, Arthur@\textsc{Schnitzler, Arthur}!zzzHerzl, Theodor@\emph{von Theodor Herzl}!1895-05-192@{19. 5. 1895}|(be}
\toendnotes[C]{\smallbreak\pagebreak[2]}
\correspDesc{Versand  durch Theodor Herzl am 19. 5. 1895 in Paris
\newline{}Erhalt  durch Arthur Schnitzler im Zeitraum [20. 5. 1895
                  – 24. 5. 1895?] in Wien}\toendnotes[C]{\smallbreak}
\Standort{CUL, Schnitzler, B 39.}
\physDesc{Brief, 1 Blatt, 3 Seiten, 1118 Zeichen
\newline{}Handschrift: schwarze Tinte, lateinische Kurrent
\newline{}Schnitzler: mit Bleistift datiert: »19/5 85« 
\newline{}Ordnung: mit Bleistift von unbekannter Hand nummeriert: »41« }
\buchAbdrucke{\weitereDrucke{Theodor Herzl: \emph{Briefe Anfang Mai 1895 – Anfang Dezember 1898}. Bearbeitet von Barbara Schäfer in Zusammenarbeit mit Sofia Gelmann, Chaya Harel, Ines Rubin und Daisy Ticho. Berlin, Frankfurt am Main, Wien: \emph{Propyläen} 1990, S. 41–42 (Briefe und Tagebücher. Herausgegeben von Alex Bein, Hermann Greive, Moshe Schaerf, Julius H. Schoeps und Johannes Wachten, 4).} }\toendnotes[C]{\smallbreak}
\pstart{}{\pb}Theurer Freund!\pend\vspace{0.5em}
\pstart
           Teweles\pwindex{Teweles, Heinrich 13.\,11.\,1856 Prag – 9.\,8.\,1927 Prein an der Rax@\textsc{Teweles, Heinrich} (13.\,11.\,1856 Prag – 9.\,8.\,1927 Prein an der Rax), \emph{Schriftsteller, Journalist, Theaterleiter}|pw} hat sich, wie nicht zu bezweifeln war,
               zur Discretion verpflichtet. Der Mensch\pwindex{Teweles, Heinrich 13.\,11.\,1856 Prag – 9.\,8.\,1927 Prein an der Rax@\textsc{Teweles, Heinrich} (13.\,11.\,1856 Prag – 9.\,8.\,1927 Prein an der Rax), \emph{Schriftsteller, Journalist, Theaterleiter}|pwv} hat mir dazu \label{K_L03862-1v}\edtext{einen
                  Brief}{\lemma{\textnormal{\emph{einen
                  Brief}}}\Cendnote{\textnormal{Heinrich Teweles\pwindex{Teweles, Heinrich 13.\,11.\,1856 Prag – 9.\,8.\,1927 Prein an der Rax@\textsc{Teweles, Heinrich} (13.\,11.\,1856 Prag – 9.\,8.\,1927 Prein an der Rax), \emph{Schriftsteller, Journalist, Theaterleiter}|pwk} an Theodor Herzl\pwindex{Herzl, Theodor 2.\,5.\,1860 Budapest – 3.\,7.\,1904 Edlach@\textsc{Herzl, Theodor} (2.\,5.\,1860 Budapest – 3.\,7.\,1904 Edlach), \emph{Schriftsteller, Journalist}|pwk}, 16. 5. 1895,
                        \emph{Central Zionist Archives Jerusalem},
                  H1:1985-2.}}}\label{K_L03862-1} geschrieben, der mich rührte, so viel gute Freundschaft
               spricht daraus. Merkwürdig, ich habe Freunde! Ich übertreibe nicht.\pend
           
\pstart
           Ich bitte Sie also, sich noch in folgender Weise für mich zu ruiniren: Schicken Sie
               den \label{K_L03862-2v}\edtext{beiliegenden Brief}{\lemma{\textnormal{\emph{beiliegenden Brief}}}\Cendnote{\textnormal{Herzl\pwindex{Teweles, Heinrich 13.\,11.\,1856 Prag – 9.\,8.\,1927 Prein an der Rax@\textsc{Teweles, Heinrich} (13.\,11.\,1856 Prag – 9.\,8.\,1927 Prein an der Rax), \emph{Schriftsteller, Journalist, Theaterleiter}|pwk} schreibt an Teweles\pwindex{Teweles, Heinrich 13.\,11.\,1856 Prag – 9.\,8.\,1927 Prein an der Rax@\textsc{Teweles, Heinrich} (13.\,11.\,1856 Prag – 9.\,8.\,1927 Prein an der Rax), \emph{Schriftsteller, Journalist, Theaterleiter}|pwk} am 19. 5. 1895: »Ich lasse diesen Brief von Wien\oindex{Wien@\textbf{Wien}, \emph{Verwaltungsgebiet}|pw} aus an Sie schicken, weil ich ihn
                     recommandiren will u. hier dabei meinen Namen aufs Couvert setzen müsste. So
                     käme man in Prag\oindex{Prag@\textbf{Prag}, \emph{Land}|pw} gleich auf die Spur. Ich
                     bitte Sie mir Ihre Privatadresse anzugeben [{\dots}],
                     damit ich weiterhin direkt mit Ihnen correspondiren könne«, \emph{Theodor Herzl an Heinrich Teweles, 19. 5. 1895}. In:
                        \emph{Briefe Anfang Mai 1895 – 1898}, S. 38–41, hier
                     S. 40. Der lange und persönlich gehaltene Brief Herzls\pwindex{Teweles, Heinrich 13.\,11.\,1856 Prag – 9.\,8.\,1927 Prein an der Rax@\textsc{Teweles, Heinrich} (13.\,11.\,1856 Prag – 9.\,8.\,1927 Prein an der Rax), \emph{Schriftsteller, Journalist, Theaterleiter}|pwk} über Freundschaft, seine berufliche, künstlerische und
                  persönliche Entwicklung, die Entstehung des Schauspiels \emph{Das neue Ghetto}\pwindex{Herzl, Theodor 2.\,5.\,1860 Budapest – 3.\,7.\,1904 Edlach@\textsc{Herzl, Theodor} (2.\,5.\,1860 Budapest – 3.\,7.\,1904 Edlach), \emph{Schriftsteller, Journalist}!neue Ghetto. Schauspiel in vier Acten@\strich\emph{Das neue Ghetto. Schauspiel in vier Acten}|pwk} und seine neuen Perspektiven auf die Stellung von Juden in der Gesellschaft lag dem vorliegenden Brief bei,
                  befand sich aber wahrscheinlich in einem Kuvert und war für Schnitzler nicht lesbar.}}}\label{K_L03862-2}{ }\label{K_L03862-3v}\edtext{recommandirt}{\lemma{\textnormal{\emph{recommandirt}}}\Cendnote{\textnormal{per Einschreiben}}}\label{K_L03862-3} an Teweles\pwindex{Teweles, Heinrich 13.\,11.\,1856 Prag – 9.\,8.\,1927 Prein an der Rax@\textsc{Teweles, Heinrich} (13.\,11.\,1856 Prag – 9.\,8.\,1927 Prein an der Rax), \emph{Schriftsteller, Journalist, Theaterleiter}|pw}. (\label{K_L03862-4v}\edtext{Er schreibt mir
               nämlich dass man meine Handschrift erkannt hat, auf dem Couvert meines ersten
                  Briefes}{\lemma{\textnormal{\emph{Er … Briefes}}}\Cendnote{\textnormal{Der Brief von Teweles\pwindex{Teweles, Heinrich 13.\,11.\,1856 Prag – 9.\,8.\,1927 Prein an der Rax@\textsc{Teweles, Heinrich} (13.\,11.\,1856 Prag – 9.\,8.\,1927 Prein an der Rax), \emph{Schriftsteller, Journalist, Theaterleiter}|pwk} beginnt folgendermaßen:
                     »Lieber Freund! An der Aufſchrift erkannte ich Ihre Schrift und freute
                     mich{ }ſchon. Sie haben mich alſo in Paris\oindex{Paris@\textbf{Paris}, \emph{Hauptstadt}|pw}
                     nicht vergeſſen und das iſt doch wol{ }ſchwerer, als daß ich Sie hier in Prag\oindex{Prag@\textbf{Prag}, \emph{Land}|pw} vergeſſe.« }}}\label{K_L03862-4})\pend
           
\pstart
           Lassen Sie ferner durch Schick\pwindex{Schik, Friedrich *~6.\,9.\,1857 Wien@\textsc{Schik, Friedrich} (*~6.\,9.\,1857 Wien), \emph{Notar, Journalist, Dramaturg}|pw} das Manuscript
               des Stückes\pwindex{Herzl, Theodor 2.\,5.\,1860 Budapest – 3.\,7.\,1904 Edlach@\textsc{Herzl, Theodor} (2.\,5.\,1860 Budapest – 3.\,7.\,1904 Edlach), \emph{Schriftsteller, Journalist}!neue Ghetto. Schauspiel in vier Acten@\strich\emph{Das neue Ghetto. Schauspiel in vier Acten}|pwv} unter der Adresse:
                  {\pb}Löbliche Direction des Königl. deutschen Landestheaters\orgindex{Ständetheater@Ständetheater|pw}{ }\pend
           
\pstart
           \centering{}\uline{Prag}\oindex{Prag@\textbf{Prag}, \emph{Land}|pw}\pend
           
\pstart
           absenden.\pend
           
\pstart
           Der an die Berliner\oindex{Berlin@\textbf{Berlin}, \emph{Hauptstadt}|pw}{ }Directoren\pwindex{Blumenthal, Oskar 13.\,3.\,1852 Berlin – 24.\,4.\,1917 ebd.@\textsc{Blumenthal, Oskar} (13.\,3.\,1852 Berlin – 24.\,4.\,1917 ebd.), \emph{Schriftsteller, Journalist, Theaterleiter}|pwv}\pwindex{Brahm, Otto 5.\,2.\,1856 Hamburg – 28.\,11.\,1912 Berlin@\textsc{Brahm, Otto} (5.\,2.\,1856 Hamburg – 28.\,11.\,1912 Berlin), \emph{Theaterleiter, Regisseur}|pwv}
               gerichtete \label{K_L03862-5v}\edtext{Vorwortbrief}{\lemma{\textnormal{\emph{Vorwortbrief}}}\Cendnote{\textnormal{Siehe die Beilage von Theodor Herzl\pwindex{Teweles, Heinrich 13.\,11.\,1856 Prag – 9.\,8.\,1927 Prein an der Rax@\textsc{Teweles, Heinrich} (13.\,11.\,1856 Prag – 9.\,8.\,1927 Prein an der Rax), \emph{Schriftsteller, Journalist, Theaterleiter}|pwk} an Arthur Schnitzler, XXXX Auszeichnungsfehler: Dokument L03843 nicht gefunden.}}}\label{K_L03862-5} ist natürlich
               herauszuschneiden.\pend
           
\pstart
           Folgenden Begleitbrief soll Schick\pwindex{Schik, Friedrich *~6.\,9.\,1857 Wien@\textsc{Schik, Friedrich} (*~6.\,9.\,1857 Wien), \emph{Notar, Journalist, Dramaturg}|pw}
               schreiben:\pend
           
\pstart
           Geehrte Direction!\pend
           
\pstart
           Beifolgend mein 4 actiges Schauspiel »das
               Ghetto\pwindex{Herzl, Theodor 2.\,5.\,1860 Budapest – 3.\,7.\,1904 Edlach@\textsc{Herzl, Theodor} (2.\,5.\,1860 Budapest – 3.\,7.\,1904 Edlach), \emph{Schriftsteller, Journalist}!neue Ghetto. Schauspiel in vier Acten@\strich\emph{Das neue Ghetto. Schauspiel in vier Acten}|pw}«.\pend
           
\pstart
           Ich stelle nur folgende Bedingungen: baldige unveränderte Aufführung noch in dieser
               Spielzeit. An Tantièmen wollen Sie den üblichen {\pb}Satz entrichten. Alle Abmachungen trifft
               mein Vertreter Herr Fr. Schick\pwindex{Schik, Friedrich *~6.\,9.\,1857 Wien@\textsc{Schik, Friedrich} (*~6.\,9.\,1857 Wien), \emph{Notar, Journalist, Dramaturg}|pw} in Wien\oindex{Wien@\textbf{Wien}, \emph{Verwaltungsgebiet}|pw}.\pend
           
\pstart
           \centering{}Hochachtungsvoll\pend
           
\pstart
           \raggedleft{}Albert Schnabel\pend
           
\pstart
           Für heute in Eile nur herzliche Grüsse{\\[\baselineskip]} von Ihrem getreuen{\\[\baselineskip]}\spacefill\mbox{Th H.}\pend
           \leftskip=0em{}
\pstart
           19 mai 95\pend
           \selectlanguage{ngerman}\endnumbering\briefempfaengerindex{Schnitzler, Arthur@\textsc{Schnitzler, Arthur}!zzzHerzl, Theodor@\emph{von Theodor Herzl}!1895-05-192@{19. 5. 1895}|)be}\mylabel{L03862h}
\begin{anhang}
\end{anhang}\newcommand{\dateiname}{L03862}\newcommand{\titel}{Theodor Herzl an Arthur Schnitzler, 19. 5. 1895}\newcommand{\editorInnen}{Selma Jahnke und Martin Anton Müller}%% latex-leseansicht-abspann.tex
%% Abspann für die Leseansicht.
%% Der Schalter \ifkorrekturansicht ist bereits durch den Vorspann gesetzt.

%% latex-abspann.tex
%% Gemeinsamer Abspann für Korrekturansicht und Leseansicht.
%% Setzt den Schalter \ifkorrekturansicht voraus (gesetzt in den
%% einbindenden Dateien latex-korrekturansicht-abspann.tex bzw.
%% latex-leseansicht-abspann.tex).
%% ---------------------------------------------------------------

\normalsize

% Das esempio-Environment wird nur in der Leseansicht benötigt
\ifkorrekturansicht\else
\newenvironment{esempio}[3]%
{
    \vspace{1.5ex}
    \rlap{\underline{#1}}
    \par
    \setlength{\parindent}{0cm}
    \nopagebreak
    \leftskip=#2cm
    \rightskip=#3cm
}
{
    \par
}
\fi

\doendnotes{C}
\bigskip
\vfill

\clearpage

\footnotesize

\ifkorrekturansicht
  \lohead{\textsc{register}}
\fi

% theindex-Environment neu definieren ohne reledmac
\makeatletter
\renewenvironment{theindex}{%
  \ifkorrekturansicht
    \section*{\indexname}%
  \else
    \subsubsection*{Index der erwähnten Entitäten}%
  \fi
  \setlength{\parindent}{0pt}%
  \setlength{\parskip}{0pt plus 0.3pt}%
  \let\item\@idxitem
}{%
  \ifkorrekturansicht\clearpage\fi
}
\makeatother

\IfFileExists{\jobname-pw.ind}{\input{\jobname-pw.ind}}{}

% Quellenangabe nur in der Leseansicht
\ifkorrekturansicht\else
% Fallback-Definitionen, falls die .tex-Datei \titel etc. nicht gesetzt hat
\providecommand{\titel}{}
\providecommand{\editorInnen}{}
\providecommand{\dateiname}{\jobname}

\vspace{3cm}

\vfill

\footnotesize
\textsc{Quelle}: \titel. Herausgegeben von {\editorInnen}. In: \emph{Arthur Schnitzler: Briefwechsel mit Autorinnen und Autoren}.
 Digitale Edition, https://schnitzler-briefe.acdh.oeaw.ac.at/{\dateiname}.html (Stand \today)
\fi

\end{document}


