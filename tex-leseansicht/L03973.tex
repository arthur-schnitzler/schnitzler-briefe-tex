%% latex-leseansicht-vorspann.tex
%% Vorspann für die Leseansicht.
%% Lädt die gemeinsame Datei latex-vorspann.tex mit nicht gesetztem Schalter.

\newif\ifkorrekturansicht
\korrekturansichtfalse

\input{../tex-inputs/latex-vorspann}


\section[Arthur Schnitzler an Berta Zuckerkandl, 21. 11. 1927]{L03973 Arthur Schnitzler an Berta Zuckerkandl, 21. 11. 1927}
\nopagebreak\mylabel{L03973v}
\rehead{ }\normalsize\beginnumbering\briefempfaengerindex{Zuckerkandl, Berta@\textsc{Zuckerkandl, Berta}!zzzSchnitzler, Arthur@\emph{von Arthur Schnitzler}!1927-11-211@{21. 11. 1927}|(be}
\toendnotes[C]{\smallbreak\pagebreak[2]}
\correspDesc{Versand  durch Arthur Schnitzler am 21. 11. 1927 in Wien
\newline{}Erhalt  durch Berta Zuckerkandl im Zeitraum [22. 11. 1927 – 26. 11. 1927?] in Paris}\toendnotes[C]{\smallbreak}
\Standort{DLA, HS.1985.1.2282.}
\physDesc{Brief, Durchschlag, 1 Blatt, 2 Seiten, 2184 Zeichen
\newline{}Schreibmaschine
\newline{}Handschrift: roter Buntstift, lateinische Kurrent (\noindent{}beschriftet: »\uline{Zuckerkandl}« und »\uline{Paris}«, zweiundzwanzig Unterstreichungen und eine
                                 Anstreichung)}\toendnotes[C]{\smallbreak}
\pstart
           \raggedleft{}{\pb}21. 11. 1927.\pend
           
\pstart{}Liebe und verehrte Freundin.\pend\vspace{0.5em}
\pstart
           Ich habe indess unter meinen Briefen nachgesehen; was ich Ihnen \label{K_L03973-1v}\edtext{neulich}{\lemma{\textnormal{\emph{neulich}}}\Cendnote{\textnormal{Siehe XXXX Auszeichnungsfehler: Dokument L03972 nicht gefunden.}}}\label{K_L03973-1} geschrieben
               stimmt beinahe vollkommen. Aus der Abschrift eines Briefes \label{K_L03973-2v}\edtext{an Herrn Hela\pwindex{Hella, Alzir 30.\,12.\,1881 Vieux Condé – 14.\,7.\,1953 Paris@\textsc{Hella, Alzir} (30.\,12.\,1881 Vieux Condé – 14.\,7.\,1953 Paris), \emph{Übersetzer}|pw}}{\lemma{\textnormal{\emph{an Herrn Hela}}}\Cendnote{\textnormal{»›Liebelei\pwindex{Schnitzler, Arthur 15. 5. 1862 Wien – 21. 10. 1931 ebd.@\textsc{Schnitzler, Arthur} (15. 5. 1862 Wien – 21. 10. 1931 ebd.), \emph{Schriftsteller, Mediziner}!Liebelei. Schauspiel in drei Akten@\strich\emph{Liebelei. Schauspiel in drei Akten}|pw}‹ ist wohl schon seinerzeit von Thorel\pwindex{Thorel, Jean 11.\,9.\,1859 Éragny – 20.\,8.\,1916 Enghien-les-Bains@\textsc{Thorel, Jean} (11.\,9.\,1859 Éragny – 20.\,8.\,1916 Enghien-les-Bains), \emph{Übersetzer, Dramatiker}|pw} schlecht übersetzt\pwindex{Schnitzler, Arthur 15. 5. 1862 Wien – 21. 10. 1931 ebd.@\textsc{Schnitzler, Arthur} (15. 5. 1862 Wien – 21. 10. 1931 ebd.), \emph{Schriftsteller, Mediziner}!Amourette. Pièce en trois actes. Adaptée de Arthur Schnitzler@\strich\emph{Amourette. Pièce en trois actes. Adaptée de Arthur Schnitzler}|pwv} worden, ich habe ein Exemplar\pwindex{Schnitzler, Arthur 15. 5. 1862 Wien – 21. 10. 1931 ebd.@\textsc{Schnitzler, Arthur} (15. 5. 1862 Wien – 21. 10. 1931 ebd.), \emph{Schriftsteller, Mediziner}!Amourette. Pièce en trois actes. Adaptée de Arthur Schnitzler@\strich\emph{Amourette. Pièce en trois actes. Adaptée de Arthur Schnitzler}|pwv} mit einigen kleinen
                     Verbesserungen, die keineswegs ausreichen würden, vor einigen Monaten nach Paris\oindex{Paris@\textbf{Paris}, \emph{Hauptstadt}|pw} an Herrn Lenormand\pwindex{Lenormand, Henri-René 3.\,5.\,1882 Paris – 16.\,2.\,1951 ebd.@\textsc{Lenormand, Henri-René} (3.\,5.\,1882 Paris – 16.\,2.\,1951 ebd.), \emph{Schriftsteller}|pw} geschickt.«, siehe
                     Schnitzler an Alzir Hella\pwindex{Hella, Alzir 30.\,12.\,1881 Vieux Condé – 14.\,7.\,1953 Paris@\textsc{Hella, Alzir} (30.\,12.\,1881 Vieux Condé – 14.\,7.\,1953 Paris), \emph{Übersetzer}|pwk}, 29. 3. 1926, \emph{Deutsches Literaturarchiv Marbach},
                  HS.1985.1.969.}}}\label{K_L03973-2} entnehme ich, dass ich »Liebelei\pwindex{Schnitzler, Arthur 15. 5. 1862 Wien – 21. 10. 1931 ebd.@\textsc{Schnitzler, Arthur} (15. 5. 1862 Wien – 21. 10. 1931 ebd.), \emph{Schriftsteller, Mediziner}!Liebelei. Schauspiel in drei Akten@\strich\emph{Liebelei. Schauspiel in drei Akten}|pw}«, die Uebersetzung\pwindex{Schnitzler, Arthur 15. 5. 1862 Wien – 21. 10. 1931 ebd.@\textsc{Schnitzler, Arthur} (15. 5. 1862 Wien – 21. 10. 1931 ebd.), \emph{Schriftsteller, Mediziner}!Amourette. Pièce en trois actes. Adaptée de Arthur Schnitzler@\strich\emph{Amourette. Pièce en trois actes. Adaptée de Arthur Schnitzler}|pwv} von Thorel\pwindex{Thorel, Jean 11.\,9.\,1859 Éragny – 20.\,8.\,1916 Enghien-les-Bains@\textsc{Thorel, Jean} (11.\,9.\,1859 Éragny – 20.\,8.\,1916 Enghien-les-Bains), \emph{Übersetzer, Dramatiker}|pw} mit den Korrekturen \label{K_L03973-3v}\edtext{an
                  Lenormand\pwindex{Lenormand, Henri-René 3.\,5.\,1882 Paris – 16.\,2.\,1951 ebd.@\textsc{Lenormand, Henri-René} (3.\,5.\,1882 Paris – 16.\,2.\,1951 ebd.), \emph{Schriftsteller}|pw}}{\lemma{\textnormal{\emph{an
                  Lenormand}}}\Cendnote{\textnormal{Die Übersetzung\pwindex{Schnitzler, Arthur 15. 5. 1862 Wien – 21. 10. 1931 ebd.@\textsc{Schnitzler, Arthur} (15. 5. 1862 Wien – 21. 10. 1931 ebd.), \emph{Schriftsteller, Mediziner}!Amourette. Pièce en trois actes. Adaptée de Arthur Schnitzler@\strich\emph{Amourette. Pièce en trois actes. Adaptée de Arthur Schnitzler}|pwkv} wurde übersandt mit dem Brief von Arthur Schnitzler an Henri-René Lenormand\pwindex{Lenormand, Henri-René 3.\,5.\,1882 Paris – 16.\,2.\,1951 ebd.@\textsc{Lenormand, Henri-René} (3.\,5.\,1882 Paris – 16.\,2.\,1951 ebd.), \emph{Schriftsteller}|pwk}, 30. 7. 1925, \emph{Deutsches Literaturarchiv Marbach},
                  HS.1985.1.1280.}}}\label{K_L03973-3} geschickt habe. So weit ich mich erinnere habe ich
               von Lenormand\pwindex{Lenormand, Henri-René 3.\,5.\,1882 Paris – 16.\,2.\,1951 ebd.@\textsc{Lenormand, Henri-René} (3.\,5.\,1882 Paris – 16.\,2.\,1951 ebd.), \emph{Schriftsteller}|pw} nie wieder etwas darüber gehört.
               (Uebrigens auch nicht von Hela\pwindex{Hella, Alzir 30.\,12.\,1881 Vieux Condé – 14.\,7.\,1953 Paris@\textsc{Hella, Alzir} (30.\,12.\,1881 Vieux Condé – 14.\,7.\,1953 Paris), \emph{Übersetzer}|pw}.)\pend
           
\pstart
           »Das weite Land\pwindex{Schnitzler, Arthur 15. 5. 1862 Wien – 21. 10. 1931 ebd.@\textsc{Schnitzler, Arthur} (15. 5. 1862 Wien – 21. 10. 1931 ebd.), \emph{Schriftsteller, Mediziner}!Le Pays de l’âme. Drame en 5 actes@\strich\emph{Le Pays de l’âme. Drame en 5 actes}|pwv}\pwindex{Schnitzler, Arthur 15. 5. 1862 Wien – 21. 10. 1931 ebd.@\textsc{Schnitzler, Arthur} (15. 5. 1862 Wien – 21. 10. 1931 ebd.), \emph{Schriftsteller, Mediziner}!weite Land. Tragikomödie in fünf Akten@\strich\emph{Das weite Land. Tragikomödie in fünf Akten}|pw}« ist übersetzt von Emma Cabir\pwindex{Cabire, Emma @\textsc{Cabire, Emma}, \emph{Übersetzerin, Redakteurin, Literaturagentin}|pw}, 27, Rue Lemercier\oindex{27, Rue Lemercier@\textbf{27, Rue Lemercier}, \emph{Wohngebäude}|pw}, Manuscript\pwindex{Schnitzler, Arthur 15. 5. 1862 Wien – 21. 10. 1931 ebd.@\textsc{Schnitzler, Arthur} (15. 5. 1862 Wien – 21. 10. 1931 ebd.), \emph{Schriftsteller, Mediziner}!Le Pays de l’âme. Drame en 5 actes@\strich\emph{Le Pays de l’âme. Drame en 5 actes}|pwv} befindet sich bei Gemier\pwindex{Gémier, Firmin 21.\,2.\,1865 Aubervilliers – 26.\,11.\,1933 Paris@\textsc{Gémier, Firmin} (21.\,2.\,1865 Aubervilliers – 26.\,11.\,1933 Paris), \emph{Theaterleiter, Schauspieler, Drehbuchautor}|pw}.\pend
           
\pstart
           »Der einsame Weg\pwindex{Schnitzler, Arthur 15. 5. 1862 Wien – 21. 10. 1931 ebd.@\textsc{Schnitzler, Arthur} (15. 5. 1862 Wien – 21. 10. 1931 ebd.), \emph{Schriftsteller, Mediziner}!?? [französische Übersetzung von Der einsame Weg]@\strich\emph{?? [französische Übersetzung von Der einsame Weg]}|pwv}\pwindex{Schnitzler, Arthur 15. 5. 1862 Wien – 21. 10. 1931 ebd.@\textsc{Schnitzler, Arthur} (15. 5. 1862 Wien – 21. 10. 1931 ebd.), \emph{Schriftsteller, Mediziner}!einsame Weg. Schauspiel in fünf Akten@\strich\emph{Der einsame Weg. Schauspiel in fünf Akten}|pw}« übersetzt von Mme. Bianqui{[}s{]}\pwindex{Bianquis, Geneviève 19.\,9.\,1886 Rouen – 24.\,3.\,1972 Antony@\textsc{Bianquis, Geneviève} (19.\,9.\,1886 Rouen – 24.\,3.\,1972 Antony), \emph{Übersetzerin, Literaturhistorikerin}|pw} , 6, Rue Mou{[}t{]}on-Duvernet
                  XIV\oindex{6, Rue Mouton Duvernet@\textbf{6, Rue Mouton Duvernet}, \emph{Wohngebäude}|pw}. Sie hat auch das »Bachusfest\pwindex{Schnitzler, Arthur 15. 5. 1862 Wien – 21. 10. 1931 ebd.@\textsc{Schnitzler, Arthur} (15. 5. 1862 Wien – 21. 10. 1931 ebd.), \emph{Schriftsteller, Mediziner}!Bacchusfest@\strich\emph{Das Bacchusfest}|pw}« übersetzt\pwindex{Schnitzler, Arthur 15. 5. 1862 Wien – 21. 10. 1931 ebd.@\textsc{Schnitzler, Arthur} (15. 5. 1862 Wien – 21. 10. 1931 ebd.), \emph{Schriftsteller, Mediziner}!?? [französische Übersetzung von Das Bacchusfest]@\strich\emph{?? [französische Übersetzung von Das Bacchusfest]}|pwv}.\pend
           
\pstart
           »Zwischenspiel\pwindex{Schnitzler, Arthur 15. 5. 1862 Wien – 21. 10. 1931 ebd.@\textsc{Schnitzler, Arthur} (15. 5. 1862 Wien – 21. 10. 1931 ebd.), \emph{Schriftsteller, Mediziner}!Zwischenspiel. Komödie in drei Akten@\strich\emph{Zwischenspiel. Komödie in drei Akten}|pw}«, übersetzt\pwindex{Schnitzler, Arthur 15. 5. 1862 Wien – 21. 10. 1931 ebd.@\textsc{Schnitzler, Arthur} (15. 5. 1862 Wien – 21. 10. 1931 ebd.), \emph{Schriftsteller, Mediziner}!?? [französische Übersetzung von Zwischenspiel]@\strich\emph{?? [französische Übersetzung von Zwischenspiel]}|pwv} von Maurice Rèmon\pwindex{Rémon, Maurice 27.\,11.\,1861 Paris – 20.\,6.\,1945 Mérignac@\textsc{Rémon, Maurice} (27.\,11.\,1861 Paris – 20.\,6.\,1945 Mérignac), \emph{Übersetzer}|pw}, 10, Rue
                  Daubigny XVII\oindex{10, Rue Daubigny@\textbf{10, Rue Daubigny}, \emph{Wohngebäude}|pw}.\pend
           
\pstart
           Vor vielen Jahren wurden der Zyklus »Lebendige Stunden\pwindex{Schnitzler, Arthur 15. 5. 1862 Wien – 21. 10. 1931 ebd.@\textsc{Schnitzler, Arthur} (15. 5. 1862 Wien – 21. 10. 1931 ebd.), \emph{Schriftsteller, Mediziner}!Lebendige Stunden. Vier Einakter@\strich\emph{Lebendige Stunden. Vier Einakter}|pw}« von Rémon\pwindex{Rémon, Maurice 27.\,11.\,1861 Paris – 20.\,6.\,1945 Mérignac@\textsc{Rémon, Maurice} (27.\,11.\,1861 Paris – 20.\,6.\,1945 Mérignac), \emph{Übersetzer}|pw} und Mme.
                  Noemi Vallentin\pwindex{Valentin, Noémi @\textsc{Valentin, Noémi}, \emph{Übersetzerin}|pw}{ }übersetzt\pwindex{Schnitzler, Arthur 15. 5. 1862 Wien – 21. 10. 1931 ebd.@\textsc{Schnitzler, Arthur} (15. 5. 1862 Wien – 21. 10. 1931 ebd.), \emph{Schriftsteller, Mediziner}!Heures vives [Einakterzyklus]@\strich\emph{Heures vives [Einakterzyklus]}|pwv}. »Letzte Masken\pwindex{Schnitzler, Arthur 15. 5. 1862 Wien – 21. 10. 1931 ebd.@\textsc{Schnitzler, Arthur} (15. 5. 1862 Wien – 21. 10. 1931 ebd.), \emph{Schriftsteller, Mediziner}!letzten Masken@\strich\emph{Die letzten Masken}|pw}\pwindex{Schnitzler, Arthur 15. 5. 1862 Wien – 21. 10. 1931 ebd.@\textsc{Schnitzler, Arthur} (15. 5. 1862 Wien – 21. 10. 1931 ebd.), \emph{Schriftsteller, Mediziner}!Derniers masques. Comédie en un act@\strich\emph{Les Derniers masques. Comédie en un act}|pw}« auch irgend einmal \label{K_L03973-4v}\edtext{aufgeführt}{\lemma{\textnormal{\emph{aufgeführt}}}\Cendnote{\textnormal{Der Einakter \emph{Die letzten
                     Masken}\pwindex{Schnitzler, Arthur 15. 5. 1862 Wien – 21. 10. 1931 ebd.@\textsc{Schnitzler, Arthur} (15. 5. 1862 Wien – 21. 10. 1931 ebd.), \emph{Schriftsteller, Mediziner}!letzten Masken@\strich\emph{Die letzten Masken}|pwk} wurde in der Übersetzung\pwindex{Schnitzler, Arthur 15. 5. 1862 Wien – 21. 10. 1931 ebd.@\textsc{Schnitzler, Arthur} (15. 5. 1862 Wien – 21. 10. 1931 ebd.), \emph{Schriftsteller, Mediziner}!Derniers masques. Comédie en un act@\strich\emph{Les Derniers masques. Comédie en un act}|pwkv} von Rémon\pwindex{Rémon, Maurice 27.\,11.\,1861 Paris – 20.\,6.\,1945 Mérignac@\textsc{Rémon, Maurice} (27.\,11.\,1861 Paris – 20.\,6.\,1945 Mérignac), \emph{Übersetzer}|pwk} unter Regie
                  von Aurélien Lugné-Poe\pwindex{Lugné-Poe, Aurélien-Marie 27.\,12.\,1869 Paris – 19.\,6.\,1940 Villeneuve-les-Avignon@\textsc{Lugné-Poe, Aurélien-Marie} (27.\,12.\,1869 Paris – 19.\,6.\,1940 Villeneuve-les-Avignon), \emph{Theaterleiter, Regisseur, Schauspieler}|pwk} ab dem 1. 4. 1912\eventindex{Théâtre Antoine-Simone Berriau@\textbf{Théâtre Antoine-Simone Berriau}!Premiere von Les Derniers masques und Ariane blessée@Premiere von Les Derniers masques und Ariane blessée|pwkv} durch das \emph{Théâtre de l'Oeuvre}\orgindex{Théâtre de l’Œuvre@Théâtre de l’Œuvre|pwk} im Théâtre Antoine\oindex{Théâtre Antoine-Simone Berriau@\textbf{Théâtre Antoine-Simone Berriau}, \emph{Theater}|pwk} aufgeführt.}}}\label{K_L03973-4}.\pend
           
\pstart
           Auch Uebersetzungen\pwindex{Schnitzler, Arthur 15. 5. 1862 Wien – 21. 10. 1931 ebd.@\textsc{Schnitzler, Arthur} (15. 5. 1862 Wien – 21. 10. 1931 ebd.), \emph{Schriftsteller, Mediziner}!Compagne. Comédie en une acte@\strich\emph{La Compagne. Comédie en une acte}|pwv}\pwindex{Schnitzler, Arthur 15. 5. 1862 Wien – 21. 10. 1931 ebd.@\textsc{Schnitzler, Arthur} (15. 5. 1862 Wien – 21. 10. 1931 ebd.), \emph{Schriftsteller, Mediziner}!Au Perroquet Vert@\strich\emph{Au Perroquet Vert}|pwv}
               von »Gefährtin\pwindex{Schnitzler, Arthur 15. 5. 1862 Wien – 21. 10. 1931 ebd.@\textsc{Schnitzler, Arthur} (15. 5. 1862 Wien – 21. 10. 1931 ebd.), \emph{Schriftsteller, Mediziner}!Gefährtin. Schauspiel in einem Akt@\strich\emph{Die Gefährtin. Schauspiel in einem Akt}|pw}« und »Kakadu\pwindex{Schnitzler, Arthur 15. 5. 1862 Wien – 21. 10. 1931 ebd.@\textsc{Schnitzler, Arthur} (15. 5. 1862 Wien – 21. 10. 1931 ebd.), \emph{Schriftsteller, Mediziner}!grüne Kakadu. Groteske in einem Akt@\strich\emph{Der grüne Kakadu. Groteske in einem Akt}|pw}« (beide vor mehr als 20 Jahren \label{K_L03973-5v}\edtext{bei Antoine\pwindex{Antoine, André 31.\,1.\,1858 Limoges – 23.\,10.\,1943 Le Pouliguen@\textsc{Antoine, André} (31.\,1.\,1858 Limoges – 23.\,10.\,1943 Le Pouliguen), \emph{Theaterleiter, Schauspieler}|pw} aufgeführt}{\lemma{\textnormal{\emph{bei Antoine aufgeführt}}}\Cendnote{\textnormal{\emph{Die Gefährtin}\pwindex{Schnitzler, Arthur 15. 5. 1862 Wien – 21. 10. 1931 ebd.@\textsc{Schnitzler, Arthur} (15. 5. 1862 Wien – 21. 10. 1931 ebd.), \emph{Schriftsteller, Mediziner}!Gefährtin. Schauspiel in einem Akt@\strich\emph{Die Gefährtin. Schauspiel in einem Akt}|pwk} wurde in einer Übersetzung\pwindex{Schnitzler, Arthur 15. 5. 1862 Wien – 21. 10. 1931 ebd.@\textsc{Schnitzler, Arthur} (15. 5. 1862 Wien – 21. 10. 1931 ebd.), \emph{Schriftsteller, Mediziner}!Compagne. Comédie en une acte@\strich\emph{La Compagne. Comédie en une acte}|pwkv} von Maurice Vaucaire\pwindex{Vaucaire, Maurice 2.\,7.\,1863 Versailles – 10.\,2.\,1918 Neuilly-sur-Seine@\textsc{Vaucaire, Maurice} (2.\,7.\,1863 Versailles – 10.\,2.\,1918 Neuilly-sur-Seine), \emph{Schriftsteller, Schauspieler, Übersetzer}|pwk} ab dem 29. 4. 1902\eventindex{Théâtre Antoine-Simone Berriau@\textbf{Théâtre Antoine-Simone Berriau}!Premiere von La Compagne, 29.4.1902@Premiere von La Compagne, 29.4.1902|pwk} im Théâtre Antoine\oindex{Théâtre Antoine-Simone Berriau@\textbf{Théâtre Antoine-Simone Berriau}, \emph{Theater}|pwk}
                  aufgeführt, \emph{Der grüne Kakadu}\pwindex{Schnitzler, Arthur 15. 5. 1862 Wien – 21. 10. 1931 ebd.@\textsc{Schnitzler, Arthur} (15. 5. 1862 Wien – 21. 10. 1931 ebd.), \emph{Schriftsteller, Mediziner}!grüne Kakadu. Groteske in einem Akt@\strich\emph{Der grüne Kakadu. Groteske in einem Akt}|pwk} in der Übersetzung\pwindex{Schnitzler, Arthur 15. 5. 1862 Wien – 21. 10. 1931 ebd.@\textsc{Schnitzler, Arthur} (15. 5. 1862 Wien – 21. 10. 1931 ebd.), \emph{Schriftsteller, Mediziner}!Au Perroquet Vert@\strich\emph{Au Perroquet Vert}|pwkv} von Stephan Epstein\pwindex{Epstein, Stephan 12.\,11.\,1866 Warschau – 1941 Paris@\textsc{Epstein, Stephan} (12.\,11.\,1866 Warschau – 1941 Paris), \emph{Schriftsteller, Dramaturg, Übersetzer}|pwk} und Émile Lutz\pwindex{Lutz, Émile 8.\,4.\,1868 Saint-Étienne-du-Rouvray – 18.\,1.\,1940 Paris@\textsc{Lutz, Émile} (8.\,4.\,1868 Saint-Étienne-du-Rouvray – 18.\,1.\,1940 Paris), \emph{Übersetzer, Dichter}|pwk} ab dem 7. 11. 1903\eventindex{Théâtre Antoine-Simone Berriau@\textbf{Théâtre Antoine-Simone Berriau}!Premiere von Au Perroquet Vert, 7.11.1903@Premiere von Au Perroquet Vert, 7.11.1903|pwk}{ }ebendort\oindex{Théâtre Antoine-Simone Berriau@\textbf{Théâtre Antoine-Simone Berriau}, \emph{Theater}|pwkv}.}}}\label{K_L03973-5})
               existieren, sind aber, wie auch alle andern, nicht in meinen Händen, von »Liebelei\pwindex{Schnitzler, Arthur 15. 5. 1862 Wien – 21. 10. 1931 ebd.@\textsc{Schnitzler, Arthur} (15. 5. 1862 Wien – 21. 10. 1931 ebd.), \emph{Schriftsteller, Mediziner}!Amourette. Pièce en trois actes. Adaptée de Arthur Schnitzler@\strich\emph{Amourette. Pièce en trois actes. Adaptée de Arthur Schnitzler}|pwv}\pwindex{Schnitzler, Arthur 15. 5. 1862 Wien – 21. 10. 1931 ebd.@\textsc{Schnitzler, Arthur} (15. 5. 1862 Wien – 21. 10. 1931 ebd.), \emph{Schriftsteller, Mediziner}!Liebelei. Schauspiel in drei Akten@\strich\emph{Liebelei. Schauspiel in drei Akten}|pw}« abgesehen.\pend
           
\pstart
           Wenn Gemier\pwindex{Gémier, Firmin 21.\,2.\,1865 Aubervilliers – 26.\,11.\,1933 Paris@\textsc{Gémier, Firmin} (21.\,2.\,1865 Aubervilliers – 26.\,11.\,1933 Paris), \emph{Theaterleiter, Schauspieler, Drehbuchautor}|pw} wirklich etwas von mir aufführen
               will, so wird er sich eher die französischen\oindex{Frankreich@\textbf{Frankreich}|pw}
               Uebersetzungen zu verschaffen vermögen als ich, man müsste sich mit \label{T_L03973-1v}\edtext{Mme.}{\lemma{\textnormal{\emph{Mme.}}}\Cendnote{\textnormal{In der Vorlage steht: »Meme.«.}}}\label{T_L03973-1}{ }Bianqui{[}s{]}\pwindex{Bianquis, Geneviève 19.\,9.\,1886 Rouen – 24.\,3.\,1972 Antony@\textsc{Bianquis, Geneviève} (19.\,9.\,1886 Rouen – 24.\,3.\,1972 Antony), \emph{Übersetzerin, Literaturhistorikerin}|pw}, mit Mme. Cabir\pwindex{Cabire, Emma @\textsc{Cabire, Emma}, \emph{Übersetzerin, Redakteurin, Literaturagentin}|pw}, mit Herrn Lenormand\pwindex{Lenormand, Henri-René 3.\,5.\,1882 Paris – 16.\,2.\,1951 ebd.@\textsc{Lenormand, Henri-René} (3.\,5.\,1882 Paris – 16.\,2.\,1951 ebd.), \emph{Schriftsteller}|pw}, mit Herrn
                  Rémon\pwindex{Rémon, Maurice 27.\,11.\,1861 Paris – 20.\,6.\,1945 Mérignac@\textsc{Rémon, Maurice} (27.\,11.\,1861 Paris – 20.\,6.\,1945 Mérignac), \emph{Übersetzer}|pw} doch wohl irgendwie in Verbindung
               setzen können. Natürlich wären auch manche von meinen noch nicht übersetzten Stücken
               in Betracht zu ziehen. Da müsste sie Herr Gemier\pwindex{Gémier, Firmin 21.\,2.\,1865 Aubervilliers – 26.\,11.\,1933 Paris@\textsc{Gémier, Firmin} (21.\,2.\,1865 Aubervilliers – 26.\,11.\,1933 Paris), \emph{Theaterleiter, Schauspieler, Drehbuchautor}|pw} oder sein Ver{\pb}trauensmann, sei dies Herr
                  Blum\pwindex{Blum, Robert 17.\,4.\,1881 Wien – 3.\,7.\,1952 Paris@\textsc{Blum, Robert} (17.\,4.\,1881 Wien – 3.\,7.\,1952 Paris), \emph{Schriftsteller, Journalist, Theaterleiter}|pw} oder jemand anderer, lesen und für
               seine Zwecke prüfen; von sämtlichen Stücken für Gemier\pwindex{Gémier, Firmin 21.\,2.\,1865 Aubervilliers – 26.\,11.\,1933 Paris@\textsc{Gémier, Firmin} (21.\,2.\,1865 Aubervilliers – 26.\,11.\,1933 Paris), \emph{Theaterleiter, Schauspieler, Drehbuchautor}|pw} Probeübersetzungen anzufertigen ginge wohl kaum an. übrigens ist ja
               die deutsche Sprache im Laufe der letzten Zeit auch in Frankreich\oindex{Frankreich@\textbf{Frankreich}|pw} verhältnismässig bekannt geworden. Jedenfalls werde ich mich sehr
               freuen, recht bald über die ganze Angelegenheit mehr zu hören.\pend
           
\pstart
           Ebenso wie die von Hela\pwindex{Hella, Alzir 30.\,12.\,1881 Vieux Condé – 14.\,7.\,1953 Paris@\textsc{Hella, Alzir} (30.\,12.\,1881 Vieux Condé – 14.\,7.\,1953 Paris), \emph{Übersetzer}|pw} besorgte Uebersetzung\pwindex{Schnitzler, Arthur 15. 5. 1862 Wien – 21. 10. 1931 ebd.@\textsc{Schnitzler, Arthur} (15. 5. 1862 Wien – 21. 10. 1931 ebd.), \emph{Schriftsteller, Mediziner}!Madame Beate et son fils@\strich\emph{Madame Beate et son fils}|pwv} von »Beate\pwindex{Schnitzler, Arthur 15. 5. 1862 Wien – 21. 10. 1931 ebd.@\textsc{Schnitzler, Arthur} (15. 5. 1862 Wien – 21. 10. 1931 ebd.), \emph{Schriftsteller, Mediziner}!Frau Beate und ihr Sohn. Novelle@\strich\emph{Frau Beate und ihr Sohn. Novelle}|pw}« ist auch noch »Casanovas Heimfahrt\pwindex{Schnitzler, Arthur 15. 5. 1862 Wien – 21. 10. 1931 ebd.@\textsc{Schnitzler, Arthur} (15. 5. 1862 Wien – 21. 10. 1931 ebd.), \emph{Schriftsteller, Mediziner}!Le Retour de Casanova@\strich\emph{Le Retour de Casanova}|pw}\pwindex{Schnitzler, Arthur 15. 5. 1862 Wien – 21. 10. 1931 ebd.@\textsc{Schnitzler, Arthur} (15. 5. 1862 Wien – 21. 10. 1931 ebd.), \emph{Schriftsteller, Mediziner}!Casanovas Heimfahrt@\strich\emph{Casanovas Heimfahrt}|pw}«, die vor etwa vier Jahren \label{K_L03973-6v}\edtext{von Nathan\pwindex{Nathan, Nicolas @\textsc{Nathan, Nicolas}, \emph{Übersetzer}|pw} erworben}{\lemma{\textnormal{\emph{von Nathan erworben}}}\Cendnote{\textnormal{Die \emph{Übersetzung}\pwindex{Schnitzler, Arthur 15. 5. 1862 Wien – 21. 10. 1931 ebd.@\textsc{Schnitzler, Arthur} (15. 5. 1862 Wien – 21. 10. 1931 ebd.), \emph{Schriftsteller, Mediziner}!Le Retour de Casanova@\strich\emph{Le Retour de Casanova}|pwk}, um deren Rechte es seit 1924 in der Korrespondenz Schnitzlers mit Nicolas Nathan\pwindex{Nathan, Nicolas @\textsc{Nathan, Nicolas}, \emph{Übersetzer}|pwk} ging, wurde schließlich von Maurice Remon\pwindex{Rémon, Maurice 27.\,11.\,1861 Paris – 20.\,6.\,1945 Mérignac@\textsc{Rémon, Maurice} (27.\,11.\,1861 Paris – 20.\,6.\,1945 Mérignac), \emph{Übersetzer}|pwk} übernommen und erst 1930 publiziert.}}}\label{K_L03973-6} wurde, unpubliziert. Wenn es Ihnen, liebe
               Freundin, keine zu grosse Mühe macht, wollten Sie nicht die Güte haben bei N. Nathan\pwindex{Nathan, Nicolas @\textsc{Nathan, Nicolas}, \emph{Übersetzer}|pw}, Hotel Roncerey, 10 Boulevard Monmartre\oindex{Hôtel Ronceray@\textbf{Hôtel Ronceray}, \emph{Hotel}|pw}, gelegentlich anzufragen?\pend
           
\pstart
           Mit den herzlichsten Grüssen{\\[\baselineskip]} Ihr\pend
           \leftskip=0em{}{\vspace{1\baselineskip}}
\pstart
           \noindent{}Frau Hofrätin Bertha Zuckerkandl,{\\}Paris\oindex{Paris@\textbf{Paris}, \emph{Hauptstadt}|pw}.\pend
           \selectlanguage{ngerman}\endnumbering\briefempfaengerindex{Zuckerkandl, Berta@\textsc{Zuckerkandl, Berta}!zzzSchnitzler, Arthur@\emph{von Arthur Schnitzler}!1927-11-211@{21. 11. 1927}|)be}\mylabel{L03973h}
\begin{anhang}
\end{anhang}\newcommand{\dateiname}{L03973}\newcommand{\titel}{Arthur Schnitzler an Berta Zuckerkandl, 21. 11. 1927}\newcommand{\editorInnen}{Herausgegeben von Jahnke, SelmaMüller, Martin Anton}%% latex-leseansicht-abspann.tex
%% Abspann für die Leseansicht.
%% Der Schalter \ifkorrekturansicht ist bereits durch den Vorspann gesetzt.

%% latex-abspann.tex
%% Gemeinsamer Abspann für Korrekturansicht und Leseansicht.
%% Setzt den Schalter \ifkorrekturansicht voraus (gesetzt in den
%% einbindenden Dateien latex-korrekturansicht-abspann.tex bzw.
%% latex-leseansicht-abspann.tex).
%% ---------------------------------------------------------------

\normalsize

% Das esempio-Environment wird nur in der Leseansicht benötigt
\ifkorrekturansicht\else
\newenvironment{esempio}[3]%
{
    \vspace{1.5ex}
    \rlap{\underline{#1}}
    \par
    \setlength{\parindent}{0cm}
    \nopagebreak
    \leftskip=#2cm
    \rightskip=#3cm
}
{
    \par
}
\fi

\doendnotes{C}
\bigskip
\vfill

\clearpage

\footnotesize

\ifkorrekturansicht
  \lohead{\textsc{register}}
\fi

% theindex-Environment neu definieren ohne reledmac
\makeatletter
\renewenvironment{theindex}{%
  \ifkorrekturansicht
    \section*{\indexname}%
  \else
    \subsubsection*{Index der erwähnten Entitäten}%
  \fi
  \setlength{\parindent}{0pt}%
  \setlength{\parskip}{0pt plus 0.3pt}%
  \let\item\@idxitem
}{%
  \ifkorrekturansicht\clearpage\fi
}
\makeatother

\IfFileExists{\jobname-pw.ind}{\input{\jobname-pw.ind}}{}

% Quellenangabe nur in der Leseansicht
\ifkorrekturansicht\else
% Fallback-Definitionen, falls die .tex-Datei \titel etc. nicht gesetzt hat
\providecommand{\titel}{}
\providecommand{\editorInnen}{}
\providecommand{\dateiname}{\jobname}

\vspace{3cm}

\vfill

\footnotesize
\textsc{Quelle}: \titel. Herausgegeben von {\editorInnen}. In: \emph{Arthur Schnitzler: Briefwechsel mit Autorinnen und Autoren}.
 Digitale Edition, https://schnitzler-briefe.acdh.oeaw.ac.at/{\dateiname}.html (Stand \today)
\fi

\end{document}


