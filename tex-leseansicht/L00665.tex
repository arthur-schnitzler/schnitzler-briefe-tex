%% latex-leseansicht-vorspann.tex
%% Vorspann für die Leseansicht.
%% Lädt die gemeinsame Datei latex-vorspann.tex mit nicht gesetztem Schalter.

\newif\ifkorrekturansicht
\korrekturansichtfalse

\input{../tex-inputs/latex-vorspann}

\begin{center}
            \textcolor{red}{ENTWURF. ENTZIFFERUNG NOCH NICHT KORREKTURGELESEN}
                      \end{center}
            
               \section[Arthur Schnitzler an Richard Beer-Hofmann, 10. 4. 1897]{ Arthur Schnitzler an Richard Beer-Hofmann,
               10. 4. 1897}\nopagebreak\mylabel{v}\rehead{ }\begin{ledgroupsized}[t]{13cm}\normalsize\beginnumbering\briefempfaengerindex{Beer-Hofmann, Richard@\textsc{Beer-Hofmann, Richard}!zzzSchnitzler, Arthur@\emph{von Arthur Schnitzler}!1897-04-101@{10. 4. 1897}|(be} \toendnotes[C]{\smallbreak\pagebreak[2]} \Standort{YCGL, MSS 31.}
\physDesc{Postkarte
\newline{}Handschrift: Bleistift, deutsche Kurrent\newline{}Versand: 1) Stempel: »\nobreak{}\oindex{Muenchen@\textbf{München}|pwk}Muenc\textcolor{gray}{hen} 2, 10 Apr \textcolor{gray}{9}7, 9–10\nobreak{}«.  2) Stempel: »\nobreak{}\oindex{I., Innere Stadt@\textbf{I., Innere Stadt}|pwk}Wien 1/1, 11{[}. 4. 97{]}, 9–10\textcolor{gray}{½}V, Bestellt\nobreak{}«. }\buchAbdrucke{\weitereDrucke{Arthur Schnitzler, Richard Beer-Hofmann: \emph{Briefwechsel 1891–1931}. Hg. Konstanze Fliedl. Wien, Zürich: \emph{Europaverlag} 1992, S. 101.} }\pstart{}{\pb}Herrn \textsc{Dr. Rich.
                     Beer-Hofmann}\pend{}\pstart{}\textsc{Wien}\oindex{Wien@\textbf{Wien}|pw}\pend{}\pstart{}\textsc{I. Wollzeile 15}\oindex{Wollzeile@\textbf{Wollzeile}|pw}.\pend{}{\bigskip}\pstart
           \noindent{}{\pb}Lieber Richard; geſtern hier im Dtſch. Th.\orgindex{Deutsches Theater Muenchen@Deutsches Theater München|pw}{ }Freiwild\pwindex{Schnitzler, Arthur 15.05.1862 – 21.10.1931@\textsc{Schnitzler, Arthur} (15.05.1862 – 21.10.1931), \emph{Schriftsteller, Mediziner}!Freiwild. Schauspiel in 3 Akten1896@\strich\emph{Freiwild. Schauspiel in 3 Akten} {[}1896{]}|pw}; der heutigen »Liebelei\pwindex{Schnitzler, Arthur 15.05.1862 – 21.10.1931@\textsc{Schnitzler, Arthur} (15.05.1862 – 21.10.1931), \emph{Schriftsteller, Mediziner}!Liebelei. Schauspiel in drei Akten9. 10. 1895@\strich\emph{Liebelei. Schauspiel in drei Akten} {[}9. 10. 1895{]}|pw}« entgleite ich. Fahre nach Zürich\oindex{Zuerich@\textbf{Zürich}|pw}. Bin Montag in Paris\oindex{Paris@\textbf{Paris}|pw}. – Schreiben Sie mir zu Goldmann\pwindex{Goldmann, Paul 31.01.1865 – 25.09.1935@\textsc{Goldmann, Paul} (31.01.1865 – 25.09.1935), \emph{Schriftsteller, Journalist}|pw}. – Grüßen Sie.\pend
           \pstart
           Herzlich Ihr{\\}\spacefill\mbox{Arthur.}\pend
           \endnumbering\briefempfaengerindex{Beer-Hofmann, Richard@\textsc{Beer-Hofmann, Richard}!zzzSchnitzler, Arthur@\emph{von Arthur Schnitzler}!1897-04-101@{10. 4. 1897}|)be}\mylabel{h}\end{ledgroupsized}  \newcommand{\dateiname}{L00665}\newcommand{\titel}{Arthur Schnitzler an Richard Beer-Hofmann, 10. 4. 1897}\newcommand{\editorInnen}{Martin Anton Müller und Gerd-Hermann Susen}%% latex-leseansicht-abspann.tex
%% Abspann für die Leseansicht.
%% Der Schalter \ifkorrekturansicht ist bereits durch den Vorspann gesetzt.

%% latex-abspann.tex
%% Gemeinsamer Abspann für Korrekturansicht und Leseansicht.
%% Setzt den Schalter \ifkorrekturansicht voraus (gesetzt in den
%% einbindenden Dateien latex-korrekturansicht-abspann.tex bzw.
%% latex-leseansicht-abspann.tex).
%% ---------------------------------------------------------------

\normalsize

% Das esempio-Environment wird nur in der Leseansicht benötigt
\ifkorrekturansicht\else
\newenvironment{esempio}[3]%
{
    \vspace{1.5ex}
    \rlap{\underline{#1}}
    \par
    \setlength{\parindent}{0cm}
    \nopagebreak
    \leftskip=#2cm
    \rightskip=#3cm
}
{
    \par
}
\fi

\doendnotes{C}
\bigskip
\vfill

\clearpage

\footnotesize

\ifkorrekturansicht
  \lohead{\textsc{register}}
\fi

% theindex-Environment neu definieren ohne reledmac
\makeatletter
\renewenvironment{theindex}{%
  \ifkorrekturansicht
    \section*{\indexname}%
  \else
    \subsubsection*{Index der erwähnten Entitäten}%
  \fi
  \setlength{\parindent}{0pt}%
  \setlength{\parskip}{0pt plus 0.3pt}%
  \let\item\@idxitem
}{%
  \ifkorrekturansicht\clearpage\fi
}
\makeatother

\IfFileExists{\jobname-pw.ind}{\input{\jobname-pw.ind}}{}

% Quellenangabe nur in der Leseansicht
\ifkorrekturansicht\else
% Fallback-Definitionen, falls die .tex-Datei \titel etc. nicht gesetzt hat
\providecommand{\titel}{}
\providecommand{\editorInnen}{}
\providecommand{\dateiname}{\jobname}

\vspace{3cm}

\vfill

\footnotesize
\textsc{Quelle}: \titel. Herausgegeben von {\editorInnen}. In: \emph{Arthur Schnitzler: Briefwechsel mit Autorinnen und Autoren}.
 Digitale Edition, https://schnitzler-briefe.acdh.oeaw.ac.at/{\dateiname}.html (Stand \today)
\fi

\end{document}


      