%% latex-korrekturansicht-vorspann.tex
%% Vorspann für die Korrekturansicht.
%% Lädt die gemeinsame Datei latex-vorspann.tex mit gesetztem Schalter.

\newif\ifkorrekturansicht
\korrekturansichttrue

\input{../tex-inputs/latex-vorspann}


\section[Stefan Großmann an Arthur Schnitzler, 27. 9. 1907]{L01711 Stefan Großmann an Arthur Schnitzler, 27. 9. 1907}
\nopagebreak\mylabel{L01711v}
\rehead{ }\normalsize\beginnumbering\briefempfaengerindex{Schnitzler, Arthur@\textsc{Schnitzler, Arthur}!zzzGrossmann, Stefan@\emph{von Stefan Großmann}!1907-09-272@{27. 9. 1907}|(be}
\toendnotes[C]{\smallbreak\pagebreak[2]}\Standort{CUL, Schnitzler, B 34.}
\physDesc{Brief, 1 Blatt, 1 Seite, 678 Zeichen (Briefpapier mit Trauerrand)
\newline{}Handschrift: schwarze Tinte, deutsche Kurrent
\newline{}Schnitzler: 1) mit Bleistift die Monatsangabe korrigiert: »Sept. –«  2) mit rotem Buntstift eine Unterstreichung
\newline{}Ordnung: mit Bleistift von unbekannter Hand nummeriert:
                                 »3« }\toendnotes[C]{\smallbreak}
\pstart
           {\pb}\textcolor{gray}{\textbf{Freie Volksbühne\orgindex{Wiener Freie Volksbuehne@Wiener Freie Volksbühne|pw}}}\pend
           
\pstart
           \textcolor{gray}{\textbf{Wien VI/\textsubscript{1}\oindex{Wien@\textbf{Wien}, \emph{A.ADM2}|pw}}}\pend
           
\pstart
           \textcolor{gray}{\textbf{Mariahilferſtraße Nr. 89\oindex{Mariahilfer Strasse@\textbf{Mariahilfer Straße}, \emph{Straße (K.STR)}|pw}.}}\hfill \textcolor{gray}{\textbf{Wien\oindex{Wien@\textbf{Wien}, \emph{A.ADM2}|pw}, am}}{ }27. \label{K_L01711-1v}\edtext{Augſt.}{\lemma{\textnormal{\emph{Augſt.}}}\Cendnote{\textnormal{Es dürfte sich um einen
                           Schreibirrtum handeln, der schon von Schnitzler korrigiert wurde.}}}\label{K_L01711-1}\textcolor{gray}{\textbf{190}}7\pend
           
\pstart
           \textcolor{gray}{\textbf{Poſtſparkaſſen-Konto Nr. 87.544.}}\pend
           
\pstart
           Herrn Arthur Schnitzler\hspace*{1.5em}Wien\oindex{Wien@\textbf{Wien}, \emph{A.ADM2}|pw}\pend
           
\pstart{}Sehr verehrter Herr.\pend\vspace{0.5em}
\pstart
           Würden Sie, verehrter Herr, einmal an einem Abend vor Mitgliedern der Freien Volksbühne\orgindex{Wiener Freie Volksbuehne@Wiener Freie Volksbühne|pw} eigene Dichtungen vorleſen woll\substVorne{}\textsuperscript{t}\substDazwischen{}e\substHinten{}n?\pend
           
\pstart
           Für eine andächtig u aufmerkſam lauſchende Zuhörerſchaft, aus der Elite der Wien\oindex{Wien@\textbf{Wien}, \emph{A.ADM2}|pw}er Arbeiterſchaft zuſammengesetzt, kann ich
               mich verbürgen.\pend
           
\pstart
           Wir würden die Vorleſung an einem Donnerstag oder Mittwochabend in einem ſchönen
               Verſammlungsſaal veranſtalten und zwar, wenn es Ihnen recht wäre, ſchon Mitte
               Oktober.\pend
           
\pstart
           \strikeout{Hierbei} Es würde uns große Freude bereiten, wenn Sie
               Ihre freundliche Entſcheidung bald bekanntgeben wollten.\pend
           
\pstart
           Mit der Versicherung \uline{dankbarer} Ergebenheit{\\[\baselineskip]}
                  f. d. Fr. V.\orgindex{Wiener Freie Volksbuehne@Wiener Freie Volksbühne|pw}\spacefill\mbox{Stefan Großmann}\pend
           \leftskip=0em{}
\pstart
           \noindent{}Wien I. Graben 29\textsuperscript{a}\oindex{Graben@\textbf{Graben}, \emph{Straße (K.STR)}|pw}\pend
           \selectlanguage{ngerman}\endnumbering\briefempfaengerindex{Schnitzler, Arthur@\textsc{Schnitzler, Arthur}!zzzGrossmann, Stefan@\emph{von Stefan Großmann}!1907-09-272@{27. 9. 1907}|)be}\mylabel{L01711h}  \normalsize

\doendnotes{C}
\bigskip
\vfill

\clearpage

\footnotesize

\lohead{\textsc{register}}

% Definiere theindex-Environment komplett neu ohne reledmac
\makeatletter
\renewenvironment{theindex}{%
  \section*{\indexname}%
  \setlength{\parindent}{0pt}%
  \setlength{\parskip}{0pt plus 0.3pt}%
  \let\item\@idxitem
}{%
  \clearpage
}
\makeatother

\IfFileExists{\jobname-pw.ind}{\input{\jobname-pw.ind}}{}

\end{document}

      