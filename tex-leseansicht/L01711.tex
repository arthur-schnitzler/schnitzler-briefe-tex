%% latex-leseansicht-vorspann.tex
%% Vorspann für die Leseansicht.
%% Lädt die gemeinsame Datei latex-vorspann.tex mit nicht gesetztem Schalter.

\newif\ifkorrekturansicht
\korrekturansichtfalse

\input{../tex-inputs/latex-vorspann}

\begin{center}
            \textcolor{red}{ENTWURF. ENTZIFFERUNG NOCH NICHT KORREKTURGELESEN}
                      \end{center}
            
               \section[Stefan Großmann an Arthur Schnitzler, 27. 9. 1907]{ Stefan Großmann an Arthur Schnitzler, 27. 9. 1907}\nopagebreak\mylabel{v}\rehead{ }\begin{ledgroupsized}[t]{13cm}\normalsize\beginnumbering\briefempfaengerindex{Schnitzler, Arthur@\textsc{Schnitzler, Arthur}!zzzGrossmann, Stefan@\emph{von Stefan Großmann}!1907-09-272@{27. 9. 1907}|(be} \toendnotes[C]{\smallbreak\pagebreak[2]} \Standort{CUL, Schnitzler, B 34.}
\physDesc{Brief, 1 Blatt (Briefpapier mit Trauerrand), 1 Seite
\newline{}Handschrift: schwarze Tinte, deutsche Kurrent
\newline{}Schnitzler: 1) mit Bleistift die Monatsangabe korrigiert: »Sept. –« 2) mit rotem Buntstift eine Unterstreichung\newline{}Ordnung: mit Bleistift von unbekannter Hand nummeriert:
                              »3« }\toendnotes[C]{\smallbreak}\pstart
           \noindent{}{\pb}\textcolor{gray}{\textbf{Freie Volksbühne\orgindex{Wiener Freie Volksbuehne@Wiener Freie Volksbühne|pw}}}\pend
           \pstart
           \textcolor{gray}{\textbf{Wien VI/\textsubscript{1}\oindex{Wien@\textbf{Wien}|pw}}}\pend
           \pstart
           \textcolor{gray}{\textbf{Mariahilferſtraße Nr. 89\oindex{Mariahilferstrasse@\textbf{Mariahilferstraße}|pw}.}}\hfill \textcolor{gray}{\textbf{Wien\oindex{Wien@\textbf{Wien}|pw}, am}}{ }27. \label{K_L01711_1v}\edtext{Augſt.}{\lemma{\textnormal{\emph{Augſt.}}}\Cendnote{\textnormal{Es dürfte sich um einen
                           Schreibirrtum handeln, der schon von Schnitzler\pwindex{Schnitzler, Arthur 15.05.1862 – 21.10.1931@\textsc{Schnitzler, Arthur} (15.05.1862 – 21.10.1931), \emph{Schriftsteller, Mediziner}|pwk} korrigiert wurde.}}}\label{K_L01711_1h}\textcolor{gray}{\textbf{190}}7\pend
           \pstart
           \textcolor{gray}{\textbf{Poſtſparkaſſen-Konto Nr. 87.544.}}\pend
           \pstart
           Herrn Arthur Schnitzler\hspace*{1.5em}Wien\oindex{Wien@\textbf{Wien}|pw}\pend
           \pstart{}Sehr verehrter Herr.\pend\pstart
           Würden Sie, verehrter Herr, einmal an einem Abend vor Mitgliedern der Freien Volksbühne\orgindex{Wiener Freie Volksbuehne@Wiener Freie Volksbühne|pw} eigene Dichtungen vorleſen woll\substVorne{}\textsuperscript{t}\substDazwischen{}e\substHinten{}n?\pend
           \pstart
           Für eine andächtig u aufmerkſam lauſchende Zuhörerſchaft, aus der Elite der Wien\oindex{Wien@\textbf{Wien}|pw}er Arbeiterſchaft zuſammengesetzt, kann ich mich
               verbürgen.\pend
           \pstart
           Wir würden die Vorleſung an einem Donnerstag oder Mittwochabend in einem ſchönen
               Verſammlungsſaal veranſtalten und zwar, wenn es Ihnen recht wäre, ſchon Mitte
               Oktober.\pend
           \pstart
           \strikeout{Hierbei} Es würde uns große Freude bereiten, wenn Sie
               Ihre freundliche Entſcheidung bald bekanntgeben wollten.\pend
           \pstart
           Mit der Versicherung \uline{dankbarer} Ergebenheit{\\[\baselineskip]}
                  f. d. Fr. V.\orgindex{Wiener Freie Volksbuehne@Wiener Freie Volksbühne|pw}\spacefill\mbox{Stefan
                  Großmann}\pend
           \leftskip=0em{}\pstart
           \noindent{}Wien I. Graben 29\textsuperscript{a}\oindex{Graben@\textbf{Graben}|pw}\pend
           \endnumbering\briefempfaengerindex{Schnitzler, Arthur@\textsc{Schnitzler, Arthur}!zzzGrossmann, Stefan@\emph{von Stefan Großmann}!1907-09-272@{27. 9. 1907}|)be}\mylabel{h}\end{ledgroupsized}  \newcommand{\dateiname}{L01711}\newcommand{\titel}{Stefan Großmann an Arthur Schnitzler, 27. 9. 1907}\newcommand{\editorInnen}{ Martin Anton Müller und Gerd-Hermann Susen}%% latex-leseansicht-abspann.tex
%% Abspann für die Leseansicht.
%% Der Schalter \ifkorrekturansicht ist bereits durch den Vorspann gesetzt.

%% latex-abspann.tex
%% Gemeinsamer Abspann für Korrekturansicht und Leseansicht.
%% Setzt den Schalter \ifkorrekturansicht voraus (gesetzt in den
%% einbindenden Dateien latex-korrekturansicht-abspann.tex bzw.
%% latex-leseansicht-abspann.tex).
%% ---------------------------------------------------------------

\normalsize

% Das esempio-Environment wird nur in der Leseansicht benötigt
\ifkorrekturansicht\else
\newenvironment{esempio}[3]%
{
    \vspace{1.5ex}
    \rlap{\underline{#1}}
    \par
    \setlength{\parindent}{0cm}
    \nopagebreak
    \leftskip=#2cm
    \rightskip=#3cm
}
{
    \par
}
\fi

\doendnotes{C}
\bigskip
\vfill

\clearpage

\footnotesize

\ifkorrekturansicht
  \lohead{\textsc{register}}
\fi

% theindex-Environment neu definieren ohne reledmac
\makeatletter
\renewenvironment{theindex}{%
  \ifkorrekturansicht
    \section*{\indexname}%
  \else
    \subsubsection*{Index der erwähnten Entitäten}%
  \fi
  \setlength{\parindent}{0pt}%
  \setlength{\parskip}{0pt plus 0.3pt}%
  \let\item\@idxitem
}{%
  \ifkorrekturansicht\clearpage\fi
}
\makeatother

\IfFileExists{\jobname-pw.ind}{\input{\jobname-pw.ind}}{}

% Quellenangabe nur in der Leseansicht
\ifkorrekturansicht\else
% Fallback-Definitionen, falls die .tex-Datei \titel etc. nicht gesetzt hat
\providecommand{\titel}{}
\providecommand{\editorInnen}{}
\providecommand{\dateiname}{\jobname}

\vspace{3cm}

\vfill

\footnotesize
\textsc{Quelle}: \titel. Herausgegeben von {\editorInnen}. In: \emph{Arthur Schnitzler: Briefwechsel mit Autorinnen und Autoren}.
 Digitale Edition, https://schnitzler-briefe.acdh.oeaw.ac.at/{\dateiname}.html (Stand \today)
\fi

\end{document}


      