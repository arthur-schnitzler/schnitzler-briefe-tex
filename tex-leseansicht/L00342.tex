%% latex-leseansicht-vorspann.tex
%% Vorspann für die Leseansicht.
%% Lädt die gemeinsame Datei latex-vorspann.tex mit nicht gesetztem Schalter.

\newif\ifkorrekturansicht
\korrekturansichtfalse

\input{../tex-inputs/latex-vorspann}


\section[Richard Beer-Hofmann an Arthur Schnitzler, 30. 6. 1894]{L00342 Richard Beer-Hofmann an Arthur Schnitzler, 30. 6. 1894}
\nopagebreak\mylabel{L00342v}
\rehead{ }\normalsize\beginnumbering\briefempfaengerindex{Schnitzler, Arthur@\textsc{Schnitzler, Arthur}!zzzBeer-Hofmann, Richard@\emph{von Richard Beer-Hofmann}!1894-06-301@{30. 6. 1894}|(be}
\toendnotes[C]{\smallbreak\pagebreak[2]}
\correspDesc{Versand  durch Richard Beer-Hofmann am 30. 6. 1894 in Bad Ischl
\newline{}Erhalt  durch Arthur Schnitzler im Zeitraum [1. 7. 1894
                  – 5. 7. 1894?] in Wien}\toendnotes[C]{\smallbreak}
\Standort{CUL, Schnitzler, B 8.}
\physDesc{Brief, 2 Blätter, 7 Seiten, 1237 Zeichen
\newline{}Handschrift: blauer Buntstift, lateinische Kurrent
\newline{}Schnitzler: mit Bleistift datiert: »30/6 94« und nummeriert: »33« }
\buchAbdrucke{\weitereDrucke{Arthur Schnitzler, Richard Beer-Hofmann: \emph{Briefwechsel 1891–1931}. Herausgegeben von Konstanze Fliedl. Wien, Zürich: \emph{Europaverlag} 1992, S. 55–56.} }\toendnotes[C]{\smallbreak}
\pstart{}{\pb}Lieber Arthur!\pend\vspace{0.5em}
\pstart
           An F.\pwindex{Fels, Friedrich Michael *~1864 Bad Dürkheim@\textsc{Fels, Friedrich Michael} (*~1864 Bad Dürkheim), \emph{Journalist}|pw} hatte ich natürlich vergessen, ordnete
               aber die Sache sofort nach Erhalt Ihres Briefes. –\pend
           
\pstart
           Unter welcher Adresse \label{K_L00342-1v}\edtext{gratulirt\eventindex{Wien@\textbf{Wien}!Hochzeit von Helene Altmann und Julius Schnitzler, 8.7.1894@Hochzeit von Helene Altmann und Julius Schnitzler, 8.7.1894|pwv}}{\lemma{\textnormal{\emph{gratulirt}}}\Cendnote{\textnormal{Schnitzlers Bruder
               Julius\pwindex{Schnitzler, Julius 13.\,7.\,1865 Wien – 29.\,6.\,1939 ebd.@\textsc{Schnitzler, Julius} (13.\,7.\,1865 Wien – 29.\,6.\,1939 ebd.), \emph{Chirurg}|pwk} und Helene
                  Altmann\pwindex{Schnitzler, Helene 16.\,7.\,1871 Budapest – September 1941 Atlantischer Ozean@\textsc{Schnitzler, Helene} (16.\,7.\,1871 Budapest – September 1941 Atlantischer Ozean)|pwk} heirateten am 8. 7. 1894.}}}\label{K_L00342-1} man Ihrem Bruder\pwindex{Schnitzler, Julius 13.\,7.\,1865 Wien – 29.\,6.\,1939 ebd.@\textsc{Schnitzler, Julius} (13.\,7.\,1865 Wien – 29.\,6.\,1939 ebd.), \emph{Chirurg}|pwv}?\pend
           
\pstart
           Bitte Sie um Folgendes: Ich brauche ein \label{K_L00342-2v}\edtext{Cachenez}{\lemma{\textnormal{\emph{Cachenez}}}\Cendnote{\textnormal{ein Schal}}}\label{K_L00342-2} welches so
               groß ist, daß {\pb}man es falten und
               als Schärpe binden kann. Es soll ganz \uline{schwarz} sein
               und zwar \uline{schwerer}{ }\uline{weicher}{ }\uline{matter} seidenstoff – nicht Atlas – womöglich schwarz
               in schwarz gemustert, vielleicht brokatartig. Wenn Sie es bei Stoll + Uhlig\orgindex{Stoll und Uhlig@Stoll {\kaufmannsund}  Uhlig|pw}{ }{\pb}beko{\geminationm}en, dann lassen Sie es mir direkt zusenden ohne zu bezahlen, beko{\geminationm}en Sie es dort nicht, oder sehen Sie irgendwo etwas
               Passendes, so lassen Sie es mir zusenden und bezahlen unterdessen. Es kann übrigens
               auch {\pb}\uline{wenn es das giebt} (?) schwarze glatte \uline{Roh}seide sein.\pend
           
\pstart
           Bahr\pwindex{Bahr, Hermann 19.\,7.\,1863 Linz – 15.\,1.\,1934 München@\textsc{Bahr, Hermann} (19.\,7.\,1863 Linz – 15.\,1.\,1934 München), \emph{Schriftsteller, Kritiker}|pw} war vorgestern zwei Stunden in Ischl\oindex{Bad Ischl@\textbf{Bad Ischl}|pw}.\pend
           
\pstart
           Kappers\pwindex{Kapper, Friedrich 21.\,4.\,1861 Wien – 22.\,7.\,1939 ebd.@\textsc{Kapper, Friedrich} (21.\,4.\,1861 Wien – 22.\,7.\,1939 ebd.), \emph{Mediziner}|pw}\pwindex{Kapper, Adele 25.\,1.\,1870 Wien – 1941 Vernichtungslager Maly Trostinez@\textsc{Kapper, Adele} (25.\,1.\,1870 Wien – 1941 Vernichtungslager Maly Trostinez)|pw} sind hier, ich predige ihm\pwindex{Kapper, Friedrich 21.\,4.\,1861 Wien – 22.\,7.\,1939 ebd.@\textsc{Kapper, Friedrich} (21.\,4.\,1861 Wien – 22.\,7.\,1939 ebd.), \emph{Mediziner}|pwv} Unmoral und beweise ihm
               wie bescheiden {\pb}er sein müsste. Paul Schulz\pwindex{Schulz, Paul 1.\,7.\,1860 Wien – 31.\,1.\,1919 Kreuzlingen@\textsc{Schulz, Paul} (1.\,7.\,1860 Wien – 31.\,1.\,1919 Kreuzlingen), \emph{Ministerialbeamter, Beamter}|pw} sprach ich; was hat der wieder
               gegen Sie? Oder vielmehr gegen das »Abschiedssouper\pwindex{Schnitzler, Arthur 15.\,5.\,1862 Wien – 21.\,10.\,1931 ebd.@\textsc{Schnitzler, Arthur} (15.\,5.\,1862 Wien – 21.\,10.\,1931 ebd.), \emph{Schriftsteller, Mediziner}!Abschiedssouper@\strich\emph{Abschiedssouper}|pw}«? Übrigens liebt er auch den Styl J. Opp{\dots}\pwindex{Oppenheim, Josef 17.\,12.\,1839 Arheiligen – 12.\,7.\,1900 Baden bei Wien@\textsc{Oppenheim, Josef} (17.\,12.\,1839 Arheiligen – 12.\,7.\,1900 Baden bei Wien), \emph{Journalist}|pw} und mag den Th. Herzl\pwindex{Herzl, Theodor 2.\,5.\,1860 Budapest – 3.\,7.\,1904 Edlach@\textsc{Herzl, Theodor} (2.\,5.\,1860 Budapest – 3.\,7.\,1904 Edlach), \emph{Schriftsteller, Journalist}|pw} nicht.\pend
           
\pstart
           {\pb}Ko{\geminationm}en
               Sie bald nach der Hochzeit Ihres Bruders\pwindex{Schnitzler, Julius 13.\,7.\,1865 Wien – 29.\,6.\,1939 ebd.@\textsc{Schnitzler, Julius} (13.\,7.\,1865 Wien – 29.\,6.\,1939 ebd.), \emph{Chirurg}|pwv}? Leopold\oindex{Hotel und Pension Rudolfshöhe (Leopold Petter)@\textbf{Hotel und Pension Rudolfshöhe (Leopold Petter)}, \emph{Hotel}|pw}?\pend
           
\pstart
           Grüßen Sie Hugo\pwindex{Hofmannsthal, Hugo von 1.\,2.\,1874 Wien – 15.\,7.\,1929 Rodaun@\textsc{Hofmannsthal, Hugo von} (1.\,2.\,1874 Wien – 15.\,7.\,1929 Rodaun), \emph{Schriftsteller}|pw}, zeigen Sie ihm aber nicht den
               Brief, er macht mir sonst Vorwürfe daß zuviel »Tatsächliches« {\pb}drinnen steht. Salten\pwindex{Salten, Felix 6.\,9.\,1869 Budapest – 8.\,10.\,1945 Zürich@\textsc{Salten, Felix} (6.\,9.\,1869 Budapest – 8.\,10.\,1945 Zürich), \emph{Schriftsteller, Journalist, Chefredakteur}|pw} auch.\pend
           
\pstart
           Herzlichst{\\[\baselineskip]}Ihr \spacefill\mbox{Richard}\pend
           \leftskip=0em{}
\pstart
           Ischl\oindex{Bad Ischl@\textbf{Bad Ischl}|pw}{ }30/VI 94\pend
           
\pstart
           Ich freu mich aufs Siegeln\pend
           \selectlanguage{ngerman}\endnumbering\briefempfaengerindex{Schnitzler, Arthur@\textsc{Schnitzler, Arthur}!zzzBeer-Hofmann, Richard@\emph{von Richard Beer-Hofmann}!1894-06-301@{30. 6. 1894}|)be}\mylabel{L00342h}  \newcommand{\dateiname}{L00342}\newcommand{\titel}{Richard Beer-Hofmann an Arthur Schnitzler, 30. 6. 1894}\newcommand{\editorInnen}{Herausgegeben von Martin Anton Müller}%% latex-leseansicht-abspann.tex
%% Abspann für die Leseansicht.
%% Der Schalter \ifkorrekturansicht ist bereits durch den Vorspann gesetzt.

%% latex-abspann.tex
%% Gemeinsamer Abspann für Korrekturansicht und Leseansicht.
%% Setzt den Schalter \ifkorrekturansicht voraus (gesetzt in den
%% einbindenden Dateien latex-korrekturansicht-abspann.tex bzw.
%% latex-leseansicht-abspann.tex).
%% ---------------------------------------------------------------

\normalsize

% Das esempio-Environment wird nur in der Leseansicht benötigt
\ifkorrekturansicht\else
\newenvironment{esempio}[3]%
{
    \vspace{1.5ex}
    \rlap{\underline{#1}}
    \par
    \setlength{\parindent}{0cm}
    \nopagebreak
    \leftskip=#2cm
    \rightskip=#3cm
}
{
    \par
}
\fi

\doendnotes{C}
\bigskip
\vfill

\clearpage

\footnotesize

\ifkorrekturansicht
  \lohead{\textsc{register}}
\fi

% theindex-Environment neu definieren ohne reledmac
\makeatletter
\renewenvironment{theindex}{%
  \ifkorrekturansicht
    \section*{\indexname}%
  \else
    \subsubsection*{Index der erwähnten Entitäten}%
  \fi
  \setlength{\parindent}{0pt}%
  \setlength{\parskip}{0pt plus 0.3pt}%
  \let\item\@idxitem
}{%
  \ifkorrekturansicht\clearpage\fi
}
\makeatother

\IfFileExists{\jobname-pw.ind}{\input{\jobname-pw.ind}}{}

% Quellenangabe nur in der Leseansicht
\ifkorrekturansicht\else
% Fallback-Definitionen, falls die .tex-Datei \titel etc. nicht gesetzt hat
\providecommand{\titel}{}
\providecommand{\editorInnen}{}
\providecommand{\dateiname}{\jobname}

\vspace{3cm}

\vfill

\footnotesize
\textsc{Quelle}: \titel. Herausgegeben von {\editorInnen}. In: \emph{Arthur Schnitzler: Briefwechsel mit Autorinnen und Autoren}.
 Digitale Edition, https://schnitzler-briefe.acdh.oeaw.ac.at/{\dateiname}.html (Stand \today)
\fi

\end{document}


