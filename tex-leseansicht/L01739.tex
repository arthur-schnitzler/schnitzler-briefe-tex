%% latex-leseansicht-vorspann.tex
%% Vorspann für die Leseansicht.
%% Lädt die gemeinsame Datei latex-vorspann.tex mit nicht gesetztem Schalter.

\newif\ifkorrekturansicht
\korrekturansichtfalse

\input{../tex-inputs/latex-vorspann}


         
         \newcommand{\erwaehntePersonen}{Personen: Paula Beer-Hofmann, Jacob Pollak, Louise Schnitzler}
         \newcommand{\erwaehnteOrte}{Orte: Wien}
         \newcommand{\erwaehnteWerke}{
               \section[Richard Beer-Hofmann an Arthur Schnitzler, {[}11. 12. 1907{]}]{ Richard Beer-Hofmann an Arthur Schnitzler, {[}11. 12. 1907{]}}\nopagebreak\mylabel{v}\rehead{ }\begin{ledgroupsized}[t]{13cm}\normalsize\beginnumbering \toendnotes[C]{\smallbreak\pagebreak[2]} \Standort{CUL, Schnitzler, B 8.}
\physDesc{Brief, 2 Blätter (Briefpapier mit Trauerrand. Die Blätter nummeriert: »I« beziehungsweise »II«), 6 Seiten
\newline{}Handschrift: Bleistift, lateinische Kurrent
\newline{}Schnitzler: mit Bleistift datiert: »11/\textcolor{gray}{1}2 907« \newline{}Ordnung: mit Bleistift von unbekannter Hand nummeriert:
                                    »161?« }\buchAbdrucke{\weitereDrucke{Arthur Schnitzler, Richard Beer-Hofmann: \emph{Briefwechsel 1891–1931}. Hg. Konstanze Fliedl. Wien, Zürich: \emph{Europaverlag} 1992, S. 187–188.} }\toendnotes[C]{\smallbreak}\pstart
           \raggedleft{}{\pb}Mittwoch\pend
           \pstart
           Lieber Arthur! Paula\pwindex{Beer-Hofmann, Paula 25.02.1879 – 30.10.1939@\textsc{Beer-Hofmann, Paula} (25.02.1879 – 30.10.1939)|pw} hat vorgestern bei Ihrer Mama\pwindex{Schnitzler, Louise 1840-07-08 – 1911-09-09@\textsc{Schnitzler, Louise} (1840-07-08 – 1911-09-09)|pwv} angefragt und – (allerdings nicht direkt
               durch Ihre Mama\pwindex{Schnitzler, Louise 1840-07-08 – 1911-09-09@\textsc{Schnitzler, Louise} (1840-07-08 – 1911-09-09)|pwv}) erfahren,
               das es ein leichter Scharlachfall ist, und daß das Fieber zurückgeht.
               Heute habe ich telephonisch mit Ihrer Mama\pwindex{Schnitzler, Louise 1840-07-08 – 1911-09-09@\textsc{Schnitzler, Louise} (1840-07-08 – 1911-09-09)|pwv} selbst gesprochen, und erfahren daß \uuline{S}ie selbst beunruhigt sind weil trotz des Zurückgehen des
               Fiebers noch i{\geminationm}er Delieriren vorhanden ist. Ihre Mama\pwindex{Schnitzler, Louise 1840-07-08 – 1911-09-09@\textsc{Schnitzler, Louise} (1840-07-08 – 1911-09-09)|pwv} versichert {\pb}mich, daß der behandelnde Arzt\pwindex{Pollak, Jacob 07.02.1860 – 25.03.1941@\textsc{Pollak, Jacob} (07.02.1860 – 25.03.1941), \emph{Mediziner}|pwv} erklärt hat, daß trotz
               dieser unangenehmen Begleiterscheinung \uline{kein} Anlass zu
               Besorgnis ist. Ich schreibe Ihnen dies Alles, weil ich {\pb}nicht weiss ob Sie dem Arzt\pwindex{Pollak, Jacob 07.02.1860 – 25.03.1941@\textsc{Pollak, Jacob} (07.02.1860 – 25.03.1941), \emph{Mediziner}|pwv} glauben. \uline{Mir} gegenüber ist kein Grund zum Schönfärben
               vorhanden. Ich wünsche vor Allem von ganzem Herzen daß es bald und rasch {\pb}besser geht, dann daß Sie sich
               nicht in nutzlosem Schwarzsehen \strikeout{sich} verzehren. Ich
               weiss ich weiss – ich habe leicht reden – aber vielleicht beruhigt es Sie doch ein
                  \uline{ganz klein} wenig, daß man mir den
               Krankheitsverlauf, als unangenehm, – als überflüssig complicirt – aber \uline{nicht} als gefährlich dargestellt hat.\pend
           \pstart
           {\pb}Nur darum schreib ich Ihnen, und
               weil ich denke, daß mitten unter Wichtigerem, dies kleine unwichtige – dass ich und
                  Paula\pwindex{Beer-Hofmann, Paula 25.02.1879 – 30.10.1939@\textsc{Beer-Hofmann, Paula} (25.02.1879 – 30.10.1939)|pw}{ }{\pb}oft im Tage an Sie Beide denken,
               und starke und gute Wünsche für Sie im Herzen haben – weil es vielleicht doch für
               eine Sekunde Ihnen angenehm sein könnte.\pend
           \pstart
           Von Herzen wie immer{\\[\baselineskip]}Ihr \spacefill\mbox{Richard}\pend
           \leftskip=0em{}
         
         \endnumbering\mylabel{h}\end{ledgroupsized}  \newcommand{\dateiname}{L01739}\newcommand{\titel}{Richard Beer-Hofmann an Arthur Schnitzler, [11. 12. 1907]}\newcommand{\editorInnen}{Martin Anton Müller und Gerd-Hermann Susen}%% latex-leseansicht-abspann.tex
%% Abspann für die Leseansicht.
%% Der Schalter \ifkorrekturansicht ist bereits durch den Vorspann gesetzt.

%% latex-abspann.tex
%% Gemeinsamer Abspann für Korrekturansicht und Leseansicht.
%% Setzt den Schalter \ifkorrekturansicht voraus (gesetzt in den
%% einbindenden Dateien latex-korrekturansicht-abspann.tex bzw.
%% latex-leseansicht-abspann.tex).
%% ---------------------------------------------------------------

\normalsize

% Das esempio-Environment wird nur in der Leseansicht benötigt
\ifkorrekturansicht\else
\newenvironment{esempio}[3]%
{
    \vspace{1.5ex}
    \rlap{\underline{#1}}
    \par
    \setlength{\parindent}{0cm}
    \nopagebreak
    \leftskip=#2cm
    \rightskip=#3cm
}
{
    \par
}
\fi

\doendnotes{C}
\bigskip
\vfill

\clearpage

\footnotesize

\ifkorrekturansicht
  \lohead{\textsc{register}}
\fi

% theindex-Environment neu definieren ohne reledmac
\makeatletter
\renewenvironment{theindex}{%
  \ifkorrekturansicht
    \section*{\indexname}%
  \else
    \subsubsection*{Index der erwähnten Entitäten}%
  \fi
  \setlength{\parindent}{0pt}%
  \setlength{\parskip}{0pt plus 0.3pt}%
  \let\item\@idxitem
}{%
  \ifkorrekturansicht\clearpage\fi
}
\makeatother

\IfFileExists{\jobname-pw.ind}{\input{\jobname-pw.ind}}{}

% Quellenangabe nur in der Leseansicht
\ifkorrekturansicht\else
% Fallback-Definitionen, falls die .tex-Datei \titel etc. nicht gesetzt hat
\providecommand{\titel}{}
\providecommand{\editorInnen}{}
\providecommand{\dateiname}{\jobname}

\vspace{3cm}

\vfill

\footnotesize
\textsc{Quelle}: \titel. Herausgegeben von {\editorInnen}. In: \emph{Arthur Schnitzler: Briefwechsel mit Autorinnen und Autoren}.
 Digitale Edition, https://schnitzler-briefe.acdh.oeaw.ac.at/{\dateiname}.html (Stand \today)
\fi

\end{document}


      