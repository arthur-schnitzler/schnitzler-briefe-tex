%% latex-korrekturansicht-vorspann.tex
%% Vorspann für die Korrekturansicht.
%% Lädt die gemeinsame Datei latex-vorspann.tex mit gesetztem Schalter.

\newif\ifkorrekturansicht
\korrekturansichttrue

\input{../tex-inputs/latex-vorspann}


\section[Richard Beer-Hofmann an Arthur Schnitzler, {[}11. 12. 1907{]}]{L01739 Richard Beer-Hofmann an Arthur Schnitzler, {[}11. 12. 1907{]}}
\nopagebreak\mylabel{L01739v}
\rehead{ }\normalsize\beginnumbering\briefempfaengerindex{Schnitzler, Arthur@\textsc{Schnitzler, Arthur}!zzzBeer-Hofmann, Richard@\emph{von Richard Beer-Hofmann}!1907-12-111@{{[}11. 12. 1907{]}}|(be}
\toendnotes[C]{\smallbreak\pagebreak[2]}\Standort{CUL, Schnitzler, B 8.}
\physDesc{Brief, 2 Blätter, 6 Seiten, 1305 Zeichen (Briefpapier mit Trauerrand)
\newline{}Handschrift: Bleistift, lateinische Kurrent (\noindent{}die Blätter nummeriert: »I«
                                   beziehungsweise »II«)
\newline{}Schnitzler: mit Bleistift datiert: »11/\textcolor{gray}{1}2 907« 
\newline{}Ordnung: mit Bleistift von unbekannter Hand nummeriert:
                                    »161?« }
\buchAbdrucke{\weitereDrucke{Arthur Schnitzler, Richard Beer-Hofmann: \emph{Briefwechsel 1891–1931}. Wien, Zürich: \emph{Europaverlag} 1992, S. 187–188.} }\toendnotes[C]{\smallbreak}
\pstart
           \raggedleft{}{\pb}Mittwoch\pend
           \vspace{0.5em}
\pstart
           Lieber Arthur!{ }Paula\pwindex{Beer-Hofmann, Paula 25.02.1879 – 30.10.1939@\textsc{Beer-Hofmann, Paula} (25.02.1879 – 30.10.1939)|pw} hat vorgestern bei Ihrer Mama\pwindex{Schnitzler, Louise 1840-07-08 – 1911-09-09@\textsc{Schnitzler, Louise} (1840-07-08 – 1911-09-09)|pwv} angefragt und –
               (allerdings nicht direkt durch Ihre Mama\pwindex{Schnitzler, Louise 1840-07-08 – 1911-09-09@\textsc{Schnitzler, Louise} (1840-07-08 – 1911-09-09)|pwv}) erfahren, das es ein leichter
               Scharlachfall ist, und daß das Fieber zurückgeht. Heute habe ich telephonisch mit
               Ihrer Mama\pwindex{Schnitzler, Louise 1840-07-08 – 1911-09-09@\textsc{Schnitzler, Louise} (1840-07-08 – 1911-09-09)|pwv} selbst
               gesprochen, und erfahren daß \uuline{S}ie selbst beunruhigt
               sind weil trotz des Zurückgehen des Fiebers noch i{\geminationm}er
               Delieriren vorhanden ist. Ihre Mama\pwindex{Schnitzler, Louise 1840-07-08 – 1911-09-09@\textsc{Schnitzler, Louise} (1840-07-08 – 1911-09-09)|pwv} versichert {\pb}mich, daß
               der behandelnde Arzt\pwindex{Pollak, Jacob 07.02.1860 – 25.03.1941@\textsc{Pollak, Jacob} (07.02.1860 – 25.03.1941), \emph{Mediziner/Medizinerin}|pwv} erklärt
               hat, daß trotz dieser unangenehmen Begleiterscheinung \uline{kein} Anlass zu Besorgnis ist. Ich schreibe Ihnen dies Alles, weil ich {\pb}nicht weiss ob Sie dem Arzt\pwindex{Pollak, Jacob 07.02.1860 – 25.03.1941@\textsc{Pollak, Jacob} (07.02.1860 – 25.03.1941), \emph{Mediziner/Medizinerin}|pwv} glauben. \uline{Mir} gegenüber ist kein Grund zum Schönfärben
               vorhanden. Ich wünsche vor Allem von ganzem Herzen daß es bald und rasch {\pb}besser geht, dann daß Sie sich
               nicht in nutzlosem Schwarzsehen \strikeout{sich} verzehren. Ich
               weiss ich weiss – ich habe leicht reden – aber vielleicht beruhigt es Sie doch ein
                  \uline{ganz klein} wenig, daß man mir den
               Krankheitsverlauf, als unangenehm, – als überflüssig complicirt – aber \uline{nicht} als gefährlich dargestellt hat.\pend
           
\pstart
           {\pb}Nur darum schreib ich Ihnen, und
               weil ich denke, daß mitten unter Wichtigerem, dies kleine unwichtige – dass ich und
                  Paula\pwindex{Beer-Hofmann, Paula 25.02.1879 – 30.10.1939@\textsc{Beer-Hofmann, Paula} (25.02.1879 – 30.10.1939)|pw}{ }{\pb}oft im Tage an Sie Beide denken,
               und starke und gute Wünsche für Sie im Herzen haben – weil es vielleicht doch für
               eine Sekunde Ihnen angenehm sein könnte.\pend
           
\pstart
           Von Herzen wie immer{\\[\baselineskip]}Ihr \spacefill\mbox{Richard}\pend
           \leftskip=0em{}\selectlanguage{ngerman}\endnumbering\briefempfaengerindex{Schnitzler, Arthur@\textsc{Schnitzler, Arthur}!zzzBeer-Hofmann, Richard@\emph{von Richard Beer-Hofmann}!1907-12-111@{{[}11. 12. 1907{]}}|)be}\mylabel{L01739h}  \normalsize

\doendnotes{C}
\bigskip
\vfill

\clearpage

\footnotesize

\lohead{\textsc{register}}

% Definiere theindex-Environment komplett neu ohne reledmac
\makeatletter
\renewenvironment{theindex}{%
  \section*{\indexname}%
  \setlength{\parindent}{0pt}%
  \setlength{\parskip}{0pt plus 0.3pt}%
  \let\item\@idxitem
}{%
  \clearpage
}
\makeatother

\IfFileExists{\jobname-pw.ind}{\input{\jobname-pw.ind}}{}

\end{document}

      