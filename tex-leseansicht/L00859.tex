%% latex-korrekturansicht-vorspann.tex
%% Vorspann für die Korrekturansicht.
%% Lädt die gemeinsame Datei latex-vorspann.tex mit gesetztem Schalter.

\newif\ifkorrekturansicht
\korrekturansichttrue

\input{../tex-inputs/latex-vorspann}


\section[Hugo von Hofmannsthal an Arthur Schnitzler, 19. 11. 1898]{L00859 Hugo von Hofmannsthal an Arthur Schnitzler, 19. 11. 1898}
\nopagebreak\mylabel{L00859v}
\rehead{ }\normalsize\beginnumbering\briefempfaengerindex{Schnitzler, Arthur@\textsc{Schnitzler, Arthur}!zzzHofmannsthal, Hugo von@\emph{von Hugo von Hofmannsthal}!1898-11-191@{19.  11. 1898}|(be}
\toendnotes[C]{\smallbreak\pagebreak[2]}\Standort{CUL, Schnitzler, B 43.}
\physDesc{Postkarte, 228 Zeichen
\newline{}Handschrift: Bleistift, deutsche Kurrent
\newline{}Versand: 1) Rohrpost  2) Stempel: »\nobreak{}\oindex{I., Innere Stadt@\textbf{I., Innere Stadt}, \emph{A.ADM3}|pwk}Wien 1/1, 19 XI 98, 3 50N\nobreak{}«.  3) Stempel: »\nobreak{}\oindex{IX., Alsergrund@\textbf{IX., Alsergrund}, \emph{A.ADM3}|pwk}Wien 9/2, 19 XI 98, 4 30N\nobreak{}«. 
\newline{}Ordnung: 1) mit Bleistift von unbekannter Hand nummeriert: »\strikeout{129}«  2) mit Bleistift von unbekannter Hand nummeriert:
                                    »126«}
\buchAbdrucke{\weitereDrucke{Hugo von Hofmannsthal, Arthur Schnitzler: \emph{Briefwechsel}. Frankfurt am Main: \emph{S. Fischer} 1964, S. 75.} }\toendnotes[C]{\smallbreak}\pstart{}{\pb}\textsc{Herrn D\textsuperscript{r} Arthur Schnitzler}\pend{}\pstart{}IX Franckgasse 1\oindex{Frankgasse 1@\textbf{Frankgasse 1}, \emph{Wohngebäude (K.WHS)}|pw}\pend{}\pstart{}Wien\oindex{Wien@\textbf{Wien}, \emph{A.ADM2}|pw}\pend{}{\bigskip}\vspace{1em}
\pstart
           \noindent{}{\pb}Wir beide ſollen heute Abend nach
               dem \label{K_L00859-1v}\edtext{Theater\oindex{Carl-Theater@\textbf{Carl-Theater}, \emph{Theater (K.THE)}|pwv}}{\lemma{\textnormal{\emph{Theater}}}\Cendnote{\textnormal{\emph{Die blonde Kathrein}\pwindex{blonde Kathrein. Ein Maerchenspiel@\emph{Die blonde Kathrein. Ein Märchenspiel}|pwk} von Richard Voß\pwindex{Voss, Richard 02.09.1851 – 10.06.1918@\textsc{Voss, Richard} (02.09.1851 – 10.06.1918), \emph{Schriftsteller/Schriftstellerin}|pwk} nach Hans
                     Christian Andersen\pwindex{Andersen, Hans Christian 02.04.1805 – 04.08.1875@\textsc{Andersen, Hans Christian} (02.04.1805 – 04.08.1875), \emph{Schriftsteller/Schriftstellerin}|pwk}, zum ersten Mal am Carl-Theater\oindex{Carl-Theater@\textbf{Carl-Theater}, \emph{Theater (K.THE)}|pwk}.}}}\label{K_L00859-1} mit Brahm\pwindex{Brahm, Otto 05.02.1856 – 28.11.1912@\textsc{Brahm, Otto} (05.02.1856 – 28.11.1912), \emph{Theaterleiter/Theaterleiterin, Regisseur/Regisseurin}|pw} im
                  »ſilbernen Brunnen\oindex{Zum silbernen Brunnen@\textbf{Zum silbernen Brunnen}, \emph{Gastgewerbegebäude (K.GGW)}|pw}« ſein. Bitte umgehende
               Antwort ob ich Sie nicht ſchon früher wo anders treffen oder abholen kann.\pend
           
\pstart
           Herzlich{\\[\baselineskip]}\spacefill\mbox{Hugo}\pend
           \leftskip=0em{}\selectlanguage{ngerman}\endnumbering\briefempfaengerindex{Schnitzler, Arthur@\textsc{Schnitzler, Arthur}!zzzHofmannsthal, Hugo von@\emph{von Hugo von Hofmannsthal}!1898-11-191@{19.  11. 1898}|)be}\mylabel{L00859h}  \normalsize

\doendnotes{C}
\bigskip
\vfill

\clearpage

\footnotesize

\lohead{\textsc{register}}

% Definiere theindex-Environment komplett neu ohne reledmac
\makeatletter
\renewenvironment{theindex}{%
  \section*{\indexname}%
  \setlength{\parindent}{0pt}%
  \setlength{\parskip}{0pt plus 0.3pt}%
  \let\item\@idxitem
}{%
  \clearpage
}
\makeatother

\IfFileExists{\jobname-pw.ind}{\input{\jobname-pw.ind}}{}

\end{document}

      