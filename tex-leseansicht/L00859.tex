%% latex-leseansicht-vorspann.tex
%% Vorspann für die Leseansicht.
%% Lädt die gemeinsame Datei latex-vorspann.tex mit nicht gesetztem Schalter.

\newif\ifkorrekturansicht
\korrekturansichtfalse

\input{../tex-inputs/latex-vorspann}


\section[Hugo von Hofmannsthal an Arthur Schnitzler, 19. 11. 1898]{L00859 Hugo von Hofmannsthal an Arthur Schnitzler, 19. 11. 1898}
\nopagebreak\mylabel{L00859v}
\rehead{ }\normalsize\beginnumbering\briefempfaengerindex{Schnitzler, Arthur@\textsc{Schnitzler, Arthur}!zzzHofmannsthal, Hugo von@\emph{von Hugo von Hofmannsthal}!1898-11-191@{19.  11. 1898}|(be}
\toendnotes[C]{\smallbreak\pagebreak[2]}
\correspDesc{Versand  durch Hugo von Hofmannsthal am 19.  11. 1898 in Wien
\newline{}Erhalt  durch Arthur Schnitzler am 19. 11. 1898 in Wien}\toendnotes[C]{\smallbreak}
\Standort{CUL, Schnitzler, B 43.}
\physDesc{Postkarte, 228 Zeichen
\newline{}Handschrift: Bleistift, deutsche Kurrent
\newline{}Versand: 1) Rohrpost  2) Stempel: »\nobreak{}\oindex{I., Innere Stadt@\textbf{I., Innere Stadt}, \emph{Verwaltungsgebiet}|pwk}Wien 1/1, 19 XI 98, 3 50N\nobreak{}«.  3) Stempel: »\nobreak{}\oindex{IX., Alsergrund@\textbf{IX., Alsergrund}, \emph{Verwaltungsgebiet}|pwk}Wien 9/2, 19 XI 98, 4 30N\nobreak{}«. 
\newline{}Ordnung: 1) mit Bleistift von unbekannter Hand nummeriert: »\strikeout{129}«  2) mit Bleistift von unbekannter Hand nummeriert:
                                    »126«}
\buchAbdrucke{\weitereDrucke{Hugo von Hofmannsthal, Arthur Schnitzler: \emph{Briefwechsel}. Herausgegeben von Therese Nickl und Heinrich Schnitzler. Frankfurt am Main: \emph{S. Fischer} 1964, S. 75.} }\toendnotes[C]{\smallbreak}\pstart{}{\pb}\textsc{Herrn D\textsuperscript{r} Arthur Schnitzler}\pend{}\pstart{}IX Franckgasse 1\oindex{Wien@\textbf{Wien}!IX., Alsergrund@\textbf{IX., Alsergrund}!Frankgasse 1@\textbf{Frankgasse 1}, \emph{Wohngebäude}|pw}\pend{}\pstart{}Wien\oindex{Wien@\textbf{Wien}, \emph{Verwaltungsgebiet}|pw}\pend{}{\bigskip}\vspace{1em}
\pstart
           \noindent{}{\pb}Wir beide{ }ſollen heute Abend nach
               dem \label{K_L00859-1v}\edtext{Theater\oindex{Wien@\textbf{Wien}!II., Leopoldstadt@\textbf{II., Leopoldstadt}!Carl-Theater@\textbf{Carl-Theater}, \emph{Theater}|pwv}}{\lemma{\textnormal{\emph{Theater}}}\Cendnote{\textnormal{\emph{Die blonde Kathrein}\pwindex{Voss, Richard 2.\,9.\,1851 Nowe Chrapowo – 10.\,6.\,1918 Berchtesgaden@\textsc{Voss, Richard} (2.\,9.\,1851 Nowe Chrapowo – 10.\,6.\,1918 Berchtesgaden), \emph{Schriftsteller}!blonde Kathrein. Ein Märchenspiel@\strich\emph{Die blonde Kathrein. Ein Märchenspiel}|pwk} von Richard Voß\pwindex{Voss, Richard 2.\,9.\,1851 Nowe Chrapowo – 10.\,6.\,1918 Berchtesgaden@\textsc{Voss, Richard} (2.\,9.\,1851 Nowe Chrapowo – 10.\,6.\,1918 Berchtesgaden), \emph{Schriftsteller}|pwk} nach Hans
                     Christian Andersen\pwindex{Andersen, Hans Christian 2.\,4.\,1805 Odense – 4.\,8.\,1875 Kopenhagen@\textsc{Andersen, Hans Christian} (2.\,4.\,1805 Odense – 4.\,8.\,1875 Kopenhagen), \emph{Schriftsteller}|pwk}, zum ersten Mal am Carl-Theater\oindex{Wien@\textbf{Wien}!II., Leopoldstadt@\textbf{II., Leopoldstadt}!Carl-Theater@\textbf{Carl-Theater}, \emph{Theater}|pwk}.}}}\label{K_L00859-1} mit Brahm\pwindex{Brahm, Otto 5.\,2.\,1856 Hamburg – 28.\,11.\,1912 Berlin@\textsc{Brahm, Otto} (5.\,2.\,1856 Hamburg – 28.\,11.\,1912 Berlin), \emph{Theaterleiter, Regisseur}|pw} im
                  »ſilbernen Brunnen\oindex{Wien@\textbf{Wien}!IX., Alsergrund@\textbf{IX., Alsergrund}!Zum silbernen Brunnen@\textbf{Zum silbernen Brunnen}, \emph{Gastgewerbegebäude}|pw}«{ }ſein. Bitte umgehende
               Antwort ob ich Sie nicht{ }ſchon früher wo anders treffen oder abholen kann.\pend
           
\pstart
           Herzlich{\\[\baselineskip]}\spacefill\mbox{Hugo}\pend
           \leftskip=0em{}\selectlanguage{ngerman}\endnumbering\briefempfaengerindex{Schnitzler, Arthur@\textsc{Schnitzler, Arthur}!zzzHofmannsthal, Hugo von@\emph{von Hugo von Hofmannsthal}!1898-11-191@{19.  11. 1898}|)be}\mylabel{L00859h}  \newcommand{\dateiname}{L00859}\newcommand{\titel}{Hugo von Hofmannsthal an Arthur Schnitzler, 19. 11. 1898}\newcommand{\editorInnen}{Martin Anton Müller und Gerd-Hermann Susen}%% latex-leseansicht-abspann.tex
%% Abspann für die Leseansicht.
%% Der Schalter \ifkorrekturansicht ist bereits durch den Vorspann gesetzt.

%% latex-abspann.tex
%% Gemeinsamer Abspann für Korrekturansicht und Leseansicht.
%% Setzt den Schalter \ifkorrekturansicht voraus (gesetzt in den
%% einbindenden Dateien latex-korrekturansicht-abspann.tex bzw.
%% latex-leseansicht-abspann.tex).
%% ---------------------------------------------------------------

\normalsize

% Das esempio-Environment wird nur in der Leseansicht benötigt
\ifkorrekturansicht\else
\newenvironment{esempio}[3]%
{
    \vspace{1.5ex}
    \rlap{\underline{#1}}
    \par
    \setlength{\parindent}{0cm}
    \nopagebreak
    \leftskip=#2cm
    \rightskip=#3cm
}
{
    \par
}
\fi

\doendnotes{C}
\bigskip
\vfill

\clearpage

\footnotesize

\ifkorrekturansicht
  \lohead{\textsc{register}}
\fi

% theindex-Environment neu definieren ohne reledmac
\makeatletter
\renewenvironment{theindex}{%
  \ifkorrekturansicht
    \section*{\indexname}%
  \else
    \subsubsection*{Index der erwähnten Entitäten}%
  \fi
  \setlength{\parindent}{0pt}%
  \setlength{\parskip}{0pt plus 0.3pt}%
  \let\item\@idxitem
}{%
  \ifkorrekturansicht\clearpage\fi
}
\makeatother

\IfFileExists{\jobname-pw.ind}{\input{\jobname-pw.ind}}{}

% Quellenangabe nur in der Leseansicht
\ifkorrekturansicht\else
% Fallback-Definitionen, falls die .tex-Datei \titel etc. nicht gesetzt hat
\providecommand{\titel}{}
\providecommand{\editorInnen}{}
\providecommand{\dateiname}{\jobname}

\vspace{3cm}

\vfill

\footnotesize
\textsc{Quelle}: \titel. Herausgegeben von {\editorInnen}. In: \emph{Arthur Schnitzler: Briefwechsel mit Autorinnen und Autoren}.
 Digitale Edition, https://schnitzler-briefe.acdh.oeaw.ac.at/{\dateiname}.html (Stand \today)
\fi

\end{document}


