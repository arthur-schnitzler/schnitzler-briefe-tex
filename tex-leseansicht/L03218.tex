%% latex-leseansicht-vorspann.tex
%% Vorspann für die Leseansicht.
%% Lädt die gemeinsame Datei latex-vorspann.tex mit nicht gesetztem Schalter.

\newif\ifkorrekturansicht
\korrekturansichtfalse

\input{../tex-inputs/latex-vorspann}


\section[ Paul Goldmann an Arthur Schnitzler, 9. 8. 1902]{L03218 Paul Goldmann an Arthur Schnitzler,  9. 8. 1902}
\nopagebreak\mylabel{L03218v}
\rehead{ }\normalsize\beginnumbering\briefempfaengerindex{Schnitzler, Arthur@\textsc{Schnitzler, Arthur}!zzzGoldmann, Paul@\emph{von Paul Goldmann}!1902-08-091@{9. 8. 1902}|(be}
\toendnotes[C]{\smallbreak\pagebreak[2]}
\correspDesc{Versand  durch Paul Goldmann am 9. 8. 1902 in Mürren
\newline{}Erhalt  durch Arthur Schnitzler am 12. 8. 1902 in Wien}\toendnotes[C]{\smallbreak}
\Standort{DLA, A:Schnitzler, HS.NZ85.1.3172.}
\physDesc{Bildpostkarte, 120 Zeichen
\newline{}Handschrift: schwarze Tinte, deutsche Kurrent
\newline{}Versand: 1) Stempel: »\nobreak{}\oindex{Mürren@\textbf{Mürren}|pwk}Mürren, 9. VIII. 02., 7\nobreak{}«.   2) Stempel: »\nobreak{}\oindex{IX., Alsergrund@\textbf{IX., Alsergrund}, \emph{Verwaltungsgebiet}|pwk}9/3 Wien 72, 12. 8. 02, 8{[}.{]} V, Bestellt\nobreak{}«. }\toendnotes[C]{\smallbreak}\pstart{}\textsc{{\pb}Herrn}\pend{}\pstart{}\textsc{Dr. Arthur Schnitzler}\pend{}\pstart{}\textsc{Wien\oindex{Wien@\textbf{Wien}, \emph{Verwaltungsgebiet}|pw}}\pend{}\pstart{}\textsc{IX. Frankgaſse 1\oindex{Wien@\textbf{Wien}!IX., Alsergrund@\textbf{IX., Alsergrund}!Frankgasse 1@\textbf{Frankgasse 1}, \emph{Wohngebäude}|pw}.}\pend{}{\bigskip}
\pstart
           \noindent{}\centering{}{\pb}\textcolor{gray}{\textbf{Mürren\oindex{Mürren@\textbf{Mürren}|pw} – Station gegen das Mittagshorn\oindex{Station gegen das Mittaghorn@\textbf{Station gegen das Mittaghorn}, \emph{Bahnhofsgebäude}|pw}}}\pend
           \vspace{1em}
\pstart
           \noindent{}{\pb}\label{K_L03218-1v}\edtext{\textsc{Dr. Rathenau\pwindex{Rathenau, Walther 29.\,9.\,1867 Berlin – 24.\,6.\,1922 ebd.@\textsc{Rathenau, Walther} (29.\,9.\,1867 Berlin – 24.\,6.\,1922 ebd.), \emph{Politiker, Industrieller}|pw}}}{\lemma{\textnormal{\emph{Dr. Rathenau}}}\Cendnote{\textnormal{Möglicherweise im Zusammenhang mit Goldmanns\pwindex{Goldmann, Paul 31.\,1.\,1865 Breslau – 25.\,9.\,1935 Wien@\textsc{Goldmann, Paul} (31.\,1.\,1865 Breslau – 25.\,9.\,1935 Wien), \emph{Schriftsteller, Journalist}|pwk} Empfehlung, Walther Rathenaus\pwindex{Rathenau, Walther 29.\,9.\,1867 Berlin – 24.\,6.\,1922 ebd.@\textsc{Rathenau, Walther} (29.\,9.\,1867 Berlin – 24.\,6.\,1922 ebd.), \emph{Politiker, Industrieller}|pwk}{ }\emph{Impressionen}\pwindex{Rathenau, Walther 29.\,9.\,1867 Berlin – 24.\,6.\,1922 ebd.@\textsc{Rathenau, Walther} (29.\,9.\,1867 Berlin – 24.\,6.\,1922 ebd.), \emph{Politiker, Industrieller}!Impressionen@\strich\emph{Impressionen}|pwk} zu lesen (vgl. XXXX Auszeichnungsfehler: Dokument L03214 nicht gefunden)? Walther Rathenau\pwindex{Rathenau, Walther 29.\,9.\,1867 Berlin – 24.\,6.\,1922 ebd.@\textsc{Rathenau, Walther} (29.\,9.\,1867 Berlin – 24.\,6.\,1922 ebd.), \emph{Politiker, Industrieller}|pwk} war zu
                  diesem Zeitpunkt noch für die \emph{AEG}\orgindex{Allgemeine Elektricitäts-Gesellschaft@Allgemeine Elektricitäts-Gesellschaft|pwk}
                  tätig.}}}\label{K_L03218-1}, \textsc{Berlin W.\oindex{Berlin@\textbf{Berlin}, \emph{Hauptstadt}|pw}}, \textsc{Victoriastraſse 3}\oindex{Viktoriastraße@\textbf{Viktoriastraße}, \emph{Straße}|pw}. Tauſend Grüße!\pend
           
\pstart
           Dein {\\[\baselineskip]}\spacefill\mbox{Paul Goldmann\textcolor{gray}{.}}\pend
           \leftskip=0em{}\selectlanguage{ngerman}\endnumbering\briefempfaengerindex{Schnitzler, Arthur@\textsc{Schnitzler, Arthur}!zzzGoldmann, Paul@\emph{von Paul Goldmann}!1902-08-091@{9. 8. 1902}|)be}\mylabel{L03218h}  \newcommand{\dateiname}{L03218}\newcommand{\titel}{Paul Goldmann an Arthur Schnitzler, 9. 8. 1902}\newcommand{\editorInnen}{Martin Anton Müller und Laura Untner}%% latex-leseansicht-abspann.tex
%% Abspann für die Leseansicht.
%% Der Schalter \ifkorrekturansicht ist bereits durch den Vorspann gesetzt.

%% latex-abspann.tex
%% Gemeinsamer Abspann für Korrekturansicht und Leseansicht.
%% Setzt den Schalter \ifkorrekturansicht voraus (gesetzt in den
%% einbindenden Dateien latex-korrekturansicht-abspann.tex bzw.
%% latex-leseansicht-abspann.tex).
%% ---------------------------------------------------------------

\normalsize

% Das esempio-Environment wird nur in der Leseansicht benötigt
\ifkorrekturansicht\else
\newenvironment{esempio}[3]%
{
    \vspace{1.5ex}
    \rlap{\underline{#1}}
    \par
    \setlength{\parindent}{0cm}
    \nopagebreak
    \leftskip=#2cm
    \rightskip=#3cm
}
{
    \par
}
\fi

\doendnotes{C}
\bigskip
\vfill

\clearpage

\footnotesize

\ifkorrekturansicht
  \lohead{\textsc{register}}
\fi

% theindex-Environment neu definieren ohne reledmac
\makeatletter
\renewenvironment{theindex}{%
  \ifkorrekturansicht
    \section*{\indexname}%
  \else
    \subsubsection*{Index der erwähnten Entitäten}%
  \fi
  \setlength{\parindent}{0pt}%
  \setlength{\parskip}{0pt plus 0.3pt}%
  \let\item\@idxitem
}{%
  \ifkorrekturansicht\clearpage\fi
}
\makeatother

\IfFileExists{\jobname-pw.ind}{\input{\jobname-pw.ind}}{}

% Quellenangabe nur in der Leseansicht
\ifkorrekturansicht\else
% Fallback-Definitionen, falls die .tex-Datei \titel etc. nicht gesetzt hat
\providecommand{\titel}{}
\providecommand{\editorInnen}{}
\providecommand{\dateiname}{\jobname}

\vspace{3cm}

\vfill

\footnotesize
\textsc{Quelle}: \titel. Herausgegeben von {\editorInnen}. In: \emph{Arthur Schnitzler: Briefwechsel mit Autorinnen und Autoren}.
 Digitale Edition, https://schnitzler-briefe.acdh.oeaw.ac.at/{\dateiname}.html (Stand \today)
\fi

\end{document}


