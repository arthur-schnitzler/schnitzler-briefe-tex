%% latex-korrekturansicht-vorspann.tex
%% Vorspann für die Korrekturansicht.
%% Lädt die gemeinsame Datei latex-vorspann.tex mit gesetztem Schalter.

\newif\ifkorrekturansicht
\korrekturansichttrue

\input{../tex-inputs/latex-vorspann}


\section[ Paul Goldmann an Arthur Schnitzler, 9. 8. 1902]{L03218 Paul Goldmann an Arthur Schnitzler, 9. 8. 1902}
\nopagebreak\mylabel{L03218v}
\rehead{ }\normalsize\beginnumbering\briefempfaengerindex{Schnitzler, Arthur@\textsc{Schnitzler, Arthur}!zzzGoldmann, Paul@\emph{von Paul Goldmann}!1902-08-091@{9. 8. 1902}|(be}
\toendnotes[C]{\smallbreak\pagebreak[2]}\Standort{DLA, A:Schnitzler, HS.NZ85.1.3172.}
\physDesc{Bildpostkarte, 120 Zeichen
\newline{}Handschrift: 1) schwarze Tinte, deutsche Kurrent\hspace{1em}2) schwarze Tinte, lateinische Kurrent (\noindent{}Adresse)\hspace{1em}
\newline{}Versand: 1) Stempel: »\nobreak{}\oindex{Muerren@\textbf{Mürren}, \emph{P.PPL}|pwk}Mürren, 9. VIII. 02., 7\nobreak{}«.   2) Stempel: »\nobreak{}\oindex{IX., Alsergrund@\textbf{IX., Alsergrund}, \emph{A.ADM3}|pwk}9/3 Wien 72, 12. 8. 02, 8{[}.{]} V, Bestellt\nobreak{}«. }\toendnotes[C]{\smallbreak}\pstart{}{\pb}Herrn\pend{}\pstart{}Dr. Arthur Schnitzler\pend{}\pstart{}Wien\oindex{Wien@\textbf{Wien}, \emph{A.ADM2}|pw}\pend{}\pstart{}IX. Frankgaſse 1\oindex{Frankgasse 1@\textbf{Frankgasse 1}, \emph{Wohngebäude (K.WHS)}|pw}.\pend{}{\bigskip}
\pstart
           \noindent{}\centering{}{\pb}\textcolor{gray}{\textbf{Mürren\oindex{Muerren@\textbf{Mürren}, \emph{P.PPL}|pw} – Station gegen das Mittagshorn\oindex{Station gegen das Mittaghorn@\textbf{Station gegen das Mittaghorn}, \emph{Bahnhofsgebäude (K.BHF)}|pw}}}\pend
           \vspace{1em}
\pstart
           \noindent{}{\pb}\label{K_L03218-1v}\edtext{\textsc{Dr. Rathenau\pwindex{Rathenau, Walther 29.09.1867 – 24.06.1922@\textsc{Rathenau, Walther} (29.09.1867 – 24.06.1922), \emph{Politiker/Politikerin, Industrieller/Industrielle}|pw}}}{\lemma{\textnormal{\emph{Dr. Rathenau}}}\Cendnote{\textnormal{Möglicherweise im Zusammenhang mit Goldmanns\pwindex{Goldmann, Paul 31.01.1865 – 25.09.1935@\textsc{Goldmann, Paul} (31.01.1865 – 25.09.1935), \emph{Schriftsteller/Schriftstellerin, Journalist/Journalistin}|pwk} Empfehlung, Walther Rathenaus\pwindex{Rathenau, Walther 29.09.1867 – 24.06.1922@\textsc{Rathenau, Walther} (29.09.1867 – 24.06.1922), \emph{Politiker/Politikerin, Industrieller/Industrielle}|pwk}{ }\emph{Impressionen}\pwindex{Impressionen@\emph{Impressionen}|pwk} zu lesen (vgl. Paul Goldmann an Arthur Schnitzler, 25. 7. [1902])? Walther Rathenau\pwindex{Rathenau, Walther 29.09.1867 – 24.06.1922@\textsc{Rathenau, Walther} (29.09.1867 – 24.06.1922), \emph{Politiker/Politikerin, Industrieller/Industrielle}|pwk} war zu
                  diesem Zeitpunkt noch für die \emph{AEG}\orgindex{Allgemeine Elektricitaets-Gesellschaft@Allgemeine Elektricitäts-Gesellschaft|pwk}
                  tätig.}}}\label{K_L03218-1}, \textsc{Berlin W.\oindex{Berlin@\textbf{Berlin}, \emph{P.PPLC}|pw}}, \textsc{Victoriastraſse 3}\oindex{Viktoriastrasse@\textbf{Viktoriastraße}, \emph{Straße (K.STR)}|pw}. Tauſend Grüße!\pend
           
\pstart
           Dein {\\[\baselineskip]}\spacefill\mbox{Paul Goldmann\textcolor{gray}{.}}\pend
           \leftskip=0em{}\selectlanguage{ngerman}\endnumbering\briefempfaengerindex{Schnitzler, Arthur@\textsc{Schnitzler, Arthur}!zzzGoldmann, Paul@\emph{von Paul Goldmann}!1902-08-091@{9. 8. 1902}|)be}\mylabel{L03218h}  \normalsize

\doendnotes{C}
\bigskip
\vfill

\clearpage

\footnotesize

\lohead{\textsc{register}}

% Definiere theindex-Environment komplett neu ohne reledmac
\makeatletter
\renewenvironment{theindex}{%
  \section*{\indexname}%
  \setlength{\parindent}{0pt}%
  \setlength{\parskip}{0pt plus 0.3pt}%
  \let\item\@idxitem
}{%
  \clearpage
}
\makeatother

\IfFileExists{\jobname-pw.ind}{\input{\jobname-pw.ind}}{}

\end{document}

      