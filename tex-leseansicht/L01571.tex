%% latex-leseansicht-vorspann.tex
%% Vorspann für die Leseansicht.
%% Lädt die gemeinsame Datei latex-vorspann.tex mit nicht gesetztem Schalter.

\newif\ifkorrekturansicht
\korrekturansichtfalse

\input{../tex-inputs/latex-vorspann}


         
         \renewcommand{\erwaehntePersonen}{Personen: Hugo von Hofmannsthal, Robert Michel,  Sophokles}
         \renewcommand{\erwaehnteOrte}{Orte: Edmund-Weiß-Gasse 7, Innsbruck, Kaltenleutgeben, Wien, XVIII., Währing}
         \renewcommand{\erwaehnteWerke}{Werke: Der Jäger}
               \section[Hugo von Hofmannsthal an Arthur Schnitzler, 2. 1. 1906]{ Hugo von Hofmannsthal an Arthur Schnitzler, 2. 1. 1906}\nopagebreak\mylabel{v}\rehead{ }\begin{ledgroupsized}[t]{13cm}\normalsize\beginnumbering \toendnotes[C]{\smallbreak\pagebreak[2]} \Standort{CUL, Schnitzler, B 43.}
\physDesc{Postkarte, 264 Zeichen
\newline{}Handschrift: schwarze Tinte, deutsche Kurrent
\newline{}Versand: 1) Stempel: »\nobreak{}\oindex{Kaltenleutgeben@\textbf{Kaltenleutgeben}|pwk}{[}Kal{]}tenleutgeben, 2. 1. {[}1906{]}\nobreak{}«.   2) Stempel: »\nobreak{}\oindex{XVIII., Waehring@\textbf{XVIII., Währing}|pwk}18/1
                                       {[}Wie{]}n, 3. 1. 06, 8.V\nobreak{}«. 
\newline{}Schnitzler: mit Bleistift datiert: »3/1 906« 
\newline{}Ordnung: 1) mit Bleistift von unbekannter Hand nummeriert: »\strikeout{220}«  2) mit Bleistift von unbekannter Hand nummeriert: »\strikeout{216}« 3) mit Bleistift von unbekannter Hand nummeriert:
                                    »259«}\buchAbdrucke{\weitereDrucke{Hugo von Hofmannsthal, Arthur Schnitzler: \emph{Briefwechsel}. Hg. Therese Nickl und Heinrich Schnitzler. Frankfurt am Main: \emph{S. Fischer} 1964, S. 225.} }\toendnotes[C]{\smallbreak}\pstart{}{\pb}\textcolor{gray}{\textbf{Absender:}}\pend{}\pstart{}\textsc{Sophokles\pwindex{Sophokles 497/496? v. u. Z. – 406/405 v. u. Z.@\textsc{Sophokles} (497/496? v. u. Z. – 406/405 v. u. Z.), \emph{Schriftsteller}|pwv}}.\pend{}{\bigskip}\pstart{}\textsc{Herrn D\textsuperscript{r} Arthur Schnitzler}\pend{}\pstart{}\textsc{Wien}\oindex{Wien@\textbf{Wien}|pw}\pend{}\pstart{}\textsc{XVIII. Spöttelgasse 7}.\oindex{Edmund-Weiss-Gasse 7@\textbf{Edmund-Weiß-Gasse 7}|pw}\pend{}{\bigskip}\pstart
           \noindent{}{\pb}lieber, bitte
               ſchreiben Sie mir doch 2 Worte über das \label{K_L01571-1v}\edtext{Stück\pwindex{Michel, Robert 24.02.1876 – 12.02.1957@\textsc{Michel, Robert} (24.02.1876 – 12.02.1957), \emph{Schriftsteller}!Jaeger@\strich\emph{Der Jäger}|pwv}}{\lemma{\textnormal{\emph{Stück}}}\Cendnote{\textnormal{\emph{Der Jäger}\pwindex{Michel, Robert 24.02.1876 – 12.02.1957@\textsc{Michel, Robert} (24.02.1876 – 12.02.1957), \emph{Schriftsteller}!Jaeger@\strich\emph{Der Jäger}|pwk} blieb in dieser Gestalt
                  unveröffentlicht und wurde, zur Novelle umgearbeitet, 1912
                  publiziert.}}}\label{K_L01571-1h} von Michel\pwindex{Michel, Robert 24.02.1876 – 12.02.1957@\textsc{Michel, Robert} (24.02.1876 – 12.02.1957), \emph{Schriftsteller}|pw}, ſchicken es
               aber bitte nicht an mich zurück ſondern gleich an ihn: \pend
           \pstart
           \textsc{Oberleutnant Robert
                     Michel\pwindex{Michel, Robert 24.02.1876 – 12.02.1957@\textsc{Michel, Robert} (24.02.1876 – 12.02.1957), \emph{Schriftsteller}|pw}}\pend
           \pstart
           \textsc{Innsbruck}\oindex{Innsbruck@\textbf{Innsbruck}|pw}\pend
           \pstart
           \textsc{Infanterie Cadettenschule.}\pend
           \pstart \spacefill\mbox{Hugo.}\pend{}\pstart
           \noindent{}2 I.\pend
           
         
         \endnumbering\mylabel{h}\end{ledgroupsized}  \newcommand{\dateiname}{L01571}\newcommand{\titel}{Hugo von Hofmannsthal an Arthur Schnitzler, 2. 1. 1906}\newcommand{\editorInnen}{Martin Anton Müller und Gerd-Hermann Susen}%% latex-leseansicht-abspann.tex
%% Abspann für die Leseansicht.
%% Der Schalter \ifkorrekturansicht ist bereits durch den Vorspann gesetzt.

%% latex-abspann.tex
%% Gemeinsamer Abspann für Korrekturansicht und Leseansicht.
%% Setzt den Schalter \ifkorrekturansicht voraus (gesetzt in den
%% einbindenden Dateien latex-korrekturansicht-abspann.tex bzw.
%% latex-leseansicht-abspann.tex).
%% ---------------------------------------------------------------

\normalsize

% Das esempio-Environment wird nur in der Leseansicht benötigt
\ifkorrekturansicht\else
\newenvironment{esempio}[3]%
{
    \vspace{1.5ex}
    \rlap{\underline{#1}}
    \par
    \setlength{\parindent}{0cm}
    \nopagebreak
    \leftskip=#2cm
    \rightskip=#3cm
}
{
    \par
}
\fi

\doendnotes{C}
\bigskip
\vfill

\clearpage

\footnotesize

\ifkorrekturansicht
  \lohead{\textsc{register}}
\fi

% theindex-Environment neu definieren ohne reledmac
\makeatletter
\renewenvironment{theindex}{%
  \ifkorrekturansicht
    \section*{\indexname}%
  \else
    \subsubsection*{Index der erwähnten Entitäten}%
  \fi
  \setlength{\parindent}{0pt}%
  \setlength{\parskip}{0pt plus 0.3pt}%
  \let\item\@idxitem
}{%
  \ifkorrekturansicht\clearpage\fi
}
\makeatother

\IfFileExists{\jobname-pw.ind}{\input{\jobname-pw.ind}}{}

% Quellenangabe nur in der Leseansicht
\ifkorrekturansicht\else
% Fallback-Definitionen, falls die .tex-Datei \titel etc. nicht gesetzt hat
\providecommand{\titel}{}
\providecommand{\editorInnen}{}
\providecommand{\dateiname}{\jobname}

\vspace{3cm}

\vfill

\footnotesize
\textsc{Quelle}: \titel. Herausgegeben von {\editorInnen}. In: \emph{Arthur Schnitzler: Briefwechsel mit Autorinnen und Autoren}.
 Digitale Edition, https://schnitzler-briefe.acdh.oeaw.ac.at/{\dateiname}.html (Stand \today)
\fi

\end{document}


      