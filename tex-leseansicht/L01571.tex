%% latex-korrekturansicht-vorspann.tex
%% Vorspann für die Korrekturansicht.
%% Lädt die gemeinsame Datei latex-vorspann.tex mit gesetztem Schalter.

\newif\ifkorrekturansicht
\korrekturansichttrue

\input{../tex-inputs/latex-vorspann}


\section[Hugo von Hofmannsthal an Arthur Schnitzler, 2. 1. 1906]{L01571 Hugo von Hofmannsthal an Arthur Schnitzler, 2. 1. 1906}
\nopagebreak\mylabel{L01571v}
\rehead{ }\normalsize\beginnumbering\briefempfaengerindex{Schnitzler, Arthur@\textsc{Schnitzler, Arthur}!zzzHofmannsthal, Hugo von@\emph{von Hugo von Hofmannsthal}!1906-01-021@{2. 1. 1906}|(be}
\toendnotes[C]{\smallbreak\pagebreak[2]}\Standort{CUL, Schnitzler, B 43.}
\physDesc{Postkarte, 264 Zeichen
\newline{}Handschrift: schwarze Tinte, deutsche Kurrent
\newline{}Versand: 1) Stempel: »\nobreak{}\oindex{Kaltenleutgeben@\textbf{Kaltenleutgeben}, \emph{P.PPLA3}|pwk}{[}Kal{]}tenleutgeben, 2. 1. {[}1906{]}\nobreak{}«.   2) Stempel: »\nobreak{}\oindex{XVIII., Waehring@\textbf{XVIII., Währing}, \emph{A.ADM3}|pwk}18/1
                                       {[}Wie{]}n, 3. 1. 06, 8.V\nobreak{}«. 
\newline{}Schnitzler: mit Bleistift datiert: »3/1 906« 
\newline{}Ordnung: 1) mit Bleistift von unbekannter Hand nummeriert: »\strikeout{220}«  2) mit Bleistift von unbekannter Hand nummeriert: »\strikeout{216}« 3) mit Bleistift von unbekannter Hand nummeriert:
                                    »259«}
\buchAbdrucke{\weitereDrucke{Hugo von Hofmannsthal, Arthur Schnitzler: \emph{Briefwechsel}. Frankfurt am Main: \emph{S. Fischer} 1964, S. 225.} }\toendnotes[C]{\smallbreak}\pstart{}{\pb}\textcolor{gray}{\textbf{Absender:}}\pend{}\pstart{}\textsc{Sophokles\pwindex{Sophokles 497/496? v. u. Z. – 406/405 v. u. Z.@\textsc{Sophokles} (497/496? v. u. Z. – 406/405 v. u. Z.), \emph{Schriftsteller/Schriftstellerin}|pwv}}.\pend{}{\bigskip}\pstart{}\textsc{Herrn D\textsuperscript{r} Arthur Schnitzler}\pend{}\pstart{}\textsc{Wien}\oindex{Wien@\textbf{Wien}, \emph{A.ADM2}|pw}\pend{}\pstart{}\textsc{XVIII. Spöttelgasse 7}.\oindex{Edmund-Weiss-Gasse 7@\textbf{Edmund-Weiß-Gasse 7}, \emph{Wohngebäude (K.WHS)}|pw}\pend{}{\bigskip}\vspace{1em}
\pstart
           \noindent{}{\pb}lieber, bitte
               ſchreiben Sie mir doch 2 Worte über das \label{K_L01571-1v}\edtext{Stück\pwindex{Jaeger@\emph{Der Jäger}|pwv}}{\lemma{\textnormal{\emph{Stück}}}\Cendnote{\textnormal{\emph{Der Jäger}\pwindex{Jaeger@\emph{Der Jäger}|pwk} blieb in dieser Gestalt
                  unveröffentlicht und wurde, zur Novelle umgearbeitet, 1912
                  publiziert.}}}\label{K_L01571-1} von Michel\pwindex{Michel, Robert 24.02.1876 – 12.02.1957@\textsc{Michel, Robert} (24.02.1876 – 12.02.1957), \emph{Schriftsteller/Schriftstellerin, Offizier/Offizierin, Krimiautor/Krimiautorin}|pw}, ſchicken es
               aber bitte nicht an mich zurück ſondern gleich an ihn: \pend
           
\pstart
           \textsc{Oberleutnant Robert
                     Michel\pwindex{Michel, Robert 24.02.1876 – 12.02.1957@\textsc{Michel, Robert} (24.02.1876 – 12.02.1957), \emph{Schriftsteller/Schriftstellerin, Offizier/Offizierin, Krimiautor/Krimiautorin}|pw}}\pend
           
\pstart
           \textsc{Innsbruck}\oindex{Innsbruck@\textbf{Innsbruck}, \emph{A.ADM2}|pw}\pend
           
\pstart
           \textsc{Infanterie Cadettenschule.}\pend
           \pstart \spacefill\mbox{Hugo.}\pend{}
\pstart
           \noindent{}2 I.\pend
           \selectlanguage{ngerman}\endnumbering\briefempfaengerindex{Schnitzler, Arthur@\textsc{Schnitzler, Arthur}!zzzHofmannsthal, Hugo von@\emph{von Hugo von Hofmannsthal}!1906-01-021@{2. 1. 1906}|)be}\mylabel{L01571h}  \normalsize

\doendnotes{C}
\bigskip
\vfill

\clearpage

\footnotesize

\lohead{\textsc{register}}

% Definiere theindex-Environment komplett neu ohne reledmac
\makeatletter
\renewenvironment{theindex}{%
  \section*{\indexname}%
  \setlength{\parindent}{0pt}%
  \setlength{\parskip}{0pt plus 0.3pt}%
  \let\item\@idxitem
}{%
  \clearpage
}
\makeatother

\IfFileExists{\jobname-pw.ind}{\input{\jobname-pw.ind}}{}

\end{document}

      