\input{../tex-inputs/latex-pdf-vorspann}
\begin{center}
            \textcolor{red}{ENTWURF. ENTZIFFERUNG NOCH NICHT KORREKTURGELESEN}
                      \end{center}
            
               \section[Hugo von Hofmannsthal an Arthur Schnitzler, 15. 11. {[}1911{]}]{ Hugo von Hofmannsthal an Arthur Schnitzler, 15. 11. {[}1911{]}}\nopagebreak\mylabel{v}\rehead{ }\begin{ledgroupsized}[t]{13cm}\normalsize\beginnumbering\briefempfaengerindex{Schnitzler, Arthur@\textsc{Schnitzler, Arthur}!zzzHofmannsthal, Hugo von@\emph{von Hugo von Hofmannsthal}!1911-11-151@{15. 11. 1911}|(be} \toendnotes[C]{\smallbreak\pagebreak[2]} \Standort{CUL, Schnitzler, B 43.}
\physDesc{Briefkarte
\newline{}Handschrift: schwarze Tinte, deutsche Kurrent
\newline{}Schnitzler: mit Bleistift die Jahreszahl ergänzt: »911« und beschriftet: »Hugo« \newline{}Ordnung: 1) mit Bleistift von unbekannter Hand nummeriert:
                              »324« 2) mit Bleistift von unbekannter Hand nummeriert: »333«}\buchAbdrucke{\weitereDrucke{Hugo von Hofmannsthal, Arthur Schnitzler: \emph{Briefwechsel}. Hg. Therese Nickl und Heinrich Schnitzler. Frankfurt am Main: \emph{S. Fischer} 1964, S. 264.} }\pstart
           \raggedleft{}{\pb}15 XI.\pend
           \pstart{}mein lieber Arthur\pend\pstart
           wenn Sie da ſind, ſo geben Sie mir doch \uline{gleich} ein
               Zeichen\pend
           \pstart
           ich muſs nun baldigſt wieder nach Berlin\oindex{Berlin@\textbf{Berlin}|pw},
               wegen \textsc{Jedermann}\pwindex{Hofmannsthal, Hugo von 01.02.1874 – 15.07.1929@\textsc{Hofmannsthal, Hugo von} (01.02.1874 – 15.07.1929), \emph{Schriftsteller}!Jedermann. Das Spiel vom Sterben des reichen Mannes1911@\strich\emph{Jedermann. Das Spiel vom Sterben des reichen Mannes} {[}1911{]}|pw}, wie {\pb}viele Monate des Lebens ſollen
               denn noch vergehen ohne daſs man etwas davon hat, vorläufig noch gleichzeitig am
               Leben zu ſein.\hspace*{1.5em}Alles Liebe an Olga\pwindex{Schnitzler, Olga 17.01.1882 – 13.01.1970@\textsc{Schnitzler, Olga} (17.01.1882 – 13.01.1970), \emph{Schauspielerin, Sängerin}|pw}.\pend
           \pstart Ihr \spacefill\mbox{Hugo.}\pend{}\endnumbering\briefempfaengerindex{Schnitzler, Arthur@\textsc{Schnitzler, Arthur}!zzzHofmannsthal, Hugo von@\emph{von Hugo von Hofmannsthal}!1911-11-151@{15. 11. 1911}|)be}\mylabel{h}\end{ledgroupsized}  \newcommand{\dateiname}{L02046}\newcommand{\titel}{Hugo von Hofmannsthal an Arthur Schnitzler, 15. 11. [1911]}\newcommand{\editorInnen}{Martin Anton Müller und Gerd-Hermann Susen}\input{../tex-inputs/latex-pdf-abspann}
      