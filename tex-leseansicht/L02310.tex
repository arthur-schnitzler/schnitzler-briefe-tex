%% latex-korrekturansicht-vorspann.tex
%% Vorspann für die Korrekturansicht.
%% Lädt die gemeinsame Datei latex-vorspann.tex mit gesetztem Schalter.

\newif\ifkorrekturansicht
\korrekturansichttrue

\input{../tex-inputs/latex-vorspann}


\section[Arthur Schnitzler an Robert Adam, 13. 11. 1918]{L02310 Arthur Schnitzler an Robert Adam, 13. 11. 1918}
\nopagebreak\mylabel{L02310v}
\rehead{ }\normalsize\beginnumbering\briefempfaengerindex{Adam, Robert@\textsc{Adam, Robert}!zzzSchnitzler, Arthur@\emph{von Arthur Schnitzler}!1918-11-131@{13. 11. 1918}|(be}
\toendnotes[C]{\smallbreak\pagebreak[2]}\Standort{DLA, 96.34.2/15.}
\physDesc{Briefkarte, , Umschlag, 486 Zeichen
\newline{}Schreibmaschine
\newline{}Handschrift: schwarze Tinte, lateinische Kurrent (\noindent{}Korrekturen, Grußformel und Unterschrift)
\newline{}Versand: Stempel: »\nobreak{}\textcolor{gray}{13.} XI. 18, 3\nobreak{}«.  }\toendnotes[C]{\smallbreak}\pstart{}{\pb}\textcolor{gray}{\textbf{D\textsuperscript{R} ARTHUR SCHNITZLER}}\pend{}\pstart{}\textcolor{gray}{\textbf{WIEN, XVIII. STERNWARTESTRASSE 71\oindex{Sternwartestrasse 71@\textbf{Sternwartestraße 71}, \emph{Wohngebäude (K.WHS)}|pw}.}}\pend{}{\bigskip}\pstart{}{\pb}Herrn\pend{}\pstart{}Landesgerichtsrat Dr. Robert Adam-Pollak\pend{}\pstart{}Wien XII\oindex{XII., Meidling@\textbf{XII., Meidling}, \emph{A.ADM3}|pw}.\pend{}\pstart{}Meidlinger Hauptstrasse 56\oindex{Meidlinger Hauptstrasse@\textbf{Meidlinger Hauptstraße}, \emph{Straße (K.STR)}|pw}.\pend{}{\bigskip}\vspace{1em}
\pstart
           
\pstart
           {\pb}\textcolor{gray}{\textbf{D\textsuperscript{R} ARTHUR SCHNITZLER}}\pend
           
\pstart
           \raggedleft{}13. 11. 1918\pend
           \pend
           
\pstart
           \textcolor{gray}{\textbf{WIEN, XVIII. STERNWARTESTRASSE 71\oindex{Sternwartestrasse 71@\textbf{Sternwartestraße 71}, \emph{Wohngebäude (K.WHS)}|pw}.}}\pend
           
\pstart{}Lieber und verehrter Herr Doktor.\pend\vspace{0.5em}
\pstart
           Man ist im Deutschen Volkstheater\oindex{Volkstheater@\textbf{Volkstheater}, \emph{Theater (K.THE)}|pw} auf die
               Einsendung Ihrer Stücke\pwindex{Yppl. Idylle in fuenf Akten@\emph{Yppl. Idylle in fünf Akten}|pwv}\pwindex{Fremde@\emph{Der Fremde}|pwv} vorbereitet \substVorne{}\textsuperscript{. Man}\substDazwischen{}und\substHinten{} hat mir zugesagt sie sofort und mit aller Aufmerksamkeit zu lesen.
               Vielleicht senden Sie sowohl den »Fremden\pwindex{Fremde@\emph{Der Fremde}|pw}« als
               auch »\substVorne{}\textsuperscript{Ue}\substDazwischen{}Y\substHinten{}ppel\pwindex{Yppl. Idylle in fuenf Akten@\emph{Yppl. Idylle in fünf Akten}|pw}« ein und beziehen sich mit ein paar Worten auf meine Rücksprache
               in der Direktion. – Auf baldiges Wiedersehen und herzliche Grüsse.\pend
           
\pstart
           {[}hs.:{]} Ihr{\\[\baselineskip]}\spacefill\mbox{Arthur Schnitzler}\pend
           \leftskip=0em{}\selectlanguage{ngerman}\endnumbering\briefempfaengerindex{Adam, Robert@\textsc{Adam, Robert}!zzzSchnitzler, Arthur@\emph{von Arthur Schnitzler}!1918-11-131@{13. 11. 1918}|)be}\mylabel{L02310h}  \normalsize

\doendnotes{C}
\bigskip
\vfill

\clearpage

\footnotesize

\lohead{\textsc{register}}

% Definiere theindex-Environment komplett neu ohne reledmac
\makeatletter
\renewenvironment{theindex}{%
  \section*{\indexname}%
  \setlength{\parindent}{0pt}%
  \setlength{\parskip}{0pt plus 0.3pt}%
  \let\item\@idxitem
}{%
  \clearpage
}
\makeatother

\IfFileExists{\jobname-pw.ind}{\input{\jobname-pw.ind}}{}

\end{document}

      