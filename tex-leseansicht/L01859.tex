%% latex-leseansicht-vorspann.tex
%% Vorspann für die Leseansicht.
%% Lädt die gemeinsame Datei latex-vorspann.tex mit nicht gesetztem Schalter.

\newif\ifkorrekturansicht
\korrekturansichtfalse

\input{../tex-inputs/latex-vorspann}


\section[Arthur Schnitzler an Albert Ehrenstein, 24. 7. 1909]{L01859 Arthur Schnitzler an Albert Ehrenstein, 24. 7. 1909}
\nopagebreak\mylabel{L01859v}
\rehead{ }\normalsize\beginnumbering\briefempfaengerindex{Ehrenstein, Albert@\textsc{Ehrenstein, Albert}!zzzSchnitzler, Arthur@\emph{von Arthur Schnitzler}!1909-07-241@{24. 7. 1909}|(be}
\toendnotes[C]{\smallbreak\pagebreak[2]}
\correspDesc{Versand  durch Arthur Schnitzler am 24. 7. 1909 in Edlach
\newline{}Erhalt  durch Albert Ehrenstein im Zeitraum [24. 7. 1909
                  – 28. 7. 1909?] \textbf{Ort fehlend} }\toendnotes[C]{\smallbreak}
\Standort{Jerusalem, The National Library of Israel, ARC. Ms. Var. 306 1 118.}
\physDesc{Briefkarte, 919 Zeichen
\newline{}Handschrift: schwarze Tinte, lateinische Kurrent}\toendnotes[C]{\smallbreak}
\pstart
           \raggedleft{}{\pb}Edlach\oindex{Edlach@\textbf{Edlach}|pw}, 24. 7. 09\pend
           
\pstart
           \textcolor{gray}{\textbf{Dr. Arthur Schnitzler}}{\\}\textcolor{gray}{\textbf{Wien XVIII. Spoettelgasse 7\oindex{Wien@\textbf{Wien}!XVIII., Währing@\textbf{XVIII., Währing}!Edmund-Weiß-Gasse 7@\textbf{Edmund-Weiß-Gasse 7}, \emph{Wohngebäude}|pw}.}}\pend
           \vspace{0.5em}
\pstart
           lieber Herr Ehrenstein, mit Auernheimer\pwindex{Auernheimer, Raoul 15.\,4.\,1876 Wien – 6.\,1.\,1948 Oakland@\textsc{Auernheimer, Raoul} (15.\,4.\,1876 Wien – 6.\,1.\,1948 Oakland), \emph{Schriftsteller, Journalist, Kritiker}|pw} hab ich dieser Tage viel über Sie gesprochen. Bei dieser
               Gelegenheit mit angenehmen Erstaunen bemerkt, daß er Ihre Sachen damals sehr
               eingehend und mit entschiedener Antheilnahme für die offenbare Eigenart gelesen hat.
               Er eri{\geminationn}erte sich vieler Details und ist durchaus bereit,
               alles weitere mit einem jetzt wohl noch etwas gestei{\pb}{[}gerten{]} Interesse durchzusehen. Eine Kritik über eine Dissertation\pwindex{\textcolor{red}{\textsuperscript{XXXX indx1}}!?? [Dissertation]@\strich\emph{?? [Dissertation]}|pwv} hat wohl wenig
               Chancen – aber immerhin denke ich, Sie senden sie ihm ein. Jetzt ist er allerdings
               noch auf Urlaub, reist auch bald von hier fort, (heute, fällt mir eben ein), Semmering\oindex{Semmering@\textbf{Semmering}, \emph{Verwaltungsgebiet}|pw}, dann Süd
                  Tirol\oindex{Südtirol@\textbf{Südtirol}, \emph{Verwaltungsgebiet}|pw}. Aber ich halte es für ganz vernünftig, we{\geminationn} sie zu Beginn des Herbstes ihn zu einer persönlichen Unterredung aufsuchen
               wollten. – Aergerlich, daß Sie mit solchen Leuten wie diesem Professor\pwindex{Fournier, August 19.\,6.\,1850 Wien – 18.\,5.\,1920 ebd.@\textsc{Fournier, August} (19.\,6.\,1850 Wien – 18.\,5.\,1920 ebd.), \emph{Historiker}|pwv} zu thun haben! Aber wer nicht –?
               (Billig, aber wahr.) Herzlich grüßend\hspace*{1.5em}Ihr
                  ergebener\spacefill\mbox{A. S.}\pend
           \selectlanguage{ngerman}\endnumbering\briefempfaengerindex{Ehrenstein, Albert@\textsc{Ehrenstein, Albert}!zzzSchnitzler, Arthur@\emph{von Arthur Schnitzler}!1909-07-241@{24. 7. 1909}|)be}\mylabel{L01859h}  \newcommand{\dateiname}{L01859}\newcommand{\titel}{Arthur Schnitzler an Albert Ehrenstein, 24. 7. 1909}\newcommand{\editorInnen}{Martin Anton Müller und Gerd-Hermann Susen}%% latex-leseansicht-abspann.tex
%% Abspann für die Leseansicht.
%% Der Schalter \ifkorrekturansicht ist bereits durch den Vorspann gesetzt.

%% latex-abspann.tex
%% Gemeinsamer Abspann für Korrekturansicht und Leseansicht.
%% Setzt den Schalter \ifkorrekturansicht voraus (gesetzt in den
%% einbindenden Dateien latex-korrekturansicht-abspann.tex bzw.
%% latex-leseansicht-abspann.tex).
%% ---------------------------------------------------------------

\normalsize

% Das esempio-Environment wird nur in der Leseansicht benötigt
\ifkorrekturansicht\else
\newenvironment{esempio}[3]%
{
    \vspace{1.5ex}
    \rlap{\underline{#1}}
    \par
    \setlength{\parindent}{0cm}
    \nopagebreak
    \leftskip=#2cm
    \rightskip=#3cm
}
{
    \par
}
\fi

\doendnotes{C}
\bigskip
\vfill

\clearpage

\footnotesize

\ifkorrekturansicht
  \lohead{\textsc{register}}
\fi

% theindex-Environment neu definieren ohne reledmac
\makeatletter
\renewenvironment{theindex}{%
  \ifkorrekturansicht
    \section*{\indexname}%
  \else
    \subsubsection*{Index der erwähnten Entitäten}%
  \fi
  \setlength{\parindent}{0pt}%
  \setlength{\parskip}{0pt plus 0.3pt}%
  \let\item\@idxitem
}{%
  \ifkorrekturansicht\clearpage\fi
}
\makeatother

\IfFileExists{\jobname-pw.ind}{\input{\jobname-pw.ind}}{}

% Quellenangabe nur in der Leseansicht
\ifkorrekturansicht\else
% Fallback-Definitionen, falls die .tex-Datei \titel etc. nicht gesetzt hat
\providecommand{\titel}{}
\providecommand{\editorInnen}{}
\providecommand{\dateiname}{\jobname}

\vspace{3cm}

\vfill

\footnotesize
\textsc{Quelle}: \titel. Herausgegeben von {\editorInnen}. In: \emph{Arthur Schnitzler: Briefwechsel mit Autorinnen und Autoren}.
 Digitale Edition, https://schnitzler-briefe.acdh.oeaw.ac.at/{\dateiname}.html (Stand \today)
\fi

\end{document}


