%% latex-korrekturansicht-vorspann.tex
%% Vorspann für die Korrekturansicht.
%% Lädt die gemeinsame Datei latex-vorspann.tex mit gesetztem Schalter.

\newif\ifkorrekturansicht
\korrekturansichttrue

\input{../tex-inputs/latex-vorspann}


\section[Arthur Schnitzler an Albert Ehrenstein, 24. 7. 1909]{L01859 Arthur Schnitzler an Albert Ehrenstein, 24. 7. 1909}
\nopagebreak\mylabel{L01859v}
\rehead{ }\normalsize\beginnumbering\briefempfaengerindex{Ehrenstein, Albert@\textsc{Ehrenstein, Albert}!zzzSchnitzler, Arthur@\emph{von Arthur Schnitzler}!1909-07-241@{24. 7. 1909}|(be}
\toendnotes[C]{\smallbreak\pagebreak[2]}\Standort{Jerusalem, The National Library of Israel, ARC. Ms. Var. 306 1 118.}
\physDesc{Briefkarte, 919 Zeichen
\newline{}Handschrift: schwarze Tinte, lateinische Kurrent}\toendnotes[C]{\smallbreak}
\pstart
           \raggedleft{}{\pb}Edlach\oindex{Edlach@\textbf{Edlach}, \emph{P.PPL}|pw}, 24. 7. 09\pend
           
\pstart
           \textcolor{gray}{\textbf{Dr. Arthur Schnitzler}}{\\}\textcolor{gray}{\textbf{Wien XVIII. Spoettelgasse 7\oindex{Edmund-Weiss-Gasse 7@\textbf{Edmund-Weiß-Gasse 7}, \emph{Wohngebäude (K.WHS)}|pw}.}}\pend
           \vspace{0.5em}
\pstart
           lieber Herr Ehrenstein, mit Auernheimer\pwindex{Auernheimer, Raoul 15.04.1876 – 06.01.1948@\textsc{Auernheimer, Raoul} (15.04.1876 – 06.01.1948), \emph{Schriftsteller/Schriftstellerin, Journalist/Journalistin, Kritiker/Kritikerin}|pw} hab ich dieser Tage viel über Sie gesprochen. Bei dieser
               Gelegenheit mit angenehmen Erstaunen bemerkt, daß er Ihre Sachen damals sehr
               eingehend und mit entschiedener Antheilnahme für die offenbare Eigenart gelesen hat.
               Er eri{\geminationn}erte sich vieler Details und ist durchaus bereit,
               alles weitere mit einem jetzt wohl noch etwas gestei{\pb}{[}gerten{]} Interesse durchzusehen. Eine Kritik über eine Dissertation\pwindex{?? [Dissertation]@\emph{?? [Dissertation]}|pwv} hat wohl wenig
               Chancen – aber immerhin denke ich, Sie senden sie ihm ein. Jetzt ist er allerdings
               noch auf Urlaub, reist auch bald von hier fort, (heute, fällt mir eben ein), Semmering\oindex{Semmering@\textbf{Semmering}, \emph{A.ADM3}|pw}, dann Süd
                  Tirol\oindex{Suedtirol@\textbf{Südtirol}, \emph{A.ADM2}|pw}. Aber ich halte es für ganz vernünftig, we{\geminationn} sie zu Beginn des Herbstes ihn zu einer persönlichen Unterredung aufsuchen
               wollten. – Aergerlich, daß Sie mit solchen Leuten wie diesem Professor\pwindex{Fournier, August 19.06.1850 – 18.05.1920@\textsc{Fournier, August} (19.06.1850 – 18.05.1920), \emph{Historiker/Historikerin}|pwv} zu thun haben! Aber wer nicht –?
               (Billig, aber wahr.) Herzlich grüßend\hspace*{1.5em}Ihr
                  ergebener\spacefill\mbox{A. S.}\pend
           \selectlanguage{ngerman}\endnumbering\briefempfaengerindex{Ehrenstein, Albert@\textsc{Ehrenstein, Albert}!zzzSchnitzler, Arthur@\emph{von Arthur Schnitzler}!1909-07-241@{24. 7. 1909}|)be}\mylabel{L01859h}  \normalsize

\doendnotes{C}
\bigskip
\vfill

\clearpage

\footnotesize

\lohead{\textsc{register}}

% Definiere theindex-Environment komplett neu ohne reledmac
\makeatletter
\renewenvironment{theindex}{%
  \section*{\indexname}%
  \setlength{\parindent}{0pt}%
  \setlength{\parskip}{0pt plus 0.3pt}%
  \let\item\@idxitem
}{%
  \clearpage
}
\makeatother

\IfFileExists{\jobname-pw.ind}{\input{\jobname-pw.ind}}{}

\end{document}

      