%% latex-leseansicht-vorspann.tex
%% Vorspann für die Leseansicht.
%% Lädt die gemeinsame Datei latex-vorspann.tex mit nicht gesetztem Schalter.

\newif\ifkorrekturansicht
\korrekturansichtfalse

\input{../tex-inputs/latex-vorspann}


\section[Hugo von Hofmannsthal an Arthur Schnitzler, 9. 8. {[}1895{]}]{L00471 Hugo von Hofmannsthal an Arthur Schnitzler, 9. 8. [1895]}
\nopagebreak\mylabel{L00471v}
\rehead{ }\normalsize\beginnumbering\briefempfaengerindex{Schnitzler, Arthur@\textsc{Schnitzler, Arthur}!zzzHofmannsthal, Hugo von@\emph{von Hugo von Hofmannsthal}!1895-08-092@{9. 8. [1895]}|(be}
\toendnotes[C]{\smallbreak\pagebreak[2]}
\correspDesc{Versand  durch Hugo von Hofmannsthal am 9. 8. [1895] in Hodonín
\newline{}Erhalt  durch Arthur Schnitzler im Zeitraum [10. 8. 1895
                  – 14. 8. 1895?] in Wien}\toendnotes[C]{\smallbreak}
\Standort{CUL, Schnitzler, B 43.}
\physDesc{Brief, 1 Blatt, 4 Seiten, 1427 Zeichen
\newline{}Handschrift: schwarze Tinte, deutsche Kurrent
\newline{}Schnitzler: mit Bleistift die Jahreszahl ergänzt: »95« und
                                 nummeriert: »74« }
\buchAbdrucke{\weitereDrucke{1) Hugo von Hofmannsthal: \emph{Briefe. 1890–1901}. Berlin: \emph{S. Fischer} 1935, S. 164–165.} \weitereDrucke{2) Hugo von Hofmannsthal, Arthur Schnitzler: \emph{Briefwechsel}. Herausgegeben von Therese Nickl und Heinrich Schnitzler. Frankfurt am Main: \emph{S. Fischer} 1964, S. 58–59.} }\toendnotes[C]{\smallbreak}
\pstart
           \raggedleft{}{\pb}Göding\oindex{Hodonín@\textbf{Hodonín}|pw}{ }9. Auguſt\pend
           
\pstart{}lieber Arthur\pend\vspace{0.5em}
\pstart
           es iſt doch{ }ſehr merkwürdig,{ }ſo wider{ }ſeine Natur zu leben, wie ich es jetzt thue,
               unter Menſchen, denen jeder Antheil{ }ſchon faſt wie Affectation erſcheint. Ich bin
               begierig, wie ich das{ }ſehen werde, wenn ich von dem unmittelbaren Zwang befreit bin.
               Euch vermuthe ich mit den däniſchen\oindex{Dänemark@\textbf{Dänemark}|pw} Buchten und
               der München\oindex{München@\textbf{München}|pw}er Bilderausſtellung in {\pb}Gedanken{ }ſo{ }ſpielend, wie mit
               Spielereien die noch in der Schachtel{ }ſind. Es kränkt mich, daſs mir der Richard\pwindex{Beer-Hofmann, Richard 11.\,7.\,1866 Wien – 26.\,9.\,1945 New York City@\textsc{Beer-Hofmann, Richard} (11.\,7.\,1866 Wien – 26.\,9.\,1945 New York City), \emph{Schriftsteller}|pw} nicht{ }ſchreibt. Seit 6 Wochen hat er
               mir \uline{einen} Brief geſchrieben, obwohl er weiß, daſs ich
               eine kindiſche Freude über jeden Brief hab, und hier wirklich wenig habe was mir
               Freud macht. Sonntag iſt das Rennen. Wenn ich an die Bretterwand hinflieg und mir das
               Genick brech (unwahrſcheinlich, {\pb}aber möglich){ }ſollt Ihr meine vielen Notizen auf Zetteln herausgeben, in
               Gedankengruppen geordnet, mit einem{ }ſehr einfachen, die Aſſociationen aufdeckenden
               Commentar. Denn meine Gedanken gehören alle zuſammen, weil ich von der Einheit der
               Welt{ }ſehr{ }ſtark durchdrungen bin. Ich glaub{ }ſogar ein Dichter iſt eben ein Menſch,
               dem in guten Stunden die Gedanken »ausgehen« wie man beim Patiencelegen{ }ſagt. – Am
                     15\textsuperscript{ten} iſt Abmarſch {\pb}nach Znaim\oindex{Znaim@\textbf{Znaim}, \emph{Hauptstadt}|pw}, dann Stockerau\oindex{Stockerau@\textbf{Stockerau}, \emph{Hauptstadt}|pw} etc. etc. Bitte alſo Briefe vom 14\textsuperscript{ten} an nach Wien\oindex{Wien@\textbf{Wien}, \emph{Verwaltungsgebiet}|pw} richten, von wo{ }ſie
               nachgeſchickt werden.\pend
           
\pstart
           Auf Wiederſehen!{\\[\baselineskip]}\spacefill\mbox{Hugo.}\pend
           \leftskip=0em{}
\pstart
           \noindent{}Bitte können Sie in Erfahrung bringen ob D\textsuperscript{r}{ }\label{K_L00471-1v}\edtext{Mamroth\pwindex{Mamroth, Fedor 21.\,2.\,1851 Breslau – 25.\,6.\,1907 Frankfurt am Main@\textsc{Mamroth, Fedor} (21.\,2.\,1851 Breslau – 25.\,6.\,1907 Frankfurt am Main), \emph{Journalist, Kritiker}|pw} nicht mehr bei der Frankf.\orgindex{Frankfurter Zeitung@Frankfurter Zeitung|pw}}{\lemma{\textnormal{\emph{Mamroth … Frankf.}}}\Cendnote{\textnormal{Siehe XXXX Auszeichnungsfehler: Dokument L02745 nicht gefunden.
                  }}}\label{K_L00471-1} iſt, oder beurlaubt? und mir das{ }ſchreiben?\pend
           \selectlanguage{ngerman}\endnumbering\briefempfaengerindex{Schnitzler, Arthur@\textsc{Schnitzler, Arthur}!zzzHofmannsthal, Hugo von@\emph{von Hugo von Hofmannsthal}!1895-08-092@{9. 8. [1895]}|)be}\mylabel{L00471h}  \newcommand{\dateiname}{L00471}\newcommand{\titel}{Hugo von Hofmannsthal an Arthur Schnitzler, 9. 8. [1895]}\newcommand{\editorInnen}{Martin Anton Müller und Gerd-Hermann Susen}%% latex-leseansicht-abspann.tex
%% Abspann für die Leseansicht.
%% Der Schalter \ifkorrekturansicht ist bereits durch den Vorspann gesetzt.

%% latex-abspann.tex
%% Gemeinsamer Abspann für Korrekturansicht und Leseansicht.
%% Setzt den Schalter \ifkorrekturansicht voraus (gesetzt in den
%% einbindenden Dateien latex-korrekturansicht-abspann.tex bzw.
%% latex-leseansicht-abspann.tex).
%% ---------------------------------------------------------------

\normalsize

% Das esempio-Environment wird nur in der Leseansicht benötigt
\ifkorrekturansicht\else
\newenvironment{esempio}[3]%
{
    \vspace{1.5ex}
    \rlap{\underline{#1}}
    \par
    \setlength{\parindent}{0cm}
    \nopagebreak
    \leftskip=#2cm
    \rightskip=#3cm
}
{
    \par
}
\fi

\doendnotes{C}
\bigskip
\vfill

\clearpage

\footnotesize

\ifkorrekturansicht
  \lohead{\textsc{register}}
\fi

% theindex-Environment neu definieren ohne reledmac
\makeatletter
\renewenvironment{theindex}{%
  \ifkorrekturansicht
    \section*{\indexname}%
  \else
    \subsubsection*{Index der erwähnten Entitäten}%
  \fi
  \setlength{\parindent}{0pt}%
  \setlength{\parskip}{0pt plus 0.3pt}%
  \let\item\@idxitem
}{%
  \ifkorrekturansicht\clearpage\fi
}
\makeatother

\IfFileExists{\jobname-pw.ind}{\input{\jobname-pw.ind}}{}

% Quellenangabe nur in der Leseansicht
\ifkorrekturansicht\else
% Fallback-Definitionen, falls die .tex-Datei \titel etc. nicht gesetzt hat
\providecommand{\titel}{}
\providecommand{\editorInnen}{}
\providecommand{\dateiname}{\jobname}

\vspace{3cm}

\vfill

\footnotesize
\textsc{Quelle}: \titel. Herausgegeben von {\editorInnen}. In: \emph{Arthur Schnitzler: Briefwechsel mit Autorinnen und Autoren}.
 Digitale Edition, https://schnitzler-briefe.acdh.oeaw.ac.at/{\dateiname}.html (Stand \today)
\fi

\end{document}


