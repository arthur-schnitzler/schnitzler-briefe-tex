%% latex-korrekturansicht-vorspann.tex
%% Vorspann für die Korrekturansicht.
%% Lädt die gemeinsame Datei latex-vorspann.tex mit gesetztem Schalter.

\newif\ifkorrekturansicht
\korrekturansichttrue

\input{../tex-inputs/latex-vorspann}


\section[ Paul Goldmann an Olga Gussmann, 21. 1. 1903]{L03529 Paul Goldmann an Olga Gussmann, 21. 1. 1903}
\nopagebreak\mylabel{L03529v}
\rehead{ }\normalsize\beginnumbering\briefempfaengerindex{Schnitzler, Olga@\textsc{Schnitzler, Olga}!zzzGoldmann, Paul@\emph{von Paul Goldmann}!1903-01-211@{21. 1. 1903}|(be}
\toendnotes[C]{\smallbreak\pagebreak[2]}\Standort{DLA, A:Schnitzler, HS.NZ85.1.5247.}
\physDesc{Postkarte, 274 Zeichen
\newline{}Handschrift: 1) blaue Tinte, deutsche Kurrent\hspace{1em}2) blaue Tinte, lateinische Kurrent (\noindent{}Adresse)\hspace{1em}
\newline{}Versand: Stempel: »\nobreak{}\oindex{Berlin@\textbf{Berlin}, \emph{P.PPLC}|pwk}Berlin, S. W. 11, 21. 1. 03, 12–1N.\nobreak{}«. Stempel: »\nobreak{}\oindex{XVIII., Waehring@\textbf{XVIII., Währing}, \emph{A.ADM3}|pwk}18/1 Wien 110, 22. 1. 03, 10. V, Bestellt\nobreak{}«.  }\toendnotes[C]{\smallbreak}\pstart{}{\pb}Frau\pend{}\pstart{}Olga Gussmann\pend{}\pstart{}Wien\oindex{Wien@\textbf{Wien}, \emph{A.ADM2}|pw}\pend{}\pstart{}Gentzgaſse 110\oindex{Gentzgasse@\textbf{Gentzgasse}, \emph{Straße (K.STR)}|pw}\pend{}\pstart{}Währing\oindex{XVIII., Waehring@\textbf{XVIII., Währing}, \emph{A.ADM3}|pw}\pend{}{\bigskip}\vspace{1em}
\pstart
           \noindent{}{\pb}Berlin\oindex{Berlin@\textbf{Berlin}, \emph{P.PPLC}|pw}, 21. Januar. Liebe Freundin, Ich danke vielm\substVorne{}\textsuperscript{ä}\substDazwischen{}a\substHinten{}ls für die Grüße aus \label{K_L03529-1v}\edtext{Salzburg\oindex{Salzburg@\textbf{Salzburg}, \emph{A.ADM2}|pw}}{\lemma{\textnormal{\emph{Salzburg}}}\Cendnote{\textnormal{Die Postkarte, die Arthur Schnitzler und Olga
                     Gussmann\pwindex{Schnitzler, Olga 17.01.1882 – 13.01.1970@\textsc{Schnitzler, Olga} (17.01.1882 – 13.01.1970), \emph{Schauspieler/Schauspielerin, Sänger/Sängerin}|pwk} von ihrem Aufenthalt in Salzburg\oindex{Salzburg@\textbf{Salzburg}, \emph{A.ADM2}|pwk} zwischen 12. 1. 1903 und 19. 1. 1903 gesandt haben, ist, wie nahezu alle Korrespondenzstücke
                     Schnitzlers an Goldmann\pwindex{Goldmann, Paul 31.01.1865 – 25.09.1935@\textsc{Goldmann, Paul} (31.01.1865 – 25.09.1935), \emph{Schriftsteller/Schriftstellerin, Journalist/Journalistin}|pwk}, nicht überliefert.}}}\label{K_L03529-1}, die ich Ihnen
               und Arthur herzlichſt erwidere. Ich ſtecke tief
               in der Arbeit. Warum höre ich gar nichts mehr von Arthur?\pend
           
\pstart
           Herzlichſt Ihr getreuer {\\[\baselineskip]}\spacefill\mbox{Paul Goldmann}\pend
           \leftskip=0em{}\selectlanguage{ngerman}\endnumbering\briefempfaengerindex{Schnitzler, Olga@\textsc{Schnitzler, Olga}!zzzGoldmann, Paul@\emph{von Paul Goldmann}!1903-01-211@{21. 1. 1903}|)be}\mylabel{L03529h}  \normalsize

\doendnotes{C}
\bigskip
\vfill

\clearpage

\footnotesize

\lohead{\textsc{register}}

% Definiere theindex-Environment komplett neu ohne reledmac
\makeatletter
\renewenvironment{theindex}{%
  \section*{\indexname}%
  \setlength{\parindent}{0pt}%
  \setlength{\parskip}{0pt plus 0.3pt}%
  \let\item\@idxitem
}{%
  \clearpage
}
\makeatother

\IfFileExists{\jobname-pw.ind}{\input{\jobname-pw.ind}}{}

\end{document}

      