%% latex-leseansicht-vorspann.tex
%% Vorspann für die Leseansicht.
%% Lädt die gemeinsame Datei latex-vorspann.tex mit nicht gesetztem Schalter.

\newif\ifkorrekturansicht
\korrekturansichtfalse

\input{../tex-inputs/latex-vorspann}

\begin{center}
            \textcolor{red}{ENTWURF, NICHT FERTIG KORRIGIERT}
                      \end{center}
            
         
         \renewcommand{\erwaehntePersonen}{Personen: Samuel Fischer, Georg Hirschfeld}
         \renewcommand{\erwaehnteInstitutionen}{Institutionen: S. Fischer Verlag, Volkstheater}
         \renewcommand{\erwaehnteOrte}{Orte: Berlin, Wien}
         \renewcommand{\erwaehnteWerke}{Werke: Der Hinterbliebene. Kurze Novellen, Schöne Seelen. Komödie in einem Akt}
               \section[Felix Salten an Arthur Schnitzler, 9. 10. 1899]{ Felix Salten an Arthur Schnitzler, 9. 10. 1899}\nopagebreak\mylabel{v}\rehead{ }\begin{ledgroupsized}[t]{13cm}\normalsize\beginnumbering \toendnotes[C]{\smallbreak\pagebreak[2]} \Standort{CUL, Schnitzler, B 89, A 2.}
\physDesc{Brief, 1 Blatt, 2 Seiten
\newline{}Handschrift: schwarze Tinte, lateinische Kurrent\newline{}Ordnung: mit Bleistift von unbekannter Hand nummeriert:
                                    »125« }\toendnotes[C]{\smallbreak}\pstart
           \raggedleft{}{\pb}Wien\oindex{Wien@\textbf{Wien}|pw}, 9. X. 99. \pend
           \pstart
           Lieber Freund, von Hirschfeld\pwindex{Hirschfeld, Georg 11.02.1873 – 17.01.1942@\textsc{Hirschfeld, Georg} (11.02.1873 – 17.01.1942), \emph{Schriftsteller}|pw}
               erfahre ich, dass Sie jetzt in Berlin\oindex{Berlin@\textbf{Berlin}|pw} sind, und
               da fällt es mir ein, ob Sie nicht jetzt Gelegenheit hätten, mit Fischer\pwindex{Fischer, Samuel 24.12.1859 – 15.10.1934@\textsc{Fischer, Samuel} (24.12.1859 – 15.10.1934), \emph{Verleger}|pw} ein Wort über meine Novellen\pwindex{Salten, Felix 06.09.1869 – 08.10.1945@\textsc{Salten, Felix} (06.09.1869 – 08.10.1945), \emph{Schriftsteller, Journalist}!Hinterbliebene. Kurze Novellen1900@\strich\emph{Der Hinterbliebene. Kurze Novellen} {[}1900{]}|pwv} zu sprechen, d. h. wenn es Ihnen sonst passt, und
               wenn es im Übrigen Ihre Meinung ist, dass die \label{K_L03301-1v}\edtext{Novellen gut}{\lemma{\textnormal{\emph{Novellen gut}}}\Cendnote{\textnormal{siehe Arthur Schnitzler an Felix Salten, 4. 9. 1899}}}\label{K_L03301-1h} sind. Um was ich Sie bitten würde, wäre eben nicht die »Empfehlung«, sondern,
               wenn die übrigen Umstände es erlauben, eine intensivere Intervention. Ich möchte \uline{sehr gerne} bei Fischer\orgindex{S. Fischer Verlag@S. Fischer Verlag|pw} verlegt werden, möchte aber von Fischer\pwindex{Fischer, Samuel 24.12.1859 – 15.10.1934@\textsc{Fischer, Samuel} (24.12.1859 – 15.10.1934), \emph{Verleger}|pw} keinen Korb bekommen. Vielleicht macht es etwas bei ihm aus, wenn
               Sie ihm sagen, dass in den \label{K_L03301-11v}\edtext{nächsten
               Monaten ein Stück\pwindex{Salten, Felix 06.09.1869 – 08.10.1945@\textsc{Salten, Felix} (06.09.1869 – 08.10.1945), \emph{Schriftsteller, Journalist}!Schoene Seelen. Komoedie in einem Akt1902@\strich\emph{Schöne Seelen. Komödie in einem Akt} {[}1902{]}|pwuv} von mir am Volkstheater\orgindex{Volkstheater@Volkstheater|pw}}{\lemma{\textnormal{\emph{nächsten … Volkstheater}}}\Cendnote{\textnormal{Es könnte sich um den Einakter \emph{Schöne Seelen}\pwindex{Salten, Felix 06.09.1869 – 08.10.1945@\textsc{Salten, Felix} (06.09.1869 – 08.10.1945), \emph{Schriftsteller, Journalist}!Schoene Seelen. Komoedie in einem Akt1902@\strich\emph{Schöne Seelen. Komödie in einem Akt} {[}1902{]}|pwk} handeln, von dem am
                     21. 2. 1901 gemeldet wurde, dass er am \emph{Deutschen Volkstheater}\orgindex{Volkstheater@Volkstheater|pwk} angenommen sei, der dann aber hier
                  nicht inszeniert wurde.}}}\label{K_L03301-11h} kommt. Bitte, schreiben Sie mir ein Wort, ob Sie
               das thun können, nur bitte, sagen Sie {\pb}es mir ganz ruhig, wenn Sie's
               aus irgend einem Grund nicht gerne thun möchten. \pend
           \pstart
           Hoffentlich sind Sie bald wieder hier. \label{K_L03301-111v}\edtext{Hirschfeld\pwindex{Hirschfeld, Georg 11.02.1873 – 17.01.1942@\textsc{Hirschfeld, Georg} (11.02.1873 – 17.01.1942), \emph{Schriftsteller}|pw} erzählt nur, dass er Sie ganz
               erfüllt von Arbeit angetroffen}{\lemma{\textnormal{\emph{Hirschfeld … angetroffen}}}\Cendnote{\textnormal{vgl. A. S.: \emph{Tagebuch}, 5. 10. 1899}}}\label{K_L03301-111h} hat; ich freue mich sehr darüber. \pend
           \pstart
           Herzlichst Ihr {\\[\baselineskip]}\spacefill\mbox{Salten}\pend
           \leftskip=0em{}
         
         \endnumbering\mylabel{h}\end{ledgroupsized}\begin{anhang}\end{anhang}\newcommand{\dateiname}{L03301}\newcommand{\titel}{Felix Salten an Arthur Schnitzler, 9. 10. 1899}\newcommand{\editorInnen}{Martin Anton Müller und Laura Untner}%% latex-leseansicht-abspann.tex
%% Abspann für die Leseansicht.
%% Der Schalter \ifkorrekturansicht ist bereits durch den Vorspann gesetzt.

%% latex-abspann.tex
%% Gemeinsamer Abspann für Korrekturansicht und Leseansicht.
%% Setzt den Schalter \ifkorrekturansicht voraus (gesetzt in den
%% einbindenden Dateien latex-korrekturansicht-abspann.tex bzw.
%% latex-leseansicht-abspann.tex).
%% ---------------------------------------------------------------

\normalsize

% Das esempio-Environment wird nur in der Leseansicht benötigt
\ifkorrekturansicht\else
\newenvironment{esempio}[3]%
{
    \vspace{1.5ex}
    \rlap{\underline{#1}}
    \par
    \setlength{\parindent}{0cm}
    \nopagebreak
    \leftskip=#2cm
    \rightskip=#3cm
}
{
    \par
}
\fi

\doendnotes{C}
\bigskip
\vfill

\clearpage

\footnotesize

\ifkorrekturansicht
  \lohead{\textsc{register}}
\fi

% theindex-Environment neu definieren ohne reledmac
\makeatletter
\renewenvironment{theindex}{%
  \ifkorrekturansicht
    \section*{\indexname}%
  \else
    \subsubsection*{Index der erwähnten Entitäten}%
  \fi
  \setlength{\parindent}{0pt}%
  \setlength{\parskip}{0pt plus 0.3pt}%
  \let\item\@idxitem
}{%
  \ifkorrekturansicht\clearpage\fi
}
\makeatother

\IfFileExists{\jobname-pw.ind}{\input{\jobname-pw.ind}}{}

% Quellenangabe nur in der Leseansicht
\ifkorrekturansicht\else
% Fallback-Definitionen, falls die .tex-Datei \titel etc. nicht gesetzt hat
\providecommand{\titel}{}
\providecommand{\editorInnen}{}
\providecommand{\dateiname}{\jobname}

\vspace{3cm}

\vfill

\footnotesize
\textsc{Quelle}: \titel. Herausgegeben von {\editorInnen}. In: \emph{Arthur Schnitzler: Briefwechsel mit Autorinnen und Autoren}.
 Digitale Edition, https://schnitzler-briefe.acdh.oeaw.ac.at/{\dateiname}.html (Stand \today)
\fi

\end{document}


      