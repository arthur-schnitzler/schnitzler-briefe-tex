%% latex-korrekturansicht-vorspann.tex
%% Vorspann für die Korrekturansicht.
%% Lädt die gemeinsame Datei latex-vorspann.tex mit gesetztem Schalter.

\newif\ifkorrekturansicht
\korrekturansichttrue

\input{../tex-inputs/latex-vorspann}


\section[ Felix Salten an Arthur Schnitzler, 9. 10. 1899]{L03301 Felix Salten an Arthur Schnitzler, 9. 10. 1899}
\nopagebreak\mylabel{L03301v}
\rehead{ }\normalsize\beginnumbering\briefempfaengerindex{Schnitzler, Arthur@\textsc{Schnitzler, Arthur}!zzzSalten, Felix@\emph{von Felix Salten}!1899-10-091@{9. 10. 1899}|(be}
\toendnotes[C]{\smallbreak\pagebreak[2]}\Standort{CUL, Schnitzler, B 89, A 2.}
\physDesc{Brief, 1 Blatt, 2 Seiten, 970 Zeichen
\newline{}Handschrift: schwarze Tinte, lateinische Kurrent
\newline{}Ordnung: mit Bleistift von unbekannter Hand nummeriert: »125« }\toendnotes[C]{\smallbreak}
\pstart
           \raggedleft{}{\pb}Wien\oindex{Wien@\textbf{Wien}, \emph{A.ADM2}|pw}, 9. X. 99.\pend
           \vspace{0.5em}
\pstart
           Lieber Freund, von Hirschfeld\pwindex{Hirschfeld, Georg 11.02.1873 – 17.01.1942@\textsc{Hirschfeld, Georg} (11.02.1873 – 17.01.1942), \emph{Schriftsteller/Schriftstellerin}|pw}
               erfahre ich, dass Sie jetzt in \label{K_L03301-1v}\edtext{Berlin\oindex{Berlin@\textbf{Berlin}, \emph{P.PPLC}|pw}}{\lemma{\textnormal{\emph{Berlin}}}\Cendnote{\textnormal{Schnitzler war zwischen 4. 10. 1899 und 11. 10. 1899 in Berlin\oindex{Berlin@\textbf{Berlin}, \emph{P.PPLC}|pwk}.}}}\label{K_L03301-1} sind, und da fällt es mir ein,
               ob Sie nicht jetzt Gelegenheit hätten, mit Fischer\pwindex{Fischer, Samuel 24.12.1859 – 15.10.1934@\textsc{Fischer, Samuel} (24.12.1859 – 15.10.1934), \emph{Verleger/Verlegerin}|pw} ein Wort über meine Novellen\pwindex{Hinterbliebene. Kurze Novellen@\emph{Der Hinterbliebene. Kurze Novellen}|pwv} zu sprechen, d. h. wenn es Ihnen sonst passt, und
               wenn es im Übrigen Ihre Meinung ist, dass die \label{K_L03301-2v}\edtext{Novellen gut}{\lemma{\textnormal{\emph{Novellen gut}}}\Cendnote{\textnormal{Siehe Arthur Schnitzler an Felix Salten, 4. 9. 1899.
               }}}\label{K_L03301-2} sind. Um was ich Sie bitten würde, wäre eben nicht die »Empfehlung«, sondern,
               wenn die übrigen Umstände es erlauben, eine intensivere Intervention. Ich möchte \uline{sehr gerne}{ }\label{K_L03301-3v}\edtext{bei Fischer\orgindex{S. Fischer Verlag@S. Fischer Verlag|pw} verlegt}{\lemma{\textnormal{\emph{bei Fischer verlegt}}}\Cendnote{\textnormal{Siehe Felix Salten an Arthur Schnitzler, [29. 8. 1899].
               }}}\label{K_L03301-3} werden, möchte aber von Fischer\pwindex{Fischer, Samuel 24.12.1859 – 15.10.1934@\textsc{Fischer, Samuel} (24.12.1859 – 15.10.1934), \emph{Verleger/Verlegerin}|pw} keinen
               Korb bekommen. Vielleicht macht es etwas bei ihm aus, wenn Sie ihm sagen, dass in den
                  \label{K_L03301-4v}\edtext{nächsten Monaten ein Stück\pwindex{Gemeine. Schauspiel in drei Aufzuegen@\emph{Der Gemeine. Schauspiel in drei Aufzügen}|pwv} von mir am Volkstheater\orgindex{Volkstheater@Volkstheater|pw}}{\lemma{\textnormal{\emph{nächsten … Volkstheater}}}\Cendnote{\textnormal{Das \emph{Deutsche Volkstheater}\orgindex{Volkstheater@Volkstheater|pwk} hatte \emph{Der
                     Gemeine}\pwindex{Gemeine. Schauspiel in drei Aufzuegen@\emph{Der Gemeine. Schauspiel in drei Aufzügen}|pwk} angenommen. Aufgrund der antimilitaristischen Haltung des Stücks\pwindex{Gemeine. Schauspiel in drei Aufzuegen@\emph{Der Gemeine. Schauspiel in drei Aufzügen}|pwkv} wurde es in Österreich\oindex{Oesterreich@\textbf{Österreich}, \emph{A.PCLI}|pwk} jedoch erst 1919 aufgeführt.}}}\label{K_L03301-4} kommt. Bitte, schreiben Sie mir ein Wort, ob
               Sie das thun können, nur bitte, sagen Sie {\pb}es mir ganz ruhig, wenn Sie’s
               aus irgend einem Grund nicht gerne thun möchten.\pend
           
\pstart
           Hoffentlich sind Sie \label{K_L03301-5v}\edtext{bald wieder
                  hier}{\lemma{\textnormal{\emph{bald wieder
                  hier}}}\Cendnote{\textnormal{Schnitzler kehrte am 12. 10. 1899 nach Wien\oindex{Wien@\textbf{Wien}, \emph{A.ADM2}|pwk} zurück.}}}\label{K_L03301-5}. \label{K_L03301-6v}\edtext{Hirschfeld\pwindex{Hirschfeld, Georg 11.02.1873 – 17.01.1942@\textsc{Hirschfeld, Georg} (11.02.1873 – 17.01.1942), \emph{Schriftsteller/Schriftstellerin}|pw} erzählt nur, dass er Sie ganz
               erfüllt von Arbeit angetroffen}{\lemma{\textnormal{\emph{Hirschfeld … angetroffen}}}\Cendnote{\textnormal{Siehe A. S.: \emph{Tagebuch}, 5. 10. 1899.
               }}}\label{K_L03301-6} hat; ich freue mich sehr darüber.\pend
           
\pstart
           Herzlichst Ihr {\\[\baselineskip]}\spacefill\mbox{Salten}\pend
           \leftskip=0em{}\selectlanguage{ngerman}\endnumbering\briefempfaengerindex{Schnitzler, Arthur@\textsc{Schnitzler, Arthur}!zzzSalten, Felix@\emph{von Felix Salten}!1899-10-091@{9. 10. 1899}|)be}\mylabel{L03301h}  \normalsize

\doendnotes{C}
\bigskip
\vfill

\clearpage

\footnotesize

\lohead{\textsc{register}}

% Definiere theindex-Environment komplett neu ohne reledmac
\makeatletter
\renewenvironment{theindex}{%
  \section*{\indexname}%
  \setlength{\parindent}{0pt}%
  \setlength{\parskip}{0pt plus 0.3pt}%
  \let\item\@idxitem
}{%
  \clearpage
}
\makeatother

\IfFileExists{\jobname-pw.ind}{\input{\jobname-pw.ind}}{}

\end{document}

      