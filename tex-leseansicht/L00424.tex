%% latex-korrekturansicht-vorspann.tex
%% Vorspann für die Korrekturansicht.
%% Lädt die gemeinsame Datei latex-vorspann.tex mit gesetztem Schalter.

\newif\ifkorrekturansicht
\korrekturansichttrue

\input{../tex-inputs/latex-vorspann}


\section[Arthur Schnitzler an Richard Beer-Hofmann, 9. 3. 1895]{L00424 Arthur Schnitzler an Richard Beer-Hofmann, 9. 3. 1895}
\nopagebreak\mylabel{L00424v}
\rehead{ }\normalsize\beginnumbering\briefempfaengerindex{Beer-Hofmann, Richard@\textsc{Beer-Hofmann, Richard}!zzzSchnitzler, Arthur@\emph{von Arthur Schnitzler}!1895-03-091@{9. 3. 1895}|(be}
\toendnotes[C]{\smallbreak\pagebreak[2]}\Standort{YCGL, MSS 31.}
\physDesc{Postkarte, 263 Zeichen
\newline{}Handschrift: Bleistift, deutsche Kurrent
\newline{}Versand: 1) Rohrpost  2) Stempel: »\nobreak{}\oindex{IX., Alsergrund@\textbf{IX., Alsergrund}, \emph{A.ADM3}|pwk}Wien 9/1, 9 III 9{[}5{]}, 3 10N\nobreak{}«.  3) Stempel: »\nobreak{}\oindex{I., Innere Stadt@\textbf{I., Innere Stadt}, \emph{A.ADM3}|pwk}Wien 1/1, 9 III 9{[}5{]}, 3 40N\nobreak{}«. }\toendnotes[C]{\smallbreak}\pstart{}{\pb}Herrn \textsc{Dr. Richard
                     Beer-Hofmann}\pend{}\pstart{}Wien\oindex{Wien@\textbf{Wien}, \emph{A.ADM2}|pw}\pend{}\pstart{}\textsc{I. Wollzeile 15\oindex{Wollzeile@\textbf{Wollzeile}, \emph{Straße (K.STR)}|pw}}.\pend{}{\bigskip}\vspace{1em}
\pstart
           \noindent{}{\pb}Lieber Richard. Wir haben Sitze für das
                  \uline{Abſchiedsconcert \textsc{Hubermann\pwindex{Huberman, Bronisław 19.12.1882 – 16.6.1947@\textsc{Huberman, Bronisław} (19.12.1882 – 16.6.1947), \emph{Schriftsteller/Schriftstellerin, Musiker/Musikerin, Violinist/Violinistin}|pw}}{ }29. März.}\pend
           
\pstart
           – \label{K_L00424-1v}\edtext{Dinſtag}{\lemma{\textnormal{\emph{Dinstag}}}\Cendnote{\textnormal{Wegen Erkrankung musste die Aufführung
                  kurzfristig um zwei Tage auf den 14. 3. 1895 verschoben werden, Schnitzler nahm trotzdem teil.}}}\label{K_L00424-1} geh ich
               mit Ihnen zu \textsc{Feodora}\pwindex{Fedora@\emph{Fédora}|pw}. Heute bin ich bei \textsc{Julius Caesar}\pwindex{Julius Caesar@\emph{Julius Caesar}|pw} in der Burg\oindex{Burgtheater@\textbf{Burgtheater}, \emph{S.THTR}|pw}, nachher im \textsc{Café}, wo ich Sie zu{ }ſehen hoffe –\pend
           \pstart Herzlich Ihr \spacefill\mbox{Arthur}\pend{}\selectlanguage{ngerman}\endnumbering\briefempfaengerindex{Beer-Hofmann, Richard@\textsc{Beer-Hofmann, Richard}!zzzSchnitzler, Arthur@\emph{von Arthur Schnitzler}!1895-03-091@{9. 3. 1895}|)be}\mylabel{L00424h}  \normalsize

\doendnotes{C}
\bigskip
\vfill

\clearpage

\footnotesize

\lohead{\textsc{register}}

% Definiere theindex-Environment komplett neu ohne reledmac
\makeatletter
\renewenvironment{theindex}{%
  \section*{\indexname}%
  \setlength{\parindent}{0pt}%
  \setlength{\parskip}{0pt plus 0.3pt}%
  \let\item\@idxitem
}{%
  \clearpage
}
\makeatother

\IfFileExists{\jobname-pw.ind}{\input{\jobname-pw.ind}}{}

\end{document}

      