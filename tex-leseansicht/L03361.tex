%% latex-korrekturansicht-vorspann.tex
%% Vorspann für die Korrekturansicht.
%% Lädt die gemeinsame Datei latex-vorspann.tex mit gesetztem Schalter.

\newif\ifkorrekturansicht
\korrekturansichttrue

\input{../tex-inputs/latex-vorspann}


\section[ Paul Goldmann an Arthur Schnitzler, 27. 1. {[}1903{]}]{L03361 Paul Goldmann an Arthur Schnitzler, 27. 1. {[}1903{]}}
\nopagebreak\mylabel{L03361v}
\rehead{ }\normalsize\beginnumbering\briefempfaengerindex{Schnitzler, Arthur@\textsc{Schnitzler, Arthur}!zzzGoldmann, Paul@\emph{von Paul Goldmann}!1903-01-273@{27. 1. {[}1903{]}}|(be}
\toendnotes[C]{\smallbreak\pagebreak[2]}\Standort{DLA, A:Schnitzler, HS.NZ85.1.3173.}
\physDesc{Brief, 1 Blatt, 2 Seiten, 638 Zeichen
\newline{}Handschrift: blaue Tinte, deutsche Kurrent
\newline{}Schnitzler: mit Bleistift das Jahr »903.« vermerkt }\toendnotes[C]{\smallbreak}
\pstart
           \raggedleft{}{\pb}\textcolor{gray}{\textbf{DESSAUERSTRASSE 19\oindex{Dessauer Strasse@\textbf{Dessauer Straße}, \emph{Straße (K.STR)}|pw}}}\pend
           
\pstart
           Berlin\oindex{Berlin@\textbf{Berlin}, \emph{P.PPLC}|pw}, 27. Januar.\pend
           
\pstart\center{}Mein lieber Freund,\pend\vspace{0.5em}
\pstart
           Ich \substVorne{}\textsuperscript{\textcolor{gray}{freu}}\substDazwischen{}habe\substHinten{} ſo viel zu thun, daß ich Dir nur in aller Eile für Deinen lieben Brief
               danken kann, der mich unendlich erfreut hat. Wann kommſt Du nach \label{K_L03361-1v}\edtext{Berlin\oindex{Berlin@\textbf{Berlin}, \emph{P.PPLC}|pw}}{\lemma{\textnormal{\emph{Berlin}}}\Cendnote{\textnormal{Schnitzler war vom 22. 2. 1903 bis zum 9. 3. 1903 in Berlin\oindex{Berlin@\textbf{Berlin}, \emph{P.PPLC}|pwk}. In dieser Zeit wohnte er im Palasthotel\oindex{Palasthotel Berlin@\textbf{Palasthotel Berlin}, \emph{Hotel (K.HTL)}|pwk}.}}}\label{K_L03361-1}? Ich ſehne mich danach, mit
               Dir all’ das zu beſprechen, was mir das Herz bedrückt. Ich bin ſeit Wochen in einem
               unbeſchreiblichen Zuſtande, gequält von Vorwürfen, Reue und Sehnſucht, die niemals
                  {\pb}wieder befriedigt werden wird. Vielleicht kannſt
               Du mir ein tröſtendes und beruhigendes Wort ſagen. Mit dem Direktor\pwindex{Gutscher, Eduard @\textsc{Gutscher, Eduard}, \emph{Hotelbesitzer/Hotelbesitzerin}|pwv} des »Palaſthotel\orgindex{Palasthotel Berlin@Palasthotel Berlin|pw}« habe ich geſprochen; er hat Dir wohl inzwiſchen
               ſelbſt geſchrieben.\pend
           
\pstart
           Herzlichſte Grüße Dir und Olga\pwindex{Schnitzler, Olga 17.01.1882 – 13.01.1970@\textsc{Schnitzler, Olga} (17.01.1882 – 13.01.1970), \emph{Schauspieler/Schauspielerin, Sänger/Sängerin}|pw}! {\\[\baselineskip]}Dein
               getreuer {\\[\baselineskip]}\spacefill\mbox{Paul Goldmn}\pend
           \leftskip=0em{}\selectlanguage{ngerman}\endnumbering\briefempfaengerindex{Schnitzler, Arthur@\textsc{Schnitzler, Arthur}!zzzGoldmann, Paul@\emph{von Paul Goldmann}!1903-01-273@{27. 1. {[}1903{]}}|)be}\mylabel{L03361h}  \normalsize

\doendnotes{C}
\bigskip
\vfill

\clearpage

\footnotesize

\lohead{\textsc{register}}

% Definiere theindex-Environment komplett neu ohne reledmac
\makeatletter
\renewenvironment{theindex}{%
  \section*{\indexname}%
  \setlength{\parindent}{0pt}%
  \setlength{\parskip}{0pt plus 0.3pt}%
  \let\item\@idxitem
}{%
  \clearpage
}
\makeatother

\IfFileExists{\jobname-pw.ind}{\input{\jobname-pw.ind}}{}

\end{document}

      