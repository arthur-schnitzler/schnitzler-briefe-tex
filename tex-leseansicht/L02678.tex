%% latex-korrekturansicht-vorspann.tex
%% Vorspann für die Korrekturansicht.
%% Lädt die gemeinsame Datei latex-vorspann.tex mit gesetztem Schalter.

\newif\ifkorrekturansicht
\korrekturansichttrue

\input{../tex-inputs/latex-vorspann}


\section[Thomas Mann an Arthur Schnitzler, 13. 5. {[}1912?{]}]{L02678 Thomas Mann an Arthur Schnitzler, 13. 5. {[}1912?{]}}
\nopagebreak\mylabel{L02678v}
\rehead{ }\normalsize\beginnumbering\briefempfaengerindex{Schnitzler, Arthur@\textsc{Schnitzler, Arthur}!zzzMann, Thomas@\emph{von Thomas Mann}!1912-05-132@{13. 5. {[}1912?{]}}|(be}
\toendnotes[C]{\smallbreak\pagebreak[2]}\Standort{DLA, A:Schnitzler, HS.1985.1.5560.}
\physDesc{Telegramm, 107 Zeichen
\newline{}Handschrift einer Schreibkraft: Bleistift, lateinische Kurrent
\newline{}Versand: »\noindent{}München\oindex{Muenchen@\textbf{München}, \emph{P.PPLA}|pw}{ }\textcolor{gray}{\textbf{Nr.}} 184 \textcolor{gray}{\textbf{Taxw.}} 21 \textcolor{gray}{\textbf{aufgegeben am}}{ }13/5{ }\textcolor{gray}{\textbf{um}}{ }7 \textcolor{gray}{\textbf{Uhr}} 28{ }\textsuperscript{n}\textcolor{gray}{\textbf{Mittag.}}{ / }\textcolor{gray}{\textbf{Aufgenommen von .......... am}}{ }14/5{ }\textcolor{gray}{\textbf{um}}{ }8\textsuperscript{50}« }
\buchAbdrucke{\weitereDrucke{Hans-Ulrich Lindken: \emph{Arthur Schnitzler. Aspekte und Akzente. Materialien zu Leben
                        und Werk}. Frankfurt am Main, Bern, Göttingen: \emph{Peter Lang} 1984, S. 408.} }\toendnotes[C]{\smallbreak}
\pstart
           \noindent{}{\pb}nehmen sie auch von mir{[},{]}
               Verehrter Herr, die Herzlichsten Glückwünsche zum Bevorstehenden \label{K_L02678-1v}\edtext{Fest{[}t{]}age}{\lemma{\textnormal{\emph{Festtage}}}\Cendnote{\textnormal{der 50. Geburtstag am 15. 5. 1912}}}\label{K_L02678-1}\pend
           \pstart \spacefill\mbox{T{[}h{]}omas Mann}\pend{}\selectlanguage{ngerman}\endnumbering\briefempfaengerindex{Schnitzler, Arthur@\textsc{Schnitzler, Arthur}!zzzMann, Thomas@\emph{von Thomas Mann}!1912-05-132@{13. 5. {[}1912?{]}}|)be}\mylabel{L02678h}  \normalsize

\doendnotes{C}
\bigskip
\vfill

\clearpage

\footnotesize

\lohead{\textsc{register}}

% Definiere theindex-Environment komplett neu ohne reledmac
\makeatletter
\renewenvironment{theindex}{%
  \section*{\indexname}%
  \setlength{\parindent}{0pt}%
  \setlength{\parskip}{0pt plus 0.3pt}%
  \let\item\@idxitem
}{%
  \clearpage
}
\makeatother

\IfFileExists{\jobname-pw.ind}{\input{\jobname-pw.ind}}{}

\end{document}

      