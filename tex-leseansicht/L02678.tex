\input{../tex-inputs/latex-pdf-vorspann}
\begin{center}
            \textcolor{red}{ENTWURF. ENTZIFFERUNG NOCH NICHT KORREKTURGELESEN}
                      \end{center}
            
               \section[Thomas Mann an Arthur Schnitzler, {[}15. 5. 1912?{]}]{ Thomas Mann an Arthur Schnitzler, {[}15. 5. 1912?{]}}\nopagebreak\mylabel{v}\rehead{ }\begin{ledgroupsized}[t]{13cm}\normalsize\beginnumbering\briefempfaengerindex{Schnitzler, Arthur@\textsc{Schnitzler, Arthur}!zzzMann, Thomas@\emph{von Thomas Mann}!1912-05-151@{{[}15. 5. 1912?{]}}|(be} \toendnotes[C]{\smallbreak\pagebreak[2]} \Standort{DLA, A:Schnitzler, HS.1985.1.5560.}
\physDesc{Telegramm
\newline{}maschinell}\buchAbdrucke{\weitereDrucke{Hans-Ulrich Lindken: \emph{Arthur Schnitzler. Aspekte und Akzente. Materialien zu Leben
                        und Werk}. Frankfurt am Main, Bern, Göttingen: \emph{Peter Lang} 1984, S. 408 (Europäische Hochschulschriften, Reihe 1, Deutsche Sprache und
                        Literatur, 754).} }\pstart
           \noindent{}{\pb}Nehmen Sie auch von mir, verehrter Herr, die herzlichsten Glückwünsche zum bevorstehenden Festtage\pend
           \pstart \spacefill\mbox{Thomas Mann}\pend{}\endnumbering\briefempfaengerindex{Schnitzler, Arthur@\textsc{Schnitzler, Arthur}!zzzMann, Thomas@\emph{von Thomas Mann}!1912-05-151@{{[}15. 5. 1912?{]}}|)be}\mylabel{h}\end{ledgroupsized}\begin{anhang}\end{anhang}\newcommand{\dateiname}{L02678}\newcommand{\titel}{Thomas Mann an Arthur Schnitzler, [15. 5. 1912?]}\newcommand{\editorInnen}{Martin Anton Müller und Laura Untner}\input{../tex-inputs/latex-pdf-abspann}
      