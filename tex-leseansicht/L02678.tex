%% latex-leseansicht-vorspann.tex
%% Vorspann für die Leseansicht.
%% Lädt die gemeinsame Datei latex-vorspann.tex mit nicht gesetztem Schalter.

\newif\ifkorrekturansicht
\korrekturansichtfalse

\input{../tex-inputs/latex-vorspann}


\section[Thomas Mann an Arthur Schnitzler, 13. 5. [1912?]]{L02678 Thomas Mann an Arthur Schnitzler, 13. 5. [1912?]}
\nopagebreak\mylabel{L02678v}
\rehead{ }\normalsize\beginnumbering\briefempfaengerindex{Schnitzler, Arthur@\textsc{Schnitzler, Arthur}!zzzMann, Thomas@\emph{von Thomas Mann}!1912-05-132@{13. 5. [1912?]}|(be}
\toendnotes[C]{\smallbreak\pagebreak[2]}
\correspDesc{Versand  durch Thomas Mann am 13. 5. [1912?] in München
\newline{}Erhalt  durch Arthur Schnitzler am 14. 5. [1912?] in Brijuni}\toendnotes[C]{\smallbreak}
\Standort{DLA, A:Schnitzler, HS.1985.1.5560.}
\physDesc{Telegramm, 107 Zeichen
\newline{}HandschriftX2 einer Schreibkraft: Bleistift, lateinische Kurrent
\newline{}Versand: »\noindent{}München\oindex{München@\textbf{München}|pw}{ }\textcolor{gray}{\textbf{Nr.}} 184 \textcolor{gray}{\textbf{Taxw.}} 21 \textcolor{gray}{\textbf{aufgegeben am}}{ }13/5{ }\textcolor{gray}{\textbf{um}}{ }7 \textcolor{gray}{\textbf{Uhr}} 28{ }\textsuperscript{n}\textcolor{gray}{\textbf{Mittag.}}{ / }\textcolor{gray}{\textbf{Aufgenommen von .......... am}}{ }14/5{ }\textcolor{gray}{\textbf{um}}{ }8\textsuperscript{50}« }
\buchAbdrucke{\weitereDrucke{Hans-Ulrich Lindken: \emph{Arthur Schnitzler. Aspekte und Akzente. Materialien zu Leben
                        und Werk}. Frankfurt am Main, Bern, Göttingen: \emph{Peter Lang} 1984, S. 408 (Europäische Hochschulschriften, Reihe 1, Deutsche Sprache und
                        Literatur, 754).} }\toendnotes[C]{\smallbreak}
\pstart
           \noindent{}{\pb}nehmen sie auch von mir{[},{]}
               Verehrter Herr, die Herzlichsten Glückwünsche zum Bevorstehenden \label{K_L02678-1v}\edtext{Fest{[}t{]}age}{\lemma{\textnormal{\emph{Festtage}}}\Cendnote{\textnormal{der 50. Geburtstag am 15. 5. 1912}}}\label{K_L02678-1}\pend
           \pstart \spacefill\mbox{T{[}h{]}omas Mann}\pend{}\selectlanguage{ngerman}\endnumbering\briefempfaengerindex{Schnitzler, Arthur@\textsc{Schnitzler, Arthur}!zzzMann, Thomas@\emph{von Thomas Mann}!1912-05-132@{13. 5. [1912?]}|)be}\mylabel{L02678h}  \newcommand{\dateiname}{L02678}\newcommand{\titel}{Thomas Mann an Arthur Schnitzler, 13. 5. [1912?]}\newcommand{\editorInnen}{Martin Anton Müller und Gerd-Hermann Susen}%% latex-leseansicht-abspann.tex
%% Abspann für die Leseansicht.
%% Der Schalter \ifkorrekturansicht ist bereits durch den Vorspann gesetzt.

%% latex-abspann.tex
%% Gemeinsamer Abspann für Korrekturansicht und Leseansicht.
%% Setzt den Schalter \ifkorrekturansicht voraus (gesetzt in den
%% einbindenden Dateien latex-korrekturansicht-abspann.tex bzw.
%% latex-leseansicht-abspann.tex).
%% ---------------------------------------------------------------

\normalsize

% Das esempio-Environment wird nur in der Leseansicht benötigt
\ifkorrekturansicht\else
\newenvironment{esempio}[3]%
{
    \vspace{1.5ex}
    \rlap{\underline{#1}}
    \par
    \setlength{\parindent}{0cm}
    \nopagebreak
    \leftskip=#2cm
    \rightskip=#3cm
}
{
    \par
}
\fi

\doendnotes{C}
\bigskip
\vfill

\clearpage

\footnotesize

\ifkorrekturansicht
  \lohead{\textsc{register}}
\fi

% theindex-Environment neu definieren ohne reledmac
\makeatletter
\renewenvironment{theindex}{%
  \ifkorrekturansicht
    \section*{\indexname}%
  \else
    \subsubsection*{Index der erwähnten Entitäten}%
  \fi
  \setlength{\parindent}{0pt}%
  \setlength{\parskip}{0pt plus 0.3pt}%
  \let\item\@idxitem
}{%
  \ifkorrekturansicht\clearpage\fi
}
\makeatother

\IfFileExists{\jobname-pw.ind}{\input{\jobname-pw.ind}}{}

% Quellenangabe nur in der Leseansicht
\ifkorrekturansicht\else
% Fallback-Definitionen, falls die .tex-Datei \titel etc. nicht gesetzt hat
\providecommand{\titel}{}
\providecommand{\editorInnen}{}
\providecommand{\dateiname}{\jobname}

\vspace{3cm}

\vfill

\footnotesize
\textsc{Quelle}: \titel. Herausgegeben von {\editorInnen}. In: \emph{Arthur Schnitzler: Briefwechsel mit Autorinnen und Autoren}.
 Digitale Edition, https://schnitzler-briefe.acdh.oeaw.ac.at/{\dateiname}.html (Stand \today)
\fi

\end{document}


