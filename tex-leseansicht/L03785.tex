%% latex-korrekturansicht-vorspann.tex
%% Vorspann für die Korrekturansicht.
%% Lädt die gemeinsame Datei latex-vorspann.tex mit gesetztem Schalter.

\newif\ifkorrekturansicht
\korrekturansichttrue

\input{../tex-inputs/latex-vorspann}


\section[Arthur Schnitzler an Stefan Zweig, 18. 3. 1913]{L03785 Arthur Schnitzler an Stefan Zweig, 18. 3. 1913}
\nopagebreak\mylabel{L03785v}
\rehead{ }\normalsize\beginnumbering\briefempfaengerindex{Zweig, Stefan@\textsc{Zweig, Stefan}!zzzSchnitzler, Arthur@\emph{von Arthur Schnitzler}!1913-03-181@{18. 3. 1913}|(be}
\toendnotes[C]{\smallbreak\pagebreak[2]}\Standort{Jerusalem, National Library of Israel, ARC. Ms. Var. 305 1 58 Stefan Zweig Collection.}
\physDesc{Briefkarte, 1 Blatt, 2 Seiten, 1237 Zeichen
\newline{}Schreibmaschine
\newline{}Handschrift: schwarze Tinte, lateinische Kurrent (\noindent{}Korrekturen und Unterschrift)}\toendnotes[C]{\smallbreak}
\pstart
           {\pb}\textcolor{gray}{\textbf{Dr. Arthur Schnitzler}}\hfill 18. 3. 1913.\pend
           
\pstart
           \textcolor{gray}{\textbf{Wien XVIII.
                        Sternwartestrasse 71\oindex{Sternwartestrasse 71@\textbf{Sternwartestraße 71}, \emph{Wohngebäude (K.WHS)}|pw}}}\pend
           
\pstart\center{}Lieber Herr Dr. Zweig.\pend\vspace{0.5em}
\pstart
           Seien Sie vielmals bedankt für Ihre Bemühungen in meiner Sache. Wenn für das »Weite Land\pwindex{weite Land. Tragikomoedie in fuenf Akten@\emph{Das weite Land. Tragikomödie in fünf Akten}|pw}« kein Theater übrig bleibt als das Des Arts\orgindex{Theâtre Hebertot@Théâtre Hébertot|pw}, so würde ich es natürlich auch
               akzeptieren, vorausgesetzt dass ich auf eine gute Darstellung rechnen könnte. Ein
               Erfolg desStücks\pwindex{weite Land. Tragikomoedie in fuenf Akten@\emph{Das weite Land. Tragikomödie in fünf Akten}|pwv} in Paris\oindex{Paris@\textbf{Paris}, \emph{P.PPLC}|pw} ist meines Erachtens nur möglich, wenn
               insbesondere der Hofreiter\pwindex{weite Land. Tragikomoedie in fuenf Akten@\emph{Das weite Land. Tragikomödie in fünf Akten}|pwv}
               durch einen Schauspieler ersten Ranges dem Verständnis der Leute nahegebracht werden
               würde.\pend
           
\pstart
           Ich weiss nicht, ob das Theater Des Arts\orgindex{Theâtre Hebertot@Théâtre Hébertot|pw} über
               eine festengagierte Truppe verfügt oder sich mit Schauspielern von Fall zu Fall
               behilft; dass man etwa {\pb}Guitry\pwindex{Guitry, Sacha 21.01.1885 – 24.07.1957@\textsc{Guitry, Sacha} (21.01.1885 – 24.07.1957), \emph{Schriftsteller/Schriftstellerin, Regisseur/Regisseurin, Schauspieler/Schauspielerin}|pw} (der das Stück\pwindex{weite Land. Tragikomoedie in fuenf Akten@\emph{Das weite Land. Tragikomödie in fünf Akten}|pwv}
               hier gesehen hat und sich dafür interessieren soll) gewänne, ist wohl ausgeschlossen,
               – nicht wahr? Wenn das Erscheinen als Buch den Verzicht auf die Aufführung bedeutet,
               möchte ich davon doch lieber vorläufig absehen. Bitte sagen Sie auch Herrn M\substVorne{}\textsuperscript{au}\substDazwischen{}o\substHinten{}ri\substVorne{}\textsuperscript{\textcolor{gray}{c}e}\substDazwischen{}sse\substHinten{}, ich sei völlig überzeugt, dass er nichts unterl\substVorne{}\textsuperscript{ie}\substDazwischen{}ä\substHinten{}ss\substVorne{}\textsuperscript{e}\substDazwischen{}t\substHinten{}, was im Interesse unserer Komödie\pwindex{weite Land. Tragikomoedie in fuenf Akten@\emph{Das weite Land. Tragikomödie in fünf Akten}|pwv} liegen k\substVorne{}\textsuperscript{a}\substDazwischen{}ö\substHinten{}nn\substVorne{}\textsuperscript{.}\substDazwischen{}te.\substHinten{}\pend
           
\pstart
           Wir haben hier eine etwas unruhige Zeit hinter uns, daHeini\pwindex{Schnitzler, Heinrich 09.08.1902 – 12.07.1982@\textsc{Schnitzler, Heinrich} (09.08.1902 – 12.07.1982), \emph{Regisseur/Regisseurin, Schauspieler/Schauspielerin}|pw} an Blinddarm operiert worden und erst gestern \introOben{}(\introOben{}bei vortrefflichem Befinden\introOben{})\introOben{} aus
               dem Sanatorium\oindex{Krankenhaus der Wiener Kaufmannschaft@\textbf{Krankenhaus der Wiener Kaufmannschaft}, \emph{Krankenhaus (K.KKH)}|pwv}
               wiederheimgekehrt ist.\pend
           
\pstart
           Mit vielen Grüssen, auch von meiner Frau\pwindex{Schnitzler, Olga 17.01.1882 – 13.01.1970@\textsc{Schnitzler, Olga} (17.01.1882 – 13.01.1970), \emph{Schauspieler/Schauspielerin, Sänger/Sängerin}|pwv}{\\[\baselineskip]}Ihr aufrichtig ergebener{\\[\baselineskip]}\spacefill\mbox{{[}hs.:{]} Arthur Schnitzler}\pend
           \leftskip=0em{}\selectlanguage{ngerman}\endnumbering\briefempfaengerindex{Zweig, Stefan@\textsc{Zweig, Stefan}!zzzSchnitzler, Arthur@\emph{von Arthur Schnitzler}!1913-03-181@{18. 3. 1913}|)be}\mylabel{L03785h}
\begin{anhang}
\end{anhang}\normalsize

\doendnotes{C}
\bigskip
\vfill

\clearpage

\footnotesize

\lohead{\textsc{register}}

% Definiere theindex-Environment komplett neu ohne reledmac
\makeatletter
\renewenvironment{theindex}{%
  \section*{\indexname}%
  \setlength{\parindent}{0pt}%
  \setlength{\parskip}{0pt plus 0.3pt}%
  \let\item\@idxitem
}{%
  \clearpage
}
\makeatother

\IfFileExists{\jobname-pw.ind}{\input{\jobname-pw.ind}}{}

\end{document}

      