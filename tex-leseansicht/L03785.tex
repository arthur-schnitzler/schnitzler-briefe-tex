%% latex-leseansicht-vorspann.tex
%% Vorspann für die Leseansicht.
%% Lädt die gemeinsame Datei latex-vorspann.tex mit nicht gesetztem Schalter.

\newif\ifkorrekturansicht
\korrekturansichtfalse

\input{../tex-inputs/latex-vorspann}


\section[Arthur Schnitzler an Stefan Zweig, 18. 3. 1913]{L03785 Arthur Schnitzler an Stefan Zweig, 18. 3. 1913}
\nopagebreak\mylabel{L03785v}
\rehead{ }\normalsize\beginnumbering\briefempfaengerindex{Zweig, Stefan@\textsc{Zweig, Stefan}!zzzSchnitzler, Arthur@\emph{von Arthur Schnitzler}!1913-03-181@{18. 3. 1913}|(be}
\toendnotes[C]{\smallbreak\pagebreak[2]}
\correspDesc{Versand  durch Arthur Schnitzler am 18. 3. 1913 in Wien
\newline{}Erhalt  durch Stefan Zweig im Zeitraum [19. 3. 1913 – 20. 3. 1913?] in Paris}\toendnotes[C]{\smallbreak}
\Standort{Jerusalem, National Library of Israel, ARC. Ms. Var. 305 1 58 Stefan Zweig Collection.}
\physDesc{Briefkarte, 1237 Zeichen
\newline{}Schreibmaschine
\newline{}Handschrift: schwarze Tinte, lateinische Kurrent (\noindent{}Korrekturen und Unterschrift)}\toendnotes[C]{\smallbreak}
\pstart
           {\pb}\textcolor{gray}{\textbf{Dr. Arthur Schnitzler}}\hfill 18. 3. 1913.\pend
           
\pstart
           \textcolor{gray}{\textbf{Wien XVIII.
                        Sternwartestrasse 71\oindex{Wien@\textbf{Wien}!XVIII., Währing@\textbf{XVIII., Währing}!Sternwartestraße 71@\textbf{Sternwartestraße 71}, \emph{Wohngebäude}|pw}}}\pend
           
\pstart\center{}Lieber Herr Dr. Zweig.\pend\vspace{0.5em}
\pstart
           Seien Sie vielmals bedankt für Ihre Bemühungen in meiner Sache. Wenn für das »Weite Land\pwindex{Schnitzler, Arthur 15.\,5.\,1862 Wien – 21.\,10.\,1931 ebd.@\textsc{Schnitzler, Arthur} (15.\,5.\,1862 Wien – 21.\,10.\,1931 ebd.), \emph{Schriftsteller, Mediziner}!weite Land. Tragikomödie in fünf Akten@\strich\emph{Das weite Land. Tragikomödie in fünf Akten}|pw}« kein Theater übrig bleibt als das Des Arts\orgindex{Théâtre Hébertot@Théâtre Hébertot|pw}, so würde ich es natürlich auch
               akzeptieren, vorausgesetzt dass ich auf eine gute Darstellung rechnen könnte. Ein
               Erfolg des Stücks\pwindex{Schnitzler, Arthur 15.\,5.\,1862 Wien – 21.\,10.\,1931 ebd.@\textsc{Schnitzler, Arthur} (15.\,5.\,1862 Wien – 21.\,10.\,1931 ebd.), \emph{Schriftsteller, Mediziner}!weite Land. Tragikomödie in fünf Akten@\strich\emph{Das weite Land. Tragikomödie in fünf Akten}|pwv} in Paris\oindex{Paris@\textbf{Paris}, \emph{Hauptstadt}|pw} ist meines Erachtens nur möglich, wenn
               insbesondere der Hofreiter\pwindex{Schnitzler, Arthur 15.\,5.\,1862 Wien – 21.\,10.\,1931 ebd.@\textsc{Schnitzler, Arthur} (15.\,5.\,1862 Wien – 21.\,10.\,1931 ebd.), \emph{Schriftsteller, Mediziner}!weite Land. Tragikomödie in fünf Akten@\strich\emph{Das weite Land. Tragikomödie in fünf Akten}|pwv}
               durch einen Schauspieler ersten Ranges dem Verständnis der Leute nahegebracht werden
               würde. Ich weiss nicht, ob das Theater Des Arts\orgindex{Théâtre Hébertot@Théâtre Hébertot|pw} über
               eine festengagierte Truppe verfügt oder sich mit Schauspielern von Fall zu Fall
               behilft; dass man etwa {\pb}Guitry\pwindex{Guitry, Sacha 21.\,1.\,1885 Sankt Petersburg – 24.\,7.\,1957 Paris@\textsc{Guitry, Sacha} (21.\,1.\,1885 Sankt Petersburg – 24.\,7.\,1957 Paris), \emph{Schriftsteller, Regisseur, Schauspieler}|pw} (der das Stück\pwindex{Schnitzler, Arthur 15.\,5.\,1862 Wien – 21.\,10.\,1931 ebd.@\textsc{Schnitzler, Arthur} (15.\,5.\,1862 Wien – 21.\,10.\,1931 ebd.), \emph{Schriftsteller, Mediziner}!weite Land. Tragikomödie in fünf Akten@\strich\emph{Das weite Land. Tragikomödie in fünf Akten}|pwv}
               hier gesehen hat und sich dafür interessieren soll) gewänne, ist wohl ausgeschlossen,
               – nicht wahr? Wenn das Erscheinen als Buch den Verzicht auf die Aufführung bedeutet,
               möchte ich davon doch lieber vorläufig absehen. Bitte sagen Sie auch Herrn M\substVorne{}\textsuperscript{au}\substDazwischen{}o\substHinten{}ri\substVorne{}\textsuperscript{\textcolor{gray}{c}e}\substDazwischen{}sse\substHinten{}, ich sei völlig überzeugt, dass er nichts unterl\substVorne{}\textsuperscript{ie}\substDazwischen{}ä\substHinten{}ss\substVorne{}\textsuperscript{e}\substDazwischen{}t\substHinten{}, was im Interesse unserer Komödie\pwindex{Schnitzler, Arthur 15.\,5.\,1862 Wien – 21.\,10.\,1931 ebd.@\textsc{Schnitzler, Arthur} (15.\,5.\,1862 Wien – 21.\,10.\,1931 ebd.), \emph{Schriftsteller, Mediziner}!weite Land. Tragikomödie in fünf Akten@\strich\emph{Das weite Land. Tragikomödie in fünf Akten}|pwv} liegen k\substVorne{}\textsuperscript{a}\substDazwischen{}ö\substHinten{}nn\substVorne{}\textsuperscript{.}\substDazwischen{}te.\substHinten{}\pend
           
\pstart
           Wir haben hier eine etwas unruhige Zeit hinter uns, da Heini\pwindex{Schnitzler, Heinrich 9.\,8.\,1902 Hinterbrühl – 12.\,7.\,1982 Wien@\textsc{Schnitzler, Heinrich} (9.\,8.\,1902 Hinterbrühl – 12.\,7.\,1982 Wien), \emph{Regisseur, Schauspieler}|pw} an \label{K_L03785-1v}\edtext{Blinddarm}{\lemma{\textnormal{\emph{Blinddarm}}}\Cendnote{\textnormal{Bei Heinrich Schnitzler\pwindex{Schnitzler, Heinrich 9.\,8.\,1902 Hinterbrühl – 12.\,7.\,1982 Wien@\textsc{Schnitzler, Heinrich} (9.\,8.\,1902 Hinterbrühl – 12.\,7.\,1982 Wien), \emph{Regisseur, Schauspieler}|pwk} wurde am 
                  4. 3. 1913 eine 
                  Blinddarmentzündung diagnostiziert. Am 9. 3. 1913
                  wurde er von seinem Onkel Julius Schnitzler\pwindex{Schnitzler, Julius 13.\,7.\,1865 Wien – 29.\,6.\,1939 ebd.@\textsc{Schnitzler, Julius} (13.\,7.\,1865 Wien – 29.\,6.\,1939 ebd.), \emph{Chirurg}|pwk} operiert.}}}\label{K_L03785-1} operiert worden und erst gestern \introOben{}(\introOben{}bei vortrefflichem Befinden\introOben{})\introOben{} aus
               dem Sanatorium\oindex{Wien@\textbf{Wien}!XVIII., Währing@\textbf{XVIII., Währing}!Krankenhaus der Wiener Kaufmannschaft@\textbf{Krankenhaus der Wiener Kaufmannschaft}, \emph{Krankenhaus}|pwv}{ }\label{T_L03785-1v}\edtext{wieder heimgekehrt}{\lemma{\textnormal{\emph{wieder heimgekehrt}}}\Cendnote{\textnormal{Im Typoskript steht: »wiederheimgekehrt«.}}}\label{T_L03785-1} ist.\pend
           
\pstart
           Mit vielen Grüssen, auch von meiner Frau\pwindex{Schnitzler, Olga 17.\,1.\,1882 Wien – 13.\,1.\,1970 Lugano@\textsc{Schnitzler, Olga} (17.\,1.\,1882 Wien – 13.\,1.\,1970 Lugano), \emph{Schauspielerin, Sängerin}|pwv}{\\[\baselineskip]}Ihr aufrichtig ergebener{\\[\baselineskip]}\spacefill\mbox{{[}hs.:{]} Arthur Schnitzler}\pend
           \leftskip=0em{}\selectlanguage{ngerman}\endnumbering\briefempfaengerindex{Zweig, Stefan@\textsc{Zweig, Stefan}!zzzSchnitzler, Arthur@\emph{von Arthur Schnitzler}!1913-03-181@{18. 3. 1913}|)be}\mylabel{L03785h}  \newcommand{\dateiname}{L03785}\newcommand{\titel}{Arthur Schnitzler an Stefan Zweig, 18. 3. 1913}\newcommand{\editorInnen}{Selma Jahnke und Martin Anton Müller}%% latex-leseansicht-abspann.tex
%% Abspann für die Leseansicht.
%% Der Schalter \ifkorrekturansicht ist bereits durch den Vorspann gesetzt.

%% latex-abspann.tex
%% Gemeinsamer Abspann für Korrekturansicht und Leseansicht.
%% Setzt den Schalter \ifkorrekturansicht voraus (gesetzt in den
%% einbindenden Dateien latex-korrekturansicht-abspann.tex bzw.
%% latex-leseansicht-abspann.tex).
%% ---------------------------------------------------------------

\normalsize

% Das esempio-Environment wird nur in der Leseansicht benötigt
\ifkorrekturansicht\else
\newenvironment{esempio}[3]%
{
    \vspace{1.5ex}
    \rlap{\underline{#1}}
    \par
    \setlength{\parindent}{0cm}
    \nopagebreak
    \leftskip=#2cm
    \rightskip=#3cm
}
{
    \par
}
\fi

\doendnotes{C}
\bigskip
\vfill

\clearpage

\footnotesize

\ifkorrekturansicht
  \lohead{\textsc{register}}
\fi

% theindex-Environment neu definieren ohne reledmac
\makeatletter
\renewenvironment{theindex}{%
  \ifkorrekturansicht
    \section*{\indexname}%
  \else
    \subsubsection*{Index der erwähnten Entitäten}%
  \fi
  \setlength{\parindent}{0pt}%
  \setlength{\parskip}{0pt plus 0.3pt}%
  \let\item\@idxitem
}{%
  \ifkorrekturansicht\clearpage\fi
}
\makeatother

\IfFileExists{\jobname-pw.ind}{\input{\jobname-pw.ind}}{}

% Quellenangabe nur in der Leseansicht
\ifkorrekturansicht\else
% Fallback-Definitionen, falls die .tex-Datei \titel etc. nicht gesetzt hat
\providecommand{\titel}{}
\providecommand{\editorInnen}{}
\providecommand{\dateiname}{\jobname}

\vspace{3cm}

\vfill

\footnotesize
\textsc{Quelle}: \titel. Herausgegeben von {\editorInnen}. In: \emph{Arthur Schnitzler: Briefwechsel mit Autorinnen und Autoren}.
 Digitale Edition, https://schnitzler-briefe.acdh.oeaw.ac.at/{\dateiname}.html (Stand \today)
\fi

\end{document}


