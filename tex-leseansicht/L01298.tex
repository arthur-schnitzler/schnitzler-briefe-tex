%% latex-leseansicht-vorspann.tex
%% Vorspann für die Leseansicht.
%% Lädt die gemeinsame Datei latex-vorspann.tex mit nicht gesetztem Schalter.

\newif\ifkorrekturansicht
\korrekturansichtfalse

\input{../tex-inputs/latex-vorspann}


\section[Hermann Bahr: Widmungsexemplar Rezensionen für Arthur Schnitzler, [21.?] 6. 1903]{L01298 Hermann Bahr: Widmungsexemplar Rezensionen für Arthur Schnitzler, [21.?] 6. 1903}
\nopagebreak\mylabel{L01298v}
\rehead{ }\normalsize\beginnumbering\briefempfaengerindex{Schnitzler, Arthur@\textsc{Schnitzler, Arthur}!zzzBahr, Hermann@\emph{von Hermann Bahr}!1903-06-212@{[21.?] 6. 1903}|(be}
\toendnotes[C]{\smallbreak\pagebreak[2]}
\correspDesc{Versand  durch Hermann Bahr am [21.?] 6. 1903 in Wien
\newline{}Erhalt  durch Arthur Schnitzler im Zeitraum [21. 6. 1903
                  – 25. 6. 1903?] in Wien}\toendnotes[C]{\smallbreak}
\Standort{DLA, G:Schnitzler, Arthur (Sammlung Heinrich Schnitzler).}
\physDesc{Widmung am Schmutztitel, 53 Zeichen
\newline{}Handschrift: schwarze Tinte, deutsche Kurrent
\newline{}Ordnung: bei der Enteignung des Exemplars 1938 von
                                 unbekannter Hand mit Bleistift ergänzte Informationen:
                                    »Dubl. zu 426.317-B« }
\buchAbdrucke{\weitereDrucke{Hermann Bahr, Arthur Schnitzler: \emph{Briefwechsel, Aufzeichnungen, Dokumente (1891–1931)}. Herausgegeben von Kurt Ifkovits und Martin Anton Müller. Göttingen: \emph{Wallstein} 2018, S. 266.} }\toendnotes[C]{\smallbreak}
\pstart
           \noindent{}{\pb}Meinem lieben Arthur\pend
           
\pstart
           herzlichſt{\\[\baselineskip]}\spacefill\mbox{HermannB.}\pend
           \leftskip=0em{}
\pstart
           \noindent{}Juni 1903.\pend
           {\vspace{1\baselineskip}}
\pstart
           \centering{}\textcolor{gray}{\textbf{Rezenſionen\pwindex{Bahr, Hermann 19.\,7.\,1863 Linz – 15.\,1.\,1934 München@\textsc{Bahr, Hermann} (19.\,7.\,1863 Linz – 15.\,1.\,1934 München), \emph{Schriftsteller, Kritiker}!Rezensionen. Wiener Theater 1901 bis 1903@\strich\emph{Rezensionen. Wiener Theater 1901 bis 1903}|pw}}}\pend
           \selectlanguage{ngerman}\vspace{1em}{\vspace{1\baselineskip}}
\pstart
           \centering{}{\pb}\textcolor{gray}{\textbf{Hermann Bahr}}\pend
           
\pstart
           \centering{}\textcolor{gray}{\textbf{\textbf{Rezenſionen}\pwindex{Bahr, Hermann 19.\,7.\,1863 Linz – 15.\,1.\,1934 München@\textsc{Bahr, Hermann} (19.\,7.\,1863 Linz – 15.\,1.\,1934 München), \emph{Schriftsteller, Kritiker}!Rezensionen. Wiener Theater 1901 bis 1903@\strich\emph{Rezensionen. Wiener Theater 1901 bis 1903}|pw}}}\pend
           
\pstart
           \centering{}\textcolor{gray}{\textbf{Wiener Theater}}\pend
           
\pstart
           \centering{}\textcolor{gray}{\textbf{1901 bis 1903}}\pend
           {\vspace{1\baselineskip}}
\pstart
           \raggedleft{}\textcolor{gray}{\textbf{»Wenn die Leute glauben,{\\}ich
                     wäre noch in Weimar\oindex{Weimar@\textbf{Weimar}, \emph{Verwaltungsgebiet}|pw},{\\} dann bin ich{ }ſchon in Erfurt\oindex{Erfurt@\textbf{Erfurt}|pw}.«\pwindex{Goethe, Johann Wolfgang von 28.\,8.\,1749 Frankfurt am Main – 22.\,3.\,1832 Weimar@\textsc{Goethe, Johann Wolfgang von} (28.\,8.\,1749 Frankfurt am Main – 22.\,3.\,1832 Weimar), \emph{Schriftsteller}!Gespräche@\strich\emph{Gespräche}|pwv}}}\pend
           
\pstart
           \raggedleft{}\textcolor{gray}{\textbf{\so{Goethe}.\pwindex{Goethe, Johann Wolfgang von 28.\,8.\,1749 Frankfurt am Main – 22.\,3.\,1832 Weimar@\textsc{Goethe, Johann Wolfgang von} (28.\,8.\,1749 Frankfurt am Main – 22.\,3.\,1832 Weimar), \emph{Schriftsteller}|pw}}}\pend
           {\vspace{1\baselineskip}}
\pstart
           \centering{}\textcolor{gray}{\textbf{\textbf{Berlin}\oindex{Berlin@\textbf{Berlin}, \emph{Hauptstadt}|pw}{ }1903}}\pend
           
\pstart
           \centering{}\textcolor{gray}{\textbf{\textbf{S. Fiſcher, Verlag}\orgindex{S. Fischer Verlag@S. Fischer Verlag|pw}}}\pend
           \selectlanguage{ngerman}\endnumbering\briefempfaengerindex{Schnitzler, Arthur@\textsc{Schnitzler, Arthur}!zzzBahr, Hermann@\emph{von Hermann Bahr}!1903-06-212@{[21.?] 6. 1903}|)be}\mylabel{L01298h}  \newcommand{\dateiname}{L01298}\newcommand{\titel}{Hermann Bahr: Widmungsexemplar Rezensionen für Arthur Schnitzler, [21.?] 6. 1903}\newcommand{\editorInnen}{Martin Anton Müller und Gerd-Hermann Susen}%% latex-leseansicht-abspann.tex
%% Abspann für die Leseansicht.
%% Der Schalter \ifkorrekturansicht ist bereits durch den Vorspann gesetzt.

%% latex-abspann.tex
%% Gemeinsamer Abspann für Korrekturansicht und Leseansicht.
%% Setzt den Schalter \ifkorrekturansicht voraus (gesetzt in den
%% einbindenden Dateien latex-korrekturansicht-abspann.tex bzw.
%% latex-leseansicht-abspann.tex).
%% ---------------------------------------------------------------

\normalsize

% Das esempio-Environment wird nur in der Leseansicht benötigt
\ifkorrekturansicht\else
\newenvironment{esempio}[3]%
{
    \vspace{1.5ex}
    \rlap{\underline{#1}}
    \par
    \setlength{\parindent}{0cm}
    \nopagebreak
    \leftskip=#2cm
    \rightskip=#3cm
}
{
    \par
}
\fi

\doendnotes{C}
\bigskip
\vfill

\clearpage

\footnotesize

\ifkorrekturansicht
  \lohead{\textsc{register}}
\fi

% theindex-Environment neu definieren ohne reledmac
\makeatletter
\renewenvironment{theindex}{%
  \ifkorrekturansicht
    \section*{\indexname}%
  \else
    \subsubsection*{Index der erwähnten Entitäten}%
  \fi
  \setlength{\parindent}{0pt}%
  \setlength{\parskip}{0pt plus 0.3pt}%
  \let\item\@idxitem
}{%
  \ifkorrekturansicht\clearpage\fi
}
\makeatother

\IfFileExists{\jobname-pw.ind}{\input{\jobname-pw.ind}}{}

% Quellenangabe nur in der Leseansicht
\ifkorrekturansicht\else
% Fallback-Definitionen, falls die .tex-Datei \titel etc. nicht gesetzt hat
\providecommand{\titel}{}
\providecommand{\editorInnen}{}
\providecommand{\dateiname}{\jobname}

\vspace{3cm}

\vfill

\footnotesize
\textsc{Quelle}: \titel. Herausgegeben von {\editorInnen}. In: \emph{Arthur Schnitzler: Briefwechsel mit Autorinnen und Autoren}.
 Digitale Edition, https://schnitzler-briefe.acdh.oeaw.ac.at/{\dateiname}.html (Stand \today)
\fi

\end{document}


