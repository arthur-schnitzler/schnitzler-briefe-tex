%% latex-korrekturansicht-vorspann.tex
%% Vorspann für die Korrekturansicht.
%% Lädt die gemeinsame Datei latex-vorspann.tex mit gesetztem Schalter.

\newif\ifkorrekturansicht
\korrekturansichttrue

\input{../tex-inputs/latex-vorspann}


\section[Hermann Bahr: Widmungsexemplar Rezensionen für Arthur Schnitzler, {[}21.?{]} 6. 1903]{L01298 Hermann Bahr: Widmungsexemplar Rezensionen für Arthur Schnitzler,
               {[}21.?{]} 6. 1903}
\nopagebreak\mylabel{L01298v}
\rehead{ }\normalsize\beginnumbering\briefempfaengerindex{Schnitzler, Arthur@\textsc{Schnitzler, Arthur}!zzzBahr, Hermann@\emph{von Hermann Bahr}!1903-06-212@{{[}21.?{]} 6. 1903}|(be}
\toendnotes[C]{\smallbreak\pagebreak[2]}\Standort{DLA, G:Schnitzler, Arthur (Sammlung Heinrich Schnitzler).}
\physDesc{Widmung am Schmutztitel, 53 Zeichen
\newline{}Handschrift: schwarze Tinte, deutsche Kurrent
\newline{}Ordnung: bei der Enteignung des Exemplars 1938 von
                                 unbekannter Hand mit Bleistift ergänzte Informationen:
                                    »Dubl. zu 426.317-B« }
\buchAbdrucke{\weitereDrucke{Hermann Bahr, Arthur Schnitzler: \emph{Briefwechsel, Aufzeichnungen, Dokumente (1891–1931)}. Göttingen: \emph{Wallstein} 2018, S. 266.} }\toendnotes[C]{\smallbreak}
\pstart
           \noindent{}{\pb}Meinem lieben Arthur\pend
           
\pstart
           herzlichſt{\\[\baselineskip]}\spacefill\mbox{HermannB.}\pend
           \leftskip=0em{}
\pstart
           \noindent{}Juni 1903.\pend
           {\vspace{1\baselineskip}}
\pstart
           \centering{}\textcolor{gray}{\textbf{Rezenſionen\pwindex{Rezensionen. Wiener Theater 1901 bis 1903@\emph{Rezensionen. Wiener Theater 1901 bis 1903}|pw}}}\pend
           \selectlanguage{ngerman}\vspace{1em}{\vspace{1\baselineskip}}
\pstart
           \centering{}{\pb}\textcolor{gray}{\textbf{Hermann Bahr}}\pend
           
\pstart
           \centering{}\textcolor{gray}{\textbf{\textbf{Rezenſionen}\pwindex{Rezensionen. Wiener Theater 1901 bis 1903@\emph{Rezensionen. Wiener Theater 1901 bis 1903}|pw}}}\pend
           
\pstart
           \centering{}\textcolor{gray}{\textbf{Wiener Theater}}\pend
           
\pstart
           \centering{}\textcolor{gray}{\textbf{1901 bis 1903}}\pend
           {\vspace{1\baselineskip}}
\pstart
           \raggedleft{}\textcolor{gray}{\textbf{»Wenn die Leute glauben,{\\}ich
                     wäre noch in Weimar\oindex{Weimar@\textbf{Weimar}, \emph{A.ADM3}|pw},{\\} dann bin ich
                     ſchon in Erfurt\oindex{Erfurt@\textbf{Erfurt}, \emph{P.PPLA}|pw}.«\pwindex{Gespraeche@\emph{Gespräche}|pwv}}}\pend
           
\pstart
           \raggedleft{}\textcolor{gray}{\textbf{\so{Goethe}.\pwindex{Goethe, Johann Wolfgang von 1749-08-28 – 1832-03-22@\textsc{Goethe, Johann Wolfgang von} (1749-08-28 – 1832-03-22), \emph{Schriftsteller/Schriftstellerin}|pw}}}\pend
           {\vspace{1\baselineskip}}
\pstart
           \centering{}\textcolor{gray}{\textbf{\textbf{Berlin}\oindex{Berlin@\textbf{Berlin}, \emph{P.PPLC}|pw}{ }1903}}\pend
           
\pstart
           \centering{}\textcolor{gray}{\textbf{\textbf{S. Fiſcher, Verlag}\orgindex{S. Fischer Verlag@S. Fischer Verlag|pw}}}\pend
           \selectlanguage{ngerman}\endnumbering\briefempfaengerindex{Schnitzler, Arthur@\textsc{Schnitzler, Arthur}!zzzBahr, Hermann@\emph{von Hermann Bahr}!1903-06-212@{{[}21.?{]} 6. 1903}|)be}\mylabel{L01298h}  \normalsize

\doendnotes{C}
\bigskip
\vfill

\clearpage

\footnotesize

\lohead{\textsc{register}}

% Definiere theindex-Environment komplett neu ohne reledmac
\makeatletter
\renewenvironment{theindex}{%
  \section*{\indexname}%
  \setlength{\parindent}{0pt}%
  \setlength{\parskip}{0pt plus 0.3pt}%
  \let\item\@idxitem
}{%
  \clearpage
}
\makeatother

\IfFileExists{\jobname-pw.ind}{\input{\jobname-pw.ind}}{}

\end{document}

      