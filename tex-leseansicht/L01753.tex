%% latex-leseansicht-vorspann.tex
%% Vorspann für die Leseansicht.
%% Lädt die gemeinsame Datei latex-vorspann.tex mit nicht gesetztem Schalter.

\newif\ifkorrekturansicht
\korrekturansichtfalse

\input{../tex-inputs/latex-vorspann}


         
         \renewcommand{\erwaehntePersonen}{Personen: Gerhart Hauptmann, Olga Schnitzler, Karl Werkmann}
         \renewcommand{\erwaehnteInstitutionen}{Institutionen: Franz-Grillparzer-Preis}
         \renewcommand{\erwaehnteOrte}{Orte: Wien}
         \renewcommand{\erwaehnteWerke}{Werke: Die Zeit, Oedipus und die Sphinx. Tragödie in drei Aufzügen, Und Pippa tanzt!, Verleihung des Grillparzer-Preises an Artur Schnitzler}
               \section[Hugo von Hofmannsthal an Arthur Schnitzler, {[}17. 1. 1908{]}]{ Hugo von Hofmannsthal an Arthur Schnitzler, {[}17. 1. 1908{]}}\nopagebreak\mylabel{v}\rehead{ }\begin{ledgroupsized}[t]{13cm}\normalsize\beginnumbering \toendnotes[C]{\smallbreak\pagebreak[2]} \Standort{CUL, Schnitzler, B 43.}
\physDesc{Brief, 1 Blatt, 4 Seiten
\newline{}Handschrift: schwarze Tinte, deutsche Kurrent
\newline{}Schnitzler: mit Bleistift datiert: »17/1 908« und beschriftet: »Hugo« \newline{}Ordnung: 1) mit Bleistift von unbekannter Hand nummeriert: »290«  2) mit Bleistift von unbekannter Hand nummeriert: »292«}\buchAbdrucke{\weitereDrucke{Hugo von Hofmannsthal, Arthur Schnitzler: \emph{Briefwechsel}. Hg. Therese Nickl und Heinrich Schnitzler. Frankfurt am Main: \emph{S. Fischer} 1964, S. 235.} }\toendnotes[C]{\smallbreak}\pstart
           \raggedleft{}{\pb}Freitag.\pend
           \pstart{}mein lieber Arthur\pend\pstart
           ich freue mich \uuline{ſehr}. (Mehr als ich gedacht hätte daſs
               ich mich freuen würde, wenn man mir vorher geſagt hätte: wird es Sie freuen, wenn A{\dots}?)\pend
           \pstart
           Es iſt beſonders lieb, daſs Sie ihn\orgindex{Franz-Grillparzer-Preis@Franz-Grillparzer-Preis|pwv} (durch den Redacteur\pwindex{Werkmann, Karl 14.09.1878 – 24.12.1951@\textsc{Werkmann, Karl} (14.09.1878 – 24.12.1951), \emph{Journalist}|pwv} der Zeit\pwindex{Zeit1902 – 1919@\emph{Die Zeit} {[}1902 – 1919{]}|pwv}) gleich \label{K_L01753_1v}\edtext{mir {\pb}verliehen\pwindex{Verleihung des Grillparzer-Preises an Artur Schnitzler16. 01. 1908@\emph{Verleihung des Grillparzer-Preises an Artur Schnitzler} {[}16. 01. 1908{]}|pwv}}{\lemma{\textnormal{\emph{mir verliehen}}}\Cendnote{\textnormal{Schnitzler\pwindex{Schnitzler, Arthur 15.05.1862 – 21.10.1931@\textsc{Schnitzler, Arthur} (15.05.1862 – 21.10.1931), \emph{Schriftsteller, Mediziner}|pwk}s erste Reaktion auf die Verleihung des \emph{Grillparzer-Preises}\orgindex{Franz-Grillparzer-Preis@Franz-Grillparzer-Preis|pwk}: »Ich hätte nicht geglaubt, daß der Preis\orgindex{Franz-Grillparzer-Preis@Franz-Grillparzer-Preis|pwv} mir
                     verliehen werden würde. Es kamen doch so viele Stücke hierfür in Betracht. Zum
                     Beispiel ›Oedipus und die Sphinx\pwindex{Hofmannsthal, Hugo von 1874-02-01 – 1929-07-15@\textsc{Hofmannsthal, Hugo von} (1874-02-01 – 1929-07-15), \emph{Schriftsteller}!Oedipus und die Sphinx. Tragoedie in drei Aufzuegen1906@\strich\emph{Oedipus und die Sphinx. Tragödie in drei Aufzügen} {[}1906{]}|pw}‹, von Hofmannsthal\pwindex{Hofmannsthal, Hugo von 1874-02-01 – 1929-07-15@\textsc{Hofmannsthal, Hugo von} (1874-02-01 – 1929-07-15), \emph{Schriftsteller}|pw}, dann ›Und
                           Pippa tanzt\pwindex{Hauptmann, Gerhart 15.11.1862 – 06.06.1946@\textsc{Hauptmann, Gerhart} (15.11.1862 – 06.06.1946), \emph{Schriftsteller}!Und Pippa tanzt1906@\strich\emph{Und Pippa tanzt{\rufezeichen}} {[}1906{]}|pw}‹, von Hauptmann\pwindex{Hauptmann, Gerhart 15.11.1862 – 06.06.1946@\textsc{Hauptmann, Gerhart} (15.11.1862 – 06.06.1946), \emph{Schriftsteller}|pw}.«
                           ({[}Karl
                        Werkmann:\pwindex{Werkmann, Karl 14.09.1878 – 24.12.1951@\textsc{Werkmann, Karl} (14.09.1878 – 24.12.1951), \emph{Journalist}|pwk}{]}{ }\emph{Verleihung des Grillparzer-Preises an Artur Schnitzler.}\pwindex{Verleihung des Grillparzer-Preises an Artur Schnitzler16. 01. 1908@\emph{Verleihung des Grillparzer-Preises an Artur Schnitzler} {[}16. 01. 1908{]}|pwk}
                     In: \emph{Die Zeit}\pwindex{Zeit1902 – 1919@\emph{Die Zeit} {[}1902 – 1919{]}|pwk}, Jg. 7, Nr. 1907,
                     Abendblatt, 15. 1. 1908, S. 2).}}}\label{K_L01753_1h}
                  haben.\hspace*{1.5em}Aber, im Ernſt, hätte ich ihn jemals
                  beko{\geminationm}en, bevor Sie ihn hatten ſo hätte ich ihn mit
               einem ſehr groben Brief zurückgeſchickt, ſo leid es mir um das Geld gethan
                  hätte.\hspace*{1.5em}Komiſch übrigens (gewiß hat {\pb}der Interviewer\pwindex{Werkmann, Karl 14.09.1878 – 24.12.1951@\textsc{Werkmann, Karl} (14.09.1878 – 24.12.1951), \emph{Journalist}|pwv}{ }ſich blöd ausgedrückt) daſs Sie ſich ſollten ſo
               quaſi »beſcheiden« ausgedrückt haben ſtatt zu ſagen: Natürlich muſs ich ihn kriegen,
               ſchon längſt hätten mir die Schweine ihn geben müſſen u. ſ. f.\pend
           \pstart
           Ich ſehne mich ſehr {\pb}nach
                  Ihnen.\hspace*{1.5em}Wie wird uns Olga\pwindex{Schnitzler, Olga 17.01.1882 – 13.01.1970@\textsc{Schnitzler, Olga} (17.01.1882 – 13.01.1970), \emph{Schauspielerin, Sängerin}|pw} dafür entſchädigen daſs ſie ſich \uline{wichtig} gemacht hat? Nun übrigens, das arme Ding, ich laſſe ſie ſchön und
               herzlich grüßen.\pend
           \pstart
           Von Herzen Ihr{\\[\baselineskip]}\spacefill\mbox{Hugo.}\pend
           \leftskip=0em{}
         
         \endnumbering\mylabel{h}\end{ledgroupsized}  \newcommand{\dateiname}{L01753}\newcommand{\titel}{Hugo von Hofmannsthal an Arthur Schnitzler, [17. 1. 1908]}\newcommand{\editorInnen}{Martin Anton Müller und Gerd-Hermann Susen}%% latex-leseansicht-abspann.tex
%% Abspann für die Leseansicht.
%% Der Schalter \ifkorrekturansicht ist bereits durch den Vorspann gesetzt.

%% latex-abspann.tex
%% Gemeinsamer Abspann für Korrekturansicht und Leseansicht.
%% Setzt den Schalter \ifkorrekturansicht voraus (gesetzt in den
%% einbindenden Dateien latex-korrekturansicht-abspann.tex bzw.
%% latex-leseansicht-abspann.tex).
%% ---------------------------------------------------------------

\normalsize

% Das esempio-Environment wird nur in der Leseansicht benötigt
\ifkorrekturansicht\else
\newenvironment{esempio}[3]%
{
    \vspace{1.5ex}
    \rlap{\underline{#1}}
    \par
    \setlength{\parindent}{0cm}
    \nopagebreak
    \leftskip=#2cm
    \rightskip=#3cm
}
{
    \par
}
\fi

\doendnotes{C}
\bigskip
\vfill

\clearpage

\footnotesize

\ifkorrekturansicht
  \lohead{\textsc{register}}
\fi

% theindex-Environment neu definieren ohne reledmac
\makeatletter
\renewenvironment{theindex}{%
  \ifkorrekturansicht
    \section*{\indexname}%
  \else
    \subsubsection*{Index der erwähnten Entitäten}%
  \fi
  \setlength{\parindent}{0pt}%
  \setlength{\parskip}{0pt plus 0.3pt}%
  \let\item\@idxitem
}{%
  \ifkorrekturansicht\clearpage\fi
}
\makeatother

\IfFileExists{\jobname-pw.ind}{\input{\jobname-pw.ind}}{}

% Quellenangabe nur in der Leseansicht
\ifkorrekturansicht\else
% Fallback-Definitionen, falls die .tex-Datei \titel etc. nicht gesetzt hat
\providecommand{\titel}{}
\providecommand{\editorInnen}{}
\providecommand{\dateiname}{\jobname}

\vspace{3cm}

\vfill

\footnotesize
\textsc{Quelle}: \titel. Herausgegeben von {\editorInnen}. In: \emph{Arthur Schnitzler: Briefwechsel mit Autorinnen und Autoren}.
 Digitale Edition, https://schnitzler-briefe.acdh.oeaw.ac.at/{\dateiname}.html (Stand \today)
\fi

\end{document}


      