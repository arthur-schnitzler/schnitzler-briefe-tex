%% latex-korrekturansicht-vorspann.tex
%% Vorspann für die Korrekturansicht.
%% Lädt die gemeinsame Datei latex-vorspann.tex mit gesetztem Schalter.

\newif\ifkorrekturansicht
\korrekturansichttrue

\input{../tex-inputs/latex-vorspann}


\section[Hugo von Hofmannsthal an Arthur Schnitzler, {[}17. 1. 1908{]}]{L01753 Hugo von Hofmannsthal an Arthur Schnitzler, {[}17. 1. 1908{]}}
\nopagebreak\mylabel{L01753v}
\rehead{ }\normalsize\beginnumbering\briefempfaengerindex{Schnitzler, Arthur@\textsc{Schnitzler, Arthur}!zzzHofmannsthal, Hugo von@\emph{von Hugo von Hofmannsthal}!1908-01-172@{17. 1. 1908}|(be}
\toendnotes[C]{\smallbreak\pagebreak[2]}\Standort{CUL, Schnitzler, B 43.}
\physDesc{Brief, 1 Blatt, 4 Seiten, 848 Zeichen
\newline{}Handschrift: schwarze Tinte, deutsche Kurrent
\newline{}Schnitzler: mit Bleistift datiert: »17/1 908« und beschriftet: »Hugo« 
\newline{}Ordnung: 1) mit Bleistift von unbekannter Hand nummeriert:
                                    »290«  2) mit Bleistift von unbekannter Hand nummeriert:
                                    »292«}
\buchAbdrucke{\weitereDrucke{Hugo von Hofmannsthal, Arthur Schnitzler: \emph{Briefwechsel}. Frankfurt am Main: \emph{S. Fischer} 1964, S. 235.} }\toendnotes[C]{\smallbreak}
\pstart
           \raggedleft{}{\pb}Freitag.\pend
           
\pstart{}mein lieber Arthur\pend\vspace{0.5em}
\pstart
           ich freue mich \uuline{ſehr}. (Mehr als ich gedacht hätte daſs
               ich mich freuen würde, wenn man mir vorher geſagt hätte: wird es Sie freuen, wenn A{\dots}?)\pend
           
\pstart
           Es iſt beſonders lieb, daſs Sie ihn\orgindex{Franz-Grillparzer-Preis@Franz-Grillparzer-Preis|pwv} (durch den Redacteur\pwindex{Werkmann, Karl 14.09.1878 – 24.12.1951@\textsc{Werkmann, Karl} (14.09.1878 – 24.12.1951), \emph{Journalist/Journalistin}|pwv} der Zeit\pwindex{Zeit@\emph{Die Zeit}|pwv})
               gleich \label{K_L01753-1v}\edtext{mir {\pb}verliehen\pwindex{Verleihung des Grillparzer-Preises an Artur Schnitzler@\emph{Verleihung des Grillparzer-Preises an Artur Schnitzler}|pwv}}{\lemma{\textnormal{\emph{mir verliehen}}}\Cendnote{\textnormal{Schnitzlers erste Reaktion auf die
                  Verleihung des \emph{Grillparzer-Preises}\orgindex{Franz-Grillparzer-Preis@Franz-Grillparzer-Preis|pwk}:
                     »Ich hätte nicht geglaubt, daß der Preis\orgindex{Franz-Grillparzer-Preis@Franz-Grillparzer-Preis|pwv} mir verliehen werden würde. Es kamen doch so
                     viele Stücke hierfür in Betracht. Zum Beispiel ›Oedipus und die Sphinx\pwindex{Oedipus und die Sphinx. Tragoedie in drei Aufzuegen@\emph{Oedipus und die Sphinx. Tragödie in drei Aufzügen}|pw}‹, von Hofmannsthal\pwindex{Hofmannsthal, Hugo von 1874-02-01 – 1929-07-15@\textsc{Hofmannsthal, Hugo von} (1874-02-01 – 1929-07-15), \emph{Schriftsteller/Schriftstellerin}|pw}, dann ›Und Pippa
                        tanzt\pwindex{Und Pippa tanzt@\emph{Und Pippa tanzt{\rufezeichen}}|pw}‹, von Hauptmann\pwindex{Hauptmann, Gerhart 15.11.1862 – 06.06.1946@\textsc{Hauptmann, Gerhart} (15.11.1862 – 06.06.1946), \emph{Schriftsteller/Schriftstellerin}|pw}.« A. S.: \emph{»Das Zeitlose ist von kürzester Dauer«}, [Karl Werkmann]: Verleihung des Grillparzer-Preises an Artur Schnitzler, 15. 1. 1908.}}}\label{K_L01753-1}
                  haben.\hspace*{1.5em}Aber, im Ernſt, hätte ich ihn jemals
                  beko{\geminationm}en, bevor Sie ihn hatten ſo hätte ich ihn mit
               einem ſehr groben Brief zurückgeſchickt, ſo leid es mir um das Geld gethan
                  hätte.\hspace*{1.5em}Komiſch übrigens (gewiß hat {\pb}der Interviewer\pwindex{Werkmann, Karl 14.09.1878 – 24.12.1951@\textsc{Werkmann, Karl} (14.09.1878 – 24.12.1951), \emph{Journalist/Journalistin}|pwv}{ }ſich blöd ausgedrückt) daſs Sie ſich ſollten ſo
               quaſi »beſcheiden« ausgedrückt haben ſtatt zu ſagen: Natürlich muſs ich ihn kriegen,
               ſchon längſt hätten mir die Schweine ihn geben müſſen u. ſ. f.\pend
           
\pstart
           Ich ſehne mich ſehr {\pb}nach
                  Ihnen.\hspace*{1.5em}Wie wird uns Olga\pwindex{Schnitzler, Olga 17.01.1882 – 13.01.1970@\textsc{Schnitzler, Olga} (17.01.1882 – 13.01.1970), \emph{Schauspieler/Schauspielerin, Sänger/Sängerin}|pw} dafür entſchädigen daſs ſie ſich \uline{wichtig} gemacht hat? Nun übrigens, das arme Ding, ich laſſe ſie ſchön
               und herzlich grüßen.\pend
           
\pstart
           Von Herzen Ihr{\\[\baselineskip]}\spacefill\mbox{Hugo.}\pend
           \leftskip=0em{}\selectlanguage{ngerman}\endnumbering\briefempfaengerindex{Schnitzler, Arthur@\textsc{Schnitzler, Arthur}!zzzHofmannsthal, Hugo von@\emph{von Hugo von Hofmannsthal}!1908-01-172@{17. 1. 1908}|)be}\mylabel{L01753h}  \normalsize

\doendnotes{C}
\bigskip
\vfill

\clearpage

\footnotesize

\lohead{\textsc{register}}

% Definiere theindex-Environment komplett neu ohne reledmac
\makeatletter
\renewenvironment{theindex}{%
  \section*{\indexname}%
  \setlength{\parindent}{0pt}%
  \setlength{\parskip}{0pt plus 0.3pt}%
  \let\item\@idxitem
}{%
  \clearpage
}
\makeatother

\IfFileExists{\jobname-pw.ind}{\input{\jobname-pw.ind}}{}

\end{document}

      