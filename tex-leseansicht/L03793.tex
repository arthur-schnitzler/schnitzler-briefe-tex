%% latex-korrekturansicht-vorspann.tex
%% Vorspann für die Korrekturansicht.
%% Lädt die gemeinsame Datei latex-vorspann.tex mit gesetztem Schalter.

\newif\ifkorrekturansicht
\korrekturansichttrue

\input{../tex-inputs/latex-vorspann}


\section[Arthur Schnitzler an Stefan Zweig, 28. 4. 1930]{L03793 Arthur Schnitzler an Stefan Zweig, 28. 4. 1930}
\nopagebreak\mylabel{L03793v}
\rehead{ }\normalsize\beginnumbering\briefempfaengerindex{Zweig, Stefan@\textsc{Zweig, Stefan}!zzzSchnitzler, Arthur@\emph{von Arthur Schnitzler}!1930-04-281@{28. 4. 1930}|(be}
\toendnotes[C]{\smallbreak\pagebreak[2]}\Standort{Jerusalem, National Library of Israel, ARC. Ms. Var. 305 1 58 Stefan Zweig Collection.}
\physDesc{Brief, 1 Blatt, 2 Seiten, 699 Zeichen
\newline{}Handschrift: schwarze Tinte, lateinische Kurrent}\toendnotes[C]{\smallbreak}
\pstart
           \raggedleft{}{\pb}Wien\oindex{Wien@\textbf{Wien}, \emph{A.ADM2}|pw}{ }28. 4. 30.\pend
           
\pstart{}lieber Doctor Stefan Zweig,\pend\vspace{0.5em}
\pstart
           erst vor \label{K_L03793-1v}\edtext{etlichen Tagen\eventindex{Burgtheater@\textbf{Burgtheater}!Auffuehrung von Das Lamm des Armen, 23.4.1930@Aufführung von Das Lamm des Armen, 23.4.1930|pwv}}{\lemma{\textnormal{\emph{etlichen Tagen}}}\Cendnote{\textnormal{Siehe A. S.: \emph{Kulturveranstaltungen}, 23. 4. 1930.}}}\label{K_L03793-1} hab ich das »Lamm\pwindex{Lamm des Armen. Tragikomoedie in drei Akten@\emph{Das Lamm des Armen. Tragikomödie in drei Akten}|pw}« auf der Bühne
               gesehen und Dank Ihnen erst heute für das schöne Werk; an dem ich eine rechte
               Publikumsfreude gehabt habe – ganz abgesehen vom ethischen, aesthetischen,
               handwerklichen und theatralischen Vergnügen, die aus dem Stücke quellen. (Ebenso
               pedantisch möchte ich bemerken, daſs mir die Eintheilung in 2 Theile, erster in
               5, zweiter in 4 Bildern lieber gewesen wäre als in »3 Akten«)\pend
           
\pstart
           Im ganzen wars ein darstellerisch wohl gelungner Abend\eventindex{Burgtheater@\textbf{Burgtheater}!Auffuehrung von Das Lamm des Armen, 23.4.1930@Aufführung von Das Lamm des Armen, 23.4.1930|pwv}, sehr reinliches Burgtheater\orgindex{Burgtheater@Burgtheater|pw} und in manchen Momenten mehr oder viel
               mehr. \pend
           
\pstart
           Hoffentlich sprech ich Sie bald wieder – ich finde {\pb}man
               thut das zu selten\pend
           
\pstart
           Sehr herzlich{\\[\baselineskip]}Ihr{\\[\baselineskip]}\spacefill\mbox{ArthSchnitzler}\pend
           \leftskip=0em{}\selectlanguage{ngerman}\endnumbering\briefempfaengerindex{Zweig, Stefan@\textsc{Zweig, Stefan}!zzzSchnitzler, Arthur@\emph{von Arthur Schnitzler}!1930-04-281@{28. 4. 1930}|)be}\mylabel{L03793h}
\begin{anhang}
\end{anhang}\normalsize

\doendnotes{C}
\bigskip
\vfill

\clearpage

\footnotesize

\lohead{\textsc{register}}

% Definiere theindex-Environment komplett neu ohne reledmac
\makeatletter
\renewenvironment{theindex}{%
  \section*{\indexname}%
  \setlength{\parindent}{0pt}%
  \setlength{\parskip}{0pt plus 0.3pt}%
  \let\item\@idxitem
}{%
  \clearpage
}
\makeatother

\IfFileExists{\jobname-pw.ind}{\input{\jobname-pw.ind}}{}

\end{document}

      