%% latex-leseansicht-vorspann.tex
%% Vorspann für die Leseansicht.
%% Lädt die gemeinsame Datei latex-vorspann.tex mit nicht gesetztem Schalter.

\newif\ifkorrekturansicht
\korrekturansichtfalse

\input{../tex-inputs/latex-vorspann}


\section[Arthur Schnitzler an Stefan Zweig, 28. 4. 1930]{L03793 Arthur Schnitzler an Stefan Zweig, 28. 4. 1930}
\nopagebreak\mylabel{L03793v}
\rehead{ }\normalsize\beginnumbering\briefempfaengerindex{Zweig, Stefan@\textsc{Zweig, Stefan}!zzzSchnitzler, Arthur@\emph{von Arthur Schnitzler}!1930-04-281@{28. 4. 1930}|(be}
\toendnotes[C]{\smallbreak\pagebreak[2]}
\correspDesc{Versand  durch Arthur Schnitzler am 28. 4. 1930 in Wien
\newline{}Erhalt  durch Stefan Zweig im Zeitraum [29. 4. 1930 – 3. 5. 1930?] in Salzburg}\toendnotes[C]{\smallbreak}
\Standort{Jerusalem, National Library of Israel, ARC. Ms. Var. 305 1 58 Stefan Zweig Collection.}
\physDesc{Brief, 1 Blatt, 2 Seiten, 704 Zeichen
\newline{}Handschrift: schwarze Tinte, lateinische Kurrent}\toendnotes[C]{\smallbreak}
\pstart
           \raggedleft{}{\pb}Wien\oindex{Wien@\textbf{Wien}, \emph{Verwaltungsgebiet}|pw}{ }28. 4. 30.\pend
           
\pstart{}lieber Doctor Stefan Zweig,\pend\vspace{0.5em}
\pstart
           erst vor \label{K_L03793-1v}\edtext{etlichen Tagen\eventindex{Burgtheater@\textbf{Burgtheater}!Aufführung von Das Lamm des Armen, 23.4.1930@Aufführung von Das Lamm des Armen, 23.4.1930|pwv}}{\lemma{\textnormal{\emph{etlichen Tagen}}}\Cendnote{\textnormal{Siehe A. S.: \emph{Kulturveranstaltungen}, 23. 4. 1930.}}}\label{K_L03793-1} hab ich das »Lamm\pwindex{Zweig, Stefan 28.\,11.\,1881 Wien – 23.\,2.\,1942 Petrópolis@\textsc{Zweig, Stefan} (28.\,11.\,1881 Wien – 23.\,2.\,1942 Petrópolis), \emph{Schriftsteller}!Lamm des Armen. Tragikomödie in drei Akten@\strich\emph{Das Lamm des Armen. Tragikomödie in drei Akten}|pw}« auf der Bühne
               gesehen und Dank Ihnen erst heute für das schöne Werk, an dem ich eine rechte
               Publikumsfreude gehabt habe – ganz abgesehen von ethischen, aesthetischen,
               handwerklichen und theatralischen Befriedigungen, die aus dem Stücke quellen. (Ebenso
               pedantisch möchte ich bemerken, daſs mir die Eintheilung in 2 Theile, erster in
               5, zweiter in 4 Bildern lieber gewesen wäre als in »3 Akten«)\pend
           
\pstart
           Im ganzen wars ein darstellerisch wohl gelungner Abend\eventindex{Burgtheater@\textbf{Burgtheater}!Aufführung von Das Lamm des Armen, 23.4.1930@Aufführung von Das Lamm des Armen, 23.4.1930|pwv}, sehr reinliches Burgtheater\orgindex{Burgtheater@Burgtheater|pw} und in manchen Momenten mehr oder viel
               mehr.\pend
           
\pstart
           Hoffentlich sprech ich Sie bald wieder – ich finde {\pb}man
               thut das zu selten.\pend
           
\pstart
           Sehr herzlich{\\[\baselineskip]}Ihr{\\[\baselineskip]}\spacefill\mbox{ArthSchnitzler}\pend
           \leftskip=0em{}\selectlanguage{ngerman}\endnumbering\briefempfaengerindex{Zweig, Stefan@\textsc{Zweig, Stefan}!zzzSchnitzler, Arthur@\emph{von Arthur Schnitzler}!1930-04-281@{28. 4. 1930}|)be}\mylabel{L03793h}  \newcommand{\dateiname}{L03793}\newcommand{\titel}{Arthur Schnitzler an Stefan Zweig, 28. 4. 1930}\newcommand{\editorInnen}{Selma Jahnke und Martin Anton Müller}%% latex-leseansicht-abspann.tex
%% Abspann für die Leseansicht.
%% Der Schalter \ifkorrekturansicht ist bereits durch den Vorspann gesetzt.

%% latex-abspann.tex
%% Gemeinsamer Abspann für Korrekturansicht und Leseansicht.
%% Setzt den Schalter \ifkorrekturansicht voraus (gesetzt in den
%% einbindenden Dateien latex-korrekturansicht-abspann.tex bzw.
%% latex-leseansicht-abspann.tex).
%% ---------------------------------------------------------------

\normalsize

% Das esempio-Environment wird nur in der Leseansicht benötigt
\ifkorrekturansicht\else
\newenvironment{esempio}[3]%
{
    \vspace{1.5ex}
    \rlap{\underline{#1}}
    \par
    \setlength{\parindent}{0cm}
    \nopagebreak
    \leftskip=#2cm
    \rightskip=#3cm
}
{
    \par
}
\fi

\doendnotes{C}
\bigskip
\vfill

\clearpage

\footnotesize

\ifkorrekturansicht
  \lohead{\textsc{register}}
\fi

% theindex-Environment neu definieren ohne reledmac
\makeatletter
\renewenvironment{theindex}{%
  \ifkorrekturansicht
    \section*{\indexname}%
  \else
    \subsubsection*{Index der erwähnten Entitäten}%
  \fi
  \setlength{\parindent}{0pt}%
  \setlength{\parskip}{0pt plus 0.3pt}%
  \let\item\@idxitem
}{%
  \ifkorrekturansicht\clearpage\fi
}
\makeatother

\IfFileExists{\jobname-pw.ind}{\input{\jobname-pw.ind}}{}

% Quellenangabe nur in der Leseansicht
\ifkorrekturansicht\else
% Fallback-Definitionen, falls die .tex-Datei \titel etc. nicht gesetzt hat
\providecommand{\titel}{}
\providecommand{\editorInnen}{}
\providecommand{\dateiname}{\jobname}

\vspace{3cm}

\vfill

\footnotesize
\textsc{Quelle}: \titel. Herausgegeben von {\editorInnen}. In: \emph{Arthur Schnitzler: Briefwechsel mit Autorinnen und Autoren}.
 Digitale Edition, https://schnitzler-briefe.acdh.oeaw.ac.at/{\dateiname}.html (Stand \today)
\fi

\end{document}


