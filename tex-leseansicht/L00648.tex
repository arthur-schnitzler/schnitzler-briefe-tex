%% latex-leseansicht-vorspann.tex
%% Vorspann für die Leseansicht.
%% Lädt die gemeinsame Datei latex-vorspann.tex mit nicht gesetztem Schalter.

\newif\ifkorrekturansicht
\korrekturansichtfalse

\input{../tex-inputs/latex-vorspann}


         
         \renewcommand{\erwaehntePersonen}{Personen: Hermann Bahr, Josef Svatopluk Machar}
         \renewcommand{\erwaehnteOrte}{Orte: Gersthofer Straße, Wien}
         \renewcommand{\erwaehnteWerke}{Werke: Das Tschaperl. Ein Wiener Stück in vier Aufzügen, Renaissance. Neue Studien zur Kritik der Moderne}
               \section[Arthur Schnitzler an Hermann Bahr, 28. 2. 1897]{ Arthur Schnitzler an Hermann Bahr, 28. 2. 1897}\nopagebreak\mylabel{v}\rehead{ }\begin{ledgroupsized}[t]{13cm}\normalsize\beginnumbering \toendnotes[C]{\smallbreak\pagebreak[2]} \Standort{TMW, HS AM 67336 Ba.}
\physDesc{Visitenkarte
\newline{}Handschrift: Bleistift, deutsche Kurrent
\newline{}Bahr: mit blauem Buntstift umseitiger Vermerk: »Machar\pwindex{Machar, Josef Svatopluk 29.02.1864 – 17.03.1942@\textsc{Machar, Josef Svatopluk} (29.02.1864 – 17.03.1942), \emph{Schriftsteller}|pw} + Renaissance\pwindex{Bahr, Hermann 19.07.1863 – 15.01.1934@\textsc{Bahr, Hermann} (19.07.1863 – 15.01.1934), \emph{Schriftsteller, Kritiker}!Renaissance. Neue Studien zur Kritik der Moderne1897@\strich\emph{Renaissance. Neue Studien zur Kritik der Moderne} {[}1897{]}|pw} + XVIII 2 Gerst. Hoferstr. 144\oindex{Gersthofer Strasse@\textbf{Gersthofer Straße}|pw}« }\buchAbdrucke{\weitereDrucke{Hermann Bahr, Arthur Schnitzler: \emph{Briefwechsel, Aufzeichnungen, Dokumente (1891–1931)}. Hg. Kurt Ifkovits und Martin Anton Müller. Göttingen: \emph{Wallstein} 2018, S. 136.} }\toendnotes[C]{\smallbreak}\pstart
           \noindent{}\centering{}{\pb}Gratulire\pwindex{Bahr, Hermann 19.07.1863 – 15.01.1934@\textsc{Bahr, Hermann} (19.07.1863 – 15.01.1934), \emph{Schriftsteller, Kritiker}!Tschaperl. Ein Wiener Stueck in vier Aufzuegen1896@\strich\emph{Das Tschaperl. Ein Wiener Stück in vier Aufzügen} {[}1896{]}|pwv} herzlich!\pend
           \pstart
           \noindent{}\centering{}\textcolor{gray}{\textbf{D\textsuperscript{r} Arthur
                  Schnitzler}}\pend
           \pstart
           28. 2. 1897\pend
           
         
         \endnumbering\mylabel{h}\end{ledgroupsized}  \newcommand{\dateiname}{L00648}\newcommand{\titel}{Arthur Schnitzler an Hermann Bahr, 28. 2. 1897}\newcommand{\editorInnen}{ Kurt Ifkovits,  Martin Anton Müller}%% latex-leseansicht-abspann.tex
%% Abspann für die Leseansicht.
%% Der Schalter \ifkorrekturansicht ist bereits durch den Vorspann gesetzt.

%% latex-abspann.tex
%% Gemeinsamer Abspann für Korrekturansicht und Leseansicht.
%% Setzt den Schalter \ifkorrekturansicht voraus (gesetzt in den
%% einbindenden Dateien latex-korrekturansicht-abspann.tex bzw.
%% latex-leseansicht-abspann.tex).
%% ---------------------------------------------------------------

\normalsize

% Das esempio-Environment wird nur in der Leseansicht benötigt
\ifkorrekturansicht\else
\newenvironment{esempio}[3]%
{
    \vspace{1.5ex}
    \rlap{\underline{#1}}
    \par
    \setlength{\parindent}{0cm}
    \nopagebreak
    \leftskip=#2cm
    \rightskip=#3cm
}
{
    \par
}
\fi

\doendnotes{C}
\bigskip
\vfill

\clearpage

\footnotesize

\ifkorrekturansicht
  \lohead{\textsc{register}}
\fi

% theindex-Environment neu definieren ohne reledmac
\makeatletter
\renewenvironment{theindex}{%
  \ifkorrekturansicht
    \section*{\indexname}%
  \else
    \subsubsection*{Index der erwähnten Entitäten}%
  \fi
  \setlength{\parindent}{0pt}%
  \setlength{\parskip}{0pt plus 0.3pt}%
  \let\item\@idxitem
}{%
  \ifkorrekturansicht\clearpage\fi
}
\makeatother

\IfFileExists{\jobname-pw.ind}{\input{\jobname-pw.ind}}{}

% Quellenangabe nur in der Leseansicht
\ifkorrekturansicht\else
% Fallback-Definitionen, falls die .tex-Datei \titel etc. nicht gesetzt hat
\providecommand{\titel}{}
\providecommand{\editorInnen}{}
\providecommand{\dateiname}{\jobname}

\vspace{3cm}

\vfill

\footnotesize
\textsc{Quelle}: \titel. Herausgegeben von {\editorInnen}. In: \emph{Arthur Schnitzler: Briefwechsel mit Autorinnen und Autoren}.
 Digitale Edition, https://schnitzler-briefe.acdh.oeaw.ac.at/{\dateiname}.html (Stand \today)
\fi

\end{document}


      