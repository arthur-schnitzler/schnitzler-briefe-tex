%% latex-korrekturansicht-vorspann.tex
%% Vorspann für die Korrekturansicht.
%% Lädt die gemeinsame Datei latex-vorspann.tex mit gesetztem Schalter.

\newif\ifkorrekturansicht
\korrekturansichttrue

\input{../tex-inputs/latex-vorspann}


\section[Arthur Schnitzler an Hermann Bahr, 28. 2. 1897]{L00648 Arthur Schnitzler an Hermann Bahr, 28. 2. 1897}
\nopagebreak\mylabel{L00648v}
\rehead{ }\normalsize\beginnumbering\briefempfaengerindex{Bahr, Hermann@\textsc{Bahr, Hermann}!zzzSchnitzler, Arthur@\emph{von Arthur Schnitzler}!1897-02-281@{28. 2. 1897}|(be}
\toendnotes[C]{\smallbreak\pagebreak[2]}\Standort{TMW, HS AM 67336 Ba.}
\physDesc{Visitenkarte, 29 Zeichen
\newline{}Handschrift: Bleistift, deutsche Kurrent
\newline{}Bahr: mit blauem Buntstift umseitiger Vermerk: »Machar\pwindex{Machar, Josef Svatopluk 29.02.1864 – 17.03.1942@\textsc{Machar, Josef Svatopluk} (29.02.1864 – 17.03.1942), \emph{Schriftsteller/Schriftstellerin}|pw} + Renaissance\pwindex{Renaissance. Neue Studien zur Kritik der Moderne@\emph{Renaissance. Neue Studien zur Kritik der Moderne}|pw} + XVIII 2 Gerst. Hoferstr. 144\oindex{Gersthofer Strasse@\textbf{Gersthofer Straße}, \emph{Straße (K.STR)}|pw}« }
\buchAbdrucke{\weitereDrucke{Hermann Bahr, Arthur Schnitzler: \emph{Briefwechsel, Aufzeichnungen, Dokumente (1891–1931)}. Göttingen: \emph{Wallstein} 2018, S. 136.} }\toendnotes[C]{\smallbreak}
\pstart
           \noindent{}\centering{}{\pb}Gratulire\pwindex{Tschaperl. Ein Wiener Stueck in vier Aufzuegen@\emph{Das Tschaperl. Ein Wiener Stück in vier Aufzügen}|pwv} herzlich!\pend
           
\pstart
           \centering{}\textcolor{gray}{\textbf{D\textsuperscript{r} Arthur Schnitzler}}\pend
           
\pstart
           28. 2. 1897\pend
           \selectlanguage{ngerman}\endnumbering\briefempfaengerindex{Bahr, Hermann@\textsc{Bahr, Hermann}!zzzSchnitzler, Arthur@\emph{von Arthur Schnitzler}!1897-02-281@{28. 2. 1897}|)be}\mylabel{L00648h}  \normalsize

\doendnotes{C}
\bigskip
\vfill

\clearpage

\footnotesize

\lohead{\textsc{register}}

% Definiere theindex-Environment komplett neu ohne reledmac
\makeatletter
\renewenvironment{theindex}{%
  \section*{\indexname}%
  \setlength{\parindent}{0pt}%
  \setlength{\parskip}{0pt plus 0.3pt}%
  \let\item\@idxitem
}{%
  \clearpage
}
\makeatother

\IfFileExists{\jobname-pw.ind}{\input{\jobname-pw.ind}}{}

\end{document}

      