%% latex-korrekturansicht-vorspann.tex
%% Vorspann für die Korrekturansicht.
%% Lädt die gemeinsame Datei latex-vorspann.tex mit gesetztem Schalter.

\newif\ifkorrekturansicht
\korrekturansichttrue

\input{../tex-inputs/latex-vorspann}


\section[ Felix Salten an Arthur Schnitzler, {[}20. 6. 1900{]}]{L03305 Felix Salten an Arthur Schnitzler, {[}20. 6. 1900{]}}
\nopagebreak\mylabel{L03305v}
\rehead{ }\normalsize\beginnumbering\briefempfaengerindex{Schnitzler, Arthur@\textsc{Schnitzler, Arthur}!zzzSalten, Felix@\emph{von Felix Salten}!1900-06-202@{{[}20. 6. 1900{]}}|(be}
\toendnotes[C]{\smallbreak\pagebreak[2]}\Standort{CUL, Schnitzler, B 89, A 2.}
\physDesc{Brief, 1 Blatt, 1 Seite, 747 Zeichen ({\pb}die Rückseite
                                 weist das Blatt als Abriss eines mit schwarzer Tinte beschriebenen
                                 Blattes aus)
\newline{}Handschrift: Bleistift, lateinische Kurrent
\newline{}Schnitzler: mit Bleistift datiert: »20/6 900.« 
\newline{}Ordnung: mit Bleistift von unbekannter Hand nummeriert: »129« }
\buchAbdrucke{\weitereDrucke{Hermann Bahr, Arthur Schnitzler: \emph{Briefwechsel, Aufzeichnungen, Dokumente (1891–1931)}. Göttingen: \emph{Wallstein} 2018, S. 176.} }\toendnotes[C]{\smallbreak}
\pstart
           \noindent{}{\pb}Lieber, ich war eben bei Ihnen, um Ihnen folgendes zu sagen:
               Überlegen Sie, ob Sie nicht lieber gleich zum \label{K_L03305-1v}\edtext{Volksth.\orgindex{Volkstheater@Volkstheater|pw}}{\lemma{\textnormal{\emph{Volksth.}}}\Cendnote{\textnormal{Seit Februar des Jahres glaubte Schnitzler,
                  \emph{Der Schleier der Beatrice}\pwindex{Schleier der Beatrice. Schauspiel in fuenf Akten@\emph{Der Schleier der Beatrice. Schauspiel in fünf Akten}|pwk} wäre vom \emph{Burgtheater}\orgindex{Burgtheater@Burgtheater|pwk} zur Uraufführung angenommen worden. Direktor Schlenther\pwindex{Schlenther, Paul 20.08.1854 – 30.04.1916@\textsc{Schlenther, Paul} (20.08.1854 – 30.04.1916), \emph{Schriftsteller/Schriftstellerin, Kritiker/Kritikerin, Theaterleiter/Theaterleiterin}|pwk}
                  teilte Schnitzler aber am 18. 6. 1900
                  mit, dass er die Annahme noch überlege. Schnitzler besprach bereits
                  am Folgetag die Sachlage mit Salten\pwindex{Salten, Felix 06.09.1869 – 08.10.1945@\textsc{Salten, Felix} (06.09.1869 – 08.10.1945), \emph{Schriftsteller/Schriftstellerin, Journalist/Journalistin, Chefredakteur/Chefredakteurin}|pwk}, vgl. A. S.: \emph{Tagebuch}, 18. 6. 1900.
                   Zu einer
                  Aufführung durch das \emph{Volkstheater}\orgindex{Volkstheater@Volkstheater|pwk} kam es
                  nicht.}}}\label{K_L03305-1} gehen wollen. In diesem Fall wäre die Nachricht von der Annahme
               Ihres Stück\pwindex{Schleier der Beatrice. Schauspiel in fuenf Akten@\emph{Der Schleier der Beatrice. Schauspiel in fünf Akten}|pwv}es am Volksth.\orgindex{Volkstheater@Volkstheater|pw} die vorläufig beste Antwort für Schlenther\pwindex{Schlenther, Paul 20.08.1854 – 30.04.1916@\textsc{Schlenther, Paul} (20.08.1854 – 30.04.1916), \emph{Schriftsteller/Schriftstellerin, Kritiker/Kritikerin, Theaterleiter/Theaterleiterin}|pw}. Und dem Volksth.\orgindex{Volkstheater@Volkstheater|pw} gegenüber wären Sie jetzt in der Lage zu sagen, dass
               Ihnen \uline{der Termin}{ }\uline{des Burgtheaters\orgindex{Burgtheater@Burgtheater|pw}}
               nicht passt, während Sie, falls Sie ein Refus von Schlenth.\pwindex{Schlenther, Paul 20.08.1854 – 30.04.1916@\textsc{Schlenther, Paul} (20.08.1854 – 30.04.1916), \emph{Schriftsteller/Schriftstellerin, Kritiker/Kritikerin, Theaterleiter/Theaterleiterin}|pw} provoziren, mit einem abgelehnten Stück\pwindex{Schleier der Beatrice. Schauspiel in fuenf Akten@\emph{Der Schleier der Beatrice. Schauspiel in fünf Akten}|pwv} zu Bukovics\pwindex{Bukovics, Emerich von 28.02.1844 – 04.07.1905@\textsc{Bukovics, Emerich von} (28.02.1844 – 04.07.1905), \emph{Journalist/Journalistin, Theaterleiter/Theaterleiterin}|pw}
               kommen, der vielleicht daraus wieder Capital schlägt, und Ihnen sagt, (von Bahr\pwindex{Bahr, Hermann 19.07.1863 – 15.01.1934@\textsc{Bahr, Hermann} (19.07.1863 – 15.01.1934), \emph{Schriftsteller/Schriftstellerin, Kritiker/Kritikerin}|pw} gehetzt) dass Sie nur das für ihn haben,
               was Schlenther\pwindex{Schlenther, Paul 20.08.1854 – 30.04.1916@\textsc{Schlenther, Paul} (20.08.1854 – 30.04.1916), \emph{Schriftsteller/Schriftstellerin, Kritiker/Kritikerin, Theaterleiter/Theaterleiterin}|pw} übrig läßt. Ganz abgesehen
               davon, dass Sch.\pwindex{Schlenther, Paul 20.08.1854 – 30.04.1916@\textsc{Schlenther, Paul} (20.08.1854 – 30.04.1916), \emph{Schriftsteller/Schriftstellerin, Kritiker/Kritikerin, Theaterleiter/Theaterleiterin}|pw} – wenn er von Ihnen keine
               Antwort kriegt, und nur hört, Ihr Stück\pwindex{Schleier der Beatrice. Schauspiel in fuenf Akten@\emph{Der Schleier der Beatrice. Schauspiel in fünf Akten}|pwv} sei am Volksth.\orgindex{Volkstheater@Volkstheater|pw} – gewiß gelaufen
               kommt. ec. ec. ec.\pend
           \pstart Herzl. \spacefill\mbox{Salten}\pend{}\selectlanguage{ngerman}\endnumbering\briefempfaengerindex{Schnitzler, Arthur@\textsc{Schnitzler, Arthur}!zzzSalten, Felix@\emph{von Felix Salten}!1900-06-202@{{[}20. 6. 1900{]}}|)be}\mylabel{L03305h}  \normalsize

\doendnotes{C}
\bigskip
\vfill

\clearpage

\footnotesize

\lohead{\textsc{register}}

% Definiere theindex-Environment komplett neu ohne reledmac
\makeatletter
\renewenvironment{theindex}{%
  \section*{\indexname}%
  \setlength{\parindent}{0pt}%
  \setlength{\parskip}{0pt plus 0.3pt}%
  \let\item\@idxitem
}{%
  \clearpage
}
\makeatother

\IfFileExists{\jobname-pw.ind}{\input{\jobname-pw.ind}}{}

\end{document}

      