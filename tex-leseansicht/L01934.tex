%% latex-korrekturansicht-vorspann.tex
%% Vorspann für die Korrekturansicht.
%% Lädt die gemeinsame Datei latex-vorspann.tex mit gesetztem Schalter.

\newif\ifkorrekturansicht
\korrekturansichttrue

\input{../tex-inputs/latex-vorspann}


\section[Arthur und Olga Schnitzler an Richard Beer-Hofmann, 25. 5. 1910]{L01934 Arthur und Olga Schnitzler an Richard Beer-Hofmann, 25. 5. 1910}
\nopagebreak\mylabel{L01934v}
\rehead{ }\normalsize\beginnumbering\briefempfaengerindex{Beer-Hofmann, Richard@\textsc{Beer-Hofmann, Richard}!zzzSchnitzler, Olga@\emph{von Olga Schnitzler}!1910-05-251@{25. 5. 1910}|(be}\briefempfaengerindex{Beer-Hofmann, Richard@\textsc{Beer-Hofmann, Richard}!zzzSchnitzler, Arthur@\emph{von Arthur Schnitzler}!1910-05-251@{25. 5. 1910}|(be}
\toendnotes[C]{\smallbreak\pagebreak[2]}\Standort{YCGL, MSS 31.}
\physDesc{Bildpostkarte, 330 Zeichen
\newline{}Handschrift Arthur Schnitzler: Bleistift, deutsche Kurrent
\newline{}Handschrift Olga Schnitzler: Bleistift, lateinische Kurrent
\newline{}Versand: Stempel: »\nobreak{}\oindex{Interlaken@\textbf{Interlaken}, \emph{P.PPLA2}|pwk}Interlaken, 25. V. 10, 9\nobreak{}«.  }\pstart{}{\pb}\textsc{Herrn Dr. Richard Beer-Hofmann}\pend{}\pstart{}\textsc{Wien XVIII.\oindex{XVIII., Waehring@\textbf{XVIII., Währing}, \emph{A.ADM3}|pw}}\pend{}\pstart{}\textsc{Hasenauerstr 59\oindex{Hasenauerstrasse 59@\textbf{Hasenauerstraße 59}, \emph{Wohngebäude (K.WHS)}|pw}}.\pend{}{\bigskip}
\pstart
           \noindent{}\centering{}{\pb}\textcolor{gray}{\textbf{Interlaken\oindex{Interlaken@\textbf{Interlaken}, \emph{P.PPLA2}|pw} – Höheweg\oindex{Hoeheweg@\textbf{Höheweg}, \emph{Straße (K.STR)}|pw}. Jungfrau\oindex{Jungfrau@\textbf{Jungfrau}, \emph{Berg (N.BRG)}|pw}.}}\pend
           \vspace{1em}
\pstart
           \raggedleft{}{\pb}25. 5. 10.\pend
           \vspace{0.5em}
\pstart
           {\pb}Hier iſt es über alle Maßen ſchön; alle Erinnerungen
               übertreffend. Es wi{\geminationm}elt von Hotels,
                  \textcolor{gray}{von} 500–1000 un\textcolor{gray}{d} 1000 \textsc{metern}, – und Ende Juni fänden Sie \introOben{}noch\introOben{} überall was Sie wollen. Wir fahren morgen nach \textsc{Territet\oindex{Territet@\textbf{Territet}, \emph{P.PPL}|pw}}.\pend
           \pstart Herzlichſt Ihr \spacefill\mbox{A.}\pend{}\selectlanguage{ngerman}\vspace{1em}
\pstart
           \noindent{}{[}hs. :{]} Es wird immer \uline{noch}
               schöner!\pend
           \pstart Herzlichste Grüsse! \spacefill\mbox{Olga.}\pend{}\selectlanguage{ngerman}\endnumbering\briefempfaengerindex{Beer-Hofmann, Richard@\textsc{Beer-Hofmann, Richard}!zzzSchnitzler, Olga@\emph{von Olga Schnitzler}!1910-05-251@{25. 5. 1910}|)be}\briefempfaengerindex{Beer-Hofmann, Richard@\textsc{Beer-Hofmann, Richard}!zzzSchnitzler, Arthur@\emph{von Arthur Schnitzler}!1910-05-251@{25. 5. 1910}|)be}\mylabel{L01934h}  \normalsize

\doendnotes{C}
\bigskip
\vfill

\clearpage

\footnotesize

\lohead{\textsc{register}}

% Definiere theindex-Environment komplett neu ohne reledmac
\makeatletter
\renewenvironment{theindex}{%
  \section*{\indexname}%
  \setlength{\parindent}{0pt}%
  \setlength{\parskip}{0pt plus 0.3pt}%
  \let\item\@idxitem
}{%
  \clearpage
}
\makeatother

\IfFileExists{\jobname-pw.ind}{\input{\jobname-pw.ind}}{}

\end{document}

      