%% latex-leseansicht-vorspann.tex
%% Vorspann für die Leseansicht.
%% Lädt die gemeinsame Datei latex-vorspann.tex mit nicht gesetztem Schalter.

\newif\ifkorrekturansicht
\korrekturansichtfalse

\input{../tex-inputs/latex-vorspann}


               \section[Friedrich M. Fels an Arthur Schnitzler, 20. 4. 1893]{ Friedrich M. Fels an Arthur Schnitzler, 20. 4. 1893}\nopagebreak\mylabel{v}\rehead{ }\begin{ledgroupsized}[t]{13cm}\normalsize\beginnumbering\briefempfaengerindex{Schnitzler, Arthur@\textsc{Schnitzler, Arthur}!zzzFels, Friedrich Michael@\emph{von Friedrich Michael Fels}!1893-04-201@{20. 4. 1893}|(be} \toendnotes[C]{\smallbreak\pagebreak[2]} \Standort{DLA, A:Schnitzler, HS.NZ85.1.2956.}
\physDesc{Brief, 1 Blatt, 2 Seiten
\newline{}Handschrift: schwarze Tinte, lateinische Kurrent
\newline{}Schnitzler: mit Bleistift nummeriert: »10« }\toendnotes[C]{\smallbreak}\pstart
           \raggedleft{}{\pb}Meran-Obermais, Erzh. Rainer\oindex{Erzherzog Rainer@\textbf{Erzherzog Rainer}|pw}{\\}20. April \label{K_L00198_1v}\edtext{1892}{\lemma{\textnormal{\emph{1892}}}\Cendnote{\textnormal{Die falsche
                                Jahresangabe von Schnitzler\pwindex{Schnitzler, Arthur 15.05.1862 – 21.10.1931@\textsc{Schnitzler, Arthur} (15.05.1862 – 21.10.1931), \emph{Schriftsteller, Mediziner}|pwk} durch
                                    »3« ersetzt.}}}\label{K_L00198_1h}\pend
           \pstart\center{}Lieber Dr Schnitzler!\pend\pstart
           Entschuldigen Sie, bitte, daſs ich so lange nichts von mir hören lieſs; we{\geminationn} ich wieder in Wien\oindex{Wien@\textbf{Wien}|pw}{ }ſein werde, werde ich Ihnen des
                    ausführlicheren über die Gründe meines höchst unliebenswürdigen und undankbaren
                    Schweigens sprechen. Ende dieses Monats werde ich zurückkehren, nachdem ich
                    vollständig genesen bin. Da aber zuvor die Angelegenheit mit der Rechnung
                    geordnet werden muſs, hätte ich folgende Bitte an Sie: Wollen Sie so freundlich
                    sein, bei den Herren der Deutschen Zeitung\orgindex{Deutsche Zeitung@Deutsche Zeitung|pw} –
                    daſs meine Anstellung ganz sicher sei, darüber hat mir Loris\pwindex{Hofmannsthal, Hugo von 01.02.1874 – 15.07.1929@\textsc{Hofmannsthal, Hugo von} (01.02.1874 – 15.07.1929), \emph{Schriftsteller}|pw} geschrieben – vielleicht zu veranlaſsen, daſs ich
                    vom 1. Mai ab eintreten ka{\geminationn} und \strikeout{zug} daſs mir, we{\geminationn}
                    das der Fall ist, umgehend eine Schrift zugeschickt werde, wodurch die D. Ztg.\orgindex{Deutsche Zeitung@Deutsche Zeitung|pw} erklärt, dem Hotelier\pwindex{Drassl, Josef @\textsc{Drassl, Josef}, \emph{Hotelier}|pwv} des Erzh. Rainer\oindex{Erzherzog Rainer@\textbf{Erzherzog Rainer}|pw}, bis zur Befriedigung seiner Ansprüche,
                    monatlich eine besti{\geminationm}te Su{\geminationm}e etwa ¼ \introOben{}oder ⅓\introOben{} meines
                    Gehaltes zuzusenden. We{\geminationn} ich nicht in kürzester
                    Kürze diese Schrift oder eine andere Sicherstellung \substVorne{}\textsuperscript{erhalten}{\allowbreak}\substDazwischen{}bieten ka{\geminationn}\substHinten{}{ }{\pb}werde ich in sehr unangenehme
                    Verwickelungen geraten und wahrscheinlich noch etwas früher, als hier sonst der
                    Fall wäre, die Strafe für all meine Thaten erhalten.\pend
           \pstart
           Bitte, grüſsen Sie mir alle Beka{\geminationn}ten, die etwa noch
                    geneigt sein sollten, einen Gruſs von mir zu empfangen, und seien Sie selbst
                    herzl. gegrüſst\pend
           \pstart
           von{\\[\baselineskip]}\spacefill\mbox{Fels}\pend
           \leftskip=0em{}\endnumbering\briefempfaengerindex{Schnitzler, Arthur@\textsc{Schnitzler, Arthur}!zzzFels, Friedrich Michael@\emph{von Friedrich Michael Fels}!1893-04-201@{20. 4. 1893}|)be}\mylabel{h}\end{ledgroupsized}  \newcommand{\dateiname}{L00198}\newcommand{\titel}{Friedrich M. Fels an Arthur Schnitzler, 20. 4. 1893}\newcommand{\editorInnen}{Martin Anton Müller und Gerd-Hermann Susen}%% latex-leseansicht-abspann.tex
%% Abspann für die Leseansicht.
%% Der Schalter \ifkorrekturansicht ist bereits durch den Vorspann gesetzt.

%% latex-abspann.tex
%% Gemeinsamer Abspann für Korrekturansicht und Leseansicht.
%% Setzt den Schalter \ifkorrekturansicht voraus (gesetzt in den
%% einbindenden Dateien latex-korrekturansicht-abspann.tex bzw.
%% latex-leseansicht-abspann.tex).
%% ---------------------------------------------------------------

\normalsize

% Das esempio-Environment wird nur in der Leseansicht benötigt
\ifkorrekturansicht\else
\newenvironment{esempio}[3]%
{
    \vspace{1.5ex}
    \rlap{\underline{#1}}
    \par
    \setlength{\parindent}{0cm}
    \nopagebreak
    \leftskip=#2cm
    \rightskip=#3cm
}
{
    \par
}
\fi

\doendnotes{C}
\bigskip
\vfill

\clearpage

\footnotesize

\ifkorrekturansicht
  \lohead{\textsc{register}}
\fi

% theindex-Environment neu definieren ohne reledmac
\makeatletter
\renewenvironment{theindex}{%
  \ifkorrekturansicht
    \section*{\indexname}%
  \else
    \subsubsection*{Index der erwähnten Entitäten}%
  \fi
  \setlength{\parindent}{0pt}%
  \setlength{\parskip}{0pt plus 0.3pt}%
  \let\item\@idxitem
}{%
  \ifkorrekturansicht\clearpage\fi
}
\makeatother

\IfFileExists{\jobname-pw.ind}{\input{\jobname-pw.ind}}{}

% Quellenangabe nur in der Leseansicht
\ifkorrekturansicht\else
% Fallback-Definitionen, falls die .tex-Datei \titel etc. nicht gesetzt hat
\providecommand{\titel}{}
\providecommand{\editorInnen}{}
\providecommand{\dateiname}{\jobname}

\vspace{3cm}

\vfill

\footnotesize
\textsc{Quelle}: \titel. Herausgegeben von {\editorInnen}. In: \emph{Arthur Schnitzler: Briefwechsel mit Autorinnen und Autoren}.
 Digitale Edition, https://schnitzler-briefe.acdh.oeaw.ac.at/{\dateiname}.html (Stand \today)
\fi

\end{document}


      