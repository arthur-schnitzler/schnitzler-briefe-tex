%% latex-korrekturansicht-vorspann.tex
%% Vorspann für die Korrekturansicht.
%% Lädt die gemeinsame Datei latex-vorspann.tex mit gesetztem Schalter.

\newif\ifkorrekturansicht
\korrekturansichttrue

\input{../tex-inputs/latex-vorspann}


\section[Arthur Schnitzler an Hugo Hofmannsthal, {[}5?.{]} 11. 1924]{L02419 Arthur Schnitzler an Hugo Hofmannsthal, {[}5?.{]} 11. 1924}
\nopagebreak\mylabel{L02419v}
\rehead{ }\normalsize\beginnumbering\briefempfaengerindex{Hofmannsthal, Hugo von@\textsc{Hofmannsthal, Hugo von}!zzzSchnitzler, Arthur@\emph{von Arthur Schnitzler}!1924-11-051@{{[}5?.{]} 11. 1924}|(be}
\toendnotes[C]{\smallbreak\pagebreak[2]}\Standort{FDH, Hs-30885,151.}
\physDesc{Postkarte, 982 Zeichen
\newline{}Handschrift: schwarze Tinte, lateinische Kurrent
\newline{}Versand: Stempel: »\nobreak{}\oindex{XVIII., Waehring@\textbf{XVIII., Währing}, \emph{A.ADM3}|pwk}\textcolor{gray}{18/1} Wien, \textcolor{gray}{5} XI 24, 6\nobreak{}«.  }
\buchAbdrucke{\weitereDrucke{Hugo von Hofmannsthal, Arthur Schnitzler: \emph{Briefwechsel}. Frankfurt am Main: \emph{S. Fischer} 1964, S. 300.} }\toendnotes[C]{\smallbreak}\pstart{}{\pb}\label{T_L02419-1v}\edtext{\textcolor{gray}{\textbf{A. S.}}}{\lemma{\textnormal{\emph{A. S.}}}\Cendnote{\textnormal{ovaler Absenderkleber}}}\label{T_L02419-1}\pend{}\pstart{}\textcolor{gray}{\textbf{WIEN, XVIII.}}\oindex{XVIII., Waehring@\textbf{XVIII., Währing}, \emph{A.ADM3}|pw}\pend{}\pstart{}\textcolor{gray}{\textbf{STERNWARTESTR. 71}}\oindex{Sternwartestrasse 71@\textbf{Sternwartestraße 71}, \emph{Wohngebäude (K.WHS)}|pw}\pend{}{\bigskip}\pstart{}an Hr Hugo v Hofma{\geminationn}sthal\pend{}\pstart{}Bad Aussee\oindex{Bad Aussee@\textbf{Bad Aussee}, \emph{P.PPLA3}|pw}\pend{}\pstart{}Steiermark\oindex{Steiermark@\textbf{Steiermark}, \emph{A.ADM1}|pw}.\pend{}{\bigskip}\vspace{1em}
\pstart
           \raggedleft{}{\pb}Wien\oindex{Wien@\textbf{Wien}, \emph{A.ADM2}|pw}, 6. 11. 24\pend
           \vspace{0.5em}
\pstart
           mein lieber Hugo – schönen Dank für Ihren Gruſs aus Aussee\oindex{Bad Aussee@\textbf{Bad Aussee}, \emph{P.PPLA3}|pw}. Über das Frl. Else\pwindex{Fraeulein Else@\emph{Fräulein Else}|pw} hör ich und les ich von allen Seiten so viel gutes, dſs ich sie im
               ganzen beinah überschätzt finden muſs – ebenso wie die K. d. V.\pwindex{Komoedie der Verfuehrung. In drei Akten@\emph{Komödie der Verführung. In drei Akten}|pw} – we{\geminationn} auch vielfach
               gewürdigt, – doch noch in höherm Maſs (und nicht immer reinen Herzens) misverstanden.
               Nun es ist das alte Lied – wir müssen es alle singen. Ich freue mich, dſs Ihr Stück
               vollendet ist. Wohl »Der Thurm\pwindex{Turm. Ein Trauerspiel@\emph{Der Turm. Ein Trauerspiel}|pw}«? Und die neue
               Arbeit –? Wa{\geminationn} werden Sie vorlesen? Wa{\geminationn} kommen Sie nach Wien\oindex{Wien@\textbf{Wien}, \emph{A.ADM2}|pw}? Was haben Sie für Winterpläne? – Ich bleibe wohl vorläufig hier; im
                  Jänner{ }soll {\pb}ich in der Schweiz\oindex{Schweiz@\textbf{Schweiz}, \emph{A.PCLI}|pw} lesen, – was ich hauptsächlich thun
               will, um mir eine Engadin\oindex{Engadin@\textbf{Engadin}, \emph{T.VAL}|pw}er Schnee- u
               Sonnenwoche \strikeout{\textcolor{gray}{ver}} »mit gutem Gewissen« vergönnen zu dürfen. – Ich dictire novellistisch und
               arbeite vorwiegend aphoristisch-fragmentistisch. Schreiben Sie bald wieder, und wärs
               nur ein Wort! Es ist so schön, von Ihnen was direct zu wissen! \pend
           
\pstart
           Alles Herzliche. Ihr \spacefill\mbox{A.}\pend
           \selectlanguage{ngerman}\endnumbering\briefempfaengerindex{Hofmannsthal, Hugo von@\textsc{Hofmannsthal, Hugo von}!zzzSchnitzler, Arthur@\emph{von Arthur Schnitzler}!1924-11-051@{{[}5?.{]} 11. 1924}|)be}\mylabel{L02419h}  \normalsize

\doendnotes{C}
\bigskip
\vfill

\clearpage

\footnotesize

\lohead{\textsc{register}}

% Definiere theindex-Environment komplett neu ohne reledmac
\makeatletter
\renewenvironment{theindex}{%
  \section*{\indexname}%
  \setlength{\parindent}{0pt}%
  \setlength{\parskip}{0pt plus 0.3pt}%
  \let\item\@idxitem
}{%
  \clearpage
}
\makeatother

\IfFileExists{\jobname-pw.ind}{\input{\jobname-pw.ind}}{}

\end{document}

      