%% latex-leseansicht-vorspann.tex
%% Vorspann für die Leseansicht.
%% Lädt die gemeinsame Datei latex-vorspann.tex mit nicht gesetztem Schalter.

\newif\ifkorrekturansicht
\korrekturansichtfalse

\input{../tex-inputs/latex-vorspann}


               \section[Arthur Schnitzler an Hugo Hofmannsthal, {[}5?.{]} 11. 1924]{ Arthur Schnitzler an Hugo Hofmannsthal,
               {[}5?.{]} 11. 1924}\nopagebreak\mylabel{v}\rehead{ }\begin{ledgroupsized}[t]{13cm}\normalsize\beginnumbering\briefempfaengerindex{Hofmannsthal, Hugo von@\textsc{Hofmannsthal, Hugo von}!zzzSchnitzler, Arthur@\emph{von Arthur Schnitzler}!1924-11-051@{{[}5?.{]} 11. 1924}|(be} \toendnotes[C]{\smallbreak\pagebreak[2]} \Standort{FDH, Hs-30885,151.}
\physDesc{Postkarte
\newline{}Handschrift: schwarze Tinte, lateinische Kurrent\newline{}Versand: Stempel: »\nobreak{}\oindex{XVIII., Waehring@\textbf{XVIII., Währing}|pwk}\textcolor{gray}{18/1} Wien, \textcolor{gray}{5} XI 24, 6\nobreak{}«.  }\buchAbdrucke{\weitereDrucke{Hugo von Hofmannsthal, Arthur Schnitzler: \emph{Briefwechsel}. Hg. Therese Nickl und Heinrich Schnitzler. Frankfurt am Main: \emph{S. Fischer} 1964, S. 300.} }\toendnotes[C]{\smallbreak}\pstart{}{\pb}\label{T_L02419-1v}\edtext{\textcolor{gray}{\textbf{A. S.}}}{\lemma{\textnormal{\emph{A. S.}}}\Cendnote{\textnormal{ovaler Absenderkleber}}}\label{T_L02419-1h}\pend{}\pstart{}\textcolor{gray}{\textbf{WIEN, XVIII.}}\oindex{XVIII., Waehring@\textbf{XVIII., Währing}|pw}\pend{}\pstart{}\textcolor{gray}{\textbf{STERNWARTESTR. 71}}\oindex{Sternwartestrasse@\textbf{Sternwartestraße}|pw}\pend{}{\bigskip}\pstart{}an Hr Hugo v Hofma{\geminationn}sthal\pend{}\pstart{}Bad Aussee\oindex{Bad Aussee@\textbf{Bad Aussee}|pw}\pend{}\pstart{}Steiermark\oindex{Steiermark@\textbf{Steiermark}|pw}.\pend{}{\bigskip}\pstart
           \raggedleft{}{\pb}Wien\oindex{Wien@\textbf{Wien}|pw}, 6. 11. 24\pend
           \pstart
           mein lieber Hugo – schönen Dank für Ihren Gruſs aus Aussee\oindex{Bad Aussee@\textbf{Bad Aussee}|pw}. Über das Frl. Else\pwindex{Schnitzler, Arthur 15.05.1862 – 21.10.1931@\textsc{Schnitzler, Arthur} (15.05.1862 – 21.10.1931), \emph{Schriftsteller, Mediziner}!Fraeulein Else01. 10. 1924@\strich\emph{Fräulein Else} {[}01. 10. 1924{]}|pw} hör
               ich und les ich von allen Seiten so viel gutes, dſs ich sie im ganzen beinah
               überschätzt finden muſs – ebenso wie die K. d. V.\pwindex{Schnitzler, Arthur 15.05.1862 – 21.10.1931@\textsc{Schnitzler, Arthur} (15.05.1862 – 21.10.1931), \emph{Schriftsteller, Mediziner}!Komoedie der Verfuehrung. In drei Akten1924@\strich\emph{Komödie der Verführung. In drei Akten} {[}1924{]}|pw} –
                  we{\geminationn} auch vielfach gewürdigt, – doch noch in höherm
               Maſs (und nicht immer reinen Herzens) misverstanden. Nun es ist das alte Lied – wir
               müssen es alle singen. Ich freue mich, dſs Ihr Stück vollendet ist. Wohl »Der Thurm\pwindex{Hofmannsthal, Hugo von 01.02.1874 – 15.07.1929@\textsc{Hofmannsthal, Hugo von} (01.02.1874 – 15.07.1929), \emph{Schriftsteller}!Turm. Ein Trauerspiel1925@\strich\emph{Der Turm. Ein Trauerspiel} {[}1925{]}|pw}«? Und die neue Arbeit –? Wa{\geminationn} werden Sie vorlesen? Wa{\geminationn}
               kommen Sie nach Wien\oindex{Wien@\textbf{Wien}|pw}? Was haben Sie für Winterpläne?
               – Ich bleibe wohl vorläufig hier; im Jänner{ }soll {\pb}ich in der Schweiz\oindex{Schweiz@\textbf{Schweiz}|pw} lesen, – was ich hauptsächlich thun will,
               um mir eine Engadin\oindex{Engadin@\textbf{Engadin}|pw}er Schnee- u Sonnenwoche \strikeout{\textcolor{gray}{ver}} »mit gutem Gewissen« vergönnen
               zu dürfen. – Ich dictire novellistisch und arbeite vorwiegend
               aphoristisch-fragmentistisch. Schreiben Sie bald wieder, und wärs nur ein Wort! Es
               ist so schön, von Ihnen was direct zu wissen! \pend
           \pstart
           Alles Herzliche. Ihr \spacefill\mbox{A.}\pend
                     \endnumbering\briefempfaengerindex{Hofmannsthal, Hugo von@\textsc{Hofmannsthal, Hugo von}!zzzSchnitzler, Arthur@\emph{von Arthur Schnitzler}!1924-11-051@{{[}5?.{]} 11. 1924}|)be}\mylabel{h}\end{ledgroupsized}  \newcommand{\dateiname}{L02419}\newcommand{\titel}{Arthur Schnitzler an Hugo Hofmannsthal, [5?.] 11. 1924}\newcommand{\editorInnen}{Martin Anton Müller und Gerd-Hermann Susen}
            \footnotesize
\begin{ledgroupsized}[t]{11.5cm}
\doendnotes{C}
\end{ledgroupsized}
         %% latex-leseansicht-abspann.tex
%% Abspann für die Leseansicht.
%% Der Schalter \ifkorrekturansicht ist bereits durch den Vorspann gesetzt.

%% latex-abspann.tex
%% Gemeinsamer Abspann für Korrekturansicht und Leseansicht.
%% Setzt den Schalter \ifkorrekturansicht voraus (gesetzt in den
%% einbindenden Dateien latex-korrekturansicht-abspann.tex bzw.
%% latex-leseansicht-abspann.tex).
%% ---------------------------------------------------------------

\normalsize

% Das esempio-Environment wird nur in der Leseansicht benötigt
\ifkorrekturansicht\else
\newenvironment{esempio}[3]%
{
    \vspace{1.5ex}
    \rlap{\underline{#1}}
    \par
    \setlength{\parindent}{0cm}
    \nopagebreak
    \leftskip=#2cm
    \rightskip=#3cm
}
{
    \par
}
\fi

\doendnotes{C}
\bigskip
\vfill

\clearpage

\footnotesize

\ifkorrekturansicht
  \lohead{\textsc{register}}
\fi

% theindex-Environment neu definieren ohne reledmac
\makeatletter
\renewenvironment{theindex}{%
  \ifkorrekturansicht
    \section*{\indexname}%
  \else
    \subsubsection*{Index der erwähnten Entitäten}%
  \fi
  \setlength{\parindent}{0pt}%
  \setlength{\parskip}{0pt plus 0.3pt}%
  \let\item\@idxitem
}{%
  \ifkorrekturansicht\clearpage\fi
}
\makeatother

\IfFileExists{\jobname-pw.ind}{\input{\jobname-pw.ind}}{}

% Quellenangabe nur in der Leseansicht
\ifkorrekturansicht\else
% Fallback-Definitionen, falls die .tex-Datei \titel etc. nicht gesetzt hat
\providecommand{\titel}{}
\providecommand{\editorInnen}{}
\providecommand{\dateiname}{\jobname}

\vspace{3cm}

\vfill

\footnotesize
\textsc{Quelle}: \titel. Herausgegeben von {\editorInnen}. In: \emph{Arthur Schnitzler: Briefwechsel mit Autorinnen und Autoren}.
 Digitale Edition, https://schnitzler-briefe.acdh.oeaw.ac.at/{\dateiname}.html (Stand \today)
\fi

\end{document}


      