%% latex-leseansicht-vorspann.tex
%% Vorspann für die Leseansicht.
%% Lädt die gemeinsame Datei latex-vorspann.tex mit nicht gesetztem Schalter.

\newif\ifkorrekturansicht
\korrekturansichtfalse

\input{../tex-inputs/latex-vorspann}


         
         \renewcommand{\erwaehntePersonen}{Personen: Richard Beer-Hofmann, Clemens von Franckenstein, Georg von Franckenstein, Hugo von Hofmannsthal, Felix Salten}
         \renewcommand{\erwaehnteOrte}{Orte: Altaussee, Marienbad, München, Nürnberg, Passau, Rothenburg ob der Tauber, Salzburg, Seewirt, Wien}
         \renewcommand{\erwaehnteWerke}{Werke: Das Bergwerk zu Falun}
               \section[Hugo von Hofmannsthal an Arthur Schnitzler, 20. 7. {[}1899{]}]{ Hugo von Hofmannsthal an Arthur Schnitzler, 20. 7. {[}1899{]}}\nopagebreak\mylabel{v}\rehead{ }\begin{ledgroupsized}[t]{13cm}\normalsize\beginnumbering\briefempfaengerindex{Schnitzler, Arthur@\textsc{Schnitzler, Arthur}!zzzHofmannsthal, Hugo von@\emph{von Hugo von Hofmannsthal}!1899-07-202@{20. 7. {[}1899{]}}|(be} \toendnotes[C]{\smallbreak\pagebreak[2]} \Standort{CUL, Schnitzler, B 43.}
\physDesc{Brief, 2 Blätter, 7 Seiten, 1520 Zeichen
\newline{}Handschrift: schwarze Tinte, deutsche Kurrent
\newline{}Ordnung: 1) mit Bleistift von unbekannter Hand nummeriert: »\strikeout{154}«  2) mit Bleistift von unbekannter Hand nummeriert:
                                    »152.«. Diese Hand dürfte auch für die
                                 Paginierung der beiden Blätter mit »1« respektive
                                    »2« verantwortlich sein}\buchAbdrucke{\weitereDrucke{1) Hugo von Hofmannsthal: \emph{Briefe. 1890–1901}. Berlin: \emph{S. Fischer} 1935, S. 288–289.} \weitereDrucke{2) Hugo von Hofmannsthal, Arthur Schnitzler: \emph{Briefwechsel}. Hg. Therese Nickl und Heinrich Schnitzler. Frankfurt am Main: \emph{S. Fischer} 1964, S. 126–127.} }\toendnotes[C]{\smallbreak}\pstart
           \noindent{}{\pb}\textcolor{gray}{\textbf{\label{T_L00949-1v}\edtext{hvH}{\lemma{\textnormal{\emph{hvH}}}\Cendnote{\textnormal{gedrucktes Monogramm mit Krone in blauer Farbe}}}\label{T_L00949-1h}}}\pend
           \pstart
           \raggedleft{}Marienbad\oindex{Marienbad@\textbf{Marienbad}|pw}\pend
           \pstart
           \raggedleft{}20 VII\pend
           \pstart{}mein lieber Arthur\pend\pstart
           ich möchte Ihnen gern einen viel ausführlicheren Brief ſchreiben, möchte auch gern
               über Richard\pwindex{Beer-Hofmann, Richard 1866-07-11 – 1945-09-26@\textsc{Beer-Hofmann, Richard} (1866-07-11 – 1945-09-26), \emph{Schriftsteller}|pw} vieles ſagen, aber ich bin ſo
               unglaublich abgeſpannt, ſobald meine tägliche wie im Fieber eintretende Arbeitszeit
               vorüber iſt, daſs {\pb}ich kaum im
               Stand bin die Feder zu halten.\pend
           \pstart
           Ich war mit meinen Nerven noch nie ſo herunter: ein geräuschvoller Speiſeſaal macht
               mir heftige phyſiſche Schmerzen im Genick und lauter ſolche Dummheiten. Ich werde
               nach dem 28\textsuperscript{ten} mindeſtens 14 Tage zu arbeiten aufhören {\pb}und das Landleben führen, da\strikeout{ſ}s mir allein ganz wohl thut: \textsc{tennys} Bad und vielerlei harmloſe Geſellſchaft. Ich gehe daher nach Alt-Auſſee\oindex{Altaussee@\textbf{Altaussee}|pw} entweder zu den \textsc{Franckensteins}\pwindex{Franckenstein, Clemens von 14.07.1875 – 19.08.1942@\textsc{Franckenstein, Clemens von} (14.07.1875 – 19.08.1942), \emph{Theaterleiter, Komponist, Dirigent}|pw}\pwindex{Franckenstein, Georg von 18.03.1878 – 14.10.1953@\textsc{Franckenstein, Georg von} (18.03.1878 – 14.10.1953), \emph{Diplomat}|pw} oder zum \textsc{Seewirth}\oindex{Seewirt@\textbf{Seewirt}|pw}. Vor einer Radreiſe, \uline{jetzt}, hätte ich bei
               meinem übermäßig montirten und ruheloſen Zuſtand direct Angſt. {\pb}Ich werd mich ſchon wieder in
               Ordnung bringen.\pend
           \pstart
           Mein Stück\pwindex{Hofmannsthal, Hugo von 1874-02-01 – 1929-07-15@\textsc{Hofmannsthal, Hugo von} (1874-02-01 – 1929-07-15), \emph{Schriftsteller}!Bergwerk zu Falun1900 – 1933@\strich\emph{Das Bergwerk zu Falun} {[}1900 – 1933{]}|pwv} iſt ein fünfactiges
               märchenartiges Trauerſpiel, in Verſen. 2 Acte ſind nahezu fertig. Ich habe noch nie
               ſo gern an etwas gearbeitet. Fangen Sie nur auch zu arbeiten an.\pend
           \pstart
           Oder machen Sie jetzt mit Salten\pwindex{Salten, Felix 06.09.1869 – 08.10.1945@\textsc{Salten, Felix} (06.09.1869 – 08.10.1945), \emph{Schriftsteller, Journalist}|pw} eine Radtour
                  {\pb}und laſſen für mich und für
                  September nur den Weg \textsc{Passau}\oindex{Passau@\textbf{Passau}|pw} – \textsc{Nürnberg}\oindex{Nuernberg@\textbf{Nürnberg}|pw} – Rothenburg\oindex{Rothenburg ob der Tauber@\textbf{Rothenburg ob der Tauber}|pw} – München\oindex{Muenchen@\textbf{München}|pw} – Salzburg\oindex{Salzburg@\textbf{Salzburg}|pw} in
               Reſerve. Das wäre ſchön!\pend
           \pstart
           Und um den 15. Auguſt träfen wir uns bei Richard\pwindex{Beer-Hofmann, Richard 1866-07-11 – 1945-09-26@\textsc{Beer-Hofmann, Richard} (1866-07-11 – 1945-09-26), \emph{Schriftsteller}|pw}, verbrächten immer den halben Tag arbeitend, gingen
               dann {\pb}nach Salzburg\oindex{Salzburg@\textbf{Salzburg}|pw}, noch mehr arbeitend und träten Anfang
                  September die Reiſe an. Mir folgen, ich bin der Geſcheidtere!\pend
           \pstart
           Herzlich Ihr{\\[\baselineskip]}\spacefill\mbox{Hugo}\pend
           \leftskip=0em{}\pstart
           \noindent{}\textsc{P. S.}\pend
           \pstart
           Es iſt nicht ernſt, daſs ich der Geſcheidtere bin. Sonſt sind Sie vielleicht
                  beleidigt.\pend
           \pstart
           \centering{}{\pb}Immer ſchreiben!\pend
           
         
         \endnumbering\mylabel{h}\end{ledgroupsized}  \newcommand{\dateiname}{L00949}\newcommand{\titel}{Hugo von Hofmannsthal an Arthur Schnitzler, 20. 7. [1899]}\newcommand{\editorInnen}{Martin Anton Müller und Gerd-Hermann Susen}%% latex-leseansicht-abspann.tex
%% Abspann für die Leseansicht.
%% Der Schalter \ifkorrekturansicht ist bereits durch den Vorspann gesetzt.

%% latex-abspann.tex
%% Gemeinsamer Abspann für Korrekturansicht und Leseansicht.
%% Setzt den Schalter \ifkorrekturansicht voraus (gesetzt in den
%% einbindenden Dateien latex-korrekturansicht-abspann.tex bzw.
%% latex-leseansicht-abspann.tex).
%% ---------------------------------------------------------------

\normalsize

% Das esempio-Environment wird nur in der Leseansicht benötigt
\ifkorrekturansicht\else
\newenvironment{esempio}[3]%
{
    \vspace{1.5ex}
    \rlap{\underline{#1}}
    \par
    \setlength{\parindent}{0cm}
    \nopagebreak
    \leftskip=#2cm
    \rightskip=#3cm
}
{
    \par
}
\fi

\doendnotes{C}
\bigskip
\vfill

\clearpage

\footnotesize

\ifkorrekturansicht
  \lohead{\textsc{register}}
\fi

% theindex-Environment neu definieren ohne reledmac
\makeatletter
\renewenvironment{theindex}{%
  \ifkorrekturansicht
    \section*{\indexname}%
  \else
    \subsubsection*{Index der erwähnten Entitäten}%
  \fi
  \setlength{\parindent}{0pt}%
  \setlength{\parskip}{0pt plus 0.3pt}%
  \let\item\@idxitem
}{%
  \ifkorrekturansicht\clearpage\fi
}
\makeatother

\IfFileExists{\jobname-pw.ind}{\input{\jobname-pw.ind}}{}

% Quellenangabe nur in der Leseansicht
\ifkorrekturansicht\else
% Fallback-Definitionen, falls die .tex-Datei \titel etc. nicht gesetzt hat
\providecommand{\titel}{}
\providecommand{\editorInnen}{}
\providecommand{\dateiname}{\jobname}

\vspace{3cm}

\vfill

\footnotesize
\textsc{Quelle}: \titel. Herausgegeben von {\editorInnen}. In: \emph{Arthur Schnitzler: Briefwechsel mit Autorinnen und Autoren}.
 Digitale Edition, https://schnitzler-briefe.acdh.oeaw.ac.at/{\dateiname}.html (Stand \today)
\fi

\end{document}


      