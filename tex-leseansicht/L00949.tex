%% latex-korrekturansicht-vorspann.tex
%% Vorspann für die Korrekturansicht.
%% Lädt die gemeinsame Datei latex-vorspann.tex mit gesetztem Schalter.

\newif\ifkorrekturansicht
\korrekturansichttrue

\input{../tex-inputs/latex-vorspann}


\section[Hugo von Hofmannsthal an Arthur Schnitzler, 20. 7. {[}1899{]}]{L00949 Hugo von Hofmannsthal an Arthur Schnitzler, 20. 7. {[}1899{]}}
\nopagebreak\mylabel{L00949v}
\rehead{ }\normalsize\beginnumbering\briefempfaengerindex{Schnitzler, Arthur@\textsc{Schnitzler, Arthur}!zzzHofmannsthal, Hugo von@\emph{von Hugo von Hofmannsthal}!1899-07-202@{20. 7. {[}1899{]}}|(be}
\toendnotes[C]{\smallbreak\pagebreak[2]}\Standort{CUL, Schnitzler, B 43.}
\physDesc{Brief, 2 Blätter, 7 Seiten, 1520 Zeichen
\newline{}Handschrift: schwarze Tinte, deutsche Kurrent
\newline{}Ordnung: 1) mit Bleistift von unbekannter Hand nummeriert: »\strikeout{154}«  2) mit Bleistift von unbekannter Hand nummeriert:
                                    »152.«. Diese Hand dürfte auch für die
                                 Paginierung der beiden Blätter mit »1« respektive
                                    »2« verantwortlich sein}
\buchAbdrucke{\weitereDrucke{1) Hugo von Hofmannsthal: \emph{Briefe. 1890–1901}. Berlin: \emph{S. Fischer} 1935, S. 288–289.} \weitereDrucke{2) Hugo von Hofmannsthal, Arthur Schnitzler: \emph{Briefwechsel}. Frankfurt am Main: \emph{S. Fischer} 1964, S. 126–127.} }\toendnotes[C]{\smallbreak}
\pstart
           {\pb}\textcolor{gray}{\textbf{\label{T_L00949-1v}\edtext{hvH}{\lemma{\textnormal{\emph{hvH}}}\Cendnote{\textnormal{gedrucktes Monogramm mit Krone in blauer Farbe}}}\label{T_L00949-1}}}\pend
           
\pstart
           \raggedleft{}Marienbad\oindex{Marienbad@\textbf{Marienbad}, \emph{P.PPL}|pw}\pend
           
\pstart
           \raggedleft{}20 VII\pend
           
\pstart{}mein lieber Arthur\pend\vspace{0.5em}
\pstart
           ich möchte Ihnen gern einen viel ausführlicheren Brief ſchreiben, möchte auch gern
               über Richard\pwindex{Beer-Hofmann, Richard 1866-07-11 – 1945-09-26@\textsc{Beer-Hofmann, Richard} (1866-07-11 – 1945-09-26), \emph{Schriftsteller/Schriftstellerin}|pw} vieles ſagen, aber ich bin ſo
               unglaublich abgeſpannt, ſobald meine tägliche wie im Fieber eintretende Arbeitszeit
               vorüber iſt, daſs {\pb}ich kaum im
               Stand bin die Feder zu halten.\pend
           
\pstart
           Ich war mit meinen Nerven noch nie ſo herunter: ein geräuschvoller Speiſeſaal macht
               mir heftige phyſiſche Schmerzen im Genick und lauter ſolche Dummheiten. Ich werde
               nach dem 28\textsuperscript{ten} mindeſtens 14 Tage zu arbeiten aufhören {\pb}und das Landleben führen, da\strikeout{ſ}s mir allein ganz wohl thut: \textsc{tennys} Bad und vielerlei harmloſe Geſellſchaft. Ich gehe daher nach Alt-Auſſee\oindex{Altaussee@\textbf{Altaussee}, \emph{A.ADM3}|pw} entweder zu den \textsc{Franckensteins}\pwindex{Franckenstein, Clemens von 14.07.1875 – 19.08.1942@\textsc{Franckenstein, Clemens von} (14.07.1875 – 19.08.1942), \emph{Theaterleiter/Theaterleiterin, Komponist/Komponistin, Dirigent/Dirigentin}|pw}\pwindex{Franckenstein, Georg von 18.03.1878 – 14.10.1953@\textsc{Franckenstein, Georg von} (18.03.1878 – 14.10.1953), \emph{Diplomat/Diplomatin}|pw} oder zum \textsc{Seewirth}\oindex{Hotel am See@\textbf{Hotel am See}, \emph{Hotel (K.HTL)}|pw}. Vor einer Radreiſe, \uline{jetzt}, hätte ich bei
               meinem übermäßig montirten und ruheloſen Zuſtand direct Angſt. {\pb}Ich werd mich ſchon wieder in
               Ordnung bringen.\pend
           
\pstart
           Mein Stück\pwindex{Bergwerk zu Falun@\emph{Das Bergwerk zu Falun}|pwv} iſt ein fünfactiges
               märchenartiges Trauerſpiel, in Verſen. 2 Acte ſind nahezu fertig. Ich habe noch nie
               ſo gern an etwas gearbeitet. Fangen Sie nur auch zu arbeiten an.\pend
           
\pstart
           Oder machen Sie jetzt mit Salten\pwindex{Salten, Felix 06.09.1869 – 08.10.1945@\textsc{Salten, Felix} (06.09.1869 – 08.10.1945), \emph{Schriftsteller/Schriftstellerin, Journalist/Journalistin, Chefredakteur/Chefredakteurin}|pw} eine Radtour
                  {\pb}und laſſen für mich und für
                  September nur den Weg \textsc{Passau}\oindex{Passau@\textbf{Passau}, \emph{P.PPLA3}|pw} – \textsc{Nürnberg}\oindex{Nuernberg@\textbf{Nürnberg}, \emph{P.PPL}|pw} – Rothenburg\oindex{Rothenburg ob der Tauber@\textbf{Rothenburg ob der Tauber}, \emph{P.PPL}|pw} – München\oindex{Muenchen@\textbf{München}, \emph{P.PPLA}|pw} – Salzburg\oindex{Salzburg@\textbf{Salzburg}, \emph{A.ADM2}|pw} in
               Reſerve. Das wäre ſchön!\pend
           
\pstart
           Und um den 15. Auguſt träfen wir uns bei Richard\pwindex{Beer-Hofmann, Richard 1866-07-11 – 1945-09-26@\textsc{Beer-Hofmann, Richard} (1866-07-11 – 1945-09-26), \emph{Schriftsteller/Schriftstellerin}|pw}, verbrächten immer den halben Tag arbeitend, gingen
               dann {\pb}nach Salzburg\oindex{Salzburg@\textbf{Salzburg}, \emph{A.ADM2}|pw}, noch mehr arbeitend und träten Anfang
                  September die Reiſe an. Mir folgen, ich bin der Geſcheidtere!\pend
           
\pstart
           Herzlich Ihr{\\[\baselineskip]}\spacefill\mbox{Hugo}\pend
           \leftskip=0em{}
\pstart
           \noindent{}\textsc{P. S.}\pend
           
\pstart
           Es iſt nicht ernſt, daſs ich der Geſcheidtere bin. Sonſt sind Sie vielleicht
                  beleidigt.\pend
           
\pstart
           \centering{}{\pb}Immer ſchreiben!\pend
           \selectlanguage{ngerman}\endnumbering\briefempfaengerindex{Schnitzler, Arthur@\textsc{Schnitzler, Arthur}!zzzHofmannsthal, Hugo von@\emph{von Hugo von Hofmannsthal}!1899-07-202@{20. 7. {[}1899{]}}|)be}\mylabel{L00949h}  \normalsize

\doendnotes{C}
\bigskip
\vfill

\clearpage

\footnotesize

\lohead{\textsc{register}}

% Definiere theindex-Environment komplett neu ohne reledmac
\makeatletter
\renewenvironment{theindex}{%
  \section*{\indexname}%
  \setlength{\parindent}{0pt}%
  \setlength{\parskip}{0pt plus 0.3pt}%
  \let\item\@idxitem
}{%
  \clearpage
}
\makeatother

\IfFileExists{\jobname-pw.ind}{\input{\jobname-pw.ind}}{}

\end{document}

      