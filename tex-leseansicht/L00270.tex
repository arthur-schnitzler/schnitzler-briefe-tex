%% latex-korrekturansicht-vorspann.tex
%% Vorspann für die Korrekturansicht.
%% Lädt die gemeinsame Datei latex-vorspann.tex mit gesetztem Schalter.

\newif\ifkorrekturansicht
\korrekturansichttrue

\input{../tex-inputs/latex-vorspann}


\section[Arthur Schnitzler an Richard Beer-Hofmann, 13. 10. 1893]{L00270 Arthur Schnitzler an Richard Beer-Hofmann, 13. 10. 1893}
\nopagebreak\mylabel{L00270v}
\rehead{ }\normalsize\beginnumbering\briefempfaengerindex{Beer-Hofmann, Richard@\textsc{Beer-Hofmann, Richard}!zzzSchnitzler, Arthur@\emph{von Arthur Schnitzler}!1893-10-131@{13. 10. 1893}|(be}
\toendnotes[C]{\smallbreak\pagebreak[2]}\Standort{YCGL, MSS 31.}
\physDesc{Postkarte, 169 Zeichen
\newline{}Handschrift: Bleistift, deutsche Kurrent
\newline{}Versand: 1) Rohrpost  2) Stempel: »\nobreak{}\oindex{I., Innere Stadt@\textbf{I., Innere Stadt}, \emph{A.ADM3}|pwk}Wien 1/1, 13 X 93, 5 40N\nobreak{}«.  3) Stempel: »\nobreak{}\oindex{I., Innere Stadt@\textbf{I., Innere Stadt}, \emph{A.ADM3}|pwk}Wien 1/1, 13 X \textcolor{gray}{9}3, 5 50N\nobreak{}«. }\pstart{}{\pb}\textsc{Herrn Dr. Rich. Beer-Hofmann}\pend{}\pstart{}Wien\oindex{Wien@\textbf{Wien}, \emph{A.ADM2}|pw}\pend{}\pstart{}\textsc{I Wollzeile 15}\oindex{Wollzeile@\textbf{Wollzeile}, \emph{Straße (K.STR)}|pw}.\pend{}{\bigskip}\vspace{1em}
\pstart
           \noindent{}{\pb}Lieber Richard, ich habe den Sitz zur Palaſt-\textsc{revolution}\pwindex{Eine Palastrevolution@\emph{Eine Palastrevolution}|pw} u hole Sie morgen we{\geminationn}s Ihnen recht, um
                  ½ 7 von \textsc{Griensteidl\oindex{Cafe Griensteidl@\textbf{Café Griensteidl}, \emph{Kaffeehaus (K.KAF)}|pw}} ab\pend
           \pstart Ihr er\textcolor{gray}{gb}\spacefill\mbox{Arth}\pend{}\selectlanguage{ngerman}\endnumbering\briefempfaengerindex{Beer-Hofmann, Richard@\textsc{Beer-Hofmann, Richard}!zzzSchnitzler, Arthur@\emph{von Arthur Schnitzler}!1893-10-131@{13. 10. 1893}|)be}\mylabel{L00270h}  \normalsize

\doendnotes{C}
\bigskip
\vfill

\clearpage

\footnotesize

\lohead{\textsc{register}}

% Definiere theindex-Environment komplett neu ohne reledmac
\makeatletter
\renewenvironment{theindex}{%
  \section*{\indexname}%
  \setlength{\parindent}{0pt}%
  \setlength{\parskip}{0pt plus 0.3pt}%
  \let\item\@idxitem
}{%
  \clearpage
}
\makeatother

\IfFileExists{\jobname-pw.ind}{\input{\jobname-pw.ind}}{}

\end{document}

      