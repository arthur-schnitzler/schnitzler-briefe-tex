%% latex-leseansicht-vorspann.tex
%% Vorspann für die Leseansicht.
%% Lädt die gemeinsame Datei latex-vorspann.tex mit nicht gesetztem Schalter.

\newif\ifkorrekturansicht
\korrekturansichtfalse

\input{../tex-inputs/latex-vorspann}


\section[Franz Goldstein und Thomas Mann an Arthur Schnitzler, 19. 8. 1931]{L03881 Franz Goldstein und Thomas Mann an Arthur Schnitzler, 19. 8. 1931}
\nopagebreak\mylabel{L03881v}
\rehead{ }\normalsize\beginnumbering\briefempfaengerindex{Schnitzler, Arthur@\textsc{Schnitzler, Arthur}!zzzMann, Thomas@\emph{von Thomas Mann}!1931-08-191@{19. 8. 1931}|(be}\briefempfaengerindex{Schnitzler, Arthur@\textsc{Schnitzler, Arthur}!zzzGoldstein, Franz@\emph{von Franz Goldstein}!1931-08-191@{19. 8. 1931}|(be}
\toendnotes[C]{\smallbreak\pagebreak[2]}
\correspDesc{Versand  durch Franz Goldstein, Thomas Mann am 19. 8. 1931 in Nida
\newline{}Übermittlung  am 20. 8. 1931 in Nida
\newline{}Weiterleitung  am 22. 8. 1931 in Wien
\newline{}Erhalt  durch Arthur Schnitzler im Zeitraum [23. 8. 1931
                  – 25. 8. 1931?] in Gmunden}\toendnotes[C]{\smallbreak}
\Standort{DLA, A:Schnitzler, 1985.1.3183,7.}
\physDesc{Bildpostkarte, 364 Zeichen
\newline{}Handschrift Franz Goldstein: blaue Tinte, deutsche Kurrent
\newline{}Handschrift Thomas Mann: blaue Tinte, deutsche Kurrent
\newline{}Versand: 1) Stempel: »\nobreak{}\oindex{Nida@\textbf{Nida}, \emph{Hauptstadt}|pwk}Nidden, 2\textcolor{gray}{0. V}{[}III. 31{]}\nobreak{}«.   2) Stempel: »\nobreak{}\oindex{XVIII., Währing@\textbf{XVIII., Währing}, \emph{Verwaltungsgebiet}|pwk}18 Wien 110, 22. VIII. 31, 11\nobreak{}«.  3) mit schwarzer Tinte Streichung der beiden Adresszeilen mit der Wiener\oindex{Wien@\textbf{Wien}, \emph{Verwaltungsgebiet}|pw} Adresse und Ersatz durch:
                                          »\noindent{}\textsc{Gmunden\oindex{Gmunden@\textbf{Gmunden}|pw}}{ / }\textsc{Hotel
                                                \uline{Austria}\oindex{Hotel Austria [Gmunden]@\textbf{Hotel Austria [Gmunden]}, \emph{Hotel}|pw}}« Schnitzler dürfte einen
                                 Nachsendeauftrag durch die Post verfügt haben. }\toendnotes[C]{\smallbreak}\pstart{}{\pb}\textsc{Herrn Dr. Arthur Schnitzler}\pend{}\pstart{}\textsc{Sternwartestr. 71\oindex{Wien@\textbf{Wien}!XVIII., Währing@\textbf{XVIII., Währing}!Sternwartestraße 71@\textbf{Sternwartestraße 71}, \emph{Wohngebäude}|pw}}\pend{}\pstart{}\textsc{Österreich\oindex{Österreich@\textbf{Österreich}|pw}}\pend{}{\bigskip}
\pstart
           {\pb}\textcolor{gray}{\textbf{Kuhrische Nerhrung\oindex{Kurische Nehrung@\textbf{Kurische Nehrung}, \emph{Landzunge}|pw}}}\hfill \textcolor{gray}{\textbf{Hohe Düne bei Nidden\oindex{Parnidis Düne@\textbf{Parnidis Düne}, \emph{Naturdenkmal}|pw}}}\pend
           \vspace{1em}
\pstart
           {\pb}\textsc{Nidden, Hôtel Herm. Blode\oindex{Hotel Hermann Blode@\textbf{Hotel Hermann Blode}, \emph{Hotel}|pw} (Litauen\oindex{Litauen@\textbf{Litauen}|pw}) 19. VIII.}\pend
           
\pstart{}Sehr verehrter Herr Dr.!\pend\vspace{0.5em}
\pstart
           Aus dem zauberhaften \textsc{Nidden}\oindex{Nida@\textbf{Nida}, \emph{Hauptstadt}|pw}, wohin ich von
               Thomas Manns\pwindex{Mann, Katia 24.\,7.\,1883 Feldafing – 25.\,4.\,1980 Kilchberg@\textsc{Mann, Katia} (24.\,7.\,1883 Feldafing – 25.\,4.\,1980 Kilchberg)|pwv} eingeladen bin – wir gedenken Ihrer wiederholt
            herzlichſt – geſtatte ich mir Ihnen{ }ſchönſte Grüße zu{ }ſenden. Stets Ihr ganz ergebner\pend
           \pstart \spacefill\mbox{Franz Goldstein}\pend{}\selectlanguage{ngerman}\vspace{1em}
\pstart
           \noindent{}{[}hs. Mann:{]} Herzlich verehrungsvollen Gruß!\pend
           \pstart \spacefill\mbox{Thomas Mann.}\pend{}\selectlanguage{ngerman}\endnumbering\briefempfaengerindex{Schnitzler, Arthur@\textsc{Schnitzler, Arthur}!zzzMann, Thomas@\emph{von Thomas Mann}!1931-08-191@{19. 8. 1931}|)be}\briefempfaengerindex{Schnitzler, Arthur@\textsc{Schnitzler, Arthur}!zzzGoldstein, Franz@\emph{von Franz Goldstein}!1931-08-191@{19. 8. 1931}|)be}\mylabel{L03881h}
\begin{anhang}
\end{anhang}\newcommand{\dateiname}{L03881}\newcommand{\titel}{Franz Goldstein und Thomas Mann an Arthur Schnitzler, 19. 8. 1931}\newcommand{\editorInnen}{Selma Jahnke und Martin Anton Müller}%% latex-leseansicht-abspann.tex
%% Abspann für die Leseansicht.
%% Der Schalter \ifkorrekturansicht ist bereits durch den Vorspann gesetzt.

%% latex-abspann.tex
%% Gemeinsamer Abspann für Korrekturansicht und Leseansicht.
%% Setzt den Schalter \ifkorrekturansicht voraus (gesetzt in den
%% einbindenden Dateien latex-korrekturansicht-abspann.tex bzw.
%% latex-leseansicht-abspann.tex).
%% ---------------------------------------------------------------

\normalsize

% Das esempio-Environment wird nur in der Leseansicht benötigt
\ifkorrekturansicht\else
\newenvironment{esempio}[3]%
{
    \vspace{1.5ex}
    \rlap{\underline{#1}}
    \par
    \setlength{\parindent}{0cm}
    \nopagebreak
    \leftskip=#2cm
    \rightskip=#3cm
}
{
    \par
}
\fi

\doendnotes{C}
\bigskip
\vfill

\clearpage

\footnotesize

\ifkorrekturansicht
  \lohead{\textsc{register}}
\fi

% theindex-Environment neu definieren ohne reledmac
\makeatletter
\renewenvironment{theindex}{%
  \ifkorrekturansicht
    \section*{\indexname}%
  \else
    \subsubsection*{Index der erwähnten Entitäten}%
  \fi
  \setlength{\parindent}{0pt}%
  \setlength{\parskip}{0pt plus 0.3pt}%
  \let\item\@idxitem
}{%
  \ifkorrekturansicht\clearpage\fi
}
\makeatother

\IfFileExists{\jobname-pw.ind}{\input{\jobname-pw.ind}}{}

% Quellenangabe nur in der Leseansicht
\ifkorrekturansicht\else
% Fallback-Definitionen, falls die .tex-Datei \titel etc. nicht gesetzt hat
\providecommand{\titel}{}
\providecommand{\editorInnen}{}
\providecommand{\dateiname}{\jobname}

\vspace{3cm}

\vfill

\footnotesize
\textsc{Quelle}: \titel. Herausgegeben von {\editorInnen}. In: \emph{Arthur Schnitzler: Briefwechsel mit Autorinnen und Autoren}.
 Digitale Edition, https://schnitzler-briefe.acdh.oeaw.ac.at/{\dateiname}.html (Stand \today)
\fi

\end{document}


