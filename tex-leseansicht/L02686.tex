%% latex-korrekturansicht-vorspann.tex
%% Vorspann für die Korrekturansicht.
%% Lädt die gemeinsame Datei latex-vorspann.tex mit gesetztem Schalter.

\newif\ifkorrekturansicht
\korrekturansichttrue

\input{../tex-inputs/latex-vorspann}


\section[Arthur Schnitzler an Paul Goldmann, 22. 11. 1896]{L02686 Arthur Schnitzler an Paul Goldmann, 22. 11. 1896}
\nopagebreak\mylabel{L02686v}
\rehead{ }\normalsize\beginnumbering\briefempfaengerindex{Goldmann, Paul@\textsc{Goldmann, Paul}!zzzSchnitzler, Arthur@\emph{von Arthur Schnitzler}!1896-11-222@{22. 11. 1896}|(be}
\toendnotes[C]{\smallbreak\pagebreak[2]}\Standort{DLA, A:Schnitzler, HS85.1.5681.}
\physDesc{Brief, Fotokopie2 Blätter, 8 Seiten, 2037 Zeichen, Fragment
\newline{}Handschrift: schwarze Tinte, deutsche Kurrent
\newline{}Zusatz: Von den Korrespondenzstücken Schnitzlers an Goldmann\pwindex{Goldmann, Paul 31.01.1865 – 25.09.1935@\textsc{Goldmann, Paul} (31.01.1865 – 25.09.1935), \emph{Schriftsteller/Schriftstellerin, Journalist/Journalistin}|pw} fehlt weitgehend jede Spur. In der Edition von
                                    Ritterlichkeit\pwindex{Ritterlichkeit@\emph{Ritterlichkeit}|pw} (1975) schreibt die Herausgeberin Rena R. Schlein\pwindex{Schlein, Rena R. *~1919-06-20@\textsc{Schlein, Rena R.} (*~1919-06-20)|pw}: »Zwei
                                    Telegramme und ein Brief Schnitzlers an Goldmann\pwindex{Goldmann, Paul 31.01.1865 – 25.09.1935@\textsc{Goldmann, Paul} (31.01.1865 – 25.09.1935), \emph{Schriftsteller/Schriftstellerin, Journalist/Journalistin}|pw} wurden mir von Dr. Leo P. Reckford\pwindex{Reckford, Leo P. 1903-05-03 – 1988-10-19@\textsc{Reckford, Leo P.} (1903-05-03 – 1988-10-19), \emph{Laryngologe/Laryngologin}|pw}, der diese Dokumente von
                                    der Familie Goldmanns\pwindex{Goldmann, Paul 31.01.1865 – 25.09.1935@\textsc{Goldmann, Paul} (31.01.1865 – 25.09.1935), \emph{Schriftsteller/Schriftstellerin, Journalist/Journalistin}|pw} zum
                                    Geschenk bekam, für meine Arbeit zur Verfügung gestellt«
                                 (S. 1). Reckford\pwindex{Reckford, Leo P. 1903-05-03 – 1988-10-19@\textsc{Reckford, Leo P.} (1903-05-03 – 1988-10-19), \emph{Laryngologe/Laryngologin}|pw} starb 1988, seine Nachkommen haben keine Kenntnis
                                 von diesen (und etwaigen weiteren) Korrespondenzstücken und sie
                                 sind auch nicht auffindbar. Rena
                                    R. Schlein\pwindex{Schlein, Rena R. *~1919-06-20@\textsc{Schlein, Rena R.} (*~1919-06-20)|pw} kam 1919 zur Welt. Ein Kontakt
                                 konnte nicht hergestellt werden. Die vorliegende Kopie besteht aus
                                 einem Doppelblatt mit zwei Seiten, die links die vierte und rechts
                                 die erste Seite des ersten Blattes umfassen. Beim Erstellen der
                                 Kopie wurde der linke Rand der linken Seite nicht ordentlich
                                 adjustiert und fehlt. Die Kopie dürfte durch Reckford\pwindex{Reckford, Leo P. 1903-05-03 – 1988-10-19@\textsc{Reckford, Leo P.} (1903-05-03 – 1988-10-19), \emph{Laryngologe/Laryngologin}|pw} oder Schlein\pwindex{Schlein, Rena R. *~1919-06-20@\textsc{Schlein, Rena R.} (*~1919-06-20)|pw} in den Besitz Heinrich Schnitzlers\pwindex{Schnitzler, Heinrich 09.08.1902 – 12.07.1982@\textsc{Schnitzler, Heinrich} (09.08.1902 – 12.07.1982), \emph{Regisseur/Regisseurin, Schauspieler/Schauspielerin}|pw} gelangt sein. 
\newline{}Editorischer Hinweis: Jene Teile des Briefes, die nicht im Fragment erhalten sind,
                                 werden mit Hilfe der Edition in Ritterlichkeit\pwindex{Ritterlichkeit@\emph{Ritterlichkeit}|pw} ergänzt. Die Verwendung des Schaft-s (»ſ«)
                                 wurde entsprechend den amtlichen Regeln auch auf die nicht
                                 erhaltenen Teile übertragen. }
\buchAbdrucke{\weitereDrucke{1) \pwindex{Ritterlichkeit@\emph{Ritterlichkeit}|pwk}Arthur Schnitzler: \emph{Ritterlichkeit. Fragment aus dem Nachlaß}. Bonn: \emph{Bouvier Verlag Herbert Grundmann} 1975, S. 6–7.} \weitereDrucke{2) Arthur Schnitzler: \emph{Briefe 1875–1912}. Frankfurt am Main: \emph{S. Fischer} 1981, S. 308–309.} }\toendnotes[C]{\smallbreak}
\pstart
           \noindent{}{\pb}So feſt ich auch von dem glücklichen Ausgang
               überzeugt war, mein liebſter Paul – ich bin doch jetzt froher als geſtern um die Zeit. Noch vor Deinem Telegra{\geminationm} haben wir im Kaffehaus von einer Redaction {\pb}das Reſultat telephoniſch erfahren. Und nun ſage mir
               ſelbſt – iſt es nicht jämmerlich, daß Menschen wie Du ſolchen Möglichkeiten
               preisgegeben ſind – oder, wie ich faſt lieber ſagen möchte, preisgegeben zu ſein
               glauben? Ich habe von Leo\pwindex{Van-Jung, Leo 15.10.1866 – 02.07.1939@\textsc{Van-Jung, Leo} (15.10.1866 – 02.07.1939), \emph{Gesangspädagoge/Gesangspädagogin, Mathematiker/Mathematikerin}|pw} manches gehört, ich
               habe auch Deine \label{K_L02686-1v}\edtext{Artikel\pwindex{Enthuellungen ueber die Affaire Dreyfus@\emph{Die Enthüllungen über die Affaire Dreyfus}|pwv}\pwindex{Affaire Dreyfus@\emph{Die Affaire Dreyfus}|pwv}\pwindex{Dreyfus, die oeffentliche Meinung und die deutsche Regierung@\emph{Dreyfus, die öffentliche Meinung und die deutsche Regierung}|pwv}}{\lemma{\textnormal{\emph{Artikel}}}\Cendnote{\textnormal{Siehe Arthur Schnitzler an Paul Goldmann, 21. 11. 1896.
               }}}\label{K_L02686-1} in der Fkt. Ztg.\pwindex{Frankfurter Zeitung@\emph{Frankfurter Zeitung}|pw} alle geleſen – Du haſt
               Dich einfach prachtvoll benommen – auf \uline{Dein} Tun und
               Schreiben hin allein müßte das Verfahren gegen Dreyfus\pwindex{Dreyfus, Alfred 1859-10-09 – 1935-07-12@\textsc{Dreyfus, Alfred} (1859-10-09 – 1935-07-12), \emph{Militär/Militärin}|pw} neu aufgenommen werden.\pend
           
\pstart
           Wenn in dieſer Sache ein Erfolg erzielt werden wird; Dir wird er zu danken ſein. Eine
               ſchönere Selbſtloſig{\pb}keit hat ſelten ein Mann in
               Deiner Lage bewieſen. Es iſt ebenſo edel als blödſinnig, daſs Du Dich geſchlagen haſt
               – wärſt Du aber erſchoſſen worden, ſo hätte die Ungeheuerlichkeit des Blödſinns alles
               andere verſchlungen. Es iſt vorbei – und ich hoffe, daſs Du keiner neuen Gefahr
                  entgegen{\pb}gehſt. Ich wünſche dringend, daß Du Dich
               durch keinen Tropf mehr beleidigt fühlen mögeſt. Und wenn Du genötigt biſt, einen zu
               inſultieren, ſo wirſt Du jedenfalls genau wiſſen, warum Du es tuſt, wirſt alſo immer
               im Recht ſein und kannſt auf die lächerliche Fälſchung verzichten, welche durch einen
               Kugelwechſel in klare Tatſachen hineingetragen wird. Du haſt ja ſchließlich auch
               bewieſen – nachdem das nun einmal notwendig zu ſein ſcheint – daß Du »Mut« haſt; alſo
               auch von dieſer Seite kann man nicht mehr an Dich heran. –\pend
           
\pstart
           {\pb}Vielleicht haſt Du Zeit und Luſt, mir näheres
               mitzuteilen; Du begreifſt es, daß Deine Seelenzuſtände in den verſchiedenen Momenten
               mich auch aufs lebhafteſte intereſſieren, auch darüber ſage mir etwas. –\pend
           
\pstart
           Auf Deinen lieben Brief von neulich antworte ich Dir dieſer Tage. Von mir iſt nur in
               Kürze zu melden, daß ich an den alten pſychiſchen Sachen in ſtörend hohem Maße
               leide. –\pend
           
\pstart
           Leb wohl, mein lieber Paul, und nochmals tauſend Glückwünſche, tauſend Grüße!
               {\\[\baselineskip]}Dein treuer \spacefill\mbox{Arthur}\pend
           \leftskip=0em{}
\pstart
           Wien\oindex{Wien@\textbf{Wien}, \emph{A.ADM2}|pw}{ }22. 11. 96.\pend
           \selectlanguage{ngerman}\endnumbering\briefempfaengerindex{Goldmann, Paul@\textsc{Goldmann, Paul}!zzzSchnitzler, Arthur@\emph{von Arthur Schnitzler}!1896-11-222@{22. 11. 1896}|)be}\mylabel{L02686h}  \normalsize

\doendnotes{C}
\bigskip
\vfill

\clearpage

\footnotesize

\lohead{\textsc{register}}

% Definiere theindex-Environment komplett neu ohne reledmac
\makeatletter
\renewenvironment{theindex}{%
  \section*{\indexname}%
  \setlength{\parindent}{0pt}%
  \setlength{\parskip}{0pt plus 0.3pt}%
  \let\item\@idxitem
}{%
  \clearpage
}
\makeatother

\IfFileExists{\jobname-pw.ind}{\input{\jobname-pw.ind}}{}

\end{document}

      