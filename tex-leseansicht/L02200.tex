\input{../tex-inputs/latex-pdf-vorspann}
\begin{center}
            \textcolor{red}{ENTWURF. ENTZIFFERUNG NOCH NICHT KORREKTURGELESEN}
                      \end{center}
            
               \section[Arthur Schnitzler an Georg Brandes, 18. 12. 1914]{ Arthur Schnitzler an Georg Brandes, 18. 12. 1914}\nopagebreak\mylabel{v}\rehead{ }\begin{ledgroupsized}[t]{13cm}\normalsize\beginnumbering\briefempfaengerindex{Brandes, Georg@\textsc{Brandes, Georg}!zzzSchnitzler, Arthur@\emph{von Arthur Schnitzler}!1914-12-181@{18. 12. 1914}|(be} \toendnotes[C]{\smallbreak\pagebreak[2]} \Standort{Kopenhagen, Det Kongelige Bibliotek, Georg Brandes Arkiv, box 125.}
\physDesc{Bildpostkarte
\newline{}Handschrift: schwarze Tinte, deutsche Kurrent\newline{}Versand: 1) Stempel: »\nobreak{}18. XII. 14\nobreak{}«.  2) Stempel: »\nobreak{}Wien 1, Überprüft\nobreak{}«. \newline{}Ordnung: mit Bleistift von unbekannter Hand nummeriert:
                                    »37« \newline{}Zusatz: Postkartenmotiv mit Olga\pwindex{Schnitzler, Olga 17.01.1882 – 13.01.1970@\textsc{Schnitzler, Olga} (17.01.1882 – 13.01.1970), \emph{Schauspielerin, Sängerin}|pw} und
                                    Heinrich\pwindex{Schnitzler, Heinrich 09.08.1902 – 12.07.1982@\textsc{Schnitzler, Heinrich} (09.08.1902 – 12.07.1982), \emph{Regisseur, Schauspieler}|pw} links vor dem Haus
                                 und Schnitzler und Lili\pwindex{Schnitzler, Lili 13.09.1909 – 26.07.1928@\textsc{Schnitzler, Lili} (13.09.1909 – 26.07.1928)|pw} auf dem
                                 Söller }\buchAbdrucke{\weitereDrucke{Georg Brandes, Arthur Schnitzler: \emph{Ein Briefwechsel}. Hg. Kurt Bergel. Bern: \emph{Francke} 1956, S. 114.} }\pstart{}{\pb}\textsc{Hrn Georg Brandes}\pend{}\pstart{}\textsc{Kopenhagen}\oindex{Kopenhagen@\textbf{Kopenhagen}|pw}\pend{}{\bigskip}\pstart
           \noindent{}\centering{}\textcolor{gray}{\textbf{{\pb}Wien, XVIII, Sternwartestr. 71\oindex{Sternwartestrasse@\textbf{Sternwartestraße}|pw}.}}\pend
           \pstart
           \raggedleft{}{\pb}18. 12. 914\pend
           \pstart
           mein lieber u verehrter Freund, ſeien Sie zu den Feiertagen und dem
                  ko{\geminationm}enden Jahr wieder einmal herzlichſst gegrüßt.
               Heute eben ko{\geminationm}en beſonders gute Nachrichten aus dem
               Nordoſten – vielleicht iſt es mit all dem Graun doch früher zu Ende als wie
               befürchtet. Hier iſt alles in ſchönſster Ordnung, – und man iſt voll Zuverſicht. Ein
               Wort von Ihnen thäte mir wohl! Wir alle {\pb}denken
               Ihrer in Freundſchaft!\pend
           \pstart
           Von Herzen Ihr{\\[\baselineskip]}\spacefill\mbox{Arthur Schnitzler.}\pend
           \leftskip=0em{}\endnumbering\briefempfaengerindex{Brandes, Georg@\textsc{Brandes, Georg}!zzzSchnitzler, Arthur@\emph{von Arthur Schnitzler}!1914-12-181@{18. 12. 1914}|)be}\mylabel{h}\end{ledgroupsized}  \newcommand{\dateiname}{L02200}\newcommand{\titel}{Arthur Schnitzler an Georg Brandes, 18. 12. 1914}\newcommand{\editorInnen}{Martin Anton Müller und Gerd-Hermann Susen}\input{../tex-inputs/latex-pdf-abspann}
      