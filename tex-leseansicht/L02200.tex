%% latex-korrekturansicht-vorspann.tex
%% Vorspann für die Korrekturansicht.
%% Lädt die gemeinsame Datei latex-vorspann.tex mit gesetztem Schalter.

\newif\ifkorrekturansicht
\korrekturansichttrue

\input{../tex-inputs/latex-vorspann}


\section[Arthur Schnitzler an Georg Brandes, 18. 12. 1914]{L02200 Arthur Schnitzler an Georg Brandes, 18. 12. 1914}
\nopagebreak\mylabel{L02200v}
\rehead{ }\normalsize\beginnumbering\briefempfaengerindex{Brandes, Georg@\textsc{Brandes, Georg}!zzzSchnitzler, Arthur@\emph{von Arthur Schnitzler}!1914-12-181@{18. 12. 1914}|(be}
\toendnotes[C]{\smallbreak\pagebreak[2]}\Standort{Kopenhagen, Det Kongelige Bibliotek, Georg Brandes Arkiv, box 125.}
\physDesc{Bildpostkarte, 467 Zeichen
\newline{}Handschrift: 1) schwarze Tinte, deutsche Kurrent\hspace{1em}2) schwarze Tinte, lateinische Kurrent (\noindent{}Adresse)\hspace{1em}
\newline{}Versand: 1) Stempel: »\nobreak{}18. XII. 14\nobreak{}«.   2) Stempel: »\nobreak{}Wien 1, Überprüft\nobreak{}«. 
\newline{}Ordnung: mit Bleistift von unbekannter Hand nummeriert:
                                    »37« 
\newline{}Zusatz: Postkartenmotiv mit Olga\pwindex{Schnitzler, Olga 17.01.1882 – 13.01.1970@\textsc{Schnitzler, Olga} (17.01.1882 – 13.01.1970), \emph{Schauspieler/Schauspielerin, Sänger/Sängerin}|pw}
                                 und Heinrich\pwindex{Schnitzler, Heinrich 09.08.1902 – 12.07.1982@\textsc{Schnitzler, Heinrich} (09.08.1902 – 12.07.1982), \emph{Regisseur/Regisseurin, Schauspieler/Schauspielerin}|pw} links vor dem
                                 Haus und Schnitzler und Lili\pwindex{Cappellini, Lili 13.09.1909 – 26.07.1928@\textsc{Cappellini, Lili} (13.09.1909 – 26.07.1928)|pw}
                                 auf dem Söller }
\buchAbdrucke{\weitereDrucke{Georg Brandes, Arthur Schnitzler: \emph{Ein Briefwechsel}. Bern: \emph{Francke} 1956, S. 114.} }\pstart{}{\pb}Hrn Georg Brandes\pend{}\pstart{}Kopenhagen\oindex{Kopenhagen@\textbf{Kopenhagen}, \emph{P.PPLC}|pw}\pend{}{\bigskip}
\pstart
           \noindent{}\centering{}{\pb}\textcolor{gray}{\textbf{Wien, XVIII, Sternwartestr. 71\oindex{Sternwartestrasse 71@\textbf{Sternwartestraße 71}, \emph{Wohngebäude (K.WHS)}|pw}.}}\pend
           \vspace{1em}
\pstart
           \raggedleft{}{\pb}18. 12. 914\pend
           \vspace{0.5em}
\pstart
           mein lieber u verehrter Freund, ſeien Sie zu den Feiertagen und dem
                  ko{\geminationm}enden Jahr wieder einmal herzlichſt gegrüßt. Heute
               eben ko{\geminationm}en beſonders gute Nachrichten aus dem Nordoſten
               – vielleicht iſt es mit all dem Graun doch früher zu Ende als wie befürchtet. Hier
               iſt alles in ſchönſster Ordnung, – und man iſt voll Zuverſicht. Ein Wort von Ihnen
               thäte mir wohl! Wir alle {\pb}denken Ihrer in
               Freundſchaft!\pend
           
\pstart
           Von Herzen Ihr{\\[\baselineskip]}\spacefill\mbox{Arthur Schnitzler.}\pend
           \leftskip=0em{}\selectlanguage{ngerman}\endnumbering\briefempfaengerindex{Brandes, Georg@\textsc{Brandes, Georg}!zzzSchnitzler, Arthur@\emph{von Arthur Schnitzler}!1914-12-181@{18. 12. 1914}|)be}\mylabel{L02200h}  \normalsize

\doendnotes{C}
\bigskip
\vfill

\clearpage

\footnotesize

\lohead{\textsc{register}}

% Definiere theindex-Environment komplett neu ohne reledmac
\makeatletter
\renewenvironment{theindex}{%
  \section*{\indexname}%
  \setlength{\parindent}{0pt}%
  \setlength{\parskip}{0pt plus 0.3pt}%
  \let\item\@idxitem
}{%
  \clearpage
}
\makeatother

\IfFileExists{\jobname-pw.ind}{\input{\jobname-pw.ind}}{}

\end{document}

      