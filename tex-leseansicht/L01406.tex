%% latex-leseansicht-vorspann.tex
%% Vorspann für die Leseansicht.
%% Lädt die gemeinsame Datei latex-vorspann.tex mit nicht gesetztem Schalter.

\newif\ifkorrekturansicht
\korrekturansichtfalse

\input{../tex-inputs/latex-vorspann}


         \renewcommand{\erwaehnteWerke}{}
               \section[Hugo von Hofmannsthal an Arthur Schnitzler, 1{[}9?{]}. 6. {[}1904{]}]{ Hugo von Hofmannsthal an Arthur Schnitzler, 1{[}9?{]}. 6. {[}1904{]}}\nopagebreak\mylabel{v}\rehead{ }\begin{ledgroupsized}[t]{13cm}\normalsize\beginnumbering \toendnotes[C]{\smallbreak\pagebreak[2]} \Standort{CUL, Schnitzler, B 43.}
\physDesc{Brief, 1 Blatt, 4 Seiten
\newline{}Handschrift: schwarze Tinte, deutsche Kurrent
\newline{}Schnitzler: mit Bleistift die Jahreszahl ergänzt: »904.« \newline{}Ordnung: 1) mit Bleistift von unbekannter Hand nummeriert: »\strikeout{238}«  2) mit Bleistift von unbekannter Hand nummeriert:
                                    »223«}\buchAbdrucke{\weitereDrucke{Hugo von Hofmannsthal, Arthur Schnitzler: \emph{Briefwechsel}. Hg. Therese Nickl und Heinrich Schnitzler. Frankfurt am Main: \emph{S. Fischer} 1964, S. 187.} }\toendnotes[C]{\smallbreak}\pstart
           \raggedleft{}R\oindex{XXXX Ortsangabe fehlt|pw}{ }1\substVorne{}\textsuperscript{5}\substDazwischen{}\textcolor{gray}{9}\substHinten{} VI.\pend
           \pstart
           {\pb}lieber, iſt es nicht ſchrecklich daſs wir in der gleichen Stadt
               leben und uns jahraus jahrein keine zehn mal ſehen!\pend
           \pstart
           Wie traurig wären wir, wenn der andere in eine andere Stadt überſiedeln würde und
               doch, man könnte kaum weniger von einander haben.\pend
           \pstart
           Ich möchte nun ſo gern einmal {\pb}mit
                  Gerty\pwindex{\textcolor{red}{\textsuperscript{XXXX1 indx}}|pw} gleich nach Tiſch zu Euch ko{\geminationm}en oder ſchon zu Tiſch ſo daſs wir zuſa{\geminationm}en dann einen Ausflug machen würden nach Eurer Gegend
               hin, die ich viel zu wenig kenne.\pend
           \pstart
           Samstag und Sonntag nicht Papa\pwindex{\textcolor{red}{\textsuperscript{XXXX1 indx}}|pwv}s wegen, aber ſonſt i{\geminationm}er.\hspace*{1.5em}Bitte bald Antwort, freue mich ſo ſehr auf Sie.{\\}\spacefill\mbox{Hugo}\pend
           \pstart
           \noindent{}\textsc{P. S.}{\\}Ich konnte die erſten paar Tage nach der \label{K_L01406_1v}\edtext{Rückkehr}{\lemma{\textnormal{\emph{Rückkehr}}}\Cendnote{\textnormal{Am
                        10. 6. 1904 kehrte er von einer mehrwöchigen Reise in die Niederlande\oindex{XXXX Ortsangabe fehlt|pwk} zurück.}}}\label{K_L01406_1h} nicht ſchreiben,
                  weil ich von der gräßlichen Du{\geminationm}heit die ich mit dem
                     \label{K_L01406_2v}\edtext{Kraus\pwindex{\textcolor{red}{\textsuperscript{XXXX1 indx}}|pw}-brief}{\lemma{\textnormal{\emph{Kraus-brief}}}\Cendnote{\textnormal{Adolph Donath\pwindex{\textcolor{red}{\textsuperscript{XXXX1 indx}}|pwk} hatte ein Buch für Detlev von Liliencron\pwindex{\textcolor{red}{\textsuperscript{XXXX1 indx}}|pwk} herausgegeben (\emph{Österreichische Dichter zum 60. Geburtstage
                           Detlev von Liliencrons}\textcolor{red}{\textsuperscript{XXXX indx}}. Hg. Adolph
                           Donath\pwindex{\textcolor{red}{\textsuperscript{XXXX1 indx}}|pwk} Wien: \emph{Konegen}XXXX ORGangabe fehlt{ }1904). Hofmannsthal\pwindex{\textcolor{red}{\textsuperscript{XXXX1 indx}}|pwk} hatte – in einem in
                     der \emph{Fackel}\textcolor{red}{\textsuperscript{XXXX indx}} abgedruckten Brief (Hugo von Hofmannsthal\pwindex{\textcolor{red}{\textsuperscript{XXXX1 indx}}|pwk}: \emph{Zur Liliencron-Feier}\textcolor{red}{\textsuperscript{XXXX indx}}. In: \emph{Die Fackel}\textcolor{red}{\textsuperscript{XXXX indx}}, Jg. 6, H. 142, 19. 5. 1904,
                        S. 24–26) die Schuld, warum er daran nicht mitgearbeitet hatte, Donath\pwindex{\textcolor{red}{\textsuperscript{XXXX1 indx}}|pwk} gegeben. Dieser veröffentlichte in
                     Folge die eigentliche Absage Hofmannsthal\pwindex{\textcolor{red}{\textsuperscript{XXXX1 indx}}|pwk}s.
                     Aus diesem ging eine Abneigung gegen Liliencron\pwindex{\textcolor{red}{\textsuperscript{XXXX1 indx}}|pwk} hervor, Hofmannsthal\pwindex{\textcolor{red}{\textsuperscript{XXXX1 indx}}|pwk}
                     war vor aller Öffentlichkeit als Lügner bloßgestellt.}}}\label{K_L01406_2h} gemacht hatte, ſo
                  degoutiert und verſti{\geminationm}t war wie möglich, außerdem
                  hatte ich noch eine \label{K_L01406_3v}\edtext{andere Du{\geminationm}heit}{\lemma{\textnormal{\emph{andere Dummheit}}}\Cendnote{\textnormal{Eventuell verbirgt sich die Erklärung hinter einer gestrichenen Stelle in den
                     Aufzeichnungen Hofmannsthal\pwindex{\textcolor{red}{\textsuperscript{XXXX1 indx}}|pwk}s
                        (S. 477). Demnach hätte er bei einem Tisch gegenüber einer Frau
                     einen \emph{faux pas} begangen.}}}\label{K_L01406_3h} gemacht, ganz {\pb}anderer Gattung aber auch ſehr
                  ärgerlich\pend
           
         
         \endnumbering\mylabel{h}\end{ledgroupsized}  \newcommand{\dateiname}{L01406}\newcommand{\titel}{Hugo von Hofmannsthal an Arthur Schnitzler, 1[9?]. 6. [1904]}\newcommand{\editorInnen}{Martin Anton Müller und Gerd-Hermann Susen}%% latex-leseansicht-abspann.tex
%% Abspann für die Leseansicht.
%% Der Schalter \ifkorrekturansicht ist bereits durch den Vorspann gesetzt.

%% latex-abspann.tex
%% Gemeinsamer Abspann für Korrekturansicht und Leseansicht.
%% Setzt den Schalter \ifkorrekturansicht voraus (gesetzt in den
%% einbindenden Dateien latex-korrekturansicht-abspann.tex bzw.
%% latex-leseansicht-abspann.tex).
%% ---------------------------------------------------------------

\normalsize

% Das esempio-Environment wird nur in der Leseansicht benötigt
\ifkorrekturansicht\else
\newenvironment{esempio}[3]%
{
    \vspace{1.5ex}
    \rlap{\underline{#1}}
    \par
    \setlength{\parindent}{0cm}
    \nopagebreak
    \leftskip=#2cm
    \rightskip=#3cm
}
{
    \par
}
\fi

\doendnotes{C}
\bigskip
\vfill

\clearpage

\footnotesize

\ifkorrekturansicht
  \lohead{\textsc{register}}
\fi

% theindex-Environment neu definieren ohne reledmac
\makeatletter
\renewenvironment{theindex}{%
  \ifkorrekturansicht
    \section*{\indexname}%
  \else
    \subsubsection*{Index der erwähnten Entitäten}%
  \fi
  \setlength{\parindent}{0pt}%
  \setlength{\parskip}{0pt plus 0.3pt}%
  \let\item\@idxitem
}{%
  \ifkorrekturansicht\clearpage\fi
}
\makeatother

\IfFileExists{\jobname-pw.ind}{\input{\jobname-pw.ind}}{}

% Quellenangabe nur in der Leseansicht
\ifkorrekturansicht\else
% Fallback-Definitionen, falls die .tex-Datei \titel etc. nicht gesetzt hat
\providecommand{\titel}{}
\providecommand{\editorInnen}{}
\providecommand{\dateiname}{\jobname}

\vspace{3cm}

\vfill

\footnotesize
\textsc{Quelle}: \titel. Herausgegeben von {\editorInnen}. In: \emph{Arthur Schnitzler: Briefwechsel mit Autorinnen und Autoren}.
 Digitale Edition, https://schnitzler-briefe.acdh.oeaw.ac.at/{\dateiname}.html (Stand \today)
\fi

\end{document}


      