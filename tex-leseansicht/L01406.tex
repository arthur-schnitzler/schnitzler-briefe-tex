%% latex-leseansicht-vorspann.tex
%% Vorspann für die Leseansicht.
%% Lädt die gemeinsame Datei latex-vorspann.tex mit nicht gesetztem Schalter.

\newif\ifkorrekturansicht
\korrekturansichtfalse

\input{../tex-inputs/latex-vorspann}


\section[Hugo von Hofmannsthal an Arthur Schnitzler, 1{[}9?{]}. 6. {[}1904{]}]{L01406 Hugo von Hofmannsthal an Arthur Schnitzler, 1[9?]. 6. [1904]}
\nopagebreak\mylabel{L01406v}
\rehead{ }\normalsize\beginnumbering\briefempfaengerindex{Schnitzler, Arthur@\textsc{Schnitzler, Arthur}!zzzHofmannsthal, Hugo von@\emph{von Hugo von Hofmannsthal}!1904-06-191@{1[9?]. 6. [1904]}|(be}
\toendnotes[C]{\smallbreak\pagebreak[2]}
\correspDesc{Versand  durch Hugo von Hofmannsthal am 1[9?]. 6. [1904] in Rodaun
\newline{}Erhalt  durch Arthur Schnitzler im Zeitraum [20. 6. 1904
                  – 24. 6. 1904?] in Wien}\toendnotes[C]{\smallbreak}
\Standort{CUL, Schnitzler, B 43.}
\physDesc{Brief, 1 Blatt, 4 Seiten, 836 Zeichen
\newline{}Handschrift: schwarze Tinte, deutsche Kurrent
\newline{}Schnitzler: mit Bleistift die Jahreszahl ergänzt: »904.« 
\newline{}Ordnung: 1) mit Bleistift von unbekannter Hand nummeriert: »\strikeout{238}«  2) mit Bleistift von unbekannter Hand nummeriert:
                                    »223«}
\buchAbdrucke{\weitereDrucke{Hugo von Hofmannsthal, Arthur Schnitzler: \emph{Briefwechsel}. Herausgegeben von Therese Nickl und Heinrich Schnitzler. Frankfurt am Main: \emph{S. Fischer} 1964, S. 187.} }\toendnotes[C]{\smallbreak}
\pstart
           \raggedleft{}R\oindex{Wien@\textbf{Wien}!XXIII., Liesing@\textbf{XXIII., Liesing}!Rodaun@\textbf{Rodaun}, \emph{Region}|pw}{ }1\substVorne{}\textsuperscript{5}\substDazwischen{}\textcolor{gray}{9}\substHinten{} VI.\pend
           \vspace{0.5em}
\pstart
           {\pb}lieber, iſt es nicht{ }ſchrecklich daſs wir in der gleichen Stadt
               leben und uns jahraus jahrein keine zehn mal{ }ſehen!\pend
           
\pstart
           Wie traurig wären wir, wenn der andere in eine andere Stadt überſiedeln würde und
               doch, man könnte kaum weniger von einander haben.\pend
           
\pstart
           Ich möchte nun{ }ſo gern einmal {\pb}mit
                  Gerty\pwindex{Hofmannsthal, Gertrude von 16.\,3.\,1880 Wien – 9.\,11.\,1959 Paddington@\textsc{Hofmannsthal, Gertrude von} (16.\,3.\,1880 Wien – 9.\,11.\,1959 Paddington)|pw} gleich nach Tiſch zu Euch ko{\geminationm}en oder{ }ſchon zu Tiſch{ }ſo daſs wir zuſa{\geminationm}en dann einen Ausflug machen würden nach Eurer Gegend
               hin, die ich viel zu wenig kenne.\pend
           
\pstart
           Samstag und Sonntag nicht Papas\pwindex{Hofmannsthal, Hugo August von 21.\,12.\,1841 Wien – 8.\,12.\,1915 ebd.@\textsc{Hofmannsthal, Hugo August von} (21.\,12.\,1841 Wien – 8.\,12.\,1915 ebd.), \emph{Bankdirektor}|pwv} wegen, aber{ }ſonſt i{\geminationm}er.\hspace*{1.5em}Bitte bald Antwort, freue mich{ }ſo{ }ſehr auf Sie.{\\}\spacefill\mbox{Hugo}\pend
           
\pstart
           \noindent{}\textsc{P. S.}{\\}Ich konnte die erſten paar Tage nach der \label{K_L01406-1v}\edtext{Rückkehr}{\lemma{\textnormal{\emph{Rückkehr}}}\Cendnote{\textnormal{Am
                     10. 6. 1904 war Hofmannsthal\pwindex{Hofmannsthal, Hugo von 1.\,2.\,1874 Wien – 15.\,7.\,1929 Rodaun@\textsc{Hofmannsthal, Hugo von} (1.\,2.\,1874 Wien – 15.\,7.\,1929 Rodaun), \emph{Schriftsteller}|pwk} von einer mehrwöchigen Reise in die Niederlande\oindex{Niederlande@\textbf{Niederlande}|pwk} zurückgekehrt.}}}\label{K_L01406-1} nicht{ }ſchreiben, weil ich von der gräßlichen Du{\geminationm}heit die
                  ich mit dem \label{K_L01406-2v}\edtext{Kraus\pwindex{Kraus, Karl 28.\,4.\,1874 Jičín – 12.\,6.\,1936 Wien@\textsc{Kraus, Karl} (28.\,4.\,1874 Jičín – 12.\,6.\,1936 Wien), \emph{Schriftsteller, Publizist, Schriftsteller}|pw}-brief}{\lemma{\textnormal{\emph{Kraus-brief}}}\Cendnote{\textnormal{Adolph Donath\pwindex{Donath, Adolph 9.\,12.\,1876 Kroměříž – 27.\,12.\,1937 Prag@\textsc{Donath, Adolph} (9.\,12.\,1876 Kroměříž – 27.\,12.\,1937 Prag), \emph{Schriftsteller, Journalist, Kritiker}|pwk} hatte ein Buch für Detlev von Liliencron\pwindex{Liliencron, Detlev von 3.\,6.\,1844 Kiel – 22.\,7.\,1909 Rahlstedt@\textsc{Liliencron, Detlev von} (3.\,6.\,1844 Kiel – 22.\,7.\,1909 Rahlstedt), \emph{Schriftsteller, Dichter, Dramatiker}|pwk} herausgegeben (\emph{Österreichische Dichter zum 60. Geburtstage
                           Detlev von Liliencrons}\pwindex{Österreichische Dichter zum 60. Geburtstage Detlev von Liliencrons@\emph{Österreichische Dichter zum 60. Geburtstage Detlev von Liliencrons}|pwk}. Herausgegeben von  Adolph
                           Donath\pwindex{Donath, Adolph 9.\,12.\,1876 Kroměříž – 27.\,12.\,1937 Prag@\textsc{Donath, Adolph} (9.\,12.\,1876 Kroměříž – 27.\,12.\,1937 Prag), \emph{Schriftsteller, Journalist, Kritiker}|pwk} Wien: \emph{Konegen}\orgindex{Carl Konegen@Carl Konegen|pwk}{ }1904). Hofmannsthal\pwindex{Hofmannsthal, Hugo von 1.\,2.\,1874 Wien – 15.\,7.\,1929 Rodaun@\textsc{Hofmannsthal, Hugo von} (1.\,2.\,1874 Wien – 15.\,7.\,1929 Rodaun), \emph{Schriftsteller}|pwk} hatte nicht daran mitgearbeitet.
                     In einem
                     in der \emph{Fackel}\pwindex{Fackel@\emph{Die Fackel}|pwk} abgedruckten Brief (Hugo von Hofmannsthal\pwindex{Hofmannsthal, Hugo von 1.\,2.\,1874 Wien – 15.\,7.\,1929 Rodaun@\textsc{Hofmannsthal, Hugo von} (1.\,2.\,1874 Wien – 15.\,7.\,1929 Rodaun), \emph{Schriftsteller}|pwk}: \emph{Zur Liliencron-Feier}\pwindex{Hofmannsthal, Hugo von 1.\,2.\,1874 Wien – 15.\,7.\,1929 Rodaun@\textsc{Hofmannsthal, Hugo von} (1.\,2.\,1874 Wien – 15.\,7.\,1929 Rodaun), \emph{Schriftsteller}!Zur Liliencron-Feier@\strich\emph{Zur Liliencron-Feier}|pwk}. In: \emph{Die Fackel}\pwindex{Fackel@\emph{Die Fackel}|pwk}, Jg. 6, H. 142,
                        19. 5. 1904, S. 24–26) gab er  Donath\pwindex{Donath, Adolph 9.\,12.\,1876 Kroměříž – 27.\,12.\,1937 Prag@\textsc{Donath, Adolph} (9.\,12.\,1876 Kroměříž – 27.\,12.\,1937 Prag), \emph{Schriftsteller, Journalist, Kritiker}|pwk} die Schuld.
                     Dieser veröffentlichte in Folge den tatsächlichen Absagebrief Hofmannsthals\pwindex{Hofmannsthal, Hugo von 1.\,2.\,1874 Wien – 15.\,7.\,1929 Rodaun@\textsc{Hofmannsthal, Hugo von} (1.\,2.\,1874 Wien – 15.\,7.\,1929 Rodaun), \emph{Schriftsteller}|pwk}, der eine
                     Abneigung gegen Liliencron\pwindex{Liliencron, Detlev von 3.\,6.\,1844 Kiel – 22.\,7.\,1909 Rahlstedt@\textsc{Liliencron, Detlev von} (3.\,6.\,1844 Kiel – 22.\,7.\,1909 Rahlstedt), \emph{Schriftsteller, Dichter, Dramatiker}|pwk} als Ursache erkennen ließ. Hofmannsthal\pwindex{Hofmannsthal, Hugo von 1.\,2.\,1874 Wien – 15.\,7.\,1929 Rodaun@\textsc{Hofmannsthal, Hugo von} (1.\,2.\,1874 Wien – 15.\,7.\,1929 Rodaun), \emph{Schriftsteller}|pwk} war vor aller Öffentlichkeit
                     als Lügner bloßgestellt.}}}\label{K_L01406-2} gemacht hatte,{ }ſo degoutiert und verſti{\geminationm}t war wie möglich, außerdem hatte ich noch eine
                     \label{K_L01406-3v}\edtext{andere Du{\geminationm}heit}{\lemma{\textnormal{\emph{andere Dummheit}}}\Cendnote{\textnormal{Eventuell verbirgt sich die Erklärung hinter einer gestrichenen Stelle in den
                     Aufzeichnungen Hofmannsthals\pwindex{Hofmannsthal, Hugo von 1.\,2.\,1874 Wien – 15.\,7.\,1929 Rodaun@\textsc{Hofmannsthal, Hugo von} (1.\,2.\,1874 Wien – 15.\,7.\,1929 Rodaun), \emph{Schriftsteller}|pwk}
                        (S. 477). Demnach hätte er bei einem Tisch gegenüber einer Frau
                     einen \emph{faux pas} begangen.}}}\label{K_L01406-3} gemacht, ganz {\pb}anderer Gattung aber auch{ }ſehr
                  ärgerlich\pend
           \selectlanguage{ngerman}\endnumbering\briefempfaengerindex{Schnitzler, Arthur@\textsc{Schnitzler, Arthur}!zzzHofmannsthal, Hugo von@\emph{von Hugo von Hofmannsthal}!1904-06-191@{1[9?]. 6. [1904]}|)be}\mylabel{L01406h}  \newcommand{\dateiname}{L01406}\newcommand{\titel}{Hugo von Hofmannsthal an Arthur Schnitzler, 1[9?]. 6. [1904]}\newcommand{\editorInnen}{Martin Anton Müller und Gerd-Hermann Susen}%% latex-leseansicht-abspann.tex
%% Abspann für die Leseansicht.
%% Der Schalter \ifkorrekturansicht ist bereits durch den Vorspann gesetzt.

%% latex-abspann.tex
%% Gemeinsamer Abspann für Korrekturansicht und Leseansicht.
%% Setzt den Schalter \ifkorrekturansicht voraus (gesetzt in den
%% einbindenden Dateien latex-korrekturansicht-abspann.tex bzw.
%% latex-leseansicht-abspann.tex).
%% ---------------------------------------------------------------

\normalsize

% Das esempio-Environment wird nur in der Leseansicht benötigt
\ifkorrekturansicht\else
\newenvironment{esempio}[3]%
{
    \vspace{1.5ex}
    \rlap{\underline{#1}}
    \par
    \setlength{\parindent}{0cm}
    \nopagebreak
    \leftskip=#2cm
    \rightskip=#3cm
}
{
    \par
}
\fi

\doendnotes{C}
\bigskip
\vfill

\clearpage

\footnotesize

\ifkorrekturansicht
  \lohead{\textsc{register}}
\fi

% theindex-Environment neu definieren ohne reledmac
\makeatletter
\renewenvironment{theindex}{%
  \ifkorrekturansicht
    \section*{\indexname}%
  \else
    \subsubsection*{Index der erwähnten Entitäten}%
  \fi
  \setlength{\parindent}{0pt}%
  \setlength{\parskip}{0pt plus 0.3pt}%
  \let\item\@idxitem
}{%
  \ifkorrekturansicht\clearpage\fi
}
\makeatother

\IfFileExists{\jobname-pw.ind}{\input{\jobname-pw.ind}}{}

% Quellenangabe nur in der Leseansicht
\ifkorrekturansicht\else
% Fallback-Definitionen, falls die .tex-Datei \titel etc. nicht gesetzt hat
\providecommand{\titel}{}
\providecommand{\editorInnen}{}
\providecommand{\dateiname}{\jobname}

\vspace{3cm}

\vfill

\footnotesize
\textsc{Quelle}: \titel. Herausgegeben von {\editorInnen}. In: \emph{Arthur Schnitzler: Briefwechsel mit Autorinnen und Autoren}.
 Digitale Edition, https://schnitzler-briefe.acdh.oeaw.ac.at/{\dateiname}.html (Stand \today)
\fi

\end{document}


