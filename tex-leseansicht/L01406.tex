%% latex-korrekturansicht-vorspann.tex
%% Vorspann für die Korrekturansicht.
%% Lädt die gemeinsame Datei latex-vorspann.tex mit gesetztem Schalter.

\newif\ifkorrekturansicht
\korrekturansichttrue

\input{../tex-inputs/latex-vorspann}


\section[Hugo von Hofmannsthal an Arthur Schnitzler, 1{[}9?{]}. 6. {[}1904{]}]{L01406 Hugo von Hofmannsthal an Arthur Schnitzler, 1{[}9?{]}. 6. {[}1904{]}}
\nopagebreak\mylabel{L01406v}
\rehead{ }\normalsize\beginnumbering\briefempfaengerindex{Schnitzler, Arthur@\textsc{Schnitzler, Arthur}!zzzHofmannsthal, Hugo von@\emph{von Hugo von Hofmannsthal}!1904-06-191@{1{[}9?{]}. 6. {[}1904{]}}|(be}
\toendnotes[C]{\smallbreak\pagebreak[2]}\Standort{CUL, Schnitzler, B 43.}
\physDesc{Brief, 1 Blatt, 4 Seiten, 836 Zeichen
\newline{}Handschrift: schwarze Tinte, deutsche Kurrent
\newline{}Schnitzler: mit Bleistift die Jahreszahl ergänzt: »904.« 
\newline{}Ordnung: 1) mit Bleistift von unbekannter Hand nummeriert: »\strikeout{238}«  2) mit Bleistift von unbekannter Hand nummeriert:
                                    »223«}
\buchAbdrucke{\weitereDrucke{Hugo von Hofmannsthal, Arthur Schnitzler: \emph{Briefwechsel}. Frankfurt am Main: \emph{S. Fischer} 1964, S. 187.} }\toendnotes[C]{\smallbreak}
\pstart
           \raggedleft{}R\oindex{Rodaun@\textbf{Rodaun}, \emph{A.ADM4}|pw}{ }1\substVorne{}\textsuperscript{5}\substDazwischen{}\textcolor{gray}{9}\substHinten{} VI.\pend
           \vspace{0.5em}
\pstart
           {\pb}lieber, iſt es nicht ſchrecklich daſs wir in der gleichen Stadt
               leben und uns jahraus jahrein keine zehn mal ſehen!\pend
           
\pstart
           Wie traurig wären wir, wenn der andere in eine andere Stadt überſiedeln würde und
               doch, man könnte kaum weniger von einander haben.\pend
           
\pstart
           Ich möchte nun ſo gern einmal {\pb}mit
                  Gerty\pwindex{Hofmannsthal, Gertrude von 16.03.1880 – 09.11.1959@\textsc{Hofmannsthal, Gertrude von} (16.03.1880 – 09.11.1959)|pw} gleich nach Tiſch zu Euch ko{\geminationm}en oder ſchon zu Tiſch ſo daſs wir zuſa{\geminationm}en dann einen Ausflug machen würden nach Eurer Gegend
               hin, die ich viel zu wenig kenne.\pend
           
\pstart
           Samstag und Sonntag nicht Papas\pwindex{Hofmannsthal, Hugo August von 21.12.1841 – 08.12.1915@\textsc{Hofmannsthal, Hugo August von} (21.12.1841 – 08.12.1915), \emph{Bankdirektor/Bankdirektorin}|pwv} wegen, aber ſonſt i{\geminationm}er.\hspace*{1.5em}Bitte bald Antwort, freue mich ſo ſehr auf Sie.{\\}\spacefill\mbox{Hugo}\pend
           
\pstart
           \noindent{}\textsc{P. S.}{\\}Ich konnte die erſten paar Tage nach der \label{K_L01406-1v}\edtext{Rückkehr}{\lemma{\textnormal{\emph{Rückkehr}}}\Cendnote{\textnormal{Am
                     10. 6. 1904 war Hofmannsthal\pwindex{Hofmannsthal, Hugo von 1874-02-01 – 1929-07-15@\textsc{Hofmannsthal, Hugo von} (1874-02-01 – 1929-07-15), \emph{Schriftsteller/Schriftstellerin}|pwk} von einer mehrwöchigen Reise in die Niederlande\oindex{Niederlande@\textbf{Niederlande}, \emph{A.PCLI}|pwk} zurückgekehrt.}}}\label{K_L01406-1} nicht
                  ſchreiben, weil ich von der gräßlichen Du{\geminationm}heit die
                  ich mit dem \label{K_L01406-2v}\edtext{Kraus\pwindex{Kraus, Karl 28.04.1874 – 12.06.1936@\textsc{Kraus, Karl} (28.04.1874 – 12.06.1936), \emph{Schriftsteller/Schriftstellerin, Publizist/Publizistin, Schriftsteller/Schriftstellerin}|pw}-brief}{\lemma{\textnormal{\emph{Kraus-brief}}}\Cendnote{\textnormal{Adolph Donath\pwindex{Donath, Adolph 09.12.1876 – 27.12.1937@\textsc{Donath, Adolph} (09.12.1876 – 27.12.1937), \emph{Schriftsteller/Schriftstellerin, Journalist/Journalistin, Kritiker/Kritikerin}|pwk} hatte ein Buch für Detlev von Liliencron\pwindex{Liliencron, Detlev von 03.06.1844 – 22.07.1909@\textsc{Liliencron, Detlev von} (03.06.1844 – 22.07.1909), \emph{Schriftsteller/Schriftstellerin, Dichter/Dichterin, Dramatiker/Dramatikerin}|pwk} herausgegeben (\emph{Österreichische Dichter zum 60. Geburtstage
                           Detlev von Liliencrons}\pwindex{Oesterreichische Dichter zum 60. Geburtstage Detlev von Liliencrons@\emph{Österreichische Dichter zum 60. Geburtstage Detlev von Liliencrons}|pwk}. Herausgegeben von  Adolph
                           Donath\pwindex{Donath, Adolph 09.12.1876 – 27.12.1937@\textsc{Donath, Adolph} (09.12.1876 – 27.12.1937), \emph{Schriftsteller/Schriftstellerin, Journalist/Journalistin, Kritiker/Kritikerin}|pwk} Wien: \emph{Konegen}\orgindex{Carl Konegen@Carl Konegen|pwk}{ }1904). Hofmannsthal\pwindex{Hofmannsthal, Hugo von 1874-02-01 – 1929-07-15@\textsc{Hofmannsthal, Hugo von} (1874-02-01 – 1929-07-15), \emph{Schriftsteller/Schriftstellerin}|pwk} hatte nicht daran mitgearbeitet.
                     In einem
                     in der \emph{Fackel}\pwindex{Fackel@\emph{Die Fackel}|pwk} abgedruckten Brief (Hugo von Hofmannsthal\pwindex{Hofmannsthal, Hugo von 1874-02-01 – 1929-07-15@\textsc{Hofmannsthal, Hugo von} (1874-02-01 – 1929-07-15), \emph{Schriftsteller/Schriftstellerin}|pwk}: \emph{Zur Liliencron-Feier}\pwindex{Zur Liliencron-Feier@\emph{Zur Liliencron-Feier}|pwk}. In: \emph{Die Fackel}\pwindex{Fackel@\emph{Die Fackel}|pwk}, Jg. 6, H. 142,
                        19. 5. 1904, S. 24–26) gab er  Donath\pwindex{Donath, Adolph 09.12.1876 – 27.12.1937@\textsc{Donath, Adolph} (09.12.1876 – 27.12.1937), \emph{Schriftsteller/Schriftstellerin, Journalist/Journalistin, Kritiker/Kritikerin}|pwk} die Schuld.
                     Dieser veröffentlichte in Folge den tatsächlichen Absagebrief Hofmannsthals\pwindex{Hofmannsthal, Hugo von 1874-02-01 – 1929-07-15@\textsc{Hofmannsthal, Hugo von} (1874-02-01 – 1929-07-15), \emph{Schriftsteller/Schriftstellerin}|pwk}, der eine
                     Abneigung gegen Liliencron\pwindex{Liliencron, Detlev von 03.06.1844 – 22.07.1909@\textsc{Liliencron, Detlev von} (03.06.1844 – 22.07.1909), \emph{Schriftsteller/Schriftstellerin, Dichter/Dichterin, Dramatiker/Dramatikerin}|pwk} als Ursache erkennen ließ. Hofmannsthal\pwindex{Hofmannsthal, Hugo von 1874-02-01 – 1929-07-15@\textsc{Hofmannsthal, Hugo von} (1874-02-01 – 1929-07-15), \emph{Schriftsteller/Schriftstellerin}|pwk} war vor aller Öffentlichkeit
                     als Lügner bloßgestellt.}}}\label{K_L01406-2} gemacht hatte, ſo degoutiert und verſti{\geminationm}t war wie möglich, außerdem hatte ich noch eine
                     \label{K_L01406-3v}\edtext{andere Du{\geminationm}heit}{\lemma{\textnormal{\emph{andere Dummheit}}}\Cendnote{\textnormal{Eventuell verbirgt sich die Erklärung hinter einer gestrichenen Stelle in den
                     Aufzeichnungen Hofmannsthals\pwindex{Hofmannsthal, Hugo von 1874-02-01 – 1929-07-15@\textsc{Hofmannsthal, Hugo von} (1874-02-01 – 1929-07-15), \emph{Schriftsteller/Schriftstellerin}|pwk}
                        (S. 477). Demnach hätte er bei einem Tisch gegenüber einer Frau
                     einen \emph{faux pas} begangen.}}}\label{K_L01406-3} gemacht, ganz {\pb}anderer Gattung aber auch ſehr
                  ärgerlich\pend
           \selectlanguage{ngerman}\endnumbering\briefempfaengerindex{Schnitzler, Arthur@\textsc{Schnitzler, Arthur}!zzzHofmannsthal, Hugo von@\emph{von Hugo von Hofmannsthal}!1904-06-191@{1{[}9?{]}. 6. {[}1904{]}}|)be}\mylabel{L01406h}  \normalsize

\doendnotes{C}
\bigskip
\vfill

\clearpage

\footnotesize

\lohead{\textsc{register}}

% Definiere theindex-Environment komplett neu ohne reledmac
\makeatletter
\renewenvironment{theindex}{%
  \section*{\indexname}%
  \setlength{\parindent}{0pt}%
  \setlength{\parskip}{0pt plus 0.3pt}%
  \let\item\@idxitem
}{%
  \clearpage
}
\makeatother

\IfFileExists{\jobname-pw.ind}{\input{\jobname-pw.ind}}{}

\end{document}

      