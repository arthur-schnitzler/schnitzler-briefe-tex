%% latex-korrekturansicht-vorspann.tex
%% Vorspann für die Korrekturansicht.
%% Lädt die gemeinsame Datei latex-vorspann.tex mit gesetztem Schalter.

\newif\ifkorrekturansicht
\korrekturansichttrue

\input{../tex-inputs/latex-vorspann}


\section[Hugo von Hofmannsthal an Arthur Schnitzler, {[}18. 2. 1893{]}]{L00179 Hugo von Hofmannsthal an Arthur Schnitzler, {[}18. 2. 1893{]}}
\nopagebreak\mylabel{L00179v}
\rehead{ }\normalsize\beginnumbering\briefempfaengerindex{Schnitzler, Arthur@\textsc{Schnitzler, Arthur}!zzzHofmannsthal, Hugo von@\emph{von Hugo von Hofmannsthal}!1893-02-182@{{[}18. 2. 1893{]}}|(be}
\toendnotes[C]{\smallbreak\pagebreak[2]}\Standort{CUL, Schnitzler, B 43.}
\physDesc{Brief, 1 Blatt, 3 Seiten, 874 Zeichen (aufgeprägtes Wappen)
\newline{}Handschrift: schwarze Tinte, deutsche Kurrent
\newline{}Schnitzler: mit Bleistift nummeriert: »38« }
\buchAbdrucke{\weitereDrucke{1) Hugo von Hofmannsthal, Arthur Schnitzler: \emph{Briefwechsel}. Frankfurt am Main: \emph{S. Fischer} 1964, S. 48.} \weitereDrucke{2) Hermann Bahr, Arthur Schnitzler: \emph{Briefwechsel, Aufzeichnungen, Dokumente (1891–1931)}. Göttingen: \emph{Wallstein} 2018, S. 33.} }
\pstart
           \raggedleft{}{\pb}Samstag abends.\pend
           
\pstart\center{}Lieber Arthur.\pend\vspace{0.5em}
\pstart
           Ich komme möglicherweiſe nach einer Geſellschaft ins Central\oindex{Cafe Central@\textbf{Café Central}, \emph{Kaffeehaus (K.KAF)}|pw}, antworte aber lieber ſo. Der Brief von Fels\pwindex{Fels, Friedrich Michael *~1864@\textsc{Fels, Friedrich Michael} (*~1864), \emph{Journalist/Journalistin}|pw} hat mich ſehr ſchmerzlich berührt. Er muſs jedenfalls
               unten erhalten werden; ich werde ihm dazu auch ſelbſt ſchriftlich zureden und hoffe
               Ihnen in den allernächſten Tagen wenigſtens circa 25 fl ſchicken zu können. Bahr\pwindex{Bahr, Hermann 19.07.1863 – 15.01.1934@\textsc{Bahr, Hermann} (19.07.1863 – 15.01.1934), \emph{Schriftsteller/Schriftstellerin, Kritiker/Kritikerin}|pw} ist momentan in Berlin\oindex{Berlin@\textbf{Berlin}, \emph{P.PPLC}|pw}, {\pb}er
               kommt \substVorne{}\textsuperscript{Dienstag}\substDazwischen{}Montag\substHinten{} abends zurück; Dienstag, ſpäteſtens Mittwoch werde ich ernſthaft und
               endgiltig mit ihm reden. Er hat allen möglichen guten Willen, nur nicht die Energie,
               um die mit ſolchen Dingen verbundenen ekelhaften kleinlichen Anſtände zu überwinden.
               Er muſs ſie aber eben haben. Also \uline{ich} für meinen
               Theil fürchte mich vor gar nichts als vor der muthloſen {\pb}Traurigkeit des Fels\pwindex{Fels, Friedrich Michael *~1864@\textsc{Fels, Friedrich Michael} (*~1864), \emph{Journalist/Journalistin}|pw}, die ja hoffentlich vor guter Luft und Ernährung weichen
               wird. Das übrige wird ſich und werden wir finden.\pend
           
\pstart
           Herzlichſt{\\[\baselineskip]}\spacefill\mbox{Loris}\pend
           \leftskip=0em{}\selectlanguage{ngerman}\endnumbering\briefempfaengerindex{Schnitzler, Arthur@\textsc{Schnitzler, Arthur}!zzzHofmannsthal, Hugo von@\emph{von Hugo von Hofmannsthal}!1893-02-182@{{[}18. 2. 1893{]}}|)be}\mylabel{L00179h}  \normalsize

\doendnotes{C}
\bigskip
\vfill

\clearpage

\footnotesize

\lohead{\textsc{register}}

% Definiere theindex-Environment komplett neu ohne reledmac
\makeatletter
\renewenvironment{theindex}{%
  \section*{\indexname}%
  \setlength{\parindent}{0pt}%
  \setlength{\parskip}{0pt plus 0.3pt}%
  \let\item\@idxitem
}{%
  \clearpage
}
\makeatother

\IfFileExists{\jobname-pw.ind}{\input{\jobname-pw.ind}}{}

\end{document}

      