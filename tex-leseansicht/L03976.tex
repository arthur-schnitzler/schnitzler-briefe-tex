%% latex-leseansicht-vorspann.tex
%% Vorspann für die Leseansicht.
%% Lädt die gemeinsame Datei latex-vorspann.tex mit nicht gesetztem Schalter.

\newif\ifkorrekturansicht
\korrekturansichtfalse

\input{../tex-inputs/latex-vorspann}


\section[Arthur Schnitzler an Berta Zuckerkandl, 27. 5. 1926]{L03976 Arthur Schnitzler an Berta Zuckerkandl, 27. 5. 1926}
\nopagebreak\mylabel{L03976v}
\rehead{ }\normalsize\beginnumbering\briefempfaengerindex{Zuckerkandl, Berta@\textsc{Zuckerkandl, Berta}!zzzSchnitzler, Arthur@\emph{von Arthur Schnitzler}!1926-05-271@{27. 5. 1926}|(be}
\toendnotes[C]{\smallbreak\pagebreak[2]}
\correspDesc{Versand  durch Arthur Schnitzler am 27. 5. 1926 in Wien
\newline{}Erhalt  durch Berta Zuckerkandl im Zeitraum [5. 2. 1926
                  – 9. 2. 1926?] in Paris}\toendnotes[C]{\smallbreak}
\Standort{DLA, HS.1985.1.2282.}
\physDesc{Brief, Durchschlag, 1 Blatt, 2 Seiten, 2028 Zeichen
\newline{}Schreibmaschine
\newline{}Handschrift: roter Buntstift, lateinische Kurrent (\noindent{}beschriftet: »\uline{Zuckerkandl}« und »\uline{Paris}«, vierzehn Unterstreichungen)}\toendnotes[C]{\smallbreak}
\pstart
           \raggedleft{}{\pb}27. 5. 1926.\pend
           
\pstart{}Verehrte liebe Frau Hofrätin.\pend\vspace{0.5em}
\pstart
           Seit einigen Tagen bin ich von meiner \label{K_L03976-1v}\edtext{Reise}{\lemma{\textnormal{\emph{Reise}}}\Cendnote{\textnormal{Schnitzler verließ Wien\oindex{Wien@\textbf{Wien}, \emph{Verwaltungsgebiet}|pwk} am 16. 4. 1926 und schiffte sich mit seiner Tochter Lili\pwindex{Cappellini, Lili 13.\,9.\,1909 Wien – 26.\,7.\,1928 Venedig@\textsc{Cappellini, Lili} (13.\,9.\,1909 Wien – 26.\,7.\,1928 Venedig)|pwk} in Triest\oindex{Triest@\textbf{Triest}, \emph{Verwaltungsgebiet}|pwk} auf der
                     \emph{Martha Washington}\orgindex{Martha Washington@Martha Washington|pwk} ein, um durch das Mittelmeer\oindex{Mittelmeer@\textbf{Mittelmeer}|pwk} eine Seereise bis Lissabon\oindex{Lissabon@\textbf{Lissabon}, \emph{Hauptstadt}|pwk} und auf der \emph{Adolf
                     Woermann}\orgindex{Adolf Woermann@Adolf Woermann|pwk} weiter nach Gran Canaria\oindex{Gran Canaria@\textbf{Gran Canaria}, \emph{Insel}|pwk} zu
                  unternehmen. Die Rückreise auf der \emph{Antonio
                     Delfino}\orgindex{Antonio Delfino@Antonio Delfino|pwk} führte nach Hamburg\oindex{Hamburg@\textbf{Hamburg}|pwk}. Ein
                  Aufenthalt von einer Woche in Berlin\oindex{Berlin@\textbf{Berlin}, \emph{Hauptstadt}|pwk} schloss
                  sich an, am 20. 5. 1926 kehrte Schnitzler
                  nach Wien\oindex{Wien@\textbf{Wien}, \emph{Verwaltungsgebiet}|pwk} zurück.}}}\label{K_L03976-1} wieder zurück, die
               sehr schön und interessant war und höre, dass Sie in Paris\oindex{Paris@\textbf{Paris}, \emph{Hauptstadt}|pw} sind und Anfang Juni wieder zurückkommen. So will ich Sie
               heute nur bitten mir, wenn Sie Zeit haben, ein Wort über den Stand meiner
               Angelegenheiten in Frankreich\oindex{Frankreich@\textbf{Frankreich}|pw} mitzuteilen,
               insbesondere möchte ich wissen, ob die »Else\pwindex{Schnitzler, Arthur 15. 5. 1862 Wien – 21. 10. 1931 ebd.@\textsc{Schnitzler, Arthur} (15. 5. 1862 Wien – 21. 10. 1931 ebd.), \emph{Schriftsteller, Mediziner}!Fräulein Else@\strich\emph{Fräulein Else}|pw}\pwindex{Schnitzler, Arthur 15. 5. 1862 Wien – 21. 10. 1931 ebd.@\textsc{Schnitzler, Arthur} (15. 5. 1862 Wien – 21. 10. 1931 ebd.), \emph{Schriftsteller, Mediziner}!Madmoiselle Else@\strich\emph{Madmoiselle Else}|pw}« nicht bald erscheinen wird. Das Honorar resp. die Garantie, die Herr
                  Delamain\pwindex{Delamain, Maurice 28.\,4.\,1883 Jarnac – 2.\,5.\,1974 Paris@\textsc{Delamain, Maurice} (28.\,4.\,1883 Jarnac – 2.\,5.\,1974 Paris), \emph{Kritiker, Rechtsanwalt, Verleger}|pw} bisher nicht geschickt hat, die
               500 Francs, sind ja indess auf etwas ganz Lächerliches zusammengeschrumpft. Man
               erzählte mir im übrigen, dass in Frankreich\oindex{Frankreich@\textbf{Frankreich}|pw}{ }jetzt die Verträge auf der Basis von Zahlungen in schweizer\oindex{Schweiz@\textbf{Schweiz}|pw} Francs durchaus abgeschlossen werden.\pend
           
\pstart
           Haben Sie vielleicht auch etwas von Nathan\pwindex{Nathan, Nicolas @\textsc{Nathan, Nicolas}, \emph{Übersetzer}|pw}
               gehört? Und von »Casanovas Heimfahrt\pwindex{Schnitzler, Arthur 15. 5. 1862 Wien – 21. 10. 1931 ebd.@\textsc{Schnitzler, Arthur} (15. 5. 1862 Wien – 21. 10. 1931 ebd.), \emph{Schriftsteller, Mediziner}!Casanovas Heimfahrt@\strich\emph{Casanovas Heimfahrt}|pw}«?\pend
           
\pstart
           Herrn Besnard\pwindex{Besnard, Lucien 19.\,1.\,1872 Nonancourt – 1955 Paris@\textsc{Besnard, Lucien} (19.\,1.\,1872 Nonancourt – 1955 Paris), \emph{Schriftsteller}|pw} bitte empfehlen Sie mich
               bestens; er hat mich sehr dringend zum Besuch des Kongresses\eventindex{Paris@\textbf{Paris}!Kongress zur Gründung der Confédération Internationale des Sociétés d'Auteurs et Compositeurs@Kongress zur Gründung der Confédération Internationale des Sociétés d'Auteurs et Compositeurs|pwv}{ }\label{K_L03976-2v}\edtext{aufgefordert}{\lemma{\textnormal{\emph{aufgefordert}}}\Cendnote{\textnormal{Ein Brief von Lucien
                     Besnard\pwindex{Besnard, Lucien 19.\,1.\,1872 Nonancourt – 1955 Paris@\textsc{Besnard, Lucien} (19.\,1.\,1872 Nonancourt – 1955 Paris), \emph{Schriftsteller}|pwk} ist nicht überliefert. Er war einer der Organisatoren des Kongresses zur Gründung der Confédération
                     Internationale des Sociétés d'Auteurs et Compositeurs\eventindex{Paris@\textbf{Paris}!Kongress zur Gründung der Confédération Internationale des Sociétés d'Auteurs et Compositeurs@Kongress zur Gründung der Confédération Internationale des Sociétés d'Auteurs et Compositeurs|pwk}, der vom
                     12. bis zum 16. 1926 in Paris\oindex{Paris@\textbf{Paris}, \emph{Hauptstadt}|pwk} stattfand und auf dem sich verschiedene nationale
                  (Bühnen-)Schriftstellerverbände in einem internationalen \emph{Verband}\orgindex{Confédération Internationale des Sociétés d'Auteurs et Compositeurs@Confédération Internationale des Sociétés d'Auteurs et Compositeurs|pwk} zusammenschlossen.}}}\label{K_L03976-2}, ich habe ihm schon von
               meiner Reise aus \label{K_L03976-3v}\edtext{geschrieben}{\lemma{\textnormal{\emph{geschrieben}}}\Cendnote{\textnormal{nicht überliefert}}}\label{K_L03976-3}, dass es mir
               leider nicht möglich sein wird nach Paris\oindex{Paris@\textbf{Paris}, \emph{Hauptstadt}|pw} zu
               kommen. Auch über den Aufschub der \label{K_L03976-4v}\edtext{»Kassian\pwindex{Schnitzler, Arthur 15. 5. 1862 Wien – 21. 10. 1931 ebd.@\textsc{Schnitzler, Arthur} (15. 5. 1862 Wien – 21. 10. 1931 ebd.), \emph{Schriftsteller, Mediziner}!tapfere Cassian. Puppenspiel in einem Akt@\strich\emph{Der tapfere Cassian. Puppenspiel in einem Akt}|pw}«-Aufführung}{\lemma{\textnormal{\emph{»Kassian«-Aufführung}}}\Cendnote{\textnormal{Das Projekt wurde nicht realisiert.}}}\label{K_L03976-4} hat er mir
               berichtet.\pend
           
\pstart
           Die Uebersetzerin des »Einsamen Weg\pwindex{Schnitzler, Arthur 15. 5. 1862 Wien – 21. 10. 1931 ebd.@\textsc{Schnitzler, Arthur} (15. 5. 1862 Wien – 21. 10. 1931 ebd.), \emph{Schriftsteller, Mediziner}!einsame Weg. Schauspiel in fünf Akten@\strich\emph{Der einsame Weg. Schauspiel in fünf Akten}|pw}\pwindex{Schnitzler, Arthur 15. 5. 1862 Wien – 21. 10. 1931 ebd.@\textsc{Schnitzler, Arthur} (15. 5. 1862 Wien – 21. 10. 1931 ebd.), \emph{Schriftsteller, Mediziner}!?? [französische Übersetzung von Der einsame Weg]@\strich\emph{?? [französische Übersetzung von Der einsame Weg]}|pw}«,
               Frau Bianquis\pwindex{Bianquis, Geneviève 19.\,9.\,1886 Rouen – 24.\,3.\,1972 Antony@\textsc{Bianquis, Geneviève} (19.\,9.\,1886 Rouen – 24.\,3.\,1972 Antony), \emph{Übersetzerin, Literaturhistorikerin}|pw}, hat mir \label{K_L03976-5v}\edtext{ein Buch geschickt}{\lemma{\textnormal{\emph{ein Buch geschickt}}}\Cendnote{\textnormal{Der Brief, mit dem Bianquis\pwindex{Bianquis, Geneviève 19.\,9.\,1886 Rouen – 24.\,3.\,1972 Antony@\textsc{Bianquis, Geneviève} (19.\,9.\,1886 Rouen – 24.\,3.\,1972 Antony), \emph{Übersetzerin, Literaturhistorikerin}|pwk} ihr Buch \emph{La poésie autrichienne
                     de Hofmannsthal à Rilke}\pwindex{Bianquis, Geneviève 19.\,9.\,1886 Rouen – 24.\,3.\,1972 Antony@\textsc{Bianquis, Geneviève} (19.\,9.\,1886 Rouen – 24.\,3.\,1972 Antony), \emph{Übersetzerin, Literaturhistorikerin}!poésie autrichienne de Hofmannsthal à Rilke@\strich\emph{La poésie autrichienne de Hofmannsthal à Rilke}|pwk} übersandte ist nicht überliefert, aber die Antwort
                  darauf: Arthur Schnitzler an Geneviève Bianquis\pwindex{Bianquis, Geneviève 19.\,9.\,1886 Rouen – 24.\,3.\,1972 Antony@\textsc{Bianquis, Geneviève} (19.\,9.\,1886 Rouen – 24.\,3.\,1972 Antony), \emph{Übersetzerin, Literaturhistorikerin}|pwk},
                  27. 5. 1926, \emph{Deutsches Literaturarchiv Marbach},
                     HS.1985.1.00387,5.}}}\label{K_L03976-5}, »La Poésie
                  autrichienne, de Hofmannsthal à Rilke\pwindex{Bianquis, Geneviève 19.\,9.\,1886 Rouen – 24.\,3.\,1972 Antony@\textsc{Bianquis, Geneviève} (19.\,9.\,1886 Rouen – 24.\,3.\,1972 Antony), \emph{Übersetzerin, Literaturhistorikerin}!poésie autrichienne de Hofmannsthal à Rilke@\strich\emph{La poésie autrichienne de Hofmannsthal à Rilke}|pw}«, in dem sie (nach oberflächlicher
               Durchsicht) vielfache, aber etwas fragwürdig geordnete Kenntnisse über die österreichische\oindex{Österreich@\textbf{Österreich}|pw} Literatur verrät; insbesondere
               die Bibliographie ist überraschend zusammengestell{[}t{]}. {\pb}\pend
           
\pstart
           Ich erhielt ferner einen \label{K_L03976-6v}\edtext{Brief}{\lemma{\textnormal{\emph{Brief}}}\Cendnote{\textnormal{Der Brief ist nicht überliefert, jedoch die Antwort: Arthur Schnitzler an Marie Lahy-Hollebecque\pwindex{Lahy-Hollebecque, Marie 7.\,1.\,1881 1. arrondissement [Paris] – 27.\,1.\,1957 14. arrondissement [Paris]@\textsc{Lahy-Hollebecque, Marie} (7.\,1.\,1881 1. arrondissement [Paris] – 27.\,1.\,1957 14. arrondissement [Paris]), \emph{Essayistin}|pwk}, 27. 5. 1926, \emph{Deutsches Literaturarchiv Marbach},
                        HS.1985.1.1255.}}}\label{K_L03976-6} von einer Frau Lahy-Hollebecque\pwindex{Lahy-Hollebecque, Marie 7.\,1.\,1881 1. arrondissement [Paris] – 27.\,1.\,1957 14. arrondissement [Paris]@\textsc{Lahy-Hollebecque, Marie} (7.\,1.\,1881 1. arrondissement [Paris] – 27.\,1.\,1957 14. arrondissement [Paris]), \emph{Essayistin}|pw}, 22,
                  Avenue de l’Observatoire\oindex{22, Avenue de l’Observatoire@\textbf{22, Avenue de l’Observatoire}, \emph{Wohngebäude}|pw}, die sich auf eine ihr im Jahre 1905
               von mir oder einem Vertreter erteilte Autorisation zur Uebersetzung des »Einsamen Wegs\pwindex{Schnitzler, Arthur 15. 5. 1862 Wien – 21. 10. 1931 ebd.@\textsc{Schnitzler, Arthur} (15. 5. 1862 Wien – 21. 10. 1931 ebd.), \emph{Schriftsteller, Mediziner}!einsame Weg. Schauspiel in fünf Akten@\strich\emph{Der einsame Weg. Schauspiel in fünf Akten}|pw}« beruft und angibt, sie hätte nun
               die Möglichkeit \label{K_L03976-7v}\edtext{das Buch\pwindex{Schnitzler, Arthur 15. 5. 1862 Wien – 21. 10. 1931 ebd.@\textsc{Schnitzler, Arthur} (15. 5. 1862 Wien – 21. 10. 1931 ebd.), \emph{Schriftsteller, Mediziner}!?? [französische Übersetzung von Der einsame Weg]@\strich\emph{?? [französische Übersetzung von Der einsame Weg]}|pwv}}{\lemma{\textnormal{\emph{das Buch}}}\Cendnote{\textnormal{Das Projekt wurde nicht realisiert.}}}\label{K_L03976-7} drucken zu
               lassen. Wenn Sie erlauben, verehrte Frau Hofrätin, schreibe ich der Dame, dass sie
               sich mit Ihnen in Verbindung setzen möge. Erinnere ich mich recht, so ist ja die Uebersetzung\pwindex{Schnitzler, Arthur 15. 5. 1862 Wien – 21. 10. 1931 ebd.@\textsc{Schnitzler, Arthur} (15. 5. 1862 Wien – 21. 10. 1931 ebd.), \emph{Schriftsteller, Mediziner}!?? [französische Übersetzung von Der einsame Weg]@\strich\emph{?? [französische Übersetzung von Der einsame Weg]}|pwv} von Frau Bianquis\pwindex{Bianquis, Geneviève 19.\,9.\,1886 Rouen – 24.\,3.\,1972 Antony@\textsc{Bianquis, Geneviève} (19.\,9.\,1886 Rouen – 24.\,3.\,1972 Antony), \emph{Übersetzerin, Literaturhistorikerin}|pw} noch keinem Theater vorgelegen.\pend
           
\pstart
           Herzlichste Grüsse und hoffentlich{\\[\baselineskip]} auf ein sehr baldiges Wiedersehen.{\\[\baselineskip]}
               Ihr getreuer\pend
           \leftskip=0em{}{\vspace{1\baselineskip}}
\pstart
           \noindent{}Frau Hofrätin Bertha Zuckerkandl,{\\}Paris\oindex{Paris@\textbf{Paris}, \emph{Hauptstadt}|pw}.\pend
           \selectlanguage{ngerman}\endnumbering\briefempfaengerindex{Zuckerkandl, Berta@\textsc{Zuckerkandl, Berta}!zzzSchnitzler, Arthur@\emph{von Arthur Schnitzler}!1926-05-271@{27. 5. 1926}|)be}\mylabel{L03976h}
\begin{anhang}
\end{anhang}\newcommand{\dateiname}{L03976}\newcommand{\titel}{Arthur Schnitzler an Berta Zuckerkandl, 27. 5. 1926}\newcommand{\editorInnen}{Herausgegeben von Jahnke, SelmaMüller, Martin Anton}%% latex-leseansicht-abspann.tex
%% Abspann für die Leseansicht.
%% Der Schalter \ifkorrekturansicht ist bereits durch den Vorspann gesetzt.

%% latex-abspann.tex
%% Gemeinsamer Abspann für Korrekturansicht und Leseansicht.
%% Setzt den Schalter \ifkorrekturansicht voraus (gesetzt in den
%% einbindenden Dateien latex-korrekturansicht-abspann.tex bzw.
%% latex-leseansicht-abspann.tex).
%% ---------------------------------------------------------------

\normalsize

% Das esempio-Environment wird nur in der Leseansicht benötigt
\ifkorrekturansicht\else
\newenvironment{esempio}[3]%
{
    \vspace{1.5ex}
    \rlap{\underline{#1}}
    \par
    \setlength{\parindent}{0cm}
    \nopagebreak
    \leftskip=#2cm
    \rightskip=#3cm
}
{
    \par
}
\fi

\doendnotes{C}
\bigskip
\vfill

\clearpage

\footnotesize

\ifkorrekturansicht
  \lohead{\textsc{register}}
\fi

% theindex-Environment neu definieren ohne reledmac
\makeatletter
\renewenvironment{theindex}{%
  \ifkorrekturansicht
    \section*{\indexname}%
  \else
    \subsubsection*{Index der erwähnten Entitäten}%
  \fi
  \setlength{\parindent}{0pt}%
  \setlength{\parskip}{0pt plus 0.3pt}%
  \let\item\@idxitem
}{%
  \ifkorrekturansicht\clearpage\fi
}
\makeatother

\IfFileExists{\jobname-pw.ind}{\input{\jobname-pw.ind}}{}

% Quellenangabe nur in der Leseansicht
\ifkorrekturansicht\else
% Fallback-Definitionen, falls die .tex-Datei \titel etc. nicht gesetzt hat
\providecommand{\titel}{}
\providecommand{\editorInnen}{}
\providecommand{\dateiname}{\jobname}

\vspace{3cm}

\vfill

\footnotesize
\textsc{Quelle}: \titel. Herausgegeben von {\editorInnen}. In: \emph{Arthur Schnitzler: Briefwechsel mit Autorinnen und Autoren}.
 Digitale Edition, https://schnitzler-briefe.acdh.oeaw.ac.at/{\dateiname}.html (Stand \today)
\fi

\end{document}


