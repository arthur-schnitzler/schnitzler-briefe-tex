%% latex-leseansicht-vorspann.tex
%% Vorspann für die Leseansicht.
%% Lädt die gemeinsame Datei latex-vorspann.tex mit nicht gesetztem Schalter.

\newif\ifkorrekturansicht
\korrekturansichtfalse

\input{../tex-inputs/latex-vorspann}


\section[Paul Goldmann an Arthur Schnitzler, 6. 10. [1895]]{L02749 Paul Goldmann an Arthur Schnitzler, 6. 10. [1895]}
\nopagebreak\mylabel{L02749v}
\rehead{ }\normalsize\beginnumbering\briefempfaengerindex{Schnitzler, Arthur@\textsc{Schnitzler, Arthur}!zzzGoldmann, Paul@\emph{von Paul Goldmann}!1895-10-063@{6. 10. [1895]}|(be}
\toendnotes[C]{\smallbreak\pagebreak[2]}
\correspDesc{Versand  durch Paul Goldmann am 6. 10. [1895] in Paris
\newline{}Erhalt  durch Arthur Schnitzler im Zeitraum [7. 10. 1895
                  – 11. 10. 1895?] in Wien}\toendnotes[C]{\smallbreak}
\Standort{DLA, A:Schnitzler, HS.NZ85.1.3165.}
\physDesc{Brief, 1 Blatt, 3 Seiten, 916 Zeichen
\newline{}Handschrift: blaue Tinte, deutsche Kurrent
\newline{}Schnitzler: mit Bleistift das Jahr »95« vermerkt }\toendnotes[C]{\smallbreak}
\pstart
           {\pb}\textcolor{gray}{\textbf{\textbf{Frankfurter Zeitung\orgindex{Frankfurter Zeitung@Frankfurter Zeitung|pw}}}}\pend
           
\pstart
           \textcolor{gray}{\textbf{(\begin{otherlanguage}{french}Gazette de Francfort\end{otherlanguage}\orgindex{Frankfurter Zeitung@Frankfurter Zeitung|pw}).}}\pend
           
\pstart
           \textcolor{gray}{\textbf{\textbf{\begin{otherlanguage}{french}Fondateur M. L.
                              Sonnemann\pwindex{Sonnemann, Leopold 29.\,10.\,1831 Höchberg – 30.\,10.\,1909 Frankfurt am Main@\textsc{Sonnemann, Leopold} (29.\,10.\,1831 Höchberg – 30.\,10.\,1909 Frankfurt am Main), \emph{Journalist, Herausgeber}|pw}\end{otherlanguage}.}}}\pend
           
\pstart
           \begin{otherlanguage}{french}\textcolor{gray}{\textbf{Journal politique, financier,}}\end{otherlanguage}\pend
           
\pstart
           \begin{otherlanguage}{french}\textcolor{gray}{\textbf{commercial et littéraire.}}\end{otherlanguage}\pend
           
\pstart
           \begin{otherlanguage}{french}\textcolor{gray}{\textbf{\textbf{Paraissant trois fois par jour.}}}\end{otherlanguage}\hfill \textsc{Paris\oindex{Paris@\textbf{Paris}, \emph{Hauptstadt}|pw}}, 6. Oktober\textcolor{gray}{.}\pend
           
\pstart
           \begin{otherlanguage}{french}\textcolor{gray}{\textbf{\textbf{Bureau à Paris\oindex{Paris@\textbf{Paris}, \emph{Hauptstadt}|pw}}}}\end{otherlanguage}\pend
           
\pstart
           \begin{otherlanguage}{french}\textcolor{gray}{\textbf{\textbf{24. Rue Feydeau\oindex{rue Feydeau@\textbf{rue Feydeau}, \emph{Straße}|pw}.}}}\end{otherlanguage}\pend
           
\pstart\center{}Mein lieber Freund,\pend\vspace{0.5em}
\pstart
           Morgen{ }ſchreibe ich Dir ausführlicher. Heut hab’ ich alle Hände voll zu thun: \label{K_L02749-1v}\edtext{\textsc{\begin{otherlanguage}{french}Grand Prix d’automne\end{otherlanguage}}}{\lemma{\textnormal{\emph{Grand Prix d’automne}}}\Cendnote{\textnormal{Gemeint ist wohl der \emph{Prix Montgomery}\orgindex{Prix Montgomery@Prix Montgomery|pwk}, ein Hindernisrennen mit Pferden, der
                  zuvor Grand Prix d’automne\orgindex{Prix Montgomery@Prix Montgomery|pwkv}
                  hieß und zwischen 6. und 10. 11. 1895 in Auteil\oindex{Auteuil@\textbf{Auteuil}, \emph{Ehemaliger Ort}|pwk} ausgetragen wurde.}}}\label{K_L02749-1}{ }\textsc{etc}. Einſtweilen will ich Dir nur von Herzen danken für
               Deine treue Berichterſtattung und Dir{ }ſagen, daß \strikeout{ich}
               all’ meine Wünſche mit Dir{ }ſind in dieſen {\pb}ereignißreichen und hoffentlich nicht allzu{ }ſchweren Tagen. Ich habe das Bedürfniß,
               einen Segensſpruch zu thun. Es iſt doch{ }ſchade, daß \strikeout{\textcolor{gray}{n}} wir den alten lieben Gott \textcolor{gray}{ſ}eines \strikeout{Antes} Amtes entſetzt haben. Zum Segnen war er{ }ſo bequem,{ }ſo handtlich. So
               empfehle ich Dich dem Schutze aller guten Mächte. Mit all’ dieſen Wünſchen wird man
               ja freilich {\pb}das Schickſal nicht vom Wege ablenken
               können, das{ }ſeinen Lauf nimmt. Aber ich glaube die Richtung zu{ }ſehen, in der dieſes
               Dein Schickſal geht, und ich glaube zu erkennen,{ }ſo{ }ſicher als ich je etwas erkannt,
               daß es die gute Richtung iſt.\pend
           
\pstart
           Glück, viel, viel, viel Glück, mein theurer Freund!\pend
           
\pstart
           Dein {\\[\baselineskip]}\spacefill\mbox{Paul Goldm\textcolor{gray}{{\geminationn}}}\pend
           \leftskip=0em{}\selectlanguage{ngerman}\endnumbering\briefempfaengerindex{Schnitzler, Arthur@\textsc{Schnitzler, Arthur}!zzzGoldmann, Paul@\emph{von Paul Goldmann}!1895-10-063@{6. 10. [1895]}|)be}\mylabel{L02749h}  \newcommand{\dateiname}{L02749}\newcommand{\titel}{Paul Goldmann an Arthur Schnitzler, 6. 10. [1895]}\newcommand{\editorInnen}{Martin Anton Müller und Laura Untner}%% latex-leseansicht-abspann.tex
%% Abspann für die Leseansicht.
%% Der Schalter \ifkorrekturansicht ist bereits durch den Vorspann gesetzt.

%% latex-abspann.tex
%% Gemeinsamer Abspann für Korrekturansicht und Leseansicht.
%% Setzt den Schalter \ifkorrekturansicht voraus (gesetzt in den
%% einbindenden Dateien latex-korrekturansicht-abspann.tex bzw.
%% latex-leseansicht-abspann.tex).
%% ---------------------------------------------------------------

\normalsize

% Das esempio-Environment wird nur in der Leseansicht benötigt
\ifkorrekturansicht\else
\newenvironment{esempio}[3]%
{
    \vspace{1.5ex}
    \rlap{\underline{#1}}
    \par
    \setlength{\parindent}{0cm}
    \nopagebreak
    \leftskip=#2cm
    \rightskip=#3cm
}
{
    \par
}
\fi

\doendnotes{C}
\bigskip
\vfill

\clearpage

\footnotesize

\ifkorrekturansicht
  \lohead{\textsc{register}}
\fi

% theindex-Environment neu definieren ohne reledmac
\makeatletter
\renewenvironment{theindex}{%
  \ifkorrekturansicht
    \section*{\indexname}%
  \else
    \subsubsection*{Index der erwähnten Entitäten}%
  \fi
  \setlength{\parindent}{0pt}%
  \setlength{\parskip}{0pt plus 0.3pt}%
  \let\item\@idxitem
}{%
  \ifkorrekturansicht\clearpage\fi
}
\makeatother

\IfFileExists{\jobname-pw.ind}{\input{\jobname-pw.ind}}{}

% Quellenangabe nur in der Leseansicht
\ifkorrekturansicht\else
% Fallback-Definitionen, falls die .tex-Datei \titel etc. nicht gesetzt hat
\providecommand{\titel}{}
\providecommand{\editorInnen}{}
\providecommand{\dateiname}{\jobname}

\vspace{3cm}

\vfill

\footnotesize
\textsc{Quelle}: \titel. Herausgegeben von {\editorInnen}. In: \emph{Arthur Schnitzler: Briefwechsel mit Autorinnen und Autoren}.
 Digitale Edition, https://schnitzler-briefe.acdh.oeaw.ac.at/{\dateiname}.html (Stand \today)
\fi

\end{document}


