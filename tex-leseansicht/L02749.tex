%% latex-korrekturansicht-vorspann.tex
%% Vorspann für die Korrekturansicht.
%% Lädt die gemeinsame Datei latex-vorspann.tex mit gesetztem Schalter.

\newif\ifkorrekturansicht
\korrekturansichttrue

\input{../tex-inputs/latex-vorspann}


\section[Paul Goldmann an Arthur Schnitzler, 6. 10. {[}1895{]}]{L02749 Paul Goldmann an Arthur Schnitzler, 6. 10. {[}1895{]}}
\nopagebreak\mylabel{L02749v}
\rehead{ }\normalsize\beginnumbering\briefempfaengerindex{Schnitzler, Arthur@\textsc{Schnitzler, Arthur}!zzzGoldmann, Paul@\emph{von Paul Goldmann}!1895-10-062@{6. 10. {[}1895{]}}|(be}
\toendnotes[C]{\smallbreak\pagebreak[2]}\Standort{DLA, A:Schnitzler, HS.NZ85.1.3165.}
\physDesc{Brief, 1 Blatt, 3 Seiten, 916 Zeichen
\newline{}Handschrift: blaue Tinte, deutsche Kurrent
\newline{}Schnitzler: mit Bleistift das Jahr »95« vermerkt }\toendnotes[C]{\smallbreak}
\pstart
           {\pb}\textcolor{gray}{\textbf{\textbf{Frankfurter Zeitung\orgindex{Frankfurter Zeitung@Frankfurter Zeitung|pw}}}}\pend
           
\pstart
           \textcolor{gray}{\textbf{(\begin{otherlanguage}{french}Gazette de Francfort\end{otherlanguage}\orgindex{Frankfurter Zeitung@Frankfurter Zeitung|pw}). }}\pend
           
\pstart
           \textcolor{gray}{\textbf{\textbf{\begin{otherlanguage}{french}Fondateur M. L.
                              Sonnemann\pwindex{Sonnemann, Leopold 1831-10-29 – 1909-10-30@\textsc{Sonnemann, Leopold} (1831-10-29 – 1909-10-30), \emph{Journalist/Journalistin, Herausgeber/Herausgeberin}|pw}\end{otherlanguage}.}}}\pend
           
\pstart
           \begin{otherlanguage}{french}\textcolor{gray}{\textbf{Journal politique, financier,}}\end{otherlanguage}\pend
           
\pstart
           \begin{otherlanguage}{french}\textcolor{gray}{\textbf{commercial et littéraire.}}\end{otherlanguage}\pend
           
\pstart
           \begin{otherlanguage}{french}\textcolor{gray}{\textbf{\textbf{Paraissant trois fois par jour.}}}\end{otherlanguage}\hfill \textsc{Paris\oindex{Paris@\textbf{Paris}, \emph{P.PPLC}|pw}}, 6. Oktober\textcolor{gray}{.}\pend
           
\pstart
           \begin{otherlanguage}{french}\textcolor{gray}{\textbf{\textbf{Bureau à Paris\oindex{Paris@\textbf{Paris}, \emph{P.PPLC}|pw}}}}\end{otherlanguage}\pend
           
\pstart
           \begin{otherlanguage}{french}\textcolor{gray}{\textbf{\textbf{24. Rue Feydeau\oindex{rue Feydeau@\textbf{rue Feydeau}, \emph{Straße (K.STR)}|pw}.}}}\end{otherlanguage}\pend
           
\pstart\center{}Mein lieber Freund,\pend\vspace{0.5em}
\pstart
           Morgen ſchreibe ich Dir ausführlicher. Heut hab’ ich alle Hände voll zu thun: \label{K_L02749-1v}\edtext{\textsc{\begin{otherlanguage}{french}Grand Prix d’automne\end{otherlanguage}}}{\lemma{\textnormal{\emph{Grand Prix d’automne}}}\Cendnote{\textnormal{Gemeint ist wohl der \emph{Prix Montgomery}\orgindex{Prix Montgomery@Prix Montgomery|pwk}, ein Hindernisrennen mit Pferden, der
                  zuvor Grand Prix d’automne\orgindex{Prix Montgomery@Prix Montgomery|pwkv}
                  hieß und zwischen 6. und 10. 11. 1895 in Auteil\oindex{Auteuil@\textbf{Auteuil}, \emph{P.PPLX}|pwk} ausgetragen wurde.}}}\label{K_L02749-1}{ }\textsc{etc}. Einſtweilen will ich Dir nur von Herzen danken für
               Deine treue Berichterſtattung und Dir ſagen, daß \strikeout{ich}
               all’ meine Wünſche mit Dir ſind in dieſen {\pb}ereignißreichen und hoffentlich nicht allzu ſchweren Tagen. Ich habe das Bedürfniß,
               einen Segensſpruch zu thun. Es iſt doch ſchade, daß \strikeout{\textcolor{gray}{n}} wir den alten lieben Gott \textcolor{gray}{ſ}eines \strikeout{Antes} Amtes entſetzt haben. Zum Segnen war er ſo bequem, ſo handtlich. So
               empfehle ich Dich dem Schutze aller guten Mächte. Mit all’ dieſen Wünſchen wird man
               ja freilich {\pb}das Schickſal nicht vom Wege ablenken
               können, das ſeinen Lauf nimmt. Aber ich glaube die Richtung zu ſehen, in der dieſes
               Dein Schickſal geht, und ich glaube zu erkennen, ſo ſicher als ich je etwas erkannt,
               daß es die gute Richtung iſt.\pend
           
\pstart
           Glück, viel, viel, viel Glück, mein theurer Freund!\pend
           
\pstart
           Dein {\\[\baselineskip]}\spacefill\mbox{Paul Goldm\textcolor{gray}{{\geminationn}}}\pend
           \leftskip=0em{}\selectlanguage{ngerman}\endnumbering\briefempfaengerindex{Schnitzler, Arthur@\textsc{Schnitzler, Arthur}!zzzGoldmann, Paul@\emph{von Paul Goldmann}!1895-10-062@{6. 10. {[}1895{]}}|)be}\mylabel{L02749h}  \normalsize

\doendnotes{C}
\bigskip
\vfill

\clearpage

\footnotesize

\lohead{\textsc{register}}

% Definiere theindex-Environment komplett neu ohne reledmac
\makeatletter
\renewenvironment{theindex}{%
  \section*{\indexname}%
  \setlength{\parindent}{0pt}%
  \setlength{\parskip}{0pt plus 0.3pt}%
  \let\item\@idxitem
}{%
  \clearpage
}
\makeatother

\IfFileExists{\jobname-pw.ind}{\input{\jobname-pw.ind}}{}

\end{document}

      