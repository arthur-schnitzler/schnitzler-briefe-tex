%% latex-leseansicht-vorspann.tex
%% Vorspann für die Leseansicht.
%% Lädt die gemeinsame Datei latex-vorspann.tex mit nicht gesetztem Schalter.

\newif\ifkorrekturansicht
\korrekturansichtfalse

\input{../tex-inputs/latex-vorspann}


         
         \renewcommand{\erwaehntePersonen}{Personen: Georg Brandes, Friedrich Düsel, Betty Herzfeld, Louise Schnitzler}
         \renewcommand{\erwaehnteOrte}{Orte: Berlin, Deutsches Theater Berlin, Dänemark, Grundlsee (Gemeinde), Kőszeg, Wien, Österreich}
         \renewcommand{\erwaehnteWerke}{Werke: Anatol, Liebelei. Schauspiel in drei Akten, Politiken, To Forestillinger af Henrik IV}
               \section[Marie Herzfeld an Arthur Schnitzler, 7. 8. 1896]{ Marie Herzfeld an Arthur Schnitzler, 7. 8. 1896}\nopagebreak\mylabel{v}\rehead{ }\begin{ledgroupsized}[t]{13cm}\normalsize\beginnumbering \toendnotes[C]{\smallbreak\pagebreak[2]} \Standort{DLA, A:Schnitzler, HS.1985.1.03436,1.}
\physDesc{Brief, 1 Blatt, 4 Seiten
\newline{}Handschrift: schwarze Tinte, lateinische Kurrent
\newline{}Schnitzler: 1) mit Bleistift Vermerk »\textsc{Herzfeld}«  2) mit rotem Buntstift »\textsc{(Brand\pwindex{Brandes, Georg 04.02.1842 – 19.02.1927@\textsc{Brandes, Georg} (04.02.1842 – 19.02.1927)|pw}}{[}es{]}«}\toendnotes[C]{\smallbreak}\pstart
           \raggedleft{}{\pb}Grundlsee\oindex{Grundlsee (Gemeinde)@\textbf{Grundlsee (Gemeinde)}|pw}, 7. Aug. 96\pend
           \pstart\center{}Sehr geehrter Herr Doktor!\pend\pstart
           Im dänischen\oindex{Daenemark@\textbf{Dänemark}|pw} Blatt »Politiken\pwindex{?? Werk@Nicht ermittelte Verfasserinnen und Verfasser!Politiken1. 1. 1884@\emph{Politiken} {[}1. 1. 1884{]}|pw}« v. 5. Aug. steht ein \label{K_L02590-1v}\edtext{Artikel\pwindex{Brandes, Georg 04.02.1842 – 19.02.1927@\textsc{Brandes, Georg} (04.02.1842 – 19.02.1927)!To Forestillinger af Henrik IV5. 8. 1896@\strich\emph{To Forestillinger af Henrik IV} {[}5. 8. 1896{]}|pwv} von Georg Brandes\pwindex{Brandes, Georg 04.02.1842 – 19.02.1927@\textsc{Brandes, Georg} (04.02.1842 – 19.02.1927)|pw} »Zwei Vorstellungen
                  Heinrich IV\pwindex{Brandes, Georg 04.02.1842 – 19.02.1927@\textsc{Brandes, Georg} (04.02.1842 – 19.02.1927)!To Forestillinger af Henrik IV5. 8. 1896@\strich\emph{To Forestillinger af Henrik IV} {[}5. 8. 1896{]}|pw}«}{\lemma{\textnormal{\emph{Artikel … Heinrich IV«}}}\Cendnote{\textnormal{G. B.
                        [=Georg Brandes]\pwindex{Duesel, Friedrich 11.02.1869 – 08.12.1945@\textsc{Düsel, Friedrich} (11.02.1869 – 08.12.1945), \emph{Kritiker}|pwk}: \emph{To Forestillinger af
                        Henrik IV}\pwindex{Brandes, Georg 04.02.1842 – 19.02.1927@\textsc{Brandes, Georg} (04.02.1842 – 19.02.1927)!To Forestillinger af Henrik IV5. 8. 1896@\strich\emph{To Forestillinger af Henrik IV} {[}5. 8. 1896{]}|pwk}. In: \emph{Politiken}\pwindex{?? Werk@Nicht ermittelte Verfasserinnen und Verfasser!Politiken1. 1. 1884@\emph{Politiken} {[}1. 1. 1884{]}|pwk},
                        5. 8. 1896, S. 1–2.}}}\label{K_L02590-1h}, in welchem
               folgende Stelle sich findet: »\label{K_L02590-2v}\edtext{Unter
               den Stücken, die ich da (›Deutsches Theater\oindex{Deutsches Theater Berlin@\textbf{Deutsches Theater Berlin}|pw}‹ in Berlin\oindex{Berlin@\textbf{Berlin}|pw}) mit vollendeter Kunst dargestellt sah, nenne
               ich das bewunderungswürdige östreichische\oindex{Oesterreich@\textbf{Österreich}|pw}
               Trauerspiel ›Liebelei\pwindex{Schnitzler, Arthur 15.05.1862 – 21.10.1931@\textsc{Schnitzler, Arthur} (15.05.1862 – 21.10.1931), \emph{Schriftsteller, Mediziner}!Liebelei. Schauspiel in drei Akten1895-10-09@\strich\emph{Liebelei. Schauspiel in drei Akten} {[}1895-10-09{]}|pw}‹ von Arthur Schnitzler\pwindex{Schnitzler, Arthur 15.05.1862 – 21.10.1931@\textsc{Schnitzler, Arthur} (15.05.1862 – 21.10.1931), \emph{Schriftsteller, Mediziner}|pw}, \strikeout{unter}
               demjenigen \strikeout{und}{ }{\pb}unter \substVorne{}\textsuperscript{den}\substDazwischen{}allen\substHinten{}{ }östr.\oindex{Oesterreich@\textbf{Österreich}|pw}
               Dichtern, dessen Talent am eigentümlichsten und sichersten ist.«}{\lemma{\textnormal{\emph{Unter … ist.«}}}\Cendnote{\textnormal{siehe A. S.: \emph{Tagebuch}, 18. 8. 1896}}}\label{K_L02590-2h} Ich weiß, dass dieser Ausspruch,
               den ich lieber genau als elegant zu übersetzen bemüht war, Ihnen Freude machen wird;
               denn man mag von Brandes\pwindex{Brandes, Georg 04.02.1842 – 19.02.1927@\textsc{Brandes, Georg} (04.02.1842 – 19.02.1927)|pw} denken, wie man will –
               ich gehöre nur \uline{sehr} bedingt zu seinen Bewunderern, –
               er ist ein geistvoller Mensch mit sehr sicherem Instinkt für das, was durchdringen
               wird, u. er hat eine so umfassende Kenntnis der modernen Erscheinungen, dass von ihm
                  be{\pb}merkt und »bewundert« zu werden etwas Auszeichnendes
               hat. Nach diesem kann es Ihnen wol höchstens als anmaßend scheinen, wenn ich Ihnen
               meine Eindrücke von Ihrem Stück\pwindex{Schnitzler, Arthur 15.05.1862 – 21.10.1931@\textsc{Schnitzler, Arthur} (15.05.1862 – 21.10.1931), \emph{Schriftsteller, Mediziner}!Liebelei. Schauspiel in drei Akten1895-10-09@\strich\emph{Liebelei. Schauspiel in drei Akten} {[}1895-10-09{]}|pwv},
               das ich – durch ein \label{K_L02590-3v}\edtext{Trauerjahr}{\lemma{\textnormal{\emph{Trauerjahr}}}\Cendnote{\textnormal{Am 2. 11. 1894 starb
                  ihre Mutter Betty Herzfeld\pwindex{Herzfeld, Betty 1834-05-20 – 1894-11-02@\textsc{Herzfeld, Betty} (1834-05-20 – 1894-11-02)|pwk}, die wie Schnitzler\pwindex{Schnitzler, Arthur 15.05.1862 – 21.10.1931@\textsc{Schnitzler, Arthur} (15.05.1862 – 21.10.1931), \emph{Schriftsteller, Mediziner}|pwk}s Mutter\pwindex{Schnitzler, Louise 1840-07-08 – 1911-09-09@\textsc{Schnitzler, Louise} (1840-07-08 – 1911-09-09)|pwkv} in Kőszeg\oindex{Kőszeg@\textbf{Kőszeg}|pwk} geboren
                  war.}}}\label{K_L02590-3h} und eine vielmonatliche Krankenpflege auch noch diesen Winter
               verhindert – erst im Mai{ }\introOben{}od Juni\introOben{} vor unserer Abreise sah, eingehend schildere.\pend
           \pstart
           Ich will nicht behaupten, dass es im Ganzen über Ihren Anatol\pwindex{Schnitzler, Arthur 15.05.1862 – 21.10.1931@\textsc{Schnitzler, Arthur} (15.05.1862 – 21.10.1931), \emph{Schriftsteller, Mediziner}!Anatol1892-10-29@\strich\emph{Anatol} {[}1892-10-29{]}|pw} Scenen steht; damit bewundere ich aber nur Anatol\pwindex{Schnitzler, Arthur 15.05.1862 – 21.10.1931@\textsc{Schnitzler, Arthur} (15.05.1862 – 21.10.1931), \emph{Schriftsteller, Mediziner}!Anatol1892-10-29@\strich\emph{Anatol} {[}1892-10-29{]}|pw}. Gewiss sind Sie mit dieser Arbeit in {\pb}die erste Linie deutscher Bühnenschriftsteller gerückt –
               obwol Ihr Talent darin noch novellistisch \strikeout{arbeitet}
               gestaltet, bei allem Gefühl für das Theatralische in besserem Sinn. Ich habe mir Ihre
                  \label{K_L02590-4v}\edtext{Erzälungen}{\lemma{\textnormal{\emph{Erzälungen}}}\Cendnote{\textnormal{keine klare Bezugnahme, die erste Zusammenstellung von
                  Prosatexten in Buchform erschien erst 1898}}}\label{K_L02590-4h}{ }hieher\oindex{Grundlsee (Gemeinde)@\textbf{Grundlsee (Gemeinde)}|pwv}
               mitgenommen und hoffe sie hier\oindex{Grundlsee (Gemeinde)@\textbf{Grundlsee (Gemeinde)}|pwv}
               in ein paar ruhigen Stunden zu lesen.\pend
           \pstart
           Mit besten Wünschen für Ihre Arbeiten, {\\[\baselineskip]}\spacefill\mbox{Marie
               Herzfeld}\pend
           \leftskip=0em{}
         
         \endnumbering\mylabel{h}\end{ledgroupsized}  \newcommand{\dateiname}{L02590}\newcommand{\titel}{Marie Herzfeld an Arthur Schnitzler, 7. 8. 1896}\newcommand{\editorInnen}{Martin Anton Müller und Laura Untner}%% latex-leseansicht-abspann.tex
%% Abspann für die Leseansicht.
%% Der Schalter \ifkorrekturansicht ist bereits durch den Vorspann gesetzt.

%% latex-abspann.tex
%% Gemeinsamer Abspann für Korrekturansicht und Leseansicht.
%% Setzt den Schalter \ifkorrekturansicht voraus (gesetzt in den
%% einbindenden Dateien latex-korrekturansicht-abspann.tex bzw.
%% latex-leseansicht-abspann.tex).
%% ---------------------------------------------------------------

\normalsize

% Das esempio-Environment wird nur in der Leseansicht benötigt
\ifkorrekturansicht\else
\newenvironment{esempio}[3]%
{
    \vspace{1.5ex}
    \rlap{\underline{#1}}
    \par
    \setlength{\parindent}{0cm}
    \nopagebreak
    \leftskip=#2cm
    \rightskip=#3cm
}
{
    \par
}
\fi

\doendnotes{C}
\bigskip
\vfill

\clearpage

\footnotesize

\ifkorrekturansicht
  \lohead{\textsc{register}}
\fi

% theindex-Environment neu definieren ohne reledmac
\makeatletter
\renewenvironment{theindex}{%
  \ifkorrekturansicht
    \section*{\indexname}%
  \else
    \subsubsection*{Index der erwähnten Entitäten}%
  \fi
  \setlength{\parindent}{0pt}%
  \setlength{\parskip}{0pt plus 0.3pt}%
  \let\item\@idxitem
}{%
  \ifkorrekturansicht\clearpage\fi
}
\makeatother

\IfFileExists{\jobname-pw.ind}{\input{\jobname-pw.ind}}{}

% Quellenangabe nur in der Leseansicht
\ifkorrekturansicht\else
% Fallback-Definitionen, falls die .tex-Datei \titel etc. nicht gesetzt hat
\providecommand{\titel}{}
\providecommand{\editorInnen}{}
\providecommand{\dateiname}{\jobname}

\vspace{3cm}

\vfill

\footnotesize
\textsc{Quelle}: \titel. Herausgegeben von {\editorInnen}. In: \emph{Arthur Schnitzler: Briefwechsel mit Autorinnen und Autoren}.
 Digitale Edition, https://schnitzler-briefe.acdh.oeaw.ac.at/{\dateiname}.html (Stand \today)
\fi

\end{document}


      