%% latex-leseansicht-vorspann.tex
%% Vorspann für die Leseansicht.
%% Lädt die gemeinsame Datei latex-vorspann.tex mit nicht gesetztem Schalter.

\newif\ifkorrekturansicht
\korrekturansichtfalse

\input{../tex-inputs/latex-vorspann}


\section[Marie Herzfeld an Arthur Schnitzler, 7.\,8.\,1896]{L02590 Marie Herzfeld an Arthur Schnitzler, 7.\,8.\,1896}
\nopagebreak\mylabel{L02590v}
\rehead{ }\normalsize\beginnumbering\briefempfaengerindex{Schnitzler, Arthur@\textsc{Schnitzler, Arthur}!zzzHerzfeld, Marie@\emph{von Marie Herzfeld}!1896-08-073@{7.\,8.\,1896}|(be}
\toendnotes[C]{\smallbreak\pagebreak[2]}
\correspDesc{Versand  durch Marie Herzfeld am 7. 8. 1896 in Grundlsee
\newline{}Erhalt  durch Arthur Schnitzler im Zeitraum [8. 8. 1896
                  – 12. 8. 1896?] in Wien}\toendnotes[C]{\smallbreak}
\Standort{DLA, A:Schnitzler, HS.1985.1.03436,1.}
\physDesc{Brief, 1 Blatt, 4 Seiten, 1674 Zeichen
\newline{}Handschrift: schwarze Tinte, lateinische Kurrent
\newline{}Schnitzler: 1) mit Bleistift Vermerk »\textsc{Herzfeld}«  2) mit rotem Buntstift »\textsc{(Brand\pwindex{Brandes, Georg 4.\,2.\,1842 Kopenhagen – 19.\,2.\,1927 ebd.@\textsc{Brandes, Georg} (4.\,2.\,1842 Kopenhagen – 19.\,2.\,1927 ebd.)|pw}}{[}es{]}«}\toendnotes[C]{\smallbreak}
\pstart
           \raggedleft{}{\pb}Grundlsee\oindex{Grundlsee [Gemeinde]@\textbf{Grundlsee [Gemeinde]}|pw}, 7. Aug. 96\pend
           
\pstart\center{}Sehr geehrter Herr Doktor!\pend\vspace{0.5em}
\pstart
           Im dänischen\oindex{Dänemark@\textbf{Dänemark}|pw} Blatt »Politiken\pwindex{Politiken@\emph{Politiken}|pw}« v. 5. Aug. steht
               ein \label{K_L02590-1v}\edtext{Artikel\pwindex{Brandes, Georg 4.\,2.\,1842 Kopenhagen – 19.\,2.\,1927 ebd.@\textsc{Brandes, Georg} (4.\,2.\,1842 Kopenhagen – 19.\,2.\,1927 ebd.)!To Forestillinger af Henrik IV@\strich\emph{To Forestillinger af Henrik IV}|pwv} von Georg Brandes\pwindex{Brandes, Georg 4.\,2.\,1842 Kopenhagen – 19.\,2.\,1927 ebd.@\textsc{Brandes, Georg} (4.\,2.\,1842 Kopenhagen – 19.\,2.\,1927 ebd.)|pw} »Zwei
                  Vorstellungen Heinrich IV\pwindex{Brandes, Georg 4.\,2.\,1842 Kopenhagen – 19.\,2.\,1927 ebd.@\textsc{Brandes, Georg} (4.\,2.\,1842 Kopenhagen – 19.\,2.\,1927 ebd.)!To Forestillinger af Henrik IV@\strich\emph{To Forestillinger af Henrik IV}|pw}«}{\lemma{\textnormal{\emph{Artikel … Heinrich IV«}}}\Cendnote{\textnormal{G. B. [ = Georg Brandes]\pwindex{Düsel, Friedrich 11.\,2.\,1869 Strelitz Alt – 8.\,12.\,1945 Berlin@\textsc{Düsel, Friedrich} (11.\,2.\,1869 Strelitz Alt – 8.\,12.\,1945 Berlin), \emph{Kritiker}|pwk}: \emph{To Forestillinger af Henrik IV}\pwindex{Brandes, Georg 4.\,2.\,1842 Kopenhagen – 19.\,2.\,1927 ebd.@\textsc{Brandes, Georg} (4.\,2.\,1842 Kopenhagen – 19.\,2.\,1927 ebd.)!To Forestillinger af Henrik IV@\strich\emph{To Forestillinger af Henrik IV}|pwk}. In: \emph{Politiken}\pwindex{Politiken@\emph{Politiken}|pwk}, 5. 8. 1896,
                  S. 1–2.}}}\label{K_L02590-1}, in welchem folgende Stelle sich findet: »\label{K_L02590-2v}\edtext{Unter den Stücken, die ich da (›Deutsches Theater\oindex{Deutsches Theater Berlin@\textbf{Deutsches Theater Berlin}, \emph{Theater}|pw}‹ in Berlin\oindex{Berlin@\textbf{Berlin}, \emph{Hauptstadt}|pw}) mit vollendeter Kunst dargestellt sah, nenne ich das
               bewunderungswürdige östreichische\oindex{Österreich@\textbf{Österreich}|pw} Trauerspiel
                  ›Liebelei\pwindex{Schnitzler, Arthur 15.\,5.\,1862 Wien – 21.\,10.\,1931 ebd.@\textsc{Schnitzler, Arthur} (15.\,5.\,1862 Wien – 21.\,10.\,1931 ebd.), \emph{Schriftsteller, Mediziner}!Liebelei. Schauspiel in drei Akten@\strich\emph{Liebelei. Schauspiel in drei Akten}|pw}‹ von Arthur Schnitzler, \strikeout{unter}
               demjenigen \strikeout{und}{ }{\pb}unter \substVorne{}\textsuperscript{den}\substDazwischen{}allen\substHinten{}{ }östr.\oindex{Österreich@\textbf{Österreich}|pw} Dichtern, dessen Talent am
               eigentümlichsten und sichersten ist.«}{\lemma{\textnormal{\emph{Unter … ist.«}}}\Cendnote{\textnormal{Siehe A. S.: \emph{Tagebuch}, 18. 8. 1896.
               }}}\label{K_L02590-2} Ich weiß, dass dieser Ausspruch, den ich lieber genau als elegant zu
               übersetzen bemüht war, Ihnen Freude machen wird; denn man mag von Brandes\pwindex{Brandes, Georg 4.\,2.\,1842 Kopenhagen – 19.\,2.\,1927 ebd.@\textsc{Brandes, Georg} (4.\,2.\,1842 Kopenhagen – 19.\,2.\,1927 ebd.)|pw} denken, wie man will – ich gehöre nur \uline{sehr} bedingt zu seinen Bewunderern, – er ist ein
               geistvoller Mensch mit sehr sicherem Instinkt für das, was durchdringen wird, u. er
               hat eine so umfassende Kenntnis der modernen Erscheinungen, dass von ihm be{\pb}merkt und »bewundert« zu werden etwas Auszeichnendes hat.
               Nach diesem kann es Ihnen wol höchstens als anmaßend scheinen, wenn ich Ihnen meine
               Eindrücke von Ihrem Stück\pwindex{Schnitzler, Arthur 15.\,5.\,1862 Wien – 21.\,10.\,1931 ebd.@\textsc{Schnitzler, Arthur} (15.\,5.\,1862 Wien – 21.\,10.\,1931 ebd.), \emph{Schriftsteller, Mediziner}!Liebelei. Schauspiel in drei Akten@\strich\emph{Liebelei. Schauspiel in drei Akten}|pwv}, das
               ich – durch ein \label{K_L02590-3v}\edtext{Trauerjahr}{\lemma{\textnormal{\emph{Trauerjahr}}}\Cendnote{\textnormal{Am 2. 11. 1894 war ihre
                  Mutter Betty Herzfeld\pwindex{Herzfeld, Betty 20.\,5.\,1834 Kőszeg – 2.\,11.\,1894 Prag@\textsc{Herzfeld, Betty} (20.\,5.\,1834 Kőszeg – 2.\,11.\,1894 Prag)|pwk} gestorben, die wie Schnitzlers{ }Mutter\pwindex{Schnitzler, Louise 8.\,7.\,1840 Kőszeg – 9.\,9.\,1911 Wien@\textsc{Schnitzler, Louise} (8.\,7.\,1840 Kőszeg – 9.\,9.\,1911 Wien)|pwkv} aus Kőszeg\oindex{Kőszeg@\textbf{Kőszeg}, \emph{Hauptstadt}|pwk} stammte.}}}\label{K_L02590-3} und eine vielmonatliche
               Krankenpflege auch noch diesen Winter verhindert – erst im Mai{ }\introOben{}od Juni\introOben{} vor unserer Abreise sah, eingehend schildere.\pend
           
\pstart
           Ich will nicht behaupten, dass es im Ganzen über Ihren Anatol\pwindex{Schnitzler, Arthur 15.\,5.\,1862 Wien – 21.\,10.\,1931 ebd.@\textsc{Schnitzler, Arthur} (15.\,5.\,1862 Wien – 21.\,10.\,1931 ebd.), \emph{Schriftsteller, Mediziner}!Anatol@\strich\emph{Anatol}|pw} Scenen steht; damit bewundere ich aber nur Anatol\pwindex{Schnitzler, Arthur 15.\,5.\,1862 Wien – 21.\,10.\,1931 ebd.@\textsc{Schnitzler, Arthur} (15.\,5.\,1862 Wien – 21.\,10.\,1931 ebd.), \emph{Schriftsteller, Mediziner}!Anatol@\strich\emph{Anatol}|pw}. Gewiss sind Sie mit dieser Arbeit in {\pb}die erste Linie deutscher Bühnenschriftsteller gerückt –
               obwol Ihr Talent darin noch novellistisch \strikeout{arbeitet}
               gestaltet, bei allem Gefühl für das Theatralische in besserem Sinn. Ich habe mir Ihre
                  \label{K_L02590-4v}\edtext{Erzälungen}{\lemma{\textnormal{\emph{Erzälungen}}}\Cendnote{\textnormal{Hier wird nicht auf bestimmte Texte Bezug genommen. Die erste Zusammenstellung von
                  Prosatexten in Buchform erschien erst 1898.
               }}}\label{K_L02590-4}{ }hieher\oindex{Grundlsee [Gemeinde]@\textbf{Grundlsee [Gemeinde]}|pwv} mitgenommen und hoffe
               sie hier\oindex{Grundlsee [Gemeinde]@\textbf{Grundlsee [Gemeinde]}|pwv} in ein paar ruhigen
               Stunden zu lesen.\pend
           
\pstart
           Mit besten Wünschen für Ihre Arbeiten, {\\[\baselineskip]}\spacefill\mbox{Marie Herzfeld}\pend
           \leftskip=0em{}\selectlanguage{ngerman}\endnumbering\briefempfaengerindex{Schnitzler, Arthur@\textsc{Schnitzler, Arthur}!zzzHerzfeld, Marie@\emph{von Marie Herzfeld}!1896-08-073@{7.\,8.\,1896}|)be}\mylabel{L02590h}  \newcommand{\dateiname}{L02590}\newcommand{\titel}{Marie Herzfeld an Arthur Schnitzler, 7. 8. 1896}\newcommand{\editorInnen}{Martin Anton Müller und Laura Untner}%% latex-leseansicht-abspann.tex
%% Abspann für die Leseansicht.
%% Der Schalter \ifkorrekturansicht ist bereits durch den Vorspann gesetzt.

%% latex-abspann.tex
%% Gemeinsamer Abspann für Korrekturansicht und Leseansicht.
%% Setzt den Schalter \ifkorrekturansicht voraus (gesetzt in den
%% einbindenden Dateien latex-korrekturansicht-abspann.tex bzw.
%% latex-leseansicht-abspann.tex).
%% ---------------------------------------------------------------

\normalsize

% Das esempio-Environment wird nur in der Leseansicht benötigt
\ifkorrekturansicht\else
\newenvironment{esempio}[3]%
{
    \vspace{1.5ex}
    \rlap{\underline{#1}}
    \par
    \setlength{\parindent}{0cm}
    \nopagebreak
    \leftskip=#2cm
    \rightskip=#3cm
}
{
    \par
}
\fi

\doendnotes{C}
\bigskip
\vfill

\clearpage

\footnotesize

\ifkorrekturansicht
  \lohead{\textsc{register}}
\fi

% theindex-Environment neu definieren ohne reledmac
\makeatletter
\renewenvironment{theindex}{%
  \ifkorrekturansicht
    \section*{\indexname}%
  \else
    \subsubsection*{Index der erwähnten Entitäten}%
  \fi
  \setlength{\parindent}{0pt}%
  \setlength{\parskip}{0pt plus 0.3pt}%
  \let\item\@idxitem
}{%
  \ifkorrekturansicht\clearpage\fi
}
\makeatother

\IfFileExists{\jobname-pw.ind}{\input{\jobname-pw.ind}}{}

% Quellenangabe nur in der Leseansicht
\ifkorrekturansicht\else
% Fallback-Definitionen, falls die .tex-Datei \titel etc. nicht gesetzt hat
\providecommand{\titel}{}
\providecommand{\editorInnen}{}
\providecommand{\dateiname}{\jobname}

\vspace{3cm}

\vfill

\footnotesize
\textsc{Quelle}: \titel. Herausgegeben von {\editorInnen}. In: \emph{Arthur Schnitzler: Briefwechsel mit Autorinnen und Autoren}.
 Digitale Edition, https://schnitzler-briefe.acdh.oeaw.ac.at/{\dateiname}.html (Stand \today)
\fi

\end{document}


