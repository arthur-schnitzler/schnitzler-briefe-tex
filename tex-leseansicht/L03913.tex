%% latex-leseansicht-vorspann.tex
%% Vorspann für die Leseansicht.
%% Lädt die gemeinsame Datei latex-vorspann.tex mit nicht gesetztem Schalter.

\newif\ifkorrekturansicht
\korrekturansichtfalse

\input{../tex-inputs/latex-vorspann}


\section[Arthur Schnitzler an Theodor Herzl, 19. 12. 1894]{L03913 Arthur Schnitzler an Theodor Herzl, 19. 12. 1894}
\nopagebreak\mylabel{L03913v}
\rehead{ }\normalsize\beginnumbering\briefempfaengerindex{Herzl, Theodor@\textsc{Herzl, Theodor}!zzzSchnitzler, Arthur@\emph{von Arthur Schnitzler}!1894-12-191@{19. 12. 1894}|(be}
\toendnotes[C]{\smallbreak\pagebreak[2]}
\correspDesc{Versand  durch Arthur Schnitzler am 19. 12. 1894 in Wien
\newline{}Erhalt  durch Theodor Herzl in Wien}\toendnotes[C]{\smallbreak}
\Standort{Jerusalem, Central Zionist Archives, H1:1924-18.}
\physDesc{,  Blätter,  Seiten
\newline{}Handschrift: , deutsche Kurrent}
\buchAbdrucke{\weitereDrucke{Arthur Schnitzler: \emph{Briefe 1875–1912}. Herausgegeben von Therese Nickl und Heinrich Schnitzler. Frankfurt am Main: \emph{S. Fischer} 1981, S. 244–245.} }\toendnotes[C]{\smallbreak}
\pstart{}{\pb}Lieber Freund!\pend\vspace{0.5em}
\pstart
           ich hätte Ihnen auch über die Maſchinenfrage ſchon neulich berichtet, we{\geminationn} ich ſelbſt{ }ſicher geweſen wäre. Von den mir zu Gebote{ }ſtehenden Schreibern kann keiner mit der
               Maſchine umgehen u. zum lernen war keiner bereit zu finden. Es wäre mir nur ein
               Annonciren in der Ztg übrig geblieben. Ich denke aber, daſs ein guter Schreiben
               dieſelben Dienſte thut. Jedenfalls nehme ich einen andern, als meinen gewöhnlichen –
               da es auf die Spur \strikeout{find} leiten könnte, we{\geminationn} plötzlich
               {\pb}am Dtſch. Th.\orgindex{Deutsches Theater Berlin@Deutsches Theater Berlin|pw} zwei von der
               gleichen Hand geſchriebenen \textsc{Mscr.} einlangten.– Noch eine Frage: ſoll ich nicht, ſtatt einem Buchbinder we{\geminationn} auch
               nur auf wenige Stunden das \textsc{Mscr.}\pwindex{Herzl, Theodor 2.\,5.\,1860 Budapest – 3.\,7.\,1904 Edlach@\textsc{Herzl, Theodor} (2.\,5.\,1860 Budapest – 3.\,7.\,1904 Edlach), \emph{Schriftsteller, Journalist}!neue Ghetto. Schauspiel in vier Acten@\strich\emph{Das neue Ghetto. Schauspiel in vier Acten}|pwv} zu überlaſſen, lieber ein ſehr{ }ſchönes Lederheft, d. h. ein in
               gutes Leder gebundenes Heft anſchaffen?–\pend
           
\pstart
           Zum Capitel: ſympath. Figuren.– Nicht aus theatral. Gründen hab ich sie gewünſcht,{ }ſondern eben aus Gründen der Wahrheit. Es \uline{gibt} ſympathiſchere Figuren, ſelbſt in den von dem Autor geſchilderten \label{K_L03913-1v}\edtext{Kreiſen}{\lemma{\textnormal{\emph{Kreisen}}}\Cendnote{\textnormal{}}}\label{K_L03913-1}. Und, denken Sie ſich doch: der ſtarke und tönende {\pb}Titel des Drama\pwindex{Herzl, Theodor 2.\,5.\,1860 Budapest – 3.\,7.\,1904 Edlach@\textsc{Herzl, Theodor} (2.\,5.\,1860 Budapest – 3.\,7.\,1904 Edlach), \emph{Schriftsteller, Journalist}!neue Ghetto. Schauspiel in vier Acten@\strich\emph{Das neue Ghetto. Schauspiel in vier Acten}|pwv}’s, der erwarten läßt, es werde alles darin zu finden{ }ſein, das eben da
               hineingehört. Nun – alles, das iſt ſelbſtverſtändlich dramatiſch unmöglich; aber die
               Beleuchtung müßte die vollig richtige ſein. Und da hab ich nun einmal der Eindruck:
               zu trüb. Ich ſagte ja auch ſchon, wen ich mir noch in das St.\pwindex{Herzl, Theodor 2.\,5.\,1860 Budapest – 3.\,7.\,1904 Edlach@\textsc{Herzl, Theodor} (2.\,5.\,1860 Budapest – 3.\,7.\,1904 Edlach), \emph{Schriftsteller, Journalist}!neue Ghetto. Schauspiel in vier Acten@\strich\emph{Das neue Ghetto. Schauspiel in vier Acten}|pwv} hereingewünſcht hätte; – u. daſs ich nie geſchrieben,
               man ſolle mir »lauter wundervolle Menschen zeigen, eri{\geminationn}ere ich mich ganz genau.–
               Vielleicht haben Sie den Studenten hineingebracht? Sie erwähnen nichts drüber. –
               – Das Maſſengrab, in welches die \textsc{Glosse\pwindex{Herzl, Theodor 2.\,5.\,1860 Budapest – 3.\,7.\,1904 Edlach@\textsc{Herzl, Theodor} (2.\,5.\,1860 Budapest – 3.\,7.\,1904 Edlach), \emph{Schriftsteller, Journalist}!Glosse. Lustspiel in einem Act@\strich\emph{Die Glosse. Lustspiel in einem Act}|pw}}{ }{\pb}verſenkt wurde, iſt hoffentlich nicht endgiltg
               zugeſchüttet. Eine Aufführung müſſte meiner Anſicht nach nicht nur
               »zurichten« ſondern mit »Gründen der Aeschetik« und Vernunft durchzuſetzen ſein.
               Haben Sie es bei \textsc{Brahm\pwindex{Brahm, Otto 5.\,2.\,1856 Hamburg – 28.\,11.\,1912 Berlin@\textsc{Brahm, Otto} (5.\,2.\,1856 Hamburg – 28.\,11.\,1912 Berlin), \emph{Theaterleiter, Regisseur}|pw}}{ }Berlin\oindex{Berlin@\textbf{Berlin}, \emph{Hauptstadt}|pw}, bei \textsc{Loewe\pwindex{Loewe, Theodor 1.\,1.\,1855 Wien – 1935 Breslau@\textsc{Loewe, Theodor} (1.\,1.\,1855 Wien – 1935 Breslau), \emph{Theaterleiter}|pw}}{ }Breslau\oindex{Breslau@\textbf{Breslau}|pw} eingereicht?–\pend
           
\pstart
           Den \textsc{Tabarin\pwindex{Herzl, Theodor 2.\,5.\,1860 Budapest – 3.\,7.\,1904 Edlach@\textsc{Herzl, Theodor} (2.\,5.\,1860 Budapest – 3.\,7.\,1904 Edlach), \emph{Schriftsteller, Journalist}!Tabarin. Schauspiel in einem Act. Frei nach Catulle Mendès@\strich\emph{Tabarin. Schauspiel in einem Act. Frei nach Catulle Mendès}|pw}} werden wir ja jetzt bald zu ſehen bekommen; ich bin ſehr begierig. Ich hab eine{ }ſo ſchöne Erinnerung an das Stück\pwindex{Herzl, Theodor 2.\,5.\,1860 Budapest – 3.\,7.\,1904 Edlach@\textsc{Herzl, Theodor} (2.\,5.\,1860 Budapest – 3.\,7.\,1904 Edlach), \emph{Schriftsteller, Journalist}!Tabarin. Schauspiel in einem Act. Frei nach Catulle Mendès@\strich\emph{Tabarin. Schauspiel in einem Act. Frei nach Catulle Mendès}|pwv}.
               Ko{\geminationm}en Sie vielleicht nach Wien\oindex{Wien@\textbf{Wien}, \emph{Verwaltungsgebiet}|pw}? Wegen \uline{eines}{ }Aktes\pwindex{Herzl, Theodor 2.\,5.\,1860 Budapest – 3.\,7.\,1904 Edlach@\textsc{Herzl, Theodor} (2.\,5.\,1860 Budapest – 3.\,7.\,1904 Edlach), \emph{Schriftsteller, Journalist}!Tabarin. Schauspiel in einem Act. Frei nach Catulle Mendès@\strich\emph{Tabarin. Schauspiel in einem Act. Frei nach Catulle Mendès}|pwv} – das iſt Ihnen wohl zu wenig! Ich möchte mit
               einem Akt im Burgtheater\oindex{Wien@\textbf{Wien}!I., Innere Stadt@\textbf{I., Innere Stadt}!Burgtheater@\textbf{Burgtheater}, \emph{Theater}|pw} ſtehn!\pend
           
\pstart
           Seien Sie vielmals herzlich gegrüßt{\\[\baselineskip]}Ihr treu ergebener \spacefill\mbox{Arthur
                  Schnitzler}\pend
           \leftskip=0em{}
\pstart
           19. 12. 94.\pend
           \selectlanguage{ngerman}\endnumbering\briefempfaengerindex{Herzl, Theodor@\textsc{Herzl, Theodor}!zzzSchnitzler, Arthur@\emph{von Arthur Schnitzler}!1894-12-191@{19. 12. 1894}|)be}\mylabel{L03913h}
\begin{anhang}
\end{anhang}\newcommand{\dateiname}{L03913}\newcommand{\titel}{Arthur Schnitzler an Theodor Herzl, 19. 12. 1894}\newcommand{\editorInnen}{Herausgegeben von Jahnke, SelmaMüller, Martin Anton}%% latex-leseansicht-abspann.tex
%% Abspann für die Leseansicht.
%% Der Schalter \ifkorrekturansicht ist bereits durch den Vorspann gesetzt.

%% latex-abspann.tex
%% Gemeinsamer Abspann für Korrekturansicht und Leseansicht.
%% Setzt den Schalter \ifkorrekturansicht voraus (gesetzt in den
%% einbindenden Dateien latex-korrekturansicht-abspann.tex bzw.
%% latex-leseansicht-abspann.tex).
%% ---------------------------------------------------------------

\normalsize

% Das esempio-Environment wird nur in der Leseansicht benötigt
\ifkorrekturansicht\else
\newenvironment{esempio}[3]%
{
    \vspace{1.5ex}
    \rlap{\underline{#1}}
    \par
    \setlength{\parindent}{0cm}
    \nopagebreak
    \leftskip=#2cm
    \rightskip=#3cm
}
{
    \par
}
\fi

\doendnotes{C}
\bigskip
\vfill

\clearpage

\footnotesize

\ifkorrekturansicht
  \lohead{\textsc{register}}
\fi

% theindex-Environment neu definieren ohne reledmac
\makeatletter
\renewenvironment{theindex}{%
  \ifkorrekturansicht
    \section*{\indexname}%
  \else
    \subsubsection*{Index der erwähnten Entitäten}%
  \fi
  \setlength{\parindent}{0pt}%
  \setlength{\parskip}{0pt plus 0.3pt}%
  \let\item\@idxitem
}{%
  \ifkorrekturansicht\clearpage\fi
}
\makeatother

\IfFileExists{\jobname-pw.ind}{\input{\jobname-pw.ind}}{}

% Quellenangabe nur in der Leseansicht
\ifkorrekturansicht\else
% Fallback-Definitionen, falls die .tex-Datei \titel etc. nicht gesetzt hat
\providecommand{\titel}{}
\providecommand{\editorInnen}{}
\providecommand{\dateiname}{\jobname}

\vspace{3cm}

\vfill

\footnotesize
\textsc{Quelle}: \titel. Herausgegeben von {\editorInnen}. In: \emph{Arthur Schnitzler: Briefwechsel mit Autorinnen und Autoren}.
 Digitale Edition, https://schnitzler-briefe.acdh.oeaw.ac.at/{\dateiname}.html (Stand \today)
\fi

\end{document}


