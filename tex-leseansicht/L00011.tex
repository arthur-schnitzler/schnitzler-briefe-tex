%% latex-korrekturansicht-vorspann.tex
%% Vorspann für die Korrekturansicht.
%% Lädt die gemeinsame Datei latex-vorspann.tex mit gesetztem Schalter.

\newif\ifkorrekturansicht
\korrekturansichttrue

\input{../tex-inputs/latex-vorspann}


\section[Arthur Schnitzler: Widmungsexemplar Alkandi’s Lied für Hugo von Hofmannsthal, {[}5.? 5. 1891{]}]{L00011 Arthur Schnitzler: Widmungsexemplar Alkandi’s Lied für Hugo von
               Hofmannsthal, {[}5.? 5. 1891{]}}
\nopagebreak\mylabel{L00011v}
\rehead{ }\normalsize\beginnumbering\briefempfaengerindex{Hofmannsthal, Hugo von@\textsc{Hofmannsthal, Hugo von}!zzzSchnitzler, Arthur@\emph{von Arthur Schnitzler}!1891-05-052@{{[}5.? 5. 1891{]}}|(be}
\toendnotes[C]{\smallbreak\pagebreak[2]}\Standort{FDH, FDH 3221.}
\physDesc{, 48 Zeichen
\newline{}Handschrift: schwarze Tinte, deutsche Kurrent
\newline{}Ordnung: mit Bleistift von unbekannter Hand beschriftet: »HvH-S.
                                    CI, 44« }
\buchAbdrucke{\weitereDrucke{Hugo von Hofmannsthal: \emph{Bibliothek}. Frankfurt am Main: \emph{S. Fischer} 2011, S. 603.} }\toendnotes[C]{\smallbreak}
\pstart
           \noindent{}{\pb}Meinem verehrten Freund \textsc{Loris}\pend
           
\pstart
           herzlichſt{\\[\baselineskip]}\spacefill\mbox{ArthSch.}\pend
           \leftskip=0em{}{\vspace{1\baselineskip}}
\pstart
           \centering{}\textcolor{gray}{\textbf{\textbf{\label{K_L00011-1v}\edtext{Alkandi’s Lied}{\lemma{\textnormal{\emph{Alkandi’s Lied}}}\Cendnote{\textnormal{Separat-Abdruck aus den Heften 17
                           und 18 der Zeitschrift \emph{An der schönen
                              blauen Donau}\pwindex{der schoenen blauen Donau@\emph{An der schönen blauen Donau}|pwk}}}}\label{K_L00011-1}}\pwindex{Alkandi s Lied@\emph{Alkandi’s Lied}|pw}.}}\pend
           
\pstart
           \centering{}\textcolor{gray}{\textbf{\so{Dramatiſches Gedicht in einem Aufzuge.}}}\pend
           
\pstart
           \centering{}\textcolor{gray}{\textbf{Von}}{\\}\textcolor{gray}{\textbf{\textbf{Arthur Schnitzler.}}}\pend
           {\vspace{1\baselineskip}}
\pstart
           \centering{}\textcolor{gray}{\textbf{Nachdruck verboten. – Den Bühnen gegenüber als Manuſcript.}}\pend
           
\pstart
           \centering{}\textcolor{gray}{\textbf{Wien\oindex{Wien@\textbf{Wien}, \emph{A.ADM2}|pw}, 1890.}}\pend
           \selectlanguage{ngerman}\endnumbering\briefempfaengerindex{Hofmannsthal, Hugo von@\textsc{Hofmannsthal, Hugo von}!zzzSchnitzler, Arthur@\emph{von Arthur Schnitzler}!1891-05-052@{{[}5.? 5. 1891{]}}|)be}\mylabel{L00011h}  \normalsize

\doendnotes{C}
\bigskip
\vfill

\clearpage

\footnotesize

\lohead{\textsc{register}}

% Definiere theindex-Environment komplett neu ohne reledmac
\makeatletter
\renewenvironment{theindex}{%
  \section*{\indexname}%
  \setlength{\parindent}{0pt}%
  \setlength{\parskip}{0pt plus 0.3pt}%
  \let\item\@idxitem
}{%
  \clearpage
}
\makeatother

\IfFileExists{\jobname-pw.ind}{\input{\jobname-pw.ind}}{}

\end{document}

      