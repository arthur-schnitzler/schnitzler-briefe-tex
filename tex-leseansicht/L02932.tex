%% latex-leseansicht-vorspann.tex
%% Vorspann für die Leseansicht.
%% Lädt die gemeinsame Datei latex-vorspann.tex mit nicht gesetztem Schalter.

\newif\ifkorrekturansicht
\korrekturansichtfalse

\input{../tex-inputs/latex-vorspann}


\section[ Paul Goldmann an Arthur Schnitzler, 20. 9. [1900]]{L02932 Paul Goldmann an Arthur Schnitzler,  20. 9. [1900]}
\nopagebreak\mylabel{L02932v}
\rehead{ }\normalsize\beginnumbering\briefempfaengerindex{Schnitzler, Arthur@\textsc{Schnitzler, Arthur}!zzzGoldmann, Paul@\emph{von Paul Goldmann}!1900-09-201@{20. 9. [1900]}|(be}
\toendnotes[C]{\smallbreak\pagebreak[2]}
\correspDesc{Versand  durch Paul Goldmann am 20. 9. [1900] in Berlin
\newline{}Erhalt  durch Arthur Schnitzler im Zeitraum [21. 9. 1900
                  – 25. 9. 1900?] in Wien}\toendnotes[C]{\smallbreak}
\Standort{DLA, A:Schnitzler, HS.NZ85.1.3170.}
\physDesc{Brief, 1 Blatt, 2 Seiten, 1315 Zeichen
\newline{}Handschrift: blaue Tinte, deutsche Kurrent
\newline{}Schnitzler: 1) mit Bleistift das Jahr »900« vermerkt  2) mit rotem Buntstift eine Unterstreichung}\toendnotes[C]{\smallbreak}
\pstart
           \noindent{}
\pstart
           {\pb}Berlin\oindex{Berlin@\textbf{Berlin}, \emph{Hauptstadt}|pw}, 20. September.\pend
           
\pstart
           \raggedleft{}\textcolor{gray}{\textbf{DESSAUERSTRASSE 19}}\oindex{Dessauer Straße@\textbf{Dessauer Straße}, \emph{Straße}|pw}\pend
           \pend
           
\pstart
           \centering{}Mein lieber Freund,\pend
           
\pstart
           Geſtern war Abendgeſellſchaft bei Frau \textsc{M.-C.\pwindex{Meyer-Cohn, Helene 30.\,12.\,1859 Lviv – 9.\,11.\,1918 Berlin@\textsc{Meyer-Cohn, Helene} (30.\,12.\,1859 Lviv – 9.\,11.\,1918 Berlin), \emph{Übersetzerin}|pwv}} Ich war geladen, \textsc{Kerr\pwindex{Kerr, Alfred 25.\,12.\,1867 Breslau – 12.\,10.\,1948 Hamburg@\textsc{Kerr, Alfred} (25.\,12.\,1867 Breslau – 12.\,10.\,1948 Hamburg), \emph{Schriftsteller, Kritiker}|pw}} auch. Nachher gingen wir zuſammen nach Hauſe. \textsc{Kerr\pwindex{Kerr, Alfred 25.\,12.\,1867 Breslau – 12.\,10.\,1948 Hamburg@\textsc{Kerr, Alfred} (25.\,12.\,1867 Breslau – 12.\,10.\,1948 Hamburg), \emph{Schriftsteller, Kritiker}|pw}} wünſchte eine Ausſprache. Ich war bereit und{ }ſagte, wie es mit mir{ }ſteht. Er
               war weniger deutlich, weil er bereits Thatſachen zu verſchweigen hat, über die ein
                  \textsc{Gentleman} nicht{ }ſpricht. Immerhin war er{ }ſo deutlich,
               daß ich heute weiß: er und das \label{K_L02932-1v}\edtext{Mädel\pwindex{Wendt, Anna @\textsc{Wendt, Anna}|pwv}}{\lemma{\textnormal{\emph{Mädel}}}\Cendnote{\textnormal{Anna Wendt\pwindex{Wendt, Anna @\textsc{Wendt, Anna}|pwk}, mit der womöglich auch Goldmann\pwindex{Goldmann, Paul 31.\,1.\,1865 Breslau – 25.\,9.\,1935 Wien@\textsc{Goldmann, Paul} (31.\,1.\,1865 Breslau – 25.\,9.\,1935 Wien), \emph{Schriftsteller, Journalist}|pwk} ein Verhältnis hatte oder ersehnte.
                  Siehe auch XXXX Auszeichnungsfehler: Dokument L02911 nicht gefunden.}}}\label{K_L02932-1}{ }ſind längſt
               einig. Ich hätte es erwarten{ }ſollen, aber ich war doch mit ein Bischen Hoffnung nach
                  Berlin\oindex{Berlin@\textbf{Berlin}, \emph{Hauptstadt}|pw} zurückgekommen. Darum traf es mich{ }ſchwer. Es iſt nicht blos der Schmerz abgewieſener Verliebtheit. Es iſt viel mehr.
               Ich frage mich: warum er und nicht ich? warum muß ich immer der Ausgeſtoßene{ }ſein?
               warum muß ich {\pb}zuſehen, wie ein Anderer\pwindex{Kerr, Alfred 25.\,12.\,1867 Breslau – 12.\,10.\,1948 Hamburg@\textsc{Kerr, Alfred} (25.\,12.\,1867 Breslau – 12.\,10.\,1948 Hamburg), \emph{Schriftsteller, Kritiker}|pwv} mit einem Schlage Liebe, Jugend,
               Schönheit, \label{K_L02932-2v}\edtext{Reichthum}{\lemma{\textnormal{\emph{Reichthum}}}\Cendnote{\textnormal{Das dürfte als Ausdruck der psychischen
                  Verfassung Goldmanns\pwindex{Goldmann, Paul 31.\,1.\,1865 Breslau – 25.\,9.\,1935 Wien@\textsc{Goldmann, Paul} (31.\,1.\,1865 Breslau – 25.\,9.\,1935 Wien), \emph{Schriftsteller, Journalist}|pwk} zu lesen sein und sich
                  nicht auf einen tatsächlichen Reichtum bei Anna
                     Wendt\pwindex{Wendt, Anna @\textsc{Wendt, Anna}|pwk} beziehen, die die Tochter eines Briefträgers war und ohne
                  Berufsbildung blieb.}}}\label{K_L02932-2}, alles Glück gewinnt? Und mein Leben{ }ſtarrt vor Öde,{ }ſo daß ich kaum mehr die Kraft habe, weiter meinen Weg zu gehen, wie bisher. Ich habe
                  heut mit wachen Augen die Nacht verbracht; und weil
               mir dieſer Fall zum Symbol wird, weil ich an ihm die Ausſichtsloſigkeit aller meiner
               Wünſche, die Unmöglichkeit, meine Lebenslage zu ändern und nur etwas von dem \label{K_L02932-3v}\edtext{Erſehnten}{\lemma{\textnormal{\emph{Ersehnten}}}\Cendnote{\textnormal{Goldmann\pwindex{Goldmann, Paul 31.\,1.\,1865 Breslau – 25.\,9.\,1935 Wien@\textsc{Goldmann, Paul} (31.\,1.\,1865 Breslau – 25.\,9.\,1935 Wien), \emph{Schriftsteller, Journalist}|pwk} schrieb
                  »Erſehnhten«.}}}\label{K_L02932-3} zu erreichen, – weil ich an ihm die
               Hoffnungsloſigkeit meines Schickſals von Neuem erkenne, – trage ich eine tiefe
               Verzweiflung in mir{\dotsfive}\pend
           
\pstart
           Viele Grüße! {\\[\baselineskip]}Dein {\\[\baselineskip]}\spacefill\mbox{Paul Goldmn}\pend
           \leftskip=0em{}\selectlanguage{ngerman}\endnumbering\briefempfaengerindex{Schnitzler, Arthur@\textsc{Schnitzler, Arthur}!zzzGoldmann, Paul@\emph{von Paul Goldmann}!1900-09-201@{20. 9. [1900]}|)be}\mylabel{L02932h}  \newcommand{\dateiname}{L02932}\newcommand{\titel}{Paul Goldmann an Arthur Schnitzler, 20. 9. [1900]}\newcommand{\editorInnen}{Martin Anton Müller und Laura Untner}%% latex-leseansicht-abspann.tex
%% Abspann für die Leseansicht.
%% Der Schalter \ifkorrekturansicht ist bereits durch den Vorspann gesetzt.

%% latex-abspann.tex
%% Gemeinsamer Abspann für Korrekturansicht und Leseansicht.
%% Setzt den Schalter \ifkorrekturansicht voraus (gesetzt in den
%% einbindenden Dateien latex-korrekturansicht-abspann.tex bzw.
%% latex-leseansicht-abspann.tex).
%% ---------------------------------------------------------------

\normalsize

% Das esempio-Environment wird nur in der Leseansicht benötigt
\ifkorrekturansicht\else
\newenvironment{esempio}[3]%
{
    \vspace{1.5ex}
    \rlap{\underline{#1}}
    \par
    \setlength{\parindent}{0cm}
    \nopagebreak
    \leftskip=#2cm
    \rightskip=#3cm
}
{
    \par
}
\fi

\doendnotes{C}
\bigskip
\vfill

\clearpage

\footnotesize

\ifkorrekturansicht
  \lohead{\textsc{register}}
\fi

% theindex-Environment neu definieren ohne reledmac
\makeatletter
\renewenvironment{theindex}{%
  \ifkorrekturansicht
    \section*{\indexname}%
  \else
    \subsubsection*{Index der erwähnten Entitäten}%
  \fi
  \setlength{\parindent}{0pt}%
  \setlength{\parskip}{0pt plus 0.3pt}%
  \let\item\@idxitem
}{%
  \ifkorrekturansicht\clearpage\fi
}
\makeatother

\IfFileExists{\jobname-pw.ind}{\input{\jobname-pw.ind}}{}

% Quellenangabe nur in der Leseansicht
\ifkorrekturansicht\else
% Fallback-Definitionen, falls die .tex-Datei \titel etc. nicht gesetzt hat
\providecommand{\titel}{}
\providecommand{\editorInnen}{}
\providecommand{\dateiname}{\jobname}

\vspace{3cm}

\vfill

\footnotesize
\textsc{Quelle}: \titel. Herausgegeben von {\editorInnen}. In: \emph{Arthur Schnitzler: Briefwechsel mit Autorinnen und Autoren}.
 Digitale Edition, https://schnitzler-briefe.acdh.oeaw.ac.at/{\dateiname}.html (Stand \today)
\fi

\end{document}


