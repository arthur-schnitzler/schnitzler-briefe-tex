%% latex-korrekturansicht-vorspann.tex
%% Vorspann für die Korrekturansicht.
%% Lädt die gemeinsame Datei latex-vorspann.tex mit gesetztem Schalter.

\newif\ifkorrekturansicht
\korrekturansichttrue

\input{../tex-inputs/latex-vorspann}


\section[Olga und Arthur Schnitzler an Richard Beer-Hofmann, 30. 8. 1912]{L02087 Olga und Arthur Schnitzler an Richard Beer-Hofmann, 30. 8. 1912}
\nopagebreak\mylabel{L02087v}
\rehead{ }\normalsize\beginnumbering\briefempfaengerindex{Beer-Hofmann, Paula@\textsc{Beer-Hofmann, Paula}!zzzSchnitzler, Olga@\emph{von Olga Schnitzler}!1912-08-301@{30. 8. 1912}|(be}\briefempfaengerindex{Beer-Hofmann, Paula@\textsc{Beer-Hofmann, Paula}!zzzSchnitzler, Arthur@\emph{von Arthur Schnitzler}!1912-08-301@{30. 8. 1912}|(be}\briefempfaengerindex{Beer-Hofmann, Richard@\textsc{Beer-Hofmann, Richard}!zzzSchnitzler, Olga@\emph{von Olga Schnitzler}!1912-08-301@{30. 8. 1912}|(be}\briefempfaengerindex{Beer-Hofmann, Richard@\textsc{Beer-Hofmann, Richard}!zzzSchnitzler, Arthur@\emph{von Arthur Schnitzler}!1912-08-301@{30. 8. 1912}|(be}
\toendnotes[C]{\smallbreak\pagebreak[2]}\Standort{YCGL, MSS 31.}
\physDesc{Bildpostkarte, 286 Zeichen
\newline{}Handschrift Arthur Schnitzler: Bleistift, deutsche Kurrent
\newline{}Handschrift Olga Schnitzler: Bleistift, lateinische Kurrent
\newline{}Versand: Stempel: »\nobreak{}\oindex{Tutzing@\textbf{Tutzing}, \emph{P.PPLA4}|pwk}Tutzing, 31 Aug \textcolor{gray}{12}\nobreak{}«.  
\newline{}Beer-Hofmann: mit blauem Buntstift das Datum der Beantwortung vermerkt:
                                    »B 2./IX 12« }\pstart{}{\pb}Herrn u Frau\pend{}\pstart{}D\textsuperscript{r} Richard Beer-Hofmann\pend{}\pstart{}Wien XVIII\oindex{XVIII., Waehring@\textbf{XVIII., Währing}, \emph{A.ADM3}|pw}\pend{}\pstart{}Hasenauerstr. 59\oindex{Hasenauerstrasse 59@\textbf{Hasenauerstraße 59}, \emph{Wohngebäude (K.WHS)}|pw}.\pend{}{\bigskip}
\pstart
           \noindent{}\centering{}{\pb}\textcolor{gray}{\textbf{Sonnenuntergang am Starnbergersee\oindex{Starnberger See@\textbf{Starnberger See}, \emph{H.LK}|pw}}}\pend
           \vspace{1em}
\pstart
           \noindent{}{\pb}{[}hs. :{]} lieber Richard, ſind Sie ſchon in Wien\oindex{Wien@\textbf{Wien}, \emph{A.ADM2}|pw}? Wie lang waren Sie in Iſchl\oindex{Bad Ischl@\textbf{Bad Ischl}, \emph{P.PPL}|pw}? Wie
               gehts Ihnen? Wir wollen hier bis \textsc{circa}{ }7. bleiben. Laſſen Sie \strikeout{uns} was von ſich
               hören!\pend
           
\pstart
           Herzlichſt{\\[\baselineskip]}Ihr \spacefill\mbox{Arthur}\pend
           \leftskip=0em{}\selectlanguage{ngerman}\vspace{1em}
\pstart
           \noindent{}{[}hs. :{]} Herzliche Grüsse!\pend
           \pstart \spacefill\mbox{Olga.}\pend{}
\pstart
           \noindent{}Tutzing\oindex{Tutzing@\textbf{Tutzing}, \emph{P.PPLA4}|pw},{\\}Hotel Simson\oindex{Hotel Simson@\textbf{Hotel Simson}, \emph{Hotel (K.HTL)}|pw}\pend
           
\pstart
           30. Aug. 1912\pend
           \selectlanguage{ngerman}\endnumbering\briefempfaengerindex{Beer-Hofmann, Paula@\textsc{Beer-Hofmann, Paula}!zzzSchnitzler, Olga@\emph{von Olga Schnitzler}!1912-08-301@{30. 8. 1912}|)be}\briefempfaengerindex{Beer-Hofmann, Paula@\textsc{Beer-Hofmann, Paula}!zzzSchnitzler, Arthur@\emph{von Arthur Schnitzler}!1912-08-301@{30. 8. 1912}|)be}\briefempfaengerindex{Beer-Hofmann, Richard@\textsc{Beer-Hofmann, Richard}!zzzSchnitzler, Olga@\emph{von Olga Schnitzler}!1912-08-301@{30. 8. 1912}|)be}\briefempfaengerindex{Beer-Hofmann, Richard@\textsc{Beer-Hofmann, Richard}!zzzSchnitzler, Arthur@\emph{von Arthur Schnitzler}!1912-08-301@{30. 8. 1912}|)be}\mylabel{L02087h}  \normalsize

\doendnotes{C}
\bigskip
\vfill

\clearpage

\footnotesize

\lohead{\textsc{register}}

% Definiere theindex-Environment komplett neu ohne reledmac
\makeatletter
\renewenvironment{theindex}{%
  \section*{\indexname}%
  \setlength{\parindent}{0pt}%
  \setlength{\parskip}{0pt plus 0.3pt}%
  \let\item\@idxitem
}{%
  \clearpage
}
\makeatother

\IfFileExists{\jobname-pw.ind}{\input{\jobname-pw.ind}}{}

\end{document}

      