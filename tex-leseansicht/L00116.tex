%% latex-leseansicht-vorspann.tex
%% Vorspann für die Leseansicht.
%% Lädt die gemeinsame Datei latex-vorspann.tex mit nicht gesetztem Schalter.

\newif\ifkorrekturansicht
\korrekturansichtfalse

\input{../tex-inputs/latex-vorspann}


\section[Arthur Schnitzler an Richard Beer-Hofmann, 22. 8. 1892]{L00116 Arthur Schnitzler an Richard Beer-Hofmann, 22. 8. 1892}
\nopagebreak\mylabel{L00116v}
\rehead{ }\normalsize\beginnumbering\briefempfaengerindex{Beer-Hofmann, Richard@\textsc{Beer-Hofmann, Richard}!zzzSchnitzler, Arthur@\emph{von Arthur Schnitzler}!1892-08-221@{22. 8. 1892}|(be}
\toendnotes[C]{\smallbreak\pagebreak[2]}
\correspDesc{Versand  durch Arthur Schnitzler am 22. 8. 1892 in Wien
\newline{}Erhalt  durch Richard Beer-Hofmann am 23. 8. 1892 in Bad Ischl}\toendnotes[C]{\smallbreak}
\Standort{YCGL, MSS 31.}
\physDesc{Brief, 1 Blatt, 3 Seiten, Kuvert, 847 Zeichen
\newline{}Handschrift: schwarze Tinte, deutsche Kurrent
\newline{}Versand: 1) Stempel: »\nobreak{}\oindex{Wien@\textbf{Wien}, \emph{Verwaltungsgebiet}|pwk}Wien 4/1, 22 8 92, 6–7N\nobreak{}«.   2) Stempel: »\nobreak{}\oindex{Bad Ischl@\textbf{Bad Ischl}|pwk}{\pb}Ischl, 23 8 9{[}2{]}, 7–8\nobreak{}«. }
\buchAbdrucke{\weitereDrucke{Arthur Schnitzler, Richard Beer-Hofmann: \emph{Briefwechsel 1891–1931}. Herausgegeben von Konstanze Fliedl. Wien, Zürich: \emph{Europaverlag} 1992, S. 37–38.} }\toendnotes[C]{\smallbreak}\pstart{}{\pb}Herrn Doctor \textsc{Rich.
                     Beer-Hofmann}\pend{}\pstart{}\textsc{Ischl\oindex{Bad Ischl@\textbf{Bad Ischl}|pw}.}\pend{}\pstart{}\textsc{Grazerstraße 6\oindex{Grazer Straße [Bad Ischl]@\textbf{Grazer Straße [Bad Ischl]}, \emph{Straße}|pw}}.\pend{}\pstart{}(oder \textsc{Kreuzplatz}\oindex{Kreuzplatz@\textbf{Kreuzplatz}, \emph{Platz}|pw})\pend{}{\bigskip}\vspace{1em}
\pstart
           \noindent{}{\pb}Mein lieber Richard! Warum{ }ſchreiben Sie Opernring 12\oindex{Wien@\textbf{Wien}!I., Innere Stadt@\textbf{I., Innere Stadt}!Opernring@\textbf{Opernring}, \emph{Straße}|pw}; da ich doch Kärnthnerring 12\oindex{Wien@\textbf{Wien}!I., Innere Stadt@\textbf{I., Innere Stadt}!Kärntnerring 12/Bösendorferstraße 11@\textbf{Kärntnerring 12/Bösendorferstraße 11}, \emph{Wohngebäude}|pw} oder Giſelastr. 11\oindex{Wien@\textbf{Wien}!I., Innere Stadt@\textbf{I., Innere Stadt}!Kärntnerring 12/Bösendorferstraße 11@\textbf{Kärntnerring 12/Bösendorferstraße 11}, \emph{Wohngebäude}|pw}
               wohne? Dadurch bekam ich erſt heute Ihren Brief. Nun kann ich Ihnen mittheilen, daſs
               ich{ }ſchon in wenig Tagen, Ende dieſer Woche, in Iſchl\oindex{Bad Ischl@\textbf{Bad Ischl}|pw} einlangen werde. Ich bleibe etwa 8-10 Tage dort und will jedenfalls
               weiter. Laſſen Sie mich Sie übrigens beneiden, {\pb}daſs
               Sie \uline{verſti{\geminationm}t}{ }ſind; es iſt das{ }ſicherſte Zeichen, daſs Sie nicht
               unglücklich{ }ſind. –\pend
           
\pstart
           Könnte unſer lieber Paul\pwindex{Goldmann, Paul 31.\,1.\,1865 Breslau – 25.\,9.\,1935 Wien@\textsc{Goldmann, Paul} (31.\,1.\,1865 Breslau – 25.\,9.\,1935 Wien), \emph{Schriftsteller, Journalist}|pw} das nicht geſagt
               haben? – Ein reizendes Feuilleton\pwindex{Goldmann, Paul 31.\,1.\,1865 Breslau – 25.\,9.\,1935 Wien@\textsc{Goldmann, Paul} (31.\,1.\,1865 Breslau – 25.\,9.\,1935 Wien), \emph{Schriftsteller, Journalist}!Spanisches Strandleben@\strich\emph{Spanisches Strandleben}|pwv} von ihm erſchien eben in der Frkf. Ztg\pwindex{Frankfurter Zeitung@\emph{Frankfurter Zeitung}|pw}; – aus San Sebastian\oindex{San Sebastian@\textbf{San Sebastian}, \emph{Hauptstadt}|pw}. –\pend
           
\pstart
           Ich freue mich{ }ſehr, Sie bald zu{ }ſehn; und da ich heute{ }ſchon in großen Worten drin
               bin,{ }ſo will ich Ihnen geſtehn, daſs ich mich aufrichtig nach Ihnen{ }ſehne.\pend
           
\pstart
           {\pb}\strikeout{Vielleicht} Viele herzliche Grüße{\\[\baselineskip]}der Ihre{\\[\baselineskip]}\spacefill\mbox{Arthur}\pend
           \leftskip=0em{}
\pstart
           22. 8. 92.\pend
           \selectlanguage{ngerman}\endnumbering\briefempfaengerindex{Beer-Hofmann, Richard@\textsc{Beer-Hofmann, Richard}!zzzSchnitzler, Arthur@\emph{von Arthur Schnitzler}!1892-08-221@{22. 8. 1892}|)be}\mylabel{L00116h}  \newcommand{\dateiname}{L00116}\newcommand{\titel}{Arthur Schnitzler an Richard Beer-Hofmann, 22. 8. 1892}\newcommand{\editorInnen}{Martin Anton Müller und Gerd-Hermann Susen}%% latex-leseansicht-abspann.tex
%% Abspann für die Leseansicht.
%% Der Schalter \ifkorrekturansicht ist bereits durch den Vorspann gesetzt.

%% latex-abspann.tex
%% Gemeinsamer Abspann für Korrekturansicht und Leseansicht.
%% Setzt den Schalter \ifkorrekturansicht voraus (gesetzt in den
%% einbindenden Dateien latex-korrekturansicht-abspann.tex bzw.
%% latex-leseansicht-abspann.tex).
%% ---------------------------------------------------------------

\normalsize

% Das esempio-Environment wird nur in der Leseansicht benötigt
\ifkorrekturansicht\else
\newenvironment{esempio}[3]%
{
    \vspace{1.5ex}
    \rlap{\underline{#1}}
    \par
    \setlength{\parindent}{0cm}
    \nopagebreak
    \leftskip=#2cm
    \rightskip=#3cm
}
{
    \par
}
\fi

\doendnotes{C}
\bigskip
\vfill

\clearpage

\footnotesize

\ifkorrekturansicht
  \lohead{\textsc{register}}
\fi

% theindex-Environment neu definieren ohne reledmac
\makeatletter
\renewenvironment{theindex}{%
  \ifkorrekturansicht
    \section*{\indexname}%
  \else
    \subsubsection*{Index der erwähnten Entitäten}%
  \fi
  \setlength{\parindent}{0pt}%
  \setlength{\parskip}{0pt plus 0.3pt}%
  \let\item\@idxitem
}{%
  \ifkorrekturansicht\clearpage\fi
}
\makeatother

\IfFileExists{\jobname-pw.ind}{\input{\jobname-pw.ind}}{}

% Quellenangabe nur in der Leseansicht
\ifkorrekturansicht\else
% Fallback-Definitionen, falls die .tex-Datei \titel etc. nicht gesetzt hat
\providecommand{\titel}{}
\providecommand{\editorInnen}{}
\providecommand{\dateiname}{\jobname}

\vspace{3cm}

\vfill

\footnotesize
\textsc{Quelle}: \titel. Herausgegeben von {\editorInnen}. In: \emph{Arthur Schnitzler: Briefwechsel mit Autorinnen und Autoren}.
 Digitale Edition, https://schnitzler-briefe.acdh.oeaw.ac.at/{\dateiname}.html (Stand \today)
\fi

\end{document}


