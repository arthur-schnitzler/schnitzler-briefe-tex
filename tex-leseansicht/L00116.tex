%% latex-korrekturansicht-vorspann.tex
%% Vorspann für die Korrekturansicht.
%% Lädt die gemeinsame Datei latex-vorspann.tex mit gesetztem Schalter.

\newif\ifkorrekturansicht
\korrekturansichttrue

\input{../tex-inputs/latex-vorspann}


\section[Arthur Schnitzler an Richard Beer-Hofmann, 22. 8. 1892]{L00116 Arthur Schnitzler an Richard Beer-Hofmann, 22. 8. 1892}
\nopagebreak\mylabel{L00116v}
\rehead{ }\normalsize\beginnumbering\briefempfaengerindex{Beer-Hofmann, Richard@\textsc{Beer-Hofmann, Richard}!zzzSchnitzler, Arthur@\emph{von Arthur Schnitzler}!1892-08-221@{22. 8. 1892}|(be}
\toendnotes[C]{\smallbreak\pagebreak[2]}\Standort{YCGL, MSS 31.}
\physDesc{Brief, 1 Blatt, 3 Seiten, Umschlag, 847 Zeichen
\newline{}Handschrift: schwarze Tinte, deutsche Kurrent
\newline{}Versand: 1) Stempel: »\nobreak{}Wien 4/1, 22 8 92, 6–7N\nobreak{}«.   2) Stempel: »\nobreak{}\oindex{Bad Ischl@\textbf{Bad Ischl}, \emph{P.PPL}|pwk}{\pb}Ischl, 23 8 9{[}2{]}, 7–8\nobreak{}«. }
\buchAbdrucke{\weitereDrucke{Arthur Schnitzler, Richard Beer-Hofmann: \emph{Briefwechsel 1891–1931}. Wien, Zürich: \emph{Europaverlag} 1992, S. 37–38.} }\toendnotes[C]{\smallbreak}\pstart{}{\pb}Herrn Doctor \textsc{Rich.
                     Beer-Hofmann}\pend{}\pstart{}\textsc{Ischl\oindex{Bad Ischl@\textbf{Bad Ischl}, \emph{P.PPL}|pw}.}\pend{}\pstart{}\textsc{Grazerstraße 6\oindex{Grazer Strasse [Bad Ischl]@\textbf{Grazer Straße [Bad Ischl]}, \emph{Straße (K.STR)}|pw}}.\pend{}\pstart{}(oder \textsc{Kreuzplatz}\oindex{Kreuzplatz@\textbf{Kreuzplatz}, \emph{Platz (K.PLT)}|pw})\pend{}{\bigskip}\vspace{1em}
\pstart
           \noindent{}{\pb}Mein lieber Richard! Warum ſchreiben Sie Opernring 12\oindex{Opernring@\textbf{Opernring}, \emph{Straße (K.STR)}|pw}; da ich doch Kärnthnerring 12\oindex{Kaerntnerring 12/Boesendorferstrasse 11@\textbf{Kärntnerring 12/Bösendorferstraße 11}, \emph{Wohngebäude (K.WHS)}|pw} oder Giſelastr. 11\oindex{Kaerntnerring 12/Boesendorferstrasse 11@\textbf{Kärntnerring 12/Bösendorferstraße 11}, \emph{Wohngebäude (K.WHS)}|pw}
               wohne? Dadurch bekam ich erſt heute Ihren Brief. Nun kann ich Ihnen mittheilen, daſs
               ich ſchon in wenig Tagen, Ende dieſer Woche, in Iſchl\oindex{Bad Ischl@\textbf{Bad Ischl}, \emph{P.PPL}|pw} einlangen werde. Ich bleibe etwa 8-10 Tage dort und will jedenfalls
               weiter. Laſſen Sie mich Sie übrigens beneiden, {\pb}daſs
               Sie \uline{verſti{\geminationm}t}{ }ſind; es iſt das ſicherſte Zeichen, daſs Sie nicht
               unglücklich ſind. –\pend
           
\pstart
           Könnte unſer lieber Paul\pwindex{Goldmann, Paul 31.01.1865 – 25.09.1935@\textsc{Goldmann, Paul} (31.01.1865 – 25.09.1935), \emph{Schriftsteller/Schriftstellerin, Journalist/Journalistin}|pw} das nicht geſagt
               haben? – Ein reizendes Feuilleton\pwindex{Spanisches Strandleben@\emph{Spanisches Strandleben}|pwv} von ihm erſchien eben in der Frkf. Ztg\pwindex{Frankfurter Zeitung@\emph{Frankfurter Zeitung}|pw}; – aus San Sebastian\oindex{San Sebastian@\textbf{San Sebastian}, \emph{P.PPLA2}|pw}. –\pend
           
\pstart
           Ich freue mich ſehr, Sie bald zu ſehn; und da ich heute ſchon in großen Worten drin
               bin, ſo will ich Ihnen geſtehn, daſs ich mich aufrichtig nach Ihnen ſehne.\pend
           
\pstart
           {\pb}\strikeout{Vielleicht} Viele herzliche Grüße{\\[\baselineskip]}der Ihre{\\[\baselineskip]}\spacefill\mbox{Arthur}\pend
           \leftskip=0em{}
\pstart
           22. 8. 92.\pend
           \selectlanguage{ngerman}\endnumbering\briefempfaengerindex{Beer-Hofmann, Richard@\textsc{Beer-Hofmann, Richard}!zzzSchnitzler, Arthur@\emph{von Arthur Schnitzler}!1892-08-221@{22. 8. 1892}|)be}\mylabel{L00116h}  \normalsize

\doendnotes{C}
\bigskip
\vfill

\clearpage

\footnotesize

\lohead{\textsc{register}}

% Definiere theindex-Environment komplett neu ohne reledmac
\makeatletter
\renewenvironment{theindex}{%
  \section*{\indexname}%
  \setlength{\parindent}{0pt}%
  \setlength{\parskip}{0pt plus 0.3pt}%
  \let\item\@idxitem
}{%
  \clearpage
}
\makeatother

\IfFileExists{\jobname-pw.ind}{\input{\jobname-pw.ind}}{}

\end{document}

      