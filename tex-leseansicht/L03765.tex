%% latex-korrekturansicht-vorspann.tex
%% Vorspann für die Korrekturansicht.
%% Lädt die gemeinsame Datei latex-vorspann.tex mit gesetztem Schalter.

\newif\ifkorrekturansicht
\korrekturansichttrue

\input{../tex-inputs/latex-vorspann}


\section[Theodor Herzl an Arthur Schnitzler, 25. 11. 1901]{L03765 Theodor Herzl an Arthur Schnitzler, 25. 11. 1901}
\nopagebreak\mylabel{L03765v}
\rehead{ }\normalsize\beginnumbering\briefempfaengerindex{Schnitzler, Arthur@\textsc{Schnitzler, Arthur}!zzzHerzl, Theodor@\emph{von Theodor Herzl}!1901-11-252@{25. 11. 1901}|(be}
\toendnotes[C]{\smallbreak\pagebreak[2]}\Standort{CUL, Schnitzler, B 39.}
\physDesc{Brief, 1 Blatt, 1 Seite, 160 Zeichen
\newline{}Handschrift: lila Tinte, lateinische Kurrent
\newline{}Ordnung: mit Bleistift von unbekannter Hand nummeriert: »60« }\toendnotes[C]{\smallbreak}
\pstart
           {\pb}\textcolor{gray}{\textbf{NEUE FREIE PRESSE}}\pwindex{Neue Freie Presse@\emph{Neue Freie Presse}|pw}\pend
           
\pstart
           \textcolor{gray}{\textbf{\textsc{Redaction:}}}\pend
           
\pstart
           \textcolor{gray}{\textbf{WIEN\oindex{Wien@\textbf{Wien}, \emph{A.ADM2}|pw}}}\pend
           
\pstart
           \textcolor{gray}{\textbf{\textsc{Kolowratring, Fichtegasse Nr. 11\oindex{Fichtegasse 11@\textbf{Fichtegasse 11}, \emph{Gebäude (K.GBD)}|pw}:}}}\pend
           
\pstart{}Lieber Freund,\pend\vspace{0.5em}
\pstart
           hiemit lade ich Sie ein, eine
      \label{K_L03765-1v}\edtext{Novelle}{\lemma{\textnormal{\emph{Novelle}}}\Cendnote{\textnormal{Eine unmittelbare Antwort Schnitzlers ist
      nicht erhalten. XXXX ÜBERPRÜFEN. Schnitzler dürfte aber abgesagt haben und eine Einreichung
      für Ostern 1902 versprochen haben, {XXXX ref}}}}\label{K_L03765-1} (nicht über 1200 Zeilen
      lang) für unsre neue
      Sonntagsnovellen-Rubrik
      zu schreiben.
      \pend
           
\pstart
           \label{K_L03765-2v}\edtext{U. A. w. g.}{\lemma{\textnormal{\emph{U. A. w. g.}}}\Cendnote{\textnormal{Um Antwort wird gebeten}}}\label{K_L03765-2}\pend
           
\pstart
           Ihr{\\[\baselineskip]}\spacefill\mbox{Herzl}\pend
           \leftskip=0em{}
\pstart
           25 XI 901\pend
           \selectlanguage{ngerman}\endnumbering\briefempfaengerindex{Schnitzler, Arthur@\textsc{Schnitzler, Arthur}!zzzHerzl, Theodor@\emph{von Theodor Herzl}!1901-11-252@{25. 11. 1901}|)be}\mylabel{L03765h}
\begin{anhang}
\end{anhang}\normalsize

\doendnotes{C}
\bigskip
\vfill

\clearpage

\footnotesize

\lohead{\textsc{register}}

% Definiere theindex-Environment komplett neu ohne reledmac
\makeatletter
\renewenvironment{theindex}{%
  \section*{\indexname}%
  \setlength{\parindent}{0pt}%
  \setlength{\parskip}{0pt plus 0.3pt}%
  \let\item\@idxitem
}{%
  \clearpage
}
\makeatother

\IfFileExists{\jobname-pw.ind}{\input{\jobname-pw.ind}}{}

\end{document}

      