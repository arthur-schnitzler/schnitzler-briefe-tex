%% latex-leseansicht-vorspann.tex
%% Vorspann für die Leseansicht.
%% Lädt die gemeinsame Datei latex-vorspann.tex mit nicht gesetztem Schalter.

\newif\ifkorrekturansicht
\korrekturansichtfalse

\input{../tex-inputs/latex-vorspann}


\section[Hugo Hofmannsthal an Arthur Schnitzler, 19. 9. 1919]{L02326 Hugo Hofmannsthal an Arthur Schnitzler, 19. 9. 1919}
\nopagebreak\mylabel{L02326v}
\rehead{ }\normalsize\beginnumbering\briefempfaengerindex{Schnitzler, Arthur@\textsc{Schnitzler, Arthur}!zzzHofmannsthal, Hugo von@\emph{von Hugo von Hofmannsthal}!1919-09-191@{19. 9. 1919}|(be}
\toendnotes[C]{\smallbreak\pagebreak[2]}
\correspDesc{Versand  durch Hugo von Hofmannsthal am 19. 9. 1919 in Bad Aussee
\newline{}Erhalt  durch Arthur Schnitzler im Zeitraum [20. 9. 1919
                  – 24. 9. 1919?] in Wien}\toendnotes[C]{\smallbreak}
\Standort{CUL, Schnitzler, B 43.}
\physDesc{Brief, 1 Blatt, 2 Seiten, 2258 Zeichen
\newline{}Handschrift: schwarze Tinte, deutsche Kurrent
\newline{}Ordnung: 1) mit Bleistift von Frieda
                                    Pollak\pwindex{Pollak, Frieda 8.\,12.\,1881 Wien – 13.\,7.\,1937 ebd.@\textsc{Pollak, Frieda} (8.\,12.\,1881 Wien – 13.\,7.\,1937 ebd.), \emph{Sekretärin}|pw} (?) mit dem Buchstaben »A«
                                 (Abgeschrieben/Abschrift) gekennzeichnet  2) mit Bleistift von unbekannter Hand nummeriert: »\strikeout{355}« 3) mit Bleistift von unbekannter Hand nummeriert:
                                    »382«}
\buchAbdrucke{\weitereDrucke{Hugo von Hofmannsthal, Arthur Schnitzler: \emph{Briefwechsel}. Herausgegeben von Therese Nickl und Heinrich Schnitzler. Frankfurt am Main: \emph{S. Fischer} 1964, S. 284.} }\toendnotes[C]{\smallbreak}
\pstart
           {\pb}Bad Auſſee\oindex{Bad Aussee@\textbf{Bad Aussee}, \emph{Hauptstadt}|pw}, den
                  19. IX 19.\pend
           
\pstart{}mein lieber Arthur\pend\vspace{0.5em}
\pstart
           ſehr oft in dieſem Sommer{ }ſind meine Gedanken zu Ihnen gegangen. In Ferleiten\oindex{Ferleiten@\textbf{Ferleiten}|pw} im Juli, wenn ich
               herumging in dem{ }ſtillen engen Thal das mir die Jahre meiner frühen Jugend{ }ſo nahe
               bringt, fielen Sie mir ein als einer von denen, die{ }ſchon damals meine Freunde waren
               und an die ich auf einem Holztisch in dem kleinen Tannenwald hinterm Gaſthaus – und
               der Holztiſch{ }ſteht noch immer da – Briefe{ }ſchrieb. Das ist{ }ſiebenundzwanzig Jahre
               her, wie{ }ſchwer faſslich! – Dann war ich dreimal in dieſem Sommer in Salzburg\oindex{Salzburg@\textbf{Salzburg}, \emph{Verwaltungsgebiet}|pw} u. nie bin ich durch den Mirabell-garten\oindex{Schloss Mirabell@\textbf{Schloss Mirabell}, \emph{Schloss}|pw} gegangen, nie nach Hellbrunn\oindex{Hellbrunn@\textbf{Hellbrunn}|pw} oder Leopoldskron\oindex{Salzburg-Leopoldskron@\textbf{Salzburg-Leopoldskron}, \emph{Teil eines besiedelten Ortes}|pw}, ohne{ }ſo
               herzlich an Sie zu denken.\pend
           
\pstart
           Das letzte Mal, daſs ich Sie geſehen habe, das war bei der \label{K_L02326-1v}\edtext{Generalprobe\eventindex{Oper@\textbf{Oper}!Generalprobe von Palestrina, 27.2.1919@Generalprobe von Palestrina, 27.2.1919|pwv}}{\lemma{\textnormal{\emph{Generalprobe}}}\Cendnote{\textnormal{Siehe A. S.: \emph{Tagebuch}, 27. 2. 1919.
               }}}\label{K_L02326-1} der Oper »\textsc{Palestrina}\pwindex{\textcolor{red}{\textsuperscript{XXXX indx1}}!Palestrina. Musikalische Legende in drei Akten@\strich\emph{Palestrina. Musikalische Legende in drei Akten}|pw}« – da waren Sie{ }ſo schwer bedrückt von dem was in der Welt vorging und{ }ſich
               anzukündigen{ }ſchien,{ }ſo bemüht u. bekümmert{ }ſah Ihr vertrautes inhaltsvolles Geſicht
               aus – ich wurde dann bald krank, da{ }ſah ich{ }ſehr oft Ihr Geſicht{ }ſo vor mir. Meine
               Krankheit war tiefergehend als{ }ſie im erſten Augenblick{ }ſchien, vom erſten
                  April bis in den Juli hinein war ich ein kranker, veränderter
               Menſch – erſt in Ferleiten\oindex{Ferleiten@\textbf{Ferleiten}|pw}, ganz ganz einſam,
               hab ich mich zu mir{ }ſelber {\pb}zurückgefunden, und nach jedem{ }ſolchen Zurückfinden iſt man ja vielleicht ein{ }ſtärkerer Menſch als je zuvor, man iſt halt um eine Windung der Schraube höher geko{\geminationm}en. –\hspace*{1.5em}So muſs ich mich
               glücklich nennen{ }ſeit Ende Juli, es iſt eine Productivität über mich
                  geko{\geminationm}en wie ich{ }ſie viele Jahre – es waren halt zu{ }ſchwere Jahre – nicht gekannt habe, es{ }ſind Arbeiten fertig geworden, andere in mir
               aufgewacht, noch andere{ }ſtark vorwärts geko{\geminationm}en – ich
               glaube es iſt einiges darunter, dem Sie Ihren Beifall geben werden, der mir immer{ }ſo
               warm u. vertraut und von Grund aus woltuend iſt.\pend
           
\pstart
           So{ }ſtark iſt dieſes Zuſtrömen von Einfällen und{ }ſo{ }ſicher endlich einmal – Sie kennen
               meine bizarre{ }ſchwierige Natur – die rhytmiſche Wiederkehr productiver Stunden, daſs
               ich Strauss\pwindex{Strauss, Richard 11.\,6.\,1864 München – 8.\,9.\,1949 Garmisch-Partenkirchen@\textsc{Strauss, Richard} (11.\,6.\,1864 München – 8.\,9.\,1949 Garmisch-Partenkirchen), \emph{Theaterleiter, Komponist, Dirigent}|pw} u. Schalk\pwindex{Schalk, Franz 27.\,5.\,1863 Wien – 3.\,9.\,1931 Edlach@\textsc{Schalk, Franz} (27.\,5.\,1863 Wien – 3.\,9.\,1931 Edlach), \emph{Theaterleiter, Dirigent}|pw} gebeten habe, mich bei den Proben der »Frau ohne Schatten\pwindex{Hofmannsthal, Hugo von 1.\,2.\,1874 Wien – 15.\,7.\,1929 Rodaun@\textsc{Hofmannsthal, Hugo von} (1.\,2.\,1874 Wien – 15.\,7.\,1929 Rodaun), \emph{Schriftsteller}!Frau ohne Schatten. Erzählung@\strich\emph{Die Frau ohne Schatten. Erzählung}|pw}« zu entſchuldigen – ich bin ja
               dort ohnedies nur das fünfte Rad am Wagen –{ }ſo komme ich erſt knapp vor der \label{K_L02326-2v}\edtext{Première\eventindex{Oper@\textbf{Oper}!Uraufführung vom Die Frau ohne Schatten, 10.10.1919@Uraufführung vom Die Frau ohne Schatten, 10.10.1919|pwv}\pwindex{Hofmannsthal, Hugo von 1.\,2.\,1874 Wien – 15.\,7.\,1929 Rodaun@\textsc{Hofmannsthal, Hugo von} (1.\,2.\,1874 Wien – 15.\,7.\,1929 Rodaun), \emph{Schriftsteller}!Frau ohne Schatten. Erzählung@\strich\emph{Die Frau ohne Schatten. Erzählung}|pwv}}{\lemma{\textnormal{\emph{Première}}}\Cendnote{\textnormal{Die Uraufführung\eventindex{Oper@\textbf{Oper}!Uraufführung vom Die Frau ohne Schatten, 10.10.1919@Uraufführung vom Die Frau ohne Schatten, 10.10.1919|pwkv} fand am
                     10. 10. 1919 in der Wiener Oper\oindex{Wien@\textbf{Wien}!I., Innere Stadt@\textbf{I., Innere Stadt}!Oper@\textbf{Oper}, \emph{Oper}|pwk}
                  statt. Schnitzler nahm zwei Tage zuvor an
                    der Generalprobe\eventindex{Oper@\textbf{Oper}!Generalprobe von Die Frau ohne Schatten, 8.10.1919@Generalprobe von Die Frau ohne Schatten, 8.10.1919|pwkv} teil.}}}\label{K_L02326-2}, dann hoffe ich Sie recht bald zu{ }ſehen. – Wie{ }ſchön
               wenn man nur{ }ſich wieder ein \uline{biſſerl} öfter{ }ſähe!\pend
           
\pstart
           Von Herzen Ihr{\\[\baselineskip]}\spacefill\mbox{Hugo.}\pend
           \leftskip=0em{}\selectlanguage{ngerman}\endnumbering\briefempfaengerindex{Schnitzler, Arthur@\textsc{Schnitzler, Arthur}!zzzHofmannsthal, Hugo von@\emph{von Hugo von Hofmannsthal}!1919-09-191@{19. 9. 1919}|)be}\mylabel{L02326h}  \newcommand{\dateiname}{L02326}\newcommand{\titel}{Hugo Hofmannsthal an Arthur Schnitzler, 19. 9. 1919}\newcommand{\editorInnen}{Martin Anton Müller und Gerd-Hermann Susen}%% latex-leseansicht-abspann.tex
%% Abspann für die Leseansicht.
%% Der Schalter \ifkorrekturansicht ist bereits durch den Vorspann gesetzt.

%% latex-abspann.tex
%% Gemeinsamer Abspann für Korrekturansicht und Leseansicht.
%% Setzt den Schalter \ifkorrekturansicht voraus (gesetzt in den
%% einbindenden Dateien latex-korrekturansicht-abspann.tex bzw.
%% latex-leseansicht-abspann.tex).
%% ---------------------------------------------------------------

\normalsize

% Das esempio-Environment wird nur in der Leseansicht benötigt
\ifkorrekturansicht\else
\newenvironment{esempio}[3]%
{
    \vspace{1.5ex}
    \rlap{\underline{#1}}
    \par
    \setlength{\parindent}{0cm}
    \nopagebreak
    \leftskip=#2cm
    \rightskip=#3cm
}
{
    \par
}
\fi

\doendnotes{C}
\bigskip
\vfill

\clearpage

\footnotesize

\ifkorrekturansicht
  \lohead{\textsc{register}}
\fi

% theindex-Environment neu definieren ohne reledmac
\makeatletter
\renewenvironment{theindex}{%
  \ifkorrekturansicht
    \section*{\indexname}%
  \else
    \subsubsection*{Index der erwähnten Entitäten}%
  \fi
  \setlength{\parindent}{0pt}%
  \setlength{\parskip}{0pt plus 0.3pt}%
  \let\item\@idxitem
}{%
  \ifkorrekturansicht\clearpage\fi
}
\makeatother

\IfFileExists{\jobname-pw.ind}{\input{\jobname-pw.ind}}{}

% Quellenangabe nur in der Leseansicht
\ifkorrekturansicht\else
% Fallback-Definitionen, falls die .tex-Datei \titel etc. nicht gesetzt hat
\providecommand{\titel}{}
\providecommand{\editorInnen}{}
\providecommand{\dateiname}{\jobname}

\vspace{3cm}

\vfill

\footnotesize
\textsc{Quelle}: \titel. Herausgegeben von {\editorInnen}. In: \emph{Arthur Schnitzler: Briefwechsel mit Autorinnen und Autoren}.
 Digitale Edition, https://schnitzler-briefe.acdh.oeaw.ac.at/{\dateiname}.html (Stand \today)
\fi

\end{document}


