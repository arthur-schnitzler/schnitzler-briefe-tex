%% latex-korrekturansicht-vorspann.tex
%% Vorspann für die Korrekturansicht.
%% Lädt die gemeinsame Datei latex-vorspann.tex mit gesetztem Schalter.

\newif\ifkorrekturansicht
\korrekturansichttrue

\input{../tex-inputs/latex-vorspann}


\section[Hugo Hofmannsthal an Arthur Schnitzler, 19. 9. 1919]{L02326 Hugo Hofmannsthal an Arthur Schnitzler, 19. 9. 1919}
\nopagebreak\mylabel{L02326v}
\rehead{ }\normalsize\beginnumbering\briefempfaengerindex{Schnitzler, Arthur@\textsc{Schnitzler, Arthur}!zzzHofmannsthal, Hugo von@\emph{von Hugo von Hofmannsthal}!1919-09-191@{19. 9. 1919}|(be}
\toendnotes[C]{\smallbreak\pagebreak[2]}\Standort{CUL, Schnitzler, B 43.}
\physDesc{Brief, 1 Blatt, 2 Seiten, 2258 Zeichen
\newline{}Handschrift: schwarze Tinte, deutsche Kurrent
\newline{}Ordnung: 1) mit Bleistift von Frieda
                                    Pollak\pwindex{Pollak, Frieda 08.12.1881 – 13.07.1937@\textsc{Pollak, Frieda} (08.12.1881 – 13.07.1937), \emph{Sekretär/Sekretärin}|pw} (?) mit dem Buchstaben »A«
                                 (Abgeschrieben/Abschrift) gekennzeichnet  2) mit Bleistift von unbekannter Hand nummeriert: »\strikeout{355}« 3) mit Bleistift von unbekannter Hand nummeriert:
                                    »382«}
\buchAbdrucke{\weitereDrucke{Hugo von Hofmannsthal, Arthur Schnitzler: \emph{Briefwechsel}. Frankfurt am Main: \emph{S. Fischer} 1964, S. 284.} }\toendnotes[C]{\smallbreak}
\pstart
           {\pb}Bad Auſſee\oindex{Bad Aussee@\textbf{Bad Aussee}, \emph{P.PPLA3}|pw}, den
                  19. IX 19.\pend
           
\pstart{}mein lieber Arthur\pend\vspace{0.5em}
\pstart
           ſehr oft in dieſem Sommer ſind meine Gedanken zu Ihnen gegangen. In Ferleiten\oindex{Ferleiten@\textbf{Ferleiten}, \emph{P.PPL}|pw} im Juli, wenn ich
               herumging in dem ſtillen engen Thal das mir die Jahre meiner frühen Jugend ſo nahe
               bringt, fielen Sie mir ein als einer von denen, die ſchon damals meine Freunde waren
               und an die ich auf einem Holztisch in dem kleinen Tannenwald hinterm Gaſthaus – und
               der Holztiſch ſteht noch immer da – Briefe ſchrieb. Das ist ſiebenundzwanzig Jahre
               her, wie ſchwer faſslich! – Dann war ich dreimal in dieſem Sommer in Salzburg\oindex{Salzburg@\textbf{Salzburg}, \emph{A.ADM2}|pw} u. nie bin ich durch den Mirabell-garten\oindex{Schloss Mirabell@\textbf{Schloss Mirabell}, \emph{Schloss (K.SLS)}|pw} gegangen, nie nach Hellbrunn\oindex{Hellbrunn@\textbf{Hellbrunn}, \emph{P.PPL}|pw} oder Leopoldskron\oindex{Salzburg-Leopoldskron@\textbf{Salzburg-Leopoldskron}, \emph{Teil eines besiedelten Ortes (A.BSOX)}|pw}, ohne ſo
               herzlich an Sie zu denken.\pend
           
\pstart
           Das letzte Mal, daſs ich Sie geſehen habe, das war bei der \label{K_L02326-1v}\edtext{Generalprobe\eventindex{Oper@\textbf{Oper}!Generalprobe von Palestrina, 27.2.1919@Generalprobe von Palestrina, 27.2.1919|pwv}}{\lemma{\textnormal{\emph{Generalprobe}}}\Cendnote{\textnormal{Siehe A. S.: \emph{Tagebuch}, 27. 2. 1919.
               }}}\label{K_L02326-1} der Oper »\textsc{Palestrina}\pwindex{Palestrina. Musikalische Legende in drei Akten@\emph{Palestrina. Musikalische Legende in drei Akten}|pw}« – da waren Sie ſo schwer bedrückt von dem was in der Welt vorging und ſich
               anzukündigen ſchien, ſo bemüht u. bekümmert ſah Ihr vertrautes inhaltsvolles Geſicht
               aus – ich wurde dann bald krank, da ſah ich ſehr oft Ihr Geſicht ſo vor mir. Meine
               Krankheit war tiefergehend als ſie im erſten Augenblick ſchien, vom erſten
                  April bis in den Juli hinein war ich ein kranker, veränderter
               Menſch – erſt in Ferleiten\oindex{Ferleiten@\textbf{Ferleiten}, \emph{P.PPL}|pw}, ganz ganz einſam,
               hab ich mich zu mir ſelber {\pb}zurückgefunden, und nach jedem ſolchen Zurückfinden iſt man ja vielleicht ein
               ſtärkerer Menſch als je zuvor, man iſt halt um eine Windung der Schraube höher geko{\geminationm}en. –\hspace*{1.5em}So muſs ich mich
               glücklich nennen ſeit Ende Juli, es iſt eine Productivität über mich
                  geko{\geminationm}en wie ich ſie viele Jahre – es waren halt zu
               ſchwere Jahre – nicht gekannt habe, es ſind Arbeiten fertig geworden, andere in mir
               aufgewacht, noch andere ſtark vorwärts geko{\geminationm}en – ich
               glaube es iſt einiges darunter, dem Sie Ihren Beifall geben werden, der mir immer ſo
               warm u. vertraut und von Grund aus woltuend iſt.\pend
           
\pstart
           So ſtark iſt dieſes Zuſtrömen von Einfällen und ſo ſicher endlich einmal – Sie kennen
               meine bizarre ſchwierige Natur – die rhytmiſche Wiederkehr productiver Stunden, daſs
               ich Strauss\pwindex{Strauss, Richard 11.06.1864 – 08.09.1949@\textsc{Strauss, Richard} (11.06.1864 – 08.09.1949), \emph{Theaterleiter/Theaterleiterin, Komponist/Komponistin, Dirigent/Dirigentin}|pw} u. Schalk\pwindex{Schalk, Franz 27.05.1863 – 03.09.1931@\textsc{Schalk, Franz} (27.05.1863 – 03.09.1931), \emph{Theaterleiter/Theaterleiterin, Dirigent/Dirigentin}|pw} gebeten habe, mich bei den Proben der »Frau ohne Schatten\pwindex{Frau ohne Schatten. Erzaehlung@\emph{Die Frau ohne Schatten. Erzählung}|pw}« zu entſchuldigen – ich bin ja
               dort ohnedies nur das fünfte Rad am Wagen – ſo komme ich erſt knapp vor der \label{K_L02326-2v}\edtext{Première\eventindex{Oper@\textbf{Oper}!Urauffuehrung vom Die Frau ohne Schatten, 10.10.1919@Uraufführung vom Die Frau ohne Schatten, 10.10.1919|pwv}\pwindex{Frau ohne Schatten. Erzaehlung@\emph{Die Frau ohne Schatten. Erzählung}|pwv}}{\lemma{\textnormal{\emph{Première}}}\Cendnote{\textnormal{Die Uraufführung\eventindex{Oper@\textbf{Oper}!Urauffuehrung vom Die Frau ohne Schatten, 10.10.1919@Uraufführung vom Die Frau ohne Schatten, 10.10.1919|pwkv} fand am
                     10. 10. 1919 in der Wiener Oper\oindex{Oper@\textbf{Oper}, \emph{Oper (K.OPR)}|pwk}
                  statt. Schnitzler nahm zwei Tage zuvor an
                    der Generalprobe\eventindex{Oper@\textbf{Oper}!Generalprobe von Die Frau ohne Schatten, 8.10.1919@Generalprobe von Die Frau ohne Schatten, 8.10.1919|pwkv} teil.}}}\label{K_L02326-2}, dann hoffe ich Sie recht bald zu ſehen. – Wie ſchön
               wenn man nur ſich wieder ein \uline{biſſerl} öfter ſähe! \pend
           
\pstart
           Von Herzen Ihr{\\[\baselineskip]}\spacefill\mbox{Hugo.}\pend
           \leftskip=0em{}\selectlanguage{ngerman}\endnumbering\briefempfaengerindex{Schnitzler, Arthur@\textsc{Schnitzler, Arthur}!zzzHofmannsthal, Hugo von@\emph{von Hugo von Hofmannsthal}!1919-09-191@{19. 9. 1919}|)be}\mylabel{L02326h}  \normalsize

\doendnotes{C}
\bigskip
\vfill

\clearpage

\footnotesize

\lohead{\textsc{register}}

% Definiere theindex-Environment komplett neu ohne reledmac
\makeatletter
\renewenvironment{theindex}{%
  \section*{\indexname}%
  \setlength{\parindent}{0pt}%
  \setlength{\parskip}{0pt plus 0.3pt}%
  \let\item\@idxitem
}{%
  \clearpage
}
\makeatother

\IfFileExists{\jobname-pw.ind}{\input{\jobname-pw.ind}}{}

\end{document}

      