%% latex-leseansicht-vorspann.tex
%% Vorspann für die Leseansicht.
%% Lädt die gemeinsame Datei latex-vorspann.tex mit nicht gesetztem Schalter.

\newif\ifkorrekturansicht
\korrekturansichtfalse

\input{../tex-inputs/latex-vorspann}


         
         \newcommand{\erwaehntePersonen}{Personen: Frieda Pollak, Franz Schalk, Richard Strauss}
         \newcommand{\erwaehnteOrte}{Orte: Bad Aussee, Ferleiten, Hellbrunn, Mirabell, Oper, Salzburg, Salzburg-Leopoldskron, Wien}
         \newcommand{\erwaehnteWerke}{Werke: Die Frau ohne Schatten. Erzählung, Palestrina. Musikalische Legende in drei Akten}
               \section[Hugo Hofmannsthal an Arthur Schnitzler, 19. 9. 1919]{ Hugo Hofmannsthal an Arthur Schnitzler, 19. 9. 1919}\nopagebreak\mylabel{v}\rehead{ }\begin{ledgroupsized}[t]{13cm}\normalsize\beginnumbering \toendnotes[C]{\smallbreak\pagebreak[2]} \Standort{CUL, Schnitzler, B 43.}
\physDesc{Brief, 1 Blatt, 2 Seiten
\newline{}Handschrift: schwarze Tinte, deutsche Kurrent\newline{}Ordnung: 1) mit Bleistift von Frieda Pollak\pwindex{Pollak, Frieda 08.12.1881 – 13.07.1937@\textsc{Pollak, Frieda} (08.12.1881 – 13.07.1937), \emph{Sekretärin}|pw} (?) mit dem Buchstaben »A« (Abgeschrieben/Abschrift) gekennzeichnet  2) mit Bleistift von unbekannter Hand nummeriert: »\strikeout{355}« 3) mit Bleistift von unbekannter Hand nummeriert: »382«}\buchAbdrucke{\weitereDrucke{Hugo von Hofmannsthal, Arthur Schnitzler: \emph{Briefwechsel}. Hg. Therese Nickl und Heinrich Schnitzler. Frankfurt am Main: \emph{S. Fischer} 1964, S. 284.} }\toendnotes[C]{\smallbreak}\pstart
           {\pb}Bad Auſſee\oindex{Bad Aussee@\textbf{Bad Aussee}|pw}, den 19. IX 19.\pend
           \pstart{}mein lieber Arthur\pend\pstart
           ſehr oft in dieſem Sommer ſind meine Gedanken zu Ihnen gegangen. In Ferleiten\oindex{Ferleiten@\textbf{Ferleiten}|pw} im Juli, wenn ich herumging in dem
               ſtillen engen Thal das mir die Jahre meiner frühen Jugend ſo nahe bringt, fielen Sie
               mir ein als einer von denen, die ſchon damals meine Freunde waren und an die ich auf
               einem Holztisch in dem kleinen Tannenwald hinterm Gaſthaus – und der Holztiſch ſteht
               noch immer da – Briefe ſchrieb. Das ist ſiebenundzwanzig Jahre her, wie ſchwer
               faſslich! – Dann war ich dreimal in dieſem Sommer in Salzburg\oindex{Salzburg@\textbf{Salzburg}|pw} u. nie bin ich durch den Mirabell-garten\oindex{Mirabell@\textbf{Mirabell}|pw} gegangen, nie nach Hellbrunn\oindex{Hellbrunn@\textbf{Hellbrunn}|pw}
               oder Leopoldskron\oindex{Salzburg-Leopoldskron@\textbf{Salzburg-Leopoldskron}|pw}, ohne ſo herzlich an Sie zu
               denken.\pend
           \pstart
           Das letzte Mal, daſs ich Sie geſehen habe, das war bei der \label{K_L02326_1v}\edtext{Generalprobe}{\lemma{\textnormal{\emph{Generalprobe}}}\Cendnote{\textnormal{siehe A. S.: \emph{Tagebuch}, 27. 2. 1919}}}\label{K_L02326_1h} der Oper »\textsc{Palestrina}\pwindex{\textcolor{red}{\textsuperscript{XXXX1 indx}}!Palestrina. Musikalische Legende in drei Akten1912@\strich\emph{Palestrina. Musikalische Legende in drei Akten} {[}1912{]}|pw}« – da waren Sie ſo schwer bedrückt von dem was in der Welt vorging und ſich
               anzukündigen ſchien, ſo bemüht u. bekümmert ſah Ihr vertrautes inhaltsvolles Geſicht
               aus – ich wurde dann bald krank, da ſah ich ſehr oft Ihr Geſicht ſo vor mir. Meine
               Krankheit war tiefergehend als ſie im erſten Augenblick ſchien, vom erſten
                  April bis in den Juli hinein war ich ein kranker, veränderter
               Menſch – erſt in Ferleiten\oindex{Ferleiten@\textbf{Ferleiten}|pw}, ganz ganz einſam, hab
               ich mich zu mir ſelber {\pb}zurückgefunden, und nach jedem ſolchen Zurückfinden iſt man ja vielleicht ein
               ſtärkerer Menſch als je zuvor, man iſt halt um eine Windung der Schraube höher geko{\geminationm}en. –\hspace*{1.5em}So muſs ich mich
               glücklich nennen ſeit Ende Juli, es iſt eine Productivität über mich
                  geko{\geminationm}en wie ich ſie viele Jahre – es waren halt zu
               ſchwere Jahre – nicht gekannt habe, es ſind Arbeiten fertig geworden, andere in mir
               aufgewacht, noch andere ſtark vorwärts geko{\geminationm}en – ich
               glaube es iſt einiges darunter, dem Sie Ihren Beifall geben werden, der mir immer ſo
               warm u. vertraut und von Grund aus woltuend iſt.\pend
           \pstart
           So ſtark iſt dieſes Zuſtrömen von Einfällen und ſo ſicher endlich einmal – Sie kennen
               meine bizarre ſchwierige Natur – die rhytmiſche Wiederkehr productiver Stunden, daſs
               ich Strauss\pwindex{Strauss, Richard 11.06.1864 – 08.09.1949@\textsc{Strauss, Richard} (11.06.1864 – 08.09.1949), \emph{Theaterleiter, Komponist, Dirigent}|pw} u. Schalk\pwindex{Schalk, Franz 27.05.1863 – 03.09.1931@\textsc{Schalk, Franz} (27.05.1863 – 03.09.1931), \emph{Theaterleiter, Dirigent}|pw} gebeten habe, mich bei den Proben der »Frau ohne Schatten\pwindex{Hofmannsthal, Hugo von 1874-02-01 – 1929-07-15@\textsc{Hofmannsthal, Hugo von} (1874-02-01 – 1929-07-15), \emph{Schriftsteller}!Frau ohne Schatten. Erzaehlung1919@\strich\emph{Die Frau ohne Schatten. Erzählung} {[}1919{]}|pw}« zu entſchuldigen – ich bin ja dort ohnedies nur das
               fünfte Rad am Wagen – ſo komme ich erſt knapp vor der \label{K_L02326_2v}\edtext{Première\pwindex{Hofmannsthal, Hugo von 1874-02-01 – 1929-07-15@\textsc{Hofmannsthal, Hugo von} (1874-02-01 – 1929-07-15), \emph{Schriftsteller}!Frau ohne Schatten. Erzaehlung1919@\strich\emph{Die Frau ohne Schatten. Erzählung} {[}1919{]}|pwv}}{\lemma{\textnormal{\emph{Première}}}\Cendnote{\textnormal{Die Uraufführung fand am
                     10. 10. 1919 in der Wiener Oper\oindex{Oper@\textbf{Oper}|pwk}
                  statt. Schnitzler\pwindex{Schnitzler, Arthur 15.05.1862 – 21.10.1931@\textsc{Schnitzler, Arthur} (15.05.1862 – 21.10.1931), \emph{Schriftsteller, Mediziner}|pwk} nahm zwei Tage zuvor an der
                  Generalprobe teil.}}}\label{K_L02326_2h}, dann hoffe ich Sie recht bald zu ſehen. – Wie ſchön
               wenn man nur ſich wieder ein \uline{biſſerl} öfter ſähe! \pend
           \pstart
           Von Herzen Ihr{\\[\baselineskip]}\spacefill\mbox{Hugo.}\pend
           \leftskip=0em{}
         
         \endnumbering\mylabel{h}\end{ledgroupsized}  \newcommand{\dateiname}{L02326}\newcommand{\titel}{Hugo Hofmannsthal an Arthur Schnitzler, 19. 9. 1919}\newcommand{\editorInnen}{Martin Anton Müller und Gerd-Hermann Susen}%% latex-leseansicht-abspann.tex
%% Abspann für die Leseansicht.
%% Der Schalter \ifkorrekturansicht ist bereits durch den Vorspann gesetzt.

%% latex-abspann.tex
%% Gemeinsamer Abspann für Korrekturansicht und Leseansicht.
%% Setzt den Schalter \ifkorrekturansicht voraus (gesetzt in den
%% einbindenden Dateien latex-korrekturansicht-abspann.tex bzw.
%% latex-leseansicht-abspann.tex).
%% ---------------------------------------------------------------

\normalsize

% Das esempio-Environment wird nur in der Leseansicht benötigt
\ifkorrekturansicht\else
\newenvironment{esempio}[3]%
{
    \vspace{1.5ex}
    \rlap{\underline{#1}}
    \par
    \setlength{\parindent}{0cm}
    \nopagebreak
    \leftskip=#2cm
    \rightskip=#3cm
}
{
    \par
}
\fi

\doendnotes{C}
\bigskip
\vfill

\clearpage

\footnotesize

\ifkorrekturansicht
  \lohead{\textsc{register}}
\fi

% theindex-Environment neu definieren ohne reledmac
\makeatletter
\renewenvironment{theindex}{%
  \ifkorrekturansicht
    \section*{\indexname}%
  \else
    \subsubsection*{Index der erwähnten Entitäten}%
  \fi
  \setlength{\parindent}{0pt}%
  \setlength{\parskip}{0pt plus 0.3pt}%
  \let\item\@idxitem
}{%
  \ifkorrekturansicht\clearpage\fi
}
\makeatother

\IfFileExists{\jobname-pw.ind}{\input{\jobname-pw.ind}}{}

% Quellenangabe nur in der Leseansicht
\ifkorrekturansicht\else
% Fallback-Definitionen, falls die .tex-Datei \titel etc. nicht gesetzt hat
\providecommand{\titel}{}
\providecommand{\editorInnen}{}
\providecommand{\dateiname}{\jobname}

\vspace{3cm}

\vfill

\footnotesize
\textsc{Quelle}: \titel. Herausgegeben von {\editorInnen}. In: \emph{Arthur Schnitzler: Briefwechsel mit Autorinnen und Autoren}.
 Digitale Edition, https://schnitzler-briefe.acdh.oeaw.ac.at/{\dateiname}.html (Stand \today)
\fi

\end{document}


      