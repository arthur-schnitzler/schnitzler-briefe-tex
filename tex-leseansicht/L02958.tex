%% latex-leseansicht-vorspann.tex
%% Vorspann für die Leseansicht.
%% Lädt die gemeinsame Datei latex-vorspann.tex mit nicht gesetztem Schalter.

\newif\ifkorrekturansicht
\korrekturansichtfalse

\input{../tex-inputs/latex-vorspann}

\begin{center}
            \textcolor{red}{ENTWURF, NICHT FERTIG KORRIGIERT}
                      \end{center}
            
         
         \newcommand{\erwaehntePersonen}{Personen: Richard Beer-Hofmann, Paul Horn, Josef Jarno, Felix Salten, Richard Specht, Ignaz Wild, Grethe Wreden}
         \newcommand{\erwaehnteInstitutionen}{}
         \newcommand{\erwaehnteOrte}{Orte: Anzenau, Bad Aussee, Bad Ischl, Hotel und Pension Rudolfshöhe (Leopold Petter), Salzburg, Strobl, Wien}
         \newcommand{\erwaehnteWerke}{Werke: Abschiedssouper, Das Märchen. Schauspiel in drei Aufzügen, Die Frage an das Schicksal}
               \section[Arthur Schnitzler an Felix Salten, 5. 7. 1893]{ Arthur Schnitzler an Felix Salten, 5. 7. 1893}\nopagebreak\mylabel{v}\rehead{ }\begin{ledgroupsized}[t]{13cm}\normalsize\beginnumbering \toendnotes[C]{\smallbreak\pagebreak[2]} \Standort{Wienbibliothek im Rathaus, ZPH 1681, 2.1.516.}
\physDesc{
\newline{}Handschrift: , deutsche Kurrent}\toendnotes[C]{\smallbreak}\pstart
           \raggedleft{}{\pb}\textsc{Pension Leopold\oindex{Hotel und Pension Rudolfshoehe (Leopold Petter)@\textbf{Hotel und Pension Rudolfshöhe (Leopold Petter)}|pw}}, 5/7 93\pend
           \pstart{}Mein lieber Salten,\pend\pstart
           das wichtigſte zuerſt: geſtern \textsc{per}\textsc{Bic.} in \textsc{Strobl\oindex{Strobl@\textbf{Strobl}|pw}}, heut in\textsc{Aussee\oindex{Bad Aussee@\textbf{Bad Aussee}|pw}} geweſen – geht im ganzen recht gut. Leider i{\geminationm}er allein. \textsc{Richard\pwindex{Beer-Hofmann, Richard 1866-07-11 – 1945-09-26@\textsc{Beer-Hofmann, Richard} (1866-07-11 – 1945-09-26), \emph{Schriftsteller}|pw}} ko{\geminationm}t nach (wie geſtern) oder auch nicht (wie heute.) – Geſchreibe noch nichts;
               und {\pb}Heute früh, einſam, in \textsc{Anzenau\oindex{Anzenau@\textbf{Anzenau}|pw}}, die Verſe meines allegor Gedichtes\textcolor{red}{\textsuperscript{\textbf{KEY}}} in Ihrem Sinne
               in regelmäßige Jamben übertragen.– – Meine Sti{\geminationm}ung recht ſchlecht. Leer, traurig.–
               Heut hab ich ſogar geweint – in \textsc{Anzenau\oindex{Anzenau@\textbf{Anzenau}|pw}}! – Außerdem hab ich durch den ſonderbarſten der Zufälle auch noch neue Dinge
               erfahren – {\pb}aus \textsc{Salzb.\oindex{Salzburg@\textbf{Salzburg}|pw}} – Alſo eigentlich ſehr alte Dinge – O Menſch, ahnen Sie etwa, wie geſcheidt ich
               war, als ich das Märchen\pwindex{Schnitzler, Arthur 15.05.1862 – 21.10.1931@\textsc{Schnitzler, Arthur} (15.05.1862 – 21.10.1931), \emph{Schriftsteller, Mediziner}!Maerchen. Schauspiel in drei Aufzuegen1893-12-01@\strich\emph{Das Märchen. Schauspiel in drei Aufzügen} {[}1893-12-01{]}|pw} ſchrieb? – Bitte, fragen Sie
               noch nichts in einem eventuellen Brief, den Sie mir ſchreiben – ich wäre nervös, we{\geminationn}
               ich es verraten müßte.– – \textsc{Jarno\pwindex{Jarno, Josef 24.08.1865 – 11.01.1932@\textsc{Jarno, Josef} (24.08.1865 – 11.01.1932), \emph{Theaterleiter, Schauspieler}|pw}} hab ich geſprochen; {\pb}Der
               hatte natürlich mein Stück\textcolor{red}{\textsuperscript{\textbf{KEY}}} überhaupt noch nicht geleſen;
               iſt ein Komödiant, aber nebſtbei ein geſcheidtes ungariſcher Jud u wahrſcheinlich ein
               großes Talent,– Jetzt iſt er vom Abſchiedsſouper\pwindex{Schnitzler, Arthur 15.05.1862 – 21.10.1931@\textsc{Schnitzler, Arthur} (15.05.1862 – 21.10.1931), \emph{Schriftsteller, Mediziner}!Abschiedssouper1892@\strich\emph{Abschiedssouper} {[}1892{]}|pw} ſehr
               entzückt, und \textsc{Wild\pwindex{Wild, Ignaz 13.05.1849 – 19.10.1909@\textsc{Wild, Ignaz} (13.05.1849 – 19.10.1909), \emph{Theaterleiter, Schauspieler}|pw}} (der Direktor) führt am Montag{\pb}»Frage\pwindex{Schnitzler, Arthur 15.05.1862 – 21.10.1931@\textsc{Schnitzler, Arthur} (15.05.1862 – 21.10.1931), \emph{Schriftsteller, Mediziner}!Frage an das Schicksal01. 05. 1890@\strich\emph{Die Frage an das Schicksal} {[}01. 05. 1890{]}|pw}« u »Abſchiedſouper\pwindex{Schnitzler, Arthur 15.05.1862 – 21.10.1931@\textsc{Schnitzler, Arthur} (15.05.1862 – 21.10.1931), \emph{Schriftsteller, Mediziner}!Abschiedssouper1892@\strich\emph{Abschiedssouper} {[}1892{]}|pw}« auf, ohne ſie geleſen
               zu haben, oh nicht wegen \textsc{Jarno\pwindex{Jarno, Josef 24.08.1865 – 11.01.1932@\textsc{Jarno, Josef} (24.08.1865 – 11.01.1932), \emph{Theaterleiter, Schauspieler}|pw}}, ſondern weil er ſich denkt, daſs mein Name (oh nicht als Dichter!!) ihm das
               Haus füllt.– – Sagen Sie’s aber noch niemandem. We{\geminationn} es ſicher iſt, aviſire ich Sie –
               Wo iſt Paul Horn\pwindex{Horn, Paul 13.02.1867 – 18.01.1936@\textsc{Horn, Paul} (13.02.1867 – 18.01.1936), \emph{Fabrikant}|pw}? Vielleicht {\pb}gibt »ſeine« Grethe\pwindex{Wreden, Grethe @\textsc{Wreden, Grethe}, \emph{Schauspielerin}|pw} die Cora\pwindex{Schnitzler, Arthur 15.05.1862 – 21.10.1931@\textsc{Schnitzler, Arthur} (15.05.1862 – 21.10.1931), \emph{Schriftsteller, Mediziner}!Frage an das Schicksal01. 05. 1890@\strich\emph{Die Frage an das Schicksal} {[}01. 05. 1890{]}|pwv}.– Wann ko{\geminationm}t \textsc{Richard Specht\pwindex{Specht, Richard 07.12.1870 – 18.03.1932@\textsc{Specht, Richard} (07.12.1870 – 18.03.1932), \emph{Schriftsteller, Journalist, Kritiker}|pw}}?– Einmal will ich mit \textsc{Rich. BHof\pwindex{Beer-Hofmann, Richard 1866-07-11 – 1945-09-26@\textsc{Beer-Hofmann, Richard} (1866-07-11 – 1945-09-26), \emph{Schriftsteller}|pw}} nach \textsc{Salzburg\oindex{Salzburg@\textbf{Salzburg}|pw}} mittells der neuen Bahn.– – Seien Sie ſo gut und ſchreiben Sie ſofort.– \pend
           \pstart
           Herzlich der Ihre {\\[\baselineskip]}\spacefill\mbox{Arthur}\pend
           \leftskip=0em{}
         
         \endnumbering\mylabel{h}\end{ledgroupsized}\begin{anhang}\end{anhang}\newcommand{\dateiname}{L02958}\newcommand{\titel}{Arthur Schnitzler an Felix Salten, 5. 7. 1893}\newcommand{\editorInnen}{Martin Anton Müller und Laura Untner}%% latex-leseansicht-abspann.tex
%% Abspann für die Leseansicht.
%% Der Schalter \ifkorrekturansicht ist bereits durch den Vorspann gesetzt.

%% latex-abspann.tex
%% Gemeinsamer Abspann für Korrekturansicht und Leseansicht.
%% Setzt den Schalter \ifkorrekturansicht voraus (gesetzt in den
%% einbindenden Dateien latex-korrekturansicht-abspann.tex bzw.
%% latex-leseansicht-abspann.tex).
%% ---------------------------------------------------------------

\normalsize

% Das esempio-Environment wird nur in der Leseansicht benötigt
\ifkorrekturansicht\else
\newenvironment{esempio}[3]%
{
    \vspace{1.5ex}
    \rlap{\underline{#1}}
    \par
    \setlength{\parindent}{0cm}
    \nopagebreak
    \leftskip=#2cm
    \rightskip=#3cm
}
{
    \par
}
\fi

\doendnotes{C}
\bigskip
\vfill

\clearpage

\footnotesize

\ifkorrekturansicht
  \lohead{\textsc{register}}
\fi

% theindex-Environment neu definieren ohne reledmac
\makeatletter
\renewenvironment{theindex}{%
  \ifkorrekturansicht
    \section*{\indexname}%
  \else
    \subsubsection*{Index der erwähnten Entitäten}%
  \fi
  \setlength{\parindent}{0pt}%
  \setlength{\parskip}{0pt plus 0.3pt}%
  \let\item\@idxitem
}{%
  \ifkorrekturansicht\clearpage\fi
}
\makeatother

\IfFileExists{\jobname-pw.ind}{\input{\jobname-pw.ind}}{}

% Quellenangabe nur in der Leseansicht
\ifkorrekturansicht\else
% Fallback-Definitionen, falls die .tex-Datei \titel etc. nicht gesetzt hat
\providecommand{\titel}{}
\providecommand{\editorInnen}{}
\providecommand{\dateiname}{\jobname}

\vspace{3cm}

\vfill

\footnotesize
\textsc{Quelle}: \titel. Herausgegeben von {\editorInnen}. In: \emph{Arthur Schnitzler: Briefwechsel mit Autorinnen und Autoren}.
 Digitale Edition, https://schnitzler-briefe.acdh.oeaw.ac.at/{\dateiname}.html (Stand \today)
\fi

\end{document}


      