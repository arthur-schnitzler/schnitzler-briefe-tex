%% latex-leseansicht-vorspann.tex
%% Vorspann für die Leseansicht.
%% Lädt die gemeinsame Datei latex-vorspann.tex mit nicht gesetztem Schalter.

\newif\ifkorrekturansicht
\korrekturansichtfalse

\input{../tex-inputs/latex-vorspann}


\section[Arthur Schnitzler an Felix Salten, 5. 7. 1893]{L02958 Arthur Schnitzler an Felix Salten, 5. 7. 1893}
\nopagebreak\mylabel{L02958v}
\rehead{ }\normalsize\beginnumbering\briefempfaengerindex{Salten, Felix@\textsc{Salten, Felix}!zzzSchnitzler, Arthur@\emph{von Arthur Schnitzler}!1893-07-052@{5. 7. 1893}|(be}
\toendnotes[C]{\smallbreak\pagebreak[2]}
\correspDesc{Versand  durch Arthur Schnitzler am 5. 7. 1893 in Bad Ischl
\newline{}Erhalt  durch Felix Salten am 6. 7. 1893 in Wien}\toendnotes[C]{\smallbreak}
\Standort{Wienbibliothek im Rathaus, ZPH 1681, 2.1.516.}
\physDesc{Brief, 2 Blätter, 6 Seiten, 1501 Zeichen (Briefpapier mit Trauerrand)
\newline{}Handschrift: schwarze Tinte, deutsche Kurrent
\newline{}Ordnung: mit Bleistift von unbekannter Hand Nummerierung der Doppelseiten des
                                 Konvoluts: »81«–»83« }
\buchAbdrucke{\weitereDrucke{Arthur Schnitzler: \emph{Briefe 1875–1912}. Herausgegeben von Therese Nickl und Heinrich Schnitzler. Frankfurt am Main: \emph{S. Fischer} 1981, S. 209–210.} }\toendnotes[C]{\smallbreak}
\pstart
           \raggedleft{}{\pb}\textsc{Pension Leopold\oindex{Hotel und Pension Rudolfshöhe (Leopold Petter)@\textbf{Hotel und Pension Rudolfshöhe (Leopold Petter)}, \emph{Hotel}|pw}}, 5/7 93.\pend
           
\pstart{}Mein lieber Salten,\pend\vspace{0.5em}
\pstart
           das wichtigſte zuerſt: geſtern{ }\textsc{per}{ }\label{K_L02958-1v}\edtext{\textsc{Bic.}}{\lemma{\textnormal{\emph{Bic.}}}\Cendnote{\textnormal{Bicycle (Fahrrad). Zu den Ausflügen siehe A. S.: \emph{Tagebuch}, 4. 7. 1893 und 5. 7. 1893.
               }}}\label{K_L02958-1} in \textsc{Strobl\oindex{Strobl@\textbf{Strobl}, \emph{Verwaltungsgebiet}|pw}}, heut in \textsc{Anzenau\oindex{Anzenau@\textbf{Anzenau}|pw}} geweſen – geht im ganzen recht gut. Leider i{\geminationm}er
               allein; \textsc{Richard\pwindex{Beer-Hofmann, Richard 11.\,7.\,1866 Wien – 26.\,9.\,1945 New York City@\textsc{Beer-Hofmann, Richard} (11.\,7.\,1866 Wien – 26.\,9.\,1945 New York City), \emph{Schriftsteller}|pw}} ko{\geminationm}t nach (wie geſtern) oder auch nicht (wie heute.) –
               Geſchrieben noch nichts; und {\pb}heute{ }früh, einſam, in \textsc{Anzenau\oindex{Anzenau@\textbf{Anzenau}|pw}}, die Verſe meines allegor
                  Gedichts\pwindex{Schnitzler, Arthur 15.\,5.\,1862 Wien – 21.\,10.\,1931 ebd.@\textsc{Schnitzler, Arthur} (15.\,5.\,1862 Wien – 21.\,10.\,1931 ebd.), \emph{Schriftsteller, Mediziner}!Artifex@\strich\emph{Artifex}|pwv} in Ihrem Sinne in regelmäßige Jamben übertragen. –\pend
           
\pstart
           – Meine Sti{\geminationm}ung recht{ }ſchlecht. Leer,
               traurig. – Heut hab ich{ }ſogar geweint – in \textsc{Anzenau\oindex{Anzenau@\textbf{Anzenau}|pw}}! – Außerdem hab ich durch den{ }ſonderbarſten der Zufälle auch noch \label{K_L02958-2v}\edtext{neue Dinge}{\lemma{\textnormal{\emph{neue Dinge}}}\Cendnote{\textnormal{Über den Aufenthalt von Marie Glümer\pwindex{Glümer, Marie 3.\,7.\,1867 Wien – 16.\,11.\,1925 München@\textsc{Glümer, Marie} (3.\,7.\,1867 Wien – 16.\,11.\,1925 München), \emph{Schauspielerin}|pwk} in Salzburg\oindex{Salzburg@\textbf{Salzburg}, \emph{Verwaltungsgebiet}|pwk}, wo sie eine
                  intime Beziehung mit Rudolf von
                     Cuny-Pierron\pwindex{Cuny-Pierron, Rudolf Eduard von 1.\,1.\,1853 Wien – 15.\,7.\,1922 Gmunden@\textsc{Cuny-Pierron, Rudolf Eduard von} (1.\,1.\,1853 Wien – 15.\,7.\,1922 Gmunden), \emph{Kaufmann}|pwk} hatte, vgl. A. S.: \emph{Tagebuch}, 4. 7. 1893.}}}\label{K_L02958-2} erfahren – {\pb}aus
                  \textsc{Salzb.\oindex{Salzburg@\textbf{Salzburg}, \emph{Verwaltungsgebiet}|pw}} – alſo eigentlich{ }ſehr alte Dinge – O Menſch, ahnen Sie etwa, wie geſcheidt ich
               war, als ich das Märchen\pwindex{Schnitzler, Arthur 15.\,5.\,1862 Wien – 21.\,10.\,1931 ebd.@\textsc{Schnitzler, Arthur} (15.\,5.\,1862 Wien – 21.\,10.\,1931 ebd.), \emph{Schriftsteller, Mediziner}!Märchen. Schauspiel in drei Aufzügen@\strich\emph{Das Märchen. Schauspiel in drei Aufzügen}|pw}{ }ſchrieb? – Bitte,
               fragen Sie noch nichts in einem eventuellen Brief, den Sie mir{ }ſchreiben – ich wäre
               nervös, we{\geminationn} ich es verraten müßte. –\pend
           
\pstart
           – \label{K_L02958-3v}\edtext{\textsc{Jarno\pwindex{Jarno, Josef 24.\,8.\,1865 Budapest – 11.\,1.\,1932 Wien@\textsc{Jarno, Josef} (24.\,8.\,1865 Budapest – 11.\,1.\,1932 Wien), \emph{Theaterleiter, Schauspieler}|pw}} hab ich geſprochen; {\pb}der hatte
               natürlich mein Stück\pwindex{Schnitzler, Arthur 15.\,5.\,1862 Wien – 21.\,10.\,1931 ebd.@\textsc{Schnitzler, Arthur} (15.\,5.\,1862 Wien – 21.\,10.\,1931 ebd.), \emph{Schriftsteller, Mediziner}!Anatol@\strich\emph{Anatol}|pwv} überhaupt
               noch nicht geleſen}{\lemma{\textnormal{\emph{Jarno … gelesen}}}\Cendnote{\textnormal{Siehe A. S.: \emph{Tagebuch}, 4. 7. 1893.
               }}}\label{K_L02958-3}; iſt ein Komödiant, aber nebſtbei ein geſcheidter ungar\oindex{Ungarn@\textbf{Ungarn}|pwv}iſcher Jud u wahrſcheinlich ein großes
               Talent. – Jetzt iſt er vom Abſchiedsſouper\pwindex{Schnitzler, Arthur 15.\,5.\,1862 Wien – 21.\,10.\,1931 ebd.@\textsc{Schnitzler, Arthur} (15.\,5.\,1862 Wien – 21.\,10.\,1931 ebd.), \emph{Schriftsteller, Mediziner}!Abschiedssouper@\strich\emph{Abschiedssouper}|pw}{ }ſehr entzückt, und \textsc{Wild\pwindex{Wild, Ignaz 13.\,5.\,1849 Třebíč – 19.\,10.\,1909 Wien@\textsc{Wild, Ignaz} (13.\,5.\,1849 Třebíč – 19.\,10.\,1909 Wien), \emph{Theaterleiter, Schauspieler}|pw}} (der Direktor\orgindex{Saisontheater Ischl@Saisontheater Ischl|pwv}) \label{K_L02958-4v}\edtext{führt am Montag{ }{\pb}»Frage\pwindex{Schnitzler, Arthur 15.\,5.\,1862 Wien – 21.\,10.\,1931 ebd.@\textsc{Schnitzler, Arthur} (15.\,5.\,1862 Wien – 21.\,10.\,1931 ebd.), \emph{Schriftsteller, Mediziner}!Frage an das Schicksal@\strich\emph{Die Frage an das Schicksal}|pw}«
               u »Abſchiedſouper\pwindex{Schnitzler, Arthur 15.\,5.\,1862 Wien – 21.\,10.\,1931 ebd.@\textsc{Schnitzler, Arthur} (15.\,5.\,1862 Wien – 21.\,10.\,1931 ebd.), \emph{Schriftsteller, Mediziner}!Abschiedssouper@\strich\emph{Abschiedssouper}|pw}« auf}{\lemma{\textnormal{\emph{führt … auf}}}\Cendnote{\textnormal{im Saisontheater\oindex{Lehártheater@\textbf{Lehártheater}, \emph{Theater}|pwk} in Bad Ischl\oindex{Bad Ischl@\textbf{Bad Ischl}|pwk} am 14. 7. 1893}}}\label{K_L02958-4}, ohne{ }ſie geleſen zu haben, oh nicht wegen \textsc{Jarno\pwindex{Jarno, Josef 24.\,8.\,1865 Budapest – 11.\,1.\,1932 Wien@\textsc{Jarno, Josef} (24.\,8.\,1865 Budapest – 11.\,1.\,1932 Wien), \emph{Theaterleiter, Schauspieler}|pw}},{ }ſondern weil er{ }ſich denkt, daſs mein Name (oh nicht als Dichter!!) ihm das
                  Haus\oindex{Lehártheater@\textbf{Lehártheater}, \emph{Theater}|pw} füllt. –\pend
           
\pstart
           – Sagen Sie’s aber noch niemandem. We{\geminationn} es{ }ſicher
               iſt, aviſire ich Sie – Wo iſt Paul Horn\pwindex{Horn, Paul 13.\,2.\,1867 Wien – 18.\,1.\,1936 Menton@\textsc{Horn, Paul} (13.\,2.\,1867 Wien – 18.\,1.\,1936 Menton), \emph{Fabrikant}|pw}?
               Vielleicht {\pb}gibt »ſeine« \label{K_L02958-5v}\edtext{Grethe\pwindex{Wreden, Grethe @\textsc{Wreden, Grethe}, \emph{Schauspielerin}|pw} die Cora\pwindex{Schnitzler, Arthur 15.\,5.\,1862 Wien – 21.\,10.\,1931 ebd.@\textsc{Schnitzler, Arthur} (15.\,5.\,1862 Wien – 21.\,10.\,1931 ebd.), \emph{Schriftsteller, Mediziner}!Frage an das Schicksal@\strich\emph{Die Frage an das Schicksal}|pwv}}{\lemma{\textnormal{\emph{Grethe die Cora}}}\Cendnote{\textnormal{Siehe XXXX Auszeichnungsfehler: Dokument L02959 nicht gefunden.
               }}}\label{K_L02958-5}. – Wann ko{\geminationm}t \textsc{Richard Specht\pwindex{Specht, Richard 7.\,12.\,1870 Wien – 18.\,3.\,1932 ebd.@\textsc{Specht, Richard} (7.\,12.\,1870 Wien – 18.\,3.\,1932 ebd.), \emph{Schriftsteller, Journalist, Kritiker}|pw}}? – Einmal will ich mit \textsc{Rich. BHof\pwindex{Beer-Hofmann, Richard 11.\,7.\,1866 Wien – 26.\,9.\,1945 New York City@\textsc{Beer-Hofmann, Richard} (11.\,7.\,1866 Wien – 26.\,9.\,1945 New York City), \emph{Schriftsteller}|pw}} nach \textsc{Salzburg\oindex{Salzburg@\textbf{Salzburg}, \emph{Verwaltungsgebiet}|pw}} mittells der \label{K_L02958-6v}\edtext{neuen Bahn\orgindex{Salzkammergut-Lokalbahn@Salzkammergut-Lokalbahn|pwv}}{\lemma{\textnormal{\emph{neuen Bahn}}}\Cendnote{\textnormal{Gemeint war die im Juni 1893 in
                  Betrieb genommene \emph{Salzkammergut-Lokalbahn}\orgindex{Salzkammergut-Lokalbahn@Salzkammergut-Lokalbahn|pwk}
                  zwischen Salzburg\oindex{Salzburg@\textbf{Salzburg}, \emph{Verwaltungsgebiet}|pwk} und Bad Ischl\oindex{Bad Ischl@\textbf{Bad Ischl}|pwk}.}}}\label{K_L02958-6}. –\pend
           
\pstart
           – Seien Sie{ }ſo gut und{ }ſchreiben Sie{ }ſofort. –\pend
           
\pstart
           Herzlich der Ihre {\\[\baselineskip]}\spacefill\mbox{Arthur}\pend
           \leftskip=0em{}\selectlanguage{ngerman}\endnumbering\briefempfaengerindex{Salten, Felix@\textsc{Salten, Felix}!zzzSchnitzler, Arthur@\emph{von Arthur Schnitzler}!1893-07-052@{5. 7. 1893}|)be}\mylabel{L02958h}  \newcommand{\dateiname}{L02958}\newcommand{\titel}{Arthur Schnitzler an Felix Salten, 5. 7. 1893}\newcommand{\editorInnen}{Martin Anton Müller und Laura Untner}%% latex-leseansicht-abspann.tex
%% Abspann für die Leseansicht.
%% Der Schalter \ifkorrekturansicht ist bereits durch den Vorspann gesetzt.

%% latex-abspann.tex
%% Gemeinsamer Abspann für Korrekturansicht und Leseansicht.
%% Setzt den Schalter \ifkorrekturansicht voraus (gesetzt in den
%% einbindenden Dateien latex-korrekturansicht-abspann.tex bzw.
%% latex-leseansicht-abspann.tex).
%% ---------------------------------------------------------------

\normalsize

% Das esempio-Environment wird nur in der Leseansicht benötigt
\ifkorrekturansicht\else
\newenvironment{esempio}[3]%
{
    \vspace{1.5ex}
    \rlap{\underline{#1}}
    \par
    \setlength{\parindent}{0cm}
    \nopagebreak
    \leftskip=#2cm
    \rightskip=#3cm
}
{
    \par
}
\fi

\doendnotes{C}
\bigskip
\vfill

\clearpage

\footnotesize

\ifkorrekturansicht
  \lohead{\textsc{register}}
\fi

% theindex-Environment neu definieren ohne reledmac
\makeatletter
\renewenvironment{theindex}{%
  \ifkorrekturansicht
    \section*{\indexname}%
  \else
    \subsubsection*{Index der erwähnten Entitäten}%
  \fi
  \setlength{\parindent}{0pt}%
  \setlength{\parskip}{0pt plus 0.3pt}%
  \let\item\@idxitem
}{%
  \ifkorrekturansicht\clearpage\fi
}
\makeatother

\IfFileExists{\jobname-pw.ind}{\input{\jobname-pw.ind}}{}

% Quellenangabe nur in der Leseansicht
\ifkorrekturansicht\else
% Fallback-Definitionen, falls die .tex-Datei \titel etc. nicht gesetzt hat
\providecommand{\titel}{}
\providecommand{\editorInnen}{}
\providecommand{\dateiname}{\jobname}

\vspace{3cm}

\vfill

\footnotesize
\textsc{Quelle}: \titel. Herausgegeben von {\editorInnen}. In: \emph{Arthur Schnitzler: Briefwechsel mit Autorinnen und Autoren}.
 Digitale Edition, https://schnitzler-briefe.acdh.oeaw.ac.at/{\dateiname}.html (Stand \today)
\fi

\end{document}


