%% latex-korrekturansicht-vorspann.tex
%% Vorspann für die Korrekturansicht.
%% Lädt die gemeinsame Datei latex-vorspann.tex mit gesetztem Schalter.

\newif\ifkorrekturansicht
\korrekturansichttrue

\input{../tex-inputs/latex-vorspann}


\section[Arthur Schnitzler an Felix Salten, 5. 7. 1893]{L02958 Arthur Schnitzler an Felix Salten, 5. 7. 1893}
\nopagebreak\mylabel{L02958v}
\rehead{ }\normalsize\beginnumbering\briefempfaengerindex{Salten, Felix@\textsc{Salten, Felix}!zzzSchnitzler, Arthur@\emph{von Arthur Schnitzler}!1893-07-052@{5. 7. 1893}|(be}
\toendnotes[C]{\smallbreak\pagebreak[2]}\Standort{Wienbibliothek im Rathaus, ZPH 1681, 2.1.516.}
\physDesc{Brief, 2 Blätter, 6 Seiten, 1501 Zeichen (Briefpapier mit Trauerrand)
\newline{}Handschrift: schwarze Tinte, deutsche Kurrent
\newline{}Ordnung: mit Bleistift von unbekannter Hand Nummerierung der Doppelseiten des
                                 Konvoluts: »81«–»83« }
\buchAbdrucke{\weitereDrucke{Arthur Schnitzler: \emph{Briefe 1875–1912}. Frankfurt am Main: \emph{S. Fischer} 1981, S. 209–210.} }\toendnotes[C]{\smallbreak}
\pstart
           \raggedleft{}{\pb}\textsc{Pension Leopold\oindex{Hotel und Pension Rudolfshoehe (Leopold Petter)@\textbf{Hotel und Pension Rudolfshöhe (Leopold Petter)}, \emph{Hotel (K.HTL)}|pw}}, 5/7 93. \pend
           
\pstart{}Mein lieber Salten,\pend\vspace{0.5em}
\pstart
           das wichtigſte zuerſt: geſtern{ }\textsc{per}{ }\label{K_L02958-1v}\edtext{\textsc{Bic.}}{\lemma{\textnormal{\emph{Bic.}}}\Cendnote{\textnormal{Bicycle (Fahrrad). Zu den Ausflügen siehe A. S.: \emph{Tagebuch}, 4. 7. 1893 und 5. 7. 1893.
               }}}\label{K_L02958-1} in \textsc{Strobl\oindex{Strobl@\textbf{Strobl}, \emph{A.ADM3}|pw}}, heut in \textsc{Anzenau\oindex{Anzenau@\textbf{Anzenau}, \emph{P.PPL}|pw}} geweſen – geht im ganzen recht gut. Leider i{\geminationm}er
               allein; \textsc{Richard\pwindex{Beer-Hofmann, Richard 1866-07-11 – 1945-09-26@\textsc{Beer-Hofmann, Richard} (1866-07-11 – 1945-09-26), \emph{Schriftsteller/Schriftstellerin}|pw}} ko{\geminationm}t nach (wie geſtern) oder auch nicht (wie heute.) –
               Geſchrieben noch nichts; und {\pb}heute{ }früh, einſam, in \textsc{Anzenau\oindex{Anzenau@\textbf{Anzenau}, \emph{P.PPL}|pw}}, die Verſe meines allegor
                  Gedichts\pwindex{Artifex@\emph{Artifex}|pwv} in Ihrem Sinne in regelmäßige Jamben übertragen. –\pend
           
\pstart
           – Meine Sti{\geminationm}ung recht ſchlecht. Leer,
               traurig. – Heut hab ich ſogar geweint – in \textsc{Anzenau\oindex{Anzenau@\textbf{Anzenau}, \emph{P.PPL}|pw}}! – Außerdem hab ich durch den ſonderbarſten der Zufälle auch noch \label{K_L02958-2v}\edtext{neue Dinge}{\lemma{\textnormal{\emph{neue Dinge}}}\Cendnote{\textnormal{Über den Aufenthalt von Marie Glümer\pwindex{Gluemer, Marie 03.07.1867 – 16.11.1925@\textsc{Glümer, Marie} (03.07.1867 – 16.11.1925), \emph{Schauspieler/Schauspielerin}|pwk} in Salzburg\oindex{Salzburg@\textbf{Salzburg}, \emph{A.ADM2}|pwk}, wo sie eine
                  intime Beziehung mit Rudolf von
                     Cuny-Pierron\pwindex{Cuny-Pierron, Rudolf Eduard von 01.01.1853 – 15.07.1922@\textsc{Cuny-Pierron, Rudolf Eduard von} (01.01.1853 – 15.07.1922), \emph{Kaufmann/Kauffrau}|pwk} hatte, vgl. A. S.: \emph{Tagebuch}, 4. 7. 1893.}}}\label{K_L02958-2} erfahren – {\pb}aus
                  \textsc{Salzb.\oindex{Salzburg@\textbf{Salzburg}, \emph{A.ADM2}|pw}} – alſo eigentlich ſehr alte Dinge – O Menſch, ahnen Sie etwa, wie geſcheidt ich
               war, als ich das Märchen\pwindex{Maerchen. Schauspiel in drei Aufzuegen@\emph{Das Märchen. Schauspiel in drei Aufzügen}|pw} ſchrieb? – Bitte,
               fragen Sie noch nichts in einem eventuellen Brief, den Sie mir ſchreiben – ich wäre
               nervös, we{\geminationn} ich es verraten müßte. –\pend
           
\pstart
           – \label{K_L02958-3v}\edtext{\textsc{Jarno\pwindex{Jarno, Josef 24.08.1865 – 11.01.1932@\textsc{Jarno, Josef} (24.08.1865 – 11.01.1932), \emph{Theaterleiter/Theaterleiterin, Schauspieler/Schauspielerin}|pw}} hab ich geſprochen; {\pb}der hatte
               natürlich mein Stück\pwindex{Anatol@\emph{Anatol}|pwv} überhaupt
               noch nicht geleſen}{\lemma{\textnormal{\emph{Jarno … geleſen}}}\Cendnote{\textnormal{Siehe A. S.: \emph{Tagebuch}, 4. 7. 1893.
               }}}\label{K_L02958-3}; iſt ein Komödiant, aber nebſtbei ein geſcheidter ungar\oindex{Ungarn@\textbf{Ungarn}, \emph{A.PCLI}|pwv}iſcher Jud u wahrſcheinlich ein großes
               Talent. – Jetzt iſt er vom Abſchiedsſouper\pwindex{Abschiedssouper@\emph{Abschiedssouper}|pw}
               ſehr entzückt, und \textsc{Wild\pwindex{Wild, Ignaz 13.05.1849 – 19.10.1909@\textsc{Wild, Ignaz} (13.05.1849 – 19.10.1909), \emph{Theaterleiter/Theaterleiterin, Schauspieler/Schauspielerin}|pw}} (der Direktor\orgindex{Saisontheater Ischl@Saisontheater Ischl|pwv}) \label{K_L02958-4v}\edtext{führt am Montag{ }{\pb}»Frage\pwindex{Frage an das Schicksal@\emph{Die Frage an das Schicksal}|pw}«
               u »Abſchiedſouper\pwindex{Abschiedssouper@\emph{Abschiedssouper}|pw}« auf}{\lemma{\textnormal{\emph{führt … auf}}}\Cendnote{\textnormal{im Saisontheater\oindex{Lehártheater@\textbf{Lehártheater}, \emph{Theater (K.THE)}|pwk} in Bad Ischl\oindex{Bad Ischl@\textbf{Bad Ischl}, \emph{P.PPL}|pwk} am 14. 7. 1893}}}\label{K_L02958-4}, ohne ſie geleſen zu haben, oh nicht wegen \textsc{Jarno\pwindex{Jarno, Josef 24.08.1865 – 11.01.1932@\textsc{Jarno, Josef} (24.08.1865 – 11.01.1932), \emph{Theaterleiter/Theaterleiterin, Schauspieler/Schauspielerin}|pw}}, ſondern weil er ſich denkt, daſs mein Name (oh nicht als Dichter!!) ihm das
                  Haus\oindex{Lehártheater@\textbf{Lehártheater}, \emph{Theater (K.THE)}|pw} füllt. –\pend
           
\pstart
           – Sagen Sie’s aber noch niemandem. We{\geminationn} es ſicher
               iſt, aviſire ich Sie – Wo iſt Paul Horn\pwindex{Horn, Paul 13.02.1867 – 18.01.1936@\textsc{Horn, Paul} (13.02.1867 – 18.01.1936), \emph{Fabrikant/Fabrikantin}|pw}?
               Vielleicht {\pb}gibt »ſeine« \label{K_L02958-5v}\edtext{Grethe\pwindex{Wreden, Grethe @\textsc{Wreden, Grethe}, \emph{Schauspieler/Schauspielerin}|pw} die Cora\pwindex{Frage an das Schicksal@\emph{Die Frage an das Schicksal}|pwv}}{\lemma{\textnormal{\emph{Grethe die Cora}}}\Cendnote{\textnormal{Siehe Arthur Schnitzler an Felix Salten, 9. 7. 1893.
               }}}\label{K_L02958-5}. – Wann ko{\geminationm}t \textsc{Richard Specht\pwindex{Specht, Richard 07.12.1870 – 18.03.1932@\textsc{Specht, Richard} (07.12.1870 – 18.03.1932), \emph{Schriftsteller/Schriftstellerin, Journalist/Journalistin, Kritiker/Kritikerin}|pw}}? – Einmal will ich mit \textsc{Rich. BHof\pwindex{Beer-Hofmann, Richard 1866-07-11 – 1945-09-26@\textsc{Beer-Hofmann, Richard} (1866-07-11 – 1945-09-26), \emph{Schriftsteller/Schriftstellerin}|pw}} nach \textsc{Salzburg\oindex{Salzburg@\textbf{Salzburg}, \emph{A.ADM2}|pw}} mittells der \label{K_L02958-6v}\edtext{neuen Bahn\orgindex{Salzkammergut-Lokalbahn@Salzkammergut-Lokalbahn|pwv}}{\lemma{\textnormal{\emph{neuen Bahn}}}\Cendnote{\textnormal{Gemeint war die im Juni 1893 in
                  Betrieb genommene \emph{Salzkammergut-Lokalbahn}\orgindex{Salzkammergut-Lokalbahn@Salzkammergut-Lokalbahn|pwk}
                  zwischen Salzburg\oindex{Salzburg@\textbf{Salzburg}, \emph{A.ADM2}|pwk} und Bad Ischl\oindex{Bad Ischl@\textbf{Bad Ischl}, \emph{P.PPL}|pwk}.}}}\label{K_L02958-6}. –\pend
           
\pstart
           – Seien Sie ſo gut und ſchreiben Sie ſofort. –\pend
           
\pstart
           Herzlich der Ihre {\\[\baselineskip]}\spacefill\mbox{Arthur}\pend
           \leftskip=0em{}\selectlanguage{ngerman}\endnumbering\briefempfaengerindex{Salten, Felix@\textsc{Salten, Felix}!zzzSchnitzler, Arthur@\emph{von Arthur Schnitzler}!1893-07-052@{5. 7. 1893}|)be}\mylabel{L02958h}  \normalsize

\doendnotes{C}
\bigskip
\vfill

\clearpage

\footnotesize

\lohead{\textsc{register}}

% Definiere theindex-Environment komplett neu ohne reledmac
\makeatletter
\renewenvironment{theindex}{%
  \section*{\indexname}%
  \setlength{\parindent}{0pt}%
  \setlength{\parskip}{0pt plus 0.3pt}%
  \let\item\@idxitem
}{%
  \clearpage
}
\makeatother

\IfFileExists{\jobname-pw.ind}{\input{\jobname-pw.ind}}{}

\end{document}

      