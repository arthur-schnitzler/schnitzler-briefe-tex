%% latex-leseansicht-vorspann.tex
%% Vorspann für die Leseansicht.
%% Lädt die gemeinsame Datei latex-vorspann.tex mit nicht gesetztem Schalter.

\newif\ifkorrekturansicht
\korrekturansichtfalse

\input{../tex-inputs/latex-vorspann}


\section[Arthur Schnitzler an Gustav Schwarzkopf, 12. 3. 1903]{L04145 Arthur Schnitzler an Gustav Schwarzkopf, 12. 3. 1903}
\nopagebreak\mylabel{L04145v}
\rehead{ }\normalsize\beginnumbering\briefempfaengerindex{Schwarzkopf, Gustav@\textsc{Schwarzkopf, Gustav}!zzzSchnitzler, Arthur@\emph{von Arthur Schnitzler}!1903-03-121@{12. 3. 1903}|(be}
\toendnotes[C]{\smallbreak\pagebreak[2]}
\correspDesc{Versand  durch Arthur Schnitzler am 12. 3. 1903 in Wien
\newline{}Erhalt  durch Gustav Schwarzkopf im Zeitraum [12. 3. 1903
                  – 15. 3. 1903?] in Wien}\toendnotes[C]{\smallbreak}
\Standort{CUL, Schnitzler, B 96.}
\physDesc{Brief, 1 Blatt, 1 Seite, 231 Zeichen
\newline{}Handschrift: schwarze Tinte, deutsche Kurrent}\toendnotes[C]{\smallbreak}
\pstart
           \raggedleft{}{\pb}12. 3. 903.\pend
           
\pstart{}lieber Guſtav,\pend\vspace{0.5em}
\pstart
           beifolgend der \label{K_L04145-1v}\edtext{ſtrenge Brief Alfred v Bergers\pwindex{Berger, Alfred von 30.\,4.\,1853 Wien – 24.\,8.\,1912 ebd.@\textsc{Berger, Alfred von} (30.\,4.\,1853 Wien – 24.\,8.\,1912 ebd.), \emph{Schriftsteller, Journalist, Theaterleiter}|pw} und das Manuscript\pwindex{Schwarzkopf, Max 12.\,6.\,1857 Wien – 14.\,4.\,1928 ebd.@\textsc{Schwarzkopf, Max} (12.\,6.\,1857 Wien – 14.\,4.\,1928 ebd.), \emph{Rechtsanwalt}!reine Tor. Gesellschaftsstück in vier Akten@\strich\emph{Der reine Tor. Gesellschaftsstück in vier Akten}|pwuv}}{\lemma{\textnormal{\emph{strenge … Manuscript}}}\Cendnote{\textnormal{Die Beilage ist nicht erhalten. Es dürfte
                  sich um das Manuskript von \emph{Der reine Tor}\pwindex{Schwarzkopf, Max 12.\,6.\,1857 Wien – 14.\,4.\,1928 ebd.@\textsc{Schwarzkopf, Max} (12.\,6.\,1857 Wien – 14.\,4.\,1928 ebd.), \emph{Rechtsanwalt}!reine Tor. Gesellschaftsstück in vier Akten@\strich\emph{Der reine Tor. Gesellschaftsstück in vier Akten}|pwk} von
                     Max Schwarzkopf\pwindex{Schwarzkopf, Max 12.\,6.\,1857 Wien – 14.\,4.\,1928 ebd.@\textsc{Schwarzkopf, Max} (12.\,6.\,1857 Wien – 14.\,4.\,1928 ebd.), \emph{Rechtsanwalt}|pwk} handeln. Am
                     14. 11. 1902 hatte Alfred von Berger\pwindex{Berger, Alfred von 30.\,4.\,1853 Wien – 24.\,8.\,1912 ebd.@\textsc{Berger, Alfred von} (30.\,4.\,1853 Wien – 24.\,8.\,1912 ebd.), \emph{Schriftsteller, Journalist, Theaterleiter}|pwk} in einem Brief an Schnitzler die Entscheidung über eine etwaige Annahme des Stückes durch
                  das \emph{Hamburger Schauspielhaus}\orgindex{Deutsches Schauspielhaus in Hamburg@Deutsches Schauspielhaus in Hamburg|pwk} hinausgezögert:
                        »Sehr geehrter Herr Doktor!{ / }Das mir von Ihnen übersandte Lustspiel ›Der
                           reine Thor\pwindex{Schwarzkopf, Max 12.\,6.\,1857 Wien – 14.\,4.\,1928 ebd.@\textsc{Schwarzkopf, Max} (12.\,6.\,1857 Wien – 14.\,4.\,1928 ebd.), \emph{Rechtsanwalt}!reine Tor. Gesellschaftsstück in vier Akten@\strich\emph{Der reine Tor. Gesellschaftsstück in vier Akten}|pw}‹ von einem Anonymus\pwindex{Schwarzkopf, Max 12.\,6.\,1857 Wien – 14.\,4.\,1928 ebd.@\textsc{Schwarzkopf, Max} (12.\,6.\,1857 Wien – 14.\,4.\,1928 ebd.), \emph{Rechtsanwalt}|pwv} habe ich sofort mit grösstem Interesse
                        gelesen. Es ist aber eine abermalige Lektüre des Stückes notwendig, bevor
                        ich über die Annahme schlüssig verden kann. Sie erhalten demnächst
                        diesbezügliche Nachricht.{ / }Mit den besten Grüssen{ / }hochachtungsvoll,{ / }Dr. Alfred v. Berger«. (\emph{Cambridge University Library}, Schnitzler, B 10.) Es liegt nahe, dass die fehlende Beilage nun das
                  zu 
                  Absageschreiben darstellte. }}}\label{K_L04145-1}.\introOben{}trafen \uline{heute} ein trotz des Datums 7/3. –\introOben{}\footnote{\noindent{}Das \label{K_L04145-2v}\toendnotes[C]{\begin{minipage}[t]{4em}{\makebox[3.6em][r]{\tiny{Fußnote}}}\end{minipage}\begin{minipage}[t]{\dimexpr\linewidth-4em}\textit{Mscpt behalt ich}\,{]} .\end{minipage}\par}Mscpt\pwindex{Schwarzkopf, Max 12.\,6.\,1857 Wien – 14.\,4.\,1928 ebd.@\textsc{Schwarzkopf, Max} (12.\,6.\,1857 Wien – 14.\,4.\,1928 ebd.), \emph{Rechtsanwalt}!reine Tor. Gesellschaftsstück in vier Akten@\strich\emph{Der reine Tor. Gesellschaftsstück in vier Akten}|pwuv} behalt ich\label{1} lieber und wir berathen neue
                     Angriffe.}\pend
           
\pstart
           Zur \label{K_L04145-3v}\edtext{Samſtag Première\eventindex{Volkstheater@\textbf{Volkstheater}!Premiere von Lebendige Stunden, 14.3.1903@Premiere von Lebendige Stunden, 14.3.1903|pwv}}{\lemma{\textnormal{\emph{Samstag Première}}}\Cendnote{\textnormal{ Die Theaterpremiere von \emph{Lebendige Stunden. Vier
                        Einakter}\pwindex{Schnitzler, Arthur 15. 5. 1862 Wien – 21. 10. 1931 ebd.@\textsc{Schnitzler, Arthur} (15. 5. 1862 Wien – 21. 10. 1931 ebd.), \emph{Schriftsteller, Mediziner}!Lebendige Stunden. Vier Einakter@\strich\emph{Lebendige Stunden. Vier Einakter}|pwk} von Arthur Schnitzler\eventindex{Volkstheater@\textbf{Volkstheater}!Premiere von Lebendige Stunden, 14.3.1903@Premiere von Lebendige Stunden, 14.3.1903|pwk} fand am 14. 3. 1903 am \emph{Volkstheater}\orgindex{Volkstheater@Volkstheater|pwk} im Volkstheater\oindex{Wien@\textbf{Wien}!VII., Neubau@\textbf{VII., Neubau}!Volkstheater@\textbf{Volkstheater}, \emph{Theater}|pwk} statt. }}}\label{K_L04145-3} erhalten Sie wenn es Ihnen nicht unangenehm
               iſt 2 Sitze geſchickt.\pend
           
\pstart
           Herzlichen Gruſs{\\[\baselineskip]} Ihr{\\[\baselineskip]}\spacefill\mbox{A.}\pend
           \leftskip=0em{}\selectlanguage{ngerman}\endnumbering\briefempfaengerindex{Schwarzkopf, Gustav@\textsc{Schwarzkopf, Gustav}!zzzSchnitzler, Arthur@\emph{von Arthur Schnitzler}!1903-03-121@{12. 3. 1903}|)be}\mylabel{L04145h}
\begin{anhang}
\end{anhang}\newcommand{\dateiname}{L04145}\newcommand{\titel}{Arthur Schnitzler an Gustav Schwarzkopf, 12. 3. 1903}\newcommand{\editorInnen}{Herausgegeben von Jahnke, SelmaMüller, Martin Anton}%% latex-leseansicht-abspann.tex
%% Abspann für die Leseansicht.
%% Der Schalter \ifkorrekturansicht ist bereits durch den Vorspann gesetzt.

%% latex-abspann.tex
%% Gemeinsamer Abspann für Korrekturansicht und Leseansicht.
%% Setzt den Schalter \ifkorrekturansicht voraus (gesetzt in den
%% einbindenden Dateien latex-korrekturansicht-abspann.tex bzw.
%% latex-leseansicht-abspann.tex).
%% ---------------------------------------------------------------

\normalsize

% Das esempio-Environment wird nur in der Leseansicht benötigt
\ifkorrekturansicht\else
\newenvironment{esempio}[3]%
{
    \vspace{1.5ex}
    \rlap{\underline{#1}}
    \par
    \setlength{\parindent}{0cm}
    \nopagebreak
    \leftskip=#2cm
    \rightskip=#3cm
}
{
    \par
}
\fi

\doendnotes{C}
\bigskip
\vfill

\clearpage

\footnotesize

\ifkorrekturansicht
  \lohead{\textsc{register}}
\fi

% theindex-Environment neu definieren ohne reledmac
\makeatletter
\renewenvironment{theindex}{%
  \ifkorrekturansicht
    \section*{\indexname}%
  \else
    \subsubsection*{Index der erwähnten Entitäten}%
  \fi
  \setlength{\parindent}{0pt}%
  \setlength{\parskip}{0pt plus 0.3pt}%
  \let\item\@idxitem
}{%
  \ifkorrekturansicht\clearpage\fi
}
\makeatother

\IfFileExists{\jobname-pw.ind}{\input{\jobname-pw.ind}}{}

% Quellenangabe nur in der Leseansicht
\ifkorrekturansicht\else
% Fallback-Definitionen, falls die .tex-Datei \titel etc. nicht gesetzt hat
\providecommand{\titel}{}
\providecommand{\editorInnen}{}
\providecommand{\dateiname}{\jobname}

\vspace{3cm}

\vfill

\footnotesize
\textsc{Quelle}: \titel. Herausgegeben von {\editorInnen}. In: \emph{Arthur Schnitzler: Briefwechsel mit Autorinnen und Autoren}.
 Digitale Edition, https://schnitzler-briefe.acdh.oeaw.ac.at/{\dateiname}.html (Stand \today)
\fi

\end{document}


