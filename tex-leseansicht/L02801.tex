%% latex-korrekturansicht-vorspann.tex
%% Vorspann für die Korrekturansicht.
%% Lädt die gemeinsame Datei latex-vorspann.tex mit gesetztem Schalter.

\newif\ifkorrekturansicht
\korrekturansichttrue

\input{../tex-inputs/latex-vorspann}


\section[ Paul Goldmann an Arthur Schnitzler, 27. 1. {[}1897{]}]{L02801 Paul Goldmann an Arthur Schnitzler, 27. 1. {[}1897{]}}
\nopagebreak\mylabel{L02801v}
\rehead{ }\normalsize\beginnumbering\briefempfaengerindex{Schnitzler, Arthur@\textsc{Schnitzler, Arthur}!zzzGoldmann, Paul@\emph{von Paul Goldmann}!1897-01-271@{27. 1. {[}1897{]}}|(be}
\toendnotes[C]{\smallbreak\pagebreak[2]}\Standort{DLA, A:Schnitzler, HS.NZ85.1.3167.}
\physDesc{Brief, 1 Blatt, 4 Seiten, 1754 Zeichen
\newline{}Handschrift: blaue Tinte, deutsche Kurrent
\newline{}Schnitzler: 1) mit Bleistift das Jahr »97« vermerkt  2) mit rotem Buntstift eine Unterstreichung}\toendnotes[C]{\smallbreak}
\pstart
           {\pb}\textcolor{gray}{\textbf{\textbf{Frankfurter Zeitung\orgindex{Frankfurter Zeitung@Frankfurter Zeitung|pw}}}}\pend
           
\pstart
           \textcolor{gray}{\textbf{(\begin{otherlanguage}{french}Gazette de Francfort\end{otherlanguage}\orgindex{Frankfurter Zeitung@Frankfurter Zeitung|pw}).}}\pend
           
\pstart
           \textcolor{gray}{\textbf{\textbf{\begin{otherlanguage}{french}Fondateur M.\end{otherlanguage}{ }L. Sonnemann\pwindex{Sonnemann, Leopold 1831-10-29 – 1909-10-30@\textsc{Sonnemann, Leopold} (1831-10-29 – 1909-10-30), \emph{Journalist/Journalistin, Herausgeber/Herausgeberin}|pw}.}}}\pend
           
\pstart
           \begin{otherlanguage}{french}\textcolor{gray}{\textbf{Journal politique, financier,}}\end{otherlanguage}\pend
           
\pstart
           \begin{otherlanguage}{french}\textcolor{gray}{\textbf{commercial et littéraire.}}\end{otherlanguage}\pend
           
\pstart
           \begin{otherlanguage}{french}\textcolor{gray}{\textbf{\textbf{Paraissant trois fois par jour.}}}\end{otherlanguage}\hfill \textsc{Paris\oindex{Paris@\textbf{Paris}, \emph{P.PPLC}|pw}}, 27. Januar.\pend
           
\pstart
           \begin{otherlanguage}{french}\textcolor{gray}{\textbf{\textbf{Bureau à Paris\oindex{Paris@\textbf{Paris}, \emph{P.PPLC}|pw}}}}\end{otherlanguage}\pend
           
\pstart
           \begin{otherlanguage}{french}\textcolor{gray}{\textbf{\textbf{24. Rue Feydeau\oindex{rue Feydeau@\textbf{rue Feydeau}, \emph{Straße (K.STR)}|pw}.}}}\end{otherlanguage}\pend
           
\pstart\center{}Mein lieber Freund,\pend\vspace{0.5em}
\pstart
           Nur wenige Worte heut!\pend
           
\pstart
           Dein lieber Brief hat mich beunruhigt. Was für \label{K_L02801-1v}\edtext{Aufregungen}{\lemma{\textnormal{\emph{Aufregungen}}}\Cendnote{\textnormal{Marie Reinhard\pwindex{Reinhard, Marie 1871-03-13 – 1899-03-18@\textsc{Reinhard, Marie} (1871-03-13 – 1899-03-18), \emph{Gesangspädagoge/Gesangspädagogin}|pwk} war im Dezember 1896 von Schnitzler
                  schwanger geworden, was sie im Jänner 1897
                  feststellten.}}}\label{K_L02801-1} ſind das\substVorne{}\textsuperscript{?}\substDazwischen{},\substHinten{} welche Du durchzumachen haſt? Ich will keine Einzelheiten wiſſen. Du wirſt
               mir ſchreiben, wenn Du ruhig biſt und Zeit haſt. Aber nur in einer Zeile ſollteſt Du
               mir ſagen: Hängt die Sache mit Frauen, mit der gewiſſen Dame\pwindex{Reinhard, Marie 1871-03-13 – 1899-03-18@\textsc{Reinhard, Marie} (1871-03-13 – 1899-03-18), \emph{Gesangspädagoge/Gesangspädagogin}|pwv} zuſammen? Oder ſind es Vorgänge nicht
               weiblicher Art? Im erſteren Falle würde ich bedeutend ruhiger \strikeout{ ſ\textcolor{gray}{a}} ſein. Das mag Dir frivol erſcheinen – Dir, der Du mitten darin ſtehſt. Aber
               ich {\pb}huldige doch der hier zu Lande üblichen Auffaſſung\substVorne{}\textsuperscript{:}\substDazwischen{},\substHinten{} daß Erlebniſſe mit Frauen ſelten ſchwere und weſentliche Schädigungen im
               Leben zurücklaſſen{\dotsfour}\pend
           
\pstart
           Innigen Dank für die Wärme, mit welcher Du Dich der \label{K_L02801-2v}\edtext{\textsc{Lorenzaccio\pwindex{Lorenzaccio. Drame romantique en cinq actes@\emph{Lorenzaccio. Drame romantique en cinq actes}|pw}}-Angelegenheit}{\lemma{\textnormal{\emph{Lorenzaccio-Angelegenheit}}}\Cendnote{\textnormal{Siehe Paul Goldmann an Arthur Schnitzler, 2. [1.? 1897].
               }}}\label{K_L02801-2} angenommen haſt! Ich weiß nicht, ob ich mich an die Arbeit machen werde. Es
               liegt eine complicirte Rechts-Situation vor. Nach fran\oindex{Frankreich@\textbf{Frankreich}, \emph{A.PCLI}|pwv}zöſiſchem Rechte iſt \textsc{Musset\pwindex{Musset, Alfred de 11.12.1810 – 02.05.1857@\textsc{Musset, Alfred de} (11.12.1810 – 02.05.1857), \emph{Schriftsteller/Schriftstellerin}|pw}} noch nicht frei (er wird es erſt in zehn Jahren), und die \label{K_L02801-3v}\edtext{Erben\pwindex{Lardin de Musset, Paul 1848-09-11 – 1908-03-28@\textsc{Lardin de Musset, Paul} (1848-09-11 – 1908-03-28), \emph{Beamter/Beamte}|pwv}}{\lemma{\textnormal{\emph{Erben}}}\Cendnote{\textnormal{Es ist unklar, mit wem Goldmann\pwindex{Goldmann, Paul 31.01.1865 – 25.09.1935@\textsc{Goldmann, Paul} (31.01.1865 – 25.09.1935), \emph{Schriftsteller/Schriftstellerin, Journalist/Journalistin}|pwk} in Kontakt stand. Die Rechte an den Werken Alfred de Mussets\pwindex{Musset, Alfred de 11.12.1810 – 02.05.1857@\textsc{Musset, Alfred de} (11.12.1810 – 02.05.1857), \emph{Schriftsteller/Schriftstellerin}|pwk} verwaltete jedenfalls
                  dessen Neffe Paul Lardin de Musset\pwindex{Lardin de Musset, Paul 1848-09-11 – 1908-03-28@\textsc{Lardin de Musset, Paul} (1848-09-11 – 1908-03-28), \emph{Beamter/Beamte}|pwk}.}}}\label{K_L02801-3}
               ſtellen unverſchämte Forderungen. Ich erwarte die Antwort eines \label{K_L02801-4v}\edtext{deutſch\oindex{Deutschland@\textbf{Deutschland}, \emph{A.PCLI}|pwv}en Advocaten}{\lemma{\textnormal{\emph{deutſchen Advocaten}}}\Cendnote{\textnormal{nicht ermittelt}}}\label{K_L02801-4} über den Fall. {\pb}Bin auch wenig zur Arbeit geſtimmt. Bin krank und
               werde täglich von der gräßlichen Angſt geplagt, \label{K_L02801-5v}\edtext{blind zu werden}{\lemma{\textnormal{\emph{blind zu werden}}}\Cendnote{\textnormal{Er hatte Augenprobleme, vgl. Paul Goldmann an Arthur Schnitzler, 2. [1.? 1897].}}}\label{K_L02801-5}{\dots}\pend
           
\pstart
           Geſtern ſandte ich Dir den »\textsc{Temps\pwindex{Le Temps@\emph{Le Temps}|pw}}« mit der ſchönen \label{K_L02801-6v}\edtext{Beſprechung\pwindex{Un vaudevilliste viennois@\emph{Un vaudevilliste viennois}|pwv}}{\lemma{\textnormal{\emph{Beſprechung}}}\Cendnote{\textnormal{Théodore de Wyzewa\pwindex{Wyzewa, Theodore de 1862-09-12 – 1917-04-07@\textsc{Wyzewa, Théodore de} (1862-09-12 – 1917-04-07), \emph{Schriftsteller/Schriftstellerin, Journalist/Journalistin}|pwk}: \emph{Un vaudevilliste viennois}\pwindex{Un vaudevilliste viennois@\emph{Un vaudevilliste viennois}|pwk}. In: \emph{Le Temps}\pwindex{Le Temps@\emph{Le Temps}|pwk}, Jg. 37, Nr. 13.023, 27. 1. 1897, S. 2.}}}\label{K_L02801-6} über Dich\pwindex{Mourir. Roman@\emph{Mourir. Roman}|pwv}. Der »\textsc{Temps\pwindex{Le Temps@\emph{Le Temps}|pw}}« iſt das angeſehenſte und geleſenſte fran\oindex{Frankreich@\textbf{Frankreich}, \emph{A.PCLI}|pwv}zöſiſche Blatt, die »Neue
                  Freie Preſſe\pwindex{Neue Freie Presse@\emph{Neue Freie Presse}|pw}« von \textsc{Paris\oindex{Paris@\textbf{Paris}, \emph{P.PPLC}|pw}}. Schreib’ dem \textsc{Wyzewa\pwindex{Wyzewa, Theodore de 1862-09-12 – 1917-04-07@\textsc{Wyzewa, Théodore de} (1862-09-12 – 1917-04-07), \emph{Schriftsteller/Schriftstellerin, Journalist/Journalistin}|pw}} (der ein Freund\pwindex{Wyzewa, Theodore de 1862-09-12 – 1917-04-07@\textsc{Wyzewa, Théodore de} (1862-09-12 – 1917-04-07), \emph{Schriftsteller/Schriftstellerin, Journalist/Journalistin}|pwv}{ }\textsc{Thorels\pwindex{Thorel, Jean 1859-09-11 – 1916-08-20@\textsc{Thorel, Jean} (1859-09-11 – 1916-08-20), \emph{Übersetzer/Übersetzerin, Dramatiker/Dramatikerin}|pw}} iſt) ein Wort des Dankes. Das
               kann gut thun, denn der Mann\pwindex{Wyzewa, Theodore de 1862-09-12 – 1917-04-07@\textsc{Wyzewa, Théodore de} (1862-09-12 – 1917-04-07), \emph{Schriftsteller/Schriftstellerin, Journalist/Journalistin}|pwv}
               hat großen Einfluß. Von \textsc{Thorel\pwindex{Thorel, Jean 1859-09-11 – 1916-08-20@\textsc{Thorel, Jean} (1859-09-11 – 1916-08-20), \emph{Übersetzer/Übersetzerin, Dramatiker/Dramatikerin}|pw}} höre ich nichts. Ich gehe dieſer Tage zu ihm{\dotsfour}\pend
           
\pstart
           Den Schluß des \label{K_L02801-7v}\edtext{Feuilletons\pwindex{Lorenzaccio@\emph{Lorenzaccio}|pwv} über \textsc{Lorenzaccio\pwindex{Lorenzaccio. Drame romantique en cinq actes@\emph{Lorenzaccio. Drame romantique en cinq actes}|pw}}}{\lemma{\textnormal{\emph{Feuilletons über Lorenzaccio}}}\Cendnote{\textnormal{Paul Goldmann\pwindex{Goldmann, Paul 31.01.1865 – 25.09.1935@\textsc{Goldmann, Paul} (31.01.1865 – 25.09.1935), \emph{Schriftsteller/Schriftstellerin, Journalist/Journalistin}|pwk}: \emph{Lorenzaccio}\pwindex{Lorenzaccio@\emph{Lorenzaccio}|pwk}. In: \emph{Frankfurter Zeitung}\pwindex{Frankfurter Zeitung@\emph{Frankfurter Zeitung}|pwk}, Jg. 41, Nr. 346, 13. 12. 1896,
                     Erstes Morgenblatt, S. 3; Nr. 347, 14. 12. 1896, Morgenblatt,
                     S. 1–2. Die Uraufführung von \emph{Lorenzaccio}\pwindex{Lorenzaccio. Drame romantique en cinq actes@\emph{Lorenzaccio. Drame romantique en cinq actes}|pwk} fand am 3. 12. 1896 im \emph{Théâtre de la Renaissance}\orgindex{Theâtre de la Renaissance@Théâtre de la Renaissance|pwk} in Paris\oindex{Paris@\textbf{Paris}, \emph{P.PPLC}|pwk} statt.}}}\label{K_L02801-7} ſende ich Dir deshalb {\pb}nicht, weil er nur mit wenigen Worten die Pariſ\oindex{Paris@\textbf{Paris}, \emph{P.PPLC}|pw}er Aufführung\pwindex{Lorenzaccio. Drame romantique en cinq actes@\emph{Lorenzaccio. Drame romantique en cinq actes}|pwv} beſpricht.\pend
           
\pstart
           Bald höre ich hoffentlich von Dir. \label{K_L02801-8v}\edtext{Arbeiteſt Du gar nichts?}{\lemma{\textnormal{\emph{Arbeiteſt Du gar nichts?}}}\Cendnote{\textnormal{Schnitzler war aufgrund der Aufregungen rund
                  um Marie Reinhards\pwindex{Reinhard, Marie 1871-03-13 – 1899-03-18@\textsc{Reinhard, Marie} (1871-03-13 – 1899-03-18), \emph{Gesangspädagoge/Gesangspädagogin}|pwk} Schwangerschaft
                  tatsächlich arbeitsunfähig, wie er mehrmals im \emph{Tagebuch}\pwindex{Tagebuch@\emph{Tagebuch}|pwk} notierte (vgl. A. S.: \emph{Tagebuch}, 17. 1. 1897 und 21. 1. 1897).}}}\label{K_L02801-8}\pend
           
\pstart
           Sei von Herzen gegrüßt!\pend
           
\pstart
           Dein treuer {\\[\baselineskip]}\spacefill\mbox{Paul Goldmann}\pend
           \leftskip=0em{}\selectlanguage{ngerman}\endnumbering\briefempfaengerindex{Schnitzler, Arthur@\textsc{Schnitzler, Arthur}!zzzGoldmann, Paul@\emph{von Paul Goldmann}!1897-01-271@{27. 1. {[}1897{]}}|)be}\mylabel{L02801h}  \normalsize

\doendnotes{C}
\bigskip
\vfill

\clearpage

\footnotesize

\lohead{\textsc{register}}

% Definiere theindex-Environment komplett neu ohne reledmac
\makeatletter
\renewenvironment{theindex}{%
  \section*{\indexname}%
  \setlength{\parindent}{0pt}%
  \setlength{\parskip}{0pt plus 0.3pt}%
  \let\item\@idxitem
}{%
  \clearpage
}
\makeatother

\IfFileExists{\jobname-pw.ind}{\input{\jobname-pw.ind}}{}

\end{document}

      