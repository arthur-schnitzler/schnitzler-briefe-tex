%% latex-leseansicht-vorspann.tex
%% Vorspann für die Leseansicht.
%% Lädt die gemeinsame Datei latex-vorspann.tex mit nicht gesetztem Schalter.

\newif\ifkorrekturansicht
\korrekturansichtfalse

\input{../tex-inputs/latex-vorspann}


         
         \renewcommand{\erwaehntePersonen}{Personen: Paul Lardin de Musset, Alfred de Musset, Marie Reinhard, Leopold Sonnemann, Jean Thorel, Théodore de Wyzewa}
         \renewcommand{\erwaehnteInstitutionen}{Institutionen: Frankfurter Zeitung, Théâtre de la Renaissance}
         \renewcommand{\erwaehnteOrte}{Orte: Deutschland, Frankreich, Paris, Wien, rue Feydeau}
         \renewcommand{\erwaehnteWerke}{Werke: Frankfurter Zeitung, Le Temps, Lorenzaccio, Lorenzaccio. Drame romantique en cinq actes, Mourir. Roman, Neue Freie Presse, Tagebuch, Un vaudevilliste viennois}
               \section[ Paul Goldmann an Arthur Schnitzler, 27. 1. {[}1897{]}]{ Paul Goldmann an Arthur Schnitzler, 27. 1. {[}1897{]}}\nopagebreak\mylabel{v}\rehead{ }\begin{ledgroupsized}[t]{13cm}\normalsize\beginnumbering \toendnotes[C]{\smallbreak\pagebreak[2]} \Standort{DLA, A:Schnitzler, HS.NZ85.1.3167.}
\physDesc{Brief, 1 Blatt, 4 Seiten, 1754 Zeichen
\newline{}Handschrift: blaue Tinte, deutsche Kurrent
\newline{}Schnitzler: 1) mit Bleistift das Jahr »97« vermerkt  2) mit rotem Buntstift eine Unterstreichung}\toendnotes[C]{\smallbreak}\pstart
           \noindent{}{\pb}\textcolor{gray}{\textbf{\textbf{Frankfurter Zeitung\orgindex{Frankfurter Zeitung@Frankfurter Zeitung|pw}}}}\pend
           \pstart
           \textcolor{gray}{\textbf{(\begin{otherlanguage}{french}Gazette de Francfort\end{otherlanguage}\orgindex{Frankfurter Zeitung@Frankfurter Zeitung|pw}).}}\pend
           \pstart
           \textcolor{gray}{\textbf{\textbf{\begin{otherlanguage}{french}Fondateur M.\end{otherlanguage}{ }L. Sonnemann\pwindex{Sonnemann, Leopold 1831-10-29 – 1909-10-30@\textsc{Sonnemann, Leopold} (1831-10-29 – 1909-10-30), \emph{Journalist, Herausgeber}|pw}.}}}\pend
           \pstart
           \begin{otherlanguage}{french}\textcolor{gray}{\textbf{Journal politique, financier,}}\end{otherlanguage}\pend
           \pstart
           \begin{otherlanguage}{french}\textcolor{gray}{\textbf{commercial et littéraire.}}\end{otherlanguage}\pend
           \pstart
           \begin{otherlanguage}{french}\textcolor{gray}{\textbf{\textbf{Paraissant trois fois par jour.}}}\end{otherlanguage}\hfill \textsc{Paris\oindex{Paris@\textbf{Paris}|pw}}, 27. Januar.\pend
           \pstart
           \begin{otherlanguage}{french}\textcolor{gray}{\textbf{\textbf{Bureau à Paris\oindex{Paris@\textbf{Paris}|pw}}}}\end{otherlanguage}\pend
           \pstart
           \begin{otherlanguage}{french}\textcolor{gray}{\textbf{\textbf{24. Rue Feydeau\oindex{rue Feydeau@\textbf{rue Feydeau}|pw}.}}}\end{otherlanguage}\pend
           \pstart\center{}Mein lieber Freund,\pend\pstart
           Nur wenige Worte heut!\pend
           \pstart
           Dein lieber Brief hat mich beunruhigt. Was für \label{K_L02801-1v}\edtext{Aufregungen}{\lemma{\textnormal{\emph{Aufregungen}}}\Cendnote{\textnormal{Marie Reinhard\pwindex{Reinhard, Marie 1871-03-13 – 1899-03-18@\textsc{Reinhard, Marie} (1871-03-13 – 1899-03-18), \emph{Gesangspädagogin}|pwk} war im Dezember 1896 von Schnitzler\pwindex{Schnitzler, Arthur 15.05.1862 – 21.10.1931@\textsc{Schnitzler, Arthur} (15.05.1862 – 21.10.1931), \emph{Schriftsteller, Mediziner}|pwk}
                  schwanger geworden, was sie im Jänner 1897
                  feststellten.}}}\label{K_L02801-1h} ſind das\substVorne{}\textsuperscript{?}\substDazwischen{},\substHinten{} welche Du durchzumachen haſt? Ich will keine Einzelheiten wiſſen. Du wirſt
               mir ſchreiben, wenn Du ruhig biſt und Zeit haſt. Aber nur in einer Zeile ſollteſt Du
               mir ſagen: Hängt die Sache mit Frauen, mit der gewiſſen Dame\pwindex{Reinhard, Marie 1871-03-13 – 1899-03-18@\textsc{Reinhard, Marie} (1871-03-13 – 1899-03-18), \emph{Gesangspädagogin}|pwv} zuſammen? Oder ſind es Vorgänge nicht
               weiblicher Art? Im erſteren Falle würde ich bedeutend ruhiger \strikeout{ ſ\textcolor{gray}{a}} ſein. Das mag Dir frivol erſcheinen – Dir, der Du mitten darin ſtehſt. Aber
               ich {\pb}huldige doch der hier zu Lande üblichen Auffaſſung\substVorne{}\textsuperscript{:}\substDazwischen{},\substHinten{} daß Erlebniſſe mit Frauen ſelten ſchwere und weſentliche Schädigungen im
               Leben zurücklaſſen{\dotsfour}\pend
           \pstart
           Innigen Dank für die Wärme, mit welcher Du Dich der \label{K_L02801-2v}\edtext{\textsc{Lorenzaccio\pwindex{Musset, Alfred de 11.12.1810 – 02.05.1857@\textsc{Musset, Alfred de} (11.12.1810 – 02.05.1857), \emph{Schriftsteller}!Lorenzaccio. Drame romantique en cinq actes1834@\strich\emph{Lorenzaccio. Drame romantique en cinq actes} {[}1834{]}|pw}}-Angelegenheit}{\lemma{\textnormal{\emph{Lorenzaccio-Angelegenheit}}}\Cendnote{\textnormal{siehe Paul Goldmann an Arthur Schnitzler, 2. [1.? 1897]}}}\label{K_L02801-2h} angenommen haſt! Ich weiß nicht, ob ich mich an die Arbeit machen werde. Es
               liegt eine complicirte Rechts-Situation vor. Nach fran\oindex{Frankreich@\textbf{Frankreich}|pwv}zöſiſchem Rechte iſt \textsc{Musset\pwindex{Musset, Alfred de 11.12.1810 – 02.05.1857@\textsc{Musset, Alfred de} (11.12.1810 – 02.05.1857), \emph{Schriftsteller}|pw}} noch nicht frei (er wird es erſt in zehn Jahren), und die \label{K_L02801-3v}\edtext{Erben\pwindex{Lardin de Musset, Paul 1848-09-11 – 1908-03-28@\textsc{Lardin de Musset, Paul} (1848-09-11 – 1908-03-28), \emph{Beamter}|pwv}}{\lemma{\textnormal{\emph{Erben}}}\Cendnote{\textnormal{Es ist unklar, mit wem Goldmann\pwindex{Goldmann, Paul 31.01.1865 – 25.09.1935@\textsc{Goldmann, Paul} (31.01.1865 – 25.09.1935), \emph{Schriftsteller, Journalist}|pwk} in Kontakt stand. Die Rechte an den Werken Alfred de Musset\pwindex{Musset, Alfred de 11.12.1810 – 02.05.1857@\textsc{Musset, Alfred de} (11.12.1810 – 02.05.1857), \emph{Schriftsteller}|pwk}s verwaltete jedenfalls
                  dessen Neffe Paul Lardin de Musset\pwindex{Lardin de Musset, Paul 1848-09-11 – 1908-03-28@\textsc{Lardin de Musset, Paul} (1848-09-11 – 1908-03-28), \emph{Beamter}|pwk}.}}}\label{K_L02801-3h}
               ſtellen unverſchämte Forderungen. Ich erwarte die Antwort eines \label{K_L02801-4v}\edtext{deutſch\oindex{Deutschland@\textbf{Deutschland}|pwv}en Advocaten}{\lemma{\textnormal{\emph{deutſchen Advocaten}}}\Cendnote{\textnormal{nicht ermittelt}}}\label{K_L02801-4h} über den Fall. {\pb}Bin auch wenig zur Arbeit geſtimmt. Bin krank und
               werde täglich von der gräßlichen Angſt geplagt, \label{K_L02801-5v}\edtext{blind zu werden}{\lemma{\textnormal{\emph{blind zu werden}}}\Cendnote{\textnormal{aufgrund seiner Augenprobleme (siehe Paul Goldmann an Arthur Schnitzler, 2. [1.? 1897])}}}\label{K_L02801-5h}{\dots}\pend
           \pstart
           Geſtern ſandte ich Dir den »\textsc{Temps\pwindex{?? Werk@Nicht ermittelte Verfasserinnen und Verfasser!Le Temps1861-04-25 – 1942-11-30@\emph{Le Temps} {[}1861-04-25 – 1942-11-30{]}|pw}}« mit der ſchönen \label{K_L02801-6v}\edtext{Beſprechung\pwindex{Wyzewa, Theodore de 1862-09-12 – 1917-04-07@\textsc{Wyzewa, Théodore de} (1862-09-12 – 1917-04-07), \emph{Schriftsteller, Journalist}!Un vaudevilliste viennois1897-01-27@\strich\emph{Un vaudevilliste viennois} {[}1897-01-27{]}|pwv}}{\lemma{\textnormal{\emph{Beſprechung}}}\Cendnote{\textnormal{Théodore de Wyzewa\pwindex{Wyzewa, Theodore de 1862-09-12 – 1917-04-07@\textsc{Wyzewa, Théodore de} (1862-09-12 – 1917-04-07), \emph{Schriftsteller, Journalist}|pwk}: \emph{Un vaudevilliste viennois}\pwindex{Wyzewa, Theodore de 1862-09-12 – 1917-04-07@\textsc{Wyzewa, Théodore de} (1862-09-12 – 1917-04-07), \emph{Schriftsteller, Journalist}!Un vaudevilliste viennois1897-01-27@\strich\emph{Un vaudevilliste viennois} {[}1897-01-27{]}|pwk}. In: \emph{Le Temps}\pwindex{?? Werk@Nicht ermittelte Verfasserinnen und Verfasser!Le Temps1861-04-25 – 1942-11-30@\emph{Le Temps} {[}1861-04-25 – 1942-11-30{]}|pwk}, Jg. 37, Nr. 13023, 27. 1. 1897, S. 2.}}}\label{K_L02801-6h} über Dich\pwindex{Schnitzler, Arthur 15.05.1862 – 21.10.1931@\textsc{Schnitzler, Arthur} (15.05.1862 – 21.10.1931), \emph{Schriftsteller, Mediziner}!Mourir. Roman1895-04-27 – 1895-06-01@\strich\emph{Mourir. Roman} {[}1895-04-27 – 1895-06-01{]}|pwv}. Der »\textsc{Temps\pwindex{?? Werk@Nicht ermittelte Verfasserinnen und Verfasser!Le Temps1861-04-25 – 1942-11-30@\emph{Le Temps} {[}1861-04-25 – 1942-11-30{]}|pw}}« iſt das angeſehenſte und geleſenſte fran\oindex{Frankreich@\textbf{Frankreich}|pwv}zöſiſche Blatt, die »Neue
                  Freie Preſſe\pwindex{Neue Freie Presse1864 – 1939@\emph{Neue Freie Presse} {[}1864 – 1939{]}|pw}« von \textsc{Paris\oindex{Paris@\textbf{Paris}|pw}}. Schreib’ dem \textsc{Wyzewa\pwindex{Wyzewa, Theodore de 1862-09-12 – 1917-04-07@\textsc{Wyzewa, Théodore de} (1862-09-12 – 1917-04-07), \emph{Schriftsteller, Journalist}|pw}} (der ein Freund\pwindex{Wyzewa, Theodore de 1862-09-12 – 1917-04-07@\textsc{Wyzewa, Théodore de} (1862-09-12 – 1917-04-07), \emph{Schriftsteller, Journalist}|pwv}{ }\textsc{Thorel\pwindex{Thorel, Jean 1859-09-11 – 1916-08-20@\textsc{Thorel, Jean} (1859-09-11 – 1916-08-20), \emph{Übersetzer, Schriftsteller}|pw}s} iſt) ein Wort des Dankes. Das
               kann gut thun, denn der Mann\pwindex{Wyzewa, Theodore de 1862-09-12 – 1917-04-07@\textsc{Wyzewa, Théodore de} (1862-09-12 – 1917-04-07), \emph{Schriftsteller, Journalist}|pwv}
               hat großen Einfluß. Von \textsc{Thorel\pwindex{Thorel, Jean 1859-09-11 – 1916-08-20@\textsc{Thorel, Jean} (1859-09-11 – 1916-08-20), \emph{Übersetzer, Schriftsteller}|pw}} höre ich nichts. Ich gehe dieſer Tage zu ihm{\dotsfour}\pend
           \pstart
           Den Schluß des \label{K_L02801-7v}\edtext{Feuilleton\pwindex{Goldmann, Paul 31.01.1865 – 25.09.1935@\textsc{Goldmann, Paul} (31.01.1865 – 25.09.1935), \emph{Schriftsteller, Journalist}!Lorenzaccio1896-12-13 – 1896-12-14@\strich\emph{Lorenzaccio} {[}1896-12-13 – 1896-12-14{]}|pwv}s über \textsc{Lorenzaccio\pwindex{Musset, Alfred de 11.12.1810 – 02.05.1857@\textsc{Musset, Alfred de} (11.12.1810 – 02.05.1857), \emph{Schriftsteller}!Lorenzaccio. Drame romantique en cinq actes1834@\strich\emph{Lorenzaccio. Drame romantique en cinq actes} {[}1834{]}|pw}}}{\lemma{\textnormal{\emph{Feuilletons über Lorenzaccio}}}\Cendnote{\textnormal{Paul Goldmann\pwindex{Goldmann, Paul 31.01.1865 – 25.09.1935@\textsc{Goldmann, Paul} (31.01.1865 – 25.09.1935), \emph{Schriftsteller, Journalist}|pwk}: \emph{Lorenzaccio}\pwindex{Goldmann, Paul 31.01.1865 – 25.09.1935@\textsc{Goldmann, Paul} (31.01.1865 – 25.09.1935), \emph{Schriftsteller, Journalist}!Lorenzaccio1896-12-13 – 1896-12-14@\strich\emph{Lorenzaccio} {[}1896-12-13 – 1896-12-14{]}|pwk}. In: \emph{Frankfurter Zeitung}\pwindex{?? Werk@Nicht ermittelte Verfasserinnen und Verfasser!Frankfurter Zeitung1856 – 1943@\emph{Frankfurter Zeitung} {[}1856 – 1943{]}|pwk}, Jg. 41, Nr. 346, 13. 12. 1896,
                     Erstes Morgenblatt, S. 3; Nr. 347, 14. 12. 1896, Morgenblatt,
                     S. 1–2. Die Uraufführung von \emph{Lorenzaccio}\pwindex{Musset, Alfred de 11.12.1810 – 02.05.1857@\textsc{Musset, Alfred de} (11.12.1810 – 02.05.1857), \emph{Schriftsteller}!Lorenzaccio. Drame romantique en cinq actes1834@\strich\emph{Lorenzaccio. Drame romantique en cinq actes} {[}1834{]}|pwk} fand am 3. 12. 1896 im \emph{Théâtre de la Renaissance}\orgindex{Theâtre de la Renaissance@Théâtre de la Renaissance|pwk} in Paris\oindex{Paris@\textbf{Paris}|pwk} statt.}}}\label{K_L02801-7h} ſende ich Dir deshalb {\pb}nicht, weil er nur mit wenigen Worten die Pariſ\oindex{Paris@\textbf{Paris}|pw}er Aufführung\pwindex{Musset, Alfred de 11.12.1810 – 02.05.1857@\textsc{Musset, Alfred de} (11.12.1810 – 02.05.1857), \emph{Schriftsteller}!Lorenzaccio. Drame romantique en cinq actes1834@\strich\emph{Lorenzaccio. Drame romantique en cinq actes} {[}1834{]}|pwv} beſpricht.\pend
           \pstart
           Bald höre ich hoffentlich von Dir. \label{K_L02801-8v}\edtext{Arbeiteſt Du gar nichts?}{\lemma{\textnormal{\emph{Arbeiteſt Du gar nichts?}}}\Cendnote{\textnormal{Schnitzler\pwindex{Schnitzler, Arthur 15.05.1862 – 21.10.1931@\textsc{Schnitzler, Arthur} (15.05.1862 – 21.10.1931), \emph{Schriftsteller, Mediziner}|pwk} war aufgrund der Aufregungen rund
                  um Marie Reinhard\pwindex{Reinhard, Marie 1871-03-13 – 1899-03-18@\textsc{Reinhard, Marie} (1871-03-13 – 1899-03-18), \emph{Gesangspädagogin}|pwk}s Schwangerschaft
                  tatsächlich arbeitsunfähig, wie er mehrmals im \emph{Tagebuch}\pwindex{Schnitzler, Arthur 15.05.1862 – 21.10.1931@\textsc{Schnitzler, Arthur} (15.05.1862 – 21.10.1931), \emph{Schriftsteller, Mediziner}!Tagebuch1981 – 2000@\strich\emph{Tagebuch} {[}1981 – 2000{]}|pwk} notierte (vgl. A. S.: \emph{Tagebuch}, 17. 1. 1897 und 21. 1. 1897).}}}\label{K_L02801-8h}\pend
           \pstart
           Sei von Herzen gegrüßt!\pend
           \pstart
           Dein treuer {\\[\baselineskip]}\spacefill\mbox{Paul Goldmann}\pend
           \leftskip=0em{}
         
         \endnumbering\mylabel{h}\end{ledgroupsized}  \newcommand{\dateiname}{L02801}\newcommand{\titel}{Paul Goldmann an Arthur Schnitzler, 27. 1. [1897]}\newcommand{\editorInnen}{Martin Anton Müller und Laura Untner}%% latex-leseansicht-abspann.tex
%% Abspann für die Leseansicht.
%% Der Schalter \ifkorrekturansicht ist bereits durch den Vorspann gesetzt.

%% latex-abspann.tex
%% Gemeinsamer Abspann für Korrekturansicht und Leseansicht.
%% Setzt den Schalter \ifkorrekturansicht voraus (gesetzt in den
%% einbindenden Dateien latex-korrekturansicht-abspann.tex bzw.
%% latex-leseansicht-abspann.tex).
%% ---------------------------------------------------------------

\normalsize

% Das esempio-Environment wird nur in der Leseansicht benötigt
\ifkorrekturansicht\else
\newenvironment{esempio}[3]%
{
    \vspace{1.5ex}
    \rlap{\underline{#1}}
    \par
    \setlength{\parindent}{0cm}
    \nopagebreak
    \leftskip=#2cm
    \rightskip=#3cm
}
{
    \par
}
\fi

\doendnotes{C}
\bigskip
\vfill

\clearpage

\footnotesize

\ifkorrekturansicht
  \lohead{\textsc{register}}
\fi

% theindex-Environment neu definieren ohne reledmac
\makeatletter
\renewenvironment{theindex}{%
  \ifkorrekturansicht
    \section*{\indexname}%
  \else
    \subsubsection*{Index der erwähnten Entitäten}%
  \fi
  \setlength{\parindent}{0pt}%
  \setlength{\parskip}{0pt plus 0.3pt}%
  \let\item\@idxitem
}{%
  \ifkorrekturansicht\clearpage\fi
}
\makeatother

\IfFileExists{\jobname-pw.ind}{\input{\jobname-pw.ind}}{}

% Quellenangabe nur in der Leseansicht
\ifkorrekturansicht\else
% Fallback-Definitionen, falls die .tex-Datei \titel etc. nicht gesetzt hat
\providecommand{\titel}{}
\providecommand{\editorInnen}{}
\providecommand{\dateiname}{\jobname}

\vspace{3cm}

\vfill

\footnotesize
\textsc{Quelle}: \titel. Herausgegeben von {\editorInnen}. In: \emph{Arthur Schnitzler: Briefwechsel mit Autorinnen und Autoren}.
 Digitale Edition, https://schnitzler-briefe.acdh.oeaw.ac.at/{\dateiname}.html (Stand \today)
\fi

\end{document}


      