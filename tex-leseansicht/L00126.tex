%% latex-korrekturansicht-vorspann.tex
%% Vorspann für die Korrekturansicht.
%% Lädt die gemeinsame Datei latex-vorspann.tex mit gesetztem Schalter.

\newif\ifkorrekturansicht
\korrekturansichttrue

\input{../tex-inputs/latex-vorspann}


\section[Arthur Schnitzler an Richard Beer-Hofmann, 2. 10. 1892]{L00126 Arthur Schnitzler an Richard Beer-Hofmann, 2. 10. 1892}
\nopagebreak\mylabel{L00126v}
\rehead{ }\normalsize\beginnumbering\briefempfaengerindex{Beer-Hofmann, Richard@\textsc{Beer-Hofmann, Richard}!zzzSchnitzler, Arthur@\emph{von Arthur Schnitzler}!1892-10-021@{2. 10. 1892}|(be}
\toendnotes[C]{\smallbreak\pagebreak[2]}\Standort{YCGL, MSS 31.}
\physDesc{Postkarte, 300 Zeichen
\newline{}Handschrift: Bleistift, deutsche Kurrent
\newline{}Versand: 1) Rohrpost  2) Stempel: »\nobreak{}\oindex{I., Innere Stadt@\textbf{I., Innere Stadt}, \emph{A.ADM3}|pwk}Wien 1/1, 2-X 93, 7 10N\nobreak{}«.  3) Stempel: »\nobreak{}\oindex{I., Innere Stadt@\textbf{I., Innere Stadt}, \emph{A.ADM3}|pwk}Wien 1/1, 2 X {[}92{]}, 7 40N\nobreak{}«. }
\buchAbdrucke{\weitereDrucke{Arthur Schnitzler, Richard Beer-Hofmann: \emph{Briefwechsel 1891–1931}. Wien, Zürich: \emph{Europaverlag} 1992, S. 39.} }\toendnotes[C]{\smallbreak}\pstart{}{\pb}Hrn \textsc{Dr. Richard Beer
                     Hofmann}\pend{}\pstart{}\textsc{Wien\oindex{Wien@\textbf{Wien}, \emph{A.ADM2}|pw}}\pend{}\pstart{}I. \textsc{Wollzeile 15}\oindex{Wollzeile@\textbf{Wollzeile}, \emph{Straße (K.STR)}|pw}\pend{}{\bigskip}\vspace{1em}
\pstart
           \noindent{}{\pb}Lieber Richard!{ }\textsc{Torres.\pwindex{Torresani-Lanzenfeld, Carl von 19.04.1846 – 16.04.1907@\textsc{Torresani-Lanzenfeld, Carl von} (19.04.1846 – 16.04.1907), \emph{Schriftsteller/Schriftstellerin, Offizier/Offizierin}|pw}} holt mich \label{K_L00126-1v}\edtext{\uline{Montag}}{\lemma{\textnormal{\emph{Montag}}}\Cendnote{\textnormal{Obzwar der Poststempel – sofern er sich
                  auf das Jahr bezieht – eindeutig 93 zeigt, scheint dies doch durch
                  den Inhalt ausgeschlossen. Schnitzler war am
                     Sonntag, 2. 10. 1892 in \emph{Gefallene
                     Engel}\pwindex{Gefallene Engel. Volkstueck in drei Aufzuegen@\emph{Gefallene Engel. Volkstück in drei Aufzügen}|pwk}, am Folgetag wurde er von Torresani\pwindex{Torresani-Lanzenfeld, Carl von 19.04.1846 – 16.04.1907@\textsc{Torresani-Lanzenfeld, Carl von} (19.04.1846 – 16.04.1907), \emph{Schriftsteller/Schriftstellerin, Offizier/Offizierin}|pwk} für das Ausstellungstheater\oindex{Internationales Ausstellungstheater im k.k. Prater@\textbf{Internationales Ausstellungstheater im k.k. Prater}, \emph{Theater (K.THE)}|pwk} abgeholt.}}}\label{K_L00126-1}{ }Nachmittag vor 5 Uhr für die Ausſtellung\oindex{Internationales Ausstellungstheater im k.k. Prater@\textbf{Internationales Ausstellungstheater im k.k. Prater}, \emph{Theater (K.THE)}|pwv} ab; bitte ko{\geminationm}en Sie auch zu mir. \uline{So{\geminationn}tag} denke ich zu den »gefallenen Engeln\pwindex{Gefallene Engel. Volkstueck in drei Aufzuegen@\emph{Gefallene Engel. Volkstück in drei Aufzügen}|pw}« zu
               gehn, wenn ich ordentliche Sitze beko{\geminationm}e. Jedenfalls bin
               ich um 5, ½ 6 zu Hauſe.\pend
           
\pstart
           Herzlich grüßend Ihr{\\[\baselineskip]}\spacefill\mbox{Arthur}\pend
           \leftskip=0em{}\selectlanguage{ngerman}\endnumbering\briefempfaengerindex{Beer-Hofmann, Richard@\textsc{Beer-Hofmann, Richard}!zzzSchnitzler, Arthur@\emph{von Arthur Schnitzler}!1892-10-021@{2. 10. 1892}|)be}\mylabel{L00126h}  \normalsize

\doendnotes{C}
\bigskip
\vfill

\clearpage

\footnotesize

\lohead{\textsc{register}}

% Definiere theindex-Environment komplett neu ohne reledmac
\makeatletter
\renewenvironment{theindex}{%
  \section*{\indexname}%
  \setlength{\parindent}{0pt}%
  \setlength{\parskip}{0pt plus 0.3pt}%
  \let\item\@idxitem
}{%
  \clearpage
}
\makeatother

\IfFileExists{\jobname-pw.ind}{\input{\jobname-pw.ind}}{}

\end{document}

      