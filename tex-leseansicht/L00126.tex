%% latex-leseansicht-vorspann.tex
%% Vorspann für die Leseansicht.
%% Lädt die gemeinsame Datei latex-vorspann.tex mit nicht gesetztem Schalter.

\newif\ifkorrekturansicht
\korrekturansichtfalse

\input{../tex-inputs/latex-vorspann}


\section[Arthur Schnitzler an Richard Beer-Hofmann, 2. 10. 1892]{L00126 Arthur Schnitzler an Richard Beer-Hofmann, 2. 10. 1892}
\nopagebreak\mylabel{L00126v}
\rehead{ }\normalsize\beginnumbering\briefempfaengerindex{Beer-Hofmann, Richard@\textsc{Beer-Hofmann, Richard}!zzzSchnitzler, Arthur@\emph{von Arthur Schnitzler}!1892-10-021@{2. 10. 1892}|(be}
\toendnotes[C]{\smallbreak\pagebreak[2]}
\correspDesc{Versand  durch Arthur Schnitzler am 2. 10. 1892 in Wien
\newline{}Erhalt  durch Richard Beer-Hofmann am 2. 10. 1892 in I., Innere Stadt}\toendnotes[C]{\smallbreak}
\Standort{YCGL, MSS 31.}
\physDesc{Postkarte, 300 Zeichen
\newline{}Handschrift: Bleistift, deutsche Kurrent
\newline{}Versand: 1) Rohrpost  2) Stempel: »\nobreak{}\oindex{I., Innere Stadt@\textbf{I., Innere Stadt}, \emph{Verwaltungsgebiet}|pwk}Wien 1/1, 2-X 93, 7 10N\nobreak{}«.  3) Stempel: »\nobreak{}\oindex{I., Innere Stadt@\textbf{I., Innere Stadt}, \emph{Verwaltungsgebiet}|pwk}Wien 1/1, 2 X {[}92{]}, 7 40N\nobreak{}«. }
\buchAbdrucke{\weitereDrucke{Arthur Schnitzler, Richard Beer-Hofmann: \emph{Briefwechsel 1891–1931}. Herausgegeben von Konstanze Fliedl. Wien, Zürich: \emph{Europaverlag} 1992, S. 39.} }\toendnotes[C]{\smallbreak}\pstart{}{\pb}Hrn \textsc{Dr. Richard Beer
                     Hofmann}\pend{}\pstart{}\textsc{Wien\oindex{Wien@\textbf{Wien}, \emph{Verwaltungsgebiet}|pw}}\pend{}\pstart{}I. \textsc{Wollzeile 15}\oindex{Wien@\textbf{Wien}!I., Innere Stadt@\textbf{I., Innere Stadt}!Wollzeile 15 (»Berthahof«)@\textbf{Wollzeile 15 (»Berthahof«)}, \emph{Wohngebäude}|pw}\pend{}{\bigskip}\vspace{1em}
\pstart
           \noindent{}{\pb}Lieber Richard!{ }\textsc{Torres.\pwindex{Torresani-Lanzenfeld, Carl von 19.\,4.\,1846 Mailand – 16.\,4.\,1907 Nago-Torbole@\textsc{Torresani-Lanzenfeld, Carl von} (19.\,4.\,1846 Mailand – 16.\,4.\,1907 Nago-Torbole), \emph{Schriftsteller, Offizier}|pw}} holt mich \label{K_L00126-1v}\edtext{\uline{Montag}}{\lemma{\textnormal{\emph{Montag}}}\Cendnote{\textnormal{Obzwar der Poststempel – sofern er sich
                  auf das Jahr bezieht – eindeutig 93 zeigt, scheint dies doch durch
                  den Inhalt ausgeschlossen. Schnitzler war am
                     Sonntag, 2. 10. 1892 in \emph{Gefallene
                     Engel}\pwindex{\textcolor{red}{\textsuperscript{XXXX indx1}}!Gefallene Engel. Volkstück in drei Aufzügen@\strich\emph{Gefallene Engel. Volkstück in drei Aufzügen}|pwk}, am Folgetag wurde er von Torresani\pwindex{Torresani-Lanzenfeld, Carl von 19.\,4.\,1846 Mailand – 16.\,4.\,1907 Nago-Torbole@\textsc{Torresani-Lanzenfeld, Carl von} (19.\,4.\,1846 Mailand – 16.\,4.\,1907 Nago-Torbole), \emph{Schriftsteller, Offizier}|pwk} für das Ausstellungstheater\oindex{Wien@\textbf{Wien}!II., Leopoldstadt@\textbf{II., Leopoldstadt}!Internationales Ausstellungstheater im k.k. Prater@\textbf{Internationales Ausstellungstheater im k.k. Prater}, \emph{Theater}|pwk} abgeholt.}}}\label{K_L00126-1}{ }Nachmittag vor 5 Uhr für die Ausſtellung\oindex{Wien@\textbf{Wien}!II., Leopoldstadt@\textbf{II., Leopoldstadt}!Internationales Ausstellungstheater im k.k. Prater@\textbf{Internationales Ausstellungstheater im k.k. Prater}, \emph{Theater}|pwv} ab; bitte ko{\geminationm}en Sie auch zu mir. \uline{So{\geminationn}tag} denke ich zu den »gefallenen Engeln\pwindex{\textcolor{red}{\textsuperscript{XXXX indx1}}!Gefallene Engel. Volkstück in drei Aufzügen@\strich\emph{Gefallene Engel. Volkstück in drei Aufzügen}|pw}« zu
               gehn, wenn ich ordentliche Sitze beko{\geminationm}e. Jedenfalls bin
               ich um 5, ½ 6 zu Hauſe.\pend
           
\pstart
           Herzlich grüßend Ihr{\\[\baselineskip]}\spacefill\mbox{Arthur}\pend
           \leftskip=0em{}\selectlanguage{ngerman}\endnumbering\briefempfaengerindex{Beer-Hofmann, Richard@\textsc{Beer-Hofmann, Richard}!zzzSchnitzler, Arthur@\emph{von Arthur Schnitzler}!1892-10-021@{2. 10. 1892}|)be}\mylabel{L00126h}  \newcommand{\dateiname}{L00126}\newcommand{\titel}{Arthur Schnitzler an Richard Beer-Hofmann, 2. 10. 1892}\newcommand{\editorInnen}{Martin Anton Müller und Gerd-Hermann Susen}%% latex-leseansicht-abspann.tex
%% Abspann für die Leseansicht.
%% Der Schalter \ifkorrekturansicht ist bereits durch den Vorspann gesetzt.

%% latex-abspann.tex
%% Gemeinsamer Abspann für Korrekturansicht und Leseansicht.
%% Setzt den Schalter \ifkorrekturansicht voraus (gesetzt in den
%% einbindenden Dateien latex-korrekturansicht-abspann.tex bzw.
%% latex-leseansicht-abspann.tex).
%% ---------------------------------------------------------------

\normalsize

% Das esempio-Environment wird nur in der Leseansicht benötigt
\ifkorrekturansicht\else
\newenvironment{esempio}[3]%
{
    \vspace{1.5ex}
    \rlap{\underline{#1}}
    \par
    \setlength{\parindent}{0cm}
    \nopagebreak
    \leftskip=#2cm
    \rightskip=#3cm
}
{
    \par
}
\fi

\doendnotes{C}
\bigskip
\vfill

\clearpage

\footnotesize

\ifkorrekturansicht
  \lohead{\textsc{register}}
\fi

% theindex-Environment neu definieren ohne reledmac
\makeatletter
\renewenvironment{theindex}{%
  \ifkorrekturansicht
    \section*{\indexname}%
  \else
    \subsubsection*{Index der erwähnten Entitäten}%
  \fi
  \setlength{\parindent}{0pt}%
  \setlength{\parskip}{0pt plus 0.3pt}%
  \let\item\@idxitem
}{%
  \ifkorrekturansicht\clearpage\fi
}
\makeatother

\IfFileExists{\jobname-pw.ind}{\input{\jobname-pw.ind}}{}

% Quellenangabe nur in der Leseansicht
\ifkorrekturansicht\else
% Fallback-Definitionen, falls die .tex-Datei \titel etc. nicht gesetzt hat
\providecommand{\titel}{}
\providecommand{\editorInnen}{}
\providecommand{\dateiname}{\jobname}

\vspace{3cm}

\vfill

\footnotesize
\textsc{Quelle}: \titel. Herausgegeben von {\editorInnen}. In: \emph{Arthur Schnitzler: Briefwechsel mit Autorinnen und Autoren}.
 Digitale Edition, https://schnitzler-briefe.acdh.oeaw.ac.at/{\dateiname}.html (Stand \today)
\fi

\end{document}


