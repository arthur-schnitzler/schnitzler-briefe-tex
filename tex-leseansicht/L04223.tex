%% latex-leseansicht-vorspann.tex
%% Vorspann für die Leseansicht.
%% Lädt die gemeinsame Datei latex-vorspann.tex mit nicht gesetztem Schalter.

\newif\ifkorrekturansicht
\korrekturansichtfalse

\input{../tex-inputs/latex-vorspann}


\section[Arthur Schnitzler an Gustav Schwarzkopf, 18. 7. 1913]{L04223 Arthur Schnitzler an Gustav Schwarzkopf, 18. 7. 1913}
\nopagebreak\mylabel{L04223v}
\rehead{ }\normalsize\beginnumbering\briefempfaengerindex{Schwarzkopf, Gustav@\textsc{Schwarzkopf, Gustav}!zzzSchnitzler, Arthur@\emph{von Arthur Schnitzler}!1913-07-181@{18. 7. 1913}|(be}
\toendnotes[C]{\smallbreak\pagebreak[2]}
\correspDesc{Versand  durch Arthur Schnitzler am 18. 7. 1913 in Wien
\newline{}Erhalt  durch Gustav Schwarzkopf im Zeitraum [19. 7. 1913 – 23. 7. 1913?] in Opatija}\toendnotes[C]{\smallbreak}
\Standort{DLA, A:Schnitzler, HS.1985.1.1897.}
\physDesc{Bildpostkarte, 379 Zeichen
\newline{}Handschrift: schwarze Tinte, deutsche Kurrent
\newline{}Versand: Stempel: »\nobreak{}\oindex{Wien@\textbf{Wien}, \emph{Verwaltungsgebiet}|pwk}Wie{[}n{]}, 18 VII 13\nobreak{}«.  }\toendnotes[C]{\smallbreak}\pstart{}{\pb}Herrn \textsc{Gustav
                     Schwarzkopf}\pend{}\pstart{}aus Wien\oindex{Wien@\textbf{Wien}, \emph{Verwaltungsgebiet}|pw},\pend{}\pstart{}\textsc{Abbazia\oindex{Opatija@\textbf{Opatija}, \emph{Hauptstadt}|pw}}\pend{}\pstart{}\textsc{Wiener
                  Heim\oindex{XXXX Ortsangabe fehlt|pw}}\pend{}{\bigskip}
\pstart
           \noindent{}{\pb}{[}Sternwartestrasse 71\oindex{Wien@\textbf{Wien}!XVIII., Währing@\textbf{XVIII., Währing}!Sternwartestraße 71@\textbf{Sternwartestraße 71}, \emph{Wohngebäude}|pw}{]}\pend
           \vspace{1em}
\pstart
           \raggedleft{}{\pb}18. 7. 913\pend
           \vspace{0.5em}
\pstart
           lieber Guſtav, we{\geminationn} alles weiter gut
               geht, hoffen wir \label{K_L04211-1v}\edtext{Donnerſtag
                  24. früh in Brioni\oindex{Brijuni@\textbf{Brijuni}|pw}}{\lemma{\textnormal{\emph{Donnerstag … Brioni}}}\Cendnote{\textnormal{Das trat ein, vgl. A. S.: \emph{Wiener Schnitzler}, 24. 7. 1913.}}}\label{K_L04211-1} zu{ }ſein; \label{K_L04211-2v}\edtext{Lili\pwindex{Cappellini, Lili 13.\,9.\,1909 Wien – 26.\,7.\,1928 Venedig@\textsc{Cappellini, Lili} (13.\,9.\,1909 Wien – 26.\,7.\,1928 Venedig)|pw}{ }ſchon So{\geminationn}tag}{\lemma{\textnormal{\emph{Lili schon Sonntag}}}\Cendnote{\textnormal{Vgl. A. S.: \emph{Tagebuch}, 19. 7. 1913.}}}\label{K_L04211-2}.
               Schreiben Sie mir bitte ev. noch hieher ein Wort, wie lange Sie noch in A.\oindex{Opatija@\textbf{Opatija}, \emph{Hauptstadt}|pw} bleiben u. \label{K_L04223-66v}\edtext{wann Sie zu uns nach Brioni\oindex{Brijuni@\textbf{Brijuni}|pw} ko{\geminationm}en}{\lemma{\textnormal{\emph{wann … kommen}}}\Cendnote{\textnormal{Schwarzkopf\pwindex{Schwarzkopf, Gustav 7.\,11.\,1853 Wien – 13.\,11.\,1939 ebd.@\textsc{Schwarzkopf, Gustav} (7.\,11.\,1853 Wien – 13.\,11.\,1939 ebd.), \emph{Schriftsteller}|pwk} kam
                  mehrfach zu Besuch, das erste Mal am 25. 7. 1913.}}}\label{K_L04223-66}?
               Vielen Dank für die Karte aus Abz\oindex{Opatija@\textbf{Opatija}, \emph{Hauptstadt}|pw}.\pend
           \pstart Auf Wiederſehen. Wir grüßen Sie herzlich. Ihr \spacefill\mbox{A.}\pend{}\selectlanguage{ngerman}\endnumbering\briefempfaengerindex{Schwarzkopf, Gustav@\textsc{Schwarzkopf, Gustav}!zzzSchnitzler, Arthur@\emph{von Arthur Schnitzler}!1913-07-181@{18. 7. 1913}|)be}\mylabel{L04223h}
\begin{anhang}
\end{anhang}\newcommand{\dateiname}{L04223}\newcommand{\titel}{Arthur Schnitzler an Gustav Schwarzkopf, 18. 7. 1913}\newcommand{\editorInnen}{Herausgegeben von Jahnke, SelmaMüller, Martin Anton}%% latex-leseansicht-abspann.tex
%% Abspann für die Leseansicht.
%% Der Schalter \ifkorrekturansicht ist bereits durch den Vorspann gesetzt.

%% latex-abspann.tex
%% Gemeinsamer Abspann für Korrekturansicht und Leseansicht.
%% Setzt den Schalter \ifkorrekturansicht voraus (gesetzt in den
%% einbindenden Dateien latex-korrekturansicht-abspann.tex bzw.
%% latex-leseansicht-abspann.tex).
%% ---------------------------------------------------------------

\normalsize

% Das esempio-Environment wird nur in der Leseansicht benötigt
\ifkorrekturansicht\else
\newenvironment{esempio}[3]%
{
    \vspace{1.5ex}
    \rlap{\underline{#1}}
    \par
    \setlength{\parindent}{0cm}
    \nopagebreak
    \leftskip=#2cm
    \rightskip=#3cm
}
{
    \par
}
\fi

\doendnotes{C}
\bigskip
\vfill

\clearpage

\footnotesize

\ifkorrekturansicht
  \lohead{\textsc{register}}
\fi

% theindex-Environment neu definieren ohne reledmac
\makeatletter
\renewenvironment{theindex}{%
  \ifkorrekturansicht
    \section*{\indexname}%
  \else
    \subsubsection*{Index der erwähnten Entitäten}%
  \fi
  \setlength{\parindent}{0pt}%
  \setlength{\parskip}{0pt plus 0.3pt}%
  \let\item\@idxitem
}{%
  \ifkorrekturansicht\clearpage\fi
}
\makeatother

\IfFileExists{\jobname-pw.ind}{\input{\jobname-pw.ind}}{}

% Quellenangabe nur in der Leseansicht
\ifkorrekturansicht\else
% Fallback-Definitionen, falls die .tex-Datei \titel etc. nicht gesetzt hat
\providecommand{\titel}{}
\providecommand{\editorInnen}{}
\providecommand{\dateiname}{\jobname}

\vspace{3cm}

\vfill

\footnotesize
\textsc{Quelle}: \titel. Herausgegeben von {\editorInnen}. In: \emph{Arthur Schnitzler: Briefwechsel mit Autorinnen und Autoren}.
 Digitale Edition, https://schnitzler-briefe.acdh.oeaw.ac.at/{\dateiname}.html (Stand \today)
\fi

\end{document}


