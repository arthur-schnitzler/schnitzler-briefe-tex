%% latex-korrekturansicht-vorspann.tex
%% Vorspann für die Korrekturansicht.
%% Lädt die gemeinsame Datei latex-vorspann.tex mit gesetztem Schalter.

\newif\ifkorrekturansicht
\korrekturansichttrue

\input{../tex-inputs/latex-vorspann}


\section[Stefan Zweig an Arthur Schnitzler, 29. 8. 1911]{L03633 Stefan Zweig an Arthur Schnitzler, 29. 8. 1911}
\nopagebreak\mylabel{L03633v}
\rehead{ }\normalsize\beginnumbering\briefempfaengerindex{Schnitzler, Arthur@\textsc{Schnitzler, Arthur}!zzzZweig, Stefan@\emph{von Stefan Zweig}!1911-08-291@{29. 8. 1911}|(be}
\toendnotes[C]{\smallbreak\pagebreak[2]}\Standort{CUL, Schnitzler, B 118.}
\physDesc{Bildpostkarte, 457 Zeichen
\newline{}Handschrift: schwarze Tinte, lateinische Kurrent
\newline{}Versand: Stempel: »\nobreak{}\oindex{Weimar@\textbf{Weimar}, \emph{A.ADM3}|pwk}Weimar, 29. 8. 11, 7—8 N\nobreak{}«.  
\newline{}Schnitzler: mit Bleistift »\textsc{Zweig}« }
\buchAbdrucke{\weitereDrucke{Stefan Zweig: \emph{Briefwechsel mit Hermann Bahr, Sigmund Freud, Rainer Maria
                        Rilke und Arthur Schnitzler}. Frankfurt am Main: \emph{S. Fischer} 1987, S. 366.} }\toendnotes[C]{\smallbreak}\pstart{}{\pb}D\textsuperscript{r} Artur
                  Schnitzler\pend{}\pstart{}Wien – Cottage\oindex{Waehringer Cottage@\textbf{Währinger Cottage}, \emph{Teil eines besiedelten Ortes (A.BSOX)}|pw}\pend{}\pstart{}Sternwartestrasse 71\oindex{Sternwartestrasse 71@\textbf{Sternwartestraße 71}, \emph{Wohngebäude (K.WHS)}|pw}\pend{}{\bigskip}
\pstart
           \noindent{}\centering{}{\pb}\textcolor{gray}{\textbf{Weimar, Goethes Gartenhaus\oindex{Gartenhaus [Goethe]@\textbf{Gartenhaus [Goethe]}, \emph{Gebäude (K.GBD)}|pw}.}}\pend
           \stanza{}\textcolor{gray}{\textbf{Übermüthig siehts nicht aus}}\textcolor{gray}{\textbf{Dieses stille Gartenhaus\oindex{Gartenhaus [Goethe]@\textbf{Gartenhaus [Goethe]}, \emph{Gebäude (K.GBD)}|pwv}}}\textcolor{gray}{\textbf{Allen die darin verkehrt}}\textcolor{gray}{\textbf{Ward ein guter Muth bescheert}}\textcolor{gray}{\textbf{Goethe\pwindex{Goethe, Johann Wolfgang von 1749-08-28 – 1832-03-22@\textsc{Goethe, Johann Wolfgang von} (1749-08-28 – 1832-03-22), \emph{Schriftsteller/Schriftstellerin}|pw}{ }1828}}\stanzaend{}\vspace{1em}
\pstart
           \noindent{}{\pb}Verehrter Herr Doktor, ich weiss nicht, ob Sie \label{K_L03633-1v}\edtext{schon einmal hier}{\lemma{\textnormal{\emph{schon einmal hier}}}\Cendnote{\textnormal{Schnitzler
                  hatte seine ›Reverenzreise‹ bereits vom 12. 8. 1906 bis zum 16. 8. 1906 gemacht,
                  kam aber kein zweites Mal nach Weimar\oindex{Weimar@\textbf{Weimar}, \emph{A.ADM3}|pwk}.}}}\label{K_L03633-1}
               waren: man kanns auch als Sommeraufenthalt nehmen, statt als blosse Reverenzreise, so
               wundervoll still ist's jetzt in den Gängen an der Ilm\oindex{Ilm@\textbf{Ilm}, \emph{H.STM}|pw}. Ich grüsse Sie und Ihre liebe Frau\pwindex{Schnitzler, Olga 17.01.1882 – 13.01.1970@\textsc{Schnitzler, Olga} (17.01.1882 – 13.01.1970), \emph{Schauspieler/Schauspielerin, Sänger/Sängerin}|pwv} herzlichst in alter Ergebenheit\pend
           \pstart \spacefill\mbox{Stefan Zweig}\pend{}
\pstart
           \noindent{}Wie \uline{wundervoll} ist Ihre \label{K_L03633-2v}\edtext{Hirtenflöte\pwindex{Hirtenfloete. Novelle@\emph{Die Hirtenflöte. Novelle}|pw}}{\lemma{\textnormal{\emph{Hirtenflöte}}}\Cendnote{\textnormal{Arthur Schnitzler: \emph{Die Hirtenflöte.
                           Novelle}\pwindex{Hirtenfloete. Novelle@\emph{Die Hirtenflöte. Novelle}|pwk}. In: \emph{Die neue
                        Rundschau}\pwindex{Neue Deutsche Rundschau@\emph{Neue Deutsche Rundschau}|pwk}, Jg. 22, H. 9, September 1911,
                        1249–1273. Zweigs\pwindex{Zweig, Stefan 28.11.1881 – 23.02.1942@\textsc{Zweig, Stefan} (28.11.1881 – 23.02.1942), \emph{Schriftsteller/Schriftstellerin}|pwk} Brief belegt,
                     dass das September-Heft bereits in der zweiten Hälfte des August ausgeliefert worden war.}}}\label{K_L03633-2}! Ich musste mir {\pb}es auf die Reise mitnehmen, um es beim
                  zweiten Lesen noch inniger zu geniessen.\pend
           \selectlanguage{ngerman}\endnumbering\briefempfaengerindex{Schnitzler, Arthur@\textsc{Schnitzler, Arthur}!zzzZweig, Stefan@\emph{von Stefan Zweig}!1911-08-291@{29. 8. 1911}|)be}\mylabel{L03633h}  \normalsize

\doendnotes{C}
\bigskip
\vfill

\clearpage

\footnotesize

\lohead{\textsc{register}}

% Definiere theindex-Environment komplett neu ohne reledmac
\makeatletter
\renewenvironment{theindex}{%
  \section*{\indexname}%
  \setlength{\parindent}{0pt}%
  \setlength{\parskip}{0pt plus 0.3pt}%
  \let\item\@idxitem
}{%
  \clearpage
}
\makeatother

\IfFileExists{\jobname-pw.ind}{\input{\jobname-pw.ind}}{}

\end{document}

      