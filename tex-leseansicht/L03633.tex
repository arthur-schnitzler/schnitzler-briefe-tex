%% latex-leseansicht-vorspann.tex
%% Vorspann für die Leseansicht.
%% Lädt die gemeinsame Datei latex-vorspann.tex mit nicht gesetztem Schalter.

\newif\ifkorrekturansicht
\korrekturansichtfalse

\input{../tex-inputs/latex-vorspann}


\section[Stefan Zweig an Arthur Schnitzler, 29. 8. 1911]{L03633 Stefan Zweig an Arthur Schnitzler, 29. 8. 1911}
\nopagebreak\mylabel{L03633v}
\rehead{ }\normalsize\beginnumbering\briefempfaengerindex{Schnitzler, Arthur@\textsc{Schnitzler, Arthur}!zzzZweig, Stefan@\emph{von Stefan Zweig}!1911-08-291@{29. 8. 1911}|(be}
\toendnotes[C]{\smallbreak\pagebreak[2]}
\correspDesc{Versand  durch Stefan Zweig am 29. 8. 1911 in Weimar
\newline{}Erhalt  durch Arthur Schnitzler im Zeitraum [30. 8. 1911
                  – 3. 9. 1911?] in Wien}\toendnotes[C]{\smallbreak}
\Standort{CUL, Schnitzler, B 118.}
\physDesc{Bildpostkarte, 457 Zeichen
\newline{}Handschrift: schwarze Tinte, lateinische Kurrent
\newline{}Versand: Stempel: »\nobreak{}\oindex{Weimar@\textbf{Weimar}, \emph{Verwaltungsgebiet}|pwk}Weimar, 29. 8. 11, 7–8 N\nobreak{}«.  
\newline{}Schnitzler: mit Bleistift »\textsc{Zweig}« }
\buchAbdrucke{\weitereDrucke{Stefan Zweig: \emph{Briefwechsel mit Hermann Bahr, Sigmund Freud, Rainer Maria
                        Rilke und Arthur Schnitzler}. Herausgegeben von Jeffrey B. Berlin, Hans-Ulrich Lindken und Donald A. Prater. Frankfurt am Main: \emph{S. Fischer} 1987, S. 366.} }\toendnotes[C]{\smallbreak}\pstart{}{\pb}D\textsuperscript{r} Artur
                  Schnitzler\pend{}\pstart{}Wien – Cottage\oindex{Wien@\textbf{Wien}!XVIII., Währing@\textbf{XVIII., Währing}!Währinger Cottage@\textbf{Währinger Cottage}, \emph{Teil eines besiedelten Ortes}|pw}\pend{}\pstart{}Sternwartestrasse 71\oindex{Wien@\textbf{Wien}!XVIII., Währing@\textbf{XVIII., Währing}!Sternwartestraße 71@\textbf{Sternwartestraße 71}, \emph{Wohngebäude}|pw}\pend{}{\bigskip}
\pstart
           \noindent{}\centering{}{\pb}\textcolor{gray}{\textbf{Weimar, Goethes Gartenhaus\oindex{Gartenhaus [Goethe]@\textbf{Gartenhaus [Goethe]}, \emph{Gebäude}|pw}.}}\pend
           \stanza{}\textcolor{gray}{\textbf{Übermüthig siehts nicht aus}}\newverse{}\textcolor{gray}{\textbf{Dieses stille Gartenhaus\oindex{Gartenhaus [Goethe]@\textbf{Gartenhaus [Goethe]}, \emph{Gebäude}|pwv}}}\newverse{}\textcolor{gray}{\textbf{Allen die darin verkehrt}}\newverse{}\textcolor{gray}{\textbf{Ward ein guter Muth bescheert}}\newverse{}\textcolor{gray}{\textbf{Goethe\pwindex{Goethe, Johann Wolfgang von 28.\,8.\,1749 Frankfurt am Main – 22.\,3.\,1832 Weimar@\textsc{Goethe, Johann Wolfgang von} (28.\,8.\,1749 Frankfurt am Main – 22.\,3.\,1832 Weimar), \emph{Schriftsteller}|pw}{ }1828}}\stanzaend{}\vspace{1em}
\pstart
           \noindent{}{\pb}Verehrter Herr Doktor, ich weiss nicht, ob Sie \label{K_L03633-1v}\edtext{schon einmal hier}{\lemma{\textnormal{\emph{schon einmal hier}}}\Cendnote{\textnormal{Schnitzler
                  hatte seine ›Reverenzreise‹ bereits vom 12. 8. 1906 bis zum 16. 8. 1906 gemacht,
                  kam aber kein zweites Mal nach Weimar\oindex{Weimar@\textbf{Weimar}, \emph{Verwaltungsgebiet}|pwk}.}}}\label{K_L03633-1}
               waren: man kanns auch als Sommeraufenthalt nehmen, statt als blosse Reverenzreise, so
               wundervoll still ist’s jetzt in den Gängen an der Ilm\oindex{Ilm@\textbf{Ilm}, \emph{Fluss}|pw}. Ich grüsse Sie und Ihre liebe Frau\pwindex{Schnitzler, Olga 17.\,1.\,1882 Wien – 13.\,1.\,1970 Lugano@\textsc{Schnitzler, Olga} (17.\,1.\,1882 Wien – 13.\,1.\,1970 Lugano), \emph{Schauspielerin, Sängerin}|pwv} herzlichst in alter Ergebenheit\pend
           \pstart \spacefill\mbox{Stefan Zweig}\pend{}
\pstart
           \noindent{}Wie \uline{wundervoll} ist Ihre \label{K_L03633-2v}\edtext{Hirtenflöte\pwindex{Schnitzler, Arthur 15.\,5.\,1862 Wien – 21.\,10.\,1931 ebd.@\textsc{Schnitzler, Arthur} (15.\,5.\,1862 Wien – 21.\,10.\,1931 ebd.), \emph{Schriftsteller, Mediziner}!Hirtenflöte. Novelle@\strich\emph{Die Hirtenflöte. Novelle}|pw}}{\lemma{\textnormal{\emph{Hirtenflöte}}}\Cendnote{\textnormal{Arthur Schnitzler: \emph{Die Hirtenflöte.
                           Novelle}\pwindex{Schnitzler, Arthur 15.\,5.\,1862 Wien – 21.\,10.\,1931 ebd.@\textsc{Schnitzler, Arthur} (15.\,5.\,1862 Wien – 21.\,10.\,1931 ebd.), \emph{Schriftsteller, Mediziner}!Hirtenflöte. Novelle@\strich\emph{Die Hirtenflöte. Novelle}|pwk}. In: \emph{Die neue
                        Rundschau}\pwindex{Neue Deutsche Rundschau@\emph{Neue Deutsche Rundschau}|pwk}, Jg. 22, H. 9, September 1911,
                        1249–1273. Zweigs\pwindex{Zweig, Stefan 28.\,11.\,1881 Wien – 23.\,2.\,1942 Petrópolis@\textsc{Zweig, Stefan} (28.\,11.\,1881 Wien – 23.\,2.\,1942 Petrópolis), \emph{Schriftsteller}|pwk} Brief belegt,
                     dass das September-Heft bereits in der zweiten Hälfte des August ausgeliefert worden war.}}}\label{K_L03633-2}! Ich musste mir {\pb}es auf die Reise mitnehmen, um es beim
                  zweiten Lesen noch inniger zu geniessen.\pend
           \selectlanguage{ngerman}\endnumbering\briefempfaengerindex{Schnitzler, Arthur@\textsc{Schnitzler, Arthur}!zzzZweig, Stefan@\emph{von Stefan Zweig}!1911-08-291@{29. 8. 1911}|)be}\mylabel{L03633h}  \newcommand{\dateiname}{L03633}\newcommand{\titel}{Stefan Zweig an Arthur Schnitzler, 29. 8. 1911}\newcommand{\editorInnen}{Selma Jahnke und Martin Anton Müller}%% latex-leseansicht-abspann.tex
%% Abspann für die Leseansicht.
%% Der Schalter \ifkorrekturansicht ist bereits durch den Vorspann gesetzt.

%% latex-abspann.tex
%% Gemeinsamer Abspann für Korrekturansicht und Leseansicht.
%% Setzt den Schalter \ifkorrekturansicht voraus (gesetzt in den
%% einbindenden Dateien latex-korrekturansicht-abspann.tex bzw.
%% latex-leseansicht-abspann.tex).
%% ---------------------------------------------------------------

\normalsize

% Das esempio-Environment wird nur in der Leseansicht benötigt
\ifkorrekturansicht\else
\newenvironment{esempio}[3]%
{
    \vspace{1.5ex}
    \rlap{\underline{#1}}
    \par
    \setlength{\parindent}{0cm}
    \nopagebreak
    \leftskip=#2cm
    \rightskip=#3cm
}
{
    \par
}
\fi

\doendnotes{C}
\bigskip
\vfill

\clearpage

\footnotesize

\ifkorrekturansicht
  \lohead{\textsc{register}}
\fi

% theindex-Environment neu definieren ohne reledmac
\makeatletter
\renewenvironment{theindex}{%
  \ifkorrekturansicht
    \section*{\indexname}%
  \else
    \subsubsection*{Index der erwähnten Entitäten}%
  \fi
  \setlength{\parindent}{0pt}%
  \setlength{\parskip}{0pt plus 0.3pt}%
  \let\item\@idxitem
}{%
  \ifkorrekturansicht\clearpage\fi
}
\makeatother

\IfFileExists{\jobname-pw.ind}{\input{\jobname-pw.ind}}{}

% Quellenangabe nur in der Leseansicht
\ifkorrekturansicht\else
% Fallback-Definitionen, falls die .tex-Datei \titel etc. nicht gesetzt hat
\providecommand{\titel}{}
\providecommand{\editorInnen}{}
\providecommand{\dateiname}{\jobname}

\vspace{3cm}

\vfill

\footnotesize
\textsc{Quelle}: \titel. Herausgegeben von {\editorInnen}. In: \emph{Arthur Schnitzler: Briefwechsel mit Autorinnen und Autoren}.
 Digitale Edition, https://schnitzler-briefe.acdh.oeaw.ac.at/{\dateiname}.html (Stand \today)
\fi

\end{document}


