%% latex-leseansicht-vorspann.tex
%% Vorspann für die Leseansicht.
%% Lädt die gemeinsame Datei latex-vorspann.tex mit nicht gesetztem Schalter.

\newif\ifkorrekturansicht
\korrekturansichtfalse

\input{../tex-inputs/latex-vorspann}


\section[Bertha von Suttner an Arthur Schnitzler, 4. 11. 1913]{L02156 Bertha von Suttner an Arthur Schnitzler, 4. 11. 1913}
\nopagebreak\mylabel{L02156v}
\rehead{ }\normalsize\beginnumbering\briefempfaengerindex{Schnitzler, Arthur@\textsc{Schnitzler, Arthur}!zzzSuttner, Bertha von@\emph{von Bertha von Suttner}!1913-11-041@{4. 11. 1913}|(be}
\toendnotes[C]{\smallbreak\pagebreak[2]}
\correspDesc{Versand  durch Bertha von Suttner am 4. 11. 1913 in Wien
\newline{}Übermittlung  am 5. 11. 1913 in Wien
\newline{}Erhalt  durch Arthur Schnitzler im Zeitraum [5. 11. 1913
                  – 7. 11. 1913?] in Wien}\toendnotes[C]{\smallbreak}
\Standort{CUL, Schnitzler, B 104.}
\physDesc{Postkarte, 446 Zeichen
\newline{}Handschrift: schwarze Tinte, deutsche Kurrent
\newline{}Versand: Stempel: »\nobreak{}\oindex{I., Innere Stadt@\textbf{I., Innere Stadt}, \emph{Verwaltungsgebiet}|pwk}1/1 Wien 1, 5. XI. 13, VII\nobreak{}«.  
\newline{}Schnitzler: mit rotem Buntstift eine Unterstreichung }\Standort{DLA, A:Schnitzler, HS.NZ85.1.4773.}
\physDesc{maschinenschriftliche Abschrift, 1 Blatt, 1 Seite, 446 Zeichen
\newline{}Schreibmaschine}\toendnotes[C]{\smallbreak}\pstart{}{\pb}\textsc{Herrn}\pend{}\pstart{}D\textsuperscript{r}{ }\textsc{Arthur}\pend{}\pstart{}\textsc{Schnitzler}\pend{}\pstart{}XVIII\oindex{XVIII., Währing@\textbf{XVIII., Währing}, \emph{Verwaltungsgebiet}|pw}\pend{}\pstart{}\textsc{\label{K_L02156-1v}\edtext{Sternwartegasse}{\lemma{\textnormal{\emph{Sternwartegasse}}}\Cendnote{\textnormal{richtig: Sternwartestraße\oindex{Wien@\textbf{Wien}!XVIII., Währing@\textbf{XVIII., Währing}!Sternwartestraße 71@\textbf{Sternwartestraße 71}, \emph{Wohngebäude}|pwk}}}}\label{K_L02156-1} 71\oindex{Wien@\textbf{Wien}!XVIII., Währing@\textbf{XVIII., Währing}!Sternwartestraße 71@\textbf{Sternwartestraße 71}, \emph{Wohngebäude}|pw}}\pend{}{\bigskip}\vspace{1em}
\pstart
           \centering{}{\pb}4/11 13\pend
           \vspace{0.5em}
\pstart
           Vielen Dank! Habe jede Zeile der intereſſanten Sendung geleſen. Ueber manches auch
               mich gründlich geärgert; beſonders über die Einſchachtlung, Etikettierg, Limitierung.
               Damit{ }ſoll man doch den fünf oder{ }ſechs Vertretern der Weltliteratur, die man
               jeweilig hat, fern bleiben!\pend
           
\pstart
           Künftige Woche mache ich mich an die \label{K_L02156-2v}\edtext{Arbeit}{\lemma{\textnormal{\emph{Arbeit}}}\Cendnote{\textnormal{Géza Baracs\pwindex{Baracs, Géza *~26.\,5.\,1862@\textsc{Baracs, Géza} (*~26.\,5.\,1862), \emph{Journalist, Lehrer, Priester}|pwk} gab unter seinem Pseudonym
                  »Clément Deltour« auf Subskription eine Reihe »Unsere Zeitgenossen«/»Nos
                  contemporains« heraus. Diese ist sehr selten, ein Beitrag über Schnitzler konnte nicht nachgewiesen werden.}}}\label{K_L02156-2}.\pend
           
\pstart
           Meinen \label{K_L02156-3v}\edtext{Beſuch}{\lemma{\textnormal{\emph{Besuch}}}\Cendnote{\textnormal{Vgl. A. S.: \emph{Tagebuch}, 29. 10. 1913.
               }}}\label{K_L02156-3} in der Sternwartegaſſe\oindex{Wien@\textbf{Wien}!XVIII., Währing@\textbf{XVIII., Währing}!Sternwartestraße 71@\textbf{Sternwartestraße 71}, \emph{Wohngebäude}|pw} habe ich{ }ſehr
               genoſſen.\pend
           
\pstart
           Auf bald!{\\[\baselineskip]}\spacefill\mbox{B. Suttner}\pend
           \leftskip=0em{}\selectlanguage{ngerman}\endnumbering\briefempfaengerindex{Schnitzler, Arthur@\textsc{Schnitzler, Arthur}!zzzSuttner, Bertha von@\emph{von Bertha von Suttner}!1913-11-041@{4. 11. 1913}|)be}\mylabel{L02156h}  \newcommand{\dateiname}{L02156}\newcommand{\titel}{Bertha von Suttner an Arthur Schnitzler, 4. 11. 1913}\newcommand{\editorInnen}{Martin Anton Müller und Gerd-Hermann Susen}%% latex-leseansicht-abspann.tex
%% Abspann für die Leseansicht.
%% Der Schalter \ifkorrekturansicht ist bereits durch den Vorspann gesetzt.

%% latex-abspann.tex
%% Gemeinsamer Abspann für Korrekturansicht und Leseansicht.
%% Setzt den Schalter \ifkorrekturansicht voraus (gesetzt in den
%% einbindenden Dateien latex-korrekturansicht-abspann.tex bzw.
%% latex-leseansicht-abspann.tex).
%% ---------------------------------------------------------------

\normalsize

% Das esempio-Environment wird nur in der Leseansicht benötigt
\ifkorrekturansicht\else
\newenvironment{esempio}[3]%
{
    \vspace{1.5ex}
    \rlap{\underline{#1}}
    \par
    \setlength{\parindent}{0cm}
    \nopagebreak
    \leftskip=#2cm
    \rightskip=#3cm
}
{
    \par
}
\fi

\doendnotes{C}
\bigskip
\vfill

\clearpage

\footnotesize

\ifkorrekturansicht
  \lohead{\textsc{register}}
\fi

% theindex-Environment neu definieren ohne reledmac
\makeatletter
\renewenvironment{theindex}{%
  \ifkorrekturansicht
    \section*{\indexname}%
  \else
    \subsubsection*{Index der erwähnten Entitäten}%
  \fi
  \setlength{\parindent}{0pt}%
  \setlength{\parskip}{0pt plus 0.3pt}%
  \let\item\@idxitem
}{%
  \ifkorrekturansicht\clearpage\fi
}
\makeatother

\IfFileExists{\jobname-pw.ind}{\input{\jobname-pw.ind}}{}

% Quellenangabe nur in der Leseansicht
\ifkorrekturansicht\else
% Fallback-Definitionen, falls die .tex-Datei \titel etc. nicht gesetzt hat
\providecommand{\titel}{}
\providecommand{\editorInnen}{}
\providecommand{\dateiname}{\jobname}

\vspace{3cm}

\vfill

\footnotesize
\textsc{Quelle}: \titel. Herausgegeben von {\editorInnen}. In: \emph{Arthur Schnitzler: Briefwechsel mit Autorinnen und Autoren}.
 Digitale Edition, https://schnitzler-briefe.acdh.oeaw.ac.at/{\dateiname}.html (Stand \today)
\fi

\end{document}


