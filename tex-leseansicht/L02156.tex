%% latex-leseansicht-vorspann.tex
%% Vorspann für die Leseansicht.
%% Lädt die gemeinsame Datei latex-vorspann.tex mit nicht gesetztem Schalter.

\newif\ifkorrekturansicht
\korrekturansichtfalse

\input{../tex-inputs/latex-vorspann}


         
         \newcommand{\erwaehntePersonen}{Personen: Géza Baracs}
         \newcommand{\erwaehnteOrte}{Orte: I., Innere Stadt, Sternwartestraße, Wien, XVIII., Währing}
         \newcommand{\erwaehnteWerke}{
               \section[Bertha von Suttner an Arthur Schnitzler, 4. 11. 1913]{ Bertha von Suttner an Arthur Schnitzler,
                    4. 11. 1913}\nopagebreak\mylabel{v}\rehead{ }\begin{ledgroupsized}[t]{13cm}\normalsize\beginnumbering \toendnotes[C]{\smallbreak\pagebreak[2]} \Standort{CUL, Schnitzler, B 104.}
\physDesc{Postkarte
\newline{}Handschrift: schwarze Tinte, deutsche Kurrent\newline{}Versand: Stempel: »\nobreak{}\oindex{I., Innere Stadt@\textbf{I., Innere Stadt}|pwk}1/1 Wien 1, 5. XI. 13, VII\nobreak{}«.  
\newline{}Schnitzler: mit rotem Buntstift eine Unterstreichung }\Standort{DLA, A:Schnitzler, HS.NZ85.1.4773.}
\physDesc{1 Blatt, 1 Seite, maschinelle Abschrift}\toendnotes[C]{\smallbreak}\pstart{}{\pb}\textsc{Herrn}\pend{}\pstart{}D\textsuperscript{r}{ }\textsc{Arthur}\pend{}\pstart{}\textsc{Schnitzler}\pend{}\pstart{}XVIII\oindex{XVIII., Waehring@\textbf{XVIII., Währing}|pw}\pend{}\pstart{}\textsc{\label{K_L02156_1v}\edtext{Sternwartegasse}{\lemma{\textnormal{\emph{Sternwartegasse}}}\Cendnote{\textnormal{richtig: Sternwartestraße\oindex{Sternwartestrasse@\textbf{Sternwartestraße}|pwk}}}}\label{K_L02156_1h} 71\oindex{Sternwartestrasse@\textbf{Sternwartestraße}|pw}}\pend{}{\bigskip}\pstart
           \centering{}{\pb}4/11 13\pend
           \pstart
           Vielen Dank! Habe jede Zeile der intereſſanten Sendung geleſen. Ueber manches
                    auch mich gründlich geärgert; beſonders über die Einſchachtlung, Etikettierg,
                    Limitierung. Damit ſoll man doch den fünf oder ſechs Vertretern der
                    Weltliteratur, die man jeweilig hat, fern bleiben!\pend
           \pstart
           Künftige Woche mache ich mich an die \label{K_L02156_2v}\edtext{Arbeit}{\lemma{\textnormal{\emph{Arbeit}}}\Cendnote{\textnormal{Géza
                            Baracs\pwindex{Baracs, Geza *~26.05.1862@\textsc{Baracs, Géza} (*~26.05.1862), \emph{Journalist, Lehrer, Priester}|pwk} gab unter seinem Pseudonym »Clément Deltour« auf
                        Subskription eine Reihe »Unsere Zeitgenossen«/»Nos contemporains« heraus.
                        Diese ist sehr selten, ein Beitrag über Schnitzler\pwindex{Schnitzler, Arthur 15.05.1862 – 21.10.1931@\textsc{Schnitzler, Arthur} (15.05.1862 – 21.10.1931), \emph{Schriftsteller, Mediziner}|pwk} konnte nicht nachgewiesen werden.}}}\label{K_L02156_2h}.\pend
           \pstart
           Meinen \label{K_L02156_3v}\edtext{Beſuch}{\lemma{\textnormal{\emph{Beſuch}}}\Cendnote{\textnormal{vgl. A. S.: \emph{Tagebuch}, 29. 10. 1913}}}\label{K_L02156_3h} in der Sternwartegaſſe\oindex{Sternwartestrasse@\textbf{Sternwartestraße}|pw} habe ich ſehr
                    genoſſen.\pend
           \pstart
           Auf bald!{\\[\baselineskip]}\spacefill\mbox{B. Suttner}\pend
           \leftskip=0em{}
         
         \endnumbering\mylabel{h}\end{ledgroupsized}  \newcommand{\dateiname}{L02156}\newcommand{\titel}{Bertha von Suttner an Arthur Schnitzler, 4. 11. 1913}\newcommand{\editorInnen}{Martin Anton Müller und Gerd-Hermann Susen}%% latex-leseansicht-abspann.tex
%% Abspann für die Leseansicht.
%% Der Schalter \ifkorrekturansicht ist bereits durch den Vorspann gesetzt.

%% latex-abspann.tex
%% Gemeinsamer Abspann für Korrekturansicht und Leseansicht.
%% Setzt den Schalter \ifkorrekturansicht voraus (gesetzt in den
%% einbindenden Dateien latex-korrekturansicht-abspann.tex bzw.
%% latex-leseansicht-abspann.tex).
%% ---------------------------------------------------------------

\normalsize

% Das esempio-Environment wird nur in der Leseansicht benötigt
\ifkorrekturansicht\else
\newenvironment{esempio}[3]%
{
    \vspace{1.5ex}
    \rlap{\underline{#1}}
    \par
    \setlength{\parindent}{0cm}
    \nopagebreak
    \leftskip=#2cm
    \rightskip=#3cm
}
{
    \par
}
\fi

\doendnotes{C}
\bigskip
\vfill

\clearpage

\footnotesize

\ifkorrekturansicht
  \lohead{\textsc{register}}
\fi

% theindex-Environment neu definieren ohne reledmac
\makeatletter
\renewenvironment{theindex}{%
  \ifkorrekturansicht
    \section*{\indexname}%
  \else
    \subsubsection*{Index der erwähnten Entitäten}%
  \fi
  \setlength{\parindent}{0pt}%
  \setlength{\parskip}{0pt plus 0.3pt}%
  \let\item\@idxitem
}{%
  \ifkorrekturansicht\clearpage\fi
}
\makeatother

\IfFileExists{\jobname-pw.ind}{\input{\jobname-pw.ind}}{}

% Quellenangabe nur in der Leseansicht
\ifkorrekturansicht\else
% Fallback-Definitionen, falls die .tex-Datei \titel etc. nicht gesetzt hat
\providecommand{\titel}{}
\providecommand{\editorInnen}{}
\providecommand{\dateiname}{\jobname}

\vspace{3cm}

\vfill

\footnotesize
\textsc{Quelle}: \titel. Herausgegeben von {\editorInnen}. In: \emph{Arthur Schnitzler: Briefwechsel mit Autorinnen und Autoren}.
 Digitale Edition, https://schnitzler-briefe.acdh.oeaw.ac.at/{\dateiname}.html (Stand \today)
\fi

\end{document}


      