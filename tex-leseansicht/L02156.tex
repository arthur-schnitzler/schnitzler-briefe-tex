%% latex-korrekturansicht-vorspann.tex
%% Vorspann für die Korrekturansicht.
%% Lädt die gemeinsame Datei latex-vorspann.tex mit gesetztem Schalter.

\newif\ifkorrekturansicht
\korrekturansichttrue

\input{../tex-inputs/latex-vorspann}


\section[Bertha von Suttner an Arthur Schnitzler, 4. 11. 1913]{L02156 Bertha von Suttner an Arthur Schnitzler, 4. 11. 1913}
\nopagebreak\mylabel{L02156v}
\rehead{ }\normalsize\beginnumbering\briefempfaengerindex{Schnitzler, Arthur@\textsc{Schnitzler, Arthur}!zzzSuttner, Bertha von@\emph{von Bertha von Suttner}!1913-11-041@{4. 11. 1913}|(be}
\toendnotes[C]{\smallbreak\pagebreak[2]}\Standort{CUL, Schnitzler, B 104.}
\physDesc{Postkarte, 446 Zeichen
\newline{}Handschrift: schwarze Tinte, deutsche Kurrent
\newline{}Versand: Stempel: »\nobreak{}\oindex{I., Innere Stadt@\textbf{I., Innere Stadt}, \emph{A.ADM3}|pwk}1/1 Wien 1, 5. XI. 13, VII\nobreak{}«.  
\newline{}Schnitzler: mit rotem Buntstift eine Unterstreichung }\Standort{DLA, A:Schnitzler, HS.NZ85.1.4773.}
\physDesc{maschinenschriftliche Abschrift1 Blatt, 1 Seite, 446 Zeichen
\newline{}Schreibmaschine}\toendnotes[C]{\smallbreak}\pstart{}{\pb}\textsc{Herrn}\pend{}\pstart{}D\textsuperscript{r}{ }\textsc{Arthur}\pend{}\pstart{}\textsc{Schnitzler}\pend{}\pstart{}XVIII\oindex{XVIII., Waehring@\textbf{XVIII., Währing}, \emph{A.ADM3}|pw}\pend{}\pstart{}\textsc{\label{K_L02156-1v}\edtext{Sternwartegasse}{\lemma{\textnormal{\emph{Sternwartegasse}}}\Cendnote{\textnormal{richtig: Sternwartestraße\oindex{Sternwartestrasse 71@\textbf{Sternwartestraße 71}, \emph{Wohngebäude (K.WHS)}|pwk}}}}\label{K_L02156-1} 71\oindex{Sternwartestrasse 71@\textbf{Sternwartestraße 71}, \emph{Wohngebäude (K.WHS)}|pw}}\pend{}{\bigskip}\vspace{1em}
\pstart
           \centering{}{\pb}4/11 13\pend
           \vspace{0.5em}
\pstart
           Vielen Dank! Habe jede Zeile der intereſſanten Sendung geleſen. Ueber manches auch
               mich gründlich geärgert; beſonders über die Einſchachtlung, Etikettierg, Limitierung.
               Damit ſoll man doch den fünf oder ſechs Vertretern der Weltliteratur, die man
               jeweilig hat, fern bleiben!\pend
           
\pstart
           Künftige Woche mache ich mich an die \label{K_L02156-2v}\edtext{Arbeit}{\lemma{\textnormal{\emph{Arbeit}}}\Cendnote{\textnormal{Géza Baracs\pwindex{Baracs, Geza *~26.05.1862@\textsc{Baracs, Géza} (*~26.05.1862), \emph{Journalist/Journalistin, Lehrer/Lehrerin, Priester/Priesterin}|pwk} gab unter seinem Pseudonym
                  »Clément Deltour« auf Subskription eine Reihe »Unsere Zeitgenossen«/»Nos
                  contemporains« heraus. Diese ist sehr selten, ein Beitrag über Schnitzler konnte nicht nachgewiesen werden.}}}\label{K_L02156-2}.\pend
           
\pstart
           Meinen \label{K_L02156-3v}\edtext{Beſuch}{\lemma{\textnormal{\emph{Beſuch}}}\Cendnote{\textnormal{Vgl. A. S.: \emph{Tagebuch}, 29. 10. 1913.
               }}}\label{K_L02156-3} in der Sternwartegaſſe\oindex{Sternwartestrasse 71@\textbf{Sternwartestraße 71}, \emph{Wohngebäude (K.WHS)}|pw} habe ich ſehr
               genoſſen.\pend
           
\pstart
           Auf bald!{\\[\baselineskip]}\spacefill\mbox{B. Suttner}\pend
           \leftskip=0em{}\selectlanguage{ngerman}\endnumbering\briefempfaengerindex{Schnitzler, Arthur@\textsc{Schnitzler, Arthur}!zzzSuttner, Bertha von@\emph{von Bertha von Suttner}!1913-11-041@{4. 11. 1913}|)be}\mylabel{L02156h}  \normalsize

\doendnotes{C}
\bigskip
\vfill

\clearpage

\footnotesize

\lohead{\textsc{register}}

% Definiere theindex-Environment komplett neu ohne reledmac
\makeatletter
\renewenvironment{theindex}{%
  \section*{\indexname}%
  \setlength{\parindent}{0pt}%
  \setlength{\parskip}{0pt plus 0.3pt}%
  \let\item\@idxitem
}{%
  \clearpage
}
\makeatother

\IfFileExists{\jobname-pw.ind}{\input{\jobname-pw.ind}}{}

\end{document}

      