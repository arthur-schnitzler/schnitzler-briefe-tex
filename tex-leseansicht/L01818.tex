%% latex-leseansicht-vorspann.tex
%% Vorspann für die Leseansicht.
%% Lädt die gemeinsame Datei latex-vorspann.tex mit nicht gesetztem Schalter.

\newif\ifkorrekturansicht
\korrekturansichtfalse

\input{../tex-inputs/latex-vorspann}


\section[Arthur Schnitzler an Richard Beer-Hofmann, {{[}}3. 12. 1908?{{]}}]{L01818 Arthur Schnitzler an Richard Beer-Hofmann, {[}3. 12. 1908?{]}}
\nopagebreak\mylabel{L01818v}
\rehead{ }\normalsize\beginnumbering\briefempfaengerindex{Beer-Hofmann, Richard@\textsc{Beer-Hofmann, Richard}!zzzSchnitzler, Arthur@\emph{von Arthur Schnitzler}!1908-12-031@{{[}3. 12. 1908?{]}}|(be}
\toendnotes[C]{\smallbreak\pagebreak[2]}
\correspDesc{Versand  durch Arthur Schnitzler am [3. 12. 1908?] in Wien
\newline{}Erhalt  durch Richard Beer-Hofmann am [3. 12. 1908?] in Wien}\toendnotes[C]{\smallbreak}
\Standort{YCGL, MSS 31.}
\physDesc{Brief, 1 Blatt, 2 Seiten, Kuvert, 266 Zeichen
\newline{}Handschrift: Bleistift, deutsche Kurrent
\newline{}Versand: ohne postalischen Übermittlungsvermerk }\toendnotes[C]{\smallbreak}\pstart{}{\pb}\textcolor{gray}{\textbf{Dr. Arthur Schnitzler}}\pend{}\pstart{}\textcolor{gray}{\textbf{Wien XVIII. Spoettelgasse 7\oindex{Wien@\textbf{Wien}!XVIII., Währing@\textbf{XVIII., Währing}!Edmund-Weiß-Gasse 7@\textbf{Edmund-Weiß-Gasse 7}, \emph{Wohngebäude}|pw}.}}\pend{}{\bigskip}\pstart{}{\pb}\textsc{Herrn Dr Richard Beer-Hofmann}\pend{}\pstart{}Wien XIX\oindex{XIX., Döbling@\textbf{XIX., Döbling}, \emph{Verwaltungsgebiet}|pw}\pend{}{\bigskip}\vspace{1em}
\pstart
           {\pb}\textcolor{gray}{\textbf{Dr. Arthur Schnitzler}}{\\}\textcolor{gray}{\textbf{Wien XVIII. Spoettelgasse 7\oindex{Wien@\textbf{Wien}!XVIII., Währing@\textbf{XVIII., Währing}!Edmund-Weiß-Gasse 7@\textbf{Edmund-Weiß-Gasse 7}, \emph{Wohngebäude}|pw}.}}\pend
           
\pstart{}lieber Richard\pend\vspace{0.5em}
\pstart
           wir fahren \label{K_L01818-1v}\edtext{morgen Freitag}{\lemma{\textnormal{\emph{morgen Freitag}}}\Cendnote{\textnormal{Das Korrespondenzstück ist undatiert.
                  Unter der Annahme, dass die hier geplante Semmering\oindex{Semmering@\textbf{Semmering}, \emph{Verwaltungsgebiet}|pwk}-Reise auf die hier besprochene Weise stattfand, trifft das für
                  den Aufenthalt vom 4. 12. 1908 bis zum 6. 12. 1908 zu.}}}\label{K_L01818-1} (11.25) auf den Semmering\oindex{Semmering@\textbf{Semmering}, \emph{Verwaltungsgebiet}|pw} (bis So{\geminationn}tag{[}){]}. We{\geminationn} Sie und Paula\pwindex{Beer-Hofmann, Paula 25.\,2.\,1879 Wien – 30.\,10.\,1939 Zürich@\textsc{Beer-Hofmann, Paula} (25.\,2.\,1879 Wien – 30.\,10.\,1939 Zürich)|pw} mitfahren wollten würd es uns beſonders {\pb}freuen. Ja wir würden Sie{ }ſogar mit dem Auto um
                  ¾ 11 abholen.\pend
           
\pstart
           Der ich wohl zu leben wünſche,\pend
           \pstart herzlichſt\spacefill\mbox{A.}\pend{}\selectlanguage{ngerman}\endnumbering\briefempfaengerindex{Beer-Hofmann, Richard@\textsc{Beer-Hofmann, Richard}!zzzSchnitzler, Arthur@\emph{von Arthur Schnitzler}!1908-12-031@{{[}3. 12. 1908?{]}}|)be}\mylabel{L01818h}  \newcommand{\dateiname}{L01818}\newcommand{\titel}{Arthur Schnitzler an Richard Beer-Hofmann, [3. 12. 1908?]}\newcommand{\editorInnen}{Martin Anton Müller und Gerd-Hermann Susen}%% latex-leseansicht-abspann.tex
%% Abspann für die Leseansicht.
%% Der Schalter \ifkorrekturansicht ist bereits durch den Vorspann gesetzt.

%% latex-abspann.tex
%% Gemeinsamer Abspann für Korrekturansicht und Leseansicht.
%% Setzt den Schalter \ifkorrekturansicht voraus (gesetzt in den
%% einbindenden Dateien latex-korrekturansicht-abspann.tex bzw.
%% latex-leseansicht-abspann.tex).
%% ---------------------------------------------------------------

\normalsize

% Das esempio-Environment wird nur in der Leseansicht benötigt
\ifkorrekturansicht\else
\newenvironment{esempio}[3]%
{
    \vspace{1.5ex}
    \rlap{\underline{#1}}
    \par
    \setlength{\parindent}{0cm}
    \nopagebreak
    \leftskip=#2cm
    \rightskip=#3cm
}
{
    \par
}
\fi

\doendnotes{C}
\bigskip
\vfill

\clearpage

\footnotesize

\ifkorrekturansicht
  \lohead{\textsc{register}}
\fi

% theindex-Environment neu definieren ohne reledmac
\makeatletter
\renewenvironment{theindex}{%
  \ifkorrekturansicht
    \section*{\indexname}%
  \else
    \subsubsection*{Index der erwähnten Entitäten}%
  \fi
  \setlength{\parindent}{0pt}%
  \setlength{\parskip}{0pt plus 0.3pt}%
  \let\item\@idxitem
}{%
  \ifkorrekturansicht\clearpage\fi
}
\makeatother

\IfFileExists{\jobname-pw.ind}{\input{\jobname-pw.ind}}{}

% Quellenangabe nur in der Leseansicht
\ifkorrekturansicht\else
% Fallback-Definitionen, falls die .tex-Datei \titel etc. nicht gesetzt hat
\providecommand{\titel}{}
\providecommand{\editorInnen}{}
\providecommand{\dateiname}{\jobname}

\vspace{3cm}

\vfill

\footnotesize
\textsc{Quelle}: \titel. Herausgegeben von {\editorInnen}. In: \emph{Arthur Schnitzler: Briefwechsel mit Autorinnen und Autoren}.
 Digitale Edition, https://schnitzler-briefe.acdh.oeaw.ac.at/{\dateiname}.html (Stand \today)
\fi

\end{document}


