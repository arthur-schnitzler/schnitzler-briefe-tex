%% latex-korrekturansicht-vorspann.tex
%% Vorspann für die Korrekturansicht.
%% Lädt die gemeinsame Datei latex-vorspann.tex mit gesetztem Schalter.

\newif\ifkorrekturansicht
\korrekturansichttrue

\input{../tex-inputs/latex-vorspann}


\section[Stefan Zweig an Arthur Schnitzler, 1{[}2/3?{]}. 6. {[}1913?{]}]{L03638 Stefan Zweig an Arthur Schnitzler,
               1{[}2/3?{]}. 6. {[}1913?{]}}
\nopagebreak\mylabel{L03638v}
\rehead{ }\normalsize\beginnumbering\briefempfaengerindex{Schnitzler, Arthur@\textsc{Schnitzler, Arthur}!zzzZweig, Stefan@\emph{von Stefan Zweig}!1913-06-131@{1{[}2/3?{]}. 6. {[}1913?{]}}|(be}
\toendnotes[C]{\smallbreak\pagebreak[2]}\Standort{CUL, Schnitzler, B 118.}
\physDesc{Bildpostkarte, 314 Zeichen
\newline{}Handschrift: blaue Tinte, lateinische Kurrent
\newline{}Versand: Stempel: »\nobreak{}\oindex{VIII., Josefstadt@\textbf{VIII., Josefstadt}, \emph{A.ADM3}|pwk}8/ Wien, 1{[}2/3?{]}. VI. {[}13?{]}, 7\nobreak{}«.  }
\buchAbdrucke{\weitereDrucke{Stefan Zweig: \emph{Briefwechsel mit Hermann Bahr, Sigmund Freud, Rainer Maria
                        Rilke und Arthur Schnitzler}. Frankfurt am Main: \emph{S. Fischer} 1987, S. 374.} }\toendnotes[C]{\smallbreak}\pstart{}{\pb}D\textsuperscript{r}
                  Artur Schnitzler\pend{}\pstart{}Wien – Cottage\oindex{Waehringer Cottage@\textbf{Währinger Cottage}, \emph{Teil eines besiedelten Ortes (A.BSOX)}|pw}\pend{}\pstart{}\label{K_L03638-1v}\edtext{Sternwartestrasse 72}{\lemma{\textnormal{\emph{Sternwartestrasse 72}}}\Cendnote{\textnormal{Zweig\pwindex{Zweig, Stefan 28.11.1881 – 23.02.1942@\textsc{Zweig, Stefan} (28.11.1881 – 23.02.1942), \emph{Schriftsteller/Schriftstellerin}|pwk} wechselt
                        bei der Adressierung seiner Schreiben an Schnitzler immer wieder zwischen der falschen Hausnummer
                           »72« und der richtigen
                  »71«.}}}\label{K_L03638-1}\oindex{Sternwartestrasse 71@\textbf{Sternwartestraße 71}, \emph{Wohngebäude (K.WHS)}|pw}\pend{}{\bigskip}
\pstart
           \noindent{}{\pb}\textcolor{gray}{\textbf{Wien – Maximilianplatz\oindex{Rooseveltplatz@\textbf{Rooseveltplatz}, \emph{Platz (K.PLT)}|pw} u. Votivkirche\oindex{Votivkirche@\textbf{Votivkirche}, \emph{Kirche (K.KRC)}|pw}}}\pend
           \vspace{1em}
\pstart
           \noindent{}{\pb}Verehrter Herr Doktor,
               ich höre eben von \label{K_L03638-2v}\edtext{Heinis\pwindex{Schnitzler, Heinrich 09.08.1902 – 12.07.1982@\textsc{Schnitzler, Heinrich} (09.08.1902 – 12.07.1982), \emph{Regisseur/Regisseurin, Schauspieler/Schauspielerin}|pw} Erkrankung}{\lemma{\textnormal{\emph{Heinis Erkrankung}}}\Cendnote{\textnormal{Am 10. 6. 1913
                     erhielten Olga\pwindex{Schnitzler, Olga 17.01.1882 – 13.01.1970@\textsc{Schnitzler, Olga} (17.01.1882 – 13.01.1970), \emph{Schauspieler/Schauspielerin, Sänger/Sängerin}|pwk} und Arthur Schnitzler, die erst am Vortag zu einer
                     Reise in die Schweiz\oindex{Schweiz@\textbf{Schweiz}, \emph{A.PCLI}|pwk} aufgebrochen waren, ein
                     Telegramm, dass ihr Sohn Heinrich
                        Schnitzler\pwindex{Schnitzler, Heinrich 09.08.1902 – 12.07.1982@\textsc{Schnitzler, Heinrich} (09.08.1902 – 12.07.1982), \emph{Regisseur/Regisseurin, Schauspieler/Schauspielerin}|pwk} an Scharlach erkrankt sei. Die Eltern\pwindex{Schnitzler, Olga 17.01.1882 – 13.01.1970@\textsc{Schnitzler, Olga} (17.01.1882 – 13.01.1970), \emph{Schauspieler/Schauspielerin, Sänger/Sängerin}|pwk} verließen das eben
                     erreichte Chur\oindex{Chur@\textbf{Chur}, \emph{P.PPLA}|pwk} sofort wieder und kehrten
                     bereits am 11. 6. 1913 nach Wien\oindex{Wien@\textbf{Wien}, \emph{A.ADM2}|pwk}
                  zurück.}}}\label{K_L03638-2} und Ihrer jähen \label{K_L03638-3v}\edtext{Rückkehr}{\lemma{\textnormal{\emph{Rückkehr}}}\Cendnote{\textnormal{Die Karte ist nicht datiert. Auf dem Poststempel läßt sich entziffern, dass sie im Juni gesendet wurde, die Angabe des Tages mit der Ziffer 1 beginnt und zweistellig ist. Dabei ist von der zweiten Ziffer nur eine obere Rundung zu sehen, wie sie bei den Zahlen 2, 3, 8 und 9 vorkommt. Der Verweis auf die »jähe Rückkehr« deutet darauf hin, dass Zweig\pwindex{Zweig, Stefan 28.11.1881 – 23.02.1942@\textsc{Zweig, Stefan} (28.11.1881 – 23.02.1942), \emph{Schriftsteller/Schriftstellerin}|pwk} die Karte bereits am 12. oder 13. 6. verfasst hat, als die Aufregung über die durchkreuzten Reisepläne von Olga\pwindex{Schnitzler, Olga 17.01.1882 – 13.01.1970@\textsc{Schnitzler, Olga} (17.01.1882 – 13.01.1970), \emph{Schauspieler/Schauspielerin, Sänger/Sängerin}|pwk} und Arthur Schnitzler noch frisch war.}}}\label{K_L03638-3}. Hoffentlich geht alles gut und rasch
               vorbei, meine innigsten Wünsche sind mit Ihnen in all diesen erregten und hoffentlich
               bald beruhigten Stunden.\pend
           
\pstart
           Ihr aufrichtig getreuer{\\[\baselineskip]}\spacefill\mbox{Stefan Zweig}\pend
           \leftskip=0em{}\selectlanguage{ngerman}\endnumbering\briefempfaengerindex{Schnitzler, Arthur@\textsc{Schnitzler, Arthur}!zzzZweig, Stefan@\emph{von Stefan Zweig}!1913-06-121@{1{[}2/3?{]}. 6. {[}1913?{]}}|)be}\mylabel{L03638h}
\begin{anhang}
\end{anhang}\normalsize

\doendnotes{C}
\bigskip
\vfill

\clearpage

\footnotesize

\lohead{\textsc{register}}

% Definiere theindex-Environment komplett neu ohne reledmac
\makeatletter
\renewenvironment{theindex}{%
  \section*{\indexname}%
  \setlength{\parindent}{0pt}%
  \setlength{\parskip}{0pt plus 0.3pt}%
  \let\item\@idxitem
}{%
  \clearpage
}
\makeatother

\IfFileExists{\jobname-pw.ind}{\input{\jobname-pw.ind}}{}

\end{document}

      