%% latex-leseansicht-vorspann.tex
%% Vorspann für die Leseansicht.
%% Lädt die gemeinsame Datei latex-vorspann.tex mit nicht gesetztem Schalter.

\newif\ifkorrekturansicht
\korrekturansichtfalse

\input{../tex-inputs/latex-vorspann}


         
         \renewcommand{\erwaehntePersonen}{Personen: Otto Brahm, Franz August Carl Maria Reisch, Felix Salten}
         \renewcommand{\erwaehnteOrte}{Orte: Meran, Niederlande, Noordwijk, Seis am Schlern, Südtirol, Villa Heufler}
         \renewcommand{\erwaehnteWerke}{Werke: Partie in Seis am Schlern}
               \section[ Arthur Schnitzler und Otto Brahm an Felix Salten, 19. 7. 1908]{ Arthur Schnitzler und Otto Brahm an Felix Salten, 19. 7. 1908}\nopagebreak\mylabel{v}\rehead{ }\begin{ledgroupsized}[t]{13cm}\normalsize\beginnumbering\briefempfaengerindex{Salten, Felix@\textsc{Salten, Felix}!zzzBrahm, Otto@\emph{von Otto Brahm}!1908-07-191@{19. 7. 1908}|(be}\briefempfaengerindex{Salten, Felix@\textsc{Salten, Felix}!zzzSchnitzler, Arthur@\emph{von Arthur Schnitzler}!1908-07-191@{19. 7. 1908}|(be} \toendnotes[C]{\smallbreak\pagebreak[2]} \Standort{Wienbibliothek im Rathaus, ZPH 1681, 2.1.516.}
\physDesc{Bildpostkarte, 274 Zeichen
\newline{}Handschrift Arthur Schnitzler: Bleistift, deutsche Kurrent\newline{}Handschrift Otto Brahm: Bleistift, lateinische Kurrent
\newline{}Versand: 1) Stempel: »\nobreak{}\oindex{Seis am Schlern@\textbf{Seis am Schlern}|pwk}Seis, 20. 7. 08\nobreak{}«.   2) Stempel: »\nobreak{}\oindex{Noordwijk@\textbf{Noordwijk}|pwk}Noordwijk 1, 22. 7. 08, 5–\textcolor{gray}{6} V.\nobreak{}«. 
\newline{}Ordnung: mit Bleistift von unbekannter Hand nummeriert: »4« }\toendnotes[C]{\smallbreak}\pstart{}{\pb}\textsc{Holland\oindex{Niederlande@\textbf{Niederlande}|pw}}\pend{}\pstart{}Hr \textsc{Felix Salten}\pend{}\pstart{}\textsc{Nordwyk\oindex{Noordwijk@\textbf{Noordwijk}|pw}}\pend{}{\bigskip}\pstart
           \noindent{}\centering{}{\pb}\textcolor{gray}{\textbf{Tirol\oindex{Suedtirol@\textbf{Südtirol}|pw}: \textbf{Villa Heufler, Seis am Schlern,}\oindex{Villa Heufler@\textbf{Villa Heufler}|pw} 1000 m. \textbf{Nach dem Aquarell\pwindex{Reisch, Franz August Carl Maria 1862-05-01 – 1942?@\textsc{Reisch, Franz August Carl Maria} (1862-05-01 – 1942?), \emph{Maler}!Partie in Seis am Schlern@\strich\emph{Partie in Seis am Schlern}|pwv} von F.
                           A. C. M. Reisch\pwindex{Reisch, Franz August Carl Maria 1862-05-01 – 1942?@\textsc{Reisch, Franz August Carl Maria} (1862-05-01 – 1942?), \emph{Maler}|pw}, Meran\oindex{Meran@\textbf{Meran}|pw}.}}}\pend
           \pstart
           {\pb}Schönen Dank für die \label{K_L03013-1v}\edtext{Karte}{\lemma{\textnormal{\emph{Karte}}}\Cendnote{\textnormal{Arthur Schnitzler und Otto Brahm an Felix Salten, 19. 7. 1908}}}\label{K_L03013-1h} aus \textsc{Nordwyk\oindex{Noordwijk@\textbf{Noordwijk}|pw}}. Wir fühlen uns hier\oindex{Seis am Schlern@\textbf{Seis am Schlern}|pwv} wohl
               und bleiben noch geraume Zeit. Laſſen Sie bald ein Wort hören, wie’s Ihnen geht!
                  {[}hs. Brahm:{]} und wie Sie dichten.\pend
           \pstart
           B. Gr\textcolor{gray}{.}{\\}\spacefill\mbox{OBrahm.}\pend
           \pstart {[}hs. Schnitzler:{]} Mit herzlichen So{\geminationm}erwünſchen und Grüßen von Haus zu Haus Ihr \spacefill\mbox{A.}\pend{}\pstart
           \raggedleft{}\label{T_L03013-1v}\edtext{19. 7. 08}{\lemma{\textnormal{\emph{19. 7. 08}}}\Cendnote{\textnormal{am linken oberen Rand quer zum
                     Text}}}\label{T_L03013-1h}\pend
           
         
         \endnumbering\mylabel{h}\end{ledgroupsized}  \newcommand{\dateiname}{L03013}\newcommand{\titel}{Arthur Schnitzler und Otto Brahm an Felix Salten, 19. 7. 1908}\newcommand{\editorInnen}{Martin Anton Müller und Laura Untner}%% latex-leseansicht-abspann.tex
%% Abspann für die Leseansicht.
%% Der Schalter \ifkorrekturansicht ist bereits durch den Vorspann gesetzt.

%% latex-abspann.tex
%% Gemeinsamer Abspann für Korrekturansicht und Leseansicht.
%% Setzt den Schalter \ifkorrekturansicht voraus (gesetzt in den
%% einbindenden Dateien latex-korrekturansicht-abspann.tex bzw.
%% latex-leseansicht-abspann.tex).
%% ---------------------------------------------------------------

\normalsize

% Das esempio-Environment wird nur in der Leseansicht benötigt
\ifkorrekturansicht\else
\newenvironment{esempio}[3]%
{
    \vspace{1.5ex}
    \rlap{\underline{#1}}
    \par
    \setlength{\parindent}{0cm}
    \nopagebreak
    \leftskip=#2cm
    \rightskip=#3cm
}
{
    \par
}
\fi

\doendnotes{C}
\bigskip
\vfill

\clearpage

\footnotesize

\ifkorrekturansicht
  \lohead{\textsc{register}}
\fi

% theindex-Environment neu definieren ohne reledmac
\makeatletter
\renewenvironment{theindex}{%
  \ifkorrekturansicht
    \section*{\indexname}%
  \else
    \subsubsection*{Index der erwähnten Entitäten}%
  \fi
  \setlength{\parindent}{0pt}%
  \setlength{\parskip}{0pt plus 0.3pt}%
  \let\item\@idxitem
}{%
  \ifkorrekturansicht\clearpage\fi
}
\makeatother

\IfFileExists{\jobname-pw.ind}{\input{\jobname-pw.ind}}{}

% Quellenangabe nur in der Leseansicht
\ifkorrekturansicht\else
% Fallback-Definitionen, falls die .tex-Datei \titel etc. nicht gesetzt hat
\providecommand{\titel}{}
\providecommand{\editorInnen}{}
\providecommand{\dateiname}{\jobname}

\vspace{3cm}

\vfill

\footnotesize
\textsc{Quelle}: \titel. Herausgegeben von {\editorInnen}. In: \emph{Arthur Schnitzler: Briefwechsel mit Autorinnen und Autoren}.
 Digitale Edition, https://schnitzler-briefe.acdh.oeaw.ac.at/{\dateiname}.html (Stand \today)
\fi

\end{document}


      