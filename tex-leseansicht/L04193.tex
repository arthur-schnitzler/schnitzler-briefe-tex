%% latex-leseansicht-vorspann.tex
%% Vorspann für die Leseansicht.
%% Lädt die gemeinsame Datei latex-vorspann.tex mit nicht gesetztem Schalter.

\newif\ifkorrekturansicht
\korrekturansichtfalse

\input{../tex-inputs/latex-vorspann}


\section[Arthur Schnitzler an Gustav Schwarzkopf, {{[}}22. 6. 1893?{{]}}]{L04193 Arthur Schnitzler an Gustav Schwarzkopf, {[}22. 6. 1893?{]}}
\nopagebreak\mylabel{L04193v}
\rehead{ }\normalsize\beginnumbering\briefempfaengerindex{Schwarzkopf, Gustav@\textsc{Schwarzkopf, Gustav}!zzzSchnitzler, Arthur@\emph{von Arthur Schnitzler}!1893-06-222@{{[}22. 6. 1893?{]}}|(be}
\toendnotes[C]{\smallbreak\pagebreak[2]}
\correspDesc{Versand  durch Arthur Schnitzler am [22. 6. 1893?] in Wien
\newline{}Erhalt  durch Gustav Schwarzkopf im Zeitraum [22. 6. 1893
                  – 25. 6. 1893?] in Wien}\toendnotes[C]{\smallbreak}
\Standort{CUL, Schnitzler, B 96.}
\physDesc{Briefkarte, 395 Zeichen (Karte mit Trauerrand)
\newline{}Handschrift: Bleistift, deutsche Kurrent}\toendnotes[C]{\smallbreak}
\pstart
           \noindent{}{\pb}Verehrteſter Herr Schwarzkopf, ich habe eben \label{K_L04193-1v}\edtext{Loris\pwindex{Hofmannsthal, Hugo von 1.\,2.\,1874 Wien – 15.\,7.\,1929 Rodaun@\textsc{Hofmannsthal, Hugo von} (1.\,2.\,1874 Wien – 15.\,7.\,1929 Rodaun), \emph{Schriftsteller}|pw} geſchrieben}{\lemma{\textnormal{\emph{Loris geschrieben}}}\Cendnote{\textnormal{Der Brief ist nicht erhalten, aber die Antwort: XXXX Auszeichnungsfehler: Dokument L00225 nicht gefunden.}}}\label{K_L04193-1}, dſs ich \label{K_L04193-2v}\edtext{heut schon ziemlich früh in
                  d\textcolor{gray}{ie}{ }Brühl\oindex{Brühl@\textbf{Brühl}, \emph{Tal}|pw} fahre}{\lemma{\textnormal{\emph{heut … fahre}}}\Cendnote{\textnormal{Die Karte ist undatiert, aber durch den Aufenthalt im Hotel Hajek\oindex{Hotel Hajek@\textbf{Hotel Hajek}, \emph{Hotel}|pwk} auf den 22. 6. 1893 zu
                  datieren.}}}\label{K_L04193-2}, und dſs er bei mir {\pb}übernachten kann. Fahren Sie auch{ }ſchon heute? Es wär hübſch, we{\geminationn} wir Abends zuſa{\geminationm}en
               wären. – \substVorne{}\textsuperscript{V }\substDazwischen{}B\substHinten{}ahnwa\textcolor{gray}{r}tezi{\geminationm}er{ }ſind unſicher
               und unangenehm. Ich wohne im Hajek\oindex{Hotel Hajek@\textbf{Hotel Hajek}, \emph{Hotel}|pw}, laſſen Sie
               mir dort hin poſt. –\pend
           \pstart Herzlichſt \spacefill\mbox{Arthur}\pend{}
\pstart
           \noindent{}{\pb}\label{T_L04193-1v}\edtext{wollen Sie mir noch vor
                     2 in d Grillp.ſtr.\oindex{Wien@\textbf{Wien}!I., Innere Stadt@\textbf{I., Innere Stadt}!Wohnung und Ordination Arthur Schnitzler Grillparzerstraße 7/3. Stock@\textbf{Wohnung und Ordination Arthur Schnitzler Grillparzerstraße 7/3. Stock}, \emph{Ordination}|pw} eine
                  Note{ }ſagen laſſen.}{\lemma{\textnormal{\emph{wollen … lassen.}}}\Cendnote{\textnormal{In der linken
                     unteren Ecke der ersten Seite.}}}\label{T_L04193-1}\pend
           \selectlanguage{ngerman}\endnumbering\briefempfaengerindex{Schwarzkopf, Gustav@\textsc{Schwarzkopf, Gustav}!zzzSchnitzler, Arthur@\emph{von Arthur Schnitzler}!1893-06-222@{{[}22. 6. 1893?{]}}|)be}\mylabel{L04193h}
\begin{anhang}
\end{anhang}\newcommand{\dateiname}{L04193}\newcommand{\titel}{Arthur Schnitzler an Gustav Schwarzkopf, [22. 6. 1893?]}\newcommand{\editorInnen}{Herausgegeben von Jahnke, SelmaMüller, Martin Anton}%% latex-leseansicht-abspann.tex
%% Abspann für die Leseansicht.
%% Der Schalter \ifkorrekturansicht ist bereits durch den Vorspann gesetzt.

%% latex-abspann.tex
%% Gemeinsamer Abspann für Korrekturansicht und Leseansicht.
%% Setzt den Schalter \ifkorrekturansicht voraus (gesetzt in den
%% einbindenden Dateien latex-korrekturansicht-abspann.tex bzw.
%% latex-leseansicht-abspann.tex).
%% ---------------------------------------------------------------

\normalsize

% Das esempio-Environment wird nur in der Leseansicht benötigt
\ifkorrekturansicht\else
\newenvironment{esempio}[3]%
{
    \vspace{1.5ex}
    \rlap{\underline{#1}}
    \par
    \setlength{\parindent}{0cm}
    \nopagebreak
    \leftskip=#2cm
    \rightskip=#3cm
}
{
    \par
}
\fi

\doendnotes{C}
\bigskip
\vfill

\clearpage

\footnotesize

\ifkorrekturansicht
  \lohead{\textsc{register}}
\fi

% theindex-Environment neu definieren ohne reledmac
\makeatletter
\renewenvironment{theindex}{%
  \ifkorrekturansicht
    \section*{\indexname}%
  \else
    \subsubsection*{Index der erwähnten Entitäten}%
  \fi
  \setlength{\parindent}{0pt}%
  \setlength{\parskip}{0pt plus 0.3pt}%
  \let\item\@idxitem
}{%
  \ifkorrekturansicht\clearpage\fi
}
\makeatother

\IfFileExists{\jobname-pw.ind}{\input{\jobname-pw.ind}}{}

% Quellenangabe nur in der Leseansicht
\ifkorrekturansicht\else
% Fallback-Definitionen, falls die .tex-Datei \titel etc. nicht gesetzt hat
\providecommand{\titel}{}
\providecommand{\editorInnen}{}
\providecommand{\dateiname}{\jobname}

\vspace{3cm}

\vfill

\footnotesize
\textsc{Quelle}: \titel. Herausgegeben von {\editorInnen}. In: \emph{Arthur Schnitzler: Briefwechsel mit Autorinnen und Autoren}.
 Digitale Edition, https://schnitzler-briefe.acdh.oeaw.ac.at/{\dateiname}.html (Stand \today)
\fi

\end{document}


