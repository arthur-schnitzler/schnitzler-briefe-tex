%% latex-leseansicht-vorspann.tex
%% Vorspann für die Leseansicht.
%% Lädt die gemeinsame Datei latex-vorspann.tex mit nicht gesetztem Schalter.

\newif\ifkorrekturansicht
\korrekturansichtfalse

\input{../tex-inputs/latex-vorspann}


         
         \newcommand{\erwaehntePersonen}{Personen: }
         \newcommand{\erwaehnteInstitutionen}{}
         \newcommand{\erwaehnteOrte}{}
         \newcommand{\erwaehnteWerke}{
               \section[Arthur Schnitzler an Georg Brandes, 31. 12. 1897]{ Arthur Schnitzler an Georg Brandes, 31. 12. 1897}\nopagebreak\mylabel{v}\rehead{ }\begin{ledgroupsized}[t]{13cm}\normalsize\beginnumbering \toendnotes[C]{\smallbreak\pagebreak[2]} \Standort{Kopenhagen, Det Kongelige Bibliotek, Georg Brandes Arkiv, box 125.}
\physDesc{Brief, 1 Blatt, 3 Seiten
\newline{}Handschrift: schwarze Tinte, deutsche Kurrent\newline{}Ordnung: mit Bleistift von unbekannter Hand
                                    nummeriert: »10. Schnitzler« }\buchAbdrucke{\weitereDrucke{Georg Brandes, Arthur Schnitzler: \emph{Ein Briefwechsel}. Hg. Kurt Bergel. Bern: \emph{Francke} 1956, S. 66.} }\pstart
           \raggedleft{}{\pb}\textsc{Wien}\oindex{XXXX Ortsangabe fehlt|pw}, 31. 12. 97{\\}IX. Frankgaſſe 1\oindex{XXXX Ortsangabe fehlt|pw}\pend
           \pstart{}Verehrteſter Herr Brandes,\pend\pstart
           was für eine erfreuliche Nachricht als erſte nach ſo langer Zeit! Sowohl
                        \textsc{Beer-Hofma{\geminationn}}\pwindex{\textcolor{red}{\textsuperscript{XXXX1 indx}}|pw} als ich ſind in Wien\oindex{XXXX Ortsangabe fehlt|pw} und freuen uns ſehr, Sie ſobald wiederzuſehen. Als Hotel wird mir
                    in {\pb}der letzten Zeit das »Reſidenz-Hotel\oindex{XXXX Ortsangabe fehlt|pw}« in der \textsc{Teinfaltstraße}\oindex{XXXX Ortsangabe fehlt|pw}, ſehr gut gelegen, empfohlen; es iſt nicht abſolut erſten Ranges,
                    ſcheint mir aber angenehmer als die großen Hotels, \textsc{Imperial}\oindex{XXXX Ortsangabe fehlt|pw}, \textsc{Grand Hotel}\oindex{XXXX Ortsangabe fehlt|pw}, \textsc{Bristol}\oindex{XXXX Ortsangabe fehlt|pw}. Vielleicht ſchreiben Sie mir noch näheres {\pb}über Ihre Wünſche; auf eine weitere
                    Nachricht von Ihrem Kommen dürfen wir ja hoffen?\pend
           \pstart
           Mit herzlichen Grüßen{\\[\baselineskip]}Ihr ergebenſter{\\[\baselineskip]}\spacefill\mbox{Arthur Schnitzler.}\pend
           \leftskip=0em{}
         
         \endnumbering\mylabel{h}\end{ledgroupsized}  \newcommand{\dateiname}{L00757}\newcommand{\titel}{Arthur Schnitzler an Georg Brandes, 31. 12. 1897}\newcommand{\editorInnen}{Martin Anton Müller und Gerd-Hermann Susen}%% latex-leseansicht-abspann.tex
%% Abspann für die Leseansicht.
%% Der Schalter \ifkorrekturansicht ist bereits durch den Vorspann gesetzt.

%% latex-abspann.tex
%% Gemeinsamer Abspann für Korrekturansicht und Leseansicht.
%% Setzt den Schalter \ifkorrekturansicht voraus (gesetzt in den
%% einbindenden Dateien latex-korrekturansicht-abspann.tex bzw.
%% latex-leseansicht-abspann.tex).
%% ---------------------------------------------------------------

\normalsize

% Das esempio-Environment wird nur in der Leseansicht benötigt
\ifkorrekturansicht\else
\newenvironment{esempio}[3]%
{
    \vspace{1.5ex}
    \rlap{\underline{#1}}
    \par
    \setlength{\parindent}{0cm}
    \nopagebreak
    \leftskip=#2cm
    \rightskip=#3cm
}
{
    \par
}
\fi

\doendnotes{C}
\bigskip
\vfill

\clearpage

\footnotesize

\ifkorrekturansicht
  \lohead{\textsc{register}}
\fi

% theindex-Environment neu definieren ohne reledmac
\makeatletter
\renewenvironment{theindex}{%
  \ifkorrekturansicht
    \section*{\indexname}%
  \else
    \subsubsection*{Index der erwähnten Entitäten}%
  \fi
  \setlength{\parindent}{0pt}%
  \setlength{\parskip}{0pt plus 0.3pt}%
  \let\item\@idxitem
}{%
  \ifkorrekturansicht\clearpage\fi
}
\makeatother

\IfFileExists{\jobname-pw.ind}{\input{\jobname-pw.ind}}{}

% Quellenangabe nur in der Leseansicht
\ifkorrekturansicht\else
% Fallback-Definitionen, falls die .tex-Datei \titel etc. nicht gesetzt hat
\providecommand{\titel}{}
\providecommand{\editorInnen}{}
\providecommand{\dateiname}{\jobname}

\vspace{3cm}

\vfill

\footnotesize
\textsc{Quelle}: \titel. Herausgegeben von {\editorInnen}. In: \emph{Arthur Schnitzler: Briefwechsel mit Autorinnen und Autoren}.
 Digitale Edition, https://schnitzler-briefe.acdh.oeaw.ac.at/{\dateiname}.html (Stand \today)
\fi

\end{document}


      