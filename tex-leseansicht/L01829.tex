%% latex-korrekturansicht-vorspann.tex
%% Vorspann für die Korrekturansicht.
%% Lädt die gemeinsame Datei latex-vorspann.tex mit gesetztem Schalter.

\newif\ifkorrekturansicht
\korrekturansichttrue

\input{../tex-inputs/latex-vorspann}


\section[Arthur Schnitzler an Adalbert Seligmann, 12. 2. 1909]{L01829 Arthur Schnitzler an Adalbert Seligmann, 12. 2. 1909}
\nopagebreak\mylabel{L01829v}
\rehead{ }\normalsize\beginnumbering\briefempfaengerindex{Seligmann, Adalbert Franz@\textsc{Seligmann, Adalbert Franz}!zzzSchnitzler, Arthur@\emph{von Arthur Schnitzler}!1909-02-121@{12. 2. 1909}|(be}
\toendnotes[C]{\smallbreak\pagebreak[2]}\Standort{Wienbibliothek im Rathaus, H.I.N.-95600.}
\physDesc{Briefkarte, , Umschlag, 217 Zeichen
\newline{}Handschrift: schwarze Tinte, lateinische Kurrent
\newline{}Versand: 1) Stempel: »\nobreak{}{[}Wien{]} 110, 1{[}2. 02. 190{]}9, 4\nobreak{}«.   2) Stempel: »\nobreak{}\oindex{IX., Alsergrund@\textbf{IX., Alsergrund}, \emph{A.ADM3}|pwk}9/3 Wien, 12. II. 09, 6, Bestellt\nobreak{}«. }\pstart{}{\pb}\textcolor{gray}{\textbf{Dr. Arthur Schnitzler}}\pend{}\pstart{}\textcolor{gray}{\textbf{Wien XVIII. Spoettelgasse 7\oindex{Edmund-Weiss-Gasse 7@\textbf{Edmund-Weiß-Gasse 7}, \emph{Wohngebäude (K.WHS)}|pw}.}}\pend{}{\bigskip}\pstart{}{\pb}Hrn A. F. Seligmann,\pend{}\pstart{}Wien IX\oindex{IX., Alsergrund@\textbf{IX., Alsergrund}, \emph{A.ADM3}|pw}\pend{}\pstart{}Schwarzspanierstr 15\oindex{Schwarzspanierstrasse@\textbf{Schwarzspanierstraße}, \emph{Straße (K.STR)}|pw}.\pend{}{\bigskip}\vspace{1em}
\pstart
           {\pb}\textcolor{gray}{\textbf{Dr. Arthur Schnitzler}}\hfill 12. 2. 09\pend
           
\pstart
           \textcolor{gray}{\textbf{Wien XVIII. Spoettelgasse 7\oindex{Edmund-Weiss-Gasse 7@\textbf{Edmund-Weiß-Gasse 7}, \emph{Wohngebäude (K.WHS)}|pw}.}}\pend
           \vspace{0.5em}
\pstart
           verehrter Freund, Ihren Brief habe ich zu gef. directen Beantwortung
               an Hrn Paul Brann\pwindex{Brann, Paul 05.01.1873 – 02.09.1955@\textsc{Brann, Paul} (05.01.1873 – 02.09.1955), \emph{Theaterleiter/Theaterleiterin}|pw}, München-Schwabing, Gedonstraße 6\oindex{Gedonstrasse@\textbf{Gedonstraße}, \emph{Straße (K.STR)}|pw}, gesandt\pend
           
\pstart
           Herzlichst grüßt Ihr ergebner{\\[\baselineskip]}\spacefill\mbox{A. S.}\pend
           \leftskip=0em{}\selectlanguage{ngerman}\endnumbering\briefempfaengerindex{Seligmann, Adalbert Franz@\textsc{Seligmann, Adalbert Franz}!zzzSchnitzler, Arthur@\emph{von Arthur Schnitzler}!1909-02-121@{12. 2. 1909}|)be}\mylabel{L01829h}  \normalsize

\doendnotes{C}
\bigskip
\vfill

\clearpage

\footnotesize

\lohead{\textsc{register}}

% Definiere theindex-Environment komplett neu ohne reledmac
\makeatletter
\renewenvironment{theindex}{%
  \section*{\indexname}%
  \setlength{\parindent}{0pt}%
  \setlength{\parskip}{0pt plus 0.3pt}%
  \let\item\@idxitem
}{%
  \clearpage
}
\makeatother

\IfFileExists{\jobname-pw.ind}{\input{\jobname-pw.ind}}{}

\end{document}

      