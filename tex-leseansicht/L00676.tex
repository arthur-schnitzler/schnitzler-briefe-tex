%% latex-leseansicht-vorspann.tex
%% Vorspann für die Leseansicht.
%% Lädt die gemeinsame Datei latex-vorspann.tex mit nicht gesetztem Schalter.

\newif\ifkorrekturansicht
\korrekturansichtfalse

\input{../tex-inputs/latex-vorspann}


\section[Oscar Blumenthal an Arthur Schnitzler, 12. 5. 1897]{L00676 Oscar Blumenthal an Arthur Schnitzler, 12. 5. 1897}
\nopagebreak\mylabel{L00676v}
\rehead{ }\normalsize\beginnumbering\briefempfaengerindex{Schnitzler, Arthur@\textsc{Schnitzler, Arthur}!zzzBlumenthal, Oskar@\emph{von Oskar Blumenthal}!1897-05-122@{12. 5. 1897}|(be}
\toendnotes[C]{\smallbreak\pagebreak[2]}
\correspDesc{Versand  durch Oscar Blumenthal am 12. 5. 1897 in Lauffen
\newline{}Erhalt  durch Arthur Schnitzler im Zeitraum [13. 5. 1897
                  – 17. 5. 1897?] in Wien}\toendnotes[C]{\smallbreak}
\Standort{CUL, Schnitzler, B 15.}
\physDesc{Brief, 1 Blatt, 2 Seiten, 1781 Zeichen
\newline{}Schreibmaschine
\newline{}Handschrift: 1) Bleistift (\noindent{}die ersten drei Unterstreichungen)\hspace{1em}2) schwarze Tinte (\noindent{}Unterschrift)\hspace{1em}
\newline{}Ordnung: mit Bleistift von unbekannter Hand nummeriert:
                                 »9« }\toendnotes[C]{\smallbreak}
\pstart
           \centering{}{\pb}\textcolor{gray}{\textbf{\textsc{Lessing-Theater}\orgindex{Lessing-Theater@Lessing-Theater|pw}}}\pend
           
\pstart
           \centering{}\textcolor{gray}{\textbf{DIRECTOR:}}{ }\textcolor{gray}{\textbf{\textsc{Dr. Oscar Blumenthal}.}}\pend
           
\pstart
           \raggedleft{}\textcolor{gray}{\textbf{BERLIN N.W. (40)\oindex{Berlin@\textbf{Berlin}, \emph{Hauptstadt}|pw}, den}}{ }12. Mai 1897.\pend
           
\pstart
           \raggedleft{}z. Zeit: LAUFEN bei ISCHL\oindex{Lauffen@\textbf{Lauffen}|pw}\pend
           
\pstart\center{}Sehr geehrter Herr Doctor!\pend\vspace{0.5em}
\pstart
           Es würde mir eine grosse Freude machen, wenn Sie mir für die nächste Spielzeit des
                  »LESSING-THEATER\orgindex{Lessing-Theater@Lessing-Theater|pw}S« – die letzte unter meiner Direction – ein neues Bühnenwerk aus Ihrer Feder
               anvertrauen würden. Ich gestatte mir, Sie darauf aufmerksam zu machen, dass gerade in
               der nächsten Saison sich der schauspielerische Besitzstand des »LESSING-THEATER\orgindex{Lessing-Theater@Lessing-Theater|pw}S« durch eine Anzahl von sehr vielverheissenden Neu-Engagements beträchtlich
               vermehrt hat. Es werden in den Verband des »LESSING-THEATER\orgindex{Lessing-Theater@Lessing-Theater|pw}S« vom ersten September ab neu eintreten: \uline{ADOLF KLEIN\pwindex{Klein, Adolf 15.\,8.\,1847 Wien – 11.\,3.\,1931 Berlin@\textsc{Klein, Adolf} (15.\,8.\,1847 Wien – 11.\,3.\,1931 Berlin), \emph{Schauspieler, Theaterdirektor}|pw}} vom Königlichen Schauspielhaus\oindex{Schauspielhaus Berlin@\textbf{Schauspielhaus Berlin}, \emph{Theater}|pw}; WILLY ROHLAND\pwindex{Rohland, Willy 15.\,10.\,1854 Coburg – 5.\,4.\,1908 Hannover@\textsc{Rohland, Willy} (15.\,10.\,1854 Coburg – 5.\,4.\,1908 Hannover)|pw}, ALFRED HALM\pwindex{Halm, Alfred 9.\,12.\,1861 Wien – 5.\,2.\,1951 Berlin@\textsc{Halm, Alfred} (9.\,12.\,1861 Wien – 5.\,2.\,1951 Berlin), \emph{Schauspieler}|pw} und HERRMANN VALENTIN\pwindex{Vallentin, Hermann 24.\,5.\,1872 Berlin – 18.\,9.\,1945 Tel Aviv@\textsc{Vallentin, Hermann} (24.\,5.\,1872 Berlin – 18.\,9.\,1945 Tel Aviv), \emph{Schauspieler}|pw} vom »Theater des Westens\oindex{Theater des Westens@\textbf{Theater des Westens}, \emph{Theater}|pw}«; PAULA CARLSEN\pwindex{Carlsen, Paula 12.\,7.\,1837 Cieplice Śląskie-Zdrój – 17.\,3.\,1900 Berlin@\textsc{Carlsen, Paula} (12.\,7.\,1837 Cieplice Śląskie-Zdrój – 17.\,3.\,1900 Berlin), \emph{Schauspielerin}|pw} vom »Neuen Theater\oindex{Neues Theater@\textbf{Neues Theater}, \emph{Theater}|pw}«; META ILLING\pwindex{Illing, Meta 27.\,2.\,1872 Berlin – 26.\,12.\,1909 Frankfurt am Main@\textsc{Illing, Meta} (27.\,2.\,1872 Berlin – 26.\,12.\,1909 Frankfurt am Main), \emph{Schauspielerin}|pw} vom »Deutschen Theater\orgindex{Deutsches Theater München@Deutsches Theater München|pw}« in München\oindex{München@\textbf{München}|pw}; MATHIEU PFEIL\pwindex{Pfeil, Mathieu 22.\,3.\,1862 Köln – 14.\,9.\,1939 Zürich@\textsc{Pfeil, Mathieu} (22.\,3.\,1862 Köln – 14.\,9.\,1939 Zürich), \emph{Regisseur, Schauspieler}|pw} vom »Irving Place-Theatre\orgindex{Irving Place Theatre@Irving Place Theatre|pw}« in New-York\oindex{New York City@\textbf{New York City}|pw}; ALBERT ULLRICH\pwindex{Ullrich, Albert 6.\,3.\,1872 Berlin – 4.\,5.\,1946 Braunschweig@\textsc{Ullrich, Albert} (6.\,3.\,1872 Berlin – 4.\,5.\,1946 Braunschweig), \emph{Schauspieler}|pw} vom »Hoftheater\orgindex{Hoftheater Meiningen@Hoftheater Meiningen|pw}« in Meiningen\oindex{Meiningen@\textbf{Meiningen}, \emph{Hauptstadt}|pw}. \uline{\label{T_L00676-1v}\edtext{LOUISE DUMONT}{\lemma{\textnormal{\emph{Louise Dumont}}}\Cendnote{\textnormal{Unterstreichung mit
                        Schreibmaschine}}}\label{T_L00676-1}\pwindex{Dumont, Louise 22.\,2.\,1862 Köln – 16.\,5.\,1932 Düsseldorf@\textsc{Dumont, Louise} (22.\,2.\,1862 Köln – 16.\,5.\,1932 Düsseldorf), \emph{Theaterleiterin, Schauspielerin}|pw}} wird nach einem neuen Uebereinkommen schon von Mitte October ab dem »LESSING-THEATER\orgindex{Lessing-Theater@Lessing-Theater|pw}« zur Verfügung stehen, und \uline{JENNY GROSS\pwindex{Groß, Jenny 5.\,9.\,1862 Andau – 8.\,5.\,1904 Berlin@\textsc{Groß, Jenny} (5.\,9.\,1862 Andau – 8.\,5.\,1904 Berlin), \emph{Schauspielerin}|pw}} schon in der ersten {\pb}Septemberwoche
               ihre künstlerische Thätigkeit wieder aufnehmen. Rechnet man hinzu die erprobten
               Kräfte des »LESSING-THEATER\orgindex{Lessing-Theater@Lessing-Theater|pw}S« – META JAEGER\pwindex{Jaeger, Meta 15.\,9.\,1867 Breslau – 25.\,4.\,1940 Berlin@\textsc{Jaeger, Meta} (15.\,9.\,1867 Breslau – 25.\,4.\,1940 Berlin), \emph{Schauspielerin}|pw} und MARIE ELSINGER\pwindex{Elsinger, Marie *~28.\,2.\,1874 St. Pölten@\textsc{Elsinger, Marie} (*~28.\,2.\,1874 St. Pölten), \emph{Schauspielerin}|pw}, PAULA
                  WIRTH\pwindex{Wirth, Paula 19.\,1.\,1869 München – 13.\,9.\,1922 Bremen@\textsc{Wirth, Paula} (19.\,1.\,1869 München – 13.\,9.\,1922 Bremen), \emph{Schauspielerin}|pw} und SOFIE PAGAY\pwindex{Pagay, Sofie 22.\,4.\,1857 Brünn – 23.\,1.\,1937 Berlin@\textsc{Pagay, Sofie} (22.\,4.\,1857 Brünn – 23.\,1.\,1937 Berlin), \emph{Schauspielerin}|pw}, FRANZ GUTHERY\pwindex{Guthery, Franz 25.\,3.\,1850 Bozen – 4.\,5.\,1900 Berlin@\textsc{Guthery, Franz} (25.\,3.\,1850 Bozen – 4.\,5.\,1900 Berlin), \emph{Schauspieler}|pw} und FRANZ SCHOENFELD\pwindex{Schönfeld, Franz Julius 6.\,11.\,1851 Karlsruhe – 11.\,6.\,1932 Berlin@\textsc{Schönfeld, Franz Julius} (6.\,11.\,1851 Karlsruhe – 11.\,6.\,1932 Berlin), \emph{Regisseur, Schauspieler}|pw}, EMANUEL
                  STOCKHAUSEN\pwindex{Stockhausen, Emanuel 19.\,3.\,1865 Hamburg – 26.\,3.\,1950 Bergen (Chiemgau)@\textsc{Stockhausen, Emanuel} (19.\,3.\,1865 Hamburg – 26.\,3.\,1950 Bergen (Chiemgau)), \emph{Schauspieler}|pw} und CARL WALDOW\pwindex{Waldow, Carl 18.\,6.\,1856 Essen – 23.\,3.\,1909 Berlin@\textsc{Waldow, Carl} (18.\,6.\,1856 Essen – 23.\,3.\,1909 Berlin), \emph{Schauspieler}|pw}, so
               ergiebt sich ein künstlerisches Ensemble, wie es sich nicht eben häufig
               zusammenfindet. Bietet sich in einer Novität eine humoristische Characterrolle von
               besonderer Kraft, so hat sich mir auch GEORG
                  ENGELS\pwindex{Engels, Georg 12.\,1.\,1846 Altona – 31.\,10.\,1907 Berlin@\textsc{Engels, Georg} (12.\,1.\,1846 Altona – 31.\,10.\,1907 Berlin), \emph{Schauspieler}|pw} wiederum für ein längeres Gastspiel zur Verfügung gestellt, und so
               bitte ich Sie freundlichst, mich durch zwei Worte wissen zu lassen, ob ich auf Ihre
               mir so werthvolle Mitarbeiterschaft für den Spielplan des »LESSING-THEATER\orgindex{Lessing-Theater@Lessing-Theater|pw}S« in der nächsten Saison hoffen darf.\pend
           
\pstart
           Mit ergebenstem Gruss{\\[\baselineskip]}\spacefill\mbox{{[}hs. Blumenthal:{]} Dr. Osc. Blumenthal.}\pend
           \leftskip=0em{}\selectlanguage{ngerman}\endnumbering\briefempfaengerindex{Schnitzler, Arthur@\textsc{Schnitzler, Arthur}!zzzBlumenthal, Oskar@\emph{von Oskar Blumenthal}!1897-05-122@{12. 5. 1897}|)be}\mylabel{L00676h}  \newcommand{\dateiname}{L00676}\newcommand{\titel}{Oscar Blumenthal an Arthur Schnitzler, 12. 5. 1897}\newcommand{\editorInnen}{Martin Anton Müller und Gerd-Hermann Susen}%% latex-leseansicht-abspann.tex
%% Abspann für die Leseansicht.
%% Der Schalter \ifkorrekturansicht ist bereits durch den Vorspann gesetzt.

%% latex-abspann.tex
%% Gemeinsamer Abspann für Korrekturansicht und Leseansicht.
%% Setzt den Schalter \ifkorrekturansicht voraus (gesetzt in den
%% einbindenden Dateien latex-korrekturansicht-abspann.tex bzw.
%% latex-leseansicht-abspann.tex).
%% ---------------------------------------------------------------

\normalsize

% Das esempio-Environment wird nur in der Leseansicht benötigt
\ifkorrekturansicht\else
\newenvironment{esempio}[3]%
{
    \vspace{1.5ex}
    \rlap{\underline{#1}}
    \par
    \setlength{\parindent}{0cm}
    \nopagebreak
    \leftskip=#2cm
    \rightskip=#3cm
}
{
    \par
}
\fi

\doendnotes{C}
\bigskip
\vfill

\clearpage

\footnotesize

\ifkorrekturansicht
  \lohead{\textsc{register}}
\fi

% theindex-Environment neu definieren ohne reledmac
\makeatletter
\renewenvironment{theindex}{%
  \ifkorrekturansicht
    \section*{\indexname}%
  \else
    \subsubsection*{Index der erwähnten Entitäten}%
  \fi
  \setlength{\parindent}{0pt}%
  \setlength{\parskip}{0pt plus 0.3pt}%
  \let\item\@idxitem
}{%
  \ifkorrekturansicht\clearpage\fi
}
\makeatother

\IfFileExists{\jobname-pw.ind}{\input{\jobname-pw.ind}}{}

% Quellenangabe nur in der Leseansicht
\ifkorrekturansicht\else
% Fallback-Definitionen, falls die .tex-Datei \titel etc. nicht gesetzt hat
\providecommand{\titel}{}
\providecommand{\editorInnen}{}
\providecommand{\dateiname}{\jobname}

\vspace{3cm}

\vfill

\footnotesize
\textsc{Quelle}: \titel. Herausgegeben von {\editorInnen}. In: \emph{Arthur Schnitzler: Briefwechsel mit Autorinnen und Autoren}.
 Digitale Edition, https://schnitzler-briefe.acdh.oeaw.ac.at/{\dateiname}.html (Stand \today)
\fi

\end{document}


