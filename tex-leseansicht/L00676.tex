%% latex-leseansicht-vorspann.tex
%% Vorspann für die Leseansicht.
%% Lädt die gemeinsame Datei latex-vorspann.tex mit nicht gesetztem Schalter.

\newif\ifkorrekturansicht
\korrekturansichtfalse

\input{../tex-inputs/latex-vorspann}


         
         \renewcommand{\erwaehntePersonen}{Personen: Paula Carlsen, Louise Dumont, Marie Elsinger, Georg Engels, Jenny Groß, Franz Guthery, Alfred Halm, Meta Illing, Meta Jaeger, Adolf Klein, Sofie Pagay, Mathieu Pfeil, Willy Rohland, Franz Julius Schönfeld, Emanuel Stockhausen, Albert Ullrich, Hermann Vallentin, Carl Waldow, Paula Wirth}
         \renewcommand{\erwaehnteInstitutionen}{Institutionen: Deutsches Theater München, Hoftheater Meiningen, Irving Place Theatre, Lessing-Theater}
         \renewcommand{\erwaehnteOrte}{Orte: Berlin, Lauffen, Meiningen, München, Neues Theater, New York City, Schauspielhaus, Theater des Westens, Wien}
         \renewcommand{\erwaehnteWerke}{
               \section[Oscar Blumenthal an Arthur Schnitzler, 12. 5. 1897]{ Oscar Blumenthal an Arthur Schnitzler, 12. 5. 1897}\nopagebreak\mylabel{v}\rehead{ }\begin{ledgroupsized}[t]{13cm}\normalsize\beginnumbering \toendnotes[C]{\smallbreak\pagebreak[2]} \Standort{CUL, Schnitzler, B 15.}
\physDesc{Brief, 1 Blatt, 2 Seiten
\newline{}Schreibmaschine
\newline{}Handschrift: 1) Bleistift (\noindent{}die ersten drei Unterstreichungen)\hspace{1em}2) schwarze Tinte (\noindent{}Unterschrift)\hspace{1em}\newline{}Ordnung: mit Bleistift von unbekannter Hand nummeriert:
                                 »9« }\toendnotes[C]{\smallbreak}\pstart
           \noindent{}\centering{}{\pb}\textcolor{gray}{\textbf{\textsc{Lessing-Theater}\orgindex{Lessing-Theater@Lessing-Theater|pw}}}\pend
           \pstart
           \noindent{}\centering{}\textcolor{gray}{\textbf{DIRECTOR:}}{ }\textcolor{gray}{\textbf{\textsc{Dr. Oscar Blumenthal}.}}\pend
           \pstart
           \noindent{}\raggedleft{}\textcolor{gray}{\textbf{BERLIN N.W. (40)\oindex{Berlin@\textbf{Berlin}|pw}, den}}{ }12. Mai 1897.\pend
           \pstart
           \noindent{}\raggedleft{}z. Zeit: LAUFEN bei ISCHL\oindex{Lauffen@\textbf{Lauffen}|pw}\pend
           \pstart\center{}Sehr geehrter Herr Doctor!\pend\pstart
           Es würde mir eine grosse Freude machen, wenn Sie mir für die nächste Spielzeit des
                  »LESSING-THEATER\orgindex{Lessing-Theater@Lessing-Theater|pw}S« — die letzte unter meiner Direction — ein neues Bühnenwerk aus Ihrer Feder
               anvertrauen würden. Ich gestatte mir, Sie darauf aufmerksam zu machen, dass gerade in
               der nächsten Saison sich der schauspielerische Besitzstand des »LESSING-THEATER\orgindex{Lessing-Theater@Lessing-Theater|pw}S« durch eine Anzahl von sehr vielverheissenden Neu-Engagements beträchtlich
               vermehrt hat. Es werden in den Verband des »LESSING-THEATER\orgindex{Lessing-Theater@Lessing-Theater|pw}S« vom ersten September ab neu eintreten: \uline{ADOLF KLEIN\pwindex{Klein, Adolf 15.08.1847 – 11.03.1931@\textsc{Klein, Adolf} (15.08.1847 – 11.03.1931), \emph{Schauspieler, Theaterdirektor}|pw}} vom Königlichen Schauspielhaus\oindex{Schauspielhaus@\textbf{Schauspielhaus}|pw}; WILLY ROHLAND\pwindex{Rohland, Willy 15.10.1854 – 5.4.1908@\textsc{Rohland, Willy} (15.10.1854 – 5.4.1908)|pw}, ALFRED HALM\pwindex{Halm, Alfred 1861-12-09 – 1951-02-05@\textsc{Halm, Alfred} (1861-12-09 – 1951-02-05), \emph{Schauspieler}|pw} und HERRMANN VALENTIN\pwindex{Vallentin, Hermann 24.05.1872 – 18.9.1945@\textsc{Vallentin, Hermann} (24.05.1872 – 18.9.1945), \emph{Schauspieler}|pw} vom »Theater des Westens\oindex{Theater des Westens@\textbf{Theater des Westens}|pw}«; PAULA CARLSEN\pwindex{Carlsen, Paula 12.7.1837 – 17.03.1900@\textsc{Carlsen, Paula} (12.7.1837 – 17.03.1900), \emph{Schauspielerin}|pw} vom »Neuen Theater\oindex{Neues Theater@\textbf{Neues Theater}|pw}«; META ILLING\pwindex{Illing, Meta 27.02.1872 – 26.12.1909@\textsc{Illing, Meta} (27.02.1872 – 26.12.1909), \emph{Schauspielerin}|pw} vom »Deutschen Theater\orgindex{Deutsches Theater Muenchen@Deutsches Theater München|pw}« in München\oindex{Muenchen@\textbf{München}|pw}; MATHIEU PFEIL\pwindex{Pfeil, Mathieu 22.03.1862 – 14.09.1939@\textsc{Pfeil, Mathieu} (22.03.1862 – 14.09.1939), \emph{Regisseur, Schauspieler}|pw} vom »Irving Place-Theatre\orgindex{Irving Place Theatre@Irving Place Theatre|pw}« in New-York\oindex{New York City@\textbf{New York City}|pw}; ALBERT ULLRICH\pwindex{Ullrich, Albert 6.3.1872 – 4.5.1946@\textsc{Ullrich, Albert} (6.3.1872 – 4.5.1946), \emph{Schauspieler}|pw} vom »Hoftheater\orgindex{Hoftheater Meiningen@Hoftheater Meiningen|pw}« in Meiningen\oindex{Meiningen@\textbf{Meiningen}|pw}. \uline{\label{T_L00676-1v}\edtext{LOUISE
                        DUMONT}{\lemma{\textnormal{\emph{Louise
                        Dumont}}}\Cendnote{\textnormal{Unterstreichung
                        mit Schreibmaschine}}}\label{T_L00676-1h}\pwindex{Dumont, Louise 22.02.1862 – 16.05.1932@\textsc{Dumont, Louise} (22.02.1862 – 16.05.1932), \emph{Theaterleiterin, Schauspielerin}|pw}} wird nach einem neuen Uebereinkommen schon von Mitte October ab dem »LESSING-THEATER\orgindex{Lessing-Theater@Lessing-Theater|pw}« zur Verfügung stehen, und \uline{JENNY GROSS\pwindex{Gross, Jenny 5.9.1862 – 08.05.1904@\textsc{Groß, Jenny} (5.9.1862 – 08.05.1904), \emph{Schauspielerin}|pw}} schon in der ersten {\pb}Septemberwoche
               ihre künstlerische Thätigkeit wieder aufnehmen. Rechnet man hinzu die erprobten
               Kräfte des »LESSING-THEATER\orgindex{Lessing-Theater@Lessing-Theater|pw}S« — META JAEGER\pwindex{Jaeger, Meta 15.09.1867 – 25.04.1940@\textsc{Jaeger, Meta} (15.09.1867 – 25.04.1940), \emph{Schauspielerin}|pw} und MARIE ELSINGER\pwindex{Elsinger, Marie *~28.02.1874@\textsc{Elsinger, Marie} (*~28.02.1874), \emph{Schauspielerin}|pw}, PAULA WIRTH\pwindex{Wirth, Paula 19.1.1869 – 13.9.1922@\textsc{Wirth, Paula} (19.1.1869 – 13.9.1922), \emph{Schauspielerin}|pw}
               und SOFIE PAGAY\pwindex{Pagay, Sofie 22.04.1857 – 23.01.1937@\textsc{Pagay, Sofie} (22.04.1857 – 23.01.1937), \emph{Schauspielerin}|pw}, FRANZ GUTHERY\pwindex{Guthery, Franz 25.3.1850 – 4.5.1900@\textsc{Guthery, Franz} (25.3.1850 – 4.5.1900), \emph{Schauspieler}|pw} und FRANZ SCHOENFELD\pwindex{Schoenfeld, Franz Julius 6.11.1851 – 11.6.1932@\textsc{Schönfeld, Franz Julius} (6.11.1851 – 11.6.1932), \emph{Schauspieler}|pw}, EMANUEL STOCKHAUSEN\pwindex{Stockhausen, Emanuel 1865-03-19 – 1950-03-26@\textsc{Stockhausen, Emanuel} (1865-03-19 – 1950-03-26), \emph{Schauspieler}|pw} und CARL WALDOW\pwindex{Waldow, Carl 18.6.1856 – 23.3.1909@\textsc{Waldow, Carl} (18.6.1856 – 23.3.1909), \emph{Schauspieler}|pw}, so ergiebt sich ein künstlerisches Ensemble, wie
               es sich nicht eben häufig zusammenfindet. Bietet sich in einer Novität eine
               humoristische Characterrolle von besonderer Kraft, so hat sich mir auch GEORG ENGELS\pwindex{Engels, Georg 12.01.1846 – 31.10.1907@\textsc{Engels, Georg} (12.01.1846 – 31.10.1907), \emph{Schauspieler}|pw} wiederum für ein längeres Gastspiel
               zur Verfügung gestellt, und so bitte ich Sie freundlichst, mich durch zwei Worte
               wissen zu lassen, ob ich auf Ihre mir so werthvolle Mitarbeiterschaft für den
               Spielplan des »LESSING-THEATER\orgindex{Lessing-Theater@Lessing-Theater|pw}S« in der nächsten Saison hoffen darf.\pend
           \pstart
           Mit ergebenstem Gruss{\\[\baselineskip]}\spacefill\mbox{{[}hs. Blumenthal:{]} Dr. Osc. Blumenthal.}\pend
           \leftskip=0em{}
         
         \endnumbering\mylabel{h}\end{ledgroupsized}  \newcommand{\dateiname}{L00676}\newcommand{\titel}{Oscar Blumenthal an Arthur Schnitzler, 12. 5. 1897}\newcommand{\editorInnen}{Martin Anton Müller und Gerd-Hermann Susen}%% latex-leseansicht-abspann.tex
%% Abspann für die Leseansicht.
%% Der Schalter \ifkorrekturansicht ist bereits durch den Vorspann gesetzt.

%% latex-abspann.tex
%% Gemeinsamer Abspann für Korrekturansicht und Leseansicht.
%% Setzt den Schalter \ifkorrekturansicht voraus (gesetzt in den
%% einbindenden Dateien latex-korrekturansicht-abspann.tex bzw.
%% latex-leseansicht-abspann.tex).
%% ---------------------------------------------------------------

\normalsize

% Das esempio-Environment wird nur in der Leseansicht benötigt
\ifkorrekturansicht\else
\newenvironment{esempio}[3]%
{
    \vspace{1.5ex}
    \rlap{\underline{#1}}
    \par
    \setlength{\parindent}{0cm}
    \nopagebreak
    \leftskip=#2cm
    \rightskip=#3cm
}
{
    \par
}
\fi

\doendnotes{C}
\bigskip
\vfill

\clearpage

\footnotesize

\ifkorrekturansicht
  \lohead{\textsc{register}}
\fi

% theindex-Environment neu definieren ohne reledmac
\makeatletter
\renewenvironment{theindex}{%
  \ifkorrekturansicht
    \section*{\indexname}%
  \else
    \subsubsection*{Index der erwähnten Entitäten}%
  \fi
  \setlength{\parindent}{0pt}%
  \setlength{\parskip}{0pt plus 0.3pt}%
  \let\item\@idxitem
}{%
  \ifkorrekturansicht\clearpage\fi
}
\makeatother

\IfFileExists{\jobname-pw.ind}{\input{\jobname-pw.ind}}{}

% Quellenangabe nur in der Leseansicht
\ifkorrekturansicht\else
% Fallback-Definitionen, falls die .tex-Datei \titel etc. nicht gesetzt hat
\providecommand{\titel}{}
\providecommand{\editorInnen}{}
\providecommand{\dateiname}{\jobname}

\vspace{3cm}

\vfill

\footnotesize
\textsc{Quelle}: \titel. Herausgegeben von {\editorInnen}. In: \emph{Arthur Schnitzler: Briefwechsel mit Autorinnen und Autoren}.
 Digitale Edition, https://schnitzler-briefe.acdh.oeaw.ac.at/{\dateiname}.html (Stand \today)
\fi

\end{document}


      