%% latex-leseansicht-vorspann.tex
%% Vorspann für die Leseansicht.
%% Lädt die gemeinsame Datei latex-vorspann.tex mit nicht gesetztem Schalter.

\newif\ifkorrekturansicht
\korrekturansichtfalse

\input{../tex-inputs/latex-vorspann}


               \section[Arthur Schnitzler an Richard Beer-Hofmann, 26. 7. 1900]{ Arthur Schnitzler an Richard Beer-Hofmann, 26. 7. 1900}\nopagebreak\mylabel{v}\rehead{ }\begin{ledgroupsized}[t]{13cm}\normalsize\beginnumbering\briefempfaengerindex{Beer-Hofmann, Richard@\textsc{Beer-Hofmann, Richard}!zzzSchnitzler, Arthur@\emph{von Arthur Schnitzler}!1900-07-261@{26. 7. 1900}|(be} \toendnotes[C]{\smallbreak\pagebreak[2]} \Standort{YCGL, MSS 31.}
\physDesc{Telegramm
\newline{}Handschrift einer Schreibkraft: Bleistift, lateinische Kurrent\newline{}Versand: »\noindent{}26/7{ }12\textsuperscript{30}{ / }\textcolor{gray}{\textbf{Von}}{ }Wien 12\oindex{Wien@\textbf{Wien}|pw}{ / }\textcolor{gray}{\textbf{Aufgabe-Nr.}}{ }18{ }\textcolor{gray}{\textbf{mit}} 78\textcolor{gray}{×} \textcolor{gray}{\textbf{Taxworten (}}21 \textcolor{gray}{\textbf{Worten {\dotsfive}
                                                Chiffern)}}{ / }\textcolor{gray}{\textbf{Aufgegeben am {\dotsfive}}}{ / }\textcolor{gray}{\textbf{um}}{ }9 \textcolor{gray}{\textbf{Uhr}}{ }50 \textcolor{gray}{\textbf{Min. {\dots}
                                                  Mittag}}« \newline{}Ordnung: mit Bleistift von unbekannter Hand datiert:
                                            »26. 7. 1900« }\toendnotes[C]{\smallbreak}\pstart{}{\pb}Richard Beerhofmann\pend{}\pstart{}\textcolor{gray}{\textbf{\textit{Alt Auſſee\oindex{Altaussee@\textbf{Altaussee}|pw}}}}\pend{}{\bigskip}\pstart
           \noindent{}{\pb}Heisst jenes \label{T_L01059_1v}\edtext{empfehlenswerthe}{\lemma{\textnormal{\emph{empfehlenswerthe}}}\Cendnote{\textnormal{im Original »Empfeleneswerthe«}}}\label{T_L01059_1h} Gast-haus Aussee\oindex{Bad Aussee@\textbf{Bad Aussee}|pw}{ }wilder Mann\oindex{Gasthaus Wilder Mann@\textbf{Gasthaus Wilder Mann}|pw}? Sehe Sie wahrscheinlich \label{K_L01059_1v}\edtext{Samstag}{\lemma{\textnormal{\emph{Samstag}}}\Cendnote{\textnormal{siehe A. S.: \emph{Tagebuch}, 28. 7. 1900}}}\label{K_L01059_1h} oder Sonntag herzlichst\pend
           \pstart \spacefill\mbox{Arthur}\pend{}\endnumbering\briefempfaengerindex{Beer-Hofmann, Richard@\textsc{Beer-Hofmann, Richard}!zzzSchnitzler, Arthur@\emph{von Arthur Schnitzler}!1900-07-261@{26. 7. 1900}|)be}\mylabel{h}\end{ledgroupsized}  \newcommand{\dateiname}{L01059}\newcommand{\titel}{Arthur Schnitzler an Richard Beer-Hofmann, 26. 7. 1900}\newcommand{\editorInnen}{Martin Anton Müller und Gerd-Hermann Susen}
            \footnotesize
\begin{ledgroupsized}[t]{11.5cm}
\doendnotes{C}
\end{ledgroupsized}
         %% latex-leseansicht-abspann.tex
%% Abspann für die Leseansicht.
%% Der Schalter \ifkorrekturansicht ist bereits durch den Vorspann gesetzt.

%% latex-abspann.tex
%% Gemeinsamer Abspann für Korrekturansicht und Leseansicht.
%% Setzt den Schalter \ifkorrekturansicht voraus (gesetzt in den
%% einbindenden Dateien latex-korrekturansicht-abspann.tex bzw.
%% latex-leseansicht-abspann.tex).
%% ---------------------------------------------------------------

\normalsize

% Das esempio-Environment wird nur in der Leseansicht benötigt
\ifkorrekturansicht\else
\newenvironment{esempio}[3]%
{
    \vspace{1.5ex}
    \rlap{\underline{#1}}
    \par
    \setlength{\parindent}{0cm}
    \nopagebreak
    \leftskip=#2cm
    \rightskip=#3cm
}
{
    \par
}
\fi

\doendnotes{C}
\bigskip
\vfill

\clearpage

\footnotesize

\ifkorrekturansicht
  \lohead{\textsc{register}}
\fi

% theindex-Environment neu definieren ohne reledmac
\makeatletter
\renewenvironment{theindex}{%
  \ifkorrekturansicht
    \section*{\indexname}%
  \else
    \subsubsection*{Index der erwähnten Entitäten}%
  \fi
  \setlength{\parindent}{0pt}%
  \setlength{\parskip}{0pt plus 0.3pt}%
  \let\item\@idxitem
}{%
  \ifkorrekturansicht\clearpage\fi
}
\makeatother

\IfFileExists{\jobname-pw.ind}{\input{\jobname-pw.ind}}{}

% Quellenangabe nur in der Leseansicht
\ifkorrekturansicht\else
% Fallback-Definitionen, falls die .tex-Datei \titel etc. nicht gesetzt hat
\providecommand{\titel}{}
\providecommand{\editorInnen}{}
\providecommand{\dateiname}{\jobname}

\vspace{3cm}

\vfill

\footnotesize
\textsc{Quelle}: \titel. Herausgegeben von {\editorInnen}. In: \emph{Arthur Schnitzler: Briefwechsel mit Autorinnen und Autoren}.
 Digitale Edition, https://schnitzler-briefe.acdh.oeaw.ac.at/{\dateiname}.html (Stand \today)
\fi

\end{document}


      