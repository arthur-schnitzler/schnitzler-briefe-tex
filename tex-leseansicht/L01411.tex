%% latex-leseansicht-vorspann.tex
%% Vorspann für die Leseansicht.
%% Lädt die gemeinsame Datei latex-vorspann.tex mit nicht gesetztem Schalter.

\newif\ifkorrekturansicht
\korrekturansichtfalse

\input{../tex-inputs/latex-vorspann}


         
         \newcommand{\erwaehntePersonen}{Personen: Ottilie Salten, Felix Salten}
         \newcommand{\erwaehnteOrte}{Orte: Porzellangasse, Wien}
         \newcommand{\erwaehnteWerke}{
               \section[Hugo von Hofmannsthal an Arthur Schnitzler, 28. 6. 1904]{ Hugo von Hofmannsthal an Arthur Schnitzler, 28. 6. 1904}\nopagebreak\mylabel{v}\rehead{ }\begin{ledgroupsized}[t]{13cm}\normalsize\beginnumbering \toendnotes[C]{\smallbreak\pagebreak[2]} \Standort{CUL, Schnitzler, B 43.}
\physDesc{Brief, 1 Blatt, 4 Seiten
\newline{}Handschrift: schwarze Tinte, deutsche Kurrent\newline{}Ordnung: 1) mit Bleistift von unbekannter Hand nummeriert: »\strikeout{240}«  2) mit Bleistift von unbekannter Hand nummeriert:
                                    »226«}\buchAbdrucke{\weitereDrucke{Hugo von Hofmannsthal, Arthur Schnitzler: \emph{Briefwechsel}. Hg. Therese Nickl und Heinrich Schnitzler. Frankfurt am Main: \emph{S. Fischer} 1964, S. 189.} }\toendnotes[C]{\smallbreak}\pstart
           \raggedleft{}{\pb}28 VI 1904.\pend
           \pstart{}mein lieber Arthur\pend\pstart
           im Grund bin ich froh, daſs ſich Ihr ſchleichendes Übelbefinden, das mich beſorgt
               gemacht hatte, in dieſer verhältnismäßig harmloſen Form erklärt hat.\pend
           \pstart
           Aber daſs ſich i{\geminationm}er wieder etwas dazwiſchenſtellt und
               dieſe kleinen Zuſa{\geminationm}enkünfte nicht will ſchneller
               aufeinander folgen laſſen. Und doch {\pb}weiß ich unter allem, was das
               Leben mit ſich bringt, faſt nichts ſo ſchönes als ein Nachmittag wie der \label{K_L01411_1v}\edtext{neulich}{\lemma{\textnormal{\emph{neulich}}}\Cendnote{\textnormal{vgl. A. S.: \emph{Tagebuch}, 22. 6. 1904}}}\label{K_L01411_1h}, ein Geſpräch, das manchmal in die tiefſten Tiefen untertaucht und ſich dann
               wieder mit harmloſer Freude an der Oberfläche hält, ein paar Lieder dazwiſchen, {\pb}der Spaziergang, alles das, faſt
               unglaublich viel und ſchön und harmoniſch.\pend
           \pstart
           Ich wollte folgendes vorſchlagen: ſind Sie Anfang nächſter Woche vielleicht wohl
               genug, um an unſerer Geſellſchaft Vergnügen zu finden, noch aber zu ſchwach, um etwas
               zu unternehmen, ſo würden {\pb}wir
               ſehr gern wieder zu Tiſch hinüberko{\geminationm}en, und uns dann für
               den gleichen Tag gegen 6\textsuperscript{h} zu Saltens\pwindex{Salten, Ottilie 07.03.1868 – 22.06.1942@\textsc{Salten, Ottilie} (07.03.1868 – 22.06.1942), \emph{Schauspielerin}|pw}\pwindex{Salten, Felix 06.09.1869 – 08.10.1945@\textsc{Salten, Felix} (06.09.1869 – 08.10.1945), \emph{Schriftsteller, Journalist}|pw} anſagen, dieſe ſpaziergangsweiſe \label{K_L01411_2v}\edtext{aufſuchen}{\lemma{\textnormal{\emph{aufſuchen}}}\Cendnote{\textnormal{Diese lebten in der Porzellangasse 45\oindex{Porzellangasse@\textbf{Porzellangasse}|pwk}.}}}\label{K_L01411_2h}.\pend
           \pstart
           Vielleicht, wenn Ihr Befinden es erlaubt, ſchlagen Sie uns dazu telegraphiſch einen
               Tag vor. Wenn nicht, ſo nicht.\pend
           \pstart
           Von Herzen Ihr{\\[\baselineskip]}\spacefill\mbox{Hugo}\pend
           \leftskip=0em{}
         
         \endnumbering\mylabel{h}\end{ledgroupsized}  \newcommand{\dateiname}{L01411}\newcommand{\titel}{Hugo von Hofmannsthal an Arthur Schnitzler, 28. 6. 1904}\newcommand{\editorInnen}{Martin Anton Müller und Gerd-Hermann Susen}%% latex-leseansicht-abspann.tex
%% Abspann für die Leseansicht.
%% Der Schalter \ifkorrekturansicht ist bereits durch den Vorspann gesetzt.

%% latex-abspann.tex
%% Gemeinsamer Abspann für Korrekturansicht und Leseansicht.
%% Setzt den Schalter \ifkorrekturansicht voraus (gesetzt in den
%% einbindenden Dateien latex-korrekturansicht-abspann.tex bzw.
%% latex-leseansicht-abspann.tex).
%% ---------------------------------------------------------------

\normalsize

% Das esempio-Environment wird nur in der Leseansicht benötigt
\ifkorrekturansicht\else
\newenvironment{esempio}[3]%
{
    \vspace{1.5ex}
    \rlap{\underline{#1}}
    \par
    \setlength{\parindent}{0cm}
    \nopagebreak
    \leftskip=#2cm
    \rightskip=#3cm
}
{
    \par
}
\fi

\doendnotes{C}
\bigskip
\vfill

\clearpage

\footnotesize

\ifkorrekturansicht
  \lohead{\textsc{register}}
\fi

% theindex-Environment neu definieren ohne reledmac
\makeatletter
\renewenvironment{theindex}{%
  \ifkorrekturansicht
    \section*{\indexname}%
  \else
    \subsubsection*{Index der erwähnten Entitäten}%
  \fi
  \setlength{\parindent}{0pt}%
  \setlength{\parskip}{0pt plus 0.3pt}%
  \let\item\@idxitem
}{%
  \ifkorrekturansicht\clearpage\fi
}
\makeatother

\IfFileExists{\jobname-pw.ind}{\input{\jobname-pw.ind}}{}

% Quellenangabe nur in der Leseansicht
\ifkorrekturansicht\else
% Fallback-Definitionen, falls die .tex-Datei \titel etc. nicht gesetzt hat
\providecommand{\titel}{}
\providecommand{\editorInnen}{}
\providecommand{\dateiname}{\jobname}

\vspace{3cm}

\vfill

\footnotesize
\textsc{Quelle}: \titel. Herausgegeben von {\editorInnen}. In: \emph{Arthur Schnitzler: Briefwechsel mit Autorinnen und Autoren}.
 Digitale Edition, https://schnitzler-briefe.acdh.oeaw.ac.at/{\dateiname}.html (Stand \today)
\fi

\end{document}


      