%% latex-leseansicht-vorspann.tex
%% Vorspann für die Leseansicht.
%% Lädt die gemeinsame Datei latex-vorspann.tex mit nicht gesetztem Schalter.

\newif\ifkorrekturansicht
\korrekturansichtfalse

\input{../tex-inputs/latex-vorspann}


\section[Hugo von Hofmannsthal an Arthur Schnitzler, 28. 6. 1904]{L01411 Hugo von Hofmannsthal an Arthur Schnitzler, 28. 6. 1904}
\nopagebreak\mylabel{L01411v}
\rehead{ }\normalsize\beginnumbering\briefempfaengerindex{Schnitzler, Arthur@\textsc{Schnitzler, Arthur}!zzzHofmannsthal, Hugo von@\emph{von Hugo von Hofmannsthal}!1904-06-281@{28. 6. 1904}|(be}
\toendnotes[C]{\smallbreak\pagebreak[2]}
\correspDesc{Versand  durch Hugo von Hofmannsthal am 28. 6. 1904 \textbf{Ort fehlend} 
\newline{}Erhalt  durch Arthur Schnitzler im Zeitraum [28. 6. 1904
                  – 2. 7. 1904?] in Wien}\toendnotes[C]{\smallbreak}
\Standort{CUL, Schnitzler, B 43.}
\physDesc{Brief, 1 Blatt, 4 Seiten, 1118 Zeichen
\newline{}Handschrift: schwarze Tinte, deutsche Kurrent
\newline{}Ordnung: 1) mit Bleistift von unbekannter Hand nummeriert: »\strikeout{240}«  2) mit Bleistift von unbekannter Hand nummeriert: »226«}
\buchAbdrucke{\weitereDrucke{Hugo von Hofmannsthal, Arthur Schnitzler: \emph{Briefwechsel}. Herausgegeben von Therese Nickl und Heinrich Schnitzler. Frankfurt am Main: \emph{S. Fischer} 1964, S. 189.} }\toendnotes[C]{\smallbreak}
\pstart
           \raggedleft{}{\pb}28 VI 1904.\pend
           
\pstart{}mein lieber Arthur\pend\vspace{0.5em}
\pstart
           im Grund bin ich froh, daſs{ }ſich Ihr{ }ſchleichendes \label{K_L01411-1v}\edtext{Übelbefinden}{\lemma{\textnormal{\emph{Übelbefinden}}}\Cendnote{\textnormal{Schnitzler litt seit
            23. 6. 1904 an einer Krankheit, von der man
               seit 
               26. 6. 1904 wusste, dass es Gelbsucht
               war. Am XXXX Auszeichnungsfehler: Dokument L01412 nicht gefunden war die Genesung
               soweit erfolgt, dass er wieder Besuche plante. Am 1. 7. 1904 war er endgültig gesund.}}}\label{K_L01411-1}, das mich beſorgt
               gemacht hatte, in dieſer verhältnismäßig harmloſen Form erklärt hat.\pend
           
\pstart
           Aber daſs{ }ſich i{\geminationm}er wieder etwas dazwiſchenſtellt und
               dieſe kleinen Zuſa{\geminationm}enkünfte nicht will{ }ſchneller
               aufeinander folgen laſſen. Und doch {\pb}weiß ich unter allem, was das
               Leben mit{ }ſich bringt, faſt nichts{ }ſo{ }ſchönes als ein Nachmittag wie der \label{K_L01411-2v}\edtext{neulich}{\lemma{\textnormal{\emph{neulich}}}\Cendnote{\textnormal{Vgl. A. S.: \emph{Tagebuch}, 22. 6. 1904.
               }}}\label{K_L01411-2}, ein Geſpräch, das manchmal in die tiefſten Tiefen untertaucht und{ }ſich dann
               wieder mit harmloſer Freude an der Oberfläche hält, ein paar Lieder dazwiſchen, {\pb}der Spaziergang, alles das, faſt
               unglaublich viel und{ }ſchön und harmoniſch.\pend
           
\pstart
           Ich wollte folgendes vorſchlagen:{ }ſind Sie Anfang nächſter Woche vielleicht wohl
               genug, um an unſerer Geſellſchaft Vergnügen zu finden, noch aber zu{ }ſchwach, um etwas
               zu unternehmen,{ }ſo würden {\pb}wir{ }ſehr gern wieder zu Tiſch hinüberko{\geminationm}en, und uns dann für
               den gleichen Tag gegen 6\textsuperscript{h} zu Saltens\pwindex{Salten, Ottilie 7.\,3.\,1868 Prag – 22.\,6.\,1942 Zürich@\textsc{Salten, Ottilie} (7.\,3.\,1868 Prag – 22.\,6.\,1942 Zürich), \emph{Schauspielerin}|pw}\pwindex{Salten, Felix 6.\,9.\,1869 Budapest – 8.\,10.\,1945 Zürich@\textsc{Salten, Felix} (6.\,9.\,1869 Budapest – 8.\,10.\,1945 Zürich), \emph{Schriftsteller, Journalist, Chefredakteur}|pw} anſagen, dieſe{ }ſpaziergangsweiſe \label{K_L01411-3v}\edtext{aufſuchen}{\lemma{\textnormal{\emph{aufsuchen}}}\Cendnote{\textnormal{Felix\pwindex{Salten, Felix 6.\,9.\,1869 Budapest – 8.\,10.\,1945 Zürich@\textsc{Salten, Felix} (6.\,9.\,1869 Budapest – 8.\,10.\,1945 Zürich), \emph{Schriftsteller, Journalist, Chefredakteur}|pwk} und Ottilie Salten\pwindex{Salten, Ottilie 7.\,3.\,1868 Prag – 22.\,6.\,1942 Zürich@\textsc{Salten, Ottilie} (7.\,3.\,1868 Prag – 22.\,6.\,1942 Zürich), \emph{Schauspielerin}|pwk} lebten im Sommer in der Starkfriedgasse 12\oindex{Starkfriedgassse@\textbf{Starkfriedgassse}, \emph{Straße}|pwk}.}}}\label{K_L01411-3}.\pend
           
\pstart
           Vielleicht, wenn Ihr Befinden es erlaubt,{ }ſchlagen Sie uns dazu telegraphiſch einen
               Tag vor. Wenn nicht,{ }ſo nicht.\pend
           
\pstart
           Von Herzen Ihr{\\[\baselineskip]}\spacefill\mbox{Hugo}\pend
           \leftskip=0em{}\selectlanguage{ngerman}\endnumbering\briefempfaengerindex{Schnitzler, Arthur@\textsc{Schnitzler, Arthur}!zzzHofmannsthal, Hugo von@\emph{von Hugo von Hofmannsthal}!1904-06-281@{28. 6. 1904}|)be}\mylabel{L01411h}  \newcommand{\dateiname}{L01411}\newcommand{\titel}{Hugo von Hofmannsthal an Arthur Schnitzler, 28. 6. 1904}\newcommand{\editorInnen}{Martin Anton Müller und Gerd-Hermann Susen}%% latex-leseansicht-abspann.tex
%% Abspann für die Leseansicht.
%% Der Schalter \ifkorrekturansicht ist bereits durch den Vorspann gesetzt.

%% latex-abspann.tex
%% Gemeinsamer Abspann für Korrekturansicht und Leseansicht.
%% Setzt den Schalter \ifkorrekturansicht voraus (gesetzt in den
%% einbindenden Dateien latex-korrekturansicht-abspann.tex bzw.
%% latex-leseansicht-abspann.tex).
%% ---------------------------------------------------------------

\normalsize

% Das esempio-Environment wird nur in der Leseansicht benötigt
\ifkorrekturansicht\else
\newenvironment{esempio}[3]%
{
    \vspace{1.5ex}
    \rlap{\underline{#1}}
    \par
    \setlength{\parindent}{0cm}
    \nopagebreak
    \leftskip=#2cm
    \rightskip=#3cm
}
{
    \par
}
\fi

\doendnotes{C}
\bigskip
\vfill

\clearpage

\footnotesize

\ifkorrekturansicht
  \lohead{\textsc{register}}
\fi

% theindex-Environment neu definieren ohne reledmac
\makeatletter
\renewenvironment{theindex}{%
  \ifkorrekturansicht
    \section*{\indexname}%
  \else
    \subsubsection*{Index der erwähnten Entitäten}%
  \fi
  \setlength{\parindent}{0pt}%
  \setlength{\parskip}{0pt plus 0.3pt}%
  \let\item\@idxitem
}{%
  \ifkorrekturansicht\clearpage\fi
}
\makeatother

\IfFileExists{\jobname-pw.ind}{\input{\jobname-pw.ind}}{}

% Quellenangabe nur in der Leseansicht
\ifkorrekturansicht\else
% Fallback-Definitionen, falls die .tex-Datei \titel etc. nicht gesetzt hat
\providecommand{\titel}{}
\providecommand{\editorInnen}{}
\providecommand{\dateiname}{\jobname}

\vspace{3cm}

\vfill

\footnotesize
\textsc{Quelle}: \titel. Herausgegeben von {\editorInnen}. In: \emph{Arthur Schnitzler: Briefwechsel mit Autorinnen und Autoren}.
 Digitale Edition, https://schnitzler-briefe.acdh.oeaw.ac.at/{\dateiname}.html (Stand \today)
\fi

\end{document}


