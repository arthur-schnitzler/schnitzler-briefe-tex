%% latex-korrekturansicht-vorspann.tex
%% Vorspann für die Korrekturansicht.
%% Lädt die gemeinsame Datei latex-vorspann.tex mit gesetztem Schalter.

\newif\ifkorrekturansicht
\korrekturansichttrue

\input{../tex-inputs/latex-vorspann}


\section[Hugo von Hofmannsthal an Arthur Schnitzler, 28. 6. 1904]{L01411 Hugo von Hofmannsthal an Arthur Schnitzler, 28. 6. 1904}
\nopagebreak\mylabel{L01411v}
\rehead{ }\normalsize\beginnumbering\briefempfaengerindex{Schnitzler, Arthur@\textsc{Schnitzler, Arthur}!zzzHofmannsthal, Hugo von@\emph{von Hugo von Hofmannsthal}!1904-06-281@{28. 6. 1904}|(be}
\toendnotes[C]{\smallbreak\pagebreak[2]}\Standort{CUL, Schnitzler, B 43.}
\physDesc{Brief, 1 Blatt, 4 Seiten, 1118 Zeichen
\newline{}Handschrift: schwarze Tinte, deutsche Kurrent
\newline{}Ordnung: 1) mit Bleistift von unbekannter Hand nummeriert: »\strikeout{240}«  2) mit Bleistift von unbekannter Hand nummeriert: »226«}
\buchAbdrucke{\weitereDrucke{Hugo von Hofmannsthal, Arthur Schnitzler: \emph{Briefwechsel}. Frankfurt am Main: \emph{S. Fischer} 1964, S. 189.} }\toendnotes[C]{\smallbreak}
\pstart
           \raggedleft{}{\pb}28 VI 1904.\pend
           
\pstart{}mein lieber Arthur\pend\vspace{0.5em}
\pstart
           im Grund bin ich froh, daſs ſich Ihr ſchleichendes \label{K_L01411-1v}\edtext{Übelbefinden}{\lemma{\textnormal{\emph{Übelbefinden}}}\Cendnote{\textnormal{Schnitzler litt seit
            23. 6. 1904 an einer Krankheit, von der man
               seit 
               26. 6. 1904 wusste, dass es Gelbsucht
               war. Am 30. 6. 1904 war die Genesung
               soweit erfolgt, dass er wieder Besuche plante. Am 1. 7. 1904 war er endgültig gesund.}}}\label{K_L01411-1}, das mich beſorgt
               gemacht hatte, in dieſer verhältnismäßig harmloſen Form erklärt hat.\pend
           
\pstart
           Aber daſs ſich i{\geminationm}er wieder etwas dazwiſchenſtellt und
               dieſe kleinen Zuſa{\geminationm}enkünfte nicht will ſchneller
               aufeinander folgen laſſen. Und doch {\pb}weiß ich unter allem, was das
               Leben mit ſich bringt, faſt nichts ſo ſchönes als ein Nachmittag wie der \label{K_L01411-2v}\edtext{neulich}{\lemma{\textnormal{\emph{neulich}}}\Cendnote{\textnormal{Vgl. A. S.: \emph{Tagebuch}, 22. 6. 1904.
               }}}\label{K_L01411-2}, ein Geſpräch, das manchmal in die tiefſten Tiefen untertaucht und ſich dann
               wieder mit harmloſer Freude an der Oberfläche hält, ein paar Lieder dazwiſchen, {\pb}der Spaziergang, alles das, faſt
               unglaublich viel und ſchön und harmoniſch.\pend
           
\pstart
           Ich wollte folgendes vorſchlagen: ſind Sie Anfang nächſter Woche vielleicht wohl
               genug, um an unſerer Geſellſchaft Vergnügen zu finden, noch aber zu ſchwach, um etwas
               zu unternehmen, ſo würden {\pb}wir
               ſehr gern wieder zu Tiſch hinüberko{\geminationm}en, und uns dann für
               den gleichen Tag gegen 6\textsuperscript{h} zu Saltens\pwindex{Salten, Ottilie 07.03.1868 – 22.06.1942@\textsc{Salten, Ottilie} (07.03.1868 – 22.06.1942), \emph{Schauspieler/Schauspielerin}|pw}\pwindex{Salten, Felix 06.09.1869 – 08.10.1945@\textsc{Salten, Felix} (06.09.1869 – 08.10.1945), \emph{Schriftsteller/Schriftstellerin, Journalist/Journalistin, Chefredakteur/Chefredakteurin}|pw} anſagen, dieſe ſpaziergangsweiſe \label{K_L01411-3v}\edtext{aufſuchen}{\lemma{\textnormal{\emph{aufſuchen}}}\Cendnote{\textnormal{Felix\pwindex{Salten, Felix 06.09.1869 – 08.10.1945@\textsc{Salten, Felix} (06.09.1869 – 08.10.1945), \emph{Schriftsteller/Schriftstellerin, Journalist/Journalistin, Chefredakteur/Chefredakteurin}|pwk} und Ottilie Salten\pwindex{Salten, Ottilie 07.03.1868 – 22.06.1942@\textsc{Salten, Ottilie} (07.03.1868 – 22.06.1942), \emph{Schauspieler/Schauspielerin}|pwk} lebten im Sommer in der Starkfriedgasse 12\oindex{Starkfriedgassse@\textbf{Starkfriedgassse}, \emph{Straße (K.STR)}|pwk}.}}}\label{K_L01411-3}.\pend
           
\pstart
           Vielleicht, wenn Ihr Befinden es erlaubt, ſchlagen Sie uns dazu telegraphiſch einen
               Tag vor. Wenn nicht, ſo nicht.\pend
           
\pstart
           Von Herzen Ihr{\\[\baselineskip]}\spacefill\mbox{Hugo}\pend
           \leftskip=0em{}\selectlanguage{ngerman}\endnumbering\briefempfaengerindex{Schnitzler, Arthur@\textsc{Schnitzler, Arthur}!zzzHofmannsthal, Hugo von@\emph{von Hugo von Hofmannsthal}!1904-06-281@{28. 6. 1904}|)be}\mylabel{L01411h}  \normalsize

\doendnotes{C}
\bigskip
\vfill

\clearpage

\footnotesize

\lohead{\textsc{register}}

% Definiere theindex-Environment komplett neu ohne reledmac
\makeatletter
\renewenvironment{theindex}{%
  \section*{\indexname}%
  \setlength{\parindent}{0pt}%
  \setlength{\parskip}{0pt plus 0.3pt}%
  \let\item\@idxitem
}{%
  \clearpage
}
\makeatother

\IfFileExists{\jobname-pw.ind}{\input{\jobname-pw.ind}}{}

\end{document}

      