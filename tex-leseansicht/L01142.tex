%% latex-korrekturansicht-vorspann.tex
%% Vorspann für die Korrekturansicht.
%% Lädt die gemeinsame Datei latex-vorspann.tex mit gesetztem Schalter.

\newif\ifkorrekturansicht
\korrekturansichttrue

\input{../tex-inputs/latex-vorspann}


\section[Arthur Schnitzler an Hugo von Hofmannsthal, {[}4.? 7. 1901{]}]{L01142 Arthur Schnitzler an Hugo von Hofmannsthal, {[}4.? 7. 1901{]}}
\nopagebreak\mylabel{L01142v}
\rehead{ }\normalsize\beginnumbering\briefempfaengerindex{Hofmannsthal, Hugo von@\textsc{Hofmannsthal, Hugo von}!zzzSchnitzler, Arthur@\emph{von Arthur Schnitzler}!1901-07-043@{{[}4.? 7. 1901{]}}|(be}
\toendnotes[C]{\smallbreak\pagebreak[2]}\Standort{FDH, Hs-30885,95.}
\physDesc{Brief, 1 Blatt, 4 Seiten, 999 Zeichen
\newline{}Handschrift: schwarze Tinte, deutsche Kurrent
\newline{}Ordnung: mit Bleistift von unbekannter Hand datiert: »Juni 1901« }
\buchAbdrucke{\weitereDrucke{Hugo von Hofmannsthal, Arthur Schnitzler: \emph{Briefwechsel}. Frankfurt am Main: \emph{S. Fischer} 1964, S. 148–149.} }\toendnotes[C]{\smallbreak}
\pstart
           \noindent{}{\pb}Jüdiſcher Millionärsſohn, auf den Geldſäcken ſeiner Ahnen
               herumprotzender Comoediendichter, Freimaurer und Erniedriger des k. u. k. Hofburgtheaters\orgindex{Burgtheater@Burgtheater|pw}, das hat Ihnen noch gefehlt, daſs
               Sie anonyme \label{K_L01142-1v}\edtext{Schmähkarten}{\lemma{\textnormal{\emph{Schmähkarten}}}\Cendnote{\textnormal{Siehe Hugo von Hofmannsthal an Arthur Schnitzler, 24. 6. 1901.
               }}}\label{K_L01142-1} an anſtändige ſich das Brod mühſelig verdienende deutſche Dichter ſenden, die
               zeitlebens gegen die Macht des Kapitals, gegen die Über{\pb}hebung der Großen, gegen den am Mark des Volks zehrenden Adel und Militarismus
               gekämpft haben! Aber ich werde mich nicht abhalten laſſen. Das nächſte Jahr geht es
               nicht mehr gegen die Infanterieleutenants\pwindex{Lieutenant Gustl. Novelle@\emph{Lieutenant Gustl. Novelle}|pwv}, ſondern gegen die Cavallerieleutenants, insbeſondre
               gegen die in der Reſerve! –\pend
           
\pstart
           Wie gehts Ihnen? Schade dſs {\pb}wir in I{\geminationn}sbruck\oindex{Innsbruck@\textbf{Innsbruck}, \emph{A.ADM2}|pw} nur ſo aneinander \label{K_L01142-2v}\edtext{vorübergesauſt}{\lemma{\textnormal{\emph{vorübergesauſt}}}\Cendnote{\textnormal{Siehe A. S.: \emph{Tagebuch}, 27. 6. 1901.
               }}}\label{K_L01142-2} und geſäuſelt ſind. \label{K_L01142-3v}\edtext{Ich bin
               jetzt in St. Anton\oindex{St. Anton am Arlberg@\textbf{St. Anton am Arlberg}, \emph{A.ADM3}|pw}}{\lemma{\textnormal{\emph{Ich … Anton}}}\Cendnote{\textnormal{Schnitzler hielt sich zwischen 30. 6. 1901 und 12. 7. 1901 in St. Anton am Arlberg\oindex{St. Anton am Arlberg@\textbf{St. Anton am Arlberg}, \emph{A.ADM3}|pwk} auf. Da er an Richard Beer-Hofmann\pwindex{Beer-Hofmann, Richard 1866-07-11 – 1945-09-26@\textsc{Beer-Hofmann, Richard} (1866-07-11 – 1945-09-26), \emph{Schriftsteller/Schriftstellerin}|pwk} am 4. 7. 1901 einen Brief mit teilweise
                  ähnlichem Inhalt sandte, könnte dieses Korrespondenzstück zeitnah entstanden
                  sein.}}}\label{K_L01142-3}, friere, und hoffe bald in den Süden zu radeln. In Salzburg\oindex{Salzburg@\textbf{Salzburg}, \emph{A.ADM2}|pw} hab ich gearbeitet, jetzt weniger. Laſſen Sie recht
               bald von ſich hören aber mehr. (An meine Wien\oindex{Wien@\textbf{Wien}, \emph{A.ADM2}|pw}er
               Adreſſe.) Die Schweſtern\pwindex{Schnitzler, Olga 17.01.1882 – 13.01.1970@\textsc{Schnitzler, Olga} (17.01.1882 – 13.01.1970), \emph{Schauspieler/Schauspielerin, Sänger/Sängerin}|pwv}\pwindex{Steinrueck, Elisabeth 19.11.1885 – 07.04.1920@\textsc{Steinrück, Elisabeth} (19.11.1885 – 07.04.1920)|pwv} grüßen Sie. Ich grüße Sie herzlich und bitte Sie auch Ihre {\pb}Frau\pwindex{Hofmannsthal, Gertrude von 16.03.1880 – 09.11.1959@\textsc{Hofmannsthal, Gertrude von} (16.03.1880 – 09.11.1959)|pwv} zu grüßen.\pend
           
\pstart
           Ihr{\\}\spacefill\mbox{Arthur}\pend
           \selectlanguage{ngerman}\endnumbering\briefempfaengerindex{Hofmannsthal, Hugo von@\textsc{Hofmannsthal, Hugo von}!zzzSchnitzler, Arthur@\emph{von Arthur Schnitzler}!1901-07-043@{{[}4.? 7. 1901{]}}|)be}\mylabel{L01142h}  \normalsize

\doendnotes{C}
\bigskip
\vfill

\clearpage

\footnotesize

\lohead{\textsc{register}}

% Definiere theindex-Environment komplett neu ohne reledmac
\makeatletter
\renewenvironment{theindex}{%
  \section*{\indexname}%
  \setlength{\parindent}{0pt}%
  \setlength{\parskip}{0pt plus 0.3pt}%
  \let\item\@idxitem
}{%
  \clearpage
}
\makeatother

\IfFileExists{\jobname-pw.ind}{\input{\jobname-pw.ind}}{}

\end{document}

      