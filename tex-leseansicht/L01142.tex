%% latex-leseansicht-vorspann.tex
%% Vorspann für die Leseansicht.
%% Lädt die gemeinsame Datei latex-vorspann.tex mit nicht gesetztem Schalter.

\newif\ifkorrekturansicht
\korrekturansichtfalse

\input{../tex-inputs/latex-vorspann}


         
         \renewcommand{\erwaehntePersonen}{Personen: Richard Beer-Hofmann, Hugo von Hofmannsthal, Gertrude von Hofmannsthal, Olga Schnitzler, Elisabeth Steinrück}
         \renewcommand{\erwaehnteInstitutionen}{Institutionen: Burgtheater}
         \renewcommand{\erwaehnteOrte}{Orte: Innsbruck, Salzburg, St. Anton am Arlberg, Wien}
         \renewcommand{\erwaehnteWerke}{Werke: Lieutenant Gustl. Novelle}
               \section[Arthur Schnitzler an Hugo von Hofmannsthal, {[}4.? 7. 1901{]}]{ Arthur Schnitzler an Hugo von Hofmannsthal, {[}4.? 7. 1901{]}}\nopagebreak\mylabel{v}\rehead{ }\begin{ledgroupsized}[t]{13cm}\normalsize\beginnumbering\briefempfaengerindex{Hofmannsthal, Hugo von@\textsc{Hofmannsthal, Hugo von}!zzzSchnitzler, Arthur@\emph{von Arthur Schnitzler}!1901-07-043@{{[}4.? 7. 1901{]}}|(be} \toendnotes[C]{\smallbreak\pagebreak[2]} \Standort{FDH, Hs-30885,95.}
\physDesc{Brief, 1 Blatt, 4 Seiten, 999 Zeichen
\newline{}Handschrift: schwarze Tinte, deutsche Kurrent
\newline{}Ordnung: mit Bleistift von unbekannter Hand datiert: »Juni 1901« }\buchAbdrucke{\weitereDrucke{Hugo von Hofmannsthal, Arthur Schnitzler: \emph{Briefwechsel}. Hg. Therese Nickl und Heinrich Schnitzler. Frankfurt am Main: \emph{S. Fischer} 1964, S. 148–149.} }\toendnotes[C]{\smallbreak}\pstart
           \noindent{}{\pb}Jüdiſcher Millionärsſohn, auf den Geldſäcken ſeiner Ahnen
               herumprotzender Comoediendichter, Freimaurer und Erniedriger des k. u. k. Hofburgtheaters\orgindex{Burgtheater@Burgtheater|pw}, das hat Ihnen noch gefehlt, daſs
               Sie anonyme \label{K_L01142-1v}\edtext{Schmähkarten}{\lemma{\textnormal{\emph{Schmähkarten}}}\Cendnote{\textnormal{Siehe Hugo von Hofmannsthal an Arthur Schnitzler, 24. 6. 1901.
               }}}\label{K_L01142-1h} an anſtändige ſich das Brod mühſelig verdienende deutſche Dichter ſenden, die
               zeitlebens gegen die Macht des Kapitals, gegen die Über{\pb}hebung der Großen, gegen den am Mark des Volks zehrenden Adel und Militarismus
               gekämpft haben! Aber ich werde mich nicht abhalten laſſen. Das nächſte Jahr geht es
               nicht mehr gegen die Infanterieleutenants\pwindex{Schnitzler, Arthur 15.05.1862 – 21.10.1931@\textsc{Schnitzler, Arthur} (15.05.1862 – 21.10.1931), \emph{Schriftsteller, Mediziner}!Lieutenant Gustl. Novelle1900-12-25@\strich\emph{Lieutenant Gustl. Novelle} {[}1900-12-25{]}|pwv}, ſondern gegen die Cavallerieleutenants, insbeſondre
               gegen die in der Reſerve! –\pend
           \pstart
           Wie gehts Ihnen? Schade dſs {\pb}wir in I{\geminationn}sbruck\oindex{Innsbruck@\textbf{Innsbruck}|pw} nur ſo aneinander \label{K_L01142-2v}\edtext{vorübergesauſt}{\lemma{\textnormal{\emph{vorübergesauſt}}}\Cendnote{\textnormal{Siehe A. S.: \emph{Tagebuch}, 27. 6. 1901.
               }}}\label{K_L01142-2h} und geſäuſelt ſind. \label{K_L01142-3v}\edtext{Ich bin
               jetzt in St. Anton\oindex{St. Anton am Arlberg@\textbf{St. Anton am Arlberg}|pw}}{\lemma{\textnormal{\emph{Ich … Anton}}}\Cendnote{\textnormal{Schnitzler\pwindex{Schnitzler, Arthur 15.05.1862 – 21.10.1931@\textsc{Schnitzler, Arthur} (15.05.1862 – 21.10.1931), \emph{Schriftsteller, Mediziner}|pwk} hielt sich zwischen 30. 6. 1901 und 12. 7. 1901 in St. Anton am Arlberg\oindex{St. Anton am Arlberg@\textbf{St. Anton am Arlberg}|pwk} auf. Da er an Richard Beer-Hofmann\pwindex{Beer-Hofmann, Richard 1866-07-11 – 1945-09-26@\textsc{Beer-Hofmann, Richard} (1866-07-11 – 1945-09-26), \emph{Schriftsteller}|pwk} am 4. 7. 1901 einen Brief mit teilweise
                  ähnlichem Inhalt sandte, könnte dieses Korrespondenzstück zeitnah entstanden
                  sein.}}}\label{K_L01142-3h}, friere, und hoffe bald in den Süden zu radeln. In Salzburg\oindex{Salzburg@\textbf{Salzburg}|pw} hab ich gearbeitet, jetzt weniger. Laſſen Sie recht
               bald von ſich hören aber mehr. (An meine Wien\oindex{Wien@\textbf{Wien}|pw}er
               Adreſſe.) Die Schweſtern\pwindex{Schnitzler, Olga 17.01.1882 – 13.01.1970@\textsc{Schnitzler, Olga} (17.01.1882 – 13.01.1970), \emph{Schauspielerin, Sängerin}|pwv}\pwindex{Steinrueck, Elisabeth 19.11.1885 – 07.04.1920@\textsc{Steinrück, Elisabeth} (19.11.1885 – 07.04.1920)|pwv} grüßen Sie. Ich grüße Sie herzlich und bitte Sie auch Ihre {\pb}Frau\pwindex{Hofmannsthal, Gertrude von 16.03.1880 – 09.11.1959@\textsc{Hofmannsthal, Gertrude von} (16.03.1880 – 09.11.1959)|pwv} zu grüßen.\pend
           \pstart
           Ihr{\\}\spacefill\mbox{Arthur}\pend
           
         
         \endnumbering\mylabel{h}\end{ledgroupsized}  \newcommand{\dateiname}{L01142}\newcommand{\titel}{Arthur Schnitzler an Hugo von Hofmannsthal, [4.? 7. 1901]}\newcommand{\editorInnen}{Martin Anton Müller und Gerd-Hermann Susen}%% latex-leseansicht-abspann.tex
%% Abspann für die Leseansicht.
%% Der Schalter \ifkorrekturansicht ist bereits durch den Vorspann gesetzt.

%% latex-abspann.tex
%% Gemeinsamer Abspann für Korrekturansicht und Leseansicht.
%% Setzt den Schalter \ifkorrekturansicht voraus (gesetzt in den
%% einbindenden Dateien latex-korrekturansicht-abspann.tex bzw.
%% latex-leseansicht-abspann.tex).
%% ---------------------------------------------------------------

\normalsize

% Das esempio-Environment wird nur in der Leseansicht benötigt
\ifkorrekturansicht\else
\newenvironment{esempio}[3]%
{
    \vspace{1.5ex}
    \rlap{\underline{#1}}
    \par
    \setlength{\parindent}{0cm}
    \nopagebreak
    \leftskip=#2cm
    \rightskip=#3cm
}
{
    \par
}
\fi

\doendnotes{C}
\bigskip
\vfill

\clearpage

\footnotesize

\ifkorrekturansicht
  \lohead{\textsc{register}}
\fi

% theindex-Environment neu definieren ohne reledmac
\makeatletter
\renewenvironment{theindex}{%
  \ifkorrekturansicht
    \section*{\indexname}%
  \else
    \subsubsection*{Index der erwähnten Entitäten}%
  \fi
  \setlength{\parindent}{0pt}%
  \setlength{\parskip}{0pt plus 0.3pt}%
  \let\item\@idxitem
}{%
  \ifkorrekturansicht\clearpage\fi
}
\makeatother

\IfFileExists{\jobname-pw.ind}{\input{\jobname-pw.ind}}{}

% Quellenangabe nur in der Leseansicht
\ifkorrekturansicht\else
% Fallback-Definitionen, falls die .tex-Datei \titel etc. nicht gesetzt hat
\providecommand{\titel}{}
\providecommand{\editorInnen}{}
\providecommand{\dateiname}{\jobname}

\vspace{3cm}

\vfill

\footnotesize
\textsc{Quelle}: \titel. Herausgegeben von {\editorInnen}. In: \emph{Arthur Schnitzler: Briefwechsel mit Autorinnen und Autoren}.
 Digitale Edition, https://schnitzler-briefe.acdh.oeaw.ac.at/{\dateiname}.html (Stand \today)
\fi

\end{document}


      