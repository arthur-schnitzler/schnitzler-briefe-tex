\input{../tex-inputs/latex-pdf-vorspann}
\begin{center}
            \textcolor{red}{ENTWURF. ENTZIFFERUNG NOCH NICHT KORREKTURGELESEN}
                      \end{center}
            
               \section[Arthur Schnitzler an Hugo von Hofmannsthal, {[}4.? 7. 1901{]}]{ Arthur Schnitzler an Hugo von Hofmannsthal, {[}4.? 7. 1901{]}}\nopagebreak\mylabel{v}\rehead{ }\begin{ledgroupsized}[t]{13cm}\normalsize\beginnumbering\briefempfaengerindex{Hofmannsthal, Hugo von@\textsc{Hofmannsthal, Hugo von}!zzzSchnitzler, Arthur@\emph{von Arthur Schnitzler}!1901-07-043@{{[}4.? 7. 1901{]}}|(be} \toendnotes[C]{\smallbreak\pagebreak[2]} \Standort{FDH, Hs-30885,95.}
\physDesc{Brief, 1 Blatt, 4 Seiten
\newline{}Handschrift: schwarze Tinte, deutsche Kurrent\newline{}Ordnung: mit Bleistift von unbekannter Hand datiert: »Juni 1901« }\buchAbdrucke{\weitereDrucke{Hugo von Hofmannsthal, Arthur Schnitzler: \emph{Briefwechsel}. Hg. Therese Nickl und Heinrich Schnitzler. Frankfurt am Main: \emph{S. Fischer} 1964, S. 148–149.} }\toendnotes[C]{\smallbreak}\pstart
           \noindent{}{\pb}Jüdiſcher Millionärsſohn, auf den Geldſäcken ſeiner Ahnen
               herumprotzender Comoediendichter, Freimaurer und Erniedriger des k. u. k. Hofburgtheaters\orgindex{Burgtheater@Burgtheater|pw}, das hat Ihnen noch gefehlt, daſs Sie
               anonyme Schmähkarten an anſtändige ſich das Brod mühſelig verdienende deutſche
               Dichter ſenden, die zeitlebens gegen die Macht des Kapitals, gegen die Über{\pb}hebung der Großen, gegen den am Mark des Volks zehrenden
               Adel und Militarismus gekämpft haben! Aber ich werde mich nicht abhalten laſſen. Das
               nächſte Jahr geht es nicht mehr gegen die Infanterieleutenants\pwindex{Schnitzler, Arthur 15.05.1862 – 21.10.1931@\textsc{Schnitzler, Arthur} (15.05.1862 – 21.10.1931), \emph{Schriftsteller, Mediziner}!Lieutenant Gustl. Novelle25. 12. 1900@\strich\emph{Lieutenant Gustl. Novelle} {[}25. 12. 1900{]}|pwv}, ſondern gegen die
               Cavallerieleutenants, insbeſondre gegen die in der Reſerve! –\pend
           \pstart
           Wie gehts Ihnen? Schade dſs {\pb}wir in I{\geminationn}sbruck\oindex{Innsbruck@\textbf{Innsbruck}|pw} nur ſo aneinander
               vorübergesauſt und geſäuſelt ſind. \label{K_L01142_1v}\edtext{Ich
               bin jetzt in St. Anton\oindex{St. Anton am Arlberg@\textbf{St. Anton am Arlberg}|pw}}{\lemma{\textnormal{\emph{Ich … Anton}}}\Cendnote{\textnormal{Schnitzler\pwindex{Schnitzler, Arthur 15.05.1862 – 21.10.1931@\textsc{Schnitzler, Arthur} (15.05.1862 – 21.10.1931), \emph{Schriftsteller, Mediziner}|pwk} hielt sich von circa
                     4. 7. 1901 bis vermutlich 9. 7. 1901 in St. Anton am Arlberg\oindex{St. Anton am Arlberg@\textbf{St. Anton am Arlberg}|pwk} auf. Nachdem er an Richard Beer-Hofmann\pwindex{Beer-Hofmann, Richard 11.07.1866 – 26.09.1945@\textsc{Beer-Hofmann, Richard} (11.07.1866 – 26.09.1945), \emph{Schriftsteller}|pwk} am 4. 7. 1901
                  einen Brief mit teilweise ähnlichem Inhalt sandte, könnte dieses
                  Korrespondenzstück zeitnah entstanden sein.}}}\label{K_L01142_1h}, friere, und hoffe bald in den
               Süden zu radeln. In Salzburg\oindex{Salzburg@\textbf{Salzburg}|pw} hab ich gearbeitet,
               jetzt weniger. Laſſen Sie recht bald von ſich hören aber mehr. (An meine Wien\oindex{Wien@\textbf{Wien}|pw}er Adreſſe.) Die Schweſtern\pwindex{Schnitzler, Olga 17.01.1882 – 13.01.1970@\textsc{Schnitzler, Olga} (17.01.1882 – 13.01.1970), \emph{Schauspielerin, Sängerin}|pwv}\pwindex{Steinrueck, Elisabeth 19.11.1885 – 07.04.1920@\textsc{Steinrück, Elisabeth} (19.11.1885 – 07.04.1920)|pwv} grüßen Sie. Ich grüße Sie
               herzlich und bitte Sie auch Ihre {\pb}Frau\pwindex{Hofmannsthal, Gertrude von 16.03.1880 – 09.11.1959@\textsc{Hofmannsthal, Gertrude von} (16.03.1880 – 09.11.1959)|pwv} zu grüßen.\pend
           \pstart
           Ihr{\\}\spacefill\mbox{Arthur}\pend
           \endnumbering\briefempfaengerindex{Hofmannsthal, Hugo von@\textsc{Hofmannsthal, Hugo von}!zzzSchnitzler, Arthur@\emph{von Arthur Schnitzler}!1901-07-043@{{[}4.? 7. 1901{]}}|)be}\mylabel{h}\end{ledgroupsized}  \newcommand{\dateiname}{L01142}\newcommand{\titel}{Arthur Schnitzler an Hugo von Hofmannsthal, [4.? 7. 1901]}\newcommand{\editorInnen}{Martin Anton Müller und Gerd-Hermann Susen}\input{../tex-inputs/latex-pdf-abspann}
      