%% latex-korrekturansicht-vorspann.tex
%% Vorspann für die Korrekturansicht.
%% Lädt die gemeinsame Datei latex-vorspann.tex mit gesetztem Schalter.

\newif\ifkorrekturansicht
\korrekturansichttrue

\input{../tex-inputs/latex-vorspann}


\section[Max Burckhard an Arthur Schnitzler, {[}Juni 1907?{]}]{L01681 Max Burckhard an Arthur Schnitzler, {[}Juni 1907?{]}}
\nopagebreak\mylabel{L01681v}
\rehead{ }\normalsize\beginnumbering\briefempfaengerindex{Schnitzler, Arthur@\textsc{Schnitzler, Arthur}!zzzBurckhard, Max Eugen@\emph{von Max Eugen Burckhard}!1907-06-281@{{[}Juni 1907?{]}}|(be}
\toendnotes[C]{\smallbreak\pagebreak[2]}\Standort{CUL, Schnitzler, B 20.}
\physDesc{Brief, 1 Blatt, 1 Seite, 320 Zeichen
\newline{}Handschrift: schwarze Tinte, deutsche Kurrent
\newline{}Ordnung: von Schnitzler mit Bleistift datiert: »So{\geminationm}er 907«, von unbekannter Hand mit Bleistift nummeriert:
                                    »18« }\toendnotes[C]{\smallbreak}
\pstart
           {\pb}\textcolor{gray}{\textbf{D\textsuperscript{r.} Max Burckhard}}\hfill \textcolor{gray}{\textbf{Wien, IX. Porzellangasse 48\oindex{Porzellangasse@\textbf{Porzellangasse}, \emph{Straße (K.STR)}|pw}{ }..........}}\pend
           
\pstart
           \raggedleft{}\textcolor{gray}{\textbf{St. Gilgen\oindex{St. Gilgen@\textbf{St. Gilgen}, \emph{A.ADM3}|pw}}}\hspace*{3.5em}\pend
           
\pstart{}Sehr verehrter lieber Herr Doctor!\pend\vspace{0.5em}
\pstart
           Das \label{K_L01681-1v}\edtext{Wirtshaus}{\lemma{\textnormal{\emph{Wirtshaus}}}\Cendnote{\textnormal{Schnitzler war am 28. 6. 1907
                  in der Unterkunft. Entsprechend dürfte die Empfehlung vorher übermittelt worden
                  sein. Die Angabe Schnitzlers »So{\geminationm}er 907«, sofern sie sich nicht einzig am
                  Zeitpunkt der Reise orientieren sollte, erlaubt eine Einschränkung auf
                     Juni.}}}\label{K_L01681-1} heißt »Die
                  Wochein\oindex{Die Wochein@\textbf{Die Wochein}, \emph{Hotel (K.HTL)}|pw}«, hat einen See \introOben{}(Wocheinersee\oindex{Wocheiner See@\textbf{Wocheiner See}, \emph{See (N.SEE)}|pw})\introOben{}{ }\uline{u.} gute Küche, liegt 2 Stunden ober Veldes\oindex{Bled@\textbf{Bled}, \emph{P.PPLA}|pw} (leider geht jetzt eine Bahn hin), es wird von der Frau\pwindex{Stoehr, Friederike *~01.11.1865@\textsc{Stöhr, Friederike} (*~01.11.1865), \emph{Hotelier/Hotelière}|pwv} des Malers Stöhr\pwindex{Stoehr, Ernst 01.11.1860 – 17.06.1917@\textsc{Stöhr, Ernst} (01.11.1860 – 17.06.1917), \emph{Maler/Malerin}|pw} bewirtſchaftet. Es ſoll \uline{nicht} heiß ſein im So{\geminationm}er.
               Schöne Gemsjagden, alſo auch Gemſen vorhanden!\pend
           
\pstart
           Herzlichſt{\\[\baselineskip]}\spacefill\mbox{DrBurckhard}\pend
           \leftskip=0em{}\selectlanguage{ngerman}\endnumbering\briefempfaengerindex{Schnitzler, Arthur@\textsc{Schnitzler, Arthur}!zzzBurckhard, Max Eugen@\emph{von Max Eugen Burckhard}!1907-06-011@{{[}Juni 1907?{]}}|)be}\mylabel{L01681h}  \normalsize

\doendnotes{C}
\bigskip
\vfill

\clearpage

\footnotesize

\lohead{\textsc{register}}

% Definiere theindex-Environment komplett neu ohne reledmac
\makeatletter
\renewenvironment{theindex}{%
  \section*{\indexname}%
  \setlength{\parindent}{0pt}%
  \setlength{\parskip}{0pt plus 0.3pt}%
  \let\item\@idxitem
}{%
  \clearpage
}
\makeatother

\IfFileExists{\jobname-pw.ind}{\input{\jobname-pw.ind}}{}

\end{document}

      