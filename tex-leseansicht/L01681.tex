%% latex-leseansicht-vorspann.tex
%% Vorspann für die Leseansicht.
%% Lädt die gemeinsame Datei latex-vorspann.tex mit nicht gesetztem Schalter.

\newif\ifkorrekturansicht
\korrekturansichtfalse

\input{../tex-inputs/latex-vorspann}


         
         \renewcommand{\erwaehntePersonen}{Personen: Max Eugen Burckhard, Friederike Stöhr, Ernst Stöhr}
         \renewcommand{\erwaehnteOrte}{Orte: Bled, Die Wochein, Porzellangasse, St. Gilgen, Wien, Wocheiner See}
         \renewcommand{\erwaehnteWerke}{}
               \section[Max Burckhard an Arthur Schnitzler, {[}Juni 1907?{]}]{ Max Burckhard an Arthur Schnitzler, {[}Juni 1907?{]}}\nopagebreak\mylabel{v}\rehead{ }\begin{ledgroupsized}[t]{13cm}\normalsize\beginnumbering\briefempfaengerindex{Schnitzler, Arthur@\textsc{Schnitzler, Arthur}!zzzBurckhard, Max Eugen@\emph{von Max Eugen Burckhard}!1907-06-281@{{[}Juni 1907?{]}}|(be} \toendnotes[C]{\smallbreak\pagebreak[2]} \Standort{CUL, Schnitzler, B 20.}
\physDesc{Brief, 1 Blatt, 1 Seite, 320 Zeichen
\newline{}Handschrift: schwarze Tinte, deutsche Kurrent
\newline{}Ordnung: von Schnitzler mit Bleistift datiert: »So{\geminationm}er 907«, von unbekannter Hand mit Bleistift nummeriert:
                                    »18« }\toendnotes[C]{\smallbreak}\pstart
           \noindent{}{\pb}\textcolor{gray}{\textbf{D\textsuperscript{r.} Max Burckhard}}\hfill \textcolor{gray}{\textbf{Wien, IX. Porzellangasse 48\oindex{Porzellangasse@\textbf{Porzellangasse}|pw}{ }..........}}\pend
           \pstart
           \raggedleft{}\textcolor{gray}{\textbf{St. Gilgen\oindex{St. Gilgen@\textbf{St. Gilgen}|pw}}}\hspace*{3.5em}\pend
           \pstart{}Sehr verehrter lieber Herr Doctor!\pend\pstart
           Das \label{K_L01681-1v}\edtext{Wirtshaus}{\lemma{\textnormal{\emph{Wirtshaus}}}\Cendnote{\textnormal{Schnitzler\pwindex{Schnitzler, Arthur 15.05.1862 – 21.10.1931@\textsc{Schnitzler, Arthur} (15.05.1862 – 21.10.1931), \emph{Schriftsteller, Mediziner}|pwk} war am 28. 6. 1907
                  in der Unterkunft. Entsprechend dürfte die Empfehlung vorher übermittelt worden
                  sein. Die Angabe Schnitzlers\pwindex{Schnitzler, Arthur 15.05.1862 – 21.10.1931@\textsc{Schnitzler, Arthur} (15.05.1862 – 21.10.1931), \emph{Schriftsteller, Mediziner}|pwk} »So{\geminationm}er 907«, sofern sie sich nicht einzig am
                  Zeitpunkt der Reise orientieren sollte, erlaubt eine Einschränkung auf
                     Juni.}}}\label{K_L01681-1h} heißt »Die
                  Wochein\oindex{Die Wochein@\textbf{Die Wochein}|pw}«, hat einen See \introOben{}(Wocheinersee\oindex{Wocheiner See@\textbf{Wocheiner See}|pw})\introOben{}{ }\uline{u.} gute Küche, liegt 2 Stunden ober Veldes\oindex{Bled@\textbf{Bled}|pw} (leider geht jetzt eine Bahn hin), es wird von der Frau\pwindex{Stoehr, Friederike *~01.11.1865@\textsc{Stöhr, Friederike} (*~01.11.1865), \emph{Hotelière}|pwv} des Malers Stöhr\pwindex{Stoehr, Ernst 01.11.1860 – 17.06.1917@\textsc{Stöhr, Ernst} (01.11.1860 – 17.06.1917), \emph{Maler}|pw} bewirtſchaftet. Es ſoll \uline{nicht} heiß ſein im So{\geminationm}er.
               Schöne Gemsjagden, alſo auch Gemſen vorhanden!\pend
           \pstart
           Herzlichſt{\\[\baselineskip]}\spacefill\mbox{DrBurckhard}\pend
           \leftskip=0em{}
         
         \endnumbering\mylabel{h}\end{ledgroupsized}  \newcommand{\dateiname}{L01681}\newcommand{\titel}{Max Burckhard an Arthur Schnitzler, [Juni 1907?]}\newcommand{\editorInnen}{Martin Anton Müller und Gerd-Hermann Susen}%% latex-leseansicht-abspann.tex
%% Abspann für die Leseansicht.
%% Der Schalter \ifkorrekturansicht ist bereits durch den Vorspann gesetzt.

%% latex-abspann.tex
%% Gemeinsamer Abspann für Korrekturansicht und Leseansicht.
%% Setzt den Schalter \ifkorrekturansicht voraus (gesetzt in den
%% einbindenden Dateien latex-korrekturansicht-abspann.tex bzw.
%% latex-leseansicht-abspann.tex).
%% ---------------------------------------------------------------

\normalsize

% Das esempio-Environment wird nur in der Leseansicht benötigt
\ifkorrekturansicht\else
\newenvironment{esempio}[3]%
{
    \vspace{1.5ex}
    \rlap{\underline{#1}}
    \par
    \setlength{\parindent}{0cm}
    \nopagebreak
    \leftskip=#2cm
    \rightskip=#3cm
}
{
    \par
}
\fi

\doendnotes{C}
\bigskip
\vfill

\clearpage

\footnotesize

\ifkorrekturansicht
  \lohead{\textsc{register}}
\fi

% theindex-Environment neu definieren ohne reledmac
\makeatletter
\renewenvironment{theindex}{%
  \ifkorrekturansicht
    \section*{\indexname}%
  \else
    \subsubsection*{Index der erwähnten Entitäten}%
  \fi
  \setlength{\parindent}{0pt}%
  \setlength{\parskip}{0pt plus 0.3pt}%
  \let\item\@idxitem
}{%
  \ifkorrekturansicht\clearpage\fi
}
\makeatother

\IfFileExists{\jobname-pw.ind}{\input{\jobname-pw.ind}}{}

% Quellenangabe nur in der Leseansicht
\ifkorrekturansicht\else
% Fallback-Definitionen, falls die .tex-Datei \titel etc. nicht gesetzt hat
\providecommand{\titel}{}
\providecommand{\editorInnen}{}
\providecommand{\dateiname}{\jobname}

\vspace{3cm}

\vfill

\footnotesize
\textsc{Quelle}: \titel. Herausgegeben von {\editorInnen}. In: \emph{Arthur Schnitzler: Briefwechsel mit Autorinnen und Autoren}.
 Digitale Edition, https://schnitzler-briefe.acdh.oeaw.ac.at/{\dateiname}.html (Stand \today)
\fi

\end{document}


      