%% latex-leseansicht-vorspann.tex
%% Vorspann für die Leseansicht.
%% Lädt die gemeinsame Datei latex-vorspann.tex mit nicht gesetztem Schalter.

\newif\ifkorrekturansicht
\korrekturansichtfalse

\input{../tex-inputs/latex-vorspann}


         
         \newcommand{\erwaehntePersonen}{Personen: Else Lasker-Schüler}
         \newcommand{\erwaehnteOrte}{Orte: Pension Bleckmann, Wien}
         \newcommand{\erwaehnteWerke}{
               \section[Paul Lasker-Schüler an Arthur Schnitzler, 28. 4. {[}1925?{]}]{ Paul Lasker-Schüler an Arthur Schnitzler, 28. 4. {[}1925?{]}}\nopagebreak\mylabel{v}\rehead{ }\begin{ledgroupsized}[t]{13cm}\normalsize\beginnumbering \toendnotes[C]{\smallbreak\pagebreak[2]} \Standort{DLA, A:Schnitzler, HS.1985.1.3876.}
\physDesc{Brief, 1 Blatt, 1 Seite, maschinelle Abschrift
\newline{}Schreibmaschine}\toendnotes[C]{\smallbreak}\pstart
           {\pb}\label{K_L02654-1v}\edtext{28. IV.}{\lemma{\textnormal{\emph{28. IV.}}}\Cendnote{\textnormal{Es ist kein Besuch Paul Lasker-Schüler\pwindex{Lasker-Schueler, Paul 1899-08-24 – 1927-12-14@\textsc{Lasker-Schüler, Paul} (1899-08-24 – 1927-12-14)|pwk}s bei Schnitzler\pwindex{Schnitzler, Arthur 15.05.1862 – 21.10.1931@\textsc{Schnitzler, Arthur} (15.05.1862 – 21.10.1931), \emph{Schriftsteller, Mediziner}|pwk} bekannt, durch den das Datum des Briefes gesichert bestimmt
                     werden könnte. Da der Brief nur in Abschrift vorliegt, lässt sich nicht 
                     mit Gewissheit ausschließen, dass der Abschreiber, die Abschreiberin bei der Entzifferung der Monatsangabe keinen
                     Fehler gemacht hat. Mit Hilfe der freundlichen Auskunft von Karl Jürgen
                     Skrodzki lässt sich folgende Argumentation führen, warum der Brief 1925
                     entstanden sein muss. Unter der Annahme, dass die Monatsangabe stimmt, kommen nur
                     die Jahre 1924 und 1925 in Betracht, da sich
                     hier Paul Lasker-Schüler\pwindex{Lasker-Schueler, Paul 1899-08-24 – 1927-12-14@\textsc{Lasker-Schüler, Paul} (1899-08-24 – 1927-12-14)|pwk} im April in Wien\oindex{Wien@\textbf{Wien}|pwk}
                     aufhielt. Der Brief wurde mit großer Wahrscheinlichkeit nicht vor jenem Else Lasker-Schüler\pwindex{Lasker-Schueler, Else 11.02.1869 – 22.01.1945@\textsc{Lasker-Schüler, Else} (11.02.1869 – 22.01.1945), \emph{Dichterin}|pwk}s an Schnitzler\pwindex{Schnitzler, Arthur 15.05.1862 – 21.10.1931@\textsc{Schnitzler, Arthur} (15.05.1862 – 21.10.1931), \emph{Schriftsteller, Mediziner}|pwk}
                     (10. 12. 1924) verfasst. Paul Lasker-Schüler hätte ohne
                     diese Vorarbeit seiner Mutter\pwindex{Lasker-Schueler, Else 11.02.1869 – 22.01.1945@\textsc{Lasker-Schüler, Else} (11.02.1869 – 22.01.1945), \emph{Dichterin}|pwkv} vermutlich nicht an Schnitzler\pwindex{Schnitzler, Arthur 15.05.1862 – 21.10.1931@\textsc{Schnitzler, Arthur} (15.05.1862 – 21.10.1931), \emph{Schriftsteller, Mediziner}|pwk} geschrieben.}}}\label{K_L02654-1h}\pend
           \pstart
           \centering{}Paul Lasker-Schüler\pend
           {\bigskip}\pstart
           \noindent{}bittet vielmals darum, empfangen zu werden, wenn es irgend
               möglich vielleicht noch heute, denn es handelt sich um einen \label{K_L02654-2v}\edtext{medizinischen Ratschlag}{\lemma{\textnormal{\emph{medizinischen Ratschlag}}}\Cendnote{\textnormal{Im Dezember 1925 erkrankte Paul Lasker-Schüler an Tuberkulose.}}}\label{K_L02654-2h}.\pend
           \pstart
           Ich bitte Sie vielmals Herr Doktor, mir meine Aufdringlichkeit nicht übel zu nehmen.
               Meine Adresse ist Pension Bleckmann\oindex{Pension Bleckmann@\textbf{Pension Bleckmann}|pw}{ }{\\}Thelephon 26 206.\pend
           
         
         \endnumbering\mylabel{h}\end{ledgroupsized}  \newcommand{\dateiname}{L02654}\newcommand{\titel}{Paul Lasker-Schüler an Arthur Schnitzler, 28. 4. [1925?]}\newcommand{\editorInnen}{Martin Anton Müller und Laura Untner}%% latex-leseansicht-abspann.tex
%% Abspann für die Leseansicht.
%% Der Schalter \ifkorrekturansicht ist bereits durch den Vorspann gesetzt.

%% latex-abspann.tex
%% Gemeinsamer Abspann für Korrekturansicht und Leseansicht.
%% Setzt den Schalter \ifkorrekturansicht voraus (gesetzt in den
%% einbindenden Dateien latex-korrekturansicht-abspann.tex bzw.
%% latex-leseansicht-abspann.tex).
%% ---------------------------------------------------------------

\normalsize

% Das esempio-Environment wird nur in der Leseansicht benötigt
\ifkorrekturansicht\else
\newenvironment{esempio}[3]%
{
    \vspace{1.5ex}
    \rlap{\underline{#1}}
    \par
    \setlength{\parindent}{0cm}
    \nopagebreak
    \leftskip=#2cm
    \rightskip=#3cm
}
{
    \par
}
\fi

\doendnotes{C}
\bigskip
\vfill

\clearpage

\footnotesize

\ifkorrekturansicht
  \lohead{\textsc{register}}
\fi

% theindex-Environment neu definieren ohne reledmac
\makeatletter
\renewenvironment{theindex}{%
  \ifkorrekturansicht
    \section*{\indexname}%
  \else
    \subsubsection*{Index der erwähnten Entitäten}%
  \fi
  \setlength{\parindent}{0pt}%
  \setlength{\parskip}{0pt plus 0.3pt}%
  \let\item\@idxitem
}{%
  \ifkorrekturansicht\clearpage\fi
}
\makeatother

\IfFileExists{\jobname-pw.ind}{\input{\jobname-pw.ind}}{}

% Quellenangabe nur in der Leseansicht
\ifkorrekturansicht\else
% Fallback-Definitionen, falls die .tex-Datei \titel etc. nicht gesetzt hat
\providecommand{\titel}{}
\providecommand{\editorInnen}{}
\providecommand{\dateiname}{\jobname}

\vspace{3cm}

\vfill

\footnotesize
\textsc{Quelle}: \titel. Herausgegeben von {\editorInnen}. In: \emph{Arthur Schnitzler: Briefwechsel mit Autorinnen und Autoren}.
 Digitale Edition, https://schnitzler-briefe.acdh.oeaw.ac.at/{\dateiname}.html (Stand \today)
\fi

\end{document}


      