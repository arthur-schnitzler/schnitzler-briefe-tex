%% latex-leseansicht-vorspann.tex
%% Vorspann für die Leseansicht.
%% Lädt die gemeinsame Datei latex-vorspann.tex mit nicht gesetztem Schalter.

\newif\ifkorrekturansicht
\korrekturansichtfalse

\input{../tex-inputs/latex-vorspann}


\section[Paul Lasker-Schüler an Arthur Schnitzler, 28. 4. [1925?]]{L02654 Paul Lasker-Schüler an Arthur Schnitzler, 28. 4. [1925?]}
\nopagebreak\mylabel{L02654v}
\rehead{ }\normalsize\beginnumbering\briefempfaengerindex{Schnitzler, Arthur@\textsc{Schnitzler, Arthur}!zzzLasker-Schüler, Paul@\emph{von Paul Lasker-Schüler}!1925-04-281@{28. 4. [1925?]}|(be}
\toendnotes[C]{\smallbreak\pagebreak[2]}
\correspDesc{Versand  durch Paul Lasker-Schüler am 28. 4. [1925?] in Wien
\newline{}Erhalt  durch Arthur Schnitzler im Zeitraum [28. 4. 1925
                  – 2. 5. 1925?] in Wien}\toendnotes[C]{\smallbreak}
\Standort{DLA, A:Schnitzler, HS.1985.1.3876.}
\physDesc{Brief, maschinenschriftliche Abschrift, 1 Blatt, 1 Seite, 306 Zeichen
\newline{}Schreibmaschine}\toendnotes[C]{\smallbreak}
\pstart
           {\pb}\label{K_L02654-1v}\edtext{28. IV.}{\lemma{\textnormal{\emph{28. IV.}}}\Cendnote{\textnormal{Es ist kein Besuch Paul Lasker-Schülers\pwindex{Lasker-Schüler, Paul 24.\,8.\,1899 Berlin – 14.\,12.\,1927 ebd.@\textsc{Lasker-Schüler, Paul} (24.\,8.\,1899 Berlin – 14.\,12.\,1927 ebd.)|pwk} bei Schnitzler bekannt, durch den das Datum des Briefes
                     gesichert bestimmt werden könnte. Da der Brief nur in Abschrift vorliegt, lässt
                     sich nicht mit Gewissheit ausschließen, dass die Schreibkraft, die die Abschrift übernahm,
                     bei der Entzifferung der Monatsangabe keinen Fehler gemacht hat. Mit Hilfe der
                     freundlichen Auskunft von Karl Jürgen Skrodzki lässt sich folgende
                     Argumentation führen, warum der Brief 1925 entstanden sein muss.
                     Unter der Annahme, dass die Monatsangabe stimmt, kommen nur die Jahre
                        1924 und 1925 in Betracht, da sich in dieser Zeit Paul Lasker-Schüler\pwindex{Lasker-Schüler, Paul 24.\,8.\,1899 Berlin – 14.\,12.\,1927 ebd.@\textsc{Lasker-Schüler, Paul} (24.\,8.\,1899 Berlin – 14.\,12.\,1927 ebd.)|pwk} im April in Wien\oindex{Wien@\textbf{Wien}, \emph{Verwaltungsgebiet}|pwk} aufhielt. Der Brief wurde mit großer
                     Wahrscheinlichkeit nicht vor jenem Else
                        Lasker-Schülers\pwindex{Lasker-Schüler, Else 11.\,2.\,1869 Elberfeld – 22.\,1.\,1945 Jerusalem@\textsc{Lasker-Schüler, Else} (11.\,2.\,1869 Elberfeld – 22.\,1.\,1945 Jerusalem), \emph{Dichterin}|pwk} an Schnitzler
                        (XXXX Auszeichnungsfehler: Dokument L02653 nicht gefunden)
                     verfasst. Paul Lasker-Schüler hätte ohne diese Vorarbeit seiner Mutter\pwindex{Lasker-Schüler, Else 11.\,2.\,1869 Elberfeld – 22.\,1.\,1945 Jerusalem@\textsc{Lasker-Schüler, Else} (11.\,2.\,1869 Elberfeld – 22.\,1.\,1945 Jerusalem), \emph{Dichterin}|pwkv} vermutlich nicht
                     an Schnitzler geschrieben.}}}\label{K_L02654-1}\pend
           \vspace{0.5em}
\pstart
           \centering{}Paul Lasker-Schüler\pend
           {\vspace{1\baselineskip}}
\pstart
           bittet vielmals darum, empfangen zu werden, wenn es irgend möglich vielleicht noch
                  heute, denn es handelt sich um einen \label{K_L02654-2v}\edtext{medizinischen Ratschlag}{\lemma{\textnormal{\emph{medizinischen Ratschlag}}}\Cendnote{\textnormal{Im Dezember 1925 erkrankte Paul
                  Lasker-Schüler an Tuberkulose.}}}\label{K_L02654-2}.\pend
           
\pstart
           Ich bitte Sie vielmals Herr Doktor, mir meine Aufdringlichkeit nicht übel zu nehmen.
               Meine Adresse ist Pension Bleckmann\oindex{Wien@\textbf{Wien}!IX., Alsergrund@\textbf{IX., Alsergrund}!Pension Bleckmann@\textbf{Pension Bleckmann}, \emph{Hotel}|pw}{ }{\\}Thelephon 26 206.\pend
           \selectlanguage{ngerman}\endnumbering\briefempfaengerindex{Schnitzler, Arthur@\textsc{Schnitzler, Arthur}!zzzLasker-Schüler, Paul@\emph{von Paul Lasker-Schüler}!1925-04-281@{28. 4. [1925?]}|)be}\mylabel{L02654h}  \newcommand{\dateiname}{L02654}\newcommand{\titel}{Paul Lasker-Schüler an Arthur Schnitzler, 28. 4. [1925?]}\newcommand{\editorInnen}{Martin Anton Müller und Laura Untner}%% latex-leseansicht-abspann.tex
%% Abspann für die Leseansicht.
%% Der Schalter \ifkorrekturansicht ist bereits durch den Vorspann gesetzt.

%% latex-abspann.tex
%% Gemeinsamer Abspann für Korrekturansicht und Leseansicht.
%% Setzt den Schalter \ifkorrekturansicht voraus (gesetzt in den
%% einbindenden Dateien latex-korrekturansicht-abspann.tex bzw.
%% latex-leseansicht-abspann.tex).
%% ---------------------------------------------------------------

\normalsize

% Das esempio-Environment wird nur in der Leseansicht benötigt
\ifkorrekturansicht\else
\newenvironment{esempio}[3]%
{
    \vspace{1.5ex}
    \rlap{\underline{#1}}
    \par
    \setlength{\parindent}{0cm}
    \nopagebreak
    \leftskip=#2cm
    \rightskip=#3cm
}
{
    \par
}
\fi

\doendnotes{C}
\bigskip
\vfill

\clearpage

\footnotesize

\ifkorrekturansicht
  \lohead{\textsc{register}}
\fi

% theindex-Environment neu definieren ohne reledmac
\makeatletter
\renewenvironment{theindex}{%
  \ifkorrekturansicht
    \section*{\indexname}%
  \else
    \subsubsection*{Index der erwähnten Entitäten}%
  \fi
  \setlength{\parindent}{0pt}%
  \setlength{\parskip}{0pt plus 0.3pt}%
  \let\item\@idxitem
}{%
  \ifkorrekturansicht\clearpage\fi
}
\makeatother

\IfFileExists{\jobname-pw.ind}{\input{\jobname-pw.ind}}{}

% Quellenangabe nur in der Leseansicht
\ifkorrekturansicht\else
% Fallback-Definitionen, falls die .tex-Datei \titel etc. nicht gesetzt hat
\providecommand{\titel}{}
\providecommand{\editorInnen}{}
\providecommand{\dateiname}{\jobname}

\vspace{3cm}

\vfill

\footnotesize
\textsc{Quelle}: \titel. Herausgegeben von {\editorInnen}. In: \emph{Arthur Schnitzler: Briefwechsel mit Autorinnen und Autoren}.
 Digitale Edition, https://schnitzler-briefe.acdh.oeaw.ac.at/{\dateiname}.html (Stand \today)
\fi

\end{document}


