%% latex-korrekturansicht-vorspann.tex
%% Vorspann für die Korrekturansicht.
%% Lädt die gemeinsame Datei latex-vorspann.tex mit gesetztem Schalter.

\newif\ifkorrekturansicht
\korrekturansichttrue

\input{../tex-inputs/latex-vorspann}


\section[Paul Lasker-Schüler an Arthur Schnitzler, 28. 4. {[}1925?{]}]{L02654 Paul Lasker-Schüler an Arthur Schnitzler, 28. 4. {[}1925?{]}}
\nopagebreak\mylabel{L02654v}
\rehead{ }\normalsize\beginnumbering\briefempfaengerindex{Schnitzler, Arthur@\textsc{Schnitzler, Arthur}!zzzLasker-Schueler, Paul@\emph{von Paul Lasker-Schüler}!1925-04-281@{28. 4. {[}1925?{]}}|(be}
\toendnotes[C]{\smallbreak\pagebreak[2]}\Standort{DLA, A:Schnitzler, HS.1985.1.3876.}
\physDesc{Brief, maschinenschriftliche Abschrift1 Blatt, 1 Seite, 306 Zeichen
\newline{}Schreibmaschine}\toendnotes[C]{\smallbreak}
\pstart
           {\pb}\label{K_L02654-1v}\edtext{28. IV.}{\lemma{\textnormal{\emph{28. IV.}}}\Cendnote{\textnormal{Es ist kein Besuch Paul Lasker-Schülers\pwindex{Lasker-Schueler, Paul 1899-08-24 – 1927-12-14@\textsc{Lasker-Schüler, Paul} (1899-08-24 – 1927-12-14)|pwk} bei Schnitzler bekannt, durch den das Datum des Briefes
                     gesichert bestimmt werden könnte. Da der Brief nur in Abschrift vorliegt, lässt
                     sich nicht mit Gewissheit ausschließen, dass die Schreibkraft, die die Abschrift übernahm,
                     bei der Entzifferung der Monatsangabe keinen Fehler gemacht hat. Mit Hilfe der
                     freundlichen Auskunft von Karl Jürgen Skrodzki lässt sich folgende
                     Argumentation führen, warum der Brief 1925 entstanden sein muss.
                     Unter der Annahme, dass die Monatsangabe stimmt, kommen nur die Jahre
                        1924 und 1925 in Betracht, da sich in dieser Zeit Paul Lasker-Schüler\pwindex{Lasker-Schueler, Paul 1899-08-24 – 1927-12-14@\textsc{Lasker-Schüler, Paul} (1899-08-24 – 1927-12-14)|pwk} im April in Wien\oindex{Wien@\textbf{Wien}, \emph{A.ADM2}|pwk} aufhielt. Der Brief wurde mit großer
                     Wahrscheinlichkeit nicht vor jenem Else
                        Lasker-Schülers\pwindex{Lasker-Schueler, Else 11.02.1869 – 22.01.1945@\textsc{Lasker-Schüler, Else} (11.02.1869 – 22.01.1945), \emph{Dichter/Dichterin}|pwk} an Schnitzler
                        (10. 12. 1924)
                     verfasst. Paul Lasker-Schüler hätte ohne diese Vorarbeit seiner Mutter\pwindex{Lasker-Schueler, Else 11.02.1869 – 22.01.1945@\textsc{Lasker-Schüler, Else} (11.02.1869 – 22.01.1945), \emph{Dichter/Dichterin}|pwkv} vermutlich nicht
                     an Schnitzler geschrieben.}}}\label{K_L02654-1}\pend
           \vspace{0.5em}
\pstart
           \centering{}Paul Lasker-Schüler\pend
           {\vspace{1\baselineskip}}
\pstart
           bittet vielmals darum, empfangen zu werden, wenn es irgend möglich vielleicht noch
                  heute, denn es handelt sich um einen \label{K_L02654-2v}\edtext{medizinischen Ratschlag}{\lemma{\textnormal{\emph{medizinischen Ratschlag}}}\Cendnote{\textnormal{Im Dezember 1925 erkrankte Paul
                  Lasker-Schüler an Tuberkulose.}}}\label{K_L02654-2}.\pend
           
\pstart
           Ich bitte Sie vielmals Herr Doktor, mir meine Aufdringlichkeit nicht übel zu nehmen.
               Meine Adresse ist Pension Bleckmann\oindex{Pension Bleckmann@\textbf{Pension Bleckmann}, \emph{Hotel (K.HTL)}|pw}{ }{\\}Thelephon 26 206.\pend
           \selectlanguage{ngerman}\endnumbering\briefempfaengerindex{Schnitzler, Arthur@\textsc{Schnitzler, Arthur}!zzzLasker-Schueler, Paul@\emph{von Paul Lasker-Schüler}!1925-04-281@{28. 4. {[}1925?{]}}|)be}\mylabel{L02654h}  \normalsize

\doendnotes{C}
\bigskip
\vfill

\clearpage

\footnotesize

\lohead{\textsc{register}}

% Definiere theindex-Environment komplett neu ohne reledmac
\makeatletter
\renewenvironment{theindex}{%
  \section*{\indexname}%
  \setlength{\parindent}{0pt}%
  \setlength{\parskip}{0pt plus 0.3pt}%
  \let\item\@idxitem
}{%
  \clearpage
}
\makeatother

\IfFileExists{\jobname-pw.ind}{\input{\jobname-pw.ind}}{}

\end{document}

      