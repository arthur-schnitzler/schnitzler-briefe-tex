%% latex-leseansicht-vorspann.tex
%% Vorspann für die Leseansicht.
%% Lädt die gemeinsame Datei latex-vorspann.tex mit nicht gesetztem Schalter.

\newif\ifkorrekturansicht
\korrekturansichtfalse

\input{../tex-inputs/latex-vorspann}


\section[ Paul Goldmann an Arthur Schnitzler, 4. 9. 1897]{L02822 Paul Goldmann an Arthur Schnitzler,  4. 9. 1897}
\nopagebreak\mylabel{L02822v}
\rehead{ }\normalsize\beginnumbering\briefempfaengerindex{Schnitzler, Arthur@\textsc{Schnitzler, Arthur}!zzzGoldmann, Paul@\emph{von Paul Goldmann}!1897-09-041@{4. 9. 1897}|(be}
\toendnotes[C]{\smallbreak\pagebreak[2]}
\correspDesc{Versand  durch Paul Goldmann am 4. 9. 1897 in München
\newline{}Erhalt  durch Arthur Schnitzler am 5. 9. 1897 in Wien}\toendnotes[C]{\smallbreak}
\Standort{DLA, A:Schnitzler, HS.NZ85.1.3167.}
\physDesc{Postkarte, 611 Zeichen
\newline{}Handschrift: schwarze Tinte, deutsche Kurrent
\newline{}Versand: 1) Stempel: »\nobreak{}\oindex{München@\textbf{München}|pwk}Muenchen 1, 4{[}.{]}{ }\textcolor{gray}{9}{[}. 1897{]}, 6–{[}7{]}\nobreak{}«.   2) Stempel: »\nobreak{}\oindex{IX., Alsergrund@\textbf{IX., Alsergrund}, \emph{Verwaltungsgebiet}|pwk}Wien 9/3 72, 5. 9. 1897, 11. V, Bestellt\nobreak{}«. 
\newline{}Schnitzler: mit Bleistift das Jahr »97« vermerkt }\toendnotes[C]{\smallbreak}\pstart{}\textsc{{\pb}Herrn}\pend{}\pstart{}\textsc{Dr. Arthur Schnitzler}\pend{}\pstart{}\textsc{Wien\oindex{Wien@\textbf{Wien}, \emph{Verwaltungsgebiet}|pw}}\pend{}\pstart{}\textsc{IX. Frankgaſse 1\oindex{Wien@\textbf{Wien}!IX., Alsergrund@\textbf{IX., Alsergrund}!Frankgasse 1@\textbf{Frankgasse 1}, \emph{Wohngebäude}|pw}.}\pend{}{\bigskip}\vspace{1em}
\pstart
           {\pb}\textsc{Muenchen\oindex{München@\textbf{München}|pw}}, 4. September.\pend
           \vspace{0.5em}
\pstart
           Mein lieber \strikeout{\textcolor{gray}{F}\textcolor{gray}{×}\-\textcolor{gray}{×}} Freund, Ich fand hier im \textsc{Hotel\oindex{Hotel Marienbad [München]@\textbf{Hotel Marienbad [München]}, \emph{Hotel}|pwv}} eine Karte von der \label{K_L02822-1v}\edtext{Frau\pwindex{Freudenthal, Rosa 1862 – 18.\,6.\,1905 Berlin@\textsc{Freudenthal, Rosa} (1862 – 18.\,6.\,1905 Berlin)|pwv} des Rechtsgelehrten\pwindex{Freudenthal, Hermann 1852/1853 – 12.\,9.\,1925 Berlin@\textsc{Freudenthal, Hermann} (1852/1853 – 12.\,9.\,1925 Berlin), \emph{Rechtsanwalt}|pwv}}{\lemma{\textnormal{\emph{Frau des Rechtsgelehrten}}}\Cendnote{\textnormal{Rosa Freudenthal\pwindex{Freudenthal, Rosa 1862 – 18.\,6.\,1905 Berlin@\textsc{Freudenthal, Rosa} (1862 – 18.\,6.\,1905 Berlin)|pwk} war die Ehefrau des Anwalts Hermann Freudenthal\pwindex{Freudenthal, Hermann 1852/1853 – 12.\,9.\,1925 Berlin@\textsc{Freudenthal, Hermann} (1852/1853 – 12.\,9.\,1925 Berlin), \emph{Rechtsanwalt}|pwk}. Schnitzler hatte seit dem 2. 7. 1897 ein
                  Verhältnis mit ihr.}}}\label{K_L02822-1}. Bitte, danke ihr in meinem Namen,{ }ſage ihr, daß es{ }ſehr
               lieb war, an mich gedacht zu haben, und daß die Karte{ }ſehr herzig geſchrieben war.
               Euch Allen geht es in Wien\oindex{Wien@\textbf{Wien}, \emph{Verwaltungsgebiet}|pw} hoffentlich gut. Mir
               aber iſt das Herz \strikeout{\textcolor{gray}{w}u}{ }\strikeout{\textcolor{gray}{w}u} wund vom Abſchiednehmen. Und ich bin wieder einſam in
               der großen kalten Welt. Und es regnet draußen. Viele treue Grüße Dir, der \label{K_L02822-2v}\edtext{Familie \textsc{Altmann}\pwindex{Altmann, Emma 22.\,10.\,1849 Budapest – 31.\,12.\,1930 Wien@\textsc{Altmann, Emma} (22.\,10.\,1849 Budapest – 31.\,12.\,1930 Wien)|pwv}}{\lemma{\textnormal{\emph{Familie Altmann}}}\Cendnote{\textnormal{Schnitzler verbrachte Ende August und Anfang September 1897 Zeit mit Emma
                     Altmann\pwindex{Altmann, Emma 22.\,10.\,1849 Budapest – 31.\,12.\,1930 Wien@\textsc{Altmann, Emma} (22.\,10.\,1849 Budapest – 31.\,12.\,1930 Wien)|pwk}, der Mutter seiner Schwägerin Helene\pwindex{Schnitzler, Helene 16.\,7.\,1871 Budapest – September 1941 Atlantischer Ozean@\textsc{Schnitzler, Helene} (16.\,7.\,1871 Budapest – September 1941 Atlantischer Ozean)|pwk}, Ehefrau von Julius
                     Schnitzler\pwindex{Schnitzler, Julius 13.\,7.\,1865 Wien – 29.\,6.\,1939 ebd.@\textsc{Schnitzler, Julius} (13.\,7.\,1865 Wien – 29.\,6.\,1939 ebd.), \emph{Chirurg}|pwk}.}}}\label{K_L02822-2}, der Frau\pwindex{Freudenthal, Rosa 1862 – 18.\,6.\,1905 Berlin@\textsc{Freudenthal, Rosa} (1862 – 18.\,6.\,1905 Berlin)|pwv} des Rechtsgelehrten\pwindex{Freudenthal, Hermann 1852/1853 – 12.\,9.\,1925 Berlin@\textsc{Freudenthal, Hermann} (1852/1853 – 12.\,9.\,1925 Berlin), \emph{Rechtsanwalt}|pwv}{ }\textsc{etc.}\pend
           \pstart Dein \spacefill\mbox{Paul Goldm}\pend{}
\pstart
           \noindent{}\label{T_L02822-1v}\edtext{In Frankfurt\oindex{Frankfurt am Main@\textbf{Frankfurt am Main}, \emph{Hauptstadt}|pw} bin ich Dienstag oder Mittwoch, Adreſſe: \textsc{Rossertstraſse 15\oindex{Rossertstraße@\textbf{Rossertstraße}, \emph{Straße}|pw}}}{\lemma{\textnormal{\emph{In … Rossertstrasse 15}}}\Cendnote{\textnormal{entlang der oberen Kante, verkehrt zum
                     Text}}}\label{T_L02822-1}\pend
           \selectlanguage{ngerman}\endnumbering\briefempfaengerindex{Schnitzler, Arthur@\textsc{Schnitzler, Arthur}!zzzGoldmann, Paul@\emph{von Paul Goldmann}!1897-09-041@{4. 9. 1897}|)be}\mylabel{L02822h}  \newcommand{\dateiname}{L02822}\newcommand{\titel}{Paul Goldmann an Arthur Schnitzler, 4. 9. 1897}\newcommand{\editorInnen}{Martin Anton Müller und Laura Untner}%% latex-leseansicht-abspann.tex
%% Abspann für die Leseansicht.
%% Der Schalter \ifkorrekturansicht ist bereits durch den Vorspann gesetzt.

%% latex-abspann.tex
%% Gemeinsamer Abspann für Korrekturansicht und Leseansicht.
%% Setzt den Schalter \ifkorrekturansicht voraus (gesetzt in den
%% einbindenden Dateien latex-korrekturansicht-abspann.tex bzw.
%% latex-leseansicht-abspann.tex).
%% ---------------------------------------------------------------

\normalsize

% Das esempio-Environment wird nur in der Leseansicht benötigt
\ifkorrekturansicht\else
\newenvironment{esempio}[3]%
{
    \vspace{1.5ex}
    \rlap{\underline{#1}}
    \par
    \setlength{\parindent}{0cm}
    \nopagebreak
    \leftskip=#2cm
    \rightskip=#3cm
}
{
    \par
}
\fi

\doendnotes{C}
\bigskip
\vfill

\clearpage

\footnotesize

\ifkorrekturansicht
  \lohead{\textsc{register}}
\fi

% theindex-Environment neu definieren ohne reledmac
\makeatletter
\renewenvironment{theindex}{%
  \ifkorrekturansicht
    \section*{\indexname}%
  \else
    \subsubsection*{Index der erwähnten Entitäten}%
  \fi
  \setlength{\parindent}{0pt}%
  \setlength{\parskip}{0pt plus 0.3pt}%
  \let\item\@idxitem
}{%
  \ifkorrekturansicht\clearpage\fi
}
\makeatother

\IfFileExists{\jobname-pw.ind}{\input{\jobname-pw.ind}}{}

% Quellenangabe nur in der Leseansicht
\ifkorrekturansicht\else
% Fallback-Definitionen, falls die .tex-Datei \titel etc. nicht gesetzt hat
\providecommand{\titel}{}
\providecommand{\editorInnen}{}
\providecommand{\dateiname}{\jobname}

\vspace{3cm}

\vfill

\footnotesize
\textsc{Quelle}: \titel. Herausgegeben von {\editorInnen}. In: \emph{Arthur Schnitzler: Briefwechsel mit Autorinnen und Autoren}.
 Digitale Edition, https://schnitzler-briefe.acdh.oeaw.ac.at/{\dateiname}.html (Stand \today)
\fi

\end{document}


