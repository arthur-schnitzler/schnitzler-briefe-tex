%% latex-leseansicht-vorspann.tex
%% Vorspann für die Leseansicht.
%% Lädt die gemeinsame Datei latex-vorspann.tex mit nicht gesetztem Schalter.

\newif\ifkorrekturansicht
\korrekturansichtfalse

\input{../tex-inputs/latex-vorspann}


         
         \newcommand{\erwaehntePersonen}{Personen: Emma Altmann, Rosa Freudenthal, Hermann Freudenthal, Helene Schnitzler, Julius Schnitzler}
         \newcommand{\erwaehnteOrte}{Orte: Frankfurt am Main, Frankgasse, Hotel Marienbad, München, Rossertstraße, Wien}
         \newcommand{\erwaehnteWerke}{
               \section[ Paul Goldmann an Arthur Schnitzler, 4. 9. 1897]{ Paul Goldmann an Arthur Schnitzler, 4. 9. 1897}\nopagebreak\mylabel{v}\rehead{ }\begin{ledgroupsized}[t]{13cm}\normalsize\beginnumbering \toendnotes[C]{\smallbreak\pagebreak[2]} \Standort{DLA, A:Schnitzler, HS.NZ85.1.3167.}
\physDesc{Postkarte
\newline{}Handschrift: 1) schwarze Tinte, deutsche Kurrent\hspace{1em}2) schwarze Tinte, lateinische Kurrent (\noindent{}Adresse)\hspace{1em}\newline{}Versand: 1) Stempel: »\nobreak{}\oindex{Muenchen@\textbf{München}|pwk}Muenchen 1, 4{[}.{]}{ }\textcolor{gray}{9}{[}. 1897{]}, 6–{[}7{]}\nobreak{}«.   2) Stempel: »\nobreak{}Wien 9/3 72, 5. 9. 1897, 11. V, Bestellt\nobreak{}«. 
\newline{}Schnitzler: mit Bleistift das Jahr »97« vermerkt }\toendnotes[C]{\smallbreak}\pstart{}{\pb}\textcolor{gray}{\textbf{An}}\pend{}\pstart{}Herrn\pend{}\pstart{}Dr. Arthur Schnitzler\pend{}\pstart{}\textcolor{gray}{\textbf{in}}{ }Wien\oindex{Wien@\textbf{Wien}|pw}\pend{}\pstart{}IX. Frankgaſse 1\oindex{Frankgasse@\textbf{Frankgasse}|pw}. \pend{}{\bigskip}\pstart
           {\pb}\textsc{Muenchen\oindex{Muenchen@\textbf{München}|pw}}, 4. September.\pend
           \pstart
           Mein lieber \strikeout{\textcolor{gray}{F}\textcolor{gray}{×}\-\textcolor{gray}{×}}
                  Freund, Ich fand hier im \textsc{Hotel\oindex{Hotel Marienbad@\textbf{Hotel Marienbad}|pwv}} eine Karte von der \label{K_L02822-11v}\edtext{Frau\pwindex{Freudenthal, Rosa 1862 – 18.06.1905@\textsc{Freudenthal, Rosa} (1862 – 18.06.1905)|pwv} des Rechtsgelehrten\pwindex{Freudenthal, Hermann 1852/1853 – 12.09.1925@\textsc{Freudenthal, Hermann} (1852/1853 – 12.09.1925), \emph{Rechtsanwalt}|pwv}}{\lemma{\textnormal{\emph{Frau des Rechtsgelehrten}}}\Cendnote{\textnormal{Rosa Freudenthal\pwindex{Freudenthal, Rosa 1862 – 18.06.1905@\textsc{Freudenthal, Rosa} (1862 – 18.06.1905)|pwk}, Ehefrau des Anwalts Hermann Freudenthal\pwindex{Freudenthal, Hermann 1852/1853 – 12.09.1925@\textsc{Freudenthal, Hermann} (1852/1853 – 12.09.1925), \emph{Rechtsanwalt}|pwk}, mit der Schnitzler\pwindex{Schnitzler, Arthur 15.05.1862 – 21.10.1931@\textsc{Schnitzler, Arthur} (15.05.1862 – 21.10.1931), \emph{Schriftsteller, Mediziner}|pwk} seit dem 2. 7. 1897 ein
                  Verhältnis hatte}}}\label{K_L02822-11h}. Bitte, danke ihr in meinem Namen, ſage ihr, daß es ſehr
               lieb war, an mich gedacht zu haben, und daß die Karte ſehr herzig geſchrieben war.
               Euch Allen geht es in Wien\oindex{Wien@\textbf{Wien}|pw} hoffentlich gut. Mir
               aber iſt das Herz \strikeout{\textcolor{gray}{w}u}{ }\strikeout{\textcolor{gray}{w}u} wund vom Abſchiednehmen. Und ich bin wieder einſam in
               der großen kalten Welt. Und es regnet draußen. Viele treue Grüße Dir, der \label{K_L02822-3v}\edtext{Familie \textsc{Altmann}\pwindex{Altmann, Emma 22.10.1849 – 31.12.1930@\textsc{Altmann, Emma} (22.10.1849 – 31.12.1930)|pwv}}{\lemma{\textnormal{\emph{Familie Altmann}}}\Cendnote{\textnormal{Schnitzler\pwindex{Schnitzler, Arthur 15.05.1862 – 21.10.1931@\textsc{Schnitzler, Arthur} (15.05.1862 – 21.10.1931), \emph{Schriftsteller, Mediziner}|pwk} verbrachte Ende August und Anfang
                     September 1897 Zeit mit Emma
                     Altmann\pwindex{Altmann, Emma 22.10.1849 – 31.12.1930@\textsc{Altmann, Emma} (22.10.1849 – 31.12.1930)|pwk}, der Mutter seiner Schwägerin Helene\pwindex{Schnitzler, Helene 16.07.1871 – September 1941@\textsc{Schnitzler, Helene} (16.07.1871 – September 1941)|pwk}, Ehefrau von Julius
                     Schnitzler\pwindex{Schnitzler, Julius 13.07.1865 – 29.06.1939@\textsc{Schnitzler, Julius} (13.07.1865 – 29.06.1939), \emph{Chirurg}|pwk}.}}}\label{K_L02822-3h}, der Frau\pwindex{Freudenthal, Rosa 1862 – 18.06.1905@\textsc{Freudenthal, Rosa} (1862 – 18.06.1905)|pwv} des Rechtsgelehrten\pwindex{Freudenthal, Hermann 1852/1853 – 12.09.1925@\textsc{Freudenthal, Hermann} (1852/1853 – 12.09.1925), \emph{Rechtsanwalt}|pwv}{ }\textsc{etc.}\pend
           \pstart Dein \spacefill\mbox{Paul Goldm}\pend{}\pstart
           \noindent{}\label{T_L02822-1v}\edtext{In Frankfurt\oindex{Frankfurt am Main@\textbf{Frankfurt am Main}|pw} bin ich Dienstag oder Mittwoch, Adreſſe: \textsc{Rossertstraſse 15\oindex{Rossertstrasse@\textbf{Rossertstraße}|pw}}}{\lemma{\textnormal{\emph{In … Rossertstraſse 15}}}\Cendnote{\textnormal{entlang der oberen Kante, verkehrt zum
                     Text}}}\label{T_L02822-1h}\pend
           
         
         \endnumbering\mylabel{h}\end{ledgroupsized}  \newcommand{\dateiname}{L02822}\newcommand{\titel}{Paul Goldmann an Arthur Schnitzler, 4. 9. 1897}\newcommand{\editorInnen}{Martin Anton Müller und Laura Untner}%% latex-leseansicht-abspann.tex
%% Abspann für die Leseansicht.
%% Der Schalter \ifkorrekturansicht ist bereits durch den Vorspann gesetzt.

%% latex-abspann.tex
%% Gemeinsamer Abspann für Korrekturansicht und Leseansicht.
%% Setzt den Schalter \ifkorrekturansicht voraus (gesetzt in den
%% einbindenden Dateien latex-korrekturansicht-abspann.tex bzw.
%% latex-leseansicht-abspann.tex).
%% ---------------------------------------------------------------

\normalsize

% Das esempio-Environment wird nur in der Leseansicht benötigt
\ifkorrekturansicht\else
\newenvironment{esempio}[3]%
{
    \vspace{1.5ex}
    \rlap{\underline{#1}}
    \par
    \setlength{\parindent}{0cm}
    \nopagebreak
    \leftskip=#2cm
    \rightskip=#3cm
}
{
    \par
}
\fi

\doendnotes{C}
\bigskip
\vfill

\clearpage

\footnotesize

\ifkorrekturansicht
  \lohead{\textsc{register}}
\fi

% theindex-Environment neu definieren ohne reledmac
\makeatletter
\renewenvironment{theindex}{%
  \ifkorrekturansicht
    \section*{\indexname}%
  \else
    \subsubsection*{Index der erwähnten Entitäten}%
  \fi
  \setlength{\parindent}{0pt}%
  \setlength{\parskip}{0pt plus 0.3pt}%
  \let\item\@idxitem
}{%
  \ifkorrekturansicht\clearpage\fi
}
\makeatother

\IfFileExists{\jobname-pw.ind}{\input{\jobname-pw.ind}}{}

% Quellenangabe nur in der Leseansicht
\ifkorrekturansicht\else
% Fallback-Definitionen, falls die .tex-Datei \titel etc. nicht gesetzt hat
\providecommand{\titel}{}
\providecommand{\editorInnen}{}
\providecommand{\dateiname}{\jobname}

\vspace{3cm}

\vfill

\footnotesize
\textsc{Quelle}: \titel. Herausgegeben von {\editorInnen}. In: \emph{Arthur Schnitzler: Briefwechsel mit Autorinnen und Autoren}.
 Digitale Edition, https://schnitzler-briefe.acdh.oeaw.ac.at/{\dateiname}.html (Stand \today)
\fi

\end{document}


      