%% latex-korrekturansicht-vorspann.tex
%% Vorspann für die Korrekturansicht.
%% Lädt die gemeinsame Datei latex-vorspann.tex mit gesetztem Schalter.

\newif\ifkorrekturansicht
\korrekturansichttrue

\input{../tex-inputs/latex-vorspann}


\section[ Paul Goldmann an Arthur Schnitzler, 4. 9. 1897]{L02822 Paul Goldmann an Arthur Schnitzler, 4. 9. 1897}
\nopagebreak\mylabel{L02822v}
\rehead{ }\normalsize\beginnumbering\briefempfaengerindex{Schnitzler, Arthur@\textsc{Schnitzler, Arthur}!zzzGoldmann, Paul@\emph{von Paul Goldmann}!1897-09-041@{4. 9. 1897}|(be}
\toendnotes[C]{\smallbreak\pagebreak[2]}\Standort{DLA, A:Schnitzler, HS.NZ85.1.3167.}
\physDesc{Postkarte, 611 Zeichen
\newline{}Handschrift: 1) schwarze Tinte, deutsche Kurrent\hspace{1em}2) schwarze Tinte, lateinische Kurrent (\noindent{}Adresse)\hspace{1em}
\newline{}Versand: 1) Stempel: »\nobreak{}\oindex{Muenchen@\textbf{München}, \emph{P.PPLA}|pwk}Muenchen 1, 4{[}.{]}{ }\textcolor{gray}{9}{[}. 1897{]}, 6–{[}7{]}\nobreak{}«.   2) Stempel: »\nobreak{}\oindex{IX., Alsergrund@\textbf{IX., Alsergrund}, \emph{A.ADM3}|pwk}Wien 9/3 72, 5. 9. 1897, 11. V, Bestellt\nobreak{}«. 
\newline{}Schnitzler: mit Bleistift das Jahr »97« vermerkt }\toendnotes[C]{\smallbreak}\pstart{}{\pb}Herrn\pend{}\pstart{}Dr. Arthur Schnitzler\pend{}\pstart{}Wien\oindex{Wien@\textbf{Wien}, \emph{A.ADM2}|pw}\pend{}\pstart{}IX. Frankgaſse 1\oindex{Frankgasse 1@\textbf{Frankgasse 1}, \emph{Wohngebäude (K.WHS)}|pw}. \pend{}{\bigskip}\vspace{1em}
\pstart
           {\pb}\textsc{Muenchen\oindex{Muenchen@\textbf{München}, \emph{P.PPLA}|pw}}, 4. September.\pend
           \vspace{0.5em}
\pstart
           Mein lieber \strikeout{\textcolor{gray}{F}\textcolor{gray}{×}\-\textcolor{gray}{×}} Freund, Ich fand hier im \textsc{Hotel\oindex{Hotel Marienbad [Muenchen]@\textbf{Hotel Marienbad [München]}, \emph{Hotel (K.HTL)}|pwv}} eine Karte von der \label{K_L02822-1v}\edtext{Frau\pwindex{Freudenthal, Rosa 1862 – 18.06.1905@\textsc{Freudenthal, Rosa} (1862 – 18.06.1905)|pwv} des Rechtsgelehrten\pwindex{Freudenthal, Hermann 1852/1853 – 12.09.1925@\textsc{Freudenthal, Hermann} (1852/1853 – 12.09.1925), \emph{Rechtsanwalt/Rechtsanwältin}|pwv}}{\lemma{\textnormal{\emph{Frau des Rechtsgelehrten}}}\Cendnote{\textnormal{Rosa Freudenthal\pwindex{Freudenthal, Rosa 1862 – 18.06.1905@\textsc{Freudenthal, Rosa} (1862 – 18.06.1905)|pwk} war die Ehefrau des Anwalts Hermann Freudenthal\pwindex{Freudenthal, Hermann 1852/1853 – 12.09.1925@\textsc{Freudenthal, Hermann} (1852/1853 – 12.09.1925), \emph{Rechtsanwalt/Rechtsanwältin}|pwk}. Schnitzler hatte seit dem 2. 7. 1897 ein
                  Verhältnis mit ihr.}}}\label{K_L02822-1}. Bitte, danke ihr in meinem Namen, ſage ihr, daß es ſehr
               lieb war, an mich gedacht zu haben, und daß die Karte ſehr herzig geſchrieben war.
               Euch Allen geht es in Wien\oindex{Wien@\textbf{Wien}, \emph{A.ADM2}|pw} hoffentlich gut. Mir
               aber iſt das Herz \strikeout{\textcolor{gray}{w}u}{ }\strikeout{\textcolor{gray}{w}u} wund vom Abſchiednehmen. Und ich bin wieder einſam in
               der großen kalten Welt. Und es regnet draußen. Viele treue Grüße Dir, der \label{K_L02822-2v}\edtext{Familie \textsc{Altmann}\pwindex{Altmann, Emma 22.10.1849 – 31.12.1930@\textsc{Altmann, Emma} (22.10.1849 – 31.12.1930)|pwv}}{\lemma{\textnormal{\emph{Familie Altmann}}}\Cendnote{\textnormal{Schnitzler verbrachte Ende August und Anfang
                     September 1897 Zeit mit Emma
                     Altmann\pwindex{Altmann, Emma 22.10.1849 – 31.12.1930@\textsc{Altmann, Emma} (22.10.1849 – 31.12.1930)|pwk}, der Mutter seiner Schwägerin Helene\pwindex{Schnitzler, Helene 16.07.1871 – September 1941@\textsc{Schnitzler, Helene} (16.07.1871 – September 1941)|pwk}, Ehefrau von Julius
                     Schnitzler\pwindex{Schnitzler, Julius 13.07.1865 – 29.06.1939@\textsc{Schnitzler, Julius} (13.07.1865 – 29.06.1939), \emph{Chirurg/Chirurgin}|pwk}.}}}\label{K_L02822-2}, der Frau\pwindex{Freudenthal, Rosa 1862 – 18.06.1905@\textsc{Freudenthal, Rosa} (1862 – 18.06.1905)|pwv} des Rechtsgelehrten\pwindex{Freudenthal, Hermann 1852/1853 – 12.09.1925@\textsc{Freudenthal, Hermann} (1852/1853 – 12.09.1925), \emph{Rechtsanwalt/Rechtsanwältin}|pwv}{ }\textsc{etc.}\pend
           \pstart Dein \spacefill\mbox{Paul Goldm}\pend{}
\pstart
           \noindent{}\label{T_L02822-1v}\edtext{In Frankfurt\oindex{Frankfurt am Main@\textbf{Frankfurt am Main}, \emph{P.PPLA3}|pw} bin ich Dienstag oder Mittwoch, Adreſſe: \textsc{Rossertstraſse 15\oindex{Rossertstrasse@\textbf{Rossertstraße}, \emph{Straße (K.STR)}|pw}}}{\lemma{\textnormal{\emph{In … Rossertstraſse 15}}}\Cendnote{\textnormal{entlang der oberen Kante, verkehrt zum
                     Text}}}\label{T_L02822-1}\pend
           \selectlanguage{ngerman}\endnumbering\briefempfaengerindex{Schnitzler, Arthur@\textsc{Schnitzler, Arthur}!zzzGoldmann, Paul@\emph{von Paul Goldmann}!1897-09-041@{4. 9. 1897}|)be}\mylabel{L02822h}  \normalsize

\doendnotes{C}
\bigskip
\vfill

\clearpage

\footnotesize

\lohead{\textsc{register}}

% Definiere theindex-Environment komplett neu ohne reledmac
\makeatletter
\renewenvironment{theindex}{%
  \section*{\indexname}%
  \setlength{\parindent}{0pt}%
  \setlength{\parskip}{0pt plus 0.3pt}%
  \let\item\@idxitem
}{%
  \clearpage
}
\makeatother

\IfFileExists{\jobname-pw.ind}{\input{\jobname-pw.ind}}{}

\end{document}

      