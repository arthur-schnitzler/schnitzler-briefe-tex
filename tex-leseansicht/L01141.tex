%% latex-leseansicht-vorspann.tex
%% Vorspann für die Leseansicht.
%% Lädt die gemeinsame Datei latex-vorspann.tex mit nicht gesetztem Schalter.

\newif\ifkorrekturansicht
\korrekturansichtfalse

\input{../tex-inputs/latex-vorspann}


               \section[Arthur Schnitzler an Edith Brandes, 4. 7. 1901]{ Arthur Schnitzler an Edith Brandes, 4. 7. 1901}\nopagebreak\mylabel{v}\rehead{ }\begin{ledgroupsized}[t]{13cm}\normalsize\beginnumbering\briefempfaengerindex{Philipp, Edith@\textsc{Philipp, Edith}!zzzSchnitzler, Arthur@\emph{von Arthur Schnitzler}!1901-07-042@{4. 7. 1901}|(be} \toendnotes[C]{\smallbreak\pagebreak[2]} \Standort{CUL, Schnitzler, B 17 (2).}
\physDesc{Brief, maschinelle Abschrift\newline{}Zusatz: Original nicht nachweisbar }\buchAbdrucke{\weitereDrucke{Georg Brandes, Arthur Schnitzler: \emph{Ein Briefwechsel}. Hg. Kurt Bergel. Bern: \emph{Francke} 1956, S. 89–90.} }\toendnotes[C]{\smallbreak}\pstart{}{\pb}Verehrtes Fräulein,\pend\pstart
           Sie sind mir natürlich auch längst nicht unbekannt, ich kenne sogar Ihr Bild, und
                    vor fünf Jahren hätte ich Sie persönlich kennen lernen können, wenn ich lang
                    genug in Kopenhagen\oindex{Kopenhagen@\textbf{Kopenhagen}|pw} geblieben wäre. Es freut
                    mich natürlich sehr, dass Sie etwas von mir für Ihr Album wollen. Aber da wir
                    nun beinahe gute Bekannte sind, frag ich Sie lieber gleich, was Sie denn am
                    liebsten möchten. Ich meine das so. Vielleicht haben Sie Sympathie für irgend
                    eines von meinen Büchern und wünschen, dass ich Ihnen aus einem solchen Buch
                    etwas in Ihr Album schreiben soll? (Da müssten Sie natürlich warten bis ich
                    wieder in Wien\oindex{Wien@\textbf{Wien}|pw} bin, weil ich meine Werke nicht
                    auswendig kenne.) Oder Sie wünschen lieber irgend eine der ungeheuer
                    tiefsinnigen Lebensweisheiten, von denen wir Dichter bekanntlich überfliessen?
                    Oder eine von den graziösen Geistreichigkeiten, die wir zu tausenden vorrätig
                    haben, die man auch beliebig drehen kann und die immer umgekehrt gerade so wahr
                    sind? In Wahrheit hätte ja nur \uline{eine} Art von
                    Albumblättern wirklichen Wert: eins, auf dem geschrieben stünde, was \uline{nur} der eine der es schreibt zu \uline{nur} dem einen sagen könnte, der es verlangt. Wie
                    leid tut es mir, Sie nicht gut genug zu kennen, um Ihnen ein solches anzubieten
                    – und Sie bitten zu müssen, – bis dahin – ein andres zu wählen und entgegen zu
                    nehmen.\pend
           \pstart
           Wenn Sie mir eine Zeile antworten, adressieren Sie sie gütigst nach Wien\oindex{Wien@\textbf{Wien}|pw}; ich bin auf Reisen und vielleicht schon in wenigen
                    Tagen sehr fern von hier.\pend
           \pstart
           Jetzt danke ich Ihnen noch für das viele freundliche \label{T_L01141_1v}\edtext{das}{\lemma{\textnormal{\emph{das}}}\Cendnote{\textnormal{die Abschrift
                        hat: »dass«}}}\label{T_L01141_1h} Sie mir in Ihrem Briefe gesagt haben,
                    bitte Sie, sehr herzlich Ihren Vater\pwindex{Brandes, Georg 04.02.1842 – 19.02.1927@\textsc{Brandes, Georg} (04.02.1842 – 19.02.1927)|pwv} zu grüssen und bin\pend
           \pstart
           Ihr aufrichtig ergebener{\\[\baselineskip]}\spacefill\mbox{Arthur Schnitzler}\pend
           \leftskip=0em{}\pstart
           St. Anton a/Arlberg\oindex{St. Anton am Arlberg@\textbf{St. Anton am Arlberg}|pw}{\\}4. 7. 901.\pend
                     \endnumbering\briefempfaengerindex{Philipp, Edith@\textsc{Philipp, Edith}!zzzSchnitzler, Arthur@\emph{von Arthur Schnitzler}!1901-07-042@{4. 7. 1901}|)be}\mylabel{h}\end{ledgroupsized}  \newcommand{\dateiname}{L01141}\newcommand{\titel}{Arthur Schnitzler an Edith Brandes, 4. 7. 1901}\newcommand{\editorInnen}{Martin Anton Müller und Gerd-Hermann Susen}
            \footnotesize
\begin{ledgroupsized}[t]{11.5cm}
\doendnotes{C}
\end{ledgroupsized}
         %% latex-leseansicht-abspann.tex
%% Abspann für die Leseansicht.
%% Der Schalter \ifkorrekturansicht ist bereits durch den Vorspann gesetzt.

%% latex-abspann.tex
%% Gemeinsamer Abspann für Korrekturansicht und Leseansicht.
%% Setzt den Schalter \ifkorrekturansicht voraus (gesetzt in den
%% einbindenden Dateien latex-korrekturansicht-abspann.tex bzw.
%% latex-leseansicht-abspann.tex).
%% ---------------------------------------------------------------

\normalsize

% Das esempio-Environment wird nur in der Leseansicht benötigt
\ifkorrekturansicht\else
\newenvironment{esempio}[3]%
{
    \vspace{1.5ex}
    \rlap{\underline{#1}}
    \par
    \setlength{\parindent}{0cm}
    \nopagebreak
    \leftskip=#2cm
    \rightskip=#3cm
}
{
    \par
}
\fi

\doendnotes{C}
\bigskip
\vfill

\clearpage

\footnotesize

\ifkorrekturansicht
  \lohead{\textsc{register}}
\fi

% theindex-Environment neu definieren ohne reledmac
\makeatletter
\renewenvironment{theindex}{%
  \ifkorrekturansicht
    \section*{\indexname}%
  \else
    \subsubsection*{Index der erwähnten Entitäten}%
  \fi
  \setlength{\parindent}{0pt}%
  \setlength{\parskip}{0pt plus 0.3pt}%
  \let\item\@idxitem
}{%
  \ifkorrekturansicht\clearpage\fi
}
\makeatother

\IfFileExists{\jobname-pw.ind}{\input{\jobname-pw.ind}}{}

% Quellenangabe nur in der Leseansicht
\ifkorrekturansicht\else
% Fallback-Definitionen, falls die .tex-Datei \titel etc. nicht gesetzt hat
\providecommand{\titel}{}
\providecommand{\editorInnen}{}
\providecommand{\dateiname}{\jobname}

\vspace{3cm}

\vfill

\footnotesize
\textsc{Quelle}: \titel. Herausgegeben von {\editorInnen}. In: \emph{Arthur Schnitzler: Briefwechsel mit Autorinnen und Autoren}.
 Digitale Edition, https://schnitzler-briefe.acdh.oeaw.ac.at/{\dateiname}.html (Stand \today)
\fi

\end{document}


      