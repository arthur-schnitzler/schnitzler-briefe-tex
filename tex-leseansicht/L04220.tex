%% latex-leseansicht-vorspann.tex
%% Vorspann für die Leseansicht.
%% Lädt die gemeinsame Datei latex-vorspann.tex mit nicht gesetztem Schalter.

\newif\ifkorrekturansicht
\korrekturansichtfalse

\input{../tex-inputs/latex-vorspann}


\section[Marie Holzer an Arthur Schnitzler, 23. 4. 1908]{L04220 Marie Holzer an Arthur Schnitzler, 23. 4. 1908}
\nopagebreak\mylabel{L04220v}
\rehead{ }\normalsize\beginnumbering\briefempfaengerindex{Schnitzler, Arthur@\textsc{Schnitzler, Arthur}!zzzHolzer, Marie@\emph{von Marie Holzer}!1908-04-231@{23. 4. 1908}|(be}
\toendnotes[C]{\smallbreak\pagebreak[2]}
\correspDesc{Versand  durch Marie Holzer am 23. 4. 1908 in Prag
\newline{}Erhalt  durch Arthur Schnitzler im Zeitraum [24. 4. 1908
                  – 28. 4. 1908?] in Wien}\toendnotes[C]{\smallbreak}
\Standort{CUL, Schnitzler, B 45.}
\physDesc{Brief, 1 Blatt, 4 Seiten, 2127 Zeichen
\newline{}Handschrift: blaue Tinte, deutsche Kurrent
\newline{}Schnitzler: mit Bleistift Vermerk »Abſch.« und »\textsc{Marie Holzer}« 
\newline{}Ordnung: mit Bleistift von unbekannter Hand: »Marie H« }\toendnotes[C]{\smallbreak}
\pstart{}{\pb}Sehr geehrter Herr
                  Doktor!\pend\vspace{0.5em}
\pstart
           Wir{ }ſaßen in einer kleinen Geſellſchaft und ich erzählte den Inhalt Ihrer Novelle:
                  \label{K_L04220-1v}\edtext{Der
                  Tod des Junggesellen\pwindex{Schnitzler, Arthur 15. 5. 1862 Wien – 21. 10. 1931 ebd.@\textsc{Schnitzler, Arthur} (15. 5. 1862 Wien – 21. 10. 1931 ebd.), \emph{Schriftsteller, Mediziner}!Tod des Junggesellen. Novelle@\strich\emph{Der Tod des Junggesellen. Novelle}|pw}}{\lemma{\textnormal{\emph{Der
                  Tod des Junggesellen}}}\Cendnote{\textnormal{Dieser Text
                  war gerade erschienen: Arthur
                        Schnitzler: \emph{Der Tod des Junggesellen.
                        Novelle}\pwindex{Schnitzler, Arthur 15. 5. 1862 Wien – 21. 10. 1931 ebd.@\textsc{Schnitzler, Arthur} (15. 5. 1862 Wien – 21. 10. 1931 ebd.), \emph{Schriftsteller, Mediziner}!Tod des Junggesellen. Novelle@\strich\emph{Der Tod des Junggesellen. Novelle}|pwk}. In: \emph{Österreichische
                        Rundschau}\pwindex{Österreichische Rundschau@\emph{Österreichische Rundschau}|pwk}, Bd. 15, 1. 4. 1908,
                     S. 19–26.}}}\label{K_L04220-1}, die ich dreimal geleſen und jedesmal von neuem
               miterlebt.\pend
           
\pstart
           \label{K_L04220-2v}\edtext{Das ſchreibt man, aber das tut man
                  nicht}{\lemma{\textnormal{\emph{Das schreibt … nicht}}}\Cendnote{\textnormal{In der Novelle erfahren drei
                  Männer davon, dass ein gerade Verstorbener sie mit ihren jeweiligen Ehefrauen
                  betrogen hat.}}}\label{K_L04220-2}!{ }ſagte ein temperamentoller Ehemann, ein Pflichtenmenſch,
               deſſen Gefühle ſich nur in der Skala des Erlaubten bewegen. »Künstler und
               Menſch ſind ſo grundverſchiedene Elemente, die man haarscharf auseinanderhalten muß,«{ }ſagte eine Malerin.\pend
           
\pstart
           Er\pwindex{Schnitzler, Arthur 15. 5. 1862 Wien – 21. 10. 1931 ebd.@\textsc{Schnitzler, Arthur} (15. 5. 1862 Wien – 21. 10. 1931 ebd.), \emph{Schriftsteller, Mediziner}!Tod des Junggesellen. Novelle@\strich\emph{Der Tod des Junggesellen. Novelle}|pwv} würde gewiß nicht so
               handeln, wenn die Frau {\pb}die er liebte
               – – – – \pend
           
\pstart
           Im Affekt iſt man niemals Herr ſeiner ſelbst!{ }ſo ſagten – alle – alle –\pend
           
\pstart
           Und ich drang mit meiner Verteidigung nicht durch, daß man ſich kraft ſeines Weſens,
               kraft ſeiner Überzeugung zu einem Standpunkt durchgerungen, den man unter allen
               Umſtänden einnimmt.\pend
           
\pstart
           Man \label{K_L04220-3v}\edtext{verwies auf »Das Märchen\pwindex{Schnitzler, Arthur 15. 5. 1862 Wien – 21. 10. 1931 ebd.@\textsc{Schnitzler, Arthur} (15. 5. 1862 Wien – 21. 10. 1931 ebd.), \emph{Schriftsteller, Mediziner}!Märchen. Schauspiel in drei Aufzügen@\strich\emph{Das Märchen. Schauspiel in drei Aufzügen}|pw}}{\lemma{\textnormal{\emph{verwies auf »Das Märchen}}}\Cendnote{\textnormal{In
                     Schnitzlers frühem Theaterstück kommt der
                  Protagonist nicht damit klar, dass seine Geliebte schon vor ihm sexuell aktiv
                  war.}}}\label{K_L04220-3}«.\pend
           
\pstart
           Daß Sie der Welt und dem landläufigen Empfindungsvermögen, das in tief eingewurzelten
               Vorurteilen ſeine Begründung hat im Märchen\pwindex{Schnitzler, Arthur 15. 5. 1862 Wien – 21. 10. 1931 ebd.@\textsc{Schnitzler, Arthur} (15. 5. 1862 Wien – 21. 10. 1931 ebd.), \emph{Schriftsteller, Mediziner}!Märchen. Schauspiel in drei Aufzügen@\strich\emph{Das Märchen. Schauspiel in drei Aufzügen}|pw}
               einen Spiegel vorgehalten und vielleicht auch damals ſo gefühlte, iſt möglich,
               heute ſtehn Sie aber ir{\pb}gendwo
               anders, ſagte ich. Daß durch viele Ihrer Arbeiten, als Leitmotiv derſelbe Gedanke
               durchklingt, und daß gerade dieſer Gedanke Ihre tiefinnerſte Überzeugung iſt, daß wir
               gewiſſe Gefülle nicht unter die Herrschaft des Willens zwingen können, und daß die
               Einſicht allmählich durchdringt, daß wir niemals ein ſchrankenloses Recht auf den
               Nächſten haben. Daß gerade Sie ein reifes Verſtehen für die ſchlagenden Wetter, tief
               drunten im dunkleſten Schacht der menſchlichen Pſyche haben, die Sie ſehen und
               begreifen, trotz des Vorſchutzes von Kultur und Beherrſchung, trotz jahrtauſende
               alter Vorurteile. {\pb}Und daß gerade indem
               einen Punkt, Menſch und Künſtler dieſelbe Sprache ſprechen. Ja, daß Sie der Apoſtel
               dieses wunderlichen Gedankens ſind, zu dem die andern erſt langſam, langſam finden
               müſſen.\pend
           
\pstart
           Wir gingen alle verſti{\geminationm}t auseinander. Keiner hatte den
               anderen überzeugt. Da ich aber das ſichere Gefühl habe, daß ich Sie verstehe – ſo
               möchte ich mir die Frage erlauben – wer von uns hatte recht?\pend
           
\pstart
           Mit freundlichem Gruß{\\[\baselineskip]}\spacefill\mbox{Frau Marie Holzer.}\pend
           \leftskip=0em{}
\pstart
           \textsc{Prag-Weinberge Kroneng. 78\oindex{Korunní 1153/78@\textbf{Korunní 1153/78}, \emph{Wohngebäude}|pw}}, den 23. 4.  1908.\pend
           \selectlanguage{ngerman}\endnumbering\briefempfaengerindex{Schnitzler, Arthur@\textsc{Schnitzler, Arthur}!zzzHolzer, Marie@\emph{von Marie Holzer}!1908-04-231@{23. 4. 1908}|)be}\mylabel{L04220h}
\begin{anhang}
\end{anhang}\newcommand{\dateiname}{L04220}\newcommand{\titel}{Marie Holzer an Arthur Schnitzler, 23. 4. 1908}\newcommand{\editorInnen}{Herausgegeben von Jahnke, SelmaMüller, Martin Anton}%% latex-leseansicht-abspann.tex
%% Abspann für die Leseansicht.
%% Der Schalter \ifkorrekturansicht ist bereits durch den Vorspann gesetzt.

%% latex-abspann.tex
%% Gemeinsamer Abspann für Korrekturansicht und Leseansicht.
%% Setzt den Schalter \ifkorrekturansicht voraus (gesetzt in den
%% einbindenden Dateien latex-korrekturansicht-abspann.tex bzw.
%% latex-leseansicht-abspann.tex).
%% ---------------------------------------------------------------

\normalsize

% Das esempio-Environment wird nur in der Leseansicht benötigt
\ifkorrekturansicht\else
\newenvironment{esempio}[3]%
{
    \vspace{1.5ex}
    \rlap{\underline{#1}}
    \par
    \setlength{\parindent}{0cm}
    \nopagebreak
    \leftskip=#2cm
    \rightskip=#3cm
}
{
    \par
}
\fi

\doendnotes{C}
\bigskip
\vfill

\clearpage

\footnotesize

\ifkorrekturansicht
  \lohead{\textsc{register}}
\fi

% theindex-Environment neu definieren ohne reledmac
\makeatletter
\renewenvironment{theindex}{%
  \ifkorrekturansicht
    \section*{\indexname}%
  \else
    \subsubsection*{Index der erwähnten Entitäten}%
  \fi
  \setlength{\parindent}{0pt}%
  \setlength{\parskip}{0pt plus 0.3pt}%
  \let\item\@idxitem
}{%
  \ifkorrekturansicht\clearpage\fi
}
\makeatother

\IfFileExists{\jobname-pw.ind}{\input{\jobname-pw.ind}}{}

% Quellenangabe nur in der Leseansicht
\ifkorrekturansicht\else
% Fallback-Definitionen, falls die .tex-Datei \titel etc. nicht gesetzt hat
\providecommand{\titel}{}
\providecommand{\editorInnen}{}
\providecommand{\dateiname}{\jobname}

\vspace{3cm}

\vfill

\footnotesize
\textsc{Quelle}: \titel. Herausgegeben von {\editorInnen}. In: \emph{Arthur Schnitzler: Briefwechsel mit Autorinnen und Autoren}.
 Digitale Edition, https://schnitzler-briefe.acdh.oeaw.ac.at/{\dateiname}.html (Stand \today)
\fi

\end{document}


