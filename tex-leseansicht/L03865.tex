%% latex-leseansicht-vorspann.tex
%% Vorspann für die Leseansicht.
%% Lädt die gemeinsame Datei latex-vorspann.tex mit nicht gesetztem Schalter.

\newif\ifkorrekturansicht
\korrekturansichtfalse

\input{../tex-inputs/latex-vorspann}


\section[Theodor Herzl an Arthur Schnitzler, {{[}}9. 11. 1895?{{]}}]{L03865 Theodor Herzl an Arthur Schnitzler, {[}9. 11. 1895?{]}}
\nopagebreak\mylabel{L03865v}
\rehead{ }\normalsize\beginnumbering\briefempfaengerindex{Schnitzler, Arthur@\textsc{Schnitzler, Arthur}!zzzHerzl, Theodor@\emph{von Theodor Herzl}!1895-11-091@{{[}9. 11. 1895?{]}}|(be}
\toendnotes[C]{\smallbreak\pagebreak[2]}
\correspDesc{Versand  durch Theodor Herzl am [9. 11. 1895?] in Wien
\newline{}Erhalt  durch Arthur Schnitzler im Zeitraum [9. 11. 1895 –
                  10. 11. 1895?] in Wien}\toendnotes[C]{\smallbreak}
\Standort{CUL, Schnitzler, B 39.}
\physDesc{Visitenkarte, 225 Zeichen
\newline{}Handschrift: blaue Tinte, lateinische Kurrent
\newline{}Schnitzler: mit Bleistift datiert: »Nov 95« 
\newline{}Ordnung: mit Bleistift von unbekannter Hand nummeriert: »44a« }
\buchAbdrucke{\weitereDrucke{Theodor Herzl: \emph{Briefe Anfang Mai 1895 – Anfang Dezember 1898}. Bearbeitet von Barbara Schäfer in Zusammenarbeit mit Sofia Gelmann, Chaya Harel, Ines Rubin und Daisy Ticho. Berlin, Frankfurt am Main, Wien: \emph{Propyläen} 1990, S. 63 (Briefe und Tagebücher. Herausgegeben von Alex Bein, Hermann Greive, Moshe Schaerf, Julius H. Schoeps und Johannes Wachten, 4).} }\toendnotes[C]{\smallbreak}
\pstart
           \centering{}{\pb}\textcolor{gray}{\textbf{D\textsuperscript{r}\textsuperscript{.} Theodor Herzl}}\pend
           
\pstart
           \raggedleft{}{\pb}Samstag.\pend
           \vspace{0.5em}
\pstart
           Lieber Freund, machen Sie uns \label{K_L03865-1v}\edtext{morgen{ }\introOben{}Sonntag\introOben{}}{\lemma{\textnormal{\emph{morgen Sonntag}}}\Cendnote{\textnormal{Vgl. A. S.: \emph{Tagebuch}, 10. 11. 1895. Aus dem bekannten Datum der Essenseinladung ergibt sich die Datierung der Sendung auf den 9. 11. 1895, den Vortag der Zusammenkunft.}}}\label{K_L03865-1}{ }Mittag das Vergnügen, mit Sudermann\pwindex{Sudermann, Hermann 30.\,9.\,1857 Macikai – 21.\,11.\,1928 Berlin@\textsc{Sudermann, Hermann} (30.\,9.\,1857 Macikai – 21.\,11.\,1928 Berlin), \emph{Schriftsteller}|pw} u. Burckhard\pwindex{Burckhard, Max Eugen 14.\,7.\,1854 Korneuburg – 16.\,3.\,1912 Wien@\textsc{Burckhard, Max Eugen} (14.\,7.\,1854 Korneuburg – 16.\,3.\,1912 Wien), \emph{Schriftsteller, Rechtswissenschaftler, Theaterleiter}|pw} bei uns zu speisen. Anfang 2 Uhr.\pend
           
\pstart
           \label{K_L03865-2v}\edtext{U. A. w. g.}{\lemma{\textnormal{\emph{U. A. w. g.}}}\Cendnote{\textnormal{Um Antwort wird gebeten.}}}\label{K_L03865-2}\pend
           
\pstart
           Herzlichen Gruss von Ihrem{\\[\baselineskip]}\spacefill\mbox{Th. H.}\pend
           \leftskip=0em{}
\pstart
           \noindent{}Beifolgende \label{K_L03865-3v}\edtext{Ruhmeskarte}{\lemma{\textnormal{\emph{Ruhmeskarte}}}\Cendnote{\textnormal{Beilage nicht überliefert.}}}\label{K_L03865-3} erledigen Sie
                  gefälligst selbst.\pend
           \selectlanguage{ngerman}\endnumbering\briefempfaengerindex{Schnitzler, Arthur@\textsc{Schnitzler, Arthur}!zzzHerzl, Theodor@\emph{von Theodor Herzl}!1895-11-091@{{[}9. 11. 1895?{]}}|)be}\mylabel{L03865h}
\begin{anhang}
\end{anhang}\newcommand{\dateiname}{L03865}\newcommand{\titel}{Theodor Herzl an Arthur Schnitzler, [9. 11. 1895?]}\newcommand{\editorInnen}{Selma Jahnke und Martin Anton Müller}%% latex-leseansicht-abspann.tex
%% Abspann für die Leseansicht.
%% Der Schalter \ifkorrekturansicht ist bereits durch den Vorspann gesetzt.

%% latex-abspann.tex
%% Gemeinsamer Abspann für Korrekturansicht und Leseansicht.
%% Setzt den Schalter \ifkorrekturansicht voraus (gesetzt in den
%% einbindenden Dateien latex-korrekturansicht-abspann.tex bzw.
%% latex-leseansicht-abspann.tex).
%% ---------------------------------------------------------------

\normalsize

% Das esempio-Environment wird nur in der Leseansicht benötigt
\ifkorrekturansicht\else
\newenvironment{esempio}[3]%
{
    \vspace{1.5ex}
    \rlap{\underline{#1}}
    \par
    \setlength{\parindent}{0cm}
    \nopagebreak
    \leftskip=#2cm
    \rightskip=#3cm
}
{
    \par
}
\fi

\doendnotes{C}
\bigskip
\vfill

\clearpage

\footnotesize

\ifkorrekturansicht
  \lohead{\textsc{register}}
\fi

% theindex-Environment neu definieren ohne reledmac
\makeatletter
\renewenvironment{theindex}{%
  \ifkorrekturansicht
    \section*{\indexname}%
  \else
    \subsubsection*{Index der erwähnten Entitäten}%
  \fi
  \setlength{\parindent}{0pt}%
  \setlength{\parskip}{0pt plus 0.3pt}%
  \let\item\@idxitem
}{%
  \ifkorrekturansicht\clearpage\fi
}
\makeatother

\IfFileExists{\jobname-pw.ind}{\input{\jobname-pw.ind}}{}

% Quellenangabe nur in der Leseansicht
\ifkorrekturansicht\else
% Fallback-Definitionen, falls die .tex-Datei \titel etc. nicht gesetzt hat
\providecommand{\titel}{}
\providecommand{\editorInnen}{}
\providecommand{\dateiname}{\jobname}

\vspace{3cm}

\vfill

\footnotesize
\textsc{Quelle}: \titel. Herausgegeben von {\editorInnen}. In: \emph{Arthur Schnitzler: Briefwechsel mit Autorinnen und Autoren}.
 Digitale Edition, https://schnitzler-briefe.acdh.oeaw.ac.at/{\dateiname}.html (Stand \today)
\fi

\end{document}


