%% latex-korrekturansicht-vorspann.tex
%% Vorspann für die Korrekturansicht.
%% Lädt die gemeinsame Datei latex-vorspann.tex mit gesetztem Schalter.

\newif\ifkorrekturansicht
\korrekturansichttrue

\input{../tex-inputs/latex-vorspann}


\section[Arthur Schnitzler an Georg Brandes, 7. 7. 1925]{L02444 Arthur Schnitzler an Georg Brandes, 7. 7. 1925}
\nopagebreak\mylabel{L02444v}
\rehead{ }\normalsize\beginnumbering\briefempfaengerindex{Brandes, Georg@\textsc{Brandes, Georg}!zzzSchnitzler, Arthur@\emph{von Arthur Schnitzler}!1925-07-071@{7. 7. 1925}|(be}
\toendnotes[C]{\smallbreak\pagebreak[2]}\Standort{Kopenhagen, Det Kongelige Bibliotek, Georg Brandes Arkiv, box 125.}
\physDesc{Brief, 6 Blätter, 6 Seiten, 4791 Zeichen
\newline{}Handschrift: schwarze Tinte, lateinische Kurrent (\noindent{}Text und Nummerierung der Blätter: »II« bis
                                   »VI«)
\newline{}Ordnung: mit Bleistift von unbekannter Hand nummeriert:
                                    »52.« und die weiteren Blätter datiert: »7/7 25« }
\buchAbdrucke{\weitereDrucke{1) Georg Brandes, Arthur Schnitzler: \emph{Ein Briefwechsel}. Bern: \emph{Francke} 1956, S. 147–149.} \weitereDrucke{2) Arthur Schnitzler: \emph{Briefe 1913–1931}. Frankfurt am Main: \emph{S. Fischer} 1984, S. 411–414.} }\toendnotes[C]{\smallbreak}
\pstart
           \raggedleft{}{\pb}Wien\oindex{Wien@\textbf{Wien}, \emph{A.ADM2}|pw}, 7. Juli 1925\pend
           \vspace{0.5em}
\pstart
           mein lieber und verehrter Freund, Sie haben mich während Ihres
               diesmaligen Aufenthalts in Wien\oindex{Wien@\textbf{Wien}, \emph{A.ADM2}|pw} »nicht heiter«
               gefunden, – und so muß ich fast befürchten, daß Sie nicht ganz bemerkt haben, wie
               glücklich mich Ihre Anwesenheit gemacht hat und wie froh ich war, daß Sie mir Ihre
               Sympathie – eines der Geschenke, für die ich dem Schicksal besonders dankbar bin –
               all die Jahre hindurch, die wir einander schon kennen, ungehindert erhalten haben.
               Darf ich Ihnen heute in diesen Zeilen zum Ausdruck bringen, was von Angesicht zu
               Angesicht auszusprechen, was in meinem Betragen zu verdeutlichen ich, mehr meinem
               ganzen Wesen nach, als aus vorübergehenden Sti{\geminationm}ungen
               heraus, nicht so recht im Stande war und bin? Es ist richtig, (und es bewegt mich
               sehr, daſs Sie es empfunden haben, we{\geminationn} es mir auch ein
               bischen leid thut), daſs ich {\pb}zuweilen ein wenig
               melancholisch bin, oder doch bedrückt. Hauptanlaß wohl mein Ohrenleiden, an dem nicht
               nur die langsam aber sicher zunehmende Schwerhörigkeit, sondern, mehr noch, die
               ununterbrochenen subjectiven Geräusche, ein Klingen, ein Sausen, und ein \strikeout{\textcolor{gray}{nicht}} stetes Vogelzwitschern (das sich bis zu einem mäßigen Papageiengekreisch
               verstärken kann) recht quälend sind. Und, sonderbar genug, es gibt doch Stunden, ja
               Tage, an denen mir diese Geräusche, – so continuirlich sie i{\geminationm}er (seit bald dreißig Jahren!) kaum zu Bewußtsein ko{\geminationm}en. Im ganzen verläuft ja die Sache etwas langsamer,
               als ich zu Beginn der Erkrankung gefürchtet habe – man gewöhnts auch allmälig \label{T_L02444-1v}\edtext{(}{\lemma{\textnormal{\emph{(}}}\Cendnote{\textnormal{öffnende Klammer am Zeilenende gestrichen und in der neuen Zeile erneut
                  ausgeführt}}}\label{T_L02444-1}zu mindesten manchmal) aber es ist doch schli{\geminationm}, daſs mir insbesondere der Theaterbesuch schon
               ziemlich vergällt ist und auch bei musikalischen Genüssen viel, sehr viel entgeht.
               Und schli{\geminationm}, {\pb}daß es
               eine eigentliche »Stille« für mich längst nicht mehr gibt. Glücklicherweise werd ich
               im Schlafen nicht gestört, – wenn auch diese Geräusche auf mancherlei, oft ganz
               phantastische Art sich in meine Träume drängen.\pend
           
\pstart
           Auch meine persönliche Existenz ist ja nicht ganz einfach, wie Sie wissen; aber es
               würde zu weit führen, da in Einzelheiten einzugehen; – an Conflicten seelischer Art
               mangelt es ja in diesen Grenzjahren (es ist vielleicht kühn, mit 63 noch von
               Grenzjahren zu reden, aber gerade Sie werden mich verstehen) nie.\pend
           
\pstart
           Dabei fühl ich doch, daſs ich im Grunde nicht klagen dürfte (ich thu’s auch
               selten), – besonders darum weil meine beiden Kinder\pwindex{Schnitzler, Heinrich 09.08.1902 – 12.07.1982@\textsc{Schnitzler, Heinrich} (09.08.1902 – 12.07.1982), \emph{Regisseur/Regisseurin, Schauspieler/Schauspielerin}|pwv}\pwindex{Cappellini, Lili 13.09.1909 – 26.07.1928@\textsc{Cappellini, Lili} (13.09.1909 – 26.07.1928)|pwv} sehr wohl gerathen sind
               (auch steh ich jetzt mit meiner früheren Gattin\pwindex{Schnitzler, Olga 17.01.1882 – 13.01.1970@\textsc{Schnitzler, Olga} (17.01.1882 – 13.01.1970), \emph{Schauspieler/Schauspielerin, Sänger/Sängerin}|pwv}, die in Baden-Baden\oindex{Baden-Baden@\textbf{Baden-Baden}, \emph{P.PPL}|pw} lebt, in sehr freundschaftlichen, natürlich nicht immer unge{\pb}trübten Beziehungen), und ferner weil ich mich in
               meiner Schaffenslust eher noch wachsen als abnehmen fühle. Auch an äußeren Erfolgen
               fehlt es nicht; und nach einer Periode, die sich ein wenig bedenklich anließ, glaub
               ich auch materiell – ach nicht durch das Vorhandensein eines Vermögens – wer besitzt
                  de{\geminationn} jetzt etwas!, – aber durch das Ansteigen meiner
               Einnahmen, – mit Ruhe in die Zukunft blicken zu dürfen. Und blasirt bin ich ja nicht
               – mir macht eigentlich alles mehr Freude als es mir in meiner Jugend gemacht hat, –
               jede Blume, jeder Spaziergang, jedes schöne Buch und Herzlichkeit mancher Art, die
               mir entgegengebracht wird. »So wollen wirs de{\geminationn} noch eine
               Weile weiter treiben« wie ein sehr Großer gesagt haben kö{\geminationn}te und wahrscheinlich irgendwo gesagt hat – und Sie sollen wissen, liebster Freund,
               daſs ich, we{\geminationn} auch gelegentlich ein wenig verdüstert,
                  {\pb}mich gar nicht übel befinde; – und hoffentlich
               mach ich auch Ihnen einen vergnügtem Eindruck, we{\geminationn} wir
               uns wiedersehen.\pend
           
\pstart
           Wie gut begreife ich, daſs Sie nicht nach »Leningrad\oindex{Sankt Petersburg@\textbf{Sankt Petersburg}, \emph{P.PPLA}|pw}« gehen wollen – auch ich, (selbst we{\geminationn}
               ich dort nicht reden müßte,) hätte nicht die geringste Lust dazu. Kennen Sie das \introOben{}(kleine)\introOben{}{ }Buch\pwindex{ABC des Kommunismus@\emph{Das ABC des Kommunismus}|pwv}\pwindex{ABC des Kommunismus@\emph{Das ABC des Kommunismus}|pwv} von Bucharin\pwindex{Bucharin, Nikolaj Ivanovic 9.10.1888 – 15.03.1938@\textsc{Bucharin, Nikolaj Ivanovič} (9.10.1888 – 15.03.1938), \emph{Politiker/Politikerin, Journalist/Journalistin}|pw} über den Bolschewismus? Wenn die
               deutsche Übersetzung nicht \label{T_L02444-2v}\edtext{etwa}{\lemma{\textnormal{\emph{etwa}}}\Cendnote{\textnormal{unsicher zu lesen; wohl zur Verdeutlichung
                  gestrichen und über der Zeile wiederholt}}}\label{T_L02444-2} zu dem Zwecke gefälscht ist, um
               die Idee – (die Idee!!) des Bolschewismus zu compromittieren, da{\geminationn} hat es Bucharin\pwindex{Bucharin, Nikolaj Ivanovic 9.10.1888 – 15.03.1938@\textsc{Bucharin, Nikolaj Ivanovič} (9.10.1888 – 15.03.1938), \emph{Politiker/Politikerin, Journalist/Journalistin}|pw}
               selbst in unübertrefflicher Weise gethan. –\pend
           
\pstart
           Ihren Brief hab ich in Bozen\oindex{Bozen@\textbf{Bozen}, \emph{P.PPLA2}|pw} erhalten, (Bolzano\oindex{Bozen@\textbf{Bozen}, \emph{P.PPLA2}|pw}) von wo ich erst vor ein paar Tagen nach
                  Wien\oindex{Wien@\textbf{Wien}, \emph{A.ADM2}|pw} zurückgekehrt bin. Ich bleibe nur den
                  Juli über hier, und fahre im August wahrscheinlich
               wieder in die Dolomiten\oindex{Dolomiten@\textbf{Dolomiten}, \emph{Gebirge (N.GBR)}|pw}. Für den Herbst steht
               mir allerlei bevor: in {\pb}\label{K_L02444-1v}\edtext{Berlin\oindex{Berlin@\textbf{Berlin}, \emph{P.PPLC}|pw}}{\lemma{\textnormal{\emph{Berlin}}}\Cendnote{\textnormal{Die geplante Inszenierung von Victor Barnowsky\pwindex{Barnowsky, Victor 10.09.1875 – 09.08.1952@\textsc{Barnowsky, Victor} (10.09.1875 – 09.08.1952), \emph{Theaterleiter/Theaterleiterin, Regisseur/Regisseurin, Schauspieler/Schauspielerin}|pwk} wurde nicht realisiert
                  (vgl. \emph{Briefe 1913–1931}, S. 468).}}}\label{K_L02444-1} die
               Aufführung der Komödie der Verführung\pwindex{Komoedie der Verfuehrung. In drei Akten@\emph{Komödie der Verführung. In drei Akten}|pw}, – in
                  \label{K_L02444-2v}\edtext{Wien\oindex{Wien@\textbf{Wien}, \emph{A.ADM2}|pw} Reprisen}{\lemma{\textnormal{\emph{Wien Reprisen}}}\Cendnote{\textnormal{\emph{Das weite Land}\pwindex{weite Land. Tragikomoedie in fuenf Akten@\emph{Das weite Land. Tragikomödie in fünf Akten}|pwk} wurde ab
                  4. 9. 1925 am \emph{Deutschen
                     Volkstheater}\orgindex{Volkstheater@Volkstheater|pwk} gegeben, wo auch \emph{Der einsame
                     Weg}\pwindex{einsame Weg. Schauspiel in fuenf Akten@\emph{Der einsame Weg. Schauspiel in fünf Akten}|pwk} am 14. 11. 1925 im Zuge eines Gastspiels von Albert Bassermann\pwindex{Bassermann, Albert 07.09.1867 – 15.05.1952@\textsc{Bassermann, Albert} (07.09.1867 – 15.05.1952), \emph{Schauspieler/Schauspielerin}|pwk} aufgeführt wurde.}}}\label{K_L02444-2}
               von »Das weite Land\pwindex{weite Land. Tragikomoedie in fuenf Akten@\emph{Das weite Land. Tragikomödie in fünf Akten}|pw}« und »der einsame Weg\pwindex{einsame Weg. Schauspiel in fuenf Akten@\emph{Der einsame Weg. Schauspiel in fünf Akten}|pw}«, vielleicht auch ein neues Stück\pwindex{Gang zum Weiher. Dramatische Dichtung@\emph{Der Gang zum Weiher. Dramatische Dichtung}|pwv} (in Versen). Ein paar Novellen\pwindex{Frau des Richters. Novelle@\emph{Die Frau des Richters. Novelle}|pwv}\pwindex{Traumnovelle@\emph{Traumnovelle}|pwv} sind auch fertig. In
                  \label{K_L02444-3v}\edtext{Paris\oindex{Paris@\textbf{Paris}, \emph{P.PPLC}|pw}}{\lemma{\textnormal{\emph{Paris}}}\Cendnote{\textnormal{Nicht realisiert, das Stück wurde erst
                     1931 gegeben.}}}\label{K_L02444-3} wird vielleicht »das weite Land\pwindex{weite Land. Tragikomoedie in fuenf Akten@\emph{Das weite Land. Tragikomödie in fünf Akten}|pw}« gespielt werden; und nach Amerika\oindex{Amerika@\textbf{Amerika}, \emph{kein passender Code gefunden}|pw} bin ich zur Premiere des »einsamen Wegs\pwindex{einsame Weg. Schauspiel in fuenf Akten@\emph{Der einsame Weg. Schauspiel in fünf Akten}|pw}« im \label{K_L02444-4v}\edtext{Guild Theater\orgindex{Guild Theatre@Guild Theatre|pw}}{\lemma{\textnormal{\emph{Guild Theater}}}\Cendnote{\textnormal{Das Stück\pwindex{einsame Weg. Schauspiel in fuenf Akten@\emph{Der einsame Weg. Schauspiel in fünf Akten}|pwkv} wurde erst 1931 auf den Spielplan
                  genommen.}}}\label{K_L02444-4} u. des »Ruf des Lebens\pwindex{Ruf des Lebens. Schauspiel in drei Akten@\emph{Der Ruf des Lebens. Schauspiel in drei Akten}|pw}« am
                  \label{K_L02444-5v}\edtext{Astor Theater\orgindex{Astor Theatre@Astor Theatre|pw}}{\lemma{\textnormal{\emph{Astor Theater}}}\Cendnote{\textnormal{Produziert vom \emph{Astor Theater}\orgindex{Astor Theatre@Astor Theatre|pwk}, wurde \emph{The Call of
                     Life}\pwindex{Ruf des Lebens. Schauspiel in drei Akten@\emph{Der Ruf des Lebens. Schauspiel in drei Akten}|pwk} im Comedy Theatre\oindex{Comedy Theatre@\textbf{Comedy Theatre}, \emph{Theater (K.THE)}|pwk} am
                     9. 10. 1925 zum ersten Mal und in Folge neunzehnmal gegeben. Die
                  Bearbeitung stammte von Dorothy
                  Donnelli\pwindex{Donnelly, Dorothy 28.01.1880 – 03.01.1928@\textsc{Donnelly, Dorothy} (28.01.1880 – 03.01.1928), \emph{Schauspieler/Schauspielerin}|pwk}.}}}\label{K_L02444-5} eingeladen (Ich werde aber kaum hinreisen.) –\pend
           
\pstart
           – Ich lese immer noch, aufs stärkste angeregt, Ihren wunderbaren Julius Caesar\pwindex{Gaius Julius Cæsar@\emph{Gaius Julius Cæsar}|pw}. Und erwarte Ihr »Hellas\pwindex{Hellas@\emph{Hellas}|pw}«. –\pend
           
\pstart
           Bleiben Sie mir weiter, und lange noch der Freund, der Sie mir immer waren; es ist
               schön zu wissen, daß Sie auf der Welt sind! Ich grüße Sie von Herzen!\pend
           
\pstart
           Ihr{\\[\baselineskip]}\spacefill\mbox{Arthur Schnitzler}\pend
           \leftskip=0em{}\selectlanguage{ngerman}\endnumbering\briefempfaengerindex{Brandes, Georg@\textsc{Brandes, Georg}!zzzSchnitzler, Arthur@\emph{von Arthur Schnitzler}!1925-07-071@{7. 7. 1925}|)be}\mylabel{L02444h}  \normalsize

\doendnotes{C}
\bigskip
\vfill

\clearpage

\footnotesize

\lohead{\textsc{register}}

% Definiere theindex-Environment komplett neu ohne reledmac
\makeatletter
\renewenvironment{theindex}{%
  \section*{\indexname}%
  \setlength{\parindent}{0pt}%
  \setlength{\parskip}{0pt plus 0.3pt}%
  \let\item\@idxitem
}{%
  \clearpage
}
\makeatother

\IfFileExists{\jobname-pw.ind}{\input{\jobname-pw.ind}}{}

\end{document}

      