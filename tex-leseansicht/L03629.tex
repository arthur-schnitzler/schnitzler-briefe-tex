%% latex-korrekturansicht-vorspann.tex
%% Vorspann für die Korrekturansicht.
%% Lädt die gemeinsame Datei latex-vorspann.tex mit gesetztem Schalter.

\newif\ifkorrekturansicht
\korrekturansichttrue

\input{../tex-inputs/latex-vorspann}


\section[Stefan Zweig an Arthur Schnitzler, 19. 11. 1911]{L03629 Stefan Zweig an Arthur Schnitzler, 19. 11. 1911}
\nopagebreak\mylabel{L03629v}
\rehead{ }\normalsize\beginnumbering\briefempfaengerindex{Schnitzler, Arthur@\textsc{Schnitzler, Arthur}!zzzZweig, Stefan@\emph{von Stefan Zweig}!1911-11-191@{19. 11. 1911}|(be}
\toendnotes[C]{\smallbreak\pagebreak[2]}\Standort{CUL, Schnitzler, B 118.}
\physDesc{Bildpostkarte, 517 Zeichen
\newline{}Handschrift: schwarze Tinte, lateinische Kurrent
\newline{}Versand: Stempel: »\nobreak{}\oindex{Mals@\textbf{Mals}|pwk}Mals-Bozen, 19. XI. 11\nobreak{}«.  }
\buchAbdrucke{\weitereDrucke{Stefan Zweig: \emph{Briefwechsel mit Hermann Bahr, Sigmund Freud, Rainer Maria
                        Rilke und Arthur Schnitzler}. Frankfurt am Main: \emph{S. Fischer} 1987, S. 368.} }\toendnotes[C]{\smallbreak}\pstart{}{\pb}D\textsuperscript{r} Artur
                  Schnitzler\pend{}\pstart{}Wien – Cottage\oindex{Waehringer Cottage@\textbf{Währinger Cottage}, \emph{Teil eines besiedelten Ortes (A.BSOX)}|pw}\pend{}\pstart{}\label{K_L03629-1v}\edtext{Sternwartestrasse 72}{\lemma{\textnormal{\emph{Sternwartestrasse 72}}}\Cendnote{\textnormal{Zweig\pwindex{Zweig, Stefan 28.11.1881 – 23.02.1942@\textsc{Zweig, Stefan} (28.11.1881 – 23.02.1942), \emph{Schriftsteller/Schriftstellerin}|pwk} wechselt bei der Adressierung
                        seiner Schreiben an Schnitzler immer
                        wieder zwischen der falschen Hausnummer »72« und der
                        richtigen »71«.}}}\label{K_L03629-1}\oindex{Sternwartestrasse 71@\textbf{Sternwartestraße 71}, \emph{Wohngebäude (K.WHS)}|pw}\pend{}{\bigskip}
\pstart
           \noindent{}\centering{}\textcolor{gray}{\textbf{{\pb}Südtirol, Kurort Meran\oindex{Meran@\textbf{Meran}, \emph{P.PPLA3}|pw} geg Vinschgau\oindex{Val Venosta@\textbf{Val Venosta}, \emph{T.VAL}|pw}.
               }}\pend
           
\pstart
           \centering{}\textcolor{gray}{\textbf{Zielspitze\oindex{Zielspitze@\textbf{Zielspitze}, \emph{Berg (N.BRG)}|pw}, 3006 m.\hspace*{2em}Gfallwand\oindex{Gfallwand@\textbf{Gfallwand}, \emph{T.MT}|pw}, 3179 m.\hspace*{2em}Blasius-Spitze\oindex{Blasiuszeiger@\textbf{Blasiuszeiger}, \emph{Berg (N.BRG)}|pw}.\hspace*{2em}Roteck\oindex{Roteck@\textbf{Roteck}, \emph{T.MT}|pw} 3331 m.\hspace*{2em}Tschigat\oindex{Tschigat@\textbf{Tschigat}, \emph{T.PK}|pw}, 2999 m.}}\pend
           {\vspace{1\baselineskip}}\vspace{1em}
\pstart
           \noindent{}{\pb}Verehrter Herr Doktor, ich habe Morisse\pwindex{Morisse, Paul 1866-03-11 – 1946-09-28@\textsc{Morisse, Paul} (1866-03-11 – 1946-09-28), \emph{Übersetzer/Übersetzerin}|pw} nochmals geschrieben, er möge \label{K_L03629-2v}\edtext{bei einer Übertragung}{\lemma{\textnormal{\emph{bei einer Übertragung}}}\Cendnote{\textnormal{Vgl. Stefan Zweig an Arthur Schnitzler, 6. [11.?] 1911.}}}\label{K_L03629-2} womöglich gemeinsam mit einem Theaterroutinièr\pwindex{Charasson, Henriette 1884-01-06 – 1972-12-24@\textsc{Charasson, Henriette} (1884-01-06 – 1972-12-24), \emph{Schriftsteller/Schriftstellerin}|pwv} vorgehen
               und zweifle nicht, dass er diesen Rat befolgen würde. Er ist sehr tüchtig und hat den
               Vorteil einige Jahre in \uline{Wien}\oindex{Wien@\textbf{Wien}, \emph{A.ADM2}|pw} gelebt zu haben und das Specifisch-Wienerische\oindex{Wien@\textbf{Wien}, \emph{A.ADM2}|pw} besser wiederzugeben. Ich hoffe Sie bald darüber nach {\pb}meiner Rückkunft sprechen zu können und
               grüsse Ihre Frau Gemahlin\pwindex{Schnitzler, Olga 17.01.1882 – 13.01.1970@\textsc{Schnitzler, Olga} (17.01.1882 – 13.01.1970), \emph{Schauspieler/Schauspielerin, Sänger/Sängerin}|pwv} und
               Sie viele Male als Ihr getreuer\pend
           \pstart \spacefill\mbox{Stefan Zweig}\pend{}\selectlanguage{ngerman}\endnumbering\briefempfaengerindex{Schnitzler, Arthur@\textsc{Schnitzler, Arthur}!zzzZweig, Stefan@\emph{von Stefan Zweig}!1911-11-191@{19. 11. 1911}|)be}\mylabel{L03629h}  \normalsize

\doendnotes{C}
\bigskip
\vfill

\clearpage

\footnotesize

\lohead{\textsc{register}}

% Definiere theindex-Environment komplett neu ohne reledmac
\makeatletter
\renewenvironment{theindex}{%
  \section*{\indexname}%
  \setlength{\parindent}{0pt}%
  \setlength{\parskip}{0pt plus 0.3pt}%
  \let\item\@idxitem
}{%
  \clearpage
}
\makeatother

\IfFileExists{\jobname-pw.ind}{\input{\jobname-pw.ind}}{}

\end{document}

      