%% latex-leseansicht-vorspann.tex
%% Vorspann für die Leseansicht.
%% Lädt die gemeinsame Datei latex-vorspann.tex mit nicht gesetztem Schalter.

\newif\ifkorrekturansicht
\korrekturansichtfalse

\input{../tex-inputs/latex-vorspann}


\section[Stefan Zweig an Arthur Schnitzler, 19. 11. 1911]{L03629 Stefan Zweig an Arthur Schnitzler, 19. 11. 1911}
\nopagebreak\mylabel{L03629v}
\rehead{ }\normalsize\beginnumbering\briefempfaengerindex{Schnitzler, Arthur@\textsc{Schnitzler, Arthur}!zzzZweig, Stefan@\emph{von Stefan Zweig}!1911-11-191@{19. 11. 1911}|(be}
\toendnotes[C]{\smallbreak\pagebreak[2]}
\correspDesc{Versand  durch Stefan Zweig am 19. 11. 1911 in Mals
\newline{}Erhalt  durch Arthur Schnitzler im Zeitraum [20. 11. 1911 – 24. 11. 1911?] in Wien}\toendnotes[C]{\smallbreak}
\Standort{CUL, Schnitzler, B 118.}
\physDesc{Bildpostkarte, 517 Zeichen
\newline{}Handschrift: schwarze Tinte, lateinische Kurrent
\newline{}Versand: Stempel: »\nobreak{}\oindex{Mals@\textbf{Mals}|pwk}Mals-Bozen, 19. XI. 11\nobreak{}«.  }
\buchAbdrucke{\weitereDrucke{Stefan Zweig: \emph{Briefwechsel mit Hermann Bahr, Sigmund Freud, Rainer Maria
                        Rilke und Arthur Schnitzler}. Herausgegeben von Jeffrey B. Berlin, Hans-Ulrich Lindken und Donald A. Prater. Frankfurt am Main: \emph{S. Fischer} 1987, S. 368.} }\toendnotes[C]{\smallbreak}\pstart{}{\pb}D\textsuperscript{r} Artur
                  Schnitzler\pend{}\pstart{}Wien – Cottage\oindex{Wien@\textbf{Wien}!XVIII., Währing@\textbf{XVIII., Währing}!Währinger Cottage@\textbf{Währinger Cottage}, \emph{Teil eines besiedelten Ortes}|pw}\pend{}\pstart{}\label{K_L03629-1v}\edtext{Sternwartestrasse 72}{\lemma{\textnormal{\emph{Sternwartestrasse 72}}}\Cendnote{\textnormal{Zweig\pwindex{Zweig, Stefan 28.\,11.\,1881 Wien – 23.\,2.\,1942 Petrópolis@\textsc{Zweig, Stefan} (28.\,11.\,1881 Wien – 23.\,2.\,1942 Petrópolis), \emph{Schriftsteller}|pwk} wechselt bei der Adressierung
                        seiner Schreiben an Schnitzler immer
                        wieder zwischen der falschen Hausnummer »72« und der
                        richtigen »71«.}}}\label{K_L03629-1}\oindex{Wien@\textbf{Wien}!XVIII., Währing@\textbf{XVIII., Währing}!Sternwartestraße 71@\textbf{Sternwartestraße 71}, \emph{Wohngebäude}|pw}\pend{}{\bigskip}
\pstart
           \noindent{}\centering{}\textcolor{gray}{\textbf{{\pb}Südtirol, Kurort Meran\oindex{Meran@\textbf{Meran}, \emph{Hauptstadt}|pw} geg Vinschgau\oindex{Val Venosta@\textbf{Val Venosta}, \emph{Tal}|pw}.}}\pend
           
\pstart
           \centering{}\textcolor{gray}{\textbf{Zielspitze\oindex{Zielspitze@\textbf{Zielspitze}, \emph{Berg}|pw}, 3006 m.\hspace*{2em}Gfallwand\oindex{Gfallwand@\textbf{Gfallwand}, \emph{Berg}|pw}, 3179 m.\hspace*{2em}Blasius-Spitze\oindex{Blasiuszeiger@\textbf{Blasiuszeiger}, \emph{Berg}|pw}.\hspace*{2em}Roteck\oindex{Roteck@\textbf{Roteck}, \emph{Berg}|pw} 3331 m.\hspace*{2em}Tschigat\oindex{Tschigat@\textbf{Tschigat}, \emph{Bergspitze}|pw}, 2999 m.}}\pend
           {\vspace{1\baselineskip}}\vspace{1em}
\pstart
           \noindent{}{\pb}Verehrter Herr Doktor, ich habe Morisse\pwindex{Morisse, Paul 11.\,3.\,1866 Rouen – 28.\,9.\,1946 Paris@\textsc{Morisse, Paul} (11.\,3.\,1866 Rouen – 28.\,9.\,1946 Paris), \emph{Übersetzer}|pw} nochmals geschrieben, er möge \label{K_L03629-2v}\edtext{bei einer Übertragung}{\lemma{\textnormal{\emph{bei einer Übertragung}}}\Cendnote{\textnormal{Vgl. XXXX Auszeichnungsfehler: Dokument L03630 nicht gefunden.}}}\label{K_L03629-2} womöglich gemeinsam mit einem Theaterroutinièr\pwindex{Charasson, Henriette 6.\,1.\,1884 Le Havre – 24.\,12.\,1972 Châteauroux@\textsc{Charasson, Henriette} (6.\,1.\,1884 Le Havre – 24.\,12.\,1972 Châteauroux), \emph{Schriftstellerin}|pwv} vorgehen
               und zweifle nicht, dass er diesen Rat befolgen würde. Er ist sehr tüchtig und hat den
               Vorteil einige Jahre in \uline{Wien}\oindex{Wien@\textbf{Wien}, \emph{Verwaltungsgebiet}|pw} gelebt zu haben und das Specifisch-Wienerische\oindex{Wien@\textbf{Wien}, \emph{Verwaltungsgebiet}|pw} besser wiederzugeben. Ich hoffe Sie bald darüber nach {\pb}meiner Rückkunft sprechen zu können und
               grüsse Ihre Frau Gemahlin\pwindex{Schnitzler, Olga 17.\,1.\,1882 Wien – 13.\,1.\,1970 Lugano@\textsc{Schnitzler, Olga} (17.\,1.\,1882 Wien – 13.\,1.\,1970 Lugano), \emph{Schauspielerin, Sängerin}|pwv} und
               Sie viele Male als Ihr getreuer\pend
           \pstart \spacefill\mbox{Stefan Zweig}\pend{}\selectlanguage{ngerman}\endnumbering\briefempfaengerindex{Schnitzler, Arthur@\textsc{Schnitzler, Arthur}!zzzZweig, Stefan@\emph{von Stefan Zweig}!1911-11-191@{19. 11. 1911}|)be}\mylabel{L03629h}  \newcommand{\dateiname}{L03629}\newcommand{\titel}{Stefan Zweig an Arthur Schnitzler, 19. 11. 1911}\newcommand{\editorInnen}{Selma Jahnke und Martin Anton Müller}%% latex-leseansicht-abspann.tex
%% Abspann für die Leseansicht.
%% Der Schalter \ifkorrekturansicht ist bereits durch den Vorspann gesetzt.

%% latex-abspann.tex
%% Gemeinsamer Abspann für Korrekturansicht und Leseansicht.
%% Setzt den Schalter \ifkorrekturansicht voraus (gesetzt in den
%% einbindenden Dateien latex-korrekturansicht-abspann.tex bzw.
%% latex-leseansicht-abspann.tex).
%% ---------------------------------------------------------------

\normalsize

% Das esempio-Environment wird nur in der Leseansicht benötigt
\ifkorrekturansicht\else
\newenvironment{esempio}[3]%
{
    \vspace{1.5ex}
    \rlap{\underline{#1}}
    \par
    \setlength{\parindent}{0cm}
    \nopagebreak
    \leftskip=#2cm
    \rightskip=#3cm
}
{
    \par
}
\fi

\doendnotes{C}
\bigskip
\vfill

\clearpage

\footnotesize

\ifkorrekturansicht
  \lohead{\textsc{register}}
\fi

% theindex-Environment neu definieren ohne reledmac
\makeatletter
\renewenvironment{theindex}{%
  \ifkorrekturansicht
    \section*{\indexname}%
  \else
    \subsubsection*{Index der erwähnten Entitäten}%
  \fi
  \setlength{\parindent}{0pt}%
  \setlength{\parskip}{0pt plus 0.3pt}%
  \let\item\@idxitem
}{%
  \ifkorrekturansicht\clearpage\fi
}
\makeatother

\IfFileExists{\jobname-pw.ind}{\input{\jobname-pw.ind}}{}

% Quellenangabe nur in der Leseansicht
\ifkorrekturansicht\else
% Fallback-Definitionen, falls die .tex-Datei \titel etc. nicht gesetzt hat
\providecommand{\titel}{}
\providecommand{\editorInnen}{}
\providecommand{\dateiname}{\jobname}

\vspace{3cm}

\vfill

\footnotesize
\textsc{Quelle}: \titel. Herausgegeben von {\editorInnen}. In: \emph{Arthur Schnitzler: Briefwechsel mit Autorinnen und Autoren}.
 Digitale Edition, https://schnitzler-briefe.acdh.oeaw.ac.at/{\dateiname}.html (Stand \today)
\fi

\end{document}


