%% latex-leseansicht-vorspann.tex
%% Vorspann für die Leseansicht.
%% Lädt die gemeinsame Datei latex-vorspann.tex mit nicht gesetztem Schalter.

\newif\ifkorrekturansicht
\korrekturansichtfalse

\input{../tex-inputs/latex-vorspann}

\begin{center}
            \textcolor{red}{ENTWURF, NICHT FERTIG KORRIGIERT}
                      \end{center}
            
         
         \renewcommand{\erwaehntePersonen}{Personen: Adele Sandrock, Christian Sandrock, Johanna Simonetta Sandrock}
         \renewcommand{\erwaehnteInstitutionen}{Institutionen: Oseh Chesed (Humanitas)}
         \renewcommand{\erwaehnteOrte}{Orte: Wien}
         \renewcommand{\erwaehnteWerke}{Werke: Tagebuch}
               \section[Felix Salten an Arthur Schnitzler, {[}26. 1. 1895{]}]{ Felix Salten an Arthur Schnitzler, {[}26. 1. 1895{]}}\nopagebreak\mylabel{v}\rehead{ }\begin{ledgroupsized}[t]{13cm}\normalsize\beginnumbering \toendnotes[C]{\smallbreak\pagebreak[2]} \Standort{CUL, Schnitzler, B 89, A 1.}
\physDesc{Brief, 1 Blatt, 3 Seiten, 981 Zeichen
\newline{}Handschrift: Bleistift, lateinische Kurrent
\newline{}Schnitzler: mit Bleistift datiert: »26/1 95« 
\newline{}Ordnung: mit Bleistift von unbekannter Hand nummeriert:
                                    »50« }\toendnotes[C]{\smallbreak}\pstart
           \noindent{}{\pb}Lieber Freund, ich habe die \label{K_L03149-1v}\edtext{grösste Verzweiflung\pwindex{Sandrock, Adele 1863-08-19 – 1937-08-30@\textsc{Sandrock, Adele} (1863-08-19 – 1937-08-30), \emph{Schauspielerin}|pwv}}{\lemma{\textnormal{\emph{grösste Verzweiflung}}}\Cendnote{\textnormal{Der Brief deckt sich über Teile mit dem,
                  was Schnitzler\pwindex{Schnitzler, Arthur 15.05.1862 – 21.10.1931@\textsc{Schnitzler, Arthur} (15.05.1862 – 21.10.1931), \emph{Schriftsteller, Mediziner}|pwk} im \emph{Tagebucheintrag}\pwindex{Schnitzler, Arthur 15.05.1862 – 21.10.1931@\textsc{Schnitzler, Arthur} (15.05.1862 – 21.10.1931), \emph{Schriftsteller, Mediziner}!Tagebuch1981 – 2000@\strich\emph{Tagebuch} {[}1981 – 2000{]}|pwk} zum selben Tag (26. 1. 1895) erwähnt,
                  von Salten\pwindex{Salten, Felix 06.09.1869 – 08.10.1945@\textsc{Salten, Felix} (06.09.1869 – 08.10.1945), \emph{Schriftsteller, Journalist}|pwk} im Kaffeehaus erfahren zu
                  haben.}}}\label{K_L03149-1h} vorgefunden. Weinkrämpfe, Zerknirschung, kurz Alles. \pend
           \pstart
           Die Sache lief darauf hinaus, dass mir erklärt wurde, wenn nicht morgen
               um 12, so \strikeout{eine} eine Leiche, ec. ec. Sehr
               viel Details von menschlicher Wichtigkeit! Bruder\pwindex{Sandrock, Christian 1865-01-23 – 1924@\textsc{Sandrock, Christian} (1865-01-23 – 1924), \emph{Schriftsteller, Bildender Künstler}|pwv}, Mutter\pwindex{Sandrock, Johanna Simonetta 27.6.1833 – 6.4.1917@\textsc{Sandrock, Johanna Simonetta} (27.6.1833 – 6.4.1917), \emph{Schauspielerin}|pwv} ec. \pend
           \pstart
           Der Schluss war, dass sie\pwindex{Sandrock, Adele 1863-08-19 – 1937-08-30@\textsc{Sandrock, Adele} (1863-08-19 – 1937-08-30), \emph{Schauspielerin}|pwv}
               sagte, bitte geh’ nach Hause{\dotstwo} Darauf ich, – es ist noch
               früh. – {[}»{]}Bitte, geh’ ich möchte mich niederlegen.« Darauf ich:
               Wann sehen wir uns wieder? Sie\pwindex{Sandrock, Adele 1863-08-19 – 1937-08-30@\textsc{Sandrock, Adele} (1863-08-19 – 1937-08-30), \emph{Schauspielerin}|pwv}: Nie!!! Ich: Ist das Ernst? Sie\pwindex{Sandrock, Adele 1863-08-19 – 1937-08-30@\textsc{Sandrock, Adele} (1863-08-19 – 1937-08-30), \emph{Schauspielerin}|pwv} »\textcolor{gray}{Ni{\geminationm}er}!
                  {\pb}denn ich kann nicht.
               Darauf bin ich ohne Gruß \textcolor{gray}{f}ort. \pend
           \pstart
           Die Sache macht mir den Eindruck, dass zwar noch einiges zu überstehen sein wird,
               jedoch schließlich wird sich All das geben. Es braucht nur Vorsicht. \pend
           \pstart
           Morgen hoffe ich Sie zu sehen. Vielleicht geben Sie mir Nachricht, wann
               ich zu Ihnen kommen soll, \strikeout{N\textcolor{gray}{achm}} oder kommen \substVorne{}\textsuperscript{Nach}\substDazwischen{}selbst\substHinten{}{ }{\pb}zu mir. Ich werde bis gegen
                  12\textsuperscript{h.} zu Hause sein. \pend
           \pstart
           Jetzt gehe ich zur Humanitas\orgindex{Oseh Chesed (Humanitas)@Oseh Chesed (Humanitas)|pw}, aus dringendem
               Bedürfnis nach einer Stunde unter Leuten, die keine tragischen Gebärden haben. \pend
           \pstart
           Herzlich Ihr {\\[\baselineskip]}\spacefill\mbox{Salten}\pend
           \leftskip=0em{}
         
         \endnumbering\mylabel{h}\end{ledgroupsized}\begin{anhang}\end{anhang}\newcommand{\dateiname}{L03149}\newcommand{\titel}{Felix Salten an Arthur Schnitzler, [26. 1. 1895]}\newcommand{\editorInnen}{Martin Anton Müller und Laura Untner}%% latex-leseansicht-abspann.tex
%% Abspann für die Leseansicht.
%% Der Schalter \ifkorrekturansicht ist bereits durch den Vorspann gesetzt.

%% latex-abspann.tex
%% Gemeinsamer Abspann für Korrekturansicht und Leseansicht.
%% Setzt den Schalter \ifkorrekturansicht voraus (gesetzt in den
%% einbindenden Dateien latex-korrekturansicht-abspann.tex bzw.
%% latex-leseansicht-abspann.tex).
%% ---------------------------------------------------------------

\normalsize

% Das esempio-Environment wird nur in der Leseansicht benötigt
\ifkorrekturansicht\else
\newenvironment{esempio}[3]%
{
    \vspace{1.5ex}
    \rlap{\underline{#1}}
    \par
    \setlength{\parindent}{0cm}
    \nopagebreak
    \leftskip=#2cm
    \rightskip=#3cm
}
{
    \par
}
\fi

\doendnotes{C}
\bigskip
\vfill

\clearpage

\footnotesize

\ifkorrekturansicht
  \lohead{\textsc{register}}
\fi

% theindex-Environment neu definieren ohne reledmac
\makeatletter
\renewenvironment{theindex}{%
  \ifkorrekturansicht
    \section*{\indexname}%
  \else
    \subsubsection*{Index der erwähnten Entitäten}%
  \fi
  \setlength{\parindent}{0pt}%
  \setlength{\parskip}{0pt plus 0.3pt}%
  \let\item\@idxitem
}{%
  \ifkorrekturansicht\clearpage\fi
}
\makeatother

\IfFileExists{\jobname-pw.ind}{\input{\jobname-pw.ind}}{}

% Quellenangabe nur in der Leseansicht
\ifkorrekturansicht\else
% Fallback-Definitionen, falls die .tex-Datei \titel etc. nicht gesetzt hat
\providecommand{\titel}{}
\providecommand{\editorInnen}{}
\providecommand{\dateiname}{\jobname}

\vspace{3cm}

\vfill

\footnotesize
\textsc{Quelle}: \titel. Herausgegeben von {\editorInnen}. In: \emph{Arthur Schnitzler: Briefwechsel mit Autorinnen und Autoren}.
 Digitale Edition, https://schnitzler-briefe.acdh.oeaw.ac.at/{\dateiname}.html (Stand \today)
\fi

\end{document}


      