%% latex-korrekturansicht-vorspann.tex
%% Vorspann für die Korrekturansicht.
%% Lädt die gemeinsame Datei latex-vorspann.tex mit gesetztem Schalter.

\newif\ifkorrekturansicht
\korrekturansichttrue

\input{../tex-inputs/latex-vorspann}


\section[ Felix Salten an Arthur Schnitzler, {[}26. 1. 1895{]}]{L03149 Felix Salten an Arthur Schnitzler, {[}26. 1. 1895{]}}
\nopagebreak\mylabel{L03149v}
\rehead{ }\normalsize\beginnumbering\briefempfaengerindex{Schnitzler, Arthur@\textsc{Schnitzler, Arthur}!zzzSalten, Felix@\emph{von Felix Salten}!1895-01-261@{{[}26. 1. 1895{]}}|(be}
\toendnotes[C]{\smallbreak\pagebreak[2]}\Standort{CUL, Schnitzler, B 89, A 1.}
\physDesc{Brief, 1 Blatt, 3 Seiten, 976 Zeichen
\newline{}Handschrift: Bleistift, lateinische Kurrent
\newline{}Schnitzler: mit Bleistift datiert: »26/1 95« 
\newline{}Ordnung: mit Bleistift von unbekannter Hand nummeriert: »50« }\toendnotes[C]{\smallbreak}
\pstart
           \noindent{}{\pb}Lieber Freund, ich habe die \label{K_L03149-1v}\edtext{grösste Verzweiflung\pwindex{Sandrock, Adele 1863-08-19 – 1937-08-30@\textsc{Sandrock, Adele} (1863-08-19 – 1937-08-30), \emph{Schauspieler/Schauspielerin}|pwv}}{\lemma{\textnormal{\emph{grösste Verzweiflung}}}\Cendnote{\textnormal{Der Brief deckt sich über Teile mit dem,
                  was Schnitzler im \emph{Tagebucheintrag}\pwindex{Tagebuch@\emph{Tagebuch}|pwk} zum 26. 1. 1895 erwähnte, von Salten\pwindex{Salten, Felix 06.09.1869 – 08.10.1945@\textsc{Salten, Felix} (06.09.1869 – 08.10.1945), \emph{Schriftsteller/Schriftstellerin, Journalist/Journalistin, Chefredakteur/Chefredakteurin}|pwk} im Kaffeehaus erfahren zu haben.}}}\label{K_L03149-1} vorgefunden.
               Weinkrämpfe, Zerknirschung, kurz Alles.\pend
           
\pstart
           Die Sache lief darauf hinaus, dass mir erklärt wurde, wenn nicht morgen um 12, so \strikeout{eine} eine Leiche, ec. ec. Sehr viel Details von menschlicher Wichtigkeit:
                  Bruder\pwindex{Sandrock, Christian 1865-01-23 – 1924@\textsc{Sandrock, Christian} (1865-01-23 – 1924), \emph{Schriftsteller/Schriftstellerin, Maler/Malerin}|pwv}, Mutter\pwindex{Sandrock, Johanna Simonetta 27.6.1833 – 6.4.1917@\textsc{Sandrock, Johanna Simonetta} (27.6.1833 – 6.4.1917), \emph{Schauspieler/Schauspielerin}|pwv} ec.\pend
           
\pstart
           Der Schluss war, dass sie\pwindex{Sandrock, Adele 1863-08-19 – 1937-08-30@\textsc{Sandrock, Adele} (1863-08-19 – 1937-08-30), \emph{Schauspieler/Schauspielerin}|pwv}
               sagte, bitte geh’ nach Hause,. Darauf ich, – es ist noch früh. –
                  {[}»{]}Bitte, geh’ ich möchte mich niederlegen.« Darauf ich: Wann
               sehen wir uns wieder? Sie\pwindex{Sandrock, Adele 1863-08-19 – 1937-08-30@\textsc{Sandrock, Adele} (1863-08-19 – 1937-08-30), \emph{Schauspieler/Schauspielerin}|pwv}:
               Nie!!! Ich: Ist das Ernst? Sie\pwindex{Sandrock, Adele 1863-08-19 – 1937-08-30@\textsc{Sandrock, Adele} (1863-08-19 – 1937-08-30), \emph{Schauspieler/Schauspielerin}|pwv} »\textcolor{gray}{Ni{\geminationm}er}! {\pb}denn ich kann
                  nicht.{[}«{]} Darauf bin ich ohne Gruß \textcolor{gray}{f}ort.\pend
           
\pstart
           Die Sache macht mir den Eindruck, dass zwar noch \textcolor{gray}{ei}niges zu
               überstehen sein wird, jedoch schließlich wird sich All das geben. Es braucht nur
               Vorsicht.\pend
           
\pstart
           \label{K_L03149-2v}\edtext{Morgen hoffe ich Sie zu sehen}{\lemma{\textnormal{\emph{Morgen … sehen}}}\Cendnote{\textnormal{Siehe A. S.: \emph{Tagebuch}, 27. 1. 1895.
               }}}\label{K_L03149-2}. Vielleicht geben Sie mir Nachricht, wann ich zu Ihnen kommen soll, oder
               kommen \substVorne{}\textsuperscript{N\textcolor{gray}{achm}}\substDazwischen{}selbst\substHinten{}{ }{\pb}zu mir. Ich werde bis gegen
                     12\textsuperscript{h.}{ }zu Hause\oindex{Hoerlgasse 16@\textbf{Hörlgasse 16}, \emph{Wohngebäude (K.WHS)}|pwv} sein.\pend
           
\pstart
           Jetzt gehe ich zur Humanitas\orgindex{Oseh Chesed (Humanitas)@Oseh Chesed (Humanitas)|pw}, aus
                  dringende\textcolor{gray}{m} Bedürfnis nach einer Stunde unter Leuten, die keine
               tragischen Gebärden haben.\pend
           
\pstart
           Herzlich Ihr {\\[\baselineskip]}\spacefill\mbox{Salten}\pend
           \leftskip=0em{}\selectlanguage{ngerman}\endnumbering\briefempfaengerindex{Schnitzler, Arthur@\textsc{Schnitzler, Arthur}!zzzSalten, Felix@\emph{von Felix Salten}!1895-01-261@{{[}26. 1. 1895{]}}|)be}\mylabel{L03149h}  \normalsize

\doendnotes{C}
\bigskip
\vfill

\clearpage

\footnotesize

\lohead{\textsc{register}}

% Definiere theindex-Environment komplett neu ohne reledmac
\makeatletter
\renewenvironment{theindex}{%
  \section*{\indexname}%
  \setlength{\parindent}{0pt}%
  \setlength{\parskip}{0pt plus 0.3pt}%
  \let\item\@idxitem
}{%
  \clearpage
}
\makeatother

\IfFileExists{\jobname-pw.ind}{\input{\jobname-pw.ind}}{}

\end{document}

      