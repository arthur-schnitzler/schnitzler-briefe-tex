%% latex-korrekturansicht-vorspann.tex
%% Vorspann für die Korrekturansicht.
%% Lädt die gemeinsame Datei latex-vorspann.tex mit gesetztem Schalter.

\newif\ifkorrekturansicht
\korrekturansichttrue

\input{../tex-inputs/latex-vorspann}


\section[Arthur Schnitzler an Richard Beer-Hofmann, 26. 6. 1899]{L00930 Arthur Schnitzler an Richard Beer-Hofmann, 26. 6. 1899}
\nopagebreak\mylabel{L00930v}
\rehead{ }\normalsize\beginnumbering\briefempfaengerindex{Beer-Hofmann, Richard@\textsc{Beer-Hofmann, Richard}!zzzSchnitzler, Arthur@\emph{von Arthur Schnitzler}!1899-06-261@{26. 6. 1899}|(be}
\toendnotes[C]{\smallbreak\pagebreak[2]}\Standort{YCGL, MSS 31.}
\physDesc{Bildpostkarte, 144 Zeichen
\newline{}Handschrift: Bleistift, deutsche Kurrent
\newline{}Versand: 1) Stempel: »\nobreak{}\oindex{Orahovica@\textbf{Orahovica}, \emph{P.PPLA2}|pwk}Orahovica, 26. \textcolor{gray}{6}. {[}99{]}\nobreak{}«.   2) Stempel: »\nobreak{}\oindex{Seeboden@\textbf{Seeboden}, \emph{A.ADM3}|pwk}Seeboden, 28. {[}6{]}. 9\textcolor{gray}{9}\nobreak{}«. }\pstart{}{\pb}\textsc{Dr. Rich. Beer-Hofmann}\pend{}\pstart{}\textsc{Seeboden am Millstätter}see\oindex{Seeboden@\textbf{Seeboden}, \emph{A.ADM3}|pw}\pend{}\pstart{}\textsc{Villa Platzer}\oindex{Villa Platzer@\textbf{Villa Platzer}, \emph{Gebäude (K.GBD)}|pw}\pend{}\pstart{}\textsc{Kärnthen}\oindex{Kaernten@\textbf{Kärnten}, \emph{A.ADM1}|pw}\pend{}{\bigskip}
\pstart
           \noindent{}\centering{}{\pb}\textcolor{gray}{\textbf{Pozdrav iz Orahovice\oindex{Orahovica@\textbf{Orahovica}, \emph{P.PPLA2}|pw}.}}\pend
           \vspace{1em}
\pstart
           \noindent{}{\pb}Herzlichſte Grüße. Ich hoffe in Wien\oindex{Wien@\textbf{Wien}, \emph{A.ADM2}|pw} einen Brief von Ihnen zu finden. Ihr \spacefill\mbox{Arth}\pend
           \selectlanguage{ngerman}\endnumbering\briefempfaengerindex{Beer-Hofmann, Richard@\textsc{Beer-Hofmann, Richard}!zzzSchnitzler, Arthur@\emph{von Arthur Schnitzler}!1899-06-261@{26. 6. 1899}|)be}\mylabel{L00930h}  \normalsize

\doendnotes{C}
\bigskip
\vfill

\clearpage

\footnotesize

\lohead{\textsc{register}}

% Definiere theindex-Environment komplett neu ohne reledmac
\makeatletter
\renewenvironment{theindex}{%
  \section*{\indexname}%
  \setlength{\parindent}{0pt}%
  \setlength{\parskip}{0pt plus 0.3pt}%
  \let\item\@idxitem
}{%
  \clearpage
}
\makeatother

\IfFileExists{\jobname-pw.ind}{\input{\jobname-pw.ind}}{}

\end{document}

      