%% latex-korrekturansicht-vorspann.tex
%% Vorspann für die Korrekturansicht.
%% Lädt die gemeinsame Datei latex-vorspann.tex mit gesetztem Schalter.

\newif\ifkorrekturansicht
\korrekturansichttrue

\input{../tex-inputs/latex-vorspann}


\section[Karl Kraus an Arthur Schnitzler, 8. 7. 1894]{L00348 Karl Kraus an Arthur Schnitzler, 8. 7. 1894}
\nopagebreak\mylabel{L00348v}
\rehead{ }\normalsize\beginnumbering\briefempfaengerindex{Schnitzler, Arthur@\textsc{Schnitzler, Arthur}!zzzKraus, Karl@\emph{von Karl Kraus}!1894-07-081@{9. 7. 1894}|(be}
\toendnotes[C]{\smallbreak\pagebreak[2]}\Standort{CUL, Schnitzler, B 55.}
\physDesc{Postkarte, 452 Zeichen
\newline{}Handschrift: Bleistift, deutsche Kurrent
\newline{}Versand: 1) Stempel: »\nobreak{}\oindex{Bad Ischl@\textbf{Bad Ischl}, \emph{P.PPL}|pwk}Ischl, 9/7 94, 7–F\nobreak{}«.   2) Stempel: »\nobreak{}\oindex{IX., Alsergrund@\textbf{IX., Alsergrund}, \emph{A.ADM3}|pwk}Wien 9/\textcolor{gray}{3}, 10. 7. 94, 8.V, Beste{[}llt{]}\nobreak{}«. 
\newline{}Schnitzler: mit Bleistift datiert: »9/7 94« }
\buchAbdrucke{\weitereDrucke{\emph{Literatur und Kritik}, Bd. 49, Oktober 1970, S. 521.} }\toendnotes[C]{\smallbreak}\pstart{}{\pb}Herrn\pend{}\pstart{}D\textsuperscript{r} Arthur Schnitzler\pend{}\pstart{}Wien IX.\oindex{IX., Alsergrund@\textbf{IX., Alsergrund}, \emph{A.ADM3}|pw}\pend{}\pstart{}Frankgasse 1\oindex{Frankgasse 1@\textbf{Frankgasse 1}, \emph{Wohngebäude (K.WHS)}|pw}\pend{}{\bigskip}\vspace{1em}
\pstart
           \noindent{}{\pb}Lieber Schnitzler, im »Prager
                  Tagblatt\pwindex{Prager Tagblatt@\emph{Prager Tagblatt}|pw}« vom \uline{Samstag}, 7.{ }ſteht eine (halb günſtige) \label{K_L00348-1v}\edtext{Kritik\pwindex{Maerchen@\emph{Das Märchen}|pwv}}{\lemma{\textnormal{\emph{Kritik}}}\Cendnote{\textnormal{[O. V.]: \emph{Das Märchen}\pwindex{Maerchen@\emph{Das Märchen}|pwk}. In: \emph{Prager Tagblatt}\pwindex{Prager Tagblatt@\emph{Prager Tagblatt}|pwk}, Jg. 18, Nr. 185,
                        7. 7. 1894, S. 8.}}}\label{K_L00348-1} Ihres »Märchen\pwindex{Maerchen. Schauspiel in drei Aufzuegen@\emph{Das Märchen. Schauspiel in drei Aufzügen}|pw}«. Ich wollt’ Ihnen den Ausschnitt ſchicken, erfahre aber
               eben, daſs das Blatt hier subabonniert ist. Seien Sie mir herzlichst gegrüßt!
               Hoffentlich ſehen wir uns bald. Ihr \spacefill\mbox{Kraus,}\pend
           
\pstart
           {[}({]}Ischl, Grazerſtr 133\oindex{Grazer Strasse [Bad Ischl]@\textbf{Grazer Straße [Bad Ischl]}, \emph{Straße (K.STR)}|pw}, Café Walter\oindex{Cafe Walther@\textbf{Café Walther}, \emph{Kaffeehaus (K.KAF)}|pw}, 8. VII.)\pend
           
\pstart
           \label{T_L00348-1v}\edtext{Der kl. \uline{Rosner}\pwindex{Rosner, Karl Peter 05.02.1873 – 06.05.1951@\textsc{Rosner, Karl Peter} (05.02.1873 – 06.05.1951), \emph{Schriftsteller/Schriftstellerin}|pw} fragt mich heute nach Ihrer Adreſſe; er will Ihnen ſeine »Gefühle\pwindex{Gefuehle@\emph{Gefühle}|pw}« ſchicken.}{\lemma{\textnormal{\emph{Der … ſchicken.}}}\Cendnote{\textnormal{quer am rechten Rand}}}\label{T_L00348-1}\pend
           \selectlanguage{ngerman}\endnumbering\briefempfaengerindex{Schnitzler, Arthur@\textsc{Schnitzler, Arthur}!zzzKraus, Karl@\emph{von Karl Kraus}!1894-07-081@{9. 7. 1894}|)be}\mylabel{L00348h}  \normalsize

\doendnotes{C}
\bigskip
\vfill

\clearpage

\footnotesize

\lohead{\textsc{register}}

% Definiere theindex-Environment komplett neu ohne reledmac
\makeatletter
\renewenvironment{theindex}{%
  \section*{\indexname}%
  \setlength{\parindent}{0pt}%
  \setlength{\parskip}{0pt plus 0.3pt}%
  \let\item\@idxitem
}{%
  \clearpage
}
\makeatother

\IfFileExists{\jobname-pw.ind}{\input{\jobname-pw.ind}}{}

\end{document}

      