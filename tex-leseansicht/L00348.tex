%% latex-leseansicht-vorspann.tex
%% Vorspann für die Leseansicht.
%% Lädt die gemeinsame Datei latex-vorspann.tex mit nicht gesetztem Schalter.

\newif\ifkorrekturansicht
\korrekturansichtfalse

\input{../tex-inputs/latex-vorspann}


         
         \newcommand{\erwaehntePersonen}{Personen: Karl Peter Rosner}
         \newcommand{\erwaehnteInstitutionen}{}
         \newcommand{\erwaehnteOrte}{Orte: Bad Ischl, Café Walther, Frankgasse, Grazer Straße, IX., Alsergrund, Wien}
         \newcommand{\erwaehnteWerke}{Werke: Das Märchen, Das Märchen. Schauspiel in drei Aufzügen, Gefühle, Prager Tagblatt}
               \section[Karl Kraus an Arthur Schnitzler, 8. 7. 1894]{ Karl Kraus an Arthur Schnitzler, 8. 7. 1894}\nopagebreak\mylabel{v}\rehead{ }\begin{ledgroupsized}[t]{13cm}\normalsize\beginnumbering \toendnotes[C]{\smallbreak\pagebreak[2]} \Standort{CUL, Schnitzler, B 55.}
\physDesc{Postkarte
\newline{}Handschrift: Bleistift, deutsche Kurrent\newline{}Versand: 1) Stempel: »\nobreak{}\oindex{Bad Ischl@\textbf{Bad Ischl}|pwk}Ischl, 9/7 94, 7–F\nobreak{}«.   2) Stempel: »\nobreak{}\oindex{IX., Alsergrund@\textbf{IX., Alsergrund}|pwk}Wien 9/\textcolor{gray}{3}, 10. 7. 94, 8.V, Beste{[}llt{]}\nobreak{}«. 
\newline{}Schnitzler: mit Bleistift datiert: »9/7 94« }\buchAbdrucke{\weitereDrucke{\emph{Karl Kraus und Arthur Schnitzler. Eine Dokumentation.} Hg. Reinhard Urbach. In: \emph{Literatur und Kritik}, Bd. 49, Oktober 1970, S. 521.} }\toendnotes[C]{\smallbreak}\pstart{}{\pb}Herrn\pend{}\pstart{}D\textsuperscript{r} Arthur Schnitzler\pend{}\pstart{}Wien IX.\oindex{IX., Alsergrund@\textbf{IX., Alsergrund}|pw}\pend{}\pstart{}Frankgasse 1\oindex{Frankgasse@\textbf{Frankgasse}|pw}\pend{}{\bigskip}\pstart
           \noindent{}{\pb}Lieber Schnitzler, im »Prager
                        Tagblatt\pwindex{?? Werk@Nicht ermittelte Verfasserinnen und Verfasser!Prager Tagblatt1876 – 1939@\emph{Prager Tagblatt} {[}1876 – 1939{]}|pw}« vom \uline{Samstag}, 7.{ }ſteht eine (halb günſtige) \label{K_L00348_1v}\edtext{Kritik\pwindex{?? Werk@Nicht ermittelte Verfasserinnen und Verfasser!Maerchen07. 07. 1894@\emph{Das Märchen} {[}07. 07. 1894{]}|pwv}}{\lemma{\textnormal{\emph{Kritik}}}\Cendnote{\textnormal{[O. V.:] \emph{Das Märchen}\pwindex{?? Werk@Nicht ermittelte Verfasserinnen und Verfasser!Maerchen07. 07. 1894@\emph{Das Märchen} {[}07. 07. 1894{]}|pwk}. In: \emph{Prager Tagblatt}\pwindex{?? Werk@Nicht ermittelte Verfasserinnen und Verfasser!Prager Tagblatt1876 – 1939@\emph{Prager Tagblatt} {[}1876 – 1939{]}|pwk}, Jg. 18, Nr. 185,
                                7. 7. 1894, S. 8.}}}\label{K_L00348_1h} Ihres »Märchen\pwindex{Schnitzler, Arthur 15.05.1862 – 21.10.1931@\textsc{Schnitzler, Arthur} (15.05.1862 – 21.10.1931), \emph{Schriftsteller, Mediziner}!Maerchen. Schauspiel in drei Aufzuegen1893-12-01@\strich\emph{Das Märchen. Schauspiel in drei Aufzügen} {[}1893-12-01{]}|pw}«. Ich wollt’ Ihnen den Ausschnitt ſchicken, erfahre
                    aber eben, daſs das Blatt hier subabonniert ist. Seien Sie mir herzlichst
                    gegrüßt! Hoffentlich ſehen wir uns bald. Ihr \spacefill\mbox{Kraus,}\pend
           \pstart
           {[}({]}Ischl, Grazerſtr 133\oindex{Grazer Strasse@\textbf{Grazer Straße}|pw}, Café Walter\oindex{Cafe Walther@\textbf{Café Walther}|pw}, 8. VII.)\pend
           \pstart
           \label{T_L00348_1v}\edtext{Der kl. \uline{Rosner}\pwindex{Rosner, Karl Peter 05.02.1873 – 06.05.1951@\textsc{Rosner, Karl Peter} (05.02.1873 – 06.05.1951), \emph{Schriftsteller}|pw} fragt mich heute nach Ihrer Adreſſe; er will Ihnen ſeine »Gefühle\pwindex{Rosner, Karl Peter 05.02.1873 – 06.05.1951@\textsc{Rosner, Karl Peter} (05.02.1873 – 06.05.1951), \emph{Schriftsteller}!Gefuehle1894@\strich\emph{Gefühle} {[}1894{]}|pw}« ſchicken.}{\lemma{\textnormal{\emph{Der … ſchicken.}}}\Cendnote{\textnormal{quer am rechten Rand}}}\label{T_L00348_1h}\pend
           
         
         \endnumbering\mylabel{h}\end{ledgroupsized}  \newcommand{\dateiname}{L00348}\newcommand{\titel}{Karl Kraus an Arthur Schnitzler, 8. 7. 1894}\newcommand{\editorInnen}{Martin Anton Müller und Gerd-Hermann Susen}%% latex-leseansicht-abspann.tex
%% Abspann für die Leseansicht.
%% Der Schalter \ifkorrekturansicht ist bereits durch den Vorspann gesetzt.

%% latex-abspann.tex
%% Gemeinsamer Abspann für Korrekturansicht und Leseansicht.
%% Setzt den Schalter \ifkorrekturansicht voraus (gesetzt in den
%% einbindenden Dateien latex-korrekturansicht-abspann.tex bzw.
%% latex-leseansicht-abspann.tex).
%% ---------------------------------------------------------------

\normalsize

% Das esempio-Environment wird nur in der Leseansicht benötigt
\ifkorrekturansicht\else
\newenvironment{esempio}[3]%
{
    \vspace{1.5ex}
    \rlap{\underline{#1}}
    \par
    \setlength{\parindent}{0cm}
    \nopagebreak
    \leftskip=#2cm
    \rightskip=#3cm
}
{
    \par
}
\fi

\doendnotes{C}
\bigskip
\vfill

\clearpage

\footnotesize

\ifkorrekturansicht
  \lohead{\textsc{register}}
\fi

% theindex-Environment neu definieren ohne reledmac
\makeatletter
\renewenvironment{theindex}{%
  \ifkorrekturansicht
    \section*{\indexname}%
  \else
    \subsubsection*{Index der erwähnten Entitäten}%
  \fi
  \setlength{\parindent}{0pt}%
  \setlength{\parskip}{0pt plus 0.3pt}%
  \let\item\@idxitem
}{%
  \ifkorrekturansicht\clearpage\fi
}
\makeatother

\IfFileExists{\jobname-pw.ind}{\input{\jobname-pw.ind}}{}

% Quellenangabe nur in der Leseansicht
\ifkorrekturansicht\else
% Fallback-Definitionen, falls die .tex-Datei \titel etc. nicht gesetzt hat
\providecommand{\titel}{}
\providecommand{\editorInnen}{}
\providecommand{\dateiname}{\jobname}

\vspace{3cm}

\vfill

\footnotesize
\textsc{Quelle}: \titel. Herausgegeben von {\editorInnen}. In: \emph{Arthur Schnitzler: Briefwechsel mit Autorinnen und Autoren}.
 Digitale Edition, https://schnitzler-briefe.acdh.oeaw.ac.at/{\dateiname}.html (Stand \today)
\fi

\end{document}


      