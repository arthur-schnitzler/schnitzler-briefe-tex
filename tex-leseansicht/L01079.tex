%% latex-korrekturansicht-vorspann.tex
%% Vorspann für die Korrekturansicht.
%% Lädt die gemeinsame Datei latex-vorspann.tex mit gesetztem Schalter.

\newif\ifkorrekturansicht
\korrekturansichttrue

\input{../tex-inputs/latex-vorspann}


\section[Arthur Schnitzler an Richard Beer-Hofmann, 19. 10. 1900]{L01079 Arthur Schnitzler an Richard Beer-Hofmann, 19. 10. 1900}
\nopagebreak\mylabel{L01079v}
\rehead{ }\normalsize\beginnumbering\briefempfaengerindex{Beer-Hofmann, Richard@\textsc{Beer-Hofmann, Richard}!zzzSchnitzler, Arthur@\emph{von Arthur Schnitzler}!1900-10-191@{19. 10. 1900}|(be}
\toendnotes[C]{\smallbreak\pagebreak[2]}\Standort{YCGL, MSS 31.}
\physDesc{Telegramm, 179 Zeichen
\newline{}maschinell}\pstart{}{\pb}richard beerhofmann\pend{}\pstart{}badenwien julienhof\oindex{Julienhof@\textbf{Julienhof}, \emph{Hotel (K.HTL)}|pw}\pend{}{\bigskip}\vspace{1em}
\pstart
           {\pb}+ ''''''''' de wien\oindex{Wien@\textbf{Wien}, \emph{A.ADM2}|pw} 72 + 679 18 10{ }9m – \pend
           \vspace{0.5em}
\pstart
           = ich komme heute nachmittags nach vier aber noch nicht definitiv sondern erkundigen
               herzlichst \spacefill\mbox{arthur +''}\pend
           \selectlanguage{ngerman}\endnumbering\briefempfaengerindex{Beer-Hofmann, Richard@\textsc{Beer-Hofmann, Richard}!zzzSchnitzler, Arthur@\emph{von Arthur Schnitzler}!1900-10-191@{19. 10. 1900}|)be}\mylabel{L01079h}  \normalsize

\doendnotes{C}
\bigskip
\vfill

\clearpage

\footnotesize

\lohead{\textsc{register}}

% Definiere theindex-Environment komplett neu ohne reledmac
\makeatletter
\renewenvironment{theindex}{%
  \section*{\indexname}%
  \setlength{\parindent}{0pt}%
  \setlength{\parskip}{0pt plus 0.3pt}%
  \let\item\@idxitem
}{%
  \clearpage
}
\makeatother

\IfFileExists{\jobname-pw.ind}{\input{\jobname-pw.ind}}{}

\end{document}

      