%% latex-leseansicht-vorspann.tex
%% Vorspann für die Leseansicht.
%% Lädt die gemeinsame Datei latex-vorspann.tex mit nicht gesetztem Schalter.

\newif\ifkorrekturansicht
\korrekturansichtfalse

\input{../tex-inputs/latex-vorspann}

\begin{center}
            \textcolor{red}{ENTWURF, NICHT FERTIG KORRIGIERT}
                      \end{center}
            
         
         \renewcommand{\erwaehntePersonen}{Personen:  ?? [Partner von Theodore Rottenberg, Ende 1902/Anfang 1903], Theodore Rottenberg, Olga Schnitzler}
         \renewcommand{\erwaehnteOrte}{Orte: Berlin, Dessauer Straße, Palasthotel Berlin, Wien}
         \renewcommand{\erwaehnteWerke}{}
               \section[ Paul Goldmann an Arthur Schnitzler, 17. 2. {[}1903{]}]{ Paul Goldmann an Arthur Schnitzler, 17. 2. {[}1903{]}}\nopagebreak\mylabel{v}\rehead{ }\begin{ledgroupsized}[t]{13cm}\normalsize\beginnumbering \toendnotes[C]{\smallbreak\pagebreak[2]} \Standort{DLA, A:Schnitzler, HS.NZ85.1.3173.}
\physDesc{Brief, 1 Blatt, 4 Seiten, 1599 Zeichen
\newline{}Handschrift: blaue Tinte, deutsche Kurrent
\newline{}Schnitzler: 1) mit Bleistift das Jahr »903« vermerkt  2) mit rotem Buntstift eine Unterstreichung}\toendnotes[C]{\smallbreak}\pstart
           \noindent{}\raggedleft{}{\pb}\textcolor{gray}{\textbf{DESSAUERSTRASSE 19\oindex{Dessauer Strasse@\textbf{Dessauer Straße}|pw}}}\pend
           \pstart
           Berlin\oindex{Berlin@\textbf{Berlin}|pw}, 17. Februar.\pend
           \pstart\center{}Mein lieber Freund,\pend\pstart
           Ich freue mich unendlich, Dich \label{K_L03363-1v}\edtext{bald
               hier zu ſehen}{\lemma{\textnormal{\emph{bald
               hier zu ſehen}}}\Cendnote{\textnormal{Schnitzler\pwindex{Schnitzler, Arthur 15.05.1862 – 21.10.1931@\textsc{Schnitzler, Arthur} (15.05.1862 – 21.10.1931), \emph{Schriftsteller, Mediziner}|pwk} war von 22. 2. 1903 bis 9. 3. 1903 in Berlin\oindex{Berlin@\textbf{Berlin}|pwk}. Goldmann\pwindex{Goldmann, Paul 31.01.1865 – 25.09.1935@\textsc{Goldmann, Paul} (31.01.1865 – 25.09.1935), \emph{Schriftsteller, Journalist}|pwk} traf er mehrfach, zumindest am 22. 2. 1903, 24. 2. 1903, 25. 2. 1903, 3. 3. 1903, 4. 3. 1903, 7. 3. 1903 und am 9. 3. 1903.}}}\label{K_L03363-1h},
               und werde Dich, wenn ich nichts Gegentheiliges höre, am Sonntag{ }Vormittag gegen 12 Uhr im \label{K_L03363-2v}\edtext{Palaſthotel\oindex{Palasthotel Berlin@\textbf{Palasthotel Berlin}|pw}}{\lemma{\textnormal{\emph{Palaſthotel}}}\Cendnote{\textnormal{Schnitzler\pwindex{Schnitzler, Arthur 15.05.1862 – 21.10.1931@\textsc{Schnitzler, Arthur} (15.05.1862 – 21.10.1931), \emph{Schriftsteller, Mediziner}|pwk}s Unterkunft}}}\label{K_L03363-2h} aufſuchen. Du
               kannſt Dir gar nicht denken, wie ſehr ich mich danach ſehne, mit Dir zu beſprechen,
                  \label{K_L03363-3v}\edtext{was mein Herz bedrückt}{\lemma{\textnormal{\emph{was mein Herz bedrückt}}}\Cendnote{\textnormal{Bezug auf Theodore Rottenberg\pwindex{Rottenberg, Theodore 1875-09-07 – 1945-04-05@\textsc{Rottenberg, Theodore} (1875-09-07 – 1945-04-05)|pwk}s Trennung von ihm, über die sie dann auch sprachen
                     (vgl. A. S.: \emph{Tagebuch}, 22. 2. 1903).}}}\label{K_L03363-3h}.
               Freilich, viel wirſt auch Du mir nicht helfen können. {\pb}Denn Du kannſt mir ja auch nicht das Verlorene
               wiederbringen; und das allein wäre die Heilung. Aber jede Hoffnung iſt vergeblich.
               Ich bin aus dem Leben dieſer Frau\pwindex{Rottenberg, Theodore 1875-09-07 – 1945-04-05@\textsc{Rottenberg, Theodore} (1875-09-07 – 1945-04-05)|pwv}, \strikeout{die noch} für die ich vor wenig Monaten
               noch Alles bedeutet habe, vollkommen ausgeſtrichen. Sie hat ihr Leben ganz auf den
                  Andern\pwindex{?? [Partner von Theodore Rottenberg, Ende 1902/Anfang 1903] @\textsc{?? [Partner von Theodore Rottenberg, Ende 1902/Anfang 1903]}|pwv} übertragen, und
               ich höre nur, wie glücklich ſie mit ihm iſt. Ich ſelbſt aber bekomme nicht einmal
               mehr ein Lebenszeichen. Alle meine Briefe, – flehende, reuige, verzweifelte Briefe –
                  {\pb}bleiben ohne Antwort und ſelbſt die Möglichkeit,
               indirekt Nachrichten\strikeout{\textcolor{gray}{×}} von ihr zu erhalten, ſchneidet ſie\pwindex{Rottenberg, Theodore 1875-09-07 – 1945-04-05@\textsc{Rottenberg, Theodore} (1875-09-07 – 1945-04-05)|pwv} mir ab. Ich verzehre mich in Sehnſucht. Ich warte – und
               ich warte vergebens. Jeder Tag bringt sie dem Andern\pwindex{?? [Partner von Theodore Rottenberg, Ende 1902/Anfang 1903] @\textsc{?? [Partner von Theodore Rottenberg, Ende 1902/Anfang 1903]}|pwv}{ }\strikeout{nä} näher und treibt ſie weiter von mir fort. Und ich
               muß mir ſagen, daß ich ſelbſt an Allem ſchuld \introOben{}bin\introOben{}, daß ich
               die zärtlichſte und hingebendſte Geliebte in einer finſteren Laune fortgeſtoßen habe,
               nicht ahnend, {\pb}welch’ koſtbaren Schatz ich beſaß,
               was ich jetzt erſt, zu ſpät, eingeſehen habe. Ein Wahnſinniger war ich, – ein
               verblendeter Thor – ein unerfahrener dummer Junge trotz meiner 38 Jahre!{\dotsfour}\pend
           \pstart
           Reiſe glücklich nach Berlin\oindex{Berlin@\textbf{Berlin}|pw}, grüße Olga\pwindex{Schnitzler, Olga 17.01.1882 – 13.01.1970@\textsc{Schnitzler, Olga} (17.01.1882 – 13.01.1970), \emph{Schauspielerin, Sängerin}|pw} vielmals (auf deren \label{K_L03363-4v}\edtext{Ankunft}{\lemma{\textnormal{\emph{Ankunft}}}\Cendnote{\textnormal{Olga Gussmann\pwindex{Schnitzler, Olga 17.01.1882 – 13.01.1970@\textsc{Schnitzler, Olga} (17.01.1882 – 13.01.1970), \emph{Schauspielerin, Sängerin}|pwk} kam am 4. 3. 1903 in Berlin\oindex{Berlin@\textbf{Berlin}|pwk} an und reiste am 9. 3. 1903 gemeinsam
                  mit Schnitzler\pwindex{Schnitzler, Arthur 15.05.1862 – 21.10.1931@\textsc{Schnitzler, Arthur} (15.05.1862 – 21.10.1931), \emph{Schriftsteller, Mediziner}|pwk} zurück nach Wien\oindex{Wien@\textbf{Wien}|pwk}.}}}\label{K_L03363-4h} ich mich auch ſchon ſehr freue) und ſei ſelbſt
               von Herzen gegrüßt von {\\[\baselineskip]}Deinem getreuen {\\[\baselineskip]}\spacefill\mbox{Paul Goldmann}\pend
           \leftskip=0em{}
         
         \endnumbering\mylabel{h}\end{ledgroupsized}\begin{anhang}\end{anhang}\newcommand{\dateiname}{L03363}\newcommand{\titel}{Paul Goldmann an Arthur Schnitzler, 17. 2. [1903]}\newcommand{\editorInnen}{Martin Anton Müller und Laura Untner}%% latex-leseansicht-abspann.tex
%% Abspann für die Leseansicht.
%% Der Schalter \ifkorrekturansicht ist bereits durch den Vorspann gesetzt.

%% latex-abspann.tex
%% Gemeinsamer Abspann für Korrekturansicht und Leseansicht.
%% Setzt den Schalter \ifkorrekturansicht voraus (gesetzt in den
%% einbindenden Dateien latex-korrekturansicht-abspann.tex bzw.
%% latex-leseansicht-abspann.tex).
%% ---------------------------------------------------------------

\normalsize

% Das esempio-Environment wird nur in der Leseansicht benötigt
\ifkorrekturansicht\else
\newenvironment{esempio}[3]%
{
    \vspace{1.5ex}
    \rlap{\underline{#1}}
    \par
    \setlength{\parindent}{0cm}
    \nopagebreak
    \leftskip=#2cm
    \rightskip=#3cm
}
{
    \par
}
\fi

\doendnotes{C}
\bigskip
\vfill

\clearpage

\footnotesize

\ifkorrekturansicht
  \lohead{\textsc{register}}
\fi

% theindex-Environment neu definieren ohne reledmac
\makeatletter
\renewenvironment{theindex}{%
  \ifkorrekturansicht
    \section*{\indexname}%
  \else
    \subsubsection*{Index der erwähnten Entitäten}%
  \fi
  \setlength{\parindent}{0pt}%
  \setlength{\parskip}{0pt plus 0.3pt}%
  \let\item\@idxitem
}{%
  \ifkorrekturansicht\clearpage\fi
}
\makeatother

\IfFileExists{\jobname-pw.ind}{\input{\jobname-pw.ind}}{}

% Quellenangabe nur in der Leseansicht
\ifkorrekturansicht\else
% Fallback-Definitionen, falls die .tex-Datei \titel etc. nicht gesetzt hat
\providecommand{\titel}{}
\providecommand{\editorInnen}{}
\providecommand{\dateiname}{\jobname}

\vspace{3cm}

\vfill

\footnotesize
\textsc{Quelle}: \titel. Herausgegeben von {\editorInnen}. In: \emph{Arthur Schnitzler: Briefwechsel mit Autorinnen und Autoren}.
 Digitale Edition, https://schnitzler-briefe.acdh.oeaw.ac.at/{\dateiname}.html (Stand \today)
\fi

\end{document}


      