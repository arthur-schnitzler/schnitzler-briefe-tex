%% latex-korrekturansicht-vorspann.tex
%% Vorspann für die Korrekturansicht.
%% Lädt die gemeinsame Datei latex-vorspann.tex mit gesetztem Schalter.

\newif\ifkorrekturansicht
\korrekturansichttrue

\input{../tex-inputs/latex-vorspann}


\section[Arthur Schnitzler: Widmungsexemplar Ruf des Lebens für Hugo von Hofmannsthal, 21. 3. 1909]{L01833 Arthur Schnitzler: Widmungsexemplar Ruf des Lebens für Hugo von
               Hofmannsthal, 21. 3. 1909}
\nopagebreak\mylabel{L01833v}
\rehead{ }\normalsize\beginnumbering\briefempfaengerindex{Hofmannsthal, Hugo von@\textsc{Hofmannsthal, Hugo von}!zzzSchnitzler, Arthur@\emph{von Arthur Schnitzler}!1909-03-211@{21. 3. 1909}|(be}
\toendnotes[C]{\smallbreak\pagebreak[2]}\Standort{FDH, FDH 3233.}
\physDesc{Widmung am Schmutztitel, 34 Zeichen
\newline{}Handschrift: schwarze Tinte, deutsche Kurrent
\newline{}Ordnung: mit Bleistift von unbekannter Hand beschriftet: »HvH-S.
                                    CI, 54« }
\buchAbdrucke{\weitereDrucke{Hugo von Hofmannsthal: \emph{Bibliothek}. Frankfurt am Main: \emph{S. Fischer} 2011, S. 606.} }\pstart \spacefill\mbox{{\pb}Arthur Schnitzler}\pend{}
\pstart
           Wien\oindex{Wien@\textbf{Wien}, \emph{A.ADM2}|pw}{ }21. 3. 09.\pend
           \selectlanguage{ngerman}\vspace{1em}{\vspace{1\baselineskip}}
\pstart
           {\pb}\textcolor{gray}{\textbf{Der Ruf des Lebens\pwindex{Ruf des Lebens. Schauspiel in drei Akten@\emph{Der Ruf des Lebens. Schauspiel in drei Akten}|pw}}}\pend
           
\pstart
           \textcolor{gray}{\textbf{Schauſpiel in drei Akten von Arthur Schnitzler}}\pend
           {\vspace{1\baselineskip}}
\pstart
           \textcolor{gray}{\textbf{Als Bühnen-Manuſkript gedruckt und vervielfältigt. Das Recht der
                  Aufführung ist nur von S. Fiſcher, Verlag\orgindex{S. Fischer Verlag@S. Fischer Verlag|pw}
                  (Theaterabteilung) in Berlin W., Bülowſtr. 91\oindex{Buelowstrasse@\textbf{Bülowstraße}, \emph{Straße (K.STR)}|pw}
                  zu erwerben.}}\pend
           \selectlanguage{ngerman}\endnumbering\briefempfaengerindex{Hofmannsthal, Hugo von@\textsc{Hofmannsthal, Hugo von}!zzzSchnitzler, Arthur@\emph{von Arthur Schnitzler}!1909-03-211@{21. 3. 1909}|)be}\mylabel{L01833h}  \normalsize

\doendnotes{C}
\bigskip
\vfill

\clearpage

\footnotesize

\lohead{\textsc{register}}

% Definiere theindex-Environment komplett neu ohne reledmac
\makeatletter
\renewenvironment{theindex}{%
  \section*{\indexname}%
  \setlength{\parindent}{0pt}%
  \setlength{\parskip}{0pt plus 0.3pt}%
  \let\item\@idxitem
}{%
  \clearpage
}
\makeatother

\IfFileExists{\jobname-pw.ind}{\input{\jobname-pw.ind}}{}

\end{document}

      