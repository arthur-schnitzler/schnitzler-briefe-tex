%% latex-korrekturansicht-vorspann.tex
%% Vorspann für die Korrekturansicht.
%% Lädt die gemeinsame Datei latex-vorspann.tex mit gesetztem Schalter.

\newif\ifkorrekturansicht
\korrekturansichttrue

\input{../tex-inputs/latex-vorspann}


\section[ Felix Salten an Arthur Schnitzler, 23. 9. 1898]{L03284 Felix Salten an Arthur Schnitzler, 23. 9. 1898}
\nopagebreak\mylabel{L03284v}
\rehead{ }\normalsize\beginnumbering\briefempfaengerindex{Schnitzler, Arthur@\textsc{Schnitzler, Arthur}!zzzSalten, Felix@\emph{von Felix Salten}!1898-09-231@{23. 9. 1898}|(be}
\toendnotes[C]{\smallbreak\pagebreak[2]}\Standort{CUL, Schnitzler, B 89, A 2.}
\physDesc{Brief, 1 Blatt, 1 Seite, 451 Zeichen
\newline{}Handschrift: schwarze Tinte, lateinische Kurrent
\newline{}Ordnung: mit Bleistift von unbekannter Hand nummeriert: »108« }\toendnotes[C]{\smallbreak}
\pstart
           \raggedleft{}{\pb}Hietzing\oindex{XIII., Hietzing@\textbf{XIII., Hietzing}, \emph{A.ADM3}|pw}, 23./IX. 98\pend
           \vspace{0.5em}
\pstart
           Lieber Arthur, von Frau Schmittlein\pwindex{Schmittlein, Ferdinande 1856-03-26 – 15.07.1915@\textsc{Schmittlein, Ferdinande} (1856-03-26 – 15.07.1915), \emph{Schauspieler/Schauspielerin}|pw} höre ich, dass die \label{K_L03284-1v}\edtext{Rollen zum »Vermächtnis\pwindex{Vermaechtnis. Schauspiel in drei Akten@\emph{Das Vermächtnis. Schauspiel in drei Akten}|pw}«}{\lemma{\textnormal{\emph{Rollen zum »Vermächtnis«}}}\Cendnote{\textnormal{Die Premiere fand am 30. 11. 1898 am \emph{Burgtheater}\orgindex{Burgtheater@Burgtheater|pwk} statt.}}}\label{K_L03284-1} schon eingetheilt
               sind. Vielleicht theilen Sie mir, bitte, mit, ob \label{K_L03284-2v}\edtext{Frl. Metzl\pwindex{Salten, Ottilie 07.03.1868 – 22.06.1942@\textsc{Salten, Ottilie} (07.03.1868 – 22.06.1942), \emph{Schauspieler/Schauspielerin}|pw}}{\lemma{\textnormal{\emph{Frl. Metzl}}}\Cendnote{\textnormal{Ottilie Metzl\pwindex{Salten, Ottilie 07.03.1868 – 22.06.1942@\textsc{Salten, Ottilie} (07.03.1868 – 22.06.1942), \emph{Schauspieler/Schauspielerin}|pwk} machte sich Hoffnung auf die Kinderrolle des Lulu\pwindex{Vermaechtnis. Schauspiel in drei Akten@\emph{Das Vermächtnis. Schauspiel in drei Akten}|pwkv}, die
                  ihr zwischenzeitlich aberkannt wurde, die sie aber letztlich doch spielte, 
                   vgl. Arthur Schnitzler an Felix Salten, 24. 9. 1898.}}}\label{K_L03284-2} nichts bekommt. Ich möchte
               ihr doch gerne etwas Tröstendes und Beruhigendes sagen, ehe sie’s erfährt. Denn ich
               habe ihr nach Ihrer Zusage sehr viel Hoffnung auf die Rolle gemacht, so dass \strikeout{es} sie es diesmal besonders schmerzlich empfinden wird,
               übergangen zu werden.\pend
           
\pstart
           Herzlichst {\\[\baselineskip]}Ihr {\\[\baselineskip]}\spacefill\mbox{Salten}\pend
           \leftskip=0em{}\selectlanguage{ngerman}\endnumbering\briefempfaengerindex{Schnitzler, Arthur@\textsc{Schnitzler, Arthur}!zzzSalten, Felix@\emph{von Felix Salten}!1898-09-231@{23. 9. 1898}|)be}\mylabel{L03284h}  \normalsize

\doendnotes{C}
\bigskip
\vfill

\clearpage

\footnotesize

\lohead{\textsc{register}}

% Definiere theindex-Environment komplett neu ohne reledmac
\makeatletter
\renewenvironment{theindex}{%
  \section*{\indexname}%
  \setlength{\parindent}{0pt}%
  \setlength{\parskip}{0pt plus 0.3pt}%
  \let\item\@idxitem
}{%
  \clearpage
}
\makeatother

\IfFileExists{\jobname-pw.ind}{\input{\jobname-pw.ind}}{}

\end{document}

      