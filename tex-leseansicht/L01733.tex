%% latex-korrekturansicht-vorspann.tex
%% Vorspann für die Korrekturansicht.
%% Lädt die gemeinsame Datei latex-vorspann.tex mit gesetztem Schalter.

\newif\ifkorrekturansicht
\korrekturansichttrue

\input{../tex-inputs/latex-vorspann}


\section[Hugo von Hofmannsthal an Arthur Schnitzler, 16. {[}11. 1907?{]}]{L01733 Hugo von Hofmannsthal an Arthur Schnitzler, 16. {[}11. 1907?{]}}
\nopagebreak\mylabel{L01733v}
\rehead{ }\normalsize\beginnumbering\briefempfaengerindex{Schnitzler, Arthur@\textsc{Schnitzler, Arthur}!zzzHofmannsthal, Hugo von@\emph{von Hugo von Hofmannsthal}!1907-11-162@{16. {[}11. 1907?{]}}|(be}
\toendnotes[C]{\smallbreak\pagebreak[2]}\Standort{CUL, Schnitzler, B 43.}
\physDesc{Postkarte, 153 Zeichen
\newline{}Handschrift: 1) schwarze Tinte, deutsche Kurrent\hspace{1em}2) schwarze Tinte, lateinische Kurrent (\noindent{}Adresse)\hspace{1em}
\newline{}Versand: Stempel: »\nobreak{}\oindex{Rodaun@\textbf{Rodaun}, \emph{A.ADM4}|pwk}Rodaun, 16. X\textcolor{gray}{I}. 0{[}7{]}, 12\nobreak{}«.  
\newline{}Schnitzler: mit Bleistift datiert: »16/10 907« 
\newline{}Ordnung: 1) mit Bleistift von unbekannter Hand nummeriert: »285«  2) mit Bleistift von unbekannter Hand nummeriert: »285«}
\buchAbdrucke{\weitereDrucke{Hugo von Hofmannsthal, Arthur Schnitzler: \emph{Briefwechsel}. Frankfurt am Main: \emph{S. Fischer} 1964, S. 232.} }\toendnotes[C]{\smallbreak}\pstart{}{\pb}Herrn D\textsuperscript{r} Arthur Schnitzler\pend{}\pstart{}Wien\oindex{Wien@\textbf{Wien}, \emph{A.ADM2}|pw}\pend{}\pstart{}XVII Spöttelgasse 7\oindex{Edmund-Weiss-Gasse 7@\textbf{Edmund-Weiß-Gasse 7}, \emph{Wohngebäude (K.WHS)}|pw}\pend{}\pstart{}neben Türkenschanzstrasse\oindex{Tuerkenschanzstrasse@\textbf{Türkenschanzstraße}, \emph{Straße (K.STR)}|pw}.
               \pend{}{\bigskip}\vspace{1em}
\pstart
           \noindent{}{\pb}Alſo wir ko{\geminationm}en ganz beſtimmt \label{K_L01733-1v}\edtext{Montag}{\lemma{\textnormal{\emph{Montag}}}\Cendnote{\textnormal{Die
                  handschriftliche Datierung Schnitzlers dürfte auf einer falschen Entzifferung des
                  Stempels beruhen. Nachdem aber die angesprochenen Details sich nicht mit den
                  sonstigen Dokumenten in Einklang bringen lassen, ist ein kleiner Punkt beim
                  Stempel als Überrest eines »I« zu deuten und die Karte in den November zu
                  verlegen.}}}\label{K_L01733-1} ſchon gegen \label{K_L01733-2v}\edtext{¾ 7}{\lemma{\textnormal{\emph{¾ 7}}}\Cendnote{\textnormal{18 Uhr 45}}}\label{K_L01733-2}. \label{K_L01733-3v}\edtext{Dienstag}{\lemma{\textnormal{\emph{Dienstag}}}\Cendnote{\textnormal{Er reiste am Mittwoch, den
                     20. 11. 1907 zuerst nach Dresden\oindex{Dresden@\textbf{Dresden}, \emph{P.PPLA}|pwk} und
                  dann, nach drei Tagen, weiter nach Berlin\oindex{Berlin@\textbf{Berlin}, \emph{P.PPLC}|pwk} und
                     Weimar\oindex{Weimar@\textbf{Weimar}, \emph{A.ADM3}|pwk}. Am 17. 12. 1907 kehrte er
                  zurück.}}}\label{K_L01733-3} reiſe ich.\pend
           
\pstart
           Herzlich{\\[\baselineskip]}\spacefill\mbox{Hugo.}\pend
           \leftskip=0em{}\selectlanguage{ngerman}\endnumbering\briefempfaengerindex{Schnitzler, Arthur@\textsc{Schnitzler, Arthur}!zzzHofmannsthal, Hugo von@\emph{von Hugo von Hofmannsthal}!1907-11-162@{16. {[}11. 1907?{]}}|)be}\mylabel{L01733h}  \normalsize

\doendnotes{C}
\bigskip
\vfill

\clearpage

\footnotesize

\lohead{\textsc{register}}

% Definiere theindex-Environment komplett neu ohne reledmac
\makeatletter
\renewenvironment{theindex}{%
  \section*{\indexname}%
  \setlength{\parindent}{0pt}%
  \setlength{\parskip}{0pt plus 0.3pt}%
  \let\item\@idxitem
}{%
  \clearpage
}
\makeatother

\IfFileExists{\jobname-pw.ind}{\input{\jobname-pw.ind}}{}

\end{document}

      