%% latex-leseansicht-vorspann.tex
%% Vorspann für die Leseansicht.
%% Lädt die gemeinsame Datei latex-vorspann.tex mit nicht gesetztem Schalter.

\newif\ifkorrekturansicht
\korrekturansichtfalse

\input{../tex-inputs/latex-vorspann}


         
         \renewcommand{\erwaehntePersonen}{Personen: Hugo von Hofmannsthal}
         \renewcommand{\erwaehnteOrte}{Orte: Berlin, Dresden, Edmund-Weiß-Gasse 7, Rodaun, Türkenschanzstraße, Weimar, Wien}
         \renewcommand{\erwaehnteWerke}{}
               \section[Hugo von Hofmannsthal an Arthur Schnitzler, 16. {[}11. 1907?{]}]{ Hugo von Hofmannsthal an Arthur Schnitzler, 16. {[}11. 1907?{]}}\nopagebreak\mylabel{v}\rehead{ }\begin{ledgroupsized}[t]{13cm}\normalsize\beginnumbering\briefempfaengerindex{Schnitzler, Arthur@\textsc{Schnitzler, Arthur}!zzzHofmannsthal, Hugo von@\emph{von Hugo von Hofmannsthal}!1907-11-162@{16. {[}11. 1907?{]}}|(be} \toendnotes[C]{\smallbreak\pagebreak[2]} \Standort{CUL, Schnitzler, B 43.}
\physDesc{Postkarte, 153 Zeichen
\newline{}Handschrift: 1) schwarze Tinte, deutsche Kurrent\hspace{1em}2) schwarze Tinte, lateinische Kurrent (\noindent{}Adresse)\hspace{1em}
\newline{}Versand: Stempel: »\nobreak{}\oindex{Rodaun@\textbf{Rodaun}|pwk}Rodaun, 16. X\textcolor{gray}{I}. 0{[}7{]}, 12\nobreak{}«.  
\newline{}Schnitzler: mit Bleistift datiert: »16/10 907« 
\newline{}Ordnung: 1) mit Bleistift von unbekannter Hand nummeriert:
                                    »285«  2) mit Bleistift von unbekannter Hand nummeriert:
                                    »285«}\buchAbdrucke{\weitereDrucke{Hugo von Hofmannsthal, Arthur Schnitzler: \emph{Briefwechsel}. Hg. Therese Nickl und Heinrich Schnitzler. Frankfurt am Main: \emph{S. Fischer} 1964, S. 232.} }\toendnotes[C]{\smallbreak}\pstart{}{\pb}Herrn D\textsuperscript{r} Arthur Schnitzler\pend{}\pstart{}Wien\oindex{Wien@\textbf{Wien}|pw}\pend{}\pstart{}XVII Spöttelgasse 7\oindex{Edmund-Weiss-Gasse 7@\textbf{Edmund-Weiß-Gasse 7}|pw}\pend{}\pstart{}neben Türkenschanzstrasse\oindex{Tuerkenschanzstrasse@\textbf{Türkenschanzstraße}|pw}.
               \pend{}{\bigskip}\pstart
           \noindent{}{\pb}Alſo wir ko{\geminationm}en ganz beſtimmt \label{K_L01733-1v}\edtext{Montag}{\lemma{\textnormal{\emph{Montag}}}\Cendnote{\textnormal{Die
                  handschriftliche Datierung Schnitzlers dürfte auf einer falschen Entzifferung des
                  Stempels beruhen. Nachdem aber die angesprochenen Details sich nicht mit den
                  sonstigen Dokumenten ein Einklang bringen lassen, ist ein kleiner Punkt beim
                  Stempel als Überrest eines »I« zu deuten und die Karte in den November zu
                  verlegen.}}}\label{K_L01733-1h} ſchon gegen \label{K_L01733-2v}\edtext{¾ 7}{\lemma{\textnormal{\emph{¾ 7}}}\Cendnote{\textnormal{18 Uhr 45}}}\label{K_L01733-2h}. \label{K_L01733-3v}\edtext{Dienstag}{\lemma{\textnormal{\emph{Dienstag}}}\Cendnote{\textnormal{Er reist am Mittwoch, den
                     20. 11. 1907 zuerst nach Dresden\oindex{Dresden@\textbf{Dresden}|pwk} und
                  dann, nach drei Tagen, weiter nach Berlin\oindex{Berlin@\textbf{Berlin}|pwk} und
                     Weimar\oindex{Weimar@\textbf{Weimar}|pwk}. Am 17. 12. 1907 ist er
                  zurück.}}}\label{K_L01733-3h} reiſe ich.\pend
           \pstart
           Herzlich{\\[\baselineskip]}\spacefill\mbox{Hugo.}\pend
           \leftskip=0em{}
         
         \endnumbering\mylabel{h}\end{ledgroupsized}  \newcommand{\dateiname}{L01733}\newcommand{\titel}{Hugo von Hofmannsthal an Arthur Schnitzler, 16. [11. 1907?]}\newcommand{\editorInnen}{Martin Anton Müller und Gerd-Hermann Susen}%% latex-leseansicht-abspann.tex
%% Abspann für die Leseansicht.
%% Der Schalter \ifkorrekturansicht ist bereits durch den Vorspann gesetzt.

%% latex-abspann.tex
%% Gemeinsamer Abspann für Korrekturansicht und Leseansicht.
%% Setzt den Schalter \ifkorrekturansicht voraus (gesetzt in den
%% einbindenden Dateien latex-korrekturansicht-abspann.tex bzw.
%% latex-leseansicht-abspann.tex).
%% ---------------------------------------------------------------

\normalsize

% Das esempio-Environment wird nur in der Leseansicht benötigt
\ifkorrekturansicht\else
\newenvironment{esempio}[3]%
{
    \vspace{1.5ex}
    \rlap{\underline{#1}}
    \par
    \setlength{\parindent}{0cm}
    \nopagebreak
    \leftskip=#2cm
    \rightskip=#3cm
}
{
    \par
}
\fi

\doendnotes{C}
\bigskip
\vfill

\clearpage

\footnotesize

\ifkorrekturansicht
  \lohead{\textsc{register}}
\fi

% theindex-Environment neu definieren ohne reledmac
\makeatletter
\renewenvironment{theindex}{%
  \ifkorrekturansicht
    \section*{\indexname}%
  \else
    \subsubsection*{Index der erwähnten Entitäten}%
  \fi
  \setlength{\parindent}{0pt}%
  \setlength{\parskip}{0pt plus 0.3pt}%
  \let\item\@idxitem
}{%
  \ifkorrekturansicht\clearpage\fi
}
\makeatother

\IfFileExists{\jobname-pw.ind}{\input{\jobname-pw.ind}}{}

% Quellenangabe nur in der Leseansicht
\ifkorrekturansicht\else
% Fallback-Definitionen, falls die .tex-Datei \titel etc. nicht gesetzt hat
\providecommand{\titel}{}
\providecommand{\editorInnen}{}
\providecommand{\dateiname}{\jobname}

\vspace{3cm}

\vfill

\footnotesize
\textsc{Quelle}: \titel. Herausgegeben von {\editorInnen}. In: \emph{Arthur Schnitzler: Briefwechsel mit Autorinnen und Autoren}.
 Digitale Edition, https://schnitzler-briefe.acdh.oeaw.ac.at/{\dateiname}.html (Stand \today)
\fi

\end{document}


      