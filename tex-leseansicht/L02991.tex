%% latex-leseansicht-vorspann.tex
%% Vorspann für die Leseansicht.
%% Lädt die gemeinsame Datei latex-vorspann.tex mit nicht gesetztem Schalter.

\newif\ifkorrekturansicht
\korrekturansichtfalse

\input{../tex-inputs/latex-vorspann}

\begin{center}
            \textcolor{red}{ENTWURF, NICHT FERTIG KORRIGIERT}
                      \end{center}
            
         
         \renewcommand{\erwaehntePersonen}{Personen: Hugo Haberfeld, Richard Klein, Bertha Klein, Felix Salten, Heinrich Schnitzler}
         \renewcommand{\erwaehnteInstitutionen}{Institutionen: Galerie Pisko}
         \renewcommand{\erwaehnteOrte}{Orte: Wien}
         \renewcommand{\erwaehnteWerke}{}
               \section[Arthur Schnitzler an Felix Salten, 13. 4. 1904]{ Arthur Schnitzler an Felix Salten, 13. 4. 1904}\nopagebreak\mylabel{v}\rehead{ }\begin{ledgroupsized}[t]{13cm}\normalsize\beginnumbering \toendnotes[C]{\smallbreak\pagebreak[2]} \Standort{Wienbibliothek im Rathaus, ZPH 1681, 2.1.516.}
\physDesc{Brief, 1 Blatt, 3 Seiten, 623 Zeichen
\newline{}Handschrift: Bleistift, deutsche Kurrent
\newline{}Ordnung: mit Bleistift von unbekannter Hand Nummerierung der Blätter des
                                 Konvoluts: »32« }\toendnotes[C]{\smallbreak}\pstart
           \raggedleft{}{\pb}13. 4. 904\pend
           \pstart
           lieber Freund, ein Vetter, oder wenigſtens beinah ein Vetter von
               mir, \textsc{Richard Klein\pwindex{Klein, Richard *~07.08.1873@\textsc{Klein, Richard} (*~07.08.1873), \emph{Bildender Künstler}|pw}} ſtellt bei Pisko\orgindex{Galerie Pisko@Galerie Pisko|pw} aus, ſeine Mutter\pwindex{Klein, Bertha 14.02.1849 – 16.01.1907@\textsc{Klein, Bertha} (14.02.1849 – 16.01.1907)|pwv} ſchreibt mir, ich
               möchte Sie bitten, dieſe Ausſtellg zu beſuchen.– Was hiemit geſchieht. Aber ich
               denke, nicht Sie ſondern \textsc{Haberfeld\pwindex{Haberfeld, Hugo 1875-11-24 – 1946@\textsc{Haberfeld, Hugo} (1875-11-24 – 1946), \emph{Galerist, Kritiker}|pw}} ſchrei{\pb}ben über dergleichen. (Was ich
               auch meiner Tante\pwindex{Klein, Bertha 14.02.1849 – 16.01.1907@\textsc{Klein, Bertha} (14.02.1849 – 16.01.1907)|pwuv}
               ſchreibe.) \pend
           \pstart
           Unser Bub\pwindex{Schnitzler, Heinrich 09.08.1902 – 12.07.1982@\textsc{Schnitzler, Heinrich} (09.08.1902 – 12.07.1982), \emph{Regisseur, Schauspieler}|pwv} hat die Maſern –
               trotzdem in dieſer Woche die Erkrankungsfälle ſchon ſinken. Was ſchert ſich ſo ein
               Bub um die Statiſtik. Ich denke mir oft, wie gefrozzelt ſich die Leute vorkommen, die
               krank werden, während eine {\pb}Epidemie im
               »Erlöſchen« iſt. (»Der letzte Fall«, Novelle.–) \pend
           \pstart
           Grüß Sie Gott. {\\[\baselineskip]}Herzlich Ihr {\\[\baselineskip]}\spacefill\mbox{A.}\pend
           \leftskip=0em{}
         
         \endnumbering\mylabel{h}\end{ledgroupsized}\begin{anhang}\end{anhang}\newcommand{\dateiname}{L02991}\newcommand{\titel}{Arthur Schnitzler an Felix Salten, 13. 4. 1904}\newcommand{\editorInnen}{Martin Anton Müller und Laura Untner}%% latex-leseansicht-abspann.tex
%% Abspann für die Leseansicht.
%% Der Schalter \ifkorrekturansicht ist bereits durch den Vorspann gesetzt.

%% latex-abspann.tex
%% Gemeinsamer Abspann für Korrekturansicht und Leseansicht.
%% Setzt den Schalter \ifkorrekturansicht voraus (gesetzt in den
%% einbindenden Dateien latex-korrekturansicht-abspann.tex bzw.
%% latex-leseansicht-abspann.tex).
%% ---------------------------------------------------------------

\normalsize

% Das esempio-Environment wird nur in der Leseansicht benötigt
\ifkorrekturansicht\else
\newenvironment{esempio}[3]%
{
    \vspace{1.5ex}
    \rlap{\underline{#1}}
    \par
    \setlength{\parindent}{0cm}
    \nopagebreak
    \leftskip=#2cm
    \rightskip=#3cm
}
{
    \par
}
\fi

\doendnotes{C}
\bigskip
\vfill

\clearpage

\footnotesize

\ifkorrekturansicht
  \lohead{\textsc{register}}
\fi

% theindex-Environment neu definieren ohne reledmac
\makeatletter
\renewenvironment{theindex}{%
  \ifkorrekturansicht
    \section*{\indexname}%
  \else
    \subsubsection*{Index der erwähnten Entitäten}%
  \fi
  \setlength{\parindent}{0pt}%
  \setlength{\parskip}{0pt plus 0.3pt}%
  \let\item\@idxitem
}{%
  \ifkorrekturansicht\clearpage\fi
}
\makeatother

\IfFileExists{\jobname-pw.ind}{\input{\jobname-pw.ind}}{}

% Quellenangabe nur in der Leseansicht
\ifkorrekturansicht\else
% Fallback-Definitionen, falls die .tex-Datei \titel etc. nicht gesetzt hat
\providecommand{\titel}{}
\providecommand{\editorInnen}{}
\providecommand{\dateiname}{\jobname}

\vspace{3cm}

\vfill

\footnotesize
\textsc{Quelle}: \titel. Herausgegeben von {\editorInnen}. In: \emph{Arthur Schnitzler: Briefwechsel mit Autorinnen und Autoren}.
 Digitale Edition, https://schnitzler-briefe.acdh.oeaw.ac.at/{\dateiname}.html (Stand \today)
\fi

\end{document}


      