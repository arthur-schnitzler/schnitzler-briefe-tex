%% latex-korrekturansicht-vorspann.tex
%% Vorspann für die Korrekturansicht.
%% Lädt die gemeinsame Datei latex-vorspann.tex mit gesetztem Schalter.

\newif\ifkorrekturansicht
\korrekturansichttrue

\input{../tex-inputs/latex-vorspann}


\section[ Arthur Schnitzler an Felix Salten, 13. 4. 1904]{L02991 Arthur Schnitzler an Felix Salten, 13. 4. 1904}
\nopagebreak\mylabel{L02991v}
\rehead{ }\normalsize\beginnumbering\briefempfaengerindex{Salten, Felix@\textsc{Salten, Felix}!zzzSchnitzler, Arthur@\emph{von Arthur Schnitzler}!1904-04-131@{13. 4. 1904}|(be}
\toendnotes[C]{\smallbreak\pagebreak[2]}\Standort{Wienbibliothek im Rathaus, ZPH 1681, 2.1.516.}
\physDesc{Brief, 1 Blatt, 3 Seiten, 620 Zeichen
\newline{}Handschrift: Bleistift, deutsche Kurrent
\newline{}Ordnung: mit Bleistift von unbekannter Hand Nummerierung der Doppelseiten des
                                 Konvoluts: »32«–»33« }
\buchAbdrucke{\weitereDrucke{Arthur Schnitzler: \emph{Briefe 1875–1912}. Frankfurt am Main: \emph{S. Fischer} 1981, S. 481.} }\toendnotes[C]{\smallbreak}
\pstart
           \raggedleft{}{\pb}13. 4. 904\pend
           \vspace{0.5em}
\pstart
           lieber Freund, ein Vetter, oder wenigſtens \label{K_L02991-1v}\edtext{beinah ein Vetter}{\lemma{\textnormal{\emph{beinah ein Vetter}}}\Cendnote{\textnormal{Der Vater\pwindex{Klein, Johann 1838-10-15 – 1927-05-18@\textsc{Klein, Johann} (1838-10-15 – 1927-05-18), \emph{Großindustrieller/Großindustrielle, Bankier/Bankierin}|pwkv} von Richard Klein\pwindex{Klein, Richard *~07.08.1873@\textsc{Klein, Richard} (*~07.08.1873), \emph{Maler/Malerin}|pwk} war der Bruder von Rosalie Schnitzler\pwindex{Schnitzler, Rosalie 1812/1813 – 08.11.1878@\textsc{Schnitzler, Rosalie} (1812/1813 – 08.11.1878)|pwk}, Arthur Schnitzlers Großmutter väterlicherseits.}}}\label{K_L02991-1} von
               mir, \textsc{Richard Klein\pwindex{Klein, Richard *~07.08.1873@\textsc{Klein, Richard} (*~07.08.1873), \emph{Maler/Malerin}|pw}}{[},{]} ſtellt bei Pisko\orgindex{Galerie Pisko@Galerie Pisko|pw} aus,
               ſeine Mutter\pwindex{Klein, Bertha 14.02.1849 – 16.01.1907@\textsc{Klein, Bertha} (14.02.1849 – 16.01.1907)|pwv} ſchreibt mir,
               ich möchte Sie bitten, dieſe \label{K_L02991-4v}\edtext{Ausſtellg\orgindex{[Ausstellung von Josef Beyer, Richard Klein, Lazar Krestin, Paul Ress und Karl Schade]@[Ausstellung von Josef Beyer, Richard Klein, Lazar Krestin, Paul Reß und Karl Schade]|pwv}}{\lemma{\textnormal{\emph{Ausſtellg}}}\Cendnote{\textnormal{Die Ausstellung mit vier weiteren Künstlern
                  wurde am 16. 4. 1904 eröffnet. Weder ein Besuch Schnitzlers oder Saltens\pwindex{Salten, Felix 06.09.1869 – 08.10.1945@\textsc{Salten, Felix} (06.09.1869 – 08.10.1945), \emph{Schriftsteller/Schriftstellerin, Journalist/Journalistin, Chefredakteur/Chefredakteurin}|pwk} noch eine Besprechung konnten nachgewiesen
                  werden.}}}\label{K_L02991-4} zu beſuchen. – Was hiemit geſchieht. Aber ich denke, nicht Sie
               ſondern \label{K_L02991-2v}\edtext{\textsc{Haberfeld\pwindex{Haberfeld, Hugo 1875-11-24 – 1946@\textsc{Haberfeld, Hugo} (1875-11-24 – 1946), \emph{Galerist/Galeristin, Kunstkritiker/Kunstkritikerin}|pw}}{ }\label{T_L02991-5v}\edtext{ſchrei{\pb}bt}{\lemma{\textnormal{\emph{ſchreibt}}}\Cendnote{\textnormal{Schnitzler ist beim Seitenwechsel
                     ein Grammatikfehler unterlaufen, er schrieb:
                     »schreiben«.}}}\label{T_L02991-5} über dergleichen}{\lemma{\textnormal{\emph{Haberfeld … dergleichen}}}\Cendnote{\textnormal{Siehe Felix Salten an Arthur Schnitzler, [14. 4. 1904].
               }}}\label{K_L02991-2}. (Was ich auch meiner Tante\pwindex{Klein, Bertha 14.02.1849 – 16.01.1907@\textsc{Klein, Bertha} (14.02.1849 – 16.01.1907)|pwuv} ſchreibe.)\pend
           
\pstart
           Unser Bub\pwindex{Schnitzler, Heinrich 09.08.1902 – 12.07.1982@\textsc{Schnitzler, Heinrich} (09.08.1902 – 12.07.1982), \emph{Regisseur/Regisseurin, Schauspieler/Schauspielerin}|pwv} hat die Maſern –
               trotzdem in dieſer Woche die Erkrankungsfälle ſchon ſinken. Was ſchert ſich ſo ein
               Bub um die Statiſtik. Ich denke mir oft, wie gefrozzelt ſich die Leute vorkommen, die
               krank werden, während eine {\pb}Epidemie im
               »Erlöſchen« iſt. (»Der letzte Fall«, Novelle. –)\pend
           
\pstart
           Grüß Sie Gott. {\\[\baselineskip]}Herzlich Ihr {\\[\baselineskip]}\spacefill\mbox{A.}\pend
           \leftskip=0em{}\selectlanguage{ngerman}\endnumbering\briefempfaengerindex{Salten, Felix@\textsc{Salten, Felix}!zzzSchnitzler, Arthur@\emph{von Arthur Schnitzler}!1904-04-131@{13. 4. 1904}|)be}\mylabel{L02991h}  \normalsize

\doendnotes{C}
\bigskip
\vfill

\clearpage

\footnotesize

\lohead{\textsc{register}}

% Definiere theindex-Environment komplett neu ohne reledmac
\makeatletter
\renewenvironment{theindex}{%
  \section*{\indexname}%
  \setlength{\parindent}{0pt}%
  \setlength{\parskip}{0pt plus 0.3pt}%
  \let\item\@idxitem
}{%
  \clearpage
}
\makeatother

\IfFileExists{\jobname-pw.ind}{\input{\jobname-pw.ind}}{}

\end{document}

      