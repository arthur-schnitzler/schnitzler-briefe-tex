%% latex-korrekturansicht-vorspann.tex
%% Vorspann für die Korrekturansicht.
%% Lädt die gemeinsame Datei latex-vorspann.tex mit gesetztem Schalter.

\newif\ifkorrekturansicht
\korrekturansichttrue

\input{../tex-inputs/latex-vorspann}


\section[Albert Ehrenstein an Arthur Schnitzler, 22. 11. 1909]{L01887 Albert Ehrenstein an Arthur Schnitzler, 22. 11. 1909}
\nopagebreak\mylabel{L01887v}
\rehead{ }\normalsize\beginnumbering\briefempfaengerindex{Schnitzler, Arthur@\textsc{Schnitzler, Arthur}!zzzEhrenstein, Albert@\emph{von Albert Ehrenstein}!1909-11-221@{22. 11. 1909}|(be}
\toendnotes[C]{\smallbreak\pagebreak[2]}\Standort{CUL, Schnitzler, B 30.}
\physDesc{Brief, 1 Blatt, 3 Seiten, 1391 Zeichen
\newline{}Handschrift: schwarze Tinte, deutsche Kurrent
\newline{}Schnitzler: mit Bleistift beschriftet: »\textsc{Ehrenste\textcolor{gray}{in}}« }
\buchAbdrucke{\weitereDrucke{Albert Ehrenstein: \emph{Briefe}. München: \emph{Boer} 1989, S. 35–36.} }\toendnotes[C]{\smallbreak}
\pstart
           {\pb}XVI. \textsc{Ottakringerstr.}
                        114\oindex{Ottakringer Strasse@\textbf{Ottakringer Straße}, \emph{Straße (K.STR)}|pw}. \hfill 22. XI. 09. \pend
           
\pstart{}Sehr geehrter Herr Doktor,\pend\vspace{0.5em}
\pstart
           Herr Alfred Polgar\pwindex{Polgar, Alfred 17.10.1873 – 24.04.1955@\textsc{Polgar, Alfred} (17.10.1873 – 24.04.1955), \emph{Schriftsteller/Schriftstellerin, Journalist/Journalistin, Kritiker/Kritikerin}|pw}, dem ich, wie Sie wiſſen,
               Arbeiten unterbreitete, fand großen Gefallen an denſelben und ſchickte mir, der ich
               ihn übrigens nicht perſönlich kenne, eine in ſchmeichelhafter Weiſe abgefaßte
               Empfehlung – aber zu meiner Überraſchung an Herrn Profeſſor Bie\pwindex{Bie, Oskar 09.02.1864 – 21.04.1938@\textsc{Bie, Oskar} (09.02.1864 – 21.04.1938), \emph{Schriftsteller/Schriftstellerin, Journalist/Journalistin, Redakteur/Redakteurin}|pw} für die N. Rundſchau\pwindex{neue Rundschau@\emph{Die neue Rundschau}|pw}.
               Ich konnte nicht umhin, von derſelben Gebrauch zu machen (ſchon um das mir
               entgegengebrachte Wohlwollen nicht zu kränken), obwohl ich in erſter Linie, die Rundſchau\pwindex{neue Rundschau@\emph{Die neue Rundschau}|pw} und Herrn Profeſſor Bie\pwindex{Bie, Oskar 09.02.1864 – 21.04.1938@\textsc{Bie, Oskar} (09.02.1864 – 21.04.1938), \emph{Schriftsteller/Schriftstellerin, Journalist/Journalistin, Redakteur/Redakteurin}|pw} betreffend, auf die {\pb}von Ihnen mir freundlichſt in Ausſicht
               geſtellte Fürſprache bei letzterem rechne. Vorgeſtern ſandte ich 6 Skizzen (Saccumum\pwindex{Saccumum@\emph{Saccumum}|pw}, Mitgefühl\pwindex{Mitgefuehl@\emph{Mitgefühl}|pw}, Die alte Geſchichte\pwindex{alte Geschichte@\emph{Die alte Geschichte}|pw}, Tubutſch\pwindex{Tubutsch@\emph{Tubutsch}|pw}, Baber\pwindex{Tod des Zehir eddin Muhammed Baber@\emph{Tod des Zehir eddin Muhammed Baber}|pw} u. Tai-gin\pwindex{Tai-Gin@\emph{Tai-Gin}|pw}) an Herrn Profeſſor
                  Bie\pwindex{Bie, Oskar 09.02.1864 – 21.04.1938@\textsc{Bie, Oskar} (09.02.1864 – 21.04.1938), \emph{Schriftsteller/Schriftstellerin, Journalist/Journalistin, Redakteur/Redakteurin}|pw}.\pend
           
\pstart
           Nun weiß ich nicht, ob Sie, ſehr geehrter Herr Doktor, ſchon in Berlin\oindex{Berlin@\textbf{Berlin}, \emph{P.PPLC}|pw} waren und die Liebenswürdigkeit gehabt haben, meinen
               Skizzenband »\label{K_L01887-1v}\edtext{Zuſchauer und
                  Tyrannen}{\lemma{\textnormal{\emph{Zuſchauer und
                  Tyrannen}}}\Cendnote{\textnormal{Unter diesem Titel
                     veröffentlichte Ehrenstein\pwindex{Ehrenstein, Albert 23.12.1886 – 08.04.1950@\textsc{Ehrenstein, Albert} (23.12.1886 – 08.04.1950), \emph{Schriftsteller/Schriftstellerin}|pwk} keine Novellensammlung, doch ist in seinem Nachlass ein Entwurf
                  der dafür vorgesehenen 19 Novellen überliefert.}}}\label{K_L01887-1}« – den ich Ihnen vor etwa
               14 Tagen mit einem Begleitſchreiben zukommenließ – oder eine ſtrenge Auswahl meiner
               Novelletten Ihrem Verleger\pwindex{Fischer, Samuel 24.12.1859 – 15.10.1934@\textsc{Fischer, Samuel} (24.12.1859 – 15.10.1934), \emph{Verleger/Verlegerin}|pwv}
               zu geben, oder ob dies noch bevorſteht?\pend
           
\pstart
           Jedenfalls möchte ich Sie höflichſt {\pb}bitten, nicht bloß bei dem Herrn Fiſcher\pwindex{Fischer, Samuel 24.12.1859 – 15.10.1934@\textsc{Fischer, Samuel} (24.12.1859 – 15.10.1934), \emph{Verleger/Verlegerin}|pw},
               ſondern, wenn es angängig iſt, auch bei dem Herrn Profeſſor Bie\pwindex{Bie, Oskar 09.02.1864 – 21.04.1938@\textsc{Bie, Oskar} (09.02.1864 – 21.04.1938), \emph{Schriftsteller/Schriftstellerin, Journalist/Journalistin, Redakteur/Redakteurin}|pw} für mich zu wirken.\pend
           
\pstart
           Für Ihre gewiß erfolgreichen Interventionen im Voraus dankend, bin ich mit dem
               Ausdrucke vorzüglichſter Hochachtung \pend
           
\pstart
           Ihr ergebenſter{\\[\baselineskip]}\spacefill\mbox{Albert Ehrenstein.}\pend
           \leftskip=0em{}\selectlanguage{ngerman}\endnumbering\briefempfaengerindex{Schnitzler, Arthur@\textsc{Schnitzler, Arthur}!zzzEhrenstein, Albert@\emph{von Albert Ehrenstein}!1909-11-221@{22. 11. 1909}|)be}\mylabel{L01887h}  \normalsize

\doendnotes{C}
\bigskip
\vfill

\clearpage

\footnotesize

\lohead{\textsc{register}}

% Definiere theindex-Environment komplett neu ohne reledmac
\makeatletter
\renewenvironment{theindex}{%
  \section*{\indexname}%
  \setlength{\parindent}{0pt}%
  \setlength{\parskip}{0pt plus 0.3pt}%
  \let\item\@idxitem
}{%
  \clearpage
}
\makeatother

\IfFileExists{\jobname-pw.ind}{\input{\jobname-pw.ind}}{}

\end{document}

      