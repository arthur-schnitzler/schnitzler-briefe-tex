%% latex-leseansicht-vorspann.tex
%% Vorspann für die Leseansicht.
%% Lädt die gemeinsame Datei latex-vorspann.tex mit nicht gesetztem Schalter.

\newif\ifkorrekturansicht
\korrekturansichtfalse

\input{../tex-inputs/latex-vorspann}


         
         \newcommand{\erwaehntePersonen}{Personen: Oskar Bie, Samuel Fischer, Alfred Polgar}
         \newcommand{\erwaehnteOrte}{Orte: Berlin, Ottakringerstraße, Wien}
         \newcommand{\erwaehnteWerke}{Werke: Die alte Geschichte, Die neue Rundschau, Mitgefühl, Saccumum, Tai-Gin, Tod des Zehir eddin Muhammed Baber, Tubutsch}
               \section[Albert Ehrenstein an Arthur Schnitzler, 22. 11. 1909]{ Albert Ehrenstein an Arthur Schnitzler, 22. 11. 1909}\nopagebreak\mylabel{v}\rehead{ }\begin{ledgroupsized}[t]{13cm}\normalsize\beginnumbering \toendnotes[C]{\smallbreak\pagebreak[2]} \Standort{CUL, Schnitzler, B 30.}
\physDesc{Brief, 1 Blatt, 3 Seiten
\newline{}Handschrift: schwarze Tinte, deutsche Kurrent
\newline{}Schnitzler: mit Bleistift beschriftet: »\textsc{Ehrenste\textcolor{gray}{in}}« }\buchAbdrucke{\weitereDrucke{Albert Ehrenstein: \emph{Briefe}. Hg. Hanni Mittelmann. München: \emph{Boer} 1989, S. 35–36 (Werke, 1).} }\toendnotes[C]{\smallbreak}\pstart
           \noindent{}{\pb}XVI. \textsc{Ottakringerstr.} 114\oindex{Ottakringerstrasse@\textbf{Ottakringerstraße}|pw}.
                        \hfill 22. XI. 09.
                        \pend
           \pstart{}Sehr geehrter Herr Doktor,\pend\pstart
           Herr Alfred Polgar\pwindex{Polgar, Alfred 17.10.1873 – 24.04.1955@\textsc{Polgar, Alfred} (17.10.1873 – 24.04.1955), \emph{Schriftsteller, Journalist, Kritiker}|pw}, dem ich, wie Sie wiſſen,
                    Arbeiten unterbreitete, fand großen Gefallen an denſelben und ſchickte mir, der
                    ich ihn übrigens nicht perſönlich kenne, eine in ſchmeichelhafter Weiſe
                    abgefaßte Empfehlung – aber zu meiner Überraſchung an Herrn Profeſſor Bie\pwindex{Bie, Oskar 09.02.1864 – 21.04.1938@\textsc{Bie, Oskar} (09.02.1864 – 21.04.1938), \emph{Schriftsteller, Journalist, Redakteur}|pw} für die N. Rundſchau\pwindex{?? Werk@Nicht ermittelte Verfasserinnen und Verfasser!neue Rundschau1904@\emph{Die neue Rundschau} {[}1904{]}|pw}. Ich konnte nicht umhin, von derſelben Gebrauch zu machen
                    (ſchon um das mir entgegengebrachte Wohlwollen nicht zu kränken), obwohl ich in
                    erſter Linie, die Rundſchau\pwindex{?? Werk@Nicht ermittelte Verfasserinnen und Verfasser!neue Rundschau1904@\emph{Die neue Rundschau} {[}1904{]}|pw} und Herrn
                    Profeſſor Bie\pwindex{Bie, Oskar 09.02.1864 – 21.04.1938@\textsc{Bie, Oskar} (09.02.1864 – 21.04.1938), \emph{Schriftsteller, Journalist, Redakteur}|pw} betreffend, auf die {\pb}von Ihnen mir freundlichſt in Ausſicht geſtellte
                    Fürſprache bei letzterem rechne. Vorgeſtern ſandte ich 6 Skizzen (Saccumum\pwindex{Ehrenstein, Albert 23.12.1886 – 08.04.1950@\textsc{Ehrenstein, Albert} (23.12.1886 – 08.04.1950), \emph{Schriftsteller}!Saccumum28. 02. 1911@\strich\emph{Saccumum} {[}28. 02. 1911{]}|pw}, Mitgefühl\pwindex{Ehrenstein, Albert 23.12.1886 – 08.04.1950@\textsc{Ehrenstein, Albert} (23.12.1886 – 08.04.1950), \emph{Schriftsteller}!Mitgefuehl23. 11. 1910@\strich\emph{Mitgefühl} {[}23. 11. 1910{]}|pw}, Die alte Geſchichte\pwindex{Ehrenstein, Albert 23.12.1886 – 08.04.1950@\textsc{Ehrenstein, Albert} (23.12.1886 – 08.04.1950), \emph{Schriftsteller}!alte Geschichte1912.10@\strich\emph{Die alte Geschichte} {[}1912.10{]}|pw}, Tubutſch\pwindex{Ehrenstein, Albert 23.12.1886 – 08.04.1950@\textsc{Ehrenstein, Albert} (23.12.1886 – 08.04.1950), \emph{Schriftsteller}!Tubutsch1911@\strich\emph{Tubutsch} {[}1911{]}|pw}, Baber\pwindex{Ehrenstein, Albert 23.12.1886 – 08.04.1950@\textsc{Ehrenstein, Albert} (23.12.1886 – 08.04.1950), \emph{Schriftsteller}!Tod des Zehir eddin Muhammed Baber1912@\strich\emph{Tod des Zehir eddin Muhammed Baber} {[}1912{]}|pw} u. Tai-gin\pwindex{Ehrenstein, Albert 23.12.1886 – 08.04.1950@\textsc{Ehrenstein, Albert} (23.12.1886 – 08.04.1950), \emph{Schriftsteller}!Tai-Gin1912@\strich\emph{Tai-Gin} {[}1912{]}|pw}) an Herrn Profeſſor
                        Bie\pwindex{Bie, Oskar 09.02.1864 – 21.04.1938@\textsc{Bie, Oskar} (09.02.1864 – 21.04.1938), \emph{Schriftsteller, Journalist, Redakteur}|pw}.\pend
           \pstart
           Nun weiß ich nicht, ob Sie, ſehr geehrter Herr Doktor, ſchon in Berlin\oindex{Berlin@\textbf{Berlin}|pw} waren und die Liebenswürdigkeit gehabt haben,
                    meinen Skizzenband »\label{K_L01887_1v}\edtext{Zuſchauer und
                        Tyrannen}{\lemma{\textnormal{\emph{Zuſchauer und
                        Tyrannen}}}\Cendnote{\textnormal{Unter diesem Titel
                        veröffentlichte er keine Novellensammlung, doch ist in seinem Nachlass ein
                        Entwurf der dafür vorgesehenen 19 Novellen überliefert.}}}\label{K_L01887_1h}« – den ich
                    Ihnen vor etwa 14 Tagen mit einem Begleitſchreiben zukommenließ – oder eine
                    ſtrenge Auswahl meiner Novelletten Ihrem Verleger\pwindex{Fischer, Samuel 24.12.1859 – 15.10.1934@\textsc{Fischer, Samuel} (24.12.1859 – 15.10.1934), \emph{Verleger}|pwv} zu geben, oder ob dies noch bevorſteht?\pend
           \pstart
           Jedenfalls möchte ich Sie höflichſt {\pb}bitten, nicht bloß
                    bei dem Herrn Fiſcher\pwindex{Fischer, Samuel 24.12.1859 – 15.10.1934@\textsc{Fischer, Samuel} (24.12.1859 – 15.10.1934), \emph{Verleger}|pw}, ſondern, wenn es
                    angängig iſt, auch bei dem Herrn Profeſſor Bie\pwindex{Bie, Oskar 09.02.1864 – 21.04.1938@\textsc{Bie, Oskar} (09.02.1864 – 21.04.1938), \emph{Schriftsteller, Journalist, Redakteur}|pw} für mich zu wirken.\pend
           \pstart
           Für Ihre gewiß erfolgreichen Interventionen im Voraus dankend, bin ich mit dem
                    Ausdrucke vorzüglichſter Hochachtung \pend
           \pstart
           Ihr ergebenſter{\\[\baselineskip]}\spacefill\mbox{Albert Ehrenstein.}\pend
           \leftskip=0em{}
         
         \endnumbering\mylabel{h}\end{ledgroupsized}  \newcommand{\dateiname}{L01887}\newcommand{\titel}{Albert Ehrenstein an Arthur Schnitzler, 22. 11. 1909}\newcommand{\editorInnen}{Martin Anton Müller und Gerd-Hermann Susen}%% latex-leseansicht-abspann.tex
%% Abspann für die Leseansicht.
%% Der Schalter \ifkorrekturansicht ist bereits durch den Vorspann gesetzt.

%% latex-abspann.tex
%% Gemeinsamer Abspann für Korrekturansicht und Leseansicht.
%% Setzt den Schalter \ifkorrekturansicht voraus (gesetzt in den
%% einbindenden Dateien latex-korrekturansicht-abspann.tex bzw.
%% latex-leseansicht-abspann.tex).
%% ---------------------------------------------------------------

\normalsize

% Das esempio-Environment wird nur in der Leseansicht benötigt
\ifkorrekturansicht\else
\newenvironment{esempio}[3]%
{
    \vspace{1.5ex}
    \rlap{\underline{#1}}
    \par
    \setlength{\parindent}{0cm}
    \nopagebreak
    \leftskip=#2cm
    \rightskip=#3cm
}
{
    \par
}
\fi

\doendnotes{C}
\bigskip
\vfill

\clearpage

\footnotesize

\ifkorrekturansicht
  \lohead{\textsc{register}}
\fi

% theindex-Environment neu definieren ohne reledmac
\makeatletter
\renewenvironment{theindex}{%
  \ifkorrekturansicht
    \section*{\indexname}%
  \else
    \subsubsection*{Index der erwähnten Entitäten}%
  \fi
  \setlength{\parindent}{0pt}%
  \setlength{\parskip}{0pt plus 0.3pt}%
  \let\item\@idxitem
}{%
  \ifkorrekturansicht\clearpage\fi
}
\makeatother

\IfFileExists{\jobname-pw.ind}{\input{\jobname-pw.ind}}{}

% Quellenangabe nur in der Leseansicht
\ifkorrekturansicht\else
% Fallback-Definitionen, falls die .tex-Datei \titel etc. nicht gesetzt hat
\providecommand{\titel}{}
\providecommand{\editorInnen}{}
\providecommand{\dateiname}{\jobname}

\vspace{3cm}

\vfill

\footnotesize
\textsc{Quelle}: \titel. Herausgegeben von {\editorInnen}. In: \emph{Arthur Schnitzler: Briefwechsel mit Autorinnen und Autoren}.
 Digitale Edition, https://schnitzler-briefe.acdh.oeaw.ac.at/{\dateiname}.html (Stand \today)
\fi

\end{document}


      