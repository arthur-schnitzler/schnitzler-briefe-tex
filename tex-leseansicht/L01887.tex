%% latex-leseansicht-vorspann.tex
%% Vorspann für die Leseansicht.
%% Lädt die gemeinsame Datei latex-vorspann.tex mit nicht gesetztem Schalter.

\newif\ifkorrekturansicht
\korrekturansichtfalse

\input{../tex-inputs/latex-vorspann}


\section[Albert Ehrenstein an Arthur Schnitzler, 22. 11. 1909]{L01887 Albert Ehrenstein an Arthur Schnitzler, 22. 11. 1909}
\nopagebreak\mylabel{L01887v}
\rehead{ }\normalsize\beginnumbering\briefempfaengerindex{Schnitzler, Arthur@\textsc{Schnitzler, Arthur}!zzzEhrenstein, Albert@\emph{von Albert Ehrenstein}!1909-11-221@{22. 11. 1909}|(be}
\toendnotes[C]{\smallbreak\pagebreak[2]}
\correspDesc{Versand  durch Albert Ehrenstein am 22. 11. 1909 in Wien
\newline{}Erhalt  durch Arthur Schnitzler im Zeitraum [22. 11. 1909 – 26. 11. 1909?] in Wien}\toendnotes[C]{\smallbreak}
\Standort{CUL, Schnitzler, B 30.}
\physDesc{Brief, 1 Blatt, 3 Seiten, 1391 Zeichen
\newline{}Handschrift: schwarze Tinte, deutsche Kurrent
\newline{}Schnitzler: mit Bleistift beschriftet: »\textsc{Ehrenste\textcolor{gray}{in}}« }
\buchAbdrucke{\weitereDrucke{Albert Ehrenstein: \emph{Briefe}. Herausgegeben von Hanni Mittelmann. München: \emph{Boer} 1989, S. 35–36 (Werke, 1).} }\toendnotes[C]{\smallbreak}
\pstart
           {\pb}XVI. \textsc{Ottakringerstr.}
                        114\oindex{Wien@\textbf{Wien}!XVI., Ottakring@\textbf{XVI., Ottakring}!Ottakringer Straße@\textbf{Ottakringer Straße}, \emph{Straße}|pw}\oindex{Wien@\textbf{Wien}!XVII., Hernals@\textbf{XVII., Hernals}!Ottakringer Straße@\textbf{Ottakringer Straße}, \emph{Straße}|pw}.\hfill 22. XI. 09.\pend
           
\pstart{}Sehr geehrter Herr Doktor,\pend\vspace{0.5em}
\pstart
           Herr Alfred Polgar\pwindex{Polgar, Alfred 17.\,10.\,1873 Wien – 24.\,4.\,1955 Zürich@\textsc{Polgar, Alfred} (17.\,10.\,1873 Wien – 24.\,4.\,1955 Zürich), \emph{Schriftsteller, Journalist, Kritiker}|pw}, dem ich, wie Sie wiſſen,
               Arbeiten unterbreitete, fand großen Gefallen an denſelben und{ }ſchickte mir, der ich
               ihn übrigens nicht perſönlich kenne, eine in{ }ſchmeichelhafter Weiſe abgefaßte
               Empfehlung – aber zu meiner Überraſchung an Herrn Profeſſor Bie\pwindex{Bie, Oskar 9.\,2.\,1864 Breslau – 21.\,4.\,1938 Berlin@\textsc{Bie, Oskar} (9.\,2.\,1864 Breslau – 21.\,4.\,1938 Berlin), \emph{Schriftsteller, Journalist, Redakteur}|pw} für die N. Rundſchau\pwindex{neue Rundschau@\emph{Die neue Rundschau}|pw}.
               Ich konnte nicht umhin, von derſelben Gebrauch zu machen (ſchon um das mir
               entgegengebrachte Wohlwollen nicht zu kränken), obwohl ich in erſter Linie, die Rundſchau\pwindex{neue Rundschau@\emph{Die neue Rundschau}|pw} und Herrn Profeſſor Bie\pwindex{Bie, Oskar 9.\,2.\,1864 Breslau – 21.\,4.\,1938 Berlin@\textsc{Bie, Oskar} (9.\,2.\,1864 Breslau – 21.\,4.\,1938 Berlin), \emph{Schriftsteller, Journalist, Redakteur}|pw} betreffend, auf die {\pb}von Ihnen mir freundlichſt in Ausſicht
               geſtellte Fürſprache bei letzterem rechne. Vorgeſtern{ }ſandte ich 6 Skizzen (Saccumum\pwindex{Ehrenstein, Albert 23.\,12.\,1886 Wien – 8.\,4.\,1950 New York City@\textsc{Ehrenstein, Albert} (23.\,12.\,1886 Wien – 8.\,4.\,1950 New York City), \emph{Schriftsteller}!Saccumum@\strich\emph{Saccumum}|pw}, Mitgefühl\pwindex{Ehrenstein, Albert 23.\,12.\,1886 Wien – 8.\,4.\,1950 New York City@\textsc{Ehrenstein, Albert} (23.\,12.\,1886 Wien – 8.\,4.\,1950 New York City), \emph{Schriftsteller}!Mitgefühl@\strich\emph{Mitgefühl}|pw}, Die alte Geſchichte\pwindex{Ehrenstein, Albert 23.\,12.\,1886 Wien – 8.\,4.\,1950 New York City@\textsc{Ehrenstein, Albert} (23.\,12.\,1886 Wien – 8.\,4.\,1950 New York City), \emph{Schriftsteller}!alte Geschichte@\strich\emph{Die alte Geschichte}|pw}, Tubutſch\pwindex{Ehrenstein, Albert 23.\,12.\,1886 Wien – 8.\,4.\,1950 New York City@\textsc{Ehrenstein, Albert} (23.\,12.\,1886 Wien – 8.\,4.\,1950 New York City), \emph{Schriftsteller}!Tubutsch@\strich\emph{Tubutsch}|pw}, Baber\pwindex{Ehrenstein, Albert 23.\,12.\,1886 Wien – 8.\,4.\,1950 New York City@\textsc{Ehrenstein, Albert} (23.\,12.\,1886 Wien – 8.\,4.\,1950 New York City), \emph{Schriftsteller}!Tod des Zehir eddin Muhammed Baber@\strich\emph{Tod des Zehir eddin Muhammed Baber}|pw} u. Tai-gin\pwindex{Ehrenstein, Albert 23.\,12.\,1886 Wien – 8.\,4.\,1950 New York City@\textsc{Ehrenstein, Albert} (23.\,12.\,1886 Wien – 8.\,4.\,1950 New York City), \emph{Schriftsteller}!Tai-Gin@\strich\emph{Tai-Gin}|pw}) an Herrn Profeſſor
                  Bie\pwindex{Bie, Oskar 9.\,2.\,1864 Breslau – 21.\,4.\,1938 Berlin@\textsc{Bie, Oskar} (9.\,2.\,1864 Breslau – 21.\,4.\,1938 Berlin), \emph{Schriftsteller, Journalist, Redakteur}|pw}.\pend
           
\pstart
           Nun weiß ich nicht, ob Sie,{ }ſehr geehrter Herr Doktor,{ }ſchon in Berlin\oindex{Berlin@\textbf{Berlin}, \emph{Hauptstadt}|pw} waren und die Liebenswürdigkeit gehabt haben, meinen
               Skizzenband »\label{K_L01887-1v}\edtext{Zuſchauer und
                  Tyrannen}{\lemma{\textnormal{\emph{Zuschauer und
                  Tyrannen}}}\Cendnote{\textnormal{Unter diesem Titel
                     veröffentlichte Ehrenstein\pwindex{Ehrenstein, Albert 23.\,12.\,1886 Wien – 8.\,4.\,1950 New York City@\textsc{Ehrenstein, Albert} (23.\,12.\,1886 Wien – 8.\,4.\,1950 New York City), \emph{Schriftsteller}|pwk} keine Novellensammlung, doch ist in seinem Nachlass ein Entwurf
                  der dafür vorgesehenen 19 Novellen überliefert.}}}\label{K_L01887-1}« – den ich Ihnen vor etwa
               14 Tagen mit einem Begleitſchreiben zukommenließ – oder eine{ }ſtrenge Auswahl meiner
               Novelletten Ihrem Verleger\pwindex{Fischer, Samuel 24.\,12.\,1859 Liptovský Mikuláš – 15.\,10.\,1934 Berlin@\textsc{Fischer, Samuel} (24.\,12.\,1859 Liptovský Mikuláš – 15.\,10.\,1934 Berlin), \emph{Verleger}|pwv}
               zu geben, oder ob dies noch bevorſteht?\pend
           
\pstart
           Jedenfalls möchte ich Sie höflichſt {\pb}bitten, nicht bloß bei dem Herrn Fiſcher\pwindex{Fischer, Samuel 24.\,12.\,1859 Liptovský Mikuláš – 15.\,10.\,1934 Berlin@\textsc{Fischer, Samuel} (24.\,12.\,1859 Liptovský Mikuláš – 15.\,10.\,1934 Berlin), \emph{Verleger}|pw},{ }ſondern, wenn es angängig iſt, auch bei dem Herrn Profeſſor Bie\pwindex{Bie, Oskar 9.\,2.\,1864 Breslau – 21.\,4.\,1938 Berlin@\textsc{Bie, Oskar} (9.\,2.\,1864 Breslau – 21.\,4.\,1938 Berlin), \emph{Schriftsteller, Journalist, Redakteur}|pw} für mich zu wirken.\pend
           
\pstart
           Für Ihre gewiß erfolgreichen Interventionen im Voraus dankend, bin ich mit dem
               Ausdrucke vorzüglichſter Hochachtung\pend
           
\pstart
           Ihr ergebenſter{\\[\baselineskip]}\spacefill\mbox{Albert Ehrenstein.}\pend
           \leftskip=0em{}\selectlanguage{ngerman}\endnumbering\briefempfaengerindex{Schnitzler, Arthur@\textsc{Schnitzler, Arthur}!zzzEhrenstein, Albert@\emph{von Albert Ehrenstein}!1909-11-221@{22. 11. 1909}|)be}\mylabel{L01887h}  \newcommand{\dateiname}{L01887}\newcommand{\titel}{Albert Ehrenstein an Arthur Schnitzler, 22. 11. 1909}\newcommand{\editorInnen}{Martin Anton Müller und Gerd-Hermann Susen}%% latex-leseansicht-abspann.tex
%% Abspann für die Leseansicht.
%% Der Schalter \ifkorrekturansicht ist bereits durch den Vorspann gesetzt.

%% latex-abspann.tex
%% Gemeinsamer Abspann für Korrekturansicht und Leseansicht.
%% Setzt den Schalter \ifkorrekturansicht voraus (gesetzt in den
%% einbindenden Dateien latex-korrekturansicht-abspann.tex bzw.
%% latex-leseansicht-abspann.tex).
%% ---------------------------------------------------------------

\normalsize

% Das esempio-Environment wird nur in der Leseansicht benötigt
\ifkorrekturansicht\else
\newenvironment{esempio}[3]%
{
    \vspace{1.5ex}
    \rlap{\underline{#1}}
    \par
    \setlength{\parindent}{0cm}
    \nopagebreak
    \leftskip=#2cm
    \rightskip=#3cm
}
{
    \par
}
\fi

\doendnotes{C}
\bigskip
\vfill

\clearpage

\footnotesize

\ifkorrekturansicht
  \lohead{\textsc{register}}
\fi

% theindex-Environment neu definieren ohne reledmac
\makeatletter
\renewenvironment{theindex}{%
  \ifkorrekturansicht
    \section*{\indexname}%
  \else
    \subsubsection*{Index der erwähnten Entitäten}%
  \fi
  \setlength{\parindent}{0pt}%
  \setlength{\parskip}{0pt plus 0.3pt}%
  \let\item\@idxitem
}{%
  \ifkorrekturansicht\clearpage\fi
}
\makeatother

\IfFileExists{\jobname-pw.ind}{\input{\jobname-pw.ind}}{}

% Quellenangabe nur in der Leseansicht
\ifkorrekturansicht\else
% Fallback-Definitionen, falls die .tex-Datei \titel etc. nicht gesetzt hat
\providecommand{\titel}{}
\providecommand{\editorInnen}{}
\providecommand{\dateiname}{\jobname}

\vspace{3cm}

\vfill

\footnotesize
\textsc{Quelle}: \titel. Herausgegeben von {\editorInnen}. In: \emph{Arthur Schnitzler: Briefwechsel mit Autorinnen und Autoren}.
 Digitale Edition, https://schnitzler-briefe.acdh.oeaw.ac.at/{\dateiname}.html (Stand \today)
\fi

\end{document}


