%% latex-leseansicht-vorspann.tex
%% Vorspann für die Leseansicht.
%% Lädt die gemeinsame Datei latex-vorspann.tex mit nicht gesetztem Schalter.

\newif\ifkorrekturansicht
\korrekturansichtfalse

\input{../tex-inputs/latex-vorspann}


\section[Stefan Zweig an Arthur Schnitzler, 10. 11. 1926]{L03672 Stefan Zweig an Arthur Schnitzler, 10. 11. 1926}
\nopagebreak\mylabel{L03672v}
\rehead{ }\normalsize\beginnumbering\briefempfaengerindex{Schnitzler, Arthur@\textsc{Schnitzler, Arthur}!zzzZweig, Stefan@\emph{von Stefan Zweig}!1926-11-101@{10. 11. 1926}|(be}
\toendnotes[C]{\smallbreak\pagebreak[2]}
\correspDesc{Versand  durch Stefan Zweig am 10. 11. 1926 in Salzburg
\newline{}Erhalt  durch Arthur Schnitzler im Zeitraum [11. 11. 1926 – 15. 11. 1926?] in Wien}\toendnotes[C]{\smallbreak}
\Standort{CUL, Schnitzler, B 118.}
\physDesc{Briefkarte, 980 Zeichen
\newline{}Handschrift: blaue Tinte, lateinische Kurrent
\newline{}Schnitzler: 1) mit Bleistift »\textsc{Zweig}«  2) mit rotem Buntstift zwei Unterstreichungen}
\buchAbdrucke{\weitereDrucke{Stefan Zweig: \emph{Briefwechsel mit Hermann Bahr, Sigmund Freud, Rainer Maria
                        Rilke und Arthur Schnitzler}. Herausgegeben von Jeffrey B. Berlin, Hans-Ulrich Lindken und Donald A. Prater. Frankfurt am Main: \emph{S. Fischer} 1987, S. 424–425.} }
\pstart
           {\pb}\textcolor{gray}{\textbf{SZ}}\hfill \substVorne{}\textsuperscript{\textcolor{gray}{\textbf{VIII.
                              KOCHGASSE 8\oindex{Wien@\textbf{Wien}!VIII., Josefstadt@\textbf{VIII., Josefstadt}!Kochgasse 8@\textbf{Kochgasse 8}, \emph{Wohngebäude}|pw}}}}\substDazwischen{}Salzburg\oindex{Salzburg@\textbf{Salzburg}, \emph{Verwaltungsgebiet}|pw}{ }10. Nov 26\substHinten{}\pend
           \vspace{0.5em}
\pstart
           Lieber verehrter Herr Doktor, gewisse Beziehungen vermag die Zeit
               nicht zu ändern – ich bin jetzt, Gottseisgeklagt, 45 Jahre alt, aber dennoch, wenn
               Sie zu mir sprechen, bin ich noch immer der schüchterne flaumbärtige Bursch, der
               rückwärts ins Parterre gedrückt \strikeout{auf} zu dem berühmten
               Dichter auf der Bühne \introOben{}empor\introOben{}sah. Ein zustimmendes Wort von
               Ihnen macht mich noch genau so beglückt und all die Freundschaft, die stolz gefühlte
                  {\pb}Sicherheit Ihrer Neigung kann nichts
               ändern an diesem dankbaren Aufblick. Und eigentlich möchte ich’s nicht anders. Fast
               alle, zu denen ich einst aufgeblickt, haben mich enttäuscht durch ihr Werk oder durch
               ihre menschliche Haltung – darum bin ich so froh, dass sich gerade an Ihnen meine
               Stellung, meine wirklich aufschauende, niemals änderte und niemals ändern wird. Ich
               liebe Sie sehr und bin froh, dass Sie es wissen: vielleicht kann ich das Alles einmal
               besser ausdrücken als gerade Blick in Blick.\pend
           
\pstart
           Dankbarst, treulichst Ihr{\\[\baselineskip]}\spacefill\mbox{Stefan Zweig}\pend
           \leftskip=0em{}\selectlanguage{ngerman}\endnumbering\briefempfaengerindex{Schnitzler, Arthur@\textsc{Schnitzler, Arthur}!zzzZweig, Stefan@\emph{von Stefan Zweig}!1926-11-101@{10. 11. 1926}|)be}\mylabel{L03672h}  \newcommand{\dateiname}{L03672}\newcommand{\titel}{Stefan Zweig an Arthur Schnitzler, 10. 11. 1926}\newcommand{\editorInnen}{Selma Jahnke und Martin Anton Müller}%% latex-leseansicht-abspann.tex
%% Abspann für die Leseansicht.
%% Der Schalter \ifkorrekturansicht ist bereits durch den Vorspann gesetzt.

%% latex-abspann.tex
%% Gemeinsamer Abspann für Korrekturansicht und Leseansicht.
%% Setzt den Schalter \ifkorrekturansicht voraus (gesetzt in den
%% einbindenden Dateien latex-korrekturansicht-abspann.tex bzw.
%% latex-leseansicht-abspann.tex).
%% ---------------------------------------------------------------

\normalsize

% Das esempio-Environment wird nur in der Leseansicht benötigt
\ifkorrekturansicht\else
\newenvironment{esempio}[3]%
{
    \vspace{1.5ex}
    \rlap{\underline{#1}}
    \par
    \setlength{\parindent}{0cm}
    \nopagebreak
    \leftskip=#2cm
    \rightskip=#3cm
}
{
    \par
}
\fi

\doendnotes{C}
\bigskip
\vfill

\clearpage

\footnotesize

\ifkorrekturansicht
  \lohead{\textsc{register}}
\fi

% theindex-Environment neu definieren ohne reledmac
\makeatletter
\renewenvironment{theindex}{%
  \ifkorrekturansicht
    \section*{\indexname}%
  \else
    \subsubsection*{Index der erwähnten Entitäten}%
  \fi
  \setlength{\parindent}{0pt}%
  \setlength{\parskip}{0pt plus 0.3pt}%
  \let\item\@idxitem
}{%
  \ifkorrekturansicht\clearpage\fi
}
\makeatother

\IfFileExists{\jobname-pw.ind}{\input{\jobname-pw.ind}}{}

% Quellenangabe nur in der Leseansicht
\ifkorrekturansicht\else
% Fallback-Definitionen, falls die .tex-Datei \titel etc. nicht gesetzt hat
\providecommand{\titel}{}
\providecommand{\editorInnen}{}
\providecommand{\dateiname}{\jobname}

\vspace{3cm}

\vfill

\footnotesize
\textsc{Quelle}: \titel. Herausgegeben von {\editorInnen}. In: \emph{Arthur Schnitzler: Briefwechsel mit Autorinnen und Autoren}.
 Digitale Edition, https://schnitzler-briefe.acdh.oeaw.ac.at/{\dateiname}.html (Stand \today)
\fi

\end{document}


