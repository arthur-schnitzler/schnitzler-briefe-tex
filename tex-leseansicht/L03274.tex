%% latex-leseansicht-vorspann.tex
%% Vorspann für die Leseansicht.
%% Lädt die gemeinsame Datei latex-vorspann.tex mit nicht gesetztem Schalter.

\newif\ifkorrekturansicht
\korrekturansichtfalse

\input{../tex-inputs/latex-vorspann}


\section[ Felix Salten an Arthur Schnitzler, 3. 9. [1897]]{L03274 Felix Salten an Arthur Schnitzler,  3. 9. [1897]}
\nopagebreak\mylabel{L03274v}
\rehead{ }\normalsize\beginnumbering\briefempfaengerindex{Schnitzler, Arthur@\textsc{Schnitzler, Arthur}!zzzSalten, Felix@\emph{von Felix Salten}!1897-09-031@{3. 9. [1897]}|(be}
\toendnotes[C]{\smallbreak\pagebreak[2]}
\correspDesc{Versand  durch Felix Salten am 3. 9. [1897] in Salzburg
\newline{}Erhalt  durch Arthur Schnitzler im Zeitraum [4. 9. 1897
                  – 8. 9. 1897?] in Wien}\toendnotes[C]{\smallbreak}
\Standort{CUL, Schnitzler, B 89, A 2.}
\physDesc{Brief, 1 Blatt, 2 Seiten, 1333 Zeichen
\newline{}Handschrift: Bleistift, lateinische Kurrent
\newline{}Schnitzler: mit Bleistift die Jahreszahl ergänzt: »97« 
\newline{}Ordnung: mit Bleistift von unbekannter Hand nummeriert: »97« }\toendnotes[C]{\smallbreak}
\pstart
           {\pb}\textcolor{gray}{\textbf{Café Tomaselli\oindex{Café Tomaselli@\textbf{Café Tomaselli}, \emph{Kaffeehaus}|pw}}}\hfill den 3. September\pend
           
\pstart
           \textcolor{gray}{\textbf{SALZBURG\oindex{Salzburg@\textbf{Salzburg}, \emph{Verwaltungsgebiet}|pw}}}\pend
           
\pstart
           \textcolor{gray}{\textbf{\textbf{gegründet 1753.}}}\pend
           \vspace{0.5em}
\pstart
           lieber Arthur, es ist so schönes Wetter, dass ich noch ein paar Tage
                  hier\oindex{Salzburg@\textbf{Salzburg}, \emph{Verwaltungsgebiet}|pwv} geblieben bin. So habe
               ich noch \label{K_L03274-1v}\edtext{Leo Fan-Jung\pwindex{Van-Jung, Leo 15.\,10.\,1866 Odessa – 2.\,7.\,1939 Riga@\textsc{Van-Jung, Leo} (15.\,10.\,1866 Odessa – 2.\,7.\,1939 Riga), \emph{Gesangspädagoge, Mathematiker}|pw}}{\lemma{\textnormal{\emph{Leo Fan-Jung}}}\Cendnote{\textnormal{Vgl. XXXX Auszeichnungsfehler: Dokument L00720 nicht gefunden.
               }}}\label{K_L03274-1} und \label{K_L03274-2v}\edtext{Goldmann\pwindex{Goldmann, Paul 31.\,1.\,1865 Breslau – 25.\,9.\,1935 Wien@\textsc{Goldmann, Paul} (31.\,1.\,1865 Breslau – 25.\,9.\,1935 Wien), \emph{Schriftsteller, Journalist}|pw} gesehen}{\lemma{\textnormal{\emph{Goldmann gesehen}}}\Cendnote{\textnormal{Siehe XXXX Auszeichnungsfehler: Dokument L02829 nicht gefunden.
               }}}\label{K_L03274-2}. G.\pwindex{Goldmann, Paul 31.\,1.\,1865 Breslau – 25.\,9.\,1935 Wien@\textsc{Goldmann, Paul} (31.\,1.\,1865 Breslau – 25.\,9.\,1935 Wien), \emph{Schriftsteller, Journalist}|pw} habe ich unverändert gefunden und
               er hat wieder einen schönen Eindruck gemacht. Das ist doch Einer, von dem man sagen
               kann, er sei ein absolut guter Mensch. Er war sehr lieb zu mir, was mir wolgethan
               hat. Im Allgemeinen ist meine Sti{\geminationm}ung nicht gut. Ich
               sehe von diesem schönen Platz aus nach Wien\oindex{Wien@\textbf{Wien}, \emph{Verwaltungsgebiet}|pw}\textcolor{gray}{,} wie in einen dunkeln, unangenehmen Nebel hinein. Ich weiß nicht,
               was werden wird, und fühle meine Sorgen, auch wenn mir am wohlsten ist, wie man den
               leisen Druck permanenter Kopfschmerzen immer spürt und sich schließlich daran
               gewöhnt. Doch möchte ich gerne einmal freier athmen können, – ich glaube, {\pb}es käme da noch Manches
               heraus, was gut an mir ist. Für den Winter mache ich mir die strengsten Pläne, und
               denke sie auch auszuführen. Der \label{K_L03274-3v}\edtext{Gedanke ans Sterben}{\lemma{\textnormal{\emph{Gedanke ans Sterben}}}\Cendnote{\textnormal{Siehe XXXX Auszeichnungsfehler: Dokument L03266 nicht gefunden.
               }}}\label{K_L03274-3}, der mir, wie Sie wissen, eine zeitlang abhanden gekommen, ist jetzt wieder
               so lebhaft in mir. Ich finde, dass das in vielen Beziehungen gut ist, der macht uns
               das Leben leichter, und macht es bewußter. Darüber wäre noch viel zu sagen.\pend
           
\pstart
           Wie geht es bei Ihnen? Arbeiten \textcolor{gray}{S}ie? Und \label{K_L03274-4v}\edtext{verläuft die Sache glatt}{\lemma{\textnormal{\emph{verläuft die Sache glatt}}}\Cendnote{\textnormal{Bei seiner Lebensgefährtin Marie Reinhard\pwindex{Reinhard, Marie 13.\,3.\,1871 Wien – 18.\,3.\,1899 ebd.@\textsc{Reinhard, Marie} (13.\,3.\,1871 Wien – 18.\,3.\,1899 ebd.), \emph{Gesangspädagogin}|pwk} stand die Entbindung kurz bevor. Das Kind\pwindex{?? [Totgeborener Sohn von Arthur Schnitzler und Marie Reinhard] 24.\,9.\,1897 Endresstraße 68 – 24.\,9.\,1897 ebd.@\textsc{?? [Totgeborener Sohn von Arthur Schnitzler und Marie Reinhard]} (24.\,9.\,1897 Endresstraße 68 – 24.\,9.\,1897 ebd.)|pwkv} kam am 24. 9. 1897 tot auf die Welt. }}}\label{K_L03274-4}?
               Schreiben Sie mir ein Wort darüber. Ich bin voraussichtlich Dienstag in Wien\oindex{Wien@\textbf{Wien}, \emph{Verwaltungsgebiet}|pw}. Herzliche Grüße\pend
           
\pstart
           Ihr{\\[\baselineskip]}\spacefill\mbox{Salten}\pend
           \leftskip=0em{}
\pstart
           \noindent{}Ich wohne jetzt: Erzherzog Karl\oindex{Hotel Erzherzog Karl [Salzburg]@\textbf{Hotel Erzherzog Karl [Salzburg]}, \emph{Hotel}|pw}\pend
           \selectlanguage{ngerman}\endnumbering\briefempfaengerindex{Schnitzler, Arthur@\textsc{Schnitzler, Arthur}!zzzSalten, Felix@\emph{von Felix Salten}!1897-09-031@{3. 9. [1897]}|)be}\mylabel{L03274h}  \newcommand{\dateiname}{L03274}\newcommand{\titel}{Felix Salten an Arthur Schnitzler, 3. 9. [1897]}\newcommand{\editorInnen}{Martin Anton Müller und Laura Untner}%% latex-leseansicht-abspann.tex
%% Abspann für die Leseansicht.
%% Der Schalter \ifkorrekturansicht ist bereits durch den Vorspann gesetzt.

%% latex-abspann.tex
%% Gemeinsamer Abspann für Korrekturansicht und Leseansicht.
%% Setzt den Schalter \ifkorrekturansicht voraus (gesetzt in den
%% einbindenden Dateien latex-korrekturansicht-abspann.tex bzw.
%% latex-leseansicht-abspann.tex).
%% ---------------------------------------------------------------

\normalsize

% Das esempio-Environment wird nur in der Leseansicht benötigt
\ifkorrekturansicht\else
\newenvironment{esempio}[3]%
{
    \vspace{1.5ex}
    \rlap{\underline{#1}}
    \par
    \setlength{\parindent}{0cm}
    \nopagebreak
    \leftskip=#2cm
    \rightskip=#3cm
}
{
    \par
}
\fi

\doendnotes{C}
\bigskip
\vfill

\clearpage

\footnotesize

\ifkorrekturansicht
  \lohead{\textsc{register}}
\fi

% theindex-Environment neu definieren ohne reledmac
\makeatletter
\renewenvironment{theindex}{%
  \ifkorrekturansicht
    \section*{\indexname}%
  \else
    \subsubsection*{Index der erwähnten Entitäten}%
  \fi
  \setlength{\parindent}{0pt}%
  \setlength{\parskip}{0pt plus 0.3pt}%
  \let\item\@idxitem
}{%
  \ifkorrekturansicht\clearpage\fi
}
\makeatother

\IfFileExists{\jobname-pw.ind}{\input{\jobname-pw.ind}}{}

% Quellenangabe nur in der Leseansicht
\ifkorrekturansicht\else
% Fallback-Definitionen, falls die .tex-Datei \titel etc. nicht gesetzt hat
\providecommand{\titel}{}
\providecommand{\editorInnen}{}
\providecommand{\dateiname}{\jobname}

\vspace{3cm}

\vfill

\footnotesize
\textsc{Quelle}: \titel. Herausgegeben von {\editorInnen}. In: \emph{Arthur Schnitzler: Briefwechsel mit Autorinnen und Autoren}.
 Digitale Edition, https://schnitzler-briefe.acdh.oeaw.ac.at/{\dateiname}.html (Stand \today)
\fi

\end{document}


