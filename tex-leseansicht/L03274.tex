%% latex-leseansicht-vorspann.tex
%% Vorspann für die Leseansicht.
%% Lädt die gemeinsame Datei latex-vorspann.tex mit nicht gesetztem Schalter.

\newif\ifkorrekturansicht
\korrekturansichtfalse

\input{../tex-inputs/latex-vorspann}

\begin{center}
            \textcolor{red}{ENTWURF, NICHT FERTIG KORRIGIERT}
                      \end{center}
            
         
         \renewcommand{\erwaehntePersonen}{Personen: Paul Goldmann, Leo Van-Jung}
         \renewcommand{\erwaehnteOrte}{Orte: Café Tomaselli, Hotel Erzherzog Karl, Salzburg, Wien}
         \renewcommand{\erwaehnteWerke}{}
               \section[Felix Salten an Arthur Schnitzler, 3. 9. {[}1897{]}]{ Felix Salten an Arthur Schnitzler, 3. 9. {[}1897{]}}\nopagebreak\mylabel{v}\rehead{ }\begin{ledgroupsized}[t]{13cm}\normalsize\beginnumbering \toendnotes[C]{\smallbreak\pagebreak[2]} \Standort{CUL, Schnitzler, B 89, A 2.}
\physDesc{Brief, 1 Blatt, 2 Seiten, 1336 Zeichen
\newline{}Handschrift: Bleistift, lateinische Kurrent
\newline{}Ordnung: mit Bleistift von unbekannter Hand nummeriert:
                                    »97« }\toendnotes[C]{\smallbreak}\pstart
           \noindent{}{\pb}\textcolor{gray}{\textbf{Café Tomaselli\oindex{Cafe Tomaselli@\textbf{Café Tomaselli}|pw}}}\hfill den 3. September\pend
           \pstart
           \textcolor{gray}{\textbf{SALZBURG\oindex{Salzburg@\textbf{Salzburg}|pw}}}\pend
           \pstart
           \textcolor{gray}{\textbf{* gegründet 1753 *}}\pend
           \pstart
           Lieber Arthur, es ist so schönes Wetter, dass ich noch ein paar Tage
                  hier\oindex{Salzburg@\textbf{Salzburg}|pw} geblieben bin. So habe ich noch Leo Fan Jung\pwindex{Van-Jung, Leo 15.10.1866 – 02.07.1939@\textsc{Van-Jung, Leo} (15.10.1866 – 02.07.1939), \emph{Gesangspädagoge, Mathematiker}|pw} und \label{K_L03274-1v}\edtext{Goldmann\pwindex{Goldmann, Paul 31.01.1865 – 25.09.1935@\textsc{Goldmann, Paul} (31.01.1865 – 25.09.1935), \emph{Schriftsteller, Journalist}|pw} gesehen}{\lemma{\textnormal{\emph{Goldmann gesehen}}}\Cendnote{\textnormal{siehe Paul Goldmann an Arthur Schnitzler, 15. 10. [1897]}}}\label{K_L03274-1h}. G.\pwindex{Goldmann, Paul 31.01.1865 – 25.09.1935@\textsc{Goldmann, Paul} (31.01.1865 – 25.09.1935), \emph{Schriftsteller, Journalist}|pw} habe ich unverändert gefunden und
               er hat wieder einen schönen Eindruck gemacht. Das ist doch Einer, von dem man sagen
               kann, er sei ein absolut guter Mensch. Er war sehr lieb zu mir, was mir wolgethan
               hat. Im Allgemeinen ist meine Sti{\geminationm}ung nicht gut. Ich
               sehe von diesem schönen Platz aus nach Wien\oindex{Wien@\textbf{Wien}|pw} wie in
               einen dunkeln, unangenehmen Nebel hinein. Ich weiß nicht, was werden wird, und fühle
               meine Sorgen, auch wenn mir am wohlsten ist, wie man den leisen Druck permanenter
               Kopfschmerzen immer spürt und sich schließlich daran gewöhnt. Doch möchte ich gerne
               einmal freier athmen können, – ich glaube, {\pb}es käme da noch Manches
               heraus, was gut an mir ist. Für den Winter mache ich mir die stengsten Pläne, und
               denke sie auch auszuführen. Der Gedanke ans Sterben, der mir, wie Sie wissen, eine
               zeitlang abhanden gekommen, ist jetzt wieder so lebhaft in mir. Ich finde, dass das
               in vielen Beziehungen gut ist, der macht uns das Leben leichter, und macht es
               bewußter. Darüber wäre noch viel zu sagen.\pend
           \pstart
           Wie geht es bei Ihnen? Arbeiten \textcolor{gray}{Sie}? Und verläuft die Sache glatt?
               Schreiben Sie mir ein Wort darüber. Ich bin voraussichtlich Dinestag in Wien\oindex{Wien@\textbf{Wien}|pw}. Herzliche Grüße\pend
           \pstart
           Ihr{\\[\baselineskip]}\spacefill\mbox{Salten}\pend
           \leftskip=0em{}\pstart
           \noindent{}Ich wohne jetzt: Erzherzog Karl\oindex{Hotel Erzherzog Karl@\textbf{Hotel Erzherzog Karl}|pw}\pend
           
         
         \endnumbering\mylabel{h}\end{ledgroupsized}\begin{anhang}\end{anhang}\newcommand{\dateiname}{L03274}\newcommand{\titel}{Felix Salten an Arthur Schnitzler, 3. 9. [1897]}\newcommand{\editorInnen}{Martin Anton Müller und Laura Untner}%% latex-leseansicht-abspann.tex
%% Abspann für die Leseansicht.
%% Der Schalter \ifkorrekturansicht ist bereits durch den Vorspann gesetzt.

%% latex-abspann.tex
%% Gemeinsamer Abspann für Korrekturansicht und Leseansicht.
%% Setzt den Schalter \ifkorrekturansicht voraus (gesetzt in den
%% einbindenden Dateien latex-korrekturansicht-abspann.tex bzw.
%% latex-leseansicht-abspann.tex).
%% ---------------------------------------------------------------

\normalsize

% Das esempio-Environment wird nur in der Leseansicht benötigt
\ifkorrekturansicht\else
\newenvironment{esempio}[3]%
{
    \vspace{1.5ex}
    \rlap{\underline{#1}}
    \par
    \setlength{\parindent}{0cm}
    \nopagebreak
    \leftskip=#2cm
    \rightskip=#3cm
}
{
    \par
}
\fi

\doendnotes{C}
\bigskip
\vfill

\clearpage

\footnotesize

\ifkorrekturansicht
  \lohead{\textsc{register}}
\fi

% theindex-Environment neu definieren ohne reledmac
\makeatletter
\renewenvironment{theindex}{%
  \ifkorrekturansicht
    \section*{\indexname}%
  \else
    \subsubsection*{Index der erwähnten Entitäten}%
  \fi
  \setlength{\parindent}{0pt}%
  \setlength{\parskip}{0pt plus 0.3pt}%
  \let\item\@idxitem
}{%
  \ifkorrekturansicht\clearpage\fi
}
\makeatother

\IfFileExists{\jobname-pw.ind}{\input{\jobname-pw.ind}}{}

% Quellenangabe nur in der Leseansicht
\ifkorrekturansicht\else
% Fallback-Definitionen, falls die .tex-Datei \titel etc. nicht gesetzt hat
\providecommand{\titel}{}
\providecommand{\editorInnen}{}
\providecommand{\dateiname}{\jobname}

\vspace{3cm}

\vfill

\footnotesize
\textsc{Quelle}: \titel. Herausgegeben von {\editorInnen}. In: \emph{Arthur Schnitzler: Briefwechsel mit Autorinnen und Autoren}.
 Digitale Edition, https://schnitzler-briefe.acdh.oeaw.ac.at/{\dateiname}.html (Stand \today)
\fi

\end{document}


      