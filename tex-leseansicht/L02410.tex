%% latex-leseansicht-vorspann.tex
%% Vorspann für die Leseansicht.
%% Lädt die gemeinsame Datei latex-vorspann.tex mit nicht gesetztem Schalter.

\newif\ifkorrekturansicht
\korrekturansichtfalse

\input{../tex-inputs/latex-vorspann}


\section[Arthur Schnitzler an Richard Beer-Hofmann, 27. 2. 1924]{L02410 Arthur Schnitzler an Richard Beer-Hofmann, 27. 2. 1924}
\nopagebreak\mylabel{L02410v}
\rehead{ }\normalsize\beginnumbering\briefempfaengerindex{Beer-Hofmann, Richard@\textsc{Beer-Hofmann, Richard}!zzzSchnitzler, Arthur@\emph{von Arthur Schnitzler}!1924-02-271@{27. 2. 1924}|(be}
\toendnotes[C]{\smallbreak\pagebreak[2]}
\correspDesc{Versand  durch Arthur Schnitzler am 27. 2. 1924 in Wien
\newline{}Erhalt  durch Richard Beer-Hofmann im Zeitraum [27. 2. 1924
                  – 2. 3. 1924?] in Wien}\toendnotes[C]{\smallbreak}
\Standort{YCGL, MSS 31.}
\physDesc{Brief, 1 Blatt, 1 Seite, 922 Zeichen
\newline{}Schreibmaschine
\newline{}Handschrift: Bleistift, lateinische Kurrent (\noindent{}Anrede, Einfügung, Schlussformel, Unterschrift)}\toendnotes[C]{\smallbreak}
\pstart
           \raggedleft{}{\pb}Wien\oindex{Wien@\textbf{Wien}, \emph{Verwaltungsgebiet}|pw}, 27. 2. 1924.\pend
           
\pstart\center{}{[}hs.:{]} lieber Richard\pend\vspace{0.5em}
\pstart
           {[}ms.:{]} An den Bundestheaterkommissär\pwindex{Renkin, Albert 11.\,10.\,1878 Verviers – 20.\,12.\,1962 Salzburg@\textsc{Renkin, Albert} (11.\,10.\,1878 Verviers – 20.\,12.\,1962 Salzburg), \emph{Leitender Beamter}|pwv} etc.\pend
           
\pstart
           Auf unsere vor zirka vier Wochen gestellte Frage bezüglich der perzentuellen Höhe, in
               der Lustbarkeitssteuer und Pensionszuschläge von den Tantièmen abgezogen werden, ist
               leider bisher noch keine Antwort eingelangt. Daher gestatten wir uns unsere Frage zu
               wiederholen, ebenso wie das Ersuchen um getrennte Aufstellung von Tageseinnahme und
               Abonnementsquote, so wie diese in den früheren Verrechnungen üblich war. Wir möchten
               bei dieser Gele{[}ge{]}nheit nicht unser Befremden verhehlen, dass die
               Erledigung dieser Angelegenheit, insbesondere aber die Beantwortung unserer wohl
               begründeten Frage bezüglich der per\strikeout{n}zentuellen Abzüge
               (in welchem Falle die einfache Mitteilung von zwei Ziffern genügt hätte) so lange
               hinausgezogen wird.\pend
           
\pstart
           \raggedleft{}{[}hs.:{]} \uline{          } Unterschrift\pend
           
\pstart
           {[}ms.:{]} Ich erbitte Ihr Einverständnis zur Absendung dieses
               Briefes durch Unterzeichnung dieses Blattes.\pend
           
\pstart
           {[}hs.:{]} Herzlichst{\\[\baselineskip]}Ihr{\\[\baselineskip]}\spacefill\mbox{A.}\pend
           \leftskip=0em{}\selectlanguage{ngerman}\endnumbering\briefempfaengerindex{Beer-Hofmann, Richard@\textsc{Beer-Hofmann, Richard}!zzzSchnitzler, Arthur@\emph{von Arthur Schnitzler}!1924-02-271@{27. 2. 1924}|)be}\mylabel{L02410h}  \newcommand{\dateiname}{L02410}\newcommand{\titel}{Arthur Schnitzler an Richard Beer-Hofmann, 27. 2. 1924}\newcommand{\editorInnen}{Martin Anton Müller und Gerd-Hermann Susen}%% latex-leseansicht-abspann.tex
%% Abspann für die Leseansicht.
%% Der Schalter \ifkorrekturansicht ist bereits durch den Vorspann gesetzt.

%% latex-abspann.tex
%% Gemeinsamer Abspann für Korrekturansicht und Leseansicht.
%% Setzt den Schalter \ifkorrekturansicht voraus (gesetzt in den
%% einbindenden Dateien latex-korrekturansicht-abspann.tex bzw.
%% latex-leseansicht-abspann.tex).
%% ---------------------------------------------------------------

\normalsize

% Das esempio-Environment wird nur in der Leseansicht benötigt
\ifkorrekturansicht\else
\newenvironment{esempio}[3]%
{
    \vspace{1.5ex}
    \rlap{\underline{#1}}
    \par
    \setlength{\parindent}{0cm}
    \nopagebreak
    \leftskip=#2cm
    \rightskip=#3cm
}
{
    \par
}
\fi

\doendnotes{C}
\bigskip
\vfill

\clearpage

\footnotesize

\ifkorrekturansicht
  \lohead{\textsc{register}}
\fi

% theindex-Environment neu definieren ohne reledmac
\makeatletter
\renewenvironment{theindex}{%
  \ifkorrekturansicht
    \section*{\indexname}%
  \else
    \subsubsection*{Index der erwähnten Entitäten}%
  \fi
  \setlength{\parindent}{0pt}%
  \setlength{\parskip}{0pt plus 0.3pt}%
  \let\item\@idxitem
}{%
  \ifkorrekturansicht\clearpage\fi
}
\makeatother

\IfFileExists{\jobname-pw.ind}{\input{\jobname-pw.ind}}{}

% Quellenangabe nur in der Leseansicht
\ifkorrekturansicht\else
% Fallback-Definitionen, falls die .tex-Datei \titel etc. nicht gesetzt hat
\providecommand{\titel}{}
\providecommand{\editorInnen}{}
\providecommand{\dateiname}{\jobname}

\vspace{3cm}

\vfill

\footnotesize
\textsc{Quelle}: \titel. Herausgegeben von {\editorInnen}. In: \emph{Arthur Schnitzler: Briefwechsel mit Autorinnen und Autoren}.
 Digitale Edition, https://schnitzler-briefe.acdh.oeaw.ac.at/{\dateiname}.html (Stand \today)
\fi

\end{document}


