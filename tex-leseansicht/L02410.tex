%% latex-korrekturansicht-vorspann.tex
%% Vorspann für die Korrekturansicht.
%% Lädt die gemeinsame Datei latex-vorspann.tex mit gesetztem Schalter.

\newif\ifkorrekturansicht
\korrekturansichttrue

\input{../tex-inputs/latex-vorspann}


\section[Arthur Schnitzler an Richard Beer-Hofmann, 27. 2. 1924]{L02410 Arthur Schnitzler an Richard Beer-Hofmann, 27. 2. 1924}
\nopagebreak\mylabel{L02410v}
\rehead{ }\normalsize\beginnumbering\briefempfaengerindex{Beer-Hofmann, Richard@\textsc{Beer-Hofmann, Richard}!zzzSchnitzler, Arthur@\emph{von Arthur Schnitzler}!1924-02-271@{27. 2. 1924}|(be}
\toendnotes[C]{\smallbreak\pagebreak[2]}\Standort{YCGL, MSS 31.}
\physDesc{Brief, 1 Blatt, 1 Seite, 922 Zeichen
\newline{}Schreibmaschine
\newline{}Handschrift: Bleistift, lateinische Kurrent (\noindent{}Anrede, Einfügung, Schlussformel, Unterschrift)}\toendnotes[C]{\smallbreak}
\pstart
           \raggedleft{}{\pb}Wien\oindex{Wien@\textbf{Wien}, \emph{A.ADM2}|pw}, 27. 2. 1924.\pend
           
\pstart\center{}{[}hs.:{]} lieber Richard\pend\vspace{0.5em}
\pstart
           {[}ms.:{]} An den Bundestheaterkommissär\pwindex{Renkin, Albert 11.10.1878 – 20.12.1962@\textsc{Renkin, Albert} (11.10.1878 – 20.12.1962), \emph{Leitender Beamter/Leitende Beamte}|pwv} etc.\pend
           
\pstart
           Auf unsere vor zirka vier Wochen gestellte Frage bezüglich der perzentuellen Höhe, in
               der Lustbarkeitssteuer und Pensionszuschläge von den Tantièmen abgezogen werden, ist
               leider bisher noch keine Antwort eingelangt. Daher gestatten wir uns unsere Frage zu
               wiederholen, ebenso wie das Ersuchen um getrennte Aufstellung von Tageseinnahme und
               Abonnementsquote, so wie diese in den früheren Verrechnungen üblich war. Wir möchten
               bei dieser Gele{[}ge{]}nheit nicht unser Befremden verhehlen, dass die
               Erledigung dieser Angelegenheit, insbesondere aber die Beantwortung unserer wohl
               begründeten Frage bezüglich der per\strikeout{n}zentuellen Abzüge
               (in welchem Falle die einfache Mitteilung von zwei Ziffern genügt hätte) so lange
               hinausgezogen wird.\pend
           
\pstart
           \raggedleft{}{[}hs.:{]} \uline{          } Unterschrift\pend
           
\pstart
           {[}ms.:{]} Ich erbitte Ihr Einverständnis zur Absendung dieses
               Briefes durch Unterzeichnung dieses Blattes.\pend
           
\pstart
           {[}hs.:{]} Herzlichst{\\[\baselineskip]}Ihr{\\[\baselineskip]}\spacefill\mbox{A.}\pend
           \leftskip=0em{}\selectlanguage{ngerman}\endnumbering\briefempfaengerindex{Beer-Hofmann, Richard@\textsc{Beer-Hofmann, Richard}!zzzSchnitzler, Arthur@\emph{von Arthur Schnitzler}!1924-02-271@{27. 2. 1924}|)be}\mylabel{L02410h}  \normalsize

\doendnotes{C}
\bigskip
\vfill

\clearpage

\footnotesize

\lohead{\textsc{register}}

% Definiere theindex-Environment komplett neu ohne reledmac
\makeatletter
\renewenvironment{theindex}{%
  \section*{\indexname}%
  \setlength{\parindent}{0pt}%
  \setlength{\parskip}{0pt plus 0.3pt}%
  \let\item\@idxitem
}{%
  \clearpage
}
\makeatother

\IfFileExists{\jobname-pw.ind}{\input{\jobname-pw.ind}}{}

\end{document}

      