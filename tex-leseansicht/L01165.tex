%% latex-leseansicht-vorspann.tex
%% Vorspann für die Leseansicht.
%% Lädt die gemeinsame Datei latex-vorspann.tex mit nicht gesetztem Schalter.

\newif\ifkorrekturansicht
\korrekturansichtfalse

\input{../tex-inputs/latex-vorspann}


         
         \renewcommand{\erwaehntePersonen}{Personen: Georg Brandes, Paul Goldmann}
         \renewcommand{\erwaehnteOrte}{Orte: Berlin, Dänemark, Hotel Savoy, Kopenhagen, Schloss Welsperg, Welsberg-Taisten, Wildbad Waldbrunn}
         \renewcommand{\erwaehnteWerke}{}
               \section[Paul Goldmann und Arthur Schnitzler an Georg Brandes, 21. 8. 1901]{ Paul Goldmann und Arthur Schnitzler an Georg Brandes,
               21. 8. 1901}\nopagebreak\mylabel{v}\rehead{ }\begin{ledgroupsized}[t]{13cm}\normalsize\beginnumbering \toendnotes[C]{\smallbreak\pagebreak[2]} \Standort{Kopenhagen, Det Kongelige Bibliotek, Georg Brandes Arkiv, box 125.}
\physDesc{Bildpostkarte, 121 Zeichen
\newline{}Handschrift Arthur Schnitzler: schwarze Tinte\newline{}Handschrift Paul Goldmann: schwarze Tinte, lateinische Kurrent
\newline{}Versand: 1) Stempel: »\nobreak{}\oindex{Wildbad Waldbrunn@\textbf{Wildbad Waldbrunn}|pwk}Böhm’s Hotel »Wildbad«, Waldbrunn
                                       Pustertal u. Hôtel, 22. Aug. 1901\nobreak{}«.   2) Stempel: »\nobreak{}\oindex{Welsberg-Taisten@\textbf{Welsberg-Taisten}|pwk}We{[}lsberg{]}, 22. 8. 0\textcolor{gray}{1}\nobreak{}«.  3) Stempel: »\nobreak{}25. \textcolor{gray}{8}{[}. 1901{]}, 7 L\nobreak{}«.  4) Stempel: »\nobreak{}\oindex{Kopenhagen@\textbf{Kopenhagen}|pwk}Kjobenhavn, 2\textcolor{gray}{×}. {[}8. 1901{]}, 30MB\nobreak{}«.  5) nachgesandt nach »Savoy Hotel\oindex{Hotel Savoy@\textbf{Hotel Savoy}|pw}, \uuline{Berlin}\oindex{Berlin@\textbf{Berlin}|pw}{ }\uline{W}«
\newline{}Ordnung: 1) mit schwarzem Stift von unbekannter Hand: »S«,
                                 eventuell zur alphabetischen Einordnung  2) mit Bleistift von unbekannter Hand nummeriert:
                                    »28.«, datiert: »22/8–07« und Vermerk: »Schnitzler«}\buchAbdrucke{\weitereDrucke{\emph{31. August [1907].} In: Georg Brandes, Arthur Schnitzler: \emph{Ein Briefwechsel}. Hg. Kurt Bergel. Bern: \emph{Francke} 1956, S. 96.} }\pstart{}{\pb}Herrn\pend{}\pstart{}Dr. Georg Brandes\pend{}\pstart{}Kopenhagen\oindex{Kopenhagen@\textbf{Kopenhagen}|pw}\pend{}\pstart{}Dänemark\oindex{Daenemark@\textbf{Dänemark}|pw}\pend{}{\bigskip}\pstart
           \noindent{}\centering{}\textcolor{gray}{\textbf{{\pb}Blick auf Schloss Welsberg\oindex{Schloss Welsperg@\textbf{Schloss Welsperg}|pw} vom Hôtel Wildbad
                        Waldbrunn\oindex{Wildbad Waldbrunn@\textbf{Wildbad Waldbrunn}|pw} aus}}\pend
           \pstart
           \raggedleft{}Welsberg im Pusterthal\oindex{Welsberg-Taisten@\textbf{Welsberg-Taisten}|pw}{\\}21. August\pend
           \pstart Herzliche Grüsse \spacefill\mbox{Dr. Paul Goldmann}\pend{}\pstart
           \noindent{}\spacefill\mbox{{[}hs. Schnitzler:{]} ArthurSchnitzler}\pend
           
         
         \endnumbering\mylabel{h}\end{ledgroupsized}  \newcommand{\dateiname}{L01165}\newcommand{\titel}{Paul Goldmann und Arthur Schnitzler an Georg Brandes, 21. 8. 1901}\newcommand{\editorInnen}{Martin Anton Müller und Gerd-Hermann Susen}%% latex-leseansicht-abspann.tex
%% Abspann für die Leseansicht.
%% Der Schalter \ifkorrekturansicht ist bereits durch den Vorspann gesetzt.

%% latex-abspann.tex
%% Gemeinsamer Abspann für Korrekturansicht und Leseansicht.
%% Setzt den Schalter \ifkorrekturansicht voraus (gesetzt in den
%% einbindenden Dateien latex-korrekturansicht-abspann.tex bzw.
%% latex-leseansicht-abspann.tex).
%% ---------------------------------------------------------------

\normalsize

% Das esempio-Environment wird nur in der Leseansicht benötigt
\ifkorrekturansicht\else
\newenvironment{esempio}[3]%
{
    \vspace{1.5ex}
    \rlap{\underline{#1}}
    \par
    \setlength{\parindent}{0cm}
    \nopagebreak
    \leftskip=#2cm
    \rightskip=#3cm
}
{
    \par
}
\fi

\doendnotes{C}
\bigskip
\vfill

\clearpage

\footnotesize

\ifkorrekturansicht
  \lohead{\textsc{register}}
\fi

% theindex-Environment neu definieren ohne reledmac
\makeatletter
\renewenvironment{theindex}{%
  \ifkorrekturansicht
    \section*{\indexname}%
  \else
    \subsubsection*{Index der erwähnten Entitäten}%
  \fi
  \setlength{\parindent}{0pt}%
  \setlength{\parskip}{0pt plus 0.3pt}%
  \let\item\@idxitem
}{%
  \ifkorrekturansicht\clearpage\fi
}
\makeatother

\IfFileExists{\jobname-pw.ind}{\input{\jobname-pw.ind}}{}

% Quellenangabe nur in der Leseansicht
\ifkorrekturansicht\else
% Fallback-Definitionen, falls die .tex-Datei \titel etc. nicht gesetzt hat
\providecommand{\titel}{}
\providecommand{\editorInnen}{}
\providecommand{\dateiname}{\jobname}

\vspace{3cm}

\vfill

\footnotesize
\textsc{Quelle}: \titel. Herausgegeben von {\editorInnen}. In: \emph{Arthur Schnitzler: Briefwechsel mit Autorinnen und Autoren}.
 Digitale Edition, https://schnitzler-briefe.acdh.oeaw.ac.at/{\dateiname}.html (Stand \today)
\fi

\end{document}


      