%% latex-korrekturansicht-vorspann.tex
%% Vorspann für die Korrekturansicht.
%% Lädt die gemeinsame Datei latex-vorspann.tex mit gesetztem Schalter.

\newif\ifkorrekturansicht
\korrekturansichttrue

\input{../tex-inputs/latex-vorspann}


\section[Richard Dehmel an Arthur Schnitzler, 8. 6. 1908]{L01775 Richard Dehmel an Arthur Schnitzler, 8. 6. 1908}
\nopagebreak\mylabel{L01775v}
\rehead{ }\normalsize\beginnumbering\briefempfaengerindex{Schnitzler, Arthur@\textsc{Schnitzler, Arthur}!zzzDehmel, Richard@\emph{von Richard Dehmel}!1908-06-081@{8. 6. 1908}|(be}
\toendnotes[C]{\smallbreak\pagebreak[2]}\Standort{DLA, A:Schnitzler, HS.NZ85.1.2730.}
\physDesc{Brief, maschinenschriftliche Abschrift1 Blatt, 1 Seite, 410 Zeichen
\newline{}Schreibmaschine
\newline{}Zusatz: Original nicht nachweisbar }\toendnotes[C]{\smallbreak}
\pstart
           \raggedleft{}{\pb}Braunwald\oindex{Braunwald@\textbf{Braunwald}, \emph{P.PPL}|pw},
                  8. 6. 1908.\pend
           
\pstart{}Verehrter Herr Schnitzler!\pend\vspace{0.5em}
\pstart
           Möge der Titel Ihres Romans\pwindex{Weg ins Freie. Roman@\emph{Der Weg ins Freie. Roman}|pwv}
               mir ein Omen sein. Ich sitze nämlich auf einem Schweiz\oindex{Schweiz@\textbf{Schweiz}, \emph{A.PCLI}|pw}er Berg in dickem Nebel, und es wird wohl noch eine Woche dauern, bis
               der Regen herunter ist. Da kann ich also Ihrem »Weg
                  ins Freie\pwindex{Weg ins Freie. Roman@\emph{Der Weg ins Freie. Roman}|pw}« – (zum Glück konnte ich mich nicht entschliessen, ihn in der Neuen Rdschau\pwindex{neue Rundschau@\emph{Die neue Rundschau}|pw} zu lesen) – die verständnisvollste
               Andacht widmen.\pend
           
\pstart
           Mit schönstem Dank{\\[\baselineskip]}Ihr{\\[\baselineskip]}\spacefill\mbox{Dehmel.}\pend
           \leftskip=0em{}\selectlanguage{ngerman}\endnumbering\briefempfaengerindex{Schnitzler, Arthur@\textsc{Schnitzler, Arthur}!zzzDehmel, Richard@\emph{von Richard Dehmel}!1908-06-081@{8. 6. 1908}|)be}\mylabel{L01775h}  \normalsize

\doendnotes{C}
\bigskip
\vfill

\clearpage

\footnotesize

\lohead{\textsc{register}}

% Definiere theindex-Environment komplett neu ohne reledmac
\makeatletter
\renewenvironment{theindex}{%
  \section*{\indexname}%
  \setlength{\parindent}{0pt}%
  \setlength{\parskip}{0pt plus 0.3pt}%
  \let\item\@idxitem
}{%
  \clearpage
}
\makeatother

\IfFileExists{\jobname-pw.ind}{\input{\jobname-pw.ind}}{}

\end{document}

      