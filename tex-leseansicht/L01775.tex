%% latex-leseansicht-vorspann.tex
%% Vorspann für die Leseansicht.
%% Lädt die gemeinsame Datei latex-vorspann.tex mit nicht gesetztem Schalter.

\newif\ifkorrekturansicht
\korrekturansichtfalse

\input{../tex-inputs/latex-vorspann}


\section[Richard Dehmel an Arthur Schnitzler, 8. 6. 1908]{L01775 Richard Dehmel an Arthur Schnitzler, 8. 6. 1908}
\nopagebreak\mylabel{L01775v}
\rehead{ }\normalsize\beginnumbering\briefempfaengerindex{Schnitzler, Arthur@\textsc{Schnitzler, Arthur}!zzzDehmel, Richard@\emph{von Richard Dehmel}!1908-06-081@{8. 6. 1908}|(be}
\toendnotes[C]{\smallbreak\pagebreak[2]}
\correspDesc{Versand  durch Richard Dehmel am 8. 6. 1908 in Braunwald
\newline{}Erhalt  durch Arthur Schnitzler im Zeitraum [9. 6. 1908
                  – 13. 6. 1908?] in Wien}\toendnotes[C]{\smallbreak}
\Standort{DLA, A:Schnitzler, HS.NZ85.1.2730.}
\physDesc{Brief, maschinenschriftliche Abschrift, 1 Blatt, 1 Seite, 410 Zeichen
\newline{}Schreibmaschine
\newline{}Zusatz: Original nicht nachweisbar }\toendnotes[C]{\smallbreak}
\pstart
           \raggedleft{}{\pb}Braunwald\oindex{Braunwald@\textbf{Braunwald}|pw},
                  8. 6. 1908.\pend
           
\pstart{}Verehrter Herr Schnitzler!\pend\vspace{0.5em}
\pstart
           Möge der Titel Ihres Romans\pwindex{Schnitzler, Arthur 15.\,5.\,1862 Wien – 21.\,10.\,1931 ebd.@\textsc{Schnitzler, Arthur} (15.\,5.\,1862 Wien – 21.\,10.\,1931 ebd.), \emph{Schriftsteller, Mediziner}!Weg ins Freie. Roman@\strich\emph{Der Weg ins Freie. Roman}|pwv}
               mir ein Omen sein. Ich sitze nämlich auf einem Schweiz\oindex{Schweiz@\textbf{Schweiz}|pw}er Berg in dickem Nebel, und es wird wohl noch eine Woche dauern, bis
               der Regen herunter ist. Da kann ich also Ihrem »Weg
                  ins Freie\pwindex{Schnitzler, Arthur 15.\,5.\,1862 Wien – 21.\,10.\,1931 ebd.@\textsc{Schnitzler, Arthur} (15.\,5.\,1862 Wien – 21.\,10.\,1931 ebd.), \emph{Schriftsteller, Mediziner}!Weg ins Freie. Roman@\strich\emph{Der Weg ins Freie. Roman}|pw}« – (zum Glück konnte ich mich nicht entschliessen, ihn in der Neuen Rdschau\pwindex{neue Rundschau@\emph{Die neue Rundschau}|pw} zu lesen) – die verständnisvollste
               Andacht widmen.\pend
           
\pstart
           Mit schönstem Dank{\\[\baselineskip]}Ihr{\\[\baselineskip]}\spacefill\mbox{Dehmel.}\pend
           \leftskip=0em{}\selectlanguage{ngerman}\endnumbering\briefempfaengerindex{Schnitzler, Arthur@\textsc{Schnitzler, Arthur}!zzzDehmel, Richard@\emph{von Richard Dehmel}!1908-06-081@{8. 6. 1908}|)be}\mylabel{L01775h}  \newcommand{\dateiname}{L01775}\newcommand{\titel}{Richard Dehmel an Arthur Schnitzler, 8. 6. 1908}\newcommand{\editorInnen}{Martin Anton Müller und Gerd-Hermann Susen}%% latex-leseansicht-abspann.tex
%% Abspann für die Leseansicht.
%% Der Schalter \ifkorrekturansicht ist bereits durch den Vorspann gesetzt.

%% latex-abspann.tex
%% Gemeinsamer Abspann für Korrekturansicht und Leseansicht.
%% Setzt den Schalter \ifkorrekturansicht voraus (gesetzt in den
%% einbindenden Dateien latex-korrekturansicht-abspann.tex bzw.
%% latex-leseansicht-abspann.tex).
%% ---------------------------------------------------------------

\normalsize

% Das esempio-Environment wird nur in der Leseansicht benötigt
\ifkorrekturansicht\else
\newenvironment{esempio}[3]%
{
    \vspace{1.5ex}
    \rlap{\underline{#1}}
    \par
    \setlength{\parindent}{0cm}
    \nopagebreak
    \leftskip=#2cm
    \rightskip=#3cm
}
{
    \par
}
\fi

\doendnotes{C}
\bigskip
\vfill

\clearpage

\footnotesize

\ifkorrekturansicht
  \lohead{\textsc{register}}
\fi

% theindex-Environment neu definieren ohne reledmac
\makeatletter
\renewenvironment{theindex}{%
  \ifkorrekturansicht
    \section*{\indexname}%
  \else
    \subsubsection*{Index der erwähnten Entitäten}%
  \fi
  \setlength{\parindent}{0pt}%
  \setlength{\parskip}{0pt plus 0.3pt}%
  \let\item\@idxitem
}{%
  \ifkorrekturansicht\clearpage\fi
}
\makeatother

\IfFileExists{\jobname-pw.ind}{\input{\jobname-pw.ind}}{}

% Quellenangabe nur in der Leseansicht
\ifkorrekturansicht\else
% Fallback-Definitionen, falls die .tex-Datei \titel etc. nicht gesetzt hat
\providecommand{\titel}{}
\providecommand{\editorInnen}{}
\providecommand{\dateiname}{\jobname}

\vspace{3cm}

\vfill

\footnotesize
\textsc{Quelle}: \titel. Herausgegeben von {\editorInnen}. In: \emph{Arthur Schnitzler: Briefwechsel mit Autorinnen und Autoren}.
 Digitale Edition, https://schnitzler-briefe.acdh.oeaw.ac.at/{\dateiname}.html (Stand \today)
\fi

\end{document}


