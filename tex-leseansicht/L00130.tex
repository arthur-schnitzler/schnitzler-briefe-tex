%% latex-leseansicht-vorspann.tex
%% Vorspann für die Leseansicht.
%% Lädt die gemeinsame Datei latex-vorspann.tex mit nicht gesetztem Schalter.

\newif\ifkorrekturansicht
\korrekturansichtfalse

\input{../tex-inputs/latex-vorspann}


               \section[Karl Kraus an Arthur Schnitzler, 31. 10. 1892]{ Karl Kraus an Arthur Schnitzler, 31. 10. 1892}\nopagebreak\mylabel{v}\rehead{ }\begin{ledgroupsized}[t]{13cm}\normalsize\beginnumbering\briefempfaengerindex{Schnitzler, Arthur@\textsc{Schnitzler, Arthur}!zzzKraus, Karl@\emph{von Karl Kraus}!1892-10-311@{31. 10. 1892}|(be} \toendnotes[C]{\smallbreak\pagebreak[2]} \Standort{CUL, Schnitzler, B 55.}
\physDesc{Brief, 1 Blatt, 3 Seiten
\newline{}Handschrift: schwarze Tinte, deutsche Kurrent
\newline{}Schnitzler: mit Bleistift beschriftet: »\textsc{Karl Kraus}« }\buchAbdrucke{\weitereDrucke{\emph{Karl Kraus und Arthur Schnitzler. Eine Dokumentation.} Hg. Reinhard Urbach. In: \emph{Literatur und Kritik}, Bd. 49, Oktober 1970, S. 513.} }\toendnotes[C]{\smallbreak}\pstart
           {\pb}am 31. Oktober
                  1892.\pend
           \pstart\center{}Sehr verehrter Herr Doctor!\pend\pstart
           Herzlichſten und aufrichtigſten Dank für die Überſendung Ihres Buches\pwindex{Schnitzler, Arthur 15.05.1862 – 21.10.1931@\textsc{Schnitzler, Arthur} (15.05.1862 – 21.10.1931), \emph{Schriftsteller, Mediziner}!Anatol1892-10-29 – 1892-10-29@\strich\emph{Anatol} {[}1892-10-29 – 1892-10-29{]}|pwv} und für die liebenswürdige Widmung!\pend
           \pstart
           Sie können ſich vorſtellen, \uline{wie} ich mich damit
               gefreut habe. Das iſt ja ein prächtiges Buch\pwindex{Schnitzler, Arthur 15.05.1862 – 21.10.1931@\textsc{Schnitzler, Arthur} (15.05.1862 – 21.10.1931), \emph{Schriftsteller, Mediziner}!Anatol1892-10-29 – 1892-10-29@\strich\emph{Anatol} {[}1892-10-29 – 1892-10-29{]}|pwv}! und der Prolog\pwindex{Prolog [zum Anatol]1892@\emph{Prolog [zum Anatol]} {[}1892{]}|pwv} von Loris\pwindex{Hofmannsthal, Hugo von 01.02.1874 – 15.07.1929@\textsc{Hofmannsthal, Hugo von} (01.02.1874 – 15.07.1929), \emph{Schriftsteller}|pw} iſt
               ſehr herzig. Aber ich bezahle Sie mit Undank. Denn – denken Sie ſich nur nur: ich –
               will – {\pb}eine – Kritik\pwindex{Kraus, Karl 28.04.1874 – 12.06.1936@\textsc{Kraus, Karl} (28.04.1874 – 12.06.1936), \emph{Schriftsteller, Publizist}!Arthur Schnitzler, Anatol01. 01. 1893@\strich\emph{Arthur Schnitzler, Anatol} {[}01. 01. 1893{]}|pwv} – drüber ſchreiben!! Nun ja, wenn ein
               Buch einmal \uline{in meine Klauen} kommt!\pend
           \pstart
           U. zw. entweder »\uline{Geſellſchaft}\pwindex{Gesellschaft. Monatsschrift1885 – 1902@\emph{Die Gesellschaft. Monatsschrift}|pw}« (\label{K_L00130_1v}\edtext{Dezemberheft}{\lemma{\textnormal{\emph{Dezemberheft}}}\Cendnote{\textnormal{Die Rezension erschien erst im ersten Heft
                  des neuen Jahres (Karl Kraus\pwindex{Kraus, Karl 28.04.1874 – 12.06.1936@\textsc{Kraus, Karl} (28.04.1874 – 12.06.1936), \emph{Schriftsteller, Publizist}|pwk}: \emph{Arthur Schnitzler, Anatol}\pwindex{Kraus, Karl 28.04.1874 – 12.06.1936@\textsc{Kraus, Karl} (28.04.1874 – 12.06.1936), \emph{Schriftsteller, Publizist}!Arthur Schnitzler, Anatol01. 01. 1893@\strich\emph{Arthur Schnitzler, Anatol} {[}01. 01. 1893{]}|pwk}. In: \emph{Die
                        Gesellschaft}\pwindex{Gesellschaft. Monatsschrift1885 – 1902@\emph{Die Gesellschaft. Monatsschrift}|pwk}, Jg. 9, H. 1, 1. 1. 1893, S. 109–110).
                  Die Verschiebung auf das Januarheft könnte dadurch verursacht sein, dass im
                  Dezember bereits zwei Rezensionen von Kraus\pwindex{Kraus, Karl 28.04.1874 – 12.06.1936@\textsc{Kraus, Karl} (28.04.1874 – 12.06.1936), \emph{Schriftsteller, Publizist}|pwk}
                  erschienen.}}}\label{K_L00130_1h}) oder »\uline{W\textsuperscript{r.} Allgemeine}\orgindex{Wiener Allgemeine Zeitung@Wiener Allgemeine Zeitung|pw}« – oder Feuilleton mit anderen Sachen.\pend
           \pstart
           Auguſtheft der »\uline{Geſellſchaft}\pwindex{Gesellschaft. Monatsschrift1885 – 1902@\emph{Die Gesellschaft. Monatsschrift}|pw} (Burgtheater\oindex{Burgtheater@\textbf{Burgtheater}|pw}aufsatz\pwindex{Kraus, Karl 28.04.1874 – 12.06.1936@\textsc{Kraus, Karl} (28.04.1874 – 12.06.1936), \emph{Schriftsteller, Publizist}!Burgtheater und die letzte Saison01. 08. 1892@\strich\emph{Das Burgtheater und die letzte Saison} {[}01. 08. 1892{]}|pwv}) bekam ich unlängſt zurück
               und ſende Ihnen noch heute. Er iſt leider in nicht ſehr salonfähigem Zuſtand, und
               leider – mein \uline{einziges Exemplar!}\pend
           \pstart
           {\pb}Ich hab’ Sie (von weitem allerdings) bei
               der \label{K_L00130_2v}\edtext{Premiere}{\lemma{\textnormal{\emph{Premiere}}}\Cendnote{\textnormal{am 29. 10. 1892 im Deutschen Volkstheater\oindex{Volkstheater@\textbf{Volkstheater}|pwk}; ein Besuch Schnitzler\pwindex{Schnitzler, Arthur 15.05.1862 – 21.10.1931@\textsc{Schnitzler, Arthur} (15.05.1862 – 21.10.1931), \emph{Schriftsteller, Mediziner}|pwk}s ist nicht in seinem \emph{Tagebuch}\pwindex{Schnitzler, Arthur 15.05.1862 – 21.10.1931@\textsc{Schnitzler, Arthur} (15.05.1862 – 21.10.1931), \emph{Schriftsteller, Mediziner}!Tagebuch1981 – 2000@\strich\emph{Tagebuch} {[}1981 – 2000{]}|pwk} verzeichnet.}}}\label{K_L00130_2h} der »Orientreise\pwindex{\textcolor{red}{\textsuperscript{XXXX1 indx}}!Orientreise1892@\strich\emph{Die Orientreise} {[}1892{]}|pw}\pwindex{\textcolor{red}{\textsuperscript{XXXX1 indx}}!Orientreise1892@\strich\emph{Die Orientreise} {[}1892{]}|pw}« gesehn. Nun, \uline{das} iſt doch ein Schund? \uline{Wie} hat es \uline{Ihnen} ge- resp. missfallen?\pend
           \pstart
           Ach, nochmals ergebenſt Dank für Ihre Liebenswürdigkeit und schönſten Gruß\pend
           \pstart
           von Ihrem{\\[\baselineskip]}hochachtungsvollen{\\[\baselineskip]}\spacefill\mbox{Karl Kraus}\pend
           \leftskip=0em{}\pstart
           \noindent{}I. Maximilianstr. 13\textsuperscript{I.}\oindex{Mahlerstrasse@\textbf{Mahlerstraße}|pw}\pend
                     \endnumbering\briefempfaengerindex{Schnitzler, Arthur@\textsc{Schnitzler, Arthur}!zzzKraus, Karl@\emph{von Karl Kraus}!1892-10-311@{31. 10. 1892}|)be}\mylabel{h}\end{ledgroupsized}  \newcommand{\dateiname}{L00130}\newcommand{\titel}{Karl Kraus an Arthur Schnitzler, 31. 10. 1892}\newcommand{\editorInnen}{Martin Anton Müller und Gerd-Hermann Susen}
            \footnotesize
\begin{ledgroupsized}[t]{11.5cm}
\doendnotes{C}
\end{ledgroupsized}
         %% latex-leseansicht-abspann.tex
%% Abspann für die Leseansicht.
%% Der Schalter \ifkorrekturansicht ist bereits durch den Vorspann gesetzt.

%% latex-abspann.tex
%% Gemeinsamer Abspann für Korrekturansicht und Leseansicht.
%% Setzt den Schalter \ifkorrekturansicht voraus (gesetzt in den
%% einbindenden Dateien latex-korrekturansicht-abspann.tex bzw.
%% latex-leseansicht-abspann.tex).
%% ---------------------------------------------------------------

\normalsize

% Das esempio-Environment wird nur in der Leseansicht benötigt
\ifkorrekturansicht\else
\newenvironment{esempio}[3]%
{
    \vspace{1.5ex}
    \rlap{\underline{#1}}
    \par
    \setlength{\parindent}{0cm}
    \nopagebreak
    \leftskip=#2cm
    \rightskip=#3cm
}
{
    \par
}
\fi

\doendnotes{C}
\bigskip
\vfill

\clearpage

\footnotesize

\ifkorrekturansicht
  \lohead{\textsc{register}}
\fi

% theindex-Environment neu definieren ohne reledmac
\makeatletter
\renewenvironment{theindex}{%
  \ifkorrekturansicht
    \section*{\indexname}%
  \else
    \subsubsection*{Index der erwähnten Entitäten}%
  \fi
  \setlength{\parindent}{0pt}%
  \setlength{\parskip}{0pt plus 0.3pt}%
  \let\item\@idxitem
}{%
  \ifkorrekturansicht\clearpage\fi
}
\makeatother

\IfFileExists{\jobname-pw.ind}{\input{\jobname-pw.ind}}{}

% Quellenangabe nur in der Leseansicht
\ifkorrekturansicht\else
% Fallback-Definitionen, falls die .tex-Datei \titel etc. nicht gesetzt hat
\providecommand{\titel}{}
\providecommand{\editorInnen}{}
\providecommand{\dateiname}{\jobname}

\vspace{3cm}

\vfill

\footnotesize
\textsc{Quelle}: \titel. Herausgegeben von {\editorInnen}. In: \emph{Arthur Schnitzler: Briefwechsel mit Autorinnen und Autoren}.
 Digitale Edition, https://schnitzler-briefe.acdh.oeaw.ac.at/{\dateiname}.html (Stand \today)
\fi

\end{document}


      