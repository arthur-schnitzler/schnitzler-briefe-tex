%% latex-leseansicht-vorspann.tex
%% Vorspann für die Leseansicht.
%% Lädt die gemeinsame Datei latex-vorspann.tex mit nicht gesetztem Schalter.

\newif\ifkorrekturansicht
\korrekturansichtfalse

\input{../tex-inputs/latex-vorspann}


\section[Karl Kraus an Arthur Schnitzler, 31. 10. 1892]{L00130 Karl Kraus an Arthur Schnitzler, 31. 10. 1892}
\nopagebreak\mylabel{L00130v}
\rehead{ }\normalsize\beginnumbering\briefempfaengerindex{Schnitzler, Arthur@\textsc{Schnitzler, Arthur}!zzzKraus, Karl@\emph{von Karl Kraus}!1892-10-311@{31. 10. 1892}|(be}
\toendnotes[C]{\smallbreak\pagebreak[2]}
\correspDesc{Versand  durch Karl Kraus am 31. 10. 1892 in Wien
\newline{}Erhalt  durch Arthur Schnitzler im Zeitraum [31. 10. 1892 – 4. 11. 1892?] in Wien}\toendnotes[C]{\smallbreak}
\Standort{CUL, Schnitzler, B 55.}
\physDesc{Brief, 1 Blatt, 3 Seiten, 1001 Zeichen
\newline{}Handschrift: schwarze Tinte, deutsche Kurrent
\newline{}Schnitzler: mit Bleistift beschriftet: »\textsc{Karl Kraus}« }
\buchAbdrucke{\weitereDrucke{\emph{Karl Kraus und Arthur Schnitzler. Eine Dokumentation.}Herausgegeben von Reinhard Urbach In: \emph{Literatur und Kritik}, Bd. 49, Oktober 1970, S. 513.} }\toendnotes[C]{\smallbreak}
\pstart
           {\pb}am 31. Oktober 1892.\pend
           
\pstart\center{}Sehr verehrter Herr Doctor!\pend\vspace{0.5em}
\pstart
           Herzlichſten und aufrichtigſten Dank für die Überſendung Ihres Buches\pwindex{Schnitzler, Arthur 15.\,5.\,1862 Wien – 21.\,10.\,1931 ebd.@\textsc{Schnitzler, Arthur} (15.\,5.\,1862 Wien – 21.\,10.\,1931 ebd.), \emph{Schriftsteller, Mediziner}!Anatol@\strich\emph{Anatol}|pwv} und für die liebenswürdige
               Widmung!\pend
           
\pstart
           Sie können{ }ſich vorſtellen, \uline{wie} ich mich damit
               gefreut habe. Das iſt ja ein prächtiges Buch\pwindex{Schnitzler, Arthur 15.\,5.\,1862 Wien – 21.\,10.\,1931 ebd.@\textsc{Schnitzler, Arthur} (15.\,5.\,1862 Wien – 21.\,10.\,1931 ebd.), \emph{Schriftsteller, Mediziner}!Anatol@\strich\emph{Anatol}|pwv}! und der Prolog\pwindex{Hofmannsthal, Hugo von 1.\,2.\,1874 Wien – 15.\,7.\,1929 Rodaun@\textsc{Hofmannsthal, Hugo von} (1.\,2.\,1874 Wien – 15.\,7.\,1929 Rodaun), \emph{Schriftsteller}!Prolog [zum Anatol]@\strich\emph{Prolog [zum Anatol]}|pwv} von Loris\pwindex{Hofmannsthal, Hugo von 1.\,2.\,1874 Wien – 15.\,7.\,1929 Rodaun@\textsc{Hofmannsthal, Hugo von} (1.\,2.\,1874 Wien – 15.\,7.\,1929 Rodaun), \emph{Schriftsteller}|pw} iſt{ }ſehr herzig. Aber ich bezahle Sie mit Undank. Denn – denken Sie{ }ſich nur nur: ich –
               will – {\pb}eine – Kritik\pwindex{Kraus, Karl 28.\,4.\,1874 Jičín – 12.\,6.\,1936 Wien@\textsc{Kraus, Karl} (28.\,4.\,1874 Jičín – 12.\,6.\,1936 Wien), \emph{Schriftsteller, Publizist, Schriftsteller}!Arthur Schnitzler, Anatol@\strich\emph{Arthur Schnitzler, Anatol}|pwv} – drüber{ }ſchreiben!! Nun ja, wenn
               ein Buch einmal \uline{in meine Klauen} kommt!\pend
           
\pstart
           U. zw. entweder »\uline{Geſellſchaft}\pwindex{Gesellschaft. Monatsschrift für Litteratur, Kunst und Sozialpolitik@\emph{Die Gesellschaft. Monatsschrift für Litteratur, Kunst und Sozialpolitik}|pw}« (\label{K_L00130-1v}\edtext{Dezemberheft}{\lemma{\textnormal{\emph{Dezemberheft}}}\Cendnote{\textnormal{Die Rezension erschien erst im ersten Heft
                  des neuen Jahres (Karl Kraus\pwindex{Kraus, Karl 28.\,4.\,1874 Jičín – 12.\,6.\,1936 Wien@\textsc{Kraus, Karl} (28.\,4.\,1874 Jičín – 12.\,6.\,1936 Wien), \emph{Schriftsteller, Publizist, Schriftsteller}|pwk}: \emph{Arthur Schnitzler, Anatol}\pwindex{Kraus, Karl 28.\,4.\,1874 Jičín – 12.\,6.\,1936 Wien@\textsc{Kraus, Karl} (28.\,4.\,1874 Jičín – 12.\,6.\,1936 Wien), \emph{Schriftsteller, Publizist, Schriftsteller}!Arthur Schnitzler, Anatol@\strich\emph{Arthur Schnitzler, Anatol}|pwk}. In: \emph{Die Gesellschaft}\pwindex{Gesellschaft. Monatsschrift für Litteratur, Kunst und Sozialpolitik@\emph{Die Gesellschaft. Monatsschrift für Litteratur, Kunst und Sozialpolitik}|pwk}, Jg. 9, H. 1, 1. 1. 1893,
                     S. 109–110). Die Verschiebung auf das Januarheft könnte dadurch
                  verursacht gewesen sein, dass im Dezember bereits zwei Rezensionen von Kraus\pwindex{Kraus, Karl 28.\,4.\,1874 Jičín – 12.\,6.\,1936 Wien@\textsc{Kraus, Karl} (28.\,4.\,1874 Jičín – 12.\,6.\,1936 Wien), \emph{Schriftsteller, Publizist, Schriftsteller}|pwk} erschienen waren.}}}\label{K_L00130-1}) oder »\uline{W\textsuperscript{r.} Allgemeine}\orgindex{Wiener Allgemeine Zeitung@Wiener Allgemeine Zeitung|pw}« – oder Feuilleton mit anderen Sachen.\pend
           
\pstart
           Auguſtheft der »\uline{Geſellſchaft}\pwindex{Gesellschaft. Monatsschrift für Litteratur, Kunst und Sozialpolitik@\emph{Die Gesellschaft. Monatsschrift für Litteratur, Kunst und Sozialpolitik}|pw} (Burgtheater\oindex{Wien@\textbf{Wien}!I., Innere Stadt@\textbf{I., Innere Stadt}!Burgtheater@\textbf{Burgtheater}, \emph{Theater}|pw}aufsatz\pwindex{Kraus, Karl 28.\,4.\,1874 Jičín – 12.\,6.\,1936 Wien@\textsc{Kraus, Karl} (28.\,4.\,1874 Jičín – 12.\,6.\,1936 Wien), \emph{Schriftsteller, Publizist, Schriftsteller}!Burgtheater und die letzte Saison@\strich\emph{Das Burgtheater und die letzte Saison}|pwv}) bekam ich unlängſt
               zurück und{ }ſende Ihnen noch heute. Er iſt leider in nicht{ }ſehr salonfähigem Zuſtand,
               und leider – mein \uline{einziges Exemplar!}\pend
           
\pstart
           {\pb}Ich hab’ Sie (von weitem allerdings) bei
               der \label{K_L00130-2v}\edtext{Premiere}{\lemma{\textnormal{\emph{Premiere}}}\Cendnote{\textnormal{Diese fand am 29. 10. 1892 im Deutschen Volkstheater\oindex{Wien@\textbf{Wien}!VII., Neubau@\textbf{VII., Neubau}!Volkstheater@\textbf{Volkstheater}, \emph{Theater}|pwk} statt. Ein Besuch Schnitzlers ist nicht in seinem \emph{Tagebuch}\pwindex{Schnitzler, Arthur 15.\,5.\,1862 Wien – 21.\,10.\,1931 ebd.@\textsc{Schnitzler, Arthur} (15.\,5.\,1862 Wien – 21.\,10.\,1931 ebd.), \emph{Schriftsteller, Mediziner}!Tagebuch@\strich\emph{Tagebuch}|pwk} verzeichnet. In der Aufstellung seiner Theaterbesuche (\emph{CUL}, A 179) fehlt das Blatt mit den 
               Einträgen zu Oktober bis Dezember 1892.}}}\label{K_L00130-2} der »Orientreise\pwindex{\textcolor{red}{\textsuperscript{XXXX indx1}}!Orientreise. Schwank in drei Akten@\strich\emph{Die Orientreise. Schwank in drei Akten}|pw}\pwindex{\textcolor{red}{\textsuperscript{XXXX indx1}}!Orientreise. Schwank in drei Akten@\strich\emph{Die Orientreise. Schwank in drei Akten}|pw}« gesehn. Nun, \uline{das} iſt doch ein
               Schund? \uline{Wie} hat es \uline{Ihnen} ge- resp. missfallen?\pend
           
\pstart
           Ach, nochmals ergebenſt Dank für Ihre Liebenswürdigkeit und schönſten Gruß\pend
           
\pstart
           von Ihrem{\\[\baselineskip]}hochachtungsvollen{\\[\baselineskip]}\spacefill\mbox{Karl Kraus}\pend
           \leftskip=0em{}
\pstart
           \noindent{}I. Maximilianstr. 13\textsuperscript{I.}\oindex{Wien@\textbf{Wien}!I., Innere Stadt@\textbf{I., Innere Stadt}!Mahlerstraße@\textbf{Mahlerstraße}, \emph{Straße}|pw}\pend
           \selectlanguage{ngerman}\endnumbering\briefempfaengerindex{Schnitzler, Arthur@\textsc{Schnitzler, Arthur}!zzzKraus, Karl@\emph{von Karl Kraus}!1892-10-311@{31. 10. 1892}|)be}\mylabel{L00130h}  \newcommand{\dateiname}{L00130}\newcommand{\titel}{Karl Kraus an Arthur Schnitzler, 31. 10. 1892}\newcommand{\editorInnen}{Martin Anton Müller und Gerd-Hermann Susen}%% latex-leseansicht-abspann.tex
%% Abspann für die Leseansicht.
%% Der Schalter \ifkorrekturansicht ist bereits durch den Vorspann gesetzt.

%% latex-abspann.tex
%% Gemeinsamer Abspann für Korrekturansicht und Leseansicht.
%% Setzt den Schalter \ifkorrekturansicht voraus (gesetzt in den
%% einbindenden Dateien latex-korrekturansicht-abspann.tex bzw.
%% latex-leseansicht-abspann.tex).
%% ---------------------------------------------------------------

\normalsize

% Das esempio-Environment wird nur in der Leseansicht benötigt
\ifkorrekturansicht\else
\newenvironment{esempio}[3]%
{
    \vspace{1.5ex}
    \rlap{\underline{#1}}
    \par
    \setlength{\parindent}{0cm}
    \nopagebreak
    \leftskip=#2cm
    \rightskip=#3cm
}
{
    \par
}
\fi

\doendnotes{C}
\bigskip
\vfill

\clearpage

\footnotesize

\ifkorrekturansicht
  \lohead{\textsc{register}}
\fi

% theindex-Environment neu definieren ohne reledmac
\makeatletter
\renewenvironment{theindex}{%
  \ifkorrekturansicht
    \section*{\indexname}%
  \else
    \subsubsection*{Index der erwähnten Entitäten}%
  \fi
  \setlength{\parindent}{0pt}%
  \setlength{\parskip}{0pt plus 0.3pt}%
  \let\item\@idxitem
}{%
  \ifkorrekturansicht\clearpage\fi
}
\makeatother

\IfFileExists{\jobname-pw.ind}{\input{\jobname-pw.ind}}{}

% Quellenangabe nur in der Leseansicht
\ifkorrekturansicht\else
% Fallback-Definitionen, falls die .tex-Datei \titel etc. nicht gesetzt hat
\providecommand{\titel}{}
\providecommand{\editorInnen}{}
\providecommand{\dateiname}{\jobname}

\vspace{3cm}

\vfill

\footnotesize
\textsc{Quelle}: \titel. Herausgegeben von {\editorInnen}. In: \emph{Arthur Schnitzler: Briefwechsel mit Autorinnen und Autoren}.
 Digitale Edition, https://schnitzler-briefe.acdh.oeaw.ac.at/{\dateiname}.html (Stand \today)
\fi

\end{document}


