%% latex-korrekturansicht-vorspann.tex
%% Vorspann für die Korrekturansicht.
%% Lädt die gemeinsame Datei latex-vorspann.tex mit gesetztem Schalter.

\newif\ifkorrekturansicht
\korrekturansichttrue

\input{../tex-inputs/latex-vorspann}


\section[Karl Kraus an Arthur Schnitzler, 31. 10. 1892]{L00130 Karl Kraus an Arthur Schnitzler, 31. 10. 1892}
\nopagebreak\mylabel{L00130v}
\rehead{ }\normalsize\beginnumbering\briefempfaengerindex{Schnitzler, Arthur@\textsc{Schnitzler, Arthur}!zzzKraus, Karl@\emph{von Karl Kraus}!1892-10-311@{31. 10. 1892}|(be}
\toendnotes[C]{\smallbreak\pagebreak[2]}\Standort{CUL, Schnitzler, B 55.}
\physDesc{Brief, 1 Blatt, 3 Seiten, 1001 Zeichen
\newline{}Handschrift: schwarze Tinte, deutsche Kurrent
\newline{}Schnitzler: mit Bleistift beschriftet: »\textsc{Karl Kraus}« }
\buchAbdrucke{\weitereDrucke{\emph{Literatur und Kritik}, Bd. 49, Oktober 1970, S. 513.} }\toendnotes[C]{\smallbreak}
\pstart
           {\pb}am 31. Oktober
                  1892.\pend
           
\pstart\center{}Sehr verehrter Herr Doctor!\pend\vspace{0.5em}
\pstart
           Herzlichſten und aufrichtigſten Dank für die Überſendung Ihres Buches\pwindex{Anatol@\emph{Anatol}|pwv} und für die liebenswürdige
               Widmung!\pend
           
\pstart
           Sie können ſich vorſtellen, \uline{wie} ich mich damit
               gefreut habe. Das iſt ja ein prächtiges Buch\pwindex{Anatol@\emph{Anatol}|pwv}! und der Prolog\pwindex{Prolog [zum Anatol]@\emph{Prolog [zum Anatol]}|pwv} von Loris\pwindex{Hofmannsthal, Hugo von 1874-02-01 – 1929-07-15@\textsc{Hofmannsthal, Hugo von} (1874-02-01 – 1929-07-15), \emph{Schriftsteller/Schriftstellerin}|pw} iſt
               ſehr herzig. Aber ich bezahle Sie mit Undank. Denn – denken Sie ſich nur nur: ich –
               will – {\pb}eine – Kritik\pwindex{Arthur Schnitzler, Anatol@\emph{Arthur Schnitzler, Anatol}|pwv} – drüber ſchreiben!! Nun ja, wenn
               ein Buch einmal \uline{in meine Klauen} kommt!\pend
           
\pstart
           U. zw. entweder »\uline{Geſellſchaft}\pwindex{Gesellschaft. Monatsschrift fuer Litteratur, Kunst und Sozialpolitik@\emph{Die Gesellschaft. Monatsschrift für Litteratur, Kunst und Sozialpolitik}|pw}« (\label{K_L00130-1v}\edtext{Dezemberheft}{\lemma{\textnormal{\emph{Dezemberheft}}}\Cendnote{\textnormal{Die Rezension erschien erst im ersten Heft
                  des neuen Jahres (Karl Kraus\pwindex{Kraus, Karl 28.04.1874 – 12.06.1936@\textsc{Kraus, Karl} (28.04.1874 – 12.06.1936), \emph{Schriftsteller/Schriftstellerin, Publizist/Publizistin, Schriftsteller/Schriftstellerin}|pwk}: \emph{Arthur Schnitzler, Anatol}\pwindex{Arthur Schnitzler, Anatol@\emph{Arthur Schnitzler, Anatol}|pwk}. In: \emph{Die Gesellschaft}\pwindex{Gesellschaft. Monatsschrift fuer Litteratur, Kunst und Sozialpolitik@\emph{Die Gesellschaft. Monatsschrift für Litteratur, Kunst und Sozialpolitik}|pwk}, Jg. 9, H. 1, 1. 1. 1893,
                     S. 109–110). Die Verschiebung auf das Januarheft könnte dadurch
                  verursacht gewesen sein, dass im Dezember bereits zwei Rezensionen von Kraus\pwindex{Kraus, Karl 28.04.1874 – 12.06.1936@\textsc{Kraus, Karl} (28.04.1874 – 12.06.1936), \emph{Schriftsteller/Schriftstellerin, Publizist/Publizistin, Schriftsteller/Schriftstellerin}|pwk} erschienen waren.}}}\label{K_L00130-1}) oder »\uline{W\textsuperscript{r.} Allgemeine}\orgindex{Wiener Allgemeine Zeitung@Wiener Allgemeine Zeitung|pw}« – oder Feuilleton mit anderen Sachen.\pend
           
\pstart
           Auguſtheft der »\uline{Geſellſchaft}\pwindex{Gesellschaft. Monatsschrift fuer Litteratur, Kunst und Sozialpolitik@\emph{Die Gesellschaft. Monatsschrift für Litteratur, Kunst und Sozialpolitik}|pw} (Burgtheater\oindex{Burgtheater@\textbf{Burgtheater}, \emph{S.THTR}|pw}aufsatz\pwindex{Burgtheater und die letzte Saison@\emph{Das Burgtheater und die letzte Saison}|pwv}) bekam ich unlängſt
               zurück und ſende Ihnen noch heute. Er iſt leider in nicht ſehr salonfähigem Zuſtand,
               und leider – mein \uline{einziges Exemplar!}\pend
           
\pstart
           {\pb}Ich hab’ Sie (von weitem allerdings) bei
               der \label{K_L00130-2v}\edtext{Premiere}{\lemma{\textnormal{\emph{Premiere}}}\Cendnote{\textnormal{Diese fand am 29. 10. 1892 im Deutschen Volkstheater\oindex{Volkstheater@\textbf{Volkstheater}, \emph{Theater (K.THE)}|pwk} statt. Ein Besuch Schnitzlers ist nicht in seinem \emph{Tagebuch}\pwindex{Tagebuch@\emph{Tagebuch}|pwk} verzeichnet. In der Aufstellung seiner Theaterbesuche (\emph{CUL}, A 179) fehlt das Blatt mit den 
               Einträgen zu Oktober bis Dezember 1892.}}}\label{K_L00130-2} der »Orientreise\pwindex{Orientreise. Schwank in drei Akten@\emph{Die Orientreise. Schwank in drei Akten}|pw}« gesehn. Nun, \uline{das} iſt doch ein
               Schund? \uline{Wie} hat es \uline{Ihnen} ge- resp. missfallen?\pend
           
\pstart
           Ach, nochmals ergebenſt Dank für Ihre Liebenswürdigkeit und schönſten Gruß\pend
           
\pstart
           von Ihrem{\\[\baselineskip]}hochachtungsvollen{\\[\baselineskip]}\spacefill\mbox{Karl Kraus}\pend
           \leftskip=0em{}
\pstart
           \noindent{}I. Maximilianstr. 13\textsuperscript{I.}\oindex{Mahlerstrasse@\textbf{Mahlerstraße}, \emph{Straße (K.STR)}|pw}\pend
           \selectlanguage{ngerman}\endnumbering\briefempfaengerindex{Schnitzler, Arthur@\textsc{Schnitzler, Arthur}!zzzKraus, Karl@\emph{von Karl Kraus}!1892-10-311@{31. 10. 1892}|)be}\mylabel{L00130h}  \normalsize

\doendnotes{C}
\bigskip
\vfill

\clearpage

\footnotesize

\lohead{\textsc{register}}

% Definiere theindex-Environment komplett neu ohne reledmac
\makeatletter
\renewenvironment{theindex}{%
  \section*{\indexname}%
  \setlength{\parindent}{0pt}%
  \setlength{\parskip}{0pt plus 0.3pt}%
  \let\item\@idxitem
}{%
  \clearpage
}
\makeatother

\IfFileExists{\jobname-pw.ind}{\input{\jobname-pw.ind}}{}

\end{document}

      