%% latex-korrekturansicht-vorspann.tex
%% Vorspann für die Korrekturansicht.
%% Lädt die gemeinsame Datei latex-vorspann.tex mit gesetztem Schalter.

\newif\ifkorrekturansicht
\korrekturansichttrue

\input{../tex-inputs/latex-vorspann}


\section[Richard Beer-Hofmann an Arthur Schnitzler, 5. 9. 1904]{L01436 Richard Beer-Hofmann an Arthur Schnitzler, 5. 9. 1904}
\nopagebreak\mylabel{L01436v}
\rehead{ }\normalsize\beginnumbering\briefempfaengerindex{Schnitzler, Arthur@\textsc{Schnitzler, Arthur}!zzzBeer-Hofmann, Richard@\emph{von Richard Beer-Hofmann}!1904-09-051@{5. 9. 1904}|(be}
\toendnotes[C]{\smallbreak\pagebreak[2]}\Standort{CUL, Schnitzler, B 8.}
\physDesc{Brief, 1 Blatt, 1 Seite, 542 Zeichen
\newline{}Handschrift: schwarze Tinte, lateinische Kurrent
\newline{}Ordnung: mit Bleistift von unbekannter Hand nummeriert:
                                    »187« }\toendnotes[C]{\smallbreak}
\pstart
           \noindent{}\centering{}{\pb}Aussee\oindex{Bad Aussee@\textbf{Bad Aussee}, \emph{P.PPLA3}|pw}{ }5./IX 04\pend
           
\pstart
           Lieber Arthur! Wenn das Wetter nicht zu scheusslich ist, bin ich
                  Mittwoch{ }\uline{11.46} in Lueg\oindex{Lueg@\textbf{Lueg}, \emph{Teil eines besiedelten Ortes (A.BSOX)}|pw}. Um \uline{4.02} fahre ich von Lueg\oindex{Lueg@\textbf{Lueg}, \emph{Teil eines besiedelten Ortes (A.BSOX)}|pw} nach Ischl\oindex{Bad Ischl@\textbf{Bad Ischl}, \emph{P.PPL}|pw} (\uline{5.02}), und von dort (\substVorne{}\textsuperscript{\uline{6.56}}\substDazwischen{}6.05\substHinten{}) nach Aussee\oindex{Bad Aussee@\textbf{Bad Aussee}, \emph{P.PPLA3}|pw} (\uline{7.15}). Bei späterer Abfahrt von Lueg\oindex{Lueg@\textbf{Lueg}, \emph{Teil eines besiedelten Ortes (A.BSOX)}|pw} hätte ich
               keinen guten Anschluss nach Aussee\oindex{Bad Aussee@\textbf{Bad Aussee}, \emph{P.PPLA3}|pw}. Vielleicht
               fahren Sie dann \strikeout{d} statt Donnerstag früh,
                  Mittwoch Nachmittag mit mir. In Aussee\oindex{Bad Aussee@\textbf{Bad Aussee}, \emph{P.PPLA3}|pw} wohnen Sie nicht Elisabeth\oindex{Bade-Hotel Elisabeth@\textbf{Bade-Hotel Elisabeth}, \emph{Hotel (K.HTL)}|pw}, das
               um diese Zeit im Veröden sein dürfte. Vielleicht »Post\oindex{Gasthaus Post@\textbf{Gasthaus Post}, \emph{Hotel (K.HTL)}|pw}« (Ich glaube jetzt »Franz Carl\oindex{Gasthaus Post@\textbf{Gasthaus Post}, \emph{Hotel (K.HTL)}|pw}«)
               wo Sie schon einmal \label{K_L01436-1v}\edtext{wohnten}{\lemma{\textnormal{\emph{wohnten}}}\Cendnote{\textnormal{vom 28. 7. 1900 bis zum 1. 8. 1900}}}\label{K_L01436-1}. Oder »Hackinger\oindex{Hackinger s Hotel zum Kaiser von Oesterreich@\textbf{Hackinger’s Hotel zum Kaiser von Österreich}, \emph{Hotel (K.HTL)}|pw}«, »Erzherz. Johann\oindex{Erzherzog Johann@\textbf{Erzherzog Johann}, \emph{Gastgewerbegebäude (K.GGW)}|pw}«?\pend
           
\pstart
           Herzlichst Ihr{\\[\baselineskip]}\spacefill\mbox{Richard}\pend
           \leftskip=0em{}\selectlanguage{ngerman}\endnumbering\briefempfaengerindex{Schnitzler, Arthur@\textsc{Schnitzler, Arthur}!zzzBeer-Hofmann, Richard@\emph{von Richard Beer-Hofmann}!1904-09-051@{5. 9. 1904}|)be}\mylabel{L01436h}  \normalsize

\doendnotes{C}
\bigskip
\vfill

\clearpage

\footnotesize

\lohead{\textsc{register}}

% Definiere theindex-Environment komplett neu ohne reledmac
\makeatletter
\renewenvironment{theindex}{%
  \section*{\indexname}%
  \setlength{\parindent}{0pt}%
  \setlength{\parskip}{0pt plus 0.3pt}%
  \let\item\@idxitem
}{%
  \clearpage
}
\makeatother

\IfFileExists{\jobname-pw.ind}{\input{\jobname-pw.ind}}{}

\end{document}

      