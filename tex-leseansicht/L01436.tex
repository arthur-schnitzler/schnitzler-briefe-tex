\input{../tex-inputs/latex-pdf-vorspann}
\begin{center}
            \textcolor{red}{ENTWURF. ENTZIFFERUNG NOCH NICHT KORREKTURGELESEN}
                      \end{center}
            
               \section[Richard Beer-Hofmann an Arthur Schnitzler, 5. 9. 1904]{ Richard Beer-Hofmann an Arthur Schnitzler, 5. 9. 1904}\nopagebreak\mylabel{v}\rehead{ }\begin{ledgroupsized}[t]{13cm}\normalsize\beginnumbering\briefempfaengerindex{Schnitzler, Arthur@\textsc{Schnitzler, Arthur}!zzzBeer-Hofmann, Richard@\emph{von Richard Beer-Hofmann}!1904-09-051@{5. 9. 1904}|(be} \toendnotes[C]{\smallbreak\pagebreak[2]} \Standort{CUL, Schnitzler, B 8.}
\physDesc{Brief, 1 Blatt, 1 Seite
\newline{}Handschrift: schwarze Tinte, lateinische Kurrent\newline{}Ordnung: mit Bleistift von unbekannter Hand nummeriert:
                                    »187« }\toendnotes[C]{\smallbreak}\pstart
           \noindent{}\centering{}{\pb}Aussee\oindex{Bad Aussee@\textbf{Bad Aussee}|pw}{ }5./IX 04\pend
           \pstart
           \noindent{}Lieber Arthur! Wenn das Wetter nicht zu scheusslich ist, bin ich
                  Mittwoch{ }\uline{11.46} in Lueg\oindex{Lueg am Wolfgangsee@\textbf{Lueg am Wolfgangsee}|pw}. Um \uline{4.02} fahre ich von Lueg\oindex{Lueg am Wolfgangsee@\textbf{Lueg am Wolfgangsee}|pw} nach Ischl\oindex{Bad Ischl@\textbf{Bad Ischl}|pw} (\uline{5.02}), und von dort (\substVorne{}\textsuperscript{\uline{6.56}}\substDazwischen{}6.05\substHinten{}) nach Aussee\oindex{Bad Aussee@\textbf{Bad Aussee}|pw} (\uline{7.15}). Bei späterer Abfahrt von Lueg\oindex{Lueg am Wolfgangsee@\textbf{Lueg am Wolfgangsee}|pw} hätte ich
               keinen guten Anschluss nach Aussee\oindex{Bad Aussee@\textbf{Bad Aussee}|pw}. Vielleicht
               fahren Sie dann \strikeout{d} statt Donnerstag früh,
                  Mittwoch Nachmittag mit mir. In Aussee\oindex{Bad Aussee@\textbf{Bad Aussee}|pw} wohnen Sie nicht Elisabeth\oindex{Bade-Hotel Elisabeth@\textbf{Bade-Hotel Elisabeth}|pw}, das um
               diese Zeit im Veröden sein dürfte. Vielleicht »Post\oindex{Gasthaus Post@\textbf{Gasthaus Post}|pw}« (Ich glaube jetzt »Franz Carl\oindex{Gasthaus Post@\textbf{Gasthaus Post}|pw}«) wo
               Sie schon einmal \label{K_L01436_1v}\edtext{wohnten}{\lemma{\textnormal{\emph{wohnten}}}\Cendnote{\textnormal{vom 28. 7. 1900 bis zum 1. 8. 1900}}}\label{K_L01436_1h}. Oder »Hackinger\oindex{Hackinger s Hotel zum Kaiser von Oesterreich@\textbf{Hackinger’s Hotel zum Kaiser von Österreich}|pw}«, »Erzherz.
               Johann\oindex{Erzherzog Johann@\textbf{Erzherzog Johann}|pw}«?\pend
           \pstart
           Herzlichst Ihr{\\[\baselineskip]}\spacefill\mbox{Richard}\pend
           \leftskip=0em{}\endnumbering\briefempfaengerindex{Schnitzler, Arthur@\textsc{Schnitzler, Arthur}!zzzBeer-Hofmann, Richard@\emph{von Richard Beer-Hofmann}!1904-09-051@{5. 9. 1904}|)be}\mylabel{h}\end{ledgroupsized}  \newcommand{\dateiname}{L01436}\newcommand{\titel}{Richard Beer-Hofmann an Arthur Schnitzler, 5. 9. 1904}\newcommand{\editorInnen}{Martin Anton Müller und Gerd-Hermann Susen}\input{../tex-inputs/latex-pdf-abspann}
      