%% latex-leseansicht-vorspann.tex
%% Vorspann für die Leseansicht.
%% Lädt die gemeinsame Datei latex-vorspann.tex mit nicht gesetztem Schalter.

\newif\ifkorrekturansicht
\korrekturansichtfalse

\input{../tex-inputs/latex-vorspann}


\section[Theodor Herzl an Arthur Schnitzler, 4. 3. 1895]{L03852 Theodor Herzl an Arthur Schnitzler, 4. 3. 1895}
\nopagebreak\mylabel{L03852v}
\rehead{ }\normalsize\beginnumbering\briefempfaengerindex{Schnitzler, Arthur@\textsc{Schnitzler, Arthur}!zzzHerzl, Theodor@\emph{von Theodor Herzl}!1895-03-041@{4. 3. 1895}|(be}
\toendnotes[C]{\smallbreak\pagebreak[2]}
\correspDesc{Versand  durch Theodor Herzl am 4. 3. 1895 in Paris
\newline{}Erhalt  durch Arthur Schnitzler im Zeitraum [5. 3. 1895
                  – 9. 3. 1895?] in Wien}\toendnotes[C]{\smallbreak}
\Standort{CUL, Schnitzler, B 39.}
\physDesc{Brief, 1 Blatt, 2 Seiten, 756 Zeichen
\newline{}Handschrift: schwarze Tinte, lateinische Kurrent
\newline{}Ordnung: mit Bleistift von unbekannter Hand nummeriert: »31« }
\buchAbdrucke{\weitereDrucke{Theodor Herzl: \emph{Briefe und
                        autobiographische Notizen 1866–1895}. Bearbeitet von Johannes Wachten in Zusammenarbeit mit Chaya Harel, Daisy Tycho und Manfred Winkler. Berlin, Frankfurt am Main, Wien: \emph{Propyläen} 1983, S. 577 (Briefe und Tagebücher. Herausgegeben von Alex Bein, Hermann Greive, Moshe Schaerf, Julius H. Schoeps und Johannes Wachten, 1).} }\toendnotes[C]{\smallbreak}
\pstart
           \raggedleft{}{\pb}Palais Bourbon\oindex{Palais Bourbon@\textbf{Palais Bourbon}, \emph{Regierungsgebäude}|pw}\pend
           
\pstart
           \raggedleft{}4. März 95\pend
           
\pstart{}Mein lieber Freund!\pend\vspace{0.5em}
\pstart
           Bitte geben Sie Müllern\pwindex{Müller-Guttenbrunn, Adam 22.\,10.\,1852 Zăbrani – 5.\,1.\,1923 Wien@\textsc{Müller-Guttenbrunn, Adam} (22.\,10.\,1852 Zăbrani – 5.\,1.\,1923 Wien), \emph{Schriftsteller, Theaterleiter, Beamter}|pw} einen Tritt in den
               Hintern.\pend
           
\pstart
           Die \label{K_L03852-1v}\edtext{Erklärungsfrist}{\lemma{\textnormal{\emph{Erklärungsfrist}}}\Cendnote{\textnormal{Herzl\pwindex{Herzl, Theodor 2.\,5.\,1860 Budapest – 3.\,7.\,1904 Edlach@\textsc{Herzl, Theodor} (2.\,5.\,1860 Budapest – 3.\,7.\,1904 Edlach), \emph{Schriftsteller, Journalist}|pwk} hatte Schnitzler am XXXX Auszeichnungsfehler: Dokument L03849 nicht gefunden gebeten, dem Leiter des \emph{Raimund-Theaters}\orgindex{Raimund-Theater@Raimund-Theater|pwk}{ }Adam Müller-Guttenbrunn\pwindex{Müller-Guttenbrunn, Adam 22.\,10.\,1852 Zăbrani – 5.\,1.\,1923 Wien@\textsc{Müller-Guttenbrunn, Adam} (22.\,10.\,1852 Zăbrani – 5.\,1.\,1923 Wien), \emph{Schriftsteller, Theaterleiter, Beamter}|pwk} acht Tage Zeit zu
                  geben, um zu entscheiden, ob er Herzls\pwindex{Herzl, Theodor 2.\,5.\,1860 Budapest – 3.\,7.\,1904 Edlach@\textsc{Herzl, Theodor} (2.\,5.\,1860 Budapest – 3.\,7.\,1904 Edlach), \emph{Schriftsteller, Journalist}|pwk} durch Schnitzler
                  eingereichtes Schauspiel\pwindex{Herzl, Theodor 2.\,5.\,1860 Budapest – 3.\,7.\,1904 Edlach@\textsc{Herzl, Theodor} (2.\,5.\,1860 Budapest – 3.\,7.\,1904 Edlach), \emph{Schriftsteller, Journalist}!neue Ghetto. Schauspiel in vier Acten@\strich\emph{Das neue Ghetto. Schauspiel in vier Acten}|pwkv} zur Aufführung bringen werde. Am 20. 2. 1895 bestätigte Herzl\pwindex{Herzl, Theodor 2.\,5.\,1860 Budapest – 3.\,7.\,1904 Edlach@\textsc{Herzl, Theodor} (2.\,5.\,1860 Budapest – 3.\,7.\,1904 Edlach), \emph{Schriftsteller, Journalist}|pwk} per Telegramm die Aufforderung zur Einreichung endgültig, vgl. XXXX Auszeichnungsfehler: Dokument L03850 nicht gefunden.}}}\label{K_L03852-1} ist doch schon
               um.\pend
           
\pstart
           Ich lese heute mit vielem Vergnügen, dass Ihr Stück\pwindex{Schnitzler, Arthur 15.\,5.\,1862 Wien – 21.\,10.\,1931 ebd.@\textsc{Schnitzler, Arthur} (15.\,5.\,1862 Wien – 21.\,10.\,1931 ebd.), \emph{Schriftsteller, Mediziner}!Liebelei. Schauspiel in drei Akten@\strich\emph{Liebelei. Schauspiel in drei Akten}|pwv}{ }\label{K_L03852-2v}\edtext{die nächste Burg\orgindex{Burgtheater@Burgtheater|pwv}novität}{\lemma{\textnormal{\emph{die nächste Burgnovität}}}\Cendnote{\textnormal{Vgl. \emph{Neue Freie Presse}\pwindex{Neue Freie Presse@\emph{Neue Freie Presse}|pwk}, Nr. 10.964,
                        3. 3. 1895, Morgenblatt, S. 8. Tatsächlich sollte die Uraufführung der Liebelei\eventindex{Burgtheater@\textbf{Burgtheater}!Uraufführung von Liebelei, Premiere von Rechte der Seele, 9.10.1895@Uraufführung von Liebelei, Premiere von Rechte der Seele, 9.10.1895|pwk} erst am 9. 10. 1895 stattfinden.}}}\label{K_L03852-2} ist. Glück auf!\pend
           
\pstart
           Aus leider bitteren Erfahrungen heraus gebe ich Ihnen folgenden brüderlichen Rath:
               machen Sie vor den Schauspielern auf den Proben gar keine Bemerkungen aber halten Sie
               den Nacken steif u. lassen Sie sich durch keine »Autorität« unterkriegen wenn man
               etwas gegen Ihre {\pb}Absichten machen will.
               Sie haben nachher die Verantwortung, also setzen Sie auch Ihren ganzen Willen durch.
               Der Widerstand der Theaterleute – wenn’s überhaupt einen solchen geben wird – ist
               immer nur ein scheinbarer.\pend
           
\pstart
           Schreiben Sie mir bald u. ausführlich {\\[\baselineskip]}Herzlich Ihr {\\[\baselineskip]}\spacefill\mbox{Th Herzl}\pend
           \leftskip=0em{}\selectlanguage{ngerman}\endnumbering\briefempfaengerindex{Schnitzler, Arthur@\textsc{Schnitzler, Arthur}!zzzHerzl, Theodor@\emph{von Theodor Herzl}!1895-03-041@{4. 3. 1895}|)be}\mylabel{L03852h}
\begin{anhang}
\end{anhang}\newcommand{\dateiname}{L03852}\newcommand{\titel}{Theodor Herzl an Arthur Schnitzler, 4. 3. 1895}\newcommand{\editorInnen}{Selma Jahnke und Martin Anton Müller}%% latex-leseansicht-abspann.tex
%% Abspann für die Leseansicht.
%% Der Schalter \ifkorrekturansicht ist bereits durch den Vorspann gesetzt.

%% latex-abspann.tex
%% Gemeinsamer Abspann für Korrekturansicht und Leseansicht.
%% Setzt den Schalter \ifkorrekturansicht voraus (gesetzt in den
%% einbindenden Dateien latex-korrekturansicht-abspann.tex bzw.
%% latex-leseansicht-abspann.tex).
%% ---------------------------------------------------------------

\normalsize

% Das esempio-Environment wird nur in der Leseansicht benötigt
\ifkorrekturansicht\else
\newenvironment{esempio}[3]%
{
    \vspace{1.5ex}
    \rlap{\underline{#1}}
    \par
    \setlength{\parindent}{0cm}
    \nopagebreak
    \leftskip=#2cm
    \rightskip=#3cm
}
{
    \par
}
\fi

\doendnotes{C}
\bigskip
\vfill

\clearpage

\footnotesize

\ifkorrekturansicht
  \lohead{\textsc{register}}
\fi

% theindex-Environment neu definieren ohne reledmac
\makeatletter
\renewenvironment{theindex}{%
  \ifkorrekturansicht
    \section*{\indexname}%
  \else
    \subsubsection*{Index der erwähnten Entitäten}%
  \fi
  \setlength{\parindent}{0pt}%
  \setlength{\parskip}{0pt plus 0.3pt}%
  \let\item\@idxitem
}{%
  \ifkorrekturansicht\clearpage\fi
}
\makeatother

\IfFileExists{\jobname-pw.ind}{\input{\jobname-pw.ind}}{}

% Quellenangabe nur in der Leseansicht
\ifkorrekturansicht\else
% Fallback-Definitionen, falls die .tex-Datei \titel etc. nicht gesetzt hat
\providecommand{\titel}{}
\providecommand{\editorInnen}{}
\providecommand{\dateiname}{\jobname}

\vspace{3cm}

\vfill

\footnotesize
\textsc{Quelle}: \titel. Herausgegeben von {\editorInnen}. In: \emph{Arthur Schnitzler: Briefwechsel mit Autorinnen und Autoren}.
 Digitale Edition, https://schnitzler-briefe.acdh.oeaw.ac.at/{\dateiname}.html (Stand \today)
\fi

\end{document}


