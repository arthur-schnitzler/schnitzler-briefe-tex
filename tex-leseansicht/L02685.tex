%% latex-korrekturansicht-vorspann.tex
%% Vorspann für die Korrekturansicht.
%% Lädt die gemeinsame Datei latex-vorspann.tex mit gesetztem Schalter.

\newif\ifkorrekturansicht
\korrekturansichttrue

\input{../tex-inputs/latex-vorspann}


\section[Arthur Schnitzler an Paul Goldmann, 22. 11. 1896]{L02685 Arthur Schnitzler an Paul Goldmann, 22. 11. 1896}
\nopagebreak\mylabel{L02685v}
\rehead{ }\normalsize\beginnumbering\briefempfaengerindex{Goldmann, Paul@\textsc{Goldmann, Paul}!zzzSchnitzler, Arthur@\emph{von Arthur Schnitzler}!1896-11-221@{22. 11. 1896}|(be}
\toendnotes[C]{\smallbreak\pagebreak[2]}\buchAlsQuelle{\pwindex{Ritterlichkeit@\emph{Ritterlichkeit}|pwk}Arthur Schnitzler: \emph{Ritterlichkeit. Fragment aus dem Nachlaß}. Bonn: \emph{Bouvier Verlag Herbert Grundmann} 1975, S. 6.}
\buchAbdrucke{\weitereDrucke{Arthur Schnitzler: \emph{Briefe 1875–1912}. Frankfurt am Main: \emph{S. Fischer} 1981, S. 307.} }\toendnotes[C]{\smallbreak}
\pstart
           \noindent{}{\pb}\label{K_L02685-1v}\edtext{ALSO}{\lemma{\textnormal{\emph{Also}}}\Cendnote{\textnormal{Von den Korrespondenzstücken Schnitzlers an Goldmann\pwindex{Goldmann, Paul 31.01.1865 – 25.09.1935@\textsc{Goldmann, Paul} (31.01.1865 – 25.09.1935), \emph{Schriftsteller/Schriftstellerin, Journalist/Journalistin}|pwk} fehlt weitgehend jede Spur. In der Edition von \emph{Ritterlichkeit}\pwindex{Ritterlichkeit@\emph{Ritterlichkeit}|pwk} (1975) schreibt die Herausgeberin Rena R. Schlein\pwindex{Schlein, Rena R. *~1919-06-20@\textsc{Schlein, Rena R.} (*~1919-06-20)|pwk}: »Zwei Telegramme und ein Brief Schnitzlers an Goldmann\pwindex{Goldmann, Paul 31.01.1865 – 25.09.1935@\textsc{Goldmann, Paul} (31.01.1865 – 25.09.1935), \emph{Schriftsteller/Schriftstellerin, Journalist/Journalistin}|pw} wurden mir von Dr. Leo P. Reckford\pwindex{Reckford, Leo P. 1903-05-03 – 1988-10-19@\textsc{Reckford, Leo P.} (1903-05-03 – 1988-10-19), \emph{Laryngologe/Laryngologin}|pw}, der diese Dokumente von der Familie
                           Goldmanns\pwindex{Goldmann, Paul 31.01.1865 – 25.09.1935@\textsc{Goldmann, Paul} (31.01.1865 – 25.09.1935), \emph{Schriftsteller/Schriftstellerin, Journalist/Journalistin}|pw} zum Geschenk bekam, für
                        meine Arbeit zur Verfügung gestellt« (S. 1). Reckford\pwindex{Reckford, Leo P. 1903-05-03 – 1988-10-19@\textsc{Reckford, Leo P.} (1903-05-03 – 1988-10-19), \emph{Laryngologe/Laryngologin}|pwk} starb 1988, seine
                     Nachkommen haben keine Kenntnis von diesen (und etwaigen weiteren)
                     Korrespondenzstücken und sie sind auch nicht auffindbar. Rena R. Schlein\pwindex{Schlein, Rena R. *~1919-06-20@\textsc{Schlein, Rena R.} (*~1919-06-20)|pwk} kam 1919 zur Welt. Ein
                     Kontakt konnte nicht hergestellt werden. Während von dem anderen Telegramm
                        (Arthur Schnitzler an Paul Goldmann, 22. 11. 1896) eine Fotokopie und von dem
                     Brief Teile als Fotokopie (Arthur Schnitzler an Paul Goldmann, 22. 11. 1896) im
                     Nachlass Schnitzlers liegen, gibt es für
                     dieses Telegramm keine erhaltene Vorlage.}}}\label{K_L02685-1} DAZU SCHREIB ICH EXTRA \label{K_L02685-2v}\edtext{STUECKE GEGENS
                     DUELL\pwindex{Liebelei. Schauspiel in drei Akten@\emph{Liebelei. Schauspiel in drei Akten}|pwv}\pwindex{Freiwild. Schauspiel in 3 Akten@\emph{Freiwild. Schauspiel in 3 Akten}|pwv}}{\lemma{\textnormal{\emph{Stuecke gegens
                     Duell}}}\Cendnote{\textnormal{\emph{Liebelei}\pwindex{Liebelei. Schauspiel in drei Akten@\emph{Liebelei. Schauspiel in drei Akten}|pwk} und \emph{Freiwild}\pwindex{Freiwild. Schauspiel in 3 Akten@\emph{Freiwild. Schauspiel in 3 Akten}|pwk}}}}\label{K_L02685-2}\pend
           
\pstart
           TAUSEND GRUESSE UND GLUECKWUENSCHE{\\[\baselineskip]}\spacefill\mbox{ARTHUR}\pend
           \leftskip=0em{}\selectlanguage{ngerman}\endnumbering\briefempfaengerindex{Goldmann, Paul@\textsc{Goldmann, Paul}!zzzSchnitzler, Arthur@\emph{von Arthur Schnitzler}!1896-11-221@{22. 11. 1896}|)be}\mylabel{L02685h}  \normalsize

\doendnotes{C}
\bigskip
\vfill

\clearpage

\footnotesize

\lohead{\textsc{register}}

% Definiere theindex-Environment komplett neu ohne reledmac
\makeatletter
\renewenvironment{theindex}{%
  \section*{\indexname}%
  \setlength{\parindent}{0pt}%
  \setlength{\parskip}{0pt plus 0.3pt}%
  \let\item\@idxitem
}{%
  \clearpage
}
\makeatother

\IfFileExists{\jobname-pw.ind}{\input{\jobname-pw.ind}}{}

\end{document}

      