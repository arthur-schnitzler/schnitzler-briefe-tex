%% latex-leseansicht-vorspann.tex
%% Vorspann für die Leseansicht.
%% Lädt die gemeinsame Datei latex-vorspann.tex mit nicht gesetztem Schalter.

\newif\ifkorrekturansicht
\korrekturansichtfalse

\input{../tex-inputs/latex-vorspann}


\section[Arthur Schnitzler an Paul Goldmann, 22. 11. 1896]{L02685 Arthur Schnitzler an Paul Goldmann, 22. 11. 1896}
\nopagebreak\mylabel{L02685v}
\rehead{ }\normalsize\beginnumbering\briefempfaengerindex{Goldmann, Paul@\textsc{Goldmann, Paul}!zzzSchnitzler, Arthur@\emph{von Arthur Schnitzler}!1896-11-221@{22. 11. 1896}|(be}
\toendnotes[C]{\smallbreak\pagebreak[2]}
\correspDesc{Versand  durch Arthur Schnitzler am 22. 11. 1896 in Wien
\newline{}Erhalt  durch Paul Goldmann am 22. 11. 1896 in Paris}\toendnotes[C]{\smallbreak}
\buchAlsQuelle{\pwindex{Schnitzler, Arthur 15.\,5.\,1862 Wien – 21.\,10.\,1931 ebd.@\textsc{Schnitzler, Arthur} (15.\,5.\,1862 Wien – 21.\,10.\,1931 ebd.), \emph{Schriftsteller, Mediziner}!Ritterlichkeit@\strich\emph{Ritterlichkeit}|pwk}Arthur Schnitzler: \emph{Ritterlichkeit. Fragment aus dem Nachlaß}. Bonn: \emph{Bouvier Verlag Herbert Grundmann} 1975, S. 6 (Abhandlungen zur Kunst-, Musik- und
                        Literaturwissenschaft, 176).}
\buchAbdrucke{\weitereDrucke{Arthur Schnitzler: \emph{Briefe 1875–1912}. Herausgegeben von Therese Nickl und Heinrich Schnitzler. Frankfurt am Main: \emph{S. Fischer} 1981, S. 307.} }\toendnotes[C]{\smallbreak}
\pstart
           \noindent{}{\pb}\label{K_L02685-1v}\edtext{ALSO}{\lemma{\textnormal{\emph{Also}}}\Cendnote{\textnormal{Von den Korrespondenzstücken Schnitzlers an Goldmann\pwindex{Goldmann, Paul 31.\,1.\,1865 Breslau – 25.\,9.\,1935 Wien@\textsc{Goldmann, Paul} (31.\,1.\,1865 Breslau – 25.\,9.\,1935 Wien), \emph{Schriftsteller, Journalist}|pwk} fehlt weitgehend jede Spur. In der Edition von \emph{Ritterlichkeit}\pwindex{Schnitzler, Arthur 15.\,5.\,1862 Wien – 21.\,10.\,1931 ebd.@\textsc{Schnitzler, Arthur} (15.\,5.\,1862 Wien – 21.\,10.\,1931 ebd.), \emph{Schriftsteller, Mediziner}!Ritterlichkeit@\strich\emph{Ritterlichkeit}|pwk} (1975) schreibt die Herausgeberin Rena R. Schlein\pwindex{Schlein, Rena R. *~20.\,6.\,1919 Wien@\textsc{Schlein, Rena R.} (*~20.\,6.\,1919 Wien)|pwk}: »Zwei Telegramme und ein Brief Schnitzlers an Goldmann\pwindex{Goldmann, Paul 31.\,1.\,1865 Breslau – 25.\,9.\,1935 Wien@\textsc{Goldmann, Paul} (31.\,1.\,1865 Breslau – 25.\,9.\,1935 Wien), \emph{Schriftsteller, Journalist}|pw} wurden mir von Dr. Leo P. Reckford\pwindex{Reckford, Leo P. 3.\,5.\,1903 Wien – 19.\,10.\,1988 Manhattan@\textsc{Reckford, Leo P.} (3.\,5.\,1903 Wien – 19.\,10.\,1988 Manhattan), \emph{Laryngologe}|pw}, der diese Dokumente von der Familie
                           Goldmanns\pwindex{Goldmann, Paul 31.\,1.\,1865 Breslau – 25.\,9.\,1935 Wien@\textsc{Goldmann, Paul} (31.\,1.\,1865 Breslau – 25.\,9.\,1935 Wien), \emph{Schriftsteller, Journalist}|pw} zum Geschenk bekam, für
                        meine Arbeit zur Verfügung gestellt« (S. 1). Reckford\pwindex{Reckford, Leo P. 3.\,5.\,1903 Wien – 19.\,10.\,1988 Manhattan@\textsc{Reckford, Leo P.} (3.\,5.\,1903 Wien – 19.\,10.\,1988 Manhattan), \emph{Laryngologe}|pwk} starb 1988, seine
                     Nachkommen haben keine Kenntnis von diesen (und etwaigen weiteren)
                     Korrespondenzstücken und sie sind auch nicht auffindbar. Rena R. Schlein\pwindex{Schlein, Rena R. *~20.\,6.\,1919 Wien@\textsc{Schlein, Rena R.} (*~20.\,6.\,1919 Wien)|pwk} kam 1919 zur Welt. Ein
                     Kontakt konnte nicht hergestellt werden. Während von dem anderen Telegramm
                        (XXXX Auszeichnungsfehler: Dokument L02685 nicht gefunden) eine Fotokopie und von dem
                     Brief Teile als Fotokopie (XXXX Auszeichnungsfehler: Dokument L02686 nicht gefunden) im
                     Nachlass Schnitzlers liegen, gibt es für
                     dieses Telegramm keine erhaltene Vorlage.}}}\label{K_L02685-1} DAZU SCHREIB ICH EXTRA \label{K_L02685-2v}\edtext{STUECKE GEGENS
                     DUELL\pwindex{Schnitzler, Arthur 15.\,5.\,1862 Wien – 21.\,10.\,1931 ebd.@\textsc{Schnitzler, Arthur} (15.\,5.\,1862 Wien – 21.\,10.\,1931 ebd.), \emph{Schriftsteller, Mediziner}!Liebelei. Schauspiel in drei Akten@\strich\emph{Liebelei. Schauspiel in drei Akten}|pwv}\pwindex{Schnitzler, Arthur 15.\,5.\,1862 Wien – 21.\,10.\,1931 ebd.@\textsc{Schnitzler, Arthur} (15.\,5.\,1862 Wien – 21.\,10.\,1931 ebd.), \emph{Schriftsteller, Mediziner}!Freiwild. Schauspiel in 3 Akten@\strich\emph{Freiwild. Schauspiel in 3 Akten}|pwv}}{\lemma{\textnormal{\emph{Stuecke gegens
                     Duell}}}\Cendnote{\textnormal{\emph{Liebelei}\pwindex{Schnitzler, Arthur 15.\,5.\,1862 Wien – 21.\,10.\,1931 ebd.@\textsc{Schnitzler, Arthur} (15.\,5.\,1862 Wien – 21.\,10.\,1931 ebd.), \emph{Schriftsteller, Mediziner}!Liebelei. Schauspiel in drei Akten@\strich\emph{Liebelei. Schauspiel in drei Akten}|pwk} und \emph{Freiwild}\pwindex{Schnitzler, Arthur 15.\,5.\,1862 Wien – 21.\,10.\,1931 ebd.@\textsc{Schnitzler, Arthur} (15.\,5.\,1862 Wien – 21.\,10.\,1931 ebd.), \emph{Schriftsteller, Mediziner}!Freiwild. Schauspiel in 3 Akten@\strich\emph{Freiwild. Schauspiel in 3 Akten}|pwk}}}}\label{K_L02685-2}\pend
           
\pstart
           TAUSEND GRUESSE UND GLUECKWUENSCHE{\\[\baselineskip]}\spacefill\mbox{ARTHUR}\pend
           \leftskip=0em{}\selectlanguage{ngerman}\endnumbering\briefempfaengerindex{Goldmann, Paul@\textsc{Goldmann, Paul}!zzzSchnitzler, Arthur@\emph{von Arthur Schnitzler}!1896-11-221@{22. 11. 1896}|)be}\mylabel{L02685h}  \newcommand{\dateiname}{L02685}\newcommand{\titel}{Arthur Schnitzler an Paul Goldmann, 22. 11. 1896}\newcommand{\editorInnen}{Martin Anton Müller und Laura Untner}%% latex-leseansicht-abspann.tex
%% Abspann für die Leseansicht.
%% Der Schalter \ifkorrekturansicht ist bereits durch den Vorspann gesetzt.

%% latex-abspann.tex
%% Gemeinsamer Abspann für Korrekturansicht und Leseansicht.
%% Setzt den Schalter \ifkorrekturansicht voraus (gesetzt in den
%% einbindenden Dateien latex-korrekturansicht-abspann.tex bzw.
%% latex-leseansicht-abspann.tex).
%% ---------------------------------------------------------------

\normalsize

% Das esempio-Environment wird nur in der Leseansicht benötigt
\ifkorrekturansicht\else
\newenvironment{esempio}[3]%
{
    \vspace{1.5ex}
    \rlap{\underline{#1}}
    \par
    \setlength{\parindent}{0cm}
    \nopagebreak
    \leftskip=#2cm
    \rightskip=#3cm
}
{
    \par
}
\fi

\doendnotes{C}
\bigskip
\vfill

\clearpage

\footnotesize

\ifkorrekturansicht
  \lohead{\textsc{register}}
\fi

% theindex-Environment neu definieren ohne reledmac
\makeatletter
\renewenvironment{theindex}{%
  \ifkorrekturansicht
    \section*{\indexname}%
  \else
    \subsubsection*{Index der erwähnten Entitäten}%
  \fi
  \setlength{\parindent}{0pt}%
  \setlength{\parskip}{0pt plus 0.3pt}%
  \let\item\@idxitem
}{%
  \ifkorrekturansicht\clearpage\fi
}
\makeatother

\IfFileExists{\jobname-pw.ind}{\input{\jobname-pw.ind}}{}

% Quellenangabe nur in der Leseansicht
\ifkorrekturansicht\else
% Fallback-Definitionen, falls die .tex-Datei \titel etc. nicht gesetzt hat
\providecommand{\titel}{}
\providecommand{\editorInnen}{}
\providecommand{\dateiname}{\jobname}

\vspace{3cm}

\vfill

\footnotesize
\textsc{Quelle}: \titel. Herausgegeben von {\editorInnen}. In: \emph{Arthur Schnitzler: Briefwechsel mit Autorinnen und Autoren}.
 Digitale Edition, https://schnitzler-briefe.acdh.oeaw.ac.at/{\dateiname}.html (Stand \today)
\fi

\end{document}


