%% latex-korrekturansicht-vorspann.tex
%% Vorspann für die Korrekturansicht.
%% Lädt die gemeinsame Datei latex-vorspann.tex mit gesetztem Schalter.

\newif\ifkorrekturansicht
\korrekturansichttrue

\input{../tex-inputs/latex-vorspann}


\section[Arthur Schnitzler an Richard Beer-Hofmann, 18. 4. 1901]{L01109 Arthur Schnitzler an Richard Beer-Hofmann, 18. 4. 1901}
\nopagebreak\mylabel{L01109v}
\rehead{ }\normalsize\beginnumbering\briefempfaengerindex{Beer-Hofmann, Richard@\textsc{Beer-Hofmann, Richard}!zzzSchnitzler, Arthur@\emph{von Arthur Schnitzler}!1901-04-181@{18. 4. 1901}|(be}
\toendnotes[C]{\smallbreak\pagebreak[2]}\Standort{YCGL, MSS 31.}
\physDesc{Bildpostkarte, 74 Zeichen
\newline{}Handschrift: Bleistift, deutsche Kurrent
\newline{}Versand: 1) Stempel: »\nobreak{}\oindex{Bologna@\textbf{Bologna}, \emph{P.PPLA}|pwk}Bologna Ferrovia, 18 4–01, 1M\nobreak{}«.   2) Stempel: »\nobreak{}\oindex{I., Innere Stadt@\textbf{I., Innere Stadt}, \emph{A.ADM3}|pwk}Wien 1/1 1, 19 4. 1901, 8–9½V, Bestellt\nobreak{}«. 
\newline{}Ordnung: mit Bleistift von unbekannter Hand datiert: »18. 4.« }\pstart{}{\pb}Dr. \textsc{Richard Beer-Hofmann}\pend{}\pstart{}\textsc{Wien}\oindex{Wien@\textbf{Wien}, \emph{A.ADM2}|pw}\pend{}\pstart{}\textsc{I. Wollzeile 15}\oindex{Wollzeile@\textbf{Wollzeile}, \emph{Straße (K.STR)}|pw}.\pend{}\pstart{}\textsc{Austria}\oindex{Oesterreich@\textbf{Österreich}, \emph{A.PCLI}|pw}\pend{}{\bigskip}
\pstart
           \noindent{}\raggedleft{}{\pb}\textcolor{gray}{\textbf{Ricordo di Bologna\oindex{Bologna@\textbf{Bologna}, \emph{P.PPLA}|pw}.}}\pend
           
\pstart
           \centering{}\textcolor{gray}{\textbf{Chiesa di S. Francesco\oindex{Basilica di San Francesco@\textbf{Basilica di San Francesco}, \emph{Kirche (K.KRC)}|pw}.}}\pend
           \vspace{1em}
\pstart
           \noindent{}{\pb}Herzliche Grüße.\pend
           \pstart Ihr\spacefill\mbox{Arth.}\pend{}\selectlanguage{ngerman}\endnumbering\briefempfaengerindex{Beer-Hofmann, Richard@\textsc{Beer-Hofmann, Richard}!zzzSchnitzler, Arthur@\emph{von Arthur Schnitzler}!1901-04-181@{18. 4. 1901}|)be}\mylabel{L01109h}  \normalsize

\doendnotes{C}
\bigskip
\vfill

\clearpage

\footnotesize

\lohead{\textsc{register}}

% Definiere theindex-Environment komplett neu ohne reledmac
\makeatletter
\renewenvironment{theindex}{%
  \section*{\indexname}%
  \setlength{\parindent}{0pt}%
  \setlength{\parskip}{0pt plus 0.3pt}%
  \let\item\@idxitem
}{%
  \clearpage
}
\makeatother

\IfFileExists{\jobname-pw.ind}{\input{\jobname-pw.ind}}{}

\end{document}

      