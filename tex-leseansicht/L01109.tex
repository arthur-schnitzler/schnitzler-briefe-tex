%% latex-leseansicht-vorspann.tex
%% Vorspann für die Leseansicht.
%% Lädt die gemeinsame Datei latex-vorspann.tex mit nicht gesetztem Schalter.

\newif\ifkorrekturansicht
\korrekturansichtfalse

\input{../tex-inputs/latex-vorspann}


         
         \newcommand{\erwaehntePersonen}{Personen: Richard Beer-Hofmann}
         \newcommand{\erwaehnteOrte}{Orte: Basilica di San Francesco, Bologna, I., Innere Stadt, Wien, Wollzeile, Österreich}
         \newcommand{\erwaehnteWerke}{
               \section[Arthur Schnitzler an Richard Beer-Hofmann, 18. 4. 1901]{ Arthur Schnitzler an Richard Beer-Hofmann, 18. 4. 1901}\nopagebreak\mylabel{v}\rehead{ }\begin{ledgroupsized}[t]{13cm}\normalsize\beginnumbering \toendnotes[C]{\smallbreak\pagebreak[2]} \Standort{YCGL, MSS 31.}
\physDesc{Bildpostkarte
\newline{}Handschrift: Bleistift, deutsche Kurrent\newline{}Versand: 1) Stempel: »\nobreak{}\oindex{Bologna@\textbf{Bologna}|pwk}Bologna Ferrovia, 18 4–01, 1M\nobreak{}«.   2) Stempel: »\nobreak{}\oindex{I., Innere Stadt@\textbf{I., Innere Stadt}|pwk}Wien 1/1 1, 19 4. 1901, 8–9½V, Bestellt\nobreak{}«. \newline{}Ordnung: mit Bleistift von unbekannter Hand
                                    datiert: »18. 4.« }\pstart{}{\pb}Dr. \textsc{Richard
                            Beer-Hofmann}\pend{}\pstart{}\textsc{Wien}\oindex{Wien@\textbf{Wien}|pw}\pend{}\pstart{}\textsc{I. Wollzeile 15}\oindex{Wollzeile@\textbf{Wollzeile}|pw}.\pend{}\pstart{}\textsc{Austria}\oindex{Oesterreich@\textbf{Österreich}|pw}\pend{}{\bigskip}\pstart
           \noindent{}\raggedleft{}{\pb}\textcolor{gray}{\textbf{Ricordo di Bologna\oindex{Bologna@\textbf{Bologna}|pw}.}}\pend
           \pstart
           \noindent{}\centering{}\textcolor{gray}{\textbf{Chiesa di S. Francesco\oindex{Basilica di San Francesco@\textbf{Basilica di San Francesco}|pw}.}}\pend
           \pstart
           Herzliche Grüße.\pend
           \pstart Ihr\spacefill\mbox{Arth.}\pend{}
         
         \endnumbering\mylabel{h}\end{ledgroupsized}  \newcommand{\dateiname}{L01109}\newcommand{\titel}{Arthur Schnitzler an Richard Beer-Hofmann, 18. 4. 1901}\newcommand{\editorInnen}{Martin Anton Müller und Gerd-Hermann Susen}%% latex-leseansicht-abspann.tex
%% Abspann für die Leseansicht.
%% Der Schalter \ifkorrekturansicht ist bereits durch den Vorspann gesetzt.

%% latex-abspann.tex
%% Gemeinsamer Abspann für Korrekturansicht und Leseansicht.
%% Setzt den Schalter \ifkorrekturansicht voraus (gesetzt in den
%% einbindenden Dateien latex-korrekturansicht-abspann.tex bzw.
%% latex-leseansicht-abspann.tex).
%% ---------------------------------------------------------------

\normalsize

% Das esempio-Environment wird nur in der Leseansicht benötigt
\ifkorrekturansicht\else
\newenvironment{esempio}[3]%
{
    \vspace{1.5ex}
    \rlap{\underline{#1}}
    \par
    \setlength{\parindent}{0cm}
    \nopagebreak
    \leftskip=#2cm
    \rightskip=#3cm
}
{
    \par
}
\fi

\doendnotes{C}
\bigskip
\vfill

\clearpage

\footnotesize

\ifkorrekturansicht
  \lohead{\textsc{register}}
\fi

% theindex-Environment neu definieren ohne reledmac
\makeatletter
\renewenvironment{theindex}{%
  \ifkorrekturansicht
    \section*{\indexname}%
  \else
    \subsubsection*{Index der erwähnten Entitäten}%
  \fi
  \setlength{\parindent}{0pt}%
  \setlength{\parskip}{0pt plus 0.3pt}%
  \let\item\@idxitem
}{%
  \ifkorrekturansicht\clearpage\fi
}
\makeatother

\IfFileExists{\jobname-pw.ind}{\input{\jobname-pw.ind}}{}

% Quellenangabe nur in der Leseansicht
\ifkorrekturansicht\else
% Fallback-Definitionen, falls die .tex-Datei \titel etc. nicht gesetzt hat
\providecommand{\titel}{}
\providecommand{\editorInnen}{}
\providecommand{\dateiname}{\jobname}

\vspace{3cm}

\vfill

\footnotesize
\textsc{Quelle}: \titel. Herausgegeben von {\editorInnen}. In: \emph{Arthur Schnitzler: Briefwechsel mit Autorinnen und Autoren}.
 Digitale Edition, https://schnitzler-briefe.acdh.oeaw.ac.at/{\dateiname}.html (Stand \today)
\fi

\end{document}


      