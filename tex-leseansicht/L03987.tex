%% latex-leseansicht-vorspann.tex
%% Vorspann für die Leseansicht.
%% Lädt die gemeinsame Datei latex-vorspann.tex mit nicht gesetztem Schalter.

\newif\ifkorrekturansicht
\korrekturansichtfalse

\input{../tex-inputs/latex-vorspann}


\section[Arthur Schnitzler an Berta Zuckerkandl, 25. 3. 1915]{L03987 Arthur Schnitzler an Berta Zuckerkandl, 25. 3. 1915}
\nopagebreak\mylabel{L03987v}
\rehead{ }\normalsize\beginnumbering\briefempfaengerindex{Zuckerkandl, Berta@\textsc{Zuckerkandl, Berta}!zzzSchnitzler, Arthur@\emph{von Arthur Schnitzler}!1915-03-251@{25. 3. 1915}|(be}
\toendnotes[C]{\smallbreak\pagebreak[2]}
\correspDesc{Versand  durch Arthur Schnitzler am 25. 3. 1915 in Wien
\newline{}Erhalt  durch Berta Zuckerkandl im Zeitraum [25. 3. 1915 – 28. 3. 1915?] in Wien}\toendnotes[C]{\smallbreak}
\Standort{Wien, Österreichische Nationalbibliothek, 405/B78/2 LIT MAG.}
\physDesc{Briefkarte, 874 Zeichen
\newline{}Handschrift: Bleistift, lateinische Kurrent}\toendnotes[C]{\smallbreak}
\pstart
           {\pb}\textcolor{gray}{\textbf{D\textsuperscript{r} Arthur Schnitzler
                     }}\hfill 25. 3. 1915\pend
           
\pstart
           \textcolor{gray}{\textbf{Wien, XVIII.
                        Sternwartestrasse 71\oindex{Wien@\textbf{Wien}!XVIII., Währing@\textbf{XVIII., Währing}!Sternwartestraße 71@\textbf{Sternwartestraße 71}, \emph{Wohngebäude}|pw}.}}\pend
           \vspace{0.5em}
\pstart
           Verehrte gnädige Frau,{ }Olga\pwindex{Schnitzler, Olga 17.\,1.\,1882 Wien – 13.\,1.\,1970 Lugano@\textsc{Schnitzler, Olga} (17.\,1.\,1882 Wien – 13.\,1.\,1970 Lugano), \emph{Schauspielerin, Sängerin}|pw} speist bei Bachrachs\pwindex{Bachrach, Eugenie 4.\,3.\,1857 Wien – 4.\,12.\,1937 Purkersdorf@\textsc{Bachrach, Eugenie} (4.\,3.\,1857 Wien – 4.\,12.\,1937 Purkersdorf)|pw}\pwindex{Zuckerkandl, Marianne 6.\,8.\,1882 Wien – 1964 Ascona@\textsc{Zuckerkandl, Marianne} (6.\,8.\,1882 Wien – 1964 Ascona), \emph{Übersetzerin}|pw} – ich ko{\geminationm}e eben \label{K_L03987-1v}\edtext{vom Anninger\oindex{Anninger@\textbf{Anninger}, \emph{Berg}|pw}}{\lemma{\textnormal{\emph{vom Anninger}}}\Cendnote{\textnormal{Vgl. A. S.: \emph{Tagebuch}, 25. 3. 1912.}}}\label{K_L03987-1} – so war ich so kühn \label{K_L03987-2v}\edtext{Brief}{\lemma{\textnormal{\emph{Brief}}}\Cendnote{\textnormal{nicht überliefert}}}\label{K_L03987-2} zu eröffnen. Vor allem
               wünsch ich gute Besserung – ferner, da wir Krieg u
                  Frieden\pwindex{Tolstoi, Lew Nikolajewitsch 9.\,9.\,1828 Yasnaya Polyana – 20.\,11.\,1910 Lev Tolstoy@\textsc{Tolstoi, Lew Nikolajewitsch} (9.\,9.\,1828 Yasnaya Polyana – 20.\,11.\,1910 Lev Tolstoy), \emph{Schriftsteller}!Krieg und Frieden@\strich\emph{Krieg und Frieden}|pw} schmählicher Weise nicht besitzten, erlaube ich mir die herrlichen
               Tolstoi\pwindex{Tolstoi, Lew Nikolajewitsch 9.\,9.\,1828 Yasnaya Polyana – 20.\,11.\,1910 Lev Tolstoy@\textsc{Tolstoi, Lew Nikolajewitsch} (9.\,9.\,1828 Yasnaya Polyana – 20.\,11.\,1910 Lev Tolstoy), \emph{Schriftsteller}|pw}ſchen Novellen, und – da wir schon in Rußland\oindex{Russland@\textbf{Russland}|pw} sind (\label{K_L03987-3v}\edtext{nebbich}{\lemma{\textnormal{\emph{nebbich}}}\Cendnote{\textnormal{jiddisch: wenn schon}}}\label{K_L03987-3}) – die sehr schönen Tschechow\pwindex{Čechov, Anton Pavlovič 17.\,1.\,1860 Taganrog – 15.\,7.\,1904 Badenweiler@\textsc{Čechov, Anton Pavlovič} (17.\,1.\,1860 Taganrog – 15.\,7.\,1904 Badenweiler), \emph{Schriftsteller}|pw}ſchen zu übersenden. Auch den wundervollen Ulenspiegel\pwindex{Coster, Charles de 20.\,8.\,1827 München – 7.\,5.\,1879 Ixelles@\textsc{Coster, Charles de} (20.\,8.\,1827 München – 7.\,5.\,1879 Ixelles), \emph{Schriftsteller}!Tyll Ulenspiegel und Lamm Goedzak@\strich\emph{Tyll Ulenspiegel und Lamm Goedzak}|pw}{ }{\pb}leg ich bei – de{\geminationn} mir ist als sagten Sie einmal, Sie hätten ihn noch
               nicht gelesen. Alles passt in unsre große Zeit. (Was werden wir nur anfangen, we{\geminationn} sie – noch größer wird? –) Furchtbar was Sie \label{K_L03987-4v}\edtext{von
               Prz\oindex{Przemyśl@\textbf{Przemyśl}, \emph{Hauptstadt}|pw}}{\lemma{\textnormal{\emph{von
               Prz}}}\Cendnote{\textnormal{Am 22. 3. 1915 kapitulierte die in der Garnison Przemyśl\oindex{Przemyśl@\textbf{Przemyśl}, \emph{Hauptstadt}|pwk} stationierte
                  \emph{österreichisch-ungarische Armee}\orgindex{Streitkräfte von Österreich-Ungarn@Streitkräfte von Österreich-Ungarn|pwk} nach vier Monaten Belagerung. Hermann Kusmanek\pwindex{Kusmanek, Hermann 16.\,9.\,1860 Sibiu – 7.\,8.\,1934 Wien@\textsc{Kusmanek, Hermann} (16.\,9.\,1860 Sibiu – 7.\,8.\,1934 Wien), \emph{General}|pwk} übergab die Festung an die russischen\oindex{Russland@\textbf{Russland}|pwk} Gegner und trat eine Kriegsgefangenschaft
                  an, die bis 1918 dauerte.}}}\label{K_L03987-4} schreiben! Ob von dem Schwert, das der tapfre Kusmanek\pwindex{Kusmanek, Hermann 16.\,9.\,1860 Sibiu – 7.\,8.\,1934 Wien@\textsc{Kusmanek, Hermann} (16.\,9.\,1860 Sibiu – 7.\,8.\,1934 Wien), \emph{General}|pw} behalten durfte, auch nur Einer wieder lebendig
               wird? – Und so ließe sich noch allerlei sagen – (sprach Frau Censur und strich
               auch das vorige.)\pend
           
\pstart
           Auf baldiges Wiedersehen und nochmals – gute Besserung{\\[\baselineskip]}herzliche
               Grüße{\\[\baselineskip]}Ihr \spacefill\mbox{A. S.}\pend
           \leftskip=0em{}\selectlanguage{ngerman}\endnumbering\briefempfaengerindex{Zuckerkandl, Berta@\textsc{Zuckerkandl, Berta}!zzzSchnitzler, Arthur@\emph{von Arthur Schnitzler}!1915-03-251@{25. 3. 1915}|)be}\mylabel{L03987h}
\begin{anhang}
\end{anhang}\newcommand{\dateiname}{L03987}\newcommand{\titel}{Arthur Schnitzler an Berta Zuckerkandl, 25. 3. 1915}\newcommand{\editorInnen}{Herausgegeben von Jahnke, SelmaMüller, Martin Anton}%% latex-leseansicht-abspann.tex
%% Abspann für die Leseansicht.
%% Der Schalter \ifkorrekturansicht ist bereits durch den Vorspann gesetzt.

%% latex-abspann.tex
%% Gemeinsamer Abspann für Korrekturansicht und Leseansicht.
%% Setzt den Schalter \ifkorrekturansicht voraus (gesetzt in den
%% einbindenden Dateien latex-korrekturansicht-abspann.tex bzw.
%% latex-leseansicht-abspann.tex).
%% ---------------------------------------------------------------

\normalsize

% Das esempio-Environment wird nur in der Leseansicht benötigt
\ifkorrekturansicht\else
\newenvironment{esempio}[3]%
{
    \vspace{1.5ex}
    \rlap{\underline{#1}}
    \par
    \setlength{\parindent}{0cm}
    \nopagebreak
    \leftskip=#2cm
    \rightskip=#3cm
}
{
    \par
}
\fi

\doendnotes{C}
\bigskip
\vfill

\clearpage

\footnotesize

\ifkorrekturansicht
  \lohead{\textsc{register}}
\fi

% theindex-Environment neu definieren ohne reledmac
\makeatletter
\renewenvironment{theindex}{%
  \ifkorrekturansicht
    \section*{\indexname}%
  \else
    \subsubsection*{Index der erwähnten Entitäten}%
  \fi
  \setlength{\parindent}{0pt}%
  \setlength{\parskip}{0pt plus 0.3pt}%
  \let\item\@idxitem
}{%
  \ifkorrekturansicht\clearpage\fi
}
\makeatother

\IfFileExists{\jobname-pw.ind}{\input{\jobname-pw.ind}}{}

% Quellenangabe nur in der Leseansicht
\ifkorrekturansicht\else
% Fallback-Definitionen, falls die .tex-Datei \titel etc. nicht gesetzt hat
\providecommand{\titel}{}
\providecommand{\editorInnen}{}
\providecommand{\dateiname}{\jobname}

\vspace{3cm}

\vfill

\footnotesize
\textsc{Quelle}: \titel. Herausgegeben von {\editorInnen}. In: \emph{Arthur Schnitzler: Briefwechsel mit Autorinnen und Autoren}.
 Digitale Edition, https://schnitzler-briefe.acdh.oeaw.ac.at/{\dateiname}.html (Stand \today)
\fi

\end{document}


