%% latex-leseansicht-vorspann.tex
%% Vorspann für die Leseansicht.
%% Lädt die gemeinsame Datei latex-vorspann.tex mit nicht gesetztem Schalter.

\newif\ifkorrekturansicht
\korrekturansichtfalse

\input{../tex-inputs/latex-vorspann}


         
         \renewcommand{\erwaehntePersonen}{Personen: Paul Goldmann, Leopold Sonnemann, Friedrich Uhl}
         \renewcommand{\erwaehnteInstitutionen}{Institutionen: Frankfurter Zeitung, Volkstheater}
         \renewcommand{\erwaehnteOrte}{Orte: Paris, Wien, rue Richelieu}
         \renewcommand{\erwaehnteWerke}{Werke: Das Märchen. Schauspiel in drei Aufzügen}
               \section[Paul Goldmann an Arthur Schnitzler, 6. 12. {[}1893{]}]{ Paul Goldmann an Arthur Schnitzler, 6. 12. {[}1893{]}}\nopagebreak\mylabel{v}\rehead{ }\begin{ledgroupsized}[t]{13cm}\normalsize\beginnumbering\briefempfaengerindex{Schnitzler, Arthur@\textsc{Schnitzler, Arthur}!zzzGoldmann, Paul@\emph{von Paul Goldmann}!1893-12-061@{6. 12. {[}1893{]}}|(be} \toendnotes[C]{\smallbreak\pagebreak[2]} \Standort{DLA, A:Schnitzler, HS.NZ85.1.3163.}
\physDesc{Brief, 1 Blatt, 2 Seiten, 483 Zeichen
\newline{}Handschrift: schwarze Tinte, deutsche Kurrent
\newline{}Schnitzler: mit Bleistift das Jahr »93« vermerkt }\toendnotes[C]{\smallbreak}\pstart
           \noindent{}{\pb}\textcolor{gray}{\textbf{\textbf{Frankfurter Zeitung\orgindex{Frankfurter Zeitung@Frankfurter Zeitung|pw}.}}}\pend
           \pstart
           \textcolor{gray}{\textbf{\textbf{(\begin{otherlanguage}{french}Gazette de Francfort\end{otherlanguage}\orgindex{Frankfurter Zeitung@Frankfurter Zeitung|pw}.)}}}\pend
           \pstart
           \textcolor{gray}{\textbf{\begin{otherlanguage}{french}Directeur\end{otherlanguage}{ }\textbf{M. L. Sonnemann\pwindex{Sonnemann, Leopold 1831-10-29 – 1909-10-30@\textsc{Sonnemann, Leopold} (1831-10-29 – 1909-10-30), \emph{Journalist, Herausgeber}|pw}.}}}\hfill \textsc{Paris\oindex{Paris@\textbf{Paris}|pw}}, 6. December\textcolor{gray}{.}\pend
           \pstart
           \begin{otherlanguage}{french}\textcolor{gray}{\textbf{Journal politique, financier,}}\end{otherlanguage}\pend
           \pstart
           \begin{otherlanguage}{french}\textcolor{gray}{\textbf{commercial et litteraire.}}\end{otherlanguage}\pend
           \pstart
           \begin{otherlanguage}{french}\textcolor{gray}{\textbf{\textbf{Paraissant trois fois par jour}}}\end{otherlanguage}\pend
           \pstart
           \begin{otherlanguage}{french}\textcolor{gray}{\textbf{\textbf{Bureaux à Paris\oindex{Paris@\textbf{Paris}|pw}:}}}\end{otherlanguage}\pend
           \pstart
           \begin{otherlanguage}{french}\textcolor{gray}{\textbf{\textbf{rue Richelieu 75\oindex{rue Richelieu@\textbf{rue Richelieu}|pw}.}}}\end{otherlanguage}\pend
           \pstart\center{}Mein lieber Freund!\pend\pstart
           Beilegend eine Zuſchrift \textsc{Uhls\pwindex{Uhl, Friedrich 14.05.1825 – 20.01.1906@\textsc{Uhl, Friedrich} (14.05.1825 – 20.01.1906), \emph{Journalist}|pw}}, die ich heut erhielt. Bitte\textcolor{gray}{,} ſende ſie mir ſofort zurück.\pend
           \pstart
           Und ſchreib’ mir doch endlich einmal zwei Worte.\pend
           \pstart
           Iſt es wahr, daß das Volkstheater\orgindex{Volkstheater@Volkstheater|pw} Dich gleich
               nach der zweiten Vorſtellung \label{K_L02722-1v}\edtext{abgeſetzt}{\lemma{\textnormal{\emph{abgeſetzt}}}\Cendnote{\textnormal{Bereits bei der zweiten
                  und letzten Vorstellung des \emph{Märchens}\pwindex{Schnitzler, Arthur 15.05.1862 – 21.10.1931@\textsc{Schnitzler, Arthur} (15.05.1862 – 21.10.1931), \emph{Schriftsteller, Mediziner}!Maerchen. Schauspiel in drei Aufzuegen1893-12-01@\strich\emph{Das Märchen. Schauspiel in drei Aufzügen} {[}1893-12-01{]}|pwk} am 2. 12. 1893 war kaum
                  Publikum vor Ort. Die Absetzung stand zu diesem Zeitpunkt aufgrund der Schwäche
                  des dritten Akts\pwindex{Schnitzler, Arthur 15.05.1862 – 21.10.1931@\textsc{Schnitzler, Arthur} (15.05.1862 – 21.10.1931), \emph{Schriftsteller, Mediziner}!Maerchen. Schauspiel in drei Aufzuegen1893-12-01@\strich\emph{Das Märchen. Schauspiel in drei Aufzügen} {[}1893-12-01{]}|pwkv} bereits
                  fest. Das Theater\orgindex{Volkstheater@Volkstheater|pwkv} hatte zu
                  verstehen gegeben, dass das Stück\pwindex{Schnitzler, Arthur 15.05.1862 – 21.10.1931@\textsc{Schnitzler, Arthur} (15.05.1862 – 21.10.1931), \emph{Schriftsteller, Mediziner}!Maerchen. Schauspiel in drei Aufzuegen1893-12-01@\strich\emph{Das Märchen. Schauspiel in drei Aufzügen} {[}1893-12-01{]}|pwkv} in einer abgeänderten Fassung wiederaufgenommen würde. Schnitzler\pwindex{Schnitzler, Arthur 15.05.1862 – 21.10.1931@\textsc{Schnitzler, Arthur} (15.05.1862 – 21.10.1931), \emph{Schriftsteller, Mediziner}|pwk}, der bereits für den zweiten Abend
                  den dritten Akt\pwindex{Schnitzler, Arthur 15.05.1862 – 21.10.1931@\textsc{Schnitzler, Arthur} (15.05.1862 – 21.10.1931), \emph{Schriftsteller, Mediziner}!Maerchen. Schauspiel in drei Aufzuegen1893-12-01@\strich\emph{Das Märchen. Schauspiel in drei Aufzügen} {[}1893-12-01{]}|pwkv} gekürzt
                  hatte, unternahm es, den Akt\pwindex{Schnitzler, Arthur 15.05.1862 – 21.10.1931@\textsc{Schnitzler, Arthur} (15.05.1862 – 21.10.1931), \emph{Schriftsteller, Mediziner}!Maerchen. Schauspiel in drei Aufzuegen1893-12-01@\strich\emph{Das Märchen. Schauspiel in drei Aufzügen} {[}1893-12-01{]}|pwkv} umzuschreiben. Zu einer Wiederaufnahme kam es trotzdem nicht. Vgl. A. S.: \emph{Tagebuch}, 2. 12. 1893 und 5. 12. 1893.}}}\label{K_L02722-1h}? Das ſieht der \label{K_L02722-2v}\edtext{feigen und gemeinen Bande\orgindex{Volkstheater@Volkstheater|pwv}}{\lemma{\textnormal{\emph{feigen … Bande}}}\Cendnote{\textnormal{Goldmann\pwindex{Goldmann, Paul 31.01.1865 – 25.09.1935@\textsc{Goldmann, Paul} (31.01.1865 – 25.09.1935), \emph{Schriftsteller, Journalist}|pwk} hielt wenig von der künstlerischen
                  Zugangsweise des Theaters\orgindex{Volkstheater@Volkstheater|pwkv},
                     vgl. Paul Goldmann an Arthur Schnitzler, 18. 8. [1893].}}}\label{K_L02722-2h} ganz
               ähnlich. Wahrſcheinlich ha\textcolor{gray}{b}en die Frauen der Actionäre {\pb}proteſtirt. Die Verherrlichung einer Gefallenen!
                  \label{K_L02722-3v}\edtext{\textsc{\begin{otherlanguage}{french}Pensez donc\end{otherlanguage}}!}{\lemma{\textnormal{\emph{Pensez donc!}}}\Cendnote{\textnormal{französisch: Man stelle sich
                  vor!}}}\label{K_L02722-3h}\pend
           \pstart
           Weiter ſchreiben, liebſter Freund, weiter ſchreiben!\pend
           \pstart
           Dein {\\[\baselineskip]}treuer {\\[\baselineskip]}\spacefill\mbox{Paul Goldmann}\pend
           \leftskip=0em{}
         
         \endnumbering\mylabel{h}\end{ledgroupsized}  \newcommand{\dateiname}{L02722}\newcommand{\titel}{Paul Goldmann an Arthur Schnitzler, 6. 12. [1893]}\newcommand{\editorInnen}{Martin Anton Müller und Laura Untner}%% latex-leseansicht-abspann.tex
%% Abspann für die Leseansicht.
%% Der Schalter \ifkorrekturansicht ist bereits durch den Vorspann gesetzt.

%% latex-abspann.tex
%% Gemeinsamer Abspann für Korrekturansicht und Leseansicht.
%% Setzt den Schalter \ifkorrekturansicht voraus (gesetzt in den
%% einbindenden Dateien latex-korrekturansicht-abspann.tex bzw.
%% latex-leseansicht-abspann.tex).
%% ---------------------------------------------------------------

\normalsize

% Das esempio-Environment wird nur in der Leseansicht benötigt
\ifkorrekturansicht\else
\newenvironment{esempio}[3]%
{
    \vspace{1.5ex}
    \rlap{\underline{#1}}
    \par
    \setlength{\parindent}{0cm}
    \nopagebreak
    \leftskip=#2cm
    \rightskip=#3cm
}
{
    \par
}
\fi

\doendnotes{C}
\bigskip
\vfill

\clearpage

\footnotesize

\ifkorrekturansicht
  \lohead{\textsc{register}}
\fi

% theindex-Environment neu definieren ohne reledmac
\makeatletter
\renewenvironment{theindex}{%
  \ifkorrekturansicht
    \section*{\indexname}%
  \else
    \subsubsection*{Index der erwähnten Entitäten}%
  \fi
  \setlength{\parindent}{0pt}%
  \setlength{\parskip}{0pt plus 0.3pt}%
  \let\item\@idxitem
}{%
  \ifkorrekturansicht\clearpage\fi
}
\makeatother

\IfFileExists{\jobname-pw.ind}{\input{\jobname-pw.ind}}{}

% Quellenangabe nur in der Leseansicht
\ifkorrekturansicht\else
% Fallback-Definitionen, falls die .tex-Datei \titel etc. nicht gesetzt hat
\providecommand{\titel}{}
\providecommand{\editorInnen}{}
\providecommand{\dateiname}{\jobname}

\vspace{3cm}

\vfill

\footnotesize
\textsc{Quelle}: \titel. Herausgegeben von {\editorInnen}. In: \emph{Arthur Schnitzler: Briefwechsel mit Autorinnen und Autoren}.
 Digitale Edition, https://schnitzler-briefe.acdh.oeaw.ac.at/{\dateiname}.html (Stand \today)
\fi

\end{document}


      