%% latex-leseansicht-vorspann.tex
%% Vorspann für die Leseansicht.
%% Lädt die gemeinsame Datei latex-vorspann.tex mit nicht gesetztem Schalter.

\newif\ifkorrekturansicht
\korrekturansichtfalse

\input{../tex-inputs/latex-vorspann}


\section[Arthur Schnitzler an Berta Zuckerkandl, 2. 4. 1925]{L03957 Arthur Schnitzler an Berta Zuckerkandl, 2. 4. 1925}
\nopagebreak\mylabel{L03957v}
\rehead{ }\normalsize\beginnumbering\briefempfaengerindex{Zuckerkandl, Berta@\textsc{Zuckerkandl, Berta}!zzzSchnitzler, Arthur@\emph{von Arthur Schnitzler}!1925-04-021@{2. 4. 1925}|(be}
\toendnotes[C]{\smallbreak\pagebreak[2]}
\correspDesc{Versand  durch Arthur Schnitzler am 2. 4. 1925 in Wien
\newline{}Erhalt  durch Berta Zuckerkandl im Zeitraum [3. 4. 1925
                  – 7. 4. 1925?] in Paris}\toendnotes[C]{\smallbreak}
\Standort{DLA, HS.1985.1.2282.}
\physDesc{Brief, Durchschlag, 2 Blätter, 3 Seiten, 3298 Zeichen
\newline{}Schreibmaschine
\newline{}Handschrift: 1) roter Buntstift, lateinische Kurrent (\noindent{}beschriftet: »\uline{Zuckerkandl}«, einundzwanzig Unterstreichungen)\hspace{1em}2) Bleistift, lateinische Kurrent (\noindent{}beschriftet und datiert auf S. 3: \uline{Zuckerkandl} und »2/4 25«)\hspace{1em}}\toendnotes[C]{\smallbreak}
\pstart
           \raggedleft{}{\pb}2. 4. 1925.\pend
           
\pstart{}Liebe und verehrte Frau Hofrätin.\pend\vspace{0.5em}
\pstart
           Meine \label{K_L03957-1v}\edtext{Karte}{\lemma{\textnormal{\emph{Karte}}}\Cendnote{\textnormal{nicht überliefert}}}\label{K_L03957-1} haben Sie wohl erhalten und sind
               wohl mit mir einverstanden, dass wir vorläufig einmal die Abschrift von ein oder zwei
               Akten der Bianquis’schen\pwindex{Bianquis, Geneviève 19.\,9.\,1886 Rouen – 24.\,3.\,1972 Antony@\textsc{Bianquis, Geneviève} (19.\,9.\,1886 Rouen – 24.\,3.\,1972 Antony), \emph{Übersetzerin, Literaturhistorikerin}|pw}{ }Uebersetzung\pwindex{Schnitzler, Arthur 15. 5. 1862 Wien – 21. 10. 1931 ebd.@\textsc{Schnitzler, Arthur} (15. 5. 1862 Wien – 21. 10. 1931 ebd.), \emph{Schriftsteller, Mediziner}!?? [französische Übersetzung von Der einsame Weg]@\strich\emph{?? [französische Übersetzung von Der einsame Weg]}|pwv} abschreiben
               lassen. Sollten sie es aber für richtig oder auch nur im geringsten aussichtsvoll
               halten gleich die Abschrift des Ganzen\pwindex{Schnitzler, Arthur 15. 5. 1862 Wien – 21. 10. 1931 ebd.@\textsc{Schnitzler, Arthur} (15. 5. 1862 Wien – 21. 10. 1931 ebd.), \emph{Schriftsteller, Mediziner}!?? [französische Übersetzung von Der einsame Weg]@\strich\emph{?? [französische Übersetzung von Der einsame Weg]}|pwv}\pwindex{Schnitzler, Arthur 15. 5. 1862 Wien – 21. 10. 1931 ebd.@\textsc{Schnitzler, Arthur} (15. 5. 1862 Wien – 21. 10. 1931 ebd.), \emph{Schriftsteller, Mediziner}!einsame Weg. Schauspiel in fünf Akten@\strich\emph{Der einsame Weg. Schauspiel in fünf Akten}|pwv} anfertigen zu lassen, so gebe ich Ihnen Vollmacht in
               jeder Ihnen geeignet erscheinenden Weise zu verfügen und komme für die Kosten auf.\pend
           
\pstart
           »Das weite Land\pwindex{Schnitzler, Arthur 15. 5. 1862 Wien – 21. 10. 1931 ebd.@\textsc{Schnitzler, Arthur} (15. 5. 1862 Wien – 21. 10. 1931 ebd.), \emph{Schriftsteller, Mediziner}!weite Land. Tragikomödie in fünf Akten@\strich\emph{Das weite Land. Tragikomödie in fünf Akten}|pw}« dürfte ja an sich mehr Chancen
               haben, wenigstens auf der französischen\oindex{Frankreich@\textbf{Frankreich}|pw} Bühne und
               es ist geradezu rührend, dass Lenormand\pwindex{Lenormand, Henri-René 3.\,5.\,1882 Paris – 16.\,2.\,1951 ebd.@\textsc{Lenormand, Henri-René} (3.\,5.\,1882 Paris – 16.\,2.\,1951 ebd.), \emph{Schriftsteller}|pw} sich
               die Mühe nimmt die Uebersetzung in Stand zu bringen. Bitte danken Sie ihm vielmals in
               meinem Namen; eben lese ich sein schönes \label{K_L03957-2v}\edtext{Novellenbuch\pwindex{Lenormand, Henri-René 3.\,5.\,1882 Paris – 16.\,2.\,1951 ebd.@\textsc{Lenormand, Henri-René} (3.\,5.\,1882 Paris – 16.\,2.\,1951 ebd.), \emph{Schriftsteller}!Le Penseur et la Crétine. Récits@\strich\emph{Le Penseur et la Crétine. Récits}|pwuv}}{\lemma{\textnormal{\emph{Novellenbuch}}}\Cendnote{\textnormal{Es dürfte sich um \emph{Le Penseur et la Crétine}\pwindex{Lenormand, Henri-René 3.\,5.\,1882 Paris – 16.\,2.\,1951 ebd.@\textsc{Lenormand, Henri-René} (3.\,5.\,1882 Paris – 16.\,2.\,1951 ebd.), \emph{Schriftsteller}!Le Penseur et la Crétine. Récits@\strich\emph{Le Penseur et la Crétine. Récits}|pwk} handeln, denn \emph{L'Armée secrète}\pwindex{Lenormand, Henri-René 3.\,5.\,1882 Paris – 16.\,2.\,1951 ebd.@\textsc{Lenormand, Henri-René} (3.\,5.\,1882 Paris – 16.\,2.\,1951 ebd.), \emph{Schriftsteller}!L'Armée secrète suivi de Le Juge intérieur et de Fidélité@\strich\emph{L'Armée secrète suivi de Le Juge intérieur et de Fidélité}|pwk}, die neueste Novellensammlung Lenormands\pwindex{Lenormand, Henri-René 3.\,5.\,1882 Paris – 16.\,2.\,1951 ebd.@\textsc{Lenormand, Henri-René} (3.\,5.\,1882 Paris – 16.\,2.\,1951 ebd.), \emph{Schriftsteller}|pwk}, erschien erst am Ende des Jahres
                     1925 gedruckt.}}}\label{K_L03957-2}, das er so freundlich war mir zu schicken.\pend
           
\pstart
           Was »Liebelei\pwindex{Schnitzler, Arthur 15. 5. 1862 Wien – 21. 10. 1931 ebd.@\textsc{Schnitzler, Arthur} (15. 5. 1862 Wien – 21. 10. 1931 ebd.), \emph{Schriftsteller, Mediziner}!Liebelei. Schauspiel in drei Akten@\strich\emph{Liebelei. Schauspiel in drei Akten}|pw}« anbelangt, so hat die
               wahrscheinlich \label{K_L03957-3v}\edtext{einzige Aufführung\eventindex{Dunkirk@\textbf{Dunkirk}!Aufführung von Amourette. Piéce an trois actes@Aufführung von Amourette. Piéce an trois actes|pwv}}{\lemma{\textnormal{\emph{einzige Aufführung}}}\Cendnote{\textnormal{Zur Aufführung\eventindex{Dunkirk@\textbf{Dunkirk}!Aufführung von Amourette. Piéce an trois actes@Aufführung von Amourette. Piéce an trois actes|pwkv} der \emph{Liebelei}\pwindex{Schnitzler, Arthur 15. 5. 1862 Wien – 21. 10. 1931 ebd.@\textsc{Schnitzler, Arthur} (15. 5. 1862 Wien – 21. 10. 1931 ebd.), \emph{Schriftsteller, Mediziner}!Liebelei. Schauspiel in drei Akten@\strich\emph{Liebelei. Schauspiel in drei Akten}|pwk}\pwindex{Schnitzler, Arthur 15. 5. 1862 Wien – 21. 10. 1931 ebd.@\textsc{Schnitzler, Arthur} (15. 5. 1862 Wien – 21. 10. 1931 ebd.), \emph{Schriftsteller, Mediziner}!Amourette. Pièce en trois actes. Adaptée de Arthur Schnitzler@\strich\emph{Amourette. Pièce en trois actes. Adaptée de Arthur Schnitzler}|pwk} im Kurhaus von Dunkirk\oindex{Dunkirk@\textbf{Dunkirk}, \emph{Hauptstadt}|pwk} am
                  29. 8. 1902 vgl. Karl Zieger: \emph{Arthur Schnitzler et la
                        France 1894–1938. Enquête sur une réception},
                     Villeneuve d’Ascq: \emph{Presses Universitaires du
                        Septentrion} 2012, S. 38.}}}\label{K_L03957-3} in der
               schlechten Uebersetzung\pwindex{Schnitzler, Arthur 15. 5. 1862 Wien – 21. 10. 1931 ebd.@\textsc{Schnitzler, Arthur} (15. 5. 1862 Wien – 21. 10. 1931 ebd.), \emph{Schriftsteller, Mediziner}!Amourette. Pièce en trois actes. Adaptée de Arthur Schnitzler@\strich\emph{Amourette. Pièce en trois actes. Adaptée de Arthur Schnitzler}|pwv} von
                  Jean Thorel\pwindex{Thorel, Jean 11.\,9.\,1859 Éragny – 20.\,8.\,1916 Enghien-les-Bains@\textsc{Thorel, Jean} (11.\,9.\,1859 Éragny – 20.\,8.\,1916 Enghien-les-Bains), \emph{Übersetzer, Dramatiker}|pw} im Jahre 1902 in
                  Dunkerke\oindex{Dunkirk@\textbf{Dunkirk}, \emph{Hauptstadt}|pw} stattgefunden und ich glaube mich
               auch zu erinnern (sonst hätte ich ja auch von der Aufführung\eventindex{Dunkirk@\textbf{Dunkirk}!Aufführung von Amourette. Piéce an trois actes@Aufführung von Amourette. Piéce an trois actes|pwv} nie etwas erfahren) die Summe von zehn Francs
               erhalten zu haben{\dotstwo} Man wird ja jedesfalls eine neue
               Uebersetzung anfertigen müssen; dass das Verfügungsrecht für Frankreich\oindex{Frankreich@\textbf{Frankreich}|pw} längst wieder mir allein gehört, haben wir ja schon
               festgestellt. Ich bin ja auch der Ansicht, dass eine Aufführung der »Liebelei\pwindex{Schnitzler, Arthur 15. 5. 1862 Wien – 21. 10. 1931 ebd.@\textsc{Schnitzler, Arthur} (15. 5. 1862 Wien – 21. 10. 1931 ebd.), \emph{Schriftsteller, Mediziner}!Liebelei. Schauspiel in drei Akten@\strich\emph{Liebelei. Schauspiel in drei Akten}|pw}« an einem guten Pariser\oindex{Paris@\textbf{Paris}, \emph{Hauptstadt}|pw} Theater das Wünschenswerteste wäre. Dazu müsste man die »Literatur\pwindex{Schnitzler, Arthur 15. 5. 1862 Wien – 21. 10. 1931 ebd.@\textsc{Schnitzler, Arthur} (15. 5. 1862 Wien – 21. 10. 1931 ebd.), \emph{Schriftsteller, Mediziner}!Literatur@\strich\emph{Literatur}|pw}« {\pb}geben.
               (Wie es übrigens vor dem Krieg{ }Lugné Poes\pwindex{Lugné-Poe, Aurélien-Marie 27.\,12.\,1869 Paris – 19.\,6.\,1940 Villeneuve-les-Avignon@\textsc{Lugné-Poe, Aurélien-Marie} (27.\,12.\,1869 Paris – 19.\,6.\,1940 Villeneuve-les-Avignon), \emph{Theaterleiter, Regisseur, Schauspieler}|pw} Absicht war).\pend
           
\pstart
           Zur Einakterfrage kann ich begreiflicherweise nichts Neues bemerken. Nach wie vor
               halte ich \label{K_L03957-4v}\edtext{trotz Géraldy\pwindex{Géraldy, Paul 6.\,3.\,1885 Paris – 9.\,3.\,1983 Neuilly-sur-Seine@\textsc{Géraldy, Paul} (6.\,3.\,1885 Paris – 9.\,3.\,1983 Neuilly-sur-Seine), \emph{Schriftsteller}|pw}}{\lemma{\textnormal{\emph{trotz Géraldy}}}\Cendnote{\textnormal{Paul Géraldy\pwindex{Géraldy, Paul 6.\,3.\,1885 Paris – 9.\,3.\,1983 Neuilly-sur-Seine@\textsc{Géraldy, Paul} (6.\,3.\,1885 Paris – 9.\,3.\,1983 Neuilly-sur-Seine), \emph{Schriftsteller}|pwk} hatte Berta Zuckerkandl\pwindex{Zuckerkandl, Berta 13.\,4.\,1864 Wien – 16.\,10.\,1945 Paris@\textsc{Zuckerkandl, Berta} (13.\,4.\,1864 Wien – 16.\,10.\,1945 Paris), \emph{Schriftstellerin, Journalistin, Übersetzerin}|pwk} gegenüber \emph{Die große Szene}\pwindex{Schnitzler, Arthur 15. 5. 1862 Wien – 21. 10. 1931 ebd.@\textsc{Schnitzler, Arthur} (15. 5. 1862 Wien – 21. 10. 1931 ebd.), \emph{Schriftsteller, Mediziner}!Große Szene@\strich\emph{Große Szene}|pwk} als nicht repräsentativ genug bezeichnet, das geht hervor
                  aus dem Brief: Arthur Schnitzler an Paul Géraldy\pwindex{Géraldy, Paul 6.\,3.\,1885 Paris – 9.\,3.\,1983 Neuilly-sur-Seine@\textsc{Géraldy, Paul} (6.\,3.\,1885 Paris – 9.\,3.\,1983 Neuilly-sur-Seine), \emph{Schriftsteller}|pwk}, 31. 7. 1924, \emph{Deutsches Literaturarchiv Marbach},
                     HS.1985.1.811,8.}}}\label{K_L03957-4} »Die grosse
                  Szene\pwindex{Schnitzler, Arthur 15. 5. 1862 Wien – 21. 10. 1931 ebd.@\textsc{Schnitzler, Arthur} (15. 5. 1862 Wien – 21. 10. 1931 ebd.), \emph{Schriftsteller, Mediziner}!Große Szene@\strich\emph{Große Szene}|pw}{[}«{]} für den wirkamsten, vorausgesetzt, dass sich der grosse
               Schauspieler für die Hauptrolle findet. Der »Kak{[}a{]}du\pwindex{Schnitzler, Arthur 15. 5. 1862 Wien – 21. 10. 1931 ebd.@\textsc{Schnitzler, Arthur} (15. 5. 1862 Wien – 21. 10. 1931 ebd.), \emph{Schriftsteller, Mediziner}!grüne Kakadu. Groteske in einem Akt@\strich\emph{Der grüne Kakadu. Groteske in einem Akt}|pw}« kommt gleichfalls in Betracht, obwohl er
               schon bei Antoine\pwindex{Antoine, André 31.\,1.\,1858 Limoges – 23.\,10.\,1943 Le Pouliguen@\textsc{Antoine, André} (31.\,1.\,1858 Limoges – 23.\,10.\,1943 Le Pouliguen), \emph{Theaterleiter, Schauspieler}|pw}{ }\label{K_L03957-5v}\edtext{gespielt wurde}{\lemma{\textnormal{\emph{gespielt wurde}}}\Cendnote{\textnormal{\emph{Der grüne Kakadu}\pwindex{Schnitzler, Arthur 15. 5. 1862 Wien – 21. 10. 1931 ebd.@\textsc{Schnitzler, Arthur} (15. 5. 1862 Wien – 21. 10. 1931 ebd.), \emph{Schriftsteller, Mediziner}!grüne Kakadu. Groteske in einem Akt@\strich\emph{Der grüne Kakadu. Groteske in einem Akt}|pwk} wurde in Übersetzung\pwindex{Schnitzler, Arthur 15. 5. 1862 Wien – 21. 10. 1931 ebd.@\textsc{Schnitzler, Arthur} (15. 5. 1862 Wien – 21. 10. 1931 ebd.), \emph{Schriftsteller, Mediziner}!Au Perroquet Vert@\strich\emph{Au Perroquet Vert}|pwkv} von Stephan Epstein\pwindex{Epstein, Stephan 12.\,11.\,1866 Warschau – 1941 Paris@\textsc{Epstein, Stephan} (12.\,11.\,1866 Warschau – 1941 Paris), \emph{Schriftsteller, Dramaturg, Übersetzer}|pwk} und Lutz
                     Émile\pwindex{Lutz, Émile 8.\,4.\,1868 Saint-Étienne-du-Rouvray – 18.\,1.\,1940 Paris@\textsc{Lutz, Émile} (8.\,4.\,1868 Saint-Étienne-du-Rouvray – 18.\,1.\,1940 Paris), \emph{Übersetzer, Dichter}|pwk} ab dem 7. 11. 1903\eventindex{Théâtre Antoine-Simone Berriau@\textbf{Théâtre Antoine-Simone Berriau}!Premiere von Au Perroquet Vert, 7.11.1903@Premiere von Au Perroquet Vert, 7.11.1903|pwkv} zwölf Mal am Théâtre Antoine\oindex{Théâtre Antoine-Simone Berriau@\textbf{Théâtre Antoine-Simone Berriau}, \emph{Theater}|pwk}
                  gegeben.}}}\label{K_L03957-5}. Immerhin könnte man auch an die »Frau mit dem Dolche\pwindex{Schnitzler, Arthur 15. 5. 1862 Wien – 21. 10. 1931 ebd.@\textsc{Schnitzler, Arthur} (15. 5. 1862 Wien – 21. 10. 1931 ebd.), \emph{Schriftsteller, Mediziner}!Frau mit dem Dolche@\strich\emph{Die Frau mit dem Dolche}|pw}« denken; – die Schwierigkeit wird eben immer bleiben den
               guten Uebersetzer zu entdecken. Wie schade, dass die meisten französischen\oindex{Frankreich@\textbf{Frankreich}|pw} Poeten nicht deutsch können. Ich erinnere
               übrigens daran, dass auch »Die Stunde der
                  Erkenntnis\pwindex{Schnitzler, Arthur 15. 5. 1862 Wien – 21. 10. 1931 ebd.@\textsc{Schnitzler, Arthur} (15. 5. 1862 Wien – 21. 10. 1931 ebd.), \emph{Schriftsteller, Mediziner}!Stunde des Erkennens@\strich\emph{Stunde des Erkennens}|pw}« von Mad. Bianquis\pwindex{Bianquis, Geneviève 19.\,9.\,1886 Rouen – 24.\,3.\,1972 Antony@\textsc{Bianquis, Geneviève} (19.\,9.\,1886 Rouen – 24.\,3.\,1972 Antony), \emph{Übersetzerin, Literaturhistorikerin}|pw}{ }\label{K_L03957-6v}\edtext{übersetzt\pwindex{Schnitzler, Arthur 15. 5. 1862 Wien – 21. 10. 1931 ebd.@\textsc{Schnitzler, Arthur} (15. 5. 1862 Wien – 21. 10. 1931 ebd.), \emph{Schriftsteller, Mediziner}!?? [französische Übersetzung von Stunde des Erkennens]@\strich\emph{?? [französische Übersetzung von Stunde des Erkennens]}|pwv} (?)}{\lemma{\textnormal{\emph{übersetzt (?)}}}\Cendnote{\textnormal{Diese Übersetzung\pwindex{Schnitzler, Arthur 15. 5. 1862 Wien – 21. 10. 1931 ebd.@\textsc{Schnitzler, Arthur} (15. 5. 1862 Wien – 21. 10. 1931 ebd.), \emph{Schriftsteller, Mediziner}!?? [französische Übersetzung von Stunde des Erkennens]@\strich\emph{?? [französische Übersetzung von Stunde des Erkennens]}|pwkv} taucht ansonsten in der Korrespondenz nicht
                  auf (im Gegensatz zu einer Übersetzung\pwindex{Schnitzler, Arthur 15. 5. 1862 Wien – 21. 10. 1931 ebd.@\textsc{Schnitzler, Arthur} (15. 5. 1862 Wien – 21. 10. 1931 ebd.), \emph{Schriftsteller, Mediziner}!?? [französische Übersetzung von Stunde des Erkennens]@\strich\emph{?? [französische Übersetzung von Stunde des Erkennens]}|pwkv}, die Maurice Rémon\pwindex{Rémon, Maurice 27.\,11.\,1861 Paris – 20.\,6.\,1945 Mérignac@\textsc{Rémon, Maurice} (27.\,11.\,1861 Paris – 20.\,6.\,1945 Mérignac), \emph{Übersetzer}|pwk}
                  anfertigt, vgl. XXXX Auszeichnungsfehler: Dokument L03956 nicht gefunden).}}}\label{K_L03957-6}
               ist und dass Lugné Poé\pwindex{Lugné-Poe, Aurélien-Marie 27.\,12.\,1869 Paris – 19.\,6.\,1940 Villeneuve-les-Avignon@\textsc{Lugné-Poe, Aurélien-Marie} (27.\,12.\,1869 Paris – 19.\,6.\,1940 Villeneuve-les-Avignon), \emph{Theaterleiter, Regisseur, Schauspieler}|pw} eine \label{K_L03957-7v}\edtext{Aufführung in Betracht}{\lemma{\textnormal{\emph{Aufführung in Betracht}}}\Cendnote{\textnormal{Vgl. dazu Arthur Schnitzler an Maurice
                  Rémon\pwindex{Rémon, Maurice 27.\,11.\,1861 Paris – 20.\,6.\,1945 Mérignac@\textsc{Rémon, Maurice} (27.\,11.\,1861 Paris – 20.\,6.\,1945 Mérignac), \emph{Übersetzer}|pwk}, 24. 11. 1924, \emph{Deutsches Literaturarchiv Marbach},
                  HS.1985.1.1686.}}}\label{K_L03957-7} zog.\pend
           
\pstart
           Im Ganzen halte ich die Aufführung eines abendfüllenden Stückes oder eines
               Einakterzyklus an irgend einem andern guten Theater für erstrebenswerter als die
               Aufführung eines Einakter an der Comédie
                  française\orgindex{Comédie-Française@Comédie-Française|pw}. Von Balzagette\pwindex{Bazalgette, Léon 8.\,5.\,1873 Paris – 31.\,12.\,1928 ebd.@\textsc{Bazalgette, Léon} (8.\,5.\,1873 Paris – 31.\,12.\,1928 ebd.), \emph{Schriftsteller, Übersetzer}|pw} habe ich
                  \label{K_L03957-8v}\edtext{einen Brief}{\lemma{\textnormal{\emph{einen Brief}}}\Cendnote{\textnormal{Der entsprechende Brief von Léon Bazalgette\pwindex{Bazalgette, Léon 8.\,5.\,1873 Paris – 31.\,12.\,1928 ebd.@\textsc{Bazalgette, Léon} (8.\,5.\,1873 Paris – 31.\,12.\,1928 ebd.), \emph{Schriftsteller, Übersetzer}|pwk}, der für den Verlag \emph{Rieder e Cie}\orgindex{F. Rieder et Cie@F. Rieder et Cie|pwk} die Publikation der \emph{Übersetzung}\pwindex{Schnitzler, Arthur 15. 5. 1862 Wien – 21. 10. 1931 ebd.@\textsc{Schnitzler, Arthur} (15. 5. 1862 Wien – 21. 10. 1931 ebd.), \emph{Schriftsteller, Mediziner}!Mourir. Roman [1925]@\strich\emph{Mourir. Roman [1925]}|pwk} von \emph{Sterben}\pwindex{Schnitzler, Arthur 15. 5. 1862 Wien – 21. 10. 1931 ebd.@\textsc{Schnitzler, Arthur} (15. 5. 1862 Wien – 21. 10. 1931 ebd.), \emph{Schriftsteller, Mediziner}!Sterben. Novelle@\strich\emph{Sterben. Novelle}|pwk} betreute,
                  ist nicht überliefert, aber die Antwort darauf: Arthur Schnitzler an Léon
                     Bazalgette\pwindex{Bazalgette, Léon 8.\,5.\,1873 Paris – 31.\,12.\,1928 ebd.@\textsc{Bazalgette, Léon} (8.\,5.\,1873 Paris – 31.\,12.\,1928 ebd.), \emph{Schriftsteller, Übersetzer}|pwk}, 2. 4. 1925, \emph{Deutsches Literaturarchiv Marbach},
                     HS.1985.1.303,4.}}}\label{K_L03957-8}: Er liest eben die Korrekturbogen von »Sterben\pwindex{Schnitzler, Arthur 15. 5. 1862 Wien – 21. 10. 1931 ebd.@\textsc{Schnitzler, Arthur} (15. 5. 1862 Wien – 21. 10. 1931 ebd.), \emph{Schriftsteller, Mediziner}!Mourir. Roman [1925]@\strich\emph{Mourir. Roman [1925]}|pwv}\pwindex{Schnitzler, Arthur 15. 5. 1862 Wien – 21. 10. 1931 ebd.@\textsc{Schnitzler, Arthur} (15. 5. 1862 Wien – 21. 10. 1931 ebd.), \emph{Schriftsteller, Mediziner}!Sterben. Novelle@\strich\emph{Sterben. Novelle}|pw}«. Von Grasset\pwindex{Grasset, Bernard 6.\,3.\,1881 Chambéry – 20.\,10.\,1955 Paris@\textsc{Grasset, Bernard} (6.\,3.\,1881 Chambéry – 20.\,10.\,1955 Paris), \emph{Verleger}|pw} keinerlei Nachricht; auch
               von Nathan\pwindex{Nathan, Nicolas @\textsc{Nathan, Nicolas}, \emph{Übersetzer}|pw} habe ich nichts weiter gehört\strikeout{.} seit »Casanovas
                  Heimfahrt\pwindex{Schnitzler, Arthur 15. 5. 1862 Wien – 21. 10. 1931 ebd.@\textsc{Schnitzler, Arthur} (15. 5. 1862 Wien – 21. 10. 1931 ebd.), \emph{Schriftsteller, Mediziner}!Casanovas Heimfahrt@\strich\emph{Casanovas Heimfahrt}|pw}«.\pend
           
\pstart
           \label{K_L03957-9v}\edtext{Wann kommen Sie zurück}{\lemma{\textnormal{\emph{Wann kommen Sie zurück}}}\Cendnote{\textnormal{Das erste richtige Treffen nach Zuckerkandls\pwindex{Zuckerkandl, Berta 13.\,4.\,1864 Wien – 16.\,10.\,1945 Paris@\textsc{Zuckerkandl, Berta} (13.\,4.\,1864 Wien – 16.\,10.\,1945 Paris), \emph{Schriftstellerin, Journalistin, Übersetzerin}|pwk} Heimkehr fand am 29. 5. 1925
                  statt, man sah sich aber bereits bei der Generalprobe\eventindex{Burgtheater@\textbf{Burgtheater}!Generalprobe von Der Schleier der Beatrice, 22.5.1925@Generalprobe von Der Schleier der Beatrice, 22.5.1925|pwkv} von \emph{Der Schleier der Beatrice}\pwindex{Schnitzler, Arthur 15. 5. 1862 Wien – 21. 10. 1931 ebd.@\textsc{Schnitzler, Arthur} (15. 5. 1862 Wien – 21. 10. 1931 ebd.), \emph{Schriftsteller, Mediziner}!Schleier der Beatrice. Schauspiel in fünf Akten@\strich\emph{Der Schleier der Beatrice. Schauspiel in fünf Akten}|pwk} am 22. 5. 1925, vgl. XXXX Auszeichnungsfehler: Dokument L04006 nicht gefunden.}}}\label{K_L03957-9}, verehrte Freundin? Seien Sie sehr herzlich gegrüsst{\\[\baselineskip]} und
               immer wieder vielmals für Ihre Bemühungen bedankt.{\\[\baselineskip]} Wie immer{\\[\baselineskip]} der
               Ihrige\pend
           \leftskip=0em{}{\vspace{1\baselineskip}}
\pstart
           \noindent{}{\pb}Eben kommt \label{K_L03957-10v}\edtext{der beigeschlossene Brief}{\lemma{\textnormal{\emph{der … Brief}}}\Cendnote{\textnormal{nicht überliefert}}}\label{K_L03957-10} aus Paris\oindex{Paris@\textbf{Paris}, \emph{Hauptstadt}|pw}. Ich muss bemerken, dass ich Mme Maury\pwindex{Maury, Geneviève 23.\,5.\,1886 Vevey – 21.\,8.\,1956 Paris@\textsc{Maury, Geneviève} (23.\,5.\,1886 Vevey – 21.\,8.\,1956 Paris), \emph{Übersetzerin, Bibliothekarin, Schriftstellerin}|pw} keinerlei keine bestimmte Autorisation erteilt
                  hatte, \label{K_L03957-11v}\edtext{ich erwähnte nur}{\lemma{\textnormal{\emph{ich erwähnte nur}}}\Cendnote{\textnormal{Arthur Schnitzler an Geneviève Maury\pwindex{Maury, Geneviève 23.\,5.\,1886 Vevey – 21.\,8.\,1956 Paris@\textsc{Maury, Geneviève} (23.\,5.\,1886 Vevey – 21.\,8.\,1956 Paris), \emph{Übersetzerin, Bibliothekarin, Schriftstellerin}|pwk}, 26. 3. 1925, \emph{Deutsches Literaturarchiv Marbach},
                        HS.1985.1.1390. }}}\label{K_L03957-11}, dass ich prinzipiell gegen eine Uebersetzung
                  einzelner Novellen nichts einzuwenden hätte, aber vor allem um einen
                  Honorarvorschlag ersuchte. Wenn Sie es für richtig halten, verehrte Freundin, so
                  setzen sie sich vielleicht mit Mme Maury\pwindex{Maury, Geneviève 23.\,5.\,1886 Vevey – 21.\,8.\,1956 Paris@\textsc{Maury, Geneviève} (23.\,5.\,1886 Vevey – 21.\,8.\,1956 Paris), \emph{Übersetzerin, Bibliothekarin, Schriftstellerin}|pw}
                  schon {[}in{]} Rücksicht auf die in ihrem Brief erwähnte
                  Mademoiselle Bianquis\pwindex{Bianquis, Geneviève 19.\,9.\,1886 Rouen – 24.\,3.\,1972 Antony@\textsc{Bianquis, Geneviève} (19.\,9.\,1886 Rouen – 24.\,3.\,1972 Antony), \emph{Übersetzerin, Literaturhistorikerin}|pw} in Verbindung.
                  Andernfalls senden sie mir gütigst den Brief zurück und ich antworte ihr
                  persönlich, glaube aber den Honorarvorschlag nicht annehmen zu sollen.\pend
           
\pstart
           Ihr\pend
           {\vspace{1\baselineskip}}
\pstart
           \label{K_L03957-12v}\edtext{1 Beilage}{\lemma{\textnormal{\emph{1 Beilage}}}\Cendnote{\textnormal{Brief von Geneviève
                        Maury\pwindex{Maury, Geneviève 23.\,5.\,1886 Vevey – 21.\,8.\,1956 Paris@\textsc{Maury, Geneviève} (23.\,5.\,1886 Vevey – 21.\,8.\,1956 Paris), \emph{Übersetzerin, Bibliothekarin, Schriftstellerin}|pwk}, nicht überliefert.}}}\label{K_L03957-12}.\pend
           \selectlanguage{ngerman}\endnumbering\briefempfaengerindex{Zuckerkandl, Berta@\textsc{Zuckerkandl, Berta}!zzzSchnitzler, Arthur@\emph{von Arthur Schnitzler}!1925-04-021@{2. 4. 1925}|)be}\mylabel{L03957h}
\begin{anhang}
\end{anhang}\newcommand{\dateiname}{L03957}\newcommand{\titel}{Arthur Schnitzler an Berta Zuckerkandl, 2. 4. 1925}\newcommand{\editorInnen}{Herausgegeben von Jahnke, SelmaMüller, Martin Anton}%% latex-leseansicht-abspann.tex
%% Abspann für die Leseansicht.
%% Der Schalter \ifkorrekturansicht ist bereits durch den Vorspann gesetzt.

%% latex-abspann.tex
%% Gemeinsamer Abspann für Korrekturansicht und Leseansicht.
%% Setzt den Schalter \ifkorrekturansicht voraus (gesetzt in den
%% einbindenden Dateien latex-korrekturansicht-abspann.tex bzw.
%% latex-leseansicht-abspann.tex).
%% ---------------------------------------------------------------

\normalsize

% Das esempio-Environment wird nur in der Leseansicht benötigt
\ifkorrekturansicht\else
\newenvironment{esempio}[3]%
{
    \vspace{1.5ex}
    \rlap{\underline{#1}}
    \par
    \setlength{\parindent}{0cm}
    \nopagebreak
    \leftskip=#2cm
    \rightskip=#3cm
}
{
    \par
}
\fi

\doendnotes{C}
\bigskip
\vfill

\clearpage

\footnotesize

\ifkorrekturansicht
  \lohead{\textsc{register}}
\fi

% theindex-Environment neu definieren ohne reledmac
\makeatletter
\renewenvironment{theindex}{%
  \ifkorrekturansicht
    \section*{\indexname}%
  \else
    \subsubsection*{Index der erwähnten Entitäten}%
  \fi
  \setlength{\parindent}{0pt}%
  \setlength{\parskip}{0pt plus 0.3pt}%
  \let\item\@idxitem
}{%
  \ifkorrekturansicht\clearpage\fi
}
\makeatother

\IfFileExists{\jobname-pw.ind}{\input{\jobname-pw.ind}}{}

% Quellenangabe nur in der Leseansicht
\ifkorrekturansicht\else
% Fallback-Definitionen, falls die .tex-Datei \titel etc. nicht gesetzt hat
\providecommand{\titel}{}
\providecommand{\editorInnen}{}
\providecommand{\dateiname}{\jobname}

\vspace{3cm}

\vfill

\footnotesize
\textsc{Quelle}: \titel. Herausgegeben von {\editorInnen}. In: \emph{Arthur Schnitzler: Briefwechsel mit Autorinnen und Autoren}.
 Digitale Edition, https://schnitzler-briefe.acdh.oeaw.ac.at/{\dateiname}.html (Stand \today)
\fi

\end{document}


