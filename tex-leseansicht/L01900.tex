%% latex-korrekturansicht-vorspann.tex
%% Vorspann für die Korrekturansicht.
%% Lädt die gemeinsame Datei latex-vorspann.tex mit gesetztem Schalter.

\newif\ifkorrekturansicht
\korrekturansichttrue

\input{../tex-inputs/latex-vorspann}


\section[Richard Beer-Hofmann an Arthur Schnitzler, {[}13. 12. 1909?{]}]{L01900 Richard Beer-Hofmann an Arthur Schnitzler, {[}13. 12. 1909?{]}}
\nopagebreak\mylabel{L01900v}
\rehead{ }\normalsize\beginnumbering\briefempfaengerindex{Schnitzler, Arthur@\textsc{Schnitzler, Arthur}!zzzBeer-Hofmann, Richard@\emph{von Richard Beer-Hofmann}!1909-12-131@{{[}13. 12. 1909?{]}}|(be}
\toendnotes[C]{\smallbreak\pagebreak[2]}\Standort{CUL, Schnitzler, B 8.}
\physDesc{Manuskript, 13 Blätter, 13 Seiten, 16991 Zeichen
\newline{}Schreibmaschine\noindent{}Text mit Paginierung
\newline{}Handschrift: Bleistift, deutsche Kurrent (\noindent{}Korrekturen)}
\buchAbdrucke{\weitereDrucke{Arthur Schnitzler, Richard Beer-Hofmann: \emph{Briefwechsel 1891–1931}. Wien, Zürich: \emph{Europaverlag} 1992, S. 198–206.} }\toendnotes[C]{\smallbreak}
\pstart
           \noindent{}\centering{}{\pb}\uline{DAS ECHO DES LEBENS.}\pwindex{Echo des Lebens@\emph{Das Echo des Lebens}|pw}\pend
           
\pstart
           \centering{}Ein Epilog zur \label{K_L01900-1v}\edtext{Generalprobe}{\lemma{\textnormal{\emph{Generalprobe}}}\Cendnote{\textnormal{Der Text ist
                  undatiert, wurde Schnitzler aber am 13. 12. 1909 von Beer-Hofmann\pwindex{Beer-Hofmann, Richard 1866-07-11 – 1945-09-26@\textsc{Beer-Hofmann, Richard} (1866-07-11 – 1945-09-26), \emph{Schriftsteller/Schriftstellerin}|pwk} vorgelesen. Damit ist
                  anzunehmen, dass er ihn zu diesem Zeitpunkt erhalten hat.}}}\label{K_L01900-1} des
                  Stückes{\\}»Der Ruf des Lebens\pwindex{Ruf des Lebens. Schauspiel in drei Akten@\emph{Der Ruf des Lebens. Schauspiel in drei Akten}|pw}«.\pend
           {\vspace{1\baselineskip}}
\pstart
           Am Tage der Aufführung. Vier Uhr Nachmittags. Da es der 11. Dezember
               ist, dämmert es bereits merklich. Der Vorhang ist hochgezogen. Die Bühne trägt die
               Dekoration des 2. Aktes\pwindex{Ruf des Lebens. Schauspiel in drei Akten@\emph{Der Ruf des Lebens. Schauspiel in drei Akten}|pwv}. Die
               Figuren des Stückes, die noch vor kurzem, in der unmateriellen Wirklichkeit, die
               ihnen die Worte ihres Schöpfers gaben, bewegt aufrecht standen – lehnen nun, in der
               materiellen Unwirklichkeit, die ihnen gestern – bei der Generalprobe – die
               Schauspieler gaben, etwas blass und müde an den Wänden umher. Nur »der alte Moser«
               liegt von rechts nach links, die ganze Bühne überquerend, wie ein Schlagbaum am
               Boden. Auf dem Fensterbrett scheint der Oberleib des Obersten zu stehen. Man kann
               augenblicklich nicht erkennen, ob er einen Unterleib besitzt. Irene, die man befragen
               könnte, liegt neben dem alten Moser auf dem Boden. Falls Max sie fragen sollte, wird
               sie es verneinen.\pend
           
\pstart
           Vorne, hart am Souffleurkasten, ist eine dünne, frisch gestrichene grüne Barriere
               aufgestellt, wohl, um zu verhindern, dass die Person\introOben{}en\introOben{} des
               Stückes dem Publikum zu nahe gehen. Der Souffleurkasten scheint besetzt – nach der
               Unruhe, die in ihm herrscht; (als sässe jemand darinn, dem er zu eng
               ist).\pend
           
\pstart
           Eine Pause.\pend
           
\pstart
           Dann, eine ungeduldige Stimme aus dem Souffleurkasten:\pend
           
\pstart
           {\pb}So fangen Sie doch an!\pend
           
\pstart
           Marie: (mit etwas starren Augen, leise, und ein wenig verlegen) Verzeihen Sie, Herr
               – – – ich weiss gar nicht, wie ich Sie nennen soll – –\pend
           
\pstart
           \substVorne{}\textsuperscript{Stimme}\substDazwischen{}Souffleur\substHinten{}: Souffleur! Nennen Sie mich nur so. Für Sie bin ich es augenblicklich – was
               ich sonst bin, kommt hier nicht in Betracht. Fangen Sie doch an!\pend
           
\pstart
           Marie: Verzeihen Sie, Herr Souffleur – aber – – ich bin vielleicht nicht ganz
               berechtigt, Sie das zu fragen – aber wieso sind wir da?\pend
           
\pstart
           Katharina: Ja! Wieso sind wir da?\pend
           
\pstart
           Der Oberst: Sie fragen nach den letzten Dingen – liebe Marie! Nach unserem
               Dasein.\pend
           
\pstart
           Max: (zu Albrecht leise) Der Oberst ist ein gar zu witziger Kopf!\pend
           
\pstart
           Der Oberst: Die \strikeout{ewig} Fragen nach den letzten Dingen,
               für den letzten Akt, liebe Marie! Vorher, ist jede Tiefe, eine Grube, die sich der
               Dichter gräbt.\pend
           
\pstart
           \substVorne{}\textsuperscript{Stimme}\substDazwischen{}Souffleur\substHinten{}: (ärgerlich) Dann graben Sie doch nicht, Herr Oberst!\pend
           
\pstart
           \label{T_L01900-1v}\edtext{Marie: Aber}{\lemma{\textnormal{\emph{Marie: Aber}}}\Cendnote{\textnormal{geändert aus: »Marie; Aber«}}}\label{T_L01900-1} ich habe ja
               nur ganz unschuldig gefragt – – –\pend
           
\pstart
           Die Oberstin (hebt den Kopf, sehr hart) \introOben{}»\introOben{}Unschuldig\introOben{}«\introOben{}? Sie? (Sie lacht auf, und lässt den Kopf wieder
               sinken.)\pend
           
\pstart
           Katharina: Auch ich habe nur leichthin – – –\pend
           
\pstart
           Die Oberstin: (wie vorhin) »Leichthin«? \uline{Das} passt für
               Sie. »Leichthin«.\pend
           
\pstart
           \substVorne{}\textsuperscript{Stimme}\substDazwischen{}Souffleur\substHinten{}: (ärgerlich zur Oberstin) Fangen Sie nicht wieder an – –!\pend
           
\pstart
           Unteroffizier Sebastian (sehr militärisch, aber mit eingefetteter Stimme): Melde
               gehorsamst, wir sollen doch anfangen!\pend
           
\pstart
           \substVorne{}\textsuperscript{Stimme}\substDazwischen{}Souffleur\substHinten{}: Reden Sie nichts drein – Sie – wie heissen \substVorne{}\textsuperscript{s}\substDazwischen{}S\substHinten{}ie, Unteroffizier, ich habe Ihren Namen vergessen.\pend
           
\pstart
           {\pb}Sebastian: Melde gehorsamst, Herr
               Soffleur, Sebastian!\pend
           
\pstart
           \substVorne{}\textsuperscript{Stimme}\substDazwischen{}Souffleur\substHinten{}: Anfangen!\pend
           
\pstart
           Der Arzt: Herr Souffleur, ich kann Fräulein Marie nicht Unrecht geben – – –\pend
           
\pstart
           Oberst: \uline{Das} haben wir gemerkt!\pend
           
\pstart
           Der Arzt: – – es ist doch für uns alle – sozusagen – eine Lebensfrage, zu wissen,
               wieso wir da sind?\pend
           
\pstart
           Der Adjunkt / Max (gleichzeitig) Fräulein Marie hat Recht.\pend
           
\pstart
           \substVorne{}\textsuperscript{Stimme}\substDazwischen{}Souffleur\substHinten{}: Gut, ich will versuchen – – –\pend
           
\pstart
           Der Oberst: Der Dichter ist unser Schöpfer! Sie, mein Herr, sind der Versucher!\pend
           
\pstart
           \substVorne{}\textsuperscript{Stimme}\substDazwischen{}Souffleur\substHinten{}: Graben Sie nicht, Herr Oberst! Ich will versuchen es Ihnen zu sagen. Geben
               Sie Acht!\pend
           
\pstart
           Der alte Moser (hebt den Kopf): Acht? Nein, neunundsiebzig Jahre bin ich alt – –\pend
           
\pstart
           Der Arzt: Sie irren sich, Herr Moser, Sie sind jetzt gar nicht mehr alt; Sie sind
               ganz jung tot.\pend
           
\pstart
           Der alte Moser: Ich will nicht tot sein.\pend
           
\pstart
           Der Arzt: Herr Moser! Ich bin Ihr Arzt – Sie müssen tot sein! Sie sind dazu
               verpflichtet. Nicht nur sich selbst gegenüber – – –\pend
           
\pstart
           \introOben{}\strikeout{Stimme}: \introOben{}Souffleur: Passen Sie auf, Sie
               werden mich nicht verstehen! Wissen Sie, wie lange die kleinen Teilchen im
               Resonnanzboden einer Violine nicht zur Ruhe kommen und noch
               fortschwingen, wenn für unser Ohr der Bogenstrich, der sie erschütterte, längst
               verklungen ist?\pend
           
\pstart
           Albrecht: Wir reiten morgen in den Tod, Herr Souffleur – und Sie – prüfen uns
               Physik?\pend
           
\pstart
           {\pb}Souffleur: Es war nur eine
               rhetorische Frage – – –\pend
           
\pstart
           (Der Hintergrund wird durchsichtig; im hellen Sonnenlicht erblickt man ein Dorf,
               lieblich an Hängen gelagert. Eine tiefsinnige Stimme sagt:\pend
           
\pstart
           Rhetorisch! Ja!\pend
           
\pstart
           Eine Frauenstimme (wiederholt in kurzen Atemstössen, ekstatisch): Rhetorisch! Ja, das
               ist er!\pend
           
\pstart
           (Die Landschaft entschwindet.)\pend
           
\pstart
           Arzt: War das Grünau, Herr Adjunkt?\pend
           
\pstart
           Adjunkt: Nein, lieber Doktor: Grinzing!\pend
           
\pstart
           Souffleur: Nun sehen Sie: Von allen Worten, die Sie gestern auf der Generalprobe
               sprachen, schwingt noch die Luft; sie hallen noch von den Mauern und den gemalten
               Leinenwänden wieder, und wer feine Ohren hat, kann sie hören.\pend
           
\pstart
           Arzt: Aber Herr Souffleur, das würde ja nur erklären, wieso unsere Stimmen da sind;
               aber woher nehmen Sie denn unsere Leiber?\pend
           
\pstart
           Souffleur: Ich könnte sagen: »Von die zwei Gulden!« Aber Sie würden mich nicht
               verstehen, und es würde, überdies, vielleicht mein Inkognito lüften.\pend
           
\pstart
           Der alte Moser: (hebt den Kopf) Nicht lüften! Sind Sie toll, ich kann den Tod davon
               haben. (er legt sich wieder hin).\pend
           
\pstart
           Arzt: (energisch) Herr Moser, ich ordiniere Ihnen tot zu sein. Wenn Sie meine
               Verordnungen nicht befolgen, stehe ich für nichts!\pend
           
\pstart
           Souffleur: Ihre Leiber also, Herr Arzt? Nun denn: Wenn Sie einen Gegenstand, zum
               Beispiel das Fensterkreuz dort – – –\pend
           
\pstart
           Max: Hier ist keines, Herr Souffleur! Sonst bleibt der Herr Oberst beim
               Hereinspringen hängen!\pend
           
\pstart
           {\pb}Souffleur: Lassen Sie mich
               ausreden!\pend
           
\pstart
           Oberstin: (höhnisch) \uline{Wir} – \uline{Sie}? Haha!\pend
           
\pstart
           Souffleur: Wenn Sie ein Fensterkreuz fest ins Auge fassen, und dann den Blick von ihm
               abwenden, so werden Sie – wenn Sie genau beobachten – noch einen Augenblick lang, vor
               sich in der Luft, das Bild des Fensterkreuzes sehen. Ich sage: »Das Bild«! Aber
               wissen wir, ob es mehr oder weniger Bild ist, als das Fensterkreuz, das wir vorhin
               sahen – nicht? – – Sagen Sie doch: »Jo«! – Von der Leiblichkeit, die gestern auf der
               Probe Schauspieler den Worten des Dichters liehen, schweben die Formen noch
               durcheinander in diesem Raum, und haben – eine Weile – eine Art von Leiblichkeit.\pend
           
\pstart
           Sebastian: Herr Souffleur, melde gehorsamst, dass doch seither die gestrige
               Abendvorstellung – ganz ausverkauft – hier war!\pend
           
\pstart
           Souffleur: (nach einer Pause) Sie haben recht – aber ich fange an zu zweifeln, ob Sie
               wirklich »Sebastian« heissen!\pend
           
\pstart
           Sebastian: Melde gehorsamst: So soll ich le – – (abbrechend zu Max) Fürchten Sie
               nichts, Herr Leutnant – auch wir haben einander zugeschworen, auch wir sind
               totgeweiht!\pend
           
\pstart
           Oberst: Wie kommt es, Unteroffizier, dass ich gar niemanden vom Regiment
               erblicke?\pend
           
\pstart
           Oberstin: (höhnisch) Weil Du mit dem Rücken gegen den Kasernhof stehst!\pend
           
\pstart
           Oberst: Aber es sollten doch Kürrassiere unseres Regimentes vorbeigehen und grüssen,
               und ich sollte »Gute Nacht« wünschen – – – wo sind denn Alle – – Unteroffizier!\pend
           
\pstart
           Sebastian: Melde gehorsamst: Das ganze Regiment ist nach Hause gefahren um Abschied
               zu nehmen, und wir haben einander {\pb}zugeschworen, dass keiner zurückkommt. Wir sind Alle blaue Kürrassiere.\pend
           
\pstart
           Katharina: (gerührt zu Sebastian) Geben Sie mir Ihre Hand, Sebastian\pend
           
\pstart
           Sebastian: (gibt die Hand nicht)\pend
           
\pstart
           Katharina: Abschied nehmen ist süss!\pend
           
\pstart
           Sebastian: Melde gehorsamst, es kann auch eine Woche dauern!\pend
           
\pstart
           Katharina: Ich dachte, Du seist ein lustiger Bursche?\pend
           
\pstart
           Sebastian: Zu Befehl! Auch lustig bin ich, aber dann sag ich nicht »Abschied
               nehmen«.\pend
           
\pstart
           Marie: (kommt nach vorne, bückt sich zum Souffleurkasten) Ich danke Ihnen, Herr
               Souffleur – jetzt versteh ich mein Dasein! (sie wendet sich und tritt der Oberstin
               auf ihr Kleid)\pend
           
\pstart
           Oberstin: (wütend) Jetzt treten Sie mir noch die Schleppe ab! \uline{Das} ist zu viel! Sie – Sie – Hyäne des Schlachtfeldes!\pend
           
\pstart
           Maria: (sanft) Was habe ich Ihnen getan, Frau Oberst?\pend
           
\pstart
           Oberstin: Sie fragen noch? Mein Mann erschiesst mich, und an meiner Leiche hängen Sie
               sich an den Hals meines Geliebten?\pend
           
\pstart
           Albrecht: (zu Max) Das sind \so{Deine} Zusammenhänge, Max!\pend
           
\pstart
           Marie: Und \so{mein} Schicksal vergessen Sie? Ich habe mich ihm
               hingegeben, und aus meinen Armen ist er gegangen, sich umbringen für eine Andere?
                  \label{K_L01900-2v}\edtext{Was bin \strikeout{ich} denn dann ich ihm gewesen?}{\lemma{\textnormal{\emph{Was … gewesen?}}}\Cendnote{\textnormal{Anspielung auf die Rede von Christine am Ende von \emph{Liebelei}\pwindex{Liebelei. Schauspiel in drei Akten@\emph{Liebelei. Schauspiel in drei Akten}|pwk}: »Und ich {\dots} was bin denn ich?
                     was bin denn ich ihm gewesen{\dots}«.}}}\label{K_L01900-2}\pend
           
\pstart
           Oberst: (hebt den Kopf wie ein altes Schlachtross, das bekannte Signale hört) Was
               sind \uline{das} für Töne?! Herr Leutnant!\pend
           
\pstart
           Max: Zu Befehl, Herr Oberst!\pend
           
\pstart
           Oberst: Ich komme um eine Kleinigkeit zu holen.\pend
           
\pstart
           Max: Herr Oberst?\pend
           
\pstart
           {\pb}Oberst: Meine Frau hat sich
               gestern bei Ihnen vergessen!\pend
           
\pstart
           Max: Herr Oberst scherzen.\pend
           
\pstart
           Oberst: (stark) Sie 
               \uline{h a t}
                sich vergessen.\pend
           
\pstart
           Sebastian: Melde gehorsamst, Herr Oberst, da liegt sie \uline{noch}. (er will sie aufheben.)\pend
           
\pstart
           Oberst: Lassen Sie! (zu Max) Ich will \strikeout{nicht}, dass man
               sie später bei Ihnen finde!\pend
           
\pstart
           Marie: (will auf Max zueilen) Max!\pend
           
\pstart
           Der alte Moser: (hält sie am Fuss fest) Geh nicht weg von mir, Marie! Ich hab dich
               gequält; ich bin ein alter kranker Mann von neunundsiebzig Jahren! Vergib mir!\pend
           
\pstart
           Marie: (sanft) Ich vergebe Dir!\pend
           
\pstart
           Oberstin: (wild) Sie vergibt \so{Ihnen}, Sie vergibt \so{Sie}, sie vergibt in allen Fällen! Ein sanftes Mädchen, Ihre
               Tochter! Und Sie, Herr Rittmeister, sollten sich auch schämen! Hören Sie auf mich zu
               zwicken! Sonst steh ich auf!\pend
           
\pstart
           Oberst: Irene!? Welcher Rittmeister zwickt dich?\pend
           
\pstart
           Max: Herr Oberst, wir sind beide vom Regiment der Geweihten.\pend
           
\pstart
           Oberstin: Der Herr Moser!\pend
           
\pstart
           Oberst: Woher \label{T_L01900-2v}\edtext{weisst}{\lemma{\textnormal{\emph{weisst}}}\Cendnote{\textnormal{korrigiert aus: »wiesst« der
                  Vorlage.}}}\label{T_L01900-2} Du Irene, dass er Rittmeister ist?\pend
           
\pstart
           Sebastian: Melde gehorsamst: Am Zwick, Herr Oberst!\pend
           
\pstart
           Katharina: Stecken Sie mir die Locken auf, Sebastian.!\pend
           
\pstart
           Sebastian: (fängt an, sie zu frisieren.)\pend
           
\pstart
           Oberst: Herr Rittmeister, Sie werden diesen Mord (auf Irene weisend) auf sich nehmen
               und sich morgen Früh standrechtlich erschiessen lassen. (bitter höhnend) Es wird
               Ihnen nicht schwer fallen, Sie sind ja das Sterben gewöhnt!\pend
           
\pstart
           Der alte Moser: Ich bin ein alter Mann von neunundsiebzig Jahren – –\pend
           
\pstart
           {\pb}Oberst: Und erst Rittmeister?\pend
           
\pstart
           Katharina: Noch diese Locke, Sebastian!\pend
           
\pstart
           Oberst: Wo haben Sie gedient, Herr Rittmeister?\pend
           
\pstart
           Der alte Moser: Wir sind alle blaue Kürrassiere!\pend
           
\pstart
           Oberst: Oh, über die verschlungenen Schicksalswege! So sind Sie der Rittmeister
               Moser, den ich erfunden habe?\pend
           
\pstart
           Katharina: Noch eine Locke hier, Sebastian!\pend
           
\pstart
           Sebastian: (eine Locke aufsteckend) \so{Noch} ein Dreh!\pend
           
\pstart
           Oberst: Da wären Sie ja an allem Schuld, was uns heute trifft!\pend
           
\pstart
           \introOben{}\strikeout{Rittmeister}\introOben{}Der alte Moser: Wenn Sie mich aber erfunden haben!\pend
           
\pstart
           Oberst: (bitter) Das hat Sie nicht gehindert wirklich zu sein.\pend
           
\pstart
           Arzt: Jetzt erkennen wir es, Herr Moser! Sie sind an allem schuld! Und an Ihrem
               eigenen Tod trifft Ihre arme Tochter kein Verschulden. Sie selbst haben sich
               umgebracht. Wären Sie damals, anstatt feige davonzulaufen, den Heldentod gestorben –
               Sie hätten nie geheiratet, hätten nie eine Tochter (die Sie notgedrungen vergiften
               musste) gehabt – und wären heute, – mit Ausnahme kleiner Altersbeschwerden frisch und
               gesund!\pend
           
\pstart
           Sebastian: (eine neue Locke aufsteckend, sehr begeistert) \uline{Noch} ein Dreh!\pend
           
\pstart
           Oberstin: Um Ihretwillen, Sie alter Feigling, muss mein Ma\substVorne{}\textsuperscript{nn}\substDazwischen{}x\substHinten{} sterben!\pend
           
\pstart
           Oberst: Um Ihretwillen, Herr Rittmeister, ist ein ganzes Regiment totgeweiht!\pend
           
\pstart
           Der Arzt: Sie sind an allem schuld, Herr Moser!\pend
           
\pstart
           Der alte Moser: Jetzt ist’s zu viel und vor allem, Herr Doktor, sagen Sie nicht Herr
               Moser, sondern Rittmeister zu mir. \so{Mein} Davonlaufen ist an
               allem schuld? Ja, {\pb}dann ist mein
               Vater daran schuld, weil er meine Mutter geheiratet hat, und so fort, bis auf Adam
               und Eva! Muss ich alter Mann von neunundsiebzig Jahren Ihnen sagen, dass man sich
               nicht auf Kausalitäten einlassen soll, weil sonst eine Konfusion herauskommt!\pend
           
\pstart
           Souffleur: Es \so{giebt} keine Kausalitäten!\pend
           
\pstart
           Der alte Moser: Lassen Sie mich ausreden!\pend
           
\pstart
           Souffleur: Sie \substVorne{}\textsuperscript{werden}\substDazwischen{}wären\substHinten{} der Erste, dem \uline{ich} das gestattet hätte! Es
               gibt keine Ursachen, und keine Wirkungen! Eine Wirkung ist eine Ursache, die noch
               lebt, sonst könnte sie nicht mehr wirken!\pend
           
\pstart
           Sebastian: (selig fri\strikeout{e}sierend): \so{Noch} ein Dreh!\pend
           
\pstart
           Der alte Moser: Das versteh’ ich nicht! Aber, zum Teufel, merken Sie denn nicht, dass
               Sie Alle meinem Davonlaufen Ihr Leben verdanken? Glauben Sie, mein Heldentod hatte
               den Dichter interessiert?\pend
           
\pstart
           Sebastian: (jubilierend) \so{Noch} ein Dreh!\pend
           
\pstart
           Oberst: Ihr Davonlaufen, ist doch überhaupt eine witzige Erfindung von \so{mir}! \so{Ich} kann darauf stolz sein,
               dass Sie davongelaufen sind, aber doch nicht \so{Sie}.\pend
           
\pstart
           Sebastian: (dem Wahnsinn nah) \so{Noch} ein Dreh!\pend
           
\pstart
           Arzt: Nun, Herr Oberst, wenn Sie sich aber gar so viel einbilden auf das Elend, das
               Sie mit Ihrer witzigen Erfindung angerichtet haben, so muss ich Ihnen schon sagen:
                  \so{wir} haben keinen Grund Ihnen dankbar zu sein.
               Ertrinkende sind keine Menschen für ein Drama. Und in dem Stück sind fast alle am
               Ersaufen. Katharina und der Herr Moser und die beiden Herrn Leutnants und Sie auch,
               Herr Oberst. Für das, was einer tut, der weiss, dass er morgen {\pb}sterben soll, kann man ihn nicht
               mehr zur Rechenschaft ziehen; das ist kein Lebender mehr. Der hat nicht mehr freien
               Willen, den kommandiert nur seine Todesangst, und wo es nicht freien Willen gibt –
               gibt es kein Drama!\pend
           
\pstart
           Oberst: Schade, dass Ihre Weisheit so kurzen Atem hat, wie der Herr Moser. Es \label{T_L01900-3v}\edtext{\so{gab}}{\lemma{\textnormal{\emph{gab}}}\Cendnote{\textnormal{zusätzlich noch handschriftlich
                  unterstrichen}}}\label{T_L01900-3} keines – aber es \label{T_L01900-4v}\edtext{\so{gibt}}{\lemma{\textnormal{\emph{gibt}}}\Cendnote{\textnormal{zusätzlich noch handschriftlich
                  unterstrichen}}}\label{T_L01900-4} eines: Sie selbst spielen ja darin!\pend
           
\pstart
           Sebastian: (entrüstet) \so{Den} Dreh mach’ ich nicht mit!\pend
           
\pstart
           Arzt: So werde ich Ihnen nach der Vorstellung sagen – – –\pend
           
\pstart
           Oberst: Das können Sie nicht! Sie sind nur \so{in} der
               Vorstellung!\pend
           
\pstart
           Sebastian: (schreit) Aufhören!\pend
           
\pstart
           Arzt: So werde ich Ihnen sagen, dass Tod und Leben Voraussetzungen sind – nicht
               Stoffe. Tod und Leben sind unverantwortlich und stehen niemandem Rede; und dass einer
               Dichter ist, heisst nur, dass er manchmal – nicht zu oft – nach ihnen fragen darf.
               Fragen! Ohne Antwort zu bekommen! Und schön muss er fragen – sehr schön!\pend
           
\pstart
           Oberst: Sie wollen mich wohl belehren, Herr Doktor? Schweigen Sie endlich!\pend
           
\pstart
           Arzt: Herr Oberst haben Ihrem Regiment zu befehlen – nicht mir!\pend
           
\pstart
           Oberst: Sie irren, mein Lieber! Auch Ihnen! Sie sind – wie ich selbst – von \uline{meinen} Gnaden! (er springt ins Zimmer, die Maske fällt
               ab – der Dichter steht da).\pend
           
\pstart
           Sebastian: (wimmernd in die Knie sinkend) \so{Kein} Dreh mehr!
               Keiner mehr! Wenn \so{ich} schon nicht mehr mit kann!\pend
           
\pstart
           Dichter: (nach vorne kommend, zündet sich eine Virginia an, und sagt zum Souffleur,
               in den Kasten hinunter, scharf): {\pb}Die letzte Rede des Arztes haben doch \so{Sie}
               souffliert!\pend
           
\pstart
           Souffleur: Ja mein Lieber – ebenso wie Ihre Reden!\pend
           
\pstart
           Sebastian: (wimmert auf, der Arzt unterstützt ihn und fühlt ihm den Puls).\pend
           
\pstart
           Dichter: Wieso? Ja so! Natürlich! Aber hören Sie auf, Sie sehen doch, der Mann stirbt
               bereits an Ihren Drehs!\pend
           
\pstart
           Souffleur: An meinen? Doch auch an Ihren!\pend
           
\pstart
           Dichter: Ja, ja! Aber nach dem Nachtmahl ist mir das zu anstrengend. (Er will sich
               auf die Barriere setzen).\pend
           
\pstart
           Souffleur: (aufschreiend) Nicht auf die Barriere! Das ist keine, das ist der letzte
               Akt: grün und stark gestrichen! Auf den wird sich die Kritik setzen!\pend
           
\pstart
           Dichter: Keine Witze jetzt! Sagen Sie übrigens: Es ist ja sehr ehrenvoll für mich,
               aber – haben Sie wirklich nichts anderes zu tun als sich meine Figuren herzunehmen,
               und mit ihnen Schindluder zu treiben? Fischer\pwindex{Fischer, Samuel 24.12.1859 – 15.10.1934@\textsc{Fischer, Samuel} (24.12.1859 – 15.10.1934), \emph{Verleger/Verlegerin}|pw}
               würde sagen: »Ein ausgeruhter Kopp!« Bei Ihnen kann man sich ja nicht einmal
               revanchieren. Bis zu Ihrer Generalprobe bin ich bestenfalls so alt wie der alte
               Moser! Warum verschwenden Sie so viel Geist – –\pend
           
\pstart
           Arzt: (um Sebastian bemüht) Ein Glas Wasser, bitte.\pend
           
\pstart
           Dichter: (fortfahrend) – – da sehen Sie – so viel Geist an fertige Figuren, die ihn
               wirklich nicht brauchen? Geben Sie Ihren davon – ich meine den noch unfertigen! Bei
               einer Pentalogie kann man nie davon genug haben! Uebrigens, mir fällt ein: Sie
               könnten eine Familienfideikomisstiftung aus der Pent\strikeout{h}alogie machen. Immer der älteste Sohn hat daran zu schreiben, und wenn einer
               Ihrer Nachkommen sie wirklich fertig macht, so wird{ }{\pb}er enterbt – weil er aus der Art
               geschlagen ist! Wissen Sie: \so{wenn} Sie schon das Stück
               kritisieren wollen, gestalten Sie nicht an meinen Gestalten herum – sondern schreiben
               Sie einen Essay für die »Rundschau\orgindex{Neue Rundschau, Neue Deutsche Rundschau, Freie Buehne@Neue Rundschau, Neue Deutsche Rundschau, Freie Bühne|pw}« – das wird
               wenigstens klarer und deutlicher sein.\pend
           
\pstart
           Souffleur: Deutlicher vielleicht – aber das war nicht meine Absicht!\pend
           
\pstart
           Dichter: Hetzen Sie doch nicht den Satz zu Tode – er ist sehr gut!\pend
           
\pstart
           Souffleur: Eigentlich, lieber Arthur, ist es recht unfreundlich von Ihnen, mir das
               von der Pentalogie – wenn auch im Scherze – so vor allen den Leuten zu sagen! Sie
               hätten mir das auch unter vier Augen sagen können – das wäre liebenswürdiger
               gewesen.\pend
           
\pstart
           Dichter: Liebenswürdiger vielleicht! Aber das war nicht meine Absicht!\pend
           
\pstart
           Souffleur: Hetzen Sie doch den Satz nicht zu Tode! Er ist sehr gut!\pend
           
\pstart
           Dichter: Uebrigens, das, was Sie den Arzt da sagen lassen, von der Kausalität – ist
               recht couragiert von Ihnen. Sie spielen den Krieg in Feindesland! \so{Sie} – als Verteidiger der Kausalität! Wissen Sie, was Sie
               sind??\pend
           
\pstart
           Souffleur: Meiner Bescheiden\introOben{}heit\introOben{}, lieber Arthur, ist es wohl
               zuzutrauen, dass ich weiss, was ich bin!\pend
           
\pstart
           Dichter: Sie sind: »\label{K_L01900-3v}\edtext{Grachi de
               seditione quaerentes}{\lemma{\textnormal{\emph{Grachi … quaerentes}}}\Cendnote{\textnormal{Umwandlung einer
                  lateinischen Redewendung in den Singular: »Quis tulerit Gracchos de seditione
                  quaerentes« (»Wer ertrüge es, wenn die Gracchen sich über Aufruhr
                  beklagen«).}}}\label{K_L01900-3}«! Bombenwerfer, die über Knallbonbons sich beklagen! \so{Sie} verlangen Kausalität in einem Drama! Ich krieg
               ordentlich eine Wut, wenn ich mir das vorstelle! (er bricht erbittert ein Stück von
               seiner Virginia, die nicht brennt, ab) Ausgerechnet \so{Sie}
               machen mir Vorwürfe! {\pb}\so{Sie}, der Sie – – Sie – (wütend auflachend) \so{Sie}, \so{Sie}: »Es geschah« Sie!\pend
           
\pstart
           Sebastian: (interessiert aufhorchend) Eschkenasi?? Von welchem Eschkenasi sind
               Sie – –\pend
           
\pstart
           Souffleur: (milde) Bestehen Sie noch immer darauf, dass Sie \strikeout{»Sesbas} »Sebastian« heissen?\pend
           
\pstart
           Sebastian: (hat sich aufgerichtet; respektlos, in herzlicher Gemütlichkeit,
               fraternisierend) Sind Sie nicht bös, Herr Dichter – und Herr Eschkenasi – wir sind
               Alle blaue Kürrassiere!\pend
           
\pstart
           \centering{}\uline{Der Vorhang fällt.}\pend
           \selectlanguage{ngerman}\endnumbering\briefempfaengerindex{Schnitzler, Arthur@\textsc{Schnitzler, Arthur}!zzzBeer-Hofmann, Richard@\emph{von Richard Beer-Hofmann}!1909-12-131@{{[}13. 12. 1909?{]}}|)be}\mylabel{L01900h}  \normalsize

\doendnotes{C}
\bigskip
\vfill

\clearpage

\footnotesize

\lohead{\textsc{register}}

% Definiere theindex-Environment komplett neu ohne reledmac
\makeatletter
\renewenvironment{theindex}{%
  \section*{\indexname}%
  \setlength{\parindent}{0pt}%
  \setlength{\parskip}{0pt plus 0.3pt}%
  \let\item\@idxitem
}{%
  \clearpage
}
\makeatother

\IfFileExists{\jobname-pw.ind}{\input{\jobname-pw.ind}}{}

\end{document}

      