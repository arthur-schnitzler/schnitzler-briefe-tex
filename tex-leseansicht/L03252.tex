%% latex-leseansicht-vorspann.tex
%% Vorspann für die Leseansicht.
%% Lädt die gemeinsame Datei latex-vorspann.tex mit nicht gesetztem Schalter.

\newif\ifkorrekturansicht
\korrekturansichtfalse

\input{../tex-inputs/latex-vorspann}


\section[ Paul Goldmann an Arthur Schnitzler, 27. 9. 1909]{L03252 Paul Goldmann an Arthur Schnitzler,  27. 9. 1909}
\nopagebreak\mylabel{L03252v}
\rehead{ }\normalsize\beginnumbering\briefempfaengerindex{Schnitzler, Arthur@\textsc{Schnitzler, Arthur}!zzzGoldmann, Paul@\emph{von Paul Goldmann}!1909-09-271@{27. 9. 1909}|(be}
\toendnotes[C]{\smallbreak\pagebreak[2]}
\correspDesc{Versand  durch Paul Goldmann am 27. 9. 1909 in Wien
\newline{}Erhalt  durch Arthur Schnitzler im Zeitraum [27. 9. 1909
                  – 1. 10. 1909?] in Wien}\toendnotes[C]{\smallbreak}
\Standort{DLA, A:Schnitzler, HS.NZ85.1.3175.}
\physDesc{Bildpostkarte, 250 Zeichen
\newline{}Handschrift: schwarze Tinte, deutsche Kurrent
\newline{}Versand: Stempel: »\nobreak{}\oindex{Wien@\textbf{Wien}, \emph{Verwaltungsgebiet}|pwk}Wien, {[}2{]}7. IX. 09, –9\nobreak{}«.  }\toendnotes[C]{\smallbreak}\pstart{}\textsc{{\pb}Herrn}\pend{}\pstart{}\textsc{Dr. Arthur Schnitzler}\pend{}\pstart{}\textsc{Wien\oindex{Wien@\textbf{Wien}, \emph{Verwaltungsgebiet}|pw}}\pend{}\pstart{}\textsc{XVIII. Spöttelgaſse 7\oindex{Wien@\textbf{Wien}!XVIII., Währing@\textbf{XVIII., Währing}!Edmund-Weiß-Gasse 7@\textbf{Edmund-Weiß-Gasse 7}, \emph{Wohngebäude}|pw}.}\pend{}{\bigskip}
\pstart
           \noindent{}\centering{}{\pb}\textcolor{gray}{\textbf{WIEN\oindex{Wien@\textbf{Wien}, \emph{Verwaltungsgebiet}|pw}. DENKMAL DES PRINZEN EUGEN\oindex{Prinz-Eugen-Reiterdenkmal@\textbf{Prinz-Eugen-Reiterdenkmal}, \emph{Monument}|pw}.}}\pend
           
\pstart
           \centering{}\textcolor{gray}{\textbf{\label{K_L03252-1v}\edtext{H. Junker\pwindex{Junker, Hermann 1838 Frankfurt am Main – 1899 ebd.@\textsc{Junker, Hermann} (1838 Frankfurt am Main – 1899 ebd.), \emph{Maler}|pw}\pwindex{Junker, Hermann 21.\,3.\,1867 Frankfurt am Main – 1928 Berlin@\textsc{Junker, Hermann} (21.\,3.\,1867 Frankfurt am Main – 1928 Berlin), \emph{Maler}|pw}}{\lemma{\textnormal{\emph{H. Junker}}}\Cendnote{\textnormal{Ob es sich bei dem Maler des
                     Postkartenmotivs um Hermann Junker d.
                        Ä.\pwindex{Junker, Hermann 1838 Frankfurt am Main – 1899 ebd.@\textsc{Junker, Hermann} (1838 Frankfurt am Main – 1899 ebd.), \emph{Maler}|pwk} (1838–1899) oder
                        Hermann Junker d. J.\pwindex{Junker, Hermann 21.\,3.\,1867 Frankfurt am Main – 1928 Berlin@\textsc{Junker, Hermann} (21.\,3.\,1867 Frankfurt am Main – 1928 Berlin), \emph{Maler}|pwk} (1867–1938) handelt, ist unklar.
                     Das abgedruckte Bild ist jedenfalls vor 1900
                     entstanden.}}}\label{K_L03252-1}.}}\pend
           \vspace{1em}
\pstart
           {\pb}Montag 27. 9.\pend
           \vspace{0.5em}
\pstart
           Lieber Freund, Ich habe die Abſicht, Dich, wenn ich von Dir nichts
               Gegenteiliges höre, morgen, Dienſtag, Nachmittag nach 5 Uhr zu \label{K_L03252-2v}\edtext{beſuchen}{\lemma{\textnormal{\emph{besuchen}}}\Cendnote{\textnormal{Siehe A. S.: \emph{Tagebuch}, 28. 9. 1909. }}}\label{K_L03252-2}.
               Herzliche Grüße Dir u. Deiner Frau\pwindex{Schnitzler, Olga 17.\,1.\,1882 Wien – 13.\,1.\,1970 Lugano@\textsc{Schnitzler, Olga} (17.\,1.\,1882 Wien – 13.\,1.\,1970 Lugano), \emph{Schauspielerin, Sängerin}|pwv}!\pend
           
\pstart
           Dein {\\[\baselineskip]}\spacefill\mbox{Paul Goldmann.}\pend
           \leftskip=0em{}\selectlanguage{ngerman}\endnumbering\briefempfaengerindex{Schnitzler, Arthur@\textsc{Schnitzler, Arthur}!zzzGoldmann, Paul@\emph{von Paul Goldmann}!1909-09-271@{27. 9. 1909}|)be}\mylabel{L03252h}  \newcommand{\dateiname}{L03252}\newcommand{\titel}{Paul Goldmann an Arthur Schnitzler, 27. 9. 1909}\newcommand{\editorInnen}{Martin Anton Müller und Laura Untner}%% latex-leseansicht-abspann.tex
%% Abspann für die Leseansicht.
%% Der Schalter \ifkorrekturansicht ist bereits durch den Vorspann gesetzt.

%% latex-abspann.tex
%% Gemeinsamer Abspann für Korrekturansicht und Leseansicht.
%% Setzt den Schalter \ifkorrekturansicht voraus (gesetzt in den
%% einbindenden Dateien latex-korrekturansicht-abspann.tex bzw.
%% latex-leseansicht-abspann.tex).
%% ---------------------------------------------------------------

\normalsize

% Das esempio-Environment wird nur in der Leseansicht benötigt
\ifkorrekturansicht\else
\newenvironment{esempio}[3]%
{
    \vspace{1.5ex}
    \rlap{\underline{#1}}
    \par
    \setlength{\parindent}{0cm}
    \nopagebreak
    \leftskip=#2cm
    \rightskip=#3cm
}
{
    \par
}
\fi

\doendnotes{C}
\bigskip
\vfill

\clearpage

\footnotesize

\ifkorrekturansicht
  \lohead{\textsc{register}}
\fi

% theindex-Environment neu definieren ohne reledmac
\makeatletter
\renewenvironment{theindex}{%
  \ifkorrekturansicht
    \section*{\indexname}%
  \else
    \subsubsection*{Index der erwähnten Entitäten}%
  \fi
  \setlength{\parindent}{0pt}%
  \setlength{\parskip}{0pt plus 0.3pt}%
  \let\item\@idxitem
}{%
  \ifkorrekturansicht\clearpage\fi
}
\makeatother

\IfFileExists{\jobname-pw.ind}{\input{\jobname-pw.ind}}{}

% Quellenangabe nur in der Leseansicht
\ifkorrekturansicht\else
% Fallback-Definitionen, falls die .tex-Datei \titel etc. nicht gesetzt hat
\providecommand{\titel}{}
\providecommand{\editorInnen}{}
\providecommand{\dateiname}{\jobname}

\vspace{3cm}

\vfill

\footnotesize
\textsc{Quelle}: \titel. Herausgegeben von {\editorInnen}. In: \emph{Arthur Schnitzler: Briefwechsel mit Autorinnen und Autoren}.
 Digitale Edition, https://schnitzler-briefe.acdh.oeaw.ac.at/{\dateiname}.html (Stand \today)
\fi

\end{document}


