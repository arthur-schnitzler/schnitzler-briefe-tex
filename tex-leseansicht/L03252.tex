%% latex-korrekturansicht-vorspann.tex
%% Vorspann für die Korrekturansicht.
%% Lädt die gemeinsame Datei latex-vorspann.tex mit gesetztem Schalter.

\newif\ifkorrekturansicht
\korrekturansichttrue

\input{../tex-inputs/latex-vorspann}


\section[ Paul Goldmann an Arthur Schnitzler, 27. 9. 1909]{L03252 Paul Goldmann an Arthur Schnitzler, 27. 9. 1909}
\nopagebreak\mylabel{L03252v}
\rehead{ }\normalsize\beginnumbering\briefempfaengerindex{Schnitzler, Arthur@\textsc{Schnitzler, Arthur}!zzzGoldmann, Paul@\emph{von Paul Goldmann}!1909-09-271@{27. 9. 1909}|(be}
\toendnotes[C]{\smallbreak\pagebreak[2]}\Standort{DLA, A:Schnitzler, HS.NZ85.1.3175.}
\physDesc{Bildpostkarte, 250 Zeichen
\newline{}Handschrift: 1) schwarze Tinte, deutsche Kurrent\hspace{1em}2) schwarze Tinte, lateinische Kurrent (\noindent{}Adresse)\hspace{1em}
\newline{}Versand: Stempel: »\nobreak{}Wien, {[}2{]}7. IX. 09, –9\nobreak{}«.  }\toendnotes[C]{\smallbreak}\pstart{}{\pb}Herrn\pend{}\pstart{}Dr. Arthur Schnitzler\pend{}\pstart{}Wien\oindex{Wien@\textbf{Wien}, \emph{A.ADM2}|pw}\pend{}\pstart{}XVIII. Spöttelgaſse 7\oindex{Edmund-Weiss-Gasse 7@\textbf{Edmund-Weiß-Gasse 7}, \emph{Wohngebäude (K.WHS)}|pw}.\pend{}{\bigskip}
\pstart
           \noindent{}\centering{}{\pb}\textcolor{gray}{\textbf{WIEN\oindex{Wien@\textbf{Wien}, \emph{A.ADM2}|pw}. DENKMAL DES PRINZEN EUGEN\oindex{Prinz-Eugen-Reiterdenkmal@\textbf{Prinz-Eugen-Reiterdenkmal}, \emph{Monument (K.MON)}|pw}.}}\pend
           
\pstart
           \centering{}\textcolor{gray}{\textbf{\label{K_L03252-1v}\edtext{H. Junker\pwindex{Junker, Hermann 1838 – 1899@\textsc{Junker, Hermann} (1838 – 1899), \emph{Maler/Malerin}|pw}\pwindex{Junker, Hermann 1867-03-21 – 1928@\textsc{Junker, Hermann} (1867-03-21 – 1928), \emph{Maler/Malerin}|pw}}{\lemma{\textnormal{\emph{H. Junker}}}\Cendnote{\textnormal{Ob es sich bei dem Maler des
                     Postkartenmotivs um Hermann Junker d.
                        Ä.\pwindex{Junker, Hermann 1838 – 1899@\textsc{Junker, Hermann} (1838 – 1899), \emph{Maler/Malerin}|pwk} (1838–1899) oder
                        Hermann Junker d. J.\pwindex{Junker, Hermann 1867-03-21 – 1928@\textsc{Junker, Hermann} (1867-03-21 – 1928), \emph{Maler/Malerin}|pwk} (1867–1938) handelt, ist unklar.
                     Das abgedruckte Bild ist jedenfalls vor 1900
                     entstanden.}}}\label{K_L03252-1}.}}\pend
           \vspace{1em}
\pstart
           {\pb}Montag 27. 9.\pend
           \vspace{0.5em}
\pstart
           Lieber Freund, Ich habe die Abſicht, Dich, wenn ich von Dir nichts
               Gegenteiliges höre, morgen, Dienſtag, Nachmittag nach 5 Uhr zu \label{K_L03252-2v}\edtext{beſuchen}{\lemma{\textnormal{\emph{beſuchen}}}\Cendnote{\textnormal{Siehe A. S.: \emph{Tagebuch}, 28. 9. 1909. }}}\label{K_L03252-2}.
               Herzliche Grüße Dir u. Deiner Frau\pwindex{Schnitzler, Olga 17.01.1882 – 13.01.1970@\textsc{Schnitzler, Olga} (17.01.1882 – 13.01.1970), \emph{Schauspieler/Schauspielerin, Sänger/Sängerin}|pwv}!\pend
           
\pstart
           Dein {\\[\baselineskip]}\spacefill\mbox{Paul Goldmann.}\pend
           \leftskip=0em{}\selectlanguage{ngerman}\endnumbering\briefempfaengerindex{Schnitzler, Arthur@\textsc{Schnitzler, Arthur}!zzzGoldmann, Paul@\emph{von Paul Goldmann}!1909-09-271@{27. 9. 1909}|)be}\mylabel{L03252h}  \normalsize

\doendnotes{C}
\bigskip
\vfill

\clearpage

\footnotesize

\lohead{\textsc{register}}

% Definiere theindex-Environment komplett neu ohne reledmac
\makeatletter
\renewenvironment{theindex}{%
  \section*{\indexname}%
  \setlength{\parindent}{0pt}%
  \setlength{\parskip}{0pt plus 0.3pt}%
  \let\item\@idxitem
}{%
  \clearpage
}
\makeatother

\IfFileExists{\jobname-pw.ind}{\input{\jobname-pw.ind}}{}

\end{document}

      