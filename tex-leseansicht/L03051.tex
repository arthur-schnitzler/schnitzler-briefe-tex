%% latex-leseansicht-vorspann.tex
%% Vorspann für die Leseansicht.
%% Lädt die gemeinsame Datei latex-vorspann.tex mit nicht gesetztem Schalter.

\newif\ifkorrekturansicht
\korrekturansichtfalse

\input{../tex-inputs/latex-vorspann}


\section[ Felix Salten: Widmungsexemplar Schauen und Spielen für Arthur Schnitzler, 22. 9. 1921]{L03051 Felix Salten: Widmungsexemplar Schauen und Spielen für Arthur
               Schnitzler,  22. 9. 1921}
\nopagebreak\mylabel{L03051v}
\rehead{ }\normalsize\beginnumbering\briefempfaengerindex{Schnitzler, Arthur@\textsc{Schnitzler, Arthur}!zzzSalten, Felix@\emph{von Felix Salten}!1921-09-221@{22. 9. 1921}|(be}
\toendnotes[C]{\smallbreak\pagebreak[2]}
\correspDesc{Versand  durch Felix Salten am 22. 9. 1921 in Wien
\newline{}Erhalt  durch Arthur Schnitzler im Zeitraum [22.–30. 9.1921?] in Wien}\toendnotes[C]{\smallbreak}
\Standort{DLA, G:Schnitzler, Arthur (Sammlung Heinrich Schnitzler).}
\physDesc{Widmung am Titelblatt, 70 Zeichen
\newline{}Handschrift: schwarze Tinte, lateinische Kurrent}
\pstart
           \noindent{}{\pb}Meinem lieben Arthur Schnitzler\pend
           
\pstart
           herzlichst {\\[\baselineskip]}\spacefill\mbox{Felix Salten}\pend
           \leftskip=0em{}
\pstart
           Wien\oindex{Wien@\textbf{Wien}, \emph{Verwaltungsgebiet}|pw}, 22. 9. 21\pend
           {\vspace{1\baselineskip}}
\pstart
           \centering{}\textcolor{gray}{\textbf{\so{Felix Salten}}}\pend
           
\pstart
           \centering{}\textcolor{gray}{\textbf{\textsc{\textbf{Schauen und Spielen\pwindex{Salten, Felix 6.\,9.\,1869 Budapest – 8.\,10.\,1945 Zürich@\textsc{Salten, Felix} (6.\,9.\,1869 Budapest – 8.\,10.\,1945 Zürich), \emph{Schriftsteller, Journalist, Chefredakteur}!Schauen und Spielen. Studien zur Kritik des modernen Theaters@\strich\emph{Schauen und Spielen. Studien zur Kritik des modernen Theaters}|pw}}}}}\pend
           
\pstart
           \centering{}\textcolor{gray}{\textbf{Erſter Band}}\pend
           
\pstart
           \centering{}\textcolor{gray}{\textbf{\so{Ergebniſſe}{ }{\\}\so{Erlebniſſe}\pwindex{Salten, Felix 6.\,9.\,1869 Budapest – 8.\,10.\,1945 Zürich@\textsc{Salten, Felix} (6.\,9.\,1869 Budapest – 8.\,10.\,1945 Zürich), \emph{Schriftsteller, Journalist, Chefredakteur}!Schauen und Spielen. Studien zur Kritik des modernen Theaters. Erster Band. Ergebnisse Erlebnisse@\strich\emph{Schauen und Spielen. Studien zur Kritik des modernen Theaters. Erster Band. Ergebnisse Erlebnisse}|pw}}}\pend
           {\vspace{1\baselineskip}}
\pstart
           \raggedleft{}\textcolor{gray}{\textbf{Bloße Vernunft, die{ }ſich am Kunſtwerk}}\pend
           
\pstart
           \raggedleft{}\textcolor{gray}{\textbf{reibt, begeht allemal Unzucht.}}\pend
           {\vspace{1\baselineskip}}
\pstart
           \centering{}\textcolor{gray}{\textbf{\so{1921}}}\pend
           
\pstart
           \centering{}\textcolor{gray}{\textbf{Wien\oindex{Wien@\textbf{Wien}, \emph{Verwaltungsgebiet}|pw} * WILA\orgindex{Wiener Literarische Anstalt@Wiener Literarische Anstalt|pw} * Leipzig\oindex{Leipzig@\textbf{Leipzig}, \emph{Hauptstadt}|pw}}}\pend
           \selectlanguage{ngerman}\endnumbering\briefempfaengerindex{Schnitzler, Arthur@\textsc{Schnitzler, Arthur}!zzzSalten, Felix@\emph{von Felix Salten}!1921-09-221@{22. 9. 1921}|)be}\mylabel{L03051h}  \newcommand{\dateiname}{L03051}\newcommand{\titel}{Felix Salten: Widmungsexemplar Schauen und Spielen für Arthur Schnitzler, 22. 9. 1921}\newcommand{\editorInnen}{Martin Anton Müller und Laura Untner}%% latex-leseansicht-abspann.tex
%% Abspann für die Leseansicht.
%% Der Schalter \ifkorrekturansicht ist bereits durch den Vorspann gesetzt.

%% latex-abspann.tex
%% Gemeinsamer Abspann für Korrekturansicht und Leseansicht.
%% Setzt den Schalter \ifkorrekturansicht voraus (gesetzt in den
%% einbindenden Dateien latex-korrekturansicht-abspann.tex bzw.
%% latex-leseansicht-abspann.tex).
%% ---------------------------------------------------------------

\normalsize

% Das esempio-Environment wird nur in der Leseansicht benötigt
\ifkorrekturansicht\else
\newenvironment{esempio}[3]%
{
    \vspace{1.5ex}
    \rlap{\underline{#1}}
    \par
    \setlength{\parindent}{0cm}
    \nopagebreak
    \leftskip=#2cm
    \rightskip=#3cm
}
{
    \par
}
\fi

\doendnotes{C}
\bigskip
\vfill

\clearpage

\footnotesize

\ifkorrekturansicht
  \lohead{\textsc{register}}
\fi

% theindex-Environment neu definieren ohne reledmac
\makeatletter
\renewenvironment{theindex}{%
  \ifkorrekturansicht
    \section*{\indexname}%
  \else
    \subsubsection*{Index der erwähnten Entitäten}%
  \fi
  \setlength{\parindent}{0pt}%
  \setlength{\parskip}{0pt plus 0.3pt}%
  \let\item\@idxitem
}{%
  \ifkorrekturansicht\clearpage\fi
}
\makeatother

\IfFileExists{\jobname-pw.ind}{\input{\jobname-pw.ind}}{}

% Quellenangabe nur in der Leseansicht
\ifkorrekturansicht\else
% Fallback-Definitionen, falls die .tex-Datei \titel etc. nicht gesetzt hat
\providecommand{\titel}{}
\providecommand{\editorInnen}{}
\providecommand{\dateiname}{\jobname}

\vspace{3cm}

\vfill

\footnotesize
\textsc{Quelle}: \titel. Herausgegeben von {\editorInnen}. In: \emph{Arthur Schnitzler: Briefwechsel mit Autorinnen und Autoren}.
 Digitale Edition, https://schnitzler-briefe.acdh.oeaw.ac.at/{\dateiname}.html (Stand \today)
\fi

\end{document}


