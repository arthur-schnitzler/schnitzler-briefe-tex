%% latex-korrekturansicht-vorspann.tex
%% Vorspann für die Korrekturansicht.
%% Lädt die gemeinsame Datei latex-vorspann.tex mit gesetztem Schalter.

\newif\ifkorrekturansicht
\korrekturansichttrue

\input{../tex-inputs/latex-vorspann}


\section[Bertha von Suttner an Arthur Schnitzler, {[}20. 12. 1913?{]}]{L02164 Bertha von Suttner an Arthur Schnitzler, {[}20. 12. 1913?{]}}
\nopagebreak\mylabel{L02164v}
\rehead{ }\normalsize\beginnumbering\briefempfaengerindex{Schnitzler, Arthur@\textsc{Schnitzler, Arthur}!zzzSuttner, Bertha von@\emph{von Bertha von Suttner}!1913-12-201@{{[}20. 12. 1913?{]}}|(be}
\toendnotes[C]{\smallbreak\pagebreak[2]}\Standort{CUL, Schnitzler, B 104.}
\physDesc{Briefkarte, 56 Zeichen (Krone in Golddruck )
\newline{}Handschrift: schwarze Tinte, deutsche Kurrent}\Standort{DLA, A:Schnitzler, HS.NZ85.1.4773.}
\physDesc{maschinenschriftliche Abschrift1 Blatt, 1 Seite, 56 Zeichen
\newline{}Schreibmaschine}\toendnotes[C]{\smallbreak}
\pstart
           \raggedleft{}{\pb}\textsc{Samstag}\pend
           \vspace{0.5em}
\pstart
           Ich freue mich ſehr auf \label{K_L02164-1v}\edtext{Montag}{\lemma{\textnormal{\emph{Montag}}}\Cendnote{\textnormal{Siehe A. S.: \emph{Tagebuch}, 22. 12. 1913.
               }}}\label{K_L02164-1}{ }N. M.\pend
           
\pstart
           Ihre{\\[\baselineskip]}\spacefill\mbox{B. Suttner}\pend
           \leftskip=0em{}\selectlanguage{ngerman}\endnumbering\briefempfaengerindex{Schnitzler, Arthur@\textsc{Schnitzler, Arthur}!zzzSuttner, Bertha von@\emph{von Bertha von Suttner}!1913-12-201@{{[}20. 12. 1913?{]}}|)be}\mylabel{L02164h}  \normalsize

\doendnotes{C}
\bigskip
\vfill

\clearpage

\footnotesize

\lohead{\textsc{register}}

% Definiere theindex-Environment komplett neu ohne reledmac
\makeatletter
\renewenvironment{theindex}{%
  \section*{\indexname}%
  \setlength{\parindent}{0pt}%
  \setlength{\parskip}{0pt plus 0.3pt}%
  \let\item\@idxitem
}{%
  \clearpage
}
\makeatother

\IfFileExists{\jobname-pw.ind}{\input{\jobname-pw.ind}}{}

\end{document}

      