%% latex-korrekturansicht-vorspann.tex
%% Vorspann für die Korrekturansicht.
%% Lädt die gemeinsame Datei latex-vorspann.tex mit gesetztem Schalter.

\newif\ifkorrekturansicht
\korrekturansichttrue

\input{../tex-inputs/latex-vorspann}


\section[Albert Ehrenstein an Arthur Schnitzler, 1. 7. 1909]{L01852 Albert Ehrenstein an Arthur Schnitzler, 1. 7. 1909}
\nopagebreak\mylabel{L01852v}
\rehead{ }\normalsize\beginnumbering\briefempfaengerindex{Schnitzler, Arthur@\textsc{Schnitzler, Arthur}!zzzEhrenstein, Albert@\emph{von Albert Ehrenstein}!1909-07-011@{1. 7. 1909}|(be}
\toendnotes[C]{\smallbreak\pagebreak[2]}\Standort{CUL, Schnitzler, B 30.}
\physDesc{Brief, 1 Blatt, 3 Seiten, 1352 Zeichen
\newline{}Handschrift: schwarze Tinte, deutsche Kurrent
\newline{}Schnitzler: mit Bleistift beschriftet: »\textsc{Ehrenste\textcolor{gray}{in}}« }\toendnotes[C]{\smallbreak}
\pstart
           
\pstart
           {\pb}Wien, XVI. \textsc{Ottakringerstr
                           114}\oindex{Ottakringer Strasse@\textbf{Ottakringer Straße}, \emph{Straße (K.STR)}|pw}\pend
           
\pstart
           \raggedleft{}\textsc{1. Juli 09}.\pend
           \pend
           
\pstart{}\textsc{Sehr geehrter Herr Doktor,}\pend\vspace{0.5em}
\pstart
           ohne läſtig fallen zu wollen, wäre es mir ſehr angenehm, wenn Sie, ſehr geehrter Herr
               Doktor, meinen drei\pwindex{Apaturien@\emph{Apaturien}|pwv}\pwindex{Tubutsch@\emph{Tubutsch}|pwv}\pwindex{Tod des Zehir eddin Muhammed Baber@\emph{Tod des Zehir eddin Muhammed Baber}|pwv} ebenſo länglichen als mißlungenen novelliſtiſchen Verſuchen, im Laufe
               der nächſten Wochen auf die eine oder die andere Art nahe zu treten die Güte haben
               möchten. Nach den Betrachtungen, die über H.
                  Mann\pwindex{Mann, Heinrich 27.03.1871 – 11.03.1950@\textsc{Mann, Heinrich} (27.03.1871 – 11.03.1950), \emph{Schriftsteller/Schriftstellerin}|pw} anzuſtellen ich unvorſichtig genug war, ſehne ich mich keineswegs. Da
                  {\pb}der Erdgeiſt\orgindex{Erdgeist@Erdgeist|pw} eingegangen iſt und mir dabei mein noch nicht abgedrucktes und
               abſchriftloſes Manuſkript einer Skizze verloren ging, meine Diſſertation\pwindex{Lage in Ungarn (Siebenbuergen und Serbien ausgenommen) im Jahre 1790@\emph{Die Lage in Ungarn (Siebenbürgen und Serbien ausgenommen) im Jahre 1790}|pwv}, ſo konſervativ wie meine
               andern Arbeiten gehalten war, begegnete ich bei dem betreffenden Hofrat\pwindex{Fournier, August 19.06.1850 – 18.05.1920@\textsc{Fournier, August} (19.06.1850 – 18.05.1920), \emph{Historiker/Historikerin}|pwv} namenloſen Chikanen. Ich werde allen
               möglichen Namen- und Zahlenkram lernen müſſen und doch nicht viel Chancen bei der
               Prüfung haben, wenn nicht irgend was augenfälliges von mir in der Zeit\pwindex{Zeit@\emph{Die Zeit}|pw} oder Preſſe\orgindex{Neue Freie Presse@Neue Freie Presse|pw} oder ſonſt
               einer reſpektabeln Zeitung erſcheint. Sollten Sie, {\pb}ſehr geehrter Herr Doktor mir in dieſer
               unverſchuldeten Zwangslage im mindeſten Beihilfe leiſten können, wäre ich ſo
               glücklich wie nur ein Menſch ſein kann, der die Namen ſämtlicher Erzbiſchöfe von Köln\oindex{Koeln@\textbf{Köln}, \emph{P.PPLA2}|pw} und dergleichen Ungeheuerlichkeiten ſeinem
               Gedächtniſſe einzuverleiben das Vergnügen hat.\pend
           
\pstart
           Indem ich um Entſchuldigung dieſes in der Eile hingeworfenen Briefes bitte, verbleibe
               ich\pend
           
\pstart
           Ihr ergebenſter{\\[\baselineskip]}\spacefill\mbox{Albert Ehrenstein.}\pend
           \leftskip=0em{}\selectlanguage{ngerman}\endnumbering\briefempfaengerindex{Schnitzler, Arthur@\textsc{Schnitzler, Arthur}!zzzEhrenstein, Albert@\emph{von Albert Ehrenstein}!1909-07-011@{1. 7. 1909}|)be}\mylabel{L01852h}  \normalsize

\doendnotes{C}
\bigskip
\vfill

\clearpage

\footnotesize

\lohead{\textsc{register}}

% Definiere theindex-Environment komplett neu ohne reledmac
\makeatletter
\renewenvironment{theindex}{%
  \section*{\indexname}%
  \setlength{\parindent}{0pt}%
  \setlength{\parskip}{0pt plus 0.3pt}%
  \let\item\@idxitem
}{%
  \clearpage
}
\makeatother

\IfFileExists{\jobname-pw.ind}{\input{\jobname-pw.ind}}{}

\end{document}

      