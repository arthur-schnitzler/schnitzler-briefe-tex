%% latex-leseansicht-vorspann.tex
%% Vorspann für die Leseansicht.
%% Lädt die gemeinsame Datei latex-vorspann.tex mit nicht gesetztem Schalter.

\newif\ifkorrekturansicht
\korrekturansichtfalse

\input{../tex-inputs/latex-vorspann}


\section[Albert Ehrenstein an Arthur Schnitzler, 1. 7. 1909]{L01852 Albert Ehrenstein an Arthur Schnitzler, 1. 7. 1909}
\nopagebreak\mylabel{L01852v}
\rehead{ }\normalsize\beginnumbering\briefempfaengerindex{Schnitzler, Arthur@\textsc{Schnitzler, Arthur}!zzzEhrenstein, Albert@\emph{von Albert Ehrenstein}!1909-07-011@{1. 7. 1909}|(be}
\toendnotes[C]{\smallbreak\pagebreak[2]}
\correspDesc{Versand  durch Albert Ehrenstein am 1. 7. 1909 in Wien
\newline{}Weiterleitung  in Wien
\newline{}Erhalt  durch Arthur Schnitzler im Zeitraum [3. 7. 1909
                  – 7. 7. 1909?] in Edlach}\toendnotes[C]{\smallbreak}
\Standort{CUL, Schnitzler, B 30.}
\physDesc{Brief, 1 Blatt, 3 Seiten, 1352 Zeichen
\newline{}Handschrift: schwarze Tinte, deutsche Kurrent
\newline{}Schnitzler: mit Bleistift beschriftet: »\textsc{Ehrenste\textcolor{gray}{in}}« }\toendnotes[C]{\smallbreak}
\pstart
           
\pstart
           {\pb}Wien, XVI. \textsc{Ottakringerstr
                           114}\oindex{Wien@\textbf{Wien}!XVI., Ottakring@\textbf{XVI., Ottakring}!Ottakringer Straße@\textbf{Ottakringer Straße}, \emph{Straße}|pw}\oindex{Wien@\textbf{Wien}!XVII., Hernals@\textbf{XVII., Hernals}!Ottakringer Straße@\textbf{Ottakringer Straße}, \emph{Straße}|pw}\pend
           
\pstart
           \raggedleft{}\textsc{1. Juli 09}.\pend
           \pend
           
\pstart{}\textsc{Sehr geehrter Herr Doktor,}\pend\vspace{0.5em}
\pstart
           ohne läſtig fallen zu wollen, wäre es mir{ }ſehr angenehm, wenn Sie,{ }ſehr geehrter Herr
               Doktor, meinen drei\pwindex{Ehrenstein, Albert 23.\,12.\,1886 Wien – 8.\,4.\,1950 New York City@\textsc{Ehrenstein, Albert} (23.\,12.\,1886 Wien – 8.\,4.\,1950 New York City), \emph{Schriftsteller}!Apaturien@\strich\emph{Apaturien}|pwv}\pwindex{Ehrenstein, Albert 23.\,12.\,1886 Wien – 8.\,4.\,1950 New York City@\textsc{Ehrenstein, Albert} (23.\,12.\,1886 Wien – 8.\,4.\,1950 New York City), \emph{Schriftsteller}!Tubutsch@\strich\emph{Tubutsch}|pwv}\pwindex{Ehrenstein, Albert 23.\,12.\,1886 Wien – 8.\,4.\,1950 New York City@\textsc{Ehrenstein, Albert} (23.\,12.\,1886 Wien – 8.\,4.\,1950 New York City), \emph{Schriftsteller}!Tod des Zehir eddin Muhammed Baber@\strich\emph{Tod des Zehir eddin Muhammed Baber}|pwv} ebenſo länglichen als mißlungenen novelliſtiſchen Verſuchen, im Laufe
               der nächſten Wochen auf die eine oder die andere Art nahe zu treten die Güte haben
               möchten. Nach den Betrachtungen, die über H.
                  Mann\pwindex{Mann, Heinrich 27.\,3.\,1871 Lübeck – 11.\,3.\,1950 Santa Monica@\textsc{Mann, Heinrich} (27.\,3.\,1871 Lübeck – 11.\,3.\,1950 Santa Monica), \emph{Schriftsteller}|pw} anzuſtellen ich unvorſichtig genug war,{ }ſehne ich mich keineswegs. Da
                  {\pb}der Erdgeiſt\orgindex{Erdgeist@Erdgeist|pw} eingegangen iſt und mir dabei mein noch nicht abgedrucktes und
               abſchriftloſes Manuſkript einer Skizze verloren ging, meine Diſſertation\pwindex{Ehrenstein, Albert 23.\,12.\,1886 Wien – 8.\,4.\,1950 New York City@\textsc{Ehrenstein, Albert} (23.\,12.\,1886 Wien – 8.\,4.\,1950 New York City), \emph{Schriftsteller}!Lage in Ungarn (Siebenbürgen und Serbien ausgenommen) im Jahre 1790@\strich\emph{Die Lage in Ungarn (Siebenbürgen und Serbien ausgenommen) im Jahre 1790}|pwv},{ }ſo konſervativ wie meine
               andern Arbeiten gehalten war, begegnete ich bei dem betreffenden Hofrat\pwindex{Fournier, August 19.\,6.\,1850 Wien – 18.\,5.\,1920 ebd.@\textsc{Fournier, August} (19.\,6.\,1850 Wien – 18.\,5.\,1920 ebd.), \emph{Historiker}|pwv} namenloſen Chikanen. Ich werde allen
               möglichen Namen- und Zahlenkram lernen müſſen und doch nicht viel Chancen bei der
               Prüfung haben, wenn nicht irgend was augenfälliges von mir in der Zeit\pwindex{Zeit@\emph{Die Zeit}|pw} oder Preſſe\orgindex{Neue Freie Presse@Neue Freie Presse|pw} oder{ }ſonſt
               einer reſpektabeln Zeitung erſcheint. Sollten Sie, {\pb}ſehr geehrter Herr Doktor mir in dieſer
               unverſchuldeten Zwangslage im mindeſten Beihilfe leiſten können, wäre ich{ }ſo
               glücklich wie nur ein Menſch{ }ſein kann, der die Namen{ }ſämtlicher Erzbiſchöfe von Köln\oindex{Köln@\textbf{Köln}, \emph{Hauptstadt}|pw} und dergleichen Ungeheuerlichkeiten{ }ſeinem
               Gedächtniſſe einzuverleiben das Vergnügen hat.\pend
           
\pstart
           Indem ich um Entſchuldigung dieſes in der Eile hingeworfenen Briefes bitte, verbleibe
               ich\pend
           
\pstart
           Ihr ergebenſter{\\[\baselineskip]}\spacefill\mbox{Albert Ehrenstein.}\pend
           \leftskip=0em{}\selectlanguage{ngerman}\endnumbering\briefempfaengerindex{Schnitzler, Arthur@\textsc{Schnitzler, Arthur}!zzzEhrenstein, Albert@\emph{von Albert Ehrenstein}!1909-07-011@{1. 7. 1909}|)be}\mylabel{L01852h}  \newcommand{\dateiname}{L01852}\newcommand{\titel}{Albert Ehrenstein an Arthur Schnitzler, 1. 7. 1909}\newcommand{\editorInnen}{Martin Anton Müller und Gerd-Hermann Susen}%% latex-leseansicht-abspann.tex
%% Abspann für die Leseansicht.
%% Der Schalter \ifkorrekturansicht ist bereits durch den Vorspann gesetzt.

%% latex-abspann.tex
%% Gemeinsamer Abspann für Korrekturansicht und Leseansicht.
%% Setzt den Schalter \ifkorrekturansicht voraus (gesetzt in den
%% einbindenden Dateien latex-korrekturansicht-abspann.tex bzw.
%% latex-leseansicht-abspann.tex).
%% ---------------------------------------------------------------

\normalsize

% Das esempio-Environment wird nur in der Leseansicht benötigt
\ifkorrekturansicht\else
\newenvironment{esempio}[3]%
{
    \vspace{1.5ex}
    \rlap{\underline{#1}}
    \par
    \setlength{\parindent}{0cm}
    \nopagebreak
    \leftskip=#2cm
    \rightskip=#3cm
}
{
    \par
}
\fi

\doendnotes{C}
\bigskip
\vfill

\clearpage

\footnotesize

\ifkorrekturansicht
  \lohead{\textsc{register}}
\fi

% theindex-Environment neu definieren ohne reledmac
\makeatletter
\renewenvironment{theindex}{%
  \ifkorrekturansicht
    \section*{\indexname}%
  \else
    \subsubsection*{Index der erwähnten Entitäten}%
  \fi
  \setlength{\parindent}{0pt}%
  \setlength{\parskip}{0pt plus 0.3pt}%
  \let\item\@idxitem
}{%
  \ifkorrekturansicht\clearpage\fi
}
\makeatother

\IfFileExists{\jobname-pw.ind}{\input{\jobname-pw.ind}}{}

% Quellenangabe nur in der Leseansicht
\ifkorrekturansicht\else
% Fallback-Definitionen, falls die .tex-Datei \titel etc. nicht gesetzt hat
\providecommand{\titel}{}
\providecommand{\editorInnen}{}
\providecommand{\dateiname}{\jobname}

\vspace{3cm}

\vfill

\footnotesize
\textsc{Quelle}: \titel. Herausgegeben von {\editorInnen}. In: \emph{Arthur Schnitzler: Briefwechsel mit Autorinnen und Autoren}.
 Digitale Edition, https://schnitzler-briefe.acdh.oeaw.ac.at/{\dateiname}.html (Stand \today)
\fi

\end{document}


