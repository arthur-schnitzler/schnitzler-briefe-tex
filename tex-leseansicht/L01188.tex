%% latex-korrekturansicht-vorspann.tex
%% Vorspann für die Korrekturansicht.
%% Lädt die gemeinsame Datei latex-vorspann.tex mit gesetztem Schalter.

\newif\ifkorrekturansicht
\korrekturansichttrue

\input{../tex-inputs/latex-vorspann}


\section[Arthur Schnitzler an Richard Beer-Hofmann, 25. 11. 1901]{L01188 Arthur Schnitzler an Richard Beer-Hofmann, 25. 11. 1901}
\nopagebreak\mylabel{L01188v}
\rehead{ }\normalsize\beginnumbering\briefempfaengerindex{Beer-Hofmann, Richard@\textsc{Beer-Hofmann, Richard}!zzzSchnitzler, Arthur@\emph{von Arthur Schnitzler}!1901-11-251@{25. 11. 1901}|(be}
\toendnotes[C]{\smallbreak\pagebreak[2]}\Standort{YCGL, MSS 31.}
\physDesc{Brief, 1 Blatt, 3 Seiten, Umschlag, 475 Zeichen
\newline{}Handschrift: Bleistift, deutsche Kurrent
\newline{}Versand: 1) Rohrpost  2) Stempel: »\nobreak{}\oindex{IX., Alsergrund@\textbf{IX., Alsergrund}, \emph{A.ADM3}|pwk}Wien 9/1, 25 XI 01, 3 50N\nobreak{}«.  3) Stempel: »\nobreak{}\oindex{I., Innere Stadt@\textbf{I., Innere Stadt}, \emph{A.ADM3}|pwk}{\pb}Wien 1/1, 25 XI 01, 4 10\textcolor{gray}{N}\nobreak{}«. }
\buchAbdrucke{\weitereDrucke{Arthur Schnitzler, Richard Beer-Hofmann: \emph{Briefwechsel 1891–1931}. Wien, Zürich: \emph{Europaverlag} 1992, S. 156–157.} }\toendnotes[C]{\smallbreak}\pstart{}{\pb}Herrn \textsc{Dr. Richard
                     Beer-Hofmann}\pend{}\pstart{}Wien\oindex{Wien@\textbf{Wien}, \emph{A.ADM2}|pw}\pend{}\pstart{}\textsc{I. Wollzeile 15\oindex{Wollzeile@\textbf{Wollzeile}, \emph{Straße (K.STR)}|pw}}\pend{}{\bigskip}\vspace{1em}
\pstart
           \raggedleft{}{\pb}25. 11. 901\pend
           
\pstart{}lieber Richard.\pend\vspace{0.5em}
\pstart
           Ich war heute Vormittag bei Hugo\pwindex{Hofmannsthal, Hugo von 1874-02-01 – 1929-07-15@\textsc{Hofmannsthal, Hugo von} (1874-02-01 – 1929-07-15), \emph{Schriftsteller/Schriftstellerin}|pw}.\pend
           
\pstart
           Wollen Sie, daſs ich Ihnen beiden Mittwoch oder Donnerſtag{ }Nachmittag gegen 6 meine \label{K_L01188-1v}\edtext{4
                  Stücke\pwindex{Lebendige Stunden@\emph{Lebendige Stunden}|pwv}\pwindex{letzten Masken@\emph{Die letzten Masken}|pwv}\pwindex{Literatur@\emph{Literatur}|pwv}\pwindex{Frau mit dem Dolche@\emph{Die Frau mit dem Dolche}|pwv} vorleſe}{\lemma{\textnormal{\emph{4
                  Stücke vorleſe}}}\Cendnote{\textnormal{Vgl. A. S.: \emph{Tagebuch}, 14. 12. 1901.
               }}}\label{K_L01188-1}? Wir (Sie u ich{[}){]}{ }{\pb}könnten dann Abends zuſammen
               hereinfahren. (Eventuell auch zuſa{\geminationm}en hinaus, wenn Sie
               nicht aus Wohnungsgründen früher draußen ſein müſſen.)\pend
           
\pstart
           Alſo Mittwoch oder Donnerſtag oder in dieſer Woche gar
               nicht.\pend
           
\pstart
           {\pb}Schreiben Sie mir, ich benachrichtige dann Hugo\pwindex{Hofmannsthal, Hugo von 1874-02-01 – 1929-07-15@\textsc{Hofmannsthal, Hugo von} (1874-02-01 – 1929-07-15), \emph{Schriftsteller/Schriftstellerin}|pw}.\pend
           
\pstart
           Herzlichst{\\[\baselineskip]}Ihr{\\[\baselineskip]}\spacefill\mbox{Arthur}\pend
           \leftskip=0em{}\selectlanguage{ngerman}\endnumbering\briefempfaengerindex{Beer-Hofmann, Richard@\textsc{Beer-Hofmann, Richard}!zzzSchnitzler, Arthur@\emph{von Arthur Schnitzler}!1901-11-251@{25. 11. 1901}|)be}\mylabel{L01188h}  \normalsize

\doendnotes{C}
\bigskip
\vfill

\clearpage

\footnotesize

\lohead{\textsc{register}}

% Definiere theindex-Environment komplett neu ohne reledmac
\makeatletter
\renewenvironment{theindex}{%
  \section*{\indexname}%
  \setlength{\parindent}{0pt}%
  \setlength{\parskip}{0pt plus 0.3pt}%
  \let\item\@idxitem
}{%
  \clearpage
}
\makeatother

\IfFileExists{\jobname-pw.ind}{\input{\jobname-pw.ind}}{}

\end{document}

      