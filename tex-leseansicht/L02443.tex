%% latex-leseansicht-vorspann.tex
%% Vorspann für die Leseansicht.
%% Lädt die gemeinsame Datei latex-vorspann.tex mit nicht gesetztem Schalter.

\newif\ifkorrekturansicht
\korrekturansichtfalse

\input{../tex-inputs/latex-vorspann}


               \section[Georg Brandes an Arthur Schnitzler, 21. 6. 1925]{ Georg Brandes an Arthur Schnitzler, 21. 6. 1925}\nopagebreak\mylabel{v}\rehead{ }\begin{ledgroupsized}[t]{13cm}\normalsize\beginnumbering\briefempfaengerindex{Schnitzler, Arthur@\textsc{Schnitzler, Arthur}!zzzBrandes, Georg@\emph{von Georg Brandes}!1925-06-211@{21. 6. 1925}|(be} \toendnotes[C]{\smallbreak\pagebreak[2]} \Standort{CUL, Schnitzler, B 17.}
\physDesc{Brief, 1 Blatt, 3 Seiten
\newline{}Handschrift: schwarze Tinte, lateinische Kurrent
\newline{}Schnitzler: mit rotem Buntstift vereinzelte Unterstreichungen \newline{}Ordnung: mit Bleistift von unbekannter Hand nummeriert: »59« }\buchAbdrucke{\weitereDrucke{Georg Brandes, Arthur Schnitzler: \emph{Ein Briefwechsel}. Hg. Kurt Bergel. Bern: \emph{Francke} 1956, S. 146–147.} }\toendnotes[C]{\smallbreak}\pstart
           \raggedleft{}{\pb}Kopenhagen\oindex{Kopenhagen@\textbf{Kopenhagen}|pw}{ }21 Juni 25\pend
           \pstart{}Mein lieber Freund\pend\pstart
           Sie waren diesmal wieder sehr gütig gegen mich in Wien\oindex{Wien@\textbf{Wien}|pw}. Ich ging nach Salzburg\oindex{Salzburg@\textbf{Salzburg}|pw}, verlor
                    aber dort vier Wochen mit Bronchitis, bin hier, und kann über die Gesundheit
                    nicht klagen, obwol der Sommer hier kalt und unheimlich ist.\pend
           \pstart
           Ich hätte Ihnen sehr gerne mein kleines Buch Hellas\pwindex{Brandes, Georg 04.02.1842 – 19.02.1927@\textsc{Brandes, Georg} (04.02.1842 – 19.02.1927)!Hellas1925@\strich\emph{Hellas} {[}1925{]}|pw} geschickt, aber leider durch allerlei Verlegerschwierigkeiten
                    lässt die deutsche Uebersetzung auf sich warten.\pend
           \pstart
           Es war schön, Sie und Ihr Haus wieder zu sehn. Es that mir leid zu merken, dass
                    Ihre Stimmung nicht heiter war. Sie waren nicht deshalb weniger liebenswürdig,
                    aber ich gönnte {\pb}Ihnen mehr
                    Lebensfreude.\pend
           \pstart
           Man hat ja seitdem ein älteres Schauspiel\pwindex{Schnitzler, Arthur 15.05.1862 – 21.10.1931@\textsc{Schnitzler, Arthur} (15.05.1862 – 21.10.1931), \emph{Schriftsteller, Mediziner}!Schleier der Beatrice. Schauspiel in fuenf Akten1900-12-01 – 1900-12-01@\strich\emph{Der Schleier der Beatrice. Schauspiel in fünf Akten} {[}1900-12-01 – 1900-12-01{]}|pwv} von Ihnen im Burgtheater\oindex{Burgtheater@\textbf{Burgtheater}|pw}\label{K_L02443_1v}\edtext{aufgeführt}{\lemma{\textnormal{\emph{aufgeführt}}}\Cendnote{\textnormal{Erste Wien\oindex{Wien@\textbf{Wien}|pwk}er Aufführung am
                            23. 5. 1925}}}\label{K_L02443_1h}; ich hoffe, dass die Poesie des Stückes\pwindex{Schnitzler, Arthur 15.05.1862 – 21.10.1931@\textsc{Schnitzler, Arthur} (15.05.1862 – 21.10.1931), \emph{Schriftsteller, Mediziner}!Schleier der Beatrice. Schauspiel in fuenf Akten1900-12-01 – 1900-12-01@\strich\emph{Der Schleier der Beatrice. Schauspiel in fünf Akten} {[}1900-12-01 – 1900-12-01{]}|pwv} zu ihrem Rechte kam.
                    Es muss doch ein angenehmes Gefühl sein, auf viele Menschen zugleich zu wirken.
                    Sie sind diesem Genuss gegenüber wol etwas verwöhnt und blasirt, aber nicht
                    desto weniger!\pend
           \pstart
           Ich wurde eingeladen, die Festlichkeiten wegen des 200 jährigen Bestehens der Academie der Wissenschaften\orgindex{Akademie der Wissenschaften@Akademie der Wissenschaften|pw} in Leningrad\oindex{Sankt Petersburg@\textbf{Sankt Petersburg}|pw} (!) mitzumachen; sie strecken sich in Petersburg\oindex{Sankt Petersburg@\textbf{Sankt Petersburg}|pw} und Moskau\oindex{Moskau@\textbf{Moskau}|pw} von 6–16 September, aber ich wollte
                    als Gast nicht heucheln, und Entzücken über den {\pb}jetzigen Zustand in Russland\oindex{Russland@\textbf{Russland}|pw} wäre meinerseits Heuchelei. Reden
                    müsste ich ja, und das schreckte mich. Sonst hätte ich gerne die zwei Städte\oindex{Sankt Petersburg@\textbf{Sankt Petersburg}|pwv}\oindex{Moskau@\textbf{Moskau}|pwv} unter den
                    veränderten Umständen wiedergesehen.\pend
           \pstart
           Sie waren sehr lieb so wol gegen meine Begleiterin\pwindex{Rung, Gertrud 26.03.1882 – 25.04.1959@\textsc{Rung, Gertrud} (26.03.1882 – 25.04.1959), \emph{Übersetzerin, Sekretärin}|pwv} wie gegen mich.\pend
           \pstart
           Leider reist jetzt Fru Rung\pwindex{Rung, Gertrud 26.03.1882 – 25.04.1959@\textsc{Rung, Gertrud} (26.03.1882 – 25.04.1959), \emph{Übersetzerin, Sekretärin}|pw} mit ihrem Gatten\pwindex{Rung, Otto 16.06.1874 – 19.10.1945@\textsc{Rung, Otto} (16.06.1874 – 19.10.1945), \emph{Schriftsteller}|pwv} und ihrer Cousine\pwindex{?? [Kusine von Gertrud Rung] *~1925@\textsc{?? [Kusine von Gertrud Rung]} (*~1925)|pwv} auf 6 Wochen nach
                        Italien\oindex{Italien@\textbf{Italien}|pw}. Ich kann ohne sie meine
                    Correspondenz nicht bewältigen.\pend
           \pstart
           Sie wissen kaum, wie dankbar ich mich im Innersten für Ihre vieljährige
                    Freundschaft fühle.\pend
           \pstart Ihr \spacefill\mbox{Georg Brandes}\pend{}\endnumbering\briefempfaengerindex{Schnitzler, Arthur@\textsc{Schnitzler, Arthur}!zzzBrandes, Georg@\emph{von Georg Brandes}!1925-06-211@{21. 6. 1925}|)be}\mylabel{h}\end{ledgroupsized}  \newcommand{\dateiname}{L02443}\newcommand{\titel}{Georg Brandes an Arthur Schnitzler, 21. 6. 1925}\newcommand{\editorInnen}{Martin Anton Müller und Gerd-Hermann Susen}
            \footnotesize
\begin{ledgroupsized}[t]{11.5cm}
\doendnotes{C}
\end{ledgroupsized}
         %% latex-leseansicht-abspann.tex
%% Abspann für die Leseansicht.
%% Der Schalter \ifkorrekturansicht ist bereits durch den Vorspann gesetzt.

%% latex-abspann.tex
%% Gemeinsamer Abspann für Korrekturansicht und Leseansicht.
%% Setzt den Schalter \ifkorrekturansicht voraus (gesetzt in den
%% einbindenden Dateien latex-korrekturansicht-abspann.tex bzw.
%% latex-leseansicht-abspann.tex).
%% ---------------------------------------------------------------

\normalsize

% Das esempio-Environment wird nur in der Leseansicht benötigt
\ifkorrekturansicht\else
\newenvironment{esempio}[3]%
{
    \vspace{1.5ex}
    \rlap{\underline{#1}}
    \par
    \setlength{\parindent}{0cm}
    \nopagebreak
    \leftskip=#2cm
    \rightskip=#3cm
}
{
    \par
}
\fi

\doendnotes{C}
\bigskip
\vfill

\clearpage

\footnotesize

\ifkorrekturansicht
  \lohead{\textsc{register}}
\fi

% theindex-Environment neu definieren ohne reledmac
\makeatletter
\renewenvironment{theindex}{%
  \ifkorrekturansicht
    \section*{\indexname}%
  \else
    \subsubsection*{Index der erwähnten Entitäten}%
  \fi
  \setlength{\parindent}{0pt}%
  \setlength{\parskip}{0pt plus 0.3pt}%
  \let\item\@idxitem
}{%
  \ifkorrekturansicht\clearpage\fi
}
\makeatother

\IfFileExists{\jobname-pw.ind}{\input{\jobname-pw.ind}}{}

% Quellenangabe nur in der Leseansicht
\ifkorrekturansicht\else
% Fallback-Definitionen, falls die .tex-Datei \titel etc. nicht gesetzt hat
\providecommand{\titel}{}
\providecommand{\editorInnen}{}
\providecommand{\dateiname}{\jobname}

\vspace{3cm}

\vfill

\footnotesize
\textsc{Quelle}: \titel. Herausgegeben von {\editorInnen}. In: \emph{Arthur Schnitzler: Briefwechsel mit Autorinnen und Autoren}.
 Digitale Edition, https://schnitzler-briefe.acdh.oeaw.ac.at/{\dateiname}.html (Stand \today)
\fi

\end{document}


      