%% latex-leseansicht-vorspann.tex
%% Vorspann für die Leseansicht.
%% Lädt die gemeinsame Datei latex-vorspann.tex mit nicht gesetztem Schalter.

\newif\ifkorrekturansicht
\korrekturansichtfalse

\input{../tex-inputs/latex-vorspann}


\section[ Felix Salten an Arthur Schnitzler, 17. 7. 1897]{L03269 Felix Salten an Arthur Schnitzler,  17. 7. 1897}
\nopagebreak\mylabel{L03269v}
\rehead{ }\normalsize\beginnumbering\briefempfaengerindex{Schnitzler, Arthur@\textsc{Schnitzler, Arthur}!zzzSalten, Felix@\emph{von Felix Salten}!1897-07-171@{17. 7. 1897}|(be}
\toendnotes[C]{\smallbreak\pagebreak[2]}
\correspDesc{Versand  durch Felix Salten am 17. 7. 1897 in Wien
\newline{}Erhalt  durch Arthur Schnitzler am 18. 7. 1897 in Ischl}\toendnotes[C]{\smallbreak}
\Standort{CUL, Schnitzler, B 89, A 2.}
\physDesc{Postkarte, 279 Zeichen
\newline{}Handschrift: Bleistift, lateinische Kurrent
\newline{}Versand: Stempel: »\nobreak{}\oindex{I., Innere Stadt@\textbf{I., Innere Stadt}, \emph{Verwaltungsgebiet}|pwk}Wien 1/1 1, 17. 7. 97, 11–12 N\nobreak{}«. Stempel: »\nobreak{}\oindex{Bad Ischl@\textbf{Bad Ischl}|pwk}Ischl, 18\textcolor{gray}{.} 7. 97\nobreak{}«.  
\newline{}Schnitzler: mit Bleistift datiert: »17. 7\textcolor{gray}{. 97}« 
\newline{}Ordnung: mit Bleistift von unbekannter Hand nummeriert: »92« }\toendnotes[C]{\smallbreak}\pstart{}{\pb}Herrn D\textsuperscript{r} Arthur Schnitzler\pend{}\pstart{}Ischl\oindex{Bad Ischl@\textbf{Bad Ischl}|pw}\pend{}\pstart{}Kaltenbach\oindex{Kaltenbach@\textbf{Kaltenbach}, \emph{Teil eines besiedelten Ortes}|pw}, Pension Rudolfshöhe\oindex{Hotel und Pension Rudolfshöhe (Leopold Petter)@\textbf{Hotel und Pension Rudolfshöhe (Leopold Petter)}, \emph{Hotel}|pw}.\pend{}{\bigskip}\vspace{1em}
\pstart
           \noindent{}{\pb}Lieber Freund, viel Dank für Ihren Brief. Die \label{K_L03269-1v}\edtext{Sache G. H.\pwindex{Hirschfeld, Georg 11.\,2.\,1873 Berlin – 17.\,1.\,1942 München@\textsc{Hirschfeld, Georg} (11.\,2.\,1873 Berlin – 17.\,1.\,1942 München), \emph{Schriftsteller}|pw}}{\lemma{\textnormal{\emph{Sache G. H.}}}\Cendnote{\textnormal{Wenige Tage zuvor war die Annahme von
                  Georg Hirschfelds\pwindex{Hirschfeld, Georg 11.\,2.\,1873 Berlin – 17.\,1.\,1942 München@\textsc{Hirschfeld, Georg} (11.\,2.\,1873 Berlin – 17.\,1.\,1942 München), \emph{Schriftsteller}|pwk} neuem Stück \emph{Agnes Jordan}\pwindex{Hirschfeld, Georg 11.\,2.\,1873 Berlin – 17.\,1.\,1942 München@\textsc{Hirschfeld, Georg} (11.\,2.\,1873 Berlin – 17.\,1.\,1942 München), \emph{Schriftsteller}!Agnes Jordan. Schauspiel in fünf Akten@\strich\emph{Agnes Jordan. Schauspiel in fünf Akten}|pwk} am \emph{Deutschen Theater Berlin}\orgindex{Deutsches Theater Berlin@Deutsches Theater Berlin|pwk} gemeldet worden (vgl.
                     [O. V.]: \emph{Theater und Kunst}\pwindex{Theater und Kunst [Agnes Jordan angenommen]@\emph{Theater und Kunst [Agnes Jordan angenommen]}|pwk}. In: \emph{Neues Wiener Journal}\pwindex{Neues Wiener Journal@\emph{Neues Wiener Journal}|pwk}, Nr. 1337, 14. 7. 1897, S. 6). Das \emph{Burgtheater}\orgindex{Burgtheater@Burgtheater|pwk} zog in diesen Tagen die Annahme des Stücks\pwindex{Hirschfeld, Georg 11.\,2.\,1873 Berlin – 17.\,1.\,1942 München@\textsc{Hirschfeld, Georg} (11.\,2.\,1873 Berlin – 17.\,1.\,1942 München), \emph{Schriftsteller}!Agnes Jordan. Schauspiel in fünf Akten@\strich\emph{Agnes Jordan. Schauspiel in fünf Akten}|pwkv}
                  zurück, was Schnitzler durch einen Brief von
                     Hirschfeld\pwindex{Hirschfeld, Georg 11.\,2.\,1873 Berlin – 17.\,1.\,1942 München@\textsc{Hirschfeld, Georg} (11.\,2.\,1873 Berlin – 17.\,1.\,1942 München), \emph{Schriftsteller}|pwk} vom 12. 7. 1897
                  erfuhr: »Denken Sie, bald nach Ihrem Brief bekam ich endlich Burckhards\pwindex{Burckhard, Max Eugen 14.\,7.\,1854 Korneuburg – 16.\,3.\,1912 Wien@\textsc{Burckhard, Max Eugen} (14.\,7.\,1854 Korneuburg – 16.\,3.\,1912 Wien), \emph{Schriftsteller, Rechtswissenschaftler, Theaterleiter}|pw} Brief, in dem er mir
                     auseinanderſetzte, mit allem Lob, aller Achtung, daß er das Stück\pwindex{Hirschfeld, Georg 11.\,2.\,1873 Berlin – 17.\,1.\,1942 München@\textsc{Hirschfeld, Georg} (11.\,2.\,1873 Berlin – 17.\,1.\,1942 München), \emph{Schriftsteller}!Agnes Jordan. Schauspiel in fünf Akten@\strich\emph{Agnes Jordan. Schauspiel in fünf Akten}|pwv}{ }\uline{nicht} nehmen könnte. Zenſurbedenken, und wenn
                     dieſe fortfielen, ›ſociale‹ Bedenken, ein Teil des Publikums würde oſtentativ
                     Bravo klatſchen, der andere dadurch – beleidigt{ }ſein.« (\emph{CUL}, B42) Im Hintergrund der Entscheidung dürfte
                  aus Sicht Saltens\pwindex{Salten, Felix 6.\,9.\,1869 Budapest – 8.\,10.\,1945 Zürich@\textsc{Salten, Felix} (6.\,9.\,1869 Budapest – 8.\,10.\,1945 Zürich), \emph{Schriftsteller, Journalist, Chefredakteur}|pwk} und Schnitzlers{ }Hermann
                     Bahr\pwindex{Bahr, Hermann 19.\,7.\,1863 Linz – 15.\,1.\,1934 München@\textsc{Bahr, Hermann} (19.\,7.\,1863 Linz – 15.\,1.\,1934 München), \emph{Schriftsteller, Kritiker}|pwk} gestanden sein, der das Stück für »antisemitisch«
                  hielt (A. S.: \emph{Tagebuch}, 21. 6. 1897, vgl. Hermann Bahr, Arthur Schnitzler: \emph{Briefwechsel, Aufzeichnungen, Dokumente (1891–1931)}, Arthur Schnitzler an Marie Reinhard, 25. 6. 1897) und in engem Austausch
                  mit dem Direktor Max Burckhard\pwindex{Burckhard, Max Eugen 14.\,7.\,1854 Korneuburg – 16.\,3.\,1912 Wien@\textsc{Burckhard, Max Eugen} (14.\,7.\,1854 Korneuburg – 16.\,3.\,1912 Wien), \emph{Schriftsteller, Rechtswissenschaftler, Theaterleiter}|pwk} stand (vgl. XXXX Auszeichnungsfehler: Dokument L03270 nicht gefunden, XXXX Auszeichnungsfehler: Dokument L03271 nicht gefunden).}}}\label{K_L03269-1} wusste ich schon, da H.\pwindex{Hirschfeld, Georg 11.\,2.\,1873 Berlin – 17.\,1.\,1942 München@\textsc{Hirschfeld, Georg} (11.\,2.\,1873 Berlin – 17.\,1.\,1942 München), \emph{Schriftsteller}|pw} mir schrieb. Auch ich habe die bewussten
               Einflüße sofort erkannt, und mich sehr geärgert. Mein \label{K_L03269-2v}\edtext{Buch\pwindex{Salten, Felix 6.\,9.\,1869 Budapest – 8.\,10.\,1945 Zürich@\textsc{Salten, Felix} (6.\,9.\,1869 Budapest – 8.\,10.\,1945 Zürich), \emph{Schriftsteller, Journalist, Chefredakteur}!Hinterbliebene. Kurze Novellen@\strich\emph{Der Hinterbliebene. Kurze Novellen}|pwuv}}{\lemma{\textnormal{\emph{Buch}}}\Cendnote{\textnormal{Eventuell spricht er vom Novellenband \emph{Der Hinterbliebene}\pwindex{Salten, Felix 6.\,9.\,1869 Budapest – 8.\,10.\,1945 Zürich@\textsc{Salten, Felix} (6.\,9.\,1869 Budapest – 8.\,10.\,1945 Zürich), \emph{Schriftsteller, Journalist, Chefredakteur}!Hinterbliebene. Kurze Novellen@\strich\emph{Der Hinterbliebene. Kurze Novellen}|pwk} (vgl. XXXX Auszeichnungsfehler: Dokument L03181 nicht gefunden).}}}\label{K_L03269-2} ist noch nicht fertig. Auf
               Wiedersehen\pend
           \pstart \spacefill\mbox{Salten}\pend{}\selectlanguage{ngerman}\endnumbering\briefempfaengerindex{Schnitzler, Arthur@\textsc{Schnitzler, Arthur}!zzzSalten, Felix@\emph{von Felix Salten}!1897-07-171@{17. 7. 1897}|)be}\mylabel{L03269h}  \newcommand{\dateiname}{L03269}\newcommand{\titel}{Felix Salten an Arthur Schnitzler, 17. 7. 1897}\newcommand{\editorInnen}{Martin Anton Müller und Laura Untner}%% latex-leseansicht-abspann.tex
%% Abspann für die Leseansicht.
%% Der Schalter \ifkorrekturansicht ist bereits durch den Vorspann gesetzt.

%% latex-abspann.tex
%% Gemeinsamer Abspann für Korrekturansicht und Leseansicht.
%% Setzt den Schalter \ifkorrekturansicht voraus (gesetzt in den
%% einbindenden Dateien latex-korrekturansicht-abspann.tex bzw.
%% latex-leseansicht-abspann.tex).
%% ---------------------------------------------------------------

\normalsize

% Das esempio-Environment wird nur in der Leseansicht benötigt
\ifkorrekturansicht\else
\newenvironment{esempio}[3]%
{
    \vspace{1.5ex}
    \rlap{\underline{#1}}
    \par
    \setlength{\parindent}{0cm}
    \nopagebreak
    \leftskip=#2cm
    \rightskip=#3cm
}
{
    \par
}
\fi

\doendnotes{C}
\bigskip
\vfill

\clearpage

\footnotesize

\ifkorrekturansicht
  \lohead{\textsc{register}}
\fi

% theindex-Environment neu definieren ohne reledmac
\makeatletter
\renewenvironment{theindex}{%
  \ifkorrekturansicht
    \section*{\indexname}%
  \else
    \subsubsection*{Index der erwähnten Entitäten}%
  \fi
  \setlength{\parindent}{0pt}%
  \setlength{\parskip}{0pt plus 0.3pt}%
  \let\item\@idxitem
}{%
  \ifkorrekturansicht\clearpage\fi
}
\makeatother

\IfFileExists{\jobname-pw.ind}{\input{\jobname-pw.ind}}{}

% Quellenangabe nur in der Leseansicht
\ifkorrekturansicht\else
% Fallback-Definitionen, falls die .tex-Datei \titel etc. nicht gesetzt hat
\providecommand{\titel}{}
\providecommand{\editorInnen}{}
\providecommand{\dateiname}{\jobname}

\vspace{3cm}

\vfill

\footnotesize
\textsc{Quelle}: \titel. Herausgegeben von {\editorInnen}. In: \emph{Arthur Schnitzler: Briefwechsel mit Autorinnen und Autoren}.
 Digitale Edition, https://schnitzler-briefe.acdh.oeaw.ac.at/{\dateiname}.html (Stand \today)
\fi

\end{document}


