%% latex-korrekturansicht-vorspann.tex
%% Vorspann für die Korrekturansicht.
%% Lädt die gemeinsame Datei latex-vorspann.tex mit gesetztem Schalter.

\newif\ifkorrekturansicht
\korrekturansichttrue

\input{../tex-inputs/latex-vorspann}


\section[ Felix Salten an Arthur Schnitzler, 17. 7. 1897]{L03269 Felix Salten an Arthur Schnitzler, 17. 7. 1897}
\nopagebreak\mylabel{L03269v}
\rehead{ }\normalsize\beginnumbering\briefempfaengerindex{Schnitzler, Arthur@\textsc{Schnitzler, Arthur}!zzzSalten, Felix@\emph{von Felix Salten}!1897-07-171@{17. 7. 1897}|(be}
\toendnotes[C]{\smallbreak\pagebreak[2]}\Standort{CUL, Schnitzler, B 89, A 2.}
\physDesc{Postkarte, 279 Zeichen
\newline{}Handschrift: Bleistift, lateinische Kurrent
\newline{}Versand: Stempel: »\nobreak{}\oindex{I., Innere Stadt@\textbf{I., Innere Stadt}, \emph{A.ADM3}|pwk}Wien 1/1 1, 17. 7. 97, 11–12 N\nobreak{}«. Stempel: »\nobreak{}\oindex{Bad Ischl@\textbf{Bad Ischl}, \emph{P.PPL}|pwk}Ischl, 18\textcolor{gray}{.} 7. 97\nobreak{}«.  
\newline{}Schnitzler: mit Bleistift datiert: »17. 7\textcolor{gray}{. 97}« 
\newline{}Ordnung: mit Bleistift von unbekannter Hand nummeriert: »92« }\toendnotes[C]{\smallbreak}\pstart{}{\pb}Herrn D\textsuperscript{r} Arthur Schnitzler\pend{}\pstart{}Ischl\oindex{Bad Ischl@\textbf{Bad Ischl}, \emph{P.PPL}|pw}\pend{}\pstart{}Kaltenbach\oindex{Kaltenbach@\textbf{Kaltenbach}, \emph{Teil eines besiedelten Ortes (A.BSOX)}|pw}, Pension Rudolfshöhe\oindex{Hotel und Pension Rudolfshoehe (Leopold Petter)@\textbf{Hotel und Pension Rudolfshöhe (Leopold Petter)}, \emph{Hotel (K.HTL)}|pw}.\pend{}{\bigskip}\vspace{1em}
\pstart
           \noindent{}{\pb}Lieber Freund, viel Dank für Ihren Brief. Die \label{K_L03269-1v}\edtext{Sache G. H.\pwindex{Hirschfeld, Georg 11.02.1873 – 17.01.1942@\textsc{Hirschfeld, Georg} (11.02.1873 – 17.01.1942), \emph{Schriftsteller/Schriftstellerin}|pw}}{\lemma{\textnormal{\emph{Sache G. H.}}}\Cendnote{\textnormal{Wenige Tage zuvor war die Annahme von
                  Georg Hirschfelds\pwindex{Hirschfeld, Georg 11.02.1873 – 17.01.1942@\textsc{Hirschfeld, Georg} (11.02.1873 – 17.01.1942), \emph{Schriftsteller/Schriftstellerin}|pwk} neuem Stück \emph{Agnes Jordan}\pwindex{Agnes Jordan. Schauspiel in fuenf Akten@\emph{Agnes Jordan. Schauspiel in fünf Akten}|pwk} am \emph{Deutschen Theater Berlin}\orgindex{Deutsches Theater Berlin@Deutsches Theater Berlin|pwk} gemeldet worden (vgl.
                     [O. V.]: \emph{Theater und Kunst}\pwindex{Theater und Kunst [Agnes Jordan angenommen]@\emph{Theater und Kunst [Agnes Jordan angenommen]}|pwk}. In: \emph{Neues Wiener Journal}\pwindex{Neues Wiener Journal@\emph{Neues Wiener Journal}|pwk}, Nr. 1337, 14. 7. 1897, S. 6). Das \emph{Burgtheater}\orgindex{Burgtheater@Burgtheater|pwk} zog in diesen Tagen die Annahme des Stücks\pwindex{Agnes Jordan. Schauspiel in fuenf Akten@\emph{Agnes Jordan. Schauspiel in fünf Akten}|pwkv}
                  zurück, was Schnitzler durch einen Brief von
                     Hirschfeld\pwindex{Hirschfeld, Georg 11.02.1873 – 17.01.1942@\textsc{Hirschfeld, Georg} (11.02.1873 – 17.01.1942), \emph{Schriftsteller/Schriftstellerin}|pwk} vom 12. 7. 1897
                  erfuhr: »Denken Sie, bald nach Ihrem Brief bekam ich endlich Burckhards\pwindex{Burckhard, Max Eugen 14.07.1854 – 16.03.1912@\textsc{Burckhard, Max Eugen} (14.07.1854 – 16.03.1912), \emph{Schriftsteller/Schriftstellerin, Rechtswissenschaftler/Rechtswissenschaftlerin, Theaterleiter/Theaterleiterin}|pw} Brief, in dem er mir
                     auseinanderſetzte, mit allem Lob, aller Achtung, daß er das Stück\pwindex{Agnes Jordan. Schauspiel in fuenf Akten@\emph{Agnes Jordan. Schauspiel in fünf Akten}|pwv}{ }\uline{nicht} nehmen könnte. Zenſurbedenken, und wenn
                     dieſe fortfielen, ›ſociale‹ Bedenken, ein Teil des Publikums würde oſtentativ
                     Bravo klatſchen, der andere dadurch – beleidigt ſein.« (\emph{CUL}, B42) Im Hintergrund der Entscheidung dürfte
                  aus Sicht Saltens\pwindex{Salten, Felix 06.09.1869 – 08.10.1945@\textsc{Salten, Felix} (06.09.1869 – 08.10.1945), \emph{Schriftsteller/Schriftstellerin, Journalist/Journalistin, Chefredakteur/Chefredakteurin}|pwk} und Schnitzlers{ }Hermann
                     Bahr\pwindex{Bahr, Hermann 19.07.1863 – 15.01.1934@\textsc{Bahr, Hermann} (19.07.1863 – 15.01.1934), \emph{Schriftsteller/Schriftstellerin, Kritiker/Kritikerin}|pwk} gestanden sein, der das Stück für »antisemitisch«
                  hielt (A. S.: \emph{Tagebuch}, 21. 6. 1897, vgl. Hermann Bahr, Arthur Schnitzler: \emph{Briefwechsel, Aufzeichnungen, Dokumente (1891–1931)}, Arthur Schnitzler an Marie Reinhard, 25. 6. 1897) und in engem Austausch
                  mit dem Direktor Max Burckhard\pwindex{Burckhard, Max Eugen 14.07.1854 – 16.03.1912@\textsc{Burckhard, Max Eugen} (14.07.1854 – 16.03.1912), \emph{Schriftsteller/Schriftstellerin, Rechtswissenschaftler/Rechtswissenschaftlerin, Theaterleiter/Theaterleiterin}|pwk} stand (vgl. Felix Salten an Arthur Schnitzler, 22. 7. 1897, Felix Salten an Arthur Schnitzler, 23. 7. 1897).}}}\label{K_L03269-1} wusste ich schon, da H.\pwindex{Hirschfeld, Georg 11.02.1873 – 17.01.1942@\textsc{Hirschfeld, Georg} (11.02.1873 – 17.01.1942), \emph{Schriftsteller/Schriftstellerin}|pw} mir schrieb. Auch ich habe die bewussten
               Einflüße sofort erkannt, und mich sehr geärgert. Mein \label{K_L03269-2v}\edtext{Buch\pwindex{Hinterbliebene. Kurze Novellen@\emph{Der Hinterbliebene. Kurze Novellen}|pwuv}}{\lemma{\textnormal{\emph{Buch}}}\Cendnote{\textnormal{Eventuell spricht er vom Novellenband \emph{Der Hinterbliebene}\pwindex{Hinterbliebene. Kurze Novellen@\emph{Der Hinterbliebene. Kurze Novellen}|pwk} (vgl. Felix Salten an Arthur Schnitzler, [30. 10. 1896]).}}}\label{K_L03269-2} ist noch nicht fertig. Auf
               Wiedersehen\pend
           \pstart \spacefill\mbox{Salten}\pend{}\selectlanguage{ngerman}\endnumbering\briefempfaengerindex{Schnitzler, Arthur@\textsc{Schnitzler, Arthur}!zzzSalten, Felix@\emph{von Felix Salten}!1897-07-171@{17. 7. 1897}|)be}\mylabel{L03269h}  \normalsize

\doendnotes{C}
\bigskip
\vfill

\clearpage

\footnotesize

\lohead{\textsc{register}}

% Definiere theindex-Environment komplett neu ohne reledmac
\makeatletter
\renewenvironment{theindex}{%
  \section*{\indexname}%
  \setlength{\parindent}{0pt}%
  \setlength{\parskip}{0pt plus 0.3pt}%
  \let\item\@idxitem
}{%
  \clearpage
}
\makeatother

\IfFileExists{\jobname-pw.ind}{\input{\jobname-pw.ind}}{}

\end{document}

      