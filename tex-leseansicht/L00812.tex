%% latex-korrekturansicht-vorspann.tex
%% Vorspann für die Korrekturansicht.
%% Lädt die gemeinsame Datei latex-vorspann.tex mit gesetztem Schalter.

\newif\ifkorrekturansicht
\korrekturansichttrue

\input{../tex-inputs/latex-vorspann}


\section[Richard Beer-Hofmann an Arthur Schnitzler, 4. 7. 1898]{L00812 Richard Beer-Hofmann an Arthur Schnitzler, 4. 7. 1898}
\nopagebreak\mylabel{L00812v}
\rehead{ }\normalsize\beginnumbering\briefempfaengerindex{Schnitzler, Arthur@\textsc{Schnitzler, Arthur}!zzzBeer-Hofmann, Richard@\emph{von Richard Beer-Hofmann}!1898-07-041@{4. 7. 1898}|(be}
\toendnotes[C]{\smallbreak\pagebreak[2]}\Standort{CUL, Schnitzler, B 8.}
\physDesc{Telegramm, 154 Zeichen
\newline{}maschinell
\newline{}Versand: Stempel des Telegrammbeamten Nikorowicz\pwindex{Johann, Nikorowicz @\textsc{Johann, Nikorowicz}, \emph{Telegrafenbeamter/Telegrafenbeamtin}|pw} }
\buchAbdrucke{\weitereDrucke{Arthur Schnitzler, Richard Beer-Hofmann: \emph{Briefwechsel 1891–1931}. Wien, Zürich: \emph{Europaverlag} 1992, S. 122.} }\toendnotes[C]{\smallbreak}
\pstart
           {\pb}win\oindex{Wien@\textbf{Wien}, \emph{A.ADM2}|pw} fr steindorfosziachersee\oindex{Steindorf am Ossiacher See@\textbf{Steindorf am Ossiacher See}, \emph{A.ADM3}|pw} 5 22 4/7{ }3 45n = \pend
           \vspace{0.5em}
\pstart
           schicken sye mir bitte sofort \label{K_L00812-1v}\edtext{nummero
                  sechzehn}{\lemma{\textnormal{\emph{nummero
                  sechzehn}}}\Cendnote{\textnormal{Diese Nummer war am
                     1. 7. 1898 erschienen.}}}\label{K_L00812-1} der wiener
                  rundschau\orgindex{Wiener Rundschau@Wiener Rundschau|pw} von \label{K_L00812-2v}\edtext{ersten juny}{\lemma{\textnormal{\emph{ersten juny}}}\Cendnote{\textnormal{Wie aus dem Brief vom 3. 7. 1898 hervorgeht,
                  meinte er den 1. 7. 1898.}}}\label{K_L00812-2} bryef unterwegs herzlychst
                  \spacefill\mbox{rychard .+}\pend
           \selectlanguage{ngerman}\endnumbering\briefempfaengerindex{Schnitzler, Arthur@\textsc{Schnitzler, Arthur}!zzzBeer-Hofmann, Richard@\emph{von Richard Beer-Hofmann}!1898-07-041@{4. 7. 1898}|)be}\mylabel{L00812h}  \normalsize

\doendnotes{C}
\bigskip
\vfill

\clearpage

\footnotesize

\lohead{\textsc{register}}

% Definiere theindex-Environment komplett neu ohne reledmac
\makeatletter
\renewenvironment{theindex}{%
  \section*{\indexname}%
  \setlength{\parindent}{0pt}%
  \setlength{\parskip}{0pt plus 0.3pt}%
  \let\item\@idxitem
}{%
  \clearpage
}
\makeatother

\IfFileExists{\jobname-pw.ind}{\input{\jobname-pw.ind}}{}

\end{document}

      