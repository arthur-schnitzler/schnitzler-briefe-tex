%% latex-leseansicht-vorspann.tex
%% Vorspann für die Leseansicht.
%% Lädt die gemeinsame Datei latex-vorspann.tex mit nicht gesetztem Schalter.

\newif\ifkorrekturansicht
\korrekturansichtfalse

\input{../tex-inputs/latex-vorspann}


\section[Arthur Schnitzler an Richard Beer-Hofmann, {[}1901?{]}]{L01089 Arthur Schnitzler an Richard Beer-Hofmann, {[}1901?{]}}
\nopagebreak\mylabel{L01089v}
\rehead{ }\normalsize\beginnumbering\briefempfaengerindex{Beer-Hofmann, Richard@\textsc{Beer-Hofmann, Richard}!zzzSchnitzler, Arthur@\emph{von Arthur Schnitzler}!1901-12-311@{{[}1901?{]}}|(be}
\toendnotes[C]{\smallbreak\pagebreak[2]}
\correspDesc{Versand  durch Arthur Schnitzler im Zeitraum [1901?] in Wien
\newline{}Erhalt  durch Richard Beer-Hofmann in Wien}\toendnotes[C]{\smallbreak}
\Standort{YCGL, MSS 31.}
\physDesc{Brief, 1 Blatt, 3 Seiten, Kuvert, 311 Zeichen
\newline{}Handschrift: Bleistift, deutsche Kurrent
\newline{}Versand: ohne postalischen Übermittlungsvermerk }\toendnotes[C]{\smallbreak}\pstart{}{\pb}Hrn \textsc{Dr Rich.
                     Beer-Hofmann}\pend{}\pstart{}Wien\oindex{Wien@\textbf{Wien}, \emph{Verwaltungsgebiet}|pw}\pend{}\pstart{}\textsc{I. Wollzeile 15\oindex{Wien@\textbf{Wien}!I., Innere Stadt@\textbf{I., Innere Stadt}!Wollzeile 15 (»Berthahof«)@\textbf{Wollzeile 15 (»Berthahof«)}, \emph{Wohngebäude}|pw}}.\pend{}{\bigskip}\vspace{1em}
\pstart{}{\pb}lieber Richard\pend\vspace{0.5em}
\pstart
           ſein Sie nicht bös, daſs ich Ihnen da einen Herrn\pwindex{?? [Person in traurigen Verhältnissen] @\textsc{?? [Person in traurigen Verhältnissen]}|pwv}{ }ſchicke, der mich ungeheuer dringend drum bittet und
               der thatſächlich {\pb}in{ }ſehr traurigen Verhältniſſen zu{ }ſein{ }ſcheint. Vielleicht können Sie auch etwas für ihn thun.\pend
           
\pstart
           {\pb}Herzlichſt Ihr{\\[\baselineskip]}\spacefill\mbox{ArthS}\pend
           \leftskip=0em{}
\pstart
           \noindent{}\label{K_L01089-1v}\edtext{Samſtg}{\lemma{\textnormal{\emph{Samstg}}}\Cendnote{\textnormal{Bislang konnte dieses Dokument noch nicht
                     genauer datiert werden. Es wird unter den Korrespondenzstücken des Jahres
                     1901 aufbewahrt, was die einzige vornehmbare zeitliche
                     Einordnung darstellt.}}}\label{K_L01089-1} hoffentlich Club\orgindex{Wiener Schachclub@Wiener Schachclub|pwv}.\pend
           \selectlanguage{ngerman}\endnumbering\briefempfaengerindex{Beer-Hofmann, Richard@\textsc{Beer-Hofmann, Richard}!zzzSchnitzler, Arthur@\emph{von Arthur Schnitzler}!1901-01-011@{{[}1901?{]}}|)be}\mylabel{L01089h}  \newcommand{\dateiname}{L01089}\newcommand{\titel}{Arthur Schnitzler an Richard Beer-Hofmann, [1901?]}\newcommand{\editorInnen}{Martin Anton Müller und Gerd-Hermann Susen}%% latex-leseansicht-abspann.tex
%% Abspann für die Leseansicht.
%% Der Schalter \ifkorrekturansicht ist bereits durch den Vorspann gesetzt.

%% latex-abspann.tex
%% Gemeinsamer Abspann für Korrekturansicht und Leseansicht.
%% Setzt den Schalter \ifkorrekturansicht voraus (gesetzt in den
%% einbindenden Dateien latex-korrekturansicht-abspann.tex bzw.
%% latex-leseansicht-abspann.tex).
%% ---------------------------------------------------------------

\normalsize

% Das esempio-Environment wird nur in der Leseansicht benötigt
\ifkorrekturansicht\else
\newenvironment{esempio}[3]%
{
    \vspace{1.5ex}
    \rlap{\underline{#1}}
    \par
    \setlength{\parindent}{0cm}
    \nopagebreak
    \leftskip=#2cm
    \rightskip=#3cm
}
{
    \par
}
\fi

\doendnotes{C}
\bigskip
\vfill

\clearpage

\footnotesize

\ifkorrekturansicht
  \lohead{\textsc{register}}
\fi

% theindex-Environment neu definieren ohne reledmac
\makeatletter
\renewenvironment{theindex}{%
  \ifkorrekturansicht
    \section*{\indexname}%
  \else
    \subsubsection*{Index der erwähnten Entitäten}%
  \fi
  \setlength{\parindent}{0pt}%
  \setlength{\parskip}{0pt plus 0.3pt}%
  \let\item\@idxitem
}{%
  \ifkorrekturansicht\clearpage\fi
}
\makeatother

\IfFileExists{\jobname-pw.ind}{\input{\jobname-pw.ind}}{}

% Quellenangabe nur in der Leseansicht
\ifkorrekturansicht\else
% Fallback-Definitionen, falls die .tex-Datei \titel etc. nicht gesetzt hat
\providecommand{\titel}{}
\providecommand{\editorInnen}{}
\providecommand{\dateiname}{\jobname}

\vspace{3cm}

\vfill

\footnotesize
\textsc{Quelle}: \titel. Herausgegeben von {\editorInnen}. In: \emph{Arthur Schnitzler: Briefwechsel mit Autorinnen und Autoren}.
 Digitale Edition, https://schnitzler-briefe.acdh.oeaw.ac.at/{\dateiname}.html (Stand \today)
\fi

\end{document}


