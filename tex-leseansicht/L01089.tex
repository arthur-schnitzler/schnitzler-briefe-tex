\input{../tex-inputs/latex-pdf-vorspann}
\begin{center}
            \textcolor{red}{ENTWURF. ENTZIFFERUNG NOCH NICHT KORREKTURGELESEN}
                      \end{center}
            
               \section[Arthur Schnitzler an Richard Beer-Hofmann, {[}1901?{]}]{ Arthur Schnitzler an Richard Beer-Hofmann, {[}1901?{]}}\nopagebreak\mylabel{v}\rehead{ }\begin{ledgroupsized}[t]{13cm}\normalsize\beginnumbering\briefempfaengerindex{Beer-Hofmann, Richard@\textsc{Beer-Hofmann, Richard}!zzzSchnitzler, Arthur@\emph{von Arthur Schnitzler}!1901-01-011@{{[}1901?{]}}|(be} \toendnotes[C]{\smallbreak\pagebreak[2]} \Standort{YCGL, MSS 31.}
\physDesc{Brief, 1 Blatt, 3 Seiten, Umschlag
\newline{}Handschrift: Bleistift, deutsche Kurrent\newline{}Versand: ohne postalischen Übermittlungsvermerk }\toendnotes[C]{\smallbreak}\pstart{}{\pb}Hrn \textsc{Dr Rich.
                            Beer-Hofmann}\pend{}\pstart{}Wien\oindex{Wien@\textbf{Wien}|pw}\pend{}\pstart{}\textsc{I. Wollzeile 15\oindex{Wollzeile@\textbf{Wollzeile}|pw}}.\pend{}{\bigskip}\pstart{}{\pb}lieber Richard\pend\pstart
           ſein Sie nicht bös, daſs ich Ihnen da einen Herrn\pwindex{?? [Person in traurigen Verhaeltnissen] @\textsc{?? [Person in traurigen Verhältnissen]}|pwv} ſchicke, der mich ungeheuer dringend drum
                    bittet und der thatſächlich {\pb}in ſehr traurigen
                    Verhältniſſen zu ſein ſcheint. Vielleicht können Sie auch etwas für ihn
                    thun.\pend
           \pstart
           {\pb}Herzlichſt Ihr{\\[\baselineskip]}\spacefill\mbox{ArthS}\pend
           \leftskip=0em{}\pstart
           \noindent{}\label{K_L01089_1v}\edtext{Samſtg}{\lemma{\textnormal{\emph{Samſtg}}}\Cendnote{\textnormal{Bislang
                            konnte dieses Dokument noch nicht genauer datiert werden. Da es unter
                            den Korrespondenzstücken des Jahres 1901 aufbewahrt wird,
                            ist dies die einzige vorgenommene zeitliche Einordnung.}}}\label{K_L01089_1h}
                        hoffentlich Club\orgindex{Wiener Schachclub@Wiener Schachclub|pwv}.\pend
           \endnumbering\briefempfaengerindex{Beer-Hofmann, Richard@\textsc{Beer-Hofmann, Richard}!zzzSchnitzler, Arthur@\emph{von Arthur Schnitzler}!1901-01-011@{{[}1901?{]}}|)be}\mylabel{h}\end{ledgroupsized}  \newcommand{\dateiname}{L01089}\newcommand{\titel}{Arthur Schnitzler an Richard Beer-Hofmann, [1901?]}\newcommand{\editorInnen}{Martin Anton Müller und Gerd-Hermann Susen}\input{../tex-inputs/latex-pdf-abspann}
      