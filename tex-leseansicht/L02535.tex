%% latex-leseansicht-vorspann.tex
%% Vorspann für die Leseansicht.
%% Lädt die gemeinsame Datei latex-vorspann.tex mit nicht gesetztem Schalter.

\newif\ifkorrekturansicht
\korrekturansichtfalse

\input{../tex-inputs/latex-vorspann}

\begin{center}
            \textcolor{red}{ENTWURF. ENTZIFFERUNG NOCH NICHT KORREKTURGELESEN}
                      \end{center}
            
               \section[Gerty von Hofmannsthal an Arthur Schnitzler, 9. 4. 1930]{ Gerty von Hofmannsthal an Arthur Schnitzler, 9. 4. 1930}\nopagebreak\mylabel{v}\rehead{ }\begin{ledgroupsized}[t]{13cm}\normalsize\beginnumbering\briefempfaengerindex{Schnitzler, Arthur@\textsc{Schnitzler, Arthur}!zzzHofmannsthal, Gertrude von@\emph{von Gertrude von Hofmannsthal}!1930-04-091@{9. 4. 1930}|(be} \toendnotes[C]{\smallbreak\pagebreak[2]} \Standort{CUL, Schnitzler, B 43.}
\physDesc{Brief, 1 Blatt (Briefpapier mit Trauerrand), 2 Seiten
\newline{}Schreibmaschine
\newline{}Handschrift: schwarze Tinte, deutsche Kurrent (\noindent{}Unterschrift)
\newline{}Schnitzler: mit rotem Buntstift mehrere Unterstreichungen }\toendnotes[C]{\smallbreak}\pstart
           \noindent{}{\pb}IV Mozartgasse 4\hfill Wien\oindex{Wien@\textbf{Wien}|pw} d. 9/IV 30\pend
           \pstart
           Telephon U 43384\pend
           \pstart
           Lieber Arthur, ich habe heute versucht Sie anzurufen hörte
                    aber, dass Sie eine andere \label{K_L02535_1v}\edtext{Geheimnummer}{\lemma{\textnormal{\emph{Geheimnummer}}}\Cendnote{\textnormal{vgl. Arthur Schnitzler an Gerty von Hofmannsthal, 17. 2. 1931}}}\label{K_L02535_1h} haben, wahrscheinlich sind Sie zu viel angerufen worden, darum sage ich
                    Ihnen heute meine Bitte schriftlich\pend
           \pstart
           Mein Advokat Dr Weinmann\pwindex{Weinmann, Leonhard *~1877-09-24@\textsc{Weinmann, Leonhard} (*~1877-09-24), \emph{Rechtsanwalt}|pw} würde so sehr eine
                    Unterredung mit Ihnen wünschen, es handelt sich wegen der Erbsteuer um
                    Bestimmung der Autoreneinkünfte, die man möglichst gering angeben muss, weil es
                    als Kapital angesehen wird (was wirklich recht ungerecht ist, finde ich dass es
                    doch sicher sehr schwankend sein wird) Ich konnte Dr W.\pwindex{Weinmann, Leonhard *~1877-09-24@\textsc{Weinmann, Leonhard} (*~1877-09-24), \emph{Rechtsanwalt}|pw} niemanden andern nennen als Sie, als bester Freund und
                    auch als Autor, der competent ist seine Meinung zu sagen. Was die Opern betrifft
                    hat Schalk\pwindex{Schalk, Franz 27.05.1863 – 03.09.1931@\textsc{Schalk, Franz} (27.05.1863 – 03.09.1931), \emph{Theaterleiter, Dirigent}|pw} eine Art Gutachten gegeben.
                        Dr W.\pwindex{Weinmann, Leonhard *~1877-09-24@\textsc{Weinmann, Leonhard} (*~1877-09-24), \emph{Rechtsanwalt}|pw} wird Ihnen das alles besser
                    erklären können als ich. Wollen Sie also die grosse Güte haben den Mann einmal
                    in nächster Zeit zu \label{T_L02535_1v}\edtext{einer}{\lemma{\textnormal{\emph{einer}}}\Cendnote{\textnormal{Sie schreibt:
                        »einen«}}}\label{T_L02535_1h} Ihnen passenden Stunde zu empfangen?
                    Natürlich müsste ich es einige Tage früher wissen, da der Mann\pwindex{Weinmann, Leonhard *~1877-09-24@\textsc{Weinmann, Leonhard} (*~1877-09-24), \emph{Rechtsanwalt}|pw} sehr beschäftigt ist und auch oft Verhandlungen hat.
                    Bitte rufen Sie mich einmal zwischen 10–11 vorm an, wo ich fast
                    immer zuhaus bin und lassen Sie mich ein Wort wissen.\pend
           \pstart
           Ich war drei Wochen in Berlin\oindex{Berlin@\textbf{Berlin}|pw}, habe Olga\pwindex{Schnitzler, Olga 17.01.1882 – 13.01.1970@\textsc{Schnitzler, Olga} (17.01.1882 – 13.01.1970), \emph{Schauspielerin, Sängerin}|pw} gesehen, die ich sehr wohl fand und war
                    entzückt über die Wohnung, die ich so besonders geschmackvoll fand. Heini\pwindex{Schnitzler, Heinrich 09.08.1902 – 12.07.1982@\textsc{Schnitzler, Heinrich} (09.08.1902 – 12.07.1982), \emph{Regisseur, Schauspieler}|pw} konnte ich leider nicht sehen. Raimund\pwindex{Hofmannsthal, Raimund von 26.5.1906 – 20.03.1974@\textsc{Hofmannsthal, Raimund von} (26.5.1906 – 20.03.1974)|pw} ist jetzt bis auf \label{T_L02535_2v}\edtext{weiteres}{\lemma{\textnormal{\emph{weiteres}}}\Cendnote{\textnormal{Sie schreibt: »wieteres«}}}\label{T_L02535_2h} in Berlin\oindex{Berlin@\textbf{Berlin}|pw} bei einer Filmsache und ich glaube dass
                    es aussichtsreich ist. Ich selbst bin seit gestern in der neuen Wohnung und
                    gewöhne mich langsam. Es hat gegenüber der Stallburggasse\oindex{Stallburggasse@\textbf{Stallburggasse}|pw} viele Vorteile.\pend
           \pstart
           Ich hoffe Sie schauen sichs einmal an. Sie werden viele bekannte Dinge hier
                    vorfinden, die Sie an die Elternwohnung\pwindex{Hofmannsthal, Hugo August von 21.12.1841 – 08.12.1915@\textsc{Hofmannsthal, Hugo August von} (21.12.1841 – 08.12.1915), \emph{Bankdirektor}|pw}\pwindex{Hofmannsthal, Anna von 27.01.1849 – 22.03.1904@\textsc{Hofmannsthal, Anna von} (27.01.1849 – 22.03.1904)|pw} und an Hugo\pwindex{Hofmannsthal, Hugo von 01.02.1874 – 15.07.1929@\textsc{Hofmannsthal, Hugo von} (01.02.1874 – 15.07.1929), \emph{Schriftsteller}|pw}{ }{\pb}erinnern werden! – – – – alles das
                    ist ja so traurig!\pend
           \pstart
           Viel Herzliches{\\[\baselineskip]}Ihre{\\[\baselineskip]}\spacefill\mbox{{[}hs.:{]} Gerty}\pend
           \leftskip=0em{}\endnumbering\briefempfaengerindex{Schnitzler, Arthur@\textsc{Schnitzler, Arthur}!zzzHofmannsthal, Gertrude von@\emph{von Gertrude von Hofmannsthal}!1930-04-091@{9. 4. 1930}|)be}\mylabel{h}\end{ledgroupsized}  \newcommand{\dateiname}{L02535}\newcommand{\titel}{Gerty von Hofmannsthal an Arthur Schnitzler, 9. 4. 1930}\newcommand{\editorInnen}{Martin Anton Müller und Gerd-Hermann Susen}%% latex-leseansicht-abspann.tex
%% Abspann für die Leseansicht.
%% Der Schalter \ifkorrekturansicht ist bereits durch den Vorspann gesetzt.

%% latex-abspann.tex
%% Gemeinsamer Abspann für Korrekturansicht und Leseansicht.
%% Setzt den Schalter \ifkorrekturansicht voraus (gesetzt in den
%% einbindenden Dateien latex-korrekturansicht-abspann.tex bzw.
%% latex-leseansicht-abspann.tex).
%% ---------------------------------------------------------------

\normalsize

% Das esempio-Environment wird nur in der Leseansicht benötigt
\ifkorrekturansicht\else
\newenvironment{esempio}[3]%
{
    \vspace{1.5ex}
    \rlap{\underline{#1}}
    \par
    \setlength{\parindent}{0cm}
    \nopagebreak
    \leftskip=#2cm
    \rightskip=#3cm
}
{
    \par
}
\fi

\doendnotes{C}
\bigskip
\vfill

\clearpage

\footnotesize

\ifkorrekturansicht
  \lohead{\textsc{register}}
\fi

% theindex-Environment neu definieren ohne reledmac
\makeatletter
\renewenvironment{theindex}{%
  \ifkorrekturansicht
    \section*{\indexname}%
  \else
    \subsubsection*{Index der erwähnten Entitäten}%
  \fi
  \setlength{\parindent}{0pt}%
  \setlength{\parskip}{0pt plus 0.3pt}%
  \let\item\@idxitem
}{%
  \ifkorrekturansicht\clearpage\fi
}
\makeatother

\IfFileExists{\jobname-pw.ind}{\input{\jobname-pw.ind}}{}

% Quellenangabe nur in der Leseansicht
\ifkorrekturansicht\else
% Fallback-Definitionen, falls die .tex-Datei \titel etc. nicht gesetzt hat
\providecommand{\titel}{}
\providecommand{\editorInnen}{}
\providecommand{\dateiname}{\jobname}

\vspace{3cm}

\vfill

\footnotesize
\textsc{Quelle}: \titel. Herausgegeben von {\editorInnen}. In: \emph{Arthur Schnitzler: Briefwechsel mit Autorinnen und Autoren}.
 Digitale Edition, https://schnitzler-briefe.acdh.oeaw.ac.at/{\dateiname}.html (Stand \today)
\fi

\end{document}


      