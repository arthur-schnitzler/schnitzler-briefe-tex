%% latex-korrekturansicht-vorspann.tex
%% Vorspann für die Korrekturansicht.
%% Lädt die gemeinsame Datei latex-vorspann.tex mit gesetztem Schalter.

\newif\ifkorrekturansicht
\korrekturansichttrue

\input{../tex-inputs/latex-vorspann}


\section[Gerty Hofmannsthal an Arthur Schnitzler, 9. 4. 1930]{L02535 Gerty Hofmannsthal an Arthur Schnitzler, 9. 4. 1930}
\nopagebreak\mylabel{L02535v}
\rehead{ }\normalsize\beginnumbering\briefempfaengerindex{Schnitzler, Arthur@\textsc{Schnitzler, Arthur}!zzzHofmannsthal, Gertrude von@\emph{von Gertrude von Hofmannsthal}!1930-04-091@{9. 4. 1930}|(be}
\toendnotes[C]{\smallbreak\pagebreak[2]}\Standort{CUL, Schnitzler, B 43.}
\physDesc{Brief, 1 Blatt, 2 Seiten, 1760 Zeichen (Briefpapier mit Trauerrand)
\newline{}Schreibmaschine
\newline{}Handschrift: schwarze Tinte, deutsche Kurrent (\noindent{}Unterschrift)
\newline{}Schnitzler: mit rotem Buntstift mehrere Unterstreichungen }\toendnotes[C]{\smallbreak}
\pstart
           {\pb}IV Mozartgasse 4\hfill Wien\oindex{Wien@\textbf{Wien}, \emph{A.ADM2}|pw} d. 9/IV 30\pend
           
\pstart
           Telephon U 43384\pend
           \vspace{0.5em}
\pstart
           Lieber Arthur, ich habe heute versucht Sie anzurufen hörte aber,
               dass Sie eine andere \label{K_L02535-1v}\edtext{Geheimnummer}{\lemma{\textnormal{\emph{Geheimnummer}}}\Cendnote{\textnormal{Vgl. Arthur Schnitzler an Gerty Hofmannsthal, 17. 2. 1931.
               }}}\label{K_L02535-1} haben, wahrscheinlich sind Sie zu viel angerufen worden, darum sage ich Ihnen
               heute meine Bitte schriftlich\pend
           
\pstart
           Mein Advokat Dr Weinmann\pwindex{Weinmann, Leonhard *~1877-09-24@\textsc{Weinmann, Leonhard} (*~1877-09-24), \emph{Rechtsanwalt/Rechtsanwältin}|pw} würde so sehr eine
               Unterredung mit Ihnen wünschen, es handelt sich wegen der Erbsteuer um Bestimmung der
               Autoreneinkünfte, die man möglichst gering angeben muss, weil es als Kapital
               angesehen wird (was wirklich recht ungerecht ist, finde ich dass es doch sicher sehr
               schwankend sein wird) Ich konnte Dr W.\pwindex{Weinmann, Leonhard *~1877-09-24@\textsc{Weinmann, Leonhard} (*~1877-09-24), \emph{Rechtsanwalt/Rechtsanwältin}|pw}
               niemanden andern nennen als Sie, als bester Freund und auch als Autor, der competent
               ist seine Meinung zu sagen. Was die Opern betrifft hat Schalk\pwindex{Schalk, Franz 27.05.1863 – 03.09.1931@\textsc{Schalk, Franz} (27.05.1863 – 03.09.1931), \emph{Theaterleiter/Theaterleiterin, Dirigent/Dirigentin}|pw} eine Art Gutachten gegeben. Dr W.\pwindex{Weinmann, Leonhard *~1877-09-24@\textsc{Weinmann, Leonhard} (*~1877-09-24), \emph{Rechtsanwalt/Rechtsanwältin}|pw} wird Ihnen das alles besser erklären können als ich.
               Wollen Sie also die grosse Güte haben den Mann einmal in nächster Zeit zu \label{T_L02535-1v}\edtext{einer}{\lemma{\textnormal{\emph{einer}}}\Cendnote{\textnormal{Sie schreibt: »einen«}}}\label{T_L02535-1} Ihnen passenden
               Stunde zu empfangen? Natürlich müsste ich es einige Tage früher wissen, da der Mann\pwindex{Weinmann, Leonhard *~1877-09-24@\textsc{Weinmann, Leonhard} (*~1877-09-24), \emph{Rechtsanwalt/Rechtsanwältin}|pw} sehr beschäftigt ist und auch oft
               Verhandlungen hat. Bitte rufen Sie mich einmal zwischen 10–11 vorm an,
               wo ich fast immer zuhaus bin und lassen Sie mich ein Wort wissen.\pend
           
\pstart
           Ich war drei Wochen in Berlin\oindex{Berlin@\textbf{Berlin}, \emph{P.PPLC}|pw}, habe Olga\pwindex{Schnitzler, Olga 17.01.1882 – 13.01.1970@\textsc{Schnitzler, Olga} (17.01.1882 – 13.01.1970), \emph{Schauspieler/Schauspielerin, Sänger/Sängerin}|pw} gesehen, die ich sehr wohl fand und war
               entzückt über die Wohnung, die ich so besonders geschmackvoll fand. Heini\pwindex{Schnitzler, Heinrich 09.08.1902 – 12.07.1982@\textsc{Schnitzler, Heinrich} (09.08.1902 – 12.07.1982), \emph{Regisseur/Regisseurin, Schauspieler/Schauspielerin}|pw} konnte ich leider nicht sehen. Raimund\pwindex{Hofmannsthal, Raimund von 26.5.1906 – 20.03.1974@\textsc{Hofmannsthal, Raimund von} (26.5.1906 – 20.03.1974)|pw} ist jetzt bis auf \label{T_L02535-2v}\edtext{weiteres}{\lemma{\textnormal{\emph{weiteres}}}\Cendnote{\textnormal{Sie
                  schreibt: »wieteres«}}}\label{T_L02535-2} in Berlin\oindex{Berlin@\textbf{Berlin}, \emph{P.PPLC}|pw} bei einer Filmsache und ich glaube dass es aussichtsreich ist. Ich
               selbst bin seit gestern in der neuen Wohnung und gewöhne mich langsam. Es hat
               gegenüber der Stallburggasse\oindex{Stallburggasse@\textbf{Stallburggasse}, \emph{Straße (K.STR)}|pw} viele
               Vorteile.\pend
           
\pstart
           Ich hoffe Sie schauen sichs einmal an. Sie werden viele bekannte Dinge hier
               vorfinden, die Sie an die Elternwohnung\pwindex{Hofmannsthal, Hugo August von 21.12.1841 – 08.12.1915@\textsc{Hofmannsthal, Hugo August von} (21.12.1841 – 08.12.1915), \emph{Bankdirektor/Bankdirektorin}|pw}\pwindex{Hofmannsthal, Anna von 27.01.1849 – 22.03.1904@\textsc{Hofmannsthal, Anna von} (27.01.1849 – 22.03.1904)|pw} und an Hugo\pwindex{Hofmannsthal, Hugo von 1874-02-01 – 1929-07-15@\textsc{Hofmannsthal, Hugo von} (1874-02-01 – 1929-07-15), \emph{Schriftsteller/Schriftstellerin}|pw}{ }{\pb}erinnern werden! – – – – alles das ist
               ja so traurig!\pend
           
\pstart
           Viel Herzliches{\\[\baselineskip]}Ihre{\\[\baselineskip]}\spacefill\mbox{{[}hs.:{]} Gerty}\pend
           \leftskip=0em{}\selectlanguage{ngerman}\endnumbering\briefempfaengerindex{Schnitzler, Arthur@\textsc{Schnitzler, Arthur}!zzzHofmannsthal, Gertrude von@\emph{von Gertrude von Hofmannsthal}!1930-04-091@{9. 4. 1930}|)be}\mylabel{L02535h}  \normalsize

\doendnotes{C}
\bigskip
\vfill

\clearpage

\footnotesize

\lohead{\textsc{register}}

% Definiere theindex-Environment komplett neu ohne reledmac
\makeatletter
\renewenvironment{theindex}{%
  \section*{\indexname}%
  \setlength{\parindent}{0pt}%
  \setlength{\parskip}{0pt plus 0.3pt}%
  \let\item\@idxitem
}{%
  \clearpage
}
\makeatother

\IfFileExists{\jobname-pw.ind}{\input{\jobname-pw.ind}}{}

\end{document}

      