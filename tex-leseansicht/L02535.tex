%% latex-leseansicht-vorspann.tex
%% Vorspann für die Leseansicht.
%% Lädt die gemeinsame Datei latex-vorspann.tex mit nicht gesetztem Schalter.

\newif\ifkorrekturansicht
\korrekturansichtfalse

\input{../tex-inputs/latex-vorspann}


\section[Gerty Hofmannsthal an Arthur Schnitzler, 9. 4. 1930]{L02535 Gerty Hofmannsthal an Arthur Schnitzler, 9. 4. 1930}
\nopagebreak\mylabel{L02535v}
\rehead{ }\normalsize\beginnumbering\briefempfaengerindex{Schnitzler, Arthur@\textsc{Schnitzler, Arthur}!zzzHofmannsthal, Gertrude von@\emph{von Gertrude von Hofmannsthal}!1930-04-091@{9. 4. 1930}|(be}
\toendnotes[C]{\smallbreak\pagebreak[2]}
\correspDesc{Versand  durch Gerty Hofmannsthal am 9. 4. 1930 in Wien
\newline{}Erhalt  durch Arthur Schnitzler im Zeitraum [9. 4. 1930
                  – 13. 4. 1930?] in Wien}\toendnotes[C]{\smallbreak}
\Standort{CUL, Schnitzler, B 43.}
\physDesc{Brief, 1 Blatt, 2 Seiten, 1760 Zeichen (Briefpapier mit Trauerrand)
\newline{}Schreibmaschine
\newline{}Handschrift: schwarze Tinte, deutsche Kurrent (\noindent{}Unterschrift)
\newline{}Schnitzler: mit rotem Buntstift mehrere Unterstreichungen }\toendnotes[C]{\smallbreak}
\pstart
           {\pb}IV Mozartgasse 4\hfill Wien\oindex{Wien@\textbf{Wien}, \emph{Verwaltungsgebiet}|pw} d. 9/IV 30\pend
           
\pstart
           Telephon U 43384\pend
           \vspace{0.5em}
\pstart
           Lieber Arthur, ich habe heute versucht Sie anzurufen hörte aber,
               dass Sie eine andere \label{K_L02535-1v}\edtext{Geheimnummer}{\lemma{\textnormal{\emph{Geheimnummer}}}\Cendnote{\textnormal{Vgl. XXXX Auszeichnungsfehler: Dokument L02543 nicht gefunden.
               }}}\label{K_L02535-1} haben, wahrscheinlich sind Sie zu viel angerufen worden, darum sage ich Ihnen
               heute meine Bitte schriftlich\pend
           
\pstart
           Mein Advokat Dr Weinmann\pwindex{Weinmann, Leonhard *~24.\,9.\,1877@\textsc{Weinmann, Leonhard} (*~24.\,9.\,1877), \emph{Rechtsanwalt}|pw} würde so sehr eine
               Unterredung mit Ihnen wünschen, es handelt sich wegen der Erbsteuer um Bestimmung der
               Autoreneinkünfte, die man möglichst gering angeben muss, weil es als Kapital
               angesehen wird (was wirklich recht ungerecht ist, finde ich dass es doch sicher sehr
               schwankend sein wird) Ich konnte Dr W.\pwindex{Weinmann, Leonhard *~24.\,9.\,1877@\textsc{Weinmann, Leonhard} (*~24.\,9.\,1877), \emph{Rechtsanwalt}|pw}
               niemanden andern nennen als Sie, als bester Freund und auch als Autor, der competent
               ist seine Meinung zu sagen. Was die Opern betrifft hat Schalk\pwindex{Schalk, Franz 27.\,5.\,1863 Wien – 3.\,9.\,1931 Edlach@\textsc{Schalk, Franz} (27.\,5.\,1863 Wien – 3.\,9.\,1931 Edlach), \emph{Theaterleiter, Dirigent}|pw} eine Art Gutachten gegeben. Dr W.\pwindex{Weinmann, Leonhard *~24.\,9.\,1877@\textsc{Weinmann, Leonhard} (*~24.\,9.\,1877), \emph{Rechtsanwalt}|pw} wird Ihnen das alles besser erklären können als ich.
               Wollen Sie also die grosse Güte haben den Mann einmal in nächster Zeit zu \label{T_L02535-1v}\edtext{einer}{\lemma{\textnormal{\emph{einer}}}\Cendnote{\textnormal{Sie schreibt: »einen«}}}\label{T_L02535-1} Ihnen passenden
               Stunde zu empfangen? Natürlich müsste ich es einige Tage früher wissen, da der Mann\pwindex{Weinmann, Leonhard *~24.\,9.\,1877@\textsc{Weinmann, Leonhard} (*~24.\,9.\,1877), \emph{Rechtsanwalt}|pw} sehr beschäftigt ist und auch oft
               Verhandlungen hat. Bitte rufen Sie mich einmal zwischen 10–11 vorm an,
               wo ich fast immer zuhaus bin und lassen Sie mich ein Wort wissen.\pend
           
\pstart
           Ich war drei Wochen in Berlin\oindex{Berlin@\textbf{Berlin}, \emph{Hauptstadt}|pw}, habe Olga\pwindex{Schnitzler, Olga 17.\,1.\,1882 Wien – 13.\,1.\,1970 Lugano@\textsc{Schnitzler, Olga} (17.\,1.\,1882 Wien – 13.\,1.\,1970 Lugano), \emph{Schauspielerin, Sängerin}|pw} gesehen, die ich sehr wohl fand und war
               entzückt über die Wohnung, die ich so besonders geschmackvoll fand. Heini\pwindex{Schnitzler, Heinrich 9.\,8.\,1902 Hinterbrühl – 12.\,7.\,1982 Wien@\textsc{Schnitzler, Heinrich} (9.\,8.\,1902 Hinterbrühl – 12.\,7.\,1982 Wien), \emph{Regisseur, Schauspieler}|pw} konnte ich leider nicht sehen. Raimund\pwindex{Hofmannsthal, Raimund von 26.\,5.\,1906 Rodaun – 20.\,3.\,1974 London@\textsc{Hofmannsthal, Raimund von} (26.\,5.\,1906 Rodaun – 20.\,3.\,1974 London)|pw} ist jetzt bis auf \label{T_L02535-2v}\edtext{weiteres}{\lemma{\textnormal{\emph{weiteres}}}\Cendnote{\textnormal{Sie
                  schreibt: »wieteres«}}}\label{T_L02535-2} in Berlin\oindex{Berlin@\textbf{Berlin}, \emph{Hauptstadt}|pw} bei einer Filmsache und ich glaube dass es aussichtsreich ist. Ich
               selbst bin seit gestern in der neuen Wohnung und gewöhne mich langsam. Es hat
               gegenüber der Stallburggasse\oindex{Wien@\textbf{Wien}!I., Innere Stadt@\textbf{I., Innere Stadt}!Stallburggasse@\textbf{Stallburggasse}, \emph{Straße}|pw} viele
               Vorteile.\pend
           
\pstart
           Ich hoffe Sie schauen sichs einmal an. Sie werden viele bekannte Dinge hier
               vorfinden, die Sie an die Elternwohnung\pwindex{Hofmannsthal, Hugo August von 21.\,12.\,1841 Wien – 8.\,12.\,1915 ebd.@\textsc{Hofmannsthal, Hugo August von} (21.\,12.\,1841 Wien – 8.\,12.\,1915 ebd.), \emph{Bankdirektor}|pw}\pwindex{Hofmannsthal, Anna von 27.\,1.\,1849 Wien – 22.\,3.\,1904 Sanatorium Fürth@\textsc{Hofmannsthal, Anna von} (27.\,1.\,1849 Wien – 22.\,3.\,1904 Sanatorium Fürth)|pw} und an Hugo\pwindex{Hofmannsthal, Hugo von 1.\,2.\,1874 Wien – 15.\,7.\,1929 Rodaun@\textsc{Hofmannsthal, Hugo von} (1.\,2.\,1874 Wien – 15.\,7.\,1929 Rodaun), \emph{Schriftsteller}|pw}{ }{\pb}erinnern werden! – – – – alles das ist
               ja so traurig!\pend
           
\pstart
           Viel Herzliches{\\[\baselineskip]}Ihre{\\[\baselineskip]}\spacefill\mbox{{[}hs.:{]} Gerty}\pend
           \leftskip=0em{}\selectlanguage{ngerman}\endnumbering\briefempfaengerindex{Schnitzler, Arthur@\textsc{Schnitzler, Arthur}!zzzHofmannsthal, Gertrude von@\emph{von Gertrude von Hofmannsthal}!1930-04-091@{9. 4. 1930}|)be}\mylabel{L02535h}  \newcommand{\dateiname}{L02535}\newcommand{\titel}{Gerty Hofmannsthal an Arthur Schnitzler, 9. 4. 1930}\newcommand{\editorInnen}{Martin Anton Müller und Gerd-Hermann Susen}%% latex-leseansicht-abspann.tex
%% Abspann für die Leseansicht.
%% Der Schalter \ifkorrekturansicht ist bereits durch den Vorspann gesetzt.

%% latex-abspann.tex
%% Gemeinsamer Abspann für Korrekturansicht und Leseansicht.
%% Setzt den Schalter \ifkorrekturansicht voraus (gesetzt in den
%% einbindenden Dateien latex-korrekturansicht-abspann.tex bzw.
%% latex-leseansicht-abspann.tex).
%% ---------------------------------------------------------------

\normalsize

% Das esempio-Environment wird nur in der Leseansicht benötigt
\ifkorrekturansicht\else
\newenvironment{esempio}[3]%
{
    \vspace{1.5ex}
    \rlap{\underline{#1}}
    \par
    \setlength{\parindent}{0cm}
    \nopagebreak
    \leftskip=#2cm
    \rightskip=#3cm
}
{
    \par
}
\fi

\doendnotes{C}
\bigskip
\vfill

\clearpage

\footnotesize

\ifkorrekturansicht
  \lohead{\textsc{register}}
\fi

% theindex-Environment neu definieren ohne reledmac
\makeatletter
\renewenvironment{theindex}{%
  \ifkorrekturansicht
    \section*{\indexname}%
  \else
    \subsubsection*{Index der erwähnten Entitäten}%
  \fi
  \setlength{\parindent}{0pt}%
  \setlength{\parskip}{0pt plus 0.3pt}%
  \let\item\@idxitem
}{%
  \ifkorrekturansicht\clearpage\fi
}
\makeatother

\IfFileExists{\jobname-pw.ind}{\input{\jobname-pw.ind}}{}

% Quellenangabe nur in der Leseansicht
\ifkorrekturansicht\else
% Fallback-Definitionen, falls die .tex-Datei \titel etc. nicht gesetzt hat
\providecommand{\titel}{}
\providecommand{\editorInnen}{}
\providecommand{\dateiname}{\jobname}

\vspace{3cm}

\vfill

\footnotesize
\textsc{Quelle}: \titel. Herausgegeben von {\editorInnen}. In: \emph{Arthur Schnitzler: Briefwechsel mit Autorinnen und Autoren}.
 Digitale Edition, https://schnitzler-briefe.acdh.oeaw.ac.at/{\dateiname}.html (Stand \today)
\fi

\end{document}


