%% latex-leseansicht-vorspann.tex
%% Vorspann für die Leseansicht.
%% Lädt die gemeinsame Datei latex-vorspann.tex mit nicht gesetztem Schalter.

\newif\ifkorrekturansicht
\korrekturansichtfalse

\input{../tex-inputs/latex-vorspann}


\section[ Felix Salten an Arthur Schnitzler, 15. 10. 1902]{L03335 Felix Salten an Arthur Schnitzler,  15. 10. 1902}
\nopagebreak\mylabel{L03335v}
\rehead{ }\normalsize\beginnumbering\briefempfaengerindex{Schnitzler, Arthur@\textsc{Schnitzler, Arthur}!zzzSalten, Felix@\emph{von Felix Salten}!1902-10-153@{15. 10. 1902}|(be}
\toendnotes[C]{\smallbreak\pagebreak[2]}
\correspDesc{Versand  durch Felix Salten am 15. 10. 1902 in Wien
\newline{}Erhalt  durch Arthur Schnitzler am [16. 10. 1902] in Berlin}\toendnotes[C]{\smallbreak}
\Standort{CUL, Schnitzler, B 89, A 2.}
\physDesc{Brief, 1 Blatt, 2 Seiten, 1723 Zeichen
\newline{}Handschrift: blaue Tinte, lateinische Kurrent
\newline{}Ordnung: mit Bleistift von unbekannter Hand nummeriert: »160« }\toendnotes[C]{\smallbreak}
\pstart
           {\pb}\textcolor{gray}{\textbf{DIE}}\pend
           
\pstart
           \textcolor{gray}{\textbf{ZEIT\orgindex{Zeit@Die Zeit|pw}}}\pend
           
\pstart
           \textcolor{gray}{\textbf{\textbf{Wien\oindex{Wien@\textbf{Wien}, \emph{Verwaltungsgebiet}|pw}er Tageszeitung}}}\hfill \textcolor{gray}{\textbf{\emph{WIEN\oindex{Wien@\textbf{Wien}, \emph{Verwaltungsgebiet}|pw}}}}{ }15. Octob. 02\pend
           
\pstart
           \textcolor{gray}{\textbf{Herausgeber:}}\hfill \textcolor{gray}{\textbf{\emph{I., Wipplingerstrasse 38\oindex{Wien@\textbf{Wien}!I., Innere Stadt@\textbf{I., Innere Stadt}!Wipplingerstraße@\textbf{Wipplingerstraße}, \emph{Straße}|pw}}}}\pend
           
\pstart
           \textcolor{gray}{\textbf{\textbf{Prof. Dr. I. Singer\pwindex{Singer, Isidor 16.\,1.\,1857 Budapest – 8.\,12.\,1927 Wien@\textsc{Singer, Isidor} (16.\,1.\,1857 Budapest – 8.\,12.\,1927 Wien), \emph{Journalist, Herausgeber, Soziologe}|pw}}}}\pend
           
\pstart
           \textcolor{gray}{\textbf{\textbf{Dr. Heinrich Kanner\pwindex{Kanner, Heinrich 9.\,11.\,1864 Galați – 15.\,2.\,1930 Wien@\textsc{Kanner, Heinrich} (9.\,11.\,1864 Galați – 15.\,2.\,1930 Wien), \emph{Herausgeber, Publizist}|pw}}}}\pend
           
\pstart
           \textcolor{gray}{\textbf{\textbf{Redaction.}}}\pend
           
\pstart
           \textcolor{gray}{\textbf{Telegramm-Adresse: \so{Zeit}\orgindex{Zeit@Die Zeit|pw}\so{,}{ }\so{Wien}\oindex{Wien@\textbf{Wien}, \emph{Verwaltungsgebiet}|pw}}}\pend
           
\pstart
           \textcolor{gray}{\textbf{Interurbanes Telephon Nr. 15.988}}\pend
           
\pstart
           \textcolor{gray}{\textbf{= Telephone Nr. 17.040, 17.041 =}}\pend
           \vspace{0.5em}
\pstart
           Lieber Freund, ich habe sehr bedauert, dass mich die Satzcorrectur
               zum »\label{K_L03335-1v}\edtext{Fünfkreuzertanz\pwindex{Salten, Felix 6.\,9.\,1869 Budapest – 8.\,10.\,1945 Zürich@\textsc{Salten, Felix} (6.\,9.\,1869 Budapest – 8.\,10.\,1945 Zürich), \emph{Schriftsteller, Journalist, Chefredakteur}!Fünfkreuzertanz@\strich\emph{Fünfkreuzertanz}|pw}}{\lemma{\textnormal{\emph{Fünfkreuzertanz}}}\Cendnote{\textnormal{Felix Salten\pwindex{Salten, Felix 6.\,9.\,1869 Budapest – 8.\,10.\,1945 Zürich@\textsc{Salten, Felix} (6.\,9.\,1869 Budapest – 8.\,10.\,1945 Zürich), \emph{Schriftsteller, Journalist, Chefredakteur}|pwk}: \emph{Fünfkreuzertanz}\pwindex{Salten, Felix 6.\,9.\,1869 Budapest – 8.\,10.\,1945 Zürich@\textsc{Salten, Felix} (6.\,9.\,1869 Budapest – 8.\,10.\,1945 Zürich), \emph{Schriftsteller, Journalist, Chefredakteur}!Fünfkreuzertanz@\strich\emph{Fünfkreuzertanz}|pwk}. In: \emph{Die Zeit}\pwindex{Zeit@\emph{Die Zeit}|pwk}, Jg. 1, Nr. 16, 12. 10. 1902,
                     Morgenblatt, S. 2–3.}}}\label{K_L03335-1}« Samstag bis
                  2 Uhr in der Redaction\orgindex{Zeit@Die Zeit|pwv}\oindex{Wien@\textbf{Wien}!I., Innere Stadt@\textbf{I., Innere Stadt}!Wipplingerstraße@\textbf{Wipplingerstraße}, \emph{Straße}|pwv} aufhielt, so dass ich Sie nicht mehr sehen konnte. Ich bitte Sie nun um einige
               Kleinigkeiten, die Sie gelegentlich, ohne Mühe ausrichten, und für die ich Ihnen sehr
               dankbar wäre. Erstens Herrn D\textsuperscript{r}{ }\label{K_L03335-2v}\edtext{Löwenfeld\pwindex{Löwenfeld, Raphael 11.\,2.\,1854 Poznan – 28.\,12.\,1910 Berlin@\textsc{Löwenfeld, Raphael} (11.\,2.\,1854 Poznan – 28.\,12.\,1910 Berlin), \emph{Theaterleiter}|pw} bestens von mir zu grüßen}{\lemma{\textnormal{\emph{Löwenfeld … grüßen}}}\Cendnote{\textnormal{Schnitzler sah Raphael Löwenfeld\pwindex{Löwenfeld, Raphael 11.\,2.\,1854 Poznan – 28.\,12.\,1910 Berlin@\textsc{Löwenfeld, Raphael} (11.\,2.\,1854 Poznan – 28.\,12.\,1910 Berlin), \emph{Theaterleiter}|pwk} am 15. 10. 1902 und am 17. 10. 1902.}}}\label{K_L03335-2},
               und ihm zu sagen, dass ich seinen \label{K_L03335-3v}\edtext{Aufsatz\pwindex{Löwenfeld, Raphael 11.\,2.\,1854 Poznan – 28.\,12.\,1910 Berlin@\textsc{Löwenfeld, Raphael} (11.\,2.\,1854 Poznan – 28.\,12.\,1910 Berlin), \emph{Theaterleiter}!Aufsatz über volkstümliche Klassikervorstellungen]@\strich\emph{[Aufsatz über volkstümliche Klassikervorstellungen]}|pwv} über volksthümliche
               Claßikervorstellungen schon sehnlichst erwarte}{\lemma{\textnormal{\emph{Aufsatz … erwarte}}}\Cendnote{\textnormal{nicht nachgewiesen}}}\label{K_L03335-3}. Dann erkundigen Sie sich, bitte,
               nach dem Schauspieler Paul Paschen\pwindex{Paschen, Paul @\textsc{Paschen, Paul}, \emph{Schauspieler, Filmschauspieler, Stimmbildner}|pw} (Schillertheater\orgindex{Schiller-Theater@Schiller-Theater|pw}) was das für ein Mensch ist. Ich
               habe durch \label{K_L03335-4v}\edtext{Geh. Rt.}{\lemma{\textnormal{\emph{Geh. Rt.}}}\Cendnote{\textnormal{Geheimrat}}}\label{K_L03335-4}{ }Forster\pwindex{Forster @\textsc{Forster}|pw} einen \label{K_L03335-5v}\edtext{Artikel\pwindex{Paschen, Paul @\textsc{Paschen, Paul}, \emph{Schauspieler, Filmschauspieler, Stimmbildner}!Artikel über Kulissenton]@\strich\emph{[Artikel über Kulissenton]}|pwv} von ihm bekommen über
               die Schweinerei des Coulissentones}{\lemma{\textnormal{\emph{Artikel … Coulissentones}}}\Cendnote{\textnormal{nicht
                  nachgewiesen}}}\label{K_L03335-5}. Zuletzt noch – wenn \label{K_L03335-6v}\edtext{bei Fischer\orgindex{S. Fischer Verlag@S. Fischer Verlag|pw} eine
               endgültige Entscheidung}{\lemma{\textnormal{\emph{bei … Entscheidung}}}\Cendnote{\textnormal{Bezug auf die
                  Veröffentlichung von Saltens\pwindex{Salten, Felix 6.\,9.\,1869 Budapest – 8.\,10.\,1945 Zürich@\textsc{Salten, Felix} (6.\,9.\,1869 Budapest – 8.\,10.\,1945 Zürich), \emph{Schriftsteller, Journalist, Chefredakteur}|pwk}{ }\emph{Die kleine Veronika}\pwindex{Salten, Felix 6.\,9.\,1869 Budapest – 8.\,10.\,1945 Zürich@\textsc{Salten, Felix} (6.\,9.\,1869 Budapest – 8.\,10.\,1945 Zürich), \emph{Schriftsteller, Journalist, Chefredakteur}!kleine Veronika@\strich\emph{Die kleine Veronika}|pwk} bei \emph{S. Fischer}\orgindex{S. Fischer Verlag@S. Fischer Verlag|pwk}, siehe A. S.: \emph{Tagebuch}, 15. 10. 1902 und XXXX Auszeichnungsfehler: Dokument L02979 nicht gefunden. }}}\label{K_L03335-6} getroffen ist, depeschiren Sie mir, bitte. Ich bin
               sehr neugierig, wie Sie sich leicht denken können. Ich muß nun den {\pb}»Moloch\pwindex{Wassermann, Jakob 10.\,3.\,1873 Fürth – 1.\,1.\,1934 Altaussee@\textsc{Wassermann, Jakob} (10.\,3.\,1873 Fürth – 1.\,1.\,1934 Altaussee), \emph{Schriftsteller}!Moloch@\strich\emph{Der Moloch}|pw}« trotzdem ich ihn das erste Mal refüsirt habe, \label{K_L03335-7v}\edtext{besprechen\pwindex{Zeit@\emph{Die Zeit}|pwv}}{\lemma{\textnormal{\emph{besprechen}}}\Cendnote{\textnormal{Felix Salten\pwindex{Salten, Felix 6.\,9.\,1869 Budapest – 8.\,10.\,1945 Zürich@\textsc{Salten, Felix} (6.\,9.\,1869 Budapest – 8.\,10.\,1945 Zürich), \emph{Schriftsteller, Journalist, Chefredakteur}|pwk}: \emph{Ein Gesellschaftsroman}\pwindex{Zeit@\emph{Die Zeit}|pwk}. In: \emph{Die Zeit}\pwindex{Zeit@\emph{Die Zeit}|pwk}, Jg. 1, Nr. 81, 19. 12. 1902, Morgenblatt, S. 1–2.}}}\label{K_L03335-7}. Hugo Ganz\pwindex{Ganz, Hugo 24.\,4.\,1862 Mainz – 2.\,1.\,1922 Wien@\textsc{Ganz, Hugo} (24.\,4.\,1862 Mainz – 2.\,1.\,1922 Wien), \emph{Schriftsteller, Journalist}|pw} hätte ihn übel zugerichtet, und bat mich schließlich
               darum, weil er \label{K_L03335-8v}\edtext{Herzl\pwindex{Herzl, Theodor 2.\,5.\,1860 Budapest – 3.\,7.\,1904 Edlach@\textsc{Herzl, Theodor} (2.\,5.\,1860 Budapest – 3.\,7.\,1904 Edlach), \emph{Schriftsteller, Journalist}|pw}’s \substVorne{}\textsuperscript{r}\substDazwischen{}R\substHinten{}oman »Altneuland\pwindex{Herzl, Theodor 2.\,5.\,1860 Budapest – 3.\,7.\,1904 Edlach@\textsc{Herzl, Theodor} (2.\,5.\,1860 Budapest – 3.\,7.\,1904 Edlach), \emph{Schriftsteller, Journalist}!Altneuland. Roman@\strich\emph{Altneuland. Roman}|pw}« übernommen}{\lemma{\textnormal{\emph{Herzl’s … übernommen}}}\Cendnote{\textnormal{Lector\pwindex{Ganz, Hugo 24.\,4.\,1862 Mainz – 2.\,1.\,1922 Wien@\textsc{Ganz, Hugo} (24.\,4.\,1862 Mainz – 2.\,1.\,1922 Wien), \emph{Schriftsteller, Journalist}|pwk} [ = Hugo Ganz\pwindex{Ganz, Hugo 24.\,4.\,1862 Mainz – 2.\,1.\,1922 Wien@\textsc{Ganz, Hugo} (24.\,4.\,1862 Mainz – 2.\,1.\,1922 Wien), \emph{Schriftsteller, Journalist}|pwk}]: \emph{»Altneuland«}\pwindex{Ganz, Hugo 24.\,4.\,1862 Mainz – 2.\,1.\,1922 Wien@\textsc{Ganz, Hugo} (24.\,4.\,1862 Mainz – 2.\,1.\,1922 Wien), \emph{Schriftsteller, Journalist}!Altneuland«@\strich\emph{»Altneuland«}|pwk}. In: \emph{Die Zeit}\pwindex{Zeit@\emph{Die Zeit}|pwk},
                     Jg. 1, Nr. 39, 5. 11. 1902, Morgenblatt,
                     S. 1–2.}}}\label{K_L03335-8} hat. Ich habe aufmerksam gemacht, dass ich das Buch\pwindex{Herzl, Theodor 2.\,5.\,1860 Budapest – 3.\,7.\,1904 Edlach@\textsc{Herzl, Theodor} (2.\,5.\,1860 Budapest – 3.\,7.\,1904 Edlach), \emph{Schriftsteller, Journalist}!Altneuland. Roman@\strich\emph{Altneuland. Roman}|pwv} nicht loben kann, und da
               man daran keinen Anstoß nahm, habe ich weiter keine Ursache, mit meiner ganzen
               Meinung über W.\pwindex{Wassermann, Jakob 10.\,3.\,1873 Fürth – 1.\,1.\,1934 Altaussee@\textsc{Wassermann, Jakob} (10.\,3.\,1873 Fürth – 1.\,1.\,1934 Altaussee), \emph{Schriftsteller}|pw} zurückzuhalten. Bei alledem
               hat W.\pwindex{Wassermann, Jakob 10.\,3.\,1873 Fürth – 1.\,1.\,1934 Altaussee@\textsc{Wassermann, Jakob} (10.\,3.\,1873 Fürth – 1.\,1.\,1934 Altaussee), \emph{Schriftsteller}|pw} noch Glück. Erstens ist er aus Ganz\pwindex{Ganz, Hugo 24.\,4.\,1862 Mainz – 2.\,1.\,1922 Wien@\textsc{Ganz, Hugo} (24.\,4.\,1862 Mainz – 2.\,1.\,1922 Wien), \emph{Schriftsteller, Journalist}|pw}’ Händen entwischt, zweitens nützt ihm die
               Raserei Trebitsch\pwindex{Trebitsch, Siegfried 22.\,12.\,1868 Wien – 3.\,6.\,1956 Zürich@\textsc{Trebitsch, Siegfried} (22.\,12.\,1868 Wien – 3.\,6.\,1956 Zürich), \emph{Schriftsteller, Übersetzer}|pw}’s bei mir, der schon glaubt,
               der Tag der nächsten Woche, an welchem mein Moloch-F.\pwindex{Zeit@\emph{Die Zeit}|pwv} erscheint, sei der Tag des Herrn Trebitsch\pwindex{Trebitsch, Siegfried 22.\,12.\,1868 Wien – 3.\,6.\,1956 Zürich@\textsc{Trebitsch, Siegfried} (22.\,12.\,1868 Wien – 3.\,6.\,1956 Zürich), \emph{Schriftsteller, Übersetzer}|pw}.\pend
           
\pstart
           \label{K_L03335-9v}\edtext{Gettke\pwindex{Gettke, Ernst 8.\,10.\,1841 Berlin – 4.\,12.\,1912 ebd.@\textsc{Gettke, Ernst} (8.\,10.\,1841 Berlin – 4.\,12.\,1912 ebd.), \emph{Schriftsteller, Theaterleiter, Regisseur}|pw} ist seit c\textsuperscript{a} 14 Tagen im Besitz Ihres Vertrages}{\lemma{\textnormal{\emph{Gettke … Vertrages}}}\Cendnote{\textnormal{Siehe A. S.: \emph{Tagebuch}, 29. 10. 1902. }}}\label{K_L03335-9}. Ich
               besuche ihn heute, und mache ihm von der inzwischen
               eingetretenen \label{K_L03335-10v}\edtext{Änderung der Dinge}{\lemma{\textnormal{\emph{Änderung der Dinge}}}\Cendnote{\textnormal{Bezug auf eine mögliche Aufführung von \emph{Liebelei}\pwindex{Schnitzler, Arthur 15.\,5.\,1862 Wien – 21.\,10.\,1931 ebd.@\textsc{Schnitzler, Arthur} (15.\,5.\,1862 Wien – 21.\,10.\,1931 ebd.), \emph{Schriftsteller, Mediziner}!Liebelei. Schauspiel in drei Akten@\strich\emph{Liebelei. Schauspiel in drei Akten}|pwk}, für die das \emph{Burgtheater}\orgindex{Burgtheater@Burgtheater|pwk} noch das ausschließliche Aufführungsrecht
                  hatte, vgl. XXXX Auszeichnungsfehler: Dokument L02979 nicht gefunden. Die Premiere
                  am \emph{Raimundtheater}\orgindex{Raimund-Theater@Raimund-Theater|pwk} fand am 7. 3. 1903
                  statt.}}}\label{K_L03335-10} Mittheilung. Das schiebt allerdings die Premiere im R. Th\oindex{Wien@\textbf{Wien}!VI., Mariahilf@\textbf{VI., Mariahilf}!Raimund-Theater@\textbf{Raimund-Theater}, \emph{Theater}|pw}. ein wenig hinaus!\pend
           
\pstart
           Hoffentlich schreiben Sie mir bald!\pend
           
\pstart
           Herzlichst Ihr {\\[\baselineskip]}\spacefill\mbox{Salten}\pend
           \leftskip=0em{}\selectlanguage{ngerman}\endnumbering\briefempfaengerindex{Schnitzler, Arthur@\textsc{Schnitzler, Arthur}!zzzSalten, Felix@\emph{von Felix Salten}!1902-10-153@{15. 10. 1902}|)be}\mylabel{L03335h}  \newcommand{\dateiname}{L03335}\newcommand{\titel}{Felix Salten an Arthur Schnitzler, 15. 10. 1902}\newcommand{\editorInnen}{Martin Anton Müller und Laura Untner}%% latex-leseansicht-abspann.tex
%% Abspann für die Leseansicht.
%% Der Schalter \ifkorrekturansicht ist bereits durch den Vorspann gesetzt.

%% latex-abspann.tex
%% Gemeinsamer Abspann für Korrekturansicht und Leseansicht.
%% Setzt den Schalter \ifkorrekturansicht voraus (gesetzt in den
%% einbindenden Dateien latex-korrekturansicht-abspann.tex bzw.
%% latex-leseansicht-abspann.tex).
%% ---------------------------------------------------------------

\normalsize

% Das esempio-Environment wird nur in der Leseansicht benötigt
\ifkorrekturansicht\else
\newenvironment{esempio}[3]%
{
    \vspace{1.5ex}
    \rlap{\underline{#1}}
    \par
    \setlength{\parindent}{0cm}
    \nopagebreak
    \leftskip=#2cm
    \rightskip=#3cm
}
{
    \par
}
\fi

\doendnotes{C}
\bigskip
\vfill

\clearpage

\footnotesize

\ifkorrekturansicht
  \lohead{\textsc{register}}
\fi

% theindex-Environment neu definieren ohne reledmac
\makeatletter
\renewenvironment{theindex}{%
  \ifkorrekturansicht
    \section*{\indexname}%
  \else
    \subsubsection*{Index der erwähnten Entitäten}%
  \fi
  \setlength{\parindent}{0pt}%
  \setlength{\parskip}{0pt plus 0.3pt}%
  \let\item\@idxitem
}{%
  \ifkorrekturansicht\clearpage\fi
}
\makeatother

\IfFileExists{\jobname-pw.ind}{\input{\jobname-pw.ind}}{}

% Quellenangabe nur in der Leseansicht
\ifkorrekturansicht\else
% Fallback-Definitionen, falls die .tex-Datei \titel etc. nicht gesetzt hat
\providecommand{\titel}{}
\providecommand{\editorInnen}{}
\providecommand{\dateiname}{\jobname}

\vspace{3cm}

\vfill

\footnotesize
\textsc{Quelle}: \titel. Herausgegeben von {\editorInnen}. In: \emph{Arthur Schnitzler: Briefwechsel mit Autorinnen und Autoren}.
 Digitale Edition, https://schnitzler-briefe.acdh.oeaw.ac.at/{\dateiname}.html (Stand \today)
\fi

\end{document}


