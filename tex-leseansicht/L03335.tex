%% latex-leseansicht-vorspann.tex
%% Vorspann für die Leseansicht.
%% Lädt die gemeinsame Datei latex-vorspann.tex mit nicht gesetztem Schalter.

\newif\ifkorrekturansicht
\korrekturansichtfalse

\input{../tex-inputs/latex-vorspann}


         
         \renewcommand{\erwaehntePersonen}{Personen:  Forster, Hugo Ganz, Ernst Gettke, Theodor Herzl, Heinrich Kanner, Raphael Löwenfeld, Paul Paschen, Felix Salten, Isidor Singer, Siegfried Trebitsch, Jakob Wassermann}
         \renewcommand{\erwaehnteInstitutionen}{Institutionen: Burgtheater, Die Zeit, S. Fischer Verlag, Schiller-Theater}
         \renewcommand{\erwaehnteOrte}{Orte: Berlin, Raimund-Theater, Wien, Wipplingerstraße}
         \renewcommand{\erwaehnteWerke}{Werke: Altneuland. Roman, Der Moloch, Die Zeit, Die kleine Veronika, Fünfkreuzertanz, Liebelei. Schauspiel in drei Akten, [Artikel über Kulissenton], [Aufsatz über volkstümliche Klassikervorstellungen], »Altneuland«}
               \section[ Felix Salten an Arthur Schnitzler, 15. 10. 1902]{ Felix Salten an Arthur Schnitzler, 15. 10. 1902}\nopagebreak\mylabel{v}\rehead{ }\begin{ledgroupsized}[t]{13cm}\normalsize\beginnumbering\briefempfaengerindex{Schnitzler, Arthur@\textsc{Schnitzler, Arthur}!zzzSalten, Felix@\emph{von Felix Salten}!1902-10-153@{15. 10. 1902}|(be} \toendnotes[C]{\smallbreak\pagebreak[2]} \Standort{CUL, Schnitzler, B 89, A 2.}
\physDesc{Brief, 1 Blatt, 2 Seiten, 1723 Zeichen
\newline{}Handschrift: blaue Tinte, lateinische Kurrent
\newline{}Ordnung: mit Bleistift von unbekannter Hand nummeriert: »160« }\toendnotes[C]{\smallbreak}\pstart
           \noindent{}{\pb}\textcolor{gray}{\textbf{DIE}}\pend
           \pstart
           \textcolor{gray}{\textbf{ZEIT\orgindex{Zeit@Die Zeit|pw}}}\pend
           \pstart
           \textcolor{gray}{\textbf{\textbf{Wien\oindex{Wien@\textbf{Wien}|pw}er Tageszeitung}}}\hfill \textcolor{gray}{\textbf{\emph{WIEN\oindex{Wien@\textbf{Wien}|pw}}}}{ }15. Octob. 02\pend
           \pstart
           \textcolor{gray}{\textbf{Herausgeber:}}\hfill \textcolor{gray}{\textbf{\emph{I., Wipplingerstrasse 38\oindex{Wipplingerstrasse@\textbf{Wipplingerstraße}|pw}}}}\pend
           \pstart
           \textcolor{gray}{\textbf{\textbf{Prof. Dr. I. Singer\pwindex{Singer, Isidor 16.01.1857 – 08.12.1927@\textsc{Singer, Isidor} (16.01.1857 – 08.12.1927), \emph{Journalist, Herausgeber, Soziologe}|pw}}}}\pend
           \pstart
           \textcolor{gray}{\textbf{\textbf{Dr. Heinrich Kanner\pwindex{Kanner, Heinrich 09.11.1864 – 15.02.1930@\textsc{Kanner, Heinrich} (09.11.1864 – 15.02.1930), \emph{Herausgeber, Publizist}|pw}}}}\pend
           \pstart
           \textcolor{gray}{\textbf{\textbf{Redaction.}}}\pend
           \pstart
           \textcolor{gray}{\textbf{Telegramm-Adresse: \so{Zeit}\orgindex{Zeit@Die Zeit|pw}\so{,}{ }\so{Wien}\oindex{Wien@\textbf{Wien}|pw}}}\pend
           \pstart
           \textcolor{gray}{\textbf{Interurbanes Telephon Nr. 15.988}}\pend
           \pstart
           \textcolor{gray}{\textbf{= Telephone Nr. 17.040, 17.041 =}}\pend
           \pstart
           Lieber Freund, ich habe sehr bedauert, dass mich die
               Satzcorrectur zum »\label{K_L03335-1v}\edtext{Fünfkreuzertanz\pwindex{Salten, Felix 06.09.1869 – 08.10.1945@\textsc{Salten, Felix} (06.09.1869 – 08.10.1945), \emph{Schriftsteller, Journalist}!Fuenfkreuzertanz1902-10-12@\strich\emph{Fünfkreuzertanz} {[}1902-10-12{]}|pw}}{\lemma{\textnormal{\emph{Fünfkreuzertanz}}}\Cendnote{\textnormal{Felix Salten\pwindex{Salten, Felix 06.09.1869 – 08.10.1945@\textsc{Salten, Felix} (06.09.1869 – 08.10.1945), \emph{Schriftsteller, Journalist}|pwk}: \emph{Fünfkreuzertanz}\pwindex{Salten, Felix 06.09.1869 – 08.10.1945@\textsc{Salten, Felix} (06.09.1869 – 08.10.1945), \emph{Schriftsteller, Journalist}!Fuenfkreuzertanz1902-10-12@\strich\emph{Fünfkreuzertanz} {[}1902-10-12{]}|pwk}. In: \emph{Die Zeit}\pwindex{Zeit1902-09-27 – 1919@\emph{Die Zeit} {[}1902-09-27 – 1919{]}|pwk}, Jg. 1, Nr. 16, 12. 10. 1902,
                     Morgenblatt, S. 2–3.}}}\label{K_L03335-1h}« Samstag bis
                  2 Uhr in der Redaction\orgindex{Zeit@Die Zeit|pwv}\oindex{Wipplingerstrasse@\textbf{Wipplingerstraße}|pwv} aufhielt, so dass ich Sie nicht mehr sehen konnte. Ich bitte Sie nun um einige
               Kleinigkeiten, die Sie gelegentlich, ohne Mühe ausrichten, und für die ich Ihnen sehr
               dankbar wäre. Erstens Herrn D\textsuperscript{r}{ }\label{K_L03335-2v}\edtext{Löwenfeld\pwindex{Loewenfeld, Raphael 11.02.1854 – 28.12.1910@\textsc{Löwenfeld, Raphael} (11.02.1854 – 28.12.1910), \emph{Theaterleiter}|pw} bestens von mir zu grüßen}{\lemma{\textnormal{\emph{Löwenfeld … grüßen}}}\Cendnote{\textnormal{Schnitzler\pwindex{Schnitzler, Arthur 15.05.1862 – 21.10.1931@\textsc{Schnitzler, Arthur} (15.05.1862 – 21.10.1931), \emph{Schriftsteller, Mediziner}|pwk} sah Raphael Löwenfeld\pwindex{Loewenfeld, Raphael 11.02.1854 – 28.12.1910@\textsc{Löwenfeld, Raphael} (11.02.1854 – 28.12.1910), \emph{Theaterleiter}|pwk} am 15. 10. 1902 und am 17. 10. 1902.}}}\label{K_L03335-2h},
               und ihm zu sagen, dass ich seinen \label{K_L03335-3v}\edtext{Aufsatz\pwindex{Loewenfeld, Raphael 11.02.1854 – 28.12.1910@\textsc{Löwenfeld, Raphael} (11.02.1854 – 28.12.1910), \emph{Theaterleiter}!Aufsatz ueber volkstuemliche Klassikervorstellungen]1902@\strich\emph{[Aufsatz über volkstümliche Klassikervorstellungen]} {[}1902{]}|pwv} über volksthümliche
               Claßikervorstellungen schon sehnlichst erwarte}{\lemma{\textnormal{\emph{Aufsatz … erwarte}}}\Cendnote{\textnormal{nicht nachgewiesen}}}\label{K_L03335-3h}. Dann erkundigen Sie sich, bitte,
               nach dem Schauspieler Paul Paschen\pwindex{Paschen, Paul @\textsc{Paschen, Paul}, \emph{Schauspieler, Filmschauspieler, Stimmbildner}|pw} (Schillertheater\orgindex{Schiller-Theater@Schiller-Theater|pw}) was das für ein Mensch ist. Ich
               habe durch \label{K_L03335-4v}\edtext{Geh. Rt.}{\lemma{\textnormal{\emph{Geh. Rt.}}}\Cendnote{\textnormal{Geheimrat}}}\label{K_L03335-4h}{ }Forster\pwindex{Forster @\textsc{Forster}|pw} einen \label{K_L03335-5v}\edtext{Artikel\pwindex{Paschen, Paul @\textsc{Paschen, Paul}, \emph{Schauspieler, Filmschauspieler, Stimmbildner}!Artikel ueber Kulissenton]1902@\strich\emph{[Artikel über Kulissenton]} {[}1902{]}|pwv} von ihm bekommen über
               die Schweinerei des Coulissentones}{\lemma{\textnormal{\emph{Artikel … Coulissentones}}}\Cendnote{\textnormal{nicht
                  nachgewiesen}}}\label{K_L03335-5h}. Zuletzt noch – wenn \label{K_L03335-6v}\edtext{bei Fischer\orgindex{S. Fischer Verlag@S. Fischer Verlag|pw} eine
               endgültige Entscheidung}{\lemma{\textnormal{\emph{bei … Entscheidung}}}\Cendnote{\textnormal{Bezug auf die
                  Veröffentlichung von Salten\pwindex{Salten, Felix 06.09.1869 – 08.10.1945@\textsc{Salten, Felix} (06.09.1869 – 08.10.1945), \emph{Schriftsteller, Journalist}|pwk}s \emph{Die kleine Veronika}\pwindex{Salten, Felix 06.09.1869 – 08.10.1945@\textsc{Salten, Felix} (06.09.1869 – 08.10.1945), \emph{Schriftsteller, Journalist}!kleine Veronika1902-12-01@\strich\emph{Die kleine Veronika} {[}1902-12-01{]}|pwk} bei \emph{S.
                     Fischer}\orgindex{S. Fischer Verlag@S. Fischer Verlag|pwk}, siehe A. S.: \emph{Tagebuch}, 15. 10. 1902 und Arthur Schnitzler an Felix Salten, 16. 10. 1902}}}\label{K_L03335-6h} getroffen ist, depeschiren Sie mir, bitte. Ich bin sehr neugierig, wie Sie
               sich leicht denken können. Ich muß nun den {\pb}»Moloch\pwindex{Wassermann, Jakob 10.03.1873 – 01.01.1934@\textsc{Wassermann, Jakob} (10.03.1873 – 01.01.1934), \emph{Schriftsteller}!Moloch1902-12@\strich\emph{Der Moloch} {[}1902-12{]}|pw}« trotzdem ich ihn das erste Mal refüsirt habe, \label{K_L03335-7v}\edtext{besprechen\pwindex{Zeit1902-09-27 – 1919@\emph{Die Zeit} {[}1902-09-27 – 1919{]}|pwv}}{\lemma{\textnormal{\emph{besprechen}}}\Cendnote{\textnormal{Felix Salten\pwindex{Salten, Felix 06.09.1869 – 08.10.1945@\textsc{Salten, Felix} (06.09.1869 – 08.10.1945), \emph{Schriftsteller, Journalist}|pwk}: \emph{Ein Gesellschaftsroman}\pwindex{Zeit1902-09-27 – 1919@\emph{Die Zeit} {[}1902-09-27 – 1919{]}|pwk}. In: \emph{Die Zeit}\pwindex{Zeit1902-09-27 – 1919@\emph{Die Zeit} {[}1902-09-27 – 1919{]}|pwk}, Jg. 1, Nr. 81, 19. 12. 1902, Morgenblatt, S. 1–2.}}}\label{K_L03335-7h}. Hugo Ganz\pwindex{Ganz, Hugo 24.04.1862 – 02.01.1922@\textsc{Ganz, Hugo} (24.04.1862 – 02.01.1922), \emph{Schriftsteller, Journalist}|pw} hätte ihn übel zugerichtet, und bat mich schließlich
               darum, weil er \label{K_L03335-8v}\edtext{Herzl\pwindex{Herzl, Theodor 1860-05-02 – 1904-07-03@\textsc{Herzl, Theodor} (1860-05-02 – 1904-07-03), \emph{Schriftsteller, Journalist}|pw}’s \substVorne{}\textsuperscript{r}\substDazwischen{}R\substHinten{}oman »Altneuland\pwindex{Herzl, Theodor 1860-05-02 – 1904-07-03@\textsc{Herzl, Theodor} (1860-05-02 – 1904-07-03), \emph{Schriftsteller, Journalist}!Altneuland. Roman1902@\strich\emph{Altneuland. Roman} {[}1902{]}|pw}« übernommen}{\lemma{\textnormal{\emph{Herzl’s … übernommen}}}\Cendnote{\textnormal{Lector\pwindex{Ganz, Hugo 24.04.1862 – 02.01.1922@\textsc{Ganz, Hugo} (24.04.1862 – 02.01.1922), \emph{Schriftsteller, Journalist}|pwk} [ = Hugo Ganz\pwindex{Ganz, Hugo 24.04.1862 – 02.01.1922@\textsc{Ganz, Hugo} (24.04.1862 – 02.01.1922), \emph{Schriftsteller, Journalist}|pwk}]: \emph{»Altneuland«}\pwindex{Altneuland«1902-11-05@\emph{»Altneuland«} {[}1902-11-05{]}|pwk}. In: \emph{Die Zeit}\pwindex{Zeit1902-09-27 – 1919@\emph{Die Zeit} {[}1902-09-27 – 1919{]}|pwk},
                     Jg. 1, Nr. 39, 5. 11. 1902, Morgenblatt,
                     S. 1–2.}}}\label{K_L03335-8h} hat. Ich habe aufmerksam gemacht, dass ich das Buch\pwindex{Herzl, Theodor 1860-05-02 – 1904-07-03@\textsc{Herzl, Theodor} (1860-05-02 – 1904-07-03), \emph{Schriftsteller, Journalist}!Altneuland. Roman1902@\strich\emph{Altneuland. Roman} {[}1902{]}|pwv} nicht loben kann, und da
               man daran keinen Anstoß nahm, habe ich weiter keine Ursache, mit meiner ganzen
               Meinung über 
               W.\pwindex{Wassermann, Jakob 10.03.1873 – 01.01.1934@\textsc{Wassermann, Jakob} (10.03.1873 – 01.01.1934), \emph{Schriftsteller}|pw}
                zurückzuhalten. Bei alledem hat W.\pwindex{Wassermann, Jakob 10.03.1873 – 01.01.1934@\textsc{Wassermann, Jakob} (10.03.1873 – 01.01.1934), \emph{Schriftsteller}|pw}
               noch Glück. Erstens ist er aus Ganz\pwindex{Ganz, Hugo 24.04.1862 – 02.01.1922@\textsc{Ganz, Hugo} (24.04.1862 – 02.01.1922), \emph{Schriftsteller, Journalist}|pw}’ Händen
               entwischt, zweitens nützt ihm die Raserei Trebitsch\pwindex{Trebitsch, Siegfried 22.12.1868 – 03.06.1956@\textsc{Trebitsch, Siegfried} (22.12.1868 – 03.06.1956), \emph{Schriftsteller, Übersetzer}|pw}’s bei mir, der schon glaubt, der Tag der nächsten Woche, an
               welchem mein Moloch-F.\pwindex{Zeit1902-09-27 – 1919@\emph{Die Zeit} {[}1902-09-27 – 1919{]}|pwv}
               erscheint, sei der Tag des Herrn Trebitsch\pwindex{Trebitsch, Siegfried 22.12.1868 – 03.06.1956@\textsc{Trebitsch, Siegfried} (22.12.1868 – 03.06.1956), \emph{Schriftsteller, Übersetzer}|pw}.\pend
           \pstart
           \label{K_L03335-10v}\edtext{Gettke\pwindex{Gettke, Ernst 08.10.1841 – 04.12.1912@\textsc{Gettke, Ernst} (08.10.1841 – 04.12.1912), \emph{Schriftsteller, Theaterleiter, Regisseur}|pw} ist seit c\textsuperscript{a} 14 Tagen
               im Besitz Ihres Vertrages}{\lemma{\textnormal{\emph{Gettke … Vertrages}}}\Cendnote{\textnormal{siehe A. S.: \emph{Tagebuch}, 29. 10. 1902}}}\label{K_L03335-10h}. Ich besuche ihn heute, und mache ihm von der
               inzwischen eingetretenen \label{K_L03335-11v}\edtext{Änderung der
                  Dinge}{\lemma{\textnormal{\emph{Änderung der
                  Dinge}}}\Cendnote{\textnormal{Bezug auf eine mögliche
                  Aufführung der \emph{Liebelei}\pwindex{Schnitzler, Arthur 15.05.1862 – 21.10.1931@\textsc{Schnitzler, Arthur} (15.05.1862 – 21.10.1931), \emph{Schriftsteller, Mediziner}!Liebelei. Schauspiel in drei Akten1895-10-09@\strich\emph{Liebelei. Schauspiel in drei Akten} {[}1895-10-09{]}|pwk}, für die das \emph{Burgtheater}\orgindex{Burgtheater@Burgtheater|pwk} noch das ausschließliche
                  Aufführungsrecht hatte, vgl. Arthur Schnitzler an Felix Salten, 16. 10. 1902. Die Premiere am Raimundtheater\oindex{Raimund-Theater@\textbf{Raimund-Theater}|pwk} fand am
                     7. 3. 1903 statt.}}}\label{K_L03335-11h} Mittheilung. Das
               schiebt allerdings die Premiere im R. Th\oindex{Raimund-Theater@\textbf{Raimund-Theater}|pw}. ein
               wenig hinaus!\pend
           \pstart
           Hoffentlich schreiben Sie mir bald! \pend
           \pstart
           Herzlichst Ihr {\\[\baselineskip]}\spacefill\mbox{Salten}\pend
           \leftskip=0em{}
         
         \endnumbering\mylabel{h}\end{ledgroupsized}  \newcommand{\dateiname}{L03335}\newcommand{\titel}{Felix Salten an Arthur Schnitzler, 15. 10. 1902}\newcommand{\editorInnen}{Martin Anton Müller und Laura Untner}%% latex-leseansicht-abspann.tex
%% Abspann für die Leseansicht.
%% Der Schalter \ifkorrekturansicht ist bereits durch den Vorspann gesetzt.

%% latex-abspann.tex
%% Gemeinsamer Abspann für Korrekturansicht und Leseansicht.
%% Setzt den Schalter \ifkorrekturansicht voraus (gesetzt in den
%% einbindenden Dateien latex-korrekturansicht-abspann.tex bzw.
%% latex-leseansicht-abspann.tex).
%% ---------------------------------------------------------------

\normalsize

% Das esempio-Environment wird nur in der Leseansicht benötigt
\ifkorrekturansicht\else
\newenvironment{esempio}[3]%
{
    \vspace{1.5ex}
    \rlap{\underline{#1}}
    \par
    \setlength{\parindent}{0cm}
    \nopagebreak
    \leftskip=#2cm
    \rightskip=#3cm
}
{
    \par
}
\fi

\doendnotes{C}
\bigskip
\vfill

\clearpage

\footnotesize

\ifkorrekturansicht
  \lohead{\textsc{register}}
\fi

% theindex-Environment neu definieren ohne reledmac
\makeatletter
\renewenvironment{theindex}{%
  \ifkorrekturansicht
    \section*{\indexname}%
  \else
    \subsubsection*{Index der erwähnten Entitäten}%
  \fi
  \setlength{\parindent}{0pt}%
  \setlength{\parskip}{0pt plus 0.3pt}%
  \let\item\@idxitem
}{%
  \ifkorrekturansicht\clearpage\fi
}
\makeatother

\IfFileExists{\jobname-pw.ind}{\input{\jobname-pw.ind}}{}

% Quellenangabe nur in der Leseansicht
\ifkorrekturansicht\else
% Fallback-Definitionen, falls die .tex-Datei \titel etc. nicht gesetzt hat
\providecommand{\titel}{}
\providecommand{\editorInnen}{}
\providecommand{\dateiname}{\jobname}

\vspace{3cm}

\vfill

\footnotesize
\textsc{Quelle}: \titel. Herausgegeben von {\editorInnen}. In: \emph{Arthur Schnitzler: Briefwechsel mit Autorinnen und Autoren}.
 Digitale Edition, https://schnitzler-briefe.acdh.oeaw.ac.at/{\dateiname}.html (Stand \today)
\fi

\end{document}


      