%% latex-leseansicht-vorspann.tex
%% Vorspann für die Leseansicht.
%% Lädt die gemeinsame Datei latex-vorspann.tex mit nicht gesetztem Schalter.

\newif\ifkorrekturansicht
\korrekturansichtfalse

\input{../tex-inputs/latex-vorspann}

\begin{center}
            \textcolor{red}{ENTWURF, NICHT FERTIG KORRIGIERT}
                      \end{center}
            
         
         \renewcommand{\erwaehntePersonen}{Personen:  Forster, Hugo Ganz, Ernst Gettke, Theodor Herzl, Heinrich Kanner, Raphael Löwenfeld, Paul Paschen, Isidor Singer, Siegfried Trebitsch, Jakob Wassermann}
         \renewcommand{\erwaehnteInstitutionen}{Institutionen: Die Zeit, Raimund-Theater, S. Fischer Verlag, Schiller-Theater}
         \renewcommand{\erwaehnteOrte}{Orte: Berlin, Wien, Wipplingerstraße}
         \renewcommand{\erwaehnteWerke}{Werke: Altneuland. Roman, Der Moloch, Die Zeit, Fünfkreuzertanz, »Altneuland«}
               \section[Felix Salten an Arthur Schnitzler, 15. 10. 1902]{ Felix Salten an Arthur Schnitzler, 15. 10. 1902}\nopagebreak\mylabel{v}\rehead{ }\begin{ledgroupsized}[t]{13cm}\normalsize\beginnumbering \toendnotes[C]{\smallbreak\pagebreak[2]} \Standort{CUL, Schnitzler, B 89, A 2.}
\physDesc{Brief, 1 Blatt, 2 Seiten, 1738 Zeichen
\newline{}Handschrift: blaue Tinte, lateinische Kurrent
\newline{}Ordnung: mit Bleistift von unbekannter Hand nummeriert:
                                    »160« }\toendnotes[C]{\smallbreak}\pstart
           \noindent{}{\pb}\textcolor{gray}{\textbf{DIE}}\pend
           \pstart
           \textcolor{gray}{\textbf{ZEIT\orgindex{Zeit@Die Zeit|pw}}}\pend
           \pstart
           \textcolor{gray}{\textbf{Wien\oindex{Wien@\textbf{Wien}|pw}er Tageszeitung}}\hfill \textcolor{gray}{\textbf{WIEN\oindex{Wien@\textbf{Wien}|pw}}}{ }15. Octob. 02\pend
           \pstart
           \textcolor{gray}{\textbf{Herausgeber: }}\hfill \textcolor{gray}{\textbf{I., Wipplingerstrasse 38\oindex{Wipplingerstrasse@\textbf{Wipplingerstraße}|pw}}}\pend
           \pstart
           \textcolor{gray}{\textbf{Prof. Dr. I. Singer\pwindex{Singer, Isidor 16.01.1857 – 08.12.1927@\textsc{Singer, Isidor} (16.01.1857 – 08.12.1927), \emph{Journalist, Herausgeber, Soziologe}|pw}}}\pend
           \pstart
           \textcolor{gray}{\textbf{Dr. Heinrich Kanner\pwindex{Kanner, Heinrich 09.11.1864 – 15.02.1930@\textsc{Kanner, Heinrich} (09.11.1864 – 15.02.1930), \emph{Herausgeber, Publizist}|pw}}}\pend
           \pstart
           \textcolor{gray}{\textbf{Redaction.}}\pend
           \pstart
           \textcolor{gray}{\textbf{Telegramm-Adresse: \so{Zeit}\orgindex{Zeit@Die Zeit|pw}\so{,{ }}\so{Wien}\oindex{Wien@\textbf{Wien}|pw}}}\pend
           \pstart
           \textcolor{gray}{\textbf{Interurbanes Telephon Nr. 15.988}}\pend
           \pstart
           \textcolor{gray}{\textbf{= Telephone Nr. 17.040, 17.041 =}}\pend
           \pstart
           Lieber Freund, ich habe sehr bedauert, dass mich die Satzcorrectur
               zum »\label{K_L03335-11v}\edtext{Fünfkreuzertanz\pwindex{Salten, Felix 06.09.1869 – 08.10.1945@\textsc{Salten, Felix} (06.09.1869 – 08.10.1945), \emph{Schriftsteller, Journalist}!Fuenfkreuzertanz1902-10-12@\strich\emph{Fünfkreuzertanz} {[}1902-10-12{]}|pw}}{\lemma{\textnormal{\emph{Fünfkreuzertanz}}}\Cendnote{\textnormal{Felix Salten\pwindex{Salten, Felix 06.09.1869 – 08.10.1945@\textsc{Salten, Felix} (06.09.1869 – 08.10.1945), \emph{Schriftsteller, Journalist}|pwk}: \emph{Fünfkreuzertanz}\pwindex{Salten, Felix 06.09.1869 – 08.10.1945@\textsc{Salten, Felix} (06.09.1869 – 08.10.1945), \emph{Schriftsteller, Journalist}!Fuenfkreuzertanz1902-10-12@\strich\emph{Fünfkreuzertanz} {[}1902-10-12{]}|pwk}. In: \emph{Die Zeit}\pwindex{Zeit1902-09-27 – 1919@\emph{Die Zeit} {[}1902-09-27 – 1919{]}|pwk}, Jg. 1, Nr. 16, 12. 10. 1902, Morgenblatt,
                     S. 2–3.}}}\label{K_L03335-11h}« Samstag bis 2 Uhr in der Redaction
               aufhielt, so dass ich Sie nicht mehr sehen konnte. Ich bitte Sie nun um einige
               Kleinigkeiten, die Sie gelegentlich, ohne Mühe ausrichten, und für die ich Ihnen sehr
               dankbar wäre. Erstens Herrn D\textsuperscript{r} \label{K_L03335-12v}\edtext{Löwenfeld\pwindex{Loewenfeld, Raphael 11.02.1854 – 28.12.1910@\textsc{Löwenfeld, Raphael} (11.02.1854 – 28.12.1910), \emph{Theaterleiter}|pw} bestens von mir zu grüßen}{\lemma{\textnormal{\emph{Löwenfeld … grüßen}}}\Cendnote{\textnormal{vgl. A. S.: \emph{Tagebuch}, 17. 10. 1902}}}\label{K_L03335-12h}, und ihm zu sagen, dass ich seinen \label{K_L03335-111v}\edtext{Aufsatz über volksthümliche Claßikervorstellungen schon
               sehnlichst erwarte}{\lemma{\textnormal{\emph{Aufsatz … erwarte}}}\Cendnote{\textnormal{nicht
                  nachgewiesen}}}\label{K_L03335-111h}. Dann erkundigen Sie sich, bitte, nach dem Schauspieler Paul Paschen\pwindex{Paschen, Paul @\textsc{Paschen, Paul}, \emph{Schauspieler, Filmschauspieler, Stimmbildner}|pw} (Schillertheater\orgindex{Schiller-Theater@Schiller-Theater|pw}) was das für ein Mensch ist. Ich habe durch Geh. Rt. Forster\pwindex{Forster @\textsc{Forster}|pw} einen \label{K_L03335-124v}\edtext{Artikel von ihm bekommen über die Schweinerei des
                  Coulissentones}{\lemma{\textnormal{\emph{Artikel … Coulissentones}}}\Cendnote{\textnormal{Abdruck nicht
                  nachgewiesen}}}\label{K_L03335-124h}. Zuletzt noch – wenn bei Fischer\orgindex{S. Fischer Verlag@S. Fischer Verlag|pw} eine endgültige Entscheidung getroffen ist, depeschiren Sie mir,
               bitte. Ich bin sehr neugierig, wie Sie sich leicht denken können. Ich muß nun den {\pb}»Moloch\pwindex{Wassermann, Jakob 10.03.1873 – 01.01.1934@\textsc{Wassermann, Jakob} (10.03.1873 – 01.01.1934), \emph{Schriftsteller}!Moloch1902-12@\strich\emph{Der Moloch} {[}1902-12{]}|pw}« trotzdem ich ihn das erste Mal refüsirt habe, besprechen\pwindex{Zeit1902-09-27 – 1919@\emph{Die Zeit} {[}1902-09-27 – 1919{]}|pwv}. Hugo Ganz\pwindex{Ganz, Hugo 24.04.1862 – 02.01.1922@\textsc{Ganz, Hugo} (24.04.1862 – 02.01.1922), \emph{Schriftsteller, Journalist}|pw} hätte ihn übel zugerichtet, und bat mich schließlich
               darum, weil er \label{K_L03335-13v}\edtext{Herzl\pwindex{Herzl, Theodor 1860-05-02 – 1904-07-03@\textsc{Herzl, Theodor} (1860-05-02 – 1904-07-03), \emph{Schriftsteller, Journalist}|pw}’s Roman »Altneuland\pwindex{Herzl, Theodor 1860-05-02 – 1904-07-03@\textsc{Herzl, Theodor} (1860-05-02 – 1904-07-03), \emph{Schriftsteller, Journalist}!Altneuland. Roman1902@\strich\emph{Altneuland. Roman} {[}1902{]}|pw}« übernommen}{\lemma{\textnormal{\emph{Herzl’s … übernommen}}}\Cendnote{\textnormal{Lector\pwindex{Ganz, Hugo 24.04.1862 – 02.01.1922@\textsc{Ganz, Hugo} (24.04.1862 – 02.01.1922), \emph{Schriftsteller, Journalist}|pwk} [ =Hugo Ganz\pwindex{Ganz, Hugo 24.04.1862 – 02.01.1922@\textsc{Ganz, Hugo} (24.04.1862 – 02.01.1922), \emph{Schriftsteller, Journalist}|pwk}]: \emph{»Altneuland«}\pwindex{Altneuland«1902-11-05@\emph{»Altneuland«} {[}1902-11-05{]}|pwk}. In: \emph{Die Zeit}\pwindex{Zeit1902-09-27 – 1919@\emph{Die Zeit} {[}1902-09-27 – 1919{]}|pwk},
                     Jg. 1, Nr. 39, 5. 11. 1902, Morgenblatt, S. 1–2.}}}\label{K_L03335-13h}
               hat. Ich habe aufmerksam gemacht, dass ich das Buch nicht loben kann, und da man
               daran keinen Anstoß nahm, habe ich weiter keine Ursache, mich meiner ganzen Meinung
               über W.\pwindex{Wassermann, Jakob 10.03.1873 – 01.01.1934@\textsc{Wassermann, Jakob} (10.03.1873 – 01.01.1934), \emph{Schriftsteller}|pw} zurückzuhalten. Bei alledem hat W.\pwindex{Wassermann, Jakob 10.03.1873 – 01.01.1934@\textsc{Wassermann, Jakob} (10.03.1873 – 01.01.1934), \emph{Schriftsteller}|pw} noch Glück. Erstens ist er aus Ganz\pwindex{Ganz, Hugo 24.04.1862 – 02.01.1922@\textsc{Ganz, Hugo} (24.04.1862 – 02.01.1922), \emph{Schriftsteller, Journalist}|pw}’ Händen entwischt, zweitens nützt ihm die
               Raserei Trebitsch\pwindex{Trebitsch, Siegfried 22.12.1868 – 03.06.1956@\textsc{Trebitsch, Siegfried} (22.12.1868 – 03.06.1956), \emph{Schriftsteller, Übersetzer}|pw}’s bei mir, der schon glaubt,
               der Tag der nächsten Woche, an welchem mein Moloch-F.\pwindex{Zeit1902-09-27 – 1919@\emph{Die Zeit} {[}1902-09-27 – 1919{]}|pwv} erscheint, sei der Tag des Herrn Trebitsch\pwindex{Trebitsch, Siegfried 22.12.1868 – 03.06.1956@\textsc{Trebitsch, Siegfried} (22.12.1868 – 03.06.1956), \emph{Schriftsteller, Übersetzer}|pw}. \pend
           \pstart
           \label{K_L03335-51v}\edtext{Gettke\pwindex{Gettke, Ernst 08.10.1841 – 04.12.1912@\textsc{Gettke, Ernst} (08.10.1841 – 04.12.1912), \emph{Schriftsteller, Theaterleiter, Regisseur}|pw} ist seit ca. 14 Tagen im Besitz Ihres
                  Vertrages}{\lemma{\textnormal{\emph{Gettke … Vertrages}}}\Cendnote{\textnormal{vgl. A. S.: \emph{Tagebuch}, 29. 10. 1902}}}\label{K_L03335-51h}. Ich besuche ihn heute, und mache ihm von der inzwischen eingetretenen
               Änderung der Dinge Mittheilung. Das schiebt allerdings die Premiere im R. Th\orgindex{Raimund-Theater@Raimund-Theater|pw}. ein wenig hinaus! \pend
           \pstart
           Hoffentlich schreiben Sie mir bald! \pend
           \pstart
           Herzlichst Ihr {\\[\baselineskip]}\spacefill\mbox{Salten}\pend
           \leftskip=0em{}
         
         \endnumbering\mylabel{h}\end{ledgroupsized}\begin{anhang}\end{anhang}\newcommand{\dateiname}{L03335}\newcommand{\titel}{Felix Salten an Arthur Schnitzler, 15. 10. 1902}\newcommand{\editorInnen}{Martin Anton Müller und Laura Untner}%% latex-leseansicht-abspann.tex
%% Abspann für die Leseansicht.
%% Der Schalter \ifkorrekturansicht ist bereits durch den Vorspann gesetzt.

%% latex-abspann.tex
%% Gemeinsamer Abspann für Korrekturansicht und Leseansicht.
%% Setzt den Schalter \ifkorrekturansicht voraus (gesetzt in den
%% einbindenden Dateien latex-korrekturansicht-abspann.tex bzw.
%% latex-leseansicht-abspann.tex).
%% ---------------------------------------------------------------

\normalsize

% Das esempio-Environment wird nur in der Leseansicht benötigt
\ifkorrekturansicht\else
\newenvironment{esempio}[3]%
{
    \vspace{1.5ex}
    \rlap{\underline{#1}}
    \par
    \setlength{\parindent}{0cm}
    \nopagebreak
    \leftskip=#2cm
    \rightskip=#3cm
}
{
    \par
}
\fi

\doendnotes{C}
\bigskip
\vfill

\clearpage

\footnotesize

\ifkorrekturansicht
  \lohead{\textsc{register}}
\fi

% theindex-Environment neu definieren ohne reledmac
\makeatletter
\renewenvironment{theindex}{%
  \ifkorrekturansicht
    \section*{\indexname}%
  \else
    \subsubsection*{Index der erwähnten Entitäten}%
  \fi
  \setlength{\parindent}{0pt}%
  \setlength{\parskip}{0pt plus 0.3pt}%
  \let\item\@idxitem
}{%
  \ifkorrekturansicht\clearpage\fi
}
\makeatother

\IfFileExists{\jobname-pw.ind}{\input{\jobname-pw.ind}}{}

% Quellenangabe nur in der Leseansicht
\ifkorrekturansicht\else
% Fallback-Definitionen, falls die .tex-Datei \titel etc. nicht gesetzt hat
\providecommand{\titel}{}
\providecommand{\editorInnen}{}
\providecommand{\dateiname}{\jobname}

\vspace{3cm}

\vfill

\footnotesize
\textsc{Quelle}: \titel. Herausgegeben von {\editorInnen}. In: \emph{Arthur Schnitzler: Briefwechsel mit Autorinnen und Autoren}.
 Digitale Edition, https://schnitzler-briefe.acdh.oeaw.ac.at/{\dateiname}.html (Stand \today)
\fi

\end{document}


      