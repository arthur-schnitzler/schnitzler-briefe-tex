%% latex-korrekturansicht-vorspann.tex
%% Vorspann für die Korrekturansicht.
%% Lädt die gemeinsame Datei latex-vorspann.tex mit gesetztem Schalter.

\newif\ifkorrekturansicht
\korrekturansichttrue

\input{../tex-inputs/latex-vorspann}


\section[ Felix Salten an Arthur Schnitzler, 15. 10. 1902]{L03335 Felix Salten an Arthur Schnitzler, 15. 10. 1902}
\nopagebreak\mylabel{L03335v}
\rehead{ }\normalsize\beginnumbering\briefempfaengerindex{Schnitzler, Arthur@\textsc{Schnitzler, Arthur}!zzzSalten, Felix@\emph{von Felix Salten}!1902-10-153@{15. 10. 1902}|(be}
\toendnotes[C]{\smallbreak\pagebreak[2]}\Standort{CUL, Schnitzler, B 89, A 2.}
\physDesc{Brief, 1 Blatt, 2 Seiten, 1723 Zeichen
\newline{}Handschrift: blaue Tinte, lateinische Kurrent
\newline{}Ordnung: mit Bleistift von unbekannter Hand nummeriert: »160« }\toendnotes[C]{\smallbreak}
\pstart
           {\pb}\textcolor{gray}{\textbf{DIE}}\pend
           
\pstart
           \textcolor{gray}{\textbf{ZEIT\orgindex{Zeit@Die Zeit|pw}}}\pend
           
\pstart
           \textcolor{gray}{\textbf{\textbf{Wien\oindex{Wien@\textbf{Wien}, \emph{A.ADM2}|pw}er Tageszeitung}}}\hfill \textcolor{gray}{\textbf{\emph{WIEN\oindex{Wien@\textbf{Wien}, \emph{A.ADM2}|pw}}}}{ }15. Octob. 02\pend
           
\pstart
           \textcolor{gray}{\textbf{Herausgeber:}}\hfill \textcolor{gray}{\textbf{\emph{I., Wipplingerstrasse 38\oindex{Wipplingerstrasse@\textbf{Wipplingerstraße}, \emph{Straße (K.STR)}|pw}}}}\pend
           
\pstart
           \textcolor{gray}{\textbf{\textbf{Prof. Dr. I. Singer\pwindex{Singer, Isidor 16.01.1857 – 08.12.1927@\textsc{Singer, Isidor} (16.01.1857 – 08.12.1927), \emph{Journalist/Journalistin, Herausgeber/Herausgeberin, Soziologe/Soziologin}|pw}}}}\pend
           
\pstart
           \textcolor{gray}{\textbf{\textbf{Dr. Heinrich Kanner\pwindex{Kanner, Heinrich 09.11.1864 – 15.02.1930@\textsc{Kanner, Heinrich} (09.11.1864 – 15.02.1930), \emph{Herausgeber/Herausgeberin, Publizist/Publizistin}|pw}}}}\pend
           
\pstart
           \textcolor{gray}{\textbf{\textbf{Redaction.}}}\pend
           
\pstart
           \textcolor{gray}{\textbf{Telegramm-Adresse: \so{Zeit}\orgindex{Zeit@Die Zeit|pw}\so{,}{ }\so{Wien}\oindex{Wien@\textbf{Wien}, \emph{A.ADM2}|pw}}}\pend
           
\pstart
           \textcolor{gray}{\textbf{Interurbanes Telephon Nr. 15.988}}\pend
           
\pstart
           \textcolor{gray}{\textbf{= Telephone Nr. 17.040, 17.041 =}}\pend
           \vspace{0.5em}
\pstart
           Lieber Freund, ich habe sehr bedauert, dass mich die Satzcorrectur
               zum »\label{K_L03335-1v}\edtext{Fünfkreuzertanz\pwindex{Fuenfkreuzertanz@\emph{Fünfkreuzertanz}|pw}}{\lemma{\textnormal{\emph{Fünfkreuzertanz}}}\Cendnote{\textnormal{Felix Salten\pwindex{Salten, Felix 06.09.1869 – 08.10.1945@\textsc{Salten, Felix} (06.09.1869 – 08.10.1945), \emph{Schriftsteller/Schriftstellerin, Journalist/Journalistin, Chefredakteur/Chefredakteurin}|pwk}: \emph{Fünfkreuzertanz}\pwindex{Fuenfkreuzertanz@\emph{Fünfkreuzertanz}|pwk}. In: \emph{Die Zeit}\pwindex{Zeit@\emph{Die Zeit}|pwk}, Jg. 1, Nr. 16, 12. 10. 1902,
                     Morgenblatt, S. 2–3.}}}\label{K_L03335-1}« Samstag bis
                  2 Uhr in der Redaction\orgindex{Zeit@Die Zeit|pwv}\oindex{Wipplingerstrasse@\textbf{Wipplingerstraße}, \emph{Straße (K.STR)}|pwv} aufhielt, so dass ich Sie nicht mehr sehen konnte. Ich bitte Sie nun um einige
               Kleinigkeiten, die Sie gelegentlich, ohne Mühe ausrichten, und für die ich Ihnen sehr
               dankbar wäre. Erstens Herrn D\textsuperscript{r}{ }\label{K_L03335-2v}\edtext{Löwenfeld\pwindex{Loewenfeld, Raphael 11.02.1854 – 28.12.1910@\textsc{Löwenfeld, Raphael} (11.02.1854 – 28.12.1910), \emph{Theaterleiter/Theaterleiterin}|pw} bestens von mir zu grüßen}{\lemma{\textnormal{\emph{Löwenfeld … grüßen}}}\Cendnote{\textnormal{Schnitzler sah Raphael Löwenfeld\pwindex{Loewenfeld, Raphael 11.02.1854 – 28.12.1910@\textsc{Löwenfeld, Raphael} (11.02.1854 – 28.12.1910), \emph{Theaterleiter/Theaterleiterin}|pwk} am 15. 10. 1902 und am 17. 10. 1902.}}}\label{K_L03335-2},
               und ihm zu sagen, dass ich seinen \label{K_L03335-3v}\edtext{Aufsatz\pwindex{Aufsatz ueber volkstuemliche Klassikervorstellungen]@\emph{[Aufsatz über volkstümliche Klassikervorstellungen]}|pwv} über volksthümliche
               Claßikervorstellungen schon sehnlichst erwarte}{\lemma{\textnormal{\emph{Aufsatz … erwarte}}}\Cendnote{\textnormal{nicht nachgewiesen}}}\label{K_L03335-3}. Dann erkundigen Sie sich, bitte,
               nach dem Schauspieler Paul Paschen\pwindex{Paschen, Paul @\textsc{Paschen, Paul}, \emph{Schauspieler/Schauspielerin, Filmschauspieler/Filmschauspielerin, Stimmbildner/Stimmbildnerin}|pw} (Schillertheater\orgindex{Schiller-Theater@Schiller-Theater|pw}) was das für ein Mensch ist. Ich
               habe durch \label{K_L03335-4v}\edtext{Geh. Rt.}{\lemma{\textnormal{\emph{Geh. Rt.}}}\Cendnote{\textnormal{Geheimrat}}}\label{K_L03335-4}{ }Forster\pwindex{Forster @\textsc{Forster}|pw} einen \label{K_L03335-5v}\edtext{Artikel\pwindex{Artikel ueber Kulissenton]@\emph{[Artikel über Kulissenton]}|pwv} von ihm bekommen über
               die Schweinerei des Coulissentones}{\lemma{\textnormal{\emph{Artikel … Coulissentones}}}\Cendnote{\textnormal{nicht
                  nachgewiesen}}}\label{K_L03335-5}. Zuletzt noch – wenn \label{K_L03335-6v}\edtext{bei Fischer\orgindex{S. Fischer Verlag@S. Fischer Verlag|pw} eine
               endgültige Entscheidung}{\lemma{\textnormal{\emph{bei … Entscheidung}}}\Cendnote{\textnormal{Bezug auf die
                  Veröffentlichung von Saltens\pwindex{Salten, Felix 06.09.1869 – 08.10.1945@\textsc{Salten, Felix} (06.09.1869 – 08.10.1945), \emph{Schriftsteller/Schriftstellerin, Journalist/Journalistin, Chefredakteur/Chefredakteurin}|pwk}{ }\emph{Die kleine Veronika}\pwindex{kleine Veronika@\emph{Die kleine Veronika}|pwk} bei \emph{S. Fischer}\orgindex{S. Fischer Verlag@S. Fischer Verlag|pwk}, siehe A. S.: \emph{Tagebuch}, 15. 10. 1902 und Arthur Schnitzler an Felix Salten, 16. 10. 1902. }}}\label{K_L03335-6} getroffen ist, depeschiren Sie mir, bitte. Ich bin
               sehr neugierig, wie Sie sich leicht denken können. Ich muß nun den {\pb}»Moloch\pwindex{Moloch@\emph{Der Moloch}|pw}« trotzdem ich ihn das erste Mal refüsirt habe, \label{K_L03335-7v}\edtext{besprechen\pwindex{Zeit@\emph{Die Zeit}|pwv}}{\lemma{\textnormal{\emph{besprechen}}}\Cendnote{\textnormal{Felix Salten\pwindex{Salten, Felix 06.09.1869 – 08.10.1945@\textsc{Salten, Felix} (06.09.1869 – 08.10.1945), \emph{Schriftsteller/Schriftstellerin, Journalist/Journalistin, Chefredakteur/Chefredakteurin}|pwk}: \emph{Ein Gesellschaftsroman}\pwindex{Zeit@\emph{Die Zeit}|pwk}. In: \emph{Die Zeit}\pwindex{Zeit@\emph{Die Zeit}|pwk}, Jg. 1, Nr. 81, 19. 12. 1902, Morgenblatt, S. 1–2.}}}\label{K_L03335-7}. Hugo Ganz\pwindex{Ganz, Hugo 24.04.1862 – 02.01.1922@\textsc{Ganz, Hugo} (24.04.1862 – 02.01.1922), \emph{Schriftsteller/Schriftstellerin, Journalist/Journalistin}|pw} hätte ihn übel zugerichtet, und bat mich schließlich
               darum, weil er \label{K_L03335-8v}\edtext{Herzl\pwindex{Herzl, Theodor 1860-05-02 – 1904-07-03@\textsc{Herzl, Theodor} (1860-05-02 – 1904-07-03), \emph{Schriftsteller/Schriftstellerin, Journalist/Journalistin}|pw}’s \substVorne{}\textsuperscript{r}\substDazwischen{}R\substHinten{}oman »Altneuland\pwindex{Altneuland. Roman@\emph{Altneuland. Roman}|pw}« übernommen}{\lemma{\textnormal{\emph{Herzl’s … übernommen}}}\Cendnote{\textnormal{Lector\pwindex{Ganz, Hugo 24.04.1862 – 02.01.1922@\textsc{Ganz, Hugo} (24.04.1862 – 02.01.1922), \emph{Schriftsteller/Schriftstellerin, Journalist/Journalistin}|pwk} [ = Hugo Ganz\pwindex{Ganz, Hugo 24.04.1862 – 02.01.1922@\textsc{Ganz, Hugo} (24.04.1862 – 02.01.1922), \emph{Schriftsteller/Schriftstellerin, Journalist/Journalistin}|pwk}]: \emph{»Altneuland«}\pwindex{Altneuland«@\emph{»Altneuland«}|pwk}. In: \emph{Die Zeit}\pwindex{Zeit@\emph{Die Zeit}|pwk},
                     Jg. 1, Nr. 39, 5. 11. 1902, Morgenblatt,
                     S. 1–2.}}}\label{K_L03335-8} hat. Ich habe aufmerksam gemacht, dass ich das Buch\pwindex{Altneuland. Roman@\emph{Altneuland. Roman}|pwv} nicht loben kann, und da
               man daran keinen Anstoß nahm, habe ich weiter keine Ursache, mit meiner ganzen
               Meinung über W.\pwindex{Wassermann, Jakob 10.03.1873 – 01.01.1934@\textsc{Wassermann, Jakob} (10.03.1873 – 01.01.1934), \emph{Schriftsteller/Schriftstellerin}|pw} zurückzuhalten. Bei alledem
               hat W.\pwindex{Wassermann, Jakob 10.03.1873 – 01.01.1934@\textsc{Wassermann, Jakob} (10.03.1873 – 01.01.1934), \emph{Schriftsteller/Schriftstellerin}|pw} noch Glück. Erstens ist er aus Ganz\pwindex{Ganz, Hugo 24.04.1862 – 02.01.1922@\textsc{Ganz, Hugo} (24.04.1862 – 02.01.1922), \emph{Schriftsteller/Schriftstellerin, Journalist/Journalistin}|pw}’ Händen entwischt, zweitens nützt ihm die
               Raserei Trebitsch\pwindex{Trebitsch, Siegfried 22.12.1868 – 03.06.1956@\textsc{Trebitsch, Siegfried} (22.12.1868 – 03.06.1956), \emph{Schriftsteller/Schriftstellerin, Übersetzer/Übersetzerin}|pw}’s bei mir, der schon glaubt,
               der Tag der nächsten Woche, an welchem mein Moloch-F.\pwindex{Zeit@\emph{Die Zeit}|pwv} erscheint, sei der Tag des Herrn Trebitsch\pwindex{Trebitsch, Siegfried 22.12.1868 – 03.06.1956@\textsc{Trebitsch, Siegfried} (22.12.1868 – 03.06.1956), \emph{Schriftsteller/Schriftstellerin, Übersetzer/Übersetzerin}|pw}.\pend
           
\pstart
           \label{K_L03335-9v}\edtext{Gettke\pwindex{Gettke, Ernst 08.10.1841 – 04.12.1912@\textsc{Gettke, Ernst} (08.10.1841 – 04.12.1912), \emph{Schriftsteller/Schriftstellerin, Theaterleiter/Theaterleiterin, Regisseur/Regisseurin}|pw} ist seit c\textsuperscript{a} 14 Tagen im Besitz Ihres Vertrages}{\lemma{\textnormal{\emph{Gettke … Vertrages}}}\Cendnote{\textnormal{Siehe A. S.: \emph{Tagebuch}, 29. 10. 1902. }}}\label{K_L03335-9}. Ich
               besuche ihn heute, und mache ihm von der inzwischen
               eingetretenen \label{K_L03335-10v}\edtext{Änderung der Dinge}{\lemma{\textnormal{\emph{Änderung der Dinge}}}\Cendnote{\textnormal{Bezug auf eine mögliche Aufführung von \emph{Liebelei}\pwindex{Liebelei. Schauspiel in drei Akten@\emph{Liebelei. Schauspiel in drei Akten}|pwk}, für die das \emph{Burgtheater}\orgindex{Burgtheater@Burgtheater|pwk} noch das ausschließliche Aufführungsrecht
                  hatte, vgl. Arthur Schnitzler an Felix Salten, 16. 10. 1902. Die Premiere
                  am \emph{Raimundtheater}\orgindex{Raimund-Theater@Raimund-Theater|pwk} fand am 7. 3. 1903
                  statt.}}}\label{K_L03335-10} Mittheilung. Das schiebt allerdings die Premiere im R. Th\oindex{Raimund-Theater@\textbf{Raimund-Theater}, \emph{Theater (K.THE)}|pw}. ein wenig hinaus!\pend
           
\pstart
           Hoffentlich schreiben Sie mir bald! \pend
           
\pstart
           Herzlichst Ihr {\\[\baselineskip]}\spacefill\mbox{Salten}\pend
           \leftskip=0em{}\selectlanguage{ngerman}\endnumbering\briefempfaengerindex{Schnitzler, Arthur@\textsc{Schnitzler, Arthur}!zzzSalten, Felix@\emph{von Felix Salten}!1902-10-153@{15. 10. 1902}|)be}\mylabel{L03335h}  \normalsize

\doendnotes{C}
\bigskip
\vfill

\clearpage

\footnotesize

\lohead{\textsc{register}}

% Definiere theindex-Environment komplett neu ohne reledmac
\makeatletter
\renewenvironment{theindex}{%
  \section*{\indexname}%
  \setlength{\parindent}{0pt}%
  \setlength{\parskip}{0pt plus 0.3pt}%
  \let\item\@idxitem
}{%
  \clearpage
}
\makeatother

\IfFileExists{\jobname-pw.ind}{\input{\jobname-pw.ind}}{}

\end{document}

      