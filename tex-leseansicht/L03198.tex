%% latex-leseansicht-vorspann.tex
%% Vorspann für die Leseansicht.
%% Lädt die gemeinsame Datei latex-vorspann.tex mit nicht gesetztem Schalter.

\newif\ifkorrekturansicht
\korrekturansichtfalse

\input{../tex-inputs/latex-vorspann}


\section[ Paul Goldmann an Arthur Schnitzler, 25. 2. {[}1902{]}]{L03198 Paul Goldmann an Arthur Schnitzler,  25. 2. [1902]}
\nopagebreak\mylabel{L03198v}
\rehead{ }\normalsize\beginnumbering\briefempfaengerindex{Schnitzler, Arthur@\textsc{Schnitzler, Arthur}!zzzGoldmann, Paul@\emph{von Paul Goldmann}!1902-02-251@{25. 2. [1902]}|(be}
\toendnotes[C]{\smallbreak\pagebreak[2]}
\correspDesc{Versand  durch Paul Goldmann am 25. 2. [1902] in Berlin
\newline{}Erhalt  durch Arthur Schnitzler im Zeitraum [26. 2. 1902
                  – 2. 3. 1902?] in Wien?}\toendnotes[C]{\smallbreak}
\Standort{DLA, A:Schnitzler, HS.NZ85.1.3172.}
\physDesc{Brief, 2 Blätter, 8 Seiten, 3647 Zeichen
\newline{}Handschrift: blaue Tinte, deutsche Kurrent
\newline{}Schnitzler: 1) mit Bleistift das Jahr »902« vermerkt  2) mit rotem Buntstift fünf Unterstreichungen}\toendnotes[C]{\smallbreak}
\pstart
           \raggedleft{}{\pb}\textcolor{gray}{\textbf{DESSAUERSTRASSE 19}}\oindex{Dessauer Straße@\textbf{Dessauer Straße}, \emph{Straße}|pw}\pend
           
\pstart
           Berlin\oindex{Berlin@\textbf{Berlin}, \emph{Hauptstadt}|pw}, 25. Februar.\pend
           
\pstart{}Mein lieber Freund,\pend\vspace{0.5em}
\pstart
           Ich komme leider erſt heut dazu, Deinen lieben Brief
               zu beantworten, der mir große Freude bereitet hat, weil er mir wieder einmal
               eingehenderen Bericht über Dein Ergehen gab. Ich habe eine ganze Woche lang an einem
                  \label{K_L03198-1v}\edtext{Feuilleton\pwindex{Goldmann, Paul 31.\,1.\,1865 Breslau – 25.\,9.\,1935 Wien@\textsc{Goldmann, Paul} (31.\,1.\,1865 Breslau – 25.\,9.\,1935 Wien), \emph{Schriftsteller, Journalist}!Berliner Theater. »Der Herr von Abadessa« von Felix Dörmann im Königlichen Schauspielhause@\strich\emph{Berliner Theater. »Der Herr von Abadessa« von Felix Dörmann im Königlichen Schauspielhause}|pwv} über den »Herrn von \textsc{Abadessa}\pwindex{Dörmann, Felix 29.\,5.\,1870 Wien – 26.\,10.\,1928 ebd.@\textsc{Dörmann, Felix} (29.\,5.\,1870 Wien – 26.\,10.\,1928 ebd.), \emph{Schriftsteller}!Herr von Abadessa. Ein Abenteurerstreich in drei Akten@\strich\emph{Der Herr von Abadessa. Ein Abenteurerstreich in drei Akten}|pw}«}{\lemma{\textnormal{\emph{Feuilleton … Abadessa«}}}\Cendnote{\textnormal{Paul Goldmann\pwindex{Goldmann, Paul 31.\,1.\,1865 Breslau – 25.\,9.\,1935 Wien@\textsc{Goldmann, Paul} (31.\,1.\,1865 Breslau – 25.\,9.\,1935 Wien), \emph{Schriftsteller, Journalist}|pwk}: \emph{Berliner Theater. »Der Herr von Abadessa« von Felix Dörmann
                        im Königlichen Schauspielhause}\pwindex{Goldmann, Paul 31.\,1.\,1865 Breslau – 25.\,9.\,1935 Wien@\textsc{Goldmann, Paul} (31.\,1.\,1865 Breslau – 25.\,9.\,1935 Wien), \emph{Schriftsteller, Journalist}!Berliner Theater. »Der Herr von Abadessa« von Felix Dörmann im Königlichen Schauspielhause@\strich\emph{Berliner Theater. »Der Herr von Abadessa« von Felix Dörmann im Königlichen Schauspielhause}|pwk}. In: \emph{Neue Freie Presse}\pwindex{Neue Freie Presse@\emph{Neue Freie Presse}|pwk}, Nr. 13.472, 25. 2. 1902, Morgenblatt, S. 1–4.}}}\label{K_L03198-1} (bezüglich deſſen
               ich \label{K_L03198-2v}\edtext{Deine Anſicht}{\lemma{\textnormal{\emph{Deine Ansicht}}}\Cendnote{\textnormal{Schnitzler fand es\pwindex{Dörmann, Felix 29.\,5.\,1870 Wien – 26.\,10.\,1928 ebd.@\textsc{Dörmann, Felix} (29.\,5.\,1870 Wien – 26.\,10.\,1928 ebd.), \emph{Schriftsteller}!Herr von Abadessa. Ein Abenteurerstreich in drei Akten@\strich\emph{Der Herr von Abadessa. Ein Abenteurerstreich in drei Akten}|pwkv} schlecht, vgl. A. S.: \emph{Tagebuch}, 17. 12. 1901.}}}\label{K_L03198-2} vollſtändig theile) geſchrieben und zu nichts Anderem Zeit gefunden. Jetzt
               fürchte ich, daß die Rieſenarbeit vergeblich geweſen iſt, weil ich{ }ſehr{ }ſcharf über
                  \textsc{Dörmann\pwindex{Dörmann, Felix 29.\,5.\,1870 Wien – 26.\,10.\,1928 ebd.@\textsc{Dörmann, Felix} (29.\,5.\,1870 Wien – 26.\,10.\,1928 ebd.), \emph{Schriftsteller}|pw}} abgeurtheilt habe und weil man mir kaum erlauben {\pb}wird, über einen früheren Mitarbeiter der N. Fr. Pr.\orgindex{Neue Freie Presse@Neue Freie Presse|pw}{ }ſcharf zu urtheilen.\pend
           
\pstart
           Es freut mich{ }ſehr, zu hören, daß es \label{K_L03198-3v}\edtext{\textsc{Olga\pwindex{Schnitzler, Olga 17.\,1.\,1882 Wien – 13.\,1.\,1970 Lugano@\textsc{Schnitzler, Olga} (17.\,1.\,1882 Wien – 13.\,1.\,1970 Lugano), \emph{Schauspielerin, Sängerin}|pw}} beſſer geht}{\lemma{\textnormal{\emph{Olga besser geht}}}\Cendnote{\textnormal{Siehe XXXX Auszeichnungsfehler: Dokument L03193 nicht gefunden.
               }}}\label{K_L03198-3}. Nächſtens{ }ſchreibe ich ihr wirklich. Ich zweifle nicht, daß dieſe Ausſicht
               die Beſſerung im Befinden der verehrten Freundin\pwindex{Schnitzler, Olga 17.\,1.\,1882 Wien – 13.\,1.\,1970 Lugano@\textsc{Schnitzler, Olga} (17.\,1.\,1882 Wien – 13.\,1.\,1970 Lugano), \emph{Schauspielerin, Sängerin}|pwv} beſchleunigen wird. Wie unendlich gern ich im März mit Euch\pwindex{Schnitzler, Olga 17.\,1.\,1882 Wien – 13.\,1.\,1970 Lugano@\textsc{Schnitzler, Olga} (17.\,1.\,1882 Wien – 13.\,1.\,1970 Lugano), \emph{Schauspielerin, Sängerin}|pwv} in die \label{K_L03198-4v}\edtext{Berge\oindex{Hinterbrühl@\textbf{Hinterbrühl}, \emph{Hauptstadt}|pwv}}{\lemma{\textnormal{\emph{Berge}}}\Cendnote{\textnormal{Mit einigen Unterbrechungen hielten sich
                     Schnitzler, die schwangere Olga Gussmann\pwindex{Schnitzler, Olga 17.\,1.\,1882 Wien – 13.\,1.\,1970 Lugano@\textsc{Schnitzler, Olga} (17.\,1.\,1882 Wien – 13.\,1.\,1970 Lugano), \emph{Schauspielerin, Sängerin}|pwk} und womöglich auch deren
                  Schwester Elisabeth Gussmann\pwindex{Steinrück, Elisabeth 19.\,11.\,1885 – 7.\,4.\,1920 Partenkirchen@\textsc{Steinrück, Elisabeth} (19.\,11.\,1885 – 7.\,4.\,1920 Partenkirchen)|pwk} zwischen 21. 3. 1902 und 31. 3. 1902 in der
                  neuen Unterkunft\oindex{Hauptstraße 56@\textbf{Hauptstraße 56}, \emph{Wohngebäude}|pwkv} in der
                     Hinterbrühl\oindex{Hinterbrühl@\textbf{Hinterbrühl}, \emph{Hauptstadt}|pwk} auf. Siehe XXXX Auszeichnungsfehler: Dokument L03192 nicht gefunden.}}}\label{K_L03198-4} gehen möchte,
               brauche ich nicht erſt zu{ }ſagen. Ich habe die ganze Reiſe bereits in der Phantaſie
               gemacht und dabei{ }ſehr{ }ſchöne Stunden mit Euch\pwindex{Schnitzler, Olga 17.\,1.\,1882 Wien – 13.\,1.\,1970 Lugano@\textsc{Schnitzler, Olga} (17.\,1.\,1882 Wien – 13.\,1.\,1970 Lugano), \emph{Schauspielerin, Sängerin}|pwv} verlebt. In der Wirklichkeit werde ich{ }ſie nicht machen
               können. Ich könnte höchſtens zu Oſtern ein paar Tage fort. Und der Weg von hier nach
                  Salzburg\oindex{Salzburg@\textbf{Salzburg}, \emph{Verwaltungsgebiet}|pw} oder gar {\pb}nach Südtirol\oindex{Südtirol@\textbf{Südtirol}, \emph{Verwaltungsgebiet}|pw}
               iſt für die drei oder vier Tage Urlaub, die ich mir nehmen könnte, allzu weit. Etwas
               Anderes wäre \substVorne{}\textsuperscript{ich}\substDazwischen{}es\substHinten{}, wenn Ihr\pwindex{Schnitzler, Olga 17.\,1.\,1882 Wien – 13.\,1.\,1970 Lugano@\textsc{Schnitzler, Olga} (17.\,1.\,1882 Wien – 13.\,1.\,1970 Lugano), \emph{Schauspielerin, Sängerin}|pwv} nach Deutſchland\oindex{Deutschland@\textbf{Deutschland}|pw} kommen könntet (Sächſiſche Schweiz\oindex{Sächsische Schweiz@\textbf{Sächsische Schweiz}, \emph{Region}|pw}\strikeout{,} oder Wiesbaden\oindex{Wiesbaden@\textbf{Wiesbaden}|pw}). Da könnte ich um Oſtern herum ein paar Tage mit Euch\pwindex{Schnitzler, Olga 17.\,1.\,1882 Wien – 13.\,1.\,1970 Lugano@\textsc{Schnitzler, Olga} (17.\,1.\,1882 Wien – 13.\,1.\,1970 Lugano), \emph{Schauspielerin, Sängerin}|pw}{ }ſein. Aber daran iſt ja wohl kaum zu denken. Ich
               wenigſtens würde{ }ſicher nicht nach \textsc{Wiesbaden\oindex{Wiesbaden@\textbf{Wiesbaden}|pw}} kommen, wenn ich nach Südtirol\oindex{Südtirol@\textbf{Südtirol}, \emph{Verwaltungsgebiet}|pw} gehen
               könnte.\pend
           
\pstart
           In der \label{K_L03198-5v}\edtext{Affaire \textsc{Matassich\pwindex{Mattachich, Géza von 19.\,12.\,1867 Tomaševac – 29.\,9.\,1923 Paris@\textsc{Mattachich, Géza von} (19.\,12.\,1867 Tomaševac – 29.\,9.\,1923 Paris), \emph{Oberstleutnant}|pw}}}{\lemma{\textnormal{\emph{Affaire Matassich}}}\Cendnote{\textnormal{Siehe XXXX Auszeichnungsfehler: Dokument L03197 nicht gefunden.
               }}}\label{K_L03198-5} haſt Du vollkommen Recht. Es war bei mir nur{ }ſo eine Regung, als ich die Rede\pwindex{Daszyński, Ignacy 26.\,10.\,1866 Zbarazh – 31.\,10.\,1936 Cieszyn@\textsc{Daszyński, Ignacy} (26.\,10.\,1866 Zbarazh – 31.\,10.\,1936 Cieszyn), \emph{Politiker, Ministerpräsident}!?? [Rede über die Mattachich-Affaire]@\strich\emph{?? [Rede über die Mattachich-Affaire]}|pwv}{ }\textsc{Daszinskys\pwindex{Daszyński, Ignacy 26.\,10.\,1866 Zbarazh – 31.\,10.\,1936 Cieszyn@\textsc{Daszyński, Ignacy} (26.\,10.\,1866 Zbarazh – 31.\,10.\,1936 Cieszyn), \emph{Politiker, Ministerpräsident}|pw}} las. {\pb}Namentlich{ }ſchien es mir, es{ }ſei für Dich eine{ }ſchöne Gelegenheit, Dich bei den Herrn für die \label{K_L03198-6v}\edtext{Entziehung der Charge}{\lemma{\textnormal{\emph{Entziehung der Charge}}}\Cendnote{\textnormal{Bezug auf die \emph{Lieutenant
                     Gustl}\pwindex{Schnitzler, Arthur 15.\,5.\,1862 Wien – 21.\,10.\,1931 ebd.@\textsc{Schnitzler, Arthur} (15.\,5.\,1862 Wien – 21.\,10.\,1931 ebd.), \emph{Schriftsteller, Mediziner}!Lieutenant Gustl. Novelle@\strich\emph{Lieutenant Gustl. Novelle}|pwk}-Affäre, siehe XXXX Auszeichnungsfehler: Dokument L02655 nicht gefunden.}}}\label{K_L03198-6} zu revanchiren. Du weißt, ich bin rachſüchtig. Jetzt bin ich{ }ſehr zufrieden,
               daß Du von der gefährlichen Geſchichte die Hände wegläßt.\pend
           
\pstart
           Die »Lebendigen Stunden\pwindex{Schnitzler, Arthur 15.\,5.\,1862 Wien – 21.\,10.\,1931 ebd.@\textsc{Schnitzler, Arthur} (15.\,5.\,1862 Wien – 21.\,10.\,1931 ebd.), \emph{Schriftsteller, Mediziner}!Lebendige Stunden. Vier Einakter@\strich\emph{Lebendige Stunden. Vier Einakter}|pw}« werden{ }ſich hoffentlich
               in der nächſten Saiſon über die deutſchen Bühnen bewegen. Vielleicht iſt die{ }ſchon
               vorgerückte Saiſon daran{ }ſchuld, daß es einſtweilen nicht recht vorwärts geht. In der
                  Berlin\oindex{Berlin@\textbf{Berlin}, \emph{Hauptstadt}|pw}er Geſellſchaft höre ich überall mit
               Entzücken davon{ }ſprechen. {\pb}\label{K_L03198-7v}\edtext{\textsc{Kochs\pwindex{Koch, Max 22.\,12.\,1855 München – 19.\,12.\,1931 Breslau@\textsc{Koch, Max} (22.\,12.\,1855 München – 19.\,12.\,1931 Breslau), \emph{Literarhistoriker}|pw}}{ }Kritik\pwindex{Koch, Max 22.\,12.\,1855 München – 19.\,12.\,1931 Breslau@\textsc{Koch, Max} (22.\,12.\,1855 München – 19.\,12.\,1931 Breslau), \emph{Literarhistoriker}!?? [Kritik zu Lebendige Stunden]@\strich\emph{?? [Kritik zu Lebendige Stunden]}|pwv}}{\lemma{\textnormal{\emph{Kochs Kritik}}}\Cendnote{\textnormal{nicht nachgewiesen}}}\label{K_L03198-7}{ }ſende ich Dir
               anbei zurück. Es freut mich, daß{ }ſie{ }ſo günſtig ausgefallen iſt. \strikeout{\textcolor{gray}{×}\-\textcolor{gray}{×}\-\textcolor{gray}{×}\-\textcolor{gray}{×}\-\textcolor{gray}{×}\-\textcolor{gray}{×}\-\textcolor{gray}{×}\-\textcolor{gray}{×}\-\textcolor{gray}{×}\-\textcolor{gray}{×}\-\textcolor{gray}{×}\-\textcolor{gray}{×}\-\textcolor{gray}{×}\-\textcolor{gray}{×}\-\textcolor{gray}{×}\-\textcolor{gray}{×}} Sonſt{ }ſcheint mir dieſer Kritiker\pwindex{Koch, Max 22.\,12.\,1855 München – 19.\,12.\,1931 Breslau@\textsc{Koch, Max} (22.\,12.\,1855 München – 19.\,12.\,1931 Breslau), \emph{Literarhistoriker}|pwv} ein recht unbedeutender Kopf zu{ }ſein.\pend
           
\pstart
           Ich danke Dir für Deine freundlichen Worte über mein \label{K_L03198-8v}\edtext{Opern-Feuilleton\pwindex{Goldmann, Paul 31.\,1.\,1865 Breslau – 25.\,9.\,1935 Wien@\textsc{Goldmann, Paul} (31.\,1.\,1865 Breslau – 25.\,9.\,1935 Wien), \emph{Schriftsteller, Journalist}!Berliner Theater. (»Heilmar« von Wilhelm Kienzl im königlichen Opernhause.)@\strich\emph{Berliner Theater. (»Heilmar« von Wilhelm Kienzl im königlichen Opernhause.)}|pwv}}{\lemma{\textnormal{\emph{Opern-Feuilleton}}}\Cendnote{\textnormal{Paul Goldmann\pwindex{Goldmann, Paul 31.\,1.\,1865 Breslau – 25.\,9.\,1935 Wien@\textsc{Goldmann, Paul} (31.\,1.\,1865 Breslau – 25.\,9.\,1935 Wien), \emph{Schriftsteller, Journalist}|pwk}: \emph{Berliner Theater. (»Heilmar« von Wilhelm Kienzl im
                        königlichen Opernhause)}\pwindex{Goldmann, Paul 31.\,1.\,1865 Breslau – 25.\,9.\,1935 Wien@\textsc{Goldmann, Paul} (31.\,1.\,1865 Breslau – 25.\,9.\,1935 Wien), \emph{Schriftsteller, Journalist}!Berliner Theater. (»Heilmar« von Wilhelm Kienzl im königlichen Opernhause.)@\strich\emph{Berliner Theater. (»Heilmar« von Wilhelm Kienzl im königlichen Opernhause.)}|pwk}. In: \emph{Neue
                        Freie Presse}\pwindex{Neue Freie Presse@\emph{Neue Freie Presse}|pwk}, Nr. 13.458, 11. 2. 1902,
                     Morgenblatt, S. 1–4.}}}\label{K_L03198-8} und halte Deine Ausſtellung bezüglich der
               allzu großen Länge einzelner Abſätze für nur zu berechtigt. Ich fühle es{ }ſelber, daß
               es mein{ }ſchwerſter{ }ſchriftſtelleriſcher Fehler iſt, nicht kurz{ }ſein zu können. Aber
               beim Schreiben werde ich von einem beinahe krankhaften Drang befallen, Alles bis auf
               den Grund auszuſchöpfen. {\pb}Daher kommen die Längen,
               über die ich dann erſchreckt bin, wenn ich die Arbeit gedruckt{ }ſehe. Wie lernt man,
               kurz zu{ }ſein? Kannſt Du mir nicht ein Mittel{ }ſagen?\pend
           
\pstart
           Mein Onkel\pwindex{Mamroth, Fedor 21.\,2.\,1851 Breslau – 25.\,6.\,1907 Frankfurt am Main@\textsc{Mamroth, Fedor} (21.\,2.\,1851 Breslau – 25.\,6.\,1907 Frankfurt am Main), \emph{Journalist, Kritiker}|pwv}{ }ſchreibt mir mit
               höchſtem Enthuſiasmus von einem im Wiener Verlag\orgindex{Wiener Verlag@Wiener Verlag|pw}
               erſchienenen Buch \label{K_L03198-9v}\edtext{»Chriſtiania-\textsc{Bohême}\pwindex{Jæger, Hans 2.\,9.\,1854 Drammen – 8.\,2.\,1910 Oslo@\textsc{Jæger, Hans} (2.\,9.\,1854 Drammen – 8.\,2.\,1910 Oslo), \emph{Anarchist, Autor}!Christiania-Bohême@\strich\emph{Christiania-Bohême}|pwv}«}{\lemma{\textnormal{\emph{»Christiania-Bohême«}}}\Cendnote{\textnormal{Hans Jæger\pwindex{Jæger, Hans 2.\,9.\,1854 Drammen – 8.\,2.\,1910 Oslo@\textsc{Jæger, Hans} (2.\,9.\,1854 Drammen – 8.\,2.\,1910 Oslo), \emph{Anarchist, Autor}|pwk}: \emph{Christiania-Bohême}\pwindex{Jæger, Hans 2.\,9.\,1854 Drammen – 8.\,2.\,1910 Oslo@\textsc{Jæger, Hans} (2.\,9.\,1854 Drammen – 8.\,2.\,1910 Oslo), \emph{Anarchist, Autor}!Christiania-Bohême@\strich\emph{Christiania-Bohême}|pwk}. Wien\oindex{Wien@\textbf{Wien}, \emph{Verwaltungsgebiet}|pwk}: \emph{Wiener Verlag}\orgindex{Wiener Verlag@Wiener Verlag|pwk}{ }1902 (zuerst 1885, \emph{Fra Kristiania-Bohêmen}\pwindex{Jæger, Hans 2.\,9.\,1854 Drammen – 8.\,2.\,1910 Oslo@\textsc{Jæger, Hans} (2.\,9.\,1854 Drammen – 8.\,2.\,1910 Oslo), \emph{Anarchist, Autor}!Fra Kristiania-Bohêmen@\strich\emph{Fra Kristiania-Bohêmen}|pwk}).}}}\label{K_L03198-9} von
                  \textsc{Hans Jaeger\pwindex{Jæger, Hans 2.\,9.\,1854 Drammen – 8.\,2.\,1910 Oslo@\textsc{Jæger, Hans} (2.\,9.\,1854 Drammen – 8.\,2.\,1910 Oslo), \emph{Anarchist, Autor}|pw}}.\pend
           
\pstart
           Hörſt Du etwas von dem neuen Blatt, der \label{K_L03198-10v}\edtext{»Zeit\orgindex{Zeit@Die Zeit|pw}\orgindex{Zeit. Wiener Wochenschrift@Die Zeit. Wiener Wochenschrift|pw}«}{\lemma{\textnormal{\emph{»Zeit«}}}\Cendnote{\textnormal{Siehe XXXX Auszeichnungsfehler: Dokument L03193 nicht gefunden.
               }}}\label{K_L03198-10}?\pend
           
\pstart
           Im Sommer haſt Du mir ein \label{K_L03198-11v}\edtext{Buch\pwindex{?? [Buch über den Talmud]@\emph{?? [Buch über den Talmud]}|pwv}}{\lemma{\textnormal{\emph{Buch}}}\Cendnote{\textnormal{nicht ermittelt}}}\label{K_L03198-11} geſtohlen; das
               über den {\pb}Talmud\pwindex{?? [Buch über den Talmud]@\emph{?? [Buch über den Talmud]}|pw}. Ich brauche es und{ }ſchreibe \introOben{}heut\introOben{} an \label{K_L03198-12v}\edtext{\textsc{Richard}\pwindex{Beer-Hofmann, Richard 11.\,7.\,1866 Wien – 26.\,9.\,1945 New York City@\textsc{Beer-Hofmann, Richard} (11.\,7.\,1866 Wien – 26.\,9.\,1945 New York City), \emph{Schriftsteller}|pw}}{\lemma{\textnormal{\emph{Richard}}}\Cendnote{\textnormal{Goldmann\pwindex{Goldmann, Paul 31.\,1.\,1865 Breslau – 25.\,9.\,1935 Wien@\textsc{Goldmann, Paul} (31.\,1.\,1865 Breslau – 25.\,9.\,1935 Wien), \emph{Schriftsteller, Journalist}|pwk} schrieb Beer-Hofmannn\pwindex{Beer-Hofmann, Richard 11.\,7.\,1866 Wien – 26.\,9.\,1945 New York City@\textsc{Beer-Hofmann, Richard} (11.\,7.\,1866 Wien – 26.\,9.\,1945 New York City), \emph{Schriftsteller}|pwk} noch am selben Tag, vgl. \emph{Houghton Library}\orgindex{Houghton Library@Houghton Library|pwk},
                     Harvard (Signatur 825.978). Dem Brief ist zu
                  entnehmen, dass Goldmann\pwindex{Goldmann, Paul 31.\,1.\,1865 Breslau – 25.\,9.\,1935 Wien@\textsc{Goldmann, Paul} (31.\,1.\,1865 Breslau – 25.\,9.\,1935 Wien), \emph{Schriftsteller, Journalist}|pwk} das Buch\pwindex{?? [Buch über den Talmud]@\emph{?? [Buch über den Talmud]}|pwkv} von Beer-Hofmann\pwindex{Beer-Hofmann, Richard 11.\,7.\,1866 Wien – 26.\,9.\,1945 New York City@\textsc{Beer-Hofmann, Richard} (11.\,7.\,1866 Wien – 26.\,9.\,1945 New York City), \emph{Schriftsteller}|pwk} im Sommer 1901
                  geschenkt bekommen hatte, nicht aber der Titel.}}}\label{K_L03198-12}, er möge mir doch Titel und
               Verlag angeben, damit ich es mir kommen laſſen kann. Da ich aber dieſe Anfrage an \textsc{Richard}\pwindex{Beer-Hofmann, Richard 11.\,7.\,1866 Wien – 26.\,9.\,1945 New York City@\textsc{Beer-Hofmann, Richard} (11.\,7.\,1866 Wien – 26.\,9.\,1945 New York City), \emph{Schriftsteller}|pw} für ein völlig ausſichtsloſes Unternehmen halte, bitte ich Dich (wenn Du das
                  Buch\pwindex{?? [Buch über den Talmud]@\emph{?? [Buch über den Talmud]}|pwv} nicht{ }ſelber
               brauchſt), mir es gelegentlich zu{ }ſchicken. Iſt \textsc{Richard\pwindex{Beer-Hofmann, Richard 11.\,7.\,1866 Wien – 26.\,9.\,1945 New York City@\textsc{Beer-Hofmann, Richard} (11.\,7.\,1866 Wien – 26.\,9.\,1945 New York City), \emph{Schriftsteller}|pw}} wieder ganz \label{K_L03198-13v}\edtext{geſund}{\lemma{\textnormal{\emph{gesund}}}\Cendnote{\textnormal{Siehe XXXX Auszeichnungsfehler: Dokument L03195 nicht gefunden.
               }}}\label{K_L03198-13}?\pend
           
\pstart
           Ich{ }ſende Dir anbei zwei \label{K_L03198-14v}\edtext{Feuilletons\pwindex{Meyer-Benfey, Heinrich 14.\,3.\,1869 Liebenburg – 30.\,12.\,1945 Buxtehude@\textsc{Meyer-Benfey, Heinrich} (14.\,3.\,1869 Liebenburg – 30.\,12.\,1945 Buxtehude), \emph{Germanist}!Moderne Religion@\strich\emph{Moderne Religion}|pwv}}{\lemma{\textnormal{\emph{Feuilletons}}}\Cendnote{\textnormal{Die Beilage ist nicht erhalten. Es handelte
                  sich um folgendes zweiteiliges Feuilleton\pwindex{Meyer-Benfey, Heinrich 14.\,3.\,1869 Liebenburg – 30.\,12.\,1945 Buxtehude@\textsc{Meyer-Benfey, Heinrich} (14.\,3.\,1869 Liebenburg – 30.\,12.\,1945 Buxtehude), \emph{Germanist}!Moderne Religion@\strich\emph{Moderne Religion}|pwkv} von Heinrich
                     Meyer-Benfey\pwindex{Meyer-Benfey, Heinrich 14.\,3.\,1869 Liebenburg – 30.\,12.\,1945 Buxtehude@\textsc{Meyer-Benfey, Heinrich} (14.\,3.\,1869 Liebenburg – 30.\,12.\,1945 Buxtehude), \emph{Germanist}|pwk}: \emph{Moderne Religion}\pwindex{Meyer-Benfey, Heinrich 14.\,3.\,1869 Liebenburg – 30.\,12.\,1945 Buxtehude@\textsc{Meyer-Benfey, Heinrich} (14.\,3.\,1869 Liebenburg – 30.\,12.\,1945 Buxtehude), \emph{Germanist}!Moderne Religion@\strich\emph{Moderne Religion}|pwk}. In: \emph{Frankfurter Zeitung und Handelsblatt}\pwindex{Frankfurter Zeitung@\emph{Frankfurter Zeitung}|pwk}, Jg. 46, Nr. 50,
                        19. 2. 1902, Erstes Morgenblatt, S. 1–3, und
                     Nr. 51, 20. 2. 1902, Erstes Morgenblatt,
                     S. 1–3.}}}\label{K_L03198-14}{ }\strikeout{de\textcolor{gray}{r}} der Frankfurter Ztg.\pwindex{Frankfurter Zeitung@\emph{Frankfurter Zeitung}|pw} über »Moderne Religion\pwindex{Meyer-Benfey, Heinrich 14.\,3.\,1869 Liebenburg – 30.\,12.\,1945 Buxtehude@\textsc{Meyer-Benfey, Heinrich} (14.\,3.\,1869 Liebenburg – 30.\,12.\,1945 Buxtehude), \emph{Germanist}!Moderne Religion@\strich\emph{Moderne Religion}|pwv}«, die mich
               zum Nachdenken {\pb}ſehr angeregt haben.\pend
           
\pstart
           Schreib’ mir bald, grüße die Mädels\pwindex{Schnitzler, Olga 17.\,1.\,1882 Wien – 13.\,1.\,1970 Lugano@\textsc{Schnitzler, Olga} (17.\,1.\,1882 Wien – 13.\,1.\,1970 Lugano), \emph{Schauspielerin, Sängerin}|pwv}\pwindex{Steinrück, Elisabeth 19.\,11.\,1885 – 7.\,4.\,1920 Partenkirchen@\textsc{Steinrück, Elisabeth} (19.\,11.\,1885 – 7.\,4.\,1920 Partenkirchen)|pwv} und{ }ſei{ }ſelbſt vielmals und von Herzen
               gegrüßt! {\\[\baselineskip]}Dein {\\[\baselineskip]}\spacefill\mbox{Paul Goldm}\pend
           \leftskip=0em{}\selectlanguage{ngerman}\endnumbering\briefempfaengerindex{Schnitzler, Arthur@\textsc{Schnitzler, Arthur}!zzzGoldmann, Paul@\emph{von Paul Goldmann}!1902-02-251@{25. 2. [1902]}|)be}\mylabel{L03198h}  \newcommand{\dateiname}{L03198}\newcommand{\titel}{Paul Goldmann an Arthur Schnitzler, 25. 2. [1902]}\newcommand{\editorInnen}{Martin Anton Müller und Laura Untner}%% latex-leseansicht-abspann.tex
%% Abspann für die Leseansicht.
%% Der Schalter \ifkorrekturansicht ist bereits durch den Vorspann gesetzt.

%% latex-abspann.tex
%% Gemeinsamer Abspann für Korrekturansicht und Leseansicht.
%% Setzt den Schalter \ifkorrekturansicht voraus (gesetzt in den
%% einbindenden Dateien latex-korrekturansicht-abspann.tex bzw.
%% latex-leseansicht-abspann.tex).
%% ---------------------------------------------------------------

\normalsize

% Das esempio-Environment wird nur in der Leseansicht benötigt
\ifkorrekturansicht\else
\newenvironment{esempio}[3]%
{
    \vspace{1.5ex}
    \rlap{\underline{#1}}
    \par
    \setlength{\parindent}{0cm}
    \nopagebreak
    \leftskip=#2cm
    \rightskip=#3cm
}
{
    \par
}
\fi

\doendnotes{C}
\bigskip
\vfill

\clearpage

\footnotesize

\ifkorrekturansicht
  \lohead{\textsc{register}}
\fi

% theindex-Environment neu definieren ohne reledmac
\makeatletter
\renewenvironment{theindex}{%
  \ifkorrekturansicht
    \section*{\indexname}%
  \else
    \subsubsection*{Index der erwähnten Entitäten}%
  \fi
  \setlength{\parindent}{0pt}%
  \setlength{\parskip}{0pt plus 0.3pt}%
  \let\item\@idxitem
}{%
  \ifkorrekturansicht\clearpage\fi
}
\makeatother

\IfFileExists{\jobname-pw.ind}{\input{\jobname-pw.ind}}{}

% Quellenangabe nur in der Leseansicht
\ifkorrekturansicht\else
% Fallback-Definitionen, falls die .tex-Datei \titel etc. nicht gesetzt hat
\providecommand{\titel}{}
\providecommand{\editorInnen}{}
\providecommand{\dateiname}{\jobname}

\vspace{3cm}

\vfill

\footnotesize
\textsc{Quelle}: \titel. Herausgegeben von {\editorInnen}. In: \emph{Arthur Schnitzler: Briefwechsel mit Autorinnen und Autoren}.
 Digitale Edition, https://schnitzler-briefe.acdh.oeaw.ac.at/{\dateiname}.html (Stand \today)
\fi

\end{document}


