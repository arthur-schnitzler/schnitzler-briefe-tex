%% latex-leseansicht-vorspann.tex
%% Vorspann für die Leseansicht.
%% Lädt die gemeinsame Datei latex-vorspann.tex mit nicht gesetztem Schalter.

\newif\ifkorrekturansicht
\korrekturansichtfalse

\input{../tex-inputs/latex-vorspann}

\begin{center}
            \textcolor{red}{ENTWURF, NICHT FERTIG KORRIGIERT}
                      \end{center}
            
         
         \renewcommand{\erwaehntePersonen}{Personen: Richard Beer-Hofmann, Ignacy Daszyński, Felix Dörmann, Paul Goldmann, Hans Jæger, Max Koch, Fedor Mamroth, Géza von Mattachich, Heinrich Meyer-Benfey, Olga Schnitzler, Elisabeth Steinrück}
         \renewcommand{\erwaehnteInstitutionen}{Institutionen: Die Zeit, Die Zeit. Wiener Wochenschrift, Houghton Library, Neue Freie Presse, Wiener Verlag}
         \renewcommand{\erwaehnteOrte}{Orte: Berlin, Dessauer Straße, Deutschland, Hauptstraße 56, Hinterbrühl, Salzburg, Sächsische Schweiz, Südtirol, Wien, Wiesbaden}
         \renewcommand{\erwaehnteWerke}{Werke: ?? [Buch über den Talmud], ?? [Kritik zu Lebendige Stunden], ?? [Rede über die Mattachich-Affaire], Berliner Theater. (»Heilmar« von Wilhelm Kienzl im königlichen Opernhause.), Berliner Theater. »Der Herr von Abadessa« von Felix Dörmann im Königlichen Schauspielhause, Christiania-Bohême, Der Herr von Abadessa. Ein Abenteurerstreich in drei Akten, Fra Kristiania-Bohêmen, Frankfurter Zeitung, Lebendige Stunden. Vier Einakter, Lieutenant Gustl. Novelle, Moderne Religion, Neue Freie Presse}
               \section[ Paul Goldmann an Arthur Schnitzler, 25. 2. {[}1902{]}]{ Paul Goldmann an Arthur Schnitzler, 25. 2. {[}1902{]}}\nopagebreak\mylabel{v}\rehead{ }\begin{ledgroupsized}[t]{13cm}\normalsize\beginnumbering\briefempfaengerindex{Schnitzler, Arthur@\textsc{Schnitzler, Arthur}!zzzGoldmann, Paul@\emph{von Paul Goldmann}!1902-02-251@{25. 2. {[}1902{]}}|(be} \toendnotes[C]{\smallbreak\pagebreak[2]} \Standort{DLA, A:Schnitzler, HS.NZ85.1.3172.}
\physDesc{Brief, 2 Blätter, 8 Seiten, 3649 Zeichen
\newline{}Handschrift: blaue Tinte, deutsche Kurrent
\newline{}Schnitzler: 1) mit Bleistift das Jahr »902« vermerkt  2) mit rotem Buntstift fünf Unterstreichungen}\toendnotes[C]{\smallbreak}\pstart
           \noindent{}\raggedleft{}{\pb}\textcolor{gray}{\textbf{DESSAUERSTRASSE 19}}\oindex{Dessauer Strasse@\textbf{Dessauer Straße}|pw}\pend
           \pstart
           Berlin\oindex{Berlin@\textbf{Berlin}|pw}, 25. Februar.\pend
           \pstart{}Mein lieber Freund,\pend\pstart
           Ich komme leider erſt heut dazu, Deinen lieben Brief
               zu beantworten, der mir große Freude bereitet hat, weil er mir wieder einmal
               eingehenderen Bericht über Dein Ergehen gab. Ich habe eine ganze Woche lang an einem
                  \label{K_L03198-1v}\edtext{Feuilleton\pwindex{Goldmann, Paul 31.01.1865 – 25.09.1935@\textsc{Goldmann, Paul} (31.01.1865 – 25.09.1935), \emph{Schriftsteller, Journalist}!Berliner Theater. »Der Herr von Abadessa« von Felix Doermann im Koeniglichen Schauspielhause1902-02-25@\strich\emph{Berliner Theater. »Der Herr von Abadessa« von Felix Dörmann im Königlichen Schauspielhause} {[}1902-02-25{]}|pwv} über den »Herrn von \textsc{Abadessa}\pwindex{Doermann, Felix 29.05.1870 – 26.10.1928@\textsc{Dörmann, Felix} (29.05.1870 – 26.10.1928), \emph{Schriftsteller}!Herr von Abadessa. Ein Abenteurerstreich in drei Akten1901@\strich\emph{Der Herr von Abadessa. Ein Abenteurerstreich in drei Akten} {[}1901{]}|pw}«}{\lemma{\textnormal{\emph{Feuilleton … Abadessa«}}}\Cendnote{\textnormal{Paul Goldmann\pwindex{Goldmann, Paul 31.01.1865 – 25.09.1935@\textsc{Goldmann, Paul} (31.01.1865 – 25.09.1935), \emph{Schriftsteller, Journalist}|pwk}: \emph{Berliner Theater. »Der Herr von Abadessa« von Felix Dörmann
                        im Königlichen Schauspielhause}\pwindex{Goldmann, Paul 31.01.1865 – 25.09.1935@\textsc{Goldmann, Paul} (31.01.1865 – 25.09.1935), \emph{Schriftsteller, Journalist}!Berliner Theater. »Der Herr von Abadessa« von Felix Doermann im Koeniglichen Schauspielhause1902-02-25@\strich\emph{Berliner Theater. »Der Herr von Abadessa« von Felix Dörmann im Königlichen Schauspielhause} {[}1902-02-25{]}|pwk}. In: \emph{Neue Freie Presse}\pwindex{Neue Freie Presse1864 – 1939@\emph{Neue Freie Presse} {[}1864 – 1939{]}|pwk}, Nr. 13.472, 25. 2. 1902, Morgenblatt, S. 1–4.}}}\label{K_L03198-1h} (bezüglich deſſen
               ich \label{K_L03198-2v}\edtext{Deine Anſicht}{\lemma{\textnormal{\emph{Deine Anſicht}}}\Cendnote{\textnormal{Schnitzler\pwindex{Schnitzler, Arthur 15.05.1862 – 21.10.1931@\textsc{Schnitzler, Arthur} (15.05.1862 – 21.10.1931), \emph{Schriftsteller, Mediziner}|pwk} fand es schlecht, vgl. A. S.: \emph{Tagebuch}, 17. 12. 1901}}}\label{K_L03198-2h} vollſtändig theile) geſchrieben und zu nichts Anderem Zeit gefunden. Jetzt
               fürchte ich, daß die Rieſenarbeit vergeblich geweſen iſt, weil ich ſehr ſcharf über
                  \textsc{Dörmann\pwindex{Doermann, Felix 29.05.1870 – 26.10.1928@\textsc{Dörmann, Felix} (29.05.1870 – 26.10.1928), \emph{Schriftsteller}|pw}} abgeurtheilt habe und weil man mir kaum erlauben {\pb}wird, über einen früheren Mitarbeiter der N. Fr. Pr.\orgindex{Neue Freie Presse@Neue Freie Presse|pw} ſcharf zu urtheilen.\pend
           \pstart
           Es freut mich ſehr, zu hören, daß es \label{K_L03198-3v}\edtext{\textsc{Olga\pwindex{Schnitzler, Olga 17.01.1882 – 13.01.1970@\textsc{Schnitzler, Olga} (17.01.1882 – 13.01.1970), \emph{Schauspielerin, Sängerin}|pw}} beſſer geht}{\lemma{\textnormal{\emph{Olga beſſer geht}}}\Cendnote{\textnormal{siehe Paul Goldmann an Arthur Schnitzler, 16. 1. [1902]}}}\label{K_L03198-3h}. Nächſtens ſchreibe ich ihr wirklich. Ich zweifle nicht, daß dieſe Ausſicht
               die Beſſerung im Befinden der verehrten Freundin\pwindex{Schnitzler, Olga 17.01.1882 – 13.01.1970@\textsc{Schnitzler, Olga} (17.01.1882 – 13.01.1970), \emph{Schauspielerin, Sängerin}|pwv} beſchleunigen wird. Wie unendlich gern ich im März mit Euch\pwindex{Schnitzler, Olga 17.01.1882 – 13.01.1970@\textsc{Schnitzler, Olga} (17.01.1882 – 13.01.1970), \emph{Schauspielerin, Sängerin}|pwv} in die \label{K_L03198-4v}\edtext{Berge\oindex{Hinterbruehl@\textbf{Hinterbrühl}|pwv}}{\lemma{\textnormal{\emph{Berge}}}\Cendnote{\textnormal{Mit einigen Unterbrechungen hielten sich
                     Schnitzler\pwindex{Schnitzler, Arthur 15.05.1862 – 21.10.1931@\textsc{Schnitzler, Arthur} (15.05.1862 – 21.10.1931), \emph{Schriftsteller, Mediziner}|pwk}, die schwangere Olga Gussmann\pwindex{Schnitzler, Olga 17.01.1882 – 13.01.1970@\textsc{Schnitzler, Olga} (17.01.1882 – 13.01.1970), \emph{Schauspielerin, Sängerin}|pwk} und womöglich auch deren
                  Schwester Elisabeth Gussmann\pwindex{Steinrueck, Elisabeth 19.11.1885 – 07.04.1920@\textsc{Steinrück, Elisabeth} (19.11.1885 – 07.04.1920)|pwk} zwischen 21. 3. 1902 und 31. 3. 1902 in der
                  neuen Unterkunft\oindex{Hauptstrasse 56@\textbf{Hauptstraße 56}|pwkv} in der
                     Hinterbrühl\oindex{Hinterbruehl@\textbf{Hinterbrühl}|pwk} auf. Siehe Paul Goldmann an Arthur Schnitzler, 14. 1. [1902].}}}\label{K_L03198-4h} gehen möchte,
               brauche ich nicht erſt zu ſagen. Ich habe die ganze Reiſe bereits in der Phantaſie
               gemacht und dabei ſehr ſchöne Stunden mit Euch\pwindex{Schnitzler, Olga 17.01.1882 – 13.01.1970@\textsc{Schnitzler, Olga} (17.01.1882 – 13.01.1970), \emph{Schauspielerin, Sängerin}|pwv} verlebt. In der Wirklichkeit werde ich ſie nicht machen
               können. Ich könnte höchſtens zu Oſtern ein paar Tage fort. Und der Weg von hier nach
                  Salzburg\oindex{Salzburg@\textbf{Salzburg}|pw} oder gar {\pb}nach Südtirol\oindex{Suedtirol@\textbf{Südtirol}|pw}
               iſt für die drei oder vier Tage Urlaub, die ich mir nehmen könnte, allzu weit. Etwas
               Anderes wäre \substVorne{}\textsuperscript{ich}\substDazwischen{}es\substHinten{}, wenn Ihr\pwindex{Schnitzler, Olga 17.01.1882 – 13.01.1970@\textsc{Schnitzler, Olga} (17.01.1882 – 13.01.1970), \emph{Schauspielerin, Sängerin}|pwv} nach Deutſchland\oindex{Deutschland@\textbf{Deutschland}|pw} kommen könntet (Sächſiſche Schweiz\oindex{Saechsische Schweiz@\textbf{Sächsische Schweiz}|pw}\strikeout{,} oder Wiesbaden\oindex{Wiesbaden@\textbf{Wiesbaden}|pw}). Da könnte ich um Oſtern herum ein paar Tage mit Euch\pwindex{Schnitzler, Olga 17.01.1882 – 13.01.1970@\textsc{Schnitzler, Olga} (17.01.1882 – 13.01.1970), \emph{Schauspielerin, Sängerin}|pw} ſein. Aber daran iſt ja wohl kaum zu denken. Ich
               wenigſtens würde ſicher nicht nach \textsc{Wiesbaden\oindex{Wiesbaden@\textbf{Wiesbaden}|pw}} kommen, wenn ich nach Südtirol\oindex{Suedtirol@\textbf{Südtirol}|pw} gehen
               könnte.\pend
           \pstart
           In der \label{K_L03198-5v}\edtext{Affaire \textsc{Matassich\pwindex{Mattachich, Geza von 1867-12-19 – 1923-09-29@\textsc{Mattachich, Géza von} (1867-12-19 – 1923-09-29), \emph{Oberstleutnant}|pw}}}{\lemma{\textnormal{\emph{Affaire Matassich}}}\Cendnote{\textnormal{siehe Paul Goldmann an Arthur Schnitzler, 10. 2. [1902]}}}\label{K_L03198-5h} haſt Du vollkommen Recht. Es war bei mir nur ſo eine Regung, als ich die Rede\pwindex{Daszyński, Ignacy 1866-10-26 – 1936-10-31@\textsc{Daszyński, Ignacy} (1866-10-26 – 1936-10-31), \emph{Politiker, Ministerpräsident}!?? [Rede ueber die Mattachich-Affaire]1902-02-08@\strich\emph{?? [Rede über die Mattachich-Affaire]} {[}1902-02-08{]}|pwv}{ }\textsc{Daszinsky\pwindex{Daszyński, Ignacy 1866-10-26 – 1936-10-31@\textsc{Daszyński, Ignacy} (1866-10-26 – 1936-10-31), \emph{Politiker, Ministerpräsident}|pw}s} las. {\pb}Namentlich ſchien es mir, es ſei für Dich eine
               ſchöne Gelegenheit, Dich bei den Herrn für die \label{K_L03198-6v}\edtext{Entziehung der Charge}{\lemma{\textnormal{\emph{Entziehung der Charge}}}\Cendnote{\textnormal{Bezug auf die \emph{Lieutenant
                     Gustl}\pwindex{Schnitzler, Arthur 15.05.1862 – 21.10.1931@\textsc{Schnitzler, Arthur} (15.05.1862 – 21.10.1931), \emph{Schriftsteller, Mediziner}!Lieutenant Gustl. Novelle1900-12-25@\strich\emph{Lieutenant Gustl. Novelle} {[}1900-12-25{]}|pwk}-Affaire, siehe Paul Goldmann an Arthur Schnitzler, [20. 6. 1901]}}}\label{K_L03198-6h} zu revanchiren. Du weißt, ich bin rachſüchtig. Jetzt bin ich ſehr zufrieden,
               daß Du von der gefährlichen Geſchichte die Hände wegläßt.\pend
           \pstart
           Die »Lebendigen Stunden\pwindex{Schnitzler, Arthur 15.05.1862 – 21.10.1931@\textsc{Schnitzler, Arthur} (15.05.1862 – 21.10.1931), \emph{Schriftsteller, Mediziner}!Lebendige Stunden. Vier Einakter1901-12-23@\strich\emph{Lebendige Stunden. Vier Einakter} {[}1901-12-23{]}|pw}« werden ſich hoffentlich
               in der nächſten Saiſon über die deutſchen Bühnen bewegen. Vielleicht iſt die ſchon
               vorgerückte Saiſon daran ſchuld, daß es einſtweilen nicht recht vorwärts geht. In der
                  Berlin\oindex{Berlin@\textbf{Berlin}|pw}er Geſellſchaft höre ich überall mit
               Entzücken davon ſprechen. {\pb}\label{K_L03198-7v}\edtext{\textsc{Koch\pwindex{Koch, Max 22.12.1855 – 19.12.1931@\textsc{Koch, Max} (22.12.1855 – 19.12.1931), \emph{Literarhistoriker}|pw}s}{ }Kritik\pwindex{Koch, Max 22.12.1855 – 19.12.1931@\textsc{Koch, Max} (22.12.1855 – 19.12.1931), \emph{Literarhistoriker}!?? [Kritik zu Lebendige Stunden]zwischen 5. 1. und 24. 2. 1902@\strich\emph{?? [Kritik zu Lebendige Stunden]} {[}zwischen 5. 1. und 24. 2. 1902{]}|pwv}}{\lemma{\textnormal{\emph{Kochs Kritik}}}\Cendnote{\textnormal{nicht nachgewiesen}}}\label{K_L03198-7h} ſende ich Dir anbei zurück. Es
               freut mich, daß ſie ſo günſtig ausgefallen iſt. \strikeout{\textcolor{gray}{×}\-\textcolor{gray}{×}\-\textcolor{gray}{×}\-\textcolor{gray}{×}\-\textcolor{gray}{×}\-\textcolor{gray}{×}\-\textcolor{gray}{×}\-\textcolor{gray}{×}\-\textcolor{gray}{×}\-\textcolor{gray}{×}\-\textcolor{gray}{×}\-\textcolor{gray}{×}\-\textcolor{gray}{×}\-\textcolor{gray}{×}\-\textcolor{gray}{×}\-\textcolor{gray}{×}} Sonſt ſcheint mir dieſer Kritiker\pwindex{Koch, Max 22.12.1855 – 19.12.1931@\textsc{Koch, Max} (22.12.1855 – 19.12.1931), \emph{Literarhistoriker}|pwv} ein recht unbedeutender Kopf zu ſein.\pend
           \pstart
           Ich danke Dir für Deine freundlichen Worte über mein \label{K_L03198-8v}\edtext{Opern-Feuilleton\pwindex{Goldmann, Paul 31.01.1865 – 25.09.1935@\textsc{Goldmann, Paul} (31.01.1865 – 25.09.1935), \emph{Schriftsteller, Journalist}!Berliner Theater. (»Heilmar« von Wilhelm Kienzl im koeniglichen Opernhause.)1902-02-11@\strich\emph{Berliner Theater. (»Heilmar« von Wilhelm Kienzl im königlichen Opernhause.)} {[}1902-02-11{]}|pwv}}{\lemma{\textnormal{\emph{Opern-Feuilleton}}}\Cendnote{\textnormal{Paul Goldmann\pwindex{Goldmann, Paul 31.01.1865 – 25.09.1935@\textsc{Goldmann, Paul} (31.01.1865 – 25.09.1935), \emph{Schriftsteller, Journalist}|pwk}: \emph{Berliner Theater. (»Heilmar« von Wilhelm Kienzl im
                        königlichen Opernhause.)}\pwindex{Goldmann, Paul 31.01.1865 – 25.09.1935@\textsc{Goldmann, Paul} (31.01.1865 – 25.09.1935), \emph{Schriftsteller, Journalist}!Berliner Theater. (»Heilmar« von Wilhelm Kienzl im koeniglichen Opernhause.)1902-02-11@\strich\emph{Berliner Theater. (»Heilmar« von Wilhelm Kienzl im königlichen Opernhause.)} {[}1902-02-11{]}|pwk}. In: \emph{Neue
                        Freie Presse}\pwindex{Neue Freie Presse1864 – 1939@\emph{Neue Freie Presse} {[}1864 – 1939{]}|pwk}, Nr. 13.458, 11. 2. 1902,
                     Morgenblatt, S. 1–4.}}}\label{K_L03198-8h} und halte Deine Ausſtellung bezüglich der
               allzu großen Länge einzelner Abſätze für nur zu berechtigt. Ich fühle es ſelber, daß
               es mein ſchwerſter ſchriftſtelleriſcher Fehler iſt, nicht kurz ſein zu können. Aber
               beim Schreiben werde ich von einem beinahe krankhaften Drang befallen, Alles bis auf
               den Grund auszuſchöpfen. {\pb}Daher kommen die Längen,
               über die ich dann erſchreckt bin, wenn ich die Arbeit gedruckt ſehe. Wie lernt man,
               kurz zu ſein? Kannſt Du mir nicht ein Mittel ſagen?\pend
           \pstart
           Mein Onkel\pwindex{Mamroth, Fedor 21.02.1851 – 25.06.1907@\textsc{Mamroth, Fedor} (21.02.1851 – 25.06.1907), \emph{Journalist, Kritiker}|pwv} ſchreibt mir mit
               höchſtem Enthuſiasmus von einem im Wiener Verlag\orgindex{Wiener Verlag@Wiener Verlag|pw}
               erſchienenen Buch \label{K_L03198-9v}\edtext{»Chriſtiania-\textsc{Bohême}\pwindex{Jæger, Hans 1854-09-02 – 1910-02-08@\textsc{Jæger, Hans} (1854-09-02 – 1910-02-08), \emph{Anarchist, Autor}!Christiania-Bohême1902@\strich\emph{Christiania-Bohême} {[}1902{]}|pwv}«}{\lemma{\textnormal{\emph{»Chriſtiania-Bohême«}}}\Cendnote{\textnormal{Hans Jæger\pwindex{Jæger, Hans 1854-09-02 – 1910-02-08@\textsc{Jæger, Hans} (1854-09-02 – 1910-02-08), \emph{Anarchist, Autor}|pwk}: \emph{Christiania-Bohême}\pwindex{Jæger, Hans 1854-09-02 – 1910-02-08@\textsc{Jæger, Hans} (1854-09-02 – 1910-02-08), \emph{Anarchist, Autor}!Christiania-Bohême1902@\strich\emph{Christiania-Bohême} {[}1902{]}|pwk}. Wien\oindex{Wien@\textbf{Wien}|pwk}: \emph{Wiener Verlag}\orgindex{Wiener Verlag@Wiener Verlag|pwk}{ }1902 (zuerst 1885, \emph{Fra Kristiania-Bohêmen}\pwindex{Jæger, Hans 1854-09-02 – 1910-02-08@\textsc{Jæger, Hans} (1854-09-02 – 1910-02-08), \emph{Anarchist, Autor}!Fra Kristiania-Bohêmen1885@\strich\emph{Fra Kristiania-Bohêmen} {[}1885{]}|pwk}).}}}\label{K_L03198-9h} von
                  \textsc{Hans Jaeger\pwindex{Jæger, Hans 1854-09-02 – 1910-02-08@\textsc{Jæger, Hans} (1854-09-02 – 1910-02-08), \emph{Anarchist, Autor}|pw}}.\pend
           \pstart
           Hörſt Du etwas von dem neuen Blatt, der \label{K_L03198-10v}\edtext{»Zeit\orgindex{Zeit@Die Zeit|pw}\orgindex{Zeit. Wiener Wochenschrift@Die Zeit. Wiener Wochenschrift|pw}«}{\lemma{\textnormal{\emph{»Zeit«}}}\Cendnote{\textnormal{siehe Paul Goldmann an Arthur Schnitzler, 16. 1. [1902]}}}\label{K_L03198-10h}?\pend
           \pstart
           Im Sommer haſt Du mir ein \label{K_L03198-11v}\edtext{Buch\pwindex{?? Werk@Nicht ermittelte Verfasserinnen und Verfasser!?? [Buch ueber den Talmud]@\emph{?? [Buch über den Talmud]}|pwv}}{\lemma{\textnormal{\emph{Buch}}}\Cendnote{\textnormal{nicht ermittelt}}}\label{K_L03198-11h} geſtohlen: das
               über den {\pb}Talmud\pwindex{?? Werk@Nicht ermittelte Verfasserinnen und Verfasser!?? [Buch ueber den Talmud]@\emph{?? [Buch über den Talmud]}|pw}. Ich brauche es und ſchreibe \introOben{}heut\introOben{} an \label{K_L03198-12v}\edtext{\textsc{Richard}\pwindex{Beer-Hofmann, Richard 1866-07-11 – 1945-09-26@\textsc{Beer-Hofmann, Richard} (1866-07-11 – 1945-09-26), \emph{Schriftsteller}|pw}}{\lemma{\textnormal{\emph{Richard}}}\Cendnote{\textnormal{Goldmann\pwindex{Goldmann, Paul 31.01.1865 – 25.09.1935@\textsc{Goldmann, Paul} (31.01.1865 – 25.09.1935), \emph{Schriftsteller, Journalist}|pwk} schrieb Beer-Hofmannn\pwindex{Beer-Hofmann, Richard 1866-07-11 – 1945-09-26@\textsc{Beer-Hofmann, Richard} (1866-07-11 – 1945-09-26), \emph{Schriftsteller}|pwk} noch am selben Tag, vgl. \emph{Houghton Library}\orgindex{Houghton Library@Houghton Library|pwk},
                     Harvard (Signatur 825.978). Dem Brief ist zu
                  entnehmen, dass Goldmann\pwindex{Goldmann, Paul 31.01.1865 – 25.09.1935@\textsc{Goldmann, Paul} (31.01.1865 – 25.09.1935), \emph{Schriftsteller, Journalist}|pwk} das Buch\pwindex{?? Werk@Nicht ermittelte Verfasserinnen und Verfasser!?? [Buch ueber den Talmud]@\emph{?? [Buch über den Talmud]}|pwkv} von Beer-Hofmann\pwindex{Beer-Hofmann, Richard 1866-07-11 – 1945-09-26@\textsc{Beer-Hofmann, Richard} (1866-07-11 – 1945-09-26), \emph{Schriftsteller}|pwk} im Sommer 1901
                  geschenkt bekommen hatte, nicht aber der Titel.}}}\label{K_L03198-12h}, er möge mir doch Titel und
               Verlag angeben, damit ich es mir kommen laſſen kann. Da ich aber dieſe Anfrage an \textsc{Richard}\pwindex{Beer-Hofmann, Richard 1866-07-11 – 1945-09-26@\textsc{Beer-Hofmann, Richard} (1866-07-11 – 1945-09-26), \emph{Schriftsteller}|pw} für ein völlig ausſichtsloſes Unternehmen halte, bitte ich Dich (wenn Du das
                  Buch\pwindex{?? Werk@Nicht ermittelte Verfasserinnen und Verfasser!?? [Buch ueber den Talmud]@\emph{?? [Buch über den Talmud]}|pwv} nicht ſelber
               brauchſt), mir es gelegentlich zu ſchicken. Iſt \textsc{Richard\pwindex{Beer-Hofmann, Richard 1866-07-11 – 1945-09-26@\textsc{Beer-Hofmann, Richard} (1866-07-11 – 1945-09-26), \emph{Schriftsteller}|pw}} wieder ganz \label{K_L03198-13v}\edtext{geſund}{\lemma{\textnormal{\emph{geſund}}}\Cendnote{\textnormal{siehe Paul Goldmann an Arthur Schnitzler, 25. 1. [1902]}}}\label{K_L03198-13h}?\pend
           \pstart
           Ich ſende Dir anbei zwei \label{K_L03198-14v}\edtext{Feuilletons\pwindex{Meyer-Benfey, Heinrich 1869-03-14 – 1945-12-30@\textsc{Meyer-Benfey, Heinrich} (1869-03-14 – 1945-12-30), \emph{Germanist}!Moderne Religion1902-02-19 – 1902-02-20@\strich\emph{Moderne Religion} {[}1902-02-19 – 1902-02-20{]}|pwv}}{\lemma{\textnormal{\emph{Feuilletons}}}\Cendnote{\textnormal{Beilage nicht erhalten. Es handelte
                  sich um folgendes zweiteiliges Feuilleton\pwindex{Meyer-Benfey, Heinrich 1869-03-14 – 1945-12-30@\textsc{Meyer-Benfey, Heinrich} (1869-03-14 – 1945-12-30), \emph{Germanist}!Moderne Religion1902-02-19 – 1902-02-20@\strich\emph{Moderne Religion} {[}1902-02-19 – 1902-02-20{]}|pwkv} von Heinrich
                     Meyer-Benfey\pwindex{Meyer-Benfey, Heinrich 1869-03-14 – 1945-12-30@\textsc{Meyer-Benfey, Heinrich} (1869-03-14 – 1945-12-30), \emph{Germanist}|pwk}: \emph{Moderne Religion}\pwindex{Meyer-Benfey, Heinrich 1869-03-14 – 1945-12-30@\textsc{Meyer-Benfey, Heinrich} (1869-03-14 – 1945-12-30), \emph{Germanist}!Moderne Religion1902-02-19 – 1902-02-20@\strich\emph{Moderne Religion} {[}1902-02-19 – 1902-02-20{]}|pwk}. In: \emph{Frankfurter Zeitung und Handelsblatt}\pwindex{?? Werk@Nicht ermittelte Verfasserinnen und Verfasser!Frankfurter Zeitung1856 – 1943@\emph{Frankfurter Zeitung} {[}1856 – 1943{]}|pwk}, Jg. 46, Nr. 50,
                        19. 2. 1902, Erstes Morgenblatt, S. 1–3 und
                     Nr. 51, 20. 2. 1902, Erstes Morgenblatt,
                     S. 1–3.}}}\label{K_L03198-14h}{ }\strikeout{de\textcolor{gray}{r}} der Frankfurter Ztg.\pwindex{?? Werk@Nicht ermittelte Verfasserinnen und Verfasser!Frankfurter Zeitung1856 – 1943@\emph{Frankfurter Zeitung} {[}1856 – 1943{]}|pw} über »Moderne Religion\pwindex{Meyer-Benfey, Heinrich 1869-03-14 – 1945-12-30@\textsc{Meyer-Benfey, Heinrich} (1869-03-14 – 1945-12-30), \emph{Germanist}!Moderne Religion1902-02-19 – 1902-02-20@\strich\emph{Moderne Religion} {[}1902-02-19 – 1902-02-20{]}|pwv}«, die mich
               zum Nachdenken {\pb}ſehr angeregt haben.\pend
           \pstart
           Schreib’ mir bald, grüße die Mädels\pwindex{Schnitzler, Olga 17.01.1882 – 13.01.1970@\textsc{Schnitzler, Olga} (17.01.1882 – 13.01.1970), \emph{Schauspielerin, Sängerin}|pwv}\pwindex{Steinrueck, Elisabeth 19.11.1885 – 07.04.1920@\textsc{Steinrück, Elisabeth} (19.11.1885 – 07.04.1920)|pwv} und ſei ſelbſt vielmals und von Herzen
               gegrüßt! {\\[\baselineskip]}Dein {\\[\baselineskip]}\spacefill\mbox{Paul Goldmnn}\pend
           \leftskip=0em{}
         
         \endnumbering\mylabel{h}\end{ledgroupsized}\begin{anhang}\end{anhang}\newcommand{\dateiname}{L03198}\newcommand{\titel}{Paul Goldmann an Arthur Schnitzler, 25. 2. [1902]}\newcommand{\editorInnen}{Martin Anton Müller und Laura Untner}%% latex-leseansicht-abspann.tex
%% Abspann für die Leseansicht.
%% Der Schalter \ifkorrekturansicht ist bereits durch den Vorspann gesetzt.

%% latex-abspann.tex
%% Gemeinsamer Abspann für Korrekturansicht und Leseansicht.
%% Setzt den Schalter \ifkorrekturansicht voraus (gesetzt in den
%% einbindenden Dateien latex-korrekturansicht-abspann.tex bzw.
%% latex-leseansicht-abspann.tex).
%% ---------------------------------------------------------------

\normalsize

% Das esempio-Environment wird nur in der Leseansicht benötigt
\ifkorrekturansicht\else
\newenvironment{esempio}[3]%
{
    \vspace{1.5ex}
    \rlap{\underline{#1}}
    \par
    \setlength{\parindent}{0cm}
    \nopagebreak
    \leftskip=#2cm
    \rightskip=#3cm
}
{
    \par
}
\fi

\doendnotes{C}
\bigskip
\vfill

\clearpage

\footnotesize

\ifkorrekturansicht
  \lohead{\textsc{register}}
\fi

% theindex-Environment neu definieren ohne reledmac
\makeatletter
\renewenvironment{theindex}{%
  \ifkorrekturansicht
    \section*{\indexname}%
  \else
    \subsubsection*{Index der erwähnten Entitäten}%
  \fi
  \setlength{\parindent}{0pt}%
  \setlength{\parskip}{0pt plus 0.3pt}%
  \let\item\@idxitem
}{%
  \ifkorrekturansicht\clearpage\fi
}
\makeatother

\IfFileExists{\jobname-pw.ind}{\input{\jobname-pw.ind}}{}

% Quellenangabe nur in der Leseansicht
\ifkorrekturansicht\else
% Fallback-Definitionen, falls die .tex-Datei \titel etc. nicht gesetzt hat
\providecommand{\titel}{}
\providecommand{\editorInnen}{}
\providecommand{\dateiname}{\jobname}

\vspace{3cm}

\vfill

\footnotesize
\textsc{Quelle}: \titel. Herausgegeben von {\editorInnen}. In: \emph{Arthur Schnitzler: Briefwechsel mit Autorinnen und Autoren}.
 Digitale Edition, https://schnitzler-briefe.acdh.oeaw.ac.at/{\dateiname}.html (Stand \today)
\fi

\end{document}


      