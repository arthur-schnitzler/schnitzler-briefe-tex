%% latex-leseansicht-vorspann.tex
%% Vorspann für die Leseansicht.
%% Lädt die gemeinsame Datei latex-vorspann.tex mit nicht gesetztem Schalter.

\newif\ifkorrekturansicht
\korrekturansichtfalse

\input{../tex-inputs/latex-vorspann}


         
         \renewcommand{\erwaehntePersonen}{Personen: Hermann Bahr, Alfred Deutsch-German, Markus Hajek, Felix Salten, Isidor Singer}
         \renewcommand{\erwaehnteInstitutionen}{Institutionen: Die Zeit, Die Zeit. Wiener Wochenschrift, Neues Wiener Journal}
         \renewcommand{\erwaehnteOrte}{Orte: Kuranstalt Dr. Konried, Reichenau an der Rax, Wien}
         \renewcommand{\erwaehnteWerke}{Werke: Die Zeit, Neues Wiener Journal, Reigen. Zehn Dialoge, Rezensionen. Wiener Theater 1901 bis 1903}
               \section[Arthur Schnitzler an Hermann Bahr, 6. 4. 1903]{ Arthur Schnitzler an Hermann Bahr, 6. 4. 1903}\nopagebreak\mylabel{v}\rehead{ }\begin{ledgroupsized}[t]{13cm}\normalsize\beginnumbering \toendnotes[C]{\smallbreak\pagebreak[2]} \Standort{TMW, HS AM 23354 Ba.}
\physDesc{Brief, 2 Blätter, 7 Seiten
\newline{}Handschrift: schwarze Tinte, deutsche Kurrent\newline{}Ordnung: Lochung }\buchAbdrucke{\weitereDrucke{1) Arthur Schnitzler: \emph{Briefe 1875–1912}. Hg. Therese Nickl und Heinrich Schnitzler. Frankfurt am Main: \emph{S. Fischer} 1981, S. 458–460.} \weitereDrucke{2) \emph{6. 4. 1903.} In: Arthur Schnitzler: \emph{The Letters of Arthur Schnitzler to Hermann Bahr}. Edited, annotated, and with an introduction, by Donald G.
                        Daviau. Chapel Hill: \emph{The University of North Carolina Press} 1978, S. 77–78 (University of North Carolina studies in the Germanic languages
                        and literatures, 89).} \weitereDrucke{3) Hermann Bahr, Arthur Schnitzler: \emph{Briefwechsel, Aufzeichnungen, Dokumente (1891–1931)}. Hg. Kurt Ifkovits und Martin Anton Müller. Göttingen: \emph{Wallstein} 2018, S. 264–265.} }\toendnotes[C]{\smallbreak}\pstart
           \raggedleft{}{\pb}Wien\oindex{Wien@\textbf{Wien}|pw}, 6. 4. 903.\pend
           \pstart{}lieber Hermann,\pend\pstart
           ich glaube wir befinden uns beide in einer ſehr ähnlichen Situation der
               Oeffentlichkeit gegenüber: was immer wir thun oder unterlaſſen werden – eine
               compact-vertrackte Majorität wird ſchimpfen. Es wird alſo immer notwendiger find ich
               ſich ausſchließlich nach dem zu richten, was wir ſelbſt für das vernünftige halten –
               auf die Gefahr hin dſs wir uns ge{\pb}legentlich irren.
               Willſt du mir deinen neuen Band\pwindex{Bahr, Hermann 19.07.1863 – 15.01.1934@\textsc{Bahr, Hermann} (19.07.1863 – 15.01.1934), \emph{Schriftsteller, Kritiker}!Rezensionen. Wiener Theater 1901 bis 19031903@\strich\emph{Rezensionen. Wiener Theater 1901 bis 1903} {[}1903{]}|pwv}
               widmen, ſo ſeh ich darin nichts andres als den neueſten Ausdruck für die Herzlichkeit
               unsrer Beziehungen, zu der wir uns ja wahrhaftig ſchwer genug durchgerungen haben.
               Ich freu mich nun umſo mehr, daſs wir ſo weit ſind daſs wir einander wirklich
               verſtehen und – was in dieſen Jahren {\pb}doch eigentlich recht
               ſelten vorko{\geminationm}t, uns – ich ſchließe von mir wohl nicht
               ganz verfehlt auf dich – einander jenſeits von Literatur und allerlei Getriebe – gern
               haben. \uline{Ich} für meinen Theil nehme alſo die Gefahr auf
               mich, neuerdings als mit dir vercliquet angeſehen zu werden, \introOben{}–\introOben{} (ob\substVorne{}\textsuperscript{ſ}\substDazwischen{}z\substHinten{}war ich nachweiſen könnte, daſs ich nie eine lobende Kritik über dich
               geſchrieben habe) – und {\pb}mehr als das – ich danke dir aufrichtg für deine liebenswürdg Abſicht. Eine Bitte
               füg ich bei, obwohl ſie recht überflüſſig ſein dürfte: ſage mir nichts »freundliches«
               oder »ſchönes« in deinem Widmungswort\pwindex{Bahr, Hermann 19.07.1863 – 15.01.1934@\textsc{Bahr, Hermann} (19.07.1863 – 15.01.1934), \emph{Schriftsteller, Kritiker}!Rezensionen. Wiener Theater 1901 bis 19031903@\strich\emph{Rezensionen. Wiener Theater 1901 bis 1903} {[}1903{]}|pwv}. Die Thatſache der Zu\substVorne{}\textsuperscript{n}\substDazwischen{}ei\substHinten{}gnung allein iſt mir Freude genug.\pend
           \pstart
           Eben erſt merke ich, daſs du mir auf einer Extraſeite den Wortlaut der Widmung schon
               mitgetheilt haſt. Sie iſt einfach und ſchön. Ich danke dir.\pend
           \pstart
           {\pb}Die Nachricht des N. Wr. Journ\pwindex{Neues Wiener Journal1893 – 1939@\emph{Neues Wiener Journal} {[}1893 – 1939{]}|pw} ist unwahr, mindeſtens um ſehr geraume
               Zeit verfrüht. Erinnerſt du dich, dſs wir gerade am Tag vorher mit einem Herrn\pwindex{Deutsch-German, Alfred 1870-09-27 – 1943@\textsc{Deutsch-German, Alfred} (1870-09-27 – 1943), \emph{Schriftsteller, Journalist}|pwv} des N. Wr. J.\orgindex{Neues Wiener Journal@Neues Wiener Journal|pw} über die Büberei geſprochen haben, die \substVorne{}\textsuperscript{die}\substDazwischen{}durch\substHinten{}{ }\substVorne{}\textsuperscript{\textcolor{gray}{den}}\substDazwischen{}die\substHinten{} journaliſtiſchen Einmiſchung ins Privatleben verübt werden? – In meinem Fall
               war es ja zufällig gleichgiltig; aber es hätte ebenſo gut eine freche Indiscretion
               ſein {[}können.{]}\pend
           \pstart
           \damage{–} Wie ſteht es mit deinen \label{K_L01287_1v}\edtext{Reiſe-
               u Erholungsplänen}{\lemma{\textnormal{\emph{Reiſe-
               u Erholungsplänen}}}\Cendnote{\textnormal{Bahr\pwindex{Bahr, Hermann 19.07.1863 – 15.01.1934@\textsc{Bahr, Hermann} (19.07.1863 – 15.01.1934), \emph{Schriftsteller, Kritiker}|pwk} hielt sich vom 18. bis
                     25. 5. in der Kuranstalt Konried in
                     Reichenau an der Rax\oindex{Kuranstalt Dr. Konried@\textbf{Kuranstalt Dr. Konried}|pwk} auf, Schnitzler\pwindex{Schnitzler, Arthur 15.05.1862 – 21.10.1931@\textsc{Schnitzler, Arthur} (15.05.1862 – 21.10.1931), \emph{Schriftsteller, Mediziner}|pwk}
                  war zu der Zeit vor allem in Wien\oindex{Wien@\textbf{Wien}|pwk}.}}}\label{K_L01287_1h}? Ich
               hoffe dich {\pb}jedenfalls
               ſehr bald zu ſehen; i{\geminationm}erhin verſtändige mich; denn ich
               möchte we{\geminationn}’s dir nicht unangenehm iſt, auch ganz gern
               ein paar Tage in die Reichenauer Gegend\oindex{Reichenau an der Rax@\textbf{Reichenau an der Rax}|pw}.\pend
           \pstart
           Zum Cap. Reigen\pwindex{Schnitzler, Arthur 15.05.1862 – 21.10.1931@\textsc{Schnitzler, Arthur} (15.05.1862 – 21.10.1931), \emph{Schriftsteller, Mediziner}!Reigen. Zehn Dialoge1900@\strich\emph{Reigen. Zehn Dialoge} {[}1900{]}|pw}: Salten\pwindex{Salten, Felix 06.09.1869 – 08.10.1945@\textsc{Salten, Felix} (06.09.1869 – 08.10.1945), \emph{Schriftsteller, Journalist}|pw} hat ſein Feuill. vorläufig in der Zeit\pwindex{Zeit1902 – 1919@\emph{Die Zeit} {[}1902 – 1919{]}|pw} auch noch nicht unterbringen können. Warum?{\dotstwo}
               Mein – Schwager\pwindex{Hajek, Markus 25.11.1861 – 04.04.1941@\textsc{Hajek, Markus} (25.11.1861 – 04.04.1941), \emph{Mediziner, Laryngologe}|pwv} war entſetzt,
               als er durch \label{K_L01287_2v}\edtext{Singer\pwindex{Singer, Isidor 16.01.1857 – 08.12.1927@\textsc{Singer, Isidor} (16.01.1857 – 08.12.1927), \emph{Journalist, Herausgeber, Soziologe}|pw}}{\lemma{\textnormal{\emph{Singer}}}\Cendnote{\textnormal{Isidor Singer\pwindex{Singer, Isidor 16.01.1857 – 08.12.1927@\textsc{Singer, Isidor} (16.01.1857 – 08.12.1927), \emph{Journalist, Herausgeber, Soziologe}|pwk}, der Herausgeber der
                  Wochenschrift und der gleichnamigen Tageszeitung \emph{Die Zeit}\orgindex{Zeit. Wiener Wochenschrift@Die Zeit. Wiener Wochenschrift|pwk}\orgindex{Zeit@Die Zeit|pwk}.}}}\label{K_L01287_2h} erfuhr, daſs von dieſem verderblichen Buch an
                  her{\pb}vorragender
               Stelle Notiz genommen werden ſolle u rieth ihm dringend ab. Singer\pwindex{Singer, Isidor 16.01.1857 – 08.12.1927@\textsc{Singer, Isidor} (16.01.1857 – 08.12.1927), \emph{Journalist, Herausgeber, Soziologe}|pw}: »Sehn Sie, ſogar der Schwager\pwindex{Hajek, Markus 25.11.1861 – 04.04.1941@\textsc{Hajek, Markus} (25.11.1861 – 04.04.1941), \emph{Mediziner, Laryngologe}|pwv}{\dots}«\pend
           \pstart
           Man ernenne doch endlich den Storch zum Ehrenbürger der Menſchheit.\pend
           \pstart
           herzlichen Gruſs{\\[\baselineskip]}dein getreuer{\\[\baselineskip]}\spacefill\mbox{Arthur}\pend
           \leftskip=0em{}
         
         \endnumbering\mylabel{h}\end{ledgroupsized}  \newcommand{\dateiname}{L01287}\newcommand{\titel}{Arthur Schnitzler an Hermann Bahr, 6. 4. 1903}\newcommand{\editorInnen}{ Kurt Ifkovits,  Martin Anton Müller}%% latex-leseansicht-abspann.tex
%% Abspann für die Leseansicht.
%% Der Schalter \ifkorrekturansicht ist bereits durch den Vorspann gesetzt.

%% latex-abspann.tex
%% Gemeinsamer Abspann für Korrekturansicht und Leseansicht.
%% Setzt den Schalter \ifkorrekturansicht voraus (gesetzt in den
%% einbindenden Dateien latex-korrekturansicht-abspann.tex bzw.
%% latex-leseansicht-abspann.tex).
%% ---------------------------------------------------------------

\normalsize

% Das esempio-Environment wird nur in der Leseansicht benötigt
\ifkorrekturansicht\else
\newenvironment{esempio}[3]%
{
    \vspace{1.5ex}
    \rlap{\underline{#1}}
    \par
    \setlength{\parindent}{0cm}
    \nopagebreak
    \leftskip=#2cm
    \rightskip=#3cm
}
{
    \par
}
\fi

\doendnotes{C}
\bigskip
\vfill

\clearpage

\footnotesize

\ifkorrekturansicht
  \lohead{\textsc{register}}
\fi

% theindex-Environment neu definieren ohne reledmac
\makeatletter
\renewenvironment{theindex}{%
  \ifkorrekturansicht
    \section*{\indexname}%
  \else
    \subsubsection*{Index der erwähnten Entitäten}%
  \fi
  \setlength{\parindent}{0pt}%
  \setlength{\parskip}{0pt plus 0.3pt}%
  \let\item\@idxitem
}{%
  \ifkorrekturansicht\clearpage\fi
}
\makeatother

\IfFileExists{\jobname-pw.ind}{\input{\jobname-pw.ind}}{}

% Quellenangabe nur in der Leseansicht
\ifkorrekturansicht\else
% Fallback-Definitionen, falls die .tex-Datei \titel etc. nicht gesetzt hat
\providecommand{\titel}{}
\providecommand{\editorInnen}{}
\providecommand{\dateiname}{\jobname}

\vspace{3cm}

\vfill

\footnotesize
\textsc{Quelle}: \titel. Herausgegeben von {\editorInnen}. In: \emph{Arthur Schnitzler: Briefwechsel mit Autorinnen und Autoren}.
 Digitale Edition, https://schnitzler-briefe.acdh.oeaw.ac.at/{\dateiname}.html (Stand \today)
\fi

\end{document}


      