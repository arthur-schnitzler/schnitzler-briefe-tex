%% latex-leseansicht-vorspann.tex
%% Vorspann für die Leseansicht.
%% Lädt die gemeinsame Datei latex-vorspann.tex mit nicht gesetztem Schalter.

\newif\ifkorrekturansicht
\korrekturansichtfalse

\input{../tex-inputs/latex-vorspann}


         
         \renewcommand{\erwaehntePersonen}{Personen:  ?? [Wärter von Peter Altenberg], Peter Altenberg, Hermann Bahr, Georg Engländer, Karl Richter}
         \renewcommand{\erwaehnteOrte}{Orte: Kurhaus Semmering, Otto-Wagner-Spital, Salzburg, Semmering, Sternwartestraße, Wien}
         \renewcommand{\erwaehnteWerke}{}
               \section[Arthur Schnitzler an Hermann Bahr, 22. 4. 1913]{ Arthur Schnitzler an Hermann Bahr, 22. 4. 1913}\nopagebreak\mylabel{v}\rehead{ }\begin{ledgroupsized}[t]{13cm}\normalsize\beginnumbering \toendnotes[C]{\smallbreak\pagebreak[2]} \Standort{TMW, HS AM 23393 Ba.}
\physDesc{Brief, 2 Blätter, 3 Seiten
\newline{}Schreibmaschine
\newline{}Handschrift: schwarze Tinte, lateinische Kurrent (\noindent{}Korrekturen, Unterschrift)\newline{}Ordnung: Lochung }\Standort{DLA, A:Schnitzler, 85.1.294/4.}
\physDesc{Brief, Durchschlag, 2 Blätter, 3 Seiten
\newline{}Schreibmaschine
\newline{}Handschrift: Bleistift, deutsche Kurrent (\noindent{}Streichung »dass der
                                    Aufenth.«)}\buchAbdrucke{\weitereDrucke{1) Arthur Schnitzler: \emph{Briefe 1913–1931}. Hg. Peter Michael Braunwarth, Richard Miklin, Susanne Pertlik und Heinrich Schnitzler. Frankfurt am Main: \emph{S. Fischer} 1984, S. 20–22.} \weitereDrucke{2) \emph{22. 4. 1913.} In: Arthur Schnitzler: \emph{The Letters of Arthur Schnitzler to Hermann Bahr}. Edited, annotated, and with an introduction, by Donald G.
                        Daviau. Chapel Hill: \emph{The University of North Carolina Press} 1978, S. 110–111 (University of North Carolina studies in the Germanic languages
                        and literatures, 89).} \weitereDrucke{3) Hermann Bahr, Arthur Schnitzler: \emph{Briefwechsel, Aufzeichnungen, Dokumente (1891–1931)}. Hg. Kurt Ifkovits und Martin Anton Müller. Göttingen: \emph{Wallstein} 2018, S. 484–485.} }\toendnotes[C]{\smallbreak}\pstart
           \noindent{}{\pb}\textcolor{gray}{\textbf{Dr. Arthur Schnitzler}}\hfill 22. 4. 1913.\pend
           \pstart
           \textcolor{gray}{\textbf{Wien XVIII. Sternwartestrasse 71\oindex{Sternwartestrasse@\textbf{Sternwartestraße}|pw}}}\pend
           \pstart{}Lieber Hermann.\pend\pstart
           Ich habe \label{K_L02129_1v}\edtext{nun Altenberg\pwindex{Altenberg, Peter 09.03.1859 – 08.01.1919@\textsc{Altenberg, Peter} (09.03.1859 – 08.01.1919), \emph{Schriftsteller}|pw}, seinen Bruder\pwindex{Englaender, Georg 03.04.1862 – 10.04.1927@\textsc{Engländer, Georg} (03.04.1862 – 10.04.1927), \emph{Privatbeamter}|pwv} und seinen Arzt\pwindex{Richter, Karl 09.03.1862 – 25.06.1937@\textsc{Richter, Karl} (09.03.1862 – 25.06.1937), \emph{Mediziner, Sanatoriumsleiter}|pwv} gesprochen}{\lemma{\textnormal{\emph{nun … gesprochen}}}\Cendnote{\textnormal{am 20. 4. 1913}}}\label{K_L02129_1h} und glaube ein klares Bild von der ganzen Sache zu haben. Altenberg\pwindex{Altenberg, Peter 09.03.1859 – 08.01.1919@\textsc{Altenberg, Peter} (09.03.1859 – 08.01.1919), \emph{Schriftsteller}|pw} ist vor zirka 4–5 Monaten wegen eines akuten
               alkoholischen Irreseins nach Steinhof\oindex{Otto-Wagner-Spital@\textbf{Otto-Wagner-Spital}|pw} gebracht
               worden. Die schweren Erscheinungen, Verfolgungsideen etc., die, erst in der Anstalt
               selbst auftraten, dürften (was mir ärztlicherseits allerdings nicht gesagt wurde) auf
               die plötzliche vollkommene Abstinenz zurückzuführen gewesen sein (die man jetzt, ich
               weiss nicht recht warum, statt der früher geübten allmählichen Entwöhnung in vielen
               Fällen anwendet). Ich habe Altenberg\pwindex{Altenberg, Peter 09.03.1859 – 08.01.1919@\textsc{Altenberg, Peter} (09.03.1859 – 08.01.1919), \emph{Schriftsteller}|pw} geistig
               frischer gefunden als seit langer Zeit, nur eben sehr erregt, weil er schon gerne auf
               den Semmering\oindex{Semmering@\textbf{Semmering}|pw} möchte. Freilich besteht die Gefahr,
               besser die Sicherheit, dass er ohne ärztliche Aufsicht sofort wieder zu trinken und
               bald auch wieder alkoholisch {\pb}zu \label{K_L02129_2v}\edtext{exzedieren}{\lemma{\textnormal{\emph{exzedieren}}}\Cendnote{\textnormal{übertreiben}}}\label{K_L02129_2h} anfängt. Diese Gefahr wird aber gerade so
               wie heute in acht Tagen, in vier Wochen und in einem halben Jahr bestehen. Dazu
               kommt, dass seine steigende Erregung wegen der Internierung in Steinhof\oindex{Otto-Wagner-Spital@\textbf{Otto-Wagner-Spital}|pw} seinem allgemeinen Zustand kaum förderlich sein dürfte.
               Dies alles habe ich auch Peter Altenbergs\pwindex{Altenberg, Peter 09.03.1859 – 08.01.1919@\textsc{Altenberg, Peter} (09.03.1859 – 08.01.1919), \emph{Schriftsteller}|pw}{ }Bruder\pwindex{Englaender, Georg 03.04.1862 – 10.04.1927@\textsc{Engländer, Georg} (03.04.1862 – 10.04.1927), \emph{Privatbeamter}|pwv} gesagt, und da \strikeout{auch} der Chefarzt\pwindex{Richter, Karl 09.03.1862 – 25.06.1937@\textsc{Richter, Karl} (09.03.1862 – 25.06.1937), \emph{Mediziner, Sanatoriumsleiter}|pwv} gegen P. A.’s\pwindex{Altenberg, Peter 09.03.1859 – 08.01.1919@\textsc{Altenberg, Peter} (09.03.1859 – 08.01.1919), \emph{Schriftsteller}|pw}
               Entlassung nichts einzuwenden hat, wenn der Bruder\pwindex{Englaender, Georg 03.04.1862 – 10.04.1927@\textsc{Engländer, Georg} (03.04.1862 – 10.04.1927), \emph{Privatbeamter}|pwv} die Verantwortung übernimmt, (man muss allerdings
               fragen, wofür?), so dürfte P. A.\pwindex{Altenberg, Peter 09.03.1859 – 08.01.1919@\textsc{Altenberg, Peter} (09.03.1859 – 08.01.1919), \emph{Schriftsteller}|pw} in wenigen Tagen
               die Reise auf den Semmering\oindex{Semmering@\textbf{Semmering}|pw} antreten können. Der Bruder\pwindex{Englaender, Georg 03.04.1862 – 10.04.1927@\textsc{Engländer, Georg} (03.04.1862 – 10.04.1927), \emph{Privatbeamter}|pwv} möchte nur, was ich sehr
               vernünftig finde, dass P. A.\pwindex{Altenberg, Peter 09.03.1859 – 08.01.1919@\textsc{Altenberg, Peter} (09.03.1859 – 08.01.1919), \emph{Schriftsteller}|pw} wenigstens anfänglich
               nicht im Hotel, sondern im Kurhaus\oindex{Kurhaus Semmering@\textbf{Kurhaus Semmering}|pw}, also unter recht
               bescheidener ärztlicher Aufsicht wohne. Für den Fall, dass sich das nicht durchführen
               liesse, wäre auch die Begleitung durch einen Wärter in Erwägung zu ziehen. P. A.\pwindex{Altenberg, Peter 09.03.1859 – 08.01.1919@\textsc{Altenberg, Peter} (09.03.1859 – 08.01.1919), \emph{Schriftsteller}|pw} möchte selbst sehr gern seinen Wärter\pwindex{?? [Waerter von Peter Altenberg] @\textsc{?? [Wärter von Peter Altenberg]}|pwv} aus dem Sanatorium für
               ein paar Tage mitnehmen, wenn dem nicht, wie es den Anschein hat, von Seiten der
               Anstalt Schwierigkei{\pb}ten entgegengesetzt würden. Es hat meiner Ansicht nach wirklich keinen Sinn Peter Altenberg\pwindex{Altenberg, Peter 09.03.1859 – 08.01.1919@\textsc{Altenberg, Peter} (09.03.1859 – 08.01.1919), \emph{Schriftsteller}|pw} länger in Steinhof\oindex{Otto-Wagner-Spital@\textbf{Otto-Wagner-Spital}|pw} zu halten, wenn auch kaum zu bezweifeln ist, dass nach
               einiger Zeit ihm ein neues Delirium und wahrscheinlich eine neuerliche Internierung,
               die ja dann der Umgebung wegen nicht zu vermeiden ist, bevorstehen dürfte. Von den
               Degenerationserscheinungen, die man nach allerlei Gerüchten hätte befürchten können
               habe ich bei Altenberg\pwindex{Altenberg, Peter 09.03.1859 – 08.01.1919@\textsc{Altenberg, Peter} (09.03.1859 – 08.01.1919), \emph{Schriftsteller}|pw} nicht das Geringste
               bemerkt, und ich glaube, wenn auch vielleicht die \label{T_L02129_1v}\edtext{\uline{plötzliche}}{\lemma{\textnormal{\emph{plötzliche}}}\Cendnote{\textnormal{handschriftliche Unterstreichung}}}\label{T_L02129_1h} Abstinenz zu
               Beginn der Anstaltsbehandlung nicht ausschliesslich von Vorteil \substVorne{}\textsuperscript{war, dass der Aufenthalt im Ganzen}{\allowbreak}\substDazwischen{}für ihn gewesen war\substHinten{}, – die geänderte Lebensweise im weiteren Verlauf und alles was damit
               zusammenhängt hat ihm sicher nur gut getan. Was natürlich kein Anlass ist den
               Aufenthalt ohne Notwendigkeit zu verlängern.\pend
           \pstart
           Herzlichen Gruss{\\[\baselineskip]}Dein{\\[\baselineskip]}\spacefill\mbox{{[}hs.:{]} Arthur}\pend
           \leftskip=0em{}\pstart
           \noindent{}{[}ms.:{]} Herrn Hermann Bahr, Salzburg\oindex{Salzburg@\textbf{Salzburg}|pw}.\pend
           
         
         \endnumbering\mylabel{h}\end{ledgroupsized}  \newcommand{\dateiname}{L02129}\newcommand{\titel}{Arthur Schnitzler an Hermann Bahr, 22. 4. 1913}\newcommand{\editorInnen}{ Kurt Ifkovits,  Martin Anton Müller}%% latex-leseansicht-abspann.tex
%% Abspann für die Leseansicht.
%% Der Schalter \ifkorrekturansicht ist bereits durch den Vorspann gesetzt.

%% latex-abspann.tex
%% Gemeinsamer Abspann für Korrekturansicht und Leseansicht.
%% Setzt den Schalter \ifkorrekturansicht voraus (gesetzt in den
%% einbindenden Dateien latex-korrekturansicht-abspann.tex bzw.
%% latex-leseansicht-abspann.tex).
%% ---------------------------------------------------------------

\normalsize

% Das esempio-Environment wird nur in der Leseansicht benötigt
\ifkorrekturansicht\else
\newenvironment{esempio}[3]%
{
    \vspace{1.5ex}
    \rlap{\underline{#1}}
    \par
    \setlength{\parindent}{0cm}
    \nopagebreak
    \leftskip=#2cm
    \rightskip=#3cm
}
{
    \par
}
\fi

\doendnotes{C}
\bigskip
\vfill

\clearpage

\footnotesize

\ifkorrekturansicht
  \lohead{\textsc{register}}
\fi

% theindex-Environment neu definieren ohne reledmac
\makeatletter
\renewenvironment{theindex}{%
  \ifkorrekturansicht
    \section*{\indexname}%
  \else
    \subsubsection*{Index der erwähnten Entitäten}%
  \fi
  \setlength{\parindent}{0pt}%
  \setlength{\parskip}{0pt plus 0.3pt}%
  \let\item\@idxitem
}{%
  \ifkorrekturansicht\clearpage\fi
}
\makeatother

\IfFileExists{\jobname-pw.ind}{\input{\jobname-pw.ind}}{}

% Quellenangabe nur in der Leseansicht
\ifkorrekturansicht\else
% Fallback-Definitionen, falls die .tex-Datei \titel etc. nicht gesetzt hat
\providecommand{\titel}{}
\providecommand{\editorInnen}{}
\providecommand{\dateiname}{\jobname}

\vspace{3cm}

\vfill

\footnotesize
\textsc{Quelle}: \titel. Herausgegeben von {\editorInnen}. In: \emph{Arthur Schnitzler: Briefwechsel mit Autorinnen und Autoren}.
 Digitale Edition, https://schnitzler-briefe.acdh.oeaw.ac.at/{\dateiname}.html (Stand \today)
\fi

\end{document}


      