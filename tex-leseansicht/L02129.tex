%% latex-leseansicht-vorspann.tex
%% Vorspann für die Leseansicht.
%% Lädt die gemeinsame Datei latex-vorspann.tex mit nicht gesetztem Schalter.

\newif\ifkorrekturansicht
\korrekturansichtfalse

\input{../tex-inputs/latex-vorspann}


\section[Arthur Schnitzler an Hermann Bahr, 22. 4. 1913]{L02129 Arthur Schnitzler an Hermann Bahr, 22. 4. 1913}
\nopagebreak\mylabel{L02129v}
\rehead{ }\normalsize\beginnumbering\briefempfaengerindex{Bahr, Hermann@\textsc{Bahr, Hermann}!zzzSchnitzler, Arthur@\emph{von Arthur Schnitzler}!1913-04-222@{22. 4. 1913}|(be}
\toendnotes[C]{\smallbreak\pagebreak[2]}
\correspDesc{Versand  durch Arthur Schnitzler am 22. 4. 1913 in Wien
\newline{}Erhalt  durch Hermann Bahr im Zeitraum [23. 4. 1913
                  – 27. 4. 1913?] in Salzburg}\toendnotes[C]{\smallbreak}
\Standort{TMW, HS AM 23393 Ba.}
\physDesc{Brief, 2 Blätter, 3 Seiten, 2757 Zeichen
\newline{}Schreibmaschine
\newline{}Handschrift: schwarze Tinte, lateinische Kurrent (\noindent{}Korrekturen, Unterschrift)
\newline{}Ordnung: Lochung }\Standort{DLA, A:Schnitzler, 85.1.294/4.}
\physDesc{Brief, Durchschlag, 2 Blätter, 3 Seiten, 2757 Zeichen
\newline{}Schreibmaschine
\newline{}Handschrift: Bleistift, deutsche Kurrent (\noindent{}Streichung »dass der Aufenth.«)}
\buchAbdrucke{\weitereDrucke{1) Arthur Schnitzler: \emph{Briefe 1913–1931}. Herausgegeben von Peter Michael Braunwarth, Richard Miklin, Susanne Pertlik und Heinrich Schnitzler. Frankfurt am Main: \emph{S. Fischer} 1984, S. 20–22.} \weitereDrucke{2) \emph{22. 4. 1913.} In: Arthur Schnitzler: \emph{The Letters of Arthur Schnitzler to Hermann Bahr}. Edited, annotated, and with an introduction, by Donald G. Daviau. Chapel Hill: \emph{The University of North Carolina Press} 1978, S. 110–111 (University of North Carolina studies in the Germanic languages
                        and literatures, 89).} \weitereDrucke{3) Hermann Bahr, Arthur Schnitzler: \emph{Briefwechsel, Aufzeichnungen, Dokumente (1891–1931)}. Herausgegeben von Kurt Ifkovits und Martin Anton Müller. Göttingen: \emph{Wallstein} 2018, S. 484–485.} }\toendnotes[C]{\smallbreak}
\pstart
           {\pb}\textcolor{gray}{\textbf{Dr. Arthur Schnitzler}}\hfill 22. 4. 1913.\pend
           
\pstart
           \textcolor{gray}{\textbf{Wien XVIII. Sternwartestrasse 71\oindex{Wien@\textbf{Wien}!XVIII., Währing@\textbf{XVIII., Währing}!Sternwartestraße 71@\textbf{Sternwartestraße 71}, \emph{Wohngebäude}|pw}}}\pend
           
\pstart{}Lieber Hermann.\pend\vspace{0.5em}
\pstart
           Ich habe \label{K_L02129-1v}\edtext{nun Altenberg\pwindex{Altenberg, Peter 9.\,3.\,1859 Wien – 8.\,1.\,1919 ebd.@\textsc{Altenberg, Peter} (9.\,3.\,1859 Wien – 8.\,1.\,1919 ebd.), \emph{Schriftsteller}|pw}, seinen Bruder\pwindex{Engländer, Georg 3.\,4.\,1862 Wien – 10.\,4.\,1927 ebd.@\textsc{Engländer, Georg} (3.\,4.\,1862 Wien – 10.\,4.\,1927 ebd.), \emph{Privatbeamter}|pwv} und seinen Arzt\pwindex{Richter, Karl 9.\,3.\,1862 Bruntál – 25.\,6.\,1937 Wien@\textsc{Richter, Karl} (9.\,3.\,1862 Bruntál – 25.\,6.\,1937 Wien), \emph{Mediziner, Sanatoriumsleiter}|pwv} gesprochen}{\lemma{\textnormal{\emph{nun … gesprochen}}}\Cendnote{\textnormal{am 20. 4. 1913}}}\label{K_L02129-1} und glaube ein klares Bild von der ganzen Sache zu haben. Altenberg\pwindex{Altenberg, Peter 9.\,3.\,1859 Wien – 8.\,1.\,1919 ebd.@\textsc{Altenberg, Peter} (9.\,3.\,1859 Wien – 8.\,1.\,1919 ebd.), \emph{Schriftsteller}|pw} ist vor zirka 4–5 Monaten wegen eines akuten
               alkoholischen Irreseins nach Steinhof\oindex{Wien@\textbf{Wien}!XIV., Penzing@\textbf{XIV., Penzing}!Otto-Wagner-Spital@\textbf{Otto-Wagner-Spital}, \emph{Krankenhaus}|pw} gebracht
               worden. Die schweren Erscheinungen, Verfolgungsideen etc., die, erst in der Anstalt
               selbst auftraten, dürften (was mir ärztlicherseits allerdings nicht gesagt wurde) auf
               die plötzliche vollkommene Abstinenz zurückzuführen gewesen sein (die man jetzt, ich
               weiss nicht recht warum, statt der früher geübten allmählichen Entwöhnung in vielen
               Fällen anwendet). Ich habe Altenberg\pwindex{Altenberg, Peter 9.\,3.\,1859 Wien – 8.\,1.\,1919 ebd.@\textsc{Altenberg, Peter} (9.\,3.\,1859 Wien – 8.\,1.\,1919 ebd.), \emph{Schriftsteller}|pw} geistig
               frischer gefunden als seit langer Zeit, nur eben sehr erregt, weil er schon gerne auf
               den Semmering\oindex{Semmering@\textbf{Semmering}, \emph{Verwaltungsgebiet}|pw} möchte. Freilich besteht die
               Gefahr, besser die Sicherheit, dass er ohne ärztliche Aufsicht sofort wieder zu
               trinken und bald auch wieder alkoholisch {\pb}zu \label{K_L02129-2v}\edtext{exzedieren}{\lemma{\textnormal{\emph{exzedieren}}}\Cendnote{\textnormal{übertreiben}}}\label{K_L02129-2} anfängt. Diese Gefahr wird aber gerade so wie
               heute in acht Tagen, in vier Wochen und in einem halben Jahr bestehen. Dazu kommt,
               dass seine steigende Erregung wegen der Internierung in Steinhof\oindex{Wien@\textbf{Wien}!XIV., Penzing@\textbf{XIV., Penzing}!Otto-Wagner-Spital@\textbf{Otto-Wagner-Spital}, \emph{Krankenhaus}|pw} seinem allgemeinen Zustand kaum förderlich sein dürfte.
               Dies alles habe ich auch Peter Altenbergs\pwindex{Altenberg, Peter 9.\,3.\,1859 Wien – 8.\,1.\,1919 ebd.@\textsc{Altenberg, Peter} (9.\,3.\,1859 Wien – 8.\,1.\,1919 ebd.), \emph{Schriftsteller}|pw}{ }Bruder\pwindex{Engländer, Georg 3.\,4.\,1862 Wien – 10.\,4.\,1927 ebd.@\textsc{Engländer, Georg} (3.\,4.\,1862 Wien – 10.\,4.\,1927 ebd.), \emph{Privatbeamter}|pwv} gesagt, und da \strikeout{auch} der Chefarzt\pwindex{Richter, Karl 9.\,3.\,1862 Bruntál – 25.\,6.\,1937 Wien@\textsc{Richter, Karl} (9.\,3.\,1862 Bruntál – 25.\,6.\,1937 Wien), \emph{Mediziner, Sanatoriumsleiter}|pwv} gegen P.
                  A.’s\pwindex{Altenberg, Peter 9.\,3.\,1859 Wien – 8.\,1.\,1919 ebd.@\textsc{Altenberg, Peter} (9.\,3.\,1859 Wien – 8.\,1.\,1919 ebd.), \emph{Schriftsteller}|pw} Entlassung nichts einzuwenden hat, wenn der Bruder\pwindex{Engländer, Georg 3.\,4.\,1862 Wien – 10.\,4.\,1927 ebd.@\textsc{Engländer, Georg} (3.\,4.\,1862 Wien – 10.\,4.\,1927 ebd.), \emph{Privatbeamter}|pwv} die Verantwortung übernimmt, (man
               muss allerdings fragen, wofür?), so dürfte P.
                  A.\pwindex{Altenberg, Peter 9.\,3.\,1859 Wien – 8.\,1.\,1919 ebd.@\textsc{Altenberg, Peter} (9.\,3.\,1859 Wien – 8.\,1.\,1919 ebd.), \emph{Schriftsteller}|pw} in wenigen Tagen die Reise auf den Semmering\oindex{Semmering@\textbf{Semmering}, \emph{Verwaltungsgebiet}|pw} antreten können. Der Bruder\pwindex{Engländer, Georg 3.\,4.\,1862 Wien – 10.\,4.\,1927 ebd.@\textsc{Engländer, Georg} (3.\,4.\,1862 Wien – 10.\,4.\,1927 ebd.), \emph{Privatbeamter}|pwv} möchte nur, was ich sehr vernünftig finde, dass P. A.\pwindex{Altenberg, Peter 9.\,3.\,1859 Wien – 8.\,1.\,1919 ebd.@\textsc{Altenberg, Peter} (9.\,3.\,1859 Wien – 8.\,1.\,1919 ebd.), \emph{Schriftsteller}|pw} wenigstens anfänglich nicht im Hotel,
               sondern im Kurhaus\oindex{Kurhaus Semmering@\textbf{Kurhaus Semmering}, \emph{Hotel}|pw}, also unter recht bescheidener
               ärztlicher Aufsicht wohne. Für den Fall, dass sich das nicht durchführen liesse, wäre
               auch die Begleitung durch einen Wärter in Erwägung zu ziehen. P. A.\pwindex{Altenberg, Peter 9.\,3.\,1859 Wien – 8.\,1.\,1919 ebd.@\textsc{Altenberg, Peter} (9.\,3.\,1859 Wien – 8.\,1.\,1919 ebd.), \emph{Schriftsteller}|pw} möchte selbst sehr gern seinen Wärter\pwindex{?? [Wärter von Peter Altenberg] @\textsc{?? [Wärter von Peter Altenberg]}|pwv} aus dem Sanatorium für ein paar Tage
               mitnehmen, wenn dem nicht, wie es den Anschein hat, von Seiten der Anstalt
               Schwierigkei{\pb}ten
               entgegengesetzt würden. Es hat meiner Ansicht nach wirklich keinen Sinn Peter Altenberg\pwindex{Altenberg, Peter 9.\,3.\,1859 Wien – 8.\,1.\,1919 ebd.@\textsc{Altenberg, Peter} (9.\,3.\,1859 Wien – 8.\,1.\,1919 ebd.), \emph{Schriftsteller}|pw} länger in Steinhof\oindex{Wien@\textbf{Wien}!XIV., Penzing@\textbf{XIV., Penzing}!Otto-Wagner-Spital@\textbf{Otto-Wagner-Spital}, \emph{Krankenhaus}|pw} zu halten, wenn auch kaum zu bezweifeln ist, dass nach
               einiger Zeit ihm ein neues Delirium und wahrscheinlich eine neuerliche Internierung,
               die ja dann der Umgebung wegen nicht zu vermeiden ist, bevorstehen dürfte. Von den
               Degenerationserscheinungen, die man nach allerlei Gerüchten hätte befürchten können
               habe ich bei Altenberg\pwindex{Altenberg, Peter 9.\,3.\,1859 Wien – 8.\,1.\,1919 ebd.@\textsc{Altenberg, Peter} (9.\,3.\,1859 Wien – 8.\,1.\,1919 ebd.), \emph{Schriftsteller}|pw} nicht das Geringste
               bemerkt, und ich glaube, wenn auch vielleicht die \label{T_L02129-1v}\edtext{\uline{plötzliche}}{\lemma{\textnormal{\emph{plötzliche}}}\Cendnote{\textnormal{handschriftliche Unterstreichung}}}\label{T_L02129-1}
               Abstinenz zu Beginn der Anstaltsbehandlung nicht ausschliesslich von Vorteil \substVorne{}\textsuperscript{war, dass der Aufenthalt im Ganzen}\substDazwischen{}für ihn gewesen war\substHinten{}, – die geänderte Lebensweise im weiteren Verlauf und alles was damit
               zusammenhängt hat ihm sicher nur gut getan. Was natürlich kein Anlass ist den
               Aufenthalt ohne Notwendigkeit zu verlängern.\pend
           
\pstart
           Herzlichen Gruss{\\[\baselineskip]}Dein{\\[\baselineskip]}\spacefill\mbox{{[}hs.:{]} Arthur}\pend
           \leftskip=0em{}
\pstart
           \noindent{}{[}ms.:{]} Herrn Hermann Bahr, Salzburg\oindex{Salzburg@\textbf{Salzburg}, \emph{Verwaltungsgebiet}|pw}.\pend
           \selectlanguage{ngerman}\endnumbering\briefempfaengerindex{Bahr, Hermann@\textsc{Bahr, Hermann}!zzzSchnitzler, Arthur@\emph{von Arthur Schnitzler}!1913-04-222@{22. 4. 1913}|)be}\mylabel{L02129h}  \newcommand{\dateiname}{L02129}\newcommand{\titel}{Arthur Schnitzler an Hermann Bahr, 22. 4. 1913}\newcommand{\editorInnen}{Herausgegeben von Martin Anton Müller}%% latex-leseansicht-abspann.tex
%% Abspann für die Leseansicht.
%% Der Schalter \ifkorrekturansicht ist bereits durch den Vorspann gesetzt.

%% latex-abspann.tex
%% Gemeinsamer Abspann für Korrekturansicht und Leseansicht.
%% Setzt den Schalter \ifkorrekturansicht voraus (gesetzt in den
%% einbindenden Dateien latex-korrekturansicht-abspann.tex bzw.
%% latex-leseansicht-abspann.tex).
%% ---------------------------------------------------------------

\normalsize

% Das esempio-Environment wird nur in der Leseansicht benötigt
\ifkorrekturansicht\else
\newenvironment{esempio}[3]%
{
    \vspace{1.5ex}
    \rlap{\underline{#1}}
    \par
    \setlength{\parindent}{0cm}
    \nopagebreak
    \leftskip=#2cm
    \rightskip=#3cm
}
{
    \par
}
\fi

\doendnotes{C}
\bigskip
\vfill

\clearpage

\footnotesize

\ifkorrekturansicht
  \lohead{\textsc{register}}
\fi

% theindex-Environment neu definieren ohne reledmac
\makeatletter
\renewenvironment{theindex}{%
  \ifkorrekturansicht
    \section*{\indexname}%
  \else
    \subsubsection*{Index der erwähnten Entitäten}%
  \fi
  \setlength{\parindent}{0pt}%
  \setlength{\parskip}{0pt plus 0.3pt}%
  \let\item\@idxitem
}{%
  \ifkorrekturansicht\clearpage\fi
}
\makeatother

\IfFileExists{\jobname-pw.ind}{\input{\jobname-pw.ind}}{}

% Quellenangabe nur in der Leseansicht
\ifkorrekturansicht\else
% Fallback-Definitionen, falls die .tex-Datei \titel etc. nicht gesetzt hat
\providecommand{\titel}{}
\providecommand{\editorInnen}{}
\providecommand{\dateiname}{\jobname}

\vspace{3cm}

\vfill

\footnotesize
\textsc{Quelle}: \titel. Herausgegeben von {\editorInnen}. In: \emph{Arthur Schnitzler: Briefwechsel mit Autorinnen und Autoren}.
 Digitale Edition, https://schnitzler-briefe.acdh.oeaw.ac.at/{\dateiname}.html (Stand \today)
\fi

\end{document}


