%% latex-korrekturansicht-vorspann.tex
%% Vorspann für die Korrekturansicht.
%% Lädt die gemeinsame Datei latex-vorspann.tex mit gesetztem Schalter.

\newif\ifkorrekturansicht
\korrekturansichttrue

\input{../tex-inputs/latex-vorspann}


\section[ Paul Goldmann an Arthur Schnitzler, 28. 4. 1897]{L02810 Paul Goldmann an Arthur Schnitzler, 28. 4. 1897}
\nopagebreak\mylabel{L02810v}
\rehead{ }\normalsize\beginnumbering\briefempfaengerindex{Schnitzler, Arthur@\textsc{Schnitzler, Arthur}!zzzGoldmann, Paul@\emph{von Paul Goldmann}!1897-04-282@{28. 4. 1897}|(be}
\toendnotes[C]{\smallbreak\pagebreak[2]}\Standort{DLA, A:Schnitzler, HS.NZ85.1.3167.}
\physDesc{Brief, 1 Blatt, 2 Seiten, 939 Zeichen
\newline{}Handschrift: schwarze Tinte, deutsche Kurrent}\toendnotes[C]{\smallbreak}
\pstart
           {\pb}\textcolor{gray}{\textbf{Frankfurter Zeitung}}\orgindex{Frankfurter Zeitung@Frankfurter Zeitung|pw}\hfill \textcolor{gray}{\textbf{Frankfurt a. M.\oindex{Frankfurt am Main@\textbf{Frankfurt am Main}, \emph{P.PPLA3}|pw},}}{ }28. April \textcolor{gray}{\textbf{189}}7.\pend
           
\pstart
           \textcolor{gray}{\textbf{und}}\pend
           
\pstart
           \textcolor{gray}{\textbf{Handelsblatt.}}\pend
           
\pstart
           \textcolor{gray}{\textbf{Redaktion\orgindex{Frankfurter Zeitung@Frankfurter Zeitung|pwv}.\noindent{}\textcolor{gray}{\textbf{Für die Redaktion\orgindex{Frankfurter Zeitung@Frankfurter Zeitung|pwv} beſtimmte Briefe und Sendungen wolle man
                                 \so{nicht} an die Perſon eines Redakteurs,
                              ſondern ſtets \textbf{an die Redaktion der Frankfurter Zeitung\orgindex{Frankfurter Zeitung@Frankfurter Zeitung|pw}} adreſſiren.}}}}\pend
           
\pstart
           \textcolor{gray}{\textbf{Telegramm-Adreſſe:}}\pend
           
\pstart
           \textcolor{gray}{\textbf{Zeitung\orgindex{Frankfurter Zeitung@Frankfurter Zeitung|pwv}{ }Frankfurt Main\oindex{Frankfurt am Main@\textbf{Frankfurt am Main}, \emph{P.PPLA3}|pw}. }}\pend
           
\pstart\center{}Mein lieber Freund,\pend\vspace{0.5em}
\pstart
           Meine Familie wird mich vor Ende der Woche kaum fortlaſſen und ſo werde ich Dich wohl
               vor Montag oder Dienſtag nicht wiederſehen. Auch thut mir die Ruhe wahrlich noth. Ich war
               und bin noch zum Theil in einem ſchlimmen körperlichen Zuſtande. Ich danke Dir für
               Deinen lieben Brief und freue mich, daß Ihr Euch in \textsc{Paris\oindex{Paris@\textbf{Paris}, \emph{P.PPLC}|pw}} zurechtfindet. Freitag{ }Abend ſolltet Ihr ins \label{K_L02810-1v}\edtext{\begin{otherlanguage}{french}\textsc{Concert Parisien\oindex{Concert Parisien@\textbf{Concert Parisien}, \emph{Konzertsaal (K.KNZ)}|pw}}\end{otherlanguage}}{\lemma{\textnormal{\emph{Concert Parisien}}}\Cendnote{\textnormal{Das taten sie\pwindex{Reinhard, Marie 1871-03-13 – 1899-03-18@\textsc{Reinhard, Marie} (1871-03-13 – 1899-03-18), \emph{Gesangspädagoge/Gesangspädagogin}|pwkv}, vgl. A. S.: \emph{Tagebuch}, 30. 4. 1897.}}}\label{K_L02810-1} zum \label{K_L02810-2v}\edtext{\begin{otherlanguage}{french}\textsc{vendredi classique}\end{otherlanguage}}{\lemma{\textnormal{\emph{vendredi classique}}}\Cendnote{\textnormal{Der »\begin{otherlanguage}{french}vendredi
                     classique\end{otherlanguage}« war eine Veranstaltungsreihe des Concert Parisien\oindex{Concert Parisien@\textbf{Concert Parisien}, \emph{Konzertsaal (K.KNZ)}|pwk}, genauso wie beispielsweise der
                     »\begin{otherlanguage}{french}lundi moderne\end{otherlanguage}«.}}}\label{K_L02810-2} gehen, um \label{K_L02810-3v}\edtext{\textsc{Villé\pwindex{Ville, Dora @\textsc{Villé, Dora}, \emph{Sänger/Sängerin}|pw}}}{\lemma{\textnormal{\emph{Villé}}}\Cendnote{\textnormal{Dora Villé\pwindex{Ville, Dora @\textsc{Villé, Dora}, \emph{Sänger/Sängerin}|pwk}, Sängerin beim »vendredi classique\oindex{Concert Parisien@\textbf{Concert Parisien}, \emph{Konzertsaal (K.KNZ)}|pwkv}«}}}\label{K_L02810-3}
               zu hören. {\pb}Sagte ich Dir, daß Du das \label{K_L02810-4v}\edtext{\textsc{Hotel de Ville\oindex{Hôtel de Ville@\textbf{Hôtel de Ville}, \emph{Regierungsgebäude (K.RGB)}|pw}} und das \textsc{Panthéon\oindex{Pantheon@\textbf{Panthéon}, \emph{Gebäude (K.GBD)}|pw}} beſichtigen}{\lemma{\textnormal{\emph{Hotel … beſichtigen}}}\Cendnote{\textnormal{Das Panthéon\oindex{Pantheon@\textbf{Panthéon}, \emph{Gebäude (K.GBD)}|pwk} hatte Schnitzler bereits am 17. 4. 1897 besucht. Eine Besichtigung des Hôtel de Ville\oindex{Hôtel de Ville@\textbf{Hôtel de Ville}, \emph{Regierungsgebäude (K.RGB)}|pwk} (in dem sich das Paris\oindex{Paris@\textbf{Paris}, \emph{P.PPLC}|pwk}er Rathaus befindet) ist nicht bekannt.}}}\label{K_L02810-4}
               ſollſt?\pend
           
\pstart
           Hier nichts Neues. Oder doch: Ich ſoll als Feuilleton-Correſpondent der Frankfurter Zeitung\orgindex{Frankfurter Zeitung@Frankfurter Zeitung|pw} über kurz oder lang nach \textsc{Berlin\oindex{Berlin@\textbf{Berlin}, \emph{P.PPLC}|pw}} gehen\substVorne{}\textsuperscript{\textcolor{gray}{?}}\substDazwischen{}.\substHinten{} (ganz unter uns, nicht wahr?) Soll ich? \textsc{Paris\oindex{Paris@\textbf{Paris}, \emph{P.PPLC}|pw}} iſt ſo ſchön!\pend
           
\pstart
           Wenn Du Zeit haſt, ſo ſchreib’ mir noch ein Wort über Euer Ergehen ins \textsc{Hotel Deutscher Kaiser\oindex{Hotel Deutscher Kaiser@\textbf{Hotel Deutscher Kaiser}, \emph{Hotel (K.HTL)}|pw}}. Wenn Du zu faul biſt, ſo ſchreib’ mir nicht.\pend
           
\pstart
           Grüß’ Dich Gott! Viele Grüße an Deine Freundin\pwindex{Reinhard, Marie 1871-03-13 – 1899-03-18@\textsc{Reinhard, Marie} (1871-03-13 – 1899-03-18), \emph{Gesangspädagoge/Gesangspädagogin}|pwv}!\pend
           
\pstart
           Dein treuer {\\[\baselineskip]}\spacefill\mbox{Paul Goldmn}\pend
           \leftskip=0em{}
\pstart
           \noindent{}Was macht der \label{K_L02810-5v}\edtext{blonde junge Muſiker\pwindex{?? [blonder junger Musiker in Paris] @\textsc{?? [blonder junger Musiker in Paris]}|pwv}}{\lemma{\textnormal{\emph{blonde junge Muſiker}}}\Cendnote{\textnormal{nicht identifiziert}}}\label{K_L02810-5}?\pend
           \selectlanguage{ngerman}\endnumbering\briefempfaengerindex{Schnitzler, Arthur@\textsc{Schnitzler, Arthur}!zzzGoldmann, Paul@\emph{von Paul Goldmann}!1897-04-282@{28. 4. 1897}|)be}\mylabel{L02810h}  \normalsize

\doendnotes{C}
\bigskip
\vfill

\clearpage

\footnotesize

\lohead{\textsc{register}}

% Definiere theindex-Environment komplett neu ohne reledmac
\makeatletter
\renewenvironment{theindex}{%
  \section*{\indexname}%
  \setlength{\parindent}{0pt}%
  \setlength{\parskip}{0pt plus 0.3pt}%
  \let\item\@idxitem
}{%
  \clearpage
}
\makeatother

\IfFileExists{\jobname-pw.ind}{\input{\jobname-pw.ind}}{}

\end{document}

      