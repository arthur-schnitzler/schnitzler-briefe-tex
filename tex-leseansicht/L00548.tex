%% latex-korrekturansicht-vorspann.tex
%% Vorspann für die Korrekturansicht.
%% Lädt die gemeinsame Datei latex-vorspann.tex mit gesetztem Schalter.

\newif\ifkorrekturansicht
\korrekturansichttrue

\input{../tex-inputs/latex-vorspann}


\section[Arthur Schnitzler an Richard Beer-Hofmann, 2. 6. 1896]{L00548 Arthur Schnitzler an Richard Beer-Hofmann, 2. 6. 1896}
\nopagebreak\mylabel{L00548v}
\rehead{ }\normalsize\beginnumbering\briefempfaengerindex{Beer-Hofmann, Richard@\textsc{Beer-Hofmann, Richard}!zzzSchnitzler, Arthur@\emph{von Arthur Schnitzler}!1896-06-022@{2. 6. 1896}|(be}
\toendnotes[C]{\smallbreak\pagebreak[2]}\Standort{YCGL, MSS 31.}
\physDesc{Postkarte, 219 Zeichen
\newline{}Handschrift: Bleistift, deutsche Kurrent
\newline{}Versand: 1) Rohrpost  2) Stempel: »\nobreak{}\oindex{IX., Alsergrund@\textbf{IX., Alsergrund}, \emph{A.ADM3}|pwk}Wien 9/1, 2. VI. 96, 2 50N\nobreak{}«.  3) Stempel: »\nobreak{}\oindex{I., Innere Stadt@\textbf{I., Innere Stadt}, \emph{A.ADM3}|pwk}Wien 1/1, 2. VI. 96, 3 30N\nobreak{}«. }\pstart{}{\pb}Herrn \textsc{Dr. Richard Beer
                     Hofmann}\pend{}\pstart{}Wien\oindex{Wien@\textbf{Wien}, \emph{A.ADM2}|pw}\pend{}\pstart{}\textsc{I Wollzeile 15\oindex{Wollzeile@\textbf{Wollzeile}, \emph{Straße (K.STR)}|pw}}\pend{}{\bigskip}\vspace{1em}
\pstart
           \noindent{}{\pb}Lieber Richard,{ }Hugo\pwindex{Hofmannsthal, Hugo von 1874-02-01 – 1929-07-15@\textsc{Hofmannsthal, Hugo von} (1874-02-01 – 1929-07-15), \emph{Schriftsteller/Schriftstellerin}|pw}{ }ſchreibt mir, er wird
                  8–½ 9 im \textsc{Grstdl}\oindex{Cafe Griensteidl@\textbf{Café Griensteidl}, \emph{Kaffeehaus (K.KAF)}|pw}{ }ſein, eventuell vorher bei Ihnen anläuten. Eine
               event. Aenderg können Sie dem Hugo\pwindex{Hofmannsthal, Hugo von 1874-02-01 – 1929-07-15@\textsc{Hofmannsthal, Hugo von} (1874-02-01 – 1929-07-15), \emph{Schriftsteller/Schriftstellerin}|pw} bis
                  5{ }ſagen laſſen.\pend
           
\pstart
           Herzlich{\\[\baselineskip]}Ihr\spacefill\mbox{Arthur}\pend
           \leftskip=0em{}\selectlanguage{ngerman}\endnumbering\briefempfaengerindex{Beer-Hofmann, Richard@\textsc{Beer-Hofmann, Richard}!zzzSchnitzler, Arthur@\emph{von Arthur Schnitzler}!1896-06-022@{2. 6. 1896}|)be}\mylabel{L00548h}  \normalsize

\doendnotes{C}
\bigskip
\vfill

\clearpage

\footnotesize

\lohead{\textsc{register}}

% Definiere theindex-Environment komplett neu ohne reledmac
\makeatletter
\renewenvironment{theindex}{%
  \section*{\indexname}%
  \setlength{\parindent}{0pt}%
  \setlength{\parskip}{0pt plus 0.3pt}%
  \let\item\@idxitem
}{%
  \clearpage
}
\makeatother

\IfFileExists{\jobname-pw.ind}{\input{\jobname-pw.ind}}{}

\end{document}

      