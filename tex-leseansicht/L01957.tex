%% latex-leseansicht-vorspann.tex
%% Vorspann für die Leseansicht.
%% Lädt die gemeinsame Datei latex-vorspann.tex mit nicht gesetztem Schalter.

\newif\ifkorrekturansicht
\korrekturansichtfalse

\input{../tex-inputs/latex-vorspann}


         
         \newcommand{\erwaehntePersonen}{Personen: Olga Schnitzler}
         \newcommand{\erwaehnteOrte}{Orte: Bad Aussee, Gasthaus Gosauschmied, Großer und Kleiner Donnerkogel, Sternwartestraße, Vorderer Gosausee, Wien}
         \newcommand{\erwaehnteWerke}{
               \section[Julie und Jakob Wassermann, Irene und Raoul Auernheimer, Gerty und Hugo von Hofmannsthal an Arthur und Olga Schnitzler, 20. 9. 1910]{ Julie und Jakob Wassermann, Irene und Raoul Auernheimer, Gerty und Hugo
               von Hofmannsthal an Arthur und Olga Schnitzler, 20. 9. 1910}\nopagebreak\mylabel{v}\rehead{ }\begin{ledgroupsized}[t]{13cm}\normalsize\beginnumbering \toendnotes[C]{\smallbreak\pagebreak[2]} \Standort{CUL, Schnitzler, B 108.}
\physDesc{Bildpostkarte
\newline{}Handschrift Jakob Wassermann: Bleistift, lateinische Kurrent\newline{}Handschrift Julie Wassermann: Bleistift, lateinische Kurrent\newline{}Handschrift Gertrude von Hofmannsthal: Bleistift, lateinische Kurrent\newline{}Handschrift Hugo von Hofmannsthal: Bleistift\newline{}Handschrift Irene Auernheimer: Bleistift, lateinische Kurrent\newline{}Handschrift Raoul Auernheimer: Bleistift\newline{}Versand: 1) Stempel: »\nobreak{}\oindex{Gasthaus Gosauschmied@\textbf{Gasthaus Gosauschmied}|pwk}Gasthof–Gosauschmied Gosau
                                       Salzkammergut\nobreak{}«.   2) Stempel: »\nobreak{}20. IX. 10\nobreak{}«. 
\newline{}Schnitzler: mit Bleistift die archivalische Einordnung bei Wassermann
                                 markiert: »W.«? }\pstart{}{\pb}Herrn D\textsuperscript{r}
                  Arthur Schnitzler (u. ſ. f.) \pend{}\pstart{}Wien\oindex{Wien@\textbf{Wien}|pw}\pend{}\pstart{}XIX Sternwartestrasse 7\substVorne{}\textsuperscript{2}\substDazwischen{}1\substHinten{}\oindex{Sternwartestrasse@\textbf{Sternwartestraße}|pw}\pend{}{\bigskip}\pstart
           \noindent{}\centering{}\textcolor{gray}{\textbf{{\pb}Gosau\oindex{Vorderer Gosausee@\textbf{Vorderer Gosausee}|pw}. Gasthof
                        Gosauschmied\oindex{Gasthaus Gosauschmied@\textbf{Gasthaus Gosauschmied}|pw} mit den Donnerkogln\oindex{Grosser und Kleiner Donnerkogel@\textbf{Großer und Kleiner Donnerkogel}|pw}.}}\pend
           \pstart
           {\pb}Alles Liebe von einer schönen Landpartie:
               Es grüßt Sie beide \pend
           \pstart
           \spacefill\mbox{Julie Wassermann}{\\[\baselineskip]}\spacefill\mbox{{[}hs. Raoul Auernheimer:{]} Raoul Auernheimer}\pend
           \leftskip=0em{}\pstart
           \noindent{}{[}hs. Gertrude von Hofmannsthal:{]} Wie schön, wenn Ihr auch mit wäret. \pend
           \pstart \spacefill\mbox{Gerty.}\pend{}\pstart
           \noindent{}{[}hs. Jakob Wassermann:{]} Herzlichst grüsst \pend
           \pstart
           \spacefill\mbox{JakobW.}{\\[\baselineskip]}\spacefill\mbox{{[}hs. Hugo von Hofmannsthal:{]} Hugo}\pend
           \leftskip=0em{}\pstart
           \noindent{}{[}hs. Irene Auernheimer:{]} Die herzlichsten Grüsse von \pend
           \pstart \spacefill\mbox{IreneAuernheimer}\pend{}
         
         \endnumbering\mylabel{h}\end{ledgroupsized}  \newcommand{\dateiname}{L01957}\newcommand{\titel}{Julie und Jakob Wassermann, Irene und Raoul Auernheimer, Gerty und Hugo von Hofmannsthal an Arthur und Olga Schnitzler, 20. 9. 1910}\newcommand{\editorInnen}{Martin Anton Müller und Gerd-Hermann Susen}%% latex-leseansicht-abspann.tex
%% Abspann für die Leseansicht.
%% Der Schalter \ifkorrekturansicht ist bereits durch den Vorspann gesetzt.

%% latex-abspann.tex
%% Gemeinsamer Abspann für Korrekturansicht und Leseansicht.
%% Setzt den Schalter \ifkorrekturansicht voraus (gesetzt in den
%% einbindenden Dateien latex-korrekturansicht-abspann.tex bzw.
%% latex-leseansicht-abspann.tex).
%% ---------------------------------------------------------------

\normalsize

% Das esempio-Environment wird nur in der Leseansicht benötigt
\ifkorrekturansicht\else
\newenvironment{esempio}[3]%
{
    \vspace{1.5ex}
    \rlap{\underline{#1}}
    \par
    \setlength{\parindent}{0cm}
    \nopagebreak
    \leftskip=#2cm
    \rightskip=#3cm
}
{
    \par
}
\fi

\doendnotes{C}
\bigskip
\vfill

\clearpage

\footnotesize

\ifkorrekturansicht
  \lohead{\textsc{register}}
\fi

% theindex-Environment neu definieren ohne reledmac
\makeatletter
\renewenvironment{theindex}{%
  \ifkorrekturansicht
    \section*{\indexname}%
  \else
    \subsubsection*{Index der erwähnten Entitäten}%
  \fi
  \setlength{\parindent}{0pt}%
  \setlength{\parskip}{0pt plus 0.3pt}%
  \let\item\@idxitem
}{%
  \ifkorrekturansicht\clearpage\fi
}
\makeatother

\IfFileExists{\jobname-pw.ind}{\input{\jobname-pw.ind}}{}

% Quellenangabe nur in der Leseansicht
\ifkorrekturansicht\else
% Fallback-Definitionen, falls die .tex-Datei \titel etc. nicht gesetzt hat
\providecommand{\titel}{}
\providecommand{\editorInnen}{}
\providecommand{\dateiname}{\jobname}

\vspace{3cm}

\vfill

\footnotesize
\textsc{Quelle}: \titel. Herausgegeben von {\editorInnen}. In: \emph{Arthur Schnitzler: Briefwechsel mit Autorinnen und Autoren}.
 Digitale Edition, https://schnitzler-briefe.acdh.oeaw.ac.at/{\dateiname}.html (Stand \today)
\fi

\end{document}


      