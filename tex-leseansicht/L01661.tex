%% latex-korrekturansicht-vorspann.tex
%% Vorspann für die Korrekturansicht.
%% Lädt die gemeinsame Datei latex-vorspann.tex mit gesetztem Schalter.

\newif\ifkorrekturansicht
\korrekturansichttrue

\input{../tex-inputs/latex-vorspann}


\section[Arthur Schnitzler: Widmungsexemplar Dämmerseelen für Hugo von Hofmannsthal, {[}2.?{]} 3. 1907]{L01661 Arthur Schnitzler: Widmungsexemplar Dämmerseelen für Hugo von
               Hofmannsthal, {[}2.?{]} 3. 1907}
\nopagebreak\mylabel{L01661v}
\rehead{ }\normalsize\beginnumbering\briefempfaengerindex{Hofmannsthal, Hugo von@\textsc{Hofmannsthal, Hugo von}!zzzSchnitzler, Arthur@\emph{von Arthur Schnitzler}!1907-03-022@{{[}2.?{]} 3. 1907}|(be}
\toendnotes[C]{\smallbreak\pagebreak[2]}\Standort{FDH, FDH 3222.}
\physDesc{Widmung am Vorsatzblatt, 40 Zeichen
\newline{}Handschrift: schwarze Tinte, deutsche Kurrent}
\buchAbdrucke{\weitereDrucke{Hugo von Hofmannsthal: \emph{Bibliothek}. Frankfurt am Main: \emph{S. Fischer} 2011, S. 603.} }\toendnotes[C]{\smallbreak}
\pstart
           \noindent{}{\pb}Meinem lieben Hugo\pend
           \pstart \spacefill\mbox{Arthur}\pend{}
\pstart
           Wien\oindex{Wien@\textbf{Wien}, \emph{A.ADM2}|pw}{ }\label{K_L01661-1v}\edtext{März 907}{\lemma{\textnormal{\emph{März 907}}}\Cendnote{\textnormal{Am 7. 3. 1907 notierte Schnitzler das Erscheinen. Am
                        8. 3. 1907 wurde es vom \emph{Börsenblatt für
                        den deutschen Buchhandel}\pwindex{Boersenblatt fuer den Deutschen Buchhandel@\emph{Börsenblatt für den Deutschen Buchhandel}|pwk} als Neuerscheinung gemeldet. Die Datierung erfolgt analog
                  zum Widmungsexemplar für Salten\pwindex{Salten, Felix 06.09.1869 – 08.10.1945@\textsc{Salten, Felix} (06.09.1869 – 08.10.1945), \emph{Schriftsteller/Schriftstellerin, Journalist/Journalistin, Chefredakteur/Chefredakteurin}|pwk} vom 2. 3. 1907.}}}\label{K_L01661-1}.\pend
           \selectlanguage{ngerman}\vspace{1em}{\vspace{1\baselineskip}}
\pstart
           \centering{}{\pb}\textcolor{gray}{\textbf{\so{Dämmerſeelen}}}\pwindex{Daemmerseelen. Novellen@\emph{Dämmerseelen. Novellen}|pw}\pend
           
\pstart
           \centering{}\textcolor{gray}{\textbf{Novellen}}{\\}\textcolor{gray}{\textbf{von}}{\\}\textcolor{gray}{\textbf{\so{Arthur Schnitzler}}}\pend
           {\vspace{1\baselineskip}}
\pstart
           \centering{}\textcolor{gray}{\textbf{S. Fiſcher, Verlag\orgindex{S. Fischer Verlag@S. Fischer Verlag|pw}, Berlin\oindex{Berlin@\textbf{Berlin}, \emph{P.PPLC}|pw}}}\pend
           
\pstart
           \centering{}\textcolor{gray}{\textbf{1907}}\pend
           \selectlanguage{ngerman}\endnumbering\briefempfaengerindex{Hofmannsthal, Hugo von@\textsc{Hofmannsthal, Hugo von}!zzzSchnitzler, Arthur@\emph{von Arthur Schnitzler}!1907-03-022@{{[}2.?{]} 3. 1907}|)be}\mylabel{L01661h}  \normalsize

\doendnotes{C}
\bigskip
\vfill

\clearpage

\footnotesize

\lohead{\textsc{register}}

% Definiere theindex-Environment komplett neu ohne reledmac
\makeatletter
\renewenvironment{theindex}{%
  \section*{\indexname}%
  \setlength{\parindent}{0pt}%
  \setlength{\parskip}{0pt plus 0.3pt}%
  \let\item\@idxitem
}{%
  \clearpage
}
\makeatother

\IfFileExists{\jobname-pw.ind}{\input{\jobname-pw.ind}}{}

\end{document}

      