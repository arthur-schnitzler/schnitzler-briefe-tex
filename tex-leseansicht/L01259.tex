%% latex-leseansicht-vorspann.tex
%% Vorspann für die Leseansicht.
%% Lädt die gemeinsame Datei latex-vorspann.tex mit nicht gesetztem Schalter.

\newif\ifkorrekturansicht
\korrekturansichtfalse

\input{../tex-inputs/latex-vorspann}


\section[Richard Beer-Hofmann an Arthur Schnitzler, {[}24. 12. 1902{]}]{L01259 Richard Beer-Hofmann an Arthur Schnitzler, {[}24. 12. 1902{]}}
\nopagebreak\mylabel{L01259v}
\rehead{ }\normalsize\beginnumbering\briefempfaengerindex{Schnitzler, Arthur@\textsc{Schnitzler, Arthur}!zzzBeer-Hofmann, Richard@\emph{von Richard Beer-Hofmann}!1902-12-241@{{[}24. 12. 1902{]}}|(be}
\toendnotes[C]{\smallbreak\pagebreak[2]}
\correspDesc{Versand  durch Richard Beer-Hofmann am [24. 12. 1902] in Rodaun
\newline{}Erhalt  durch Arthur Schnitzler im Zeitraum [25. 12. 1902 – 29. 12. 1902?] in Wien}\toendnotes[C]{\smallbreak}
\Standort{CUL, Schnitzler, B 8.}
\physDesc{Visitenkarte, 48 Zeichen (Visitenkarte mit Trauerrand )
\newline{}Handschrift: schwarze Tinte, lateinische Kurrent
\newline{}Ordnung: mit Bleistift von unbekannter Hand nummeriert:
                                    »179a« }
\buchAbdrucke{\weitereDrucke{Arthur Schnitzler, Richard Beer-Hofmann: \emph{Briefwechsel 1891–1931}. Herausgegeben von Konstanze Fliedl. Wien, Zürich: \emph{Europaverlag} 1992, S. 159.} }\toendnotes[C]{\smallbreak}
\pstart
           \noindent{}{\pb}Anstatt \label{K_L01259-1v}\edtext{der aus Pontresina\oindex{Pontresina@\textbf{Pontresina}|pw}}{\lemma{\textnormal{\emph{der aus Pontresina}}}\Cendnote{\textnormal{Es handelt sich möglicherweise um eine
                  beim gemeinsamen Aufenthalt 1900 gekaufte Uhr, die durch das
                  Weihnachtsgeschenk ersetzt werden sollte.}}}\label{K_L01259-1} zu tragen.\pend
           \pstart Von Herzen Ihr\pend{}
\pstart
           \centering{}{\pb}\textcolor{gray}{\textbf{\textsc{Richard \strikeout{\label{T_L01259-1v}\edtext{Beer-Hofmann}{\lemma{\textnormal{\emph{Beer-Hofmann}}}\Cendnote{\textnormal{Streichung mit Tinte}}}\label{T_L01259-1}}}}}\pend
           
\pstart
           \raggedleft{}\textcolor{gray}{\textbf{RODAUN\oindex{Wien@\textbf{Wien}!XXIII., Liesing@\textbf{XXIII., Liesing}!Rodaun@\textbf{Rodaun}, \emph{Region}|pw}, bei Wien\oindex{Wien@\textbf{Wien}, \emph{Verwaltungsgebiet}|pw}.}}\pend
           \selectlanguage{ngerman}\endnumbering\briefempfaengerindex{Schnitzler, Arthur@\textsc{Schnitzler, Arthur}!zzzBeer-Hofmann, Richard@\emph{von Richard Beer-Hofmann}!1902-12-241@{{[}24. 12. 1902{]}}|)be}\mylabel{L01259h}  \newcommand{\dateiname}{L01259}\newcommand{\titel}{Richard Beer-Hofmann an Arthur Schnitzler, [24. 12. 1902]}\newcommand{\editorInnen}{Martin Anton Müller und Gerd-Hermann Susen}%% latex-leseansicht-abspann.tex
%% Abspann für die Leseansicht.
%% Der Schalter \ifkorrekturansicht ist bereits durch den Vorspann gesetzt.

%% latex-abspann.tex
%% Gemeinsamer Abspann für Korrekturansicht und Leseansicht.
%% Setzt den Schalter \ifkorrekturansicht voraus (gesetzt in den
%% einbindenden Dateien latex-korrekturansicht-abspann.tex bzw.
%% latex-leseansicht-abspann.tex).
%% ---------------------------------------------------------------

\normalsize

% Das esempio-Environment wird nur in der Leseansicht benötigt
\ifkorrekturansicht\else
\newenvironment{esempio}[3]%
{
    \vspace{1.5ex}
    \rlap{\underline{#1}}
    \par
    \setlength{\parindent}{0cm}
    \nopagebreak
    \leftskip=#2cm
    \rightskip=#3cm
}
{
    \par
}
\fi

\doendnotes{C}
\bigskip
\vfill

\clearpage

\footnotesize

\ifkorrekturansicht
  \lohead{\textsc{register}}
\fi

% theindex-Environment neu definieren ohne reledmac
\makeatletter
\renewenvironment{theindex}{%
  \ifkorrekturansicht
    \section*{\indexname}%
  \else
    \subsubsection*{Index der erwähnten Entitäten}%
  \fi
  \setlength{\parindent}{0pt}%
  \setlength{\parskip}{0pt plus 0.3pt}%
  \let\item\@idxitem
}{%
  \ifkorrekturansicht\clearpage\fi
}
\makeatother

\IfFileExists{\jobname-pw.ind}{\input{\jobname-pw.ind}}{}

% Quellenangabe nur in der Leseansicht
\ifkorrekturansicht\else
% Fallback-Definitionen, falls die .tex-Datei \titel etc. nicht gesetzt hat
\providecommand{\titel}{}
\providecommand{\editorInnen}{}
\providecommand{\dateiname}{\jobname}

\vspace{3cm}

\vfill

\footnotesize
\textsc{Quelle}: \titel. Herausgegeben von {\editorInnen}. In: \emph{Arthur Schnitzler: Briefwechsel mit Autorinnen und Autoren}.
 Digitale Edition, https://schnitzler-briefe.acdh.oeaw.ac.at/{\dateiname}.html (Stand \today)
\fi

\end{document}


