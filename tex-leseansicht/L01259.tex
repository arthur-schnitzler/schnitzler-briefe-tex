\input{../tex-inputs/latex-pdf-vorspann}
\begin{center}
            \textcolor{red}{ENTWURF. ENTZIFFERUNG NOCH NICHT KORREKTURGELESEN}
                      \end{center}
            
               \section[Richard Beer-Hofmann an Arthur Schnitzler, {[}24. 12. 1902{]}]{ Richard Beer-Hofmann an Arthur Schnitzler, {[}24. 12. 1902{]}}\nopagebreak\mylabel{v}\rehead{ }\begin{ledgroupsized}[t]{13cm}\normalsize\beginnumbering\briefempfaengerindex{Schnitzler, Arthur@\textsc{Schnitzler, Arthur}!zzzBeer-Hofmann, Richard@\emph{von Richard Beer-Hofmann}!1902-12-241@{{[}24. 12. 1902{]}}|(be} \toendnotes[C]{\smallbreak\pagebreak[2]} \Standort{CUL, Schnitzler, B 8.}
\physDesc{Visitenkarte mit Trauerrand
\newline{}Handschrift: schwarze Tinte, lateinische Kurrent\newline{}Ordnung: mit Bleistift von unbekannter Hand nummeriert:
                                    »179a« }\buchAbdrucke{\weitereDrucke{Arthur Schnitzler, Richard Beer-Hofmann: \emph{Briefwechsel 1891–1931}. Hg. Konstanze Fliedl. Wien, Zürich: \emph{Europaverlag} 1992, S. 159.} }\toendnotes[C]{\smallbreak}\pstart
           \noindent{}{\pb}Anstatt \label{K_L01259_1v}\edtext{der aus Pontresina\oindex{Pontresina@\textbf{Pontresina}|pw}}{\lemma{\textnormal{\emph{der aus Pontresina}}}\Cendnote{\textnormal{Es handelt sich möglicherweise um eine
                  beim gemeinsamen Aufenthalt 1900 gekaufte Uhr, die durch das
                  Weihnachtsgeschenk ersetzt werden sollte.}}}\label{K_L01259_1h} zu tragen.\pend
           \pstart Von Herzen Ihr\pend{}\pstart
           \centering{}{\pb}\textcolor{gray}{\textbf{\textsc{Richard \strikeout{\label{T_L01259-1v}\edtext{Beer-Hofmann}{\lemma{\textnormal{\emph{Beer-Hofmann}}}\Cendnote{\textnormal{Streichung mit Tinte}}}\label{T_L01259-1h}}}}}\pend
           \pstart
           \noindent{}\raggedleft{}\textcolor{gray}{\textbf{RODAUN\oindex{Rodaun@\textbf{Rodaun}|pw}, bei Wien\oindex{Wien@\textbf{Wien}|pw}.}}\pend
           \endnumbering\briefempfaengerindex{Schnitzler, Arthur@\textsc{Schnitzler, Arthur}!zzzBeer-Hofmann, Richard@\emph{von Richard Beer-Hofmann}!1902-12-241@{{[}24. 12. 1902{]}}|)be}\mylabel{h}\end{ledgroupsized}  \newcommand{\dateiname}{L01259}\newcommand{\titel}{Richard Beer-Hofmann an Arthur Schnitzler, [24. 12. 1902]}\newcommand{\editorInnen}{Martin Anton Müller und Gerd-Hermann Susen}\input{../tex-inputs/latex-pdf-abspann}
      