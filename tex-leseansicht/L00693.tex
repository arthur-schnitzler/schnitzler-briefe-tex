%% latex-leseansicht-vorspann.tex
%% Vorspann für die Leseansicht.
%% Lädt die gemeinsame Datei latex-vorspann.tex mit nicht gesetztem Schalter.

\newif\ifkorrekturansicht
\korrekturansichtfalse

\input{../tex-inputs/latex-vorspann}


         
         \newcommand{\erwaehntePersonen}{Personen: Richard Beer-Hofmann, Johann Wolfgang von Goethe, Wolfgang Amadeus Mozart}
         \newcommand{\erwaehnteInstitutionen}{}
         \newcommand{\erwaehnteOrte}{Orte: Bad Fusch, Italien, Salzburg, Wien}
         \newcommand{\erwaehnteWerke}{Werke: Faust. Eine Tragödie, Italienische Reise, W. A. Mozart, Über den Sprachgebrauch bei den Dichtern der Pléjade}
               \section[Hugo von Hofmannsthal an Arthur Schnitzler, 6. 7. {[}1897{]}]{ Hugo von Hofmannsthal an Arthur Schnitzler, 6. 7. {[}1897{]}}\nopagebreak\mylabel{v}\rehead{ }\begin{ledgroupsized}[t]{13cm}\normalsize\beginnumbering \toendnotes[C]{\smallbreak\pagebreak[2]} \Standort{CUL, Schnitzler, B 43.}
\physDesc{Brief, 1 Blatt (gedrucktes Wappen in blauer Farbe), 3 Seiten
\newline{}Handschrift: schwarze Tinte, deutsche Kurrent
\newline{}Schnitzler: mit Bleistift die Jahreszahl ergänzt: »97« \newline{}Ordnung: mit Bleistift von unbekannter Hand nummeriert:
                                        »92« }\buchAbdrucke{\weitereDrucke{Hugo von Hofmannsthal, Arthur Schnitzler: \emph{Briefwechsel}. Hg. Therese Nickl und Heinrich Schnitzler. Frankfurt am Main: \emph{S. Fischer} 1964, S. 88.} }\toendnotes[C]{\smallbreak}\pstart
           \raggedleft{}{\pb}Bad Fuſch\oindex{Bad Fusch@\textbf{Bad Fusch}|pw}, 6. July.\pend
           \pstart{}mein lieber Arthur,\pend\pstart
           ich lebe ſehr ſtill und recht zufrieden, verſuche hie und da Verſe zu machen und
                    komme mir merkwürdig unſicher und entwöhnt vor, ſchmiere an meiner Doctorsarbeit\pwindex{Hofmannsthal, Hugo von 1874-02-01 – 1929-07-15@\textsc{Hofmannsthal, Hugo von} (1874-02-01 – 1929-07-15), \emph{Schriftsteller}!Ueber den Sprachgebrauch bei den Dichtern der Plejade1899@\strich\emph{Über den Sprachgebrauch bei den Dichtern der Pléjade} {[}1899{]}|pwv} und finde daſs
                        »Fauſt\pwindex{Goethe, Johann Wolfgang von 1749-08-28 – 1832-03-22@\textsc{Goethe, Johann Wolfgang von} (1749-08-28 – 1832-03-22), \emph{Schriftsteller}!Faust. Eine Tragoedie1808@\strich\emph{Faust. Eine Tragödie} {[}1808{]}|pw}« von Goethe\pwindex{Goethe, Johann Wolfgang von 1749-08-28 – 1832-03-22@\textsc{Goethe, Johann Wolfgang von} (1749-08-28 – 1832-03-22), \emph{Schriftsteller}|pw} ein ſehr angenehmes Buch iſt, in welchem das Schöne und das
                    Kluge wundervoll ineinander aufgehen, was man denn wohl heitere {\pb}Weisheit nennen kann. Anders
                    wieder die italieniſche Reiſe\pwindex{Goethe, Johann Wolfgang von 1749-08-28 – 1832-03-22@\textsc{Goethe, Johann Wolfgang von} (1749-08-28 – 1832-03-22), \emph{Schriftsteller}!Italienische Reise1816 – 1817@\strich\emph{Italienische Reise} {[}1816 – 1817{]}|pw}, die einem einen
                    guten Begriff von der Friſche und kraftvollen Naivetät eines drei- oder
                    vierundvierzigjährigen Menſchen geben kann.\pend
           \pstart
           Die Mozart\pwindex{Mozart, Wolfgang Amadeus 27.01.1756 – 05.12.1791@\textsc{Mozart, Wolfgang Amadeus} (27.01.1756 – 05.12.1791), \emph{Komponist}|pw}biographie\pwindex{\textcolor{red}{\textsuperscript{XXXX1 indx}}!W. A. Mozart1856 – 1859@\strich\emph{W. A. Mozart} {[}1856 – 1859{]}|pwv} enthält viel
                    weniger menſchliches, als ich erwartet hätte, zumindeſt in dieſem Theil; nur
                    hübſche kindiſche Briefe aus Italien\oindex{Italien@\textbf{Italien}|pw}.
                    Vielleicht ſchicken Sie mir gelegentlich hieher den 2\textsuperscript{ten} Band, ich Ihnen {\pb}den erſten. Denn nach Salzburg\oindex{Salzburg@\textbf{Salzburg}|pw} ko{\geminationm} ich nur mit einem ſehr kleinen Koffer. Daſs mir
                        Richard\pwindex{Beer-Hofmann, Richard 1866-07-11 – 1945-09-26@\textsc{Beer-Hofmann, Richard} (1866-07-11 – 1945-09-26), \emph{Schriftsteller}|pw} abſolut nicht ſchreibt, bedeutet
                    doch wohl nichts beſonderes, am wenigſten daſs er viel arbeitet?\pend
           \pstart
           Ich wäre ſehr froh über einige Nachricht von Euch beiden.\pend
           \pstart
           Herzlich der Ihre{\\[\baselineskip]}\spacefill\mbox{Hugo.}\pend
           \leftskip=0em{}
         
         \endnumbering\mylabel{h}\end{ledgroupsized}  \newcommand{\dateiname}{L00693}\newcommand{\titel}{Hugo von Hofmannsthal an Arthur Schnitzler, 6. 7. [1897]}\newcommand{\editorInnen}{Martin Anton Müller und Gerd-Hermann Susen}%% latex-leseansicht-abspann.tex
%% Abspann für die Leseansicht.
%% Der Schalter \ifkorrekturansicht ist bereits durch den Vorspann gesetzt.

%% latex-abspann.tex
%% Gemeinsamer Abspann für Korrekturansicht und Leseansicht.
%% Setzt den Schalter \ifkorrekturansicht voraus (gesetzt in den
%% einbindenden Dateien latex-korrekturansicht-abspann.tex bzw.
%% latex-leseansicht-abspann.tex).
%% ---------------------------------------------------------------

\normalsize

% Das esempio-Environment wird nur in der Leseansicht benötigt
\ifkorrekturansicht\else
\newenvironment{esempio}[3]%
{
    \vspace{1.5ex}
    \rlap{\underline{#1}}
    \par
    \setlength{\parindent}{0cm}
    \nopagebreak
    \leftskip=#2cm
    \rightskip=#3cm
}
{
    \par
}
\fi

\doendnotes{C}
\bigskip
\vfill

\clearpage

\footnotesize

\ifkorrekturansicht
  \lohead{\textsc{register}}
\fi

% theindex-Environment neu definieren ohne reledmac
\makeatletter
\renewenvironment{theindex}{%
  \ifkorrekturansicht
    \section*{\indexname}%
  \else
    \subsubsection*{Index der erwähnten Entitäten}%
  \fi
  \setlength{\parindent}{0pt}%
  \setlength{\parskip}{0pt plus 0.3pt}%
  \let\item\@idxitem
}{%
  \ifkorrekturansicht\clearpage\fi
}
\makeatother

\IfFileExists{\jobname-pw.ind}{\input{\jobname-pw.ind}}{}

% Quellenangabe nur in der Leseansicht
\ifkorrekturansicht\else
% Fallback-Definitionen, falls die .tex-Datei \titel etc. nicht gesetzt hat
\providecommand{\titel}{}
\providecommand{\editorInnen}{}
\providecommand{\dateiname}{\jobname}

\vspace{3cm}

\vfill

\footnotesize
\textsc{Quelle}: \titel. Herausgegeben von {\editorInnen}. In: \emph{Arthur Schnitzler: Briefwechsel mit Autorinnen und Autoren}.
 Digitale Edition, https://schnitzler-briefe.acdh.oeaw.ac.at/{\dateiname}.html (Stand \today)
\fi

\end{document}


      