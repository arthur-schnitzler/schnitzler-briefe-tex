%% latex-korrekturansicht-vorspann.tex
%% Vorspann für die Korrekturansicht.
%% Lädt die gemeinsame Datei latex-vorspann.tex mit gesetztem Schalter.

\newif\ifkorrekturansicht
\korrekturansichttrue

\input{../tex-inputs/latex-vorspann}


\section[Hugo von Hofmannsthal an Arthur Schnitzler, 6. 7. {[}1897{]}]{L00693 Hugo von Hofmannsthal an Arthur Schnitzler, 6. 7. {[}1897{]}}
\nopagebreak\mylabel{L00693v}
\rehead{ }\normalsize\beginnumbering\briefempfaengerindex{Schnitzler, Arthur@\textsc{Schnitzler, Arthur}!zzzHofmannsthal, Hugo von@\emph{von Hugo von Hofmannsthal}!1897-07-061@{6. 7. {[}1897{]}}|(be}
\toendnotes[C]{\smallbreak\pagebreak[2]}\Standort{CUL, Schnitzler, B 43.}
\physDesc{Brief, 1 Blatt, 3 Seiten, 1022 Zeichen (gedrucktes Wappen in blauer Farbe)
\newline{}Handschrift: schwarze Tinte, deutsche Kurrent
\newline{}Schnitzler: mit Bleistift die Jahreszahl ergänzt: »97« 
\newline{}Ordnung: mit Bleistift von unbekannter Hand nummeriert:
                                    »92« }
\buchAbdrucke{\weitereDrucke{Hugo von Hofmannsthal, Arthur Schnitzler: \emph{Briefwechsel}. Frankfurt am Main: \emph{S. Fischer} 1964, S. 88.} }\toendnotes[C]{\smallbreak}
\pstart
           \raggedleft{}{\pb}Bad Fuſch\oindex{Bad Fusch@\textbf{Bad Fusch}, \emph{A.ADM3}|pw}, 6. July.\pend
           
\pstart{}mein lieber Arthur,\pend\vspace{0.5em}
\pstart
           ich lebe ſehr ſtill und recht zufrieden, verſuche hie und da Verſe zu machen und
               komme mir merkwürdig unſicher und entwöhnt vor, ſchmiere an meiner Doctorsarbeit\pwindex{Ueber den Sprachgebrauch bei den Dichtern der Plejade@\emph{Über den Sprachgebrauch bei den Dichtern der Pléjade}|pwv} und finde daſs »Fauſt\pwindex{Faust. Eine Tragoedie@\emph{Faust. Eine Tragödie}|pw}« von Goethe\pwindex{Goethe, Johann Wolfgang von 1749-08-28 – 1832-03-22@\textsc{Goethe, Johann Wolfgang von} (1749-08-28 – 1832-03-22), \emph{Schriftsteller/Schriftstellerin}|pw} ein ſehr angenehmes Buch iſt, in welchem das Schöne und das Kluge
               wundervoll ineinander aufgehen, was man denn wohl heitere {\pb}Weisheit nennen kann. Anders
               wieder die italieniſche Reiſe\pwindex{Italienische Reise@\emph{Italienische Reise}|pw}, die einem einen
               guten Begriff von der Friſche und kraftvollen Naivetät eines drei- oder
               vierundvierzigjährigen Menſchen geben kann.\pend
           
\pstart
           Die Mozart\pwindex{Mozart, Wolfgang Amadeus 27.01.1756 – 05.12.1791@\textsc{Mozart, Wolfgang Amadeus} (27.01.1756 – 05.12.1791), \emph{Komponist/Komponistin}|pw}biographie\pwindex{W. A. Mozart@\emph{W. A. Mozart}|pwv} enthält viel weniger
               menſchliches, als ich erwartet hätte, zumindeſt in dieſem Theil; nur hübſche
               kindiſche Briefe aus Italien\oindex{Italien@\textbf{Italien}, \emph{A.PCLI}|pw}. Vielleicht
               ſchicken Sie mir gelegentlich hieher den 2\textsuperscript{ten} Band, ich
               Ihnen {\pb}den erſten. Denn nach Salzburg\oindex{Salzburg@\textbf{Salzburg}, \emph{A.ADM2}|pw} ko{\geminationm} ich nur
               mit einem ſehr kleinen Koffer. Daſs mir Richard\pwindex{Beer-Hofmann, Richard 1866-07-11 – 1945-09-26@\textsc{Beer-Hofmann, Richard} (1866-07-11 – 1945-09-26), \emph{Schriftsteller/Schriftstellerin}|pw} abſolut nicht ſchreibt, bedeutet doch wohl nichts beſonderes, am
               wenigſten daſs er viel arbeitet?\pend
           
\pstart
           Ich wäre ſehr froh über einige Nachricht von Euch beiden.\pend
           
\pstart
           Herzlich der Ihre{\\[\baselineskip]}\spacefill\mbox{Hugo.}\pend
           \leftskip=0em{}\selectlanguage{ngerman}\endnumbering\briefempfaengerindex{Schnitzler, Arthur@\textsc{Schnitzler, Arthur}!zzzHofmannsthal, Hugo von@\emph{von Hugo von Hofmannsthal}!1897-07-061@{6. 7. {[}1897{]}}|)be}\mylabel{L00693h}  \normalsize

\doendnotes{C}
\bigskip
\vfill

\clearpage

\footnotesize

\lohead{\textsc{register}}

% Definiere theindex-Environment komplett neu ohne reledmac
\makeatletter
\renewenvironment{theindex}{%
  \section*{\indexname}%
  \setlength{\parindent}{0pt}%
  \setlength{\parskip}{0pt plus 0.3pt}%
  \let\item\@idxitem
}{%
  \clearpage
}
\makeatother

\IfFileExists{\jobname-pw.ind}{\input{\jobname-pw.ind}}{}

\end{document}

      