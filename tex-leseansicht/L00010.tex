%% latex-korrekturansicht-vorspann.tex
%% Vorspann für die Korrekturansicht.
%% Lädt die gemeinsame Datei latex-vorspann.tex mit gesetztem Schalter.

\newif\ifkorrekturansicht
\korrekturansichttrue

\input{../tex-inputs/latex-vorspann}


\section[Arthur Schnitzler an Wilhelm Bölsche, {[}5. 5. 1891{]}]{L00010 Arthur Schnitzler an Wilhelm Bölsche, {[}5. 5. 1891{]}}
\nopagebreak\mylabel{L00010v}
\rehead{ }\normalsize\beginnumbering\briefempfaengerindex{Boelsche, Wilhelm@\textsc{Bölsche, Wilhelm}!zzzSchnitzler, Arthur@\emph{von Arthur Schnitzler}!1891-05-051@{{[}5. 5. 1891{]}}|(be}
\toendnotes[C]{\smallbreak\pagebreak[2]}\Standort{Wrocław, Biblioteka Uniwersytecka, Böl.Pis 1772.}
\physDesc{Brief, 1 Blatt, 2 Seiten, 328 Zeichen
\newline{}Handschrift: schwarze Tinte, deutsche Kurrent}
\buchAbdrucke{\weitereDrucke{1) \emph{Germanica Wratislaviensia} (1987) Nr. 77, S. 465.} \weitereDrucke{2) Wilhelm Bölsche: \emph{Briefwechsel. Mit Autoren der Freien Bühne}. Berlin: \emph{Weidler} 2010, S. 671.} }\toendnotes[C]{\smallbreak}
\pstart{}{\pb}Sehr geehrter Herr Redakteur,\pend\vspace{0.5em}
\pstart
           ich \label{K_L00010-1v}\edtext{ſende}{\lemma{\textnormal{\emph{ſende}}}\Cendnote{\textnormal{Vgl. A. S.: \emph{Tagebuch}, 5. 5. 1891.
               }}}\label{K_L00010-1} Ihnen hier eine Skizze\pwindex{Sohn. Aus den Papieren eines Arztes@\emph{Der Sohn. Aus den Papieren eines Arztes}|pwv}, vielleicht finden Sie dieſelbe für Ihre Zeitſchrift\pwindex{Freie Buehne fuer den Entwickelungskampf der Zeit@\emph{Freie Bühne für den Entwickelungskampf der Zeit}|pwv} geeignet, was mir zur beſondern Ehre gereichte.
               Können Sie das Ding nicht brauchen, ſo haben Sie wohl die Liebens{\pb}würdigkeit, es bald an mich zurückzuſenden.\pend
           
\pstart
           Hochachtungsvoll{\\[\baselineskip]}\spacefill\mbox{Dr. Arthur Schnitzler}\pend
           \leftskip=0em{}
\pstart
           \noindent{}\textsc{Wien, I. Giselastraße 11}\oindex{Ordination Arthur Schnitzler [Boesendorferstrasse 11]@\textbf{Ordination Arthur Schnitzler [Bösendorferstraße 11]}, \emph{Ordination}|pw}.\pend
           \selectlanguage{ngerman}\endnumbering\briefempfaengerindex{Boelsche, Wilhelm@\textsc{Bölsche, Wilhelm}!zzzSchnitzler, Arthur@\emph{von Arthur Schnitzler}!1891-05-051@{{[}5. 5. 1891{]}}|)be}\mylabel{L00010h}  \normalsize

\doendnotes{C}
\bigskip
\vfill

\clearpage

\footnotesize

\lohead{\textsc{register}}

% Definiere theindex-Environment komplett neu ohne reledmac
\makeatletter
\renewenvironment{theindex}{%
  \section*{\indexname}%
  \setlength{\parindent}{0pt}%
  \setlength{\parskip}{0pt plus 0.3pt}%
  \let\item\@idxitem
}{%
  \clearpage
}
\makeatother

\IfFileExists{\jobname-pw.ind}{\input{\jobname-pw.ind}}{}

\end{document}

      