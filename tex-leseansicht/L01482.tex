%% latex-korrekturansicht-vorspann.tex
%% Vorspann für die Korrekturansicht.
%% Lädt die gemeinsame Datei latex-vorspann.tex mit gesetztem Schalter.

\newif\ifkorrekturansicht
\korrekturansichttrue

\input{../tex-inputs/latex-vorspann}


\section[Max Burckhard an Arthur Schnitzler, 24. {[}12. 1904{]}]{L01482 Max Burckhard an Arthur Schnitzler, 24. {[}12. 1904{]}}
\nopagebreak\mylabel{L01482v}
\rehead{ }\normalsize\beginnumbering\briefempfaengerindex{Schnitzler, Arthur@\textsc{Schnitzler, Arthur}!zzzBurckhard, Max Eugen@\emph{von Max Eugen Burckhard}!1904-12-241@{24. {[}12. 1904{]}}|(be}
\toendnotes[C]{\smallbreak\pagebreak[2]}\Standort{CUL, Schnitzler, B 20.}
\physDesc{Telegramm, 95 Zeichen
\newline{}maschinell
\newline{}Versand: aufgenommen von »\textcolor{gray}{\textbf{\textit{C.
                                             Neuberg\textcolor{gray}{er}\pwindex{Neuberger, Karoline 1865-01-11 – 1944-12-01@\textsc{Neuberger, Karoline} (1865-01-11 – 1944-12-01), \emph{Telegrafenbeamter/Telegrafenbeamtin}|pw}}}}« 
\newline{}Schnitzler: mit Bleistift datiert: »24/12 904« }
\pstart
           {\pb}fr st gilgen\oindex{St. Gilgen@\textbf{St. Gilgen}, \emph{A.ADM3}|pw} 35 13 24{ }2 35 n\pend
           \vspace{0.5em}
\pstart
           wetter herrlich. – was ists? herzliche weihnachtsgruesse\pend
           \pstart \spacefill\mbox{= burckhard. +}\pend{}\selectlanguage{ngerman}\endnumbering\briefempfaengerindex{Schnitzler, Arthur@\textsc{Schnitzler, Arthur}!zzzBurckhard, Max Eugen@\emph{von Max Eugen Burckhard}!1904-12-241@{24. {[}12. 1904{]}}|)be}\mylabel{L01482h}  \normalsize

\doendnotes{C}
\bigskip
\vfill

\clearpage

\footnotesize

\lohead{\textsc{register}}

% Definiere theindex-Environment komplett neu ohne reledmac
\makeatletter
\renewenvironment{theindex}{%
  \section*{\indexname}%
  \setlength{\parindent}{0pt}%
  \setlength{\parskip}{0pt plus 0.3pt}%
  \let\item\@idxitem
}{%
  \clearpage
}
\makeatother

\IfFileExists{\jobname-pw.ind}{\input{\jobname-pw.ind}}{}

\end{document}

      