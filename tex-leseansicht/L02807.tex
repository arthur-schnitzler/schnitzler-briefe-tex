%% latex-leseansicht-vorspann.tex
%% Vorspann für die Leseansicht.
%% Lädt die gemeinsame Datei latex-vorspann.tex mit nicht gesetztem Schalter.

\newif\ifkorrekturansicht
\korrekturansichtfalse

\input{../tex-inputs/latex-vorspann}


\section[ Paul Goldmann an Arthur Schnitzler, 4. 4. {[}1897{]}]{L02807 Paul Goldmann an Arthur Schnitzler,  4. 4. [1897]}
\nopagebreak\mylabel{L02807v}
\rehead{ }\normalsize\beginnumbering\briefempfaengerindex{Schnitzler, Arthur@\textsc{Schnitzler, Arthur}!zzzGoldmann, Paul@\emph{von Paul Goldmann}!1897-04-041@{4. 4. [1897]}|(be}
\toendnotes[C]{\smallbreak\pagebreak[2]}
\correspDesc{Versand  durch Paul Goldmann am 4. 4. [1897] in Paris
\newline{}Erhalt  durch Arthur Schnitzler im Zeitraum [5. 4. 1897
                  – 9. 4. 1897?] in Wien}\toendnotes[C]{\smallbreak}
\Standort{DLA, A:Schnitzler, HS.NZ85.1.3167.}
\physDesc{Brief, 1 Blatt, 4 Seiten, 1544 Zeichen
\newline{}Handschrift: blaue Tinte, deutsche Kurrent
\newline{}Schnitzler: mit Bleistift das Jahr »97« vermerkt }\toendnotes[C]{\smallbreak}
\pstart
           {\pb}\textcolor{gray}{\textbf{\textbf{Frankfurter Zeitung\orgindex{Frankfurter Zeitung@Frankfurter Zeitung|pw}}}}\pend
           
\pstart
           \textcolor{gray}{\textbf{(\begin{otherlanguage}{french}Gazette de Francfort\end{otherlanguage}\orgindex{Frankfurter Zeitung@Frankfurter Zeitung|pw}).}}\pend
           
\pstart
           \textcolor{gray}{\textbf{\textbf{\begin{otherlanguage}{french}Fondateur M.\end{otherlanguage}{ }L. Sonnemann\pwindex{Sonnemann, Leopold 29.\,10.\,1831 Höchberg – 30.\,10.\,1909 Frankfurt am Main@\textsc{Sonnemann, Leopold} (29.\,10.\,1831 Höchberg – 30.\,10.\,1909 Frankfurt am Main), \emph{Journalist, Herausgeber}|pw}.}}}\pend
           
\pstart
           \begin{otherlanguage}{french}\textcolor{gray}{\textbf{Journal politique, financier,}}\end{otherlanguage}\pend
           
\pstart
           \begin{otherlanguage}{french}\textcolor{gray}{\textbf{commercial et littéraire.}}\end{otherlanguage}\pend
           
\pstart
           \begin{otherlanguage}{french}\textcolor{gray}{\textbf{\textbf{Paraissant trois fois par jour.}}}\end{otherlanguage}\hfill \textsc{Paris\oindex{Paris@\textbf{Paris}, \emph{Hauptstadt}|pw}}, 4. April.\pend
           
\pstart
           \begin{otherlanguage}{french}\textcolor{gray}{\textbf{\textbf{Bureau à Paris\oindex{Paris@\textbf{Paris}, \emph{Hauptstadt}|pw}}}}\end{otherlanguage}\pend
           
\pstart
           \begin{otherlanguage}{french}\textcolor{gray}{\textbf{\textbf{24. Rue Feydeau\oindex{rue Feydeau@\textbf{rue Feydeau}, \emph{Straße}|pw}.}}}\end{otherlanguage}\pend
           
\pstart\center{}Mein lieber Freund,\pend\vspace{0.5em}
\pstart
           Hoffentlich erreichen dieſe Zeilen Dich noch. Sie{ }ſollen Dir nichts{ }ſagen, als daß
               ich Dir von ganzem Herzen glückliche \label{K_L02807-1v}\edtext{Reiſe}{\lemma{\textnormal{\emph{Reise}}}\Cendnote{\textnormal{Schnitzler reiste am 7. 4. 1897 nach München\oindex{München@\textbf{München}|pwk} ab, am 10. 4. 1897 ging es
                  für ihn weiter nach Zürich\oindex{Zürich@\textbf{Zürich}|pwk}. Vom 12. 4. 1897 bis zum 24. 5. 1897 war er in
                     Paris\oindex{Paris@\textbf{Paris}, \emph{Hauptstadt}|pwk}, dann bis 1. 6. 1897 in London\oindex{London@\textbf{London}, \emph{Hauptstadt}|pwk}. Am 2. 6. 1897 kam er wieder nach Wien\oindex{Wien@\textbf{Wien}, \emph{Verwaltungsgebiet}|pwk}.}}}\label{K_L02807-1} wünſche und daß ich mich unendlich auf das
               Wiederſehen mit Dir freue (obwohl es nicht nöthig iſt, das zu{ }ſagen) {\dotsfour}\pend
           
\pstart
           Ich denke ans \textsc{Hotel de l’Athénée\oindex{Hotel de l’Athénée@\textbf{Hotel de l’Athénée}, \emph{Hotel}|pw}}. Im Centrum der Stadt\oindex{Paris@\textbf{Paris}, \emph{Hauptstadt}|pwv},
               hinter der Oper\oindex{Opéra Garnier@\textbf{Opéra Garnier}, \emph{Oper}|pwv} gelegen.
               Größtentheils engl\oindex{England@\textbf{England}, \emph{Land}|pwv}iſche und fran\oindex{Frankreich@\textbf{Frankreich}|pwv}zöſiſche Kundſchaft. Nie hat{ }ſich noch ein {\pb}\strikeout{Englä\oindex{England@\textbf{England}, \emph{Land}|pwv}\textcolor{gray}{×}}{ }Öſterreich\oindex{Österreich@\textbf{Österreich}|pw}er dorthin verirrt. Preis: ein Zimmer
               im vierten Stock (\label{K_L02807-2v}\edtext{\begin{otherlanguage}{french}\textsc{Ascenseur}\end{otherlanguage}}{\lemma{\textnormal{\emph{Ascenseur}}}\Cendnote{\textnormal{französisch: Aufzug}}}\label{K_L02807-2}) 7 \textsc{Francs}, in einem niedrigen Stockwerk natürlich theurer.
               Gegenwärtig iſt das Haus\oindex{Hotel de l’Athénée@\textbf{Hotel de l’Athénée}, \emph{Hotel}|pwv}
               (welches als vortrefflich bekannt iſt) bis unters Dach gefüllt. Man hat mir aber
               verſprochen, daß, wenn ich drei Tage vorher Deine Ankunft melde, man mir zwei Zimmer
               reſerviren wird. Im Centrum\oindex{Paris@\textbf{Paris}, \emph{Hauptstadt}|pwv}
               mußt Du wohnen, ich hab’ mir das überlegt: Du verlierſt{ }ſonſt zuviel Zeit. Auch
               könnte \uline{ich} Dich{ }ſonſt zu{ }ſelten{ }ſehen.\pend
           
\pstart
           {\pb}Wenn ich das Reiſegeld habe (was zurſtunde mehr als
               fraglich iſt) und wenn im \label{K_L02807-3v}\edtext{Orient}{\lemma{\textnormal{\emph{Orient}}}\Cendnote{\textnormal{Siehe XXXX Auszeichnungsfehler: Dokument L02805 nicht gefunden.
               }}}\label{K_L02807-3} kein Krieg ausbricht, fahre ich nach Frankfurt\oindex{Frankfurt am Main@\textbf{Frankfurt am Main}, \emph{Hauptstadt}|pw} um den 19. April herum und
               bleibe 10 bis 14 Tage.\pend
           
\pstart
           Damenſtrohhüte? Wird \strikeout{de} das Fräulein\pwindex{Reinhard, Marie 13.\,3.\,1871 Wien – 18.\,3.\,1899 ebd.@\textsc{Reinhard, Marie} (13.\,3.\,1871 Wien – 18.\,3.\,1899 ebd.), \emph{Gesangspädagogin}|pwv} im \textsc{Louvre}\oindex{Musée du Louvre@\textbf{Musée du Louvre}, \emph{Museum}|pw} oder \label{K_L02807-4v}\edtext{\textsc{Bon Marché\oindex{Le Bon Marché@\textbf{Le Bon Marché}, \emph{Geschäft}|pw}}}{\lemma{\textnormal{\emph{Bon Marché}}}\Cendnote{\textnormal{Kaufhaus\oindex{Le Bon Marché@\textbf{Le Bon Marché}, \emph{Geschäft}|pwkv} im siebten \begin{otherlanguage}{french}Arrondissement\end{otherlanguage}\oindex{7. arrondissement [Paris]@\textbf{7. arrondissement [Paris]}|pwk}}}}\label{K_L02807-4} kaufen. \strikeout{A} Außerdem kann{ }ſie{ }ſonſt zwiſchen
               tauſend und einigen Geſchäften wählen.\pend
           
\pstart
           Cylinder? Den{ }ſollſt Du gewiß mitbringen, wenn Du hier Beſuche machen willſt. Wenn
                  {\pb}Du ihn nicht mitbringſt,{ }ſo{ }ſchadet es auch
               nichts.\pend
           
\pstart
           Grüß’ Dich Gott, liebſter Freund,{ }ſchreib’ mir ein Wort \strikeout{\textcolor{gray}{×}} von unterwegs und komme{ }ſo bald als möglich!\pend
           
\pstart
           Dein treuer {\\[\baselineskip]}\spacefill\mbox{Paul Goldmann.}\pend
           \leftskip=0em{}
\pstart
           \noindent{}Habe natürlich keinem Menſchen eine Sylbe von Deiner bevorſtehenden Ankunft
                  geſagt.\pend
           \selectlanguage{ngerman}\endnumbering\briefempfaengerindex{Schnitzler, Arthur@\textsc{Schnitzler, Arthur}!zzzGoldmann, Paul@\emph{von Paul Goldmann}!1897-04-041@{4. 4. [1897]}|)be}\mylabel{L02807h}  \newcommand{\dateiname}{L02807}\newcommand{\titel}{Paul Goldmann an Arthur Schnitzler, 4. 4. [1897]}\newcommand{\editorInnen}{Martin Anton Müller und Laura Untner}%% latex-leseansicht-abspann.tex
%% Abspann für die Leseansicht.
%% Der Schalter \ifkorrekturansicht ist bereits durch den Vorspann gesetzt.

%% latex-abspann.tex
%% Gemeinsamer Abspann für Korrekturansicht und Leseansicht.
%% Setzt den Schalter \ifkorrekturansicht voraus (gesetzt in den
%% einbindenden Dateien latex-korrekturansicht-abspann.tex bzw.
%% latex-leseansicht-abspann.tex).
%% ---------------------------------------------------------------

\normalsize

% Das esempio-Environment wird nur in der Leseansicht benötigt
\ifkorrekturansicht\else
\newenvironment{esempio}[3]%
{
    \vspace{1.5ex}
    \rlap{\underline{#1}}
    \par
    \setlength{\parindent}{0cm}
    \nopagebreak
    \leftskip=#2cm
    \rightskip=#3cm
}
{
    \par
}
\fi

\doendnotes{C}
\bigskip
\vfill

\clearpage

\footnotesize

\ifkorrekturansicht
  \lohead{\textsc{register}}
\fi

% theindex-Environment neu definieren ohne reledmac
\makeatletter
\renewenvironment{theindex}{%
  \ifkorrekturansicht
    \section*{\indexname}%
  \else
    \subsubsection*{Index der erwähnten Entitäten}%
  \fi
  \setlength{\parindent}{0pt}%
  \setlength{\parskip}{0pt plus 0.3pt}%
  \let\item\@idxitem
}{%
  \ifkorrekturansicht\clearpage\fi
}
\makeatother

\IfFileExists{\jobname-pw.ind}{\input{\jobname-pw.ind}}{}

% Quellenangabe nur in der Leseansicht
\ifkorrekturansicht\else
% Fallback-Definitionen, falls die .tex-Datei \titel etc. nicht gesetzt hat
\providecommand{\titel}{}
\providecommand{\editorInnen}{}
\providecommand{\dateiname}{\jobname}

\vspace{3cm}

\vfill

\footnotesize
\textsc{Quelle}: \titel. Herausgegeben von {\editorInnen}. In: \emph{Arthur Schnitzler: Briefwechsel mit Autorinnen und Autoren}.
 Digitale Edition, https://schnitzler-briefe.acdh.oeaw.ac.at/{\dateiname}.html (Stand \today)
\fi

\end{document}


