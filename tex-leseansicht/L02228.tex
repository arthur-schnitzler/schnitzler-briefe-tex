%% latex-korrekturansicht-vorspann.tex
%% Vorspann für die Korrekturansicht.
%% Lädt die gemeinsame Datei latex-vorspann.tex mit gesetztem Schalter.

\newif\ifkorrekturansicht
\korrekturansichttrue

\input{../tex-inputs/latex-vorspann}


\section[Arthur Schnitzler an Robert Adam, 18. 5. 1916]{L02228 Arthur Schnitzler an Robert Adam, 18. 5. 1916}
\nopagebreak\mylabel{L02228v}
\rehead{ }\normalsize\beginnumbering\briefempfaengerindex{Adam, Robert@\textsc{Adam, Robert}!zzzSchnitzler, Arthur@\emph{von Arthur Schnitzler}!1916-05-181@{18. 5. 1916}|(be}
\toendnotes[C]{\smallbreak\pagebreak[2]}\Standort{DLA, 96.34.1/17.}
\physDesc{Postkarte, 185 Zeichen
\newline{}Handschrift: schwarze Tinte, deutsche Kurrent
\newline{}Versand: Stempel: »\nobreak{}\oindex{XVIII., Waehring@\textbf{XVIII., Währing}, \emph{A.ADM3}|pwk}18/1 Wien, 18. V. 16\nobreak{}«.  }\pstart{}{\pb}\textcolor{gray}{\textbf{Dr. Arthur Schnitzler}}\pend{}\pstart{}\textcolor{gray}{\textbf{Wien XVIII. Sternwartestrasse 71\oindex{Sternwartestrasse 71@\textbf{Sternwartestraße 71}, \emph{Wohngebäude (K.WHS)}|pw}}}\pend{}{\bigskip}\pstart{}Hrn \textsc{Dr. Robert Adam Pollak}\pend{}\pstart{}Wien \substVorne{}\textsuperscript{II}\substDazwischen{}X\substHinten{}II\oindex{XII., Meidling@\textbf{XII., Meidling}, \emph{A.ADM3}|pw}.\pend{}\pstart{}\textsc{Meidlinger Hptstr} 56\oindex{Meidlinger Hauptstrasse@\textbf{Meidlinger Hauptstraße}, \emph{Straße (K.STR)}|pw}.\pend{}{\bigskip}\vspace{1em}
\pstart
           \raggedleft{}{\pb}18. 5. 16.\pend
           
\pstart{}verehrter Herr Doktor,\pend\vspace{0.5em}
\pstart
           darf ich Sie bitten morgen ſchon um ½ 7{ }ſtatt 7 zu ko{\geminationm}en?\pend
           
\pstart
           herzlichſt grüßen\textcolor{gray}{d}{\\[\baselineskip]}Ihr ergebn\textcolor{gray}{er}{\\[\baselineskip]}\spacefill\mbox{ArthSchnitzler}\pend
           \leftskip=0em{}\selectlanguage{ngerman}\endnumbering\briefempfaengerindex{Adam, Robert@\textsc{Adam, Robert}!zzzSchnitzler, Arthur@\emph{von Arthur Schnitzler}!1916-05-181@{18. 5. 1916}|)be}\mylabel{L02228h}  \normalsize

\doendnotes{C}
\bigskip
\vfill

\clearpage

\footnotesize

\lohead{\textsc{register}}

% Definiere theindex-Environment komplett neu ohne reledmac
\makeatletter
\renewenvironment{theindex}{%
  \section*{\indexname}%
  \setlength{\parindent}{0pt}%
  \setlength{\parskip}{0pt plus 0.3pt}%
  \let\item\@idxitem
}{%
  \clearpage
}
\makeatother

\IfFileExists{\jobname-pw.ind}{\input{\jobname-pw.ind}}{}

\end{document}

      