%% latex-leseansicht-vorspann.tex
%% Vorspann für die Leseansicht.
%% Lädt die gemeinsame Datei latex-vorspann.tex mit nicht gesetztem Schalter.

\newif\ifkorrekturansicht
\korrekturansichtfalse

\input{../tex-inputs/latex-vorspann}


\section[Arthur Schnitzler an Stefan Zweig, 3[1?]. 7. 1929]{L03762 Arthur Schnitzler an Stefan Zweig, 3[1?]. 7. 1929}
\nopagebreak\mylabel{L03762v}
\rehead{ }\normalsize\beginnumbering\briefempfaengerindex{Zweig, Stefan@\textsc{Zweig, Stefan}!zzzSchnitzler, Arthur@\emph{von Arthur Schnitzler}!1929-07-311@{3[1?]. 7. 1929}|(be}
\toendnotes[C]{\smallbreak\pagebreak[2]}
\correspDesc{Versand  durch Arthur Schnitzler am 3[1?]. 7. 1929 in Wien
\newline{}Erhalt  durch Stefan Zweig im Zeitraum [1. 8. 1929
                  – 5. 8. 1929?] in Salzburg}\toendnotes[C]{\smallbreak}
\Standort{Jerusalem, National Library of Israel, ARC. Ms. Var. 305 1 58 Stefan Zweig Collection.}
\physDesc{Brief, 1 Blatt, 2 Seiten, 674 Zeichen (Briefpapier mit Trauerrand)
\newline{}Handschrift: Bleistift, lateinische Kurrent}\toendnotes[C]{\smallbreak}
\pstart
           \raggedleft{}{\pb}Wien\oindex{Wien@\textbf{Wien}, \emph{Verwaltungsgebiet}|pw}{ }3{[}1{]}/7 929\pend
           
\pstart{}lieber Doctor Stefan Zweig\pend\vspace{0.5em}
\pstart
           Die Widmung\pwindex{Zweig, Stefan 28.\,11.\,1881 Wien – 23.\,2.\,1942 Petrópolis@\textsc{Zweig, Stefan} (28.\,11.\,1881 Wien – 23.\,2.\,1942 Petrópolis), \emph{Schriftsteller}!Joseph Fouché. Bildnis eines politischen Menschen@\strich\emph{Joseph Fouché. Bildnis eines politischen Menschen}|pwv} wird mir besonders
               ehrenvoll sein – das \label{K_L03762-1v}\edtext{Probecapitel\pwindex{Zweig, Stefan 28.\,11.\,1881 Wien – 23.\,2.\,1942 Petrópolis@\textsc{Zweig, Stefan} (28.\,11.\,1881 Wien – 23.\,2.\,1942 Petrópolis), \emph{Schriftsteller}!Joseph Fouché, Herzog von Otranto. Der Endkampf mit Napoleon@\strich\emph{Joseph Fouché, Herzog von Otranto. Der Endkampf mit Napoleon}|pwv}}{\lemma{\textnormal{\emph{Probecapitel}}}\Cendnote{\textnormal{Stefan Zweig\pwindex{Zweig, Stefan 28.\,11.\,1881 Wien – 23.\,2.\,1942 Petrópolis@\textsc{Zweig, Stefan} (28.\,11.\,1881 Wien – 23.\,2.\,1942 Petrópolis), \emph{Schriftsteller}|pwk}: \emph{Joseph Fouché, Herzog von Otranto. Der Endkampf mit
                        Napoleon}\pwindex{Zweig, Stefan 28.\,11.\,1881 Wien – 23.\,2.\,1942 Petrópolis@\textsc{Zweig, Stefan} (28.\,11.\,1881 Wien – 23.\,2.\,1942 Petrópolis), \emph{Schriftsteller}!Joseph Fouché, Herzog von Otranto. Der Endkampf mit Napoleon@\strich\emph{Joseph Fouché, Herzog von Otranto. Der Endkampf mit Napoleon}|pwk}. In: \emph{Neue Freie Presse}\pwindex{Neue Freie Presse@\emph{Neue Freie Presse}|pwk},
                     Nr. 23.232, 19. 5. 1929, Morgenblatt, S. 33–42.}}}\label{K_L03762-1}, in
               d N. Fr. P.\pwindex{Neue Freie Presse@\emph{Neue Freie Presse}|pw} fand ich ganz außerordentlich{\dotstwo}\pend
           
\pstart
           Danke noch sehr für Ihre freundliche \label{K_L03762-2v}\edtext{Auskunft}{\lemma{\textnormal{\emph{Auskunft}}}\Cendnote{\textnormal{Siehe XXXX Auszeichnungsfehler: Dokument L03690 nicht gefunden.}}}\label{K_L03762-2} wegen der
               Versteigung. Haben Sie schon etwas geschickt? – Daſs nicht alle Manuscripte gleich
               viel werth sein können – darüber dürften auch die Autoren sich klar sein; – {\pb}am Ende ist dies auch bei Bilder Auctionen der Fall –
               trotzdem sind dort glaub ich Koppelungen nicht üblich – und ich bin auch überzeugt,
               daſs sie der Wohlth\textcolor{gray}{ä}tigkeitsaction von Vortheil sein werden. Aber
               ich werde nicht frondiren.\pend
           
\pstart
           Auf Wiedersehen, mein lieber Doctor Stefan Zweig und sehr herzliche Grüße{\\[\baselineskip]}\spacefill\mbox{ArthSchnitzler}\pend
           \leftskip=0em{}\selectlanguage{ngerman}\endnumbering\briefempfaengerindex{Zweig, Stefan@\textsc{Zweig, Stefan}!zzzSchnitzler, Arthur@\emph{von Arthur Schnitzler}!1929-07-311@{3[1?]. 7. 1929}|)be}\mylabel{L03762h}  \newcommand{\dateiname}{L03762}\newcommand{\titel}{Arthur Schnitzler an Stefan Zweig, 3[1?]. 7. 1929}\newcommand{\editorInnen}{Selma Jahnke und Martin Anton Müller}%% latex-leseansicht-abspann.tex
%% Abspann für die Leseansicht.
%% Der Schalter \ifkorrekturansicht ist bereits durch den Vorspann gesetzt.

%% latex-abspann.tex
%% Gemeinsamer Abspann für Korrekturansicht und Leseansicht.
%% Setzt den Schalter \ifkorrekturansicht voraus (gesetzt in den
%% einbindenden Dateien latex-korrekturansicht-abspann.tex bzw.
%% latex-leseansicht-abspann.tex).
%% ---------------------------------------------------------------

\normalsize

% Das esempio-Environment wird nur in der Leseansicht benötigt
\ifkorrekturansicht\else
\newenvironment{esempio}[3]%
{
    \vspace{1.5ex}
    \rlap{\underline{#1}}
    \par
    \setlength{\parindent}{0cm}
    \nopagebreak
    \leftskip=#2cm
    \rightskip=#3cm
}
{
    \par
}
\fi

\doendnotes{C}
\bigskip
\vfill

\clearpage

\footnotesize

\ifkorrekturansicht
  \lohead{\textsc{register}}
\fi

% theindex-Environment neu definieren ohne reledmac
\makeatletter
\renewenvironment{theindex}{%
  \ifkorrekturansicht
    \section*{\indexname}%
  \else
    \subsubsection*{Index der erwähnten Entitäten}%
  \fi
  \setlength{\parindent}{0pt}%
  \setlength{\parskip}{0pt plus 0.3pt}%
  \let\item\@idxitem
}{%
  \ifkorrekturansicht\clearpage\fi
}
\makeatother

\IfFileExists{\jobname-pw.ind}{\input{\jobname-pw.ind}}{}

% Quellenangabe nur in der Leseansicht
\ifkorrekturansicht\else
% Fallback-Definitionen, falls die .tex-Datei \titel etc. nicht gesetzt hat
\providecommand{\titel}{}
\providecommand{\editorInnen}{}
\providecommand{\dateiname}{\jobname}

\vspace{3cm}

\vfill

\footnotesize
\textsc{Quelle}: \titel. Herausgegeben von {\editorInnen}. In: \emph{Arthur Schnitzler: Briefwechsel mit Autorinnen und Autoren}.
 Digitale Edition, https://schnitzler-briefe.acdh.oeaw.ac.at/{\dateiname}.html (Stand \today)
\fi

\end{document}


