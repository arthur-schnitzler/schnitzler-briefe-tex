%% latex-korrekturansicht-vorspann.tex
%% Vorspann für die Korrekturansicht.
%% Lädt die gemeinsame Datei latex-vorspann.tex mit gesetztem Schalter.

\newif\ifkorrekturansicht
\korrekturansichttrue

\input{../tex-inputs/latex-vorspann}


\section[Arthur Schnitzler an Stefan Zweig, 3{[}1{]}. 7. 1929]{L03762 Arthur Schnitzler an Stefan Zweig, 3{[}1{]}. 7. 1929}
\nopagebreak\mylabel{L03762v}
\rehead{ }\normalsize\beginnumbering\briefempfaengerindex{Zweig, Stefan@\textsc{Zweig, Stefan}!zzzSchnitzler, Arthur@\emph{von Arthur Schnitzler}!1929-07-311@{3{[}1{]}. 7. 1929}|(be}
\toendnotes[C]{\smallbreak\pagebreak[2]}\Standort{Jerusalem, National Library of Israel, ARC. Ms. Var. 305 1 58 Stefan Zweig Collection.}
\physDesc{Brief, 1 Blatt, 2 Seiten, 678 Zeichen
\newline{}Handschrift: Bleistift, lateinische Kurrent}\toendnotes[C]{\smallbreak}
\pstart
           \raggedleft{}{\pb}Wien\oindex{Wien@\textbf{Wien}, \emph{A.ADM2}|pw}{ }3\textcolor{gray}{1.} 7 929\pend
           
\pstart{}lieber Doctor Stefan Zweig,\pend\vspace{0.5em}
\pstart
           die Widmung\pwindex{Joseph Fouche. Bildnis eines politischen Menschen@\emph{Joseph Fouché. Bildnis eines politischen Menschen}|pwv} wird mir besonders ehrenvoll sein – das \label{K_L03768-1v}\edtext{Probecapitel\pwindex{Joseph Fouche, Herzog von Otranto. Der Endkampf mit Napoleon@\emph{Joseph Fouché, Herzog von Otranto. Der Endkampf mit Napoleon}|pwv}}{\lemma{\textnormal{\emph{Probecapitel}}}\Cendnote{\textnormal{Stefan Zweig\pwindex{Zweig, Stefan 28.11.1881 – 23.02.1942@\textsc{Zweig, Stefan} (28.11.1881 – 23.02.1942), \emph{Schriftsteller/Schriftstellerin}|pwk}: 
                        \emph{Joseph Fouché, Herzog von Otranto. Der Endkampf mit Napoleon}\pwindex{Joseph Fouche, Herzog von Otranto. Der Endkampf mit Napoleon@\emph{Joseph Fouché, Herzog von Otranto. Der Endkampf mit Napoleon}|pwk}. In: \emph{Neue Freie Presse}\pwindex{Neue Freie Presse@\emph{Neue Freie Presse}|pwk},
                        Nr. 23.232, 19. 5. 1929, Morgenblatt, S. 33–42.}}}\label{K_L03768-1}, in d N. Fr. P.\pwindex{Neue Freie Presse@\emph{Neue Freie Presse}|pw} fand ich ganz außerordentlich. \pend
           
\pstart
           Danke noch sehr für Ihre freundliche Auskunft wegen der Versteigung. Haben Sie schon
               etwas geschickt? – Daſs nicht alle Manuscripte gleich viel wert sein können – darüber
               dürften auch die Autoren sich klar sein; – {\pb}am Ende ist
               dies auch bei Bilder Auctionen der Fall – trotzdem sind dort glaub ich Koppelungen
               recht üblich – und ich bin auch überzeugt, daſs sie der
               Wohlth\textcolor{gray}{ä}tigkeitsaction von Vortheil sein werden. Aber ich werde
               nicht frondiren.\pend
           
\pstart
           Auf Wiedersehen, mein lieber Doctor Stefan Zweig und sehr herzliche Grüße{\\[\baselineskip]}\spacefill\mbox{ArthSchnitzler}\pend
           \leftskip=0em{}\selectlanguage{ngerman}\endnumbering\briefempfaengerindex{Zweig, Stefan@\textsc{Zweig, Stefan}!zzzSchnitzler, Arthur@\emph{von Arthur Schnitzler}!1929-07-311@{3{[}1{]}. 7. 1929}|)be}\mylabel{L03762h}
\begin{anhang}
\end{anhang}\normalsize

\doendnotes{C}
\bigskip
\vfill

\clearpage

\footnotesize

\lohead{\textsc{register}}

% Definiere theindex-Environment komplett neu ohne reledmac
\makeatletter
\renewenvironment{theindex}{%
  \section*{\indexname}%
  \setlength{\parindent}{0pt}%
  \setlength{\parskip}{0pt plus 0.3pt}%
  \let\item\@idxitem
}{%
  \clearpage
}
\makeatother

\IfFileExists{\jobname-pw.ind}{\input{\jobname-pw.ind}}{}

\end{document}

      