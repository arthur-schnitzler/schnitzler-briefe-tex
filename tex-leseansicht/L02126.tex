%% latex-korrekturansicht-vorspann.tex
%% Vorspann für die Korrekturansicht.
%% Lädt die gemeinsame Datei latex-vorspann.tex mit gesetztem Schalter.

\newif\ifkorrekturansicht
\korrekturansichttrue

\input{../tex-inputs/latex-vorspann}


\section[Peter Altenberg an Arthur Schnitzler, {[}20.?{]} 4. 1913]{L02126 Peter Altenberg an Arthur Schnitzler, {[}20.?{]} 4. 1913}
\nopagebreak\mylabel{L02126v}
\rehead{ }\normalsize\beginnumbering\briefempfaengerindex{Schnitzler, Arthur@\textsc{Schnitzler, Arthur}!zzzAltenberg, Peter@\emph{von Peter Altenberg}!1913-04-202@{{[}20.?{]} 4. 1913}|(be}
\toendnotes[C]{\smallbreak\pagebreak[2]}\Standort{DLA, A:Schnitzler, HS.NZ85.1.2342, S. 14.}
\physDesc{Karte, maschinenschriftliche Abschrift1 Blatt, 1 Seite, 42 Zeichen
\newline{}Schreibmaschine}\toendnotes[C]{\smallbreak}
\pstart
           \raggedleft{}{\pb}?/4. 1913\pend
           \vspace{0.5em}
\pstart
           \uline{Tiefsten Dank} für \label{K_L02126-1v}\edtext{Besuch}{\lemma{\textnormal{\emph{Besuch}}}\Cendnote{\textnormal{Am 20. 4. 1913 besuchte
                     Schnitzler{ }Altenberg\pwindex{Altenberg, Peter 09.03.1859 – 08.01.1919@\textsc{Altenberg, Peter} (09.03.1859 – 08.01.1919), \emph{Schriftsteller/Schriftstellerin}|pwk} in der
                  Psychiatrie der Landesheilanstalt Am Steinhof\oindex{Otto-Wagner-Spital@\textbf{Otto-Wagner-Spital}, \emph{Krankenhaus (K.KKH)}|pwk}}}}\label{K_L02126-1}! Ihr\hspace*{1.5em}\spacefill\mbox{P. A.}\pend
           \selectlanguage{ngerman}\endnumbering\briefempfaengerindex{Schnitzler, Arthur@\textsc{Schnitzler, Arthur}!zzzAltenberg, Peter@\emph{von Peter Altenberg}!1913-04-202@{{[}20.?{]} 4. 1913}|)be}\mylabel{L02126h}  \normalsize

\doendnotes{C}
\bigskip
\vfill

\clearpage

\footnotesize

\lohead{\textsc{register}}

% Definiere theindex-Environment komplett neu ohne reledmac
\makeatletter
\renewenvironment{theindex}{%
  \section*{\indexname}%
  \setlength{\parindent}{0pt}%
  \setlength{\parskip}{0pt plus 0.3pt}%
  \let\item\@idxitem
}{%
  \clearpage
}
\makeatother

\IfFileExists{\jobname-pw.ind}{\input{\jobname-pw.ind}}{}

\end{document}

      