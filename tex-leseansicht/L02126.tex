%% latex-leseansicht-vorspann.tex
%% Vorspann für die Leseansicht.
%% Lädt die gemeinsame Datei latex-vorspann.tex mit nicht gesetztem Schalter.

\newif\ifkorrekturansicht
\korrekturansichtfalse

\input{../tex-inputs/latex-vorspann}


\section[Peter Altenberg an Arthur Schnitzler, {[}20.?{]} 4. 1913]{L02126 Peter Altenberg an Arthur Schnitzler, [20.?] 4. 1913}
\nopagebreak\mylabel{L02126v}
\rehead{ }\normalsize\beginnumbering\briefempfaengerindex{Schnitzler, Arthur@\textsc{Schnitzler, Arthur}!zzzAltenberg, Peter@\emph{von Peter Altenberg}!1913-04-202@{[20.?] 4. 1913}|(be}
\toendnotes[C]{\smallbreak\pagebreak[2]}
\correspDesc{Versand  durch Peter Altenberg am [20.?] 4. 1913 in Wien
\newline{}Erhalt  durch Arthur Schnitzler im Zeitraum [20. 4. 1913
                  – 24. 4. 1913?] in Wien}\toendnotes[C]{\smallbreak}
\Standort{DLA, A:Schnitzler, HS.NZ85.1.2342, S. 14.}
\physDesc{Karte, maschinenschriftliche Abschrift, 1 Blatt, 1 Seite, 42 Zeichen
\newline{}Schreibmaschine}\toendnotes[C]{\smallbreak}
\pstart
           \raggedleft{}{\pb}?/4. 1913\pend
           \vspace{0.5em}
\pstart
           \uline{Tiefsten Dank} für \label{K_L02126-1v}\edtext{Besuch}{\lemma{\textnormal{\emph{Besuch}}}\Cendnote{\textnormal{Am 20. 4. 1913 besuchte
                     Schnitzler{ }Altenberg\pwindex{Altenberg, Peter 9.\,3.\,1859 Wien – 8.\,1.\,1919 ebd.@\textsc{Altenberg, Peter} (9.\,3.\,1859 Wien – 8.\,1.\,1919 ebd.), \emph{Schriftsteller}|pwk} in der
                  Psychiatrie der Landesheilanstalt Am Steinhof\oindex{Wien@\textbf{Wien}!XIV., Penzing@\textbf{XIV., Penzing}!Otto-Wagner-Spital@\textbf{Otto-Wagner-Spital}, \emph{Krankenhaus}|pwk}}}}\label{K_L02126-1}! Ihr\hspace*{1.5em}\spacefill\mbox{P. A.}\pend
           \selectlanguage{ngerman}\endnumbering\briefempfaengerindex{Schnitzler, Arthur@\textsc{Schnitzler, Arthur}!zzzAltenberg, Peter@\emph{von Peter Altenberg}!1913-04-202@{[20.?] 4. 1913}|)be}\mylabel{L02126h}  \newcommand{\dateiname}{L02126}\newcommand{\titel}{Peter Altenberg an Arthur Schnitzler, [20.?] 4. 1913}\newcommand{\editorInnen}{Martin Anton Müller und Gerd-Hermann Susen}%% latex-leseansicht-abspann.tex
%% Abspann für die Leseansicht.
%% Der Schalter \ifkorrekturansicht ist bereits durch den Vorspann gesetzt.

%% latex-abspann.tex
%% Gemeinsamer Abspann für Korrekturansicht und Leseansicht.
%% Setzt den Schalter \ifkorrekturansicht voraus (gesetzt in den
%% einbindenden Dateien latex-korrekturansicht-abspann.tex bzw.
%% latex-leseansicht-abspann.tex).
%% ---------------------------------------------------------------

\normalsize

% Das esempio-Environment wird nur in der Leseansicht benötigt
\ifkorrekturansicht\else
\newenvironment{esempio}[3]%
{
    \vspace{1.5ex}
    \rlap{\underline{#1}}
    \par
    \setlength{\parindent}{0cm}
    \nopagebreak
    \leftskip=#2cm
    \rightskip=#3cm
}
{
    \par
}
\fi

\doendnotes{C}
\bigskip
\vfill

\clearpage

\footnotesize

\ifkorrekturansicht
  \lohead{\textsc{register}}
\fi

% theindex-Environment neu definieren ohne reledmac
\makeatletter
\renewenvironment{theindex}{%
  \ifkorrekturansicht
    \section*{\indexname}%
  \else
    \subsubsection*{Index der erwähnten Entitäten}%
  \fi
  \setlength{\parindent}{0pt}%
  \setlength{\parskip}{0pt plus 0.3pt}%
  \let\item\@idxitem
}{%
  \ifkorrekturansicht\clearpage\fi
}
\makeatother

\IfFileExists{\jobname-pw.ind}{\input{\jobname-pw.ind}}{}

% Quellenangabe nur in der Leseansicht
\ifkorrekturansicht\else
% Fallback-Definitionen, falls die .tex-Datei \titel etc. nicht gesetzt hat
\providecommand{\titel}{}
\providecommand{\editorInnen}{}
\providecommand{\dateiname}{\jobname}

\vspace{3cm}

\vfill

\footnotesize
\textsc{Quelle}: \titel. Herausgegeben von {\editorInnen}. In: \emph{Arthur Schnitzler: Briefwechsel mit Autorinnen und Autoren}.
 Digitale Edition, https://schnitzler-briefe.acdh.oeaw.ac.at/{\dateiname}.html (Stand \today)
\fi

\end{document}


