%% latex-korrekturansicht-vorspann.tex
%% Vorspann für die Korrekturansicht.
%% Lädt die gemeinsame Datei latex-vorspann.tex mit gesetztem Schalter.

\newif\ifkorrekturansicht
\korrekturansichttrue

\input{../tex-inputs/latex-vorspann}


\section[Arthur Schnitzler an Robert Adam, 21. 1. 1919]{L02320 Arthur Schnitzler an Robert Adam, 21. 1. 1919}
\nopagebreak\mylabel{L02320v}
\rehead{ }\normalsize\beginnumbering\briefempfaengerindex{Adam, Robert@\textsc{Adam, Robert}!zzzSchnitzler, Arthur@\emph{von Arthur Schnitzler}!1919-01-211@{21. 1. 1919}|(be}
\toendnotes[C]{\smallbreak\pagebreak[2]}\Standort{DLA, 96.34.2/17.}
\physDesc{Postkarte, 413 Zeichen
\newline{}Handschrift: schwarze Tinte, deutsche Kurrent
\newline{}Versand: Stempel: »\nobreak{}4\nobreak{}«.  }\pstart{}{\pb}A. S. Wien XVIII, \textsc{Sternwartestrasse} 71\oindex{Sternwartestrasse 71@\textbf{Sternwartestraße 71}, \emph{Wohngebäude (K.WHS)}|pw}\pend{}{\bigskip}\pstart{}Herrn Landesgerichtsrath\pend{}\pstart{}\textsc{Dr. Robert Adam Pollak}\pend{}\pstart{}\textsc{Wien} XII\oindex{XII., Meidling@\textbf{XII., Meidling}, \emph{A.ADM3}|pw}.\pend{}\pstart{}\textsc{Meidlinger Hptstr.} 58\oindex{Meidlinger Hauptstrasse@\textbf{Meidlinger Hauptstraße}, \emph{Straße (K.STR)}|pw}\pend{}{\bigskip}\vspace{1em}
\pstart
           \raggedleft{}{\pb}21. 1. 1919\pend
           
\pstart{}verehrteſter Herr Doktor,\pend\vspace{0.5em}
\pstart
           ſollten Sie eben am Freitag (24.) Abends nach
                  6 Zeit haben, ſo ſind Sie mir willko{\geminationm}en.
               Gottes Mühlen ſind \textsc{Express}angelegenheiten gegen
               direktoriale Entſchlüſſe. Davon weiſs ich manches Lied zu ſingen – ohne Lyriker zu
               ſein.\pend
           
\pstart
           Herzlich grüßt Sie Ihr ergebener{\\[\baselineskip]}\spacefill\mbox{Arthur Schnitzler}\pend
           \leftskip=0em{}\selectlanguage{ngerman}\endnumbering\briefempfaengerindex{Adam, Robert@\textsc{Adam, Robert}!zzzSchnitzler, Arthur@\emph{von Arthur Schnitzler}!1919-01-211@{21. 1. 1919}|)be}\mylabel{L02320h}  \normalsize

\doendnotes{C}
\bigskip
\vfill

\clearpage

\footnotesize

\lohead{\textsc{register}}

% Definiere theindex-Environment komplett neu ohne reledmac
\makeatletter
\renewenvironment{theindex}{%
  \section*{\indexname}%
  \setlength{\parindent}{0pt}%
  \setlength{\parskip}{0pt plus 0.3pt}%
  \let\item\@idxitem
}{%
  \clearpage
}
\makeatother

\IfFileExists{\jobname-pw.ind}{\input{\jobname-pw.ind}}{}

\end{document}

      