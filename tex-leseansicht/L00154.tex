%% latex-korrekturansicht-vorspann.tex
%% Vorspann für die Korrekturansicht.
%% Lädt die gemeinsame Datei latex-vorspann.tex mit gesetztem Schalter.

\newif\ifkorrekturansicht
\korrekturansichttrue

\input{../tex-inputs/latex-vorspann}


\section[Friedrich M. Fels an Arthur Schnitzler, {[}1. 1. 1893?{]}]{L00154 Friedrich M. Fels an Arthur Schnitzler, {[}1. 1. 1893?{]}}
\nopagebreak\mylabel{L00154v}
\rehead{ }\normalsize\beginnumbering\briefempfaengerindex{Schnitzler, Arthur@\textsc{Schnitzler, Arthur}!zzzFels, Friedrich Michael@\emph{von Friedrich Michael Fels}!1893-01-013@{{[}1. 1. 1893?{]}}|(be}
\toendnotes[C]{\smallbreak\pagebreak[2]}\Standort{DLA, A:Schnitzler, HS.NZ85.1.2956.}
\physDesc{Brief, 1 Blatt, 1 Seite, 708 Zeichen
\newline{}Handschrift: schwarze Tinte, lateinische Kurrent
\newline{}Schnitzler: mit Bleistift datiert: »93« und nummeriert: »6« }\toendnotes[C]{\smallbreak}
\pstart
           \noindent{}{\pb}Lieber Doktor Arthur! Das \label{K_L00154-1v}\edtext{Verfehlen}{\lemma{\textnormal{\emph{Verfehlen}}}\Cendnote{\textnormal{Vgl. A. S.: \emph{Tagebuch}, 1. 1. 1893: »Bei
                        Fels\pwindex{Fels, Friedrich Michael *~1864@\textsc{Fels, Friedrich Michael} (*~1864), \emph{Journalist/Journalistin}|pw}; verschlossene Thür. (Er
                     krank.)«. Möglicherweise ist dieses undatierte Korrespondenzstück im
                  Anschluss an dieses Ereignis verfasst.}}}\label{K_L00154-1} heute war mir sehr unangenehm; de{\geminationn} kaum waren Sie in der \label{K_L00154-2v}\edtext{Reisnerstraſse\oindex{Reisnerstrasse@\textbf{Reisnerstraße}, \emph{Straße (K.STR)}|pw}}{\lemma{\textnormal{\emph{Reisnerstrasse}}}\Cendnote{\textnormal{Hier befand sich die Redaktion der \emph{Allgemeinen Kunst-Chronik}\orgindex{Allgemeine Kunst-Chronik@Allgemeine Kunst-Chronik|pwk}.}}}\label{K_L00154-2}, als ich hin
               kam. So ko{\geminationn}te ich den eckelhalften Weg in die Leopoldstadt\oindex{II., Leopoldstadt@\textbf{II., Leopoldstadt}, \emph{A.ADM3}|pw} nicht verhindern. Natürlich hatte ich
               gleich eine kleine Freude, als mir der Alte\pwindex{Lauser, Wilhelm 15.6.1836 – 11.11.1902@\textsc{Lauser, Wilhelm} (15.6.1836 – 11.11.1902), \emph{Schriftsteller/Schriftstellerin, Herausgeber/Herausgeberin}|pwv} eröffnete, we{\geminationn} ich noch
               ein paar Tage krank und arbeitsunfähig sei, er genötigt sei, die Stelle aufzugeben.
               Also jetzt \uline{muſs} ich gesund sein. We{\geminationn} ich ich nur eſsen kö{\geminationn}te?
               Große und wichtige Frage: darf ich baden?\pend
           
\pstart
           Künftig werde ich, um bei meinen 70 fl zu bleiben, schon um zehn oder halb elf aufs
               Bureau ko{\geminationm}en; Sie kö{\geminationn}en
               also zu früherer Zeit ko{\geminationm}en, vielleicht morgen?\pend
           
\pstart
           Herzlichst{\\[\baselineskip]}\spacefill\mbox{Fels}\pend
           \leftskip=0em{}
\pstart
           \noindent{}Das muſs ich kriegen: 1. Appetit, 2. die Möglichkeit zu gehen, ohne
                  umzufallen.\pend
           \selectlanguage{ngerman}\endnumbering\briefempfaengerindex{Schnitzler, Arthur@\textsc{Schnitzler, Arthur}!zzzFels, Friedrich Michael@\emph{von Friedrich Michael Fels}!1893-01-013@{{[}1. 1. 1893?{]}}|)be}\mylabel{L00154h}  \normalsize

\doendnotes{C}
\bigskip
\vfill

\clearpage

\footnotesize

\lohead{\textsc{register}}

% Definiere theindex-Environment komplett neu ohne reledmac
\makeatletter
\renewenvironment{theindex}{%
  \section*{\indexname}%
  \setlength{\parindent}{0pt}%
  \setlength{\parskip}{0pt plus 0.3pt}%
  \let\item\@idxitem
}{%
  \clearpage
}
\makeatother

\IfFileExists{\jobname-pw.ind}{\input{\jobname-pw.ind}}{}

\end{document}

      