%% latex-leseansicht-vorspann.tex
%% Vorspann für die Leseansicht.
%% Lädt die gemeinsame Datei latex-vorspann.tex mit nicht gesetztem Schalter.

\newif\ifkorrekturansicht
\korrekturansichtfalse

\input{../tex-inputs/latex-vorspann}


\section[Arthur Schnitzler an Hugo von Hofmannsthal, 27. 7. 1899]{L00951 Arthur Schnitzler an Hugo von Hofmannsthal, 27. 7. 1899}
\nopagebreak\mylabel{L00951v}
\rehead{ }\normalsize\beginnumbering\briefempfaengerindex{Hofmannsthal, Hugo von@\textsc{Hofmannsthal, Hugo von}!zzzSchnitzler, Arthur@\emph{von Arthur Schnitzler}!1899-07-271@{27. 7. 1899}|(be}
\toendnotes[C]{\smallbreak\pagebreak[2]}
\correspDesc{Versand  durch Arthur Schnitzler am 27. 7. 1899 in Velden am Wörthersee
\newline{}Erhalt  durch Hugo von Hofmannsthal im Zeitraum [28. 7. 1899
                  – 1. 8. 1899?] in Marienbad}\toendnotes[C]{\smallbreak}
\Standort{FDH, Hs-30885,85.}
\physDesc{Brief, 1 Blatt, 4 Seiten, 1278 Zeichen
\newline{}Handschrift: Bleistift, deutsche Kurrent}
\buchAbdrucke{\weitereDrucke{Hugo von Hofmannsthal, Arthur Schnitzler: \emph{Briefwechsel}. Herausgegeben von Therese Nickl und Heinrich Schnitzler. Frankfurt am Main: \emph{S. Fischer} 1964, S. 127–128.} }\toendnotes[C]{\smallbreak}
\pstart
           \raggedleft{}{\pb}\textsc{Velden, Pension Pundschu}\oindex{Pension Pundschu@\textbf{Pension Pundschu}, \emph{Hotel}|pw}{\\}27. 7. 99.\pend
           \vspace{0.5em}
\pstart
           mein lieber Hugo; etwa am 5. Auguſt{ }ſoll von \textsc{Toblach}\oindex{Toblach@\textbf{Toblach}, \emph{Verwaltungsgebiet}|pw} aus die Fußtour angetreten werden, Richard\pwindex{Beer-Hofmann, Richard 11.\,7.\,1866 Wien – 26.\,9.\,1945 New York City@\textsc{Beer-Hofmann, Richard} (11.\,7.\,1866 Wien – 26.\,9.\,1945 New York City), \emph{Schriftsteller}|pw}, (der bis dahin mit der Novelle\pwindex{Beer-Hofmann, Richard 11.\,7.\,1866 Wien – 26.\,9.\,1945 New York City@\textsc{Beer-Hofmann, Richard} (11.\,7.\,1866 Wien – 26.\,9.\,1945 New York City), \emph{Schriftsteller}!Tod Georgs@\strich\emph{Der Tod Georgs}|pwv} fertig iſt und der neulich, in viel beſſrer Sti{\geminationm}g als ich vermuthet, hier war, und den ich So{\geminationn}tag am \textsc{Millstätter\oindex{Millstätter See@\textbf{Millstätter See}, \emph{See}|pw}}ſee sehe), \textsc{Wassermann}\pwindex{Wassermann, Jakob 10.\,3.\,1873 Fürth – 1.\,1.\,1934 Altaussee@\textsc{Wassermann, Jakob} (10.\,3.\,1873 Fürth – 1.\,1.\,1934 Altaussee), \emph{Schriftsteller}|pw}, ich, (am End auch Rob. Hirſchfeld\pwindex{Hirschfeld, Robert 17.\,9.\,1857 Žďár nad Sázavou – 2.\,4.\,1914 Salzburg@\textsc{Hirschfeld, Robert} (17.\,9.\,1857 Žďár nad Sázavou – 2.\,4.\,1914 Salzburg), \emph{Journalist, Musikkritiker}|pw} und
                  we{\geminationn} er{ }ſich dazu entſchließt Gustav Schwk.\pwindex{Schwarzkopf, Gustav 7.\,11.\,1853 Wien – 13.\,11.\,1939 ebd.@\textsc{Schwarzkopf, Gustav} (7.\,11.\,1853 Wien – 13.\,11.\,1939 ebd.), \emph{Schriftsteller}|pw}); südtiroliſ\oindex{Südtirol@\textbf{Südtirol}, \emph{Verwaltungsgebiet}|pw}che Päſſe, Ende etwa 15. Auguſt in Trient\oindex{Trient@\textbf{Trient}|pw}, \textsc{resp}. Bozen\oindex{Bozen@\textbf{Bozen}, \emph{Hauptstadt}|pw}. Zweite {\pb}Hälfte
                  Auguſt verbring ich in Iſchl\oindex{Bad Ischl@\textbf{Bad Ischl}|pw}. \substVorne{}\textsuperscript{I}\substDazwischen{}S\substHinten{}o käme dann, wie es ja auch Ihnen lieb wäre, unſere thüringiſche\oindex{Thüringen@\textbf{Thüringen}, \emph{Land}|pw} Radpartie Anfang September. Bleiben
               wir aber dabei, wenns möglich.\pend
           
\pstart
           – Ich habe zu arbeiten begonnen; das Stück\pwindex{Schnitzler, Arthur 15.\,5.\,1862 Wien – 21.\,10.\,1931 ebd.@\textsc{Schnitzler, Arthur} (15.\,5.\,1862 Wien – 21.\,10.\,1931 ebd.), \emph{Schriftsteller, Mediziner}!Schleier der Beatrice. Schauspiel in fünf Akten@\strich\emph{Der Schleier der Beatrice. Schauspiel in fünf Akten}|pwv}; es war doch weiter als ich gedacht, und wenn ich
               auch auf der Reiſe arbeiten kann, bin ich im Herbſt am Ende {\pb}fertig. Manchmal{ }ſcheints mir dſs es was werden könnte –
               oft aber bin ich wie vor den Kopf geſchlagen. Das Gefühl hab ich halt noch immer, dſs
               ich nicht weiß – für wen eigentlich –?\pend
           
\pstart
           – Schreiben Sie mir gleich ein Wort nach \uline{\textsc{Toblach}, Südbahnhotel\oindex{Südbahnhotel [Toblach]@\textbf{Südbahnhotel [Toblach]}, \emph{Hotel}|pw}}. Wo werden Sie in der 2. Hälfte Auguſt{ }ſein? Und was Ihr Stück\pwindex{Hofmannsthal, Hugo von 1.\,2.\,1874 Wien – 15.\,7.\,1929 Rodaun@\textsc{Hofmannsthal, Hugo von} (1.\,2.\,1874 Wien – 15.\,7.\,1929 Rodaun), \emph{Schriftsteller}!Bergwerk zu Falun@\strich\emph{Das Bergwerk zu Falun}|pwv} anlangt,{ }ſo darf {\pb}man ja
               da wirklich{ }ſagen: »Glück auf –«?\pend
           
\pstart
           Das Bad hier war prächtig; nun freu ich mich aber, dſs ich wieder woanders hinkomme.
                  Waſſerm.\pwindex{Wassermann, Jakob 10.\,3.\,1873 Fürth – 1.\,1.\,1934 Altaussee@\textsc{Wassermann, Jakob} (10.\,3.\,1873 Fürth – 1.\,1.\,1934 Altaussee), \emph{Schriftsteller}|pw}{ }ſchreibt{ }ſeinen Roman\pwindex{Wassermann, Jakob 10.\,3.\,1873 Fürth – 1.\,1.\,1934 Altaussee@\textsc{Wassermann, Jakob} (10.\,3.\,1873 Fürth – 1.\,1.\,1934 Altaussee), \emph{Schriftsteller}!Geschichte der jungen Renate Fuchs@\strich\emph{Die Geschichte der jungen Renate Fuchs}|pwv} ab. –\pend
           
\pstart
           – In \textsc{Tobl}.\oindex{Toblach@\textbf{Toblach}, \emph{Verwaltungsgebiet}|pw} bin ich noch mit Mama\pwindex{Schnitzler, Louise 8.\,7.\,1840 Kőszeg – 9.\,9.\,1911 Wien@\textsc{Schnitzler, Louise} (8.\,7.\,1840 Kőszeg – 9.\,9.\,1911 Wien)|pwv} u Schweſter\pwindex{Hajek, Gisela 20.\,12.\,1867 Wien – 3.\,2.\,1953 Cambridge@\textsc{Hajek, Gisela} (20.\,12.\,1867 Wien – 3.\,2.\,1953 Cambridge)|pwv}.\pend
           \pstart Herzlichſt Ihr \spacefill\mbox{Arth}\pend{}\selectlanguage{ngerman}\endnumbering\briefempfaengerindex{Hofmannsthal, Hugo von@\textsc{Hofmannsthal, Hugo von}!zzzSchnitzler, Arthur@\emph{von Arthur Schnitzler}!1899-07-271@{27. 7. 1899}|)be}\mylabel{L00951h}  \newcommand{\dateiname}{L00951}\newcommand{\titel}{Arthur Schnitzler an Hugo von Hofmannsthal, 27. 7. 1899}\newcommand{\editorInnen}{Martin Anton Müller und Gerd-Hermann Susen}%% latex-leseansicht-abspann.tex
%% Abspann für die Leseansicht.
%% Der Schalter \ifkorrekturansicht ist bereits durch den Vorspann gesetzt.

%% latex-abspann.tex
%% Gemeinsamer Abspann für Korrekturansicht und Leseansicht.
%% Setzt den Schalter \ifkorrekturansicht voraus (gesetzt in den
%% einbindenden Dateien latex-korrekturansicht-abspann.tex bzw.
%% latex-leseansicht-abspann.tex).
%% ---------------------------------------------------------------

\normalsize

% Das esempio-Environment wird nur in der Leseansicht benötigt
\ifkorrekturansicht\else
\newenvironment{esempio}[3]%
{
    \vspace{1.5ex}
    \rlap{\underline{#1}}
    \par
    \setlength{\parindent}{0cm}
    \nopagebreak
    \leftskip=#2cm
    \rightskip=#3cm
}
{
    \par
}
\fi

\doendnotes{C}
\bigskip
\vfill

\clearpage

\footnotesize

\ifkorrekturansicht
  \lohead{\textsc{register}}
\fi

% theindex-Environment neu definieren ohne reledmac
\makeatletter
\renewenvironment{theindex}{%
  \ifkorrekturansicht
    \section*{\indexname}%
  \else
    \subsubsection*{Index der erwähnten Entitäten}%
  \fi
  \setlength{\parindent}{0pt}%
  \setlength{\parskip}{0pt plus 0.3pt}%
  \let\item\@idxitem
}{%
  \ifkorrekturansicht\clearpage\fi
}
\makeatother

\IfFileExists{\jobname-pw.ind}{\input{\jobname-pw.ind}}{}

% Quellenangabe nur in der Leseansicht
\ifkorrekturansicht\else
% Fallback-Definitionen, falls die .tex-Datei \titel etc. nicht gesetzt hat
\providecommand{\titel}{}
\providecommand{\editorInnen}{}
\providecommand{\dateiname}{\jobname}

\vspace{3cm}

\vfill

\footnotesize
\textsc{Quelle}: \titel. Herausgegeben von {\editorInnen}. In: \emph{Arthur Schnitzler: Briefwechsel mit Autorinnen und Autoren}.
 Digitale Edition, https://schnitzler-briefe.acdh.oeaw.ac.at/{\dateiname}.html (Stand \today)
\fi

\end{document}


