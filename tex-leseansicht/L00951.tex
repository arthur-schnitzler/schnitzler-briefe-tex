%% latex-leseansicht-vorspann.tex
%% Vorspann für die Leseansicht.
%% Lädt die gemeinsame Datei latex-vorspann.tex mit nicht gesetztem Schalter.

\newif\ifkorrekturansicht
\korrekturansichtfalse

\input{../tex-inputs/latex-vorspann}


         
         \renewcommand{\erwaehntePersonen}{Personen: Richard Beer-Hofmann, Gisela Hajek, Robert Hirschfeld, Hugo von Hofmannsthal, Louise Schnitzler, Gustav Schwarzkopf, Jakob Wassermann}
         \renewcommand{\erwaehnteOrte}{Orte: Bad Ischl, Bozen, Marienbad, Millstätter See, Pension Pundschu, Südbahnhotel, Südtirol, Thüringen, Toblach, Trient, Velden am Wörthersee}
         \renewcommand{\erwaehnteWerke}{Werke: Das Bergwerk zu Falun, Der Schleier der Beatrice. Schauspiel in fünf Akten, Der Tod Georgs, Die Geschichte der jungen Renate Fuchs}
               \section[Arthur Schnitzler an Hugo von Hofmannsthal, 27. 7. 1899]{ Arthur Schnitzler an Hugo von Hofmannsthal, 27. 7. 1899}\nopagebreak\mylabel{v}\rehead{ }\begin{ledgroupsized}[t]{13cm}\normalsize\beginnumbering\briefempfaengerindex{Hofmannsthal, Hugo von@\textsc{Hofmannsthal, Hugo von}!zzzSchnitzler, Arthur@\emph{von Arthur Schnitzler}!1899-07-271@{27. 7. 1899}|(be} \toendnotes[C]{\smallbreak\pagebreak[2]} \Standort{FDH, Hs-30885,85.}
\physDesc{Brief, 1 Blatt, 4 Seiten, 1278 Zeichen
\newline{}Handschrift: Bleistift, deutsche Kurrent}\buchAbdrucke{\weitereDrucke{Hugo von Hofmannsthal, Arthur Schnitzler: \emph{Briefwechsel}. Hg. Therese Nickl und Heinrich Schnitzler. Frankfurt am Main: \emph{S. Fischer} 1964, S. 127–128.} }\toendnotes[C]{\smallbreak}\pstart
           \raggedleft{}{\pb}\textsc{Velden, Pension Pundschu}\oindex{Pension Pundschu@\textbf{Pension Pundschu}|pw}{\\}27. 7. 99.\pend
           \pstart
           mein lieber Hugo; etwa am 5. Auguſt{ }ſoll von \textsc{Toblach}\oindex{Toblach@\textbf{Toblach}|pw} aus die Fußtour angetreten werden, Richard\pwindex{Beer-Hofmann, Richard 1866-07-11 – 1945-09-26@\textsc{Beer-Hofmann, Richard} (1866-07-11 – 1945-09-26), \emph{Schriftsteller}|pw}, (der bis dahin mit der Novelle\pwindex{Beer-Hofmann, Richard 1866-07-11 – 1945-09-26@\textsc{Beer-Hofmann, Richard} (1866-07-11 – 1945-09-26), \emph{Schriftsteller}!Tod Georgs1900@\strich\emph{Der Tod Georgs} {[}1900{]}|pwv} fertig iſt und der neulich, in viel beſſrer Sti{\geminationm}g als ich vermuthet, hier war, und den ich So{\geminationn}tag am \textsc{Millstätter\oindex{Millstaetter See@\textbf{Millstätter See}|pw}}ſee sehe), \textsc{Wassermann}\pwindex{Wassermann, Jakob 10.03.1873 – 01.01.1934@\textsc{Wassermann, Jakob} (10.03.1873 – 01.01.1934), \emph{Schriftsteller}|pw}, ich, (am End auch Rob. Hirſchfeld\pwindex{Hirschfeld, Robert 17.09.1857 – 02.04.1914@\textsc{Hirschfeld, Robert} (17.09.1857 – 02.04.1914), \emph{Journalist, Musikkritiker}|pw} und
                  we{\geminationn} er ſich dazu entſchließt Gustav Schwk.\pwindex{Schwarzkopf, Gustav 07.11.1853 – 13.11.1939@\textsc{Schwarzkopf, Gustav} (07.11.1853 – 13.11.1939), \emph{Schriftsteller}|pw}); südtiroliſ\oindex{Suedtirol@\textbf{Südtirol}|pw}che Päſſe, Ende etwa 15. Auguſt in Trient\oindex{Trient@\textbf{Trient}|pw}, \textsc{resp}. Bozen\oindex{Bozen@\textbf{Bozen}|pw}. Zweite {\pb}Hälfte
                  Auguſt verbring ich in Iſchl\oindex{Bad Ischl@\textbf{Bad Ischl}|pw}. \substVorne{}\textsuperscript{I}\substDazwischen{}S\substHinten{}o käme dann, wie es ja auch Ihnen lieb wäre, unſere thüringiſche\oindex{Thueringen@\textbf{Thüringen}|pw} Radpartie Anfang September. Bleiben
               wir aber dabei, wenns möglich.\pend
           \pstart
           – Ich habe zu arbeiten begonnen; das Stück\pwindex{Schnitzler, Arthur 15.05.1862 – 21.10.1931@\textsc{Schnitzler, Arthur} (15.05.1862 – 21.10.1931), \emph{Schriftsteller, Mediziner}!Schleier der Beatrice. Schauspiel in fuenf Akten1900-12-01@\strich\emph{Der Schleier der Beatrice. Schauspiel in fünf Akten} {[}1900-12-01{]}|pwv}; es war doch weiter als ich gedacht, und wenn ich
               auch auf der Reiſe arbeiten kann, bin ich im Herbſt am Ende {\pb}fertig. Manchmal ſcheints mir dſs es was werden könnte –
               oft aber bin ich wie vor den Kopf geſchlagen. Das Gefühl hab ich halt noch immer, dſs
               ich nicht weiß – für wen eigentlich –?\pend
           \pstart
           – Schreiben Sie mir gleich ein Wort nach \uline{\textsc{Toblach}, Südbahnhotel\oindex{Suedbahnhotel@\textbf{Südbahnhotel}|pw}}. Wo werden Sie in der 2. Hälfte Auguſt{ }ſein? Und was Ihr Stück\pwindex{Hofmannsthal, Hugo von 1874-02-01 – 1929-07-15@\textsc{Hofmannsthal, Hugo von} (1874-02-01 – 1929-07-15), \emph{Schriftsteller}!Bergwerk zu Falun1900 – 1933@\strich\emph{Das Bergwerk zu Falun} {[}1900 – 1933{]}|pwv} anlangt, ſo darf {\pb}man ja
               da wirklich ſagen: »Glück auf –«?\pend
           \pstart
           Das Bad hier war prächtig; nun freu ich mich aber, dſs ich wieder woanders hinkomme.
                  Waſſerm.\pwindex{Wassermann, Jakob 10.03.1873 – 01.01.1934@\textsc{Wassermann, Jakob} (10.03.1873 – 01.01.1934), \emph{Schriftsteller}|pw}{ }ſchreibt ſeinen Roman\pwindex{Wassermann, Jakob 10.03.1873 – 01.01.1934@\textsc{Wassermann, Jakob} (10.03.1873 – 01.01.1934), \emph{Schriftsteller}!Geschichte der jungen Renate Fuchs1900@\strich\emph{Die Geschichte der jungen Renate Fuchs} {[}1900{]}|pwv} ab. –\pend
           \pstart
           – In \textsc{Tobl}.\oindex{Toblach@\textbf{Toblach}|pw} bin ich noch mit Mama\pwindex{Schnitzler, Louise 1840-07-08 – 1911-09-09@\textsc{Schnitzler, Louise} (1840-07-08 – 1911-09-09)|pwv} u Schweſter\pwindex{Hajek, Gisela 20.12.1867 – 03.02.1953@\textsc{Hajek, Gisela} (20.12.1867 – 03.02.1953)|pwv}.\pend
           \pstart Herzlichſt Ihr \spacefill\mbox{Arth}\pend{}
         
         \endnumbering\mylabel{h}\end{ledgroupsized}  \newcommand{\dateiname}{L00951}\newcommand{\titel}{Arthur Schnitzler an Hugo von Hofmannsthal, 27. 7. 1899}\newcommand{\editorInnen}{Martin Anton Müller und Gerd-Hermann Susen}%% latex-leseansicht-abspann.tex
%% Abspann für die Leseansicht.
%% Der Schalter \ifkorrekturansicht ist bereits durch den Vorspann gesetzt.

%% latex-abspann.tex
%% Gemeinsamer Abspann für Korrekturansicht und Leseansicht.
%% Setzt den Schalter \ifkorrekturansicht voraus (gesetzt in den
%% einbindenden Dateien latex-korrekturansicht-abspann.tex bzw.
%% latex-leseansicht-abspann.tex).
%% ---------------------------------------------------------------

\normalsize

% Das esempio-Environment wird nur in der Leseansicht benötigt
\ifkorrekturansicht\else
\newenvironment{esempio}[3]%
{
    \vspace{1.5ex}
    \rlap{\underline{#1}}
    \par
    \setlength{\parindent}{0cm}
    \nopagebreak
    \leftskip=#2cm
    \rightskip=#3cm
}
{
    \par
}
\fi

\doendnotes{C}
\bigskip
\vfill

\clearpage

\footnotesize

\ifkorrekturansicht
  \lohead{\textsc{register}}
\fi

% theindex-Environment neu definieren ohne reledmac
\makeatletter
\renewenvironment{theindex}{%
  \ifkorrekturansicht
    \section*{\indexname}%
  \else
    \subsubsection*{Index der erwähnten Entitäten}%
  \fi
  \setlength{\parindent}{0pt}%
  \setlength{\parskip}{0pt plus 0.3pt}%
  \let\item\@idxitem
}{%
  \ifkorrekturansicht\clearpage\fi
}
\makeatother

\IfFileExists{\jobname-pw.ind}{\input{\jobname-pw.ind}}{}

% Quellenangabe nur in der Leseansicht
\ifkorrekturansicht\else
% Fallback-Definitionen, falls die .tex-Datei \titel etc. nicht gesetzt hat
\providecommand{\titel}{}
\providecommand{\editorInnen}{}
\providecommand{\dateiname}{\jobname}

\vspace{3cm}

\vfill

\footnotesize
\textsc{Quelle}: \titel. Herausgegeben von {\editorInnen}. In: \emph{Arthur Schnitzler: Briefwechsel mit Autorinnen und Autoren}.
 Digitale Edition, https://schnitzler-briefe.acdh.oeaw.ac.at/{\dateiname}.html (Stand \today)
\fi

\end{document}


      