%% latex-korrekturansicht-vorspann.tex
%% Vorspann für die Korrekturansicht.
%% Lädt die gemeinsame Datei latex-vorspann.tex mit gesetztem Schalter.

\newif\ifkorrekturansicht
\korrekturansichttrue

\input{../tex-inputs/latex-vorspann}


\section[Arthur Schnitzler an Hugo von Hofmannsthal, 27. 7. 1899]{L00951 Arthur Schnitzler an Hugo von Hofmannsthal, 27. 7. 1899}
\nopagebreak\mylabel{L00951v}
\rehead{ }\normalsize\beginnumbering\briefempfaengerindex{Hofmannsthal, Hugo von@\textsc{Hofmannsthal, Hugo von}!zzzSchnitzler, Arthur@\emph{von Arthur Schnitzler}!1899-07-271@{27. 7. 1899}|(be}
\toendnotes[C]{\smallbreak\pagebreak[2]}\Standort{FDH, Hs-30885,85.}
\physDesc{Brief, 1 Blatt, 4 Seiten, 1278 Zeichen
\newline{}Handschrift: Bleistift, deutsche Kurrent}
\buchAbdrucke{\weitereDrucke{Hugo von Hofmannsthal, Arthur Schnitzler: \emph{Briefwechsel}. Frankfurt am Main: \emph{S. Fischer} 1964, S. 127–128.} }\toendnotes[C]{\smallbreak}
\pstart
           \raggedleft{}{\pb}\textsc{Velden, Pension Pundschu}\oindex{Pension Pundschu@\textbf{Pension Pundschu}, \emph{Hotel (K.HTL)}|pw}{\\}27. 7. 99.\pend
           \vspace{0.5em}
\pstart
           mein lieber Hugo; etwa am 5. Auguſt{ }ſoll von \textsc{Toblach}\oindex{Toblach@\textbf{Toblach}, \emph{A.ADM3}|pw} aus die Fußtour angetreten werden, Richard\pwindex{Beer-Hofmann, Richard 1866-07-11 – 1945-09-26@\textsc{Beer-Hofmann, Richard} (1866-07-11 – 1945-09-26), \emph{Schriftsteller/Schriftstellerin}|pw}, (der bis dahin mit der Novelle\pwindex{Tod Georgs@\emph{Der Tod Georgs}|pwv} fertig iſt und der neulich, in viel beſſrer Sti{\geminationm}g als ich vermuthet, hier war, und den ich So{\geminationn}tag am \textsc{Millstätter\oindex{Millstaetter See@\textbf{Millstätter See}, \emph{See (N.SEE)}|pw}}ſee sehe), \textsc{Wassermann}\pwindex{Wassermann, Jakob 10.03.1873 – 01.01.1934@\textsc{Wassermann, Jakob} (10.03.1873 – 01.01.1934), \emph{Schriftsteller/Schriftstellerin}|pw}, ich, (am End auch Rob. Hirſchfeld\pwindex{Hirschfeld, Robert 17.09.1857 – 02.04.1914@\textsc{Hirschfeld, Robert} (17.09.1857 – 02.04.1914), \emph{Journalist/Journalistin, Musikkritiker/Musikkritikerin}|pw} und
                  we{\geminationn} er ſich dazu entſchließt Gustav Schwk.\pwindex{Schwarzkopf, Gustav 07.11.1853 – 13.11.1939@\textsc{Schwarzkopf, Gustav} (07.11.1853 – 13.11.1939), \emph{Schriftsteller/Schriftstellerin}|pw}); südtiroliſ\oindex{Suedtirol@\textbf{Südtirol}, \emph{A.ADM2}|pw}che Päſſe, Ende etwa 15. Auguſt in Trient\oindex{Trient@\textbf{Trient}, \emph{P.PPLA}|pw}, \textsc{resp}. Bozen\oindex{Bozen@\textbf{Bozen}, \emph{P.PPLA2}|pw}. Zweite {\pb}Hälfte
                  Auguſt verbring ich in Iſchl\oindex{Bad Ischl@\textbf{Bad Ischl}, \emph{P.PPL}|pw}. \substVorne{}\textsuperscript{I}\substDazwischen{}S\substHinten{}o käme dann, wie es ja auch Ihnen lieb wäre, unſere thüringiſche\oindex{Thueringen@\textbf{Thüringen}, \emph{A.ADM1}|pw} Radpartie Anfang September. Bleiben
               wir aber dabei, wenns möglich.\pend
           
\pstart
           – Ich habe zu arbeiten begonnen; das Stück\pwindex{Schleier der Beatrice. Schauspiel in fuenf Akten@\emph{Der Schleier der Beatrice. Schauspiel in fünf Akten}|pwv}; es war doch weiter als ich gedacht, und wenn ich
               auch auf der Reiſe arbeiten kann, bin ich im Herbſt am Ende {\pb}fertig. Manchmal ſcheints mir dſs es was werden könnte –
               oft aber bin ich wie vor den Kopf geſchlagen. Das Gefühl hab ich halt noch immer, dſs
               ich nicht weiß – für wen eigentlich –?\pend
           
\pstart
           – Schreiben Sie mir gleich ein Wort nach \uline{\textsc{Toblach}, Südbahnhotel\oindex{Suedbahnhotel [Toblach]@\textbf{Südbahnhotel [Toblach]}, \emph{Hotel (K.HTL)}|pw}}. Wo werden Sie in der 2. Hälfte Auguſt{ }ſein? Und was Ihr Stück\pwindex{Bergwerk zu Falun@\emph{Das Bergwerk zu Falun}|pwv} anlangt, ſo darf {\pb}man ja
               da wirklich ſagen: »Glück auf –«?\pend
           
\pstart
           Das Bad hier war prächtig; nun freu ich mich aber, dſs ich wieder woanders hinkomme.
                  Waſſerm.\pwindex{Wassermann, Jakob 10.03.1873 – 01.01.1934@\textsc{Wassermann, Jakob} (10.03.1873 – 01.01.1934), \emph{Schriftsteller/Schriftstellerin}|pw}{ }ſchreibt ſeinen Roman\pwindex{Geschichte der jungen Renate Fuchs@\emph{Die Geschichte der jungen Renate Fuchs}|pwv} ab. –\pend
           
\pstart
           – In \textsc{Tobl}.\oindex{Toblach@\textbf{Toblach}, \emph{A.ADM3}|pw} bin ich noch mit Mama\pwindex{Schnitzler, Louise 1840-07-08 – 1911-09-09@\textsc{Schnitzler, Louise} (1840-07-08 – 1911-09-09)|pwv} u Schweſter\pwindex{Hajek, Gisela 20.12.1867 – 03.02.1953@\textsc{Hajek, Gisela} (20.12.1867 – 03.02.1953)|pwv}.\pend
           \pstart Herzlichſt Ihr \spacefill\mbox{Arth}\pend{}\selectlanguage{ngerman}\endnumbering\briefempfaengerindex{Hofmannsthal, Hugo von@\textsc{Hofmannsthal, Hugo von}!zzzSchnitzler, Arthur@\emph{von Arthur Schnitzler}!1899-07-271@{27. 7. 1899}|)be}\mylabel{L00951h}  \normalsize

\doendnotes{C}
\bigskip
\vfill

\clearpage

\footnotesize

\lohead{\textsc{register}}

% Definiere theindex-Environment komplett neu ohne reledmac
\makeatletter
\renewenvironment{theindex}{%
  \section*{\indexname}%
  \setlength{\parindent}{0pt}%
  \setlength{\parskip}{0pt plus 0.3pt}%
  \let\item\@idxitem
}{%
  \clearpage
}
\makeatother

\IfFileExists{\jobname-pw.ind}{\input{\jobname-pw.ind}}{}

\end{document}

      