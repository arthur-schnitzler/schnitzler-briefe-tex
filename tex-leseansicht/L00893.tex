%% latex-leseansicht-vorspann.tex
%% Vorspann für die Leseansicht.
%% Lädt die gemeinsame Datei latex-vorspann.tex mit nicht gesetztem Schalter.

\newif\ifkorrekturansicht
\korrekturansichtfalse

\input{../tex-inputs/latex-vorspann}


\section[Richard Beer-Hofmann an Arthur Schnitzler, 24. 2. 1899]{L00893 Richard Beer-Hofmann an Arthur Schnitzler, 24. 2. 1899}
\nopagebreak\mylabel{L00893v}
\rehead{ }\normalsize\beginnumbering\briefempfaengerindex{Schnitzler, Arthur@\textsc{Schnitzler, Arthur}!zzzBeer-Hofmann, Richard@\emph{von Richard Beer-Hofmann}!1899-02-242@{24. 2. 1899}|(be}
\toendnotes[C]{\smallbreak\pagebreak[2]}
\correspDesc{Versand  durch Richard Beer-Hofmann am 24. 2. 1899 in Wien
\newline{}Erhalt  durch Arthur Schnitzler im Zeitraum [24. 2. 1899
                  – 28. 2. 1899?] in Wien}\toendnotes[C]{\smallbreak}
\Standort{CUL, Schnitzler, B 8.}
\physDesc{Brief, 1 Blatt, 1 Seite, 146 Zeichen
\newline{}Handschrift: schwarze Tinte, lateinische Kurrent
\newline{}Ordnung: mit Bleistift von unbekannter Hand nummeriert:
                                    »126« }\toendnotes[C]{\smallbreak}
\pstart
           \raggedleft{}{\pb}24/II 99\pend
           \vspace{0.5em}
\pstart
           Lieber Arthur! Gemischtes \label{K_L00893-1v}\edtext{Hausbrot}{\lemma{\textnormal{\emph{Hausbrot}}}\Cendnote{\textnormal{»Hausbrot«
                  als ein immer im Schrank verfügbares Lebensmittel steht sinnbildlich für eine
                  immer gern genossene Kost. Hier vermutlich in Anspielung auf die bevorstehende
                   Uraufführung\eventindex{Burgtheater@\textbf{Burgtheater}!Uraufführung von Der grüne Kakadu – Paracelsus – Die Gefährtin. Drei Einakter, 1.3.1899@Uraufführung von Der grüne Kakadu – Paracelsus – Die Gefährtin. Drei Einakter, 1.3.1899|pwkv} der drei Einakter \emph{Der grüne Kakadu –
                     Paracelsus – Die Gefährtin}\pwindex{Schnitzler, Arthur 15.\,5.\,1862 Wien – 21.\,10.\,1931 ebd.@\textsc{Schnitzler, Arthur} (15.\,5.\,1862 Wien – 21.\,10.\,1931 ebd.), \emph{Schriftsteller, Mediziner}!grüne Kakadu – Paracelsus – Die Gefährtin. Drei Einakter@\strich\emph{Der grüne Kakadu – Paracelsus – Die Gefährtin. Drei Einakter}|pwk} am 1. 3. 1899, denen er wünscht, auf Dauer im Repertoire des \emph{Burgtheaters}\orgindex{Burgtheater@Burgtheater|pwk} zu bleiben.}}}\label{K_L00893-1}, \uline{sehr} dünn, und \uline{sehr}
               fett, Ecksitz, Mittelgang, 7\textsuperscript{te} Reihe (= 2. R. Parquet.).
               Wenn er ganz durch ist. –\pend
           \pstart \spacefill\mbox{Richard}\pend{}\selectlanguage{ngerman}\endnumbering\briefempfaengerindex{Schnitzler, Arthur@\textsc{Schnitzler, Arthur}!zzzBeer-Hofmann, Richard@\emph{von Richard Beer-Hofmann}!1899-02-242@{24. 2. 1899}|)be}\mylabel{L00893h}  \newcommand{\dateiname}{L00893}\newcommand{\titel}{Richard Beer-Hofmann an Arthur Schnitzler, 24. 2. 1899}\newcommand{\editorInnen}{Martin Anton Müller und Gerd-Hermann Susen}%% latex-leseansicht-abspann.tex
%% Abspann für die Leseansicht.
%% Der Schalter \ifkorrekturansicht ist bereits durch den Vorspann gesetzt.

%% latex-abspann.tex
%% Gemeinsamer Abspann für Korrekturansicht und Leseansicht.
%% Setzt den Schalter \ifkorrekturansicht voraus (gesetzt in den
%% einbindenden Dateien latex-korrekturansicht-abspann.tex bzw.
%% latex-leseansicht-abspann.tex).
%% ---------------------------------------------------------------

\normalsize

% Das esempio-Environment wird nur in der Leseansicht benötigt
\ifkorrekturansicht\else
\newenvironment{esempio}[3]%
{
    \vspace{1.5ex}
    \rlap{\underline{#1}}
    \par
    \setlength{\parindent}{0cm}
    \nopagebreak
    \leftskip=#2cm
    \rightskip=#3cm
}
{
    \par
}
\fi

\doendnotes{C}
\bigskip
\vfill

\clearpage

\footnotesize

\ifkorrekturansicht
  \lohead{\textsc{register}}
\fi

% theindex-Environment neu definieren ohne reledmac
\makeatletter
\renewenvironment{theindex}{%
  \ifkorrekturansicht
    \section*{\indexname}%
  \else
    \subsubsection*{Index der erwähnten Entitäten}%
  \fi
  \setlength{\parindent}{0pt}%
  \setlength{\parskip}{0pt plus 0.3pt}%
  \let\item\@idxitem
}{%
  \ifkorrekturansicht\clearpage\fi
}
\makeatother

\IfFileExists{\jobname-pw.ind}{\input{\jobname-pw.ind}}{}

% Quellenangabe nur in der Leseansicht
\ifkorrekturansicht\else
% Fallback-Definitionen, falls die .tex-Datei \titel etc. nicht gesetzt hat
\providecommand{\titel}{}
\providecommand{\editorInnen}{}
\providecommand{\dateiname}{\jobname}

\vspace{3cm}

\vfill

\footnotesize
\textsc{Quelle}: \titel. Herausgegeben von {\editorInnen}. In: \emph{Arthur Schnitzler: Briefwechsel mit Autorinnen und Autoren}.
 Digitale Edition, https://schnitzler-briefe.acdh.oeaw.ac.at/{\dateiname}.html (Stand \today)
\fi

\end{document}


