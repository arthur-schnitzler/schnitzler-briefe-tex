%% latex-leseansicht-vorspann.tex
%% Vorspann für die Leseansicht.
%% Lädt die gemeinsame Datei latex-vorspann.tex mit nicht gesetztem Schalter.

\newif\ifkorrekturansicht
\korrekturansichtfalse

\input{../tex-inputs/latex-vorspann}


\section[Arthur Schnitzler an Stefan Zweig, 12. 11. 1915]{L03791 Arthur Schnitzler an Stefan Zweig, 12. 11. 1915}
\nopagebreak\mylabel{L03791v}
\rehead{ }\normalsize\beginnumbering\briefempfaengerindex{Zweig, Stefan@\textsc{Zweig, Stefan}!zzzSchnitzler, Arthur@\emph{von Arthur Schnitzler}!1915-11-121@{12. 11. 1915}|(be}
\toendnotes[C]{\smallbreak\pagebreak[2]}
\correspDesc{Versand  durch Arthur Schnitzler am 12. 11. 1915 in Wien
\newline{}Erhalt  durch Stefan Zweig im Zeitraum [12. 11. 1915 – 15. 11. 1915?] in Wien}\toendnotes[C]{\smallbreak}
\Standort{Jerusalem, National Library of Israel, ARC. Ms. Var. 305 1 58 Stefan Zweig Collection.}
\physDesc{Postkarte, 184 Zeichen
\newline{}Handschrift: schwarze Tinte, deutsche Kurrent
\newline{}Versand: Stempel: »\nobreak{}\oindex{Wien@\textbf{Wien}, \emph{Verwaltungsgebiet}|pwk}Wien 13, 1
                                       \textcolor{gray}{2}. XI. 15, 2\textsuperscript{20}\nobreak{}«.  }\toendnotes[C]{\smallbreak}\pstart{}{\pb}\textcolor{gray}{\textbf{Dr. Arthur
                  Schnitzler}}\pend{}\pstart{}\textcolor{gray}{\textbf{Wien XVIII.
                        Sternwartestrasse 71\oindex{Wien@\textbf{Wien}!XVIII., Währing@\textbf{XVIII., Währing}!Sternwartestraße 71@\textbf{Sternwartestraße 71}, \emph{Wohngebäude}|pw}}}\pend{}{\bigskip}\pstart{}Hrn.\pend{}\pstart{}\textsc{Dr Stefan Zweig}\pend{}\pstart{}Wien VIII\oindex{VIII., Josefstadt@\textbf{VIII., Josefstadt}, \emph{Verwaltungsgebiet}|pw}\pend{}\pstart{}\textsc{Kochgasse 8\oindex{Wien@\textbf{Wien}!VIII., Josefstadt@\textbf{VIII., Josefstadt}!Kochgasse 8@\textbf{Kochgasse 8}, \emph{Wohngebäude}|pw}}.\pend{}{\bigskip}\vspace{1em}
\pstart
           \raggedleft{}{\pb}12. 11. 91\textcolor{gray}{5}.\pend
           
\pstart{}lieber Herr Doktor!\pend\vspace{0.5em}
\pstart
           wollen Sie uns \label{K_L03791-1v}\edtext{So{\geminationn}tag}{\lemma{\textnormal{\emph{Sonntag}}}\Cendnote{\textnormal{Vgl. A. S.: \emph{Tagebuch}, 14. 11. 1915.}}}\label{K_L03791-1} zum Nachtmahl das Vergnügen{ }ſchenken? Wir wären ſehr erfreut. Vor acht!\pend
           
\pstart
           Herzlichſt Ihr{\\[\baselineskip]}\spacefill\mbox{ArthSchn}\pend
           \leftskip=0em{}\selectlanguage{ngerman}\endnumbering\briefempfaengerindex{Zweig, Stefan@\textsc{Zweig, Stefan}!zzzSchnitzler, Arthur@\emph{von Arthur Schnitzler}!1915-11-121@{12. 11. 1915}|)be}\mylabel{L03791h}  \newcommand{\dateiname}{L03791}\newcommand{\titel}{Arthur Schnitzler an Stefan Zweig, 12. 11. 1915}\newcommand{\editorInnen}{Selma Jahnke und Martin Anton Müller}%% latex-leseansicht-abspann.tex
%% Abspann für die Leseansicht.
%% Der Schalter \ifkorrekturansicht ist bereits durch den Vorspann gesetzt.

%% latex-abspann.tex
%% Gemeinsamer Abspann für Korrekturansicht und Leseansicht.
%% Setzt den Schalter \ifkorrekturansicht voraus (gesetzt in den
%% einbindenden Dateien latex-korrekturansicht-abspann.tex bzw.
%% latex-leseansicht-abspann.tex).
%% ---------------------------------------------------------------

\normalsize

% Das esempio-Environment wird nur in der Leseansicht benötigt
\ifkorrekturansicht\else
\newenvironment{esempio}[3]%
{
    \vspace{1.5ex}
    \rlap{\underline{#1}}
    \par
    \setlength{\parindent}{0cm}
    \nopagebreak
    \leftskip=#2cm
    \rightskip=#3cm
}
{
    \par
}
\fi

\doendnotes{C}
\bigskip
\vfill

\clearpage

\footnotesize

\ifkorrekturansicht
  \lohead{\textsc{register}}
\fi

% theindex-Environment neu definieren ohne reledmac
\makeatletter
\renewenvironment{theindex}{%
  \ifkorrekturansicht
    \section*{\indexname}%
  \else
    \subsubsection*{Index der erwähnten Entitäten}%
  \fi
  \setlength{\parindent}{0pt}%
  \setlength{\parskip}{0pt plus 0.3pt}%
  \let\item\@idxitem
}{%
  \ifkorrekturansicht\clearpage\fi
}
\makeatother

\IfFileExists{\jobname-pw.ind}{\input{\jobname-pw.ind}}{}

% Quellenangabe nur in der Leseansicht
\ifkorrekturansicht\else
% Fallback-Definitionen, falls die .tex-Datei \titel etc. nicht gesetzt hat
\providecommand{\titel}{}
\providecommand{\editorInnen}{}
\providecommand{\dateiname}{\jobname}

\vspace{3cm}

\vfill

\footnotesize
\textsc{Quelle}: \titel. Herausgegeben von {\editorInnen}. In: \emph{Arthur Schnitzler: Briefwechsel mit Autorinnen und Autoren}.
 Digitale Edition, https://schnitzler-briefe.acdh.oeaw.ac.at/{\dateiname}.html (Stand \today)
\fi

\end{document}


