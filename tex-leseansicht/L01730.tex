\input{../tex-inputs/latex-pdf-vorspann}
\begin{center}
            \textcolor{red}{ENTWURF. ENTZIFFERUNG NOCH NICHT KORREKTURGELESEN}
                      \end{center}
            
               \section[Max Burckhard: Widmungsexemplar Quer durch das Leben für Arthur Schnitzler, 14. 11. 1907]{ Max Burckhard: Widmungsexemplar Quer durch das Leben für Arthur Schnitzler,
                    14. 11. 1907}\nopagebreak\mylabel{v}\rehead{ }\begin{ledgroupsized}[t]{13cm}\normalsize\beginnumbering\briefempfaengerindex{Schnitzler, Arthur@\textsc{Schnitzler, Arthur}!zzzBurckhard, Max Eugen@\emph{von Max Eugen Burckhard}!1907-11-141@{14. 11. 1907}|(be} \toendnotes[C]{\smallbreak\pagebreak[2]} \Standort{DLA, G:Schnitzler, Arthur (Sammlung Heinrich Schnitzler).}
\physDesc{Widmung am Titelblatt
\newline{}Handschrift: schwarze Tinte, deutsche Kurrent\newline{}Ordnung: bei der Enteignung des Exemplars 1938 von unbekannter Hand mit Bleistift
                                    als Dublette markiert: »= 455782-B« }\pstart
           \noindent{}{\pb}Arthur Schnitzler herzlich in treuer
                    Verehrung\pend
           \pstart \spacefill\mbox{D\textsuperscript{r} Burckhard}\pend{}\pstart
           St. Gilgen\oindex{St. Gilgen@\textbf{St. Gilgen}|pw}{ }14/11 07\pend
           {\bigskip}\pstart
           \noindent{}\centering{}\textcolor{gray}{\textbf{QUER DURCH DAS LEBEN\pwindex{Burckhard, Max Eugen 14.07.1854 – 16.03.1912@\textsc{Burckhard, Max Eugen} (14.07.1854 – 16.03.1912), \emph{Schriftsteller, Rechtswissenschaftler, Theaterleiter}!Quer durch das Leben. Fuenfzig Aufsaetze1908 – 1908@\strich\emph{Quer durch das Leben. Fünfzig Aufsätze} {[}1908 – 1908{]}|pw}.}}\pend
           \pstart
           \noindent{}\centering{}\textcolor{gray}{\textbf{FÜNFZIG
                    AUFSÄTZE}}\pend
           \pstart
           \noindent{}\centering{}\textcolor{gray}{\textbf{VON}}{\\}\textcolor{gray}{\textbf{MAX BURCKHARD}}.\pend
           {\bigskip}\pstart
           \noindent{}\textcolor{gray}{\textbf{WIEN\oindex{Wien@\textbf{Wien}|pw}.}}\hfill \textcolor{gray}{\textbf{LEIPZIG\oindex{Leipzig@\textbf{Leipzig}|pw}.}}\pend
           \pstart
           \textcolor{gray}{\textbf{\so{F. Tempsky}\orgindex{F. Tempsky@F. Tempsky|pw}.}}\hfill \textcolor{gray}{\textbf{\so{G. Freytag G. m. b. H.}\orgindex{G. Freytag@G. Freytag|pw}}}\pend
           \pstart
           \centering{}\textcolor{gray}{\textbf{1908.}}\pend
           \endnumbering\briefempfaengerindex{Schnitzler, Arthur@\textsc{Schnitzler, Arthur}!zzzBurckhard, Max Eugen@\emph{von Max Eugen Burckhard}!1907-11-141@{14. 11. 1907}|)be}\mylabel{h}\end{ledgroupsized}  \newcommand{\dateiname}{L01730}\newcommand{\titel}{Max Burckhard: Widmungsexemplar Quer durch das Leben für Arthur Schnitzler, 14. 11. 1907}\newcommand{\editorInnen}{Martin Anton Müller und Gerd-Hermann Susen}\input{../tex-inputs/latex-pdf-abspann}
      