%% latex-leseansicht-vorspann.tex
%% Vorspann für die Leseansicht.
%% Lädt die gemeinsame Datei latex-vorspann.tex mit nicht gesetztem Schalter.

\newif\ifkorrekturansicht
\korrekturansichtfalse

\input{../tex-inputs/latex-vorspann}


\section[Max Burckhard: Widmungsexemplar Quer durch das Leben für Arthur Schnitzler, 14. 11. 1907]{L01730 Max Burckhard: Widmungsexemplar Quer durch das Leben für Arthur
               Schnitzler, 14. 11. 1907}
\nopagebreak\mylabel{L01730v}
\rehead{ }\normalsize\beginnumbering\briefempfaengerindex{Schnitzler, Arthur@\textsc{Schnitzler, Arthur}!zzzBurckhard, Max Eugen@\emph{von Max Eugen Burckhard}!1907-11-141@{14. 11. 1907}|(be}
\toendnotes[C]{\smallbreak\pagebreak[2]}
\correspDesc{Versand  durch Max Burckhard am 14. 11. 1907 in St. Gilgen
\newline{}Erhalt  durch Arthur Schnitzler im Zeitraum [15. 11. 1907 – 19. 11. 1907?] in Wien}\toendnotes[C]{\smallbreak}
\Standort{DLA, G:Schnitzler, Arthur (Sammlung Heinrich Schnitzler).}
\physDesc{Widmung am Titelblatt, 79 Zeichen
\newline{}Handschrift: schwarze Tinte, deutsche Kurrent
\newline{}Ordnung: bei der Enteignung des Exemplars 1938 von
                                 unbekannter Hand mit Bleistift als Dublette markiert:
                                    »= 455782-B« }
\pstart
           \noindent{}{\pb}Arthur Schnitzler herzlich in treuer
               Verehrung\pend
           \pstart \spacefill\mbox{D\textsuperscript{r} Burckhard}\pend{}
\pstart
           St. Gilgen\oindex{St. Gilgen@\textbf{St. Gilgen}, \emph{Verwaltungsgebiet}|pw}{ }14/11 07\pend
           {\vspace{1\baselineskip}}
\pstart
           \centering{}\textcolor{gray}{\textbf{QUER DURCH DAS LEBEN\pwindex{Burckhard, Max Eugen 14.\,7.\,1854 Korneuburg – 16.\,3.\,1912 Wien@\textsc{Burckhard, Max Eugen} (14.\,7.\,1854 Korneuburg – 16.\,3.\,1912 Wien), \emph{Schriftsteller, Rechtswissenschaftler, Theaterleiter}!Quer durch das Leben. Fünfzig Aufsätze@\strich\emph{Quer durch das Leben. Fünfzig Aufsätze}|pw}.}}\pend
           
\pstart
           \centering{}\textcolor{gray}{\textbf{FÜNFZIG AUFSÄTZE}}\pend
           
\pstart
           \centering{}\textcolor{gray}{\textbf{VON}}{\\}\textcolor{gray}{\textbf{MAX BURCKHARD}}.\pend
           {\vspace{1\baselineskip}}
\pstart
           
\pstart
           \textcolor{gray}{\textbf{WIEN\oindex{Wien@\textbf{Wien}, \emph{Verwaltungsgebiet}|pw}.}}\pend
           
\pstart
           \raggedleft{}\textcolor{gray}{\textbf{LEIPZIG\oindex{Leipzig@\textbf{Leipzig}, \emph{Hauptstadt}|pw}.}}\pend
           \pend
           
\pstart
           
\pstart
           \textcolor{gray}{\textbf{\so{F. TEMPSKY}\orgindex{F. Tempsky@F. Tempsky|pw}.}}\pend
           
\pstart
           \raggedleft{}\textcolor{gray}{\textbf{\so{G. FREYTAG G. M. B. H.}\orgindex{G. Freytag@G. Freytag|pw}}}\pend
           \pend
           
\pstart
           \centering{}\textcolor{gray}{\textbf{1908.}}\pend
           \selectlanguage{ngerman}\endnumbering\briefempfaengerindex{Schnitzler, Arthur@\textsc{Schnitzler, Arthur}!zzzBurckhard, Max Eugen@\emph{von Max Eugen Burckhard}!1907-11-141@{14. 11. 1907}|)be}\mylabel{L01730h}  \newcommand{\dateiname}{L01730}\newcommand{\titel}{Max Burckhard: Widmungsexemplar Quer durch das Leben für Arthur Schnitzler, 14. 11. 1907}\newcommand{\editorInnen}{Martin Anton Müller und Gerd-Hermann Susen}%% latex-leseansicht-abspann.tex
%% Abspann für die Leseansicht.
%% Der Schalter \ifkorrekturansicht ist bereits durch den Vorspann gesetzt.

%% latex-abspann.tex
%% Gemeinsamer Abspann für Korrekturansicht und Leseansicht.
%% Setzt den Schalter \ifkorrekturansicht voraus (gesetzt in den
%% einbindenden Dateien latex-korrekturansicht-abspann.tex bzw.
%% latex-leseansicht-abspann.tex).
%% ---------------------------------------------------------------

\normalsize

% Das esempio-Environment wird nur in der Leseansicht benötigt
\ifkorrekturansicht\else
\newenvironment{esempio}[3]%
{
    \vspace{1.5ex}
    \rlap{\underline{#1}}
    \par
    \setlength{\parindent}{0cm}
    \nopagebreak
    \leftskip=#2cm
    \rightskip=#3cm
}
{
    \par
}
\fi

\doendnotes{C}
\bigskip
\vfill

\clearpage

\footnotesize

\ifkorrekturansicht
  \lohead{\textsc{register}}
\fi

% theindex-Environment neu definieren ohne reledmac
\makeatletter
\renewenvironment{theindex}{%
  \ifkorrekturansicht
    \section*{\indexname}%
  \else
    \subsubsection*{Index der erwähnten Entitäten}%
  \fi
  \setlength{\parindent}{0pt}%
  \setlength{\parskip}{0pt plus 0.3pt}%
  \let\item\@idxitem
}{%
  \ifkorrekturansicht\clearpage\fi
}
\makeatother

\IfFileExists{\jobname-pw.ind}{\input{\jobname-pw.ind}}{}

% Quellenangabe nur in der Leseansicht
\ifkorrekturansicht\else
% Fallback-Definitionen, falls die .tex-Datei \titel etc. nicht gesetzt hat
\providecommand{\titel}{}
\providecommand{\editorInnen}{}
\providecommand{\dateiname}{\jobname}

\vspace{3cm}

\vfill

\footnotesize
\textsc{Quelle}: \titel. Herausgegeben von {\editorInnen}. In: \emph{Arthur Schnitzler: Briefwechsel mit Autorinnen und Autoren}.
 Digitale Edition, https://schnitzler-briefe.acdh.oeaw.ac.at/{\dateiname}.html (Stand \today)
\fi

\end{document}


