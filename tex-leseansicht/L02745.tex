%% latex-leseansicht-vorspann.tex
%% Vorspann für die Leseansicht.
%% Lädt die gemeinsame Datei latex-vorspann.tex mit nicht gesetztem Schalter.

\newif\ifkorrekturansicht
\korrekturansichtfalse

\input{../tex-inputs/latex-vorspann}


         
         \renewcommand{\erwaehntePersonen}{Personen: Lou Andreas-Salomé, Fedor Mamroth, Felix Salten, Leopold Sonnemann}
         \renewcommand{\erwaehnteInstitutionen}{Institutionen: Frankfurter Zeitung}
         \renewcommand{\erwaehnteOrte}{Orte: Bad Ischl, Bad Krankenheil, Bad Tölz, Paris, Villa Carlo, rue Feydeau}
         \renewcommand{\erwaehnteWerke}{}
               \section[Paul Goldmann an Arthur Schnitzler, 19. 8. {[}1895{]}]{ Paul Goldmann an Arthur Schnitzler, 19. 8. {[}1895{]}}\nopagebreak\mylabel{v}\rehead{ }\begin{ledgroupsized}[t]{13cm}\normalsize\beginnumbering \toendnotes[C]{\smallbreak\pagebreak[2]} \Standort{DLA, A:Schnitzler, HS.NZ85.1.3165.}
\physDesc{Brief, 1 Blatt, 4 Seiten, 1483 Zeichen
\newline{}Handschrift: schwarze Tinte, deutsche Kurrent
\newline{}Schnitzler: 1) mit Bleistift das Jahr »95« vermerkt  2) mit rotem Buntstift zwei Unterstreichungen}\toendnotes[C]{\smallbreak}\pstart
           \noindent{}{\pb}\textcolor{gray}{\textbf{\textbf{Frankfurter Zeitung\orgindex{Frankfurter Zeitung@Frankfurter Zeitung|pw}}}}\pend
           \pstart
           \textcolor{gray}{\textbf{(\begin{otherlanguage}{french}Gazette de Francfort\end{otherlanguage}\orgindex{Frankfurter Zeitung@Frankfurter Zeitung|pw}). }}\pend
           \pstart
           \textcolor{gray}{\textbf{\textbf{\begin{otherlanguage}{french}Fondateur M. L.
                                 Sonnemann\pwindex{Sonnemann, Leopold 1831-10-29 – 1909-10-30@\textsc{Sonnemann, Leopold} (1831-10-29 – 1909-10-30), \emph{Journalist, Herausgeber}|pw}\end{otherlanguage}.}}}\hfill \textsc{Toelz\oindex{Bad Toelz@\textbf{Bad Tölz}|pw}}, 19. Auguſt.\pend
           \pstart
           \begin{otherlanguage}{french}\textcolor{gray}{\textbf{Journal politique, financier,}}\end{otherlanguage}\pend
           \pstart
           \begin{otherlanguage}{french}\textcolor{gray}{\textbf{commercial et littéraire.}}\end{otherlanguage}\pend
           \pstart
           \begin{otherlanguage}{french}\textcolor{gray}{\textbf{\textbf{Paraissant trois fois par jour.}}}\end{otherlanguage}\pend
           \pstart
           \begin{otherlanguage}{french}\textcolor{gray}{\textbf{\textbf{Bureau à Paris\oindex{Paris@\textbf{Paris}|pw}}}}\end{otherlanguage}\pend
           \pstart
           \begin{otherlanguage}{french}\textcolor{gray}{\textbf{\textbf{24. Rue Feydeau\oindex{rue Feydeau@\textbf{rue Feydeau}|pw}.}}}\end{otherlanguage}\pend
           \pstart\center{}Mein lieber Freund,\pend\pstart
           Alſo von Herzen Glück auf den Weg – auf den guten Weg, der Dich zu mir führen ſoll.
               Ich freue mich auf unſer Wiederſehn und ich fürchte mich zugleich davor \substVorne{}\textsuperscript{.}\substDazwischen{}–\substHinten{} aus Gründen, die Du gewiß verſtehſt, ohne daß ich ſie ſage{\dotsfive}\pend
           \pstart
           Ich wohne in \textsc{Krankenheil\oindex{Bad Krankenheil@\textbf{Bad Krankenheil}|pw}}, \textsc{Villa Carlo\oindex{Villa Carlo@\textbf{Villa Carlo}|pw}}. Aber Du telegraphirſt mir wohl am Tage vor Deiner Ankunft, {\pb}damit ich nur ja zu Hauſe bin.\pend
           \pstart
           Deine Fahrt wird ſchön ſein. Wenn ich nur wüßte, was man thun könnte, damit Du gutes
               Wetter haſt!\pend
           \pstart
           Wenn Du die Frau \textsc{Andreas\pwindex{Andreas-Salome, Lou 12.02.1861 – 05.02.1937@\textsc{Andreas-Salomé, Lou} (12.02.1861 – 05.02.1937), \emph{Schriftstellerin}|pw}} ſiehſt, ſo grüße ſie von mir recht herzlich. Ich möchte ſie gern einmal
               wiederſehen, wüßte ich nur wie und wo?\pend
           \pstart
           \label{K_L02745-1v}\edtext{\textsc{Mamroth\pwindex{Mamroth, Fedor 21.02.1851 – 25.06.1907@\textsc{Mamroth, Fedor} (21.02.1851 – 25.06.1907), \emph{Journalist, Kritiker}|pw}} iſt it noch bei der »Frankfurter Zeitung\orgindex{Frankfurter Zeitung@Frankfurter Zeitung|pw}}{\lemma{\textnormal{\emph{Mamroth … Zeitung}}}\Cendnote{\textnormal{siehe Hugo von Hofmannsthal an Arthur Schnitzler, 9. 8. [1895]}}}\label{K_L02745-1h}«, auch tritt er ſeinen großen Urlaub erſt nächſtens an\textcolor{gray}{.}{ }{\pb}Hingegen war er in der letzten Zeit mehrmals vom
                  Büreau\orgindex{Frankfurter Zeitung@Frankfurter Zeitung|pwv} abweſend, und ich
               müßte den präciſen Zeitpunkt wiſſen, um die Anfrage \strikeout{gena\textcolor{gray}{u}} genau beantworten zu können{\dotssix}\pend
           \pstart
           Ich bin heut ſo traurig u. hoffnungslos. Bin hier ganz
               allein u. habe Muße, über mich nachzudenken. Das iſt ſchrecklich. Ich ſchreibe Dir
               das nur, um Dich darauf vorzubereiten, daß Du mich nicht in jener guten Stimmung
               treffen wirſt, von der Dein lieber Brief ſpricht.\pend
           \pstart
           {\pb}Das ganze Jahr über habe \strikeout{i\textcolor{gray}{c}} ich mich auf das Wiederſehn mit Dir gefreut. Jetzt ſolls kaum mehr eine Woche
               dauern. Merkwürdig, wie die Begebenheiten langſam und geräuſchlos heranrücken! Es
               ſcheint Alles ſtill zu ſtehen, und nun auf einmal iſts nur noch eine Woche! {\dotsfive}\pend
           \pstart
           Grüß’ Dich Gott, mein lieber Freund!\pend
           \pstart
           Dein {\\[\baselineskip]}\spacefill\mbox{Paul Goldmann}\pend
           \leftskip=0em{}\pstart
           \noindent{}Grüße an Herrn \textsc{Salten\pwindex{Salten, Felix 06.09.1869 – 08.10.1945@\textsc{Salten, Felix} (06.09.1869 – 08.10.1945), \emph{Schriftsteller, Journalist}|pw}}!\pend
           
         
         \endnumbering\mylabel{h}\end{ledgroupsized}  \newcommand{\dateiname}{L02745}\newcommand{\titel}{Paul Goldmann an Arthur Schnitzler, 19. 8. [1895]}\newcommand{\editorInnen}{Martin Anton Müller und Laura Untner}%% latex-leseansicht-abspann.tex
%% Abspann für die Leseansicht.
%% Der Schalter \ifkorrekturansicht ist bereits durch den Vorspann gesetzt.

%% latex-abspann.tex
%% Gemeinsamer Abspann für Korrekturansicht und Leseansicht.
%% Setzt den Schalter \ifkorrekturansicht voraus (gesetzt in den
%% einbindenden Dateien latex-korrekturansicht-abspann.tex bzw.
%% latex-leseansicht-abspann.tex).
%% ---------------------------------------------------------------

\normalsize

% Das esempio-Environment wird nur in der Leseansicht benötigt
\ifkorrekturansicht\else
\newenvironment{esempio}[3]%
{
    \vspace{1.5ex}
    \rlap{\underline{#1}}
    \par
    \setlength{\parindent}{0cm}
    \nopagebreak
    \leftskip=#2cm
    \rightskip=#3cm
}
{
    \par
}
\fi

\doendnotes{C}
\bigskip
\vfill

\clearpage

\footnotesize

\ifkorrekturansicht
  \lohead{\textsc{register}}
\fi

% theindex-Environment neu definieren ohne reledmac
\makeatletter
\renewenvironment{theindex}{%
  \ifkorrekturansicht
    \section*{\indexname}%
  \else
    \subsubsection*{Index der erwähnten Entitäten}%
  \fi
  \setlength{\parindent}{0pt}%
  \setlength{\parskip}{0pt plus 0.3pt}%
  \let\item\@idxitem
}{%
  \ifkorrekturansicht\clearpage\fi
}
\makeatother

\IfFileExists{\jobname-pw.ind}{\input{\jobname-pw.ind}}{}

% Quellenangabe nur in der Leseansicht
\ifkorrekturansicht\else
% Fallback-Definitionen, falls die .tex-Datei \titel etc. nicht gesetzt hat
\providecommand{\titel}{}
\providecommand{\editorInnen}{}
\providecommand{\dateiname}{\jobname}

\vspace{3cm}

\vfill

\footnotesize
\textsc{Quelle}: \titel. Herausgegeben von {\editorInnen}. In: \emph{Arthur Schnitzler: Briefwechsel mit Autorinnen und Autoren}.
 Digitale Edition, https://schnitzler-briefe.acdh.oeaw.ac.at/{\dateiname}.html (Stand \today)
\fi

\end{document}


      