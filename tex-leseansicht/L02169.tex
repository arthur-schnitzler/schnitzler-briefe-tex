%% latex-leseansicht-vorspann.tex
%% Vorspann für die Leseansicht.
%% Lädt die gemeinsame Datei latex-vorspann.tex mit nicht gesetztem Schalter.

\newif\ifkorrekturansicht
\korrekturansichtfalse

\input{../tex-inputs/latex-vorspann}


\section[Arthur Schnitzler an Hermann Bahr, 30. 3. 1914]{L02169 Arthur Schnitzler an Hermann Bahr, 30. 3. 1914}
\nopagebreak\mylabel{L02169v}
\rehead{ }\normalsize\beginnumbering\briefempfaengerindex{Bahr, Hermann@\textsc{Bahr, Hermann}!zzzSchnitzler, Arthur@\emph{von Arthur Schnitzler}!1914-03-301@{30. 3. 1914}|(be}
\toendnotes[C]{\smallbreak\pagebreak[2]}
\correspDesc{Versand  durch Arthur Schnitzler am 30. 3. 1914 in Wien
\newline{}Erhalt  durch Hermann Bahr im Zeitraum [30. 3. 1914
                  – 3. 4. 1914?] \textbf{Ort fehlend} }\toendnotes[C]{\smallbreak}
\Standort{TMW, HS AM 60140 Ba.}
\physDesc{Briefkarte, 695 Zeichen
\newline{}Handschrift: schwarze Tinte, deutsche Kurrent
\newline{}Bahr: das Urteil über Anna
                                    Bahr-Mildenburg\pwindex{Bahr-Mildenburg, Anna 29.\,11.\,1872 Wien – 27.\,1.\,1947 ebd.@\textsc{Bahr-Mildenburg, Anna} (29.\,11.\,1872 Wien – 27.\,1.\,1947 ebd.), \emph{Sängerin}|pw} seitlich mit rotem Buntstift
                                 hervorgehoben }
\buchAbdrucke{\weitereDrucke{1) \emph{30. 3. 1914, Abschrift.} In: Arthur Schnitzler: \emph{The Letters of Arthur Schnitzler to Hermann Bahr}. Edited, annotated, and with an introduction, by Donald G. Daviau. Chapel Hill: \emph{The University of North Carolina Press} 1978, S. 113 (University of North Carolina studies in the Germanic languages
                        and literatures, 89).} \weitereDrucke{2) Hermann Bahr, Arthur Schnitzler: \emph{Briefwechsel, Aufzeichnungen, Dokumente (1891–1931)}. Herausgegeben von Kurt Ifkovits und Martin Anton Müller. Göttingen: \emph{Wallstein} 2018, S. 493.} }\toendnotes[C]{\smallbreak}
\pstart
           {\pb}\textcolor{gray}{\textbf{Dr. Arthur Schnitzler}}\hfill 30. 3. 914\pend
           
\pstart
           \textcolor{gray}{\textbf{Wien XVIII. Sternwartestrasse 71\oindex{Wien@\textbf{Wien}!XVIII., Währing@\textbf{XVIII., Währing}!Sternwartestraße 71@\textbf{Sternwartestraße 71}, \emph{Wohngebäude}|pw}}}\pend
           
\pstart{}mein lieber Hermann,\pend\vspace{0.5em}
\pstart
           deine Reiſe- u Aufenthaltspläne laſſen wenig Hoffnung übrig, daſs man einander
               wenigſtens im Laufe des So{\geminationm}ers begegnete – nachdem unſer
               Winterverſuch leider misglückt war. Wir wollen Anfang Mai nach Florenz\oindex{Florenz@\textbf{Florenz}|pw};{ }ſpäter (13.) von \textsc{Genua}\oindex{Genua@\textbf{Genua}|pw} aus zu Schiff nach Antwerpen\oindex{Antwerpen@\textbf{Antwerpen}, \emph{Region}|pw}, {\pb}über Holland\oindex{Niederlande@\textbf{Niederlande}|pw} zurück. Juni u Juli großentheils Wien\oindex{Wien@\textbf{Wien}, \emph{Verwaltungsgebiet}|pw}. Dann Gebirge. (Engadin\oindex{Engadin@\textbf{Engadin}, \emph{Tal}|pw}?) –\pend
           
\pstart
           Am \label{K_L02169-1v}\edtext{Freitag}{\lemma{\textnormal{\emph{Freitag}}}\Cendnote{\textnormal{27. 3. 1914}}}\label{K_L02169-1} haben wir\pwindex{Schnitzler, Olga 17.\,1.\,1882 Wien – 13.\,1.\,1970 Lugano@\textsc{Schnitzler, Olga} (17.\,1.\,1882 Wien – 13.\,1.\,1970 Lugano), \emph{Schauspielerin, Sängerin}|pwv}, nach
               ziemlich langer Zeit, deine Frau\pwindex{Bahr-Mildenburg, Anna 29.\,11.\,1872 Wien – 27.\,1.\,1947 ebd.@\textsc{Bahr-Mildenburg, Anna} (29.\,11.\,1872 Wien – 27.\,1.\,1947 ebd.), \emph{Sängerin}|pwv} wieder{ }ſingen gehört. \label{K_L02169-2v}\edtext{Gurrelieder\pwindex{Schönberg, Arnold 13.\,9.\,1874 Wien – 13.\,7.\,1951 Los Angeles@\textsc{Schönberg, Arnold} (13.\,9.\,1874 Wien – 13.\,7.\,1951 Los Angeles), \emph{Komponist}!Gurre-Lieder@\strich\emph{Gurre-Lieder}|pw}}{\lemma{\textnormal{\emph{Gurrelieder}}}\Cendnote{\textnormal{von Arnold Schönberg\pwindex{Schönberg, Arnold 13.\,9.\,1874 Wien – 13.\,7.\,1951 Los Angeles@\textsc{Schönberg, Arnold} (13.\,9.\,1874 Wien – 13.\,7.\,1951 Los Angeles), \emph{Komponist}|pwk}, am 27. 3. 1914 mit Anna
                     Bahr-Mildenburg\pwindex{Bahr-Mildenburg, Anna 29.\,11.\,1872 Wien – 27.\,1.\,1947 ebd.@\textsc{Bahr-Mildenburg, Anna} (29.\,11.\,1872 Wien – 27.\,1.\,1947 ebd.), \emph{Sängerin}|pwk}}}}\label{K_L02169-2}. Was ſie\pwindex{Bahr-Mildenburg, Anna 29.\,11.\,1872 Wien – 27.\,1.\,1947 ebd.@\textsc{Bahr-Mildenburg, Anna} (29.\,11.\,1872 Wien – 27.\,1.\,1947 ebd.), \emph{Sängerin}|pwv} geboten
               hat, gehört einfach zu dem \uline{größten}, was man \uline{je} im Conzertſaal \substVorne{}\textsuperscript{gehört}\substDazwischen{}erlebt\substHinten{} hat. Schade daſs du nicht dabei warſt.\pend
           \pstart Wir\pwindex{Schnitzler, Olga 17.\,1.\,1882 Wien – 13.\,1.\,1970 Lugano@\textsc{Schnitzler, Olga} (17.\,1.\,1882 Wien – 13.\,1.\,1970 Lugano), \emph{Schauspielerin, Sängerin}|pwv} grüßen dich herzlichſt!
               Und{ }ſage deiner Gattin\pwindex{Bahr-Mildenburg, Anna 29.\,11.\,1872 Wien – 27.\,1.\,1947 ebd.@\textsc{Bahr-Mildenburg, Anna} (29.\,11.\,1872 Wien – 27.\,1.\,1947 ebd.), \emph{Sängerin}|pwv} daſs
               wir{ }ſie bewundern. Auf Wiederſehen doch hoffentlich einmal! Dein
                  \spacefill\mbox{Arthur}\pend{}\selectlanguage{ngerman}\endnumbering\briefempfaengerindex{Bahr, Hermann@\textsc{Bahr, Hermann}!zzzSchnitzler, Arthur@\emph{von Arthur Schnitzler}!1914-03-301@{30. 3. 1914}|)be}\mylabel{L02169h}  \newcommand{\dateiname}{L02169}\newcommand{\titel}{Arthur Schnitzler an Hermann Bahr, 30. 3. 1914}\newcommand{\editorInnen}{Herausgegeben von Martin Anton Müller}%% latex-leseansicht-abspann.tex
%% Abspann für die Leseansicht.
%% Der Schalter \ifkorrekturansicht ist bereits durch den Vorspann gesetzt.

%% latex-abspann.tex
%% Gemeinsamer Abspann für Korrekturansicht und Leseansicht.
%% Setzt den Schalter \ifkorrekturansicht voraus (gesetzt in den
%% einbindenden Dateien latex-korrekturansicht-abspann.tex bzw.
%% latex-leseansicht-abspann.tex).
%% ---------------------------------------------------------------

\normalsize

% Das esempio-Environment wird nur in der Leseansicht benötigt
\ifkorrekturansicht\else
\newenvironment{esempio}[3]%
{
    \vspace{1.5ex}
    \rlap{\underline{#1}}
    \par
    \setlength{\parindent}{0cm}
    \nopagebreak
    \leftskip=#2cm
    \rightskip=#3cm
}
{
    \par
}
\fi

\doendnotes{C}
\bigskip
\vfill

\clearpage

\footnotesize

\ifkorrekturansicht
  \lohead{\textsc{register}}
\fi

% theindex-Environment neu definieren ohne reledmac
\makeatletter
\renewenvironment{theindex}{%
  \ifkorrekturansicht
    \section*{\indexname}%
  \else
    \subsubsection*{Index der erwähnten Entitäten}%
  \fi
  \setlength{\parindent}{0pt}%
  \setlength{\parskip}{0pt plus 0.3pt}%
  \let\item\@idxitem
}{%
  \ifkorrekturansicht\clearpage\fi
}
\makeatother

\IfFileExists{\jobname-pw.ind}{\input{\jobname-pw.ind}}{}

% Quellenangabe nur in der Leseansicht
\ifkorrekturansicht\else
% Fallback-Definitionen, falls die .tex-Datei \titel etc. nicht gesetzt hat
\providecommand{\titel}{}
\providecommand{\editorInnen}{}
\providecommand{\dateiname}{\jobname}

\vspace{3cm}

\vfill

\footnotesize
\textsc{Quelle}: \titel. Herausgegeben von {\editorInnen}. In: \emph{Arthur Schnitzler: Briefwechsel mit Autorinnen und Autoren}.
 Digitale Edition, https://schnitzler-briefe.acdh.oeaw.ac.at/{\dateiname}.html (Stand \today)
\fi

\end{document}


