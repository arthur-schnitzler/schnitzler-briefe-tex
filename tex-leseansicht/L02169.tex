%% latex-korrekturansicht-vorspann.tex
%% Vorspann für die Korrekturansicht.
%% Lädt die gemeinsame Datei latex-vorspann.tex mit gesetztem Schalter.

\newif\ifkorrekturansicht
\korrekturansichttrue

\input{../tex-inputs/latex-vorspann}


\section[Arthur Schnitzler an Hermann Bahr, 30. 3. 1914]{L02169 Arthur Schnitzler an Hermann Bahr, 30. 3. 1914}
\nopagebreak\mylabel{L02169v}
\rehead{ }\normalsize\beginnumbering\briefempfaengerindex{Bahr, Hermann@\textsc{Bahr, Hermann}!zzzSchnitzler, Arthur@\emph{von Arthur Schnitzler}!1914-03-301@{30. 3. 1914}|(be}
\toendnotes[C]{\smallbreak\pagebreak[2]}\Standort{TMW, HS AM 60140 Ba.}
\physDesc{Briefkarte, 695 Zeichen
\newline{}Handschrift: schwarze Tinte, deutsche Kurrent
\newline{}Bahr: das Urteil über Anna
                                    Bahr-Mildenburg\pwindex{Bahr-Mildenburg, Anna 29.11.1872 – 27.01.1947@\textsc{Bahr-Mildenburg, Anna} (29.11.1872 – 27.01.1947), \emph{Sänger/Sängerin}|pw} seitlich mit rotem Buntstift
                                 hervorgehoben }
\buchAbdrucke{\weitereDrucke{1) Arthur Schnitzler: \emph{The Letters of Arthur Schnitzler to Hermann Bahr}. Chapel Hill: \emph{The University of North Carolina Press} 1978, S. 113.} \weitereDrucke{2) Hermann Bahr, Arthur Schnitzler: \emph{Briefwechsel, Aufzeichnungen, Dokumente (1891–1931)}. Göttingen: \emph{Wallstein} 2018, S. 493.} }\toendnotes[C]{\smallbreak}
\pstart
           {\pb}\textcolor{gray}{\textbf{Dr. Arthur Schnitzler}}\hfill 30. 3. 914\pend
           
\pstart
           \textcolor{gray}{\textbf{Wien XVIII. Sternwartestrasse 71\oindex{Sternwartestrasse 71@\textbf{Sternwartestraße 71}, \emph{Wohngebäude (K.WHS)}|pw}}}\pend
           
\pstart{}mein lieber Hermann, \pend\vspace{0.5em}
\pstart
           deine Reiſe- u Aufenthaltspläne laſſen wenig Hoffnung übrig, daſs man einander
               wenigſtens im Laufe des So{\geminationm}ers begegnete – nachdem unſer
               Winterverſuch leider misglückt war. Wir wollen Anfang Mai nach Florenz\oindex{Florenz@\textbf{Florenz}, \emph{P.PPLA}|pw}; ſpäter (13.) von \textsc{Genua}\oindex{Genua@\textbf{Genua}, \emph{P.PPLA}|pw} aus zu Schiff nach Antwerpen\oindex{Antwerpen@\textbf{Antwerpen}, \emph{A.ADM4}|pw}, {\pb}über Holland\oindex{Niederlande@\textbf{Niederlande}, \emph{A.PCLI}|pw} zurück. Juni u Juli großentheils Wien\oindex{Wien@\textbf{Wien}, \emph{A.ADM2}|pw}. Dann Gebirge. (Engadin\oindex{Engadin@\textbf{Engadin}, \emph{T.VAL}|pw}?) – \pend
           
\pstart
           Am \label{K_L02169-1v}\edtext{Freitag}{\lemma{\textnormal{\emph{Freitag}}}\Cendnote{\textnormal{27. 3. 1914}}}\label{K_L02169-1} haben wir\pwindex{Schnitzler, Olga 17.01.1882 – 13.01.1970@\textsc{Schnitzler, Olga} (17.01.1882 – 13.01.1970), \emph{Schauspieler/Schauspielerin, Sänger/Sängerin}|pwv}, nach
               ziemlich langer Zeit, deine Frau\pwindex{Bahr-Mildenburg, Anna 29.11.1872 – 27.01.1947@\textsc{Bahr-Mildenburg, Anna} (29.11.1872 – 27.01.1947), \emph{Sänger/Sängerin}|pwv} wieder ſingen gehört. \label{K_L02169-2v}\edtext{Gurrelieder\pwindex{Gurre-Lieder@\emph{Gurre-Lieder}|pw}}{\lemma{\textnormal{\emph{Gurrelieder}}}\Cendnote{\textnormal{von Arnold Schönberg\pwindex{Schoenberg, Arnold 13.09.1874 – 13.07.1951@\textsc{Schönberg, Arnold} (13.09.1874 – 13.07.1951), \emph{Komponist/Komponistin}|pwk}, am 27. 3. 1914 mit Anna
                     Bahr-Mildenburg\pwindex{Bahr-Mildenburg, Anna 29.11.1872 – 27.01.1947@\textsc{Bahr-Mildenburg, Anna} (29.11.1872 – 27.01.1947), \emph{Sänger/Sängerin}|pwk}}}}\label{K_L02169-2}. Was ſie\pwindex{Bahr-Mildenburg, Anna 29.11.1872 – 27.01.1947@\textsc{Bahr-Mildenburg, Anna} (29.11.1872 – 27.01.1947), \emph{Sänger/Sängerin}|pwv} geboten
               hat, gehört einfach zu dem \uline{größten}, was man \uline{je} im Conzertſaal \substVorne{}\textsuperscript{gehört}\substDazwischen{}erlebt\substHinten{} hat. Schade daſs du nicht dabei warſt.\pend
           \pstart Wir\pwindex{Schnitzler, Olga 17.01.1882 – 13.01.1970@\textsc{Schnitzler, Olga} (17.01.1882 – 13.01.1970), \emph{Schauspieler/Schauspielerin, Sänger/Sängerin}|pwv} grüßen dich herzlichſt!
               Und ſage deiner Gattin\pwindex{Bahr-Mildenburg, Anna 29.11.1872 – 27.01.1947@\textsc{Bahr-Mildenburg, Anna} (29.11.1872 – 27.01.1947), \emph{Sänger/Sängerin}|pwv} daſs
               wir ſie bewundern. Auf Wiederſehen doch hoffentlich einmal! Dein
                  \spacefill\mbox{Arthur}\pend{}\selectlanguage{ngerman}\endnumbering\briefempfaengerindex{Bahr, Hermann@\textsc{Bahr, Hermann}!zzzSchnitzler, Arthur@\emph{von Arthur Schnitzler}!1914-03-301@{30. 3. 1914}|)be}\mylabel{L02169h}  \normalsize

\doendnotes{C}
\bigskip
\vfill

\clearpage

\footnotesize

\lohead{\textsc{register}}

% Definiere theindex-Environment komplett neu ohne reledmac
\makeatletter
\renewenvironment{theindex}{%
  \section*{\indexname}%
  \setlength{\parindent}{0pt}%
  \setlength{\parskip}{0pt plus 0.3pt}%
  \let\item\@idxitem
}{%
  \clearpage
}
\makeatother

\IfFileExists{\jobname-pw.ind}{\input{\jobname-pw.ind}}{}

\end{document}

      