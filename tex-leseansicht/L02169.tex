%% latex-leseansicht-vorspann.tex
%% Vorspann für die Leseansicht.
%% Lädt die gemeinsame Datei latex-vorspann.tex mit nicht gesetztem Schalter.

\newif\ifkorrekturansicht
\korrekturansichtfalse

\input{../tex-inputs/latex-vorspann}


         
         \renewcommand{\erwaehntePersonen}{Personen: Hermann Bahr, Anna Bahr-Mildenburg, Olga Schnitzler, Arnold Schönberg}
         \renewcommand{\erwaehnteOrte}{Orte: Antwerpen, Engadin, Florenz, Genua, Niederlande, Sternwartestraße, Wien}
         \renewcommand{\erwaehnteWerke}{Werke: Gurre-Lieder}
               \section[Arthur Schnitzler an Hermann Bahr, 30. 3. 1914]{ Arthur Schnitzler an Hermann Bahr, 30. 3. 1914}\nopagebreak\mylabel{v}\rehead{ }\begin{ledgroupsized}[t]{13cm}\normalsize\beginnumbering \toendnotes[C]{\smallbreak\pagebreak[2]} \Standort{TMW, HS AM 60140 Ba.}
\physDesc{Briefkarte
\newline{}Handschrift: schwarze Tinte, deutsche Kurrent
\newline{}Bahr: das Urteil über Anna
                                    Bahr-Mildenburg\pwindex{Bahr-Mildenburg, Anna 29.11.1872 – 27.01.1947@\textsc{Bahr-Mildenburg, Anna} (29.11.1872 – 27.01.1947), \emph{Sängerin}|pw} seitlich mit rotem Buntstift
                                 hervorgehoben }\buchAbdrucke{\weitereDrucke{1) \emph{30. 3. 1914, Abschrift.} In: Arthur Schnitzler: \emph{The Letters of Arthur Schnitzler to Hermann Bahr}. Edited, annotated, and with an introduction, by Donald G.
                        Daviau. Chapel Hill: \emph{The University of North Carolina Press} 1978, S. 113 (University of North Carolina studies in the Germanic languages
                        and literatures, 89).} \weitereDrucke{2) Hermann Bahr, Arthur Schnitzler: \emph{Briefwechsel, Aufzeichnungen, Dokumente (1891–1931)}. Hg. Kurt Ifkovits und Martin Anton Müller. Göttingen: \emph{Wallstein} 2018, S. 493.} }\toendnotes[C]{\smallbreak}\pstart
           \noindent{}{\pb}\textcolor{gray}{\textbf{Dr. Arthur Schnitzler}}\hfill 30. 3. 914\pend
           \pstart
           \textcolor{gray}{\textbf{Wien XVIII. Sternwartestrasse 71\oindex{Sternwartestrasse@\textbf{Sternwartestraße}|pw}}}\pend
           \pstart{}mein lieber Hermann, \pend\pstart
           deine Reiſe- u Aufenthaltspläne laſſen wenig Hoffnung übrig, daſs man einander
               wenigſtens im Laufe des So{\geminationm}ers begegnete – nachdem unſer
               Winterverſuch leider misglückt war. Wir wollen Anfang Mai nach Florenz\oindex{Florenz@\textbf{Florenz}|pw}; ſpäter (13.) von \textsc{Genua}\oindex{Genua@\textbf{Genua}|pw} aus zu Schiff nach Antwerpen\oindex{Antwerpen@\textbf{Antwerpen}|pw}, {\pb}über Holland\oindex{Niederlande@\textbf{Niederlande}|pw} zurück. Juni u Juli großentheils Wien\oindex{Wien@\textbf{Wien}|pw}. Dann Gebirge. (Engadin\oindex{Engadin@\textbf{Engadin}|pw}?)
               – \pend
           \pstart
           Am \label{K_L02169_1v}\edtext{Freitag}{\lemma{\textnormal{\emph{Freitag}}}\Cendnote{\textnormal{27. 3. 1914}}}\label{K_L02169_1h} haben wir\pwindex{Schnitzler, Olga 17.01.1882 – 13.01.1970@\textsc{Schnitzler, Olga} (17.01.1882 – 13.01.1970), \emph{Schauspielerin, Sängerin}|pwv}, nach ziemlich
               langer Zeit, deine Frau\pwindex{Bahr-Mildenburg, Anna 29.11.1872 – 27.01.1947@\textsc{Bahr-Mildenburg, Anna} (29.11.1872 – 27.01.1947), \emph{Sängerin}|pwv} wieder
               ſingen gehört. \label{K_L02169_2v}\edtext{Gurrelieder\pwindex{Schoenberg, Arnold 13.09.1874 – 14.07.1951@\textsc{Schönberg, Arnold} (13.09.1874 – 14.07.1951), \emph{Komponist}!Gurre-Lieder1913@\strich\emph{Gurre-Lieder} {[}1913{]}|pw}}{\lemma{\textnormal{\emph{Gurrelieder}}}\Cendnote{\textnormal{von Arnold Schönberg\pwindex{Schoenberg, Arnold 13.09.1874 – 14.07.1951@\textsc{Schönberg, Arnold} (13.09.1874 – 14.07.1951), \emph{Komponist}|pwk}, am 27. 3. 1914 mit Anna Bahr-Mildenburg\pwindex{Bahr-Mildenburg, Anna 29.11.1872 – 27.01.1947@\textsc{Bahr-Mildenburg, Anna} (29.11.1872 – 27.01.1947), \emph{Sängerin}|pwk}}}}\label{K_L02169_2h}. Was ſie\pwindex{Bahr-Mildenburg, Anna 29.11.1872 – 27.01.1947@\textsc{Bahr-Mildenburg, Anna} (29.11.1872 – 27.01.1947), \emph{Sängerin}|pwv} geboten hat, gehört einfach zu dem \uline{größten}, was man \uline{je} im
               Conzertſaal \substVorne{}\textsuperscript{gehört}{\allowbreak}\substDazwischen{}erlebt\substHinten{} hat. Schade daſs du nicht dabei warſt.\pend
           \pstart Wir\pwindex{Schnitzler, Olga 17.01.1882 – 13.01.1970@\textsc{Schnitzler, Olga} (17.01.1882 – 13.01.1970), \emph{Schauspielerin, Sängerin}|pwv} grüßen dich herzlichſt! Und
               ſage deiner Gattin\pwindex{Bahr-Mildenburg, Anna 29.11.1872 – 27.01.1947@\textsc{Bahr-Mildenburg, Anna} (29.11.1872 – 27.01.1947), \emph{Sängerin}|pwv} daſs wir ſie
               bewundern. Auf Wiederſehen doch hoffentlich einmal! Dein \spacefill\mbox{Arthur}\pend{}
         
         \endnumbering\mylabel{h}\end{ledgroupsized}  \newcommand{\dateiname}{L02169}\newcommand{\titel}{Arthur Schnitzler an Hermann Bahr, 30. 3. 1914}\newcommand{\editorInnen}{ Kurt Ifkovits,  Martin Anton Müller}%% latex-leseansicht-abspann.tex
%% Abspann für die Leseansicht.
%% Der Schalter \ifkorrekturansicht ist bereits durch den Vorspann gesetzt.

%% latex-abspann.tex
%% Gemeinsamer Abspann für Korrekturansicht und Leseansicht.
%% Setzt den Schalter \ifkorrekturansicht voraus (gesetzt in den
%% einbindenden Dateien latex-korrekturansicht-abspann.tex bzw.
%% latex-leseansicht-abspann.tex).
%% ---------------------------------------------------------------

\normalsize

% Das esempio-Environment wird nur in der Leseansicht benötigt
\ifkorrekturansicht\else
\newenvironment{esempio}[3]%
{
    \vspace{1.5ex}
    \rlap{\underline{#1}}
    \par
    \setlength{\parindent}{0cm}
    \nopagebreak
    \leftskip=#2cm
    \rightskip=#3cm
}
{
    \par
}
\fi

\doendnotes{C}
\bigskip
\vfill

\clearpage

\footnotesize

\ifkorrekturansicht
  \lohead{\textsc{register}}
\fi

% theindex-Environment neu definieren ohne reledmac
\makeatletter
\renewenvironment{theindex}{%
  \ifkorrekturansicht
    \section*{\indexname}%
  \else
    \subsubsection*{Index der erwähnten Entitäten}%
  \fi
  \setlength{\parindent}{0pt}%
  \setlength{\parskip}{0pt plus 0.3pt}%
  \let\item\@idxitem
}{%
  \ifkorrekturansicht\clearpage\fi
}
\makeatother

\IfFileExists{\jobname-pw.ind}{\input{\jobname-pw.ind}}{}

% Quellenangabe nur in der Leseansicht
\ifkorrekturansicht\else
% Fallback-Definitionen, falls die .tex-Datei \titel etc. nicht gesetzt hat
\providecommand{\titel}{}
\providecommand{\editorInnen}{}
\providecommand{\dateiname}{\jobname}

\vspace{3cm}

\vfill

\footnotesize
\textsc{Quelle}: \titel. Herausgegeben von {\editorInnen}. In: \emph{Arthur Schnitzler: Briefwechsel mit Autorinnen und Autoren}.
 Digitale Edition, https://schnitzler-briefe.acdh.oeaw.ac.at/{\dateiname}.html (Stand \today)
\fi

\end{document}


      