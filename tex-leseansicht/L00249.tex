%% latex-leseansicht-vorspann.tex
%% Vorspann für die Leseansicht.
%% Lädt die gemeinsame Datei latex-vorspann.tex mit nicht gesetztem Schalter.

\newif\ifkorrekturansicht
\korrekturansichtfalse

\input{../tex-inputs/latex-vorspann}


         
         \renewcommand{\erwaehntePersonen}{Personen: Richard Beer-Hofmann, Otto Erich Hartleben, Felix Salten}
         \renewcommand{\erwaehnteOrte}{Orte: Bad Ischl, Schulgasse, Wien}
         \renewcommand{\erwaehnteWerke}{Werke: Camelias, Das Kind, Der Tod Georgs, Die Erziehung zur Ehe, Die kleine Komödie, Novellen}
               \section[Arthur Schnitzler an Richard Beer-Hofmann, 3. 8. 1893]{ Arthur Schnitzler an Richard Beer-Hofmann, 3. 8. 1893}\nopagebreak\mylabel{v}\rehead{ }\begin{ledgroupsized}[t]{13cm}\normalsize\beginnumbering \toendnotes[C]{\smallbreak\pagebreak[2]} \Standort{CUL, Schnitzler, B 8.1, S. 16–17.}
\physDesc{Brief, Maschinenschriftliche Abschrift, 1 Blatt, 1 Seite, 1780 Zeichen
\newline{}Schreibmaschine
\newline{}Handschrift: Bleistift, deutsche Kurrent (\noindent{}eine Korrektur)
\newline{}Ordnung: von unbekannter Hand nummeriert mit: »30« }\buchAbdrucke{\weitereDrucke{Arthur Schnitzler, Richard Beer-Hofmann: \emph{Briefwechsel 1891–1931}. Hg. Konstanze Fliedl. Wien, Zürich: \emph{Europaverlag} 1992, S. 49–50.} }\toendnotes[C]{\smallbreak}\pstart
           \raggedleft{}{\pb}Wien\oindex{Wien@\textbf{Wien}|pw}, 3. 8. 93.\pend
           \pstart
           Lieber Richard, eben habe ich die Camelia\pwindex{Beer-Hofmann, Richard 1866-07-11 – 1945-09-26@\textsc{Beer-Hofmann, Richard} (1866-07-11 – 1945-09-26), \emph{Schriftsteller}!Camelias1893@\strich\emph{Camelias} {[}1893{]}|pw}’s wiedergelesen und kann Sie versichern, dass sie die gefährliche
               Probe des Wiedererlebens aufs glücklichste bestanden haben. Die Skizze ist eine
               Stiefschwester Ihres »Kind\pwindex{Beer-Hofmann, Richard 1866-07-11 – 1945-09-26@\textsc{Beer-Hofmann, Richard} (1866-07-11 – 1945-09-26), \emph{Schriftsteller}!Kind1893@\strich\emph{Das Kind} {[}1893{]}|pw}’s«; das Blut des
               Vaters pulsirt drin und dass Sie nun eine neue Muse haben, darf Sie gegen die
               frühere, mit der Sie die Camelias\pwindex{Beer-Hofmann, Richard 1866-07-11 – 1945-09-26@\textsc{Beer-Hofmann, Richard} (1866-07-11 – 1945-09-26), \emph{Schriftsteller}!Camelias1893@\strich\emph{Camelias} {[}1893{]}|pw} gezeugt haben,
               nicht ungerecht machen. Dagegen muss ich aber bemerken, dass mir die Miederstelle
               noch unangenehmer auffiel, als das erste Mal; sie ist absolut überflüssig und
               ausschliesslich widerlich. Mit demselben Recht dürften Sie darauf bestehen, den
               abendlichen Stuhlgang Ihres Helden zu schildern; ja beinahe mit mehr Recht; denn er
               ist natürlicher und berechtigter als das Mieder. Zur Charakteristik Freddys\pwindex{Beer-Hofmann, Richard 1866-07-11 – 1945-09-26@\textsc{Beer-Hofmann, Richard} (1866-07-11 – 1945-09-26), \emph{Schriftsteller}!Camelias1893@\strich\emph{Camelias} {[}1893{]}|pwv} gehört es auch absolut
               nicht. Sie sollten Freddy\pwindex{Beer-Hofmann, Richard 1866-07-11 – 1945-09-26@\textsc{Beer-Hofmann, Richard} (1866-07-11 – 1945-09-26), \emph{Schriftsteller}!Camelias1893@\strich\emph{Camelias} {[}1893{]}|pwv} auch
               etwas älter machen; denn es ist mir unangenehm, dass man sich mit 38 Jahren schon so
               fürchterlich {\pb}in der Decadence fühlen
               soll; – oder, was einfacher ist, gehen Sie bei dem Gefühl des Altseins von Freddy\pwindex{Beer-Hofmann, Richard 1866-07-11 – 1945-09-26@\textsc{Beer-Hofmann, Richard} (1866-07-11 – 1945-09-26), \emph{Schriftsteller}!Camelias1893@\strich\emph{Camelias} {[}1893{]}|pwv} mehr auf das
               psychologische \label{T_L00249_1v}\edtext{als}{\lemma{\textnormal{\emph{als}}}\Cendnote{\textnormal{korrigiert aus: »aus«}}}\label{T_L00249_1h}
               auf die ganz groben körperlichen Dinge. Kurzum, ich will mir nicht von Ihrer Novellette\pwindex{Beer-Hofmann, Richard 1866-07-11 – 1945-09-26@\textsc{Beer-Hofmann, Richard} (1866-07-11 – 1945-09-26), \emph{Schriftsteller}!Camelias1893@\strich\emph{Camelias} {[}1893{]}|pwv} die Möglichkeit
               nehmen lassen, in sieben Jahren ein junges Mädel zu heiraten! Verstehen Sie? – Aber
               das wesentliche: die Camelia\pwindex{Beer-Hofmann, Richard 1866-07-11 – 1945-09-26@\textsc{Beer-Hofmann, Richard} (1866-07-11 – 1945-09-26), \emph{Schriftsteller}!Camelias1893@\strich\emph{Camelias} {[}1893{]}|pw}’s gehören in Ihr
                  Buch\pwindex{Beer-Hofmann, Richard 1866-07-11 – 1945-09-26@\textsc{Beer-Hofmann, Richard} (1866-07-11 – 1945-09-26), \emph{Schriftsteller}!Novellen1. 12. 1893@\strich\emph{Novellen} {[}1. 12. 1893{]}|pwv}. –\pend
           \pstart
           – Haben Sie das Kind\pwindex{Beer-Hofmann, Richard 1866-07-11 – 1945-09-26@\textsc{Beer-Hofmann, Richard} (1866-07-11 – 1945-09-26), \emph{Schriftsteller}!Kind1893@\strich\emph{Das Kind} {[}1893{]}|pw} vorgelesen? – Schreiben Sie
               mir darüber! – Ich habe keine Einberufung. Werde vielleicht mit Salten\pwindex{Salten, Felix 06.09.1869 – 08.10.1945@\textsc{Salten, Felix} (06.09.1869 – 08.10.1945), \emph{Schriftsteller, Journalist}|pw} eine Bicycletour machen. –\pend
           \pstart
           Gibts was neues in Ischl\oindex{Bad Ischl@\textbf{Bad Ischl}|pw}? –\pend
           \pstart
           Las »Die Erziehung zur Ehe\pwindex{Hartleben, Otto Erich 03.06.1864 – 11.02.1905@\textsc{Hartleben, Otto Erich} (03.06.1864 – 11.02.1905), \emph{Schriftsteller}!Erziehung zur Ehe1893@\strich\emph{Die Erziehung zur Ehe} {[}1893{]}|pw}« von Hartleben\pwindex{Hartleben, Otto Erich 03.06.1864 – 11.02.1905@\textsc{Hartleben, Otto Erich} (03.06.1864 – 11.02.1905), \emph{Schriftsteller}|pw}; gefiel mir bis zum letzten Akt ganz
               ausnehmend. –\pend
           \pstart
           Meine Briefnovellette\pwindex{Schnitzler, Arthur 15.05.1862 – 21.10.1931@\textsc{Schnitzler, Arthur} (15.05.1862 – 21.10.1931), \emph{Schriftsteller, Mediziner}!kleine Komoedie1895-08-01@\strich\emph{Die kleine Komödie} {[}1895-08-01{]}|pwv} ist bis
               auf ein paar Zeilen fertig. Hoffentlich bring ich doch wieder einmal ein Stück
               zusammen. –\pend
           \pstart
           »Wieder einmal« – Grössenwahn? –\pend
           \pstart Herzlich Ihr \spacefill\mbox{Arthur.}\pend{}\pstart
           \noindent{}Grüssen Sie das nothwendige. Götterliebling\pwindex{Beer-Hofmann, Richard 1866-07-11 – 1945-09-26@\textsc{Beer-Hofmann, Richard} (1866-07-11 – 1945-09-26), \emph{Schriftsteller}!Tod Georgs1900@\strich\emph{Der Tod Georgs} {[}1900{]}|pw}? –\pend
           \pstart
           (nach Ischl, Schulg.\oindex{Schulgasse@\textbf{Schulgasse}|pw})\pend
           
         
         \endnumbering\mylabel{h}\end{ledgroupsized}  \newcommand{\dateiname}{L00249}\newcommand{\titel}{Arthur Schnitzler an Richard Beer-Hofmann, 3. 8. 1893}\newcommand{\editorInnen}{Martin Anton Müller und Gerd-Hermann Susen}%% latex-leseansicht-abspann.tex
%% Abspann für die Leseansicht.
%% Der Schalter \ifkorrekturansicht ist bereits durch den Vorspann gesetzt.

%% latex-abspann.tex
%% Gemeinsamer Abspann für Korrekturansicht und Leseansicht.
%% Setzt den Schalter \ifkorrekturansicht voraus (gesetzt in den
%% einbindenden Dateien latex-korrekturansicht-abspann.tex bzw.
%% latex-leseansicht-abspann.tex).
%% ---------------------------------------------------------------

\normalsize

% Das esempio-Environment wird nur in der Leseansicht benötigt
\ifkorrekturansicht\else
\newenvironment{esempio}[3]%
{
    \vspace{1.5ex}
    \rlap{\underline{#1}}
    \par
    \setlength{\parindent}{0cm}
    \nopagebreak
    \leftskip=#2cm
    \rightskip=#3cm
}
{
    \par
}
\fi

\doendnotes{C}
\bigskip
\vfill

\clearpage

\footnotesize

\ifkorrekturansicht
  \lohead{\textsc{register}}
\fi

% theindex-Environment neu definieren ohne reledmac
\makeatletter
\renewenvironment{theindex}{%
  \ifkorrekturansicht
    \section*{\indexname}%
  \else
    \subsubsection*{Index der erwähnten Entitäten}%
  \fi
  \setlength{\parindent}{0pt}%
  \setlength{\parskip}{0pt plus 0.3pt}%
  \let\item\@idxitem
}{%
  \ifkorrekturansicht\clearpage\fi
}
\makeatother

\IfFileExists{\jobname-pw.ind}{\input{\jobname-pw.ind}}{}

% Quellenangabe nur in der Leseansicht
\ifkorrekturansicht\else
% Fallback-Definitionen, falls die .tex-Datei \titel etc. nicht gesetzt hat
\providecommand{\titel}{}
\providecommand{\editorInnen}{}
\providecommand{\dateiname}{\jobname}

\vspace{3cm}

\vfill

\footnotesize
\textsc{Quelle}: \titel. Herausgegeben von {\editorInnen}. In: \emph{Arthur Schnitzler: Briefwechsel mit Autorinnen und Autoren}.
 Digitale Edition, https://schnitzler-briefe.acdh.oeaw.ac.at/{\dateiname}.html (Stand \today)
\fi

\end{document}


      