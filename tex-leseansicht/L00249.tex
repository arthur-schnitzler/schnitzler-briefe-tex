%% latex-korrekturansicht-vorspann.tex
%% Vorspann für die Korrekturansicht.
%% Lädt die gemeinsame Datei latex-vorspann.tex mit gesetztem Schalter.

\newif\ifkorrekturansicht
\korrekturansichttrue

\input{../tex-inputs/latex-vorspann}


\section[Arthur Schnitzler an Richard Beer-Hofmann, 3. 8. 1893]{L00249 Arthur Schnitzler an Richard Beer-Hofmann, 3. 8. 1893}
\nopagebreak\mylabel{L00249v}
\rehead{ }\normalsize\beginnumbering\briefempfaengerindex{Beer-Hofmann, Richard@\textsc{Beer-Hofmann, Richard}!zzzSchnitzler, Arthur@\emph{von Arthur Schnitzler}!1893-08-031@{3. 8. 1893}|(be}
\toendnotes[C]{\smallbreak\pagebreak[2]}\Standort{CUL, Schnitzler, B 8.1, S. 16–17.}
\physDesc{Brief, maschinenschriftliche Abschrift1 Blatt, 1 Seite, 1780 Zeichen
\newline{}Schreibmaschine
\newline{}Handschrift: Bleistift, deutsche Kurrent (\noindent{}eine Korrektur)
\newline{}Ordnung: von unbekannter Hand nummeriert mit: »30« }
\buchAbdrucke{\weitereDrucke{Arthur Schnitzler, Richard Beer-Hofmann: \emph{Briefwechsel 1891–1931}. Wien, Zürich: \emph{Europaverlag} 1992, S. 49–50.} }\toendnotes[C]{\smallbreak}
\pstart
           \raggedleft{}{\pb}Wien\oindex{Wien@\textbf{Wien}, \emph{A.ADM2}|pw}, 3. 8. 93.\pend
           \vspace{0.5em}
\pstart
           Lieber Richard, eben habe ich die Camelia\pwindex{Camelias@\emph{Camelias}|pw}’s wiedergelesen und kann Sie versichern, dass sie die gefährliche
               Probe des Wiedererlebens aufs glücklichste bestanden haben. Die Skizze ist eine
               Stiefschwester Ihres »Kind\pwindex{Kind@\emph{Das Kind}|pw}’s«; das Blut des
               Vaters pulsirt drin und dass Sie nun eine neue Muse haben, darf Sie gegen die
               frühere, mit der Sie die Camelias\pwindex{Camelias@\emph{Camelias}|pw} gezeugt haben,
               nicht ungerecht machen. Dagegen muss ich aber bemerken, dass mir die Miederstelle
               noch unangenehmer auffiel, als das erste Mal; sie ist absolut überflüssig und
               ausschliesslich widerlich. Mit demselben Recht dürften Sie darauf bestehen, den
               abendlichen Stuhlgang Ihres Helden zu schildern; ja beinahe mit mehr Recht; denn er
               ist natürlicher und berechtigter als das Mieder. Zur Charakteristik Freddys\pwindex{Camelias@\emph{Camelias}|pwv} gehört es auch absolut
               nicht. Sie sollten Freddy\pwindex{Camelias@\emph{Camelias}|pwv} auch
               etwas älter machen; denn es ist mir unangenehm, dass man sich mit 38 Jahren schon so
               fürchterlich {\pb}in der Decadence fühlen
               soll; – oder, was einfacher ist, gehen Sie bei dem Gefühl des Altseins von Freddy\pwindex{Camelias@\emph{Camelias}|pwv} mehr auf das
               psychologische \label{T_L00249-1v}\edtext{als}{\lemma{\textnormal{\emph{als}}}\Cendnote{\textnormal{korrigiert aus: »aus«}}}\label{T_L00249-1}
               auf die ganz groben körperlichen Dinge. Kurzum, ich will mir nicht von Ihrer Novellette\pwindex{Camelias@\emph{Camelias}|pwv} die Möglichkeit
               nehmen lassen, in sieben Jahren ein junges Mädel zu heiraten! Verstehen Sie? – Aber
               das wesentliche: die Camelia\pwindex{Camelias@\emph{Camelias}|pw}’s gehören in Ihr
                  Buch\pwindex{Novellen@\emph{Novellen}|pwv}. –\pend
           
\pstart
           – Haben Sie das Kind\pwindex{Kind@\emph{Das Kind}|pw} vorgelesen? – Schreiben Sie
               mir darüber! – Ich habe keine Einberufung. Werde vielleicht mit Salten\pwindex{Salten, Felix 06.09.1869 – 08.10.1945@\textsc{Salten, Felix} (06.09.1869 – 08.10.1945), \emph{Schriftsteller/Schriftstellerin, Journalist/Journalistin, Chefredakteur/Chefredakteurin}|pw} eine Bicycletour machen. –\pend
           
\pstart
           Gibts was neues in Ischl\oindex{Bad Ischl@\textbf{Bad Ischl}, \emph{P.PPL}|pw}? –\pend
           
\pstart
           Las »Die Erziehung zur Ehe\pwindex{Erziehung zur Ehe@\emph{Die Erziehung zur Ehe}|pw}« von Hartleben\pwindex{Hartleben, Otto Erich 03.06.1864 – 11.02.1905@\textsc{Hartleben, Otto Erich} (03.06.1864 – 11.02.1905), \emph{Schriftsteller/Schriftstellerin}|pw}; gefiel mir bis zum letzten Akt ganz
               ausnehmend. –\pend
           
\pstart
           Meine Briefnovellette\pwindex{kleine Komoedie@\emph{Die kleine Komödie}|pwv} ist bis
               auf ein paar Zeilen fertig. Hoffentlich bring ich doch wieder einmal ein Stück
               zusammen. –\pend
           
\pstart
           »Wieder einmal« – Grössenwahn? –\pend
           \pstart Herzlich Ihr \spacefill\mbox{Arthur.}\pend{}
\pstart
           \noindent{}Grüssen Sie das nothwendige. Götterliebling\pwindex{Tod Georgs@\emph{Der Tod Georgs}|pw}? –\pend
           
\pstart
           (nach Ischl, Schulg.\oindex{Schulgasse@\textbf{Schulgasse}, \emph{Straße (K.STR)}|pw})\pend
           \selectlanguage{ngerman}\endnumbering\briefempfaengerindex{Beer-Hofmann, Richard@\textsc{Beer-Hofmann, Richard}!zzzSchnitzler, Arthur@\emph{von Arthur Schnitzler}!1893-08-031@{3. 8. 1893}|)be}\mylabel{L00249h}  \normalsize

\doendnotes{C}
\bigskip
\vfill

\clearpage

\footnotesize

\lohead{\textsc{register}}

% Definiere theindex-Environment komplett neu ohne reledmac
\makeatletter
\renewenvironment{theindex}{%
  \section*{\indexname}%
  \setlength{\parindent}{0pt}%
  \setlength{\parskip}{0pt plus 0.3pt}%
  \let\item\@idxitem
}{%
  \clearpage
}
\makeatother

\IfFileExists{\jobname-pw.ind}{\input{\jobname-pw.ind}}{}

\end{document}

      