%% latex-leseansicht-vorspann.tex
%% Vorspann für die Leseansicht.
%% Lädt die gemeinsame Datei latex-vorspann.tex mit nicht gesetztem Schalter.

\newif\ifkorrekturansicht
\korrekturansichtfalse

\input{../tex-inputs/latex-vorspann}


\section[Arthur Schnitzler an Richard Beer-Hofmann, 3. 8. 1893]{L00249 Arthur Schnitzler an Richard Beer-Hofmann, 3. 8. 1893}
\nopagebreak\mylabel{L00249v}
\rehead{ }\normalsize\beginnumbering\briefempfaengerindex{Beer-Hofmann, Richard@\textsc{Beer-Hofmann, Richard}!zzzSchnitzler, Arthur@\emph{von Arthur Schnitzler}!1893-08-031@{3. 8. 1893}|(be}
\toendnotes[C]{\smallbreak\pagebreak[2]}
\correspDesc{Versand  durch Arthur Schnitzler am 3. 8. 1893 in Wien
\newline{}Erhalt  durch Richard Beer-Hofmann im Zeitraum [4. 8. 1893
                  – 8. 8. 1893?] in Bad Ischl}\toendnotes[C]{\smallbreak}
\Standort{CUL, Schnitzler, B 8.1, S. 16–17.}
\physDesc{Brief, maschinenschriftliche Abschrift, 1 Blatt, 1 Seite, 1780 Zeichen
\newline{}Schreibmaschine
\newline{}Handschrift: Bleistift, deutsche Kurrent (\noindent{}eine Korrektur)
\newline{}Ordnung: von unbekannter Hand nummeriert mit: »30« }
\buchAbdrucke{\weitereDrucke{Arthur Schnitzler, Richard Beer-Hofmann: \emph{Briefwechsel 1891–1931}. Herausgegeben von Konstanze Fliedl. Wien, Zürich: \emph{Europaverlag} 1992, S. 49–50.} }\toendnotes[C]{\smallbreak}
\pstart
           \raggedleft{}{\pb}Wien\oindex{Wien@\textbf{Wien}, \emph{Verwaltungsgebiet}|pw}, 3. 8. 93.\pend
           \vspace{0.5em}
\pstart
           Lieber Richard, eben habe ich die Camelia\pwindex{Beer-Hofmann, Richard 11.\,7.\,1866 Wien – 26.\,9.\,1945 New York City@\textsc{Beer-Hofmann, Richard} (11.\,7.\,1866 Wien – 26.\,9.\,1945 New York City), \emph{Schriftsteller}!Camelias@\strich\emph{Camelias}|pw}’s wiedergelesen und kann Sie versichern, dass sie die gefährliche
               Probe des Wiedererlebens aufs glücklichste bestanden haben. Die Skizze ist eine
               Stiefschwester Ihres »Kind\pwindex{Beer-Hofmann, Richard 11.\,7.\,1866 Wien – 26.\,9.\,1945 New York City@\textsc{Beer-Hofmann, Richard} (11.\,7.\,1866 Wien – 26.\,9.\,1945 New York City), \emph{Schriftsteller}!Kind@\strich\emph{Das Kind}|pw}’s«; das Blut des
               Vaters pulsirt drin und dass Sie nun eine neue Muse haben, darf Sie gegen die
               frühere, mit der Sie die Camelias\pwindex{Beer-Hofmann, Richard 11.\,7.\,1866 Wien – 26.\,9.\,1945 New York City@\textsc{Beer-Hofmann, Richard} (11.\,7.\,1866 Wien – 26.\,9.\,1945 New York City), \emph{Schriftsteller}!Camelias@\strich\emph{Camelias}|pw} gezeugt haben,
               nicht ungerecht machen. Dagegen muss ich aber bemerken, dass mir die Miederstelle
               noch unangenehmer auffiel, als das erste Mal; sie ist absolut überflüssig und
               ausschliesslich widerlich. Mit demselben Recht dürften Sie darauf bestehen, den
               abendlichen Stuhlgang Ihres Helden zu schildern; ja beinahe mit mehr Recht; denn er
               ist natürlicher und berechtigter als das Mieder. Zur Charakteristik Freddys\pwindex{Beer-Hofmann, Richard 11.\,7.\,1866 Wien – 26.\,9.\,1945 New York City@\textsc{Beer-Hofmann, Richard} (11.\,7.\,1866 Wien – 26.\,9.\,1945 New York City), \emph{Schriftsteller}!Camelias@\strich\emph{Camelias}|pwv} gehört es auch absolut
               nicht. Sie sollten Freddy\pwindex{Beer-Hofmann, Richard 11.\,7.\,1866 Wien – 26.\,9.\,1945 New York City@\textsc{Beer-Hofmann, Richard} (11.\,7.\,1866 Wien – 26.\,9.\,1945 New York City), \emph{Schriftsteller}!Camelias@\strich\emph{Camelias}|pwv} auch
               etwas älter machen; denn es ist mir unangenehm, dass man sich mit 38 Jahren schon so
               fürchterlich {\pb}in der Decadence fühlen
               soll; – oder, was einfacher ist, gehen Sie bei dem Gefühl des Altseins von Freddy\pwindex{Beer-Hofmann, Richard 11.\,7.\,1866 Wien – 26.\,9.\,1945 New York City@\textsc{Beer-Hofmann, Richard} (11.\,7.\,1866 Wien – 26.\,9.\,1945 New York City), \emph{Schriftsteller}!Camelias@\strich\emph{Camelias}|pwv} mehr auf das
               psychologische \label{T_L00249-1v}\edtext{als}{\lemma{\textnormal{\emph{als}}}\Cendnote{\textnormal{korrigiert aus: »aus«}}}\label{T_L00249-1}
               auf die ganz groben körperlichen Dinge. Kurzum, ich will mir nicht von Ihrer Novellette\pwindex{Beer-Hofmann, Richard 11.\,7.\,1866 Wien – 26.\,9.\,1945 New York City@\textsc{Beer-Hofmann, Richard} (11.\,7.\,1866 Wien – 26.\,9.\,1945 New York City), \emph{Schriftsteller}!Camelias@\strich\emph{Camelias}|pwv} die Möglichkeit
               nehmen lassen, in sieben Jahren ein junges Mädel zu heiraten! Verstehen Sie? – Aber
               das wesentliche: die Camelia\pwindex{Beer-Hofmann, Richard 11.\,7.\,1866 Wien – 26.\,9.\,1945 New York City@\textsc{Beer-Hofmann, Richard} (11.\,7.\,1866 Wien – 26.\,9.\,1945 New York City), \emph{Schriftsteller}!Camelias@\strich\emph{Camelias}|pw}’s gehören in Ihr
                  Buch\pwindex{Beer-Hofmann, Richard 11.\,7.\,1866 Wien – 26.\,9.\,1945 New York City@\textsc{Beer-Hofmann, Richard} (11.\,7.\,1866 Wien – 26.\,9.\,1945 New York City), \emph{Schriftsteller}!Novellen@\strich\emph{Novellen}|pwv}. –\pend
           
\pstart
           – Haben Sie das Kind\pwindex{Beer-Hofmann, Richard 11.\,7.\,1866 Wien – 26.\,9.\,1945 New York City@\textsc{Beer-Hofmann, Richard} (11.\,7.\,1866 Wien – 26.\,9.\,1945 New York City), \emph{Schriftsteller}!Kind@\strich\emph{Das Kind}|pw} vorgelesen? – Schreiben Sie
               mir darüber! – Ich habe keine Einberufung. Werde vielleicht mit Salten\pwindex{Salten, Felix 6.\,9.\,1869 Budapest – 8.\,10.\,1945 Zürich@\textsc{Salten, Felix} (6.\,9.\,1869 Budapest – 8.\,10.\,1945 Zürich), \emph{Schriftsteller, Journalist, Chefredakteur}|pw} eine Bicycletour machen. –\pend
           
\pstart
           Gibts was neues in Ischl\oindex{Bad Ischl@\textbf{Bad Ischl}|pw}? –\pend
           
\pstart
           Las »Die Erziehung zur Ehe\pwindex{Hartleben, Otto Erich 3.\,6.\,1864 Clausthal-Zellerfeld – 11.\,2.\,1905 Salò@\textsc{Hartleben, Otto Erich} (3.\,6.\,1864 Clausthal-Zellerfeld – 11.\,2.\,1905 Salò), \emph{Schriftsteller}!Erziehung zur Ehe@\strich\emph{Die Erziehung zur Ehe}|pw}« von Hartleben\pwindex{Hartleben, Otto Erich 3.\,6.\,1864 Clausthal-Zellerfeld – 11.\,2.\,1905 Salò@\textsc{Hartleben, Otto Erich} (3.\,6.\,1864 Clausthal-Zellerfeld – 11.\,2.\,1905 Salò), \emph{Schriftsteller}|pw}; gefiel mir bis zum letzten Akt ganz
               ausnehmend. –\pend
           
\pstart
           Meine Briefnovellette\pwindex{Schnitzler, Arthur 15.\,5.\,1862 Wien – 21.\,10.\,1931 ebd.@\textsc{Schnitzler, Arthur} (15.\,5.\,1862 Wien – 21.\,10.\,1931 ebd.), \emph{Schriftsteller, Mediziner}!kleine Komödie@\strich\emph{Die kleine Komödie}|pwv} ist bis
               auf ein paar Zeilen fertig. Hoffentlich bring ich doch wieder einmal ein Stück
               zusammen. –\pend
           
\pstart
           »Wieder einmal« – Grössenwahn? –\pend
           \pstart Herzlich Ihr \spacefill\mbox{Arthur.}\pend{}
\pstart
           \noindent{}Grüssen Sie das nothwendige. Götterliebling\pwindex{Beer-Hofmann, Richard 11.\,7.\,1866 Wien – 26.\,9.\,1945 New York City@\textsc{Beer-Hofmann, Richard} (11.\,7.\,1866 Wien – 26.\,9.\,1945 New York City), \emph{Schriftsteller}!Tod Georgs@\strich\emph{Der Tod Georgs}|pw}? –\pend
           
\pstart
           (nach Ischl, Schulg.\oindex{Schulgasse@\textbf{Schulgasse}, \emph{Straße}|pw})\pend
           \selectlanguage{ngerman}\endnumbering\briefempfaengerindex{Beer-Hofmann, Richard@\textsc{Beer-Hofmann, Richard}!zzzSchnitzler, Arthur@\emph{von Arthur Schnitzler}!1893-08-031@{3. 8. 1893}|)be}\mylabel{L00249h}  \newcommand{\dateiname}{L00249}\newcommand{\titel}{Arthur Schnitzler an Richard Beer-Hofmann, 3. 8. 1893}\newcommand{\editorInnen}{Martin Anton Müller und Gerd-Hermann Susen}%% latex-leseansicht-abspann.tex
%% Abspann für die Leseansicht.
%% Der Schalter \ifkorrekturansicht ist bereits durch den Vorspann gesetzt.

%% latex-abspann.tex
%% Gemeinsamer Abspann für Korrekturansicht und Leseansicht.
%% Setzt den Schalter \ifkorrekturansicht voraus (gesetzt in den
%% einbindenden Dateien latex-korrekturansicht-abspann.tex bzw.
%% latex-leseansicht-abspann.tex).
%% ---------------------------------------------------------------

\normalsize

% Das esempio-Environment wird nur in der Leseansicht benötigt
\ifkorrekturansicht\else
\newenvironment{esempio}[3]%
{
    \vspace{1.5ex}
    \rlap{\underline{#1}}
    \par
    \setlength{\parindent}{0cm}
    \nopagebreak
    \leftskip=#2cm
    \rightskip=#3cm
}
{
    \par
}
\fi

\doendnotes{C}
\bigskip
\vfill

\clearpage

\footnotesize

\ifkorrekturansicht
  \lohead{\textsc{register}}
\fi

% theindex-Environment neu definieren ohne reledmac
\makeatletter
\renewenvironment{theindex}{%
  \ifkorrekturansicht
    \section*{\indexname}%
  \else
    \subsubsection*{Index der erwähnten Entitäten}%
  \fi
  \setlength{\parindent}{0pt}%
  \setlength{\parskip}{0pt plus 0.3pt}%
  \let\item\@idxitem
}{%
  \ifkorrekturansicht\clearpage\fi
}
\makeatother

\IfFileExists{\jobname-pw.ind}{\input{\jobname-pw.ind}}{}

% Quellenangabe nur in der Leseansicht
\ifkorrekturansicht\else
% Fallback-Definitionen, falls die .tex-Datei \titel etc. nicht gesetzt hat
\providecommand{\titel}{}
\providecommand{\editorInnen}{}
\providecommand{\dateiname}{\jobname}

\vspace{3cm}

\vfill

\footnotesize
\textsc{Quelle}: \titel. Herausgegeben von {\editorInnen}. In: \emph{Arthur Schnitzler: Briefwechsel mit Autorinnen und Autoren}.
 Digitale Edition, https://schnitzler-briefe.acdh.oeaw.ac.at/{\dateiname}.html (Stand \today)
\fi

\end{document}


