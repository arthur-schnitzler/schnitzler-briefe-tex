%% latex-korrekturansicht-vorspann.tex
%% Vorspann für die Korrekturansicht.
%% Lädt die gemeinsame Datei latex-vorspann.tex mit gesetztem Schalter.

\newif\ifkorrekturansicht
\korrekturansichttrue

\input{../tex-inputs/latex-vorspann}


\section[Arthur Schnitzler: Widmungsexemplar Der einsame Weg für Hermann Bahr, {[}10.?{]} 2. 1904]{L01371 Arthur Schnitzler: Widmungsexemplar Der einsame Weg für Hermann Bahr,
               {[}10.?{]} 2. 1904}
\nopagebreak\mylabel{L01371v}
\rehead{ }\normalsize\beginnumbering\briefempfaengerindex{Bahr, Hermann@\textsc{Bahr, Hermann}!zzzSchnitzler, Arthur@\emph{von Arthur Schnitzler}!1904-02-101@{{[}10.?{]} 2. 1904}|(be}
\toendnotes[C]{\smallbreak\pagebreak[2]}\Standort{Salzburg, Universitätsbibliothek, 32321-I.}
\physDesc{Widmung am Vorsatzblatt, 47 Zeichen
\newline{}Handschrift: schwarze Tinte, deutsche Kurrent}
\buchAbdrucke{\weitereDrucke{Hermann Bahr, Arthur Schnitzler: \emph{Briefwechsel, Aufzeichnungen, Dokumente (1891–1931)}. Göttingen: \emph{Wallstein} 2018, S. 298.} }\toendnotes[C]{\smallbreak}
\pstart
           \noindent{}{\pb}Meinem lieben Herma{\geminationn}\pend
           \pstart \spacefill\mbox{ArthSchn}\pend{}
\pstart
           \noindent{}Berlin\oindex{Berlin@\textbf{Berlin}, \emph{P.PPLC}|pw}{ }\label{K_L01371-1v}\edtext{Feber}{\lemma{\textnormal{\emph{Feber}}}\Cendnote{\textnormal{Vgl. A. S.: \emph{Tagebuch}, 10. 2. 1904.
                     }}}\label{K_L01371-1} 904. \pend
           \selectlanguage{ngerman}\vspace{1em}{\vspace{1\baselineskip}}
\pstart
           \centering{}{\pb}\textcolor{gray}{\textbf{\textbf{Der einſame Weg\pwindex{einsame Weg. Schauspiel in fuenf Akten@\emph{Der einsame Weg. Schauspiel in fünf Akten}|pw}}}}\pend
           
\pstart
           \centering{}\textcolor{gray}{\textbf{Schauſpiel in fünf Acten}}\pend
           
\pstart
           \centering{}\textcolor{gray}{\textbf{von}}\pend
           
\pstart
           \centering{}\textcolor{gray}{\textbf{\textbf{Arthur Schnitzler}}}\pend
           {\vspace{1\baselineskip}}
\pstart
           \centering{}\textcolor{gray}{\textbf{Zweite Auflage}}\pend
           {\vspace{1\baselineskip}}
\pstart
           \centering{}\textcolor{gray}{\textbf{Berlin\oindex{Berlin@\textbf{Berlin}, \emph{P.PPLC}|pw}{ }1904}}\pend
           
\pstart
           \centering{}\textcolor{gray}{\textbf{S. Fiſcher, Verlag\orgindex{S. Fischer Verlag@S. Fischer Verlag|pw}}}\pend
           \selectlanguage{ngerman}\endnumbering\briefempfaengerindex{Bahr, Hermann@\textsc{Bahr, Hermann}!zzzSchnitzler, Arthur@\emph{von Arthur Schnitzler}!1904-02-101@{{[}10.?{]} 2. 1904}|)be}\mylabel{L01371h}  \normalsize

\doendnotes{C}
\bigskip
\vfill

\clearpage

\footnotesize

\lohead{\textsc{register}}

% Definiere theindex-Environment komplett neu ohne reledmac
\makeatletter
\renewenvironment{theindex}{%
  \section*{\indexname}%
  \setlength{\parindent}{0pt}%
  \setlength{\parskip}{0pt plus 0.3pt}%
  \let\item\@idxitem
}{%
  \clearpage
}
\makeatother

\IfFileExists{\jobname-pw.ind}{\input{\jobname-pw.ind}}{}

\end{document}

      