%% latex-korrekturansicht-vorspann.tex
%% Vorspann für die Korrekturansicht.
%% Lädt die gemeinsame Datei latex-vorspann.tex mit gesetztem Schalter.

\newif\ifkorrekturansicht
\korrekturansichttrue

\input{../tex-inputs/latex-vorspann}


\section[Arthur Schnitzler an Richard Beer-Hofmann, 6. 9. 1901]{L01167 Arthur Schnitzler an Richard Beer-Hofmann, 6. 9. 1901}
\nopagebreak\mylabel{L01167v}
\rehead{ }\normalsize\beginnumbering\briefempfaengerindex{Beer-Hofmann, Richard@\textsc{Beer-Hofmann, Richard}!zzzSchnitzler, Arthur@\emph{von Arthur Schnitzler}!1901-09-061@{6. 9. 1901}|(be}
\toendnotes[C]{\smallbreak\pagebreak[2]}\Standort{YCGL, MSS 31.}
\physDesc{Postkarte, 244 Zeichen
\newline{}Handschrift: schwarze Tinte, deutsche Kurrent
\newline{}Versand: 1) Stempel: »\nobreak{}\oindex{IX., Alsergrund@\textbf{IX., Alsergrund}, \emph{A.ADM3}|pwk}Wien 9/1 66, 6. 9. \textcolor{gray}{0}1\nobreak{}«.   2) Stempel: »\nobreak{}\oindex{Poertschach am Woerthersee@\textbf{Pörtschach am Wörthersee}, \emph{P.PPL}|pwk}\textcolor{gray}{Pörtsc}hach am See , 7 \textcolor{gray}{9} 0{[}0{]}1\nobreak{}«. 
\newline{}Ordnung: mit Bleistift von unbekannter Hand datiert: »6. 9.« }\toendnotes[C]{\smallbreak}\pstart{}{\pb}Herrn \textsc{Dr. Rich.
                     Beer-Hofmann}\pend{}\pstart{}\textsc{Pörtschach\oindex{Poertschach am Woerthersee@\textbf{Pörtschach am Wörthersee}, \emph{P.PPL}|pw}}\pend{}\pstart{}\textsc{Villa Arnstein\oindex{Villa Arnstein@\textbf{Villa Arnstein}, \emph{Wohngebäude (K.WHS)}|pw}.}\pend{}{\bigskip}\vspace{1em}
\pstart
           \noindent{}{\pb}lieber Richard, heute nur die kurze Frage, ob Sie in den \label{K_L01167-1v}\edtext{Club\orgindex{?? [Wiener Club September 1901]@?? [Wiener Club September 1901]|pwv}}{\lemma{\textnormal{\emph{Club}}}\Cendnote{\textnormal{Von
                     welcher Clubmitgliedschaft\orgindex{?? [Wiener Club September 1901]@?? [Wiener Club September 1901]|pwkv} hier und in den folgenden Korrespondenzstücken die Rede
                  ist, konnte nicht ermittelt werden. Eine Mitgliedschaft im \emph{Schachclub}\orgindex{Wiener Schachclub@Wiener Schachclub|pwk} bestand bereits 1899 (vgl. Felix Salten an Arthur Schnitzler, [18. 11. 1899]), sodass diese hier
                  nicht gemeint sein dürfte. Im \emph{Tagebuch}\pwindex{Tagebuch@\emph{Tagebuch}|pwk}
                  erwähnt Schnitzler in den kommenden Monaten
                  keinen Club. In Korrespondenzstücken erwähnte er am [14. 9. 1901?] und am 21. 9. 1901
                     einen Club\orgindex{?? [Wiener Club September 1901]@?? [Wiener Club September 1901]|pwkv}, aber explizit ohne notwendigerweise Mitglied werden
                     zu wollen.}}}\label{K_L01167-1} eintreten werden? hat wohl nicht mehr viel Sinn. Ich bin nicht
               ſehr dafür. Wie gehts? Ich habe dictirt.\pend
           
\pstart
           Herzlichſt Ihr{\\[\baselineskip]}\spacefill\mbox{Arthur}\pend
           \leftskip=0em{}
\pstart
           6. 9. 901.\pend
           \selectlanguage{ngerman}\endnumbering\briefempfaengerindex{Beer-Hofmann, Richard@\textsc{Beer-Hofmann, Richard}!zzzSchnitzler, Arthur@\emph{von Arthur Schnitzler}!1901-09-061@{6. 9. 1901}|)be}\mylabel{L01167h}  \normalsize

\doendnotes{C}
\bigskip
\vfill

\clearpage

\footnotesize

\lohead{\textsc{register}}

% Definiere theindex-Environment komplett neu ohne reledmac
\makeatletter
\renewenvironment{theindex}{%
  \section*{\indexname}%
  \setlength{\parindent}{0pt}%
  \setlength{\parskip}{0pt plus 0.3pt}%
  \let\item\@idxitem
}{%
  \clearpage
}
\makeatother

\IfFileExists{\jobname-pw.ind}{\input{\jobname-pw.ind}}{}

\end{document}

      