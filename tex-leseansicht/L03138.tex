%% latex-korrekturansicht-vorspann.tex
%% Vorspann für die Korrekturansicht.
%% Lädt die gemeinsame Datei latex-vorspann.tex mit gesetztem Schalter.

\newif\ifkorrekturansicht
\korrekturansichttrue

\input{../tex-inputs/latex-vorspann}


\section[Felix Salten an Arthur Schnitzler, {[}15.? 6. 1894{]}]{L03138 Felix Salten an Arthur Schnitzler, {[}15.? 6. 1894{]}}
\nopagebreak\mylabel{L03138v}
\rehead{ }\normalsize\beginnumbering\briefempfaengerindex{Schnitzler, Arthur@\textsc{Schnitzler, Arthur}!zzzSalten, Felix@\emph{von Felix Salten}!1894-06-151@{{[}15.? 6. 1894{]}}|(be}
\toendnotes[C]{\smallbreak\pagebreak[2]}\Standort{CUL, Schnitzler, B 89, A 1.}
\physDesc{Brief, 1 Blatt, 1 Seite, 484 Zeichen
\newline{}Handschrift: schwarze Tinte, lateinische Kurrent
\newline{}Schnitzler: mit Bleistift datiert: »Juni 94« 
\newline{}Ordnung: mit Bleistift von unbekannter Hand nummeriert: »39« }\toendnotes[C]{\smallbreak}
\pstart{}{\pb}Lieber Freund!\pend\vspace{0.5em}
\pstart
           a.) \label{K_L03138-1v}\edtext{werde ich sogleich thun}{\lemma{\textnormal{\emph{werde ich sogleich thun}}}\Cendnote{\textnormal{Bezug unklar}}}\label{K_L03138-1}, und mich bemühen, dass
               die Sache am Ende sich nicht jährt, ehe sie geordnet ist.\pend
           
\pstart
           b.) \label{K_L03138-2v}\edtext{soll in den nächsten Tagen
                  erfolgen}{\lemma{\textnormal{\emph{soll … erfolgen}}}\Cendnote{\textnormal{Bezug unklar}}}\label{K_L03138-2}, bin
               nicht Schuld, dass es noch nicht geschehen.\pend
           
\pstart
           c.) \label{K_L03138-3v}\edtext{Dörmann\pwindex{Doermann, Felix 29.05.1870 – 26.10.1928@\textsc{Dörmann, Felix} (29.05.1870 – 26.10.1928), \emph{Schriftsteller/Schriftstellerin}|pw} frägt an}{\lemma{\textnormal{\emph{Dörmann frägt an}}}\Cendnote{\textnormal{Felix Dörmann\pwindex{Doermann, Felix 29.05.1870 – 26.10.1928@\textsc{Dörmann, Felix} (29.05.1870 – 26.10.1928), \emph{Schriftsteller/Schriftstellerin}|pwk} arbeitete in dieser Zeit und
                  bis Ende Juni 1894 (von Wien\oindex{Wien@\textbf{Wien}, \emph{A.ADM2}|pwk} aus?) an der Zeitschrift \emph{Neue Deutsche Rundschau}\pwindex{Neue Deutsche Rundschau@\emph{Neue Deutsche Rundschau}|pwk} mit. Darin findet sich in der Zeit jedoch kein
                  Abdruck dieses\pwindex{Anfang vom Ende@\emph{Anfang vom Ende}|pwkv} oder anderer
                  Gedichte von Schnitzler. In einem Brief vom
                     20. 6. 1894 bat Dörmann\pwindex{Doermann, Felix 29.05.1870 – 26.10.1928@\textsc{Dörmann, Felix} (29.05.1870 – 26.10.1928), \emph{Schriftsteller/Schriftstellerin}|pwk}{ }Schnitzler, ihm »ein paar andere
                     ſchöne Verse [zu] ſchicken«, was darauf hindeutet, dass Schnitzler ihm – auf Saltens\pwindex{Salten, Felix 06.09.1869 – 08.10.1945@\textsc{Salten, Felix} (06.09.1869 – 08.10.1945), \emph{Schriftsteller/Schriftstellerin, Journalist/Journalistin, Chefredakteur/Chefredakteurin}|pwk} Aufforderung hin – etwas geschickt hatte. Da
                     Dörmanns\pwindex{Doermann, Felix 29.05.1870 – 26.10.1928@\textsc{Dörmann, Felix} (29.05.1870 – 26.10.1928), \emph{Schriftsteller/Schriftstellerin}|pwk} Engagement bei der Monatsschrift
                  aber kurz vor dem Ende stand, überrascht es nicht, dass aus der Sache nichts
                  wurde.}}}\label{K_L03138-3}, ob er Ihr Gedicht »Dass all das Schöne nun längst zu Ende\pwindex{Anfang vom Ende@\emph{Anfang vom Ende}|pwv}« bringen darf.
               Schreiben Sie ihm vielleicht eine Karte.\pend
           
\pstart
           c.) Sind Sie \label{K_L03138-4v}\edtext{morgen bei »Therese
                  Krones\pwindex{Therese Krones. Genrebild mit Gesang und Tanz in drei Akten@\emph{Therese Krones. Genrebild mit Gesang und Tanz in drei Akten}|pw}?«}{\lemma{\textnormal{\emph{morgen … Krones?«}}}\Cendnote{\textnormal{Diese Stelle erlaubt die genauere
                  Datierung, da die Premiere von \emph{Therese Krones}\pwindex{Therese Krones. Genrebild mit Gesang und Tanz in drei Akten@\emph{Therese Krones. Genrebild mit Gesang und Tanz in drei Akten}|pwk}
                  am 16. 6. 1894 am
                     \emph{Deutschen Volkstheater}\orgindex{Volkstheater@Volkstheater|pwk} stattfand. Sowohl Schnitzler als auch Salten\pwindex{Salten, Felix 06.09.1869 – 08.10.1945@\textsc{Salten, Felix} (06.09.1869 – 08.10.1945), \emph{Schriftsteller/Schriftstellerin, Journalist/Journalistin, Chefredakteur/Chefredakteurin}|pwk} nahmen teil. Danach waren sie gemeinsam mit Adele Sandrock\pwindex{Sandrock, Adele 1863-08-19 – 1937-08-30@\textsc{Sandrock, Adele} (1863-08-19 – 1937-08-30), \emph{Schauspieler/Schauspielerin}|pwk} und deren Mutter Johanna Simonetta Sandrock\pwindex{Sandrock, Johanna Simonetta 27.6.1833 – 6.4.1917@\textsc{Sandrock, Johanna Simonetta} (27.6.1833 – 6.4.1917), \emph{Schauspieler/Schauspielerin}|pwk} im Riedhof\oindex{Riedhof@\textbf{Riedhof}, \emph{Lokal (K.LKL)}|pwk}.}}}\label{K_L03138-4} Ich bin auf alle Fälle da, und
                  \introOben{}wir\introOben{} soupiren dann zusammen? Wenn nicht Arkaden Café\oindex{Cafe Arkaden@\textbf{Café Arkaden}, \emph{Kaffeehaus (K.KAF)}|pw}!\pend
           
\pstart
           Herzlichst Ihr {\\[\baselineskip]}\spacefill\mbox{Salten}\pend
           \leftskip=0em{}\selectlanguage{ngerman}\endnumbering\briefempfaengerindex{Schnitzler, Arthur@\textsc{Schnitzler, Arthur}!zzzSalten, Felix@\emph{von Felix Salten}!1894-06-151@{{[}15.? 6. 1894{]}}|)be}\mylabel{L03138h}  \normalsize

\doendnotes{C}
\bigskip
\vfill

\clearpage

\footnotesize

\lohead{\textsc{register}}

% Definiere theindex-Environment komplett neu ohne reledmac
\makeatletter
\renewenvironment{theindex}{%
  \section*{\indexname}%
  \setlength{\parindent}{0pt}%
  \setlength{\parskip}{0pt plus 0.3pt}%
  \let\item\@idxitem
}{%
  \clearpage
}
\makeatother

\IfFileExists{\jobname-pw.ind}{\input{\jobname-pw.ind}}{}

\end{document}

      