%% latex-leseansicht-vorspann.tex
%% Vorspann für die Leseansicht.
%% Lädt die gemeinsame Datei latex-vorspann.tex mit nicht gesetztem Schalter.

\newif\ifkorrekturansicht
\korrekturansichtfalse

\input{../tex-inputs/latex-vorspann}


         
         \renewcommand{\erwaehntePersonen}{Personen: Felix Dörmann, Felix Salten, Adele Sandrock, Johanna Simonetta Sandrock}
         \renewcommand{\erwaehnteOrte}{Orte: Café Arkaden, Riedhof, Volkstheater, Wien}
         \renewcommand{\erwaehnteWerke}{Werke: Anfang vom Ende, Neue Deutsche Rundschau, Therese Krones. Genrebild mit Gesang und Tanz in drei Akten}
               \section[Felix Salten an Arthur Schnitzler, {[}15.? 6. 1894{]}]{ Felix Salten an Arthur Schnitzler, {[}15.? 6. 1894{]}}\nopagebreak\mylabel{v}\rehead{ }\begin{ledgroupsized}[t]{13cm}\normalsize\beginnumbering \toendnotes[C]{\smallbreak\pagebreak[2]} \Standort{CUL, Schnitzler, B 89, A 1.}
\physDesc{Brief, 1 Blatt, 1 Seite, 484 Zeichen
\newline{}Handschrift: schwarze Tinte, lateinische Kurrent
\newline{}Schnitzler: mit Bleistift datiert: »Juni 94« 
\newline{}Ordnung: mit Bleistift von unbekannter Hand nummeriert: »39« }\toendnotes[C]{\smallbreak}\pstart{}{\pb}Lieber Freund!\pend\pstart
           a.) \label{K_L03138-1v}\edtext{werde ich sogleich thun}{\lemma{\textnormal{\emph{werde ich sogleich thun}}}\Cendnote{\textnormal{Bezug unklar}}}\label{K_L03138-1h}, und mich bemühen, dass
               die Sache am Ende sich nicht jährt, ehe sie geordnet ist.\pend
           \pstart
           b.) \label{K_L03138-2v}\edtext{soll in den nächsten Tagen
                  erfolgen}{\lemma{\textnormal{\emph{soll … erfolgen}}}\Cendnote{\textnormal{Bezug unklar}}}\label{K_L03138-2h}, bin
               nicht Schuld, dass es noch nicht geschehen.\pend
           \pstart
           c.) \label{K_L03138-3v}\edtext{Dörmann\pwindex{Doermann, Felix 29.05.1870 – 26.10.1928@\textsc{Dörmann, Felix} (29.05.1870 – 26.10.1928), \emph{Schriftsteller}|pw} frägt an}{\lemma{\textnormal{\emph{Dörmann frägt an}}}\Cendnote{\textnormal{Felix Dörmann\pwindex{Doermann, Felix 29.05.1870 – 26.10.1928@\textsc{Dörmann, Felix} (29.05.1870 – 26.10.1928), \emph{Schriftsteller}|pwk} arbeitete in dieser Zeit und
                  bis Ende Juni 1894 (von Wien\oindex{Wien@\textbf{Wien}|pwk} aus?) an der Zeitschrift \emph{Neue Deutsche Rundschau}\pwindex{Neue Deutsche Rundschau1894-01-01 – 1903-12-31@\emph{Neue Deutsche Rundschau} {[}1894-01-01 – 1903-12-31{]}|pwk} mit. Darin findet sich in der Zeit jedoch kein
                  Abdruck dieses\pwindex{Schnitzler, Arthur 15.05.1862 – 21.10.1931@\textsc{Schnitzler, Arthur} (15.05.1862 – 21.10.1931), \emph{Schriftsteller, Mediziner}!Anfang vom Ende1892-03-03@\strich\emph{Anfang vom Ende} {[}1892-03-03{]}|pwkv} oder anderer
                  Gedichte von Schnitzler\pwindex{Schnitzler, Arthur 15.05.1862 – 21.10.1931@\textsc{Schnitzler, Arthur} (15.05.1862 – 21.10.1931), \emph{Schriftsteller, Mediziner}|pwk}. In einem Brief vom
                     20. 6. 1894 bat Dörmann\pwindex{Doermann, Felix 29.05.1870 – 26.10.1928@\textsc{Dörmann, Felix} (29.05.1870 – 26.10.1928), \emph{Schriftsteller}|pwk}{ }Schnitzler\pwindex{Schnitzler, Arthur 15.05.1862 – 21.10.1931@\textsc{Schnitzler, Arthur} (15.05.1862 – 21.10.1931), \emph{Schriftsteller, Mediziner}|pwk}, ihm »ein paar andere
                     ſchöne Verse [zu] ſchicken«, was darauf hindeutet, dass Schnitzler\pwindex{Schnitzler, Arthur 15.05.1862 – 21.10.1931@\textsc{Schnitzler, Arthur} (15.05.1862 – 21.10.1931), \emph{Schriftsteller, Mediziner}|pwk} ihm – auf Salten\pwindex{Salten, Felix 06.09.1869 – 08.10.1945@\textsc{Salten, Felix} (06.09.1869 – 08.10.1945), \emph{Schriftsteller, Journalist}|pwk}s Aufforderung hin – etwas geschickt hatte. Nachdem
                     Dörmann\pwindex{Doermann, Felix 29.05.1870 – 26.10.1928@\textsc{Dörmann, Felix} (29.05.1870 – 26.10.1928), \emph{Schriftsteller}|pwk}s Engagement bei der Monatsschrift
                  aber kurz vor dem Ende stand, überrascht es nicht, dass aus der Sache nichts
                  wurde.}}}\label{K_L03138-3h}, ob er Ihr Gedicht »Dass all das Schöne nun längst zu Ende\pwindex{Schnitzler, Arthur 15.05.1862 – 21.10.1931@\textsc{Schnitzler, Arthur} (15.05.1862 – 21.10.1931), \emph{Schriftsteller, Mediziner}!Anfang vom Ende1892-03-03@\strich\emph{Anfang vom Ende} {[}1892-03-03{]}|pwv}« bringen darf.
               Schreiben Sie ihm vielleicht eine Karte.\pend
           \pstart
           c.) Sind Sie \label{K_L03138-4v}\edtext{morgen bei »Therese
                  Krones\pwindex{\textcolor{red}{\textsuperscript{XXXX1 indx}}!Therese Krones. Genrebild mit Gesang und Tanz in drei Akten1862@\strich\emph{Therese Krones. Genrebild mit Gesang und Tanz in drei Akten} {[}1862{]}|pw}?«}{\lemma{\textnormal{\emph{morgen … Krones?«}}}\Cendnote{\textnormal{Das erlaubt die genauere
                  Datierung, da die Premiere von \emph{Therese Krones}\pwindex{\textcolor{red}{\textsuperscript{XXXX1 indx}}!Therese Krones. Genrebild mit Gesang und Tanz in drei Akten1862@\strich\emph{Therese Krones. Genrebild mit Gesang und Tanz in drei Akten} {[}1862{]}|pwk}
                  am 16. 6. 1894 am
                     Deutschen Volkstheater\oindex{Volkstheater@\textbf{Volkstheater}|pwk} stattfand. Sowohl Schnitzler\pwindex{Schnitzler, Arthur 15.05.1862 – 21.10.1931@\textsc{Schnitzler, Arthur} (15.05.1862 – 21.10.1931), \emph{Schriftsteller, Mediziner}|pwk} als auch Salten\pwindex{Salten, Felix 06.09.1869 – 08.10.1945@\textsc{Salten, Felix} (06.09.1869 – 08.10.1945), \emph{Schriftsteller, Journalist}|pwk} nahmen teil. Danach waren sie gemeinsam mit Adele Sandrock\pwindex{Sandrock, Adele 1863-08-19 – 1937-08-30@\textsc{Sandrock, Adele} (1863-08-19 – 1937-08-30), \emph{Schauspielerin}|pwk} und deren Mutter Johanna Simonetta Sandrock\pwindex{Sandrock, Johanna Simonetta 27.6.1833 – 6.4.1917@\textsc{Sandrock, Johanna Simonetta} (27.6.1833 – 6.4.1917), \emph{Schauspielerin}|pwk} im Riedhof\oindex{Riedhof@\textbf{Riedhof}|pwk}.}}}\label{K_L03138-4h} Ich bin auf alle Fälle da, und
                  \introOben{}wir\introOben{} soupiren dann zusammen? Wenn nicht Arkaden Café\oindex{Cafe Arkaden@\textbf{Café Arkaden}|pw}!\pend
           \pstart
           Herzlichst Ihr {\\[\baselineskip]}\spacefill\mbox{Salten}\pend
           \leftskip=0em{}
         
         \endnumbering\mylabel{h}\end{ledgroupsized}  \newcommand{\dateiname}{L03138}\newcommand{\titel}{Felix Salten an Arthur Schnitzler, [15.? 6. 1894]}\newcommand{\editorInnen}{Martin Anton Müller und Laura Untner}%% latex-leseansicht-abspann.tex
%% Abspann für die Leseansicht.
%% Der Schalter \ifkorrekturansicht ist bereits durch den Vorspann gesetzt.

%% latex-abspann.tex
%% Gemeinsamer Abspann für Korrekturansicht und Leseansicht.
%% Setzt den Schalter \ifkorrekturansicht voraus (gesetzt in den
%% einbindenden Dateien latex-korrekturansicht-abspann.tex bzw.
%% latex-leseansicht-abspann.tex).
%% ---------------------------------------------------------------

\normalsize

% Das esempio-Environment wird nur in der Leseansicht benötigt
\ifkorrekturansicht\else
\newenvironment{esempio}[3]%
{
    \vspace{1.5ex}
    \rlap{\underline{#1}}
    \par
    \setlength{\parindent}{0cm}
    \nopagebreak
    \leftskip=#2cm
    \rightskip=#3cm
}
{
    \par
}
\fi

\doendnotes{C}
\bigskip
\vfill

\clearpage

\footnotesize

\ifkorrekturansicht
  \lohead{\textsc{register}}
\fi

% theindex-Environment neu definieren ohne reledmac
\makeatletter
\renewenvironment{theindex}{%
  \ifkorrekturansicht
    \section*{\indexname}%
  \else
    \subsubsection*{Index der erwähnten Entitäten}%
  \fi
  \setlength{\parindent}{0pt}%
  \setlength{\parskip}{0pt plus 0.3pt}%
  \let\item\@idxitem
}{%
  \ifkorrekturansicht\clearpage\fi
}
\makeatother

\IfFileExists{\jobname-pw.ind}{\input{\jobname-pw.ind}}{}

% Quellenangabe nur in der Leseansicht
\ifkorrekturansicht\else
% Fallback-Definitionen, falls die .tex-Datei \titel etc. nicht gesetzt hat
\providecommand{\titel}{}
\providecommand{\editorInnen}{}
\providecommand{\dateiname}{\jobname}

\vspace{3cm}

\vfill

\footnotesize
\textsc{Quelle}: \titel. Herausgegeben von {\editorInnen}. In: \emph{Arthur Schnitzler: Briefwechsel mit Autorinnen und Autoren}.
 Digitale Edition, https://schnitzler-briefe.acdh.oeaw.ac.at/{\dateiname}.html (Stand \today)
\fi

\end{document}


      