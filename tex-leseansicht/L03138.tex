%% latex-leseansicht-vorspann.tex
%% Vorspann für die Leseansicht.
%% Lädt die gemeinsame Datei latex-vorspann.tex mit nicht gesetztem Schalter.

\newif\ifkorrekturansicht
\korrekturansichtfalse

\input{../tex-inputs/latex-vorspann}

\begin{center}
            \textcolor{red}{ENTWURF, NICHT FERTIG KORRIGIERT}
                      \end{center}
            
         
         \renewcommand{\erwaehntePersonen}{Personen: Felix Dörmann}
         \renewcommand{\erwaehnteOrte}{Orte: Café Arkaden, Volkstheater, Wien}
         \renewcommand{\erwaehnteWerke}{Werke: Anfang vom Ende, Therese Krones. Genrebild mit Gesang und Tanz in drei Akten}
               \section[Felix Salten an Arthur Schnitzler, {[}15.? 6. 1894{]}]{ Felix Salten an Arthur Schnitzler, {[}15.? 6. 1894{]}}\nopagebreak\mylabel{v}\rehead{ }\begin{ledgroupsized}[t]{13cm}\normalsize\beginnumbering \toendnotes[C]{\smallbreak\pagebreak[2]} \Standort{CUL, Schnitzler, B 89, A 1.}
\physDesc{Brief, 1 Blatt, 1 Seite
\newline{}Handschrift: schwarze Tinte, lateinische Kurrent
\newline{}Schnitzler: mit Bleistift datiert: »Juni
                                 94« \newline{}Ordnung: mit Bleistift von unbekannter Hand nummeriert: »39« }\toendnotes[C]{\smallbreak}\pstart
           \noindent{}{\pb}Lieber Freund! a) werde ich sogleich thun, und mich bemühen,
               dass die Sache am Ende sich nicht jährt, ehe sie geordnet ist. \pend
           \pstart
           b) soll in den nächsten Tagen erfolgen, bin nicht Schuld, dass es noch nicht
               geschehen. \pend
           \pstart
           c) \label{K_L03138-11v}\edtext{Dörmann\pwindex{Doermann, Felix 29.05.1870 – 26.10.1928@\textsc{Dörmann, Felix} (29.05.1870 – 26.10.1928), \emph{Schriftsteller}|pw} frägt an}{\lemma{\textnormal{\emph{Dörmann frägt an}}}\Cendnote{\textnormal{XXXX}}}\label{K_L03138-11h}, ob er Ihr Gedicht »dass all das Schöne nun längst zu Ende\pwindex{Schnitzler, Arthur 15.05.1862 – 21.10.1931@\textsc{Schnitzler, Arthur} (15.05.1862 – 21.10.1931), \emph{Schriftsteller, Mediziner}!Anfang vom Ende1892-03-03@\strich\emph{Anfang vom Ende} {[}1892-03-03{]}|pwv}« bringen darf. Schreiben Sie ihm
               vielleicht eine Karte. \pend
           \pstart
           d) Sind Sie \label{K_L03138-1v}\edtext{morgen bei »Therese Krones\pwindex{\textcolor{red}{\textsuperscript{XXXX1 indx}}!Therese Krones. Genrebild mit Gesang und Tanz in drei Akten1862@\strich\emph{Therese Krones. Genrebild mit Gesang und Tanz in drei Akten} {[}1862{]}|pw}?«}{\lemma{\textnormal{\emph{morgen … Krones?«}}}\Cendnote{\textnormal{Das erlaubt die genauere Datierung, da die Premiere von \emph{Therese Krones}\pwindex{\textcolor{red}{\textsuperscript{XXXX1 indx}}!Therese Krones. Genrebild mit Gesang und Tanz in drei Akten1862@\strich\emph{Therese Krones. Genrebild mit Gesang und Tanz in drei Akten} {[}1862{]}|pwk} am 16. 6. 1894 am Deutschen Volkstheater\oindex{Volkstheater@\textbf{Volkstheater}|pwk} stattfand. Sowohl Schnitzler\pwindex{Schnitzler, Arthur 15.05.1862 – 21.10.1931@\textsc{Schnitzler, Arthur} (15.05.1862 – 21.10.1931), \emph{Schriftsteller, Mediziner}|pwk} wie Salten\pwindex{Salten, Felix 06.09.1869 – 08.10.1945@\textsc{Salten, Felix} (06.09.1869 – 08.10.1945), \emph{Schriftsteller, Journalist}|pwk} nahmen teil.}}}\label{K_L03138-1h} Ich bin auf alle Fälle da, und
                  \introOben{}wir\introOben{} soupiren dann zusammen? Wenn nicht Arkaden Café\oindex{Cafe Arkaden@\textbf{Café Arkaden}|pw}! \pend
           \pstart
           Herzlichst Ihr {\\[\baselineskip]}\spacefill\mbox{Salten}\pend
           \leftskip=0em{}
         
         \endnumbering\mylabel{h}\end{ledgroupsized}\begin{anhang}\end{anhang}\newcommand{\dateiname}{L03138}\newcommand{\titel}{Felix Salten an Arthur Schnitzler, [15.? 6. 1894]}\newcommand{\editorInnen}{Martin Anton Müller und Laura Untner}%% latex-leseansicht-abspann.tex
%% Abspann für die Leseansicht.
%% Der Schalter \ifkorrekturansicht ist bereits durch den Vorspann gesetzt.

%% latex-abspann.tex
%% Gemeinsamer Abspann für Korrekturansicht und Leseansicht.
%% Setzt den Schalter \ifkorrekturansicht voraus (gesetzt in den
%% einbindenden Dateien latex-korrekturansicht-abspann.tex bzw.
%% latex-leseansicht-abspann.tex).
%% ---------------------------------------------------------------

\normalsize

% Das esempio-Environment wird nur in der Leseansicht benötigt
\ifkorrekturansicht\else
\newenvironment{esempio}[3]%
{
    \vspace{1.5ex}
    \rlap{\underline{#1}}
    \par
    \setlength{\parindent}{0cm}
    \nopagebreak
    \leftskip=#2cm
    \rightskip=#3cm
}
{
    \par
}
\fi

\doendnotes{C}
\bigskip
\vfill

\clearpage

\footnotesize

\ifkorrekturansicht
  \lohead{\textsc{register}}
\fi

% theindex-Environment neu definieren ohne reledmac
\makeatletter
\renewenvironment{theindex}{%
  \ifkorrekturansicht
    \section*{\indexname}%
  \else
    \subsubsection*{Index der erwähnten Entitäten}%
  \fi
  \setlength{\parindent}{0pt}%
  \setlength{\parskip}{0pt plus 0.3pt}%
  \let\item\@idxitem
}{%
  \ifkorrekturansicht\clearpage\fi
}
\makeatother

\IfFileExists{\jobname-pw.ind}{\input{\jobname-pw.ind}}{}

% Quellenangabe nur in der Leseansicht
\ifkorrekturansicht\else
% Fallback-Definitionen, falls die .tex-Datei \titel etc. nicht gesetzt hat
\providecommand{\titel}{}
\providecommand{\editorInnen}{}
\providecommand{\dateiname}{\jobname}

\vspace{3cm}

\vfill

\footnotesize
\textsc{Quelle}: \titel. Herausgegeben von {\editorInnen}. In: \emph{Arthur Schnitzler: Briefwechsel mit Autorinnen und Autoren}.
 Digitale Edition, https://schnitzler-briefe.acdh.oeaw.ac.at/{\dateiname}.html (Stand \today)
\fi

\end{document}


      