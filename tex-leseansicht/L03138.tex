%% latex-leseansicht-vorspann.tex
%% Vorspann für die Leseansicht.
%% Lädt die gemeinsame Datei latex-vorspann.tex mit nicht gesetztem Schalter.

\newif\ifkorrekturansicht
\korrekturansichtfalse

\input{../tex-inputs/latex-vorspann}


\section[Felix Salten an Arthur Schnitzler, {{[}}15.? 6. 1894{{]}}]{L03138 Felix Salten an Arthur Schnitzler, {[}15.? 6. 1894{]}}
\nopagebreak\mylabel{L03138v}
\rehead{ }\normalsize\beginnumbering\briefempfaengerindex{Schnitzler, Arthur@\textsc{Schnitzler, Arthur}!zzzSalten, Felix@\emph{von Felix Salten}!1894-06-151@{{[}15.? 6. 1894{]}}|(be}
\toendnotes[C]{\smallbreak\pagebreak[2]}
\correspDesc{Versand  durch Felix Salten am [15.? 6. 1894] in Wien
\newline{}Erhalt  durch Arthur Schnitzler am [15.? 6. 1894] in Wien}\toendnotes[C]{\smallbreak}
\Standort{CUL, Schnitzler, B 89, A 1.}
\physDesc{Brief, 1 Blatt, 1 Seite, 484 Zeichen
\newline{}Handschrift: schwarze Tinte, lateinische Kurrent
\newline{}Schnitzler: mit Bleistift datiert: »Juni 94« 
\newline{}Ordnung: mit Bleistift von unbekannter Hand nummeriert: »39« }\toendnotes[C]{\smallbreak}
\pstart{}{\pb}Lieber Freund!\pend\vspace{0.5em}
\pstart
           a.) \label{K_L03138-1v}\edtext{werde ich sogleich thun}{\lemma{\textnormal{\emph{werde ich sogleich thun}}}\Cendnote{\textnormal{Bezug unklar}}}\label{K_L03138-1}, und mich bemühen, dass
               die Sache am Ende sich nicht jährt, ehe sie geordnet ist.\pend
           
\pstart
           b.) \label{K_L03138-2v}\edtext{soll in den nächsten Tagen
                  erfolgen}{\lemma{\textnormal{\emph{soll … erfolgen}}}\Cendnote{\textnormal{Bezug unklar}}}\label{K_L03138-2}, bin
               nicht Schuld, dass es noch nicht geschehen.\pend
           
\pstart
           c.) \label{K_L03138-3v}\edtext{Dörmann\pwindex{Dörmann, Felix 29.\,5.\,1870 Wien – 26.\,10.\,1928 ebd.@\textsc{Dörmann, Felix} (29.\,5.\,1870 Wien – 26.\,10.\,1928 ebd.), \emph{Schriftsteller}|pw} frägt an}{\lemma{\textnormal{\emph{Dörmann frägt an}}}\Cendnote{\textnormal{Felix Dörmann\pwindex{Dörmann, Felix 29.\,5.\,1870 Wien – 26.\,10.\,1928 ebd.@\textsc{Dörmann, Felix} (29.\,5.\,1870 Wien – 26.\,10.\,1928 ebd.), \emph{Schriftsteller}|pwk} arbeitete in dieser Zeit und
                  bis Ende Juni 1894 (von Wien\oindex{Wien@\textbf{Wien}, \emph{Verwaltungsgebiet}|pwk} aus?) an der Zeitschrift \emph{Neue Deutsche Rundschau}\pwindex{Neue Deutsche Rundschau@\emph{Neue Deutsche Rundschau}|pwk} mit. Darin findet sich in der Zeit jedoch kein
                  Abdruck dieses\pwindex{Schnitzler, Arthur 15.\,5.\,1862 Wien – 21.\,10.\,1931 ebd.@\textsc{Schnitzler, Arthur} (15.\,5.\,1862 Wien – 21.\,10.\,1931 ebd.), \emph{Schriftsteller, Mediziner}!Anfang vom Ende@\strich\emph{Anfang vom Ende}|pwkv} oder anderer
                  Gedichte von Schnitzler. In einem Brief vom
                     20. 6. 1894 bat Dörmann\pwindex{Dörmann, Felix 29.\,5.\,1870 Wien – 26.\,10.\,1928 ebd.@\textsc{Dörmann, Felix} (29.\,5.\,1870 Wien – 26.\,10.\,1928 ebd.), \emph{Schriftsteller}|pwk}{ }Schnitzler, ihm »ein paar andere{ }ſchöne Verse [zu]{ }ſchicken«, was darauf hindeutet, dass Schnitzler ihm – auf Saltens\pwindex{Salten, Felix 6.\,9.\,1869 Budapest – 8.\,10.\,1945 Zürich@\textsc{Salten, Felix} (6.\,9.\,1869 Budapest – 8.\,10.\,1945 Zürich), \emph{Schriftsteller, Journalist, Chefredakteur}|pwk} Aufforderung hin – etwas geschickt hatte. Da
                     Dörmanns\pwindex{Dörmann, Felix 29.\,5.\,1870 Wien – 26.\,10.\,1928 ebd.@\textsc{Dörmann, Felix} (29.\,5.\,1870 Wien – 26.\,10.\,1928 ebd.), \emph{Schriftsteller}|pwk} Engagement bei der Monatsschrift
                  aber kurz vor dem Ende stand, überrascht es nicht, dass aus der Sache nichts
                  wurde.}}}\label{K_L03138-3}, ob er Ihr Gedicht »Dass all das Schöne nun längst zu Ende\pwindex{Schnitzler, Arthur 15.\,5.\,1862 Wien – 21.\,10.\,1931 ebd.@\textsc{Schnitzler, Arthur} (15.\,5.\,1862 Wien – 21.\,10.\,1931 ebd.), \emph{Schriftsteller, Mediziner}!Anfang vom Ende@\strich\emph{Anfang vom Ende}|pwv}« bringen darf.
               Schreiben Sie ihm vielleicht eine Karte.\pend
           
\pstart
           c.) Sind Sie \label{K_L03138-4v}\edtext{morgen bei »Therese
                  Krones\pwindex{\textcolor{red}{\textsuperscript{XXXX indx1}}!Therese Krones. Genrebild mit Gesang und Tanz in drei Akten@\strich\emph{Therese Krones. Genrebild mit Gesang und Tanz in drei Akten}|pw}?«}{\lemma{\textnormal{\emph{morgen … Krones?«}}}\Cendnote{\textnormal{Diese Stelle erlaubt die genauere
                  Datierung, da die Premiere von \emph{Therese Krones}\pwindex{\textcolor{red}{\textsuperscript{XXXX indx1}}!Therese Krones. Genrebild mit Gesang und Tanz in drei Akten@\strich\emph{Therese Krones. Genrebild mit Gesang und Tanz in drei Akten}|pwk}
                  am 16. 6. 1894 am
                     \emph{Deutschen Volkstheater}\orgindex{Volkstheater@Volkstheater|pwk} stattfand. Sowohl Schnitzler als auch Salten\pwindex{Salten, Felix 6.\,9.\,1869 Budapest – 8.\,10.\,1945 Zürich@\textsc{Salten, Felix} (6.\,9.\,1869 Budapest – 8.\,10.\,1945 Zürich), \emph{Schriftsteller, Journalist, Chefredakteur}|pwk} nahmen teil. Danach waren sie gemeinsam mit Adele Sandrock\pwindex{Sandrock, Adele 19.\,8.\,1863 Rotterdam – 30.\,8.\,1937 Berlin@\textsc{Sandrock, Adele} (19.\,8.\,1863 Rotterdam – 30.\,8.\,1937 Berlin), \emph{Schauspielerin}|pwk} und deren Mutter Johanna Simonetta Sandrock\pwindex{Sandrock, Johanna Simonetta 27.\,6.\,1833 Amsterdam – 6.\,4.\,1917 Berlin@\textsc{Sandrock, Johanna Simonetta} (27.\,6.\,1833 Amsterdam – 6.\,4.\,1917 Berlin), \emph{Schauspielerin}|pwk} im Riedhof\oindex{Wien@\textbf{Wien}!VIII., Josefstadt@\textbf{VIII., Josefstadt}!Riedhof@\textbf{Riedhof}, \emph{Lokal}|pwk}.}}}\label{K_L03138-4} Ich bin auf alle Fälle da, und
                  \introOben{}wir\introOben{} soupiren dann zusammen? Wenn nicht Arkaden Café\oindex{Wien@\textbf{Wien}!I., Innere Stadt@\textbf{I., Innere Stadt}!Café Arkaden@\textbf{Café Arkaden}, \emph{Kaffeehaus}|pw}!\pend
           
\pstart
           Herzlichst Ihr {\\[\baselineskip]}\spacefill\mbox{Salten}\pend
           \leftskip=0em{}\selectlanguage{ngerman}\endnumbering\briefempfaengerindex{Schnitzler, Arthur@\textsc{Schnitzler, Arthur}!zzzSalten, Felix@\emph{von Felix Salten}!1894-06-151@{{[}15.? 6. 1894{]}}|)be}\mylabel{L03138h}  \newcommand{\dateiname}{L03138}\newcommand{\titel}{Felix Salten an Arthur Schnitzler, [15.? 6. 1894]}\newcommand{\editorInnen}{Martin Anton Müller und Laura Untner}%% latex-leseansicht-abspann.tex
%% Abspann für die Leseansicht.
%% Der Schalter \ifkorrekturansicht ist bereits durch den Vorspann gesetzt.

%% latex-abspann.tex
%% Gemeinsamer Abspann für Korrekturansicht und Leseansicht.
%% Setzt den Schalter \ifkorrekturansicht voraus (gesetzt in den
%% einbindenden Dateien latex-korrekturansicht-abspann.tex bzw.
%% latex-leseansicht-abspann.tex).
%% ---------------------------------------------------------------

\normalsize

% Das esempio-Environment wird nur in der Leseansicht benötigt
\ifkorrekturansicht\else
\newenvironment{esempio}[3]%
{
    \vspace{1.5ex}
    \rlap{\underline{#1}}
    \par
    \setlength{\parindent}{0cm}
    \nopagebreak
    \leftskip=#2cm
    \rightskip=#3cm
}
{
    \par
}
\fi

\doendnotes{C}
\bigskip
\vfill

\clearpage

\footnotesize

\ifkorrekturansicht
  \lohead{\textsc{register}}
\fi

% theindex-Environment neu definieren ohne reledmac
\makeatletter
\renewenvironment{theindex}{%
  \ifkorrekturansicht
    \section*{\indexname}%
  \else
    \subsubsection*{Index der erwähnten Entitäten}%
  \fi
  \setlength{\parindent}{0pt}%
  \setlength{\parskip}{0pt plus 0.3pt}%
  \let\item\@idxitem
}{%
  \ifkorrekturansicht\clearpage\fi
}
\makeatother

\IfFileExists{\jobname-pw.ind}{\input{\jobname-pw.ind}}{}

% Quellenangabe nur in der Leseansicht
\ifkorrekturansicht\else
% Fallback-Definitionen, falls die .tex-Datei \titel etc. nicht gesetzt hat
\providecommand{\titel}{}
\providecommand{\editorInnen}{}
\providecommand{\dateiname}{\jobname}

\vspace{3cm}

\vfill

\footnotesize
\textsc{Quelle}: \titel. Herausgegeben von {\editorInnen}. In: \emph{Arthur Schnitzler: Briefwechsel mit Autorinnen und Autoren}.
 Digitale Edition, https://schnitzler-briefe.acdh.oeaw.ac.at/{\dateiname}.html (Stand \today)
\fi

\end{document}


