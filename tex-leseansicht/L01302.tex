%% latex-korrekturansicht-vorspann.tex
%% Vorspann für die Korrekturansicht.
%% Lädt die gemeinsame Datei latex-vorspann.tex mit gesetztem Schalter.

\newif\ifkorrekturansicht
\korrekturansichttrue

\input{../tex-inputs/latex-vorspann}


\section[Arthur Schnitzler an Hermann Bahr, 1{[}3{]}. 7. 1903]{L01302 Arthur Schnitzler an Hermann Bahr, 1{[}3{]}. 7. 1903}
\nopagebreak\mylabel{L01302v}
\rehead{ }\normalsize\beginnumbering\briefempfaengerindex{Bahr, Hermann@\textsc{Bahr, Hermann}!zzzSchnitzler, Arthur@\emph{von Arthur Schnitzler}!1903-07-131@{1{[}3{]}. 7. 1903}|(be}
\toendnotes[C]{\smallbreak\pagebreak[2]}\Standort{TMW, HS AM 60165 Ba.}
\physDesc{Briefkarte, 426 Zeichen
\newline{}Handschrift: schwarze Tinte, deutsche Kurrent
\newline{}Ordnung: Lochung }
\buchAbdrucke{\weitereDrucke{1) Arthur Schnitzler: \emph{The Letters of Arthur Schnitzler to Hermann Bahr}. Chapel Hill: \emph{The University of North Carolina Press} 1978, S. 79.} \weitereDrucke{2) Hermann Bahr, Arthur Schnitzler: \emph{Briefwechsel, Aufzeichnungen, Dokumente (1891–1931)}. Göttingen: \emph{Wallstein} 2018, S. 267.} }\toendnotes[C]{\smallbreak}
\pstart
           {\pb}1\damage{3}. 7. 903.\pend
           \vspace{0.5em}
\pstart
           lieber Hermann,{ }Salten\pwindex{Salten, Felix 06.09.1869 – 08.10.1945@\textsc{Salten, Felix} (06.09.1869 – 08.10.1945), \emph{Schriftsteller/Schriftstellerin, Journalist/Journalistin, Chefredakteur/Chefredakteurin}|pw}{ }\damage{ü}bermittelt mir deine freundliche Frage, ob ich was dagegen hätte, we{\geminationn} du den Reigen\pwindex{Reigen. Zehn Dialoge@\emph{Reigen. Zehn Dialoge}|pw}
               öffentlich vorzuleſen versuchteſt. Im Gegentheil, es wird {\pb}mir \damage{ſe}hr angenehm ſein. Nu\damage{r} werde ich zum erſten Mal bedauern – daſs ich der Verfaſſer bin – weil ich
               nemlich nicht als Zuhörer meiner eigenen Sachen unter dem Publikum ſitzen kann! Auf
                  Wiederſehen\hspace*{1.5em}dein getreuer{\\}\spacefill\mbox{A. S.}\pend
           
\pstart
           \noindent{}\label{T_L01302-1v}\edtext{Prächtig war dein Dialog\pwindex{Dialog vom Tragischen@\emph{Dialog vom Tragischen}|pw} in der \textsc{N. D. R}\pwindex{neue Rundschau@\emph{Die neue Rundschau}|pw}! –}{\lemma{\textnormal{\emph{Prächtig … N. D. R! –}}}\Cendnote{\textnormal{auf der ersten Seite, am
                     unteren Seitenrand, verkehrt zum Text}}}\label{T_L01302-1}\pend
           \selectlanguage{ngerman}\endnumbering\briefempfaengerindex{Bahr, Hermann@\textsc{Bahr, Hermann}!zzzSchnitzler, Arthur@\emph{von Arthur Schnitzler}!1903-07-131@{1{[}3{]}. 7. 1903}|)be}\mylabel{L01302h}  \normalsize

\doendnotes{C}
\bigskip
\vfill

\clearpage

\footnotesize

\lohead{\textsc{register}}

% Definiere theindex-Environment komplett neu ohne reledmac
\makeatletter
\renewenvironment{theindex}{%
  \section*{\indexname}%
  \setlength{\parindent}{0pt}%
  \setlength{\parskip}{0pt plus 0.3pt}%
  \let\item\@idxitem
}{%
  \clearpage
}
\makeatother

\IfFileExists{\jobname-pw.ind}{\input{\jobname-pw.ind}}{}

\end{document}

      