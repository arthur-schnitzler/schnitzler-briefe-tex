\input{../tex-inputs/latex-pdf-vorspann}
\begin{center}
            \textcolor{red}{ENTWURF. ENTZIFFERUNG NOCH NICHT KORREKTURGELESEN}
                      \end{center}
            
               \section[Paul Goldmann an Arthur Schnitzler, {[}20. 1. 1895?{]}]{ Paul Goldmann an Arthur Schnitzler, {[}20. 1. 1895?{]}}\nopagebreak\mylabel{v}\rehead{ }\begin{ledgroupsized}[t]{13cm}\normalsize\beginnumbering\briefempfaengerindex{Schnitzler, Arthur@\textsc{Schnitzler, Arthur}!zzzGoldmann, Paul@\emph{von Paul Goldmann}!1895-01-201@{{[}20. 1. 1895?{]}}|(be} \toendnotes[C]{\smallbreak\pagebreak[2]} \Standort{DLA, A:Schnitzler, HS.NZ85.1.3165.}
\physDesc{Telegramm
\newline{}maschinell
\newline{}Schnitzler: mit Bleistift datiert: »Januar 9\strikeout{4}« \newline{}Ordnung: 1) von unbekannter Hand mit Tinte einen Strich bei der ergänzten
                                 Jahreszahl hinzugefügt, eventuell in der Absicht, aus der
                                 gestrichenen »4« eine »5« zu machen 2) beschnitten}\toendnotes[C]{\smallbreak}\pstart
           \centering{}{\pb}w\oindex{Wien@\textbf{Wien}|pw} fr paris\oindex{Paris@\textbf{Paris}|pw}
                  30298 20{ }1/38=\pend
           \pstart
           hab meine \label{K_L02692-1v}\edtext{innige freude}{\lemma{\textnormal{\emph{innige freude}}}\Cendnote{\textnormal{Das Datum lässt sich durch das
                  Zusammenlesen dreier Indizien ermitteln: Die Datierung Schnitzler\pwindex{Schnitzler, Arthur 15.05.1862 – 21.10.1931@\textsc{Schnitzler, Arthur} (15.05.1862 – 21.10.1931), \emph{Schriftsteller, Mediziner}|pwk}s auf »Januar«, die Angabe des
                  Tages »20« in der Adressierungszeile des Telegramms und dem
                  Umstand, dass am 19. 1. 1895 bekannt wurde, dass \emph{Liebelei}\pwindex{Schnitzler, Arthur 15.05.1862 – 21.10.1931@\textsc{Schnitzler, Arthur} (15.05.1862 – 21.10.1931), \emph{Schriftsteller, Mediziner}!Liebelei. Schauspiel in drei Akten9. 10. 1895@\strich\emph{Liebelei. Schauspiel in drei Akten} {[}9. 10. 1895{]}|pwk} vom \emph{Burgtheater}\orgindex{Burgtheater@Burgtheater|pwk} angenommen
                  worden war.}}}\label{K_L02692-1h} dran nun wirds rasch aufwaerts gehen\pend
           \pstart haendedruck und glueckwuensche = \spacefill\mbox{goldmann. +}\pend{}\endnumbering\briefempfaengerindex{Schnitzler, Arthur@\textsc{Schnitzler, Arthur}!zzzGoldmann, Paul@\emph{von Paul Goldmann}!1895-01-201@{{[}20. 1. 1895?{]}}|)be}\mylabel{h}\end{ledgroupsized}\begin{anhang}\end{anhang}\newcommand{\dateiname}{L02692}\newcommand{\titel}{Paul Goldmann an Arthur Schnitzler, [20. 1. 1895?]}\newcommand{\editorInnen}{Martin Anton Müller und Laura Untner}\input{../tex-inputs/latex-pdf-abspann}
      