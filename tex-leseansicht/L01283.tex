%% latex-leseansicht-vorspann.tex
%% Vorspann für die Leseansicht.
%% Lädt die gemeinsame Datei latex-vorspann.tex mit nicht gesetztem Schalter.

\newif\ifkorrekturansicht
\korrekturansichtfalse

\input{../tex-inputs/latex-vorspann}


\section[Hermann Bahr an Arthur Schnitzler, {{[}}30. 3. 1903{{]}}]{L01283 Hermann Bahr an Arthur Schnitzler, {[}30. 3. 1903{]}}
\nopagebreak\mylabel{L01283v}
\rehead{ }\normalsize\beginnumbering\briefempfaengerindex{Schnitzler, Arthur@\textsc{Schnitzler, Arthur}!zzzBahr, Hermann@\emph{von Hermann Bahr}!1903-03-302@{{[}30. 3. 1903{]}}|(be}
\toendnotes[C]{\smallbreak\pagebreak[2]}
\correspDesc{Versand  durch Hermann Bahr am [30. 3. 1903] in Wien
\newline{}Erhalt  durch Arthur Schnitzler im Zeitraum [30. 3. 1903
                  – 3. 4. 1903?] in Wien}\toendnotes[C]{\smallbreak}
\Standort{CUL, Schnitzler, B 5b.}
\physDesc{Brief, 1 Blatt, 2 Seiten, 702 Zeichen
\newline{}Handschrift: schwarze Tinte, deutsche Kurrent
\newline{}Schnitzler: mit Bleistift datiert: »Ende März 903.« 
\newline{}Ordnung: mit Bleistift von unbekannter Hand nummeriert:
                                    »97« }
\buchAbdrucke{\weitereDrucke{Hermann Bahr, Arthur Schnitzler: \emph{Briefwechsel, Aufzeichnungen, Dokumente (1891–1931)}. Herausgegeben von Kurt Ifkovits und Martin Anton Müller. Göttingen: \emph{Wallstein} 2018, S. 258.} }
\pstart
           \raggedleft{}{\pb}Montag\pend
           
\pstart\center{}Lieber Arthur!\pend\vspace{0.5em}
\pstart
           Ich hatte{ }ſogleich bei Pötzl\pwindex{Pötzl, Eduard 17.\,3.\,1851 Wien – 20.\,8.\,1914 Mödling@\textsc{Pötzl, Eduard} (17.\,3.\,1851 Wien – 20.\,8.\,1914 Mödling), \emph{Schriftsteller, Journalist}|pw} (ſchriftlich,
               damit er es nicht ableugnen kann) ein Feuilleton über den Reigen\pwindex{Schnitzler, Arthur 15.\,5.\,1862 Wien – 21.\,10.\,1931 ebd.@\textsc{Schnitzler, Arthur} (15.\,5.\,1862 Wien – 21.\,10.\,1931 ebd.), \emph{Schriftsteller, Mediziner}!Reigen. Zehn Dialoge@\strich\emph{Reigen. Zehn Dialoge}|pw} angemeldet, um es ihm wenigstens zu erſchweren, daß er von
               anderer Seite etwas über das Buch bringt. Darauf erhalte ich eben folgende Antwort,
               die ich mir gelegentlich zurückerbitte. Ich gehe nun heute oder morgen mit der Sache
               zu Wilhelm Singer\pwindex{Singer, Wilhelm 26.\,11.\,1847 Bzenec – 10.\,10.\,1917 Wien@\textsc{Singer, Wilhelm} (26.\,11.\,1847 Bzenec – 10.\,10.\,1917 Wien), \emph{Journalist, Chefredakteur}|pw}, der mir Recht geben, über
                  P.\pwindex{Pötzl, Eduard 17.\,3.\,1851 Wien – 20.\,8.\,1914 Mödling@\textsc{Pötzl, Eduard} (17.\,3.\,1851 Wien – 20.\,8.\,1914 Mödling), \emph{Schriftsteller, Journalist}|pw} wahnſinnig{ }ſchimpfen und zuletzt
               entſcheiden wird, daß Leute wie wir – nemlich {\pb}Er,
               Ich und Du – viel zu hoch{ }ſtehen, um uns mit{ }ſolchen Burſchen einzulaſſen, das heißt
               daß es alſo bei P\pwindex{Pötzl, Eduard 17.\,3.\,1851 Wien – 20.\,8.\,1914 Mödling@\textsc{Pötzl, Eduard} (17.\,3.\,1851 Wien – 20.\,8.\,1914 Mödling), \emph{Schriftsteller, Journalist}|pw}’s Entſcheidung bleibt.\pend
           
\pstart
           Jedenfalls aber bitte ich Dich nochmals mir baldigſt ein Exemplar zu{ }ſchicken.\pend
           
\pstart
           Herzlichſt{\\[\baselineskip]}Dein{\\[\baselineskip]}\spacefill\mbox{Hermann}\pend
           \leftskip=0em{}\selectlanguage{ngerman}\endnumbering\briefempfaengerindex{Schnitzler, Arthur@\textsc{Schnitzler, Arthur}!zzzBahr, Hermann@\emph{von Hermann Bahr}!1903-03-302@{{[}30. 3. 1903{]}}|)be}\mylabel{L01283h}  \newcommand{\dateiname}{L01283}\newcommand{\titel}{Hermann Bahr an Arthur Schnitzler, [30. 3. 1903]}\newcommand{\editorInnen}{Herausgegeben von Martin Anton Müller}%% latex-leseansicht-abspann.tex
%% Abspann für die Leseansicht.
%% Der Schalter \ifkorrekturansicht ist bereits durch den Vorspann gesetzt.

%% latex-abspann.tex
%% Gemeinsamer Abspann für Korrekturansicht und Leseansicht.
%% Setzt den Schalter \ifkorrekturansicht voraus (gesetzt in den
%% einbindenden Dateien latex-korrekturansicht-abspann.tex bzw.
%% latex-leseansicht-abspann.tex).
%% ---------------------------------------------------------------

\normalsize

% Das esempio-Environment wird nur in der Leseansicht benötigt
\ifkorrekturansicht\else
\newenvironment{esempio}[3]%
{
    \vspace{1.5ex}
    \rlap{\underline{#1}}
    \par
    \setlength{\parindent}{0cm}
    \nopagebreak
    \leftskip=#2cm
    \rightskip=#3cm
}
{
    \par
}
\fi

\doendnotes{C}
\bigskip
\vfill

\clearpage

\footnotesize

\ifkorrekturansicht
  \lohead{\textsc{register}}
\fi

% theindex-Environment neu definieren ohne reledmac
\makeatletter
\renewenvironment{theindex}{%
  \ifkorrekturansicht
    \section*{\indexname}%
  \else
    \subsubsection*{Index der erwähnten Entitäten}%
  \fi
  \setlength{\parindent}{0pt}%
  \setlength{\parskip}{0pt plus 0.3pt}%
  \let\item\@idxitem
}{%
  \ifkorrekturansicht\clearpage\fi
}
\makeatother

\IfFileExists{\jobname-pw.ind}{\input{\jobname-pw.ind}}{}

% Quellenangabe nur in der Leseansicht
\ifkorrekturansicht\else
% Fallback-Definitionen, falls die .tex-Datei \titel etc. nicht gesetzt hat
\providecommand{\titel}{}
\providecommand{\editorInnen}{}
\providecommand{\dateiname}{\jobname}

\vspace{3cm}

\vfill

\footnotesize
\textsc{Quelle}: \titel. Herausgegeben von {\editorInnen}. In: \emph{Arthur Schnitzler: Briefwechsel mit Autorinnen und Autoren}.
 Digitale Edition, https://schnitzler-briefe.acdh.oeaw.ac.at/{\dateiname}.html (Stand \today)
\fi

\end{document}


