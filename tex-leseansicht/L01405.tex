%% latex-leseansicht-vorspann.tex
%% Vorspann für die Leseansicht.
%% Lädt die gemeinsame Datei latex-vorspann.tex mit nicht gesetztem Schalter.

\newif\ifkorrekturansicht
\korrekturansichtfalse

\input{../tex-inputs/latex-vorspann}


         
         \newcommand{\erwaehntePersonen}{Personen: Richard Beer-Hofmann, Alexandre fils Dumas,  Leb}
         \newcommand{\erwaehnteOrte}{Orte: Burgtheater, Rodaun, Wien}
         \newcommand{\erwaehnteWerke}{Werke: Vater und Sohn}
               \section[Arthur Schnitzler an Richard Beer-Hofmann, 9. 6. 1904]{ Arthur Schnitzler an Richard Beer-Hofmann, 9. 6. 1904}\nopagebreak\mylabel{v}\rehead{ }\begin{ledgroupsized}[t]{13cm}\normalsize\beginnumbering \toendnotes[C]{\smallbreak\pagebreak[2]} \Standort{YCGL, MSS 31.}
\physDesc{Telegramm
\newline{}Handschrift  Leb: 1) Bleistift, deutsche Kurrent\hspace{1em}2) Bleistift, lateinische Kurrent (\noindent{}Adresse)\hspace{1em}\newline{}Versand: »\noindent{}\textcolor{gray}{\textbf{Aufgegeben am}}{ }9/VI{ }\textcolor{gray}{\textbf{um}}{ }4 \textcolor{gray}{\textbf{Uhr {\dots} Min.}} N \textcolor{gray}{\textbf{Mi{[}ttag{]}}}{ / }\textcolor{gray}{\textbf{Eingelangt von}} L. \textcolor{gray}{\textbf{auf Leitung Nr.}}{ }43 \textcolor{gray}{\textbf{am}}{ }9/VI\textcolor{gray}{\textbf{190}}4 { }\textcolor{gray}{\textbf{um}}{ }4 \textcolor{gray}{\textbf{Uhr}} 45 \textcolor{gray}{\textbf{Min.}} N \textcolor{gray}{\textbf{Mittag}}{ / }\textcolor{gray}{\textbf{Aufgenommen durch}}{ }\textsc{Leb\pwindex{Leb @\textsc{Leb}, \emph{Telegrafenbeamter/Telegrafenbeamtin}|pw}}{ / }\textcolor{gray}{\textbf{Von}}{ }Wien 113\oindex{Wien@\textbf{Wien}|pw}{ }\textcolor{gray}{\textbf{Aufgabe-Nr.}} 64{ }\textcolor{gray}{\textbf{mit}} 21 \textcolor{gray}{\textbf{Taxworten ({\dots} Worten {\dots} Chiffern)}}« }\toendnotes[C]{\smallbreak}\pstart{}{\pb}Richard Beerhofma{\geminationn}\pend{}\pstart{}Rodaun\oindex{Rodaun@\textbf{Rodaun}|pw}\pend{}{\bigskip}\pstart
           \noindent{}{\pb}Wären gern gekommen haben aber \label{K_L01405-1v}\edtext{Sitze Burgtheater\oindex{Burgtheater@\textbf{Burgtheater}|pw}}{\lemma{\textnormal{\emph{Sitze Burgtheater}}}\Cendnote{\textnormal{für \emph{Vater und Sohn}\pwindex{Dumas, Alexandre fils 28.07.1824 – 27.11.1895@\textsc{Dumas, Alexandre fils} (28.07.1824 – 27.11.1895), \emph{Schriftsteller}!Vater und Sohn1859@\strich\emph{Vater und Sohn} {[}1859{]}|pwk} von Alexandre Dumas
                     fils\pwindex{Dumas, Alexandre fils 28.07.1824 – 27.11.1895@\textsc{Dumas, Alexandre fils} (28.07.1824 – 27.11.1895), \emph{Schriftsteller}|pwk}.}}}\label{K_L01405-1h}. Grüßen Sie die verſa{\geminationm}elten auf
               ſehr baldiges Wiederſehen herzlichſt\pend
           \pstart \spacefill\mbox{Arthur}\pend{}
         
         \endnumbering\mylabel{h}\end{ledgroupsized}  \newcommand{\dateiname}{L01405}\newcommand{\titel}{Arthur Schnitzler an Richard Beer-Hofmann, 9. 6. 1904}\newcommand{\editorInnen}{Martin Anton Müller und Gerd-Hermann Susen}%% latex-leseansicht-abspann.tex
%% Abspann für die Leseansicht.
%% Der Schalter \ifkorrekturansicht ist bereits durch den Vorspann gesetzt.

%% latex-abspann.tex
%% Gemeinsamer Abspann für Korrekturansicht und Leseansicht.
%% Setzt den Schalter \ifkorrekturansicht voraus (gesetzt in den
%% einbindenden Dateien latex-korrekturansicht-abspann.tex bzw.
%% latex-leseansicht-abspann.tex).
%% ---------------------------------------------------------------

\normalsize

% Das esempio-Environment wird nur in der Leseansicht benötigt
\ifkorrekturansicht\else
\newenvironment{esempio}[3]%
{
    \vspace{1.5ex}
    \rlap{\underline{#1}}
    \par
    \setlength{\parindent}{0cm}
    \nopagebreak
    \leftskip=#2cm
    \rightskip=#3cm
}
{
    \par
}
\fi

\doendnotes{C}
\bigskip
\vfill

\clearpage

\footnotesize

\ifkorrekturansicht
  \lohead{\textsc{register}}
\fi

% theindex-Environment neu definieren ohne reledmac
\makeatletter
\renewenvironment{theindex}{%
  \ifkorrekturansicht
    \section*{\indexname}%
  \else
    \subsubsection*{Index der erwähnten Entitäten}%
  \fi
  \setlength{\parindent}{0pt}%
  \setlength{\parskip}{0pt plus 0.3pt}%
  \let\item\@idxitem
}{%
  \ifkorrekturansicht\clearpage\fi
}
\makeatother

\IfFileExists{\jobname-pw.ind}{\input{\jobname-pw.ind}}{}

% Quellenangabe nur in der Leseansicht
\ifkorrekturansicht\else
% Fallback-Definitionen, falls die .tex-Datei \titel etc. nicht gesetzt hat
\providecommand{\titel}{}
\providecommand{\editorInnen}{}
\providecommand{\dateiname}{\jobname}

\vspace{3cm}

\vfill

\footnotesize
\textsc{Quelle}: \titel. Herausgegeben von {\editorInnen}. In: \emph{Arthur Schnitzler: Briefwechsel mit Autorinnen und Autoren}.
 Digitale Edition, https://schnitzler-briefe.acdh.oeaw.ac.at/{\dateiname}.html (Stand \today)
\fi

\end{document}


      