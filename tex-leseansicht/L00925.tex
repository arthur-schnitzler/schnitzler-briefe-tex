%% latex-leseansicht-vorspann.tex
%% Vorspann für die Leseansicht.
%% Lädt die gemeinsame Datei latex-vorspann.tex mit nicht gesetztem Schalter.

\newif\ifkorrekturansicht
\korrekturansichtfalse

\input{../tex-inputs/latex-vorspann}


         
         \renewcommand{\erwaehntePersonen}{Personen: André Antoine,  Bech, Georg Brandes, Marie Reinhard, Émile Soutif}
         \renewcommand{\erwaehnteOrte}{Orte: Berlin, Dresden, Kopenhagen, Paris, Théâtre Antoine-Simone Berriau, Wien}
         \renewcommand{\erwaehnteWerke}{Werke: Der grüne Kakadu. Groteske in einem Akt, Die Nächste, Tagebuch}
               \section[Arthur Schnitzler an Georg Brandes, 15. 6. 1899]{ Arthur Schnitzler an Georg Brandes, 15. 6. 1899}\nopagebreak\mylabel{v}\rehead{ }\begin{ledgroupsized}[t]{13cm}\normalsize\beginnumbering \toendnotes[C]{\smallbreak\pagebreak[2]} \Standort{Kopenhagen, Det Kongelige Bibliotek, Georg Brandes Arkiv, box 125.}
\physDesc{Briefkarte
\newline{}Handschrift: schwarze Tinte, deutsche Kurrent\newline{}Ordnung: mit Bleistift von unbekannter Hand nummeriert:
                                    »18.« und datiert: »15/6 99« }\buchAbdrucke{\weitereDrucke{Georg Brandes, Arthur Schnitzler: \emph{Ein Briefwechsel}. Hg. Kurt Bergel. Bern: \emph{Francke} 1956, S. 78–79.} }\toendnotes[C]{\smallbreak}\pstart
           \noindent{}{\pb}Verehrter Herr Brandes, ich denke, die Adreſſe \textsc{Antoine}\pwindex{Antoine, Andre 1858-01-31 – 1943-10-23@\textsc{Antoine, André} (1858-01-31 – 1943-10-23), \emph{Theaterleiter, Schauspieler}|pw}\textsc{, Direktor} des \textsc{theatre Antoine}\oindex{Theâtre Antoine-Simone Berriau@\textbf{Théâtre Antoine-Simone Berriau}|pw} in \textsc{Paris}\oindex{Paris@\textbf{Paris}|pw} genügt; ich weiſs wenigſtens keine andere. Noch einmal wiederhole ich, daſs ich
               Sie um nichts andres bitte, als \textsc{Antoine}\pwindex{Antoine, Andre 1858-01-31 – 1943-10-23@\textsc{Antoine, André} (1858-01-31 – 1943-10-23), \emph{Theaterleiter, Schauspieler}|pw}{ }\uline{zum \introOben{}baldigen\introOben{} Leſen des \textsc{Manuscriptes}} aufzufordern; Ihr Name iſt in Paris\oindex{Berlin@\textbf{Berlin}|pw}{ }ſo berühmt wie anderswo (muß ich Ihnen das wirklich
               ſagen?) mich ke{\geminationn}t dort kein Menſch. Ich ſelbſt habe mich
               um eine Überſetzung des »Kakadu\pwindex{Schnitzler, Arthur 15.05.1862 – 21.10.1931@\textsc{Schnitzler, Arthur} (15.05.1862 – 21.10.1931), \emph{Schriftsteller, Mediziner}!gruene Kakadu. Groteske in einem Akt1. 3. 1899@\strich\emph{Der grüne Kakadu. Groteske in einem Akt} {[}1. 3. 1899{]}|pw}« nicht bemüht; zwei
               Herren, einer, \textsc{Soutif}\pwindex{Soutif, Emile @\textsc{Soutif, Émile}, \emph{Lehrer}|pw} in Dresden\oindex{Dresden@\textbf{Dresden}|pw}, ein zweiter \textsc{Bech}\pwindex{Bech @\textsc{Bech}, \emph{Übersetzer/Übersetzerin}|pw}, in Paris\oindex{Paris@\textbf{Paris}|pw}{ }{\pb}haben ſich an mich um Erlaubnis gewandt; und we{\geminationn} es ſich machen ließe, wäre mir eine Pariſ\oindex{Paris@\textbf{Paris}|pw}er Aufführung natürlich ſehr erwünſcht. –\pend
           \pstart
           In den letzten Tagen habe ich wieder zu arbeiten begonnen; eine kleine Novelle\pwindex{Schnitzler, Arthur 15.05.1862 – 21.10.1931@\textsc{Schnitzler, Arthur} (15.05.1862 – 21.10.1931), \emph{Schriftsteller, Mediziner}!Naechste1899@\strich\emph{Die Nächste} {[}1899{]}|pwv}\pwindex{Schnitzler, Arthur 15.05.1862 – 21.10.1931@\textsc{Schnitzler, Arthur} (15.05.1862 – 21.10.1931), \emph{Schriftsteller, Mediziner}!Naechste1899@\strich\emph{Die Nächste} {[}1899{]}|pwv}, die ich gerade zu \label{K_L00925_1v}\edtext{\uline{jener} Zeit}{\lemma{\textnormal{\emph{jener Zeit}}}\Cendnote{\textnormal{Gemeint ist die postum veröffentlichte Novelle \emph{Die Nächste}\pwindex{Schnitzler, Arthur 15.05.1862 – 21.10.1931@\textsc{Schnitzler, Arthur} (15.05.1862 – 21.10.1931), \emph{Schriftsteller, Mediziner}!Naechste1899@\strich\emph{Die Nächste} {[}1899{]}|pwk}\pwindex{Schnitzler, Arthur 15.05.1862 – 21.10.1931@\textsc{Schnitzler, Arthur} (15.05.1862 – 21.10.1931), \emph{Schriftsteller, Mediziner}!Naechste1899@\strich\emph{Die Nächste} {[}1899{]}|pwk}. An der Novelle arbeitete er am 15. 3. 1899 – drei Tage
                  vor dem Tod Marie Reinhards\pwindex{Reinhard, Marie 1871-03-13 – 1899-03-18@\textsc{Reinhard, Marie} (1871-03-13 – 1899-03-18), \emph{Gesangspädagogin}|pwk}, danach hält das \emph{Tagebuch}\pwindex{Schnitzler, Arthur 15.05.1862 – 21.10.1931@\textsc{Schnitzler, Arthur} (15.05.1862 – 21.10.1931), \emph{Schriftsteller, Mediziner}!Tagebuch1981 – 2000@\strich\emph{Tagebuch} {[}1981 – 2000{]}|pwk} am 12. 6. 1899 die Weiterarbeit fest. 
                  Er beendete sie »vorläufig« am 6. 7. 1899.}}}\label{K_L00925_1h}{ }\substVorne{}\textsuperscript{begonn}{\allowbreak}\substDazwischen{}angefa\substHinten{}ngen hatte, und in der mir heute alle möglichen Ahnungen zu zittern
               ſcheinen.\pend
           \pstart
           Ich freue mich, daſs Sie endlich außer Bette ſind; ich hoffe und wünſche \introOben{}Ihnen\introOben{} für weiterhin alles gute und ſchöne.\pend
           \pstart Ihr \spacefill\mbox{Arthur Schnitzler}\pend{}\pstart
           15. 6. 99.\pend
           
         
         \endnumbering\mylabel{h}\end{ledgroupsized}  \newcommand{\dateiname}{L00925}\newcommand{\titel}{Arthur Schnitzler an Georg Brandes, 15. 6. 1899}\newcommand{\editorInnen}{Martin Anton Müller und Gerd-Hermann Susen}%% latex-leseansicht-abspann.tex
%% Abspann für die Leseansicht.
%% Der Schalter \ifkorrekturansicht ist bereits durch den Vorspann gesetzt.

%% latex-abspann.tex
%% Gemeinsamer Abspann für Korrekturansicht und Leseansicht.
%% Setzt den Schalter \ifkorrekturansicht voraus (gesetzt in den
%% einbindenden Dateien latex-korrekturansicht-abspann.tex bzw.
%% latex-leseansicht-abspann.tex).
%% ---------------------------------------------------------------

\normalsize

% Das esempio-Environment wird nur in der Leseansicht benötigt
\ifkorrekturansicht\else
\newenvironment{esempio}[3]%
{
    \vspace{1.5ex}
    \rlap{\underline{#1}}
    \par
    \setlength{\parindent}{0cm}
    \nopagebreak
    \leftskip=#2cm
    \rightskip=#3cm
}
{
    \par
}
\fi

\doendnotes{C}
\bigskip
\vfill

\clearpage

\footnotesize

\ifkorrekturansicht
  \lohead{\textsc{register}}
\fi

% theindex-Environment neu definieren ohne reledmac
\makeatletter
\renewenvironment{theindex}{%
  \ifkorrekturansicht
    \section*{\indexname}%
  \else
    \subsubsection*{Index der erwähnten Entitäten}%
  \fi
  \setlength{\parindent}{0pt}%
  \setlength{\parskip}{0pt plus 0.3pt}%
  \let\item\@idxitem
}{%
  \ifkorrekturansicht\clearpage\fi
}
\makeatother

\IfFileExists{\jobname-pw.ind}{\input{\jobname-pw.ind}}{}

% Quellenangabe nur in der Leseansicht
\ifkorrekturansicht\else
% Fallback-Definitionen, falls die .tex-Datei \titel etc. nicht gesetzt hat
\providecommand{\titel}{}
\providecommand{\editorInnen}{}
\providecommand{\dateiname}{\jobname}

\vspace{3cm}

\vfill

\footnotesize
\textsc{Quelle}: \titel. Herausgegeben von {\editorInnen}. In: \emph{Arthur Schnitzler: Briefwechsel mit Autorinnen und Autoren}.
 Digitale Edition, https://schnitzler-briefe.acdh.oeaw.ac.at/{\dateiname}.html (Stand \today)
\fi

\end{document}


      