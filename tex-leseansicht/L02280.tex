%% latex-leseansicht-vorspann.tex
%% Vorspann für die Leseansicht.
%% Lädt die gemeinsame Datei latex-vorspann.tex mit nicht gesetztem Schalter.

\newif\ifkorrekturansicht
\korrekturansichtfalse

\input{../tex-inputs/latex-vorspann}


         
         \renewcommand{\erwaehntePersonen}{Personen: Samuel Fischer, Frieda Pollak, Olga Schnitzler}
         \renewcommand{\erwaehnteOrte}{Orte: Deutschland, Rodaun, Stallburggasse, Sternwartestraße, Volkstheater, Wien, XVIII., Währing}
         \renewcommand{\erwaehnteWerke}{Werke: Die prosaischen Schriften, Fink und Fliederbusch. Komödie in drei Akten}
               \section[Hugo von Hofmannsthal an Arthur Schnitzler, 13. 11. {[}1917{]}]{ Hugo von Hofmannsthal an Arthur Schnitzler, 13. 11. {[}1917{]}}\nopagebreak\mylabel{v}\rehead{ }\begin{ledgroupsized}[t]{13cm}\normalsize\beginnumbering \toendnotes[C]{\smallbreak\pagebreak[2]} \Standort{CUL, Schnitzler, B 43.}
\physDesc{Kartenbrief
\newline{}Handschrift: 1) schwarze Tinte, deutsche Kurrent\hspace{1em}2) schwarze Tinte, lateinische Kurrent (\noindent{}Adresse)\hspace{1em}\newline{}Versand: Stempel: »\nobreak{}\oindex{Rodaun@\textbf{Rodaun}|pwk}Rodaun, 13. 1{[}1. 1917{]}, 2 N\nobreak{}«.  
\newline{}Schnitzler: 1) mit Bleistift beschriftet: »\textsc{Hugo}«, datiert »18?«  2) mit rotem Buntstift eine Unterstreichung\newline{}Ordnung: 1) mit Bleistift von Frieda
                                    Pollak\pwindex{Pollak, Frieda 08.12.1881 – 13.07.1937@\textsc{Pollak, Frieda} (08.12.1881 – 13.07.1937), \emph{Sekretärin}|pw} (?) mit dem Buchstaben »A«
                                 (Abgeschrieben/Abschrift) gekennzeichnet  2) mit Bleistift von unbekannter Hand nummeriert:
                                    »390«}\buchAbdrucke{\weitereDrucke{Hugo von Hofmannsthal, Arthur Schnitzler: \emph{Briefwechsel}. Hg. Therese Nickl und Heinrich Schnitzler. Frankfurt am Main: \emph{S. Fischer} 1964, S. 282.} }\toendnotes[C]{\smallbreak}\pstart{}{\pb}Herrn D\textsuperscript{r} Arthur Schnitzler\pend{}\pstart{}Wien\oindex{Wien@\textbf{Wien}|pw}\pend{}\pstart{}XVIII\oindex{XVIII., Waehring@\textbf{XVIII., Währing}|pw}\pend{}\pstart{}Sternwartestrasse 71\oindex{Sternwartestrasse@\textbf{Sternwartestraße}|pw}\pend{}{\bigskip}\pstart
           \raggedleft{}{\pb}\textsc{R}\oindex{Rodaun@\textbf{Rodaun}|pw}{ }\label{K_L02280_1v}\edtext{12 XI}{\lemma{\textnormal{\emph{12 XI}}}\Cendnote{\textnormal{Hier ist ein Irrtum des Verfassers
                     anzunehmen. Sowohl der Poststempel als auch der Verweis auf die »morgige«
                     Uraufführung verweisen auf den 13. 11. 1917 als Tag der
                     Abfassung.}}}\label{K_L02280_1h}\pend
           \pstart
           mein lieber Arthur\hspace*{1.5em}der \label{K_L02280_2v}\edtext{dritte}{\lemma{\textnormal{\emph{dritte}}}\Cendnote{\textnormal{Der erste erschien
                     1907, der zweite 1914, der dritte Ende November
                     1917.}}}\label{K_L02280_2h} Band meiner Proſaarbeiten\pwindex{Hofmannsthal, Hugo von 1874-02-01 – 1929-07-15@\textsc{Hofmannsthal, Hugo von} (1874-02-01 – 1929-07-15), \emph{Schriftsteller}!prosaischen Schriften1907 – 1917@\strich\emph{Die prosaischen Schriften} {[}1907 – 1917{]}|pw} wird in dieſen Tagen durch Fiſcher\pwindex{Fischer, Samuel 24.12.1859 – 15.10.1934@\textsc{Fischer, Samuel} (24.12.1859 – 15.10.1934), \emph{Verleger}|pw} an Sie geſchickt werden, bitte nehmen Sie ihn wie aus meiner Hand,
               ich habe den Auftrag gegeben, diesmal direct zu ſchicken, weil man ja weder Papier
               noch Spagat mehr hat, um von Haus Bücher zu verſenden. Und ſo iſt man ſchließlich
               auch voneinander abgeſchnitten, durch die Einſchränkung der Verkehrsmittel u. die
               Unmöglichkeit, eine Abendmahlzeit herzuſtellen.\pend
           \pstart
           Wenn ich \label{K_L02280_3v}\edtext{aus Deutſchland\oindex{Deutschland@\textbf{Deutschland}|pw} zurückkomme}{\lemma{\textnormal{\emph{aus … zurückkomme}}}\Cendnote{\textnormal{Die Reise dauerte vom 20. 11. 1917 bis zum
                     8. 12. 1917.}}}\label{K_L02280_3h}, Mitte December, ſo hoffe ich
               daſs Sie u. Olga\pwindex{Schnitzler, Olga 17.01.1882 – 13.01.1970@\textsc{Schnitzler, Olga} (17.01.1882 – 13.01.1970), \emph{Schauspielerin, Sängerin}|pw} einmal gegen Abend in unſere
               kleine Stadtwohnung\oindex{Stallburggasse@\textbf{Stallburggasse}|pwv} ko{\geminationm}en werden. Indeſſen freue ich mich auf \label{K_L02280_4v}\edtext{morgen Abend}{\lemma{\textnormal{\emph{morgen Abend}}}\Cendnote{\textnormal{Uraufführung von \emph{Fink und
                     Fliederbusch}\pwindex{Schnitzler, Arthur 15.05.1862 – 21.10.1931@\textsc{Schnitzler, Arthur} (15.05.1862 – 21.10.1931), \emph{Schriftsteller, Mediziner}!Fink und Fliederbusch. Komoedie in drei Akten1917@\strich\emph{Fink und Fliederbusch. Komödie in drei Akten} {[}1917{]}|pwk} am 14. 11. 1917 im Deutschen
                     Volkstheater\oindex{Volkstheater@\textbf{Volkstheater}|pwk}.}}}\label{K_L02280_4h}, und werde für das Ernſte u. für den Spaß in Ihrer
                  Comödie\pwindex{Schnitzler, Arthur 15.05.1862 – 21.10.1931@\textsc{Schnitzler, Arthur} (15.05.1862 – 21.10.1931), \emph{Schriftsteller, Mediziner}!Fink und Fliederbusch. Komoedie in drei Akten1917@\strich\emph{Fink und Fliederbusch. Komödie in drei Akten} {[}1917{]}|pwv} ein guter Zuhörer
               ſein.\pend
           \pstart Herzlich Ihr\spacefill\mbox{Hugo}\pend{}
         
         \endnumbering\mylabel{h}\end{ledgroupsized}  \newcommand{\dateiname}{L02280}\newcommand{\titel}{Hugo von Hofmannsthal an Arthur Schnitzler, 13. 11. [1917]}\newcommand{\editorInnen}{Martin Anton Müller und Gerd-Hermann Susen}%% latex-leseansicht-abspann.tex
%% Abspann für die Leseansicht.
%% Der Schalter \ifkorrekturansicht ist bereits durch den Vorspann gesetzt.

%% latex-abspann.tex
%% Gemeinsamer Abspann für Korrekturansicht und Leseansicht.
%% Setzt den Schalter \ifkorrekturansicht voraus (gesetzt in den
%% einbindenden Dateien latex-korrekturansicht-abspann.tex bzw.
%% latex-leseansicht-abspann.tex).
%% ---------------------------------------------------------------

\normalsize

% Das esempio-Environment wird nur in der Leseansicht benötigt
\ifkorrekturansicht\else
\newenvironment{esempio}[3]%
{
    \vspace{1.5ex}
    \rlap{\underline{#1}}
    \par
    \setlength{\parindent}{0cm}
    \nopagebreak
    \leftskip=#2cm
    \rightskip=#3cm
}
{
    \par
}
\fi

\doendnotes{C}
\bigskip
\vfill

\clearpage

\footnotesize

\ifkorrekturansicht
  \lohead{\textsc{register}}
\fi

% theindex-Environment neu definieren ohne reledmac
\makeatletter
\renewenvironment{theindex}{%
  \ifkorrekturansicht
    \section*{\indexname}%
  \else
    \subsubsection*{Index der erwähnten Entitäten}%
  \fi
  \setlength{\parindent}{0pt}%
  \setlength{\parskip}{0pt plus 0.3pt}%
  \let\item\@idxitem
}{%
  \ifkorrekturansicht\clearpage\fi
}
\makeatother

\IfFileExists{\jobname-pw.ind}{\input{\jobname-pw.ind}}{}

% Quellenangabe nur in der Leseansicht
\ifkorrekturansicht\else
% Fallback-Definitionen, falls die .tex-Datei \titel etc. nicht gesetzt hat
\providecommand{\titel}{}
\providecommand{\editorInnen}{}
\providecommand{\dateiname}{\jobname}

\vspace{3cm}

\vfill

\footnotesize
\textsc{Quelle}: \titel. Herausgegeben von {\editorInnen}. In: \emph{Arthur Schnitzler: Briefwechsel mit Autorinnen und Autoren}.
 Digitale Edition, https://schnitzler-briefe.acdh.oeaw.ac.at/{\dateiname}.html (Stand \today)
\fi

\end{document}


      