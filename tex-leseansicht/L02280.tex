%% latex-leseansicht-vorspann.tex
%% Vorspann für die Leseansicht.
%% Lädt die gemeinsame Datei latex-vorspann.tex mit nicht gesetztem Schalter.

\newif\ifkorrekturansicht
\korrekturansichtfalse

\input{../tex-inputs/latex-vorspann}


\section[Hugo von Hofmannsthal an Arthur Schnitzler, 13. 11. [1917]]{L02280 Hugo von Hofmannsthal an Arthur Schnitzler, 13. 11. [1917]}
\nopagebreak\mylabel{L02280v}
\rehead{ }\normalsize\beginnumbering\briefempfaengerindex{Schnitzler, Arthur@\textsc{Schnitzler, Arthur}!zzzHofmannsthal, Hugo von@\emph{von Hugo von Hofmannsthal}!1917-11-131@{13. 11. [1917]}|(be}
\toendnotes[C]{\smallbreak\pagebreak[2]}
\correspDesc{Versand  durch Hugo von Hofmannsthal am 13. 11. [1917] in Rodaun
\newline{}Erhalt  durch Arthur Schnitzler im Zeitraum [14. 11. 1917 – 18. 11. 1917?] in Wien}\toendnotes[C]{\smallbreak}
\Standort{CUL, Schnitzler, B 43.}
\physDesc{Kartenbrief, 782 Zeichen
\newline{}Handschrift: schwarze Tinte, deutsche Kurrent
\newline{}Versand: Stempel: »\nobreak{}\oindex{Wien@\textbf{Wien}!XXIII., Liesing@\textbf{XXIII., Liesing}!Rodaun@\textbf{Rodaun}, \emph{Region}|pwk}Rodaun, 13. 1{[}1. 1917{]}, 2 N\nobreak{}«.  
\newline{}Schnitzler: 1) mit Bleistift beschriftet: »\textsc{Hugo}«, datiert »18?«  2) mit rotem Buntstift eine Unterstreichung
\newline{}Ordnung: 1) mit Bleistift von Frieda
                                    Pollak\pwindex{Pollak, Frieda 8.\,12.\,1881 Wien – 13.\,7.\,1937 ebd.@\textsc{Pollak, Frieda} (8.\,12.\,1881 Wien – 13.\,7.\,1937 ebd.), \emph{Sekretärin}|pw} (?) mit dem Buchstaben »A«
                                 (Abgeschrieben/Abschrift) gekennzeichnet  2) mit Bleistift von unbekannter Hand nummeriert:
                                    »390«}
\buchAbdrucke{\weitereDrucke{Hugo von Hofmannsthal, Arthur Schnitzler: \emph{Briefwechsel}. Herausgegeben von Therese Nickl und Heinrich Schnitzler. Frankfurt am Main: \emph{S. Fischer} 1964, S. 282.} }\toendnotes[C]{\smallbreak}\pstart{}\textsc{{\pb}Herrn D\textsuperscript{r} Arthur Schnitzler}\pend{}\pstart{}\textsc{Wien\oindex{Wien@\textbf{Wien}, \emph{Verwaltungsgebiet}|pw}}\pend{}\pstart{}\textsc{XVIII\oindex{XVIII., Währing@\textbf{XVIII., Währing}, \emph{Verwaltungsgebiet}|pw}}\pend{}\pstart{}\textsc{Sternwartestrasse 71\oindex{Wien@\textbf{Wien}!XVIII., Währing@\textbf{XVIII., Währing}!Sternwartestraße 71@\textbf{Sternwartestraße 71}, \emph{Wohngebäude}|pw}}\pend{}{\bigskip}\vspace{1em}
\pstart
           \raggedleft{}{\pb}\textsc{R}\oindex{Wien@\textbf{Wien}!XXIII., Liesing@\textbf{XXIII., Liesing}!Rodaun@\textbf{Rodaun}, \emph{Region}|pw}{ }\label{K_L02280-1v}\edtext{12 XI}{\lemma{\textnormal{\emph{12 XI}}}\Cendnote{\textnormal{Hier ist ein Irrtum des Verfassers
                     anzunehmen. Sowohl der Poststempel als auch der Verweis auf die »morgige«
                      Uraufführung\eventindex{Volkstheater@\textbf{Volkstheater}!Uraufführung von Fink und Fliederbusch, 14.11.1917@Uraufführung von Fink und Fliederbusch, 14.11.1917|pwkv} verweisen auf den 13. 11. 1917 als Tag der
                     Abfassung.}}}\label{K_L02280-1}\pend
           \vspace{0.5em}
\pstart
           mein lieber Arthur\hspace*{1.5em}der \label{K_L02280-2v}\edtext{dritte}{\lemma{\textnormal{\emph{dritte}}}\Cendnote{\textnormal{Der erste erschien
                     1907, der zweite 1914, der dritte Ende November 1917.}}}\label{K_L02280-2} Band meiner Proſaarbeiten\pwindex{Hofmannsthal, Hugo von 1.\,2.\,1874 Wien – 15.\,7.\,1929 Rodaun@\textsc{Hofmannsthal, Hugo von} (1.\,2.\,1874 Wien – 15.\,7.\,1929 Rodaun), \emph{Schriftsteller}!prosaischen Schriften@\strich\emph{Die prosaischen Schriften}|pw} wird in dieſen Tagen durch Fiſcher\pwindex{Fischer, Samuel 24.\,12.\,1859 Liptovský Mikuláš – 15.\,10.\,1934 Berlin@\textsc{Fischer, Samuel} (24.\,12.\,1859 Liptovský Mikuláš – 15.\,10.\,1934 Berlin), \emph{Verleger}|pw} an Sie geſchickt werden, bitte nehmen Sie ihn wie aus meiner Hand,
               ich habe den Auftrag gegeben, diesmal direct zu{ }ſchicken, weil man ja weder Papier
               noch Spagat mehr hat, um von Haus Bücher zu verſenden. Und{ }ſo iſt man{ }ſchließlich
               auch voneinander abgeſchnitten, durch die Einſchränkung der Verkehrsmittel u. die
               Unmöglichkeit, eine Abendmahlzeit herzuſtellen.\pend
           
\pstart
           Wenn ich \label{K_L02280-3v}\edtext{aus Deutſchland\oindex{Deutschland@\textbf{Deutschland}|pw} zurückkomme}{\lemma{\textnormal{\emph{aus … zurückkomme}}}\Cendnote{\textnormal{Die Reise dauerte vom 20. 11. 1917 bis zum
                     8. 12. 1917.}}}\label{K_L02280-3}, Mitte December,{ }ſo hoffe ich
               daſs Sie u. Olga\pwindex{Schnitzler, Olga 17.\,1.\,1882 Wien – 13.\,1.\,1970 Lugano@\textsc{Schnitzler, Olga} (17.\,1.\,1882 Wien – 13.\,1.\,1970 Lugano), \emph{Schauspielerin, Sängerin}|pw} einmal gegen Abend in unſere
                kleine Stadtwohnung\oindex{Wien@\textbf{Wien}!I., Innere Stadt@\textbf{I., Innere Stadt}!Stallburggasse@\textbf{Stallburggasse}, \emph{Straße}|pwv} ko{\geminationm}en werden. Indeſſen freue ich mich auf \label{K_L02280-4v}\edtext{morgen Abend}{\lemma{\textnormal{\emph{morgen Abend}}}\Cendnote{\textnormal{Uraufführung\eventindex{Volkstheater@\textbf{Volkstheater}!Uraufführung von Fink und Fliederbusch, 14.11.1917@Uraufführung von Fink und Fliederbusch, 14.11.1917|pwkv} von \emph{Fink und
                     Fliederbusch}\pwindex{Schnitzler, Arthur 15.\,5.\,1862 Wien – 21.\,10.\,1931 ebd.@\textsc{Schnitzler, Arthur} (15.\,5.\,1862 Wien – 21.\,10.\,1931 ebd.), \emph{Schriftsteller, Mediziner}!Fink und Fliederbusch. Komödie in drei Akten@\strich\emph{Fink und Fliederbusch. Komödie in drei Akten}|pwk} am 14. 11. 1917 im Deutschen
                     Volkstheater\oindex{Wien@\textbf{Wien}!VII., Neubau@\textbf{VII., Neubau}!Volkstheater@\textbf{Volkstheater}, \emph{Theater}|pwk}.}}}\label{K_L02280-4}, und werde für das Ernſte u. für den Spaß in Ihrer
                  Comödie\pwindex{Schnitzler, Arthur 15.\,5.\,1862 Wien – 21.\,10.\,1931 ebd.@\textsc{Schnitzler, Arthur} (15.\,5.\,1862 Wien – 21.\,10.\,1931 ebd.), \emph{Schriftsteller, Mediziner}!Fink und Fliederbusch. Komödie in drei Akten@\strich\emph{Fink und Fliederbusch. Komödie in drei Akten}|pwv} ein guter Zuhörer{ }ſein.\pend
           \pstart Herzlich Ihr\spacefill\mbox{Hugo}\pend{}\selectlanguage{ngerman}\endnumbering\briefempfaengerindex{Schnitzler, Arthur@\textsc{Schnitzler, Arthur}!zzzHofmannsthal, Hugo von@\emph{von Hugo von Hofmannsthal}!1917-11-131@{13. 11. [1917]}|)be}\mylabel{L02280h}  \newcommand{\dateiname}{L02280}\newcommand{\titel}{Hugo von Hofmannsthal an Arthur Schnitzler, 13. 11. [1917]}\newcommand{\editorInnen}{Martin Anton Müller und Gerd-Hermann Susen}%% latex-leseansicht-abspann.tex
%% Abspann für die Leseansicht.
%% Der Schalter \ifkorrekturansicht ist bereits durch den Vorspann gesetzt.

%% latex-abspann.tex
%% Gemeinsamer Abspann für Korrekturansicht und Leseansicht.
%% Setzt den Schalter \ifkorrekturansicht voraus (gesetzt in den
%% einbindenden Dateien latex-korrekturansicht-abspann.tex bzw.
%% latex-leseansicht-abspann.tex).
%% ---------------------------------------------------------------

\normalsize

% Das esempio-Environment wird nur in der Leseansicht benötigt
\ifkorrekturansicht\else
\newenvironment{esempio}[3]%
{
    \vspace{1.5ex}
    \rlap{\underline{#1}}
    \par
    \setlength{\parindent}{0cm}
    \nopagebreak
    \leftskip=#2cm
    \rightskip=#3cm
}
{
    \par
}
\fi

\doendnotes{C}
\bigskip
\vfill

\clearpage

\footnotesize

\ifkorrekturansicht
  \lohead{\textsc{register}}
\fi

% theindex-Environment neu definieren ohne reledmac
\makeatletter
\renewenvironment{theindex}{%
  \ifkorrekturansicht
    \section*{\indexname}%
  \else
    \subsubsection*{Index der erwähnten Entitäten}%
  \fi
  \setlength{\parindent}{0pt}%
  \setlength{\parskip}{0pt plus 0.3pt}%
  \let\item\@idxitem
}{%
  \ifkorrekturansicht\clearpage\fi
}
\makeatother

\IfFileExists{\jobname-pw.ind}{\input{\jobname-pw.ind}}{}

% Quellenangabe nur in der Leseansicht
\ifkorrekturansicht\else
% Fallback-Definitionen, falls die .tex-Datei \titel etc. nicht gesetzt hat
\providecommand{\titel}{}
\providecommand{\editorInnen}{}
\providecommand{\dateiname}{\jobname}

\vspace{3cm}

\vfill

\footnotesize
\textsc{Quelle}: \titel. Herausgegeben von {\editorInnen}. In: \emph{Arthur Schnitzler: Briefwechsel mit Autorinnen und Autoren}.
 Digitale Edition, https://schnitzler-briefe.acdh.oeaw.ac.at/{\dateiname}.html (Stand \today)
\fi

\end{document}


