%% latex-leseansicht-vorspann.tex
%% Vorspann für die Leseansicht.
%% Lädt die gemeinsame Datei latex-vorspann.tex mit nicht gesetztem Schalter.

\newif\ifkorrekturansicht
\korrekturansichtfalse

\input{../tex-inputs/latex-vorspann}


\section[Frank Wedekind an Arthur Schnitzler, 24. 12. 1909]{L01909 Frank Wedekind an Arthur Schnitzler, 24. 12. 1909}
\nopagebreak\mylabel{L01909v}
\rehead{ }\normalsize\beginnumbering\briefempfaengerindex{Schnitzler, Arthur@\textsc{Schnitzler, Arthur}!zzzWedekind, Frank@\emph{von Frank Wedekind}!1909-12-242@{24. 12. 1909}|(be}
\toendnotes[C]{\smallbreak\pagebreak[2]}
\correspDesc{Versand  durch Frank Wedekind am 24. 12. 1909 in München
\newline{}Erhalt  durch Arthur Schnitzler im Zeitraum [25. 12. 1909 – 29. 12. 1909?] in Wien}\toendnotes[C]{\smallbreak}
\Standort{CUL, Schnitzler, B 111.}
\physDesc{Brief, 1 Blatt, 4 Seiten, 1490 Zeichen
\newline{}Handschrift: schwarze Tinte, deutsche Kurrent
\newline{}Schnitzler: mit Bleistift beschriftet: »\textsc{Wedekind}« }
\buchAbdrucke{\weitereDrucke{\emph{Frank Wedekinds Korrespondenz digital}. (7. 10. 2024) \url{https://briefedition.wedekind.h-da.de/view/document/single.xhtml?contentType=1&documentId=1551}.} }\toendnotes[C]{\smallbreak}
\pstart{}{\pb}Sehr verehrter Herr Doctor!\pend\vspace{0.5em}
\pstart
           Darf ich Sie aufrichtig und herzlich bitten, es nur nicht als Theilnahmsloſigkeit
               auszulegen, daß wir nicht zu Ihnen kamen. Am Tage als wir zu{ }ſpielen aufhörten, bekam
               meine Frau\pwindex{Wedekind, Tilly 11.\,4.\,1886 Graz – 20.\,4.\,1970 München@\textsc{Wedekind, Tilly} (11.\,4.\,1886 Graz – 20.\,4.\,1970 München), \emph{Schauspielerin}|pwv} die Nachricht, daß
               unſere Kleine\pwindex{Wedekind, Pamela 12.\,12.\,1906 Berlin – 9.\,4.\,1986 Ambach@\textsc{Wedekind, Pamela} (12.\,12.\,1906 Berlin – 9.\,4.\,1986 Ambach), \emph{Schauspielerin, Übersetzerin}|pwv}, die in Graz\oindex{Graz@\textbf{Graz}, \emph{Verwaltungsgebiet}|pw} war, arg erkältet{ }ſei. {\pb}Meine Frau\pwindex{Wedekind, Tilly 11.\,4.\,1886 Graz – 20.\,4.\,1970 München@\textsc{Wedekind, Tilly} (11.\,4.\,1886 Graz – 20.\,4.\,1970 München), \emph{Schauspielerin}|pwv} reiſte Hals über Kopf ohne{ }ſich einen Augenblick Ruhe
               zu gönnen hin, um \label{T_L01909-1v}\edtext{ſie\pwindex{Wedekind, Pamela 12.\,12.\,1906 Berlin – 9.\,4.\,1986 Ambach@\textsc{Wedekind, Pamela} (12.\,12.\,1906 Berlin – 9.\,4.\,1986 Ambach), \emph{Schauspielerin, Übersetzerin}|pwv}}{\lemma{\textnormal{\emph{sie}}}\Cendnote{\textnormal{Wedekind\pwindex{Wedekind, Frank 24.\,7.\,1864 Hannover – 9.\,3.\,1918 München@\textsc{Wedekind, Frank} (24.\,7.\,1864 Hannover – 9.\,3.\,1918 München), \emph{Schriftsteller, Schauspieler, Schriftsteller}|pwk}{ }schreibt: »Sie«.}}}\label{T_L01909-1} zu holen
               und als{ }ſie mit ihr nach Wien\oindex{Wien@\textbf{Wien}, \emph{Verwaltungsgebiet}|pw} kam fand ich es für
               dringend geboten, ohne Aufenthalt nach Hauſe\oindex{München@\textbf{München}|pwv} zurückzukehren. Am Dienſtag hoffte ich Sie
               wenigſtens allein noch aufſuchen zu können, aber auch dazu fehlte mir buchſtäblich
               die Zeit. So muß ich Ihnen meinen herzlichen Dank für die liebenswürdige
               Aufmerkſamkeit {\pb}die Sie für meine Arbeit
               übrig hatten, nun{ }ſchriftlich ausſprechen. Dieſe Gelegenheit kann ich aber nicht
               vorbeiziehen laſſen ohne Ihnen zu{ }ſagen, daß ich Ihnen die reichſten, künſtleriſch
               höchſten Genüſſe verdanke, die uns die deutſche Sprache{ }ſeit zwanzig Jahren bietet,
               und daß ich für viele Ihrer Werke die bedingungsloſe Verehrung fühle, die ich{ }ſonſt
               nur für Vergangenes aufbringen kann. So weit ich weiß kennen wir uns{ }ſeit \label{K_L01909-1v}\edtext{bald zehn Jahren}{\lemma{\textnormal{\emph{bald zehn Jahren}}}\Cendnote{\textnormal{Vgl. A. S.: \emph{Tagebuch}, 16. 11. 1901.
               }}}\label{K_L01909-1} und haben uns in dieſen zehn Jahren {\pb}\label{K_L01909-2v}\edtext{zwei mal geſehen}{\lemma{\textnormal{\emph{zwei mal gesehen}}}\Cendnote{\textnormal{Siehe A. S.: \emph{Tagebuch}, 1. 5. 1907, 15. 9. 1909.
               }}}\label{K_L01909-2}. Sie werden es mir daher nicht verdenken, daß ich die Gelegenheit wahrneme,
               Ihnen mein Herz auszuſchütten. An mir{ }ſoll es doch gewiß nicht liegen, daß wir uns
               nicht öfter begegnen.\pend
           
\pstart
           Wollen Sie bitte Ihrer verehrten Frau Gemahlin\pwindex{Schnitzler, Olga 17.\,1.\,1882 Wien – 13.\,1.\,1970 Lugano@\textsc{Schnitzler, Olga} (17.\,1.\,1882 Wien – 13.\,1.\,1970 Lugano), \emph{Schauspielerin, Sängerin}|pwv} meiner Frau\pwindex{Wedekind, Tilly 11.\,4.\,1886 Graz – 20.\,4.\,1970 München@\textsc{Wedekind, Tilly} (11.\,4.\,1886 Graz – 20.\,4.\,1970 München), \emph{Schauspielerin}|pwv} und meine ergebenſten Empfehlungen ausſprechen.\pend
           
\pstart
           Ihr ergebener{\\[\baselineskip]}\spacefill\mbox{FrankWedekind.}\pend
           \leftskip=0em{}
\pstart
           Heiliger Abend 1909.\pend
           \selectlanguage{ngerman}\endnumbering\briefempfaengerindex{Schnitzler, Arthur@\textsc{Schnitzler, Arthur}!zzzWedekind, Frank@\emph{von Frank Wedekind}!1909-12-242@{24. 12. 1909}|)be}\mylabel{L01909h}  \newcommand{\dateiname}{L01909}\newcommand{\titel}{Frank Wedekind an Arthur Schnitzler, 24. 12. 1909}\newcommand{\editorInnen}{Martin Anton Müller und Gerd-Hermann Susen}%% latex-leseansicht-abspann.tex
%% Abspann für die Leseansicht.
%% Der Schalter \ifkorrekturansicht ist bereits durch den Vorspann gesetzt.

%% latex-abspann.tex
%% Gemeinsamer Abspann für Korrekturansicht und Leseansicht.
%% Setzt den Schalter \ifkorrekturansicht voraus (gesetzt in den
%% einbindenden Dateien latex-korrekturansicht-abspann.tex bzw.
%% latex-leseansicht-abspann.tex).
%% ---------------------------------------------------------------

\normalsize

% Das esempio-Environment wird nur in der Leseansicht benötigt
\ifkorrekturansicht\else
\newenvironment{esempio}[3]%
{
    \vspace{1.5ex}
    \rlap{\underline{#1}}
    \par
    \setlength{\parindent}{0cm}
    \nopagebreak
    \leftskip=#2cm
    \rightskip=#3cm
}
{
    \par
}
\fi

\doendnotes{C}
\bigskip
\vfill

\clearpage

\footnotesize

\ifkorrekturansicht
  \lohead{\textsc{register}}
\fi

% theindex-Environment neu definieren ohne reledmac
\makeatletter
\renewenvironment{theindex}{%
  \ifkorrekturansicht
    \section*{\indexname}%
  \else
    \subsubsection*{Index der erwähnten Entitäten}%
  \fi
  \setlength{\parindent}{0pt}%
  \setlength{\parskip}{0pt plus 0.3pt}%
  \let\item\@idxitem
}{%
  \ifkorrekturansicht\clearpage\fi
}
\makeatother

\IfFileExists{\jobname-pw.ind}{\input{\jobname-pw.ind}}{}

% Quellenangabe nur in der Leseansicht
\ifkorrekturansicht\else
% Fallback-Definitionen, falls die .tex-Datei \titel etc. nicht gesetzt hat
\providecommand{\titel}{}
\providecommand{\editorInnen}{}
\providecommand{\dateiname}{\jobname}

\vspace{3cm}

\vfill

\footnotesize
\textsc{Quelle}: \titel. Herausgegeben von {\editorInnen}. In: \emph{Arthur Schnitzler: Briefwechsel mit Autorinnen und Autoren}.
 Digitale Edition, https://schnitzler-briefe.acdh.oeaw.ac.at/{\dateiname}.html (Stand \today)
\fi

\end{document}


