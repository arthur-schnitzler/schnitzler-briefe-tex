%% latex-leseansicht-vorspann.tex
%% Vorspann für die Leseansicht.
%% Lädt die gemeinsame Datei latex-vorspann.tex mit nicht gesetztem Schalter.

\newif\ifkorrekturansicht
\korrekturansichtfalse

\input{../tex-inputs/latex-vorspann}


         
         \newcommand{\erwaehntePersonen}{Personen: Olga Schnitzler, Tilly Wedekind, Pamela Wedekind}
         \newcommand{\erwaehnteOrte}{Orte: Graz, München, Wien}
         \newcommand{\erwaehnteWerke}{
               \section[Frank Wedekind an Arthur Schnitzler, 24. 12. 1909]{ Frank Wedekind an Arthur Schnitzler, 24. 12. 1909}\nopagebreak\mylabel{v}\rehead{ }\begin{ledgroupsized}[t]{13cm}\normalsize\beginnumbering \toendnotes[C]{\smallbreak\pagebreak[2]} \Standort{CUL, Schnitzler, B 111.}
\physDesc{Brief, 1 Blatt, 4 Seiten
\newline{}Handschrift: schwarze Tinte, deutsche Kurrent
\newline{}Schnitzler: mit Bleistift beschriftet: »\textsc{Wedekind}« }\toendnotes[C]{\smallbreak}\pstart{}{\pb}Sehr verehrter Herr Doctor!\pend\pstart
           Darf ich Sie aufrichtig und herzlich bitten, es nur nicht als Theilnahmsloſigkeit
               auszulegen, daß wir nicht zu Ihnen kamen. Am Tage als wir zu ſpielen aufhörten, bekam
               meine Frau\pwindex{Wedekind, Tilly 11.04.1886 – 20.04.1970@\textsc{Wedekind, Tilly} (11.04.1886 – 20.04.1970), \emph{Schauspielerin}|pwv} die Nachricht, daß
               unſere Kleine\pwindex{Wedekind, Pamela 12.12.1906 – 09.04.1986@\textsc{Wedekind, Pamela} (12.12.1906 – 09.04.1986), \emph{Schauspielerin, Übersetzerin}|pwv}, die in Graz\oindex{Graz@\textbf{Graz}|pw} war, arg erkältet ſei. {\pb}Meine Frau\pwindex{Wedekind, Tilly 11.04.1886 – 20.04.1970@\textsc{Wedekind, Tilly} (11.04.1886 – 20.04.1970), \emph{Schauspielerin}|pwv} reiſte Hals über Kopf ohne ſich einen Augenblick Ruhe
               zu gönnen hin, um \label{T_L01909_1v}\edtext{ſie\pwindex{Wedekind, Pamela 12.12.1906 – 09.04.1986@\textsc{Wedekind, Pamela} (12.12.1906 – 09.04.1986), \emph{Schauspielerin, Übersetzerin}|pwv}}{\lemma{\textnormal{\emph{ſie}}}\Cendnote{\textnormal{Wedekind\pwindex{Wedekind, Frank 24.07.1864 – 09.03.1918@\textsc{Wedekind, Frank} (24.07.1864 – 09.03.1918), \emph{Schriftsteller, Schauspieler}|pwk}{ }schreibt: »Sie«.}}}\label{T_L01909_1h} zu holen
               und als ſie mit ihr nach Wien\oindex{Wien@\textbf{Wien}|pw} kam fand ich es für
               dringend geboten, ohne Aufenthalt nach Hauſe\oindex{Muenchen@\textbf{München}|pwv} zurückzukehren. Am Dienſtag hoffte ich Sie
               wenigſtens allein noch aufſuchen zu können, aber auch dazu fehlte mir buchſtäblich
               die Zeit. So muß ich Ihnen meinen herzlichen Dank für die liebenswürdige
               Aufmerkſamkeit {\pb}die Sie für meine Arbeit
               übrig hatten, nun ſchriftlich ausſprechen. Dieſe Gelegenheit kann ich aber nicht
               vorbeiziehen laſſen ohne Ihnen zu ſagen, daß ich Ihnen die reichſten, künſtleriſch
               höchſten Genüſſe verdanke, die uns die deutſche Sprache ſeit zwanzig Jahren bietet,
               und daß ich für viele Ihrer Werke die bedingungsloſe Verehrung fühle, die ich ſonſt
               nur für Vergangenes aufbringen kann. So weit ich weiß kennen wir uns ſeit \label{K_L01909_1v}\edtext{bald zehn Jahren}{\lemma{\textnormal{\emph{bald zehn Jahren}}}\Cendnote{\textnormal{vgl. A. S.: \emph{Tagebuch}, 16. 11. 1901}}}\label{K_L01909_1h} und haben
               uns in dieſen zehn Jahren {\pb}\label{K_L01909_2v}\edtext{zwei mal geſehen}{\lemma{\textnormal{\emph{zwei mal geſehen}}}\Cendnote{\textnormal{siehe A. S.: \emph{Tagebuch}, 1. 5. 1907, siehe A. S.: \emph{Tagebuch}, 15. 9. 1909}}}\label{K_L01909_2h}. Sie werden es mir daher nicht
               verdenken, daß ich die Gelegenheit wahrneme, Ihnen mein Herz auszuſchütten. An mir
               ſoll es doch gewiß nicht liegen, daß wir uns nicht öfter begegnen.\pend
           \pstart
           Wollen Sie bitte Ihrer verehrten Frau
                  Gemahlin\pwindex{Schnitzler, Olga 17.01.1882 – 13.01.1970@\textsc{Schnitzler, Olga} (17.01.1882 – 13.01.1970), \emph{Schauspielerin, Sängerin}|pwv} meiner Frau\pwindex{Wedekind, Tilly 11.04.1886 – 20.04.1970@\textsc{Wedekind, Tilly} (11.04.1886 – 20.04.1970), \emph{Schauspielerin}|pwv} und
               meine ergebenſten Empfehlungen ausſprechen.\pend
           \pstart
           Ihr ergebener{\\[\baselineskip]}\spacefill\mbox{FrankWedekind.}\pend
           \leftskip=0em{}\pstart
           Heiliger Abend 1909.\pend
           
         
         \endnumbering\mylabel{h}\end{ledgroupsized}  \newcommand{\dateiname}{L01909}\newcommand{\titel}{Frank Wedekind an Arthur Schnitzler, 24. 12. 1909}\newcommand{\editorInnen}{Martin Anton Müller und Gerd-Hermann Susen}%% latex-leseansicht-abspann.tex
%% Abspann für die Leseansicht.
%% Der Schalter \ifkorrekturansicht ist bereits durch den Vorspann gesetzt.

%% latex-abspann.tex
%% Gemeinsamer Abspann für Korrekturansicht und Leseansicht.
%% Setzt den Schalter \ifkorrekturansicht voraus (gesetzt in den
%% einbindenden Dateien latex-korrekturansicht-abspann.tex bzw.
%% latex-leseansicht-abspann.tex).
%% ---------------------------------------------------------------

\normalsize

% Das esempio-Environment wird nur in der Leseansicht benötigt
\ifkorrekturansicht\else
\newenvironment{esempio}[3]%
{
    \vspace{1.5ex}
    \rlap{\underline{#1}}
    \par
    \setlength{\parindent}{0cm}
    \nopagebreak
    \leftskip=#2cm
    \rightskip=#3cm
}
{
    \par
}
\fi

\doendnotes{C}
\bigskip
\vfill

\clearpage

\footnotesize

\ifkorrekturansicht
  \lohead{\textsc{register}}
\fi

% theindex-Environment neu definieren ohne reledmac
\makeatletter
\renewenvironment{theindex}{%
  \ifkorrekturansicht
    \section*{\indexname}%
  \else
    \subsubsection*{Index der erwähnten Entitäten}%
  \fi
  \setlength{\parindent}{0pt}%
  \setlength{\parskip}{0pt plus 0.3pt}%
  \let\item\@idxitem
}{%
  \ifkorrekturansicht\clearpage\fi
}
\makeatother

\IfFileExists{\jobname-pw.ind}{\input{\jobname-pw.ind}}{}

% Quellenangabe nur in der Leseansicht
\ifkorrekturansicht\else
% Fallback-Definitionen, falls die .tex-Datei \titel etc. nicht gesetzt hat
\providecommand{\titel}{}
\providecommand{\editorInnen}{}
\providecommand{\dateiname}{\jobname}

\vspace{3cm}

\vfill

\footnotesize
\textsc{Quelle}: \titel. Herausgegeben von {\editorInnen}. In: \emph{Arthur Schnitzler: Briefwechsel mit Autorinnen und Autoren}.
 Digitale Edition, https://schnitzler-briefe.acdh.oeaw.ac.at/{\dateiname}.html (Stand \today)
\fi

\end{document}


      