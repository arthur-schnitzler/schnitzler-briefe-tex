%% latex-leseansicht-vorspann.tex
%% Vorspann für die Leseansicht.
%% Lädt die gemeinsame Datei latex-vorspann.tex mit nicht gesetztem Schalter.

\newif\ifkorrekturansicht
\korrekturansichtfalse

\input{../tex-inputs/latex-vorspann}


\section[Richard Beer-Hofmann an Arthur Schnitzler, 14. 10. 1892]{L00127 Richard Beer-Hofmann an Arthur Schnitzler, 14. 10. 1892}
\nopagebreak\mylabel{L00127v}
\rehead{ }\normalsize\beginnumbering\briefempfaengerindex{Schnitzler, Arthur@\textsc{Schnitzler, Arthur}!zzzBeer-Hofmann, Richard@\emph{von Richard Beer-Hofmann}!1892-10-141@{14. 10. 1892}|(be}
\toendnotes[C]{\smallbreak\pagebreak[2]}
\correspDesc{Versand  durch Richard Beer-Hofmann am 14. 10. 1892 in Wien
\newline{}Erhalt  durch Arthur Schnitzler im Zeitraum [14. 10. 1892 – 18. 10. 1892?] in Wien}\toendnotes[C]{\smallbreak}
\Standort{CUL, Schnitzler, B 8.}
\physDesc{Brief, 1 Blatt, 2 Seiten, 287 Zeichen
\newline{}Handschrift: blauer Buntstift, lateinische Kurrent
\newline{}Schnitzler: mit Bleistift datiert: »14/10 92« 
\newline{}Ordnung: mit Bleistift von unbekannter Hand nummeriert:
                                    »11« }
\buchAbdrucke{\weitereDrucke{Arthur Schnitzler, Richard Beer-Hofmann: \emph{Briefwechsel 1891–1931}. Herausgegeben von Konstanze Fliedl. Wien, Zürich: \emph{Europaverlag} 1992, S. 39–40.} }\toendnotes[C]{\smallbreak}
\pstart{}{\pb}Lieber Arthur!\pend\vspace{0.5em}
\pstart
           Ich bin seit gestern hier; Ich möchte heute zur »\label{K_L00127-1v}\edtext{Cameliendame\pwindex{\textcolor{red}{\textsuperscript{XXXX indx1}}!Dame aux camélias (théâtre)@\strich\emph{La Dame aux camélias (théâtre)}|pw}}{\lemma{\textnormal{\emph{Cameliendame}}}\Cendnote{\textnormal{Schnitzler dürfte das Gastspiel von Sarah Bernhardt\pwindex{Bernhardt, Sarah 22.\,10.\,1844 Paris – 26.\,3.\,1923 ebd.@\textsc{Bernhardt, Sarah} (22.\,10.\,1844 Paris – 26.\,3.\,1923 ebd.), \emph{Schauspielerin}|pwk} am Theater an der Wien\oindex{Wien@\textbf{Wien}!VI., Mariahilf@\textbf{VI., Mariahilf}!Theater an der Wien@\textbf{Theater an der Wien}, \emph{Theater}|pwk}
                  nicht besucht haben.}}}\label{K_L00127-1}« gehen; wenn es Ihnen möglich ist ko{\geminationm}en Sie so um \label{K_L00127-2v}\edtext{¼ 6}{\lemma{\textnormal{\emph{¼ 6}}}\Cendnote{\textnormal{17 Uhr 15}}}\label{K_L00127-2} zu mir und bringen mir dabei auch mein Opernglas mit.\pend
           
\pstart
           {\pb}Sie waren doch noch nicht
               dabei?\pend
           
\pstart
           Ich warte also bis ¼ 6.\pend
           
\pstart
           Herzlichst{\\[\baselineskip]}\spacefill\mbox{Richard}\pend
           \leftskip=0em{}
\pstart
           14/X 92\pend
           
\pstart
           Pardon für die zwei »dabei«.\pend
           \selectlanguage{ngerman}\endnumbering\briefempfaengerindex{Schnitzler, Arthur@\textsc{Schnitzler, Arthur}!zzzBeer-Hofmann, Richard@\emph{von Richard Beer-Hofmann}!1892-10-141@{14. 10. 1892}|)be}\mylabel{L00127h}  \newcommand{\dateiname}{L00127}\newcommand{\titel}{Richard Beer-Hofmann an Arthur Schnitzler, 14. 10. 1892}\newcommand{\editorInnen}{Martin Anton Müller und Gerd-Hermann Susen}%% latex-leseansicht-abspann.tex
%% Abspann für die Leseansicht.
%% Der Schalter \ifkorrekturansicht ist bereits durch den Vorspann gesetzt.

%% latex-abspann.tex
%% Gemeinsamer Abspann für Korrekturansicht und Leseansicht.
%% Setzt den Schalter \ifkorrekturansicht voraus (gesetzt in den
%% einbindenden Dateien latex-korrekturansicht-abspann.tex bzw.
%% latex-leseansicht-abspann.tex).
%% ---------------------------------------------------------------

\normalsize

% Das esempio-Environment wird nur in der Leseansicht benötigt
\ifkorrekturansicht\else
\newenvironment{esempio}[3]%
{
    \vspace{1.5ex}
    \rlap{\underline{#1}}
    \par
    \setlength{\parindent}{0cm}
    \nopagebreak
    \leftskip=#2cm
    \rightskip=#3cm
}
{
    \par
}
\fi

\doendnotes{C}
\bigskip
\vfill

\clearpage

\footnotesize

\ifkorrekturansicht
  \lohead{\textsc{register}}
\fi

% theindex-Environment neu definieren ohne reledmac
\makeatletter
\renewenvironment{theindex}{%
  \ifkorrekturansicht
    \section*{\indexname}%
  \else
    \subsubsection*{Index der erwähnten Entitäten}%
  \fi
  \setlength{\parindent}{0pt}%
  \setlength{\parskip}{0pt plus 0.3pt}%
  \let\item\@idxitem
}{%
  \ifkorrekturansicht\clearpage\fi
}
\makeatother

\IfFileExists{\jobname-pw.ind}{\input{\jobname-pw.ind}}{}

% Quellenangabe nur in der Leseansicht
\ifkorrekturansicht\else
% Fallback-Definitionen, falls die .tex-Datei \titel etc. nicht gesetzt hat
\providecommand{\titel}{}
\providecommand{\editorInnen}{}
\providecommand{\dateiname}{\jobname}

\vspace{3cm}

\vfill

\footnotesize
\textsc{Quelle}: \titel. Herausgegeben von {\editorInnen}. In: \emph{Arthur Schnitzler: Briefwechsel mit Autorinnen und Autoren}.
 Digitale Edition, https://schnitzler-briefe.acdh.oeaw.ac.at/{\dateiname}.html (Stand \today)
\fi

\end{document}


