%% latex-leseansicht-vorspann.tex
%% Vorspann für die Leseansicht.
%% Lädt die gemeinsame Datei latex-vorspann.tex mit nicht gesetztem Schalter.

\newif\ifkorrekturansicht
\korrekturansichtfalse

\input{../tex-inputs/latex-vorspann}


         
         \renewcommand{\erwaehntePersonen}{Personen: Richard Beer-Hofmann, Paul Goldmann, Alfred Kerr}
         \renewcommand{\erwaehnteOrte}{Orte: Bad Aussee, Berghaus Moserboden, Berlin, Bludenz, Bormio, Hotel und Pension Rudolfshöhe (Leopold Petter), Innsbruck, Iseosee, Italien, Küblis, Pontresina, Salzburg, Schruns, Schweiz, Solda, Sulzfluh, Trafoi, Zell am See, Österreichischer Hof}
         \renewcommand{\erwaehnteWerke}{}
               \section[ Arthur Schnitzler an Paul Goldmann, 30. 7. 1900]{ Arthur Schnitzler an Paul Goldmann, 30. 7. 1900}\nopagebreak\mylabel{v}\rehead{ }\begin{ledgroupsized}[t]{13cm}\normalsize\beginnumbering\briefempfaengerindex{Goldmann, Paul@\textsc{Goldmann, Paul}!zzzSchnitzler, Arthur@\emph{von Arthur Schnitzler}!1900-07-301@{30. 7. 1900}|(be} \toendnotes[C]{\smallbreak\pagebreak[2]} \Standort{Berlin, Akademie der Künste, Alfred Kerr-Archiv, 2487.}
\physDesc{Brief, 1 Blatt, 4 Seiten, 928 Zeichen
\newline{}Handschrift: Bleistift, deutsche Kurrent
\newline{}Ordnung: mit Bleistift nummeriert: »69« }\toendnotes[C]{\smallbreak}\pstart
           \raggedleft{}{\pb}\textsc{Aussee\oindex{Bad Aussee@\textbf{Bad Aussee}|pw}}{ }30/7 900.\pend
           \pstart
           Mein lieber Paul, wir (Richard\pwindex{Beer-Hofmann, Richard 1866-07-11 – 1945-09-26@\textsc{Beer-Hofmann, Richard} (1866-07-11 – 1945-09-26), \emph{Schriftsteller}|pw} u. ich) haben heute
                  folgend\textcolor{gray}{e} Punkte feſtgestellt:\pend
           \pstart
           So{\geminationn}tag 12. Rendezvous \uline{Salzburg\oindex{Salzburg@\textbf{Salzburg}|pw}} (Oeſterr. Hof\oindex{Oesterreichischer Hof@\textbf{Österreichischer Hof}|pw}.)\pend
           \pstart
           13.{ }Salzburg\oindex{Salzburg@\textbf{Salzburg}|pw}.\pend
           \pstart
           14. Nach Zell am
                  See\oindex{Zell am See@\textbf{Zell am See}|pw}, Moſerboden\oindex{Berghaus Moserboden@\textbf{Berghaus Moserboden}|pw}\pend
           \pstart
           15. – Moſerboden\oindex{Berghaus Moserboden@\textbf{Berghaus Moserboden}|pw}
               – Zell am See\oindex{Zell am See@\textbf{Zell am See}|pw} – \uuline{Innsbruck}\oindex{Innsbruck@\textbf{Innsbruck}|pw}\pend
           \pstart
           16.{ }Innsbruck\oindex{Innsbruck@\textbf{Innsbruck}|pw} Bahn Bludenz\oindex{Bludenz@\textbf{Bludenz}|pw} Poſt Schruns\oindex{Schruns@\textbf{Schruns}|pw}.\pend
           \pstart
           {\pb}17., event. 18.{ }Schruns\oindex{Schruns@\textbf{Schruns}|pw}. –\pend
           \pstart
           19.{ }Schruns\oindex{Schruns@\textbf{Schruns}|pw} – Sulzfluh\oindex{Sulzfluh@\textbf{Sulzfluh}|pw}\pend
           \pstart
           20.\hspace*{3.5em}– Küblis\oindex{Kueblis@\textbf{Küblis}|pw}\pend
           \pstart
           21.{ }22. auf einen noch nicht beſti{\geminationm}ten Weg nach Pontreſina\oindex{Pontresina@\textbf{Pontresina}|pw}.\pend
           \pstart
           Von dort nach Übereinko{\geminationm}en \introOben{}1)\introOben{}
               entweder Bormio\oindex{Bormio@\textbf{Bormio}|pw}, \introOben{}a)\introOben{}
               von Bormio\oindex{Bormio@\textbf{Bormio}|pw} gegen Trafoi\oindex{Trafoi@\textbf{Trafoi}|pw}, Sulden\oindex{Solda@\textbf{Solda}|pw}{ }\textsc{etc} b) von Bormio\oindex{Bormio@\textbf{Bormio}|pw}
               nach dem \textsc{Lago d’Iseo\oindex{Iseosee@\textbf{Iseosee}|pw}} oder 2) von Pontreſina\oindex{Pontresina@\textbf{Pontresina}|pw} weſtlich in die
                  Schweiz\oindex{Schweiz@\textbf{Schweiz}|pw}{ }{\pb}oder 3) von Pontreſina\oindex{Pontresina@\textbf{Pontresina}|pw} nach Italien\oindex{Italien@\textbf{Italien}|pw}. –\pend
           \pstart
           Am ſchönſten wäre alſo, \label{K_L03176-1v}\edtext{we{\geminationn} Du am 12. od. 13. in Salzburg\oindex{Salzburg@\textbf{Salzburg}|pw}
                  einträfſt}{\lemma{\textnormal{\emph{wenn … einträfſt}}}\Cendnote{\textnormal{Siehe Paul Goldmann an Arthur Schnitzler, 16. 6. [1900].
               }}}\label{K_L03176-1h}; we{\geminationn} Dir das nicht möglich am 15. in Innsbruck\oindex{Innsbruck@\textbf{Innsbruck}|pw}. –\pend
           \pstart
           Vielleicht \label{K_L03176-2v}\edtext{ſendeſt Du dieſen Brief gleich an \textsc{Kerr\pwindex{Kerr, Alfred 25.12.1867 – 12.10.1948@\textsc{Kerr, Alfred} (25.12.1867 – 12.10.1948), \emph{Schriftsteller, Kritiker}|pw}}}{\lemma{\textnormal{\emph{ſendeſt … Kerr}}}\Cendnote{\textnormal{Das geschah, er ist im Nachlass Kerrs\pwindex{Kerr, Alfred 25.12.1867 – 12.10.1948@\textsc{Kerr, Alfred} (25.12.1867 – 12.10.1948), \emph{Schriftsteller, Kritiker}|pwk} überliefert.}}}\label{K_L03176-2h}, für den dasſelbe gilt. Ich weiſs nicht, wohin ich zu adreſſiren hätte.\pend
           \pstart
           {\pb}Ich fahre morgen nach
                  \uline{Iſchl, Rudolfshöhe}\oindex{Hotel und Pension Rudolfshoehe (Leopold Petter)@\textbf{Hotel und Pension Rudolfshöhe (Leopold Petter)}|pw}, wo mich \label{K_L03176-3v}\edtext{bis auf weiteres}{\lemma{\textnormal{\emph{bis auf weiteres}}}\Cendnote{\textnormal{Schnitzler\pwindex{Schnitzler, Arthur 15.05.1862 – 21.10.1931@\textsc{Schnitzler, Arthur} (15.05.1862 – 21.10.1931), \emph{Schriftsteller, Mediziner}|pwk} reiste am 10. 8. 1900 nach Salzburg\oindex{Salzburg@\textbf{Salzburg}|pwk} ab.}}}\label{K_L03176-3h} Nachrichten treffen.\pend
           \pstart
           Bitte um recht baldige Verſtändg \textsc{resp} Einverſtändnis.\pend
           \pstart
           Herzlichſt \textcolor{gray}{G}ruß{\\[\baselineskip]}Dein {\\[\baselineskip]}\spacefill\mbox{Arthur}\pend
           \leftskip=0em{}
         
         \endnumbering\mylabel{h}\end{ledgroupsized}  \newcommand{\dateiname}{L03176}\newcommand{\titel}{Arthur Schnitzler an Paul Goldmann, 30. 7. 1900}\newcommand{\editorInnen}{Martin Anton Müller und Laura Untner}%% latex-leseansicht-abspann.tex
%% Abspann für die Leseansicht.
%% Der Schalter \ifkorrekturansicht ist bereits durch den Vorspann gesetzt.

%% latex-abspann.tex
%% Gemeinsamer Abspann für Korrekturansicht und Leseansicht.
%% Setzt den Schalter \ifkorrekturansicht voraus (gesetzt in den
%% einbindenden Dateien latex-korrekturansicht-abspann.tex bzw.
%% latex-leseansicht-abspann.tex).
%% ---------------------------------------------------------------

\normalsize

% Das esempio-Environment wird nur in der Leseansicht benötigt
\ifkorrekturansicht\else
\newenvironment{esempio}[3]%
{
    \vspace{1.5ex}
    \rlap{\underline{#1}}
    \par
    \setlength{\parindent}{0cm}
    \nopagebreak
    \leftskip=#2cm
    \rightskip=#3cm
}
{
    \par
}
\fi

\doendnotes{C}
\bigskip
\vfill

\clearpage

\footnotesize

\ifkorrekturansicht
  \lohead{\textsc{register}}
\fi

% theindex-Environment neu definieren ohne reledmac
\makeatletter
\renewenvironment{theindex}{%
  \ifkorrekturansicht
    \section*{\indexname}%
  \else
    \subsubsection*{Index der erwähnten Entitäten}%
  \fi
  \setlength{\parindent}{0pt}%
  \setlength{\parskip}{0pt plus 0.3pt}%
  \let\item\@idxitem
}{%
  \ifkorrekturansicht\clearpage\fi
}
\makeatother

\IfFileExists{\jobname-pw.ind}{\input{\jobname-pw.ind}}{}

% Quellenangabe nur in der Leseansicht
\ifkorrekturansicht\else
% Fallback-Definitionen, falls die .tex-Datei \titel etc. nicht gesetzt hat
\providecommand{\titel}{}
\providecommand{\editorInnen}{}
\providecommand{\dateiname}{\jobname}

\vspace{3cm}

\vfill

\footnotesize
\textsc{Quelle}: \titel. Herausgegeben von {\editorInnen}. In: \emph{Arthur Schnitzler: Briefwechsel mit Autorinnen und Autoren}.
 Digitale Edition, https://schnitzler-briefe.acdh.oeaw.ac.at/{\dateiname}.html (Stand \today)
\fi

\end{document}


      