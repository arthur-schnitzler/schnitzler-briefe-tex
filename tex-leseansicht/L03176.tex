%% latex-korrekturansicht-vorspann.tex
%% Vorspann für die Korrekturansicht.
%% Lädt die gemeinsame Datei latex-vorspann.tex mit gesetztem Schalter.

\newif\ifkorrekturansicht
\korrekturansichttrue

\input{../tex-inputs/latex-vorspann}


\section[ Arthur Schnitzler an Paul Goldmann, 30. 7. 1900]{L03176 Arthur Schnitzler an Paul Goldmann, 30. 7. 1900}
\nopagebreak\mylabel{L03176v}
\rehead{ }\normalsize\beginnumbering\briefempfaengerindex{Goldmann, Paul@\textsc{Goldmann, Paul}!zzzSchnitzler, Arthur@\emph{von Arthur Schnitzler}!1900-07-301@{30. 7. 1900}|(be}
\toendnotes[C]{\smallbreak\pagebreak[2]}\Standort{Berlin, Akademie der Künste, Alfred Kerr-Archiv, 2487.}
\physDesc{Brief, 1 Blatt, 4 Seiten, 928 Zeichen
\newline{}Handschrift: Bleistift, deutsche Kurrent
\newline{}Ordnung: mit Bleistift nummeriert: »69« }\toendnotes[C]{\smallbreak}
\pstart
           \raggedleft{}{\pb}\textsc{Aussee\oindex{Bad Aussee@\textbf{Bad Aussee}, \emph{P.PPLA3}|pw}}{ }30/7 900.\pend
           \vspace{0.5em}
\pstart
           Mein lieber Paul, wir (Richard\pwindex{Beer-Hofmann, Richard 1866-07-11 – 1945-09-26@\textsc{Beer-Hofmann, Richard} (1866-07-11 – 1945-09-26), \emph{Schriftsteller/Schriftstellerin}|pw} u. ich) haben heute
                  folgend\textcolor{gray}{e} Punkte feſtgestellt:\pend
           
\pstart
           So{\geminationn}tag 12. Rendezvous \uline{Salzburg\oindex{Salzburg@\textbf{Salzburg}, \emph{A.ADM2}|pw}} (Oeſterr. Hof\oindex{Oesterreichischer Hof@\textbf{Österreichischer Hof}, \emph{Hotel (K.HTL)}|pw}.)\pend
           
\pstart
           13.{ }Salzburg\oindex{Salzburg@\textbf{Salzburg}, \emph{A.ADM2}|pw}.\pend
           
\pstart
           14. Nach Zell am
                  See\oindex{Zell am See@\textbf{Zell am See}, \emph{P.PPLA3}|pw}, Moſerboden\oindex{Berghaus Moserboden@\textbf{Berghaus Moserboden}, \emph{S.GHSE}|pw}\pend
           
\pstart
           15. – Moſerboden\oindex{Berghaus Moserboden@\textbf{Berghaus Moserboden}, \emph{S.GHSE}|pw}
               – Zell am See\oindex{Zell am See@\textbf{Zell am See}, \emph{P.PPLA3}|pw} – \uuline{Innsbruck}\oindex{Innsbruck@\textbf{Innsbruck}, \emph{A.ADM2}|pw}\pend
           
\pstart
           16.{ }Innsbruck\oindex{Innsbruck@\textbf{Innsbruck}, \emph{A.ADM2}|pw} Bahn Bludenz\oindex{Bludenz@\textbf{Bludenz}, \emph{P.PPLA2}|pw} Poſt Schruns\oindex{Schruns@\textbf{Schruns}, \emph{A.ADM3}|pw}.\pend
           
\pstart
           {\pb}17., event. 18.{ }Schruns\oindex{Schruns@\textbf{Schruns}, \emph{A.ADM3}|pw}. –\pend
           
\pstart
           19.{ }Schruns\oindex{Schruns@\textbf{Schruns}, \emph{A.ADM3}|pw} – Sulzfluh\oindex{Sulzfluh@\textbf{Sulzfluh}, \emph{T.PK}|pw}\pend
           
\pstart
           20.\hspace*{3.5em}– Küblis\oindex{Kueblis@\textbf{Küblis}, \emph{P.PPL}|pw}\pend
           
\pstart
           21.{ }22. auf einen noch nicht beſti{\geminationm}ten Weg nach Pontreſina\oindex{Pontresina@\textbf{Pontresina}, \emph{P.PPL}|pw}.\pend
           
\pstart
           Von dort nach Übereinko{\geminationm}en \introOben{}1)\introOben{}
               entweder Bormio\oindex{Bormio@\textbf{Bormio}, \emph{P.PPLA3}|pw}, \introOben{}a)\introOben{}
               von Bormio\oindex{Bormio@\textbf{Bormio}, \emph{P.PPLA3}|pw} gegen Trafoi\oindex{Trafoi@\textbf{Trafoi}, \emph{P.PPL}|pw}, Sulden\oindex{Solda@\textbf{Solda}, \emph{P.PPL}|pw}{ }\textsc{etc} b) von Bormio\oindex{Bormio@\textbf{Bormio}, \emph{P.PPLA3}|pw}
               nach dem \textsc{Lago d’Iseo\oindex{Iseosee@\textbf{Iseosee}, \emph{H.LK}|pw}} oder 2) von Pontreſina\oindex{Pontresina@\textbf{Pontresina}, \emph{P.PPL}|pw} weſtlich in die
                  Schweiz\oindex{Schweiz@\textbf{Schweiz}, \emph{A.PCLI}|pw}{ }{\pb}oder 3) von Pontreſina\oindex{Pontresina@\textbf{Pontresina}, \emph{P.PPL}|pw} nach Italien\oindex{Italien@\textbf{Italien}, \emph{A.PCLI}|pw}. –\pend
           
\pstart
           Am ſchönſten wäre alſo, \label{K_L03176-1v}\edtext{we{\geminationn} Du am 12. od. 13. in Salzburg\oindex{Salzburg@\textbf{Salzburg}, \emph{A.ADM2}|pw}
                  einträfſt}{\lemma{\textnormal{\emph{wenn … einträfſt}}}\Cendnote{\textnormal{Siehe Paul Goldmann an Arthur Schnitzler, 16. 6. [1900].
               }}}\label{K_L03176-1}; we{\geminationn} Dir das nicht möglich am 15. in Innsbruck\oindex{Innsbruck@\textbf{Innsbruck}, \emph{A.ADM2}|pw}. –\pend
           
\pstart
           Vielleicht \label{K_L03176-2v}\edtext{ſendeſt Du dieſen Brief gleich an \textsc{Kerr\pwindex{Kerr, Alfred 25.12.1867 – 12.10.1948@\textsc{Kerr, Alfred} (25.12.1867 – 12.10.1948), \emph{Schriftsteller/Schriftstellerin, Kritiker/Kritikerin}|pw}}}{\lemma{\textnormal{\emph{ſendeſt … Kerr}}}\Cendnote{\textnormal{Das geschah, er ist im Nachlass Kerrs\pwindex{Kerr, Alfred 25.12.1867 – 12.10.1948@\textsc{Kerr, Alfred} (25.12.1867 – 12.10.1948), \emph{Schriftsteller/Schriftstellerin, Kritiker/Kritikerin}|pwk} überliefert.}}}\label{K_L03176-2}, für den dasſelbe gilt. Ich weiſs nicht, wohin ich zu adreſſiren hätte.\pend
           
\pstart
           {\pb}Ich fahre morgen nach
                  \uline{Iſchl, Rudolfshöhe}\oindex{Hotel und Pension Rudolfshoehe (Leopold Petter)@\textbf{Hotel und Pension Rudolfshöhe (Leopold Petter)}, \emph{Hotel (K.HTL)}|pw}, wo mich \label{K_L03176-3v}\edtext{bis auf weiteres}{\lemma{\textnormal{\emph{bis auf weiteres}}}\Cendnote{\textnormal{Schnitzler reiste am 10. 8. 1900 nach Salzburg\oindex{Salzburg@\textbf{Salzburg}, \emph{A.ADM2}|pwk} ab.}}}\label{K_L03176-3} Nachrichten treffen.\pend
           
\pstart
           Bitte um recht baldige Verſtändg \textsc{resp} Einverſtändnis.\pend
           
\pstart
           Herzlichſt \textcolor{gray}{G}ruß{\\[\baselineskip]}Dein {\\[\baselineskip]}\spacefill\mbox{Arthur}\pend
           \leftskip=0em{}\selectlanguage{ngerman}\endnumbering\briefempfaengerindex{Goldmann, Paul@\textsc{Goldmann, Paul}!zzzSchnitzler, Arthur@\emph{von Arthur Schnitzler}!1900-07-301@{30. 7. 1900}|)be}\mylabel{L03176h}  \normalsize

\doendnotes{C}
\bigskip
\vfill

\clearpage

\footnotesize

\lohead{\textsc{register}}

% Definiere theindex-Environment komplett neu ohne reledmac
\makeatletter
\renewenvironment{theindex}{%
  \section*{\indexname}%
  \setlength{\parindent}{0pt}%
  \setlength{\parskip}{0pt plus 0.3pt}%
  \let\item\@idxitem
}{%
  \clearpage
}
\makeatother

\IfFileExists{\jobname-pw.ind}{\input{\jobname-pw.ind}}{}

\end{document}

      