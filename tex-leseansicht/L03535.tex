%% latex-leseansicht-vorspann.tex
%% Vorspann für die Leseansicht.
%% Lädt die gemeinsame Datei latex-vorspann.tex mit nicht gesetztem Schalter.

\newif\ifkorrekturansicht
\korrekturansichtfalse

\input{../tex-inputs/latex-vorspann}


\section[ Paul Goldmann an Olga Gussmann, 15. 11. {[}1901{]}]{L03535 Paul Goldmann an Olga Gussmann,  15. 11. [1901]}
\nopagebreak\mylabel{L03535v}
\rehead{ }\normalsize\beginnumbering\briefempfaengerindex{Schnitzler, Olga@\textsc{Schnitzler, Olga}!zzzGoldmann, Paul@\emph{von Paul Goldmann}!1901-11-151@{15. 11. [1901]}|(be}
\toendnotes[C]{\smallbreak\pagebreak[2]}
\correspDesc{Versand  durch Paul Goldmann am 15. 11. [1901] in Berlin
\newline{}Erhalt  durch Olga Gussmann im Zeitraum [16. 11. 1901 – 20. 11. 1901?] in Wien}\toendnotes[C]{\smallbreak}
\Standort{DLA, A:Schnitzler, HS.NZ85.1.5247.}
\physDesc{Brief, 1 Blatt, 4 Seiten, 2212 Zeichen
\newline{}Handschrift: blaue Tinte, deutsche Kurrent}\toendnotes[C]{\smallbreak}
\pstart
           \raggedleft{}{\pb}\textcolor{gray}{\textbf{DESSAUERSTRASSE 19\oindex{Dessauer Straße@\textbf{Dessauer Straße}, \emph{Straße}|pw}}}\pend
           
\pstart
           Berlin\oindex{Berlin@\textbf{Berlin}, \emph{Hauptstadt}|pw}, 15. November.\pend
           
\pstart{}Liebes Fräulein \textsc{Olga},\pend\vspace{0.5em}
\pstart
           Ich danke Ihnen für Ihren lieben Brief und freue mich, daß Sie und \textsc{Arthur} ein paar \label{K_L03535-1v}\edtext{frohe und friedliche
                  Tage}{\lemma{\textnormal{\emph{frohe … Tage}}}\Cendnote{\textnormal{Schnitzler und Olga Gussmann\pwindex{Schnitzler, Olga 17.\,1.\,1882 Wien – 13.\,1.\,1970 Lugano@\textsc{Schnitzler, Olga} (17.\,1.\,1882 Wien – 13.\,1.\,1970 Lugano), \emph{Schauspielerin, Sängerin}|pwk} waren erst am Vortag, dem 14. 11. 1901, aus Payerbach\oindex{Hotel Edlacherhof@\textbf{Hotel Edlacherhof}, \emph{Hotel}|pwk} nach Wien\oindex{Wien@\textbf{Wien}, \emph{Verwaltungsgebiet}|pwk} zurückgekehrt, wo sie vier Tage verlebt hatten.}}}\label{K_L03535-1} haben
               verleben können. Ihre Schilderungen{ }ſind{ }ſehr eindrucksvoll, und an Ihren Worten iſt
               ein Schimmer von Glück haften geblieben.\pend
           
\pstart
           Ihr Brief erfordert eine ausführliche Beantwortung, und{ }ſie{ }ſoll Ihnen werden,{ }ſobald
               die Arbeit mir ein wenig Luft läßt.\pend
           
\pstart
           Eines aber muß ich mir gleich von der Seele{ }ſchreiben. Ich danke Ihnen für {\pb}die Offenheit, mit der Sie zu mir über meine
               Feuilletons{ }ſprechen, und werde Ihnen mit derſelben Offenheit antworten. Und da muß
               ich Ihnen{ }ſagen, daß Ihre Äußerungen mich außerordentlich geſchmerzt, – daß{ }ſie mich
               in einem Punkte getroffen haben, \strikeout{\textcolor{gray}{wo}} an dem ich überaus empfindlich bin. Oder, um es etwas weniger{ }ſentimental
               auszudrücken: Ich bin \strikeout{\textcolor{gray}{×}\-\textcolor{gray}{×}\-\textcolor{gray}{×}\-\textcolor{gray}{×}\-\textcolor{gray}{×}} verblüfft, von Ihnen{ }ſo ganz und gar nicht verſtanden zu werden. Ich bin
               verblüfft, daß Sie nicht begreifen, wieviel ehrliche Kunſtbegeiſterung, welch’ heißes
               Wahrheitsſtreben in meinen \label{K_L03535-2v}\edtext{Kritiken\pwindex{Hauptmann, Gerhart 15.\,11.\,1862 Szczawno-Zdrój – 6.\,6.\,1946 Jagniątków@\textsc{Hauptmann, Gerhart} (15.\,11.\,1862 Szczawno-Zdrój – 6.\,6.\,1946 Jagniątków), \emph{Schriftsteller}!Einsame Menschen. Drama@\strich\emph{Einsame Menschen. Drama}|pwv}\pwindex{Goldmann, Paul 31.\,1.\,1865 Breslau – 25.\,9.\,1935 Wien@\textsc{Goldmann, Paul} (31.\,1.\,1865 Breslau – 25.\,9.\,1935 Wien), \emph{Schriftsteller, Journalist}!Berliner Brief. [»Schluck und Jau« von Gerhart Hauptmann am Deutschen Theater]@\strich\emph{Berliner Brief. [»Schluck und Jau« von Gerhart Hauptmann am Deutschen Theater]}|pwv}\pwindex{Goldmann, Paul 31.\,1.\,1865 Breslau – 25.\,9.\,1935 Wien@\textsc{Goldmann, Paul} (31.\,1.\,1865 Breslau – 25.\,9.\,1935 Wien), \emph{Schriftsteller, Journalist}!Michael Kramer.«@\strich\emph{»Michael Kramer.«}|pwv}
               über \textsc{Hauptmann\pwindex{Hauptmann, Gerhart 15.\,11.\,1862 Szczawno-Zdrój – 6.\,6.\,1946 Jagniątków@\textsc{Hauptmann, Gerhart} (15.\,11.\,1862 Szczawno-Zdrój – 6.\,6.\,1946 Jagniątków), \emph{Schriftsteller}|pw}}}{\lemma{\textnormal{\emph{Kritiken
               über Hauptmann}}}\Cendnote{\textnormal{Der unmittelbare Auslöser der
                  Auseinandersetzung war diese Rezension: Paul Goldmann\pwindex{Goldmann, Paul 31.\,1.\,1865 Breslau – 25.\,9.\,1935 Wien@\textsc{Goldmann, Paul} (31.\,1.\,1865 Breslau – 25.\,9.\,1935 Wien), \emph{Schriftsteller, Journalist}|pwk}: \emph{Berliner Theater. »Einsame Menschen« im Deutschen Theater}\pwindex{Goldmann, Paul 31.\,1.\,1865 Breslau – 25.\,9.\,1935 Wien@\textsc{Goldmann, Paul} (31.\,1.\,1865 Breslau – 25.\,9.\,1935 Wien), \emph{Schriftsteller, Journalist}!Berliner Theater. »Einsame Menschen« im Deutschen Theater@\strich\emph{Berliner Theater. »Einsame Menschen« im Deutschen Theater}|pwk}.
                     In: \emph{Neue Freie Presse}\pwindex{Neue Freie Presse@\emph{Neue Freie Presse}|pwk}, Nr. 13.345, 19. 10. 1901, Morgenblatt, S. 1–3. Dabei
                  dürften auch frühere Feuilletons thematisiert worden sein: Paul Goldmann\pwindex{Goldmann, Paul 31.\,1.\,1865 Breslau – 25.\,9.\,1935 Wien@\textsc{Goldmann, Paul} (31.\,1.\,1865 Breslau – 25.\,9.\,1935 Wien), \emph{Schriftsteller, Journalist}|pwk}: \emph{Berliner Brief}\pwindex{Goldmann, Paul 31.\,1.\,1865 Breslau – 25.\,9.\,1935 Wien@\textsc{Goldmann, Paul} (31.\,1.\,1865 Breslau – 25.\,9.\,1935 Wien), \emph{Schriftsteller, Journalist}!Berliner Brief. [»Schluck und Jau« von Gerhart Hauptmann am Deutschen Theater]@\strich\emph{Berliner Brief. [»Schluck und Jau« von Gerhart Hauptmann am Deutschen Theater]}|pwk}. In: \emph{Neue Freie Presse}\pwindex{Neue Freie Presse@\emph{Neue Freie Presse}|pwk}, Nr. 12.735, 6. 2. 1900, Morgenblatt, S. 1–3. Paul Goldmann\pwindex{Goldmann, Paul 31.\,1.\,1865 Breslau – 25.\,9.\,1935 Wien@\textsc{Goldmann, Paul} (31.\,1.\,1865 Breslau – 25.\,9.\,1935 Wien), \emph{Schriftsteller, Journalist}|pwk}: \emph{»Michael Kramer«}\pwindex{Goldmann, Paul 31.\,1.\,1865 Breslau – 25.\,9.\,1935 Wien@\textsc{Goldmann, Paul} (31.\,1.\,1865 Breslau – 25.\,9.\,1935 Wien), \emph{Schriftsteller, Journalist}!Michael Kramer.«@\strich\emph{»Michael Kramer.«}|pwk}. In: \emph{Neue Freie Presse}\pwindex{Neue Freie Presse@\emph{Neue Freie Presse}|pwk}, Nr. 13.055, 28. 12. 1900, Morgenblatt, S. 1–3. Siehe auch XXXX Auszeichnungsfehler: Dokument L03090 nicht gefunden, XXXX Auszeichnungsfehler: Dokument L03091 nicht gefunden und XXXX Auszeichnungsfehler: Dokument L03092 nicht gefunden.}}}\label{K_L03535-2}{ }ſich {\pb}ausdrückt. Ich bin verblüfft, daß Sie in einem
               Falle, wo Ihre und meine Meinung{ }ſich gegenüberſtehen, nicht einen Augenblick \substVorne{}\textsuperscript{\textcolor{gray}{den} Fall}\substDazwischen{}die Frage\substHinten{} in Erwägung ziehen, ob nicht vielleicht Sie im Unrecht{ }ſind, und daß Sie
               ohneweiters eine Auslegung{ }ſich zurechtmachen, die mich (ich kann es nicht anders{ }ſagen) in meiner \strikeout{\textcolor{gray}{kr}itiſch} Ehre als Kritiker trifft. Denn ich würde es für
               unehrenhaft halten, wenn ich, wie Sie meinen, in meinem Kampf gegen \textsc{Hauptmann\pwindex{Hauptmann, Gerhart 15.\,11.\,1862 Szczawno-Zdrój – 6.\,6.\,1946 Jagniątków@\textsc{Hauptmann, Gerhart} (15.\,11.\,1862 Szczawno-Zdrój – 6.\,6.\,1946 Jagniątków), \emph{Schriftsteller}|pw}} mich auch nur im Mindeſten durch perſönliche Motive leiten ließe. Wenn Sie
               meine Angriffe gegen \textsc{Hauptmann\pwindex{Hauptmann, Gerhart 15.\,11.\,1862 Szczawno-Zdrój – 6.\,6.\,1946 Jagniątków@\textsc{Hauptmann, Gerhart} (15.\,11.\,1862 Szczawno-Zdrój – 6.\,6.\,1946 Jagniątków), \emph{Schriftsteller}|pw}} perſönlich {\pb}finden,{ }ſo wiſſen Sie wohl nicht,
               was perſönliche Angriffe{ }ſind. Meine Einwendungen{ }ſind einer abſolut{ }ſachlichen Art;
               und wenn{ }ſie im heftigen Tone vorgebracht werden,{ }ſo kommt dieſer Ton von meinem
               Temperament, –{ }ſo kommt er von der Erbitterung her, die mich erfüllt, einen{ }ſo
               minderwerthigen Geiſt, wie \textsc{Gerhart Hauptmann\pwindex{Hauptmann, Gerhart 15.\,11.\,1862 Szczawno-Zdrój – 6.\,6.\,1946 Jagniątków@\textsc{Hauptmann, Gerhart} (15.\,11.\,1862 Szczawno-Zdrój – 6.\,6.\,1946 Jagniątków), \emph{Schriftsteller}|pw}}, zum großen Dichter erhoben zu{ }ſehen. Und daß Sie mir dieſe Erbitterung nicht
               glauben wollen, daß Sie nach perſönlichen Motiven{ }ſuchen, – Sie, eine Freundin, – das
               hat mich verblüfft, das hat mich{ }ſchwer gekränkt{\dotsfive}\pend
           \pstart Grüßen Sie, bitte, \textsc{Liesl\pwindex{Steinrück, Elisabeth 19.\,11.\,1885 – 7.\,4.\,1920 Partenkirchen@\textsc{Steinrück, Elisabeth} (19.\,11.\,1885 – 7.\,4.\,1920 Partenkirchen)|pw}}; und{ }ſeien Sie{ }ſammt \textsc{Arthur} herzlichſt gegrüßt von Ihrem \spacefill\mbox{Paul Goldmann}\pend{}\selectlanguage{ngerman}\endnumbering\briefempfaengerindex{Schnitzler, Olga@\textsc{Schnitzler, Olga}!zzzGoldmann, Paul@\emph{von Paul Goldmann}!1901-11-151@{15. 11. [1901]}|)be}\mylabel{L03535h}  \newcommand{\dateiname}{L03535}\newcommand{\titel}{Paul Goldmann an Olga Gussmann, 15. 11. [1901]}\newcommand{\editorInnen}{Martin Anton Müller und Laura Untner}%% latex-leseansicht-abspann.tex
%% Abspann für die Leseansicht.
%% Der Schalter \ifkorrekturansicht ist bereits durch den Vorspann gesetzt.

%% latex-abspann.tex
%% Gemeinsamer Abspann für Korrekturansicht und Leseansicht.
%% Setzt den Schalter \ifkorrekturansicht voraus (gesetzt in den
%% einbindenden Dateien latex-korrekturansicht-abspann.tex bzw.
%% latex-leseansicht-abspann.tex).
%% ---------------------------------------------------------------

\normalsize

% Das esempio-Environment wird nur in der Leseansicht benötigt
\ifkorrekturansicht\else
\newenvironment{esempio}[3]%
{
    \vspace{1.5ex}
    \rlap{\underline{#1}}
    \par
    \setlength{\parindent}{0cm}
    \nopagebreak
    \leftskip=#2cm
    \rightskip=#3cm
}
{
    \par
}
\fi

\doendnotes{C}
\bigskip
\vfill

\clearpage

\footnotesize

\ifkorrekturansicht
  \lohead{\textsc{register}}
\fi

% theindex-Environment neu definieren ohne reledmac
\makeatletter
\renewenvironment{theindex}{%
  \ifkorrekturansicht
    \section*{\indexname}%
  \else
    \subsubsection*{Index der erwähnten Entitäten}%
  \fi
  \setlength{\parindent}{0pt}%
  \setlength{\parskip}{0pt plus 0.3pt}%
  \let\item\@idxitem
}{%
  \ifkorrekturansicht\clearpage\fi
}
\makeatother

\IfFileExists{\jobname-pw.ind}{\input{\jobname-pw.ind}}{}

% Quellenangabe nur in der Leseansicht
\ifkorrekturansicht\else
% Fallback-Definitionen, falls die .tex-Datei \titel etc. nicht gesetzt hat
\providecommand{\titel}{}
\providecommand{\editorInnen}{}
\providecommand{\dateiname}{\jobname}

\vspace{3cm}

\vfill

\footnotesize
\textsc{Quelle}: \titel. Herausgegeben von {\editorInnen}. In: \emph{Arthur Schnitzler: Briefwechsel mit Autorinnen und Autoren}.
 Digitale Edition, https://schnitzler-briefe.acdh.oeaw.ac.at/{\dateiname}.html (Stand \today)
\fi

\end{document}


