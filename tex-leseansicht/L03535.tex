%% latex-leseansicht-vorspann.tex
%% Vorspann für die Leseansicht.
%% Lädt die gemeinsame Datei latex-vorspann.tex mit nicht gesetztem Schalter.

\newif\ifkorrekturansicht
\korrekturansichtfalse

\input{../tex-inputs/latex-vorspann}

\begin{center}
            \textcolor{red}{ENTWURF, NICHT FERTIG KORRIGIERT}
                      \end{center}
            
         
         \renewcommand{\erwaehntePersonen}{Personen: Gerhart Hauptmann, Olga Schnitzler, Elisabeth Steinrück}
         \renewcommand{\erwaehnteOrte}{Orte: Berlin, Wien}
         \renewcommand{\erwaehnteWerke}{}
               \section[ Paul Goldmann an Olga XXXX Gussmann/Schnitzler, 15. 11. {[}XXXX{]}]{ Paul Goldmann an Olga XXXX Gussmann/Schnitzler, 15. 11. {[}XXXX{]}}\nopagebreak\mylabel{v}\rehead{ }\begin{ledgroupsized}[t]{13cm}\normalsize\beginnumbering \toendnotes[C]{\smallbreak\pagebreak[2]} \Standort{DLA, A:Schnitzler, HS.1985.1.5247.}
\physDesc{,  Blätter,  Seiten
\newline{}Handschrift: , deutsche Kurrent}\toendnotes[C]{\smallbreak}\pstart
           \noindent{}{\pb}\pend
           \textcolor{gray}{\textbf{DESSAUERSTRASSE 19\oindex{XXXX Ortsangabe fehlt|pw}}}\textcolor{red}{\textsuperscript{\textbf{KEY}}}\pstart
           Berlin\oindex{Berlin@\textbf{Berlin}|pw}, 15. November.\pend
           \pstart{}Liebes Fräulein \textsc{Olga},\pend\pstart
           Ich danke Ihnen für Ihren lieben Brief und freue mich, daß Sie und \textsc{Arthur\pwindex{Schnitzler, Arthur 15.05.1862 – 21.10.1931@\textsc{Schnitzler, Arthur} (15.05.1862 – 21.10.1931), \emph{Schriftsteller, Mediziner}|pw}} ein paar frohe und friedliche Tage haben verleben können. Ihre Schilderungen
               ſind ſehr eindrucksvoll, und an Ihren Worten iſt ein Schimmer von Glück haften
               geblieben.\pend
           \pstart
           Ihr Brief erfordert eine ausführliche Beantwortung, und ſie ſoll Ihnen werden, ſobald
               die Arbeit mir ein wenig Luft läßt.\pend
           \pstart
           Eines aber muß ich mir gleich von der Seele ſchreiben. Ich danke Ihnen für {\pb}die Offenheit, mit der Sie zu
               mir über meine Feuilletons ſprechen, und werde Ihnen mit derſelben Offenheit
               antworten. Und da muß ich Ihnen ſagen, daß Ihre Äußerungen mich außerordentlich
               geſchmerzt, – daß ſie mich in einem Punkte getroffen haben, \strikeout{\textcolor{gray}{w}} an dem ich überaus empfindlich bin. Oder, um es etwas weniger ſentimental
               auszudeuten: Ich bin \strikeout{X} verblüfft, von Ihnen ſo ganz und gar nicht verſtanden zu werden. Ich
               bin verblüfft, daß Sie nicht begreifen, wieviel ehrliche Kunſtbegeiſterung, welch’
               heißes Wahrheitsſtreben in meinen Kritiken über \textsc{Hauptmann\pwindex{Hauptmann, Gerhart 15.11.1862 – 06.06.1946@\textsc{Hauptmann, Gerhart} (15.11.1862 – 06.06.1946), \emph{Schriftsteller}|pw}} ſich {\pb} ausdrückt. Ich bin
               verblüfft, daß Sie in einem Falle, wo Ihre und meine Meinung ſich gegenüberſtehen,
                  die Frage nicht einen Augenblick \strikeout{\textcolor{gray}{den} Fall} in Erwägung ziehen, ob nicht vielleicht Sie im
               Unrecht ſind, und daß Sie ohneweiters eine Auslegung ſich zurechtmachen, die mich
               (ich kann es nicht anders ſagen) in meiner \strikeout{\textcolor{gray}{kr}itiſch} Ehre als Kritiker trifft. Denn ich würde es für
               unehrenhaft halten, wenn ich, wie Sie meinen, in meinem Kampf gegen \textsc{Hauptmann\pwindex{Hauptmann, Gerhart 15.11.1862 – 06.06.1946@\textsc{Hauptmann, Gerhart} (15.11.1862 – 06.06.1946), \emph{Schriftsteller}|pw}} mich auch nur im Mindeſten durch perſönliche Motive leiten ließe. Wenn Sie
               meine Angriffe gegen \textsc{Hauptmann\pwindex{Hauptmann, Gerhart 15.11.1862 – 06.06.1946@\textsc{Hauptmann, Gerhart} (15.11.1862 – 06.06.1946), \emph{Schriftsteller}|pw}} perſönlich {\pb} finden, ſo
               wiſſen Sie wohl nicht, was perſönliche Angriffe ſind. Meine Einwendungen ſind einer
               abſolut ſachliche\label{XXXXv}\edtext{{[}n{]}[Kommentar: Goldmann\u0020schrieb\u0020sachlicher]}{\lemma{\textnormal{\emph{XXXX Lemmafehler}}}\Cendnote{\textnormal{}}}\label{XXXX} Art; und wenn ſie im heftigen Tone
               vorgebracht werden, ſo kommt dieſer Ton von meinem Temperament, – ſo kommt er von der
               Erbitterung her, die mich erfüllt, einen ſo minderwerthigen Geiſt, wie \textsc{Gerhart Hauptmann\textcolor{red}{\textsuperscript{\textbf{KEY}}}}, zum großen Dichter erhoben zu ſehen. Und daß Sie mir die Erbitterung nicht
               glauben wollen, daß Sie nach perſönlichen Motiven ſuchen, – Sie, eine Freundin, – das
               hat mich verblüfft, das hat mich ſchwer gekränkt{\dotsfive}{\\[\baselineskip]}Grüßen Sie, bitte, \textsc{Liesl\pwindex{Steinrueck, Elisabeth 19.11.1885 – 07.04.1920@\textsc{Steinrück, Elisabeth} (19.11.1885 – 07.04.1920)|pw}}; und ſeien Sie\pend
           \leftskip=0em{}\pstart
           {\\[\baselineskip]}ſammt \textsc{Arthur\pwindex{Schnitzler, Arthur 15.05.1862 – 21.10.1931@\textsc{Schnitzler, Arthur} (15.05.1862 – 21.10.1931), \emph{Schriftsteller, Mediziner}|pw}} herzlichſt gegrüßt von\pend
           \leftskip=0em{}\pstart
           {\\[\baselineskip]}Ihrem \spacefill\mbox{Paul Goldmann}\pend
           \leftskip=0em{}
         
         \endnumbering\mylabel{h}\end{ledgroupsized}\begin{anhang}\end{anhang}\newcommand{\dateiname}{L03535}\newcommand{\titel}{Paul Goldmann an Olga XXXX Gussmann/Schnitzler, 15. 11. [XXXX]}\newcommand{\editorInnen}{Martin Anton Müller und Laura Untner}%% latex-leseansicht-abspann.tex
%% Abspann für die Leseansicht.
%% Der Schalter \ifkorrekturansicht ist bereits durch den Vorspann gesetzt.

%% latex-abspann.tex
%% Gemeinsamer Abspann für Korrekturansicht und Leseansicht.
%% Setzt den Schalter \ifkorrekturansicht voraus (gesetzt in den
%% einbindenden Dateien latex-korrekturansicht-abspann.tex bzw.
%% latex-leseansicht-abspann.tex).
%% ---------------------------------------------------------------

\normalsize

% Das esempio-Environment wird nur in der Leseansicht benötigt
\ifkorrekturansicht\else
\newenvironment{esempio}[3]%
{
    \vspace{1.5ex}
    \rlap{\underline{#1}}
    \par
    \setlength{\parindent}{0cm}
    \nopagebreak
    \leftskip=#2cm
    \rightskip=#3cm
}
{
    \par
}
\fi

\doendnotes{C}
\bigskip
\vfill

\clearpage

\footnotesize

\ifkorrekturansicht
  \lohead{\textsc{register}}
\fi

% theindex-Environment neu definieren ohne reledmac
\makeatletter
\renewenvironment{theindex}{%
  \ifkorrekturansicht
    \section*{\indexname}%
  \else
    \subsubsection*{Index der erwähnten Entitäten}%
  \fi
  \setlength{\parindent}{0pt}%
  \setlength{\parskip}{0pt plus 0.3pt}%
  \let\item\@idxitem
}{%
  \ifkorrekturansicht\clearpage\fi
}
\makeatother

\IfFileExists{\jobname-pw.ind}{\input{\jobname-pw.ind}}{}

% Quellenangabe nur in der Leseansicht
\ifkorrekturansicht\else
% Fallback-Definitionen, falls die .tex-Datei \titel etc. nicht gesetzt hat
\providecommand{\titel}{}
\providecommand{\editorInnen}{}
\providecommand{\dateiname}{\jobname}

\vspace{3cm}

\vfill

\footnotesize
\textsc{Quelle}: \titel. Herausgegeben von {\editorInnen}. In: \emph{Arthur Schnitzler: Briefwechsel mit Autorinnen und Autoren}.
 Digitale Edition, https://schnitzler-briefe.acdh.oeaw.ac.at/{\dateiname}.html (Stand \today)
\fi

\end{document}


      