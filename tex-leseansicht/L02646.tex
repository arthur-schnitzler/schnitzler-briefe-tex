%% latex-leseansicht-vorspann.tex
%% Vorspann für die Leseansicht.
%% Lädt die gemeinsame Datei latex-vorspann.tex mit nicht gesetztem Schalter.

\newif\ifkorrekturansicht
\korrekturansichtfalse

\input{../tex-inputs/latex-vorspann}


         
         \renewcommand{\erwaehntePersonen}{Personen:  ?? [Frau, die mit Goldmann in der Straßenbahn spricht, Ende November 1889],  ?? [Mann, der Gespräch über Schnitzler in der Straßenbahn belauscht, Ende November 1889], Paul Goldmann, Fedor Mamroth}
         \renewcommand{\erwaehnteInstitutionen}{Institutionen: An der schönen blauen Donau, Josef Eberle Stein-, Buch und Musikaliendruckerei}
         \renewcommand{\erwaehnteOrte}{Orte: Berggasse, Seidengasse, Wien}
         \renewcommand{\erwaehnteWerke}{}
               \section[Paul Goldmann an Arthur Schnitzler, 6. 12. 1889]{ Paul Goldmann an Arthur Schnitzler, 6. 12. 1889}\nopagebreak\mylabel{v}\rehead{ }\begin{ledgroupsized}[t]{13cm}\normalsize\beginnumbering\briefempfaengerindex{Schnitzler, Arthur@\textsc{Schnitzler, Arthur}!zzzGoldmann, Paul@\emph{von Paul Goldmann}!1889-12-061@{6. 12. 1889}|(be} \toendnotes[C]{\smallbreak\pagebreak[2]} \Standort{DLA, A:Schnitzler, HS.NZ85.1.3162.}
\physDesc{Brief, 1 Blatt, 4 Seiten, 3126 Zeichen
\newline{}Handschrift: blaue Tinte, deutsche Kurrent}\toendnotes[C]{\smallbreak}\pstart
           \noindent{}\centering{}{\pb}\textcolor{gray}{\textbf{\textbf{Adminiſtration: VII.
                           Seidengaſſe 7\oindex{Seidengasse@\textbf{Seidengasse}|pw}} (Jos. Eberle {\kaufmannsund} Co.\orgindex{Josef Eberle Stein-, Buch und Musikaliendruckerei@Josef Eberle Stein-, Buch und Musikaliendruckerei|pw})}}\pend
           \pstart
           \noindent{}\centering{}\textcolor{gray}{\textbf{An der Schönen Blauen Donau\orgindex{der schoenen blauen Donau@An der schönen blauen Donau|pw}}}\pend
           \pstart
           \noindent{}\centering{}\textcolor{gray}{\textbf{Chef-Redacteur: Dr. F.
                        Mamroth\pwindex{Mamroth, Fedor 21.02.1851 – 25.06.1907@\textsc{Mamroth, Fedor} (21.02.1851 – 25.06.1907), \emph{Journalist, Kritiker}|pw}. – Redaction: IX.,
                        Berggaſſe 31\oindex{Berggasse@\textbf{Berggasse}|pw}.}}\pend
           \pstart
           \raggedleft{}\textcolor{gray}{\textbf{Wien\oindex{Wien@\textbf{Wien}|pw}, den}}{ }6. December \textcolor{gray}{\textbf{18}}89.\pend
           \pstart\center{}Lieber Freund!\pend\pstart
           Sie haben Recht, es iſt ein fatales Zuſammentreffen geweſen. Aber – ich habe mir die
               Sache reiflich überlegt – es trifft mich nicht ſoviel Schuld, als Sie meinen.
               Zunächſt habe ich ja des Geſpräch nicht geſucht; zweitens iſt das ſelbe nicht, wie
               Ihr Gewährsmann\pwindex{?? [Mann, der Gespraech ueber Schnitzler in der Strassenbahn belauscht, Ende November 1889] *~Ende November 1889@\textsc{?? [Mann, der Gespräch über Schnitzler in der Straßenbahn belauscht, Ende November 1889]} (*~Ende November 1889)|pwv}
                  angi\textcolor{gray}{e}bt, »laut und lebhaft« geführt worden; überdies hatte ich
               von der Anweſenheit eines Dritten\pwindex{?? [Mann, der Gespraech ueber Schnitzler in der Strassenbahn belauscht, Ende November 1889] *~Ende November 1889@\textsc{?? [Mann, der Gespräch über Schnitzler in der Straßenbahn belauscht, Ende November 1889]} (*~Ende November 1889)|pwv} natürlich keine Ahnung; Sachen, die Sie irgendwie kompromittiren
               könnten, ſind ſelbſtverſtändlich nicht geſprochen worden; es iſt eben nur Ihr Name
               genannt worden, da es ja unmöglich iſt, die Nennung des Namens von demjenigen zu
               umgehen, über den man ſpricht. Soweit kann man in ſeiner Vorſicht unmöglich gehen,
               daß man von Perſonen, von denen man ganz {\pb}allgemein
               und unverfänglich ſpricht, nur die Anfangs-Buchſtaben nennt; überdies bitte ich Sie,
               ſich zu überlegen, wie beleidigend ein ſolches Verfahren der betreffenden Dame\pwindex{?? [Frau, die mit Goldmann in der Strassenbahn spricht, Ende November 1889] @\textsc{?? [Frau, die mit Goldmann in der Straßenbahn spricht, Ende November 1889]}|pwv} gegenüber iſt, mit der
               man ſpricht, und wie lächerlich man ſich ſelbſt dadurch macht. Schuld trägt nur der
               Zufall, der es gefügt hat, daß ein Geſpräch zwiſchen der Betreffenden\pwindex{?? [Frau, die mit Goldmann in der Strassenbahn spricht, Ende November 1889] @\textsc{?? [Frau, die mit Goldmann in der Straßenbahn spricht, Ende November 1889]}|pwv} und mir überhaupt auf der
               Tramway geführt wurde. Und Schuld trägt ferner der Dritte\pwindex{?? [Mann, der Gespraech ueber Schnitzler in der Strassenbahn belauscht, Ende November 1889] *~Ende November 1889@\textsc{?? [Mann, der Gespräch über Schnitzler in der Straßenbahn belauscht, Ende November 1889]} (*~Ende November 1889)|pwv}, der indiskret genug war, auf ein
               nicht für ihn beſtimmtes Geſpräch zu hören, darüber einem And\textcolor{gray}{re}n
               zu berichten und offenbar in einer Weise zu berichten, welche das jenige, was an \strikeout{f} und für ſich nicht \introOben{}für Sie\introOben{}
               kompromittirend war, erſt dazu machte. An \uline{deſſen}
               Adreſſe alſo hätten Sie ſich, wie ich meine, mit Ihren Vorwürfen wenden müſſen, und
               nicht an die meinige.\pend
           \pstart
           Sie werden begreifen, daß Ihr Brief mich, der ich mich ſchuldlos fühle, ſehr
               verſtimmt hat. Ich begreife vollkommen, wie peinlich Ihnen jene Unterredung geweſen
               iſt; ich bedaure auch von ganzem Herzen, daß ich der unſchuldige Anlaß war, daß Ihnen
               ein Ärgerniß bereitet wurde. Aber ich finde es – ganz offen geſtanden – {\pb}nicht recht ſreundſchaftlich von Ihnen gehandelt,
               daß Sie mich ohneweiters für Alles verantwortlich machen und mich in einer etwas
               odioſen Form zur Rechenſchaft ziehen, odios vor allem deshalb, weil, wie Sie
               jedenfalls wiſſen, \strikeout{\textcolor{gray}{e}}für einen Herrn mit etwas ausgebildeter Empfindlichkeit, es nichts
               Verletzenderes gibt, als eine Rüge und eine Belehrung, die mir beide in Ihrem Briefe
               ertheilt werden. Wäre ich an Ihrer Stelle geweſen, ſo glaube ich, daß ich nicht ſo
               vorgegangen wäre. Ich hätte entweder ganz darüber geſchwiegen, oder aber ich hätte
               die Sache in jenem gewiſſen Tone ſcherzhaften Vorwurfs zur Sprache gebracht und es
               dem Tacte des and\textcolor{gray}{re}n Theiles überlaſſen, ſich das, was darin Rüge
               und Belehrung iſt, ſelbſt herauszufinden.\pend
           \pstart
           Daß Sie \strikeout{Keines} keinen von dieſen beiden Wegen
               eingeſchlagen haben, verletzt mich ſehr. Es reſultirt daraus, wie geſagt, eine
               gewiſſe Verſtimmung gegen Sie. Und da es mir ſchwer fallen würde, dieſelbe zu
               verbergen, ſo bitte ich Sie, \strikeout{\textcolor{gray}{d}} mir zu geſtatten, daß ich für die nächſten Wochen von einem {\pb}\label{K_L02646-1v}\edtext{Zuſammenſein}{\lemma{\textnormal{\emph{Zuſammenſein}}}\Cendnote{\textnormal{Der Kontaktabbruch hielt nur bis zum nächsten Tag (7. 12. 1889).}}}\label{K_L02646-1h}
               mit Ihnen abſehe. Es fällt mir freilich ſchwer, Ihre ſo lieb gewordene Geſellſchaft
               mir zu verſagen; aber Sie haben mich da in eine Zwangslage verſetzt, aus der ich
               keinen andern Ausweg ſehe, als dieſen.\pend
           \pstart
           Ich grüße Sie herzlichſt! {\\[\baselineskip]}Ihr {\\[\baselineskip]}\spacefill\mbox{Dr. Paul Goldmann.}\pend
           \leftskip=0em{}
         
         \endnumbering\mylabel{h}\end{ledgroupsized}  \newcommand{\dateiname}{L02646}\newcommand{\titel}{Paul Goldmann an Arthur Schnitzler, 6. 12. 1889}\newcommand{\editorInnen}{Martin Anton Müller und Laura Untner}%% latex-leseansicht-abspann.tex
%% Abspann für die Leseansicht.
%% Der Schalter \ifkorrekturansicht ist bereits durch den Vorspann gesetzt.

%% latex-abspann.tex
%% Gemeinsamer Abspann für Korrekturansicht und Leseansicht.
%% Setzt den Schalter \ifkorrekturansicht voraus (gesetzt in den
%% einbindenden Dateien latex-korrekturansicht-abspann.tex bzw.
%% latex-leseansicht-abspann.tex).
%% ---------------------------------------------------------------

\normalsize

% Das esempio-Environment wird nur in der Leseansicht benötigt
\ifkorrekturansicht\else
\newenvironment{esempio}[3]%
{
    \vspace{1.5ex}
    \rlap{\underline{#1}}
    \par
    \setlength{\parindent}{0cm}
    \nopagebreak
    \leftskip=#2cm
    \rightskip=#3cm
}
{
    \par
}
\fi

\doendnotes{C}
\bigskip
\vfill

\clearpage

\footnotesize

\ifkorrekturansicht
  \lohead{\textsc{register}}
\fi

% theindex-Environment neu definieren ohne reledmac
\makeatletter
\renewenvironment{theindex}{%
  \ifkorrekturansicht
    \section*{\indexname}%
  \else
    \subsubsection*{Index der erwähnten Entitäten}%
  \fi
  \setlength{\parindent}{0pt}%
  \setlength{\parskip}{0pt plus 0.3pt}%
  \let\item\@idxitem
}{%
  \ifkorrekturansicht\clearpage\fi
}
\makeatother

\IfFileExists{\jobname-pw.ind}{\input{\jobname-pw.ind}}{}

% Quellenangabe nur in der Leseansicht
\ifkorrekturansicht\else
% Fallback-Definitionen, falls die .tex-Datei \titel etc. nicht gesetzt hat
\providecommand{\titel}{}
\providecommand{\editorInnen}{}
\providecommand{\dateiname}{\jobname}

\vspace{3cm}

\vfill

\footnotesize
\textsc{Quelle}: \titel. Herausgegeben von {\editorInnen}. In: \emph{Arthur Schnitzler: Briefwechsel mit Autorinnen und Autoren}.
 Digitale Edition, https://schnitzler-briefe.acdh.oeaw.ac.at/{\dateiname}.html (Stand \today)
\fi

\end{document}


      