%% latex-korrekturansicht-vorspann.tex
%% Vorspann für die Korrekturansicht.
%% Lädt die gemeinsame Datei latex-vorspann.tex mit gesetztem Schalter.

\newif\ifkorrekturansicht
\korrekturansichttrue

\input{../tex-inputs/latex-vorspann}


\section[Peter Altenberg an Arthur Schnitzler, 30. 11. 1898]{L00861 Peter Altenberg an Arthur Schnitzler, 30. 11. 1898}
\nopagebreak\mylabel{L00861v}
\rehead{ }\normalsize\beginnumbering\briefempfaengerindex{Schnitzler, Arthur@\textsc{Schnitzler, Arthur}!zzzAltenberg, Peter@\emph{von Peter Altenberg}!1898-11-301@{30. 11. 1898}|(be}
\toendnotes[C]{\smallbreak\pagebreak[2]}\Standort{CUL, Schnitzler, B 2.}
\physDesc{Brief, 1 Blatt, 2 Seiten, 719 Zeichen
\newline{}Handschrift: schwarze Tinte, deutsche Kurrent
\newline{}Schnitzler: mit rotem Buntstift eine Unterstreichung 
\newline{}Ordnung: mit Bleistift von unbekannter Hand nummeriert:
                                 »7« }\toendnotes[C]{\smallbreak}
\pstart{}{\pb}Lieber \textsc{D\textsuperscript{r.}} Arthur Schnitzler:\pend\vspace{0.5em}
\pstart
           Mit beſonderem Vergnügen ergreife ich die Gelegenheit, Ihnen etwas Angenehmes,
               Freundliches zu ſagen. Ihr Stück\pwindex{Vermaechtnis. Schauspiel in drei Akten@\emph{Das Vermächtnis. Schauspiel in drei Akten}|pwv} hat mir ganz \label{K_L00861-1v}\edtext{außerordentlich gefallen}{\lemma{\textnormal{\emph{außerordentlich gefallen}}}\Cendnote{\textnormal{\emph{Das Vermächtnis}\pwindex{Vermaechtnis. Schauspiel in drei Akten@\emph{Das Vermächtnis. Schauspiel in drei Akten}|pwk} wurde am
                  30. 11. 1898 zum ersten Mal am \emph{Burgtheater}\orgindex{Burgtheater@Burgtheater|pwk} gegeben, das Schreiben Altenbergs\pwindex{Altenberg, Peter 09.03.1859 – 08.01.1919@\textsc{Altenberg, Peter} (09.03.1859 – 08.01.1919), \emph{Schriftsteller/Schriftstellerin}|pwk} dürfte also nach Ende der Vorstellung (gegen 21 Uhr 30)
                  verfasst worden sein.}}}\label{K_L00861-1} und habe ich im Theater\oindex{Burgtheater@\textbf{Burgtheater}, \emph{S.THTR}|pwv}{ }ſelbſt dieſer Empfindung in zügelloſer Weiſe
               Ausdruck gegeben. Dieſe Geſtalt des Profeſſors Loſati\pwindex{Vermaechtnis. Schauspiel in drei Akten@\emph{Das Vermächtnis. Schauspiel in drei Akten}|pwv}, noch dazu von Hartmann\pwindex{Hartmann, Ernst 08.01.1844 – 10.10.1911@\textsc{Hartmann, Ernst} (08.01.1844 – 10.10.1911), \emph{Schauspieler/Schauspielerin}|pw} in dieſer letzten Vollkommenheit lebendig gemacht, iſt wirklich
               wunderbar ausgeführt.\pend
           
\pstart
           {\pb}Ich hätte entſchieden dieſes Stück
               betitelt: »\uline{Profeſſor Loſati}\pwindex{Vermaechtnis. Schauspiel in drei Akten@\emph{Das Vermächtnis. Schauspiel in drei Akten}|pwv}«. Der 3. Akt\pwindex{Vermaechtnis. Schauspiel in drei Akten@\emph{Das Vermächtnis. Schauspiel in drei Akten}|pwv} mit den
               Karakteren des Profeſſors u. ſeiner Tochter iſt meiſterhaft.\pend
           
\pstart
           Ich war ganz hingeriſſen.\pend
           
\pstart
           Es iſt entſchieden Ihre kraftvollſte Arbeit\pwindex{Vermaechtnis. Schauspiel in drei Akten@\emph{Das Vermächtnis. Schauspiel in drei Akten}|pwv}. Einfach vorzüglich.\pend
           
\pstart
           Ich ſpreche Ihnen meine allerherzlichſte Gratulation aus.\pend
           \pstart \spacefill\mbox{Peter Altenberg}\pend{}
\pstart
           30. November 98.\pend
           \selectlanguage{ngerman}\endnumbering\briefempfaengerindex{Schnitzler, Arthur@\textsc{Schnitzler, Arthur}!zzzAltenberg, Peter@\emph{von Peter Altenberg}!1898-11-301@{30. 11. 1898}|)be}\mylabel{L00861h}  \normalsize

\doendnotes{C}
\bigskip
\vfill

\clearpage

\footnotesize

\lohead{\textsc{register}}

% Definiere theindex-Environment komplett neu ohne reledmac
\makeatletter
\renewenvironment{theindex}{%
  \section*{\indexname}%
  \setlength{\parindent}{0pt}%
  \setlength{\parskip}{0pt plus 0.3pt}%
  \let\item\@idxitem
}{%
  \clearpage
}
\makeatother

\IfFileExists{\jobname-pw.ind}{\input{\jobname-pw.ind}}{}

\end{document}

      