\input{../tex-inputs/latex-pdf-vorspann}
\begin{center}
            \textcolor{red}{ENTWURF. ENTZIFFERUNG NOCH NICHT KORREKTURGELESEN}
                      \end{center}
            
               \section[Arthur Schnitzler an Hugo von Hofmannsthal, {[}10. 1. 1899{]}]{ Arthur Schnitzler an Hugo von Hofmannsthal, {[}10. 1. 1899{]}}\nopagebreak\mylabel{v}\rehead{ }\begin{ledgroupsized}[t]{13cm}\normalsize\beginnumbering\briefempfaengerindex{Hofmannsthal, Hugo von@\textsc{Hofmannsthal, Hugo von}!zzzSchnitzler, Arthur@\emph{von Arthur Schnitzler}!1899-01-102@{{[}10. 1. 1899{]}}|(be} \toendnotes[C]{\smallbreak\pagebreak[2]} \Standort{FDH, Hs-30885,79.}
\physDesc{Brief, 1 Blatt, 3 Seiten
\newline{}Handschrift: Bleistift, deutsche Kurrent\newline{}Ordnung: mit Bleistift von unbekannter Hand datiert: »Anf. 99, 98?« }\buchAbdrucke{\weitereDrucke{Hugo von Hofmannsthal, Arthur Schnitzler: \emph{Briefwechsel}. Hg. Therese Nickl und Heinrich Schnitzler. Frankfurt am Main: \emph{S. Fischer} 1964, S. 116–117.} }\toendnotes[C]{\smallbreak}\pstart
           \raggedleft{}{\pb}\uline{Dinſtg.}\pend
           \pstart
           Mein lieber Hugo, ich wußte gar nicht, dſs Sie ſchon da ſind.
                    Morgen ko{\geminationm} ich jedenfalls ins \textsc{Pfob\oindex{Cafe Pfob@\textbf{Café Pfob}|pw}} u freu mich Sie endlich wiederzuſehn. \textsc{Pfob\oindex{Cafe Pfob@\textbf{Café Pfob}|pw}} iſt allerdgs wenig. Vor \textsc{Pfob\oindex{Cafe Pfob@\textbf{Café Pfob}|pw}} will ich morgen komiſcherweiſe ins Jantſchtheater\oindex{Jantsch-Theater@\textbf{Jantsch-Theater}|pw} zu Theodora\pwindex{\textcolor{red}{\textsuperscript{XXXX1 indx}}!Theodora1884@\strich\emph{Theodora} {[}1884{]}|pw}, u zw hab
                    ich mit \textsc{Wassermann}\pwindex{Wassermann, Jakob 10.03.1873 – 01.01.1934@\textsc{Wassermann, Jakob} (10.03.1873 – 01.01.1934), \emph{Schriftsteller}|pw} vor ½ 8 im Vorraum des Theaters Rendezvous. Vielleicht hat er
                    eine {\pb}geſchenkte Loge; ev. kaufen wir uns Billetts.
                    Vielleicht ſind Sie auch vor ½ 8 im Vorraum. Eine gute Schauſpielerin\pwindex{Anders, Elisabeth @\textsc{Anders, Elisabeth}, \emph{Schauspielerin}|pwv}{ }ſoll die Theodora\pwindex{\textcolor{red}{\textsuperscript{XXXX1 indx}}!Theodora1884@\strich\emph{Theodora} {[}1884{]}|pwv}{ }ſpielen.\pend
           \pstart
           Mir iſt es wieder innerlich recht miſerabel gegangen; aber mit dem Arbeiten
                    beſſer. Im übrigen muſs ich über Burg\orgindex{Burgtheater@Burgtheater|pw} mit Ihnen
                    reden. Denken Sie, dſs der Kakadu\pwindex{Schnitzler, Arthur 15.05.1862 – 21.10.1931@\textsc{Schnitzler, Arthur} (15.05.1862 – 21.10.1931), \emph{Schriftsteller, Mediziner}!gruene Kakadu. Groteske in einem Akt1.3.1899 – 1.3.1899@\strich\emph{Der grüne Kakadu. Groteske in einem Akt} {[}1.3.1899 – 1.3.1899{]}|pw}{ }{\pb}nicht unbeträchtliche Chancen hat! – Aber das alles
                    mündlich –\pend
           \pstart Von Herzen Ihr \spacefill\mbox{Arthur}\pend{}\endnumbering\briefempfaengerindex{Hofmannsthal, Hugo von@\textsc{Hofmannsthal, Hugo von}!zzzSchnitzler, Arthur@\emph{von Arthur Schnitzler}!1899-01-102@{{[}10. 1. 1899{]}}|)be}\mylabel{h}\end{ledgroupsized}  \newcommand{\dateiname}{L00878}\newcommand{\titel}{Arthur Schnitzler an Hugo von Hofmannsthal, [10. 1. 1899]}\newcommand{\editorInnen}{Martin Anton Müller und Gerd-Hermann Susen}\input{../tex-inputs/latex-pdf-abspann}
      