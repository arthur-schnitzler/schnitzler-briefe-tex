%% latex-leseansicht-vorspann.tex
%% Vorspann für die Leseansicht.
%% Lädt die gemeinsame Datei latex-vorspann.tex mit nicht gesetztem Schalter.

\newif\ifkorrekturansicht
\korrekturansichtfalse

\input{../tex-inputs/latex-vorspann}


\section[Arthur Schnitzler an Hugo von Hofmannsthal, {{[}}10. 1. 1899{{]}}]{L00878 Arthur Schnitzler an Hugo von Hofmannsthal, {[}10. 1. 1899{]}}
\nopagebreak\mylabel{L00878v}
\rehead{ }\normalsize\beginnumbering\briefempfaengerindex{Hofmannsthal, Hugo von@\textsc{Hofmannsthal, Hugo von}!zzzSchnitzler, Arthur@\emph{von Arthur Schnitzler}!1899-01-102@{{[}10. 1. 1899{]}}|(be}
\toendnotes[C]{\smallbreak\pagebreak[2]}
\correspDesc{Versand  durch Arthur Schnitzler am [10. 1. 1899] in Wien
\newline{}Erhalt  durch Hugo von Hofmannsthal im Zeitraum [10. 1. 1899
                  – 14. 1. 1899?] in Wien}\toendnotes[C]{\smallbreak}
\Standort{FDH, Hs-30885,79.}
\physDesc{Brief, 1 Blatt, 3 Seiten, 693 Zeichen
\newline{}Handschrift: Bleistift, deutsche Kurrent
\newline{}Ordnung: mit Bleistift von unbekannter Hand datiert: »Anf. 99, 98?« }
\buchAbdrucke{\weitereDrucke{Hugo von Hofmannsthal, Arthur Schnitzler: \emph{Briefwechsel}. Herausgegeben von Therese Nickl und Heinrich Schnitzler. Frankfurt am Main: \emph{S. Fischer} 1964, S. 116–117.} }\toendnotes[C]{\smallbreak}
\pstart
           \raggedleft{}{\pb}\uline{Dinſtg.}\pend
           \vspace{0.5em}
\pstart
           Mein lieber Hugo, ich wußte gar nicht, dſs Sie{ }ſchon da{ }ſind. Morgen
                  ko{\geminationm} ich jedenfalls ins \textsc{Pfob\oindex{Wien@\textbf{Wien}!I., Innere Stadt@\textbf{I., Innere Stadt}!Café Pfob@\textbf{Café Pfob}, \emph{Kaffeehaus}|pw}} u freu mich Sie endlich wiederzuſehn. \textsc{Pfob\oindex{Wien@\textbf{Wien}!I., Innere Stadt@\textbf{I., Innere Stadt}!Café Pfob@\textbf{Café Pfob}, \emph{Kaffeehaus}|pw}} iſt allerdgs wenig. Vor \textsc{Pfob\oindex{Wien@\textbf{Wien}!I., Innere Stadt@\textbf{I., Innere Stadt}!Café Pfob@\textbf{Café Pfob}, \emph{Kaffeehaus}|pw}} will ich morgen komiſcherweiſe ins Jantſchtheater\oindex{Jantsch-Theater@\textbf{Jantsch-Theater}, \emph{Theater}|pw} zu Theodora\pwindex{\textcolor{red}{\textsuperscript{XXXX indx1}}!Theodora@\strich\emph{Theodora}|pw}, u zw hab
               ich mit \textsc{Wassermann}\pwindex{Wassermann, Jakob 10.\,3.\,1873 Fürth – 1.\,1.\,1934 Altaussee@\textsc{Wassermann, Jakob} (10.\,3.\,1873 Fürth – 1.\,1.\,1934 Altaussee), \emph{Schriftsteller}|pw} vor ½ 8 im Vorraum des Theaters Rendezvous. Vielleicht hat er eine
                  {\pb}geſchenkte Loge; ev. kaufen wir uns Billetts.
               Vielleicht{ }ſind Sie auch vor ½ 8 im Vorraum. Eine gute Schauſpielerin\pwindex{Anders, Elisabeth @\textsc{Anders, Elisabeth}, \emph{Schauspielerin}|pwv}{ }ſoll die Theodora\pwindex{\textcolor{red}{\textsuperscript{XXXX indx1}}!Theodora@\strich\emph{Theodora}|pwv}{ }ſpielen.\pend
           
\pstart
           Mir iſt es wieder innerlich recht miſerabel gegangen; aber mit dem Arbeiten beſſer.
               Im übrigen muſs ich über Burg\orgindex{Burgtheater@Burgtheater|pw} mit Ihnen reden.
               Denken Sie, dſs der Kakadu\pwindex{Schnitzler, Arthur 15.\,5.\,1862 Wien – 21.\,10.\,1931 ebd.@\textsc{Schnitzler, Arthur} (15.\,5.\,1862 Wien – 21.\,10.\,1931 ebd.), \emph{Schriftsteller, Mediziner}!grüne Kakadu. Groteske in einem Akt@\strich\emph{Der grüne Kakadu. Groteske in einem Akt}|pw}{ }{\pb}nicht unbeträchtliche Chancen hat! – Aber das alles
               mündlich –\pend
           \pstart Von Herzen Ihr \spacefill\mbox{Arthur}\pend{}\selectlanguage{ngerman}\endnumbering\briefempfaengerindex{Hofmannsthal, Hugo von@\textsc{Hofmannsthal, Hugo von}!zzzSchnitzler, Arthur@\emph{von Arthur Schnitzler}!1899-01-102@{{[}10. 1. 1899{]}}|)be}\mylabel{L00878h}  \newcommand{\dateiname}{L00878}\newcommand{\titel}{Arthur Schnitzler an Hugo von Hofmannsthal, [10. 1. 1899]}\newcommand{\editorInnen}{Martin Anton Müller und Gerd-Hermann Susen}%% latex-leseansicht-abspann.tex
%% Abspann für die Leseansicht.
%% Der Schalter \ifkorrekturansicht ist bereits durch den Vorspann gesetzt.

%% latex-abspann.tex
%% Gemeinsamer Abspann für Korrekturansicht und Leseansicht.
%% Setzt den Schalter \ifkorrekturansicht voraus (gesetzt in den
%% einbindenden Dateien latex-korrekturansicht-abspann.tex bzw.
%% latex-leseansicht-abspann.tex).
%% ---------------------------------------------------------------

\normalsize

% Das esempio-Environment wird nur in der Leseansicht benötigt
\ifkorrekturansicht\else
\newenvironment{esempio}[3]%
{
    \vspace{1.5ex}
    \rlap{\underline{#1}}
    \par
    \setlength{\parindent}{0cm}
    \nopagebreak
    \leftskip=#2cm
    \rightskip=#3cm
}
{
    \par
}
\fi

\doendnotes{C}
\bigskip
\vfill

\clearpage

\footnotesize

\ifkorrekturansicht
  \lohead{\textsc{register}}
\fi

% theindex-Environment neu definieren ohne reledmac
\makeatletter
\renewenvironment{theindex}{%
  \ifkorrekturansicht
    \section*{\indexname}%
  \else
    \subsubsection*{Index der erwähnten Entitäten}%
  \fi
  \setlength{\parindent}{0pt}%
  \setlength{\parskip}{0pt plus 0.3pt}%
  \let\item\@idxitem
}{%
  \ifkorrekturansicht\clearpage\fi
}
\makeatother

\IfFileExists{\jobname-pw.ind}{\input{\jobname-pw.ind}}{}

% Quellenangabe nur in der Leseansicht
\ifkorrekturansicht\else
% Fallback-Definitionen, falls die .tex-Datei \titel etc. nicht gesetzt hat
\providecommand{\titel}{}
\providecommand{\editorInnen}{}
\providecommand{\dateiname}{\jobname}

\vspace{3cm}

\vfill

\footnotesize
\textsc{Quelle}: \titel. Herausgegeben von {\editorInnen}. In: \emph{Arthur Schnitzler: Briefwechsel mit Autorinnen und Autoren}.
 Digitale Edition, https://schnitzler-briefe.acdh.oeaw.ac.at/{\dateiname}.html (Stand \today)
\fi

\end{document}


