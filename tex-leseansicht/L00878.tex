%% latex-korrekturansicht-vorspann.tex
%% Vorspann für die Korrekturansicht.
%% Lädt die gemeinsame Datei latex-vorspann.tex mit gesetztem Schalter.

\newif\ifkorrekturansicht
\korrekturansichttrue

\input{../tex-inputs/latex-vorspann}


\section[Arthur Schnitzler an Hugo von Hofmannsthal, {[}10. 1. 1899{]}]{L00878 Arthur Schnitzler an Hugo von Hofmannsthal, {[}10. 1. 1899{]}}
\nopagebreak\mylabel{L00878v}
\rehead{ }\normalsize\beginnumbering\briefempfaengerindex{Hofmannsthal, Hugo von@\textsc{Hofmannsthal, Hugo von}!zzzSchnitzler, Arthur@\emph{von Arthur Schnitzler}!1899-01-102@{{[}10. 1. 1899{]}}|(be}
\toendnotes[C]{\smallbreak\pagebreak[2]}\Standort{FDH, Hs-30885,79.}
\physDesc{Brief, 1 Blatt, 3 Seiten, 693 Zeichen
\newline{}Handschrift: Bleistift, deutsche Kurrent
\newline{}Ordnung: mit Bleistift von unbekannter Hand datiert: »Anf. 99, 98?« }
\buchAbdrucke{\weitereDrucke{Hugo von Hofmannsthal, Arthur Schnitzler: \emph{Briefwechsel}. Frankfurt am Main: \emph{S. Fischer} 1964, S. 116–117.} }\toendnotes[C]{\smallbreak}
\pstart
           \raggedleft{}{\pb}\uline{Dinſtg.}\pend
           \vspace{0.5em}
\pstart
           Mein lieber Hugo, ich wußte gar nicht, dſs Sie ſchon da ſind. Morgen
                  ko{\geminationm} ich jedenfalls ins \textsc{Pfob\oindex{Cafe Pfob@\textbf{Café Pfob}, \emph{Kaffeehaus (K.KAF)}|pw}} u freu mich Sie endlich wiederzuſehn. \textsc{Pfob\oindex{Cafe Pfob@\textbf{Café Pfob}, \emph{Kaffeehaus (K.KAF)}|pw}} iſt allerdgs wenig. Vor \textsc{Pfob\oindex{Cafe Pfob@\textbf{Café Pfob}, \emph{Kaffeehaus (K.KAF)}|pw}} will ich morgen komiſcherweiſe ins Jantſchtheater\oindex{Jantsch-Theater@\textbf{Jantsch-Theater}, \emph{Theater (K.THE)}|pw} zu Theodora\pwindex{Theodora@\emph{Theodora}|pw}, u zw hab
               ich mit \textsc{Wassermann}\pwindex{Wassermann, Jakob 10.03.1873 – 01.01.1934@\textsc{Wassermann, Jakob} (10.03.1873 – 01.01.1934), \emph{Schriftsteller/Schriftstellerin}|pw} vor ½ 8 im Vorraum des Theaters Rendezvous. Vielleicht hat er eine
                  {\pb}geſchenkte Loge; ev. kaufen wir uns Billetts.
               Vielleicht ſind Sie auch vor ½ 8 im Vorraum. Eine gute Schauſpielerin\pwindex{Anders, Elisabeth @\textsc{Anders, Elisabeth}, \emph{Schauspieler/Schauspielerin}|pwv}{ }ſoll die Theodora\pwindex{Theodora@\emph{Theodora}|pwv}{ }ſpielen.\pend
           
\pstart
           Mir iſt es wieder innerlich recht miſerabel gegangen; aber mit dem Arbeiten beſſer.
               Im übrigen muſs ich über Burg\orgindex{Burgtheater@Burgtheater|pw} mit Ihnen reden.
               Denken Sie, dſs der Kakadu\pwindex{gruene Kakadu. Groteske in einem Akt@\emph{Der grüne Kakadu. Groteske in einem Akt}|pw}{ }{\pb}nicht unbeträchtliche Chancen hat! – Aber das alles
               mündlich –\pend
           \pstart Von Herzen Ihr \spacefill\mbox{Arthur}\pend{}\selectlanguage{ngerman}\endnumbering\briefempfaengerindex{Hofmannsthal, Hugo von@\textsc{Hofmannsthal, Hugo von}!zzzSchnitzler, Arthur@\emph{von Arthur Schnitzler}!1899-01-102@{{[}10. 1. 1899{]}}|)be}\mylabel{L00878h}  \normalsize

\doendnotes{C}
\bigskip
\vfill

\clearpage

\footnotesize

\lohead{\textsc{register}}

% Definiere theindex-Environment komplett neu ohne reledmac
\makeatletter
\renewenvironment{theindex}{%
  \section*{\indexname}%
  \setlength{\parindent}{0pt}%
  \setlength{\parskip}{0pt plus 0.3pt}%
  \let\item\@idxitem
}{%
  \clearpage
}
\makeatother

\IfFileExists{\jobname-pw.ind}{\input{\jobname-pw.ind}}{}

\end{document}

      