%% latex-leseansicht-vorspann.tex
%% Vorspann für die Leseansicht.
%% Lädt die gemeinsame Datei latex-vorspann.tex mit nicht gesetztem Schalter.

\newif\ifkorrekturansicht
\korrekturansichtfalse

\input{../tex-inputs/latex-vorspann}


\section[Georg Brandes an Arthur Schnitzler, 21. 4. 1926]{L02470 Georg Brandes an Arthur Schnitzler, 21. 4. 1926}
\nopagebreak\mylabel{L02470v}
\rehead{ }\normalsize\beginnumbering\briefempfaengerindex{Schnitzler, Arthur@\textsc{Schnitzler, Arthur}!zzzBrandes, Georg@\emph{von Georg Brandes}!1926-04-211@{21. 4. 1926}|(be}
\toendnotes[C]{\smallbreak\pagebreak[2]}
\correspDesc{Versand  durch Georg Brandes am 21. 4. 1926 in Kopenhagen
\newline{}Weiterleitung  in Wien
\newline{}Erhalt  durch Arthur Schnitzler am 12. 5. 1926 in Berlin}\toendnotes[C]{\smallbreak}
\Standort{CUL, Schnitzler, B 17.}
\physDesc{Briefkarte, 1282 Zeichen
\newline{}Handschrift: schwarze Tinte, lateinische Kurrent
\newline{}Schnitzler: mit rotem Buntstift eine Unterstreichung 
\newline{}Ordnung: mit Bleistift von unbekannter Hand nummeriert:
                                    »62« }
\buchAbdrucke{\weitereDrucke{Georg Brandes, Arthur Schnitzler: \emph{Ein Briefwechsel}. Herausgegeben von Kurt Bergel. Bern: \emph{Francke} 1956, S. 152–153.} }\toendnotes[C]{\smallbreak}
\pstart
           \raggedleft{}{\pb}Kopenhagen\oindex{Kopenhagen@\textbf{Kopenhagen}, \emph{Hauptstadt}|pw}{ }21 April 26\pend
           \vspace{0.5em}
\pstart
           Mein liebster Freund Sie sind einer der wenigen Menschen, dem ich
               nur Gutes verdanke, einen wahren geistigen Reichtum. Heute las ich zum zweiten Male –
               nach Monaten – Ihr tiefsinniges Drama über den \uline{Weiher}\pwindex{Schnitzler, Arthur 15.\,5.\,1862 Wien – 21.\,10.\,1931 ebd.@\textsc{Schnitzler, Arthur} (15.\,5.\,1862 Wien – 21.\,10.\,1931 ebd.), \emph{Schriftsteller, Mediziner}!Gang zum Weiher. Dramatische Dichtung@\strich\emph{Der Gang zum Weiher. Dramatische Dichtung}|pw}, und verstand es inniger als das erste Mal, hatte meine Freude daran. Sie haben
               dort eine Saite angeschlagen, die in der Gegenwart selten \substVorne{}\textsuperscript{geworden ist}\substDazwischen{}gehört wird\substHinten{}; Verse klingen heutzutage selten von der Bühne, und Sie sind zu den
               ausführlicheren Repliken älterer Zeiten zurückgekehrt. Aber Sie meistern diesen Stil,
               und Sie \substVorne{}\textsuperscript{\textcolor{gray}{fesslen}}\substDazwischen{}fesseln\substHinten{}. Das Stück\pwindex{Schnitzler, Arthur 15.\,5.\,1862 Wien – 21.\,10.\,1931 ebd.@\textsc{Schnitzler, Arthur} (15.\,5.\,1862 Wien – 21.\,10.\,1931 ebd.), \emph{Schriftsteller, Mediziner}!Gang zum Weiher. Dramatische Dichtung@\strich\emph{Der Gang zum Weiher. Dramatische Dichtung}|pwv} ist ein
               schönes Ganzes.\pend
           
\pstart
           {\pb}Ich habe keine Zeitungen in
               deutscher Sprache, weiss deshalb nicht, ob das Stück\pwindex{Schnitzler, Arthur 15.\,5.\,1862 Wien – 21.\,10.\,1931 ebd.@\textsc{Schnitzler, Arthur} (15.\,5.\,1862 Wien – 21.\,10.\,1931 ebd.), \emph{Schriftsteller, Mediziner}!Gang zum Weiher. Dramatische Dichtung@\strich\emph{Der Gang zum Weiher. Dramatische Dichtung}|pwv} aufgeführt worden noch ob es Erfolg hatte. Sie
               wissen, dass ich Ihnen jeglichen Erfolg wünsche. – Ich denke mir, dass ich
                  Anfang Mai um meiner Gesundheit willen nach Karlsbad\oindex{Karlsbad@\textbf{Karlsbad}|pw} reise. Ich bin wol mehr als ein Dutzend Mal vor dem
               Kriege dort gewesen. Jetzt wird es wol dort, wie überall, \strikeout{dort} ärmer sein. Die Sprache trennt mich leider von Ihnen. Mein deutscher
               Verleger, Erich Reiss\pwindex{Reiss, Erich 24.\,1.\,1887 Berlin – 8.\,5.\,1951 New York City@\textsc{Reiss, Erich} (24.\,1.\,1887 Berlin – 8.\,5.\,1951 New York City), \emph{Verleger}|pw}, hat Fallissement
               gemacht. Alles was er mir schuldig war, seit Jahren, ist in Rauch aufgegangen.\pend
           
\pstart
           Ich hoffe, dass es Ihnen und den Kindern\pwindex{Schnitzler, Heinrich 9.\,8.\,1902 Hinterbrühl – 12.\,7.\,1982 Wien@\textsc{Schnitzler, Heinrich} (9.\,8.\,1902 Hinterbrühl – 12.\,7.\,1982 Wien), \emph{Regisseur, Schauspieler}|pwv}\pwindex{Cappellini, Lili 13.\,9.\,1909 Wien – 26.\,7.\,1928 Venedig@\textsc{Cappellini, Lili} (13.\,9.\,1909 Wien – 26.\,7.\,1928 Venedig)|pwv} gut geht. – Frau Gertrud Rung\pwindex{Rung, Gertrud 26.\,3.\,1882 Kopenhagen – 25.\,4.\,1959@\textsc{Rung, Gertrud} (26.\,3.\,1882 Kopenhagen – 25.\,4.\,1959), \emph{Übersetzerin, Sekretärin}|pw}, die Sie freundlich empfingen, liebt Sie sehr. Ihr Freund
                  \spacefill\mbox{Georg Brandes}\pend
           \selectlanguage{ngerman}\endnumbering\briefempfaengerindex{Schnitzler, Arthur@\textsc{Schnitzler, Arthur}!zzzBrandes, Georg@\emph{von Georg Brandes}!1926-04-211@{21. 4. 1926}|)be}\mylabel{L02470h}  \newcommand{\dateiname}{L02470}\newcommand{\titel}{Georg Brandes an Arthur Schnitzler, 21. 4. 1926}\newcommand{\editorInnen}{Martin Anton Müller und Gerd-Hermann Susen}%% latex-leseansicht-abspann.tex
%% Abspann für die Leseansicht.
%% Der Schalter \ifkorrekturansicht ist bereits durch den Vorspann gesetzt.

%% latex-abspann.tex
%% Gemeinsamer Abspann für Korrekturansicht und Leseansicht.
%% Setzt den Schalter \ifkorrekturansicht voraus (gesetzt in den
%% einbindenden Dateien latex-korrekturansicht-abspann.tex bzw.
%% latex-leseansicht-abspann.tex).
%% ---------------------------------------------------------------

\normalsize

% Das esempio-Environment wird nur in der Leseansicht benötigt
\ifkorrekturansicht\else
\newenvironment{esempio}[3]%
{
    \vspace{1.5ex}
    \rlap{\underline{#1}}
    \par
    \setlength{\parindent}{0cm}
    \nopagebreak
    \leftskip=#2cm
    \rightskip=#3cm
}
{
    \par
}
\fi

\doendnotes{C}
\bigskip
\vfill

\clearpage

\footnotesize

\ifkorrekturansicht
  \lohead{\textsc{register}}
\fi

% theindex-Environment neu definieren ohne reledmac
\makeatletter
\renewenvironment{theindex}{%
  \ifkorrekturansicht
    \section*{\indexname}%
  \else
    \subsubsection*{Index der erwähnten Entitäten}%
  \fi
  \setlength{\parindent}{0pt}%
  \setlength{\parskip}{0pt plus 0.3pt}%
  \let\item\@idxitem
}{%
  \ifkorrekturansicht\clearpage\fi
}
\makeatother

\IfFileExists{\jobname-pw.ind}{\input{\jobname-pw.ind}}{}

% Quellenangabe nur in der Leseansicht
\ifkorrekturansicht\else
% Fallback-Definitionen, falls die .tex-Datei \titel etc. nicht gesetzt hat
\providecommand{\titel}{}
\providecommand{\editorInnen}{}
\providecommand{\dateiname}{\jobname}

\vspace{3cm}

\vfill

\footnotesize
\textsc{Quelle}: \titel. Herausgegeben von {\editorInnen}. In: \emph{Arthur Schnitzler: Briefwechsel mit Autorinnen und Autoren}.
 Digitale Edition, https://schnitzler-briefe.acdh.oeaw.ac.at/{\dateiname}.html (Stand \today)
\fi

\end{document}


