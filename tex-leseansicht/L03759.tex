%% latex-leseansicht-vorspann.tex
%% Vorspann für die Leseansicht.
%% Lädt die gemeinsame Datei latex-vorspann.tex mit nicht gesetztem Schalter.

\newif\ifkorrekturansicht
\korrekturansichtfalse

\input{../tex-inputs/latex-vorspann}


\section[Olga Schnitzler an Stefan Zweig, 21. 12. 1916]{L03759 Olga Schnitzler an Stefan Zweig, 21. 12. 1916}
\nopagebreak\mylabel{L03759v}
\rehead{ }\normalsize\beginnumbering\briefempfaengerindex{Zweig, Stefan@\textsc{Zweig, Stefan}!zzzSchnitzler, Olga@\emph{von Olga Schnitzler}!1916-12-212@{21. 12. 1916}|(be}
\toendnotes[C]{\smallbreak\pagebreak[2]}
\correspDesc{Versand  durch Olga Schnitzler am 21. 12. 1916 in Wien
\newline{}Erhalt  durch Stefan Zweig im Zeitraum [21. 12. 1916 – 24. 12. 1916?] in Wien}\toendnotes[C]{\smallbreak}
\Standort{Jerusalem, National Library of Israel, ARC. Ms. Var. 305 1 58 Stefan Zweig Collection.}
\physDesc{Briefkarte, 833 Zeichen
\newline{}Handschrift: schwarze Tinte, lateinische Kurrent}\toendnotes[C]{\smallbreak}
\pstart
           \raggedleft{}{\pb}21. Dec. 1916.\pend
           \vspace{0.5em}
\pstart
           Lieber Herr Doctor, meine liebe Hofrätin\pwindex{Zuckerkandl, Berta 13.\,4.\,1864 Wien – 16.\,10.\,1945 Paris@\textsc{Zuckerkandl, Berta} (13.\,4.\,1864 Wien – 16.\,10.\,1945 Paris), \emph{Schriftstellerin, Journalistin, Übersetzerin}|pwv} erzält mir heute Abend, dass Sie Ihnen von dieser
               übeln \label{K_L03759-1v}\edtext{Klatscherei}{\lemma{\textnormal{\emph{Klatscherei}}}\Cendnote{\textnormal{Olga
                     Schnitzler\pwindex{Schnitzler, Olga 17.\,1.\,1882 Wien – 13.\,1.\,1970 Lugano@\textsc{Schnitzler, Olga} (17.\,1.\,1882 Wien – 13.\,1.\,1970 Lugano), \emph{Schauspielerin, Sängerin}|pwk} hatte es schwer und tat sich schwer, mit ihrer Gesangskarriere
                  aus dem Schatten des berühmten Ehemanns zu treten. Entsprechend empfindlich reagierte sie auf Gerüchte,
                  die ihr Können in Frage stellten. Vgl. A. S.: \emph{Tagebuch}, 4. 12. 1916. Eventuell 
                  könnte es sich auch um ein spezifischeres Gerücht handeln, 
                  Olga Schnitzler\pwindex{Schnitzler, Olga 17.\,1.\,1882 Wien – 13.\,1.\,1970 Lugano@\textsc{Schnitzler, Olga} (17.\,1.\,1882 Wien – 13.\,1.\,1970 Lugano), \emph{Schauspielerin, Sängerin}|pwk} in einer intimen Beziehung vermutend.}}}\label{K_L03759-1} berichtet hat und dass Sie dieses
               Gerede richtigstellen wollen. Ich bitte Sie sehr, – tun sie es nicht, – das gibt der
               Sache eine Bedeutung, die sie nicht hat und nicht haben darf. Mein Instinct hat sich
               damals, nach dem ersten Erschrecken, bald gegen den »Warner« {\pb}gewendet, – ich finde, ein Mann, der alle paar Jahre auf 2 Stunden in meinem
                  Hause ist, darf so etwas gewiss nicht
               tun, – kaum hat ein erprobter Freund das Recht dazu. Die Hofrätin\pwindex{Zuckerkandl, Berta 13.\,4.\,1864 Wien – 16.\,10.\,1945 Paris@\textsc{Zuckerkandl, Berta} (13.\,4.\,1864 Wien – 16.\,10.\,1945 Paris), \emph{Schriftstellerin, Journalistin, Übersetzerin}|pwv} hat mir ihren Eindruck von Ihrer aufrichtigen Freude
               an meinem \label{K_L03759-3v}\edtext{Concert\textcolor{gray}{-}Abend\eventindex{Wiener Konzerthaus@\textbf{Wiener Konzerthaus}!Gesangskonzert von Olga Schnitzler, 18.11.1916@Gesangskonzert von Olga Schnitzler, 18.11.1916|pwv}}{\lemma{\textnormal{\emph{Concert-Abend}}}\Cendnote{\textnormal{Am 18. 11. 1916 war Olga Schnitzler\pwindex{Schnitzler, Olga 17.\,1.\,1882 Wien – 13.\,1.\,1970 Lugano@\textsc{Schnitzler, Olga} (17.\,1.\,1882 Wien – 13.\,1.\,1970 Lugano), \emph{Schauspielerin, Sängerin}|pwk} an einem Liederkonzert\eventindex{Wiener Konzerthaus@\textbf{Wiener Konzerthaus}!Gesangskonzert von Olga Schnitzler, 18.11.1916@Gesangskonzert von Olga Schnitzler, 18.11.1916|pwk} im Wiener Konzerthaus\oindex{Wien@\textbf{Wien}!III., Landstraße@\textbf{III., Landstraße}!Wiener Konzerthaus@\textbf{Wiener Konzerthaus}, \emph{Konzertsaal}|pwk}
                  beteiligt gewesen.}}}\label{K_L03759-3} mitgeteilt, — das genügt mir vollkommen, und so lasse ich
               mir Ihre freundlichen Worte auch nicht entstellen. Man hat seinen Weg zu gehen,
               darauf kommt es an. Seien sie herzlich gegrüsst!\pend
           \pstart \spacefill\mbox{OlgaSchnitzler.}\pend{}\selectlanguage{ngerman}\endnumbering\briefempfaengerindex{Zweig, Stefan@\textsc{Zweig, Stefan}!zzzSchnitzler, Olga@\emph{von Olga Schnitzler}!1916-12-212@{21. 12. 1916}|)be}\mylabel{L03759h}  \newcommand{\dateiname}{L03759}\newcommand{\titel}{Olga Schnitzler an Stefan Zweig, 21. 12. 1916}\newcommand{\editorInnen}{Selma Jahnke und Martin Anton Müller}%% latex-leseansicht-abspann.tex
%% Abspann für die Leseansicht.
%% Der Schalter \ifkorrekturansicht ist bereits durch den Vorspann gesetzt.

%% latex-abspann.tex
%% Gemeinsamer Abspann für Korrekturansicht und Leseansicht.
%% Setzt den Schalter \ifkorrekturansicht voraus (gesetzt in den
%% einbindenden Dateien latex-korrekturansicht-abspann.tex bzw.
%% latex-leseansicht-abspann.tex).
%% ---------------------------------------------------------------

\normalsize

% Das esempio-Environment wird nur in der Leseansicht benötigt
\ifkorrekturansicht\else
\newenvironment{esempio}[3]%
{
    \vspace{1.5ex}
    \rlap{\underline{#1}}
    \par
    \setlength{\parindent}{0cm}
    \nopagebreak
    \leftskip=#2cm
    \rightskip=#3cm
}
{
    \par
}
\fi

\doendnotes{C}
\bigskip
\vfill

\clearpage

\footnotesize

\ifkorrekturansicht
  \lohead{\textsc{register}}
\fi

% theindex-Environment neu definieren ohne reledmac
\makeatletter
\renewenvironment{theindex}{%
  \ifkorrekturansicht
    \section*{\indexname}%
  \else
    \subsubsection*{Index der erwähnten Entitäten}%
  \fi
  \setlength{\parindent}{0pt}%
  \setlength{\parskip}{0pt plus 0.3pt}%
  \let\item\@idxitem
}{%
  \ifkorrekturansicht\clearpage\fi
}
\makeatother

\IfFileExists{\jobname-pw.ind}{\input{\jobname-pw.ind}}{}

% Quellenangabe nur in der Leseansicht
\ifkorrekturansicht\else
% Fallback-Definitionen, falls die .tex-Datei \titel etc. nicht gesetzt hat
\providecommand{\titel}{}
\providecommand{\editorInnen}{}
\providecommand{\dateiname}{\jobname}

\vspace{3cm}

\vfill

\footnotesize
\textsc{Quelle}: \titel. Herausgegeben von {\editorInnen}. In: \emph{Arthur Schnitzler: Briefwechsel mit Autorinnen und Autoren}.
 Digitale Edition, https://schnitzler-briefe.acdh.oeaw.ac.at/{\dateiname}.html (Stand \today)
\fi

\end{document}


