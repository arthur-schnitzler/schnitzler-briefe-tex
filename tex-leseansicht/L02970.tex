%% latex-korrekturansicht-vorspann.tex
%% Vorspann für die Korrekturansicht.
%% Lädt die gemeinsame Datei latex-vorspann.tex mit gesetztem Schalter.

\newif\ifkorrekturansicht
\korrekturansichttrue

\input{../tex-inputs/latex-vorspann}


\section[ Arthur Schnitzler an Felix Salten, 16. 9. 1901]{L02970 Arthur Schnitzler an Felix Salten, 16. 9. 1901}
\nopagebreak\mylabel{L02970v}
\rehead{ }\normalsize\beginnumbering\briefempfaengerindex{Salten, Felix@\textsc{Salten, Felix}!zzzSchnitzler, Arthur@\emph{von Arthur Schnitzler}!1901-09-161@{16. 9. 1901}|(be}
\toendnotes[C]{\smallbreak\pagebreak[2]}\Standort{Wienbibliothek im Rathaus, ZPH 1681, 2.1.516.}
\physDesc{Brief, 1 Blatt, 2 Seiten, 359 Zeichen
\newline{}Handschrift: Bleistift, deutsche Kurrent
\newline{}Ordnung: mit Bleistift von unbekannter Hand nummeriert: »23« }\toendnotes[C]{\smallbreak}
\pstart
           \raggedleft{}{\pb}16. \substVorne{}\textsuperscript{1}\substDazwischen{}9\substHinten{}. 901\pend
           \vspace{0.5em}
\pstart
           Lieber Freund, der kleine Herr \label{K_L02970-1v}\edtext{Lanz\pwindex{Lantz, Adolf 10.11.1882 – 19.08.1949@\textsc{Lantz, Adolf} (10.11.1882 – 19.08.1949), \emph{Schriftsteller/Schriftstellerin, Theaterleiter/Theaterleiterin, Dramaturg/Dramaturgin}|pwu}}{\lemma{\textnormal{\emph{Lanz}}}\Cendnote{\textnormal{Adolf Lantz\pwindex{Lantz, Adolf 10.11.1882 – 19.08.1949@\textsc{Lantz, Adolf} (10.11.1882 – 19.08.1949), \emph{Schriftsteller/Schriftstellerin, Theaterleiter/Theaterleiterin, Dramaturg/Dramaturgin}|pwk}?}}}\label{K_L02970-1}, der Ihnen \label{K_L02970-2v}\edtext{ſ. Z.}{\lemma{\textnormal{\emph{ſ. Z.}}}\Cendnote{\textnormal{seiner Zeit}}}\label{K_L02970-2} einige Manuſcripte überreicht laßt Sie
               durch mich bitten, diese Manuscripte bei Ihrem Hausmeiſter\pwindex{?? [Hausmeister von Felix Salten in der Kochgasse 1901] @\textsc{?? [Hausmeister von Felix Salten in der Kochgasse 1901]}|pwv} zu \substVorne{}\textsuperscript{über}\substDazwischen{}hinter\substHinten{}legen, wo er ſie ſich abholen möchte. –\pend
           
\pstart
           Warum hab ich Sie a\textcolor{gray}{uch}{ }\label{K_L02970-3v}\edtext{Samſtag}{\lemma{\textnormal{\emph{Samſtag}}}\Cendnote{\textnormal{Vgl. Arthur Schnitzler an Felix Salten, [14. 9. 1901?].
               }}}\label{K_L02970-3} nicht geſehen? Sollten ſie ſchon im Club\orgindex{?? [Wiener Club September 1901]@?? [Wiener Club September 1901]|pwv}{ }{\pb}geweſen ſein? –\pend
           
\pstart
           Ich ſchreibe \label{K_L02970-4v}\edtext{2 Einakter\pwindex{Puppenspieler. Studie in einem Aufzuge@\emph{Der Puppenspieler. Studie in einem Aufzuge}|pwv}\pwindex{letzten Masken@\emph{Die letzten Masken}|pwv}}{\lemma{\textnormal{\emph{2 Einakter}}}\Cendnote{\textnormal{\emph{Der Puppenspieler}\pwindex{Puppenspieler. Studie in einem Aufzuge@\emph{Der Puppenspieler. Studie in einem Aufzuge}|pwk} und \emph{Die letzten Masken}\pwindex{letzten Masken@\emph{Die letzten Masken}|pwk}, vgl. A. S.: \emph{Tagebuch}, 16. 9. 1901.}}}\label{K_L02970-4}, die zu den 3 andren\pwindex{Literatur@\emph{Literatur}|pwv}\pwindex{Frau mit dem Dolche@\emph{Die Frau mit dem Dolche}|pwv}\pwindex{Lebendige Stunden@\emph{Lebendige Stunden}|pwv} gehören.\pend
           
\pstart
           Herzlichſt Ihr {\\[\baselineskip]}\spacefill\mbox{ArthSch}\pend
           \leftskip=0em{}\selectlanguage{ngerman}\endnumbering\briefempfaengerindex{Salten, Felix@\textsc{Salten, Felix}!zzzSchnitzler, Arthur@\emph{von Arthur Schnitzler}!1901-09-161@{16. 9. 1901}|)be}\mylabel{L02970h}  \normalsize

\doendnotes{C}
\bigskip
\vfill

\clearpage

\footnotesize

\lohead{\textsc{register}}

% Definiere theindex-Environment komplett neu ohne reledmac
\makeatletter
\renewenvironment{theindex}{%
  \section*{\indexname}%
  \setlength{\parindent}{0pt}%
  \setlength{\parskip}{0pt plus 0.3pt}%
  \let\item\@idxitem
}{%
  \clearpage
}
\makeatother

\IfFileExists{\jobname-pw.ind}{\input{\jobname-pw.ind}}{}

\end{document}

      