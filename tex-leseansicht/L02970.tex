%% latex-leseansicht-vorspann.tex
%% Vorspann für die Leseansicht.
%% Lädt die gemeinsame Datei latex-vorspann.tex mit nicht gesetztem Schalter.

\newif\ifkorrekturansicht
\korrekturansichtfalse

\input{../tex-inputs/latex-vorspann}


\section[ Arthur Schnitzler an Felix Salten, 16. 9. 1901]{L02970 Arthur Schnitzler an Felix Salten,  16. 9. 1901}
\nopagebreak\mylabel{L02970v}
\rehead{ }\normalsize\beginnumbering\briefempfaengerindex{Salten, Felix@\textsc{Salten, Felix}!zzzSchnitzler, Arthur@\emph{von Arthur Schnitzler}!1901-09-161@{16. 9. 1901}|(be}
\toendnotes[C]{\smallbreak\pagebreak[2]}
\correspDesc{Versand  durch Arthur Schnitzler am 16. 9. 1901 in Wien
\newline{}Erhalt  durch Felix Salten im Zeitraum [16. 9. 1901 – 19. 9. 1901?] in Wien}\toendnotes[C]{\smallbreak}
\Standort{Wienbibliothek im Rathaus, ZPH 1681, 2.1.516.}
\physDesc{Brief, 1 Blatt, 2 Seiten, 359 Zeichen
\newline{}Handschrift: Bleistift, deutsche Kurrent
\newline{}Ordnung: mit Bleistift von unbekannter Hand nummeriert: »23« }\toendnotes[C]{\smallbreak}
\pstart
           \raggedleft{}{\pb}16. \substVorne{}\textsuperscript{1}\substDazwischen{}9\substHinten{}. 901\pend
           \vspace{0.5em}
\pstart
           Lieber Freund, der kleine Herr \label{K_L02970-1v}\edtext{Lanz\pwindex{Lantz, Adolf 10.\,11.\,1882 Wien – 19.\,8.\,1949 London@\textsc{Lantz, Adolf} (10.\,11.\,1882 Wien – 19.\,8.\,1949 London), \emph{Schriftsteller, Theaterleiter, Dramaturg}|pwu}}{\lemma{\textnormal{\emph{Lanz}}}\Cendnote{\textnormal{Adolf Lantz\pwindex{Lantz, Adolf 10.\,11.\,1882 Wien – 19.\,8.\,1949 London@\textsc{Lantz, Adolf} (10.\,11.\,1882 Wien – 19.\,8.\,1949 London), \emph{Schriftsteller, Theaterleiter, Dramaturg}|pwk}?}}}\label{K_L02970-1}, der Ihnen \label{K_L02970-2v}\edtext{ſ. Z.}{\lemma{\textnormal{\emph{s. Z.}}}\Cendnote{\textnormal{seiner Zeit}}}\label{K_L02970-2} einige Manuſcripte überreicht laßt Sie
               durch mich bitten, diese Manuscripte bei Ihrem Hausmeiſter\pwindex{?? [Hausmeister von Felix Salten in der Kochgasse 1901] @\textsc{?? [Hausmeister von Felix Salten in der Kochgasse 1901]}|pwv} zu \substVorne{}\textsuperscript{über}\substDazwischen{}hinter\substHinten{}legen, wo er{ }ſie{ }ſich abholen möchte. –\pend
           
\pstart
           Warum hab ich Sie a\textcolor{gray}{uch}{ }\label{K_L02970-3v}\edtext{Samſtag}{\lemma{\textnormal{\emph{Samstag}}}\Cendnote{\textnormal{Vgl. XXXX Auszeichnungsfehler: Dokument L02968 nicht gefunden.
               }}}\label{K_L02970-3} nicht geſehen? Sollten{ }ſie{ }ſchon im Club\orgindex{?? [Wiener Club September 1901]@?? [Wiener Club September 1901]|pwv}{ }{\pb}geweſen{ }ſein? –\pend
           
\pstart
           Ich{ }ſchreibe \label{K_L02970-4v}\edtext{2 Einakter\pwindex{Schnitzler, Arthur 15.\,5.\,1862 Wien – 21.\,10.\,1931 ebd.@\textsc{Schnitzler, Arthur} (15.\,5.\,1862 Wien – 21.\,10.\,1931 ebd.), \emph{Schriftsteller, Mediziner}!Puppenspieler. Studie in einem Aufzuge@\strich\emph{Der Puppenspieler. Studie in einem Aufzuge}|pwv}\pwindex{Schnitzler, Arthur 15.\,5.\,1862 Wien – 21.\,10.\,1931 ebd.@\textsc{Schnitzler, Arthur} (15.\,5.\,1862 Wien – 21.\,10.\,1931 ebd.), \emph{Schriftsteller, Mediziner}!letzten Masken@\strich\emph{Die letzten Masken}|pwv}}{\lemma{\textnormal{\emph{2 Einakter}}}\Cendnote{\textnormal{\emph{Der Puppenspieler}\pwindex{Schnitzler, Arthur 15.\,5.\,1862 Wien – 21.\,10.\,1931 ebd.@\textsc{Schnitzler, Arthur} (15.\,5.\,1862 Wien – 21.\,10.\,1931 ebd.), \emph{Schriftsteller, Mediziner}!Puppenspieler. Studie in einem Aufzuge@\strich\emph{Der Puppenspieler. Studie in einem Aufzuge}|pwk} und \emph{Die letzten Masken}\pwindex{Schnitzler, Arthur 15.\,5.\,1862 Wien – 21.\,10.\,1931 ebd.@\textsc{Schnitzler, Arthur} (15.\,5.\,1862 Wien – 21.\,10.\,1931 ebd.), \emph{Schriftsteller, Mediziner}!letzten Masken@\strich\emph{Die letzten Masken}|pwk}, vgl. A. S.: \emph{Tagebuch}, 16. 9. 1901.}}}\label{K_L02970-4}, die zu den 3 andren\pwindex{Schnitzler, Arthur 15.\,5.\,1862 Wien – 21.\,10.\,1931 ebd.@\textsc{Schnitzler, Arthur} (15.\,5.\,1862 Wien – 21.\,10.\,1931 ebd.), \emph{Schriftsteller, Mediziner}!Literatur@\strich\emph{Literatur}|pwv}\pwindex{Schnitzler, Arthur 15.\,5.\,1862 Wien – 21.\,10.\,1931 ebd.@\textsc{Schnitzler, Arthur} (15.\,5.\,1862 Wien – 21.\,10.\,1931 ebd.), \emph{Schriftsteller, Mediziner}!Frau mit dem Dolche@\strich\emph{Die Frau mit dem Dolche}|pwv}\pwindex{Schnitzler, Arthur 15.\,5.\,1862 Wien – 21.\,10.\,1931 ebd.@\textsc{Schnitzler, Arthur} (15.\,5.\,1862 Wien – 21.\,10.\,1931 ebd.), \emph{Schriftsteller, Mediziner}!Lebendige Stunden@\strich\emph{Lebendige Stunden}|pwv} gehören.\pend
           
\pstart
           Herzlichſt Ihr {\\[\baselineskip]}\spacefill\mbox{ArthSch}\pend
           \leftskip=0em{}\selectlanguage{ngerman}\endnumbering\briefempfaengerindex{Salten, Felix@\textsc{Salten, Felix}!zzzSchnitzler, Arthur@\emph{von Arthur Schnitzler}!1901-09-161@{16. 9. 1901}|)be}\mylabel{L02970h}  \newcommand{\dateiname}{L02970}\newcommand{\titel}{Arthur Schnitzler an Felix Salten, 16. 9. 1901}\newcommand{\editorInnen}{Martin Anton Müller und Laura Untner}%% latex-leseansicht-abspann.tex
%% Abspann für die Leseansicht.
%% Der Schalter \ifkorrekturansicht ist bereits durch den Vorspann gesetzt.

%% latex-abspann.tex
%% Gemeinsamer Abspann für Korrekturansicht und Leseansicht.
%% Setzt den Schalter \ifkorrekturansicht voraus (gesetzt in den
%% einbindenden Dateien latex-korrekturansicht-abspann.tex bzw.
%% latex-leseansicht-abspann.tex).
%% ---------------------------------------------------------------

\normalsize

% Das esempio-Environment wird nur in der Leseansicht benötigt
\ifkorrekturansicht\else
\newenvironment{esempio}[3]%
{
    \vspace{1.5ex}
    \rlap{\underline{#1}}
    \par
    \setlength{\parindent}{0cm}
    \nopagebreak
    \leftskip=#2cm
    \rightskip=#3cm
}
{
    \par
}
\fi

\doendnotes{C}
\bigskip
\vfill

\clearpage

\footnotesize

\ifkorrekturansicht
  \lohead{\textsc{register}}
\fi

% theindex-Environment neu definieren ohne reledmac
\makeatletter
\renewenvironment{theindex}{%
  \ifkorrekturansicht
    \section*{\indexname}%
  \else
    \subsubsection*{Index der erwähnten Entitäten}%
  \fi
  \setlength{\parindent}{0pt}%
  \setlength{\parskip}{0pt plus 0.3pt}%
  \let\item\@idxitem
}{%
  \ifkorrekturansicht\clearpage\fi
}
\makeatother

\IfFileExists{\jobname-pw.ind}{\input{\jobname-pw.ind}}{}

% Quellenangabe nur in der Leseansicht
\ifkorrekturansicht\else
% Fallback-Definitionen, falls die .tex-Datei \titel etc. nicht gesetzt hat
\providecommand{\titel}{}
\providecommand{\editorInnen}{}
\providecommand{\dateiname}{\jobname}

\vspace{3cm}

\vfill

\footnotesize
\textsc{Quelle}: \titel. Herausgegeben von {\editorInnen}. In: \emph{Arthur Schnitzler: Briefwechsel mit Autorinnen und Autoren}.
 Digitale Edition, https://schnitzler-briefe.acdh.oeaw.ac.at/{\dateiname}.html (Stand \today)
\fi

\end{document}


