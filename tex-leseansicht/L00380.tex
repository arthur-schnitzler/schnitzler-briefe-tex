%% latex-korrekturansicht-vorspann.tex
%% Vorspann für die Korrekturansicht.
%% Lädt die gemeinsame Datei latex-vorspann.tex mit gesetztem Schalter.

\newif\ifkorrekturansicht
\korrekturansichttrue

\input{../tex-inputs/latex-vorspann}


\section[Arthur Schnitzler an Richard Beer-Hofmann, 9. 10. 1894]{L00380 Arthur Schnitzler an Richard Beer-Hofmann, 9. 10. 1894}
\nopagebreak\mylabel{L00380v}
\rehead{ }\normalsize\beginnumbering\briefempfaengerindex{Beer-Hofmann, Richard@\textsc{Beer-Hofmann, Richard}!zzzSchnitzler, Arthur@\emph{von Arthur Schnitzler}!1894-10-091@{9. 10. 1894}|(be}
\toendnotes[C]{\smallbreak\pagebreak[2]}\Standort{YCGL, MSS 31.}
\physDesc{Postkarte, 513 Zeichen
\newline{}Handschrift: 1) Bleistift, deutsche Kurrent\hspace{1em}2) Bleistift, lateinische Kurrent (\noindent{}Adresse)\hspace{1em}
\newline{}Versand: 1) Stempel: »\nobreak{}\oindex{IX., Alsergrund@\textbf{IX., Alsergrund}, \emph{A.ADM3}|pwk}Wien 9/3, 9. 10. 94, 5–6 N\nobreak{}«.   2) Stempel: »\nobreak{}\oindex{Rom@\textbf{Rom}, \emph{P.PPLC}|pwk}Roma, 11 10 {[}94{]}, 8 M\nobreak{}«. }
\buchAbdrucke{\weitereDrucke{Hermann Bahr, Arthur Schnitzler: \emph{Briefwechsel, Aufzeichnungen, Dokumente (1891–1931)}. Göttingen: \emph{Wallstein} 2018.} }\toendnotes[C]{\smallbreak}\pstart{}{\pb}Italien\oindex{Italien@\textbf{Italien}, \emph{A.PCLI}|pw}\pend{}\pstart{}Dr. Rich Beer-Hofmann\pend{}\pstart{}Rom\oindex{Rom@\textbf{Rom}, \emph{P.PPLC}|pw}\pend{}\pstart{}Hotel Quirinal\oindex{Hotel Quirinale@\textbf{Hotel Quirinale}, \emph{Hotel (K.HTL)}|pw}\pend{}{\bigskip}\vspace{1em}
\pstart
           {\pb}\textcolor{gray}{Wien}\oindex{Wien@\textbf{Wien}, \emph{A.ADM2}|pw}\pend
           
\pstart
           \raggedleft{}Dienstag, 9. 10. 94.\pend
           \vspace{0.5em}
\pstart
           Lieber Richard, bitte theilen Sie mir mit, ob Sie meinen Brief Rom\oindex{Rom@\textbf{Rom}, \emph{P.PPLC}|pw}{ }\textsc{a post. ferm} der »Lieber Bekannter« anfing, nicht erhalten
               haben. Und die 2 Karten nach Pallanza\oindex{Pallanza@\textbf{Pallanza}, \emph{P.PPL}|pw}? –\pend
           
\pstart
           \textsc{Bahr}\pwindex{Bahr, Hermann 19.07.1863 – 15.01.1934@\textsc{Bahr, Hermann} (19.07.1863 – 15.01.1934), \emph{Schriftsteller/Schriftstellerin, Kritiker/Kritikerin}|pw}: Wien, VIII \textsc{Lammgasse} 3\oindex{Lammgasse@\textbf{Lammgasse}, \emph{Straße (K.STR)}|pw}. Er hat ſich ſehr über Ihr Telegr. gefreut. Erſte Nu{\geminationm}er\orgindex{Zeit. Wiener Wochenschrift@Die Zeit. Wiener Wochenschrift|pwv} wohlgelungen. \textsc{\label{K_L00380-1v}\edtext{Helferich\pwindex{Heilbut, Emil 02.04.1861 – 16.02.1921@\textsc{Heilbut, Emil} (02.04.1861 – 16.02.1921), \emph{Journalist/Journalistin, Kunstschriftsteller/Kunstschriftstellerin, Sammler/Sammlerin}|pw}\pwindex{Schoene Frauen«@\emph{»Schöne Frauen«}|pwv}}{\lemma{\textnormal{\emph{Helferich}}}\Cendnote{\textnormal{Hermann Helferich\pwindex{Heilbut, Emil 02.04.1861 – 16.02.1921@\textsc{Heilbut, Emil} (02.04.1861 – 16.02.1921), \emph{Journalist/Journalistin, Kunstschriftsteller/Kunstschriftstellerin, Sammler/Sammlerin}|pwk}: \emph{»Schöne Frauen«}\pwindex{Schoene Frauen«@\emph{»Schöne Frauen«}|pwk}. In: \emph{Die
                           Zeit}\pwindex{Zeit. Wiener Wochenschrift@\emph{Die Zeit. Wiener Wochenschrift}|pwk}, Bd. 1, Nr. 1, 6. 10. 1894, S. 7–8.}}}\label{K_L00380-1}} famos; \label{K_L00380-2v}\edtext{\textsc{Bahr}\pwindex{Bahr, Hermann 19.07.1863 – 15.01.1934@\textsc{Bahr, Hermann} (19.07.1863 – 15.01.1934), \emph{Schriftsteller/Schriftstellerin, Kritiker/Kritikerin}|pw}’s Sachen}{\lemma{\textnormal{\emph{Bahr’s Sachen}}}\Cendnote{\textnormal{Bahr\pwindex{Bahr, Hermann 19.07.1863 – 15.01.1934@\textsc{Bahr, Hermann} (19.07.1863 – 15.01.1934), \emph{Schriftsteller/Schriftstellerin, Kritiker/Kritikerin}|pwk} hat, unter verschiedenen Namen und
                  Kürzeln, zwei Aufsätze, zwei Buch- und drei Theaterbesprechungen im ersten
                  Heft.}}}\label{K_L00380-2}, beſonders \label{K_L00380-3v}\edtext{Burgtheater\pwindex{Burgtheater [Das fuenfte Jahr]@\emph{Burgtheater [Das fünfte Jahr]}|pw}}{\lemma{\textnormal{\emph{Burgtheater}}}\Cendnote{\textnormal{Hermann Bahr\pwindex{Bahr, Hermann 19.07.1863 – 15.01.1934@\textsc{Bahr, Hermann} (19.07.1863 – 15.01.1934), \emph{Schriftsteller/Schriftstellerin, Kritiker/Kritikerin}|pwk}: \emph{Burgtheater}\pwindex{Burgtheater [Das fuenfte Jahr]@\emph{Burgtheater [Das fünfte Jahr]}|pwk}. In: \emph{Die
                        Zeit}\pwindex{Zeit. Wiener Wochenschrift@\emph{Die Zeit. Wiener Wochenschrift}|pwk}, Bd. 1, Nr. 1, 6. 10. 1894, S. 9–10.}}}\label{K_L00380-3}
               vorzüglich. –\pend
           
\pstart
           Schmetterlingsſchlacht\pwindex{Schmetterlingsschlacht. Komoedie in 4 Akten@\emph{Die Schmetterlingsschlacht. Komödie in 4 Akten}|pw} noch nicht geſehen, will
               Freitag gehen. – Schreiben Sie mehr, wann ko{\geminationm}en Sie?\pend
           \pstart Herzlichen Gruſs\hspace*{1.5em}Ihr \spacefill\mbox{Arthur}\pend{}\selectlanguage{ngerman}\endnumbering\briefempfaengerindex{Beer-Hofmann, Richard@\textsc{Beer-Hofmann, Richard}!zzzSchnitzler, Arthur@\emph{von Arthur Schnitzler}!1894-10-091@{9. 10. 1894}|)be}\mylabel{L00380h}  \normalsize

\doendnotes{C}
\bigskip
\vfill

\clearpage

\footnotesize

\lohead{\textsc{register}}

% Definiere theindex-Environment komplett neu ohne reledmac
\makeatletter
\renewenvironment{theindex}{%
  \section*{\indexname}%
  \setlength{\parindent}{0pt}%
  \setlength{\parskip}{0pt plus 0.3pt}%
  \let\item\@idxitem
}{%
  \clearpage
}
\makeatother

\IfFileExists{\jobname-pw.ind}{\input{\jobname-pw.ind}}{}

\end{document}

      