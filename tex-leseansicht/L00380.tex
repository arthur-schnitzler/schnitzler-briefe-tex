%% latex-leseansicht-vorspann.tex
%% Vorspann für die Leseansicht.
%% Lädt die gemeinsame Datei latex-vorspann.tex mit nicht gesetztem Schalter.

\newif\ifkorrekturansicht
\korrekturansichtfalse

\input{../tex-inputs/latex-vorspann}


               \section[Arthur Schnitzler an Richard Beer-Hofmann, 9. 10. 1894]{ Arthur Schnitzler an Richard Beer-Hofmann, 9. 10. 1894}\nopagebreak\mylabel{v}\rehead{ }\begin{ledgroupsized}[t]{13cm}\normalsize\beginnumbering\briefempfaengerindex{Beer-Hofmann, Richard@\textsc{Beer-Hofmann, Richard}!zzzSchnitzler, Arthur@\emph{von Arthur Schnitzler}!1894-10-091@{9. 10. 1894}|(be} \toendnotes[C]{\smallbreak\pagebreak[2]} \Standort{YCGL, MSS 31.}
\physDesc{Postkarte
\newline{}Handschrift: Bleistift, deutsche Kurrent\newline{}Versand: 1) Stempel: »\nobreak{}\oindex{IX., Alsergrund@\textbf{IX., Alsergrund}|pwk}Wien 9/3, 9. 10. 94, 5–6 N\nobreak{}«.  2) Stempel: »\nobreak{}\oindex{Rom@\textbf{Rom}|pwk}Roma, 11 10 {[}94{]}, 8 M\nobreak{}«. }\buchAbdrucke{\weitereDrucke{Hermann Bahr, Arthur Schnitzler: \emph{Briefwechsel, Aufzeichnungen, Dokumente
                                (1891–1931)}. Hg. Kurt Ifkovits und Martin Anton Müller. Göttingen: \emph{Wallstein} 2018.} }\toendnotes[C]{\smallbreak}\pstart{}{\pb}\textsc{Italien\oindex{Italien@\textbf{Italien}|pw}}\pend{}\pstart{}\textsc{Dr. Rich Beer-Hofmann}\pend{}\pstart{}\textsc{Rom\oindex{Rom@\textbf{Rom}|pw}}\pend{}\pstart{}\textsc{Hotel Quirinal\oindex{Hotel Quirinale@\textbf{Hotel Quirinale}|pw}}\pend{}{\bigskip}\pstart
           \noindent{}{\pb}\textcolor{gray}{Wien}\oindex{Wien@\textbf{Wien}|pw}\pend
           \pstart
           \raggedleft{}Dienstag, 9. 10. 94.\pend
           \pstart
           Lieber Richard, bitte theilen Sie mir mit, ob Sie meinen Brief
                        Rom\oindex{Rom@\textbf{Rom}|pw}{ }\textsc{a post. ferm} der »Lieber Bekannter« anfing, nicht
                    erhalten haben. Und die 2 Karten nach Pallanza\oindex{Pallanza@\textbf{Pallanza}|pw}? –\pend
           \pstart
           \textsc{Bahr}\pwindex{Bahr, Hermann 19.07.1863 – 15.01.1934@\textsc{Bahr, Hermann} (19.07.1863 – 15.01.1934), \emph{Schriftsteller, Kritiker}|pw}: Wien, VIII \textsc{Lammgasse} 3\oindex{Lammgasse@\textbf{Lammgasse}|pw}. Er hat ſich ſehr über Ihr Telegr.
                    gefreut. Erſte Nu{\geminationm}er\orgindex{Zeit. Wiener Wochenschrift@Die Zeit. Wiener Wochenschrift|pwv} wohlgelungen. \textsc{\label{K_L00380_1v}\edtext{Helferich\pwindex{Heilbut, Emil 02.04.1861 – 16.02.1921@\textsc{Heilbut, Emil} (02.04.1861 – 16.02.1921), \emph{Journalist, Kunstschriftsteller, Sammler}|pw}\pwindex{Heilbut, Emil 02.04.1861 – 16.02.1921@\textsc{Heilbut, Emil} (02.04.1861 – 16.02.1921), \emph{Journalist, Kunstschriftsteller, Sammler}!Schoene Frauen«06. 10. 1894@\strich\emph{»Schöne Frauen«} {[}06. 10. 1894{]}|pwv}}{\lemma{\textnormal{\emph{Helferich}}}\Cendnote{\textnormal{Hermann Helferich\pwindex{Heilbut, Emil 02.04.1861 – 16.02.1921@\textsc{Heilbut, Emil} (02.04.1861 – 16.02.1921), \emph{Journalist, Kunstschriftsteller, Sammler}|pwk}: \emph{»Schöne Frauen«}\pwindex{Heilbut, Emil 02.04.1861 – 16.02.1921@\textsc{Heilbut, Emil} (02.04.1861 – 16.02.1921), \emph{Journalist, Kunstschriftsteller, Sammler}!Schoene Frauen«06. 10. 1894@\strich\emph{»Schöne Frauen«} {[}06. 10. 1894{]}|pwk}. In: \emph{Die Zeit}\pwindex{Zeit. Wiener Wochenschrift1894 – 1904@\emph{Die Zeit. Wiener Wochenschrift}|pwk}, Bd. 1, Nr. 1,
                                    6. 10. 1894, S. 7–8.}}}\label{K_L00380_1h}} famos; \label{K_L00380_2v}\edtext{\textsc{Bahr}\pwindex{Bahr, Hermann 19.07.1863 – 15.01.1934@\textsc{Bahr, Hermann} (19.07.1863 – 15.01.1934), \emph{Schriftsteller, Kritiker}|pw}’s Sachen}{\lemma{\textnormal{\emph{Bahr’s Sachen}}}\Cendnote{\textnormal{Bahr\pwindex{Bahr, Hermann 19.07.1863 – 15.01.1934@\textsc{Bahr, Hermann} (19.07.1863 – 15.01.1934), \emph{Schriftsteller, Kritiker}|pwk} hat, unter verschiedenen Namen und Kürzeln,
                        zwei Aufsätze, zwei Buch- und drei Theaterbesprechungen im ersten
                        Heft.}}}\label{K_L00380_2h}, beſonders \label{K_L00380_3v}\edtext{Burgtheater\pwindex{Bahr, Hermann 19.07.1863 – 15.01.1934@\textsc{Bahr, Hermann} (19.07.1863 – 15.01.1934), \emph{Schriftsteller, Kritiker}!Burgtheater [Das fuenfte Jahr]06. 10. 1894@\strich\emph{Burgtheater [Das fünfte Jahr]} {[}06. 10. 1894{]}|pw}}{\lemma{\textnormal{\emph{Burgtheater}}}\Cendnote{\textnormal{Hermann Bahr\pwindex{Bahr, Hermann 19.07.1863 – 15.01.1934@\textsc{Bahr, Hermann} (19.07.1863 – 15.01.1934), \emph{Schriftsteller, Kritiker}|pwk}: \emph{Burgtheater}\pwindex{Bahr, Hermann 19.07.1863 – 15.01.1934@\textsc{Bahr, Hermann} (19.07.1863 – 15.01.1934), \emph{Schriftsteller, Kritiker}!Burgtheater [Das fuenfte Jahr]06. 10. 1894@\strich\emph{Burgtheater [Das fünfte Jahr]} {[}06. 10. 1894{]}|pwk}. In: \emph{Die Zeit}\pwindex{Zeit. Wiener Wochenschrift1894 – 1904@\emph{Die Zeit. Wiener Wochenschrift}|pwk}, Bd. 1, Nr. 1, 6. 10. 1894,
                            S. 9–10.}}}\label{K_L00380_3h} vorzüglich. –\pend
           \pstart
           Schmetterlingsſchlacht\pwindex{\textcolor{red}{\textsuperscript{XXXX1 indx}}!Schmetterlingsschlacht1894@\strich\emph{Die Schmetterlingsschlacht} {[}1894{]}|pw} noch nicht geſehen,
                    will Freitag gehen. – Schreiben Sie mehr, wann ko{\geminationm}en Sie?\pend
           \pstart Herzlichen Gruſs\hspace*{1.5em}Ihr
                        \spacefill\mbox{Arthur}\pend{}\endnumbering\briefempfaengerindex{Beer-Hofmann, Richard@\textsc{Beer-Hofmann, Richard}!zzzSchnitzler, Arthur@\emph{von Arthur Schnitzler}!1894-10-091@{9. 10. 1894}|)be}\mylabel{h}\end{ledgroupsized}  \newcommand{\dateiname}{L00380}\newcommand{\titel}{Arthur Schnitzler an Richard Beer-Hofmann, 9. 10. 1894}\newcommand{\editorInnen}{ Martin Anton Müller und Gerd-Hermann Susen}
            \footnotesize
\begin{ledgroupsized}[t]{11.5cm}
\doendnotes{C}
\end{ledgroupsized}
         %% latex-leseansicht-abspann.tex
%% Abspann für die Leseansicht.
%% Der Schalter \ifkorrekturansicht ist bereits durch den Vorspann gesetzt.

%% latex-abspann.tex
%% Gemeinsamer Abspann für Korrekturansicht und Leseansicht.
%% Setzt den Schalter \ifkorrekturansicht voraus (gesetzt in den
%% einbindenden Dateien latex-korrekturansicht-abspann.tex bzw.
%% latex-leseansicht-abspann.tex).
%% ---------------------------------------------------------------

\normalsize

% Das esempio-Environment wird nur in der Leseansicht benötigt
\ifkorrekturansicht\else
\newenvironment{esempio}[3]%
{
    \vspace{1.5ex}
    \rlap{\underline{#1}}
    \par
    \setlength{\parindent}{0cm}
    \nopagebreak
    \leftskip=#2cm
    \rightskip=#3cm
}
{
    \par
}
\fi

\doendnotes{C}
\bigskip
\vfill

\clearpage

\footnotesize

\ifkorrekturansicht
  \lohead{\textsc{register}}
\fi

% theindex-Environment neu definieren ohne reledmac
\makeatletter
\renewenvironment{theindex}{%
  \ifkorrekturansicht
    \section*{\indexname}%
  \else
    \subsubsection*{Index der erwähnten Entitäten}%
  \fi
  \setlength{\parindent}{0pt}%
  \setlength{\parskip}{0pt plus 0.3pt}%
  \let\item\@idxitem
}{%
  \ifkorrekturansicht\clearpage\fi
}
\makeatother

\IfFileExists{\jobname-pw.ind}{\input{\jobname-pw.ind}}{}

% Quellenangabe nur in der Leseansicht
\ifkorrekturansicht\else
% Fallback-Definitionen, falls die .tex-Datei \titel etc. nicht gesetzt hat
\providecommand{\titel}{}
\providecommand{\editorInnen}{}
\providecommand{\dateiname}{\jobname}

\vspace{3cm}

\vfill

\footnotesize
\textsc{Quelle}: \titel. Herausgegeben von {\editorInnen}. In: \emph{Arthur Schnitzler: Briefwechsel mit Autorinnen und Autoren}.
 Digitale Edition, https://schnitzler-briefe.acdh.oeaw.ac.at/{\dateiname}.html (Stand \today)
\fi

\end{document}


      