%% latex-leseansicht-vorspann.tex
%% Vorspann für die Leseansicht.
%% Lädt die gemeinsame Datei latex-vorspann.tex mit nicht gesetztem Schalter.

\newif\ifkorrekturansicht
\korrekturansichtfalse

\input{../tex-inputs/latex-vorspann}


\section[Arthur Schnitzler u. a. an Hermann Bahr, 14. 12. 1903]{L01351 Arthur Schnitzler u. a. an Hermann Bahr, 14. 12. 1903}
\nopagebreak\mylabel{L01351v}
\rehead{ }\normalsize\beginnumbering\briefempfaengerindex{Bahr, Hermann@\textsc{Bahr, Hermann}!zzzBeer-Hofmann, Richard@\emph{von Richard Beer-Hofmann}!1903-12-142@{14. 12. 1903}|(be}\briefempfaengerindex{Bahr, Hermann@\textsc{Bahr, Hermann}!zzzHofmannsthal, Gertrude von@\emph{von Gertrude von Hofmannsthal}!1903-12-142@{14. 12. 1903}|(be}\briefempfaengerindex{Bahr, Hermann@\textsc{Bahr, Hermann}!zzzHofmannsthal, Hugo von@\emph{von Hugo von Hofmannsthal}!1903-12-142@{14. 12. 1903}|(be}\briefempfaengerindex{Bahr, Hermann@\textsc{Bahr, Hermann}!zzzSchnitzler, Olga@\emph{von Olga Schnitzler}!1903-12-142@{14. 12. 1903}|(be}\briefempfaengerindex{Bahr, Hermann@\textsc{Bahr, Hermann}!zzzSchnitzler, Arthur@\emph{von Arthur Schnitzler}!1903-12-142@{14. 12. 1903}|(be}
\toendnotes[C]{\smallbreak\pagebreak[2]}
\correspDesc{Versand  durch Arthur Schnitzler, Olga Schnitzler, Hugo von Hofmannsthal, Gerty von Hofmannsthal, Richard Beer-Hofmann am 14. 12. 1903 in Wien
\newline{}Erhalt  durch Hermann Bahr am 15. 12. 03 in Wien}\toendnotes[C]{\smallbreak}
\Standort{TMW, HS AM 49103 Ba.}
\physDesc{Bildpostkarte, 142 Zeichen
\newline{}Handschrift Arthur Schnitzler: Bleistift, deutsche Kurrent
\newline{}Handschrift Olga Schnitzler: Bleistift
\newline{}Handschrift Hugo von Hofmannsthal: Bleistift, lateinische Kurrent
\newline{}Handschrift Gertrude von Hofmannsthal: Bleistift
\newline{}Handschrift Richard Beer-Hofmann: Bleistift
\newline{}Versand: 1) Stempel: »\nobreak{}\oindex{XVIII., Währing@\textbf{XVIII., Währing}, \emph{Verwaltungsgebiet}|pwk}18/1 Wien, 15. 12. 03, 6–7 V\nobreak{}«.   2) Stempel: »\nobreak{}Bestellt, \oindex{XIII., Hietzing@\textbf{XIII., Hietzing}, \emph{Verwaltungsgebiet}|pwk}Wien 13/7, 15{[}.{]} 12. 03, 11.\nobreak{}«. 
\newline{}Ordnung: Lochung }
\buchAbdrucke{\weitereDrucke{1) Hugo von Hofmannsthal, Gerty von Hofmannsthal, Hermann Bahr: \emph{Briefwechsel 1891–1934}. Herausgegeben und kommentiert von Elsbeth Dangel-Pelloquin. Göttingen: \emph{Wallstein} 2013, S. 229.} \weitereDrucke{2) Hermann Bahr, Arthur Schnitzler: \emph{Briefwechsel, Aufzeichnungen, Dokumente (1891–1931)}. Herausgegeben von Kurt Ifkovits und Martin Anton Müller. Göttingen: \emph{Wallstein} 2018, S. 286.} }\toendnotes[C]{\smallbreak}\pstart{}{\pb}Herrn Hermann
                  Bahr\pend{}\pstart{}Wien Ob St Veit\oindex{Wien@\textbf{Wien}!XIII., Hietzing@\textbf{XIII., Hietzing}!Ober Sankt Veit@\textbf{Ober Sankt Veit}, \emph{Ehemaliger Ort}|pw}\pend{}\pstart{}Veitliſſengaſſe\oindex{Wien@\textbf{Wien}!XIII., Hietzing@\textbf{XIII., Hietzing}!Veitlissengasse@\textbf{Veitlissengasse}, \emph{Straße}|pw}\pend{}{\bigskip}
\pstart
           \noindent{}\centering{}{\pb}\textcolor{gray}{\textbf{Hietzing\oindex{XIV., Penzing@\textbf{XIV., Penzing}, \emph{Verwaltungsgebiet}|pw}, den {\dots}
                  19{\dots}}}\pend
           
\pstart
           \centering{}\textcolor{gray}{\textbf{Auhofstrasse 1\oindex{Wien@\textbf{Wien}!XIII., Hietzing@\textbf{XIII., Hietzing}!Auhofstraße@\textbf{Auhofstraße}, \emph{Straße}|pw}.}}\pend
           
\pstart
           \textcolor{gray}{\textbf{Schöner Gruss vom reizenden Etablissement ›Ottakringerbräu\oindex{Wien@\textbf{Wien}!XIII., Hietzing@\textbf{XIII., Hietzing}!Ottakringer Bräu@\textbf{Ottakringer Bräu}, \emph{Bierhaus}|pw}‹ ›zum schwarzen Hahn‹, wo es so gemüthlich
                  ist.}}\pend
           \vspace{1em}
\pstart
           \noindent{}{\pb}Dem Meister\pwindex{Bahr, Hermann 19.\,7.\,1863 Linz – 15.\,1.\,1934 München@\textsc{Bahr, Hermann} (19.\,7.\,1863 Linz – 15.\,1.\,1934 München), \emph{Schriftsteller, Kritiker}!Meister. Komödie in drei Akten@\strich\emph{Der Meister. Komödie in drei Akten}|pw}\pwindex{Bahr, Hermann 19.\,7.\,1863 Linz – 15.\,1.\,1934 München@\textsc{Bahr, Hermann} (19.\,7.\,1863 Linz – 15.\,1.\,1934 München), \emph{Schriftsteller, Kritiker}!Meister. Komödie in drei Akten@\strich\emph{Der Meister. Komödie in drei Akten}|pw}. – 10{\%}!\pend
           
\pstart
           14. 12. 03\pend
           
\pstart
           \label{K_L01351-1v}\edtext{Unter sich\pwindex{Bahr, Hermann 19.\,7.\,1863 Linz – 15.\,1.\,1934 München@\textsc{Bahr, Hermann} (19.\,7.\,1863 Linz – 15.\,1.\,1934 München), \emph{Schriftsteller, Kritiker}!Unter sich. Ein Arme-Leut’-Stück@\strich\emph{Unter sich. Ein Arme-Leut’-Stück}|pw}}{\lemma{\textnormal{\emph{Unter sich}}}\Cendnote{\textnormal{Die \emph{11
                     Scharfrichter}\orgindex{elf Scharfrichter@Die elf Scharfrichter|pwk} hatten Bahrs\pwindex{Bahr, Hermann 19.\,7.\,1863 Linz – 15.\,1.\,1934 München@\textsc{Bahr, Hermann} (19.\,7.\,1863 Linz – 15.\,1.\,1934 München), \emph{Schriftsteller, Kritiker}|pwk}{ }Schauspiel \emph{Unter
                     sich}\pwindex{Bahr, Hermann 19.\,7.\,1863 Linz – 15.\,1.\,1934 München@\textsc{Bahr, Hermann} (19.\,7.\,1863 Linz – 15.\,1.\,1934 München), \emph{Schriftsteller, Kritiker}!Unter sich. Ein Arme-Leut’-Stück@\strich\emph{Unter sich. Ein Arme-Leut’-Stück}|pwk} am 9. 12. 1903 in Wien\oindex{Wien@\textbf{Wien}, \emph{Verwaltungsgebiet}|pwk} gegeben.}}}\label{K_L01351-1}:\pend
           \pstart \spacefill\mbox{Arthur}\pend{}
\pstart
           \noindent{}\spacefill\mbox{{[}hs. Schnitzler:{]} Olga S.}{\\}\spacefill\mbox{{[}hs. Hofmannsthal:{]} Hugo (ich bin nicht der \label{K_L01351-2v}\edtext{Onkel}{\lemma{\textnormal{\emph{Onkel}}}\Cendnote{\textnormal{Figur aus \emph{Unter sich}\pwindex{Bahr, Hermann 19.\,7.\,1863 Linz – 15.\,1.\,1934 München@\textsc{Bahr, Hermann} (19.\,7.\,1863 Linz – 15.\,1.\,1934 München), \emph{Schriftsteller, Kritiker}!Unter sich. Ein Arme-Leut’-Stück@\strich\emph{Unter sich. Ein Arme-Leut’-Stück}|pwk}.}}}\label{K_L01351-2}\pwindex{Bahr, Hermann 19.\,7.\,1863 Linz – 15.\,1.\,1934 München@\textsc{Bahr, Hermann} (19.\,7.\,1863 Linz – 15.\,1.\,1934 München), \emph{Schriftsteller, Kritiker}!Unter sich. Ein Arme-Leut’-Stück@\strich\emph{Unter sich. Ein Arme-Leut’-Stück}|pwv})}{\\}\spacefill\mbox{{[}hs. Hofmannsthal:{]} Gerty}{\\}\spacefill\mbox{{[}hs. Beer-Hofmann:{]} Richard}\pend
           \selectlanguage{ngerman}\endnumbering\briefempfaengerindex{Bahr, Hermann@\textsc{Bahr, Hermann}!zzzBeer-Hofmann, Richard@\emph{von Richard Beer-Hofmann}!1903-12-142@{14. 12. 1903}|)be}\briefempfaengerindex{Bahr, Hermann@\textsc{Bahr, Hermann}!zzzHofmannsthal, Gertrude von@\emph{von Gertrude von Hofmannsthal}!1903-12-142@{14. 12. 1903}|)be}\briefempfaengerindex{Bahr, Hermann@\textsc{Bahr, Hermann}!zzzHofmannsthal, Hugo von@\emph{von Hugo von Hofmannsthal}!1903-12-142@{14. 12. 1903}|)be}\briefempfaengerindex{Bahr, Hermann@\textsc{Bahr, Hermann}!zzzSchnitzler, Olga@\emph{von Olga Schnitzler}!1903-12-142@{14. 12. 1903}|)be}\briefempfaengerindex{Bahr, Hermann@\textsc{Bahr, Hermann}!zzzSchnitzler, Arthur@\emph{von Arthur Schnitzler}!1903-12-142@{14. 12. 1903}|)be}\mylabel{L01351h}  \newcommand{\dateiname}{L01351}\newcommand{\titel}{Arthur Schnitzler u. a. an Hermann Bahr, 14. 12. 1903}\newcommand{\editorInnen}{Herausgegeben von Martin Anton Müller}%% latex-leseansicht-abspann.tex
%% Abspann für die Leseansicht.
%% Der Schalter \ifkorrekturansicht ist bereits durch den Vorspann gesetzt.

%% latex-abspann.tex
%% Gemeinsamer Abspann für Korrekturansicht und Leseansicht.
%% Setzt den Schalter \ifkorrekturansicht voraus (gesetzt in den
%% einbindenden Dateien latex-korrekturansicht-abspann.tex bzw.
%% latex-leseansicht-abspann.tex).
%% ---------------------------------------------------------------

\normalsize

% Das esempio-Environment wird nur in der Leseansicht benötigt
\ifkorrekturansicht\else
\newenvironment{esempio}[3]%
{
    \vspace{1.5ex}
    \rlap{\underline{#1}}
    \par
    \setlength{\parindent}{0cm}
    \nopagebreak
    \leftskip=#2cm
    \rightskip=#3cm
}
{
    \par
}
\fi

\doendnotes{C}
\bigskip
\vfill

\clearpage

\footnotesize

\ifkorrekturansicht
  \lohead{\textsc{register}}
\fi

% theindex-Environment neu definieren ohne reledmac
\makeatletter
\renewenvironment{theindex}{%
  \ifkorrekturansicht
    \section*{\indexname}%
  \else
    \subsubsection*{Index der erwähnten Entitäten}%
  \fi
  \setlength{\parindent}{0pt}%
  \setlength{\parskip}{0pt plus 0.3pt}%
  \let\item\@idxitem
}{%
  \ifkorrekturansicht\clearpage\fi
}
\makeatother

\IfFileExists{\jobname-pw.ind}{\input{\jobname-pw.ind}}{}

% Quellenangabe nur in der Leseansicht
\ifkorrekturansicht\else
% Fallback-Definitionen, falls die .tex-Datei \titel etc. nicht gesetzt hat
\providecommand{\titel}{}
\providecommand{\editorInnen}{}
\providecommand{\dateiname}{\jobname}

\vspace{3cm}

\vfill

\footnotesize
\textsc{Quelle}: \titel. Herausgegeben von {\editorInnen}. In: \emph{Arthur Schnitzler: Briefwechsel mit Autorinnen und Autoren}.
 Digitale Edition, https://schnitzler-briefe.acdh.oeaw.ac.at/{\dateiname}.html (Stand \today)
\fi

\end{document}


