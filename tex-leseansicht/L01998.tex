%% latex-leseansicht-vorspann.tex
%% Vorspann für die Leseansicht.
%% Lädt die gemeinsame Datei latex-vorspann.tex mit nicht gesetztem Schalter.

\newif\ifkorrekturansicht
\korrekturansichtfalse

\input{../tex-inputs/latex-vorspann}


         
         \renewcommand{\erwaehntePersonen}{Personen: Franz Blei, Georg von Lukács, Charles-Louis Philippe}
         \renewcommand{\erwaehnteInstitutionen}{Institutionen: Neue Freie Presse}
         \renewcommand{\erwaehnteOrte}{Orte: Carrara, Florenz, Forte dei Marmi, Lido, Massa, München, Pietrasanta, Pisa, Tengstraße, Viareggio, Wien}
         \renewcommand{\erwaehnteWerke}{Werke: Die neue Rundschau, Über Sehnsucht und Form}
               \section[Franz Blei an Arthur Schnitzler, {[}Anfang? Januar 1911{]}]{ Franz Blei an Arthur Schnitzler, {[}Anfang? Januar 1911{]}}\nopagebreak\mylabel{v}\rehead{ }\begin{ledgroupsized}[t]{13cm}\normalsize\beginnumbering\briefempfaengerindex{Schnitzler, Arthur@\textsc{Schnitzler, Arthur}!zzzBlei, Franz@\emph{von Franz Blei}!1911-01-011@{{[}Anfang? Januar 1911{]}}|(be} \toendnotes[C]{\smallbreak\pagebreak[2]} \Standort{CUL, Schnitzler, B 14.}
\physDesc{Brief, 1 Blatt, 2 Seiten, 1068 Zeichen
\newline{}Handschrift: grüne Tinte, lateinische Kurrent
\newline{}Schnitzler: mit Bleistift datiert: »Jänner 911« 
\newline{}Ordnung: mit Bleistift von unbekannter Hand nummeriert:
                                 »7« }\toendnotes[C]{\smallbreak}\pstart
           \noindent{}{\pb}\textcolor{gray}{\textbf{DR. FRANZ BLEI}}\hfill \textcolor{gray}{\textbf{MÜNCHEN}}\oindex{Muenchen@\textbf{München}|pw}\pend
           \pstart
           \raggedleft{}\textcolor{gray}{\textbf{Tengstraße 41}}\oindex{Tengstrasse@\textbf{Tengstraße}|pw}\pend
           \pstart{}Verehrter Herr Doktor,\pend\pstart
           den \label{K_L01998-1v}\edtext{Aufsatz\pwindex{Ueber Sehnsucht und Form01. 02. 1911@\emph{Über Sehnsucht und Form} {[}01. 02. 1911{]}|pwv}}{\lemma{\textnormal{\emph{Aufsatz}}}\Cendnote{\textnormal{Ein Teil davon erschien unmittelbar nach
                  dem Brief gedruckt: Georg von Lukacs\pwindex{Lukács, Georg von 13.04.1885 – 04.06.1971@\textsc{Lukács, Georg von} (13.04.1885 – 04.06.1971), \emph{Philosoph}|pwk}: \emph{Über Sehnsucht und Form}\pwindex{Ueber Sehnsucht und Form01. 02. 1911@\emph{Über Sehnsucht und Form} {[}01. 02. 1911{]}|pwk}. In: \emph{Die neue Rundschau}\pwindex{?? Werk@Nicht ermittelte Verfasserinnen und Verfasser!neue Rundschau1904@\emph{Die neue Rundschau} {[}1904{]}|pwk}, Jg. 22, H. 2, Februar
                     1911, S. 192–198.
               }}}\label{K_L01998-1h} über Ch. L. Philippe\pwindex{Philippe, Charles-Louis 04.08.1874 – 21.12.1909@\textsc{Philippe, Charles-Louis} (04.08.1874 – 21.12.1909), \emph{Schriftsteller}|pw}, von D\textsuperscript{r} Georg von \label{T_L01998-1v}\edtext{Lukacs}{\lemma{\textnormal{\emph{Lukacs}}}\Cendnote{\textnormal{korrigiert aus: »Lukasc«}}}\label{T_L01998-1h}\pwindex{Lukács, Georg von 13.04.1885 – 04.06.1971@\textsc{Lukács, Georg von} (13.04.1885 – 04.06.1971), \emph{Philosoph}|pw} schicke ich Ihnen, sowie ich ihn \label{K_L01998-2v}\edtext{zurück{[}be{]}komme}{\lemma{\textnormal{\emph{zurückbekomme}}}\Cendnote{\textnormal{Zwei Briefe von Blei\pwindex{Blei, Franz 18.01.1871 – 10.07.1942@\textsc{Blei, Franz} (18.01.1871 – 10.07.1942), \emph{Schriftsteller}|pwk} an Lukács\pwindex{Lukács, Georg von 13.04.1885 – 04.06.1971@\textsc{Lukács, Georg von} (13.04.1885 – 04.06.1971), \emph{Philosoph}|pwk} lassen diesen Brief näher eingrenzen.
                  Am 26. 12. 1909 schrieb Blei\pwindex{Blei, Franz 18.01.1871 – 10.07.1942@\textsc{Blei, Franz} (18.01.1871 – 10.07.1942), \emph{Schriftsteller}|pwk},
                     Schnitzler\pwindex{Schnitzler, Arthur 15.05.1862 – 21.10.1931@\textsc{Schnitzler, Arthur} (15.05.1862 – 21.10.1931), \emph{Schriftsteller, Mediziner}|pwk} habe ihn bei einem Treffen in
                     München\oindex{Muenchen@\textbf{München}|pwk} um den Text gebeten (Georg Lukács\pwindex{Lukács, Georg von 13.04.1885 – 04.06.1971@\textsc{Lukács, Georg von} (13.04.1885 – 04.06.1971), \emph{Philosoph}|pwk}: \emph{Briefwechsel 1902–1917}. Herausgegeben von Éva Karádi und Éva Fekete. Stuttgart:
                        \emph{Metzler}{ }1982, S. 189). Am 6. 1. 1910 schrieb Lukács\pwindex{Lukács, Georg von 13.04.1885 – 04.06.1971@\textsc{Lukács, Georg von} (13.04.1885 – 04.06.1971), \emph{Philosoph}|pwk}, er nehme die Vermittlung zur \emph{Neuen Freien Presse}\orgindex{Neue Freie Presse@Neue Freie Presse|pwk} an (ebd.,
                     S. 196). Da eine solche Vermittlung nicht stattgefunden hat,
                  der Text aber schon ab Mitte des Monats für \emph{Die
                     neue Rundschau}\pwindex{?? Werk@Nicht ermittelte Verfasserinnen und Verfasser!neue Rundschau1904@\emph{Die neue Rundschau} {[}1904{]}|pwk} blockiert gewesen sein musste, ist der Brief davor
                  anzusiedeln.}}}\label{K_L01998-2h} – ich fürchte zwar, er wird für die N. F. P.\orgindex{Neue Freie Presse@Neue Freie Presse|pw} zu lang sein, so etwa 10 Spalten. Aber, er wird doch Sie
               interessiren.\pend
           \pstart
           An Forte dei Marmi\oindex{Forte dei Marmi@\textbf{Forte dei Marmi}|pw} will ich Sie noch erinnern.
               Weg: Florenz\oindex{Florenz@\textbf{Florenz}|pw}–Pisa\oindex{Pisa@\textbf{Pisa}|pw}–Pietrasanta\oindex{Pietrasanta@\textbf{Pietrasanta}|pw}. Von da im Wägelchen.
               Ein Haus (5 Zimmer) mit Garten kostet für die Saison (ein Wort zu grossartig für das
               ganz unluxuriöse Forte\oindex{Forte dei Marmi@\textbf{Forte dei Marmi}|pw}), d. h.
                  1. Juni bis Ende September 400–500 francs. Die \label{K_L01998-3v}\edtext{Capana}{\lemma{\textnormal{\emph{Capana}}}\Cendnote{\textnormal{italienisch capanna: Hütte}}}\label{K_L01998-3h} für diese Zeit etwa 80 frs.
               Die Person, die kommt, um einzukaufen, zu kochen, aufzuräumen, bekommt 1 Lira pro Tag
               – wenn sie im Haus schläft 20 frs im Monat. {\pb}Der sehr schöne Strand ist 4–5 Stunden
               lang, reicht von Viareggio\oindex{Viareggio@\textbf{Viareggio}|pw} bis Massa Carrara\oindex{Massa@\textbf{Massa}|pw}. Es giebt Wälder und die sehr
               schönen Carrara\oindex{Carrara@\textbf{Carrara}|pw}berge. Es regnet so gut wie nie
               und die Wärme ist immer erträglich. –\pend
           \pstart
           Pensionen nehmen 7 frs pro Tag den erwachsenen Menschen, Kinder 3 frs.\pend
           \pstart
           Es ist sehr schön, sehr still da und sehr viel Raum. An den Lido\oindex{Lido@\textbf{Lido}|pw} dürfen Sie nicht denken.\pend
           \pstart
           Das ist alles was über Forte\oindex{Forte dei Marmi@\textbf{Forte dei Marmi}|pw} zu sagen ist.\pend
           \pstart
           Herzlich \textcolor{gray}{Ihr} ergebener{\\[\baselineskip]}\spacefill\mbox{Frz Blei}\pend
           \leftskip=0em{}
         
         \endnumbering\mylabel{h}\end{ledgroupsized}  \newcommand{\dateiname}{L01998}\newcommand{\titel}{Franz Blei an Arthur Schnitzler, [Anfang? Januar 1911]}\newcommand{\editorInnen}{Martin Anton Müller und Gerd-Hermann Susen}%% latex-leseansicht-abspann.tex
%% Abspann für die Leseansicht.
%% Der Schalter \ifkorrekturansicht ist bereits durch den Vorspann gesetzt.

%% latex-abspann.tex
%% Gemeinsamer Abspann für Korrekturansicht und Leseansicht.
%% Setzt den Schalter \ifkorrekturansicht voraus (gesetzt in den
%% einbindenden Dateien latex-korrekturansicht-abspann.tex bzw.
%% latex-leseansicht-abspann.tex).
%% ---------------------------------------------------------------

\normalsize

% Das esempio-Environment wird nur in der Leseansicht benötigt
\ifkorrekturansicht\else
\newenvironment{esempio}[3]%
{
    \vspace{1.5ex}
    \rlap{\underline{#1}}
    \par
    \setlength{\parindent}{0cm}
    \nopagebreak
    \leftskip=#2cm
    \rightskip=#3cm
}
{
    \par
}
\fi

\doendnotes{C}
\bigskip
\vfill

\clearpage

\footnotesize

\ifkorrekturansicht
  \lohead{\textsc{register}}
\fi

% theindex-Environment neu definieren ohne reledmac
\makeatletter
\renewenvironment{theindex}{%
  \ifkorrekturansicht
    \section*{\indexname}%
  \else
    \subsubsection*{Index der erwähnten Entitäten}%
  \fi
  \setlength{\parindent}{0pt}%
  \setlength{\parskip}{0pt plus 0.3pt}%
  \let\item\@idxitem
}{%
  \ifkorrekturansicht\clearpage\fi
}
\makeatother

\IfFileExists{\jobname-pw.ind}{\input{\jobname-pw.ind}}{}

% Quellenangabe nur in der Leseansicht
\ifkorrekturansicht\else
% Fallback-Definitionen, falls die .tex-Datei \titel etc. nicht gesetzt hat
\providecommand{\titel}{}
\providecommand{\editorInnen}{}
\providecommand{\dateiname}{\jobname}

\vspace{3cm}

\vfill

\footnotesize
\textsc{Quelle}: \titel. Herausgegeben von {\editorInnen}. In: \emph{Arthur Schnitzler: Briefwechsel mit Autorinnen und Autoren}.
 Digitale Edition, https://schnitzler-briefe.acdh.oeaw.ac.at/{\dateiname}.html (Stand \today)
\fi

\end{document}


      