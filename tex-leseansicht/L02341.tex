\input{../tex-inputs/latex-pdf-vorspann}
\begin{center}
            \textcolor{red}{ENTWURF. ENTZIFFERUNG NOCH NICHT KORREKTURGELESEN}
                      \end{center}
            
               \section[Arthur Schnitzler an Robert Adam, 12. 6. 1920]{ Arthur Schnitzler an Robert Adam, 12. 6. 1920}\nopagebreak\mylabel{v}\rehead{ }\begin{ledgroupsized}[t]{13cm}\normalsize\beginnumbering\briefempfaengerindex{Adam, Robert@\textsc{Adam, Robert}!zzzSchnitzler, Arthur@\emph{von Arthur Schnitzler}!1920-06-121@{12. 6. 1920}|(be} \toendnotes[C]{\smallbreak\pagebreak[2]} \Standort{DLA, 96.34.2/20.}
\physDesc{Postkarte
\newline{}Handschrift: schwarze Tinte, lateinische Kurrent\newline{}Versand: Stempel: »\nobreak{}Wien 66, 15. VI. 20, X\nobreak{}«.  }\toendnotes[C]{\smallbreak}\pstart{}{\pb}\textcolor{gray}{\textbf{D\textsuperscript{R}
                                ARTHUR SCHNITZLER}}\pend{}\pstart{}\textcolor{gray}{\textbf{WIEN, XVIII.
                                    STERNWARTESTRASSE 71\oindex{Sternwartestrasse@\textbf{Sternwartestraße}|pw}.}}\pend{}{\bigskip}\pstart{}Hrn Dr. Rob. Ad. Pollak\pend{}\pstart{}Ob-Landesger-Rath\pend{}\pstart{}Wien XII\oindex{XII., Meidling@\textbf{XII., Meidling}|pw}.\pend{}\pstart{}Meidlinger Hptstr. 58\oindex{Meidlinger Hauptstrasse@\textbf{Meidlinger Hauptstraße}|pw}.\pend{}{\bigskip}\pstart
           \raggedleft{}{\pb}12. 6. 1920\pend
           \pstart\center{}Verehrter Herr Doctor\pend\pstart
           Vielen Dank für die liebenswürdige Übersendg Ihres \label{K_L02341_1v}\edtext{Artikels über
                        Rechtsprincipien\pwindex{Adam, Robert 20.04.1877 – 16.10.1961@\textsc{Adam, Robert} (20.04.1877 – 16.10.1961), \emph{Schriftsteller, Richter}!Ueber Rechtsprinzipien. Eine analytische Untersuchung1920@\strich\emph{Über Rechtsprinzipien. Eine analytische Untersuchung} {[}1920{]}|pwv}}{\lemma{\textnormal{\emph{Artikels über Rechtsprincipien}}}\Cendnote{\textnormal{Robert Adam Pollak\pwindex{Adam, Robert 20.04.1877 – 16.10.1961@\textsc{Adam, Robert} (20.04.1877 – 16.10.1961), \emph{Schriftsteller, Richter}|pwk}: \emph{Ueber Rechtsprinzipien. Eine Analytische
                                Untersuchung}\pwindex{Adam, Robert 20.04.1877 – 16.10.1961@\textsc{Adam, Robert} (20.04.1877 – 16.10.1961), \emph{Schriftsteller, Richter}!Ueber Rechtsprinzipien. Eine analytische Untersuchung1920@\strich\emph{Über Rechtsprinzipien. Eine analytische Untersuchung} {[}1920{]}|pwk}. In: \emph{Archiv für
                                Rechts- und Wirtschaftsphilosophie}\pwindex{Archiv fuer Rechts- und Wirtschaftsphilosophie1907 – 1933@\emph{Archiv für Rechts- und Wirtschaftsphilosophie}|pwk}, Jg. 13, H. 2,
                                1919, S. 110–135. }}}\label{K_L02341_1h}, den ich mit starkem Interesse lese.\pend
           \pstart
           W\textcolor{gray}{o}\textcolor{gray}{l}eben Sie für den Sommer? Hab ich vielleicht wieder einmal
                    das Vergnügen Sie zu sehen?\pend
           \pstart
           Herzlichst grüsst Sie{\\[\baselineskip]}Ihr sehr ergeben\textcolor{gray}{er}{\\[\baselineskip]}\spacefill\mbox{ArthurSchnitzler}\pend
           \leftskip=0em{}\endnumbering\briefempfaengerindex{Adam, Robert@\textsc{Adam, Robert}!zzzSchnitzler, Arthur@\emph{von Arthur Schnitzler}!1920-06-121@{12. 6. 1920}|)be}\mylabel{h}\end{ledgroupsized}  \newcommand{\dateiname}{L02341}\newcommand{\titel}{Arthur Schnitzler an Robert Adam, 12. 6. 1920}\newcommand{\editorInnen}{Martin Anton Müller und Gerd-Hermann Susen}\input{../tex-inputs/latex-pdf-abspann}
      