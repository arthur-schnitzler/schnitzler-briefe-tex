%% latex-korrekturansicht-vorspann.tex
%% Vorspann für die Korrekturansicht.
%% Lädt die gemeinsame Datei latex-vorspann.tex mit gesetztem Schalter.

\newif\ifkorrekturansicht
\korrekturansichttrue

\input{../tex-inputs/latex-vorspann}


\section[Arthur Schnitzler an Robert Adam, 12. 6. 1920]{L02341 Arthur Schnitzler an Robert Adam, 12. 6. 1920}
\nopagebreak\mylabel{L02341v}
\rehead{ }\normalsize\beginnumbering\briefempfaengerindex{Adam, Robert@\textsc{Adam, Robert}!zzzSchnitzler, Arthur@\emph{von Arthur Schnitzler}!1920-06-121@{12. 6. 1920}|(be}
\toendnotes[C]{\smallbreak\pagebreak[2]}\Standort{DLA, 96.34.2/20.}
\physDesc{Postkarte, 363 Zeichen
\newline{}Handschrift: schwarze Tinte, lateinische Kurrent
\newline{}Versand: Stempel: »\nobreak{}Wien 66, 15. VI. 20, X\nobreak{}«.  }\toendnotes[C]{\smallbreak}\pstart{}{\pb}\textcolor{gray}{\textbf{D\textsuperscript{R} ARTHUR SCHNITZLER}}\pend{}\pstart{}\textcolor{gray}{\textbf{WIEN, XVIII. STERNWARTESTRASSE 71\oindex{Sternwartestrasse 71@\textbf{Sternwartestraße 71}, \emph{Wohngebäude (K.WHS)}|pw}.}}\pend{}{\bigskip}\pstart{}Hrn Dr. Rob. Ad. Pollak\pend{}\pstart{}Ob-Landesger-Rath\pend{}\pstart{}Wien XII\oindex{XII., Meidling@\textbf{XII., Meidling}, \emph{A.ADM3}|pw}.\pend{}\pstart{}Meidlinger Hptstr. 58\oindex{Meidlinger Hauptstrasse@\textbf{Meidlinger Hauptstraße}, \emph{Straße (K.STR)}|pw}.\pend{}{\bigskip}\vspace{1em}
\pstart
           \raggedleft{}{\pb}12. 6. 1920\pend
           
\pstart\center{}Verehrter Herr Doctor\pend\vspace{0.5em}
\pstart
           Vielen Dank für die liebenswürdige Übersendg Ihres \label{K_L02341-1v}\edtext{Artikels über Rechtsprincipien\pwindex{Ueber Rechtsprinzipien. Eine analytische Untersuchung@\emph{Über Rechtsprinzipien. Eine analytische Untersuchung}|pwv}}{\lemma{\textnormal{\emph{Artikels über Rechtsprincipien}}}\Cendnote{\textnormal{Robert Adam Pollak\pwindex{Adam, Robert 20.04.1877 – 16.10.1961@\textsc{Adam, Robert} (20.04.1877 – 16.10.1961), \emph{Schriftsteller/Schriftstellerin, Richter/Richterin}|pwk}: \emph{Ueber Rechtsprinzipien. Eine Analytische Untersuchung}\pwindex{Ueber Rechtsprinzipien. Eine analytische Untersuchung@\emph{Über Rechtsprinzipien. Eine analytische Untersuchung}|pwk}. In:
                        \emph{Archiv für Rechts- und
                        Wirtschaftsphilosophie}\pwindex{Archiv fuer Rechts- und Wirtschaftsphilosophie@\emph{Archiv für Rechts- und Wirtschaftsphilosophie}|pwk}, Jg. 13, H. 2, 1919, S. 110–135.
               }}}\label{K_L02341-1}, den ich mit starkem Interesse lese.\pend
           
\pstart
           W\textcolor{gray}{o}\textcolor{gray}{l}eben Sie für den Sommer? Hab ich vielleicht wieder einmal das
               Vergnügen Sie zu sehen?\pend
           
\pstart
           Herzlichst grüsst Sie{\\[\baselineskip]}Ihr sehr ergeben\textcolor{gray}{er}{\\[\baselineskip]}\spacefill\mbox{ArthurSchnitzler}\pend
           \leftskip=0em{}\selectlanguage{ngerman}\endnumbering\briefempfaengerindex{Adam, Robert@\textsc{Adam, Robert}!zzzSchnitzler, Arthur@\emph{von Arthur Schnitzler}!1920-06-121@{12. 6. 1920}|)be}\mylabel{L02341h}  \normalsize

\doendnotes{C}
\bigskip
\vfill

\clearpage

\footnotesize

\lohead{\textsc{register}}

% Definiere theindex-Environment komplett neu ohne reledmac
\makeatletter
\renewenvironment{theindex}{%
  \section*{\indexname}%
  \setlength{\parindent}{0pt}%
  \setlength{\parskip}{0pt plus 0.3pt}%
  \let\item\@idxitem
}{%
  \clearpage
}
\makeatother

\IfFileExists{\jobname-pw.ind}{\input{\jobname-pw.ind}}{}

\end{document}

      