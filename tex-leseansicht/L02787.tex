%% latex-leseansicht-vorspann.tex
%% Vorspann für die Leseansicht.
%% Lädt die gemeinsame Datei latex-vorspann.tex mit nicht gesetztem Schalter.

\newif\ifkorrekturansicht
\korrekturansichtfalse

\input{../tex-inputs/latex-vorspann}


         
         \newcommand{\erwaehntePersonen}{Personen:  ?? [Wirt im Café Tannhäuser], Henning von Brüsewitz, Jean-Louis Forain, Alix von Hessen-Darmstadt,  Nikolaus II. von Russland, Theodor Siepmann, Leopold Sonnemann, Jean Thorel, Leo Van-Jung}
         \newcommand{\erwaehnteInstitutionen}{Institutionen: Frankfurter Zeitung, Société des Auteurs et Compositeurs Dramatiques, Theater in der Josefstadt}
         \newcommand{\erwaehnteOrte}{Orte: Berlin, Café Tannhäuser, Frankreich, Kaiserstraße, Karlsruhe, Karlstraße, Paris, Russland, Wien, rue Feydeau}
         \newcommand{\erwaehnteWerke}{Werke: Amourette. Pièce en trois actes. Adaptée de Arthur Schnitzler, Frankfurter Zeitung, Liebelei. Schauspiel in drei Akten, Tages-Rundschau [Offizier hat Bürger niedergestochen]}
               \section[ Paul Goldmann an Arthur Schnitzler, 17. 10. {[}1896{]}]{ Paul Goldmann an Arthur Schnitzler, 17. 10. {[}1896{]}}\nopagebreak\mylabel{v}\rehead{ }\begin{ledgroupsized}[t]{13cm}\normalsize\beginnumbering \toendnotes[C]{\smallbreak\pagebreak[2]} \Standort{DLA, A:Schnitzler, HS.NZ85.1.3166.}
\physDesc{Brief, 1 Blatt, 3 Seiten
\newline{}Handschrift: blaue Tinte, deutsche Kurrent\newline{}Beilage: Zeitungsausschnitt, beschnitten, am Spaltenumbruch
                                 zusammengeklebt. Mit blauem Buntstift wurde der Beginn markiert.
                                 Die Rückseite ist offensichtlich nicht relevant. 
\newline{}Schnitzler: 1) mit Bleistift das Jahr »96« vermerkt  2) mit rotem Buntstift zwei Unterstreichungen}\toendnotes[C]{\smallbreak}\pstart
           \noindent{}{\pb}\textcolor{gray}{\textbf{\textbf{Frankfurter Zeitung\orgindex{Frankfurter Zeitung@Frankfurter Zeitung|pw}}}}\pend
           \pstart
           \textcolor{gray}{\textbf{(\begin{otherlanguage}{french}Gazette de Francfort\end{otherlanguage}\orgindex{Frankfurter Zeitung@Frankfurter Zeitung|pw}).}}\pend
           \pstart
           \textcolor{gray}{\textbf{\textbf{\begin{otherlanguage}{french}Fondateur M.\end{otherlanguage}{ }L. Sonnemann\pwindex{Sonnemann, Leopold 1831-10-29 – 1909-10-30@\textsc{Sonnemann, Leopold} (1831-10-29 – 1909-10-30), \emph{Journalist, Herausgeber}|pw}.}}}\pend
           \pstart
           \begin{otherlanguage}{french}\textcolor{gray}{\textbf{Journal\pwindex{?? Werk@Nicht ermittelte Verfasserinnen und Verfasser!Frankfurter Zeitung1856 – 1943@\emph{Frankfurter Zeitung} {[}1856 – 1943{]}|pwv} politique,
                        financier,}}\end{otherlanguage}\pend
           \pstart
           \begin{otherlanguage}{french}\textcolor{gray}{\textbf{commercial et littéraire.}}\end{otherlanguage}\pend
           \pstart
           \begin{otherlanguage}{french}\textcolor{gray}{\textbf{\textbf{Paraissant trois fois par jour.}}}\end{otherlanguage}\hfill \textsc{Paris\oindex{Paris@\textbf{Paris}|pw}}, 17. October.\pend
           \pstart
           \begin{otherlanguage}{french}\textcolor{gray}{\textbf{\textbf{Bureau à Paris\oindex{Paris@\textbf{Paris}|pw}}}}\end{otherlanguage}\pend
           \pstart
           \begin{otherlanguage}{french}\textcolor{gray}{\textbf{\textbf{24. Rue Feydeau\oindex{rue Feydeau@\textbf{rue Feydeau}|pw}.}}}\end{otherlanguage}\pend
           \pstart\center{}Mein lieber Freund,\pend\pstart
           Warum höre ich ſo gar nichts mehr von Dir? Deine lieben Nachrichten fehlen mir ſehr.
               Eine ſo lange Pauſe haſt Du noch nie gemacht. Ich bin in Sorgen. Biſt Du unwohl? Oder
               iſt Dir ſonſt etwas Verſtimmendes \label{K_L02787-1v}\edtext{zugeſtoßen}{\lemma{\textnormal{\emph{zugeſtoßen}}}\Cendnote{\textnormal{keine Vorkommnisse
                  bekannt}}}\label{K_L02787-1h}? Du mußt mir gleich ſchreiben.\pend
           \pstart
           Anbei eine \label{K_L02787-2v}\edtext{Beſcheinigung}{\lemma{\textnormal{\emph{Beſcheinigung}}}\Cendnote{\textnormal{Beilage nicht erhalten}}}\label{K_L02787-2h} von \textsc{Thorel\pwindex{Thorel, Jean 1859-09-11 – 1916-08-20@\textsc{Thorel, Jean} (1859-09-11 – 1916-08-20), \emph{Übersetzer, Dramatiker}|pw}}, dem ich die 500 \textsc{Fr.} ausgehändigt. Dieſe
               Beſcheinigung habe ich mir ausſtellen laſſen, um gegenüber der \textsc{Société des Auteurs Dramatiques\orgindex{Societe des Auteurs et Compositeurs Dramatiques@Société des Auteurs et Compositeurs Dramatiques|pw}} (durch welche hier das \textsc{Tantièmen}-Geſchäft geht) {\pb}den \uline{Darlehens}-Character des von Dir gezahlten Betrages zu conſtatiren. Heb’ Dir das
               Billet gut auf!\pend
           \pstart
           Die Überſetzung\pwindex{Thorel, Jean 1859-09-11 – 1916-08-20@\textsc{Thorel, Jean} (1859-09-11 – 1916-08-20), \emph{Übersetzer, Dramatiker}!Amourette. Piece en trois actes. Adaptee de Arthur Schnitzler1897@\strich\emph{Amourette. Pièce en trois actes. Adaptée de Arthur Schnitzler} {[}Übersetzung, 1897{]}|pwv} iſt ſeit geſtern in meinen Händen. Ich will ſie ein wenig
               durchſchauen, dann ſoll ſie copirt werden, und dann bekommſt Du die Copie. Große
               Schwierigkeiten macht uns das \label{K_L02787-3v}\edtext{»Joſefſtädter Theater\orgindex{Theater in der Josefstadt@Theater in der Josefstadt|pw}«}{\lemma{\textnormal{\emph{»Joſefſtädter Theater«}}}\Cendnote{\textnormal{das in der \emph{Liebelei}\pwindex{Schnitzler, Arthur 15.05.1862 – 21.10.1931@\textsc{Schnitzler, Arthur} (15.05.1862 – 21.10.1931), \emph{Schriftsteller, Mediziner}!Liebelei. Schauspiel in drei Akten1895-10-09@\strich\emph{Liebelei. Schauspiel in drei Akten} {[}1895-10-09{]}|pwk}
                  mehrmals namentlich erwähnt wird}}}\label{K_L02787-3h}. In \textsc{Paris\oindex{Paris@\textbf{Paris}|pw}} hat natürlich kein Menſch eine Ahnung, was für ein Ding das iſt? Wie ſoll man
               das alſo im Franzöſiſchen umſchreiben, um dem Publicum den Eindruck des {\pb}Vorſtadt-\textsc{Milieus} zu geben?
               Vielleicht einfach: »\label{K_L02787-4v}\edtext{\begin{otherlanguage}{french}\textsc{un théâtre
                        du faubourg\orgindex{Theater in der Josefstadt@Theater in der Josefstadt|pwv}}\end{otherlanguage}}{\lemma{\textnormal{\emph{un théâtre
                        du faubourg}}}\Cendnote{\textnormal{französisch: Vorstadttheater}}}\label{K_L02787-4h}«?
               Oder fällt Dir was Beſſeres ein.\pend
           \pstart
           Anbei auch ein \label{K_L02787-18v}\edtext{Ausſchnitt\pwindex{?? Werk@Nicht ermittelte Verfasserinnen und Verfasser!Tages-Rundschau [Offizier hat Buerger niedergestochen]1896-10-14@\emph{Tages-Rundschau [Offizier hat Bürger niedergestochen]} {[}1896-10-14{]}|pwv}}{\lemma{\textnormal{\emph{Ausſchnitt}}}\Cendnote{\textnormal{siehe unten}}}\label{K_L02787-18h} aus unſerem Blatte\pwindex{?? Werk@Nicht ermittelte Verfasserinnen und Verfasser!Frankfurter Zeitung1856 – 1943@\emph{Frankfurter Zeitung} {[}1856 – 1943{]}|pwv} über eine dieſer Tage
               vorgefallene Säbel-Affaire. Wenn Du \strikeout{\textcolor{gray}{h}} das noch nicht geleſen haſt, wirds Dich intereſſiren.\pend
           \pstart
           Wie ſtehts mit \label{K_L02787-5v}\edtext{Berlin\oindex{Berlin@\textbf{Berlin}|pw}}{\lemma{\textnormal{\emph{Berlin}}}\Cendnote{\textnormal{Siehe dazu vor allem \emph{Der Briefwechsel Arthur Schnitzler — Otto Brahm}.
                     Vollständige Ausgabe. Herausgegeben, eingeleitet und erläutert von Oskar
                     Seidlin. Tübingen: \emph{Niemeyer}{ }1975, S. 14 ff.}}}\label{K_L02787-5h}?\pend
           \pstart
           Durch die verfluchten \label{K_L02787-7v}\edtext{Ruſſ\oindex{Russland@\textbf{Russland}|pw}enfeſte}{\lemma{\textnormal{\emph{Ruſſenfeſte}}}\Cendnote{\textnormal{Goldmann\pwindex{Goldmann, Paul 31.01.1865 – 25.09.1935@\textsc{Goldmann, Paul} (31.01.1865 – 25.09.1935), \emph{Schriftsteller, Journalist}|pwk} bezog sich hier wohl auf den Frankreich\oindex{Frankreich@\textbf{Frankreich}|pwk}-Besuch des Zaren Nikolaus II.\pwindex{Nikolaus II. von Russland 1868-05-06 – 1918-07-17@\textsc{Nikolaus II. von Russland} (1868-05-06 – 1918-07-17), \emph{Zar}|pwk} und der Kaiserin Alexandra Fjodorowna\pwindex{Hessen-Darmstadt, Alix von 1872-06-06 – 1918-07-17@\textsc{Hessen-Darmstadt, Alix von} (1872-06-06 – 1918-07-17), \emph{Kaiserin, Prinzessin}|pwk} zwischen dem 5. und 9. 10. 1896 und den damit
                  einhergehenden »Zarentagen« in Paris\oindex{Paris@\textbf{Paris}|pwk}.}}}\label{K_L02787-7h}
               habe ich noch keine Zeit gehabt, zu \textsc{Forain\pwindex{Forain, Jean-Louis 1852-10-23 – 1931-07-11@\textsc{Forain, Jean-Louis} (1852-10-23 – 1931-07-11), \emph{Maler, Grafiker, Karikaturist}|pw}} zu gehen. Das bleibt für nächſte Woche.\pend
           \pstart
           Viele treue Grüße!\pend
           \pstart
           Dein {\\[\baselineskip]}\spacefill\mbox{Paul Goldmann}\pend
           \leftskip=0em{}\pstart
           \noindent{}\textsc{Leo Fanjung\pwindex{Van-Jung, Leo 15.10.1866 – 02.07.1939@\textsc{Van-Jung, Leo} (15.10.1866 – 02.07.1939), \emph{Gesangspädagoge, Mathematiker}|pw}} war hier, mit dem ich mich rieſig geſreut habe. Welch’
                     {[}ein{]} liebes Kind!\pend
           {\bigskip}\pstart
           \noindent{}{\pb}\label{K_L02787-999v}\edtext{Wie ſchon mitgetheilt}{\lemma{\textnormal{\emph{Wie ſchon mitgetheilt}}}\Cendnote{\textnormal{Der beiliegende Ausschnitt\pwindex{?? Werk@Nicht ermittelte Verfasserinnen und Verfasser!Tages-Rundschau [Offizier hat Buerger niedergestochen]1896-10-14@\emph{Tages-Rundschau [Offizier hat Bürger niedergestochen]} {[}1896-10-14{]}|pwkv} ist gedruckt und aus der \emph{Frankfurter Zeitung}\pwindex{?? Werk@Nicht ermittelte Verfasserinnen und Verfasser!Frankfurter Zeitung1856 – 1943@\emph{Frankfurter Zeitung} {[}1856 – 1943{]}|pwk} ausgeschnitten: \emph{Tages-Rundschau}\pwindex{?? Werk@Nicht ermittelte Verfasserinnen und Verfasser!Tages-Rundschau [Offizier hat Buerger niedergestochen]1896-10-14@\emph{Tages-Rundschau [Offizier hat Bürger niedergestochen]} {[}1896-10-14{]}|pwk}. In: \emph{Frankfurter Zeitung}\pwindex{?? Werk@Nicht ermittelte Verfasserinnen und Verfasser!Frankfurter Zeitung1856 – 1943@\emph{Frankfurter Zeitung} {[}1856 – 1943{]}|pwk}, Jg. 41, Nr. 286,
                        14. 10. 1896, Abendblatt, S. 1.}}}\label{K_L02787-999h} wurde, hat in \so{Karlsruhe}\oindex{Karlsruhe@\textbf{Karlsruhe}|pw} ein \so{Offizier}\pwindex{Bruesewitz, Henning von 1862 – 1900-01-24@\textsc{Brüsewitz, Henning von} (1862 – 1900-01-24), \emph{Militär}|pwv} einen \so{Bürger}\pwindex{Siepmann, Theodor †~1896-10-11@\textsc{Siepmann, Theodor} (†~1896-10-11), \emph{Handwerker, Mechaniker}|pwv} ohne jede Veranlaſſung \so{niedergeſtochen}. Ueber den
               traurigen Vorgang erhalten wir von einem Augenzeugen zugleich nach den Mittheilungen
               weiterer Augenzeugen eine Darſtellung, die durchaus den Eindruck der Glaubwürdigkeit
               macht. Wir geben ſie nachſtehend wieder, da der Vorgang zu einigen Bemerkungen an
               dieſer Stelle Veranlaſſung gibt. Der Augenzeuge ſchreibt:\pend
           \pstart
           Premierlieutenant v. Brüſewitz\pwindex{Bruesewitz, Henning von 1862 – 1900-01-24@\textsc{Brüsewitz, Henning von} (1862 – 1900-01-24), \emph{Militär}|pw} begann mit Siepmann\pwindex{Siepmann, Theodor †~1896-10-11@\textsc{Siepmann, Theodor} (†~1896-10-11), \emph{Handwerker, Mechaniker}|pw} einen Wortwechſel, weil dieſer
               angeblich beim Niederſitzen an ſeinen Stuhl geſtoßen ſein ſoll, was übrigens ſelbſt
               von den mit Siepmann\pwindex{Siepmann, Theodor †~1896-10-11@\textsc{Siepmann, Theodor} (†~1896-10-11), \emph{Handwerker, Mechaniker}|pw} am gleichen Tiſche
               ſitzenden Perſonen nicht bemerkt wurde. Siepmann\pwindex{Siepmann, Theodor †~1896-10-11@\textsc{Siepmann, Theodor} (†~1896-10-11), \emph{Handwerker, Mechaniker}|pw} erwiderte, er wiſſe nichts davon, daß er v. Brüſewitz\pwindex{Bruesewitz, Henning von 1862 – 1900-01-24@\textsc{Brüsewitz, Henning von} (1862 – 1900-01-24), \emph{Militär}|pw} angerempelt habe. Dieſer rief hierauf den
                  \label{K_L02787-19v}\edtext{Wirth\pwindex{?? [Wirt im Cafe Tannhaeuser] @\textsc{?? [Wirt im Café Tannhäuser]}|pwv}}{\lemma{\textnormal{\emph{Wirth}}}\Cendnote{\textnormal{nicht identifiziert}}}\label{K_L02787-19h} und forderte
               ihn auf, Siepmann\pwindex{Siepmann, Theodor †~1896-10-11@\textsc{Siepmann, Theodor} (†~1896-10-11), \emph{Handwerker, Mechaniker}|pw} hinauszuweiſen, der nicht
               wiſſe, wie er ſich zu betragen habe. Der Wirth\pwindex{?? [Wirt im Cafe Tannhaeuser] @\textsc{?? [Wirt im Café Tannhäuser]}|pwv} ſuchte die Beiden durch Zureden zu beruhigen, was ihm
               anſcheinend auch gelang. Siepmann\pwindex{Siepmann, Theodor †~1896-10-11@\textsc{Siepmann, Theodor} (†~1896-10-11), \emph{Handwerker, Mechaniker}|pw} verließ
               dann das Lokal\oindex{Cafe Tannhaeuser@\textbf{Café Tannhäuser}|pwv}, kam aber
               gleich darauf wieder herein und ſetzte ſich. Nach kurzer Zeit rief v. Brüſewitz\pwindex{Bruesewitz, Henning von 1862 – 1900-01-24@\textsc{Brüsewitz, Henning von} (1862 – 1900-01-24), \emph{Militär}|pw} ſehr laut: »Sie haben mich in brüſker Weiſe
               angerempelt und ſich nicht entſchuldigt.« Siepmann\pwindex{Siepmann, Theodor †~1896-10-11@\textsc{Siepmann, Theodor} (†~1896-10-11), \emph{Handwerker, Mechaniker}|pw} erwiderte: »Ich weiß nichts davon.« Daraufhin ſprang v. Brüſewitz\pwindex{Bruesewitz, Henning von 1862 – 1900-01-24@\textsc{Brüsewitz, Henning von} (1862 – 1900-01-24), \emph{Militär}|pw} auf, ſtellte ſich vor Siepmann\pwindex{Siepmann, Theodor †~1896-10-11@\textsc{Siepmann, Theodor} (†~1896-10-11), \emph{Handwerker, Mechaniker}|pw} hin und ſchrie: »Wollen Sie mich um
               Entſchuldigung bitten, ja oder nein, ja oder nein, ja oder nein?« Siepmann\pwindex{Siepmann, Theodor †~1896-10-11@\textsc{Siepmann, Theodor} (†~1896-10-11), \emph{Handwerker, Mechaniker}|pw} blieb ruhig ſitzen und erwiderte ſchließlich:
               »Keine Antwort wird Ihnen auch genügen.« Daraufhin trat v. Brüſewitz\pwindex{Bruesewitz, Henning von 1862 – 1900-01-24@\textsc{Brüsewitz, Henning von} (1862 – 1900-01-24), \emph{Militär}|pw} 2 bis 3 Schritte zurück, ſchrie: »\so{Nein, das genügt mir ganz und gar nicht}«, riß den Säbel aus
               der Scheide und wollte mit hochgeſchwungener Waffe auf Siepmann\pwindex{Siepmann, Theodor †~1896-10-11@\textsc{Siepmann, Theodor} (†~1896-10-11), \emph{Handwerker, Mechaniker}|pw} eindringen. Der Wirth\pwindex{?? [Wirt im Cafe Tannhaeuser] @\textsc{?? [Wirt im Café Tannhäuser]}|pwv} und der Kellner fielen ihm jedoch in den Arm und
               hielten ihn feſt, während Siepmann\pwindex{Siepmann, Theodor †~1896-10-11@\textsc{Siepmann, Theodor} (†~1896-10-11), \emph{Handwerker, Mechaniker}|pw} das Lokal\oindex{Cafe Tannhaeuser@\textbf{Café Tannhäuser}|pwv} verließ und auf den Hof
               ging. v. Brüſewitz\pwindex{Bruesewitz, Henning von 1862 – 1900-01-24@\textsc{Brüsewitz, Henning von} (1862 – 1900-01-24), \emph{Militär}|pw} ſteckte ſeinen Säbel ein,
               ſetzte die Mütze auf, zog den Mantel an und rief dabei: »\so{Meine Ehre iſt kaput}, ich bin ein todter Mann; morgen kann ich meinen Abſchied
               einreichen.« Mit dieſen Worten verließ er das Lokal\oindex{Cafe Tannhaeuser@\textbf{Café Tannhäuser}|pwv} durch die nach der Karlſtraße\oindex{Karlstrasse@\textbf{Karlstraße}|pw} führende Thür. Dort ſtand ein Schutzmann\pwindex{Siepmann, Theodor †~1896-10-11@\textsc{Siepmann, Theodor} (†~1896-10-11), \emph{Handwerker, Mechaniker}|pwv}, bei dem ſich v. Brüſewitz\pwindex{Bruesewitz, Henning von 1862 – 1900-01-24@\textsc{Brüsewitz, Henning von} (1862 – 1900-01-24), \emph{Militär}|pw} erkundigte, ob Siepmann\pwindex{Siepmann, Theodor †~1896-10-11@\textsc{Siepmann, Theodor} (†~1896-10-11), \emph{Handwerker, Mechaniker}|pw}
               das Lokal\oindex{Cafe Tannhaeuser@\textbf{Café Tannhäuser}|pwv} verlaſſen habe.
               Als dieſer das verneinte, ſagte v. Brüſewitz\pwindex{Bruesewitz, Henning von 1862 – 1900-01-24@\textsc{Brüsewitz, Henning von} (1862 – 1900-01-24), \emph{Militär}|pw}:
               »den muß ich abpaſſen.« \so{Er holte dann zwei Feldwebel herbei}, denen er befahl, an der Thüre zu bleiben, \so{da er bedroht ſei}. Er ſelbſt ging von der Kaiſerſtraße\oindex{Kaiserstrasse@\textbf{Kaiserstraße}|pw} aus wieder in den zu den vordern Lokalen führenden Gang hinein.
               Inzwiſchen hatten der Wirth\pwindex{?? [Wirt im Cafe Tannhaeuser] @\textsc{?? [Wirt im Café Tannhäuser]}|pwv}
               und ein anderer Herr dem Siepmann\pwindex{Siepmann, Theodor †~1896-10-11@\textsc{Siepmann, Theodor} (†~1896-10-11), \emph{Handwerker, Mechaniker}|pw} im Hofe
               zugeredet, er ſolle, um die Sache gütlich zu erledigen, am andern Tage zu v. Brüſewitz\pwindex{Bruesewitz, Henning von 1862 – 1900-01-24@\textsc{Brüsewitz, Henning von} (1862 – 1900-01-24), \emph{Militär}|pw} gehen und ſich entſchuldigen,
               wozu er auch bereit ſchien. Er bat den Wirth\pwindex{?? [Wirt im Cafe Tannhaeuser] @\textsc{?? [Wirt im Café Tannhäuser]}|pwv}, ihm ſeinen Hut zu holen. Der Wirth\pwindex{?? [Wirt im Cafe Tannhaeuser] @\textsc{?? [Wirt im Café Tannhäuser]}|pwv} holte den Hut, und wollte Siepmann\pwindex{Siepmann, Theodor †~1896-10-11@\textsc{Siepmann, Theodor} (†~1896-10-11), \emph{Handwerker, Mechaniker}|pw} vom Hofe auf den nach der Kaiſerſtraße\oindex{Kaiserstrasse@\textbf{Kaiserstraße}|pw} führenden Hausflur laſſen. Als er
               die Thür öffnete, ſtand v. Brüſewitz\pwindex{Bruesewitz, Henning von 1862 – 1900-01-24@\textsc{Brüsewitz, Henning von} (1862 – 1900-01-24), \emph{Militär}|pw} direkt
               vor der Thür und wollte mit den Worten: »Wo iſt der Schuft\pwindex{Siepmann, Theodor †~1896-10-11@\textsc{Siepmann, Theodor} (†~1896-10-11), \emph{Handwerker, Mechaniker}|pwv}?« in den Hof eindringen. Der Wirth\pwindex{?? [Wirt im Cafe Tannhaeuser] @\textsc{?? [Wirt im Café Tannhäuser]}|pwv} faßte ihn am Arme und
               rief ihm laut zu: »Herr Lieutenant\pwindex{Bruesewitz, Henning von 1862 – 1900-01-24@\textsc{Brüsewitz, Henning von} (1862 – 1900-01-24), \emph{Militär}|pwv}, der Mann\pwindex{Siepmann, Theodor †~1896-10-11@\textsc{Siepmann, Theodor} (†~1896-10-11), \emph{Handwerker, Mechaniker}|pwv} will ſich ja entſchuldigen.« Von
                  Brüſewitz\pwindex{Bruesewitz, Henning von 1862 – 1900-01-24@\textsc{Brüsewitz, Henning von} (1862 – 1900-01-24), \emph{Militär}|pw} erwiderte nichts, zog, als er Siepmann\pwindex{Siepmann, Theodor †~1896-10-11@\textsc{Siepmann, Theodor} (†~1896-10-11), \emph{Handwerker, Mechaniker}|pw} erblickte, den Säbel und ging auf ihn los. Siepmann\pwindex{Siepmann, Theodor †~1896-10-11@\textsc{Siepmann, Theodor} (†~1896-10-11), \emph{Handwerker, Mechaniker}|pw} ergriff die Flucht und rief: »Ich bitte um
               Verzeihung, verzeihen Sie mir.« Am Ende des nur wenige Schritte langen Hofes, holte
                  v. Brüſewitz\pwindex{Bruesewitz, Henning von 1862 – 1900-01-24@\textsc{Brüsewitz, Henning von} (1862 – 1900-01-24), \emph{Militär}|pw} den Siepmann,\pwindex{Siepmann, Theodor †~1896-10-11@\textsc{Siepmann, Theodor} (†~1896-10-11), \emph{Handwerker, Mechaniker}|pw} der die Thüre zum Lokal\oindex{Cafe Tannhaeuser@\textbf{Café Tannhäuser}|pwv} nicht fand, ein und stach ihn
               nieder. Als er die blutige Waffe wieder einſteckte, ſagte er: »\so{So, jetzt iſt meine Ehre gerettet,}« und begab ſich dann durch das Lokal\oindex{Cafe Tannhaeuser@\textbf{Café Tannhäuser}|pwv} ungehindert auf die
               Straße. Siepmann\pwindex{Siepmann, Theodor †~1896-10-11@\textsc{Siepmann, Theodor} (†~1896-10-11), \emph{Handwerker, Mechaniker}|pw} wurde von einigen Herren in
               die Portierſtube auf ein Bett gebracht, wo er nach etwa einer halben Stunde
               verſchied. Der Säbel war auf der rechten Seite ungefähr 30 \textsc{cm}
               tief eingedrungen und hatte die Leber und wahrſcheinlich noch andere Organe
               durchbohrt. Die Wunde war abſolut tödtlich, und die ärztliche Hilfe war
               vergeblich.\pend
           
         
         \endnumbering\mylabel{h}\end{ledgroupsized}  \newcommand{\dateiname}{L02787}\newcommand{\titel}{Paul Goldmann an Arthur Schnitzler, 17. 10. [1896]}\newcommand{\editorInnen}{Martin Anton Müller und Laura Untner}%% latex-leseansicht-abspann.tex
%% Abspann für die Leseansicht.
%% Der Schalter \ifkorrekturansicht ist bereits durch den Vorspann gesetzt.

%% latex-abspann.tex
%% Gemeinsamer Abspann für Korrekturansicht und Leseansicht.
%% Setzt den Schalter \ifkorrekturansicht voraus (gesetzt in den
%% einbindenden Dateien latex-korrekturansicht-abspann.tex bzw.
%% latex-leseansicht-abspann.tex).
%% ---------------------------------------------------------------

\normalsize

% Das esempio-Environment wird nur in der Leseansicht benötigt
\ifkorrekturansicht\else
\newenvironment{esempio}[3]%
{
    \vspace{1.5ex}
    \rlap{\underline{#1}}
    \par
    \setlength{\parindent}{0cm}
    \nopagebreak
    \leftskip=#2cm
    \rightskip=#3cm
}
{
    \par
}
\fi

\doendnotes{C}
\bigskip
\vfill

\clearpage

\footnotesize

\ifkorrekturansicht
  \lohead{\textsc{register}}
\fi

% theindex-Environment neu definieren ohne reledmac
\makeatletter
\renewenvironment{theindex}{%
  \ifkorrekturansicht
    \section*{\indexname}%
  \else
    \subsubsection*{Index der erwähnten Entitäten}%
  \fi
  \setlength{\parindent}{0pt}%
  \setlength{\parskip}{0pt plus 0.3pt}%
  \let\item\@idxitem
}{%
  \ifkorrekturansicht\clearpage\fi
}
\makeatother

\IfFileExists{\jobname-pw.ind}{\input{\jobname-pw.ind}}{}

% Quellenangabe nur in der Leseansicht
\ifkorrekturansicht\else
% Fallback-Definitionen, falls die .tex-Datei \titel etc. nicht gesetzt hat
\providecommand{\titel}{}
\providecommand{\editorInnen}{}
\providecommand{\dateiname}{\jobname}

\vspace{3cm}

\vfill

\footnotesize
\textsc{Quelle}: \titel. Herausgegeben von {\editorInnen}. In: \emph{Arthur Schnitzler: Briefwechsel mit Autorinnen und Autoren}.
 Digitale Edition, https://schnitzler-briefe.acdh.oeaw.ac.at/{\dateiname}.html (Stand \today)
\fi

\end{document}


      