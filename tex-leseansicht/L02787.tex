%% latex-leseansicht-vorspann.tex
%% Vorspann für die Leseansicht.
%% Lädt die gemeinsame Datei latex-vorspann.tex mit nicht gesetztem Schalter.

\newif\ifkorrekturansicht
\korrekturansichtfalse

\input{../tex-inputs/latex-vorspann}


\section[ Paul Goldmann an Arthur Schnitzler, 17. 10. [1896]]{L02787 Paul Goldmann an Arthur Schnitzler,  17. 10. [1896]}
\nopagebreak\mylabel{L02787v}
\rehead{ }\normalsize\beginnumbering\briefempfaengerindex{Schnitzler, Arthur@\textsc{Schnitzler, Arthur}!zzzGoldmann, Paul@\emph{von Paul Goldmann}!1896-10-172@{17. 10. [1896]}|(be}
\toendnotes[C]{\smallbreak\pagebreak[2]}
\correspDesc{Versand  durch Paul Goldmann am 17. 10. [1896] in Paris
\newline{}Erhalt  durch Arthur Schnitzler im Zeitraum [18. 10. 1896 – 22. 10. 1896?] in Wien}\toendnotes[C]{\smallbreak}
\Standort{DLA, A:Schnitzler, HS.NZ85.1.3166.}
\physDesc{Brief, 1 Blatt, 3 Seiten, 5232 Zeichen
\newline{}Handschrift: blaue Tinte, deutsche Kurrent
\newline{}Beilage: Zeitungsausschnitt, beschnitten, am Spaltenumbruch
                                 zusammengeklebt. Mit blauem Buntstift wurde der Beginn markiert.
                                 Die Rückseite ist offensichtlich nicht relevant. 
\newline{}Schnitzler: 1) mit Bleistift das Jahr »96« vermerkt  2) mit rotem Buntstift zwei Unterstreichungen}\toendnotes[C]{\smallbreak}
\pstart
           {\pb}\textcolor{gray}{\textbf{\textbf{Frankfurter Zeitung\orgindex{Frankfurter Zeitung@Frankfurter Zeitung|pw}}}}\pend
           
\pstart
           \textcolor{gray}{\textbf{(\begin{otherlanguage}{french}Gazette de Francfort\end{otherlanguage}\orgindex{Frankfurter Zeitung@Frankfurter Zeitung|pw}).}}\pend
           
\pstart
           \textcolor{gray}{\textbf{\textbf{\begin{otherlanguage}{french}Fondateur M.\end{otherlanguage}{ }L. Sonnemann\pwindex{Sonnemann, Leopold 29.\,10.\,1831 Höchberg – 30.\,10.\,1909 Frankfurt am Main@\textsc{Sonnemann, Leopold} (29.\,10.\,1831 Höchberg – 30.\,10.\,1909 Frankfurt am Main), \emph{Journalist, Herausgeber}|pw}.}}}\pend
           
\pstart
           \begin{otherlanguage}{french}\textcolor{gray}{\textbf{Journal\pwindex{Frankfurter Zeitung@\emph{Frankfurter Zeitung}|pwv} politique,
                        financier,}}\end{otherlanguage}\pend
           
\pstart
           \begin{otherlanguage}{french}\textcolor{gray}{\textbf{commercial et littéraire.}}\end{otherlanguage}\pend
           
\pstart
           \begin{otherlanguage}{french}\textcolor{gray}{\textbf{\textbf{Paraissant trois fois par jour.}}}\end{otherlanguage}\hfill \textsc{Paris\oindex{Paris@\textbf{Paris}, \emph{Hauptstadt}|pw}}, 17. October.\pend
           
\pstart
           \begin{otherlanguage}{french}\textcolor{gray}{\textbf{\textbf{Bureau à Paris\oindex{Paris@\textbf{Paris}, \emph{Hauptstadt}|pw}}}}\end{otherlanguage}\pend
           
\pstart
           \begin{otherlanguage}{french}\textcolor{gray}{\textbf{\textbf{24. Rue Feydeau\oindex{rue Feydeau@\textbf{rue Feydeau}, \emph{Straße}|pw}.}}}\end{otherlanguage}\pend
           
\pstart\center{}Mein lieber Freund,\pend\vspace{0.5em}
\pstart
           Warum höre ich{ }ſo gar nichts mehr von Dir? Deine lieben Nachrichten fehlen mir{ }ſehr.
               Eine{ }ſo lange Pauſe haſt Du noch nie gemacht. Ich bin in Sorgen. Biſt Du unwohl? Oder
               iſt Dir{ }ſonſt etwas Verſtimmendes \label{K_L02787-1v}\edtext{zugeſtoßen}{\lemma{\textnormal{\emph{zugestoßen}}}\Cendnote{\textnormal{keine Vorkommnisse
                  bekannt}}}\label{K_L02787-1}? Du mußt mir gleich{ }ſchreiben.\pend
           
\pstart
           Anbei eine \label{K_L02787-2v}\edtext{Beſcheinigung}{\lemma{\textnormal{\emph{Bescheinigung}}}\Cendnote{\textnormal{Beilage nicht erhalten}}}\label{K_L02787-2} von \textsc{Thorel\pwindex{Thorel, Jean 11.\,9.\,1859 Éragny – 20.\,8.\,1916 Enghien-les-Bains@\textsc{Thorel, Jean} (11.\,9.\,1859 Éragny – 20.\,8.\,1916 Enghien-les-Bains), \emph{Übersetzer, Dramatiker}|pw}}, dem ich die 500 \textsc{Fr.} ausgehändigt. Dieſe
               Beſcheinigung habe ich mir ausſtellen laſſen, um gegenüber der \textsc{Société des Auteurs Dramatiques\orgindex{Société des Auteurs et Compositeurs Dramatiques@Société des Auteurs et Compositeurs Dramatiques|pw}} (durch welche hier das \textsc{Tantièmen}-Geſchäft geht) {\pb}den \uline{Darlehens}-Character des von Dir gezahlten Betrages zu conſtatiren. Heb’ Dir das
               Billet gut auf!\pend
           
\pstart
           Die Überſetzung\pwindex{Schnitzler, Arthur 15.\,5.\,1862 Wien – 21.\,10.\,1931 ebd.@\textsc{Schnitzler, Arthur} (15.\,5.\,1862 Wien – 21.\,10.\,1931 ebd.), \emph{Schriftsteller, Mediziner}!Amourette. Pièce en trois actes. Adaptée de Arthur Schnitzler@\strich\emph{Amourette. Pièce en trois actes. Adaptée de Arthur Schnitzler}|pwv} iſt{ }ſeit geſtern in meinen Händen. Ich will{ }ſie ein wenig
               durchſchauen, dann{ }ſoll{ }ſie copirt werden, und dann bekommſt Du die Copie. Große
               Schwierigkeiten macht uns das \label{K_L02787-3v}\edtext{»Joſefſtädter Theater\orgindex{Theater in der Josefstadt@Theater in der Josefstadt|pw}«}{\lemma{\textnormal{\emph{»Josefstädter Theater«}}}\Cendnote{\textnormal{das in \emph{Liebelei}\pwindex{Schnitzler, Arthur 15.\,5.\,1862 Wien – 21.\,10.\,1931 ebd.@\textsc{Schnitzler, Arthur} (15.\,5.\,1862 Wien – 21.\,10.\,1931 ebd.), \emph{Schriftsteller, Mediziner}!Liebelei. Schauspiel in drei Akten@\strich\emph{Liebelei. Schauspiel in drei Akten}|pwk}
                  mehrmals namentlich erwähnt wird}}}\label{K_L02787-3}. In \textsc{Paris\oindex{Paris@\textbf{Paris}, \emph{Hauptstadt}|pw}} hat natürlich kein Menſch eine Ahnung, was für ein Ding das iſt? Wie{ }ſoll man
               das alſo im Franzöſiſchen umſchreiben, um dem Publicum den Eindruck des {\pb}Vorſtadt-\textsc{Milieus} zu geben?
               Vielleicht einfach: »\label{K_L02787-4v}\edtext{\begin{otherlanguage}{french}\textsc{un théâtre
                        du faubourg\orgindex{Theater in der Josefstadt@Theater in der Josefstadt|pwv}}\end{otherlanguage}}{\lemma{\textnormal{\emph{un théâtre
                        du faubourg}}}\Cendnote{\textnormal{französisch: Vorstadttheater}}}\label{K_L02787-4}«?
               Oder fällt Dir was Beſſeres ein.\pend
           
\pstart
           Anbei auch ein \label{K_L02787-5v}\edtext{Ausſchnitt\pwindex{Tages-Rundschau [Offizier hat Bürger niedergestochen]@\emph{Tages-Rundschau [Offizier hat Bürger niedergestochen]}|pwv}}{\lemma{\textnormal{\emph{Ausschnitt}}}\Cendnote{\textnormal{Siehe unten.}}}\label{K_L02787-5} aus unſerem Blatte\pwindex{Frankfurter Zeitung@\emph{Frankfurter Zeitung}|pwv} über eine dieſer Tage
               vorgefallene Säbel-Affaire. Wenn Du \strikeout{\textcolor{gray}{h}} das noch nicht geleſen haſt, wirds Dich intereſſiren.\pend
           
\pstart
           Wie{ }ſtehts mit \label{K_L02787-6v}\edtext{Berlin\oindex{Berlin@\textbf{Berlin}, \emph{Hauptstadt}|pw}}{\lemma{\textnormal{\emph{Berlin}}}\Cendnote{\textnormal{Siehe dazu vor allem \emph{Der Briefwechsel Arthur Schnitzler – Otto Brahm}.
                     Vollständige Ausgabe. Herausgegeben, eingeleitet und erläutert von Oskar
                     Seidlin. Tübingen: \emph{Niemeyer}{ }1975, S. 14 ff.
               }}}\label{K_L02787-6}?\pend
           
\pstart
           Durch die verfluchten \label{K_L02787-7v}\edtext{Ruſſ\oindex{Russland@\textbf{Russland}|pw}enfeſte}{\lemma{\textnormal{\emph{Russenfeste}}}\Cendnote{\textnormal{Goldmann\pwindex{Goldmann, Paul 31.\,1.\,1865 Breslau – 25.\,9.\,1935 Wien@\textsc{Goldmann, Paul} (31.\,1.\,1865 Breslau – 25.\,9.\,1935 Wien), \emph{Schriftsteller, Journalist}|pwk} bezog sich hier wohl auf den Frankreich\oindex{Frankreich@\textbf{Frankreich}|pwk}-Besuch des Zaren Nikolaus II.\pwindex{Nikolaus II. von Russland 6.\,5.\,1868 Pushkin – 17.\,7.\,1918 Jekaterinburg@\textsc{Nikolaus II. von Russland} (6.\,5.\,1868 Pushkin – 17.\,7.\,1918 Jekaterinburg), \emph{Zar}|pwk} und der Kaiserin Alexandra Fjodorowna\pwindex{Hessen-Darmstadt, Alix von 6.\,6.\,1872 Darmstadt – 17.\,7.\,1918 Jekaterinburg@\textsc{Hessen-Darmstadt, Alix von} (6.\,6.\,1872 Darmstadt – 17.\,7.\,1918 Jekaterinburg), \emph{Kaiserin, Prinzessin}|pwk} zwischen dem 5. und 9. 10. 1896 und den damit
                  einhergehenden »Zarentagen« in Paris\oindex{Paris@\textbf{Paris}, \emph{Hauptstadt}|pwk}.}}}\label{K_L02787-7}
               habe ich noch keine Zeit gehabt, zu \textsc{Forain\pwindex{Forain, Jean-Louis 23.\,10.\,1852 Reims – 11.\,7.\,1931 Paris@\textsc{Forain, Jean-Louis} (23.\,10.\,1852 Reims – 11.\,7.\,1931 Paris), \emph{Maler, Grafiker, Karikaturist}|pw}} zu gehen. Das bleibt für nächſte Woche.\pend
           
\pstart
           Viele treue Grüße!\pend
           
\pstart
           Dein {\\[\baselineskip]}\spacefill\mbox{Paul Goldmann}\pend
           \leftskip=0em{}
\pstart
           \noindent{}\textsc{Leo Fanjung\pwindex{Van-Jung, Leo 15.\,10.\,1866 Odessa – 2.\,7.\,1939 Riga@\textsc{Van-Jung, Leo} (15.\,10.\,1866 Odessa – 2.\,7.\,1939 Riga), \emph{Gesangspädagoge, Mathematiker}|pw}} war hier, mit dem ich mich rieſig geſreut habe. Welch’
                     {[}ein{]} liebes Kind!\pend
           \selectlanguage{ngerman}\vspace{1em}{\vspace{1\baselineskip}}
\pstart
           {\pb}\label{K_L02787-8v}\edtext{Wie{ }ſchon mitgetheilt}{\lemma{\textnormal{\emph{Wie schon mitgetheilt}}}\Cendnote{\textnormal{Der beiliegende Ausschnitt\pwindex{Tages-Rundschau [Offizier hat Bürger niedergestochen]@\emph{Tages-Rundschau [Offizier hat Bürger niedergestochen]}|pwkv} ist aus der \emph{Frankfurter Zeitung}\pwindex{Frankfurter Zeitung@\emph{Frankfurter Zeitung}|pwk} ausgeschnitten: \emph{Tages-Rundschau}\pwindex{Tages-Rundschau [Offizier hat Bürger niedergestochen]@\emph{Tages-Rundschau [Offizier hat Bürger niedergestochen]}|pwk}. In: \emph{Frankfurter Zeitung}\pwindex{Frankfurter Zeitung@\emph{Frankfurter Zeitung}|pwk}, Jg. 41, Nr. 286,
                        14. 10. 1896, Abendblatt, S. 1.}}}\label{K_L02787-8} wurde, hat in \so{Karlsruhe}\oindex{Karlsruhe@\textbf{Karlsruhe}, \emph{Hauptstadt}|pw} ein \so{Offizier}\pwindex{Brüsewitz, Henning von 1862 – 24.\,1.\,1900@\textsc{Brüsewitz, Henning von} (1862 – 24.\,1.\,1900), \emph{Militär, Offizier}|pwv} einen \so{Bürger}\pwindex{Siepmann, Theodor †~11.\,10.\,1896 Karlsruhe@\textsc{Siepmann, Theodor} (†~11.\,10.\,1896 Karlsruhe), \emph{Handwerker, Mechaniker}|pwv} ohne jede Veranlaſſung \so{niedergeſtochen}. Ueber den
               traurigen Vorgang erhalten wir von einem Augenzeugen zugleich nach den Mittheilungen
               weiterer Augenzeugen eine Darſtellung, die durchaus den Eindruck der Glaubwürdigkeit
               macht. Wir geben{ }ſie nachſtehend wieder, da der Vorgang zu einigen Bemerkungen an
               dieſer Stelle Veranlaſſung gibt. Der Augenzeuge{ }ſchreibt:\pend
           
\pstart
           Premierlieutenant v. Brüſewitz\pwindex{Brüsewitz, Henning von 1862 – 24.\,1.\,1900@\textsc{Brüsewitz, Henning von} (1862 – 24.\,1.\,1900), \emph{Militär, Offizier}|pw} begann mit Siepmann\pwindex{Siepmann, Theodor †~11.\,10.\,1896 Karlsruhe@\textsc{Siepmann, Theodor} (†~11.\,10.\,1896 Karlsruhe), \emph{Handwerker, Mechaniker}|pw} einen Wortwechſel, weil dieſer
               angeblich beim Niederſitzen an{ }ſeinen Stuhl geſtoßen{ }ſein{ }ſoll, was übrigens{ }ſelbſt
               von den mit Siepmann\pwindex{Siepmann, Theodor †~11.\,10.\,1896 Karlsruhe@\textsc{Siepmann, Theodor} (†~11.\,10.\,1896 Karlsruhe), \emph{Handwerker, Mechaniker}|pw} am gleichen Tiſche{ }ſitzenden Perſonen nicht bemerkt wurde. Siepmann\pwindex{Siepmann, Theodor †~11.\,10.\,1896 Karlsruhe@\textsc{Siepmann, Theodor} (†~11.\,10.\,1896 Karlsruhe), \emph{Handwerker, Mechaniker}|pw} erwiderte, er wiſſe nichts davon, daß er v. Brüſewitz\pwindex{Brüsewitz, Henning von 1862 – 24.\,1.\,1900@\textsc{Brüsewitz, Henning von} (1862 – 24.\,1.\,1900), \emph{Militär, Offizier}|pw} angerempelt habe. Dieſer rief hierauf den
                  \label{K_L02787-9v}\edtext{Wirth\pwindex{?? [Wirt im Café Tannhäuser] @\textsc{?? [Wirt im Café Tannhäuser]}|pwv}}{\lemma{\textnormal{\emph{Wirth}}}\Cendnote{\textnormal{nicht identifiziert}}}\label{K_L02787-9} und forderte
               ihn auf, Siepmann\pwindex{Siepmann, Theodor †~11.\,10.\,1896 Karlsruhe@\textsc{Siepmann, Theodor} (†~11.\,10.\,1896 Karlsruhe), \emph{Handwerker, Mechaniker}|pw} hinauszuweiſen, der nicht
               wiſſe, wie er{ }ſich zu betragen habe. Der Wirth\pwindex{?? [Wirt im Café Tannhäuser] @\textsc{?? [Wirt im Café Tannhäuser]}|pwv}{ }ſuchte die Beiden durch Zureden zu beruhigen, was ihm
               anſcheinend auch gelang. Siepmann\pwindex{Siepmann, Theodor †~11.\,10.\,1896 Karlsruhe@\textsc{Siepmann, Theodor} (†~11.\,10.\,1896 Karlsruhe), \emph{Handwerker, Mechaniker}|pw} verließ
               dann das Lokal\oindex{Café Tannhäuser@\textbf{Café Tannhäuser}, \emph{Kaffeehaus}|pwv}, kam aber
               gleich darauf wieder herein und{ }ſetzte{ }ſich. Nach kurzer Zeit rief v. Brüſewitz\pwindex{Brüsewitz, Henning von 1862 – 24.\,1.\,1900@\textsc{Brüsewitz, Henning von} (1862 – 24.\,1.\,1900), \emph{Militär, Offizier}|pw}{ }ſehr laut: »Sie haben mich in brüſker Weiſe
               angerempelt und{ }ſich nicht entſchuldigt.« Siepmann\pwindex{Siepmann, Theodor †~11.\,10.\,1896 Karlsruhe@\textsc{Siepmann, Theodor} (†~11.\,10.\,1896 Karlsruhe), \emph{Handwerker, Mechaniker}|pw} erwiderte: »Ich weiß nichts davon.« Daraufhin{ }ſprang v. Brüſewitz\pwindex{Brüsewitz, Henning von 1862 – 24.\,1.\,1900@\textsc{Brüsewitz, Henning von} (1862 – 24.\,1.\,1900), \emph{Militär, Offizier}|pw} auf,{ }ſtellte{ }ſich vor Siepmann\pwindex{Siepmann, Theodor †~11.\,10.\,1896 Karlsruhe@\textsc{Siepmann, Theodor} (†~11.\,10.\,1896 Karlsruhe), \emph{Handwerker, Mechaniker}|pw} hin und{ }ſchrie: »Wollen Sie mich um
               Entſchuldigung bitten, ja oder nein, ja oder nein, ja oder nein?« Siepmann\pwindex{Siepmann, Theodor †~11.\,10.\,1896 Karlsruhe@\textsc{Siepmann, Theodor} (†~11.\,10.\,1896 Karlsruhe), \emph{Handwerker, Mechaniker}|pw} blieb ruhig{ }ſitzen und erwiderte{ }ſchließlich:
               »Keine Antwort wird Ihnen auch genügen.« Daraufhin trat v. Brüſewitz\pwindex{Brüsewitz, Henning von 1862 – 24.\,1.\,1900@\textsc{Brüsewitz, Henning von} (1862 – 24.\,1.\,1900), \emph{Militär, Offizier}|pw} 2 bis 3 Schritte zurück,{ }ſchrie: »\so{Nein, das genügt mir ganz und gar nicht}«, riß den Säbel aus
               der Scheide und wollte mit hochgeſchwungener Waffe auf Siepmann\pwindex{Siepmann, Theodor †~11.\,10.\,1896 Karlsruhe@\textsc{Siepmann, Theodor} (†~11.\,10.\,1896 Karlsruhe), \emph{Handwerker, Mechaniker}|pw} eindringen. Der Wirth\pwindex{?? [Wirt im Café Tannhäuser] @\textsc{?? [Wirt im Café Tannhäuser]}|pwv} und der Kellner fielen ihm jedoch in den Arm und
               hielten ihn feſt, während Siepmann\pwindex{Siepmann, Theodor †~11.\,10.\,1896 Karlsruhe@\textsc{Siepmann, Theodor} (†~11.\,10.\,1896 Karlsruhe), \emph{Handwerker, Mechaniker}|pw} das Lokal\oindex{Café Tannhäuser@\textbf{Café Tannhäuser}, \emph{Kaffeehaus}|pwv} verließ und auf den Hof
               ging. v. Brüſewitz\pwindex{Brüsewitz, Henning von 1862 – 24.\,1.\,1900@\textsc{Brüsewitz, Henning von} (1862 – 24.\,1.\,1900), \emph{Militär, Offizier}|pw}{ }ſteckte{ }ſeinen Säbel ein,{ }ſetzte die Mütze auf, zog den Mantel an und rief dabei: »\so{Meine
                  Ehre iſt kaput}, ich bin ein todter Mann; morgen kann ich meinen Abſchied
               einreichen.« Mit dieſen Worten verließ er das Lokal\oindex{Café Tannhäuser@\textbf{Café Tannhäuser}, \emph{Kaffeehaus}|pwv} durch die nach der Karlſtraße\oindex{Karlstraße [Karlsruhe]@\textbf{Karlstraße [Karlsruhe]}, \emph{Straße}|pw} führende Thür. Dort{ }ſtand ein Schutzmann\pwindex{Siepmann, Theodor †~11.\,10.\,1896 Karlsruhe@\textsc{Siepmann, Theodor} (†~11.\,10.\,1896 Karlsruhe), \emph{Handwerker, Mechaniker}|pwv}, bei dem{ }ſich v. Brüſewitz\pwindex{Brüsewitz, Henning von 1862 – 24.\,1.\,1900@\textsc{Brüsewitz, Henning von} (1862 – 24.\,1.\,1900), \emph{Militär, Offizier}|pw} erkundigte, ob Siepmann\pwindex{Siepmann, Theodor †~11.\,10.\,1896 Karlsruhe@\textsc{Siepmann, Theodor} (†~11.\,10.\,1896 Karlsruhe), \emph{Handwerker, Mechaniker}|pw}
               das Lokal\oindex{Café Tannhäuser@\textbf{Café Tannhäuser}, \emph{Kaffeehaus}|pwv} verlaſſen habe.
               Als dieſer das verneinte,{ }ſagte v. Brüſewitz\pwindex{Brüsewitz, Henning von 1862 – 24.\,1.\,1900@\textsc{Brüsewitz, Henning von} (1862 – 24.\,1.\,1900), \emph{Militär, Offizier}|pw}:
               »den muß ich abpaſſen.« \so{Er holte dann zwei Feldwebel herbei}, denen er befahl, an der Thüre zu bleiben, \so{da er
                  bedroht{ }ſei}. Er{ }ſelbſt ging von der Kaiſerſtraße\oindex{Kaiserstraße [Karlsruhe]@\textbf{Kaiserstraße [Karlsruhe]}, \emph{Straße}|pw} aus wieder in den zu den vordern Lokalen führenden Gang hinein.
               Inzwiſchen hatten der Wirth\pwindex{?? [Wirt im Café Tannhäuser] @\textsc{?? [Wirt im Café Tannhäuser]}|pwv}
               und ein anderer Herr dem Siepmann\pwindex{Siepmann, Theodor †~11.\,10.\,1896 Karlsruhe@\textsc{Siepmann, Theodor} (†~11.\,10.\,1896 Karlsruhe), \emph{Handwerker, Mechaniker}|pw} im Hofe
               zugeredet, er{ }ſolle, um die Sache gütlich zu erledigen, am andern Tage zu v. Brüſewitz\pwindex{Brüsewitz, Henning von 1862 – 24.\,1.\,1900@\textsc{Brüsewitz, Henning von} (1862 – 24.\,1.\,1900), \emph{Militär, Offizier}|pw} gehen und{ }ſich entſchuldigen,
               wozu er auch bereit{ }ſchien. Er bat den Wirth\pwindex{?? [Wirt im Café Tannhäuser] @\textsc{?? [Wirt im Café Tannhäuser]}|pwv}, ihm{ }ſeinen Hut zu holen. Der Wirth\pwindex{?? [Wirt im Café Tannhäuser] @\textsc{?? [Wirt im Café Tannhäuser]}|pwv} holte den Hut, und wollte Siepmann\pwindex{Siepmann, Theodor †~11.\,10.\,1896 Karlsruhe@\textsc{Siepmann, Theodor} (†~11.\,10.\,1896 Karlsruhe), \emph{Handwerker, Mechaniker}|pw} vom Hofe auf den nach der Kaiſerſtraße\oindex{Kaiserstraße [Karlsruhe]@\textbf{Kaiserstraße [Karlsruhe]}, \emph{Straße}|pw} führenden Hausflur laſſen. Als er
               die Thür öffnete,{ }ſtand v. Brüſewitz\pwindex{Brüsewitz, Henning von 1862 – 24.\,1.\,1900@\textsc{Brüsewitz, Henning von} (1862 – 24.\,1.\,1900), \emph{Militär, Offizier}|pw} direkt
               vor der Thür und wollte mit den Worten: »Wo iſt der Schuft\pwindex{Siepmann, Theodor †~11.\,10.\,1896 Karlsruhe@\textsc{Siepmann, Theodor} (†~11.\,10.\,1896 Karlsruhe), \emph{Handwerker, Mechaniker}|pwv}?« in den Hof eindringen. Der Wirth\pwindex{?? [Wirt im Café Tannhäuser] @\textsc{?? [Wirt im Café Tannhäuser]}|pwv} faßte ihn am Arme und
               rief ihm laut zu: »Herr Lieutenant\pwindex{Brüsewitz, Henning von 1862 – 24.\,1.\,1900@\textsc{Brüsewitz, Henning von} (1862 – 24.\,1.\,1900), \emph{Militär, Offizier}|pwv}, der Mann\pwindex{Siepmann, Theodor †~11.\,10.\,1896 Karlsruhe@\textsc{Siepmann, Theodor} (†~11.\,10.\,1896 Karlsruhe), \emph{Handwerker, Mechaniker}|pwv} will{ }ſich ja entſchuldigen.« Von
                  Brüſewitz\pwindex{Brüsewitz, Henning von 1862 – 24.\,1.\,1900@\textsc{Brüsewitz, Henning von} (1862 – 24.\,1.\,1900), \emph{Militär, Offizier}|pw} erwiderte nichts, zog, als er Siepmann\pwindex{Siepmann, Theodor †~11.\,10.\,1896 Karlsruhe@\textsc{Siepmann, Theodor} (†~11.\,10.\,1896 Karlsruhe), \emph{Handwerker, Mechaniker}|pw} erblickte, den Säbel und ging auf ihn los. Siepmann\pwindex{Siepmann, Theodor †~11.\,10.\,1896 Karlsruhe@\textsc{Siepmann, Theodor} (†~11.\,10.\,1896 Karlsruhe), \emph{Handwerker, Mechaniker}|pw} ergriff die Flucht und rief: »Ich bitte um
               Verzeihung, verzeihen Sie mir.« Am Ende des nur wenige Schritte langen Hofes, holte
                  v. Brüſewitz\pwindex{Brüsewitz, Henning von 1862 – 24.\,1.\,1900@\textsc{Brüsewitz, Henning von} (1862 – 24.\,1.\,1900), \emph{Militär, Offizier}|pw} den Siepmann,\pwindex{Siepmann, Theodor †~11.\,10.\,1896 Karlsruhe@\textsc{Siepmann, Theodor} (†~11.\,10.\,1896 Karlsruhe), \emph{Handwerker, Mechaniker}|pw} der die Thüre zum Lokal\oindex{Café Tannhäuser@\textbf{Café Tannhäuser}, \emph{Kaffeehaus}|pwv} nicht fand, ein und stach ihn
               nieder. Als er die blutige Waffe wieder einſteckte,{ }ſagte er: »\so{So, jetzt iſt meine Ehre gerettet,}« und begab{ }ſich dann durch das Lokal\oindex{Café Tannhäuser@\textbf{Café Tannhäuser}, \emph{Kaffeehaus}|pwv} ungehindert auf die
               Straße. Siepmann\pwindex{Siepmann, Theodor †~11.\,10.\,1896 Karlsruhe@\textsc{Siepmann, Theodor} (†~11.\,10.\,1896 Karlsruhe), \emph{Handwerker, Mechaniker}|pw} wurde von einigen Herren in
               die Portierſtube auf ein Bett gebracht, wo er nach etwa einer halben Stunde
               verſchied. Der Säbel war auf der rechten Seite ungefähr 30 \textsc{cm}
               tief eingedrungen und hatte die Leber und wahrſcheinlich noch andere Organe
               durchbohrt. Die Wunde war abſolut tödtlich, und die ärztliche Hilfe war
               vergeblich.\pend
           \selectlanguage{ngerman}\endnumbering\briefempfaengerindex{Schnitzler, Arthur@\textsc{Schnitzler, Arthur}!zzzGoldmann, Paul@\emph{von Paul Goldmann}!1896-10-172@{17. 10. [1896]}|)be}\mylabel{L02787h}  \newcommand{\dateiname}{L02787}\newcommand{\titel}{Paul Goldmann an Arthur Schnitzler, 17. 10. [1896]}\newcommand{\editorInnen}{Martin Anton Müller und Laura Untner}%% latex-leseansicht-abspann.tex
%% Abspann für die Leseansicht.
%% Der Schalter \ifkorrekturansicht ist bereits durch den Vorspann gesetzt.

%% latex-abspann.tex
%% Gemeinsamer Abspann für Korrekturansicht und Leseansicht.
%% Setzt den Schalter \ifkorrekturansicht voraus (gesetzt in den
%% einbindenden Dateien latex-korrekturansicht-abspann.tex bzw.
%% latex-leseansicht-abspann.tex).
%% ---------------------------------------------------------------

\normalsize

% Das esempio-Environment wird nur in der Leseansicht benötigt
\ifkorrekturansicht\else
\newenvironment{esempio}[3]%
{
    \vspace{1.5ex}
    \rlap{\underline{#1}}
    \par
    \setlength{\parindent}{0cm}
    \nopagebreak
    \leftskip=#2cm
    \rightskip=#3cm
}
{
    \par
}
\fi

\doendnotes{C}
\bigskip
\vfill

\clearpage

\footnotesize

\ifkorrekturansicht
  \lohead{\textsc{register}}
\fi

% theindex-Environment neu definieren ohne reledmac
\makeatletter
\renewenvironment{theindex}{%
  \ifkorrekturansicht
    \section*{\indexname}%
  \else
    \subsubsection*{Index der erwähnten Entitäten}%
  \fi
  \setlength{\parindent}{0pt}%
  \setlength{\parskip}{0pt plus 0.3pt}%
  \let\item\@idxitem
}{%
  \ifkorrekturansicht\clearpage\fi
}
\makeatother

\IfFileExists{\jobname-pw.ind}{\input{\jobname-pw.ind}}{}

% Quellenangabe nur in der Leseansicht
\ifkorrekturansicht\else
% Fallback-Definitionen, falls die .tex-Datei \titel etc. nicht gesetzt hat
\providecommand{\titel}{}
\providecommand{\editorInnen}{}
\providecommand{\dateiname}{\jobname}

\vspace{3cm}

\vfill

\footnotesize
\textsc{Quelle}: \titel. Herausgegeben von {\editorInnen}. In: \emph{Arthur Schnitzler: Briefwechsel mit Autorinnen und Autoren}.
 Digitale Edition, https://schnitzler-briefe.acdh.oeaw.ac.at/{\dateiname}.html (Stand \today)
\fi

\end{document}


