%% latex-korrekturansicht-vorspann.tex
%% Vorspann für die Korrekturansicht.
%% Lädt die gemeinsame Datei latex-vorspann.tex mit gesetztem Schalter.

\newif\ifkorrekturansicht
\korrekturansichttrue

\input{../tex-inputs/latex-vorspann}


\section[Max Burckhard: Widmungsexemplar Im Paradiese für Arthur Schnitzler, 2{[}2?{]}. 12. 1906]{L01645 Max Burckhard: Widmungsexemplar Im Paradiese für Arthur Schnitzler,
               2{[}2?{]}. 12. 1906}
\nopagebreak\mylabel{L01645v}
\rehead{ }\normalsize\beginnumbering\briefempfaengerindex{Schnitzler, Arthur@\textsc{Schnitzler, Arthur}!zzzBurckhard, Max Eugen@\emph{von Max Eugen Burckhard}!1906-12-221@{2{[}2?{]}. 12. 1906}|(be}
\toendnotes[C]{\smallbreak\pagebreak[2]}\Standort{DLA, G:Schnitzler, Arthur (Sammlung Heinrich Schnitzler).}
\physDesc{Widmung am Schmutztitel, 56 Zeichen
\newline{}Handschrift: schwarze Tinte, deutsche Kurrent
\newline{}Ordnung: bei der Enteignung des Exemplars 1938 von
                                 unbekannter Hand mit Bleistift ergänzte Kenntlichmachung als
                                 Dublette: »= 452.154-B« }\toendnotes[C]{\smallbreak}
\pstart
           \noindent{}{\pb}Arthur Schnitzler{\\}in herzlicher Verehrung \pend
           \pstart \spacefill\mbox{Max Burckhard}\pend{}{\vspace{1\baselineskip}}
\pstart
           \centering{}\textcolor{gray}{\textbf{Im Paradiese\pwindex{Im Paradiese. Komoedie in 4 Akten@\emph{Im Paradiese. Komödie in 4 Akten}|pw}.}}\pend
           \selectlanguage{ngerman}\vspace{1em}{\vspace{1\baselineskip}}
\pstart
           \centering{}{\pb}\textcolor{gray}{\textbf{Max Burckhard.}}\pend
           
\pstart
           \centering{}\textcolor{gray}{\textbf{Im Paradiese\pwindex{Im Paradiese. Komoedie in 4 Akten@\emph{Im Paradiese. Komödie in 4 Akten}|pw}.}}\pend
           
\pstart
           \centering{}\textcolor{gray}{\textbf{Komödie in 4 Akten.}}\pend
           {\vspace{1\baselineskip}}
\pstart
           \centering{}\textcolor{gray}{\textbf{Wiener Verlag, G. m. b. H.\orgindex{Wiener Verlag@Wiener Verlag|pw}}}\pend
           
\pstart
           \centering{}\textcolor{gray}{\textbf{WIEN\oindex{Wien@\textbf{Wien}, \emph{A.ADM2}|pw}{ }UND{ }LEIPZIG\oindex{Leipzig@\textbf{Leipzig}, \emph{P.PPLA3}|pw}}}\pend
           
\pstart
           \centering{}\textcolor{gray}{\textbf{\label{K_L01645-1v}\edtext{1907}{\lemma{\textnormal{\emph{1907}}}\Cendnote{\textnormal{Das Buch 
                        ist vordatiert, vgl. A. S.: \emph{Tagebuch}, 22. 12. 1906.
                     }}}\label{K_L01645-1}.}}\pend
           \selectlanguage{ngerman}\endnumbering\briefempfaengerindex{Schnitzler, Arthur@\textsc{Schnitzler, Arthur}!zzzBurckhard, Max Eugen@\emph{von Max Eugen Burckhard}!1906-12-221@{2{[}2?{]}. 12. 1906}|)be}\mylabel{L01645h}  \normalsize

\doendnotes{C}
\bigskip
\vfill

\clearpage

\footnotesize

\lohead{\textsc{register}}

% Definiere theindex-Environment komplett neu ohne reledmac
\makeatletter
\renewenvironment{theindex}{%
  \section*{\indexname}%
  \setlength{\parindent}{0pt}%
  \setlength{\parskip}{0pt plus 0.3pt}%
  \let\item\@idxitem
}{%
  \clearpage
}
\makeatother

\IfFileExists{\jobname-pw.ind}{\input{\jobname-pw.ind}}{}

\end{document}

      