%% latex-korrekturansicht-vorspann.tex
%% Vorspann für die Korrekturansicht.
%% Lädt die gemeinsame Datei latex-vorspann.tex mit gesetztem Schalter.

\newif\ifkorrekturansicht
\korrekturansichttrue

\input{../tex-inputs/latex-vorspann}


\section[Thomas Mann an Arthur Schnitzler, {[}nach dem 18. 11. 1929{]}]{L02524 Thomas Mann an Arthur Schnitzler, {[}nach dem 18. 11. 1929{]}}
\nopagebreak\mylabel{L02524v}
\rehead{ }\normalsize\beginnumbering\briefempfaengerindex{Schnitzler, Arthur@\textsc{Schnitzler, Arthur}!zzzMann, Thomas@\emph{von Thomas Mann}!1929-11-191@{{[}nach dem 18. 11. 1929{]}}|(be}
\toendnotes[C]{\smallbreak\pagebreak[2]}\Standort{CUL, Schnitzler, B 67.}
\physDesc{Briefkarte, 31 Zeichen
\newline{}Handschrift: schwarze Tinte, deutsche Kurrent (\noindent{}Unterschrift und Nachschrift)}
\buchAbdrucke{\weitereDrucke{\emph{Modern Austrian Literature}, Jg. 7 (1974) Nr. 1/2, S. 26.} }\toendnotes[C]{\smallbreak}
\pstart
           \noindent{}
\pstart
           {\pb}\textcolor{gray}{\textbf{Dr. Thomas Mann}}\pend
           
\pstart
           \raggedleft{}\textcolor{gray}{\textbf{München\oindex{Muenchen@\textbf{München}, \emph{P.PPLA}|pw}, den \label{K_L02524-1v}\edtext{14. November 1929}{\lemma{\textnormal{\emph{14. November 1929}}}\Cendnote{\textnormal{Entgegen der gedruckten Datierung
                        ist anzunehmen, dass Mann\pwindex{Mann, Thomas 06.06.1875 – 12.08.1955@\textsc{Mann, Thomas} (06.06.1875 – 12.08.1955), \emph{Schriftsteller/Schriftstellerin}|pwk} auf die
                        Gratulation vom 18. 11. 1929 geantwortet hat.}}}\label{K_L02524-1}}}\pend
           \pend
           
\pstart
           \centering{}\textcolor{gray}{\textbf{Herzlich danke ich für die mir anläßlich der Verleihung des Nobelpreises\orgindex{Nobelpreis@Nobelpreis|pw} gewidmeten Glückwünsche.}}\pend
           \pstart \spacefill\mbox{Thomas Mann}\pend{}
\pstart
           \noindent{}In treuer Verehrung!\pend
           \selectlanguage{ngerman}\endnumbering\briefempfaengerindex{Schnitzler, Arthur@\textsc{Schnitzler, Arthur}!zzzMann, Thomas@\emph{von Thomas Mann}!1929-11-191@{{[}nach dem 18. 11. 1929{]}}|)be}\mylabel{L02524h}  \normalsize

\doendnotes{C}
\bigskip
\vfill

\clearpage

\footnotesize

\lohead{\textsc{register}}

% Definiere theindex-Environment komplett neu ohne reledmac
\makeatletter
\renewenvironment{theindex}{%
  \section*{\indexname}%
  \setlength{\parindent}{0pt}%
  \setlength{\parskip}{0pt plus 0.3pt}%
  \let\item\@idxitem
}{%
  \clearpage
}
\makeatother

\IfFileExists{\jobname-pw.ind}{\input{\jobname-pw.ind}}{}

\end{document}

      