%% latex-leseansicht-vorspann.tex
%% Vorspann für die Leseansicht.
%% Lädt die gemeinsame Datei latex-vorspann.tex mit nicht gesetztem Schalter.

\newif\ifkorrekturansicht
\korrekturansichtfalse

\input{../tex-inputs/latex-vorspann}


\section[Stefan Zweig an Arthur Schnitzler, 27. 10. 1911]{L03637 Stefan Zweig an Arthur Schnitzler, 27. 10. 1911}
\nopagebreak\mylabel{L03637v}
\rehead{ }\normalsize\beginnumbering\briefempfaengerindex{Schnitzler, Arthur@\textsc{Schnitzler, Arthur}!zzzZweig, Stefan@\emph{von Stefan Zweig}!1911-10-271@{27. 10. 1911}|(be}
\toendnotes[C]{\smallbreak\pagebreak[2]}
\correspDesc{Versand  durch Stefan Zweig am 27. 10. 1911 in Wien
\newline{}Erhalt  durch Arthur Schnitzler im Zeitraum [27. 10. 1911 – 29. 10. 1911?] in Wien}\toendnotes[C]{\smallbreak}
\Standort{CUL, Schnitzler, B 118.}
\physDesc{Brief, 1 Blatt, 1 Seite, 996 Zeichen
\newline{}Handschrift: blaue Tinte, lateinische Kurrent}
\buchAbdrucke{\weitereDrucke{Stefan Zweig: \emph{Briefwechsel mit Hermann Bahr, Sigmund Freud, Rainer
                                Maria Rilke und Arthur Schnitzler}. Herausgegeben von Jeffrey B. Berlin, Hans-Ulrich Lindken und Donald A. Prater. Frankfurt am Main: \emph{S. Fischer} 1987, S. 367.} }\toendnotes[C]{\smallbreak}
\pstart
           {\pb}\textcolor{gray}{\textbf{SZ}}\hfill \textcolor{gray}{\textbf{VIII. KOCHGASSE 8\oindex{Wien@\textbf{Wien}!VIII., Josefstadt@\textbf{VIII., Josefstadt}!Kochgasse 8@\textbf{Kochgasse 8}, \emph{Wohngebäude}|pw}}}\pend
           
\pstart
           \raggedleft{}\textcolor{gray}{\textbf{WIEN\oindex{Wien@\textbf{Wien}, \emph{Verwaltungsgebiet}|pw},}}{ }27. Oct. 11\pend
           \vspace{0.5em}
\pstart
           Sehr verehrter Herr Doktor, für das Gautier\pwindex{Gautier, Théophile 30.\,8.\,1811 Tarbes – 23.\,10.\,1872 Neuilly-sur-Seine@\textsc{Gautier, Théophile} (30.\,8.\,1811 Tarbes – 23.\,10.\,1872 Neuilly-sur-Seine), \emph{Schriftsteller, Kritiker, Maler}|pw} Denkmal habe ich gar nichts gezeichnet, sondern
                    nur unterzeichnet, weil ich nicht glaube, dass sich in Frankreich\oindex{Frankreich@\textbf{Frankreich}|pw} drei Francs für ein \introOben{}deutsches\introOben{}{ }Hebbel\pwindex{Hebbel, Friedrich 18.\,3.\,1813 Wesselburen – 13.\,12.\,1863 Wien@\textsc{Hebbel, Friedrich} (18.\,3.\,1813 Wesselburen – 13.\,12.\,1863 Wien), \emph{Schriftsteller}|pw} oder Kleist­\pwindex{Kleist, Heinrich von 18.\,10.\,1777 Frankfurt (Oder) – 21.\,11.\,1811 Kleiner Wannsee@\textsc{Kleist, Heinrich von} (18.\,10.\,1777 Frankfurt (Oder) – 21.\,11.\,1811 Kleiner Wannsee), \emph{Schriftsteller}|pw}-Denkmal zusa{\geminationm}enfinden würden. Frankreich\oindex{Frankreich@\textbf{Frankreich}|pw} braucht uns
                    wahrhaftig nicht.\pend
           
\pstart
           Noch eines: ich bitte Sie, officiell von meiner Mitteilung \label{K_L03637-1v}\edtext{wegen Ihres Buches}{\lemma{\textnormal{\emph{wegen Ihres Buches}}}\Cendnote{\textnormal{Der Verleger Samuel Fischer\pwindex{Fischer, Samuel 24.\,12.\,1859 Liptovský Mikuláš – 15.\,10.\,1934 Berlin@\textsc{Fischer, Samuel} (24.\,12.\,1859 Liptovský Mikuláš – 15.\,10.\,1934 Berlin), \emph{Verleger}|pwk} plante anläßlich des bevorstehenden 50. Geburtstages des Autors die Herausgabe einer
                        Monografie über Schnitzlers Werk
                         und suchte dafür einen Verfasser,
                            vgl. Hermann Bahr, Arthur Schnitzler: \emph{Briefwechsel, Aufzeichnungen, Dokumente (1891–1931)}, S. Fischer an Arthur Schnitzler, 27. 5. 1910.}}}\label{K_L03637-1} bei
                        Fischer\pwindex{Fischer, Samuel 24.\,12.\,1859 Liptovský Mikuláš – 15.\,10.\,1934 Berlin@\textsc{Fischer, Samuel} (24.\,12.\,1859 Liptovský Mikuláš – 15.\,10.\,1934 Berlin), \emph{Verleger}|pw} nichts zu wissen. Ich hatte
                    nur eine Art Schuldgefühl, dass ich selbst dieses Buch nicht übernahm (ich
                    stecke in einer \label{K_L03637-2v}\edtext{Arbeit}{\lemma{\textnormal{\emph{Arbeit}}}\Cendnote{\textnormal{Möglicherweise ist Zweigs\pwindex{Zweig, Stefan 28.\,11.\,1881 Wien – 23.\,2.\,1942 Petrópolis@\textsc{Zweig, Stefan} (28.\,11.\,1881 Wien – 23.\,2.\,1942 Petrópolis), \emph{Schriftsteller}|pwk} Novellenband \emph{Erstes Erlebnis}\pwindex{Zweig, Stefan 28.\,11.\,1881 Wien – 23.\,2.\,1942 Petrópolis@\textsc{Zweig, Stefan} (28.\,11.\,1881 Wien – 23.\,2.\,1942 Petrópolis), \emph{Schriftsteller}!Erstes Erlebnis. Vier Geschichten aus Kinderland@\strich\emph{Erstes Erlebnis. Vier Geschichten aus Kinderland}|pwk} gemeint, der im November 1911 bei \emph{Fischer}\orgindex{S. Fischer Verlag@S. Fischer Verlag|pwk}
                        erschien.}}}\label{K_L03637-2}, fühle mich übrigens nicht ganz zulänglich) und suchte dies
                    zu tilgen, indem ich den empfahl, der mir der Beste dünkte: Auernheimer\pwindex{Auernheimer, Raoul 15.\,4.\,1876 Wien – 6.\,1.\,1948 Oakland@\textsc{Auernheimer, Raoul} (15.\,4.\,1876 Wien – 6.\,1.\,1948 Oakland), \emph{Schriftsteller, Journalist, Kritiker}|pw}. Ich hoffe, es wird bald zustandekommen.
                    Auch die \label{K_L03637-3v}\edtext{Feier der
                        Fünfzigjährigen}{\lemma{\textnormal{\emph{Feier der Fünfzigjährigen}}}\Cendnote{\textnormal{Stefan Zweig\pwindex{Zweig, Stefan 28.\,11.\,1881 Wien – 23.\,2.\,1942 Petrópolis@\textsc{Zweig, Stefan} (28.\,11.\,1881 Wien – 23.\,2.\,1942 Petrópolis), \emph{Schriftsteller}|pwk} hatte angeregt, die
                        Generation wichtiger Schriftsteller, die sich in diesen Jahren in der
                        Lebensmitte befanden, zu ehren, indem im gesamten deutschsprachigen Raum am
                        jeweiligen 50. Geburtstag eines Autors dessen Stücke aufgeführt werden
                        sollten, vgl. Stefan Zweig\pwindex{Zweig, Stefan 28.\,11.\,1881 Wien – 23.\,2.\,1942 Petrópolis@\textsc{Zweig, Stefan} (28.\,11.\,1881 Wien – 23.\,2.\,1942 Petrópolis), \emph{Schriftsteller}|pwk}: \emph{Den Fünfzigjährigen! Eine öffentliche
                            Anregung}\pwindex{Zweig, Stefan 28.\,11.\,1881 Wien – 23.\,2.\,1942 Petrópolis@\textsc{Zweig, Stefan} (28.\,11.\,1881 Wien – 23.\,2.\,1942 Petrópolis), \emph{Schriftsteller}!Den Fünfzigjährigen Eine öffentliche Anregung@\strich\emph{Den Fünfzigjährigen{\rufezeichen} Eine öffentliche Anregung}|pwk}. In: \emph{Berliner
                                Tageblatt}\pwindex{Berliner Tageblatt@\emph{Berliner Tageblatt}|pwk}, Jg. 40, Nr. 464, 12. 9. 1911,
                            S. [2–3]. Namentlich erwähnt werden in dem Artikel Gerhart Hauptmann\pwindex{Hauptmann, Gerhart 15.\,11.\,1862 Szczawno-Zdrój – 6.\,6.\,1946 Jagniątków@\textsc{Hauptmann, Gerhart} (15.\,11.\,1862 Szczawno-Zdrój – 6.\,6.\,1946 Jagniątków), \emph{Schriftsteller}|pwk}, Arthur Schnitzler, Richard Dehmel\pwindex{Dehmel, Richard 18.\,11.\,1863 Hermsdorf – 8.\,2.\,1920 Blankenese@\textsc{Dehmel, Richard} (18.\,11.\,1863 Hermsdorf – 8.\,2.\,1920 Blankenese), \emph{Schriftsteller, Schriftsteller, Krimiautor}|pwk}, Maurice
                            Maeterlinck\pwindex{Maeterlinck, Maurice 29.\,8.\,1862 Gent – 6.\,5.\,1949 Nizza@\textsc{Maeterlinck, Maurice} (29.\,8.\,1862 Gent – 6.\,5.\,1949 Nizza), \emph{Schriftsteller}|pwk} und Hermann
                        Bahr\pwindex{Bahr, Hermann 1858/1859 – 1939 Wien@\textsc{Bahr, Hermann} (1858/1859 – 1939 Wien), \emph{Privatbeamter}|pwk}.}}}\label{K_L03637-3} wird organisiert werden.\pend
           
\pstart
           Ich bin \label{K_L03637-4v}\edtext{Ende November wieder in Wien\oindex{Wien@\textbf{Wien}, \emph{Verwaltungsgebiet}|pw}}{\lemma{\textnormal{\emph{Ende … Wien}}}\Cendnote{\textnormal{Er reiste nach Paris\oindex{Paris@\textbf{Paris}, \emph{Hauptstadt}|pwk}. Am 12. 12. 1911 fand das nächste belegte
                        Treffen statt. }}}\label{K_L03637-4} und freue mich dann innig, mich bei Ihnen wieder
                    anmelden zu können. Viele Grüsse Ihrer verehrten Frau Gemahlin\pwindex{Schnitzler, Olga 17.\,1.\,1882 Wien – 13.\,1.\,1970 Lugano@\textsc{Schnitzler, Olga} (17.\,1.\,1882 Wien – 13.\,1.\,1970 Lugano), \emph{Schauspielerin, Sängerin}|pwv} von Ihrem getreuen\pend
           \pstart \spacefill\mbox{Stefan Zweig}\pend{}
\pstart
           \noindent{}Mein »Haus am Meer\pwindex{Zweig, Stefan 28.\,11.\,1881 Wien – 23.\,2.\,1942 Petrópolis@\textsc{Zweig, Stefan} (28.\,11.\,1881 Wien – 23.\,2.\,1942 Petrópolis), \emph{Schriftsteller}!Haus am Meer. Ein Schauspiel in zwei Teilen (drei Aufzügen)@\strich\emph{Das Haus am Meer. Ein Schauspiel in zwei Teilen (drei Aufzügen)}|pw}« findet gute Freunde.
                        Ein paar der grossen Bühnen sind mir schon so viel wie sicher!\pend
           \selectlanguage{ngerman}\endnumbering\briefempfaengerindex{Schnitzler, Arthur@\textsc{Schnitzler, Arthur}!zzzZweig, Stefan@\emph{von Stefan Zweig}!1911-10-271@{27. 10. 1911}|)be}\mylabel{L03637h}  \newcommand{\dateiname}{L03637}\newcommand{\titel}{Stefan Zweig an Arthur Schnitzler, 27. 10. 1911}\newcommand{\editorInnen}{Selma Jahnke und Martin Anton Müller}%% latex-leseansicht-abspann.tex
%% Abspann für die Leseansicht.
%% Der Schalter \ifkorrekturansicht ist bereits durch den Vorspann gesetzt.

%% latex-abspann.tex
%% Gemeinsamer Abspann für Korrekturansicht und Leseansicht.
%% Setzt den Schalter \ifkorrekturansicht voraus (gesetzt in den
%% einbindenden Dateien latex-korrekturansicht-abspann.tex bzw.
%% latex-leseansicht-abspann.tex).
%% ---------------------------------------------------------------

\normalsize

% Das esempio-Environment wird nur in der Leseansicht benötigt
\ifkorrekturansicht\else
\newenvironment{esempio}[3]%
{
    \vspace{1.5ex}
    \rlap{\underline{#1}}
    \par
    \setlength{\parindent}{0cm}
    \nopagebreak
    \leftskip=#2cm
    \rightskip=#3cm
}
{
    \par
}
\fi

\doendnotes{C}
\bigskip
\vfill

\clearpage

\footnotesize

\ifkorrekturansicht
  \lohead{\textsc{register}}
\fi

% theindex-Environment neu definieren ohne reledmac
\makeatletter
\renewenvironment{theindex}{%
  \ifkorrekturansicht
    \section*{\indexname}%
  \else
    \subsubsection*{Index der erwähnten Entitäten}%
  \fi
  \setlength{\parindent}{0pt}%
  \setlength{\parskip}{0pt plus 0.3pt}%
  \let\item\@idxitem
}{%
  \ifkorrekturansicht\clearpage\fi
}
\makeatother

\IfFileExists{\jobname-pw.ind}{\input{\jobname-pw.ind}}{}

% Quellenangabe nur in der Leseansicht
\ifkorrekturansicht\else
% Fallback-Definitionen, falls die .tex-Datei \titel etc. nicht gesetzt hat
\providecommand{\titel}{}
\providecommand{\editorInnen}{}
\providecommand{\dateiname}{\jobname}

\vspace{3cm}

\vfill

\footnotesize
\textsc{Quelle}: \titel. Herausgegeben von {\editorInnen}. In: \emph{Arthur Schnitzler: Briefwechsel mit Autorinnen und Autoren}.
 Digitale Edition, https://schnitzler-briefe.acdh.oeaw.ac.at/{\dateiname}.html (Stand \today)
\fi

\end{document}


