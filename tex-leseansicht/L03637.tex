%% latex-korrekturansicht-vorspann.tex
%% Vorspann für die Korrekturansicht.
%% Lädt die gemeinsame Datei latex-vorspann.tex mit gesetztem Schalter.

\newif\ifkorrekturansicht
\korrekturansichttrue

\input{../tex-inputs/latex-vorspann}


\section[Stefan Zweig an Arthur Schnitzler, 27. 10. 1911]{L03637 Stefan Zweig an Arthur Schnitzler, 27. 10. 1911}
\nopagebreak\mylabel{L03637v}
\rehead{ }\normalsize\beginnumbering\briefempfaengerindex{Schnitzler, Arthur@\textsc{Schnitzler, Arthur}!zzzZweig, Stefan@\emph{von Stefan Zweig}!1911-10-271@{27. 10. 1911}|(be}
\toendnotes[C]{\smallbreak\pagebreak[2]}\Standort{CUL, Schnitzler, B 118.}
\physDesc{Brief, 1 Blatt, 1 Seite, 995 Zeichen
\newline{}Handschrift: blaue Tinte, lateinische Kurrent}
\buchAbdrucke{\weitereDrucke{Stefan Zweig: \emph{Briefwechsel mit Hermann Bahr, Sigmund Freud, Rainer
                                Maria Rilke und Arthur Schnitzler}. Frankfurt am Main: \emph{S. Fischer} 1987, S. 367.} }\toendnotes[C]{\smallbreak}
\pstart
           {\pb}\textcolor{gray}{\textbf{SZ}}\hfill \textcolor{gray}{\textbf{VIII. KOCHGASSE 8\oindex{Kochgasse 8@\textbf{Kochgasse 8}, \emph{Wohngebäude (K.WHS)}|pw}}}\pend
           
\pstart
           \raggedleft{}\textcolor{gray}{\textbf{WIEN\oindex{Wien@\textbf{Wien}, \emph{A.ADM2}|pw},}}{ }27. Oct. 11\pend
           \vspace{0.5em}
\pstart
           Sehr verehrter Herr Doktor, für das Gautier\pwindex{Gautier, Theophile 1811-08-30 – 1872-10-23@\textsc{Gautier, Théophile} (1811-08-30 – 1872-10-23), \emph{Schriftsteller/Schriftstellerin, Kritiker/Kritikerin, Maler/Malerin}|pw} Denkmal habe ich gar nichts gezeichnet, sondern
                    nur unterzeichnet, weil ich nicht glaube, dass sich in Frankreich\oindex{Frankreich@\textbf{Frankreich}, \emph{A.PCLI}|pw} drei Francs für ein \introOben{}deutsches\introOben{}{ }Hebbel\pwindex{Hebbel, Friedrich 18.03.1813 – 13.12.1863@\textsc{Hebbel, Friedrich} (18.03.1813 – 13.12.1863), \emph{Schriftsteller/Schriftstellerin}|pw} oder Kleist­\pwindex{Kleist, Heinrich von 18.10.1777 – 21.11.1811@\textsc{Kleist, Heinrich von} (18.10.1777 – 21.11.1811), \emph{Schriftsteller/Schriftstellerin}|pw}-Denkmal zusa{\geminationm}enfinden würden. Frankreich\oindex{Frankreich@\textbf{Frankreich}, \emph{A.PCLI}|pw} braucht uns
                    wahrhaftig nicht. \pend
           
\pstart
           Noch eines: ich bitte Sie, officiell von meiner Mitteilung \label{K_L03637-1v}\edtext{wegen Ihres Buches}{\lemma{\textnormal{\emph{wegen Ihres Buches}}}\Cendnote{\textnormal{Der Verleger Samuel Fischer\pwindex{Fischer, Samuel 24.12.1859 – 15.10.1934@\textsc{Fischer, Samuel} (24.12.1859 – 15.10.1934), \emph{Verleger/Verlegerin}|pwk} plante anläßlich des bevorstehenden 50. Geburtstages des Autors die Herausgabe einer
                        Monografie über Schnitzlers Werk
                         und suchte dafür einen Verfasser,
                            vgl. Hermann Bahr, Arthur Schnitzler: \emph{Briefwechsel, Aufzeichnungen, Dokumente (1891–1931)}, S. Fischer an Arthur Schnitzler, 27. 5. 1910.}}}\label{K_L03637-1} bei
                        Fischer\pwindex{Fischer, Samuel 24.12.1859 – 15.10.1934@\textsc{Fischer, Samuel} (24.12.1859 – 15.10.1934), \emph{Verleger/Verlegerin}|pw} nichts zu wissen. Ich hatte
                    nur eine Art Schuldgefühl, dass ich selbst dieses Buch nicht übernahm (ich
                    stecke in einer \label{K_L03637-2v}\edtext{Arbeit}{\lemma{\textnormal{\emph{Arbeit}}}\Cendnote{\textnormal{Möglicherweise ist Zweigs\pwindex{Zweig, Stefan 28.11.1881 – 23.02.1942@\textsc{Zweig, Stefan} (28.11.1881 – 23.02.1942), \emph{Schriftsteller/Schriftstellerin}|pwk} Novellenband \emph{Erstes Erlebnis}\pwindex{Erstes Erlebnis. Vier Geschichten aus Kinderland@\emph{Erstes Erlebnis. Vier Geschichten aus Kinderland}|pwk} gemeint, der im November
                            1911 bei \emph{Fischer}\orgindex{S. Fischer Verlag@S. Fischer Verlag|pwk}
                        erschien.}}}\label{K_L03637-2}, fühle mich übrigens nicht ganz zulänglich) und suchte dies
                    zu tilgen, indem ich den empfahl, der mir der Beste dünkte: Auernheimer\pwindex{Auernheimer, Raoul 15.04.1876 – 06.01.1948@\textsc{Auernheimer, Raoul} (15.04.1876 – 06.01.1948), \emph{Schriftsteller/Schriftstellerin, Journalist/Journalistin, Kritiker/Kritikerin}|pw}. Ich hoffe, es wird bald zustandekommen.
                    Auch die \label{K_L03637-3v}\edtext{Feier der
                        Fünfzigjährigen}{\lemma{\textnormal{\emph{Feier der Fünfzigjährigen}}}\Cendnote{\textnormal{Stefan Zweig\pwindex{Zweig, Stefan 28.11.1881 – 23.02.1942@\textsc{Zweig, Stefan} (28.11.1881 – 23.02.1942), \emph{Schriftsteller/Schriftstellerin}|pwk} hatte angeregt, die
                        Generation wichtiger Schriftsteller, die sich in diesen Jahren in der
                        Lebensmitte befanden, zu ehren, indem im gesamten deutschsprachigen Raum am
                        jeweiligen 50. Geburtstag eines Autors dessen Stücke aufgeführt werden
                        sollten, vgl. Stefan Zweig\pwindex{Zweig, Stefan 28.11.1881 – 23.02.1942@\textsc{Zweig, Stefan} (28.11.1881 – 23.02.1942), \emph{Schriftsteller/Schriftstellerin}|pwk}: \emph{Den Fünfzigjährigen! Eine öffentliche
                            Anregung}\pwindex{Den Fuenfzigjaehrigen Eine oeffentliche Anregung@\emph{Den Fünfzigjährigen{\rufezeichen} Eine öffentliche Anregung}|pwk}. In: \emph{Berliner
                                Tageblatt}\pwindex{Berliner Tageblatt@\emph{Berliner Tageblatt}|pwk}, Jg. 40, Nr. 464, 12. 9. 1911,
                            S. [2–3]. Namentlich erwähnt werden in dem Artikel Gerhart Hauptmann\pwindex{Hauptmann, Gerhart 15.11.1862 – 06.06.1946@\textsc{Hauptmann, Gerhart} (15.11.1862 – 06.06.1946), \emph{Schriftsteller/Schriftstellerin}|pwk}, Arthur Schnitzler, Richard Dehmel\pwindex{Dehmel, Richard 18.11.1863 – 08.02.1920@\textsc{Dehmel, Richard} (18.11.1863 – 08.02.1920), \emph{Schriftsteller/Schriftstellerin, Schriftsteller/Schriftstellerin, Krimiautor/Krimiautorin}|pwk}, Maurice
                            Maeterlinck\pwindex{Maeterlinck, Maurice 29.08.1862 – 06.05.1949@\textsc{Maeterlinck, Maurice} (29.08.1862 – 06.05.1949), \emph{Schriftsteller/Schriftstellerin}|pwk} und Hermann
                        Bahr\pwindex{Bahr, Hermann 1858/1859 – 1939@\textsc{Bahr, Hermann} (1858/1859 – 1939), \emph{Privatbeamter/Privatbeamtin}|pwk}.}}}\label{K_L03637-3} wird organisiert werden. \pend
           
\pstart
           Ich bin \label{K_L03637-4v}\edtext{Ende
                        November wieder in Wien\oindex{Wien@\textbf{Wien}, \emph{A.ADM2}|pw}}{\lemma{\textnormal{\emph{Ende … Wien}}}\Cendnote{\textnormal{Er reiste nach Paris\oindex{Paris@\textbf{Paris}, \emph{P.PPLC}|pwk}. Am 12. 12. 1911 fand das nächste belegte
                        Treffen statt. }}}\label{K_L03637-4} und freue mich dann innig, mich bei Ihnen wieder
                    anmelden zu können. Viele Grüsse Ihrer verehrten Frau Gemahlin\pwindex{Schnitzler, Olga 17.01.1882 – 13.01.1970@\textsc{Schnitzler, Olga} (17.01.1882 – 13.01.1970), \emph{Schauspieler/Schauspielerin, Sänger/Sängerin}|pwv} von Ihrem getreuen \pend
           \pstart \spacefill\mbox{Stefan Zweig}\pend{}
\pstart
           \noindent{}Mein »Haus am Meer\pwindex{Haus am Meer. Ein Schauspiel in zwei Teilen (drei Aufzuegen)@\emph{Das Haus am Meer. Ein Schauspiel in zwei Teilen (drei Aufzügen)}|pw}« findet gute Freunde.
                        Ein paar der grossen Bühnen sind mir schon so viel wie sicher!\pend
           \selectlanguage{ngerman}\endnumbering\briefempfaengerindex{Schnitzler, Arthur@\textsc{Schnitzler, Arthur}!zzzZweig, Stefan@\emph{von Stefan Zweig}!1911-10-271@{27. 10. 1911}|)be}\mylabel{L03637h}  \normalsize

\doendnotes{C}
\bigskip
\vfill

\clearpage

\footnotesize

\lohead{\textsc{register}}

% Definiere theindex-Environment komplett neu ohne reledmac
\makeatletter
\renewenvironment{theindex}{%
  \section*{\indexname}%
  \setlength{\parindent}{0pt}%
  \setlength{\parskip}{0pt plus 0.3pt}%
  \let\item\@idxitem
}{%
  \clearpage
}
\makeatother

\IfFileExists{\jobname-pw.ind}{\input{\jobname-pw.ind}}{}

\end{document}

      