%% latex-leseansicht-vorspann.tex
%% Vorspann für die Leseansicht.
%% Lädt die gemeinsame Datei latex-vorspann.tex mit nicht gesetztem Schalter.

\newif\ifkorrekturansicht
\korrekturansichtfalse

\input{../tex-inputs/latex-vorspann}


\section[Arthur Schnitzler an Stefan Zweig, 9. 11. 1911]{L03789 Arthur Schnitzler an Stefan Zweig, 9. 11. 1911}
\nopagebreak\mylabel{L03789v}
\rehead{ }\normalsize\beginnumbering\briefempfaengerindex{Zweig, Stefan@\textsc{Zweig, Stefan}!zzzSchnitzler, Arthur@\emph{von Arthur Schnitzler}!1911-11-091@{9. 11. 1911}|(be}
\toendnotes[C]{\smallbreak\pagebreak[2]}
\correspDesc{Versand  durch Arthur Schnitzler am 9. 11. 1911 in Hamburg
\newline{}Erhalt  durch Stefan Zweig im Zeitraum [10. 11. 1911 – 14. 11. 1911?] in Wien}\toendnotes[C]{\smallbreak}
\Standort{Jerusalem, National Library of Israel, ARC. Ms. Var. 305 1 58 Stefan Zweig Collection.}
\physDesc{Bildpostkarte, 457 Zeichen
\newline{}Handschrift: schwarze Tinte, lateinische Kurrent
\newline{}Versand: Stempel: »\nobreak{}\oindex{Hamburg@\textbf{Hamburg}|pwk}Hamburg, 9. 11. 11, 10–11M\nobreak{}«.  }\toendnotes[C]{\smallbreak}\pstart{}{\pb}Herrn Dr. Stephan Zweig\pend{}\pstart{}Wien VIII\oindex{VIII., Josefstadt@\textbf{VIII., Josefstadt}, \emph{Verwaltungsgebiet}|pw}\pend{}\pstart{}Kochgasse 8\oindex{Wien@\textbf{Wien}!VIII., Josefstadt@\textbf{VIII., Josefstadt}!Kochgasse 8@\textbf{Kochgasse 8}, \emph{Wohngebäude}|pw}\pend{}{\bigskip}
\pstart
           {\pb}\textcolor{gray}{\textbf{Hamburg, Hotel Atlantic\oindex{Hotel Atlantic@\textbf{Hotel Atlantic}, \emph{Hotel}|pw}}}\hfill \textcolor{gray}{\textbf{Restaurant Pfordte}}\oindex{Restaurant Pfordte im Hotel Atlantic@\textbf{Restaurant Pfordte im Hotel Atlantic}, \emph{Restaurant}|pw}\pend
           \vspace{1em}
\pstart
           \centering{}{\pb}Hamburg\oindex{Hamburg@\textbf{Hamburg}|pw}, 9. 11. 911\pend
           \vspace{0.5em}
\pstart
           {\pb}lieber Herr Doctor, außer Herrn Morisse\pwindex{Morisse, Paul 11.\,3.\,1866 Rouen – 28.\,9.\,1946 Paris@\textsc{Morisse, Paul} (11.\,3.\,1866 Rouen – 28.\,9.\,1946 Paris), \emph{Übersetzer}|pw} haben sich noch andre Übersetzer gemeldet, darunter Rémon\pwindex{Rémon, Maurice 27.\,11.\,1861 Paris – 20.\,6.\,1945 Mérignac@\textsc{Rémon, Maurice} (27.\,11.\,1861 Paris – 20.\,6.\,1945 Mérignac), \emph{Übersetzer}|pw}, der schon einiges\pwindex{Schnitzler, Arthur 15.\,5.\,1862 Wien – 21.\,10.\,1931 ebd.@\textsc{Schnitzler, Arthur} (15.\,5.\,1862 Wien – 21.\,10.\,1931 ebd.), \emph{Schriftsteller, Mediziner}!femme au poignard@\strich\emph{La femme au poignard}|pwv}\pwindex{Schnitzler, Arthur 15.\,5.\,1862 Wien – 21.\,10.\,1931 ebd.@\textsc{Schnitzler, Arthur} (15.\,5.\,1862 Wien – 21.\,10.\,1931 ebd.), \emph{Schriftsteller, Mediziner}!Morts se taisent@\strich\emph{Les Morts se taisent}|pwv}\pwindex{Schnitzler, Arthur 15.\,5.\,1862 Wien – 21.\,10.\,1931 ebd.@\textsc{Schnitzler, Arthur} (15.\,5.\,1862 Wien – 21.\,10.\,1931 ebd.), \emph{Schriftsteller, Mediziner}!Jour de gloire@\strich\emph{Jour de gloire}|pwv} von mir übersetzt
               und auch zur Veröffentlichg gebracht hat. Entschieden hab ich mich noch nicht. Hat
               Morisse\pwindex{Morisse, Paul 11.\,3.\,1866 Rouen – 28.\,9.\,1946 Paris@\textsc{Morisse, Paul} (11.\,3.\,1866 Rouen – 28.\,9.\,1946 Paris), \emph{Übersetzer}|pw} schon \uline{Stücke} übersetzt? Vielen Dank für Ihr
               freundliches Interesse. Gratuliere zu den schönen Annahmen. Lassen Sie sich’s in Meran\oindex{Meran@\textbf{Meran}, \emph{Hauptstadt}|pw} wohl ergehen. Herzliche Grüße. Ihr\pend
           \pstart \spacefill\mbox{Arthur Schnitzler}\pend{}\selectlanguage{ngerman}\endnumbering\briefempfaengerindex{Zweig, Stefan@\textsc{Zweig, Stefan}!zzzSchnitzler, Arthur@\emph{von Arthur Schnitzler}!1911-11-091@{9. 11. 1911}|)be}\mylabel{L03789h}  \newcommand{\dateiname}{L03789}\newcommand{\titel}{Arthur Schnitzler an Stefan Zweig, 9. 11. 1911}\newcommand{\editorInnen}{Selma Jahnke und Martin Anton Müller}%% latex-leseansicht-abspann.tex
%% Abspann für die Leseansicht.
%% Der Schalter \ifkorrekturansicht ist bereits durch den Vorspann gesetzt.

%% latex-abspann.tex
%% Gemeinsamer Abspann für Korrekturansicht und Leseansicht.
%% Setzt den Schalter \ifkorrekturansicht voraus (gesetzt in den
%% einbindenden Dateien latex-korrekturansicht-abspann.tex bzw.
%% latex-leseansicht-abspann.tex).
%% ---------------------------------------------------------------

\normalsize

% Das esempio-Environment wird nur in der Leseansicht benötigt
\ifkorrekturansicht\else
\newenvironment{esempio}[3]%
{
    \vspace{1.5ex}
    \rlap{\underline{#1}}
    \par
    \setlength{\parindent}{0cm}
    \nopagebreak
    \leftskip=#2cm
    \rightskip=#3cm
}
{
    \par
}
\fi

\doendnotes{C}
\bigskip
\vfill

\clearpage

\footnotesize

\ifkorrekturansicht
  \lohead{\textsc{register}}
\fi

% theindex-Environment neu definieren ohne reledmac
\makeatletter
\renewenvironment{theindex}{%
  \ifkorrekturansicht
    \section*{\indexname}%
  \else
    \subsubsection*{Index der erwähnten Entitäten}%
  \fi
  \setlength{\parindent}{0pt}%
  \setlength{\parskip}{0pt plus 0.3pt}%
  \let\item\@idxitem
}{%
  \ifkorrekturansicht\clearpage\fi
}
\makeatother

\IfFileExists{\jobname-pw.ind}{\input{\jobname-pw.ind}}{}

% Quellenangabe nur in der Leseansicht
\ifkorrekturansicht\else
% Fallback-Definitionen, falls die .tex-Datei \titel etc. nicht gesetzt hat
\providecommand{\titel}{}
\providecommand{\editorInnen}{}
\providecommand{\dateiname}{\jobname}

\vspace{3cm}

\vfill

\footnotesize
\textsc{Quelle}: \titel. Herausgegeben von {\editorInnen}. In: \emph{Arthur Schnitzler: Briefwechsel mit Autorinnen und Autoren}.
 Digitale Edition, https://schnitzler-briefe.acdh.oeaw.ac.at/{\dateiname}.html (Stand \today)
\fi

\end{document}


