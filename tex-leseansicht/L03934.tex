%% latex-leseansicht-vorspann.tex
%% Vorspann für die Leseansicht.
%% Lädt die gemeinsame Datei latex-vorspann.tex mit nicht gesetztem Schalter.

\newif\ifkorrekturansicht
\korrekturansichtfalse

\input{../tex-inputs/latex-vorspann}


\section[Arthur Schnitzler an Theodor Herzl, {[}zwischen 5. und 7. 4. 1896?{]}]{L03934 Arthur Schnitzler an Theodor Herzl, {[}zwischen 5. und 7. 4. 1896?{]}}
\nopagebreak\mylabel{L03934v}
\rehead{ }\normalsize\beginnumbering\briefempfaengerindex{Herzl, Theodor@\textsc{Herzl, Theodor}!zzzSchnitzler, Arthur@\emph{von Arthur Schnitzler}!1896-04-071@{{[}zwischen 5. und 7. 4. 1896?{]}}|(be}
\toendnotes[C]{\smallbreak\pagebreak[2]}
\correspDesc{Versand  durch Arthur Schnitzler im Zeitraum [zwischen 5. und 7. 4. 1896?] in Wien
\newline{}Erhalt  durch Theodor Herzl in Wien}\toendnotes[C]{\smallbreak}
\Standort{Jerusalem, Central Zionist Archives, H1:1925-19.}
\physDesc{Brief, 1 Blatt, 2 Seiten
\newline{}Handschrift: schwarze Tinte, deutsche Kurrent}\toendnotes[C]{\smallbreak}
\pstart{}{\pb}Mein lieber Freund,\pend\vspace{0.5em}
\pstart
           für eine künſtleriſche Freude, wie Sie{ }ſie mir und manchen anderen Verſtehern
      durch Ihre \label{K_L03934-1v}\edtext{traurig-ſchöne Oſterphantasie\pwindex{Herzl, Theodor 2.\,5.\,1860 Budapest – 3.\,7.\,1904 Edlach@\textsc{Herzl, Theodor} (2.\,5.\,1860 Budapest – 3.\,7.\,1904 Edlach), \emph{Schriftsteller, Journalist}!Frühling im Elend@\strich\emph{Frühling im Elend}|pwv}}{\lemma{\textnormal{\emph{traurig-schöne Osterphantasie}}}\Cendnote{\textnormal{Das vorliegende
      Korrespondenzstück ist undatiert. Herzl\pwindex{Herzl, Theodor 2.\,5.\,1860 Budapest – 3.\,7.\,1904 Edlach@\textsc{Herzl, Theodor} (2.\,5.\,1860 Budapest – 3.\,7.\,1904 Edlach), \emph{Schriftsteller, Journalist}|pwk} hat in den für die Korrespondenz relevanten Jahren 1893 bis 1902 weitgehend jährlich einen
         Text zu den Ostersonntagausgaben der \emph{Neuen Freie Presse}\pwindex{Neue Freie Presse@\emph{Neue Freie Presse}|pwk} beigesteuert. Trotzdem lassen sich auf kaum einen die von Schnitzler gebrauchten
         Beschreibungen »traurig-ſchön« und »Osterphantasie«  anuwenden. Die Identifikation mit dem Beitrag \emph{Frühling im Elend}\pwindex{Herzl, Theodor 2.\,5.\,1860 Budapest – 3.\,7.\,1904 Edlach@\textsc{Herzl, Theodor} (2.\,5.\,1860 Budapest – 3.\,7.\,1904 Edlach), \emph{Schriftsteller, Journalist}!Frühling im Elend@\strich\emph{Frühling im Elend}|pwk} (\emph{Neue Freie Presse}\pwindex{Neue Freie Presse@\emph{Neue Freie Presse}|pwk}, Nr. 11.357, 5. 4. 1896, Morgenblatt, S. 7–8.)
         gelingt durch die Danksagung Herzls\pwindex{Herzl, Theodor 2.\,5.\,1860 Budapest – 3.\,7.\,1904 Edlach@\textsc{Herzl, Theodor} (2.\,5.\,1860 Budapest – 3.\,7.\,1904 Edlach), \emph{Schriftsteller, Journalist}|pwk} vom XXXX Auszeichnungsfehler: Dokument L03866 nicht gefunden.}}}\label{K_L03934-1}
      bereitet haben, ſollte man irgendwie
      dankbar ſein können. Man ka{\geminationn}
      es nur, indem man ſagt: Wie ſchön!
               {\pb}Ich habe zwar die angenehm Ueberzeugung, daſs Sie dasſelbe fühlen, aber
               dieſes Feu{[}i{]}lleton\pwindex{Herzl, Theodor 2.\,5.\,1860 Budapest – 3.\,7.\,1904 Edlach@\textsc{Herzl, Theodor} (2.\,5.\,1860 Budapest – 3.\,7.\,1904 Edlach), \emph{Schriftsteller, Journalist}!Frühling im Elend@\strich\emph{Frühling im Elend}|pwv} iſt nochmal ſchöner
         als Sie glauben.\pend
           \pstart Herzliche Grüße! Ihr \spacefill\mbox{ArthurSchn}\pend{}\selectlanguage{ngerman}\endnumbering\briefempfaengerindex{Herzl, Theodor@\textsc{Herzl, Theodor}!zzzSchnitzler, Arthur@\emph{von Arthur Schnitzler}!1896-04-051@{{[}zwischen 5. und 7. 4. 1896?{]}}|)be}\mylabel{L03934h}
\begin{anhang}
\end{anhang}\newcommand{\dateiname}{L03934}\newcommand{\titel}{Arthur Schnitzler an Theodor Herzl, [zwischen 5. und 7. 4. 1896?]}\newcommand{\editorInnen}{Herausgegeben von Jahnke, SelmaMüller, Martin Anton}%% latex-leseansicht-abspann.tex
%% Abspann für die Leseansicht.
%% Der Schalter \ifkorrekturansicht ist bereits durch den Vorspann gesetzt.

%% latex-abspann.tex
%% Gemeinsamer Abspann für Korrekturansicht und Leseansicht.
%% Setzt den Schalter \ifkorrekturansicht voraus (gesetzt in den
%% einbindenden Dateien latex-korrekturansicht-abspann.tex bzw.
%% latex-leseansicht-abspann.tex).
%% ---------------------------------------------------------------

\normalsize

% Das esempio-Environment wird nur in der Leseansicht benötigt
\ifkorrekturansicht\else
\newenvironment{esempio}[3]%
{
    \vspace{1.5ex}
    \rlap{\underline{#1}}
    \par
    \setlength{\parindent}{0cm}
    \nopagebreak
    \leftskip=#2cm
    \rightskip=#3cm
}
{
    \par
}
\fi

\doendnotes{C}
\bigskip
\vfill

\clearpage

\footnotesize

\ifkorrekturansicht
  \lohead{\textsc{register}}
\fi

% theindex-Environment neu definieren ohne reledmac
\makeatletter
\renewenvironment{theindex}{%
  \ifkorrekturansicht
    \section*{\indexname}%
  \else
    \subsubsection*{Index der erwähnten Entitäten}%
  \fi
  \setlength{\parindent}{0pt}%
  \setlength{\parskip}{0pt plus 0.3pt}%
  \let\item\@idxitem
}{%
  \ifkorrekturansicht\clearpage\fi
}
\makeatother

\IfFileExists{\jobname-pw.ind}{\input{\jobname-pw.ind}}{}

% Quellenangabe nur in der Leseansicht
\ifkorrekturansicht\else
% Fallback-Definitionen, falls die .tex-Datei \titel etc. nicht gesetzt hat
\providecommand{\titel}{}
\providecommand{\editorInnen}{}
\providecommand{\dateiname}{\jobname}

\vspace{3cm}

\vfill

\footnotesize
\textsc{Quelle}: \titel. Herausgegeben von {\editorInnen}. In: \emph{Arthur Schnitzler: Briefwechsel mit Autorinnen und Autoren}.
 Digitale Edition, https://schnitzler-briefe.acdh.oeaw.ac.at/{\dateiname}.html (Stand \today)
\fi

\end{document}


