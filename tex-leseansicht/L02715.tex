%% latex-leseansicht-vorspann.tex
%% Vorspann für die Leseansicht.
%% Lädt die gemeinsame Datei latex-vorspann.tex mit nicht gesetztem Schalter.

\newif\ifkorrekturansicht
\korrekturansichtfalse

\input{../tex-inputs/latex-vorspann}


         
         \renewcommand{\erwaehntePersonen}{Personen: Paul Goldmann}
         \renewcommand{\erwaehnteOrte}{Orte: Getreidegasse, Grillparzerstraße, Hotel Goldenes Horn, Salzburg, Wien}
         \renewcommand{\erwaehnteWerke}{}
               \section[Paul Goldmann an Arthur Schnitzler, 12. 9. 1893]{ Paul Goldmann an Arthur Schnitzler, 12. 9. 1893}\nopagebreak\mylabel{v}\rehead{ }\begin{ledgroupsized}[t]{13cm}\normalsize\beginnumbering\briefempfaengerindex{Schnitzler, Arthur@\textsc{Schnitzler, Arthur}!zzzGoldmann, Paul@\emph{von Paul Goldmann}!1893-09-121@{12. 9. 1893}|(be} \toendnotes[C]{\smallbreak\pagebreak[2]} \Standort{DLA, A:Schnitzler, HS.NZ85.1.3163.}
\physDesc{Kartenbrief, 391 Zeichen
\newline{}Handschrift: schwarze Tinte, deutsche Kurrent
\newline{}Versand: 1) Stempel: »\nobreak{}\oindex{Salzburg@\textbf{Salzburg}|pwk}Salzburg Stadt, 12/9 93, 6 A\nobreak{}«.   2) Stempel: »\nobreak{}Wien 1/1 1, {[}1{]}3/9.  93, 8–9½ V\textcolor{gray}{.}, {[}B{]}estellt\nobreak{}«. 
\newline{}Schnitzler: mit schwarzer Tinte das Jahr »93« vermerkt }\toendnotes[C]{\smallbreak}\pstart{}{\pb}\textsc{Herrn}\pend{}\pstart{}\textsc{Dr. Arthur \label{T_L02715-1v}\edtext{Schnitzler}{\lemma{\textnormal{\emph{Schnitzler}}}\Cendnote{\textnormal{zur Verdeutlichung
                        aufgrund von Wasserflecken neuerlich der Nachname
                        »Schnitzler« darüber geschrieben}}}\label{T_L02715-1h}}\pend{}\pstart{}\textsc{Wien\oindex{Wien@\textbf{Wien}|pw}}\pend{}\pstart{}\textsc{I. Grillparzerstraſse 7}\oindex{Grillparzerstrasse@\textbf{Grillparzerstraße}|pw}.\pend{}{\bigskip}\pstart
           \noindent{}\raggedleft{}{\pb}\textsc{Salzburg\oindex{Salzburg@\textbf{Salzburg}|pw}}, 12. September. \pend
           \pstart\center{}Mein lieber Freund!\pend\pstart
           Ich bin in \textsc{Salzburg\oindex{Salzburg@\textbf{Salzburg}|pw}}, Hotel Goldenes Horn\oindex{Hotel Goldenes Horn@\textbf{Hotel Goldenes Horn}|pw}, \label{K_L02715-1v}\edtext{Getreidemarkt\oindex{Getreidegasse@\textbf{Getreidegasse}|pwv}}{\lemma{\textnormal{\emph{Getreidemarkt}}}\Cendnote{\textnormal{Getreidegasse\oindex{Getreidegasse@\textbf{Getreidegasse}|pwk}, vgl. Paul Goldmann an Arthur Schnitzler, 14. 9. [1893]}}}\label{K_L02715-1h}, und erwarte Dich mit Ungeduld. Bin geſtern{ }Abend angekommen und werde etwa acht Tage bleiben. Die Freude, Dich zu
               ſehen, wirſt Du mir nicht vorenthalten, nicht wahr? Nur bitte ich um vorherige
               telegraphiſche Nachricht.\pend
           \pstart
           In Treue{\\[\baselineskip]}Dein {\\[\baselineskip]}\spacefill\mbox{Paul Goldmann.}\pend
           \leftskip=0em{}
         
         \endnumbering\mylabel{h}\end{ledgroupsized}  \newcommand{\dateiname}{L02715}\newcommand{\titel}{Paul Goldmann an Arthur Schnitzler, 12. 9. 1893}\newcommand{\editorInnen}{Martin Anton Müller und Laura Untner}%% latex-leseansicht-abspann.tex
%% Abspann für die Leseansicht.
%% Der Schalter \ifkorrekturansicht ist bereits durch den Vorspann gesetzt.

%% latex-abspann.tex
%% Gemeinsamer Abspann für Korrekturansicht und Leseansicht.
%% Setzt den Schalter \ifkorrekturansicht voraus (gesetzt in den
%% einbindenden Dateien latex-korrekturansicht-abspann.tex bzw.
%% latex-leseansicht-abspann.tex).
%% ---------------------------------------------------------------

\normalsize

% Das esempio-Environment wird nur in der Leseansicht benötigt
\ifkorrekturansicht\else
\newenvironment{esempio}[3]%
{
    \vspace{1.5ex}
    \rlap{\underline{#1}}
    \par
    \setlength{\parindent}{0cm}
    \nopagebreak
    \leftskip=#2cm
    \rightskip=#3cm
}
{
    \par
}
\fi

\doendnotes{C}
\bigskip
\vfill

\clearpage

\footnotesize

\ifkorrekturansicht
  \lohead{\textsc{register}}
\fi

% theindex-Environment neu definieren ohne reledmac
\makeatletter
\renewenvironment{theindex}{%
  \ifkorrekturansicht
    \section*{\indexname}%
  \else
    \subsubsection*{Index der erwähnten Entitäten}%
  \fi
  \setlength{\parindent}{0pt}%
  \setlength{\parskip}{0pt plus 0.3pt}%
  \let\item\@idxitem
}{%
  \ifkorrekturansicht\clearpage\fi
}
\makeatother

\IfFileExists{\jobname-pw.ind}{\input{\jobname-pw.ind}}{}

% Quellenangabe nur in der Leseansicht
\ifkorrekturansicht\else
% Fallback-Definitionen, falls die .tex-Datei \titel etc. nicht gesetzt hat
\providecommand{\titel}{}
\providecommand{\editorInnen}{}
\providecommand{\dateiname}{\jobname}

\vspace{3cm}

\vfill

\footnotesize
\textsc{Quelle}: \titel. Herausgegeben von {\editorInnen}. In: \emph{Arthur Schnitzler: Briefwechsel mit Autorinnen und Autoren}.
 Digitale Edition, https://schnitzler-briefe.acdh.oeaw.ac.at/{\dateiname}.html (Stand \today)
\fi

\end{document}


      