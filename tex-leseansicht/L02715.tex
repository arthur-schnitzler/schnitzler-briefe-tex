%% latex-korrekturansicht-vorspann.tex
%% Vorspann für die Korrekturansicht.
%% Lädt die gemeinsame Datei latex-vorspann.tex mit gesetztem Schalter.

\newif\ifkorrekturansicht
\korrekturansichttrue

\input{../tex-inputs/latex-vorspann}


\section[Paul Goldmann an Arthur Schnitzler, 12. 9. 1893]{L02715 Paul Goldmann an Arthur Schnitzler, 12. 9. 1893}
\nopagebreak\mylabel{L02715v}
\rehead{ }\normalsize\beginnumbering\briefempfaengerindex{Schnitzler, Arthur@\textsc{Schnitzler, Arthur}!zzzGoldmann, Paul@\emph{von Paul Goldmann}!1893-09-121@{12. 9. 1893}|(be}
\toendnotes[C]{\smallbreak\pagebreak[2]}\Standort{DLA, A:Schnitzler, HS.NZ85.1.3163.}
\physDesc{Kartenbrief, 391 Zeichen
\newline{}Handschrift: schwarze Tinte, deutsche Kurrent
\newline{}Versand: 1) Stempel: »\nobreak{}\oindex{Salzburg@\textbf{Salzburg}, \emph{A.ADM2}|pwk}Salzburg Stadt, 12/9 93, 6 A\nobreak{}«.   2) Stempel: »\nobreak{}Wien 1/1 1, {[}1{]}3/9.  93, 8–9½ V\textcolor{gray}{.}, {[}B{]}estellt\nobreak{}«. 
\newline{}Schnitzler: mit schwarzer Tinte das Jahr »93« vermerkt }\toendnotes[C]{\smallbreak}\pstart{}{\pb}\textsc{Herrn}\pend{}\pstart{}\textsc{Dr. Arthur \label{T_L02715-1v}\edtext{Schnitzler}{\lemma{\textnormal{\emph{Schnitzler}}}\Cendnote{\textnormal{zur Verdeutlichung
                        aufgrund von Wasserflecken neuerlich der Nachname
                        »Schnitzler« darüber geschrieben}}}\label{T_L02715-1}}\pend{}\pstart{}\textsc{Wien\oindex{Wien@\textbf{Wien}, \emph{A.ADM2}|pw}}\pend{}\pstart{}\textsc{I. Grillparzerstraſse 7}\oindex{Grillparzerstrasse@\textbf{Grillparzerstraße}, \emph{R.ST}|pw}.\pend{}{\bigskip}\vspace{1em}
\pstart
           \raggedleft{}{\pb}\textsc{Salzburg\oindex{Salzburg@\textbf{Salzburg}, \emph{A.ADM2}|pw}}, 12. September. \pend
           
\pstart\center{}Mein lieber Freund!\pend\vspace{0.5em}
\pstart
           Ich bin in \textsc{Salzburg\oindex{Salzburg@\textbf{Salzburg}, \emph{A.ADM2}|pw}}, Hotel Goldenes Horn\oindex{Hotel Goldenes Horn@\textbf{Hotel Goldenes Horn}, \emph{Hotel (K.HTL)}|pw}, \label{K_L02715-1v}\edtext{Getreidemarkt\oindex{Getreidegasse@\textbf{Getreidegasse}, \emph{R.RD}|pwv}}{\lemma{\textnormal{\emph{Getreidemarkt}}}\Cendnote{\textnormal{Getreidegasse\oindex{Getreidegasse@\textbf{Getreidegasse}, \emph{R.RD}|pwk}, vgl. Paul Goldmann an Arthur Schnitzler, 14. 9. [1893].
               }}}\label{K_L02715-1}, und erwarte Dich mit Ungeduld. Bin geſtern{ }Abend angekommen und werde etwa acht Tage bleiben. Die Freude, Dich zu
               ſehen, wirſt Du mir nicht vorenthalten, nicht wahr? Nur bitte ich um vorherige
               telegraphiſche Nachricht.\pend
           
\pstart
           In Treue{\\[\baselineskip]}Dein {\\[\baselineskip]}\spacefill\mbox{Paul Goldmann.}\pend
           \leftskip=0em{}\selectlanguage{ngerman}\endnumbering\briefempfaengerindex{Schnitzler, Arthur@\textsc{Schnitzler, Arthur}!zzzGoldmann, Paul@\emph{von Paul Goldmann}!1893-09-121@{12. 9. 1893}|)be}\mylabel{L02715h}  \normalsize

\doendnotes{C}
\bigskip
\vfill

\clearpage

\footnotesize

\lohead{\textsc{register}}

% Definiere theindex-Environment komplett neu ohne reledmac
\makeatletter
\renewenvironment{theindex}{%
  \section*{\indexname}%
  \setlength{\parindent}{0pt}%
  \setlength{\parskip}{0pt plus 0.3pt}%
  \let\item\@idxitem
}{%
  \clearpage
}
\makeatother

\IfFileExists{\jobname-pw.ind}{\input{\jobname-pw.ind}}{}

\end{document}

      