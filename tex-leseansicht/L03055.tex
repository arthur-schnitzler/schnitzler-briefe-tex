%% latex-korrekturansicht-vorspann.tex
%% Vorspann für die Korrekturansicht.
%% Lädt die gemeinsame Datei latex-vorspann.tex mit gesetztem Schalter.

\newif\ifkorrekturansicht
\korrekturansichttrue

\input{../tex-inputs/latex-vorspann}


\section[ Paul Goldmann an Arthur Schnitzler, 22. 1. {[}1901{]}]{L03055 Paul Goldmann an Arthur Schnitzler, 22. 1. {[}1901{]}}
\nopagebreak\mylabel{L03055v}
\rehead{ }\normalsize\beginnumbering\briefempfaengerindex{Schnitzler, Arthur@\textsc{Schnitzler, Arthur}!zzzGoldmann, Paul@\emph{von Paul Goldmann}!1901-01-221@{22. 1. {[}1901{]}}|(be}
\toendnotes[C]{\smallbreak\pagebreak[2]}\Standort{DLA, A:Schnitzler, HS.NZ85.1.3171.}
\physDesc{Brief, 1 Blatt, 4 Seiten, 1730 Zeichen
\newline{}Handschrift: blaue Tinte, deutsche Kurrent
\newline{}Schnitzler: mit rotem Buntstift sechs Unterstreichungen }\toendnotes[C]{\smallbreak}
\pstart
           \raggedleft{}{\pb}\textcolor{gray}{\textbf{DESSAUERSTRASSE 19}}\oindex{Dessauer Strasse@\textbf{Dessauer Straße}, \emph{Straße (K.STR)}|pw}\pend
           
\pstart
           Berlin\oindex{Berlin@\textbf{Berlin}, \emph{P.PPLC}|pw}, 22. Januar.\pend
           
\pstart\center{}Mein lieber Freund,\pend\vspace{0.5em}
\pstart
           \textsc{Mizzi Glümer\pwindex{Gluemer, Marie 03.07.1867 – 16.11.1925@\textsc{Glümer, Marie} (03.07.1867 – 16.11.1925), \emph{Schauspieler/Schauspielerin}|pw}} iſt krank, liegt zu Bett und ſieht ſo elend aus, daß ich erſchrocken bin (Unter
                  uns!)\textcolor{gray}{.} Du ſollteſt dem armen Mädel\pwindex{Gluemer, Marie 03.07.1867 – 16.11.1925@\textsc{Glümer, Marie} (03.07.1867 – 16.11.1925), \emph{Schauspieler/Schauspielerin}|pwv} einen guten Brief ſchreiben.\pend
           
\pstart
           \textsc{Brahm\pwindex{Brahm, Otto 05.02.1856 – 28.11.1912@\textsc{Brahm, Otto} (05.02.1856 – 28.11.1912), \emph{Theaterleiter/Theaterleiterin, Regisseur/Regisseurin}|pw}} ſagte mir bei einer der letzten \begin{otherlanguage}{french}\textsc{Premièren}\end{otherlanguage}, er möchte von Dir einen oder zwei \label{K_L03055-1v}\edtext{Einakter}{\lemma{\textnormal{\emph{Einakter}}}\Cendnote{\textnormal{Am 12. 2. 1901 hatte sich Brahm\pwindex{Brahm, Otto 05.02.1856 – 28.11.1912@\textsc{Brahm, Otto} (05.02.1856 – 28.11.1912), \emph{Theaterleiter/Theaterleiterin, Regisseur/Regisseurin}|pwk} bei Schnitzler
                  für den Erhalt von \emph{Zum großen Wurstel}\pwindex{Zum grossen Wurstel. Burleske in einem Akt@\emph{Zum großen Wurstel. Burleske in einem Akt}|pwk} (noch
                  unter dem Entstehungstitel »Marionetten\pwindex{Zum grossen Wurstel. Burleske in einem Akt@\emph{Zum großen Wurstel. Burleske in einem Akt}|pwkv}«) bedankt, vgl. \emph{Der Briefwechsel
                        Arthur Schnitzler – Otto Brahm}. Vollständige Ausgabe. Herausgegeben,
                     eingeleitet und erläutert von Oskar Seidlin. Tübingen:
                        \emph{Niemeyer}{ }1975, S. 88.}}}\label{K_L03055-1} haben. Wer iſt \textsc{Stefan Vacano\pwindex{Vacano, Stefan 1874-12-12 – 1963@\textsc{Vacano, Stefan} (1874-12-12 – 1963), \emph{Schriftsteller/Schriftstellerin}|pw}}? Ich kann mir die Aufführung ſeines Stück\pwindex{Tag@\emph{Der Tag}|pwv}es nur durch \label{K_L03055-2v}\edtext{Beziehungen}{\lemma{\textnormal{\emph{Beziehungen}}}\Cendnote{\textnormal{Der
                     Wien\oindex{Wien@\textbf{Wien}, \emph{A.ADM2}|pwk}er Stefan Vacano\pwindex{Vacano, Stefan 1874-12-12 – 1963@\textsc{Vacano, Stefan} (1874-12-12 – 1963), \emph{Schriftsteller/Schriftstellerin}|pwk} war Theaterdichter und als Regieassistent bei Otto Brahm\pwindex{Brahm, Otto 05.02.1856 – 28.11.1912@\textsc{Brahm, Otto} (05.02.1856 – 28.11.1912), \emph{Theaterleiter/Theaterleiterin, Regisseur/Regisseurin}|pwk} tätig. Möglicherweise war die
                  Freundschaft zwischen den beiden Männer\pwindex{Vacano, Stefan 1874-12-12 – 1963@\textsc{Vacano, Stefan} (1874-12-12 – 1963), \emph{Schriftsteller/Schriftstellerin}|pwk}\pwindex{Brahm, Otto 05.02.1856 – 28.11.1912@\textsc{Brahm, Otto} (05.02.1856 – 28.11.1912), \emph{Theaterleiter/Theaterleiterin, Regisseur/Regisseurin}|pwk}n auch persönlicher Natur, da sich diese Stelle durchaus auch als
                  leicht verklausulierter Hinweis auf Brahms\pwindex{Brahm, Otto 05.02.1856 – 28.11.1912@\textsc{Brahm, Otto} (05.02.1856 – 28.11.1912), \emph{Theaterleiter/Theaterleiterin, Regisseur/Regisseurin}|pwk}
                  Homosexualität lesen lässt. Brahm\pwindex{Brahm, Otto 05.02.1856 – 28.11.1912@\textsc{Brahm, Otto} (05.02.1856 – 28.11.1912), \emph{Theaterleiter/Theaterleiterin, Regisseur/Regisseurin}|pwk} agierte
                  auch als Vacanos\pwindex{Vacano, Stefan 1874-12-12 – 1963@\textsc{Vacano, Stefan} (1874-12-12 – 1963), \emph{Schriftsteller/Schriftstellerin}|pwk} Förderer. So gelang etwa
                     Vacanos\pwindex{Vacano, Stefan 1874-12-12 – 1963@\textsc{Vacano, Stefan} (1874-12-12 – 1963), \emph{Schriftsteller/Schriftstellerin}|pwk} Vierakter \emph{Der Tag}\pwindex{Tag@\emph{Der Tag}|pwk} am 19. 1. 1901 am
                     \emph{Deutschen Theater}\orgindex{Deutsches Theater Berlin@Deutsches Theater Berlin|pwk} in Berlin\oindex{Berlin@\textbf{Berlin}, \emph{P.PPLC}|pwk} zur Uraufführung. }}}\label{K_L03055-2} zwiſchen \textsc{Brahm\pwindex{Brahm, Otto 05.02.1856 – 28.11.1912@\textsc{Brahm, Otto} (05.02.1856 – 28.11.1912), \emph{Theaterleiter/Theaterleiterin, Regisseur/Regisseurin}|pw}} und ihm erklären, die nicht blos diejenigen des Theaterdirektors\pwindex{Brahm, Otto 05.02.1856 – 28.11.1912@\textsc{Brahm, Otto} (05.02.1856 – 28.11.1912), \emph{Theaterleiter/Theaterleiterin, Regisseur/Regisseurin}|pwv} zum Autor\pwindex{Vacano, Stefan 1874-12-12 – 1963@\textsc{Vacano, Stefan} (1874-12-12 – 1963), \emph{Schriftsteller/Schriftstellerin}|pwv} ſind. Der {\pb}Dichter\pwindex{Vacano, Stefan 1874-12-12 – 1963@\textsc{Vacano, Stefan} (1874-12-12 – 1963), \emph{Schriftsteller/Schriftstellerin}|pwv} des »Tag\pwindex{Tag@\emph{Der Tag}|pw}« ſieht auch danach aus. \textsc{Brahm\pwindex{Brahm, Otto 05.02.1856 – 28.11.1912@\textsc{Brahm, Otto} (05.02.1856 – 28.11.1912), \emph{Theaterleiter/Theaterleiterin, Regisseur/Regisseurin}|pw}} gleichfalls.\pend
           
\pstart
           Von \textsc{Olga G.\pwindex{Schnitzler, Olga 17.01.1882 – 13.01.1970@\textsc{Schnitzler, Olga} (17.01.1882 – 13.01.1970), \emph{Schauspieler/Schauspielerin, Sänger/Sängerin}|pw}} erhielt ich einen beinahe ſchwermüthigen Brief. Angenehmes Liebesglück! Warum
                  \label{K_L03055-3v}\edtext{quälſt}{\lemma{\textnormal{\emph{quälſt}}}\Cendnote{\textnormal{Das dürfte als Bezugnahme auf Schnitzlers Zögern, sich fix zu binden, zu verstehen
                  sein.}}}\label{K_L03055-3} Du das Mädel\pwindex{Schnitzler, Olga 17.01.1882 – 13.01.1970@\textsc{Schnitzler, Olga} (17.01.1882 – 13.01.1970), \emph{Schauspieler/Schauspielerin, Sänger/Sängerin}|pwv}
               ſo?\pend
           
\pstart
           Es wäre ſchön, wenn Du in den \strikeout{\textcolor{gray}{B}} Anſichts- und Poſtkarten-Verkehr, den Du mit mir unterhältſt, auch einmal
               durch Abſendung eines Briefes eine erfriſchende Abwechſelung brächteſt. Ich wüßte
               beiſpielsweiſe gern, was \textsc{Richard\pwindex{Beer-Hofmann, Richard 1866-07-11 – 1945-09-26@\textsc{Beer-Hofmann, Richard} (1866-07-11 – 1945-09-26), \emph{Schriftsteller/Schriftstellerin}|pw}} macht. Selbſtverſtändlich ſchreibt er mir nicht. Er wird mir niemals ſo lange
               nicht ſchreiben können, als {\pb}ich im Stande ſein
               werde, mich darüber zu empören. In meiner \label{K_L03055-4v}\edtext{Kritik\pwindex{Michael Kramer.«@\emph{»Michael Kramer.«}|pwv}}{\lemma{\textnormal{\emph{Kritik}}}\Cendnote{\textnormal{Paul Goldmann\pwindex{Goldmann, Paul 31.01.1865 – 25.09.1935@\textsc{Goldmann, Paul} (31.01.1865 – 25.09.1935), \emph{Schriftsteller/Schriftstellerin, Journalist/Journalistin}|pwk}: \emph{Feuilleton. »Michael Kramer«}\pwindex{Michael Kramer.«@\emph{»Michael Kramer.«}|pwk}. In: \emph{Neue Freie Presse}\pwindex{Neue Freie Presse@\emph{Neue Freie Presse}|pwk}, Nr. 13.055, 28. 12. 1900, Morgenblatt, S. 1–3.}}}\label{K_L03055-4} über »Michael Kramer\pwindex{Michael Kramer. Drama@\emph{Michael Kramer. Drama}|pw}« ſoll er, wie ich höre, –
               Schadenfreude gefunden haben. Es iſt intereſſant, daß dieſer feinſte \strikeout{\textcolor{gray}{×}\-\textcolor{gray}{×}\-\textcolor{gray}{×}\-\textcolor{gray}{×}}{ }Menſchenkenner\pwindex{Beer-Hofmann, Richard 1866-07-11 – 1945-09-26@\textsc{Beer-Hofmann, Richard} (1866-07-11 – 1945-09-26), \emph{Schriftsteller/Schriftstellerin}|pwv} gerade mich
               weniger kennt, als irgend Jemand, und daß gerade dieſer bewundernswürdig geſcheite
                  Menſch\pwindex{Beer-Hofmann, Richard 1866-07-11 – 1945-09-26@\textsc{Beer-Hofmann, Richard} (1866-07-11 – 1945-09-26), \emph{Schriftsteller/Schriftstellerin}|pwv} ſo dumm über mich
               urtheilt. Ich werde für ihn einen Commentar über mich ſchreiben. Bitte ſag’ ihm das,
               – und daß ich ihn ſehr vermiſſe und daß ich viel darum gäbe, könnte ich ihn immer in
               meiner Nähe haben.\pend
           
\pstart
           {\pb}Ich bin vollſtändig ohne Verkehr, – vollſtändig
               einſam. \textsc{Kerr\pwindex{Kerr, Alfred 25.12.1867 – 12.10.1948@\textsc{Kerr, Alfred} (25.12.1867 – 12.10.1948), \emph{Schriftsteller/Schriftstellerin, Kritiker/Kritikerin}|pw}} benimmt ſich blödſinnig. Seit Du aus Berlin\oindex{Berlin@\textbf{Berlin}, \emph{P.PPLC}|pw}
               fort biſt, habe ich ihn nicht mehr geſprochen. Wenn er mich im Theater ſieht, drückt
               er mir raſch die Hand und läuft weg{\dotssix}\pend
           
\pstart
           Schreib’ mir bald!\pend
           
\pstart
           Viele treue Grüße! {\\[\baselineskip]}Dein {\\[\baselineskip]}\spacefill\mbox{Paul Goldmann.}\pend
           \leftskip=0em{}\selectlanguage{ngerman}\endnumbering\briefempfaengerindex{Schnitzler, Arthur@\textsc{Schnitzler, Arthur}!zzzGoldmann, Paul@\emph{von Paul Goldmann}!1901-01-221@{22. 1. {[}1901{]}}|)be}\mylabel{L03055h}  \normalsize

\doendnotes{C}
\bigskip
\vfill

\clearpage

\footnotesize

\lohead{\textsc{register}}

% Definiere theindex-Environment komplett neu ohne reledmac
\makeatletter
\renewenvironment{theindex}{%
  \section*{\indexname}%
  \setlength{\parindent}{0pt}%
  \setlength{\parskip}{0pt plus 0.3pt}%
  \let\item\@idxitem
}{%
  \clearpage
}
\makeatother

\IfFileExists{\jobname-pw.ind}{\input{\jobname-pw.ind}}{}

\end{document}

      