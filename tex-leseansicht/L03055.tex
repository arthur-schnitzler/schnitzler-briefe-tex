%% latex-leseansicht-vorspann.tex
%% Vorspann für die Leseansicht.
%% Lädt die gemeinsame Datei latex-vorspann.tex mit nicht gesetztem Schalter.

\newif\ifkorrekturansicht
\korrekturansichtfalse

\input{../tex-inputs/latex-vorspann}


\section[ Paul Goldmann an Arthur Schnitzler, 22. 1. {[}1901{]}]{L03055 Paul Goldmann an Arthur Schnitzler,  22. 1. [1901]}
\nopagebreak\mylabel{L03055v}
\rehead{ }\normalsize\beginnumbering\briefempfaengerindex{Schnitzler, Arthur@\textsc{Schnitzler, Arthur}!zzzGoldmann, Paul@\emph{von Paul Goldmann}!1901-01-221@{22. 1. [1901]}|(be}
\toendnotes[C]{\smallbreak\pagebreak[2]}
\correspDesc{Versand  durch Paul Goldmann am 22. 1. [1901] in Berlin
\newline{}Erhalt  durch Arthur Schnitzler im Zeitraum [23. 1. 1901
                  – 27. 1. 1901?] in Wien}\toendnotes[C]{\smallbreak}
\Standort{DLA, A:Schnitzler, HS.NZ85.1.3171.}
\physDesc{Brief, 1 Blatt, 4 Seiten, 1730 Zeichen
\newline{}Handschrift: blaue Tinte, deutsche Kurrent
\newline{}Schnitzler: mit rotem Buntstift sechs Unterstreichungen }\toendnotes[C]{\smallbreak}
\pstart
           \raggedleft{}{\pb}\textcolor{gray}{\textbf{DESSAUERSTRASSE 19}}\oindex{Dessauer Straße@\textbf{Dessauer Straße}, \emph{Straße}|pw}\pend
           
\pstart
           Berlin\oindex{Berlin@\textbf{Berlin}, \emph{Hauptstadt}|pw}, 22. Januar.\pend
           
\pstart\center{}Mein lieber Freund,\pend\vspace{0.5em}
\pstart
           \textsc{Mizzi Glümer\pwindex{Glümer, Marie 3.\,7.\,1867 Wien – 16.\,11.\,1925 München@\textsc{Glümer, Marie} (3.\,7.\,1867 Wien – 16.\,11.\,1925 München), \emph{Schauspielerin}|pw}} iſt krank, liegt zu Bett und{ }ſieht{ }ſo elend aus, daß ich erſchrocken bin (Unter
                  uns!)\textcolor{gray}{.} Du{ }ſollteſt dem armen Mädel\pwindex{Glümer, Marie 3.\,7.\,1867 Wien – 16.\,11.\,1925 München@\textsc{Glümer, Marie} (3.\,7.\,1867 Wien – 16.\,11.\,1925 München), \emph{Schauspielerin}|pwv} einen guten Brief{ }ſchreiben.\pend
           
\pstart
           \textsc{Brahm\pwindex{Brahm, Otto 5.\,2.\,1856 Hamburg – 28.\,11.\,1912 Berlin@\textsc{Brahm, Otto} (5.\,2.\,1856 Hamburg – 28.\,11.\,1912 Berlin), \emph{Theaterleiter, Regisseur}|pw}}{ }ſagte mir bei einer der letzten \begin{otherlanguage}{french}\textsc{Premièren}\end{otherlanguage}, er möchte von Dir einen oder zwei \label{K_L03055-1v}\edtext{Einakter}{\lemma{\textnormal{\emph{Einakter}}}\Cendnote{\textnormal{Am 12. 2. 1901 hatte sich Brahm\pwindex{Brahm, Otto 5.\,2.\,1856 Hamburg – 28.\,11.\,1912 Berlin@\textsc{Brahm, Otto} (5.\,2.\,1856 Hamburg – 28.\,11.\,1912 Berlin), \emph{Theaterleiter, Regisseur}|pwk} bei Schnitzler
                  für den Erhalt von \emph{Zum großen Wurstel}\pwindex{Schnitzler, Arthur 15.\,5.\,1862 Wien – 21.\,10.\,1931 ebd.@\textsc{Schnitzler, Arthur} (15.\,5.\,1862 Wien – 21.\,10.\,1931 ebd.), \emph{Schriftsteller, Mediziner}!Zum großen Wurstel. Burleske in einem Akt@\strich\emph{Zum großen Wurstel. Burleske in einem Akt}|pwk} (noch
                  unter dem Entstehungstitel »Marionetten\pwindex{Schnitzler, Arthur 15.\,5.\,1862 Wien – 21.\,10.\,1931 ebd.@\textsc{Schnitzler, Arthur} (15.\,5.\,1862 Wien – 21.\,10.\,1931 ebd.), \emph{Schriftsteller, Mediziner}!Zum großen Wurstel. Burleske in einem Akt@\strich\emph{Zum großen Wurstel. Burleske in einem Akt}|pwkv}«) bedankt, vgl. \emph{Der Briefwechsel
                        Arthur Schnitzler – Otto Brahm}. Vollständige Ausgabe. Herausgegeben,
                     eingeleitet und erläutert von Oskar Seidlin. Tübingen:
                        \emph{Niemeyer}{ }1975, S. 88.}}}\label{K_L03055-1} haben. Wer iſt \textsc{Stefan Vacano\pwindex{Vacano, Stefan 12.\,12.\,1874 Wien – 1963@\textsc{Vacano, Stefan} (12.\,12.\,1874 Wien – 1963), \emph{Schriftsteller}|pw}}? Ich kann mir die Aufführung{ }ſeines Stück\pwindex{Vacano, Stefan 12.\,12.\,1874 Wien – 1963@\textsc{Vacano, Stefan} (12.\,12.\,1874 Wien – 1963), \emph{Schriftsteller}!Tag@\strich\emph{Der Tag}|pwv}es nur durch \label{K_L03055-2v}\edtext{Beziehungen}{\lemma{\textnormal{\emph{Beziehungen}}}\Cendnote{\textnormal{Der
                     Wien\oindex{Wien@\textbf{Wien}, \emph{Verwaltungsgebiet}|pwk}er Stefan Vacano\pwindex{Vacano, Stefan 12.\,12.\,1874 Wien – 1963@\textsc{Vacano, Stefan} (12.\,12.\,1874 Wien – 1963), \emph{Schriftsteller}|pwk} war Theaterdichter und als Regieassistent bei Otto Brahm\pwindex{Brahm, Otto 5.\,2.\,1856 Hamburg – 28.\,11.\,1912 Berlin@\textsc{Brahm, Otto} (5.\,2.\,1856 Hamburg – 28.\,11.\,1912 Berlin), \emph{Theaterleiter, Regisseur}|pwk} tätig. Möglicherweise war die
                  Freundschaft zwischen den beiden Männer\pwindex{Vacano, Stefan 12.\,12.\,1874 Wien – 1963@\textsc{Vacano, Stefan} (12.\,12.\,1874 Wien – 1963), \emph{Schriftsteller}|pwk}\pwindex{Brahm, Otto 5.\,2.\,1856 Hamburg – 28.\,11.\,1912 Berlin@\textsc{Brahm, Otto} (5.\,2.\,1856 Hamburg – 28.\,11.\,1912 Berlin), \emph{Theaterleiter, Regisseur}|pwk}n auch persönlicher Natur, da sich diese Stelle durchaus auch als
                  leicht verklausulierter Hinweis auf Brahms\pwindex{Brahm, Otto 5.\,2.\,1856 Hamburg – 28.\,11.\,1912 Berlin@\textsc{Brahm, Otto} (5.\,2.\,1856 Hamburg – 28.\,11.\,1912 Berlin), \emph{Theaterleiter, Regisseur}|pwk}
                  Homosexualität lesen lässt. Brahm\pwindex{Brahm, Otto 5.\,2.\,1856 Hamburg – 28.\,11.\,1912 Berlin@\textsc{Brahm, Otto} (5.\,2.\,1856 Hamburg – 28.\,11.\,1912 Berlin), \emph{Theaterleiter, Regisseur}|pwk} agierte
                  auch als Vacanos\pwindex{Vacano, Stefan 12.\,12.\,1874 Wien – 1963@\textsc{Vacano, Stefan} (12.\,12.\,1874 Wien – 1963), \emph{Schriftsteller}|pwk} Förderer. So gelang etwa
                     Vacanos\pwindex{Vacano, Stefan 12.\,12.\,1874 Wien – 1963@\textsc{Vacano, Stefan} (12.\,12.\,1874 Wien – 1963), \emph{Schriftsteller}|pwk} Vierakter \emph{Der Tag}\pwindex{Vacano, Stefan 12.\,12.\,1874 Wien – 1963@\textsc{Vacano, Stefan} (12.\,12.\,1874 Wien – 1963), \emph{Schriftsteller}!Tag@\strich\emph{Der Tag}|pwk} am 19. 1. 1901 am
                  \emph{Deutschen Theater}\orgindex{Deutsches Theater Berlin@Deutsches Theater Berlin|pwk} in Berlin\oindex{Berlin@\textbf{Berlin}, \emph{Hauptstadt}|pwk} zur Uraufführung\eventindex{Deutsches Theater Berlin@\textbf{Deutsches Theater Berlin}!Uraufführung von Der Tag, 19.1.1901@Uraufführung von Der Tag, 19.1.1901|pwkv}. }}}\label{K_L03055-2} zwiſchen \textsc{Brahm\pwindex{Brahm, Otto 5.\,2.\,1856 Hamburg – 28.\,11.\,1912 Berlin@\textsc{Brahm, Otto} (5.\,2.\,1856 Hamburg – 28.\,11.\,1912 Berlin), \emph{Theaterleiter, Regisseur}|pw}} und ihm erklären, die nicht blos diejenigen des Theaterdirektors\pwindex{Brahm, Otto 5.\,2.\,1856 Hamburg – 28.\,11.\,1912 Berlin@\textsc{Brahm, Otto} (5.\,2.\,1856 Hamburg – 28.\,11.\,1912 Berlin), \emph{Theaterleiter, Regisseur}|pwv} zum Autor\pwindex{Vacano, Stefan 12.\,12.\,1874 Wien – 1963@\textsc{Vacano, Stefan} (12.\,12.\,1874 Wien – 1963), \emph{Schriftsteller}|pwv}{ }ſind. Der {\pb}Dichter\pwindex{Vacano, Stefan 12.\,12.\,1874 Wien – 1963@\textsc{Vacano, Stefan} (12.\,12.\,1874 Wien – 1963), \emph{Schriftsteller}|pwv} des »Tag\pwindex{Vacano, Stefan 12.\,12.\,1874 Wien – 1963@\textsc{Vacano, Stefan} (12.\,12.\,1874 Wien – 1963), \emph{Schriftsteller}!Tag@\strich\emph{Der Tag}|pw}«{ }ſieht auch danach aus. \textsc{Brahm\pwindex{Brahm, Otto 5.\,2.\,1856 Hamburg – 28.\,11.\,1912 Berlin@\textsc{Brahm, Otto} (5.\,2.\,1856 Hamburg – 28.\,11.\,1912 Berlin), \emph{Theaterleiter, Regisseur}|pw}} gleichfalls.\pend
           
\pstart
           Von \textsc{Olga G.\pwindex{Schnitzler, Olga 17.\,1.\,1882 Wien – 13.\,1.\,1970 Lugano@\textsc{Schnitzler, Olga} (17.\,1.\,1882 Wien – 13.\,1.\,1970 Lugano), \emph{Schauspielerin, Sängerin}|pw}} erhielt ich einen beinahe{ }ſchwermüthigen Brief. Angenehmes Liebesglück! Warum
                  \label{K_L03055-3v}\edtext{quälſt}{\lemma{\textnormal{\emph{quälst}}}\Cendnote{\textnormal{Das dürfte als Bezugnahme auf Schnitzlers Zögern, sich fix zu binden, zu verstehen
                  sein.}}}\label{K_L03055-3} Du das Mädel\pwindex{Schnitzler, Olga 17.\,1.\,1882 Wien – 13.\,1.\,1970 Lugano@\textsc{Schnitzler, Olga} (17.\,1.\,1882 Wien – 13.\,1.\,1970 Lugano), \emph{Schauspielerin, Sängerin}|pwv}{ }ſo?\pend
           
\pstart
           Es wäre{ }ſchön, wenn Du in den \strikeout{\textcolor{gray}{B}} Anſichts- und Poſtkarten-Verkehr, den Du mit mir unterhältſt, auch einmal
               durch Abſendung eines Briefes eine erfriſchende Abwechſelung brächteſt. Ich wüßte
               beiſpielsweiſe gern, was \textsc{Richard\pwindex{Beer-Hofmann, Richard 11.\,7.\,1866 Wien – 26.\,9.\,1945 New York City@\textsc{Beer-Hofmann, Richard} (11.\,7.\,1866 Wien – 26.\,9.\,1945 New York City), \emph{Schriftsteller}|pw}} macht. Selbſtverſtändlich{ }ſchreibt er mir nicht. Er wird mir niemals{ }ſo lange
               nicht{ }ſchreiben können, als {\pb}ich im Stande{ }ſein
               werde, mich darüber zu empören. In meiner \label{K_L03055-4v}\edtext{Kritik\pwindex{Goldmann, Paul 31.\,1.\,1865 Breslau – 25.\,9.\,1935 Wien@\textsc{Goldmann, Paul} (31.\,1.\,1865 Breslau – 25.\,9.\,1935 Wien), \emph{Schriftsteller, Journalist}!Michael Kramer.«@\strich\emph{»Michael Kramer.«}|pwv}}{\lemma{\textnormal{\emph{Kritik}}}\Cendnote{\textnormal{Paul Goldmann\pwindex{Goldmann, Paul 31.\,1.\,1865 Breslau – 25.\,9.\,1935 Wien@\textsc{Goldmann, Paul} (31.\,1.\,1865 Breslau – 25.\,9.\,1935 Wien), \emph{Schriftsteller, Journalist}|pwk}: \emph{Feuilleton. »Michael Kramer«}\pwindex{Goldmann, Paul 31.\,1.\,1865 Breslau – 25.\,9.\,1935 Wien@\textsc{Goldmann, Paul} (31.\,1.\,1865 Breslau – 25.\,9.\,1935 Wien), \emph{Schriftsteller, Journalist}!Michael Kramer.«@\strich\emph{»Michael Kramer.«}|pwk}. In: \emph{Neue Freie Presse}\pwindex{Neue Freie Presse@\emph{Neue Freie Presse}|pwk}, Nr. 13.055, 28. 12. 1900, Morgenblatt, S. 1–3.}}}\label{K_L03055-4} über »Michael Kramer\pwindex{\textcolor{red}{\textsuperscript{XXXX indx1}}!Michael Kramer. Drama@\strich\emph{Michael Kramer. Drama}|pw}«{ }ſoll er, wie ich höre, –
               Schadenfreude gefunden haben. Es iſt intereſſant, daß dieſer feinſte \strikeout{\textcolor{gray}{×}\-\textcolor{gray}{×}\-\textcolor{gray}{×}\-\textcolor{gray}{×}}{ }Menſchenkenner\pwindex{Beer-Hofmann, Richard 11.\,7.\,1866 Wien – 26.\,9.\,1945 New York City@\textsc{Beer-Hofmann, Richard} (11.\,7.\,1866 Wien – 26.\,9.\,1945 New York City), \emph{Schriftsteller}|pwv} gerade mich
               weniger kennt, als irgend Jemand, und daß gerade dieſer bewundernswürdig geſcheite
                  Menſch\pwindex{Beer-Hofmann, Richard 11.\,7.\,1866 Wien – 26.\,9.\,1945 New York City@\textsc{Beer-Hofmann, Richard} (11.\,7.\,1866 Wien – 26.\,9.\,1945 New York City), \emph{Schriftsteller}|pwv}{ }ſo dumm über mich
               urtheilt. Ich werde für ihn einen Commentar über mich{ }ſchreiben. Bitte{ }ſag’ ihm das,
               – und daß ich ihn{ }ſehr vermiſſe und daß ich viel darum gäbe, könnte ich ihn immer in
               meiner Nähe haben.\pend
           
\pstart
           {\pb}Ich bin vollſtändig ohne Verkehr, – vollſtändig
               einſam. \textsc{Kerr\pwindex{Kerr, Alfred 25.\,12.\,1867 Breslau – 12.\,10.\,1948 Hamburg@\textsc{Kerr, Alfred} (25.\,12.\,1867 Breslau – 12.\,10.\,1948 Hamburg), \emph{Schriftsteller, Kritiker}|pw}} benimmt{ }ſich blödſinnig. Seit Du aus Berlin\oindex{Berlin@\textbf{Berlin}, \emph{Hauptstadt}|pw}
               fort biſt, habe ich ihn nicht mehr geſprochen. Wenn er mich im Theater{ }ſieht, drückt
               er mir raſch die Hand und läuft weg{\dotssix}\pend
           
\pstart
           Schreib’ mir bald!\pend
           
\pstart
           Viele treue Grüße! {\\[\baselineskip]}Dein {\\[\baselineskip]}\spacefill\mbox{Paul Goldmann.}\pend
           \leftskip=0em{}\selectlanguage{ngerman}\endnumbering\briefempfaengerindex{Schnitzler, Arthur@\textsc{Schnitzler, Arthur}!zzzGoldmann, Paul@\emph{von Paul Goldmann}!1901-01-221@{22. 1. [1901]}|)be}\mylabel{L03055h}  \newcommand{\dateiname}{L03055}\newcommand{\titel}{Paul Goldmann an Arthur Schnitzler, 22. 1. [1901]}\newcommand{\editorInnen}{Martin Anton Müller und Laura Untner}%% latex-leseansicht-abspann.tex
%% Abspann für die Leseansicht.
%% Der Schalter \ifkorrekturansicht ist bereits durch den Vorspann gesetzt.

%% latex-abspann.tex
%% Gemeinsamer Abspann für Korrekturansicht und Leseansicht.
%% Setzt den Schalter \ifkorrekturansicht voraus (gesetzt in den
%% einbindenden Dateien latex-korrekturansicht-abspann.tex bzw.
%% latex-leseansicht-abspann.tex).
%% ---------------------------------------------------------------

\normalsize

% Das esempio-Environment wird nur in der Leseansicht benötigt
\ifkorrekturansicht\else
\newenvironment{esempio}[3]%
{
    \vspace{1.5ex}
    \rlap{\underline{#1}}
    \par
    \setlength{\parindent}{0cm}
    \nopagebreak
    \leftskip=#2cm
    \rightskip=#3cm
}
{
    \par
}
\fi

\doendnotes{C}
\bigskip
\vfill

\clearpage

\footnotesize

\ifkorrekturansicht
  \lohead{\textsc{register}}
\fi

% theindex-Environment neu definieren ohne reledmac
\makeatletter
\renewenvironment{theindex}{%
  \ifkorrekturansicht
    \section*{\indexname}%
  \else
    \subsubsection*{Index der erwähnten Entitäten}%
  \fi
  \setlength{\parindent}{0pt}%
  \setlength{\parskip}{0pt plus 0.3pt}%
  \let\item\@idxitem
}{%
  \ifkorrekturansicht\clearpage\fi
}
\makeatother

\IfFileExists{\jobname-pw.ind}{\input{\jobname-pw.ind}}{}

% Quellenangabe nur in der Leseansicht
\ifkorrekturansicht\else
% Fallback-Definitionen, falls die .tex-Datei \titel etc. nicht gesetzt hat
\providecommand{\titel}{}
\providecommand{\editorInnen}{}
\providecommand{\dateiname}{\jobname}

\vspace{3cm}

\vfill

\footnotesize
\textsc{Quelle}: \titel. Herausgegeben von {\editorInnen}. In: \emph{Arthur Schnitzler: Briefwechsel mit Autorinnen und Autoren}.
 Digitale Edition, https://schnitzler-briefe.acdh.oeaw.ac.at/{\dateiname}.html (Stand \today)
\fi

\end{document}


