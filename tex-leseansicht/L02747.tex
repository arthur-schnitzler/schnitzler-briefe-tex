%% latex-leseansicht-vorspann.tex
%% Vorspann für die Leseansicht.
%% Lädt die gemeinsame Datei latex-vorspann.tex mit nicht gesetztem Schalter.

\newif\ifkorrekturansicht
\korrekturansichtfalse

\input{../tex-inputs/latex-vorspann}


\section[Paul Goldmann an Arthur Schnitzler, 12. 9. [1895]]{L02747 Paul Goldmann an Arthur Schnitzler, 12. 9. [1895]}
\nopagebreak\mylabel{L02747v}
\rehead{ }\normalsize\beginnumbering\briefempfaengerindex{Schnitzler, Arthur@\textsc{Schnitzler, Arthur}!zzzGoldmann, Paul@\emph{von Paul Goldmann}!1895-09-122@{12. 9. [1895]}|(be}
\toendnotes[C]{\smallbreak\pagebreak[2]}
\correspDesc{Versand  durch Paul Goldmann am 12. 9. [1895] in Paris
\newline{}Erhalt  durch Arthur Schnitzler im Zeitraum [13. 9. 1895
                  – 17. 9. 1895?] in Wien}\toendnotes[C]{\smallbreak}
\Standort{DLA, A:Schnitzler, HS.NZ85.1.3165.}
\physDesc{Brief, 1 Blatt, 2 Seiten, 865 Zeichen
\newline{}Handschrift: blaue Tinte, deutsche Kurrent
\newline{}Schnitzler: mit Bleistift das Jahr »95« vermerkt }\toendnotes[C]{\smallbreak}
\pstart
           {\pb}\textcolor{gray}{\textbf{\textbf{Frankfurter Zeitung\orgindex{Frankfurter Zeitung@Frankfurter Zeitung|pw}}}}\pend
           
\pstart
           \textcolor{gray}{\textbf{(\begin{otherlanguage}{french}Gazette de Francfort\end{otherlanguage}\orgindex{Frankfurter Zeitung@Frankfurter Zeitung|pw}).}}\pend
           
\pstart
           \textcolor{gray}{\textbf{\textbf{\begin{otherlanguage}{french}Fondateur M. L.
                                 Sonnemann\pwindex{Sonnemann, Leopold 29.\,10.\,1831 Höchberg – 30.\,10.\,1909 Frankfurt am Main@\textsc{Sonnemann, Leopold} (29.\,10.\,1831 Höchberg – 30.\,10.\,1909 Frankfurt am Main), \emph{Journalist, Herausgeber}|pw}\end{otherlanguage}.}}}\hfill \textsc{Paris\oindex{Paris@\textbf{Paris}, \emph{Hauptstadt}|pw}}, 12. September.\pend
           
\pstart
           \begin{otherlanguage}{french}\textcolor{gray}{\textbf{Journal politique, financier,}}\end{otherlanguage}\pend
           
\pstart
           \begin{otherlanguage}{french}\textcolor{gray}{\textbf{commercial et littéraire.}}\end{otherlanguage}\pend
           
\pstart
           \begin{otherlanguage}{french}\textcolor{gray}{\textbf{\textbf{Paraissant trois fois par jour.}}}\end{otherlanguage}\pend
           
\pstart
           \begin{otherlanguage}{french}\textcolor{gray}{\textbf{\textbf{Bureau à Paris\oindex{Paris@\textbf{Paris}, \emph{Hauptstadt}|pw}}}}\end{otherlanguage}\pend
           
\pstart
           \begin{otherlanguage}{french}\textcolor{gray}{\textbf{\textbf{24. Rue Feydeau\oindex{rue Feydeau@\textbf{rue Feydeau}, \emph{Straße}|pw}.}}}\end{otherlanguage}\pend
           
\pstart\center{}Mein lieber Freund,\pend\vspace{0.5em}
\pstart
           Seit geſtern bin ich wieder in \textsc{Paris\oindex{Paris@\textbf{Paris}, \emph{Hauptstadt}|pw}}, und all’ das Schöne der letzten Wochen iſt nicht mehr wahr. Es waren köſtliche
               Stunden mit Euch zuſammen, und mein Herz iſt noch warm \strikeout{\textcolor{gray}{×}} von all dem Lieben, das Ihr mir gegeben. Tauſend Dank dafür!\pend
           
\pstart
           Hier will es gar nicht recht gehen. \strikeout{\textcolor{gray}{×}\-\textcolor{gray}{×}\-\textcolor{gray}{×}} Körper und Seele wollen nicht mehr in das bisherige Leben hinein, und ich muß
               alle Kraft zuſammennehmen, um mich zu überwinden.\pend
           
\pstart
           {\pb}Bitte,{ }ſchreib’ mir gleich, wie es mit dem Burgtheater\orgindex{Burgtheater@Burgtheater|pw}{ }ſteht. Die letzte \label{K_L02747-1v}\edtext{Correſpondenz von \textsc{Uhl\pwindex{Uhl, Friedrich 14.\,5.\,1825 Cieszyn – 20.\,1.\,1906 Mondsee@\textsc{Uhl, Friedrich} (14.\,5.\,1825 Cieszyn – 20.\,1.\,1906 Mondsee), \emph{Journalist}|pw}}\pwindex{Wiener Brief [Die neue Saison im Burgtheater]@\emph{Wiener Brief [Die neue Saison im Burgtheater]}|pwv}}{\lemma{\textnormal{\emph{Correspondenz von Uhl}}}\Cendnote{\textnormal{[Friedrich Uhl\pwindex{Uhl, Friedrich 14.\,5.\,1825 Cieszyn – 20.\,1.\,1906 Mondsee@\textsc{Uhl, Friedrich} (14.\,5.\,1825 Cieszyn – 20.\,1.\,1906 Mondsee), \emph{Journalist}|pwk}]: \emph{Wiener Brief}\pwindex{Wiener Brief [Die neue Saison im Burgtheater]@\emph{Wiener Brief [Die neue Saison im Burgtheater]}|pwk}. In: \emph{Frankfurter Zeitung}\pwindex{Frankfurter Zeitung@\emph{Frankfurter Zeitung}|pwk}, Jg. 40, Nr. 249, 8. 9. 1895, Zweites
                     Morgenblatt, S. 1; vgl. XXXX Auszeichnungsfehler: Dokument L02748 nicht gefunden.}}}\label{K_L02747-1} bei uns dürfte wohl günſtigen Einfluß haben durch die Drohung, Rechenſchaft
               am Ende des Jahres zu fordern.\pend
           
\pstart
           \textsc{Wolff}\pwindex{Wolff, Theodor 2.\,8.\,1868 Berlin – 23.\,9.\,1943 ebd.@\textsc{Wolff, Theodor} (2.\,8.\,1868 Berlin – 23.\,9.\,1943 ebd.), \emph{Schriftsteller, Journalist}|pw} (vom »Berliner Tageblatt\orgindex{Berliner Tageblatt@Berliner Tageblatt|pw}«) erzählte mir, er
               habe in Berlin\oindex{Berlin@\textbf{Berlin}, \emph{Hauptstadt}|pw} jetzt gehört, daß Dein Stück\pwindex{Schnitzler, Arthur 15.\,5.\,1862 Wien – 21.\,10.\,1931 ebd.@\textsc{Schnitzler, Arthur} (15.\,5.\,1862 Wien – 21.\,10.\,1931 ebd.), \emph{Schriftsteller, Mediziner}!Liebelei. Schauspiel in drei Akten@\strich\emph{Liebelei. Schauspiel in drei Akten}|pwv} unter den erſten \strikeout{d\textcolor{gray}{r}} darankommen{ }ſolle.\pend
           
\pstart
           Und{ }ſchreibe mir, wie es Dir{ }ſonſt geht?\pend
           
\pstart
           Viele treue Grüße! {\\[\baselineskip]}Dein {\\[\baselineskip]}\spacefill\mbox{Paul Goldmann}\pend
           \leftskip=0em{}
\pstart
           \noindent{}\textsc{Frischauer\pwindex{Frischauer, Berthold 9.\,9.\,1851 Brünn – 4.\,2.\,1924 Wien@\textsc{Frischauer, Berthold} (9.\,9.\,1851 Brünn – 4.\,2.\,1924 Wien), \emph{Journalist}|pw}} kommt wirklich an \textsc{Herzls\pwindex{Herzl, Theodor 2.\,5.\,1860 Budapest – 3.\,7.\,1904 Edlach@\textsc{Herzl, Theodor} (2.\,5.\,1860 Budapest – 3.\,7.\,1904 Edlach), \emph{Schriftsteller, Journalist}|pw}}{ }\label{K_L02747-2v}\edtext{Stelle}{\lemma{\textnormal{\emph{Stelle}}}\Cendnote{\textnormal{als Paris\oindex{Paris@\textbf{Paris}, \emph{Hauptstadt}|pwk}er Korrespondent\pwindex{Frischauer, Berthold 9.\,9.\,1851 Brünn – 4.\,2.\,1924 Wien@\textsc{Frischauer, Berthold} (9.\,9.\,1851 Brünn – 4.\,2.\,1924 Wien), \emph{Journalist}|pwkv} der \emph{Neuen Freien Presse}\orgindex{Neue Freie Presse@Neue Freie Presse|pwk}}}}\label{K_L02747-2}.\pend
           \selectlanguage{ngerman}\endnumbering\briefempfaengerindex{Schnitzler, Arthur@\textsc{Schnitzler, Arthur}!zzzGoldmann, Paul@\emph{von Paul Goldmann}!1895-09-122@{12. 9. [1895]}|)be}\mylabel{L02747h}  \newcommand{\dateiname}{L02747}\newcommand{\titel}{Paul Goldmann an Arthur Schnitzler, 12. 9. [1895]}\newcommand{\editorInnen}{Martin Anton Müller und Laura Untner}%% latex-leseansicht-abspann.tex
%% Abspann für die Leseansicht.
%% Der Schalter \ifkorrekturansicht ist bereits durch den Vorspann gesetzt.

%% latex-abspann.tex
%% Gemeinsamer Abspann für Korrekturansicht und Leseansicht.
%% Setzt den Schalter \ifkorrekturansicht voraus (gesetzt in den
%% einbindenden Dateien latex-korrekturansicht-abspann.tex bzw.
%% latex-leseansicht-abspann.tex).
%% ---------------------------------------------------------------

\normalsize

% Das esempio-Environment wird nur in der Leseansicht benötigt
\ifkorrekturansicht\else
\newenvironment{esempio}[3]%
{
    \vspace{1.5ex}
    \rlap{\underline{#1}}
    \par
    \setlength{\parindent}{0cm}
    \nopagebreak
    \leftskip=#2cm
    \rightskip=#3cm
}
{
    \par
}
\fi

\doendnotes{C}
\bigskip
\vfill

\clearpage

\footnotesize

\ifkorrekturansicht
  \lohead{\textsc{register}}
\fi

% theindex-Environment neu definieren ohne reledmac
\makeatletter
\renewenvironment{theindex}{%
  \ifkorrekturansicht
    \section*{\indexname}%
  \else
    \subsubsection*{Index der erwähnten Entitäten}%
  \fi
  \setlength{\parindent}{0pt}%
  \setlength{\parskip}{0pt plus 0.3pt}%
  \let\item\@idxitem
}{%
  \ifkorrekturansicht\clearpage\fi
}
\makeatother

\IfFileExists{\jobname-pw.ind}{\input{\jobname-pw.ind}}{}

% Quellenangabe nur in der Leseansicht
\ifkorrekturansicht\else
% Fallback-Definitionen, falls die .tex-Datei \titel etc. nicht gesetzt hat
\providecommand{\titel}{}
\providecommand{\editorInnen}{}
\providecommand{\dateiname}{\jobname}

\vspace{3cm}

\vfill

\footnotesize
\textsc{Quelle}: \titel. Herausgegeben von {\editorInnen}. In: \emph{Arthur Schnitzler: Briefwechsel mit Autorinnen und Autoren}.
 Digitale Edition, https://schnitzler-briefe.acdh.oeaw.ac.at/{\dateiname}.html (Stand \today)
\fi

\end{document}


