%% latex-leseansicht-vorspann.tex
%% Vorspann für die Leseansicht.
%% Lädt die gemeinsame Datei latex-vorspann.tex mit nicht gesetztem Schalter.

\newif\ifkorrekturansicht
\korrekturansichtfalse

\input{../tex-inputs/latex-vorspann}


         
         \newcommand{\erwaehntePersonen}{Personen: Berthold Frischauer, Theodor Herzl, Leopold Sonnemann, Friedrich Uhl, Theodor Wolff}
         \newcommand{\erwaehnteInstitutionen}{Institutionen: Berliner Tageblatt, Burgtheater, Frankfurter Zeitung, Neue Freie Presse}
         \newcommand{\erwaehnteOrte}{Orte: Berlin, Paris, Wien, rue Feydeau}
         \newcommand{\erwaehnteWerke}{Werke: Frankfurter Zeitung, Liebelei. Schauspiel in drei Akten, Wiener Brief [Die neue Saison im Burgtheater]}
               \section[Paul Goldmann an Arthur Schnitzler, 12. 9. {[}1895{]}]{ Paul Goldmann an Arthur Schnitzler, 12. 9. {[}1895{]}}\nopagebreak\mylabel{v}\rehead{ }\begin{ledgroupsized}[t]{13cm}\normalsize\beginnumbering \toendnotes[C]{\smallbreak\pagebreak[2]} \Standort{DLA, A:Schnitzler, HS.NZ85.1.3165.}
\physDesc{Brief, 1 Blatt, 2 Seiten
\newline{}Handschrift: blaue Tinte, deutsche Kurrent
\newline{}Schnitzler: mit Bleistift das Jahr »95« vermerkt }\toendnotes[C]{\smallbreak}\pstart
           \noindent{}{\pb}\textcolor{gray}{\textbf{\textbf{Frankfurter Zeitung\orgindex{Frankfurter Zeitung@Frankfurter Zeitung|pw}}}}\pend
           \pstart
           \textcolor{gray}{\textbf{(\begin{otherlanguage}{french}Gazette de Francfort\end{otherlanguage}\orgindex{Frankfurter Zeitung@Frankfurter Zeitung|pw}). }}\pend
           \pstart
           \textcolor{gray}{\textbf{\textbf{\begin{otherlanguage}{french}Fondateur M. L.
                                 Sonnemann\pwindex{Sonnemann, Leopold 1831-10-29 – 1909-10-30@\textsc{Sonnemann, Leopold} (1831-10-29 – 1909-10-30), \emph{Journalist, Herausgeber}|pw}\end{otherlanguage}.}}}\hfill \textsc{Paris\oindex{Paris@\textbf{Paris}|pw}}, 12. September.\pend
           \pstart
           \begin{otherlanguage}{french}\textcolor{gray}{\textbf{Journal politique, financier,}}\end{otherlanguage}\pend
           \pstart
           \begin{otherlanguage}{french}\textcolor{gray}{\textbf{commercial et littéraire.}}\end{otherlanguage}\pend
           \pstart
           \begin{otherlanguage}{french}\textcolor{gray}{\textbf{\textbf{Paraissant trois fois par jour.}}}\end{otherlanguage}\pend
           \pstart
           \begin{otherlanguage}{french}\textcolor{gray}{\textbf{\textbf{Bureau à Paris\oindex{Paris@\textbf{Paris}|pw}}}}\end{otherlanguage}\pend
           \pstart
           \begin{otherlanguage}{french}\textcolor{gray}{\textbf{\textbf{24. Rue Feydeau\oindex{rue Feydeau@\textbf{rue Feydeau}|pw}.}}}\end{otherlanguage}\pend
           \pstart\center{}Mein lieber Freund,\pend\pstart
           Seit geſtern bin ich wieder in \textsc{Paris\oindex{Paris@\textbf{Paris}|pw}}, und all’ das Schöne der letzten Wochen iſt nicht mehr wahr. Es waren köſtliche
               Stunden mit Euch zuſammen, und mein Herz iſt noch warm \strikeout{\textcolor{gray}{×}} von all dem Lieben, das Ihr mir gegeben. Tauſend Dank dafür!\pend
           \pstart
           Hier will es gar nicht recht gehen. \strikeout{\textcolor{gray}{×}\-\textcolor{gray}{×}\-\textcolor{gray}{×}} Körper und Seele wollen nicht mehr in das bisherige Leben hinein, und ich muß
               alle Kraft zuſammennehmen, um mich zu überwinden.\pend
           \pstart
           {\pb}Bitte, ſchreib’ mir gleich, wie es mit dem Burgtheater\orgindex{Burgtheater@Burgtheater|pw} ſteht. Die letzte \label{K_L02747-2v}\edtext{Correſpondenz von \textsc{Uhl\pwindex{Uhl, Friedrich 14.05.1825 – 20.01.1906@\textsc{Uhl, Friedrich} (14.05.1825 – 20.01.1906), \emph{Journalist}|pw}}\pwindex{Wiener Brief [Die neue Saison im Burgtheater]1895-09-08@\emph{Wiener Brief [Die neue Saison im Burgtheater]} {[}1895-09-08{]}|pwv}}{\lemma{\textnormal{\emph{Correſpondenz von Uhl}}}\Cendnote{\textnormal{[Friedrich Uhl\pwindex{Uhl, Friedrich 14.05.1825 – 20.01.1906@\textsc{Uhl, Friedrich} (14.05.1825 – 20.01.1906), \emph{Journalist}|pwk}]: \emph{Wiener Brief}\pwindex{Wiener Brief [Die neue Saison im Burgtheater]1895-09-08@\emph{Wiener Brief [Die neue Saison im Burgtheater]} {[}1895-09-08{]}|pwk}. In: \emph{Frankfurter Zeitung}\pwindex{?? Werk@Nicht ermittelte Verfasserinnen und Verfasser!Frankfurter Zeitung1856 – 1943@\emph{Frankfurter Zeitung} {[}1856 – 1943{]}|pwk}, Jg. 40, Nr. 249, 8. 9. 1895, Zweites
                     Morgenblatt, S. 1; vgl. Paul Goldmann an Arthur Schnitzler, 23. 9. [1895]}}}\label{K_L02747-2h} bei uns dürſte wohl günſtigen Einfluß haben durch die Drohung, Rechenſchaft
               am Ende des Jahres zu fordern.\pend
           \pstart
           \textsc{Wolff}\pwindex{Wolff, Theodor 1868-08-02 – 1943-09-23@\textsc{Wolff, Theodor} (1868-08-02 – 1943-09-23), \emph{Schriftsteller, Journalist}|pw} (vom »Berliner Tageblatt\orgindex{Berliner Tageblatt@Berliner Tageblatt|pw}«) erzählte mir, er
               habe in Berlin\oindex{Berlin@\textbf{Berlin}|pw} jetzt gehört, daß Dein Stück\pwindex{Schnitzler, Arthur 15.05.1862 – 21.10.1931@\textsc{Schnitzler, Arthur} (15.05.1862 – 21.10.1931), \emph{Schriftsteller, Mediziner}!Liebelei. Schauspiel in drei Akten1895-10-09@\strich\emph{Liebelei. Schauspiel in drei Akten} {[}1895-10-09{]}|pwv} unter den erſten \strikeout{d\textcolor{gray}{r}} darankommen ſolle.\pend
           \pstart
           Und ſchreibe mir, wie es Dir ſonſt geht?\pend
           \pstart
           Viele treue Grüße! {\\[\baselineskip]}Dein {\\[\baselineskip]}\spacefill\mbox{Paul Goldmann}\pend
           \leftskip=0em{}\pstart
           \noindent{}\textsc{Frischauer\pwindex{Frischauer, Berthold 1851-09-09 – 1924-02-04@\textsc{Frischauer, Berthold} (1851-09-09 – 1924-02-04), \emph{Journalist}|pw}} kommt wirklich an \textsc{Herzl\pwindex{Herzl, Theodor 1860-05-02 – 1904-07-03@\textsc{Herzl, Theodor} (1860-05-02 – 1904-07-03), \emph{Schriftsteller, Journalist}|pw}s}{ }\label{K_L02747-1v}\edtext{Stelle}{\lemma{\textnormal{\emph{Stelle}}}\Cendnote{\textnormal{als Paris\oindex{Paris@\textbf{Paris}|pwk}er Korrespondent\pwindex{Frischauer, Berthold 1851-09-09 – 1924-02-04@\textsc{Frischauer, Berthold} (1851-09-09 – 1924-02-04), \emph{Journalist}|pwkv} der \emph{Neuen Freien Presse}\orgindex{Neue Freie Presse@Neue Freie Presse|pwk}}}}\label{K_L02747-1h}.\pend
           
         
         \endnumbering\mylabel{h}\end{ledgroupsized}  \newcommand{\dateiname}{L02747}\newcommand{\titel}{Paul Goldmann an Arthur Schnitzler, 12. 9. [1895]}\newcommand{\editorInnen}{Martin Anton Müller und Laura Untner}%% latex-leseansicht-abspann.tex
%% Abspann für die Leseansicht.
%% Der Schalter \ifkorrekturansicht ist bereits durch den Vorspann gesetzt.

%% latex-abspann.tex
%% Gemeinsamer Abspann für Korrekturansicht und Leseansicht.
%% Setzt den Schalter \ifkorrekturansicht voraus (gesetzt in den
%% einbindenden Dateien latex-korrekturansicht-abspann.tex bzw.
%% latex-leseansicht-abspann.tex).
%% ---------------------------------------------------------------

\normalsize

% Das esempio-Environment wird nur in der Leseansicht benötigt
\ifkorrekturansicht\else
\newenvironment{esempio}[3]%
{
    \vspace{1.5ex}
    \rlap{\underline{#1}}
    \par
    \setlength{\parindent}{0cm}
    \nopagebreak
    \leftskip=#2cm
    \rightskip=#3cm
}
{
    \par
}
\fi

\doendnotes{C}
\bigskip
\vfill

\clearpage

\footnotesize

\ifkorrekturansicht
  \lohead{\textsc{register}}
\fi

% theindex-Environment neu definieren ohne reledmac
\makeatletter
\renewenvironment{theindex}{%
  \ifkorrekturansicht
    \section*{\indexname}%
  \else
    \subsubsection*{Index der erwähnten Entitäten}%
  \fi
  \setlength{\parindent}{0pt}%
  \setlength{\parskip}{0pt plus 0.3pt}%
  \let\item\@idxitem
}{%
  \ifkorrekturansicht\clearpage\fi
}
\makeatother

\IfFileExists{\jobname-pw.ind}{\input{\jobname-pw.ind}}{}

% Quellenangabe nur in der Leseansicht
\ifkorrekturansicht\else
% Fallback-Definitionen, falls die .tex-Datei \titel etc. nicht gesetzt hat
\providecommand{\titel}{}
\providecommand{\editorInnen}{}
\providecommand{\dateiname}{\jobname}

\vspace{3cm}

\vfill

\footnotesize
\textsc{Quelle}: \titel. Herausgegeben von {\editorInnen}. In: \emph{Arthur Schnitzler: Briefwechsel mit Autorinnen und Autoren}.
 Digitale Edition, https://schnitzler-briefe.acdh.oeaw.ac.at/{\dateiname}.html (Stand \today)
\fi

\end{document}


      