%% latex-korrekturansicht-vorspann.tex
%% Vorspann für die Korrekturansicht.
%% Lädt die gemeinsame Datei latex-vorspann.tex mit gesetztem Schalter.

\newif\ifkorrekturansicht
\korrekturansichttrue

\input{../tex-inputs/latex-vorspann}


\section[ Felix Salten an Arthur Schnitzler, 8. 8. 1900]{L03309 Felix Salten an Arthur Schnitzler, 8. 8. 1900}
\nopagebreak\mylabel{L03309v}
\rehead{ }\normalsize\beginnumbering\briefempfaengerindex{Schnitzler, Arthur@\textsc{Schnitzler, Arthur}!zzzSalten, Felix@\emph{von Felix Salten}!1900-08-081@{8. 8. 1900}|(be}
\toendnotes[C]{\smallbreak\pagebreak[2]}\Standort{CUL, Schnitzler, B 89, A 2.}
\physDesc{Brief, 1 Blatt, 1 Seite, 406 Zeichen
\newline{}Handschrift: schwarze Tinte, lateinische Kurrent
\newline{}Ordnung: mit Bleistift von unbekannter Hand nummeriert: »133« }\toendnotes[C]{\smallbreak}
\pstart
           \raggedleft{}{\pb}Karlsbad\oindex{Karlsbad@\textbf{Karlsbad}, \emph{P.PPLA}|pw}, 8./VIII. 00.\pend
           \vspace{0.5em}
\pstart
           Lieber Arthur, ich bitte, \label{K_L03309-1v}\edtext{eingeschloßenen Brief}{\lemma{\textnormal{\emph{eingeschloßenen Brief}}}\Cendnote{\textnormal{Die Beilage ist nicht erhalten.}}}\label{K_L03309-1} an Osc. Mayer\pwindex{Mayer, Oskar 1876 – 15.05.1915@\textsc{Mayer, Oskar} (1876 – 15.05.1915), \emph{Schriftsteller/Schriftstellerin, Beamter/Beamte}|pw} gütigst befördern zu wollen, dessen Adreße mir
               leider nicht bekannt ist, und der ja wol noch in Ischl\oindex{Bad Ischl@\textbf{Bad Ischl}, \emph{P.PPL}|pw} oder mit Ihnen in Salzburg\oindex{Salzburg@\textbf{Salzburg}, \emph{A.ADM2}|pw} sich
               befindet. Es handelt sich um eine mir ganz unerfindliche Geschichte, die ich gerne so
               oder so aufgeklärt sähe.\pend
           
\pstart
           Vielen Dank und herzlichste Grüße. Ich bin vermuthlich Sonntag oder Montag in \label{K_L03309-2v}\edtext{Ischl\oindex{Bad Ischl@\textbf{Bad Ischl}, \emph{P.PPL}|pw}}{\lemma{\textnormal{\emph{Ischl}}}\Cendnote{\textnormal{Siehe Felix Salten an Arthur Schnitzler, 7. 8. 1900.
               }}}\label{K_L03309-2}.\pend
           
\pstart
           Ihr {\\[\baselineskip]}\spacefill\mbox{Salten}\pend
           \leftskip=0em{}\selectlanguage{ngerman}\endnumbering\briefempfaengerindex{Schnitzler, Arthur@\textsc{Schnitzler, Arthur}!zzzSalten, Felix@\emph{von Felix Salten}!1900-08-081@{8. 8. 1900}|)be}\mylabel{L03309h}  \normalsize

\doendnotes{C}
\bigskip
\vfill

\clearpage

\footnotesize

\lohead{\textsc{register}}

% Definiere theindex-Environment komplett neu ohne reledmac
\makeatletter
\renewenvironment{theindex}{%
  \section*{\indexname}%
  \setlength{\parindent}{0pt}%
  \setlength{\parskip}{0pt plus 0.3pt}%
  \let\item\@idxitem
}{%
  \clearpage
}
\makeatother

\IfFileExists{\jobname-pw.ind}{\input{\jobname-pw.ind}}{}

\end{document}

      