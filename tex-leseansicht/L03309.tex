%% latex-leseansicht-vorspann.tex
%% Vorspann für die Leseansicht.
%% Lädt die gemeinsame Datei latex-vorspann.tex mit nicht gesetztem Schalter.

\newif\ifkorrekturansicht
\korrekturansichtfalse

\input{../tex-inputs/latex-vorspann}


\section[ Felix Salten an Arthur Schnitzler, 8. 8. 1900]{L03309 Felix Salten an Arthur Schnitzler,  8. 8. 1900}
\nopagebreak\mylabel{L03309v}
\rehead{ }\normalsize\beginnumbering\briefempfaengerindex{Schnitzler, Arthur@\textsc{Schnitzler, Arthur}!zzzSalten, Felix@\emph{von Felix Salten}!1900-08-081@{8. 8. 1900}|(be}
\toendnotes[C]{\smallbreak\pagebreak[2]}
\correspDesc{Versand  durch Felix Salten am 8. 8. 1900 in Karlsbad
\newline{}Erhalt  durch Arthur Schnitzler im Zeitraum [9. 8. 1900
                  – 13. 8. 1900?] in Bad Ischl}\toendnotes[C]{\smallbreak}
\Standort{CUL, Schnitzler, B 89, A 2.}
\physDesc{Brief, 1 Blatt, 1 Seite, 406 Zeichen
\newline{}Handschrift: schwarze Tinte, lateinische Kurrent
\newline{}Ordnung: mit Bleistift von unbekannter Hand nummeriert: »133« }\toendnotes[C]{\smallbreak}
\pstart
           \raggedleft{}{\pb}Karlsbad\oindex{Karlsbad@\textbf{Karlsbad}|pw}, 8./VIII. 00.\pend
           \vspace{0.5em}
\pstart
           Lieber Arthur, ich bitte, \label{K_L03309-1v}\edtext{eingeschloßenen Brief}{\lemma{\textnormal{\emph{eingeschloßenen Brief}}}\Cendnote{\textnormal{Die Beilage ist nicht erhalten.}}}\label{K_L03309-1} an Osc. Mayer\pwindex{Mayer, Oskar 1876 – 15.\,5.\,1915 München@\textsc{Mayer, Oskar} (1876 – 15.\,5.\,1915 München), \emph{Schriftsteller, Beamter}|pw} gütigst befördern zu wollen, dessen Adreße mir
               leider nicht bekannt ist, und der ja wol noch in Ischl\oindex{Bad Ischl@\textbf{Bad Ischl}|pw} oder mit Ihnen in Salzburg\oindex{Salzburg@\textbf{Salzburg}, \emph{Verwaltungsgebiet}|pw} sich
               befindet. Es handelt sich um eine mir ganz unerfindliche Geschichte, die ich gerne so
               oder so aufgeklärt sähe.\pend
           
\pstart
           Vielen Dank und herzlichste Grüße. Ich bin vermuthlich Sonntag oder Montag in \label{K_L03309-2v}\edtext{Ischl\oindex{Bad Ischl@\textbf{Bad Ischl}|pw}}{\lemma{\textnormal{\emph{Ischl}}}\Cendnote{\textnormal{Siehe XXXX Auszeichnungsfehler: Dokument L03308 nicht gefunden.
               }}}\label{K_L03309-2}.\pend
           
\pstart
           Ihr {\\[\baselineskip]}\spacefill\mbox{Salten}\pend
           \leftskip=0em{}\selectlanguage{ngerman}\endnumbering\briefempfaengerindex{Schnitzler, Arthur@\textsc{Schnitzler, Arthur}!zzzSalten, Felix@\emph{von Felix Salten}!1900-08-081@{8. 8. 1900}|)be}\mylabel{L03309h}  \newcommand{\dateiname}{L03309}\newcommand{\titel}{Felix Salten an Arthur Schnitzler, 8. 8. 1900}\newcommand{\editorInnen}{Martin Anton Müller und Laura Untner}%% latex-leseansicht-abspann.tex
%% Abspann für die Leseansicht.
%% Der Schalter \ifkorrekturansicht ist bereits durch den Vorspann gesetzt.

%% latex-abspann.tex
%% Gemeinsamer Abspann für Korrekturansicht und Leseansicht.
%% Setzt den Schalter \ifkorrekturansicht voraus (gesetzt in den
%% einbindenden Dateien latex-korrekturansicht-abspann.tex bzw.
%% latex-leseansicht-abspann.tex).
%% ---------------------------------------------------------------

\normalsize

% Das esempio-Environment wird nur in der Leseansicht benötigt
\ifkorrekturansicht\else
\newenvironment{esempio}[3]%
{
    \vspace{1.5ex}
    \rlap{\underline{#1}}
    \par
    \setlength{\parindent}{0cm}
    \nopagebreak
    \leftskip=#2cm
    \rightskip=#3cm
}
{
    \par
}
\fi

\doendnotes{C}
\bigskip
\vfill

\clearpage

\footnotesize

\ifkorrekturansicht
  \lohead{\textsc{register}}
\fi

% theindex-Environment neu definieren ohne reledmac
\makeatletter
\renewenvironment{theindex}{%
  \ifkorrekturansicht
    \section*{\indexname}%
  \else
    \subsubsection*{Index der erwähnten Entitäten}%
  \fi
  \setlength{\parindent}{0pt}%
  \setlength{\parskip}{0pt plus 0.3pt}%
  \let\item\@idxitem
}{%
  \ifkorrekturansicht\clearpage\fi
}
\makeatother

\IfFileExists{\jobname-pw.ind}{\input{\jobname-pw.ind}}{}

% Quellenangabe nur in der Leseansicht
\ifkorrekturansicht\else
% Fallback-Definitionen, falls die .tex-Datei \titel etc. nicht gesetzt hat
\providecommand{\titel}{}
\providecommand{\editorInnen}{}
\providecommand{\dateiname}{\jobname}

\vspace{3cm}

\vfill

\footnotesize
\textsc{Quelle}: \titel. Herausgegeben von {\editorInnen}. In: \emph{Arthur Schnitzler: Briefwechsel mit Autorinnen und Autoren}.
 Digitale Edition, https://schnitzler-briefe.acdh.oeaw.ac.at/{\dateiname}.html (Stand \today)
\fi

\end{document}


