%% latex-korrekturansicht-vorspann.tex
%% Vorspann für die Korrekturansicht.
%% Lädt die gemeinsame Datei latex-vorspann.tex mit gesetztem Schalter.

\newif\ifkorrekturansicht
\korrekturansichttrue

\input{../tex-inputs/latex-vorspann}


\section[Arthur Schnitzler an Hermann Bahr, 16. 4. 1907]{L01669 Arthur Schnitzler an Hermann Bahr, 16. 4. 1907}
\nopagebreak\mylabel{L01669v}
\rehead{ }\normalsize\beginnumbering\briefempfaengerindex{Bahr, Hermann@\textsc{Bahr, Hermann}!zzzSchnitzler, Arthur@\emph{von Arthur Schnitzler}!1907-04-161@{16. 4. 1907}|(be}
\toendnotes[C]{\smallbreak\pagebreak[2]}\Standort{TMW, HS AM 60172 Ba.}
\physDesc{Briefkarte, 288 Zeichen
\newline{}Handschrift: schwarze Tinte, deutsche Kurrent
\newline{}Ordnung: Lochung }
\buchAbdrucke{\weitereDrucke{1) Arthur Schnitzler: \emph{The Letters of Arthur Schnitzler to Hermann Bahr}. Chapel Hill: \emph{The University of North Carolina Press} 1978, S. 99.} \weitereDrucke{2) Hermann Bahr, Arthur Schnitzler: \emph{Briefwechsel, Aufzeichnungen, Dokumente (1891–1931)}. Göttingen: \emph{Wallstein} 2018, S. 391.} }\toendnotes[C]{\smallbreak}
\pstart
           {\pb}\textcolor{gray}{\textbf{Dr. Arthur Schnitzler}}\hfill 16. 4. 907\pend
           
\pstart
           \textcolor{gray}{\textbf{Wien XVIII. Spoettelgasse 7\oindex{Edmund-Weiss-Gasse 7@\textbf{Edmund-Weiß-Gasse 7}, \emph{Wohngebäude (K.WHS)}|pw}.}}\pend
           \vspace{0.5em}
\pstart
           lieber Hermann, ich ſchlage dir vor, \label{K_L01669-1v}\edtext{Samſtag}{\lemma{\textnormal{\emph{Samſtag}}}\Cendnote{\textnormal{Siehe A. S.: \emph{Tagebuch}, 20. 4. 1907.
               }}}\label{K_L01669-1} zu mir zu kommen und natürlich mit uns\pwindex{Schnitzler, Heinrich 09.08.1902 – 12.07.1982@\textsc{Schnitzler, Heinrich} (09.08.1902 – 12.07.1982), \emph{Regisseur/Regisseurin, Schauspieler/Schauspielerin}|pwv}\pwindex{Schnitzler, Olga 17.01.1882 – 13.01.1970@\textsc{Schnitzler, Olga} (17.01.1882 – 13.01.1970), \emph{Schauspieler/Schauspielerin, Sänger/Sängerin}|pwv} zu ſpeiſen. Paſſt dir der Samſtag nicht,
               ſo theil es mir bitte gleich mit, u auch \uline{wie lange}{ }{\pb}du überhaupt \label{K_L01669-2v}\edtext{in Wien\oindex{Wien@\textbf{Wien}, \emph{A.ADM2}|pw}
                  bleibſt}{\lemma{\textnormal{\emph{in Wien
                  bleibſt}}}\Cendnote{\textnormal{Vom 1. 5. 1907 bis zum
                     8. 5. 1907 urlaubte Bahr\pwindex{Bahr, Hermann 19.07.1863 – 15.01.1934@\textsc{Bahr, Hermann} (19.07.1863 – 15.01.1934), \emph{Schriftsteller/Schriftstellerin, Kritiker/Kritikerin}|pwk} an
                  der oberen Adria\oindex{Adriatisches Meer@\textbf{Adriatisches Meer}, \emph{Meer (N.MER)}|pwk}.}}}\label{K_L01669-2}.\pend
           
\pstart
           Ich freue mich ſehr dich wieder zu ſehen.{\\[\baselineskip]}Herzlichſt mit Grüßen von uns\pwindex{Schnitzler, Heinrich 09.08.1902 – 12.07.1982@\textsc{Schnitzler, Heinrich} (09.08.1902 – 12.07.1982), \emph{Regisseur/Regisseurin, Schauspieler/Schauspielerin}|pwv}\pwindex{Schnitzler, Olga 17.01.1882 – 13.01.1970@\textsc{Schnitzler, Olga} (17.01.1882 – 13.01.1970), \emph{Schauspieler/Schauspielerin, Sänger/Sängerin}|pwv} allen{\\[\baselineskip]}dein{\\[\baselineskip]}\spacefill\mbox{Arthur}\pend
           \leftskip=0em{}\selectlanguage{ngerman}\endnumbering\briefempfaengerindex{Bahr, Hermann@\textsc{Bahr, Hermann}!zzzSchnitzler, Arthur@\emph{von Arthur Schnitzler}!1907-04-161@{16. 4. 1907}|)be}\mylabel{L01669h}  \normalsize

\doendnotes{C}
\bigskip
\vfill

\clearpage

\footnotesize

\lohead{\textsc{register}}

% Definiere theindex-Environment komplett neu ohne reledmac
\makeatletter
\renewenvironment{theindex}{%
  \section*{\indexname}%
  \setlength{\parindent}{0pt}%
  \setlength{\parskip}{0pt plus 0.3pt}%
  \let\item\@idxitem
}{%
  \clearpage
}
\makeatother

\IfFileExists{\jobname-pw.ind}{\input{\jobname-pw.ind}}{}

\end{document}

      