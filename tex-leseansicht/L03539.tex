%% latex-leseansicht-vorspann.tex
%% Vorspann für die Leseansicht.
%% Lädt die gemeinsame Datei latex-vorspann.tex mit nicht gesetztem Schalter.

\newif\ifkorrekturansicht
\korrekturansichtfalse

\input{../tex-inputs/latex-vorspann}

\begin{center}
            \textcolor{red}{ENTWURF, NICHT FERTIG KORRIGIERT}
                      \end{center}
            
         
         \renewcommand{\erwaehntePersonen}{Personen: Franziska Goldmann, Paul Goldmann}
         \renewcommand{\erwaehnteOrte}{Orte: Berlin, Wien}
         \renewcommand{\erwaehnteWerke}{Werke: Fräulein Else}
               \section[Franziska Goldmann an Arthur Schnitzler, {[}Ende Oktober 1925?{]}]{ Franziska Goldmann an Arthur Schnitzler, {[}Ende Oktober 1925?{]}}\nopagebreak\mylabel{v}\rehead{ }\begin{ledgroupsized}[t]{13cm}\normalsize\beginnumbering\briefempfaengerindex{Schnitzler, Arthur@\textsc{Schnitzler, Arthur}!zzzGoldmann, Franziska@\emph{von Franziska Goldmann}!1925-10-311@{{[}Ende Oktober 1925?{]}}|(be} \toendnotes[C]{\smallbreak\pagebreak[2]} \Standort{DLA, A:Schnitzler, HS.NZ85.1.3161.}
\physDesc{Brief, 1 Blatt, 2 Seiten, 512 Zeichen
\newline{}Handschrift: schwarze Tinte, lateinische Kurrent
\newline{}Schnitzler: 1) mit Bleistift Vermerk »Franz
                                          Gold\textcolor{gray}{ma}{[}nn{]}\pwindex{Goldmann, Franziska 1911-05-29 – 1963-08-19@\textsc{Goldmann, Franziska} (1911-05-29 – 1963-08-19), \emph{Schauspielerin}|pw}«  2) mit rotem Buntstift drei Unterstreichungen}\toendnotes[C]{\smallbreak}\pstart\center{}{\pb}Sehr geehrter Herr Dr.\pend\pstart
           Bitte entschuldigen Sie, daß ich Ihnen erst jetzt für die Mühe danke, die Sie sich
               machten, indem Sie mir Ihr reizendes \label{K_L03539-1v}\edtext{Buch\pwindex{Schnitzler, Arthur 15.05.1862 – 21.10.1931@\textsc{Schnitzler, Arthur} (15.05.1862 – 21.10.1931), \emph{Schriftsteller, Mediziner}!Fraeulein Else01. 10. 1924@\strich\emph{Fräulein Else} {[}01. 10. 1924{]}|pwuv}}{\lemma{\textnormal{\emph{Buch}}}\Cendnote{\textnormal{In Goldmann\pwindex{Goldmann, Paul 31.01.1865 – 25.09.1935@\textsc{Goldmann, Paul} (31.01.1865 – 25.09.1935), \emph{Schriftsteller, Journalist}|pwk}s Brief vom 24. 10. 1925 ist zu lesen: »Franzi\pwindex{Goldmann, Franziska 1911-05-29 – 1963-08-19@\textsc{Goldmann, Franziska} (1911-05-29 – 1963-08-19), \emph{Schauspielerin}|pw} iſt bereits in ›Fräulein
                  Elſe\pwindex{Schnitzler, Arthur 15.05.1862 – 21.10.1931@\textsc{Schnitzler, Arthur} (15.05.1862 – 21.10.1931), \emph{Schriftsteller, Mediziner}!Fraeulein Else01. 10. 1924@\strich\emph{Fräulein Else} {[}01. 10. 1924{]}|pw}‹ vertieft u. erklärt, es ſei das Schönſte, das ſie je geleſen habe, –
                  dankt Dir auch für die eigenhändige Widmung, mit der ſie in ihrer Klaſſe großen
                  Eindruck zu machen hofft.« Aufgrund der Ähnlichkeit der Schilderungen ist davon
                  auszugehen, dass der Brief von Franziska
                     Goldmann\pwindex{Goldmann, Franziska 1911-05-29 – 1963-08-19@\textsc{Goldmann, Franziska} (1911-05-29 – 1963-08-19), \emph{Schauspielerin}|pwk} ungefähr zur selben Zeit, Ende Oktober 1925, entstand.}}}\label{K_L03539-1h} schickten. Ich war aber sehr neugierig
               darauf und wollte es zuerst auslesen. Es hat mir \substVorne{}\textsuperscript{\textcolor{gray}{f}}\substDazwischen{}v\substHinten{}on Anfang bis Ende den größten Spaß gemacht, besonders der Schluß, den ich
               sehr aufregend und tragisch finde\substVorne{}\textsuperscript{.}\substDazwischen{},\substHinten{} und ist eins der schönsten Bücher, die ich gelesen habe. Über die Widmung
               sind meine \label{T_L03539-1v}\edtext{sämtlichen}{\lemma{\textnormal{\emph{sämtlichen}}}\Cendnote{\textnormal{korrigiert aus »samtlichen«}}}\label{T_L03539-1h} Freunde zersprungen.\pend
           \pstart
           {\pb}Mit nochmals vielem herzlichen Dank {\\[\baselineskip]}Ihre {\\[\baselineskip]}\spacefill\mbox{Franzi Goldmann}\pend
           \leftskip=0em{}
         
         \endnumbering\mylabel{h}\end{ledgroupsized}  \newcommand{\dateiname}{L03539}\newcommand{\titel}{Franziska Goldmann an Arthur Schnitzler, [Ende Oktober 1925?]}\newcommand{\editorInnen}{Martin Anton Müller und Laura Untner}%% latex-leseansicht-abspann.tex
%% Abspann für die Leseansicht.
%% Der Schalter \ifkorrekturansicht ist bereits durch den Vorspann gesetzt.

%% latex-abspann.tex
%% Gemeinsamer Abspann für Korrekturansicht und Leseansicht.
%% Setzt den Schalter \ifkorrekturansicht voraus (gesetzt in den
%% einbindenden Dateien latex-korrekturansicht-abspann.tex bzw.
%% latex-leseansicht-abspann.tex).
%% ---------------------------------------------------------------

\normalsize

% Das esempio-Environment wird nur in der Leseansicht benötigt
\ifkorrekturansicht\else
\newenvironment{esempio}[3]%
{
    \vspace{1.5ex}
    \rlap{\underline{#1}}
    \par
    \setlength{\parindent}{0cm}
    \nopagebreak
    \leftskip=#2cm
    \rightskip=#3cm
}
{
    \par
}
\fi

\doendnotes{C}
\bigskip
\vfill

\clearpage

\footnotesize

\ifkorrekturansicht
  \lohead{\textsc{register}}
\fi

% theindex-Environment neu definieren ohne reledmac
\makeatletter
\renewenvironment{theindex}{%
  \ifkorrekturansicht
    \section*{\indexname}%
  \else
    \subsubsection*{Index der erwähnten Entitäten}%
  \fi
  \setlength{\parindent}{0pt}%
  \setlength{\parskip}{0pt plus 0.3pt}%
  \let\item\@idxitem
}{%
  \ifkorrekturansicht\clearpage\fi
}
\makeatother

\IfFileExists{\jobname-pw.ind}{\input{\jobname-pw.ind}}{}

% Quellenangabe nur in der Leseansicht
\ifkorrekturansicht\else
% Fallback-Definitionen, falls die .tex-Datei \titel etc. nicht gesetzt hat
\providecommand{\titel}{}
\providecommand{\editorInnen}{}
\providecommand{\dateiname}{\jobname}

\vspace{3cm}

\vfill

\footnotesize
\textsc{Quelle}: \titel. Herausgegeben von {\editorInnen}. In: \emph{Arthur Schnitzler: Briefwechsel mit Autorinnen und Autoren}.
 Digitale Edition, https://schnitzler-briefe.acdh.oeaw.ac.at/{\dateiname}.html (Stand \today)
\fi

\end{document}


      