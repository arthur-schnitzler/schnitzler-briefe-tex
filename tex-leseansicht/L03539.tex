%% latex-korrekturansicht-vorspann.tex
%% Vorspann für die Korrekturansicht.
%% Lädt die gemeinsame Datei latex-vorspann.tex mit gesetztem Schalter.

\newif\ifkorrekturansicht
\korrekturansichttrue

\input{../tex-inputs/latex-vorspann}


\section[Franziska Goldmann an Arthur Schnitzler, {[}Ende Oktober 1925?{]}]{L03539 Franziska Goldmann an Arthur Schnitzler, {[}Ende Oktober 1925?{]}}
\nopagebreak\mylabel{L03539v}
\rehead{ }\normalsize\beginnumbering\briefempfaengerindex{Schnitzler, Arthur@\textsc{Schnitzler, Arthur}!zzzGoldmann, Franziska@\emph{von Franziska Goldmann}!1925-10-311@{{[}Ende Oktober 1925?{]}}|(be}
\toendnotes[C]{\smallbreak\pagebreak[2]}\Standort{DLA, A:Schnitzler, HS.NZ85.1.3161.}
\physDesc{Brief, 1 Blatt, 2 Seiten, 513 Zeichen
\newline{}Handschrift: schwarze Tinte, lateinische Kurrent
\newline{}Schnitzler: 1) mit Bleistift Vermerk »Franz{[}iska{]}
                                          Gold\textcolor{gray}{ma}{[}nn{]}\pwindex{Goldmann, Franziska 1911-05-29 – 1963-08-19@\textsc{Goldmann, Franziska} (1911-05-29 – 1963-08-19), \emph{Schauspieler/Schauspielerin}|pw}«  2) mit rotem Buntstift drei Unterstreichungen}\toendnotes[C]{\smallbreak}
\pstart\center{}{\pb}Sehr geehrter Herr Dr.\pend\vspace{0.5em}
\pstart
           Bitte entschuldigen Sie, daß ich Ihnen erst jetzt für die Mühe danke, die Sie sich
               machten, indem Sie mir Ihr reizendes \label{K_L03539-1v}\edtext{Buch\pwindex{Fraeulein Else@\emph{Fräulein Else}|pwuv}}{\lemma{\textnormal{\emph{Buch}}}\Cendnote{\textnormal{In Goldmanns\pwindex{Goldmann, Paul 31.01.1865 – 25.09.1935@\textsc{Goldmann, Paul} (31.01.1865 – 25.09.1935), \emph{Schriftsteller/Schriftstellerin, Journalist/Journalistin}|pwk} Brief vom 24. 10. 1925 ist zu lesen: »Franzi\pwindex{Goldmann, Franziska 1911-05-29 – 1963-08-19@\textsc{Goldmann, Franziska} (1911-05-29 – 1963-08-19), \emph{Schauspieler/Schauspielerin}|pw} iſt bereits in ›Fräulein Elſe\pwindex{Fraeulein Else@\emph{Fräulein Else}|pw}‹ vertieft u. erklärt, es ſei das
                     Schönſte, das ſie je geleſen habe, – dankt Dir auch für die eigenhändige
                     Widmung, mit der ſie in ihrer Klaſſe großen Eindruck zu machen hofft.«
                  Aufgrund der Ähnlichkeit der Schilderungen ist davon auszugehen, dass der Brief
                  von Franziska Goldmann\pwindex{Goldmann, Franziska 1911-05-29 – 1963-08-19@\textsc{Goldmann, Franziska} (1911-05-29 – 1963-08-19), \emph{Schauspieler/Schauspielerin}|pwk} ungefähr zur selben
                  Zeit, Ende Oktober 1925, verfasst wurde.}}}\label{K_L03539-1}
               schickten. Ich war aber sehr neugierig darauf und wollte es zuerst auslesen. Es hat
               mir \substVorne{}\textsuperscript{\textcolor{gray}{f}}\substDazwischen{}v\substHinten{}on Anfang bis Ende den größten Spaß gemacht, besonders der Schluß, den ich
               sehr aufregend und tragisch finde, und {[}es{]} ist eins der schönsten
               Bücher, die ich gelesen habe. Über die Widmung sind meine \label{T_L03539-1v}\edtext{sämtlichen}{\lemma{\textnormal{\emph{sämtlichen}}}\Cendnote{\textnormal{korrigiert aus »samtlichen«}}}\label{T_L03539-1} Freunde zersprungen.\pend
           
\pstart
           {\pb}Mit nochmals vielem herzlichen Dank {\\[\baselineskip]}Ihre {\\[\baselineskip]}\spacefill\mbox{Franzi Goldmann}\pend
           \leftskip=0em{}\selectlanguage{ngerman}\endnumbering\briefempfaengerindex{Schnitzler, Arthur@\textsc{Schnitzler, Arthur}!zzzGoldmann, Franziska@\emph{von Franziska Goldmann}!1925-10-311@{{[}Ende Oktober 1925?{]}}|)be}\mylabel{L03539h}  \normalsize

\doendnotes{C}
\bigskip
\vfill

\clearpage

\footnotesize

\lohead{\textsc{register}}

% Definiere theindex-Environment komplett neu ohne reledmac
\makeatletter
\renewenvironment{theindex}{%
  \section*{\indexname}%
  \setlength{\parindent}{0pt}%
  \setlength{\parskip}{0pt plus 0.3pt}%
  \let\item\@idxitem
}{%
  \clearpage
}
\makeatother

\IfFileExists{\jobname-pw.ind}{\input{\jobname-pw.ind}}{}

\end{document}

      