%% latex-korrekturansicht-vorspann.tex
%% Vorspann für die Korrekturansicht.
%% Lädt die gemeinsame Datei latex-vorspann.tex mit gesetztem Schalter.

\newif\ifkorrekturansicht
\korrekturansichttrue

\input{../tex-inputs/latex-vorspann}


\section[ Felix Salten an Arthur Schnitzler, {[}5. 9. 1896{]}]{L03180 Felix Salten an Arthur Schnitzler, {[}5. 9. 1896{]}}
\nopagebreak\mylabel{L03180v}
\rehead{ }\normalsize\beginnumbering\briefempfaengerindex{Schnitzler, Arthur@\textsc{Schnitzler, Arthur}!zzzSalten, Felix@\emph{von Felix Salten}!1896-09-052@{{[}5. 9. 1896{]}}|(be}
\toendnotes[C]{\smallbreak\pagebreak[2]}\Standort{CUL, Schnitzler, B 89, A 1.}
\physDesc{Brief, 1 Blatt, 1 Seite, 224 Zeichen
\newline{}Handschrift: schwarze Tinte, lateinische Kurrent
\newline{}Schnitzler: mit Bleistift datiert: »5/9 96« 
\newline{}Ordnung: mit Bleistift von unbekannter Hand nummeriert: »79« }\toendnotes[C]{\smallbreak}
\pstart
           {\pb}\textcolor{gray}{\textbf{\textbf{»Wiener Allgemeine
                        Zeitung\orgindex{Wiener Allgemeine Zeitung@Wiener Allgemeine Zeitung|pw}«}}}\pend
           
\pstart
           \textcolor{gray}{\textbf{Redaction und Adminiſtration:}}\pend
           
\pstart
           \textcolor{gray}{\textbf{Wien\oindex{Wien@\textbf{Wien}, \emph{A.ADM2}|pw}}}\pend
           
\pstart
           \textcolor{gray}{\textbf{\textbf{IX}/3,{ }\textbf{\so{Univerſitätsſtraße Nr. 6}}\oindex{Universitaetsstrasse@\textbf{Universitätsstraße}, \emph{Straße (K.STR)}|pw}\textbf{.}}}\pend
           
\pstart
           \textcolor{gray}{\textbf{Ankündigungs-Bureau:}}\pend
           
\pstart
           \textcolor{gray}{\textbf{\textbf{\so{I. Schulerſtraße Nr. 14}\oindex{Schulerstrasse@\textbf{Schulerstraße}, \emph{Straße (K.STR)}|pw}.
                     }}}\pend
           
\pstart
           \textcolor{gray}{\textbf{Telegramm-Adreſſe: »Allgemeine, Wien\oindex{Wien@\textbf{Wien}, \emph{A.ADM2}|pw}«.}}\pend
           
\pstart
           \textcolor{gray}{\textbf{Telephon der Redaction: Nr. 2180.}}\pend
           
\pstart
           \textcolor{gray}{\textbf{\hspace*{1.5em}„\hspace*{1.5em}„\hspace*{1.5em} Adminiſtration: Nr. 805.}}\pend
           \vspace{0.5em}
\pstart
           Lieber Arthur leider gibt’s keinen \label{K_L03180-1v}\edtext{Sitz heute}{\lemma{\textnormal{\emph{Sitz heute}}}\Cendnote{\textnormal{vermutlich für die Vorstellung von Ferdinand Raimunds\pwindex{Raimund, Ferdinand 01.06.1790 – 05.09.1836@\textsc{Raimund, Ferdinand} (01.06.1790 – 05.09.1836), \emph{Schauspieler/Schauspielerin, Dramatiker/Dramatikerin}|pwk}{ }\emph{Das Mädchen aus der Feenwelt oder Der Bauer als Millionär}\pwindex{Maedchen aus der Feenwelt oder Der Bauer als Millionaer@\emph{Das Mädchen aus der Feenwelt oder Der Bauer als Millionär}|pwk}
                  im Raimund-Theater\oindex{Raimund-Theater@\textbf{Raimund-Theater}, \emph{Theater (K.THE)}|pwk}, vgl. A. S.: \emph{Tagebuch}, 5. 9. 1896.}}}\label{K_L03180-1}. St-g.\pwindex{Sternberg, Julian 08.11.1868 – 28.06. 1945@\textsc{Sternberg, Julian} (08.11.1868 – 28.06. 1945), \emph{Journalist/Journalistin}|pw} hat mir ihn nicht gegeben {\kaufmannsund} mich hoch {\kaufmannsund} theuer
               gebeten, ich möge ihm denselben laßen.\pend
           
\pstart
           Also wenns nicht regnet \label{K_L03180-2v}\edtext{morgen{ }8'40{ }Südbahn\orgindex{Suedbahnstrecke@Südbahnstrecke|pw}}{\lemma{\textnormal{\emph{morgen 8'40 Südbahn}}}\Cendnote{\textnormal{Von wo aus der gemeinsame Radausflug starten sollte, vgl. A. S.: \emph{Tagebuch}, 6. 8. 1896.
               }}}\label{K_L03180-2}.\pend
           
\pstart
           Jedenfalls heute{ }Abend noch im \label{K_L03180-3v}\edtext{Caféhaus}{\lemma{\textnormal{\emph{Caféhaus}}}\Cendnote{\textnormal{Welches gemeint ist, konnte nicht bestimmt werden.}}}\label{K_L03180-3}\pend
           
\pstart
           herzlichst {\\[\baselineskip]}\spacefill\mbox{Salten}\pend
           \leftskip=0em{}\selectlanguage{ngerman}\endnumbering\briefempfaengerindex{Schnitzler, Arthur@\textsc{Schnitzler, Arthur}!zzzSalten, Felix@\emph{von Felix Salten}!1896-09-052@{{[}5. 9. 1896{]}}|)be}\mylabel{L03180h}  \normalsize

\doendnotes{C}
\bigskip
\vfill

\clearpage

\footnotesize

\lohead{\textsc{register}}

% Definiere theindex-Environment komplett neu ohne reledmac
\makeatletter
\renewenvironment{theindex}{%
  \section*{\indexname}%
  \setlength{\parindent}{0pt}%
  \setlength{\parskip}{0pt plus 0.3pt}%
  \let\item\@idxitem
}{%
  \clearpage
}
\makeatother

\IfFileExists{\jobname-pw.ind}{\input{\jobname-pw.ind}}{}

\end{document}

      