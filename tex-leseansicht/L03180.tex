%% latex-leseansicht-vorspann.tex
%% Vorspann für die Leseansicht.
%% Lädt die gemeinsame Datei latex-vorspann.tex mit nicht gesetztem Schalter.

\newif\ifkorrekturansicht
\korrekturansichtfalse

\input{../tex-inputs/latex-vorspann}


         
         \renewcommand{\erwaehntePersonen}{Personen: Ferdinand Raimund, Julian Sternberg}
         \renewcommand{\erwaehnteInstitutionen}{Institutionen: Südbahnstrecke, Wiener Allgemeine Zeitung}
         \renewcommand{\erwaehnteOrte}{Orte: Raimund-Theater, Schulerstraße, Universitätsstraße, Wien}
         \renewcommand{\erwaehnteWerke}{Werke: Das Mädchen aus der Feenwelt oder Der Bauer als Millionär}
               \section[ Felix Salten an Arthur Schnitzler, {[}5. 9. 1896{]}]{ Felix Salten an Arthur Schnitzler, {[}5. 9. 1896{]}}\nopagebreak\mylabel{v}\rehead{ }\begin{ledgroupsized}[t]{13cm}\normalsize\beginnumbering \toendnotes[C]{\smallbreak\pagebreak[2]} \Standort{CUL, Schnitzler, B 89, A 1.}
\physDesc{Brief, 1 Blatt, 1 Seite, 224 Zeichen
\newline{}Handschrift: schwarze Tinte, lateinische Kurrent
\newline{}Schnitzler: mit Bleistift datiert: »5/9 96« 
\newline{}Ordnung: mit Bleistift von unbekannter Hand nummeriert: »79« }\toendnotes[C]{\smallbreak}\pstart
           \noindent{}{\pb}\textcolor{gray}{\textbf{\textbf{»Wiener Allgemeine
                        Zeitung\orgindex{Wiener Allgemeine Zeitung@Wiener Allgemeine Zeitung|pw}«}}}\pend
           \pstart
           \textcolor{gray}{\textbf{Redaction und Adminiſtration:}}\pend
           \pstart
           \textcolor{gray}{\textbf{Wien\oindex{Wien@\textbf{Wien}|pw}}}\pend
           \pstart
           \textcolor{gray}{\textbf{\textbf{IX}/3,{ }\textbf{\so{Univerſitätsſtraße Nr. 6}}\oindex{Universitaetsstrasse@\textbf{Universitätsstraße}|pw}\textbf{.}}}\pend
           \pstart
           \textcolor{gray}{\textbf{Ankündigungs-Bureau:}}\pend
           \pstart
           \textcolor{gray}{\textbf{\textbf{\so{I. Schulerſtraße Nr. 14}\oindex{Schulerstrasse@\textbf{Schulerstraße}|pw}.
                     }}}\pend
           \pstart
           \textcolor{gray}{\textbf{Telegramm-Adreſſe: »Allgemeine, Wien\oindex{Wien@\textbf{Wien}|pw}«.}}\pend
           \pstart
           \textcolor{gray}{\textbf{Telephon der Redaction: Nr. 2180.}}\pend
           \pstart
           \textcolor{gray}{\textbf{\hspace*{1.5em}„\hspace*{1.5em}„\hspace*{1.5em} Adminiſtration: Nr. 805.}}\pend
           \pstart
           Lieber Arthur leider gibt’s keinen \label{K_L03180-1v}\edtext{Sitz heute}{\lemma{\textnormal{\emph{Sitz heute}}}\Cendnote{\textnormal{vermutlich für die Vorstellung von Ferdinand Raimund\pwindex{Raimund, Ferdinand 01.06.1790 – 05.09.1836@\textsc{Raimund, Ferdinand} (01.06.1790 – 05.09.1836), \emph{Schauspieler, Dramatiker}|pwk}s \emph{Das Mädchen aus der Feenwelt oder Der Bauer als Millionär}\pwindex{Raimund, Ferdinand 01.06.1790 – 05.09.1836@\textsc{Raimund, Ferdinand} (01.06.1790 – 05.09.1836), \emph{Schauspieler, Dramatiker}!Maedchen aus der Feenwelt oder Der Bauer als Millionaer@\strich\emph{Das Mädchen aus der Feenwelt oder Der Bauer als Millionär}|pwk}
                  im Raimund-Theater\oindex{Raimund-Theater@\textbf{Raimund-Theater}|pwk}, vgl. A. S.: \emph{Tagebuch}, 5. 9. 1896}}}\label{K_L03180-1h}. St-g.\pwindex{Sternberg, Julian 08.11.1868 – 28.06. 1945@\textsc{Sternberg, Julian} (08.11.1868 – 28.06. 1945), \emph{Journalist}|pw} hat mir ihn nicht gegeben {\kaufmannsund} mich hoch {\kaufmannsund} theuer
               gebeten, ich möge ihm denselben laßen.\pend
           \pstart
           Also wenns nicht regnet \label{K_L03180-2v}\edtext{morgen{ }8'40{ }Südbahn\orgindex{Suedbahnstrecke@Südbahnstrecke|pw}}{\lemma{\textnormal{\emph{morgen 8'40 Südbahn}}}\Cendnote{\textnormal{für einen gemeinsamen Radausflug, vgl. A. S.: \emph{Tagebuch}, 6. 8. 1896}}}\label{K_L03180-2h}.\pend
           \pstart
           Jedenfalls heute{ }Abend noch im \label{K_L03180-3v}\edtext{Caféhaus}{\lemma{\textnormal{\emph{Caféhaus}}}\Cendnote{\textnormal{Welches gemeint ist, konnte nicht bestimmt werden.}}}\label{K_L03180-3h}\pend
           \pstart
           herzlichst {\\[\baselineskip]}\spacefill\mbox{Salten}\pend
           \leftskip=0em{}
         
         \endnumbering\mylabel{h}\end{ledgroupsized}  \newcommand{\dateiname}{L03180}\newcommand{\titel}{Felix Salten an Arthur Schnitzler, [5. 9. 1896]}\newcommand{\editorInnen}{Martin Anton Müller und Laura Untner}%% latex-leseansicht-abspann.tex
%% Abspann für die Leseansicht.
%% Der Schalter \ifkorrekturansicht ist bereits durch den Vorspann gesetzt.

%% latex-abspann.tex
%% Gemeinsamer Abspann für Korrekturansicht und Leseansicht.
%% Setzt den Schalter \ifkorrekturansicht voraus (gesetzt in den
%% einbindenden Dateien latex-korrekturansicht-abspann.tex bzw.
%% latex-leseansicht-abspann.tex).
%% ---------------------------------------------------------------

\normalsize

% Das esempio-Environment wird nur in der Leseansicht benötigt
\ifkorrekturansicht\else
\newenvironment{esempio}[3]%
{
    \vspace{1.5ex}
    \rlap{\underline{#1}}
    \par
    \setlength{\parindent}{0cm}
    \nopagebreak
    \leftskip=#2cm
    \rightskip=#3cm
}
{
    \par
}
\fi

\doendnotes{C}
\bigskip
\vfill

\clearpage

\footnotesize

\ifkorrekturansicht
  \lohead{\textsc{register}}
\fi

% theindex-Environment neu definieren ohne reledmac
\makeatletter
\renewenvironment{theindex}{%
  \ifkorrekturansicht
    \section*{\indexname}%
  \else
    \subsubsection*{Index der erwähnten Entitäten}%
  \fi
  \setlength{\parindent}{0pt}%
  \setlength{\parskip}{0pt plus 0.3pt}%
  \let\item\@idxitem
}{%
  \ifkorrekturansicht\clearpage\fi
}
\makeatother

\IfFileExists{\jobname-pw.ind}{\input{\jobname-pw.ind}}{}

% Quellenangabe nur in der Leseansicht
\ifkorrekturansicht\else
% Fallback-Definitionen, falls die .tex-Datei \titel etc. nicht gesetzt hat
\providecommand{\titel}{}
\providecommand{\editorInnen}{}
\providecommand{\dateiname}{\jobname}

\vspace{3cm}

\vfill

\footnotesize
\textsc{Quelle}: \titel. Herausgegeben von {\editorInnen}. In: \emph{Arthur Schnitzler: Briefwechsel mit Autorinnen und Autoren}.
 Digitale Edition, https://schnitzler-briefe.acdh.oeaw.ac.at/{\dateiname}.html (Stand \today)
\fi

\end{document}


      