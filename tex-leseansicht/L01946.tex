%% latex-leseansicht-vorspann.tex
%% Vorspann für die Leseansicht.
%% Lädt die gemeinsame Datei latex-vorspann.tex mit nicht gesetztem Schalter.

\newif\ifkorrekturansicht
\korrekturansichtfalse

\input{../tex-inputs/latex-vorspann}


         
         \renewcommand{\erwaehntePersonen}{Personen: Peter Altenberg, Raoul Auernheimer, Oskar Bie, Paul Ernst, Johann Wolfgang von Goethe, Franz Grillparzer, Stefan Großmann, Hanns von Gumppenberg, Abraham Halbert, Friedrich Hebbel, Karl Borromäus Heinrich, Hermann Hesse, Camill Hoffmann,  Homer, Heinrich von Kleist, Karl Kraus, Alfred Kubin, Otto König, Philipp Langmann, Thomas Mann, Georg Müller, Alfred Polgar, Ferdinand von Saar, Paul Scheerbart, Friedrich von Schiller, Adalbert Franz Seligmann, William Shakespeare, Otto Soyka, Richard Specht, Hans von Weber, Joseph Victor Widmann}
         \renewcommand{\erwaehnteInstitutionen}{Institutionen: Albert Langen, Der Merker, Der Sturm, Egon Fleischel {\kaufmannsund}  Co., Erich-Reiss-Verlag, Hyperion, Insel-Verlag, Ministerium für Inneres, Ministerium für Unterricht, Neue Freie Presse, Wiener Verlag}
         \renewcommand{\erwaehnteOrte}{Orte: Berlin, Deutschland, III., Landstraße, Ungarn, Vrádište, Wien, Österreich}
         \renewcommand{\erwaehnteWerke}{Werke: Bastei-Szene. Erste Szene des dritten Aufzuges aus der dramatischen Historie: »Der junge Medardus«, Buddenbrooks, Das Haus Delorme. Eine Familienszene, Der Schleier der Beatrice. Schauspiel in fünf Akten, Der Weg ins Freie. Roman, Der einsame Weg. Schauspiel in fünf Akten, Der grüne Kakadu. Groteske in einem Akt, Der junge Medardus. Dramatische Historie in einem Vorspiel und fünf Aufzügen, Die Fackel, Die Parasiten der Parasiten, Die Wahlverwandtschaften, Graf Cilli, Liebelei. Erstes Bild, Liebelei. Schauspiel in drei Akten, Neue Freie Presse, Reigen. Zehn Dialoge, Tod eines Seebären, Wanderers Lied, Widmungen zur Feier des siebzigsten Geburtstages Ferdinand von Saar’s., Zum großen Wurstel. Burleske in einem Akt}
               \section[Albert Ehrenstein an Arthur Schnitzler, 12. 7. 1910]{ Albert Ehrenstein an Arthur Schnitzler, 12. 7. 1910}\nopagebreak\mylabel{v}\rehead{ }\begin{ledgroupsized}[t]{13cm}\normalsize\beginnumbering \toendnotes[C]{\smallbreak\pagebreak[2]} \Standort{CUL, Schnitzler, B 30.}
\physDesc{Brief, 2 Blätter, 7 Seiten, 5914 Zeichen
\newline{}Handschrift: schwarze Tinte, deutsche Kurrent
\newline{}Schnitzler: mit Bleistift beschriftet: »\textsc{Ehrenstein}« }\buchAbdrucke{\weitereDrucke{Albert Ehrenstein: \emph{Briefe}. Hg. Hanni Mittelmann. München: \emph{Boer} 1989, S. 45–48 (Werke, 1).} }\toendnotes[C]{\smallbreak}\pstart
           {\pb}\textsc{Vradist bei Holics\oindex{Vrádište@\textbf{Vrádište}|pw}, }\hfill \textsc{12. Juli 1910}\pend
           \pstart
           \textsc{Ungarn\oindex{Ungarn@\textbf{Ungarn}|pw}}\pend
           \pstart{}\textsc{Hochverehrter Herr Doktor,}\pend\pstart
           ich glaube, es wird, Sie vielleicht intereſſieren, wenn ich wieder einmal über meine
               literariſchen Miß- und Erfolge Nachricht gebe. Kraus\pwindex{Kraus, Karl 28.04.1874 – 12.06.1936@\textsc{Kraus, Karl} (28.04.1874 – 12.06.1936), \emph{Schriftsteller, Publizist}|pw}, mit dem ich übrigens bereits ſehr ſchlecht ſtehe, weil wir beide, wie
               Sie wiſſen, recht unverträglich ſind, hat einmal ein \label{K_L01946_1v}\edtext{Gedicht\pwindex{Ehrenstein, Albert 23.12.1886 – 08.04.1950@\textsc{Ehrenstein, Albert} (23.12.1886 – 08.04.1950), \emph{Schriftsteller}!Wanderers Lied18. 02. 1910@\strich\emph{Wanderers Lied} {[}18. 02. 1910{]}|pwv}}{\lemma{\textnormal{\emph{Gedicht}}}\Cendnote{\textnormal{Albert Ehrenstein\pwindex{Ehrenstein, Albert 23.12.1886 – 08.04.1950@\textsc{Ehrenstein, Albert} (23.12.1886 – 08.04.1950), \emph{Schriftsteller}|pwk}: \emph{Wanderers Lied}\pwindex{Ehrenstein, Albert 23.12.1886 – 08.04.1950@\textsc{Ehrenstein, Albert} (23.12.1886 – 08.04.1950), \emph{Schriftsteller}!Wanderers Lied18. 02. 1910@\strich\emph{Wanderers Lied} {[}18. 02. 1910{]}|pwk}. In: \emph{Die
                        Fackel}\pwindex{Fackel1899-04 – 1936@\emph{Die Fackel} {[}1899-04 – 1936{]}|pwk}, Jg. 11, Nr. 296–297, 18. 2. 1910,
                  S. 36.}}}\label{K_L01946_1h} von mir gebracht, ein anderes akzeptiert, der
               honorarfeindliche Berlin\oindex{Berlin@\textbf{Berlin}|pw}er »Sturm\orgindex{Sturm@Der Sturm|pw}« zwei minderwertige Skizzen\pwindex{Ehrenstein, Albert 23.12.1886 – 08.04.1950@\textsc{Ehrenstein, Albert} (23.12.1886 – 08.04.1950), \emph{Schriftsteller}!Parasiten der Parasiten31. 03. 1910@\strich\emph{Die Parasiten der Parasiten} {[}31. 03. 1910{]}|pwv}\pwindex{Ehrenstein, Albert 23.12.1886 – 08.04.1950@\textsc{Ehrenstein, Albert} (23.12.1886 – 08.04.1950), \emph{Schriftsteller}!Tod eines Seebaeren16. 06. 1910@\strich\emph{Tod eines Seebären} {[}16. 06. 1910{]}|pwv}. Im übrigen ein Debacle
               auf der ganzen Linie. Die Verlage Reiß\orgindex{Erich-Reiss-Verlag@Erich-Reiss-Verlag|pw}, Fleiſchel\orgindex{Egon Fleischel und Co.@Egon Fleischel {\kaufmannsund}  Co.|pw}, Langen\orgindex{Albert Langen@Albert Langen|pw}, v. Weber\pwindex{Weber, Hans von 22.04.1872 – 22.04.1924@\textsc{Weber, Hans von} (22.04.1872 – 22.04.1924), \emph{Verleger}|pw}\orgindex{Hyperion@Hyperion|pwv} haben meine Sachen ohne weitere Begründung refuſiert, Georg Müller\pwindex{Mueller, Georg 29.12.1877 – 29.12.1917@\textsc{Müller, Georg} (29.12.1877 – 29.12.1917), \emph{Verleger}|pw} iſt trotz der Intervention der Herren Alfred Kubin\pwindex{Kubin, Alfred 10.04.1877 – 20.08.1959@\textsc{Kubin, Alfred} (10.04.1877 – 20.08.1959), \emph{Schriftsteller, Bildender Künstler, Bildender Künstler}|pw} und A. Halbert\pwindex{Halbert, Abraham 16.09.1881 – 15.10.1965@\textsc{Halbert, Abraham} (16.09.1881 – 15.10.1965), \emph{Schriftsteller, Journalist}|pw} zu einer höflichen Ablehnung geſchritten, der Inſelverlag\orgindex{Insel-Verlag@Insel-Verlag|pw} reagierte nach einer Empfehlung durch
                  \label{K_L01946_2v}\edtext{Paul Ernſt\pwindex{Ernst, Paul 07.03.1866 – 13.05.1933@\textsc{Ernst, Paul} (07.03.1866 – 13.05.1933), \emph{Schriftsteller}|pw}}{\lemma{\textnormal{\emph{Paul Ernſt}}}\Cendnote{\textnormal{Vgl. den Brief Ehrensteins\pwindex{Ehrenstein, Albert 23.12.1886 – 08.04.1950@\textsc{Ehrenstein, Albert} (23.12.1886 – 08.04.1950), \emph{Schriftsteller}|pwk} an Paul Ernst\pwindex{Ernst, Paul 07.03.1866 – 13.05.1933@\textsc{Ernst, Paul} (07.03.1866 – 13.05.1933), \emph{Schriftsteller}|pwk} vom 16. 5. 1910,
                     abgedruckt in: A. E.\pwindex{Ehrenstein, Albert 23.12.1886 – 08.04.1950@\textsc{Ehrenstein, Albert} (23.12.1886 – 08.04.1950), \emph{Schriftsteller}|pwk}: \emph{Briefe}, S. 39.}}}\label{K_L01946_2h}{ }{\pb}ähnlich ſauer. An komiſchen Werturteilen
               fehlte es nicht, Soyka\pwindex{Soyka, Otto 09.05.1881 – 02.12.1955@\textsc{Soyka, Otto} (09.05.1881 – 02.12.1955), \emph{Schriftsteller}|pw}{ }ſchimpfte mich ein Genie, Paul Ernſt\pwindex{Ernst, Paul 07.03.1866 – 13.05.1933@\textsc{Ernst, Paul} (07.03.1866 – 13.05.1933), \emph{Schriftsteller}|pw} gab zuerſt reichliches Lob von ſich, um ſchließlich
               bei dem \textsc{Cliché} »frühreifes Wien\oindex{Wien@\textbf{Wien}|pw}er Talent, das längſtens in fünf Jahren abgeſtorben ſein wird« zu enden.
               Angeſichts Ihrer Anſicht, vieles bei mir ſei noch unreif, erinnert mich dieſer
               Widerſpruch lebhaft daran, daß Auernheimer\pwindex{Auernheimer, Raoul 15.04.1876 – 06.01.1948@\textsc{Auernheimer, Raoul} (15.04.1876 – 06.01.1948), \emph{Schriftsteller, Journalist, Kritiker}|pw}
               meine Th. Mann\pwindex{Mann, Thomas 06.06.1875 – 12.08.1955@\textsc{Mann, Thomas} (06.06.1875 – 12.08.1955), \emph{Schriftsteller}|pw}-kritik dithyrambiſch nannte,
                  Polgar\pwindex{Polgar, Alfred 17.10.1873 – 24.04.1955@\textsc{Polgar, Alfred} (17.10.1873 – 24.04.1955), \emph{Schriftsteller, Journalist, Kritiker}|pw}{ }ſie für ein abſcheuliches Pamphlet erklärte, jener
               mich als phantaſtiſchen Schriftſteller rubrizierte, Großmann\pwindex{Grossmann, Stefan 19.05.1875 – 03.01.1935@\textsc{Großmann, Stefan} (19.05.1875 – 03.01.1935), \emph{Schriftsteller, Journalist}|pw}{ }ſich durch meinen Realismus abgeſtoßen fühlte. Die
               Prognoſe des D\textsuperscript{r} Ernſt\pwindex{Ernst, Paul 07.03.1866 – 13.05.1933@\textsc{Ernst, Paul} (07.03.1866 – 13.05.1933), \emph{Schriftsteller}|pw}{ }ſcheint mir \introOben{}jedenfalls\introOben{}
               unzutreffend: nach fünfjähriger Stagnation ſind mir meine lyriſchen Fähigkeiten heuer
               wiedergekehrt. Immerhin hat eine Ballade\pwindex{Ehrenstein, Albert 23.12.1886 – 08.04.1950@\textsc{Ehrenstein, Albert} (23.12.1886 – 08.04.1950), \emph{Schriftsteller}!Graf Cilli1910@\strich\emph{Graf Cilli} {[}1910{]}|pwv}, die ich im Mai fabrizierte, bereits den Rekord von
               zwölf Retournierungen. Ich möchte ſie mit einigen anderen kleinen Arbeiten {\pb}Ihnen unterbreiten: Ich halte die Sachen
               nämlich nicht für ſo ſchlecht wie die vereinigten Redaktionsphiliſter, deren
               Autogramme zu ſammeln mein Schickſal zu ſein ſcheint. Die Herren Heſſe\pwindex{Hesse, Hermann 02.07.1877 – 09.08.1962@\textsc{Hesse, Hermann} (02.07.1877 – 09.08.1962), \emph{Schriftsteller}|pw}, \label{K_L01946_3v}\edtext{Gumppenberg\pwindex{Gumppenberg, Hanns von 04.12.1866 – 29.03.1928@\textsc{Gumppenberg, Hanns von} (04.12.1866 – 29.03.1928), \emph{Schriftsteller, Kritiker}|pw}}{\lemma{\textnormal{\emph{Gumppenberg}}}\Cendnote{\textnormal{Vgl. den Brief Ehrensteins\pwindex{Ehrenstein, Albert 23.12.1886 – 08.04.1950@\textsc{Ehrenstein, Albert} (23.12.1886 – 08.04.1950), \emph{Schriftsteller}|pwk} an Hanns von Gumppenberg\pwindex{Gumppenberg, Hanns von 04.12.1866 – 29.03.1928@\textsc{Gumppenberg, Hanns von} (04.12.1866 – 29.03.1928), \emph{Schriftsteller, Kritiker}|pwk} vom
                        16. 5. 1910, abgedruckt in: A. E.\pwindex{Ehrenstein, Albert 23.12.1886 – 08.04.1950@\textsc{Ehrenstein, Albert} (23.12.1886 – 08.04.1950), \emph{Schriftsteller}|pwk}: \emph{Briefe}, S. 38.}}}\label{K_L01946_3h}, K. B. Heinrich\pwindex{Heinrich, Karl Borromaeus 22.07.1884 – 25.10.1938@\textsc{Heinrich, Karl Borromäus} (22.07.1884 – 25.10.1938), \emph{Schriftsteller}|pw}, Scheerbart\pwindex{Scheerbart, Paul 08.01.1863 – 15.10.1915@\textsc{Scheerbart, Paul} (08.01.1863 – 15.10.1915), \emph{Schriftsteller}|pw}, Lang-\pwindex{Langmann, Philipp 05.02.1862 – 22.05.1931@\textsc{Langmann, Philipp} (05.02.1862 – 22.05.1931), \emph{Schriftsteller, Journalist, Schriftsteller}|pw}, Wid-\pwindex{Widmann, Joseph Victor 20.02.1842 – 06.11.1911@\textsc{Widmann, Joseph Victor} (20.02.1842 – 06.11.1911), \emph{Schriftsteller, Journalist}|pw}, Hoff-\pwindex{Hoffmann, Camill 31.10.1878 – 01.10.1944@\textsc{Hoffmann, Camill} (31.10.1878 – 01.10.1944), \emph{Schriftsteller, Journalist}|pw} und Großmann\pwindex{Grossmann, Stefan 19.05.1875 – 03.01.1935@\textsc{Großmann, Stefan} (19.05.1875 – 03.01.1935), \emph{Schriftsteller, Journalist}|pw} behaupten einhellig eine intenſive
               Nichteignung meiner Arbeiten für Ihre reſpektiven Blätter. Bie\pwindex{Bie, Oskar 09.02.1864 – 21.04.1938@\textsc{Bie, Oskar} (09.02.1864 – 21.04.1938), \emph{Schriftsteller, Journalist, Redakteur}|pw} verwechſelt mich konſtant mit R. Auernheimer\pwindex{Auernheimer, Raoul 15.04.1876 – 06.01.1948@\textsc{Auernheimer, Raoul} (15.04.1876 – 06.01.1948), \emph{Schriftsteller, Journalist, Kritiker}|pw}, Wien III\oindex{III., Landstrasse@\textbf{III., Landstraße}|pw},
               und verlangt immer wieder duftige Wien\oindex{Wien@\textbf{Wien}|pw}er Ware, die
               ich natürlich nicht herſtellen kann. Kurz, es dürfte kein namhaftes Organ in Öſterreich\oindex{Oesterreich@\textbf{Österreich}|pw} und Deutſchland\oindex{Deutschland@\textbf{Deutschland}|pw} geben, das mich nicht mit ſeinen nichtsſagenden
               Ablehnungsformularen beglückt hätte. — Ein Herr König\pwindex{Koenig, Otto 12.05.1881 – 13.09.1955@\textsc{König, Otto} (12.05.1881 – 13.09.1955), \emph{Journalist, Volksbildner}|pw} vom »Merker\orgindex{Merker@Der Merker|pw}« möchte für den
               Spätherbſt eine kritiſche Studie über Sie, den Dramatiker, von mir haben, aber ſein
               Blatt zahlt ſpät und ſchlecht, und mit meiner Betrachtungsweiſe wäre wohl eher noch
               der Autor als der päpſtliche Merker\orgindex{Merker@Der Merker|pw}{ }{\pb}einverſtanden. Ich würde Sie nämlich,
               trotzdem Ihre Stücke oftmals von der Bühne her auf mich ſtark gewirkt haben,
               ebenſowenig einen Dramatiker nennen wie etwa Grillparzer\pwindex{Grillparzer, Franz 15.01.1791 – 21.01.1872@\textsc{Grillparzer, Franz} (15.01.1791 – 21.01.1872), \emph{Schriftsteller, Beamter}|pw} oder irgend einen anderen öſterreichiſchen\oindex{Oesterreich@\textbf{Österreich}|pw} Dichter. Ich würde ſagen, Sie ſeien im Grunde genommen ein
               Lyriker, ein Stimmungsdichter, der ſich zu\introOben{}r\introOben{}{ }\strikeout{ſeiner} Erreichung ſeiner Zwecke oft des Dialoges,
               noch häufiger der epiſchen Form bedient. »Der einſame
                  Weg\pwindex{Schnitzler, Arthur 15.05.1862 – 21.10.1931@\textsc{Schnitzler, Arthur} (15.05.1862 – 21.10.1931), \emph{Schriftsteller, Mediziner}!einsame Weg. Schauspiel in fuenf Akten1904@\strich\emph{Der einsame Weg. Schauspiel in fünf Akten} {[}1904{]}|pw}« zum Beiſpiel iſt nichts \introOben{}anderes\introOben{} als eine
               wunderſchöne, dialogiſierte Novelle, in der ebenſo wie in den ähnlichen Wahlverwandtſchaften\pwindex{Goethe, Johann Wolfgang von 1749-08-28 – 1832-03-22@\textsc{Goethe, Johann Wolfgang von} (1749-08-28 – 1832-03-22), \emph{Schriftsteller}!Wahlverwandtschaften1809@\strich\emph{Die Wahlverwandtschaften} {[}1809{]}|pw} (aber auch bei Homer\pwindex{Homer @\textsc{Homer}, \emph{Schriftsteller}|pw} und den Buddenbrooks\pwindex{Mann, Thomas 06.06.1875 – 12.08.1955@\textsc{Mann, Thomas} (06.06.1875 – 12.08.1955), \emph{Schriftsteller}!Buddenbrooks1901@\strich\emph{Buddenbrooks} {[}1901{]}|pw}) ein Ausſterben der feiner organiſierten Individuen, ein \substVorne{}\textsuperscript{Überleben}{\allowbreak}\substDazwischen{}Amlebenbleiben\substHinten{} der gangbareren Typen zu regiſtrieren iſt. Jene unerbittliche Logik, jene
               unabwendbaren Reſultate ineinanderwachſender Motive, zu denen Shakeſpeare\pwindex{Shakespeare, William 23.4.1564? – 03.05.1616@\textsc{Shakespeare, William} (23.4.1564? – 03.05.1616), \emph{Schauspieler, Schriftsteller}|pw} kam, hat von deutſchen \substVorne{}\textsuperscript{Dichtern}{\allowbreak}\substDazwischen{}Dramatikern\substHinten{} nicht einmal Kleiſt\pwindex{Kleist, Heinrich von 18.10.1777 – 21.11.1811@\textsc{Kleist, Heinrich von} (18.10.1777 – 21.11.1811), \emph{Schriftsteller}|pw}; Hebbel\pwindex{Hebbel, Friedrich 18.03.1813 – 13.12.1863@\textsc{Hebbel, Friedrich} (18.03.1813 – 13.12.1863), \emph{Schriftsteller}|pw} und Schiller\pwindex{Schiller, Friedrich von 10.11.1759 – 09.05.1805@\textsc{Schiller, Friedrich von} (10.11.1759 – 09.05.1805), \emph{Schriftsteller, Historiker, Philosoph}|pw}{ }ſind Dialektiker, {\pb}Goethe\pwindex{Goethe, Johann Wolfgang von 1749-08-28 – 1832-03-22@\textsc{Goethe, Johann Wolfgang von} (1749-08-28 – 1832-03-22), \emph{Schriftsteller}|pw} iſt – ich weiß kein höheres Lob für
               Ihren muſikaliſchen, ſtets melodiſchen Stil – Lyriker. Diejenigen Ihrer Werke, die
               auf den Einfall und Einfälle geſtellt ſind, wie die meiſten Ihrer Einakter und
               Dialoge, wüßte ich nicht zu beſprechen. Mit Mathematik befaſſe ich mich nicht gern,
               und wenn, ſo würde ich den »Reigen\pwindex{Schnitzler, Arthur 15.05.1862 – 21.10.1931@\textsc{Schnitzler, Arthur} (15.05.1862 – 21.10.1931), \emph{Schriftsteller, Mediziner}!Reigen. Zehn Dialoge1900@\strich\emph{Reigen. Zehn Dialoge} {[}1900{]}|pw}« als
               Vertreter hinſtellen und beklopfen. Behaupten, es gebräche der Compoſition an
               Vollſtändigkeit, ſei man ſchon Algebraiker genug, die Prinzipien der Combination und
               Permutation anzuwenden, hätte der Cirkus komplett ſein müſſen, die Dörfer Sodom und
               Gomorrha nicht außer Betracht bleiben dürfen.\pend
           \pstart
           Über die Vollkommenheit wieder, repräſentiert durch den »einſamen Weg\pwindex{Schnitzler, Arthur 15.05.1862 – 21.10.1931@\textsc{Schnitzler, Arthur} (15.05.1862 – 21.10.1931), \emph{Schriftsteller, Mediziner}!einsame Weg. Schauspiel in fuenf Akten1904@\strich\emph{Der einsame Weg. Schauspiel in fünf Akten} {[}1904{]}|pw}«, »großen
                  Wurſtel\pwindex{Schnitzler, Arthur 15.05.1862 – 21.10.1931@\textsc{Schnitzler, Arthur} (15.05.1862 – 21.10.1931), \emph{Schriftsteller, Mediziner}!Zum grossen Wurstel. Burleske in einem Akt23. 04. 1905@\strich\emph{Zum großen Wurstel. Burleske in einem Akt} {[}23. 04. 1905{]}|pw}\pwindex{Schnitzler, Arthur 15.05.1862 – 21.10.1931@\textsc{Schnitzler, Arthur} (15.05.1862 – 21.10.1931), \emph{Schriftsteller, Mediziner}!Zum grossen Wurstel. Burleske in einem Akt23. 04. 1905@\strich\emph{Zum großen Wurstel. Burleske in einem Akt} {[}23. 04. 1905{]}|pw}« und »Schleier der Beatrice\pwindex{Schnitzler, Arthur 15.05.1862 – 21.10.1931@\textsc{Schnitzler, Arthur} (15.05.1862 – 21.10.1931), \emph{Schriftsteller, Mediziner}!Schleier der Beatrice. Schauspiel in fuenf Akten1900-12-01@\strich\emph{Der Schleier der Beatrice. Schauspiel in fünf Akten} {[}1900-12-01{]}|pw}«
               (deſſen Helden übrigens \introOben{}\strikeout{der unlogiſchere, ſentimentalere}\introOben{}Altenberg\pwindex{Altenberg, Peter 09.03.1859 – 08.01.1919@\textsc{Altenberg, Peter} (09.03.1859 – 08.01.1919), \emph{Schriftsteller}|pw} nicht zum Selbſtmord hätten ſchreiten
               laſſen, \introOben{}bloß\introOben{} weil die Vertreterin der {\pb}\strikeout{der} Weiblichkeit von einem anderen Mann träumte) –
               über das Vollendete läßt ſich wenig ſagen. Vor allem aber gebricht es mir an
               Material, ich kenne nicht jenen Schauſpielereinakter\pwindex{Schnitzler, Arthur 15.05.1862 – 21.10.1931@\textsc{Schnitzler, Arthur} (15.05.1862 – 21.10.1931), \emph{Schriftsteller, Mediziner}!Haus Delorme. Eine Familienszene1977@\strich\emph{Das Haus Delorme. Eine Familienszene} {[}1977{]}|pwv}, der in Berlin\oindex{Berlin@\textbf{Berlin}|pw} zu
               einem \label{K_L01946_4v}\edtext{Skandal}{\lemma{\textnormal{\emph{Skandal}}}\Cendnote{\textnormal{\emph{Das Haus Delorme}\pwindex{Schnitzler, Arthur 15.05.1862 – 21.10.1931@\textsc{Schnitzler, Arthur} (15.05.1862 – 21.10.1931), \emph{Schriftsteller, Mediziner}!Haus Delorme. Eine Familienszene1977@\strich\emph{Das Haus Delorme. Eine Familienszene} {[}1977{]}|pwk} wurde kurz vor der Premiere
                  im März 1904 zurückgezogen, wobei Schnitzler\pwindex{Schnitzler, Arthur 15.05.1862 – 21.10.1931@\textsc{Schnitzler, Arthur} (15.05.1862 – 21.10.1931), \emph{Schriftsteller, Mediziner}|pwk} selbst als Grund nannte, die Schauspieler hätten ihr eigenes
                  Milieu nicht darstellen mögen (\emph{Briefe} I,488–489).}}}\label{K_L01946_4h} führte, und was mich
               noch mehr intereſſierte: ich kenne bis auf das Bruchſtück in einem \label{K_L01946_5v}\edtext{Widmungsbuche\pwindex{Widmungen zur Feier des siebzigsten Geburtstages Ferdinand von Saar s.1902-11-14@\emph{Widmungen zur Feier des siebzigsten Geburtstages Ferdinand von Saar’s.} {[}1902-11-14{]}|pwv}}{\lemma{\textnormal{\emph{Widmungsbuche}}}\Cendnote{\textnormal{Arthur Schnitzler\pwindex{Schnitzler, Arthur 15.05.1862 – 21.10.1931@\textsc{Schnitzler, Arthur} (15.05.1862 – 21.10.1931), \emph{Schriftsteller, Mediziner}|pwk}: \emph{Liebelei. Erstes Bild}\pwindex{Schnitzler, Arthur 15.05.1862 – 21.10.1931@\textsc{Schnitzler, Arthur} (15.05.1862 – 21.10.1931), \emph{Schriftsteller, Mediziner}!Liebelei. Erstes Bild1902-11-14@\strich\emph{Liebelei. Erstes Bild} {[}1902-11-14{]}|pwk}. In: \emph{Widmungen zur Feier des siebzigsten Geburtstages Ferdinand von Saar\pwindex{Saar, Ferdinand von 30.09.1833 – 24.07.1906@\textsc{Saar, Ferdinand von} (30.09.1833 – 24.07.1906), \emph{Schriftsteller}|pwk}’s}\pwindex{Widmungen zur Feier des siebzigsten Geburtstages Ferdinand von Saar s.1902-11-14@\emph{Widmungen zur Feier des siebzigsten Geburtstages Ferdinand von Saar’s.} {[}1902-11-14{]}|pwk}. Hg. Richard Specht\pwindex{Specht, Richard 07.12.1870 – 18.03.1932@\textsc{Specht, Richard} (07.12.1870 – 18.03.1932), \emph{Schriftsteller, Journalist, Kritiker}|pwk}. Buchschmuck A. F. Seligmann\pwindex{Seligmann, Adalbert Franz 02.04.1862 – 13.12.1945@\textsc{Seligmann, Adalbert Franz} (02.04.1862 – 13.12.1945), \emph{Bildender Künstler, Publizist}|pwk}. Wien: \emph{Wiener Verlag}\orgindex{Wiener Verlag@Wiener Verlag|pwk} 1903, S. 175–196.}}}\label{K_L01946_5h} die erſte
               Faſſung der »Liebelei\pwindex{Schnitzler, Arthur 15.05.1862 – 21.10.1931@\textsc{Schnitzler, Arthur} (15.05.1862 – 21.10.1931), \emph{Schriftsteller, Mediziner}!Liebelei. Schauspiel in drei Akten1895-10-09@\strich\emph{Liebelei. Schauspiel in drei Akten} {[}1895-10-09{]}|pw}« nicht, die mir in dieſer
               Form, nach dem Fragment beurteilt, viel höheren Wert zu beſitzen ſcheint. (Dieſelbe
               legere Technik fand ich in den in der »N. Fr. Preſſe\orgindex{Neue Freie Presse@Neue Freie Presse|pw}« veröffentlichten \label{K_L01946_6v}\edtext{Szenen}{\lemma{\textnormal{\emph{Szenen}}}\Cendnote{\textnormal{Arthur Schnitzler\pwindex{Schnitzler, Arthur 15.05.1862 – 21.10.1931@\textsc{Schnitzler, Arthur} (15.05.1862 – 21.10.1931), \emph{Schriftsteller, Mediziner}|pwk}: \emph{Bastei-Szene. Erste Szene des dritten Aufzuges aus der
                        dramatischen Historie: »\so{Der junge Medardus}}\pwindex{Schnitzler, Arthur 15.05.1862 – 21.10.1931@\textsc{Schnitzler, Arthur} (15.05.1862 – 21.10.1931), \emph{Schriftsteller, Mediziner}!Bastei-Szene. Erste Szene des dritten Aufzuges aus der dramatischen Historie:
                  »Der junge Medardus«27. 03. 1910@\strich\emph{Bastei-Szene. Erste Szene des dritten Aufzuges aus der dramatischen Historie: »Der junge Medardus«} {[}27. 03. 1910{]}|pwk}. In: \emph{Neue Freie Presse}\pwindex{Neue Freie Presse1864 – 1939@\emph{Neue Freie Presse} {[}1864 – 1939{]}|pwk}, Nr. 16378,
                        27. 3. 1910, S. 32–39.}}}\label{K_L01946_6h} aus dem »Medardus\pwindex{Schnitzler, Arthur 15.05.1862 – 21.10.1931@\textsc{Schnitzler, Arthur} (15.05.1862 – 21.10.1931), \emph{Schriftsteller, Mediziner}!junge Medardus. Dramatische Historie in einem Vorspiel und fuenf
                  Aufzuegen1910-10-26@\strich\emph{Der junge Medardus. Dramatische Historie in einem Vorspiel und fünf Aufzügen} {[}1910-10-26{]}|pw}« wieder, die andererſeits wieder eine
               gewiſſe und vielleicht luſtige Ähnlichkeit mit dem »Kakadu\pwindex{Schnitzler, Arthur 15.05.1862 – 21.10.1931@\textsc{Schnitzler, Arthur} (15.05.1862 – 21.10.1931), \emph{Schriftsteller, Mediziner}!gruene Kakadu. Groteske in einem Akt1. 3. 1899@\strich\emph{Der grüne Kakadu. Groteske in einem Akt} {[}1. 3. 1899{]}|pw}« beſitzen.) \textsc{Summa summarum} möchte ich ſehr
               gern ein Eſſay über Sie ſchreiben (ſchon weil ich Ihnen womöglich jedes Gefallen an
               der vorliegenden Form des »Wegs ins Freie\pwindex{Schnitzler, Arthur 15.05.1862 – 21.10.1931@\textsc{Schnitzler, Arthur} (15.05.1862 – 21.10.1931), \emph{Schriftsteller, Mediziner}!Weg ins Freie. Roman1.1.1908 – 1.6.1908@\strich\emph{Der Weg ins Freie. Roman} {[}1.1.1908 – 1.6.1908{]}|pw}«
               benehmen will), aber weder ſcheint mir {\pb}der »Merker\orgindex{Merker@Der Merker|pw}« das geeignete Blatt, noch könnte
               ich ohne einiges biographiſche und entwicklungsgeſchichtliche Material ſo ſchnell
               etwa Ihrer und meiner Würdiges zu Tage befördern. Wenigſtens kaum vor März
                  1911, denn meine Studien machen nur langſame Fortſchritte. Zwar ſind die
               geographiſch-hiſtoriſchen Arbeiten bereits approbiert, das kleine philoſophiſche
               Rigoroſum bereits hinter mir und ſo ſteht zu befürchten, daß ich im
                  Oktober zum Dr. phil. degradiert werde. Aber ich \substVorne{}\textsuperscript{fürchte,}{\allowbreak}\substDazwischen{}beſorge\substHinten{} nicht über genügend ſtarke Protektion zu verfügen, um ins Miniſterium des Unterrichts\orgindex{Ministerium fuer Unterricht@Ministerium für Unterricht|pw} oder Inneren\orgindex{Ministerium fuer Inneres@Ministerium für Inneres|pw} kommen zu können und es müßte alſo im
                  Jänner{ }ſchreckliche, überdies nicht gerade viel Chancen
               bietende Lehramtsprüfungen ablegen\pend
           \pstart
           Ihr Hochachtungsvoll und ergebenſt grüßender{\\[\baselineskip]}\spacefill\mbox{Albert Ehrenstein.}\pend
           \leftskip=0em{}
         
         \endnumbering\mylabel{h}\end{ledgroupsized}  \newcommand{\dateiname}{L01946}\newcommand{\titel}{Albert Ehrenstein an Arthur Schnitzler, 12. 7. 1910}\newcommand{\editorInnen}{Martin Anton Müller und Gerd-Hermann Susen}%% latex-leseansicht-abspann.tex
%% Abspann für die Leseansicht.
%% Der Schalter \ifkorrekturansicht ist bereits durch den Vorspann gesetzt.

%% latex-abspann.tex
%% Gemeinsamer Abspann für Korrekturansicht und Leseansicht.
%% Setzt den Schalter \ifkorrekturansicht voraus (gesetzt in den
%% einbindenden Dateien latex-korrekturansicht-abspann.tex bzw.
%% latex-leseansicht-abspann.tex).
%% ---------------------------------------------------------------

\normalsize

% Das esempio-Environment wird nur in der Leseansicht benötigt
\ifkorrekturansicht\else
\newenvironment{esempio}[3]%
{
    \vspace{1.5ex}
    \rlap{\underline{#1}}
    \par
    \setlength{\parindent}{0cm}
    \nopagebreak
    \leftskip=#2cm
    \rightskip=#3cm
}
{
    \par
}
\fi

\doendnotes{C}
\bigskip
\vfill

\clearpage

\footnotesize

\ifkorrekturansicht
  \lohead{\textsc{register}}
\fi

% theindex-Environment neu definieren ohne reledmac
\makeatletter
\renewenvironment{theindex}{%
  \ifkorrekturansicht
    \section*{\indexname}%
  \else
    \subsubsection*{Index der erwähnten Entitäten}%
  \fi
  \setlength{\parindent}{0pt}%
  \setlength{\parskip}{0pt plus 0.3pt}%
  \let\item\@idxitem
}{%
  \ifkorrekturansicht\clearpage\fi
}
\makeatother

\IfFileExists{\jobname-pw.ind}{\input{\jobname-pw.ind}}{}

% Quellenangabe nur in der Leseansicht
\ifkorrekturansicht\else
% Fallback-Definitionen, falls die .tex-Datei \titel etc. nicht gesetzt hat
\providecommand{\titel}{}
\providecommand{\editorInnen}{}
\providecommand{\dateiname}{\jobname}

\vspace{3cm}

\vfill

\footnotesize
\textsc{Quelle}: \titel. Herausgegeben von {\editorInnen}. In: \emph{Arthur Schnitzler: Briefwechsel mit Autorinnen und Autoren}.
 Digitale Edition, https://schnitzler-briefe.acdh.oeaw.ac.at/{\dateiname}.html (Stand \today)
\fi

\end{document}


      