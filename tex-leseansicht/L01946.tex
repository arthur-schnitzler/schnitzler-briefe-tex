%% latex-leseansicht-vorspann.tex
%% Vorspann für die Leseansicht.
%% Lädt die gemeinsame Datei latex-vorspann.tex mit nicht gesetztem Schalter.

\newif\ifkorrekturansicht
\korrekturansichtfalse

\input{../tex-inputs/latex-vorspann}


\section[Albert Ehrenstein an Arthur Schnitzler, 12. 7. 1910]{L01946 Albert Ehrenstein an Arthur Schnitzler, 12. 7. 1910}
\nopagebreak\mylabel{L01946v}
\rehead{ }\normalsize\beginnumbering\briefempfaengerindex{Schnitzler, Arthur@\textsc{Schnitzler, Arthur}!zzzEhrenstein, Albert@\emph{von Albert Ehrenstein}!1910-07-122@{12. 7. 1910}|(be}
\toendnotes[C]{\smallbreak\pagebreak[2]}
\correspDesc{Versand  durch Albert Ehrenstein am 12. 7. 1910 in Vrádište
\newline{}Erhalt  durch Arthur Schnitzler im Zeitraum [13. 7. 1910
                  – 17. 7. 1910?] in Wien}\toendnotes[C]{\smallbreak}
\Standort{CUL, Schnitzler, B 30.}
\physDesc{Brief, 2 Blätter, 7 Seiten, 5914 Zeichen
\newline{}Handschrift: schwarze Tinte, deutsche Kurrent
\newline{}Schnitzler: mit Bleistift beschriftet: »\textsc{Ehrenstein}« }
\buchAbdrucke{\weitereDrucke{Albert Ehrenstein: \emph{Briefe}. Herausgegeben von Hanni Mittelmann. München: \emph{Boer} 1989, S. 45–48 (Werke, 1).} }\toendnotes[C]{\smallbreak}
\pstart
           
\pstart
           {\pb}\textsc{Vradist bei Holics\oindex{Vrádište@\textbf{Vrádište}|pw},}\pend
           
\pstart
           \raggedleft{}\textsc{12. Juli 1910}\pend
           \pend
           
\pstart
           \textsc{Ungarn\oindex{Ungarn@\textbf{Ungarn}|pw}}\pend
           
\pstart{}\textsc{Hochverehrter Herr Doktor,}\pend\vspace{0.5em}
\pstart
           ich glaube, es wird, Sie vielleicht intereſſieren, wenn ich wieder einmal über meine
               literariſchen Miß- und Erfolge Nachricht gebe. Kraus\pwindex{Kraus, Karl 28.\,4.\,1874 Jičín – 12.\,6.\,1936 Wien@\textsc{Kraus, Karl} (28.\,4.\,1874 Jičín – 12.\,6.\,1936 Wien), \emph{Schriftsteller, Publizist, Schriftsteller}|pw}, mit dem ich übrigens bereits{ }ſehr{ }ſchlecht{ }ſtehe, weil wir beide, wie
               Sie wiſſen, recht unverträglich{ }ſind, hat einmal ein \label{K_L01946-1v}\edtext{Gedicht\pwindex{Ehrenstein, Albert 23.\,12.\,1886 Wien – 8.\,4.\,1950 New York City@\textsc{Ehrenstein, Albert} (23.\,12.\,1886 Wien – 8.\,4.\,1950 New York City), \emph{Schriftsteller}!Wanderers Lied@\strich\emph{Wanderers Lied}|pwv}}{\lemma{\textnormal{\emph{Gedicht}}}\Cendnote{\textnormal{Albert Ehrenstein\pwindex{Ehrenstein, Albert 23.\,12.\,1886 Wien – 8.\,4.\,1950 New York City@\textsc{Ehrenstein, Albert} (23.\,12.\,1886 Wien – 8.\,4.\,1950 New York City), \emph{Schriftsteller}|pwk}: \emph{Wanderers Lied}\pwindex{Ehrenstein, Albert 23.\,12.\,1886 Wien – 8.\,4.\,1950 New York City@\textsc{Ehrenstein, Albert} (23.\,12.\,1886 Wien – 8.\,4.\,1950 New York City), \emph{Schriftsteller}!Wanderers Lied@\strich\emph{Wanderers Lied}|pwk}. In: \emph{Die
                        Fackel}\pwindex{Fackel@\emph{Die Fackel}|pwk}, Jg. 11, Nr. 296–297, 18. 2. 1910,
                  S. 36.}}}\label{K_L01946-1} von mir gebracht, ein anderes akzeptiert, der
               honorarfeindliche Berlin\oindex{Berlin@\textbf{Berlin}, \emph{Hauptstadt}|pw}er »Sturm\orgindex{Sturm@Der Sturm|pw}« zwei minderwertige Skizzen\pwindex{Ehrenstein, Albert 23.\,12.\,1886 Wien – 8.\,4.\,1950 New York City@\textsc{Ehrenstein, Albert} (23.\,12.\,1886 Wien – 8.\,4.\,1950 New York City), \emph{Schriftsteller}!Parasiten der Parasiten@\strich\emph{Die Parasiten der Parasiten}|pwv}\pwindex{Ehrenstein, Albert 23.\,12.\,1886 Wien – 8.\,4.\,1950 New York City@\textsc{Ehrenstein, Albert} (23.\,12.\,1886 Wien – 8.\,4.\,1950 New York City), \emph{Schriftsteller}!Tod eines Seebären@\strich\emph{Tod eines Seebären}|pwv}. Im übrigen ein Debacle
               auf der ganzen Linie. Die Verlage Reiß\orgindex{Erich-Reiss-Verlag@Erich-Reiss-Verlag|pw}, Fleiſchel\orgindex{Egon Fleischel und Co.@Egon Fleischel {\kaufmannsund}  Co.|pw}, Langen\orgindex{Albert Langen@Albert Langen|pw}, v. Weber\pwindex{Weber, Hans von 22.\,4.\,1872 Dresden – 22.\,4.\,1924 München@\textsc{Weber, Hans von} (22.\,4.\,1872 Dresden – 22.\,4.\,1924 München), \emph{Verleger}|pw}\orgindex{Hyperion@Hyperion|pwv} haben meine Sachen ohne weitere Begründung refuſiert, Georg Müller\pwindex{Müller, Georg 29.\,12.\,1877 Mainz – 29.\,12.\,1917 München@\textsc{Müller, Georg} (29.\,12.\,1877 Mainz – 29.\,12.\,1917 München), \emph{Verleger}|pw} iſt trotz der Intervention der Herren Alfred Kubin\pwindex{Kubin, Alfred 10.\,4.\,1877 Litoměřice – 20.\,8.\,1959 Zwickledt@\textsc{Kubin, Alfred} (10.\,4.\,1877 Litoměřice – 20.\,8.\,1959 Zwickledt), \emph{Schriftsteller, Maler, Grafiker}|pw} und A. Halbert\pwindex{Halbert, Abraham 16.\,9.\,1881 Botoșani – 15.\,10.\,1965 Hamburg@\textsc{Halbert, Abraham} (16.\,9.\,1881 Botoșani – 15.\,10.\,1965 Hamburg), \emph{Schriftsteller, Journalist}|pw} zu einer höflichen Ablehnung geſchritten, der Inſelverlag\orgindex{Insel Verlag@Insel Verlag|pw} reagierte nach einer Empfehlung durch
                  \label{K_L01946-2v}\edtext{Paul Ernſt\pwindex{Ernst, Paul 7.\,3.\,1866 Elbingerode – 13.\,5.\,1933 Sankt Georgen an der Stiefing@\textsc{Ernst, Paul} (7.\,3.\,1866 Elbingerode – 13.\,5.\,1933 Sankt Georgen an der Stiefing), \emph{Schriftsteller}|pw}}{\lemma{\textnormal{\emph{Paul Ernst}}}\Cendnote{\textnormal{Vgl. den Brief Ehrensteins\pwindex{Ehrenstein, Albert 23.\,12.\,1886 Wien – 8.\,4.\,1950 New York City@\textsc{Ehrenstein, Albert} (23.\,12.\,1886 Wien – 8.\,4.\,1950 New York City), \emph{Schriftsteller}|pwk} an Paul Ernst\pwindex{Ernst, Paul 7.\,3.\,1866 Elbingerode – 13.\,5.\,1933 Sankt Georgen an der Stiefing@\textsc{Ernst, Paul} (7.\,3.\,1866 Elbingerode – 13.\,5.\,1933 Sankt Georgen an der Stiefing), \emph{Schriftsteller}|pwk} vom 16. 5. 1910,
                     abgedruckt in: A. E.\pwindex{Ehrenstein, Albert 23.\,12.\,1886 Wien – 8.\,4.\,1950 New York City@\textsc{Ehrenstein, Albert} (23.\,12.\,1886 Wien – 8.\,4.\,1950 New York City), \emph{Schriftsteller}|pwk}: \emph{Briefe}, S. 39.}}}\label{K_L01946-2}{ }{\pb}ähnlich{ }ſauer. An komiſchen Werturteilen
               fehlte es nicht, Soyka\pwindex{Soyka, Otto 9.\,5.\,1881 Wien – 2.\,12.\,1955 ebd.@\textsc{Soyka, Otto} (9.\,5.\,1881 Wien – 2.\,12.\,1955 ebd.), \emph{Schriftsteller}|pw}{ }ſchimpfte mich ein Genie, Paul Ernſt\pwindex{Ernst, Paul 7.\,3.\,1866 Elbingerode – 13.\,5.\,1933 Sankt Georgen an der Stiefing@\textsc{Ernst, Paul} (7.\,3.\,1866 Elbingerode – 13.\,5.\,1933 Sankt Georgen an der Stiefing), \emph{Schriftsteller}|pw} gab zuerſt reichliches Lob von{ }ſich, um{ }ſchließlich
               bei dem \textsc{Cliché} »frühreifes Wien\oindex{Wien@\textbf{Wien}, \emph{Verwaltungsgebiet}|pw}er Talent, das längſtens in fünf Jahren abgeſtorben{ }ſein wird« zu enden.
               Angeſichts Ihrer Anſicht, vieles bei mir{ }ſei noch unreif, erinnert mich dieſer
               Widerſpruch lebhaft daran, daß Auernheimer\pwindex{Auernheimer, Raoul 15.\,4.\,1876 Wien – 6.\,1.\,1948 Oakland@\textsc{Auernheimer, Raoul} (15.\,4.\,1876 Wien – 6.\,1.\,1948 Oakland), \emph{Schriftsteller, Journalist, Kritiker}|pw}
               meine Th. Mann\pwindex{Mann, Thomas 6.\,6.\,1875 Lübeck – 12.\,8.\,1955 Zürich@\textsc{Mann, Thomas} (6.\,6.\,1875 Lübeck – 12.\,8.\,1955 Zürich), \emph{Schriftsteller}|pw}-kritik dithyrambiſch nannte,
                  Polgar\pwindex{Polgar, Alfred 17.\,10.\,1873 Wien – 24.\,4.\,1955 Zürich@\textsc{Polgar, Alfred} (17.\,10.\,1873 Wien – 24.\,4.\,1955 Zürich), \emph{Schriftsteller, Journalist, Kritiker}|pw}{ }ſie für ein abſcheuliches Pamphlet erklärte, jener
               mich als phantaſtiſchen Schriftſteller rubrizierte, Großmann\pwindex{Großmann, Stefan 19.\,5.\,1875 Wien – 3.\,1.\,1935 ebd.@\textsc{Großmann, Stefan} (19.\,5.\,1875 Wien – 3.\,1.\,1935 ebd.), \emph{Schriftsteller, Journalist}|pw}{ }ſich durch meinen Realismus abgeſtoßen fühlte. Die
               Prognoſe des D\textsuperscript{r}{ }Ernſt\pwindex{Ernst, Paul 7.\,3.\,1866 Elbingerode – 13.\,5.\,1933 Sankt Georgen an der Stiefing@\textsc{Ernst, Paul} (7.\,3.\,1866 Elbingerode – 13.\,5.\,1933 Sankt Georgen an der Stiefing), \emph{Schriftsteller}|pw}{ }ſcheint mir \introOben{}jedenfalls\introOben{}
               unzutreffend: nach fünfjähriger Stagnation{ }ſind mir meine lyriſchen Fähigkeiten heuer
               wiedergekehrt. Immerhin hat eine Ballade\pwindex{Ehrenstein, Albert 23.\,12.\,1886 Wien – 8.\,4.\,1950 New York City@\textsc{Ehrenstein, Albert} (23.\,12.\,1886 Wien – 8.\,4.\,1950 New York City), \emph{Schriftsteller}!Graf Cilli@\strich\emph{Graf Cilli}|pwv}, die ich im Mai fabrizierte, bereits den Rekord von
               zwölf Retournierungen. Ich möchte{ }ſie mit einigen anderen kleinen Arbeiten {\pb}Ihnen unterbreiten: Ich halte die Sachen
               nämlich nicht für{ }ſo{ }ſchlecht wie die vereinigten Redaktionsphiliſter, deren
               Autogramme zu{ }ſammeln mein Schickſal zu{ }ſein{ }ſcheint. Die Herren Heſſe\pwindex{Hesse, Hermann 2.\,7.\,1877 Calw – 9.\,8.\,1962 Montagnola@\textsc{Hesse, Hermann} (2.\,7.\,1877 Calw – 9.\,8.\,1962 Montagnola), \emph{Schriftsteller}|pw}, \label{K_L01946-3v}\edtext{Gumppenberg\pwindex{Gumppenberg, Hanns von 4.\,12.\,1866 Landshut – 29.\,3.\,1928 München@\textsc{Gumppenberg, Hanns von} (4.\,12.\,1866 Landshut – 29.\,3.\,1928 München), \emph{Schriftsteller, Kritiker}|pw}}{\lemma{\textnormal{\emph{Gumppenberg}}}\Cendnote{\textnormal{Vgl. den Brief Ehrensteins\pwindex{Ehrenstein, Albert 23.\,12.\,1886 Wien – 8.\,4.\,1950 New York City@\textsc{Ehrenstein, Albert} (23.\,12.\,1886 Wien – 8.\,4.\,1950 New York City), \emph{Schriftsteller}|pwk} an Hanns von Gumppenberg\pwindex{Gumppenberg, Hanns von 4.\,12.\,1866 Landshut – 29.\,3.\,1928 München@\textsc{Gumppenberg, Hanns von} (4.\,12.\,1866 Landshut – 29.\,3.\,1928 München), \emph{Schriftsteller, Kritiker}|pwk} vom
                        16. 5. 1910, abgedruckt in: A. E.\pwindex{Ehrenstein, Albert 23.\,12.\,1886 Wien – 8.\,4.\,1950 New York City@\textsc{Ehrenstein, Albert} (23.\,12.\,1886 Wien – 8.\,4.\,1950 New York City), \emph{Schriftsteller}|pwk}: \emph{Briefe}, S. 38.}}}\label{K_L01946-3}, K. B. Heinrich\pwindex{Heinrich, Karl Borromäus 22.\,7.\,1884 Hangenham – 25.\,10.\,1938 Einsiedeln@\textsc{Heinrich, Karl Borromäus} (22.\,7.\,1884 Hangenham – 25.\,10.\,1938 Einsiedeln), \emph{Schriftsteller, Schriftsteller}|pw}, Scheerbart\pwindex{Scheerbart, Paul 8.\,1.\,1863 Danzig – 15.\,10.\,1915 Berlin@\textsc{Scheerbart, Paul} (8.\,1.\,1863 Danzig – 15.\,10.\,1915 Berlin), \emph{Schriftsteller}|pw}, Lang-\pwindex{Langmann, Philipp 5.\,2.\,1862 Brünn – 22.\,5.\,1931 Wien@\textsc{Langmann, Philipp} (5.\,2.\,1862 Brünn – 22.\,5.\,1931 Wien), \emph{Schriftsteller, Journalist}|pw}, Wid-\pwindex{Widmann, Joseph Victor 20.\,2.\,1842 Brněnské Ivanovice – 6.\,11.\,1911 Bern@\textsc{Widmann, Joseph Victor} (20.\,2.\,1842 Brněnské Ivanovice – 6.\,11.\,1911 Bern), \emph{Schriftsteller, Journalist}|pw}, Hoff-\pwindex{Hoffmann, Camill 31.\,10.\,1878 Kolín – 1.\,10.\,1944 Konzentrationslager Auschwitz-Birkenau@\textsc{Hoffmann, Camill} (31.\,10.\,1878 Kolín – 1.\,10.\,1944 Konzentrationslager Auschwitz-Birkenau), \emph{Schriftsteller, Journalist}|pw} und Großmann\pwindex{Großmann, Stefan 19.\,5.\,1875 Wien – 3.\,1.\,1935 ebd.@\textsc{Großmann, Stefan} (19.\,5.\,1875 Wien – 3.\,1.\,1935 ebd.), \emph{Schriftsteller, Journalist}|pw} behaupten einhellig eine intenſive
               Nichteignung meiner Arbeiten für Ihre reſpektiven Blätter. Bie\pwindex{Bie, Oskar 9.\,2.\,1864 Breslau – 21.\,4.\,1938 Berlin@\textsc{Bie, Oskar} (9.\,2.\,1864 Breslau – 21.\,4.\,1938 Berlin), \emph{Schriftsteller, Journalist, Redakteur}|pw} verwechſelt mich konſtant mit R. Auernheimer\pwindex{Auernheimer, Raoul 15.\,4.\,1876 Wien – 6.\,1.\,1948 Oakland@\textsc{Auernheimer, Raoul} (15.\,4.\,1876 Wien – 6.\,1.\,1948 Oakland), \emph{Schriftsteller, Journalist, Kritiker}|pw}, Wien III\oindex{III., Landstraße@\textbf{III., Landstraße}, \emph{Verwaltungsgebiet}|pw},
               und verlangt immer wieder duftige Wien\oindex{Wien@\textbf{Wien}, \emph{Verwaltungsgebiet}|pw}er Ware, die
               ich natürlich nicht herſtellen kann. Kurz, es dürfte kein namhaftes Organ in Öſterreich\oindex{Österreich@\textbf{Österreich}|pw} und Deutſchland\oindex{Deutschland@\textbf{Deutschland}|pw} geben, das mich nicht mit{ }ſeinen nichtsſagenden
               Ablehnungsformularen beglückt hätte. – Ein Herr König\pwindex{Koenig, Otto Martin Julius 12.\,5.\,1881 Wien – 13.\,9.\,1955 Klosterneuburg@\textsc{Koenig, Otto Martin Julius} (12.\,5.\,1881 Wien – 13.\,9.\,1955 Klosterneuburg), \emph{Journalist, Volksbildner}|pw} vom »Merker\orgindex{Merker@Der Merker|pw}« möchte für den
               Spätherbſt eine kritiſche Studie über Sie, den Dramatiker, von mir haben, aber{ }ſein
               Blatt zahlt{ }ſpät und{ }ſchlecht, und mit meiner Betrachtungsweiſe wäre wohl eher noch
               der Autor als der päpſtliche Merker\orgindex{Merker@Der Merker|pw}{ }{\pb}einverſtanden. Ich würde Sie nämlich,
               trotzdem Ihre Stücke oftmals von der Bühne her auf mich{ }ſtark gewirkt haben,
               ebenſowenig einen Dramatiker nennen wie etwa Grillparzer\pwindex{Grillparzer, Franz 15.\,1.\,1791 Wien – 21.\,1.\,1872 ebd.@\textsc{Grillparzer, Franz} (15.\,1.\,1791 Wien – 21.\,1.\,1872 ebd.), \emph{Schriftsteller, Beamter}|pw} oder irgend einen anderen öſterreichiſchen\oindex{Österreich@\textbf{Österreich}|pw} Dichter. Ich würde{ }ſagen, Sie{ }ſeien im Grunde genommen ein
               Lyriker, ein Stimmungsdichter, der{ }ſich zu\introOben{}r\introOben{}{ }\strikeout{ſeiner} Erreichung{ }ſeiner Zwecke oft des Dialoges,
               noch häufiger der epiſchen Form bedient. »Der einſame
                  Weg\pwindex{Schnitzler, Arthur 15.\,5.\,1862 Wien – 21.\,10.\,1931 ebd.@\textsc{Schnitzler, Arthur} (15.\,5.\,1862 Wien – 21.\,10.\,1931 ebd.), \emph{Schriftsteller, Mediziner}!einsame Weg. Schauspiel in fünf Akten@\strich\emph{Der einsame Weg. Schauspiel in fünf Akten}|pw}« zum Beiſpiel iſt nichts \introOben{}anderes\introOben{} als eine
               wunderſchöne, dialogiſierte Novelle, in der ebenſo wie in den ähnlichen Wahlverwandtſchaften\pwindex{Goethe, Johann Wolfgang von 28.\,8.\,1749 Frankfurt am Main – 22.\,3.\,1832 Weimar@\textsc{Goethe, Johann Wolfgang von} (28.\,8.\,1749 Frankfurt am Main – 22.\,3.\,1832 Weimar), \emph{Schriftsteller}!Wahlverwandtschaften@\strich\emph{Die Wahlverwandtschaften}|pw} (aber auch bei Homer\pwindex{Homer @\textsc{Homer}, \emph{Schriftsteller}|pw} und den Buddenbrooks\pwindex{Mann, Thomas 6.\,6.\,1875 Lübeck – 12.\,8.\,1955 Zürich@\textsc{Mann, Thomas} (6.\,6.\,1875 Lübeck – 12.\,8.\,1955 Zürich), \emph{Schriftsteller}!Buddenbrooks@\strich\emph{Buddenbrooks}|pw}) ein Ausſterben der feiner organiſierten Individuen, ein \substVorne{}\textsuperscript{Überleben}\substDazwischen{}Amlebenbleiben\substHinten{} der gangbareren Typen zu regiſtrieren iſt. Jene unerbittliche Logik, jene
               unabwendbaren Reſultate ineinanderwachſender Motive, zu denen Shakeſpeare\pwindex{Shakespeare, William 23.\,4.\,1564? Stratford-upon-Avon – 3.\,5.\,1616 ebd.@\textsc{Shakespeare, William} (23.\,4.\,1564? Stratford-upon-Avon – 3.\,5.\,1616 ebd.), \emph{Schauspieler, Dramatiker}|pw} kam, hat von deutſchen \substVorne{}\textsuperscript{Dichtern}\substDazwischen{}Dramatikern\substHinten{} nicht einmal Kleiſt\pwindex{Kleist, Heinrich von 18.\,10.\,1777 Frankfurt (Oder) – 21.\,11.\,1811 Kleiner Wannsee@\textsc{Kleist, Heinrich von} (18.\,10.\,1777 Frankfurt (Oder) – 21.\,11.\,1811 Kleiner Wannsee), \emph{Schriftsteller}|pw}; Hebbel\pwindex{Hebbel, Friedrich 18.\,3.\,1813 Wesselburen – 13.\,12.\,1863 Wien@\textsc{Hebbel, Friedrich} (18.\,3.\,1813 Wesselburen – 13.\,12.\,1863 Wien), \emph{Schriftsteller}|pw} und Schiller\pwindex{Schiller, Friedrich von 10.\,11.\,1759 Marbach am Neckar – 9.\,5.\,1805 Weimar@\textsc{Schiller, Friedrich von} (10.\,11.\,1759 Marbach am Neckar – 9.\,5.\,1805 Weimar), \emph{Schriftsteller, Historiker, Philosoph}|pw}{ }ſind Dialektiker, {\pb}Goethe\pwindex{Goethe, Johann Wolfgang von 28.\,8.\,1749 Frankfurt am Main – 22.\,3.\,1832 Weimar@\textsc{Goethe, Johann Wolfgang von} (28.\,8.\,1749 Frankfurt am Main – 22.\,3.\,1832 Weimar), \emph{Schriftsteller}|pw} iſt – ich weiß kein höheres Lob für
               Ihren muſikaliſchen,{ }ſtets melodiſchen Stil – Lyriker. Diejenigen Ihrer Werke, die
               auf den Einfall und Einfälle geſtellt{ }ſind, wie die meiſten Ihrer Einakter und
               Dialoge, wüßte ich nicht zu beſprechen. Mit Mathematik befaſſe ich mich nicht gern,
               und wenn,{ }ſo würde ich den »Reigen\pwindex{Schnitzler, Arthur 15.\,5.\,1862 Wien – 21.\,10.\,1931 ebd.@\textsc{Schnitzler, Arthur} (15.\,5.\,1862 Wien – 21.\,10.\,1931 ebd.), \emph{Schriftsteller, Mediziner}!Reigen. Zehn Dialoge@\strich\emph{Reigen. Zehn Dialoge}|pw}« als
               Vertreter hinſtellen und beklopfen. Behaupten, es gebräche der Compoſition an
               Vollſtändigkeit,{ }ſei man{ }ſchon Algebraiker genug, die Prinzipien der Combination und
               Permutation anzuwenden, hätte der Cirkus komplett{ }ſein müſſen, die Dörfer Sodom und
               Gomorrha nicht außer Betracht bleiben dürfen.\pend
           
\pstart
           Über die Vollkommenheit wieder, repräſentiert durch den »einſamen Weg\pwindex{Schnitzler, Arthur 15.\,5.\,1862 Wien – 21.\,10.\,1931 ebd.@\textsc{Schnitzler, Arthur} (15.\,5.\,1862 Wien – 21.\,10.\,1931 ebd.), \emph{Schriftsteller, Mediziner}!einsame Weg. Schauspiel in fünf Akten@\strich\emph{Der einsame Weg. Schauspiel in fünf Akten}|pw}«, »großen
                  Wurſtel\pwindex{Schnitzler, Arthur 15.\,5.\,1862 Wien – 21.\,10.\,1931 ebd.@\textsc{Schnitzler, Arthur} (15.\,5.\,1862 Wien – 21.\,10.\,1931 ebd.), \emph{Schriftsteller, Mediziner}!Zum großen Wurstel. Burleske in einem Akt@\strich\emph{Zum großen Wurstel. Burleske in einem Akt}|pw}« und »Schleier der Beatrice\pwindex{Schnitzler, Arthur 15.\,5.\,1862 Wien – 21.\,10.\,1931 ebd.@\textsc{Schnitzler, Arthur} (15.\,5.\,1862 Wien – 21.\,10.\,1931 ebd.), \emph{Schriftsteller, Mediziner}!Schleier der Beatrice. Schauspiel in fünf Akten@\strich\emph{Der Schleier der Beatrice. Schauspiel in fünf Akten}|pw}«
               (deſſen Helden übrigens \introOben{}\strikeout{der unlogiſchere,{ }ſentimentalere{ }}\introOben{}Altenberg\pwindex{Altenberg, Peter 9.\,3.\,1859 Wien – 8.\,1.\,1919 ebd.@\textsc{Altenberg, Peter} (9.\,3.\,1859 Wien – 8.\,1.\,1919 ebd.), \emph{Schriftsteller}|pw} nicht zum Selbſtmord hätten{ }ſchreiten
               laſſen, \introOben{}bloß\introOben{} weil die Vertreterin der {\pb}\strikeout{der} Weiblichkeit von einem anderen Mann träumte) –
               über das Vollendete läßt{ }ſich wenig{ }ſagen. Vor allem aber gebricht es mir an
               Material, ich kenne nicht jenen Schauſpielereinakter\pwindex{Schnitzler, Arthur 15.\,5.\,1862 Wien – 21.\,10.\,1931 ebd.@\textsc{Schnitzler, Arthur} (15.\,5.\,1862 Wien – 21.\,10.\,1931 ebd.), \emph{Schriftsteller, Mediziner}!Haus Delorme. Eine Familienszene@\strich\emph{Das Haus Delorme. Eine Familienszene}|pwv}, der in Berlin\oindex{Berlin@\textbf{Berlin}, \emph{Hauptstadt}|pw} zu
               einem \label{K_L01946-4v}\edtext{Skandal}{\lemma{\textnormal{\emph{Skandal}}}\Cendnote{\textnormal{\emph{Das Haus Delorme}\pwindex{Schnitzler, Arthur 15.\,5.\,1862 Wien – 21.\,10.\,1931 ebd.@\textsc{Schnitzler, Arthur} (15.\,5.\,1862 Wien – 21.\,10.\,1931 ebd.), \emph{Schriftsteller, Mediziner}!Haus Delorme. Eine Familienszene@\strich\emph{Das Haus Delorme. Eine Familienszene}|pwk} wurde kurz vor der Premiere
                  im März 1904 zurückgezogen, wobei Schnitzler selbst als Grund nannte, die Schauspieler hätten ihr eigenes
                  Milieu nicht darstellen mögen (\emph{Briefe 1875–1912}, S. 488–489).}}}\label{K_L01946-4} führte, und was mich
               noch mehr intereſſierte: ich kenne bis auf das Bruchſtück in einem \label{K_L01946-5v}\edtext{Widmungsbuche\pwindex{Widmungen zur Feier des siebzigsten Geburtstages Ferdinand von Saar’s.@\emph{Widmungen zur Feier des siebzigsten Geburtstages Ferdinand von Saar’s.}|pwv}}{\lemma{\textnormal{\emph{Widmungsbuche}}}\Cendnote{\textnormal{Arthur Schnitzler: \emph{Liebelei. Erstes Bild}\pwindex{Schnitzler, Arthur 15.\,5.\,1862 Wien – 21.\,10.\,1931 ebd.@\textsc{Schnitzler, Arthur} (15.\,5.\,1862 Wien – 21.\,10.\,1931 ebd.), \emph{Schriftsteller, Mediziner}!Liebelei. Erstes Bild@\strich\emph{Liebelei. Erstes Bild}|pwk}. In: \emph{Widmungen zur Feier des siebzigsten Geburtstages Ferdinand von Saar\pwindex{Saar, Ferdinand von 30.\,9.\,1833 Wien – 24.\,7.\,1906 ebd.@\textsc{Saar, Ferdinand von} (30.\,9.\,1833 Wien – 24.\,7.\,1906 ebd.), \emph{Schriftsteller}|pwk}’s}\pwindex{Widmungen zur Feier des siebzigsten Geburtstages Ferdinand von Saar’s.@\emph{Widmungen zur Feier des siebzigsten Geburtstages Ferdinand von Saar’s.}|pwk}. Herausgegeben von  Richard Specht\pwindex{Specht, Richard 7.\,12.\,1870 Wien – 18.\,3.\,1932 ebd.@\textsc{Specht, Richard} (7.\,12.\,1870 Wien – 18.\,3.\,1932 ebd.), \emph{Schriftsteller, Journalist, Kritiker}|pwk}. Buchschmuck A. F. Seligmann\pwindex{Seligmann, Adalbert Franz 2.\,4.\,1862 Wien – 13.\,12.\,1945 ebd.@\textsc{Seligmann, Adalbert Franz} (2.\,4.\,1862 Wien – 13.\,12.\,1945 ebd.), \emph{Maler, Publizist}|pwk}. Wien: \emph{Wiener Verlag}\orgindex{Wiener Verlag@Wiener Verlag|pwk}{ }1903, S. 175–196.}}}\label{K_L01946-5} die erſte
               Faſſung der »Liebelei\pwindex{Schnitzler, Arthur 15.\,5.\,1862 Wien – 21.\,10.\,1931 ebd.@\textsc{Schnitzler, Arthur} (15.\,5.\,1862 Wien – 21.\,10.\,1931 ebd.), \emph{Schriftsteller, Mediziner}!Liebelei. Schauspiel in drei Akten@\strich\emph{Liebelei. Schauspiel in drei Akten}|pw}« nicht, die mir in dieſer
               Form, nach dem Fragment beurteilt, viel höheren Wert zu beſitzen{ }ſcheint. (Dieſelbe
               legere Technik fand ich in den in der »N. Fr. Preſſe\orgindex{Neue Freie Presse@Neue Freie Presse|pw}« veröffentlichten \label{K_L01946-6v}\edtext{Szenen}{\lemma{\textnormal{\emph{Szenen}}}\Cendnote{\textnormal{Arthur Schnitzler: \emph{Bastei-Szene. Erste Szene des dritten Aufzuges aus der
                        dramatischen Historie: »\so{Der junge Medardus}}\pwindex{Schnitzler, Arthur 15.\,5.\,1862 Wien – 21.\,10.\,1931 ebd.@\textsc{Schnitzler, Arthur} (15.\,5.\,1862 Wien – 21.\,10.\,1931 ebd.), \emph{Schriftsteller, Mediziner}!Bastei-Szene. Erste Szene des dritten Aufzuges aus der dramatischen Historie: »Der junge Medardus«@\strich\emph{Bastei-Szene. Erste Szene des dritten Aufzuges aus der dramatischen Historie: »Der junge Medardus«}|pwk}. In: \emph{Neue Freie Presse}\pwindex{Neue Freie Presse@\emph{Neue Freie Presse}|pwk}, Nr. 16.378,
                        27. 3. 1910, S. 32–39.}}}\label{K_L01946-6} aus dem »Medardus\pwindex{Schnitzler, Arthur 15.\,5.\,1862 Wien – 21.\,10.\,1931 ebd.@\textsc{Schnitzler, Arthur} (15.\,5.\,1862 Wien – 21.\,10.\,1931 ebd.), \emph{Schriftsteller, Mediziner}!junge Medardus. Dramatische Historie in einem Vorspiel und fünf Aufzügen@\strich\emph{Der junge Medardus. Dramatische Historie in einem Vorspiel und fünf Aufzügen}|pw}« wieder, die andererſeits wieder eine
               gewiſſe und vielleicht luſtige Ähnlichkeit mit dem »Kakadu\pwindex{Schnitzler, Arthur 15.\,5.\,1862 Wien – 21.\,10.\,1931 ebd.@\textsc{Schnitzler, Arthur} (15.\,5.\,1862 Wien – 21.\,10.\,1931 ebd.), \emph{Schriftsteller, Mediziner}!grüne Kakadu. Groteske in einem Akt@\strich\emph{Der grüne Kakadu. Groteske in einem Akt}|pw}« beſitzen.) \textsc{Summa summarum} möchte ich{ }ſehr
               gern ein Eſſay über Sie{ }ſchreiben (ſchon weil ich Ihnen womöglich jedes Gefallen an
               der vorliegenden Form des »Wegs ins Freie\pwindex{Schnitzler, Arthur 15.\,5.\,1862 Wien – 21.\,10.\,1931 ebd.@\textsc{Schnitzler, Arthur} (15.\,5.\,1862 Wien – 21.\,10.\,1931 ebd.), \emph{Schriftsteller, Mediziner}!Weg ins Freie. Roman@\strich\emph{Der Weg ins Freie. Roman}|pw}«
               benehmen will), aber weder{ }ſcheint mir {\pb}der »Merker\orgindex{Merker@Der Merker|pw}« das geeignete Blatt, noch könnte
               ich ohne einiges biographiſche und entwicklungsgeſchichtliche Material{ }ſo{ }ſchnell
               etwa Ihrer und meiner Würdiges zu Tage befördern. Wenigſtens kaum vor März 1911, denn meine Studien machen nur langſame Fortſchritte. Zwar{ }ſind die
               geographiſch-hiſtoriſchen Arbeiten bereits approbiert, das kleine philoſophiſche
               Rigoroſum bereits hinter mir und{ }ſo{ }ſteht zu befürchten, daß ich im
                  Oktober zum Dr. phil. degradiert werde. Aber ich \substVorne{}\textsuperscript{fürchte,}\substDazwischen{}beſorge\substHinten{} nicht über genügend{ }ſtarke Protektion zu verfügen, um ins Miniſterium des Unterrichts\orgindex{Ministerium für Unterricht@Ministerium für Unterricht|pw} oder Inneren\orgindex{Ministerium für Inneres@Ministerium für Inneres|pw} kommen zu können und es müßte alſo im
                  Jänner{ }ſchreckliche, überdies nicht gerade viel Chancen
               bietende Lehramtsprüfungen ablegen\pend
           
\pstart
           Ihr Hochachtungsvoll und ergebenſt grüßender{\\[\baselineskip]}\spacefill\mbox{Albert Ehrenstein.}\pend
           \leftskip=0em{}\selectlanguage{ngerman}\endnumbering\briefempfaengerindex{Schnitzler, Arthur@\textsc{Schnitzler, Arthur}!zzzEhrenstein, Albert@\emph{von Albert Ehrenstein}!1910-07-122@{12. 7. 1910}|)be}\mylabel{L01946h}  \newcommand{\dateiname}{L01946}\newcommand{\titel}{Albert Ehrenstein an Arthur Schnitzler, 12. 7. 1910}\newcommand{\editorInnen}{Martin Anton Müller und Gerd-Hermann Susen}%% latex-leseansicht-abspann.tex
%% Abspann für die Leseansicht.
%% Der Schalter \ifkorrekturansicht ist bereits durch den Vorspann gesetzt.

%% latex-abspann.tex
%% Gemeinsamer Abspann für Korrekturansicht und Leseansicht.
%% Setzt den Schalter \ifkorrekturansicht voraus (gesetzt in den
%% einbindenden Dateien latex-korrekturansicht-abspann.tex bzw.
%% latex-leseansicht-abspann.tex).
%% ---------------------------------------------------------------

\normalsize

% Das esempio-Environment wird nur in der Leseansicht benötigt
\ifkorrekturansicht\else
\newenvironment{esempio}[3]%
{
    \vspace{1.5ex}
    \rlap{\underline{#1}}
    \par
    \setlength{\parindent}{0cm}
    \nopagebreak
    \leftskip=#2cm
    \rightskip=#3cm
}
{
    \par
}
\fi

\doendnotes{C}
\bigskip
\vfill

\clearpage

\footnotesize

\ifkorrekturansicht
  \lohead{\textsc{register}}
\fi

% theindex-Environment neu definieren ohne reledmac
\makeatletter
\renewenvironment{theindex}{%
  \ifkorrekturansicht
    \section*{\indexname}%
  \else
    \subsubsection*{Index der erwähnten Entitäten}%
  \fi
  \setlength{\parindent}{0pt}%
  \setlength{\parskip}{0pt plus 0.3pt}%
  \let\item\@idxitem
}{%
  \ifkorrekturansicht\clearpage\fi
}
\makeatother

\IfFileExists{\jobname-pw.ind}{\input{\jobname-pw.ind}}{}

% Quellenangabe nur in der Leseansicht
\ifkorrekturansicht\else
% Fallback-Definitionen, falls die .tex-Datei \titel etc. nicht gesetzt hat
\providecommand{\titel}{}
\providecommand{\editorInnen}{}
\providecommand{\dateiname}{\jobname}

\vspace{3cm}

\vfill

\footnotesize
\textsc{Quelle}: \titel. Herausgegeben von {\editorInnen}. In: \emph{Arthur Schnitzler: Briefwechsel mit Autorinnen und Autoren}.
 Digitale Edition, https://schnitzler-briefe.acdh.oeaw.ac.at/{\dateiname}.html (Stand \today)
\fi

\end{document}


