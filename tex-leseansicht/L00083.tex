%% latex-leseansicht-vorspann.tex
%% Vorspann für die Leseansicht.
%% Lädt die gemeinsame Datei latex-vorspann.tex mit nicht gesetztem Schalter.

\newif\ifkorrekturansicht
\korrekturansichtfalse

\input{../tex-inputs/latex-vorspann}


\section[Hugo von Hofmannsthal an Arthur Schnitzler, {{[}}17. 3. 1892{{]}}]{L00083 Hugo von Hofmannsthal an Arthur Schnitzler, {[}17. 3. 1892{]}}
\nopagebreak\mylabel{L00083v}
\rehead{ }\normalsize\beginnumbering\briefempfaengerindex{Schnitzler, Arthur@\textsc{Schnitzler, Arthur}!zzzHofmannsthal, Hugo von@\emph{von Hugo von Hofmannsthal}!1892-03-171@{{[}17. 3. 1892{]}}|(be}
\toendnotes[C]{\smallbreak\pagebreak[2]}
\correspDesc{Versand  durch Hugo von Hofmannsthal am [17. 3. 1892] in Wien
\newline{}Erhalt  durch Arthur Schnitzler im Zeitraum [17. 3. 1892
                  – 21. 3. 1892?] in Wien}\toendnotes[C]{\smallbreak}
\Standort{CUL, Schnitzler, B 43.}
\physDesc{Briefkarte, 545 Zeichen (aufgeprägtes Wappen)
\newline{}Handschrift: schwarze Tinte, deutsche Kurrent
\newline{}Schnitzler: mit Bleistift das Datum ergänzt: »Mitte März 92« und nummeriert: »19« }
\buchAbdrucke{\weitereDrucke{Hugo von Hofmannsthal, Arthur Schnitzler: \emph{Briefwechsel}. Herausgegeben von Therese Nickl und Heinrich Schnitzler. Frankfurt am Main: \emph{S. Fischer} 1964, S. 17.} }\toendnotes[C]{\smallbreak}
\pstart
           \raggedleft{}{\pb}Donnerstag.\pend
           \vspace{0.5em}
\pstart
           Thatſachen: 1.) Frl. Herzfeld\pwindex{Herzfeld, Marie 20.\,3.\,1855 Kőszeg – 22.\,9.\,1940 Mining@\textsc{Herzfeld, Marie} (20.\,3.\,1855 Kőszeg – 22.\,9.\,1940 Mining), \emph{Schriftstellerin, Übersetzerin}|pw}{ }ſagt mir, daſs die \textsc{Revue}\orgindex{Allgemeine Theater-Revue für Bühne und Welt. Illustrierte Halbmonatsschrift@Allgemeine Theater-Revue für Bühne und Welt. Illustrierte Halbmonatsschrift|pwv} von Fried\pwindex{Fried, Alfred Hermann 11.\,11.\,1864 Wien – 4.\,5.\,1921 ebd.@\textsc{Fried, Alfred Hermann} (11.\,11.\,1864 Wien – 4.\,5.\,1921 ebd.), \emph{Schriftsteller, Verleger, Publizist}|pw} in jeder Beziehung ernſt zu
               nehmen iſt. 2.) Wegen Schwarzkopfs\pwindex{Schwarzkopf, Gustav 7.\,11.\,1853 Wien – 13.\,11.\,1939 ebd.@\textsc{Schwarzkopf, Gustav} (7.\,11.\,1853 Wien – 13.\,11.\,1939 ebd.), \emph{Schriftsteller}|pw} Empfehlung
               an Bonz\orgindex{Adolf Bonz und Comp.@Adolf Bonz {\kaufmannsund}  Comp.|pw} müſſen wir noch{ }ſprechen.\pend
           
\pstart
           3.) Dem Bératon\pwindex{Bératon, Ferry 6.\,12.\,1859 Wien – 11.\,2.\,1900 Venedig@\textsc{Bératon, Ferry} (6.\,12.\,1859 Wien – 11.\,2.\,1900 Venedig), \emph{Schriftsteller, Journalist, Maler}|pw} werde ich{ }ſo bald als möglich
               10 fl{ }ſchicken.\pend
           
\pstart
           4.) Wäre es nicht möglich, daſs ich Sonntag um \damage{\textcolor{gray}{4}} zu Ihnen komme, daſs auch Salten\pwindex{Salten, Felix 6.\,9.\,1869 Budapest – 8.\,10.\,1945 Zürich@\textsc{Salten, Felix} (6.\,9.\,1869 Budapest – 8.\,10.\,1945 Zürich), \emph{Schriftsteller, Journalist, Chefredakteur}|pw}
               beſtimmt kommt und daſs ich Euch etwas\pwindex{Hofmannsthal, Hugo von 1.\,2.\,1874 Wien – 15.\,7.\,1929 Rodaun@\textsc{Hofmannsthal, Hugo von} (1.\,2.\,1874 Wien – 15.\,7.\,1929 Rodaun), \emph{Schriftsteller}!Tod des Tizian. Ein Bruchstück@\strich\emph{Der Tod des Tizian. Ein Bruchstück}|pwv} vorle\substVorne{}\textsuperscript{ſen}\substDazwischen{}ſe\substHinten{}, was ich zum Druck verſprochen habe, aber nicht gern ohne Euch fortſchicken
               möchte?, wenn nicht Sonntag,{ }ſo machen Sie einen anderen Vorſchlag.\pend
           
\pstart
           Herzlichſt{\\[\baselineskip]}\spacefill\mbox{Loris.}\pend
           \leftskip=0em{}
\pstart
           \noindent{}Beiliegend, danke, Nietzſche\pwindex{Nietzsche, Friedrich 15.\,10.\,1844 Röcken – 25.\,8.\,1900 Weimar@\textsc{Nietzsche, Friedrich} (15.\,10.\,1844 Röcken – 25.\,8.\,1900 Weimar), \emph{Schriftsteller, Philosoph}|pw}.\pend
           \selectlanguage{ngerman}\endnumbering\briefempfaengerindex{Schnitzler, Arthur@\textsc{Schnitzler, Arthur}!zzzHofmannsthal, Hugo von@\emph{von Hugo von Hofmannsthal}!1892-03-171@{{[}17. 3. 1892{]}}|)be}\mylabel{L00083h}  \newcommand{\dateiname}{L00083}\newcommand{\titel}{Hugo von Hofmannsthal an Arthur Schnitzler, [17. 3. 1892]}\newcommand{\editorInnen}{Martin Anton Müller und Gerd-Hermann Susen}%% latex-leseansicht-abspann.tex
%% Abspann für die Leseansicht.
%% Der Schalter \ifkorrekturansicht ist bereits durch den Vorspann gesetzt.

%% latex-abspann.tex
%% Gemeinsamer Abspann für Korrekturansicht und Leseansicht.
%% Setzt den Schalter \ifkorrekturansicht voraus (gesetzt in den
%% einbindenden Dateien latex-korrekturansicht-abspann.tex bzw.
%% latex-leseansicht-abspann.tex).
%% ---------------------------------------------------------------

\normalsize

% Das esempio-Environment wird nur in der Leseansicht benötigt
\ifkorrekturansicht\else
\newenvironment{esempio}[3]%
{
    \vspace{1.5ex}
    \rlap{\underline{#1}}
    \par
    \setlength{\parindent}{0cm}
    \nopagebreak
    \leftskip=#2cm
    \rightskip=#3cm
}
{
    \par
}
\fi

\doendnotes{C}
\bigskip
\vfill

\clearpage

\footnotesize

\ifkorrekturansicht
  \lohead{\textsc{register}}
\fi

% theindex-Environment neu definieren ohne reledmac
\makeatletter
\renewenvironment{theindex}{%
  \ifkorrekturansicht
    \section*{\indexname}%
  \else
    \subsubsection*{Index der erwähnten Entitäten}%
  \fi
  \setlength{\parindent}{0pt}%
  \setlength{\parskip}{0pt plus 0.3pt}%
  \let\item\@idxitem
}{%
  \ifkorrekturansicht\clearpage\fi
}
\makeatother

\IfFileExists{\jobname-pw.ind}{\input{\jobname-pw.ind}}{}

% Quellenangabe nur in der Leseansicht
\ifkorrekturansicht\else
% Fallback-Definitionen, falls die .tex-Datei \titel etc. nicht gesetzt hat
\providecommand{\titel}{}
\providecommand{\editorInnen}{}
\providecommand{\dateiname}{\jobname}

\vspace{3cm}

\vfill

\footnotesize
\textsc{Quelle}: \titel. Herausgegeben von {\editorInnen}. In: \emph{Arthur Schnitzler: Briefwechsel mit Autorinnen und Autoren}.
 Digitale Edition, https://schnitzler-briefe.acdh.oeaw.ac.at/{\dateiname}.html (Stand \today)
\fi

\end{document}


