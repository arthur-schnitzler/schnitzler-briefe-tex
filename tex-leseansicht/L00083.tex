%% latex-leseansicht-vorspann.tex
%% Vorspann für die Leseansicht.
%% Lädt die gemeinsame Datei latex-vorspann.tex mit nicht gesetztem Schalter.

\newif\ifkorrekturansicht
\korrekturansichtfalse

\input{../tex-inputs/latex-vorspann}


               \section[Hugo von Hofmannsthal an Arthur Schnitzler, {[}17. 3. 1892{]}]{ Hugo von Hofmannsthal an Arthur Schnitzler, {[}17. 3. 1892{]}}\nopagebreak\mylabel{v}\rehead{ }\begin{ledgroupsized}[t]{13cm}\normalsize\beginnumbering\briefempfaengerindex{Schnitzler, Arthur@\textsc{Schnitzler, Arthur}!zzzHofmannsthal, Hugo von@\emph{von Hugo von Hofmannsthal}!1892-03-171@{{[}17. 3. 1892{]}}|(be} \toendnotes[C]{\smallbreak\pagebreak[2]} \Standort{CUL, Schnitzler, B 43.}
\physDesc{Briefkarte mit aufgeprägtem Wappen
\newline{}Handschrift: schwarze Tinte, deutsche Kurrent
\newline{}Schnitzler: mit Bleistift das Datum ergänzt: »Mitte März 92« und nummeriert: »19« }\buchAbdrucke{\weitereDrucke{Hugo von Hofmannsthal, Arthur Schnitzler: \emph{Briefwechsel}. Hg. Therese Nickl und Heinrich Schnitzler. Frankfurt am Main: \emph{S. Fischer} 1964, S. 17.} }\toendnotes[C]{\smallbreak}\pstart
           \raggedleft{}{\pb}Donnerstag.\pend
           \pstart
           Thatſachen: 1.) Frl. Herzfeld\pwindex{Herzfeld, Marie 20.03.1855 – 22.09.1940@\textsc{Herzfeld, Marie} (20.03.1855 – 22.09.1940), \emph{Schriftstellerin, Übersetzerin}|pw}{ }ſagt mir, daſs die \textsc{Revue}\orgindex{Allgemeine Theater-Revue fuer Buehne und Welt@Allgemeine Theater-Revue für Bühne und Welt|pwv} von Fried\pwindex{Fried, Alfred Hermann 11.11.1864 – 04.05.1921@\textsc{Fried, Alfred Hermann} (11.11.1864 – 04.05.1921), \emph{Schriftsteller/Schriftstellerin, Verleger/Verlegerin, Publizist/Publizistin}|pw} in jeder Beziehung ernſt zu
                    nehmen iſt. 2.) Wegen Schwarzkopf\pwindex{Schwarzkopf, Gustav 07.11.1853 – 13.11.1939@\textsc{Schwarzkopf, Gustav} (07.11.1853 – 13.11.1939), \emph{Schriftsteller}|pw}s
                    Empfehlung an Bonz\orgindex{Adolf Bonz und Comp.@Adolf Bonz {\kaufmannsund}  Comp.|pw} müſſen wir noch
                    ſprechen.\pend
           \pstart
           3.) Dem Bératon\pwindex{Beraton, Ferry 06.12.1859 – 11.02.1900@\textsc{Bératon, Ferry} (06.12.1859 – 11.02.1900), \emph{Schriftsteller/Schriftstellerin, Journalist/Journalistin, Bildender Künstler/Bildende Künstlerin >> Maler/Malerin}|pw} werde ich ſo bald als möglich
                    10 fl ſchicken.\pend
           \pstart
           4.) Wäre es nicht möglich, daſs ich Sonntag um \damage{\textcolor{gray}{4}} zu Ihnen komme, daſs auch Salten\pwindex{Salten, Felix 06.09.1869 – 08.10.1945@\textsc{Salten, Felix} (06.09.1869 – 08.10.1945), \emph{Schriftsteller, Journalist}|pw}
                    beſtimmt kommt und daſs ich Euch etwas\pwindex{Hofmannsthal, Hugo von 01.02.1874 – 15.07.1929@\textsc{Hofmannsthal, Hugo von} (01.02.1874 – 15.07.1929), \emph{Schriftsteller}!Tod des Tizian1892.10@\strich\emph{Der Tod des Tizian} {[}1892.10{]}|pwv} vorle\substVorne{}\textsuperscript{ſen}\substDazwischen{}ſe\substHinten{}, was ich zum Druck verſprochen habe, aber nicht gern ohne Euch
                    fortſchicken möchte?, wenn nicht Sonntag, ſo machen Sie einen anderen
                    Vorſchlag.\pend
           \pstart
           Herzlichſt{\\[\baselineskip]}\spacefill\mbox{Loris.}\pend
           \leftskip=0em{}\pstart
           \noindent{}Beiliegend, danke, Nietzſche\pwindex{Nietzsche, Friedrich 15.10.1844 – 25.08.1900@\textsc{Nietzsche, Friedrich} (15.10.1844 – 25.08.1900), \emph{Schriftsteller, Philosoph}|pw}.\pend
                     \endnumbering\briefempfaengerindex{Schnitzler, Arthur@\textsc{Schnitzler, Arthur}!zzzHofmannsthal, Hugo von@\emph{von Hugo von Hofmannsthal}!1892-03-171@{{[}17. 3. 1892{]}}|)be}\mylabel{h}\end{ledgroupsized}  \newcommand{\dateiname}{L00083}\newcommand{\titel}{Hugo von Hofmannsthal an Arthur Schnitzler, [17. 3. 1892]}\newcommand{\editorInnen}{Martin Anton Müller und Gerd-Hermann Susen}%% latex-leseansicht-abspann.tex
%% Abspann für die Leseansicht.
%% Der Schalter \ifkorrekturansicht ist bereits durch den Vorspann gesetzt.

%% latex-abspann.tex
%% Gemeinsamer Abspann für Korrekturansicht und Leseansicht.
%% Setzt den Schalter \ifkorrekturansicht voraus (gesetzt in den
%% einbindenden Dateien latex-korrekturansicht-abspann.tex bzw.
%% latex-leseansicht-abspann.tex).
%% ---------------------------------------------------------------

\normalsize

% Das esempio-Environment wird nur in der Leseansicht benötigt
\ifkorrekturansicht\else
\newenvironment{esempio}[3]%
{
    \vspace{1.5ex}
    \rlap{\underline{#1}}
    \par
    \setlength{\parindent}{0cm}
    \nopagebreak
    \leftskip=#2cm
    \rightskip=#3cm
}
{
    \par
}
\fi

\doendnotes{C}
\bigskip
\vfill

\clearpage

\footnotesize

\ifkorrekturansicht
  \lohead{\textsc{register}}
\fi

% theindex-Environment neu definieren ohne reledmac
\makeatletter
\renewenvironment{theindex}{%
  \ifkorrekturansicht
    \section*{\indexname}%
  \else
    \subsubsection*{Index der erwähnten Entitäten}%
  \fi
  \setlength{\parindent}{0pt}%
  \setlength{\parskip}{0pt plus 0.3pt}%
  \let\item\@idxitem
}{%
  \ifkorrekturansicht\clearpage\fi
}
\makeatother

\IfFileExists{\jobname-pw.ind}{\input{\jobname-pw.ind}}{}

% Quellenangabe nur in der Leseansicht
\ifkorrekturansicht\else
% Fallback-Definitionen, falls die .tex-Datei \titel etc. nicht gesetzt hat
\providecommand{\titel}{}
\providecommand{\editorInnen}{}
\providecommand{\dateiname}{\jobname}

\vspace{3cm}

\vfill

\footnotesize
\textsc{Quelle}: \titel. Herausgegeben von {\editorInnen}. In: \emph{Arthur Schnitzler: Briefwechsel mit Autorinnen und Autoren}.
 Digitale Edition, https://schnitzler-briefe.acdh.oeaw.ac.at/{\dateiname}.html (Stand \today)
\fi

\end{document}


      