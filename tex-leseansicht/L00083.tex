%% latex-korrekturansicht-vorspann.tex
%% Vorspann für die Korrekturansicht.
%% Lädt die gemeinsame Datei latex-vorspann.tex mit gesetztem Schalter.

\newif\ifkorrekturansicht
\korrekturansichttrue

\input{../tex-inputs/latex-vorspann}


\section[Hugo von Hofmannsthal an Arthur Schnitzler, {[}17. 3. 1892{]}]{L00083 Hugo von Hofmannsthal an Arthur Schnitzler, {[}17. 3. 1892{]}}
\nopagebreak\mylabel{L00083v}
\rehead{ }\normalsize\beginnumbering\briefempfaengerindex{Schnitzler, Arthur@\textsc{Schnitzler, Arthur}!zzzHofmannsthal, Hugo von@\emph{von Hugo von Hofmannsthal}!1892-03-171@{{[}17. 3. 1892{]}}|(be}
\toendnotes[C]{\smallbreak\pagebreak[2]}\Standort{CUL, Schnitzler, B 43.}
\physDesc{Briefkarte, 545 Zeichen (aufgeprägtes Wappen)
\newline{}Handschrift: schwarze Tinte, deutsche Kurrent
\newline{}Schnitzler: mit Bleistift das Datum ergänzt: »Mitte März 92« und nummeriert: »19« }
\buchAbdrucke{\weitereDrucke{Hugo von Hofmannsthal, Arthur Schnitzler: \emph{Briefwechsel}. Frankfurt am Main: \emph{S. Fischer} 1964, S. 17.} }\toendnotes[C]{\smallbreak}
\pstart
           \raggedleft{}{\pb}Donnerstag.\pend
           \vspace{0.5em}
\pstart
           Thatſachen: 1.) Frl. Herzfeld\pwindex{Herzfeld, Marie 20.03.1855 – 22.09.1940@\textsc{Herzfeld, Marie} (20.03.1855 – 22.09.1940), \emph{Schriftsteller/Schriftstellerin, Übersetzer/Übersetzerin}|pw}{ }ſagt mir, daſs die \textsc{Revue}\orgindex{Allgemeine Theater-Revue fuer Buehne und Welt. Illustrierte Halbmonatsschrift@Allgemeine Theater-Revue für Bühne und Welt. Illustrierte Halbmonatsschrift|pwv} von Fried\pwindex{Fried, Alfred Hermann 11.11.1864 – 04.05.1921@\textsc{Fried, Alfred Hermann} (11.11.1864 – 04.05.1921), \emph{Schriftsteller/Schriftstellerin, Verleger/Verlegerin, Publizist/Publizistin}|pw} in jeder Beziehung ernſt zu
               nehmen iſt. 2.) Wegen Schwarzkopfs\pwindex{Schwarzkopf, Gustav 07.11.1853 – 13.11.1939@\textsc{Schwarzkopf, Gustav} (07.11.1853 – 13.11.1939), \emph{Schriftsteller/Schriftstellerin}|pw} Empfehlung
               an Bonz\orgindex{Adolf Bonz und Comp.@Adolf Bonz {\kaufmannsund}  Comp.|pw} müſſen wir noch ſprechen.\pend
           
\pstart
           3.) Dem Bératon\pwindex{Beraton, Ferry 06.12.1859 – 11.02.1900@\textsc{Bératon, Ferry} (06.12.1859 – 11.02.1900), \emph{Schriftsteller/Schriftstellerin, Journalist/Journalistin, Maler/Malerin}|pw} werde ich ſo bald als möglich
               10 fl ſchicken.\pend
           
\pstart
           4.) Wäre es nicht möglich, daſs ich Sonntag um \damage{\textcolor{gray}{4}} zu Ihnen komme, daſs auch Salten\pwindex{Salten, Felix 06.09.1869 – 08.10.1945@\textsc{Salten, Felix} (06.09.1869 – 08.10.1945), \emph{Schriftsteller/Schriftstellerin, Journalist/Journalistin, Chefredakteur/Chefredakteurin}|pw}
               beſtimmt kommt und daſs ich Euch etwas\pwindex{Tod des Tizian. Ein Bruchstueck@\emph{Der Tod des Tizian. Ein Bruchstück}|pwv} vorle\substVorne{}\textsuperscript{ſen}\substDazwischen{}ſe\substHinten{}, was ich zum Druck verſprochen habe, aber nicht gern ohne Euch fortſchicken
               möchte?, wenn nicht Sonntag, ſo machen Sie einen anderen Vorſchlag.\pend
           
\pstart
           Herzlichſt{\\[\baselineskip]}\spacefill\mbox{Loris.}\pend
           \leftskip=0em{}
\pstart
           \noindent{}Beiliegend, danke, Nietzſche\pwindex{Nietzsche, Friedrich 15.10.1844 – 25.08.1900@\textsc{Nietzsche, Friedrich} (15.10.1844 – 25.08.1900), \emph{Schriftsteller/Schriftstellerin, Philosoph/Philosophin}|pw}.\pend
           \selectlanguage{ngerman}\endnumbering\briefempfaengerindex{Schnitzler, Arthur@\textsc{Schnitzler, Arthur}!zzzHofmannsthal, Hugo von@\emph{von Hugo von Hofmannsthal}!1892-03-171@{{[}17. 3. 1892{]}}|)be}\mylabel{L00083h}  \normalsize

\doendnotes{C}
\bigskip
\vfill

\clearpage

\footnotesize

\lohead{\textsc{register}}

% Definiere theindex-Environment komplett neu ohne reledmac
\makeatletter
\renewenvironment{theindex}{%
  \section*{\indexname}%
  \setlength{\parindent}{0pt}%
  \setlength{\parskip}{0pt plus 0.3pt}%
  \let\item\@idxitem
}{%
  \clearpage
}
\makeatother

\IfFileExists{\jobname-pw.ind}{\input{\jobname-pw.ind}}{}

\end{document}

      