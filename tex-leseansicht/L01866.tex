%% latex-korrekturansicht-vorspann.tex
%% Vorspann für die Korrekturansicht.
%% Lädt die gemeinsame Datei latex-vorspann.tex mit gesetztem Schalter.

\newif\ifkorrekturansicht
\korrekturansichttrue

\input{../tex-inputs/latex-vorspann}


\section[Richard Beer-Hofmann an Arthur Schnitzler, 21. 8. 1909]{L01866 Richard Beer-Hofmann an Arthur Schnitzler, 21. 8. 1909}
\nopagebreak\mylabel{L01866v}
\rehead{ }\normalsize\beginnumbering\briefempfaengerindex{Schnitzler, Arthur@\textsc{Schnitzler, Arthur}!zzzBeer-Hofmann, Richard@\emph{von Richard Beer-Hofmann}!1909-08-211@{21. 8. 1909}|(be}
\toendnotes[C]{\smallbreak\pagebreak[2]}\Standort{CUL, Schnitzler, B 8.}
\physDesc{Bildpostkarte, 72 Zeichen
\newline{}Handschrift: schwarze Tinte, lateinische Kurrent
\newline{}Versand: 1) Stempel: »\nobreak{}\oindex{Venedig@\textbf{Venedig}, \emph{P.PPLA}|pwk}S Elisabetta di Lido
                                       (Venezia), 21 8 {[}1909{]}, 2S\nobreak{}«.   2) Stempel: »\nobreak{}\oindex{Edlach@\textbf{Edlach}, \emph{P.PPL}|pwk}Ed{[}lach{]} b.
                                       Reichenau, 25. VIII. 09, XII\nobreak{}«.  3) mit schwarzer Tinte von unbekannter Hand die Ortsangabe gestrichen und
                                 ersetzt durch: »\noindent{}\textsc{Wien}\oindex{Wien@\textbf{Wien}, \emph{A.ADM2}|pw},{ / }\textsc{XIII Spoettelgasse}\oindex{Edmund-Weiss-Gasse 7@\textbf{Edmund-Weiß-Gasse 7}, \emph{Wohngebäude (K.WHS)}|pw}«
\newline{}Schnitzler: mit Bleistift beschriftet: »\textsc{Beerh.}« 
\newline{}Ordnung: mit Bleistift von unbekannter Hand nummeriert:
                                    »222« }\pstart{}{\pb}Herrn\pend{}\pstart{}D\textsuperscript{r} Arthur Schnitzler\pend{}\pstart{}Edlach\oindex{Edlach@\textbf{Edlach}, \emph{P.PPL}|pw}\pend{}\pstart{} bei Reichenau\oindex{Reichenau an der Rax@\textbf{Reichenau an der Rax}, \emph{A.ADM3}|pw}\pend{}\pstart{}Austria\oindex{Oesterreich@\textbf{Österreich}, \emph{A.PCLI}|pw}\pend{}{\bigskip}
\pstart
           \noindent{}\centering{}{\pb}\textcolor{gray}{\textbf{Venezia\oindex{Venedig@\textbf{Venedig}, \emph{P.PPLA}|pw} – Panorama dal Campanile di S. Giorgio Maggiore\oindex{San Giorgio Maggiore@\textbf{San Giorgio Maggiore}, \emph{Kirche (K.KRC)}|pw}}}\pend
           \vspace{1em}
\pstart
           \noindent{}{\pb}Herzliche Grüsse\pend
           \pstart \spacefill\mbox{Richard}\pend{}\selectlanguage{ngerman}\endnumbering\briefempfaengerindex{Schnitzler, Arthur@\textsc{Schnitzler, Arthur}!zzzBeer-Hofmann, Richard@\emph{von Richard Beer-Hofmann}!1909-08-211@{21. 8. 1909}|)be}\mylabel{L01866h}  \normalsize

\doendnotes{C}
\bigskip
\vfill

\clearpage

\footnotesize

\lohead{\textsc{register}}

% Definiere theindex-Environment komplett neu ohne reledmac
\makeatletter
\renewenvironment{theindex}{%
  \section*{\indexname}%
  \setlength{\parindent}{0pt}%
  \setlength{\parskip}{0pt plus 0.3pt}%
  \let\item\@idxitem
}{%
  \clearpage
}
\makeatother

\IfFileExists{\jobname-pw.ind}{\input{\jobname-pw.ind}}{}

\end{document}

      