%% latex-korrekturansicht-vorspann.tex
%% Vorspann für die Korrekturansicht.
%% Lädt die gemeinsame Datei latex-vorspann.tex mit gesetztem Schalter.

\newif\ifkorrekturansicht
\korrekturansichttrue

\input{../tex-inputs/latex-vorspann}


\section[Hermann Bahr an Arthur Schnitzler, 16. 3. 1904]{L01385 Hermann Bahr an Arthur Schnitzler, 16. 3. 1904}
\nopagebreak\mylabel{L01385v}
\rehead{ }\normalsize\beginnumbering\briefempfaengerindex{Schnitzler, Arthur@\textsc{Schnitzler, Arthur}!zzzBahr, Hermann@\emph{von Hermann Bahr}!1904-03-161@{16. 3. 1904}|(be}
\toendnotes[C]{\smallbreak\pagebreak[2]}\Standort{CUL, Schnitzler, B 5b.}
\physDesc{Bildpostkarte, 130 Zeichen
\newline{}Handschrift: Bleistift, deutsche Kurrent
\newline{}Versand: Stempel: »\nobreak{}\oindex{Athen@\textbf{Athen}, \emph{P.PPLC}|pwk}\griechisch{ΑΘHNAI}, \griechisch{16 ΜAΡΤ. 1904}\nobreak{}«.  
\newline{}Schnitzler: mit Bleistift datiert: »März 90\textcolor{gray}{4}« 
\newline{}Ordnung: mit Bleistift von unbekannter Hand nummeriert:
                                    »114« }
\buchAbdrucke{\weitereDrucke{Hermann Bahr, Arthur Schnitzler: \emph{Briefwechsel, Aufzeichnungen, Dokumente (1891–1931)}. Göttingen: \emph{Wallstein} 2018, S. 305.} }\toendnotes[C]{\smallbreak}\pstart{}{\pb}\griechisch{πρός τόν κύριον}\pend{}\pstart{}\textsc{D\textsuperscript{r} Arthur Schnitzler}\pend{}\pstart{}\griechisch{Βιέννη}\oindex{Wien@\textbf{Wien}, \emph{A.ADM2}|pw}\pend{}\pstart{}\textsc{Wien XVIII}\oindex{XVIII., Waehring@\textbf{XVIII., Währing}, \emph{A.ADM3}|pw}\pend{}\pstart{}\textsc{Spöttelgasse 7}\oindex{Edmund-Weiss-Gasse 7@\textbf{Edmund-Weiß-Gasse 7}, \emph{Wohngebäude (K.WHS)}|pw}\pend{}\pstart{}\griechisch{Αυστρία}\oindex{Oesterreich@\textbf{Österreich}, \emph{A.PCLI}|pw}\pend{}{\bigskip}
\pstart
           \noindent{}\centering{}{\pb}\textcolor{gray}{\textbf{Athène\oindex{Athen@\textbf{Athen}, \emph{P.PPLC}|pw}. \begin{otherlanguage}{french}Temple de
                     Thesée\end{otherlanguage}{ }\griechisch{Θησείον}}}\pend
           \vspace{1em}
\pstart
           {\pb}Herzlichſt{\\[\baselineskip]}mit vielen Grüßen an Deine Frau\pwindex{Schnitzler, Olga 17.01.1882 – 13.01.1970@\textsc{Schnitzler, Olga} (17.01.1882 – 13.01.1970), \emph{Schauspieler/Schauspielerin, Sänger/Sängerin}|pwv}{\\[\baselineskip]}\spacefill\mbox{Herm}\pend
           \leftskip=0em{}
\pstart
           \noindent{}Du mußt hieher.\pend
           \selectlanguage{ngerman}\endnumbering\briefempfaengerindex{Schnitzler, Arthur@\textsc{Schnitzler, Arthur}!zzzBahr, Hermann@\emph{von Hermann Bahr}!1904-03-161@{16. 3. 1904}|)be}\mylabel{L01385h}  \normalsize

\doendnotes{C}
\bigskip
\vfill

\clearpage

\footnotesize

\lohead{\textsc{register}}

% Definiere theindex-Environment komplett neu ohne reledmac
\makeatletter
\renewenvironment{theindex}{%
  \section*{\indexname}%
  \setlength{\parindent}{0pt}%
  \setlength{\parskip}{0pt plus 0.3pt}%
  \let\item\@idxitem
}{%
  \clearpage
}
\makeatother

\IfFileExists{\jobname-pw.ind}{\input{\jobname-pw.ind}}{}

\end{document}

      