%% latex-leseansicht-vorspann.tex
%% Vorspann für die Leseansicht.
%% Lädt die gemeinsame Datei latex-vorspann.tex mit nicht gesetztem Schalter.

\newif\ifkorrekturansicht
\korrekturansichtfalse

\input{../tex-inputs/latex-vorspann}

\begin{center}
            \textcolor{red}{ENTWURF. ENTZIFFERUNG NOCH NICHT KORREKTURGELESEN}
                      \end{center}
            
               \section[Arthur Schnitzler an Richard Beer-Hofmann, 17. 8. 1901]{ Arthur Schnitzler an Richard Beer-Hofmann, 17. 8. 1901}\nopagebreak\mylabel{v}\rehead{ }\begin{ledgroupsized}[t]{13cm}\normalsize\beginnumbering\briefempfaengerindex{Beer-Hofmann, Richard@\textsc{Beer-Hofmann, Richard}!zzzSchnitzler, Arthur@\emph{von Arthur Schnitzler}!1901-08-171@{17. 8. 1901}|(be} \toendnotes[C]{\smallbreak\pagebreak[2]} \Standort{YCGL, MSS 31.}
\physDesc{Brief, 1 Blatt, 3 Seiten, Umschlag
\newline{}Handschrift: Bleistift, deutsche Kurrent\newline{}Versand: 1) Stempel: »\nobreak{}\oindex{Welsberg-Taisten@\textbf{Welsberg-Taisten}|pwk}Welsberg, 17. 8. 01\nobreak{}«.  2) Stempel: »\nobreak{}\oindex{Wildbad Waldbrunn@\textbf{Wildbad Waldbrunn}|pwk}{\pb}Grand Hôtel Wildbad
                              Waldbrunn Pusterthal, 17 AUG 19\textcolor{gray}{01}\nobreak{}«. 3) Stempel: »\nobreak{}\oindex{Poertschach@\textbf{Pörtschach}|pwk}Pörtschach {[}am See{]}, 18 {[}8 01{]}\nobreak{}«. }\buchAbdrucke{\weitereDrucke{Arthur Schnitzler, Richard Beer-Hofmann: \emph{Briefwechsel 1891–1931}. Hg. Konstanze Fliedl. Wien, Zürich: \emph{Europaverlag} 1992, S. 154–155.} }\toendnotes[C]{\smallbreak}\pstart{}{\pb}\textsc{Dr. Richard Beer-Hofma{\geminationn}}\pend{}\pstart{}\textsc{Pörtschach\oindex{Poertschach@\textbf{Pörtschach}|pw}}\pend{}\pstart{}\textsc{Villa Arnstein\oindex{Villa Arnstein@\textbf{Villa Arnstein}|pw}.}\pend{}{\bigskip}\pstart
           \raggedleft{}{\pb}\textsc{Welsberg, Waldbrunn}\oindex{Wildbad Waldbrunn@\textbf{Wildbad Waldbrunn}|pw}{\\}17. 8. 901\pend
           \pstart
           mein lieber Richard, ſeit vorgeſtern bin ich hier u finde es
               unverſtändlich, dſs dieſer Ort nicht populärer iſt: \textsc{Waldbrunn}\oindex{Wildbad Waldbrunn@\textbf{Wildbad Waldbrunn}|pw} liegt eine ¼ Std höher als \textsc{Welsberg}\oindex{Welsberg-Taisten@\textbf{Welsberg-Taisten}|pw}, hat einen ſchönen Ausblick und gleich hinter dem Hotel (Penſion 3.50 alles
               wirklich gut) herrlichen Wald. Paul\pwindex{Goldmann, Paul 31.01.1865 – 25.09.1935@\textsc{Goldmann, Paul} (31.01.1865 – 25.09.1935), \emph{Schriftsteller, Journalist}|pw} iſt noch am
                  Gardaſee\oindex{Lago di Garda@\textbf{Lago di Garda}|pw} und ko{\geminationm}t
               morgen. Es hätte keinen Sinn, wenn Sie nur auf ein paar Stunden {\pb}kämen; würden Sie ſich aber zu einem längern
               Aufenthalt (6–8 Tage) entſchließen, ſo würde ich auch meinen Aufenthalt verlängern.
               Unter andern Umſtänden führe ich in etwa 10 Tagen von hier ab; ich würde Sie dann in
                  Pörtſchach\oindex{Poertschach@\textbf{Pörtschach}|pw} beſuchen (mit Paul\pwindex{Goldmann, Paul 31.01.1865 – 25.09.1935@\textsc{Goldmann, Paul} (31.01.1865 – 25.09.1935), \emph{Schriftsteller, Journalist}|pw} denk ich) oder wir treffen uns in Villach\oindex{Villach@\textbf{Villach}|pw}? {\pb}Aber das weitaus
               ſympathiſcheſte wäre doch, we{\geminationn} Sie hieherkämen, die
               beiden jungen Damen\pwindex{Schnitzler, Olga 17.01.1882 – 13.01.1970@\textsc{Schnitzler, Olga} (17.01.1882 – 13.01.1970), \emph{Schauspielerin, Sängerin}|pwv}\pwindex{Steinrueck, Elisabeth 19.11.1885 – 07.04.1920@\textsc{Steinrück, Elisabeth} (19.11.1885 – 07.04.1920)|pwv}, die
               mit mir zugleich hier ſind, würden Sie gewiſs nicht ſtören.\pend
           \pstart
           Jedenfalls ſchreiben Sie mir gleich ein Wort hieher.\pend
           \pstart
           Von \textsc{Kerr}\pwindex{Kerr, Alfred 25.12.1867 – 12.10.1948@\textsc{Kerr, Alfred} (25.12.1867 – 12.10.1948), \emph{Schriftsteller, Kritiker}|pw} hab ich keine Nachricht.\pend
           \pstart
           Von Herzen{\\[\baselineskip]}Ihr\spacefill\mbox{Arthur}\pend
           \leftskip=0em{}\endnumbering\briefempfaengerindex{Beer-Hofmann, Richard@\textsc{Beer-Hofmann, Richard}!zzzSchnitzler, Arthur@\emph{von Arthur Schnitzler}!1901-08-171@{17. 8. 1901}|)be}\mylabel{h}\end{ledgroupsized}  \newcommand{\dateiname}{L01162}\newcommand{\titel}{Arthur Schnitzler an Richard Beer-Hofmann, 17. 8. 1901}\newcommand{\editorInnen}{Martin Anton Müller und Gerd-Hermann Susen}%% latex-leseansicht-abspann.tex
%% Abspann für die Leseansicht.
%% Der Schalter \ifkorrekturansicht ist bereits durch den Vorspann gesetzt.

%% latex-abspann.tex
%% Gemeinsamer Abspann für Korrekturansicht und Leseansicht.
%% Setzt den Schalter \ifkorrekturansicht voraus (gesetzt in den
%% einbindenden Dateien latex-korrekturansicht-abspann.tex bzw.
%% latex-leseansicht-abspann.tex).
%% ---------------------------------------------------------------

\normalsize

% Das esempio-Environment wird nur in der Leseansicht benötigt
\ifkorrekturansicht\else
\newenvironment{esempio}[3]%
{
    \vspace{1.5ex}
    \rlap{\underline{#1}}
    \par
    \setlength{\parindent}{0cm}
    \nopagebreak
    \leftskip=#2cm
    \rightskip=#3cm
}
{
    \par
}
\fi

\doendnotes{C}
\bigskip
\vfill

\clearpage

\footnotesize

\ifkorrekturansicht
  \lohead{\textsc{register}}
\fi

% theindex-Environment neu definieren ohne reledmac
\makeatletter
\renewenvironment{theindex}{%
  \ifkorrekturansicht
    \section*{\indexname}%
  \else
    \subsubsection*{Index der erwähnten Entitäten}%
  \fi
  \setlength{\parindent}{0pt}%
  \setlength{\parskip}{0pt plus 0.3pt}%
  \let\item\@idxitem
}{%
  \ifkorrekturansicht\clearpage\fi
}
\makeatother

\IfFileExists{\jobname-pw.ind}{\input{\jobname-pw.ind}}{}

% Quellenangabe nur in der Leseansicht
\ifkorrekturansicht\else
% Fallback-Definitionen, falls die .tex-Datei \titel etc. nicht gesetzt hat
\providecommand{\titel}{}
\providecommand{\editorInnen}{}
\providecommand{\dateiname}{\jobname}

\vspace{3cm}

\vfill

\footnotesize
\textsc{Quelle}: \titel. Herausgegeben von {\editorInnen}. In: \emph{Arthur Schnitzler: Briefwechsel mit Autorinnen und Autoren}.
 Digitale Edition, https://schnitzler-briefe.acdh.oeaw.ac.at/{\dateiname}.html (Stand \today)
\fi

\end{document}


      