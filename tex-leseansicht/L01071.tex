%% latex-leseansicht-vorspann.tex
%% Vorspann für die Leseansicht.
%% Lädt die gemeinsame Datei latex-vorspann.tex mit nicht gesetztem Schalter.

\newif\ifkorrekturansicht
\korrekturansichtfalse

\input{../tex-inputs/latex-vorspann}


\section[Richard Beer-Hofmann an Arthur Schnitzler, 6. 9. 1900]{L01071 Richard Beer-Hofmann an Arthur Schnitzler, 6. 9. 1900}
\nopagebreak\mylabel{L01071v}
\rehead{ }\normalsize\beginnumbering\briefempfaengerindex{Schnitzler, Arthur@\textsc{Schnitzler, Arthur}!zzzBeer-Hofmann, Richard@\emph{von Richard Beer-Hofmann}!1900-09-061@{6. 9. 1900}|(be}
\toendnotes[C]{\smallbreak\pagebreak[2]}
\correspDesc{Versand  durch Richard Beer-Hofmann am 6. 9. 1900 in Altaussee
\newline{}Erhalt  durch Arthur Schnitzler im Zeitraum [7. 9. 1900
                  – 11. 9. 1900?] in Wien}\toendnotes[C]{\smallbreak}
\Standort{CUL, Schnitzler, B 8.}
\physDesc{Brief, 1 Blatt, 1 Seite, 214 Zeichen
\newline{}Handschrift: Bleistift, lateinische Kurrent
\newline{}Ordnung: mit Bleistift von unbekannter Hand nummeriert:
                                    »158« }\toendnotes[C]{\smallbreak}
\pstart
           \raggedleft{}{\pb}Alt-Aussee\oindex{Altaussee@\textbf{Altaussee}, \emph{Verwaltungsgebiet}|pw}{ }6/IX 1900\pend
           \vspace{0.5em}
\pstart
           Lieber Arthur! In Eile: Ich bleibe noch bis ungefähr
                  18. hier u. ko{\geminationm}e dann nach Wien\oindex{Wien@\textbf{Wien}, \emph{Verwaltungsgebiet}|pw}. Dh: ich \uline{will}
               das thun. Werde ich \label{K_L01071-1v}\edtext{Paul\pwindex{Goldmann, Paul 31.\,1.\,1865 Breslau – 25.\,9.\,1935 Wien@\textsc{Goldmann, Paul} (31.\,1.\,1865 Breslau – 25.\,9.\,1935 Wien), \emph{Schriftsteller, Journalist}|pw} dann noch in Wien\oindex{Wien@\textbf{Wien}, \emph{Verwaltungsgebiet}|pw} treffen}{\lemma{\textnormal{\emph{Paul … treffen}}}\Cendnote{\textnormal{Sie
                  trafen sich nicht mehr, vgl. XXXX Auszeichnungsfehler: Dokument L02931 nicht gefunden.}}}\label{K_L01071-1}?\pend
           
\pstart
           Schreiben Sie mir, bitte, zwei Zeilen.\pend
           \pstart Von Herzen Ihr \spacefill\mbox{R.}\pend{}\selectlanguage{ngerman}\endnumbering\briefempfaengerindex{Schnitzler, Arthur@\textsc{Schnitzler, Arthur}!zzzBeer-Hofmann, Richard@\emph{von Richard Beer-Hofmann}!1900-09-061@{6. 9. 1900}|)be}\mylabel{L01071h}  \newcommand{\dateiname}{L01071}\newcommand{\titel}{Richard Beer-Hofmann an Arthur Schnitzler, 6. 9. 1900}\newcommand{\editorInnen}{Martin Anton Müller und Gerd-Hermann Susen}%% latex-leseansicht-abspann.tex
%% Abspann für die Leseansicht.
%% Der Schalter \ifkorrekturansicht ist bereits durch den Vorspann gesetzt.

%% latex-abspann.tex
%% Gemeinsamer Abspann für Korrekturansicht und Leseansicht.
%% Setzt den Schalter \ifkorrekturansicht voraus (gesetzt in den
%% einbindenden Dateien latex-korrekturansicht-abspann.tex bzw.
%% latex-leseansicht-abspann.tex).
%% ---------------------------------------------------------------

\normalsize

% Das esempio-Environment wird nur in der Leseansicht benötigt
\ifkorrekturansicht\else
\newenvironment{esempio}[3]%
{
    \vspace{1.5ex}
    \rlap{\underline{#1}}
    \par
    \setlength{\parindent}{0cm}
    \nopagebreak
    \leftskip=#2cm
    \rightskip=#3cm
}
{
    \par
}
\fi

\doendnotes{C}
\bigskip
\vfill

\clearpage

\footnotesize

\ifkorrekturansicht
  \lohead{\textsc{register}}
\fi

% theindex-Environment neu definieren ohne reledmac
\makeatletter
\renewenvironment{theindex}{%
  \ifkorrekturansicht
    \section*{\indexname}%
  \else
    \subsubsection*{Index der erwähnten Entitäten}%
  \fi
  \setlength{\parindent}{0pt}%
  \setlength{\parskip}{0pt plus 0.3pt}%
  \let\item\@idxitem
}{%
  \ifkorrekturansicht\clearpage\fi
}
\makeatother

\IfFileExists{\jobname-pw.ind}{\input{\jobname-pw.ind}}{}

% Quellenangabe nur in der Leseansicht
\ifkorrekturansicht\else
% Fallback-Definitionen, falls die .tex-Datei \titel etc. nicht gesetzt hat
\providecommand{\titel}{}
\providecommand{\editorInnen}{}
\providecommand{\dateiname}{\jobname}

\vspace{3cm}

\vfill

\footnotesize
\textsc{Quelle}: \titel. Herausgegeben von {\editorInnen}. In: \emph{Arthur Schnitzler: Briefwechsel mit Autorinnen und Autoren}.
 Digitale Edition, https://schnitzler-briefe.acdh.oeaw.ac.at/{\dateiname}.html (Stand \today)
\fi

\end{document}


