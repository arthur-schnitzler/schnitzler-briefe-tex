%% latex-korrekturansicht-vorspann.tex
%% Vorspann für die Korrekturansicht.
%% Lädt die gemeinsame Datei latex-vorspann.tex mit gesetztem Schalter.

\newif\ifkorrekturansicht
\korrekturansichttrue

\input{../tex-inputs/latex-vorspann}


\section[Richard Beer-Hofmann an Arthur Schnitzler, 6. 9. 1900]{L01071 Richard Beer-Hofmann an Arthur Schnitzler, 6. 9. 1900}
\nopagebreak\mylabel{L01071v}
\rehead{ }\normalsize\beginnumbering\briefempfaengerindex{Schnitzler, Arthur@\textsc{Schnitzler, Arthur}!zzzBeer-Hofmann, Richard@\emph{von Richard Beer-Hofmann}!1900-09-061@{6. 9. 1900}|(be}
\toendnotes[C]{\smallbreak\pagebreak[2]}\Standort{CUL, Schnitzler, B 8.}
\physDesc{Brief, 1 Blatt, 1 Seite, 214 Zeichen
\newline{}Handschrift: Bleistift, lateinische Kurrent
\newline{}Ordnung: mit Bleistift von unbekannter Hand nummeriert:
                                    »158« }\toendnotes[C]{\smallbreak}
\pstart
           \raggedleft{}{\pb}Alt-Aussee\oindex{Altaussee@\textbf{Altaussee}, \emph{A.ADM3}|pw}{ }6/IX 1900\pend
           \vspace{0.5em}
\pstart
           Lieber Arthur! In Eile: Ich bleibe noch bis ungefähr
                  18. hier u. ko{\geminationm}e dann nach Wien\oindex{Wien@\textbf{Wien}, \emph{A.ADM2}|pw}. Dh: ich \uline{will}
               das thun. Werde ich \label{K_L01071-1v}\edtext{Paul\pwindex{Goldmann, Paul 31.01.1865 – 25.09.1935@\textsc{Goldmann, Paul} (31.01.1865 – 25.09.1935), \emph{Schriftsteller/Schriftstellerin, Journalist/Journalistin}|pw} dann noch in Wien\oindex{Wien@\textbf{Wien}, \emph{A.ADM2}|pw} treffen}{\lemma{\textnormal{\emph{Paul … treffen}}}\Cendnote{\textnormal{Sie
                  trafen sich nicht mehr, vgl. Paul Goldmann an Arthur Schnitzler, 19. 9. [1900].}}}\label{K_L01071-1}?\pend
           
\pstart
           Schreiben Sie mir, bitte, zwei Zeilen.\pend
           \pstart Von Herzen Ihr \spacefill\mbox{R.}\pend{}\selectlanguage{ngerman}\endnumbering\briefempfaengerindex{Schnitzler, Arthur@\textsc{Schnitzler, Arthur}!zzzBeer-Hofmann, Richard@\emph{von Richard Beer-Hofmann}!1900-09-061@{6. 9. 1900}|)be}\mylabel{L01071h}  \normalsize

\doendnotes{C}
\bigskip
\vfill

\clearpage

\footnotesize

\lohead{\textsc{register}}

% Definiere theindex-Environment komplett neu ohne reledmac
\makeatletter
\renewenvironment{theindex}{%
  \section*{\indexname}%
  \setlength{\parindent}{0pt}%
  \setlength{\parskip}{0pt plus 0.3pt}%
  \let\item\@idxitem
}{%
  \clearpage
}
\makeatother

\IfFileExists{\jobname-pw.ind}{\input{\jobname-pw.ind}}{}

\end{document}

      