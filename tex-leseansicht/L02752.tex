%% latex-leseansicht-vorspann.tex
%% Vorspann für die Leseansicht.
%% Lädt die gemeinsame Datei latex-vorspann.tex mit nicht gesetztem Schalter.

\newif\ifkorrekturansicht
\korrekturansichtfalse

\input{../tex-inputs/latex-vorspann}


         
         \newcommand{\erwaehntePersonen}{Personen: Hermann Bahr, Richard Beer-Hofmann, Hugo von Hofmannsthal, Leopold Sonnemann}
         \newcommand{\erwaehnteInstitutionen}{Institutionen: Frankfurter Zeitung}
         \newcommand{\erwaehnteOrte}{Orte: Paris, Wien, rue Feydeau}
         \newcommand{\erwaehnteWerke}{Werke: Burgtheater (Liebelei, Schauspiel in drei Acten von Arthur Schnitzler. Rechte der Seele, Schauspiel in einem Act von Guiseppe Giacosa. Zum ersten Mal aufgeführt am 9. October), Die Zeit. Wiener Wochenschrift, Liebelei. Schauspiel in drei Akten}
               \section[Paul Goldmann an Arthur Schnitzler, 14. 10. {[}1895{]}]{ Paul Goldmann an Arthur Schnitzler, 14. 10. {[}1895{]}}\nopagebreak\mylabel{v}\rehead{ }\begin{ledgroupsized}[t]{13cm}\normalsize\beginnumbering \toendnotes[C]{\smallbreak\pagebreak[2]} \Standort{DLA, A:Schnitzler, HS.NZ85.1.3165.}
\physDesc{Brief, 1 Blatt, 2 Seiten
\newline{}Handschrift: blaue Tinte, deutsche Kurrent
\newline{}Schnitzler: 1) mit Bleistift das Jahr » 95« vermerkt  2) mit rotem Buntstift eine Unterstreichung}\toendnotes[C]{\smallbreak}\pstart
           \noindent{}{\pb}\textcolor{gray}{\textbf{\textbf{Frankfurter Zeitung\orgindex{Frankfurter Zeitung@Frankfurter Zeitung|pw}}}}\pend
           \pstart
           \textcolor{gray}{\textbf{(\begin{otherlanguage}{french}Gazette de Francfort\end{otherlanguage}\orgindex{Frankfurter Zeitung@Frankfurter Zeitung|pw}). }}\pend
           \pstart
           \textcolor{gray}{\textbf{\textbf{\begin{otherlanguage}{french}Fondateur M. L.
                                 Sonnemann\pwindex{Sonnemann, Leopold 1831-10-29 – 1909-10-30@\textsc{Sonnemann, Leopold} (1831-10-29 – 1909-10-30), \emph{Journalist, Herausgeber}|pw}\end{otherlanguage}.}}}\hfill \textsc{Paris\oindex{Paris@\textbf{Paris}|pw}}, 14. October.\pend
           \pstart
           \begin{otherlanguage}{french}\textcolor{gray}{\textbf{Journal politique, financier,}}\end{otherlanguage}\pend
           \pstart
           \begin{otherlanguage}{french}\textcolor{gray}{\textbf{commercial et littéraire.}}\end{otherlanguage}\pend
           \pstart
           \begin{otherlanguage}{french}\textcolor{gray}{\textbf{\textbf{Paraissant trois fois par jour.}}}\end{otherlanguage}\pend
           \pstart
           \begin{otherlanguage}{french}\textcolor{gray}{\textbf{\textbf{Bureau à Paris\oindex{Paris@\textbf{Paris}|pw}}}}\end{otherlanguage}\pend
           \pstart
           \begin{otherlanguage}{french}\textcolor{gray}{\textbf{\textbf{24. Rue Feydeau\oindex{rue Feydeau@\textbf{rue Feydeau}|pw}.}}}\end{otherlanguage}\pend
           \pstart\center{}Mein lieber Freund,\pend\pstart
           Dank für Deinen lieben Brief! Schreib’ mir ausführlicher, ſobald Du kannſt, aber
               nicht früher: ich warte gern.\pend
           \pstart
           Ich ſchreibe Dir heut nur, weil ich ſoeben \textsc{Bahr\pwindex{Bahr, Hermann 19.07.1863 – 15.01.1934@\textsc{Bahr, Hermann} (19.07.1863 – 15.01.1934), \emph{Schriftsteller, Kritiker}|pw}s}{ }\label{K_L02752-1v}\edtext{Referat\pwindex{Bahr, Hermann 19.07.1863 – 15.01.1934@\textsc{Bahr, Hermann} (19.07.1863 – 15.01.1934), \emph{Schriftsteller, Kritiker}!Burgtheater (Liebelei, Schauspiel in drei Acten von Arthur Schnitzler. Rechte der Seele, Schauspiel in einem Act von Guiseppe Giacosa. Zum ersten Mal aufgefuehrt am 9. October)1895-10-12@\strich\emph{Burgtheater (Liebelei, Schauspiel in drei Acten von Arthur Schnitzler. Rechte der Seele, Schauspiel in einem Act von Guiseppe Giacosa. Zum ersten Mal aufgeführt am 9. October)} {[}1895-10-12{]}|pwv}}{\lemma{\textnormal{\emph{Referat}}}\Cendnote{\textnormal{Hermann Bahr\pwindex{Bahr, Hermann 19.07.1863 – 15.01.1934@\textsc{Bahr, Hermann} (19.07.1863 – 15.01.1934), \emph{Schriftsteller, Kritiker}|pwk}: \emph{Burgtheater (Liebelei, Schauspiel in drei Acten von Arthur
                        Schnitzler. Rechte der Seele, Schauspiel in einem Act von Guiseppe Giacosa.
                        Zum ersten Mal aufgeführt am 9. October)}\pwindex{Bahr, Hermann 19.07.1863 – 15.01.1934@\textsc{Bahr, Hermann} (19.07.1863 – 15.01.1934), \emph{Schriftsteller, Kritiker}!Burgtheater (Liebelei, Schauspiel in drei Acten von Arthur Schnitzler. Rechte der Seele, Schauspiel in einem Act von Guiseppe Giacosa. Zum ersten Mal aufgefuehrt am 9. October)1895-10-12@\strich\emph{Burgtheater (Liebelei, Schauspiel in drei Acten von Arthur Schnitzler. Rechte der Seele, Schauspiel in einem Act von Guiseppe Giacosa. Zum ersten Mal aufgeführt am 9. October)} {[}1895-10-12{]}|pwk}. In: \emph{Die Zeit}\pwindex{Zeit. Wiener Wochenschrift1894 – 1904@\emph{Die Zeit. Wiener Wochenschrift} {[}1894 – 1904{]}|pwk}, Bd. 5, Nr. 54, 12. 10. 1895, S. 27–28.}}}\label{K_L02752-1h} geleſen habe. Das iſt keine
               Kritik, das iſt ein Bubenſtreich. Ich ſehe von der Dummheit und Gemeinheit ab, mit
               der die literariſche Beurtheilung\pwindex{Bahr, Hermann 19.07.1863 – 15.01.1934@\textsc{Bahr, Hermann} (19.07.1863 – 15.01.1934), \emph{Schriftsteller, Kritiker}!Burgtheater (Liebelei, Schauspiel in drei Acten von Arthur Schnitzler. Rechte der Seele, Schauspiel in einem Act von Guiseppe Giacosa. Zum ersten Mal aufgefuehrt am 9. October)1895-10-12@\strich\emph{Burgtheater (Liebelei, Schauspiel in drei Acten von Arthur Schnitzler. Rechte der Seele, Schauspiel in einem Act von Guiseppe Giacosa. Zum ersten Mal aufgeführt am 9. October)} {[}1895-10-12{]}|pwv} abgefaßt iſt. Aber dieſer Artikel\pwindex{Bahr, Hermann 19.07.1863 – 15.01.1934@\textsc{Bahr, Hermann} (19.07.1863 – 15.01.1934), \emph{Schriftsteller, Kritiker}!Burgtheater (Liebelei, Schauspiel in drei Acten von Arthur Schnitzler. Rechte der Seele, Schauspiel in einem Act von Guiseppe Giacosa. Zum ersten Mal aufgefuehrt am 9. October)1895-10-12@\strich\emph{Burgtheater (Liebelei, Schauspiel in drei Acten von Arthur Schnitzler. Rechte der Seele, Schauspiel in einem Act von Guiseppe Giacosa. Zum ersten Mal aufgeführt am 9. October)} {[}1895-10-12{]}|pwv} enthält \label{K_L02752-55v}\edtext{perſönliche Beleidigungen}{\lemma{\textnormal{\emph{perſönliche Beleidigungen}}}\Cendnote{\textnormal{Die Kritik\pwindex{Bahr, Hermann 19.07.1863 – 15.01.1934@\textsc{Bahr, Hermann} (19.07.1863 – 15.01.1934), \emph{Schriftsteller, Kritiker}!Burgtheater (Liebelei, Schauspiel in drei Acten von Arthur Schnitzler. Rechte der Seele, Schauspiel in einem Act von Guiseppe Giacosa. Zum ersten Mal aufgefuehrt am 9. October)1895-10-12@\strich\emph{Burgtheater (Liebelei, Schauspiel in drei Acten von Arthur Schnitzler. Rechte der Seele, Schauspiel in einem Act von Guiseppe Giacosa. Zum ersten Mal aufgeführt am 9. October)} {[}1895-10-12{]}|pwkv} lässt sich in diesem Satz zusammenfassen: Schnitzler\pwindex{Schnitzler, Arthur 15.05.1862 – 21.10.1931@\textsc{Schnitzler, Arthur} (15.05.1862 – 21.10.1931), \emph{Schriftsteller, Mediziner}|pwk} »weiß die neuen Elemente unserer Stadt\oindex{Wien@\textbf{Wien}|pwv} zu fühlen, auch zu
                     schildern; ›dramatisieren‹ kann er sie noch nicht.« (S. 27) Woran Goldmann\pwindex{Goldmann, Paul 31.01.1865 – 25.09.1935@\textsc{Goldmann, Paul} (31.01.1865 – 25.09.1935), \emph{Schriftsteller, Journalist}|pwk} die persönliche Beleidigung
                  festmacht, ist nicht zu bestimmen; eventuell in der behaupteten Nähe von Schnitzler\pwindex{Schnitzler, Arthur 15.05.1862 – 21.10.1931@\textsc{Schnitzler, Arthur} (15.05.1862 – 21.10.1931), \emph{Schriftsteller, Mediziner}|pwk} und den Lebemännern, die er
                  schildert, oder in dieser Aussage: »›Er ist für eine andere gestorben! für
                     eine Frau, die er geliebt hat – ihr Mann hat ihn umgebracht! Und ich – was bin
                     ich denn? Was war denn ich? Was bin denn ich ihm gewesen?‹ Diese Klage\pwindex{Schnitzler, Arthur 15.05.1862 – 21.10.1931@\textsc{Schnitzler, Arthur} (15.05.1862 – 21.10.1931), \emph{Schriftsteller, Mediziner}!Liebelei. Schauspiel in drei Akten1895-10-09@\strich\emph{Liebelei. Schauspiel in drei Akten} {[}1895-10-09{]}|pwv} hat einen so
                     innigen und echten Ton, dass man merkt, sie kommt dem Autor\pwindex{Schnitzler, Arthur 15.05.1862 – 21.10.1931@\textsc{Schnitzler, Arthur} (15.05.1862 – 21.10.1931), \emph{Schriftsteller, Mediziner}|pwv} vom Herzen; das sehr wien\oindex{Wien@\textbf{Wien}|pw}erische Elend, an dem Leben so daneben
                     vorbeizuleben, hat er, das vernimmt man, wohl an sich selbst gespürt.«
                  (ebd.)}}}\label{K_L02752-55h} gegen Dich. {\pb}Ich habe vor
               Entrüſtung gezittert, als ich das las. Wäre ich in Wien\oindex{Wien@\textbf{Wien}|pw}, ſo würde \uline{ich} den Menſchen\pwindex{Bahr, Hermann 19.07.1863 – 15.01.1934@\textsc{Bahr, Hermann} (19.07.1863 – 15.01.1934), \emph{Schriftsteller, Kritiker}|pwv} zur Rechenſchaft gezogen haben. Du
               ſelbſt kannſt kaum etwas machen, da die Welt Dir in jedem Falle Unrecht geben würde.
               Aber ich halte es für abſolut unumgänglich, daß Du Deine perſönlichen Beziehungen zu
               dem Burſchen\pwindex{Bahr, Hermann 19.07.1863 – 15.01.1934@\textsc{Bahr, Hermann} (19.07.1863 – 15.01.1934), \emph{Schriftsteller, Kritiker}|pwv} abbrichſt.
                  \label{K_L02752-44v}\edtext{Das Gleiche erwarte ich von \textsc{Richard\pwindex{Beer-Hofmann, Richard 1866-07-11 – 1945-09-26@\textsc{Beer-Hofmann, Richard} (1866-07-11 – 1945-09-26), \emph{Schriftsteller}|pw}}}{\lemma{\textnormal{\emph{Das … Richard}}}\Cendnote{\textnormal{Auch Schnitzler\pwindex{Schnitzler, Arthur 15.05.1862 – 21.10.1931@\textsc{Schnitzler, Arthur} (15.05.1862 – 21.10.1931), \emph{Schriftsteller, Mediziner}|pwk} war über den »freundschaftlichen Verkehr« Beer-Hofmann\pwindex{Beer-Hofmann, Richard 1866-07-11 – 1945-09-26@\textsc{Beer-Hofmann, Richard} (1866-07-11 – 1945-09-26), \emph{Schriftsteller}|pwk}s und Hofmannsthal\pwindex{Hofmannsthal, Hugo von 1874-02-01 – 1929-07-15@\textsc{Hofmannsthal, Hugo von} (1874-02-01 – 1929-07-15), \emph{Schriftsteller}|pwk}s mit Bahr\pwindex{Bahr, Hermann 19.07.1863 – 15.01.1934@\textsc{Bahr, Hermann} (19.07.1863 – 15.01.1934), \emph{Schriftsteller, Kritiker}|pwk} verärgert, vgl. A. S.: \emph{Tagebuch}, 6. 11. 1895.}}}\label{K_L02752-44h}. Ein Bube\pwindex{Bahr, Hermann 19.07.1863 – 15.01.1934@\textsc{Bahr, Hermann} (19.07.1863 – 15.01.1934), \emph{Schriftsteller, Kritiker}|pwv}, der mit Schmutz wirft, gehört nicht in Eure Geſellſchaft.\pend
           \pstart
           Viele treue Grüße! Dein {\\[\baselineskip]}\spacefill\mbox{Paul Goldmann.}\pend
           \leftskip=0em{}
         
         \endnumbering\mylabel{h}\end{ledgroupsized}  \newcommand{\dateiname}{L02752}\newcommand{\titel}{Paul Goldmann an Arthur Schnitzler, 14. 10. [1895]}\newcommand{\editorInnen}{Martin Anton Müller und Laura Untner}%% latex-leseansicht-abspann.tex
%% Abspann für die Leseansicht.
%% Der Schalter \ifkorrekturansicht ist bereits durch den Vorspann gesetzt.

%% latex-abspann.tex
%% Gemeinsamer Abspann für Korrekturansicht und Leseansicht.
%% Setzt den Schalter \ifkorrekturansicht voraus (gesetzt in den
%% einbindenden Dateien latex-korrekturansicht-abspann.tex bzw.
%% latex-leseansicht-abspann.tex).
%% ---------------------------------------------------------------

\normalsize

% Das esempio-Environment wird nur in der Leseansicht benötigt
\ifkorrekturansicht\else
\newenvironment{esempio}[3]%
{
    \vspace{1.5ex}
    \rlap{\underline{#1}}
    \par
    \setlength{\parindent}{0cm}
    \nopagebreak
    \leftskip=#2cm
    \rightskip=#3cm
}
{
    \par
}
\fi

\doendnotes{C}
\bigskip
\vfill

\clearpage

\footnotesize

\ifkorrekturansicht
  \lohead{\textsc{register}}
\fi

% theindex-Environment neu definieren ohne reledmac
\makeatletter
\renewenvironment{theindex}{%
  \ifkorrekturansicht
    \section*{\indexname}%
  \else
    \subsubsection*{Index der erwähnten Entitäten}%
  \fi
  \setlength{\parindent}{0pt}%
  \setlength{\parskip}{0pt plus 0.3pt}%
  \let\item\@idxitem
}{%
  \ifkorrekturansicht\clearpage\fi
}
\makeatother

\IfFileExists{\jobname-pw.ind}{\input{\jobname-pw.ind}}{}

% Quellenangabe nur in der Leseansicht
\ifkorrekturansicht\else
% Fallback-Definitionen, falls die .tex-Datei \titel etc. nicht gesetzt hat
\providecommand{\titel}{}
\providecommand{\editorInnen}{}
\providecommand{\dateiname}{\jobname}

\vspace{3cm}

\vfill

\footnotesize
\textsc{Quelle}: \titel. Herausgegeben von {\editorInnen}. In: \emph{Arthur Schnitzler: Briefwechsel mit Autorinnen und Autoren}.
 Digitale Edition, https://schnitzler-briefe.acdh.oeaw.ac.at/{\dateiname}.html (Stand \today)
\fi

\end{document}


      