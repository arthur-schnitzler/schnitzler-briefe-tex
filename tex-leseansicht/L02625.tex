%% latex-leseansicht-vorspann.tex
%% Vorspann für die Leseansicht.
%% Lädt die gemeinsame Datei latex-vorspann.tex mit nicht gesetztem Schalter.

\newif\ifkorrekturansicht
\korrekturansichtfalse

\input{../tex-inputs/latex-vorspann}


         
         \renewcommand{\erwaehntePersonen}{Personen: Henri Albert, Albrecht Altdorfer, Theodor Herzl, Ludwig Quidde, Leopold Sonnemann,  Wilhelm II. von Preußen}
         \renewcommand{\erwaehnteInstitutionen}{Institutionen: Frankfurter Zeitung}
         \renewcommand{\erwaehnteOrte}{Orte: Bad Aussee, Bad Ischl, München, Paris, Wien, rue Feydeau}
         \renewcommand{\erwaehnteWerke}{Werke: Anatol, Caligula – Eine Studie über römischen Cäsarenwahnsinn, Das Märchen. Schauspiel in drei Aufzügen, L'Idée libre. Revue mensuelle de Littérature et d'Art, La Revue blanche, Laubwald mit dem heiligen Georg, Les Emplettes de Noël, Les Lettres allemandes. Drames Nouveaux, Mercure de France, Tagebuch, Weihnachts-Einkäufe}
               \section[Paul Goldmann an Arthur Schnitzler, 15. 6. {[}1894{]}]{ Paul Goldmann an Arthur Schnitzler, 15. 6. {[}1894{]}}\nopagebreak\mylabel{v}\rehead{ }\begin{ledgroupsized}[t]{13cm}\normalsize\beginnumbering \toendnotes[C]{\smallbreak\pagebreak[2]} \Standort{DLA, A:Schnitzler, HS.NZ85.1.3164.}
\physDesc{Brief, 1 Blatt, 3 Seiten, 1084 Zeichen
\newline{}Handschrift: schwarze Tinte, deutsche Kurrent
\newline{}Schnitzler: 1) mit Bleistift auf dem ersten Blatt die Jahreszahl »94« vermerkt  2) mit rotem Buntstift eine Unterstreichung}\toendnotes[C]{\smallbreak}\pstart
           \noindent{}{\pb}\textcolor{gray}{\textbf{Frankfurter Zeitung\orgindex{Frankfurter Zeitung@Frankfurter Zeitung|pw}.}}\pend
           \pstart
           \textcolor{gray}{\textbf{(Gazette de
                     Francfort\orgindex{Frankfurter Zeitung@Frankfurter Zeitung|pw}).}}\pend
           \pstart
           \textcolor{gray}{\textbf{Fondateur \textbf{M. L. Sonnemann\pwindex{Sonnemann, Leopold 1831-10-29 – 1909-10-30@\textsc{Sonnemann, Leopold} (1831-10-29 – 1909-10-30), \emph{Journalist, Herausgeber}|pw}}.}}\pend
           \pstart
           \textcolor{gray}{\textbf{\begin{otherlanguage}{french}Journal politique, financier,\end{otherlanguage}}}\hfill \textsc{Paris\oindex{Paris@\textbf{Paris}|pw}}, 15. Juni.\pend
           \pstart
           \textcolor{gray}{\textbf{\begin{otherlanguage}{french}commercial et littéraire.\end{otherlanguage}}}\pend
           \pstart
           \textcolor{gray}{\textbf{\begin{otherlanguage}{french}\textbf{Paraissant trois fois par jour.}\end{otherlanguage}}}\pend
           \pstart
           \textcolor{gray}{\textbf{\begin{otherlanguage}{french}\textbf{Bureau à Paris\oindex{Paris@\textbf{Paris}|pw}:}\end{otherlanguage}}}\pend
           \pstart
           \textcolor{gray}{\textbf{\begin{otherlanguage}{french}24. Rue Feydeau\oindex{rue Feydeau@\textbf{rue Feydeau}|pw}.\end{otherlanguage}}}\pend
           \pstart{}Mein lieber Freund,\pend\pstart
           Ich bin ſehr beſchäftigt. Darum nur wenige Zeilen.\pend
           \pstart
           1.) Wärmſten Dank für Deinen lieben Brief aus \label{K_L02625-1v}\edtext{\textsc{Muenchen\oindex{Muenchen@\textbf{München}|pw}}}{\lemma{\textnormal{\emph{Muenchen}}}\Cendnote{\textnormal{Zwischen 2. 6. 1894 und 8. 6. 1894 hielt sich Schnitzler\pwindex{Schnitzler, Arthur 15.05.1862 – 21.10.1931@\textsc{Schnitzler, Arthur} (15.05.1862 – 21.10.1931), \emph{Schriftsteller, Mediziner}|pwk} in München\oindex{Muenchen@\textbf{München}|pwk}
                  auf.}}}\label{K_L02625-1h}. Er erklärt Manches und läßt Manches im Unklaren. All’ das iſt ſehr
               ſchwer brieflich abzumachen. Auch das, was mich erregt, läßt ſich kaum ſo
               niederſchreiben. Ich möchte mit Dir ſprechen, aber vielleicht iſt es am Beſten gar
               nicht mehr darüber zu {\pb}reden. Die Dinge müſſen ihren
               Lauf gehen.\pend
           \pstart
           2.) Haſt Du die \label{K_L02625-5v}\edtext{»\textsc{Revue Blanche\pwindex{?? Werk@Nicht ermittelte Verfasserinnen und Verfasser!Revue blanche1889 – 1903@\emph{La Revue blanche} {[}1889 – 1903{]}|pw}}«}{\lemma{\textnormal{\emph{»Revue Blanche«}}}\Cendnote{\textnormal{Die wohl für den \emph{Mercure de France}\pwindex{?? Werk@Nicht ermittelte Verfasserinnen und Verfasser!Mercure de France1890 – 1965@\emph{Mercure de France} {[}1890 – 1965{]}|pwk} gedachte (siehe Paul Goldmann an Arthur Schnitzler, 29. 5. [1894]) Besprechung\pwindex{Albert, Henri 1869-11-16 – 1921-08-03@\textsc{Albert, Henri} (1869-11-16 – 1921-08-03), \emph{Journalist, Kritiker, Übersetzer}!Lettres allemandes. Drames Nouveaux1894-06@\strich\emph{Les Lettres allemandes. Drames Nouveaux} {[}1894-06{]}|pwkv} von Schnitzler\pwindex{Schnitzler, Arthur 15.05.1862 – 21.10.1931@\textsc{Schnitzler, Arthur} (15.05.1862 – 21.10.1931), \emph{Schriftsteller, Mediziner}|pwk}s Schauspiel \emph{Das Märchen}\pwindex{Schnitzler, Arthur 15.05.1862 – 21.10.1931@\textsc{Schnitzler, Arthur} (15.05.1862 – 21.10.1931), \emph{Schriftsteller, Mediziner}!Maerchen. Schauspiel in drei Aufzuegen1893-12-01@\strich\emph{Das Märchen. Schauspiel in drei Aufzügen} {[}1893-12-01{]}|pwk}
                  erschien in der \emph{Revue blanche}\pwindex{?? Werk@Nicht ermittelte Verfasserinnen und Verfasser!Revue blanche1889 – 1903@\emph{La Revue blanche} {[}1889 – 1903{]}|pwk}: Henri Albert\pwindex{Albert, Henri 1869-11-16 – 1921-08-03@\textsc{Albert, Henri} (1869-11-16 – 1921-08-03), \emph{Journalist, Kritiker, Übersetzer}|pwk}: \emph{Les Lettres allemandes. Drames Nouveaux}\pwindex{Albert, Henri 1869-11-16 – 1921-08-03@\textsc{Albert, Henri} (1869-11-16 – 1921-08-03), \emph{Journalist, Kritiker, Übersetzer}!Lettres allemandes. Drames Nouveaux1894-06@\strich\emph{Les Lettres allemandes. Drames Nouveaux} {[}1894-06{]}|pwk}. In: \emph{La Revue Blanche}\pwindex{?? Werk@Nicht ermittelte Verfasserinnen und Verfasser!Revue blanche1889 – 1903@\emph{La Revue blanche} {[}1889 – 1903{]}|pwk}, Jg. 6, Nr. 32,
                        Juni 1984, S. 556–560, hier S. 560. Dem \emph{Tagebuch}\pwindex{Schnitzler, Arthur 15.05.1862 – 21.10.1931@\textsc{Schnitzler, Arthur} (15.05.1862 – 21.10.1931), \emph{Schriftsteller, Mediziner}!Tagebuch1981 – 2000@\strich\emph{Tagebuch} {[}1981 – 2000{]}|pwk} ist zu entnehmen, dass Schnitzler\pwindex{Schnitzler, Arthur 15.05.1862 – 21.10.1931@\textsc{Schnitzler, Arthur} (15.05.1862 – 21.10.1931), \emph{Schriftsteller, Mediziner}|pwk} die Besprechung\pwindex{Albert, Henri 1869-11-16 – 1921-08-03@\textsc{Albert, Henri} (1869-11-16 – 1921-08-03), \emph{Journalist, Kritiker, Übersetzer}!Lettres allemandes. Drames Nouveaux1894-06@\strich\emph{Les Lettres allemandes. Drames Nouveaux} {[}1894-06{]}|pwkv} las (vgl. A. S.: \emph{Tagebuch}, 11. 6. 1894).}}}\label{K_L02625-5h} erhalten.\pend
           \pstart
           3.) Können wir \label{K_L02625-7v}\edtext{im Auguſt zuſammenreiſen}{\lemma{\textnormal{\emph{im Auguſt zuſammenreiſen}}}\Cendnote{\textnormal{Von 23. 8. 1894 bis 3. 9. 1894
                  verbrachten Schnitzler\pwindex{Schnitzler, Arthur 15.05.1862 – 21.10.1931@\textsc{Schnitzler, Arthur} (15.05.1862 – 21.10.1931), \emph{Schriftsteller, Mediziner}|pwk} und Goldmann\pwindex{Goldmann, Paul 31.01.1865 – 25.09.1935@\textsc{Goldmann, Paul} (31.01.1865 – 25.09.1935), \emph{Schriftsteller, Journalist}|pwk} einige Zeit gemeinsam in Bad Ischl\oindex{Bad Ischl@\textbf{Bad Ischl}|pwk} und Bad
                  Aussee\oindex{Bad Aussee@\textbf{Bad Aussee}|pwk}.}}}\label{K_L02625-7h}? Bitte, antworte mir umgehend, denn ich muß jetzt bereits
               anfangen, eventuelle Vorkehrungen zu treffen.\pend
           \pstart
           4.) Was weißt Du von \textsc{Muenchen\oindex{Muenchen@\textbf{München}|pw}} zu erzählen? Haſt Du den \textsc{Altdorfer\pwindex{Altdorfer, Albrecht 1480 – 1538-02-12@\textsc{Altdorfer, Albrecht} (1480 – 1538-02-12), \emph{Bildender Künstler, Kupferstecher, Baumeister}|pw}\pwindex{Altdorfer, Albrecht 1480 – 1538-02-12@\textsc{Altdorfer, Albrecht} (1480 – 1538-02-12), \emph{Bildender Künstler, Kupferstecher, Baumeister}!Laubwald mit dem heiligen Georg1510@\strich\emph{Laubwald mit dem heiligen Georg} {[}1510{]}|pwv}} geſehen, von dem ich Dir \label{K_L02625-2v}\edtext{ſchrieb}{\lemma{\textnormal{\emph{ſchrieb}}}\Cendnote{\textnormal{siehe Paul Goldmann an Arthur Schnitzler, 1. 6. [1894]}}}\label{K_L02625-2h}? Wie gehts Dir \strikeout{J} geſundheitlich?\pend
           \pstart
           {\pb}5.) \textsc{Herzl\pwindex{Herzl, Theodor 1860-05-02 – 1904-07-03@\textsc{Herzl, Theodor} (1860-05-02 – 1904-07-03), \emph{Schriftsteller, Journalist}|pw}}, den ich verſchiedentlich von Dir gegrüßt, läßt Dich verſchiedentlich wieder
               grüßen. Desgleichen \textsc{Henri Albert\pwindex{Albert, Henri 1869-11-16 – 1921-08-03@\textsc{Albert, Henri} (1869-11-16 – 1921-08-03), \emph{Journalist, Kritiker, Übersetzer}|pw}}. Ich habe dieſer Tage den \label{K_L02625-3v}\edtext{Bürſten-Abzug}{\lemma{\textnormal{\emph{Bürſten-Abzug}}}\Cendnote{\textnormal{Probeabzug}}}\label{K_L02625-3h} der
                  \label{K_L02625-4v}\edtext{»\textsc{Emplettes de Noël\pwindex{Albert, Henri 1869-11-16 – 1921-08-03@\textsc{Albert, Henri} (1869-11-16 – 1921-08-03), \emph{Journalist, Kritiker, Übersetzer}!Emplettes de Noel1894-05 – 1894-06@\strich\emph{Les Emplettes de Noël} {[}Übersetzung, 1894-05 – 1894-06{]}|pw}}«}{\lemma{\textnormal{\emph{»Emplettes de Noël«}}}\Cendnote{\textnormal{Henri Albert\pwindex{Albert, Henri 1869-11-16 – 1921-08-03@\textsc{Albert, Henri} (1869-11-16 – 1921-08-03), \emph{Journalist, Kritiker, Übersetzer}|pwk}s Übersetzung\pwindex{Albert, Henri 1869-11-16 – 1921-08-03@\textsc{Albert, Henri} (1869-11-16 – 1921-08-03), \emph{Journalist, Kritiker, Übersetzer}!Emplettes de Noel1894-05 – 1894-06@\strich\emph{Les Emplettes de Noël} {[}Übersetzung, 1894-05 – 1894-06{]}|pwkv} von Schnitzler\pwindex{Schnitzler, Arthur 15.05.1862 – 21.10.1931@\textsc{Schnitzler, Arthur} (15.05.1862 – 21.10.1931), \emph{Schriftsteller, Mediziner}|pwk}s \emph{Anatol}\pwindex{Schnitzler, Arthur 15.05.1862 – 21.10.1931@\textsc{Schnitzler, Arthur} (15.05.1862 – 21.10.1931), \emph{Schriftsteller, Mediziner}!Anatol1892-10-29@\strich\emph{Anatol} {[}1892-10-29{]}|pwk}-Einakter \emph{Weihnachts-Einkäufe}\pwindex{Schnitzler, Arthur 15.05.1862 – 21.10.1931@\textsc{Schnitzler, Arthur} (15.05.1862 – 21.10.1931), \emph{Schriftsteller, Mediziner}!Weihnachts-Einkaeufe24. 12. 1891@\strich\emph{Weihnachts-Einkäufe} {[}24. 12. 1891{]}|pwk}}}}\label{K_L02625-4h} geſehen, die \label{K_L02625-6v}\edtext{in der »\textsc{Idée Libre\pwindex{?? Werk@Nicht ermittelte Verfasserinnen und Verfasser!L'Idee libre. Revue mensuelle de Litterature et d'Art1892 – 1895@\emph{L'Idée libre. Revue mensuelle de Littérature et d'Art} {[}1892 – 1895{]}|pw}}« erſcheinen}{\lemma{\textnormal{\emph{in … erſcheinen}}}\Cendnote{\textnormal{Arthur Schnitzler\pwindex{Schnitzler, Arthur 15.05.1862 – 21.10.1931@\textsc{Schnitzler, Arthur} (15.05.1862 – 21.10.1931), \emph{Schriftsteller, Mediziner}|pwk}: \emph{Les Emplettes de Noël}\pwindex{Albert, Henri 1869-11-16 – 1921-08-03@\textsc{Albert, Henri} (1869-11-16 – 1921-08-03), \emph{Journalist, Kritiker, Übersetzer}!Emplettes de Noel1894-05 – 1894-06@\strich\emph{Les Emplettes de Noël} {[}Übersetzung, 1894-05 – 1894-06{]}|pwk}. Übersetzt von Henri Albert\pwindex{Albert, Henri 1869-11-16 – 1921-08-03@\textsc{Albert, Henri} (1869-11-16 – 1921-08-03), \emph{Journalist, Kritiker, Übersetzer}|pwk}. In: \emph{L'Idée
                        libre. Revue mensuelle de Littérature et d'Art}\pwindex{?? Werk@Nicht ermittelte Verfasserinnen und Verfasser!L'Idee libre. Revue mensuelle de Litterature et d'Art1892 – 1895@\emph{L'Idée libre. Revue mensuelle de Littérature et d'Art} {[}1892 – 1895{]}|pwk}, Jg. 3, Nr. 5–6,
                        Mai–Juni 1984, S. 215–225. Am 21. 7. 1894 notierte Schnitzler\pwindex{Schnitzler, Arthur 15.05.1862 – 21.10.1931@\textsc{Schnitzler, Arthur} (15.05.1862 – 21.10.1931), \emph{Schriftsteller, Mediziner}|pwk} in seinem \emph{Tagebuch}\pwindex{Schnitzler, Arthur 15.05.1862 – 21.10.1931@\textsc{Schnitzler, Arthur} (15.05.1862 – 21.10.1931), \emph{Schriftsteller, Mediziner}!Tagebuch1981 – 2000@\strich\emph{Tagebuch} {[}1981 – 2000{]}|pwk}: »Schlecht übersetzt.« Albert\pwindex{Albert, Henri 1869-11-16 – 1921-08-03@\textsc{Albert, Henri} (1869-11-16 – 1921-08-03), \emph{Journalist, Kritiker, Übersetzer}|pwk} gegenüber dürfte er aber ein anderes Urteil geäußert
                  haben, denn dieser antwortete ihm in einem Brief am 6. 8. 1894: »Dass Ihnen meine Uebersetzung\pwindex{Albert, Henri 1869-11-16 – 1921-08-03@\textsc{Albert, Henri} (1869-11-16 – 1921-08-03), \emph{Journalist, Kritiker, Übersetzer}!Emplettes de Noel1894-05 – 1894-06@\strich\emph{Les Emplettes de Noël} {[}Übersetzung, 1894-05 – 1894-06{]}|pwv} so gut gefallen hat, hat mich hoch
                     erfreut.« (\emph{DLA}, HS.1985.1.2331,3)}}}\label{K_L02625-6h} werden, da die
               andern auf Monat und Jahr hinaus keinen Platz haben.\pend
           \pstart
           6.) \label{K_L02625-8v}\edtext{Lies »\textsc{Caligula\pwindex{Quidde, Ludwig 1858-03-23 – 1941-03-05@\textsc{Quidde, Ludwig} (1858-03-23 – 1941-03-05), \emph{Politiker, Herausgeber, Historiker}!Caligula – Eine Studie ueber roemischen Caesarenwahnsinn1894@\strich\emph{Caligula – Eine Studie über römischen Cäsarenwahnsinn} {[}1894{]}|pw}}« von \textsc{Quidde\pwindex{Quidde, Ludwig 1858-03-23 – 1941-03-05@\textsc{Quidde, Ludwig} (1858-03-23 – 1941-03-05), \emph{Politiker, Herausgeber, Historiker}|pw}}!}{\lemma{\textnormal{\emph{Lies … Quidde!}}}\Cendnote{\textnormal{Eine Lektüre der kleinen Studie
                  über den Cäsarenwahn durch Schnitzler\pwindex{Schnitzler, Arthur 15.05.1862 – 21.10.1931@\textsc{Schnitzler, Arthur} (15.05.1862 – 21.10.1931), \emph{Schriftsteller, Mediziner}|pwk}, die
                  von den Zeitgenossinnen und Zeitgenossen als Schmähschrift gegen Wilhelm II.\pwindex{Wilhelm II. von Preussen 27.1.1859 – 4.6.1941@\textsc{Wilhelm II. von Preußen} (27.1.1859 – 4.6.1941), \emph{Kaiser}|pwk} gelesen wurde, ist bislang nicht belegt.}}}\label{K_L02625-8h}\pend
           \pstart
           7.) Viele treue Grüße! {\\[\baselineskip]}Dein {\\[\baselineskip]}\spacefill\mbox{Paul Goldmann}\pend
           \leftskip=0em{}
         
         \endnumbering\mylabel{h}\end{ledgroupsized}  \newcommand{\dateiname}{L02625}\newcommand{\titel}{Paul Goldmann an Arthur Schnitzler, 15. 6. [1894]}\newcommand{\editorInnen}{Martin Anton Müller und Laura Untner}%% latex-leseansicht-abspann.tex
%% Abspann für die Leseansicht.
%% Der Schalter \ifkorrekturansicht ist bereits durch den Vorspann gesetzt.

%% latex-abspann.tex
%% Gemeinsamer Abspann für Korrekturansicht und Leseansicht.
%% Setzt den Schalter \ifkorrekturansicht voraus (gesetzt in den
%% einbindenden Dateien latex-korrekturansicht-abspann.tex bzw.
%% latex-leseansicht-abspann.tex).
%% ---------------------------------------------------------------

\normalsize

% Das esempio-Environment wird nur in der Leseansicht benötigt
\ifkorrekturansicht\else
\newenvironment{esempio}[3]%
{
    \vspace{1.5ex}
    \rlap{\underline{#1}}
    \par
    \setlength{\parindent}{0cm}
    \nopagebreak
    \leftskip=#2cm
    \rightskip=#3cm
}
{
    \par
}
\fi

\doendnotes{C}
\bigskip
\vfill

\clearpage

\footnotesize

\ifkorrekturansicht
  \lohead{\textsc{register}}
\fi

% theindex-Environment neu definieren ohne reledmac
\makeatletter
\renewenvironment{theindex}{%
  \ifkorrekturansicht
    \section*{\indexname}%
  \else
    \subsubsection*{Index der erwähnten Entitäten}%
  \fi
  \setlength{\parindent}{0pt}%
  \setlength{\parskip}{0pt plus 0.3pt}%
  \let\item\@idxitem
}{%
  \ifkorrekturansicht\clearpage\fi
}
\makeatother

\IfFileExists{\jobname-pw.ind}{\input{\jobname-pw.ind}}{}

% Quellenangabe nur in der Leseansicht
\ifkorrekturansicht\else
% Fallback-Definitionen, falls die .tex-Datei \titel etc. nicht gesetzt hat
\providecommand{\titel}{}
\providecommand{\editorInnen}{}
\providecommand{\dateiname}{\jobname}

\vspace{3cm}

\vfill

\footnotesize
\textsc{Quelle}: \titel. Herausgegeben von {\editorInnen}. In: \emph{Arthur Schnitzler: Briefwechsel mit Autorinnen und Autoren}.
 Digitale Edition, https://schnitzler-briefe.acdh.oeaw.ac.at/{\dateiname}.html (Stand \today)
\fi

\end{document}


      