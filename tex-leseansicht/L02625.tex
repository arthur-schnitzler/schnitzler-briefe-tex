%% latex-leseansicht-vorspann.tex
%% Vorspann für die Leseansicht.
%% Lädt die gemeinsame Datei latex-vorspann.tex mit nicht gesetztem Schalter.

\newif\ifkorrekturansicht
\korrekturansichtfalse

\input{../tex-inputs/latex-vorspann}


\section[Paul Goldmann an Arthur Schnitzler, 15. 6. [1894]]{L02625 Paul Goldmann an Arthur Schnitzler, 15. 6. [1894]}
\nopagebreak\mylabel{L02625v}
\rehead{ }\normalsize\beginnumbering\briefempfaengerindex{Schnitzler, Arthur@\textsc{Schnitzler, Arthur}!zzzGoldmann, Paul@\emph{von Paul Goldmann}!1894-06-153@{15. 6. [1894]}|(be}
\toendnotes[C]{\smallbreak\pagebreak[2]}
\correspDesc{Versand  durch Paul Goldmann am 15. 6. [1894] in Paris
\newline{}Erhalt  durch Arthur Schnitzler im Zeitraum [16. 6. 1894
                  – 20. 6. 1894?] in Wien}\toendnotes[C]{\smallbreak}
\Standort{DLA, A:Schnitzler, HS.NZ85.1.3164.}
\physDesc{Brief, 1 Blatt, 3 Seiten, 1084 Zeichen
\newline{}Handschrift: schwarze Tinte, deutsche Kurrent
\newline{}Schnitzler: 1) mit Bleistift auf dem ersten Blatt die Jahreszahl »94« vermerkt  2) mit rotem Buntstift eine Unterstreichung}\toendnotes[C]{\smallbreak}
\pstart
           {\pb}\textcolor{gray}{\textbf{Frankfurter Zeitung\orgindex{Frankfurter Zeitung@Frankfurter Zeitung|pw}}}\pend
           
\pstart
           \textcolor{gray}{\textbf{(Gazette de
                     Francfort\orgindex{Frankfurter Zeitung@Frankfurter Zeitung|pw}).}}\pend
           
\pstart
           \textcolor{gray}{\textbf{Fondateur \textbf{M. L. Sonnemann\pwindex{Sonnemann, Leopold 29.\,10.\,1831 Höchberg – 30.\,10.\,1909 Frankfurt am Main@\textsc{Sonnemann, Leopold} (29.\,10.\,1831 Höchberg – 30.\,10.\,1909 Frankfurt am Main), \emph{Journalist, Herausgeber}|pw}}.}}\pend
           
\pstart
           \textcolor{gray}{\textbf{\begin{otherlanguage}{french}Journal politique, financier,\end{otherlanguage}}}\hfill \textsc{Paris\oindex{Paris@\textbf{Paris}, \emph{Hauptstadt}|pw}}, 15. Juni.\pend
           
\pstart
           \textcolor{gray}{\textbf{\begin{otherlanguage}{french}commercial et littéraire.\end{otherlanguage}}}\pend
           
\pstart
           \textcolor{gray}{\textbf{\begin{otherlanguage}{french}\textbf{Paraissant trois fois par jour.}\end{otherlanguage}}}\pend
           
\pstart
           \textcolor{gray}{\textbf{\begin{otherlanguage}{french}\textbf{Bureau à Paris\oindex{Paris@\textbf{Paris}, \emph{Hauptstadt}|pw}:}\end{otherlanguage}}}\pend
           
\pstart
           \textcolor{gray}{\textbf{\begin{otherlanguage}{french}24. Rue Feydeau\oindex{rue Feydeau@\textbf{rue Feydeau}, \emph{Straße}|pw}.\end{otherlanguage}}}\pend
           
\pstart{}Mein lieber Freund,\pend\vspace{0.5em}
\pstart
           Ich bin{ }ſehr beſchäftigt. Darum nur wenige Zeilen.\pend
           
\pstart
           1.) Wärmſten Dank für Deinen lieben Brief aus \label{K_L02625-1v}\edtext{\textsc{Muenchen\oindex{München@\textbf{München}|pw}}}{\lemma{\textnormal{\emph{Muenchen}}}\Cendnote{\textnormal{Zwischen 2. 6. 1894 und 8. 6. 1894 hielt sich Schnitzler in München\oindex{München@\textbf{München}|pwk}
                  auf.}}}\label{K_L02625-1}. Er erklärt Manches und läßt Manches im Unklaren. All’ das iſt{ }ſehr{ }ſchwer brieflich abzumachen. Auch das, was mich erregt, läßt{ }ſich kaum{ }ſo
               niederſchreiben. Ich möchte mit Dir{ }ſprechen, aber vielleicht iſt es am Beſten gar
               nicht mehr darüber zu {\pb}reden. Die Dinge müſſen ihren
               Lauf gehen.\pend
           
\pstart
           2.) Haſt Du die \label{K_L02625-2v}\edtext{»\textsc{Revue Blanche\pwindex{Revue blanche@\emph{La Revue blanche}|pw}}«}{\lemma{\textnormal{\emph{»Revue Blanche«}}}\Cendnote{\textnormal{Die wohl für den \emph{Mercure de France}\pwindex{Mercure de France@\emph{Mercure de France}|pwk} gedachte (siehe XXXX Auszeichnungsfehler: Dokument L02621 nicht gefunden) Besprechung\pwindex{Albert, Henri 16.\,11.\,1869 Niederbronn-les-Bains – 3.\,8.\,1921 Straßburg@\textsc{Albert, Henri} (16.\,11.\,1869 Niederbronn-les-Bains – 3.\,8.\,1921 Straßburg), \emph{Journalist, Kritiker, Übersetzer}!Lettres allemandes. Drames Nouveaux@\strich\emph{Les Lettres allemandes. Drames Nouveaux}|pwkv} von Schnitzlers Schauspiel \emph{Das Märchen}\pwindex{Schnitzler, Arthur 15.\,5.\,1862 Wien – 21.\,10.\,1931 ebd.@\textsc{Schnitzler, Arthur} (15.\,5.\,1862 Wien – 21.\,10.\,1931 ebd.), \emph{Schriftsteller, Mediziner}!Märchen. Schauspiel in drei Aufzügen@\strich\emph{Das Märchen. Schauspiel in drei Aufzügen}|pwk}
                  erschien in der \emph{Revue blanche}\pwindex{Revue blanche@\emph{La Revue blanche}|pwk}: Henri Albert\pwindex{Albert, Henri 16.\,11.\,1869 Niederbronn-les-Bains – 3.\,8.\,1921 Straßburg@\textsc{Albert, Henri} (16.\,11.\,1869 Niederbronn-les-Bains – 3.\,8.\,1921 Straßburg), \emph{Journalist, Kritiker, Übersetzer}|pwk}: \emph{Les Lettres allemandes. Drames Nouveaux}\pwindex{Albert, Henri 16.\,11.\,1869 Niederbronn-les-Bains – 3.\,8.\,1921 Straßburg@\textsc{Albert, Henri} (16.\,11.\,1869 Niederbronn-les-Bains – 3.\,8.\,1921 Straßburg), \emph{Journalist, Kritiker, Übersetzer}!Lettres allemandes. Drames Nouveaux@\strich\emph{Les Lettres allemandes. Drames Nouveaux}|pwk}. In: \emph{La Revue Blanche}\pwindex{Revue blanche@\emph{La Revue blanche}|pwk}, Jg. 6, Nr. 32,
                        Juni 1984, S. 556–560, hier S. 560. Dem \emph{Tagebuch}\pwindex{Schnitzler, Arthur 15.\,5.\,1862 Wien – 21.\,10.\,1931 ebd.@\textsc{Schnitzler, Arthur} (15.\,5.\,1862 Wien – 21.\,10.\,1931 ebd.), \emph{Schriftsteller, Mediziner}!Tagebuch@\strich\emph{Tagebuch}|pwk} ist zu entnehmen, dass Schnitzler die Besprechung\pwindex{Albert, Henri 16.\,11.\,1869 Niederbronn-les-Bains – 3.\,8.\,1921 Straßburg@\textsc{Albert, Henri} (16.\,11.\,1869 Niederbronn-les-Bains – 3.\,8.\,1921 Straßburg), \emph{Journalist, Kritiker, Übersetzer}!Lettres allemandes. Drames Nouveaux@\strich\emph{Les Lettres allemandes. Drames Nouveaux}|pwkv} gelesen hat (vgl. A. S.: \emph{Tagebuch}, 11. 6. 1894).}}}\label{K_L02625-2} erhalten.\pend
           
\pstart
           3.) Können wir \label{K_L02625-3v}\edtext{im Auguſt zuſammenreiſen}{\lemma{\textnormal{\emph{im August zusammenreisen}}}\Cendnote{\textnormal{Vom 23. 8. 1894 bis zum 3. 9. 1894
                  verbrachten Schnitzler und Goldmann\pwindex{Goldmann, Paul 31.\,1.\,1865 Breslau – 25.\,9.\,1935 Wien@\textsc{Goldmann, Paul} (31.\,1.\,1865 Breslau – 25.\,9.\,1935 Wien), \emph{Schriftsteller, Journalist}|pwk} einige Zeit gemeinsam in Bad Ischl\oindex{Bad Ischl@\textbf{Bad Ischl}|pwk} und Bad
                  Aussee\oindex{Bad Aussee@\textbf{Bad Aussee}, \emph{Hauptstadt}|pwk}.}}}\label{K_L02625-3}? Bitte, antworte mir umgehend, denn ich muß jetzt bereits
               anfangen, eventuelle Vorkehrungen zu treffen.\pend
           
\pstart
           4.) Was weißt Du von \textsc{Muenchen\oindex{München@\textbf{München}|pw}} zu erzählen? Haſt Du den \textsc{Altdorfer\pwindex{Altdorfer, Albrecht 1480 – 12.\,2.\,1538@\textsc{Altdorfer, Albrecht} (1480 – 12.\,2.\,1538), \emph{Maler, Kupferstecher, Baumeister}|pw}\pwindex{Altdorfer, Albrecht 1480 – 12.\,2.\,1538@\textsc{Altdorfer, Albrecht} (1480 – 12.\,2.\,1538), \emph{Maler, Kupferstecher, Baumeister}!Laubwald mit dem heiligen Georg@\strich\emph{Laubwald mit dem heiligen Georg}|pwv}} geſehen, von dem ich Dir \label{K_L02625-4v}\edtext{ſchrieb}{\lemma{\textnormal{\emph{schrieb}}}\Cendnote{\textnormal{Siehe XXXX Auszeichnungsfehler: Dokument L02623 nicht gefunden.
               }}}\label{K_L02625-4}? Wie gehts Dir \strikeout{J} geſundheitlich?\pend
           
\pstart
           {\pb}5.) \textsc{Herzl\pwindex{Herzl, Theodor 2.\,5.\,1860 Budapest – 3.\,7.\,1904 Edlach@\textsc{Herzl, Theodor} (2.\,5.\,1860 Budapest – 3.\,7.\,1904 Edlach), \emph{Schriftsteller, Journalist}|pw}}, den ich verſchiedentlich von Dir gegrüßt, läßt Dich verſchiedentlich wieder
               grüßen. Desgleichen \textsc{Henri Albert\pwindex{Albert, Henri 16.\,11.\,1869 Niederbronn-les-Bains – 3.\,8.\,1921 Straßburg@\textsc{Albert, Henri} (16.\,11.\,1869 Niederbronn-les-Bains – 3.\,8.\,1921 Straßburg), \emph{Journalist, Kritiker, Übersetzer}|pw}}. Ich habe dieſer Tage den \label{K_L02625-5v}\edtext{Bürſten-Abzug}{\lemma{\textnormal{\emph{Bürsten-Abzug}}}\Cendnote{\textnormal{Probeabzug}}}\label{K_L02625-5} der
                  \label{K_L02625-6v}\edtext{»\textsc{Emplettes de Noël\pwindex{Schnitzler, Arthur 15.\,5.\,1862 Wien – 21.\,10.\,1931 ebd.@\textsc{Schnitzler, Arthur} (15.\,5.\,1862 Wien – 21.\,10.\,1931 ebd.), \emph{Schriftsteller, Mediziner}!Emplettes de Noël@\strich\emph{Les Emplettes de Noël}|pw}}«}{\lemma{\textnormal{\emph{»Emplettes de Noël«}}}\Cendnote{\textnormal{Henri Alberts\pwindex{Albert, Henri 16.\,11.\,1869 Niederbronn-les-Bains – 3.\,8.\,1921 Straßburg@\textsc{Albert, Henri} (16.\,11.\,1869 Niederbronn-les-Bains – 3.\,8.\,1921 Straßburg), \emph{Journalist, Kritiker, Übersetzer}|pwk}{ }Übersetzung\pwindex{Schnitzler, Arthur 15.\,5.\,1862 Wien – 21.\,10.\,1931 ebd.@\textsc{Schnitzler, Arthur} (15.\,5.\,1862 Wien – 21.\,10.\,1931 ebd.), \emph{Schriftsteller, Mediziner}!Emplettes de Noël@\strich\emph{Les Emplettes de Noël}|pwkv} von Schnitzlers{ }\emph{Anatol}\pwindex{Schnitzler, Arthur 15.\,5.\,1862 Wien – 21.\,10.\,1931 ebd.@\textsc{Schnitzler, Arthur} (15.\,5.\,1862 Wien – 21.\,10.\,1931 ebd.), \emph{Schriftsteller, Mediziner}!Anatol@\strich\emph{Anatol}|pwk}-Einakter \emph{Weihnachts-Einkäufe}\pwindex{Schnitzler, Arthur 15.\,5.\,1862 Wien – 21.\,10.\,1931 ebd.@\textsc{Schnitzler, Arthur} (15.\,5.\,1862 Wien – 21.\,10.\,1931 ebd.), \emph{Schriftsteller, Mediziner}!Weihnachts-Einkäufe@\strich\emph{Weihnachts-Einkäufe}|pwk}}}}\label{K_L02625-6} geſehen, die \label{K_L02625-7v}\edtext{in der »\textsc{Idée Libre\pwindex{Idée libre. Revue mensuelle de Littérature et d'Art@\emph{L’Idée libre. Revue mensuelle de Littérature et d'Art}|pw}}« erſcheinen}{\lemma{\textnormal{\emph{in … erscheinen}}}\Cendnote{\textnormal{Arthur Schnitzler: \emph{Les Emplettes de Noël}\pwindex{Schnitzler, Arthur 15.\,5.\,1862 Wien – 21.\,10.\,1931 ebd.@\textsc{Schnitzler, Arthur} (15.\,5.\,1862 Wien – 21.\,10.\,1931 ebd.), \emph{Schriftsteller, Mediziner}!Emplettes de Noël@\strich\emph{Les Emplettes de Noël}|pwk}. Übersetzt von Henri Albert\pwindex{Albert, Henri 16.\,11.\,1869 Niederbronn-les-Bains – 3.\,8.\,1921 Straßburg@\textsc{Albert, Henri} (16.\,11.\,1869 Niederbronn-les-Bains – 3.\,8.\,1921 Straßburg), \emph{Journalist, Kritiker, Übersetzer}|pwk}. In: \emph{L’Idée
                        libre. Revue mensuelle de Littérature et d’Art}\pwindex{Idée libre. Revue mensuelle de Littérature et d'Art@\emph{L’Idée libre. Revue mensuelle de Littérature et d'Art}|pwk}, Jg. 3, Nr. 5–6,
                        Mai–Juni 1984, S. 215–225. Am 21. 7. 1894 notierte Schnitzler in seinem \emph{Tagebuch}\pwindex{Schnitzler, Arthur 15.\,5.\,1862 Wien – 21.\,10.\,1931 ebd.@\textsc{Schnitzler, Arthur} (15.\,5.\,1862 Wien – 21.\,10.\,1931 ebd.), \emph{Schriftsteller, Mediziner}!Tagebuch@\strich\emph{Tagebuch}|pwk}: »Schlecht übersetzt.« Albert\pwindex{Albert, Henri 16.\,11.\,1869 Niederbronn-les-Bains – 3.\,8.\,1921 Straßburg@\textsc{Albert, Henri} (16.\,11.\,1869 Niederbronn-les-Bains – 3.\,8.\,1921 Straßburg), \emph{Journalist, Kritiker, Übersetzer}|pwk} gegenüber dürfte er aber ein anderes Urteil geäußert
                  haben, denn dieser antwortete ihm in einem Brief am 6. 8. 1894: »Dass Ihnen meine Uebersetzung\pwindex{Schnitzler, Arthur 15.\,5.\,1862 Wien – 21.\,10.\,1931 ebd.@\textsc{Schnitzler, Arthur} (15.\,5.\,1862 Wien – 21.\,10.\,1931 ebd.), \emph{Schriftsteller, Mediziner}!Emplettes de Noël@\strich\emph{Les Emplettes de Noël}|pwv} so gut gefallen hat, hat mich hoch
                     erfreut.« (\emph{DLA}, HS.1985.1.2331,3.)}}}\label{K_L02625-7} werden, da die
               andern auf Monat und Jahr hinaus keinen Platz haben.\pend
           
\pstart
           6.) \label{K_L02625-8v}\edtext{Lies »\textsc{Caligula\pwindex{Quidde, Ludwig 23.\,3.\,1858 Bremen – 5.\,3.\,1941@\textsc{Quidde, Ludwig} (23.\,3.\,1858 Bremen – 5.\,3.\,1941), \emph{Politiker, Herausgeber, Historiker}!Caligula – Eine Studie über römischen Cäsarenwahnsinn@\strich\emph{Caligula – Eine Studie über römischen Cäsarenwahnsinn}|pw}}« von \textsc{Quidde\pwindex{Quidde, Ludwig 23.\,3.\,1858 Bremen – 5.\,3.\,1941@\textsc{Quidde, Ludwig} (23.\,3.\,1858 Bremen – 5.\,3.\,1941), \emph{Politiker, Herausgeber, Historiker}|pw}}!}{\lemma{\textnormal{\emph{Lies … Quidde!}}}\Cendnote{\textnormal{Eine Lektüre der kleinen Studie
                  über den Cäsarenwahn durch Schnitzler, die
                  von den Zeitgenossinnen und Zeitgenossen als Schmähschrift gegen Wilhelm II.\pwindex{Wilhelm II. von Preußen 27.\,1.\,1859 Potsdam – 4.\,6.\,1941 Gemeente Utrechtse Heuvelrug@\textsc{Wilhelm II. von Preußen} (27.\,1.\,1859 Potsdam – 4.\,6.\,1941 Gemeente Utrechtse Heuvelrug), \emph{Kaiser}|pwk} gelesen wurde, ist bislang nicht belegt.}}}\label{K_L02625-8}\pend
           
\pstart
           7.) Viele treue Grüße! {\\[\baselineskip]}Dein {\\[\baselineskip]}\spacefill\mbox{Paul Goldmann}\pend
           \leftskip=0em{}\selectlanguage{ngerman}\endnumbering\briefempfaengerindex{Schnitzler, Arthur@\textsc{Schnitzler, Arthur}!zzzGoldmann, Paul@\emph{von Paul Goldmann}!1894-06-153@{15. 6. [1894]}|)be}\mylabel{L02625h}  \newcommand{\dateiname}{L02625}\newcommand{\titel}{Paul Goldmann an Arthur Schnitzler, 15. 6. [1894]}\newcommand{\editorInnen}{Martin Anton Müller und Laura Untner}%% latex-leseansicht-abspann.tex
%% Abspann für die Leseansicht.
%% Der Schalter \ifkorrekturansicht ist bereits durch den Vorspann gesetzt.

%% latex-abspann.tex
%% Gemeinsamer Abspann für Korrekturansicht und Leseansicht.
%% Setzt den Schalter \ifkorrekturansicht voraus (gesetzt in den
%% einbindenden Dateien latex-korrekturansicht-abspann.tex bzw.
%% latex-leseansicht-abspann.tex).
%% ---------------------------------------------------------------

\normalsize

% Das esempio-Environment wird nur in der Leseansicht benötigt
\ifkorrekturansicht\else
\newenvironment{esempio}[3]%
{
    \vspace{1.5ex}
    \rlap{\underline{#1}}
    \par
    \setlength{\parindent}{0cm}
    \nopagebreak
    \leftskip=#2cm
    \rightskip=#3cm
}
{
    \par
}
\fi

\doendnotes{C}
\bigskip
\vfill

\clearpage

\footnotesize

\ifkorrekturansicht
  \lohead{\textsc{register}}
\fi

% theindex-Environment neu definieren ohne reledmac
\makeatletter
\renewenvironment{theindex}{%
  \ifkorrekturansicht
    \section*{\indexname}%
  \else
    \subsubsection*{Index der erwähnten Entitäten}%
  \fi
  \setlength{\parindent}{0pt}%
  \setlength{\parskip}{0pt plus 0.3pt}%
  \let\item\@idxitem
}{%
  \ifkorrekturansicht\clearpage\fi
}
\makeatother

\IfFileExists{\jobname-pw.ind}{\input{\jobname-pw.ind}}{}

% Quellenangabe nur in der Leseansicht
\ifkorrekturansicht\else
% Fallback-Definitionen, falls die .tex-Datei \titel etc. nicht gesetzt hat
\providecommand{\titel}{}
\providecommand{\editorInnen}{}
\providecommand{\dateiname}{\jobname}

\vspace{3cm}

\vfill

\footnotesize
\textsc{Quelle}: \titel. Herausgegeben von {\editorInnen}. In: \emph{Arthur Schnitzler: Briefwechsel mit Autorinnen und Autoren}.
 Digitale Edition, https://schnitzler-briefe.acdh.oeaw.ac.at/{\dateiname}.html (Stand \today)
\fi

\end{document}


