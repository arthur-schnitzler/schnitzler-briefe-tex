%% latex-korrekturansicht-vorspann.tex
%% Vorspann für die Korrekturansicht.
%% Lädt die gemeinsame Datei latex-vorspann.tex mit gesetztem Schalter.

\newif\ifkorrekturansicht
\korrekturansichttrue

\input{../tex-inputs/latex-vorspann}


\section[Paul Goldmann an Arthur Schnitzler, 15. 6. {[}1894{]}]{L02625 Paul Goldmann an Arthur Schnitzler, 15. 6. {[}1894{]}}
\nopagebreak\mylabel{L02625v}
\rehead{ }\normalsize\beginnumbering\briefempfaengerindex{Schnitzler, Arthur@\textsc{Schnitzler, Arthur}!zzzGoldmann, Paul@\emph{von Paul Goldmann}!1894-06-153@{15. 6. {[}1894{]}}|(be}
\toendnotes[C]{\smallbreak\pagebreak[2]}\Standort{DLA, A:Schnitzler, HS.NZ85.1.3164.}
\physDesc{Brief, 1 Blatt, 3 Seiten, 1084 Zeichen
\newline{}Handschrift: schwarze Tinte, deutsche Kurrent
\newline{}Schnitzler: 1) mit Bleistift auf dem ersten Blatt die Jahreszahl »94« vermerkt  2) mit rotem Buntstift eine Unterstreichung}\toendnotes[C]{\smallbreak}
\pstart
           {\pb}\textcolor{gray}{\textbf{Frankfurter Zeitung\orgindex{Frankfurter Zeitung@Frankfurter Zeitung|pw}}}\pend
           
\pstart
           \textcolor{gray}{\textbf{(Gazette de
                     Francfort\orgindex{Frankfurter Zeitung@Frankfurter Zeitung|pw}).}}\pend
           
\pstart
           \textcolor{gray}{\textbf{Fondateur \textbf{M. L. Sonnemann\pwindex{Sonnemann, Leopold 1831-10-29 – 1909-10-30@\textsc{Sonnemann, Leopold} (1831-10-29 – 1909-10-30), \emph{Journalist/Journalistin, Herausgeber/Herausgeberin}|pw}}.}}\pend
           
\pstart
           \textcolor{gray}{\textbf{\begin{otherlanguage}{french}Journal politique, financier,\end{otherlanguage}}}\hfill \textsc{Paris\oindex{Paris@\textbf{Paris}, \emph{P.PPLC}|pw}}, 15. Juni.\pend
           
\pstart
           \textcolor{gray}{\textbf{\begin{otherlanguage}{french}commercial et littéraire.\end{otherlanguage}}}\pend
           
\pstart
           \textcolor{gray}{\textbf{\begin{otherlanguage}{french}\textbf{Paraissant trois fois par jour.}\end{otherlanguage}}}\pend
           
\pstart
           \textcolor{gray}{\textbf{\begin{otherlanguage}{french}\textbf{Bureau à Paris\oindex{Paris@\textbf{Paris}, \emph{P.PPLC}|pw}:}\end{otherlanguage}}}\pend
           
\pstart
           \textcolor{gray}{\textbf{\begin{otherlanguage}{french}24. Rue Feydeau\oindex{rue Feydeau@\textbf{rue Feydeau}, \emph{Straße (K.STR)}|pw}.\end{otherlanguage}}}\pend
           
\pstart{}Mein lieber Freund,\pend\vspace{0.5em}
\pstart
           Ich bin ſehr beſchäftigt. Darum nur wenige Zeilen.\pend
           
\pstart
           1.) Wärmſten Dank für Deinen lieben Brief aus \label{K_L02625-1v}\edtext{\textsc{Muenchen\oindex{Muenchen@\textbf{München}, \emph{P.PPLA}|pw}}}{\lemma{\textnormal{\emph{Muenchen}}}\Cendnote{\textnormal{Zwischen 2. 6. 1894 und 8. 6. 1894 hielt sich Schnitzler in München\oindex{Muenchen@\textbf{München}, \emph{P.PPLA}|pwk}
                  auf.}}}\label{K_L02625-1}. Er erklärt Manches und läßt Manches im Unklaren. All’ das iſt ſehr
               ſchwer brieflich abzumachen. Auch das, was mich erregt, läßt ſich kaum ſo
               niederſchreiben. Ich möchte mit Dir ſprechen, aber vielleicht iſt es am Beſten gar
               nicht mehr darüber zu {\pb}reden. Die Dinge müſſen ihren
               Lauf gehen.\pend
           
\pstart
           2.) Haſt Du die \label{K_L02625-2v}\edtext{»\textsc{Revue Blanche\pwindex{Revue blanche@\emph{La Revue blanche}|pw}}«}{\lemma{\textnormal{\emph{»Revue Blanche«}}}\Cendnote{\textnormal{Die wohl für den \emph{Mercure de France}\pwindex{Mercure de France@\emph{Mercure de France}|pwk} gedachte (siehe Paul Goldmann an Arthur Schnitzler, 29. 5. [1894]) Besprechung\pwindex{Lettres allemandes. Drames Nouveaux@\emph{Les Lettres allemandes. Drames Nouveaux}|pwkv} von Schnitzlers Schauspiel \emph{Das Märchen}\pwindex{Maerchen. Schauspiel in drei Aufzuegen@\emph{Das Märchen. Schauspiel in drei Aufzügen}|pwk}
                  erschien in der \emph{Revue blanche}\pwindex{Revue blanche@\emph{La Revue blanche}|pwk}: Henri Albert\pwindex{Albert, Henri 1869-11-16 – 1921-08-03@\textsc{Albert, Henri} (1869-11-16 – 1921-08-03), \emph{Journalist/Journalistin, Kritiker/Kritikerin, Übersetzer/Übersetzerin}|pwk}: \emph{Les Lettres allemandes. Drames Nouveaux}\pwindex{Lettres allemandes. Drames Nouveaux@\emph{Les Lettres allemandes. Drames Nouveaux}|pwk}. In: \emph{La Revue Blanche}\pwindex{Revue blanche@\emph{La Revue blanche}|pwk}, Jg. 6, Nr. 32,
                        Juni 1984, S. 556–560, hier S. 560. Dem \emph{Tagebuch}\pwindex{Tagebuch@\emph{Tagebuch}|pwk} ist zu entnehmen, dass Schnitzler die Besprechung\pwindex{Lettres allemandes. Drames Nouveaux@\emph{Les Lettres allemandes. Drames Nouveaux}|pwkv} gelesen hat (vgl. A. S.: \emph{Tagebuch}, 11. 6. 1894).}}}\label{K_L02625-2} erhalten.\pend
           
\pstart
           3.) Können wir \label{K_L02625-3v}\edtext{im Auguſt zuſammenreiſen}{\lemma{\textnormal{\emph{im Auguſt zuſammenreiſen}}}\Cendnote{\textnormal{Vom 23. 8. 1894 bis zum 3. 9. 1894
                  verbrachten Schnitzler und Goldmann\pwindex{Goldmann, Paul 31.01.1865 – 25.09.1935@\textsc{Goldmann, Paul} (31.01.1865 – 25.09.1935), \emph{Schriftsteller/Schriftstellerin, Journalist/Journalistin}|pwk} einige Zeit gemeinsam in Bad Ischl\oindex{Bad Ischl@\textbf{Bad Ischl}, \emph{P.PPL}|pwk} und Bad
                  Aussee\oindex{Bad Aussee@\textbf{Bad Aussee}, \emph{P.PPLA3}|pwk}.}}}\label{K_L02625-3}? Bitte, antworte mir umgehend, denn ich muß jetzt bereits
               anfangen, eventuelle Vorkehrungen zu treffen.\pend
           
\pstart
           4.) Was weißt Du von \textsc{Muenchen\oindex{Muenchen@\textbf{München}, \emph{P.PPLA}|pw}} zu erzählen? Haſt Du den \textsc{Altdorfer\pwindex{Altdorfer, Albrecht 1480 – 1538-02-12@\textsc{Altdorfer, Albrecht} (1480 – 1538-02-12), \emph{Maler/Malerin, Kupferstecher/Kupferstecherin, Baumeister/Baumeisterin}|pw}\pwindex{Laubwald mit dem heiligen Georg@\emph{Laubwald mit dem heiligen Georg}|pwv}} geſehen, von dem ich Dir \label{K_L02625-4v}\edtext{ſchrieb}{\lemma{\textnormal{\emph{ſchrieb}}}\Cendnote{\textnormal{Siehe Paul Goldmann an Arthur Schnitzler, 1. 6. [1894].
               }}}\label{K_L02625-4}? Wie gehts Dir \strikeout{J} geſundheitlich?\pend
           
\pstart
           {\pb}5.) \textsc{Herzl\pwindex{Herzl, Theodor 1860-05-02 – 1904-07-03@\textsc{Herzl, Theodor} (1860-05-02 – 1904-07-03), \emph{Schriftsteller/Schriftstellerin, Journalist/Journalistin}|pw}}, den ich verſchiedentlich von Dir gegrüßt, läßt Dich verſchiedentlich wieder
               grüßen. Desgleichen \textsc{Henri Albert\pwindex{Albert, Henri 1869-11-16 – 1921-08-03@\textsc{Albert, Henri} (1869-11-16 – 1921-08-03), \emph{Journalist/Journalistin, Kritiker/Kritikerin, Übersetzer/Übersetzerin}|pw}}. Ich habe dieſer Tage den \label{K_L02625-5v}\edtext{Bürſten-Abzug}{\lemma{\textnormal{\emph{Bürſten-Abzug}}}\Cendnote{\textnormal{Probeabzug}}}\label{K_L02625-5} der
                  \label{K_L02625-6v}\edtext{»\textsc{Emplettes de Noël\pwindex{Emplettes de Noel@\emph{Les Emplettes de Noël}|pw}}«}{\lemma{\textnormal{\emph{»Emplettes de Noël«}}}\Cendnote{\textnormal{Henri Alberts\pwindex{Albert, Henri 1869-11-16 – 1921-08-03@\textsc{Albert, Henri} (1869-11-16 – 1921-08-03), \emph{Journalist/Journalistin, Kritiker/Kritikerin, Übersetzer/Übersetzerin}|pwk}{ }Übersetzung\pwindex{Emplettes de Noel@\emph{Les Emplettes de Noël}|pwkv} von Schnitzlers{ }\emph{Anatol}\pwindex{Anatol@\emph{Anatol}|pwk}-Einakter \emph{Weihnachts-Einkäufe}\pwindex{Weihnachts-Einkaeufe@\emph{Weihnachts-Einkäufe}|pwk}}}}\label{K_L02625-6} geſehen, die \label{K_L02625-7v}\edtext{in der »\textsc{Idée Libre\pwindex{L'Idee libre. Revue mensuelle de Litterature et d'Art@\emph{L'Idée libre. Revue mensuelle de Littérature et d'Art}|pw}}« erſcheinen}{\lemma{\textnormal{\emph{in … erſcheinen}}}\Cendnote{\textnormal{Arthur Schnitzler: \emph{Les Emplettes de Noël}\pwindex{Emplettes de Noel@\emph{Les Emplettes de Noël}|pwk}. Übersetzt von Henri Albert\pwindex{Albert, Henri 1869-11-16 – 1921-08-03@\textsc{Albert, Henri} (1869-11-16 – 1921-08-03), \emph{Journalist/Journalistin, Kritiker/Kritikerin, Übersetzer/Übersetzerin}|pwk}. In: \emph{L'Idée
                        libre. Revue mensuelle de Littérature et d'Art}\pwindex{L'Idee libre. Revue mensuelle de Litterature et d'Art@\emph{L'Idée libre. Revue mensuelle de Littérature et d'Art}|pwk}, Jg. 3, Nr. 5–6,
                        Mai–Juni 1984, S. 215–225. Am 21. 7. 1894 notierte Schnitzler in seinem \emph{Tagebuch}\pwindex{Tagebuch@\emph{Tagebuch}|pwk}: »Schlecht übersetzt.« Albert\pwindex{Albert, Henri 1869-11-16 – 1921-08-03@\textsc{Albert, Henri} (1869-11-16 – 1921-08-03), \emph{Journalist/Journalistin, Kritiker/Kritikerin, Übersetzer/Übersetzerin}|pwk} gegenüber dürfte er aber ein anderes Urteil geäußert
                  haben, denn dieser antwortete ihm in einem Brief am 6. 8. 1894: »Dass Ihnen meine Uebersetzung\pwindex{Emplettes de Noel@\emph{Les Emplettes de Noël}|pwv} so gut gefallen hat, hat mich hoch
                     erfreut.« (\emph{DLA}, HS.1985.1.2331,3.)}}}\label{K_L02625-7} werden, da die
               andern auf Monat und Jahr hinaus keinen Platz haben.\pend
           
\pstart
           6.) \label{K_L02625-8v}\edtext{Lies »\textsc{Caligula\pwindex{Caligula – Eine Studie ueber roemischen Caesarenwahnsinn@\emph{Caligula – Eine Studie über römischen Cäsarenwahnsinn}|pw}}« von \textsc{Quidde\pwindex{Quidde, Ludwig 1858-03-23 – 1941-03-05@\textsc{Quidde, Ludwig} (1858-03-23 – 1941-03-05), \emph{Politiker/Politikerin, Herausgeber/Herausgeberin, Historiker/Historikerin}|pw}}!}{\lemma{\textnormal{\emph{Lies … Quidde!}}}\Cendnote{\textnormal{Eine Lektüre der kleinen Studie
                  über den Cäsarenwahn durch Schnitzler, die
                  von den Zeitgenossinnen und Zeitgenossen als Schmähschrift gegen Wilhelm II.\pwindex{Wilhelm II. von Preussen 27.1.1859 – 4.6.1941@\textsc{Wilhelm II. von Preußen} (27.1.1859 – 4.6.1941), \emph{Kaiser/Kaiserin}|pwk} gelesen wurde, ist bislang nicht belegt.}}}\label{K_L02625-8}\pend
           
\pstart
           7.) Viele treue Grüße! {\\[\baselineskip]}Dein {\\[\baselineskip]}\spacefill\mbox{Paul Goldmann}\pend
           \leftskip=0em{}\selectlanguage{ngerman}\endnumbering\briefempfaengerindex{Schnitzler, Arthur@\textsc{Schnitzler, Arthur}!zzzGoldmann, Paul@\emph{von Paul Goldmann}!1894-06-153@{15. 6. {[}1894{]}}|)be}\mylabel{L02625h}  \normalsize

\doendnotes{C}
\bigskip
\vfill

\clearpage

\footnotesize

\lohead{\textsc{register}}

% Definiere theindex-Environment komplett neu ohne reledmac
\makeatletter
\renewenvironment{theindex}{%
  \section*{\indexname}%
  \setlength{\parindent}{0pt}%
  \setlength{\parskip}{0pt plus 0.3pt}%
  \let\item\@idxitem
}{%
  \clearpage
}
\makeatother

\IfFileExists{\jobname-pw.ind}{\input{\jobname-pw.ind}}{}

\end{document}

      