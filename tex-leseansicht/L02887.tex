%% latex-leseansicht-vorspann.tex
%% Vorspann für die Leseansicht.
%% Lädt die gemeinsame Datei latex-vorspann.tex mit nicht gesetztem Schalter.

\newif\ifkorrekturansicht
\korrekturansichtfalse

\input{../tex-inputs/latex-vorspann}


         
         \renewcommand{\erwaehntePersonen}{Personen: Johann Peter Eckermann, Johann Wolfgang von Goethe, Paul Goldmann, Friedrich von Müller}
         \renewcommand{\erwaehnteInstitutionen}{Institutionen: F. A. Brockhaus (Leipzig), J.G. Cotta’sche Buchhandlung Nachfolger}
         \renewcommand{\erwaehnteOrte}{Orte: Florenz, Hôtel National, Italien, Leipzig, Mailand, Piazza della Scala, Stuttgart, Wiesbaden}
         \renewcommand{\erwaehnteWerke}{Werke: Der Schleier der Beatrice. Schauspiel in fünf Akten, Der Weg ins Freie. Roman, Gespräche mit Goethe in den letzten Jahren seines Lebens. 3 Bde., Goethes Unterhaltungen mit dem Kanzler Friedrich von Müller, Tagebuch}
               \section[ Paul Goldmann an Arthur Schnitzler, 25. 9. {[}1899{]}]{ Paul Goldmann an Arthur Schnitzler, 25. 9. {[}1899{]}}\nopagebreak\mylabel{v}\rehead{ }\begin{ledgroupsized}[t]{13cm}\normalsize\beginnumbering\briefempfaengerindex{Schnitzler, Arthur@\textsc{Schnitzler, Arthur}!zzzGoldmann, Paul@\emph{von Paul Goldmann}!1899-09-251@{25. 9. {[}1899{]}}|(be} \toendnotes[C]{\smallbreak\pagebreak[2]} \Standort{DLA, A:Schnitzler, HS.NZ85.1.3169.}
\physDesc{Brief, 1 Blatt, 3 Seiten, 1017 Zeichen
\newline{}Handschrift: schwarze Tinte, deutsche Kurrent
\newline{}Schnitzler: 1) mit Bleistift das Jahr »99« vermerkt  2) mit rotem Buntstift eine Unterstreichung}\toendnotes[C]{\smallbreak}\pstart
           \noindent{}{\pb}\textcolor{gray}{\textbf{\textbf{HÔTEL NATIONAL\oindex{Hôtel National@\textbf{Hôtel National}|pw}}}}\hfill \textcolor{gray}{\textbf{\textit{\label{K_L02887-1v}\edtext{\begin{otherlanguage}{french}VENTE DES BILLETS DE CHEMIN DE FER\end{otherlanguage}}{\lemma{\textnormal{\emph{Vente … fer}}}\Cendnote{\textnormal{französisch: Verkauf von
                              Eisenbahnbillets }}}\label{K_L02887-1h}}}}\pend
           \pstart
           \textcolor{gray}{\textbf{\begin{otherlanguage}{french}\so{MILAN}\oindex{Mailand@\textbf{Mailand}|pw}\end{otherlanguage}}}\hfill \textcolor{gray}{\textbf{\textit{\label{K_L02887-2v}\edtext{\begin{otherlanguage}{french}BUREAU DE POSTE DANS LA MAISON\end{otherlanguage}}{\lemma{\textnormal{\emph{Bureau … maison}}}\Cendnote{\textnormal{französisch: Postamt im
                              Haus}}}\label{K_L02887-2h}}}}\pend
           \pstart
           Place de la Scala\oindex{Piazza della Scala@\textbf{Piazza della Scala}|pw}\hfill \textcolor{gray}{\textbf{\textit{\begin{otherlanguage}{french}COOK\textsuperscript{S}
                              COUPONS\end{otherlanguage}}}}\pend
           \pstart
           \textcolor{gray}{\textbf{\emph{\begin{otherlanguage}{french}Lumière Eléctrique\end{otherlanguage}}}}\hfill Mailand\oindex{Mailand@\textbf{Mailand}|pw}{ }25. September.\pend
           \pstart\center{}Mein lieber Freund,\pend\pstart
           Wie geht es Dir? Biſt Du wieder hergeſtellt? Wie fühlſt Du Dich in \label{K_L02887-3v}\edtext{Wiesbaden\oindex{Wiesbaden@\textbf{Wiesbaden}|pw}}{\lemma{\textnormal{\emph{Wiesbaden}}}\Cendnote{\textnormal{Schnitzler\pwindex{Schnitzler, Arthur 15.05.1862 – 21.10.1931@\textsc{Schnitzler, Arthur} (15.05.1862 – 21.10.1931), \emph{Schriftsteller, Mediziner}|pwk} war zwischen 24. 9. 1899 und 3. 10. 1899 in Wiesbaden\oindex{Wiesbaden@\textbf{Wiesbaden}|pwk}. Dem \emph{Tagebuch}\pwindex{Schnitzler, Arthur 15.05.1862 – 21.10.1931@\textsc{Schnitzler, Arthur} (15.05.1862 – 21.10.1931), \emph{Schriftsteller, Mediziner}!Tagebuch1981 – 2000@\strich\emph{Tagebuch} {[}1981 – 2000{]}|pwk} ist zu entnehmen, dass er in dieser Zeit an dem Text, der zum
                  Roman \emph{Der Weg ins Freie}\pwindex{Schnitzler, Arthur 15.05.1862 – 21.10.1931@\textsc{Schnitzler, Arthur} (15.05.1862 – 21.10.1931), \emph{Schriftsteller, Mediziner}!Weg ins Freie. Roman1.1.1908 – 1.6.1908@\strich\emph{Der Weg ins Freie. Roman} {[}1.1.1908 – 1.6.1908{]}|pwk} (vgl. A. S.: \emph{Tagebuch}, 27. 9. 1899) wurde, und dem
                  Schauspiel \emph{Der Schleier der Beatrice}\pwindex{Schnitzler, Arthur 15.05.1862 – 21.10.1931@\textsc{Schnitzler, Arthur} (15.05.1862 – 21.10.1931), \emph{Schriftsteller, Mediziner}!Schleier der Beatrice. Schauspiel in fuenf Akten1900-12-01@\strich\emph{Der Schleier der Beatrice. Schauspiel in fünf Akten} {[}1900-12-01{]}|pwk} (vgl. A. S.: \emph{Tagebuch}, 2. 10. 1899)
                  arbeitete.}}}\label{K_L02887-3h}? Rückt die Arbeit vom Fleck? Und haſt Du wieder Talent?\pend
           \pstart
           Hier iſt Sommer, – helles, frohes Licht und linde Luft. Du \strikeout{haſ} hätteſt Dir doch einen Ruck geben und {\pb}mitkommen ſollen. Es hätte Dir wohlgethan. Und dieſes ſanfte Entzücken in dieſem
                  Italien\oindex{Italien@\textbf{Italien}|pw}! Und dieſe Fülle des Lebens in Mailand\oindex{Mailand@\textbf{Mailand}|pw}!\pend
           \pstart
           Während der Fahrt las ich mit hohem Genuß \label{K_L02887-4v}\edtext{\textsc{Mueller\pwindex{Mueller, Friedrich von 1779-04-13 – 1849-10-21@\textsc{Müller, Friedrich von} (1779-04-13 – 1849-10-21), \emph{Politiker}|pw}s}{ }Geſpräche mit \textsc{Goethe\pwindex{Goethe, Johann Wolfgang von 1749-08-28 – 1832-03-22@\textsc{Goethe, Johann Wolfgang von} (1749-08-28 – 1832-03-22), \emph{Schriftsteller}|pw}}\pwindex{Mueller, Friedrich von 1779-04-13 – 1849-10-21@\textsc{Müller, Friedrich von} (1779-04-13 – 1849-10-21), \emph{Politiker}!Goethes Unterhaltungen mit dem Kanzler Friedrich von Mueller1870@\strich\emph{Goethes Unterhaltungen mit dem Kanzler Friedrich von Müller} {[}1870{]}|pwv}}{\lemma{\textnormal{\emph{Muellers … Goethe}}}\Cendnote{\textnormal{Friedrich von Müller\pwindex{Mueller, Friedrich von 1779-04-13 – 1849-10-21@\textsc{Müller, Friedrich von} (1779-04-13 – 1849-10-21), \emph{Politiker}|pwk}: \emph{Goethes Unterhaltungen mit dem Kanzler Friedrich von
                        Müller}\pwindex{Mueller, Friedrich von 1779-04-13 – 1849-10-21@\textsc{Müller, Friedrich von} (1779-04-13 – 1849-10-21), \emph{Politiker}!Goethes Unterhaltungen mit dem Kanzler Friedrich von Mueller1870@\strich\emph{Goethes Unterhaltungen mit dem Kanzler Friedrich von Müller} {[}1870{]}|pwk}. Stuttgart\oindex{Stuttgart@\textbf{Stuttgart}|pwk}: \emph{Cotta}\orgindex{J.G. Cotta sche Buchhandlung Nachfolger@J.G. Cotta’sche Buchhandlung Nachfolger|pwk}{ }1870.}}}\label{K_L02887-4h}. Das iſt kein für die Unſterblichkeit zurecht gemachter \textsc{Goethe\pwindex{Goethe, Johann Wolfgang von 1749-08-28 – 1832-03-22@\textsc{Goethe, Johann Wolfgang von} (1749-08-28 – 1832-03-22), \emph{Schriftsteller}|pw}}, wie der \strikeout{v}{ }\label{K_L02887-5v}\edtext{\textsc{Eckermann\pwindex{Eckermann, Johann Peter 21.09.1792 – 03.12.1854@\textsc{Eckermann, Johann Peter} (21.09.1792 – 03.12.1854), \emph{Sekretär}!Gespraeche mit Goethe in den letzten Jahren seines Lebens. 3 Bde.1836 – 1848@\strich\emph{Gespräche mit Goethe in den letzten Jahren seines Lebens. 3 Bde.} {[}1836 – 1848{]}|pwv}\pwindex{Eckermann, Johann Peter 21.09.1792 – 03.12.1854@\textsc{Eckermann, Johann Peter} (21.09.1792 – 03.12.1854), \emph{Sekretär}|pw}s}}{\lemma{\textnormal{\emph{Eckermanns}}}\Cendnote{\textnormal{Johann Peter Eckermann\pwindex{Eckermann, Johann Peter 21.09.1792 – 03.12.1854@\textsc{Eckermann, Johann Peter} (21.09.1792 – 03.12.1854), \emph{Sekretär}|pwk}: \emph{Gespräche mit Goethe in den letzten Jahren seines
                        Lebens}\pwindex{Eckermann, Johann Peter 21.09.1792 – 03.12.1854@\textsc{Eckermann, Johann Peter} (21.09.1792 – 03.12.1854), \emph{Sekretär}!Gespraeche mit Goethe in den letzten Jahren seines Lebens. 3 Bde.1836 – 1848@\strich\emph{Gespräche mit Goethe in den letzten Jahren seines Lebens. 3 Bde.} {[}1836 – 1848{]}|pwk}. 3 Bde. Leipzig\oindex{Leipzig@\textbf{Leipzig}|pwk}: \emph{Brockhaus}\orgindex{F. A. Brockhaus (Leipzig)@F. A. Brockhaus (Leipzig)|pwk}{ }1836, 1848.}}}\label{K_L02887-5h}, ſondern \textsc{Goethe\pwindex{Goethe, Johann Wolfgang von 1749-08-28 – 1832-03-22@\textsc{Goethe, Johann Wolfgang von} (1749-08-28 – 1832-03-22), \emph{Schriftsteller}|pw}} als Menſch, mit all’ ſeinen \strikeout{Sch\textcolor{gray}{w}} Schwächen auch und manchen Widerwärtigkeiten. Selbſt Antiſemit war er, der
               Schuft! \textsc{Mueller\pwindex{Mueller, Friedrich von 1779-04-13 – 1849-10-21@\textsc{Müller, Friedrich von} (1779-04-13 – 1849-10-21), \emph{Politiker}!Goethes Unterhaltungen mit dem Kanzler Friedrich von Mueller1870@\strich\emph{Goethes Unterhaltungen mit dem Kanzler Friedrich von Müller} {[}1870{]}|pwv}\pwindex{Mueller, Friedrich von 1779-04-13 – 1849-10-21@\textsc{Müller, Friedrich von} (1779-04-13 – 1849-10-21), \emph{Politiker}|pw}} ſieht ihn nicht als Gott an, wie \textsc{Eckermann\pwindex{Eckermann, Johann Peter 21.09.1792 – 03.12.1854@\textsc{Eckermann, Johann Peter} (21.09.1792 – 03.12.1854), \emph{Sekretär}!Gespraeche mit Goethe in den letzten Jahren seines Lebens. 3 Bde.1836 – 1848@\strich\emph{Gespräche mit Goethe in den letzten Jahren seines Lebens. 3 Bde.} {[}1836 – 1848{]}|pwv}\pwindex{Eckermann, Johann Peter 21.09.1792 – 03.12.1854@\textsc{Eckermann, Johann Peter} (21.09.1792 – 03.12.1854), \emph{Sekretär}|pw}}, {\pb}ſondern fühlt ſich ihm mehr gleich und iſt
               darum kritiſcher. Und doch wieder, alle die goldenen Worte, die das Buch\pwindex{Mueller, Friedrich von 1779-04-13 – 1849-10-21@\textsc{Müller, Friedrich von} (1779-04-13 – 1849-10-21), \emph{Politiker}!Goethes Unterhaltungen mit dem Kanzler Friedrich von Mueller1870@\strich\emph{Goethes Unterhaltungen mit dem Kanzler Friedrich von Müller} {[}1870{]}|pwv} enthält! {\dots}\pend
           \pstart
           Schreib mir nach \begin{otherlanguage}{italian}\textsc{Firenze\oindex{Florenz@\textbf{Florenz}|pw}, \label{K_L02887-6v}\edtext{ferma in posta}{\lemma{\textnormal{\emph{ferma in posta}}}\Cendnote{\textnormal{italienisch: postlagernd}}}\label{K_L02887-6h}}\end{otherlanguage}!\pend
           \pstart
           Viele treue Grüße! {\\[\baselineskip]}Dein {\\[\baselineskip]}\spacefill\mbox{Paul Goldmnn}\pend
           \leftskip=0em{}
         
         \endnumbering\mylabel{h}\end{ledgroupsized}  \newcommand{\dateiname}{L02887}\newcommand{\titel}{Paul Goldmann an Arthur Schnitzler, 25. 9. [1899]}\newcommand{\editorInnen}{Martin Anton Müller und Laura Untner}%% latex-leseansicht-abspann.tex
%% Abspann für die Leseansicht.
%% Der Schalter \ifkorrekturansicht ist bereits durch den Vorspann gesetzt.

%% latex-abspann.tex
%% Gemeinsamer Abspann für Korrekturansicht und Leseansicht.
%% Setzt den Schalter \ifkorrekturansicht voraus (gesetzt in den
%% einbindenden Dateien latex-korrekturansicht-abspann.tex bzw.
%% latex-leseansicht-abspann.tex).
%% ---------------------------------------------------------------

\normalsize

% Das esempio-Environment wird nur in der Leseansicht benötigt
\ifkorrekturansicht\else
\newenvironment{esempio}[3]%
{
    \vspace{1.5ex}
    \rlap{\underline{#1}}
    \par
    \setlength{\parindent}{0cm}
    \nopagebreak
    \leftskip=#2cm
    \rightskip=#3cm
}
{
    \par
}
\fi

\doendnotes{C}
\bigskip
\vfill

\clearpage

\footnotesize

\ifkorrekturansicht
  \lohead{\textsc{register}}
\fi

% theindex-Environment neu definieren ohne reledmac
\makeatletter
\renewenvironment{theindex}{%
  \ifkorrekturansicht
    \section*{\indexname}%
  \else
    \subsubsection*{Index der erwähnten Entitäten}%
  \fi
  \setlength{\parindent}{0pt}%
  \setlength{\parskip}{0pt plus 0.3pt}%
  \let\item\@idxitem
}{%
  \ifkorrekturansicht\clearpage\fi
}
\makeatother

\IfFileExists{\jobname-pw.ind}{\input{\jobname-pw.ind}}{}

% Quellenangabe nur in der Leseansicht
\ifkorrekturansicht\else
% Fallback-Definitionen, falls die .tex-Datei \titel etc. nicht gesetzt hat
\providecommand{\titel}{}
\providecommand{\editorInnen}{}
\providecommand{\dateiname}{\jobname}

\vspace{3cm}

\vfill

\footnotesize
\textsc{Quelle}: \titel. Herausgegeben von {\editorInnen}. In: \emph{Arthur Schnitzler: Briefwechsel mit Autorinnen und Autoren}.
 Digitale Edition, https://schnitzler-briefe.acdh.oeaw.ac.at/{\dateiname}.html (Stand \today)
\fi

\end{document}


      