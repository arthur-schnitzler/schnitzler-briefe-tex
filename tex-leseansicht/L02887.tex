%% latex-korrekturansicht-vorspann.tex
%% Vorspann für die Korrekturansicht.
%% Lädt die gemeinsame Datei latex-vorspann.tex mit gesetztem Schalter.

\newif\ifkorrekturansicht
\korrekturansichttrue

\input{../tex-inputs/latex-vorspann}


\section[ Paul Goldmann an Arthur Schnitzler, 25. 9. {[}1899{]}]{L02887 Paul Goldmann an Arthur Schnitzler, 25. 9. {[}1899{]}}
\nopagebreak\mylabel{L02887v}
\rehead{ }\normalsize\beginnumbering\briefempfaengerindex{Schnitzler, Arthur@\textsc{Schnitzler, Arthur}!zzzGoldmann, Paul@\emph{von Paul Goldmann}!1899-09-251@{25. 9. {[}1899{]}}|(be}
\toendnotes[C]{\smallbreak\pagebreak[2]}\Standort{DLA, A:Schnitzler, HS.NZ85.1.3169.}
\physDesc{Brief, 1 Blatt, 3 Seiten, 1017 Zeichen
\newline{}Handschrift: schwarze Tinte, deutsche Kurrent
\newline{}Schnitzler: 1) mit Bleistift das Jahr »99« vermerkt  2) mit rotem Buntstift eine Unterstreichung}\toendnotes[C]{\smallbreak}
\pstart
           {\pb}\textcolor{gray}{\textbf{\textbf{HÔTEL NATIONAL\oindex{Hôtel National@\textbf{Hôtel National}, \emph{Hotel (K.HTL)}|pw}}}}\hfill \textcolor{gray}{\textbf{\textit{\label{K_L02887-1v}\edtext{\begin{otherlanguage}{french}VENTE DES BILLETS DE CHEMIN DE FER\end{otherlanguage}}{\lemma{\textnormal{\emph{Vente … fer}}}\Cendnote{\textnormal{französisch: Verkauf von
                              Eisenbahnbillets }}}\label{K_L02887-1}}}}\pend
           
\pstart
           \textcolor{gray}{\textbf{\begin{otherlanguage}{french}\so{MILAN}\oindex{Mailand@\textbf{Mailand}, \emph{P.PPLA}|pw}\end{otherlanguage}}}\hfill \textcolor{gray}{\textbf{\textit{\label{K_L02887-2v}\edtext{\begin{otherlanguage}{french}BUREAU DE POSTE DANS LA MAISON\end{otherlanguage}}{\lemma{\textnormal{\emph{Bureau … maison}}}\Cendnote{\textnormal{französisch: Postamt im
                              Haus}}}\label{K_L02887-2}}}}\pend
           
\pstart
           Place de la Scala\oindex{Piazza della Scala@\textbf{Piazza della Scala}, \emph{Platz (K.PLT)}|pw}\hfill \textcolor{gray}{\textbf{\textit{\begin{otherlanguage}{french}COOK\textsuperscript{S}
                              COUPONS\end{otherlanguage}}}}\pend
           
\pstart
           \textcolor{gray}{\textbf{\emph{\begin{otherlanguage}{french}Lumière Eléctrique\end{otherlanguage}}}}\hfill Mailand\oindex{Mailand@\textbf{Mailand}, \emph{P.PPLA}|pw}{ }25. September.\pend
           
\pstart\center{}Mein lieber Freund,\pend\vspace{0.5em}
\pstart
           Wie geht es Dir? Biſt Du wieder hergeſtellt? Wie fühlſt Du Dich in \label{K_L02887-3v}\edtext{Wiesbaden\oindex{Wiesbaden@\textbf{Wiesbaden}, \emph{P.PPLA}|pw}}{\lemma{\textnormal{\emph{Wiesbaden}}}\Cendnote{\textnormal{Schnitzler war zwischen 24. 9. 1899 und 3. 10. 1899 in Wiesbaden\oindex{Wiesbaden@\textbf{Wiesbaden}, \emph{P.PPLA}|pwk}. Dem \emph{Tagebuch}\pwindex{Tagebuch@\emph{Tagebuch}|pwk} ist zu entnehmen, dass er in dieser Zeit an dem Text, der zum
                  Roman \emph{Der Weg ins Freie}\pwindex{Weg ins Freie. Roman@\emph{Der Weg ins Freie. Roman}|pwk} (vgl. A. S.: \emph{Tagebuch}, 27. 9. 1899) wurde, und dem
                  Schauspiel \emph{Der Schleier der Beatrice}\pwindex{Schleier der Beatrice. Schauspiel in fuenf Akten@\emph{Der Schleier der Beatrice. Schauspiel in fünf Akten}|pwk} (vgl. A. S.: \emph{Tagebuch}, 2. 10. 1899)
                  arbeitete.}}}\label{K_L02887-3}? Rückt die Arbeit vom Fleck? Und haſt Du wieder Talent?\pend
           
\pstart
           Hier iſt Sommer, – helles, frohes Licht und linde Luft. Du \strikeout{haſ} hätteſt Dir doch einen Ruck geben und {\pb}mitkommen ſollen. Es hätte Dir wohlgethan. Und dieſes ſanfte Entzücken in dieſem
                  Italien\oindex{Italien@\textbf{Italien}, \emph{A.PCLI}|pw}! Und dieſe Fülle des Lebens in Mailand\oindex{Mailand@\textbf{Mailand}, \emph{P.PPLA}|pw}!\pend
           
\pstart
           Während der Fahrt las ich mit hohem Genuß \label{K_L02887-4v}\edtext{\textsc{Muellers\pwindex{Mueller, Friedrich von 1779-04-13 – 1849-10-21@\textsc{Müller, Friedrich von} (1779-04-13 – 1849-10-21), \emph{Kanzler/Kanzlerin}|pw}}{ }Geſpräche mit \textsc{Goethe\pwindex{Goethe, Johann Wolfgang von 1749-08-28 – 1832-03-22@\textsc{Goethe, Johann Wolfgang von} (1749-08-28 – 1832-03-22), \emph{Schriftsteller/Schriftstellerin}|pw}}\pwindex{Goethes Unterhaltungen mit dem Kanzler Friedrich von Mueller@\emph{Goethes Unterhaltungen mit dem Kanzler Friedrich von Müller}|pwv}}{\lemma{\textnormal{\emph{Muellers … Goethe}}}\Cendnote{\textnormal{Friedrich von Müller\pwindex{Mueller, Friedrich von 1779-04-13 – 1849-10-21@\textsc{Müller, Friedrich von} (1779-04-13 – 1849-10-21), \emph{Kanzler/Kanzlerin}|pwk}: \emph{Goethes Unterhaltungen mit dem Kanzler Friedrich von
                        Müller}\pwindex{Goethes Unterhaltungen mit dem Kanzler Friedrich von Mueller@\emph{Goethes Unterhaltungen mit dem Kanzler Friedrich von Müller}|pwk}. Stuttgart\oindex{Stuttgart@\textbf{Stuttgart}, \emph{P.PPLA}|pwk}: \emph{Cotta}\orgindex{J.G. Cotta sche Buchhandlung Nachfolger@J.G. Cotta’sche Buchhandlung Nachfolger|pwk}{ }1870.}}}\label{K_L02887-4}. Das iſt kein für die Unſterblichkeit zurecht gemachter \textsc{Goethe\pwindex{Goethe, Johann Wolfgang von 1749-08-28 – 1832-03-22@\textsc{Goethe, Johann Wolfgang von} (1749-08-28 – 1832-03-22), \emph{Schriftsteller/Schriftstellerin}|pw}}, wie der \strikeout{v}{ }\label{K_L02887-5v}\edtext{\textsc{Eckermanns\pwindex{Gespraeche mit Goethe in den letzten Jahren seines Lebens@\emph{Gespräche mit Goethe in den letzten Jahren seines Lebens}|pwv}\pwindex{Eckermann, Johann Peter 21.09.1792 – 03.12.1854@\textsc{Eckermann, Johann Peter} (21.09.1792 – 03.12.1854), \emph{Sekretär/Sekretärin}|pw}}}{\lemma{\textnormal{\emph{Eckermanns}}}\Cendnote{\textnormal{Johann Peter Eckermann\pwindex{Eckermann, Johann Peter 21.09.1792 – 03.12.1854@\textsc{Eckermann, Johann Peter} (21.09.1792 – 03.12.1854), \emph{Sekretär/Sekretärin}|pwk}: \emph{Gespräche mit Goethe in den letzten Jahren seines
                        Lebens}\pwindex{Gespraeche mit Goethe in den letzten Jahren seines Lebens@\emph{Gespräche mit Goethe in den letzten Jahren seines Lebens}|pwk}. 3 Bde. Leipzig\oindex{Leipzig@\textbf{Leipzig}, \emph{P.PPLA3}|pwk}: \emph{Brockhaus}\orgindex{F. A. Brockhaus [Leipzig]@F. A. Brockhaus [Leipzig]|pwk}{ }1836, 1848.}}}\label{K_L02887-5}, ſondern \textsc{Goethe\pwindex{Goethe, Johann Wolfgang von 1749-08-28 – 1832-03-22@\textsc{Goethe, Johann Wolfgang von} (1749-08-28 – 1832-03-22), \emph{Schriftsteller/Schriftstellerin}|pw}} als Menſch, mit all’ ſeinen \strikeout{Sch\textcolor{gray}{w}} Schwächen auch und manchen Widerwärtigkeiten. Selbſt Antiſemit war er, der
               Schuft! \textsc{Mueller\pwindex{Goethes Unterhaltungen mit dem Kanzler Friedrich von Mueller@\emph{Goethes Unterhaltungen mit dem Kanzler Friedrich von Müller}|pwv}\pwindex{Mueller, Friedrich von 1779-04-13 – 1849-10-21@\textsc{Müller, Friedrich von} (1779-04-13 – 1849-10-21), \emph{Kanzler/Kanzlerin}|pw}} ſieht ihn nicht als Gott an, wie \textsc{Eckermann\pwindex{Gespraeche mit Goethe in den letzten Jahren seines Lebens@\emph{Gespräche mit Goethe in den letzten Jahren seines Lebens}|pwv}\pwindex{Eckermann, Johann Peter 21.09.1792 – 03.12.1854@\textsc{Eckermann, Johann Peter} (21.09.1792 – 03.12.1854), \emph{Sekretär/Sekretärin}|pw}}, {\pb}ſondern fühlt ſich ihm mehr gleich und iſt
               darum kritiſcher. Und doch wieder, alle die goldenen Worte, die das Buch\pwindex{Goethes Unterhaltungen mit dem Kanzler Friedrich von Mueller@\emph{Goethes Unterhaltungen mit dem Kanzler Friedrich von Müller}|pwv} enthält! {\dots}\pend
           
\pstart
           Schreib mir nach \begin{otherlanguage}{italian}\textsc{Firenze\oindex{Florenz@\textbf{Florenz}, \emph{P.PPLA}|pw}, \label{K_L02887-6v}\edtext{ferma in posta}{\lemma{\textnormal{\emph{ferma in posta}}}\Cendnote{\textnormal{italienisch: postlagernd}}}\label{K_L02887-6}}\end{otherlanguage}!\pend
           
\pstart
           Viele treue Grüße! {\\[\baselineskip]}Dein {\\[\baselineskip]}\spacefill\mbox{Paul Goldmnn}\pend
           \leftskip=0em{}\selectlanguage{ngerman}\endnumbering\briefempfaengerindex{Schnitzler, Arthur@\textsc{Schnitzler, Arthur}!zzzGoldmann, Paul@\emph{von Paul Goldmann}!1899-09-251@{25. 9. {[}1899{]}}|)be}\mylabel{L02887h}  \normalsize

\doendnotes{C}
\bigskip
\vfill

\clearpage

\footnotesize

\lohead{\textsc{register}}

% Definiere theindex-Environment komplett neu ohne reledmac
\makeatletter
\renewenvironment{theindex}{%
  \section*{\indexname}%
  \setlength{\parindent}{0pt}%
  \setlength{\parskip}{0pt plus 0.3pt}%
  \let\item\@idxitem
}{%
  \clearpage
}
\makeatother

\IfFileExists{\jobname-pw.ind}{\input{\jobname-pw.ind}}{}

\end{document}

      