%% latex-leseansicht-vorspann.tex
%% Vorspann für die Leseansicht.
%% Lädt die gemeinsame Datei latex-vorspann.tex mit nicht gesetztem Schalter.

\newif\ifkorrekturansicht
\korrekturansichtfalse

\input{../tex-inputs/latex-vorspann}


\section[Arthur Schnitzler an Berta Zuckerkandl, 20. 6. 1911]{L03950 Arthur Schnitzler an Berta Zuckerkandl, 20. 6. 1911}
\nopagebreak\mylabel{L03950v}
\rehead{ }\normalsize\beginnumbering\briefempfaengerindex{Zuckerkandl, Berta@\textsc{Zuckerkandl, Berta}!zzzSchnitzler, Arthur@\emph{von Arthur Schnitzler}!1911-06-201@{20. 6. 1911}|(be}
\toendnotes[C]{\smallbreak\pagebreak[2]}
\correspDesc{Versand  durch Arthur Schnitzler am 20. 6. 1911 in Wien
\newline{}Erhalt  durch Berta Zuckerkandl am 22. 6. 1911 in Paris}\toendnotes[C]{\smallbreak}
\Standort{DLA, HS.1985.1.2282.}
\physDesc{Brief, Durchschlag, 1 Blatt, 2 Seiten, 951 Zeichen
\newline{}Schreibmaschine
\newline{}Handschrift Schreibkraft: Bleistift, deutsche Kurrent (\noindent{}Monats- und Jahresangabe: »Juni 1911.«,
                      Korrekturen)
\newline{}Handschrift Arthur Schnitzler: roter Buntstift, lateinische Kurrent (\noindent{}beschriftet: »\uline{Zuckerkandl}« und
                      »Frk«, vier Unterstreichungen)}\toendnotes[C]{\smallbreak}
\pstart
           \raggedleft{}{\pb}20.{ }\introOben{}Juni 1911.\introOben{}\pend
           
\pstart{} Verehrte gnädige Frau.\pend\vspace{0.5em}
\pstart
           Hier sende ich Ihnen das \label{K_L03950-1v}\edtext{Scenarium des
            »Medardus\pwindex{Schnitzler, Arthur 15.\,5.\,1862 Wien – 21.\,10.\,1931 ebd.@\textsc{Schnitzler, Arthur} (15.\,5.\,1862 Wien – 21.\,10.\,1931 ebd.), \emph{Schriftsteller, Mediziner}!junge Medardus. Dramatische Historie in einem Vorspiel und fünf
            Aufzügen@\strich\emph{Der junge Medardus. Dramatische Historie in einem Vorspiel und fünf Aufzügen}|pw}« }{\lemma{\textnormal{\emph{Scenarium des
            »Medardus« }}}\Cendnote{\textnormal{Die Beilage des Briefes ist nicht überliefert. Schnitzler vermerkte am 16. 6. 1911 im \emph{Tagebuch}\pwindex{Schnitzler, Arthur 15.\,5.\,1862 Wien – 21.\,10.\,1931 ebd.@\textsc{Schnitzler, Arthur} (15.\,5.\,1862 Wien – 21.\,10.\,1931 ebd.), \emph{Schriftsteller, Mediziner}!Tagebuch@\strich\emph{Tagebuch}|pwk} die Arbeit am Szenarium und am 20. 6. 1911 dessen Absenden an Berta Zuckerkandl\pwindex{Zuckerkandl, Berta 13.\,4.\,1864 Wien – 16.\,10.\,1945 Paris@\textsc{Zuckerkandl, Berta} (13.\,4.\,1864 Wien – 16.\,10.\,1945 Paris), \emph{Schriftstellerin, Journalistin, Übersetzerin}|pwk}.}}}\label{K_L03950-1}ein und hoffe, dass es Ihren Wünschen
          leidlich entsprechen dürfte. Die am Schlusse angefügte Bemerkung dürfte wohl der Erwägung
          wert sein. Hoffentlich habe ich bald das Vergnügen mehr von der Sache zu hören und danke
          Ihnen für Ihr Interesse sowohl als für Ihre Bemühungen.\pend
           
\pstart
           Der Ordnung wegen möchte ich es hier auch noch schriftlich niederlegen, dass ich für den
          Fall, dass wir zu einem Resultate kommen sollten, auf die Hälfte der Tantiemen und der
          eventuellen anderen Honorare, Buchausgabe und dergl., Anspruch mache. Wegen der Verteilung
          der andern Hälfte werden sie sich wohl selbst, verehrte gnädige Frau, mit Ihrer
          präsumptiven Hilfskraft ins Einvernehmen setzen. Ich schreibe an den Verlag\orgindex{S. Fischer Verlag@S. Fischer Verlag|pwv} wegen der an Sie zu sendenden Exemplare\pwindex{Schnitzler, Arthur 15.\,5.\,1862 Wien – 21.\,10.\,1931 ebd.@\textsc{Schnitzler, Arthur} (15.\,5.\,1862 Wien – 21.\,10.\,1931 ebd.), \emph{Schriftsteller, Mediziner}!junge Medardus. Dramatische Historie in einem Vorspiel und fünf
            Aufzügen@\strich\emph{Der junge Medardus. Dramatische Historie in einem Vorspiel und fünf Aufzügen}|pwv}. Es stehen Ihnen natürlich
          nach Bedarf auch weitere zur Verfügung.\pend
           
\pstart
           {\pb}Mit herzlichen Grüssen{\\[\baselineskip]} Ihr sehr ergebener\pend
           \leftskip=0em{}{\vspace{1\baselineskip}}
\pstart
           \noindent{}Frau Berta Zuckerkandl, \label{K_L03950-2v}\edtext{Wien\oindex{Wien@\textbf{Wien}, \emph{Verwaltungsgebiet}|pw}}{\lemma{\textnormal{\emph{Wien}}}\Cendnote{\textnormal{Aus dem \emph{Tagebuch}\pwindex{Schnitzler, Arthur 15.\,5.\,1862 Wien – 21.\,10.\,1931 ebd.@\textsc{Schnitzler, Arthur} (15.\,5.\,1862 Wien – 21.\,10.\,1931 ebd.), \emph{Schriftsteller, Mediziner}!Tagebuch@\strich\emph{Tagebuch}|pwk}eintrag vom 20. 6. 1911 geht hervor, dass Schnitzler den Brief nicht nach Wien\oindex{Wien@\textbf{Wien}, \emph{Verwaltungsgebiet}|pwk}
              sondern nach Paris\oindex{Paris@\textbf{Paris}, \emph{Hauptstadt}|pwk} sandte, wohin Zuckerkandl\pwindex{Zuckerkandl, Berta 13.\,4.\,1864 Wien – 16.\,10.\,1945 Paris@\textsc{Zuckerkandl, Berta} (13.\,4.\,1864 Wien – 16.\,10.\,1945 Paris), \emph{Schriftstellerin, Journalistin, Übersetzerin}|pwk} am vorangegangenen Samstag aufzubrechen geplant
              hatte, vgl. XXXX Auszeichnungsfehler: Dokument L03996 nicht gefunden.}}}\label{K_L03950-2}.\pend
           \selectlanguage{ngerman}\endnumbering\briefempfaengerindex{Zuckerkandl, Berta@\textsc{Zuckerkandl, Berta}!zzzSchnitzler, Arthur@\emph{von Arthur Schnitzler}!1911-06-201@{20. 6. 1911}|)be}\mylabel{L03950h}
\begin{anhang}
\end{anhang}\newcommand{\dateiname}{L03950}\newcommand{\titel}{Arthur Schnitzler an Berta Zuckerkandl, 20. 6. 1911}\newcommand{\editorInnen}{Herausgegeben von Jahnke, SelmaMüller, Martin Anton}%% latex-leseansicht-abspann.tex
%% Abspann für die Leseansicht.
%% Der Schalter \ifkorrekturansicht ist bereits durch den Vorspann gesetzt.

%% latex-abspann.tex
%% Gemeinsamer Abspann für Korrekturansicht und Leseansicht.
%% Setzt den Schalter \ifkorrekturansicht voraus (gesetzt in den
%% einbindenden Dateien latex-korrekturansicht-abspann.tex bzw.
%% latex-leseansicht-abspann.tex).
%% ---------------------------------------------------------------

\normalsize

% Das esempio-Environment wird nur in der Leseansicht benötigt
\ifkorrekturansicht\else
\newenvironment{esempio}[3]%
{
    \vspace{1.5ex}
    \rlap{\underline{#1}}
    \par
    \setlength{\parindent}{0cm}
    \nopagebreak
    \leftskip=#2cm
    \rightskip=#3cm
}
{
    \par
}
\fi

\doendnotes{C}
\bigskip
\vfill

\clearpage

\footnotesize

\ifkorrekturansicht
  \lohead{\textsc{register}}
\fi

% theindex-Environment neu definieren ohne reledmac
\makeatletter
\renewenvironment{theindex}{%
  \ifkorrekturansicht
    \section*{\indexname}%
  \else
    \subsubsection*{Index der erwähnten Entitäten}%
  \fi
  \setlength{\parindent}{0pt}%
  \setlength{\parskip}{0pt plus 0.3pt}%
  \let\item\@idxitem
}{%
  \ifkorrekturansicht\clearpage\fi
}
\makeatother

\IfFileExists{\jobname-pw.ind}{\input{\jobname-pw.ind}}{}

% Quellenangabe nur in der Leseansicht
\ifkorrekturansicht\else
% Fallback-Definitionen, falls die .tex-Datei \titel etc. nicht gesetzt hat
\providecommand{\titel}{}
\providecommand{\editorInnen}{}
\providecommand{\dateiname}{\jobname}

\vspace{3cm}

\vfill

\footnotesize
\textsc{Quelle}: \titel. Herausgegeben von {\editorInnen}. In: \emph{Arthur Schnitzler: Briefwechsel mit Autorinnen und Autoren}.
 Digitale Edition, https://schnitzler-briefe.acdh.oeaw.ac.at/{\dateiname}.html (Stand \today)
\fi

\end{document}


