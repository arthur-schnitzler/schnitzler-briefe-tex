%% latex-korrekturansicht-vorspann.tex
%% Vorspann für die Korrekturansicht.
%% Lädt die gemeinsame Datei latex-vorspann.tex mit gesetztem Schalter.

\newif\ifkorrekturansicht
\korrekturansichttrue

\input{../tex-inputs/latex-vorspann}


\section[Elsa Plessner an Arthur Schnitzler, 17. 10. 1896]{L03705 Elsa Plessner an Arthur Schnitzler, 17. 10. 1896}
\nopagebreak\mylabel{L03705v}
\rehead{ }\normalsize\beginnumbering\briefempfaengerindex{Schnitzler, Arthur@\textsc{Schnitzler, Arthur}!zzzPlessner, Elsa@\emph{von Elsa Plessner}!1896-10-173@{17. 10. 1896}|(be}
\toendnotes[C]{\smallbreak\pagebreak[2]}\Standort{DLA, A:Schnitzler, HS.1985.1.419.}
\physDesc{Brief,  Blätter, 3 Seiten, 1246 Zeichen
\newline{}Handschrift: , lateinische Kurrent
\newline{}Schnitzler: zwei Unterstreichungen }\toendnotes[C]{\smallbreak}
\pstart
           {\pb} I. Bäckerstrasse N\textsuperscript{o}
                     1\oindex{Baeckerstrasse 1@\textbf{Bäckerstraße 1}, \emph{Wohngebäude (K.WHS)}|pw}, den 1.17. 10. 96.\pend
           
\pstart{}Hochverehrter Herr Doctor!\pend\vspace{0.5em}
\pstart
           Gestern Abends beim \label{K_L03705-1v}\edtext{Dörmann\pwindex{Doermann, Felix 29.05.1870 – 26.10.1928@\textsc{Dörmann, Felix} (29.05.1870 – 26.10.1928), \emph{Schriftsteller/Schriftstellerin}|pw}-Premièren\pwindex{Sein Sohn. Schauspiel in vier Acten@\emph{Sein Sohn. Schauspiel in vier Acten}|pwv}feste\eventindex{Raimund-Theater@\textbf{Raimund-Theater}!Urauffuehrung von Sein Sohn, 16.10.1896@Uraufführung von Sein Sohn, 16.10.1896|pw}}{\lemma{\textnormal{\emph{Dörmann-Premièrenfeste}}}\Cendnote{\textnormal{Am 16. 10. 1896 hatte im Raimund-Theater\oindex{Raimund-Theater@\textbf{Raimund-Theater}, \emph{Theater (K.THE)}|pwk} die Uraufführung von Felix
                     Dörmanns\pwindex{Doermann, Felix 29.05.1870 – 26.10.1928@\textsc{Dörmann, Felix} (29.05.1870 – 26.10.1928), \emph{Schriftsteller/Schriftstellerin}|pwk} Drama \emph{Sein Sohn}\pwindex{Sein Sohn. Schauspiel in vier Acten@\emph{Sein Sohn. Schauspiel in vier Acten}|pwk}
                  stattgefunden, die auch Schnitzler besucht
                  hatte, vgl. A. S.: \emph{Tagebuch}, 16. 10. 1896.}}}\label{K_L03705-1}
               erfuhr ich von Herrn Dr. Leo Hirschfeld\pwindex{Feld, Leo 14.02.1869 – 05.09.1924@\textsc{Feld, Leo} (14.02.1869 – 05.09.1924), \emph{Schriftsteller/Schriftstellerin, Übersetzer/Übersetzerin, Dirigent/Dirigentin}|pw}, dass
               Director Brahm\pwindex{Brahm, Otto 05.02.1856 – 28.11.1912@\textsc{Brahm, Otto} (05.02.1856 – 28.11.1912), \emph{Theaterleiter/Theaterleiterin, Regisseur/Regisseurin}|pw} sich in Wien\oindex{Wien@\textbf{Wien}, \emph{A.ADM2}|pw} befindet. Sie können sich denken, wie erstaunt und erfreut
               ich war, denn ich ventilirte mit Mama\pwindex{Plessner, Clementine 1855-12-07 – 1943-02-27@\textsc{Plessner, Clementine} (1855-12-07 – 1943-02-27), \emph{Schauspieler/Schauspielerin, Filmschauspieler/Filmschauspielerin}|pwv} bereits die Frage einer kurzen Reise nach {\pb}Berlin\oindex{Berlin@\textbf{Berlin}, \emph{P.PPLC}|pw}. Da Sie mir einmal den keinen Finger
               gereicht haben, so bitte ich Sie, jetzt, falls Sie meine Arbeit\pwindex{Heimweh [dreiaktige Tragikomoedie]@\emph{Heimweh [dreiaktige Tragikomödie]}|pwv} dessen würdig erachten, Ihre ganze,
               vielvermögende Hand dabei ins Spiel zu bringen und mir mitzutheilen, ob und wie ich
               mit Herrn Director Brahm\pwindex{Brahm, Otto 05.02.1856 – 28.11.1912@\textsc{Brahm, Otto} (05.02.1856 – 28.11.1912), \emph{Theaterleiter/Theaterleiterin, Regisseur/Regisseurin}|pw} diesbezüglich \introOben{}(meiner Arbeit\pwindex{Heimweh [dreiaktige Tragikomoedie]@\emph{Heimweh [dreiaktige Tragikomödie]}|pwv})\introOben{} mich in \strikeout{D} directes
               Einvernehmen setzen soll. Sie sind doch einmal der gute Geist – {\pb}der
               liebe Herrgott muss sich noch viel mehr Bitten gefallen lassen! Von Dankbarkeit und
               s. w. will und kann ich Ihnen nicht reden, weil wir doch Beide wißen, was dran ist –
               aber wenn ich auch nicht rede – Sie werden sehen – – !! Wirklich! Verehrter, einziger
               Herr Doctor, wenn Sie mir den Herrn Director\pwindex{Brahm, Otto 05.02.1856 – 28.11.1912@\textsc{Brahm, Otto} (05.02.1856 – 28.11.1912), \emph{Theaterleiter/Theaterleiterin, Regisseur/Regisseurin}|pwv} auf \strikeout{45} 1 Stunde
               festnageln könnten, daß ich ihm mein Stück\pwindex{Heimweh [dreiaktige Tragikomoedie]@\emph{Heimweh [dreiaktige Tragikomödie]}|pwv} vorlese – –  wenn Sie das thun würden!! Geht's? – Sie
               haben doch so viel Einfluß!! – Bitte!\pend
           
\pstart
           N. B. Ohne Unbescheidenheit. Ich soll \uline{gut}
               vorlesen wie man sagt! – – Bitte um Nachricht! – \label{K_L03705-2v}\edtext{\begin{otherlanguage}{french}Sans phrase\end{otherlanguage}}{\lemma{\textnormal{\emph{Sans phrase}}}\Cendnote{\textnormal{französisch: ohne Umschweife}}}\label{K_L03705-2} in
               Ewigkeit ergeben\pend
           \pstart \spacefill\mbox{Elsa Plessner}\pend{}\selectlanguage{ngerman}\endnumbering\briefempfaengerindex{Schnitzler, Arthur@\textsc{Schnitzler, Arthur}!zzzPlessner, Elsa@\emph{von Elsa Plessner}!1896-10-173@{17. 10. 1896}|)be}\mylabel{L03705h}
\begin{anhang}
\end{anhang}\normalsize

\doendnotes{C}
\bigskip
\vfill

\clearpage

\footnotesize

\lohead{\textsc{register}}

% Definiere theindex-Environment komplett neu ohne reledmac
\makeatletter
\renewenvironment{theindex}{%
  \section*{\indexname}%
  \setlength{\parindent}{0pt}%
  \setlength{\parskip}{0pt plus 0.3pt}%
  \let\item\@idxitem
}{%
  \clearpage
}
\makeatother

\IfFileExists{\jobname-pw.ind}{\input{\jobname-pw.ind}}{}

\end{document}

      