%% latex-korrekturansicht-vorspann.tex
%% Vorspann für die Korrekturansicht.
%% Lädt die gemeinsame Datei latex-vorspann.tex mit gesetztem Schalter.

\newif\ifkorrekturansicht
\korrekturansichttrue

\input{../tex-inputs/latex-vorspann}


\section[Georg Brandes an Arthur Schnitzler, 5. 3. 1913]{L02116 Georg Brandes an Arthur Schnitzler, 5. 3. 1913}
\nopagebreak\mylabel{L02116v}
\rehead{ }\normalsize\beginnumbering\briefempfaengerindex{Schnitzler, Arthur@\textsc{Schnitzler, Arthur}!zzzBrandes, Georg@\emph{von Georg Brandes}!1913-03-051@{5. 3. 1913}|(be}
\toendnotes[C]{\smallbreak\pagebreak[2]}\Standort{CUL, Schnitzler, B 17.}
\physDesc{Postkarte, 1388 Zeichen
\newline{}Handschrift: schwarze Tinte, lateinische Kurrent
\newline{}Versand: Stempel: »\nobreak{}\oindex{Taormina@\textbf{Taormina}, \emph{P.PPLA3}|pwk}Taormina Messina, 6 3 13\nobreak{}«.  
\newline{}Ordnung: mit Bleistift von unbekannter Hand nummeriert:
                                    »41« }
\buchAbdrucke{\weitereDrucke{Georg Brandes, Arthur Schnitzler: \emph{Ein Briefwechsel}. Bern: \emph{Francke} 1956, S. 107.} }\toendnotes[C]{\smallbreak}\pstart{}{\pb}Herrn Dr. Arthur
                  Schnitzler\pend{}\pstart{}71 Sternwartestrasse Wien XVIII\oindex{Sternwartestrasse 71@\textbf{Sternwartestraße 71}, \emph{Wohngebäude (K.WHS)}|pw}\pend{}\pstart{}Vienna\oindex{Wien@\textbf{Wien}, \emph{A.ADM2}|pw}\pend{}\pstart{}Austria\oindex{Oesterreich@\textbf{Österreich}, \emph{A.PCLI}|pw}\pend{}{\bigskip}\vspace{1em}
\pstart
           \raggedleft{}{\pb}5 März 13\pend
           
\pstart{}Mein verehrtester Freund\pend\vspace{0.5em}
\pstart
           Ich erhalte hier (Hotel Métropole, Taormina\oindex{Grand Hotel Metropol@\textbf{Grand Hotel Metropol}, \emph{Hotel (K.HTL)}|pw})
               Ihren liebenswürdigen Brief, der mir zeigt, dass ich Unrecht hatte zu glauben, was
               die Professorin Zuckerkandl\pwindex{Zuckerkandl, Berta 13.04.1864 – 16.10.1945@\textsc{Zuckerkandl, Berta} (13.04.1864 – 16.10.1945), \emph{Journalist/Journalistin, Übersetzer/Übersetzerin}|pw} mir in Wien\oindex{Wien@\textbf{Wien}, \emph{A.ADM2}|pw} über den Anlass Ihres Schauspiels\pwindex{Professor Bernhardi. Komoedie in fuenf Akten@\emph{Professor Bernhardi. Komödie in fünf Akten}|pwv} erzählte. Ich bitte Sie meinen
               Irrthum zu entschuldigen. Man sollte nie Vertrauen an dergleichen Mittheilungen
               haben.\pend
           
\pstart
           Ich habe nie die Uebersetzung jenes vor Monaten geschriebenen Artikels\pwindex{Theater und Schauspiele in Deutschland@\emph{Theater und Schauspiele in Deutschland}|pwv} gesehen, und ich hatte sogar ganz
               vergessen, dass ich vor Monaten den Photographen\pwindex{?? [Fotograf in Paris] *~1912@\textsc{?? [Fotograf in Paris]} (*~1912)|pwv} in Paris\oindex{Paris@\textbf{Paris}, \emph{P.PPLC}|pw}
               bat, Ihnen mein Bild\pwindex{Georg Brandes]@\emph{[Georg Brandes]}|pwv} zu
               senden.\pend
           
\pstart
           Es geht mir mit Ihnen heute, wie es mir wöchentlich mit meiner liebsten Freundin\pwindex{Knudtzon, Bertha 24.02.1850 – 04.05.1923@\textsc{Knudtzon, Bertha} (24.02.1850 – 04.05.1923)|pwv} geht, die
               augenblicklich, auf einer Seereise begriffen, sich in Hongkong\oindex{Hong Kong@\textbf{Hong Kong}, \emph{P.PPLC}|pw} befindet. Wenn Ihre Antworten kommen, verstehe ich sie
               kaum, weil ich meine alten Briefe ganz vergessen habe.\pend
           
\pstart
           Ich war nach Paris\oindex{Paris@\textbf{Paris}, \emph{P.PPLC}|pw} in Pallanza\oindex{Pallanza@\textbf{Pallanza}, \emph{P.PPL}|pw}, Rom\oindex{Rom@\textbf{Rom}, \emph{P.PPLC}|pw}, Neapel\oindex{Neapel@\textbf{Neapel}, \emph{P.PPLA}|pw}, Palermo\oindex{Palermo@\textbf{Palermo}, \emph{P.PPLA}|pw} und längere Zeit in Tunis\oindex{Tunis@\textbf{Tunis}, \emph{P.PPLC}|pw}, das
               mir sehr gefiel trotz des ungünstigsten Wetters.\pend
           
\pstart
           Ich soll im April in Neapel\oindex{Neapel@\textbf{Neapel}, \emph{P.PPLA}|pw} und Rom\oindex{Rom@\textbf{Rom}, \emph{P.PPLC}|pw} reden, denke etwa am 1 Mai in Kopenhagen\oindex{Kopenhagen@\textbf{Kopenhagen}, \emph{P.PPLC}|pw} zurück zu sein. Hier bleibe ich
               ungefähr {\pb}drei Wochen. Hier hab
               ich endlich Sonne gefunden.\pend
           
\pstart
           Habe ich mich auch unrichtig ausgedrückt, können Sie wenigstens nicht meine
               freundschaftliche Gesinnung bezweifeln.\pend
           
\pstart
           Ihre werthe und liebe Frau Gemahlin\pwindex{Schnitzler, Olga 17.01.1882 – 13.01.1970@\textsc{Schnitzler, Olga} (17.01.1882 – 13.01.1970), \emph{Schauspieler/Schauspielerin, Sänger/Sängerin}|pwv} und die beiden mir so lieben Beer-Hofmanns\pwindex{Beer-Hofmann, Richard 1866-07-11 – 1945-09-26@\textsc{Beer-Hofmann, Richard} (1866-07-11 – 1945-09-26), \emph{Schriftsteller/Schriftstellerin}|pw}\pwindex{Beer-Hofmann, Paula 25.02.1879 – 30.10.1939@\textsc{Beer-Hofmann, Paula} (25.02.1879 – 30.10.1939)|pw} bitte ich an mich zu erinnern.\pend
           \pstart Ihr ergebener \spacefill\mbox{Georg Brandes}\pend{}\selectlanguage{ngerman}\endnumbering\briefempfaengerindex{Schnitzler, Arthur@\textsc{Schnitzler, Arthur}!zzzBrandes, Georg@\emph{von Georg Brandes}!1913-03-051@{5. 3. 1913}|)be}\mylabel{L02116h}  \normalsize

\doendnotes{C}
\bigskip
\vfill

\clearpage

\footnotesize

\lohead{\textsc{register}}

% Definiere theindex-Environment komplett neu ohne reledmac
\makeatletter
\renewenvironment{theindex}{%
  \section*{\indexname}%
  \setlength{\parindent}{0pt}%
  \setlength{\parskip}{0pt plus 0.3pt}%
  \let\item\@idxitem
}{%
  \clearpage
}
\makeatother

\IfFileExists{\jobname-pw.ind}{\input{\jobname-pw.ind}}{}

\end{document}

      