%% latex-leseansicht-vorspann.tex
%% Vorspann für die Leseansicht.
%% Lädt die gemeinsame Datei latex-vorspann.tex mit nicht gesetztem Schalter.

\newif\ifkorrekturansicht
\korrekturansichtfalse

\input{../tex-inputs/latex-vorspann}


               \section[Georg Brandes an Arthur Schnitzler, 5. 3. 1913]{ Georg Brandes an Arthur Schnitzler, 5. 3. 1913}\nopagebreak\mylabel{v}\rehead{ }\begin{ledgroupsized}[t]{13cm}\normalsize\beginnumbering\briefempfaengerindex{Schnitzler, Arthur@\textsc{Schnitzler, Arthur}!zzzBrandes, Georg@\emph{von Georg Brandes}!1913-03-051@{5. 3. 1913}|(be} \toendnotes[C]{\smallbreak\pagebreak[2]} \Standort{CUL, Schnitzler, B 17.}
\physDesc{Postkarte
\newline{}Handschrift: schwarze Tinte, lateinische Kurrent\newline{}Versand: Stempel: »\nobreak{}\oindex{Taormina@\textbf{Taormina}|pwk}Taormina Messina, 6 3 13\nobreak{}«.  \newline{}Ordnung: mit Bleistift von unbekannter Hand nummeriert: »41« }\buchAbdrucke{\weitereDrucke{Georg Brandes, Arthur Schnitzler: \emph{Ein Briefwechsel}. Hg. Kurt Bergel. Bern: \emph{Francke} 1956, S. 107.} }\toendnotes[C]{\smallbreak}\pstart{}{\pb}Herrn Dr. Arthur
                        Schnitzler\pend{}\pstart{}71 Sternwartestrasse Wien XVIII\oindex{Sternwartestrasse@\textbf{Sternwartestraße}|pw}\pend{}\pstart{}Vienna\oindex{Wien@\textbf{Wien}|pw}\pend{}\pstart{}Austria\oindex{Oesterreich@\textbf{Österreich}|pw}\pend{}{\bigskip}\pstart
           \raggedleft{}{\pb}5 März 13\pend
           \pstart{}Mein verehrtester Freund\pend\pstart
           Ich erhalte hier (Hotel Métropole, Taormina\oindex{Grand Hotel Metropol@\textbf{Grand Hotel Metropol}|pw})
                    Ihren liebenswürdigen Brief, der mir zeigt, dass ich Unrecht hatte zu glauben,
                    was die Professorin Zuckerkandl\pwindex{Zuckerkandl, Berta 13.04.1864 – 16.10.1945@\textsc{Zuckerkandl, Berta} (13.04.1864 – 16.10.1945), \emph{Journalistin, Übersetzerin}|pw} mir in Wien\oindex{Wien@\textbf{Wien}|pw} über den Anlass Ihres Schauspiels\pwindex{Schnitzler, Arthur 15.05.1862 – 21.10.1931@\textsc{Schnitzler, Arthur} (15.05.1862 – 21.10.1931), \emph{Schriftsteller, Mediziner}!Professor Bernhardi. Komoedie in fuenf Akten1912@\strich\emph{Professor Bernhardi. Komödie in fünf Akten} {[}1912{]}|pwv} erzählte. Ich bitte Sie
                    meinen Irrthum zu entschuldigen. Man sollte nie Vertrauen an dergleichen
                    Mittheilungen haben.\pend
           \pstart
           Ich habe nie die Uebersetzung jenes vor Monaten geschriebenen Artikels\pwindex{Brandes, Georg 04.02.1842 – 19.02.1927@\textsc{Brandes, Georg} (04.02.1842 – 19.02.1927)!Theater und Schauspiele in Deutschland01. 02. 1913@\strich\emph{Theater und Schauspiele in Deutschland} {[}01. 02. 1913{]}|pwv} gesehen, und ich hatte sogar
                    ganz vergessen, dass ich vor Monaten den Photographen\pwindex{?? [Fotograf in Paris] *~1912@\textsc{?? [Fotograf in Paris]} (*~1912)|pwv} in Paris\oindex{Paris@\textbf{Paris}|pw} bat, Ihnen mein Bild\pwindex{?? [Fotograf in Paris] *~1912@\textsc{?? [Fotograf in Paris]} (*~1912)!Georg Brandes]1913@\strich\emph{[Georg Brandes]} {[}1913{]}|pwv} zu senden.\pend
           \pstart
           Es geht mir mit Ihnen heute, wie es mir wöchentlich mit meiner liebsten Freundin\pwindex{Knudtzon, Bertha 24.02.1850 – 04.05.1923@\textsc{Knudtzon, Bertha} (24.02.1850 – 04.05.1923)|pwv} geht, die
                    augenblicklich, auf einer Seereise begriffen, sich in Hongkong\oindex{Hong Kong@\textbf{Hong Kong}|pw} befindet. Wenn Ihre Antworten kommen, verstehe
                    ich sie kaum, weil ich meine alten Briefe ganz vergessen habe.\pend
           \pstart
           Ich war nach Paris\oindex{Paris@\textbf{Paris}|pw} in Pallanza\oindex{Pallanza@\textbf{Pallanza}|pw}, Rom\oindex{Rom@\textbf{Rom}|pw}, Neapel\oindex{Neapel@\textbf{Neapel}|pw}, Palermo\oindex{Palermo@\textbf{Palermo}|pw} und längere Zeit in Tunis\oindex{Tunis@\textbf{Tunis}|pw},
                    das mir sehr gefiel trotz des ungünstigsten Wetters.\pend
           \pstart
           Ich soll im April in Neapel\oindex{Neapel@\textbf{Neapel}|pw} und Rom\oindex{Rom@\textbf{Rom}|pw} reden, denke etwa am 1 Mai in
                        Kopenhagen\oindex{Kopenhagen@\textbf{Kopenhagen}|pw} zurück zu sein. Hier bleibe
                    ich ungefähr {\pb}drei Wochen.
                    Hier hab ich endlich Sonne gefunden.\pend
           \pstart
           Habe ich mich auch unrichtig ausgedrückt, können Sie wenigstens nicht meine
                    freundschaftliche Gesinnung bezweifeln.\pend
           \pstart
           Ihre werthe und liebe Frau Gemahlin\pwindex{Schnitzler, Olga 17.01.1882 – 13.01.1970@\textsc{Schnitzler, Olga} (17.01.1882 – 13.01.1970), \emph{Schauspielerin, Sängerin}|pwv} und die beiden mir so lieben Beer-Hofmanns\pwindex{Beer-Hofmann, Richard 11.07.1866 – 26.09.1945@\textsc{Beer-Hofmann, Richard} (11.07.1866 – 26.09.1945), \emph{Schriftsteller}|pw}\pwindex{Beer-Hofmann, Paula 25.02.1879 – 30.10.1939@\textsc{Beer-Hofmann, Paula} (25.02.1879 – 30.10.1939)|pw} bitte ich an mich zu erinnern.\pend
           \pstart Ihr ergebener \spacefill\mbox{Georg Brandes}\pend{}\endnumbering\briefempfaengerindex{Schnitzler, Arthur@\textsc{Schnitzler, Arthur}!zzzBrandes, Georg@\emph{von Georg Brandes}!1913-03-051@{5. 3. 1913}|)be}\mylabel{h}\end{ledgroupsized}  \newcommand{\dateiname}{L02116}\newcommand{\titel}{Georg Brandes an Arthur Schnitzler, 5. 3. 1913}\newcommand{\editorInnen}{Martin Anton Müller und Gerd-Hermann Susen}%% latex-leseansicht-abspann.tex
%% Abspann für die Leseansicht.
%% Der Schalter \ifkorrekturansicht ist bereits durch den Vorspann gesetzt.

%% latex-abspann.tex
%% Gemeinsamer Abspann für Korrekturansicht und Leseansicht.
%% Setzt den Schalter \ifkorrekturansicht voraus (gesetzt in den
%% einbindenden Dateien latex-korrekturansicht-abspann.tex bzw.
%% latex-leseansicht-abspann.tex).
%% ---------------------------------------------------------------

\normalsize

% Das esempio-Environment wird nur in der Leseansicht benötigt
\ifkorrekturansicht\else
\newenvironment{esempio}[3]%
{
    \vspace{1.5ex}
    \rlap{\underline{#1}}
    \par
    \setlength{\parindent}{0cm}
    \nopagebreak
    \leftskip=#2cm
    \rightskip=#3cm
}
{
    \par
}
\fi

\doendnotes{C}
\bigskip
\vfill

\clearpage

\footnotesize

\ifkorrekturansicht
  \lohead{\textsc{register}}
\fi

% theindex-Environment neu definieren ohne reledmac
\makeatletter
\renewenvironment{theindex}{%
  \ifkorrekturansicht
    \section*{\indexname}%
  \else
    \subsubsection*{Index der erwähnten Entitäten}%
  \fi
  \setlength{\parindent}{0pt}%
  \setlength{\parskip}{0pt plus 0.3pt}%
  \let\item\@idxitem
}{%
  \ifkorrekturansicht\clearpage\fi
}
\makeatother

\IfFileExists{\jobname-pw.ind}{\input{\jobname-pw.ind}}{}

% Quellenangabe nur in der Leseansicht
\ifkorrekturansicht\else
% Fallback-Definitionen, falls die .tex-Datei \titel etc. nicht gesetzt hat
\providecommand{\titel}{}
\providecommand{\editorInnen}{}
\providecommand{\dateiname}{\jobname}

\vspace{3cm}

\vfill

\footnotesize
\textsc{Quelle}: \titel. Herausgegeben von {\editorInnen}. In: \emph{Arthur Schnitzler: Briefwechsel mit Autorinnen und Autoren}.
 Digitale Edition, https://schnitzler-briefe.acdh.oeaw.ac.at/{\dateiname}.html (Stand \today)
\fi

\end{document}


      