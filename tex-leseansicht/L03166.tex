%% latex-leseansicht-vorspann.tex
%% Vorspann für die Leseansicht.
%% Lädt die gemeinsame Datei latex-vorspann.tex mit nicht gesetztem Schalter.

\newif\ifkorrekturansicht
\korrekturansichtfalse

\input{../tex-inputs/latex-vorspann}


         
         \renewcommand{\erwaehntePersonen}{Personen: Giuseppe Giacosa, Ludmilla Karplus, Siegmund Karplus}
         \renewcommand{\erwaehnteOrte}{Orte: Burgtheater, Ronacher, Wien}
         \renewcommand{\erwaehnteWerke}{Werke: Liebelei. Schauspiel in drei Akten, Rechte der Seele. Schauspiel in einem Act}
               \section[ Felix Salten an Arthur Schnitzler, {[}16. 11. 1895{]}]{ Felix Salten an Arthur Schnitzler, {[}16. 11. 1895{]}}\nopagebreak\mylabel{v}\rehead{ }\begin{ledgroupsized}[t]{13cm}\normalsize\beginnumbering \toendnotes[C]{\smallbreak\pagebreak[2]} \Standort{CUL, Schnitzler, B 89, A 1.}
\physDesc{Brief, 1 Blatt, 1 Seite, 147 Zeichen
\newline{}Handschrift: Bleistift, lateinische Kurrent
\newline{}Schnitzler: mit Bleistift datiert: »16/11 95« 
\newline{}Ordnung: mit Bleistift von unbekannter Hand nummeriert: »66« }\toendnotes[C]{\smallbreak}\pstart
           \noindent{}{\pb}Ich will Ihnen nur sagen:\pend
           \settowidth{\longeste}{Sonntag, denx24.}\settowidth{\longestz}{»Rechte der Seele«}\settowidth{\longestd}{}\settowidth{\longestv}{}\settowidth{\longestf}{}\addtolength\longeste{1em}
        \addtolength\longestz{1em}
      \pstart\noindent\makebox[\the\longeste][l]{\label{K_L03166-1v}\edtext{Sonntag, den 24.}{\lemma{\textnormal{\emph{Sonntag, den 24.}}}\Cendnote{\textnormal{Seit dem 9. 10. 1895 wurden Giuseppe
                           Giacosa\pwindex{Giacosa, Giuseppe 21.10.1847 – 02.09.1906@\textsc{Giacosa, Giuseppe} (21.10.1847 – 02.09.1906), \emph{Schriftsteller}|pwk}s \emph{Rechte der Seele}\pwindex{Giacosa, Giuseppe 21.10.1847 – 02.09.1906@\textsc{Giacosa, Giuseppe} (21.10.1847 – 02.09.1906), \emph{Schriftsteller}!Rechte der Seele. Schauspiel in einem Act1895-10-09@\strich\emph{Rechte der Seele. Schauspiel in einem Act} {[}1895-10-09{]}|pwk} und
                           Schnitzler\pwindex{Schnitzler, Arthur 15.05.1862 – 21.10.1931@\textsc{Schnitzler, Arthur} (15.05.1862 – 21.10.1931), \emph{Schriftsteller, Mediziner}|pwk}s \emph{Liebelei}\pwindex{Schnitzler, Arthur 15.05.1862 – 21.10.1931@\textsc{Schnitzler, Arthur} (15.05.1862 – 21.10.1931), \emph{Schriftsteller, Mediziner}!Liebelei. Schauspiel in drei Akten1895-10-09@\strich\emph{Liebelei. Schauspiel in drei Akten} {[}1895-10-09{]}|pwk} am Burgtheater\oindex{Burgtheater@\textbf{Burgtheater}|pwk} gemeinsam gespielt. Am 24. 11. 1895 wurde die \emph{Liebelei}\pwindex{Schnitzler, Arthur 15.05.1862 – 21.10.1931@\textsc{Schnitzler, Arthur} (15.05.1862 – 21.10.1931), \emph{Schriftsteller, Mediziner}!Liebelei. Schauspiel in drei Akten1895-10-09@\strich\emph{Liebelei. Schauspiel in drei Akten} {[}1895-10-09{]}|pwk} zum elften Mal gegeben.}}}\label{K_L03166-1h}}\makebox[\the\longestz][l]{»Rechte der Seele\pwindex{Giacosa, Giuseppe 21.10.1847 – 02.09.1906@\textsc{Giacosa, Giuseppe} (21.10.1847 – 02.09.1906), \emph{Schriftsteller}!Rechte der Seele. Schauspiel in einem Act1895-10-09@\strich\emph{Rechte der Seele. Schauspiel in einem Act} {[}1895-10-09{]}|pw}«}
                  \pend\pstart\noindent\makebox[\the\longeste][l]{}\makebox[\the\longestz][l]{»Liebelei\pwindex{Schnitzler, Arthur 15.05.1862 – 21.10.1931@\textsc{Schnitzler, Arthur} (15.05.1862 – 21.10.1931), \emph{Schriftsteller, Mediziner}!Liebelei. Schauspiel in drei Akten1895-10-09@\strich\emph{Liebelei. Schauspiel in drei Akten} {[}1895-10-09{]}|pw}« –}
                  \pend\pstart
           Über so was kann ich mich \label{T_L03166-1v}\edtext{riesig}{\lemma{\textnormal{\emph{riesig}}}\Cendnote{\textnormal{»riesig« dürfte absichtlich mit größerer
            Schrift geschrieben sein}}}\label{T_L03166-1h}{ }\label{K_L03166-2v}\edtext{amusiren}{\lemma{\textnormal{\emph{amusiren}}}\Cendnote{\textnormal{Eventuell 
            fand er die Paarung der Titel im Sinne von »Liebelei« als »Recht der Seele« vergnüglich?}}}\label{K_L03166-2h}. Ihr {\\}\spacefill\mbox{Salten}\pend
           \pstart
           \noindent{}Wie ist’s \label{K_L03166-3v}\edtext{heute mit Ronacher\oindex{Ronacher@\textbf{Ronacher}|pw}}{\lemma{\textnormal{\emph{heute mit Ronacher}}}\Cendnote{\textnormal{Schnitzler\pwindex{Schnitzler, Arthur 15.05.1862 – 21.10.1931@\textsc{Schnitzler, Arthur} (15.05.1862 – 21.10.1931), \emph{Schriftsteller, Mediziner}|pwk} besuchte an diesem Abend den
                     Polterabend von Ludmilla Kaufmann\pwindex{Karplus, Ludmilla *~12.10.1866@\textsc{Karplus, Ludmilla} (*~12.10.1866)|pwk}, die
                     am Folgetag 
                     den Rechtsanwalt Siegmund Karplus\pwindex{Karplus, Siegmund 12.11.1861 – 16.08.1928@\textsc{Karplus, Siegmund} (12.11.1861 – 16.08.1928), \emph{Rechtsanwalt}|pwk}
                     heiratete. Ein Besuch der Hochzeit erwähnt Schnitzler\pwindex{Schnitzler, Arthur 15.05.1862 – 21.10.1931@\textsc{Schnitzler, Arthur} (15.05.1862 – 21.10.1931), \emph{Schriftsteller, Mediziner}|pwk} nicht, stattdessen 
                     besuchte er am 17. 11. 1895 das Ronacher\oindex{Ronacher@\textbf{Ronacher}|pwk}, so dass das Korrespondenzstück auch in der Nacht vom 16. auf den 17. gelaufen sein 
                     und sich auf den 17. beziehen könnte. Auffällig ist, dass sich auch für das folgende
                     Korrespondenzstück eine ähnliche Argumentation rechtfertigen lässt, siehe Felix Salten an Arthur Schnitzler, [12?. 12. 1895].
                  }}}\label{K_L03166-3h}?\pend
           
         
         \endnumbering\mylabel{h}\end{ledgroupsized}  \newcommand{\dateiname}{L03166}\newcommand{\titel}{Felix Salten an Arthur Schnitzler, [16. 11. 1895]}\newcommand{\editorInnen}{Martin Anton Müller und Laura Untner}%% latex-leseansicht-abspann.tex
%% Abspann für die Leseansicht.
%% Der Schalter \ifkorrekturansicht ist bereits durch den Vorspann gesetzt.

%% latex-abspann.tex
%% Gemeinsamer Abspann für Korrekturansicht und Leseansicht.
%% Setzt den Schalter \ifkorrekturansicht voraus (gesetzt in den
%% einbindenden Dateien latex-korrekturansicht-abspann.tex bzw.
%% latex-leseansicht-abspann.tex).
%% ---------------------------------------------------------------

\normalsize

% Das esempio-Environment wird nur in der Leseansicht benötigt
\ifkorrekturansicht\else
\newenvironment{esempio}[3]%
{
    \vspace{1.5ex}
    \rlap{\underline{#1}}
    \par
    \setlength{\parindent}{0cm}
    \nopagebreak
    \leftskip=#2cm
    \rightskip=#3cm
}
{
    \par
}
\fi

\doendnotes{C}
\bigskip
\vfill

\clearpage

\footnotesize

\ifkorrekturansicht
  \lohead{\textsc{register}}
\fi

% theindex-Environment neu definieren ohne reledmac
\makeatletter
\renewenvironment{theindex}{%
  \ifkorrekturansicht
    \section*{\indexname}%
  \else
    \subsubsection*{Index der erwähnten Entitäten}%
  \fi
  \setlength{\parindent}{0pt}%
  \setlength{\parskip}{0pt plus 0.3pt}%
  \let\item\@idxitem
}{%
  \ifkorrekturansicht\clearpage\fi
}
\makeatother

\IfFileExists{\jobname-pw.ind}{\input{\jobname-pw.ind}}{}

% Quellenangabe nur in der Leseansicht
\ifkorrekturansicht\else
% Fallback-Definitionen, falls die .tex-Datei \titel etc. nicht gesetzt hat
\providecommand{\titel}{}
\providecommand{\editorInnen}{}
\providecommand{\dateiname}{\jobname}

\vspace{3cm}

\vfill

\footnotesize
\textsc{Quelle}: \titel. Herausgegeben von {\editorInnen}. In: \emph{Arthur Schnitzler: Briefwechsel mit Autorinnen und Autoren}.
 Digitale Edition, https://schnitzler-briefe.acdh.oeaw.ac.at/{\dateiname}.html (Stand \today)
\fi

\end{document}


      