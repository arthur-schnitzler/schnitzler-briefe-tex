%% latex-leseansicht-vorspann.tex
%% Vorspann für die Leseansicht.
%% Lädt die gemeinsame Datei latex-vorspann.tex mit nicht gesetztem Schalter.

\newif\ifkorrekturansicht
\korrekturansichtfalse

\input{../tex-inputs/latex-vorspann}


\section[ Felix Salten an Arthur Schnitzler, [22. 12. 1904?]]{L03403 Felix Salten an Arthur Schnitzler,  [22. 12. 1904?]}
\nopagebreak\mylabel{L03403v}
\rehead{ }\normalsize\beginnumbering\briefempfaengerindex{Schnitzler, Arthur@\textsc{Schnitzler, Arthur}!zzzSalten, Felix@\emph{von Felix Salten}!1904-12-221@{{[}22. 12. 1904?{]}}|(be}
\toendnotes[C]{\smallbreak\pagebreak[2]}
\correspDesc{Versand  durch Felix Salten am [22. 12. 1904?] in Wien
\newline{}Erhalt  durch Arthur Schnitzler am [23. 12. 1904?] in Wien}\toendnotes[C]{\smallbreak}
\Standort{CUL, Schnitzler, B 89, B 1.}
\physDesc{Bildpostkarte, 213 Zeichen
\newline{}Handschrift: schwarze Tinte, lateinische Kurrent
\newline{}Ordnung: mit Bleistift von unbekannter Hand nummeriert: »196« }\toendnotes[C]{\smallbreak}\pstart{}{\pb}Herrn D\textsuperscript{r} Arthur Schnitzler\pend{}\pstart{}Wien XVIII.\oindex{XVIII., Währing@\textbf{XVIII., Währing}, \emph{Verwaltungsgebiet}|pw}\pend{}\pstart{}Spöttelgaße 7\oindex{Wien@\textbf{Wien}!XVIII., Währing@\textbf{XVIII., Währing}!Edmund-Weiß-Gasse 7@\textbf{Edmund-Weiß-Gasse 7}, \emph{Wohngebäude}|pw}\pend{}{\bigskip}
\pstart
           \noindent{}\centering{}{\pb}\textcolor{gray}{\textbf{Johann Benedickter\pwindex{Benedickter, Johann @\textsc{Benedickter, Johann}, \emph{Gastwirt}|pw}’s}}\pend
           
\pstart
           \centering{}\textcolor{gray}{\textbf{Restaurant u. Weinhandlung »zum Riedhof«}}\oindex{Wien@\textbf{Wien}!VIII., Josefstadt@\textbf{VIII., Josefstadt}!Riedhof@\textbf{Riedhof}, \emph{Lokal}|pw}\pend
           
\pstart
           \centering{}\textcolor{gray}{\textbf{WIEN\oindex{Wien@\textbf{Wien}, \emph{Verwaltungsgebiet}|pw}}}\pend
           
\pstart
           \centering{}\textcolor{gray}{\textbf{VIII, Schlösselgasse 14\oindex{Wien@\textbf{Wien}!VIII., Josefstadt@\textbf{VIII., Josefstadt}!Schlösselgasse@\textbf{Schlösselgasse}, \emph{Straße}|pw}}}\pend
           
\pstart
           \centering{}\textcolor{gray}{\textbf{Wickenburggasse 15}}\oindex{Wien@\textbf{Wien}!VIII., Josefstadt@\textbf{VIII., Josefstadt}!Wickenburggasse@\textbf{Wickenburggasse}, \emph{Straße}|pw}\pend
           
\pstart
           \centering{}\textcolor{gray}{\textbf{Garten mit Spiegelveranda}}\pend
           
\pstart
           \centering{}\textcolor{gray}{\textbf{Marmorsaal}}\pend
           \vspace{1em}
\pstart
           {\pb}zwischen ¾ 11–11\pend
           \vspace{0.5em}
\pstart
           Sind etwas verspätet \label{K_L03403-1v}\edtext{gekommen}{\lemma{\textnormal{\emph{gekommen}}}\Cendnote{\textnormal{in den Riedhof\oindex{Wien@\textbf{Wien}!VIII., Josefstadt@\textbf{VIII., Josefstadt}!Riedhof@\textbf{Riedhof}, \emph{Lokal}|pwk}; vgl. XXXX Auszeichnungsfehler: Dokument L03402 nicht gefunden.}}}\label{K_L03403-1}, weil Otti\pwindex{Salten, Ottilie 7.\,3.\,1868 Prag – 22.\,6.\,1942 Zürich@\textsc{Salten, Ottilie} (7.\,3.\,1868 Prag – 22.\,6.\,1942 Zürich), \emph{Schauspielerin}|pw} nach dem Conzert\pwindex{\textcolor{red}{\textsuperscript{XXXX indx1}}!3. Sinfonie in d-Moll@\strich\emph{3. Sinfonie in d-Moll}|pwv}{ }\label{K_L03403-2v}\edtext{des Kind\pwindex{Rehmann, Anna Katharina 18.\,8.\,1904 Wien – 27.\,3.\,1977 Zürich@\textsc{Rehmann, Anna Katharina} (18.\,8.\,1904 Wien – 27.\,3.\,1977 Zürich), \emph{Schauspielerin, Übersetzerin}|pwv}es wegen}{\lemma{\textnormal{\emph{des Kindes wegen}}}\Cendnote{\textnormal{Siehe XXXX Auszeichnungsfehler: Dokument L03400 nicht gefunden.
               }}}\label{K_L03403-2} nochmals \label{K_L03403-3v}\edtext{nach Hause}{\lemma{\textnormal{\emph{nach Hause}}}\Cendnote{\textnormal{Die beiden Wörter durch einen 
               langen Strich miteinander verbunden. Möglich wäre auch eine Lesart »nachhause«.}}}\label{K_L03403-3} mußte, und sind sehr erstaunt, dass Sie es so eilig
               hatten.\pend
           \pstart \spacefill\mbox{S.}\pend{}\selectlanguage{ngerman}\endnumbering\briefempfaengerindex{Schnitzler, Arthur@\textsc{Schnitzler, Arthur}!zzzSalten, Felix@\emph{von Felix Salten}!1904-12-221@{{[}22. 12. 1904?{]}}|)be}\mylabel{L03403h}  \newcommand{\dateiname}{L03403}\newcommand{\titel}{Felix Salten an Arthur Schnitzler, [22. 12. 1904?]}\newcommand{\editorInnen}{Martin Anton Müller und Laura Untner}%% latex-leseansicht-abspann.tex
%% Abspann für die Leseansicht.
%% Der Schalter \ifkorrekturansicht ist bereits durch den Vorspann gesetzt.

%% latex-abspann.tex
%% Gemeinsamer Abspann für Korrekturansicht und Leseansicht.
%% Setzt den Schalter \ifkorrekturansicht voraus (gesetzt in den
%% einbindenden Dateien latex-korrekturansicht-abspann.tex bzw.
%% latex-leseansicht-abspann.tex).
%% ---------------------------------------------------------------

\normalsize

% Das esempio-Environment wird nur in der Leseansicht benötigt
\ifkorrekturansicht\else
\newenvironment{esempio}[3]%
{
    \vspace{1.5ex}
    \rlap{\underline{#1}}
    \par
    \setlength{\parindent}{0cm}
    \nopagebreak
    \leftskip=#2cm
    \rightskip=#3cm
}
{
    \par
}
\fi

\doendnotes{C}
\bigskip
\vfill

\clearpage

\footnotesize

\ifkorrekturansicht
  \lohead{\textsc{register}}
\fi

% theindex-Environment neu definieren ohne reledmac
\makeatletter
\renewenvironment{theindex}{%
  \ifkorrekturansicht
    \section*{\indexname}%
  \else
    \subsubsection*{Index der erwähnten Entitäten}%
  \fi
  \setlength{\parindent}{0pt}%
  \setlength{\parskip}{0pt plus 0.3pt}%
  \let\item\@idxitem
}{%
  \ifkorrekturansicht\clearpage\fi
}
\makeatother

\IfFileExists{\jobname-pw.ind}{\input{\jobname-pw.ind}}{}

% Quellenangabe nur in der Leseansicht
\ifkorrekturansicht\else
% Fallback-Definitionen, falls die .tex-Datei \titel etc. nicht gesetzt hat
\providecommand{\titel}{}
\providecommand{\editorInnen}{}
\providecommand{\dateiname}{\jobname}

\vspace{3cm}

\vfill

\footnotesize
\textsc{Quelle}: \titel. Herausgegeben von {\editorInnen}. In: \emph{Arthur Schnitzler: Briefwechsel mit Autorinnen und Autoren}.
 Digitale Edition, https://schnitzler-briefe.acdh.oeaw.ac.at/{\dateiname}.html (Stand \today)
\fi

\end{document}


