%% latex-leseansicht-vorspann.tex
%% Vorspann für die Leseansicht.
%% Lädt die gemeinsame Datei latex-vorspann.tex mit nicht gesetztem Schalter.

\newif\ifkorrekturansicht
\korrekturansichtfalse

\input{../tex-inputs/latex-vorspann}

\begin{center}
            \textcolor{red}{ENTWURF, NICHT FERTIG KORRIGIERT}
                      \end{center}
            
         
         \renewcommand{\erwaehntePersonen}{Personen: Johann Benedickter, Anna Katharina Rehmann, Ottilie Salten}
         \renewcommand{\erwaehnteOrte}{Orte: Edmund-Weiß-Gasse, Riedhof, Schlösselgasse, Wickenburggasse, Wien, XVIII., Währing}
         \renewcommand{\erwaehnteWerke}{Werke: Symphonie Nr. 3 D-Moll}
               \section[Felix Salten an Arthur Schnitzler, {[}22. 12. 1904?{]}]{ Felix Salten an Arthur Schnitzler, {[}22. 12. 1904?{]}}\nopagebreak\mylabel{v}\rehead{ }\begin{ledgroupsized}[t]{13cm}\normalsize\beginnumbering \toendnotes[C]{\smallbreak\pagebreak[2]} \Standort{CUL, Schnitzler, B 89, B 1.}
\physDesc{Bildpostkarte
\newline{}Handschrift: schwarze Tinte, lateinische Kurrent\newline{}Ordnung: mit Bleistift von unbekannter Hand nummeriert:
                                    »196« }\toendnotes[C]{\smallbreak}\pstart{}{\pb}Herrn D\textsuperscript{r} Arthur Schnitzler\pend{}\pstart{}Wien XVIII.\oindex{XVIII., Waehring@\textbf{XVIII., Währing}|pw}\pend{}\pstart{}Spöttelgaſse 7\oindex{Edmund-Weiss-Gasse@\textbf{Edmund-Weiß-Gasse}|pw}\pend{}{\bigskip}\pstart
           \noindent{}\centering{}{\pb}\textcolor{gray}{\textbf{Johann Benedickter\pwindex{Benedickter, Johann @\textsc{Benedickter, Johann}, \emph{Gastwirt}|pw}’s}}\pend
           \pstart
           \noindent{}\centering{}\textcolor{gray}{\textbf{Restaurant u. Weinhandlung »zum Riedhof«}}\oindex{Riedhof@\textbf{Riedhof}|pw}\pend
           \pstart
           \noindent{}\centering{}\textcolor{gray}{\textbf{WIEN\oindex{Wien@\textbf{Wien}|pw}}}\pend
           \pstart
           \noindent{}\centering{}\textcolor{gray}{\textbf{VIII, Schlösselgasse 14\oindex{Schloesselgasse@\textbf{Schlösselgasse}|pw}}}\pend
           \pstart
           \noindent{}\centering{}\textcolor{gray}{\textbf{Wickenburggasse 15 }}\oindex{Wickenburggasse@\textbf{Wickenburggasse}|pw}\pend
           \pstart
           \noindent{}\centering{}\textcolor{gray}{\textbf{Garten mit Spiegelveranda}}\pend
           \pstart
           \noindent{}\centering{}\textcolor{gray}{\textbf{Marmorsaal}}\pend
           \pstart
           zwischen ¾ 11–11\pend
           \pstart
           Sind etwas verspätet gekommen, weil Otti\pwindex{Salten, Ottilie 07.03.1868 – 22.06.1942@\textsc{Salten, Ottilie} (07.03.1868 – 22.06.1942), \emph{Schauspielerin}|pw} nach dem
               Konzert\pwindex{\textcolor{red}{\textsuperscript{XXXX1 indx}}!Symphonie Nr. 3 D-Moll1902@\strich\emph{Symphonie Nr. 3 D-Moll} {[}1902{]}|pwv} des Kindes\pwindex{Rehmann, Anna Katharina 18.08.1904 – 27.03.1977@\textsc{Rehmann, Anna Katharina} (18.08.1904 – 27.03.1977), \emph{Schauspielerin}|pwv} wegen nochmals
               nach Hause mußte, und sind sehr erstaunt, dass Sie es so eilig hatten. \pend
           \pstart \spacefill\mbox{S.}\pend{}
         
         \endnumbering\mylabel{h}\end{ledgroupsized}\begin{anhang}\end{anhang}\newcommand{\dateiname}{L03403}\newcommand{\titel}{Felix Salten an Arthur Schnitzler, [22. 12. 1904?]}\newcommand{\editorInnen}{Martin Anton Müller und Laura Untner}%% latex-leseansicht-abspann.tex
%% Abspann für die Leseansicht.
%% Der Schalter \ifkorrekturansicht ist bereits durch den Vorspann gesetzt.

%% latex-abspann.tex
%% Gemeinsamer Abspann für Korrekturansicht und Leseansicht.
%% Setzt den Schalter \ifkorrekturansicht voraus (gesetzt in den
%% einbindenden Dateien latex-korrekturansicht-abspann.tex bzw.
%% latex-leseansicht-abspann.tex).
%% ---------------------------------------------------------------

\normalsize

% Das esempio-Environment wird nur in der Leseansicht benötigt
\ifkorrekturansicht\else
\newenvironment{esempio}[3]%
{
    \vspace{1.5ex}
    \rlap{\underline{#1}}
    \par
    \setlength{\parindent}{0cm}
    \nopagebreak
    \leftskip=#2cm
    \rightskip=#3cm
}
{
    \par
}
\fi

\doendnotes{C}
\bigskip
\vfill

\clearpage

\footnotesize

\ifkorrekturansicht
  \lohead{\textsc{register}}
\fi

% theindex-Environment neu definieren ohne reledmac
\makeatletter
\renewenvironment{theindex}{%
  \ifkorrekturansicht
    \section*{\indexname}%
  \else
    \subsubsection*{Index der erwähnten Entitäten}%
  \fi
  \setlength{\parindent}{0pt}%
  \setlength{\parskip}{0pt plus 0.3pt}%
  \let\item\@idxitem
}{%
  \ifkorrekturansicht\clearpage\fi
}
\makeatother

\IfFileExists{\jobname-pw.ind}{\input{\jobname-pw.ind}}{}

% Quellenangabe nur in der Leseansicht
\ifkorrekturansicht\else
% Fallback-Definitionen, falls die .tex-Datei \titel etc. nicht gesetzt hat
\providecommand{\titel}{}
\providecommand{\editorInnen}{}
\providecommand{\dateiname}{\jobname}

\vspace{3cm}

\vfill

\footnotesize
\textsc{Quelle}: \titel. Herausgegeben von {\editorInnen}. In: \emph{Arthur Schnitzler: Briefwechsel mit Autorinnen und Autoren}.
 Digitale Edition, https://schnitzler-briefe.acdh.oeaw.ac.at/{\dateiname}.html (Stand \today)
\fi

\end{document}


      