%% latex-korrekturansicht-vorspann.tex
%% Vorspann für die Korrekturansicht.
%% Lädt die gemeinsame Datei latex-vorspann.tex mit gesetztem Schalter.

\newif\ifkorrekturansicht
\korrekturansichttrue

\input{../tex-inputs/latex-vorspann}


\section[ Felix Salten an Arthur Schnitzler, {[}22. 12. 1904?{]}]{L03403 Felix Salten an Arthur Schnitzler, {[}22. 12. 1904?{]}}
\nopagebreak\mylabel{L03403v}
\rehead{ }\normalsize\beginnumbering\briefempfaengerindex{Schnitzler, Arthur@\textsc{Schnitzler, Arthur}!zzzSalten, Felix@\emph{von Felix Salten}!1904-12-221@{{[}22. 12. 1904?{]}}|(be}
\toendnotes[C]{\smallbreak\pagebreak[2]}\Standort{CUL, Schnitzler, B 89, B 1.}
\physDesc{Bildpostkarte, 213 Zeichen
\newline{}Handschrift: schwarze Tinte, lateinische Kurrent
\newline{}Ordnung: mit Bleistift von unbekannter Hand nummeriert: »196« }\toendnotes[C]{\smallbreak}\pstart{}{\pb}Herrn D\textsuperscript{r} Arthur Schnitzler\pend{}\pstart{}Wien XVIII.\oindex{XVIII., Waehring@\textbf{XVIII., Währing}, \emph{A.ADM3}|pw}\pend{}\pstart{}Spöttelgaße 7\oindex{Edmund-Weiss-Gasse 7@\textbf{Edmund-Weiß-Gasse 7}, \emph{Wohngebäude (K.WHS)}|pw}\pend{}{\bigskip}
\pstart
           \noindent{}\centering{}{\pb}\textcolor{gray}{\textbf{Johann Benedickter\pwindex{Benedickter, Johann @\textsc{Benedickter, Johann}, \emph{Gastwirt/Gastwirtin}|pw}’s}}\pend
           
\pstart
           \centering{}\textcolor{gray}{\textbf{Restaurant u. Weinhandlung »zum Riedhof«}}\oindex{Riedhof@\textbf{Riedhof}, \emph{Lokal (K.LKL)}|pw}\pend
           
\pstart
           \centering{}\textcolor{gray}{\textbf{WIEN\oindex{Wien@\textbf{Wien}, \emph{A.ADM2}|pw}}}\pend
           
\pstart
           \centering{}\textcolor{gray}{\textbf{VIII, Schlösselgasse 14\oindex{Schloesselgasse@\textbf{Schlösselgasse}, \emph{Straße (K.STR)}|pw}}}\pend
           
\pstart
           \centering{}\textcolor{gray}{\textbf{Wickenburggasse 15}}\oindex{Wickenburggasse@\textbf{Wickenburggasse}, \emph{Straße (K.STR)}|pw}\pend
           
\pstart
           \centering{}\textcolor{gray}{\textbf{Garten mit Spiegelveranda}}\pend
           
\pstart
           \centering{}\textcolor{gray}{\textbf{Marmorsaal}}\pend
           \vspace{1em}
\pstart
           {\pb}zwischen ¾ 11–11\pend
           \vspace{0.5em}
\pstart
           Sind etwas verspätet \label{K_L03403-1v}\edtext{gekommen}{\lemma{\textnormal{\emph{gekommen}}}\Cendnote{\textnormal{in den Riedhof\oindex{Riedhof@\textbf{Riedhof}, \emph{Lokal (K.LKL)}|pwk}; vgl. Felix Salten an Arthur Schnitzler, [20. 12. 1904].}}}\label{K_L03403-1}, weil Otti\pwindex{Salten, Ottilie 07.03.1868 – 22.06.1942@\textsc{Salten, Ottilie} (07.03.1868 – 22.06.1942), \emph{Schauspieler/Schauspielerin}|pw} nach dem Conzert\pwindex{3. Sinfonie in d-Moll@\emph{3. Sinfonie in d-Moll}|pwv}{ }\label{K_L03403-2v}\edtext{des Kind\pwindex{Rehmann, Anna Katharina 18.08.1904 – 27.03.1977@\textsc{Rehmann, Anna Katharina} (18.08.1904 – 27.03.1977), \emph{Schauspieler/Schauspielerin, Übersetzer/Übersetzerin}|pwv}es wegen}{\lemma{\textnormal{\emph{des Kindes wegen}}}\Cendnote{\textnormal{Siehe Felix Salten an Arthur Schnitzler, [15. 12. 1904].
               }}}\label{K_L03403-2} nochmals \label{K_L03403-3v}\edtext{nach Hause}{\lemma{\textnormal{\emph{nach Hause}}}\Cendnote{\textnormal{Die beiden Wörter durch einen 
               langen Strich miteinander verbunden. Möglich wäre auch eine Lesart »nachhause«.}}}\label{K_L03403-3} mußte, und sind sehr erstaunt, dass Sie es so eilig
               hatten.\pend
           \pstart \spacefill\mbox{S.}\pend{}\selectlanguage{ngerman}\endnumbering\briefempfaengerindex{Schnitzler, Arthur@\textsc{Schnitzler, Arthur}!zzzSalten, Felix@\emph{von Felix Salten}!1904-12-221@{{[}22. 12. 1904?{]}}|)be}\mylabel{L03403h}  \normalsize

\doendnotes{C}
\bigskip
\vfill

\clearpage

\footnotesize

\lohead{\textsc{register}}

% Definiere theindex-Environment komplett neu ohne reledmac
\makeatletter
\renewenvironment{theindex}{%
  \section*{\indexname}%
  \setlength{\parindent}{0pt}%
  \setlength{\parskip}{0pt plus 0.3pt}%
  \let\item\@idxitem
}{%
  \clearpage
}
\makeatother

\IfFileExists{\jobname-pw.ind}{\input{\jobname-pw.ind}}{}

\end{document}

      