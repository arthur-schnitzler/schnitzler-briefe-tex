%% latex-korrekturansicht-vorspann.tex
%% Vorspann für die Korrekturansicht.
%% Lädt die gemeinsame Datei latex-vorspann.tex mit gesetztem Schalter.

\newif\ifkorrekturansicht
\korrekturansichttrue

\input{../tex-inputs/latex-vorspann}


\section[Felix Salten an Arthur Schnitzler, {[}13.? 4. 1892{]}]{L03184 Felix Salten an Arthur Schnitzler, {[}13.? 4. 1892{]}}
\nopagebreak\mylabel{L03184v}
\rehead{ }\normalsize\beginnumbering\briefempfaengerindex{Schnitzler, Arthur@\textsc{Schnitzler, Arthur}!zzzSalten, Felix@\emph{von Felix Salten}!1892-04-132@{{[}13.? 4. 1892{]}}|(be}
\toendnotes[C]{\smallbreak\pagebreak[2]}\Standort{CUL, Schnitzler, B 89, A 1.}
\physDesc{Brief, 1 Blatt, 2 Seiten, 430 Zeichen
\newline{}Handschrift: Bleistift, lateinische Kurrent
\newline{}Schnitzler: mit Bleistift datiert: »April 92« 
\newline{}Ordnung: mit Bleistift von unbekannter Hand nummeriert: »10« }\toendnotes[C]{\smallbreak}
\pstart
           \noindent{}{\pb}lieber Arthur! Ich ging vorbei, vergaß natürlich, dass Sie \label{K_L03184-1v}\edtext{Burgring 1\oindex{Wohnung und Ordination Johann Schnitzler Burgring 1@\textbf{Wohnung und Ordination Johann Schnitzler Burgring 1}, \emph{Ordination}|pw} ordiniren}{\lemma{\textnormal{\emph{Burgring 1 ordiniren}}}\Cendnote{\textnormal{Schnitzler vertrat seinen Vater\pwindex{Schnitzler, Johann 10.04.1835 – 02.05.1893@\textsc{Schnitzler, Johann} (10.04.1835 – 02.05.1893), \emph{Laryngologe/Laryngologin}|pwkv} in dessen Praxis\oindex{Wohnung und Ordination Johann Schnitzler Burgring 1@\textbf{Wohnung und Ordination Johann Schnitzler Burgring 1}, \emph{Ordination}|pwkv}, vgl. A. S.: \emph{Tagebuch}, 11. 4. 1892.}}}\label{K_L03184-1}. Ihre Handschuhe brachte ich zurück,
               u. sagen wollte ich Ihnen, dass ich \label{K_L03184-2v}\edtext{Abends}{\lemma{\textnormal{\emph{Abends}}}\Cendnote{\textnormal{Schnitzler datiert auf »April
                     92«. Durch den Hinweis auf die Ordination\oindex{Wohnung und Ordination Johann Schnitzler Burgring 1@\textbf{Wohnung und Ordination Johann Schnitzler Burgring 1}, \emph{Ordination}|pwkv} am Burgring 1\oindex{Wohnung und Ordination Johann Schnitzler Burgring 1@\textbf{Wohnung und Ordination Johann Schnitzler Burgring 1}, \emph{Ordination}|pwk} lässt sich der Zeitpunkt etwas genauer bestimmen. Am 11. 4. 1892 waren
                  beide gemeinsam im Prater\oindex{Prater@\textbf{Prater}, \emph{Park (K.PRK)}|pwk}, wo Salten\pwindex{Salten, Felix 06.09.1869 – 08.10.1945@\textsc{Salten, Felix} (06.09.1869 – 08.10.1945), \emph{Schriftsteller/Schriftstellerin, Journalist/Journalistin, Chefredakteur/Chefredakteurin}|pwk} die erwähnten Handschuhe ausgeliehen haben könnte,
                  und zwei Tage später kam Salten\pwindex{Salten, Felix 06.09.1869 – 08.10.1945@\textsc{Salten, Felix} (06.09.1869 – 08.10.1945), \emph{Schriftsteller/Schriftstellerin, Journalist/Journalistin, Chefredakteur/Chefredakteurin}|pwk} zum
                  Abendessen, womit dieses Korrespondenzstück mutmaßlich datiert werden
                  kann.}}}\label{K_L03184-2} wahrscheinlich komme, doch erst gegen 11 Uhr. Jetzt bin
               ich müde und ruhe mich {\pb}ein
               wenig aus und lese die Neue fr Pr.\pwindex{Neue Freie Presse@\emph{Neue Freie Presse}|pw} u. bilde mir
               ein, ich »bin mein mich innig liebender{[}«{]}\pend
           
\pstart
           \centering{}Arthur Schnitzler.\pend
           
\pstart
           Habe heute gearbeitet{[},{]} aber wenig, gehe jetzt nach
               Hause, wieder arbeiten.\pend
           
\pstart
           Loris\pwindex{Hofmannsthal, Hugo von 1874-02-01 – 1929-07-15@\textsc{Hofmannsthal, Hugo von} (1874-02-01 – 1929-07-15), \emph{Schriftsteller/Schriftstellerin}|pw}, Beer
                  Hofmann\pwindex{Beer-Hofmann, Richard 1866-07-11 – 1945-09-26@\textsc{Beer-Hofmann, Richard} (1866-07-11 – 1945-09-26), \emph{Schriftsteller/Schriftstellerin}|pw}?\pend
           \selectlanguage{ngerman}\endnumbering\briefempfaengerindex{Schnitzler, Arthur@\textsc{Schnitzler, Arthur}!zzzSalten, Felix@\emph{von Felix Salten}!1892-04-132@{{[}13.? 4. 1892{]}}|)be}\mylabel{L03184h}  \normalsize

\doendnotes{C}
\bigskip
\vfill

\clearpage

\footnotesize

\lohead{\textsc{register}}

% Definiere theindex-Environment komplett neu ohne reledmac
\makeatletter
\renewenvironment{theindex}{%
  \section*{\indexname}%
  \setlength{\parindent}{0pt}%
  \setlength{\parskip}{0pt plus 0.3pt}%
  \let\item\@idxitem
}{%
  \clearpage
}
\makeatother

\IfFileExists{\jobname-pw.ind}{\input{\jobname-pw.ind}}{}

\end{document}

      