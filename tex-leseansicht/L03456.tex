%% latex-korrekturansicht-vorspann.tex
%% Vorspann für die Korrekturansicht.
%% Lädt die gemeinsame Datei latex-vorspann.tex mit gesetztem Schalter.

\newif\ifkorrekturansicht
\korrekturansichttrue

\input{../tex-inputs/latex-vorspann}


\section[ Paul Goldmann an Arthur Schnitzler, 18. 11. {[}1904{]}]{L03456 Paul Goldmann an Arthur Schnitzler, 18. 11. {[}1904{]}}
\nopagebreak\mylabel{L03456v}
\rehead{ }\normalsize\beginnumbering\briefempfaengerindex{Schnitzler, Arthur@\textsc{Schnitzler, Arthur}!zzzGoldmann, Paul@\emph{von Paul Goldmann}!1904-11-181@{18. 11. {[}1904{]}}|(be}
\toendnotes[C]{\smallbreak\pagebreak[2]}\Standort{DLA, A:Schnitzler, HS.NZ85.1.3174.}
\physDesc{Brief, 1 Blatt, 2 Seiten, 670 Zeichen
\newline{}Handschrift: blaue Tinte, deutsche Kurrent
\newline{}Schnitzler: 1) mit Bleistift das Jahr »904« vermerkt  2) mit rotem Buntstift eine Unterstreichung}\toendnotes[C]{\smallbreak}
\pstart
           \raggedleft{}{\pb}\textcolor{gray}{\textbf{DESSAUERSTRASSE 19\oindex{Dessauer Strasse@\textbf{Dessauer Straße}, \emph{Straße (K.STR)}|pw}}}\pend
           
\pstart
           Berlin\oindex{Berlin@\textbf{Berlin}, \emph{P.PPLC}|pw}, \substVorne{}\textsuperscript{2}\substDazwischen{}1\substHinten{}8. November.\pend
           
\pstart\center{}Mein lieber Freund,\pend\vspace{0.5em}
\pstart
           Ich \strikeout{\textcolor{gray}{×}} danke Dir für Deinen Brief und werde mich ſehr freuen, Dich \label{K_L03456-1v}\edtext{bald zu ſehen}{\lemma{\textnormal{\emph{bald zu ſehen}}}\Cendnote{\textnormal{Schnitzler war seit 13. 11. 1904 in Berlin\oindex{Berlin@\textbf{Berlin}, \emph{P.PPLC}|pwk}. Am \emph{Kleinen Theater}\orgindex{Kleines Theater@Kleines Theater|pwk} stand die Uraufführung von \emph{Der tapfere Cassian}\pwindex{tapfere Cassian. Puppenspiel in einem Akt@\emph{Der tapfere Cassian. Puppenspiel in einem Akt}|pwk} und \emph{Das Haus Delorme}\pwindex{Haus Delorme. Eine Familienszene@\emph{Das Haus Delorme. Eine Familienszene}|pwk} bevor, dazu sollte \emph{Der
                     grüne Kakadu}\pwindex{gruene Kakadu. Groteske in einem Akt@\emph{Der grüne Kakadu. Groteske in einem Akt}|pwk} neu gegeben werden. Kurzfristig wurde \emph{Das Haus Delorme}\pwindex{Haus Delorme. Eine Familienszene@\emph{Das Haus Delorme. Eine Familienszene}|pwk} noch vom Programm genommen, die beiden
                  anderen Stücke wurden erstmals am 22. 11. 1904 aufgeführt. Zu einem Treffen Schnitzlers und Goldmanns\pwindex{Goldmann, Paul 31.01.1865 – 25.09.1935@\textsc{Goldmann, Paul} (31.01.1865 – 25.09.1935), \emph{Schriftsteller/Schriftstellerin, Journalist/Journalistin}|pwk} kam es am Montag, dem 21. 11. 1904, doch –
                  anders als hier Goldmann\pwindex{Goldmann, Paul 31.01.1865 – 25.09.1935@\textsc{Goldmann, Paul} (31.01.1865 – 25.09.1935), \emph{Schriftsteller/Schriftstellerin, Journalist/Journalistin}|pwk} vorgeschlagen –
                  vermutlich ohne den ebenfalls in Berlin\oindex{Berlin@\textbf{Berlin}, \emph{P.PPLC}|pwk}
                  weilenden Richard Beer-Hofmann\pwindex{Beer-Hofmann, Richard 1866-07-11 – 1945-09-26@\textsc{Beer-Hofmann, Richard} (1866-07-11 – 1945-09-26), \emph{Schriftsteller/Schriftstellerin}|pwk}. Am 23. 11. 1904, dem Tag
                  nach der Aufführung, sahen sich die beiden erneut. An diesem Tag dürften sie
                  gemeinsam eine Reaktion auf eine Meldung über die Absetzung von \emph{Das Haus Delorme}\pwindex{Haus Delorme. Eine Familienszene@\emph{Das Haus Delorme. Eine Familienszene}|pwk}\pwindex{Schnitzlers »Haus Delorme«@\emph{Schnitzlers »Haus Delorme«}|pwkv} verfasst haben, vgl. [O. V.]: \emph{Schnitzlers »Haus Delorme«}\pwindex{Schnitzlers »Haus Delorme«@\emph{Schnitzlers »Haus Delorme«}|pwk}. In: \emph{Berliner Tageblatt und -Handelszeitung}\pwindex{Berliner Tageblatt@\emph{Berliner Tageblatt}|pwk}, Jg. 33, Nr. 595, 22. 11. 1904, Abend-Ausgabe, S. 2. Im \emph{Tagebuch}\pwindex{Tagebuch@\emph{Tagebuch}|pwk} erwähnte Schnitzler die Meldung als »infame Notiz«
                     (22. 11. 1904).
                  Der mit Bleistift abgefasste Text ist aus der Perspektive Schnitzlers verfasst, wurde aber von Goldmanns\pwindex{Goldmann, Paul 31.01.1865 – 25.09.1935@\textsc{Goldmann, Paul} (31.01.1865 – 25.09.1935), \emph{Schriftsteller/Schriftstellerin, Journalist/Journalistin}|pwk} Hand niedergeschrieben. Zumindest eine Korrektur
                     (»die Meldung von Seiten der Cenſur«) wurde von Schnitzler vorgenommen, auch die letzten drei Worte stammen
                  von ihm. Das Blatt mit dem Text findet sich heute gemeinsam mit dem vorliegenden
                  Brief im Nachlass Schnitzlers: »{\pb}\strikeout{\textcolor{gray}{E}{ }}Sehr geehrte Redaktion\orgindex{Berliner Tageblatt@Berliner Tageblatt|pwv}, Geſtatten Sie mir, zur Richtigſtellung der
                           Meldungen\pwindex{Schnitzlers »Haus Delorme«@\emph{Schnitzlers »Haus Delorme«}|pwv}, die
                        Sie geſtern bezüglich \strikeout{d} meines noch unveröffentlichten Einakters
                           ›Das Haus \textsc{Delorme}\pwindex{Haus Delorme. Eine Familienszene@\emph{Das Haus Delorme. Eine Familienszene}|pw}‹ publizirt haben, Ihnen Folgendes mitzutheilen: \strikeout{Es iſt \textcolor{gray}{manc}he} Es entſpricht
                        nicht den Thatſachen, daß die Schauſpieler ſich geweigert haben, \strikeout{daß} das Stück\pwindex{Haus Delorme. Eine Familienszene@\emph{Das Haus Delorme. Eine Familienszene}|pwv} zu ſpielen. Freitag war noch Probe. \strikeout{Abends infolge die das Cenſur-} Am Freitag{ }Abend, vor der auf Sonnabend
                        angeſetzten Generalprobe, \strikeout{\textcolor{gray}{er}} erfolgte \substVorne{}\textsuperscript{das Cenſurverbot}\substDazwischen{}die Meldung von Seiten der Cenſur\substHinten{}. Nur aus dieſem Grunde wurde das Stück\pwindex{Haus Delorme. Eine Familienszene@\emph{Das Haus Delorme. Eine Familienszene}|pwv} abgeſetzt. Der Inhalt des Stück\pwindex{Haus Delorme. Eine Familienszene@\emph{Das Haus Delorme. Eine Familienszene}|pwv}es iſt in \strikeout{der Ihrem Blatte\pwindex{Berliner Tageblatt@\emph{Berliner Tageblatt}|pw}} Ihrem Berichte\pwindex{Schnitzlers »Haus Delorme«@\emph{Schnitzlers »Haus Delorme«}|pwv} unrichtig wiedergegeben.{ / }Mit vorzgl Hoch« Abgeschickt wurde dieses Protestschreiben aller Wahrscheinlichkeit nach
                  nicht. Am 24. 11. 1904 war Schnitzler wieder in Wien\oindex{Wien@\textbf{Wien}, \emph{A.ADM2}|pwk} und gab zwei Interviews\pwindex{Arthur Schnitzlers »Haus Delorme«@\emph{Arthur Schnitzlers »Haus Delorme«}|pwkv}\pwindex{Haus Delorme. (Eine Richtigstellung von Arthur Schnitzler.)@\emph{Haus Delorme. (Eine Richtigstellung von Arthur Schnitzler.)}|pwkv} zur Causa (A. S.: \emph{»Das Zeitlose ist von kürzester Dauer«}, [Ludwig Klinenberger]: Arthur Schnitzlers »Haus Delorme«, 25. 11. 1904 und A. S.: \emph{»Das Zeitlose ist von kürzester Dauer«}, [Marco Brociner]: Haus Delorme. (Eine Richtigstellung von Arthur Schnitzler), 25. 11. 1904). Schnitzlers hier getätigten Aussagen wurden
                  am 26. 11. 1904 im \emph{Berliner
                     Tageblatt}\pwindex{Berliner Tageblatt@\emph{Berliner Tageblatt}|pwk} aufgegriffen, zugleich wurde auf der eigenen Darstellung
                  beharrt.}}}\label{K_L03456-1}. Samſtag{ }zwiſchen 6 und 7 bitte ich Dich nicht zu kommen. Ich muß
                  Abends ins Theater\oindex{Lessing-Theater@\textbf{Lessing-Theater}, \emph{Theater (K.THE)}|pwv} (\label{K_L03456-2v}\edtext{\textsc{Dreyer\pwindex{Siebzehnjaehrige@\emph{Die Siebzehnjährige}|pwv}\pwindex{Dreyer, Max 25.09.1862 – 27.11.1946@\textsc{Dreyer, Max} (25.09.1862 – 27.11.1946), \emph{Schriftsteller/Schriftstellerin}|pw}}}{\lemma{\textnormal{\emph{Dreyer}}}\Cendnote{\textnormal{Die Uraufführung von Max Dreyers\pwindex{Dreyer, Max 25.09.1862 – 27.11.1946@\textsc{Dreyer, Max} (25.09.1862 – 27.11.1946), \emph{Schriftsteller/Schriftstellerin}|pwk}{ }\emph{Die Siebzehnjährige}\pwindex{Siebzehnjaehrige@\emph{Die Siebzehnjährige}|pwk} fand am 20. 11. 1904 am Berlin\oindex{Berlin@\textbf{Berlin}, \emph{P.PPLC}|pwk}er \emph{Lessing-Theater}\orgindex{Lessing-Theater@Lessing-Theater|pwk} statt. Goldmann\pwindex{Goldmann, Paul 31.01.1865 – 25.09.1935@\textsc{Goldmann, Paul} (31.01.1865 – 25.09.1935), \emph{Schriftsteller/Schriftstellerin, Journalist/Journalistin}|pwk} nahm vermutlich an der Generalprobe
                  teil.}}}\label{K_L03456-2}) und muß gerade in dieſer Stunde meine \label{K_L03456-3v}\edtext{Telegramme\pwindex{Theater- und Kunstnachrichten [Die Siebzehnjaehrige]@\emph{Theater- und Kunstnachrichten [Die Siebzehnjährige]}|pwv}}{\lemma{\textnormal{\emph{Telegramme}}}\Cendnote{\textnormal{[Paul Goldmann\pwindex{Goldmann, Paul 31.01.1865 – 25.09.1935@\textsc{Goldmann, Paul} (31.01.1865 – 25.09.1935), \emph{Schriftsteller/Schriftstellerin, Journalist/Journalistin}|pwk}]: \emph{Theater- und Kunstnachrichten}\pwindex{Theater- und Kunstnachrichten [Die Siebzehnjaehrige]@\emph{Theater- und Kunstnachrichten [Die Siebzehnjährige]}|pwk}. In: \emph{Neue Freie Presse}\pwindex{Neue Freie Presse@\emph{Neue Freie Presse}|pwk}, Nr. 14.455, 20. 11. 1904, Morgenblatt, S. 12. Für welche weiteren
                  Zeitungen Goldmann\pwindex{Goldmann, Paul 31.01.1865 – 25.09.1935@\textsc{Goldmann, Paul} (31.01.1865 – 25.09.1935), \emph{Schriftsteller/Schriftstellerin, Journalist/Journalistin}|pwk} Theatertelegramme
                  schrieb, wie die Mehrzahlform »Telegramme« hier wohl zu verstehen
                  ist, ist nicht geklärt.}}}\label{K_L03456-3} raſch fertigſtellen. {\pb}Sonntag bin ich leider auch nicht frei, – wohl aber
                  Montag{ }Abend. Ich habe heut mit \textsc{Richard\pwindex{Beer-Hofmann, Richard 1866-07-11 – 1945-09-26@\textsc{Beer-Hofmann, Richard} (1866-07-11 – 1945-09-26), \emph{Schriftsteller/Schriftstellerin}|pw}} telephoniſch ein Beiſammenſein für Montag{ }Abend verabredet, und es wäre ſehr ſchön, wenn Du auch dabei ſein
               könnteſt. Geht das nicht, ſo triffſt Du mich jedenfalls Montag{ }zwiſchen 6 u. 7 Uhr{ }zu Hauſe\oindex{Dessauer Strasse@\textbf{Dessauer Straße}, \emph{Straße (K.STR)}|pwv}. Oder, wenn Du mir
               ſagen kannſt, wo ich Dich um 5 Uhr treffen kann, komme ich auch zu
               Dir.\pend
           
\pstart
           Herzlichſt {\\[\baselineskip]}Dein {\\[\baselineskip]}\spacefill\mbox{Paul Goldmann.}\pend
           \leftskip=0em{}\selectlanguage{ngerman}\endnumbering\briefempfaengerindex{Schnitzler, Arthur@\textsc{Schnitzler, Arthur}!zzzGoldmann, Paul@\emph{von Paul Goldmann}!1904-11-181@{18. 11. {[}1904{]}}|)be}\mylabel{L03456h}  \normalsize

\doendnotes{C}
\bigskip
\vfill

\clearpage

\footnotesize

\lohead{\textsc{register}}

% Definiere theindex-Environment komplett neu ohne reledmac
\makeatletter
\renewenvironment{theindex}{%
  \section*{\indexname}%
  \setlength{\parindent}{0pt}%
  \setlength{\parskip}{0pt plus 0.3pt}%
  \let\item\@idxitem
}{%
  \clearpage
}
\makeatother

\IfFileExists{\jobname-pw.ind}{\input{\jobname-pw.ind}}{}

\end{document}

      