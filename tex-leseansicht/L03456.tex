%% latex-leseansicht-vorspann.tex
%% Vorspann für die Leseansicht.
%% Lädt die gemeinsame Datei latex-vorspann.tex mit nicht gesetztem Schalter.

\newif\ifkorrekturansicht
\korrekturansichtfalse

\input{../tex-inputs/latex-vorspann}

\begin{center}
            \textcolor{red}{ENTWURF, NICHT FERTIG KORRIGIERT}
                      \end{center}
            
         
         \renewcommand{\erwaehntePersonen}{Personen: Richard Beer-Hofmann, Max Dreyer}
         \renewcommand{\erwaehnteOrte}{Orte: Berlin, Dessauer Straße, Kleines Theater, Lessing-Theater}
         \renewcommand{\erwaehnteWerke}{Werke: Der grüne Kakadu. Groteske in einem Akt, Der tapfere Cassian. Puppenspiel in einem Akt, Die Siebzehnjährige, Neue Freie Presse, Theater- und Kunstnachrichten [Die Siebzehnjährige]}
               \section[ Paul Goldmann an Arthur Schnitzler, 18. 11. {[}1904{]}]{ Paul Goldmann an Arthur Schnitzler, 18. 11. {[}1904{]}}\nopagebreak\mylabel{v}\rehead{ }\begin{ledgroupsized}[t]{13cm}\normalsize\beginnumbering \toendnotes[C]{\smallbreak\pagebreak[2]} \Standort{DLA, A:Schnitzler, HS.NZ85.1.3174.}
\physDesc{Brief, 1 Blatt, 2 Seiten, 666 Zeichen
\newline{}Handschrift: blaue Tinte, deutsche Kurrent
\newline{}Schnitzler: 1) mit Bleistift das Jahr »{[}1{]}904« vermerkt  2) mit rotem Buntstift eine Unterstreichung}\toendnotes[C]{\smallbreak}\pstart
           \noindent{}\raggedleft{}{\pb}\textcolor{gray}{\textbf{DESSAUERSTRASSE 19\oindex{Dessauer Strasse@\textbf{Dessauer Straße}|pw}}}\pend
           \pstart
           Berlin\oindex{Berlin@\textbf{Berlin}|pw}, \substVorne{}\textsuperscript{2}\substDazwischen{}1\substHinten{}8. November.\pend
           \pstart\center{}Mein lieber Freund,\pend\pstart
           Ich danke Dir für Deinen Brief und werde mich ſehr freuen, Dich \label{K_L03456-3v}\edtext{bald zu ſehen}{\lemma{\textnormal{\emph{bald zu ſehen}}}\Cendnote{\textnormal{Für die Uraufführung von \emph{Der
                     tapfere Cassian}\pwindex{Schnitzler, Arthur 15.05.1862 – 21.10.1931@\textsc{Schnitzler, Arthur} (15.05.1862 – 21.10.1931), \emph{Schriftsteller, Mediziner}!tapfere Cassian. Puppenspiel in einem Akt01. 02. 1904@\strich\emph{Der tapfere Cassian. Puppenspiel in einem Akt} {[}01. 02. 1904{]}|pwk} und \emph{Der grüne Kakadu}\pwindex{Schnitzler, Arthur 15.05.1862 – 21.10.1931@\textsc{Schnitzler, Arthur} (15.05.1862 – 21.10.1931), \emph{Schriftsteller, Mediziner}!gruene Kakadu. Groteske in einem Akt1. 3. 1899@\strich\emph{Der grüne Kakadu. Groteske in einem Akt} {[}1. 3. 1899{]}|pwk}
                  am 22. 11. 1904 im
                     Kleinen Theater\oindex{Kleines Theater@\textbf{Kleines Theater}|pwk} war Schnitzler\pwindex{Schnitzler, Arthur 15.05.1862 – 21.10.1931@\textsc{Schnitzler, Arthur} (15.05.1862 – 21.10.1931), \emph{Schriftsteller, Mediziner}|pwk} zwischen 13. 11. 1904 und 23. 11. 1904 in Berlin\oindex{Berlin@\textbf{Berlin}|pwk}. Goldmann\pwindex{Goldmann, Paul 31.01.1865 – 25.09.1935@\textsc{Goldmann, Paul} (31.01.1865 – 25.09.1935), \emph{Schriftsteller, Journalist}|pwk} besuchte er am
                     21. 11. 1904 (ohne
                     Richard Beer-Hofmann\pwindex{Beer-Hofmann, Richard 1866-07-11 – 1945-09-26@\textsc{Beer-Hofmann, Richard} (1866-07-11 – 1945-09-26), \emph{Schriftsteller}|pwk}) und am 23. 11. 1904.}}}\label{K_L03456-3h}.
                  Samſtag{ }zwiſchen 6 und 7 bitte ich Dich nicht zu kommen. Ich muß
                  Abends ins Theater\oindex{Lessing-Theater@\textbf{Lessing-Theater}|pwv} (\label{K_L03456-1v}\edtext{\textsc{Dreyer\pwindex{Dreyer, Max 25.09.1862 – 27.11.1946@\textsc{Dreyer, Max} (25.09.1862 – 27.11.1946), \emph{Schriftsteller}!Siebzehnjaehrige1904@\strich\emph{Die Siebzehnjährige} {[}1904{]}|pwv}\pwindex{Dreyer, Max 25.09.1862 – 27.11.1946@\textsc{Dreyer, Max} (25.09.1862 – 27.11.1946), \emph{Schriftsteller}|pw}}}{\lemma{\textnormal{\emph{Dreyer}}}\Cendnote{\textnormal{Die Uraufführung von Max Dreyer\pwindex{Dreyer, Max 25.09.1862 – 27.11.1946@\textsc{Dreyer, Max} (25.09.1862 – 27.11.1946), \emph{Schriftsteller}|pwk}s \emph{Die
                     Siebzehnjährige}\pwindex{Dreyer, Max 25.09.1862 – 27.11.1946@\textsc{Dreyer, Max} (25.09.1862 – 27.11.1946), \emph{Schriftsteller}!Siebzehnjaehrige1904@\strich\emph{Die Siebzehnjährige} {[}1904{]}|pwk} fand am 20. 11. 1904 am Berlin\oindex{Berlin@\textbf{Berlin}|pwk}er Lessing-Theater\oindex{Lessing-Theater@\textbf{Lessing-Theater}|pwk} statt. Goldmann\pwindex{Goldmann, Paul 31.01.1865 – 25.09.1935@\textsc{Goldmann, Paul} (31.01.1865 – 25.09.1935), \emph{Schriftsteller, Journalist}|pwk}
                  nahm vermutlich an der Generalprobe teil.}}}\label{K_L03456-1h}) und muß gerade in dieſer Stunde
               meine \label{K_L03456-2v}\edtext{Telegramm\pwindex{Theater- und Kunstnachrichten [Die Siebzehnjaehrige]1904-09-20@\emph{Theater- und Kunstnachrichten [Die Siebzehnjährige]} {[}1904-09-20{]}|pwv}e}{\lemma{\textnormal{\emph{Telegramme}}}\Cendnote{\textnormal{[Paul Goldmann\pwindex{Goldmann, Paul 31.01.1865 – 25.09.1935@\textsc{Goldmann, Paul} (31.01.1865 – 25.09.1935), \emph{Schriftsteller, Journalist}|pwk}]: \emph{Theater- und Kunstnachrichten}\pwindex{Theater- und Kunstnachrichten [Die Siebzehnjaehrige]1904-09-20@\emph{Theater- und Kunstnachrichten [Die Siebzehnjährige]} {[}1904-09-20{]}|pwk}. In: \emph{Neue Freie Presse}\pwindex{Neue Freie Presse1864 – 1939@\emph{Neue Freie Presse} {[}1864 – 1939{]}|pwk}, Nr. 14455, 20. 11. 1904, Morgenblatt, S. 12.}}}\label{K_L03456-2h} raſch fertigſtellen.
                  {\pb}Sonntag bin ich leider auch nicht frei – wohl aber
                  Montag{ }Abend. Ich habe heut mit \textsc{Richard\pwindex{Beer-Hofmann, Richard 1866-07-11 – 1945-09-26@\textsc{Beer-Hofmann, Richard} (1866-07-11 – 1945-09-26), \emph{Schriftsteller}|pw}} telphoniſch ein Beiſammenſein fürMontag{ }Abend verabredet, und es wäre ſehr ſchön, wenn Du auch dabei ſein
               könnteſt. Geht das nicht, ſo triffſt Du mich jedenfalls Montagzwiſchen 6 u. 7 Uhrzu Hauſe\oindex{Dessauer Strasse@\textbf{Dessauer Straße}|pwv}. Oder, wenn Du mir
               ſagen kannſt, wo ich Dich um 5 Uhr teffen kann, komme ich auch zu
               Dir.\pend
           \pstart
           Herzlichſt {\\[\baselineskip]}Dein {\\[\baselineskip]}\spacefill\mbox{Paul Goldmann.}\pend
           \leftskip=0em{}
         
         \endnumbering\mylabel{h}\end{ledgroupsized}\begin{anhang}\end{anhang}\newcommand{\dateiname}{L03456}\newcommand{\titel}{Paul Goldmann an Arthur Schnitzler, 18. 11. [1904]}\newcommand{\editorInnen}{Martin Anton Müller und Laura Untner}%% latex-leseansicht-abspann.tex
%% Abspann für die Leseansicht.
%% Der Schalter \ifkorrekturansicht ist bereits durch den Vorspann gesetzt.

%% latex-abspann.tex
%% Gemeinsamer Abspann für Korrekturansicht und Leseansicht.
%% Setzt den Schalter \ifkorrekturansicht voraus (gesetzt in den
%% einbindenden Dateien latex-korrekturansicht-abspann.tex bzw.
%% latex-leseansicht-abspann.tex).
%% ---------------------------------------------------------------

\normalsize

% Das esempio-Environment wird nur in der Leseansicht benötigt
\ifkorrekturansicht\else
\newenvironment{esempio}[3]%
{
    \vspace{1.5ex}
    \rlap{\underline{#1}}
    \par
    \setlength{\parindent}{0cm}
    \nopagebreak
    \leftskip=#2cm
    \rightskip=#3cm
}
{
    \par
}
\fi

\doendnotes{C}
\bigskip
\vfill

\clearpage

\footnotesize

\ifkorrekturansicht
  \lohead{\textsc{register}}
\fi

% theindex-Environment neu definieren ohne reledmac
\makeatletter
\renewenvironment{theindex}{%
  \ifkorrekturansicht
    \section*{\indexname}%
  \else
    \subsubsection*{Index der erwähnten Entitäten}%
  \fi
  \setlength{\parindent}{0pt}%
  \setlength{\parskip}{0pt plus 0.3pt}%
  \let\item\@idxitem
}{%
  \ifkorrekturansicht\clearpage\fi
}
\makeatother

\IfFileExists{\jobname-pw.ind}{\input{\jobname-pw.ind}}{}

% Quellenangabe nur in der Leseansicht
\ifkorrekturansicht\else
% Fallback-Definitionen, falls die .tex-Datei \titel etc. nicht gesetzt hat
\providecommand{\titel}{}
\providecommand{\editorInnen}{}
\providecommand{\dateiname}{\jobname}

\vspace{3cm}

\vfill

\footnotesize
\textsc{Quelle}: \titel. Herausgegeben von {\editorInnen}. In: \emph{Arthur Schnitzler: Briefwechsel mit Autorinnen und Autoren}.
 Digitale Edition, https://schnitzler-briefe.acdh.oeaw.ac.at/{\dateiname}.html (Stand \today)
\fi

\end{document}


      