%% latex-leseansicht-vorspann.tex
%% Vorspann für die Leseansicht.
%% Lädt die gemeinsame Datei latex-vorspann.tex mit nicht gesetztem Schalter.

\newif\ifkorrekturansicht
\korrekturansichtfalse

\input{../tex-inputs/latex-vorspann}

\begin{center}
            \textcolor{red}{ENTWURF, NICHT FERTIG KORRIGIERT}
                      \end{center}
            
         
         \renewcommand{\erwaehntePersonen}{Personen: Richard Beer-Hofmann, Max Dreyer, Paul Goldmann}
         \renewcommand{\erwaehnteInstitutionen}{Institutionen: Berliner Tageblatt, Kleines Theater}
         \renewcommand{\erwaehnteOrte}{Orte: Berlin, Dessauer Straße, Lessing-Theater, Wien}
         \renewcommand{\erwaehnteWerke}{Werke: Arthur Schnitzlers »Haus Delorme«, Berliner Tageblatt, Das Haus Delorme. Eine Familienszene, Der grüne Kakadu. Groteske in einem Akt, Der tapfere Cassian. Puppenspiel in einem Akt, Die Siebzehnjährige, Haus Delorme. (Eine Richtigstellung von Arthur Schnitzler.), Neue Freie Presse, Schnitzlers »Haus Delorme«, Tagebuch, Theater- und Kunstnachrichten [Die Siebzehnjährige]}
               \section[ Paul Goldmann an Arthur Schnitzler, 18. 11. {[}1904{]}]{ Paul Goldmann an Arthur Schnitzler, 18. 11. {[}1904{]}}\nopagebreak\mylabel{v}\rehead{ }\begin{ledgroupsized}[t]{13cm}\normalsize\beginnumbering\briefempfaengerindex{Schnitzler, Arthur@\textsc{Schnitzler, Arthur}!zzzGoldmann, Paul@\emph{von Paul Goldmann}!1904-11-181@{18. 11. {[}1904{]}}|(be} \toendnotes[C]{\smallbreak\pagebreak[2]} \Standort{DLA, A:Schnitzler, HS.NZ85.1.3174.}
\physDesc{Brief, 1 Blatt, 2 Seiten, 670 Zeichen
\newline{}Handschrift: blaue Tinte, deutsche Kurrent
\newline{}Schnitzler: 1) mit Bleistift das Jahr »904« vermerkt  2) mit rotem Buntstift eine Unterstreichung}\toendnotes[C]{\smallbreak}\pstart
           \noindent{}\raggedleft{}{\pb}\textcolor{gray}{\textbf{DESSAUERSTRASSE 19\oindex{Dessauer Strasse@\textbf{Dessauer Straße}|pw}}}\pend
           \pstart
           Berlin\oindex{Berlin@\textbf{Berlin}|pw}, \substVorne{}\textsuperscript{2}\substDazwischen{}1\substHinten{}8. November.\pend
           \pstart\center{}Mein lieber Freund,\pend\pstart
           Ich \strikeout{\textcolor{gray}{×}} danke Dir für Deinen Brief und werde mich ſehr freuen, Dich \label{K_L03456-1v}\edtext{bald zu ſehen}{\lemma{\textnormal{\emph{bald zu ſehen}}}\Cendnote{\textnormal{Schnitzler\pwindex{Schnitzler, Arthur 15.05.1862 – 21.10.1931@\textsc{Schnitzler, Arthur} (15.05.1862 – 21.10.1931), \emph{Schriftsteller, Mediziner}|pwk} kam am 13. 11. 1904 nach Berlin\oindex{Berlin@\textbf{Berlin}|pwk}. Am \emph{Kleinen Theater}\orgindex{Kleines Theater@Kleines Theater|pwk} stand die Uraufführung von \emph{Der tapfere Cassian}\pwindex{Schnitzler, Arthur 15.05.1862 – 21.10.1931@\textsc{Schnitzler, Arthur} (15.05.1862 – 21.10.1931), \emph{Schriftsteller, Mediziner}!tapfere Cassian. Puppenspiel in einem Akt01. 02. 1904@\strich\emph{Der tapfere Cassian. Puppenspiel in einem Akt} {[}01. 02. 1904{]}|pwk} und \emph{Das Haus Delorme}\pwindex{Schnitzler, Arthur 15.05.1862 – 21.10.1931@\textsc{Schnitzler, Arthur} (15.05.1862 – 21.10.1931), \emph{Schriftsteller, Mediziner}!Haus Delorme. Eine Familienszene1977@\strich\emph{Das Haus Delorme. Eine Familienszene} {[}1977{]}|pwk} bevor, dazu sollte \emph{Der
                     grüne Kakadu}\pwindex{Schnitzler, Arthur 15.05.1862 – 21.10.1931@\textsc{Schnitzler, Arthur} (15.05.1862 – 21.10.1931), \emph{Schriftsteller, Mediziner}!gruene Kakadu. Groteske in einem Akt1. 3. 1899@\strich\emph{Der grüne Kakadu. Groteske in einem Akt} {[}1. 3. 1899{]}|pwk}\pwindex{Schnitzler, Arthur 15.05.1862 – 21.10.1931@\textsc{Schnitzler, Arthur} (15.05.1862 – 21.10.1931), \emph{Schriftsteller, Mediziner}!gruene Kakadu. Groteske in einem Akt1. 3. 1899@\strich\emph{Der grüne Kakadu. Groteske in einem Akt} {[}1. 3. 1899{]}|pwk} neu gegeben werden. Kurzfristig wurde \emph{Das Haus Delorme}\pwindex{Schnitzler, Arthur 15.05.1862 – 21.10.1931@\textsc{Schnitzler, Arthur} (15.05.1862 – 21.10.1931), \emph{Schriftsteller, Mediziner}!Haus Delorme. Eine Familienszene1977@\strich\emph{Das Haus Delorme. Eine Familienszene} {[}1977{]}|pwk} noch vom Programm genommen, die beiden
                  anderen Stücke wurden erstmals am 22. 11. 1904 aufgeführt. Zu einem Treffen Schnitzler\pwindex{Schnitzler, Arthur 15.05.1862 – 21.10.1931@\textsc{Schnitzler, Arthur} (15.05.1862 – 21.10.1931), \emph{Schriftsteller, Mediziner}|pwk}s und Goldmann\pwindex{Goldmann, Paul 31.01.1865 – 25.09.1935@\textsc{Goldmann, Paul} (31.01.1865 – 25.09.1935), \emph{Schriftsteller, Journalist}|pwk}s kam es am Montag, dem 21. 11. 1904, doch
                  anders als im vorliegenden Brief von Goldmann\pwindex{Goldmann, Paul 31.01.1865 – 25.09.1935@\textsc{Goldmann, Paul} (31.01.1865 – 25.09.1935), \emph{Schriftsteller, Journalist}|pwk} vorgeschlagen, vermutlich ohne den ebenfalls in Berlin\oindex{Berlin@\textbf{Berlin}|pwk} weilenden Richard
                     Beer-Hofmann\pwindex{Beer-Hofmann, Richard 1866-07-11 – 1945-09-26@\textsc{Beer-Hofmann, Richard} (1866-07-11 – 1945-09-26), \emph{Schriftsteller}|pwk}. Am 23. 11. 1904, dem Tag nach der Aufführung, sahen sich die beiden
                  erneut. An diesem Tag dürften sie gemeinsam eine Reaktion auf eine Meldung über die Absetzung von Das Haus
                     Delorme\pwindex{?? Werk@Nicht ermittelte Verfasserinnen und Verfasser!Schnitzlers »Haus Delorme«1904-11-22@\emph{Schnitzlers »Haus Delorme«} {[}1904-11-22{]}|pwkv} ([O. V.]: \emph{Schnitzlers »Haus
                        Delorme«}\pwindex{?? Werk@Nicht ermittelte Verfasserinnen und Verfasser!Schnitzlers »Haus Delorme«1904-11-22@\emph{Schnitzlers »Haus Delorme«} {[}1904-11-22{]}|pwk}. In: \emph{Berliner Tageblatt und
                        -Handelszeitung}\pwindex{?? Werk@Nicht ermittelte Verfasserinnen und Verfasser!Berliner Tageblatt1872 – 1939@\emph{Berliner Tageblatt} {[}1872 – 1939{]}|pwk}, Jg. 33, Nr. 595, 22. 11. 1904, Abend-Ausgabe, S. 2.) verfasst haben. Schnitzler\pwindex{Schnitzler, Arthur 15.05.1862 – 21.10.1931@\textsc{Schnitzler, Arthur} (15.05.1862 – 21.10.1931), \emph{Schriftsteller, Mediziner}|pwk} erwähnt die Meldung als »infame
                     Notiz« in seinem \emph{Tagebuch}\pwindex{\textcolor{red}{\textsuperscript{XXXX1 indx}}!Tagebuch1981 – 2000@\strich\emph{Tagebuch} {[}Hrsg., 1981 – 2000{]}|pwk} (22. 11. 1904). Der mit
                  Bleistift abgefasste Text ist aus der Perspektive Schnitzler\pwindex{Schnitzler, Arthur 15.05.1862 – 21.10.1931@\textsc{Schnitzler, Arthur} (15.05.1862 – 21.10.1931), \emph{Schriftsteller, Mediziner}|pwk}s geschrieben, wurde aber von der Hand Goldmann\pwindex{Goldmann, Paul 31.01.1865 – 25.09.1935@\textsc{Goldmann, Paul} (31.01.1865 – 25.09.1935), \emph{Schriftsteller, Journalist}|pwk}s niedergeschrieben. Zumindest eine Korrektur
                     (»die Meldung von Seite der Cenſur«) wurde von Schnitzler\pwindex{Schnitzler, Arthur 15.05.1862 – 21.10.1931@\textsc{Schnitzler, Arthur} (15.05.1862 – 21.10.1931), \emph{Schriftsteller, Mediziner}|pwk} vorgenommen, auch die letzten drei Worte stammen
                  von ihm. Das Blatt mit dem Text findet sich heute gemeinsam mit dem vorliegenden
                  Brief im Nachlass Schnitzler\pwindex{Schnitzler, Arthur 15.05.1862 – 21.10.1931@\textsc{Schnitzler, Arthur} (15.05.1862 – 21.10.1931), \emph{Schriftsteller, Mediziner}|pwk}s: »{\pb}\strikeout{\textcolor{gray}{E}{ }}Sehr geehrte Redaktion\orgindex{Berliner Tageblatt@Berliner Tageblatt|pwv}, Geſtatten Sie mir, zur
                        Richtigſtellung der Meldungen\pwindex{?? Werk@Nicht ermittelte Verfasserinnen und Verfasser!Schnitzlers »Haus Delorme«1904-11-22@\emph{Schnitzlers »Haus Delorme«} {[}1904-11-22{]}|pwv}, die Sie geſtern bezüglich
                           \strikeout{d} meines noch unveröffentlichten
                        Einakters ›Das Haus \textsc{Delorme}\pwindex{Schnitzler, Arthur 15.05.1862 – 21.10.1931@\textsc{Schnitzler, Arthur} (15.05.1862 – 21.10.1931), \emph{Schriftsteller, Mediziner}!Haus Delorme. Eine Familienszene1977@\strich\emph{Das Haus Delorme. Eine Familienszene} {[}1977{]}|pw}‹ publizirt haben, Ihnen Folgendes mitzutheilen: \strikeout{Es iſt \textcolor{gray}{manc}he} Es entſpricht
                        nicht den Thatſachen, daß die Schauſpieler ſich geweigert haben, \strikeout{daß} das Stück\pwindex{Schnitzler, Arthur 15.05.1862 – 21.10.1931@\textsc{Schnitzler, Arthur} (15.05.1862 – 21.10.1931), \emph{Schriftsteller, Mediziner}!Haus Delorme. Eine Familienszene1977@\strich\emph{Das Haus Delorme. Eine Familienszene} {[}1977{]}|pwv} zu ſpielen. Freitag war noch Probe. \strikeout{Abends infolge die das Cenſur-} Am Freitag{ }Abend, vor der auf Sonnabend
                        angeſetzten Generalprobe, \strikeout{\textcolor{gray}{er}} erfolgte \substVorne{}\textsuperscript{das Cenſurverbot}{\allowbreak}\substDazwischen{}die Meldung von Seiten der Cenſur\substHinten{}. Nur aus dieſem Grunde wurde das Stück\pwindex{Schnitzler, Arthur 15.05.1862 – 21.10.1931@\textsc{Schnitzler, Arthur} (15.05.1862 – 21.10.1931), \emph{Schriftsteller, Mediziner}!Haus Delorme. Eine Familienszene1977@\strich\emph{Das Haus Delorme. Eine Familienszene} {[}1977{]}|pwv} abgeſetzt. Der Inhalt des Stück\pwindex{Schnitzler, Arthur 15.05.1862 – 21.10.1931@\textsc{Schnitzler, Arthur} (15.05.1862 – 21.10.1931), \emph{Schriftsteller, Mediziner}!Haus Delorme. Eine Familienszene1977@\strich\emph{Das Haus Delorme. Eine Familienszene} {[}1977{]}|pwv}es iſt in \strikeout{der Ihrem Blatte\pwindex{?? Werk@Nicht ermittelte Verfasserinnen und Verfasser!Berliner Tageblatt1872 – 1939@\emph{Berliner Tageblatt} {[}1872 – 1939{]}|pw}} Ihrem Berichte\pwindex{?? Werk@Nicht ermittelte Verfasserinnen und Verfasser!Schnitzlers »Haus Delorme«1904-11-22@\emph{Schnitzlers »Haus Delorme«} {[}1904-11-22{]}|pwv} unrichtig wiedergegeben.{ / }Mit vorzgl Hoch« Dieses Protestschreiben dürfte aller Wahrscheinlichkeit nach nicht
                     abgeschickt worden sein. Am 24. 11. 1904 war Schnitzler\pwindex{Schnitzler, Arthur 15.05.1862 – 21.10.1931@\textsc{Schnitzler, Arthur} (15.05.1862 – 21.10.1931), \emph{Schriftsteller, Mediziner}|pwk} wieder
                     in Wien\oindex{Wien@\textbf{Wien}|pwk} und gab zwei Interviews\pwindex{Arthur Schnitzlers »Haus Delorme«1904-11-25@\emph{Arthur Schnitzlers »Haus Delorme«} {[}1904-11-25{]}|pwkv}\pwindex{Haus Delorme. (Eine Richtigstellung von Arthur Schnitzler.)1904-11-25@\emph{Haus Delorme. (Eine Richtigstellung von Arthur Schnitzler.)} {[}1904-11-25{]}|pwkv} zur Causa (A. S.: \emph{»Das Zeitlose ist von kürzester Dauer«}, [Ludwig Klinenberger]: Arthur Schnitzlers »Haus Delorme«, 25. 11. 1904 und A. S.: \emph{»Das Zeitlose ist von kürzester Dauer«}, [Marco Brociner]: Haus Delorme. (Eine Richtigstellung von Arthur Schnitzler), 25. 11. 1904). Schnitzler\pwindex{Schnitzler, Arthur 15.05.1862 – 21.10.1931@\textsc{Schnitzler, Arthur} (15.05.1862 – 21.10.1931), \emph{Schriftsteller, Mediziner}|pwk}s 
                     hier getätigten Aussagen wurden am
                     26. 11. 1904 im \emph{Berliner
                        Tageblatt}\pwindex{?? Werk@Nicht ermittelte Verfasserinnen und Verfasser!Berliner Tageblatt1872 – 1939@\emph{Berliner Tageblatt} {[}1872 – 1939{]}|pwk} aufgegriffen, zugleich wurde auf der eigenen Darstellung zu
                  beharrt.}}}\label{K_L03456-1h}. Samſtag{ }zwiſchen 6 und 7 bitte ich Dich nicht zu kommen. Ich muß
                  Abends ins Theater\oindex{Lessing-Theater@\textbf{Lessing-Theater}|pwv} (\label{K_L03456-2v}\edtext{\textsc{Dreyer\pwindex{Dreyer, Max 25.09.1862 – 27.11.1946@\textsc{Dreyer, Max} (25.09.1862 – 27.11.1946), \emph{Schriftsteller}!Siebzehnjaehrige1904@\strich\emph{Die Siebzehnjährige} {[}1904{]}|pwv}\pwindex{Dreyer, Max 25.09.1862 – 27.11.1946@\textsc{Dreyer, Max} (25.09.1862 – 27.11.1946), \emph{Schriftsteller}|pw}}}{\lemma{\textnormal{\emph{Dreyer}}}\Cendnote{\textnormal{Die Uraufführung von Max Dreyer\pwindex{Dreyer, Max 25.09.1862 – 27.11.1946@\textsc{Dreyer, Max} (25.09.1862 – 27.11.1946), \emph{Schriftsteller}|pwk}s \emph{Die
                     Siebzehnjährige}\pwindex{Dreyer, Max 25.09.1862 – 27.11.1946@\textsc{Dreyer, Max} (25.09.1862 – 27.11.1946), \emph{Schriftsteller}!Siebzehnjaehrige1904@\strich\emph{Die Siebzehnjährige} {[}1904{]}|pwk} fand am 20. 11. 1904 am Berlin\oindex{Berlin@\textbf{Berlin}|pwk}er Lessing-Theater\oindex{Lessing-Theater@\textbf{Lessing-Theater}|pwk} statt. Goldmann\pwindex{Goldmann, Paul 31.01.1865 – 25.09.1935@\textsc{Goldmann, Paul} (31.01.1865 – 25.09.1935), \emph{Schriftsteller, Journalist}|pwk}
                  nahm vermutlich an der Generalprobe teil.}}}\label{K_L03456-2h}) und muß gerade in dieſer Stunde
               meine \label{K_L03456-3v}\edtext{Telegramme\pwindex{Theater- und Kunstnachrichten [Die Siebzehnjaehrige]1904-09-20@\emph{Theater- und Kunstnachrichten [Die Siebzehnjährige]} {[}1904-09-20{]}|pwv}}{\lemma{\textnormal{\emph{Telegramme}}}\Cendnote{\textnormal{[Paul Goldmann\pwindex{Goldmann, Paul 31.01.1865 – 25.09.1935@\textsc{Goldmann, Paul} (31.01.1865 – 25.09.1935), \emph{Schriftsteller, Journalist}|pwk}]: \emph{Theater- und Kunstnachrichten}\pwindex{Theater- und Kunstnachrichten [Die Siebzehnjaehrige]1904-09-20@\emph{Theater- und Kunstnachrichten [Die Siebzehnjährige]} {[}1904-09-20{]}|pwk}. In: \emph{Neue Freie Presse}\pwindex{Neue Freie Presse1864 – 1939@\emph{Neue Freie Presse} {[}1864 – 1939{]}|pwk}, Nr. 14.455, 20. 11. 1904, Morgenblatt, S. 12. Für welche weiteren
                  Zeitungen Goldmann\pwindex{Goldmann, Paul 31.01.1865 – 25.09.1935@\textsc{Goldmann, Paul} (31.01.1865 – 25.09.1935), \emph{Schriftsteller, Journalist}|pwk} Theatertelegramme
                  schrieb, wie die Mehrzahlform »Telegramme« hier wohl zu verstehen
                  ist, ist nicht geklärt.}}}\label{K_L03456-3h} raſch fertigſtellen. {\pb}Sonntag bin ich leider auch nicht frei, – wohl aber
                  Montag{ }Abend. Ich habe heut mit \textsc{Richard\pwindex{Beer-Hofmann, Richard 1866-07-11 – 1945-09-26@\textsc{Beer-Hofmann, Richard} (1866-07-11 – 1945-09-26), \emph{Schriftsteller}|pw}} telephoniſch ein Beiſammenſein für Montag{ }Abend verabredet, und es wäre ſehr ſchön, wenn Du auch dabei ſein
               könnteſt. Geht das nicht, ſo triffſt Du mich jedenfalls Montag{ }zwiſchen 6 u. 7 Uhr{ }zu Hauſe\oindex{Dessauer Strasse@\textbf{Dessauer Straße}|pwv}. Oder, wenn Du mir
               ſagen kannſt, wo ich Dich um 5 Uhr treffen kann, komme ich auch zu
               Dir.\pend
           \pstart
           Herzlichſt {\\[\baselineskip]}Dein {\\[\baselineskip]}\spacefill\mbox{Paul Goldmann.}\pend
           \leftskip=0em{}
         
         \endnumbering\mylabel{h}\end{ledgroupsized}\begin{anhang}\end{anhang}\newcommand{\dateiname}{L03456}\newcommand{\titel}{Paul Goldmann an Arthur Schnitzler, 18. 11. [1904]}\newcommand{\editorInnen}{Martin Anton Müller und Laura Untner}%% latex-leseansicht-abspann.tex
%% Abspann für die Leseansicht.
%% Der Schalter \ifkorrekturansicht ist bereits durch den Vorspann gesetzt.

%% latex-abspann.tex
%% Gemeinsamer Abspann für Korrekturansicht und Leseansicht.
%% Setzt den Schalter \ifkorrekturansicht voraus (gesetzt in den
%% einbindenden Dateien latex-korrekturansicht-abspann.tex bzw.
%% latex-leseansicht-abspann.tex).
%% ---------------------------------------------------------------

\normalsize

% Das esempio-Environment wird nur in der Leseansicht benötigt
\ifkorrekturansicht\else
\newenvironment{esempio}[3]%
{
    \vspace{1.5ex}
    \rlap{\underline{#1}}
    \par
    \setlength{\parindent}{0cm}
    \nopagebreak
    \leftskip=#2cm
    \rightskip=#3cm
}
{
    \par
}
\fi

\doendnotes{C}
\bigskip
\vfill

\clearpage

\footnotesize

\ifkorrekturansicht
  \lohead{\textsc{register}}
\fi

% theindex-Environment neu definieren ohne reledmac
\makeatletter
\renewenvironment{theindex}{%
  \ifkorrekturansicht
    \section*{\indexname}%
  \else
    \subsubsection*{Index der erwähnten Entitäten}%
  \fi
  \setlength{\parindent}{0pt}%
  \setlength{\parskip}{0pt plus 0.3pt}%
  \let\item\@idxitem
}{%
  \ifkorrekturansicht\clearpage\fi
}
\makeatother

\IfFileExists{\jobname-pw.ind}{\input{\jobname-pw.ind}}{}

% Quellenangabe nur in der Leseansicht
\ifkorrekturansicht\else
% Fallback-Definitionen, falls die .tex-Datei \titel etc. nicht gesetzt hat
\providecommand{\titel}{}
\providecommand{\editorInnen}{}
\providecommand{\dateiname}{\jobname}

\vspace{3cm}

\vfill

\footnotesize
\textsc{Quelle}: \titel. Herausgegeben von {\editorInnen}. In: \emph{Arthur Schnitzler: Briefwechsel mit Autorinnen und Autoren}.
 Digitale Edition, https://schnitzler-briefe.acdh.oeaw.ac.at/{\dateiname}.html (Stand \today)
\fi

\end{document}


      