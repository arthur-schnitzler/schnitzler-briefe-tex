%% latex-korrekturansicht-vorspann.tex
%% Vorspann für die Korrekturansicht.
%% Lädt die gemeinsame Datei latex-vorspann.tex mit gesetztem Schalter.

\newif\ifkorrekturansicht
\korrekturansichttrue

\input{../tex-inputs/latex-vorspann}


\section[ Paul Goldmann an Arthur Schnitzler, 15. 5. 1906]{L03245 Paul Goldmann an Arthur Schnitzler, 15. 5. 1906}
\nopagebreak\mylabel{L03245v}
\rehead{ }\normalsize\beginnumbering\briefempfaengerindex{Schnitzler, Arthur@\textsc{Schnitzler, Arthur}!zzzGoldmann, Paul@\emph{von Paul Goldmann}!1906-05-152@{15. 5. 1906}|(be}
\toendnotes[C]{\smallbreak\pagebreak[2]}\Standort{DLA, A:Schnitzler, HS.NZ85.1.3175.}
\physDesc{Bildpostkarte, 290 Zeichen
\newline{}Handschrift: 1) blaue Tinte, deutsche Kurrent\hspace{1em}2) blaue Tinte, lateinische Kurrent (\noindent{}Adresse)\hspace{1em}
\newline{}Versand: Stempel: »\nobreak{}\oindex{Berlin@\textbf{Berlin}, \emph{P.PPLC}|pwk}Berlin\textcolor{gray}{,} W 9, 15. 5. 06, 1–2N\textcolor{gray}{.}\nobreak{}«.  
\newline{}Schnitzler: mit Bleistift das Jahr »906« vermerkt }\toendnotes[C]{\smallbreak}\pstart{}{\pb}Herrn\pend{}\pstart{}Dr. Arthur Schnitzler\pend{}\pstart{}Wien\oindex{Wien@\textbf{Wien}, \emph{A.ADM2}|pw}\pend{}\pstart{}XVIII. Spöttelgaſse 7\oindex{Edmund-Weiss-Gasse 7@\textbf{Edmund-Weiß-Gasse 7}, \emph{Wohngebäude (K.WHS)}|pw}.\pend{}{\bigskip}
\pstart
           \noindent{}\centering{}{\pb}\label{K_L03245-1v}\edtext{{[}Zwei Bäume vor einem Gutshof{]}}{\lemma{\textnormal{\emph{Zwei … Gutshof}}}\Cendnote{\textnormal{»Nordwestdeutscher
                     Kunstverlag«, »Intarsial Karten Serie 1307«}}}\label{K_L03245-1}\pend
           \vspace{1em}
\pstart
           \noindent{}{\pb}Berlin\oindex{Berlin@\textbf{Berlin}, \emph{P.PPLC}|pw}, 15. Mai.
                  Lieber Freund, Haſt Du nicht heut oder morgen{ }\label{K_L03245-2v}\edtext{Geburtstag}{\lemma{\textnormal{\emph{Geburtstag}}}\Cendnote{\textnormal{Am 15. 5. 1906 feierte
                  Schnitzler seinen 44. Geburtstag.}}}\label{K_L03245-2}? Wenn ja, ſo ſende ich Dir herzliche Glückwünſche, – und wenn nicht,
               auch. Herzliche Grüße an Dich, Frau\pwindex{Schnitzler, Olga 17.01.1882 – 13.01.1970@\textsc{Schnitzler, Olga} (17.01.1882 – 13.01.1970), \emph{Schauspieler/Schauspielerin, Sänger/Sängerin}|pwv} und Kind\pwindex{Schnitzler, Heinrich 09.08.1902 – 12.07.1982@\textsc{Schnitzler, Heinrich} (09.08.1902 – 12.07.1982), \emph{Regisseur/Regisseurin, Schauspieler/Schauspielerin}|pwv}
               von\pend
           
\pstart
           Deinem getreuen {\\[\baselineskip]}\spacefill\mbox{Paul Goldmann.}\pend
           \leftskip=0em{}\selectlanguage{ngerman}\endnumbering\briefempfaengerindex{Schnitzler, Arthur@\textsc{Schnitzler, Arthur}!zzzGoldmann, Paul@\emph{von Paul Goldmann}!1906-05-152@{15. 5. 1906}|)be}\mylabel{L03245h}  \normalsize

\doendnotes{C}
\bigskip
\vfill

\clearpage

\footnotesize

\lohead{\textsc{register}}

% Definiere theindex-Environment komplett neu ohne reledmac
\makeatletter
\renewenvironment{theindex}{%
  \section*{\indexname}%
  \setlength{\parindent}{0pt}%
  \setlength{\parskip}{0pt plus 0.3pt}%
  \let\item\@idxitem
}{%
  \clearpage
}
\makeatother

\IfFileExists{\jobname-pw.ind}{\input{\jobname-pw.ind}}{}

\end{document}

      