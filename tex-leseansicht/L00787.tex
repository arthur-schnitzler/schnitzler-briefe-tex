%% latex-korrekturansicht-vorspann.tex
%% Vorspann für die Korrekturansicht.
%% Lädt die gemeinsame Datei latex-vorspann.tex mit gesetztem Schalter.

\newif\ifkorrekturansicht
\korrekturansichttrue

\input{../tex-inputs/latex-vorspann}


\section[Arthur Schnitzler an Georg Brandes, 27. 3. 1898]{L00787 Arthur Schnitzler an Georg Brandes, 27. 3. 1898}
\nopagebreak\mylabel{L00787v}
\rehead{ }\normalsize\beginnumbering\briefempfaengerindex{Brandes, Georg@\textsc{Brandes, Georg}!zzzSchnitzler, Arthur@\emph{von Arthur Schnitzler}!1898-03-271@{27. 3. 1898}|(be}
\toendnotes[C]{\smallbreak\pagebreak[2]}\Standort{Kopenhagen, Det Kongelige Bibliotek, Georg Brandes Arkiv, box 125.}
\physDesc{Brief, 3 Blätter, 12 Seiten, 3696 Zeichen
\newline{}Handschrift: schwarze Tinte, deutsche Kurrent
\newline{}Ordnung: mit Bleistift von unbekannter Hand nummeriert: »11.
                                    Schnitzler« sowie das Datum unterhalb der Datierung
                                 wiederholt: »27–3–98«; auf dem zweiten und dritten Blatt ebenfalls mit
                                 Bleistift: »27/3 98« }
\buchAbdrucke{\weitereDrucke{1) Georg Brandes, Arthur Schnitzler: \emph{Ein Briefwechsel}. Bern: \emph{Francke} 1956, S. 67–69.} \weitereDrucke{2) Arthur Schnitzler: \emph{Briefe 1875–1912}. Frankfurt am Main: \emph{S. Fischer} 1981, S. 348–350.} }\toendnotes[C]{\smallbreak}
\pstart
           \raggedleft{}{\pb}Wien\oindex{Wien@\textbf{Wien}, \emph{A.ADM2}|pw}, 27. 3. 98\pend
           
\pstart{}Verehrteſter Herr Brandes,\pend\vspace{0.5em}
\pstart
           es war wirklich nicht nothwendig uns für etwas zu danken, was uns ſelbſt ſo viel
               Freude gemacht hat wie die Möglichkeit während Ihres Wien\oindex{Wien@\textbf{Wien}, \emph{A.ADM2}|pw}er Aufenthalts einige Stunden mit Ihnen zu verbringen; jedenfalls aber
               freut mich Ihre liebe Nachricht aus Sicilien\oindex{Sizilien@\textbf{Sizilien}, \emph{A.ADM1}|pw}, die
               mir von Ihrem Wohlbefinden ſo ange{\pb}nehme Kunde
               gibt. Über Ihre Aufnahme in Rom\oindex{Rom@\textbf{Rom}, \emph{P.PPLC}|pw} hatte ich ſchon
               irgendwo geleſen; der ungeſtörte Fortgang Ihrer Reiſe ließ mich auch vermuthen, daſs
               Sie von Hauſe günſtige Mittheilungen erhielten, was mir nun durch Ihren Brief
               erfreulich beſtätigt wird. Wir haben auch aus Kopenhagen\oindex{Kopenhagen@\textbf{Kopenhagen}, \emph{P.PPLC}|pw} Ihre Bücher geſchickt bekommen; herzlichen Dank dafür. Den
                  \label{K_L00787-1v}\edtext{Band}{\lemma{\textnormal{\emph{Band}}}\Cendnote{\textnormal{1897 erschien von \emph{Die Hauptströmungen
                     der Literatur des neunzehnten Jahrhunderts}\pwindex{Hauptstroemungen der Literatur des neunzehnten Jahrhunderts@\emph{Hauptströmungen der Literatur des neunzehnten Jahrhunderts}|pwk} im Verlag \emph{Barsdorf}\orgindex{Hermann Barsdorf@Hermann Barsdorf|pwk} eine »fünfte, gänzlich neu bearbeitete und
                     bedeutend vermehrte Auflage« in 27 Lieferungen.}}}\label{K_L00787-1} aus den Hauptſtrömungen\pwindex{Hauptstroemungen der Literatur des neunzehnten Jahrhunderts@\emph{Hauptströmungen der Literatur des neunzehnten Jahrhunderts}|pw} hab ich ſchon gekannt, in der
               früheren {\pb}Ausgabe; dagegen habe ich Ihre Rede über das Nationalgefühl\pwindex{Nationalgefuehl@\emph{Nationalgefühl}|pwv}
               zum erſten Mal geleſen. Ich glaube dſs ſie als ein wahres Muſter ihrer Gattung gelten
               kann, da ſie ſchwungvoll und ſachlich zugleich iſt.\pend
           
\pstart
           Die \label{K_L00787-2v}\edtext{Aufnahme}{\lemma{\textnormal{\emph{Aufnahme}}}\Cendnote{\textnormal{\emph{Freiwild}\pwindex{Freiwild. Schauspiel in 3 Akten@\emph{Freiwild. Schauspiel in 3 Akten}|pwk} wurde vom 4. 2. 1898 bis zum
                     26. 2. 1898 am Carl-Theater in
                     Wien\oindex{Carl-Theater@\textbf{Carl-Theater}, \emph{Theater (K.THE)}|pwk} gegeben.}}}\label{K_L00787-2} des »Freiwild\pwindex{Freiwild. Schauspiel in 3 Akten@\emph{Freiwild. Schauspiel in 3 Akten}|pw}«,
               nach der Sie ſich erkundigen, war hier am erſten Abend eine ſehr gute; die Kritik war
               im ganzen wenig wohlwollend. Sie wiſſen, daſs ich ſelbſt {\pb}eine geringe Meinung von dem künſtleriſchen Werth
               dieſes Stücks\pwindex{Freiwild. Schauspiel in 3 Akten@\emph{Freiwild. Schauspiel in 3 Akten}|pwv} habe; aber davon
               war wenig die Rede. Dagegen \strikeout{flo} iſt bei der
               Beſprechung der angeblichen Tendenz ſo viel Bornirtheit und Verlogenheit aufgeflogen
               – wie Staubwolken, wenn ein galoppirendes Roſs über die Landſtraße jagt. Insbeſondre
               die antiſemitiſchen Blätter leiſteten unglaubliches in Denunziationen. Es iſt
               ſchließlich ſo weit geko{\geminationm}en, daſs die Direktion {\pb}des Theaters\oindex{Carl-Theater@\textbf{Carl-Theater}, \emph{Theater (K.THE)}|pwv} nach ſieben Vorſtellungen »auf einen Wink von
               oben«, (über den man mir ſelbſt nur unter 4 Augen Aufſchluß geben wollte, was ich
               nicht annahm) das Stück abſetzte. –\pend
           
\pstart
           Mein neues Schauspiel\pwindex{Vermaechtnis. Schauspiel in drei Akten@\emph{Das Vermächtnis. Schauspiel in drei Akten}|pwv} ko{\geminationm}t im Herbſt in der Burg\oindex{Burgtheater@\textbf{Burgtheater}, \emph{S.THTR}|pw} dran (we{\geminationn} die Hofcensur nichts dawider
               hat); jetzt habe ich ein paar einaktige Sachen\pwindex{gruene Kakadu – Paracelsus – Die Gefaehrtin. Drei Einakter@\emph{Der grüne Kakadu – Paracelsus – Die Gefährtin. Drei Einakter}|pwv} geſchrieben und möchte bald wieder an was größeres
               gehen. Bei dem neuen Schauſpiel\pwindex{Vermaechtnis. Schauspiel in drei Akten@\emph{Das Vermächtnis. Schauspiel in drei Akten}|pwv} iſt mir ſtärker als je ein Grundmangel {\pb}meines Schaffens zum Bewußtſein gekommen. Ich
               finde nemlich, daſs mir die \label{K_L00787-3v}\edtext{Nebenfiguren meiſtens nicht übel gelingen; hingegen iſt meine Hauptperson \strikeout{\textcolor{gray}{meiſtens}} i{\geminationm}er irgend wer, dem was ſehr trauriges
                  paſſirt}{\lemma{\textnormal{\emph{Nebenfiguren … paſſirt}}}\Cendnote{\textnormal{Vgl. A. S.: \emph{Tagebuch}, 21. 2. 1898.
               }}}\label{K_L00787-3} – und nicht viel mehr. Sie holt ihre Bedeutung aus ihrem Schickſal, nicht aus
               ihrem Weſen.\pend
           
\pstart
           Die »Luſt\pwindex{Lust@\emph{Lust}|pw}« von d’Annuncio\pwindex{DAnnunzio, Gabriele 12.03.1863 – 01.03.1938@\textsc{D’Annunzio, Gabriele} (12.03.1863 – 01.03.1938), \emph{Schriftsteller/Schriftstellerin}|pw}, die Sie auf der Reiſe geleſen haben, war mir auch nicht
               ſympathiſch. Vor allem ſchien mir einiger \textsc{Snobismus}{ }{\pb}drin zu ſtecken; auch Bildungs\textsc{snobismus}. Dagegen wäre möglicherweiſe nichts einzuwenden,
                  we{\geminationn} nicht gewiſſe künſtleriſche Schwächen daraus
               hervorgingen. Ein Dichter hat gewiſs das Recht zu ſagen: Sie ſah aus wie die \textsc{Madonna}\pwindex{Sixtinische Madonna@\emph{Sixtinische Madonna}|pw} von \textsc{Rafael}\pwindex{Raffaello Sanzio da Urbino 28. 3. oder 6. 4. 1483 – 6.04.1520@\textsc{Raffaello Sanzio da Urbino} (28. 3. oder 6. 4. 1483 – 6.04.1520), \emph{Maler/Malerin}|pw} in \textsc{Dresden}\oindex{Dresden@\textbf{Dresden}, \emph{P.PPLA}|pw} oder er erinnerte mich an ein Portrait von Rembrandt\pwindex{Rembrandt van Rijn 15.07.1606 – 04.10.1669@\textsc{Rembrandt van Rijn} (15.07.1606 – 04.10.1669), \emph{Maler/Malerin}|pw}; – aber er darf nicht verlangen, daſs ich mir was vorſtellen ſoll,
                  we{\geminationn} er ſchildert: Sie hat Hände wie die {\pb}Dame auf dem Bild eines unbeka{\geminationn}ten Malers das in einer unbekannten Galerie in einer
               ganz kleinen italieniſchen\oindex{Italien@\textbf{Italien}, \emph{A.PCLI}|pw}{ }Stadt hängt. Derartiges findet ſich in der »Luſt\pwindex{Lust@\emph{Lust}|pw}« nicht gerade ſelten. – Was ich aber ſonſt
               von d’Annuncio\pwindex{DAnnunzio, Gabriele 12.03.1863 – 01.03.1938@\textsc{D’Annunzio, Gabriele} (12.03.1863 – 01.03.1938), \emph{Schriftsteller/Schriftstellerin}|pw} kenne, hat mich mit Bewunderung
               erfüllt. Ich meine den »Triumph des Todes\pwindex{Triumph des Todes@\emph{Triumph des Todes}|pw}« und
               die »Unſchuldige\pwindex{Unschuldige@\emph{Unschuldige}|pw}.« –\pend
           
\pstart
           Wie lange bleiben Sie noch in Italien\oindex{Italien@\textbf{Italien}, \emph{A.PCLI}|pw}? Werden
               wir bald wieder von {\pb}Ihnen hören? Ich brauche die
               »Wir« nicht näher zu bezeichnen. Paul Goldmann\pwindex{Goldmann, Paul 31.01.1865 – 25.09.1935@\textsc{Goldmann, Paul} (31.01.1865 – 25.09.1935), \emph{Schriftsteller/Schriftstellerin, Journalist/Journalistin}|pw}
               geht auf etwa ein halbes Jahr nach China\oindex{China@\textbf{China}, \emph{A.PCLI}|pw} und Japan\oindex{Japan@\textbf{Japan}, \emph{A.PCLI}|pw}, im Auftrag ſeines Blattes\orgindex{Frankfurter Zeitung@Frankfurter Zeitung|pwv}; er ſchifft ſich am
                  5. April in Genua\oindex{Genua@\textbf{Genua}, \emph{P.PPLA}|pw} ein. Ich will
               in der Charwoche per Rad vom Bre{\geminationn}er\oindex{Brenner@\textbf{Brenner}, \emph{T.PASS}|pw} aus durchs Ampezzothal\oindex{Valle DAmpezzo@\textbf{Valle d’Ampezzo}, \emph{T.VAL}|pw}
               nach Venedig\oindex{Venedig@\textbf{Venedig}, \emph{P.PPLA}|pw}.\pend
           
\pstart
           Von meiner Mama\pwindex{Schnitzler, Louise 1840-07-08 – 1911-09-09@\textsc{Schnitzler, Louise} (1840-07-08 – 1911-09-09)|pwv} und Beer-Hofmann\pwindex{Beer-Hofmann, Richard 1866-07-11 – 1945-09-26@\textsc{Beer-Hofmann, Richard} (1866-07-11 – 1945-09-26), \emph{Schriftsteller/Schriftstellerin}|pw} habe ich Ihnen die beſten Grüße zu
               ſagen; {\pb}mögen Sie, verehrteſter Herr Brandes,
               angenehmes denken und angenehmes erleben und uns, wenn Sie ſich auf der Rückreiſe
               wieder in Wien\oindex{Wien@\textbf{Wien}, \emph{A.ADM2}|pw} aufhalten (was dringend gewünſcht
               wird) mancherlei davon erzählen.\pend
           
\pstart
           Herzlichſt ergeben{\\[\baselineskip]}Ihr \spacefill\mbox{ArthurSchnitzler}\pend
           \leftskip=0em{}\selectlanguage{ngerman}\endnumbering\briefempfaengerindex{Brandes, Georg@\textsc{Brandes, Georg}!zzzSchnitzler, Arthur@\emph{von Arthur Schnitzler}!1898-03-271@{27. 3. 1898}|)be}\mylabel{L00787h}  \normalsize

\doendnotes{C}
\bigskip
\vfill

\clearpage

\footnotesize

\lohead{\textsc{register}}

% Definiere theindex-Environment komplett neu ohne reledmac
\makeatletter
\renewenvironment{theindex}{%
  \section*{\indexname}%
  \setlength{\parindent}{0pt}%
  \setlength{\parskip}{0pt plus 0.3pt}%
  \let\item\@idxitem
}{%
  \clearpage
}
\makeatother

\IfFileExists{\jobname-pw.ind}{\input{\jobname-pw.ind}}{}

\end{document}

      