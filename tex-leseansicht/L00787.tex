%% latex-leseansicht-vorspann.tex
%% Vorspann für die Leseansicht.
%% Lädt die gemeinsame Datei latex-vorspann.tex mit nicht gesetztem Schalter.

\newif\ifkorrekturansicht
\korrekturansichtfalse

\input{../tex-inputs/latex-vorspann}


               \section[Arthur Schnitzler an Georg Brandes, 27. 3. 1898]{ Arthur Schnitzler an Georg Brandes, 27. 3. 1898}\nopagebreak\mylabel{v}\rehead{ }\begin{ledgroupsized}[t]{13cm}\normalsize\beginnumbering\briefempfaengerindex{Brandes, Georg@\textsc{Brandes, Georg}!zzzSchnitzler, Arthur@\emph{von Arthur Schnitzler}!1898-03-271@{27. 3. 1898}|(be} \toendnotes[C]{\smallbreak\pagebreak[2]} \Standort{Kopenhagen, Det Kongelige Bibliotek, Georg Brandes Arkiv, box 125.}
\physDesc{Brief, 3 Blätter, 12 Seiten
\newline{}Handschrift: schwarze Tinte, deutsche Kurrent\newline{}Ordnung: mit Bleistift von unbekannter Hand nummeriert: »11.
                                    Schnitzler« sowie das Datum unterhalb der Datierung
                                 wiederholt: »27–3–98«; auf dem zweiten und dritten Blatt ebenfalls mit
                                 Bleistift: »27/3 98« }\buchAbdrucke{\weitereDrucke{1) Georg Brandes, Arthur Schnitzler: \emph{Ein Briefwechsel}. Hg. Kurt Bergel. Bern: \emph{Francke} 1956, S. 67–69.} \weitereDrucke{2) Arthur Schnitzler: \emph{Briefe 1875–1912}. Hg. Therese Nickl und Heinrich Schnitzler. Frankfurt am Main: \emph{S. Fischer} 1981, S. 348–350.} }\toendnotes[C]{\smallbreak}\pstart
           \raggedleft{}{\pb}Wien\oindex{Wien@\textbf{Wien}|pw}, 27. 3. 98\pend
           \pstart{}Verehrteſter Herr Brandes,\pend\pstart
           es war wirklich nicht nothwendig uns für etwas zu danken, was uns ſelbſt ſo viel
               Freude gemacht hat wie die Möglichkeit während Ihres Wien\oindex{Wien@\textbf{Wien}|pw}er Aufenthalts einige Stunden mit Ihnen zu verbringen; jedenfalls aber
               freut mich Ihre liebe Nachricht aus Sicilien\oindex{Sizilien@\textbf{Sizilien}|pw}, die
               mir von Ihrem Wohlbefinden ſo ange{\pb}nehme Kunde
               gibt. Über Ihre Aufnahme in Rom\oindex{Rom@\textbf{Rom}|pw} hatte ich ſchon
               irgendwo geleſen; der ungeſtörte Fortgang Ihrer Reiſe ließ mich auch vermuthen, daſs
               Sie von Hauſe günſtige Mittheilungen erhielten, was mir nun durch Ihren Brief
               erfreulich beſtätigt wird. Wir haben auch aus Kopenhagen\oindex{Kopenhagen@\textbf{Kopenhagen}|pw} Ihre Bücher geſchickt bekommen; herzlichen Dank dafür. Den
                  \label{K_L00787_1v}\edtext{Band}{\lemma{\textnormal{\emph{Band}}}\Cendnote{\textnormal{1897 erschien von \emph{Die Hauptströmungen
                     der Literatur des neunzehnten Jahrhunderts}\pwindex{Brandes, Georg 04.02.1842 – 19.02.1927@\textsc{Brandes, Georg} (04.02.1842 – 19.02.1927)!Hauptstroemungen der Literatur des neunzehnten Jahrhunderts1872@\strich\emph{Hauptströmungen der Literatur des neunzehnten Jahrhunderts} {[}1872{]}|pwk} im Verlag \emph{Barsdorf}\orgindex{Hermann Barsdorf@Hermann Barsdorf|pwk} eine »fünfte, gänzlich neu bearbeitete und
                     bedeutend vermehrte Auflage« in 27 Lieferungen.}}}\label{K_L00787_1h} aus den Hauptſtrömungen\pwindex{Brandes, Georg 04.02.1842 – 19.02.1927@\textsc{Brandes, Georg} (04.02.1842 – 19.02.1927)!Hauptstroemungen der Literatur des neunzehnten Jahrhunderts1872@\strich\emph{Hauptströmungen der Literatur des neunzehnten Jahrhunderts} {[}1872{]}|pw} hab ich ſchon gekannt, in der
               früheren {\pb}Ausgabe; dagegen habe ich Ihre Rede über das Nationalgefühl\pwindex{Brandes, Georg 04.02.1842 – 19.02.1927@\textsc{Brandes, Georg} (04.02.1842 – 19.02.1927)!Nationalgefuehl01. 02. 1894@\strich\emph{Nationalgefühl} {[}01. 02. 1894{]}|pwv} zum
               erſten Mal geleſen. Ich glaube dſs ſie als ein wahres Muſter ihrer Gattung gelten
               kann, da ſie ſchwungvoll und ſachlich zugleich iſt.\pend
           \pstart
           Die \label{K_L00787_2v}\edtext{Aufnahme}{\lemma{\textnormal{\emph{Aufnahme}}}\Cendnote{\textnormal{\emph{Freiwild}\pwindex{Schnitzler, Arthur 15.05.1862 – 21.10.1931@\textsc{Schnitzler, Arthur} (15.05.1862 – 21.10.1931), \emph{Schriftsteller, Mediziner}!Freiwild. Schauspiel in 3 Akten1896@\strich\emph{Freiwild. Schauspiel in 3 Akten} {[}1896{]}|pwk} wurde vom 4. 2. 1898 bis
                  zum 26. 2. 1898 am Carl-Theater in
                     Wien\oindex{Carl-Theater@\textbf{Carl-Theater}|pwk} gegeben.}}}\label{K_L00787_2h} des »Freiwild\pwindex{Schnitzler, Arthur 15.05.1862 – 21.10.1931@\textsc{Schnitzler, Arthur} (15.05.1862 – 21.10.1931), \emph{Schriftsteller, Mediziner}!Freiwild. Schauspiel in 3 Akten1896@\strich\emph{Freiwild. Schauspiel in 3 Akten} {[}1896{]}|pw}«,
               nach der Sie ſich erkundigen, war hier am erſten Abend eine ſehr gute; die Kritik war
               im ganzen wenig wohlwollend. Sie wiſſen, daſs ich ſelbſt {\pb}eine geringe Meinung von dem künſtleriſchen Werth
               dieſes Stücks\pwindex{Schnitzler, Arthur 15.05.1862 – 21.10.1931@\textsc{Schnitzler, Arthur} (15.05.1862 – 21.10.1931), \emph{Schriftsteller, Mediziner}!Freiwild. Schauspiel in 3 Akten1896@\strich\emph{Freiwild. Schauspiel in 3 Akten} {[}1896{]}|pwv} habe; aber davon
               war wenig die Rede. Dagegen \strikeout{flo} iſt bei der
               Beſprechung der angeblichen Tendenz ſo viel Bornirtheit und Verlogenheit aufgeflogen
               – wie Staubwolken, wenn ein galoppirendes Roſs über die Landſtraße jagt. Insbeſondre
               die antiſemitiſchen Blätter leiſteten unglaubliches in Denunziationen. Es iſt
               ſchließlich ſo weit geko{\geminationm}en, daſs die Direktion {\pb}des Theaters\oindex{Carl-Theater@\textbf{Carl-Theater}|pwv} nach ſieben Vorſtellungen »auf einen Wink von oben«, (über den man
               mir ſelbſt nur unter 4 Augen Aufſchluß geben wollte, was ich nicht annahm) das Stück
               abſetzte. –\pend
           \pstart
           Mein neues Schauspiel\pwindex{Schnitzler, Arthur 15.05.1862 – 21.10.1931@\textsc{Schnitzler, Arthur} (15.05.1862 – 21.10.1931), \emph{Schriftsteller, Mediziner}!Vermaechtnis. Schauspiel in drei Akten1898@\strich\emph{Das Vermächtnis. Schauspiel in drei Akten} {[}1898{]}|pwv} ko{\geminationm}t im Herbſt in der Burg\oindex{Burgtheater@\textbf{Burgtheater}|pw}
               dran (we{\geminationn} die Hofcensur nichts dawider hat); jetzt habe
               ich ein paar einaktige Sachen\pwindex{Schnitzler, Arthur 15.05.1862 – 21.10.1931@\textsc{Schnitzler, Arthur} (15.05.1862 – 21.10.1931), \emph{Schriftsteller, Mediziner}!gruene Kakadu – Paracelsus – Die Gefaehrtin. Drei Einakter1.3.1899 – 1.3.1899@\strich\emph{Der grüne Kakadu – Paracelsus – Die Gefährtin. Drei Einakter} {[}1.3.1899 – 1.3.1899{]}|pwv}
               geſchrieben und möchte bald wieder an was größeres gehen. Bei dem neuen Schauſpiel\pwindex{Schnitzler, Arthur 15.05.1862 – 21.10.1931@\textsc{Schnitzler, Arthur} (15.05.1862 – 21.10.1931), \emph{Schriftsteller, Mediziner}!Vermaechtnis. Schauspiel in drei Akten1898@\strich\emph{Das Vermächtnis. Schauspiel in drei Akten} {[}1898{]}|pwv} iſt mir ſtärker als je
               ein Grundmangel {\pb}meines Schaffens zum Bewußtſein
               gekommen. Ich finde nemlich, daſs mir die \label{K_L00787_3v}\edtext{Nebenfiguren meiſtens nicht übel gelingen; hingegen iſt meine
               Hauptperson \strikeout{\textcolor{gray}{meiſtens}} i{\geminationm}er irgend wer, dem was ſehr trauriges
                  paſſirt}{\lemma{\textnormal{\emph{Nebenfiguren … paſſirt}}}\Cendnote{\textnormal{vgl. A. S.: \emph{Tagebuch}, 21. 2. 1898}}}\label{K_L00787_3h} – und nicht viel mehr.
               Sie holt ihre Bedeutung aus ihrem Schickſal, nicht aus ihrem Weſen.\pend
           \pstart
           Die »Luſt\pwindex{DAnnunzio, Gabriele 12.03.1863 – 01.03.1938@\textsc{D’Annunzio, Gabriele} (12.03.1863 – 01.03.1938), \emph{Schriftsteller}!Lust1889@\strich\emph{Lust} {[}1889{]}|pw}« von d’Annuncio\pwindex{DAnnunzio, Gabriele 12.03.1863 – 01.03.1938@\textsc{D’Annunzio, Gabriele} (12.03.1863 – 01.03.1938), \emph{Schriftsteller}|pw}, die Sie auf der Reiſe geleſen haben, war mir auch nicht
               ſympathiſch. Vor allem ſchien mir einiger \textsc{Snobismus}{ }{\pb}drin zu ſtecken; auch Bildungs\textsc{snobismus}. Dagegen wäre möglicherweiſe nichts einzuwenden,
                  we{\geminationn} nicht gewiſſe künſtleriſche Schwächen daraus
               hervorgingen. Ein Dichter hat gewiſs das Recht zu ſagen: Sie ſah aus wie die \textsc{Madonna}\pwindex{Raffaello Sanzio da Urbino 28. 3. oder 6. 4. 1483 – 6.04.1520@\textsc{Raffaello Sanzio da Urbino} (28. 3. oder 6. 4. 1483 – 6.04.1520), \emph{Maler}!Sixtinische Madonna1512 – 1513@\strich\emph{Sixtinische Madonna} {[}1512 – 1513{]}|pw} von \textsc{Rafael}\pwindex{Raffaello Sanzio da Urbino 28. 3. oder 6. 4. 1483 – 6.04.1520@\textsc{Raffaello Sanzio da Urbino} (28. 3. oder 6. 4. 1483 – 6.04.1520), \emph{Maler}|pw} in \textsc{Dresden}\oindex{Dresden@\textbf{Dresden}|pw} oder er erinnerte mich an ein Portrait von Rembrandt\pwindex{Rembrandt van Rijn 15.07.1606 – 04.10.1669@\textsc{Rembrandt van Rijn} (15.07.1606 – 04.10.1669), \emph{Maler}|pw}; – aber er darf nicht verlangen, daſs ich mir was vorſtellen ſoll,
                  we{\geminationn} er ſchildert: Sie hat Hände wie die {\pb}Dame auf dem Bild eines unbeka{\geminationn}ten Malers das in einer unbekannten Galerie in einer
               ganz kleinen italieniſchen\oindex{Italien@\textbf{Italien}|pw}{ }Stadt hängt. Derartiges findet ſich in der »Luſt\pwindex{DAnnunzio, Gabriele 12.03.1863 – 01.03.1938@\textsc{D’Annunzio, Gabriele} (12.03.1863 – 01.03.1938), \emph{Schriftsteller}!Lust1889@\strich\emph{Lust} {[}1889{]}|pw}« nicht gerade ſelten. – Was ich aber ſonſt von
                  d’Annuncio\pwindex{DAnnunzio, Gabriele 12.03.1863 – 01.03.1938@\textsc{D’Annunzio, Gabriele} (12.03.1863 – 01.03.1938), \emph{Schriftsteller}|pw} kenne, hat mich mit Bewunderung
               erfüllt. Ich meine den »Triumph des Todes\pwindex{DAnnunzio, Gabriele 12.03.1863 – 01.03.1938@\textsc{D’Annunzio, Gabriele} (12.03.1863 – 01.03.1938), \emph{Schriftsteller}!Triumph des Todes1894@\strich\emph{Triumph des Todes} {[}1894{]}|pw}« und die
                  »Unſchuldige\pwindex{DAnnunzio, Gabriele 12.03.1863 – 01.03.1938@\textsc{D’Annunzio, Gabriele} (12.03.1863 – 01.03.1938), \emph{Schriftsteller}!Unschuldige1892@\strich\emph{Unschuldige} {[}1892{]}|pw}.« –\pend
           \pstart
           Wie lange bleiben Sie noch in Italien\oindex{Italien@\textbf{Italien}|pw}? Werden wir
               bald wieder von {\pb}Ihnen hören? Ich brauche die
               »Wir« nicht näher zu bezeichnen. Paul Goldmann\pwindex{Goldmann, Paul 31.01.1865 – 25.09.1935@\textsc{Goldmann, Paul} (31.01.1865 – 25.09.1935), \emph{Schriftsteller, Journalist}|pw}
               geht auf etwa ein halbes Jahr nach China\oindex{China@\textbf{China}|pw} und Japan\oindex{Japan@\textbf{Japan}|pw}, im Auftrag ſeines Blattes\orgindex{Frankfurter Zeitung@Frankfurter Zeitung|pwv}; er ſchifft ſich am 5. April in Genua\oindex{Genua@\textbf{Genua}|pw} ein. Ich will in der Charwoche per Rad vom Bre{\geminationn}er\oindex{Brenner@\textbf{Brenner}|pw} aus durchs
                  Ampezzothal nach Venedig\oindex{Venedig@\textbf{Venedig}|pw}.\pend
           \pstart
           Von meiner Mama\pwindex{Schnitzler, Louise 08.07.1840 – 09.09.1911@\textsc{Schnitzler, Louise} (08.07.1840 – 09.09.1911)|pwv} und Beer-Hofmann\pwindex{Beer-Hofmann, Richard 11.07.1866 – 26.09.1945@\textsc{Beer-Hofmann, Richard} (11.07.1866 – 26.09.1945), \emph{Schriftsteller}|pw} habe ich Ihnen die beſten Grüße zu
               ſagen; {\pb}mögen Sie, verehrteſter Herr Brandes,
               angenehmes denken und angenehmes erleben und uns, wenn Sie ſich auf der Rückreiſe
               wieder in Wien\oindex{Wien@\textbf{Wien}|pw} aufhalten (was dringend gewünſcht
               wird) mancherlei davon erzählen.\pend
           \pstart
           Herzlichſt ergeben{\\[\baselineskip]}Ihr \spacefill\mbox{ArthurSchnitzler}\pend
           \leftskip=0em{}          \endnumbering\briefempfaengerindex{Brandes, Georg@\textsc{Brandes, Georg}!zzzSchnitzler, Arthur@\emph{von Arthur Schnitzler}!1898-03-271@{27. 3. 1898}|)be}\mylabel{h}\end{ledgroupsized}  \newcommand{\dateiname}{L00787}\newcommand{\titel}{Arthur Schnitzler an Georg Brandes, 27. 3. 1898}\newcommand{\editorInnen}{Martin Anton Müller und Gerd-Hermann Susen}
            \footnotesize
\begin{ledgroupsized}[t]{11.5cm}
\doendnotes{C}
\end{ledgroupsized}
         %% latex-leseansicht-abspann.tex
%% Abspann für die Leseansicht.
%% Der Schalter \ifkorrekturansicht ist bereits durch den Vorspann gesetzt.

%% latex-abspann.tex
%% Gemeinsamer Abspann für Korrekturansicht und Leseansicht.
%% Setzt den Schalter \ifkorrekturansicht voraus (gesetzt in den
%% einbindenden Dateien latex-korrekturansicht-abspann.tex bzw.
%% latex-leseansicht-abspann.tex).
%% ---------------------------------------------------------------

\normalsize

% Das esempio-Environment wird nur in der Leseansicht benötigt
\ifkorrekturansicht\else
\newenvironment{esempio}[3]%
{
    \vspace{1.5ex}
    \rlap{\underline{#1}}
    \par
    \setlength{\parindent}{0cm}
    \nopagebreak
    \leftskip=#2cm
    \rightskip=#3cm
}
{
    \par
}
\fi

\doendnotes{C}
\bigskip
\vfill

\clearpage

\footnotesize

\ifkorrekturansicht
  \lohead{\textsc{register}}
\fi

% theindex-Environment neu definieren ohne reledmac
\makeatletter
\renewenvironment{theindex}{%
  \ifkorrekturansicht
    \section*{\indexname}%
  \else
    \subsubsection*{Index der erwähnten Entitäten}%
  \fi
  \setlength{\parindent}{0pt}%
  \setlength{\parskip}{0pt plus 0.3pt}%
  \let\item\@idxitem
}{%
  \ifkorrekturansicht\clearpage\fi
}
\makeatother

\IfFileExists{\jobname-pw.ind}{\input{\jobname-pw.ind}}{}

% Quellenangabe nur in der Leseansicht
\ifkorrekturansicht\else
% Fallback-Definitionen, falls die .tex-Datei \titel etc. nicht gesetzt hat
\providecommand{\titel}{}
\providecommand{\editorInnen}{}
\providecommand{\dateiname}{\jobname}

\vspace{3cm}

\vfill

\footnotesize
\textsc{Quelle}: \titel. Herausgegeben von {\editorInnen}. In: \emph{Arthur Schnitzler: Briefwechsel mit Autorinnen und Autoren}.
 Digitale Edition, https://schnitzler-briefe.acdh.oeaw.ac.at/{\dateiname}.html (Stand \today)
\fi

\end{document}


      