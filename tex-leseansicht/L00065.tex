%% latex-korrekturansicht-vorspann.tex
%% Vorspann für die Korrekturansicht.
%% Lädt die gemeinsame Datei latex-vorspann.tex mit gesetztem Schalter.

\newif\ifkorrekturansicht
\korrekturansichttrue

\input{../tex-inputs/latex-vorspann}


\section[Hugo von Hofmannsthal an Arthur Schnitzler, {[}25. – 29. 1. 1892?{]}]{L00065 Hugo von Hofmannsthal an Arthur Schnitzler, {[}25. – 29. 1.
               1892?{]}}
\nopagebreak\mylabel{L00065v}
\rehead{ }\normalsize\beginnumbering\briefempfaengerindex{Schnitzler, Arthur@\textsc{Schnitzler, Arthur}!zzzHofmannsthal, Hugo von@\emph{von Hugo von Hofmannsthal}!1892-01-291@{{[}25. – 29. 1.
                  1892?{]}}|(be}
\toendnotes[C]{\smallbreak\pagebreak[2]}\Standort{CUL, Schnitzler, B 43.}
\physDesc{Briefkarte, 225 Zeichen
\newline{}Handschrift: Bleistift, deutsche Kurrent
\newline{}Schnitzler: mit Bleistift datiert: »Anfg 92.« 
\newline{}Ordnung: mit Bleistift von unbekannter Hand nummeriert: »13« }
\buchAbdrucke{\weitereDrucke{Hugo von Hofmannsthal, Arthur Schnitzler: \emph{Briefwechsel}. Frankfurt am Main: \emph{S. Fischer} 1964, S. 14.} }\toendnotes[C]{\smallbreak}
\pstart
           {\pb}\textcolor{gray}{\textbf{\label{T_L00065-1v}\edtext{AvH}{\lemma{\textnormal{\emph{AvH}}}\Cendnote{\textnormal{Monogramm der Mutter Anna von Hofmannsthal\pwindex{Hofmannsthal, Anna von 27.01.1849 – 22.03.1904@\textsc{Hofmannsthal, Anna von} (27.01.1849 – 22.03.1904)|pwk} mit Krone in Golddruck}}}\label{T_L00065-1}}}\pend
           
\pstart{}Lieber Freund.\pend\vspace{0.5em}
\pstart
           Bitte ſchreiben Sie ſich auch da hinein. Näheres \label{K_L00065-1v}\edtext{Sonntag}{\lemma{\textnormal{\emph{Sonntag}}}\Cendnote{\textnormal{Das erste
                  Treffen nach dem Erscheinen von \emph{Der Sohn}\pwindex{Sohn. Aus den Papieren eines Arztes@\emph{Der Sohn. Aus den Papieren eines Arztes}|pwk}
                  lässt sich für den 31. 1. 1892 belegen, wodurch sich dieses Korrespondenzstück
                  zeitlich vorne und hinten eingrenzen lässt. Eine weitere kleine Einschränkung gibt
                  der Umstand, dass am Vortag nicht mehr von »Sonntag« sondern von »morgen« die Rede
                  gewesen sein dürfte, was den 30. ausschließt.}}}\label{K_L00065-1}. Die Idee und
               die \label{K_L00065-2v}\edtext{3 letzten Zeilen}{\lemma{\textnormal{\emph{3 letzten Zeilen}}}\Cendnote{\textnormal{Das stützt die Datierung Schnitzlers, da \emph{Der
                     Sohn}\pwindex{Sohn. Aus den Papieren eines Arztes@\emph{Der Sohn. Aus den Papieren eines Arztes}|pwk} im Januarheft der \emph{Freien Bühne}\pwindex{Freie Buehne fuer modernes Leben@\emph{Freie Bühne für modernes Leben}|pwk}
                  erschienen ist. Schnitzler vermerkte dies am 24. 1. 1892 im \emph{Tagebuch}\pwindex{Tagebuch@\emph{Tagebuch}|pwk}, weswegen anzunehmen ist, dass auch
                     Hofmannsthal\pwindex{Hofmannsthal, Hugo von 1874-02-01 – 1929-07-15@\textsc{Hofmannsthal, Hugo von} (1874-02-01 – 1929-07-15), \emph{Schriftsteller/Schriftstellerin}|pwk} in etwa zu dieser Zeit die
                  Möglichkeit hatte, die Geschichte zu lesen.}}}\label{K_L00065-2} vom »Sohn\pwindex{Sohn. Aus den Papieren eines Arztes@\emph{Der Sohn. Aus den Papieren eines Arztes}|pw}« ſind ganz 1892; das übrige etwas älter, aber
               gar nicht {\pb}bös. Ich hoffe, daſs
               Sie gut aufgelegt ſind\pend
           
\pstart
           Herzlichſt{\\[\baselineskip]}\spacefill\mbox{Loris}\pend
           \leftskip=0em{}\selectlanguage{ngerman}\endnumbering\briefempfaengerindex{Schnitzler, Arthur@\textsc{Schnitzler, Arthur}!zzzHofmannsthal, Hugo von@\emph{von Hugo von Hofmannsthal}!1892-01-251@{{[}25. – 29. 1.
                  1892?{]}}|)be}\mylabel{L00065h}  \normalsize

\doendnotes{C}
\bigskip
\vfill

\clearpage

\footnotesize

\lohead{\textsc{register}}

% Definiere theindex-Environment komplett neu ohne reledmac
\makeatletter
\renewenvironment{theindex}{%
  \section*{\indexname}%
  \setlength{\parindent}{0pt}%
  \setlength{\parskip}{0pt plus 0.3pt}%
  \let\item\@idxitem
}{%
  \clearpage
}
\makeatother

\IfFileExists{\jobname-pw.ind}{\input{\jobname-pw.ind}}{}

\end{document}

      