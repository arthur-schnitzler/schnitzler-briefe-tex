%% latex-korrekturansicht-vorspann.tex
%% Vorspann für die Korrekturansicht.
%% Lädt die gemeinsame Datei latex-vorspann.tex mit gesetztem Schalter.

\newif\ifkorrekturansicht
\korrekturansichttrue

\input{../tex-inputs/latex-vorspann}


\section[Arthur Schnitzler an Richard Beer-Hofmann, 7. 10. 1907]{L01715 Arthur Schnitzler an Richard Beer-Hofmann, 7. 10. 1907}
\nopagebreak\mylabel{L01715v}
\rehead{ }\normalsize\beginnumbering\briefempfaengerindex{Beer-Hofmann, Richard@\textsc{Beer-Hofmann, Richard}!zzzSchnitzler, Arthur@\emph{von Arthur Schnitzler}!1907-10-072@{7. 10. 1907}|(be}
\toendnotes[C]{\smallbreak\pagebreak[2]}\Standort{YCGL, MSS 31.}
\physDesc{Postkarte, 205 Zeichen
\newline{}Handschrift: Bleistift, deutsche Kurrent
\newline{}Versand: 1) Rohrpost  2) Stempel: »\nobreak{}\oindex{XVIII., Waehring@\textbf{XVIII., Währing}, \emph{A.ADM3}|pwk}18/1 Wien 111, 7 X 07, 7–\nobreak{}«.  3) Stempel: »\nobreak{}\oindex{XVIII., Waehring@\textbf{XVIII., Währing}, \emph{A.ADM3}|pwk}Wien 18/1, 7 X 07, 7\textsuperscript{10}\nobreak{}«. }\pstart{}{\pb}\textcolor{gray}{\textbf{Dr. Arthur Schnitzler}}\pend{}\pstart{}\textcolor{gray}{\textbf{Wien XVIII. Spoettelgasse 7\oindex{Edmund-Weiss-Gasse 7@\textbf{Edmund-Weiß-Gasse 7}, \emph{Wohngebäude (K.WHS)}|pw}.}}\pend{}{\bigskip}\pstart{}Herrn \textsc{Dr. Richard Beerhofmann}\pend{}\pstart{}Wien XVIII\oindex{XVIII., Waehring@\textbf{XVIII., Währing}, \emph{A.ADM3}|pw}\pend{}\pstart{}\textsc{Hasenauerstraße 59}\oindex{Hasenauerstrasse 59@\textbf{Hasenauerstraße 59}, \emph{Wohngebäude (K.WHS)}|pw}. \pend{}{\bigskip}\vspace{1em}
\pstart
           \noindent{}{\pb}Wir bitten Sie und Paula\pwindex{Beer-Hofmann, Paula 25.02.1879 – 30.10.1939@\textsc{Beer-Hofmann, Paula} (25.02.1879 – 30.10.1939)|pw} morgen Dinſtag{ }Abend bei uns zu ſpeiſen. Paul
                  Goldm.\pwindex{Goldmann, Paul 31.01.1865 – 25.09.1935@\textsc{Goldmann, Paul} (31.01.1865 – 25.09.1935), \emph{Schriftsteller/Schriftstellerin, Journalist/Journalistin}|pw} u Leo Vanj.\pwindex{Van-Jung, Leo 15.10.1866 – 02.07.1939@\textsc{Van-Jung, Leo} (15.10.1866 – 02.07.1939), \emph{Gesangspädagoge/Gesangspädagogin, Mathematiker/Mathematikerin}|pw} ſind verſtändigt.
                  Ko{\geminationm}en Sie nicht ſpät. (½ 8 etwa.) \pend
           
\pstart
           Herzlichſt Ihr{\\[\baselineskip]}\spacefill\mbox{A.}\pend
           \leftskip=0em{}\selectlanguage{ngerman}\endnumbering\briefempfaengerindex{Beer-Hofmann, Richard@\textsc{Beer-Hofmann, Richard}!zzzSchnitzler, Arthur@\emph{von Arthur Schnitzler}!1907-10-072@{7. 10. 1907}|)be}\mylabel{L01715h}  \normalsize

\doendnotes{C}
\bigskip
\vfill

\clearpage

\footnotesize

\lohead{\textsc{register}}

% Definiere theindex-Environment komplett neu ohne reledmac
\makeatletter
\renewenvironment{theindex}{%
  \section*{\indexname}%
  \setlength{\parindent}{0pt}%
  \setlength{\parskip}{0pt plus 0.3pt}%
  \let\item\@idxitem
}{%
  \clearpage
}
\makeatother

\IfFileExists{\jobname-pw.ind}{\input{\jobname-pw.ind}}{}

\end{document}

      