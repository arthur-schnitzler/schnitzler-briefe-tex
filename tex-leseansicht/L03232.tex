%% latex-leseansicht-vorspann.tex
%% Vorspann für die Leseansicht.
%% Lädt die gemeinsame Datei latex-vorspann.tex mit nicht gesetztem Schalter.

\newif\ifkorrekturansicht
\korrekturansichtfalse

\input{../tex-inputs/latex-vorspann}


         
         \renewcommand{\erwaehntePersonen}{Personen: Louise Schnitzler, Olga Schnitzler}
         \renewcommand{\erwaehnteOrte}{Orte: Berlin, Edmund-Weiß-Gasse 7, Wien}
         \renewcommand{\erwaehnteWerke}{}
               \section[ Paul Goldmann an Arthur Schnitzler, 12. 5. 1905]{ Paul Goldmann an Arthur Schnitzler, 12. 5. 1905}\nopagebreak\mylabel{v}\rehead{ }\begin{ledgroupsized}[t]{13cm}\normalsize\beginnumbering \toendnotes[C]{\smallbreak\pagebreak[2]} \Standort{DLA, A:Schnitzler, HS.NZ85.1.3175.}
\physDesc{Postkarte, 538 Zeichen
\newline{}Handschrift: 1) blaue Tinte, deutsche Kurrent\hspace{1em}2) blaue Tinte, lateinische Kurrent (\noindent{}Adresse)\hspace{1em}
\newline{}Versand: 1) Stempel: »\nobreak{}\oindex{Berlin@\textbf{Berlin}|pwk}Berlin SW 11, 12. 5. 05, 4–5N\nobreak{}«.   2) Stempel: »\nobreak{}Wien 110, {[}1{]}\textcolor{gray}{3}. 5. 05, Bestellt\nobreak{}«. 
\newline{}Schnitzler: mit Bleistift das Jahr »{[}19{]}05« vermerkt }\toendnotes[C]{\smallbreak}\pstart{}{\pb}Herrn\pend{}\pstart{}Dr. Arthur Schnitzler\pend{}\pstart{}Wien\oindex{Wien@\textbf{Wien}|pw}\pend{}\pstart{}XVIII. Spöttelgaſse 7\oindex{Edmund-Weiss-Gasse 7@\textbf{Edmund-Weiß-Gasse 7}|pw}.\pend{}{\bigskip}\pstart
           \noindent{}{\pb}Berlin\oindex{Berlin@\textbf{Berlin}|pw}, 12. Mai.
                  Lieber Freund, Ich habe ſehr bedauert, Dich in
                  \label{K_L03232-1v}\edtext{Wien\oindex{Wien@\textbf{Wien}|pw} nicht angetroffen}{\lemma{\textnormal{\emph{Wien nicht angetroffen}}}\Cendnote{\textnormal{siehe Paul Goldmann an Arthur Schnitzler, 3. 5. 1905}}}\label{K_L03232-1h} zu haben, und danke Dir nachträglich für die Einladung, die mich nicht
               erreicht hat. Hoffentlich gibt mir der \label{K_L03232-2v}\edtext{Sommer}{\lemma{\textnormal{\emph{Sommer}}}\Cendnote{\textnormal{Am 31. 7. 1905 besuchte
                     Goldmann\pwindex{Goldmann, Paul 31.01.1865 – 25.09.1935@\textsc{Goldmann, Paul} (31.01.1865 – 25.09.1935), \emph{Schriftsteller, Journalist}|pwk}{ }Schnitzler\pwindex{Schnitzler, Arthur 15.05.1862 – 21.10.1931@\textsc{Schnitzler, Arthur} (15.05.1862 – 21.10.1931), \emph{Schriftsteller, Mediziner}|pwk} in Wien\oindex{Wien@\textbf{Wien}|pwk}.}}}\label{K_L03232-2h} Gelegenheit, Dich zu ſehen. Laß’ mich jedenfalls wiſſen, wo
               Du biſt. Mit Deiner Mutter\pwindex{Schnitzler, Louise 1840-07-08 – 1911-09-09@\textsc{Schnitzler, Louise} (1840-07-08 – 1911-09-09)|pwv}
               habe ich ſo halb und halb ein Zuſammentreffen verabredet. Herzliche Grüße an Dich und
               Deine Frau\pwindex{Schnitzler, Olga 17.01.1882 – 13.01.1970@\textsc{Schnitzler, Olga} (17.01.1882 – 13.01.1970), \emph{Schauspielerin, Sängerin}|pwv} von Deinem {\\}\spacefill\mbox{Paul Goldmann.}\pend
           \pstart
           \noindent{}Haſt Du nicht dieſer Tage Deinen \label{K_L03232-3v}\edtext{Geburtstag}{\lemma{\textnormal{\emph{Geburtstag}}}\Cendnote{\textnormal{Schnitzler\pwindex{Schnitzler, Arthur 15.05.1862 – 21.10.1931@\textsc{Schnitzler, Arthur} (15.05.1862 – 21.10.1931), \emph{Schriftsteller, Mediziner}|pwk} wurde am 15. 5. 1905 43 Jahre alt.}}}\label{K_L03232-3h}? Wenn ja, ſo gratulire ich \strikeout{D\textcolor{gray}{ic}h} Dir herzlich.\pend
           
         
         \endnumbering\mylabel{h}\end{ledgroupsized}  \newcommand{\dateiname}{L03232}\newcommand{\titel}{Paul Goldmann an Arthur Schnitzler, 12. 5. 1905}\newcommand{\editorInnen}{Martin Anton Müller und Laura Untner}%% latex-leseansicht-abspann.tex
%% Abspann für die Leseansicht.
%% Der Schalter \ifkorrekturansicht ist bereits durch den Vorspann gesetzt.

%% latex-abspann.tex
%% Gemeinsamer Abspann für Korrekturansicht und Leseansicht.
%% Setzt den Schalter \ifkorrekturansicht voraus (gesetzt in den
%% einbindenden Dateien latex-korrekturansicht-abspann.tex bzw.
%% latex-leseansicht-abspann.tex).
%% ---------------------------------------------------------------

\normalsize

% Das esempio-Environment wird nur in der Leseansicht benötigt
\ifkorrekturansicht\else
\newenvironment{esempio}[3]%
{
    \vspace{1.5ex}
    \rlap{\underline{#1}}
    \par
    \setlength{\parindent}{0cm}
    \nopagebreak
    \leftskip=#2cm
    \rightskip=#3cm
}
{
    \par
}
\fi

\doendnotes{C}
\bigskip
\vfill

\clearpage

\footnotesize

\ifkorrekturansicht
  \lohead{\textsc{register}}
\fi

% theindex-Environment neu definieren ohne reledmac
\makeatletter
\renewenvironment{theindex}{%
  \ifkorrekturansicht
    \section*{\indexname}%
  \else
    \subsubsection*{Index der erwähnten Entitäten}%
  \fi
  \setlength{\parindent}{0pt}%
  \setlength{\parskip}{0pt plus 0.3pt}%
  \let\item\@idxitem
}{%
  \ifkorrekturansicht\clearpage\fi
}
\makeatother

\IfFileExists{\jobname-pw.ind}{\input{\jobname-pw.ind}}{}

% Quellenangabe nur in der Leseansicht
\ifkorrekturansicht\else
% Fallback-Definitionen, falls die .tex-Datei \titel etc. nicht gesetzt hat
\providecommand{\titel}{}
\providecommand{\editorInnen}{}
\providecommand{\dateiname}{\jobname}

\vspace{3cm}

\vfill

\footnotesize
\textsc{Quelle}: \titel. Herausgegeben von {\editorInnen}. In: \emph{Arthur Schnitzler: Briefwechsel mit Autorinnen und Autoren}.
 Digitale Edition, https://schnitzler-briefe.acdh.oeaw.ac.at/{\dateiname}.html (Stand \today)
\fi

\end{document}


      