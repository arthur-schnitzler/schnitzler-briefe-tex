%% latex-leseansicht-vorspann.tex
%% Vorspann für die Leseansicht.
%% Lädt die gemeinsame Datei latex-vorspann.tex mit nicht gesetztem Schalter.

\newif\ifkorrekturansicht
\korrekturansichtfalse

\input{../tex-inputs/latex-vorspann}


\section[ Arthur Schnitzler an Felix Salten, 12. 1. 1905]{L02996 Arthur Schnitzler an Felix Salten,  12. 1. 1905}
\nopagebreak\mylabel{L02996v}
\rehead{ }\normalsize\beginnumbering\briefempfaengerindex{Salten, Felix@\textsc{Salten, Felix}!zzzSchnitzler, Arthur@\emph{von Arthur Schnitzler}!1905-01-121@{12. 1. 1905}|(be}
\toendnotes[C]{\smallbreak\pagebreak[2]}
\correspDesc{Versand  durch Arthur Schnitzler am 12. 1. 1905 in Wien
\newline{}Erhalt  durch Felix Salten im Zeitraum [12. 1. 1905
                  – 15. 1. 1905?] in Wien}\toendnotes[C]{\smallbreak}
\Standort{Wienbibliothek im Rathaus, ZPH 1681, 2.1.516.}
\physDesc{Kartenbrief, 364 Zeichen
\newline{}Handschrift: schwarze Tinte, deutsche Kurrent
\newline{}Versand: Stempel: »\nobreak{}\oindex{VIII., Josefstadt@\textbf{VIII., Josefstadt}, \emph{Verwaltungsgebiet}|pwk}18/1 Wien 110, 12. II. 16, XI\nobreak{}«.  
\newline{}Ordnung: mit Bleistift von unbekannter Hand nummeriert: »19« }\toendnotes[C]{\smallbreak}\pstart{}{\pb}\textsc{Herrn Felix Salten}\pend{}\pstart{}Wien IX\oindex{IX., Alsergrund@\textbf{IX., Alsergrund}, \emph{Verwaltungsgebiet}|pw}\pend{}\pstart{}\textsc{Porzellang. 45\oindex{Wien@\textbf{Wien}!IX., Alsergrund@\textbf{IX., Alsergrund}!Porzellangasse@\textbf{Porzellangasse}, \emph{Straße}|pw}.}\pend{}{\bigskip}\vspace{1em}
\pstart
           \raggedleft{}{\pb}\label{K_L02996-1v}\edtext{12. 1. 905}{\lemma{\textnormal{\emph{12. 1. 905}}}\Cendnote{\textnormal{Der Poststempel war bei der Jahreszahl um je eine Stelle
                     verdreht.}}}\label{K_L02996-1}\pend
           \vspace{0.5em}
\pstart
           lieber, herzlichen Dank für Ihren \label{K_L02996-2v}\edtext{Brief}{\lemma{\textnormal{\emph{Brief}}}\Cendnote{\textnormal{XXXX Auszeichnungsfehler: Dokument L03405 nicht gefunden.
               }}}\label{K_L02996-2}. Ich habe der \textsc{Hervay\pwindex{Hervay von Kirchberg, Elvira Leontine 18.\,7.\,1860 Poznan – nach 1929@\textsc{Hervay von Kirchberg, Elvira Leontine} (18.\,7.\,1860 Poznan – nach 1929)|pw}}{ }ſelbſt geſchrieben, ganz ehrlich, die Gründe; warum ich überhaupt, nicht nur
               für{ }ſie, nicht leſe.\pend
           
\pstart
           Die \textsc{S.\pwindex{Sandrock, Adele 19.\,8.\,1863 Rotterdam – 30.\,8.\,1937 Berlin@\textsc{Sandrock, Adele} (19.\,8.\,1863 Rotterdam – 30.\,8.\,1937 Berlin), \emph{Schauspielerin}|pw}}{ }ſchrieb mir geſtern, dſs die Nieſe\pwindex{Niese, Hansi 10.\,11.\,1875 Wien – 4.\,4.\,1934 ebd.@\textsc{Niese, Hansi} (10.\,11.\,1875 Wien – 4.\,4.\,1934 ebd.), \emph{Schauspielerin}|pw} auch mit Freuden zugeſagt habe.\pend
           
\pstart
           Auf bald. Bei uns im Hauſe lag oder liegt alles; jetzt die Kinderfrau\pwindex{Loew, Anna *~11.\,4.\,1888 Ješín@\textsc{Loew, Anna} (*~11.\,4.\,1888 Ješín), \emph{Kinderbetreuerin, Dienstbotin}|pwv} und Olga\pwindex{Schnitzler, Olga 17.\,1.\,1882 Wien – 13.\,1.\,1970 Lugano@\textsc{Schnitzler, Olga} (17.\,1.\,1882 Wien – 13.\,1.\,1970 Lugano), \emph{Schauspielerin, Sängerin}|pw}.\pend
           
\pstart
           Herzlichſt Ihr {\\[\baselineskip]}\spacefill\mbox{A.}\pend
           \leftskip=0em{}\selectlanguage{ngerman}\endnumbering\briefempfaengerindex{Salten, Felix@\textsc{Salten, Felix}!zzzSchnitzler, Arthur@\emph{von Arthur Schnitzler}!1905-01-121@{12. 1. 1905}|)be}\mylabel{L02996h}  \newcommand{\dateiname}{L02996}\newcommand{\titel}{Arthur Schnitzler an Felix Salten, 12. 1. 1905}\newcommand{\editorInnen}{Martin Anton Müller und Laura Untner}%% latex-leseansicht-abspann.tex
%% Abspann für die Leseansicht.
%% Der Schalter \ifkorrekturansicht ist bereits durch den Vorspann gesetzt.

%% latex-abspann.tex
%% Gemeinsamer Abspann für Korrekturansicht und Leseansicht.
%% Setzt den Schalter \ifkorrekturansicht voraus (gesetzt in den
%% einbindenden Dateien latex-korrekturansicht-abspann.tex bzw.
%% latex-leseansicht-abspann.tex).
%% ---------------------------------------------------------------

\normalsize

% Das esempio-Environment wird nur in der Leseansicht benötigt
\ifkorrekturansicht\else
\newenvironment{esempio}[3]%
{
    \vspace{1.5ex}
    \rlap{\underline{#1}}
    \par
    \setlength{\parindent}{0cm}
    \nopagebreak
    \leftskip=#2cm
    \rightskip=#3cm
}
{
    \par
}
\fi

\doendnotes{C}
\bigskip
\vfill

\clearpage

\footnotesize

\ifkorrekturansicht
  \lohead{\textsc{register}}
\fi

% theindex-Environment neu definieren ohne reledmac
\makeatletter
\renewenvironment{theindex}{%
  \ifkorrekturansicht
    \section*{\indexname}%
  \else
    \subsubsection*{Index der erwähnten Entitäten}%
  \fi
  \setlength{\parindent}{0pt}%
  \setlength{\parskip}{0pt plus 0.3pt}%
  \let\item\@idxitem
}{%
  \ifkorrekturansicht\clearpage\fi
}
\makeatother

\IfFileExists{\jobname-pw.ind}{\input{\jobname-pw.ind}}{}

% Quellenangabe nur in der Leseansicht
\ifkorrekturansicht\else
% Fallback-Definitionen, falls die .tex-Datei \titel etc. nicht gesetzt hat
\providecommand{\titel}{}
\providecommand{\editorInnen}{}
\providecommand{\dateiname}{\jobname}

\vspace{3cm}

\vfill

\footnotesize
\textsc{Quelle}: \titel. Herausgegeben von {\editorInnen}. In: \emph{Arthur Schnitzler: Briefwechsel mit Autorinnen und Autoren}.
 Digitale Edition, https://schnitzler-briefe.acdh.oeaw.ac.at/{\dateiname}.html (Stand \today)
\fi

\end{document}


