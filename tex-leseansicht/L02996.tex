%% latex-korrekturansicht-vorspann.tex
%% Vorspann für die Korrekturansicht.
%% Lädt die gemeinsame Datei latex-vorspann.tex mit gesetztem Schalter.

\newif\ifkorrekturansicht
\korrekturansichttrue

\input{../tex-inputs/latex-vorspann}


\section[ Arthur Schnitzler an Felix Salten, 12. 1. 1905]{L02996 Arthur Schnitzler an Felix Salten, 12. 1. 1905}
\nopagebreak\mylabel{L02996v}
\rehead{ }\normalsize\beginnumbering\briefempfaengerindex{Salten, Felix@\textsc{Salten, Felix}!zzzSchnitzler, Arthur@\emph{von Arthur Schnitzler}!1905-01-121@{12. 1. 1905}|(be}
\toendnotes[C]{\smallbreak\pagebreak[2]}\Standort{Wienbibliothek im Rathaus, ZPH 1681, 2.1.516.}
\physDesc{Kartenbrief, 364 Zeichen
\newline{}Handschrift: schwarze Tinte, deutsche Kurrent
\newline{}Versand: Stempel: »\nobreak{}\oindex{VIII., Josefstadt@\textbf{VIII., Josefstadt}, \emph{A.ADM3}|pwk}18/1 Wien 110, 12. II. 16, XI\nobreak{}«.  
\newline{}Ordnung: mit Bleistift von unbekannter Hand nummeriert: »19« }\toendnotes[C]{\smallbreak}\pstart{}{\pb}\textsc{Herrn Felix Salten}\pend{}\pstart{}Wien IX\oindex{IX., Alsergrund@\textbf{IX., Alsergrund}, \emph{A.ADM3}|pw}\pend{}\pstart{}\textsc{Porzellang. 45\oindex{Porzellangasse@\textbf{Porzellangasse}, \emph{Straße (K.STR)}|pw}.}\pend{}{\bigskip}\vspace{1em}
\pstart
           \raggedleft{}{\pb}\label{K_L02996-1v}\edtext{12. 1. 905}{\lemma{\textnormal{\emph{12. 1. 905}}}\Cendnote{\textnormal{Der Poststempel war bei der Jahreszahl um je eine Stelle
                     verdreht.}}}\label{K_L02996-1}\pend
           \vspace{0.5em}
\pstart
           lieber, herzlichen Dank für Ihren \label{K_L02996-2v}\edtext{Brief}{\lemma{\textnormal{\emph{Brief}}}\Cendnote{\textnormal{Felix Salten an Arthur Schnitzler, 11. 1. 1905.
               }}}\label{K_L02996-2}. Ich habe der \textsc{Hervay\pwindex{Hervay von Kirchberg, Elvira Leontine 18.07.1860 – nach 1929@\textsc{Hervay von Kirchberg, Elvira Leontine} (18.07.1860 – nach 1929)|pw}} ſelbſt geſchrieben, ganz ehrlich, die Gründe; warum ich überhaupt, nicht nur
               für ſie, nicht leſe.\pend
           
\pstart
           Die \textsc{S.\pwindex{Sandrock, Adele 1863-08-19 – 1937-08-30@\textsc{Sandrock, Adele} (1863-08-19 – 1937-08-30), \emph{Schauspieler/Schauspielerin}|pw}} ſchrieb mir geſtern, dſs die Nieſe\pwindex{Niese, Hansi 10.11.1875 – 04.04.1934@\textsc{Niese, Hansi} (10.11.1875 – 04.04.1934), \emph{Schauspieler/Schauspielerin}|pw} auch mit Freuden zugeſagt habe.\pend
           
\pstart
           Auf bald. Bei uns im Hauſe lag oder liegt alles; jetzt die Kinderfrau\pwindex{Loew, Anna *~11.04.1888@\textsc{Loew, Anna} (*~11.04.1888), \emph{Kinderbetreuer/Kinderbetreuerin, Dienstbote/Dienstbotin}|pwv} und Olga\pwindex{Schnitzler, Olga 17.01.1882 – 13.01.1970@\textsc{Schnitzler, Olga} (17.01.1882 – 13.01.1970), \emph{Schauspieler/Schauspielerin, Sänger/Sängerin}|pw}.\pend
           
\pstart
           Herzlichſt Ihr {\\[\baselineskip]}\spacefill\mbox{A.}\pend
           \leftskip=0em{}\selectlanguage{ngerman}\endnumbering\briefempfaengerindex{Salten, Felix@\textsc{Salten, Felix}!zzzSchnitzler, Arthur@\emph{von Arthur Schnitzler}!1905-01-121@{12. 1. 1905}|)be}\mylabel{L02996h}  \normalsize

\doendnotes{C}
\bigskip
\vfill

\clearpage

\footnotesize

\lohead{\textsc{register}}

% Definiere theindex-Environment komplett neu ohne reledmac
\makeatletter
\renewenvironment{theindex}{%
  \section*{\indexname}%
  \setlength{\parindent}{0pt}%
  \setlength{\parskip}{0pt plus 0.3pt}%
  \let\item\@idxitem
}{%
  \clearpage
}
\makeatother

\IfFileExists{\jobname-pw.ind}{\input{\jobname-pw.ind}}{}

\end{document}

      