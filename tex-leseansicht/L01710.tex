%% latex-korrekturansicht-vorspann.tex
%% Vorspann für die Korrekturansicht.
%% Lädt die gemeinsame Datei latex-vorspann.tex mit gesetztem Schalter.

\newif\ifkorrekturansicht
\korrekturansichttrue

\input{../tex-inputs/latex-vorspann}


\section[Richard Beer-Hofmann an Arthur Schnitzler, {[}27. 9. 1907{]}]{L01710 Richard Beer-Hofmann an Arthur Schnitzler, {[}27. 9. 1907{]}}
\nopagebreak\mylabel{L01710v}
\rehead{ }\normalsize\beginnumbering\briefempfaengerindex{Schnitzler, Arthur@\textsc{Schnitzler, Arthur}!zzzBeer-Hofmann, Richard@\emph{von Richard Beer-Hofmann}!1907-09-271@{{[}27. 9. 1907{]}}|(be}
\toendnotes[C]{\smallbreak\pagebreak[2]}\Standort{CUL, Schnitzler, B 8.}
\physDesc{Sonderfall, 1 Blatt, 2 Seiten, 621 Zeichen (Manuskript )
\newline{}Handschrift: Bleistift, lateinische Kurrent
\newline{}Schnitzler: mit Bleistift datiert: »Oct 907« 
\newline{}Ordnung: 1) mit Bleistift von Olga
                                    Schnitzler\pwindex{Schnitzler, Olga 17.01.1882 – 13.01.1970@\textsc{Schnitzler, Olga} (17.01.1882 – 13.01.1970), \emph{Schauspieler/Schauspielerin, Sänger/Sängerin}|pw} (?) betitelt: »Auf das Feuilleton\pwindex{Arthur Schnitzler@\emph{Arthur Schnitzler}|pwv} von Berger\pwindex{Berger, Alfred von 30.04.1853 – 24.08.1912@\textsc{Berger, Alfred von} (30.04.1853 – 24.08.1912), \emph{Schriftsteller/Schriftstellerin, Journalist/Journalistin, Theaterleiter/Theaterleiterin}|pw} über
                                    Arthur.«  2) mit Bleistift von unbekannter Hand nummeriert:
                                    »278a«}
\buchAbdrucke{\weitereDrucke{Arthur Schnitzler, Richard Beer-Hofmann: \emph{Briefwechsel 1891–1931}. Wien, Zürich: \emph{Europaverlag} 1992, S. 185.} }\toendnotes[C]{\smallbreak}\stanza{}{\pb}\label{K_L01710-1v}\edtext{Wie das Schicksal es auch füge}{\lemma{\textnormal{\emph{Wie … füge}}}\Cendnote{\textnormal{Schnitzler bekam das Gedicht am 27. 9. 1907
                              vorgelesen. Mutmaßlich entspricht das dem Tag, an dem er dieses Blatt geschenkt
                              bekam.}}}\label{K_L01710-1}, –\pwindex{Wie das Schicksal es auch fuege]@\emph{[Wie das Schicksal es auch füge]}|pwv}Alfred\pwindex{Berger, Alfred von 30.04.1853 – 24.08.1912@\textsc{Berger, Alfred von} (30.04.1853 – 24.08.1912), \emph{Schriftsteller/Schriftstellerin, Journalist/Journalistin, Theaterleiter/Theaterleiterin}|pw} kann nichts mehr
                        passieren!\pwindex{Wie das Schicksal es auch fuege]@\emph{[Wie das Schicksal es auch füge]}|pwv}Wahrheit mischt er hold mit
                        Lüge –\pwindex{Wie das Schicksal es auch fuege]@\emph{[Wie das Schicksal es auch füge]}|pwv}\label{K_L01710-2v}\edtext{Schreibt Kritik}{\lemma{\textnormal{\emph{Schreibt Kritik}}}\Cendnote{\textnormal{In seinem Feuilleton \emph{Arthur Schnitzler}\pwindex{Arthur Schnitzler@\emph{Arthur Schnitzler}|pwk} schrieb Alfred von Berger\pwindex{Berger, Alfred von 30.04.1853 – 24.08.1912@\textsc{Berger, Alfred von} (30.04.1853 – 24.08.1912), \emph{Schriftsteller/Schriftstellerin, Journalist/Journalistin, Theaterleiter/Theaterleiterin}|pwk}, Schnitzlers ganzes Werk bestehe nur
                           aus drei Dingen, Sex, Tod und (Schau-)Spiel (\emph{Neue Freie Presse}\pwindex{Neue Freie Presse@\emph{Neue Freie Presse}|pwk}, Nr. 15.467,
                              22. 9. 1907, S. 1–2).}}}\label{K_L01710-2} mit
                     Hintertüren.\pwindex{Wie das Schicksal es auch fuege]@\emph{[Wie das Schicksal es auch füge]}|pwv}\stanzaend{}\stanza{}Vorn ist’s eine
                        Ruhmespforte\pwindex{Wie das Schicksal es auch fuege]@\emph{[Wie das Schicksal es auch füge]}|pwv}Hinten wirds ein
                        Hochgericht,\pwindex{Wie das Schicksal es auch fuege]@\emph{[Wie das Schicksal es auch füge]}|pwv}Rückversichert sind die
                        Worte –\pwindex{Wie das Schicksal es auch fuege]@\emph{[Wie das Schicksal es auch füge]}|pwv}\uline{Alles} sagt er – und sagt’s nicht!\pwindex{Wie das Schicksal es auch fuege]@\emph{[Wie das Schicksal es auch füge]}|pwv}\stanzaend{}\stanza{}Wird es eine
                        Ehrenkette?\pwindex{Wie das Schicksal es auch fuege]@\emph{[Wie das Schicksal es auch füge]}|pwv}Flicht er Ihnen einen
                        Strick?\pwindex{Wie das Schicksal es auch fuege]@\emph{[Wie das Schicksal es auch füge]}|pwv}Selber weiss er’s nicht –
                        ich wette –\pwindex{Wie das Schicksal es auch fuege]@\emph{[Wie das Schicksal es auch füge]}|pwv}Dieser Janus der
                        Kritik.\pwindex{Wie das Schicksal es auch fuege]@\emph{[Wie das Schicksal es auch füge]}|pwv}\stanzaend{}\stanza{}{\pb}Doch im ganzen,
                        ungefährlich\pwindex{Wie das Schicksal es auch fuege]@\emph{[Wie das Schicksal es auch füge]}|pwv}wird die Sache – wie mir
                        scheint –\pwindex{Wie das Schicksal es auch fuege]@\emph{[Wie das Schicksal es auch füge]}|pwv}Danken Sie ihm nur \uline{so} ehrlich,\pwindex{Wie das Schicksal es auch fuege]@\emph{[Wie das Schicksal es auch füge]}|pwv}Als er’s selbst mit Ihnen
                        meint.\pwindex{Wie das Schicksal es auch fuege]@\emph{[Wie das Schicksal es auch füge]}|pwv}\stanzaend{}\stanza{}Alfreds\pwindex{Berger, Alfred von 30.04.1853 – 24.08.1912@\textsc{Berger, Alfred von} (30.04.1853 – 24.08.1912), \emph{Schriftsteller/Schriftstellerin, Journalist/Journalistin, Theaterleiter/Theaterleiterin}|pw}s Lob, und Alfreds\pwindex{Berger, Alfred von 30.04.1853 – 24.08.1912@\textsc{Berger, Alfred von} (30.04.1853 – 24.08.1912), \emph{Schriftsteller/Schriftstellerin, Journalist/Journalistin, Theaterleiter/Theaterleiterin}|pw}s Tadel\pwindex{Wie das Schicksal es auch fuege]@\emph{[Wie das Schicksal es auch füge]}|pwv}Rührt Sie ja nicht! – Gott
                        sei Dank!\pwindex{Wie das Schicksal es auch fuege]@\emph{[Wie das Schicksal es auch füge]}|pwv}– Doch – welch hoher
                        Seelenadel,\pwindex{Wie das Schicksal es auch fuege]@\emph{[Wie das Schicksal es auch füge]}|pwv}Spricht aus Alfreds\pwindex{Berger, Alfred von 30.04.1853 – 24.08.1912@\textsc{Berger, Alfred von} (30.04.1853 – 24.08.1912), \emph{Schriftsteller/Schriftstellerin, Journalist/Journalistin, Theaterleiter/Theaterleiterin}|pw}s Lotterbank!\pwindex{Wie das Schicksal es auch fuege]@\emph{[Wie das Schicksal es auch füge]}|pwv}\stanzaend{}\pstart \spacefill\mbox{R. B-H.}\pend{}\selectlanguage{ngerman}\endnumbering\briefempfaengerindex{Schnitzler, Arthur@\textsc{Schnitzler, Arthur}!zzzBeer-Hofmann, Richard@\emph{von Richard Beer-Hofmann}!1907-09-271@{{[}27. 9. 1907{]}}|)be}\mylabel{L01710h}  \normalsize

\doendnotes{C}
\bigskip
\vfill

\clearpage

\footnotesize

\lohead{\textsc{register}}

% Definiere theindex-Environment komplett neu ohne reledmac
\makeatletter
\renewenvironment{theindex}{%
  \section*{\indexname}%
  \setlength{\parindent}{0pt}%
  \setlength{\parskip}{0pt plus 0.3pt}%
  \let\item\@idxitem
}{%
  \clearpage
}
\makeatother

\IfFileExists{\jobname-pw.ind}{\input{\jobname-pw.ind}}{}

\end{document}

      