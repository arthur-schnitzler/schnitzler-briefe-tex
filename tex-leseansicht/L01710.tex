%% latex-leseansicht-vorspann.tex
%% Vorspann für die Leseansicht.
%% Lädt die gemeinsame Datei latex-vorspann.tex mit nicht gesetztem Schalter.

\newif\ifkorrekturansicht
\korrekturansichtfalse

\input{../tex-inputs/latex-vorspann}


\section[Richard Beer-Hofmann an Arthur Schnitzler, {{[}}27. 9. 1907{{]}}]{L01710 Richard Beer-Hofmann an Arthur Schnitzler, {[}27. 9. 1907{]}}
\nopagebreak\mylabel{L01710v}
\rehead{ }\normalsize\beginnumbering\briefempfaengerindex{Schnitzler, Arthur@\textsc{Schnitzler, Arthur}!zzzBeer-Hofmann, Richard@\emph{von Richard Beer-Hofmann}!1907-09-271@{{[}27. 9. 1907{]}}|(be}
\toendnotes[C]{\smallbreak\pagebreak[2]}
\correspDesc{Versand  durch Richard Beer-Hofmann am [27. 9. 1907] in Wien
\newline{}Erhalt  durch Arthur Schnitzler am [27. 9. 1907] in Wien}\toendnotes[C]{\smallbreak}
\Standort{CUL, Schnitzler, B 8.}
\physDesc{Sonderfall, 1 Blatt, 2 Seiten, 621 Zeichen (Manuskript )
\newline{}Handschrift: Bleistift, lateinische Kurrent
\newline{}Schnitzler: mit Bleistift datiert: »Oct 907« 
\newline{}Ordnung: 1) mit Bleistift von Olga
                                    Schnitzler\pwindex{Schnitzler, Olga 17.\,1.\,1882 Wien – 13.\,1.\,1970 Lugano@\textsc{Schnitzler, Olga} (17.\,1.\,1882 Wien – 13.\,1.\,1970 Lugano), \emph{Schauspielerin, Sängerin}|pw} (?) betitelt: »Auf das Feuilleton\pwindex{Berger, Alfred von 30.\,4.\,1853 Wien – 24.\,8.\,1912 ebd.@\textsc{Berger, Alfred von} (30.\,4.\,1853 Wien – 24.\,8.\,1912 ebd.), \emph{Schriftsteller, Journalist, Theaterleiter}!Arthur Schnitzler@\strich\emph{Arthur Schnitzler}|pwv} von Berger\pwindex{Berger, Alfred von 30.\,4.\,1853 Wien – 24.\,8.\,1912 ebd.@\textsc{Berger, Alfred von} (30.\,4.\,1853 Wien – 24.\,8.\,1912 ebd.), \emph{Schriftsteller, Journalist, Theaterleiter}|pw} über
                                    Arthur.«  2) mit Bleistift von unbekannter Hand nummeriert:
                                    »278a«}
\buchAbdrucke{\weitereDrucke{Arthur Schnitzler, Richard Beer-Hofmann: \emph{Briefwechsel 1891–1931}. Herausgegeben von Konstanze Fliedl. Wien, Zürich: \emph{Europaverlag} 1992, S. 185.} }\toendnotes[C]{\smallbreak}\stanza{}{\pb}\label{K_L01710-1v}\edtext{Wie das Schicksal es auch füge}{\lemma{\textnormal{\emph{Wie … füge}}}\Cendnote{\textnormal{Schnitzler bekam das Gedicht am 27. 9. 1907
                              vorgelesen. Mutmaßlich entspricht das dem Tag, an dem er dieses Blatt geschenkt
                              bekam.}}}\label{K_L01710-1}, –\pwindex{Beer-Hofmann, Richard 11.\,7.\,1866 Wien – 26.\,9.\,1945 New York City@\textsc{Beer-Hofmann, Richard} (11.\,7.\,1866 Wien – 26.\,9.\,1945 New York City), \emph{Schriftsteller}!Wie das Schicksal es auch füge]@\strich\emph{[Wie das Schicksal es auch füge]}|pwv}\newverse{}Alfred\pwindex{Berger, Alfred von 30.\,4.\,1853 Wien – 24.\,8.\,1912 ebd.@\textsc{Berger, Alfred von} (30.\,4.\,1853 Wien – 24.\,8.\,1912 ebd.), \emph{Schriftsteller, Journalist, Theaterleiter}|pw} kann nichts mehr
                        passieren!\pwindex{Beer-Hofmann, Richard 11.\,7.\,1866 Wien – 26.\,9.\,1945 New York City@\textsc{Beer-Hofmann, Richard} (11.\,7.\,1866 Wien – 26.\,9.\,1945 New York City), \emph{Schriftsteller}!Wie das Schicksal es auch füge]@\strich\emph{[Wie das Schicksal es auch füge]}|pwv}\newverse{}Wahrheit mischt er hold mit
                        Lüge –\pwindex{Beer-Hofmann, Richard 11.\,7.\,1866 Wien – 26.\,9.\,1945 New York City@\textsc{Beer-Hofmann, Richard} (11.\,7.\,1866 Wien – 26.\,9.\,1945 New York City), \emph{Schriftsteller}!Wie das Schicksal es auch füge]@\strich\emph{[Wie das Schicksal es auch füge]}|pwv}\newverse{}\label{K_L01710-2v}\edtext{Schreibt Kritik}{\lemma{\textnormal{\emph{Schreibt Kritik}}}\Cendnote{\textnormal{In seinem Feuilleton \emph{Arthur Schnitzler}\pwindex{Berger, Alfred von 30.\,4.\,1853 Wien – 24.\,8.\,1912 ebd.@\textsc{Berger, Alfred von} (30.\,4.\,1853 Wien – 24.\,8.\,1912 ebd.), \emph{Schriftsteller, Journalist, Theaterleiter}!Arthur Schnitzler@\strich\emph{Arthur Schnitzler}|pwk} schrieb Alfred von Berger\pwindex{Berger, Alfred von 30.\,4.\,1853 Wien – 24.\,8.\,1912 ebd.@\textsc{Berger, Alfred von} (30.\,4.\,1853 Wien – 24.\,8.\,1912 ebd.), \emph{Schriftsteller, Journalist, Theaterleiter}|pwk}, Schnitzlers ganzes Werk bestehe nur
                           aus drei Dingen, Sex, Tod und (Schau-)Spiel (\emph{Neue Freie Presse}\pwindex{Neue Freie Presse@\emph{Neue Freie Presse}|pwk}, Nr. 15.467,
                              22. 9. 1907, S. 1–2).}}}\label{K_L01710-2} mit
                     Hintertüren.\pwindex{Beer-Hofmann, Richard 11.\,7.\,1866 Wien – 26.\,9.\,1945 New York City@\textsc{Beer-Hofmann, Richard} (11.\,7.\,1866 Wien – 26.\,9.\,1945 New York City), \emph{Schriftsteller}!Wie das Schicksal es auch füge]@\strich\emph{[Wie das Schicksal es auch füge]}|pwv}\stanzaend{}\stanza{}Vorn ist’s eine
                        Ruhmespforte\pwindex{Beer-Hofmann, Richard 11.\,7.\,1866 Wien – 26.\,9.\,1945 New York City@\textsc{Beer-Hofmann, Richard} (11.\,7.\,1866 Wien – 26.\,9.\,1945 New York City), \emph{Schriftsteller}!Wie das Schicksal es auch füge]@\strich\emph{[Wie das Schicksal es auch füge]}|pwv}\newverse{}Hinten wirds ein
                        Hochgericht,\pwindex{Beer-Hofmann, Richard 11.\,7.\,1866 Wien – 26.\,9.\,1945 New York City@\textsc{Beer-Hofmann, Richard} (11.\,7.\,1866 Wien – 26.\,9.\,1945 New York City), \emph{Schriftsteller}!Wie das Schicksal es auch füge]@\strich\emph{[Wie das Schicksal es auch füge]}|pwv}\newverse{}Rückversichert sind die
                        Worte –\pwindex{Beer-Hofmann, Richard 11.\,7.\,1866 Wien – 26.\,9.\,1945 New York City@\textsc{Beer-Hofmann, Richard} (11.\,7.\,1866 Wien – 26.\,9.\,1945 New York City), \emph{Schriftsteller}!Wie das Schicksal es auch füge]@\strich\emph{[Wie das Schicksal es auch füge]}|pwv}\newverse{}\uline{Alles} sagt er – und sagt’s nicht!\pwindex{Beer-Hofmann, Richard 11.\,7.\,1866 Wien – 26.\,9.\,1945 New York City@\textsc{Beer-Hofmann, Richard} (11.\,7.\,1866 Wien – 26.\,9.\,1945 New York City), \emph{Schriftsteller}!Wie das Schicksal es auch füge]@\strich\emph{[Wie das Schicksal es auch füge]}|pwv}\stanzaend{}\stanza{}Wird es eine
                        Ehrenkette?\pwindex{Beer-Hofmann, Richard 11.\,7.\,1866 Wien – 26.\,9.\,1945 New York City@\textsc{Beer-Hofmann, Richard} (11.\,7.\,1866 Wien – 26.\,9.\,1945 New York City), \emph{Schriftsteller}!Wie das Schicksal es auch füge]@\strich\emph{[Wie das Schicksal es auch füge]}|pwv}\newverse{}Flicht er Ihnen einen
                        Strick?\pwindex{Beer-Hofmann, Richard 11.\,7.\,1866 Wien – 26.\,9.\,1945 New York City@\textsc{Beer-Hofmann, Richard} (11.\,7.\,1866 Wien – 26.\,9.\,1945 New York City), \emph{Schriftsteller}!Wie das Schicksal es auch füge]@\strich\emph{[Wie das Schicksal es auch füge]}|pwv}\newverse{}Selber weiss er’s nicht –
                        ich wette –\pwindex{Beer-Hofmann, Richard 11.\,7.\,1866 Wien – 26.\,9.\,1945 New York City@\textsc{Beer-Hofmann, Richard} (11.\,7.\,1866 Wien – 26.\,9.\,1945 New York City), \emph{Schriftsteller}!Wie das Schicksal es auch füge]@\strich\emph{[Wie das Schicksal es auch füge]}|pwv}\newverse{}Dieser Janus der
                        Kritik.\pwindex{Beer-Hofmann, Richard 11.\,7.\,1866 Wien – 26.\,9.\,1945 New York City@\textsc{Beer-Hofmann, Richard} (11.\,7.\,1866 Wien – 26.\,9.\,1945 New York City), \emph{Schriftsteller}!Wie das Schicksal es auch füge]@\strich\emph{[Wie das Schicksal es auch füge]}|pwv}\stanzaend{}\stanza{}{\pb}Doch im ganzen,
                        ungefährlich\pwindex{Beer-Hofmann, Richard 11.\,7.\,1866 Wien – 26.\,9.\,1945 New York City@\textsc{Beer-Hofmann, Richard} (11.\,7.\,1866 Wien – 26.\,9.\,1945 New York City), \emph{Schriftsteller}!Wie das Schicksal es auch füge]@\strich\emph{[Wie das Schicksal es auch füge]}|pwv}\newverse{}wird die Sache – wie mir
                        scheint –\pwindex{Beer-Hofmann, Richard 11.\,7.\,1866 Wien – 26.\,9.\,1945 New York City@\textsc{Beer-Hofmann, Richard} (11.\,7.\,1866 Wien – 26.\,9.\,1945 New York City), \emph{Schriftsteller}!Wie das Schicksal es auch füge]@\strich\emph{[Wie das Schicksal es auch füge]}|pwv}\newverse{}Danken Sie ihm nur \uline{so} ehrlich,\pwindex{Beer-Hofmann, Richard 11.\,7.\,1866 Wien – 26.\,9.\,1945 New York City@\textsc{Beer-Hofmann, Richard} (11.\,7.\,1866 Wien – 26.\,9.\,1945 New York City), \emph{Schriftsteller}!Wie das Schicksal es auch füge]@\strich\emph{[Wie das Schicksal es auch füge]}|pwv}\newverse{}Als er’s selbst mit Ihnen
                        meint.\pwindex{Beer-Hofmann, Richard 11.\,7.\,1866 Wien – 26.\,9.\,1945 New York City@\textsc{Beer-Hofmann, Richard} (11.\,7.\,1866 Wien – 26.\,9.\,1945 New York City), \emph{Schriftsteller}!Wie das Schicksal es auch füge]@\strich\emph{[Wie das Schicksal es auch füge]}|pwv}\stanzaend{}\stanza{}Alfreds\pwindex{Berger, Alfred von 30.\,4.\,1853 Wien – 24.\,8.\,1912 ebd.@\textsc{Berger, Alfred von} (30.\,4.\,1853 Wien – 24.\,8.\,1912 ebd.), \emph{Schriftsteller, Journalist, Theaterleiter}|pw}s Lob, und Alfreds\pwindex{Berger, Alfred von 30.\,4.\,1853 Wien – 24.\,8.\,1912 ebd.@\textsc{Berger, Alfred von} (30.\,4.\,1853 Wien – 24.\,8.\,1912 ebd.), \emph{Schriftsteller, Journalist, Theaterleiter}|pw}s Tadel\pwindex{Beer-Hofmann, Richard 11.\,7.\,1866 Wien – 26.\,9.\,1945 New York City@\textsc{Beer-Hofmann, Richard} (11.\,7.\,1866 Wien – 26.\,9.\,1945 New York City), \emph{Schriftsteller}!Wie das Schicksal es auch füge]@\strich\emph{[Wie das Schicksal es auch füge]}|pwv}\newverse{}Rührt Sie ja nicht! – Gott
                        sei Dank!\pwindex{Beer-Hofmann, Richard 11.\,7.\,1866 Wien – 26.\,9.\,1945 New York City@\textsc{Beer-Hofmann, Richard} (11.\,7.\,1866 Wien – 26.\,9.\,1945 New York City), \emph{Schriftsteller}!Wie das Schicksal es auch füge]@\strich\emph{[Wie das Schicksal es auch füge]}|pwv}\newverse{}– Doch – welch hoher
                        Seelenadel,\pwindex{Beer-Hofmann, Richard 11.\,7.\,1866 Wien – 26.\,9.\,1945 New York City@\textsc{Beer-Hofmann, Richard} (11.\,7.\,1866 Wien – 26.\,9.\,1945 New York City), \emph{Schriftsteller}!Wie das Schicksal es auch füge]@\strich\emph{[Wie das Schicksal es auch füge]}|pwv}\newverse{}Spricht aus Alfreds\pwindex{Berger, Alfred von 30.\,4.\,1853 Wien – 24.\,8.\,1912 ebd.@\textsc{Berger, Alfred von} (30.\,4.\,1853 Wien – 24.\,8.\,1912 ebd.), \emph{Schriftsteller, Journalist, Theaterleiter}|pw}s Lotterbank!\pwindex{Beer-Hofmann, Richard 11.\,7.\,1866 Wien – 26.\,9.\,1945 New York City@\textsc{Beer-Hofmann, Richard} (11.\,7.\,1866 Wien – 26.\,9.\,1945 New York City), \emph{Schriftsteller}!Wie das Schicksal es auch füge]@\strich\emph{[Wie das Schicksal es auch füge]}|pwv}\stanzaend{}\pstart \spacefill\mbox{R. B-H.}\pend{}\selectlanguage{ngerman}\endnumbering\briefempfaengerindex{Schnitzler, Arthur@\textsc{Schnitzler, Arthur}!zzzBeer-Hofmann, Richard@\emph{von Richard Beer-Hofmann}!1907-09-271@{{[}27. 9. 1907{]}}|)be}\mylabel{L01710h}  \newcommand{\dateiname}{L01710}\newcommand{\titel}{Richard Beer-Hofmann an Arthur Schnitzler, [27. 9. 1907]}\newcommand{\editorInnen}{Martin Anton Müller und Gerd-Hermann Susen}%% latex-leseansicht-abspann.tex
%% Abspann für die Leseansicht.
%% Der Schalter \ifkorrekturansicht ist bereits durch den Vorspann gesetzt.

%% latex-abspann.tex
%% Gemeinsamer Abspann für Korrekturansicht und Leseansicht.
%% Setzt den Schalter \ifkorrekturansicht voraus (gesetzt in den
%% einbindenden Dateien latex-korrekturansicht-abspann.tex bzw.
%% latex-leseansicht-abspann.tex).
%% ---------------------------------------------------------------

\normalsize

% Das esempio-Environment wird nur in der Leseansicht benötigt
\ifkorrekturansicht\else
\newenvironment{esempio}[3]%
{
    \vspace{1.5ex}
    \rlap{\underline{#1}}
    \par
    \setlength{\parindent}{0cm}
    \nopagebreak
    \leftskip=#2cm
    \rightskip=#3cm
}
{
    \par
}
\fi

\doendnotes{C}
\bigskip
\vfill

\clearpage

\footnotesize

\ifkorrekturansicht
  \lohead{\textsc{register}}
\fi

% theindex-Environment neu definieren ohne reledmac
\makeatletter
\renewenvironment{theindex}{%
  \ifkorrekturansicht
    \section*{\indexname}%
  \else
    \subsubsection*{Index der erwähnten Entitäten}%
  \fi
  \setlength{\parindent}{0pt}%
  \setlength{\parskip}{0pt plus 0.3pt}%
  \let\item\@idxitem
}{%
  \ifkorrekturansicht\clearpage\fi
}
\makeatother

\IfFileExists{\jobname-pw.ind}{\input{\jobname-pw.ind}}{}

% Quellenangabe nur in der Leseansicht
\ifkorrekturansicht\else
% Fallback-Definitionen, falls die .tex-Datei \titel etc. nicht gesetzt hat
\providecommand{\titel}{}
\providecommand{\editorInnen}{}
\providecommand{\dateiname}{\jobname}

\vspace{3cm}

\vfill

\footnotesize
\textsc{Quelle}: \titel. Herausgegeben von {\editorInnen}. In: \emph{Arthur Schnitzler: Briefwechsel mit Autorinnen und Autoren}.
 Digitale Edition, https://schnitzler-briefe.acdh.oeaw.ac.at/{\dateiname}.html (Stand \today)
\fi

\end{document}


