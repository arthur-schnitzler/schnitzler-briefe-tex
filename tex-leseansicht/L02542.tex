\input{../tex-inputs/latex-pdf-vorspann}
\begin{center}
            \textcolor{red}{ENTWURF. ENTZIFFERUNG NOCH NICHT KORREKTURGELESEN}
                      \end{center}
            
               \section[Gerty von Hofmannsthal an Arthur Schnitzler, 16. 2. 1931]{ Gerty von Hofmannsthal an Arthur Schnitzler, 16. 2. 1931}\nopagebreak\mylabel{v}\rehead{ }\begin{ledgroupsized}[t]{13cm}\normalsize\beginnumbering\briefempfaengerindex{Schnitzler, Arthur@\textsc{Schnitzler, Arthur}!zzzHofmannsthal, Gertrude von@\emph{von Gertrude von Hofmannsthal}!1931-02-161@{16. 2. 1931}|(be} \toendnotes[C]{\smallbreak\pagebreak[2]} \Standort{CUL, Schnitzler, B 43.}
\physDesc{Brief, 1 Blatt, 2 Seiten
\newline{}Handschrift: blaue Tinte, lateinische Kurrent
\newline{}Schnitzler: mit rotem Buntstift beschriftet »\textsc{Hugo}« und mehrere Unterstreichungen }\toendnotes[C]{\smallbreak}\pstart
           \raggedleft{}{\pb}Mozartg. 4\oindex{Mozartgasse@\textbf{Mozartgasse}|pw}{ }16/II 31\pend
           \pstart
           Lieber Arthur, wie sehr freute ich mich über den grossen starken
                  \label{K_L02542_1v}\edtext{Erfolg Ihres Stückes\pwindex{Schnitzler, Arthur 15.05.1862 – 21.10.1931@\textsc{Schnitzler, Arthur} (15.05.1862 – 21.10.1931), \emph{Schriftsteller, Mediziner}!Gang zum Weiher. Dramatische Dichtung1926@\strich\emph{Der Gang zum Weiher. Dramatische Dichtung} {[}1926{]}|pwv}}{\lemma{\textnormal{\emph{Erfolg Ihres Stückes}}}\Cendnote{\textnormal{Die Uraufführung von \emph{Der Gang zum Weiher}\pwindex{Schnitzler, Arthur 15.05.1862 – 21.10.1931@\textsc{Schnitzler, Arthur} (15.05.1862 – 21.10.1931), \emph{Schriftsteller, Mediziner}!Gang zum Weiher. Dramatische Dichtung1926@\strich\emph{Der Gang zum Weiher. Dramatische Dichtung} {[}1926{]}|pwk} fand am 14. 2. 1931 im Burgtheater\oindex{Burgtheater@\textbf{Burgtheater}|pwk} statt.}}}\label{K_L02542_1h} den ich in sämmtlichen Zeitungen verfolgte –
               wie schön und entspannend für Sie! Ich weiss ja wie aufregend diese letzten Tage vor
               einer Erstaufführung sind und wollte Sie daher auch gar nicht stören, Ihnen zu sagen,
               dass ich wieder in Wien\oindex{Wien@\textbf{Wien}|pw} bin, dass ich seit
                  20 September verreist war, in Heidelberg\oindex{Heidelberg@\textbf{Heidelberg}|pw}, Basel\oindex{Basel@\textbf{Basel}|pw}, Zürich\oindex{Zuerich@\textbf{Zürich}|pw} und München\oindex{Muenchen@\textbf{München}|pw}! Ich war
               eigentlich nur eine kurze Zwischenzeit in Wien\oindex{Wien@\textbf{Wien}|pw} vom
               späten Herbst bis gegen Weihnachten! Damals nahm ich mir fest vor, Ihnen von Berlin\oindex{Berlin@\textbf{Berlin}|pw} zu berichten (wo ich viel mit Olga\pwindex{Schnitzler, Olga 17.01.1882 – 13.01.1970@\textsc{Schnitzler, Olga} (17.01.1882 – 13.01.1970), \emph{Schauspielerin, Sängerin}|pw} war und Heini\pwindex{Schnitzler, Heinrich 09.08.1902 – 12.07.1982@\textsc{Schnitzler, Heinrich} (09.08.1902 – 12.07.1982), \emph{Regisseur, Schauspieler}|pw} knapp vor seiner \label{K_L02542_2v}\edtext{Heirat}{\lemma{\textnormal{\emph{Heirat}}}\Cendnote{\textnormal{Heinrich Schnitzler\pwindex{Schnitzler, Heinrich 09.08.1902 – 12.07.1982@\textsc{Schnitzler, Heinrich} (09.08.1902 – 12.07.1982), \emph{Regisseur, Schauspieler}|pwk} und Ruth Albu\pwindex{Schnitzler, Ruth 04.04.1908 – 27.02.2000@\textsc{Schnitzler, Ruth} (04.04.1908 – 27.02.2000), \emph{Schauspielerin}|pwk} heirateten am 29. 10. 1930.}}}\label{K_L02542_2h} wiedersah) aber i{\geminationm}er {\pb}fehlte
               mir die Courage Sie anzurufen da ich Ihre Arbeitsstunden nicht wusste!\pend
           \pstart
           Ich hoffe \substVorne{}\textsuperscript{in}\substDazwischen{}für eine\substHinten{} de\substVorne{}\textsuperscript{n}\substDazwischen{}r\substHinten{} nächsten Aufführungen von Buschbeck\pwindex{Buschbeck, Erhard 06.01.1889 – 02.09.1960@\textsc{Buschbeck, Erhard} (06.01.1889 – 02.09.1960), \emph{Schriftsteller/Schriftstellerin, Dramaturg/Dramaturgin}|pw}
               einmal zwei Plätze verlangen zu können und freue mich sehr darauf!\pend
           \pstart
           Wenn Sie mir einmal vorschlagen wollen wann ich Sie besuchen darf, tue ich es mit
               grosser Freude nur bitte sagen Sie es mir ein bissl früher, damit ich mich freihalte
               – ich verstehe aber auch so gut wenn Sie jetzt \uline{Ruhe}
               haben wollen!\pend
           \pstart
           Alles Liebe und nochmals herzl. Glückwünsche zur gelungenen Aufführung\pwindex{Schnitzler, Arthur 15.05.1862 – 21.10.1931@\textsc{Schnitzler, Arthur} (15.05.1862 – 21.10.1931), \emph{Schriftsteller, Mediziner}!Gang zum Weiher. Dramatische Dichtung1926@\strich\emph{Der Gang zum Weiher. Dramatische Dichtung} {[}1926{]}|pwv}{\\[\baselineskip]}Ihre{\\[\baselineskip]}\spacefill\mbox{Gerty}\pend
           \leftskip=0em{}\endnumbering\briefempfaengerindex{Schnitzler, Arthur@\textsc{Schnitzler, Arthur}!zzzHofmannsthal, Gertrude von@\emph{von Gertrude von Hofmannsthal}!1931-02-161@{16. 2. 1931}|)be}\mylabel{h}\end{ledgroupsized}  \newcommand{\dateiname}{L02542}\newcommand{\titel}{Gerty von Hofmannsthal an Arthur Schnitzler, 16. 2. 1931}\newcommand{\editorInnen}{Martin Anton Müller und Gerd-Hermann Susen}\input{../tex-inputs/latex-pdf-abspann}
      