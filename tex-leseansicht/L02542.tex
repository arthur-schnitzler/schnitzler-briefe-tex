%% latex-leseansicht-vorspann.tex
%% Vorspann für die Leseansicht.
%% Lädt die gemeinsame Datei latex-vorspann.tex mit nicht gesetztem Schalter.

\newif\ifkorrekturansicht
\korrekturansichtfalse

\input{../tex-inputs/latex-vorspann}


\section[Gerty Hofmannsthal an Arthur Schnitzler, 16. 2. 1931]{L02542 Gerty Hofmannsthal an Arthur Schnitzler, 16. 2. 1931}
\nopagebreak\mylabel{L02542v}
\rehead{ }\normalsize\beginnumbering\briefempfaengerindex{Schnitzler, Arthur@\textsc{Schnitzler, Arthur}!zzzHofmannsthal, Gertrude von@\emph{von Gertrude von Hofmannsthal}!1931-02-161@{16. 2. 1931}|(be}
\toendnotes[C]{\smallbreak\pagebreak[2]}
\correspDesc{Versand  durch Gerty Hofmannsthal am 16. 2. 1931 in Wien
\newline{}Erhalt  durch Arthur Schnitzler im Zeitraum [16. 2. 1931
                  – 20. 2. 1931?] in Wien}\toendnotes[C]{\smallbreak}
\Standort{CUL, Schnitzler, B 43.}
\physDesc{Brief, 1 Blatt, 2 Seiten, 1154 Zeichen
\newline{}Handschrift: blaue Tinte, lateinische Kurrent
\newline{}Schnitzler: mit rotem Buntstift beschriftet »\textsc{Hugo}« und mehrere Unterstreichungen }\toendnotes[C]{\smallbreak}
\pstart
           \raggedleft{}{\pb}Mozartg. 4\oindex{Wien@\textbf{Wien}!IV., Wieden@\textbf{IV., Wieden}!Mozartgasse@\textbf{Mozartgasse}, \emph{Straße}|pw}{ }16/II 31\pend
           \vspace{0.5em}
\pstart
           Lieber Arthur, wie sehr freute ich mich über den grossen starken
                  \label{K_L02542-1v}\edtext{Erfolg Ihres Stückes\pwindex{Schnitzler, Arthur 15.\,5.\,1862 Wien – 21.\,10.\,1931 ebd.@\textsc{Schnitzler, Arthur} (15.\,5.\,1862 Wien – 21.\,10.\,1931 ebd.), \emph{Schriftsteller, Mediziner}!Gang zum Weiher. Dramatische Dichtung@\strich\emph{Der Gang zum Weiher. Dramatische Dichtung}|pwv}}{\lemma{\textnormal{\emph{Erfolg Ihres Stückes}}}\Cendnote{\textnormal{Die Uraufführung\eventindex{Burgtheater@\textbf{Burgtheater}!Uraufführung von Der Gang zum Weiher, 14.2.1931@Uraufführung von Der Gang zum Weiher, 14.2.1931|pwkv} von \emph{Der Gang zum Weiher}\pwindex{Schnitzler, Arthur 15.\,5.\,1862 Wien – 21.\,10.\,1931 ebd.@\textsc{Schnitzler, Arthur} (15.\,5.\,1862 Wien – 21.\,10.\,1931 ebd.), \emph{Schriftsteller, Mediziner}!Gang zum Weiher. Dramatische Dichtung@\strich\emph{Der Gang zum Weiher. Dramatische Dichtung}|pwk} fand am 14. 2. 1931 im \emph{Burgtheater}\orgindex{Burgtheater@Burgtheater|pwk} statt.}}}\label{K_L02542-1} den ich in sämmtlichen Zeitungen
               verfolgte – wie schön und entspannend für Sie! Ich weiss ja wie aufregend diese
               letzten Tage vor einer Erstaufführung sind und wollte Sie daher auch gar nicht
               stören, Ihnen zu sagen, dass ich wieder in Wien\oindex{Wien@\textbf{Wien}, \emph{Verwaltungsgebiet}|pw}
               bin, dass ich seit 20 September verreist war, in Heidelberg\oindex{Heidelberg@\textbf{Heidelberg}, \emph{Hauptstadt}|pw}, Basel\oindex{Basel@\textbf{Basel}|pw}, Zürich\oindex{Zürich@\textbf{Zürich}|pw} und München\oindex{München@\textbf{München}|pw}! Ich war eigentlich nur eine kurze Zwischenzeit in Wien\oindex{Wien@\textbf{Wien}, \emph{Verwaltungsgebiet}|pw} vom späten Herbst bis gegen Weihnachten! Damals nahm ich
               mir fest vor, Ihnen von Berlin\oindex{Berlin@\textbf{Berlin}, \emph{Hauptstadt}|pw} zu berichten (wo
               ich viel mit Olga\pwindex{Schnitzler, Olga 17.\,1.\,1882 Wien – 13.\,1.\,1970 Lugano@\textsc{Schnitzler, Olga} (17.\,1.\,1882 Wien – 13.\,1.\,1970 Lugano), \emph{Schauspielerin, Sängerin}|pw} war und Heini\pwindex{Schnitzler, Heinrich 9.\,8.\,1902 Hinterbrühl – 12.\,7.\,1982 Wien@\textsc{Schnitzler, Heinrich} (9.\,8.\,1902 Hinterbrühl – 12.\,7.\,1982 Wien), \emph{Regisseur, Schauspieler}|pw} knapp vor seiner \label{K_L02542-2v}\edtext{Heirat}{\lemma{\textnormal{\emph{Heirat}}}\Cendnote{\textnormal{Heinrich Schnitzler\pwindex{Schnitzler, Heinrich 9.\,8.\,1902 Hinterbrühl – 12.\,7.\,1982 Wien@\textsc{Schnitzler, Heinrich} (9.\,8.\,1902 Hinterbrühl – 12.\,7.\,1982 Wien), \emph{Regisseur, Schauspieler}|pwk} und Ruth Albu\pwindex{Schnitzler, Ruth 4.\,4.\,1908 Berlin – 27.\,2.\,2000 Santa Barbara@\textsc{Schnitzler, Ruth} (4.\,4.\,1908 Berlin – 27.\,2.\,2000 Santa Barbara), \emph{Schauspielerin}|pwk} hatten am 29. 10. 1930 geheiratet.}}}\label{K_L02542-2} wiedersah) aber i{\geminationm}er {\pb}fehlte
               mir die Courage Sie anzurufen da ich Ihre Arbeitsstunden nicht wusste!\pend
           
\pstart
           Ich hoffe \substVorne{}\textsuperscript{in}\substDazwischen{}für eine\substHinten{} de\substVorne{}\textsuperscript{n}\substDazwischen{}r\substHinten{} nächsten Aufführungen von Buschbeck\pwindex{Buschbeck, Erhard 6.\,1.\,1889 Salzburg – 2.\,9.\,1960 Wien@\textsc{Buschbeck, Erhard} (6.\,1.\,1889 Salzburg – 2.\,9.\,1960 Wien), \emph{Schriftsteller, Dramaturg}|pw}
               einmal zwei Plätze verlangen zu können und freue mich sehr darauf!\pend
           
\pstart
           Wenn Sie mir einmal vorschlagen wollen wann ich Sie besuchen darf, tue ich es mit
               grosser Freude nur bitte sagen Sie es mir ein bissl früher, damit ich mich freihalte
               – ich verstehe aber auch so gut wenn Sie jetzt \uline{Ruhe}
               haben wollen!\pend
           
\pstart
           Alles Liebe und nochmals herzl. Glückwünsche zur gelungenen Aufführung\pwindex{Schnitzler, Arthur 15.\,5.\,1862 Wien – 21.\,10.\,1931 ebd.@\textsc{Schnitzler, Arthur} (15.\,5.\,1862 Wien – 21.\,10.\,1931 ebd.), \emph{Schriftsteller, Mediziner}!Gang zum Weiher. Dramatische Dichtung@\strich\emph{Der Gang zum Weiher. Dramatische Dichtung}|pwv}{\\[\baselineskip]}Ihre{\\[\baselineskip]}\spacefill\mbox{Gerty}\pend
           \leftskip=0em{}\selectlanguage{ngerman}\endnumbering\briefempfaengerindex{Schnitzler, Arthur@\textsc{Schnitzler, Arthur}!zzzHofmannsthal, Gertrude von@\emph{von Gertrude von Hofmannsthal}!1931-02-161@{16. 2. 1931}|)be}\mylabel{L02542h}  \newcommand{\dateiname}{L02542}\newcommand{\titel}{Gerty Hofmannsthal an Arthur Schnitzler, 16. 2. 1931}\newcommand{\editorInnen}{Martin Anton Müller und Gerd-Hermann Susen}%% latex-leseansicht-abspann.tex
%% Abspann für die Leseansicht.
%% Der Schalter \ifkorrekturansicht ist bereits durch den Vorspann gesetzt.

%% latex-abspann.tex
%% Gemeinsamer Abspann für Korrekturansicht und Leseansicht.
%% Setzt den Schalter \ifkorrekturansicht voraus (gesetzt in den
%% einbindenden Dateien latex-korrekturansicht-abspann.tex bzw.
%% latex-leseansicht-abspann.tex).
%% ---------------------------------------------------------------

\normalsize

% Das esempio-Environment wird nur in der Leseansicht benötigt
\ifkorrekturansicht\else
\newenvironment{esempio}[3]%
{
    \vspace{1.5ex}
    \rlap{\underline{#1}}
    \par
    \setlength{\parindent}{0cm}
    \nopagebreak
    \leftskip=#2cm
    \rightskip=#3cm
}
{
    \par
}
\fi

\doendnotes{C}
\bigskip
\vfill

\clearpage

\footnotesize

\ifkorrekturansicht
  \lohead{\textsc{register}}
\fi

% theindex-Environment neu definieren ohne reledmac
\makeatletter
\renewenvironment{theindex}{%
  \ifkorrekturansicht
    \section*{\indexname}%
  \else
    \subsubsection*{Index der erwähnten Entitäten}%
  \fi
  \setlength{\parindent}{0pt}%
  \setlength{\parskip}{0pt plus 0.3pt}%
  \let\item\@idxitem
}{%
  \ifkorrekturansicht\clearpage\fi
}
\makeatother

\IfFileExists{\jobname-pw.ind}{\input{\jobname-pw.ind}}{}

% Quellenangabe nur in der Leseansicht
\ifkorrekturansicht\else
% Fallback-Definitionen, falls die .tex-Datei \titel etc. nicht gesetzt hat
\providecommand{\titel}{}
\providecommand{\editorInnen}{}
\providecommand{\dateiname}{\jobname}

\vspace{3cm}

\vfill

\footnotesize
\textsc{Quelle}: \titel. Herausgegeben von {\editorInnen}. In: \emph{Arthur Schnitzler: Briefwechsel mit Autorinnen und Autoren}.
 Digitale Edition, https://schnitzler-briefe.acdh.oeaw.ac.at/{\dateiname}.html (Stand \today)
\fi

\end{document}


