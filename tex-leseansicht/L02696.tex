%% latex-leseansicht-vorspann.tex
%% Vorspann für die Leseansicht.
%% Lädt die gemeinsame Datei latex-vorspann.tex mit nicht gesetztem Schalter.

\newif\ifkorrekturansicht
\korrekturansichtfalse

\input{../tex-inputs/latex-vorspann}


\section[Paul Goldmann an Arthur Schnitzler, 10. 8. 1892]{L02696 Paul Goldmann an Arthur Schnitzler, 10. 8. 1892}
\nopagebreak\mylabel{L02696v}
\rehead{ }\normalsize\beginnumbering\briefempfaengerindex{Schnitzler, Arthur@\textsc{Schnitzler, Arthur}!zzzGoldmann, Paul@\emph{von Paul Goldmann}!1892-08-101@{10. 8. 1892}|(be}
\toendnotes[C]{\smallbreak\pagebreak[2]}
\correspDesc{Versand  durch Paul Goldmann am 10. 8. 1892 in San Sebastian
\newline{}Erhalt  durch Arthur Schnitzler am 13. 8. 1892 in Wien}\toendnotes[C]{\smallbreak}
\Standort{DLA, A:Schnitzler, HS.NZ85.1.3163.}
\physDesc{Postkarte, 692 Zeichen
\newline{}Handschrift: schwarze Tinte, lateinische Kurrent
\newline{}Versand: 1) Stempel: »\nobreak{}10 Ago. 92, Amb. Des\textcolor{gray}{c}.\nobreak{}«.   2) Stempel: »\nobreak{}\oindex{Wien@\textbf{Wien}, \emph{Verwaltungsgebiet}|pwk}Wien 1/1, 13{[}.{]} 8. 92, 9–10½ V., Bestellt\nobreak{}«. 
\newline{}Schnitzler: mit Bleistift das Datum »10/8/92« vermerkt sowie die Jahresangabe »92« der Datumsangabe ergänzt }\toendnotes[C]{\smallbreak}\pstart{}{\pb}Autriche\oindex{Österreich@\textbf{Österreich}|pw}! \pend{}\pstart{}\textcolor{gray}{\textbf{\begin{otherlanguage}{french}A\end{otherlanguage}}} Herrn Dr. Arthur Schnitzler\pend{}\pstart{}I. Giselastraſse 11\oindex{Wien@\textbf{Wien}!I., Innere Stadt@\textbf{I., Innere Stadt}!Ordination Arthur Schnitzler [Bösendorferstraße 11]@\textbf{Ordination Arthur Schnitzler [Bösendorferstraße 11]}, \emph{Ordination}|pw}\pend{}\pstart{}Wien\oindex{Wien@\textbf{Wien}, \emph{Verwaltungsgebiet}|pw}. \pend{}{\bigskip}\vspace{1em}
\pstart
           \raggedleft{}{\pb}San Sebastian\oindex{San Sebastian@\textbf{San Sebastian}, \emph{Hauptstadt}|pw}, 10 août\pend
           \vspace{0.5em}
\pstart
           \label{K_L02696-1v}\edtext{Me voilà donc en Espagne\oindex{Spanien@\textbf{Spanien}|pw}, mon bien cher ami. J’ai passé trois jours dans ce
               petit paradis\oindex{San Sebastian@\textbf{San Sebastian}, \emph{Hauptstadt}|pwv} au golfe de Biscaya\oindex{Biskaya@\textbf{Biskaya}|pw}. J’ai vu des choses on ne peut
               plus espagn\oindex{Spanien@\textbf{Spanien}|pwv}oles. J’ai assisté
               aux grandes courses de taureaux, j’ai regardé la reine\pwindex{Maria Christina von Österreich 21.\,7.\,1858 Židlochovice – 6.\,2.\,1929 Madrid@\textsc{Maria Christina von Österreich} (21.\,7.\,1858 Židlochovice – 6.\,2.\,1929 Madrid), \emph{Königin}|pwv} prendre son bain et le petit roi\pwindex{Alfons XIII. 17.\,5.\,1886 Madrid – 18.\,2.\,1941 Rom@\textsc{Alfons XIII.} (17.\,5.\,1886 Madrid – 18.\,2.\,1941 Rom), \emph{König}|pwv} jouant dans le \strikeout{sabl} sable, j’ai fumé des cigares de Havanah\oindex{Havanna@\textbf{Havanna}, \emph{Hauptstadt}|pw} et j’ai bu du vin d’Andalousie\oindex{Andalusia@\textbf{Andalusia}, \emph{Land}|pw}. Mais je t’assure, que, le premièr moment de curiosité passé, mon
               cœur était rongé de soucis et d’inquiétude nerveuse comme avant. Peut-être que tant
               cela sara beau dans le souvenir, mais dans la présesence ça ne c’est point.
               Meilleures amitiés. Bien à toi.}{\lemma{\textnormal{\emph{Me … toi.}}}\Cendnote{\textnormal{französisch: »Nun also in Spanien\oindex{Spanien@\textbf{Spanien}|pwk}, mein
                  lieber Freund: Ich habe drei Tage in diesem kleinen Paradies\oindex{San Sebastian@\textbf{San Sebastian}, \emph{Hauptstadt}|pwkv} am Golf von Biscaya\oindex{Biskaya@\textbf{Biskaya}|pwk} verbracht. Ich habe Dinge gesehen, die
                     spanischer\oindex{Spanien@\textbf{Spanien}|pwkv} nicht sein
                  könnten. Ich war bei den großen Stierrennen dabei, habe der Königin\pwindex{Maria Christina von Österreich 21.\,7.\,1858 Židlochovice – 6.\,2.\,1929 Madrid@\textsc{Maria Christina von Österreich} (21.\,7.\,1858 Židlochovice – 6.\,2.\,1929 Madrid), \emph{Königin}|pwkv} beim Baden und dem kleinen König\pwindex{Alfons XIII. 17.\,5.\,1886 Madrid – 18.\,2.\,1941 Rom@\textsc{Alfons XIII.} (17.\,5.\,1886 Madrid – 18.\,2.\,1941 Rom), \emph{König}|pwkv} beim
                  Sandspielen zugesehen, ich habe Havanna\oindex{Havanna@\textbf{Havanna}, \emph{Hauptstadt}|pwk}-Zigarren geraucht und Wein aus Andalusien\oindex{Andalusia@\textbf{Andalusia}, \emph{Land}|pwk} getrunken. Aber sei versichert, dass mein Herz, nachdem der
                  erste Eindruck der Neugierde vorüber war, von Sorgen und nervöser Unruhe
                  zerfressen war wie zuvor. Vielleicht ist es in der Erinnerung schön, aber in der
                  Gegenwart ist es das nicht. Mit besten Grüßen. Alles Gute für dich.«}}}\label{K_L02696-1}\pend
           \pstart \label{T_L02696-1v}\edtext{Ton \spacefill\mbox{Paul Goldmann.}}{\lemma{\textnormal{\emph{Ton Paul Goldmann.}}}\Cendnote{\textnormal{seitlich am rechten Rand}}}\label{T_L02696-1}\pend{}\selectlanguage{ngerman}\endnumbering\briefempfaengerindex{Schnitzler, Arthur@\textsc{Schnitzler, Arthur}!zzzGoldmann, Paul@\emph{von Paul Goldmann}!1892-08-101@{10. 8. 1892}|)be}\mylabel{L02696h}  \newcommand{\dateiname}{L02696}\newcommand{\titel}{Paul Goldmann an Arthur Schnitzler, 10. 8. 1892}\newcommand{\editorInnen}{Martin Anton Müller und Laura Untner}%% latex-leseansicht-abspann.tex
%% Abspann für die Leseansicht.
%% Der Schalter \ifkorrekturansicht ist bereits durch den Vorspann gesetzt.

%% latex-abspann.tex
%% Gemeinsamer Abspann für Korrekturansicht und Leseansicht.
%% Setzt den Schalter \ifkorrekturansicht voraus (gesetzt in den
%% einbindenden Dateien latex-korrekturansicht-abspann.tex bzw.
%% latex-leseansicht-abspann.tex).
%% ---------------------------------------------------------------

\normalsize

% Das esempio-Environment wird nur in der Leseansicht benötigt
\ifkorrekturansicht\else
\newenvironment{esempio}[3]%
{
    \vspace{1.5ex}
    \rlap{\underline{#1}}
    \par
    \setlength{\parindent}{0cm}
    \nopagebreak
    \leftskip=#2cm
    \rightskip=#3cm
}
{
    \par
}
\fi

\doendnotes{C}
\bigskip
\vfill

\clearpage

\footnotesize

\ifkorrekturansicht
  \lohead{\textsc{register}}
\fi

% theindex-Environment neu definieren ohne reledmac
\makeatletter
\renewenvironment{theindex}{%
  \ifkorrekturansicht
    \section*{\indexname}%
  \else
    \subsubsection*{Index der erwähnten Entitäten}%
  \fi
  \setlength{\parindent}{0pt}%
  \setlength{\parskip}{0pt plus 0.3pt}%
  \let\item\@idxitem
}{%
  \ifkorrekturansicht\clearpage\fi
}
\makeatother

\IfFileExists{\jobname-pw.ind}{\input{\jobname-pw.ind}}{}

% Quellenangabe nur in der Leseansicht
\ifkorrekturansicht\else
% Fallback-Definitionen, falls die .tex-Datei \titel etc. nicht gesetzt hat
\providecommand{\titel}{}
\providecommand{\editorInnen}{}
\providecommand{\dateiname}{\jobname}

\vspace{3cm}

\vfill

\footnotesize
\textsc{Quelle}: \titel. Herausgegeben von {\editorInnen}. In: \emph{Arthur Schnitzler: Briefwechsel mit Autorinnen und Autoren}.
 Digitale Edition, https://schnitzler-briefe.acdh.oeaw.ac.at/{\dateiname}.html (Stand \today)
\fi

\end{document}


