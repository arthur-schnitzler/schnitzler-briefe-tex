%% latex-korrekturansicht-vorspann.tex
%% Vorspann für die Korrekturansicht.
%% Lädt die gemeinsame Datei latex-vorspann.tex mit gesetztem Schalter.

\newif\ifkorrekturansicht
\korrekturansichttrue

\input{../tex-inputs/latex-vorspann}


\section[Paul Goldmann an Arthur Schnitzler, 10. 8. 1892]{L02696 Paul Goldmann an Arthur Schnitzler, 10. 8. 1892}
\nopagebreak\mylabel{L02696v}
\rehead{ }\normalsize\beginnumbering\briefempfaengerindex{Schnitzler, Arthur@\textsc{Schnitzler, Arthur}!zzzGoldmann, Paul@\emph{von Paul Goldmann}!1892-08-101@{10. 8. 1892}|(be}
\toendnotes[C]{\smallbreak\pagebreak[2]}\Standort{DLA, A:Schnitzler, HS.NZ85.1.3163.}
\physDesc{Postkarte, 692 Zeichen
\newline{}Handschrift: schwarze Tinte, lateinische Kurrent
\newline{}Versand: 1) Stempel: »\nobreak{}10 Ago. 92, Amb. Des\textcolor{gray}{c}.\nobreak{}«.   2) Stempel: »\nobreak{}Wien 1/1, 13{[}.{]} 8. 92, 9–10½ V., Bestellt\nobreak{}«. 
\newline{}Schnitzler: mit Bleistift das Datum »10/8/92« vermerkt sowie die Jahresangabe »92« der Datumsangabe ergänzt }\toendnotes[C]{\smallbreak}\pstart{}{\pb}Autriche\oindex{Oesterreich@\textbf{Österreich}, \emph{A.PCLI}|pw}! \pend{}\pstart{}\textcolor{gray}{\textbf{\begin{otherlanguage}{french}A\end{otherlanguage}}} Herrn Dr. Arthur Schnitzler\pend{}\pstart{}I. Giselastraſse 11\oindex{Ordination Arthur Schnitzler [Boesendorferstrasse 11]@\textbf{Ordination Arthur Schnitzler [Bösendorferstraße 11]}, \emph{Ordination}|pw}\pend{}\pstart{}Wien\oindex{Wien@\textbf{Wien}, \emph{A.ADM2}|pw}. \pend{}{\bigskip}\vspace{1em}
\pstart
           \raggedleft{}{\pb}San Sebastian\oindex{San Sebastian@\textbf{San Sebastian}, \emph{P.PPLA2}|pw}, 10 août\pend
           \vspace{0.5em}
\pstart
           \label{K_L02696-1v}\edtext{Me voilà donc en Espagne\oindex{Spanien@\textbf{Spanien}, \emph{A.PCLI}|pw}, mon bien cher ami. J’ai passé trois jours dans ce
               petit paradis\oindex{San Sebastian@\textbf{San Sebastian}, \emph{P.PPLA2}|pwv} au golfe de Biscaya\oindex{Biskaya@\textbf{Biskaya}, \emph{Bucht (N.BCT)}|pw}. J’ai vu des choses on ne peut
               plus espagn\oindex{Spanien@\textbf{Spanien}, \emph{A.PCLI}|pwv}oles. J’ai assisté
               aux grandes courses de taureaux, j’ai regardé la reine\pwindex{Maria Christina von Oesterreich 1858-07-21 – 1929-02-06@\textsc{Maria Christina von Österreich} (1858-07-21 – 1929-02-06), \emph{König/Königin}|pwv} prendre son bain et le petit roi\pwindex{Alfons XIII. 1886-05-17 – 1941-02-18@\textsc{Alfons XIII.} (1886-05-17 – 1941-02-18), \emph{König/Königin}|pwv} jouant dans le \strikeout{sabl} sable, j’ai fumé des cigares de Havanah\oindex{Havanna@\textbf{Havanna}, \emph{P.PPLC}|pw} et j’ai bu du vin d’Andalousie\oindex{Andalusia@\textbf{Andalusia}, \emph{A.ADM1}|pw}. Mais je t’assure, que, le premièr moment de curiosité passé, mon
               cœur était rongé de soucis et d’inquiétude nerveuse comme avant. Peut-être que tant
               cela sara beau dans le souvenir, mais dans la présesence ça ne c’est point.
               Meilleures amitiés. Bien à toi.}{\lemma{\textnormal{\emph{Me … toi.}}}\Cendnote{\textnormal{französisch: »Nun also in Spanien\oindex{Spanien@\textbf{Spanien}, \emph{A.PCLI}|pwk}, mein
                  lieber Freund: Ich habe drei Tage in diesem kleinen Paradies\oindex{San Sebastian@\textbf{San Sebastian}, \emph{P.PPLA2}|pwkv} am Golf von Biscaya\oindex{Biskaya@\textbf{Biskaya}, \emph{Bucht (N.BCT)}|pwk} verbracht. Ich habe Dinge gesehen, die
                     spanischer\oindex{Spanien@\textbf{Spanien}, \emph{A.PCLI}|pwkv} nicht sein
                  könnten. Ich war bei den großen Stierrennen dabei, habe der Königin\pwindex{Maria Christina von Oesterreich 1858-07-21 – 1929-02-06@\textsc{Maria Christina von Österreich} (1858-07-21 – 1929-02-06), \emph{König/Königin}|pwkv} beim Baden und dem kleinen König\pwindex{Alfons XIII. 1886-05-17 – 1941-02-18@\textsc{Alfons XIII.} (1886-05-17 – 1941-02-18), \emph{König/Königin}|pwkv} beim
                  Sandspielen zugesehen, ich habe Havanna\oindex{Havanna@\textbf{Havanna}, \emph{P.PPLC}|pwk}-Zigarren geraucht und Wein aus Andalusien\oindex{Andalusia@\textbf{Andalusia}, \emph{A.ADM1}|pwk} getrunken. Aber sei versichert, dass mein Herz, nachdem der
                  erste Eindruck der Neugierde vorüber war, von Sorgen und nervöser Unruhe
                  zerfressen war wie zuvor. Vielleicht ist es in der Erinnerung schön, aber in der
                  Gegenwart ist es das nicht. Mit besten Grüßen. Alles Gute für dich.«}}}\label{K_L02696-1}\pend
           \pstart \label{T_L02696-1v}\edtext{Ton \spacefill\mbox{Paul Goldmann.}}{\lemma{\textnormal{\emph{Ton Paul Goldmann.}}}\Cendnote{\textnormal{seitlich am rechten Rand}}}\label{T_L02696-1}\pend{}\selectlanguage{ngerman}\endnumbering\briefempfaengerindex{Schnitzler, Arthur@\textsc{Schnitzler, Arthur}!zzzGoldmann, Paul@\emph{von Paul Goldmann}!1892-08-101@{10. 8. 1892}|)be}\mylabel{L02696h}  \normalsize

\doendnotes{C}
\bigskip
\vfill

\clearpage

\footnotesize

\lohead{\textsc{register}}

% Definiere theindex-Environment komplett neu ohne reledmac
\makeatletter
\renewenvironment{theindex}{%
  \section*{\indexname}%
  \setlength{\parindent}{0pt}%
  \setlength{\parskip}{0pt plus 0.3pt}%
  \let\item\@idxitem
}{%
  \clearpage
}
\makeatother

\IfFileExists{\jobname-pw.ind}{\input{\jobname-pw.ind}}{}

\end{document}

      