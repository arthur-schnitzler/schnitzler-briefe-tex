%% latex-leseansicht-vorspann.tex
%% Vorspann für die Leseansicht.
%% Lädt die gemeinsame Datei latex-vorspann.tex mit nicht gesetztem Schalter.

\newif\ifkorrekturansicht
\korrekturansichtfalse

\input{../tex-inputs/latex-vorspann}


         
         \renewcommand{\erwaehntePersonen}{Personen: Alphonse Allais, Lou Andreas-Salomé, Hermann Bahr, Richard Beer-Hofmann, Otto Brahm, Max Eugen Burckhard, Siegmund Feldmann, Ludwig Fulda, Paul Goldmann, Theodor Herzl, Hugo von Hofmannsthal, Heinrich Kanner, Fedor Mamroth, Felix Salten, Adele Sandrock, Louise Schnitzler, Julius Schnitzler, Helene Schnitzler, Else Singer, Leopold Sonnemann, Hermann Sudermann, Friedrich Gustav Triesch, Pierre Veber}
         \renewcommand{\erwaehnteInstitutionen}{Institutionen: Burgtheater, Die Zeit. Wiener Wochenschrift, Frankfurter Zeitung, Lessing-Theater, Raimund-Theater, Wiener Allgemeine Zeitung}
         \renewcommand{\erwaehnteOrte}{Orte: Berlin, Paris, Wien, rue Feydeau}
         \renewcommand{\erwaehnteWerke}{Werke: Das verlorene Paradies. Schauspiel in drei Aufzügen, Die Schmetterlingsschlacht. Komödie in 4 Akten, Die neuen Menschen. Ein Schauspiel, Kunst und Leben. [Raimundtheater. Ottilie von Triesch], Le Journal, Lettre à M. Alphonse Allais sur les omnibus, Liebelei. Schauspiel in drei Akten, Ottilie. Schauspiel in vier Akten, Sterben. Novelle, Tabarin. Schauspiel in einem Act. Frei nach Catulle Mendès, Tagebuch}
               \section[Paul Goldmann an Arthur Schnitzler, 25. 10. {[}1894{]}]{ Paul Goldmann an Arthur Schnitzler, 25. 10. {[}1894{]}}\nopagebreak\mylabel{v}\rehead{ }\begin{ledgroupsized}[t]{13cm}\normalsize\beginnumbering \toendnotes[C]{\smallbreak\pagebreak[2]} \Standort{DLA, A:Schnitzler, HS.NZ85.1.3164.}
\physDesc{Brief, 3 Blätter, 12 Seiten, 5936 Zeichen
\newline{}Handschrift: schwarze Tinte, deutsche Kurrent
\newline{}Schnitzler: 1) mit Bleistift auf dem ersten Blatt die Jahreszahl »94« vermerkt  2) mit rotem Buntstift fünf Unterstreichungen}\toendnotes[C]{\smallbreak}\pstart
           \noindent{}{\pb}\textcolor{gray}{\textbf{\textbf{Frankfurter Zeitung\orgindex{Frankfurter Zeitung@Frankfurter Zeitung|pw}}}}\pend
           \pstart
           \textcolor{gray}{\textbf{(Gazette de
                     Francfort\orgindex{Frankfurter Zeitung@Frankfurter Zeitung|pw}).}}\pend
           \pstart
           \textcolor{gray}{\textbf{\textbf{\begin{otherlanguage}{french}Fondateur\end{otherlanguage} M. L. Sonnemann\pwindex{Sonnemann, Leopold 1831-10-29 – 1909-10-30@\textsc{Sonnemann, Leopold} (1831-10-29 – 1909-10-30), \emph{Journalist, Herausgeber}|pw}.}}}\pend
           \pstart
           \textcolor{gray}{\textbf{\begin{otherlanguage}{french}Journal politique, financier,\end{otherlanguage}}}\hfill \textsc{Paris\oindex{Paris@\textbf{Paris}|pw}}, 25. Oktober.\pend
           \pstart
           \textcolor{gray}{\textbf{\begin{otherlanguage}{french}commercial et littéraire.\end{otherlanguage}}}\pend
           \pstart
           \textcolor{gray}{\textbf{\begin{otherlanguage}{french}\textbf{Paraissant trois fois par jour.}\end{otherlanguage}}}\pend
           \pstart
           \textcolor{gray}{\textbf{\begin{otherlanguage}{french}\textbf{Bureaux à Paris\oindex{Paris@\textbf{Paris}|pw}:}\end{otherlanguage}}}\pend
           \pstart
           \textcolor{gray}{\textbf{\textbf{24. \begin{otherlanguage}{french}Rue
                              Feydeau\end{otherlanguage}\oindex{rue Feydeau@\textbf{rue Feydeau}|pw}.}}}\pend
           \pstart\center{}Mein lieber Freund,\pend\pstart
           Ich hatte mich ſehr nach einem ausführlichen Briefe von \strikeout{D\textcolor{gray}{e}} Dir geſehnt. Sein Ausbleiben machte mir Sorge, und ich war in meinen
               Grübeleien ſchon zu allerlei traurigen Maximen gelangt. Da kam er endlich, und er
               brachte mir ſoviel Liebes und Gutes, daß ich ihn mit einer wahren Freude geleſen
               habe. Nun wollte ich gleich antworten. Aber ſchlimme Dinge miſchten ſich dazwiſchen.
               Meine Augen ſind ſeit acht Tagen erkrankt. Der Arzt ſcheint eine \textsc{\label{K_L02616-1v}\edtext{Iritis}{\lemma{\textnormal{\emph{Iritis}}}\Cendnote{\textnormal{}}}\label{K_L02616-1h}} zu fürchten. {\pb}Die Sache wird täglich
               ſchlimmer; aber es ſind bisher doch nur Vorſymptome da. So habe ich Dir nicht
               geantwortet, nicht weil meine Sehkraft bereits angegriffen iſt, ſondern weil ich
               tief, tief verzweifelt bin. Heut iſt es mir endlich
               gelungen, meine Depreſſion zu überwinden und den ſeeliſchen Rapport mit Dir
               herzuſtellen.\pend
           \pstart
           So laß’ Dich alſo zunächſt von ganzem Herzen beglückwünſchen, daß das \label{K_L02616-2v}\edtext{Werk\pwindex{Schnitzler, Arthur 15.05.1862 – 21.10.1931@\textsc{Schnitzler, Arthur} (15.05.1862 – 21.10.1931), \emph{Schriftsteller, Mediziner}!Liebelei. Schauspiel in drei Akten1895-10-09@\strich\emph{Liebelei. Schauspiel in drei Akten} {[}1895-10-09{]}|pwv} nun endlich
                  vollendet}{\lemma{\textnormal{\emph{Werk … vollendet}}}\Cendnote{\textnormal{}}}\label{K_L02616-2h} iſt. Als wirs ſo \label{K_L02616-3v}\edtext{zuſammen beſprachen}{\lemma{\textnormal{\emph{zuſammen beſprachen}}}\Cendnote{\textnormal{}}}\label{K_L02616-3h}, hatte ich die Empfindung, daß Du es {\pb}gut machen müßteſt. Es lag in Deinem Ton ſoviel
               Sicherheit – trotz allen Suchens. \strikeout{Un} Und ich fand
               Dich auch ganz über dem Stoff ſtehend. Die Idee, die Du entworfen, iſt glänzend, in
               all’ ihrer Einfachheit. Daß Du im Stande ſein würdeſt, die Form mit Leben zu füllen,
               war ſicher. Kurzum, ich fuhr weg und erzählte meinem Onkel\pwindex{Mamroth, Fedor 21.02.1851 – 25.06.1907@\textsc{Mamroth, Fedor} (21.02.1851 – 25.06.1907), \emph{Journalist, Kritiker}|pwv}: »Du wirſt ſehen, in ein, zwei Jahren wird er ſein
               Meiſterſtück liefern.« Darum überraſcht mich nichts am Beifall der Freunde\pwindex{Salten, Felix 06.09.1869 – 08.10.1945@\textsc{Salten, Felix} (06.09.1869 – 08.10.1945), \emph{Schriftsteller, Journalist}|pwv}\pwindex{Hofmannsthal, Hugo von 1874-02-01 – 1929-07-15@\textsc{Hofmannsthal, Hugo von} (1874-02-01 – 1929-07-15), \emph{Schriftsteller}|pwv}. Mir iſt, als hätten ſie
               meine Anſicht beſtätigt. Nur möcht’ ichs gerne leſen. Dein \label{K_L02616-4v}\edtext{Original-{\pb}Manuſkript\pwindex{Schnitzler, Arthur 15.05.1862 – 21.10.1931@\textsc{Schnitzler, Arthur} (15.05.1862 – 21.10.1931), \emph{Schriftsteller, Mediziner}!Liebelei. Schauspiel in drei Akten1895-10-09@\strich\emph{Liebelei. Schauspiel in drei Akten} {[}1895-10-09{]}|pwv}}{\lemma{\textnormal{\emph{Original-Manuſkript}}}\Cendnote{\textnormal{}}}\label{K_L02616-4h} iſt nicht zu entziffern. Aber Du läßt
               wohl noch eine zweite Abſchrift machen. Ich rathe Dir, es zugleich in einem Berlin\oindex{Berlin@\textbf{Berlin}|pw}er Theater (\textsc{Brahm\pwindex{Brahm, Otto 05.02.1856 – 28.11.1912@\textsc{Brahm, Otto} (05.02.1856 – 28.11.1912), \emph{Theaterleiter, Regisseur}|pw}\orgindex{Lessing-Theater@Lessing-Theater|pwv}}) \label{K_L02616-5v}\edtext{einzureichen}{\lemma{\textnormal{\emph{einzureichen}}}\Cendnote{\textnormal{}}}\label{K_L02616-5h}. Dann ſchickſt Du mirs, bitte, vorher;
               ich gebe Dir mein Wort: in drei Tagen haſt Dus wieder. Ich freue mich für Dich, und
               ich bin glücklich in dem Gedanken, wie es jetzt mit Dir vorwärts gehen wird. Dabei
               bin ich merkwürdiger Weiſe gar nicht neidiſch – wie auf alle Anderen – ſondern nur
               froh. Es iſt, als geſchähe in meinem eigenen Leben etwas Gutes.\pend
           \pstart
           {\pb}Selbſtverſtändlich mußt Du das Stück\pwindex{Schnitzler, Arthur 15.05.1862 – 21.10.1931@\textsc{Schnitzler, Arthur} (15.05.1862 – 21.10.1931), \emph{Schriftsteller, Mediziner}!Liebelei. Schauspiel in drei Akten1895-10-09@\strich\emph{Liebelei. Schauspiel in drei Akten} {[}1895-10-09{]}|pwv} dem Burgtheater\orgindex{Burgtheater@Burgtheater|pw}{ }\label{K_L02616-6v}\edtext{einreichen}{\lemma{\textnormal{\emph{einreichen}}}\Cendnote{\textnormal{}}}\label{K_L02616-6h}. Wenn es Wien\oindex{Wien@\textbf{Wien}|pw}eriſch iſt,
               ſo müßte es doch logiſcher Weiſe noch beſſer dafür paſſen, als die \strikeout{\textcolor{gray}{×}\-\textcolor{gray}{×}\-\textcolor{gray}{×}\-\textcolor{gray}{×}s}{ }Berlin\oindex{Berlin@\textbf{Berlin}|pw}eriſchen Stücke (\textsc{\label{K_L02616-7v}\edtext{Sudermann\pwindex{Sudermann, Hermann 30.09.1857 – 21.11.1928@\textsc{Sudermann, Hermann} (30.09.1857 – 21.11.1928), \emph{Schriftsteller}|pw}\pwindex{Sudermann, Hermann 30.09.1857 – 21.11.1928@\textsc{Sudermann, Hermann} (30.09.1857 – 21.11.1928), \emph{Schriftsteller}!Schmetterlingsschlacht. Komoedie in 4 Akten1894-10-06@\strich\emph{Die Schmetterlingsschlacht. Komödie in 4 Akten} {[}1894-10-06{]}|pwv}}{\lemma{\textnormal{\emph{Sudermann}}}\Cendnote{\textnormal{}}}\label{K_L02616-7h}}, \textsc{\label{K_L02616-8v}\edtext{Fulda\pwindex{Fulda, Ludwig 15.07.1862 – 30.03.1939@\textsc{Fulda, Ludwig} (15.07.1862 – 30.03.1939), \emph{Schriftsteller, Übersetzer}|pw}\pwindex{Fulda, Ludwig 15.07.1862 – 30.03.1939@\textsc{Fulda, Ludwig} (15.07.1862 – 30.03.1939), \emph{Schriftsteller, Übersetzer}!verlorene Paradies. Schauspiel in drei Aufzuegen1890@\strich\emph{Das verlorene Paradies. Schauspiel in drei Aufzügen} {[}1890{]}|pwv}}{\lemma{\textnormal{\emph{Fulda}}}\Cendnote{\textnormal{}}}\label{K_L02616-8h}}). Daß \label{K_L02616-9v}\edtext{\textsc{Bahr\pwindex{Bahr, Hermann 19.07.1863 – 15.01.1934@\textsc{Bahr, Hermann} (19.07.1863 – 15.01.1934), \emph{Schriftsteller, Kritiker}|pw}} Dich ins \textsc{Raimund}-Theater\orgindex{Raimund-Theater@Raimund-Theater|pw}}{\lemma{\textnormal{\emph{Bahr … Raimund-Theater}}}\Cendnote{\textnormal{}}}\label{K_L02616-9h} weiſen möchte, iſt mir durchaus
               erklärlich. Das Burgtheater\orgindex{Burgtheater@Burgtheater|pw} iſt für die große
               Literatur da\strikeout{, Du aber} (\textsc{Bahr\pwindex{Bahr, Hermann 19.07.1863 – 15.01.1934@\textsc{Bahr, Hermann} (19.07.1863 – 15.01.1934), \emph{Schriftsteller, Kritiker}|pw}}, Neue Menſchen\pwindex{Bahr, Hermann 19.07.1863 – 15.01.1934@\textsc{Bahr, Hermann} (19.07.1863 – 15.01.1934), \emph{Schriftsteller, Kritiker}!neuen Menschen. Ein Schauspiel1887@\strich\emph{Die neuen Menschen. Ein Schauspiel} {[}1887{]}|pw}), Du aber ſollſt zum
               Dichter von Volksſtücken geſtempelt werden. Ich bin auch überzeugt, er wird \textsc{Burckhardt\pwindex{Burckhard, Max Eugen 14.07.1854 – 16.03.1912@\textsc{Burckhard, Max Eugen} (14.07.1854 – 16.03.1912), \emph{Schriftsteller, Rechtswissenschaftler, Theaterleiter}|pw}} gegen Dich zu beeinfluſſen ſuchen. {\pb}Der
               Schuft! So ſehr ich dagegen ankämpfe, mein Haß gegen den Burſchen\pwindex{Bahr, Hermann 19.07.1863 – 15.01.1934@\textsc{Bahr, Hermann} (19.07.1863 – 15.01.1934), \emph{Schriftsteller, Kritiker}|pwv} wächſt beinahe täglich. Es iſt ein
                  \strikeout{\textcolor{gray}{m}}{ }\strikeout{unl} unlauterer Menſch. Man braucht ihn nur \label{K_L02616-10v}\edtext{in der »Zeit\orgindex{Zeit. Wiener Wochenschrift@Die Zeit. Wiener Wochenschrift|pw}«}{\lemma{\textnormal{\emph{in der »Zeit«}}}\Cendnote{\textnormal{}}}\label{K_L02616-10h} zu beobachten. Alles,
               was von \textsc{Kanner\pwindex{Kanner, Heinrich 09.11.1864 – 15.02.1930@\textsc{Kanner, Heinrich} (09.11.1864 – 15.02.1930), \emph{Herausgeber, Publizist}|pw}} kommt, iſt nämlich, originell und muthig. In \label{K_L02616-11v}\edtext{\textsc{Bahr\pwindex{Bahr, Hermann 19.07.1863 – 15.01.1934@\textsc{Bahr, Hermann} (19.07.1863 – 15.01.1934), \emph{Schriftsteller, Kritiker}|pw}s} Reſſort}{\lemma{\textnormal{\emph{Bahrs Reſſort}}}\Cendnote{\textnormal{}}}\label{K_L02616-11h} gibt es nichts als berechnetes Laviren, verbunden mit
               frechem literariſchem Pontificiren. Socialpolitiſch und politiſch iſt die Revüe\orgindex{Zeit. Wiener Wochenschrift@Die Zeit. Wiener Wochenschrift|pwv} vorzüglich; literariſch
               finde ich ſie talent- und \strikeout{int} intereſſelos redigirt;
               da gibt es nur einen \textsc{Bahr\pwindex{Bahr, Hermann 19.07.1863 – 15.01.1934@\textsc{Bahr, Hermann} (19.07.1863 – 15.01.1934), \emph{Schriftsteller, Kritiker}|pw}}, \strikeout{de\textcolor{gray}{r}} alles Andere iſt als Relief befandelt. \strikeout{D\textcolor{gray}{er}}{ }{\pb}Er wird das ſchöne Unternehmen\orgindex{Zeit. Wiener Wochenschrift@Die Zeit. Wiener Wochenschrift|pwv} ſchon umbringen.\pend
           \pstart
           »\label{K_L02616-12v}\edtext{Sterben\pwindex{Schnitzler, Arthur 15.05.1862 – 21.10.1931@\textsc{Schnitzler, Arthur} (15.05.1862 – 21.10.1931), \emph{Schriftsteller, Mediziner}!Sterben. Novelle1894-10-01 – 1894-12-01@\strich\emph{Sterben. Novelle} {[}1894-10-01 – 1894-12-01{]}|pw}}{\lemma{\textnormal{\emph{Sterben}}}\Cendnote{\textnormal{}}}\label{K_L02616-12h}« habe ich geleſen. Es hat mich tief,
               tief ergriffen. Wenn Du wüßteſt, was für einen goldenen Reifeton Deine Kunſt jetzt
               hat! Dieſe klare und noble Einfachheit! Dieſe Gemüthstiefe! Und dieſer ſcharfe
               Verſtand, der in des Lebens dunkelſte Gründe dringt! Soweit ich bisher urtheilen
               kann, iſt es eine große Leiſtung, wohl Deine größte biſher. Nur Eines meine ich – ich
               weiß nicht, ob der Eindruck bis zum Schluß vorhalten wird – Du ſollteſt aus der
               verfluchten Illegitimtät heraus. Das bringt etwas {\pb}Halbes hinein. Wenn das Mädl ſeine Frau wäre, ſo \strikeout{\textcolor{gray}{×}} wäre es noch ergreifender, noch allgemein menſchlicher. Ich glaube, daß es
               nichts ſchaden könnte, bis nach Weihnachten mit dem Buche\pwindex{Schnitzler, Arthur 15.05.1862 – 21.10.1931@\textsc{Schnitzler, Arthur} (15.05.1862 – 21.10.1931), \emph{Schriftsteller, Mediziner}!Sterben. Novelle1894-10-01 – 1894-12-01@\strich\emph{Sterben. Novelle} {[}1894-10-01 – 1894-12-01{]}|pwv} zu warten. Vor
                  Weihnachten kommſt Du in den großen Schwall hinein, nachher tritt es
               beſſer hervor.\pend
           \pstart
           Das \label{K_L02616-13v}\edtext{Stück\pwindex{Triesch, Friedrich Gustav 16.06.1845 – 24.05.1907@\textsc{Triesch, Friedrich Gustav} (16.06.1845 – 24.05.1907), \emph{Schriftsteller}!Ottilie. Schauspiel in vier Akten1892@\strich\emph{Ottilie. Schauspiel in vier Akten} {[}1892{]}|pwv} von \textsc{Triesch\pwindex{Triesch, Friedrich Gustav 16.06.1845 – 24.05.1907@\textsc{Triesch, Friedrich Gustav} (16.06.1845 – 24.05.1907), \emph{Schriftsteller}|pw}}}{\lemma{\textnormal{\emph{Stück von Triesch}}}\Cendnote{\textnormal{}}}\label{K_L02616-13h} hat \textsc{Bahr\pwindex{Bahr, Hermann 19.07.1863 – 15.01.1934@\textsc{Bahr, Hermann} (19.07.1863 – 15.01.1934), \emph{Schriftsteller, Kritiker}|pw}} in der »Zeit\orgindex{Zeit. Wiener Wochenschrift@Die Zeit. Wiener Wochenschrift|pw}« feſt \label{K_L02616-14v}\edtext{gelobt\pwindex{Kunst und Leben. [Raimundtheater. Ottilie von Triesch]20. 10. 1894@\emph{Kunst und Leben. [Raimundtheater. Ottilie von Triesch]} {[}20. 10. 1894{]}|pwv}}{\lemma{\textnormal{\emph{gelobt}}}\Cendnote{\textnormal{}}}\label{K_L02616-14h}. Verhält ſich eben mit der \label{K_L02616-15v}\edtext{\textsc{Clique}}{\lemma{\textnormal{\emph{Clique}}}\Cendnote{\textnormal{}}}\label{K_L02616-15h}, der Herr. Pfui, pfui!\pend
           \pstart
           Das »\textsc{Journal\pwindex{?? Werk@Nicht ermittelte Verfasserinnen und Verfasser!Le Journal1892@\emph{Le Journal} {[}1892{]}|pw}}« iſt, ſeit Du es abonnirt haſt, recht ſchwach. Es iſt, als geſchähe es
               abſichtlich. Vergiß’ nicht, {\pb}die Humoriſten zu
               leſen: \textsc{Allais\pwindex{Allais, Alphonse 1854-10-20 – 28.10.1905@\textsc{Allais, Alphonse} (1854-10-20 – 28.10.1905), \emph{Schriftsteller}|pw}}, \textsc{Bill Sharp\pwindex{Veber, Pierre 1869-05-15 – 1942-08-20@\textsc{Veber, Pierre} (1869-05-15 – 1942-08-20), \emph{Schriftsteller}|pw}}{ }\textsc{etc.} Des Letzteren »\label{K_L02616-16v}\edtext{Briefe an \textsc{Allais\pwindex{Allais, Alphonse 1854-10-20 – 28.10.1905@\textsc{Allais, Alphonse} (1854-10-20 – 28.10.1905), \emph{Schriftsteller}|pw}} über die Zündhölzchen\pwindex{Lettre à M. Alphonse Allais sur les omnibus1894-10-18@\emph{Lettre à M. Alphonse Allais sur les omnibus} {[}1894-10-18{]}|pw}}{\lemma{\textnormal{\emph{Briefe … Zündhölzchen}}}\Cendnote{\textnormal{}}}\label{K_L02616-16h} und \label{K_L02616-17v}\edtext{über die Omnibuſſe\pwindex{Lettre à M. Alphonse Allais sur les omnibus1894-10-18@\emph{Lettre à M. Alphonse Allais sur les omnibus} {[}1894-10-18{]}|pw}«}{\lemma{\textnormal{\emph{über die Omnibuſſe«}}}\Cendnote{\textnormal{}}}\label{K_L02616-17h} waren köſtlich. Freilich muß man ein wenig \label{T_L02616-1v}\edtext{Lokalkenntniß}{\lemma{\textnormal{\emph{Lokalkenntniß}}}\Cendnote{\textnormal{}}}\label{T_L02616-1h} haben, um das in ſeiner ganzen Größe zu würdigen. Du haſt
                  \textsc{30 fr. 40 ct.} bei mir gut. Was ſoll damit geſchehen? Ein
               paar Sachen habe ich für Dich geſammelt, wie ich Dir verſprochen. Es iſt nicht viel
               Bedeutendes drunter, aber allerlei {\pb}Kurioſes. Es iſt
               natürlich lächerlich, daß ich Dir zugemuthet habe, über das Alles mir zu berichten.
               Schreib’ mir nur ein allgemeines Wort, obs Dir ſo recht iſt. Dann fahre ich fort.\pend
           \pstart
           \label{K_L02616-18v}\edtext{Das mit dem \strikeout{ſeh} ſechzehnjährigen Mädel\pwindex{Singer, Else 25.06.1878 – 1943?@\textsc{Singer, Else} (25.06.1878 – 1943?), \emph{Schriftstellerin, Sprachlehrerin}|pwv}}{\lemma{\textnormal{\emph{Das … Mädel}}}\Cendnote{\textnormal{gemeint war wohl Else Singer\pwindex{Singer, Else 25.06.1878 – 1943?@\textsc{Singer, Else} (25.06.1878 – 1943?), \emph{Schriftstellerin, Sprachlehrerin}|pwk}, mit der Schnitzler\pwindex{Schnitzler, Arthur 15.05.1862 – 21.10.1931@\textsc{Schnitzler, Arthur} (15.05.1862 – 21.10.1931), \emph{Schriftsteller, Mediziner}|pwk} zu dieser Zeit viel Kontakt hatte (vgl. \emph{Tagebuch}\pwindex{\textcolor{red}{\textsuperscript{XXXX1 indx}}!Tagebuch1981 – 2000@\strich\emph{Tagebuch} {[}Hrsg., 1981 – 2000{]}|pwk}); konkreter Bezug unklar}}}\label{K_L02616-18h} hat mich gerührt.
               Liebes, kleines Ding!\pend
           \pstart
           Die Frau \textsc{Andreas\pwindex{Andreas-Salome, Lou 12.02.1861 – 05.02.1937@\textsc{Andreas-Salomé, Lou} (12.02.1861 – 05.02.1937), \emph{Schriftstellerin}|pw}} ſprach ich hier noch einmal. Ich glaube, ſie hat mich lieb gehabt. Nun iſt ſie
               im Groll von mir geſchieden, weil ich ſie zurückgeſtoßen habe. Und allſogleich ſtellt
                  {\pb}ſich bei mir die Reue ein. Aber ſie hat
               unwiderruflich mit mir gebrochen.\pend
           \pstart
           Grüß’ mir \textsc{Richard\pwindex{Beer-Hofmann, Richard 1866-07-11 – 1945-09-26@\textsc{Beer-Hofmann, Richard} (1866-07-11 – 1945-09-26), \emph{Schriftsteller}|pw}} und \textsc{Loris\pwindex{Hofmannsthal, Hugo von 1874-02-01 – 1929-07-15@\textsc{Hofmannsthal, Hugo von} (1874-02-01 – 1929-07-15), \emph{Schriftsteller}|pw}}.\pend
           \pstart
           \textsc{Herzl\pwindex{Herzl, Theodor 1860-05-02 – 1904-07-03@\textsc{Herzl, Theodor} (1860-05-02 – 1904-07-03), \emph{Schriftsteller, Journalist}|pw}} ſehe ich kaum. Bin wieder ganz mit ihm auseinander. Er war ſeit ſeiner
               Rückkunft einmal bei mir, um mir anzuzeigen, daß \label{K_L02616-19v}\edtext{»\textsc{Tabarin\pwindex{Herzl, Theodor 1860-05-02 – 1904-07-03@\textsc{Herzl, Theodor} (1860-05-02 – 1904-07-03), \emph{Schriftsteller, Journalist}!Tabarin. Schauspiel in einem Act. Frei nach Catulle Mendes1884@\strich\emph{Tabarin. Schauspiel in einem Act. Frei nach Catulle Mendès} {[}1884{]}|pw}}« werde aufgeführt}{\lemma{\textnormal{\emph{»Tabarin« werde aufgeführt}}}\Cendnote{\textnormal{}}}\label{K_L02616-19h} werden, was
               mich neidiſch machen ſollte. Seitdem verkehrt er täglich mit \textsc{Feldmann\pwindex{Feldmann, Siegmund @\textsc{Feldmann, Siegmund}, \emph{Schriftsteller, Journalist}|pw}} und läßt ſich bei mir nicht mehr ſehen. So habe ich ihn auch links liegen
               laſſen.\pend
           \pstart
           Aber Deinen Gruß und {\pb}Dein Lob habe ich ihm
               ausgerichtet. Das hat ihn ſehr gefreut.\pend
           \pstart
           Meine Sachen ſammeln? Ich weiß genau, daß ſie es nicht werth ſind. Aber mir thut es
               wohl, wenn Du mir das Gegentheil ſchreibſt. Natürlich werde ich ſie nicht
               ſammeln.\pend
           \pstart
           Bitte, mich Frl. \textsc{Sandrock\pwindex{Sandrock, Adele 1863-08-19 – 1937-08-30@\textsc{Sandrock, Adele} (1863-08-19 – 1937-08-30), \emph{Schauspielerin}|pw}} zu empfehlen.\pend
           \pstart
           Bitte, mich Deiner Frau Mutter\pwindex{Schnitzler, Louise 1840-07-08 – 1911-09-09@\textsc{Schnitzler, Louise} (1840-07-08 – 1911-09-09)|pwv} recht herzlich zu empfehlen. Bitte, Deinen Bruder\pwindex{Schnitzler, Julius 13.07.1865 – 29.06.1939@\textsc{Schnitzler, Julius} (13.07.1865 – 29.06.1939), \emph{Chirurg}|pwv} und Deine entzückende kleine Schwägerin\pwindex{Schnitzler, Helene 16.07.1871 – September 1941@\textsc{Schnitzler, Helene} (16.07.1871 – September 1941)|pwv} recht herzlich von
               mir zu grüßen.\pend
           \pstart
           Und ſei Du ſelbſt von Herzen gegrüßt! Dein{\\[\baselineskip]}treuer \spacefill\mbox{Paul
                  Goldmann}\pend
           \leftskip=0em{}\pstart
           \noindent{}\label{T_L02616-2v}\edtext{\textsc{Salten\pwindex{Salten, Felix 06.09.1869 – 08.10.1945@\textsc{Salten, Felix} (06.09.1869 – 08.10.1945), \emph{Schriftsteller, Journalist}|pw}} laſſe ich zu ſeiner \label{K_L02616-20v}\edtext{neuen
                     Stellung\orgindex{Wiener Allgemeine Zeitung@Wiener Allgemeine Zeitung|pwv}}{\lemma{\textnormal{\emph{neuen
                     Stellung}}}\Cendnote{\textnormal{}}}\label{K_L02616-20h} gratuliren}{\lemma{\textnormal{\emph{Salten … gratuliren}}}\Cendnote{\textnormal{}}}\label{T_L02616-2h}.\pend
           \pstart
           {\pb}\label{T_L02616-3v}\edtext{Wenn Du vom Burgtheater\orgindex{Burgtheater@Burgtheater|pw} Antwort haſt, erbitte ich \uline{umgehende} Mittheilung}{\lemma{\textnormal{\emph{Wenn … Mittheilung}}}\Cendnote{\textnormal{}}}\label{T_L02616-3h}.\pend
           
         
         \endnumbering\mylabel{h}\end{ledgroupsized}  \newcommand{\dateiname}{L02616}\newcommand{\titel}{Paul Goldmann an Arthur Schnitzler, 25. 10. [1894]}\newcommand{\editorInnen}{Martin Anton Müller und Laura Untner}%% latex-leseansicht-abspann.tex
%% Abspann für die Leseansicht.
%% Der Schalter \ifkorrekturansicht ist bereits durch den Vorspann gesetzt.

%% latex-abspann.tex
%% Gemeinsamer Abspann für Korrekturansicht und Leseansicht.
%% Setzt den Schalter \ifkorrekturansicht voraus (gesetzt in den
%% einbindenden Dateien latex-korrekturansicht-abspann.tex bzw.
%% latex-leseansicht-abspann.tex).
%% ---------------------------------------------------------------

\normalsize

% Das esempio-Environment wird nur in der Leseansicht benötigt
\ifkorrekturansicht\else
\newenvironment{esempio}[3]%
{
    \vspace{1.5ex}
    \rlap{\underline{#1}}
    \par
    \setlength{\parindent}{0cm}
    \nopagebreak
    \leftskip=#2cm
    \rightskip=#3cm
}
{
    \par
}
\fi

\doendnotes{C}
\bigskip
\vfill

\clearpage

\footnotesize

\ifkorrekturansicht
  \lohead{\textsc{register}}
\fi

% theindex-Environment neu definieren ohne reledmac
\makeatletter
\renewenvironment{theindex}{%
  \ifkorrekturansicht
    \section*{\indexname}%
  \else
    \subsubsection*{Index der erwähnten Entitäten}%
  \fi
  \setlength{\parindent}{0pt}%
  \setlength{\parskip}{0pt plus 0.3pt}%
  \let\item\@idxitem
}{%
  \ifkorrekturansicht\clearpage\fi
}
\makeatother

\IfFileExists{\jobname-pw.ind}{\input{\jobname-pw.ind}}{}

% Quellenangabe nur in der Leseansicht
\ifkorrekturansicht\else
% Fallback-Definitionen, falls die .tex-Datei \titel etc. nicht gesetzt hat
\providecommand{\titel}{}
\providecommand{\editorInnen}{}
\providecommand{\dateiname}{\jobname}

\vspace{3cm}

\vfill

\footnotesize
\textsc{Quelle}: \titel. Herausgegeben von {\editorInnen}. In: \emph{Arthur Schnitzler: Briefwechsel mit Autorinnen und Autoren}.
 Digitale Edition, https://schnitzler-briefe.acdh.oeaw.ac.at/{\dateiname}.html (Stand \today)
\fi

\end{document}


      