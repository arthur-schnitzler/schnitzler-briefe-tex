%% latex-leseansicht-vorspann.tex
%% Vorspann für die Leseansicht.
%% Lädt die gemeinsame Datei latex-vorspann.tex mit nicht gesetztem Schalter.

\newif\ifkorrekturansicht
\korrekturansichtfalse

\input{../tex-inputs/latex-vorspann}


\section[Paul Goldmann an Arthur Schnitzler, 25. 10. [1894]]{L02616 Paul Goldmann an Arthur Schnitzler, 25. 10. [1894]}
\nopagebreak\mylabel{L02616v}
\rehead{ }\normalsize\beginnumbering\briefempfaengerindex{Schnitzler, Arthur@\textsc{Schnitzler, Arthur}!zzzGoldmann, Paul@\emph{von Paul Goldmann}!1894-10-251@{25. 10. [1894]}|(be}
\toendnotes[C]{\smallbreak\pagebreak[2]}
\correspDesc{Versand  durch Paul Goldmann am 25. 10. [1894] in Paris
\newline{}Erhalt  durch Arthur Schnitzler im Zeitraum [26. 10. 1894 – 30. 10. 1894?] in Wien}\toendnotes[C]{\smallbreak}
\Standort{DLA, A:Schnitzler, HS.NZ85.1.3164.}
\physDesc{Brief, 3 Blätter, 12 Seiten, 5936 Zeichen
\newline{}Handschrift: schwarze Tinte, deutsche Kurrent
\newline{}Schnitzler: 1) mit Bleistift auf dem ersten Blatt die Jahreszahl »94« vermerkt  2) mit rotem Buntstift fünf Unterstreichungen}\toendnotes[C]{\smallbreak}
\pstart
           {\pb}\textcolor{gray}{\textbf{\textbf{Frankfurter Zeitung\orgindex{Frankfurter Zeitung@Frankfurter Zeitung|pw}}}}\pend
           
\pstart
           \textcolor{gray}{\textbf{(Gazette de
                     Francfort\orgindex{Frankfurter Zeitung@Frankfurter Zeitung|pw}).}}\pend
           
\pstart
           \textcolor{gray}{\textbf{\textbf{\begin{otherlanguage}{french}Fondateur\end{otherlanguage} M. L. Sonnemann\pwindex{Sonnemann, Leopold 29.\,10.\,1831 Höchberg – 30.\,10.\,1909 Frankfurt am Main@\textsc{Sonnemann, Leopold} (29.\,10.\,1831 Höchberg – 30.\,10.\,1909 Frankfurt am Main), \emph{Journalist, Herausgeber}|pw}.}}}\pend
           
\pstart
           \textcolor{gray}{\textbf{\begin{otherlanguage}{french}Journal politique, financier,\end{otherlanguage}}}\hfill \textsc{Paris\oindex{Paris@\textbf{Paris}, \emph{Hauptstadt}|pw}}, 25. Oktober.\pend
           
\pstart
           \textcolor{gray}{\textbf{\begin{otherlanguage}{french}commercial et littéraire.\end{otherlanguage}}}\pend
           
\pstart
           \textcolor{gray}{\textbf{\begin{otherlanguage}{french}\textbf{Paraissant trois fois par jour.}\end{otherlanguage}}}\pend
           
\pstart
           \textcolor{gray}{\textbf{\begin{otherlanguage}{french}\textbf{Bureaux à Paris\oindex{Paris@\textbf{Paris}, \emph{Hauptstadt}|pw}:}\end{otherlanguage}}}\pend
           
\pstart
           \textcolor{gray}{\textbf{\textbf{24. \begin{otherlanguage}{french}Rue
                              Feydeau\end{otherlanguage}\oindex{rue Feydeau@\textbf{rue Feydeau}, \emph{Straße}|pw}.}}}\pend
           
\pstart\center{}Mein lieber Freund,\pend\vspace{0.5em}
\pstart
           Ich hatte mich{ }ſehr nach einem ausführlichen Briefe von \strikeout{D\textcolor{gray}{e}} Dir geſehnt. Sein Ausbleiben machte mir Sorge, und ich war in meinen
               Grübeleien{ }ſchon zu allerlei traurigen Maximen gelangt. Da kam er endlich, und er
               brachte mir{ }ſoviel Liebes und Gutes, daß ich ihn mit einer wahren Freude geleſen
               habe. Nun wollte ich gleich antworten. Aber{ }ſchlimme Dinge miſchten{ }ſich dazwiſchen.
               Meine Augen{ }ſind{ }ſeit acht Tagen erkrankt. Der Arzt{ }ſcheint eine \textsc{\label{K_L02616-1v}\edtext{Iritis}{\lemma{\textnormal{\emph{Iritis}}}\Cendnote{\textnormal{Entzündung der Regenbogenhaut}}}\label{K_L02616-1}} zu fürchten. {\pb}Die Sache wird täglich{ }ſchlimmer; aber es{ }ſind bisher doch nur Vorſymptome da. So habe ich Dir nicht
               geantwortet, nicht weil meine Sehkraft bereits angegriffen iſt,{ }ſondern weil ich
               tief, tief verzweifelt bin. Heut iſt es mir endlich
               gelungen, meine Depreſſion zu überwinden und den{ }ſeeliſchen Rapport mit Dir
               herzuſtellen.\pend
           
\pstart
           So laß’ Dich alſo zunächſt von ganzem Herzen beglückwünſchen, daß das \label{K_L02616-2v}\edtext{Werk\pwindex{Schnitzler, Arthur 15.\,5.\,1862 Wien – 21.\,10.\,1931 ebd.@\textsc{Schnitzler, Arthur} (15.\,5.\,1862 Wien – 21.\,10.\,1931 ebd.), \emph{Schriftsteller, Mediziner}!Liebelei. Schauspiel in drei Akten@\strich\emph{Liebelei. Schauspiel in drei Akten}|pwv} nun endlich
               vollendet}{\lemma{\textnormal{\emph{Werk … vollendet}}}\Cendnote{\textnormal{Am 14. 10. 1894 hatte Schnitzler{ }\emph{Liebelei}\pwindex{Schnitzler, Arthur 15.\,5.\,1862 Wien – 21.\,10.\,1931 ebd.@\textsc{Schnitzler, Arthur} (15.\,5.\,1862 Wien – 21.\,10.\,1931 ebd.), \emph{Schriftsteller, Mediziner}!Liebelei. Schauspiel in drei Akten@\strich\emph{Liebelei. Schauspiel in drei Akten}|pwk}{ }Hugo von Hofmannsthal\pwindex{Hofmannsthal, Hugo von 1.\,2.\,1874 Wien – 15.\,7.\,1929 Rodaun@\textsc{Hofmannsthal, Hugo von} (1.\,2.\,1874 Wien – 15.\,7.\,1929 Rodaun), \emph{Schriftsteller}|pwk} und Felix Salten\pwindex{Salten, Felix 6.\,9.\,1869 Budapest – 8.\,10.\,1945 Zürich@\textsc{Salten, Felix} (6.\,9.\,1869 Budapest – 8.\,10.\,1945 Zürich), \emph{Schriftsteller, Journalist, Chefredakteur}|pwk} vorgelesen. Diese urteilten, das Stück\pwindex{Schnitzler, Arthur 15.\,5.\,1862 Wien – 21.\,10.\,1931 ebd.@\textsc{Schnitzler, Arthur} (15.\,5.\,1862 Wien – 21.\,10.\,1931 ebd.), \emph{Schriftsteller, Mediziner}!Liebelei. Schauspiel in drei Akten@\strich\emph{Liebelei. Schauspiel in drei Akten}|pwkv} sei bis auf wenige
                  Formulierungen fertig. Schnitzler hatte die
                  Fertigstellung des Stück\pwindex{Schnitzler, Arthur 15.\,5.\,1862 Wien – 21.\,10.\,1931 ebd.@\textsc{Schnitzler, Arthur} (15.\,5.\,1862 Wien – 21.\,10.\,1931 ebd.), \emph{Schriftsteller, Mediziner}!Liebelei. Schauspiel in drei Akten@\strich\emph{Liebelei. Schauspiel in drei Akten}|pwkv}es
                  bereits zehn Tage vorher (am 4. 10. 1894) im \emph{Tagebuch}\pwindex{Schnitzler, Arthur 15.\,5.\,1862 Wien – 21.\,10.\,1931 ebd.@\textsc{Schnitzler, Arthur} (15.\,5.\,1862 Wien – 21.\,10.\,1931 ebd.), \emph{Schriftsteller, Mediziner}!Tagebuch@\strich\emph{Tagebuch}|pwk}
                  notiert.}}}\label{K_L02616-2} iſt. Als wirs{ }ſo \label{K_L02616-3v}\edtext{zuſammen beſprachen}{\lemma{\textnormal{\emph{zusammen besprachen}}}\Cendnote{\textnormal{Siehe A. S.: \emph{Tagebuch}, 30. 8. 1894.
                  }}}\label{K_L02616-3}, hatte ich die Empfindung, daß Du es {\pb}gut machen müßteſt. Es lag in Deinem Ton{ }ſoviel
               Sicherheit – trotz allen Suchens. \strikeout{Un} Und ich fand
               Dich auch ganz über dem Stoff{ }ſtehend. Die Idee, die Du entworfen, iſt glänzend, in
               all’ ihrer Einfachheit. Daß Du im Stande{ }ſein würdeſt, die Form mit Leben zu füllen,
               war{ }ſicher. Kurzum, ich fuhr weg und erzählte meinem Onkel\pwindex{Mamroth, Fedor 21.\,2.\,1851 Breslau – 25.\,6.\,1907 Frankfurt am Main@\textsc{Mamroth, Fedor} (21.\,2.\,1851 Breslau – 25.\,6.\,1907 Frankfurt am Main), \emph{Journalist, Kritiker}|pwv}: »Du wirſt{ }ſehen, in ein, zwei Jahren wird er{ }ſein
               Meiſterſtück liefern.« Darum überraſcht mich nichts am Beifall der Freunde\pwindex{Salten, Felix 6.\,9.\,1869 Budapest – 8.\,10.\,1945 Zürich@\textsc{Salten, Felix} (6.\,9.\,1869 Budapest – 8.\,10.\,1945 Zürich), \emph{Schriftsteller, Journalist, Chefredakteur}|pwv}\pwindex{Hofmannsthal, Hugo von 1.\,2.\,1874 Wien – 15.\,7.\,1929 Rodaun@\textsc{Hofmannsthal, Hugo von} (1.\,2.\,1874 Wien – 15.\,7.\,1929 Rodaun), \emph{Schriftsteller}|pwv}. Mir iſt, als hätten{ }ſie
               meine Anſicht beſtätigt. Nur möcht’ ichs gerne leſen. Dein \label{K_L02616-4v}\edtext{Original-{\pb}Manuſkript\pwindex{Schnitzler, Arthur 15.\,5.\,1862 Wien – 21.\,10.\,1931 ebd.@\textsc{Schnitzler, Arthur} (15.\,5.\,1862 Wien – 21.\,10.\,1931 ebd.), \emph{Schriftsteller, Mediziner}!Liebelei. Schauspiel in drei Akten@\strich\emph{Liebelei. Schauspiel in drei Akten}|pwv}}{\lemma{\textnormal{\emph{Original-Manuskript}}}\Cendnote{\textnormal{Goldmann\pwindex{Goldmann, Paul 31.\,1.\,1865 Breslau – 25.\,9.\,1935 Wien@\textsc{Goldmann, Paul} (31.\,1.\,1865 Breslau – 25.\,9.\,1935 Wien), \emph{Schriftsteller, Journalist}|pwk} dürfte hier eine (zutreffende)
                  Annahme äußern, nicht ein Urteil nachdem er das Manuskript der Handschrift H\textsuperscript{2} eingesehen hatte. Vgl. A. S.: \emph{Liebelei}\pwindex{Schnitzler, Arthur 15.\,5.\,1862 Wien – 21.\,10.\,1931 ebd.@\textsc{Schnitzler, Arthur} (15.\,5.\,1862 Wien – 21.\,10.\,1931 ebd.), \emph{Schriftsteller, Mediziner}!Liebelei. Schauspiel in drei Akten@\strich\emph{Liebelei. Schauspiel in drei Akten}|pwk}. Historisch-kritische Ausgabe. Herausgegeben von
                        Peter Michael Braunwarth, Gerhard Hubmann und Isabella Schwentner. Berlin,
                        Boston: \emph{de Gruyter}{ }2014. (Werke in historisch-kritischen Ausgaben, hg. Konstanze
                        Fliedl), S. 333–915.}}}\label{K_L02616-4} iſt nicht zu entziffern. Aber Du läßt
               wohl noch eine zweite Abſchrift machen. Ich rathe Dir, es zugleich in einem Berlin\oindex{Berlin@\textbf{Berlin}, \emph{Hauptstadt}|pw}er Theater (\textsc{Brahm\pwindex{Brahm, Otto 5.\,2.\,1856 Hamburg – 28.\,11.\,1912 Berlin@\textsc{Brahm, Otto} (5.\,2.\,1856 Hamburg – 28.\,11.\,1912 Berlin), \emph{Theaterleiter, Regisseur}|pw}\orgindex{Lessing-Theater@Lessing-Theater|pwv}}) \label{K_L02616-5v}\edtext{einzureichen}{\lemma{\textnormal{\emph{einzureichen}}}\Cendnote{\textnormal{Brahm\pwindex{Brahm, Otto 5.\,2.\,1856 Hamburg – 28.\,11.\,1912 Berlin@\textsc{Brahm, Otto} (5.\,2.\,1856 Hamburg – 28.\,11.\,1912 Berlin), \emph{Theaterleiter, Regisseur}|pwk} leitete das \emph{Lessing-Theater}\orgindex{Lessing-Theater@Lessing-Theater|pwk}. Schnitzler folgte dem Rat Goldmanns\pwindex{Goldmann, Paul 31.\,1.\,1865 Breslau – 25.\,9.\,1935 Wien@\textsc{Goldmann, Paul} (31.\,1.\,1865 Breslau – 25.\,9.\,1935 Wien), \emph{Schriftsteller, Journalist}|pwk} nicht. Stattdessen legt die Korrespondenz zwischen Schnitzler
                   und Brahm\pwindex{Brahm, Otto 5.\,2.\,1856 Hamburg – 28.\,11.\,1912 Berlin@\textsc{Brahm, Otto} (5.\,2.\,1856 Hamburg – 28.\,11.\,1912 Berlin), \emph{Theaterleiter, Regisseur}|pwk} nahe, dass der Theaterdirektor\pwindex{Brahm, Otto 5.\,2.\,1856 Hamburg – 28.\,11.\,1912 Berlin@\textsc{Brahm, Otto} (5.\,2.\,1856 Hamburg – 28.\,11.\,1912 Berlin), \emph{Theaterleiter, Regisseur}|pwkv}, nachdem
                   \emph{Liebelei}\pwindex{Schnitzler, Arthur 15.\,5.\,1862 Wien – 21.\,10.\,1931 ebd.@\textsc{Schnitzler, Arthur} (15.\,5.\,1862 Wien – 21.\,10.\,1931 ebd.), \emph{Schriftsteller, Mediziner}!Liebelei. Schauspiel in drei Akten@\strich\emph{Liebelei. Schauspiel in drei Akten}|pwk} vom \emph{Burgtheater}\orgindex{Burgtheater@Burgtheater|pwk} akzeptiert worden war, selbst aktiv
                  wurde.}}}\label{K_L02616-5}. Dann{ }ſchickſt Du mirs, bitte, vorher;
               ich gebe Dir mein Wort: in drei Tagen haſt Dus wieder. Ich freue mich für Dich, und
               ich bin glücklich in dem Gedanken, wie es jetzt mit Dir vorwärts gehen wird. Dabei
               bin ich merkwürdiger Weiſe gar nicht neidiſch – wie auf alle Anderen –{ }ſondern nur
               froh. Es iſt, als geſchähe in meinem eigenen Leben etwas Gutes.\pend
           
\pstart
           {\pb}Selbſtverſtändlich mußt Du das Stück\pwindex{Schnitzler, Arthur 15.\,5.\,1862 Wien – 21.\,10.\,1931 ebd.@\textsc{Schnitzler, Arthur} (15.\,5.\,1862 Wien – 21.\,10.\,1931 ebd.), \emph{Schriftsteller, Mediziner}!Liebelei. Schauspiel in drei Akten@\strich\emph{Liebelei. Schauspiel in drei Akten}|pwv} dem Burgtheater\orgindex{Burgtheater@Burgtheater|pw}{ }\label{K_L02616-6v}\edtext{einreichen}{\lemma{\textnormal{\emph{einreichen}}}\Cendnote{\textnormal{Am 27. 10. 1894 erhielt Schnitzler
                  eine Abschrift von \emph{Liebelei}\pwindex{Schnitzler, Arthur 15.\,5.\,1862 Wien – 21.\,10.\,1931 ebd.@\textsc{Schnitzler, Arthur} (15.\,5.\,1862 Wien – 21.\,10.\,1931 ebd.), \emph{Schriftsteller, Mediziner}!Liebelei. Schauspiel in drei Akten@\strich\emph{Liebelei. Schauspiel in drei Akten}|pwk}, am XXXX Auszeichnungsfehler: Dokument L00395 nicht gefunden gratulierte Burckhard\pwindex{Burckhard, Max Eugen 14.\,7.\,1854 Korneuburg – 16.\,3.\,1912 Wien@\textsc{Burckhard, Max Eugen} (14.\,7.\,1854 Korneuburg – 16.\,3.\,1912 Wien), \emph{Schriftsteller, Rechtswissenschaftler, Theaterleiter}|pwk} und deutete die Annahme an. Sofern
                  es nicht eine weitere Abschrift gab, hatte er also schnell gelesen.}}}\label{K_L02616-6}. Wenn es Wien\oindex{Wien@\textbf{Wien}, \emph{Verwaltungsgebiet}|pw}eriſch iſt,{ }ſo müßte es doch logiſcher Weiſe noch beſſer dafür paſſen, als die \strikeout{\textcolor{gray}{×}\-\textcolor{gray}{×}\-\textcolor{gray}{×}\-\textcolor{gray}{×}s}{ }Berlin\oindex{Berlin@\textbf{Berlin}, \emph{Hauptstadt}|pw}eriſchen Stücke (\textsc{\label{K_L02616-7v}\edtext{Sudermann\pwindex{Sudermann, Hermann 30.\,9.\,1857 Macikai – 21.\,11.\,1928 Berlin@\textsc{Sudermann, Hermann} (30.\,9.\,1857 Macikai – 21.\,11.\,1928 Berlin), \emph{Schriftsteller}|pw}\pwindex{Sudermann, Hermann 30.\,9.\,1857 Macikai – 21.\,11.\,1928 Berlin@\textsc{Sudermann, Hermann} (30.\,9.\,1857 Macikai – 21.\,11.\,1928 Berlin), \emph{Schriftsteller}!Schmetterlingsschlacht. Komödie in 4 Akten@\strich\emph{Die Schmetterlingsschlacht. Komödie in 4 Akten}|pwv}}{\lemma{\textnormal{\emph{Sudermann}}}\Cendnote{\textnormal{\emph{Die Schmetterlingsschlacht}\pwindex{Sudermann, Hermann 30.\,9.\,1857 Macikai – 21.\,11.\,1928 Berlin@\textsc{Sudermann, Hermann} (30.\,9.\,1857 Macikai – 21.\,11.\,1928 Berlin), \emph{Schriftsteller}!Schmetterlingsschlacht. Komödie in 4 Akten@\strich\emph{Die Schmetterlingsschlacht. Komödie in 4 Akten}|pwk} von Hermann Sudermann\pwindex{Sudermann, Hermann 30.\,9.\,1857 Macikai – 21.\,11.\,1928 Berlin@\textsc{Sudermann, Hermann} (30.\,9.\,1857 Macikai – 21.\,11.\,1928 Berlin), \emph{Schriftsteller}|pwk} hatte am
                      6.\,10.\,1894 die Uraufführung\eventindex{Burgtheater@\textbf{Burgtheater}!Uraufführung der Schmetterlingsschlacht, 6.10.1894@Uraufführung der Schmetterlingsschlacht, 6.10.1894|pwkv} am \emph{Burgtheater}\orgindex{Burgtheater@Burgtheater|pwk}.}}}\label{K_L02616-7}}, \textsc{\label{K_L02616-8v}\edtext{Fulda\pwindex{Fulda, Ludwig 15.\,7.\,1862 Frankfurt am Main – 30.\,3.\,1939 Berlin@\textsc{Fulda, Ludwig} (15.\,7.\,1862 Frankfurt am Main – 30.\,3.\,1939 Berlin), \emph{Schriftsteller, Übersetzer}|pw}\pwindex{Fulda, Ludwig 15.\,7.\,1862 Frankfurt am Main – 30.\,3.\,1939 Berlin@\textsc{Fulda, Ludwig} (15.\,7.\,1862 Frankfurt am Main – 30.\,3.\,1939 Berlin), \emph{Schriftsteller, Übersetzer}!verlorene Paradies. Schauspiel in drei Aufzügen@\strich\emph{Das verlorene Paradies. Schauspiel in drei Aufzügen}|pwv}}{\lemma{\textnormal{\emph{Fulda}}}\Cendnote{\textnormal{\emph{Das verlorene Paradies}\pwindex{Fulda, Ludwig 15.\,7.\,1862 Frankfurt am Main – 30.\,3.\,1939 Berlin@\textsc{Fulda, Ludwig} (15.\,7.\,1862 Frankfurt am Main – 30.\,3.\,1939 Berlin), \emph{Schriftsteller, Übersetzer}!verlorene Paradies. Schauspiel in drei Aufzügen@\strich\emph{Das verlorene Paradies. Schauspiel in drei Aufzügen}|pwk} von Ludwig Fulda\pwindex{Fulda, Ludwig 15.\,7.\,1862 Frankfurt am Main – 30.\,3.\,1939 Berlin@\textsc{Fulda, Ludwig} (15.\,7.\,1862 Frankfurt am Main – 30.\,3.\,1939 Berlin), \emph{Schriftsteller, Übersetzer}|pwk} wurde erstmals am
                     25.\,1.\,1891 am \emph{Burgtheater}\orgindex{Burgtheater@Burgtheater|pwk}
                     gegeben und befand sich noch 1894 auf dem Spielplan.}}}\label{K_L02616-8}}). Daß \label{K_L02616-9v}\edtext{\textsc{Bahr\pwindex{Bahr, Hermann 19.\,7.\,1863 Linz – 15.\,1.\,1934 München@\textsc{Bahr, Hermann} (19.\,7.\,1863 Linz – 15.\,1.\,1934 München), \emph{Schriftsteller, Kritiker}|pw}} Dich ins \textsc{Raimund}-Theater\orgindex{Raimund-Theater@Raimund-Theater|pw}}{\lemma{\textnormal{\emph{Bahr … Raimund-Theater}}}\Cendnote{\textnormal{Siehe A. S.: \emph{Tagebuch}, 16. 10. 1894 und XXXX Auszeichnungsfehler: Dokument L00387 nicht gefunden.
               }}}\label{K_L02616-9} weiſen möchte, iſt mir durchaus
               erklärlich. Das Burgtheater\orgindex{Burgtheater@Burgtheater|pw} iſt für die große
               Literatur da\strikeout{, Du aber} (\textsc{Bahr\pwindex{Bahr, Hermann 19.\,7.\,1863 Linz – 15.\,1.\,1934 München@\textsc{Bahr, Hermann} (19.\,7.\,1863 Linz – 15.\,1.\,1934 München), \emph{Schriftsteller, Kritiker}|pw}}, Neue Menſchen\pwindex{Bahr, Hermann 19.\,7.\,1863 Linz – 15.\,1.\,1934 München@\textsc{Bahr, Hermann} (19.\,7.\,1863 Linz – 15.\,1.\,1934 München), \emph{Schriftsteller, Kritiker}!neuen Menschen. Ein Schauspiel@\strich\emph{Die neuen Menschen. Ein Schauspiel}|pw}), Du aber{ }ſollſt zum
               Dichter von Volksſtücken geſtempelt werden. Ich bin auch überzeugt, er wird \textsc{Burckhardt\pwindex{Burckhard, Max Eugen 14.\,7.\,1854 Korneuburg – 16.\,3.\,1912 Wien@\textsc{Burckhard, Max Eugen} (14.\,7.\,1854 Korneuburg – 16.\,3.\,1912 Wien), \emph{Schriftsteller, Rechtswissenschaftler, Theaterleiter}|pw}} gegen Dich zu beeinfluſſen{ }ſuchen. {\pb}Der
               Schuft! So{ }ſehr ich dagegen ankämpfe, mein Haß gegen den Burſchen\pwindex{Bahr, Hermann 19.\,7.\,1863 Linz – 15.\,1.\,1934 München@\textsc{Bahr, Hermann} (19.\,7.\,1863 Linz – 15.\,1.\,1934 München), \emph{Schriftsteller, Kritiker}|pwv} wächſt beinahe täglich. Es iſt ein
                  \strikeout{\textcolor{gray}{m}}{ }\strikeout{unl} unlauterer Menſch. Man braucht ihn nur \label{K_L02616-10v}\edtext{in der »Zeit\orgindex{Zeit. Wiener Wochenschrift@Die Zeit. Wiener Wochenschrift|pw}«}{\lemma{\textnormal{\emph{in der »Zeit«}}}\Cendnote{\textnormal{Die \emph{Zeit}\pwindex{Zeit. Wiener Wochenschrift@\emph{Die Zeit. Wiener Wochenschrift}|pwk} erschien ab 6.\,10.\,1894 wöchentlich,
                  wodurch Goldmann\pwindex{Goldmann, Paul 31.\,1.\,1865 Breslau – 25.\,9.\,1935 Wien@\textsc{Goldmann, Paul} (31.\,1.\,1865 Breslau – 25.\,9.\,1935 Wien), \emph{Schriftsteller, Journalist}|pwk} die ersten drei Hefte
                  gekannt haben dürfte.}}}\label{K_L02616-10} zu beobachten. Alles,
               was von \textsc{Kanner\pwindex{Kanner, Heinrich 9.\,11.\,1864 Galați – 15.\,2.\,1930 Wien@\textsc{Kanner, Heinrich} (9.\,11.\,1864 Galați – 15.\,2.\,1930 Wien), \emph{Herausgeber, Publizist}|pw}} kommt, iſt nämlich, originell und muthig. In \label{K_L02616-11v}\edtext{\textsc{Bahrs\pwindex{Bahr, Hermann 19.\,7.\,1863 Linz – 15.\,1.\,1934 München@\textsc{Bahr, Hermann} (19.\,7.\,1863 Linz – 15.\,1.\,1934 München), \emph{Schriftsteller, Kritiker}|pw}} Reſſort}{\lemma{\textnormal{\emph{Bahrs Ressort}}}\Cendnote{\textnormal{Bahr\pwindex{Bahr, Hermann 19.\,7.\,1863 Linz – 15.\,1.\,1934 München@\textsc{Bahr, Hermann} (19.\,7.\,1863 Linz – 15.\,1.\,1934 München), \emph{Schriftsteller, Kritiker}|pwk} verantwortete den Kulturteil.}}}\label{K_L02616-11} gibt es nichts als berechnetes Laviren, verbunden mit
               frechem literariſchem Pontificiren. Socialpolitiſch und politiſch iſt die Revüe\orgindex{Zeit. Wiener Wochenschrift@Die Zeit. Wiener Wochenschrift|pwv} vorzüglich; literariſch
               finde ich{ }ſie talent- und \strikeout{int} intereſſelos redigirt;
               da gibt es nur einen \textsc{Bahr\pwindex{Bahr, Hermann 19.\,7.\,1863 Linz – 15.\,1.\,1934 München@\textsc{Bahr, Hermann} (19.\,7.\,1863 Linz – 15.\,1.\,1934 München), \emph{Schriftsteller, Kritiker}|pw}}, \strikeout{de\textcolor{gray}{r}} alles Andere iſt als Relief behandelt. \strikeout{D\textcolor{gray}{er}}{ }{\pb}Er wird das{ }ſchöne Unternehmen\orgindex{Zeit. Wiener Wochenschrift@Die Zeit. Wiener Wochenschrift|pwv}{ }ſchon umbringen.\pend
           
\pstart
           »\label{K_L02616-12v}\edtext{Sterben\pwindex{Schnitzler, Arthur 15.\,5.\,1862 Wien – 21.\,10.\,1931 ebd.@\textsc{Schnitzler, Arthur} (15.\,5.\,1862 Wien – 21.\,10.\,1931 ebd.), \emph{Schriftsteller, Mediziner}!Sterben. Novelle@\strich\emph{Sterben. Novelle}|pw}}{\lemma{\textnormal{\emph{Sterben}}}\Cendnote{\textnormal{Goldmann\pwindex{Goldmann, Paul 31.\,1.\,1865 Breslau – 25.\,9.\,1935 Wien@\textsc{Goldmann, Paul} (31.\,1.\,1865 Breslau – 25.\,9.\,1935 Wien), \emph{Schriftsteller, Journalist}|pwk} bezog sich auf den ersten Teil des
                  Erstdrucks von \emph{Sterben}\pwindex{Schnitzler, Arthur 15.\,5.\,1862 Wien – 21.\,10.\,1931 ebd.@\textsc{Schnitzler, Arthur} (15.\,5.\,1862 Wien – 21.\,10.\,1931 ebd.), \emph{Schriftsteller, Mediziner}!Sterben. Novelle@\strich\emph{Sterben. Novelle}|pwk}, der im Oktober-Heft der \emph{Neuen
                        Deutschen Rundschau}\pwindex{Neue Deutsche Rundschau@\emph{Neue Deutsche Rundschau}|pwk} enthalten war (Jg. 5, H. 10, S. 969–988).
                  Zwei weitere Teile folgten bis Dezember. Die
                  Buchausgabe erschien im November 1894, auf 1895 vordatiert. Die von Goldmann\pwindex{Goldmann, Paul 31.\,1.\,1865 Breslau – 25.\,9.\,1935 Wien@\textsc{Goldmann, Paul} (31.\,1.\,1865 Breslau – 25.\,9.\,1935 Wien), \emph{Schriftsteller, Journalist}|pwk} vorgeschlagenen Änderungen wurden nicht berücksichtigt.}}}\label{K_L02616-12}« habe ich geleſen. Es hat mich tief,
               tief ergriffen. Wenn Du wüßteſt, was für einen goldenen Reifeton Deine Kunſt jetzt
               hat! Dieſe klare und noble Einfachheit! Dieſe Gemüthstiefe! Und dieſer{ }ſcharfe
               Verſtand, der in des Lebens dunkelſte Gründe dringt! Soweit ich bisher urtheilen
               kann, iſt es eine große Leiſtung, wohl Deine größte biſher. Nur Eines meine ich – ich
               weiß nicht, ob der Eindruck bis zum Schluß vorhalten wird – Du{ }ſollteſt aus der
               verfluchten Illegitimtät heraus. Das bringt etwas {\pb}Halbes hinein. Wenn das Mädl{ }ſeine Frau wäre,{ }ſo \strikeout{\textcolor{gray}{×}} wäre es noch ergreifender, noch allgemein menſchlicher. Ich glaube, daß es
               nichts{ }ſchaden könnte, bis nach Weihnachten mit dem Buche\pwindex{Schnitzler, Arthur 15.\,5.\,1862 Wien – 21.\,10.\,1931 ebd.@\textsc{Schnitzler, Arthur} (15.\,5.\,1862 Wien – 21.\,10.\,1931 ebd.), \emph{Schriftsteller, Mediziner}!Sterben. Novelle@\strich\emph{Sterben. Novelle}|pwv} zu warten. Vor
                  Weihnachten kommſt Du in den großen Schwall hinein, nachher tritt es
               beſſer hervor.\pend
           
\pstart
           Das \label{K_L02616-13v}\edtext{Stück\pwindex{Triesch, Friedrich Gustav 16.\,6.\,1845 Wien – 24.\,5.\,1907 ebd.@\textsc{Triesch, Friedrich Gustav} (16.\,6.\,1845 Wien – 24.\,5.\,1907 ebd.), \emph{Schriftsteller}!Ottilie. Schauspiel in vier Akten@\strich\emph{Ottilie. Schauspiel in vier Akten}|pwv} von \textsc{Triesch\pwindex{Triesch, Friedrich Gustav 16.\,6.\,1845 Wien – 24.\,5.\,1907 ebd.@\textsc{Triesch, Friedrich Gustav} (16.\,6.\,1845 Wien – 24.\,5.\,1907 ebd.), \emph{Schriftsteller}|pw}}}{\lemma{\textnormal{\emph{Stück von Triesch}}}\Cendnote{\textnormal{Am 16. 10. 1894 hatte am \emph{Raimund-Theater}\orgindex{Raimund-Theater@Raimund-Theater|pwk} die Premiere von \emph{Ottilie. Schauspiel in vier Akten}\pwindex{Triesch, Friedrich Gustav 16.\,6.\,1845 Wien – 24.\,5.\,1907 ebd.@\textsc{Triesch, Friedrich Gustav} (16.\,6.\,1845 Wien – 24.\,5.\,1907 ebd.), \emph{Schriftsteller}!Ottilie. Schauspiel in vier Akten@\strich\emph{Ottilie. Schauspiel in vier Akten}|pwk} stattgefunden. Schnitzler hatte die Aufführung besucht und
                  im \emph{Tagebuch}\pwindex{Schnitzler, Arthur 15.\,5.\,1862 Wien – 21.\,10.\,1931 ebd.@\textsc{Schnitzler, Arthur} (15.\,5.\,1862 Wien – 21.\,10.\,1931 ebd.), \emph{Schriftsteller, Mediziner}!Tagebuch@\strich\emph{Tagebuch}|pwk} notiert:
                  »bodenlos«.}}}\label{K_L02616-13} hat \textsc{Bahr\pwindex{Bahr, Hermann 19.\,7.\,1863 Linz – 15.\,1.\,1934 München@\textsc{Bahr, Hermann} (19.\,7.\,1863 Linz – 15.\,1.\,1934 München), \emph{Schriftsteller, Kritiker}|pw}} in der »Zeit\orgindex{Zeit. Wiener Wochenschrift@Die Zeit. Wiener Wochenschrift|pw}« feſt \label{K_L02616-14v}\edtext{gelobt\pwindex{Bahr, Hermann 19.\,7.\,1863 Linz – 15.\,1.\,1934 München@\textsc{Bahr, Hermann} (19.\,7.\,1863 Linz – 15.\,1.\,1934 München), \emph{Schriftsteller, Kritiker}!Kunst und Leben. [Raimundtheater. Ottilie von Triesch]@\strich\emph{Kunst und Leben. [Raimundtheater. Ottilie von Triesch]}|pwv}}{\lemma{\textnormal{\emph{gelobt}}}\Cendnote{\textnormal{H. B.\pwindex{Bahr, Hermann 19.\,7.\,1863 Linz – 15.\,1.\,1934 München@\textsc{Bahr, Hermann} (19.\,7.\,1863 Linz – 15.\,1.\,1934 München), \emph{Schriftsteller, Kritiker}|pwk}: \emph{Kunst und Leben. [Raimundtheater]}\pwindex{Bahr, Hermann 19.\,7.\,1863 Linz – 15.\,1.\,1934 München@\textsc{Bahr, Hermann} (19.\,7.\,1863 Linz – 15.\,1.\,1934 München), \emph{Schriftsteller, Kritiker}!Kunst und Leben. [Raimundtheater. Ottilie von Triesch]@\strich\emph{Kunst und Leben. [Raimundtheater. Ottilie von Triesch]}|pwk}. In: \emph{Die Zeit}\orgindex{Zeit. Wiener Wochenschrift@Die Zeit. Wiener Wochenschrift|pwk}, Jg. 1, H. 3, 20.\,10.\,1894,
                     S. 44. Vgl. A. S.: \emph{Tagebuch}, 7. 10. 1894.}}}\label{K_L02616-14}. Verhält{ }ſich eben mit der \label{K_L02616-15v}\edtext{\textsc{Clique}}{\lemma{\textnormal{\emph{Clique}}}\Cendnote{\textnormal{Goldmann\pwindex{Goldmann, Paul 31.\,1.\,1865 Breslau – 25.\,9.\,1935 Wien@\textsc{Goldmann, Paul} (31.\,1.\,1865 Breslau – 25.\,9.\,1935 Wien), \emph{Schriftsteller, Journalist}|pwk} bezieht sich abschätzig auf die
                  momentanen Akteure der Theater, nicht unbedingt auf eine spezifische
                  Gruppe von namentlich bekannten Personen.}}}\label{K_L02616-15}, der Herr. Pfui, pfui!\pend
           
\pstart
           Das »\textsc{Journal\pwindex{Le Journal@\emph{Le Journal}|pw}}« iſt,{ }ſeit Du es abonnirt haſt, recht{ }ſchwach. Es iſt, als geſchähe es
               abſichtlich. Vergiß’ nicht, {\pb}die Humoriſten zu
               leſen: \textsc{Allais\pwindex{Allais, Alphonse 20.\,10.\,1854 Honfleur – 28.\,10.\,1905 Paris@\textsc{Allais, Alphonse} (20.\,10.\,1854 Honfleur – 28.\,10.\,1905 Paris), \emph{Schriftsteller}|pw}}, \textsc{Bill Sharp\pwindex{Veber, Pierre 15.\,5.\,1869 Paris – 20.\,8.\,1942 ebd.@\textsc{Veber, Pierre} (15.\,5.\,1869 Paris – 20.\,8.\,1942 ebd.), \emph{Schriftsteller}|pw}}{ }\textsc{etc.} Des Letzteren »\label{K_L02616-16v}\edtext{Briefe an \textsc{Allais\pwindex{Allais, Alphonse 20.\,10.\,1854 Honfleur – 28.\,10.\,1905 Paris@\textsc{Allais, Alphonse} (20.\,10.\,1854 Honfleur – 28.\,10.\,1905 Paris), \emph{Schriftsteller}|pw}} über die Zündhölzchen\pwindex{Veber, Pierre 15.\,5.\,1869 Paris – 20.\,8.\,1942 ebd.@\textsc{Veber, Pierre} (15.\,5.\,1869 Paris – 20.\,8.\,1942 ebd.), \emph{Schriftsteller}!Lettre à M. Alphonse Allais sur les omnibus@\strich\emph{Lettre à M. Alphonse Allais sur les omnibus}|pw}}{\lemma{\textnormal{\emph{Briefe … Zündhölzchen}}}\Cendnote{\textnormal{Bill Sharp [ = Pierre Veber]\pwindex{Veber, Pierre 15.\,5.\,1869 Paris – 20.\,8.\,1942 ebd.@\textsc{Veber, Pierre} (15.\,5.\,1869 Paris – 20.\,8.\,1942 ebd.), \emph{Schriftsteller}|pwk}: \emph{Lettre à M. Alphonse Allais sur les
                        allumettes}\pwindex{Veber, Pierre 15.\,5.\,1869 Paris – 20.\,8.\,1942 ebd.@\textsc{Veber, Pierre} (15.\,5.\,1869 Paris – 20.\,8.\,1942 ebd.), \emph{Schriftsteller}!Lettre à M. Alphonse Allais sur les omnibus@\strich\emph{Lettre à M. Alphonse Allais sur les omnibus}|pwk}. In: \emph{Le Journal}\pwindex{Le Journal@\emph{Le Journal}|pwk}, Jg. 3,
                     Nr. 732, 29.\,9.\,1894, S. 1–2.}}}\label{K_L02616-16} und \label{K_L02616-17v}\edtext{über die Omnibuſſe\pwindex{Veber, Pierre 15.\,5.\,1869 Paris – 20.\,8.\,1942 ebd.@\textsc{Veber, Pierre} (15.\,5.\,1869 Paris – 20.\,8.\,1942 ebd.), \emph{Schriftsteller}!Lettre à M. Alphonse Allais sur les omnibus@\strich\emph{Lettre à M. Alphonse Allais sur les omnibus}|pw}«}{\lemma{\textnormal{\emph{über die Omnibusse«}}}\Cendnote{\textnormal{Bill Sharp [ = Pierre Veber]\pwindex{Veber, Pierre 15.\,5.\,1869 Paris – 20.\,8.\,1942 ebd.@\textsc{Veber, Pierre} (15.\,5.\,1869 Paris – 20.\,8.\,1942 ebd.), \emph{Schriftsteller}|pwk}: \emph{Lettre à M. Alphonse Allais sur les
                        omnibus}\pwindex{Veber, Pierre 15.\,5.\,1869 Paris – 20.\,8.\,1942 ebd.@\textsc{Veber, Pierre} (15.\,5.\,1869 Paris – 20.\,8.\,1942 ebd.), \emph{Schriftsteller}!Lettre à M. Alphonse Allais sur les omnibus@\strich\emph{Lettre à M. Alphonse Allais sur les omnibus}|pwk}. In: \emph{Le Journal}\pwindex{Le Journal@\emph{Le Journal}|pwk}, Jg. 3,
                     Nr. 751, 18.\,10.\,1894, S. 1–2.}}}\label{K_L02616-17} waren köſtlich. Freilich muß man ein wenig \label{T_L02616-1v}\edtext{Lokalkenntniß}{\lemma{\textnormal{\emph{Lokalkenntniß}}}\Cendnote{\textnormal{Goldmann\pwindex{Goldmann, Paul 31.\,1.\,1865 Breslau – 25.\,9.\,1935 Wien@\textsc{Goldmann, Paul} (31.\,1.\,1865 Breslau – 25.\,9.\,1935 Wien), \emph{Schriftsteller, Journalist}|pwk} schrieb: »Lokalkenntniß
                           zu«.}}}\label{T_L02616-1} haben, um das in{ }ſeiner ganzen Größe zu würdigen. Du haſt
                  \textsc{30 fr. 40 ct.} bei mir gut. Was{ }ſoll damit geſchehen? Ein
               paar Sachen habe ich für Dich geſammelt, wie ich Dir verſprochen. Es iſt nicht viel
               Bedeutendes drunter, aber allerlei {\pb}Kurioſes. Es iſt
               natürlich lächerlich, daß ich Dir zugemuthet habe, über das Alles mir zu berichten.
               Schreib’ mir nur ein allgemeines Wort, obs Dir{ }ſo recht iſt. Dann fahre ich fort.\pend
           
\pstart
           \label{K_L02616-18v}\edtext{Das mit dem \strikeout{ſeh}{ }ſechzehnjährigen Mädel\pwindex{Singer, Else 25.\,6.\,1878 Wien – 1943?@\textsc{Singer, Else} (25.\,6.\,1878 Wien – 1943?), \emph{Schriftstellerin, Sprachlehrerin}|pwv}}{\lemma{\textnormal{\emph{Das … Mädel}}}\Cendnote{\textnormal{Goldmann\pwindex{Goldmann, Paul 31.\,1.\,1865 Breslau – 25.\,9.\,1935 Wien@\textsc{Goldmann, Paul} (31.\,1.\,1865 Breslau – 25.\,9.\,1935 Wien), \emph{Schriftsteller, Journalist}|pwk} bezieht sich wohl auf Else Singer\pwindex{Singer, Else 25.\,6.\,1878 Wien – 1943?@\textsc{Singer, Else} (25.\,6.\,1878 Wien – 1943?), \emph{Schriftstellerin, Sprachlehrerin}|pwk}, mit der Schnitzler zu dieser Zeit viel Kontakt hatte (vgl. \emph{Tagebuch}\pwindex{Schnitzler, Arthur 15.\,5.\,1862 Wien – 21.\,10.\,1931 ebd.@\textsc{Schnitzler, Arthur} (15.\,5.\,1862 Wien – 21.\,10.\,1931 ebd.), \emph{Schriftsteller, Mediziner}!Tagebuch@\strich\emph{Tagebuch}|pwk}).}}}\label{K_L02616-18} hat mich gerührt.
               Liebes, kleines Ding!\pend
           
\pstart
           Die Frau \textsc{Andreas\pwindex{Andreas-Salomé, Lou 12.\,2.\,1861 Sankt Petersburg – 5.\,2.\,1937 Göttingen@\textsc{Andreas-Salomé, Lou} (12.\,2.\,1861 Sankt Petersburg – 5.\,2.\,1937 Göttingen), \emph{Schriftstellerin}|pw}}{ }ſprach ich hier noch einmal. Ich glaube,{ }ſie hat mich lieb gehabt. Nun iſt{ }ſie
               im Groll von mir geſchieden, weil ich{ }ſie zurückgeſtoßen habe. Und allſogleich{ }ſtellt
                  {\pb}ſich bei mir die Reue ein. Aber{ }ſie hat
               unwiderruflich mit mir gebrochen.\pend
           
\pstart
           Grüß’ mir \textsc{Richard\pwindex{Beer-Hofmann, Richard 11.\,7.\,1866 Wien – 26.\,9.\,1945 New York City@\textsc{Beer-Hofmann, Richard} (11.\,7.\,1866 Wien – 26.\,9.\,1945 New York City), \emph{Schriftsteller}|pw}} und \textsc{Loris\pwindex{Hofmannsthal, Hugo von 1.\,2.\,1874 Wien – 15.\,7.\,1929 Rodaun@\textsc{Hofmannsthal, Hugo von} (1.\,2.\,1874 Wien – 15.\,7.\,1929 Rodaun), \emph{Schriftsteller}|pw}}.\pend
           
\pstart
           \textsc{Herzl\pwindex{Herzl, Theodor 2.\,5.\,1860 Budapest – 3.\,7.\,1904 Edlach@\textsc{Herzl, Theodor} (2.\,5.\,1860 Budapest – 3.\,7.\,1904 Edlach), \emph{Schriftsteller, Journalist}|pw}}{ }ſehe ich kaum. Bin wieder ganz mit ihm auseinander. Er war{ }ſeit{ }ſeiner
               Rückkunft einmal bei mir, um mir anzuzeigen, daß \label{K_L02616-19v}\edtext{»\textsc{Tabarin\pwindex{Herzl, Theodor 2.\,5.\,1860 Budapest – 3.\,7.\,1904 Edlach@\textsc{Herzl, Theodor} (2.\,5.\,1860 Budapest – 3.\,7.\,1904 Edlach), \emph{Schriftsteller, Journalist}!Tabarin. Schauspiel in einem Act. Frei nach Catulle Mendès@\strich\emph{Tabarin. Schauspiel in einem Act. Frei nach Catulle Mendès}|pw}}« werde aufgeführt}{\lemma{\textnormal{\emph{»Tabarin« werde aufgeführt}}}\Cendnote{\textnormal{Der Einakter \emph{Tabarin}\pwindex{Herzl, Theodor 2.\,5.\,1860 Budapest – 3.\,7.\,1904 Edlach@\textsc{Herzl, Theodor} (2.\,5.\,1860 Budapest – 3.\,7.\,1904 Edlach), \emph{Schriftsteller, Journalist}!Tabarin. Schauspiel in einem Act. Frei nach Catulle Mendès@\strich\emph{Tabarin. Schauspiel in einem Act. Frei nach Catulle Mendès}|pwk} war seit Anfang Oktober
                  als Novität für die Saison 1894/1895 am \emph{Burgtheater}\orgindex{Burgtheater@Burgtheater|pwk} angekündigt. Die Premiere fand am 2.\,5.\,1895 statt, Schnitzler besuchte die Aufführung am 7. 5. 1895.}}}\label{K_L02616-19} werden, was
               mich neidiſch machen{ }ſollte. Seitdem verkehrt er täglich mit \textsc{Feldmann\pwindex{Feldmann, Siegmund @\textsc{Feldmann, Siegmund}, \emph{Schriftsteller, Journalist}|pw}} und läßt{ }ſich bei mir nicht mehr{ }ſehen. So habe ich ihn auch links liegen
               laſſen.\pend
           
\pstart
           Aber Deinen Gruß und {\pb}Dein Lob habe ich ihm
               ausgerichtet. Das hat ihn{ }ſehr gefreut.\pend
           
\pstart
           Meine Sachen{ }ſammeln? Ich weiß genau, daß{ }ſie es nicht werth{ }ſind. Aber mir thut es
               wohl, wenn Du mir das Gegentheil{ }ſchreibſt. Natürlich werde ich{ }ſie nicht{ }ſammeln.\pend
           
\pstart
           Bitte, mich Frl. \textsc{Sandrock\pwindex{Sandrock, Adele 19.\,8.\,1863 Rotterdam – 30.\,8.\,1937 Berlin@\textsc{Sandrock, Adele} (19.\,8.\,1863 Rotterdam – 30.\,8.\,1937 Berlin), \emph{Schauspielerin}|pw}} zu empfehlen.\pend
           
\pstart
           Bitte, mich Deiner Frau Mutter\pwindex{Schnitzler, Louise 8.\,7.\,1840 Kőszeg – 9.\,9.\,1911 Wien@\textsc{Schnitzler, Louise} (8.\,7.\,1840 Kőszeg – 9.\,9.\,1911 Wien)|pwv} recht herzlich zu empfehlen. Bitte, Deinen Bruder\pwindex{Schnitzler, Julius 13.\,7.\,1865 Wien – 29.\,6.\,1939 ebd.@\textsc{Schnitzler, Julius} (13.\,7.\,1865 Wien – 29.\,6.\,1939 ebd.), \emph{Chirurg}|pwv} und Deine entzückende kleine Schwägerin\pwindex{Schnitzler, Helene 16.\,7.\,1871 Budapest – September 1941 Atlantischer Ozean@\textsc{Schnitzler, Helene} (16.\,7.\,1871 Budapest – September 1941 Atlantischer Ozean)|pwv} recht herzlich von
               mir zu grüßen.\pend
           
\pstart
           Und{ }ſei Du{ }ſelbſt von Herzen gegrüßt! Dein{\\[\baselineskip]}treuer \spacefill\mbox{Paul
                  Goldmann}\pend
           \leftskip=0em{}
\pstart
           \noindent{}\label{T_L02616-2v}\edtext{\textsc{Salten\pwindex{Salten, Felix 6.\,9.\,1869 Budapest – 8.\,10.\,1945 Zürich@\textsc{Salten, Felix} (6.\,9.\,1869 Budapest – 8.\,10.\,1945 Zürich), \emph{Schriftsteller, Journalist, Chefredakteur}|pw}} laſſe ich zu{ }ſeiner \label{K_L02616-20v}\edtext{neuen
                     Stellung\orgindex{Wiener Allgemeine Zeitung@Wiener Allgemeine Zeitung|pwv}}{\lemma{\textnormal{\emph{neuen
                     Stellung}}}\Cendnote{\textnormal{Felix Salten\pwindex{Salten, Felix 6.\,9.\,1869 Budapest – 8.\,10.\,1945 Zürich@\textsc{Salten, Felix} (6.\,9.\,1869 Budapest – 8.\,10.\,1945 Zürich), \emph{Schriftsteller, Journalist, Chefredakteur}|pwk} war seit Oktober 1894 bei der \emph{Wiener Allgemeinen Zeitung}\orgindex{Wiener Allgemeine Zeitung@Wiener Allgemeine Zeitung|pwk} engagiert.}}}\label{K_L02616-20} gratuliren}{\lemma{\textnormal{\emph{Salten … gratuliren}}}\Cendnote{\textnormal{entlang des linken
                              Blattrandes}}}\label{T_L02616-2}.\pend
           
\pstart
           {\pb}\label{T_L02616-3v}\edtext{Wenn Du vom Burgtheater\orgindex{Burgtheater@Burgtheater|pw} Antwort haſt, erbitte ich \uline{umgehende} Mittheilung}{\lemma{\textnormal{\emph{Wenn … Mittheilung}}}\Cendnote{\textnormal{am
                     oberen Rand der ersten Seite, verkehrt zum Text}}}\label{T_L02616-3}.\pend
           \selectlanguage{ngerman}\endnumbering\briefempfaengerindex{Schnitzler, Arthur@\textsc{Schnitzler, Arthur}!zzzGoldmann, Paul@\emph{von Paul Goldmann}!1894-10-251@{25. 10. [1894]}|)be}\mylabel{L02616h}  \newcommand{\dateiname}{L02616}\newcommand{\titel}{Paul Goldmann an Arthur Schnitzler, 25. 10. [1894]}\newcommand{\editorInnen}{Martin Anton Müller und Laura Untner}%% latex-leseansicht-abspann.tex
%% Abspann für die Leseansicht.
%% Der Schalter \ifkorrekturansicht ist bereits durch den Vorspann gesetzt.

%% latex-abspann.tex
%% Gemeinsamer Abspann für Korrekturansicht und Leseansicht.
%% Setzt den Schalter \ifkorrekturansicht voraus (gesetzt in den
%% einbindenden Dateien latex-korrekturansicht-abspann.tex bzw.
%% latex-leseansicht-abspann.tex).
%% ---------------------------------------------------------------

\normalsize

% Das esempio-Environment wird nur in der Leseansicht benötigt
\ifkorrekturansicht\else
\newenvironment{esempio}[3]%
{
    \vspace{1.5ex}
    \rlap{\underline{#1}}
    \par
    \setlength{\parindent}{0cm}
    \nopagebreak
    \leftskip=#2cm
    \rightskip=#3cm
}
{
    \par
}
\fi

\doendnotes{C}
\bigskip
\vfill

\clearpage

\footnotesize

\ifkorrekturansicht
  \lohead{\textsc{register}}
\fi

% theindex-Environment neu definieren ohne reledmac
\makeatletter
\renewenvironment{theindex}{%
  \ifkorrekturansicht
    \section*{\indexname}%
  \else
    \subsubsection*{Index der erwähnten Entitäten}%
  \fi
  \setlength{\parindent}{0pt}%
  \setlength{\parskip}{0pt plus 0.3pt}%
  \let\item\@idxitem
}{%
  \ifkorrekturansicht\clearpage\fi
}
\makeatother

\IfFileExists{\jobname-pw.ind}{\input{\jobname-pw.ind}}{}

% Quellenangabe nur in der Leseansicht
\ifkorrekturansicht\else
% Fallback-Definitionen, falls die .tex-Datei \titel etc. nicht gesetzt hat
\providecommand{\titel}{}
\providecommand{\editorInnen}{}
\providecommand{\dateiname}{\jobname}

\vspace{3cm}

\vfill

\footnotesize
\textsc{Quelle}: \titel. Herausgegeben von {\editorInnen}. In: \emph{Arthur Schnitzler: Briefwechsel mit Autorinnen und Autoren}.
 Digitale Edition, https://schnitzler-briefe.acdh.oeaw.ac.at/{\dateiname}.html (Stand \today)
\fi

\end{document}


