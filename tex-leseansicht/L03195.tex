%% latex-leseansicht-vorspann.tex
%% Vorspann für die Leseansicht.
%% Lädt die gemeinsame Datei latex-vorspann.tex mit nicht gesetztem Schalter.

\newif\ifkorrekturansicht
\korrekturansichtfalse

\input{../tex-inputs/latex-vorspann}

\begin{center}
            \textcolor{red}{ENTWURF, NICHT FERTIG KORRIGIERT}
                      \end{center}
            
         
         \renewcommand{\erwaehntePersonen}{Personen: Richard Beer-Hofmann, Emerich von Bukovics, Theodor Herzl, Heinrich Kanner, Fedor Mamroth, Olga Schnitzler}
         \renewcommand{\erwaehnteInstitutionen}{Institutionen: Burgtheater, Deutsches Theater Berlin, Die Zeit, Volkstheater}
         \renewcommand{\erwaehnteOrte}{Orte: Berlin, Dessauer Straße, Frankfurt am Main, Frankreich, Rodaun, Wien}
         \renewcommand{\erwaehnteWerke}{Werke: Berliner Theater. (»Lebendige Stunden« von Arthur Schnitzler.), Der Schleier der Beatrice. Schauspiel in fünf Akten, Lebendige Stunden. Vier Einakter, Neue Freie Presse, Tagebuch}
               \section[ Paul Goldmann an Arthur Schnitzler, 25. 1. {[}1902{]}]{ Paul Goldmann an Arthur Schnitzler, 25. 1. {[}1902{]}}\nopagebreak\mylabel{v}\rehead{ }\begin{ledgroupsized}[t]{13cm}\normalsize\beginnumbering \toendnotes[C]{\smallbreak\pagebreak[2]} \Standort{DLA, A:Schnitzler, HS.NZ85.1.3172.}
\physDesc{Brief, 2 Blätter, 7 Seiten
\newline{}Handschrift: blaue Tinte, deutsche Kurrent
\newline{}Schnitzler: 1) mit Bleistift das Jahr »{[}1{]}90\textcolor{gray}{2}« vermerkt  2) mit rotem Buntstift drei Unterstreichungen}\toendnotes[C]{\smallbreak}\pstart
           \noindent{}\raggedleft{}{\pb}\textcolor{gray}{\textbf{DESSAUERSTRASSE 19}}\oindex{Dessauer Strasse@\textbf{Dessauer Straße}|pw}\pend
           \pstart
           Berlin\oindex{Berlin@\textbf{Berlin}|pw}, 25. Januar.\pend
           \pstart\center{}Mein lieber Freund,\pend\pstart
           Wir wollen die Debatte ſchließen. Nur Eines noch: Ich habe Dir nicht vorgeworfen, daß
               Du von Dir mehr erfüllt biſt, als von mir. Es iſt ſelbſtverſtändlich, daß Jeder von
               ſich mehr erfüllt iſt als von einem Anderen. Ich meine nur, daß \strikeout{\textcolor{gray}{ich in Deinen}} weil Du von Dir bedeutend mehr erfüllt biſt, als es die Regel iſt, der Platz,
               den ich in Deinem Denken und Empfinden einnehme, auch bedeutend geringer iſt, als ein
               Freund vom Freunde in der Regel beanſpruchen kann. Das iſt eine Nuancen-Frage; und
               über dieſe läßt ſich nicht discutiren. Wir wollen auch nicht mehr darüber reden,
               weder ſchriftlich, \strikeout{noch mündlich}.\pend
           \pstart
           Was Du mir über \strikeout{D} mein \label{K_L03195-1v}\edtext{Feuilleton\pwindex{Goldmann, Paul 31.01.1865 – 25.09.1935@\textsc{Goldmann, Paul} (31.01.1865 – 25.09.1935), \emph{Schriftsteller, Journalist}!Berliner Theater. (»Lebendige Stunden« von Arthur Schnitzler.)1902-01-22@\strich\emph{Berliner Theater. (»Lebendige Stunden« von Arthur Schnitzler.)} {[}1902-01-22{]}|pwv}}{\lemma{\textnormal{\emph{Feuilleton}}}\Cendnote{\textnormal{Paul Goldmann\pwindex{Goldmann, Paul 31.01.1865 – 25.09.1935@\textsc{Goldmann, Paul} (31.01.1865 – 25.09.1935), \emph{Schriftsteller, Journalist}|pwk}: \emph{Berliner Theater. (»Lebendige Stunden« von Arthur
                        Schnitzler.)}\pwindex{Goldmann, Paul 31.01.1865 – 25.09.1935@\textsc{Goldmann, Paul} (31.01.1865 – 25.09.1935), \emph{Schriftsteller, Journalist}!Berliner Theater. (»Lebendige Stunden« von Arthur Schnitzler.)1902-01-22@\strich\emph{Berliner Theater. (»Lebendige Stunden« von Arthur Schnitzler.)} {[}1902-01-22{]}|pwk}. In: \emph{Neue Freie
                        Presse}\pwindex{Neue Freie Presse1864 – 1939@\emph{Neue Freie Presse} {[}1864 – 1939{]}|pwk}, Nr. 13438, 22. 1. 1902,
                     Morgenblatt, S. 1–4.}}}\label{K_L03195-1h} ſchreibſt, könnte eine neue große Debatte
               hervorrufen. Auch hier wieder thuſt Du mir {\pb}Unrecht
               vom Anfang bis zum Ende. Die Mühe, die ich mir genommen, Deine Dichtungen\pwindex{Schnitzler, Arthur 15.05.1862 – 21.10.1931@\textsc{Schnitzler, Arthur} (15.05.1862 – 21.10.1931), \emph{Schriftsteller, Mediziner}!Lebendige Stunden. Vier Einakter1901-12-23@\strich\emph{Lebendige Stunden. Vier Einakter} {[}1901-12-23{]}|pwv} bis in die feinſten \textsc{Nuancen} zu durchdenken und zu ergründen, ſiehſt Du nicht.
               Wenigſtens erwähnſt Du ſie mit keinem Worte. Hingegen ſchreibſt Du mir, ich ſei
               »liebenswürdig« gegen Dich geweſen. Mein lieber Freund, ich bin nicht liebenswürdig
               gegen Dich geweſen\strikeout{,} und weigere mich entſchieden,
               jemals liebenswürdig gegen Dich zu ſein. Ich habe Dir das Höchſte \strikeout{\textcolor{gray}{in}} gegeben, was ich Dir geben kann: Wahrheit. Ich bilde mir natürlich nicht ein,
               die objektive Wahrheit gefunden zu haben; aber die ſubjektive Wahrheit, wie ich ſie
               empfunden habe, habe ich ausgedrückt. Von meinem Standpunkte aus iſt in dieſer Kritik\pwindex{Goldmann, Paul 31.01.1865 – 25.09.1935@\textsc{Goldmann, Paul} (31.01.1865 – 25.09.1935), \emph{Schriftsteller, Journalist}!Berliner Theater. (»Lebendige Stunden« von Arthur Schnitzler.)1902-01-22@\strich\emph{Berliner Theater. (»Lebendige Stunden« von Arthur Schnitzler.)} {[}1902-01-22{]}|pwv} jedes Wort wahr. Auch
               der Satz, den Du hervorhebſt, iſt wahr. Ich habe Dich als {\pb}Dramatiker zu kritiſiren gehabt, nicht als
               Novelliſten. Ich habe von Dir das große dramatiſche Werk verlangt, das Du meiner
               feſten Überzeugung nach leiſten kannſt, – das Du allein leiſten kannſt von allen
               deutſchen Schriftſtellern Deiner Generation. Der »\textsc{Schleier der Beatrice\pwindex{Schnitzler, Arthur 15.05.1862 – 21.10.1931@\textsc{Schnitzler, Arthur} (15.05.1862 – 21.10.1931), \emph{Schriftsteller, Mediziner}!Schleier der Beatrice. Schauspiel in fuenf Akten1900-12-01@\strich\emph{Der Schleier der Beatrice. Schauspiel in fünf Akten} {[}1900-12-01{]}|pw}}« iſt dieſes große Werk nicht. Trotz alles Starken und Glänzenden, das dieſes
                  Drama\pwindex{Schnitzler, Arthur 15.05.1862 – 21.10.1931@\textsc{Schnitzler, Arthur} (15.05.1862 – 21.10.1931), \emph{Schriftsteller, Mediziner}!Schleier der Beatrice. Schauspiel in fuenf Akten1900-12-01@\strich\emph{Der Schleier der Beatrice. Schauspiel in fünf Akten} {[}1900-12-01{]}|pwv} enthält, iſt es ein
               großes Drama nicht geworden, weil auch hier \strikeout{ein} die
               Liebſchaft als Haupthema behandelt iſt und alles Andere nur als Epiſode in der
               Liebſchaft erſcheint. Auch auf dieſes Drama\pwindex{Schnitzler, Arthur 15.05.1862 – 21.10.1931@\textsc{Schnitzler, Arthur} (15.05.1862 – 21.10.1931), \emph{Schriftsteller, Mediziner}!Schleier der Beatrice. Schauspiel in fuenf Akten1900-12-01@\strich\emph{Der Schleier der Beatrice. Schauspiel in fünf Akten} {[}1900-12-01{]}|pwv} paßt durchaus der \label{K_L03195-5v}\edtext{fran\oindex{Frankreich@\textbf{Frankreich}|pwv}zöſiſche Satz}{\lemma{\textnormal{\emph{franzöſiſche Satz}}}\Cendnote{\textnormal{»Arthur Schnitzler’s Dichtungen
                     handeln fast immer zunächst von einer Liebschaft und von allem Andern nebenbei.
                     Man könnte diese Kunst unter Variirung einer bekannten Erklärung des Wesens der
                     Kunst definiren, als: ›\begin{otherlanguage}{french}Un coin de la vie, vu à travers
                        une amourette\end{otherlanguage}‹ [eine Seite des Lebens, aus der Perspektive einer
                     Romanze betrachtet]. Diese Art der Darstellung jedoch gibt ein unrichtiges
                     Bild. Denn die Liebe, obwol sie eine nicht unwichtige Angelegenheit des Daseins
                     bildet, ist doch immer nur eine Episode im Leben, während in Arthur
                     Schnitzler’s Schriften umgekehrt das Leben oft als eine Episode in der Liebe
                     erscheint. (S. 4)«}}}\label{K_L03195-5h}, den ich niedergeſchrieben habe, – auf dieſes
                  Drama\pwindex{Schnitzler, Arthur 15.05.1862 – 21.10.1931@\textsc{Schnitzler, Arthur} (15.05.1862 – 21.10.1931), \emph{Schriftsteller, Mediziner}!Schleier der Beatrice. Schauspiel in fuenf Akten1900-12-01@\strich\emph{Der Schleier der Beatrice. Schauspiel in fünf Akten} {[}1900-12-01{]}|pwv}{ }paßt er erſt recht, weil Du hier auf dem Wege zum
               höchſten warſt und \substVorne{}\textsuperscript{weil}\substDazwischen{}weil\substHinten{} Dich dieſe einſeitige Betrachtungsweiſe, die immer und {\pb}vor Allem nach \strikeout{ne}
               neuen Spezialfällen der Liebe Ausblick hält, gerade hier verhindert hat, das Höchſte
               zu erreichen. Ich hätte das auch in meinem Feuilleton\pwindex{Goldmann, Paul 31.01.1865 – 25.09.1935@\textsc{Goldmann, Paul} (31.01.1865 – 25.09.1935), \emph{Schriftsteller, Journalist}!Berliner Theater. (»Lebendige Stunden« von Arthur Schnitzler.)1902-01-22@\strich\emph{Berliner Theater. (»Lebendige Stunden« von Arthur Schnitzler.)} {[}1902-01-22{]}|pwv}\strikeout{mehr} ausgeführt, wenn ich auf der gewählten Spalte
               noch Platz gehabt hätte zu dieſer Ausführung. Wenn Dich demnächſt wieder Leute
               fragen, ob ich Deine Werke der letzten Jahre denn nicht kenne, ſo bitte ich Dich,
               ihnen das zu ſagen.\pend
           \pstart
           Von \textsc{Herzl\pwindex{Herzl, Theodor 1860-05-02 – 1904-07-03@\textsc{Herzl, Theodor} (1860-05-02 – 1904-07-03), \emph{Schriftsteller, Journalist}|pw}} erhielt ich einen Brief, den ich Dir nicht ſchicken kann, weil ich ihn der
               Curioſität halber meinem Onkel\pwindex{Mamroth, Fedor 21.02.1851 – 25.06.1907@\textsc{Mamroth, Fedor} (21.02.1851 – 25.06.1907), \emph{Journalist, Kritiker}|pwv} geſandt habe. Ich citire aus dem Gedächtniß folgenden Satz: »Die
               Grenzlinie (in meinem Feuilleton\pwindex{Goldmann, Paul 31.01.1865 – 25.09.1935@\textsc{Goldmann, Paul} (31.01.1865 – 25.09.1935), \emph{Schriftsteller, Journalist}!Berliner Theater. (»Lebendige Stunden« von Arthur Schnitzler.)1902-01-22@\strich\emph{Berliner Theater. (»Lebendige Stunden« von Arthur Schnitzler.)} {[}1902-01-22{]}|pwv} über »Lebendige Stunden\pwindex{Schnitzler, Arthur 15.05.1862 – 21.10.1931@\textsc{Schnitzler, Arthur} (15.05.1862 – 21.10.1931), \emph{Schriftsteller, Mediziner}!Lebendige Stunden. Vier Einakter1901-12-23@\strich\emph{Lebendige Stunden. Vier Einakter} {[}1901-12-23{]}|pw}«)
               zwiſchen aufrichtiger und geſchriebener Meinung {\pb}habe ich ſehr wohl bemerkt; \substVorne{}\textsuperscript{aber}\substDazwischen{}aber\substHinten{} (wenn irgendeine Unaufrichtigkeit entſchuldbar iſt, ſo iſt es die durch eine
               alte Freundſchaft gebotene.« Ich habe dieſen unſinnigen Vorwurf der Unaufrichtigkeit
                  \introOben{}in einem
                  Briefe\introOben{}
               mit Entſchiedenheit zurückgewieſen.\pend
           \pstart
           Zu meiner Freude ſehe ich »Lebendige Stunden\pwindex{Schnitzler, Arthur 15.05.1862 – 21.10.1931@\textsc{Schnitzler, Arthur} (15.05.1862 – 21.10.1931), \emph{Schriftsteller, Mediziner}!Lebendige Stunden. Vier Einakter1901-12-23@\strich\emph{Lebendige Stunden. Vier Einakter} {[}1901-12-23{]}|pw}«
               ſtändig auf dem Theater\orgindex{Deutsches Theater Berlin@Deutsches Theater Berlin|pw}zettel. Ich hoffe, daß
               dies einen Kaſſenerfolg bedeutet. Haben \label{K_L03195-123v}\edtext{andere deutſche Bühnen}{\lemma{\textnormal{\emph{andere deutſche Bühnen}}}\Cendnote{\textnormal{Im Herbst 1901 hatte das Wien\oindex{Wien@\textbf{Wien}|pwk}er \emph{Volkstheater}\orgindex{Volkstheater@Volkstheater|pwk} unter der Leitung von Emerich von Bukovics\pwindex{Bukovics, Emerich von 28.02.1844 – 04.07.1905@\textsc{Bukovics, Emerich von} (28.02.1844 – 04.07.1905), \emph{Journalist, Theaterleiter}|pwk} die Stücke\pwindex{Schnitzler, Arthur 15.05.1862 – 21.10.1931@\textsc{Schnitzler, Arthur} (15.05.1862 – 21.10.1931), \emph{Schriftsteller, Mediziner}!Lebendige Stunden. Vier Einakter1901-12-23@\strich\emph{Lebendige Stunden. Vier Einakter} {[}1901-12-23{]}|pwkv} angenommen. Die Premiere fand am 14. 3. 1903
                  statt.}}}\label{K_L03195-123h} die Stücke\pwindex{Schnitzler, Arthur 15.05.1862 – 21.10.1931@\textsc{Schnitzler, Arthur} (15.05.1862 – 21.10.1931), \emph{Schriftsteller, Mediziner}!Lebendige Stunden. Vier Einakter1901-12-23@\strich\emph{Lebendige Stunden. Vier Einakter} {[}1901-12-23{]}|pwv}
               bereits erworben? Wie hat ſich das \label{K_L03195-43v}\edtext{Burgtheater\orgindex{Burgtheater@Burgtheater|pw}}{\lemma{\textnormal{\emph{Burgtheater}}}\Cendnote{\textnormal{Schnitzler\pwindex{Schnitzler, Arthur 15.05.1862 – 21.10.1931@\textsc{Schnitzler, Arthur} (15.05.1862 – 21.10.1931), \emph{Schriftsteller, Mediziner}|pwk} notierte noch am 28. 11. 1901 im \emph{Tagebuch}\pwindex{Schnitzler, Arthur 15.05.1862 – 21.10.1931@\textsc{Schnitzler, Arthur} (15.05.1862 – 21.10.1931), \emph{Schriftsteller, Mediziner}!Tagebuch1981 – 2000@\strich\emph{Tagebuch} {[}1981 – 2000{]}|pwk}: »Ich merke deutlich dass man
                     weiss das Burgth.\orgindex{Burgtheater@Burgtheater|pw} ist mir
                     verschlossen.―« Siehe auch Hermann Bahr an Arthur Schnitzler, 9. 1. 1902.}}}\label{K_L03195-43h} verhalten?\pend
           \pstart
           Daß \label{K_L03195-23v}\edtext{\textsc{Olga\pwindex{Schnitzler, Olga 17.01.1882 – 13.01.1970@\textsc{Schnitzler, Olga} (17.01.1882 – 13.01.1970), \emph{Schauspielerin, Sängerin}|pw}} immer noch bettlägerig}{\lemma{\textnormal{\emph{Olga … bettlägerig}}}\Cendnote{\textnormal{siehe Paul Goldmann an Arthur Schnitzler, 16. 1. [1902]}}}\label{K_L03195-23h} iſt, bedaure ich unendlich. Ich bitte Dich, ſie herzlichſt zu grüßen. Kann
               ich ihr vielleicht irgend Etwas zu leſen ſchicken? {\pb}An \textsc{Richard\pwindex{Beer-Hofmann, Richard 1866-07-11 – 1945-09-26@\textsc{Beer-Hofmann, Richard} (1866-07-11 – 1945-09-26), \emph{Schriftsteller}|pw}} ſchreibe ich, ſobald ich kann. Bitte grüße ihn inzwiſchen vielmals. Dieſe
                  \label{K_L03195-44v}\edtext{Krankheit}{\lemma{\textnormal{\emph{Krankheit}}}\Cendnote{\textnormal{siehe A. S.: \emph{Tagebuch}, 19. 1. 1902}}}\label{K_L03195-44h} kommt wahrſcheinlich von der Feuchtigkeit in dem verfluchten Neſt\oindex{Rodaun@\textbf{Rodaun}|pwv}, in das er ohne jede\textcolor{gray}{r}
               Nothwendigkeit hat hinausziehen müſſen. Hoffentlich hat er keine Schmerzen
               gelitten.\pend
           \pstart
           Ich ſelbſt habe wieder einmal eine bittere Enttäuſchung \label{K_L03195-334v}\edtext{\substVorne{}\textsuperscript{e\textcolor{gray}{rlebt}.}{\allowbreak}\substDazwischen{}erlebt.\substHinten{}}{\lemma{\textnormal{\emph{erlebt.erlebt.}}}\Cendnote{\textnormal{in der Vorlage ist der Punkt nicht
                  eindeutig durchgestrichen, jedoch war das wohl intendiert}}}\label{K_L03195-334h}{ }\textsc{Kanner\pwindex{Kanner, Heinrich 09.11.1864 – 15.02.1930@\textsc{Kanner, Heinrich} (09.11.1864 – 15.02.1930), \emph{Herausgeber, Publizist}|pw}} war hier\oindex{Berlin@\textbf{Berlin}|pwv}, um für ſein
               neues \label{K_L03195-2121v}\edtext{Blatt\orgindex{Zeit@Die Zeit|pwv}}{\lemma{\textnormal{\emph{Blatt}}}\Cendnote{\textnormal{siehe Paul Goldmann an Arthur Schnitzler, 16. 1. [1902]}}}\label{K_L03195-2121h} Engagements \strikeout{zu} abzuſchließen. Wenn es
               irgendwo Jemanden gibt, den er verſuchen \uline{müßte}, zu
               gewinnen, ſo bin \uline{ich} es. Ich war erſtaunt, daß er mir
               keinen Antrag machte. Jetzt hat er in {\pb}Frankfurt\oindex{Frankfurt am Main@\textbf{Frankfurt am Main}|pw} meinem Onkel\pwindex{Mamroth, Fedor 21.02.1851 – 25.06.1907@\textsc{Mamroth, Fedor} (21.02.1851 – 25.06.1907), \emph{Journalist, Kritiker}|pwv} geſagt,er wolle mich nicht haben, weil
               in dem neuen Unternehmen\orgindex{Zeit@Die Zeit|pwv} ihn
               mein Peſſimismus zu ſehr bedrücken würde. \strikeout{Der} Dieſes
               Urtheil iſt blödſinnig. Aber es läßt ſich nichts dagegen machen. Ich aber ſage nur:
               Wenn ſelbſt die einzigen Leute, \strikeout{mit denen ich} zu
               denen ich aus geiſtigen und moraliſchen Gründen gehöre, mich nicht haben wollen, –
               wozu habe ich dann mein Leben lang gearbeitet, und welche Zukunft habe ich zu
               erwarten?\pend
           \pstart
           Sei vielmals und von Herzen gegrüßt! Dein {\\[\baselineskip]}\spacefill\mbox{Paul Goldmn}\pend
           \leftskip=0em{}
         
         \endnumbering\mylabel{h}\end{ledgroupsized}\begin{anhang}\end{anhang}\newcommand{\dateiname}{L03195}\newcommand{\titel}{Paul Goldmann an Arthur Schnitzler, 25. 1. [1902]}\newcommand{\editorInnen}{Martin Anton Müller und Laura Untner}%% latex-leseansicht-abspann.tex
%% Abspann für die Leseansicht.
%% Der Schalter \ifkorrekturansicht ist bereits durch den Vorspann gesetzt.

%% latex-abspann.tex
%% Gemeinsamer Abspann für Korrekturansicht und Leseansicht.
%% Setzt den Schalter \ifkorrekturansicht voraus (gesetzt in den
%% einbindenden Dateien latex-korrekturansicht-abspann.tex bzw.
%% latex-leseansicht-abspann.tex).
%% ---------------------------------------------------------------

\normalsize

% Das esempio-Environment wird nur in der Leseansicht benötigt
\ifkorrekturansicht\else
\newenvironment{esempio}[3]%
{
    \vspace{1.5ex}
    \rlap{\underline{#1}}
    \par
    \setlength{\parindent}{0cm}
    \nopagebreak
    \leftskip=#2cm
    \rightskip=#3cm
}
{
    \par
}
\fi

\doendnotes{C}
\bigskip
\vfill

\clearpage

\footnotesize

\ifkorrekturansicht
  \lohead{\textsc{register}}
\fi

% theindex-Environment neu definieren ohne reledmac
\makeatletter
\renewenvironment{theindex}{%
  \ifkorrekturansicht
    \section*{\indexname}%
  \else
    \subsubsection*{Index der erwähnten Entitäten}%
  \fi
  \setlength{\parindent}{0pt}%
  \setlength{\parskip}{0pt plus 0.3pt}%
  \let\item\@idxitem
}{%
  \ifkorrekturansicht\clearpage\fi
}
\makeatother

\IfFileExists{\jobname-pw.ind}{\input{\jobname-pw.ind}}{}

% Quellenangabe nur in der Leseansicht
\ifkorrekturansicht\else
% Fallback-Definitionen, falls die .tex-Datei \titel etc. nicht gesetzt hat
\providecommand{\titel}{}
\providecommand{\editorInnen}{}
\providecommand{\dateiname}{\jobname}

\vspace{3cm}

\vfill

\footnotesize
\textsc{Quelle}: \titel. Herausgegeben von {\editorInnen}. In: \emph{Arthur Schnitzler: Briefwechsel mit Autorinnen und Autoren}.
 Digitale Edition, https://schnitzler-briefe.acdh.oeaw.ac.at/{\dateiname}.html (Stand \today)
\fi

\end{document}


      