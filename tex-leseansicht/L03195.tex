%% latex-korrekturansicht-vorspann.tex
%% Vorspann für die Korrekturansicht.
%% Lädt die gemeinsame Datei latex-vorspann.tex mit gesetztem Schalter.

\newif\ifkorrekturansicht
\korrekturansichttrue

\input{../tex-inputs/latex-vorspann}


\section[ Paul Goldmann an Arthur Schnitzler, 25. 1. {[}1902{]}]{L03195 Paul Goldmann an Arthur Schnitzler, 25. 1. {[}1902{]}}
\nopagebreak\mylabel{L03195v}
\rehead{ }\normalsize\beginnumbering\briefempfaengerindex{Schnitzler, Arthur@\textsc{Schnitzler, Arthur}!zzzGoldmann, Paul@\emph{von Paul Goldmann}!1902-01-252@{25. 1. {[}1902{]}}|(be}
\toendnotes[C]{\smallbreak\pagebreak[2]}\Standort{DLA, A:Schnitzler, HS.NZ85.1.3172.}
\physDesc{Brief, 2 Blätter, 7 Seiten, 4597 Zeichen
\newline{}Handschrift: blaue Tinte, deutsche Kurrent
\newline{}Schnitzler: 1) mit Bleistift das Jahr »90\textcolor{gray}{2}« vermerkt  2) mit rotem Buntstift drei Unterstreichungen}\toendnotes[C]{\smallbreak}
\pstart
           \raggedleft{}{\pb}\textcolor{gray}{\textbf{DESSAUERSTRASSE 19}}\oindex{Dessauer Strasse@\textbf{Dessauer Straße}, \emph{Straße (K.STR)}|pw}\pend
           
\pstart
           Berlin\oindex{Berlin@\textbf{Berlin}, \emph{P.PPLC}|pw}, 25. Januar.\pend
           
\pstart\center{}Mein lieber Freund,\pend\vspace{0.5em}
\pstart
           Wir wollen die Debatte ſchließen. Nur Eines noch: Ich habe Dir nicht vorgeworfen, daß
               Du von Dir mehr erfüllt biſt, als von mir. Es iſt ſelbſtverſtändlich, daß Jeder von
               ſich mehr erfüllt iſt als von einem Anderen. Ich meine nur, daß \strikeout{\textcolor{gray}{ich in Deinen}} weil Du von Dir bedeutend mehr erfüllt biſt, als es die Regel iſt, der Platz,
               den ich in Deinem Denken und Empfinden einnehme, auch bedeutend geringer iſt, als ein
               Freund vom Freunde in der Regel beanſpruchen kann. Das iſt eine \textsc{Nuancen}-Frage; und über dieſe läßt ſich nicht discutiren. Wir wollen auch
               nicht mehr darüber reden, weder ſchriftlich, \uline{noch
                  mündlich}.\pend
           
\pstart
           Was Du mir über \strikeout{D} mein \label{K_L03195-1v}\edtext{Feuilleton\pwindex{Berliner Theater. (»Lebendige Stunden« von Arthur Schnitzler.)@\emph{Berliner Theater. (»Lebendige Stunden« von Arthur Schnitzler.)}|pwv}}{\lemma{\textnormal{\emph{Feuilleton}}}\Cendnote{\textnormal{Paul Goldmann\pwindex{Goldmann, Paul 31.01.1865 – 25.09.1935@\textsc{Goldmann, Paul} (31.01.1865 – 25.09.1935), \emph{Schriftsteller/Schriftstellerin, Journalist/Journalistin}|pwk}: \emph{Berliner Theater. (»Lebendige Stunden« von Arthur
                        Schnitzler)}\pwindex{Berliner Theater. (»Lebendige Stunden« von Arthur Schnitzler.)@\emph{Berliner Theater. (»Lebendige Stunden« von Arthur Schnitzler.)}|pwk}. In: \emph{Neue Freie
                        Presse}\pwindex{Neue Freie Presse@\emph{Neue Freie Presse}|pwk}, Nr. 13.438, 22. 1. 1902,
                     Morgenblatt, S. 1–4.}}}\label{K_L03195-1} ſchreibſt, könnte eine neue große Debatte
               hervorrufen. Auch hier wieder thuſt Du mir {\pb}Unrecht
               vom Anfang bis zum Ende. Die Mühe, die ich mir genommen, Deine Dichtungen\pwindex{Lebendige Stunden. Vier Einakter@\emph{Lebendige Stunden. Vier Einakter}|pwv} bis in die feinſten \textsc{Nuancen} zu durchdenken und zu ergründen, ſiehſt Du nicht.
               Wenigſtens erwähnſt Du ſie mit keinem Worte. Hingegen ſchreibſt Du mir, ich ſei
               »liebenswürdig« gegen Dich geweſen. Mein lieber Freund, ich bin nicht liebenswürdig
               gegen Dich geweſen\strikeout{,} und weigere mich entſchieden,
               jemals liebenswürdig gegen Dich zu ſein. Ich habe Dir das Höchſte \strikeout{\textcolor{gray}{in}} gegeben, was ich Dir geben kann: Wahrheit. Ich bilde mir natürlich nicht ein,
               die objektive Wahrheit gefunden zu haben; aber die ſubjektive Wahrheit, wie ich ſie
               empfunden habe, habe ich ausgedrückt. Von meinem Standpunkte aus iſt in dieſer Kritik\pwindex{Berliner Theater. (»Lebendige Stunden« von Arthur Schnitzler.)@\emph{Berliner Theater. (»Lebendige Stunden« von Arthur Schnitzler.)}|pwv} jedes Wort \substVorne{}\textsuperscript{\textcolor{gray}{wa}h}\substDazwischen{}wahr\substHinten{}. Auch der Satz, den Du hervorhebſt, iſt wahr. Ich habe Dich als {\pb}Dramatiker zu kritiſiren gehabt, nicht als
               Novelliſten. Ich habe von Dir das große dramatiſche Werk verlangt, das Du meiner
               feſten Überzeugung nach leiſten kannſt, – das Du allein leiſten kannſt von allen
               deutſchen Schriftſtellern Deiner Generation. Der »Schleier der \textsc{Beatrice}\pwindex{Schleier der Beatrice. Schauspiel in fuenf Akten@\emph{Der Schleier der Beatrice. Schauspiel in fünf Akten}|pw}« iſt dieſes große Werk nicht. Trotz alles Starken und Glänzenden, das dieſes
                  Drama\pwindex{Schleier der Beatrice. Schauspiel in fuenf Akten@\emph{Der Schleier der Beatrice. Schauspiel in fünf Akten}|pwv} enthält, iſt es ein
               großes Drama nicht geworden, weil auch hier \strikeout{ein} die
               Liebſchaft als Hauptthema behandelt iſt und alles Andere nur als Epiſode in der
               Liebſchaft erſcheint. Auch auf dieſes Drama\pwindex{Schleier der Beatrice. Schauspiel in fuenf Akten@\emph{Der Schleier der Beatrice. Schauspiel in fünf Akten}|pwv} paßt durchaus der \label{K_L03195-2v}\edtext{fran\oindex{Frankreich@\textbf{Frankreich}, \emph{A.PCLI}|pwv}zöſiſche Satz}{\lemma{\textnormal{\emph{franzöſiſche Satz}}}\Cendnote{\textnormal{»Arthur Schnitzler’s
                        Dichtungen handeln fast immer zunächst von einer Liebschaft und von allem
                        Andern nebenbei. Man könnte diese Kunst unter Variirung einer bekannten
                        Erklärung des Wesens der Kunst definiren, als: ›\begin{otherlanguage}{french}\textsc{Un coin de la vie, vu à travers une amourette}\end{otherlanguage}‹. Diese Art der Darstellung jedoch gibt ein unrichtiges Bild.
                        Denn die Liebe, obwol sie eine nicht unwichtige Angelegenheit des Daseins
                        bildet, ist doch immer nur eine Episode im Leben, während in Arthur
                        Schnitzler’s Schriften umgekehrt das Leben oft als eine Episode in der Liebe
                        erscheint.\pwindex{Berliner Theater. (»Lebendige Stunden« von Arthur Schnitzler.)@\emph{Berliner Theater. (»Lebendige Stunden« von Arthur Schnitzler.)}|pwv}« (S. 4) Der französische Satz kann übersetzt werden als:
                  ›Eine Ecke des Lebens, aus der Perspektive einer Liebelei betrachtet‹. Es ist ein
                  verfremdetes Zitat im Nachklang von Émile
                     Zola\pwindex{Zola, Emile 02.04.1840 – 29.09.1902@\textsc{Zola, Émile} (02.04.1840 – 29.09.1902), \emph{Schriftsteller/Schriftstellerin, Journalist/Journalistin}|pwk}, bei dem es lautet: »\begin{otherlanguage}{french}Un œuvre d’art est un coin de la création vu à travers
                        un tempérament.\end{otherlanguage}« (Ein Kunstwerk ist eine Ecke der Schöpfung, vermittels einer Stimmung
                  wahrgenommen.) }}}\label{K_L03195-2}, den ich niedergeſchrieben habe, – auf dieſes Drama\pwindex{Schleier der Beatrice. Schauspiel in fuenf Akten@\emph{Der Schleier der Beatrice. Schauspiel in fünf Akten}|pwv}{ }paßt er erſt recht, weil Du hier auf dem Wege zum
               Höchſten warſt und \substVorne{}\textsuperscript{weil}\substDazwischen{}weil\substHinten{} Dich dieſe einſeitige Betrachtungsweiſe, die immer und {\pb}vor Allem nach \strikeout{ne}
               neuen Spezialfällen der Liebe Ausblick hält, gerade hier verhindert hat, das Höchſte
               zu erreichen. Ich hätte das auch in meinem Feuilleton\pwindex{Berliner Theater. (»Lebendige Stunden« von Arthur Schnitzler.)@\emph{Berliner Theater. (»Lebendige Stunden« von Arthur Schnitzler.)}|pwv}{ }\strikeout{mehr} ausgeführt, wenn ich auf der zwölften Spalte
               noch Platz gehabt hätte zu dieſer Ausführung. Wenn Dich demnächſt wieder Leute
               fragen, ob ich Deine Werke der letzten Jahre denn nicht kenne, ſo bitte ich Dich,
               ihnen das zu ſagen.\pend
           
\pstart
           Von \textsc{Herzl\pwindex{Herzl, Theodor 1860-05-02 – 1904-07-03@\textsc{Herzl, Theodor} (1860-05-02 – 1904-07-03), \emph{Schriftsteller/Schriftstellerin, Journalist/Journalistin}|pw}} erhielt ich einen Brief, den ich Dir nicht ſchicken kann, weil ich ihn der
               Curioſität halber meinem Onkel\pwindex{Mamroth, Fedor 21.02.1851 – 25.06.1907@\textsc{Mamroth, Fedor} (21.02.1851 – 25.06.1907), \emph{Journalist/Journalistin, Kritiker/Kritikerin}|pwv} geſandt habe. Ich citire aus dem Gedächtniß folgenden Satz: »Die
               Grenzlinie (in meinem Feuilleton\pwindex{Berliner Theater. (»Lebendige Stunden« von Arthur Schnitzler.)@\emph{Berliner Theater. (»Lebendige Stunden« von Arthur Schnitzler.)}|pwv} über »Lebendige Stunden\pwindex{Lebendige Stunden. Vier Einakter@\emph{Lebendige Stunden. Vier Einakter}|pw}«)
               zwiſchen aufrichtiger und geſchriebener Meinung {\pb}habe ich ſehr wohl bemerkt; \substVorne{}\textsuperscript{aber}\substDazwischen{}aber\substHinten{} (wenn irgendeine Unaufrichtigkeit entſchuldbar iſt, ſo iſt es die durch eine
               alte Freundſchaft gebotene.« Ich habe dieſen unſinnigen Vorwurf der Unaufrichtigkeit
                  \introOben{}in einem
                  Briefe\introOben{}
               mit Entſchiedenheit zurückgewieſen.\pend
           
\pstart
           Zu meiner Freude ſehe ich »Lebendige Stunden\pwindex{Lebendige Stunden. Vier Einakter@\emph{Lebendige Stunden. Vier Einakter}|pw}«
               ſtändig auf dem Theater\orgindex{Deutsches Theater Berlin@Deutsches Theater Berlin|pw}zettel. Ich hoffe, daß
               dies einen Kaſſenerfolg bedeutet. Haben \label{K_L03195-3v}\edtext{andere deutſche Bühnen}{\lemma{\textnormal{\emph{andere deutſche Bühnen}}}\Cendnote{\textnormal{Im Herbst 1901 hatte das Wien\oindex{Wien@\textbf{Wien}, \emph{A.ADM2}|pwk}er \emph{Volkstheater}\orgindex{Volkstheater@Volkstheater|pwk} unter der Leitung von Emerich von Bukovics\pwindex{Bukovics, Emerich von 28.02.1844 – 04.07.1905@\textsc{Bukovics, Emerich von} (28.02.1844 – 04.07.1905), \emph{Journalist/Journalistin, Theaterleiter/Theaterleiterin}|pwk} die Stücke\pwindex{Lebendige Stunden. Vier Einakter@\emph{Lebendige Stunden. Vier Einakter}|pwkv} angenommen, die Premiere fand aber erst am 14. 3. 1903 statt.
                  Siehe auch Paul Goldmann an Arthur Schnitzler, 13. 12. [1901].}}}\label{K_L03195-3} die Stücke\pwindex{Lebendige Stunden. Vier Einakter@\emph{Lebendige Stunden. Vier Einakter}|pwv} bereits erworben? Wie
               hat ſich das \label{K_L03195-4v}\edtext{Burgtheater\orgindex{Burgtheater@Burgtheater|pw}}{\lemma{\textnormal{\emph{Burgtheater}}}\Cendnote{\textnormal{Schnitzler hat zum 28. 11. 1901 im \emph{Tagebuch}\pwindex{Tagebuch@\emph{Tagebuch}|pwk} notiert: »Ich merke deutlich dass man
                     weiss das Burgth.\orgindex{Burgtheater@Burgtheater|pw} ist mir
                     verschlossen. –« Das war eine Folge des öffentlich ausgetragenen Streits
                  um die unklare Annahme und spätere Zurückgabe von \emph{Der Schleier der Beatrice}\pwindex{Schleier der Beatrice. Schauspiel in fuenf Akten@\emph{Der Schleier der Beatrice. Schauspiel in fünf Akten}|pwk} durch Paul
                     Schlenther\pwindex{Schlenther, Paul 20.08.1854 – 30.04.1916@\textsc{Schlenther, Paul} (20.08.1854 – 30.04.1916), \emph{Schriftsteller/Schriftstellerin, Kritiker/Kritikerin, Theaterleiter/Theaterleiterin}|pwk}. Siehe auch Richard Beer-Hofmann an Arthur Schnitzler, 14. 9. 1900 und
                     Hermann Bahr an Arthur Schnitzler, 9. 1. 1902.}}}\label{K_L03195-4} verhalten?\pend
           
\pstart
           Daß \label{K_L03195-5v}\edtext{\textsc{Olga\pwindex{Schnitzler, Olga 17.01.1882 – 13.01.1970@\textsc{Schnitzler, Olga} (17.01.1882 – 13.01.1970), \emph{Schauspieler/Schauspielerin, Sänger/Sängerin}|pw}} immer noch bettlägerig}{\lemma{\textnormal{\emph{Olga … bettlägerig}}}\Cendnote{\textnormal{Siehe Paul Goldmann an Arthur Schnitzler, 16. 1. [1902].
               }}}\label{K_L03195-5} iſt, bedaure ich unendlich. Ich bitte Dich, ſie herzlichſt zu grüßen. Kann
               ich ihr vielleicht irgend Etwas zu leſen ſchicken?\pend
           
\pstart
           {\pb}An \textsc{Richard\pwindex{Beer-Hofmann, Richard 1866-07-11 – 1945-09-26@\textsc{Beer-Hofmann, Richard} (1866-07-11 – 1945-09-26), \emph{Schriftsteller/Schriftstellerin}|pw}} ſchreibe ich, ſobald ich kann. Bitte grüße ihn inzwiſchen vielmals. Dieſe
                  \label{K_L03195-6v}\edtext{Krankheit}{\lemma{\textnormal{\emph{Krankheit}}}\Cendnote{\textnormal{Siehe A. S.: \emph{Tagebuch}, 19. 1. 1902.
               }}}\label{K_L03195-6} kommt wahrſcheinlich von der Feuchtigkeit in dem verfluchten Neſt\oindex{Rodaun@\textbf{Rodaun}, \emph{A.ADM4}|pwv}, in das er ohne jede\textcolor{gray}{r}
               Nothwendigkeit hat hinausziehen müſſen. Hoffentlich hat er keine Schmerzen
               gelitten.\pend
           
\pstart
           Ich ſelbſt habe wieder einmal eine bittere Enttäuſchung \substVorne{}\textsuperscript{e\textcolor{gray}{rlebt}.}\substDazwischen{}erlebt.\substHinten{}{ }\textsc{Kanner\pwindex{Kanner, Heinrich 09.11.1864 – 15.02.1930@\textsc{Kanner, Heinrich} (09.11.1864 – 15.02.1930), \emph{Herausgeber/Herausgeberin, Publizist/Publizistin}|pw}} war hier\oindex{Berlin@\textbf{Berlin}, \emph{P.PPLC}|pwv}, um für ſein
               neues \label{K_L03195-7v}\edtext{Blatt\orgindex{Zeit@Die Zeit|pwv}}{\lemma{\textnormal{\emph{Blatt}}}\Cendnote{\textnormal{Siehe Paul Goldmann an Arthur Schnitzler, 16. 1. [1902].
               }}}\label{K_L03195-7} Engagements \strikeout{zu} abzuſchließen. Wenn es
               irgendwo Jemanden gibt, den er verſuchen \uline{müßte}, zu
               gewinnen, ſo bin \uline{ich} es. Ich war erſtaunt, daß er mir
               keinen Antrag machte. Jetzt hat er in {\pb}Frankfurt\oindex{Frankfurt am Main@\textbf{Frankfurt am Main}, \emph{P.PPLA3}|pw} meinem Onkel\pwindex{Mamroth, Fedor 21.02.1851 – 25.06.1907@\textsc{Mamroth, Fedor} (21.02.1851 – 25.06.1907), \emph{Journalist/Journalistin, Kritiker/Kritikerin}|pwv} geſagt, er wolle mich nicht haben,
               weil in dem neuen Unternehmen\orgindex{Zeit@Die Zeit|pwv}
               ihn mein Peſſimismus zu ſehr bedrücken würde. \strikeout{Der}
               Dieſes Urtheil iſt blödſinnig. Aber es läßt ſich nichts dagegen machen. Ich aber ſage
                  \textcolor{gray}{mi}r: Wenn ſelbſt die einzigen Leute, \strikeout{mit denen ich} zu denen ich aus geiſtigen und moraliſchen Gründen gehöre,
               mich nicht haben wollen, – wozu habe ich dann mein Leben lang gearbeitet, und welche
               Zukunft habe ich zu erwarten?\pend
           
\pstart
           Sei vielmals und von Herzen gegrüßt! Dein {\\[\baselineskip]}\spacefill\mbox{Paul Goldmn}\pend
           \leftskip=0em{}\selectlanguage{ngerman}\endnumbering\briefempfaengerindex{Schnitzler, Arthur@\textsc{Schnitzler, Arthur}!zzzGoldmann, Paul@\emph{von Paul Goldmann}!1902-01-252@{25. 1. {[}1902{]}}|)be}\mylabel{L03195h}  \normalsize

\doendnotes{C}
\bigskip
\vfill

\clearpage

\footnotesize

\lohead{\textsc{register}}

% Definiere theindex-Environment komplett neu ohne reledmac
\makeatletter
\renewenvironment{theindex}{%
  \section*{\indexname}%
  \setlength{\parindent}{0pt}%
  \setlength{\parskip}{0pt plus 0.3pt}%
  \let\item\@idxitem
}{%
  \clearpage
}
\makeatother

\IfFileExists{\jobname-pw.ind}{\input{\jobname-pw.ind}}{}

\end{document}

      