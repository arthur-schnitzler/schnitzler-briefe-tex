%% latex-korrekturansicht-vorspann.tex
%% Vorspann für die Korrekturansicht.
%% Lädt die gemeinsame Datei latex-vorspann.tex mit gesetztem Schalter.

\newif\ifkorrekturansicht
\korrekturansichttrue

\input{../tex-inputs/latex-vorspann}


\section[Arthur Schnitzler an Richard Beer-Hofmann, 10. 9. 1899]{L00972 Arthur Schnitzler an Richard Beer-Hofmann, 10. 9. 1899}
\nopagebreak\mylabel{L00972v}
\rehead{ }\normalsize\beginnumbering\briefempfaengerindex{Beer-Hofmann, Richard@\textsc{Beer-Hofmann, Richard}!zzzSchnitzler, Arthur@\emph{von Arthur Schnitzler}!1899-09-101@{10. 9. 1899}|(be}
\toendnotes[C]{\smallbreak\pagebreak[2]}\Standort{YCGL, MSS 31.}
\physDesc{Postkarte, 260 Zeichen
\newline{}Handschrift: Bleistift, deutsche Kurrent
\newline{}Versand: 1) Stempel: »\nobreak{}\oindex{Bad Ischl@\textbf{Bad Ischl}, \emph{P.PPL}|pwk}Ischl, 10. 9. 99, 7–8 V\nobreak{}«.   2) Stempel: »\nobreak{}\oindex{Brixen@\textbf{Brixen}, \emph{P.PPLA3}|pwk}Brixen, 11. 9. 99, 6.V\nobreak{}«. }\pstart{}{\pb}\textsc{Herrn Dr. Richard Beer-Hofmann}\pend{}\pstart{}\textsc{Vahrn}\oindex{Vahrn@\textbf{Vahrn}, \emph{P.PPLA3}|pw} bei \textsc{Brixen\oindex{Brixen@\textbf{Brixen}, \emph{P.PPLA3}|pw}}\pend{}\pstart{}\textsc{Tirol}\oindex{Tirol@\textbf{Tirol}, \emph{A.ADM1}|pw}\pend{}{\bigskip}\vspace{1em}
\pstart
           \noindent{}{\pb}lieber Richard, eben iſt ein Brief an Sie nach \textsc{Sachsenburg}\oindex{Sachsenburg@\textbf{Sachsenburg}, \emph{A.ADM3}|pw} abgegangen. Er enthält nichts wichtiges; nur d\substVorne{}\textsuperscript{\textcolor{gray}{en Umſtand}}\substDazwischen{}ie Bitte\substHinten{}, Sie möchten mir nach München\oindex{Muenchen@\textbf{München}, \emph{P.PPLA}|pw}{ }ſchreiben, wo ich Mittwoch u
                  Donnerstag{ }ſein will.\pend
           \pstart Herzlich Ihr \spacefill\mbox{A.}\pend{}\selectlanguage{ngerman}\endnumbering\briefempfaengerindex{Beer-Hofmann, Richard@\textsc{Beer-Hofmann, Richard}!zzzSchnitzler, Arthur@\emph{von Arthur Schnitzler}!1899-09-101@{10. 9. 1899}|)be}\mylabel{L00972h}  \normalsize

\doendnotes{C}
\bigskip
\vfill

\clearpage

\footnotesize

\lohead{\textsc{register}}

% Definiere theindex-Environment komplett neu ohne reledmac
\makeatletter
\renewenvironment{theindex}{%
  \section*{\indexname}%
  \setlength{\parindent}{0pt}%
  \setlength{\parskip}{0pt plus 0.3pt}%
  \let\item\@idxitem
}{%
  \clearpage
}
\makeatother

\IfFileExists{\jobname-pw.ind}{\input{\jobname-pw.ind}}{}

\end{document}

      