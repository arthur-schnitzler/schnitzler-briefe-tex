%% latex-leseansicht-vorspann.tex
%% Vorspann für die Leseansicht.
%% Lädt die gemeinsame Datei latex-vorspann.tex mit nicht gesetztem Schalter.

\newif\ifkorrekturansicht
\korrekturansichtfalse

\input{../tex-inputs/latex-vorspann}


         
         \newcommand{\erwaehntePersonen}{Personen: Hermann Bahr, Richard Beer-Hofmann, Ferdinand von Saar}
         \newcommand{\erwaehnteInstitutionen}{}
         \newcommand{\erwaehnteOrte}{Orte: Bad Ischl, Hotel und Pension Rudolfshöhe (Leopold Petter), Wien}
         \newcommand{\erwaehnteWerke}{Werke: Die Zeit. Wiener Wochenschrift, Herr Fridolin und sein Glück, Später Ruhm}
               \section[Arthur Schnitzler an Hermann Bahr, 17. 7. 1895]{ Arthur Schnitzler an Hermann Bahr, 17. 7. 1895}\nopagebreak\mylabel{v}\rehead{ }\begin{ledgroupsized}[t]{13cm}\normalsize\beginnumbering \toendnotes[C]{\smallbreak\pagebreak[2]} \Standort{TMW, HS AM 23324 Ba.}
\physDesc{Brief, 1 Blatt, 3 Seiten
\newline{}Handschrift: schwarze Tinte, deutsche Kurrent\newline{}Ordnung: Lochung }\buchAbdrucke{\weitereDrucke{1) \emph{17. 7. 1895.} In: Arthur Schnitzler: \emph{The Letters of Arthur Schnitzler to Hermann Bahr}. Edited, annotated, and with an introduction, by Donald G.
                        Daviau. Chapel Hill: \emph{The University of North Carolina Press} 1978, S. 58 (University of North Carolina studies in the Germanic languages
                        and literatures, 89).} \weitereDrucke{2) Hermann Bahr, Arthur Schnitzler: \emph{Briefwechsel, Aufzeichnungen, Dokumente (1891–1931)}. Hg. Kurt Ifkovits und Martin Anton Müller. Göttingen: \emph{Wallstein} 2018, S. 103.} }\toendnotes[C]{\smallbreak}\pstart{}{\pb}Lieber
                  Hermann, \pend\pstart
           hier iſt alſo die Novelle\pwindex{Schnitzler, Arthur 15.05.1862 – 21.10.1931@\textsc{Schnitzler, Arthur} (15.05.1862 – 21.10.1931), \emph{Schriftsteller, Mediziner}!Spaeter Ruhm2014@\strich\emph{Später Ruhm} {[}2014{]}|pwv}. Ich
               habe viel geſtrichen, fürchte aber noch i{\geminationm}er dß ſie zu
               lang iſt. In dieſem Falle hätte ich nichts dagegen, daſs ſie in kleinerm Drucke
               erſcheint. (Wie ſ. Z. \label{K_L00463_1v}\edtext{\textsc{Saar}\pwindex{Saar, Ferdinand von 30.09.1833 – 24.07.1906@\textsc{Saar, Ferdinand von} (30.09.1833 – 24.07.1906), \emph{Schriftsteller}|pw}\pwindex{Saar, Ferdinand von 30.09.1833 – 24.07.1906@\textsc{Saar, Ferdinand von} (30.09.1833 – 24.07.1906), \emph{Schriftsteller}!Herr Fridolin und sein Glueck1894-10-06 – 1894-11-03@\strich\emph{Herr Fridolin und sein Glück} {[}1894-10-06 – 1894-11-03{]}|pwv}}{\lemma{\textnormal{\emph{Saar}}}\Cendnote{\textnormal{Ferdinand von Saar\pwindex{Saar, Ferdinand von 30.09.1833 – 24.07.1906@\textsc{Saar, Ferdinand von} (30.09.1833 – 24.07.1906), \emph{Schriftsteller}|pwk}: \emph{Herr Fridolin und sein Glück}\pwindex{Saar, Ferdinand von 30.09.1833 – 24.07.1906@\textsc{Saar, Ferdinand von} (30.09.1833 – 24.07.1906), \emph{Schriftsteller}!Herr Fridolin und sein Glueck1894-10-06 – 1894-11-03@\strich\emph{Herr Fridolin und sein Glück} {[}1894-10-06 – 1894-11-03{]}|pwk}. In: \emph{Die Zeit}\pwindex{Zeit. Wiener Wochenschrift1894 – 1904@\emph{Die Zeit. Wiener Wochenschrift} {[}1894 – 1904{]}|pwk}, Bd. 1, Nr. 1, 6. 10. 1894 – Nr. 5,
                        3. 11. 1894 (5 Teile).}}}\label{K_L00463_1h}.) Findeſt Du noch Stellen,
               die Du für entbehrlich hältſt, ſo gib ſie mir vielleicht an, ſtreiche aber
               keinesfalls ſelbſt. {\pb}Auch wenn dir ein wirkſamerer Titel einfiele, ſo wäre mir das ſehr willko{\geminationm}en. –\pend
           \pstart
           Kannſt Du die Geſchichte nicht brauchen, ſo behalte das \textsc{Manuscr}. jedenfalls freundlichſt bei Dir, bis ich nach Wien\oindex{Wien@\textbf{Wien}|pw} zurückkehre. Nachrichten erbitte ich mir an untenſtehende
               Adreſſe. Richard\pwindex{Beer-Hofmann, Richard 1866-07-11 – 1945-09-26@\textsc{Beer-Hofmann, Richard} (1866-07-11 – 1945-09-26), \emph{Schriftsteller}|pw}{ }ſagt mir übrigens, dß Du bald
                  {\pb}wieder her ko{\geminationm}ſt, da ſprechen wir uns wohl, was mich ſehr freuen
               wird.\pend
           \pstart Herzliche Grüße von Deinem ergeb \spacefill\mbox{ArthSch}\pend{}\pstart
           1\substVorne{}\textsuperscript{6}\substDazwischen{}7\substHinten{}/7. 95{\\}\textsc{Ischl, Rudolfshöhe\oindex{Hotel und Pension Rudolfshoehe (Leopold Petter)@\textbf{Hotel und Pension Rudolfshöhe (Leopold Petter)}|pw}.}\pend
           
         
         \endnumbering\mylabel{h}\end{ledgroupsized}  \newcommand{\dateiname}{L00463}\newcommand{\titel}{Arthur Schnitzler an Hermann Bahr, 17. 7. 1895}\newcommand{\editorInnen}{ Kurt Ifkovits,  Martin Anton Müller}%% latex-leseansicht-abspann.tex
%% Abspann für die Leseansicht.
%% Der Schalter \ifkorrekturansicht ist bereits durch den Vorspann gesetzt.

%% latex-abspann.tex
%% Gemeinsamer Abspann für Korrekturansicht und Leseansicht.
%% Setzt den Schalter \ifkorrekturansicht voraus (gesetzt in den
%% einbindenden Dateien latex-korrekturansicht-abspann.tex bzw.
%% latex-leseansicht-abspann.tex).
%% ---------------------------------------------------------------

\normalsize

% Das esempio-Environment wird nur in der Leseansicht benötigt
\ifkorrekturansicht\else
\newenvironment{esempio}[3]%
{
    \vspace{1.5ex}
    \rlap{\underline{#1}}
    \par
    \setlength{\parindent}{0cm}
    \nopagebreak
    \leftskip=#2cm
    \rightskip=#3cm
}
{
    \par
}
\fi

\doendnotes{C}
\bigskip
\vfill

\clearpage

\footnotesize

\ifkorrekturansicht
  \lohead{\textsc{register}}
\fi

% theindex-Environment neu definieren ohne reledmac
\makeatletter
\renewenvironment{theindex}{%
  \ifkorrekturansicht
    \section*{\indexname}%
  \else
    \subsubsection*{Index der erwähnten Entitäten}%
  \fi
  \setlength{\parindent}{0pt}%
  \setlength{\parskip}{0pt plus 0.3pt}%
  \let\item\@idxitem
}{%
  \ifkorrekturansicht\clearpage\fi
}
\makeatother

\IfFileExists{\jobname-pw.ind}{\input{\jobname-pw.ind}}{}

% Quellenangabe nur in der Leseansicht
\ifkorrekturansicht\else
% Fallback-Definitionen, falls die .tex-Datei \titel etc. nicht gesetzt hat
\providecommand{\titel}{}
\providecommand{\editorInnen}{}
\providecommand{\dateiname}{\jobname}

\vspace{3cm}

\vfill

\footnotesize
\textsc{Quelle}: \titel. Herausgegeben von {\editorInnen}. In: \emph{Arthur Schnitzler: Briefwechsel mit Autorinnen und Autoren}.
 Digitale Edition, https://schnitzler-briefe.acdh.oeaw.ac.at/{\dateiname}.html (Stand \today)
\fi

\end{document}


      