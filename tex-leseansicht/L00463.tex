%% latex-korrekturansicht-vorspann.tex
%% Vorspann für die Korrekturansicht.
%% Lädt die gemeinsame Datei latex-vorspann.tex mit gesetztem Schalter.

\newif\ifkorrekturansicht
\korrekturansichttrue

\input{../tex-inputs/latex-vorspann}


\section[Arthur Schnitzler an Hermann Bahr, 17. 7. 1895]{L00463 Arthur Schnitzler an Hermann Bahr, 17. 7. 1895}
\nopagebreak\mylabel{L00463v}
\rehead{ }\normalsize\beginnumbering\briefempfaengerindex{Bahr, Hermann@\textsc{Bahr, Hermann}!zzzSchnitzler, Arthur@\emph{von Arthur Schnitzler}!1895-07-171@{17. 7. 1895}|(be}
\toendnotes[C]{\smallbreak\pagebreak[2]}\Standort{TMW, HS AM 23324 Ba.}
\physDesc{Brief, 1 Blatt, 3 Seiten, 759 Zeichen
\newline{}Handschrift: schwarze Tinte, deutsche Kurrent
\newline{}Ordnung: Lochung }
\buchAbdrucke{\weitereDrucke{1) Arthur Schnitzler: \emph{The Letters of Arthur Schnitzler to Hermann Bahr}. Chapel Hill: \emph{The University of North Carolina Press} 1978, S. 58.} \weitereDrucke{2) Hermann Bahr, Arthur Schnitzler: \emph{Briefwechsel, Aufzeichnungen, Dokumente (1891–1931)}. Göttingen: \emph{Wallstein} 2018, S. 103.} }\toendnotes[C]{\smallbreak}
\pstart{}{\pb}Lieber Hermann,
               \pend\vspace{0.5em}
\pstart
           hier iſt alſo die Novelle\pwindex{Spaeter Ruhm@\emph{Später Ruhm}|pwv}. Ich
               habe viel geſtrichen, fürchte aber noch i{\geminationm}er dß ſie zu
               lang iſt. In dieſem Falle hätte ich nichts dagegen, daſs ſie in kleinerm Drucke
               erſcheint. (Wie ſ. Z. \label{K_L00463-1v}\edtext{\textsc{Saar}\pwindex{Saar, Ferdinand von 30.09.1833 – 24.07.1906@\textsc{Saar, Ferdinand von} (30.09.1833 – 24.07.1906), \emph{Schriftsteller/Schriftstellerin}|pw}\pwindex{Herr Fridolin und sein Glueck@\emph{Herr Fridolin und sein Glück}|pwv}}{\lemma{\textnormal{\emph{Saar}}}\Cendnote{\textnormal{Ferdinand von Saar\pwindex{Saar, Ferdinand von 30.09.1833 – 24.07.1906@\textsc{Saar, Ferdinand von} (30.09.1833 – 24.07.1906), \emph{Schriftsteller/Schriftstellerin}|pwk}: \emph{Herr Fridolin und sein Glück}\pwindex{Herr Fridolin und sein Glueck@\emph{Herr Fridolin und sein Glück}|pwk}. In: \emph{Die Zeit}\pwindex{Zeit. Wiener Wochenschrift@\emph{Die Zeit. Wiener Wochenschrift}|pwk}, Bd. 1, Nr. 1, 6. 10. 1894 – Nr. 5,
                        3. 11. 1894 (5 Teile).}}}\label{K_L00463-1}.) Findeſt Du noch Stellen,
               die Du für entbehrlich hältſt, ſo gib ſie mir vielleicht an, ſtreiche aber
               keinesfalls ſelbſt. {\pb}Auch wenn dir ein wirkſamerer Titel einfiele, ſo wäre mir das ſehr willko{\geminationm}en. –\pend
           
\pstart
           Kannſt Du die Geſchichte nicht brauchen, ſo behalte das \textsc{Manuscr}. jedenfalls freundlichſt bei Dir, bis ich nach Wien\oindex{Wien@\textbf{Wien}, \emph{A.ADM2}|pw} zurückkehre. Nachrichten erbitte ich mir an untenſtehende
               Adreſſe. Richard\pwindex{Beer-Hofmann, Richard 1866-07-11 – 1945-09-26@\textsc{Beer-Hofmann, Richard} (1866-07-11 – 1945-09-26), \emph{Schriftsteller/Schriftstellerin}|pw}{ }ſagt mir übrigens, dß Du bald {\pb}wieder her ko{\geminationm}ſt, da ſprechen wir uns wohl, was mich ſehr freuen
               wird.\pend
           \pstart Herzliche Grüße von Deinem ergeb \spacefill\mbox{ArthSch}\pend{}
\pstart
           1\substVorne{}\textsuperscript{6}\substDazwischen{}7\substHinten{}/7. 95{\\}\textsc{Ischl, Rudolfshöhe\oindex{Hotel und Pension Rudolfshoehe (Leopold Petter)@\textbf{Hotel und Pension Rudolfshöhe (Leopold Petter)}, \emph{Hotel (K.HTL)}|pw}.}\pend
           \selectlanguage{ngerman}\endnumbering\briefempfaengerindex{Bahr, Hermann@\textsc{Bahr, Hermann}!zzzSchnitzler, Arthur@\emph{von Arthur Schnitzler}!1895-07-171@{17. 7. 1895}|)be}\mylabel{L00463h}  \normalsize

\doendnotes{C}
\bigskip
\vfill

\clearpage

\footnotesize

\lohead{\textsc{register}}

% Definiere theindex-Environment komplett neu ohne reledmac
\makeatletter
\renewenvironment{theindex}{%
  \section*{\indexname}%
  \setlength{\parindent}{0pt}%
  \setlength{\parskip}{0pt plus 0.3pt}%
  \let\item\@idxitem
}{%
  \clearpage
}
\makeatother

\IfFileExists{\jobname-pw.ind}{\input{\jobname-pw.ind}}{}

\end{document}

      