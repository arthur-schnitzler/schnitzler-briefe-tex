%% latex-leseansicht-vorspann.tex
%% Vorspann für die Leseansicht.
%% Lädt die gemeinsame Datei latex-vorspann.tex mit nicht gesetztem Schalter.

\newif\ifkorrekturansicht
\korrekturansichtfalse

\input{../tex-inputs/latex-vorspann}


\section[Therese Rie-Andro an Arthur Schnitzler, 22. 12. 1929]{L02567 Therese Rie-Andro an Arthur Schnitzler, 22. 12. 1929}
\nopagebreak\mylabel{L02567v}
\rehead{ }\normalsize\beginnumbering\briefempfaengerindex{Schnitzler, Arthur@\textsc{Schnitzler, Arthur}!zzzRie, Therese@\emph{von Therese Rie}!1929-12-222@{22. 12. 1929}|(be}
\toendnotes[C]{\smallbreak\pagebreak[2]}
\correspDesc{Versand  durch Therese Rie am 22. 12. 1929 in Wien
\newline{}Erhalt  durch Arthur Schnitzler im Zeitraum [22. 12. 1929 – 26. 12. 1929?] in Wien}\toendnotes[C]{\smallbreak}
\Standort{CUL, Schnitzler, B 82.}
\physDesc{Brief, 1 Blatt, 2 Seiten, 2424 Zeichen
\newline{}Schreibmaschine
\newline{}Handschrift: schwarze Tinte, lateinische Kurrent (\noindent{}marginale Korrekturen, Schlussformel und Unterschrift)
\newline{}Schnitzler: mit rotem Buntstift beschriftet: »\textsc{So{\geminationm}erlüfte\pwindex{Schnitzler, Arthur 15.\,5.\,1862 Wien – 21.\,10.\,1931 ebd.@\textsc{Schnitzler, Arthur} (15.\,5.\,1862 Wien – 21.\,10.\,1931 ebd.), \emph{Schriftsteller, Mediziner}!Im Spiel der Sommerlüfte. In drei Aufzügen@\strich\emph{Im Spiel der Sommerlüfte. In drei Aufzügen}|pw}}« und mehrere Unterstreichungen }\toendnotes[C]{\smallbreak}
\pstart
           {\pb}\textcolor{gray}{\textbf{THERESE RIE-ANDRO}}\hfill \textcolor{gray}{\textbf{WIEN, IV.\oindex{IV., Wieden@\textbf{IV., Wieden}, \emph{Verwaltungsgebiet}|pw}}}\pend
           
\pstart
           \raggedleft{}\textcolor{gray}{\textbf{SCHÖNBURGSTRASSE 48\oindex{Wien@\textbf{Wien}!IV., Wieden@\textbf{IV., Wieden}!Schönburgstraße 48@\textbf{Schönburgstraße 48}, \emph{Wohngebäude}|pw}}}\pend
           
\pstart
           \raggedleft{}22/12/29\pend
           
\pstart{}Verehrter Herr Doktor,\pend\vspace{0.5em}
\pstart
           Ich möchte Ihnen nur danken für das bezaubernde Stück\pwindex{Schnitzler, Arthur 15.\,5.\,1862 Wien – 21.\,10.\,1931 ebd.@\textsc{Schnitzler, Arthur} (15.\,5.\,1862 Wien – 21.\,10.\,1931 ebd.), \emph{Schriftsteller, Mediziner}!Im Spiel der Sommerlüfte. In drei Aufzügen@\strich\emph{Im Spiel der Sommerlüfte. In drei Aufzügen}|pwv} Leben und Athmosphäre, das Sie \label{K_L02567-1v}\edtext{gestern}{\lemma{\textnormal{\emph{gestern}}}\Cendnote{\textnormal{Siehe A. S.: \emph{Tagebuch}, 21. 12. 1929.
               }}}\label{K_L02567-1} vor uns haben erstehen lassen. Es ist wol in jedem Ihrer Stücke so, dass sich
               einem Selbsterlebtes zur Allgemeingiltigkeit sublimiert. Oh, wie gut kannte man sie,
               diese stillen Villen, eine Stunde und doch weltenweit von der Stadt, in deren weichem
               und etwas feuchtem Grün Frauen und Kinder von Mai bis September spielten und
               träumten. Es war nicht immer ein ganz gutes Träumen, das zeigt ihr Stück\pwindex{Schnitzler, Arthur 15.\,5.\,1862 Wien – 21.\,10.\,1931 ebd.@\textsc{Schnitzler, Arthur} (15.\,5.\,1862 Wien – 21.\,10.\,1931 ebd.), \emph{Schriftsteller, Mediziner}!Im Spiel der Sommerlüfte. In drei Aufzügen@\strich\emph{Im Spiel der Sommerlüfte. In drei Aufzügen}|pwv}, das mit so zarter Hand einen
               Schleier von Gesichtern entfernt, die uns \strikeout{so} vertraut
               und im Grunde doch fremd waren: von denen unserer Mütter. Man war schon rebellischer,
               man wollte nicht mehr so pflanzenhaft passiv dahinleben, man hielt für unlebendig, wo
               nur tiefstes Verbergen war; man begriff urplötzlich hervorquellende Bitterkeiten
               nicht. Das und noch so viel anderes lehrt Ihr Stück\pwindex{Schnitzler, Arthur 15.\,5.\,1862 Wien – 21.\,10.\,1931 ebd.@\textsc{Schnitzler, Arthur} (15.\,5.\,1862 Wien – 21.\,10.\,1931 ebd.), \emph{Schriftsteller, Mediziner}!Im Spiel der Sommerlüfte. In drei Aufzügen@\strich\emph{Im Spiel der Sommerlüfte. In drei Aufzügen}|pwv} verstehen – wann hätte ein Werk von Ihnen einen nicht
               das Leben besser verstehen gelehrt!\pend
           
\pstart
           Und Gusti\pwindex{Schnitzler, Arthur 15.\,5.\,1862 Wien – 21.\,10.\,1931 ebd.@\textsc{Schnitzler, Arthur} (15.\,5.\,1862 Wien – 21.\,10.\,1931 ebd.), \emph{Schriftsteller, Mediziner}!Im Spiel der Sommerlüfte. In drei Aufzügen@\strich\emph{Im Spiel der Sommerlüfte. In drei Aufzügen}|pwv}! Ich kannte Gusti\pwindex{Schnitzler, Arthur 15.\,5.\,1862 Wien – 21.\,10.\,1931 ebd.@\textsc{Schnitzler, Arthur} (15.\,5.\,1862 Wien – 21.\,10.\,1931 ebd.), \emph{Schriftsteller, Mediziner}!Im Spiel der Sommerlüfte. In drei Aufzügen@\strich\emph{Im Spiel der Sommerlüfte. In drei Aufzügen}|pwv} persönlich; immer war man
               Freund Ihrer Gestalten. Gusti\pwindex{Schnitzler, Arthur 15.\,5.\,1862 Wien – 21.\,10.\,1931 ebd.@\textsc{Schnitzler, Arthur} (15.\,5.\,1862 Wien – 21.\,10.\,1931 ebd.), \emph{Schriftsteller, Mediziner}!Im Spiel der Sommerlüfte. In drei Aufzügen@\strich\emph{Im Spiel der Sommerlüfte. In drei Aufzügen}|pwv}
               war eine heissverehrte Freundin (bei den Eltern weniger beliebt!), die man still
               bewunderte, weil sie so gut konnte, was man selbst nicht fertig brachte, weil ihre
               Unternehmungslust nicht von den Gedanken gehemmt war, dass der Mensch in einem
               gewissen Alter doch eigentlich nur aus Ellenbogen und linken Füssen besteht. (Das
               Wort »sex-appeal« war noch nicht erfunden.) Ich hoffe, Sie haben das junge Fräulein
                  Ullrich\pwindex{Ullrich, Luise 31.\,10.\,1910 Wien – 22.\,1.\,1985 München@\textsc{Ullrich, Luise} (31.\,10.\,1910 Wien – 22.\,1.\,1985 München), \emph{Schauspielerin}|pw}, die ich bisher noch garnicht
               kannte, ebenso entzückend gefunden wie ich: so ganz echt und am meisten, wo sie
               lügt!\pend
           
\pstart
           Ueberhaupt eine Aufführung, der man anmerkte, dass nicht nur gewöhnliche Regiearbeit
               geleistet worden war. Ueber Moissi\pwindex{Moissi, Alexander 2.\,4.\,1879 Triest – 22.\,3.\,1935 Wien@\textsc{Moissi, Alexander} (2.\,4.\,1879 Triest – 22.\,3.\,1935 Wien), \emph{Schauspieler}|pw}{ }{\pb}freilich möchte ich lieber nicht
               sprechen; er ist Ihnen gewiss lieb und auch persönlich ein anziehender Mensch. Aber
               er ist immer aus Neapel\oindex{Neapel@\textbf{Neapel}|pw} an der Newa\oindex{Newa@\textbf{Newa}, \emph{Fluss}|pw} – nie aus Österreich\oindex{Österreich@\textbf{Österreich}|pw}{\dots}\pend
           
\pstart
           Ich habe noch keine Kritiken gelesen und ich denke mir, es wird einen Ueberfluss an
               schönen Worten von Seiten der Herren geben, die ja alles besser wissen. Ich möchte
               Ihnen, verehrter Herr Doktor, nur ganz einfach und persönlich sagen, wie ganz
               mitgenommen ich von jeder Szene war, und wie ganz mir Ihr Stück\pwindex{Schnitzler, Arthur 15.\,5.\,1862 Wien – 21.\,10.\,1931 ebd.@\textsc{Schnitzler, Arthur} (15.\,5.\,1862 Wien – 21.\,10.\,1931 ebd.), \emph{Schriftsteller, Mediziner}!Im Spiel der Sommerlüfte. In drei Aufzügen@\strich\emph{Im Spiel der Sommerlüfte. In drei Aufzügen}|pwv} das Shakespeare\pwindex{Shakespeare, William 23.\,4.\,1564? Stratford-upon-Avon – 3.\,5.\,1616 ebd.@\textsc{Shakespeare, William} (23.\,4.\,1564? Stratford-upon-Avon – 3.\,5.\,1616 ebd.), \emph{Schauspieler, Dramatiker}|pw}’sche Wort zu erfüllen schien: »\label{K_L02567-2v}\edtext{Sind wir ein Spiel von jedem Druck der Luft}{\lemma{\textnormal{\emph{Sind … Luft}}}\Cendnote{\textnormal{richtig: Goethe\pwindex{Goethe, Johann Wolfgang von 28.\,8.\,1749 Frankfurt am Main – 22.\,3.\,1832 Weimar@\textsc{Goethe, Johann Wolfgang von} (28.\,8.\,1749 Frankfurt am Main – 22.\,3.\,1832 Weimar), \emph{Schriftsteller}|pwk}, \emph{Faust I}\pwindex{Goethe, Johann Wolfgang von 28.\,8.\,1749 Frankfurt am Main – 22.\,3.\,1832 Weimar@\textsc{Goethe, Johann Wolfgang von} (28.\,8.\,1749 Frankfurt am Main – 22.\,3.\,1832 Weimar), \emph{Schriftsteller}!Faust. Eine Tragödie@\strich\emph{Faust. Eine Tragödie}|pwk}}}}\label{K_L02567-2}«. Denn immer noch sind es die Abenteuer der Seele, die uns am tiefsten ans
               Herz rühren!\pend
           
\pstart
           In Dankbarkeit und Verehrung{\\[\baselineskip]}{[}hs.:{]} Ihre{\\[\baselineskip]}\spacefill\mbox{ThereseRie-Andro.}\pend
           \leftskip=0em{}\selectlanguage{ngerman}\endnumbering\briefempfaengerindex{Schnitzler, Arthur@\textsc{Schnitzler, Arthur}!zzzRie, Therese@\emph{von Therese Rie}!1929-12-222@{22. 12. 1929}|)be}\mylabel{L02567h}  \newcommand{\dateiname}{L02567}\newcommand{\titel}{Therese Rie-Andro an Arthur Schnitzler, 22. 12. 1929}\newcommand{\editorInnen}{Martin Anton Müller und Gerd-Hermann Susen}%% latex-leseansicht-abspann.tex
%% Abspann für die Leseansicht.
%% Der Schalter \ifkorrekturansicht ist bereits durch den Vorspann gesetzt.

%% latex-abspann.tex
%% Gemeinsamer Abspann für Korrekturansicht und Leseansicht.
%% Setzt den Schalter \ifkorrekturansicht voraus (gesetzt in den
%% einbindenden Dateien latex-korrekturansicht-abspann.tex bzw.
%% latex-leseansicht-abspann.tex).
%% ---------------------------------------------------------------

\normalsize

% Das esempio-Environment wird nur in der Leseansicht benötigt
\ifkorrekturansicht\else
\newenvironment{esempio}[3]%
{
    \vspace{1.5ex}
    \rlap{\underline{#1}}
    \par
    \setlength{\parindent}{0cm}
    \nopagebreak
    \leftskip=#2cm
    \rightskip=#3cm
}
{
    \par
}
\fi

\doendnotes{C}
\bigskip
\vfill

\clearpage

\footnotesize

\ifkorrekturansicht
  \lohead{\textsc{register}}
\fi

% theindex-Environment neu definieren ohne reledmac
\makeatletter
\renewenvironment{theindex}{%
  \ifkorrekturansicht
    \section*{\indexname}%
  \else
    \subsubsection*{Index der erwähnten Entitäten}%
  \fi
  \setlength{\parindent}{0pt}%
  \setlength{\parskip}{0pt plus 0.3pt}%
  \let\item\@idxitem
}{%
  \ifkorrekturansicht\clearpage\fi
}
\makeatother

\IfFileExists{\jobname-pw.ind}{\input{\jobname-pw.ind}}{}

% Quellenangabe nur in der Leseansicht
\ifkorrekturansicht\else
% Fallback-Definitionen, falls die .tex-Datei \titel etc. nicht gesetzt hat
\providecommand{\titel}{}
\providecommand{\editorInnen}{}
\providecommand{\dateiname}{\jobname}

\vspace{3cm}

\vfill

\footnotesize
\textsc{Quelle}: \titel. Herausgegeben von {\editorInnen}. In: \emph{Arthur Schnitzler: Briefwechsel mit Autorinnen und Autoren}.
 Digitale Edition, https://schnitzler-briefe.acdh.oeaw.ac.at/{\dateiname}.html (Stand \today)
\fi

\end{document}


