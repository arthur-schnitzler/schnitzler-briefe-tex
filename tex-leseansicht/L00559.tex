%% latex-korrekturansicht-vorspann.tex
%% Vorspann für die Korrekturansicht.
%% Lädt die gemeinsame Datei latex-vorspann.tex mit gesetztem Schalter.

\newif\ifkorrekturansicht
\korrekturansichttrue

\input{../tex-inputs/latex-vorspann}


\section[Arthur Schnitzler an Richard Beer-Hofmann, 2. 7. 1896]{L00559 Arthur Schnitzler an Richard Beer-Hofmann, 2. 7. 1896}
\nopagebreak\mylabel{L00559v}
\rehead{ }\normalsize\beginnumbering\briefempfaengerindex{Beer-Hofmann, Richard@\textsc{Beer-Hofmann, Richard}!zzzSchnitzler, Arthur@\emph{von Arthur Schnitzler}!1896-07-021@{2. 7. 1896}|(be}
\toendnotes[C]{\smallbreak\pagebreak[2]}\Standort{YCGL, MSS 31.}
\physDesc{Postkarte, 551 Zeichen
\newline{}Handschrift: schwarze Tinte, deutsche Kurrent
\newline{}Versand: 1) Stempel: »\nobreak{}\oindex{VI., Mariahilf@\textbf{VI., Mariahilf}, \emph{A.ADM3}|pwk}Wien 6/1, 2. 7. 96, 1–2N\nobreak{}«.   2) Stempel: »\nobreak{}\oindex{St. Gilgen@\textbf{St. Gilgen}, \emph{A.ADM3}|pwk}St. Gilgen, 3 7 96\nobreak{}«. }
\buchAbdrucke{\weitereDrucke{Arthur Schnitzler, Richard Beer-Hofmann: \emph{Briefwechsel 1891–1931}. Wien, Zürich: \emph{Europaverlag} 1992, S. 92.} }\pstart{}{\pb}Herrn \textsc{Dr. Richard
                     Beer-Hofmann}\pend{}\pstart{}\textsc{Fürberg am Wolfgangsee}\oindex{Fuerberg@\textbf{Fürberg}, \emph{P.PPL}|pw}\pend{}{\bigskip}\vspace{1em}
\pstart
           \noindent{}{\pb}Lieber Richard, wenn Sie nicht in der Correſpondzkartensti{\geminationm}g ſind, raffen Sie ſich zu einem Brief auf. Paul\pwindex{Goldmann, Paul 31.01.1865 – 25.09.1935@\textsc{Goldmann, Paul} (31.01.1865 – 25.09.1935), \emph{Schriftsteller/Schriftstellerin, Journalist/Journalistin}|pw} ko{\geminationm}t nach Dänemark\oindex{Daenemark@\textbf{Dänemark}, \emph{A.PCLI}|pw}. Schreiben Sie ihm. Ich reiſe
                  Freitag Abend Hamburg\oindex{Hamburg@\textbf{Hamburg}, \emph{P.PPLA}|pw}. Dort \textsc{post rest} hoff ich Nachricht von Ihnen zu finden. Am
                  7. geht mein Schiff ab. Nach \textsc{Trondjhem}\oindex{Trondheim@\textbf{Trondheim}, \emph{P.PPLA2}|pw}{ }ſenden Sie eine \introOben{}(briefl.)\introOben{}
               Nachricht am 9. Juli; eine zweite am 18. Juli. – Telegra{\geminationm}e wiſſen Sie ja. Gehen Sie nicht nach München\oindex{Muenchen@\textbf{München}, \emph{P.PPLA}|pw}? Vielleicht doch mit mir auf der
               Rückreiſe. –\pend
           
\pstart
           Seien Sie vielmals herzlich gegrüßt u grüßen Sie Paula\pwindex{Beer-Hofmann, Paula 25.02.1879 – 30.10.1939@\textsc{Beer-Hofmann, Paula} (25.02.1879 – 30.10.1939)|pw}.{\\[\baselineskip]}Ihr \spacefill\mbox{ArthurSch}\pend
           \leftskip=0em{}\selectlanguage{ngerman}\endnumbering\briefempfaengerindex{Beer-Hofmann, Richard@\textsc{Beer-Hofmann, Richard}!zzzSchnitzler, Arthur@\emph{von Arthur Schnitzler}!1896-07-021@{2. 7. 1896}|)be}\mylabel{L00559h}  \normalsize

\doendnotes{C}
\bigskip
\vfill

\clearpage

\footnotesize

\lohead{\textsc{register}}

% Definiere theindex-Environment komplett neu ohne reledmac
\makeatletter
\renewenvironment{theindex}{%
  \section*{\indexname}%
  \setlength{\parindent}{0pt}%
  \setlength{\parskip}{0pt plus 0.3pt}%
  \let\item\@idxitem
}{%
  \clearpage
}
\makeatother

\IfFileExists{\jobname-pw.ind}{\input{\jobname-pw.ind}}{}

\end{document}

      