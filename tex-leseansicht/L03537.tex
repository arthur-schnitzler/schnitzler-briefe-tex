%% latex-korrekturansicht-vorspann.tex
%% Vorspann für die Korrekturansicht.
%% Lädt die gemeinsame Datei latex-vorspann.tex mit gesetztem Schalter.

\newif\ifkorrekturansicht
\korrekturansichttrue

\input{../tex-inputs/latex-vorspann}


\section[ Paul Goldmann an Olga Gussmann, 20. 12. {[}1900{]}]{L03537 Paul Goldmann an Olga Gussmann, 20. 12. {[}1900{]}}
\nopagebreak\mylabel{L03537v}
\rehead{ }\normalsize\beginnumbering\briefempfaengerindex{Schnitzler, Olga@\textsc{Schnitzler, Olga}!zzzGoldmann, Paul@\emph{von Paul Goldmann}!1900-12-201@{20. 12. {[}1900{]}}|(be}
\toendnotes[C]{\smallbreak\pagebreak[2]}\Standort{DLA, A:Schnitzler, HS.NZ85.1.5247.}
\physDesc{Brief, 1 Blatt, 4 Seiten, 2177 Zeichen
\newline{}Handschrift: blaue Tinte, deutsche Kurrent}\toendnotes[C]{\smallbreak}
\pstart
           \raggedleft{}{\pb}\textcolor{gray}{\textbf{DESSAUERSTRASSE 19\oindex{Dessauer Strasse@\textbf{Dessauer Straße}, \emph{Straße (K.STR)}|pw}}}\pend
           
\pstart
           Berlin\oindex{Berlin@\textbf{Berlin}, \emph{P.PPLC}|pw}, 20. Dezember.\pend
           
\pstart\center{}Verehrtes und liebes Fräulein,\pend\vspace{0.5em}
\pstart
           Die Briefe, die Sie und Ihr Schweſterchen\pwindex{Steinrueck, Elisabeth 19.11.1885 – 07.04.1920@\textsc{Steinrück, Elisabeth} (19.11.1885 – 07.04.1920)|pwv} mir geſchrieben, haben mir \strikeout{g\textcolor{gray}{×}\-\textcolor{gray}{×}} große Freude bereitet. Seit Wochen liegen ſie auf dem Schreibtiſch – ganz
               obenauf, um raſch zur Hand zu ſein für den Fall, daß die Stunde des Briefſchreibens
               kommen ſollte. Aber die Stunde iſt bisher nicht gekommen, wird auch wohl ſo bald
               nicht kommen in meinem vielgeplagten Berichterſtatter-Daſein, und das, was ich Ihnen
                  heut ſchreibe, iſt eigentlich kein Brief, ſondern
               es ſind nur drei kurze Worte des Dankes und des herzlichen Gedenkens, die doch
               endlich einmal geſagt werden mußten, Ihnen {\pb}ſowohl,
               wie dem Fräulein \textsc{Liesl\pwindex{Steinrueck, Elisabeth 19.11.1885 – 07.04.1920@\textsc{Steinrück, Elisabeth} (19.11.1885 – 07.04.1920)|pw}}.\pend
           
\pstart
           Inzwiſchen war \textsc{Dr. Schnitzler} in \strikeout{Wien\oindex{Wien@\textbf{Wien}, \emph{A.ADM2}|pw}}{ }\label{K_L03537-1v}\edtext{Berlin\oindex{Berlin@\textbf{Berlin}, \emph{P.PPLC}|pw}}{\lemma{\textnormal{\emph{Berlin}}}\Cendnote{\textnormal{Schnitzler war zwischen dem 24. 11. 1900 und dem 28. 11. 1900 in Berlin\oindex{Berlin@\textbf{Berlin}, \emph{P.PPLC}|pwk} gewesen und hatte Goldmann\pwindex{Goldmann, Paul 31.01.1865 – 25.09.1935@\textsc{Goldmann, Paul} (31.01.1865 – 25.09.1935), \emph{Schriftsteller/Schriftstellerin, Journalist/Journalistin}|pwk} dort täglich getroffen.}}}\label{K_L03537-1} und hat mir
               Mancherlei über die Rothe-Sterngaſſe\pwindex{Steinrueck, Elisabeth 19.11.1885 – 07.04.1920@\textsc{Steinrück, Elisabeth} (19.11.1885 – 07.04.1920)|pwv}\oindex{Rotensterngasse@\textbf{Rotensterngasse}, \emph{Straße (K.STR)}|pw} berichtet. Insbeſondere, daß es Ihnen gut geht und daß Sie tüchtig vorwärts
               ſtreben, was ja die Hauptſache iſt. Ich wäre gern, gern wieder einmal mit Ihnen
               zuſammen. Berlin\oindex{Berlin@\textbf{Berlin}, \emph{P.PPLC}|pw} iſt eine große Stadt, aber
                  \label{K_L03537-2v}\edtext{eine Rothe-Sterngaſſe\oindex{Rotensterngasse@\textbf{Rotensterngasse}, \emph{Straße (K.STR)}|pw} gibt es hier nicht}{\lemma{\textnormal{\emph{eine … nicht}}}\Cendnote{\textnormal{Die Stelle lässt sich auch im Kontext von Goldmanns\pwindex{Goldmann, Paul 31.01.1865 – 25.09.1935@\textsc{Goldmann, Paul} (31.01.1865 – 25.09.1935), \emph{Schriftsteller/Schriftstellerin, Journalist/Journalistin}|pwk} (unerwiderter) Schwärmerei für Elisabeth Gussmann\pwindex{Steinrueck, Elisabeth 19.11.1885 – 07.04.1920@\textsc{Steinrück, Elisabeth} (19.11.1885 – 07.04.1920)|pwkv} lesen, vgl.
                     deren Korrespondenz: \emph{DLA}, HS.1985.1.5246.
               }}}\label{K_L03537-2}. Und ich bin ſehr einſam.\pend
           
\pstart
           Sie ſollen mir bald wieder ſchreiben, Sie und Ihr Fräulein Schweſter\pwindex{Steinrueck, Elisabeth 19.11.1885 – 07.04.1920@\textsc{Steinrück, Elisabeth} (19.11.1885 – 07.04.1920)|pwv}, das Sie ſelbſt die »kleine
               Beſtie« nennen. (Ich wage kaum, es niederzuſchreiben). Auch ſollten Sie Beide\pwindex{Steinrueck, Elisabeth 19.11.1885 – 07.04.1920@\textsc{Steinrück, Elisabeth} (19.11.1885 – 07.04.1920)|pwv} nach Berlin\oindex{Berlin@\textbf{Berlin}, \emph{P.PPLC}|pw} kommen. Ich werde Sie fürſtlich aufnehmen, {\pb}und Sie dürfen bei \textsc{Josty\oindex{Cafe Josty@\textbf{Café Josty}, \emph{Kaffeehaus (K.KAF)}|pw}} einen ganzen Tag lang Indianerkrapfen mit Schlagobers eſſen.\pend
           
\pstart
           Im \label{K_L03537-3v}\edtext{Theater}{\lemma{\textnormal{\emph{Theater}}}\Cendnote{\textnormal{Friedrich Hebbels\pwindex{Hebbel, Friedrich 18.03.1813 – 13.12.1863@\textsc{Hebbel, Friedrich} (18.03.1813 – 13.12.1863), \emph{Schriftsteller/Schriftstellerin}|pwk}{ }\emph{Agnes Bernauer}\pwindex{Agnes Bernauer@\emph{Agnes Bernauer}|pwk} wurde am Berlin\oindex{Berlin@\textbf{Berlin}, \emph{P.PPLC}|pwk}er \emph{Schauspielhaus}\orgindex{Schauspielhaus Berlin@Schauspielhaus Berlin|pwk} gegeben. Tolstois\pwindex{Tolstoi, Leo N. von 09.09.1828 – 20.11.1910@\textsc{Tolstoi, Leo N. von} (09.09.1828 – 20.11.1910), \emph{Schriftsteller/Schriftstellerin, Schriftsteller/Schriftstellerin, Krimiautor/Krimiautorin}|pwk}{ }\emph{Die
                     Macht der Finsternis}\pwindex{Macht der Finsternis@\emph{Die Macht der Finsternis}|pwk} stand am Spielplan des \emph{Deutschen Theaters}\orgindex{Deutsches Theater Berlin@Deutsches Theater Berlin|pwk}. Am \emph{Berliner
                     Theater}\orgindex{Berliner Theater@Berliner Theater|pwk} wurde \emph{Frauenherrschaft. Lustspiel
                     in vier Aufzügen nach Aristophanes’ »Ekklesiazusen« und »Lysistrate«}\pwindex{Frauenherrschaft. Lustspiel in vier Aufzuegen nach Aristophanes  »Ekklesiazusen« und »Lysistrate«@\emph{Frauenherrschaft. Lustspiel in vier Aufzügen nach Aristophanes’ »Ekklesiazusen« und »Lysistrate«}|pwk} von
                     Adolf von Wilbrandt\pwindex{Wilbrandt, Adolf von 24.08.1837 – 10.06.1911@\textsc{Wilbrandt, Adolf von} (24.08.1837 – 10.06.1911), \emph{Schriftsteller/Schriftstellerin, Theaterleiter/Theaterleiterin, Schauspieler/Schauspielerin}|pwk} gespielt.}}}\label{K_L03537-3}
               erleben wir allerlei Gutes: \textsc{Tolstois\pwindex{Tolstoi, Leo N. von 09.09.1828 – 20.11.1910@\textsc{Tolstoi, Leo N. von} (09.09.1828 – 20.11.1910), \emph{Schriftsteller/Schriftstellerin, Schriftsteller/Schriftstellerin, Krimiautor/Krimiautorin}|pw}} »Macht der Finſterniß\pwindex{Macht der Finsternis@\emph{Die Macht der Finsternis}|pw}«, \textsc{Hebbel\pwindex{Hebbel, Friedrich 18.03.1813 – 13.12.1863@\textsc{Hebbel, Friedrich} (18.03.1813 – 13.12.1863), \emph{Schriftsteller/Schriftstellerin}|pw}\textcolor{gray}{’}s} herrliche »\textsc{Agnes Bernauer\pwindex{Agnes Bernauer@\emph{Agnes Bernauer}|pw}}«, ein wenig \textsc{Aristophanes\pwindex{Frauenherrschaft. Lustspiel in vier Aufzuegen nach Aristophanes  »Ekklesiazusen« und »Lysistrate«@\emph{Frauenherrschaft. Lustspiel in vier Aufzügen nach Aristophanes’ »Ekklesiazusen« und »Lysistrate«}|pwv}\pwindex{Aristophanes 0445? v. Chr. – 0385? v. Chr.@\textsc{Aristophanes} (0445? v. Chr. – 0385? v. Chr.), \emph{Schriftsteller/Schriftstellerin}|pw} etc}.\pend
           
\pstart
           Wenn Sie unſeren lieben \textsc{Dr. Arthur Schnitzler} ſehen, ſo ſagen Sie ihm: 1.) daß er mir eine Ewigkeit nicht geſchrieben hat und
               daß dies eine Infamie iſt 2.) daß \textsc{Alfred Klaar\pwindex{Klaar, Alfred 07.11.1848 – 04.11.1927@\textsc{Klaar, Alfred} (07.11.1848 – 04.11.1927), \emph{Schriftsteller/Schriftstellerin, Kritiker/Kritikerin}|pw}}, der ehemalige Kritiker der »\textsc{Bohemia\orgindex{Bohemia@Bohemia|pw}}«, ein Schmock in Reincultur, der ödeſte und blödeſte Schwätzer der
                  Jetztzeit{[},{]} Theaterkritiker und Feuilleton-Redakteur der »Voſſiſchen Zeitung\orgindex{Vossische Zeitung@Vossische Zeitung|pw}« geworden iſt. Auch ich hatte
               mich für die Stelle gemeldet, {\pb}bekam aber nicht einmal
               eine Antwort. Ich bin nämlich (aber ſagen Sie es nicht weiter!) \label{K_L03537-4v}\edtext{nicht »literariſch«}{\lemma{\textnormal{\emph{nicht »literariſch«}}}\Cendnote{\textnormal{Diesen vermeintlichen Vorbehalt gegenüber seiner Person und
                  dem Beruf des Kritikers an sich hatte Goldmann\pwindex{Goldmann, Paul 31.01.1865 – 25.09.1935@\textsc{Goldmann, Paul} (31.01.1865 – 25.09.1935), \emph{Schriftsteller/Schriftstellerin, Journalist/Journalistin}|pwk} in Briefen an Schnitzler
                  bereits mehrmals thematisiert, beispielsweise Paul Goldmann an Arthur Schnitzler, 29. 5. [1900].}}}\label{K_L03537-4}.\pend
           
\pstart
           Ich wünſche Ihnen und dem Fräulein \textsc{Liesl\pwindex{Steinrueck, Elisabeth 19.11.1885 – 07.04.1920@\textsc{Steinrück, Elisabeth} (19.11.1885 – 07.04.1920)|pw}} frohe Weihnachten, bitte Sie, meinen
               Namensvetter \textsc{Paul\pwindex{Marx, Paul 21.07.1879 – 1956-10-30@\textsc{Marx, Paul} (21.07.1879 – 1956-10-30), \emph{Regisseur/Regisseurin, Schauspieler/Schauspielerin}|pw}} zu grüßen, hoffe, bald wieder durch einen Brief erfreut zu werden, und küſſe
               Ihnen Beiden\pwindex{Steinrueck, Elisabeth 19.11.1885 – 07.04.1920@\textsc{Steinrück, Elisabeth} (19.11.1885 – 07.04.1920)|pwv} je eine Hand.
               {\\[\baselineskip]}Ihr freundſchaftlich ergebener {\\[\baselineskip]}\spacefill\mbox{Dr. Paul Goldmann.}\pend
           \leftskip=0em{}\selectlanguage{ngerman}\endnumbering\briefempfaengerindex{Schnitzler, Olga@\textsc{Schnitzler, Olga}!zzzGoldmann, Paul@\emph{von Paul Goldmann}!1900-12-201@{20. 12. {[}1900{]}}|)be}\mylabel{L03537h}  \normalsize

\doendnotes{C}
\bigskip
\vfill

\clearpage

\footnotesize

\lohead{\textsc{register}}

% Definiere theindex-Environment komplett neu ohne reledmac
\makeatletter
\renewenvironment{theindex}{%
  \section*{\indexname}%
  \setlength{\parindent}{0pt}%
  \setlength{\parskip}{0pt plus 0.3pt}%
  \let\item\@idxitem
}{%
  \clearpage
}
\makeatother

\IfFileExists{\jobname-pw.ind}{\input{\jobname-pw.ind}}{}

\end{document}

      