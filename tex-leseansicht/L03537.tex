%% latex-leseansicht-vorspann.tex
%% Vorspann für die Leseansicht.
%% Lädt die gemeinsame Datei latex-vorspann.tex mit nicht gesetztem Schalter.

\newif\ifkorrekturansicht
\korrekturansichtfalse

\input{../tex-inputs/latex-vorspann}

\begin{center}
            \textcolor{red}{ENTWURF, NICHT FERTIG KORRIGIERT}
                      \end{center}
            
         
         \renewcommand{\erwaehntePersonen}{Personen: Olga Schnitzler, Elisabeth Steinrück}
         \renewcommand{\erwaehnteOrte}{Orte: Berlin, Wien}
         \renewcommand{\erwaehnteWerke}{}
               \section[ Paul Goldmann an Olga XXXX Gussmann/Schnitzler, 20. 12. {[}XXXX{]}]{ Paul Goldmann an Olga XXXX Gussmann/Schnitzler, 20. 12. {[}XXXX{]}}\nopagebreak\mylabel{v}\rehead{ }\begin{ledgroupsized}[t]{13cm}\normalsize\beginnumbering \toendnotes[C]{\smallbreak\pagebreak[2]} \Standort{DLA, A:Schnitzler, HS.1985.1.5247.}
\physDesc{,  Blätter,  Seiten
\newline{}Handschrift: , deutsche Kurrent}\toendnotes[C]{\smallbreak}{\pb}\textcolor{gray}{\textbf{DESSAUERSTRASSE 19\oindex{XXXX Ortsangabe fehlt|pw}}}\textcolor{red}{\textsuperscript{\textbf{KEY}}}\pstart
           Berlin\oindex{Berlin@\textbf{Berlin}|pw}, 20. Dezember.\pend
           \pstart{}Verehrtes und liebes Fräulein,\pend\pstart
           Die Briefe, die Sie und Ihr Schweſterchen\textcolor{red}{\textsuperscript{\textbf{KEY}}} mir
               geſchrieben, haben mir \strikeout{gX} große Freude
               bereitet. Seit Wochen liegen ſie auf dem Schreibtiſch – ganz obenauf, um raſch zur
               Hand zu ſein für den Fall, daß die Stunde des Briefſchreibens kommen ſollte. Aber die
               Stunde iſt bisher nicht gekommen, wird auch wohl ſo bald nicht kommen in meinem
               vielgeplagten Berichterſtatter-Daſein, und das, was ich Ihnen heut
               ſchreibe, iſt eigentlich kein Brief, ſondern es ſind nur drei kurze Worte des Dankes
               und des herzlichen Gedankens, die doch endlich einmal geſagt werden mußten, Ihnen {\pb} ſowohl, wie dem Fräulein \textsc{Liesl\pwindex{Steinrueck, Elisabeth 19.11.1885 – 07.04.1920@\textsc{Steinrück, Elisabeth} (19.11.1885 – 07.04.1920)|pw}}.\pend
           \pstart
           Inzwiſchen war \textsc{Dr. Schnitzler\pwindex{Schnitzler, Arthur 15.05.1862 – 21.10.1931@\textsc{Schnitzler, Arthur} (15.05.1862 – 21.10.1931), \emph{Schriftsteller, Mediziner}|pw}} in \strikeout{Wien}\textcolor{red}{\textsuperscript{\textbf{KEY}}}Berlin\oindex{Berlin@\textbf{Berlin}|pw} und hat mir Mancherlei über die Rothe-Sterngaſſe\textcolor{red}{\textsuperscript{\textbf{KEY}}}\textcolor{red}{\textsuperscript{\textbf{KEY}}} berichtet. Insbeſondere, daß es Ihnen gut geht und daß Sie tüchtig vorwärts
               ſtreben, was ja die Hauptſache iſt. Ich wäre gern, gern wieder einmal mit Ihnen
               zuſammen. Berlin\oindex{Berlin@\textbf{Berlin}|pw} iſt eine große Stadt, aber eine
                  Rothe-Sterngaſſe\textcolor{red}{\textsuperscript{\textbf{KEY}}}\textcolor{red}{\textsuperscript{\textbf{KEY}}} gibt es hier nicht. Und ich bin ſehr einſam.\pend
           \pstart
           Sie ſollen mir bald wieder ſchreiben, Sie und Ihr Fräulein Schweſter\textcolor{red}{\textsuperscript{\textbf{KEY}}}, das Sie ſelbſt die »kleine Beſtie«
               nennen. (Ich wage kaum, es niederzuſchreiben). Auch ſollten Sie Beide\textcolor{red}{\textsuperscript{\textbf{KEY}}} nach Berlin\oindex{Berlin@\textbf{Berlin}|pw} kommen. Ich werde Sie fürſtlich aufnehmen, {\pb} und Sie dürfen bei \label{XXXXv}\edtext{\textsc{Josty\textcolor{red}{\textsuperscript{\textbf{KEY}}}}[Kommentar: Café\u0020Josty]}{\lemma{\textnormal{\emph{XXXX Lemmafehler}}}\Cendnote{\textnormal{}}}\label{XXXX} einen ganzen Tag lang Indianerkrapfen mit
               Schlagobers eſſen.\pend
           \pstart
           Im Theater erleben wir allerlei Gutes: \textsc{Tolstoi\textcolor{red}{\textsuperscript{\textbf{KEY}}}s} »Macht der
                  Finſterniß\textcolor{red}{\textsuperscript{\textbf{KEY}}}«, \textsc{Hebbel\textcolor{gray}{}\textcolor{red}{\textsuperscript{\textbf{KEY}}}\textcolor{gray}{’}s} herrliche »\textsc{Agnes Bernauer\textcolor{red}{\textsuperscript{\textbf{KEY}}}}«, ein wenig \textsc{Aristophanes\textcolor{red}{\textsuperscript{\textbf{KEY}}} etc}.\pend
           \pstart
           Wenn Sie unſeren lieben \textsc{Dr. Arthur\pwindex{Schnitzler, Arthur 15.05.1862 – 21.10.1931@\textsc{Schnitzler, Arthur} (15.05.1862 – 21.10.1931), \emph{Schriftsteller, Mediziner}|pw}}\textsc{Schnitzler\pwindex{Schnitzler, Arthur 15.05.1862 – 21.10.1931@\textsc{Schnitzler, Arthur} (15.05.1862 – 21.10.1931), \emph{Schriftsteller, Mediziner}|pw}} ſehen, ſo ſagen Sie ihm: 1.) daß er mir eine Ewigkeit nicht geſchrieben hat und
               daß dies eine Infamie iſt 2.) daß \textsc{Alfred Klaar\textcolor{red}{\textsuperscript{\textbf{KEY}}}}, der ehemalige Kritiker der »\textsc{Bohemia\textcolor{red}{\textsuperscript{\textbf{KEY}}}}«, ein Schmock in Reincultur, der ödeſte und blödeſte Schwätzer der
                  Jetztzeit{[},{]} Theaterkritiker und Feuilleton-Redakteur der »Voſſiſchen Zeitung\textcolor{red}{\textsuperscript{\textbf{KEY}}}« geworden iſt. Auch ich hatte mich für
               die Stelle gemeldet, {\pb}bekam aber
               nicht einmal eine Antwort. Ich bin nämlich (aber ſagen Sie es nicht weiter!) nicht
               »literariſch«. {\\[\baselineskip]}Ich wünſche Ihnen und dem Fräulein\pend
           \leftskip=0em{}\pstart
           {\\[\baselineskip]}\textsc{Liesl\pwindex{Steinrueck, Elisabeth 19.11.1885 – 07.04.1920@\textsc{Steinrück, Elisabeth} (19.11.1885 – 07.04.1920)|pw}} frohe Weihnachten, bitte Sie, meinen\pend
           \leftskip=0em{}\pstart
           {\\[\baselineskip]}Namensvetter \textsc{Paul\textcolor{red}{\textsuperscript{\textbf{KEY}}}} zu grüßen, hoffe,\pend
           \leftskip=0em{}\pstart
           {\\[\baselineskip]}bald wieder durch einen Brief erfreut\pend
           \leftskip=0em{}\pstart
           {\\[\baselineskip]}zu werden, und küſſe Ihnen Beiden\textcolor{red}{\textsuperscript{\textbf{KEY}}} je eine\pend
           \leftskip=0em{}\pstart
           {\\[\baselineskip]}Hand.\pend
           \leftskip=0em{}\pstart
           {\\[\baselineskip]}Ihr freundſchaftlich ergebener\pend
           \leftskip=0em{}\pstart
           {\\[\baselineskip]}\spacefill\mbox{Dr. Paul Goldmann.}\pend
           \leftskip=0em{}
         
         \endnumbering\mylabel{h}\end{ledgroupsized}\begin{anhang}\end{anhang}\newcommand{\dateiname}{L03537}\newcommand{\titel}{Paul Goldmann an Olga XXXX Gussmann/Schnitzler, 20. 12. [XXXX]}\newcommand{\editorInnen}{Martin Anton Müller und Laura Untner}%% latex-leseansicht-abspann.tex
%% Abspann für die Leseansicht.
%% Der Schalter \ifkorrekturansicht ist bereits durch den Vorspann gesetzt.

%% latex-abspann.tex
%% Gemeinsamer Abspann für Korrekturansicht und Leseansicht.
%% Setzt den Schalter \ifkorrekturansicht voraus (gesetzt in den
%% einbindenden Dateien latex-korrekturansicht-abspann.tex bzw.
%% latex-leseansicht-abspann.tex).
%% ---------------------------------------------------------------

\normalsize

% Das esempio-Environment wird nur in der Leseansicht benötigt
\ifkorrekturansicht\else
\newenvironment{esempio}[3]%
{
    \vspace{1.5ex}
    \rlap{\underline{#1}}
    \par
    \setlength{\parindent}{0cm}
    \nopagebreak
    \leftskip=#2cm
    \rightskip=#3cm
}
{
    \par
}
\fi

\doendnotes{C}
\bigskip
\vfill

\clearpage

\footnotesize

\ifkorrekturansicht
  \lohead{\textsc{register}}
\fi

% theindex-Environment neu definieren ohne reledmac
\makeatletter
\renewenvironment{theindex}{%
  \ifkorrekturansicht
    \section*{\indexname}%
  \else
    \subsubsection*{Index der erwähnten Entitäten}%
  \fi
  \setlength{\parindent}{0pt}%
  \setlength{\parskip}{0pt plus 0.3pt}%
  \let\item\@idxitem
}{%
  \ifkorrekturansicht\clearpage\fi
}
\makeatother

\IfFileExists{\jobname-pw.ind}{\input{\jobname-pw.ind}}{}

% Quellenangabe nur in der Leseansicht
\ifkorrekturansicht\else
% Fallback-Definitionen, falls die .tex-Datei \titel etc. nicht gesetzt hat
\providecommand{\titel}{}
\providecommand{\editorInnen}{}
\providecommand{\dateiname}{\jobname}

\vspace{3cm}

\vfill

\footnotesize
\textsc{Quelle}: \titel. Herausgegeben von {\editorInnen}. In: \emph{Arthur Schnitzler: Briefwechsel mit Autorinnen und Autoren}.
 Digitale Edition, https://schnitzler-briefe.acdh.oeaw.ac.at/{\dateiname}.html (Stand \today)
\fi

\end{document}


      