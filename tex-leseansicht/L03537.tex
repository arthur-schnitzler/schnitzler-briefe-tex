%% latex-leseansicht-vorspann.tex
%% Vorspann für die Leseansicht.
%% Lädt die gemeinsame Datei latex-vorspann.tex mit nicht gesetztem Schalter.

\newif\ifkorrekturansicht
\korrekturansichtfalse

\input{../tex-inputs/latex-vorspann}


\section[ Paul Goldmann an Olga Gussmann, 20. 12. [1900]]{L03537 Paul Goldmann an Olga Gussmann,  20. 12. [1900]}
\nopagebreak\mylabel{L03537v}
\rehead{ }\normalsize\beginnumbering\briefempfaengerindex{Schnitzler, Olga@\textsc{Schnitzler, Olga}!zzzGoldmann, Paul@\emph{von Paul Goldmann}!1900-12-202@{20. 12. [1900]}|(be}
\toendnotes[C]{\smallbreak\pagebreak[2]}
\correspDesc{Versand  durch Paul Goldmann am 20. 12. [1900] in Berlin
\newline{}Erhalt  durch Olga Gussmann im Zeitraum [21. 12. 1900 – 25. 12. 1900?] in Wien}\toendnotes[C]{\smallbreak}
\Standort{DLA, A:Schnitzler, HS.NZ85.1.5247.}
\physDesc{Brief, 1 Blatt, 4 Seiten, 2177 Zeichen
\newline{}Handschrift: blaue Tinte, deutsche Kurrent}\toendnotes[C]{\smallbreak}
\pstart
           \raggedleft{}{\pb}\textcolor{gray}{\textbf{DESSAUERSTRASSE 19\oindex{Dessauer Straße@\textbf{Dessauer Straße}, \emph{Straße}|pw}}}\pend
           
\pstart
           Berlin\oindex{Berlin@\textbf{Berlin}, \emph{Hauptstadt}|pw}, 20. Dezember.\pend
           
\pstart\center{}Verehrtes und liebes Fräulein,\pend\vspace{0.5em}
\pstart
           Die Briefe, die Sie und Ihr Schweſterchen\pwindex{Steinrück, Elisabeth 19.\,11.\,1885 – 7.\,4.\,1920 Partenkirchen@\textsc{Steinrück, Elisabeth} (19.\,11.\,1885 – 7.\,4.\,1920 Partenkirchen)|pwv} mir geſchrieben, haben mir \strikeout{g\textcolor{gray}{×}\-\textcolor{gray}{×}} große Freude bereitet. Seit Wochen liegen{ }ſie auf dem Schreibtiſch – ganz
               obenauf, um raſch zur Hand zu{ }ſein für den Fall, daß die Stunde des Briefſchreibens
               kommen{ }ſollte. Aber die Stunde iſt bisher nicht gekommen, wird auch wohl{ }ſo bald
               nicht kommen in meinem vielgeplagten Berichterſtatter-Daſein, und das, was ich Ihnen
                  heut{ }ſchreibe, iſt eigentlich kein Brief,{ }ſondern
               es{ }ſind nur drei kurze Worte des Dankes und des herzlichen Gedenkens, die doch
               endlich einmal geſagt werden mußten, Ihnen {\pb}ſowohl,
               wie dem Fräulein \textsc{Liesl\pwindex{Steinrück, Elisabeth 19.\,11.\,1885 – 7.\,4.\,1920 Partenkirchen@\textsc{Steinrück, Elisabeth} (19.\,11.\,1885 – 7.\,4.\,1920 Partenkirchen)|pw}}.\pend
           
\pstart
           Inzwiſchen war \textsc{Dr. Schnitzler} in \strikeout{Wien\oindex{Wien@\textbf{Wien}, \emph{Verwaltungsgebiet}|pw}}{ }\label{K_L03537-1v}\edtext{Berlin\oindex{Berlin@\textbf{Berlin}, \emph{Hauptstadt}|pw}}{\lemma{\textnormal{\emph{Berlin}}}\Cendnote{\textnormal{Schnitzler war zwischen dem 24. 11. 1900 und dem 28. 11. 1900 in Berlin\oindex{Berlin@\textbf{Berlin}, \emph{Hauptstadt}|pwk} gewesen und hatte Goldmann\pwindex{Goldmann, Paul 31.\,1.\,1865 Breslau – 25.\,9.\,1935 Wien@\textsc{Goldmann, Paul} (31.\,1.\,1865 Breslau – 25.\,9.\,1935 Wien), \emph{Schriftsteller, Journalist}|pwk} dort täglich getroffen.}}}\label{K_L03537-1} und hat mir
               Mancherlei über die Rothe-Sterngaſſe\pwindex{Steinrück, Elisabeth 19.\,11.\,1885 – 7.\,4.\,1920 Partenkirchen@\textsc{Steinrück, Elisabeth} (19.\,11.\,1885 – 7.\,4.\,1920 Partenkirchen)|pwv}\oindex{Wien@\textbf{Wien}!II., Leopoldstadt@\textbf{II., Leopoldstadt}!Rotensterngasse@\textbf{Rotensterngasse}, \emph{Straße}|pw} berichtet. Insbeſondere, daß es Ihnen gut geht und daß Sie tüchtig vorwärts{ }ſtreben, was ja die Hauptſache iſt. Ich wäre gern, gern wieder einmal mit Ihnen
               zuſammen. Berlin\oindex{Berlin@\textbf{Berlin}, \emph{Hauptstadt}|pw} iſt eine große Stadt, aber
                  \label{K_L03537-2v}\edtext{eine Rothe-Sterngaſſe\oindex{Wien@\textbf{Wien}!II., Leopoldstadt@\textbf{II., Leopoldstadt}!Rotensterngasse@\textbf{Rotensterngasse}, \emph{Straße}|pw} gibt es hier nicht}{\lemma{\textnormal{\emph{eine … nicht}}}\Cendnote{\textnormal{Die Stelle lässt sich auch im Kontext von Goldmanns\pwindex{Goldmann, Paul 31.\,1.\,1865 Breslau – 25.\,9.\,1935 Wien@\textsc{Goldmann, Paul} (31.\,1.\,1865 Breslau – 25.\,9.\,1935 Wien), \emph{Schriftsteller, Journalist}|pwk} (unerwiderter) Schwärmerei für Elisabeth Gussmann\pwindex{Steinrück, Elisabeth 19.\,11.\,1885 – 7.\,4.\,1920 Partenkirchen@\textsc{Steinrück, Elisabeth} (19.\,11.\,1885 – 7.\,4.\,1920 Partenkirchen)|pwkv} lesen, vgl.
                     deren Korrespondenz: \emph{DLA}, HS.1985.1.5246.
               }}}\label{K_L03537-2}. Und ich bin{ }ſehr einſam.\pend
           
\pstart
           Sie{ }ſollen mir bald wieder{ }ſchreiben, Sie und Ihr Fräulein Schweſter\pwindex{Steinrück, Elisabeth 19.\,11.\,1885 – 7.\,4.\,1920 Partenkirchen@\textsc{Steinrück, Elisabeth} (19.\,11.\,1885 – 7.\,4.\,1920 Partenkirchen)|pwv}, das Sie{ }ſelbſt die »kleine
               Beſtie« nennen. (Ich wage kaum, es niederzuſchreiben). Auch{ }ſollten Sie Beide\pwindex{Steinrück, Elisabeth 19.\,11.\,1885 – 7.\,4.\,1920 Partenkirchen@\textsc{Steinrück, Elisabeth} (19.\,11.\,1885 – 7.\,4.\,1920 Partenkirchen)|pwv} nach Berlin\oindex{Berlin@\textbf{Berlin}, \emph{Hauptstadt}|pw} kommen. Ich werde Sie fürſtlich aufnehmen, {\pb}und Sie dürfen bei \textsc{Josty\oindex{Café Josty@\textbf{Café Josty}, \emph{Kaffeehaus}|pw}} einen ganzen Tag lang Indianerkrapfen mit Schlagobers eſſen.\pend
           
\pstart
           Im \label{K_L03537-3v}\edtext{Theater}{\lemma{\textnormal{\emph{Theater}}}\Cendnote{\textnormal{Friedrich Hebbels\pwindex{Hebbel, Friedrich 18.\,3.\,1813 Wesselburen – 13.\,12.\,1863 Wien@\textsc{Hebbel, Friedrich} (18.\,3.\,1813 Wesselburen – 13.\,12.\,1863 Wien), \emph{Schriftsteller}|pwk}{ }\emph{Agnes Bernauer}\pwindex{Hebbel, Friedrich 18.\,3.\,1813 Wesselburen – 13.\,12.\,1863 Wien@\textsc{Hebbel, Friedrich} (18.\,3.\,1813 Wesselburen – 13.\,12.\,1863 Wien), \emph{Schriftsteller}!Agnes Bernauer@\strich\emph{Agnes Bernauer}|pwk} wurde am Berlin\oindex{Berlin@\textbf{Berlin}, \emph{Hauptstadt}|pwk}er \emph{Schauspielhaus}\orgindex{Schauspielhaus Berlin@Schauspielhaus Berlin|pwk} gegeben. Tolstois\pwindex{Tolstoi, Lew Nikolajewitsch 9.\,9.\,1828 Yasnaya Polyana – 20.\,11.\,1910 Lev Tolstoy@\textsc{Tolstoi, Lew Nikolajewitsch} (9.\,9.\,1828 Yasnaya Polyana – 20.\,11.\,1910 Lev Tolstoy), \emph{Schriftsteller}|pwk}{ }\emph{Die
                     Macht der Finsternis}\pwindex{Tolstoi, Lew Nikolajewitsch 9.\,9.\,1828 Yasnaya Polyana – 20.\,11.\,1910 Lev Tolstoy@\textsc{Tolstoi, Lew Nikolajewitsch} (9.\,9.\,1828 Yasnaya Polyana – 20.\,11.\,1910 Lev Tolstoy), \emph{Schriftsteller}!Macht der Finsternis@\strich\emph{Die Macht der Finsternis}|pwk} stand am Spielplan des \emph{Deutschen Theaters}\orgindex{Deutsches Theater Berlin@Deutsches Theater Berlin|pwk}. Am \emph{Berliner
                     Theater}\orgindex{Berliner Theater@Berliner Theater|pwk} wurde \emph{Frauenherrschaft. Lustspiel
                     in vier Aufzügen nach Aristophanes’ »Ekklesiazusen« und »Lysistrate«}\pwindex{Aristophanes 0445? v. Chr. – 0385? v. Chr.@\textsc{Aristophanes} (0445? v. Chr. – 0385? v. Chr.), \emph{Schriftsteller}!Frauenherrschaft. Lustspiel in vier Aufzügen nach Aristophanes’ »Ekklesiazusen« und »Lysistrate«@\strich\emph{Frauenherrschaft. Lustspiel in vier Aufzügen nach Aristophanes’ »Ekklesiazusen« und »Lysistrate«}|pwk} von
                     Adolf von Wilbrandt\pwindex{Wilbrandt, Adolf von 24.\,8.\,1837 Rostock – 10.\,6.\,1911 ebd.@\textsc{Wilbrandt, Adolf von} (24.\,8.\,1837 Rostock – 10.\,6.\,1911 ebd.), \emph{Schriftsteller, Theaterleiter, Schauspieler}|pwk} gespielt.}}}\label{K_L03537-3}
               erleben wir allerlei Gutes: \textsc{Tolstois\pwindex{Tolstoi, Lew Nikolajewitsch 9.\,9.\,1828 Yasnaya Polyana – 20.\,11.\,1910 Lev Tolstoy@\textsc{Tolstoi, Lew Nikolajewitsch} (9.\,9.\,1828 Yasnaya Polyana – 20.\,11.\,1910 Lev Tolstoy), \emph{Schriftsteller}|pw}} »Macht der Finſterniß\pwindex{Tolstoi, Lew Nikolajewitsch 9.\,9.\,1828 Yasnaya Polyana – 20.\,11.\,1910 Lev Tolstoy@\textsc{Tolstoi, Lew Nikolajewitsch} (9.\,9.\,1828 Yasnaya Polyana – 20.\,11.\,1910 Lev Tolstoy), \emph{Schriftsteller}!Macht der Finsternis@\strich\emph{Die Macht der Finsternis}|pw}«, \textsc{Hebbel\pwindex{Hebbel, Friedrich 18.\,3.\,1813 Wesselburen – 13.\,12.\,1863 Wien@\textsc{Hebbel, Friedrich} (18.\,3.\,1813 Wesselburen – 13.\,12.\,1863 Wien), \emph{Schriftsteller}|pw}\textcolor{gray}{’}s} herrliche »\textsc{Agnes Bernauer\pwindex{Hebbel, Friedrich 18.\,3.\,1813 Wesselburen – 13.\,12.\,1863 Wien@\textsc{Hebbel, Friedrich} (18.\,3.\,1813 Wesselburen – 13.\,12.\,1863 Wien), \emph{Schriftsteller}!Agnes Bernauer@\strich\emph{Agnes Bernauer}|pw}}«, ein wenig \textsc{Aristophanes\pwindex{Aristophanes 0445? v. Chr. – 0385? v. Chr.@\textsc{Aristophanes} (0445? v. Chr. – 0385? v. Chr.), \emph{Schriftsteller}!Frauenherrschaft. Lustspiel in vier Aufzügen nach Aristophanes’ »Ekklesiazusen« und »Lysistrate«@\strich\emph{Frauenherrschaft. Lustspiel in vier Aufzügen nach Aristophanes’ »Ekklesiazusen« und »Lysistrate«}|pwv}\pwindex{Aristophanes 0445? v. Chr. – 0385? v. Chr.@\textsc{Aristophanes} (0445? v. Chr. – 0385? v. Chr.), \emph{Schriftsteller}|pw} etc}.\pend
           
\pstart
           Wenn Sie unſeren lieben \textsc{Dr. Arthur Schnitzler}{ }ſehen,{ }ſo{ }ſagen Sie ihm: 1.) daß er mir eine Ewigkeit nicht geſchrieben hat und
               daß dies eine Infamie iſt 2.) daß \textsc{Alfred Klaar\pwindex{Klaar, Alfred 7.\,11.\,1848 Prag – 4.\,11.\,1927 Berlin@\textsc{Klaar, Alfred} (7.\,11.\,1848 Prag – 4.\,11.\,1927 Berlin), \emph{Schriftsteller, Kritiker}|pw}}, der ehemalige Kritiker der »\textsc{Bohemia\orgindex{Bohemia@Bohemia|pw}}«, ein Schmock in Reincultur, der ödeſte und blödeſte Schwätzer der
                  Jetztzeit{[},{]} Theaterkritiker und Feuilleton-Redakteur der »Voſſiſchen Zeitung\orgindex{Vossische Zeitung@Vossische Zeitung|pw}« geworden iſt. Auch ich hatte
               mich für die Stelle gemeldet, {\pb}bekam aber nicht einmal
               eine Antwort. Ich bin nämlich (aber{ }ſagen Sie es nicht weiter!) \label{K_L03537-4v}\edtext{nicht »literariſch«}{\lemma{\textnormal{\emph{nicht »literarisch«}}}\Cendnote{\textnormal{Diesen vermeintlichen Vorbehalt gegenüber seiner Person und
                  dem Beruf des Kritikers an sich hatte Goldmann\pwindex{Goldmann, Paul 31.\,1.\,1865 Breslau – 25.\,9.\,1935 Wien@\textsc{Goldmann, Paul} (31.\,1.\,1865 Breslau – 25.\,9.\,1935 Wien), \emph{Schriftsteller, Journalist}|pwk} in Briefen an Schnitzler
                  bereits mehrmals thematisiert, beispielsweise XXXX Auszeichnungsfehler: Dokument L02917 nicht gefunden.}}}\label{K_L03537-4}.\pend
           
\pstart
           Ich wünſche Ihnen und dem Fräulein \textsc{Liesl\pwindex{Steinrück, Elisabeth 19.\,11.\,1885 – 7.\,4.\,1920 Partenkirchen@\textsc{Steinrück, Elisabeth} (19.\,11.\,1885 – 7.\,4.\,1920 Partenkirchen)|pw}} frohe Weihnachten, bitte Sie, meinen
               Namensvetter \textsc{Paul\pwindex{Marx, Paul 21.\,7.\,1879 Wien – 30.\,10.\,1956 ebd.@\textsc{Marx, Paul} (21.\,7.\,1879 Wien – 30.\,10.\,1956 ebd.), \emph{Regisseur, Schauspieler}|pw}} zu grüßen, hoffe, bald wieder durch einen Brief erfreut zu werden, und küſſe
               Ihnen Beiden\pwindex{Steinrück, Elisabeth 19.\,11.\,1885 – 7.\,4.\,1920 Partenkirchen@\textsc{Steinrück, Elisabeth} (19.\,11.\,1885 – 7.\,4.\,1920 Partenkirchen)|pwv} je eine Hand.
               {\\[\baselineskip]}Ihr freundſchaftlich ergebener {\\[\baselineskip]}\spacefill\mbox{Dr. Paul Goldmann.}\pend
           \leftskip=0em{}\selectlanguage{ngerman}\endnumbering\briefempfaengerindex{Schnitzler, Olga@\textsc{Schnitzler, Olga}!zzzGoldmann, Paul@\emph{von Paul Goldmann}!1900-12-202@{20. 12. [1900]}|)be}\mylabel{L03537h}  \newcommand{\dateiname}{L03537}\newcommand{\titel}{Paul Goldmann an Olga Gussmann, 20. 12. [1900]}\newcommand{\editorInnen}{Martin Anton Müller und Laura Untner}%% latex-leseansicht-abspann.tex
%% Abspann für die Leseansicht.
%% Der Schalter \ifkorrekturansicht ist bereits durch den Vorspann gesetzt.

%% latex-abspann.tex
%% Gemeinsamer Abspann für Korrekturansicht und Leseansicht.
%% Setzt den Schalter \ifkorrekturansicht voraus (gesetzt in den
%% einbindenden Dateien latex-korrekturansicht-abspann.tex bzw.
%% latex-leseansicht-abspann.tex).
%% ---------------------------------------------------------------

\normalsize

% Das esempio-Environment wird nur in der Leseansicht benötigt
\ifkorrekturansicht\else
\newenvironment{esempio}[3]%
{
    \vspace{1.5ex}
    \rlap{\underline{#1}}
    \par
    \setlength{\parindent}{0cm}
    \nopagebreak
    \leftskip=#2cm
    \rightskip=#3cm
}
{
    \par
}
\fi

\doendnotes{C}
\bigskip
\vfill

\clearpage

\footnotesize

\ifkorrekturansicht
  \lohead{\textsc{register}}
\fi

% theindex-Environment neu definieren ohne reledmac
\makeatletter
\renewenvironment{theindex}{%
  \ifkorrekturansicht
    \section*{\indexname}%
  \else
    \subsubsection*{Index der erwähnten Entitäten}%
  \fi
  \setlength{\parindent}{0pt}%
  \setlength{\parskip}{0pt plus 0.3pt}%
  \let\item\@idxitem
}{%
  \ifkorrekturansicht\clearpage\fi
}
\makeatother

\IfFileExists{\jobname-pw.ind}{\input{\jobname-pw.ind}}{}

% Quellenangabe nur in der Leseansicht
\ifkorrekturansicht\else
% Fallback-Definitionen, falls die .tex-Datei \titel etc. nicht gesetzt hat
\providecommand{\titel}{}
\providecommand{\editorInnen}{}
\providecommand{\dateiname}{\jobname}

\vspace{3cm}

\vfill

\footnotesize
\textsc{Quelle}: \titel. Herausgegeben von {\editorInnen}. In: \emph{Arthur Schnitzler: Briefwechsel mit Autorinnen und Autoren}.
 Digitale Edition, https://schnitzler-briefe.acdh.oeaw.ac.at/{\dateiname}.html (Stand \today)
\fi

\end{document}


