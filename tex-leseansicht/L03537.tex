%% latex-leseansicht-vorspann.tex
%% Vorspann für die Leseansicht.
%% Lädt die gemeinsame Datei latex-vorspann.tex mit nicht gesetztem Schalter.

\newif\ifkorrekturansicht
\korrekturansichtfalse

\input{../tex-inputs/latex-vorspann}

\begin{center}
            \textcolor{red}{ENTWURF, NICHT FERTIG KORRIGIERT}
                      \end{center}
            
         
         \renewcommand{\erwaehntePersonen}{Personen:  Aristophanes, Paul Goldmann, Friedrich Hebbel, Alfred Klaar, Paul Marx, Olga Schnitzler, Elisabeth Steinrück, Leo N. von Tolstoi, Adolf von Wilbrandt}
         \renewcommand{\erwaehnteInstitutionen}{Institutionen: Bohemia, Deutsches Theater Berlin, Vossische Zeitung}
         \renewcommand{\erwaehnteOrte}{Orte: Berlin, Berliner Theater, Café Josty, Dessauer Straße, Rotensterngasse, Schauspielhaus, Wien}
         \renewcommand{\erwaehnteWerke}{Werke: Agnes Bernauer, Die Macht der Finsternis, Frauenherrschaft. Lustspiel in vier Aufzügen nach Aristophanes’ »Ekklesiazusen« und »Lysistrate«}
               \section[ Paul Goldmann an Olga Gussmann, 20. 12. {[}1900{]}]{ Paul Goldmann an Olga Gussmann, 20. 12. {[}1900{]}}\nopagebreak\mylabel{v}\rehead{ }\begin{ledgroupsized}[t]{13cm}\normalsize\beginnumbering \toendnotes[C]{\smallbreak\pagebreak[2]} \Standort{DLA, A:Schnitzler, HS.NZ85.1.5247.}
\physDesc{Brief, 1 Blatt, 4 Seiten, 2177 Zeichen
\newline{}Handschrift: blaue Tinte, deutsche Kurrent}\toendnotes[C]{\smallbreak}\pstart
           \noindent{}\raggedleft{}{\pb}\textcolor{gray}{\textbf{DESSAUERSTRASSE 19\oindex{Dessauer Strasse@\textbf{Dessauer Straße}|pw}}}\pend
           \pstart
           Berlin\oindex{Berlin@\textbf{Berlin}|pw}, 20. Dezember.\pend
           \pstart{}Verehrtes und liebes Fräulein,\pend\pstart
           Die Briefe, die Sie und Ihr Schweſterchen\pwindex{Steinrueck, Elisabeth 19.11.1885 – 07.04.1920@\textsc{Steinrück, Elisabeth} (19.11.1885 – 07.04.1920)|pwv} mir geſchrieben, haben mir \strikeout{g\textcolor{gray}{×}\-\textcolor{gray}{×}} große Freude bereitet. Seit Wochen liegen ſie auf dem Schreibtiſch – ganz
               obenauf, um raſch zur Hand zu ſein für den Fall, daß die Stunde des Briefſchreibens
               kommen ſollte. Aber die Stunde iſt bisher nicht gekommen, wird auch wohl ſo bald
               nicht kommen in meinem vielgeplagten Berichterſtatter-Daſein, und das, was ich Ihnen
                  heut ſchreibe, iſt eigentlich kein Brief, ſondern
               es ſind nur drei kurze Worte des Dankes und des herzlichen Gedankens, die doch
               endlich einmal geſagt werden mußten, Ihnen {\pb}ſowohl,
               wie dem Fräulein \textsc{Liesl\pwindex{Steinrueck, Elisabeth 19.11.1885 – 07.04.1920@\textsc{Steinrück, Elisabeth} (19.11.1885 – 07.04.1920)|pw}}.\pend
           \pstart
           Inzwiſchen war \textsc{Dr. Schnitzler\pwindex{Schnitzler, Arthur 15.05.1862 – 21.10.1931@\textsc{Schnitzler, Arthur} (15.05.1862 – 21.10.1931), \emph{Schriftsteller, Mediziner}|pw}} in \strikeout{Wien\oindex{Wien@\textbf{Wien}|pw}}{ }\label{K_L03537-1v}\edtext{Berlin\oindex{Berlin@\textbf{Berlin}|pw}}{\lemma{\textnormal{\emph{Berlin}}}\Cendnote{\textnormal{Da Schnitzler\pwindex{Schnitzler, Arthur 15.05.1862 – 21.10.1931@\textsc{Schnitzler, Arthur} (15.05.1862 – 21.10.1931), \emph{Schriftsteller, Mediziner}|pwk} zwischen 24. 11. 1900 und 28. 11. 1900 in Berlin\oindex{Berlin@\textbf{Berlin}|pwk} gewesen
                  war, wo er Goldmann\pwindex{Goldmann, Paul 31.01.1865 – 25.09.1935@\textsc{Goldmann, Paul} (31.01.1865 – 25.09.1935), \emph{Schriftsteller, Journalist}|pwk} täglich getroffen
                  hatte, ist der Brief auf das Jahr 1900 datierbar.}}}\label{K_L03537-1h}
               und hat mir Mancherlei über die Rothe-Sterngaſſe\pwindex{Steinrueck, Elisabeth 19.11.1885 – 07.04.1920@\textsc{Steinrück, Elisabeth} (19.11.1885 – 07.04.1920)|pwv}\oindex{Rotensterngasse@\textbf{Rotensterngasse}|pw} berichtet. Insbeſondere, daß es Ihnen gut geht und daß Sie tüchtig vorwärts
               ſtreben, was ja die Hauptſache iſt. Ich wäre gern, gern wieder einmal mit Ihnen
               zuſammen. Berlin\oindex{Berlin@\textbf{Berlin}|pw} iſt eine große Stadt, aber
                  \label{K_L03537-2v}\edtext{eine Rothe-Sterngaſſe\oindex{Rotensterngasse@\textbf{Rotensterngasse}|pw} gibt es hier nicht}{\lemma{\textnormal{\emph{eine … nicht}}}\Cendnote{\textnormal{auch zu lesen im Hinblick auf Goldmann\pwindex{Goldmann, Paul 31.01.1865 – 25.09.1935@\textsc{Goldmann, Paul} (31.01.1865 – 25.09.1935), \emph{Schriftsteller, Journalist}|pwk}s (unerwiderte) Schwärmerei für Elisabeth Gussmann\pwindex{Steinrueck, Elisabeth 19.11.1885 – 07.04.1920@\textsc{Steinrück, Elisabeth} (19.11.1885 – 07.04.1920)|pwkv}, siehe deren
                     Korrespondenz: \emph{DLA}, HS.1985.1.5246}}}\label{K_L03537-2h}. Und ich bin ſehr einſam.\pend
           \pstart
           Sie ſollen mir bald wieder ſchreiben, Sie und Ihr Fräulein Schweſter\pwindex{Steinrueck, Elisabeth 19.11.1885 – 07.04.1920@\textsc{Steinrück, Elisabeth} (19.11.1885 – 07.04.1920)|pwv}, das Sie ſelbſt die »kleine
               Beſtie« nennen. (Ich wage kaum, es niederzuſchreiben). Auch ſollten Sie Beide\pwindex{Steinrueck, Elisabeth 19.11.1885 – 07.04.1920@\textsc{Steinrück, Elisabeth} (19.11.1885 – 07.04.1920)|pwv} nach Berlin\oindex{Berlin@\textbf{Berlin}|pw} kommen. Ich werde Sie fürſtlich aufnehmen, {\pb}und Sie dürfen bei \textsc{Josty\oindex{Cafe Josty@\textbf{Café Josty}|pw}} einen ganzen Tag lang Indianerkrapfen mit Schlagobers eſſen.\pend
           \pstart
           Im \label{K_L03537-3v}\edtext{Theater}{\lemma{\textnormal{\emph{Theater}}}\Cendnote{\textnormal{Friedrich Hebbel\pwindex{Hebbel, Friedrich 18.03.1813 – 13.12.1863@\textsc{Hebbel, Friedrich} (18.03.1813 – 13.12.1863), \emph{Schriftsteller}|pwk}s \emph{Agnes Bernauer}\pwindex{Hebbel, Friedrich 18.03.1813 – 13.12.1863@\textsc{Hebbel, Friedrich} (18.03.1813 – 13.12.1863), \emph{Schriftsteller}!Agnes Bernauer1852@\strich\emph{Agnes Bernauer} {[}1852{]}|pwk} wurde am Berlin\oindex{Berlin@\textbf{Berlin}|pwk}er Schauspielhaus\oindex{Schauspielhaus@\textbf{Schauspielhaus}|pwk} gespielt. Leo N. von Tolstoi\pwindex{Tolstoi, Leo N. von 9.09.1828 – 20.11.1910@\textsc{Tolstoi, Leo N. von} (9.09.1828 – 20.11.1910), \emph{Schriftsteller}|pwk}s \emph{Die Macht der Finsternis}\pwindex{Tolstoi, Leo N. von 9.09.1828 – 20.11.1910@\textsc{Tolstoi, Leo N. von} (9.09.1828 – 20.11.1910), \emph{Schriftsteller}!Macht der Finsternis1886@\strich\emph{Die Macht der Finsternis} {[}1886{]}|pwk} stand am Spielplan des \emph{Deutschen Theater}\orgindex{Deutsches Theater Berlin@Deutsches Theater Berlin|pwk}s. Am Berliner Theater\oindex{Berliner Theater@\textbf{Berliner Theater}|pwk} wurde \emph{Frauenherrschaft. Lustspiel in vier Aufzügen nach Aristophanes’
                     »Ekklesiazusen« und »Lysistrate«}\pwindex{Wilbrandt, Adolf von 24.08.1837 – 10.06.1911@\textsc{Wilbrandt, Adolf von} (24.08.1837 – 10.06.1911), \emph{Schriftsteller, Theaterleiter, Schauspieler}!Frauenherrschaft. Lustspiel in vier Aufzuegen nach Aristophanes  »Ekklesiazusen« und »Lysistrate«1892@\strich\emph{Frauenherrschaft. Lustspiel in vier Aufzügen nach Aristophanes’ »Ekklesiazusen« und »Lysistrate«} {[}Übersetzung, 1892{]}|pwk} von Adolf von Wilbrandt\pwindex{Wilbrandt, Adolf von 24.08.1837 – 10.06.1911@\textsc{Wilbrandt, Adolf von} (24.08.1837 – 10.06.1911), \emph{Schriftsteller, Theaterleiter, Schauspieler}|pwk} gegeben.}}}\label{K_L03537-3h} erleben wir allerlei Gutes: \textsc{Tolstoi\pwindex{Tolstoi, Leo N. von 9.09.1828 – 20.11.1910@\textsc{Tolstoi, Leo N. von} (9.09.1828 – 20.11.1910), \emph{Schriftsteller}|pw}s} »Macht der Finſterniß\pwindex{Tolstoi, Leo N. von 9.09.1828 – 20.11.1910@\textsc{Tolstoi, Leo N. von} (9.09.1828 – 20.11.1910), \emph{Schriftsteller}!Macht der Finsternis1886@\strich\emph{Die Macht der Finsternis} {[}1886{]}|pw}«, \textsc{Hebbel\pwindex{Hebbel, Friedrich 18.03.1813 – 13.12.1863@\textsc{Hebbel, Friedrich} (18.03.1813 – 13.12.1863), \emph{Schriftsteller}|pw}\textcolor{gray}{’}s} herrliche »\textsc{Agnes Bernauer\pwindex{Hebbel, Friedrich 18.03.1813 – 13.12.1863@\textsc{Hebbel, Friedrich} (18.03.1813 – 13.12.1863), \emph{Schriftsteller}!Agnes Bernauer1852@\strich\emph{Agnes Bernauer} {[}1852{]}|pw}}«, ein wenig \textsc{Aristophanes\pwindex{Wilbrandt, Adolf von 24.08.1837 – 10.06.1911@\textsc{Wilbrandt, Adolf von} (24.08.1837 – 10.06.1911), \emph{Schriftsteller, Theaterleiter, Schauspieler}!Frauenherrschaft. Lustspiel in vier Aufzuegen nach Aristophanes  »Ekklesiazusen« und »Lysistrate«1892@\strich\emph{Frauenherrschaft. Lustspiel in vier Aufzügen nach Aristophanes’ »Ekklesiazusen« und »Lysistrate«} {[}Übersetzung, 1892{]}|pwv}\pwindex{Aristophanes 0445? v. Chr. – 0385? v. Chr.@\textsc{Aristophanes} (0445? v. Chr. – 0385? v. Chr.), \emph{Schriftsteller}|pw} etc}.\pend
           \pstart
           Wenn Sie unſeren lieben \textsc{Dr. Arthur Schnitzler\pwindex{Schnitzler, Arthur 15.05.1862 – 21.10.1931@\textsc{Schnitzler, Arthur} (15.05.1862 – 21.10.1931), \emph{Schriftsteller, Mediziner}|pw}} ſehen, ſo ſagen Sie ihm: 1.) daß er mir eine Ewigkeit nicht geſchrieben hat und
               daß dies eine Infamie iſt 2.) daß \textsc{Alfred Klaar\pwindex{Klaar, Alfred 07.11.1848 – 04.11.1927@\textsc{Klaar, Alfred} (07.11.1848 – 04.11.1927), \emph{Schriftsteller, Kritiker}|pw}}, der ehemalige Kritiker der »\textsc{Bohemia\orgindex{Bohemia@Bohemia|pw}}«, ein Schmock in Reincultur, der ödeſte und blödeſte Schwätzer der
                  Jetztzeit{[},{]} Theaterkritiker und Feuilleton-Redakteur der »Voſſiſchen Zeitung\orgindex{Vossische Zeitung@Vossische Zeitung|pw}« geworden iſt. Auch ich hatte
               mich für die Stelle gemeldet, {\pb}bekam aber nicht einmal
               eine Antwort. Ich bin nämlich (aber ſagen Sie es nicht weiter!) \label{K_L03537-4v}\edtext{nicht »literariſch«}{\lemma{\textnormal{\emph{nicht »literariſch«}}}\Cendnote{\textnormal{Diesen vermeintlichen Vorbehalt gegenüber seiner Person und
                  dem Beruf des Kritikers an sich hatte Goldmann\pwindex{Goldmann, Paul 31.01.1865 – 25.09.1935@\textsc{Goldmann, Paul} (31.01.1865 – 25.09.1935), \emph{Schriftsteller, Journalist}|pwk} in Briefen an Schnitzler\pwindex{Schnitzler, Arthur 15.05.1862 – 21.10.1931@\textsc{Schnitzler, Arthur} (15.05.1862 – 21.10.1931), \emph{Schriftsteller, Mediziner}|pwk}
                  bereits mehrmals thematisiert. Siehe zum Beispiel Paul Goldmann an Arthur Schnitzler, 29. 5. [1900].}}}\label{K_L03537-4h}.\pend
           \pstart
           Ich wünſche Ihnen und dem Fräulein \textsc{Liesl\pwindex{Steinrueck, Elisabeth 19.11.1885 – 07.04.1920@\textsc{Steinrück, Elisabeth} (19.11.1885 – 07.04.1920)|pw}} frohe Weihnachten, bitte Sie, meinen
               Namensvetter \textsc{Paul\pwindex{Marx, Paul 21.07.1879 – 1956-10-30@\textsc{Marx, Paul} (21.07.1879 – 1956-10-30), \emph{Regisseur, Schauspieler}|pw}} zu grüßen, hoffe, bald wieder durch einen Brief erfreut zu werden, und küſſe
               Ihnen Beiden\pwindex{Steinrueck, Elisabeth 19.11.1885 – 07.04.1920@\textsc{Steinrück, Elisabeth} (19.11.1885 – 07.04.1920)|pwv} je eine Hand.
               {\\[\baselineskip]}Ihr freundſchaftlich ergebener {\\[\baselineskip]}\spacefill\mbox{Dr. Paul Goldmann.}\pend
           \leftskip=0em{}
         
         \endnumbering\mylabel{h}\end{ledgroupsized}\begin{anhang}\end{anhang}\newcommand{\dateiname}{L03537}\newcommand{\titel}{Paul Goldmann an Olga Gussmann, 20. 12. [1900]}\newcommand{\editorInnen}{Martin Anton Müller und Laura Untner}%% latex-leseansicht-abspann.tex
%% Abspann für die Leseansicht.
%% Der Schalter \ifkorrekturansicht ist bereits durch den Vorspann gesetzt.

%% latex-abspann.tex
%% Gemeinsamer Abspann für Korrekturansicht und Leseansicht.
%% Setzt den Schalter \ifkorrekturansicht voraus (gesetzt in den
%% einbindenden Dateien latex-korrekturansicht-abspann.tex bzw.
%% latex-leseansicht-abspann.tex).
%% ---------------------------------------------------------------

\normalsize

% Das esempio-Environment wird nur in der Leseansicht benötigt
\ifkorrekturansicht\else
\newenvironment{esempio}[3]%
{
    \vspace{1.5ex}
    \rlap{\underline{#1}}
    \par
    \setlength{\parindent}{0cm}
    \nopagebreak
    \leftskip=#2cm
    \rightskip=#3cm
}
{
    \par
}
\fi

\doendnotes{C}
\bigskip
\vfill

\clearpage

\footnotesize

\ifkorrekturansicht
  \lohead{\textsc{register}}
\fi

% theindex-Environment neu definieren ohne reledmac
\makeatletter
\renewenvironment{theindex}{%
  \ifkorrekturansicht
    \section*{\indexname}%
  \else
    \subsubsection*{Index der erwähnten Entitäten}%
  \fi
  \setlength{\parindent}{0pt}%
  \setlength{\parskip}{0pt plus 0.3pt}%
  \let\item\@idxitem
}{%
  \ifkorrekturansicht\clearpage\fi
}
\makeatother

\IfFileExists{\jobname-pw.ind}{\input{\jobname-pw.ind}}{}

% Quellenangabe nur in der Leseansicht
\ifkorrekturansicht\else
% Fallback-Definitionen, falls die .tex-Datei \titel etc. nicht gesetzt hat
\providecommand{\titel}{}
\providecommand{\editorInnen}{}
\providecommand{\dateiname}{\jobname}

\vspace{3cm}

\vfill

\footnotesize
\textsc{Quelle}: \titel. Herausgegeben von {\editorInnen}. In: \emph{Arthur Schnitzler: Briefwechsel mit Autorinnen und Autoren}.
 Digitale Edition, https://schnitzler-briefe.acdh.oeaw.ac.at/{\dateiname}.html (Stand \today)
\fi

\end{document}


      