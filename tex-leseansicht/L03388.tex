%% latex-korrekturansicht-vorspann.tex
%% Vorspann für die Korrekturansicht.
%% Lädt die gemeinsame Datei latex-vorspann.tex mit gesetztem Schalter.

\newif\ifkorrekturansicht
\korrekturansichttrue

\input{../tex-inputs/latex-vorspann}


\section[ Paul Goldmann an Arthur Schnitzler, 14. 11. {[}1903{]}]{L03388 Paul Goldmann an Arthur Schnitzler, 14. 11. {[}1903{]}}
\nopagebreak\mylabel{L03388v}
\rehead{ }\normalsize\beginnumbering\briefempfaengerindex{Schnitzler, Arthur@\textsc{Schnitzler, Arthur}!zzzGoldmann, Paul@\emph{von Paul Goldmann}!1903-11-141@{14. 11. {[}1903{]}}|(be}
\toendnotes[C]{\smallbreak\pagebreak[2]}\Standort{DLA, A:Schnitzler, HS.NZ85.1.3173.}
\physDesc{Brief, 1 Blatt, 4 Seiten, 1938 Zeichen
\newline{}Handschrift: blaue Tinte, deutsche Kurrent
\newline{}Schnitzler: 1) mit Bleistift das Jahr »903.« vermerkt  2) mit Bleistift auf einem beigelegten Blatt mutmaßlich eine Antwortskizze
                                 notiert, die nur unzuverlässig zu entziffern ist: »{\pb}{ / }\textcolor{gray}{\textbf{Recept.}}{ / }\uline{Nordau\pwindex{Nordau, Max 29.07.1849 – 22.01.1923@\textsc{Nordau, Max} (29.07.1849 – 22.01.1923), \emph{Schriftsteller/Schriftstellerin, Mediziner/Medizinerin}|pw}} – hat falſch berichte d Kriti – ſind
                                          \textcolor{gray}{au} glzd. –
                                          (folgende.) – \textcolor{gray}{u}{ / }tagl: d Kritik wei\textcolor{gray}{ß} nicht
                                       anzufangen! –){ / }–{ / }Dor{[}a{]}
                                          Popper\pwindex{Popper, Emilie Dorothea 1893-10-08 – 1933-11-24@\textsc{Popper, Emilie Dorothea} (1893-10-08 – 1933-11-24), \emph{Pianist/Pianistin, Pädagoge/Pädagogin}|pw} – Paul\pwindex{Marx, Paul 21.07.1879 – 1956-10-30@\textsc{Marx, Paul} (21.07.1879 – 1956-10-30), \emph{Regisseur/Regisseurin, Schauspieler/Schauspielerin}|pwu}\textcolor{gray}{!}{ / }–{ / }Nicht leider unſich und« 3) mit rotem Buntstift zwei Unterstreichungen}\toendnotes[C]{\smallbreak}
\pstart
           \raggedleft{}{\pb}\textcolor{gray}{\textbf{DESSAUERSTRASSE 19\oindex{Dessauer Strasse@\textbf{Dessauer Straße}, \emph{Straße (K.STR)}|pw}}}\pend
           
\pstart
           Berlin\oindex{Berlin@\textbf{Berlin}, \emph{P.PPLC}|pw}, 14. November.\pend
           
\pstart\center{}Mein lieber Freund,\pend\vspace{0.5em}
\pstart
           Verzeih mir, daß ich ſo \label{K_L03388-1v}\edtext{lange nicht
                  geſchrieben}{\lemma{\textnormal{\emph{lange nicht
                  geſchrieben}}}\Cendnote{\textnormal{Vgl. Arthur Schnitzler an Hugo von Hofmannsthal, 4. 11. 1903. }}}\label{K_L03388-1} habe. Ich
               lebe ſeit meiner \label{K_L03388-2v}\edtext{Rückkehr}{\lemma{\textnormal{\emph{Rückkehr}}}\Cendnote{\textnormal{Siehe Paul Goldmann an Arthur Schnitzler, 7. 9. 1903. }}}\label{K_L03388-2} in
               fortwährend wechſelnden Stimmungen, in vielen Sorgen und Widrigkeiten. Eine große
               Müdigkeit hielt mich vom Schreiben zurück. Im Grunde \strikeout{iſ\textcolor{gray}{t}} bleibt doch immer Alles beim Alten. Wozu alſo ſchreiben?\pend
           
\pstart
           Deine lieben Nachrichten haben mir ſehr gefehlt. Warum haſt \uline{Du} mir denn nicht geſchrieben? Sind wir denn ſo formell geworden, daß
               Einer auf des Andern Brief wartet, um ihm Nachricht von ſich zu geben? Geſtern{ }{\pb}habe ich endlich durch \textsc{Liesl\pwindex{Steinrueck, Elisabeth 19.11.1885 – 07.04.1920@\textsc{Steinrück, Elisabeth} (19.11.1885 – 07.04.1920)|pw}}, die ich bei den \label{K_L03388-3v}\edtext{»Böſen Buben\oindex{Die boesen Buben@\textbf{Die bösen Buben}, \emph{Kabarett (K.KBR)}|pw}«}{\lemma{\textnormal{\emph{»Böſen Buben«}}}\Cendnote{\textnormal{Die bösen Buben\oindex{Die boesen Buben@\textbf{Die bösen Buben}, \emph{Kabarett (K.KBR)}|pwk} war der Name eines Berlin\oindex{Berlin@\textbf{Berlin}, \emph{P.PPLC}|pwk}er Kabaretts, das 1901 von Rudolf Bernauer\pwindex{Bernauer, Rudolf 20.01.1880 – 27.11.1953@\textsc{Bernauer, Rudolf} (20.01.1880 – 27.11.1953), \emph{Theaterleiter/Theaterleiterin, Schauspieler/Schauspielerin}|pwk} und Carl Meinhard\pwindex{Meinhard, Carl 28.11.1875 – 12.02.1949@\textsc{Meinhard, Carl} (28.11.1875 – 12.02.1949), \emph{Theaterleiter/Theaterleiterin}|pwk} gegründet worden war und bis
                     1905 bestand.}}}\label{K_L03388-3} ſprach, etwas Näheres über Dich
               erfahren. Ich habe zu meiner großen Freude gehört, daß es Dir, Deiner Frau\pwindex{Schnitzler, Olga 17.01.1882 – 13.01.1970@\textsc{Schnitzler, Olga} (17.01.1882 – 13.01.1970), \emph{Schauspieler/Schauspielerin, Sänger/Sängerin}|pwv} und dem Kinde\pwindex{Schnitzler, Heinrich 09.08.1902 – 12.07.1982@\textsc{Schnitzler, Heinrich} (09.08.1902 – 12.07.1982), \emph{Regisseur/Regisseurin, Schauspieler/Schauspielerin}|pwv} gut geht. Und nicht minder freue ich
               mich über die Ausſicht, Dich bald in \label{K_L03388-4v}\edtext{Berlin\oindex{Berlin@\textbf{Berlin}, \emph{P.PPLC}|pw}}{\lemma{\textnormal{\emph{Berlin}}}\Cendnote{\textnormal{Das nächste Mal war Schnitzler zwischen 5. 2. 1904 und 17. 2. 1904 in Berlin\oindex{Berlin@\textbf{Berlin}, \emph{P.PPLC}|pwk}. Goldmann\pwindex{Goldmann, Paul 31.01.1865 – 25.09.1935@\textsc{Goldmann, Paul} (31.01.1865 – 25.09.1935), \emph{Schriftsteller/Schriftstellerin, Journalist/Journalistin}|pwk} traf er
                  jedenfalls am 7. 2. 1904, 10. 2. 1904 und 16. 2. 1904.}}}\label{K_L03388-4} zu ſehen. Zu Deinen Erfolgen in der letzten Zeit
                  (\label{K_L03388-5v}\edtext{Schill\strikeout{t}ertheater\orgindex{Schiller-Theater@Schiller-Theater|pw}}{\lemma{\textnormal{\emph{Schillertheater}}}\Cendnote{\textnormal{Am 29. 10. 1903 hatte am Berlin\oindex{Berlin@\textbf{Berlin}, \emph{P.PPLC}|pwk}er \emph{Schiller-Theater}\orgindex{Schiller-Theater@Schiller-Theater|pwk} ein »Schnitzler-Abend« mit einer Aufführung von \emph{Liebelei}\pwindex{Liebelei. Schauspiel in drei Akten@\emph{Liebelei. Schauspiel in drei Akten}|pwk} und \emph{Literatur}\pwindex{Literatur@\emph{Literatur}|pwk} stattgefunden.}}}\label{K_L03388-5}, \label{K_L03388-6v}\edtext{Paris\oindex{Paris@\textbf{Paris}, \emph{P.PPLC}|pw}}{\lemma{\textnormal{\emph{Paris}}}\Cendnote{\textnormal{Die Inszenierung von \emph{Au Perroquet vert}\pwindex{Au Perroquet Vert@\emph{Au Perroquet Vert}|pwk} (\emph{Der
                     grüne Kakadu}\pwindex{gruene Kakadu. Groteske in einem Akt@\emph{Der grüne Kakadu. Groteske in einem Akt}|pwk}) wurde im Théatre Antoine\oindex{Theâtre Antoine-Simone Berriau@\textbf{Théâtre Antoine-Simone Berriau}, \emph{Theater (K.THE)}|pwk}
                  zwischen 7. 11. 1903 und 6. 12. 1903 zwölfmal gegeben.}}}\label{K_L03388-6}, \label{K_L03388-7v}\edtext{Bahrs\pwindex{Bahr, Hermann 19.07.1863 – 15.01.1934@\textsc{Bahr, Hermann} (19.07.1863 – 15.01.1934), \emph{Schriftsteller/Schriftstellerin, Kritiker/Kritikerin}|pw}{ }Vorleſung\pwindex{Reigen. Zehn Dialoge@\emph{Reigen. Zehn Dialoge}|pwv}}{\lemma{\textnormal{\emph{Bahrs Vorleſung}}}\Cendnote{\textnormal{Bahr\pwindex{Bahr, Hermann 19.07.1863 – 15.01.1934@\textsc{Bahr, Hermann} (19.07.1863 – 15.01.1934), \emph{Schriftsteller/Schriftstellerin, Kritiker/Kritikerin}|pwk} hatte eine öffentliche Vorlesung des
                     \emph{Reigen}\pwindex{Reigen. Zehn Dialoge@\emph{Reigen. Zehn Dialoge}|pwk} geplant. Letztlich wurde ihm das
                  behördlich untersagt. Vgl. A. S.: \emph{Tagebuch}, 1. 11. 1903; Hermann Bahr, Arthur Schnitzler: \emph{Briefwechsel, Aufzeichnungen, Dokumente (1891–1931)}, Aufzeichnung von Hermann Bahr, 30. 10. 1903.}}}\label{K_L03388-7})
               beglückwünſche ich Dich herzlichſt, und ich hoffe, daß das neue Stück\pwindex{einsame Weg. Schauspiel in fuenf Akten@\emph{Der einsame Weg. Schauspiel in fünf Akten}|pwv} dieſe »ſchöne« Reihe mit Glanz
               fortſetzen wird. Den \label{K_L03388-8v}\edtext{Artikel\pwindex{Deutsche Theaterstuecke in Frankreich@\emph{Deutsche Theaterstücke in Frankreich}|pwv} von \textsc{Nordau\pwindex{Nordau, Max 29.07.1849 – 22.01.1923@\textsc{Nordau, Max} (29.07.1849 – 22.01.1923), \emph{Schriftsteller/Schriftstellerin, Mediziner/Medizinerin}|pw}}}{\lemma{\textnormal{\emph{Artikel von Nordau}}}\Cendnote{\textnormal{M. N.\pwindex{Nordau, Max 29.07.1849 – 22.01.1923@\textsc{Nordau, Max} (29.07.1849 – 22.01.1923), \emph{Schriftsteller/Schriftstellerin, Mediziner/Medizinerin}|pwkv} [ = Max Nordau\pwindex{Nordau, Max 29.07.1849 – 22.01.1923@\textsc{Nordau, Max} (29.07.1849 – 22.01.1923), \emph{Schriftsteller/Schriftstellerin, Mediziner/Medizinerin}|pwk}]: \emph{Deutsche Theaterstücke in Frankreich}\pwindex{Deutsche Theaterstuecke in Frankreich@\emph{Deutsche Theaterstücke in Frankreich}|pwk}. In: \emph{Vossische Zeitung}\pwindex{Vossische Zeitung@\emph{Vossische Zeitung}|pwk}, Nr. 529, 11. 11. 1903, Morgen-Ausgabe,
                  S. [15]–[16].}}}\label{K_L03388-8} ſchickte ich Dir, weil ich es bemerkenswerth fand, daß
               dieſer {\pb}Menſch\pwindex{Nordau, Max 29.07.1849 – 22.01.1923@\textsc{Nordau, Max} (29.07.1849 – 22.01.1923), \emph{Schriftsteller/Schriftstellerin, Mediziner/Medizinerin}|pwv}, der Alles verreißt,
               ſo freundlich über Dich ſprach\pwindex{Deutsche Theaterstuecke in Frankreich@\emph{Deutsche Theaterstücke in Frankreich}|pwv}.\pend
           
\pstart
           Für Fräulein \label{K_L03388-9v}\edtext{\textsc{Dora Popper\pwindex{Popper, Emilie Dorothea 1893-10-08 – 1933-11-24@\textsc{Popper, Emilie Dorothea} (1893-10-08 – 1933-11-24), \emph{Pianist/Pianistin, Pädagoge/Pädagogin}|pw}}}{\lemma{\textnormal{\emph{Dora Popper}}}\Cendnote{\textnormal{Goldmann\pwindex{Goldmann, Paul 31.01.1865 – 25.09.1935@\textsc{Goldmann, Paul} (31.01.1865 – 25.09.1935), \emph{Schriftsteller/Schriftstellerin, Journalist/Journalistin}|pwk} bemühte sich um Presseberichte
                  über die Pianistin\pwindex{Popper, Emilie Dorothea 1893-10-08 – 1933-11-24@\textsc{Popper, Emilie Dorothea} (1893-10-08 – 1933-11-24), \emph{Pianist/Pianistin, Pädagoge/Pädagogin}|pwkv}, vgl. Paul Goldmann an Arthur Schnitzler, 13. 12. [1903]. }}}\label{K_L03388-9} habe ich
               leider nicht viel thun können. Was mir möglich war, habe ich gethan.\pend
           
\pstart
           \label{K_L03388-10v}\edtext{\textsc{Gourgauds} Geſpräche mit \textsc{Napoleon}\pwindex{Napoleons Gedanken und Erinnerungen. St. Helena 1815–18@\emph{Napoleons Gedanken und Erinnerungen. St. Helena 1815–18}|pw}}{\lemma{\textnormal{\emph{Gourgauds … Napoleon}}}\Cendnote{\textnormal{Gaspard Gourgaud\pwindex{Gourgaud, Gaspard 1783-09-14 – 1852-07-25@\textsc{Gourgaud, Gaspard} (1783-09-14 – 1852-07-25), \emph{Schriftsteller/Schriftstellerin, General/Generalin}|pwk}: \emph{Napoleons Gedanken und Erinnerungen. St. Helena
                        1815–18}\pwindex{Napoleons Gedanken und Erinnerungen. St. Helena 1815–18@\emph{Napoleons Gedanken und Erinnerungen. St. Helena 1815–18}|pwk}. Übersetzt von Heinrich
                        Conrad\pwindex{Conrad, Heinrich 19.10.1866 – 1918-12-20@\textsc{Conrad, Heinrich} (19.10.1866 – 1918-12-20), \emph{Übersetzer/Übersetzerin, Romanist/Romanistin}|pwk}. Stuttgart\oindex{Stuttgart@\textbf{Stuttgart}, \emph{P.PPLA}|pwk}: \emph{Robert Lutz}\orgindex{Robert Lutz@Robert Lutz|pwk}{ }1901.}}}\label{K_L03388-10}, die ich Dir verdanke (ich werde Dir das Buch\pwindex{Napoleons Gedanken und Erinnerungen. St. Helena 1815–18@\emph{Napoleons Gedanken und Erinnerungen. St. Helena 1815–18}|pwv} in Berlin\oindex{Berlin@\textbf{Berlin}, \emph{P.PPLC}|pw} zurückgeben) haben mir viel Genuß bereitet. Ein herrliches Buch \substVorne{}\textsuperscript{i}\substDazwischen{}ſ\substHinten{}ind \label{K_L03388-11v}\edtext{\textsc{Krapotkins} Memoiren\pwindex{Memoiren eines Revolutionaers. 2 Bde.@\emph{Memoiren eines Revolutionärs. 2 Bde.}|pw}}{\lemma{\textnormal{\emph{Krapotkins Memoiren}}}\Cendnote{\textnormal{Peter Kropotkin\pwindex{Kropotkin, Pjotr Alexejewitsch 1842-12-09 – 1921-02-08@\textsc{Kropotkin, Pjotr Alexejewitsch} (1842-12-09 – 1921-02-08), \emph{Schriftsteller/Schriftstellerin, Geograf/Geografin, Revolutionär/Revolutionärin}|pwk}: \emph{Memoiren eines Revolutionärs}\pwindex{Memoiren eines Revolutionaers. 2 Bde.@\emph{Memoiren eines Revolutionärs. 2 Bde.}|pwk}. 2 Bde. Übersetzt von Max Pannwitz\pwindex{Pannwitz, Max 1854-11-08 – 1921-08-29@\textsc{Pannwitz, Max} (1854-11-08 – 1921-08-29), \emph{Schriftsteller/Schriftstellerin, Übersetzer/Übersetzerin}|pwk}. Stuttgart\oindex{Stuttgart@\textbf{Stuttgart}, \emph{P.PPLA}|pwk}: \emph{Robert
                        Lutz}\orgindex{Robert Lutz@Robert Lutz|pwk}{ }1900. Eventuell las Schnitzler die Memoiren\pwindex{Memoiren eines Revolutionaers. 2 Bde.@\emph{Memoiren eines Revolutionärs. 2 Bde.}|pwkv}{ }1923, als er im \emph{Tagebuch}\pwindex{Tagebuch@\emph{Tagebuch}|pwk} nur notierte, »Kropotkin\pwindex{Kropotkin, Pjotr Alexejewitsch 1842-12-09 – 1921-02-08@\textsc{Kropotkin, Pjotr Alexejewitsch} (1842-12-09 – 1921-02-08), \emph{Schriftsteller/Schriftstellerin, Geograf/Geografin, Revolutionär/Revolutionärin}|pw}« zu lesen (siehe 7. 5. 1923, 14. 6. 1923).}}}\label{K_L03388-11}, \strikeout{d\textcolor{gray}{i}} (im ſelben Verlag\orgindex{Robert Lutz@Robert Lutz|pwv}e
               erſchienen), \strikeout{die} deren Lektüre ich Dir dringend
               empfehle.\pend
           
\pstart
           Mit \label{K_L03388-12v}\edtext{Frankfurt\oindex{Frankfurt am Main@\textbf{Frankfurt am Main}, \emph{P.PPLA3}|pw}\pwindex{Rottenberg, Theodore 1875-09-07 – 1945-04-05@\textsc{Rottenberg, Theodore} (1875-09-07 – 1945-04-05)|pwv}}{\lemma{\textnormal{\emph{Frankfurt}}}\Cendnote{\textnormal{Bezug auf das Verhältnis mit und zu
                     Theodore Rottenberg\pwindex{Rottenberg, Theodore 1875-09-07 – 1945-04-05@\textsc{Rottenberg, Theodore} (1875-09-07 – 1945-04-05)|pwk}, mit der Goldmann\pwindex{Goldmann, Paul 31.01.1865 – 25.09.1935@\textsc{Goldmann, Paul} (31.01.1865 – 25.09.1935), \emph{Schriftsteller/Schriftstellerin, Journalist/Journalistin}|pwk} in einer intimen Beziehung
                  stand}}}\label{K_L03388-12} bin ich in reger Correſpondenz. Hier und da fährt ein Sturm
               dazwiſchen. Ich weiß nicht, was werden ſoll. Ich mag mich an dieſe Frau\pwindex{Rottenberg, Theodore 1875-09-07 – 1945-04-05@\textsc{Rottenberg, Theodore} (1875-09-07 – 1945-04-05)|pwv} nicht durch Heirath binden, weil das
               mein \strikeout{Ru\textcolor{gray}{i}} wirthſchaftlicher {\pb}Ruin wäre und weil auch,
               infolge der Affaire\pwindex{?? [Partner von Theodore Rottenberg, Ende 1902/Anfang 1903] @\textsc{?? [Partner von Theodore Rottenberg, Ende 1902/Anfang 1903]}|pwv} in
               dieſem Winter, viel \label{K_L03388-13v}\edtext{Schmutz}{\lemma{\textnormal{\emph{Schmutz}}}\Cendnote{\textnormal{Siehe Paul Goldmann an Arthur Schnitzler, 7. 9. 1903. }}}\label{K_L03388-13} an ihr\pwindex{Rottenberg, Theodore 1875-09-07 – 1945-04-05@\textsc{Rottenberg, Theodore} (1875-09-07 – 1945-04-05)|pw} haftet; anderſeits kann ich nicht einmal
               den Gedanken ertragen, auf ſie zu verzichten.\pend
           
\pstart
           Grüße Deine Frau\pwindex{Schnitzler, Olga 17.01.1882 – 13.01.1970@\textsc{Schnitzler, Olga} (17.01.1882 – 13.01.1970), \emph{Schauspieler/Schauspielerin, Sänger/Sängerin}|pwv}
               vielmals, ſchreib mir bald und ſei ſelbſt herzlichſt gegrüßt von Deinem getreuen {\\[\baselineskip]}\spacefill\mbox{Paul Goldmann.}\pend
           \leftskip=0em{}\selectlanguage{ngerman}\endnumbering\briefempfaengerindex{Schnitzler, Arthur@\textsc{Schnitzler, Arthur}!zzzGoldmann, Paul@\emph{von Paul Goldmann}!1903-11-141@{14. 11. {[}1903{]}}|)be}\mylabel{L03388h}  \normalsize

\doendnotes{C}
\bigskip
\vfill

\clearpage

\footnotesize

\lohead{\textsc{register}}

% Definiere theindex-Environment komplett neu ohne reledmac
\makeatletter
\renewenvironment{theindex}{%
  \section*{\indexname}%
  \setlength{\parindent}{0pt}%
  \setlength{\parskip}{0pt plus 0.3pt}%
  \let\item\@idxitem
}{%
  \clearpage
}
\makeatother

\IfFileExists{\jobname-pw.ind}{\input{\jobname-pw.ind}}{}

\end{document}

      