%% latex-leseansicht-vorspann.tex
%% Vorspann für die Leseansicht.
%% Lädt die gemeinsame Datei latex-vorspann.tex mit nicht gesetztem Schalter.

\newif\ifkorrekturansicht
\korrekturansichtfalse

\input{../tex-inputs/latex-vorspann}

\begin{center}
            \textcolor{red}{ENTWURF, NICHT FERTIG KORRIGIERT}
                      \end{center}
            
         
         \renewcommand{\erwaehntePersonen}{Personen:  ?? [Partner von Theodore Rottenberg, Ende 1902/Anfang 1903], Hermann Bahr, Rudolf Bernauer, Heinrich Conrad, Gaspard Gourgaud, Pjotr Alexejewitsch Kropotkin, Paul Marx, Carl Meinhard, Max Nordau, Max Pannwitz, Emilie Dorothea Popper, Theodore Rottenberg, Olga Schnitzler, Heinrich Schnitzler, Elisabeth Steinrück}
         \renewcommand{\erwaehnteInstitutionen}{Institutionen: Robert Lutz, Schiller-Theater}
         \renewcommand{\erwaehnteOrte}{Orte: Berlin, Dessauer Straße, Die bösen Buben, Frankfurt am Main, Paris, Schiller-Theater, Stuttgart, Théâtre Antoine-Simone Berriau, Wien}
         \renewcommand{\erwaehnteWerke}{Werke: Au Perroquet Vert, Der einsame Weg. Schauspiel in fünf Akten, Der grüne Kakadu. Groteske in einem Akt, Deutsche Theaterstücke in Frankreich, Liebelei. Schauspiel in drei Akten, Literatur, Memoiren eines Revolutionärs. 2 Bde., Napoleons Gedanken und Erinnerungen. St. Helena 1815–18, Reigen. Zehn Dialoge, Tagebuch, Vossische Zeitung}
               \section[ Paul Goldmann an Arthur Schnitzler, 14. 11. {[}1903{]}]{ Paul Goldmann an Arthur Schnitzler, 14. 11. {[}1903{]}}\nopagebreak\mylabel{v}\rehead{ }\begin{ledgroupsized}[t]{13cm}\normalsize\beginnumbering \toendnotes[C]{\smallbreak\pagebreak[2]} \Standort{DLA, A:Schnitzler, HS.NZ85.1.3173.}
\physDesc{Brief, 1 Blatt, 4 Seiten, 1939 Zeichen
\newline{}Handschrift: blaue Tinte, deutsche Kurrent
\newline{}Schnitzler: 1) mit Bleistift das Jahr »{[}1{]}903.«   2) mit Bleistift auf einem beigelegten Blatt mutmaßlich eine Antwortskizze, die
                                 nur unzuverlässig zu entziffern ist: »{\pb}{ / }\textcolor{gray}{\textbf{Recept}}{ / }\uline{Nordau\pwindex{Nordau, Max 29.07.1849 – 22.01.1923@\textsc{Nordau, Max} (29.07.1849 – 22.01.1923), \emph{Schriftsteller, Mediziner}|pw}} – hat falſch\textcolor{gray}{e} berichte d Kriti –
                                       hier an tgl. – (folgende.) – –{ / }dzt: d Kritik nach nicht anzufangen! –){ / }–{ / }Dor{[}a{]}
                                          Popper\pwindex{Popper, Emilie Dorothea 1893-10-08 – 1933-11-24@\textsc{Popper, Emilie Dorothea} (1893-10-08 – 1933-11-24), \emph{Pianistin, Pädagogin}|pw} – Paul\pwindex{Marx, Paul 21.07.1879 – 1956-10-30@\textsc{Marx, Paul} (21.07.1879 – 1956-10-30), \emph{Regisseur, Schauspieler}|pwu}\textcolor{gray}{!}{ / }–{ / }Nicht beide unſtet iſ\textcolor{gray}{t}« 3) mit rotem Buntstift zwei Unterstreichungen}\toendnotes[C]{\smallbreak}\pstart
           \noindent{}\raggedleft{}{\pb}\textcolor{gray}{\textbf{DESSAUERSTRASSE 19\oindex{Dessauer Strasse@\textbf{Dessauer Straße}|pw}}}\pend
           \pstart
           Berlin\oindex{Berlin@\textbf{Berlin}|pw}, 14. November.\pend
           \pstart\center{}Mein lieber Freund,\pend\pstart
           Verzeih mir, daß ich ſo lange nicht geſchrieben habe. Ich lebe ſeit meiner \label{K_L03388-1v}\edtext{Rückkehr}{\lemma{\textnormal{\emph{Rückkehr}}}\Cendnote{\textnormal{siehe Paul Goldmann an Arthur Schnitzler, 7. 9. 1903}}}\label{K_L03388-1h} in fortwährend wechſelnden Stimmungen, in vielen Sorgen und Widrigkeiten.
               Eine große Müdigkeit hielt mich vom Schreiben zurück. Im Grunde \strikeout{iſ\textcolor{gray}{t}} bleibt doch immer Alles beim Alten. Wozu alſo ſchreiben?\pend
           \pstart
           Deine lieben Nachrichten haben mir ſehr gefehlt. Warum haſt \uline{Du} mir denn nicht geſchrieben? Sind wir denn ſo formell geworden, daß
               Einer auf des Andern Brief wartet, um ihm Nachricht von ſich zu geben? Geſtern{ }{\pb}habe ich endlich durch \textsc{Liesl\pwindex{Steinrueck, Elisabeth 19.11.1885 – 07.04.1920@\textsc{Steinrück, Elisabeth} (19.11.1885 – 07.04.1920)|pw}}, die ich bei den \label{K_L03388-2v}\edtext{»Böſen Buben\oindex{Die boesen Buben@\textbf{Die bösen Buben}|pw}«}{\lemma{\textnormal{\emph{»Böſen Buben«}}}\Cendnote{\textnormal{Die bösen Buben\oindex{Die boesen Buben@\textbf{Die bösen Buben}|pwk} war der Name eines Berlin\oindex{Berlin@\textbf{Berlin}|pwk}er Kabaretts, das 1901 von Rudolf Bernauer\pwindex{Bernauer, Rudolf 20.01.1880 – 27.11.1953@\textsc{Bernauer, Rudolf} (20.01.1880 – 27.11.1953), \emph{Theaterleiter, Schauspieler}|pwk} und Carl Meinhard\pwindex{Meinhard, Carl 28.11.1875 – 12.02.1949@\textsc{Meinhard, Carl} (28.11.1875 – 12.02.1949), \emph{Theaterleiter}|pwk} gegründet worden war und bis
                     1905 bestand.}}}\label{K_L03388-2h} ſprach, etwas Näheres über Dich
               erfahren. Ich habe zu meiner großen Freude gehört, daß es Dir, Deiner Frau\pwindex{Schnitzler, Olga 17.01.1882 – 13.01.1970@\textsc{Schnitzler, Olga} (17.01.1882 – 13.01.1970), \emph{Schauspielerin, Sängerin}|pwv} und dem Kinde\pwindex{Schnitzler, Heinrich 09.08.1902 – 12.07.1982@\textsc{Schnitzler, Heinrich} (09.08.1902 – 12.07.1982), \emph{Regisseur, Schauspieler}|pwv} gut geht. Und nicht minder freue ich
               mich über die Ausſicht, Dich bald in \label{K_L03388-3v}\edtext{Berlin\oindex{Berlin@\textbf{Berlin}|pw}}{\lemma{\textnormal{\emph{Berlin}}}\Cendnote{\textnormal{Das nächste Mal war Schnitzler\pwindex{Schnitzler, Arthur 15.05.1862 – 21.10.1931@\textsc{Schnitzler, Arthur} (15.05.1862 – 21.10.1931), \emph{Schriftsteller, Mediziner}|pwk} zwischen 5. 2. 1904 und 17. 2. 1904 in Berlin\oindex{Berlin@\textbf{Berlin}|pwk}. Goldmann\pwindex{Goldmann, Paul 31.01.1865 – 25.09.1935@\textsc{Goldmann, Paul} (31.01.1865 – 25.09.1935), \emph{Schriftsteller, Journalist}|pwk} traf er
                  jedenfalls am 7. 2. 1904, 10. 2. 1904 und 16. 2. 1904.}}}\label{K_L03388-3h} zu ſehen. Zu Deinen Erfolgen in der letzten Zeit
                  (\label{K_L03388-4v}\edtext{Schillertheater\orgindex{Schiller-Theater@Schiller-Theater|pw}}{\lemma{\textnormal{\emph{Schillertheater}}}\Cendnote{\textnormal{Am 29. 10. 1903 hatte am Berlin\oindex{Berlin@\textbf{Berlin}|pwk}er Schillertheater\oindex{Schiller-Theater@\textbf{Schiller-Theater}|pwk} ein »Schnitzler\pwindex{Schnitzler, Arthur 15.05.1862 – 21.10.1931@\textsc{Schnitzler, Arthur} (15.05.1862 – 21.10.1931), \emph{Schriftsteller, Mediziner}|pwk}-Abend« mit einer Aufführung von \emph{Liebelei}\pwindex{Schnitzler, Arthur 15.05.1862 – 21.10.1931@\textsc{Schnitzler, Arthur} (15.05.1862 – 21.10.1931), \emph{Schriftsteller, Mediziner}!Liebelei. Schauspiel in drei Akten1895-10-09@\strich\emph{Liebelei. Schauspiel in drei Akten} {[}1895-10-09{]}|pwk} und \emph{Literatur}\pwindex{Schnitzler, Arthur 15.05.1862 – 21.10.1931@\textsc{Schnitzler, Arthur} (15.05.1862 – 21.10.1931), \emph{Schriftsteller, Mediziner}!Literatur1901@\strich\emph{Literatur} {[}1901{]}|pwk} stattgefunden.}}}\label{K_L03388-4h}, \label{K_L03388-44v}\edtext{Paris\oindex{Paris@\textbf{Paris}|pw}}{\lemma{\textnormal{\emph{Paris}}}\Cendnote{\textnormal{Die Inszenierung von \emph{Au Perroquet vert}\pwindex{Schnitzler, Arthur 15.05.1862 – 21.10.1931@\textsc{Schnitzler, Arthur} (15.05.1862 – 21.10.1931), \emph{Schriftsteller, Mediziner}!Au Perroquet Vert1903-11-07@\strich\emph{Au Perroquet Vert} {[}1903-11-07{]}|pwk} (\emph{Der
                     grüne Kakadu}\pwindex{Schnitzler, Arthur 15.05.1862 – 21.10.1931@\textsc{Schnitzler, Arthur} (15.05.1862 – 21.10.1931), \emph{Schriftsteller, Mediziner}!gruene Kakadu. Groteske in einem Akt1. 3. 1899@\strich\emph{Der grüne Kakadu. Groteske in einem Akt} {[}1. 3. 1899{]}|pwk}) wurde im Théatre Antoine\oindex{Theâtre Antoine-Simone Berriau@\textbf{Théâtre Antoine-Simone Berriau}|pwk}
                  zwischen 7. 11. 1903 und 6. 12. 1903 zwölf Mal gegeben.}}}\label{K_L03388-44h}, \label{K_L03388-77v}\edtext{Bahr\pwindex{Bahr, Hermann 19.07.1863 – 15.01.1934@\textsc{Bahr, Hermann} (19.07.1863 – 15.01.1934), \emph{Schriftsteller, Kritiker}|pw}s Vorleſung\pwindex{Schnitzler, Arthur 15.05.1862 – 21.10.1931@\textsc{Schnitzler, Arthur} (15.05.1862 – 21.10.1931), \emph{Schriftsteller, Mediziner}!Reigen. Zehn Dialoge1900@\strich\emph{Reigen. Zehn Dialoge} {[}1900{]}|pwv}}{\lemma{\textnormal{\emph{Bahrs Vorleſung}}}\Cendnote{\textnormal{Bahr\pwindex{Bahr, Hermann 19.07.1863 – 15.01.1934@\textsc{Bahr, Hermann} (19.07.1863 – 15.01.1934), \emph{Schriftsteller, Kritiker}|pwk} hatte eine öffentliche Vorlesung von
                     \emph{Reigen}\pwindex{Schnitzler, Arthur 15.05.1862 – 21.10.1931@\textsc{Schnitzler, Arthur} (15.05.1862 – 21.10.1931), \emph{Schriftsteller, Mediziner}!Reigen. Zehn Dialoge1900@\strich\emph{Reigen. Zehn Dialoge} {[}1900{]}|pwk} geplant. Letztlich wurde ihm das
                  behördlich untersagt. Vgl. A. S.: \emph{Tagebuch}, 1. 11. 1903; Bahr/Schnitzler, D041436.}}}\label{K_L03388-77h})
               beglückwünſche ich Dich herzlichſt, und ich hoffe, daß das neue Stück\pwindex{Schnitzler, Arthur 15.05.1862 – 21.10.1931@\textsc{Schnitzler, Arthur} (15.05.1862 – 21.10.1931), \emph{Schriftsteller, Mediziner}!einsame Weg. Schauspiel in fuenf Akten1904@\strich\emph{Der einsame Weg. Schauspiel in fünf Akten} {[}1904{]}|pwv} dieſe »ſchöne« Reihe mit Glanz
               fortſetzen wird. Den \label{K_L03388-6v}\edtext{Artikel\pwindex{Deutsche Theaterstuecke in Frankreich1903-11-11@\emph{Deutsche Theaterstücke in Frankreich} {[}1903-11-11{]}|pwv} von \textsc{Nordau\pwindex{Nordau, Max 29.07.1849 – 22.01.1923@\textsc{Nordau, Max} (29.07.1849 – 22.01.1923), \emph{Schriftsteller, Mediziner}|pw}}}{\lemma{\textnormal{\emph{Artikel von Nordau}}}\Cendnote{\textnormal{M. N.\pwindex{Nordau, Max 29.07.1849 – 22.01.1923@\textsc{Nordau, Max} (29.07.1849 – 22.01.1923), \emph{Schriftsteller, Mediziner}|pwkv} [ = Max Nordau\pwindex{Nordau, Max 29.07.1849 – 22.01.1923@\textsc{Nordau, Max} (29.07.1849 – 22.01.1923), \emph{Schriftsteller, Mediziner}|pwk}]: \emph{Deutsche Theaterstücke in Frankreich}\pwindex{Deutsche Theaterstuecke in Frankreich1903-11-11@\emph{Deutsche Theaterstücke in Frankreich} {[}1903-11-11{]}|pwk}. In: \emph{Vossische Zeitung}\pwindex{?? Werk@Nicht ermittelte Verfasserinnen und Verfasser!Vossische Zeitung1617 – 1934@\emph{Vossische Zeitung} {[}1617 – 1934{]}|pwk}, Nr. 529, 11. 11. 1903, Morgen-Ausgabe,
                  S. [15]–[16].}}}\label{K_L03388-6h} ſchickte ich Dir, weil ich es bemerkenswerth fand, daß
               dieſer {\pb}Menſch\pwindex{Nordau, Max 29.07.1849 – 22.01.1923@\textsc{Nordau, Max} (29.07.1849 – 22.01.1923), \emph{Schriftsteller, Mediziner}|pwv}, der Alles verreißt,
               ſo freundlich über Dich ſprach\pwindex{Deutsche Theaterstuecke in Frankreich1903-11-11@\emph{Deutsche Theaterstücke in Frankreich} {[}1903-11-11{]}|pwv}.\pend
           \pstart
           Für Fräulein \label{K_L03388-7v}\edtext{\textsc{Dora Popper\pwindex{Popper, Emilie Dorothea 1893-10-08 – 1933-11-24@\textsc{Popper, Emilie Dorothea} (1893-10-08 – 1933-11-24), \emph{Pianistin, Pädagogin}|pw}}}{\lemma{\textnormal{\emph{Dora Popper}}}\Cendnote{\textnormal{Goldmann\pwindex{Goldmann, Paul 31.01.1865 – 25.09.1935@\textsc{Goldmann, Paul} (31.01.1865 – 25.09.1935), \emph{Schriftsteller, Journalist}|pwk} bemühte sich um Presseberichte für
                  die Pianistin, vgl. Paul Goldmann an Arthur Schnitzler, 13. 12. [1903].
               }}}\label{K_L03388-7h} habe ich leider nicht viel thun können. Was mir möglich war, habe ich
               gethan.\pend
           \pstart
           \label{K_L03388-8v}\edtext{\textsc{Gourgauds} Geſpräche mit \textsc{Napoleon}\pwindex{Gourgaud, Gaspard 1783-09-14 – 1852-07-25@\textsc{Gourgaud, Gaspard} (1783-09-14 – 1852-07-25), \emph{Schriftsteller, General}!Napoleons Gedanken und Erinnerungen. St. Helena 1815–181901@\strich\emph{Napoleons Gedanken und Erinnerungen. St. Helena 1815–18} {[}1901{]}|pw}}{\lemma{\textnormal{\emph{Gourgauds … Napoleon}}}\Cendnote{\textnormal{Gaspard Gourgaud\pwindex{Gourgaud, Gaspard 1783-09-14 – 1852-07-25@\textsc{Gourgaud, Gaspard} (1783-09-14 – 1852-07-25), \emph{Schriftsteller, General}|pwk}: \emph{Napoleons Gedanken und Erinnerungen. St. Helena
                        1815–18}\pwindex{Gourgaud, Gaspard 1783-09-14 – 1852-07-25@\textsc{Gourgaud, Gaspard} (1783-09-14 – 1852-07-25), \emph{Schriftsteller, General}!Napoleons Gedanken und Erinnerungen. St. Helena 1815–181901@\strich\emph{Napoleons Gedanken und Erinnerungen. St. Helena 1815–18} {[}1901{]}|pwk}. Übersetzt von Heinrich
                        Conrad\pwindex{Conrad, Heinrich 19.10.1866 – 1918-12-20@\textsc{Conrad, Heinrich} (19.10.1866 – 1918-12-20), \emph{Übersetzer, Romanist}|pwk}. Stuttgart\oindex{Stuttgart@\textbf{Stuttgart}|pwk}: \emph{Robert Lutz}\orgindex{Robert Lutz@Robert Lutz|pwk}{ }1901.}}}\label{K_L03388-8h}, die ich Dir verdanke (ich werde Dir das Buch\pwindex{Gourgaud, Gaspard 1783-09-14 – 1852-07-25@\textsc{Gourgaud, Gaspard} (1783-09-14 – 1852-07-25), \emph{Schriftsteller, General}!Napoleons Gedanken und Erinnerungen. St. Helena 1815–181901@\strich\emph{Napoleons Gedanken und Erinnerungen. St. Helena 1815–18} {[}1901{]}|pwv} in Berlin\oindex{Berlin@\textbf{Berlin}|pw} zurückgeben) haben mir viel Genuß bereitet. Ein herrliches Buch \substVorne{}\textsuperscript{i}\substDazwischen{}ſ\substHinten{}ind \label{K_L03388-9v}\edtext{\textsc{Krapotkins} Memoiren\pwindex{Kropotkin, Pjotr Alexejewitsch 1842-12-09 – 1921-02-08@\textsc{Kropotkin, Pjotr Alexejewitsch} (1842-12-09 – 1921-02-08), \emph{Geograf, Revolutionär, Anarchist}!Memoiren eines Revolutionaers. 2 Bde.1900@\strich\emph{Memoiren eines Revolutionärs. 2 Bde.} {[}1900{]}|pw}}{\lemma{\textnormal{\emph{Krapotkins Memoiren}}}\Cendnote{\textnormal{Peter Kropotkin\pwindex{Kropotkin, Pjotr Alexejewitsch 1842-12-09 – 1921-02-08@\textsc{Kropotkin, Pjotr Alexejewitsch} (1842-12-09 – 1921-02-08), \emph{Geograf, Revolutionär, Anarchist}|pwk}: \emph{Memoiren eines Revolutionärs}\pwindex{Kropotkin, Pjotr Alexejewitsch 1842-12-09 – 1921-02-08@\textsc{Kropotkin, Pjotr Alexejewitsch} (1842-12-09 – 1921-02-08), \emph{Geograf, Revolutionär, Anarchist}!Memoiren eines Revolutionaers. 2 Bde.1900@\strich\emph{Memoiren eines Revolutionärs. 2 Bde.} {[}1900{]}|pwk}. 2 Bände. Übersetzt von
                        Max Pannwitz\pwindex{Pannwitz, Max 1854-11-08 – 1921-08-29@\textsc{Pannwitz, Max} (1854-11-08 – 1921-08-29), \emph{Schriftsteller, Übersetzer}|pwk}. Stuttgart\oindex{Stuttgart@\textbf{Stuttgart}|pwk}: \emph{Robert
                        Lutz}\orgindex{Robert Lutz@Robert Lutz|pwk}{ }1900. Eventuell las Schnitzler\pwindex{Schnitzler, Arthur 15.05.1862 – 21.10.1931@\textsc{Schnitzler, Arthur} (15.05.1862 – 21.10.1931), \emph{Schriftsteller, Mediziner}|pwk} die Memoiren\pwindex{Kropotkin, Pjotr Alexejewitsch 1842-12-09 – 1921-02-08@\textsc{Kropotkin, Pjotr Alexejewitsch} (1842-12-09 – 1921-02-08), \emph{Geograf, Revolutionär, Anarchist}!Memoiren eines Revolutionaers. 2 Bde.1900@\strich\emph{Memoiren eines Revolutionärs. 2 Bde.} {[}1900{]}|pwkv}{ }1923, als er im \emph{Tagebuch}\pwindex{Schnitzler, Arthur 15.05.1862 – 21.10.1931@\textsc{Schnitzler, Arthur} (15.05.1862 – 21.10.1931), \emph{Schriftsteller, Mediziner}!Tagebuch1981 – 2000@\strich\emph{Tagebuch} {[}1981 – 2000{]}|pwk} nur notierte, »Kropotkin\pwindex{Kropotkin, Pjotr Alexejewitsch 1842-12-09 – 1921-02-08@\textsc{Kropotkin, Pjotr Alexejewitsch} (1842-12-09 – 1921-02-08), \emph{Geograf, Revolutionär, Anarchist}|pw}« zu lesen (7. 5. 1923, 14. 6. 1923).}}}\label{K_L03388-9h}, \strikeout{d\textcolor{gray}{e}} (im ſelben Verlag\orgindex{Robert Lutz@Robert Lutz|pwv}e
               erſchienen), \strikeout{die} deren Lektüre ich Dir dringend
               empfehle.\pend
           \pstart
           Mit \label{K_L03388-10v}\edtext{Frankfurt\oindex{Frankfurt am Main@\textbf{Frankfurt am Main}|pw}\pwindex{Rottenberg, Theodore 1875-09-07 – 1945-04-05@\textsc{Rottenberg, Theodore} (1875-09-07 – 1945-04-05)|pwv}}{\lemma{\textnormal{\emph{Frankfurt}}}\Cendnote{\textnormal{Bezug auf sein Verhältnis mit und zu
                     Theodore Rottenberg\pwindex{Rottenberg, Theodore 1875-09-07 – 1945-04-05@\textsc{Rottenberg, Theodore} (1875-09-07 – 1945-04-05)|pwk}, mit der Goldmann\pwindex{Goldmann, Paul 31.01.1865 – 25.09.1935@\textsc{Goldmann, Paul} (31.01.1865 – 25.09.1935), \emph{Schriftsteller, Journalist}|pwk} in einer intimen Beziehung
                  stand.}}}\label{K_L03388-10h} bin ich in reger Correſpondenz. Hier und da fährt ein Sturm
               dazwiſchen. Ich weiß nicht, was werden ſoll. Ich mag mich an dieſe Frau\pwindex{Rottenberg, Theodore 1875-09-07 – 1945-04-05@\textsc{Rottenberg, Theodore} (1875-09-07 – 1945-04-05)|pwv} nicht durch Heirath binden, weil das
               mein \strikeout{Ru\textcolor{gray}{in}} wirthſchaftlicher {\pb}Ruin wäre und weil auch,
               infolge der Affaire\pwindex{?? [Partner von Theodore Rottenberg, Ende 1902/Anfang 1903] @\textsc{?? [Partner von Theodore Rottenberg, Ende 1902/Anfang 1903]}|pwv} in
               dieſem Winter, viel \label{K_L03388-11v}\edtext{Schmutz}{\lemma{\textnormal{\emph{Schmutz}}}\Cendnote{\textnormal{siehe Paul Goldmann an Arthur Schnitzler, 7. 9. 1903}}}\label{K_L03388-11h} an ihr\pwindex{Rottenberg, Theodore 1875-09-07 – 1945-04-05@\textsc{Rottenberg, Theodore} (1875-09-07 – 1945-04-05)|pw} haftet; anderſeits kann ich
               nicht einmal den Gedanken ertragen, auf ſie zu verzichten.\pend
           \pstart
           Grüße Deine Frau\pwindex{Schnitzler, Olga 17.01.1882 – 13.01.1970@\textsc{Schnitzler, Olga} (17.01.1882 – 13.01.1970), \emph{Schauspielerin, Sängerin}|pwv}
               vielmals, ſchreib mir bald und ſei ſelbſt herzlichſt gegrüßt von Deinem getreuen {\\[\baselineskip]}\spacefill\mbox{Paul Goldmann.}\pend
           \leftskip=0em{}
         
         \endnumbering\mylabel{h}\end{ledgroupsized}\begin{anhang}\end{anhang}\newcommand{\dateiname}{L03388}\newcommand{\titel}{Paul Goldmann an Arthur Schnitzler, 14. 11. [1903]}\newcommand{\editorInnen}{Martin Anton Müller und Laura Untner}%% latex-leseansicht-abspann.tex
%% Abspann für die Leseansicht.
%% Der Schalter \ifkorrekturansicht ist bereits durch den Vorspann gesetzt.

%% latex-abspann.tex
%% Gemeinsamer Abspann für Korrekturansicht und Leseansicht.
%% Setzt den Schalter \ifkorrekturansicht voraus (gesetzt in den
%% einbindenden Dateien latex-korrekturansicht-abspann.tex bzw.
%% latex-leseansicht-abspann.tex).
%% ---------------------------------------------------------------

\normalsize

% Das esempio-Environment wird nur in der Leseansicht benötigt
\ifkorrekturansicht\else
\newenvironment{esempio}[3]%
{
    \vspace{1.5ex}
    \rlap{\underline{#1}}
    \par
    \setlength{\parindent}{0cm}
    \nopagebreak
    \leftskip=#2cm
    \rightskip=#3cm
}
{
    \par
}
\fi

\doendnotes{C}
\bigskip
\vfill

\clearpage

\footnotesize

\ifkorrekturansicht
  \lohead{\textsc{register}}
\fi

% theindex-Environment neu definieren ohne reledmac
\makeatletter
\renewenvironment{theindex}{%
  \ifkorrekturansicht
    \section*{\indexname}%
  \else
    \subsubsection*{Index der erwähnten Entitäten}%
  \fi
  \setlength{\parindent}{0pt}%
  \setlength{\parskip}{0pt plus 0.3pt}%
  \let\item\@idxitem
}{%
  \ifkorrekturansicht\clearpage\fi
}
\makeatother

\IfFileExists{\jobname-pw.ind}{\input{\jobname-pw.ind}}{}

% Quellenangabe nur in der Leseansicht
\ifkorrekturansicht\else
% Fallback-Definitionen, falls die .tex-Datei \titel etc. nicht gesetzt hat
\providecommand{\titel}{}
\providecommand{\editorInnen}{}
\providecommand{\dateiname}{\jobname}

\vspace{3cm}

\vfill

\footnotesize
\textsc{Quelle}: \titel. Herausgegeben von {\editorInnen}. In: \emph{Arthur Schnitzler: Briefwechsel mit Autorinnen und Autoren}.
 Digitale Edition, https://schnitzler-briefe.acdh.oeaw.ac.at/{\dateiname}.html (Stand \today)
\fi

\end{document}


      