%% latex-leseansicht-vorspann.tex
%% Vorspann für die Leseansicht.
%% Lädt die gemeinsame Datei latex-vorspann.tex mit nicht gesetztem Schalter.

\newif\ifkorrekturansicht
\korrekturansichtfalse

\input{../tex-inputs/latex-vorspann}


\section[ Paul Goldmann an Arthur Schnitzler, 14. 11. {[}1903{]}]{L03388 Paul Goldmann an Arthur Schnitzler,  14. 11. [1903]}
\nopagebreak\mylabel{L03388v}
\rehead{ }\normalsize\beginnumbering\briefempfaengerindex{Schnitzler, Arthur@\textsc{Schnitzler, Arthur}!zzzGoldmann, Paul@\emph{von Paul Goldmann}!1903-11-141@{14. 11. [1903]}|(be}
\toendnotes[C]{\smallbreak\pagebreak[2]}
\correspDesc{Versand  durch Paul Goldmann am 14. 11. [1903] in Berlin
\newline{}Erhalt  durch Arthur Schnitzler im Zeitraum [15. 11. 1903 – 19. 11. 1903?] in Wien}\toendnotes[C]{\smallbreak}
\Standort{DLA, A:Schnitzler, HS.NZ85.1.3173.}
\physDesc{Brief, 1 Blatt, 4 Seiten, 1938 Zeichen
\newline{}Handschrift: blaue Tinte, deutsche Kurrent
\newline{}Schnitzler: 1) mit Bleistift das Jahr »903.« vermerkt  2) mit Bleistift auf einem beigelegten Blatt mutmaßlich eine Antwortskizze
                                 notiert, die nur unzuverlässig zu entziffern ist: »{\pb}{ / }\textcolor{gray}{\textbf{Recept.}}{ / }\uline{Nordau\pwindex{Nordau, Max 29.\,7.\,1849 Budapest – 22.\,1.\,1923 Paris@\textsc{Nordau, Max} (29.\,7.\,1849 Budapest – 22.\,1.\,1923 Paris), \emph{Schriftsteller, Mediziner}|pw}} – hat falſch berichte d Kriti –{ }ſind
                                          \textcolor{gray}{au} glzd. –
                                          (folgende.) – \textcolor{gray}{u}{ / }tagl: d Kritik wei\textcolor{gray}{ß} nicht
                                       anzufangen! –){ / }–{ / }Dor{[}a{]}
                                          Popper\pwindex{Popper, Emilie Dorothea 8.\,10.\,1893 Wien – 24.\,11.\,1933 ebd.@\textsc{Popper, Emilie Dorothea} (8.\,10.\,1893 Wien – 24.\,11.\,1933 ebd.), \emph{Pianistin, Pädagogin}|pw} – Paul\pwindex{Marx, Paul 21.\,7.\,1879 Wien – 30.\,10.\,1956 ebd.@\textsc{Marx, Paul} (21.\,7.\,1879 Wien – 30.\,10.\,1956 ebd.), \emph{Regisseur, Schauspieler}|pwu}\textcolor{gray}{!}{ / }–{ / }Nicht leider unſich und« 3) mit rotem Buntstift zwei Unterstreichungen}\toendnotes[C]{\smallbreak}
\pstart
           \raggedleft{}{\pb}\textcolor{gray}{\textbf{DESSAUERSTRASSE 19\oindex{Dessauer Straße@\textbf{Dessauer Straße}, \emph{Straße}|pw}}}\pend
           
\pstart
           Berlin\oindex{Berlin@\textbf{Berlin}, \emph{Hauptstadt}|pw}, 14. November.\pend
           
\pstart\center{}Mein lieber Freund,\pend\vspace{0.5em}
\pstart
           Verzeih mir, daß ich{ }ſo \label{K_L03388-1v}\edtext{lange nicht
                  geſchrieben}{\lemma{\textnormal{\emph{lange nicht
                  geschrieben}}}\Cendnote{\textnormal{Vgl. XXXX Auszeichnungsfehler: Dokument L01335 nicht gefunden. }}}\label{K_L03388-1} habe. Ich
               lebe{ }ſeit meiner \label{K_L03388-2v}\edtext{Rückkehr}{\lemma{\textnormal{\emph{Rückkehr}}}\Cendnote{\textnormal{Siehe XXXX Auszeichnungsfehler: Dokument L03386 nicht gefunden. }}}\label{K_L03388-2} in
               fortwährend wechſelnden Stimmungen, in vielen Sorgen und Widrigkeiten. Eine große
               Müdigkeit hielt mich vom Schreiben zurück. Im Grunde \strikeout{iſ\textcolor{gray}{t}} bleibt doch immer Alles beim Alten. Wozu alſo{ }ſchreiben?\pend
           
\pstart
           Deine lieben Nachrichten haben mir{ }ſehr gefehlt. Warum haſt \uline{Du} mir denn nicht geſchrieben? Sind wir denn{ }ſo formell geworden, daß
               Einer auf des Andern Brief wartet, um ihm Nachricht von{ }ſich zu geben? Geſtern{ }{\pb}habe ich endlich durch \textsc{Liesl\pwindex{Steinrück, Elisabeth 19.\,11.\,1885 – 7.\,4.\,1920 Partenkirchen@\textsc{Steinrück, Elisabeth} (19.\,11.\,1885 – 7.\,4.\,1920 Partenkirchen)|pw}}, die ich bei den \label{K_L03388-3v}\edtext{»Böſen Buben\oindex{Die bösen Buben@\textbf{Die bösen Buben}, \emph{Kabarett}|pw}«}{\lemma{\textnormal{\emph{»Bösen Buben«}}}\Cendnote{\textnormal{Die bösen Buben\oindex{Die bösen Buben@\textbf{Die bösen Buben}, \emph{Kabarett}|pwk} war der Name eines Berlin\oindex{Berlin@\textbf{Berlin}, \emph{Hauptstadt}|pwk}er Kabaretts, das 1901 von Rudolf Bernauer\pwindex{Bernauer, Rudolf 20.\,1.\,1880 Wien – 27.\,11.\,1953 London@\textsc{Bernauer, Rudolf} (20.\,1.\,1880 Wien – 27.\,11.\,1953 London), \emph{Theaterleiter, Schauspieler}|pwk} und Carl Meinhard\pwindex{Meinhard, Carl 28.\,11.\,1875 Jihlava – 12.\,2.\,1949 Buenos Aires@\textsc{Meinhard, Carl} (28.\,11.\,1875 Jihlava – 12.\,2.\,1949 Buenos Aires), \emph{Theaterleiter}|pwk} gegründet worden war und bis
                     1905 bestand.}}}\label{K_L03388-3}{ }ſprach, etwas Näheres über Dich
               erfahren. Ich habe zu meiner großen Freude gehört, daß es Dir, Deiner Frau\pwindex{Schnitzler, Olga 17.\,1.\,1882 Wien – 13.\,1.\,1970 Lugano@\textsc{Schnitzler, Olga} (17.\,1.\,1882 Wien – 13.\,1.\,1970 Lugano), \emph{Schauspielerin, Sängerin}|pwv} und dem Kinde\pwindex{Schnitzler, Heinrich 9.\,8.\,1902 Hinterbrühl – 12.\,7.\,1982 Wien@\textsc{Schnitzler, Heinrich} (9.\,8.\,1902 Hinterbrühl – 12.\,7.\,1982 Wien), \emph{Regisseur, Schauspieler}|pwv} gut geht. Und nicht minder freue ich
               mich über die Ausſicht, Dich bald in \label{K_L03388-4v}\edtext{Berlin\oindex{Berlin@\textbf{Berlin}, \emph{Hauptstadt}|pw}}{\lemma{\textnormal{\emph{Berlin}}}\Cendnote{\textnormal{Das nächste Mal war Schnitzler zwischen 5. 2. 1904 und 17. 2. 1904 in Berlin\oindex{Berlin@\textbf{Berlin}, \emph{Hauptstadt}|pwk}. Goldmann\pwindex{Goldmann, Paul 31.\,1.\,1865 Breslau – 25.\,9.\,1935 Wien@\textsc{Goldmann, Paul} (31.\,1.\,1865 Breslau – 25.\,9.\,1935 Wien), \emph{Schriftsteller, Journalist}|pwk} traf er
                  jedenfalls am 7. 2. 1904, 10. 2. 1904 und 16. 2. 1904.}}}\label{K_L03388-4} zu{ }ſehen. Zu Deinen Erfolgen in der letzten Zeit
                  (\label{K_L03388-5v}\edtext{Schill\strikeout{t}ertheater\orgindex{Schiller-Theater@Schiller-Theater|pw}}{\lemma{\textnormal{\emph{Schillertheater}}}\Cendnote{\textnormal{Am 29. 10. 1903 hatte am Berlin\oindex{Berlin@\textbf{Berlin}, \emph{Hauptstadt}|pwk}er \emph{Schiller-Theater}\orgindex{Schiller-Theater@Schiller-Theater|pwk} ein »Schnitzler-Abend« mit einer Aufführung von \emph{Liebelei}\pwindex{Schnitzler, Arthur 15.\,5.\,1862 Wien – 21.\,10.\,1931 ebd.@\textsc{Schnitzler, Arthur} (15.\,5.\,1862 Wien – 21.\,10.\,1931 ebd.), \emph{Schriftsteller, Mediziner}!Liebelei. Schauspiel in drei Akten@\strich\emph{Liebelei. Schauspiel in drei Akten}|pwk} und \emph{Literatur}\pwindex{Schnitzler, Arthur 15.\,5.\,1862 Wien – 21.\,10.\,1931 ebd.@\textsc{Schnitzler, Arthur} (15.\,5.\,1862 Wien – 21.\,10.\,1931 ebd.), \emph{Schriftsteller, Mediziner}!Literatur@\strich\emph{Literatur}|pwk} stattgefunden.}}}\label{K_L03388-5}, \label{K_L03388-6v}\edtext{Paris\oindex{Paris@\textbf{Paris}, \emph{Hauptstadt}|pw}}{\lemma{\textnormal{\emph{Paris}}}\Cendnote{\textnormal{Die Inszenierung von \emph{Au Perroquet vert}\pwindex{Schnitzler, Arthur 15.\,5.\,1862 Wien – 21.\,10.\,1931 ebd.@\textsc{Schnitzler, Arthur} (15.\,5.\,1862 Wien – 21.\,10.\,1931 ebd.), \emph{Schriftsteller, Mediziner}!Au Perroquet Vert@\strich\emph{Au Perroquet Vert}|pwk} (\emph{Der
                     grüne Kakadu}\pwindex{Schnitzler, Arthur 15.\,5.\,1862 Wien – 21.\,10.\,1931 ebd.@\textsc{Schnitzler, Arthur} (15.\,5.\,1862 Wien – 21.\,10.\,1931 ebd.), \emph{Schriftsteller, Mediziner}!grüne Kakadu. Groteske in einem Akt@\strich\emph{Der grüne Kakadu. Groteske in einem Akt}|pwk}) wurde im Théatre Antoine\oindex{Théâtre Antoine-Simone Berriau@\textbf{Théâtre Antoine-Simone Berriau}, \emph{Theater}|pwk}
                  zwischen 7. 11. 1903 und 6. 12. 1903 zwölfmal gegeben.}}}\label{K_L03388-6}, \label{K_L03388-7v}\edtext{Bahrs\pwindex{Bahr, Hermann 19.\,7.\,1863 Linz – 15.\,1.\,1934 München@\textsc{Bahr, Hermann} (19.\,7.\,1863 Linz – 15.\,1.\,1934 München), \emph{Schriftsteller, Kritiker}|pw}{ }Vorleſung\pwindex{Schnitzler, Arthur 15.\,5.\,1862 Wien – 21.\,10.\,1931 ebd.@\textsc{Schnitzler, Arthur} (15.\,5.\,1862 Wien – 21.\,10.\,1931 ebd.), \emph{Schriftsteller, Mediziner}!Reigen. Zehn Dialoge@\strich\emph{Reigen. Zehn Dialoge}|pwv}}{\lemma{\textnormal{\emph{Bahrs Vorlesung}}}\Cendnote{\textnormal{Bahr\pwindex{Bahr, Hermann 19.\,7.\,1863 Linz – 15.\,1.\,1934 München@\textsc{Bahr, Hermann} (19.\,7.\,1863 Linz – 15.\,1.\,1934 München), \emph{Schriftsteller, Kritiker}|pwk} hatte eine öffentliche Vorlesung des
                     \emph{Reigen}\pwindex{Schnitzler, Arthur 15.\,5.\,1862 Wien – 21.\,10.\,1931 ebd.@\textsc{Schnitzler, Arthur} (15.\,5.\,1862 Wien – 21.\,10.\,1931 ebd.), \emph{Schriftsteller, Mediziner}!Reigen. Zehn Dialoge@\strich\emph{Reigen. Zehn Dialoge}|pwk} geplant. Letztlich wurde ihm das
                  behördlich untersagt. Vgl. A. S.: \emph{Tagebuch}, 1. 11. 1903; Hermann Bahr, Arthur Schnitzler: \emph{Briefwechsel, Aufzeichnungen, Dokumente (1891–1931)}, Aufzeichnung von Hermann Bahr, 30. 10. 1903.}}}\label{K_L03388-7})
               beglückwünſche ich Dich herzlichſt, und ich hoffe, daß das neue Stück\pwindex{Schnitzler, Arthur 15.\,5.\,1862 Wien – 21.\,10.\,1931 ebd.@\textsc{Schnitzler, Arthur} (15.\,5.\,1862 Wien – 21.\,10.\,1931 ebd.), \emph{Schriftsteller, Mediziner}!einsame Weg. Schauspiel in fünf Akten@\strich\emph{Der einsame Weg. Schauspiel in fünf Akten}|pwv} dieſe »ſchöne« Reihe mit Glanz
               fortſetzen wird. Den \label{K_L03388-8v}\edtext{Artikel\pwindex{Nordau, Max 29.\,7.\,1849 Budapest – 22.\,1.\,1923 Paris@\textsc{Nordau, Max} (29.\,7.\,1849 Budapest – 22.\,1.\,1923 Paris), \emph{Schriftsteller, Mediziner}!Deutsche Theaterstücke in Frankreich@\strich\emph{Deutsche Theaterstücke in Frankreich}|pwv} von \textsc{Nordau\pwindex{Nordau, Max 29.\,7.\,1849 Budapest – 22.\,1.\,1923 Paris@\textsc{Nordau, Max} (29.\,7.\,1849 Budapest – 22.\,1.\,1923 Paris), \emph{Schriftsteller, Mediziner}|pw}}}{\lemma{\textnormal{\emph{Artikel von Nordau}}}\Cendnote{\textnormal{M. N.\pwindex{Nordau, Max 29.\,7.\,1849 Budapest – 22.\,1.\,1923 Paris@\textsc{Nordau, Max} (29.\,7.\,1849 Budapest – 22.\,1.\,1923 Paris), \emph{Schriftsteller, Mediziner}|pwkv} [ = Max Nordau\pwindex{Nordau, Max 29.\,7.\,1849 Budapest – 22.\,1.\,1923 Paris@\textsc{Nordau, Max} (29.\,7.\,1849 Budapest – 22.\,1.\,1923 Paris), \emph{Schriftsteller, Mediziner}|pwk}]: \emph{Deutsche Theaterstücke in Frankreich}\pwindex{Nordau, Max 29.\,7.\,1849 Budapest – 22.\,1.\,1923 Paris@\textsc{Nordau, Max} (29.\,7.\,1849 Budapest – 22.\,1.\,1923 Paris), \emph{Schriftsteller, Mediziner}!Deutsche Theaterstücke in Frankreich@\strich\emph{Deutsche Theaterstücke in Frankreich}|pwk}. In: \emph{Vossische Zeitung}\pwindex{Vossische Zeitung@\emph{Vossische Zeitung}|pwk}, Nr. 529, 11. 11. 1903, Morgen-Ausgabe,
                  S. [15]–[16].}}}\label{K_L03388-8}{ }ſchickte ich Dir, weil ich es bemerkenswerth fand, daß
               dieſer {\pb}Menſch\pwindex{Nordau, Max 29.\,7.\,1849 Budapest – 22.\,1.\,1923 Paris@\textsc{Nordau, Max} (29.\,7.\,1849 Budapest – 22.\,1.\,1923 Paris), \emph{Schriftsteller, Mediziner}|pwv}, der Alles verreißt,{ }ſo freundlich über Dich ſprach\pwindex{Nordau, Max 29.\,7.\,1849 Budapest – 22.\,1.\,1923 Paris@\textsc{Nordau, Max} (29.\,7.\,1849 Budapest – 22.\,1.\,1923 Paris), \emph{Schriftsteller, Mediziner}!Deutsche Theaterstücke in Frankreich@\strich\emph{Deutsche Theaterstücke in Frankreich}|pwv}.\pend
           
\pstart
           Für Fräulein \label{K_L03388-9v}\edtext{\textsc{Dora Popper\pwindex{Popper, Emilie Dorothea 8.\,10.\,1893 Wien – 24.\,11.\,1933 ebd.@\textsc{Popper, Emilie Dorothea} (8.\,10.\,1893 Wien – 24.\,11.\,1933 ebd.), \emph{Pianistin, Pädagogin}|pw}}}{\lemma{\textnormal{\emph{Dora Popper}}}\Cendnote{\textnormal{Goldmann\pwindex{Goldmann, Paul 31.\,1.\,1865 Breslau – 25.\,9.\,1935 Wien@\textsc{Goldmann, Paul} (31.\,1.\,1865 Breslau – 25.\,9.\,1935 Wien), \emph{Schriftsteller, Journalist}|pwk} bemühte sich um Presseberichte
                  über die Pianistin\pwindex{Popper, Emilie Dorothea 8.\,10.\,1893 Wien – 24.\,11.\,1933 ebd.@\textsc{Popper, Emilie Dorothea} (8.\,10.\,1893 Wien – 24.\,11.\,1933 ebd.), \emph{Pianistin, Pädagogin}|pwkv}, vgl. XXXX Auszeichnungsfehler: Dokument L03389 nicht gefunden. }}}\label{K_L03388-9} habe ich
               leider nicht viel thun können. Was mir möglich war, habe ich gethan.\pend
           
\pstart
           \label{K_L03388-10v}\edtext{\textsc{Gourgauds} Geſpräche mit \textsc{Napoleon}\pwindex{Gourgaud, Gaspard 14.\,9.\,1783 Versailles – 25.\,7.\,1852 Paris@\textsc{Gourgaud, Gaspard} (14.\,9.\,1783 Versailles – 25.\,7.\,1852 Paris), \emph{Schriftsteller, General}!Napoleons Gedanken und Erinnerungen. St. Helena 1815–18@\strich\emph{Napoleons Gedanken und Erinnerungen. St. Helena 1815–18}|pw}}{\lemma{\textnormal{\emph{Gourgauds … Napoleon}}}\Cendnote{\textnormal{Gaspard Gourgaud\pwindex{Gourgaud, Gaspard 14.\,9.\,1783 Versailles – 25.\,7.\,1852 Paris@\textsc{Gourgaud, Gaspard} (14.\,9.\,1783 Versailles – 25.\,7.\,1852 Paris), \emph{Schriftsteller, General}|pwk}: \emph{Napoleons Gedanken und Erinnerungen. St. Helena
                        1815–18}\pwindex{Gourgaud, Gaspard 14.\,9.\,1783 Versailles – 25.\,7.\,1852 Paris@\textsc{Gourgaud, Gaspard} (14.\,9.\,1783 Versailles – 25.\,7.\,1852 Paris), \emph{Schriftsteller, General}!Napoleons Gedanken und Erinnerungen. St. Helena 1815–18@\strich\emph{Napoleons Gedanken und Erinnerungen. St. Helena 1815–18}|pwk}. Übersetzt von Heinrich
                        Conrad\pwindex{Conrad, Heinrich 19.\,10.\,1866 Hamburg – 20.\,12.\,1918 München@\textsc{Conrad, Heinrich} (19.\,10.\,1866 Hamburg – 20.\,12.\,1918 München), \emph{Übersetzer, Romanist}|pwk}. Stuttgart\oindex{Stuttgart@\textbf{Stuttgart}|pwk}: \emph{Robert Lutz}\orgindex{Robert Lutz@Robert Lutz|pwk}{ }1901.}}}\label{K_L03388-10}, die ich Dir verdanke (ich werde Dir das Buch\pwindex{Gourgaud, Gaspard 14.\,9.\,1783 Versailles – 25.\,7.\,1852 Paris@\textsc{Gourgaud, Gaspard} (14.\,9.\,1783 Versailles – 25.\,7.\,1852 Paris), \emph{Schriftsteller, General}!Napoleons Gedanken und Erinnerungen. St. Helena 1815–18@\strich\emph{Napoleons Gedanken und Erinnerungen. St. Helena 1815–18}|pwv} in Berlin\oindex{Berlin@\textbf{Berlin}, \emph{Hauptstadt}|pw} zurückgeben) haben mir viel Genuß bereitet. Ein herrliches Buch \substVorne{}\textsuperscript{i}\substDazwischen{}ſ\substHinten{}ind \label{K_L03388-11v}\edtext{\textsc{Krapotkins} Memoiren\pwindex{Kropotkin, Pjotr Alexejewitsch 9.\,12.\,1842 Moskau – 8.\,2.\,1921 Dmitrov@\textsc{Kropotkin, Pjotr Alexejewitsch} (9.\,12.\,1842 Moskau – 8.\,2.\,1921 Dmitrov), \emph{Schriftsteller, Geograf, Revolutionär}!Memoiren eines Revolutionärs. 2 Bde.@\strich\emph{Memoiren eines Revolutionärs. 2 Bde.}|pw}}{\lemma{\textnormal{\emph{Krapotkins Memoiren}}}\Cendnote{\textnormal{Peter Kropotkin\pwindex{Kropotkin, Pjotr Alexejewitsch 9.\,12.\,1842 Moskau – 8.\,2.\,1921 Dmitrov@\textsc{Kropotkin, Pjotr Alexejewitsch} (9.\,12.\,1842 Moskau – 8.\,2.\,1921 Dmitrov), \emph{Schriftsteller, Geograf, Revolutionär}|pwk}: \emph{Memoiren eines Revolutionärs}\pwindex{Kropotkin, Pjotr Alexejewitsch 9.\,12.\,1842 Moskau – 8.\,2.\,1921 Dmitrov@\textsc{Kropotkin, Pjotr Alexejewitsch} (9.\,12.\,1842 Moskau – 8.\,2.\,1921 Dmitrov), \emph{Schriftsteller, Geograf, Revolutionär}!Memoiren eines Revolutionärs. 2 Bde.@\strich\emph{Memoiren eines Revolutionärs. 2 Bde.}|pwk}. 2 Bde. Übersetzt von Max Pannwitz\pwindex{Pannwitz, Max 8.\,11.\,1854 Stradow (Vetschau) – 29.\,8.\,1921 Stuttgart@\textsc{Pannwitz, Max} (8.\,11.\,1854 Stradow (Vetschau) – 29.\,8.\,1921 Stuttgart), \emph{Schriftsteller, Übersetzer}|pwk}. Stuttgart\oindex{Stuttgart@\textbf{Stuttgart}|pwk}: \emph{Robert
                        Lutz}\orgindex{Robert Lutz@Robert Lutz|pwk}{ }1900. Eventuell las Schnitzler die Memoiren\pwindex{Kropotkin, Pjotr Alexejewitsch 9.\,12.\,1842 Moskau – 8.\,2.\,1921 Dmitrov@\textsc{Kropotkin, Pjotr Alexejewitsch} (9.\,12.\,1842 Moskau – 8.\,2.\,1921 Dmitrov), \emph{Schriftsteller, Geograf, Revolutionär}!Memoiren eines Revolutionärs. 2 Bde.@\strich\emph{Memoiren eines Revolutionärs. 2 Bde.}|pwkv}{ }1923, als er im \emph{Tagebuch}\pwindex{Schnitzler, Arthur 15.\,5.\,1862 Wien – 21.\,10.\,1931 ebd.@\textsc{Schnitzler, Arthur} (15.\,5.\,1862 Wien – 21.\,10.\,1931 ebd.), \emph{Schriftsteller, Mediziner}!Tagebuch@\strich\emph{Tagebuch}|pwk} nur notierte, »Kropotkin\pwindex{Kropotkin, Pjotr Alexejewitsch 9.\,12.\,1842 Moskau – 8.\,2.\,1921 Dmitrov@\textsc{Kropotkin, Pjotr Alexejewitsch} (9.\,12.\,1842 Moskau – 8.\,2.\,1921 Dmitrov), \emph{Schriftsteller, Geograf, Revolutionär}|pw}« zu lesen (siehe 7. 5. 1923, 14. 6. 1923).}}}\label{K_L03388-11}, \strikeout{d\textcolor{gray}{i}} (im{ }ſelben Verlag\orgindex{Robert Lutz@Robert Lutz|pwv}e
               erſchienen), \strikeout{die} deren Lektüre ich Dir dringend
               empfehle.\pend
           
\pstart
           Mit \label{K_L03388-12v}\edtext{Frankfurt\oindex{Frankfurt am Main@\textbf{Frankfurt am Main}, \emph{Hauptstadt}|pw}\pwindex{Rottenberg, Theodore 7.\,9.\,1875 – 5.\,4.\,1945 Limburg an der Lahn@\textsc{Rottenberg, Theodore} (7.\,9.\,1875 – 5.\,4.\,1945 Limburg an der Lahn)|pwv}}{\lemma{\textnormal{\emph{Frankfurt}}}\Cendnote{\textnormal{Bezug auf das Verhältnis mit und zu
                     Theodore Rottenberg\pwindex{Rottenberg, Theodore 7.\,9.\,1875 – 5.\,4.\,1945 Limburg an der Lahn@\textsc{Rottenberg, Theodore} (7.\,9.\,1875 – 5.\,4.\,1945 Limburg an der Lahn)|pwk}, mit der Goldmann\pwindex{Goldmann, Paul 31.\,1.\,1865 Breslau – 25.\,9.\,1935 Wien@\textsc{Goldmann, Paul} (31.\,1.\,1865 Breslau – 25.\,9.\,1935 Wien), \emph{Schriftsteller, Journalist}|pwk} in einer intimen Beziehung
                  stand}}}\label{K_L03388-12} bin ich in reger Correſpondenz. Hier und da fährt ein Sturm
               dazwiſchen. Ich weiß nicht, was werden{ }ſoll. Ich mag mich an dieſe Frau\pwindex{Rottenberg, Theodore 7.\,9.\,1875 – 5.\,4.\,1945 Limburg an der Lahn@\textsc{Rottenberg, Theodore} (7.\,9.\,1875 – 5.\,4.\,1945 Limburg an der Lahn)|pwv} nicht durch Heirath binden, weil das
               mein \strikeout{Ru\textcolor{gray}{i}} wirthſchaftlicher {\pb}Ruin wäre und weil auch,
               infolge der Affaire\pwindex{?? [Partner von Theodore Rottenberg, Ende 1902/Anfang 1903] @\textsc{?? [Partner von Theodore Rottenberg, Ende 1902/Anfang 1903]}|pwv} in
               dieſem Winter, viel \label{K_L03388-13v}\edtext{Schmutz}{\lemma{\textnormal{\emph{Schmutz}}}\Cendnote{\textnormal{Siehe XXXX Auszeichnungsfehler: Dokument L03386 nicht gefunden. }}}\label{K_L03388-13} an ihr\pwindex{Rottenberg, Theodore 7.\,9.\,1875 – 5.\,4.\,1945 Limburg an der Lahn@\textsc{Rottenberg, Theodore} (7.\,9.\,1875 – 5.\,4.\,1945 Limburg an der Lahn)|pw} haftet; anderſeits kann ich nicht einmal
               den Gedanken ertragen, auf{ }ſie zu verzichten.\pend
           
\pstart
           Grüße Deine Frau\pwindex{Schnitzler, Olga 17.\,1.\,1882 Wien – 13.\,1.\,1970 Lugano@\textsc{Schnitzler, Olga} (17.\,1.\,1882 Wien – 13.\,1.\,1970 Lugano), \emph{Schauspielerin, Sängerin}|pwv}
               vielmals,{ }ſchreib mir bald und{ }ſei{ }ſelbſt herzlichſt gegrüßt von Deinem getreuen {\\[\baselineskip]}\spacefill\mbox{Paul Goldmann.}\pend
           \leftskip=0em{}\selectlanguage{ngerman}\endnumbering\briefempfaengerindex{Schnitzler, Arthur@\textsc{Schnitzler, Arthur}!zzzGoldmann, Paul@\emph{von Paul Goldmann}!1903-11-141@{14. 11. [1903]}|)be}\mylabel{L03388h}  \newcommand{\dateiname}{L03388}\newcommand{\titel}{Paul Goldmann an Arthur Schnitzler, 14. 11. [1903]}\newcommand{\editorInnen}{Martin Anton Müller und Laura Untner}%% latex-leseansicht-abspann.tex
%% Abspann für die Leseansicht.
%% Der Schalter \ifkorrekturansicht ist bereits durch den Vorspann gesetzt.

%% latex-abspann.tex
%% Gemeinsamer Abspann für Korrekturansicht und Leseansicht.
%% Setzt den Schalter \ifkorrekturansicht voraus (gesetzt in den
%% einbindenden Dateien latex-korrekturansicht-abspann.tex bzw.
%% latex-leseansicht-abspann.tex).
%% ---------------------------------------------------------------

\normalsize

% Das esempio-Environment wird nur in der Leseansicht benötigt
\ifkorrekturansicht\else
\newenvironment{esempio}[3]%
{
    \vspace{1.5ex}
    \rlap{\underline{#1}}
    \par
    \setlength{\parindent}{0cm}
    \nopagebreak
    \leftskip=#2cm
    \rightskip=#3cm
}
{
    \par
}
\fi

\doendnotes{C}
\bigskip
\vfill

\clearpage

\footnotesize

\ifkorrekturansicht
  \lohead{\textsc{register}}
\fi

% theindex-Environment neu definieren ohne reledmac
\makeatletter
\renewenvironment{theindex}{%
  \ifkorrekturansicht
    \section*{\indexname}%
  \else
    \subsubsection*{Index der erwähnten Entitäten}%
  \fi
  \setlength{\parindent}{0pt}%
  \setlength{\parskip}{0pt plus 0.3pt}%
  \let\item\@idxitem
}{%
  \ifkorrekturansicht\clearpage\fi
}
\makeatother

\IfFileExists{\jobname-pw.ind}{\input{\jobname-pw.ind}}{}

% Quellenangabe nur in der Leseansicht
\ifkorrekturansicht\else
% Fallback-Definitionen, falls die .tex-Datei \titel etc. nicht gesetzt hat
\providecommand{\titel}{}
\providecommand{\editorInnen}{}
\providecommand{\dateiname}{\jobname}

\vspace{3cm}

\vfill

\footnotesize
\textsc{Quelle}: \titel. Herausgegeben von {\editorInnen}. In: \emph{Arthur Schnitzler: Briefwechsel mit Autorinnen und Autoren}.
 Digitale Edition, https://schnitzler-briefe.acdh.oeaw.ac.at/{\dateiname}.html (Stand \today)
\fi

\end{document}


