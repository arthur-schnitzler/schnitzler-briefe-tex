%% latex-leseansicht-vorspann.tex
%% Vorspann für die Leseansicht.
%% Lädt die gemeinsame Datei latex-vorspann.tex mit nicht gesetztem Schalter.

\newif\ifkorrekturansicht
\korrekturansichtfalse

\input{../tex-inputs/latex-vorspann}


\section[Theodor Herzl an Arthur Schnitzler, 15. 6. 1893]{L03830 Theodor Herzl an Arthur Schnitzler, 15. 6. 1893}
\nopagebreak\mylabel{L03830v}
\rehead{ }\normalsize\beginnumbering\briefempfaengerindex{Schnitzler, Arthur@\textsc{Schnitzler, Arthur}!zzzHerzl, Theodor@\emph{von Theodor Herzl}!1893-06-151@{15. 6. 1893}|(be}
\toendnotes[C]{\smallbreak\pagebreak[2]}
\correspDesc{Versand  durch Theodor Herzl am 15. 6. 1893 in Paris
\newline{}Erhalt  durch Arthur Schnitzler im Zeitraum [16. 6. 1893
                  – 20. 6. 1893?] in Wien}\toendnotes[C]{\smallbreak}
\Standort{CUL, Schnitzler, B 39.}
\physDesc{Brief, 3 Blätter, 9 Seiten, 6518 Zeichen
\newline{}Handschrift: schwarze Tinte, lateinische Kurrent
\newline{}Ordnung: 1) mit Bleistift von unbekannter Hand nummeriert: »10«  2) mit Bleistift mutmaßlich von Leon
                                    Kellner\pwindex{Kellner, Leon 17.\,4.\,1859 Tarnów – 5.\,12.\,1928 Wien@\textsc{Kellner, Leon} (17.\,4.\,1859 Tarnów – 5.\,12.\,1928 Wien), \emph{Zionist, Literaturhistoriker, Anglist}|pw} Markierung interessanter Stellen 3) mit blauem Buntstift von Leon Kellner\pwindex{Kellner, Leon 17.\,4.\,1859 Tarnów – 5.\,12.\,1928 Wien@\textsc{Kellner, Leon} (17.\,4.\,1859 Tarnów – 5.\,12.\,1928 Wien), \emph{Zionist, Literaturhistoriker, Anglist}|pw}
                                 Markierung von Stellen für die Publikation}
\buchAbdrucke{\weitereDrucke{1) \pwindex{Kellner, Leon 17.\,4.\,1859 Tarnów – 5.\,12.\,1928 Wien@\textsc{Kellner, Leon} (17.\,4.\,1859 Tarnów – 5.\,12.\,1928 Wien), \emph{Zionist, Literaturhistoriker, Anglist}!Theodor Herzls Lehrjahre (1860–1895). Nach den handschriftlichen Quellen@\strich\emph{Theodor Herzls Lehrjahre (1860–1895). Nach den handschriftlichen Quellen}|pwk}Leon Kellner: \emph{Theodor Herzls Lehrjahre (1860–1895). Nach den handschriftlichen Quellen}. Wien, Berlin: \emph{R. Löwit-Verlag} 1920, S. 114–117.} \weitereDrucke{2) Theodor Herzl: \emph{Briefe und
                        autobiographische Notizen 1866–1895}. Bearbeitet von Johannes Wachten in Zusammenarbeit mit Chaya Harel, Daisy Tycho und Manfred Winkler. Berlin, Frankfurt am Main, Wien: \emph{Propyläen} 1983, S. 528–531 (Briefe und Tagebücher. Herausgegeben von Alex Bein, Hermann Greive, Moshe Schaerf, Julius H. Schoeps und Johannes Wachten, 1).} }\toendnotes[C]{\smallbreak}
\pstart
           {\pb}\textcolor{gray}{\textbf{NOUVELLE PRESSE LIBRE}}\orgindex{Neue Freie Presse@Neue Freie Presse|pw}\hfill \textcolor{gray}{\textbf{8, Rue de Monceau}}\oindex{8, rue de Monceau@\textbf{8, rue de Monceau}, \emph{Wohngebäude}|pw}\pend
           
\pstart
           \textcolor{gray}{\textbf{D\textsuperscript{r}{ }TH. HERZL}}\hfill 15. Juni 893\pend
           
\pstart{}Lieber Freund!\pend\vspace{0.5em}
\pstart
           Mein drittes liebes Kind\pwindex{Neumann, Margarethe 20.\,5.\,1893 Paris – 15.\,3.\,1943 Konzentrationslager Theresienstadt@\textsc{Neumann, Margarethe} (20.\,5.\,1893 Paris – 15.\,3.\,1943 Konzentrationslager Theresienstadt)|pwv},
               ein Mäderl u. es heisst Greterl\pwindex{Neumann, Margarethe 20.\,5.\,1893 Paris – 15.\,3.\,1943 Konzentrationslager Theresienstadt@\textsc{Neumann, Margarethe} (20.\,5.\,1893 Paris – 15.\,3.\,1943 Konzentrationslager Theresienstadt)|pw}, ist schon
               bald \label{K_L03830-1v}\edtext{vier Wochen alt}{\lemma{\textnormal{\emph{vier Wochen alt}}}\Cendnote{\textnormal{Am 20.\,5.\,1893 war die dritte
                  Tochter Margarethe\pwindex{Neumann, Margarethe 20.\,5.\,1893 Paris – 15.\,3.\,1943 Konzentrationslager Theresienstadt@\textsc{Neumann, Margarethe} (20.\,5.\,1893 Paris – 15.\,3.\,1943 Konzentrationslager Theresienstadt)|pwk}, später genannt Trude,
                  zur Welt gekommen.}}}\label{K_L03830-1}. Meine Frau\pwindex{Herzl, Julie 1.\,2.\,1868 Budapest – 10.\,11.\,1907 Wien@\textsc{Herzl, Julie} (1.\,2.\,1868 Budapest – 10.\,11.\,1907 Wien)|pwv} hat sich fast gänzlich erholt, u. wir denken an die
               Reise wenn nur nichts dazwischen kommt. Mein ältestes Mäderl\pwindex{Hüft, Pauline 29.\,3.\,1890 – 8.\,9.\,1930@\textsc{Hüft, Pauline} (29.\,3.\,1890 – 8.\,9.\,1930)|pwv} ist seit gestern
               krank. Der Arzt\pwindex{?? [Arzt der Familie Herzl in Paris] @\textsc{?? [Arzt der Familie Herzl in Paris]}|pwv} sagte uns
                  gestern Abends dass es vielleicht Masern werden. Heute
               meint er, dass es nicht dazu kommen würde. Aber sie fiebert noch stark.\pend
           
\pstart
           Jedenfalls war die Nacht für uns schlaflos. Meine arme Frau\pwindex{Herzl, Julie 1.\,2.\,1868 Budapest – 10.\,11.\,1907 Wien@\textsc{Herzl, Julie} (1.\,2.\,1868 Budapest – 10.\,11.\,1907 Wien)|pwv} sass auf einem Sessel. Ich hatte mir
               meinen Buben\pwindex{Herzl, Hans 10.\,6.\,1891 Wien – 14.\,9.\,1930 Bordeaux@\textsc{Herzl, Hans} (10.\,6.\,1891 Wien – 14.\,9.\,1930 Bordeaux)|pwv} ins
               Schreibzimmer gelegt, um wenns noch nicht zu {\pb}spät wäre die Ansteckung zu verhindern.
               Der Kerl\pwindex{Herzl, Hans 10.\,6.\,1891 Wien – 14.\,9.\,1930 Bordeaux@\textsc{Herzl, Hans} (10.\,6.\,1891 Wien – 14.\,9.\,1930 Bordeaux)|pwv} hat die ganze Nacht
               krakehlt, erst als ich ihm Prügel fest versprach, wurde er ruhig und sagte: Nit
               weinen, wieder lieb! – Er ist zwei Jahre alt.\pend
           
\pstart
           Kinder sind immerwährend zugleich Freude u. Angst, u. aus beiden Gründen wird Einem
               durch sie das Leben lieb.\pend
           
\pstart
           Auch in der Politik darf nichts dazwischen kommen, dann reisen wir am
                  26 ds. mit dem O. E.\orgindex{Orient Express@Orient Express|pw} nach Wien\oindex{Wien@\textbf{Wien}, \emph{Verwaltungsgebiet}|pw}. Ich möchte bevor wir nach Baden\oindex{Baden bei Wien@\textbf{Baden bei Wien}, \emph{Hauptstadt}|pw} gehen einigemal in die Wiener\oindex{Wien@\textbf{Wien}, \emph{Verwaltungsgebiet}|pw} Theater gehen. Ich weiss nicht mehr, wie sich der Lieutenant in den
               Backfisch verlieben kann u. umgekehrt. Gesteh’ ichs, ich sehne mich wieder nach der
               heimischen \label{K_L03830-2v}\edtext{Imbecillität}{\lemma{\textnormal{\emph{Imbecillität}}}\Cendnote{\textnormal{Schwachsinn}}}\label{K_L03830-2}.\pend
           
\pstart
           Mit dem Flüchtling\pwindex{Herzl, Theodor 2.\,5.\,1860 Budapest – 3.\,7.\,1904 Edlach@\textsc{Herzl, Theodor} (2.\,5.\,1860 Budapest – 3.\,7.\,1904 Edlach), \emph{Schriftsteller, Journalist}!Flüchtling. Lustspiel in einem Aufzug@\strich\emph{Der Flüchtling. Lustspiel in einem Aufzug}|pw} ists sonderbar. {\pb}Mein erster \label{K_L03830-3v}\edtext{Erfolg in Berlin\oindex{Berlin@\textbf{Berlin}, \emph{Hauptstadt}|pw}}{\lemma{\textnormal{\emph{Erfolg in Berlin}}}\Cendnote{\textnormal{ Die Berlinpremiere\eventindex{Berliner Theater@\textbf{Berliner Theater}!Aufführung von Der Flüchtling und Die Eine weint, die Andere lacht, 31.5.1893@Aufführung von Der Flüchtling und Die Eine weint, die Andere lacht, 31.5.1893|pwkv} von \emph{Der
                     Flüchtling. Lustspiel in einem Aufzug}\pwindex{Herzl, Theodor 2.\,5.\,1860 Budapest – 3.\,7.\,1904 Edlach@\textsc{Herzl, Theodor} (2.\,5.\,1860 Budapest – 3.\,7.\,1904 Edlach), \emph{Schriftsteller, Journalist}!Flüchtling. Lustspiel in einem Aufzug@\strich\emph{Der Flüchtling. Lustspiel in einem Aufzug}|pwk} von Theodor Herzl\pwindex{Herzl, Theodor 2.\,5.\,1860 Budapest – 3.\,7.\,1904 Edlach@\textsc{Herzl, Theodor} (2.\,5.\,1860 Budapest – 3.\,7.\,1904 Edlach), \emph{Schriftsteller, Journalist}|pwk} fand zusammen mit einer Aufführung von \emph{Die Eine weint, die Andere lacht. Schauspiel in
                     vier Akten}\pwindex{Dumanoir, Philippe 25.\,7.\,1806 – 13.\,11.\,1865@\textsc{Dumanoir, Philippe} (25.\,7.\,1806 – 13.\,11.\,1865), \emph{Schriftsteller/Schriftstellerin, Librettist/Librettistin}!Eine weint, die Andere lacht. Schauspiel in vier Akten@\strich\emph{Die Eine weint, die Andere lacht. Schauspiel in vier Akten}|pwk}\pwindex{Kéraniou, Ange de 4.\,5.\,1829 – 1872@\textsc{Kéraniou, Ange de} (4.\,5.\,1829 – 1872), \emph{Schriftsteller}!Eine weint, die Andere lacht. Schauspiel in vier Akten@\strich\emph{Die Eine weint, die Andere lacht. Schauspiel in vier Akten}|pwk} von Philippe Dumanoir\pwindex{Dumanoir, Philippe 25.\,7.\,1806 – 13.\,11.\,1865@\textsc{Dumanoir, Philippe} (25.\,7.\,1806 – 13.\,11.\,1865), \emph{Schriftsteller/Schriftstellerin, Librettist/Librettistin}|pwk}
                  und Ange de Kéraniou\pwindex{Kéraniou, Ange de 4.\,5.\,1829 – 1872@\textsc{Kéraniou, Ange de} (4.\,5.\,1829 – 1872), \emph{Schriftsteller}|pwk} am 31. 5. 1893 am \emph{Berliner
                     Theater}\orgindex{Berliner Theater@Berliner Theater|pwk} statt. }}}\label{K_L03830-3}. Als ich einige Tage später davon hörte, nahm ich
               das Buch\pwindex{Herzl, Theodor 2.\,5.\,1860 Budapest – 3.\,7.\,1904 Edlach@\textsc{Herzl, Theodor} (2.\,5.\,1860 Budapest – 3.\,7.\,1904 Edlach), \emph{Schriftsteller, Journalist}!Flüchtling. Lustspiel in einem Aufzug@\strich\emph{Der Flüchtling. Lustspiel in einem Aufzug}|pwv} vor, las es, war über
               die saloppe Sprache entsetzt; nur das, was die vornehmen \label{K_L03830-4v}\edtext{Kritiker rügten}{\lemma{\textnormal{\emph{Kritiker rügten}}}\Cendnote{\textnormal{Der Kritiker Ludwig Berthold\pwindex{Berthold, Ludwig @\textsc{Berthold, Ludwig}, \emph{Journalist, Theaterkritiker}|pwk} bemängelte
                  die Nebenhandlung: »Nebenbei läuft noch eine sehr gleichgiltige Gesellschafterin
                  und deren vom Schnupfen befallener, immerwährend niesender Liebhaber. Diese grobe
                  Geschmacklosigkeit [...] wäre im Stande gewesen gewesen, das Interesse [...]
                  abzuschwächen, wenn nicht Frl. Nuscha
                     Butze\pwindex{Butze, Nuscha 22.\,2.\,1860 Berlin – 10.\,12.\,1913 ebd.@\textsc{Butze, Nuscha} (22.\,2.\,1860 Berlin – 10.\,12.\,1913 ebd.), \emph{Schauspielerin, Theaterdirektorin}|pwk} und Herr Ludwig Stahl\pwindex{Stahl, Ludwig 4.\,4.\,1856 Brünn – 25.\,8.\,1908 Blankenberge@\textsc{Stahl, Ludwig} (4.\,4.\,1856 Brünn – 25.\,8.\,1908 Blankenberge), \emph{Regisseur, Schauspieler}|pwk} den
                  erkälteten Herrn etwas in den Hintergrund gedrängt hätten [...].« (L. B.\pwindex{Berthold, Ludwig @\textsc{Berthold, Ludwig}, \emph{Journalist, Theaterkritiker}|pwkv}: \emph{Berliner Theaterbericht}\pwindex{Berthold, Ludwig @\textsc{Berthold, Ludwig}, \emph{Journalist, Theaterkritiker}!Berliner Theaterbericht@\strich\emph{Berliner Theaterbericht}|pwk}. In: \emph{Die Gesellschaft. Illustrirtes
                     Wochenblatt}\pwindex{Gesellschaft. Politisches illustriertes Wochenblatt@\emph{Die Gesellschaft. Politisches illustriertes Wochenblatt}|pwk}, Jg. 7, Nr. 23, 11. 6. 1893,
               S. 10)}}}\label{K_L03830-4}, hat mich ergötzt: das Niesen des Eifersüchtigen. Es ist ein
               prächtiger Bühneneinfall, denn an der Stelle ist nicht mehr Zeit, auch nur mit einem
               Wort zu verhindern, dass die \substVorne{}\textsuperscript{Leute}\substDazwischen{}Zuschauer\substHinten{} den  Streit für ernst halten, u. damit man die Angst der Margarethe, auf die
               es ankommt, beobachten könne, muss die \label{K_L03830-5v}\edtext{Contrahage}{\lemma{\textnormal{\emph{Contrahage}}}\Cendnote{\textnormal{Forderung zum
                  Duell}}}\label{K_L03830-5} spassig sein.\pend
           
\pstart
           Was sagen Sie, mit welcher \label{K_L03830-6v}\edtext{\begin{otherlanguage}{french}désinvolture\end{otherlanguage}}{\lemma{\textnormal{\emph{désinvolture}}}\Cendnote{\textnormal{französisch: Ungeniertheit}}}\label{K_L03830-6}{ }ich
               mich lobe? Deutlicher als alles sagt Ihnen dies, dass ich von einem Abgeschiedenen
               spreche.\pend
           
\pstart
           Beurtheilen Sie die Aufführung des Flüchtlings\pwindex{Herzl, Theodor 2.\,5.\,1860 Budapest – 3.\,7.\,1904 Edlach@\textsc{Herzl, Theodor} (2.\,5.\,1860 Budapest – 3.\,7.\,1904 Edlach), \emph{Schriftsteller, Journalist}!Flüchtling. Lustspiel in einem Aufzug@\strich\emph{Der Flüchtling. Lustspiel in einem Aufzug}|pw}
               nicht falsch. Sie braucht sie ebensowenig zu demüthigen, wie die Erfolge der
               gewöhnlichen Dutzendscribenten. Verstehen Sie {\pb}das Leben! Hier die Geschichte des Flüchtlings\pwindex{Herzl, Theodor 2.\,5.\,1860 Budapest – 3.\,7.\,1904 Edlach@\textsc{Herzl, Theodor} (2.\,5.\,1860 Budapest – 3.\,7.\,1904 Edlach), \emph{Schriftsteller, Journalist}!Flüchtling. Lustspiel in einem Aufzug@\strich\emph{Der Flüchtling. Lustspiel in einem Aufzug}|pw}. 1887 wollte ich eine
                  italienische\oindex{Italien@\textbf{Italien}|pw} Reise machen. Reisegeld gabs
               verflucht wenig. Groller\pwindex{Groller, Balduin 5.\,9.\,1848 Arad – 22.\,3.\,1916 Wien@\textsc{Groller, Balduin} (5.\,9.\,1848 Arad – 22.\,3.\,1916 Wien), \emph{Schriftsteller, Journalist}|pw} (Illustrirte Zeitung\orgindex{Neue Illustrierte Zeitung@Neue Illustrierte Zeitung|pw}) war charmant genug, mir \strikeout{den} damals zu sagen, ich solle, wie \label{K_L03830-7v}\edtext{für die blaue Donau\pwindex{der schönen blauen Donau@\emph{An der schönen blauen Donau}|pw}}{\lemma{\textnormal{\emph{für die blaue Donau}}}\Cendnote{\textnormal{In der Zeitschrift \emph{An der schönen Blauen Donau}\pwindex{der schönen blauen Donau@\emph{An der schönen blauen Donau}|pwk} waren von Herzl\pwindex{Herzl, Theodor 2.\,5.\,1860 Budapest – 3.\,7.\,1904 Edlach@\textsc{Herzl, Theodor} (2.\,5.\,1860 Budapest – 3.\,7.\,1904 Edlach), \emph{Schriftsteller, Journalist}|pwk} im Vorjahr der dramatische Scherz \emph{Schlechte Nachrichten}\pwindex{Herzl, Theodor 2.\,5.\,1860 Budapest – 3.\,7.\,1904 Edlach@\textsc{Herzl, Theodor} (2.\,5.\,1860 Budapest – 3.\,7.\,1904 Edlach), \emph{Schriftsteller, Journalist}!Schlechte Nachrichten. Ein dramatischer Scherz@\strich\emph{Schlechte Nachrichten. Ein dramatischer Scherz}|pwk} und die Novelle \emph{Der sechste Welttheil}\pwindex{Herzl, Theodor 2.\,5.\,1860 Budapest – 3.\,7.\,1904 Edlach@\textsc{Herzl, Theodor} (2.\,5.\,1860 Budapest – 3.\,7.\,1904 Edlach), \emph{Schriftsteller, Journalist}!sechste Welttheil@\strich\emph{Der sechste Welttheil}|pwk} erschienen (\emph{Schlechte Nachrichten. Ein dramatischer
                        Scherz}\pwindex{Herzl, Theodor 2.\,5.\,1860 Budapest – 3.\,7.\,1904 Edlach@\textsc{Herzl, Theodor} (2.\,5.\,1860 Budapest – 3.\,7.\,1904 Edlach), \emph{Schriftsteller, Journalist}!Schlechte Nachrichten. Ein dramatischer Scherz@\strich\emph{Schlechte Nachrichten. Ein dramatischer Scherz}|pwk}. In: \emph{An der schönen blauen
                        Donau}\pwindex{der schönen blauen Donau@\emph{An der schönen blauen Donau}|pwk}, Jg. 1, H. 2, 1. 2. 1886, S. 50–52 und \emph{Der sechste Welttheil}\pwindex{Herzl, Theodor 2.\,5.\,1860 Budapest – 3.\,7.\,1904 Edlach@\textsc{Herzl, Theodor} (2.\,5.\,1860 Budapest – 3.\,7.\,1904 Edlach), \emph{Schriftsteller, Journalist}!sechste Welttheil@\strich\emph{Der sechste Welttheil}|pwk}. In: \emph{An der schönen blauen Donau}\pwindex{der schönen blauen Donau@\emph{An der schönen blauen Donau}|pwk}, Jg. 1, H. 9,
                        15. 5. 1886, S. 257–259). 1887 wurde dort
                  sein Einakter \emph{Die causa Hirschkorn}\pwindex{Herzl, Theodor 2.\,5.\,1860 Budapest – 3.\,7.\,1904 Edlach@\textsc{Herzl, Theodor} (2.\,5.\,1860 Budapest – 3.\,7.\,1904 Edlach), \emph{Schriftsteller, Journalist}!causa Hirschkorn. Lustspiel in einem Act@\strich\emph{Die causa Hirschkorn. Lustspiel in einem Act}|pwk} abgedruckt (\emph{Die causa Hirschkorn. Lustspiel in einem
                        Act}\pwindex{Herzl, Theodor 2.\,5.\,1860 Budapest – 3.\,7.\,1904 Edlach@\textsc{Herzl, Theodor} (2.\,5.\,1860 Budapest – 3.\,7.\,1904 Edlach), \emph{Schriftsteller, Journalist}!causa Hirschkorn. Lustspiel in einem Act@\strich\emph{Die causa Hirschkorn. Lustspiel in einem Act}|pwk}. In: \emph{An der schönen blauen
                        Donau}\pwindex{der schönen blauen Donau@\emph{An der schönen blauen Donau}|pwk}, Jg. 2, H. 11, 1. 6. 1887,
                  S. 254–256.).}}}\label{K_L03830-7} etwas Dramatisches \label{K_L03830-8v}\edtext{für ihn schreiben}{\lemma{\textnormal{\emph{für ihn schreiben}}}\Cendnote{\textnormal{\emph{Der Flüchtling. Lustspiel in einem Act}\pwindex{Herzl, Theodor 2.\,5.\,1860 Budapest – 3.\,7.\,1904 Edlach@\textsc{Herzl, Theodor} (2.\,5.\,1860 Budapest – 3.\,7.\,1904 Edlach), \emph{Schriftsteller, Journalist}!Flüchtling. Lustspiel in einem Aufzug@\strich\emph{Der Flüchtling. Lustspiel in einem Aufzug}|pwk}.
                     In: \emph{Neue Illustrirte Zeitung}\pwindex{Neue Illustrierte Zeitung@\emph{Neue Illustrierte Zeitung}|pwk}, Jg. 15,
                     Bd. 2, Nr. 36, 5. 6. 1887, S. 567–569 und Nr. 37,
                        12. 6. 1887, S. 579–582.}}}\label{K_L03830-8}. Gerade sauste mir der
               Ihnen bekannte Einfall dieses Einakters\pwindex{Herzl, Theodor 2.\,5.\,1860 Budapest – 3.\,7.\,1904 Edlach@\textsc{Herzl, Theodor} (2.\,5.\,1860 Budapest – 3.\,7.\,1904 Edlach), \emph{Schriftsteller, Journalist}!Flüchtling. Lustspiel in einem Aufzug@\strich\emph{Der Flüchtling. Lustspiel in einem Aufzug}|pwv} durch den Kopf. Hingesetzt u. hingeschleudert. Ich glaube in drei
               Tagen. Ich wollte schon abreisen. Nicht mehr deutlich weiss ich ob ich das Honorar
               vorgeschossen bekam. Ich vermuthe es, denn ich reiste ab u. schrieb mich dann bis Neapel\oindex{Neapel@\textbf{Neapel}|pw} durch. (Freilich hat mein guter Vater\pwindex{Herzl, Jakob 14.\,3.\,1837 Zemun – 9.\,6.\,1902 Wien@\textsc{Herzl, Jakob} (14.\,3.\,1837 Zemun – 9.\,6.\,1902 Wien), \emph{Bankdirektor, Großkaufmann}|pw} auch was hergegeben.) Dieser Schmarrn\pwindex{Herzl, Theodor 2.\,5.\,1860 Budapest – 3.\,7.\,1904 Edlach@\textsc{Herzl, Theodor} (2.\,5.\,1860 Budapest – 3.\,7.\,1904 Edlach), \emph{Schriftsteller, Journalist}!Flüchtling. Lustspiel in einem Aufzug@\strich\emph{Der Flüchtling. Lustspiel in einem Aufzug}|pwv}, den ich wie alle
               meine Stücke dem Burgtheater\orgindex{Burgtheater@Burgtheater|pw} einreichte, wurde
               ich weiss nicht mehr warum – gewiss aus keinem literarischen Grunde – angenommen
               u. lag dann zwei Jahre. Förster\pwindex{Förster, August 3.\,6.\,1828 Bad Lauchstädt – 22.\,12.\,1889 Semmering@\textsc{Förster, August} (3.\,6.\,1828 Bad Lauchstädt – 22.\,12.\,1889 Semmering), \emph{Theaterleiter, Regisseur, Schauspieler}|pw} wurde
               Director. Ich war bei der Allg. Ztg\orgindex{Wiener Allgemeine Zeitung@Wiener Allgemeine Zeitung|pw}. Ich hatte
                  {\pb}in der Redaction\orgindex{Wiener Allgemeine Zeitung@Wiener Allgemeine Zeitung|pwv} einen unangenehmen Collegen\pwindex{?? [Kulturredakteur der Wiener Allgemeinen Zeitung] @\textsc{?? [Kulturredakteur der Wiener Allgemeinen Zeitung]}|pwv}, einen boshaften
               Narren, der mich molestirte wo er konnte u. mit dem ich nicht einmal auf dem
               Grussfuss stand. Dieser schrieb \label{K_L03830-9v}\edtext{eine
               hämische Notiz\pwindex{?? [Kulturredakteur der Wiener Allgemeinen Zeitung] @\textsc{?? [Kulturredakteur der Wiener Allgemeinen Zeitung]}!Theater an der Wien]@\strich\emph{[Theater an der Wien]}|pwv}}{\lemma{\textnormal{\emph{eine
               hämische Notiz}}}\Cendnote{\textnormal{Am 4. 4. 1889 berichtete
                  die \emph{Wiener Allgemeine Zeitung}\pwindex{Wiener Allgemeine Zeitung@\emph{Wiener Allgemeine Zeitung}|pwk} über eine neue
                     Aufführung\eventindex{Theater an der Wien@\textbf{Theater an der Wien}!Aufführung von O, diese Schwiegermutter, 2.4.1889@Aufführung von O, diese Schwiegermutter{\rufezeichen}, 2.4.1889|pwkv} des
                  Schwanks \emph{O, diese Schwiegermutter}\pwindex{\textcolor{red}{\textsuperscript{XXXX indx1}}!O, diese Schwiegermutter@\strich\emph{O, diese Schwiegermutter{\rufezeichen}}|pwk}\pwindex{\textcolor{red}{\textsuperscript{XXXX indx1}}!O, diese Schwiegermutter@\strich\emph{O, diese Schwiegermutter{\rufezeichen}}|pwk}. Die
                  Hauptrolle, die seit der Premiere\eventindex{Theater an der Wien@\textbf{Theater an der Wien}!Premiere von Die Zaubergeige, O, diese Schwiegermutter, 1.12.1888@Premiere von Die Zaubergeige, O, diese Schwiegermutter{\rufezeichen}, 1.12.1888|pwkv} am 1. 12. 1888 der Schauspieler Alexander Giradi\pwindex{Girardi, Alexander 5.\,12.\,1850 Graz – 20.\,4.\,1918 Wien@\textsc{Girardi, Alexander} (5.\,12.\,1850 Graz – 20.\,4.\,1918 Wien), \emph{Schauspieler}|pwk} innegehabt hatte, wurde nun von Heinrich Förster\pwindex{Förster, Heinrich 27.\,6.\,1859 Wien – 8.\,9.\,1897@\textsc{Förster, Heinrich} (27.\,6.\,1859 Wien – 8.\,9.\,1897), \emph{Schauspieler, Theaterregisseur}|pwk}, Sohn des Burgtheaterdirektors\pwindex{Förster, August 3.\,6.\,1828 Bad Lauchstädt – 22.\,12.\,1889 Semmering@\textsc{Förster, August} (3.\,6.\,1828 Bad Lauchstädt – 22.\,12.\,1889 Semmering), \emph{Theaterleiter, Regisseur, Schauspieler}|pwkv}
                  übernommen. In der Kritik\pwindex{?? [Kulturredakteur der Wiener Allgemeinen Zeitung] @\textsc{?? [Kulturredakteur der Wiener Allgemeinen Zeitung]}!Theater an der Wien]@\strich\emph{[Theater an der Wien]}|pwkv}
                  heißt es u. a.: »Sagen wir es rund heraus: Herr Förster\pwindex{Förster, Heinrich 27.\,6.\,1859 Wien – 8.\,9.\,1897@\textsc{Förster, Heinrich} (27.\,6.\,1859 Wien – 8.\,9.\,1897), \emph{Schauspieler, Theaterregisseur}|pw} hat gestern den Erwartungen
                     keineswegs entsprochen. [...], man kann sich keinen grelleren Kontrast denken,
                     als Girardi\pwindex{Girardi, Alexander 5.\,12.\,1850 Graz – 20.\,4.\,1918 Wien@\textsc{Girardi, Alexander} (5.\,12.\,1850 Graz – 20.\,4.\,1918 Wien), \emph{Schauspieler}|pw} und Herrn Förster\pwindex{Förster, Heinrich 27.\,6.\,1859 Wien – 8.\,9.\,1897@\textsc{Förster, Heinrich} (27.\,6.\,1859 Wien – 8.\,9.\,1897), \emph{Schauspieler, Theaterregisseur}|pw} in der Rolle des Henri Duval. [...] Zum
                     Bonvivant mangeln ihm alle Eigenschaften.« (\emph{[Theater an der Wien]}\pwindex{?? [Kulturredakteur der Wiener Allgemeinen Zeitung] @\textsc{?? [Kulturredakteur der Wiener Allgemeinen Zeitung]}!Theater an der Wien]@\strich\emph{[Theater an der Wien]}|pwk}. In: \emph{Wiener Allgemeine Zeitung}\pwindex{Wiener Allgemeine Zeitung@\emph{Wiener Allgemeine Zeitung}|pwk}, Nr. 3257,
                        4. 4. 1889, S. 4.)}}}\label{K_L03830-9} über Foersters\pwindex{Förster, August 3.\,6.\,1828 Bad Lauchstädt – 22.\,12.\,1889 Semmering@\textsc{Förster, August} (3.\,6.\,1828 Bad Lauchstädt – 22.\,12.\,1889 Semmering), \emph{Theaterleiter, Regisseur, Schauspieler}|pw}{ }Sohn\pwindex{Förster, Heinrich 27.\,6.\,1859 Wien – 8.\,9.\,1897@\textsc{Förster, Heinrich} (27.\,6.\,1859 Wien – 8.\,9.\,1897), \emph{Schauspieler, Theaterregisseur}|pw}. Förster\pwindex{Förster, August 3.\,6.\,1828 Bad Lauchstädt – 22.\,12.\,1889 Semmering@\textsc{Förster, August} (3.\,6.\,1828 Bad Lauchstädt – 22.\,12.\,1889 Semmering), \emph{Theaterleiter, Regisseur, Schauspieler}|pw} glaubte, dass mein »Kamerad\pwindex{?? [Kulturredakteur der Wiener Allgemeinen Zeitung] @\textsc{?? [Kulturredakteur der Wiener Allgemeinen Zeitung]}|pwv}« mich Unaufgeführten rächen wollte u. \label{K_L03830-10v}\edtext{setzte den Flüchtling\pwindex{Herzl, Theodor 2.\,5.\,1860 Budapest – 3.\,7.\,1904 Edlach@\textsc{Herzl, Theodor} (2.\,5.\,1860 Budapest – 3.\,7.\,1904 Edlach), \emph{Schriftsteller, Journalist}!Flüchtling. Lustspiel in einem Aufzug@\strich\emph{Der Flüchtling. Lustspiel in einem Aufzug}|pw} erschrocken an}{\lemma{\textnormal{\emph{setzte … an}}}\Cendnote{\textnormal{Am
                     4. 5. 1889 erlebte Herzls\pwindex{Herzl, Theodor 2.\,5.\,1860 Budapest – 3.\,7.\,1904 Edlach@\textsc{Herzl, Theodor} (2.\,5.\,1860 Budapest – 3.\,7.\,1904 Edlach), \emph{Schriftsteller, Journalist}|pwk}
                  Lustspiel \emph{Der Flüchtling}\pwindex{Herzl, Theodor 2.\,5.\,1860 Budapest – 3.\,7.\,1904 Edlach@\textsc{Herzl, Theodor} (2.\,5.\,1860 Budapest – 3.\,7.\,1904 Edlach), \emph{Schriftsteller, Journalist}!Flüchtling. Lustspiel in einem Aufzug@\strich\emph{Der Flüchtling. Lustspiel in einem Aufzug}|pwk} seine Uraufführung\eventindex{Burgtheater@\textbf{Burgtheater}!Premiere von Der Schierling, Uraufführung von Im Bunde der Dritte, Der Flüchtling, 4.5.1889@Premiere von Der Schierling, Uraufführung von Im Bunde der Dritte, Der Flüchtling, 4.5.1889|pwkv} am \emph{Burgtheater}\orgindex{Burgtheater@Burgtheater|pwk}.}}}\label{K_L03830-10}. Sind das Komödien,
               was?\pend
           
\pstart
           Noch besser die Berliner\oindex{Berlin@\textbf{Berlin}, \emph{Hauptstadt}|pw} Geschichte des Flüchtlings\pwindex{Herzl, Theodor 2.\,5.\,1860 Budapest – 3.\,7.\,1904 Edlach@\textsc{Herzl, Theodor} (2.\,5.\,1860 Budapest – 3.\,7.\,1904 Edlach), \emph{Schriftsteller, Journalist}!Flüchtling. Lustspiel in einem Aufzug@\strich\emph{Der Flüchtling. Lustspiel in einem Aufzug}|pw}. Sie wissen dass ich mit meinem Stück
                  »Der Bernhardiner\pwindex{Herzl, Theodor 2.\,5.\,1860 Budapest – 3.\,7.\,1904 Edlach@\textsc{Herzl, Theodor} (2.\,5.\,1860 Budapest – 3.\,7.\,1904 Edlach), \emph{Schriftsteller, Journalist}!Was wird man sagen?@\strich\emph{Was wird man sagen?}|pw}« –  nicht mein
               schlechtestes, was freilich nichts sagen will – am Berliner\oindex{Berlin@\textbf{Berlin}, \emph{Hauptstadt}|pw} Theater einen der beschämendsten Durchfälle am »Berliner Theater\orgindex{Berliner Theater@Berliner Theater|pw}« erlitt. Es war ein Lustspiel, das Barnay\pwindex{Barnay, Ludwig 11.\,2.\,1842 Budapest – 1.\,2.\,1924@\textsc{Barnay, Ludwig} (11.\,2.\,1842 Budapest – 1.\,2.\,1924), \emph{Schriftsteller, Schauspieler, Theaterdirektor}|pw} weil er eine Rolle für sich zurecht
               schneidern wollte als Schauspiel spielte{[}.{]}{\pb}Es war eine zu verstohlene Satire auf
               die Sentimentalität u. jene Halben, die sich vom \label{K_L03830-11v}\edtext{\begin{otherlanguage}{french}qu’ en dira-t-on\end{otherlanguage}}{\lemma{\textnormal{\emph{qu’ en dira-t-on}}}\Cendnote{\textnormal{französisch: was man sagen wird, Gerede
                  der Leute}}}\label{K_L03830-11}{ }leiten lassen. Alle Absichten wurden ins Gegentheil verkehrt,
               u. zw. unter meinen Augen. Ich war schwach genug zu allem Ja zu sagen, aber
               hauptsächlich war ich wirthschaftlich schwach. Ich brauchte \label{K_L03830-12v}\edtext{die Aufführung}{\lemma{\textnormal{\emph{die Aufführung}}}\Cendnote{\textnormal{
                  Die Theateruraufführung von \emph{Was wird man sagen?}\pwindex{Herzl, Theodor 2.\,5.\,1860 Budapest – 3.\,7.\,1904 Edlach@\textsc{Herzl, Theodor} (2.\,5.\,1860 Budapest – 3.\,7.\,1904 Edlach), \emph{Schriftsteller, Journalist}!Was wird man sagen?@\strich\emph{Was wird man sagen?}|pwk} alias \emph{Der Bernhardiner}\pwindex{Herzl, Theodor 2.\,5.\,1860 Budapest – 3.\,7.\,1904 Edlach@\textsc{Herzl, Theodor} (2.\,5.\,1860 Budapest – 3.\,7.\,1904 Edlach), \emph{Schriftsteller, Journalist}!Was wird man sagen?@\strich\emph{Was wird man sagen?}|pwk} von Theodor Herzl\pwindex{Herzl, Theodor 2.\,5.\,1860 Budapest – 3.\,7.\,1904 Edlach@\textsc{Herzl, Theodor} (2.\,5.\,1860 Budapest – 3.\,7.\,1904 Edlach), \emph{Schriftsteller, Journalist}|pwk}\eventindex{Berliner Theater@\textbf{Berliner Theater}!Uraufführung von Der Bernhardiner, 30.10.1890@Uraufführung von Der Bernhardiner, 30.10.1890|pwk} fand am 30. 10. 1890 am \emph{Berliner Theater}\orgindex{Berliner Theater@Berliner Theater|pwk} statt.}}}\label{K_L03830-12}.\pend
           
\pstart
           Barnay\pwindex{Barnay, Ludwig 11.\,2.\,1842 Budapest – 1.\,2.\,1924@\textsc{Barnay, Ludwig} (11.\,2.\,1842 Budapest – 1.\,2.\,1924), \emph{Schriftsteller, Schauspieler, Theaterdirektor}|pw} wusste wohl, dass ich ihm geschrieben
               u. gesagt hatte, dass mein Stück »Was wird man
                  sagen?\pwindex{Herzl, Theodor 2.\,5.\,1860 Budapest – 3.\,7.\,1904 Edlach@\textsc{Herzl, Theodor} (2.\,5.\,1860 Budapest – 3.\,7.\,1904 Edlach), \emph{Schriftsteller, Journalist}!Was wird man sagen?@\strich\emph{Was wird man sagen?}|pw}« ein Lustspiel sei, u. dass ich es als solches gespielt wünsche. Er
               sah auch, wie tapfer u. schweigsam ich den ganzen Misserfolg allein trug, ohne zu
               maukezen. Ich hätte aus der Verhunzung meines Stückes\pwindex{Herzl, Theodor 2.\,5.\,1860 Budapest – 3.\,7.\,1904 Edlach@\textsc{Herzl, Theodor} (2.\,5.\,1860 Budapest – 3.\,7.\,1904 Edlach), \emph{Schriftsteller, Journalist}!Was wird man sagen?@\strich\emph{Was wird man sagen?}|pw} immerhin ein Feuilleton herausfetzen können. Ich hatte aber nach der
               Niederlage die richtige Haltung {\pb}\strikeout{während} so wie ich sie vorher nicht hatte. Es wäre
               geschmacklos u. feig gewesen, die Schuld abzuwälzen.\pend
           
\pstart
           Aber wenn mich ganz Berlin\oindex{Berlin@\textbf{Berlin}, \emph{Hauptstadt}|pw} u. was dahinter steht
               – also alle deutschen Theater – für einen unfähigen Idioten tief unter nehmen wir an
                  Triesch\pwindex{Triesch, Friedrich Gustav 16.\,6.\,1845 Wien – 24.\,5.\,1907 ebd.@\textsc{Triesch, Friedrich Gustav} (16.\,6.\,1845 Wien – 24.\,5.\,1907 ebd.), \emph{Schriftsteller}|pw} halten mussten, der eine Barnay\pwindex{Barnay, Ludwig 11.\,2.\,1842 Budapest – 1.\,2.\,1924@\textsc{Barnay, Ludwig} (11.\,2.\,1842 Budapest – 1.\,2.\,1924), \emph{Schriftsteller, Schauspieler, Theaterdirektor}|pw} kannte das Unrecht, das ich ruhig
               aushielt. Nun, er lehnte dennoch ein Stück\pwindex{Herzl, Theodor 2.\,5.\,1860 Budapest – 3.\,7.\,1904 Edlach@\textsc{Herzl, Theodor} (2.\,5.\,1860 Budapest – 3.\,7.\,1904 Edlach), \emph{Schriftsteller, Journalist}!Prinzen aus Genieland. Lustspiel in 4 Akten@\strich\emph{Prinzen aus Genieland. Lustspiel in 4 Akten}|pwv} ab, mit dem ich \introOben{}vielleicht\introOben{}
               meine Revanche hätte nehmen können, obwol seine Regisseure es zur Aufführung
               empfahlen: das unter dem schlechten Titel Prinzen aus
                  Genieland\pwindex{Herzl, Theodor 2.\,5.\,1860 Budapest – 3.\,7.\,1904 Edlach@\textsc{Herzl, Theodor} (2.\,5.\,1860 Budapest – 3.\,7.\,1904 Edlach), \emph{Schriftsteller, Journalist}!Prinzen aus Genieland. Lustspiel in 4 Akten@\strich\emph{Prinzen aus Genieland. Lustspiel in 4 Akten}|pw}{ }\label{K_L03830-13v}\edtext{im Carltheater\orgindex{Carl-Theater@Carl-Theater|pw}}{\lemma{\textnormal{\emph{im Carltheater}}}\Cendnote{\textnormal{ Die Theateruraufführung von \emph{Prinzen aus Genieland.
                        Lustspiel in 4 Akten}\pwindex{Herzl, Theodor 2.\,5.\,1860 Budapest – 3.\,7.\,1904 Edlach@\textsc{Herzl, Theodor} (2.\,5.\,1860 Budapest – 3.\,7.\,1904 Edlach), \emph{Schriftsteller, Journalist}!Prinzen aus Genieland. Lustspiel in 4 Akten@\strich\emph{Prinzen aus Genieland. Lustspiel in 4 Akten}|pwk}\eventindex{Carl-Theater@\textbf{Carl-Theater}!Uraufführung von Prinzen in Genieland, 20.11.1891@Uraufführung von Prinzen in Genieland, 20.11.1891|pwk} fand am 20. 11. 1891 am \emph{Carl-Theater}\orgindex{Carl-Theater@Carl-Theater|pwk} statt.}}}\label{K_L03830-13} von den \introOben{}Possen\introOben{}Darstellern wie ich glaube nicht umgebrachte Künstlerlustspiel\pwindex{Herzl, Theodor 2.\,5.\,1860 Budapest – 3.\,7.\,1904 Edlach@\textsc{Herzl, Theodor} (2.\,5.\,1860 Budapest – 3.\,7.\,1904 Edlach), \emph{Schriftsteller, Journalist}!Prinzen aus Genieland. Lustspiel in 4 Akten@\strich\emph{Prinzen aus Genieland. Lustspiel in 4 Akten}|pwv}.\pend
           
\pstart
           Barnay\pwindex{Barnay, Ludwig 11.\,2.\,1842 Budapest – 1.\,2.\,1924@\textsc{Barnay, Ludwig} (11.\,2.\,1842 Budapest – 1.\,2.\,1924), \emph{Schriftsteller, Schauspieler, Theaterdirektor}|pw} gibt jetzt sein Theater\orgindex{Berliner Theater@Berliner Theater|pwv} auf. Er \label{K_L03830-14v}\edtext{ordnet offenbar sein Haus}{\lemma{\textnormal{\emph{ordnet … Haus}}}\Cendnote{\textnormal{Ludwig Barnay\pwindex{Barnay, Ludwig 11.\,2.\,1842 Budapest – 1.\,2.\,1924@\textsc{Barnay, Ludwig} (11.\,2.\,1842 Budapest – 1.\,2.\,1924), \emph{Schriftsteller, Schauspieler, Theaterdirektor}|pwk} verließ Berlin\oindex{Berlin@\textbf{Berlin}, \emph{Hauptstadt}|pwk}{ }1894 und gab die Leitung des \emph{Berliner
                     Theaters}\orgindex{Berliner Theater@Berliner Theater|pwk} ab.}}}\label{K_L03830-14} bevor er wieder auf Reisen geht. Vielleicht findet er,
               dass man den Journalisten {\pb}nicht
               unversöhnt herumgehen lassen darf – u. gibt \label{K_L03830-15v}\edtext{als letzte Novität}{\lemma{\textnormal{\emph{als letzte Novität}}}\Cendnote{\textnormal{So im Wortlaut im \emph{Berliner
                     Theaterbericht}\pwindex{Berthold, Ludwig @\textsc{Berthold, Ludwig}, \emph{Journalist, Theaterkritiker}!Berliner Theaterbericht@\strich\emph{Berliner Theaterbericht}|pwk} der Zeitschrift \emph{Die
                     Gesellschaft}\pwindex{Gesellschaft. Politisches illustriertes Wochenblatt@\emph{Die Gesellschaft. Politisches illustriertes Wochenblatt}|pwk} über die Premiere\eventindex{Berliner Theater@\textbf{Berliner Theater}!Aufführung von Der Flüchtling und Die Eine weint, die Andere lacht, 31.5.1893@Aufführung von Der Flüchtling und Die Eine weint, die Andere lacht, 31.5.1893|pwkv}: »Das \emph{Berliner Theater}\orgindex{Berliner Theater@Berliner Theater|pwk}
                  brachte am 31. v. M. als letzte Novität in dieser Saison ein
                  einactiges Lustspiel ›\emph{Der Flüchtling}\pwindex{Herzl, Theodor 2.\,5.\,1860 Budapest – 3.\,7.\,1904 Edlach@\textsc{Herzl, Theodor} (2.\,5.\,1860 Budapest – 3.\,7.\,1904 Edlach), \emph{Schriftsteller, Journalist}!Flüchtling. Lustspiel in einem Aufzug@\strich\emph{Der Flüchtling. Lustspiel in einem Aufzug}|pwk}‹ von Theodor Herzl\pwindex{Herzl, Theodor 2.\,5.\,1860 Budapest – 3.\,7.\,1904 Edlach@\textsc{Herzl, Theodor} (2.\,5.\,1860 Budapest – 3.\,7.\,1904 Edlach), \emph{Schriftsteller, Journalist}|pwk}.« (L. B.\pwindex{Berthold, Ludwig @\textsc{Berthold, Ludwig}, \emph{Journalist, Theaterkritiker}|pwkv}: \emph{Berliner Theaterbericht}\pwindex{Berthold, Ludwig @\textsc{Berthold, Ludwig}, \emph{Journalist, Theaterkritiker}!Berliner Theaterbericht@\strich\emph{Berliner Theaterbericht}|pwk}. In: \emph{Die Gesellschaft. Illustrirtes
                     Wochenblatt}\pwindex{Gesellschaft. Politisches illustriertes Wochenblatt@\emph{Die Gesellschaft. Politisches illustriertes Wochenblatt}|pwk}, Jg. 7, Nr. 23, 11. 6. 1893,
                  S. 10).}}}\label{K_L03830-15} seiner Direction mein Stückchen\pwindex{Herzl, Theodor 2.\,5.\,1860 Budapest – 3.\,7.\,1904 Edlach@\textsc{Herzl, Theodor} (2.\,5.\,1860 Budapest – 3.\,7.\,1904 Edlach), \emph{Schriftsteller, Journalist}!Flüchtling. Lustspiel in einem Aufzug@\strich\emph{Der Flüchtling. Lustspiel in einem Aufzug}|pwv}, ohne mich zu fragen.\pend
           
\pstart
           Verstehen Sie das Leben, Freund! Ich habe Barnay\pwindex{Barnay, Ludwig 11.\,2.\,1842 Budapest – 1.\,2.\,1924@\textsc{Barnay, Ludwig} (11.\,2.\,1842 Budapest – 1.\,2.\,1924), \emph{Schriftsteller, Schauspieler, Theaterdirektor}|pw} für die Aufmerksamkeit nicht gedankt. Er ist davon wahrscheinlich
               sehr überrascht. Wie überrascht wäre er aber, wenn er wüsste dass ich Alles verziehen
                  \strikeout{was} obschon nicht vergessen hatte. Und dass er er
               gerade durch den Fehler, den er begangen, vor dem Stahl meiner Feder immer sicher
               war. Diese Leute wissen nicht, dass wir Anderen die Zeitung nie für unsere
               Privatangelegenheiten verwenden.\pend
           
\pstart
           Ja, ich könnte Ihnen viel erzählen, auch von der Lustspielconcurrenz und anderen
               Gemeinheiten des Deutschen Volkstheaters\orgindex{Volkstheater@Volkstheater|pw} in Wien\oindex{Wien@\textbf{Wien}, \emph{Verwaltungsgebiet}|pw}. \strikeout{Ich \textcolor{gray}{×}} Es hat lange gedauert, bis die Miserablen des {\pb}Theaters mich gebrochen haben. Sie hätten
               es nie zuwege gebracht, wenn ich mich nicht um sie gekümmert hätte, sondern
               geschrieben wie ich wollte, wie mirs zu Muthe und im Sinne war. Und ich sage Ihnen
               das, damit Sie aus meinem Falle lernen. Pfeifen Sie auf das Gesindel. Schreiben Sie
                  \uline{nur}, wie es Ihnen gefällt. Bei Ihrem Talent ist
               es, dann eine innere Nothwendigkeit, dass Sie auch eines nicht fernen Tages den
               äusseren Erfolg sehen. Aber werden Sie viel glücklicher sein, wenn man Sie \strikeout{solchen} vor die grosse \label{K_L03830-16v}\edtext{\begin{otherlanguage}{french}Courtine\end{otherlanguage}}{\lemma{\textnormal{\emph{Courtine}}}\Cendnote{\textnormal{französisch: Vorhang}}}\label{K_L03830-16}{ }des Burg\oindex{Wien@\textbf{Wien}!I., Innere Stadt@\textbf{I., Innere Stadt}!Burgtheater@\textbf{Burgtheater}, \emph{Theater}|pw} oder Lessingtheaters\oindex{Lessing-Theater@\textbf{Lessing-Theater}, \emph{Theater}|pw} treten lässt? Das ist ein Glück welches täglich so u. so
               viele Mätzchenmacher haben.\pend
           
\pstart
           Auf Wiedersehen in Wien\oindex{Wien@\textbf{Wien}, \emph{Verwaltungsgebiet}|pw}{\\[\baselineskip]} Ihr Freund{\\[\baselineskip]}\spacefill\mbox{Th Herzl}\pend
           \leftskip=0em{}
\pstart
           \noindent{}Ich brauche Ihnen nicht zu sagen dass alles, was in diesem Briefe steht nur \uline{für Sie allein} geschrieben ist.\pend
           \selectlanguage{ngerman}\endnumbering\briefempfaengerindex{Schnitzler, Arthur@\textsc{Schnitzler, Arthur}!zzzHerzl, Theodor@\emph{von Theodor Herzl}!1893-06-151@{15. 6. 1893}|)be}\mylabel{L03830h}
\begin{anhang}
\end{anhang}\newcommand{\dateiname}{L03830}\newcommand{\titel}{Theodor Herzl an Arthur Schnitzler, 15. 6. 1893}\newcommand{\editorInnen}{Selma Jahnke und Martin Anton Müller}%% latex-leseansicht-abspann.tex
%% Abspann für die Leseansicht.
%% Der Schalter \ifkorrekturansicht ist bereits durch den Vorspann gesetzt.

%% latex-abspann.tex
%% Gemeinsamer Abspann für Korrekturansicht und Leseansicht.
%% Setzt den Schalter \ifkorrekturansicht voraus (gesetzt in den
%% einbindenden Dateien latex-korrekturansicht-abspann.tex bzw.
%% latex-leseansicht-abspann.tex).
%% ---------------------------------------------------------------

\normalsize

% Das esempio-Environment wird nur in der Leseansicht benötigt
\ifkorrekturansicht\else
\newenvironment{esempio}[3]%
{
    \vspace{1.5ex}
    \rlap{\underline{#1}}
    \par
    \setlength{\parindent}{0cm}
    \nopagebreak
    \leftskip=#2cm
    \rightskip=#3cm
}
{
    \par
}
\fi

\doendnotes{C}
\bigskip
\vfill

\clearpage

\footnotesize

\ifkorrekturansicht
  \lohead{\textsc{register}}
\fi

% theindex-Environment neu definieren ohne reledmac
\makeatletter
\renewenvironment{theindex}{%
  \ifkorrekturansicht
    \section*{\indexname}%
  \else
    \subsubsection*{Index der erwähnten Entitäten}%
  \fi
  \setlength{\parindent}{0pt}%
  \setlength{\parskip}{0pt plus 0.3pt}%
  \let\item\@idxitem
}{%
  \ifkorrekturansicht\clearpage\fi
}
\makeatother

\IfFileExists{\jobname-pw.ind}{\input{\jobname-pw.ind}}{}

% Quellenangabe nur in der Leseansicht
\ifkorrekturansicht\else
% Fallback-Definitionen, falls die .tex-Datei \titel etc. nicht gesetzt hat
\providecommand{\titel}{}
\providecommand{\editorInnen}{}
\providecommand{\dateiname}{\jobname}

\vspace{3cm}

\vfill

\footnotesize
\textsc{Quelle}: \titel. Herausgegeben von {\editorInnen}. In: \emph{Arthur Schnitzler: Briefwechsel mit Autorinnen und Autoren}.
 Digitale Edition, https://schnitzler-briefe.acdh.oeaw.ac.at/{\dateiname}.html (Stand \today)
\fi

\end{document}


