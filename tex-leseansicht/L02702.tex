%% latex-leseansicht-vorspann.tex
%% Vorspann für die Leseansicht.
%% Lädt die gemeinsame Datei latex-vorspann.tex mit nicht gesetztem Schalter.

\newif\ifkorrekturansicht
\korrekturansichtfalse

\input{../tex-inputs/latex-vorspann}

\begin{center}
            \textcolor{red}{ENTWURF, NICHT FERTIG KORRIGIERT}
                      \end{center}
            
               \section[Paul Goldmann an Arthur Schnitzler, 9. 10. {[}1892{]}]{ Paul Goldmann an Arthur Schnitzler, 9. 10. {[}1892{]}}\nopagebreak\mylabel{v}\rehead{ }\begin{ledgroupsized}[t]{13cm}\normalsize\beginnumbering\briefempfaengerindex{Schnitzler, Arthur@\textsc{Schnitzler, Arthur}!zzzGoldmann, Paul@\emph{von Paul Goldmann}!1892-10-091@{9. 10. {[}1892{]}}|(be} \toendnotes[C]{\smallbreak\pagebreak[2]} \Standort{DLA, A:Schnitzler, HS.NZ85.1.3163.}
\physDesc{Brief, 1 Blatt, 3 Seiten
\newline{}Handschrift: schwarze Tinte, deutsche Kurrent
\newline{}Schnitzler: mit Bleistift das Jahr »92« vermerkt }\toendnotes[C]{\smallbreak}\pstart
           \noindent{}{\pb}\textcolor{gray}{\textbf{Frankfurter Zeitung\orgindex{Frankfurter Zeitung@Frankfurter Zeitung|pw}.}}\pend
           \pstart
           \textcolor{gray}{\textbf{(Gazette de
                     Francfort\orgindex{Frankfurter Zeitung@Frankfurter Zeitung|pw}.)}}\hfill Paris\oindex{Paris@\textbf{Paris}|pw}, 9. October.\pend
           \pstart
           \textcolor{gray}{\textbf{\begin{otherlanguage}{french}Directeur\end{otherlanguage}: \textbf{M. L. Sonnemann\pwindex{Sonnemann, Leopold 1831-10-29 – 1909-10-30@\textsc{Sonnemann, Leopold} (1831-10-29 – 1909-10-30), \emph{Journalist, Herausgeber}|pw}}.}}\pend
           \pstart
           \textcolor{gray}{\textbf{\begin{otherlanguage}{french}Journal politique, financier,\end{otherlanguage}}}\pend
           \pstart
           \textcolor{gray}{\textbf{\begin{otherlanguage}{french}commercial et littéraire.\end{otherlanguage}}}\pend
           \pstart
           \textcolor{gray}{\textbf{\begin{otherlanguage}{french}\textbf{Paraissant trois fois par jour}\end{otherlanguage}}}.\pend
           \pstart
           \textcolor{gray}{\textbf{–}}\pend
           \pstart
           \textcolor{gray}{\textbf{\begin{otherlanguage}{french}\textbf{Bureaux à Paris\oindex{Paris@\textbf{Paris}|pw}:}\end{otherlanguage}}}\pend
           \pstart
           \textcolor{gray}{\textbf{\begin{otherlanguage}{french}\textbf{rue Richelieu 75.\oindex{rue Richelieu@\textbf{rue Richelieu}|pw}.}\end{otherlanguage}}}\pend
           \pstart
           \centering{}Mein lieber Arthur!\pend
           \pstart
           \noindent{}Ich brauche Dir nicht erſt zu ſchreiben, daß du in Allem auf mich zählen kannſt. Den
               Brief hebe ich auf. Aber bitte, ſchreibe mir bald. Ich ſehne mich ſchon ſehr nach
               einem Worte von Dir. Genauer Bericht, bitte! Mein Onkel\pwindex{Mamroth, Fedor 21.02.1851 – 25.06.1907@\textsc{Mamroth, Fedor} (21.02.1851 – 25.06.1907), \emph{Journalist, Kritiker}|pwv} kann Dir keine Empfehlung an den Frankfurt\oindex{Frankfurt am Main@\textbf{Frankfurt am Main}|pw}er Director\pwindex{Sonnemann, Leopold 1831-10-29 – 1909-10-30@\textsc{Sonnemann, Leopold} (1831-10-29 – 1909-10-30), \emph{Journalist, Herausgeber}|pwv} geben, weil er ſchlechter mit ihm ſteht als je. Infolge ſeiner
               letzten ſcharfen \label{K_L02702-1v}\edtext{Kritiken}{\lemma{\textnormal{\emph{Kritiken}}}\Cendnote{\textnormal{XXXX}}}\label{K_L02702-1h} iſt es ſogar zu bedrohlichen
               Auftritten zwiſchen meinem Onkel\pwindex{Mamroth, Fedor 21.02.1851 – 25.06.1907@\textsc{Mamroth, Fedor} (21.02.1851 – 25.06.1907), \emph{Journalist, Kritiker}|pwv} u. Herrn \textsc{Sonnemann\pwindex{Sonnemann, Leopold 1831-10-29 – 1909-10-30@\textsc{Sonnemann, Leopold} (1831-10-29 – 1909-10-30), \emph{Journalist, Herausgeber}|pwv}} gekommen. {\pb}Ob ich hier werde etwas thun
               können, weiß ich nicht. Jedenfalls arbeite ich daran. Läge Dir aber etwas daran, in
                  \label{K_L02702-2v}\edtext{\textsc{Breslau\oindex{Breslau@\textbf{Breslau}|pw}}}{\lemma{\textnormal{\emph{Breslau}}}\Cendnote{\textnormal{Aus 1892 sind keine
                  Bemühungen um Aufführungen in Breslau\oindex{Breslau@\textbf{Breslau}|pwk} bekannt,
                  sehr wohl jedoch aus 1890 und 1891, als Schnitzler mit
                     Theodor Loewe\pwindex{Loewe, Theodor 09.09.1855 – 1935@\textsc{Loewe, Theodor} (09.09.1855 – 1935), \emph{Theaterleiter}|pwk} wegen einer möglichen
                  Aufführung von \emph{Alkandi’s Lied}\pwindex{Schnitzler, Arthur 15.05.1862 – 21.10.1931@\textsc{Schnitzler, Arthur} (15.05.1862 – 21.10.1931), \emph{Schriftsteller, Mediziner}!Alkandi s Lied15.8.1890 – 1.9.1890@\strich\emph{Alkandi’s Lied} {[}15.8.1890 – 1.9.1890{]}|pwk} in Kontakt war.
                     Siehe A. S.: \emph{Tagebuch}, 23. 6. 1891}}}\label{K_L02702-2h} aufgeführt zu werden, ſo könnte ich vielleicht etwas richten. Kommſt Du alſo
               doch zuerſt in \label{K_L02702-3v}\edtext{\textsc{Prag\oindex{Prag@\textbf{Prag}|pw}}}{\lemma{\textnormal{\emph{Prag}}}\Cendnote{\textnormal{Siehe Paul Goldmann an Arthur Schnitzler, 27. 6. [1892]}}}\label{K_L02702-3h} daran? Und wann und bei wem das \label{K_L02702-6v}\edtext{Buch\pwindex{Schnitzler, Arthur 15.05.1862 – 21.10.1931@\textsc{Schnitzler, Arthur} (15.05.1862 – 21.10.1931), \emph{Schriftsteller, Mediziner}!Anatol1892-10-29 – 1892-10-29@\strich\emph{Anatol} {[}1892-10-29 – 1892-10-29{]}|pwv}}{\lemma{\textnormal{\emph{Buch}}}\Cendnote{\textnormal{Arthur Schnitzler: \emph{Anatol}\pwindex{Schnitzler, Arthur 15.05.1862 – 21.10.1931@\textsc{Schnitzler, Arthur} (15.05.1862 – 21.10.1931), \emph{Schriftsteller, Mediziner}!Anatol1892-10-29 – 1892-10-29@\strich\emph{Anatol} {[}1892-10-29 – 1892-10-29{]}|pwk}. Berlin\oindex{Berlin@\textbf{Berlin}|pwk}: \emph{Bibliographisches Bureau}\orgindex{Bibliographisches Bureau@Bibliographisches Bureau|pwk}{ }1892, vordatiert auf 1893.}}}\label{K_L02702-6h}? Ich weiß leider ſo gar nichts mehr. Und \label{K_L02702-4v}\edtext{mit wem warſt Du in Venedig\oindex{Venedig@\textbf{Venedig}|pw}}{\lemma{\textnormal{\emph{mit … Venedig}}}\Cendnote{\textnormal{Schnitzler war von 17. 8. 1892 bis 22. 9. 1892 mit seinem Bruder\pwindex{Schnitzler, Julius 13.07.1865 – 29.06.1939@\textsc{Schnitzler, Julius} (13.07.1865 – 29.06.1939), \emph{Chirurg}|pwkv}{ }Julius\pwindex{Schnitzler, Julius 13.07.1865 – 29.06.1939@\textsc{Schnitzler, Julius} (13.07.1865 – 29.06.1939), \emph{Chirurg}|pwk} in Venedig\oindex{Venedig@\textbf{Venedig}|pwk}. Julius\pwindex{Schnitzler, Julius 13.07.1865 – 29.06.1939@\textsc{Schnitzler, Julius} (13.07.1865 – 29.06.1939), \emph{Chirurg}|pwk} reiste jedoch
                  bereits am 20. 9. 1892
                  ab.}}}\label{K_L02702-4h}? Hätteſt du mir ein Wort geſagt, ſo würde ich meinen Urlaub verſchoben
               haben und mitgekommen ſein.\pend
           \pstart
           Bitte lies: 1.) \label{K_L02702-5v}\edtext{\textsc{Renan\pwindex{Renan, Ernest 27.02.1823 – 02.10.1892@\textsc{Renan, Ernest} (27.02.1823 – 02.10.1892), \emph{Schriftsteller, Orientalist}|pw}}: Leben Jeſu\pwindex{?? Werk@Nicht ermittelte Verfasserinnen und Verfasser!Leben Jesu. Vollstaendige Volks-Ausgabe1864 – 1864@\emph{Das Leben Jesu. Vollständige Volks-Ausgabe} {[}1864 – 1864{]}|pw}}{\lemma{\textnormal{\emph{Renan: Leben Jeſu}}}\Cendnote{\textnormal{Lektüre keiner der genannten Werke\pwindex{?? Werk@Nicht ermittelte Verfasserinnen und Verfasser!Leben Jesu. Vollstaendige Volks-Ausgabe1864 – 1864@\emph{Das Leben Jesu. Vollständige Volks-Ausgabe} {[}1864 – 1864{]}|pwkv}\pwindex{Chamfort, Sebastien Roch Nicolas 06.04.1741 – 13.04.1794@\textsc{Chamfort, Sébastien Roch Nicolas} (06.04.1741 – 13.04.1794), \emph{Schriftsteller}!Maximes et Pensees, Caracteres et Anecdotes1795 – 1795@\strich\emph{Maximes et Pensées, Caractères et Anecdotes} {[}1795 – 1795{]}|pwkv}\pwindex{Prudhomme, Sully 1839-03-16 – 1907-09-07@\textsc{Prudhomme, Sully} (1839-03-16 – 1907-09-07), \emph{Schriftsteller}!caresses1869 – 1869@\strich\emph{Les caresses} {[}1869 – 1869{]}|pwkv}
                  bekannt}}}\label{K_L02702-5h} (Kleine Volks{\pb}ausgabe) 2. \textsc{Chamfort\pwindex{Chamfort, Sebastien Roch Nicolas 06.04.1741 – 13.04.1794@\textsc{Chamfort, Sébastien Roch Nicolas} (06.04.1741 – 13.04.1794), \emph{Schriftsteller}|pw}}: \textsc{Maximes\pwindex{Chamfort, Sebastien Roch Nicolas 06.04.1741 – 13.04.1794@\textsc{Chamfort, Sébastien Roch Nicolas} (06.04.1741 – 13.04.1794), \emph{Schriftsteller}!Maximes et Pensees, Caracteres et Anecdotes1795 – 1795@\strich\emph{Maximes et Pensées, Caractères et Anecdotes} {[}1795 – 1795{]}|pw}} (\textsc{Collection des auteurs célèbres\pwindex{Chamfort, Sebastien Roch Nicolas 06.04.1741 – 13.04.1794@\textsc{Chamfort, Sébastien Roch Nicolas} (06.04.1741 – 13.04.1794), \emph{Schriftsteller}!Œuvres choises de Chamfort, tome 21869 – 1869@\strich\emph{Œuvres choises de Chamfort, tome 2} {[}1869 – 1869{]}|pw}}) 3.) In der Sammlung\pwindex{Prudhomme, Sully 1839-03-16 – 1907-09-07@\textsc{Prudhomme, Sully} (1839-03-16 – 1907-09-07), \emph{Schriftsteller}!Œuvres de Sully Prudhomme, tome 21872 – 1872@\strich\emph{Œuvres de Sully Prudhomme, tome 2} {[}1872 – 1872{]}|pwuv} der Gedichte von \textsc{Sully Prud’homme\pwindex{Prudhomme, Sully 1839-03-16 – 1907-09-07@\textsc{Prudhomme, Sully} (1839-03-16 – 1907-09-07), \emph{Schriftsteller}|pw}} dasjenige, das den Titel trägt »\textsc{Les caresses\pwindex{Prudhomme, Sully 1839-03-16 – 1907-09-07@\textsc{Prudhomme, Sully} (1839-03-16 – 1907-09-07), \emph{Schriftsteller}!caresses1869 – 1869@\strich\emph{Les caresses} {[}1869 – 1869{]}|pw}}«. Beſonders das letztere\pwindex{Prudhomme, Sully 1839-03-16 – 1907-09-07@\textsc{Prudhomme, Sully} (1839-03-16 – 1907-09-07), \emph{Schriftsteller}!caresses1869 – 1869@\strich\emph{Les caresses} {[}1869 – 1869{]}|pwv}
               wird Dir vielleicht ein wenig eine brennende Herzenswunde kühlen.\pend
           \pstart
           Grüß’ Dich Gott, liebſter Freund!\pend
           \pstart
           Ich umarme Dich und \textsc{Richard\pwindex{Beer-Hofmann, Richard 11.07.1866 – 26.09.1945@\textsc{Beer-Hofmann, Richard} (11.07.1866 – 26.09.1945), \emph{Schriftsteller}|pw}}.\pend
           \pstart
           Dein {\\[\baselineskip]}\spacefill\mbox{Paul Goldmann.}\pend
           \leftskip=0em{}\endnumbering\briefempfaengerindex{Schnitzler, Arthur@\textsc{Schnitzler, Arthur}!zzzGoldmann, Paul@\emph{von Paul Goldmann}!1892-10-091@{9. 10. {[}1892{]}}|)be}\mylabel{h}\end{ledgroupsized}\begin{anhang}\end{anhang}\newcommand{\dateiname}{L02702}\newcommand{\titel}{Paul Goldmann an Arthur Schnitzler, 9. 10. [1892]}\newcommand{\editorInnen}{Martin Anton Müller und Laura Untner}
            \footnotesize
\begin{ledgroupsized}[t]{11.5cm}
\doendnotes{C}
\end{ledgroupsized}
         %% latex-leseansicht-abspann.tex
%% Abspann für die Leseansicht.
%% Der Schalter \ifkorrekturansicht ist bereits durch den Vorspann gesetzt.

%% latex-abspann.tex
%% Gemeinsamer Abspann für Korrekturansicht und Leseansicht.
%% Setzt den Schalter \ifkorrekturansicht voraus (gesetzt in den
%% einbindenden Dateien latex-korrekturansicht-abspann.tex bzw.
%% latex-leseansicht-abspann.tex).
%% ---------------------------------------------------------------

\normalsize

% Das esempio-Environment wird nur in der Leseansicht benötigt
\ifkorrekturansicht\else
\newenvironment{esempio}[3]%
{
    \vspace{1.5ex}
    \rlap{\underline{#1}}
    \par
    \setlength{\parindent}{0cm}
    \nopagebreak
    \leftskip=#2cm
    \rightskip=#3cm
}
{
    \par
}
\fi

\doendnotes{C}
\bigskip
\vfill

\clearpage

\footnotesize

\ifkorrekturansicht
  \lohead{\textsc{register}}
\fi

% theindex-Environment neu definieren ohne reledmac
\makeatletter
\renewenvironment{theindex}{%
  \ifkorrekturansicht
    \section*{\indexname}%
  \else
    \subsubsection*{Index der erwähnten Entitäten}%
  \fi
  \setlength{\parindent}{0pt}%
  \setlength{\parskip}{0pt plus 0.3pt}%
  \let\item\@idxitem
}{%
  \ifkorrekturansicht\clearpage\fi
}
\makeatother

\IfFileExists{\jobname-pw.ind}{\input{\jobname-pw.ind}}{}

% Quellenangabe nur in der Leseansicht
\ifkorrekturansicht\else
% Fallback-Definitionen, falls die .tex-Datei \titel etc. nicht gesetzt hat
\providecommand{\titel}{}
\providecommand{\editorInnen}{}
\providecommand{\dateiname}{\jobname}

\vspace{3cm}

\vfill

\footnotesize
\textsc{Quelle}: \titel. Herausgegeben von {\editorInnen}. In: \emph{Arthur Schnitzler: Briefwechsel mit Autorinnen und Autoren}.
 Digitale Edition, https://schnitzler-briefe.acdh.oeaw.ac.at/{\dateiname}.html (Stand \today)
\fi

\end{document}


      