%% latex-korrekturansicht-vorspann.tex
%% Vorspann für die Korrekturansicht.
%% Lädt die gemeinsame Datei latex-vorspann.tex mit gesetztem Schalter.

\newif\ifkorrekturansicht
\korrekturansichttrue

\input{../tex-inputs/latex-vorspann}


\section[Paul Goldmann an Arthur Schnitzler, 9. 10. {[}1892{]}]{L02702 Paul Goldmann an Arthur Schnitzler, 9. 10. {[}1892{]}}
\nopagebreak\mylabel{L02702v}
\rehead{ }\normalsize\beginnumbering\briefempfaengerindex{Schnitzler, Arthur@\textsc{Schnitzler, Arthur}!zzzGoldmann, Paul@\emph{von Paul Goldmann}!1892-10-091@{9. 10. {[}1892{]}}|(be}
\toendnotes[C]{\smallbreak\pagebreak[2]}\Standort{DLA, A:Schnitzler, HS.NZ85.1.3163.}
\physDesc{Brief, 1 Blatt, 3 Seiten, 1264 Zeichen
\newline{}Handschrift: schwarze Tinte, deutsche Kurrent
\newline{}Schnitzler: mit Bleistift das Jahr »92« vermerkt }\toendnotes[C]{\smallbreak}
\pstart
           {\pb}\textcolor{gray}{\textbf{Frankfurter Zeitung\orgindex{Frankfurter Zeitung@Frankfurter Zeitung|pw}.}}\pend
           
\pstart
           \textcolor{gray}{\textbf{(Gazette de
                        Francfort\orgindex{Frankfurter Zeitung@Frankfurter Zeitung|pw}.)}}\hfill \textsc{Paris\oindex{Paris@\textbf{Paris}, \emph{P.PPLC}|pw}}, 9. October.\pend
           
\pstart
           \textcolor{gray}{\textbf{\begin{otherlanguage}{french}Directeur\end{otherlanguage}: \textbf{M. L. Sonnemann\pwindex{Sonnemann, Leopold 1831-10-29 – 1909-10-30@\textsc{Sonnemann, Leopold} (1831-10-29 – 1909-10-30), \emph{Journalist/Journalistin, Herausgeber/Herausgeberin}|pw}}.}}\pend
           
\pstart
           \textcolor{gray}{\textbf{\begin{otherlanguage}{french}Journal politique, financier,\end{otherlanguage}}}\pend
           
\pstart
           \textcolor{gray}{\textbf{\begin{otherlanguage}{french}commercial et litteraire.\end{otherlanguage}}}\pend
           
\pstart
           \textcolor{gray}{\textbf{\begin{otherlanguage}{french}\textbf{Paraissant trois fois par jour}\end{otherlanguage}}}\pend
           
\pstart
           \textcolor{gray}{\textbf{\begin{otherlanguage}{french}\textbf{Bureaux à Paris\oindex{Paris@\textbf{Paris}, \emph{P.PPLC}|pw}:}\end{otherlanguage}}}\pend
           
\pstart
           \textcolor{gray}{\textbf{\begin{otherlanguage}{french}\textbf{rue Richelieu 75\oindex{rue Richelieu@\textbf{rue Richelieu}, \emph{Straße (K.STR)}|pw}.}\end{otherlanguage}}}\pend
           
\pstart\center{}Mein lieber Freund!\pend\vspace{0.5em}
\pstart
           Ich brauche Dir nicht erſt zu ſchreiben, daß Du in Allem auf mich zählen kannſt. Den
               Brief hebe ich auf. Aber bitte, ſchreibe mir bald. Ich ſehne mich ſchon ſehr nach
               einem Worte von Dir. Genauer Bericht, bitte! Mein Onkel\pwindex{Mamroth, Fedor 21.02.1851 – 25.06.1907@\textsc{Mamroth, Fedor} (21.02.1851 – 25.06.1907), \emph{Journalist/Journalistin, Kritiker/Kritikerin}|pwv} kann Dir keine Empfehlung an den Frankfurt\oindex{Frankfurt am Main@\textbf{Frankfurt am Main}, \emph{P.PPLA3}|pw}er Director\pwindex{Sonnemann, Leopold 1831-10-29 – 1909-10-30@\textsc{Sonnemann, Leopold} (1831-10-29 – 1909-10-30), \emph{Journalist/Journalistin, Herausgeber/Herausgeberin}|pwv} geben, weil er ſchlechter mit ihm ſteht als je. Infolge ſeiner
               letzten ſcharfen Kritiken iſt es ſogar zu bedrohlichen Auftritten zwiſchen meinem Onkel\pwindex{Mamroth, Fedor 21.02.1851 – 25.06.1907@\textsc{Mamroth, Fedor} (21.02.1851 – 25.06.1907), \emph{Journalist/Journalistin, Kritiker/Kritikerin}|pwv} u. Herrn \textsc{Sonnemann\pwindex{Sonnemann, Leopold 1831-10-29 – 1909-10-30@\textsc{Sonnemann, Leopold} (1831-10-29 – 1909-10-30), \emph{Journalist/Journalistin, Herausgeber/Herausgeberin}|pwv}} gekommen. {\pb}Ob ich hier werde etwas thun
               können, weiß ich nicht. Jedenfalls arbeite ich daran. Läge Dir aber etwas daran, in
                  \label{K_L02702-1v}\edtext{\textsc{Breslau\oindex{Breslau@\textbf{Breslau}, \emph{P.PPLA}|pw}}}{\lemma{\textnormal{\emph{Breslau}}}\Cendnote{\textnormal{Aus dem Jahr 1892 sind
                  keine Bemühungen um Aufführungen in Breslau\oindex{Breslau@\textbf{Breslau}, \emph{P.PPLA}|pwk}
                  bekannt. Solche gab es 1890 und 1891, als Schnitzler mit Theodor Loewe\pwindex{Loewe, Theodor 1855-01-01 – 1935@\textsc{Loewe, Theodor} (1855-01-01 – 1935), \emph{Theaterleiter/Theaterleiterin}|pwk} wegen einer möglichen Aufführung von \emph{Alkandi’s Lied}\pwindex{Alkandi s Lied@\emph{Alkandi’s Lied}|pwk} in Kontakt stand. Siehe A. S.: \emph{Tagebuch}, 23. 6. 1891. }}}\label{K_L02702-1}
               aufgeführt zu werden, ſo könnte ich vielleicht etwas richten. Kommſt Du alſo doch
               zuerſt in \label{K_L02702-2v}\edtext{\textsc{Prag\oindex{Prag@\textbf{Prag}, \emph{A.ADM1}|pw}}}{\lemma{\textnormal{\emph{Prag}}}\Cendnote{\textnormal{Siehe Paul Goldmann an Arthur Schnitzler, 27. 6. [1892].
               }}}\label{K_L02702-2} daran? Und wann und bei wem das \label{K_L02702-3v}\edtext{Buch\pwindex{Anatol@\emph{Anatol}|pwv}}{\lemma{\textnormal{\emph{Buch}}}\Cendnote{\textnormal{Arthur Schnitzler: \emph{Anatol}\pwindex{Anatol@\emph{Anatol}|pwk}. Berlin\oindex{Berlin@\textbf{Berlin}, \emph{P.PPLC}|pwk}: \emph{Bibliographisches Bureau}\orgindex{Bibliographisches Bureau@Bibliographisches Bureau|pwk}{ }1892, vordatiert auf 1893.
               }}}\label{K_L02702-3}? Ich weiß leider ſo gar nichts mehr. Und \label{K_L02702-4v}\edtext{mit wem warſt Du in Venedig\oindex{Venedig@\textbf{Venedig}, \emph{P.PPLA}|pw}}{\lemma{\textnormal{\emph{mit … Venedig}}}\Cendnote{\textnormal{Schnitzler war vom 17. 8. 1892 bis zum 22. 9. 1892 mit seinem
                  Bruder Julius\pwindex{Schnitzler, Julius 13.07.1865 – 29.06.1939@\textsc{Schnitzler, Julius} (13.07.1865 – 29.06.1939), \emph{Chirurg/Chirurgin}|pwk} in Venedig\oindex{Venedig@\textbf{Venedig}, \emph{P.PPLA}|pwk}. Dieser reiste bereits am 20. 9. 1892 ab.}}}\label{K_L02702-4}? Hätteſt Du mir ein
               Wort geſagt, ſo würde ich meinen Urlaub verſchoben haben und mitgekommen ſein.\pend
           
\pstart
           Bitte lies: 1.) \label{K_L02702-5v}\edtext{\textsc{Renan\pwindex{Renan, Ernest 27.02.1823 – 02.10.1892@\textsc{Renan, Ernest} (27.02.1823 – 02.10.1892), \emph{Schriftsteller/Schriftstellerin, Orientalist/Orientalistin}|pw}}: Leben Jeſu\pwindex{Leben Jesu. Vollstaendige Volks-Ausgabe@\emph{Das Leben Jesu. Vollständige Volks-Ausgabe}|pw}}{\lemma{\textnormal{\emph{Renan: Leben Jeſu}}}\Cendnote{\textnormal{Eine Lektüre der genannten Werke\pwindex{Leben Jesu. Vollstaendige Volks-Ausgabe@\emph{Das Leben Jesu. Vollständige Volks-Ausgabe}|pwkv}\pwindex{Maximes et Pensees, Caracteres et Anecdotes@\emph{Maximes et Pensées, Caractères et Anecdotes}|pwkv}\pwindex{caresses@\emph{Les caresses}|pwkv}
                  durch Schnitzler lässt sich nachweisen, doch
                  findet sich Renan\pwindex{Renan, Ernest 27.02.1823 – 02.10.1892@\textsc{Renan, Ernest} (27.02.1823 – 02.10.1892), \emph{Schriftsteller/Schriftstellerin, Orientalist/Orientalistin}|pwk} in Schnitzlers{ }\emph{Leseliste}\pwindex{Notizen zu Lektuere und Theaterbesuchen (1879-1927)@\emph{Notizen zu Lektüre und Theaterbesuchen (1879-1927)}|pwk}.}}}\label{K_L02702-5} (Kleine Volks{\pb}ausgabe) 2.
                  \textsc{Chamfort\pwindex{Chamfort, Sebastien Roch Nicolas 06.04.1741 – 13.04.1794@\textsc{Chamfort, Sébastien Roch Nicolas} (06.04.1741 – 13.04.1794), \emph{Schriftsteller/Schriftstellerin}|pw}}: \textsc{Maximes\pwindex{Maximes et Pensees, Caracteres et Anecdotes@\emph{Maximes et Pensées, Caractères et Anecdotes}|pw}} (\textsc{Collection des auteurs célèbres\pwindex{Œuvres choises de Chamfort, tome 2@\emph{Œuvres choises de Chamfort, tome 2}|pw}}) 3.) In der \label{K_L02702-6v}\edtext{Sammlung\pwindex{Œuvres de Sully Prudhomme, tome 2@\emph{Œuvres de Sully Prudhomme, tome 2}|pwuv}}{\lemma{\textnormal{\emph{Sammlung}}}\Cendnote{\textnormal{Vermutlich bezog er sich auf diese
                  Ausgabe: Sully Prudhomme\pwindex{Prudhomme, Sully 1839-03-16 – 1907-09-07@\textsc{Prudhomme, Sully} (1839-03-16 – 1907-09-07), \emph{Schriftsteller/Schriftstellerin, Librettist/Librettistin, Nobelpreisträger/Nobelpreisträgerin}|pwk}: \emph{Les Solitudes. Poésies}\pwindex{Œuvres de Sully Prudhomme, tome 2@\emph{Œuvres de Sully Prudhomme, tome 2}|pwk}. Paris:
                        \emph{Alphonse Lemerre, Éditeur}{ }1869. \emph{Les caresses}\pwindex{caresses@\emph{Les caresses}|pwk} findet sich auf den
                  Seiten 117–119.}}}\label{K_L02702-6} der Gedichte von \textsc{Sully Prud’homme\pwindex{Prudhomme, Sully 1839-03-16 – 1907-09-07@\textsc{Prudhomme, Sully} (1839-03-16 – 1907-09-07), \emph{Schriftsteller/Schriftstellerin, Librettist/Librettistin, Nobelpreisträger/Nobelpreisträgerin}|pw}} dasjenige, das den Titel trägt »\textsc{Les caresses\pwindex{caresses@\emph{Les caresses}|pw}}«. Beſonders das letztere\pwindex{caresses@\emph{Les caresses}|pwv}
               wird Dir vielleicht ein wenig eine brennende Herzenswunde kühlen.\pend
           
\pstart
           Grüß’ Dich Gott, liebſter Freund!\pend
           
\pstart
           Ich umarme Dich und \textsc{Richard\pwindex{Beer-Hofmann, Richard 1866-07-11 – 1945-09-26@\textsc{Beer-Hofmann, Richard} (1866-07-11 – 1945-09-26), \emph{Schriftsteller/Schriftstellerin}|pw}}.\pend
           
\pstart
           Dein {\\[\baselineskip]}\spacefill\mbox{Paul Goldmann.}\pend
           \leftskip=0em{}\selectlanguage{ngerman}\endnumbering\briefempfaengerindex{Schnitzler, Arthur@\textsc{Schnitzler, Arthur}!zzzGoldmann, Paul@\emph{von Paul Goldmann}!1892-10-091@{9. 10. {[}1892{]}}|)be}\mylabel{L02702h}  \normalsize

\doendnotes{C}
\bigskip
\vfill

\clearpage

\footnotesize

\lohead{\textsc{register}}

% Definiere theindex-Environment komplett neu ohne reledmac
\makeatletter
\renewenvironment{theindex}{%
  \section*{\indexname}%
  \setlength{\parindent}{0pt}%
  \setlength{\parskip}{0pt plus 0.3pt}%
  \let\item\@idxitem
}{%
  \clearpage
}
\makeatother

\IfFileExists{\jobname-pw.ind}{\input{\jobname-pw.ind}}{}

\end{document}

      