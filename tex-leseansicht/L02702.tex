%% latex-leseansicht-vorspann.tex
%% Vorspann für die Leseansicht.
%% Lädt die gemeinsame Datei latex-vorspann.tex mit nicht gesetztem Schalter.

\newif\ifkorrekturansicht
\korrekturansichtfalse

\input{../tex-inputs/latex-vorspann}


\section[Paul Goldmann an Arthur Schnitzler, 9. 10. [1892]]{L02702 Paul Goldmann an Arthur Schnitzler, 9. 10. [1892]}
\nopagebreak\mylabel{L02702v}
\rehead{ }\normalsize\beginnumbering\briefempfaengerindex{Schnitzler, Arthur@\textsc{Schnitzler, Arthur}!zzzGoldmann, Paul@\emph{von Paul Goldmann}!1892-10-091@{9. 10. [1892]}|(be}
\toendnotes[C]{\smallbreak\pagebreak[2]}
\correspDesc{Versand  durch Paul Goldmann am 9. 10. [1892] in Paris
\newline{}Erhalt  durch Arthur Schnitzler im Zeitraum [10. 10. 1892 – 14. 10. 1892?] in Wien}\toendnotes[C]{\smallbreak}
\Standort{DLA, A:Schnitzler, HS.NZ85.1.3163.}
\physDesc{Brief, 1 Blatt, 3 Seiten, 1264 Zeichen
\newline{}Handschrift: schwarze Tinte, deutsche Kurrent
\newline{}Schnitzler: mit Bleistift das Jahr »92« vermerkt }\toendnotes[C]{\smallbreak}
\pstart
           {\pb}\textcolor{gray}{\textbf{Frankfurter Zeitung\orgindex{Frankfurter Zeitung@Frankfurter Zeitung|pw}.}}\pend
           
\pstart
           \textcolor{gray}{\textbf{(Gazette de
                        Francfort\orgindex{Frankfurter Zeitung@Frankfurter Zeitung|pw}.)}}\hfill \textsc{Paris\oindex{Paris@\textbf{Paris}, \emph{Hauptstadt}|pw}}, 9. October.\pend
           
\pstart
           \textcolor{gray}{\textbf{\begin{otherlanguage}{french}Directeur\end{otherlanguage}: \textbf{M. L. Sonnemann\pwindex{Sonnemann, Leopold 29.\,10.\,1831 Höchberg – 30.\,10.\,1909 Frankfurt am Main@\textsc{Sonnemann, Leopold} (29.\,10.\,1831 Höchberg – 30.\,10.\,1909 Frankfurt am Main), \emph{Journalist, Herausgeber}|pw}}.}}\pend
           
\pstart
           \textcolor{gray}{\textbf{\begin{otherlanguage}{french}Journal politique, financier,\end{otherlanguage}}}\pend
           
\pstart
           \textcolor{gray}{\textbf{\begin{otherlanguage}{french}commercial et litteraire.\end{otherlanguage}}}\pend
           
\pstart
           \textcolor{gray}{\textbf{\begin{otherlanguage}{french}\textbf{Paraissant trois fois par jour}\end{otherlanguage}}}\pend
           
\pstart
           \textcolor{gray}{\textbf{\begin{otherlanguage}{french}\textbf{Bureaux à Paris\oindex{Paris@\textbf{Paris}, \emph{Hauptstadt}|pw}:}\end{otherlanguage}}}\pend
           
\pstart
           \textcolor{gray}{\textbf{\begin{otherlanguage}{french}\textbf{rue Richelieu 75\oindex{rue Richelieu@\textbf{rue Richelieu}, \emph{Straße}|pw}.}\end{otherlanguage}}}\pend
           
\pstart\center{}Mein lieber Freund!\pend\vspace{0.5em}
\pstart
           Ich brauche Dir nicht erſt zu{ }ſchreiben, daß Du in Allem auf mich zählen kannſt. Den
               Brief hebe ich auf. Aber bitte,{ }ſchreibe mir bald. Ich{ }ſehne mich{ }ſchon{ }ſehr nach
               einem Worte von Dir. Genauer Bericht, bitte! Mein Onkel\pwindex{Mamroth, Fedor 21.\,2.\,1851 Breslau – 25.\,6.\,1907 Frankfurt am Main@\textsc{Mamroth, Fedor} (21.\,2.\,1851 Breslau – 25.\,6.\,1907 Frankfurt am Main), \emph{Journalist, Kritiker}|pwv} kann Dir keine Empfehlung an den Frankfurt\oindex{Frankfurt am Main@\textbf{Frankfurt am Main}, \emph{Hauptstadt}|pw}er Director\pwindex{Sonnemann, Leopold 29.\,10.\,1831 Höchberg – 30.\,10.\,1909 Frankfurt am Main@\textsc{Sonnemann, Leopold} (29.\,10.\,1831 Höchberg – 30.\,10.\,1909 Frankfurt am Main), \emph{Journalist, Herausgeber}|pwv} geben, weil er{ }ſchlechter mit ihm{ }ſteht als je. Infolge{ }ſeiner
               letzten{ }ſcharfen Kritiken iſt es{ }ſogar zu bedrohlichen Auftritten zwiſchen meinem Onkel\pwindex{Mamroth, Fedor 21.\,2.\,1851 Breslau – 25.\,6.\,1907 Frankfurt am Main@\textsc{Mamroth, Fedor} (21.\,2.\,1851 Breslau – 25.\,6.\,1907 Frankfurt am Main), \emph{Journalist, Kritiker}|pwv} u. Herrn \textsc{Sonnemann\pwindex{Sonnemann, Leopold 29.\,10.\,1831 Höchberg – 30.\,10.\,1909 Frankfurt am Main@\textsc{Sonnemann, Leopold} (29.\,10.\,1831 Höchberg – 30.\,10.\,1909 Frankfurt am Main), \emph{Journalist, Herausgeber}|pwv}} gekommen. {\pb}Ob ich hier werde etwas thun
               können, weiß ich nicht. Jedenfalls arbeite ich daran. Läge Dir aber etwas daran, in
                  \label{K_L02702-1v}\edtext{\textsc{Breslau\oindex{Breslau@\textbf{Breslau}|pw}}}{\lemma{\textnormal{\emph{Breslau}}}\Cendnote{\textnormal{Aus dem Jahr 1892 sind
                  keine Bemühungen um Aufführungen in Breslau\oindex{Breslau@\textbf{Breslau}|pwk}
                  bekannt. Solche gab es 1890 und 1891, als Schnitzler mit Theodor Loewe\pwindex{Loewe, Theodor 1.\,1.\,1855 Wien – 1935 Breslau@\textsc{Loewe, Theodor} (1.\,1.\,1855 Wien – 1935 Breslau), \emph{Theaterleiter}|pwk} wegen einer möglichen Aufführung von \emph{Alkandi’s Lied}\pwindex{Schnitzler, Arthur 15. 5. 1862 Wien – 21. 10. 1931 ebd.@\textsc{Schnitzler, Arthur} (15. 5. 1862 Wien – 21. 10. 1931 ebd.), \emph{Schriftsteller, Mediziner}!Alkandi’s Lied@\strich\emph{Alkandi’s Lied}|pwk} in Kontakt stand. Siehe A. S.: \emph{Tagebuch}, 23. 6. 1891. }}}\label{K_L02702-1}
               aufgeführt zu werden,{ }ſo könnte ich vielleicht etwas richten. Kommſt Du alſo doch
               zuerſt in \label{K_L02702-2v}\edtext{\textsc{Prag\oindex{Prag@\textbf{Prag}, \emph{Land}|pw}}}{\lemma{\textnormal{\emph{Prag}}}\Cendnote{\textnormal{Siehe XXXX Auszeichnungsfehler: Dokument L02699 nicht gefunden.
               }}}\label{K_L02702-2} daran? Und wann und bei wem das \label{K_L02702-3v}\edtext{Buch\pwindex{Schnitzler, Arthur 15. 5. 1862 Wien – 21. 10. 1931 ebd.@\textsc{Schnitzler, Arthur} (15. 5. 1862 Wien – 21. 10. 1931 ebd.), \emph{Schriftsteller, Mediziner}!Anatol@\strich\emph{Anatol}|pwv}}{\lemma{\textnormal{\emph{Buch}}}\Cendnote{\textnormal{Arthur Schnitzler: \emph{Anatol}\pwindex{Schnitzler, Arthur 15. 5. 1862 Wien – 21. 10. 1931 ebd.@\textsc{Schnitzler, Arthur} (15. 5. 1862 Wien – 21. 10. 1931 ebd.), \emph{Schriftsteller, Mediziner}!Anatol@\strich\emph{Anatol}|pwk}. Berlin\oindex{Berlin@\textbf{Berlin}, \emph{Hauptstadt}|pwk}: \emph{Bibliographisches Bureau}\orgindex{Bibliographisches Bureau@Bibliographisches Bureau|pwk}{ }1892, vordatiert auf 1893.
               }}}\label{K_L02702-3}? Ich weiß leider{ }ſo gar nichts mehr. Und \label{K_L02702-4v}\edtext{mit wem warſt Du in Venedig\oindex{Venedig@\textbf{Venedig}|pw}}{\lemma{\textnormal{\emph{mit … Venedig}}}\Cendnote{\textnormal{Schnitzler war vom 17. 8. 1892 bis zum 22. 9. 1892 mit seinem
                  Bruder Julius\pwindex{Schnitzler, Julius 13.\,7.\,1865 Wien – 29.\,6.\,1939 ebd.@\textsc{Schnitzler, Julius} (13.\,7.\,1865 Wien – 29.\,6.\,1939 ebd.), \emph{Chirurg}|pwk} in Venedig\oindex{Venedig@\textbf{Venedig}|pwk}. Dieser reiste bereits am 20. 9. 1892 ab.}}}\label{K_L02702-4}? Hätteſt Du mir ein
               Wort geſagt,{ }ſo würde ich meinen Urlaub verſchoben haben und mitgekommen{ }ſein.\pend
           
\pstart
           Bitte lies: 1.) \label{K_L02702-5v}\edtext{\textsc{Renan\pwindex{Renan, Ernest 27.\,2.\,1823 Tréguier – 2.\,10.\,1892 Paris@\textsc{Renan, Ernest} (27.\,2.\,1823 Tréguier – 2.\,10.\,1892 Paris), \emph{Schriftsteller, Orientalist}|pw}}: Leben Jeſu\pwindex{Leben Jesu. Vollständige Volks-Ausgabe@\emph{Das Leben Jesu. Vollständige Volks-Ausgabe}|pw}}{\lemma{\textnormal{\emph{Renan: Leben Jesu}}}\Cendnote{\textnormal{Eine Lektüre der genannten Werke\pwindex{Leben Jesu. Vollständige Volks-Ausgabe@\emph{Das Leben Jesu. Vollständige Volks-Ausgabe}|pwkv}\pwindex{Chamfort, Sébastien Roch Nicolas 6.\,4.\,1741 Clermont – 13.\,4.\,1794 Paris@\textsc{Chamfort, Sébastien Roch Nicolas} (6.\,4.\,1741 Clermont – 13.\,4.\,1794 Paris), \emph{Schriftsteller}!Maximes et Pensées, Caractères et Anecdotes@\strich\emph{Maximes et Pensées, Caractères et Anecdotes}|pwkv}\pwindex{Prudhomme, Sully 16.\,3.\,1839 Paris – 7.\,9.\,1907 Châtenay-Malabry@\textsc{Prudhomme, Sully} (16.\,3.\,1839 Paris – 7.\,9.\,1907 Châtenay-Malabry), \emph{Schriftsteller, Librettist, Nobelpreisträger}!caresses@\strich\emph{Les caresses}|pwkv}
                  durch Schnitzler lässt sich nachweisen, doch
                  findet sich Renan\pwindex{Renan, Ernest 27.\,2.\,1823 Tréguier – 2.\,10.\,1892 Paris@\textsc{Renan, Ernest} (27.\,2.\,1823 Tréguier – 2.\,10.\,1892 Paris), \emph{Schriftsteller, Orientalist}|pwk} in Schnitzlers{ }\emph{Leseliste}\pwindex{Schnitzler, Arthur 15. 5. 1862 Wien – 21. 10. 1931 ebd.@\textsc{Schnitzler, Arthur} (15. 5. 1862 Wien – 21. 10. 1931 ebd.), \emph{Schriftsteller, Mediziner}!Notizen zu Lektüre und Theaterbesuchen (1879-1927)@\strich\emph{Notizen zu Lektüre und Theaterbesuchen (1879-1927)}|pwk}.}}}\label{K_L02702-5} (Kleine Volks{\pb}ausgabe) 2.
                  \textsc{Chamfort\pwindex{Chamfort, Sébastien Roch Nicolas 6.\,4.\,1741 Clermont – 13.\,4.\,1794 Paris@\textsc{Chamfort, Sébastien Roch Nicolas} (6.\,4.\,1741 Clermont – 13.\,4.\,1794 Paris), \emph{Schriftsteller}|pw}}: \textsc{Maximes\pwindex{Chamfort, Sébastien Roch Nicolas 6.\,4.\,1741 Clermont – 13.\,4.\,1794 Paris@\textsc{Chamfort, Sébastien Roch Nicolas} (6.\,4.\,1741 Clermont – 13.\,4.\,1794 Paris), \emph{Schriftsteller}!Maximes et Pensées, Caractères et Anecdotes@\strich\emph{Maximes et Pensées, Caractères et Anecdotes}|pw}} (\textsc{Collection des auteurs célèbres\pwindex{Chamfort, Sébastien Roch Nicolas 6.\,4.\,1741 Clermont – 13.\,4.\,1794 Paris@\textsc{Chamfort, Sébastien Roch Nicolas} (6.\,4.\,1741 Clermont – 13.\,4.\,1794 Paris), \emph{Schriftsteller}!Œuvres choises de Chamfort, tome 2@\strich\emph{Œuvres choises de Chamfort, tome 2}|pw}}) 3.) In der \label{K_L02702-6v}\edtext{Sammlung\pwindex{Prudhomme, Sully 16.\,3.\,1839 Paris – 7.\,9.\,1907 Châtenay-Malabry@\textsc{Prudhomme, Sully} (16.\,3.\,1839 Paris – 7.\,9.\,1907 Châtenay-Malabry), \emph{Schriftsteller, Librettist, Nobelpreisträger}!Œuvres de Sully Prudhomme, tome 2@\strich\emph{Œuvres de Sully Prudhomme, tome 2}|pwuv}}{\lemma{\textnormal{\emph{Sammlung}}}\Cendnote{\textnormal{Vermutlich bezog er sich auf diese
                  Ausgabe: Sully Prudhomme\pwindex{Prudhomme, Sully 16.\,3.\,1839 Paris – 7.\,9.\,1907 Châtenay-Malabry@\textsc{Prudhomme, Sully} (16.\,3.\,1839 Paris – 7.\,9.\,1907 Châtenay-Malabry), \emph{Schriftsteller, Librettist, Nobelpreisträger}|pwk}: \emph{Les Solitudes. Poésies}\pwindex{Prudhomme, Sully 16.\,3.\,1839 Paris – 7.\,9.\,1907 Châtenay-Malabry@\textsc{Prudhomme, Sully} (16.\,3.\,1839 Paris – 7.\,9.\,1907 Châtenay-Malabry), \emph{Schriftsteller, Librettist, Nobelpreisträger}!Œuvres de Sully Prudhomme, tome 2@\strich\emph{Œuvres de Sully Prudhomme, tome 2}|pwk}. Paris:
                        \emph{Alphonse Lemerre, Éditeur}{ }1869. \emph{Les caresses}\pwindex{Prudhomme, Sully 16.\,3.\,1839 Paris – 7.\,9.\,1907 Châtenay-Malabry@\textsc{Prudhomme, Sully} (16.\,3.\,1839 Paris – 7.\,9.\,1907 Châtenay-Malabry), \emph{Schriftsteller, Librettist, Nobelpreisträger}!caresses@\strich\emph{Les caresses}|pwk} findet sich auf den
                  Seiten 117–119.}}}\label{K_L02702-6} der Gedichte von \textsc{Sully Prud’homme\pwindex{Prudhomme, Sully 16.\,3.\,1839 Paris – 7.\,9.\,1907 Châtenay-Malabry@\textsc{Prudhomme, Sully} (16.\,3.\,1839 Paris – 7.\,9.\,1907 Châtenay-Malabry), \emph{Schriftsteller, Librettist, Nobelpreisträger}|pw}} dasjenige, das den Titel trägt »\textsc{Les caresses\pwindex{Prudhomme, Sully 16.\,3.\,1839 Paris – 7.\,9.\,1907 Châtenay-Malabry@\textsc{Prudhomme, Sully} (16.\,3.\,1839 Paris – 7.\,9.\,1907 Châtenay-Malabry), \emph{Schriftsteller, Librettist, Nobelpreisträger}!caresses@\strich\emph{Les caresses}|pw}}«. Beſonders das letztere\pwindex{Prudhomme, Sully 16.\,3.\,1839 Paris – 7.\,9.\,1907 Châtenay-Malabry@\textsc{Prudhomme, Sully} (16.\,3.\,1839 Paris – 7.\,9.\,1907 Châtenay-Malabry), \emph{Schriftsteller, Librettist, Nobelpreisträger}!caresses@\strich\emph{Les caresses}|pwv}
               wird Dir vielleicht ein wenig eine brennende Herzenswunde kühlen.\pend
           
\pstart
           Grüß’ Dich Gott, liebſter Freund!\pend
           
\pstart
           Ich umarme Dich und \textsc{Richard\pwindex{Beer-Hofmann, Richard 11.\,7.\,1866 Wien – 26.\,9.\,1945 New York City@\textsc{Beer-Hofmann, Richard} (11.\,7.\,1866 Wien – 26.\,9.\,1945 New York City), \emph{Schriftsteller}|pw}}.\pend
           
\pstart
           Dein {\\[\baselineskip]}\spacefill\mbox{Paul Goldmann.}\pend
           \leftskip=0em{}\selectlanguage{ngerman}\endnumbering\briefempfaengerindex{Schnitzler, Arthur@\textsc{Schnitzler, Arthur}!zzzGoldmann, Paul@\emph{von Paul Goldmann}!1892-10-091@{9. 10. [1892]}|)be}\mylabel{L02702h}  \newcommand{\dateiname}{L02702}\newcommand{\titel}{Paul Goldmann an Arthur Schnitzler, 9. 10. [1892]}\newcommand{\editorInnen}{Martin Anton Müller und Laura Untner}%% latex-leseansicht-abspann.tex
%% Abspann für die Leseansicht.
%% Der Schalter \ifkorrekturansicht ist bereits durch den Vorspann gesetzt.

%% latex-abspann.tex
%% Gemeinsamer Abspann für Korrekturansicht und Leseansicht.
%% Setzt den Schalter \ifkorrekturansicht voraus (gesetzt in den
%% einbindenden Dateien latex-korrekturansicht-abspann.tex bzw.
%% latex-leseansicht-abspann.tex).
%% ---------------------------------------------------------------

\normalsize

% Das esempio-Environment wird nur in der Leseansicht benötigt
\ifkorrekturansicht\else
\newenvironment{esempio}[3]%
{
    \vspace{1.5ex}
    \rlap{\underline{#1}}
    \par
    \setlength{\parindent}{0cm}
    \nopagebreak
    \leftskip=#2cm
    \rightskip=#3cm
}
{
    \par
}
\fi

\doendnotes{C}
\bigskip
\vfill

\clearpage

\footnotesize

\ifkorrekturansicht
  \lohead{\textsc{register}}
\fi

% theindex-Environment neu definieren ohne reledmac
\makeatletter
\renewenvironment{theindex}{%
  \ifkorrekturansicht
    \section*{\indexname}%
  \else
    \subsubsection*{Index der erwähnten Entitäten}%
  \fi
  \setlength{\parindent}{0pt}%
  \setlength{\parskip}{0pt plus 0.3pt}%
  \let\item\@idxitem
}{%
  \ifkorrekturansicht\clearpage\fi
}
\makeatother

\IfFileExists{\jobname-pw.ind}{\input{\jobname-pw.ind}}{}

% Quellenangabe nur in der Leseansicht
\ifkorrekturansicht\else
% Fallback-Definitionen, falls die .tex-Datei \titel etc. nicht gesetzt hat
\providecommand{\titel}{}
\providecommand{\editorInnen}{}
\providecommand{\dateiname}{\jobname}

\vspace{3cm}

\vfill

\footnotesize
\textsc{Quelle}: \titel. Herausgegeben von {\editorInnen}. In: \emph{Arthur Schnitzler: Briefwechsel mit Autorinnen und Autoren}.
 Digitale Edition, https://schnitzler-briefe.acdh.oeaw.ac.at/{\dateiname}.html (Stand \today)
\fi

\end{document}


