%% latex-korrekturansicht-vorspann.tex
%% Vorspann für die Korrekturansicht.
%% Lädt die gemeinsame Datei latex-vorspann.tex mit gesetztem Schalter.

\newif\ifkorrekturansicht
\korrekturansichttrue

\input{../tex-inputs/latex-vorspann}


\section[Richard Beer-Hofmann an Arthur Schnitzler, 7. 10. 1894]{L00378 Richard Beer-Hofmann an Arthur Schnitzler, 7. 10. 1894}
\nopagebreak\mylabel{L00378v}
\rehead{ }\normalsize\beginnumbering\briefempfaengerindex{Schnitzler, Arthur@\textsc{Schnitzler, Arthur}!zzzBeer-Hofmann, Richard@\emph{von Richard Beer-Hofmann}!1894-10-071@{7. 10. 1894}|(be}
\toendnotes[C]{\smallbreak\pagebreak[2]}\Standort{CUL, Schnitzler, B 8.}
\physDesc{Bildpostkarte, 350 Zeichen
\newline{}Handschrift: Bleistift, lateinische Kurrent
\newline{}Versand: 1) Stempel: »\nobreak{}\oindex{Hotel Quirinale@\textbf{Hotel Quirinale}, \emph{Hotel (K.HTL)}|pwk}Grand Hôtel du Quirinal
                                       ROME, 7–OTT.–94, Tenu par Alessandro Marroni\nobreak{}«.   2) Stempel: »\nobreak{}\oindex{IX., Alsergrund@\textbf{IX., Alsergrund}, \emph{A.ADM3}|pwk}Wien 9/3, 9. 10. 94, 8.V, Bestellt\nobreak{}«. 
\newline{}Schnitzler: mit Bleistift nummeriert: »38« 
\newline{}Zusatz: Postkartenmotiv ist ein Lichtdruck mit Engelsburg\oindex{Engelsburg@\textbf{Engelsburg}, \emph{Museum (K.MUS)}|pw} und Petersdom\oindex{Petersdom@\textbf{Petersdom}, \emph{Kirche (K.KRC)}|pw} }\pstart{}{\pb}Herrn D\textsuperscript{r}\pend{}\pstart{}Arthur Schnitzler\pend{}\pstart{}Austria\oindex{Oesterreich@\textbf{Österreich}, \emph{A.PCLI}|pw}\pend{}\pstart{}Wien\oindex{Wien@\textbf{Wien}, \emph{A.ADM2}|pw}\pend{}\pstart{}Frankgasse 1\oindex{Frankgasse 1@\textbf{Frankgasse 1}, \emph{Wohngebäude (K.WHS)}|pw}\pend{}{\bigskip}
\pstart
           \noindent{}\centering{}{\pb}\textcolor{gray}{\textbf{Ricordo di Roma\oindex{Rom@\textbf{Rom}, \emph{P.PPLC}|pw}}}\pend
           \vspace{1em}
\pstart
           \raggedleft{}{\pb}Sonntag 7/X{ }\uline{Rom}\oindex{Rom@\textbf{Rom}, \emph{P.PPLC}|pw}\pend
           \vspace{0.5em}
\pstart
           Lieber Arthur! Warum schreiben Sie nicht? bis incl. nächsten
                  Sonntag bin ich hier – »Hôtel
                  Quirinal\oindex{Hotel Quirinale@\textbf{Hotel Quirinale}, \emph{Hotel (K.HTL)}|pw}.« Sehe aber auch auf Post nach ob nichts »posta ferma« von Ihnen.
                  Zeit\pwindex{Zeit. Wiener Wochenschrift@\emph{Die Zeit. Wiener Wochenschrift}|pw}? Schmetterlingsschlacht\pwindex{Schmetterlingsschlacht. Komoedie in 4 Akten@\emph{Die Schmetterlingsschlacht. Komödie in 4 Akten}|pw}? Bahrs’\pwindex{Bahr, Hermann 19.07.1863 – 15.01.1934@\textsc{Bahr, Hermann} (19.07.1863 – 15.01.1934), \emph{Schriftsteller/Schriftstellerin, Kritiker/Kritikerin}|pw}
               Privatadresse habe ich in unsäglicher Du{\geminationm}heit vergessen.
               In \uline{Rom}\oindex{Rom@\textbf{Rom}, \emph{P.PPLC}|pw} bin ich.\pend
           
\pstart
           Herzlichst{\\[\baselineskip]}Ihr \spacefill\mbox{Richard}\pend
           \leftskip=0em{}\selectlanguage{ngerman}\endnumbering\briefempfaengerindex{Schnitzler, Arthur@\textsc{Schnitzler, Arthur}!zzzBeer-Hofmann, Richard@\emph{von Richard Beer-Hofmann}!1894-10-071@{7. 10. 1894}|)be}\mylabel{L00378h}  \normalsize

\doendnotes{C}
\bigskip
\vfill

\clearpage

\footnotesize

\lohead{\textsc{register}}

% Definiere theindex-Environment komplett neu ohne reledmac
\makeatletter
\renewenvironment{theindex}{%
  \section*{\indexname}%
  \setlength{\parindent}{0pt}%
  \setlength{\parskip}{0pt plus 0.3pt}%
  \let\item\@idxitem
}{%
  \clearpage
}
\makeatother

\IfFileExists{\jobname-pw.ind}{\input{\jobname-pw.ind}}{}

\end{document}

      