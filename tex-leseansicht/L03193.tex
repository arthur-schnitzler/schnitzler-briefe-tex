%% latex-leseansicht-vorspann.tex
%% Vorspann für die Leseansicht.
%% Lädt die gemeinsame Datei latex-vorspann.tex mit nicht gesetztem Schalter.

\newif\ifkorrekturansicht
\korrekturansichtfalse

\input{../tex-inputs/latex-vorspann}


         
         \renewcommand{\erwaehntePersonen}{Personen: Heinrich Kanner, Alfred de Musset, Olga Schnitzler, Siegfried Trebitsch, Alice Ziegler, Arnost Ziegler}
         \renewcommand{\erwaehnteInstitutionen}{Institutionen: Deutsches Theater Berlin}
         \renewcommand{\erwaehnteOrte}{Orte: Berlin, Carl-Theater, Dessauer Straße, Wien}
         \renewcommand{\erwaehnteWerke}{Werke: Berliner Theater. (»Lebendige Stunden« von Arthur Schnitzler.), Die Zeit, Die Zeit. Wiener Wochenschrift, Kleine Chronik. [Das Wiener Gastspiel des Berliner Deutschen Theaters.], Lebendige Stunden. Vier Einakter, Lorenzaccio. Drame romantique en cinq actes, Neue Freie Presse}
               \section[ Paul Goldmann an Arthur Schnitzler, 16. 1. {[}1902{]}]{ Paul Goldmann an Arthur Schnitzler, 16. 1. {[}1902{]}}\nopagebreak\mylabel{v}\rehead{ }\begin{ledgroupsized}[t]{13cm}\normalsize\beginnumbering \toendnotes[C]{\smallbreak\pagebreak[2]} \Standort{DLA, A:Schnitzler, HS.NZ85.1.3172.}
\physDesc{Brief, 1 Blatt, 2 Seiten, 1400 Zeichen
\newline{}Handschrift: blaue Tinte, deutsche Kurrent
\newline{}Schnitzler: 1) mit Bleistift das Jahr »{[}1{]}902« vermerkt  2) mit rotem Buntstift vier Unterstreichungen}\toendnotes[C]{\smallbreak}\pstart
           \noindent{}\raggedleft{}{\pb}\textcolor{gray}{\textbf{DESSAUERSTRASSE 19}}\oindex{Dessauer Strasse@\textbf{Dessauer Straße}|pw}\pend
           \pstart
           Berlin\oindex{Berlin@\textbf{Berlin}|pw}, 16. Januar.\pend
           \pstart{}Mein lieber Freund,\pend\pstart
           Diesmal haſt \uline{Du} mich, wie ich glaube, \label{K_L03193-1v}\edtext{mißverſtanden}{\lemma{\textnormal{\emph{mißverſtanden}}}\Cendnote{\textnormal{Schnitzler\pwindex{Schnitzler, Arthur 15.05.1862 – 21.10.1931@\textsc{Schnitzler, Arthur} (15.05.1862 – 21.10.1931), \emph{Schriftsteller, Mediziner}|pwk} dürfte entweder durch Goldmann\pwindex{Goldmann, Paul 31.01.1865 – 25.09.1935@\textsc{Goldmann, Paul} (31.01.1865 – 25.09.1935), \emph{Schriftsteller, Journalist}|pwk}s abwägende Worte hinsichtlich der
                     Notiz\pwindex{Kleine Chronik. [Das Wiener Gastspiel des Berliner Deutschen
                  Theaters.]1902-01-17@\emph{Kleine Chronik. [Das Wiener Gastspiel des Berliner Deutschen Theaters.]} {[}1902-01-17{]}|pwkv} in der \emph{Neuen Freien Presse}\pwindex{Neue Freie Presse1864 – 1939@\emph{Neue Freie Presse} {[}1864 – 1939{]}|pwk} zum Gastspiel\pwindex{Schnitzler, Arthur 15.05.1862 – 21.10.1931@\textsc{Schnitzler, Arthur} (15.05.1862 – 21.10.1931), \emph{Schriftsteller, Mediziner}!Lebendige Stunden. Vier Einakter1901-12-23@\strich\emph{Lebendige Stunden. Vier Einakter} {[}1901-12-23{]}|pwkv} des \emph{Deutschen Theaters Berlin}\orgindex{Deutsches Theater Berlin@Deutsches Theater Berlin|pwk} am Wien\oindex{Wien@\textbf{Wien}|pwk}er Carl-Theater\oindex{Carl-Theater@\textbf{Carl-Theater}|pwk} verstört gewesen
                  sein, oder durch die »eiſige[] Kälte«, mit der dieser am Feuilleton\pwindex{Goldmann, Paul 31.01.1865 – 25.09.1935@\textsc{Goldmann, Paul} (31.01.1865 – 25.09.1935), \emph{Schriftsteller, Journalist}!Berliner Theater. (»Lebendige Stunden« von Arthur Schnitzler.)1902-01-22@\strich\emph{Berliner Theater. (»Lebendige Stunden« von Arthur Schnitzler.)} {[}1902-01-22{]}|pwkv} über \emph{Lebendige Stunden}\pwindex{Schnitzler, Arthur 15.05.1862 – 21.10.1931@\textsc{Schnitzler, Arthur} (15.05.1862 – 21.10.1931), \emph{Schriftsteller, Mediziner}!Lebendige Stunden. Vier Einakter1901-12-23@\strich\emph{Lebendige Stunden. Vier Einakter} {[}1901-12-23{]}|pwk} arbeitete. Siehe Paul Goldmann an Arthur Schnitzler, 14. 1. [1902]. }}}\label{K_L03193-1h}. Deine
               Standrede hat mich \strikeout{daher} überraſcht, weil mein
               letzter Brief ganz harmlos gemeint war. Aber ich mag nicht darauf erwidern. Ich habe
               keine Zeit zur Polemik; ich ſchreibe lieber an dem \textsc{Feuilleton\pwindex{Goldmann, Paul 31.01.1865 – 25.09.1935@\textsc{Goldmann, Paul} (31.01.1865 – 25.09.1935), \emph{Schriftsteller, Journalist}!Berliner Theater. (»Lebendige Stunden« von Arthur Schnitzler.)1902-01-22@\strich\emph{Berliner Theater. (»Lebendige Stunden« von Arthur Schnitzler.)} {[}1902-01-22{]}|pwv}} über Deine Stücke\pwindex{Schnitzler, Arthur 15.05.1862 – 21.10.1931@\textsc{Schnitzler, Arthur} (15.05.1862 – 21.10.1931), \emph{Schriftsteller, Mediziner}!Lebendige Stunden. Vier Einakter1901-12-23@\strich\emph{Lebendige Stunden. Vier Einakter} {[}1901-12-23{]}|pwv}
               weiter. Bin ich wirklich ſo koloſſal empfindlich? Ich finde, es iſt bequem, \strikeout{\textcolor{gray}{die}{ }\textcolor{gray}{×}\-\textcolor{gray}{×}\-\textcolor{gray}{×}\-\textcolor{gray}{×}\-\textcolor{gray}{×}\-\textcolor{gray}{×} an} irgendwelche
               Differenzen durch die Empfindlichkeit des Anderen zu erklären. Man erſpart ſich
               ſelbſt dadurch jedes Gefühl der Verantwortung. Aber es gäbe vielleicht auch eine
               andere Erklärung. Beiſpielsweiſe die, daß von Dir zu mir nicht Alles in Ordnung iſt –
               vielleicht ſchon ſeit Jahren nicht in Ordnung iſt. Außer über meine Empfindlichkeit
               ſollteſt Du auch darüber einmal nachdenken.\pend
           \pstart
           Du haſt gewünſcht, wir ſollten grob zu einander ſein. Bin ich grob genug? Aber laſſen
               wir es dabei {\pb}bewenden. Dieſe Diskuſſionen führen zu
               nichts.\pend
           \pstart
           Ich wäre Dir ſehr dankbar, wenn Du \label{K_L03193-2v}\edtext{\textsc{Trebitsch\pwindex{Trebitsch, Siegfried 22.12.1868 – 03.06.1956@\textsc{Trebitsch, Siegfried} (22.12.1868 – 03.06.1956), \emph{Schriftsteller, Übersetzer}|pw}} bewegen}{\lemma{\textnormal{\emph{Trebitsch bewegen}}}\Cendnote{\textnormal{Musset\pwindex{Musset, Alfred de 11.12.1810 – 02.05.1857@\textsc{Musset, Alfred de} (11.12.1810 – 02.05.1857), \emph{Schriftsteller}|pwk}s \emph{Lorenzaccio}\pwindex{Musset, Alfred de 11.12.1810 – 02.05.1857@\textsc{Musset, Alfred de} (11.12.1810 – 02.05.1857), \emph{Schriftsteller}!Lorenzaccio. Drame romantique en cinq actes1834@\strich\emph{Lorenzaccio. Drame romantique en cinq actes} {[}1834{]}|pwk} wurde von Siegfried
                     Trebitsch\pwindex{Trebitsch, Siegfried 22.12.1868 – 03.06.1956@\textsc{Trebitsch, Siegfried} (22.12.1868 – 03.06.1956), \emph{Schriftsteller, Übersetzer}|pwk} nicht übersetzt.}}}\label{K_L03193-2h} könnteſt, von der \textsc{Lorenzaccio\pwindex{Musset, Alfred de 11.12.1810 – 02.05.1857@\textsc{Musset, Alfred de} (11.12.1810 – 02.05.1857), \emph{Schriftsteller}!Lorenzaccio. Drame romantique en cinq actes1834@\strich\emph{Lorenzaccio. Drame romantique en cinq actes} {[}1834{]}|pwv}}-Überſetzung abzuſehen. Vielleicht mache ich mich \label{K_L03193-4v}\edtext{doch noch einmal}{\lemma{\textnormal{\emph{doch noch einmal}}}\Cendnote{\textnormal{Siehe Paul Goldmann an Arthur Schnitzler, 2. [1.? 1897]. Goldmann\pwindex{Goldmann, Paul 31.01.1865 – 25.09.1935@\textsc{Goldmann, Paul} (31.01.1865 – 25.09.1935), \emph{Schriftsteller, Journalist}|pwk} veröffentlichte zwar nie eine \emph{Lorenzaccio}\pwindex{Musset, Alfred de 11.12.1810 – 02.05.1857@\textsc{Musset, Alfred de} (11.12.1810 – 02.05.1857), \emph{Schriftsteller}!Lorenzaccio. Drame romantique en cinq actes1834@\strich\emph{Lorenzaccio. Drame romantique en cinq actes} {[}1834{]}|pwk}-Übersetzung, jedoch eine von Musset\pwindex{Musset, Alfred de 11.12.1810 – 02.05.1857@\textsc{Musset, Alfred de} (11.12.1810 – 02.05.1857), \emph{Schriftsteller}|pwk}s \emph{Il ne faut jurer
                     de rien}\textcolor{red}{\textsuperscript{XXXX indx}}: Alfred de Musset\pwindex{Musset, Alfred de 11.12.1810 – 02.05.1857@\textsc{Musset, Alfred de} (11.12.1810 – 02.05.1857), \emph{Schriftsteller}|pwk}:
                        \emph{Man soll nichts verschwören. Komödie in 3
                        Akten}\textcolor{red}{\textsuperscript{XXXX indx}} [1836/48].
                     Übersetzt von Paul Goldmann\pwindex{Goldmann, Paul 31.01.1865 – 25.09.1935@\textsc{Goldmann, Paul} (31.01.1865 – 25.09.1935), \emph{Schriftsteller, Journalist}|pwk}. Frankfurt a. M.\oindex{XXXX Ortsangabe fehlt|pwk}: \emph{Rütten {\kaufmannsund} Loening}XXXX ORGangabe fehlt{ }1902.}}}\label{K_L03193-4h} an
               dieſe Arbeit.\pend
           \pstart
           \textsc{Kanner\pwindex{Kanner, Heinrich 09.11.1864 – 15.02.1930@\textsc{Kanner, Heinrich} (09.11.1864 – 15.02.1930), \emph{Herausgeber, Publizist}|pw}}, der in \textsc{Berlin}\oindex{Berlin@\textbf{Berlin}|pw} weilt, war bei mir. Die \label{K_L03193-6v}\edtext{Umwandlung der »Zeit\pwindex{Zeit. Wiener Wochenschrift1894 – 1904@\emph{Die Zeit. Wiener Wochenschrift} {[}1894 – 1904{]}|pw}« in ein Tagesblatt\pwindex{Zeit1902-09-27 – 1919@\emph{Die Zeit} {[}1902-09-27 – 1919{]}|pwv}}{\lemma{\textnormal{\emph{Umwandlung … Tagesblatt}}}\Cendnote{\textnormal{siehe Paul Goldmann an Arthur Schnitzler und Olga
               Gussmann, 7. 7. [1901]}}}\label{K_L03193-6h} iſt beſchloſſene Sache.\pend
           \pstart
           \textsc{Alice Bondy\pwindex{Ziegler, Alice 1880-01-05 – Dezember 1943@\textsc{Ziegler, Alice} (1880-01-05 – Dezember 1943)|pw}} zeigt mir ihre \label{K_L03193-8v}\edtext{Verlobung}{\lemma{\textnormal{\emph{Verlobung}}}\Cendnote{\textnormal{Ernst Ziegler\pwindex{Ziegler, Arnost 1871-12-06 – 1943-01-02@\textsc{Ziegler, Arnost} (1871-12-06 – 1943-01-02), \emph{Bankdirektor}|pwk} und Alice Bondy\pwindex{Ziegler, Alice 1880-01-05 – Dezember 1943@\textsc{Ziegler, Alice} (1880-01-05 – Dezember 1943)|pwk} heirateten am 7. 5. 1902. In den späten 1890er-Jahren hatte
                     Goldmann\pwindex{Goldmann, Paul 31.01.1865 – 25.09.1935@\textsc{Goldmann, Paul} (31.01.1865 – 25.09.1935), \emph{Schriftsteller, Journalist}|pwk} für die damals knapp unter 20-Jährige\pwindex{Ziegler, Alice 1880-01-05 – Dezember 1943@\textsc{Ziegler, Alice} (1880-01-05 – Dezember 1943)|pwkv} geschwärmt,
                     siehe Paul Goldmann an Arthur Schnitzler, 10. 12. [1897], 19. 1. [1898] und 30. 8. 1899.}}}\label{K_L03193-8h} mit
               einem \textsc{Dr. Ziegler\pwindex{Ziegler, Arnost 1871-12-06 – 1943-01-02@\textsc{Ziegler, Arnost} (1871-12-06 – 1943-01-02), \emph{Bankdirektor}|pw}} an.\pend
           \pstart
           Es thut mir unendlich leid, daß \textsc{Olga\pwindex{Schnitzler, Olga 17.01.1882 – 13.01.1970@\textsc{Schnitzler, Olga} (17.01.1882 – 13.01.1970), \emph{Schauspielerin, Sängerin}|pw}} ſich ſo \label{K_L03193-9v}\edtext{plagen}{\lemma{\textnormal{\emph{plagen}}}\Cendnote{\textnormal{womöglich verursacht durch die
                  Schwangerschaft, siehe A. S.: \emph{Tagebuch}, 4. 1. 1902 und 8. 1. 1902}}}\label{K_L03193-9h} muß. Verſichere ſie meiner herzlichſten Antheilnahme und grüße ſie
               vielmals.\pend
           \pstart
           Auch Du ſei von Herzen gegrüßt. {\\[\baselineskip]}Dein {\\[\baselineskip]}\spacefill\mbox{Paul Goldm}\pend
           \leftskip=0em{}
         
         \endnumbering\mylabel{h}\end{ledgroupsized}  \newcommand{\dateiname}{L03193}\newcommand{\titel}{Paul Goldmann an Arthur Schnitzler, 16. 1. [1902]}\newcommand{\editorInnen}{Martin Anton Müller und Laura Untner}%% latex-leseansicht-abspann.tex
%% Abspann für die Leseansicht.
%% Der Schalter \ifkorrekturansicht ist bereits durch den Vorspann gesetzt.

%% latex-abspann.tex
%% Gemeinsamer Abspann für Korrekturansicht und Leseansicht.
%% Setzt den Schalter \ifkorrekturansicht voraus (gesetzt in den
%% einbindenden Dateien latex-korrekturansicht-abspann.tex bzw.
%% latex-leseansicht-abspann.tex).
%% ---------------------------------------------------------------

\normalsize

% Das esempio-Environment wird nur in der Leseansicht benötigt
\ifkorrekturansicht\else
\newenvironment{esempio}[3]%
{
    \vspace{1.5ex}
    \rlap{\underline{#1}}
    \par
    \setlength{\parindent}{0cm}
    \nopagebreak
    \leftskip=#2cm
    \rightskip=#3cm
}
{
    \par
}
\fi

\doendnotes{C}
\bigskip
\vfill

\clearpage

\footnotesize

\ifkorrekturansicht
  \lohead{\textsc{register}}
\fi

% theindex-Environment neu definieren ohne reledmac
\makeatletter
\renewenvironment{theindex}{%
  \ifkorrekturansicht
    \section*{\indexname}%
  \else
    \subsubsection*{Index der erwähnten Entitäten}%
  \fi
  \setlength{\parindent}{0pt}%
  \setlength{\parskip}{0pt plus 0.3pt}%
  \let\item\@idxitem
}{%
  \ifkorrekturansicht\clearpage\fi
}
\makeatother

\IfFileExists{\jobname-pw.ind}{\input{\jobname-pw.ind}}{}

% Quellenangabe nur in der Leseansicht
\ifkorrekturansicht\else
% Fallback-Definitionen, falls die .tex-Datei \titel etc. nicht gesetzt hat
\providecommand{\titel}{}
\providecommand{\editorInnen}{}
\providecommand{\dateiname}{\jobname}

\vspace{3cm}

\vfill

\footnotesize
\textsc{Quelle}: \titel. Herausgegeben von {\editorInnen}. In: \emph{Arthur Schnitzler: Briefwechsel mit Autorinnen und Autoren}.
 Digitale Edition, https://schnitzler-briefe.acdh.oeaw.ac.at/{\dateiname}.html (Stand \today)
\fi

\end{document}


      