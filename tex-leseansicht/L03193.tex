%% latex-korrekturansicht-vorspann.tex
%% Vorspann für die Korrekturansicht.
%% Lädt die gemeinsame Datei latex-vorspann.tex mit gesetztem Schalter.

\newif\ifkorrekturansicht
\korrekturansichttrue

\input{../tex-inputs/latex-vorspann}


\section[ Paul Goldmann an Arthur Schnitzler, 16. 1. {[}1902{]}]{L03193 Paul Goldmann an Arthur Schnitzler, 16. 1. {[}1902{]}}
\nopagebreak\mylabel{L03193v}
\rehead{ }\normalsize\beginnumbering\briefempfaengerindex{Schnitzler, Arthur@\textsc{Schnitzler, Arthur}!zzzGoldmann, Paul@\emph{von Paul Goldmann}!1902-01-161@{16. 1. {[}1902{]}}|(be}
\toendnotes[C]{\smallbreak\pagebreak[2]}\Standort{DLA, A:Schnitzler, HS.NZ85.1.3172.}
\physDesc{Brief, 1 Blatt, 2 Seiten, 1400 Zeichen
\newline{}Handschrift: blaue Tinte, deutsche Kurrent
\newline{}Schnitzler: 1) mit Bleistift das Jahr »902« vermerkt  2) mit rotem Buntstift vier Unterstreichungen}\toendnotes[C]{\smallbreak}
\pstart
           \raggedleft{}{\pb}\textcolor{gray}{\textbf{DESSAUERSTRASSE 19}}\oindex{Dessauer Strasse@\textbf{Dessauer Straße}, \emph{Straße (K.STR)}|pw}\pend
           
\pstart
           Berlin\oindex{Berlin@\textbf{Berlin}, \emph{P.PPLC}|pw}, 16. Januar.\pend
           
\pstart{}Mein lieber Freund,\pend\vspace{0.5em}
\pstart
           Diesmal haſt \uline{Du} mich, wie ich glaube, \label{K_L03193-1v}\edtext{mißverſtanden}{\lemma{\textnormal{\emph{mißverſtanden}}}\Cendnote{\textnormal{Schnitzler dürfte entweder durch Goldmanns\pwindex{Goldmann, Paul 31.01.1865 – 25.09.1935@\textsc{Goldmann, Paul} (31.01.1865 – 25.09.1935), \emph{Schriftsteller/Schriftstellerin, Journalist/Journalistin}|pwk} abwägende Worte hinsichtlich der
                     Notiz\pwindex{Kleine Chronik. [Das Wiener Gastspiel des Berliner Deutschen Theaters.]@\emph{Kleine Chronik. [Das Wiener Gastspiel des Berliner Deutschen Theaters.]}|pwkv} in der \emph{Neuen Freien Presse}\pwindex{Neue Freie Presse@\emph{Neue Freie Presse}|pwk} zum Gastspiel\pwindex{Lebendige Stunden. Vier Einakter@\emph{Lebendige Stunden. Vier Einakter}|pwkv} des \emph{Deutschen Theaters Berlin}\orgindex{Deutsches Theater Berlin@Deutsches Theater Berlin|pwk} am Wien\oindex{Wien@\textbf{Wien}, \emph{A.ADM2}|pwk}er Carl-Theater\oindex{Carl-Theater@\textbf{Carl-Theater}, \emph{Theater (K.THE)}|pwk} verstört gewesen
                  sein, oder durch die »eiſige[] Kälte«, mit der dieser am Feuilleton\pwindex{Berliner Theater. (»Lebendige Stunden« von Arthur Schnitzler.)@\emph{Berliner Theater. (»Lebendige Stunden« von Arthur Schnitzler.)}|pwkv} über \emph{Lebendige Stunden}\pwindex{Lebendige Stunden. Vier Einakter@\emph{Lebendige Stunden. Vier Einakter}|pwk} arbeitete. Siehe Paul Goldmann an Arthur Schnitzler, 14. 1. [1902]. }}}\label{K_L03193-1}. Deine
               Standrede hat mich \strikeout{daher} überraſcht, weil mein
               letzter Brief ganz harmlos gemeint war. Aber ich mag nicht darauf erwidern. Ich habe
               keine Zeit zur Polemik; ich ſchreibe lieber an dem \textsc{Feuilleton\pwindex{Berliner Theater. (»Lebendige Stunden« von Arthur Schnitzler.)@\emph{Berliner Theater. (»Lebendige Stunden« von Arthur Schnitzler.)}|pwv}} über Deine Stücke\pwindex{Lebendige Stunden. Vier Einakter@\emph{Lebendige Stunden. Vier Einakter}|pwv}
               weiter. Bin ich wirklich ſo koloſſal empfindlich? Ich finde, es iſt bequem, \strikeout{\textcolor{gray}{die}{ }\textcolor{gray}{×}\-\textcolor{gray}{×}\-\textcolor{gray}{×}\-\textcolor{gray}{×}\-\textcolor{gray}{×}\-\textcolor{gray}{×} an} irgendwelche
               Differenzen durch die Empfindlichkeit des Anderen zu erklären. Man erſpart ſich
               ſelbſt dadurch jedes Gefühl der Verantwortung. Aber es gäbe vielleicht auch eine
               andere Erklärung. Beiſpielsweiſe die, daß von Dir zu mir nicht Alles in Ordnung iſt –
               vielleicht ſchon ſeit Jahren nicht in Ordnung iſt. Außer über meine Empfindlichkeit
               ſollteſt Du auch darüber einmal nachdenken.\pend
           
\pstart
           Du haſt gewünſcht, wir ſollten grob zu einander ſein. Bin ich grob genug? Aber laſſen
               wir es dabei {\pb}bewenden. Dieſe Diskuſſionen führen zu
               nichts.\pend
           
\pstart
           Ich wäre Dir ſehr dankbar, wenn Du \label{K_L03193-2v}\edtext{\textsc{Trebitsch\pwindex{Trebitsch, Siegfried 22.12.1868 – 03.06.1956@\textsc{Trebitsch, Siegfried} (22.12.1868 – 03.06.1956), \emph{Schriftsteller/Schriftstellerin, Übersetzer/Übersetzerin}|pw}} bewegen}{\lemma{\textnormal{\emph{Trebitsch bewegen}}}\Cendnote{\textnormal{Mussets\pwindex{Musset, Alfred de 11.12.1810 – 02.05.1857@\textsc{Musset, Alfred de} (11.12.1810 – 02.05.1857), \emph{Schriftsteller/Schriftstellerin}|pwk}{ }\emph{Lorenzaccio}\pwindex{Lorenzaccio. Drame romantique en cinq actes@\emph{Lorenzaccio. Drame romantique en cinq actes}|pwk} wurde von Siegfried
                     Trebitsch\pwindex{Trebitsch, Siegfried 22.12.1868 – 03.06.1956@\textsc{Trebitsch, Siegfried} (22.12.1868 – 03.06.1956), \emph{Schriftsteller/Schriftstellerin, Übersetzer/Übersetzerin}|pwk} nicht übersetzt.}}}\label{K_L03193-2} könnteſt, von der \textsc{Lorenzaccio\pwindex{Lorenzaccio. Drame romantique en cinq actes@\emph{Lorenzaccio. Drame romantique en cinq actes}|pwv}}-Überſetzung abzuſehen. Vielleicht mache ich mich \label{K_L03193-3v}\edtext{doch noch einmal}{\lemma{\textnormal{\emph{doch noch einmal}}}\Cendnote{\textnormal{Siehe Paul Goldmann an Arthur Schnitzler, 2. [1.? 1897]. Goldmann\pwindex{Goldmann, Paul 31.01.1865 – 25.09.1935@\textsc{Goldmann, Paul} (31.01.1865 – 25.09.1935), \emph{Schriftsteller/Schriftstellerin, Journalist/Journalistin}|pwk} veröffentlichte zwar nie eine \emph{Lorenzaccio}\pwindex{Lorenzaccio. Drame romantique en cinq actes@\emph{Lorenzaccio. Drame romantique en cinq actes}|pwk}-Übersetzung, jedoch eine von Mussets\pwindex{Musset, Alfred de 11.12.1810 – 02.05.1857@\textsc{Musset, Alfred de} (11.12.1810 – 02.05.1857), \emph{Schriftsteller/Schriftstellerin}|pwk}{ }\emph{Il ne faut jurer
                     de rien}\pwindex{ne faut jurer de rien@\emph{Il ne faut jurer de rien}|pwk}: Alfred de Musset\pwindex{Musset, Alfred de 11.12.1810 – 02.05.1857@\textsc{Musset, Alfred de} (11.12.1810 – 02.05.1857), \emph{Schriftsteller/Schriftstellerin}|pwk}:
                        \emph{Man soll nichts verschwören. Komödie in 3
                        Akten}\pwindex{Man soll nichts verschwoeren. Komoedie in 3 Akten@\emph{Man soll nichts verschwören. Komödie in 3 Akten}|pwk} [1836/48].
                     Übersetzt von Paul Goldmann\pwindex{Goldmann, Paul 31.01.1865 – 25.09.1935@\textsc{Goldmann, Paul} (31.01.1865 – 25.09.1935), \emph{Schriftsteller/Schriftstellerin, Journalist/Journalistin}|pwk}. Frankfurt a. M.\oindex{Frankfurt am Main@\textbf{Frankfurt am Main}, \emph{P.PPLA3}|pwk}: \emph{Rütten {\kaufmannsund} Loening}\orgindex{Ruetten und Loening@Rütten {\kaufmannsund}  Loening|pwk}{ }1902.}}}\label{K_L03193-3} an
               dieſe Arbeit.\pend
           
\pstart
           \textsc{Kanner\pwindex{Kanner, Heinrich 09.11.1864 – 15.02.1930@\textsc{Kanner, Heinrich} (09.11.1864 – 15.02.1930), \emph{Herausgeber/Herausgeberin, Publizist/Publizistin}|pw}}, der in \textsc{Berlin}\oindex{Berlin@\textbf{Berlin}, \emph{P.PPLC}|pw} weilt, war bei mir. Die \label{K_L03193-4v}\edtext{Umwandlung der »Zeit\pwindex{Zeit. Wiener Wochenschrift@\emph{Die Zeit. Wiener Wochenschrift}|pw}« in ein Tagesblatt\pwindex{Zeit@\emph{Die Zeit}|pwv}}{\lemma{\textnormal{\emph{Umwandlung … Tagesblatt}}}\Cendnote{\textnormal{Siehe Paul Goldmann an Arthur Schnitzler und Olga
               Gussmann, 7. 7. [1901].
               }}}\label{K_L03193-4} iſt beſchloſſene Sache.\pend
           
\pstart
           \textsc{Alice Bondy\pwindex{Ziegler, Alice 1880-01-05 – Dezember 1943@\textsc{Ziegler, Alice} (1880-01-05 – Dezember 1943)|pw}} zeigt mir ihre \label{K_L03193-5v}\edtext{Verlobung}{\lemma{\textnormal{\emph{Verlobung}}}\Cendnote{\textnormal{Ernst Ziegler\pwindex{Ziegler, Arnost 1871-12-06 – 1943-01-02@\textsc{Ziegler, Arnost} (1871-12-06 – 1943-01-02), \emph{Bankdirektor/Bankdirektorin}|pwk} und Alice Bondy\pwindex{Ziegler, Alice 1880-01-05 – Dezember 1943@\textsc{Ziegler, Alice} (1880-01-05 – Dezember 1943)|pwk} heirateten am 7. 5. 1902. In den späten 1890er-Jahren hatte
                     Goldmann\pwindex{Goldmann, Paul 31.01.1865 – 25.09.1935@\textsc{Goldmann, Paul} (31.01.1865 – 25.09.1935), \emph{Schriftsteller/Schriftstellerin, Journalist/Journalistin}|pwk} für die damals knapp unter 20-Jährige\pwindex{Ziegler, Alice 1880-01-05 – Dezember 1943@\textsc{Ziegler, Alice} (1880-01-05 – Dezember 1943)|pwkv} geschwärmt,
                     siehe Paul Goldmann an Arthur Schnitzler, 10. 12. [1897], 19. 1. [1898] und 30. 8. 1899.}}}\label{K_L03193-5} mit
               einem \textsc{Dr. Ziegler\pwindex{Ziegler, Arnost 1871-12-06 – 1943-01-02@\textsc{Ziegler, Arnost} (1871-12-06 – 1943-01-02), \emph{Bankdirektor/Bankdirektorin}|pw}} an.\pend
           
\pstart
           Es thut mir unendlich leid, daß \textsc{Olga\pwindex{Schnitzler, Olga 17.01.1882 – 13.01.1970@\textsc{Schnitzler, Olga} (17.01.1882 – 13.01.1970), \emph{Schauspieler/Schauspielerin, Sänger/Sängerin}|pw}} ſich ſo \label{K_L03193-6v}\edtext{plagen}{\lemma{\textnormal{\emph{plagen}}}\Cendnote{\textnormal{Er dürfte auf Komplikationen oder zumindest Sorgen in der  Schwangerschaft anspielen, siehe A. S.: \emph{Tagebuch}, 4. 1. 1902 und 8. 1. 1902.
               }}}\label{K_L03193-6} muß. Verſichere ſie meiner herzlichſten Antheilnahme und grüße ſie
               vielmals.\pend
           
\pstart
           Auch Du ſei von Herzen gegrüßt. {\\[\baselineskip]}Dein {\\[\baselineskip]}\spacefill\mbox{Paul Goldm}\pend
           \leftskip=0em{}\selectlanguage{ngerman}\endnumbering\briefempfaengerindex{Schnitzler, Arthur@\textsc{Schnitzler, Arthur}!zzzGoldmann, Paul@\emph{von Paul Goldmann}!1902-01-161@{16. 1. {[}1902{]}}|)be}\mylabel{L03193h}  \normalsize

\doendnotes{C}
\bigskip
\vfill

\clearpage

\footnotesize

\lohead{\textsc{register}}

% Definiere theindex-Environment komplett neu ohne reledmac
\makeatletter
\renewenvironment{theindex}{%
  \section*{\indexname}%
  \setlength{\parindent}{0pt}%
  \setlength{\parskip}{0pt plus 0.3pt}%
  \let\item\@idxitem
}{%
  \clearpage
}
\makeatother

\IfFileExists{\jobname-pw.ind}{\input{\jobname-pw.ind}}{}

\end{document}

      