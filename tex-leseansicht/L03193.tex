%% latex-leseansicht-vorspann.tex
%% Vorspann für die Leseansicht.
%% Lädt die gemeinsame Datei latex-vorspann.tex mit nicht gesetztem Schalter.

\newif\ifkorrekturansicht
\korrekturansichtfalse

\input{../tex-inputs/latex-vorspann}


\section[ Paul Goldmann an Arthur Schnitzler, 16. 1. [1902]]{L03193 Paul Goldmann an Arthur Schnitzler,  16. 1. [1902]}
\nopagebreak\mylabel{L03193v}
\rehead{ }\normalsize\beginnumbering\briefempfaengerindex{Schnitzler, Arthur@\textsc{Schnitzler, Arthur}!zzzGoldmann, Paul@\emph{von Paul Goldmann}!1902-01-162@{16. 1. [1902]}|(be}
\toendnotes[C]{\smallbreak\pagebreak[2]}
\correspDesc{Versand  durch Paul Goldmann am 16. 1. [1902] in Berlin
\newline{}Erhalt  durch Arthur Schnitzler im Zeitraum [17. 1. 1902
                  – 21. 1. 1902?] in Wien}\toendnotes[C]{\smallbreak}
\Standort{DLA, A:Schnitzler, HS.NZ85.1.3172.}
\physDesc{Brief, 1 Blatt, 2 Seiten, 1400 Zeichen
\newline{}Handschrift: blaue Tinte, deutsche Kurrent
\newline{}Schnitzler: 1) mit Bleistift das Jahr »902« vermerkt  2) mit rotem Buntstift vier Unterstreichungen}\toendnotes[C]{\smallbreak}
\pstart
           \raggedleft{}{\pb}\textcolor{gray}{\textbf{DESSAUERSTRASSE 19}}\oindex{Dessauer Straße@\textbf{Dessauer Straße}, \emph{Straße}|pw}\pend
           
\pstart
           Berlin\oindex{Berlin@\textbf{Berlin}, \emph{Hauptstadt}|pw}, 16. Januar.\pend
           
\pstart{}Mein lieber Freund,\pend\vspace{0.5em}
\pstart
           Diesmal haſt \uline{Du} mich, wie ich glaube, \label{K_L03193-1v}\edtext{mißverſtanden}{\lemma{\textnormal{\emph{mißverstanden}}}\Cendnote{\textnormal{Schnitzler dürfte entweder durch Goldmanns\pwindex{Goldmann, Paul 31.\,1.\,1865 Breslau – 25.\,9.\,1935 Wien@\textsc{Goldmann, Paul} (31.\,1.\,1865 Breslau – 25.\,9.\,1935 Wien), \emph{Schriftsteller, Journalist}|pwk} abwägende Worte hinsichtlich der
                     Notiz\pwindex{Kleine Chronik. [Das Wiener Gastspiel des Berliner Deutschen Theaters.]@\emph{Kleine Chronik. [Das Wiener Gastspiel des Berliner Deutschen Theaters.]}|pwkv} in der \emph{Neuen Freien Presse}\pwindex{Neue Freie Presse@\emph{Neue Freie Presse}|pwk} zum Gastspiel\pwindex{Schnitzler, Arthur 15.\,5.\,1862 Wien – 21.\,10.\,1931 ebd.@\textsc{Schnitzler, Arthur} (15.\,5.\,1862 Wien – 21.\,10.\,1931 ebd.), \emph{Schriftsteller, Mediziner}!Lebendige Stunden. Vier Einakter@\strich\emph{Lebendige Stunden. Vier Einakter}|pwkv} des \emph{Deutschen Theaters Berlin}\orgindex{Deutsches Theater Berlin@Deutsches Theater Berlin|pwk} am Wien\oindex{Wien@\textbf{Wien}, \emph{Verwaltungsgebiet}|pwk}er Carl-Theater\oindex{Wien@\textbf{Wien}!II., Leopoldstadt@\textbf{II., Leopoldstadt}!Carl-Theater@\textbf{Carl-Theater}, \emph{Theater}|pwk} verstört gewesen
                  sein, oder durch die »eiſige[] Kälte«, mit der dieser am Feuilleton\pwindex{Goldmann, Paul 31.\,1.\,1865 Breslau – 25.\,9.\,1935 Wien@\textsc{Goldmann, Paul} (31.\,1.\,1865 Breslau – 25.\,9.\,1935 Wien), \emph{Schriftsteller, Journalist}!Berliner Theater. (»Lebendige Stunden« von Arthur Schnitzler.)@\strich\emph{Berliner Theater. (»Lebendige Stunden« von Arthur Schnitzler.)}|pwkv} über \emph{Lebendige Stunden}\pwindex{Schnitzler, Arthur 15.\,5.\,1862 Wien – 21.\,10.\,1931 ebd.@\textsc{Schnitzler, Arthur} (15.\,5.\,1862 Wien – 21.\,10.\,1931 ebd.), \emph{Schriftsteller, Mediziner}!Lebendige Stunden. Vier Einakter@\strich\emph{Lebendige Stunden. Vier Einakter}|pwk} arbeitete. Siehe XXXX Auszeichnungsfehler: Dokument L03192 nicht gefunden. }}}\label{K_L03193-1}. Deine
               Standrede hat mich \strikeout{daher} überraſcht, weil mein
               letzter Brief ganz harmlos gemeint war. Aber ich mag nicht darauf erwidern. Ich habe
               keine Zeit zur Polemik; ich{ }ſchreibe lieber an dem \textsc{Feuilleton\pwindex{Goldmann, Paul 31.\,1.\,1865 Breslau – 25.\,9.\,1935 Wien@\textsc{Goldmann, Paul} (31.\,1.\,1865 Breslau – 25.\,9.\,1935 Wien), \emph{Schriftsteller, Journalist}!Berliner Theater. (»Lebendige Stunden« von Arthur Schnitzler.)@\strich\emph{Berliner Theater. (»Lebendige Stunden« von Arthur Schnitzler.)}|pwv}} über Deine Stücke\pwindex{Schnitzler, Arthur 15.\,5.\,1862 Wien – 21.\,10.\,1931 ebd.@\textsc{Schnitzler, Arthur} (15.\,5.\,1862 Wien – 21.\,10.\,1931 ebd.), \emph{Schriftsteller, Mediziner}!Lebendige Stunden. Vier Einakter@\strich\emph{Lebendige Stunden. Vier Einakter}|pwv}
               weiter. Bin ich wirklich{ }ſo koloſſal empfindlich? Ich finde, es iſt bequem, \strikeout{\textcolor{gray}{die}{ }\textcolor{gray}{×}\-\textcolor{gray}{×}\-\textcolor{gray}{×}\-\textcolor{gray}{×}\-\textcolor{gray}{×}\-\textcolor{gray}{×} an} irgendwelche
               Differenzen durch die Empfindlichkeit des Anderen zu erklären. Man erſpart{ }ſich{ }ſelbſt dadurch jedes Gefühl der Verantwortung. Aber es gäbe vielleicht auch eine
               andere Erklärung. Beiſpielsweiſe die, daß von Dir zu mir nicht Alles in Ordnung iſt –
               vielleicht{ }ſchon{ }ſeit Jahren nicht in Ordnung iſt. Außer über meine Empfindlichkeit{ }ſollteſt Du auch darüber einmal nachdenken.\pend
           
\pstart
           Du haſt gewünſcht, wir{ }ſollten grob zu einander{ }ſein. Bin ich grob genug? Aber laſſen
               wir es dabei {\pb}bewenden. Dieſe Diskuſſionen führen zu
               nichts.\pend
           
\pstart
           Ich wäre Dir{ }ſehr dankbar, wenn Du \label{K_L03193-2v}\edtext{\textsc{Trebitsch\pwindex{Trebitsch, Siegfried 22.\,12.\,1868 Wien – 3.\,6.\,1956 Zürich@\textsc{Trebitsch, Siegfried} (22.\,12.\,1868 Wien – 3.\,6.\,1956 Zürich), \emph{Schriftsteller, Übersetzer}|pw}} bewegen}{\lemma{\textnormal{\emph{Trebitsch bewegen}}}\Cendnote{\textnormal{Mussets\pwindex{Musset, Alfred de 11.\,12.\,1810 Paris – 2.\,5.\,1857 ebd.@\textsc{Musset, Alfred de} (11.\,12.\,1810 Paris – 2.\,5.\,1857 ebd.), \emph{Schriftsteller}|pwk}{ }\emph{Lorenzaccio}\pwindex{Musset, Alfred de 11.\,12.\,1810 Paris – 2.\,5.\,1857 ebd.@\textsc{Musset, Alfred de} (11.\,12.\,1810 Paris – 2.\,5.\,1857 ebd.), \emph{Schriftsteller}!Lorenzaccio. Drame romantique en cinq actes@\strich\emph{Lorenzaccio. Drame romantique en cinq actes}|pwk} wurde von Siegfried
                     Trebitsch\pwindex{Trebitsch, Siegfried 22.\,12.\,1868 Wien – 3.\,6.\,1956 Zürich@\textsc{Trebitsch, Siegfried} (22.\,12.\,1868 Wien – 3.\,6.\,1956 Zürich), \emph{Schriftsteller, Übersetzer}|pwk} nicht übersetzt.}}}\label{K_L03193-2} könnteſt, von der \textsc{Lorenzaccio\pwindex{Musset, Alfred de 11.\,12.\,1810 Paris – 2.\,5.\,1857 ebd.@\textsc{Musset, Alfred de} (11.\,12.\,1810 Paris – 2.\,5.\,1857 ebd.), \emph{Schriftsteller}!Lorenzaccio. Drame romantique en cinq actes@\strich\emph{Lorenzaccio. Drame romantique en cinq actes}|pwv}}-Überſetzung abzuſehen. Vielleicht mache ich mich \label{K_L03193-3v}\edtext{doch noch einmal}{\lemma{\textnormal{\emph{doch noch einmal}}}\Cendnote{\textnormal{Siehe XXXX Auszeichnungsfehler: Dokument L02792 nicht gefunden. Goldmann\pwindex{Goldmann, Paul 31.\,1.\,1865 Breslau – 25.\,9.\,1935 Wien@\textsc{Goldmann, Paul} (31.\,1.\,1865 Breslau – 25.\,9.\,1935 Wien), \emph{Schriftsteller, Journalist}|pwk} veröffentlichte zwar nie eine \emph{Lorenzaccio}\pwindex{Musset, Alfred de 11.\,12.\,1810 Paris – 2.\,5.\,1857 ebd.@\textsc{Musset, Alfred de} (11.\,12.\,1810 Paris – 2.\,5.\,1857 ebd.), \emph{Schriftsteller}!Lorenzaccio. Drame romantique en cinq actes@\strich\emph{Lorenzaccio. Drame romantique en cinq actes}|pwk}-Übersetzung, jedoch eine von Mussets\pwindex{Musset, Alfred de 11.\,12.\,1810 Paris – 2.\,5.\,1857 ebd.@\textsc{Musset, Alfred de} (11.\,12.\,1810 Paris – 2.\,5.\,1857 ebd.), \emph{Schriftsteller}|pwk}{ }\emph{Il ne faut jurer
                     de rien}\pwindex{Musset, Alfred de 11.\,12.\,1810 Paris – 2.\,5.\,1857 ebd.@\textsc{Musset, Alfred de} (11.\,12.\,1810 Paris – 2.\,5.\,1857 ebd.), \emph{Schriftsteller}!ne faut jurer de rien@\strich\emph{Il ne faut jurer de rien}|pwk}: Alfred de Musset\pwindex{Musset, Alfred de 11.\,12.\,1810 Paris – 2.\,5.\,1857 ebd.@\textsc{Musset, Alfred de} (11.\,12.\,1810 Paris – 2.\,5.\,1857 ebd.), \emph{Schriftsteller}|pwk}:
                        \emph{Man soll nichts verschwören. Komödie in 3
                        Akten}\pwindex{Musset, Alfred de 11.\,12.\,1810 Paris – 2.\,5.\,1857 ebd.@\textsc{Musset, Alfred de} (11.\,12.\,1810 Paris – 2.\,5.\,1857 ebd.), \emph{Schriftsteller}!Man soll nichts verschwören. Komödie in 3 Akten@\strich\emph{Man soll nichts verschwören. Komödie in 3 Akten}|pwk} [1836/48].
                     Übersetzt von Paul Goldmann\pwindex{Goldmann, Paul 31.\,1.\,1865 Breslau – 25.\,9.\,1935 Wien@\textsc{Goldmann, Paul} (31.\,1.\,1865 Breslau – 25.\,9.\,1935 Wien), \emph{Schriftsteller, Journalist}|pwk}. Frankfurt a. M.\oindex{Frankfurt am Main@\textbf{Frankfurt am Main}, \emph{Hauptstadt}|pwk}: \emph{Rütten {\kaufmannsund} Loening}\orgindex{Rütten und Loening@Rütten {\kaufmannsund}  Loening|pwk}{ }1902.}}}\label{K_L03193-3} an
               dieſe Arbeit.\pend
           
\pstart
           \textsc{Kanner\pwindex{Kanner, Heinrich 9.\,11.\,1864 Galați – 15.\,2.\,1930 Wien@\textsc{Kanner, Heinrich} (9.\,11.\,1864 Galați – 15.\,2.\,1930 Wien), \emph{Herausgeber, Publizist}|pw}}, der in \textsc{Berlin}\oindex{Berlin@\textbf{Berlin}, \emph{Hauptstadt}|pw} weilt, war bei mir. Die \label{K_L03193-4v}\edtext{Umwandlung der »Zeit\pwindex{Zeit. Wiener Wochenschrift@\emph{Die Zeit. Wiener Wochenschrift}|pw}« in ein Tagesblatt\pwindex{Zeit@\emph{Die Zeit}|pwv}}{\lemma{\textnormal{\emph{Umwandlung … Tagesblatt}}}\Cendnote{\textnormal{Siehe XXXX Auszeichnungsfehler: Dokument L03072 nicht gefunden.
               }}}\label{K_L03193-4} iſt beſchloſſene Sache.\pend
           
\pstart
           \textsc{Alice Bondy\pwindex{Ziegler, Alice 5.\,1.\,1880 Prag – Dezember 1943 Konzentrationslager Auschwitz-Birkenau@\textsc{Ziegler, Alice} (5.\,1.\,1880 Prag – Dezember 1943 Konzentrationslager Auschwitz-Birkenau)|pw}} zeigt mir ihre \label{K_L03193-5v}\edtext{Verlobung}{\lemma{\textnormal{\emph{Verlobung}}}\Cendnote{\textnormal{Ernst Ziegler\pwindex{Ziegler, Arnost 6.\,12.\,1871 Polička – 2.\,1.\,1943 Terezín@\textsc{Ziegler, Arnost} (6.\,12.\,1871 Polička – 2.\,1.\,1943 Terezín), \emph{Bankdirektor}|pwk} und Alice Bondy\pwindex{Ziegler, Alice 5.\,1.\,1880 Prag – Dezember 1943 Konzentrationslager Auschwitz-Birkenau@\textsc{Ziegler, Alice} (5.\,1.\,1880 Prag – Dezember 1943 Konzentrationslager Auschwitz-Birkenau)|pwk} heirateten am 7. 5. 1902. In den späten 1890er-Jahren hatte
                     Goldmann\pwindex{Goldmann, Paul 31.\,1.\,1865 Breslau – 25.\,9.\,1935 Wien@\textsc{Goldmann, Paul} (31.\,1.\,1865 Breslau – 25.\,9.\,1935 Wien), \emph{Schriftsteller, Journalist}|pwk} für die damals knapp unter 20-Jährige\pwindex{Ziegler, Alice 5.\,1.\,1880 Prag – Dezember 1943 Konzentrationslager Auschwitz-Birkenau@\textsc{Ziegler, Alice} (5.\,1.\,1880 Prag – Dezember 1943 Konzentrationslager Auschwitz-Birkenau)|pwkv} geschwärmt,
                     siehe XXXX Auszeichnungsfehler: Dokument L02833 nicht gefunden, XXXX Auszeichnungsfehler: Dokument L02836 nicht gefunden und XXXX Auszeichnungsfehler: Dokument L02885 nicht gefunden.}}}\label{K_L03193-5} mit
               einem \textsc{Dr. Ziegler\pwindex{Ziegler, Arnost 6.\,12.\,1871 Polička – 2.\,1.\,1943 Terezín@\textsc{Ziegler, Arnost} (6.\,12.\,1871 Polička – 2.\,1.\,1943 Terezín), \emph{Bankdirektor}|pw}} an.\pend
           
\pstart
           Es thut mir unendlich leid, daß \textsc{Olga\pwindex{Schnitzler, Olga 17.\,1.\,1882 Wien – 13.\,1.\,1970 Lugano@\textsc{Schnitzler, Olga} (17.\,1.\,1882 Wien – 13.\,1.\,1970 Lugano), \emph{Schauspielerin, Sängerin}|pw}}{ }ſich{ }ſo \label{K_L03193-6v}\edtext{plagen}{\lemma{\textnormal{\emph{plagen}}}\Cendnote{\textnormal{Er dürfte auf Komplikationen oder zumindest Sorgen in der  Schwangerschaft anspielen, siehe A. S.: \emph{Tagebuch}, 4. 1. 1902 und 8. 1. 1902.
               }}}\label{K_L03193-6} muß. Verſichere{ }ſie meiner herzlichſten Antheilnahme und grüße{ }ſie
               vielmals.\pend
           
\pstart
           Auch Du{ }ſei von Herzen gegrüßt. {\\[\baselineskip]}Dein {\\[\baselineskip]}\spacefill\mbox{Paul Goldm}\pend
           \leftskip=0em{}\selectlanguage{ngerman}\endnumbering\briefempfaengerindex{Schnitzler, Arthur@\textsc{Schnitzler, Arthur}!zzzGoldmann, Paul@\emph{von Paul Goldmann}!1902-01-162@{16. 1. [1902]}|)be}\mylabel{L03193h}  \newcommand{\dateiname}{L03193}\newcommand{\titel}{Paul Goldmann an Arthur Schnitzler, 16. 1. [1902]}\newcommand{\editorInnen}{Martin Anton Müller und Laura Untner}%% latex-leseansicht-abspann.tex
%% Abspann für die Leseansicht.
%% Der Schalter \ifkorrekturansicht ist bereits durch den Vorspann gesetzt.

%% latex-abspann.tex
%% Gemeinsamer Abspann für Korrekturansicht und Leseansicht.
%% Setzt den Schalter \ifkorrekturansicht voraus (gesetzt in den
%% einbindenden Dateien latex-korrekturansicht-abspann.tex bzw.
%% latex-leseansicht-abspann.tex).
%% ---------------------------------------------------------------

\normalsize

% Das esempio-Environment wird nur in der Leseansicht benötigt
\ifkorrekturansicht\else
\newenvironment{esempio}[3]%
{
    \vspace{1.5ex}
    \rlap{\underline{#1}}
    \par
    \setlength{\parindent}{0cm}
    \nopagebreak
    \leftskip=#2cm
    \rightskip=#3cm
}
{
    \par
}
\fi

\doendnotes{C}
\bigskip
\vfill

\clearpage

\footnotesize

\ifkorrekturansicht
  \lohead{\textsc{register}}
\fi

% theindex-Environment neu definieren ohne reledmac
\makeatletter
\renewenvironment{theindex}{%
  \ifkorrekturansicht
    \section*{\indexname}%
  \else
    \subsubsection*{Index der erwähnten Entitäten}%
  \fi
  \setlength{\parindent}{0pt}%
  \setlength{\parskip}{0pt plus 0.3pt}%
  \let\item\@idxitem
}{%
  \ifkorrekturansicht\clearpage\fi
}
\makeatother

\IfFileExists{\jobname-pw.ind}{\input{\jobname-pw.ind}}{}

% Quellenangabe nur in der Leseansicht
\ifkorrekturansicht\else
% Fallback-Definitionen, falls die .tex-Datei \titel etc. nicht gesetzt hat
\providecommand{\titel}{}
\providecommand{\editorInnen}{}
\providecommand{\dateiname}{\jobname}

\vspace{3cm}

\vfill

\footnotesize
\textsc{Quelle}: \titel. Herausgegeben von {\editorInnen}. In: \emph{Arthur Schnitzler: Briefwechsel mit Autorinnen und Autoren}.
 Digitale Edition, https://schnitzler-briefe.acdh.oeaw.ac.at/{\dateiname}.html (Stand \today)
\fi

\end{document}


