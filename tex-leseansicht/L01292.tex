%% latex-korrekturansicht-vorspann.tex
%% Vorspann für die Korrekturansicht.
%% Lädt die gemeinsame Datei latex-vorspann.tex mit gesetztem Schalter.

\newif\ifkorrekturansicht
\korrekturansichttrue

\input{../tex-inputs/latex-vorspann}


\section[Arthur Schnitzler an Hugo von Hofmannsthal, 23. 5. 1903]{L01292 Arthur Schnitzler an Hugo von Hofmannsthal, 23. 5. 1903}
\nopagebreak\mylabel{L01292v}
\rehead{ }\normalsize\beginnumbering\briefempfaengerindex{Hofmannsthal, Hugo von@\textsc{Hofmannsthal, Hugo von}!zzzSchnitzler, Arthur@\emph{von Arthur Schnitzler}!1903-05-231@{23. 5. 1903}|(be}
\toendnotes[C]{\smallbreak\pagebreak[2]}\Standort{FDH, Hs-30885,102.}
\physDesc{Brief, 1 Blatt, 3 Seiten, 521 Zeichen
\newline{}Handschrift: Bleistift, deutsche Kurrent}
\buchAbdrucke{\weitereDrucke{Hugo von Hofmannsthal, Arthur Schnitzler: \emph{Briefwechsel}. Frankfurt am Main: \emph{S. Fischer} 1964, S. 168–169.} }
\pstart
           \raggedleft{}{\pb}23/5 903.\pend
           \vspace{0.5em}
\pstart
           Was ich Ihnen heute zu ſagen vergaſs, lieber Hugo, ein Frl \textsc{Maria Luggin}\pwindex{Luggin, Marie 01.07.1867 – 11.02.1945@\textsc{Luggin, Marie} (01.07.1867 – 11.02.1945), \emph{Rezitator/Rezitatorin, Sekretär/Sekretärin, Vorleser/Vorleserin}|pw} Vorleſerin, früher bei der \textsc{Ebner Eschenbach}\pwindex{Ebner-Eschenbach, Marie von 13.09.1830 – 12.03.1916@\textsc{Ebner-Eschenbach, Marie von} (13.09.1830 – 12.03.1916), \emph{Schriftsteller/Schriftstellerin, Schriftsteller/Schriftstellerin}|pw} glaub ich, jetzt bei der Generalin \textsc{v. Hueber}\pwindex{Hueber, Henriette von 11.3.1841 – 11.4.1911@\textsc{Hueber, Henriette von} (11.3.1841 – 11.4.1911)|pw}, von ſehr ſympathiſchem Weſen, will im Herbſt in kleinem Kreiſe
                  (Saal des wiſſenſch. Club\oindex{Saal des wissenschaftlichen Clubs@\textbf{Saal des wissenschaftlichen Clubs}, \emph{Veranstaltungsgebäude (K.VSB)}|pw}{[}){]}{ }{\pb}oder ſonſt wo, ungedrucktes (oder möglichſt unbekanntes)
               von beſſeren Wien\oindex{Wien@\textbf{Wien}, \emph{A.ADM2}|pw}ern \textsc{resp}{ }Oeſterreichern\oindex{Oesterreich@\textbf{Österreich}, \emph{A.PCLI}|pw} vorleſen; bat mich, bei Ihnen
               für ſie zu reden, was ich ſehr gern thue. Ich geb ihr jedenfalls was we{\geminationn} ich was habe; ka{\geminationn} ich ihr
               in Ihrem {\pb}Namen Hoffnung machen?\pend
           
\pstart
           Herzlichſt{\\[\baselineskip]}Ihr \spacefill\mbox{A.}\pend
           \leftskip=0em{}\selectlanguage{ngerman}\endnumbering\briefempfaengerindex{Hofmannsthal, Hugo von@\textsc{Hofmannsthal, Hugo von}!zzzSchnitzler, Arthur@\emph{von Arthur Schnitzler}!1903-05-231@{23. 5. 1903}|)be}\mylabel{L01292h}  \normalsize

\doendnotes{C}
\bigskip
\vfill

\clearpage

\footnotesize

\lohead{\textsc{register}}

% Definiere theindex-Environment komplett neu ohne reledmac
\makeatletter
\renewenvironment{theindex}{%
  \section*{\indexname}%
  \setlength{\parindent}{0pt}%
  \setlength{\parskip}{0pt plus 0.3pt}%
  \let\item\@idxitem
}{%
  \clearpage
}
\makeatother

\IfFileExists{\jobname-pw.ind}{\input{\jobname-pw.ind}}{}

\end{document}

      