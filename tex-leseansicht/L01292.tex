%% latex-leseansicht-vorspann.tex
%% Vorspann für die Leseansicht.
%% Lädt die gemeinsame Datei latex-vorspann.tex mit nicht gesetztem Schalter.

\newif\ifkorrekturansicht
\korrekturansichtfalse

\input{../tex-inputs/latex-vorspann}


         
         \renewcommand{\erwaehntePersonen}{Personen: Marie von Ebner-Eschenbach, Hugo von Hofmannsthal, Henriette von Hueber, Marie Luggin}
         \renewcommand{\erwaehnteOrte}{Orte: Saal des wissenschaftlichen Clubs, Wien, Österreich}
         \renewcommand{\erwaehnteWerke}{}
               \section[Arthur Schnitzler an Hugo von Hofmannsthal, 23. 5. 1903]{ Arthur Schnitzler an Hugo von Hofmannsthal, 23. 5. 1903}\nopagebreak\mylabel{v}\rehead{ }\begin{ledgroupsized}[t]{13cm}\normalsize\beginnumbering \toendnotes[C]{\smallbreak\pagebreak[2]} \Standort{FDH, Hs-30885,102.}
\physDesc{Brief, 1 Blatt, 3 Seiten
\newline{}Handschrift: Bleistift, deutsche Kurrent}\buchAbdrucke{\weitereDrucke{Hugo von Hofmannsthal, Arthur Schnitzler: \emph{Briefwechsel}. Hg. Therese Nickl und Heinrich Schnitzler. Frankfurt am Main: \emph{S. Fischer} 1964, S. 168–169.} }\pstart
           \raggedleft{}{\pb}23/5 903.\pend
           \pstart
           Was ich Ihnen heute zu ſagen vergaſs, lieber Hugo, ein Frl \textsc{Maria Luggin}\pwindex{Luggin, Marie 01.07.1867 – 11.02.1945@\textsc{Luggin, Marie} (01.07.1867 – 11.02.1945), \emph{Rezitatorin, Sekretärin, Vorleserin}|pw} Vorleſerin, früher bei der \textsc{Ebner Eschenbach}\pwindex{Ebner-Eschenbach, Marie von 13.09.1830 – 12.03.1916@\textsc{Ebner-Eschenbach, Marie von} (13.09.1830 – 12.03.1916), \emph{Schriftstellerin}|pw} glaub ich, jetzt bei der Generalin \textsc{v. Hueber}\pwindex{Hueber, Henriette von 11.3.1841 – 11.4.1911@\textsc{Hueber, Henriette von} (11.3.1841 – 11.4.1911)|pw}, von ſehr ſympathiſchem Weſen, will im Herbſt in kleinem Kreiſe
                  (Saal des wiſſenſch. Club\oindex{Saal des wissenschaftlichen Clubs@\textbf{Saal des wissenschaftlichen Clubs}|pw}{[}){]}{ }{\pb}oder ſonſt wo, ungedrucktes (oder möglichſt unbekanntes) von beſſeren Wien\oindex{Wien@\textbf{Wien}|pw}ern \textsc{resp}{ }Oeſterreichern\oindex{Oesterreich@\textbf{Österreich}|pw} vorleſen; bat
               mich, bei Ihnen für ſie zu reden, was ich ſehr gern thue. Ich geb ihr jedenfalls was
                  we{\geminationn} ich was habe; ka{\geminationn}
               ich ihr in Ihrem {\pb}Namen Hoffnung machen?\pend
           \pstart
           Herzlichſt{\\[\baselineskip]}Ihr \spacefill\mbox{A.}\pend
           \leftskip=0em{}
         
         \endnumbering\mylabel{h}\end{ledgroupsized}  \newcommand{\dateiname}{L01292}\newcommand{\titel}{Arthur Schnitzler an Hugo von Hofmannsthal, 23. 5. 1903}\newcommand{\editorInnen}{Martin Anton Müller und Gerd-Hermann Susen}%% latex-leseansicht-abspann.tex
%% Abspann für die Leseansicht.
%% Der Schalter \ifkorrekturansicht ist bereits durch den Vorspann gesetzt.

%% latex-abspann.tex
%% Gemeinsamer Abspann für Korrekturansicht und Leseansicht.
%% Setzt den Schalter \ifkorrekturansicht voraus (gesetzt in den
%% einbindenden Dateien latex-korrekturansicht-abspann.tex bzw.
%% latex-leseansicht-abspann.tex).
%% ---------------------------------------------------------------

\normalsize

% Das esempio-Environment wird nur in der Leseansicht benötigt
\ifkorrekturansicht\else
\newenvironment{esempio}[3]%
{
    \vspace{1.5ex}
    \rlap{\underline{#1}}
    \par
    \setlength{\parindent}{0cm}
    \nopagebreak
    \leftskip=#2cm
    \rightskip=#3cm
}
{
    \par
}
\fi

\doendnotes{C}
\bigskip
\vfill

\clearpage

\footnotesize

\ifkorrekturansicht
  \lohead{\textsc{register}}
\fi

% theindex-Environment neu definieren ohne reledmac
\makeatletter
\renewenvironment{theindex}{%
  \ifkorrekturansicht
    \section*{\indexname}%
  \else
    \subsubsection*{Index der erwähnten Entitäten}%
  \fi
  \setlength{\parindent}{0pt}%
  \setlength{\parskip}{0pt plus 0.3pt}%
  \let\item\@idxitem
}{%
  \ifkorrekturansicht\clearpage\fi
}
\makeatother

\IfFileExists{\jobname-pw.ind}{\input{\jobname-pw.ind}}{}

% Quellenangabe nur in der Leseansicht
\ifkorrekturansicht\else
% Fallback-Definitionen, falls die .tex-Datei \titel etc. nicht gesetzt hat
\providecommand{\titel}{}
\providecommand{\editorInnen}{}
\providecommand{\dateiname}{\jobname}

\vspace{3cm}

\vfill

\footnotesize
\textsc{Quelle}: \titel. Herausgegeben von {\editorInnen}. In: \emph{Arthur Schnitzler: Briefwechsel mit Autorinnen und Autoren}.
 Digitale Edition, https://schnitzler-briefe.acdh.oeaw.ac.at/{\dateiname}.html (Stand \today)
\fi

\end{document}


      