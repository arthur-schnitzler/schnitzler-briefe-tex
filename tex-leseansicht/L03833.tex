%% latex-leseansicht-vorspann.tex
%% Vorspann für die Leseansicht.
%% Lädt die gemeinsame Datei latex-vorspann.tex mit nicht gesetztem Schalter.

\newif\ifkorrekturansicht
\korrekturansichtfalse

\input{../tex-inputs/latex-vorspann}


\section[Theodor Herzl an Arthur Schnitzler, 8. 12. 1893]{L03833 Theodor Herzl an Arthur Schnitzler, 8. 12. 1893}
\nopagebreak\mylabel{L03833v}
\rehead{ }\normalsize\beginnumbering\briefempfaengerindex{Schnitzler, Arthur@\textsc{Schnitzler, Arthur}!zzzHerzl, Theodor@\emph{von Theodor Herzl}!1893-12-082@{8. 12. 1893}|(be}
\toendnotes[C]{\smallbreak\pagebreak[2]}
\correspDesc{Versand  durch Theodor Herzl am 8. 12. 1893 in Paris
\newline{}Erhalt  durch Arthur Schnitzler im Zeitraum [9. 12. 1893 – 13. 12. 1893?] in Wien}\toendnotes[C]{\smallbreak}
\Standort{CUL, Schnitzler, B 39.}
\physDesc{Brief, 1 Blatt, 1 Seite, 483 Zeichen
\newline{}Handschrift: schwarze Tinte, lateinische Kurrent
\newline{}Ordnung: mit Bleistift von unbekannter Hand nummeriert: »12« }
\buchAbdrucke{\weitereDrucke{Theodor Herzl: \emph{Briefe und
                        autobiographische Notizen 1866–1895}. Bearbeitet von Johannes Wachten in Zusammenarbeit mit Chaya Harel, Daisy Tycho und Manfred Winkler. Berlin, Frankfurt am Main, Wien: \emph{Propyläen} 1983, S. 541–542 (Briefe und Tagebücher. Herausgegeben von Alex Bein, Hermann Greive, Moshe Schaerf, Julius H. Schoeps und Johannes Wachten, 1).} }\toendnotes[C]{\smallbreak}
\pstart
           {\pb}\textcolor{gray}{\textbf{NOUVELLE PRESSE LIBRE }}\orgindex{Neue Freie Presse@Neue Freie Presse|pw}\hfill \textcolor{gray}{\textbf{Paris\oindex{8, rue de Monceau@\textbf{8, rue de Monceau}, \emph{Wohngebäude}|pw},
                        le}}{ }8 Dec{ }\textcolor{gray}{\textbf{189}}3\pend
           
\pstart
           \textcolor{gray}{\textbf{D\textsuperscript{r}{ }TH. HERZL}}\pend
           
\pstart{}Lieber Freund!\pend\vspace{0.5em}
\pstart
           Sieben Wochen war ich malariakrank \introOben{}– aus \label{K_L03833-1v}\edtext{Venedig\oindex{Venedig@\textbf{Venedig}|pw}–Toulon\oindex{Toulon@\textbf{Toulon}, \emph{Hauptstadt}|pw}}{\lemma{\textnormal{\emph{Venedig–Toulon}}}\Cendnote{\textnormal{Herzl\pwindex{Herzl, Theodor 2.\,5.\,1860 Budapest – 3.\,7.\,1904 Edlach@\textsc{Herzl, Theodor} (2.\,5.\,1860 Budapest – 3.\,7.\,1904 Edlach), \emph{Schriftsteller, Journalist}|pwk} war als Frankreich\oindex{Frankreich@\textbf{Frankreich}|pwk}korrespondent für die \emph{Neue Freie Presse}\orgindex{Neue Freie Presse@Neue Freie Presse|pwk} nach Toulon\oindex{Toulon@\textbf{Toulon}, \emph{Hauptstadt}|pwk} gereist und hatte an den Festivitäten zum Empfang
                     der russischen\oindex{Russland@\textbf{Russland}|pwk} Flotte dort
                     teilgenommen. In seinem \emph{Bericht}\pwindex{Herzl, Theodor 2.\,5.\,1860 Budapest – 3.\,7.\,1904 Edlach@\textsc{Herzl, Theodor} (2.\,5.\,1860 Budapest – 3.\,7.\,1904 Edlach), \emph{Schriftsteller, Journalist}!Flottenfest in Toulon@\strich\emph{Das Flottenfest in Toulon}|pwk} darüber erwähnte er auch das venezianische\oindex{Venedig@\textbf{Venedig}|pwk} Fest mit »Tausenden von Zuschauern«, das am
                        14. 10. 1893 an den Kais veranstaltet wurde
                           (\emph{Das Flottenfest in Toulon}\pwindex{Herzl, Theodor 2.\,5.\,1860 Budapest – 3.\,7.\,1904 Edlach@\textsc{Herzl, Theodor} (2.\,5.\,1860 Budapest – 3.\,7.\,1904 Edlach), \emph{Schriftsteller, Journalist}!Flottenfest in Toulon@\strich\emph{Das Flottenfest in Toulon}|pwk}.
                        In: \emph{Neue Freie Presse}\pwindex{Neue Freie Presse@\emph{Neue Freie Presse}|pwk}, Nr. 10.471,
                           16. 10. 1893, S. 5). Vermutlich hatte
                     er sich dort angesteckt.}}}\label{K_L03833-1} mitgebracht –\introOben{}, schwere Fieberwochen.
               Heute ein zermatschgerter Reconvalescent. So fasse ich mich nothgedrungen kurz. Ich
               gratulire zur \label{K_L03833-2v}\edtext{Märchenaufführung\pwindex{Schnitzler, Arthur 15.\,5.\,1862 Wien – 21.\,10.\,1931 ebd.@\textsc{Schnitzler, Arthur} (15.\,5.\,1862 Wien – 21.\,10.\,1931 ebd.), \emph{Schriftsteller, Mediziner}!Märchen. Schauspiel in drei Aufzügen@\strich\emph{Das Märchen. Schauspiel in drei Aufzügen}|pw}}{\lemma{\textnormal{\emph{Märchenaufführung}}}\Cendnote{\textnormal{
                  Die Theateruraufführung von \emph{Das Märchen. Schauspiel in drei Aufzügen}\pwindex{Schnitzler, Arthur 15.\,5.\,1862 Wien – 21.\,10.\,1931 ebd.@\textsc{Schnitzler, Arthur} (15.\,5.\,1862 Wien – 21.\,10.\,1931 ebd.), \emph{Schriftsteller, Mediziner}!Märchen. Schauspiel in drei Aufzügen@\strich\emph{Das Märchen. Schauspiel in drei Aufzügen}|pwk} von Arthur Schnitzler\eventindex{Volkstheater@\textbf{Volkstheater}!Uraufführung von Das Märchen, 1.12.1893@Uraufführung von Das Märchen, 1.12.1893|pwk} fand am 1. 12. 1893 am \emph{Volkstheater}\orgindex{Volkstheater@Volkstheater|pwk} statt.
               }}}\label{K_L03833-2}. Die Bühne ist erobert.
               Sie werden sie nicht mehr verlassen. Nur sich treu bleiben! u. ringsherum reden
               lassen.\pend
           
\pstart
           Ich grüsse Sie herzlich {\\[\baselineskip]}Ihr aufrichtig ergebener {\\[\baselineskip]}\spacefill\mbox{Herzl}\pend
           \leftskip=0em{}
\pstart
           \noindent{}Könnten Sie nur Bahrs\pwindex{Bahr, Hermann 19.\,7.\,1863 Linz – 15.\,1.\,1934 München@\textsc{Bahr, Hermann} (19.\,7.\,1863 Linz – 15.\,1.\,1934 München), \emph{Schriftsteller, Kritiker}|pw}\label{K_L03833-3v}\edtext{{ }Artikelserie über Jung-Oesterreich\pwindex{Bahr, Hermann 19.\,7.\,1863 Linz – 15.\,1.\,1934 München@\textsc{Bahr, Hermann} (19.\,7.\,1863 Linz – 15.\,1.\,1934 München), \emph{Schriftsteller, Kritiker}!junge Österreich@\strich\emph{Das junge Österreich}|pwv}}{\lemma{\textnormal{\emph{Artikelserie über Jung-Oesterreich}}}\Cendnote{\textnormal{Hermann Bahr\pwindex{Bahr, Hermann 19.\,7.\,1863 Linz – 15.\,1.\,1934 München@\textsc{Bahr, Hermann} (19.\,7.\,1863 Linz – 15.\,1.\,1934 München), \emph{Schriftsteller, Kritiker}|pwk}: \emph{Das
                           junge Österreich}\pwindex{Bahr, Hermann 19.\,7.\,1863 Linz – 15.\,1.\,1934 München@\textsc{Bahr, Hermann} (19.\,7.\,1863 Linz – 15.\,1.\,1934 München), \emph{Schriftsteller, Kritiker}!junge Österreich@\strich\emph{Das junge Österreich}|pwk}. In: \emph{Deutsche
                           Zeitung}\pwindex{Deutsche Zeitung@\emph{Deutsche Zeitung}|pwk}, Jg. 23, Nr. 7806,
                           20. 9. 1893, Morgen-Ausgabe, S. 1–2;
                        Nr. 7813, 27. 9. 1893, Morgen-Ausgabe,
                        S. 1–3; Nr. 7823, 7. 10. 1893,
                        Morgen-Ausgabe, S. 1–3.}}}\label{K_L03833-3} (Deutsche Zeitung\pwindex{Deutsche Zeitung@\emph{Deutsche Zeitung}|pw}) verschaffen? Bitte!\pend
           \selectlanguage{ngerman}\endnumbering\briefempfaengerindex{Schnitzler, Arthur@\textsc{Schnitzler, Arthur}!zzzHerzl, Theodor@\emph{von Theodor Herzl}!1893-12-082@{8. 12. 1893}|)be}\mylabel{L03833h}
\begin{anhang}
\end{anhang}\newcommand{\dateiname}{L03833}\newcommand{\titel}{Theodor Herzl an Arthur Schnitzler, 8. 12. 1893}\newcommand{\editorInnen}{Selma Jahnke und Martin Anton Müller}%% latex-leseansicht-abspann.tex
%% Abspann für die Leseansicht.
%% Der Schalter \ifkorrekturansicht ist bereits durch den Vorspann gesetzt.

%% latex-abspann.tex
%% Gemeinsamer Abspann für Korrekturansicht und Leseansicht.
%% Setzt den Schalter \ifkorrekturansicht voraus (gesetzt in den
%% einbindenden Dateien latex-korrekturansicht-abspann.tex bzw.
%% latex-leseansicht-abspann.tex).
%% ---------------------------------------------------------------

\normalsize

% Das esempio-Environment wird nur in der Leseansicht benötigt
\ifkorrekturansicht\else
\newenvironment{esempio}[3]%
{
    \vspace{1.5ex}
    \rlap{\underline{#1}}
    \par
    \setlength{\parindent}{0cm}
    \nopagebreak
    \leftskip=#2cm
    \rightskip=#3cm
}
{
    \par
}
\fi

\doendnotes{C}
\bigskip
\vfill

\clearpage

\footnotesize

\ifkorrekturansicht
  \lohead{\textsc{register}}
\fi

% theindex-Environment neu definieren ohne reledmac
\makeatletter
\renewenvironment{theindex}{%
  \ifkorrekturansicht
    \section*{\indexname}%
  \else
    \subsubsection*{Index der erwähnten Entitäten}%
  \fi
  \setlength{\parindent}{0pt}%
  \setlength{\parskip}{0pt plus 0.3pt}%
  \let\item\@idxitem
}{%
  \ifkorrekturansicht\clearpage\fi
}
\makeatother

\IfFileExists{\jobname-pw.ind}{\input{\jobname-pw.ind}}{}

% Quellenangabe nur in der Leseansicht
\ifkorrekturansicht\else
% Fallback-Definitionen, falls die .tex-Datei \titel etc. nicht gesetzt hat
\providecommand{\titel}{}
\providecommand{\editorInnen}{}
\providecommand{\dateiname}{\jobname}

\vspace{3cm}

\vfill

\footnotesize
\textsc{Quelle}: \titel. Herausgegeben von {\editorInnen}. In: \emph{Arthur Schnitzler: Briefwechsel mit Autorinnen und Autoren}.
 Digitale Edition, https://schnitzler-briefe.acdh.oeaw.ac.at/{\dateiname}.html (Stand \today)
\fi

\end{document}


