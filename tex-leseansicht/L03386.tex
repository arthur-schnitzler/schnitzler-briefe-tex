%% latex-leseansicht-vorspann.tex
%% Vorspann für die Leseansicht.
%% Lädt die gemeinsame Datei latex-vorspann.tex mit nicht gesetztem Schalter.

\newif\ifkorrekturansicht
\korrekturansichtfalse

\input{../tex-inputs/latex-vorspann}


         
         \renewcommand{\erwaehntePersonen}{Personen:  ?? [Partner von Theodore Rottenberg, Ende 1902/Anfang 1903], Paul Goldmann, Fedor Mamroth, Emilio Rizzi, Josef Rosengart, Theodore Rottenberg, Ludwig Rottenberg, Olga Schnitzler}
         \renewcommand{\erwaehnteOrte}{Orte: Hotel Centrale Rovereto, Lago di Garda, Rovereto, Salzburg, Venedig, Wien}
         \renewcommand{\erwaehnteWerke}{}
               \section[ Paul Goldmann an Arthur Schnitzler, 7. 9. 1903]{ Paul Goldmann an Arthur Schnitzler, 7. 9. 1903}\nopagebreak\mylabel{v}\rehead{ }\begin{ledgroupsized}[t]{13cm}\normalsize\beginnumbering\briefempfaengerindex{Schnitzler, Arthur@\textsc{Schnitzler, Arthur}!zzzGoldmann, Paul@\emph{von Paul Goldmann}!1903-09-071@{7. 9. 1903}|(be} \toendnotes[C]{\smallbreak\pagebreak[2]} \Standort{DLA, A:Schnitzler, HS.NZ85.1.3173.}
\physDesc{Brief, 1 Blatt, 2 Seiten, 1198 Zeichen
\newline{}Handschrift: schwarze Tinte, deutsche Kurrent}\toendnotes[C]{\smallbreak}\pstart
           \noindent{}\raggedleft{}{\pb}\textcolor{gray}{\textbf{Rovereto\oindex{Rovereto@\textbf{Rovereto}|pw}{ }\begin{otherlanguage}{italian}li\end{otherlanguage}}}{ }7. September \textcolor{gray}{\textbf{190}}3.\pend
           \pstart
           \noindent{}\raggedleft{}\textcolor{gray}{\textbf{\textbf{Hôtel Centrale\oindex{Hotel Centrale Rovereto@\textbf{Hotel Centrale Rovereto}|pw}}}}\pend
           \pstart
           \noindent{}\raggedleft{}\textcolor{gray}{\textbf{\textbf{E. RIZZI\pwindex{Rizzi, Emilio @\textsc{Rizzi, Emilio}, \emph{Hotelbesitzer}|pw}} – Rovereto\oindex{Rovereto@\textbf{Rovereto}|pw}}}\pend
           \pstart\center{}Mein lieber Freund,\pend\pstart
           Wenn Du am \label{K_L03386-1v}\edtext{15. September{ }\textsc{Wien\oindex{Wien@\textbf{Wien}|pw}} verlaſſen}{\lemma{\textnormal{\emph{15. September Wien verlaſſen}}}\Cendnote{\textnormal{Schnitzler\pwindex{Schnitzler, Arthur 15.05.1862 – 21.10.1931@\textsc{Schnitzler, Arthur} (15.05.1862 – 21.10.1931), \emph{Schriftsteller, Mediziner}|pwk} blieb den September über in Wien\oindex{Wien@\textbf{Wien}|pwk}, dürfte aber bis zum 17. [9. 1903?] geplant haben,
                  gemeinsam mit seiner frisch angetrauten Ehefrau Olga\pwindex{Schnitzler, Olga 17.01.1882 – 13.01.1970@\textsc{Schnitzler, Olga} (17.01.1882 – 13.01.1970), \emph{Schauspielerin, Sängerin}|pwk} eine Reise nach Salzburg\oindex{Salzburg@\textbf{Salzburg}|pwk} oder an
                  den Gardasee\oindex{Lago di Garda@\textbf{Lago di Garda}|pwk} zu unternehmen.
                     Vgl. \emph{Der Briefwechsel Arthur Schnitzler – Otto
                        Brahm}. Vollständige Ausgabe. Herausgegeben, eingeleitet und
                     erläutert von Oskar Seidlin. Tübingen: \emph{Niemeyer}{ }1975, S. 148.}}}\label{K_L03386-1h} willſt, würde ich wohl kaum die Freude
               haben, Dich auf meiner Rückreiſe zu ſehen. Meine Freundin\pwindex{Rottenberg, Theodore 1875-09-07 – 1945-04-05@\textsc{Rottenberg, Theodore} (1875-09-07 – 1945-04-05)|pwv} iſt vor einigen Tagen heimgefahren. Die Briefe des
                  \label{K_L03386-2v}\edtext{Mann\pwindex{Rottenberg, Ludwig 11.10.1864 – 6.5.1932@\textsc{Rottenberg, Ludwig} (11.10.1864 – 6.5.1932), \emph{Kapellmeister}|pwv}}{\lemma{\textnormal{\emph{Mann}}}\Cendnote{\textnormal{Theodore Rottenbergs\pwindex{Rottenberg, Theodore 1875-09-07 – 1945-04-05@\textsc{Rottenberg, Theodore} (1875-09-07 – 1945-04-05)|pwk} Ehemann Ludwig Rottenberg\pwindex{Rottenberg, Ludwig 11.10.1864 – 6.5.1932@\textsc{Rottenberg, Ludwig} (11.10.1864 – 6.5.1932), \emph{Kapellmeister}|pwk}}}}\label{K_L03386-2h}es wurden drohend und ſchienen eine Kataſtrophe anzukündigen. Was nach der
               Heimkehr der armen Frau\pwindex{Rottenberg, Theodore 1875-09-07 – 1945-04-05@\textsc{Rottenberg, Theodore} (1875-09-07 – 1945-04-05)|pwv}
               geſchehen iſt, weiß ich noch nicht. Auch auf meiner Seite gibt es \strikeout{ung\textcolor{gray}{e}} unvorhergeſehene Complikationen. Ich erhielt einen Brief meines Schwagers\pwindex{Rosengart, Josef 1860-02-08 – 1927-08-04@\textsc{Rosengart, Josef} (1860-02-08 – 1927-08-04), \emph{Arzt}|pwv}, der beſagt, dieſe
                  Frau\pwindex{Rottenberg, Theodore 1875-09-07 – 1945-04-05@\textsc{Rottenberg, Theodore} (1875-09-07 – 1945-04-05)|pwv} ſei nach den \label{K_L03386-3v}\edtext{Ereigniſſen dieſes Winters}{\lemma{\textnormal{\emph{Ereigniſſen … Winters}}}\Cendnote{\textnormal{Ende 1902 bis Anfang 1903 waren
                     Goldmann\pwindex{Goldmann, Paul 31.01.1865 – 25.09.1935@\textsc{Goldmann, Paul} (31.01.1865 – 25.09.1935), \emph{Schriftsteller, Journalist}|pwk} und Rottenberg\pwindex{Rottenberg, Theodore 1875-09-07 – 1945-04-05@\textsc{Rottenberg, Theodore} (1875-09-07 – 1945-04-05)|pwk} getrennt gewesen und sie war mit einem anderen Mann\pwindex{?? [Partner von Theodore Rottenberg, Ende 1902/Anfang 1903] @\textsc{?? [Partner von Theodore Rottenberg, Ende 1902/Anfang 1903]}|pwkv} eine Beziehung eingegangen.
                     Vgl. Paul Goldmann an Arthur Schnitzler, 28. 12. [1902], 17. 2. [1903] und 14. 11. [1903].}}}\label{K_L03386-3h} nicht
               mehr eine Frau, die man heirathet, und der mich vor die Wahl zwiſchen einer Heirath
               und einem Bruch mit meinem Schwager\pwindex{Rosengart, Josef 1860-02-08 – 1927-08-04@\textsc{Rosengart, Josef} (1860-02-08 – 1927-08-04), \emph{Arzt}|pwv} ſtellt{\dotstwo} Mein Onkel\pwindex{Mamroth, Fedor 21.02.1851 – 25.06.1907@\textsc{Mamroth, Fedor} (21.02.1851 – 25.06.1907), \emph{Journalist, Kritiker}|pwv}, den ich unterwegs getroffen, ſpricht
               zu mir\strikeout{, \textcolor{gray}{dem}} in dem milden und mitleidigen Tone, in dem man zu Jemandem ſpricht, der im
               Begriff iſt, ſich in ein großes Unheil zu ſtürzen. Ich weiß in dieſem {\pb}Widerſtreit der Empfindungen wieder nicht aus noch
               ein.\pend
           \pstart
           Heut fahre ich ein paar Tage nach Venedig\oindex{Venedig@\textbf{Venedig}|pw}. Vor Montag bin ich kaum
               in \label{K_L03386-4v}\edtext{Wien\oindex{Wien@\textbf{Wien}|pw}}{\lemma{\textnormal{\emph{Wien}}}\Cendnote{\textnormal{Goldmann\pwindex{Goldmann, Paul 31.01.1865 – 25.09.1935@\textsc{Goldmann, Paul} (31.01.1865 – 25.09.1935), \emph{Schriftsteller, Journalist}|pwk} war spätestens am 17. [9. 1903?] in Wien\oindex{Wien@\textbf{Wien}|pwk}. Er und Schnitzler\pwindex{Schnitzler, Arthur 15.05.1862 – 21.10.1931@\textsc{Schnitzler, Arthur} (15.05.1862 – 21.10.1931), \emph{Schriftsteller, Mediziner}|pwk} sahen sich am 18. 9. 1903, 20. 9. 1903 und 21. 9. 1903.}}}\label{K_L03386-4h}. Natürlich wirſt Du Dich in
               Deinen Reiſedispoſitionen durch mich keineswegs ſtören laſſen. Wenn Du mir etwas
               ſchreiben willſt: \textsc{Venedig\oindex{Venedig@\textbf{Venedig}|pw}, Poste restante}.\pend
           \pstart
           Ich grüße Dich und Deine Frau\pwindex{Schnitzler, Olga 17.01.1882 – 13.01.1970@\textsc{Schnitzler, Olga} (17.01.1882 – 13.01.1970), \emph{Schauspielerin, Sängerin}|pwv} auf das Herzlichſte. {\\[\baselineskip]}Dein treuer {\\[\baselineskip]}\spacefill\mbox{Paul Goldmn}\pend
           \leftskip=0em{}
         
         \endnumbering\mylabel{h}\end{ledgroupsized}  \newcommand{\dateiname}{L03386}\newcommand{\titel}{Paul Goldmann an Arthur Schnitzler, 7. 9. 1903}\newcommand{\editorInnen}{Martin Anton Müller und Laura Untner}%% latex-leseansicht-abspann.tex
%% Abspann für die Leseansicht.
%% Der Schalter \ifkorrekturansicht ist bereits durch den Vorspann gesetzt.

%% latex-abspann.tex
%% Gemeinsamer Abspann für Korrekturansicht und Leseansicht.
%% Setzt den Schalter \ifkorrekturansicht voraus (gesetzt in den
%% einbindenden Dateien latex-korrekturansicht-abspann.tex bzw.
%% latex-leseansicht-abspann.tex).
%% ---------------------------------------------------------------

\normalsize

% Das esempio-Environment wird nur in der Leseansicht benötigt
\ifkorrekturansicht\else
\newenvironment{esempio}[3]%
{
    \vspace{1.5ex}
    \rlap{\underline{#1}}
    \par
    \setlength{\parindent}{0cm}
    \nopagebreak
    \leftskip=#2cm
    \rightskip=#3cm
}
{
    \par
}
\fi

\doendnotes{C}
\bigskip
\vfill

\clearpage

\footnotesize

\ifkorrekturansicht
  \lohead{\textsc{register}}
\fi

% theindex-Environment neu definieren ohne reledmac
\makeatletter
\renewenvironment{theindex}{%
  \ifkorrekturansicht
    \section*{\indexname}%
  \else
    \subsubsection*{Index der erwähnten Entitäten}%
  \fi
  \setlength{\parindent}{0pt}%
  \setlength{\parskip}{0pt plus 0.3pt}%
  \let\item\@idxitem
}{%
  \ifkorrekturansicht\clearpage\fi
}
\makeatother

\IfFileExists{\jobname-pw.ind}{\input{\jobname-pw.ind}}{}

% Quellenangabe nur in der Leseansicht
\ifkorrekturansicht\else
% Fallback-Definitionen, falls die .tex-Datei \titel etc. nicht gesetzt hat
\providecommand{\titel}{}
\providecommand{\editorInnen}{}
\providecommand{\dateiname}{\jobname}

\vspace{3cm}

\vfill

\footnotesize
\textsc{Quelle}: \titel. Herausgegeben von {\editorInnen}. In: \emph{Arthur Schnitzler: Briefwechsel mit Autorinnen und Autoren}.
 Digitale Edition, https://schnitzler-briefe.acdh.oeaw.ac.at/{\dateiname}.html (Stand \today)
\fi

\end{document}


      