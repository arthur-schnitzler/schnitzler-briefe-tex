%% latex-leseansicht-vorspann.tex
%% Vorspann für die Leseansicht.
%% Lädt die gemeinsame Datei latex-vorspann.tex mit nicht gesetztem Schalter.

\newif\ifkorrekturansicht
\korrekturansichtfalse

\input{../tex-inputs/latex-vorspann}


\section[ Paul Goldmann an Arthur Schnitzler, 7. 9. 1903]{L03386 Paul Goldmann an Arthur Schnitzler,  7. 9. 1903}
\nopagebreak\mylabel{L03386v}
\rehead{ }\normalsize\beginnumbering\briefempfaengerindex{Schnitzler, Arthur@\textsc{Schnitzler, Arthur}!zzzGoldmann, Paul@\emph{von Paul Goldmann}!1903-09-071@{7. 9. 1903}|(be}
\toendnotes[C]{\smallbreak\pagebreak[2]}
\correspDesc{Versand  durch Paul Goldmann am 7. 9. 1903 in Rovereto
\newline{}Erhalt  durch Arthur Schnitzler im Zeitraum [8. 9. 1903
                  – 12. 9. 1903?] in Wien}\toendnotes[C]{\smallbreak}
\Standort{DLA, A:Schnitzler, HS.NZ85.1.3173.}
\physDesc{Brief, 1 Blatt, 2 Seiten, 1198 Zeichen
\newline{}Handschrift: schwarze Tinte, deutsche Kurrent}\toendnotes[C]{\smallbreak}
\pstart
           \raggedleft{}{\pb}\textcolor{gray}{\textbf{Rovereto\oindex{Rovereto@\textbf{Rovereto}|pw}{ }\begin{otherlanguage}{italian}li\end{otherlanguage}}}{ }7. September \textcolor{gray}{\textbf{190}}3.\pend
           
\pstart
           \raggedleft{}\textcolor{gray}{\textbf{\textbf{Hôtel Centrale\oindex{Hotel Centrale Rovereto@\textbf{Hotel Centrale Rovereto}, \emph{Hotel}|pw}}}}\pend
           
\pstart
           \raggedleft{}\textcolor{gray}{\textbf{\textbf{E. RIZZI\pwindex{Rizzi, Emilio @\textsc{Rizzi, Emilio}, \emph{Hotelbesitzer}|pw}} – Rovereto\oindex{Rovereto@\textbf{Rovereto}|pw}}}\pend
           
\pstart\center{}Mein lieber Freund,\pend\vspace{0.5em}
\pstart
           Wenn Du am \label{K_L03386-1v}\edtext{15. September{ }\textsc{Wien\oindex{Wien@\textbf{Wien}, \emph{Verwaltungsgebiet}|pw}} verlaſſen}{\lemma{\textnormal{\emph{15. September Wien verlassen}}}\Cendnote{\textnormal{Schnitzler blieb den September über in Wien\oindex{Wien@\textbf{Wien}, \emph{Verwaltungsgebiet}|pwk}, dürfte aber bis zum XXXX Auszeichnungsfehler: Dokument L03387 nicht gefunden geplant haben,
                  gemeinsam mit seiner frisch angetrauten Ehefrau Olga\pwindex{Schnitzler, Olga 17.\,1.\,1882 Wien – 13.\,1.\,1970 Lugano@\textsc{Schnitzler, Olga} (17.\,1.\,1882 Wien – 13.\,1.\,1970 Lugano), \emph{Schauspielerin, Sängerin}|pwk} eine Reise nach Salzburg\oindex{Salzburg@\textbf{Salzburg}, \emph{Verwaltungsgebiet}|pwk} oder an
                  den Gardasee\oindex{Lago di Garda@\textbf{Lago di Garda}, \emph{See}|pwk} zu unternehmen.
                     Vgl. \emph{Der Briefwechsel Arthur Schnitzler – Otto
                        Brahm}. Vollständige Ausgabe. Herausgegeben, eingeleitet und
                     erläutert von Oskar Seidlin. Tübingen: \emph{Niemeyer}{ }1975, S. 148.}}}\label{K_L03386-1} willſt, würde ich wohl kaum die Freude
               haben, Dich auf meiner Rückreiſe zu{ }ſehen. Meine Freundin\pwindex{Rottenberg, Theodore 7.\,9.\,1875 – 5.\,4.\,1945 Limburg an der Lahn@\textsc{Rottenberg, Theodore} (7.\,9.\,1875 – 5.\,4.\,1945 Limburg an der Lahn)|pwv} iſt vor einigen Tagen heimgefahren. Die Briefe des
                  \label{K_L03386-2v}\edtext{Mann\pwindex{Rottenberg, Ludwig 11.\,10.\,1864 Czernowitz – 6.\,5.\,1932 Frankfurt am Main@\textsc{Rottenberg, Ludwig} (11.\,10.\,1864 Czernowitz – 6.\,5.\,1932 Frankfurt am Main), \emph{Kapellmeister}|pwv}}{\lemma{\textnormal{\emph{Mann}}}\Cendnote{\textnormal{Theodore Rottenbergs\pwindex{Rottenberg, Theodore 7.\,9.\,1875 – 5.\,4.\,1945 Limburg an der Lahn@\textsc{Rottenberg, Theodore} (7.\,9.\,1875 – 5.\,4.\,1945 Limburg an der Lahn)|pwk} Ehemann Ludwig Rottenberg\pwindex{Rottenberg, Ludwig 11.\,10.\,1864 Czernowitz – 6.\,5.\,1932 Frankfurt am Main@\textsc{Rottenberg, Ludwig} (11.\,10.\,1864 Czernowitz – 6.\,5.\,1932 Frankfurt am Main), \emph{Kapellmeister}|pwk}}}}\label{K_L03386-2}es wurden drohend und{ }ſchienen eine Kataſtrophe anzukündigen. Was nach der
               Heimkehr der armen Frau\pwindex{Rottenberg, Theodore 7.\,9.\,1875 – 5.\,4.\,1945 Limburg an der Lahn@\textsc{Rottenberg, Theodore} (7.\,9.\,1875 – 5.\,4.\,1945 Limburg an der Lahn)|pwv}
               geſchehen iſt, weiß ich noch nicht. Auch auf meiner Seite gibt es \strikeout{ung\textcolor{gray}{e}} unvorhergeſehene Complikationen. Ich erhielt einen Brief meines Schwagers\pwindex{Rosengart, Josef 8.\,2.\,1860 Laupheim – 4.\,8.\,1927 Frankfurt am Main@\textsc{Rosengart, Josef} (8.\,2.\,1860 Laupheim – 4.\,8.\,1927 Frankfurt am Main), \emph{Arzt}|pwv}, der beſagt, dieſe
                  Frau\pwindex{Rottenberg, Theodore 7.\,9.\,1875 – 5.\,4.\,1945 Limburg an der Lahn@\textsc{Rottenberg, Theodore} (7.\,9.\,1875 – 5.\,4.\,1945 Limburg an der Lahn)|pwv}{ }ſei nach den \label{K_L03386-3v}\edtext{Ereigniſſen dieſes Winters}{\lemma{\textnormal{\emph{Ereignissen … Winters}}}\Cendnote{\textnormal{Ende 1902 bis Anfang 1903 waren
                     Goldmann\pwindex{Goldmann, Paul 31.\,1.\,1865 Breslau – 25.\,9.\,1935 Wien@\textsc{Goldmann, Paul} (31.\,1.\,1865 Breslau – 25.\,9.\,1935 Wien), \emph{Schriftsteller, Journalist}|pwk} und Rottenberg\pwindex{Rottenberg, Theodore 7.\,9.\,1875 – 5.\,4.\,1945 Limburg an der Lahn@\textsc{Rottenberg, Theodore} (7.\,9.\,1875 – 5.\,4.\,1945 Limburg an der Lahn)|pwk} getrennt gewesen und sie war mit einem anderen Mann\pwindex{?? [Partner von Theodore Rottenberg, Ende 1902/Anfang 1903] @\textsc{?? [Partner von Theodore Rottenberg, Ende 1902/Anfang 1903]}|pwkv} eine Beziehung eingegangen.
                     Vgl. XXXX Auszeichnungsfehler: Dokument L03231 nicht gefunden, XXXX Auszeichnungsfehler: Dokument L03363 nicht gefunden und XXXX Auszeichnungsfehler: Dokument L03388 nicht gefunden.}}}\label{K_L03386-3} nicht
               mehr eine Frau, die man heirathet, und der mich vor die Wahl zwiſchen einer Heirath
               und einem Bruch mit meinem Schwager\pwindex{Rosengart, Josef 8.\,2.\,1860 Laupheim – 4.\,8.\,1927 Frankfurt am Main@\textsc{Rosengart, Josef} (8.\,2.\,1860 Laupheim – 4.\,8.\,1927 Frankfurt am Main), \emph{Arzt}|pwv}{ }ſtellt{\dotstwo} Mein Onkel\pwindex{Mamroth, Fedor 21.\,2.\,1851 Breslau – 25.\,6.\,1907 Frankfurt am Main@\textsc{Mamroth, Fedor} (21.\,2.\,1851 Breslau – 25.\,6.\,1907 Frankfurt am Main), \emph{Journalist, Kritiker}|pwv}, den ich unterwegs getroffen,{ }ſpricht
               zu mir\strikeout{, \textcolor{gray}{dem}} in dem milden und mitleidigen Tone, in dem man zu Jemandem{ }ſpricht, der im
               Begriff iſt,{ }ſich in ein großes Unheil zu{ }ſtürzen. Ich weiß in dieſem {\pb}Widerſtreit der Empfindungen wieder nicht aus noch
               ein.\pend
           
\pstart
           Heut fahre ich ein paar Tage nach Venedig\oindex{Venedig@\textbf{Venedig}|pw}. Vor Montag bin ich kaum
               in \label{K_L03386-4v}\edtext{Wien\oindex{Wien@\textbf{Wien}, \emph{Verwaltungsgebiet}|pw}}{\lemma{\textnormal{\emph{Wien}}}\Cendnote{\textnormal{Goldmann\pwindex{Goldmann, Paul 31.\,1.\,1865 Breslau – 25.\,9.\,1935 Wien@\textsc{Goldmann, Paul} (31.\,1.\,1865 Breslau – 25.\,9.\,1935 Wien), \emph{Schriftsteller, Journalist}|pwk} war spätestens am XXXX Auszeichnungsfehler: Dokument L03387 nicht gefunden in Wien\oindex{Wien@\textbf{Wien}, \emph{Verwaltungsgebiet}|pwk}. Er und Schnitzler sahen sich am 18. 9. 1903, 20. 9. 1903 und 21. 9. 1903.}}}\label{K_L03386-4}. Natürlich wirſt Du Dich in
               Deinen Reiſedispoſitionen durch mich keineswegs{ }ſtören laſſen. Wenn Du mir etwas{ }ſchreiben willſt: \textsc{Venedig\oindex{Venedig@\textbf{Venedig}|pw}, Poste restante}.\pend
           
\pstart
           Ich grüße Dich und Deine Frau\pwindex{Schnitzler, Olga 17.\,1.\,1882 Wien – 13.\,1.\,1970 Lugano@\textsc{Schnitzler, Olga} (17.\,1.\,1882 Wien – 13.\,1.\,1970 Lugano), \emph{Schauspielerin, Sängerin}|pwv} auf das Herzlichſte. {\\[\baselineskip]}Dein treuer {\\[\baselineskip]}\spacefill\mbox{Paul Goldmn}\pend
           \leftskip=0em{}\selectlanguage{ngerman}\endnumbering\briefempfaengerindex{Schnitzler, Arthur@\textsc{Schnitzler, Arthur}!zzzGoldmann, Paul@\emph{von Paul Goldmann}!1903-09-071@{7. 9. 1903}|)be}\mylabel{L03386h}  \newcommand{\dateiname}{L03386}\newcommand{\titel}{Paul Goldmann an Arthur Schnitzler, 7. 9. 1903}\newcommand{\editorInnen}{Martin Anton Müller und Laura Untner}%% latex-leseansicht-abspann.tex
%% Abspann für die Leseansicht.
%% Der Schalter \ifkorrekturansicht ist bereits durch den Vorspann gesetzt.

%% latex-abspann.tex
%% Gemeinsamer Abspann für Korrekturansicht und Leseansicht.
%% Setzt den Schalter \ifkorrekturansicht voraus (gesetzt in den
%% einbindenden Dateien latex-korrekturansicht-abspann.tex bzw.
%% latex-leseansicht-abspann.tex).
%% ---------------------------------------------------------------

\normalsize

% Das esempio-Environment wird nur in der Leseansicht benötigt
\ifkorrekturansicht\else
\newenvironment{esempio}[3]%
{
    \vspace{1.5ex}
    \rlap{\underline{#1}}
    \par
    \setlength{\parindent}{0cm}
    \nopagebreak
    \leftskip=#2cm
    \rightskip=#3cm
}
{
    \par
}
\fi

\doendnotes{C}
\bigskip
\vfill

\clearpage

\footnotesize

\ifkorrekturansicht
  \lohead{\textsc{register}}
\fi

% theindex-Environment neu definieren ohne reledmac
\makeatletter
\renewenvironment{theindex}{%
  \ifkorrekturansicht
    \section*{\indexname}%
  \else
    \subsubsection*{Index der erwähnten Entitäten}%
  \fi
  \setlength{\parindent}{0pt}%
  \setlength{\parskip}{0pt plus 0.3pt}%
  \let\item\@idxitem
}{%
  \ifkorrekturansicht\clearpage\fi
}
\makeatother

\IfFileExists{\jobname-pw.ind}{\input{\jobname-pw.ind}}{}

% Quellenangabe nur in der Leseansicht
\ifkorrekturansicht\else
% Fallback-Definitionen, falls die .tex-Datei \titel etc. nicht gesetzt hat
\providecommand{\titel}{}
\providecommand{\editorInnen}{}
\providecommand{\dateiname}{\jobname}

\vspace{3cm}

\vfill

\footnotesize
\textsc{Quelle}: \titel. Herausgegeben von {\editorInnen}. In: \emph{Arthur Schnitzler: Briefwechsel mit Autorinnen und Autoren}.
 Digitale Edition, https://schnitzler-briefe.acdh.oeaw.ac.at/{\dateiname}.html (Stand \today)
\fi

\end{document}


