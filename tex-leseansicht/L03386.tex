%% latex-leseansicht-vorspann.tex
%% Vorspann für die Leseansicht.
%% Lädt die gemeinsame Datei latex-vorspann.tex mit nicht gesetztem Schalter.

\newif\ifkorrekturansicht
\korrekturansichtfalse

\input{../tex-inputs/latex-vorspann}

\begin{center}
            \textcolor{red}{ENTWURF, NICHT FERTIG KORRIGIERT}
                      \end{center}
            
         
         \renewcommand{\erwaehntePersonen}{Personen:  ?? [Partner von Theodore Rottenberg, Ende 1902/Anfang 1903], Fedor Mamroth, Emilio Rizzi, Josef Rosengart, Theodore Rottenberg, Ludwig Rottenberg, Olga Schnitzler}
         \renewcommand{\erwaehnteOrte}{Orte: Hotel Centrale Rovereto, Rovereto, Venedig, Wien}
         \renewcommand{\erwaehnteWerke}{}
               \section[ Paul Goldmann an Arthur Schnitzler, 7. 9. 1903]{ Paul Goldmann an Arthur Schnitzler, 7. 9. 1903}\nopagebreak\mylabel{v}\rehead{ }\begin{ledgroupsized}[t]{13cm}\normalsize\beginnumbering \toendnotes[C]{\smallbreak\pagebreak[2]} \Standort{DLA, A:Schnitzler, HS.NZ85.1.3173.}
\physDesc{Brief, 1 Blatt, 2 Seiten
\newline{}Handschrift: schwarze Tinte, deutsche Kurrent}\toendnotes[C]{\smallbreak}\pstart
           \noindent{}\raggedleft{}{\pb}\textcolor{gray}{\textbf{Rovereto\oindex{Rovereto@\textbf{Rovereto}|pw}{ }\begin{otherlanguage}{italian}li\end{otherlanguage}}}{ }7. September \textcolor{gray}{\textbf{190}}3.\pend
           \pstart
           \noindent{}\raggedleft{}\textcolor{gray}{\textbf{\textbf{Hôtel Centrale\oindex{Hotel Centrale Rovereto@\textbf{Hotel Centrale Rovereto}|pw}}}}\pend
           \pstart
           \noindent{}\raggedleft{}\textcolor{gray}{\textbf{\textbf{E. RIZZI\pwindex{Rizzi, Emilio @\textsc{Rizzi, Emilio}, \emph{Hotelbesitzer}|pw}} – ROVERETO\oindex{Rovereto@\textbf{Rovereto}|pw}}}\pend
           \pstart\center{}Mein lieber Freund,\pend\pstart
           Wenn Du am 15. SeptemberWien\oindex{Wien@\textbf{Wien}|pw} verlaſſen willſt, würde ich wohl kaum die
               Freude haben, Dich auf meiner Rückreiſe zu ſehen. Meine Freundin\pwindex{Rottenberg, Theodore 1875-09-07 – 1945-04-05@\textsc{Rottenberg, Theodore} (1875-09-07 – 1945-04-05)|pwv} iſt vor einigen Tagen
               heimgefahren. Die Briefe des \label{K_L03386-1v}\edtext{Mann\pwindex{Rottenberg, Ludwig 11.10.1864 – 6.5.1932@\textsc{Rottenberg, Ludwig} (11.10.1864 – 6.5.1932), \emph{Kapellmeister}|pwv}}{\lemma{\textnormal{\emph{Mann}}}\Cendnote{\textnormal{Theodore Rottenberg\pwindex{Rottenberg, Theodore 1875-09-07 – 1945-04-05@\textsc{Rottenberg, Theodore} (1875-09-07 – 1945-04-05)|pwk}, mit der Goldmann\pwindex{Goldmann, Paul 31.01.1865 – 25.09.1935@\textsc{Goldmann, Paul} (31.01.1865 – 25.09.1935), \emph{Schriftsteller, Journalist}|pwk} seit 1899
                  ein Verhältnis hatte, war mit Ludwig
                     Rottenberg\pwindex{Rottenberg, Ludwig 11.10.1864 – 6.5.1932@\textsc{Rottenberg, Ludwig} (11.10.1864 – 6.5.1932), \emph{Kapellmeister}|pwk} verheiratet.}}}\label{K_L03386-1h}es wurden drohend und ſchienen eine
               Kataſtrophe anzukündigen. Was nach der Heimkehr der armen Frau\pwindex{Rottenberg, Theodore 1875-09-07 – 1945-04-05@\textsc{Rottenberg, Theodore} (1875-09-07 – 1945-04-05)|pwv} geſchehen iſt, weiß ich noch nicht.
               Auch auf meiner Seite gibt es \strikeout{ung\textcolor{gray}{e}} unvorhergeſehne Complikationen. Ich erhielt einen Brief meines Schwager\pwindex{Rosengart, Josef 1860-02-08 – 1927-08-04@\textsc{Rosengart, Josef} (1860-02-08 – 1927-08-04), \emph{Arzt}|pwv}s, der beſagt, dieſe
                  Frau\pwindex{Rottenberg, Theodore 1875-09-07 – 1945-04-05@\textsc{Rottenberg, Theodore} (1875-09-07 – 1945-04-05)|pwv} ſei nach den \label{K_L03386-2v}\edtext{Ereigniſſen dieſes Winters}{\lemma{\textnormal{\emph{Ereigniſſen … Winters}}}\Cendnote{\textnormal{Bezug auf Rottenberg\pwindex{Rottenberg, Theodore 1875-09-07 – 1945-04-05@\textsc{Rottenberg, Theodore} (1875-09-07 – 1945-04-05)|pwk}s anderen Liebhaber\pwindex{?? [Partner von Theodore Rottenberg, Ende 1902/Anfang 1903] @\textsc{?? [Partner von Theodore Rottenberg, Ende 1902/Anfang 1903]}|pwk}
                  Ende 1902–Anfang 1903 (vgl. Paul Goldmann an Arthur Schnitzler, 14. 11. [1903], siehe auch etwa Paul Goldmann an Arthur Schnitzler, 28. 12. [1902] und 17. 2. [1903])}}}\label{K_L03386-2h} nicht mehr eine Frau, die man
               heirathet, und der mich vor die Wahl zwiſchen einer Heirath und einem Bruch mit
               meinem Schwager\pwindex{Rosengart, Josef 1860-02-08 – 1927-08-04@\textsc{Rosengart, Josef} (1860-02-08 – 1927-08-04), \emph{Arzt}|pwv} ſtellt. Mein
                  Onkel\pwindex{Mamroth, Fedor 21.02.1851 – 25.06.1907@\textsc{Mamroth, Fedor} (21.02.1851 – 25.06.1907), \emph{Journalist, Kritiker}|pwv}, den ich unterwegs
               getroffen, ſpricht zu mir\strikeout{, \textcolor{gray}{×}\-\textcolor{gray}{×} d\textcolor{gray}{en}} in dem milden und mitleidigen Tone, in dem man zu Jemandem ſpricht, der im
               Begriff iſt, ſich in ein großes Unheil zu ſtürzen. Ich weiß in dieſem {\pb}Widerſtreit der Empfindungen wieder nicht aus noch
               ein.\pend
           \pstart
           Heut fahre ich ein paar Tage nach Venedig\oindex{Venedig@\textbf{Venedig}|pw}. Vor Montag bin ich kaum
               in \label{K_L03386-3v}\edtext{Wien\oindex{Wien@\textbf{Wien}|pw}}{\lemma{\textnormal{\emph{Wien}}}\Cendnote{\textnormal{Goldmann\pwindex{Goldmann, Paul 31.01.1865 – 25.09.1935@\textsc{Goldmann, Paul} (31.01.1865 – 25.09.1935), \emph{Schriftsteller, Journalist}|pwk} und Schnitzler\pwindex{Schnitzler, Arthur 15.05.1862 – 21.10.1931@\textsc{Schnitzler, Arthur} (15.05.1862 – 21.10.1931), \emph{Schriftsteller, Mediziner}|pwk} sahen sich am 18. 9. 1903, 20. 9. 1903 und 21. 9. 1903 – er war also jedenfalls zwischen
                     18. 9. 1903 und
                     21. 9. 1903 in
                     Wien\oindex{Wien@\textbf{Wien}|pwk}. Am 2. 10. [1903] schrieb Goldmann\pwindex{Goldmann, Paul 31.01.1865 – 25.09.1935@\textsc{Goldmann, Paul} (31.01.1865 – 25.09.1935), \emph{Schriftsteller, Journalist}|pwk} noch einmal aus Wien\oindex{Wien@\textbf{Wien}|pwk}, er
                  blieb also vermutlich die ganze Zeit bis Anfang Oktober 1903.}}}\label{K_L03386-3h}. Natürlich wirſt Du Dich in Deinen
               Reiſedispoſitionen auch mich keineswegs ſtören laſſen. Wenn Du mir etwas ſchreiben
               willſt: \textsc{Venedig\oindex{Venedig@\textbf{Venedig}|pw}, Poste}\textsc{restante}.\pend
           \pstart
           Ich grüße Dich und Deine Frau\pwindex{Schnitzler, Olga 17.01.1882 – 13.01.1970@\textsc{Schnitzler, Olga} (17.01.1882 – 13.01.1970), \emph{Schauspielerin, Sängerin}|pwv} auf das Herzlichſte. {\\[\baselineskip]}Dein treuer {\\[\baselineskip]}\spacefill\mbox{Paul Goldmn}\pend
           \leftskip=0em{}
         
         \endnumbering\mylabel{h}\end{ledgroupsized}\begin{anhang}\end{anhang}\newcommand{\dateiname}{L03386}\newcommand{\titel}{Paul Goldmann an Arthur Schnitzler, 7. 9. 1903}\newcommand{\editorInnen}{Martin Anton Müller und Laura Untner}%% latex-leseansicht-abspann.tex
%% Abspann für die Leseansicht.
%% Der Schalter \ifkorrekturansicht ist bereits durch den Vorspann gesetzt.

%% latex-abspann.tex
%% Gemeinsamer Abspann für Korrekturansicht und Leseansicht.
%% Setzt den Schalter \ifkorrekturansicht voraus (gesetzt in den
%% einbindenden Dateien latex-korrekturansicht-abspann.tex bzw.
%% latex-leseansicht-abspann.tex).
%% ---------------------------------------------------------------

\normalsize

% Das esempio-Environment wird nur in der Leseansicht benötigt
\ifkorrekturansicht\else
\newenvironment{esempio}[3]%
{
    \vspace{1.5ex}
    \rlap{\underline{#1}}
    \par
    \setlength{\parindent}{0cm}
    \nopagebreak
    \leftskip=#2cm
    \rightskip=#3cm
}
{
    \par
}
\fi

\doendnotes{C}
\bigskip
\vfill

\clearpage

\footnotesize

\ifkorrekturansicht
  \lohead{\textsc{register}}
\fi

% theindex-Environment neu definieren ohne reledmac
\makeatletter
\renewenvironment{theindex}{%
  \ifkorrekturansicht
    \section*{\indexname}%
  \else
    \subsubsection*{Index der erwähnten Entitäten}%
  \fi
  \setlength{\parindent}{0pt}%
  \setlength{\parskip}{0pt plus 0.3pt}%
  \let\item\@idxitem
}{%
  \ifkorrekturansicht\clearpage\fi
}
\makeatother

\IfFileExists{\jobname-pw.ind}{\input{\jobname-pw.ind}}{}

% Quellenangabe nur in der Leseansicht
\ifkorrekturansicht\else
% Fallback-Definitionen, falls die .tex-Datei \titel etc. nicht gesetzt hat
\providecommand{\titel}{}
\providecommand{\editorInnen}{}
\providecommand{\dateiname}{\jobname}

\vspace{3cm}

\vfill

\footnotesize
\textsc{Quelle}: \titel. Herausgegeben von {\editorInnen}. In: \emph{Arthur Schnitzler: Briefwechsel mit Autorinnen und Autoren}.
 Digitale Edition, https://schnitzler-briefe.acdh.oeaw.ac.at/{\dateiname}.html (Stand \today)
\fi

\end{document}


      