%% latex-korrekturansicht-vorspann.tex
%% Vorspann für die Korrekturansicht.
%% Lädt die gemeinsame Datei latex-vorspann.tex mit gesetztem Schalter.

\newif\ifkorrekturansicht
\korrekturansichttrue

\input{../tex-inputs/latex-vorspann}


\section[Richard Beer-Hofmann an Arthur Schnitzler, 4. 8. 1909]{L01863 Richard Beer-Hofmann an Arthur Schnitzler, 4. 8. 1909}
\nopagebreak\mylabel{L01863v}
\rehead{ }\normalsize\beginnumbering\briefempfaengerindex{Schnitzler, Arthur@\textsc{Schnitzler, Arthur}!zzzBeer-Hofmann, Richard@\emph{von Richard Beer-Hofmann}!1909-08-041@{4. 8. 1909}|(be}
\toendnotes[C]{\smallbreak\pagebreak[2]}\Standort{CUL, Schnitzler, B 8.}
\physDesc{Brief, 1 Blatt, 2 Seiten, 589 Zeichen
\newline{}Handschrift: schwarze Tinte, lateinische Kurrent
\newline{}Schnitzler: mit Bleistift beschriftet: »\textsc{Beer Hofm.}« 
\newline{}Ordnung: mit Bleistift von unbekannter Hand nummeriert:
                                    »221« }
\buchAbdrucke{\weitereDrucke{Arthur Schnitzler, Richard Beer-Hofmann: \emph{Briefwechsel 1891–1931}. Wien, Zürich: \emph{Europaverlag} 1992, S. 194–195.} }\toendnotes[C]{\smallbreak}
\pstart
           \raggedleft{}{\pb}Wien\oindex{Wien@\textbf{Wien}, \emph{A.ADM2}|pw}{ }4./VIII. 09.\pend
           \vspace{0.5em}
\pstart
           Lieber Arthur! Dank für Ihren Brief. \label{K_L01863-1v}\edtext{Es war nicht schön}{\lemma{\textnormal{\emph{Es war nicht schön}}}\Cendnote{\textnormal{Der Tod der Tante, vgl. Arthur Schnitzler an Richard Beer-Hofmann, 31. 7. 1909 und A. S.: \emph{Tagebuch}, 7. 8. 1909.}}}\label{K_L01863-1}; man prügelt uns zu oft.\pend
           
\pstart
           Wir wollen am 9 hier weg, in Villach\oindex{Villach@\textbf{Villach}, \emph{A.ADM3}|pw} übernachten, am 10, am Lido\oindex{Lido@\textbf{Lido}, \emph{P.PPL}|pw}. Paula\pwindex{Beer-Hofmann, Paula 25.02.1879 – 30.10.1939@\textsc{Beer-Hofmann, Paula} (25.02.1879 – 30.10.1939)|pw} braucht Wärme, und Sonne,
               und die haben wir – hoffe ich – doch sicherer da unten – am Lido\oindex{Lido@\textbf{Lido}, \emph{P.PPL}|pw} meine ich. Sonst wären wir sehr gerne mit Ihnen beisa{\geminationm}en gewesen.\pend
           
\pstart
           Von Leo\pwindex{Van-Jung, Leo 15.10.1866 – 02.07.1939@\textsc{Van-Jung, Leo} (15.10.1866 – 02.07.1939), \emph{Gesangspädagoge/Gesangspädagogin, Mathematiker/Mathematikerin}|pw} hörte ich, dass es {\pb}Ihnen Allen gut geht.\pend
           
\pstart
           Ich dachte daran auf einen Tag zu Ihnen zu ko{\geminationm}en, aber
               es ist zu viel Hetze und wir sind so müde.\pend
           
\pstart
           Des Medardus\pwindex{junge Medardus. Dramatische Historie in einem Vorspiel und fuenf Aufzuegen@\emph{Der junge Medardus. Dramatische Historie in einem Vorspiel und fünf Aufzügen}|pw} Schicksal hat mich sehr gefreut.
               Wann werde ich ihn kennen lernen. Herzliche Grüsse Ihnen, Ihrer Frau\pwindex{Schnitzler, Olga 17.01.1882 – 13.01.1970@\textsc{Schnitzler, Olga} (17.01.1882 – 13.01.1970), \emph{Schauspieler/Schauspielerin, Sänger/Sängerin}|pwv} und dem Buben\pwindex{Schnitzler, Heinrich 09.08.1902 – 12.07.1982@\textsc{Schnitzler, Heinrich} (09.08.1902 – 12.07.1982), \emph{Regisseur/Regisseurin, Schauspieler/Schauspielerin}|pwv}.\pend
           
\pstart
           Ihr{\\[\baselineskip]}\spacefill\mbox{Richard}\pend
           \leftskip=0em{}\selectlanguage{ngerman}\endnumbering\briefempfaengerindex{Schnitzler, Arthur@\textsc{Schnitzler, Arthur}!zzzBeer-Hofmann, Richard@\emph{von Richard Beer-Hofmann}!1909-08-041@{4. 8. 1909}|)be}\mylabel{L01863h}  \normalsize

\doendnotes{C}
\bigskip
\vfill

\clearpage

\footnotesize

\lohead{\textsc{register}}

% Definiere theindex-Environment komplett neu ohne reledmac
\makeatletter
\renewenvironment{theindex}{%
  \section*{\indexname}%
  \setlength{\parindent}{0pt}%
  \setlength{\parskip}{0pt plus 0.3pt}%
  \let\item\@idxitem
}{%
  \clearpage
}
\makeatother

\IfFileExists{\jobname-pw.ind}{\input{\jobname-pw.ind}}{}

\end{document}

      