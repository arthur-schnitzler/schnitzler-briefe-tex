%% latex-leseansicht-vorspann.tex
%% Vorspann für die Leseansicht.
%% Lädt die gemeinsame Datei latex-vorspann.tex mit nicht gesetztem Schalter.

\newif\ifkorrekturansicht
\korrekturansichtfalse

\input{../tex-inputs/latex-vorspann}


\section[Richard Beer-Hofmann an Arthur Schnitzler, 23. 6. 1893]{L00227 Richard Beer-Hofmann an Arthur Schnitzler, 23. 6. 1893}
\nopagebreak\mylabel{L00227v}
\rehead{ }\normalsize\beginnumbering\briefempfaengerindex{Schnitzler, Arthur@\textsc{Schnitzler, Arthur}!zzzBeer-Hofmann, Richard@\emph{von Richard Beer-Hofmann}!1893-06-231@{23. 6. 1893}|(be}
\toendnotes[C]{\smallbreak\pagebreak[2]}
\correspDesc{Versand  durch Richard Beer-Hofmann am 23. 6. 1893 in Bad Ischl
\newline{}Erhalt  durch Arthur Schnitzler im Zeitraum [24. 6. 1893
                  – 28. 6. 1893?] in Wien}\toendnotes[C]{\smallbreak}
\Standort{CUL, Schnitzler, B 8.}
\physDesc{Brief, 2 Blätter, 3 Seiten, 1295 Zeichen
\newline{}Handschrift: Bleistift, lateinische Kurrent
\newline{}Schnitzler: mit Bleistift nummeriert: »18« bzw.
                                    »18a« }
\buchAbdrucke{\weitereDrucke{Arthur Schnitzler, Richard Beer-Hofmann: \emph{Briefwechsel 1891–1931}. Herausgegeben von Konstanze Fliedl. Wien, Zürich: \emph{Europaverlag} 1992, S. 45.} }
\pstart
           \noindent{}{\pb}Lieber Arthur! Bisher
               hat sich Jarno\pwindex{Jarno, Josef 24.\,8.\,1865 Budapest – 11.\,1.\,1932 Wien@\textsc{Jarno, Josef} (24.\,8.\,1865 Budapest – 11.\,1.\,1932 Wien), \emph{Theaterleiter, Schauspieler}|pw} noch nicht sehen lassen;
               übrigens ko{\geminationm}en Sie ja hoffentlich in einigen Tagen
               selbst. Bitte, wenn Sie ko{\geminationm}en bringen Sie mir ein
               Flaccon Parfüm mit; es ist bei »\uline{Weisse\orgindex{Theodor Weisse@Theodor Weisse|pw}}« am Mehlmarkt\oindex{Wien@\textbf{Wien}!I., Innere Stadt@\textbf{I., Innere Stadt}!Neuer Markt@\textbf{Neuer Markt}, \emph{Platz}|pw} Ecke der Plankengasse\oindex{Plankengasse@\textbf{Plankengasse}, \emph{Straße}|pw} erhältlich, der Name ist, \uline{glaube} ich: »\uline{Neomir du
               Phare}« oder sonst irgendwie aehnlich; auch bringen – oder wenn\strikeout{s} es Sie genirt, – schicken Sie mir 100 Stück egyptische\oindex{Ägypten@\textbf{Ägypten}|pw} echte Cigaretten irgendwelche Marke zu
               5-6 fl. höchstens (Riedhof\oindex{Wien@\textbf{Wien}!VIII., Josefstadt@\textbf{VIII., Josefstadt}!Riedhof@\textbf{Riedhof}, \emph{Lokal}|pw}, Central\oindex{Wien@\textbf{Wien}!I., Innere Stadt@\textbf{I., Innere Stadt}!Café Central@\textbf{Café Central}, \emph{Kaffeehaus}|pw}, Sacher\oindex{Wien@\textbf{Wien}!I., Innere Stadt@\textbf{I., Innere Stadt}!Hotel Sacher@\textbf{Hotel Sacher}, \emph{Hotel}|pw}, Caffée Impérial\oindex{Wien@\textbf{Wien}!I., Innere Stadt@\textbf{I., Innere Stadt}!Café Imperial@\textbf{Café Imperial}, \emph{Kaffeehaus}|pw}). Vielleicht ni{\geminationm}t Salten\pwindex{Salten, Felix 6.\,9.\,1869 Budapest – 8.\,10.\,1945 Zürich@\textsc{Salten, Felix} (6.\,9.\,1869 Budapest – 8.\,10.\,1945 Zürich), \emph{Schriftsteller, Journalist, Chefredakteur}|pw} seinen
               Urlaub auch um dieselbe Zeit? Ich sehe ein daß mir – da ich Euch
               \textcolor{gray}{d}och nicht nachlaufen kann – nichts anderes {\pb}übrig bleiben wird, als im Herbste
               gleichfalls Bycicle oder Bicycle fahren zu lernen; ich traure bereits jetzt bei dem
               Gedanken wieviel Ersparnisse an Fiakern und Omnibus-Fahrten mich das wieder kosten
               wird!\pend
           
\pstart
           Herzlichst{\\[\baselineskip]}\spacefill\mbox{Richard}\pend
           \leftskip=0em{}
\pstart
           \noindent{}Grüßen Sie nach Ermessen, und wenn Sie die Comissionen irgendwie geniren, geben
                  Sie sich keine Mühe, – es ist nicht wichtig.\pend
           
\pstart
           \raggedleft{}R.\pend
           
\pstart
           23 Juni 93 Ischl\oindex{Bad Ischl@\textbf{Bad Ischl}|pw}\pend
           
\pstart
           {\pb}Soeben fällt mir ein\substVorne{}\textsuperscript{,}\substDazwischen{}:\substHinten{} Gestern saß in der Theater-Loge ein Fräulein »Wreden\pwindex{Wreden, Grethe @\textsc{Wreden, Grethe}, \emph{Schauspielerin}|pw}«, mir »wolbekannt«, eine der
                  3 Schlafwagenconducteurstöchter wenn ich nicht irre, und P. H.\pwindex{Horn, Paul 13.\,2.\,1867 Wien – 18.\,1.\,1936 Menton@\textsc{Horn, Paul} (13.\,2.\,1867 Wien – 18.\,1.\,1936 Menton), \emph{Fabrikant}|pw}{[}s{]} gewesene Herrin? Was ist mit ihr? Soll man sie besuchen, –
                  ansprechen – ignoriren, weiß P. H.\pwindex{Horn, Paul 13.\,2.\,1867 Wien – 18.\,1.\,1936 Menton@\textsc{Horn, Paul} (13.\,2.\,1867 Wien – 18.\,1.\,1936 Menton), \emph{Fabrikant}|pw} von
                  ihrem hiesigen Aufenthalte, ko{\geminationm}t er her?\pend
           \selectlanguage{ngerman}\endnumbering\briefempfaengerindex{Schnitzler, Arthur@\textsc{Schnitzler, Arthur}!zzzBeer-Hofmann, Richard@\emph{von Richard Beer-Hofmann}!1893-06-231@{23. 6. 1893}|)be}\mylabel{L00227h}  \newcommand{\dateiname}{L00227}\newcommand{\titel}{Richard Beer-Hofmann an Arthur Schnitzler, 23. 6. 1893}\newcommand{\editorInnen}{Martin Anton Müller und Gerd-Hermann Susen}%% latex-leseansicht-abspann.tex
%% Abspann für die Leseansicht.
%% Der Schalter \ifkorrekturansicht ist bereits durch den Vorspann gesetzt.

%% latex-abspann.tex
%% Gemeinsamer Abspann für Korrekturansicht und Leseansicht.
%% Setzt den Schalter \ifkorrekturansicht voraus (gesetzt in den
%% einbindenden Dateien latex-korrekturansicht-abspann.tex bzw.
%% latex-leseansicht-abspann.tex).
%% ---------------------------------------------------------------

\normalsize

% Das esempio-Environment wird nur in der Leseansicht benötigt
\ifkorrekturansicht\else
\newenvironment{esempio}[3]%
{
    \vspace{1.5ex}
    \rlap{\underline{#1}}
    \par
    \setlength{\parindent}{0cm}
    \nopagebreak
    \leftskip=#2cm
    \rightskip=#3cm
}
{
    \par
}
\fi

\doendnotes{C}
\bigskip
\vfill

\clearpage

\footnotesize

\ifkorrekturansicht
  \lohead{\textsc{register}}
\fi

% theindex-Environment neu definieren ohne reledmac
\makeatletter
\renewenvironment{theindex}{%
  \ifkorrekturansicht
    \section*{\indexname}%
  \else
    \subsubsection*{Index der erwähnten Entitäten}%
  \fi
  \setlength{\parindent}{0pt}%
  \setlength{\parskip}{0pt plus 0.3pt}%
  \let\item\@idxitem
}{%
  \ifkorrekturansicht\clearpage\fi
}
\makeatother

\IfFileExists{\jobname-pw.ind}{\input{\jobname-pw.ind}}{}

% Quellenangabe nur in der Leseansicht
\ifkorrekturansicht\else
% Fallback-Definitionen, falls die .tex-Datei \titel etc. nicht gesetzt hat
\providecommand{\titel}{}
\providecommand{\editorInnen}{}
\providecommand{\dateiname}{\jobname}

\vspace{3cm}

\vfill

\footnotesize
\textsc{Quelle}: \titel. Herausgegeben von {\editorInnen}. In: \emph{Arthur Schnitzler: Briefwechsel mit Autorinnen und Autoren}.
 Digitale Edition, https://schnitzler-briefe.acdh.oeaw.ac.at/{\dateiname}.html (Stand \today)
\fi

\end{document}


