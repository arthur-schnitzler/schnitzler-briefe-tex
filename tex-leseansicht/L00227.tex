\input{../tex-inputs/latex-pdf-vorspann}
\begin{center}
            \textcolor{red}{ENTWURF. ENTZIFFERUNG NOCH NICHT KORREKTURGELESEN}
                      \end{center}
            
               \section[Richard Beer-Hofmann an Arthur Schnitzler, 23. 6. 1893]{ Richard Beer-Hofmann an Arthur Schnitzler, 23. 6. 1893}\nopagebreak\mylabel{v}\rehead{ }\begin{ledgroupsized}[t]{13cm}\normalsize\beginnumbering\briefempfaengerindex{Schnitzler, Arthur@\textsc{Schnitzler, Arthur}!zzzBeer-Hofmann, Richard@\emph{von Richard Beer-Hofmann}!1893-06-231@{23. 6. 1893}|(be} \toendnotes[C]{\smallbreak\pagebreak[2]} \Standort{CUL, Schnitzler, B 8.}
\physDesc{Brief, 2 Blätter, 3 Seiten
\newline{}Handschrift: Bleistift, lateinische Kurrent
\newline{}Schnitzler: mit Bleistift nummeriert: »18« bzw.
               »18a« }\buchAbdrucke{\weitereDrucke{Arthur Schnitzler, Richard Beer-Hofmann: \emph{Briefwechsel 1891–1931}. Hg. Konstanze Fliedl. Wien, Zürich: \emph{Europaverlag} 1992, S. 45.} }\pstart
           \noindent{}{\pb}Lieber Arthur! Bisher
               hat sich Jarno\pwindex{Jarno, Josef 24.08.1865 – 11.01.1932@\textsc{Jarno, Josef} (24.08.1865 – 11.01.1932), \emph{Theaterleiter, Schauspieler}|pw} noch nicht sehen lassen; übrigens
                  ko{\geminationm}en Sie ja hoffentlich in einigen Tagen selbst.
               Bitte, wenn Sie ko{\geminationm}en bringen Sie mir ein Flaccon Parfüm
               mit; es ist bei »\uline{Weisse\orgindex{Theodor Weisse@Theodor Weisse|pw}}« am Mehlmarkt\oindex{Neuer Markt@\textbf{Neuer Markt}|pw} Ecke der Plankengasse\oindex{Plankengasse@\textbf{Plankengasse}|pw} erhältlich, der Name ist, \uline{glaube} ich: »\uline{Neomir du
               Phare}« oder sonst irgendwie aehnlich; auch bringen – oder wenn\strikeout{s} es Sie genirt, – schicken Sie mir 100 Stück egyptische\oindex{Aegypten@\textbf{Ägypten}|pw} echte Cigaretten irgendwelche Marke zu
               5-6 fl. höchstens (Riedhof\oindex{Riedhof@\textbf{Riedhof}|pw}, Central\oindex{Cafe Central@\textbf{Café Central}|pw}, Sacher\oindex{Hotel Sacher@\textbf{Hotel Sacher}|pw}, Caffée Impérial\oindex{Cafe Imperial@\textbf{Café Imperial}|pw}). Vielleicht ni{\geminationm}t
                  Salten\pwindex{Salten, Felix 06.09.1869 – 08.10.1945@\textsc{Salten, Felix} (06.09.1869 – 08.10.1945), \emph{Schriftsteller, Journalist}|pw} seinen Urlaub auch um dieselbe Zeit? Ich
               sehe ein daß mir – da ich Euch \textcolor{gray}{d}och nicht nachlaufen kann – nichts
               anderes {\pb}übrig bleiben wird, als im
               Herbste gleichfalls Bycicle oder Bicycle fahren zu lernen; ich traure bereits jetzt
               bei dem Gedanken wieviel Ersparnisse an Fiakern und Omnibus-Fahrten mich das wieder
               kosten wird!\pend
           \pstart
           Herzlichst{\\[\baselineskip]}\spacefill\mbox{Richard}\pend
           \leftskip=0em{}\pstart
           \noindent{}Grüßen Sie nach Ermessen, und wenn Sie die Comissionen irgendwie geniren, geben
                  Sie sich keine Mühe, – es ist nicht wichtig.\pend
           \pstart
           \raggedleft{}R.\pend
           \pstart
           \noindent{}23 Juni 93 Ischl\oindex{Bad Ischl@\textbf{Bad Ischl}|pw}\pend
           \pstart
           {\pb}Soeben fällt mir ein\substVorne{}\textsuperscript{,}\substDazwischen{}:\substHinten{} Gestern saß in der Theater-Loge ein Fräulein »Wreden\pwindex{Wreden, Grethe @\textsc{Wreden, Grethe}, \emph{Schauspielerin}|pw}«, mir »wolbekannt«, eine der
                  3 Schlafwagenconducteurstöchter wenn ich nicht irre, und P. H.\pwindex{Horn, Paul 13.02.1867 – 18.01.1936@\textsc{Horn, Paul} (13.02.1867 – 18.01.1936), \emph{Fabrikant}|pw}{[}s{]} gewesene Herrin? Was ist mit ihr? Soll man sie besuchen, –
                  ansprechen – ignoriren, weiß P. H.\pwindex{Horn, Paul 13.02.1867 – 18.01.1936@\textsc{Horn, Paul} (13.02.1867 – 18.01.1936), \emph{Fabrikant}|pw} von ihrem
                  hiesigen Aufenthalte, ko{\geminationm}t er her?\pend
           \endnumbering\briefempfaengerindex{Schnitzler, Arthur@\textsc{Schnitzler, Arthur}!zzzBeer-Hofmann, Richard@\emph{von Richard Beer-Hofmann}!1893-06-231@{23. 6. 1893}|)be}\mylabel{h}\end{ledgroupsized}  \newcommand{\dateiname}{L00227}\newcommand{\titel}{Richard Beer-Hofmann an Arthur Schnitzler, 23. 6. 1893}\newcommand{\editorInnen}{Martin Anton Müller und Gerd-Hermann Susen}\input{../tex-inputs/latex-pdf-abspann}
      