%% latex-leseansicht-vorspann.tex
%% Vorspann für die Leseansicht.
%% Lädt die gemeinsame Datei latex-vorspann.tex mit nicht gesetztem Schalter.

\newif\ifkorrekturansicht
\korrekturansichtfalse

\input{../tex-inputs/latex-vorspann}


\section[Arthur Schnitzler an Karl Emil Franzos, 25. 6. 1892]{L03615 Arthur Schnitzler an Karl Emil Franzos, 25. 6. 1892}
\nopagebreak\mylabel{L03615v}
\rehead{ }\normalsize\beginnumbering\briefempfaengerindex{Franzos, Karl Emil@\textsc{Franzos, Karl Emil}!zzzSchnitzler, Arthur@\emph{von Arthur Schnitzler}!1892-06-251@{25. 6. 1892}|(be}
\toendnotes[C]{\smallbreak\pagebreak[2]}
\correspDesc{Versand  durch Arthur Schnitzler am 25. 6. 1892 in Wien
\newline{}Erhalt  durch Karl Emil Franzos im Zeitraum [26. 6. 1892 – 30. 6. 1892?] in Berlin}\toendnotes[C]{\smallbreak}
\Standort{Wienbibliothek im Rathaus, H.I.N.-60192.}
\physDesc{Brief, 1 Blatt, 1 Seite, 208 Zeichen
\newline{}Handschrift: schwarze Tinte, deutsche Kurrent}\toendnotes[C]{\smallbreak}
\pstart
           \raggedleft{}{\pb}25/6 92\pend
           
\pstart{}Hochgeſchätzter Herr,\pend\vspace{0.5em}
\pstart
           es wäre mir eine beſondre Ehre,  wenn Sie das beifolgende
               \label{K_L03615-1v}\edtext{Gedicht\pwindex{Schnitzler, Arthur 15.\,5.\,1862 Wien – 21.\,10.\,1931 ebd.@\textsc{Schnitzler, Arthur} (15.\,5.\,1862 Wien – 21.\,10.\,1931 ebd.), \emph{Schriftsteller, Mediziner}!Anfang vom Ende@\strich\emph{Anfang vom Ende}|pwv}}{\lemma{\textnormal{\emph{Gedicht}}}\Cendnote{\textnormal{Arthur Schnitzler: \emph{Anfang vom Ende}\pwindex{Schnitzler, Arthur 15.\,5.\,1862 Wien – 21.\,10.\,1931 ebd.@\textsc{Schnitzler, Arthur} (15.\,5.\,1862 Wien – 21.\,10.\,1931 ebd.), \emph{Schriftsteller, Mediziner}!Anfang vom Ende@\strich\emph{Anfang vom Ende}|pwk}. In:
                     \emph{Deutsche Dichtung}\pwindex{Deutsche Dichtung@\emph{Deutsche Dichtung}|pwk}, Bd. 12, Nr. 8,
                     15. 7. 1892, S. 192.
               }}}\label{K_L03615-1} der Aufnahme in der »\textsc{Deutschen Dichtung\pwindex{Deutsche Dichtung@\emph{Deutsche Dichtung}|pw}}« werth hielten.\pend
           
\pstart
           Hochachtungsvoll{\\[\baselineskip]}\spacefill\mbox{Dr. Arthur Schnitzler}\pend
           \leftskip=0em{}
\pstart
           \noindent{}Wien, I. \textsc{Giselastraße 11}\oindex{Wien@\textbf{Wien}!I., Innere Stadt@\textbf{I., Innere Stadt}!Bösendorferstraße 7@\textbf{Bösendorferstraße 7}, \emph{Wohngebäude}|pw}.\pend
           \selectlanguage{ngerman}\endnumbering\briefempfaengerindex{Franzos, Karl Emil@\textsc{Franzos, Karl Emil}!zzzSchnitzler, Arthur@\emph{von Arthur Schnitzler}!1892-06-251@{25. 6. 1892}|)be}\mylabel{L03615h}  \newcommand{\dateiname}{L03615}\newcommand{\titel}{Arthur Schnitzler an Karl Emil Franzos, 25. 6. 1892}\newcommand{\editorInnen}{Selma Jahnke und Martin Anton Müller}%% latex-leseansicht-abspann.tex
%% Abspann für die Leseansicht.
%% Der Schalter \ifkorrekturansicht ist bereits durch den Vorspann gesetzt.

%% latex-abspann.tex
%% Gemeinsamer Abspann für Korrekturansicht und Leseansicht.
%% Setzt den Schalter \ifkorrekturansicht voraus (gesetzt in den
%% einbindenden Dateien latex-korrekturansicht-abspann.tex bzw.
%% latex-leseansicht-abspann.tex).
%% ---------------------------------------------------------------

\normalsize

% Das esempio-Environment wird nur in der Leseansicht benötigt
\ifkorrekturansicht\else
\newenvironment{esempio}[3]%
{
    \vspace{1.5ex}
    \rlap{\underline{#1}}
    \par
    \setlength{\parindent}{0cm}
    \nopagebreak
    \leftskip=#2cm
    \rightskip=#3cm
}
{
    \par
}
\fi

\doendnotes{C}
\bigskip
\vfill

\clearpage

\footnotesize

\ifkorrekturansicht
  \lohead{\textsc{register}}
\fi

% theindex-Environment neu definieren ohne reledmac
\makeatletter
\renewenvironment{theindex}{%
  \ifkorrekturansicht
    \section*{\indexname}%
  \else
    \subsubsection*{Index der erwähnten Entitäten}%
  \fi
  \setlength{\parindent}{0pt}%
  \setlength{\parskip}{0pt plus 0.3pt}%
  \let\item\@idxitem
}{%
  \ifkorrekturansicht\clearpage\fi
}
\makeatother

\IfFileExists{\jobname-pw.ind}{\input{\jobname-pw.ind}}{}

% Quellenangabe nur in der Leseansicht
\ifkorrekturansicht\else
% Fallback-Definitionen, falls die .tex-Datei \titel etc. nicht gesetzt hat
\providecommand{\titel}{}
\providecommand{\editorInnen}{}
\providecommand{\dateiname}{\jobname}

\vspace{3cm}

\vfill

\footnotesize
\textsc{Quelle}: \titel. Herausgegeben von {\editorInnen}. In: \emph{Arthur Schnitzler: Briefwechsel mit Autorinnen und Autoren}.
 Digitale Edition, https://schnitzler-briefe.acdh.oeaw.ac.at/{\dateiname}.html (Stand \today)
\fi

\end{document}


