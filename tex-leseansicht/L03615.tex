%% latex-korrekturansicht-vorspann.tex
%% Vorspann für die Korrekturansicht.
%% Lädt die gemeinsame Datei latex-vorspann.tex mit gesetztem Schalter.

\newif\ifkorrekturansicht
\korrekturansichttrue

\input{../tex-inputs/latex-vorspann}


\section[Arthur Schnitzler an Karl Emil Franzos, 25. 6. 1892]{L03615 Arthur Schnitzler an Karl Emil Franzos, 25. 6. 1892}
\nopagebreak\mylabel{L03615v}
\rehead{ }\normalsize\beginnumbering\briefempfaengerindex{Franzos, Karl Emil@\textsc{Franzos, Karl Emil}!zzzSchnitzler, Arthur@\emph{von Arthur Schnitzler}!1892-06-251@{25. 6. 1892}|(be}
\toendnotes[C]{\smallbreak\pagebreak[2]}\Standort{Wienbibliothek im Rathaus, H.I.N.-60192.}
\physDesc{Brief, 1 Blatt, 1 Seite, 208 Zeichen
\newline{}Handschrift: schwarze Tinte, deutsche Kurrent}\toendnotes[C]{\smallbreak}
\pstart
           \raggedleft{}{\pb}25/6 92\pend
           
\pstart{}Hochgeſchätzter Herr,\pend\vspace{0.5em}
\pstart
           es wäre mir eine beſondre Ehre,  wenn Sie das beifolgende
               \label{K_L03615-1v}\edtext{Gedicht\pwindex{Anfang vom Ende@\emph{Anfang vom Ende}|pwv}}{\lemma{\textnormal{\emph{Gedicht}}}\Cendnote{\textnormal{Arthur Schnitzler: \emph{Anfang vom Ende}\pwindex{Anfang vom Ende@\emph{Anfang vom Ende}|pwk}. In:
                     \emph{Deutsche Dichtung}\pwindex{Deutsche Dichtung@\emph{Deutsche Dichtung}|pwk}, Bd. 12, Nr. 8,
                     15. 7. 1892, S. 192.
               }}}\label{K_L03615-1} der Aufnahme in der »\textsc{Deutschen Dichtung\pwindex{Deutsche Dichtung@\emph{Deutsche Dichtung}|pw}}« werth hielten. \pend
           
\pstart
           Hochachtungsvoll{\\[\baselineskip]}\spacefill\mbox{Dr. Arthur Schnitzler}\pend
           \leftskip=0em{}
\pstart
           \noindent{}Wien, I. \textsc{Giselastraße 11}\oindex{Boesendorferstrasse 7@\textbf{Bösendorferstraße 7}, \emph{Wohngebäude (K.WHS)}|pw}.\pend
           \selectlanguage{ngerman}\endnumbering\briefempfaengerindex{Franzos, Karl Emil@\textsc{Franzos, Karl Emil}!zzzSchnitzler, Arthur@\emph{von Arthur Schnitzler}!1892-06-251@{25. 6. 1892}|)be}\mylabel{L03615h}  \normalsize

\doendnotes{C}
\bigskip
\vfill

\clearpage

\footnotesize

\lohead{\textsc{register}}

% Definiere theindex-Environment komplett neu ohne reledmac
\makeatletter
\renewenvironment{theindex}{%
  \section*{\indexname}%
  \setlength{\parindent}{0pt}%
  \setlength{\parskip}{0pt plus 0.3pt}%
  \let\item\@idxitem
}{%
  \clearpage
}
\makeatother

\IfFileExists{\jobname-pw.ind}{\input{\jobname-pw.ind}}{}

\end{document}

      