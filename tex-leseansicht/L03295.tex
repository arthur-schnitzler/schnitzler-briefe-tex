%% latex-korrekturansicht-vorspann.tex
%% Vorspann für die Korrekturansicht.
%% Lädt die gemeinsame Datei latex-vorspann.tex mit gesetztem Schalter.

\newif\ifkorrekturansicht
\korrekturansichttrue

\input{../tex-inputs/latex-vorspann}


\section[ Felix Salten an Arthur Schnitzler, 27. 7. 1899]{L03295 Felix Salten an Arthur Schnitzler, 27. 7. 1899}
\nopagebreak\mylabel{L03295v}
\rehead{ }\normalsize\beginnumbering\briefempfaengerindex{Schnitzler, Arthur@\textsc{Schnitzler, Arthur}!zzzSalten, Felix@\emph{von Felix Salten}!1899-07-273@{27. 7. 1899}|(be}
\toendnotes[C]{\smallbreak\pagebreak[2]}\Standort{CUL, Schnitzler, B 89, A 2.}
\physDesc{Brief, 1 Blatt, 3 Seiten, 1187 Zeichen
\newline{}Handschrift: Bleistift, lateinische Kurrent
\newline{}Ordnung: mit Bleistift von unbekannter Hand nummeriert: »119« }\toendnotes[C]{\smallbreak}
\pstart
           \raggedleft{}{\pb}Wien\oindex{Wien@\textbf{Wien}, \emph{A.ADM2}|pw}, 27. Juli 99\pend
           \vspace{0.5em}
\pstart
           Lieber Freund, ich war jetzt ein paar Tage in Unterach\oindex{Unterach am Attersee@\textbf{Unterach am Attersee}, \emph{P.PPL}|pw}, wo die Otti\pwindex{Salten, Ottilie 07.03.1868 – 22.06.1942@\textsc{Salten, Ottilie} (07.03.1868 – 22.06.1942), \emph{Schauspieler/Schauspielerin}|pw}
               wohnt. Nun bin ich wieder hier, und plage mich mit der \label{K_L03295-1v}\edtext{W\textsuperscript{r} Allg Rundschau\pwindex{Wiener Allgemeine Rundschau@\emph{Wiener Allgemeine Rundschau}|pw}}{\lemma{\textnormal{\emph{W\textsuperscript{r} Allg Rundschau}}}\Cendnote{\textnormal{Siehe Felix Salten an Arthur Schnitzler, 21. 6. 1899.
               }}}\label{K_L03295-1}, die weder mir, noch dem D\textsuperscript{r}{ }Szeps\pwindex{Szeps, Moriz 04.11.1834 – 09.08.1902@\textsc{Szeps, Moriz} (04.11.1834 – 09.08.1902), \emph{Journalist/Journalistin}|pw} noch den Abonnenten Freude macht. Den
               Abonnenten nicht, weil sie literarisch ist, dem D\textsuperscript{r}{ }Szeps\pwindex{Szeps, Moriz 04.11.1834 – 09.08.1902@\textsc{Szeps, Moriz} (04.11.1834 – 09.08.1902), \emph{Journalist/Journalistin}|pw} nicht, weil die Abonnenten murren, und
               mir nicht, weil ich nun schon mit meinem Namen dabei bin, und es nicht gerne schlecht
               machen möchte. Mich verstimmt das einigermaßen, wie Sie wol denken können. Mit \label{K_L03295-2v}\edtext{Geiringer\pwindex{Geiringer, Leopold 27.06.1851 – 29.05.1900@\textsc{Geiringer, Leopold} (27.06.1851 – 29.05.1900), \emph{Schriftsteller/Schriftstellerin, Dramaturg/Dramaturgin}|pwu} ist es nichts. Es ist ganz
               wirr und nicht einen Menschen, der für Geirin{\pb}gers\pwindex{Geiringer, Leopold 27.06.1851 – 29.05.1900@\textsc{Geiringer, Leopold} (27.06.1851 – 29.05.1900), \emph{Schriftsteller/Schriftstellerin, Dramaturg/Dramaturgin}|pwu} Ideen
               Geld verlieren möchte. Deshalb sein Plan mit Beer-Hofmann\pwindex{Beer-Hofmann, Richard 1866-07-11 – 1945-09-26@\textsc{Beer-Hofmann, Richard} (1866-07-11 – 1945-09-26), \emph{Schriftsteller/Schriftstellerin}|pw}}{\lemma{\textnormal{\emph{Geiringer … Beer-Hofmann}}}\Cendnote{\textnormal{Eventuell ist der Schriftsteller und Dramaturg 
                  Leopold Geiringer\pwindex{Geiringer, Leopold 27.06.1851 – 29.05.1900@\textsc{Geiringer, Leopold} (27.06.1851 – 29.05.1900), \emph{Schriftsteller/Schriftstellerin, Dramaturg/Dramaturgin}|pwk}gemeint. Womöglich sollte
                  mit 
                  Beer-Hofmann\pwindex{Beer-Hofmann, Richard 1866-07-11 – 1945-09-26@\textsc{Beer-Hofmann, Richard} (1866-07-11 – 1945-09-26), \emph{Schriftsteller/Schriftstellerin}|pwk} ein Finanzier für ein neues Zeitschriftenprojekt
                  gewonnen werden.}}}\label{K_L03295-2}! Von mir verlangt
               er, ich solle ihm einen Capitalisten schaffen. Dann will er mir eine Redactionsstelle gegen – Gewinnstantheil –
               verleihen!!\pend
           
\pstart
           Ich arbeite wenig, denn die Zeitung\pwindex{Wiener Allgemeine Montags-Zeitung@\emph{Wiener Allgemeine Montags-Zeitung}|pwv}\pwindex{Wiener Allgemeine Zeitung@\emph{Wiener Allgemeine Zeitung}|pwv} macht mir viel Kopfzerbrechen und auch
               sonst kommt wieder einmal viel auf einmal zusammen. In ein paar Tagen fahre ich
               wieder nach Unterach\oindex{Unterach am Attersee@\textbf{Unterach am Attersee}, \emph{P.PPL}|pw}. Schreiben Sie mir aber
                  imm\textcolor{gray}{er}hin nur hierher. Das \label{K_L03295-3v}\edtext{Feuilleton\pwindex{?? [Feuilleton ueber Paul Goldmann]@\emph{?? [Feuilleton über Paul Goldmann]}|pwv} über Goldmann\pwindex{Goldmann, Paul 31.01.1865 – 25.09.1935@\textsc{Goldmann, Paul} (31.01.1865 – 25.09.1935), \emph{Schriftsteller/Schriftstellerin, Journalist/Journalistin}|pw}}{\lemma{\textnormal{\emph{Feuilleton über Goldmann}}}\Cendnote{\textnormal{In der \emph{Wiener
                  Allgemeinen Montags-Zeitung}\pwindex{Wiener Allgemeine Montags-Zeitung@\emph{Wiener Allgemeine Montags-Zeitung}|pwk} erschien kein Feuilleton über Goldmann\pwindex{Goldmann, Paul 31.01.1865 – 25.09.1935@\textsc{Goldmann, Paul} (31.01.1865 – 25.09.1935), \emph{Schriftsteller/Schriftstellerin, Journalist/Journalistin}|pwk}. Im November und Dezember 1899
                  sind zwei längere Auszüge aus Goldmanns\pwindex{Goldmann, Paul 31.01.1865 – 25.09.1935@\textsc{Goldmann, Paul} (31.01.1865 – 25.09.1935), \emph{Schriftsteller/Schriftstellerin, Journalist/Journalistin}|pwk} Reisebericht \emph{Ein Sommer in
                     China}\pwindex{Sommer in China. Reisebilder@\emph{Ein Sommer in China. Reisebilder}|pwk} erschienen, aber diese dürften hier nicht gemeint sein. Mutmaßlich
                  hat Goldmann\pwindex{Goldmann, Paul 31.01.1865 – 25.09.1935@\textsc{Goldmann, Paul} (31.01.1865 – 25.09.1935), \emph{Schriftsteller/Schriftstellerin, Journalist/Journalistin}|pwk} sich auf eine
                  Vermittlungsposition beschränkt und das »über« ist als ›ein über
                  Vermittlung von Goldmann\pwindex{Goldmann, Paul 31.01.1865 – 25.09.1935@\textsc{Goldmann, Paul} (31.01.1865 – 25.09.1935), \emph{Schriftsteller/Schriftstellerin, Journalist/Journalistin}|pwk} erhaltenes
                  Feuilleton‹ zu lesen. Die Ausgabe vom 7. 8. 1899
                  behandelte etwa ausführlich den aktuellen Stand der Dreyfus\pwindex{Dreyfus, Alfred 1859-10-09 – 1935-07-12@\textsc{Dreyfus, Alfred} (1859-10-09 – 1935-07-12), \emph{Militär/Militärin}|pwk}-Affäre, über die auch Goldmann\pwindex{Goldmann, Paul 31.01.1865 – 25.09.1935@\textsc{Goldmann, Paul} (31.01.1865 – 25.09.1935), \emph{Schriftsteller/Schriftstellerin, Journalist/Journalistin}|pwk} berichtet hat. Auch sind in dem Blatt\pwindex{Wiener Allgemeine Montags-Zeitung@\emph{Wiener Allgemeine Montags-Zeitung}|pwkv} in der kurzen Zeit seines
                  Bestehens mehrere Texte von fran\oindex{Frankreich@\textbf{Frankreich}, \emph{A.PCLI}|pwkv}zösischen Autoren erschienen, mit denen Goldmann\pwindex{Goldmann, Paul 31.01.1865 – 25.09.1935@\textsc{Goldmann, Paul} (31.01.1865 – 25.09.1935), \emph{Schriftsteller/Schriftstellerin, Journalist/Journalistin}|pwk} bereits 1893/1894 in der \emph{Frankfurter
                     Zeitung}\pwindex{Frankfurter Zeitung@\emph{Frankfurter Zeitung}|pwk} die Feuilletonreihe \emph{Neue
                     französische Humoristen}\pwindex{Neue franzoesische Humoristen@\emph{Neue französische Humoristen}|pwk} bestritten hatte (siehe Paul Goldmann an Arthur Schnitzler, 7. 9. [1896]).}}}\label{K_L03295-3} erscheint in den nächsten Tagen. Ich
               sende {\pb}es\pwindex{?? [Feuilleton ueber Paul Goldmann]@\emph{?? [Feuilleton über Paul Goldmann]}|pwv} Ihnen gleich.\pend
           
\pstart
           Auf Wiedersehen: hoffentlich bald. \label{K_L03295-4v}\edtext{Grüßen Sie Wassermann\pwindex{Wassermann, Jakob 10.03.1873 – 01.01.1934@\textsc{Wassermann, Jakob} (10.03.1873 – 01.01.1934), \emph{Schriftsteller/Schriftstellerin}|pw} und den emsigen Richard\pwindex{Beer-Hofmann, Richard 1866-07-11 – 1945-09-26@\textsc{Beer-Hofmann, Richard} (1866-07-11 – 1945-09-26), \emph{Schriftsteller/Schriftstellerin}|pw}}{\lemma{\textnormal{\emph{Grüßen … Richard}}}\Cendnote{\textnormal{Jakob Wassermann\pwindex{Wassermann, Jakob 10.03.1873 – 01.01.1934@\textsc{Wassermann, Jakob} (10.03.1873 – 01.01.1934), \emph{Schriftsteller/Schriftstellerin}|pwk} hielt sich gemeinsam mit
                     Schnitzler in Velden am Wörthersee\oindex{Velden am Woerthersee@\textbf{Velden am Wörthersee}, \emph{P.PPL}|pwk} auf. Am 28. 7. 1899 reisten sie weiter nach Villach\oindex{Villach@\textbf{Villach}, \emph{A.ADM3}|pwk}. Richard Beer-Hofmann\pwindex{Beer-Hofmann, Richard 1866-07-11 – 1945-09-26@\textsc{Beer-Hofmann, Richard} (1866-07-11 – 1945-09-26), \emph{Schriftsteller/Schriftstellerin}|pwk} hielt sich im nahegelegenen Seeboden\oindex{Seeboden@\textbf{Seeboden}, \emph{A.ADM3}|pwk} auf und traf Schnitzler in dieser Zeit ebenso. Am 5. 8. 1899 starteten Schnitzler, Wassermann\pwindex{Wassermann, Jakob 10.03.1873 – 01.01.1934@\textsc{Wassermann, Jakob} (10.03.1873 – 01.01.1934), \emph{Schriftsteller/Schriftstellerin}|pwk} und Beer-Hofmann\pwindex{Beer-Hofmann, Richard 1866-07-11 – 1945-09-26@\textsc{Beer-Hofmann, Richard} (1866-07-11 – 1945-09-26), \emph{Schriftsteller/Schriftstellerin}|pwk} in Niederdorf\oindex{Niederdorf@\textbf{Niederdorf}, \emph{P.PPLA3}|pwk} eine mehrtätige gemeinsame
                  Wanderung.}}}\label{K_L03295-4}. Frl. Metzl\pwindex{Salten, Ottilie 07.03.1868 – 22.06.1942@\textsc{Salten, Ottilie} (07.03.1868 – 22.06.1942), \emph{Schauspieler/Schauspielerin}|pw} grüßt
               Sie.\pend
           
\pstart
           Herzlichst {\\[\baselineskip]}Ihr {\\[\baselineskip]}\spacefill\mbox{Salten}\pend
           \leftskip=0em{}\selectlanguage{ngerman}\endnumbering\briefempfaengerindex{Schnitzler, Arthur@\textsc{Schnitzler, Arthur}!zzzSalten, Felix@\emph{von Felix Salten}!1899-07-273@{27. 7. 1899}|)be}\mylabel{L03295h}  \normalsize

\doendnotes{C}
\bigskip
\vfill

\clearpage

\footnotesize

\lohead{\textsc{register}}

% Definiere theindex-Environment komplett neu ohne reledmac
\makeatletter
\renewenvironment{theindex}{%
  \section*{\indexname}%
  \setlength{\parindent}{0pt}%
  \setlength{\parskip}{0pt plus 0.3pt}%
  \let\item\@idxitem
}{%
  \clearpage
}
\makeatother

\IfFileExists{\jobname-pw.ind}{\input{\jobname-pw.ind}}{}

\end{document}

      