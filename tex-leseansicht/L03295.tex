%% latex-leseansicht-vorspann.tex
%% Vorspann für die Leseansicht.
%% Lädt die gemeinsame Datei latex-vorspann.tex mit nicht gesetztem Schalter.

\newif\ifkorrekturansicht
\korrekturansichtfalse

\input{../tex-inputs/latex-vorspann}

\begin{center}
            \textcolor{red}{ENTWURF, NICHT FERTIG KORRIGIERT}
                      \end{center}
            
         
         \renewcommand{\erwaehntePersonen}{Personen: Richard Beer-Hofmann, Leopold Geiringer, Paul Goldmann, Ottilie Salten, Moriz Szeps, Jakob Wassermann}
         \renewcommand{\erwaehnteInstitutionen}{Institutionen: Wiener Allgemeine Montagszeitung}
         \renewcommand{\erwaehnteOrte}{Orte: Unterach am Attersee, Wien}
         \renewcommand{\erwaehnteWerke}{Werke: ?? [Feuilleton über Paul Goldmann]}
               \section[Felix Salten an Arthur Schnitzler, 27. 7. 1899]{ Felix Salten an Arthur Schnitzler, 27. 7. 1899}\nopagebreak\mylabel{v}\rehead{ }\begin{ledgroupsized}[t]{13cm}\normalsize\beginnumbering \toendnotes[C]{\smallbreak\pagebreak[2]} \Standort{CUL, Schnitzler, B 89, A 2.}
\physDesc{Brief, 1 Blatt, 3 Seiten
\newline{}Handschrift: Bleistift, lateinische Kurrent\newline{}Ordnung: mit Bleistift von unbekannter Hand nummeriert:
                                    »119« }\toendnotes[C]{\smallbreak}\pstart
           \raggedleft{}{\pb}Wien\oindex{Wien@\textbf{Wien}|pw}, 27. Juli 99\pend
           \pstart
           Lieber Freund, ich war jetzt ein paar Tage in Unterach\oindex{Unterach am Attersee@\textbf{Unterach am Attersee}|pw}, wo die Otti\pwindex{Salten, Ottilie 07.03.1868 – 22.06.1942@\textsc{Salten, Ottilie} (07.03.1868 – 22.06.1942), \emph{Schauspielerin}|pw}
               wohnt. Nun bin ich wieder hier, und plage mich mit der Wr Allg Rundschau\orgindex{Wiener Allgemeine Montagszeitung@Wiener Allgemeine Montagszeitung|pw}, die weder mir, noch dem D\textsuperscript{r}{ }Szeps\pwindex{Szeps, Moriz 04.11.1834 – 09.08.1902@\textsc{Szeps, Moriz} (04.11.1834 – 09.08.1902), \emph{Journalist}|pw} noch den Abonnenten Freude macht. Den Abonnenten
               nicht, weil sie literarisch ist, dem D\textsuperscript{r}{ }Szeps\pwindex{Szeps, Moriz 04.11.1834 – 09.08.1902@\textsc{Szeps, Moriz} (04.11.1834 – 09.08.1902), \emph{Journalist}|pw} nicht, weil die Abonnenten murren, und mir nicht,
               weil ich nun schon mit meinem Namen dabei bin, und es nicht gerne schlecht machen
               möchte. Mich verstimmt das einigermaßen, wie Sie wol denken können. Mit Geiringer\pwindex{Geiringer, Leopold 27.06.1851 – 29.05.1900@\textsc{Geiringer, Leopold} (27.06.1851 – 29.05.1900), \emph{Schriftsteller, Dramaturg}|pwu} ist es nichts. Es ist ganz wirr und
               nicht einen Menschen, der für Geirin{\pb}ger\pwindex{Geiringer, Leopold 27.06.1851 – 29.05.1900@\textsc{Geiringer, Leopold} (27.06.1851 – 29.05.1900), \emph{Schriftsteller, Dramaturg}|pwu}s Ideen Geld
               verlieren möchte. Deshalb sein Plan mit Beer-Hofmann\pwindex{Beer-Hofmann, Richard 1866-07-11 – 1945-09-26@\textsc{Beer-Hofmann, Richard} (1866-07-11 – 1945-09-26), \emph{Schriftsteller}|pw}! Von mir verlangt er, ich soll ihm einen Capitalisten schaffen.
               Dann will er mir eine Redactionsstelle gegen – Gewinnstantheil – verleihen!! \pend
           \pstart
           Ich arbeite wenig, denn die Zeitung\orgindex{Wiener Allgemeine Montagszeitung@Wiener Allgemeine Montagszeitung|pwv}
               macht mir viel Kopfzerbrechen und auch sonst kommt wieder einmal viel auf einmal
               zusammen. In ein paar Tagen fahre ich wieder nach Unterach\oindex{Unterach am Attersee@\textbf{Unterach am Attersee}|pw}. \pend
           \pstart
           Schreiben Sie mir aber imm\textcolor{gray}{er}hin nur hierher. Das Feuilleton\pwindex{Salten, Felix 06.09.1869 – 08.10.1945@\textsc{Salten, Felix} (06.09.1869 – 08.10.1945), \emph{Schriftsteller, Journalist}!?? [Feuilleton ueber Paul Goldmann]Ende Juli/Anfang August 1899@\strich\emph{?? [Feuilleton über Paul Goldmann]} {[}Ende Juli/Anfang August 1899{]}|pwv} über Goldmann\pwindex{Goldmann, Paul 31.01.1865 – 25.09.1935@\textsc{Goldmann, Paul} (31.01.1865 – 25.09.1935), \emph{Schriftsteller, Journalist}|pw} erscheint in den nächsten Tagen. Ich sende {\pb}es Ihnen gleich. \pend
           \pstart
           Auf Wiedersehen, hoffentlich bald. Grüßen Sie Wassermann\pwindex{Wassermann, Jakob 10.03.1873 – 01.01.1934@\textsc{Wassermann, Jakob} (10.03.1873 – 01.01.1934), \emph{Schriftsteller}|pw} und den emsigen Richard\pwindex{Beer-Hofmann, Richard 1866-07-11 – 1945-09-26@\textsc{Beer-Hofmann, Richard} (1866-07-11 – 1945-09-26), \emph{Schriftsteller}|pw}.
               Frl. Metzl\pwindex{Salten, Ottilie 07.03.1868 – 22.06.1942@\textsc{Salten, Ottilie} (07.03.1868 – 22.06.1942), \emph{Schauspielerin}|pw} grüßt Sie. \pend
           \pstart
           Herzlichst {\\[\baselineskip]}Ihr {\\[\baselineskip]}\spacefill\mbox{Salten}\pend
           \leftskip=0em{}
         
         \endnumbering\mylabel{h}\end{ledgroupsized}\begin{anhang}\end{anhang}\newcommand{\dateiname}{L03295}\newcommand{\titel}{Felix Salten an Arthur Schnitzler, 27. 7. 1899}\newcommand{\editorInnen}{Martin Anton Müller und Laura Untner}%% latex-leseansicht-abspann.tex
%% Abspann für die Leseansicht.
%% Der Schalter \ifkorrekturansicht ist bereits durch den Vorspann gesetzt.

%% latex-abspann.tex
%% Gemeinsamer Abspann für Korrekturansicht und Leseansicht.
%% Setzt den Schalter \ifkorrekturansicht voraus (gesetzt in den
%% einbindenden Dateien latex-korrekturansicht-abspann.tex bzw.
%% latex-leseansicht-abspann.tex).
%% ---------------------------------------------------------------

\normalsize

% Das esempio-Environment wird nur in der Leseansicht benötigt
\ifkorrekturansicht\else
\newenvironment{esempio}[3]%
{
    \vspace{1.5ex}
    \rlap{\underline{#1}}
    \par
    \setlength{\parindent}{0cm}
    \nopagebreak
    \leftskip=#2cm
    \rightskip=#3cm
}
{
    \par
}
\fi

\doendnotes{C}
\bigskip
\vfill

\clearpage

\footnotesize

\ifkorrekturansicht
  \lohead{\textsc{register}}
\fi

% theindex-Environment neu definieren ohne reledmac
\makeatletter
\renewenvironment{theindex}{%
  \ifkorrekturansicht
    \section*{\indexname}%
  \else
    \subsubsection*{Index der erwähnten Entitäten}%
  \fi
  \setlength{\parindent}{0pt}%
  \setlength{\parskip}{0pt plus 0.3pt}%
  \let\item\@idxitem
}{%
  \ifkorrekturansicht\clearpage\fi
}
\makeatother

\IfFileExists{\jobname-pw.ind}{\input{\jobname-pw.ind}}{}

% Quellenangabe nur in der Leseansicht
\ifkorrekturansicht\else
% Fallback-Definitionen, falls die .tex-Datei \titel etc. nicht gesetzt hat
\providecommand{\titel}{}
\providecommand{\editorInnen}{}
\providecommand{\dateiname}{\jobname}

\vspace{3cm}

\vfill

\footnotesize
\textsc{Quelle}: \titel. Herausgegeben von {\editorInnen}. In: \emph{Arthur Schnitzler: Briefwechsel mit Autorinnen und Autoren}.
 Digitale Edition, https://schnitzler-briefe.acdh.oeaw.ac.at/{\dateiname}.html (Stand \today)
\fi

\end{document}


      