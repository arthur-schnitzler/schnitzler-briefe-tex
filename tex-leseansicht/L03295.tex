%% latex-leseansicht-vorspann.tex
%% Vorspann für die Leseansicht.
%% Lädt die gemeinsame Datei latex-vorspann.tex mit nicht gesetztem Schalter.

\newif\ifkorrekturansicht
\korrekturansichtfalse

\input{../tex-inputs/latex-vorspann}


\section[ Felix Salten an Arthur Schnitzler, 27. 7. 1899]{L03295 Felix Salten an Arthur Schnitzler,  27. 7. 1899}
\nopagebreak\mylabel{L03295v}
\rehead{ }\normalsize\beginnumbering\briefempfaengerindex{Schnitzler, Arthur@\textsc{Schnitzler, Arthur}!zzzSalten, Felix@\emph{von Felix Salten}!1899-07-274@{27. 7. 1899}|(be}
\toendnotes[C]{\smallbreak\pagebreak[2]}
\correspDesc{Versand  durch Felix Salten am 27. 7. 1899 in Wien
\newline{}Erhalt  durch Arthur Schnitzler im Zeitraum [28. 7. 1899
                  – 1. 8. 1899?] in Velden am Wörthersee}\toendnotes[C]{\smallbreak}
\Standort{CUL, Schnitzler, B 89, A 2.}
\physDesc{Brief, 1 Blatt, 3 Seiten, 1186 Zeichen
\newline{}Handschrift: Bleistift, lateinische Kurrent
\newline{}Ordnung: mit Bleistift von unbekannter Hand nummeriert: »119« }\toendnotes[C]{\smallbreak}
\pstart
           \raggedleft{}{\pb}Wien\oindex{Wien@\textbf{Wien}, \emph{Verwaltungsgebiet}|pw}, 27. Juli 99\pend
           \vspace{0.5em}
\pstart
           Lieber Freund, ich war jetzt ein paar Tage in Unterach\oindex{Unterach am Attersee@\textbf{Unterach am Attersee}|pw}, wo die Otti\pwindex{Salten, Ottilie 7.\,3.\,1868 Prag – 22.\,6.\,1942 Zürich@\textsc{Salten, Ottilie} (7.\,3.\,1868 Prag – 22.\,6.\,1942 Zürich), \emph{Schauspielerin}|pw}
               wohnt. Nun bin ich wieder hier, und plage mich mit der \label{K_L03295-1v}\edtext{W\textsuperscript{r} Allg Rundschau\pwindex{Wiener Allgemeine Rundschau@\emph{Wiener Allgemeine Rundschau}|pw}}{\lemma{\textnormal{\emph{W\textsuperscript{r} Allg Rundschau}}}\Cendnote{\textnormal{Siehe XXXX Auszeichnungsfehler: Dokument L03293 nicht gefunden.
               }}}\label{K_L03295-1}, die weder mir, noch dem D\textsuperscript{r}{ }Szeps\pwindex{Szeps, Moriz 4.\,11.\,1834 Busk – 9.\,8.\,1902 Wien@\textsc{Szeps, Moriz} (4.\,11.\,1834 Busk – 9.\,8.\,1902 Wien), \emph{Journalist}|pw} noch den Abonnenten Freude macht. Den
               Abonnenten nicht, weil sie literarisch ist, dem D\textsuperscript{r}{ }Szeps\pwindex{Szeps, Moriz 4.\,11.\,1834 Busk – 9.\,8.\,1902 Wien@\textsc{Szeps, Moriz} (4.\,11.\,1834 Busk – 9.\,8.\,1902 Wien), \emph{Journalist}|pw} nicht, weil die Abonnenten murren, und
               mir nicht, weil ich nun schon mit meinem Namen dabei bin, und es nicht gerne schlecht
               machen möchte. Mich verstimmt das einigermaßen, wie Sie wol denken können. Mit \label{K_L03295-2v}\edtext{Geiringer\pwindex{Geiringer, Leopold 27.\,6.\,1851 Wien – 29.\,5.\,1900 ebd.@\textsc{Geiringer, Leopold} (27.\,6.\,1851 Wien – 29.\,5.\,1900 ebd.), \emph{Schriftsteller, Dramaturg}|pwu} ist es nichts. Es ist ganz
               wirr und nicht einen Menschen, der für Geirin{\pb}gers\pwindex{Geiringer, Leopold 27.\,6.\,1851 Wien – 29.\,5.\,1900 ebd.@\textsc{Geiringer, Leopold} (27.\,6.\,1851 Wien – 29.\,5.\,1900 ebd.), \emph{Schriftsteller, Dramaturg}|pwu} Ideen
               Geld verlieren möchte. Deshalb sein Plan mit Beer-Hofmann\pwindex{Beer-Hofmann, Richard 11.\,7.\,1866 Wien – 26.\,9.\,1945 New York City@\textsc{Beer-Hofmann, Richard} (11.\,7.\,1866 Wien – 26.\,9.\,1945 New York City), \emph{Schriftsteller}|pw}}{\lemma{\textnormal{\emph{Geiringer … Beer-Hofmann}}}\Cendnote{\textnormal{Eventuell ist der Schriftsteller und Dramaturg 
                  Leopold Geiringer\pwindex{Geiringer, Leopold 27.\,6.\,1851 Wien – 29.\,5.\,1900 ebd.@\textsc{Geiringer, Leopold} (27.\,6.\,1851 Wien – 29.\,5.\,1900 ebd.), \emph{Schriftsteller, Dramaturg}|pwk}gemeint. Womöglich sollte
                  mit 
                  Beer-Hofmann\pwindex{Beer-Hofmann, Richard 11.\,7.\,1866 Wien – 26.\,9.\,1945 New York City@\textsc{Beer-Hofmann, Richard} (11.\,7.\,1866 Wien – 26.\,9.\,1945 New York City), \emph{Schriftsteller}|pwk} ein Finanzier für ein neues Zeitschriftenprojekt
                  gewonnen werden.}}}\label{K_L03295-2}! Von mir verlangt
               er, ich solle ihm einen Capitalisten schaffen. Dann will er mir eine Redactionsstelle gegen – Gewinnstantheil –
               verleihen!!\pend
           
\pstart
           Ich arbeite wenig, denn die Zeitung\pwindex{Wiener Allgemeine Montags-Zeitung@\emph{Wiener Allgemeine Montags-Zeitung}|pwv}\pwindex{Wiener Allgemeine Zeitung@\emph{Wiener Allgemeine Zeitung}|pwv} macht mir viel Kopfzerbrechen und auch
               sonst kommt wieder einmal viel auf einmal zusammen. In ein paar Tagen fahre ich
               wieder nach Unterach\oindex{Unterach am Attersee@\textbf{Unterach am Attersee}|pw}. Schreiben Sie mir aber
                  imm\textcolor{gray}{er}hin nur hierher. Das \label{K_L03295-3v}\edtext{Feuilleton\pwindex{Salten, Felix 6.\,9.\,1869 Budapest – 8.\,10.\,1945 Zürich@\textsc{Salten, Felix} (6.\,9.\,1869 Budapest – 8.\,10.\,1945 Zürich), \emph{Schriftsteller, Journalist, Chefredakteur}!?? [Feuilleton über Paul Goldmann]@\strich\emph{?? [Feuilleton über Paul Goldmann]}|pwv} über Goldmann\pwindex{Goldmann, Paul 31.\,1.\,1865 Breslau – 25.\,9.\,1935 Wien@\textsc{Goldmann, Paul} (31.\,1.\,1865 Breslau – 25.\,9.\,1935 Wien), \emph{Schriftsteller, Journalist}|pw}}{\lemma{\textnormal{\emph{Feuilleton über Goldmann}}}\Cendnote{\textnormal{In der \emph{Wiener
                  Allgemeinen Montags-Zeitung}\pwindex{Wiener Allgemeine Montags-Zeitung@\emph{Wiener Allgemeine Montags-Zeitung}|pwk} erschien kein Feuilleton über Goldmann\pwindex{Goldmann, Paul 31.\,1.\,1865 Breslau – 25.\,9.\,1935 Wien@\textsc{Goldmann, Paul} (31.\,1.\,1865 Breslau – 25.\,9.\,1935 Wien), \emph{Schriftsteller, Journalist}|pwk}. Im November und Dezember 1899
                  sind zwei längere Auszüge aus Goldmanns\pwindex{Goldmann, Paul 31.\,1.\,1865 Breslau – 25.\,9.\,1935 Wien@\textsc{Goldmann, Paul} (31.\,1.\,1865 Breslau – 25.\,9.\,1935 Wien), \emph{Schriftsteller, Journalist}|pwk} Reisebericht \emph{Ein Sommer in
                     China}\pwindex{Goldmann, Paul 31.\,1.\,1865 Breslau – 25.\,9.\,1935 Wien@\textsc{Goldmann, Paul} (31.\,1.\,1865 Breslau – 25.\,9.\,1935 Wien), \emph{Schriftsteller, Journalist}!Sommer in China. Reisebilder@\strich\emph{Ein Sommer in China. Reisebilder}|pwk} erschienen, aber diese dürften hier nicht gemeint sein. Mutmaßlich
                  hat Goldmann\pwindex{Goldmann, Paul 31.\,1.\,1865 Breslau – 25.\,9.\,1935 Wien@\textsc{Goldmann, Paul} (31.\,1.\,1865 Breslau – 25.\,9.\,1935 Wien), \emph{Schriftsteller, Journalist}|pwk} sich auf eine
                  Vermittlungsposition beschränkt und das »über« ist als ›ein über
                  Vermittlung von Goldmann\pwindex{Goldmann, Paul 31.\,1.\,1865 Breslau – 25.\,9.\,1935 Wien@\textsc{Goldmann, Paul} (31.\,1.\,1865 Breslau – 25.\,9.\,1935 Wien), \emph{Schriftsteller, Journalist}|pwk} erhaltenes
                  Feuilleton‹ zu lesen. Die Ausgabe vom 7. 8. 1899
                  behandelte etwa ausführlich den aktuellen Stand der Dreyfus\pwindex{Dreyfus, Alfred 9.\,10.\,1859 Mulhouse – 12.\,7.\,1935 Paris@\textsc{Dreyfus, Alfred} (9.\,10.\,1859 Mulhouse – 12.\,7.\,1935 Paris), \emph{Militär}|pwk}-Affäre, über die auch Goldmann\pwindex{Goldmann, Paul 31.\,1.\,1865 Breslau – 25.\,9.\,1935 Wien@\textsc{Goldmann, Paul} (31.\,1.\,1865 Breslau – 25.\,9.\,1935 Wien), \emph{Schriftsteller, Journalist}|pwk} berichtet hat. Auch sind in dem Blatt\pwindex{Wiener Allgemeine Montags-Zeitung@\emph{Wiener Allgemeine Montags-Zeitung}|pwkv} in der kurzen Zeit seines
                  Bestehens mehrere Texte von fran\oindex{Frankreich@\textbf{Frankreich}|pwkv}zösischen Autoren erschienen, mit denen Goldmann\pwindex{Goldmann, Paul 31.\,1.\,1865 Breslau – 25.\,9.\,1935 Wien@\textsc{Goldmann, Paul} (31.\,1.\,1865 Breslau – 25.\,9.\,1935 Wien), \emph{Schriftsteller, Journalist}|pwk} bereits 1893/1894 in der \emph{Frankfurter
                     Zeitung}\pwindex{Frankfurter Zeitung@\emph{Frankfurter Zeitung}|pwk} die Feuilletonreihe \emph{Neue
                     französische Humoristen}\pwindex{Goldmann, Paul 31.\,1.\,1865 Breslau – 25.\,9.\,1935 Wien@\textsc{Goldmann, Paul} (31.\,1.\,1865 Breslau – 25.\,9.\,1935 Wien), \emph{Schriftsteller, Journalist}!Neue französische Humoristen@\strich\emph{Neue französische Humoristen}|pwk} bestritten hatte (siehe XXXX Auszeichnungsfehler: Dokument L02784 nicht gefunden).}}}\label{K_L03295-3} erscheint in den nächsten Tagen. Ich
               sende {\pb}es\pwindex{Salten, Felix 6.\,9.\,1869 Budapest – 8.\,10.\,1945 Zürich@\textsc{Salten, Felix} (6.\,9.\,1869 Budapest – 8.\,10.\,1945 Zürich), \emph{Schriftsteller, Journalist, Chefredakteur}!?? [Feuilleton über Paul Goldmann]@\strich\emph{?? [Feuilleton über Paul Goldmann]}|pwv} Ihnen gleich.\pend
           
\pstart
           Auf Wiedersehen: hoffentlich bald. \label{K_L03295-4v}\edtext{Grüßen Sie Wassermann\pwindex{Wassermann, Jakob 10.\,3.\,1873 Fürth – 1.\,1.\,1934 Altaussee@\textsc{Wassermann, Jakob} (10.\,3.\,1873 Fürth – 1.\,1.\,1934 Altaussee), \emph{Schriftsteller}|pw} und den emsigen Richard\pwindex{Beer-Hofmann, Richard 11.\,7.\,1866 Wien – 26.\,9.\,1945 New York City@\textsc{Beer-Hofmann, Richard} (11.\,7.\,1866 Wien – 26.\,9.\,1945 New York City), \emph{Schriftsteller}|pw}}{\lemma{\textnormal{\emph{Grüßen … Richard}}}\Cendnote{\textnormal{Jakob Wassermann\pwindex{Wassermann, Jakob 10.\,3.\,1873 Fürth – 1.\,1.\,1934 Altaussee@\textsc{Wassermann, Jakob} (10.\,3.\,1873 Fürth – 1.\,1.\,1934 Altaussee), \emph{Schriftsteller}|pwk} hielt sich gemeinsam mit
                   Schnitzler in Velden am Wörthersee\oindex{Velden am Wörthersee@\textbf{Velden am Wörthersee}|pwk} auf. Am 28. 7. 1899 reisten sie weiter nach Villach\oindex{Villach@\textbf{Villach}, \emph{Verwaltungsgebiet}|pwk}. Richard Beer-Hofmann\pwindex{Beer-Hofmann, Richard 11.\,7.\,1866 Wien – 26.\,9.\,1945 New York City@\textsc{Beer-Hofmann, Richard} (11.\,7.\,1866 Wien – 26.\,9.\,1945 New York City), \emph{Schriftsteller}|pwk} hielt sich im nahegelegenen Seeboden\oindex{Seeboden@\textbf{Seeboden}, \emph{Verwaltungsgebiet}|pwk} auf und traf Schnitzler in dieser Zeit ebenso. Am 5. 8. 1899 starteten Schnitzler, Wassermann\pwindex{Wassermann, Jakob 10.\,3.\,1873 Fürth – 1.\,1.\,1934 Altaussee@\textsc{Wassermann, Jakob} (10.\,3.\,1873 Fürth – 1.\,1.\,1934 Altaussee), \emph{Schriftsteller}|pwk} und Beer-Hofmann\pwindex{Beer-Hofmann, Richard 11.\,7.\,1866 Wien – 26.\,9.\,1945 New York City@\textsc{Beer-Hofmann, Richard} (11.\,7.\,1866 Wien – 26.\,9.\,1945 New York City), \emph{Schriftsteller}|pwk} in Niederdorf\oindex{Niederdorf@\textbf{Niederdorf}, \emph{Verwaltungsgebiet}|pwk} eine mehrtätige gemeinsame
                  Wanderung.}}}\label{K_L03295-4}. Frl. Metzl\pwindex{Salten, Ottilie 7.\,3.\,1868 Prag – 22.\,6.\,1942 Zürich@\textsc{Salten, Ottilie} (7.\,3.\,1868 Prag – 22.\,6.\,1942 Zürich), \emph{Schauspielerin}|pw} grüßt
               Sie.\pend
           
\pstart
           Herzlichst {\\[\baselineskip]}Ihr {\\[\baselineskip]}\spacefill\mbox{Salten}\pend
           \leftskip=0em{}\selectlanguage{ngerman}\endnumbering\briefempfaengerindex{Schnitzler, Arthur@\textsc{Schnitzler, Arthur}!zzzSalten, Felix@\emph{von Felix Salten}!1899-07-274@{27. 7. 1899}|)be}\mylabel{L03295h}  \newcommand{\dateiname}{L03295}\newcommand{\titel}{Felix Salten an Arthur Schnitzler, 27. 7. 1899}\newcommand{\editorInnen}{Martin Anton Müller und Laura Untner}%% latex-leseansicht-abspann.tex
%% Abspann für die Leseansicht.
%% Der Schalter \ifkorrekturansicht ist bereits durch den Vorspann gesetzt.

%% latex-abspann.tex
%% Gemeinsamer Abspann für Korrekturansicht und Leseansicht.
%% Setzt den Schalter \ifkorrekturansicht voraus (gesetzt in den
%% einbindenden Dateien latex-korrekturansicht-abspann.tex bzw.
%% latex-leseansicht-abspann.tex).
%% ---------------------------------------------------------------

\normalsize

% Das esempio-Environment wird nur in der Leseansicht benötigt
\ifkorrekturansicht\else
\newenvironment{esempio}[3]%
{
    \vspace{1.5ex}
    \rlap{\underline{#1}}
    \par
    \setlength{\parindent}{0cm}
    \nopagebreak
    \leftskip=#2cm
    \rightskip=#3cm
}
{
    \par
}
\fi

\doendnotes{C}
\bigskip
\vfill

\clearpage

\footnotesize

\ifkorrekturansicht
  \lohead{\textsc{register}}
\fi

% theindex-Environment neu definieren ohne reledmac
\makeatletter
\renewenvironment{theindex}{%
  \ifkorrekturansicht
    \section*{\indexname}%
  \else
    \subsubsection*{Index der erwähnten Entitäten}%
  \fi
  \setlength{\parindent}{0pt}%
  \setlength{\parskip}{0pt plus 0.3pt}%
  \let\item\@idxitem
}{%
  \ifkorrekturansicht\clearpage\fi
}
\makeatother

\IfFileExists{\jobname-pw.ind}{\input{\jobname-pw.ind}}{}

% Quellenangabe nur in der Leseansicht
\ifkorrekturansicht\else
% Fallback-Definitionen, falls die .tex-Datei \titel etc. nicht gesetzt hat
\providecommand{\titel}{}
\providecommand{\editorInnen}{}
\providecommand{\dateiname}{\jobname}

\vspace{3cm}

\vfill

\footnotesize
\textsc{Quelle}: \titel. Herausgegeben von {\editorInnen}. In: \emph{Arthur Schnitzler: Briefwechsel mit Autorinnen und Autoren}.
 Digitale Edition, https://schnitzler-briefe.acdh.oeaw.ac.at/{\dateiname}.html (Stand \today)
\fi

\end{document}


