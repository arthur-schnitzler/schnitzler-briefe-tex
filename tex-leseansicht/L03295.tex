%% latex-leseansicht-vorspann.tex
%% Vorspann für die Leseansicht.
%% Lädt die gemeinsame Datei latex-vorspann.tex mit nicht gesetztem Schalter.

\newif\ifkorrekturansicht
\korrekturansichtfalse

\input{../tex-inputs/latex-vorspann}


         
         \renewcommand{\erwaehntePersonen}{Personen: Richard Beer-Hofmann, Alfred Dreyfus, Leopold Geiringer, Paul Goldmann, Felix Salten, Ottilie Salten, Moriz Szeps, Jakob Wassermann}
         \renewcommand{\erwaehnteInstitutionen}{Institutionen: Wiener Allgemeine Montags-Zeitung}
         \renewcommand{\erwaehnteOrte}{Orte: Frankreich, Niederdorf, Seeboden, Unterach am Attersee, Velden am Wörthersee, Villach, Wien}
         \renewcommand{\erwaehnteWerke}{Werke: ?? [Feuilleton über Paul Goldmann], Ein Sommer in China. Reisebilder, Frankfurter Zeitung, Neue französische Humoristen, Wiener Allgemeine Montags-Zeitung, Wiener Allgemeine Rundschau, Wiener Allgemeine Zeitung}
               \section[ Felix Salten an Arthur Schnitzler, 27. 7. 1899]{ Felix Salten an Arthur Schnitzler, 27. 7. 1899}\nopagebreak\mylabel{v}\rehead{ }\begin{ledgroupsized}[t]{13cm}\normalsize\beginnumbering\briefempfaengerindex{Schnitzler, Arthur@\textsc{Schnitzler, Arthur}!zzzSalten, Felix@\emph{von Felix Salten}!1899-07-273@{27. 7. 1899}|(be} \toendnotes[C]{\smallbreak\pagebreak[2]} \Standort{CUL, Schnitzler, B 89, A 2.}
\physDesc{Brief, 1 Blatt, 3 Seiten, 1184 Zeichen
\newline{}Handschrift: Bleistift, lateinische Kurrent
\newline{}Ordnung: mit Bleistift von unbekannter Hand nummeriert: »119« }\toendnotes[C]{\smallbreak}\pstart
           \raggedleft{}{\pb}Wien\oindex{Wien@\textbf{Wien}|pw}, 27. Juli 99\pend
           \pstart
           Lieber Freund, ich war jetzt ein paar Tage in Unterach\oindex{Unterach am Attersee@\textbf{Unterach am Attersee}|pw}, wo die Otti\pwindex{Salten, Ottilie 07.03.1868 – 22.06.1942@\textsc{Salten, Ottilie} (07.03.1868 – 22.06.1942), \emph{Schauspielerin}|pw}
               wohnt. Nun bin ich wieder hier, und plage mich mit der \label{K_L03295-1v}\edtext{W\textsuperscript{r} Allg Rundschau\pwindex{Wiener Allgemeine Rundschau1899-07-03 – 1899-12-18@\emph{Wiener Allgemeine Rundschau} {[}1899-07-03 – 1899-12-18{]}|pw}}{\lemma{\textnormal{\emph{W\textsuperscript{r} Allg Rundschau}}}\Cendnote{\textnormal{siehe Felix Salten an Arthur Schnitzler, 21. 6. 1899}}}\label{K_L03295-1h}, die weder mir, noch dem D\textsuperscript{r}{ }Szeps\pwindex{Szeps, Moriz 04.11.1834 – 09.08.1902@\textsc{Szeps, Moriz} (04.11.1834 – 09.08.1902), \emph{Journalist}|pw} noch den Abonnenten Freude macht. Den
               Abonnenten nicht, weil sie literarisch ist, dem D\textsuperscript{r}{ }Szeps\pwindex{Szeps, Moriz 04.11.1834 – 09.08.1902@\textsc{Szeps, Moriz} (04.11.1834 – 09.08.1902), \emph{Journalist}|pw} nicht, weil die Abonnenten murren, und
               mir nicht, weil ich nun schon mit meinem Namen dabei bin, und es nicht gerne schlecht
               machen möchte. Mich verstimmt das einigermaßen, wie Sie wol denken können. Mit \label{K_L03295-3v}\edtext{Geiringer\pwindex{Geiringer, Leopold 27.06.1851 – 29.05.1900@\textsc{Geiringer, Leopold} (27.06.1851 – 29.05.1900), \emph{Schriftsteller, Dramaturg}|pwu} ist es nichts. Es ist ganz
               wirr und nicht einen Menschen, der für Geirin{\pb}ger\pwindex{Geiringer, Leopold 27.06.1851 – 29.05.1900@\textsc{Geiringer, Leopold} (27.06.1851 – 29.05.1900), \emph{Schriftsteller, Dramaturg}|pwu}s Ideen
               Geld verlieren möchte. Deshalb sein Plan mit Beer-Hofmann\pwindex{Beer-Hofmann, Richard 1866-07-11 – 1945-09-26@\textsc{Beer-Hofmann, Richard} (1866-07-11 – 1945-09-26), \emph{Schriftsteller}|pw}}{\lemma{\textnormal{\emph{Geiringer … Beer-Hofmann}}}\Cendnote{\textnormal{Eventuell der Schriftsteller und Dramaturg 
                  Leopold Geiringer\pwindex{Geiringer, Leopold 27.06.1851 – 29.05.1900@\textsc{Geiringer, Leopold} (27.06.1851 – 29.05.1900), \emph{Schriftsteller, Dramaturg}|pwk}? Womöglich sollte
                  mit 
                  Beer-Hofmann\pwindex{Beer-Hofmann, Richard 1866-07-11 – 1945-09-26@\textsc{Beer-Hofmann, Richard} (1866-07-11 – 1945-09-26), \emph{Schriftsteller}|pwk} ein Finanzier für neues Zeitschriftenprojekt
                  gewonnen werden.}}}\label{K_L03295-3h}! Von mir verlangt
               er, ich solle ihm einen Capitalisten schaffen. Dann will er mir eine Redactionsstelle gegen – Gewinnstantheil –
               verleihen!!\pend
           \pstart
           Ich arbeite wenig, denn die Zeitung\pwindex{Wiener Allgemeine Montags-Zeitung1899-07-03 – 1899-12-18@\emph{Wiener Allgemeine Montags-Zeitung} {[}1899-07-03 – 1899-12-18{]}|pwv}\pwindex{Wiener Allgemeine Zeitung1.3.1880 – 11.2.1934@\emph{Wiener Allgemeine Zeitung} {[}1.3.1880 – 11.2.1934{]}|pwv} macht mir viel Kopfzerbrechen und auch
               sonst kommt wieder einmal viel auf einmal zusammen. In ein paar Tagen fahre ich
               wieder nach Unterach\oindex{Unterach am Attersee@\textbf{Unterach am Attersee}|pw}. Schreiben Sie mir aber
                  imm\textcolor{gray}{er}hin nur hierher. Das \label{K_L03295-4v}\edtext{Feuilleton\pwindex{Salten, Felix 06.09.1869 – 08.10.1945@\textsc{Salten, Felix} (06.09.1869 – 08.10.1945), \emph{Schriftsteller, Journalist}!?? [Feuilleton ueber Paul Goldmann]Ende Juli/Anfang August 1899@\strich\emph{?? [Feuilleton über Paul Goldmann]} {[}Ende Juli/Anfang August 1899{]}|pwv} über Goldmann\pwindex{Goldmann, Paul 31.01.1865 – 25.09.1935@\textsc{Goldmann, Paul} (31.01.1865 – 25.09.1935), \emph{Schriftsteller, Journalist}|pw}}{\lemma{\textnormal{\emph{Feuilleton über Goldmann}}}\Cendnote{\textnormal{Ein Feuilleton über Goldmann\pwindex{Goldmann, Paul 31.01.1865 – 25.09.1935@\textsc{Goldmann, Paul} (31.01.1865 – 25.09.1935), \emph{Schriftsteller, Journalist}|pwk} in der \emph{Wiener
                     Allgemeinen Montags-Zeitung}\pwindex{Wiener Allgemeine Montags-Zeitung1899-07-03 – 1899-12-18@\emph{Wiener Allgemeine Montags-Zeitung} {[}1899-07-03 – 1899-12-18{]}|pwk} konnte nicht nachgewiesen werden. Im November und Dezember
                  erschienen zwei längere Auszüge aus Goldmann\pwindex{Goldmann, Paul 31.01.1865 – 25.09.1935@\textsc{Goldmann, Paul} (31.01.1865 – 25.09.1935), \emph{Schriftsteller, Journalist}|pwk}s Reisebericht \emph{Ein Sommer in
                     China}\pwindex{Goldmann, Paul 31.01.1865 – 25.09.1935@\textsc{Goldmann, Paul} (31.01.1865 – 25.09.1935), \emph{Schriftsteller, Journalist}!Sommer in China. Reisebilder1899-05-02@\strich\emph{Ein Sommer in China. Reisebilder} {[}1899-05-02{]}|pwk}, aber diese dürften hier nicht gemeint gewesen sein. Mutmaßlich
                  hatte Goldmann\pwindex{Goldmann, Paul 31.01.1865 – 25.09.1935@\textsc{Goldmann, Paul} (31.01.1865 – 25.09.1935), \emph{Schriftsteller, Journalist}|pwk} sich auf eine
                  Vermittlungsposition beschränkt und das »über« ist als ›ein über
                  Vermittlung von Goldmann\pwindex{Goldmann, Paul 31.01.1865 – 25.09.1935@\textsc{Goldmann, Paul} (31.01.1865 – 25.09.1935), \emph{Schriftsteller, Journalist}|pwk} erhaltenes
                  Feuilleton‹ zu lesen. Die Ausgabe vom 7. 8. 1899
                  behandelte etwa ausführlich den aktuellen Stand der Dreyfus\pwindex{Dreyfus, Alfred 1859-10-09 – 1935-07-12@\textsc{Dreyfus, Alfred} (1859-10-09 – 1935-07-12), \emph{Militär}|pwk}-Affäre, über die auch Goldmann\pwindex{Goldmann, Paul 31.01.1865 – 25.09.1935@\textsc{Goldmann, Paul} (31.01.1865 – 25.09.1935), \emph{Schriftsteller, Journalist}|pwk} berichtete. Auch sind in dem Blatt\pwindex{Wiener Allgemeine Montags-Zeitung1899-07-03 – 1899-12-18@\emph{Wiener Allgemeine Montags-Zeitung} {[}1899-07-03 – 1899-12-18{]}|pwkv} in der kurzen Zeit seines
                  Bestehens mehrere Texte von fran\oindex{Frankreich@\textbf{Frankreich}|pwkv}zösischen Autoren erschienen, mit denen Goldmann\pwindex{Goldmann, Paul 31.01.1865 – 25.09.1935@\textsc{Goldmann, Paul} (31.01.1865 – 25.09.1935), \emph{Schriftsteller, Journalist}|pwk} bereits 1893/1894 in der \emph{Frankfurter
                     Zeitung}\pwindex{?? Werk@Nicht ermittelte Verfasserinnen und Verfasser!Frankfurter Zeitung1856 – 1943@\emph{Frankfurter Zeitung} {[}1856 – 1943{]}|pwk} die Feuilletonreihe \emph{Neue
                     französische Humoristen}\pwindex{Goldmann, Paul 31.01.1865 – 25.09.1935@\textsc{Goldmann, Paul} (31.01.1865 – 25.09.1935), \emph{Schriftsteller, Journalist}!Neue franzoesische Humoristen1893-09-03@\strich\emph{Neue französische Humoristen} {[}1893-09-03{]}|pwk} bestritten hatte (siehe Paul Goldmann an Arthur Schnitzler, 7. 9. [1896]).}}}\label{K_L03295-4h} erscheint in den nächsten Tagen. Ich
               sende {\pb}es\pwindex{Salten, Felix 06.09.1869 – 08.10.1945@\textsc{Salten, Felix} (06.09.1869 – 08.10.1945), \emph{Schriftsteller, Journalist}!?? [Feuilleton ueber Paul Goldmann]Ende Juli/Anfang August 1899@\strich\emph{?? [Feuilleton über Paul Goldmann]} {[}Ende Juli/Anfang August 1899{]}|pwv} Ihnen gleich.\pend
           \pstart
           Auf Wiedersehen: hoffentlich bald. \label{K_L03295-5v}\edtext{Grüßen Sie Wassermann\pwindex{Wassermann, Jakob 10.03.1873 – 01.01.1934@\textsc{Wassermann, Jakob} (10.03.1873 – 01.01.1934), \emph{Schriftsteller}|pw} und den emsigen Richard\pwindex{Beer-Hofmann, Richard 1866-07-11 – 1945-09-26@\textsc{Beer-Hofmann, Richard} (1866-07-11 – 1945-09-26), \emph{Schriftsteller}|pw}}{\lemma{\textnormal{\emph{Grüßen … Richard}}}\Cendnote{\textnormal{Jakob Wassermann\pwindex{Wassermann, Jakob 10.03.1873 – 01.01.1934@\textsc{Wassermann, Jakob} (10.03.1873 – 01.01.1934), \emph{Schriftsteller}|pwk} hielt sich gemeinsam mit
                     Schnitzler\pwindex{Schnitzler, Arthur 15.05.1862 – 21.10.1931@\textsc{Schnitzler, Arthur} (15.05.1862 – 21.10.1931), \emph{Schriftsteller, Mediziner}|pwk} in Velden am Wörthersee\oindex{Velden am Woerthersee@\textbf{Velden am Wörthersee}|pwk} auf. Am 28. 7. 1899 reisten sie weiter nach Villach\oindex{Villach@\textbf{Villach}|pwk}. Richard Beer-Hofmann\pwindex{Beer-Hofmann, Richard 1866-07-11 – 1945-09-26@\textsc{Beer-Hofmann, Richard} (1866-07-11 – 1945-09-26), \emph{Schriftsteller}|pwk} hielt sich im nahegelegenen Seeboden\oindex{Seeboden@\textbf{Seeboden}|pwk} auf und traf Schnitzler\pwindex{Schnitzler, Arthur 15.05.1862 – 21.10.1931@\textsc{Schnitzler, Arthur} (15.05.1862 – 21.10.1931), \emph{Schriftsteller, Mediziner}|pwk} in dieser Zeit ebenso. Am 5. 8. 1899 starteten Schnitzler\pwindex{Schnitzler, Arthur 15.05.1862 – 21.10.1931@\textsc{Schnitzler, Arthur} (15.05.1862 – 21.10.1931), \emph{Schriftsteller, Mediziner}|pwk}, Wassermann\pwindex{Wassermann, Jakob 10.03.1873 – 01.01.1934@\textsc{Wassermann, Jakob} (10.03.1873 – 01.01.1934), \emph{Schriftsteller}|pwk} und Beer-Hofmann\pwindex{Beer-Hofmann, Richard 1866-07-11 – 1945-09-26@\textsc{Beer-Hofmann, Richard} (1866-07-11 – 1945-09-26), \emph{Schriftsteller}|pwk} in Niederdorf\oindex{Niederdorf@\textbf{Niederdorf}|pwk} eine mehrtätige gemeinsame
                  Wanderung.}}}\label{K_L03295-5h}. Frl. Metzl\pwindex{Salten, Ottilie 07.03.1868 – 22.06.1942@\textsc{Salten, Ottilie} (07.03.1868 – 22.06.1942), \emph{Schauspielerin}|pw} grüßt
               Sie.\pend
           \pstart
           Herzlichst {\\[\baselineskip]}Ihr {\\[\baselineskip]}\spacefill\mbox{Salten}\pend
           \leftskip=0em{}
         
         \endnumbering\mylabel{h}\end{ledgroupsized}  \newcommand{\dateiname}{L03295}\newcommand{\titel}{Felix Salten an Arthur Schnitzler, 27. 7. 1899}\newcommand{\editorInnen}{Martin Anton Müller und Laura Untner}%% latex-leseansicht-abspann.tex
%% Abspann für die Leseansicht.
%% Der Schalter \ifkorrekturansicht ist bereits durch den Vorspann gesetzt.

%% latex-abspann.tex
%% Gemeinsamer Abspann für Korrekturansicht und Leseansicht.
%% Setzt den Schalter \ifkorrekturansicht voraus (gesetzt in den
%% einbindenden Dateien latex-korrekturansicht-abspann.tex bzw.
%% latex-leseansicht-abspann.tex).
%% ---------------------------------------------------------------

\normalsize

% Das esempio-Environment wird nur in der Leseansicht benötigt
\ifkorrekturansicht\else
\newenvironment{esempio}[3]%
{
    \vspace{1.5ex}
    \rlap{\underline{#1}}
    \par
    \setlength{\parindent}{0cm}
    \nopagebreak
    \leftskip=#2cm
    \rightskip=#3cm
}
{
    \par
}
\fi

\doendnotes{C}
\bigskip
\vfill

\clearpage

\footnotesize

\ifkorrekturansicht
  \lohead{\textsc{register}}
\fi

% theindex-Environment neu definieren ohne reledmac
\makeatletter
\renewenvironment{theindex}{%
  \ifkorrekturansicht
    \section*{\indexname}%
  \else
    \subsubsection*{Index der erwähnten Entitäten}%
  \fi
  \setlength{\parindent}{0pt}%
  \setlength{\parskip}{0pt plus 0.3pt}%
  \let\item\@idxitem
}{%
  \ifkorrekturansicht\clearpage\fi
}
\makeatother

\IfFileExists{\jobname-pw.ind}{\input{\jobname-pw.ind}}{}

% Quellenangabe nur in der Leseansicht
\ifkorrekturansicht\else
% Fallback-Definitionen, falls die .tex-Datei \titel etc. nicht gesetzt hat
\providecommand{\titel}{}
\providecommand{\editorInnen}{}
\providecommand{\dateiname}{\jobname}

\vspace{3cm}

\vfill

\footnotesize
\textsc{Quelle}: \titel. Herausgegeben von {\editorInnen}. In: \emph{Arthur Schnitzler: Briefwechsel mit Autorinnen und Autoren}.
 Digitale Edition, https://schnitzler-briefe.acdh.oeaw.ac.at/{\dateiname}.html (Stand \today)
\fi

\end{document}


      