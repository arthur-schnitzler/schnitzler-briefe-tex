%% latex-leseansicht-vorspann.tex
%% Vorspann für die Leseansicht.
%% Lädt die gemeinsame Datei latex-vorspann.tex mit nicht gesetztem Schalter.

\newif\ifkorrekturansicht
\korrekturansichtfalse

\input{../tex-inputs/latex-vorspann}


\section[Richard Beer-Hofmann an Arthur Schnitzler, 5. 10. 1894]{L00377 Richard Beer-Hofmann an Arthur Schnitzler, 5. 10. 1894}
\nopagebreak\mylabel{L00377v}
\rehead{ }\normalsize\beginnumbering\briefempfaengerindex{Schnitzler, Arthur@\textsc{Schnitzler, Arthur}!zzzBeer-Hofmann, Richard@\emph{von Richard Beer-Hofmann}!1894-10-052@{5. 10. 1894}|(be}
\toendnotes[C]{\smallbreak\pagebreak[2]}
\correspDesc{Versand  durch Richard Beer-Hofmann am 5. 10. 1894 in Rom
\newline{}Erhalt  durch Arthur Schnitzler am 8. 10. 1894 in Wien}\toendnotes[C]{\smallbreak}
\Standort{CUL, Schnitzler, B 8.}
\physDesc{Bildpostkarte, 174 Zeichen
\newline{}Handschrift: Bleistift, lateinische Kurrent
\newline{}Versand: 1) Stempel: »\nobreak{}\oindex{Rom@\textbf{Rom}, \emph{Hauptstadt}|pwk}Roma, 5 10.–94\nobreak{}«.   2) Stempel: »\nobreak{}\oindex{IX., Alsergrund@\textbf{IX., Alsergrund}, \emph{Verwaltungsgebiet}|pwk}Wien 9/3, 8. 10. 94, 8.V, Bestellt\nobreak{}«. 
\newline{}Ordnung: von Schnitzler mit Bleistift nummeriert: »44«,
                                 von zweiter, unbekannter Hand die Nummerierung wiederholt }\pstart{}{\pb}Herrn\pend{}\pstart{}Arthur Schnitzler\pend{}\pstart{}Austria\oindex{Österreich@\textbf{Österreich}|pw}\pend{}\pstart{}Wien\oindex{Wien@\textbf{Wien}, \emph{Verwaltungsgebiet}|pw}\pend{}\pstart{}Frankgaſſe 1\oindex{Wien@\textbf{Wien}!IX., Alsergrund@\textbf{IX., Alsergrund}!Frankgasse 1@\textbf{Frankgasse 1}, \emph{Wohngebäude}|pw}\pend{}{\bigskip}
\pstart
           \noindent{}\centering{}{\pb}\textcolor{gray}{\textbf{Ricordo di Firenze\oindex{Florenz@\textbf{Florenz}|pw} – Chiesa Santa
                     Croce\oindex{Santa Croce [Florenz]@\textbf{Santa Croce [Florenz]}, \emph{Kirche}|pw}}}\pend
           \vspace{1em}
\pstart
           \noindent{}{\pb}Lieber! Gemäldegallerie 3 ist noch \uline{Florenz\oindex{Florenz@\textbf{Florenz}|pw}}, ich aber bin in Rom\oindex{Rom@\textbf{Rom}, \emph{Hauptstadt}|pw}. Bitte schreiben Sie.
               Herzlichst Ihr\pend
           \pstart \spacefill\mbox{Richard}\pend{}
\pstart
           Rom\oindex{Rom@\textbf{Rom}, \emph{Hauptstadt}|pw}{ }Freitag 5/X 94{ }abends\pend
           \selectlanguage{ngerman}\endnumbering\briefempfaengerindex{Schnitzler, Arthur@\textsc{Schnitzler, Arthur}!zzzBeer-Hofmann, Richard@\emph{von Richard Beer-Hofmann}!1894-10-052@{5. 10. 1894}|)be}\mylabel{L00377h}  \newcommand{\dateiname}{L00377}\newcommand{\titel}{Richard Beer-Hofmann an Arthur Schnitzler, 5. 10. 1894}\newcommand{\editorInnen}{Martin Anton Müller und Gerd-Hermann Susen}%% latex-leseansicht-abspann.tex
%% Abspann für die Leseansicht.
%% Der Schalter \ifkorrekturansicht ist bereits durch den Vorspann gesetzt.

%% latex-abspann.tex
%% Gemeinsamer Abspann für Korrekturansicht und Leseansicht.
%% Setzt den Schalter \ifkorrekturansicht voraus (gesetzt in den
%% einbindenden Dateien latex-korrekturansicht-abspann.tex bzw.
%% latex-leseansicht-abspann.tex).
%% ---------------------------------------------------------------

\normalsize

% Das esempio-Environment wird nur in der Leseansicht benötigt
\ifkorrekturansicht\else
\newenvironment{esempio}[3]%
{
    \vspace{1.5ex}
    \rlap{\underline{#1}}
    \par
    \setlength{\parindent}{0cm}
    \nopagebreak
    \leftskip=#2cm
    \rightskip=#3cm
}
{
    \par
}
\fi

\doendnotes{C}
\bigskip
\vfill

\clearpage

\footnotesize

\ifkorrekturansicht
  \lohead{\textsc{register}}
\fi

% theindex-Environment neu definieren ohne reledmac
\makeatletter
\renewenvironment{theindex}{%
  \ifkorrekturansicht
    \section*{\indexname}%
  \else
    \subsubsection*{Index der erwähnten Entitäten}%
  \fi
  \setlength{\parindent}{0pt}%
  \setlength{\parskip}{0pt plus 0.3pt}%
  \let\item\@idxitem
}{%
  \ifkorrekturansicht\clearpage\fi
}
\makeatother

\IfFileExists{\jobname-pw.ind}{\input{\jobname-pw.ind}}{}

% Quellenangabe nur in der Leseansicht
\ifkorrekturansicht\else
% Fallback-Definitionen, falls die .tex-Datei \titel etc. nicht gesetzt hat
\providecommand{\titel}{}
\providecommand{\editorInnen}{}
\providecommand{\dateiname}{\jobname}

\vspace{3cm}

\vfill

\footnotesize
\textsc{Quelle}: \titel. Herausgegeben von {\editorInnen}. In: \emph{Arthur Schnitzler: Briefwechsel mit Autorinnen und Autoren}.
 Digitale Edition, https://schnitzler-briefe.acdh.oeaw.ac.at/{\dateiname}.html (Stand \today)
\fi

\end{document}


