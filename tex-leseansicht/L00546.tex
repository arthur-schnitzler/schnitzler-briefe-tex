%% latex-leseansicht-vorspann.tex
%% Vorspann für die Leseansicht.
%% Lädt die gemeinsame Datei latex-vorspann.tex mit nicht gesetztem Schalter.

\newif\ifkorrekturansicht
\korrekturansichtfalse

\input{../tex-inputs/latex-vorspann}


         
         \renewcommand{\erwaehntePersonen}{Personen: Richard Beer-Hofmann, Otto Brahm, Johann Wolfgang von Goethe, Friedrich Hebbel, Hugo von Hofmannsthal, Gotthold Ephraim Lessing, Felix Salten, Gustav Schwarzkopf, Hippolyte Taine}
         \renewcommand{\erwaehnteOrte}{Orte: Dänemark, Norwegen, Schweden, Tłumacz, Wien}
         \renewcommand{\erwaehnteWerke}{Werke: ?? [Englischer Kriminalroman], Der Zerrissene, Die Frau des Weisen. Erzählung, Die Revolution, Freiwild. Schauspiel in 3 Akten, Ollapatrida des durchgetriebenen Fuchsmundi, Poesie und Leben. Aus einem Vortrage, West-östlicher Divan}
               \section[Arthur Schnitzler an Hugo von Hofmannsthal, 23. 5. 1896]{ Arthur Schnitzler an Hugo von Hofmannsthal, 23. 5. 1896}\nopagebreak\mylabel{v}\rehead{ }\begin{ledgroupsized}[t]{13cm}\normalsize\beginnumbering\briefempfaengerindex{Hofmannsthal, Hugo von@\textsc{Hofmannsthal, Hugo von}!zzzSchnitzler, Arthur@\emph{von Arthur Schnitzler}!1896-05-231@{23. 5. 1896}|(be} \toendnotes[C]{\smallbreak\pagebreak[2]} \Standort{FDH, Hs-30885,49.}
\physDesc{Brief, 1 Blatt, 4 Seiten, 1683 Zeichen
\newline{}Handschrift: schwarze Tinte, deutsche Kurrent}\buchAbdrucke{\weitereDrucke{Hugo von Hofmannsthal, Arthur Schnitzler: \emph{Briefwechsel}. Hg. Therese Nickl und Heinrich Schnitzler. Frankfurt am Main: \emph{S. Fischer} 1964, S. 66–67.} }\toendnotes[C]{\smallbreak}\pstart
           \raggedleft{}{\pb}Wien\oindex{Wien@\textbf{Wien}|pw}, 23. 5. 96.\pend
           \pstart
           Mein lieber Hugo, ich freue mich ſehr daſs Sie ſich meiner erinnert
               haben u noch mehr, daſs Sie bald zurückko{\geminationm}en. Im
                  Juni wollen wir dann doch noch ein paar Mal zuſa{\geminationm}en ſein. Und das eine Mal von den paar werde ich wohl
               das Stück\pwindex{Schnitzler, Arthur 15.05.1862 – 21.10.1931@\textsc{Schnitzler, Arthur} (15.05.1862 – 21.10.1931), \emph{Schriftsteller, Mediziner}!Freiwild. Schauspiel in 3 Akten1896@\strich\emph{Freiwild. Schauspiel in 3 Akten} {[}1896{]}|pwv} vorleſen können. Ich
               habe jetzt mehr Zuverſicht. Aber mit meinem ganzen Herzen bin ich doch nicht dabei.
               Vielleicht iſt das ſogar gut: vielleicht {\pb}iſt es ein
               Fehler von vielen meiner Sachen, daſs ich mit ihnen im Schreiben zu zärtlich geworden
               bin.\pend
           \pstart
           Ihren Artikel über Poeſie und
                  Leben\pwindex{Hofmannsthal, Hugo von 1874-02-01 – 1929-07-15@\textsc{Hofmannsthal, Hugo von} (1874-02-01 – 1929-07-15), \emph{Schriftsteller}!Poesie und Leben. Aus einem Vortrage1896-05-16@\strich\emph{Poesie und Leben. Aus einem Vortrage} {[}1896-05-16{]}|pwv} hab\textcolor{gray}{e} ich als ein ſchönes Gedicht empfunden; aber es
               kam mir vor, als we{\geminationn} Sie die Grenzen der Poeſie zu eng
               gezogen hätten, während es doch Ihre Abſicht war, ſie zu erweitern. Woher eigentlich
               dieſes ſonderbare Bedürfnis kommt, über Kunſt zu reden. Ich ſelbſt fühl es manchmal,
               und {\pb}habe nachher i{\geminationm}er oder
               oft das Gefühl etwas überflüſſiges oder gar unrechtes gethan \introOben{}zu\introOben{} haben. Es ko{\geminationm}t besti{\geminationm}t \uline{nicht allein} daher,
               daſs das Theoretiſiren einfach meinem Weſen nicht entſpricht. Und meine Sehnſucht,
               ins Klare zu kommen, iſt gewiſs auch nicht gering. Und was Goethe\pwindex{Goethe, Johann Wolfgang von 1749-08-28 – 1832-03-22@\textsc{Goethe, Johann Wolfgang von} (1749-08-28 – 1832-03-22), \emph{Schriftsteller}|pw}, Leſſing\pwindex{Lessing, Gotthold Ephraim 22.01.1729 – 15.02.1781@\textsc{Lessing, Gotthold Ephraim} (22.01.1729 – 15.02.1781), \emph{Schriftsteller, Bibliothekar}|pw}, Hebbel\pwindex{Hebbel, Friedrich 18.03.1813 – 13.12.1863@\textsc{Hebbel, Friedrich} (18.03.1813 – 13.12.1863), \emph{Schriftsteller}|pw}, was Sie und andre über Kunſt ſagen, leſe
               ich gern; manches beruhigt mich, indem es abſchließt, andres bewegt {\pb}mich, indem es Thore aufſchließt. Wir ſprechen einmal
               darüber.\pend
           \pstart
           \textsc{Brahm}\pwindex{Brahm, Otto 05.02.1856 – 28.11.1912@\textsc{Brahm, Otto} (05.02.1856 – 28.11.1912), \emph{Theaterleiter, Regisseur}|pw} iſt jetzt da, den ich perſönlich gern habe. Geſtern Abend waren er, Richard\pwindex{Beer-Hofmann, Richard 1866-07-11 – 1945-09-26@\textsc{Beer-Hofmann, Richard} (1866-07-11 – 1945-09-26), \emph{Schriftsteller}|pw}, Salten\pwindex{Salten, Felix 06.09.1869 – 08.10.1945@\textsc{Salten, Felix} (06.09.1869 – 08.10.1945), \emph{Schriftsteller, Journalist, Chefredakteur}|pw} u. Schwarzkopf\pwindex{Schwarzkopf, Gustav 07.11.1853 – 13.11.1939@\textsc{Schwarzkopf, Gustav} (07.11.1853 – 13.11.1939), \emph{Schriftsteller}|pw} bei mir. –
               Geleſen hab ich die Frzſ. Revol.\pwindex{Taine, Hippolyte 21.04.1828 – 05.03.1893@\textsc{Taine, Hippolyte} (21.04.1828 – 05.03.1893), \emph{Philosoph, Geschichtsschreiber}!Revolution1878 – 1883@\strich\emph{Die Revolution} {[}1878 – 1883{]}|pw} von \textsc{Taine}\pwindex{Taine, Hippolyte 21.04.1828 – 05.03.1893@\textsc{Taine, Hippolyte} (21.04.1828 – 05.03.1893), \emph{Philosoph, Geschichtsschreiber}|pw}, die Olla potrida des durchtriebenen
                  Fuchsmundi\pwindex{\textcolor{red}{\textsuperscript{XXXX1 indx}}!Ollapatrida des durchgetriebenen Fuchsmundi1711@\strich\emph{Ollapatrida des durchgetriebenen Fuchsmundi} {[}1711{]}|pw}, die Noten zum Divan\pwindex{Goethe, Johann Wolfgang von 1749-08-28 – 1832-03-22@\textsc{Goethe, Johann Wolfgang von} (1749-08-28 – 1832-03-22), \emph{Schriftsteller}!West-oestlicher Divan1819@\strich\emph{West-östlicher Divan} {[}1819{]}|pw} und
               einen engliſchen \label{K_L00546-1v}\edtext{Kriminalroman\pwindex{?? Werk@Nicht ermittelte Verfasserinnen und Verfasser!?? [Englischer Kriminalroman]1896@\emph{?? [Englischer Kriminalroman]} {[}1896{]}|pwv}}{\lemma{\textnormal{\emph{Kriminalroman}}}\Cendnote{\textnormal{nicht identifiziert}}}\label{K_L00546-1h}. – Mein So{\geminationm}erplan iſt jetzt Norwegen\oindex{Norwegen@\textbf{Norwegen}|pw}, Schweden\oindex{Schweden@\textbf{Schweden}|pw}, Dänemark\oindex{Daenemark@\textbf{Dänemark}|pw}; und eine Novelle\pwindex{Schnitzler, Arthur 15.05.1862 – 21.10.1931@\textsc{Schnitzler, Arthur} (15.05.1862 – 21.10.1931), \emph{Schriftsteller, Mediziner}!Frau des Weisen. Erzaehlung1897-01-02 – 1897-01-16@\strich\emph{Die Frau des Weisen. Erzählung} {[}1897-01-02 – 1897-01-16{]}|pwv}. – Jetzt iſt ein Gewitter mit Blitz und
               Donner und Abend geh ich zum »Zerriſſenen\pwindex{\textcolor{red}{\textsuperscript{XXXX1 indx}}!Zerrissene9. 4. 1844@\strich\emph{Der Zerrissene} {[}9. 4. 1844{]}|pw}«.\pend
           \pstart Herzlich der Ihre, \spacefill\mbox{AS.}\pend{}
         
         \endnumbering\mylabel{h}\end{ledgroupsized}  \newcommand{\dateiname}{L00546}\newcommand{\titel}{Arthur Schnitzler an Hugo von Hofmannsthal, 23. 5. 1896}\newcommand{\editorInnen}{Martin Anton Müller und Gerd-Hermann Susen}%% latex-leseansicht-abspann.tex
%% Abspann für die Leseansicht.
%% Der Schalter \ifkorrekturansicht ist bereits durch den Vorspann gesetzt.

%% latex-abspann.tex
%% Gemeinsamer Abspann für Korrekturansicht und Leseansicht.
%% Setzt den Schalter \ifkorrekturansicht voraus (gesetzt in den
%% einbindenden Dateien latex-korrekturansicht-abspann.tex bzw.
%% latex-leseansicht-abspann.tex).
%% ---------------------------------------------------------------

\normalsize

% Das esempio-Environment wird nur in der Leseansicht benötigt
\ifkorrekturansicht\else
\newenvironment{esempio}[3]%
{
    \vspace{1.5ex}
    \rlap{\underline{#1}}
    \par
    \setlength{\parindent}{0cm}
    \nopagebreak
    \leftskip=#2cm
    \rightskip=#3cm
}
{
    \par
}
\fi

\doendnotes{C}
\bigskip
\vfill

\clearpage

\footnotesize

\ifkorrekturansicht
  \lohead{\textsc{register}}
\fi

% theindex-Environment neu definieren ohne reledmac
\makeatletter
\renewenvironment{theindex}{%
  \ifkorrekturansicht
    \section*{\indexname}%
  \else
    \subsubsection*{Index der erwähnten Entitäten}%
  \fi
  \setlength{\parindent}{0pt}%
  \setlength{\parskip}{0pt plus 0.3pt}%
  \let\item\@idxitem
}{%
  \ifkorrekturansicht\clearpage\fi
}
\makeatother

\IfFileExists{\jobname-pw.ind}{\input{\jobname-pw.ind}}{}

% Quellenangabe nur in der Leseansicht
\ifkorrekturansicht\else
% Fallback-Definitionen, falls die .tex-Datei \titel etc. nicht gesetzt hat
\providecommand{\titel}{}
\providecommand{\editorInnen}{}
\providecommand{\dateiname}{\jobname}

\vspace{3cm}

\vfill

\footnotesize
\textsc{Quelle}: \titel. Herausgegeben von {\editorInnen}. In: \emph{Arthur Schnitzler: Briefwechsel mit Autorinnen und Autoren}.
 Digitale Edition, https://schnitzler-briefe.acdh.oeaw.ac.at/{\dateiname}.html (Stand \today)
\fi

\end{document}


      