%% latex-korrekturansicht-vorspann.tex
%% Vorspann für die Korrekturansicht.
%% Lädt die gemeinsame Datei latex-vorspann.tex mit gesetztem Schalter.

\newif\ifkorrekturansicht
\korrekturansichttrue

\input{../tex-inputs/latex-vorspann}


\section[Arthur Schnitzler an Hugo von Hofmannsthal, 23. 5. 1896]{L00546 Arthur Schnitzler an Hugo von Hofmannsthal, 23. 5. 1896}
\nopagebreak\mylabel{L00546v}
\rehead{ }\normalsize\beginnumbering\briefempfaengerindex{Hofmannsthal, Hugo von@\textsc{Hofmannsthal, Hugo von}!zzzSchnitzler, Arthur@\emph{von Arthur Schnitzler}!1896-05-231@{23. 5. 1896}|(be}
\toendnotes[C]{\smallbreak\pagebreak[2]}\Standort{FDH, Hs-30885,49.}
\physDesc{Brief, 1 Blatt, 4 Seiten, 1683 Zeichen
\newline{}Handschrift: schwarze Tinte, deutsche Kurrent}
\buchAbdrucke{\weitereDrucke{Hugo von Hofmannsthal, Arthur Schnitzler: \emph{Briefwechsel}. Frankfurt am Main: \emph{S. Fischer} 1964, S. 66–67.} }\toendnotes[C]{\smallbreak}
\pstart
           \raggedleft{}{\pb}Wien\oindex{Wien@\textbf{Wien}, \emph{A.ADM2}|pw}, 23. 5. 96.\pend
           \vspace{0.5em}
\pstart
           Mein lieber Hugo, ich freue mich ſehr daſs Sie ſich meiner erinnert
               haben u noch mehr, daſs Sie bald zurückko{\geminationm}en. Im
                  Juni wollen wir dann doch noch ein paar Mal zuſa{\geminationm}en ſein. Und das eine Mal von den paar werde ich wohl
               das Stück\pwindex{Freiwild. Schauspiel in 3 Akten@\emph{Freiwild. Schauspiel in 3 Akten}|pwv} vorleſen können. Ich
               habe jetzt mehr Zuverſicht. Aber mit meinem ganzen Herzen bin ich doch nicht dabei.
               Vielleicht iſt das ſogar gut: vielleicht {\pb}iſt es ein
               Fehler von vielen meiner Sachen, daſs ich mit ihnen im Schreiben zu zärtlich geworden
               bin.\pend
           
\pstart
           Ihren Artikel über Poeſie und
                  Leben\pwindex{Poesie und Leben. Aus einem Vortrage@\emph{Poesie und Leben. Aus einem Vortrage}|pwv} hab\textcolor{gray}{e} ich als ein ſchönes Gedicht empfunden; aber es
               kam mir vor, als we{\geminationn} Sie die Grenzen der Poeſie zu eng
               gezogen hätten, während es doch Ihre Abſicht war, ſie zu erweitern. Woher eigentlich
               dieſes ſonderbare Bedürfnis kommt, über Kunſt zu reden. Ich ſelbſt fühl es manchmal,
               und {\pb}habe nachher i{\geminationm}er oder
               oft das Gefühl etwas überflüſſiges oder gar unrechtes gethan \introOben{}zu\introOben{} haben. Es ko{\geminationm}t besti{\geminationm}t \uline{nicht allein} daher,
               daſs das Theoretiſiren einfach meinem Weſen nicht entſpricht. Und meine Sehnſucht,
               ins Klare zu kommen, iſt gewiſs auch nicht gering. Und was Goethe\pwindex{Goethe, Johann Wolfgang von 1749-08-28 – 1832-03-22@\textsc{Goethe, Johann Wolfgang von} (1749-08-28 – 1832-03-22), \emph{Schriftsteller/Schriftstellerin}|pw}, Leſſing\pwindex{Lessing, Gotthold Ephraim 22.01.1729 – 15.02.1781@\textsc{Lessing, Gotthold Ephraim} (22.01.1729 – 15.02.1781), \emph{Schriftsteller/Schriftstellerin, Kritiker/Kritikerin, Philosoph/Philosophin}|pw}, Hebbel\pwindex{Hebbel, Friedrich 18.03.1813 – 13.12.1863@\textsc{Hebbel, Friedrich} (18.03.1813 – 13.12.1863), \emph{Schriftsteller/Schriftstellerin}|pw}, was Sie und andre über Kunſt ſagen, leſe
               ich gern; manches beruhigt mich, indem es abſchließt, andres bewegt {\pb}mich, indem es Thore aufſchließt. Wir ſprechen einmal
               darüber.\pend
           
\pstart
           \textsc{Brahm}\pwindex{Brahm, Otto 05.02.1856 – 28.11.1912@\textsc{Brahm, Otto} (05.02.1856 – 28.11.1912), \emph{Theaterleiter/Theaterleiterin, Regisseur/Regisseurin}|pw} iſt jetzt da, den ich perſönlich gern habe. Geſtern Abend waren er, Richard\pwindex{Beer-Hofmann, Richard 1866-07-11 – 1945-09-26@\textsc{Beer-Hofmann, Richard} (1866-07-11 – 1945-09-26), \emph{Schriftsteller/Schriftstellerin}|pw}, Salten\pwindex{Salten, Felix 06.09.1869 – 08.10.1945@\textsc{Salten, Felix} (06.09.1869 – 08.10.1945), \emph{Schriftsteller/Schriftstellerin, Journalist/Journalistin, Chefredakteur/Chefredakteurin}|pw} u. Schwarzkopf\pwindex{Schwarzkopf, Gustav 07.11.1853 – 13.11.1939@\textsc{Schwarzkopf, Gustav} (07.11.1853 – 13.11.1939), \emph{Schriftsteller/Schriftstellerin}|pw} bei mir. –
               Geleſen hab ich die Frzſ. Revol.\pwindex{Revolution@\emph{Die Revolution}|pw} von \textsc{Taine}\pwindex{Taine, Hippolyte 21.04.1828 – 05.03.1893@\textsc{Taine, Hippolyte} (21.04.1828 – 05.03.1893), \emph{Philosoph/Philosophin, Geschichtsschreiber/Geschichtsschreiberin}|pw}, die Olla potrida des durchtriebenen
                  Fuchsmundi\pwindex{Ollapatrida des durchgetriebenen Fuchsmundi@\emph{Ollapatrida des durchgetriebenen Fuchsmundi}|pw}, die Noten zum Divan\pwindex{West-oestlicher Divan@\emph{West-östlicher Divan}|pw} und
               einen engliſchen \label{K_L00546-1v}\edtext{Kriminalroman\pwindex{?? [Englischer Kriminalroman]@\emph{?? [Englischer Kriminalroman]}|pwv}}{\lemma{\textnormal{\emph{Kriminalroman}}}\Cendnote{\textnormal{nicht identifiziert}}}\label{K_L00546-1}. – Mein So{\geminationm}erplan iſt jetzt Norwegen\oindex{Norwegen@\textbf{Norwegen}, \emph{A.PCLI}|pw}, Schweden\oindex{Schweden@\textbf{Schweden}, \emph{A.PCLI}|pw}, Dänemark\oindex{Daenemark@\textbf{Dänemark}, \emph{A.PCLI}|pw}; und eine Novelle\pwindex{Frau des Weisen. Erzaehlung@\emph{Die Frau des Weisen. Erzählung}|pwv}. – Jetzt iſt ein Gewitter mit Blitz und
               Donner und Abend geh ich zum »Zerriſſenen\pwindex{Zerrissene. Posse mit Gesang in drei Akten@\emph{Der Zerrissene. Posse mit Gesang in drei Akten}|pw}«.\pend
           \pstart Herzlich der Ihre, \spacefill\mbox{AS.}\pend{}\selectlanguage{ngerman}\endnumbering\briefempfaengerindex{Hofmannsthal, Hugo von@\textsc{Hofmannsthal, Hugo von}!zzzSchnitzler, Arthur@\emph{von Arthur Schnitzler}!1896-05-231@{23. 5. 1896}|)be}\mylabel{L00546h}  \normalsize

\doendnotes{C}
\bigskip
\vfill

\clearpage

\footnotesize

\lohead{\textsc{register}}

% Definiere theindex-Environment komplett neu ohne reledmac
\makeatletter
\renewenvironment{theindex}{%
  \section*{\indexname}%
  \setlength{\parindent}{0pt}%
  \setlength{\parskip}{0pt plus 0.3pt}%
  \let\item\@idxitem
}{%
  \clearpage
}
\makeatother

\IfFileExists{\jobname-pw.ind}{\input{\jobname-pw.ind}}{}

\end{document}

      