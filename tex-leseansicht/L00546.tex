%% latex-leseansicht-vorspann.tex
%% Vorspann für die Leseansicht.
%% Lädt die gemeinsame Datei latex-vorspann.tex mit nicht gesetztem Schalter.

\newif\ifkorrekturansicht
\korrekturansichtfalse

\input{../tex-inputs/latex-vorspann}


\section[Arthur Schnitzler an Hugo von Hofmannsthal, 23. 5. 1896]{L00546 Arthur Schnitzler an Hugo von Hofmannsthal, 23. 5. 1896}
\nopagebreak\mylabel{L00546v}
\rehead{ }\normalsize\beginnumbering\briefempfaengerindex{Hofmannsthal, Hugo von@\textsc{Hofmannsthal, Hugo von}!zzzSchnitzler, Arthur@\emph{von Arthur Schnitzler}!1896-05-231@{23. 5. 1896}|(be}
\toendnotes[C]{\smallbreak\pagebreak[2]}
\correspDesc{Versand  durch Arthur Schnitzler am 23. 5. 1896 in Wien
\newline{}Erhalt  durch Hugo von Hofmannsthal im Zeitraum [24. 5. 1896
                  – 28. 5. 1896?] in Tlumatsch}\toendnotes[C]{\smallbreak}
\Standort{FDH, Hs-30885,49.}
\physDesc{Brief, 1 Blatt, 4 Seiten, 1683 Zeichen
\newline{}Handschrift: schwarze Tinte, deutsche Kurrent}
\buchAbdrucke{\weitereDrucke{Hugo von Hofmannsthal, Arthur Schnitzler: \emph{Briefwechsel}. Herausgegeben von Therese Nickl und Heinrich Schnitzler. Frankfurt am Main: \emph{S. Fischer} 1964, S. 66–67.} }\toendnotes[C]{\smallbreak}
\pstart
           \raggedleft{}{\pb}Wien\oindex{Wien@\textbf{Wien}, \emph{Verwaltungsgebiet}|pw}, 23. 5. 96.\pend
           \vspace{0.5em}
\pstart
           Mein lieber Hugo, ich freue mich{ }ſehr daſs Sie{ }ſich meiner erinnert
               haben u noch mehr, daſs Sie bald zurückko{\geminationm}en. Im
                  Juni wollen wir dann doch noch ein paar Mal zuſa{\geminationm}en{ }ſein. Und das eine Mal von den paar werde ich wohl
               das Stück\pwindex{Schnitzler, Arthur 15.\,5.\,1862 Wien – 21.\,10.\,1931 ebd.@\textsc{Schnitzler, Arthur} (15.\,5.\,1862 Wien – 21.\,10.\,1931 ebd.), \emph{Schriftsteller, Mediziner}!Freiwild. Schauspiel in 3 Akten@\strich\emph{Freiwild. Schauspiel in 3 Akten}|pwv} vorleſen können. Ich
               habe jetzt mehr Zuverſicht. Aber mit meinem ganzen Herzen bin ich doch nicht dabei.
               Vielleicht iſt das{ }ſogar gut: vielleicht {\pb}iſt es ein
               Fehler von vielen meiner Sachen, daſs ich mit ihnen im Schreiben zu zärtlich geworden
               bin.\pend
           
\pstart
           Ihren Artikel über Poeſie und
                  Leben\pwindex{Hofmannsthal, Hugo von 1.\,2.\,1874 Wien – 15.\,7.\,1929 Rodaun@\textsc{Hofmannsthal, Hugo von} (1.\,2.\,1874 Wien – 15.\,7.\,1929 Rodaun), \emph{Schriftsteller}!Poesie und Leben. Aus einem Vortrage@\strich\emph{Poesie und Leben. Aus einem Vortrage}|pwv} hab\textcolor{gray}{e} ich als ein{ }ſchönes Gedicht empfunden; aber es
               kam mir vor, als we{\geminationn} Sie die Grenzen der Poeſie zu eng
               gezogen hätten, während es doch Ihre Abſicht war,{ }ſie zu erweitern. Woher eigentlich
               dieſes{ }ſonderbare Bedürfnis kommt, über Kunſt zu reden. Ich{ }ſelbſt fühl es manchmal,
               und {\pb}habe nachher i{\geminationm}er oder
               oft das Gefühl etwas überflüſſiges oder gar unrechtes gethan \introOben{}zu\introOben{} haben. Es ko{\geminationm}t besti{\geminationm}t \uline{nicht allein} daher,
               daſs das Theoretiſiren einfach meinem Weſen nicht entſpricht. Und meine Sehnſucht,
               ins Klare zu kommen, iſt gewiſs auch nicht gering. Und was Goethe\pwindex{Goethe, Johann Wolfgang von 28.\,8.\,1749 Frankfurt am Main – 22.\,3.\,1832 Weimar@\textsc{Goethe, Johann Wolfgang von} (28.\,8.\,1749 Frankfurt am Main – 22.\,3.\,1832 Weimar), \emph{Schriftsteller}|pw}, Leſſing\pwindex{Lessing, Gotthold Ephraim 22.\,1.\,1729 Kamenz – 15.\,2.\,1781 Braunschweig@\textsc{Lessing, Gotthold Ephraim} (22.\,1.\,1729 Kamenz – 15.\,2.\,1781 Braunschweig), \emph{Schriftsteller, Kritiker, Philosoph}|pw}, Hebbel\pwindex{Hebbel, Friedrich 18.\,3.\,1813 Wesselburen – 13.\,12.\,1863 Wien@\textsc{Hebbel, Friedrich} (18.\,3.\,1813 Wesselburen – 13.\,12.\,1863 Wien), \emph{Schriftsteller}|pw}, was Sie und andre über Kunſt{ }ſagen, leſe
               ich gern; manches beruhigt mich, indem es abſchließt, andres bewegt {\pb}mich, indem es Thore aufſchließt. Wir{ }ſprechen einmal
               darüber.\pend
           
\pstart
           \textsc{Brahm}\pwindex{Brahm, Otto 5.\,2.\,1856 Hamburg – 28.\,11.\,1912 Berlin@\textsc{Brahm, Otto} (5.\,2.\,1856 Hamburg – 28.\,11.\,1912 Berlin), \emph{Theaterleiter, Regisseur}|pw} iſt jetzt da, den ich perſönlich gern habe. Geſtern Abend waren er, Richard\pwindex{Beer-Hofmann, Richard 11.\,7.\,1866 Wien – 26.\,9.\,1945 New York City@\textsc{Beer-Hofmann, Richard} (11.\,7.\,1866 Wien – 26.\,9.\,1945 New York City), \emph{Schriftsteller}|pw}, Salten\pwindex{Salten, Felix 6.\,9.\,1869 Budapest – 8.\,10.\,1945 Zürich@\textsc{Salten, Felix} (6.\,9.\,1869 Budapest – 8.\,10.\,1945 Zürich), \emph{Schriftsteller, Journalist, Chefredakteur}|pw} u. Schwarzkopf\pwindex{Schwarzkopf, Gustav 7.\,11.\,1853 Wien – 13.\,11.\,1939 ebd.@\textsc{Schwarzkopf, Gustav} (7.\,11.\,1853 Wien – 13.\,11.\,1939 ebd.), \emph{Schriftsteller}|pw} bei mir. –
               Geleſen hab ich die Frzſ. Revol.\pwindex{Taine, Hippolyte 21.\,4.\,1828 Vouziers – 5.\,3.\,1893 Paris@\textsc{Taine, Hippolyte} (21.\,4.\,1828 Vouziers – 5.\,3.\,1893 Paris), \emph{Philosoph, Geschichtsschreiber}!Revolution@\strich\emph{Die Revolution}|pw} von \textsc{Taine}\pwindex{Taine, Hippolyte 21.\,4.\,1828 Vouziers – 5.\,3.\,1893 Paris@\textsc{Taine, Hippolyte} (21.\,4.\,1828 Vouziers – 5.\,3.\,1893 Paris), \emph{Philosoph, Geschichtsschreiber}|pw}, die Olla potrida des durchtriebenen
                  Fuchsmundi\pwindex{\textcolor{red}{\textsuperscript{XXXX indx1}}!Ollapatrida des durchgetriebenen Fuchsmundi@\strich\emph{Ollapatrida des durchgetriebenen Fuchsmundi}|pw}, die Noten zum Divan\pwindex{Goethe, Johann Wolfgang von 28.\,8.\,1749 Frankfurt am Main – 22.\,3.\,1832 Weimar@\textsc{Goethe, Johann Wolfgang von} (28.\,8.\,1749 Frankfurt am Main – 22.\,3.\,1832 Weimar), \emph{Schriftsteller}!West-östlicher Divan@\strich\emph{West-östlicher Divan}|pw} und
               einen engliſchen \label{K_L00546-1v}\edtext{Kriminalroman\pwindex{?? [Englischer Kriminalroman]@\emph{?? [Englischer Kriminalroman]}|pwv}}{\lemma{\textnormal{\emph{Kriminalroman}}}\Cendnote{\textnormal{nicht identifiziert}}}\label{K_L00546-1}. – Mein So{\geminationm}erplan iſt jetzt Norwegen\oindex{Norwegen@\textbf{Norwegen}|pw}, Schweden\oindex{Schweden@\textbf{Schweden}|pw}, Dänemark\oindex{Dänemark@\textbf{Dänemark}|pw}; und eine Novelle\pwindex{Schnitzler, Arthur 15.\,5.\,1862 Wien – 21.\,10.\,1931 ebd.@\textsc{Schnitzler, Arthur} (15.\,5.\,1862 Wien – 21.\,10.\,1931 ebd.), \emph{Schriftsteller, Mediziner}!Frau des Weisen. Erzählung@\strich\emph{Die Frau des Weisen. Erzählung}|pwv}. – Jetzt iſt ein Gewitter mit Blitz und
               Donner und Abend geh ich zum »Zerriſſenen\pwindex{\textcolor{red}{\textsuperscript{XXXX indx1}}!Zerrissene. Posse mit Gesang in drei Akten@\strich\emph{Der Zerrissene. Posse mit Gesang in drei Akten}|pw}«.\pend
           \pstart Herzlich der Ihre, \spacefill\mbox{AS.}\pend{}\selectlanguage{ngerman}\endnumbering\briefempfaengerindex{Hofmannsthal, Hugo von@\textsc{Hofmannsthal, Hugo von}!zzzSchnitzler, Arthur@\emph{von Arthur Schnitzler}!1896-05-231@{23. 5. 1896}|)be}\mylabel{L00546h}  \newcommand{\dateiname}{L00546}\newcommand{\titel}{Arthur Schnitzler an Hugo von Hofmannsthal, 23. 5. 1896}\newcommand{\editorInnen}{Martin Anton Müller und Gerd-Hermann Susen}%% latex-leseansicht-abspann.tex
%% Abspann für die Leseansicht.
%% Der Schalter \ifkorrekturansicht ist bereits durch den Vorspann gesetzt.

%% latex-abspann.tex
%% Gemeinsamer Abspann für Korrekturansicht und Leseansicht.
%% Setzt den Schalter \ifkorrekturansicht voraus (gesetzt in den
%% einbindenden Dateien latex-korrekturansicht-abspann.tex bzw.
%% latex-leseansicht-abspann.tex).
%% ---------------------------------------------------------------

\normalsize

% Das esempio-Environment wird nur in der Leseansicht benötigt
\ifkorrekturansicht\else
\newenvironment{esempio}[3]%
{
    \vspace{1.5ex}
    \rlap{\underline{#1}}
    \par
    \setlength{\parindent}{0cm}
    \nopagebreak
    \leftskip=#2cm
    \rightskip=#3cm
}
{
    \par
}
\fi

\doendnotes{C}
\bigskip
\vfill

\clearpage

\footnotesize

\ifkorrekturansicht
  \lohead{\textsc{register}}
\fi

% theindex-Environment neu definieren ohne reledmac
\makeatletter
\renewenvironment{theindex}{%
  \ifkorrekturansicht
    \section*{\indexname}%
  \else
    \subsubsection*{Index der erwähnten Entitäten}%
  \fi
  \setlength{\parindent}{0pt}%
  \setlength{\parskip}{0pt plus 0.3pt}%
  \let\item\@idxitem
}{%
  \ifkorrekturansicht\clearpage\fi
}
\makeatother

\IfFileExists{\jobname-pw.ind}{\input{\jobname-pw.ind}}{}

% Quellenangabe nur in der Leseansicht
\ifkorrekturansicht\else
% Fallback-Definitionen, falls die .tex-Datei \titel etc. nicht gesetzt hat
\providecommand{\titel}{}
\providecommand{\editorInnen}{}
\providecommand{\dateiname}{\jobname}

\vspace{3cm}

\vfill

\footnotesize
\textsc{Quelle}: \titel. Herausgegeben von {\editorInnen}. In: \emph{Arthur Schnitzler: Briefwechsel mit Autorinnen und Autoren}.
 Digitale Edition, https://schnitzler-briefe.acdh.oeaw.ac.at/{\dateiname}.html (Stand \today)
\fi

\end{document}


