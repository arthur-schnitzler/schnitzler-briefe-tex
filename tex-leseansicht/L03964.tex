%% latex-leseansicht-vorspann.tex
%% Vorspann für die Leseansicht.
%% Lädt die gemeinsame Datei latex-vorspann.tex mit nicht gesetztem Schalter.

\newif\ifkorrekturansicht
\korrekturansichtfalse

\input{../tex-inputs/latex-vorspann}


\section[Arthur Schnitzler an Berta Zuckerkandl, 11. 1. 1926]{L03964 Arthur Schnitzler an Berta Zuckerkandl, 11. 1. 1926}
\nopagebreak\mylabel{L03964v}
\rehead{ }\normalsize\beginnumbering\briefempfaengerindex{Zuckerkandl, Berta@\textsc{Zuckerkandl, Berta}!zzzSchnitzler, Arthur@\emph{von Arthur Schnitzler}!1926-01-111@{11. 1. 1926}|(be}
\toendnotes[C]{\smallbreak\pagebreak[2]}
\correspDesc{Versand  durch Arthur Schnitzler am 11. 1. 1926 in Wien
\newline{}Erhalt  durch Berta Zuckerkandl im Zeitraum [12. 1. 1926 – 16. 1. 1926?] in Paris}\toendnotes[C]{\smallbreak}
\Standort{DLA, HS.1985.1.2282.}
\physDesc{Brief, Durchschlag, 1 Blatt, 2 Seiten, 1300 Zeichen
\newline{}Schreibmaschine
\newline{}Handschrift: roter Buntstift, lateinische Kurrent (\noindent{}beschriftet: »\uline{Zuckerkandl}« und »\uline{Frkr}«, zehn Unterstreichungen)}\toendnotes[C]{\smallbreak}
\pstart
           \raggedleft{}{\pb}11. 1. 1926.\pend
           
\pstart{}Liebe und verehrte Frau Hofrätin.\pend\vspace{0.5em}
\pstart
           Besnard\pwindex{Besnard, Lucien 19.\,1.\,1872 Nonancourt – 1955 Paris@\textsc{Besnard, Lucien} (19.\,1.\,1872 Nonancourt – 1955 Paris), \emph{Schriftsteller}|pw}
               hat mir mit einem sehr liebenswürdigen \label{K_L03964-1v}\edtext{Brief}{\lemma{\textnormal{\emph{Brief}}}\Cendnote{\textnormal{Der Brief ist nicht überliefert, jedoch die Antwort: Arthur Schnitzler an Lucien Besnard\pwindex{Besnard, Lucien 19.\,1.\,1872 Nonancourt – 1955 Paris@\textsc{Besnard, Lucien} (19.\,1.\,1872 Nonancourt – 1955 Paris), \emph{Schriftsteller}|pwk}, 11. 1. 1926, \emph{Deutsches Literaturarchiv Marbach},
                     HS.1985.1.384,5.}}}\label{K_L03964-1} das \begin{otherlanguage}{french}Bulletin\end{otherlanguage}{ }geschickt, Ihrem
               freundlichen Rate folgend, \label{T_L03964-1v}\edtext{habe ich die Tantiemen zu gleichen Teilen repartiert}{\lemma{\textnormal{\emph{habe … repartiert}}}\Cendnote{\textnormal{In der Vorlage steht: »habe ich die Tantiemen folgend habe ich die Tantiemen zu gleichen Teilen repartiert«.}}}\label{T_L03964-1}. Komisch eigentlich,
               dass gerade dieses kleine Stück\pwindex{Schnitzler, Arthur 15. 5. 1862 Wien – 21. 10. 1931 ebd.@\textsc{Schnitzler, Arthur} (15. 5. 1862 Wien – 21. 10. 1931 ebd.), \emph{Schriftsteller, Mediziner}!tapfere Cassian. Puppenspiel in einem Akt@\strich\emph{Der tapfere Cassian. Puppenspiel in einem Akt}|pwv} nach so langer Zeit als \label{K_L03964-2v}\edtext{erste Pariser\oindex{Paris@\textbf{Paris}, \emph{Hauptstadt}|pw} Aufführung}{\lemma{\textnormal{\emph{erste Pariser Aufführung}}}\Cendnote{\textnormal{Das Projekt wurde nicht realisiert.}}}\label{K_L03964-2} eines
      meiner Werke herauskommt. Wissen sie vielleicht, was man dazu geben wird?\pend
           
\pstart
           Frau
               Clara Pollaczek\pwindex{Pollaczek, Clara Katharina 15.\,1.\,1875 Wien – 22.\,7.\,1951 ebd.@\textsc{Pollaczek, Clara Katharina} (15.\,1.\,1875 Wien – 22.\,7.\,1951 ebd.), \emph{Schriftstellerin}|pw} ist mit ihrer Uebersetzung\pwindex{Schnitzler, Arthur 15. 5. 1862 Wien – 21. 10. 1931 ebd.@\textsc{Schnitzler, Arthur} (15. 5. 1862 Wien – 21. 10. 1931 ebd.), \emph{Schriftsteller, Mediziner}!Madmoiselle Else@\strich\emph{Madmoiselle Else}|pwv} des »Fräulein Else\pwindex{Schnitzler, Arthur 15. 5. 1862 Wien – 21. 10. 1931 ebd.@\textsc{Schnitzler, Arthur} (15. 5. 1862 Wien – 21. 10. 1931 ebd.), \emph{Schriftsteller, Mediziner}!Fräulein Else@\strich\emph{Fräulein Else}|pw}« bald fertig, natürlich wird die Vervielfältigung noch
               einige Zeit in Anspruch nehmen; \label{K_L03964-3v}\edtext{wie lange bleiben Sie}{\lemma{\textnormal{\emph{wie lange bleiben Sie}}}\Cendnote{\textnormal{Die erste Begegnung nach der Reise fand laut \emph{Tagebuch}\pwindex{Schnitzler, Arthur 15. 5. 1862 Wien – 21. 10. 1931 ebd.@\textsc{Schnitzler, Arthur} (15. 5. 1862 Wien – 21. 10. 1931 ebd.), \emph{Schriftsteller, Mediziner}!Tagebuch@\strich\emph{Tagebuch}|pwk} am 21. 2. 1926 statt.}}}\label{K_L03964-3} noch in Paris\oindex{Paris@\textbf{Paris}, \emph{Hauptstadt}|pw}? Wird man Ihnen das
               Manuscript\pwindex{Schnitzler, Arthur 15. 5. 1862 Wien – 21. 10. 1931 ebd.@\textsc{Schnitzler, Arthur} (15. 5. 1862 Wien – 21. 10. 1931 ebd.), \emph{Schriftsteller, Mediziner}!Madmoiselle Else@\strich\emph{Madmoiselle Else}|pwv} noch dorthin senden können?\pend
           
\pstart
           Meine Reise nach Berlin\oindex{Berlin@\textbf{Berlin}, \emph{Hauptstadt}|pw} dürfte
               \label{K_L03964-4v}\edtext{im Februar}{\lemma{\textnormal{\emph{im Februar}}}\Cendnote{\textnormal{Schnitzler hielt sich vom 6. 2. 1926 bis zum 12. 2. 1926 in Berlin\oindex{Berlin@\textbf{Berlin}, \emph{Hauptstadt}|pwk} auf.}}}\label{K_L03964-4} stattfinden, \label{K_L03964-5v}\edtext{am 7. Februar}{\lemma{\textnormal{\emph{am 7. Februar}}}\Cendnote{\textnormal{
                  Die Lesung von \emph{Lieutenant Gustl}\pwindex{Schnitzler, Arthur 15. 5. 1862 Wien – 21. 10. 1931 ebd.@\textsc{Schnitzler, Arthur} (15. 5. 1862 Wien – 21. 10. 1931 ebd.), \emph{Schriftsteller, Mediziner}!Lieutenant Gustl. Novelle@\strich\emph{Lieutenant Gustl. Novelle}|pwk} durch Arthur Schnitzler, \emph{Fräulein Else}\pwindex{Schnitzler, Arthur 15. 5. 1862 Wien – 21. 10. 1931 ebd.@\textsc{Schnitzler, Arthur} (15. 5. 1862 Wien – 21. 10. 1931 ebd.), \emph{Schriftsteller, Mediziner}!Fräulein Else@\strich\emph{Fräulein Else}|pwk}
                    durch Elisabeth Bergner\pwindex{Bergner, Elisabeth 22.\,8.\,1897 Drohobych – 12.\,5.\,1986 London@\textsc{Bergner, Elisabeth} (22.\,8.\,1897 Drohobych – 12.\,5.\,1986 London), \emph{Schauspielerin}|pwk}\eventindex{Reichstag@\textbf{Reichstag}!Lesung von Lieutenant Gustl, Fräulein Else, 7.2.1926@Lesung von Lieutenant Gustl, Fräulein Else, 7.2.1926|pwk} fand am 7. 2. 1926 im Berliner\oindex{Berlin@\textbf{Berlin}, \emph{Hauptstadt}|pwk}{ }Reichstag\oindex{Reichstag@\textbf{Reichstag}, \emph{Regierungsgebäude}|pwk} statt.
               }}}\label{K_L03964-5} liest
               Elisabeth Bergner\pwindex{Bergner, Elisabeth 22.\,8.\,1897 Drohobych – 12.\,5.\,1986 London@\textsc{Bergner, Elisabeth} (22.\,8.\,1897 Drohobych – 12.\,5.\,1986 London), \emph{Schauspielerin}|pw} »Fräulein Else\pwindex{Schnitzler, Arthur 15. 5. 1862 Wien – 21. 10. 1931 ebd.@\textsc{Schnitzler, Arthur} (15. 5. 1862 Wien – 21. 10. 1931 ebd.), \emph{Schriftsteller, Mediziner}!Fräulein Else@\strich\emph{Fräulein Else}|pw}«, ich wahrscheinlich dazu den »Leutnant Gustl\pwindex{Schnitzler, Arthur 15. 5. 1862 Wien – 21. 10. 1931 ebd.@\textsc{Schnitzler, Arthur} (15. 5. 1862 Wien – 21. 10. 1931 ebd.), \emph{Schriftsteller, Mediziner}!Lieutenant Gustl. Novelle@\strich\emph{Lieutenant Gustl. Novelle}|pw}«. Möglicherweise wird Barnowsky\pwindex{Barnowsky, Victor 10.\,9.\,1875 Berlin – 9.\,8.\,1952 New York City@\textsc{Barnowsky, Victor} (10.\,9.\,1875 Berlin – 9.\,8.\,1952 New York City), \emph{Theaterleiter, Regisseur, Schauspieler}|pw} ungefähr zu gleicher
               Zeit \label{K_L03964-6v}\edtext{»Die Schwestern\pwindex{Schnitzler, Arthur 15. 5. 1862 Wien – 21. 10. 1931 ebd.@\textsc{Schnitzler, Arthur} (15. 5. 1862 Wien – 21. 10. 1931 ebd.), \emph{Schriftsteller, Mediziner}!Schwestern oder Casanova in Spa. Lustspiel in Versen@\strich\emph{Die Schwestern oder Casanova in Spa. Lustspiel in Versen}|pw}« und »Comtesse Mizi\pwindex{Schnitzler, Arthur 15. 5. 1862 Wien – 21. 10. 1931 ebd.@\textsc{Schnitzler, Arthur} (15. 5. 1862 Wien – 21. 10. 1931 ebd.), \emph{Schriftsteller, Mediziner}!Komtesse Mizzi oder: Der Familientag@\strich\emph{Komtesse Mizzi oder: Der Familientag}|pw}« aufführen}{\lemma{\textnormal{\emph{»Die … aufführen}}}\Cendnote{\textnormal{Es fand keine Inszenierung von Werken Schnitzlers durch Victor Barnowsky\pwindex{Barnowsky, Victor 10.\,9.\,1875 Berlin – 9.\,8.\,1952 New York City@\textsc{Barnowsky, Victor} (10.\,9.\,1875 Berlin – 9.\,8.\,1952 New York City), \emph{Theaterleiter, Regisseur, Schauspieler}|pwk} im Jahr 1926 statt.}}}\label{K_L03964-6}.\pend
           
\pstart
           Dass Sie, verehrte Freundin, bei
      Ihrer ungeheueren Inanspruchnahme auch für
      meine kleinen Angelegenheiten Zeit finden,
      rührt mich geradezu. Ich hoffe, Sie haben nach
      allen Richtungen hin Erfolg, für sich selbst
      hoffentlich noch mehr als für die Anderen,
      befinden sich wohl und frisch und denken meiner mit gleicher freundschaftlicher Sympathie
      wie ich Ihrer.\pend
           
\pstart
           {\pb}Mit den herzlichsten Grüssen{\\[\baselineskip]} Ihr getreuer\pend
           \leftskip=0em{}{\vspace{1\baselineskip}}
\pstart
           \noindent{}Frau Hofrätin Berta Zuckerkandl,{\\}Paris\oindex{Paris@\textbf{Paris}, \emph{Hauptstadt}|pw}.\pend
           \selectlanguage{ngerman}\endnumbering\briefempfaengerindex{Zuckerkandl, Berta@\textsc{Zuckerkandl, Berta}!zzzSchnitzler, Arthur@\emph{von Arthur Schnitzler}!1926-01-111@{11. 1. 1926}|)be}\mylabel{L03964h}
\begin{anhang}
\end{anhang}\newcommand{\dateiname}{L03964}\newcommand{\titel}{Arthur Schnitzler an Berta Zuckerkandl, 11. 1. 1926}\newcommand{\editorInnen}{Herausgegeben von Jahnke, SelmaMüller, Martin Anton}%% latex-leseansicht-abspann.tex
%% Abspann für die Leseansicht.
%% Der Schalter \ifkorrekturansicht ist bereits durch den Vorspann gesetzt.

%% latex-abspann.tex
%% Gemeinsamer Abspann für Korrekturansicht und Leseansicht.
%% Setzt den Schalter \ifkorrekturansicht voraus (gesetzt in den
%% einbindenden Dateien latex-korrekturansicht-abspann.tex bzw.
%% latex-leseansicht-abspann.tex).
%% ---------------------------------------------------------------

\normalsize

% Das esempio-Environment wird nur in der Leseansicht benötigt
\ifkorrekturansicht\else
\newenvironment{esempio}[3]%
{
    \vspace{1.5ex}
    \rlap{\underline{#1}}
    \par
    \setlength{\parindent}{0cm}
    \nopagebreak
    \leftskip=#2cm
    \rightskip=#3cm
}
{
    \par
}
\fi

\doendnotes{C}
\bigskip
\vfill

\clearpage

\footnotesize

\ifkorrekturansicht
  \lohead{\textsc{register}}
\fi

% theindex-Environment neu definieren ohne reledmac
\makeatletter
\renewenvironment{theindex}{%
  \ifkorrekturansicht
    \section*{\indexname}%
  \else
    \subsubsection*{Index der erwähnten Entitäten}%
  \fi
  \setlength{\parindent}{0pt}%
  \setlength{\parskip}{0pt plus 0.3pt}%
  \let\item\@idxitem
}{%
  \ifkorrekturansicht\clearpage\fi
}
\makeatother

\IfFileExists{\jobname-pw.ind}{\input{\jobname-pw.ind}}{}

% Quellenangabe nur in der Leseansicht
\ifkorrekturansicht\else
% Fallback-Definitionen, falls die .tex-Datei \titel etc. nicht gesetzt hat
\providecommand{\titel}{}
\providecommand{\editorInnen}{}
\providecommand{\dateiname}{\jobname}

\vspace{3cm}

\vfill

\footnotesize
\textsc{Quelle}: \titel. Herausgegeben von {\editorInnen}. In: \emph{Arthur Schnitzler: Briefwechsel mit Autorinnen und Autoren}.
 Digitale Edition, https://schnitzler-briefe.acdh.oeaw.ac.at/{\dateiname}.html (Stand \today)
\fi

\end{document}


