%% latex-leseansicht-vorspann.tex
%% Vorspann für die Leseansicht.
%% Lädt die gemeinsame Datei latex-vorspann.tex mit nicht gesetztem Schalter.

\newif\ifkorrekturansicht
\korrekturansichtfalse

\input{../tex-inputs/latex-vorspann}


               \section[Arthur Schnitzler und Leopold Kramer an Richard Beer-Hofmann, 5. 6. 1898]{ Arthur Schnitzler und Leopold Kramer an Richard Beer-Hofmann,
               5. 6. 1898}\nopagebreak\mylabel{v}\rehead{ }\begin{ledgroupsized}[t]{13cm}\normalsize\beginnumbering\briefempfaengerindex{Beer-Hofmann, Richard@\textsc{Beer-Hofmann, Richard}!zzzKramer, Leopold@\emph{von Leopold Kramer}!1898-06-051@{5. 6. 1898}|(be}\briefempfaengerindex{Beer-Hofmann, Richard@\textsc{Beer-Hofmann, Richard}!zzzSchnitzler, Arthur@\emph{von Arthur Schnitzler}!1898-06-051@{5. 6. 1898}|(be} \toendnotes[C]{\smallbreak\pagebreak[2]} \Standort{YCGL, MSS 31.}
\physDesc{Bildpostkarte
\newline{}Handschrift Arthur Schnitzler: Bleistift, deutsche Kurrent\newline{}Handschrift Leopold Kramer: Bleistift, lateinische Kurrent\newline{}Versand: Stempel: »\nobreak{}\oindex{Steindorf am Ossiacher See@\textbf{Steindorf am Ossiacher See}|pwk}Steindorf am Ossiacher See, {[}5{]} 6 {[}98{]}\nobreak{}«.  }\buchAbdrucke{\weitereDrucke{Arthur Schnitzler, Richard Beer-Hofmann: \emph{Briefwechsel 1891–1931}. Hg. Konstanze Fliedl. Wien, Zürich: \emph{Europaverlag} 1992, S. 117.} }\pstart{}{\pb}\textsc{Dr. Richard Beer-Hofmann}\pend{}\pstart{}\textsc{Steindorf\oindex{Steindorf am Ossiacher See@\textbf{Steindorf am Ossiacher See}|pw}}\pend{}\pstart{}\textsc{am Ossiacher See\oindex{Ossiacher See@\textbf{Ossiacher See}|pw}}\pend{}\pstart{}\textsc{Kärnthen\oindex{Kaernten@\textbf{Kärnten}|pw}}\pend{}{\bigskip}\pstart
           \noindent{}\centering{}\textcolor{gray}{\textbf{{\pb}Gruss aus Krieglach\oindex{Krieglach@\textbf{Krieglach}|pw}. P. K. Rosegger\pwindex{Rosegger, Peter 31.07.1843 – 26.06.1918@\textsc{Rosegger, Peter} (31.07.1843 – 26.06.1918), \emph{Schriftsteller}|pw}’s
                     Geburtshaus. P. K. Rosegger\pwindex{Rosegger, Peter 31.07.1843 – 26.06.1918@\textsc{Rosegger, Peter} (31.07.1843 – 26.06.1918), \emph{Schriftsteller}|pw}’s Villa}}\pend
           \pstart
           \raggedleft{}{\pb}So{\geminationn}tag.\pend
           \pstart
           Hier iſt das erſte Nachtquartier. I{\geminationm}er näher. Herzlichſt
               Ihr \spacefill\mbox{Arth}\pend
           \pstart
           \noindent{}{[}hs. Kramer:{]} D\textsuperscript{r} Schnitzler maltraitirt
               mich schrecklich\pend
           \pstart
           Ich komme luftleer nach Ossiach\oindex{Ossiach@\textbf{Ossiach}|pw} –
                  \spacefill\mbox{Kramer}\pend
           \endnumbering\briefempfaengerindex{Beer-Hofmann, Richard@\textsc{Beer-Hofmann, Richard}!zzzKramer, Leopold@\emph{von Leopold Kramer}!1898-06-051@{5. 6. 1898}|)be}\briefempfaengerindex{Beer-Hofmann, Richard@\textsc{Beer-Hofmann, Richard}!zzzSchnitzler, Arthur@\emph{von Arthur Schnitzler}!1898-06-051@{5. 6. 1898}|)be}\mylabel{h}\end{ledgroupsized}  \newcommand{\dateiname}{L00803}\newcommand{\titel}{Arthur Schnitzler und Leopold Kramer an Richard Beer-Hofmann, 5. 6. 1898}\newcommand{\editorInnen}{Martin Anton Müller und Gerd-Hermann Susen}%% latex-leseansicht-abspann.tex
%% Abspann für die Leseansicht.
%% Der Schalter \ifkorrekturansicht ist bereits durch den Vorspann gesetzt.

%% latex-abspann.tex
%% Gemeinsamer Abspann für Korrekturansicht und Leseansicht.
%% Setzt den Schalter \ifkorrekturansicht voraus (gesetzt in den
%% einbindenden Dateien latex-korrekturansicht-abspann.tex bzw.
%% latex-leseansicht-abspann.tex).
%% ---------------------------------------------------------------

\normalsize

% Das esempio-Environment wird nur in der Leseansicht benötigt
\ifkorrekturansicht\else
\newenvironment{esempio}[3]%
{
    \vspace{1.5ex}
    \rlap{\underline{#1}}
    \par
    \setlength{\parindent}{0cm}
    \nopagebreak
    \leftskip=#2cm
    \rightskip=#3cm
}
{
    \par
}
\fi

\doendnotes{C}
\bigskip
\vfill

\clearpage

\footnotesize

\ifkorrekturansicht
  \lohead{\textsc{register}}
\fi

% theindex-Environment neu definieren ohne reledmac
\makeatletter
\renewenvironment{theindex}{%
  \ifkorrekturansicht
    \section*{\indexname}%
  \else
    \subsubsection*{Index der erwähnten Entitäten}%
  \fi
  \setlength{\parindent}{0pt}%
  \setlength{\parskip}{0pt plus 0.3pt}%
  \let\item\@idxitem
}{%
  \ifkorrekturansicht\clearpage\fi
}
\makeatother

\IfFileExists{\jobname-pw.ind}{\input{\jobname-pw.ind}}{}

% Quellenangabe nur in der Leseansicht
\ifkorrekturansicht\else
% Fallback-Definitionen, falls die .tex-Datei \titel etc. nicht gesetzt hat
\providecommand{\titel}{}
\providecommand{\editorInnen}{}
\providecommand{\dateiname}{\jobname}

\vspace{3cm}

\vfill

\footnotesize
\textsc{Quelle}: \titel. Herausgegeben von {\editorInnen}. In: \emph{Arthur Schnitzler: Briefwechsel mit Autorinnen und Autoren}.
 Digitale Edition, https://schnitzler-briefe.acdh.oeaw.ac.at/{\dateiname}.html (Stand \today)
\fi

\end{document}


      