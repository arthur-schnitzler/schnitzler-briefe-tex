%% latex-korrekturansicht-vorspann.tex
%% Vorspann für die Korrekturansicht.
%% Lädt die gemeinsame Datei latex-vorspann.tex mit gesetztem Schalter.

\newif\ifkorrekturansicht
\korrekturansichttrue

\input{../tex-inputs/latex-vorspann}


\section[Georg Brandes an Arthur Schnitzler, {[}18. 3. 1900{]}]{L01022 Georg Brandes an Arthur Schnitzler, {[}18. 3. 1900{]}}
\nopagebreak\mylabel{L01022v}
\rehead{ }\normalsize\beginnumbering\briefempfaengerindex{Schnitzler, Arthur@\textsc{Schnitzler, Arthur}!zzzBrandes, Georg@\emph{von Georg Brandes}!1900-03-181@{{[}18. 3. 1900{]}}|(be}
\toendnotes[C]{\smallbreak\pagebreak[2]}\Standort{CUL, Schnitzler, B 17.}
\physDesc{Brief, 1 Blatt, 1 Seite, 206 Zeichen
\newline{}Handschrift: schwarze Tinte, lateinische Kurrent
\newline{}Ordnung: von Schnitzler mit Bleistift datiert: »18/3 1900, von unbekannter Hand mit Bleistift nummeriert:
                                    »17« }
\buchAbdrucke{\weitereDrucke{Georg Brandes, Arthur Schnitzler: \emph{Ein Briefwechsel}. Bern: \emph{Francke} 1956, S. 79.} }
\pstart
           \raggedleft{}{\pb}Hotel Krantz\oindex{Hotel Krantz@\textbf{Hotel Krantz}, \emph{Hotel (K.HTL)}|pw}{\\}am Mehlmarkt {\\}Sonntag\pend
           
\pstart{}Lieber Dr. Schnitzler\pend\vspace{0.5em}
\pstart
           Warum sehe ich Sie denn gar nicht dies Mal{ }{\dotstwo} Ich habe mich jeden Tag nach Ihnen gesehnt und sehe nur
               gleichgültige Gesichter.\pend
           
\pstart
           Ihr{\\[\baselineskip]}\spacefill\mbox{Georg Brandes}\pend
           \leftskip=0em{}
\pstart
           \noindent{}Montag Abend?\pend
           \selectlanguage{ngerman}\endnumbering\briefempfaengerindex{Schnitzler, Arthur@\textsc{Schnitzler, Arthur}!zzzBrandes, Georg@\emph{von Georg Brandes}!1900-03-181@{{[}18. 3. 1900{]}}|)be}\mylabel{L01022h}  \normalsize

\doendnotes{C}
\bigskip
\vfill

\clearpage

\footnotesize

\lohead{\textsc{register}}

% Definiere theindex-Environment komplett neu ohne reledmac
\makeatletter
\renewenvironment{theindex}{%
  \section*{\indexname}%
  \setlength{\parindent}{0pt}%
  \setlength{\parskip}{0pt plus 0.3pt}%
  \let\item\@idxitem
}{%
  \clearpage
}
\makeatother

\IfFileExists{\jobname-pw.ind}{\input{\jobname-pw.ind}}{}

\end{document}

      