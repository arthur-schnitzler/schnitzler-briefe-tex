%% latex-korrekturansicht-vorspann.tex
%% Vorspann für die Korrekturansicht.
%% Lädt die gemeinsame Datei latex-vorspann.tex mit gesetztem Schalter.

\newif\ifkorrekturansicht
\korrekturansichttrue

\input{../tex-inputs/latex-vorspann}


\section[Georg Brandes an Arthur Schnitzler, 10. 3. 1899]{L00905 Georg Brandes an Arthur Schnitzler, 10. 3. 1899}
\nopagebreak\mylabel{L00905v}
\rehead{ }\normalsize\beginnumbering\briefempfaengerindex{Schnitzler, Arthur@\textsc{Schnitzler, Arthur}!zzzBrandes, Georg@\emph{von Georg Brandes}!1899-03-101@{10. 3. 1899}|(be}
\toendnotes[C]{\smallbreak\pagebreak[2]}\Standort{CUL, Schnitzler, B 17.}
\physDesc{Brief, 1 Blatt, 4 Seiten, 2293 Zeichen
\newline{}Handschrift: Bleistift, lateinische Kurrent
\newline{}Ordnung: mit Bleistift von unbekannter Hand nummeriert:
                                    »14« }
\buchAbdrucke{\weitereDrucke{Georg Brandes, Arthur Schnitzler: \emph{Ein Briefwechsel}. Bern: \emph{Francke} 1956, S. 73–74.} }\toendnotes[C]{\smallbreak}
\pstart
           \raggedleft{}{\pb}Kopenhagen\oindex{Kopenhagen@\textbf{Kopenhagen}, \emph{P.PPLC}|pw}{ }10 März 99\pend
           
\pstart{}Liebster Dr. Schnitzler\pend\vspace{0.5em}
\pstart
           Ich bin leider noch im Bett; bald sind jetzt 3 Monate so vergangen. Ich schreibe
               Ihnen nur heute weil ich Jemand\pwindex{Larsen, Karl 28.07.1860 – 11.07.1931@\textsc{Larsen, Karl} (28.07.1860 – 11.07.1931), \emph{Schriftsteller/Schriftstellerin}|pwv} gestern eine Karte für Sie gab und nicht will, dass Sie sich dadurch
               im Geringsten verpflichtet glauben sollen. Es war mir nicht möglich Nein zu sagen. Es
               ist der dänische\oindex{Daenemark@\textbf{Dänemark}, \emph{A.PCLI}|pw} Schriftsteller Karl Larsen\pwindex{Larsen, Karl 28.07.1860 – 11.07.1931@\textsc{Larsen, Karl} (28.07.1860 – 11.07.1931), \emph{Schriftsteller/Schriftstellerin}|pw}, ein talentvoller Mensch,
               gewissenhafter Psycholog, sehr feinhörig in allem Sprachlichen, ein wahrer
               Phonograph, aber langweilig, weil er immer nur von \uline{sich} spricht, immer nur an seinen litterarischen \uline{Vortheil} denkt und Kritiken und öffentliches Lob haben will. Sie kennen den
               Typus.\pend
           
\pstart
           Aber er kann Ihnen jedenfalls einen {\pb}Gruss aus Kopenhagen\oindex{Kopenhagen@\textbf{Kopenhagen}, \emph{P.PPLC}|pw} bringen.\pend
           
\pstart
           So entzückt ich war über Ihr letztes grösseres Schaupiel\pwindex{Vermaechtnis. Schauspiel in drei Akten@\emph{Das Vermächtnis. Schauspiel in drei Akten}|pwv} – ich entsinne mich des Titels nicht – wo der
               junge Mann im ersten Akt stirbt – so fremd ist mir der kl. Einakter\pwindex{gruene Kakadu. Groteske in einem Akt@\emph{Der grüne Kakadu. Groteske in einem Akt}|pwv} den Sie mir kürzlich schickten.
               Ich weiss ja nicht ob irgend eine historische Notiz zu Grunde liegt, sonst aber kommt
               die Idee mir sonderbar vor, dass vornehme Leute – seien sie auch noch so abgespannt –
               eine Kneipe besuchen sollten um sich von \uline{Schauspielern
                  revolutionäre Scenen vorspielen zu lassen}. Es ist so verdammt künstlich, so
                  »\label{K_L00905-1v}\edtext{ausklamüstirt}{\lemma{\textnormal{\emph{ausklamüstirt}}}\Cendnote{\textnormal{ausklamüsern: (zu sehr) im Detail
                  ausgedacht}}}\label{K_L00905-1}«, wie die Norddeutschen\oindex{Deutschland@\textbf{Deutschland}, \emph{A.PCLI}|pw}
               sagen.\pend
           
\pstart
           Sonst wissen Sie, dass ich in Sie verliebt bin und alles was Sie machen {\pb}gut finde und jede Gelegenheit
               ergreife Sie mündlich und schriftlich zu preisen.\pend
           
\pstart
           Ist es nicht sonderbar? Mein so ruhiges und würdiges \label{K_L00905-2v}\edtext{Manifest\pwindex{Daenentum in Suedjuetland@\emph{Das Dänentum in Südjütland}|pwv}}{\lemma{\textnormal{\emph{Manifest}}}\Cendnote{\textnormal{Es erschien zuerst als \emph{Danskheden i Sønderjylland}\pwindex{Daenentum in Suedjuetland@\emph{Das Dänentum in Südjütland}|pwk} In: \emph{Tilskueren}\pwindex{Tilskueren@\emph{Tilskueren}|pwk}, Jg. 16, März 1899,
                     S. 185–199, dann als \emph{Das Dänenthum in Südjütland}\pwindex{Daenentum in Suedjuetland@\emph{Das Dänentum in Südjütland}|pwk}. In: \emph{Die Zukunft}\pwindex{Zukunft@\emph{Die Zukunft}|pwk}, Bd. 27,
                        8. 4. 1899, S. 58–71.}}}\label{K_L00905-2} an die Deutschen haben
               sowohl die Neue freie Presse\orgindex{Neue Freie Presse@Neue Freie Presse|pw} wie die Frankfurter Zeitung\orgindex{Frankfurter Zeitung@Frankfurter Zeitung|pw} abgewiesen. Nun versuche ich
               mein Glück bei Barth\pwindex{Barth, Theodor 16.07.1849 – 03.06.1909@\textsc{Barth, Theodor} (16.07.1849 – 03.06.1909), \emph{Politiker/Politikerin, Herausgeber/Herausgeberin}|pw}’s Die Nation\orgindex{Nation@Die Nation|pw}. Ich lasse es in allen Sprachen sonst erscheinen. Es
               ist ein Bogen gross über die schleswigsche\oindex{Suedschleswig@\textbf{Südschleswig}, \emph{P.PPLA3}|pw}
               Sache.\pend
           
\pstart
           Ich habe sonst wenig arbeiten können. Nur Annie
                  Vivanti\pwindex{Vivanti, Annie 07.04.1866 – 20.02.1942@\textsc{Vivanti, Annie} (07.04.1866 – 20.02.1942), \emph{Schriftsteller/Schriftstellerin}|pw} aus dem Italiänischen\oindex{Italien@\textbf{Italien}, \emph{A.PCLI}|pw} in
                  \label{K_L00905-3v}\edtext{dänische\oindex{Daenemark@\textbf{Dänemark}, \emph{A.PCLI}|pw}{ }Verse\pwindex{Gedichte]@\emph{[Gedichte]}|pwv}}{\lemma{\textnormal{\emph{dänische Verse}}}\Cendnote{\textnormal{Georg Brandes\pwindex{Brandes, Georg 04.02.1842 – 19.02.1927@\textsc{Brandes, Georg} (04.02.1842 – 19.02.1927)|pwk}: \emph{Annie Vivanti}\pwindex{Annie Vivanti@\emph{Annie Vivanti}|pwk}. In: \emph{Tilskueren}\pwindex{Tilskueren@\emph{Tilskueren}|pwk}, Jg. 16, Februar 1899,
                  S. 107–124.}}}\label{K_L00905-3} gebracht.\pend
           
\pstart
           Sie liebenswürdiger fragten mich einmal in einem Brief: Wie sind Ihre Verse\pwindex{Ungdomsvers [Jugendgedichte]@\emph{Ungdomsvers [Jugendgedichte]}|pwv}, sind sie gut? Nansen\pwindex{Nansen, Peter 20.01.1861 – 31.07.1918@\textsc{Nansen, Peter} (20.01.1861 – 31.07.1918), \emph{Schriftsteller/Schriftstellerin, Journalist/Journalistin, Verleger/Verlegerin}|pw} findet sie akademisch, ein Urtheil, das ich ein bischen
               komisch finde, denn sie {\pb}sind sehr
               persönlich, aber als Verse\pwindex{Ungdomsvers [Jugendgedichte]@\emph{Ungdomsvers [Jugendgedichte]}|pwv}
               sind sie gut. Das Einzige auf der Welt was ich kann ist dänisch\oindex{Daenemark@\textbf{Dänemark}, \emph{A.PCLI}|pw} schreiben.\pend
           
\pstart
           Ich drücke Ihre Hände. Kürzlich erfuhr ich, dass Goldmann\pwindex{Goldmann, Paul 31.01.1865 – 25.09.1935@\textsc{Goldmann, Paul} (31.01.1865 – 25.09.1935), \emph{Schriftsteller/Schriftstellerin, Journalist/Journalistin}|pw} wieder in Europa\oindex{Europa@\textbf{Europa}, \emph{Kontinent (A.KNT)}|pw} ist. Das freut
               mich.\pend
           
\pstart
           Ihr ganz ergebener{\\[\baselineskip]}\spacefill\mbox{Georg Brandes}\pend
           \leftskip=0em{}
\pstart
           \noindent{}Man fängt in nächster Woche hier an, meine Gesammelte Schriften\pwindex{Samlede Skrifter [Gesammelte Werke]@\emph{Samlede Skrifter [Gesammelte Werke]}|pw} (!) herauszugeben und glaubt an einen Erfolg.\pend
           \selectlanguage{ngerman}\endnumbering\briefempfaengerindex{Schnitzler, Arthur@\textsc{Schnitzler, Arthur}!zzzBrandes, Georg@\emph{von Georg Brandes}!1899-03-101@{10. 3. 1899}|)be}\mylabel{L00905h}  \normalsize

\doendnotes{C}
\bigskip
\vfill

\clearpage

\footnotesize

\lohead{\textsc{register}}

% Definiere theindex-Environment komplett neu ohne reledmac
\makeatletter
\renewenvironment{theindex}{%
  \section*{\indexname}%
  \setlength{\parindent}{0pt}%
  \setlength{\parskip}{0pt plus 0.3pt}%
  \let\item\@idxitem
}{%
  \clearpage
}
\makeatother

\IfFileExists{\jobname-pw.ind}{\input{\jobname-pw.ind}}{}

\end{document}

      