%% latex-leseansicht-vorspann.tex
%% Vorspann für die Leseansicht.
%% Lädt die gemeinsame Datei latex-vorspann.tex mit nicht gesetztem Schalter.

\newif\ifkorrekturansicht
\korrekturansichtfalse

\input{../tex-inputs/latex-vorspann}


\section[Georg Brandes an Arthur Schnitzler, 10. 3. 1899]{L00905 Georg Brandes an Arthur Schnitzler, 10. 3. 1899}
\nopagebreak\mylabel{L00905v}
\rehead{ }\normalsize\beginnumbering\briefempfaengerindex{Schnitzler, Arthur@\textsc{Schnitzler, Arthur}!zzzBrandes, Georg@\emph{von Georg Brandes}!1899-03-101@{10. 3. 1899}|(be}
\toendnotes[C]{\smallbreak\pagebreak[2]}
\correspDesc{Versand  durch Georg Brandes am 10. 3. 1899 \textbf{Ort fehlend} 
\newline{}Erhalt  durch Arthur Schnitzler im Zeitraum [10. 3. 1899
                  – 14. 3. 1899?] in Wien}\toendnotes[C]{\smallbreak}
\Standort{CUL, Schnitzler, B 17.}
\physDesc{Brief, 1 Blatt, 4 Seiten, 2293 Zeichen
\newline{}Handschrift: Bleistift, lateinische Kurrent
\newline{}Ordnung: mit Bleistift von unbekannter Hand nummeriert:
                                    »14« }
\buchAbdrucke{\weitereDrucke{Georg Brandes, Arthur Schnitzler: \emph{Ein Briefwechsel}. Herausgegeben von Kurt Bergel. Bern: \emph{Francke} 1956, S. 73–74.} }\toendnotes[C]{\smallbreak}
\pstart
           \raggedleft{}{\pb}Kopenhagen\oindex{Kopenhagen@\textbf{Kopenhagen}, \emph{Hauptstadt}|pw}{ }10 März 99\pend
           
\pstart{}Liebster Dr. Schnitzler\pend\vspace{0.5em}
\pstart
           Ich bin leider noch im Bett; bald sind jetzt 3 Monate so vergangen. Ich schreibe
               Ihnen nur heute weil ich Jemand\pwindex{Larsen, Karl 28.\,7.\,1860 Rendsburg – 11.\,7.\,1931 Kopenhagen@\textsc{Larsen, Karl} (28.\,7.\,1860 Rendsburg – 11.\,7.\,1931 Kopenhagen), \emph{Schriftsteller}|pwv} gestern eine Karte für Sie gab und nicht will, dass Sie sich dadurch
               im Geringsten verpflichtet glauben sollen. Es war mir nicht möglich Nein zu sagen. Es
               ist der dänische\oindex{Dänemark@\textbf{Dänemark}|pw} Schriftsteller Karl Larsen\pwindex{Larsen, Karl 28.\,7.\,1860 Rendsburg – 11.\,7.\,1931 Kopenhagen@\textsc{Larsen, Karl} (28.\,7.\,1860 Rendsburg – 11.\,7.\,1931 Kopenhagen), \emph{Schriftsteller}|pw}, ein talentvoller Mensch,
               gewissenhafter Psycholog, sehr feinhörig in allem Sprachlichen, ein wahrer
               Phonograph, aber langweilig, weil er immer nur von \uline{sich} spricht, immer nur an seinen litterarischen \uline{Vortheil} denkt und Kritiken und öffentliches Lob haben will. Sie kennen den
               Typus.\pend
           
\pstart
           Aber er kann Ihnen jedenfalls einen {\pb}Gruss aus Kopenhagen\oindex{Kopenhagen@\textbf{Kopenhagen}, \emph{Hauptstadt}|pw} bringen.\pend
           
\pstart
           So entzückt ich war über Ihr letztes grösseres Schaupiel\pwindex{Schnitzler, Arthur 15.\,5.\,1862 Wien – 21.\,10.\,1931 ebd.@\textsc{Schnitzler, Arthur} (15.\,5.\,1862 Wien – 21.\,10.\,1931 ebd.), \emph{Schriftsteller, Mediziner}!Vermächtnis. Schauspiel in drei Akten@\strich\emph{Das Vermächtnis. Schauspiel in drei Akten}|pwv} – ich entsinne mich des Titels nicht – wo der
               junge Mann im ersten Akt stirbt – so fremd ist mir der kl. Einakter\pwindex{Schnitzler, Arthur 15.\,5.\,1862 Wien – 21.\,10.\,1931 ebd.@\textsc{Schnitzler, Arthur} (15.\,5.\,1862 Wien – 21.\,10.\,1931 ebd.), \emph{Schriftsteller, Mediziner}!grüne Kakadu. Groteske in einem Akt@\strich\emph{Der grüne Kakadu. Groteske in einem Akt}|pwv} den Sie mir kürzlich schickten.
               Ich weiss ja nicht ob irgend eine historische Notiz zu Grunde liegt, sonst aber kommt
               die Idee mir sonderbar vor, dass vornehme Leute – seien sie auch noch so abgespannt –
               eine Kneipe besuchen sollten um sich von \uline{Schauspielern
                  revolutionäre Scenen vorspielen zu lassen}. Es ist so verdammt künstlich, so
                  »\label{K_L00905-1v}\edtext{ausklamüstirt}{\lemma{\textnormal{\emph{ausklamüstirt}}}\Cendnote{\textnormal{ausklamüsern: (zu sehr) im Detail
                  ausgedacht}}}\label{K_L00905-1}«, wie die Norddeutschen\oindex{Deutschland@\textbf{Deutschland}|pw}
               sagen.\pend
           
\pstart
           Sonst wissen Sie, dass ich in Sie verliebt bin und alles was Sie machen {\pb}gut finde und jede Gelegenheit
               ergreife Sie mündlich und schriftlich zu preisen.\pend
           
\pstart
           Ist es nicht sonderbar? Mein so ruhiges und würdiges \label{K_L00905-2v}\edtext{Manifest\pwindex{Brandes, Georg 4.\,2.\,1842 Kopenhagen – 19.\,2.\,1927 ebd.@\textsc{Brandes, Georg} (4.\,2.\,1842 Kopenhagen – 19.\,2.\,1927 ebd.)!Dänentum in Südjütland@\strich\emph{Das Dänentum in Südjütland}|pwv}}{\lemma{\textnormal{\emph{Manifest}}}\Cendnote{\textnormal{Es erschien zuerst als \emph{Danskheden i Sønderjylland}\pwindex{Brandes, Georg 4.\,2.\,1842 Kopenhagen – 19.\,2.\,1927 ebd.@\textsc{Brandes, Georg} (4.\,2.\,1842 Kopenhagen – 19.\,2.\,1927 ebd.)!Dänentum in Südjütland@\strich\emph{Das Dänentum in Südjütland}|pwk} In: \emph{Tilskueren}\pwindex{Tilskueren@\emph{Tilskueren}|pwk}, Jg. 16, März 1899,
                     S. 185–199, dann als \emph{Das Dänenthum in Südjütland}\pwindex{Brandes, Georg 4.\,2.\,1842 Kopenhagen – 19.\,2.\,1927 ebd.@\textsc{Brandes, Georg} (4.\,2.\,1842 Kopenhagen – 19.\,2.\,1927 ebd.)!Dänentum in Südjütland@\strich\emph{Das Dänentum in Südjütland}|pwk}. In: \emph{Die Zukunft}\pwindex{Zukunft@\emph{Die Zukunft}|pwk}, Bd. 27,
                        8. 4. 1899, S. 58–71.}}}\label{K_L00905-2} an die Deutschen haben
               sowohl die Neue freie Presse\orgindex{Neue Freie Presse@Neue Freie Presse|pw} wie die Frankfurter Zeitung\orgindex{Frankfurter Zeitung@Frankfurter Zeitung|pw} abgewiesen. Nun versuche ich
               mein Glück bei Barth\pwindex{Barth, Theodor 16.\,7.\,1849 Duderstadt – 3.\,6.\,1909 Baden-Baden@\textsc{Barth, Theodor} (16.\,7.\,1849 Duderstadt – 3.\,6.\,1909 Baden-Baden), \emph{Politiker, Herausgeber}|pw}’s Die Nation\orgindex{Nation@Die Nation|pw}. Ich lasse es in allen Sprachen sonst erscheinen. Es
               ist ein Bogen gross über die schleswigsche\oindex{Südschleswig@\textbf{Südschleswig}, \emph{Hauptstadt}|pw}
               Sache.\pend
           
\pstart
           Ich habe sonst wenig arbeiten können. Nur Annie
                  Vivanti\pwindex{Vivanti, Annie 7.\,4.\,1866 London – 20.\,2.\,1942 Turin@\textsc{Vivanti, Annie} (7.\,4.\,1866 London – 20.\,2.\,1942 Turin), \emph{Schriftstellerin}|pw} aus dem Italiänischen\oindex{Italien@\textbf{Italien}|pw} in
                  \label{K_L00905-3v}\edtext{dänische\oindex{Dänemark@\textbf{Dänemark}|pw}{ }Verse\pwindex{Vivanti, Annie 7.\,4.\,1866 London – 20.\,2.\,1942 Turin@\textsc{Vivanti, Annie} (7.\,4.\,1866 London – 20.\,2.\,1942 Turin), \emph{Schriftstellerin}!Gedichte]@\strich\emph{[Gedichte]}|pwv}}{\lemma{\textnormal{\emph{dänische Verse}}}\Cendnote{\textnormal{Georg Brandes\pwindex{Brandes, Georg 4.\,2.\,1842 Kopenhagen – 19.\,2.\,1927 ebd.@\textsc{Brandes, Georg} (4.\,2.\,1842 Kopenhagen – 19.\,2.\,1927 ebd.)|pwk}: \emph{Annie Vivanti}\pwindex{Annie Vivanti@\emph{Annie Vivanti}|pwk}. In: \emph{Tilskueren}\pwindex{Tilskueren@\emph{Tilskueren}|pwk}, Jg. 16, Februar 1899,
                  S. 107–124.}}}\label{K_L00905-3} gebracht.\pend
           
\pstart
           Sie liebenswürdiger fragten mich einmal in einem Brief: Wie sind Ihre Verse\pwindex{Brandes, Georg 4.\,2.\,1842 Kopenhagen – 19.\,2.\,1927 ebd.@\textsc{Brandes, Georg} (4.\,2.\,1842 Kopenhagen – 19.\,2.\,1927 ebd.)!Ungdomsvers [Jugendgedichte]@\strich\emph{Ungdomsvers [Jugendgedichte]}|pwv}, sind sie gut? Nansen\pwindex{Nansen, Peter 20.\,1.\,1861 Kopenhagen – 31.\,7.\,1918 Mariager@\textsc{Nansen, Peter} (20.\,1.\,1861 Kopenhagen – 31.\,7.\,1918 Mariager), \emph{Schriftsteller, Journalist, Verleger}|pw} findet sie akademisch, ein Urtheil, das ich ein bischen
               komisch finde, denn sie {\pb}sind sehr
               persönlich, aber als Verse\pwindex{Brandes, Georg 4.\,2.\,1842 Kopenhagen – 19.\,2.\,1927 ebd.@\textsc{Brandes, Georg} (4.\,2.\,1842 Kopenhagen – 19.\,2.\,1927 ebd.)!Ungdomsvers [Jugendgedichte]@\strich\emph{Ungdomsvers [Jugendgedichte]}|pwv}
               sind sie gut. Das Einzige auf der Welt was ich kann ist dänisch\oindex{Dänemark@\textbf{Dänemark}|pw} schreiben.\pend
           
\pstart
           Ich drücke Ihre Hände. Kürzlich erfuhr ich, dass Goldmann\pwindex{Goldmann, Paul 31.\,1.\,1865 Breslau – 25.\,9.\,1935 Wien@\textsc{Goldmann, Paul} (31.\,1.\,1865 Breslau – 25.\,9.\,1935 Wien), \emph{Schriftsteller, Journalist}|pw} wieder in Europa\oindex{Europa@\textbf{Europa}|pw} ist. Das freut
               mich.\pend
           
\pstart
           Ihr ganz ergebener{\\[\baselineskip]}\spacefill\mbox{Georg Brandes}\pend
           \leftskip=0em{}
\pstart
           \noindent{}Man fängt in nächster Woche hier an, meine Gesammelte Schriften\pwindex{Brandes, Georg 4.\,2.\,1842 Kopenhagen – 19.\,2.\,1927 ebd.@\textsc{Brandes, Georg} (4.\,2.\,1842 Kopenhagen – 19.\,2.\,1927 ebd.)!Samlede Skrifter [Gesammelte Werke]@\strich\emph{Samlede Skrifter [Gesammelte Werke]}|pw} (!) herauszugeben und glaubt an einen Erfolg.\pend
           \selectlanguage{ngerman}\endnumbering\briefempfaengerindex{Schnitzler, Arthur@\textsc{Schnitzler, Arthur}!zzzBrandes, Georg@\emph{von Georg Brandes}!1899-03-101@{10. 3. 1899}|)be}\mylabel{L00905h}  \newcommand{\dateiname}{L00905}\newcommand{\titel}{Georg Brandes an Arthur Schnitzler, 10. 3. 1899}\newcommand{\editorInnen}{Martin Anton Müller und Gerd-Hermann Susen}%% latex-leseansicht-abspann.tex
%% Abspann für die Leseansicht.
%% Der Schalter \ifkorrekturansicht ist bereits durch den Vorspann gesetzt.

%% latex-abspann.tex
%% Gemeinsamer Abspann für Korrekturansicht und Leseansicht.
%% Setzt den Schalter \ifkorrekturansicht voraus (gesetzt in den
%% einbindenden Dateien latex-korrekturansicht-abspann.tex bzw.
%% latex-leseansicht-abspann.tex).
%% ---------------------------------------------------------------

\normalsize

% Das esempio-Environment wird nur in der Leseansicht benötigt
\ifkorrekturansicht\else
\newenvironment{esempio}[3]%
{
    \vspace{1.5ex}
    \rlap{\underline{#1}}
    \par
    \setlength{\parindent}{0cm}
    \nopagebreak
    \leftskip=#2cm
    \rightskip=#3cm
}
{
    \par
}
\fi

\doendnotes{C}
\bigskip
\vfill

\clearpage

\footnotesize

\ifkorrekturansicht
  \lohead{\textsc{register}}
\fi

% theindex-Environment neu definieren ohne reledmac
\makeatletter
\renewenvironment{theindex}{%
  \ifkorrekturansicht
    \section*{\indexname}%
  \else
    \subsubsection*{Index der erwähnten Entitäten}%
  \fi
  \setlength{\parindent}{0pt}%
  \setlength{\parskip}{0pt plus 0.3pt}%
  \let\item\@idxitem
}{%
  \ifkorrekturansicht\clearpage\fi
}
\makeatother

\IfFileExists{\jobname-pw.ind}{\input{\jobname-pw.ind}}{}

% Quellenangabe nur in der Leseansicht
\ifkorrekturansicht\else
% Fallback-Definitionen, falls die .tex-Datei \titel etc. nicht gesetzt hat
\providecommand{\titel}{}
\providecommand{\editorInnen}{}
\providecommand{\dateiname}{\jobname}

\vspace{3cm}

\vfill

\footnotesize
\textsc{Quelle}: \titel. Herausgegeben von {\editorInnen}. In: \emph{Arthur Schnitzler: Briefwechsel mit Autorinnen und Autoren}.
 Digitale Edition, https://schnitzler-briefe.acdh.oeaw.ac.at/{\dateiname}.html (Stand \today)
\fi

\end{document}


