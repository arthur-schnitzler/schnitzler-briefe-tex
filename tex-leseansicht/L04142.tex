%% latex-leseansicht-vorspann.tex
%% Vorspann für die Leseansicht.
%% Lädt die gemeinsame Datei latex-vorspann.tex mit nicht gesetztem Schalter.

\newif\ifkorrekturansicht
\korrekturansichtfalse

\input{../tex-inputs/latex-vorspann}


\section[Arthur Schnitzler an Gustav Schwarzkopf, 9. 9. 1899]{L04142 Arthur Schnitzler an Gustav Schwarzkopf, 9. 9. 1899}
\nopagebreak\mylabel{L04142v}
\rehead{ }\normalsize\beginnumbering\briefempfaengerindex{Schwarzkopf, Gustav@\textsc{Schwarzkopf, Gustav}!zzzSchnitzler, Arthur@\emph{von Arthur Schnitzler}!1899-09-092@{9. 9. 1899}|(be}
\toendnotes[C]{\smallbreak\pagebreak[2]}
\correspDesc{Versand  durch Arthur Schnitzler am 9. 9. 1899 in Bad Ischl
\newline{}Erhalt  durch Gustav Schwarzkopf im Zeitraum [10. 9. 1899 – 14. 9. 1899?] in Wien}\toendnotes[C]{\smallbreak}
\Standort{CUL, Schnitzler, B 96.}
\physDesc{Brief, 2 Blätter, 8 Seiten, 2897 Zeichen
\newline{}Handschrift: Bleistift, deutsche Kurrent}
\buchAbdrucke{\weitereDrucke{Arthur Schnitzler: \emph{Briefe 1875–1912}. Herausgegeben von Therese Nickl und Heinrich Schnitzler. Frankfurt am Main: \emph{S. Fischer} 1981, S. 376–378.} }\toendnotes[C]{\smallbreak}
\pstart
           \raggedleft{}{\pb}Samſt 9. 9. 99.\pend
           
\pstart
           \raggedleft{}\textsc{Ischl}, Rudolfshöhe\oindex{Hotel und Pension Rudolfshöhe (Leopold Petter)@\textbf{Hotel und Pension Rudolfshöhe (Leopold Petter)}, \emph{Hotel}|pw}\pend
           \vspace{0.5em}
\pstart
           lieber Guſtav, Sie{ }ſehen, ich bin noch immer hier. Heut hab ich
               wieder einmal vorläufig, höchſt vorläufig, und in ſchlechter Sti{\geminationm}ung mein \label{K_L04142-1v}\edtext{Stück\pwindex{Schnitzler, Arthur 15. 5. 1862 Wien – 21. 10. 1931 ebd.@\textsc{Schnitzler, Arthur} (15. 5. 1862 Wien – 21. 10. 1931 ebd.), \emph{Schriftsteller, Mediziner}!Schleier der Beatrice. Schauspiel in fünf Akten@\strich\emph{Der Schleier der Beatrice. Schauspiel in fünf Akten}|pwv} abgeſchloſſen}{\lemma{\textnormal{\emph{Stück abgeschlossen}}}\Cendnote{\textnormal{Vgl. A. S.: \emph{Tagebuch}, 9. 9. 1899. }}}\label{K_L04142-1}; eine
               Zeit, in den Mittelakten gings gut; aber in den letzten Tagen bin ich wieder ganz
               gottverlaſſen,  und über die Schwächlichkeit meines Ausdrucks in einer vielleicht
               unverhältnismäßg Niedergeſchlagenheit. Da wirkt natürlich auch andres mit, das
                  \label{K_L04142-2v}\edtext{Ohr}{\lemma{\textnormal{\emph{Ohr}}}\Cendnote{\textnormal{Schnitzler litt an
                  einer Verknöcherung des Innenohrs.}}}\label{K_L04142-2}{\pb}vor allem. – \label{K_L04142-3v}\edtext{Hugo\pwindex{Hofmannsthal, Hugo von 1.\,2.\,1874 Wien – 15.\,7.\,1929 Rodaun@\textsc{Hofmannsthal, Hugo von} (1.\,2.\,1874 Wien – 15.\,7.\,1929 Rodaun), \emph{Schriftsteller}|pw}
               war bis vorgeſtern}{\lemma{\textnormal{\emph{Hugo
               war bis vorgestern}}}\Cendnote{\textnormal{Vgl. A. S.: \emph{Tagebuch}, 7. 9. 1899. }}}\label{K_L04142-3} hier;{ }ſeine Geſellſchaft that mir ſehr wohl. Er ſchrieb anfangs fleißig; dann war er
               plötzlich ganz herunter u. iſt jetzt in Altauſſee\oindex{Altaussee@\textbf{Altaussee}, \emph{Verwaltungsgebiet}|pw}. Ich bleibe noch \label{K_L04142-4v}\edtext{bis Dinſtag}{\lemma{\textnormal{\emph{bis Dinstag}}}\Cendnote{\textnormal{12. 9. 1899.
               }}}\label{K_L04142-4} hier, fahre dann nach München\oindex{München@\textbf{München}|pw}, wo ich
               mit M. E.\pwindex{Elsinger, Marie *~28.\,2.\,1874 St. Pölten@\textsc{Elsinger, Marie} (*~28.\,2.\,1874 St. Pölten), \emph{Schauspielerin}|pw} zuſa{\geminationm}entreffen und eine kleine Reiſe unternehmen ſoll, will, muſs, werde – ich
               weiſs wirklich ſelbſt nicht. Iſt übrigens vom 1. October bei \textsc{Neuma{\geminationn} Hofer\pwindex{Neumann-Hofer, Gilbert Otto 4.\,2.\,1857 Bol’shiye Berezhki – 14.\,4.\,1941 Detmold@\textsc{Neumann-Hofer, Gilbert Otto} (4.\,2.\,1857 Bol’shiye Berezhki – 14.\,4.\,1941 Detmold), \emph{Kritiker, Theaterleiter}|pw}\orgindex{Lessing-Theater@Lessing-Theater|pwv}} engagirt, u ziemlich {\pb}komiſch,
               wie ich ſowohl von M. E.\pwindex{Elsinger, Marie *~28.\,2.\,1874 St. Pölten@\textsc{Elsinger, Marie} (*~28.\,2.\,1874 St. Pölten), \emph{Schauspielerin}|pw} als von M. G.\pwindex{Glümer, Marie 3.\,7.\,1867 Wien – 16.\,11.\,1925 München@\textsc{Glümer, Marie} (3.\,7.\,1867 Wien – 16.\,11.\,1925 München), \emph{Schauspielerin}|pw} Briefe bekomme, daſs ſie dem \textsc{N. Hofer\pwindex{Neumann-Hofer, Gilbert Otto 4.\,2.\,1857 Bol’shiye Berezhki – 14.\,4.\,1941 Detmold@\textsc{Neumann-Hofer, Gilbert Otto} (4.\,2.\,1857 Bol’shiye Berezhki – 14.\,4.\,1941 Detmold), \emph{Kritiker, Theaterleiter}|pw}} die Rolle\pwindex{Schnitzler, Arthur 15. 5. 1862 Wien – 21. 10. 1931 ebd.@\textsc{Schnitzler, Arthur} (15. 5. 1862 Wien – 21. 10. 1931 ebd.), \emph{Schriftsteller, Mediziner}!Abschiedssouper@\strich\emph{Abschiedssouper}|pwv}
               zurückſchicken. – Nun hat er der Gl.\pwindex{Glümer, Marie 3.\,7.\,1867 Wien – 16.\,11.\,1925 München@\textsc{Glümer, Marie} (3.\,7.\,1867 Wien – 16.\,11.\,1925 München), \emph{Schauspielerin}|pw} die
                  Abſchieds-Annie\pwindex{Schnitzler, Arthur 15. 5. 1862 Wien – 21. 10. 1931 ebd.@\textsc{Schnitzler, Arthur} (15. 5. 1862 Wien – 21. 10. 1931 ebd.), \emph{Schriftsteller, Mediziner}!Abschiedssouper@\strich\emph{Abschiedssouper}|pwv} gegeben und
                  ko{\geminationm}t ſich wahrſcheinlich wie ein richtig
               diplomatiſcher Kerl vor, der die Beziehg ſeiner Künſtler zu verwerthen
               verſteht. – Das, war Sie ſo beißend »Zerſtreuung« zu ne{\geminationn}en pflegen, hat ſich hier auch gefunden, {\pb}gleich doppelt\pwindex{Bergmeister, Johanna 4.\,6.\,1869 Wien – 30.\,12.\,1924 Berlin@\textsc{Bergmeister, Johanna} (4.\,6.\,1869 Wien – 30.\,12.\,1924 Berlin)|pwv}\pwindex{Samek, Helene 5.\,3.\,1865 Wien – vor 1945@\textsc{Samek, Helene} (5.\,3.\,1865 Wien – vor 1945)|pwv}, aber so jüdiſch, daſs
               ich mich nach der katholiſchen Canaille wahrhaft ſehne. – Alſo wird es doch weiter
               oben zu heißen haben: \uline{will}. – \label{K_L04142-5v}\edtext{Um den 20. herum werde ich
               wahrſcheinlich in Berlin\oindex{Berlin@\textbf{Berlin}, \emph{Hauptstadt}|pw}}{\lemma{\textnormal{\emph{Um … Berlin}}}\Cendnote{\textnormal{Seine Ankunft in Berlin\oindex{Berlin@\textbf{Berlin}, \emph{Hauptstadt}|pwk}
                  verzögerte sich bis zum A. S.: \emph{Wiener Schnitzler}, 3. 10. 1899.}}}\label{K_L04142-5} ſein; anfangs dachte ich dem \label{K_L04142-6v}\edtext{Brahm\pwindex{Brahm, Otto 5.\,2.\,1856 Hamburg – 28.\,11.\,1912 Berlin@\textsc{Brahm, Otto} (5.\,2.\,1856 Hamburg – 28.\,11.\,1912 Berlin), \emph{Theaterleiter, Regisseur}|pw} mein Stück\pwindex{Schnitzler, Arthur 15. 5. 1862 Wien – 21. 10. 1931 ebd.@\textsc{Schnitzler, Arthur} (15. 5. 1862 Wien – 21. 10. 1931 ebd.), \emph{Schriftsteller, Mediziner}!Schleier der Beatrice. Schauspiel in fünf Akten@\strich\emph{Der Schleier der Beatrice. Schauspiel in fünf Akten}|pwv} dort vorleſen\eventindex{Luisenplatz 2@\textbf{Luisenplatz 2}!Private Lesung von Der Schleier der Beatrice, 7.10.1899@Private Lesung von Der Schleier der Beatrice, 7.10.1899|pwv}}{\lemma{\textnormal{\emph{Brahm … vorlesen}}}\Cendnote{\textnormal{Vgl. A. S.: \emph{Kulturveranstaltungen}, 7. 10. 1899.}}}\label{K_L04142-6} zu können; aber jetzt ko{\geminationm}ts
               mir abſolut unfertig vor. – – Auf \textsc{\label{K_L04142-7v}\edtext{Mercier\pwindex{Mercier, Auguste 8.\,12.\,1833 Arras – 3.\,3.\,1921 Paris@\textsc{Mercier, Auguste} (8.\,12.\,1833 Arras – 3.\,3.\,1921 Paris), \emph{Politiker}|pw}}{\lemma{\textnormal{\emph{Mercier}}}\Cendnote{\textnormal{Der Name des
                           ehemaligen französischen\oindex{Frankreich@\textbf{Frankreich}|pwk} Kriegsministers Auguste Mercier\pwindex{Mercier, Auguste 8.\,12.\,1833 Arras – 3.\,3.\,1921 Paris@\textsc{Mercier, Auguste} (8.\,12.\,1833 Arras – 3.\,3.\,1921 Paris), \emph{Politiker}|pwk} 
                           steht hier als Synonym for den (institutionellen) Verrat am Juden Alfred Dreyfus\pwindex{Dreyfus, Alfred 9.\,10.\,1859 Mulhouse – 12.\,7.\,1935 Paris@\textsc{Dreyfus, Alfred} (9.\,10.\,1859 Mulhouse – 12.\,7.\,1935 Paris), \emph{Militär}|pwk},
                           der seine Unschuld gerade wieder in einem Prozess beweisen musste.}}}\label{K_L04142-7}} und das andre Geſindel hab ich eine förmliche Wuth – trotz der \textsc{Neuen Freien\orgindex{Neue Freie Presse@Neue Freie Presse|pw}}{[}.{]}{ }{\pb}Sehr freuen würd es mich, we{\geminationn} ich am Mittwoch in München\oindex{München@\textbf{München}|pw}{ }\textsc{post rest.} ein kurzes Wort (das iſt doch mit möglichſter
               Beſcheidenheit ausgedrückt) von Ihnen fände. – \label{K_L04142-8v}\edtext{Ebermann\pwindex{Ebermann, Leo 16.\,7.\,1863 Draganovka – 9.\,10.\,1914 Wien@\textsc{Ebermann, Leo} (16.\,7.\,1863 Draganovka – 9.\,10.\,1914 Wien), \emph{Schriftsteller, Journalist, Rechtswissenschaftler}|pw}
               hab ich flüchtig geſprochen;}{\lemma{\textnormal{\emph{Ebermann … gesprochen;}}}\Cendnote{\textnormal{Das Treffen ist nicht im \emph{Tagebuch}\pwindex{Schnitzler, Arthur 15. 5. 1862 Wien – 21. 10. 1931 ebd.@\textsc{Schnitzler, Arthur} (15. 5. 1862 Wien – 21. 10. 1931 ebd.), \emph{Schriftsteller, Mediziner}!Tagebuch@\strich\emph{Tagebuch}|pwk} erwähnt.}}}\label{K_L04142-8} er behauptet ein \label{K_L04142-9v}\edtext{Stück\pwindex{Ebermann, Leo 16.\,7.\,1863 Draganovka – 9.\,10.\,1914 Wien@\textsc{Ebermann, Leo} (16.\,7.\,1863 Draganovka – 9.\,10.\,1914 Wien), \emph{Schriftsteller, Journalist, Rechtswissenschaftler}!?? [Versdrama für Josef Kainz]@\strich\emph{?? [Versdrama für Josef Kainz]}|pwv}}{\lemma{\textnormal{\emph{Stück}}}\Cendnote{\textnormal{Von Ebermann\pwindex{Ebermann, Leo 16.\,7.\,1863 Draganovka – 9.\,10.\,1914 Wien@\textsc{Ebermann, Leo} (16.\,7.\,1863 Draganovka – 9.\,10.\,1914 Wien), \emph{Schriftsteller, Journalist, Rechtswissenschaftler}|pwk} erschien kein Stück mehr. Am 13. 1. 1900 meldete die
                     \emph{Wiener Allgemeine Zeitung}\pwindex{Wiener Allgemeine Zeitung@\emph{Wiener Allgemeine Zeitung}|pwk} (Nr. 6555, S. 2), Ebermann\pwindex{Ebermann, Leo 16.\,7.\,1863 Draganovka – 9.\,10.\,1914 Wien@\textsc{Ebermann, Leo} (16.\,7.\,1863 Draganovka – 9.\,10.\,1914 Wien), \emph{Schriftsteller, Journalist, Rechtswissenschaftler}|pwk} habe ein Versdrama vollendet, das er für Josef Kainz\pwindex{Kainz, Josef 2.\,1.\,1858 Mosonmagyaróvár – 20.\,9.\,1910 Wien@\textsc{Kainz, Josef} (2.\,1.\,1858 Mosonmagyaróvár – 20.\,9.\,1910 Wien), \emph{Schauspieler}|pwk} geschrieben habe.}}}\label{K_L04142-9} vollendet zu haben; wie Sie ja wahrſcheinlich wiſſen
               werden. – Richard\pwindex{Beer-Hofmann, Richard 11.\,7.\,1866 Wien – 26.\,9.\,1945 New York City@\textsc{Beer-Hofmann, Richard} (11.\,7.\,1866 Wien – 26.\,9.\,1945 New York City), \emph{Schriftsteller}|pw}{[},{]} denken Sie, arbeitet bereits an ſeinem Drama\pwindex{Beer-Hofmann, Richard 11.\,7.\,1866 Wien – 26.\,9.\,1945 New York City@\textsc{Beer-Hofmann, Richard} (11.\,7.\,1866 Wien – 26.\,9.\,1945 New York City), \emph{Schriftsteller}!Graf von Charolais. Ein Trauerspiel@\strich\emph{Der Graf von Charolais. Ein Trauerspiel}|pwv}. – Morgen iſt hier \label{K_L04142-10v}\edtext{Fuhrmann Henſchel\pwindex{Hauptmann, Gerhart 15.\,11.\,1862 Szczawno-Zdrój – 6.\,6.\,1946 Jagniątków@\textsc{Hauptmann, Gerhart} (15.\,11.\,1862 Szczawno-Zdrój – 6.\,6.\,1946 Jagniątków), \emph{Schriftsteller}!Fuhrmann Henschel. Schauspiel in 5 Akten@\strich\emph{Fuhrmann Henschel. Schauspiel in 5 Akten}|pw}\eventindex{Lehártheater@\textbf{Lehártheater}!Aufführung von Fuhrmann Henschel, 10.9.1899@Aufführung von Fuhrmann Henschel, 10.9.1899|pwv}}{\lemma{\textnormal{\emph{Fuhrmann Henschel}}}\Cendnote{\textnormal{Schnitzler dürfte die Aufführung nicht
                     besucht und Maran\pwindex{Maran, Gustav 8.\,1.\,1854 Wien – 18.\,5.\,1917 Sulz im Wienerwald@\textsc{Maran, Gustav} (8.\,1.\,1854 Wien – 18.\,5.\,1917 Sulz im Wienerwald), \emph{Schauspieler}|pwk} nicht mitgespielt haben.}}}\label{K_L04142-10} – {\pb}der wirkliche von Hauptmann\pwindex{Hauptmann, Gerhart 15.\,11.\,1862 Szczawno-Zdrój – 6.\,6.\,1946 Jagniątków@\textsc{Hauptmann, Gerhart} (15.\,11.\,1862 Szczawno-Zdrój – 6.\,6.\,1946 Jagniątków), \emph{Schriftsteller}|pw} – mit \uline{Maran}\pwindex{Maran, Gustav 8.\,1.\,1854 Wien – 18.\,5.\,1917 Sulz im Wienerwald@\textsc{Maran, Gustav} (8.\,1.\,1854 Wien – 18.\,5.\,1917 Sulz im Wienerwald), \emph{Schauspieler}|pw}!! als Gaſt, offenbar in den Hauptrolle. – Der kleine Kraus\pwindex{Kraus, Karl 28.\,4.\,1874 Jičín – 12.\,6.\,1936 Wien@\textsc{Kraus, Karl} (28.\,4.\,1874 Jičín – 12.\,6.\,1936 Wien), \emph{Schriftsteller, Publizist, Schriftsteller}|pw} ſitzt im Theater\oindex{Lehártheater@\textbf{Lehártheater}, \emph{Theater}|pwv} (ich
               war bei einigen Offenbach\pwindex{Offenbach, Jacques 20.\,6.\,1819 Köln – 5.\,10.\,1880 Paris@\textsc{Offenbach, Jacques} (20.\,6.\,1819 Köln – 5.\,10.\,1880 Paris), \emph{Komponist}|pw}\eventindex{Lehártheater@\textbf{Lehártheater}!Aufführung von Pariser Leben, 27.8.1899@Aufführung von Pariser Leben, 27.8.1899|pw}\eventindex{Lehártheater@\textbf{Lehártheater}!Aufführung von La princesse de Trébizonde, 29.8.1899@Aufführung von La princesse de Trébizonde, 29.8.1899|pw}\eventindex{Lehártheater@\textbf{Lehártheater}!Aufführung von Blaubart, 1.9.1899@Aufführung von Blaubart, 1.9.1899|pw}’s, die Ihnen ſicher beſſer bekannt{ }ſind als dem \introOben{}hieſigen\introOben{}{ }Kapellmeiſter\pwindex{Raimann, Rudolf 7.\,5.\,1861 Veszprém – 26.\,9.\,1913 Wien@\textsc{Raimann, Rudolf} (7.\,5.\,1861 Veszprém – 26.\,9.\,1913 Wien), \emph{Komponist, Dirigent}|pwv}) – ſehr großartig; ſeine Stellung zu den Antiſemiten iſt doch
               das widerwärtigſte, was mir je vorgeko{\geminationm}en. Ja wenn es
               Einſicht, {\pb}Intention zu Gerechtigkeit
               wäre; aber es iſt ſchließlich auch nichts als Kriecherei – irgend was wie das, was
               ich einmal in einer Tramway erlebt habe, wie ein ſchäbiger jüdiſcher Commis vor Luëger\pwindex{Lueger, Karl 24.\,10.\,1844 Wien – 10.\,3.\,1910 ebd.@\textsc{Lueger, Karl} (24.\,10.\,1844 Wien – 10.\,3.\,1910 ebd.), \emph{Politiker}|pw} Platz machte und ſagte, »Bitte Herr Doktor« und
               entzückt war, von Luëger\pwindex{Lueger, Karl 24.\,10.\,1844 Wien – 10.\,3.\,1910 ebd.@\textsc{Lueger, Karl} (24.\,10.\,1844 Wien – 10.\,3.\,1910 ebd.), \emph{Politiker}|pw} keinen Fußtritt zu erhalten –
               kurz die Haltung des {\pb}kleinen Kraus\pwindex{Kraus, Karl 28.\,4.\,1874 Jičín – 12.\,6.\,1936 Wien@\textsc{Kraus, Karl} (28.\,4.\,1874 Jičín – 12.\,6.\,1936 Wien), \emph{Schriftsteller, Publizist, Schriftsteller}|pw} gegen die Antiſemiten – iſt nicht jüdiſch.
               (Vermeiden Sie es nach Thunlichkeit, dieſen Brief \textsc{Vergani\pwindex{Vergani, Ernst 15.\,3.\,1848 Stebnik – 19.\,2.\,1915 Emmersdorf an der Donau@\textsc{Vergani, Ernst} (15.\,3.\,1848 Stebnik – 19.\,2.\,1915 Emmersdorf an der Donau), \emph{Herausgeber}|pw}} oder \textsc{C. H. Wolff\pwindex{Wolf, Karl Hermann 27.\,1.\,1862 Cheb – 11.\,6.\,1941 Wien@\textsc{Wolf, Karl Hermann} (27.\,1.\,1862 Cheb – 11.\,6.\,1941 Wien), \emph{Schriftsteller, Herausgeber, Abgeordneter}|pw}} mitzutheilen.) – haben Sie \label{K_L04142-11v}\edtext{Muſchelkinder\pwindex{Berks, Maria von 10.\,8.\,1859 Livorno – 25.\,5.\,1910 Gorizia@\textsc{Berks, Maria von} (10.\,8.\,1859 Livorno – 25.\,5.\,1910 Gorizia), \emph{Schriftstellerin, Dramatikerin}!Muschelkinder. Schauspiel in vier Acten nach einem Roman von Guy de Maupassant@\strich\emph{Muschelkinder. Schauspiel in vier Acten nach einem Roman von Guy de Maupassant}|pw}}{\lemma{\textnormal{\emph{Muschelkinder}}}\Cendnote{\textnormal{Das von Marie von Berks\pwindex{Berks, Maria von 10.\,8.\,1859 Livorno – 25.\,5.\,1910 Gorizia@\textsc{Berks, Maria von} (10.\,8.\,1859 Livorno – 25.\,5.\,1910 Gorizia), \emph{Schriftstellerin, Dramatikerin}|pwk}
                  verfasste Schauspiel \emph{Muschelkinder}\pwindex{Berks, Maria von 10.\,8.\,1859 Livorno – 25.\,5.\,1910 Gorizia@\textsc{Berks, Maria von} (10.\,8.\,1859 Livorno – 25.\,5.\,1910 Gorizia), \emph{Schriftstellerin, Dramatikerin}!Muschelkinder. Schauspiel in vier Acten nach einem Roman von Guy de Maupassant@\strich\emph{Muschelkinder. Schauspiel in vier Acten nach einem Roman von Guy de Maupassant}|pwk} in vier Akten, nach dem Roman \emph{Pierre {\kaufmannsund} Jean}\pwindex{Maupassant, Guy de 5.\,8.\,1850 Tourville-sur-Arques – 7.\,7.\,1893 Paris@\textsc{Maupassant, Guy de} (5.\,8.\,1850 Tourville-sur-Arques – 7.\,7.\,1893 Paris), \emph{Schriftsteller}!Pierre und Jean@\strich\emph{Pierre {\kaufmannsund} Jean}|pwk} von
                  Guy de Maupassant\pwindex{Maupassant, Guy de 5.\,8.\,1850 Tourville-sur-Arques – 7.\,7.\,1893 Paris@\textsc{Maupassant, Guy de} (5.\,8.\,1850 Tourville-sur-Arques – 7.\,7.\,1893 Paris), \emph{Schriftsteller}|pwk}, hatte am 2. 9. 1899 am \emph{Volkstheater}\orgindex{Volkstheater@Volkstheater|pwk} in Wien\oindex{Wien@\textbf{Wien}, \emph{Verwaltungsgebiet}|pwk}
                  seine Premiere\eventindex{Volkstheater@\textbf{Volkstheater}!Premiere von Muschelkinder, 2.9.1899@Premiere von Muschelkinder, 2.9.1899|pwkv}.}}}\label{K_L04142-11} geſehn? –\pend
           
\pstart
           Leben Sie wohl und ſagen Sie{\\[\baselineskip]} mir, ob Sie nicht doch endlich{\\[\baselineskip]} ein
               Stück ſchreiben werden.{\\[\baselineskip]} Herzlich der Ihre{\\[\baselineskip]}\spacefill\mbox{ArtsSch.}\pend
           \leftskip=0em{}\selectlanguage{ngerman}\endnumbering\briefempfaengerindex{Schwarzkopf, Gustav@\textsc{Schwarzkopf, Gustav}!zzzSchnitzler, Arthur@\emph{von Arthur Schnitzler}!1899-09-092@{9. 9. 1899}|)be}\mylabel{L04142h}
\begin{anhang}
\end{anhang}\newcommand{\dateiname}{L04142}\newcommand{\titel}{Arthur Schnitzler an Gustav Schwarzkopf, 9. 9. 1899}\newcommand{\editorInnen}{Herausgegeben von Jahnke, SelmaMüller, Martin Anton}%% latex-leseansicht-abspann.tex
%% Abspann für die Leseansicht.
%% Der Schalter \ifkorrekturansicht ist bereits durch den Vorspann gesetzt.

%% latex-abspann.tex
%% Gemeinsamer Abspann für Korrekturansicht und Leseansicht.
%% Setzt den Schalter \ifkorrekturansicht voraus (gesetzt in den
%% einbindenden Dateien latex-korrekturansicht-abspann.tex bzw.
%% latex-leseansicht-abspann.tex).
%% ---------------------------------------------------------------

\normalsize

% Das esempio-Environment wird nur in der Leseansicht benötigt
\ifkorrekturansicht\else
\newenvironment{esempio}[3]%
{
    \vspace{1.5ex}
    \rlap{\underline{#1}}
    \par
    \setlength{\parindent}{0cm}
    \nopagebreak
    \leftskip=#2cm
    \rightskip=#3cm
}
{
    \par
}
\fi

\doendnotes{C}
\bigskip
\vfill

\clearpage

\footnotesize

\ifkorrekturansicht
  \lohead{\textsc{register}}
\fi

% theindex-Environment neu definieren ohne reledmac
\makeatletter
\renewenvironment{theindex}{%
  \ifkorrekturansicht
    \section*{\indexname}%
  \else
    \subsubsection*{Index der erwähnten Entitäten}%
  \fi
  \setlength{\parindent}{0pt}%
  \setlength{\parskip}{0pt plus 0.3pt}%
  \let\item\@idxitem
}{%
  \ifkorrekturansicht\clearpage\fi
}
\makeatother

\IfFileExists{\jobname-pw.ind}{\input{\jobname-pw.ind}}{}

% Quellenangabe nur in der Leseansicht
\ifkorrekturansicht\else
% Fallback-Definitionen, falls die .tex-Datei \titel etc. nicht gesetzt hat
\providecommand{\titel}{}
\providecommand{\editorInnen}{}
\providecommand{\dateiname}{\jobname}

\vspace{3cm}

\vfill

\footnotesize
\textsc{Quelle}: \titel. Herausgegeben von {\editorInnen}. In: \emph{Arthur Schnitzler: Briefwechsel mit Autorinnen und Autoren}.
 Digitale Edition, https://schnitzler-briefe.acdh.oeaw.ac.at/{\dateiname}.html (Stand \today)
\fi

\end{document}


