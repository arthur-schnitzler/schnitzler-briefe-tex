%% latex-leseansicht-vorspann.tex
%% Vorspann für die Leseansicht.
%% Lädt die gemeinsame Datei latex-vorspann.tex mit nicht gesetztem Schalter.

\newif\ifkorrekturansicht
\korrekturansichtfalse

\input{../tex-inputs/latex-vorspann}


\section[Arthur Schnitzler an Berta Zuckerkandl, 1. 12. 1923]{L03949 Arthur Schnitzler an Berta Zuckerkandl, 1. 12. 1923}
\nopagebreak\mylabel{L03949v}
\rehead{ }\normalsize\beginnumbering\briefempfaengerindex{Zuckerkandl, Berta@\textsc{Zuckerkandl, Berta}!zzzSchnitzler, Arthur@\emph{von Arthur Schnitzler}!1923-12-011@{1. 12. 1923}|(be}
\toendnotes[C]{\smallbreak\pagebreak[2]}
\correspDesc{Versand  durch Arthur Schnitzler am 1. 12. 1923 in Wien
\newline{}Erhalt  durch Berta Zuckerkandl im Zeitraum [1. 12. 1923
                  – 4. 12. 1923?] in Wien}\toendnotes[C]{\smallbreak}
\Standort{DLA, HS.1985.1.2282.}
\physDesc{Brief, Durchschlag, 1 Blatt, 1 Seite, 1604 Zeichen
\newline{}Schreibmaschine
\newline{}Handschrift: 1) roter Buntstift, lateinische Kurrent (\noindent{}beschriftet: »\uline{Zuckerkandl}«, vier Unterstreichungen)\hspace{1em}2) Bleistift, lateinische Kurrent (\noindent{}Korrekturen)\hspace{1em}}\toendnotes[C]{\smallbreak}
\pstart
           \raggedleft{}{\pb}1. 12. 1923.\pend
           
\pstart{}Liebe und verehrte Frau Hofrätin.\pend\vspace{0.5em}
\pstart
           Ich sende Ihnen hier eine Abschrift eines eben an Boutelleau\pwindex{Chardonne, Jacques 2.\,1.\,1884 Barbezieux-Saint-Hilaire – 29.\,5.\,1968 La Frette-sur-Seine@\textsc{Chardonne, Jacques} (2.\,1.\,1884 Barbezieux-Saint-Hilaire – 29.\,5.\,1968 La Frette-sur-Seine), \emph{Schriftsteller, Verleger}|pw}{ }\label{K_L03949-1v}\edtext{abgesandten Briefes}{\lemma{\textnormal{\emph{abgesandten Briefes}}}\Cendnote{\textnormal{Arthur
                     Schnitzler an Jacques Boutelleau\pwindex{Chardonne, Jacques 2.\,1.\,1884 Barbezieux-Saint-Hilaire – 29.\,5.\,1968 La Frette-sur-Seine@\textsc{Chardonne, Jacques} (2.\,1.\,1884 Barbezieux-Saint-Hilaire – 29.\,5.\,1968 La Frette-sur-Seine), \emph{Schriftsteller, Verleger}|pwk},
                     1. 12. 1923, Durchschlag im \emph{Deutschen Literaturarchiv Marbach},
                  HS.1985.1.1297. Es geht ausführlich um die finanziellen Bedigungen für die
                  mögliche Publikation von zwei Bänden mit einer Zusammenstellung von Einaktern bzw.
                  Novellen Schnitzlers, um die Auswahl der
                  Texte und Urheberrechtsfragen der Übersetzungen.}}}\label{K_L03949-1} und hoffe, dass Sie mit
               dessen Inhalt einverstanden sind. \label{K_L03949-2v}\edtext{Die
                  Bücher}{\lemma{\textnormal{\emph{Die
                  Bücher}}}\Cendnote{\textnormal{In seinem Brief an Jacques Boutelleau\pwindex{Chardonne, Jacques 2.\,1.\,1884 Barbezieux-Saint-Hilaire – 29.\,5.\,1968 La Frette-sur-Seine@\textsc{Chardonne, Jacques} (2.\,1.\,1884 Barbezieux-Saint-Hilaire – 29.\,5.\,1968 La Frette-sur-Seine), \emph{Schriftsteller, Verleger}|pwk} kündigt Schnitzler die Übersendung mehrerer
                  deutschsprachiger Sammlungen seiner Novellen an und darüber hinaus des Romans \emph{Frau Bertha Garlan}\pwindex{Schnitzler, Arthur 15. 5. 1862 Wien – 21. 10. 1931 ebd.@\textsc{Schnitzler, Arthur} (15. 5. 1862 Wien – 21. 10. 1931 ebd.), \emph{Schriftsteller, Mediziner}!Frau Bertha Garlan. Roman@\strich\emph{Frau Bertha Garlan. Roman}|pwk} und der beiden Dramen \emph{Zwischenspiel}\pwindex{Schnitzler, Arthur 15. 5. 1862 Wien – 21. 10. 1931 ebd.@\textsc{Schnitzler, Arthur} (15. 5. 1862 Wien – 21. 10. 1931 ebd.), \emph{Schriftsteller, Mediziner}!Zwischenspiel. Komödie in drei Akten@\strich\emph{Zwischenspiel. Komödie in drei Akten}|pwk} und \emph{Das weite Land}\pwindex{Schnitzler, Arthur 15. 5. 1862 Wien – 21. 10. 1931 ebd.@\textsc{Schnitzler, Arthur} (15. 5. 1862 Wien – 21. 10. 1931 ebd.), \emph{Schriftsteller, Mediziner}!weite Land. Tragikomödie in fünf Akten@\strich\emph{Das weite Land. Tragikomödie in fünf Akten}|pwk}.}}}\label{K_L03949-2} sind noch nicht abgeschickt.
               Vielleicht haben sie mir dazu einen Vorschlag zu machen.\pend
           
\pstart
           Was nun unsere gegenseitige Vertragsabmachung anbelangt, so möchte ich noch einmal
               rekapitulieren: Sie sind mit 15 {\%} an sämmtlichen durch Sie
               vermittelten Abschlüssen sowohl was Bücher als was Aufführungen anbelangt, beteiligt.
               Sie haben die Freundlichkeit mich von eventuellen für mich bestimmten Anträgen zu
               verständigen; gelangen Anträge direkt an mich, so übergebe ich sie entweder Ihnen zu
               weiterer Verhandlung, in welchem Fall natürlich auch die 15 {\%}ige Tantieme in Kraft bleibt, doch sind Sie auch an Abschlüssen, die ich ganz
               direkt – aber nie ohne Sie vorher zu verständigen, resp. ohne Ihren Rat einzuholen –
               erledige perzentuel u. zw. in diesem mit 5 {\%} beteiligt. Diese
               Abmachung würde vorläufig für ein Jahr \strikeout{gelten}, und automatisch auf ein weiteres verlängert \substVorne{}\textsuperscript{werden}\substDazwischen{}gelten\substHinten{},
               wenn nicht von der einen oder von der anderen Seite spätestens einen Monat vorher
               Kündigung erfolgt. Vielleicht haben Sie noch irgend etwas hiezu zu bemerken,
               verehrteste Frau Hofrätin, oder Ergänzungen vorzuschlagen, jedesfalls betrachte ich
               den 1. Dezember als den Beginn unserer quasi offiziellen Verbindung, die
               sich für mich hoffentlich ebenso erfreulich gestalten wird, als es unsere
               freundschaftliche bisher war, die gewiss für alle Zeit aufrecht bleiben wird.\pend
           
\pstart
           Mit herzlichem Dank und \label{T_L03949-1v}\edtext{Gruss}{\lemma{\textnormal{\emph{Gruss}}}\Cendnote{\textnormal{Auf dem Durchschlag steht: »Fruss«.}}}\label{T_L03949-1}{\\[\baselineskip]}Ihr sehr ergebener\pend
           \leftskip=0em{}{\vspace{1\baselineskip}}
\pstart
           \noindent{}Frau Hofrätin Bertha Zuckerkandl.\pend
           \selectlanguage{ngerman}\endnumbering\briefempfaengerindex{Zuckerkandl, Berta@\textsc{Zuckerkandl, Berta}!zzzSchnitzler, Arthur@\emph{von Arthur Schnitzler}!1923-12-011@{1. 12. 1923}|)be}\mylabel{L03949h}
\begin{anhang}
\end{anhang}\newcommand{\dateiname}{L03949}\newcommand{\titel}{Arthur Schnitzler an Berta Zuckerkandl, 1. 12. 1923}\newcommand{\editorInnen}{Herausgegeben von Jahnke, SelmaMüller, Martin Anton}%% latex-leseansicht-abspann.tex
%% Abspann für die Leseansicht.
%% Der Schalter \ifkorrekturansicht ist bereits durch den Vorspann gesetzt.

%% latex-abspann.tex
%% Gemeinsamer Abspann für Korrekturansicht und Leseansicht.
%% Setzt den Schalter \ifkorrekturansicht voraus (gesetzt in den
%% einbindenden Dateien latex-korrekturansicht-abspann.tex bzw.
%% latex-leseansicht-abspann.tex).
%% ---------------------------------------------------------------

\normalsize

% Das esempio-Environment wird nur in der Leseansicht benötigt
\ifkorrekturansicht\else
\newenvironment{esempio}[3]%
{
    \vspace{1.5ex}
    \rlap{\underline{#1}}
    \par
    \setlength{\parindent}{0cm}
    \nopagebreak
    \leftskip=#2cm
    \rightskip=#3cm
}
{
    \par
}
\fi

\doendnotes{C}
\bigskip
\vfill

\clearpage

\footnotesize

\ifkorrekturansicht
  \lohead{\textsc{register}}
\fi

% theindex-Environment neu definieren ohne reledmac
\makeatletter
\renewenvironment{theindex}{%
  \ifkorrekturansicht
    \section*{\indexname}%
  \else
    \subsubsection*{Index der erwähnten Entitäten}%
  \fi
  \setlength{\parindent}{0pt}%
  \setlength{\parskip}{0pt plus 0.3pt}%
  \let\item\@idxitem
}{%
  \ifkorrekturansicht\clearpage\fi
}
\makeatother

\IfFileExists{\jobname-pw.ind}{\input{\jobname-pw.ind}}{}

% Quellenangabe nur in der Leseansicht
\ifkorrekturansicht\else
% Fallback-Definitionen, falls die .tex-Datei \titel etc. nicht gesetzt hat
\providecommand{\titel}{}
\providecommand{\editorInnen}{}
\providecommand{\dateiname}{\jobname}

\vspace{3cm}

\vfill

\footnotesize
\textsc{Quelle}: \titel. Herausgegeben von {\editorInnen}. In: \emph{Arthur Schnitzler: Briefwechsel mit Autorinnen und Autoren}.
 Digitale Edition, https://schnitzler-briefe.acdh.oeaw.ac.at/{\dateiname}.html (Stand \today)
\fi

\end{document}


