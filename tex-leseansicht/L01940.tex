%% latex-leseansicht-vorspann.tex
%% Vorspann für die Leseansicht.
%% Lädt die gemeinsame Datei latex-vorspann.tex mit nicht gesetztem Schalter.

\newif\ifkorrekturansicht
\korrekturansichtfalse

\input{../tex-inputs/latex-vorspann}


\section[Arthur Schnitzler an Max Mell, 28. 6. 1910]{L01940 Arthur Schnitzler an Max Mell, 28. 6. 1910}
\nopagebreak\mylabel{L01940v}
\rehead{ }\normalsize\beginnumbering\briefempfaengerindex{Mell, Max@\textsc{Mell, Max}!zzzSchnitzler, Arthur@\emph{von Arthur Schnitzler}!1910-06-281@{28. 6. 1910}|(be}
\toendnotes[C]{\smallbreak\pagebreak[2]}
\correspDesc{Versand  durch Arthur Schnitzler am 28. 6. 1910 in Wien
\newline{}Erhalt  durch Max Mell im Zeitraum [28. 6. 1910
                  – 2. 7. 1910?] in Wien}\toendnotes[C]{\smallbreak}
\Standort{DLA, A:Schnitzler, HS.NZ85.1.1403.}
\physDesc{Brief, Durchschlag, 1 Blatt, 1 Seite, 940 Zeichen, Fragment
\newline{}Schreibmaschine
\newline{}Handschrift: roter Buntstift (\noindent{}»K{[}opie{]}« und zwei Unterstreichungen)}\toendnotes[C]{\smallbreak}
\pstart
           \raggedleft{}{\pb}XVIII. Spöttelg. 7\oindex{Wien@\textbf{Wien}!XVIII., Währing@\textbf{XVIII., Währing}!Edmund-Weiß-Gasse 7@\textbf{Edmund-Weiß-Gasse 7}, \emph{Wohngebäude}|pw}.{\\}Wien\oindex{Wien@\textbf{Wien}, \emph{Verwaltungsgebiet}|pw}, 28.\,6.\,1910.\pend
           
\pstart\center{}Lieber Herr Dr. Mell.\pend\vspace{0.5em}
\pstart
           Erst in den letzten Tagen bin ich dazu gekommen Ihr neues Buch\pwindex{Mell, Max 10.\,11.\,1882 Maribor – 13.\,12.\,1971 Wien@\textsc{Mell, Max} (10.\,11.\,1882 Maribor – 13.\,12.\,1971 Wien), \emph{Schriftsteller}!Jägerhaussage und andere Novellen@\strich\emph{Jägerhaussage und andere Novellen}|pwv} zu lesen, das wir bei einer Heimkehr
               von einer \label{K_L01940-1v}\edtext{Schweiz\oindex{Schweiz@\textbf{Schweiz}|pw}er Reise}{\lemma{\textnormal{\emph{Schweizer Reise}}}\Cendnote{\textnormal{Er war vom 17. 5. 1910 bis zum 2. 6. 1910 unterwegs gewesen.}}}\label{K_L01940-1}
               vorgefunden haben. Es war mir ein wirkliches Vergnügen die sichere und schöne
               Weiterentwicklung eines Talents darin ausgesprochen zu finden, dessen Anfänge ich
               schon mit Sympathie begleitet habe. Einen gewissen Manierismus, von dem sich in der
               ersten Geschichte vom jüdischen Sklaven\pwindex{Mell, Max 10.\,11.\,1882 Maribor – 13.\,12.\,1971 Wien@\textsc{Mell, Max} (10.\,11.\,1882 Maribor – 13.\,12.\,1971 Wien), \emph{Schriftsteller}!Geschichte eines jüdischen Sklaven@\strich\emph{Geschichte eines jüdischen Sklaven}|pw} noch
               Spuren finden, scheinen Sie nun gänzlich verlassen zu wollen. Die Erzählungen
               fliessen beinahe durchaus einfach und in einem unprätenziösen Verhältnis zum
               Grundeinfall dahin und der Grundeinfall selbst erweist von Fall zu Fall seine gesunde
               Art durch seine Fähigkeit allerlei kräftiges Detail zu produzieren, ohne das auch die
               bestgefundene Idee auf dem Wege hinauszusiechen pflegt.\pend
           
\pstart
           Seien Sie herzlich bedankt und gegrüsst auch von meiner Frau\pwindex{Schnitzler, Olga 17.\,1.\,1882 Wien – 13.\,1.\,1970 Lugano@\textsc{Schnitzler, Olga} (17.\,1.\,1882 Wien – 13.\,1.\,1970 Lugano), \emph{Schauspielerin, Sängerin}|pwv} und empfehlen Sie uns an Schwester\pwindex{Mell, Maria 12.\,7.\,1885 Maribor – 29.\,10.\,1954 Wien@\textsc{Mell, Maria} (12.\,7.\,1885 Maribor – 29.\,10.\,1954 Wien), \emph{Schauspielerin}|pwv}\pend
           \selectlanguage{ngerman}\endnumbering\briefempfaengerindex{Mell, Max@\textsc{Mell, Max}!zzzSchnitzler, Arthur@\emph{von Arthur Schnitzler}!1910-06-281@{28. 6. 1910}|)be}\mylabel{L01940h}  \newcommand{\dateiname}{L01940}\newcommand{\titel}{Arthur Schnitzler an Max Mell, 28. 6. 1910}\newcommand{\editorInnen}{Martin Anton Müller und Gerd-Hermann Susen}%% latex-leseansicht-abspann.tex
%% Abspann für die Leseansicht.
%% Der Schalter \ifkorrekturansicht ist bereits durch den Vorspann gesetzt.

%% latex-abspann.tex
%% Gemeinsamer Abspann für Korrekturansicht und Leseansicht.
%% Setzt den Schalter \ifkorrekturansicht voraus (gesetzt in den
%% einbindenden Dateien latex-korrekturansicht-abspann.tex bzw.
%% latex-leseansicht-abspann.tex).
%% ---------------------------------------------------------------

\normalsize

% Das esempio-Environment wird nur in der Leseansicht benötigt
\ifkorrekturansicht\else
\newenvironment{esempio}[3]%
{
    \vspace{1.5ex}
    \rlap{\underline{#1}}
    \par
    \setlength{\parindent}{0cm}
    \nopagebreak
    \leftskip=#2cm
    \rightskip=#3cm
}
{
    \par
}
\fi

\doendnotes{C}
\bigskip
\vfill

\clearpage

\footnotesize

\ifkorrekturansicht
  \lohead{\textsc{register}}
\fi

% theindex-Environment neu definieren ohne reledmac
\makeatletter
\renewenvironment{theindex}{%
  \ifkorrekturansicht
    \section*{\indexname}%
  \else
    \subsubsection*{Index der erwähnten Entitäten}%
  \fi
  \setlength{\parindent}{0pt}%
  \setlength{\parskip}{0pt plus 0.3pt}%
  \let\item\@idxitem
}{%
  \ifkorrekturansicht\clearpage\fi
}
\makeatother

\IfFileExists{\jobname-pw.ind}{\input{\jobname-pw.ind}}{}

% Quellenangabe nur in der Leseansicht
\ifkorrekturansicht\else
% Fallback-Definitionen, falls die .tex-Datei \titel etc. nicht gesetzt hat
\providecommand{\titel}{}
\providecommand{\editorInnen}{}
\providecommand{\dateiname}{\jobname}

\vspace{3cm}

\vfill

\footnotesize
\textsc{Quelle}: \titel. Herausgegeben von {\editorInnen}. In: \emph{Arthur Schnitzler: Briefwechsel mit Autorinnen und Autoren}.
 Digitale Edition, https://schnitzler-briefe.acdh.oeaw.ac.at/{\dateiname}.html (Stand \today)
\fi

\end{document}


