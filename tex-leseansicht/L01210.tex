%% latex-leseansicht-vorspann.tex
%% Vorspann für die Leseansicht.
%% Lädt die gemeinsame Datei latex-vorspann.tex mit nicht gesetztem Schalter.

\newif\ifkorrekturansicht
\korrekturansichtfalse

\input{../tex-inputs/latex-vorspann}


\section[Hugo von Hofmannsthal an Arthur Schnitzler, {{[}}28. 3. 1902{{]}}]{L01210 Hugo von Hofmannsthal an Arthur Schnitzler, {[}28. 3. 1902{]}}
\nopagebreak\mylabel{L01210v}
\rehead{ }\normalsize\beginnumbering\briefempfaengerindex{Schnitzler, Arthur@\textsc{Schnitzler, Arthur}!zzzHofmannsthal, Hugo von@\emph{von Hugo von Hofmannsthal}!1902-03-281@{28. 3. 1902}|(be}
\toendnotes[C]{\smallbreak\pagebreak[2]}
\correspDesc{Versand  durch Hugo von Hofmannsthal am 28. 3. 1902 \textbf{Ort fehlend} 
\newline{}Erhalt  durch Arthur Schnitzler im Zeitraum [28. 3. 1902
                  – 1. 4. 1902?] in Wien}\toendnotes[C]{\smallbreak}
\Standort{CUL, Schnitzler, B 43.}
\physDesc{Brief, 3 Blätter, 12 Seiten, 3021 Zeichen
\newline{}Handschrift: schwarze Tinte, deutsche Kurrent
\newline{}Schnitzler: mit Bleistift datiert: »28/3 902« 
\newline{}Ordnung: 1) mit Bleistift von unbekannter Hand nummeriert: »\strikeout{193}«  2) mit Bleistift von unbekannter Hand nummeriert:
                                    »186« und die folgenden Blätter mit
                                    »186.2.« beziehungsweise »186.3.«
                                 beschriftet}
\buchAbdrucke{\weitereDrucke{Hugo von Hofmannsthal, Arthur Schnitzler: \emph{Briefwechsel}. Herausgegeben von Therese Nickl und Heinrich Schnitzler. Frankfurt am Main: \emph{S. Fischer} 1964, S. 154–155.} }\toendnotes[C]{\smallbreak}
\pstart{}{\pb}mein lieber guter
                  Arthur,\pend\vspace{0.5em}
\pstart
           ich will Ihnen aufrichtig{ }ſagen, daſs mich Ihr Telegramm{ }ſehr verletzt hat. Ich will
               es deswegen lieber ausſprechen als verſchweigen, weil ich glaube, daſs das, was an{ }ſolchen Dingen für mich{ }ſo verletzend iſt, von Ihnen, als höchſt unwichtig, kaum {\pb}bemerkt wird und
                  da{[}ſ{]}s das Ganze in dem Moment vermieden wäre, wo Sie überhaupt
               zum Bewuſstſein davon kämen.\pend
           
\pstart
           In den 10 Jahren,{ }ſeit wir uns kennen, hab ich die unaufhörliche Freude eines intimen
               Verkehrs mit Ihnen immer unter{ }ſolchen Formen {\pb}genießen können, die Ihre
               Bequemlichkeit in Bezug auf Ort und Stunde des Zuſammentreffens etc nie tangiert
               haben. Es war nicht nur für Sie,{ }ſondern auch für mich bequemer, es war durch alle
               Umſtände gegeben, daſs Sie faſt nie zu mir gekommen{ }ſind und ich oft zu Ihnen etc.
               etc.\pend
           
\pstart
           {\pb}Und andererſeits haben Sie in
               dieſer langen Zeit wohl auch bemerken können, daſs mir ziemlich fern liegt Sie irgend
               wie durch Bekanntmachen mit Leuten etc in Anſpruch zu nehmen.\pend
           
\pstart
           Nun ereignet{ }ſich ein beſonderer ganz vereinzelter Fall: eine Frau\pwindex{Thun-Hohenstein-Salm-Reifferscheidt, Christiane von 12.\,6.\,1859 Doksy – 6.\,8.\,1935 Prag@\textsc{Thun-Hohenstein-Salm-Reifferscheidt, Christiane von} (12.\,6.\,1859 Doksy – 6.\,8.\,1935 Prag), \emph{Schriftstellerin}|pwv}, mit der ich ziemlich befreundet bin,
                  {\pb}und die wirklich eine
               merkwürdige Frau\pwindex{Thun-Hohenstein-Salm-Reifferscheidt, Christiane von 12.\,6.\,1859 Doksy – 6.\,8.\,1935 Prag@\textsc{Thun-Hohenstein-Salm-Reifferscheidt, Christiane von} (12.\,6.\,1859 Doksy – 6.\,8.\,1935 Prag), \emph{Schriftstellerin}|pwv} iſt, durch
               eine{ }ſeltene Übereinſtimmung von Güte, Vornehmheit und wirklichem Geiſt, dabei von
               der äußerſten Zurückhaltung, iſoliert und faſt menſchenſcheu, dieſe Frau\pwindex{Thun-Hohenstein-Salm-Reifferscheidt, Christiane von 12.\,6.\,1859 Doksy – 6.\,8.\,1935 Prag@\textsc{Thun-Hohenstein-Salm-Reifferscheidt, Christiane von} (12.\,6.\,1859 Doksy – 6.\,8.\,1935 Prag), \emph{Schriftstellerin}|pwv} erfreut mich (ich gebrauche das Wort
               in{ }ſeiner wirklichen Bedeutung){ }ſeit jeher durch ihre warme {\pb}und kluge Auffaſſung aller Ihrer
               Arbeiten. Und dieſe Frau\pwindex{Thun-Hohenstein-Salm-Reifferscheidt, Christiane von 12.\,6.\,1859 Doksy – 6.\,8.\,1935 Prag@\textsc{Thun-Hohenstein-Salm-Reifferscheidt, Christiane von} (12.\,6.\,1859 Doksy – 6.\,8.\,1935 Prag), \emph{Schriftstellerin}|pwv}\strikeout{,}{ }ſpricht mir, ganz ausnahmsweiſe, ihrer Art gar
               nicht entſprechend, lebhaft und mehrmals den Wunſch aus, Sie einmal zu{ }ſehen. Ich
               antworte: ganz gern, ganz leicht, einmal bei mir draußen. Es vergeht der Herbſt, der
               Winter, es {\pb}kommt das
               unfreundliche Frühjahr und da{ }ſie furchtbar an Neuralgien leidet,{ }ſagt{ }ſie:{ }ſo werde
               ich wieder nicht nach Rodaun\oindex{Wien@\textbf{Wien}!XXIII., Liesing@\textbf{XXIII., Liesing}!Rodaun@\textbf{Rodaun}, \emph{Region}|pw} kommen, und ich füge
               hinzu: und das mit dem Schnitzler wird nicht zuſammengehen. Im Augenblick fällt uns
               ein, daſs{ }ſie in ihrer Wohnung {\pb}ganz allein iſt, ihre Söhne\pwindex{Thun-Hohenstein-Salm-Reifferscheid, Josef Oswald 18.\,12.\,1878 Klášterec nad Ohří – 21.\,9.\,1942 ebd.@\textsc{Thun-Hohenstein-Salm-Reifferscheid, Josef Oswald} (18.\,12.\,1878 Klášterec nad Ohří – 21.\,9.\,1942 ebd.)|pwv}\pwindex{Thun-Hohenstein, Paul von 10.\,11.\,1884 Prag – 10.\,9.\,1963 Wien@\textsc{Thun-Hohenstein, Paul von} (10.\,11.\,1884 Prag – 10.\,9.\,1963 Wien), \emph{Schriftsteller, Ministerialbeamter}|pwv}\pwindex{Thun-Hohenstein-Salm-Reifferscheid, Adolf 31.\,8.\,1880 – 28.\,9.\,1957@\textsc{Thun-Hohenstein-Salm-Reifferscheid, Adolf} (31.\,8.\,1880 – 28.\,9.\,1957), \emph{Chirurg}|pwv} in Prag\oindex{Prag@\textbf{Prag}, \emph{Land}|pw}, ihr Mann\pwindex{Thun-Hohenstein-Salm-Reifferscheidt, Oswald 14.\,12.\,1849 Žehušice – 21.\,10.\,1913 Kärntner Straße 41@\textsc{Thun-Hohenstein-Salm-Reifferscheidt, Oswald} (14.\,12.\,1849 Žehušice – 21.\,10.\,1913 Kärntner Straße 41), \emph{Politiker, Industrieller, Großgrundbesitzer}|pwv} an der Riviera\oindex{Riviera@\textbf{Riviera}|pw}, und es kommt uns, mit der halb
               kindiſchen Freude, etwas ungewöhnliches zu arrangieren, der Gedanke an dieſes
               Frühſtück. Aus Beſcheidenheit fügt{ }ſie hinzu, man{ }ſollte, damit Sie{ }ſich nicht
               langweilen, noch {\pb}jemand
               Geſcheidten einladen der Ihnen neu und unterhaltend{ }ſein könnte, ich{ }ſchlage Kaſſner\pwindex{Kassner, Rudolf 11.\,9.\,1873 Velké Pavlovice – 1.\,4.\,1959 Sierre@\textsc{Kassner, Rudolf} (11.\,9.\,1873 Velké Pavlovice – 1.\,4.\,1959 Sierre), \emph{Schriftsteller}|pw} vor, den ich Ihnen{ }ſchon lange bekannt
               machen wollte, man wählt die Stunde des Frühſtücks, die Sie in nichts{ }ſtören kann,
               weil {\pb}ich weiß daſs Sie
               nachmittags gern arbeiten und Ruhe haben, es iſt eine Wohnung\oindex{Wien@\textbf{Wien}!I., Innere Stadt@\textbf{I., Innere Stadt}!Palais Thun-Salm@\textbf{Palais Thun-Salm}, \emph{Gebäude}|pwv} in der inneren Stadt,\pend
           
\pstart
           \numberlinefalse{}–\numberlinetrue{}\pend
           
\pstart
           ich überſchreite eine{ }ſeit 10 Jahren geübte Zurückhaltung und trage Ihnen dieſe Sache
               als herzlichen Wunſch oder Bitte von {\pb}mir vor, und Sie antworten, daſs
               Ihnen Mittagseinladungen in der nächſten Zeit unbequem{ }ſind!\pend
           
\pstart
           Ich kann wirklich nicht weiterſchreiben, weil ich zu erregt bin, und die Thränen in
               den Augen {\pb}habe, natürlich nicht
               vor Rührung{ }ſondern vor Zorn.\pend
           
\pstart
           Da Sie aus dieſer Heftigkeit vielleicht gerade bemerken, wie herzlich gern ich Sie
               habe,{ }ſo hoffe ich, daſs dieſer Brief Sie in keiner häſslichen Art ärgern wird.\pend
           
\pstart
           Von Herzen Ihr{\\[\baselineskip]}\spacefill\mbox{Hugo.}\pend
           \leftskip=0em{}\selectlanguage{ngerman}\endnumbering\briefempfaengerindex{Schnitzler, Arthur@\textsc{Schnitzler, Arthur}!zzzHofmannsthal, Hugo von@\emph{von Hugo von Hofmannsthal}!1902-03-281@{28. 3. 1902}|)be}\mylabel{L01210h}  \newcommand{\dateiname}{L01210}\newcommand{\titel}{Hugo von Hofmannsthal an Arthur Schnitzler, [28. 3. 1902]}\newcommand{\editorInnen}{Martin Anton Müller und Gerd-Hermann Susen}%% latex-leseansicht-abspann.tex
%% Abspann für die Leseansicht.
%% Der Schalter \ifkorrekturansicht ist bereits durch den Vorspann gesetzt.

%% latex-abspann.tex
%% Gemeinsamer Abspann für Korrekturansicht und Leseansicht.
%% Setzt den Schalter \ifkorrekturansicht voraus (gesetzt in den
%% einbindenden Dateien latex-korrekturansicht-abspann.tex bzw.
%% latex-leseansicht-abspann.tex).
%% ---------------------------------------------------------------

\normalsize

% Das esempio-Environment wird nur in der Leseansicht benötigt
\ifkorrekturansicht\else
\newenvironment{esempio}[3]%
{
    \vspace{1.5ex}
    \rlap{\underline{#1}}
    \par
    \setlength{\parindent}{0cm}
    \nopagebreak
    \leftskip=#2cm
    \rightskip=#3cm
}
{
    \par
}
\fi

\doendnotes{C}
\bigskip
\vfill

\clearpage

\footnotesize

\ifkorrekturansicht
  \lohead{\textsc{register}}
\fi

% theindex-Environment neu definieren ohne reledmac
\makeatletter
\renewenvironment{theindex}{%
  \ifkorrekturansicht
    \section*{\indexname}%
  \else
    \subsubsection*{Index der erwähnten Entitäten}%
  \fi
  \setlength{\parindent}{0pt}%
  \setlength{\parskip}{0pt plus 0.3pt}%
  \let\item\@idxitem
}{%
  \ifkorrekturansicht\clearpage\fi
}
\makeatother

\IfFileExists{\jobname-pw.ind}{\input{\jobname-pw.ind}}{}

% Quellenangabe nur in der Leseansicht
\ifkorrekturansicht\else
% Fallback-Definitionen, falls die .tex-Datei \titel etc. nicht gesetzt hat
\providecommand{\titel}{}
\providecommand{\editorInnen}{}
\providecommand{\dateiname}{\jobname}

\vspace{3cm}

\vfill

\footnotesize
\textsc{Quelle}: \titel. Herausgegeben von {\editorInnen}. In: \emph{Arthur Schnitzler: Briefwechsel mit Autorinnen und Autoren}.
 Digitale Edition, https://schnitzler-briefe.acdh.oeaw.ac.at/{\dateiname}.html (Stand \today)
\fi

\end{document}


