%% latex-leseansicht-vorspann.tex
%% Vorspann für die Leseansicht.
%% Lädt die gemeinsame Datei latex-vorspann.tex mit nicht gesetztem Schalter.

\newif\ifkorrekturansicht
\korrekturansichtfalse

\input{../tex-inputs/latex-vorspann}


\section[Hugo von Hofmannsthal an Arthur Schnitzler, 7. 8. 1898]{L00832 Hugo von Hofmannsthal an Arthur Schnitzler, 7. 8. 1898}
\nopagebreak\mylabel{L00832v}
\rehead{ }\normalsize\beginnumbering\briefempfaengerindex{Schnitzler, Arthur@\textsc{Schnitzler, Arthur}!zzzHofmannsthal, Hugo von@\emph{von Hugo von Hofmannsthal}!1898-08-071@{7. 8. 1898}|(be}
\toendnotes[C]{\smallbreak\pagebreak[2]}
\correspDesc{Versand  durch Hugo von Hofmannsthal am 7. 8. 1898 in Hinterbrühl
\newline{}Erhalt  durch Arthur Schnitzler am 7. 8. 1898 in Tegernsee}\toendnotes[C]{\smallbreak}
\Standort{CUL, Schnitzler, B 43.}
\physDesc{Telegramm, 215 Zeichen
\newline{}HandschriftX2 einer Schreibkraft: Bleistift, deutsche Kurrent
\newline{}Versand: »\noindent{}\begin{center}\textcolor{gray}{\textbf{\textbf{K. B. Telegraphenſtation}}}{ }\textcolor{gray}{\textbf{\textit{TEGERNSEE\oindex{Tegernsee@\textbf{Tegernsee}|pw}}}}\oindex{Telegrafenstation@\textbf{Telegrafenstation}, \emph{Bürogebäude}|pw}\end{center}{ / }\textcolor{gray}{\textbf{Nr.}}{ }310{ / }\textcolor{gray}{\textbf{Aufgegeben in}}{ }Hinterbrühl\oindex{Hinterbrühl@\textbf{Hinterbrühl}, \emph{Hauptstadt}|pw}{ / }\raggedleft{}\textcolor{gray}{\textbf{Abgefertigt}}{ }7/8{ }\textcolor{gray}{\textbf{189}}8{ }11\textcolor{gray}{\textbf{U}}10\textcolor{gray}{\textbf{M}}{ }V\textcolor{gray}{\textbf{Mttg.}}{ / }\textcolor{gray}{\textbf{Nr.}}{ }103 0 30\textcolor{gray}{\textbf{W.}}{ / }\textcolor{gray}{\textbf{den}}{ }7/8 \textcolor{gray}{\textbf{189}}8{ }9\textcolor{gray}{\textbf{U}}–\textcolor{gray}{\textbf{M}}{ }V\textcolor{gray}{\textbf{Mttg.}}« 
\newline{}Ordnung: mit Bleistift von unbekannter Hand nummeriert:
                                    »121« }
\buchAbdrucke{\weitereDrucke{Hugo von Hofmannsthal, Arthur Schnitzler: \emph{Briefwechsel}. Herausgegeben von Therese Nickl und Heinrich Schnitzler. Frankfurt am Main: \emph{S. Fischer} 1964, S. 110.} }\pstart{}{\pb}Schnitzler\pend{}\pstart{}Hotel Post\oindex{Hotel Post [Tegernsee]@\textbf{Hotel Post [Tegernsee]}, \emph{Hotel}|pw} über München\oindex{München@\textbf{München}|pw}\pend{}{\bigskip}\vspace{1em}
\pstart
           \noindent{}\raggedleft{}{\pb}\textcolor{gray}{\textbf{\textit{TEGERNSEE\oindex{Tegernsee@\textbf{Tegernsee}|pw}}}}\pend
           
\pstart
           Bin aus vielen Gründen{ }ſchon Mittwoch{ }Abend in Baſel\oindex{Basel@\textbf{Basel}|pw} bitte Drahtantwort
                  Hinterbrühl\oindex{Hinterbrühl@\textbf{Hinterbrühl}, \emph{Hauptstadt}|pw} ob{ }ſie{ }ſpäteſtens
                  Donnerſtag auch dort{ }ſein können und welches Gaſthaus\pend
           \pstart \spacefill\mbox{Hugo}\pend{}
\pstart
           \noindent{}{\pb}\textsc{Hotel National}\oindex{Hotel National [Luzern]@\textbf{Hotel National [Luzern]}, \emph{Hotel}|pw} b. d. Bahn\pend
           \selectlanguage{ngerman}\endnumbering\briefempfaengerindex{Schnitzler, Arthur@\textsc{Schnitzler, Arthur}!zzzHofmannsthal, Hugo von@\emph{von Hugo von Hofmannsthal}!1898-08-071@{7. 8. 1898}|)be}\mylabel{L00832h}  \newcommand{\dateiname}{L00832}\newcommand{\titel}{Hugo von Hofmannsthal an Arthur Schnitzler, 7. 8. 1898}\newcommand{\editorInnen}{Martin Anton Müller und Gerd-Hermann Susen}%% latex-leseansicht-abspann.tex
%% Abspann für die Leseansicht.
%% Der Schalter \ifkorrekturansicht ist bereits durch den Vorspann gesetzt.

%% latex-abspann.tex
%% Gemeinsamer Abspann für Korrekturansicht und Leseansicht.
%% Setzt den Schalter \ifkorrekturansicht voraus (gesetzt in den
%% einbindenden Dateien latex-korrekturansicht-abspann.tex bzw.
%% latex-leseansicht-abspann.tex).
%% ---------------------------------------------------------------

\normalsize

% Das esempio-Environment wird nur in der Leseansicht benötigt
\ifkorrekturansicht\else
\newenvironment{esempio}[3]%
{
    \vspace{1.5ex}
    \rlap{\underline{#1}}
    \par
    \setlength{\parindent}{0cm}
    \nopagebreak
    \leftskip=#2cm
    \rightskip=#3cm
}
{
    \par
}
\fi

\doendnotes{C}
\bigskip
\vfill

\clearpage

\footnotesize

\ifkorrekturansicht
  \lohead{\textsc{register}}
\fi

% theindex-Environment neu definieren ohne reledmac
\makeatletter
\renewenvironment{theindex}{%
  \ifkorrekturansicht
    \section*{\indexname}%
  \else
    \subsubsection*{Index der erwähnten Entitäten}%
  \fi
  \setlength{\parindent}{0pt}%
  \setlength{\parskip}{0pt plus 0.3pt}%
  \let\item\@idxitem
}{%
  \ifkorrekturansicht\clearpage\fi
}
\makeatother

\IfFileExists{\jobname-pw.ind}{\input{\jobname-pw.ind}}{}

% Quellenangabe nur in der Leseansicht
\ifkorrekturansicht\else
% Fallback-Definitionen, falls die .tex-Datei \titel etc. nicht gesetzt hat
\providecommand{\titel}{}
\providecommand{\editorInnen}{}
\providecommand{\dateiname}{\jobname}

\vspace{3cm}

\vfill

\footnotesize
\textsc{Quelle}: \titel. Herausgegeben von {\editorInnen}. In: \emph{Arthur Schnitzler: Briefwechsel mit Autorinnen und Autoren}.
 Digitale Edition, https://schnitzler-briefe.acdh.oeaw.ac.at/{\dateiname}.html (Stand \today)
\fi

\end{document}


