%% latex-leseansicht-vorspann.tex
%% Vorspann für die Leseansicht.
%% Lädt die gemeinsame Datei latex-vorspann.tex mit nicht gesetztem Schalter.

\newif\ifkorrekturansicht
\korrekturansichtfalse

\input{../tex-inputs/latex-vorspann}


\section[Arthur Schnitzler an Gustav Schwarzkopf, 10. 11. 1903]{L04146 Arthur Schnitzler an Gustav Schwarzkopf, 10. 11. 1903}
\nopagebreak\mylabel{L04146v}
\rehead{ }\normalsize\beginnumbering\briefempfaengerindex{Schwarzkopf, Gustav@\textsc{Schwarzkopf, Gustav}!zzzSchnitzler, Arthur@\emph{von Arthur Schnitzler}!1903-11-104@{10. 11. 1903}|(be}
\toendnotes[C]{\smallbreak\pagebreak[2]}
\correspDesc{Versand  durch Arthur Schnitzler am 10. 11. 1903 in Wien
\newline{}Erhalt  durch Gustav Schwarzkopf im Zeitraum [10. 11. 1903 – 13. 11. 1903?] in Wien}\toendnotes[C]{\smallbreak}
\Standort{CUL, Schnitzler, B 96.}
\physDesc{Brief, 1 Blatt, 1 Seite, 195 Zeichen
\newline{}Handschrift: Bleistift, deutsche Kurrent}\toendnotes[C]{\smallbreak}
\pstart
           \raggedleft{}{\pb}Wien\oindex{Wien@\textbf{Wien}, \emph{Verwaltungsgebiet}|pw}{ }10. 11. 903\pend
           \vspace{0.5em}
\pstart
           lieber Guſtav, ich ſende Ihnen hier den \label{K_L04146-1v}\edtext{Brief des Dr Wolff\pwindex{Wolff, Ludwig 7.\,3.\,1876 Bielsko-Biała – nach 1958 Vereinigte Staaten von Amerika [USA]@\textsc{Wolff, Ludwig} (7.\,3.\,1876 Bielsko-Biała – nach 1958 Vereinigte Staaten von Amerika [USA]), \emph{Schriftsteller, Dramaturg}|pwu}}{\lemma{\textnormal{\emph{Brief des Dr Wolff}}}\Cendnote{\textnormal{Beilage nicht erhalten.
                  Da Schnitzler sichtlich als Mittelsmann
                  fungiert, dürfte es sich um das Stück \emph{Der reine
                     Tor}\pwindex{Schwarzkopf, Max 12.\,6.\,1857 Wien – 14.\,4.\,1928 ebd.@\textsc{Schwarzkopf, Max} (12.\,6.\,1857 Wien – 14.\,4.\,1928 ebd.), \emph{Rechtsanwalt}!reine Tor. Gesellschaftsstück in vier Akten@\strich\emph{Der reine Tor. Gesellschaftsstück in vier Akten}|pwk} von Max Schwarzkopf\pwindex{Schwarzkopf, Max 12.\,6.\,1857 Wien – 14.\,4.\,1928 ebd.@\textsc{Schwarzkopf, Max} (12.\,6.\,1857 Wien – 14.\,4.\,1928 ebd.), \emph{Rechtsanwalt}|pwk} handeln,
                     vgl. XXXX Auszeichnungsfehler: Dokument L04145 nicht gefunden. Ein Theaterleiter namens Wolff
                  mit Doktortitel konnte nicht ausfindig gemacht werden. Es könnte sich um den beim
                     \emph{Theater in der Josefstadt}\orgindex{Theater in der Josefstadt@Theater in der Josefstadt|pwk} als Dramaturg
                  engagierten Ludwig Wolff\pwindex{Wolff, Ludwig 7.\,3.\,1876 Bielsko-Biała – nach 1958 Vereinigte Staaten von Amerika [USA]@\textsc{Wolff, Ludwig} (7.\,3.\,1876 Bielsko-Biała – nach 1958 Vereinigte Staaten von Amerika [USA]), \emph{Schriftsteller, Dramaturg}|pwk} gehandelt haben,
                  der aber keinen akademischen Titel trug. }}}\label{K_L04146-1}, den ich bei meiner \label{K_L04146-2v}\edtext{Rückkunft vom Se{\geminationm}ering\oindex{Semmering@\textbf{Semmering}, \emph{Verwaltungsgebiet}|pw}}{\lemma{\textnormal{\emph{Rückkunft vom Semmering}}}\Cendnote{\textnormal{Er war am Vorabend zurückgekehrt, vgl. A. S.: \emph{Wiener Schnitzler}, 9. 11. 1903.}}}\label{K_L04146-2} vorfinde.
               Das Stück\pwindex{Schwarzkopf, Max 12.\,6.\,1857 Wien – 14.\,4.\,1928 ebd.@\textsc{Schwarzkopf, Max} (12.\,6.\,1857 Wien – 14.\,4.\,1928 ebd.), \emph{Rechtsanwalt}!reine Tor. Gesellschaftsstück in vier Akten@\strich\emph{Der reine Tor. Gesellschaftsstück in vier Akten}|pwv} iſt auch da. Was
               nun?\pend
           
\pstart
           Auf baldges Wiederſehen.{\\[\baselineskip]} Herzlichſt Ihr{\\[\baselineskip]}\spacefill\mbox{ArthSch}\pend
           \leftskip=0em{}\selectlanguage{ngerman}\endnumbering\briefempfaengerindex{Schwarzkopf, Gustav@\textsc{Schwarzkopf, Gustav}!zzzSchnitzler, Arthur@\emph{von Arthur Schnitzler}!1903-11-104@{10. 11. 1903}|)be}\mylabel{L04146h}
\begin{anhang}
\end{anhang}\newcommand{\dateiname}{L04146}\newcommand{\titel}{Arthur Schnitzler an Gustav Schwarzkopf, 10. 11. 1903}\newcommand{\editorInnen}{Herausgegeben von Jahnke, SelmaMüller, Martin Anton}%% latex-leseansicht-abspann.tex
%% Abspann für die Leseansicht.
%% Der Schalter \ifkorrekturansicht ist bereits durch den Vorspann gesetzt.

%% latex-abspann.tex
%% Gemeinsamer Abspann für Korrekturansicht und Leseansicht.
%% Setzt den Schalter \ifkorrekturansicht voraus (gesetzt in den
%% einbindenden Dateien latex-korrekturansicht-abspann.tex bzw.
%% latex-leseansicht-abspann.tex).
%% ---------------------------------------------------------------

\normalsize

% Das esempio-Environment wird nur in der Leseansicht benötigt
\ifkorrekturansicht\else
\newenvironment{esempio}[3]%
{
    \vspace{1.5ex}
    \rlap{\underline{#1}}
    \par
    \setlength{\parindent}{0cm}
    \nopagebreak
    \leftskip=#2cm
    \rightskip=#3cm
}
{
    \par
}
\fi

\doendnotes{C}
\bigskip
\vfill

\clearpage

\footnotesize

\ifkorrekturansicht
  \lohead{\textsc{register}}
\fi

% theindex-Environment neu definieren ohne reledmac
\makeatletter
\renewenvironment{theindex}{%
  \ifkorrekturansicht
    \section*{\indexname}%
  \else
    \subsubsection*{Index der erwähnten Entitäten}%
  \fi
  \setlength{\parindent}{0pt}%
  \setlength{\parskip}{0pt plus 0.3pt}%
  \let\item\@idxitem
}{%
  \ifkorrekturansicht\clearpage\fi
}
\makeatother

\IfFileExists{\jobname-pw.ind}{\input{\jobname-pw.ind}}{}

% Quellenangabe nur in der Leseansicht
\ifkorrekturansicht\else
% Fallback-Definitionen, falls die .tex-Datei \titel etc. nicht gesetzt hat
\providecommand{\titel}{}
\providecommand{\editorInnen}{}
\providecommand{\dateiname}{\jobname}

\vspace{3cm}

\vfill

\footnotesize
\textsc{Quelle}: \titel. Herausgegeben von {\editorInnen}. In: \emph{Arthur Schnitzler: Briefwechsel mit Autorinnen und Autoren}.
 Digitale Edition, https://schnitzler-briefe.acdh.oeaw.ac.at/{\dateiname}.html (Stand \today)
\fi

\end{document}


