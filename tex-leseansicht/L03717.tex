%% latex-leseansicht-vorspann.tex
%% Vorspann für die Leseansicht.
%% Lädt die gemeinsame Datei latex-vorspann.tex mit nicht gesetztem Schalter.

\newif\ifkorrekturansicht
\korrekturansichtfalse

\input{../tex-inputs/latex-vorspann}


\section[Elsa Plessner an Arthur Schnitzler, 22. 10. 1898]{L03717 Elsa Plessner an Arthur Schnitzler, 22. 10. 1898}
\nopagebreak\mylabel{L03717v}
\rehead{ }\normalsize\beginnumbering\briefempfaengerindex{Schnitzler, Arthur@\textsc{Schnitzler, Arthur}!zzzPlessner, Elsa@\emph{von Elsa Plessner}!1898-10-221@{22. 10. 1898}|(be}
\toendnotes[C]{\smallbreak\pagebreak[2]}
\correspDesc{Versand  durch Elsa Plessner am 22. 10. 1898 in Wien
\newline{}Erhalt  durch Arthur Schnitzler im Zeitraum [23. 10. 1898 – 27. 10. 1898?] in Wien}\toendnotes[C]{\smallbreak}
\Standort{DLA, A:Schnitzler, HS.1985.1.419.}
\physDesc{Visitenkarte, 420 Zeichen
\newline{}Handschrift: schwarze Tinte, lateinische Kurrent}\toendnotes[C]{\smallbreak}
\pstart
           \centering{}{\pb}den 22./10. 98.\pend
           
\pstart{}Verehrter Herr Doctor!\pend\vspace{0.5em}
\pstart
           Bitte, seien Sie so lieb wie immer und theilen Sie mir \label{K_L03717-1v}\edtext{gfl.}{\lemma{\textnormal{\emph{gfl.}}}\Cendnote{\textnormal{gefällig}}}\label{K_L03717-1} mit, wie Ihre Ansicht über
               die \label{K_L03717-2v}\edtext{beifolgende Geschichte}{\lemma{\textnormal{\emph{beifolgende Geschichte}}}\Cendnote{\textnormal{Beilage nicht erhalten. Um welchen ihrer Texte es sich 
                  gehandelt hat, ist
                  nicht zu rekonstruieren.}}}\label{K_L03717-2} ausfällt. – – – Sie wissen ja, wieviel mir stets an
               Ihrem Urtheil {\pb}liegt! –\pend
           
\pstart
           In einer der \label{K_L03717-3v}\edtext{nächsten Nummern der »Wage\pwindex{Wage. Eine Wiener Wochenschrift@\emph{Die Wage. Eine Wiener Wochenschrift}|pw}« werden
               Sie eine größere Novelle\pwindex{Plessner, Elsa 22.\,8.\,1875 Wien – 7.\,5.\,1932 Alicante@\textsc{Plessner, Elsa} (22.\,8.\,1875 Wien – 7.\,5.\,1932 Alicante), \emph{Schriftstellerin}!neue Lehrer. Novelle@\strich\emph{Der neue Lehrer. Novelle}|pwv}}{\lemma{\textnormal{\emph{nächsten … Novelle}}}\Cendnote{\textnormal{Zu der hier angekündigten Publikation kam es nicht, 
                  Elsa Plessner\pwindex{Plessner, Elsa 22.\,8.\,1875 Wien – 7.\,5.\,1932 Alicante@\textsc{Plessner, Elsa} (22.\,8.\,1875 Wien – 7.\,5.\,1932 Alicante), \emph{Schriftstellerin}|pwk} zog den
                  Text zurück, wie aus dem Brief vom XXXX Auszeichnungsfehler: Dokument L03718 nicht gefunden hervorgeht. Es dürfte sich um die Novelle \emph{Der neue Lehrer}\pwindex{Plessner, Elsa 22.\,8.\,1875 Wien – 7.\,5.\,1932 Alicante@\textsc{Plessner, Elsa} (22.\,8.\,1875 Wien – 7.\,5.\,1932 Alicante), \emph{Schriftstellerin}!neue Lehrer. Novelle@\strich\emph{Der neue Lehrer. Novelle}|pwk} handeln, deren Titel Plessner\pwindex{Plessner, Elsa 22.\,8.\,1875 Wien – 7.\,5.\,1932 Alicante@\textsc{Plessner, Elsa} (22.\,8.\,1875 Wien – 7.\,5.\,1932 Alicante), \emph{Schriftstellerin}|pwk} im Brief vom XXXX Auszeichnungsfehler: Dokument L03720 nicht gefunden erstmals erwähnt. Diese stellt ihren längsten
                  überlieferten Prosatext aus der Zeit dar.}}}\label{K_L03717-3} von mir finden, deren \substVorne{}\textsuperscript{U}\substDazwischen{}Beu\substHinten{}rtheil\introOben{}ung\introOben{} von Ihrer Seite mich schon jetzt
               außerordentlich interessirt. – Besten, herzlichen Dank im Voraus\pend
           
\pstart
           \centering{}\textcolor{gray}{\textbf{ELSA PLESSNER}}\pend
           \selectlanguage{ngerman}\endnumbering\briefempfaengerindex{Schnitzler, Arthur@\textsc{Schnitzler, Arthur}!zzzPlessner, Elsa@\emph{von Elsa Plessner}!1898-10-221@{22. 10. 1898}|)be}\mylabel{L03717h}  \newcommand{\dateiname}{L03717}\newcommand{\titel}{Elsa Plessner an Arthur Schnitzler, 22. 10. 1898}\newcommand{\editorInnen}{Selma Jahnke und Martin Anton Müller}%% latex-leseansicht-abspann.tex
%% Abspann für die Leseansicht.
%% Der Schalter \ifkorrekturansicht ist bereits durch den Vorspann gesetzt.

%% latex-abspann.tex
%% Gemeinsamer Abspann für Korrekturansicht und Leseansicht.
%% Setzt den Schalter \ifkorrekturansicht voraus (gesetzt in den
%% einbindenden Dateien latex-korrekturansicht-abspann.tex bzw.
%% latex-leseansicht-abspann.tex).
%% ---------------------------------------------------------------

\normalsize

% Das esempio-Environment wird nur in der Leseansicht benötigt
\ifkorrekturansicht\else
\newenvironment{esempio}[3]%
{
    \vspace{1.5ex}
    \rlap{\underline{#1}}
    \par
    \setlength{\parindent}{0cm}
    \nopagebreak
    \leftskip=#2cm
    \rightskip=#3cm
}
{
    \par
}
\fi

\doendnotes{C}
\bigskip
\vfill

\clearpage

\footnotesize

\ifkorrekturansicht
  \lohead{\textsc{register}}
\fi

% theindex-Environment neu definieren ohne reledmac
\makeatletter
\renewenvironment{theindex}{%
  \ifkorrekturansicht
    \section*{\indexname}%
  \else
    \subsubsection*{Index der erwähnten Entitäten}%
  \fi
  \setlength{\parindent}{0pt}%
  \setlength{\parskip}{0pt plus 0.3pt}%
  \let\item\@idxitem
}{%
  \ifkorrekturansicht\clearpage\fi
}
\makeatother

\IfFileExists{\jobname-pw.ind}{\input{\jobname-pw.ind}}{}

% Quellenangabe nur in der Leseansicht
\ifkorrekturansicht\else
% Fallback-Definitionen, falls die .tex-Datei \titel etc. nicht gesetzt hat
\providecommand{\titel}{}
\providecommand{\editorInnen}{}
\providecommand{\dateiname}{\jobname}

\vspace{3cm}

\vfill

\footnotesize
\textsc{Quelle}: \titel. Herausgegeben von {\editorInnen}. In: \emph{Arthur Schnitzler: Briefwechsel mit Autorinnen und Autoren}.
 Digitale Edition, https://schnitzler-briefe.acdh.oeaw.ac.at/{\dateiname}.html (Stand \today)
\fi

\end{document}


