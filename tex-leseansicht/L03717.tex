%% latex-korrekturansicht-vorspann.tex
%% Vorspann für die Korrekturansicht.
%% Lädt die gemeinsame Datei latex-vorspann.tex mit gesetztem Schalter.

\newif\ifkorrekturansicht
\korrekturansichttrue

\input{../tex-inputs/latex-vorspann}


\section[Elsa Plessner an Arthur Schnitzler, 22. 10. 1898]{L03717 Elsa Plessner an Arthur Schnitzler, 22. 10. 1898}
\nopagebreak\mylabel{L03717v}
\rehead{ }\normalsize\beginnumbering\briefempfaengerindex{Schnitzler, Arthur@\textsc{Schnitzler, Arthur}!zzzPlessner, Elsa@\emph{von Elsa Plessner}!1898-10-221@{22. 10 1898}|(be}
\toendnotes[C]{\smallbreak\pagebreak[2]}\Standort{DLA, A:Schnitzler, HS.1985.1.419.}
\physDesc{Kartenbrief, 1 Blatt, 2 Seiten, 419 Zeichen
\newline{}Handschrift: , lateinische Kurrent
\newline{}Zusatz: Kartenbrief mit Vordruck auf S. 2: »ELSA
                                 PLESSNER« }\toendnotes[C]{\smallbreak}
\pstart
           \raggedleft{}{\pb}den 22. /10. 98.\pend
           
\pstart{}Verehrter Herr Doctor!\pend\vspace{0.5em}
\pstart
           Bitte seien Sie so lieb wie immer und theilen Sie mir gff. mit, wie Ihre Ansicht über
               die \label{K_L03717-1v}\edtext{beifolgende Geschichte}{\lemma{\textnormal{\emph{beifolgende Geschichte}}}\Cendnote{\textnormal{Welcher Text dem Schreiben beilag, ist
                  nicht zu rekonstruieren.}}}\label{K_L03717-1} ausfällt.... Sie wissen ja, wieviel mir stets an
               Ihrem Urtheil liegt!. \pend
           
\pstart
           In einer der nächsten Nummern der »Wage\pwindex{Wage. Eine Wiener Wochenschrift@\emph{Die Wage. Eine Wiener Wochenschrift}|pw}« werden
               Sie \label{K_L03717-2v}\edtext{eine größere Novelle\pwindex{neue Lehrer. Novelle@\emph{Der neue Lehrer. Novelle}|pwuv}}{\lemma{\textnormal{\emph{eine größere Novelle}}}\Cendnote{\textnormal{Elsa Plessner\pwindex{Plessner, Elsa 22.08.1875 – 01.05.1932@\textsc{Plessner, Elsa} (22.08.1875 – 01.05.1932), \emph{Schriftsteller/Schriftstellerin}|pwk} zog den
                  Text zurück, wie aus dem Brief vom 2. 1. 1899 hervorgeht. Vermutlich handelte es sich um die Novelle \emph{Der neue Lehrer}\pwindex{neue Lehrer. Novelle@\emph{Der neue Lehrer. Novelle}|pwk}, deren Titel Plessner\pwindex{Plessner, Elsa 22.08.1875 – 01.05.1932@\textsc{Plessner, Elsa} (22.08.1875 – 01.05.1932), \emph{Schriftsteller/Schriftstellerin}|pwk} im Brief vom 19. 1. 1899 erstmals erwähnt und die ihren längsten
                  überlieferten Prosatext darstellt.}}}\label{K_L03717-2} von mir finden, deren \substVorne{}\textsuperscript{U}\substDazwischen{}Beu\substHinten{}rtheil\introOben{}ung\introOben{} von Ihrer Seite mich schon jetzt
               außerordentlich interessirt. – Besten herzlichen Dank im Voraus! \pend
           \selectlanguage{ngerman}\endnumbering\briefempfaengerindex{Schnitzler, Arthur@\textsc{Schnitzler, Arthur}!zzzPlessner, Elsa@\emph{von Elsa Plessner}!1898-10-221@{22. 10 1898}|)be}\mylabel{L03717h}
\begin{anhang}
\end{anhang}\normalsize

\doendnotes{C}
\bigskip
\vfill

\clearpage

\footnotesize

\lohead{\textsc{register}}

% Definiere theindex-Environment komplett neu ohne reledmac
\makeatletter
\renewenvironment{theindex}{%
  \section*{\indexname}%
  \setlength{\parindent}{0pt}%
  \setlength{\parskip}{0pt plus 0.3pt}%
  \let\item\@idxitem
}{%
  \clearpage
}
\makeatother

\IfFileExists{\jobname-pw.ind}{\input{\jobname-pw.ind}}{}

\end{document}

      