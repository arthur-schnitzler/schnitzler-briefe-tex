%% latex-leseansicht-vorspann.tex
%% Vorspann für die Leseansicht.
%% Lädt die gemeinsame Datei latex-vorspann.tex mit nicht gesetztem Schalter.

\newif\ifkorrekturansicht
\korrekturansichtfalse

\input{../tex-inputs/latex-vorspann}


\section[Friedrich M. Fels an Arthur Schnitzler, 7. 10. 1894]{L00379 Friedrich M. Fels an Arthur Schnitzler, 7. 10. 1894}
\nopagebreak\mylabel{L00379v}
\rehead{ }\normalsize\beginnumbering\briefempfaengerindex{Schnitzler, Arthur@\textsc{Schnitzler, Arthur}!zzzFels, Friedrich Michael@\emph{von Friedrich Michael Fels}!1894-10-072@{7. 10. 1894}|(be}
\toendnotes[C]{\smallbreak\pagebreak[2]}
\correspDesc{Versand  durch Friedrich M. Fels am 7. 10. 1894 in Wien
\newline{}Erhalt  durch Arthur Schnitzler am 8. 10. 1894 in Wien}\toendnotes[C]{\smallbreak}
\Standort{DLA, A:Schnitzler, HS.NZ85.1.2956.}
\physDesc{Kartenbrief, 675 Zeichen
\newline{}Handschrift: schwarze Tinte, lateinische Kurrent
\newline{}Versand: 1) Stempel: »\nobreak{}\oindex{XVIII., Währing@\textbf{XVIII., Währing}, \emph{Verwaltungsgebiet}|pwk}Wien 18{[}/1{]}, 7 10 {[}1894{]}\nobreak{}«.   2) Stempel: »\nobreak{}\oindex{IX., Alsergrund@\textbf{IX., Alsergrund}, \emph{Verwaltungsgebiet}|pwk}Wien 9/3, 8. 10. 1894, 8.V, Bestellt\nobreak{}«. 
\newline{}Schnitzler: mit Bleistift falsch datiert: »1/10 94« und nummeriert: »15« }\toendnotes[C]{\smallbreak}\pstart{}{\pb}Herrn Dr. Arthur Schnitzler\pend{}\pstart{}Schriftsteller\pend{}\pstart{}Wien\oindex{Wien@\textbf{Wien}, \emph{Verwaltungsgebiet}|pw}\pend{}\pstart{}IX, Frankgaße 1\oindex{Wien@\textbf{Wien}!IX., Alsergrund@\textbf{IX., Alsergrund}!Frankgasse 1@\textbf{Frankgasse 1}, \emph{Wohngebäude}|pw}\pend{}{\bigskip}\vspace{1em}
\pstart
           \raggedleft{}{\pb}Wien XVIII, Gürtelstraße 90\oindex{Wien@\textbf{Wien}!IX., Alsergrund@\textbf{IX., Alsergrund}!Währinger Gürtel@\textbf{Währinger Gürtel}, \emph{Straße}|pw}\oindex{Wien@\textbf{Wien}!XVIII., Währing@\textbf{XVIII., Währing}!Währinger Gürtel@\textbf{Währinger Gürtel}, \emph{Straße}|pw} parterre Th. 9\pend
           
\pstart{}Lieber Dr. Schnitzler!\pend\vspace{0.5em}
\pstart
           Entschuldigen Sie, we{\geminationn} ich Sie schon wieder mit einer
               Bitte belästige. Bei der »\uline{Wiener Allgemeinen Zeitung\orgindex{Wiener Allgemeine Zeitung@Wiener Allgemeine Zeitung|pw}}« soll eine große Veränderung bevorstehen, wobei \damage{viel}leicht auch für mich etwas abfallen kö{\geminationn}te. Doch
               hat mein Gewährsma{\geminationn} versprechen müßen, niemanden etwas
               von der Sache zu verraten; er behauptet aber, \uline{Sie}
               wüßten ganz besti{\geminationm}t davon, da der neue Herausgeber\pwindex{Gans-Ludassy, Julius von 13.\,4.\,1858 Wien – 30.\,9.\,1922 ebd.@\textsc{Gans-Ludassy, Julius von} (13.\,4.\,1858 Wien – 30.\,9.\,1922 ebd.), \emph{Schriftsteller, Journalist, Herausgeber}|pwv} ein \label{K_L00379-1v}\edtext{Freund}{\lemma{\textnormal{\emph{Freund}}}\Cendnote{\textnormal{Julius von Gans-Ludassy\pwindex{Gans-Ludassy, Julius von 13.\,4.\,1858 Wien – 30.\,9.\,1922 ebd.@\textsc{Gans-Ludassy, Julius von} (13.\,4.\,1858 Wien – 30.\,9.\,1922 ebd.), \emph{Schriftsteller, Journalist, Herausgeber}|pwk} war mit einer
                  Kusine von Schnitzler verheiratet.}}}\label{K_L00379-1}
               von Ihnen sei etc. We{\geminationn} dies der Fall ist, sind Sie wohl
               so freundlich, mir anzugeben, an wen ich mich zu wenden habe, und ein gutes Wort für
               mich einzulegen.\pend
           
\pstart
           Mit bestem Dank und Gruß{\\[\baselineskip]}Ihr{\\[\baselineskip]}\spacefill\mbox{Fels}\pend
           \leftskip=0em{}\selectlanguage{ngerman}\endnumbering\briefempfaengerindex{Schnitzler, Arthur@\textsc{Schnitzler, Arthur}!zzzFels, Friedrich Michael@\emph{von Friedrich Michael Fels}!1894-10-072@{7. 10. 1894}|)be}\mylabel{L00379h}  \newcommand{\dateiname}{L00379}\newcommand{\titel}{Friedrich M. Fels an Arthur Schnitzler, 7. 10. 1894}\newcommand{\editorInnen}{Martin Anton Müller und Gerd-Hermann Susen}%% latex-leseansicht-abspann.tex
%% Abspann für die Leseansicht.
%% Der Schalter \ifkorrekturansicht ist bereits durch den Vorspann gesetzt.

%% latex-abspann.tex
%% Gemeinsamer Abspann für Korrekturansicht und Leseansicht.
%% Setzt den Schalter \ifkorrekturansicht voraus (gesetzt in den
%% einbindenden Dateien latex-korrekturansicht-abspann.tex bzw.
%% latex-leseansicht-abspann.tex).
%% ---------------------------------------------------------------

\normalsize

% Das esempio-Environment wird nur in der Leseansicht benötigt
\ifkorrekturansicht\else
\newenvironment{esempio}[3]%
{
    \vspace{1.5ex}
    \rlap{\underline{#1}}
    \par
    \setlength{\parindent}{0cm}
    \nopagebreak
    \leftskip=#2cm
    \rightskip=#3cm
}
{
    \par
}
\fi

\doendnotes{C}
\bigskip
\vfill

\clearpage

\footnotesize

\ifkorrekturansicht
  \lohead{\textsc{register}}
\fi

% theindex-Environment neu definieren ohne reledmac
\makeatletter
\renewenvironment{theindex}{%
  \ifkorrekturansicht
    \section*{\indexname}%
  \else
    \subsubsection*{Index der erwähnten Entitäten}%
  \fi
  \setlength{\parindent}{0pt}%
  \setlength{\parskip}{0pt plus 0.3pt}%
  \let\item\@idxitem
}{%
  \ifkorrekturansicht\clearpage\fi
}
\makeatother

\IfFileExists{\jobname-pw.ind}{\input{\jobname-pw.ind}}{}

% Quellenangabe nur in der Leseansicht
\ifkorrekturansicht\else
% Fallback-Definitionen, falls die .tex-Datei \titel etc. nicht gesetzt hat
\providecommand{\titel}{}
\providecommand{\editorInnen}{}
\providecommand{\dateiname}{\jobname}

\vspace{3cm}

\vfill

\footnotesize
\textsc{Quelle}: \titel. Herausgegeben von {\editorInnen}. In: \emph{Arthur Schnitzler: Briefwechsel mit Autorinnen und Autoren}.
 Digitale Edition, https://schnitzler-briefe.acdh.oeaw.ac.at/{\dateiname}.html (Stand \today)
\fi

\end{document}


