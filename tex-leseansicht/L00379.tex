%% latex-korrekturansicht-vorspann.tex
%% Vorspann für die Korrekturansicht.
%% Lädt die gemeinsame Datei latex-vorspann.tex mit gesetztem Schalter.

\newif\ifkorrekturansicht
\korrekturansichttrue

\input{../tex-inputs/latex-vorspann}


\section[Friedrich M. Fels an Arthur Schnitzler, 7. 10. 1894]{L00379 Friedrich M. Fels an Arthur Schnitzler, 7. 10. 1894}
\nopagebreak\mylabel{L00379v}
\rehead{ }\normalsize\beginnumbering\briefempfaengerindex{Schnitzler, Arthur@\textsc{Schnitzler, Arthur}!zzzFels, Friedrich Michael@\emph{von Friedrich Michael Fels}!1894-10-072@{7. 10. 1894}|(be}
\toendnotes[C]{\smallbreak\pagebreak[2]}\Standort{DLA, A:Schnitzler, HS.NZ85.1.2956.}
\physDesc{Kartenbrief, 675 Zeichen
\newline{}Handschrift: schwarze Tinte, lateinische Kurrent
\newline{}Versand: 1) Stempel: »\nobreak{}\oindex{XVIII., Waehring@\textbf{XVIII., Währing}, \emph{A.ADM3}|pwk}Wien 18{[}/1{]}, 7 10 {[}1894{]}\nobreak{}«.   2) Stempel: »\nobreak{}\oindex{IX., Alsergrund@\textbf{IX., Alsergrund}, \emph{A.ADM3}|pwk}Wien 9/3, 8. 10. 1894, 8.V, Bestellt\nobreak{}«. 
\newline{}Schnitzler: mit Bleistift falsch datiert: »1/10 94« und nummeriert: »15« }\toendnotes[C]{\smallbreak}\pstart{}{\pb}Herrn Dr. Arthur Schnitzler\pend{}\pstart{}Schriftsteller\pend{}\pstart{}Wien\oindex{Wien@\textbf{Wien}, \emph{A.ADM2}|pw}\pend{}\pstart{}IX, Frankgaße 1\oindex{Frankgasse 1@\textbf{Frankgasse 1}, \emph{Wohngebäude (K.WHS)}|pw}\pend{}{\bigskip}\vspace{1em}
\pstart
           \raggedleft{}{\pb}Wien XVIII, Gürtelstraße 90\oindex{Waehringer Guertel@\textbf{Währinger Gürtel}, \emph{Straße (K.STR)}|pw} parterre Th. 9 \pend
           
\pstart{}Lieber Dr. Schnitzler!\pend\vspace{0.5em}
\pstart
           Entschuldigen Sie, we{\geminationn} ich Sie schon wieder mit einer
               Bitte belästige. Bei der »\uline{Wiener Allgemeinen Zeitung\orgindex{Wiener Allgemeine Zeitung@Wiener Allgemeine Zeitung|pw}}« soll eine große Veränderung bevorstehen, wobei \damage{viel}leicht auch für mich etwas abfallen kö{\geminationn}te. Doch
               hat mein Gewährsma{\geminationn} versprechen müßen, niemanden etwas
               von der Sache zu verraten; er behauptet aber, \uline{Sie}
               wüßten ganz besti{\geminationm}t davon, da der neue Herausgeber\pwindex{Gans-Ludassy, Julius von 13.04.1858 – 30.09.1922@\textsc{Gans-Ludassy, Julius von} (13.04.1858 – 30.09.1922), \emph{Schriftsteller/Schriftstellerin, Journalist/Journalistin, Herausgeber/Herausgeberin}|pwv} ein \label{K_L00379-1v}\edtext{Freund}{\lemma{\textnormal{\emph{Freund}}}\Cendnote{\textnormal{Julius von Gans-Ludassy\pwindex{Gans-Ludassy, Julius von 13.04.1858 – 30.09.1922@\textsc{Gans-Ludassy, Julius von} (13.04.1858 – 30.09.1922), \emph{Schriftsteller/Schriftstellerin, Journalist/Journalistin, Herausgeber/Herausgeberin}|pwk} war mit einer
                  Kusine von Schnitzler verheiratet.}}}\label{K_L00379-1}
               von Ihnen sei etc. We{\geminationn} dies der Fall ist, sind Sie wohl
               so freundlich, mir anzugeben, an wen ich mich zu wenden habe, und ein gutes Wort für
               mich einzulegen.\pend
           
\pstart
           Mit bestem Dank und Gruß{\\[\baselineskip]}Ihr{\\[\baselineskip]}\spacefill\mbox{Fels}\pend
           \leftskip=0em{}\selectlanguage{ngerman}\endnumbering\briefempfaengerindex{Schnitzler, Arthur@\textsc{Schnitzler, Arthur}!zzzFels, Friedrich Michael@\emph{von Friedrich Michael Fels}!1894-10-072@{7. 10. 1894}|)be}\mylabel{L00379h}  \normalsize

\doendnotes{C}
\bigskip
\vfill

\clearpage

\footnotesize

\lohead{\textsc{register}}

% Definiere theindex-Environment komplett neu ohne reledmac
\makeatletter
\renewenvironment{theindex}{%
  \section*{\indexname}%
  \setlength{\parindent}{0pt}%
  \setlength{\parskip}{0pt plus 0.3pt}%
  \let\item\@idxitem
}{%
  \clearpage
}
\makeatother

\IfFileExists{\jobname-pw.ind}{\input{\jobname-pw.ind}}{}

\end{document}

      