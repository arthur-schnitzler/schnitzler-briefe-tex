%% latex-korrekturansicht-vorspann.tex
%% Vorspann für die Korrekturansicht.
%% Lädt die gemeinsame Datei latex-vorspann.tex mit gesetztem Schalter.

\newif\ifkorrekturansicht
\korrekturansichttrue

\input{../tex-inputs/latex-vorspann}


\section[Richard Beer-Hofmann an Arthur Schnitzler, 17. 9. 1895]{L00486 Richard Beer-Hofmann an Arthur Schnitzler, 17. 9. 1895}
\nopagebreak\mylabel{L00486v}
\rehead{ }\normalsize\beginnumbering\briefempfaengerindex{Schnitzler, Arthur@\textsc{Schnitzler, Arthur}!zzzBeer-Hofmann, Richard@\emph{von Richard Beer-Hofmann}!1895-09-171@{17. 9. 1895}|(be}
\toendnotes[C]{\smallbreak\pagebreak[2]}\Standort{CUL, Schnitzler, B 8.}
\physDesc{Brief, 1 Blatt, 4 Seiten, 1482 Zeichen
\newline{}Handschrift: Bleistift, lateinische Kurrent
\newline{}Schnitzler: mit Bleistift nummeriert: »66« }
\buchAbdrucke{\weitereDrucke{Arthur Schnitzler, Richard Beer-Hofmann: \emph{Briefwechsel 1891–1931}. Wien, Zürich: \emph{Europaverlag} 1992, S. 82.} }\toendnotes[C]{\smallbreak}
\pstart
           \raggedleft{}{\pb}Schönberg\oindex{Schoenberg im Stubaital@\textbf{Schönberg im Stubaital}, \emph{P.PPLA3}|pw}{ }17/IX 95{ }Abends\pend
           \vspace{0.5em}
\pstart
           Lieber Arthur!{ }\uline{Soeben} erhalte ich Ihren Brief. Ich bin wirklich in
               guter Sti{\geminationm}ung; hoffentlich merken Sie es an Manchem wenn
               ich nach Wien\oindex{Wien@\textbf{Wien}, \emph{A.ADM2}|pw} zurückko{\geminationm}e{[}.{]} Daß ich seit Sonntag{ }Früh allein bin wissen Sie wol. Wie das Alleinreisen von L.\pwindex{Andreas-Salome, Lou 12.02.1861 – 05.02.1937@\textsc{Andreas-Salomé, Lou} (12.02.1861 – 05.02.1937), \emph{Schriftsteller/Schriftstellerin}|pw} aufgeno{\geminationm}en
               wurde? Zu schwierig in Worte zu kleiden. Nur vorläufig: Sie geht nicht nach Kopenhagen\oindex{Kopenhagen@\textbf{Kopenhagen}, \emph{P.PPLC}|pw} – sagt sie. Aber das ist nicht
               offiziell. \strikeout{Hier will ich bis Freitag}{ }Samstag{ }\introOben{}Früh\introOben{} will ich von {\pb}hier fort nach
                  Riva\oindex{Riva del Garda@\textbf{Riva del Garda}, \emph{P.PPLA3}|pw}, – einen Tag dort bleiben und dann nach
                  Salò\oindex{Salo@\textbf{Salò}, \emph{P.PPLA3}|pw}, Südwestende des Gardasees\oindex{Lago di Garda@\textbf{Lago di Garda}, \emph{See (N.SEE)}|pw}. Vielleicht gefällt es mir aber dort nicht, dann
               vielleicht Verona\oindex{Verona@\textbf{Verona}, \emph{P.PPLA2}|pw}, das ich nicht kenne.
               Jedenfalls erwarte ich noch einen Brief hieher, einen nach \uline{Riva}\oindex{Riva del Garda@\textbf{Riva del Garda}, \emph{P.PPLA3}|pw}{ }\uline{Poste restante}.\pend
           
\pstart
           Paul Horn\pwindex{Horn, Paul 13.02.1867 – 18.01.1936@\textsc{Horn, Paul} (13.02.1867 – 18.01.1936), \emph{Fabrikant/Fabrikantin}|pw} ist mir in der Erinnerung widerlich,
               Mann mit »lustigen Streichen« in der Jugend, kein Mensch.\pend
           
\pstart
           {\pb}Wozu Brosamen wie »Alles erkundigt
               sich«? Wer verübelt uns übrigens daß wir nicht fort Litteratur reden?\pend
           
\pstart
           Wie kommt Speidel\pwindex{Speidel, Ludwig 1830-04-11 – 1906-02-03@\textsc{Speidel, Ludwig} (1830-04-11 – 1906-02-03), \emph{Journalist/Journalistin, Kritiker/Kritikerin}|pw} zu Ebermann\pwindex{Ebermann, Leo 16.07.1863 – 09.10.1914@\textsc{Ebermann, Leo} (16.07.1863 – 09.10.1914), \emph{Schriftsteller/Schriftstellerin, Journalist/Journalistin, Rechtswissenschaftler/Rechtswissenschaftlerin}|pw}? Momentan bin ich \uline{der},
               der einzige Gast im Wirtshaus\oindex{Gasthaus Jagerhof@\textbf{Gasthaus Jagerhof}, \emph{Gastgewerbegebäude (K.GGW)}|pwv}. Ich »lebe u genieße«. Nochmals: Wann \uline{frühestens} kann »Liebelei\pwindex{Liebelei. Schauspiel in drei Akten@\emph{Liebelei. Schauspiel in drei Akten}|pw}« ko{\geminationm}en, denn vielleicht verzögert sich ja meine Ankunft, in
               den October hinein.\pend
           
\pstart
           {\pb}Adieu, ich will noch vor der
               Dunkelheit ein wenig spazieren. Die Zirbelkiefer die an der Strasse steht, ko{\geminationm}t in Goethes\pwindex{Goethe, Johann Wolfgang von 1749-08-28 – 1832-03-22@\textsc{Goethe, Johann Wolfgang von} (1749-08-28 – 1832-03-22), \emph{Schriftsteller/Schriftstellerin}|pw}{ }italienischer Reise\pwindex{Italienische Reise@\emph{Italienische Reise}|pw} vor. (Reise über den Brenner\oindex{Brenner@\textbf{Brenner}, \emph{T.PASS}|pw}) »Bei Schemberg\oindex{Schoenberg im Stubaital@\textbf{Schönberg im Stubaital}, \emph{P.PPLA3}|pw}« etc. das weiß ich aus dem \label{K_L00486-1v}\edtext{Meyer\pwindex{Deutsche Alpen@\emph{Deutsche Alpen}|pwv}}{\lemma{\textnormal{\emph{Meyer}}}\Cendnote{\textnormal{»Dagegen gelangt man {[}\ldots{]} auf dem \emph{alten}, r. abgehenden (schlechten) Fahrweg, {[}\ldots{]} den sogen. \emph{Alten
                        Schönberg} (dessen Zirben schon Goethe\pwindex{Goethe, Johann Wolfgang von 1749-08-28 – 1832-03-22@\textsc{Goethe, Johann Wolfgang von} (1749-08-28 – 1832-03-22), \emph{Schriftsteller/Schriftstellerin}|pw} in seiner ›Italienischer
                        Reise\pwindex{Italienische Reise@\emph{Italienische Reise}|pw}‹ erwähnt; bei einer ›Goethe\pwindex{Goethe, Johann Wolfgang von 1749-08-28 – 1832-03-22@\textsc{Goethe, Johann Wolfgang von} (1749-08-28 – 1832-03-22), \emph{Schriftsteller/Schriftstellerin}|pw}bank‹ schöne Aussicht) hinan« (\emph{Meyers Reisebücher}\pwindex{Meyers Reisebuecher@\emph{Meyers Reisebücher}|pwk}. \emph{Deutsche Alpen. Erster Teil: Bayerisches Hochland, Allgäu,
                        Vorarlberg, Tirol, Brennerbahn, Ötztaler-, Stubaier-, und Ortlergruppe,
                        Bozen, Schlern und Rosengarten, Meran, Brenta- und Adamellogruppe;
                        Bergamasker Alpen, Gardasee.}\pwindex{Deutsche Alpen@\emph{Deutsche Alpen}|pwk} Fünfte Auflage. Mit 23 Karten, 4 Plänen
                     und 12 Panoramen. Leipzig, Wien: \emph{Bibliographisches Institut}\orgindex{Bibliographisches Institut@Bibliographisches Institut|pwk}{ }1896, S. 217).}}}\label{K_L00486-1}. Werden uns je Bäume irgendwo wachsen
               – bei Meyer\pwindex{Meyers Reisebuecher@\emph{Meyers Reisebücher}|pw}?\pend
           
\pstart
           »Laßt uns lächeln.«\pend
           
\pstart
           Herzlichst Ihr{\\[\baselineskip]}\spacefill\mbox{Richard}\pend
           \leftskip=0em{}
\pstart
           \noindent{}Ich freu mich so sehr mit Ihren Briefen\pend
           
\pstart
           »\label{K_L00486-2v}\edtext{schreiben Sie augenscharf}{\lemma{\textnormal{\emph{schreiben Sie augenscharf}}}\Cendnote{\textnormal{offenbar ein stehender Ausdruck der
                     Gruppe, der sich auch im Briefwechsel zwischen Salten\pwindex{Salten, Felix 06.09.1869 – 08.10.1945@\textsc{Salten, Felix} (06.09.1869 – 08.10.1945), \emph{Schriftsteller/Schriftstellerin, Journalist/Journalistin, Chefredakteur/Chefredakteurin}|pwk} und Hofmannsthal\pwindex{Hofmannsthal, Hugo von 1874-02-01 – 1929-07-15@\textsc{Hofmannsthal, Hugo von} (1874-02-01 – 1929-07-15), \emph{Schriftsteller/Schriftstellerin}|pwk}
                     nachweisen lässt.}}}\label{K_L00486-2}«\pend
           \selectlanguage{ngerman}\endnumbering\briefempfaengerindex{Schnitzler, Arthur@\textsc{Schnitzler, Arthur}!zzzBeer-Hofmann, Richard@\emph{von Richard Beer-Hofmann}!1895-09-171@{17. 9. 1895}|)be}\mylabel{L00486h}  \normalsize

\doendnotes{C}
\bigskip
\vfill

\clearpage

\footnotesize

\lohead{\textsc{register}}

% Definiere theindex-Environment komplett neu ohne reledmac
\makeatletter
\renewenvironment{theindex}{%
  \section*{\indexname}%
  \setlength{\parindent}{0pt}%
  \setlength{\parskip}{0pt plus 0.3pt}%
  \let\item\@idxitem
}{%
  \clearpage
}
\makeatother

\IfFileExists{\jobname-pw.ind}{\input{\jobname-pw.ind}}{}

\end{document}

      