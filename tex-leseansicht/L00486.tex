%% latex-leseansicht-vorspann.tex
%% Vorspann für die Leseansicht.
%% Lädt die gemeinsame Datei latex-vorspann.tex mit nicht gesetztem Schalter.

\newif\ifkorrekturansicht
\korrekturansichtfalse

\input{../tex-inputs/latex-vorspann}


         
         \renewcommand{\erwaehntePersonen}{Personen: Lou Andreas-Salomé, Richard Beer-Hofmann, Leo Ebermann, Johann Wolfgang von Goethe, Hugo von Hofmannsthal, Paul Horn, Felix Salten, Ludwig Speidel}
         \renewcommand{\erwaehnteInstitutionen}{Institutionen: Bibliographisches Institut}
         \renewcommand{\erwaehnteOrte}{Orte: Brenner, Gasthaus Jagerhof, Kopenhagen, Lago di Garda, Riva del Garda, Salò, Schönberg im Stubaital, Verona, Wien}
         \renewcommand{\erwaehnteWerke}{Werke: Deutsche Alpen, Italienische Reise, Liebelei. Schauspiel in drei Akten, Meyers Reisebücher}
               \section[Richard Beer-Hofmann an Arthur Schnitzler, 17. 9. 1895]{ Richard Beer-Hofmann an Arthur Schnitzler, 17. 9. 1895}\nopagebreak\mylabel{v}\rehead{ }\begin{ledgroupsized}[t]{13cm}\normalsize\beginnumbering\briefempfaengerindex{Schnitzler, Arthur@\textsc{Schnitzler, Arthur}!zzzBeer-Hofmann, Richard@\emph{von Richard Beer-Hofmann}!1895-09-171@{17. 9. 1895}|(be} \toendnotes[C]{\smallbreak\pagebreak[2]} \Standort{CUL, Schnitzler, B 8.}
\physDesc{Brief, 1 Blatt, 4 Seiten, 1482 Zeichen
\newline{}Handschrift: Bleistift, lateinische Kurrent
\newline{}Schnitzler: mit Bleistift nummeriert: »66« }\buchAbdrucke{\weitereDrucke{Arthur Schnitzler, Richard Beer-Hofmann: \emph{Briefwechsel 1891–1931}. Hg. Konstanze Fliedl. Wien, Zürich: \emph{Europaverlag} 1992, S. 82.} }\toendnotes[C]{\smallbreak}\pstart
           \raggedleft{}{\pb}Schönberg\oindex{Schoenberg im Stubaital@\textbf{Schönberg im Stubaital}|pw}{ }17/IX 95{ }Abends\pend
           \pstart
           Lieber Arthur!{ }\uline{Soeben} erhalte ich Ihren Brief. Ich bin wirklich in
               guter Sti{\geminationm}ung; hoffentlich merken Sie es an Manchem wenn
               ich nach Wien\oindex{Wien@\textbf{Wien}|pw} zurückko{\geminationm}e{[}.{]} Daß ich seit Sonntag{ }Früh allein bin wissen Sie wol. Wie das Alleinreisen von L.\pwindex{Andreas-Salome, Lou 12.02.1861 – 05.02.1937@\textsc{Andreas-Salomé, Lou} (12.02.1861 – 05.02.1937), \emph{Schriftstellerin}|pw} aufgeno{\geminationm}en
               wurde? Zu schwierig in Worte zu kleiden. Nur vorläufig: Sie geht nicht nach Kopenhagen\oindex{Kopenhagen@\textbf{Kopenhagen}|pw} – sagt sie. Aber das ist nicht
               offiziell. \strikeout{Hier will ich bis Freitag}{ }Samstag{ }\introOben{}Früh\introOben{} will ich von {\pb}hier fort nach
                  Riva\oindex{Riva del Garda@\textbf{Riva del Garda}|pw}, – einen Tag dort bleiben und dann nach
                  Salò\oindex{Salo@\textbf{Salò}|pw}, Südwestende des Gardasees\oindex{Lago di Garda@\textbf{Lago di Garda}|pw}. Vielleicht gefällt es mir aber dort nicht, dann
               vielleicht Verona\oindex{Verona@\textbf{Verona}|pw}, das ich nicht kenne.
               Jedenfalls erwarte ich noch einen Brief hieher, einen nach \uline{Riva}\oindex{Riva del Garda@\textbf{Riva del Garda}|pw}{ }\uline{Poste restante}.\pend
           \pstart
           Paul Horn\pwindex{Horn, Paul 13.02.1867 – 18.01.1936@\textsc{Horn, Paul} (13.02.1867 – 18.01.1936), \emph{Fabrikant}|pw} ist mir in der Erinnerung widerlich,
               Mann mit »lustigen Streichen« in der Jugend, kein Mensch.\pend
           \pstart
           {\pb}Wozu Brosamen wie »Alles erkundigt
               sich«? Wer verübelt uns übrigens daß wir nicht fort Litteratur reden?\pend
           \pstart
           Wie kommt Speidel\pwindex{Speidel, Ludwig 1830-04-11 – 1906-02-03@\textsc{Speidel, Ludwig} (1830-04-11 – 1906-02-03), \emph{Journalist, Kritiker}|pw} zu Ebermann\pwindex{Ebermann, Leo 16.07.1863 – 09.10.1914@\textsc{Ebermann, Leo} (16.07.1863 – 09.10.1914), \emph{Schriftsteller, Journalist, Rechtswissenschaftler}|pw}? Momentan bin ich \uline{der},
               der einzige Gast im Wirtshaus\oindex{Gasthaus Jagerhof@\textbf{Gasthaus Jagerhof}|pwv}. Ich »lebe u genieße«. Nochmals: Wann \uline{frühestens} kann »Liebelei\pwindex{Schnitzler, Arthur 15.05.1862 – 21.10.1931@\textsc{Schnitzler, Arthur} (15.05.1862 – 21.10.1931), \emph{Schriftsteller, Mediziner}!Liebelei. Schauspiel in drei Akten1895-10-09@\strich\emph{Liebelei. Schauspiel in drei Akten} {[}1895-10-09{]}|pw}« ko{\geminationm}en, denn vielleicht verzögert sich ja meine Ankunft, in
               den October hinein.\pend
           \pstart
           {\pb}Adieu, ich will noch vor der
               Dunkelheit ein wenig spazieren. Die Zirbelkiefer die an der Strasse steht, ko{\geminationm}t in Goethes\pwindex{Goethe, Johann Wolfgang von 1749-08-28 – 1832-03-22@\textsc{Goethe, Johann Wolfgang von} (1749-08-28 – 1832-03-22), \emph{Schriftsteller}|pw}{ }italienischer Reise\pwindex{Goethe, Johann Wolfgang von 1749-08-28 – 1832-03-22@\textsc{Goethe, Johann Wolfgang von} (1749-08-28 – 1832-03-22), \emph{Schriftsteller}!Italienische Reise1816 – 1817@\strich\emph{Italienische Reise} {[}1816 – 1817{]}|pw} vor. (Reise über den Brenner\oindex{Brenner@\textbf{Brenner}|pw}) »Bei Schemberg\oindex{Schoenberg im Stubaital@\textbf{Schönberg im Stubaital}|pw}« etc. das weiß ich aus dem \label{K_L00486-1v}\edtext{Meyer\pwindex{?? Werk@Nicht ermittelte Verfasserinnen und Verfasser!Deutsche Alpen1877@\emph{Deutsche Alpen} {[}1877{]}|pwv}}{\lemma{\textnormal{\emph{Meyer}}}\Cendnote{\textnormal{»Dagegen gelangt man {[}\ldots{]} auf dem \emph{alten}, r. abgehenden (schlechten) Fahrweg, {[}\ldots{]} den sogen. \emph{Alten
                        Schönberg} (dessen Zirben schon Goethe\pwindex{Goethe, Johann Wolfgang von 1749-08-28 – 1832-03-22@\textsc{Goethe, Johann Wolfgang von} (1749-08-28 – 1832-03-22), \emph{Schriftsteller}|pw} in seiner ›Italienischer
                        Reise\pwindex{Goethe, Johann Wolfgang von 1749-08-28 – 1832-03-22@\textsc{Goethe, Johann Wolfgang von} (1749-08-28 – 1832-03-22), \emph{Schriftsteller}!Italienische Reise1816 – 1817@\strich\emph{Italienische Reise} {[}1816 – 1817{]}|pw}‹ erwähnt; bei einer ›Goethe\pwindex{Goethe, Johann Wolfgang von 1749-08-28 – 1832-03-22@\textsc{Goethe, Johann Wolfgang von} (1749-08-28 – 1832-03-22), \emph{Schriftsteller}|pw}bank‹ schöne Aussicht) hinan«. (\emph{Meyers Reisebücher}\pwindex{?? Werk@Nicht ermittelte Verfasserinnen und Verfasser!Meyers ReisebuecherNone@\emph{Meyers Reisebücher} {[}None{]}|pwk}. \emph{Deutsche Alpen. Erster Teil: Bayerisches Hochland, Allgäu,
                        Vorarlberg, Tirol, Brennerbahn, Ötztaler-, Stubaier-, und Ortlergruppe,
                        Bozen, Schlern und Rosengarten, Meran, Brenta- und Adamellogruppe;
                        Bergamasker Alpen, Gardasee.}\pwindex{?? Werk@Nicht ermittelte Verfasserinnen und Verfasser!Deutsche Alpen1877@\emph{Deutsche Alpen} {[}1877{]}|pwk} Fünfte Auflage. Mit 23 Karten, 4 Plänen
                     und 12 Panoramen. Leipzig, Wien: \emph{Bibliographisches Institut}\orgindex{Bibliographisches Institut@Bibliographisches Institut|pwk}{ }1896, S. 217.)}}}\label{K_L00486-1h}. Werden uns je Bäume irgendwo wachsen
               – bei Meyer\pwindex{?? Werk@Nicht ermittelte Verfasserinnen und Verfasser!Meyers ReisebuecherNone@\emph{Meyers Reisebücher} {[}None{]}|pw}?\pend
           \pstart
           »Laßt uns lächeln.«\pend
           \pstart
           Herzlichst Ihr{\\[\baselineskip]}\spacefill\mbox{Richard}\pend
           \leftskip=0em{}\pstart
           \noindent{}Ich freu mich so sehr mit Ihren Briefen\pend
           \pstart
           »\label{K_L00486-2v}\edtext{schreiben Sie augenscharf}{\lemma{\textnormal{\emph{schreiben Sie augenscharf}}}\Cendnote{\textnormal{offenbar ein stehender Ausdruck der
                     Gruppe, der sich auch im Briefwechsel zwischen Salten\pwindex{Salten, Felix 06.09.1869 – 08.10.1945@\textsc{Salten, Felix} (06.09.1869 – 08.10.1945), \emph{Schriftsteller, Journalist}|pwk} und Hofmannsthal\pwindex{Hofmannsthal, Hugo von 1874-02-01 – 1929-07-15@\textsc{Hofmannsthal, Hugo von} (1874-02-01 – 1929-07-15), \emph{Schriftsteller}|pwk}
                     nachweisen lässt.}}}\label{K_L00486-2h}«\pend
           
         
         \endnumbering\mylabel{h}\end{ledgroupsized}  \newcommand{\dateiname}{L00486}\newcommand{\titel}{Richard Beer-Hofmann an Arthur Schnitzler, 17. 9. 1895}\newcommand{\editorInnen}{Martin Anton Müller und Gerd-Hermann Susen}%% latex-leseansicht-abspann.tex
%% Abspann für die Leseansicht.
%% Der Schalter \ifkorrekturansicht ist bereits durch den Vorspann gesetzt.

%% latex-abspann.tex
%% Gemeinsamer Abspann für Korrekturansicht und Leseansicht.
%% Setzt den Schalter \ifkorrekturansicht voraus (gesetzt in den
%% einbindenden Dateien latex-korrekturansicht-abspann.tex bzw.
%% latex-leseansicht-abspann.tex).
%% ---------------------------------------------------------------

\normalsize

% Das esempio-Environment wird nur in der Leseansicht benötigt
\ifkorrekturansicht\else
\newenvironment{esempio}[3]%
{
    \vspace{1.5ex}
    \rlap{\underline{#1}}
    \par
    \setlength{\parindent}{0cm}
    \nopagebreak
    \leftskip=#2cm
    \rightskip=#3cm
}
{
    \par
}
\fi

\doendnotes{C}
\bigskip
\vfill

\clearpage

\footnotesize

\ifkorrekturansicht
  \lohead{\textsc{register}}
\fi

% theindex-Environment neu definieren ohne reledmac
\makeatletter
\renewenvironment{theindex}{%
  \ifkorrekturansicht
    \section*{\indexname}%
  \else
    \subsubsection*{Index der erwähnten Entitäten}%
  \fi
  \setlength{\parindent}{0pt}%
  \setlength{\parskip}{0pt plus 0.3pt}%
  \let\item\@idxitem
}{%
  \ifkorrekturansicht\clearpage\fi
}
\makeatother

\IfFileExists{\jobname-pw.ind}{\input{\jobname-pw.ind}}{}

% Quellenangabe nur in der Leseansicht
\ifkorrekturansicht\else
% Fallback-Definitionen, falls die .tex-Datei \titel etc. nicht gesetzt hat
\providecommand{\titel}{}
\providecommand{\editorInnen}{}
\providecommand{\dateiname}{\jobname}

\vspace{3cm}

\vfill

\footnotesize
\textsc{Quelle}: \titel. Herausgegeben von {\editorInnen}. In: \emph{Arthur Schnitzler: Briefwechsel mit Autorinnen und Autoren}.
 Digitale Edition, https://schnitzler-briefe.acdh.oeaw.ac.at/{\dateiname}.html (Stand \today)
\fi

\end{document}


      