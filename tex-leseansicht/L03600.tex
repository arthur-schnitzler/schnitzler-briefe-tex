%% latex-korrekturansicht-vorspann.tex
%% Vorspann für die Korrekturansicht.
%% Lädt die gemeinsame Datei latex-vorspann.tex mit gesetztem Schalter.

\newif\ifkorrekturansicht
\korrekturansichttrue

\input{../tex-inputs/latex-vorspann}


\section[ Arthur Schnitzler: Widmungsexemplar Das Märchen für Felix Salten, 8. 5. 1894]{L03600 Arthur Schnitzler: Widmungsexemplar Das Märchen für Felix
               Salten, 8. 5. 1894}
\nopagebreak\mylabel{L03600v}
\rehead{ }\normalsize\beginnumbering\briefempfaengerindex{Salten, Felix@\textsc{Salten, Felix}!zzzSchnitzler, Arthur@\emph{von Arthur Schnitzler}!1894-05-083@{8. 5. 94}|(be}
\toendnotes[C]{\smallbreak\pagebreak[2]}\Standort{Wienbibliothek im Rathaus, A-57094/2.Ex., DS-2018-9503.}
\physDesc{Widmung am Vorsatzblatt, 59 Zeichen
\newline{}Handschrift: schwarze Tinte, deutsche Kurrent
\newline{}Ordnung: 1) mit schwarzer Tinte am linken Vorsatzblatt gestrichene Regalerfassung: »\noindent{}IN\textsuperscript{o} 2468 WN\textsuperscript{o} 1537{ / }XI b«  2) mit schwarzer Tinte ausgefüllter Stempel: »\noindent{}\textcolor{gray}{\textbf{\textit{Felix Salten}}}{ / }\textcolor{gray}{\textbf{\textit{Inv. Nr.}}}{ }4464{ / }\textcolor{gray}{\textbf{\textit{Werk Nr.}}}{ }2192{ / }\textcolor{gray}{\textbf{\textit{Schrank}}}{ }XIV A.
                                          Z\textcolor{gray}{.}\textcolor{gray}{\textbf{\textit{Fach}}} b«}\toendnotes[C]{\smallbreak}
\pstart
           \noindent{}{\pb}Meinem lieben \textsc{Felix Salten}\pend
           
\pstart
           herzlichſt {\\[\baselineskip]}\spacefill\mbox{Arth Sch}\pend
           \leftskip=0em{}
\pstart
           Wien\oindex{Wien@\textbf{Wien}, \emph{A.ADM2}|pw}, 8. 5. 94.\pend
           \selectlanguage{ngerman}\vspace{1em}{\vspace{1\baselineskip}}
\pstart
           \centering{}{\pb}\textcolor{gray}{\textbf{\textbf{Das Märchen\pwindex{Maerchen. Schauspiel in drei Aufzuegen@\emph{Das Märchen. Schauspiel in drei Aufzügen}|pw}.}}}\pend
           
\pstart
           \centering{}\textcolor{gray}{\textbf{\textbf{Schauſpiel in drei Aufzügen}}}{\\}\textcolor{gray}{\textbf{von}}{\\}\textcolor{gray}{\textbf{\textbf{Arthur Schnitzler.}}}\pend
           {\vspace{1\baselineskip}}
\pstart
           \centering{}\textcolor{gray}{\textbf{\textbf{Dresden}\oindex{Dresden@\textbf{Dresden}, \emph{P.PPLA}|pw} und \textbf{Leipzig}\oindex{Leipzig@\textbf{Leipzig}, \emph{P.PPLA3}|pw}}}\pend
           
\pstart
           \centering{}\textcolor{gray}{\textbf{\so{E. Pierſon’s Verlag}\orgindex{E. Pierson s Verlag@E. Pierson’s Verlag|pw}}}\pend
           
\pstart
           \centering{}\textcolor{gray}{\textbf{\label{K_L03600-1v}\edtext{1894}{\lemma{\textnormal{\emph{1894}}}\Cendnote{\textnormal{\emph{Das Märchen}\pwindex{Maerchen. Schauspiel in drei Aufzuegen@\emph{Das Märchen. Schauspiel in drei Aufzügen}|pwk} war am 5. 5. 1894 vom \emph{Börsenblatt für den deutschen Buchhandel}\pwindex{Boersenblatt fuer den Deutschen Buchhandel@\emph{Börsenblatt für den Deutschen Buchhandel}|pwk} als Neuerscheinung
                        gemeldet worden.}}}\label{K_L03600-1}.}}\pend
           \selectlanguage{ngerman}\endnumbering\briefempfaengerindex{Salten, Felix@\textsc{Salten, Felix}!zzzSchnitzler, Arthur@\emph{von Arthur Schnitzler}!1894-05-083@{8. 5. 94}|)be}\mylabel{L03600h}  \normalsize

\doendnotes{C}
\bigskip
\vfill

\clearpage

\footnotesize

\lohead{\textsc{register}}

% Definiere theindex-Environment komplett neu ohne reledmac
\makeatletter
\renewenvironment{theindex}{%
  \section*{\indexname}%
  \setlength{\parindent}{0pt}%
  \setlength{\parskip}{0pt plus 0.3pt}%
  \let\item\@idxitem
}{%
  \clearpage
}
\makeatother

\IfFileExists{\jobname-pw.ind}{\input{\jobname-pw.ind}}{}

\end{document}

      