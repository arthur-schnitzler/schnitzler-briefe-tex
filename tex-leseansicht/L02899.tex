%% latex-leseansicht-vorspann.tex
%% Vorspann für die Leseansicht.
%% Lädt die gemeinsame Datei latex-vorspann.tex mit nicht gesetztem Schalter.

\newif\ifkorrekturansicht
\korrekturansichtfalse

\input{../tex-inputs/latex-vorspann}

\begin{center}
            \textcolor{red}{ENTWURF, NICHT FERTIG KORRIGIERT}
                      \end{center}
            
         
         \renewcommand{\erwaehntePersonen}{Personen: Heinrich Heine, Alfred Kerr}
         \renewcommand{\erwaehnteOrte}{Orte: Deutschland, Frankfurt am Main, Wien}
         \renewcommand{\erwaehnteWerke}{Werke: Frankfurter Zeitung, Heine}
               \section[ Paul Goldmann an Arthur Schnitzler, 13. 12. {[}1899{]}]{ Paul Goldmann an Arthur Schnitzler, 13. 12. {[}1899{]}}\nopagebreak\mylabel{v}\rehead{ }\begin{ledgroupsized}[t]{13cm}\normalsize\beginnumbering \toendnotes[C]{\smallbreak\pagebreak[2]} \Standort{DLA, A:Schnitzler, HS.NZ85.1.3169.}
\physDesc{Brief, 1 Blatt, 2 Seiten
\newline{}Handschrift: blaue Tinte, deutsche Kurrent
\newline{}Schnitzler: 1) mit Bleistift das Jahr »99« vermerkt  2) mit rotem Buntstift eine Unterstreichung}\toendnotes[C]{\smallbreak}\pstart
           \raggedleft{}{\pb}Frankfurt\oindex{Frankfurt am Main@\textbf{Frankfurt am Main}|pw}, 13. Dezember.\pend
           \pstart\center{}Mein lieber Freund,\pend\pstart
           Da Du wohl nicht die »Frankfurter Zeitung\pwindex{?? Werk@Nicht ermittelte Verfasserinnen und Verfasser!Frankfurter Zeitung1856 – 1943@\emph{Frankfurter Zeitung} {[}1856 – 1943{]}|pw}«
               lieſt, \label{K_L02899-1v}\edtext{ſende ich Dir anbei das geſtern erſchienene Feuilleton\pwindex{Kerr, Alfred 25.12.1867 – 12.10.1948@\textsc{Kerr, Alfred} (25.12.1867 – 12.10.1948), \emph{Schriftsteller, Kritiker}!Heine1899-12-13@\strich\emph{Heine} {[}1899-12-13{]}|pwv} von \textsc{Kerr\pwindex{Kerr, Alfred 25.12.1867 – 12.10.1948@\textsc{Kerr, Alfred} (25.12.1867 – 12.10.1948), \emph{Schriftsteller, Kritiker}|pw}} über \textsc{Heine\pwindex{Heine, Heinrich 13.12.1797 – 17.02.1856@\textsc{Heine, Heinrich} (13.12.1797 – 17.02.1856), \emph{Schriftsteller}|pw}}}{\lemma{\textnormal{\emph{ſende … Heine}}}\Cendnote{\textnormal{Alfred Kerr\pwindex{Kerr, Alfred 25.12.1867 – 12.10.1948@\textsc{Kerr, Alfred} (25.12.1867 – 12.10.1948), \emph{Schriftsteller, Kritiker}|pwk}: \emph{Heine}\pwindex{Kerr, Alfred 25.12.1867 – 12.10.1948@\textsc{Kerr, Alfred} (25.12.1867 – 12.10.1948), \emph{Schriftsteller, Kritiker}!Heine1899-12-13@\strich\emph{Heine} {[}1899-12-13{]}|pwk}. In: \emph{Frankfurter
                        Zeitung}\pwindex{?? Werk@Nicht ermittelte Verfasserinnen und Verfasser!Frankfurter Zeitung1856 – 1943@\emph{Frankfurter Zeitung} {[}1856 – 1943{]}|pwk}, Jg. 44, Nr. 345, 13. 12. 1899, Erstes Morgenblatt, S. 1–2. Schnitzler\pwindex{Schnitzler, Arthur 15.05.1862 – 21.10.1931@\textsc{Schnitzler, Arthur} (15.05.1862 – 21.10.1931), \emph{Schriftsteller, Mediziner}|pwk} hatte den Brief spätestens am 15. 12. 1899 in den
                  Händen, da schrieb er an Kerr\pwindex{Kerr, Alfred 25.12.1867 – 12.10.1948@\textsc{Kerr, Alfred} (25.12.1867 – 12.10.1948), \emph{Schriftsteller, Kritiker}|pwk}:
                     »Lieber Herr Kerr\pwindex{Kerr, Alfred 25.12.1867 – 12.10.1948@\textsc{Kerr, Alfred} (25.12.1867 – 12.10.1948), \emph{Schriftsteller, Kritiker}|pw}, ich muss Ihnen
                     diesen Brief meines Freundes Goldmann\pwindex{Goldmann, Paul 31.01.1865 – 25.09.1935@\textsc{Goldmann, Paul} (31.01.1865 – 25.09.1935), \emph{Schriftsteller, Journalist}|pw}
                     doch senden – Sie werden so freundlich sein, ihm (G.\pwindex{Goldmann, Paul 31.01.1865 – 25.09.1935@\textsc{Goldmann, Paul} (31.01.1865 – 25.09.1935), \emph{Schriftsteller, Journalist}|pw}!) nie zu verrathen, daß ich es gethan, und senden
                     mir ihn (den Brief) auch bald wieder zurück. Freuen wird es Sie jedenfalls –
                     wie man überhaupt Ehrgeiz hat, – haben soll? haben muss? – das beste bleibt
                     doch zu wünschen, dass andere kluge Menschen gut über uns denken. Der Ansicht
                        G.\pwindex{Goldmann, Paul 31.01.1865 – 25.09.1935@\textsc{Goldmann, Paul} (31.01.1865 – 25.09.1935), \emph{Schriftsteller, Journalist}|pw}s über Ihr Feuilleton\pwindex{Kerr, Alfred 25.12.1867 – 12.10.1948@\textsc{Kerr, Alfred} (25.12.1867 – 12.10.1948), \emph{Schriftsteller, Kritiker}!Heine1899-12-13@\strich\emph{Heine} {[}1899-12-13{]}|pwv} schließ ich mich
                     vollkommen an – ohne sein Empfinden von ›Zurückgeworfensein in die
                     Mittelmäßigkeit‹ im geringsten berechtigt zu finden. Denn auch er gehört zu den
                     ganz vortrefflichen.« (Kerr\pwindex{Kerr, Alfred 25.12.1867 – 12.10.1948@\textsc{Kerr, Alfred} (25.12.1867 – 12.10.1948), \emph{Schriftsteller, Kritiker}|pwk}, Schnitzler\pwindex{Schnitzler, Arthur 15.05.1862 – 21.10.1931@\textsc{Schnitzler, Arthur} (15.05.1862 – 21.10.1931), \emph{Schriftsteller, Mediziner}|pwk}: \emph{»Es ist eine sehr seltsame Gefühlsmischung, die Sie erwecken.«
                        Briefwechsel 1896–1925}. Hg. v. Elgin Helmstaedt. In: \emph{Sinn und Form}, Jg. 69, H. 5,
                        September/Oktober 2017, S. 598–599)}}}\label{K_L02899-1h}. Ich halte
               dasſelbe für eines der vollendetſten Kunſtwerke, welche die neuere deutſch\oindex{Deutschland@\textbf{Deutschland}|pwv}e Journaliſtik hervorgebracht hat.
               Wenn man ſelbſt Zeitungsſchreiber von Beruf iſt, ſo fühlt man ſich tief verſtimmt
               durch \strikeout{eine} dieſe \strikeout{ſolche}{ }Arbeit\pwindex{Kerr, Alfred 25.12.1867 – 12.10.1948@\textsc{Kerr, Alfred} (25.12.1867 – 12.10.1948), \emph{Schriftsteller, Kritiker}!Heine1899-12-13@\strich\emph{Heine} {[}1899-12-13{]}|pwv}, die eine ſolche Kunſt
               des Ausdrucks, eine ſolche Kraft der Concentrirung, einen ſo unbedingt perſönlichen
               Styl und ein ſo gründliches Wiſſen bekundet. Es ſteckt thatſächlich etwas Geniales \substVorne{}\textsuperscript{darin}\substDazwischen{}darin\substHinten{} – {\pb}etwas von \textsc{Heine\pwindex{Heine, Heinrich 13.12.1797 – 17.02.1856@\textsc{Heine, Heinrich} (13.12.1797 – 17.02.1856), \emph{Schriftsteller}|pw}’s} Größe (ohne den leiſeſten
               Anklang an \textsc{Heine\pwindex{Heine, Heinrich 13.12.1797 – 17.02.1856@\textsc{Heine, Heinrich} (13.12.1797 – 17.02.1856), \emph{Schriftsteller}|pw}’s} Art), – und, wenn man ſelbſt
               Zeitungsſchreiber von Beruf iſt (ſiehe oben), ſo fühlt man ſich erbarmungslos in die
               Mittelmäßigkeit zurückgeworfen.\pend
           \pstart
           Viele treue Grüße! {\\[\baselineskip]}Dein {\\[\baselineskip]}\spacefill\mbox{Paul Goldmann}\pend
           \leftskip=0em{}
         
         \endnumbering\mylabel{h}\end{ledgroupsized}  \newcommand{\dateiname}{L02899}\newcommand{\titel}{Paul Goldmann an Arthur Schnitzler, 13. 12. [1899]}\newcommand{\editorInnen}{Martin Anton Müller und Laura Untner}%% latex-leseansicht-abspann.tex
%% Abspann für die Leseansicht.
%% Der Schalter \ifkorrekturansicht ist bereits durch den Vorspann gesetzt.

%% latex-abspann.tex
%% Gemeinsamer Abspann für Korrekturansicht und Leseansicht.
%% Setzt den Schalter \ifkorrekturansicht voraus (gesetzt in den
%% einbindenden Dateien latex-korrekturansicht-abspann.tex bzw.
%% latex-leseansicht-abspann.tex).
%% ---------------------------------------------------------------

\normalsize

% Das esempio-Environment wird nur in der Leseansicht benötigt
\ifkorrekturansicht\else
\newenvironment{esempio}[3]%
{
    \vspace{1.5ex}
    \rlap{\underline{#1}}
    \par
    \setlength{\parindent}{0cm}
    \nopagebreak
    \leftskip=#2cm
    \rightskip=#3cm
}
{
    \par
}
\fi

\doendnotes{C}
\bigskip
\vfill

\clearpage

\footnotesize

\ifkorrekturansicht
  \lohead{\textsc{register}}
\fi

% theindex-Environment neu definieren ohne reledmac
\makeatletter
\renewenvironment{theindex}{%
  \ifkorrekturansicht
    \section*{\indexname}%
  \else
    \subsubsection*{Index der erwähnten Entitäten}%
  \fi
  \setlength{\parindent}{0pt}%
  \setlength{\parskip}{0pt plus 0.3pt}%
  \let\item\@idxitem
}{%
  \ifkorrekturansicht\clearpage\fi
}
\makeatother

\IfFileExists{\jobname-pw.ind}{\input{\jobname-pw.ind}}{}

% Quellenangabe nur in der Leseansicht
\ifkorrekturansicht\else
% Fallback-Definitionen, falls die .tex-Datei \titel etc. nicht gesetzt hat
\providecommand{\titel}{}
\providecommand{\editorInnen}{}
\providecommand{\dateiname}{\jobname}

\vspace{3cm}

\vfill

\footnotesize
\textsc{Quelle}: \titel. Herausgegeben von {\editorInnen}. In: \emph{Arthur Schnitzler: Briefwechsel mit Autorinnen und Autoren}.
 Digitale Edition, https://schnitzler-briefe.acdh.oeaw.ac.at/{\dateiname}.html (Stand \today)
\fi

\end{document}


      