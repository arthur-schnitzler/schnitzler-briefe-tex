%% latex-korrekturansicht-vorspann.tex
%% Vorspann für die Korrekturansicht.
%% Lädt die gemeinsame Datei latex-vorspann.tex mit gesetztem Schalter.

\newif\ifkorrekturansicht
\korrekturansichttrue

\input{../tex-inputs/latex-vorspann}


\section[ Paul Goldmann an Arthur Schnitzler, 13. 12. {[}1899{]}]{L02899 Paul Goldmann an Arthur Schnitzler, 13. 12. {[}1899{]}}
\nopagebreak\mylabel{L02899v}
\rehead{ }\normalsize\beginnumbering\briefempfaengerindex{Schnitzler, Arthur@\textsc{Schnitzler, Arthur}!zzzGoldmann, Paul@\emph{von Paul Goldmann}!1899-12-131@{13. 12. {[}1899{]}}|(be}
\toendnotes[C]{\smallbreak\pagebreak[2]}\Standort{DLA, A:Schnitzler, HS.NZ85.1.3169.}
\physDesc{Brief, 1 Blatt, 2 Seiten, 827 Zeichen
\newline{}Handschrift: blaue Tinte, deutsche Kurrent
\newline{}Schnitzler: 1) mit Bleistift das Jahr »99« vermerkt  2) mit rotem Buntstift eine Unterstreichung}\toendnotes[C]{\smallbreak}
\pstart
           \raggedleft{}{\pb}Frankfurt\oindex{Frankfurt am Main@\textbf{Frankfurt am Main}, \emph{P.PPLA3}|pw}, 13. Dezember.\pend
           
\pstart\center{}Mein lieber Freund,\pend\vspace{0.5em}
\pstart
           Da Du wohl nicht die »Frankfurter Zeitung\pwindex{Frankfurter Zeitung@\emph{Frankfurter Zeitung}|pw}«
               lieſt, \label{K_L02899-1v}\edtext{ſende ich Dir anbei das geſtern erſchienene Feuilleton\pwindex{Heine@\emph{Heine}|pwv} von \textsc{Kerr\pwindex{Kerr, Alfred 25.12.1867 – 12.10.1948@\textsc{Kerr, Alfred} (25.12.1867 – 12.10.1948), \emph{Schriftsteller/Schriftstellerin, Kritiker/Kritikerin}|pw}} über \textsc{Heine\pwindex{Heine, Heinrich 13.12.1797 – 17.02.1856@\textsc{Heine, Heinrich} (13.12.1797 – 17.02.1856), \emph{Schriftsteller/Schriftstellerin}|pw}}}{\lemma{\textnormal{\emph{ſende … Heine}}}\Cendnote{\textnormal{Alfred Kerr\pwindex{Kerr, Alfred 25.12.1867 – 12.10.1948@\textsc{Kerr, Alfred} (25.12.1867 – 12.10.1948), \emph{Schriftsteller/Schriftstellerin, Kritiker/Kritikerin}|pwk}: \emph{Heine}\pwindex{Heine@\emph{Heine}|pwk}. In: \emph{Frankfurter
                        Zeitung}\pwindex{Frankfurter Zeitung@\emph{Frankfurter Zeitung}|pwk}, Jg. 44, Nr. 345, 13. 12. 1899,
                     Erstes Morgenblatt, S. 1–2. Schnitzler hatte den Brief spätestens am 15. 12. 1899 in den
                  Händen, da schrieb er an Kerr\pwindex{Kerr, Alfred 25.12.1867 – 12.10.1948@\textsc{Kerr, Alfred} (25.12.1867 – 12.10.1948), \emph{Schriftsteller/Schriftstellerin, Kritiker/Kritikerin}|pwk}:
                     »Lieber Herr Kerr\pwindex{Kerr, Alfred 25.12.1867 – 12.10.1948@\textsc{Kerr, Alfred} (25.12.1867 – 12.10.1948), \emph{Schriftsteller/Schriftstellerin, Kritiker/Kritikerin}|pw}, ich muss Ihnen
                     diesen Brief meines Freundes Goldmann\pwindex{Goldmann, Paul 31.01.1865 – 25.09.1935@\textsc{Goldmann, Paul} (31.01.1865 – 25.09.1935), \emph{Schriftsteller/Schriftstellerin, Journalist/Journalistin}|pw}
                     doch senden – Sie werden so freundlich sein, ihm (G.\pwindex{Goldmann, Paul 31.01.1865 – 25.09.1935@\textsc{Goldmann, Paul} (31.01.1865 – 25.09.1935), \emph{Schriftsteller/Schriftstellerin, Journalist/Journalistin}|pw}!) nie zu verrathen, daß ich es gethan, und senden
                     mir ihn (den Brief) auch bald wieder zurück. Freuen wird es Sie jedenfalls –
                     wie man überhaupt Ehrgeiz hat, – haben soll? haben muss? – das beste bleibt
                     doch zu wünschen, dass andere kluge Menschen gut über uns denken. Der Ansicht
                        G.s\pwindex{Goldmann, Paul 31.01.1865 – 25.09.1935@\textsc{Goldmann, Paul} (31.01.1865 – 25.09.1935), \emph{Schriftsteller/Schriftstellerin, Journalist/Journalistin}|pw} über Ihr Feuilleton\pwindex{Heine@\emph{Heine}|pwv} schließ ich mich
                     vollkommen an – ohne sein Empfinden von ›Zurückgeworfensein in die
                     Mittelmäßigkeit‹ im geringsten berechtigt zu finden. Denn auch er gehört zu den
                     ganz vortrefflichen.« (Kerr\pwindex{Kerr, Alfred 25.12.1867 – 12.10.1948@\textsc{Kerr, Alfred} (25.12.1867 – 12.10.1948), \emph{Schriftsteller/Schriftstellerin, Kritiker/Kritikerin}|pwk}, Schnitzler: \emph{»Es ist eine sehr seltsame
                        Gefühlsmischung, die Sie erwecken.« Briefwechsel 1896–1925}.
                     Herausgegeben von Elgin Helmstaedt. In: \emph{Sinn und Form},
                     Jg. 69, H. 5, September/Oktober 2017, S. 598–599.)}}}\label{K_L02899-1}.
               Ich halte dasſelbe für eines der vollendetſten Kunſtwerke, welche die neuere deutſch\oindex{Deutschland@\textbf{Deutschland}, \emph{A.PCLI}|pwv}e Journaliſtik
               hervorgebracht hat. Wenn man ſelbſt Zeitungsſchreiber von Beruf iſt, ſo fühlt man
               ſich tief verſtimmt durch \strikeout{eine} dieſe \strikeout{ſolche}{ }Arbeit\pwindex{Heine@\emph{Heine}|pwv}, die eine ſolche Kunſt
               des Ausdrucks, eine ſolche Kraft der Concentrirung, einen ſo unbedingt perſönlichen
               Styl und ein ſo gründliches Wiſſen bekundet. Es ſteckt thatſächlich etwas Geniales \substVorne{}\textsuperscript{darin}\substDazwischen{}darin\substHinten{} – {\pb}etwas von \textsc{Heine\pwindex{Heine, Heinrich 13.12.1797 – 17.02.1856@\textsc{Heine, Heinrich} (13.12.1797 – 17.02.1856), \emph{Schriftsteller/Schriftstellerin}|pw}’s} Größe (ohne den leiſeſten
               Anklang an \textsc{Heine\pwindex{Heine, Heinrich 13.12.1797 – 17.02.1856@\textsc{Heine, Heinrich} (13.12.1797 – 17.02.1856), \emph{Schriftsteller/Schriftstellerin}|pw}’s} Art), – und, wenn man ſelbſt
               Zeitungsſchreiber von Beruf iſt (ſiehe oben), ſo fühlt man ſich erbarmungslos in die
               Mittelmäßigkeit zurückgeworfen.\pend
           
\pstart
           Viele treue Grüße! {\\[\baselineskip]}Dein {\\[\baselineskip]}\spacefill\mbox{Paul Goldmann}\pend
           \leftskip=0em{}\selectlanguage{ngerman}\endnumbering\briefempfaengerindex{Schnitzler, Arthur@\textsc{Schnitzler, Arthur}!zzzGoldmann, Paul@\emph{von Paul Goldmann}!1899-12-131@{13. 12. {[}1899{]}}|)be}\mylabel{L02899h}  \normalsize

\doendnotes{C}
\bigskip
\vfill

\clearpage

\footnotesize

\lohead{\textsc{register}}

% Definiere theindex-Environment komplett neu ohne reledmac
\makeatletter
\renewenvironment{theindex}{%
  \section*{\indexname}%
  \setlength{\parindent}{0pt}%
  \setlength{\parskip}{0pt plus 0.3pt}%
  \let\item\@idxitem
}{%
  \clearpage
}
\makeatother

\IfFileExists{\jobname-pw.ind}{\input{\jobname-pw.ind}}{}

\end{document}

      