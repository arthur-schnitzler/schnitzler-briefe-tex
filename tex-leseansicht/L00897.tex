%% latex-korrekturansicht-vorspann.tex
%% Vorspann für die Korrekturansicht.
%% Lädt die gemeinsame Datei latex-vorspann.tex mit gesetztem Schalter.

\newif\ifkorrekturansicht
\korrekturansichttrue

\input{../tex-inputs/latex-vorspann}


\section[Jakob Julius David an Arthur Schnitzler, 3. 3. 1899]{L00897 Jakob Julius David an Arthur Schnitzler, 3. 3. 1899}
\nopagebreak\mylabel{L00897v}
\rehead{ }\normalsize\beginnumbering\briefempfaengerindex{Schnitzler, Arthur@\textsc{Schnitzler, Arthur}!zzzDavid, Jakob Julius@\emph{von Jakob Julius David}!1899-03-031@{3. 3. 1899}|(be}
\toendnotes[C]{\smallbreak\pagebreak[2]}\Standort{CUL, Schnitzler, B 25.}
\physDesc{Postkarte, 216 Zeichen
\newline{}Handschrift: schwarze Tinte, lateinische Kurrent
\newline{}Versand: 1) Stempel: »\nobreak{}\oindex{II., Leopoldstadt@\textbf{II., Leopoldstadt}, \emph{A.ADM3}|pwk}Wien 2/3, 3. 3. 99, 1–4N\nobreak{}«.   2) Stempel: »\nobreak{}\oindex{IX., Alsergrund@\textbf{IX., Alsergrund}, \emph{A.ADM3}|pwk}Wien 9/3, 3. 3. 99, 6.N\nobreak{}«. 
\newline{}Ordnung: mit Bleistift von unbekannter Hand nummeriert:
                                 »5« }\toendnotes[C]{\smallbreak}\pstart{}{\pb}Herrn D\textsuperscript{r}
                  Arthur Schnitzler\pend{}\pstart{}IX. Franckgaße N\textsuperscript{o} 1\oindex{Frankgasse 1@\textbf{Frankgasse 1}, \emph{Wohngebäude (K.WHS)}|pw}. \pend{}{\bigskip}\vspace{1em}
\pstart\center{}{\pb}Werther Herr!\pend\vspace{0.5em}
\pstart
           Schön Dank. Also \label{K_L00897-1v}\edtext{Dienstag}{\lemma{\textnormal{\emph{Dienstag}}}\Cendnote{\textnormal{An diesem Tag fand die vierte Aufführung
                  der drei Einakter \emph{Der grüne Kakadu – Paracelsus –
                     Die Gefährtin}\pwindex{gruene Kakadu – Paracelsus – Die Gefaehrtin. Drei Einakter@\emph{Der grüne Kakadu – Paracelsus – Die Gefährtin. Drei Einakter}|pwk} statt. David\pwindex{David, Jakob Julius 1859-02-06 – 1906-11-20@\textsc{David, Jakob Julius} (1859-02-06 – 1906-11-20), \emph{Schriftsteller/Schriftstellerin, Journalist/Journalistin}|pwk} dürfte die am 28. 2. 1899 erbetenen Freikarten bekommen
                  haben.}}}\label{K_L00897-1}.\pend
           
\pstart
            Seither haben Sie ja wohl auch gesehen, daß ich coram publico nicht anders \label{K_L00897-2v}\edtext{schrieb\pwindex{Aus ungleichen Tagen@\emph{Aus ungleichen Tagen}|pwv}}{\lemma{\textnormal{\emph{schrieb}}}\Cendnote{\textnormal{J. J. David\pwindex{David, Jakob Julius 1859-02-06 – 1906-11-20@\textsc{David, Jakob Julius} (1859-02-06 – 1906-11-20), \emph{Schriftsteller/Schriftstellerin, Journalist/Journalistin}|pwk}: \emph{Aus ungleichen Tagen. (»Paracelsus«, Schauspiel; »Die
                        Gefährtin«, Schauspiel; »Der grüne Kakadu«, Groteske. Drei Einacter von
                        Arthur Schnitzler. Im Burgtheater zum erstenmale aufgeführt am 1. März
                        1899)}\pwindex{Aus ungleichen Tagen@\emph{Aus ungleichen Tagen}|pwk}. In: \emph{Neues Wiener Journal}\pwindex{Neues Wiener Journal@\emph{Neues Wiener Journal}|pwk},
                     Jg. 7, Nr. 1925, 2. 3. 1899, S. 1–2.}}}\label{K_L00897-2}. Unsere Kritik!
               Ein feines Capitel! \pend
           
\pstart
           Bestens Ihr{\\[\baselineskip]}\spacefill\mbox{David}\pend
           \leftskip=0em{}\selectlanguage{ngerman}\endnumbering\briefempfaengerindex{Schnitzler, Arthur@\textsc{Schnitzler, Arthur}!zzzDavid, Jakob Julius@\emph{von Jakob Julius David}!1899-03-031@{3. 3. 1899}|)be}\mylabel{L00897h}  \normalsize

\doendnotes{C}
\bigskip
\vfill

\clearpage

\footnotesize

\lohead{\textsc{register}}

% Definiere theindex-Environment komplett neu ohne reledmac
\makeatletter
\renewenvironment{theindex}{%
  \section*{\indexname}%
  \setlength{\parindent}{0pt}%
  \setlength{\parskip}{0pt plus 0.3pt}%
  \let\item\@idxitem
}{%
  \clearpage
}
\makeatother

\IfFileExists{\jobname-pw.ind}{\input{\jobname-pw.ind}}{}

\end{document}

      