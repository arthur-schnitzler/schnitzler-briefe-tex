%% latex-leseansicht-vorspann.tex
%% Vorspann für die Leseansicht.
%% Lädt die gemeinsame Datei latex-vorspann.tex mit nicht gesetztem Schalter.

\newif\ifkorrekturansicht
\korrekturansichtfalse

\input{../tex-inputs/latex-vorspann}


\section[Arthur Schnitzler an Gustav Schwarzkopf, 12. 10. 1919]{L04159 Arthur Schnitzler an Gustav Schwarzkopf, 12. 10. 1919}
\nopagebreak\mylabel{L04159v}
\rehead{ }\normalsize\beginnumbering\briefempfaengerindex{Schwarzkopf, Gustav@\textsc{Schwarzkopf, Gustav}!zzzSchnitzler, Arthur@\emph{von Arthur Schnitzler}!1919-10-121@{12. 10. 1919}|(be}
\toendnotes[C]{\smallbreak\pagebreak[2]}
\correspDesc{Versand  durch Arthur Schnitzler am 12. 10. 1919 in Wien
\newline{}Erhalt  durch Gustav Schwarzkopf im Zeitraum [12. 10. 1919 – 15. 10. 1919?] in Wien}\toendnotes[C]{\smallbreak}
\Standort{CUL, Schnitzler, B 96.}
\physDesc{Brief, 1 Blatt, 1 Seite, 174 Zeichen
\newline{}Handschrift: Bleistift, lateinische Kurrent}\toendnotes[C]{\smallbreak}
\pstart
           \raggedleft{}{\pb}12. X. 919\pend
           
\pstart{}lieber Guſtav,\pend\vspace{0.5em}
\pstart
           aus alter Gewohnheit – nicht aus Zudringlichkeit – schick ich Ihnen diese \label{K_L04159-1v}\edtext{Sitze}{\lemma{\textnormal{\emph{Sitze}}}\Cendnote{\textnormal{Es dürfte sich um Karten für die Premiere von \emph{Freiwild}\pwindex{Schnitzler, Arthur 15. 5. 1862 Wien – 21. 10. 1931 ebd.@\textsc{Schnitzler, Arthur} (15. 5. 1862 Wien – 21. 10. 1931 ebd.), \emph{Schriftsteller, Mediziner}!Freiwild. Schauspiel in 3 Akten@\strich\emph{Freiwild. Schauspiel in 3 Akten}|pwk}\eventindex{Wiener Stadttheater [8. Wiener Gemeindebezirk]@\textbf{Wiener Stadttheater [8. Wiener Gemeindebezirk]}!Premiere von Freiwild, 14.10.1919@Premiere von Freiwild, 14.10.1919|pwk} am 14. 10. 1919 im Wiener Stadttheater\oindex{Wien@\textbf{Wien}!VIII., Josefstadt@\textbf{VIII., Josefstadt}!Wiener Stadttheater [8. Wiener Gemeindebezirk]@\textbf{Wiener Stadttheater [8. Wiener Gemeindebezirk]}, \emph{Theater}|pwk} gehandelt haben.}}}\label{K_L04159-1}. Aber es ist
               vielleicht etwas wärmer im Theater\oindex{Wien@\textbf{Wien}!VIII., Josefstadt@\textbf{VIII., Josefstadt}!Wiener Stadttheater [8. Wiener Gemeindebezirk]@\textbf{Wiener Stadttheater [8. Wiener Gemeindebezirk]}, \emph{Theater}|pwuv} als zu Haus\oindex{Wien@\textbf{Wien}!I., Innere Stadt@\textbf{I., Innere Stadt}!Tiefer Graben 17@\textbf{Tiefer Graben 17}, \emph{Wohngebäude}|pwv}\pend
           
\pstart
           Herzlichst{\\[\baselineskip]} Ihr{\\[\baselineskip]}\spacefill\mbox{A.}\pend
           \leftskip=0em{}\selectlanguage{ngerman}\endnumbering\briefempfaengerindex{Schwarzkopf, Gustav@\textsc{Schwarzkopf, Gustav}!zzzSchnitzler, Arthur@\emph{von Arthur Schnitzler}!1919-10-121@{12. 10. 1919}|)be}\mylabel{L04159h}
\begin{anhang}
\end{anhang}\newcommand{\dateiname}{L04159}\newcommand{\titel}{Arthur Schnitzler an Gustav Schwarzkopf, 12. 10. 1919}\newcommand{\editorInnen}{Herausgegeben von Jahnke, SelmaMüller, Martin Anton}%% latex-leseansicht-abspann.tex
%% Abspann für die Leseansicht.
%% Der Schalter \ifkorrekturansicht ist bereits durch den Vorspann gesetzt.

%% latex-abspann.tex
%% Gemeinsamer Abspann für Korrekturansicht und Leseansicht.
%% Setzt den Schalter \ifkorrekturansicht voraus (gesetzt in den
%% einbindenden Dateien latex-korrekturansicht-abspann.tex bzw.
%% latex-leseansicht-abspann.tex).
%% ---------------------------------------------------------------

\normalsize

% Das esempio-Environment wird nur in der Leseansicht benötigt
\ifkorrekturansicht\else
\newenvironment{esempio}[3]%
{
    \vspace{1.5ex}
    \rlap{\underline{#1}}
    \par
    \setlength{\parindent}{0cm}
    \nopagebreak
    \leftskip=#2cm
    \rightskip=#3cm
}
{
    \par
}
\fi

\doendnotes{C}
\bigskip
\vfill

\clearpage

\footnotesize

\ifkorrekturansicht
  \lohead{\textsc{register}}
\fi

% theindex-Environment neu definieren ohne reledmac
\makeatletter
\renewenvironment{theindex}{%
  \ifkorrekturansicht
    \section*{\indexname}%
  \else
    \subsubsection*{Index der erwähnten Entitäten}%
  \fi
  \setlength{\parindent}{0pt}%
  \setlength{\parskip}{0pt plus 0.3pt}%
  \let\item\@idxitem
}{%
  \ifkorrekturansicht\clearpage\fi
}
\makeatother

\IfFileExists{\jobname-pw.ind}{\input{\jobname-pw.ind}}{}

% Quellenangabe nur in der Leseansicht
\ifkorrekturansicht\else
% Fallback-Definitionen, falls die .tex-Datei \titel etc. nicht gesetzt hat
\providecommand{\titel}{}
\providecommand{\editorInnen}{}
\providecommand{\dateiname}{\jobname}

\vspace{3cm}

\vfill

\footnotesize
\textsc{Quelle}: \titel. Herausgegeben von {\editorInnen}. In: \emph{Arthur Schnitzler: Briefwechsel mit Autorinnen und Autoren}.
 Digitale Edition, https://schnitzler-briefe.acdh.oeaw.ac.at/{\dateiname}.html (Stand \today)
\fi

\end{document}


