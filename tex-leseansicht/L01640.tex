%% latex-korrekturansicht-vorspann.tex
%% Vorspann für die Korrekturansicht.
%% Lädt die gemeinsame Datei latex-vorspann.tex mit gesetztem Schalter.

\newif\ifkorrekturansicht
\korrekturansichttrue

\input{../tex-inputs/latex-vorspann}


\section[Albert Ehrenstein an Arthur Schnitzler, 30. 11. 1906]{L01640 Albert Ehrenstein an Arthur Schnitzler, 30. 11. 1906}
\nopagebreak\mylabel{L01640v}
\rehead{ }\normalsize\beginnumbering\briefempfaengerindex{Schnitzler, Arthur@\textsc{Schnitzler, Arthur}!zzzEhrenstein, Albert@\emph{von Albert Ehrenstein}!1906-11-301@{30. 11. 1906}|(be}
\toendnotes[C]{\smallbreak\pagebreak[2]}\Standort{CUL, Schnitzler, B 30.}
\physDesc{Brief, 1 Blatt, 2 Seiten, 993 Zeichen
\newline{}Handschrift: schwarze Tinte, deutsche Kurrent
\newline{}Schnitzler: Beschriftung »Ehrenste\textcolor{gray}{in}« }
\buchAbdrucke{\weitereDrucke{Albert Ehrenstein: \emph{Briefe}. München: \emph{Boer} 1989, S. 20.} }\toendnotes[C]{\smallbreak}
\pstart
           \raggedleft{}{\pb}Wien\oindex{Wien@\textbf{Wien}, \emph{A.ADM2}|pw}, den 30. Nov. 06\pend
           
\pstart{}Sehr geehrter Herr Doktor.\pend\vspace{0.5em}
\pstart
           Ihre außerordentliche Geduld, ſehr geehrter Herr Doktor, hoffe ich nicht auf eine
               allzuharte Probe geſtellt zu haben, wenn ich höflichſt bitte, meine etwas
               dilettantiſche Übertragung des euripideiſchen\pwindex{Euripides 485? – 406? v. u. Z.@\textsc{Euripides} (485? – 406? v. u. Z.), \emph{Schriftsteller/Schriftstellerin}|pw}{ }\label{K_L01640-1v}\edtext{Librettos\pwindex{Helena@\emph{Helena}|pwv}}{\lemma{\textnormal{\emph{Librettos}}}\Cendnote{\textnormal{Die Bearbeitung von \emph{Helena}\pwindex{Helena@\emph{Helena}|pwk} ist nicht erhalten.}}}\label{K_L01640-1} einiger Lektüre zu
               unterziehen. Sollte dies aber doch der Fall ſein, ſo möchte ich Euer Hochwohlgeboren
               ergebenſt erſuchen, beachten zu wollen, daß ich nicht daran denke, die Arbeit etwa in
               dieſer Form {\pb}irgendwie bekannt zu machen,
               ſondern falls ſich überhaupt das Sujet zu einer Veröffentlichung eignen ſollte, würde
               ich von den 2000 Verſen des Euripides\pwindex{Euripides 485? – 406? v. u. Z.@\textsc{Euripides} (485? – 406? v. u. Z.), \emph{Schriftsteller/Schriftstellerin}|pw} und
               meiner Überſetzung etwa 1000 weglaſſen, die vier Akte in zwei oder einen
               zuſammenziehen, was mir bei der Fülle entbehrlicher Chorlieder, bei dem Überfluſſe an
               Wiederholungen und unnützen Längen des Dialoges nicht ſchwer fiele. Indem ich Sie,
               ſehr verehrter Herr Doktor, bitte, mir dieſe Arbeit nicht übelzunehmen, verbleibe ich
               hochachtungsvoll\pend
           
\pstart
           Ihr Sie verehrender{\\[\baselineskip]}\spacefill\mbox{Albert Ehrenstein.}\pend
           \leftskip=0em{}\selectlanguage{ngerman}\endnumbering\briefempfaengerindex{Schnitzler, Arthur@\textsc{Schnitzler, Arthur}!zzzEhrenstein, Albert@\emph{von Albert Ehrenstein}!1906-11-301@{30. 11. 1906}|)be}\mylabel{L01640h}  \normalsize

\doendnotes{C}
\bigskip
\vfill

\clearpage

\footnotesize

\lohead{\textsc{register}}

% Definiere theindex-Environment komplett neu ohne reledmac
\makeatletter
\renewenvironment{theindex}{%
  \section*{\indexname}%
  \setlength{\parindent}{0pt}%
  \setlength{\parskip}{0pt plus 0.3pt}%
  \let\item\@idxitem
}{%
  \clearpage
}
\makeatother

\IfFileExists{\jobname-pw.ind}{\input{\jobname-pw.ind}}{}

\end{document}

      