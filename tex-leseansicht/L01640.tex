%% latex-leseansicht-vorspann.tex
%% Vorspann für die Leseansicht.
%% Lädt die gemeinsame Datei latex-vorspann.tex mit nicht gesetztem Schalter.

\newif\ifkorrekturansicht
\korrekturansichtfalse

\input{../tex-inputs/latex-vorspann}


\section[Albert Ehrenstein an Arthur Schnitzler, 30. 11. 1906]{L01640 Albert Ehrenstein an Arthur Schnitzler, 30. 11. 1906}
\nopagebreak\mylabel{L01640v}
\rehead{ }\normalsize\beginnumbering\briefempfaengerindex{Schnitzler, Arthur@\textsc{Schnitzler, Arthur}!zzzEhrenstein, Albert@\emph{von Albert Ehrenstein}!1906-11-301@{30. 11. 1906}|(be}
\toendnotes[C]{\smallbreak\pagebreak[2]}
\correspDesc{Versand  durch Albert Ehrenstein am 30. 11. 1906 in Wien
\newline{}Erhalt  durch Arthur Schnitzler im Zeitraum [30. 11. 1906 – 4. 12. 1906?] in Wien}\toendnotes[C]{\smallbreak}
\Standort{CUL, Schnitzler, B 30.}
\physDesc{Brief, 1 Blatt, 2 Seiten, 993 Zeichen
\newline{}Handschrift: schwarze Tinte, deutsche Kurrent
\newline{}Schnitzler: Beschriftung »Ehrenste\textcolor{gray}{in}« }
\buchAbdrucke{\weitereDrucke{Albert Ehrenstein: \emph{Briefe}. Herausgegeben von Hanni Mittelmann. München: \emph{Boer} 1989, S. 20 (Werke, 1).} }\toendnotes[C]{\smallbreak}
\pstart
           \raggedleft{}{\pb}Wien\oindex{Wien@\textbf{Wien}, \emph{Verwaltungsgebiet}|pw}, den 30. Nov. 06\pend
           
\pstart{}Sehr geehrter Herr Doktor.\pend\vspace{0.5em}
\pstart
           Ihre außerordentliche Geduld,{ }ſehr geehrter Herr Doktor, hoffe ich nicht auf eine
               allzuharte Probe geſtellt zu haben, wenn ich höflichſt bitte, meine etwas
               dilettantiſche Übertragung des euripideiſchen\pwindex{Euripides 485? Salamina – 406? v.\,u.\,Z. Pella@\textsc{Euripides} (485? Salamina – 406? v.\,u.\,Z. Pella), \emph{Schriftsteller}|pw}{ }\label{K_L01640-1v}\edtext{Librettos\pwindex{Euripides 485? Salamina – 406? v.\,u.\,Z. Pella@\textsc{Euripides} (485? Salamina – 406? v.\,u.\,Z. Pella), \emph{Schriftsteller}!Helena@\strich\emph{Helena}|pwv}}{\lemma{\textnormal{\emph{Librettos}}}\Cendnote{\textnormal{Die Bearbeitung von \emph{Helena}\pwindex{Euripides 485? Salamina – 406? v.\,u.\,Z. Pella@\textsc{Euripides} (485? Salamina – 406? v.\,u.\,Z. Pella), \emph{Schriftsteller}!Helena@\strich\emph{Helena}|pwk} ist nicht erhalten.}}}\label{K_L01640-1} einiger Lektüre zu
               unterziehen. Sollte dies aber doch der Fall{ }ſein,{ }ſo möchte ich Euer Hochwohlgeboren
               ergebenſt erſuchen, beachten zu wollen, daß ich nicht daran denke, die Arbeit etwa in
               dieſer Form {\pb}irgendwie bekannt zu machen,{ }ſondern falls{ }ſich überhaupt das Sujet zu einer Veröffentlichung eignen{ }ſollte, würde
               ich von den 2000 Verſen des Euripides\pwindex{Euripides 485? Salamina – 406? v.\,u.\,Z. Pella@\textsc{Euripides} (485? Salamina – 406? v.\,u.\,Z. Pella), \emph{Schriftsteller}|pw} und
               meiner Überſetzung etwa 1000 weglaſſen, die vier Akte in zwei oder einen
               zuſammenziehen, was mir bei der Fülle entbehrlicher Chorlieder, bei dem Überfluſſe an
               Wiederholungen und unnützen Längen des Dialoges nicht{ }ſchwer fiele. Indem ich Sie,{ }ſehr verehrter Herr Doktor, bitte, mir dieſe Arbeit nicht übelzunehmen, verbleibe ich
               hochachtungsvoll\pend
           
\pstart
           Ihr Sie verehrender{\\[\baselineskip]}\spacefill\mbox{Albert Ehrenstein.}\pend
           \leftskip=0em{}\selectlanguage{ngerman}\endnumbering\briefempfaengerindex{Schnitzler, Arthur@\textsc{Schnitzler, Arthur}!zzzEhrenstein, Albert@\emph{von Albert Ehrenstein}!1906-11-301@{30. 11. 1906}|)be}\mylabel{L01640h}  \newcommand{\dateiname}{L01640}\newcommand{\titel}{Albert Ehrenstein an Arthur Schnitzler, 30. 11. 1906}\newcommand{\editorInnen}{Martin Anton Müller und Gerd-Hermann Susen}%% latex-leseansicht-abspann.tex
%% Abspann für die Leseansicht.
%% Der Schalter \ifkorrekturansicht ist bereits durch den Vorspann gesetzt.

%% latex-abspann.tex
%% Gemeinsamer Abspann für Korrekturansicht und Leseansicht.
%% Setzt den Schalter \ifkorrekturansicht voraus (gesetzt in den
%% einbindenden Dateien latex-korrekturansicht-abspann.tex bzw.
%% latex-leseansicht-abspann.tex).
%% ---------------------------------------------------------------

\normalsize

% Das esempio-Environment wird nur in der Leseansicht benötigt
\ifkorrekturansicht\else
\newenvironment{esempio}[3]%
{
    \vspace{1.5ex}
    \rlap{\underline{#1}}
    \par
    \setlength{\parindent}{0cm}
    \nopagebreak
    \leftskip=#2cm
    \rightskip=#3cm
}
{
    \par
}
\fi

\doendnotes{C}
\bigskip
\vfill

\clearpage

\footnotesize

\ifkorrekturansicht
  \lohead{\textsc{register}}
\fi

% theindex-Environment neu definieren ohne reledmac
\makeatletter
\renewenvironment{theindex}{%
  \ifkorrekturansicht
    \section*{\indexname}%
  \else
    \subsubsection*{Index der erwähnten Entitäten}%
  \fi
  \setlength{\parindent}{0pt}%
  \setlength{\parskip}{0pt plus 0.3pt}%
  \let\item\@idxitem
}{%
  \ifkorrekturansicht\clearpage\fi
}
\makeatother

\IfFileExists{\jobname-pw.ind}{\input{\jobname-pw.ind}}{}

% Quellenangabe nur in der Leseansicht
\ifkorrekturansicht\else
% Fallback-Definitionen, falls die .tex-Datei \titel etc. nicht gesetzt hat
\providecommand{\titel}{}
\providecommand{\editorInnen}{}
\providecommand{\dateiname}{\jobname}

\vspace{3cm}

\vfill

\footnotesize
\textsc{Quelle}: \titel. Herausgegeben von {\editorInnen}. In: \emph{Arthur Schnitzler: Briefwechsel mit Autorinnen und Autoren}.
 Digitale Edition, https://schnitzler-briefe.acdh.oeaw.ac.at/{\dateiname}.html (Stand \today)
\fi

\end{document}


