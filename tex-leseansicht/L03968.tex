%% latex-leseansicht-vorspann.tex
%% Vorspann für die Leseansicht.
%% Lädt die gemeinsame Datei latex-vorspann.tex mit nicht gesetztem Schalter.

\newif\ifkorrekturansicht
\korrekturansichtfalse

\input{../tex-inputs/latex-vorspann}


\section[Arthur Schnitzler an Berta Zuckerkandl, 4. 2. 1926]{L03968 Arthur Schnitzler an Berta Zuckerkandl, 4. 2. 1926}
\nopagebreak\mylabel{L03968v}
\rehead{ }\normalsize\beginnumbering\briefempfaengerindex{Zuckerkandl, Berta@\textsc{Zuckerkandl, Berta}!zzzSchnitzler, Arthur@\emph{von Arthur Schnitzler}!1926-02-041@{4. 2. 1926}|(be}
\toendnotes[C]{\smallbreak\pagebreak[2]}
\correspDesc{Versand  durch Arthur Schnitzler am 4. 2. 1926 in Wien
\newline{}Erhalt  durch Berta Zuckerkandl im Zeitraum [4. 2. 1926 – 7. 2. 1926?] in Wien}\toendnotes[C]{\smallbreak}
\Standort{DLA, HS.1985.1.2282.}
\physDesc{Brief, Durchschlag, 1 Blatt, 1 Seite, 1069 Zeichen
\newline{}Schreibmaschine
\newline{}Handschrift: roter Buntstift, lateinische Kurrent (\noindent{}beschriftet: »\uline{Zuckerkandl}« und »Fkr«, sechs Unterstreichungen)}\toendnotes[C]{\smallbreak}
\pstart
           \raggedleft{}{\pb}4. 2. 1926.\pend
           
\pstart{}Verehrte und liebe Frau Hofrätin.\pend\vspace{0.5em}
\pstart
           Leider kann ich Ihre Ankunft in
               Wien\oindex{Wien@\textbf{Wien}, \emph{Verwaltungsgebiet}|pw}{ }\label{K_L03968-1v}\edtext{nicht mehr abwarten}{\lemma{\textnormal{\emph{nicht mehr abwarten}}}\Cendnote{\textnormal{Schnitzler hielt sich vom 6. 2. 1926 bis zum 12. 2. 1926 in Berlin\oindex{Berlin@\textbf{Berlin}, \emph{Hauptstadt}|pwk} auf.}}}\label{K_L03968-1}, so mögen Sie denn
               diesen Gruss und Dank in Ihrer Wohnung\oindex{Wien@\textbf{Wien}!I., Innere Stadt@\textbf{I., Innere Stadt}!Oppolzergasse 6@\textbf{Oppolzergasse 6}, \emph{Wohngebäude}|pwv} vorfinden. Eben habe ich \label{K_L03968-2v}\edtext{an Delamain\pwindex{Delamain, Maurice 28.\,4.\,1883 Jarnac – 2.\,5.\,1974 Paris@\textsc{Delamain, Maurice} (28.\,4.\,1883 Jarnac – 2.\,5.\,1974 Paris), \emph{Kritiker, Rechtsanwalt, Verleger}|pw}}{\lemma{\textnormal{\emph{an Delamain}}}\Cendnote{\textnormal{Arthur Schnitzler an Maurice Delamain\pwindex{Delamain, Maurice 28.\,4.\,1883 Jarnac – 2.\,5.\,1974 Paris@\textsc{Delamain, Maurice} (28.\,4.\,1883 Jarnac – 2.\,5.\,1974 Paris), \emph{Kritiker, Rechtsanwalt, Verleger}|pwk}, 4. 2. 1926, \emph{Deutsches Literaturarchiv
                        Marbach}, HS.1985.1.562,2.
               }}}\label{K_L03968-2} geschrieben, dass ich seine Vorschläge annehme und
      habe noch, Ihrem guten Ratschlag folgend,
      einiges »Liebenswürdige« beigefügt. Dass
      die Uebersetzung\pwindex{Schnitzler, Arthur 15. 5. 1862 Wien – 21. 10. 1931 ebd.@\textsc{Schnitzler, Arthur} (15. 5. 1862 Wien – 21. 10. 1931 ebd.), \emph{Schriftsteller, Mediziner}!Madmoiselle Else@\strich\emph{Madmoiselle Else}|pw} von Frau Pollaczek\pwindex{Pollaczek, Clara Katharina 15.\,1.\,1875 Wien – 22.\,7.\,1951 ebd.@\textsc{Pollaczek, Clara Katharina} (15.\,1.\,1875 Wien – 22.\,7.\,1951 ebd.), \emph{Schriftstellerin}|pw} solche
      Zustimmung findet ist in jeder Hinsicht erfreulich und lässt weitere Versuche in dieser Richtung in Erwägung ziehen. Es ist die
      Frage, ob die Uebersetzerin\pwindex{Pollaczek, Clara Katharina 15.\,1.\,1875 Wien – 22.\,7.\,1951 ebd.@\textsc{Pollaczek, Clara Katharina} (15.\,1.\,1875 Wien – 22.\,7.\,1951 ebd.), \emph{Schriftstellerin}|pwv} bei Erscheinen
               des Buches\pwindex{Schnitzler, Arthur 15. 5. 1862 Wien – 21. 10. 1931 ebd.@\textsc{Schnitzler, Arthur} (15. 5. 1862 Wien – 21. 10. 1931 ebd.), \emph{Schriftsteller, Mediziner}!Madmoiselle Else@\strich\emph{Madmoiselle Else}|pwv} gleich mit ihrem ganzen Namen
               genannt werden soll oder ob es aus mancherlei Gründen sich empfiehlt nur mit den Initialen oder mit einem Pseudonym hervorzutreten. Ueber diese Fragen und allerlei anderes wird sich Frau P.\pwindex{Pollaczek, Clara Katharina 15.\,1.\,1875 Wien – 22.\,7.\,1951 ebd.@\textsc{Pollaczek, Clara Katharina} (15.\,1.\,1875 Wien – 22.\,7.\,1951 ebd.), \emph{Schriftstellerin}|pw}, wenn Sie erlauben,
      persönlich mit Ihnen in Verbindung setzen.\pend
           
\pstart
           Meine Adresse in Berlin\oindex{Berlin@\textbf{Berlin}, \emph{Hauptstadt}|pw} ist Hotel Esplanade\oindex{Hotel Esplanade [Berlin]@\textbf{Hotel Esplanade [Berlin]}, \emph{Hotel}|pw}. Mein Aufenthalt dort auf zirka
      8 Tage präliminiert. Am Sonntag findet \label{K_L03968-3v}\edtext{die
               Vorlesung\eventindex{Reichstag@\textbf{Reichstag}!Lesung von Lieutenant Gustl, Fräulein Else, 7.2.1926@Lesung von Lieutenant Gustl, Fräulein Else, 7.2.1926|pwv}}{\lemma{\textnormal{\emph{die
               Vorlesung}}}\Cendnote{\textnormal{
                  Die Lesung von \emph{Lieutenant Gustl. Novelle}\pwindex{Schnitzler, Arthur 15. 5. 1862 Wien – 21. 10. 1931 ebd.@\textsc{Schnitzler, Arthur} (15. 5. 1862 Wien – 21. 10. 1931 ebd.), \emph{Schriftsteller, Mediziner}!Lieutenant Gustl. Novelle@\strich\emph{Lieutenant Gustl. Novelle}|pwk} durch Arthur Schnitzler und \emph{Fräulein Else}\pwindex{Schnitzler, Arthur 15. 5. 1862 Wien – 21. 10. 1931 ebd.@\textsc{Schnitzler, Arthur} (15. 5. 1862 Wien – 21. 10. 1931 ebd.), \emph{Schriftsteller, Mediziner}!Fräulein Else@\strich\emph{Fräulein Else}|pwk} durch Elisabeth Bergner\pwindex{Bergner, Elisabeth 22.\,8.\,1897 Drohobych – 12.\,5.\,1986 London@\textsc{Bergner, Elisabeth} (22.\,8.\,1897 Drohobych – 12.\,5.\,1986 London), \emph{Schauspielerin}|pwk}\eventindex{Reichstag@\textbf{Reichstag}!Lesung von Lieutenant Gustl, Fräulein Else, 7.2.1926@Lesung von Lieutenant Gustl, Fräulein Else, 7.2.1926|pwk} fand am 7. 2. 1926 im Berliner\oindex{Berlin@\textbf{Berlin}, \emph{Hauptstadt}|pwk}{ }Reichstag\oindex{Reichstag@\textbf{Reichstag}, \emph{Regierungsgebäude}|pwk} statt.}}}\label{K_L03968-3} im Reichstag\oindex{Reichstag@\textbf{Reichstag}, \emph{Regierungsgebäude}|pw} statt.\pend
           
\pstart
           Auf ein baldiges Wiedersehen und herzlichste Grüsse{\\[\baselineskip]} Ihres sehr ergebenen\pend
           \leftskip=0em{}{\vspace{1\baselineskip}}
\pstart
           \noindent{}Frau Hofrätin Berta Zuckerkandl,{\\}Wien\oindex{Wien@\textbf{Wien}, \emph{Verwaltungsgebiet}|pw}.\pend
           \selectlanguage{ngerman}\endnumbering\briefempfaengerindex{Zuckerkandl, Berta@\textsc{Zuckerkandl, Berta}!zzzSchnitzler, Arthur@\emph{von Arthur Schnitzler}!1926-02-041@{4. 2. 1926}|)be}\mylabel{L03968h}
\begin{anhang}
\end{anhang}\newcommand{\dateiname}{L03968}\newcommand{\titel}{Arthur Schnitzler an Berta Zuckerkandl, 4. 2. 1926}\newcommand{\editorInnen}{Herausgegeben von Jahnke, SelmaMüller, Martin Anton}%% latex-leseansicht-abspann.tex
%% Abspann für die Leseansicht.
%% Der Schalter \ifkorrekturansicht ist bereits durch den Vorspann gesetzt.

%% latex-abspann.tex
%% Gemeinsamer Abspann für Korrekturansicht und Leseansicht.
%% Setzt den Schalter \ifkorrekturansicht voraus (gesetzt in den
%% einbindenden Dateien latex-korrekturansicht-abspann.tex bzw.
%% latex-leseansicht-abspann.tex).
%% ---------------------------------------------------------------

\normalsize

% Das esempio-Environment wird nur in der Leseansicht benötigt
\ifkorrekturansicht\else
\newenvironment{esempio}[3]%
{
    \vspace{1.5ex}
    \rlap{\underline{#1}}
    \par
    \setlength{\parindent}{0cm}
    \nopagebreak
    \leftskip=#2cm
    \rightskip=#3cm
}
{
    \par
}
\fi

\doendnotes{C}
\bigskip
\vfill

\clearpage

\footnotesize

\ifkorrekturansicht
  \lohead{\textsc{register}}
\fi

% theindex-Environment neu definieren ohne reledmac
\makeatletter
\renewenvironment{theindex}{%
  \ifkorrekturansicht
    \section*{\indexname}%
  \else
    \subsubsection*{Index der erwähnten Entitäten}%
  \fi
  \setlength{\parindent}{0pt}%
  \setlength{\parskip}{0pt plus 0.3pt}%
  \let\item\@idxitem
}{%
  \ifkorrekturansicht\clearpage\fi
}
\makeatother

\IfFileExists{\jobname-pw.ind}{\input{\jobname-pw.ind}}{}

% Quellenangabe nur in der Leseansicht
\ifkorrekturansicht\else
% Fallback-Definitionen, falls die .tex-Datei \titel etc. nicht gesetzt hat
\providecommand{\titel}{}
\providecommand{\editorInnen}{}
\providecommand{\dateiname}{\jobname}

\vspace{3cm}

\vfill

\footnotesize
\textsc{Quelle}: \titel. Herausgegeben von {\editorInnen}. In: \emph{Arthur Schnitzler: Briefwechsel mit Autorinnen und Autoren}.
 Digitale Edition, https://schnitzler-briefe.acdh.oeaw.ac.at/{\dateiname}.html (Stand \today)
\fi

\end{document}


