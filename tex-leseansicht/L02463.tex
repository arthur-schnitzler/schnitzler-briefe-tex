%% latex-leseansicht-vorspann.tex
%% Vorspann für die Leseansicht.
%% Lädt die gemeinsame Datei latex-vorspann.tex mit nicht gesetztem Schalter.

\newif\ifkorrekturansicht
\korrekturansichtfalse

\input{../tex-inputs/latex-vorspann}


\section[Gertrud Rung an Arthur Schnitzler, 30. 12. 1925]{L02463 Gertrud Rung an Arthur Schnitzler, 30. 12. 1925}
\nopagebreak\mylabel{L02463v}
\rehead{ }\normalsize\beginnumbering\briefempfaengerindex{Schnitzler, Arthur@\textsc{Schnitzler, Arthur}!zzzRung, Gertrud@\emph{von Gertrud Rung}!1925-12-302@{30. 12. 1925}|(be}
\toendnotes[C]{\smallbreak\pagebreak[2]}
\correspDesc{Versand  durch Gertrud Rung am 30. 12. 1925 in Kopenhagen
\newline{}Erhalt  durch Arthur Schnitzler am 2. I. 26 in Wien}\toendnotes[C]{\smallbreak}
\Standort{CUL, Schnitzler, B 17.}
\physDesc{Bildpostkarte, 285 Zeichen
\newline{}Handschrift: schwarze Tinte, lateinische Kurrent
\newline{}Versand: 1) Stempel: »\nobreak{}\oindex{Kopenhagen@\textbf{Kopenhagen}, \emph{Hauptstadt}|pwk}Københaven, 925\nobreak{}«.   2) Stempel: »\nobreak{}\oindex{Wien@\textbf{Wien}, \emph{Verwaltungsgebiet}|pwk}\textcolor{gray}{Wien} 65, 2. I. 26, \textcolor{gray}{V}III\nobreak{}«.  3) mit blauem Buntstift von unbekannter Hand die Bezirksangabe der
                                 Adresse zu »XVIII« richtiggestellt
\newline{}Schnitzler: mit rotem Buntstift beschriftet: »\textsc{Brandes}« 
\newline{}Ordnung: mit Bleistift von unbekannter Hand nummeriert:
                                    »64« }
\buchAbdrucke{\weitereDrucke{Georg Brandes, Arthur Schnitzler: \emph{Ein Briefwechsel}. Herausgegeben von Kurt Bergel. Bern: \emph{Francke} 1956, S. 151.} }\pstart{}{\pb}Herrn
                  Schriftsteller\pend{}\pstart{}Dr. Arthur Schnitzler\pend{}\pstart{}Sternwartestraße 71\oindex{Wien@\textbf{Wien}!XVIII., Währing@\textbf{XVIII., Währing}!Sternwartestraße 71@\textbf{Sternwartestraße 71}, \emph{Wohngebäude}|pw}\pend{}\pstart{}Wien VIII\oindex{VIII., Josefstadt@\textbf{VIII., Josefstadt}, \emph{Verwaltungsgebiet}|pw}\pend{}\pstart{}Oesterreich\oindex{Österreich@\textbf{Österreich}|pw}\pend{}{\bigskip}
\pstart
           \noindent{}\centering{}\textcolor{gray}{\textbf{{\pb}København. Marmorbroen\oindex{Marmorbroen@\textbf{Marmorbroen}, \emph{Brücke}|pw}.}}\pend
           \vspace{1em}
\pstart
           {\pb}30–12–25\pend
           
\pstart{}Hochverehrter Herr Doktor Schnitzler\pend\vspace{0.5em}
\pstart
           In dankbarer Erinnerung schöner Stunden in Ihrem gastfreundlichen Hause sende ich
               Ihnen die herzlichsten Glückwünsche für das neue Jahr.\pend
           
\pstart
           Ihre ergebene{\\[\baselineskip]}\spacefill\mbox{Gertrud Rung}\pend
           \leftskip=0em{}\selectlanguage{ngerman}\endnumbering\briefempfaengerindex{Schnitzler, Arthur@\textsc{Schnitzler, Arthur}!zzzRung, Gertrud@\emph{von Gertrud Rung}!1925-12-302@{30. 12. 1925}|)be}\mylabel{L02463h}  \newcommand{\dateiname}{L02463}\newcommand{\titel}{Gertrud Rung an Arthur Schnitzler, 30. 12. 1925}\newcommand{\editorInnen}{Martin Anton Müller und Gerd-Hermann Susen}%% latex-leseansicht-abspann.tex
%% Abspann für die Leseansicht.
%% Der Schalter \ifkorrekturansicht ist bereits durch den Vorspann gesetzt.

%% latex-abspann.tex
%% Gemeinsamer Abspann für Korrekturansicht und Leseansicht.
%% Setzt den Schalter \ifkorrekturansicht voraus (gesetzt in den
%% einbindenden Dateien latex-korrekturansicht-abspann.tex bzw.
%% latex-leseansicht-abspann.tex).
%% ---------------------------------------------------------------

\normalsize

% Das esempio-Environment wird nur in der Leseansicht benötigt
\ifkorrekturansicht\else
\newenvironment{esempio}[3]%
{
    \vspace{1.5ex}
    \rlap{\underline{#1}}
    \par
    \setlength{\parindent}{0cm}
    \nopagebreak
    \leftskip=#2cm
    \rightskip=#3cm
}
{
    \par
}
\fi

\doendnotes{C}
\bigskip
\vfill

\clearpage

\footnotesize

\ifkorrekturansicht
  \lohead{\textsc{register}}
\fi

% theindex-Environment neu definieren ohne reledmac
\makeatletter
\renewenvironment{theindex}{%
  \ifkorrekturansicht
    \section*{\indexname}%
  \else
    \subsubsection*{Index der erwähnten Entitäten}%
  \fi
  \setlength{\parindent}{0pt}%
  \setlength{\parskip}{0pt plus 0.3pt}%
  \let\item\@idxitem
}{%
  \ifkorrekturansicht\clearpage\fi
}
\makeatother

\IfFileExists{\jobname-pw.ind}{\input{\jobname-pw.ind}}{}

% Quellenangabe nur in der Leseansicht
\ifkorrekturansicht\else
% Fallback-Definitionen, falls die .tex-Datei \titel etc. nicht gesetzt hat
\providecommand{\titel}{}
\providecommand{\editorInnen}{}
\providecommand{\dateiname}{\jobname}

\vspace{3cm}

\vfill

\footnotesize
\textsc{Quelle}: \titel. Herausgegeben von {\editorInnen}. In: \emph{Arthur Schnitzler: Briefwechsel mit Autorinnen und Autoren}.
 Digitale Edition, https://schnitzler-briefe.acdh.oeaw.ac.at/{\dateiname}.html (Stand \today)
\fi

\end{document}


