%% latex-korrekturansicht-vorspann.tex
%% Vorspann für die Korrekturansicht.
%% Lädt die gemeinsame Datei latex-vorspann.tex mit gesetztem Schalter.

\newif\ifkorrekturansicht
\korrekturansichttrue

\input{../tex-inputs/latex-vorspann}


\section[Gertrud Rung an Arthur Schnitzler, 30. 12. 1925]{L02463 Gertrud Rung an Arthur Schnitzler, 30. 12. 1925}
\nopagebreak\mylabel{L02463v}
\rehead{ }\normalsize\beginnumbering\briefempfaengerindex{Schnitzler, Arthur@\textsc{Schnitzler, Arthur}!zzzRung, Gertrud@\emph{von Gertrud Rung}!1925-12-302@{30. 12. 1925}|(be}
\toendnotes[C]{\smallbreak\pagebreak[2]}\Standort{CUL, Schnitzler, B 17.}
\physDesc{Bildpostkarte, 285 Zeichen
\newline{}Handschrift: schwarze Tinte, lateinische Kurrent
\newline{}Versand: 1) Stempel: »\nobreak{}\oindex{Kopenhagen@\textbf{Kopenhagen}, \emph{P.PPLC}|pwk}Københaven, 925\nobreak{}«.   2) Stempel: »\nobreak{}\textcolor{gray}{Wien} 65, 2. I. 26, \textcolor{gray}{V}III\nobreak{}«.  3) mit blauem Buntstift von unbekannter Hand die Bezirksangabe der
                                 Adresse zu »XVIII« richtiggestellt
\newline{}Schnitzler: mit rotem Buntstift beschriftet: »\textsc{Brandes}« 
\newline{}Ordnung: mit Bleistift von unbekannter Hand nummeriert:
                                    »64« }
\buchAbdrucke{\weitereDrucke{Georg Brandes, Arthur Schnitzler: \emph{Ein Briefwechsel}. Bern: \emph{Francke} 1956, S. 151.} }\pstart{}{\pb}Herrn
                  Schriftsteller\pend{}\pstart{}Dr. Arthur Schnitzler\pend{}\pstart{}Sternwartestraße 71\oindex{Sternwartestrasse 71@\textbf{Sternwartestraße 71}, \emph{Wohngebäude (K.WHS)}|pw}\pend{}\pstart{}Wien VIII\oindex{VIII., Josefstadt@\textbf{VIII., Josefstadt}, \emph{A.ADM3}|pw}\pend{}\pstart{}Oesterreich\oindex{Oesterreich@\textbf{Österreich}, \emph{A.PCLI}|pw}\pend{}{\bigskip}
\pstart
           \noindent{}\centering{}\textcolor{gray}{\textbf{{\pb}København. Marmorbroen\oindex{Marmorbroen@\textbf{Marmorbroen}, \emph{Brücke (K.BRK)}|pw}.}}\pend
           \vspace{1em}
\pstart
           {\pb}30–12–25\pend
           
\pstart{}Hochverehrter Herr Doktor Schnitzler\pend\vspace{0.5em}
\pstart
           In dankbarer Erinnerung schöner Stunden in Ihrem gastfreundlichen Hause sende ich
               Ihnen die herzlichsten Glückwünsche für das neue Jahr.\pend
           
\pstart
           Ihre ergebene{\\[\baselineskip]}\spacefill\mbox{Gertrud Rung}\pend
           \leftskip=0em{}\selectlanguage{ngerman}\endnumbering\briefempfaengerindex{Schnitzler, Arthur@\textsc{Schnitzler, Arthur}!zzzRung, Gertrud@\emph{von Gertrud Rung}!1925-12-302@{30. 12. 1925}|)be}\mylabel{L02463h}  \normalsize

\doendnotes{C}
\bigskip
\vfill

\clearpage

\footnotesize

\lohead{\textsc{register}}

% Definiere theindex-Environment komplett neu ohne reledmac
\makeatletter
\renewenvironment{theindex}{%
  \section*{\indexname}%
  \setlength{\parindent}{0pt}%
  \setlength{\parskip}{0pt plus 0.3pt}%
  \let\item\@idxitem
}{%
  \clearpage
}
\makeatother

\IfFileExists{\jobname-pw.ind}{\input{\jobname-pw.ind}}{}

\end{document}

      