%% latex-leseansicht-vorspann.tex
%% Vorspann für die Leseansicht.
%% Lädt die gemeinsame Datei latex-vorspann.tex mit nicht gesetztem Schalter.

\newif\ifkorrekturansicht
\korrekturansichtfalse

\input{../tex-inputs/latex-vorspann}

\begin{center}
            \textcolor{red}{ENTWURF, NICHT FERTIG KORRIGIERT}
                      \end{center}
            
         
         \newcommand{\erwaehntePersonen}{Personen: Hugo von Hofmannsthal, Felix Salten}
         \newcommand{\erwaehnteOrte}{Orte: Kaltenleutgeben, Salzburg, Wien}
         \newcommand{\erwaehnteWerke}{Werke: Der Schleier der Beatrice. Schauspiel in fünf Akten, »Der Schleier der Beatrice«. Ein Konflikt mit dem Burgtheater}
               \section[Arthur Schnitzler an Felix Salten, {[}26. 6. 1902{]}]{ Arthur Schnitzler an Felix Salten, {[}26. 6. 1902{]}}\nopagebreak\mylabel{v}\rehead{ }\begin{ledgroupsized}[t]{13cm}\normalsize\beginnumbering \toendnotes[C]{\smallbreak\pagebreak[2]} \Standort{Wienbibliothek im Rathaus, ZPH 1681, 2.1.516.}
\physDesc{
\newline{}Handschrift: , deutsche Kurrent}\toendnotes[C]{\smallbreak}\pstart
           \noindent{}{\pb}lieber Freund, wieder hat mich geſtern – ſchon auf dem Weg, das
               gräßliche Wetter abgehalten Sie in \textsc{Kaltenl.\oindex{Kaltenleutgeben@\textbf{Kaltenleutgeben}|pw}} zu beſuchen. Nun ſeh ich Sie wohl erſt, nach meiner Rückkehr, etwa gegen den
               10. Juli. Ich fahre {\pb}\label{K_L02976-1v}\edtext{morgen}{\lemma{\textnormal{\emph{morgen}}}\Cendnote{\textnormal{Das erlaubt die
               Datierung des undatierten Korrespondenzstücks.}}}\label{K_L02976-1h}{ }Salzburg\oindex{Salzburg@\textbf{Salzburg}|pw}, Hugo\pwindex{Hofmannsthal, Hugo von 1874-02-01 – 1929-07-15@\textsc{Hofmannsthal, Hugo von} (1874-02-01 – 1929-07-15), \emph{Schriftsteller}|pw}
               dürfte übermorgen nachko{\geminationm}en.– Briefe werden mir aus Wien\oindex{Wien@\textbf{Wien}|pw} nachgeſchickt. Die 
                  \textsc{Bea\pwindex{Schnitzler, Arthur 15.05.1862 – 21.10.1931@\textsc{Schnitzler, Arthur} (15.05.1862 – 21.10.1931), \emph{Schriftsteller, Mediziner}!Schleier der Beatrice. Schauspiel in fuenf Akten1900-12-01@\strich\emph{Der Schleier der Beatrice. Schauspiel in fünf Akten} {[}1900-12-01{]}|pw}}.-Sache\pwindex{\textcolor{red}{\textsuperscript{XXXX1 indx}}!Schleier der Beatrice«. Ein Konflikt mit dem Burgtheater1900-09-14@\strich\emph{»Der Schleier der Beatrice«. Ein Konflikt mit dem Burgtheater} {[}1900-09-14{]}|pwv}\pwindex{\textcolor{red}{\textsuperscript{XXXX1 indx}}!Schleier der Beatrice«. Ein Konflikt mit dem Burgtheater1900-09-14@\strich\emph{»Der Schleier der Beatrice«. Ein Konflikt mit dem Burgtheater} {[}1900-09-14{]}|pwv}\pwindex{\textcolor{red}{\textsuperscript{XXXX1 indx}}!Schleier der Beatrice«. Ein Konflikt mit dem Burgtheater1900-09-14@\strich\emph{»Der Schleier der Beatrice«. Ein Konflikt mit dem Burgtheater} {[}1900-09-14{]}|pwv}\pwindex{\textcolor{red}{\textsuperscript{XXXX1 indx}}!Schleier der Beatrice«. Ein Konflikt mit dem Burgtheater1900-09-14@\strich\emph{»Der Schleier der Beatrice«. Ein Konflikt mit dem Burgtheater} {[}1900-09-14{]}|pwv}\pwindex{Salten, Felix 06.09.1869 – 08.10.1945@\textsc{Salten, Felix} (06.09.1869 – 08.10.1945), \emph{Schriftsteller, Journalist}!Schleier der Beatrice«. Ein Konflikt mit dem Burgtheater1900-09-14@\strich\emph{»Der Schleier der Beatrice«. Ein Konflikt mit dem Burgtheater} {[}1900-09-14{]}|pwv} ka{\geminationn} ich wohl nach meiner Rückkehr noch
               ſehen, nicht wahr? Wie lange denken Sie in K.\oindex{Kaltenleutgeben@\textbf{Kaltenleutgeben}|pw} 
               {\pb}zu bleiben? \pend
           \pstart
           Ich grüße Sie herzlich {\\[\baselineskip]}Ihr {\\[\baselineskip]}\spacefill\mbox{A.}\pend
           \leftskip=0em{}
         
         \endnumbering\mylabel{h}\end{ledgroupsized}\begin{anhang}\end{anhang}\newcommand{\dateiname}{L02976}\newcommand{\titel}{Arthur Schnitzler an Felix Salten, [26. 6. 1902]}\newcommand{\editorInnen}{Martin Anton Müller und Laura Untner}%% latex-leseansicht-abspann.tex
%% Abspann für die Leseansicht.
%% Der Schalter \ifkorrekturansicht ist bereits durch den Vorspann gesetzt.

%% latex-abspann.tex
%% Gemeinsamer Abspann für Korrekturansicht und Leseansicht.
%% Setzt den Schalter \ifkorrekturansicht voraus (gesetzt in den
%% einbindenden Dateien latex-korrekturansicht-abspann.tex bzw.
%% latex-leseansicht-abspann.tex).
%% ---------------------------------------------------------------

\normalsize

% Das esempio-Environment wird nur in der Leseansicht benötigt
\ifkorrekturansicht\else
\newenvironment{esempio}[3]%
{
    \vspace{1.5ex}
    \rlap{\underline{#1}}
    \par
    \setlength{\parindent}{0cm}
    \nopagebreak
    \leftskip=#2cm
    \rightskip=#3cm
}
{
    \par
}
\fi

\doendnotes{C}
\bigskip
\vfill

\clearpage

\footnotesize

\ifkorrekturansicht
  \lohead{\textsc{register}}
\fi

% theindex-Environment neu definieren ohne reledmac
\makeatletter
\renewenvironment{theindex}{%
  \ifkorrekturansicht
    \section*{\indexname}%
  \else
    \subsubsection*{Index der erwähnten Entitäten}%
  \fi
  \setlength{\parindent}{0pt}%
  \setlength{\parskip}{0pt plus 0.3pt}%
  \let\item\@idxitem
}{%
  \ifkorrekturansicht\clearpage\fi
}
\makeatother

\IfFileExists{\jobname-pw.ind}{\input{\jobname-pw.ind}}{}

% Quellenangabe nur in der Leseansicht
\ifkorrekturansicht\else
% Fallback-Definitionen, falls die .tex-Datei \titel etc. nicht gesetzt hat
\providecommand{\titel}{}
\providecommand{\editorInnen}{}
\providecommand{\dateiname}{\jobname}

\vspace{3cm}

\vfill

\footnotesize
\textsc{Quelle}: \titel. Herausgegeben von {\editorInnen}. In: \emph{Arthur Schnitzler: Briefwechsel mit Autorinnen und Autoren}.
 Digitale Edition, https://schnitzler-briefe.acdh.oeaw.ac.at/{\dateiname}.html (Stand \today)
\fi

\end{document}


      