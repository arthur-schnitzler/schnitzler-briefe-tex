%% latex-leseansicht-vorspann.tex
%% Vorspann für die Leseansicht.
%% Lädt die gemeinsame Datei latex-vorspann.tex mit nicht gesetztem Schalter.

\newif\ifkorrekturansicht
\korrekturansichtfalse

\input{../tex-inputs/latex-vorspann}


\section[ Arthur Schnitzler an Felix Salten, {[}26. 6. 1902{]}]{L02976 Arthur Schnitzler an Felix Salten,  [26. 6. 1902]}
\nopagebreak\mylabel{L02976v}
\rehead{ }\normalsize\beginnumbering\briefempfaengerindex{Salten, Felix@\textsc{Salten, Felix}!zzzSchnitzler, Arthur@\emph{von Arthur Schnitzler}!1902-06-262@{{[}26. 6. 1902{]}}|(be}
\toendnotes[C]{\smallbreak\pagebreak[2]}
\correspDesc{Versand  durch Arthur Schnitzler am [26. 6. 1902] in Wien
\newline{}Erhalt  durch Felix Salten im Zeitraum [27. 6. 1902
                  – 1. 7. 1902?] in Kaltenleutgeben}\toendnotes[C]{\smallbreak}
\Standort{Wienbibliothek im Rathaus, ZPH 1681, 2.1.516.}
\physDesc{Brief, 1 Blatt, 2 Seiten, 420 Zeichen
\newline{}Handschrift: Bleistift, deutsche Kurrent
\newline{}Ordnung: 1) mit Bleistift von unbekannter Hand Nummerierung der Dppelseiten des Konvoluts:
                                    »8«–»9«  2) mit Bleistift datiert: »[26. 06. 1902]«}\toendnotes[C]{\smallbreak}
\pstart
           \noindent{}{\pb}lieber Freund, wieder hat mich geſtern –{ }ſchon auf dem Weg, das gräßliche Wetter abgehalten Sie in \textsc{Kaltenl.\oindex{Kaltenleutgeben@\textbf{Kaltenleutgeben}, \emph{Hauptstadt}|pw}} zu beſuchen. Nun{ }ſeh ich Sie wohl erſt, nach meiner Rückkehr, etwa gegen den
                  10. Juli. Ich fahre {\pb}\label{K_L02976-1v}\edtext{morgen}{\lemma{\textnormal{\emph{morgen}}}\Cendnote{\textnormal{Das erlaubt die Datierung des
                  undatierten Korrespondenzstücks, vgl. A. S.: \emph{Tagebuch}, 27. 6. 1902. Schnitzler kehrte am 8. 7. 1902 nach Wien\oindex{Wien@\textbf{Wien}, \emph{Verwaltungsgebiet}|pwk} zurück und
                  sah Salten\pwindex{Salten, Felix 6.\,9.\,1869 Budapest – 8.\,10.\,1945 Zürich@\textsc{Salten, Felix} (6.\,9.\,1869 Budapest – 8.\,10.\,1945 Zürich), \emph{Schriftsteller, Journalist, Chefredakteur}|pwk} nachweislich am 14. 7. 1902
                  wieder.}}}\label{K_L02976-1}{ }Salzburg\oindex{Salzburg@\textbf{Salzburg}, \emph{Verwaltungsgebiet}|pw}, Hugo\pwindex{Hofmannsthal, Hugo von 1.\,2.\,1874 Wien – 15.\,7.\,1929 Rodaun@\textsc{Hofmannsthal, Hugo von} (1.\,2.\,1874 Wien – 15.\,7.\,1929 Rodaun), \emph{Schriftsteller}|pw} dürfte übermorgen nachko{\geminationm}en. – Briefe werden mir aus Wien\oindex{Wien@\textbf{Wien}, \emph{Verwaltungsgebiet}|pw} nachgeſchickt. Die \label{K_L02976-2v}\edtext{\textsc{Bea.}\pwindex{Schnitzler, Arthur 15.\,5.\,1862 Wien – 21.\,10.\,1931 ebd.@\textsc{Schnitzler, Arthur} (15.\,5.\,1862 Wien – 21.\,10.\,1931 ebd.), \emph{Schriftsteller, Mediziner}!Schleier der Beatrice. Schauspiel in fünf Akten@\strich\emph{Der Schleier der Beatrice. Schauspiel in fünf Akten}|pw}-Sache\pwindex{\textcolor{red}{\textsuperscript{XXXX indx1}}!Erklärung [Schleier der Beatrice]@\strich\emph{Erklärung [Schleier der Beatrice]}|pwv}\pwindex{Salten, Felix 6.\,9.\,1869 Budapest – 8.\,10.\,1945 Zürich@\textsc{Salten, Felix} (6.\,9.\,1869 Budapest – 8.\,10.\,1945 Zürich), \emph{Schriftsteller, Journalist, Chefredakteur}!Erklärung [Schleier der Beatrice]@\strich\emph{Erklärung [Schleier der Beatrice]}|pwv}\pwindex{\textcolor{red}{\textsuperscript{XXXX indx1}}!Erklärung [Schleier der Beatrice]@\strich\emph{Erklärung [Schleier der Beatrice]}|pwv}\pwindex{\textcolor{red}{\textsuperscript{XXXX indx1}}!Erklärung [Schleier der Beatrice]@\strich\emph{Erklärung [Schleier der Beatrice]}|pwv}\pwindex{\textcolor{red}{\textsuperscript{XXXX indx1}}!Erklärung [Schleier der Beatrice]@\strich\emph{Erklärung [Schleier der Beatrice]}|pwv}\pwindex{\textcolor{red}{\textsuperscript{XXXX indx1}}!Erklärung [Schleier der Beatrice]@\strich\emph{Erklärung [Schleier der Beatrice]}|pwv}}{\lemma{\textnormal{\emph{Bea.-Sache}}}\Cendnote{\textnormal{Schnitzler verhandelte
                  zu diesem Zeitpunkt mit mehreren Theatern über eine Inszenierung von \emph{Der Schleier der Beatrice}\pwindex{Schnitzler, Arthur 15.\,5.\,1862 Wien – 21.\,10.\,1931 ebd.@\textsc{Schnitzler, Arthur} (15.\,5.\,1862 Wien – 21.\,10.\,1931 ebd.), \emph{Schriftsteller, Mediziner}!Schleier der Beatrice. Schauspiel in fünf Akten@\strich\emph{Der Schleier der Beatrice. Schauspiel in fünf Akten}|pwk},
                  siehe A. S.: \emph{Tagebuch}, 17. 7. 1902. Inwiefern hier Salten\pwindex{Salten, Felix 6.\,9.\,1869 Budapest – 8.\,10.\,1945 Zürich@\textsc{Salten, Felix} (6.\,9.\,1869 Budapest – 8.\,10.\,1945 Zürich), \emph{Schriftsteller, Journalist, Chefredakteur}|pwk} 
                  tätig war oder ob das überhaupt damit in Zusammenhang stand, konnte nicht geklärt werden.}}}\label{K_L02976-2} ka{\geminationn} ich wohl nach meiner Rückkehr noch{ }ſehen,
               nicht wahr? Wie lange denken Sie in K.\oindex{Kaltenleutgeben@\textbf{Kaltenleutgeben}, \emph{Hauptstadt}|pw}{ }{\pb}zu bleiben?\pend
           
\pstart
           Ich grüße Sie herzlich {\\[\baselineskip]}Ihr {\\[\baselineskip]}\spacefill\mbox{A.}\pend
           \leftskip=0em{}\selectlanguage{ngerman}\endnumbering\briefempfaengerindex{Salten, Felix@\textsc{Salten, Felix}!zzzSchnitzler, Arthur@\emph{von Arthur Schnitzler}!1902-06-262@{{[}26. 6. 1902{]}}|)be}\mylabel{L02976h}  \newcommand{\dateiname}{L02976}\newcommand{\titel}{Arthur Schnitzler an Felix Salten, [26. 6. 1902]}\newcommand{\editorInnen}{Martin Anton Müller und Laura Untner}%% latex-leseansicht-abspann.tex
%% Abspann für die Leseansicht.
%% Der Schalter \ifkorrekturansicht ist bereits durch den Vorspann gesetzt.

%% latex-abspann.tex
%% Gemeinsamer Abspann für Korrekturansicht und Leseansicht.
%% Setzt den Schalter \ifkorrekturansicht voraus (gesetzt in den
%% einbindenden Dateien latex-korrekturansicht-abspann.tex bzw.
%% latex-leseansicht-abspann.tex).
%% ---------------------------------------------------------------

\normalsize

% Das esempio-Environment wird nur in der Leseansicht benötigt
\ifkorrekturansicht\else
\newenvironment{esempio}[3]%
{
    \vspace{1.5ex}
    \rlap{\underline{#1}}
    \par
    \setlength{\parindent}{0cm}
    \nopagebreak
    \leftskip=#2cm
    \rightskip=#3cm
}
{
    \par
}
\fi

\doendnotes{C}
\bigskip
\vfill

\clearpage

\footnotesize

\ifkorrekturansicht
  \lohead{\textsc{register}}
\fi

% theindex-Environment neu definieren ohne reledmac
\makeatletter
\renewenvironment{theindex}{%
  \ifkorrekturansicht
    \section*{\indexname}%
  \else
    \subsubsection*{Index der erwähnten Entitäten}%
  \fi
  \setlength{\parindent}{0pt}%
  \setlength{\parskip}{0pt plus 0.3pt}%
  \let\item\@idxitem
}{%
  \ifkorrekturansicht\clearpage\fi
}
\makeatother

\IfFileExists{\jobname-pw.ind}{\input{\jobname-pw.ind}}{}

% Quellenangabe nur in der Leseansicht
\ifkorrekturansicht\else
% Fallback-Definitionen, falls die .tex-Datei \titel etc. nicht gesetzt hat
\providecommand{\titel}{}
\providecommand{\editorInnen}{}
\providecommand{\dateiname}{\jobname}

\vspace{3cm}

\vfill

\footnotesize
\textsc{Quelle}: \titel. Herausgegeben von {\editorInnen}. In: \emph{Arthur Schnitzler: Briefwechsel mit Autorinnen und Autoren}.
 Digitale Edition, https://schnitzler-briefe.acdh.oeaw.ac.at/{\dateiname}.html (Stand \today)
\fi

\end{document}


