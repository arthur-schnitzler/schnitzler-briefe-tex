%% latex-korrekturansicht-vorspann.tex
%% Vorspann für die Korrekturansicht.
%% Lädt die gemeinsame Datei latex-vorspann.tex mit gesetztem Schalter.

\newif\ifkorrekturansicht
\korrekturansichttrue

\input{../tex-inputs/latex-vorspann}


\section[ Arthur Schnitzler an Felix Salten, {[}26. 6. 1902{]}]{L02976 Arthur Schnitzler an Felix Salten, {[}26. 6. 1902{]}}
\nopagebreak\mylabel{L02976v}
\rehead{ }\normalsize\beginnumbering\briefempfaengerindex{Salten, Felix@\textsc{Salten, Felix}!zzzSchnitzler, Arthur@\emph{von Arthur Schnitzler}!1902-06-262@{{[}26. 6. 1902{]}}|(be}
\toendnotes[C]{\smallbreak\pagebreak[2]}\Standort{Wienbibliothek im Rathaus, ZPH 1681, 2.1.516.}
\physDesc{Brief, 1 Blatt, 2 Seiten, 420 Zeichen
\newline{}Handschrift: Bleistift, deutsche Kurrent
\newline{}Ordnung: 1) mit Bleistift von unbekannter Hand Nummerierung der Dppelseiten des Konvoluts:
                                    »8«–»9«  2) mit Bleistift datiert: »[26. 06. 1902]«}\toendnotes[C]{\smallbreak}
\pstart
           \noindent{}{\pb}lieber Freund, wieder hat mich geſtern – ſchon auf dem Weg, das gräßliche Wetter abgehalten Sie in \textsc{Kaltenl.\oindex{Kaltenleutgeben@\textbf{Kaltenleutgeben}, \emph{P.PPLA3}|pw}} zu beſuchen. Nun ſeh ich Sie wohl erſt, nach meiner Rückkehr, etwa gegen den
                  10. Juli. Ich fahre {\pb}\label{K_L02976-1v}\edtext{morgen}{\lemma{\textnormal{\emph{morgen}}}\Cendnote{\textnormal{Das erlaubt die Datierung des
                  undatierten Korrespondenzstücks, vgl. A. S.: \emph{Tagebuch}, 27. 6. 1902. Schnitzler kehrte am 8. 7. 1902 nach Wien\oindex{Wien@\textbf{Wien}, \emph{A.ADM2}|pwk} zurück und
                  sah Salten\pwindex{Salten, Felix 06.09.1869 – 08.10.1945@\textsc{Salten, Felix} (06.09.1869 – 08.10.1945), \emph{Schriftsteller/Schriftstellerin, Journalist/Journalistin, Chefredakteur/Chefredakteurin}|pwk} nachweislich am 14. 7. 1902
                  wieder.}}}\label{K_L02976-1}{ }Salzburg\oindex{Salzburg@\textbf{Salzburg}, \emph{A.ADM2}|pw}, Hugo\pwindex{Hofmannsthal, Hugo von 1874-02-01 – 1929-07-15@\textsc{Hofmannsthal, Hugo von} (1874-02-01 – 1929-07-15), \emph{Schriftsteller/Schriftstellerin}|pw} dürfte übermorgen nachko{\geminationm}en. – Briefe werden mir aus Wien\oindex{Wien@\textbf{Wien}, \emph{A.ADM2}|pw} nachgeſchickt. Die \label{K_L02976-2v}\edtext{\textsc{Bea.}\pwindex{Schleier der Beatrice. Schauspiel in fuenf Akten@\emph{Der Schleier der Beatrice. Schauspiel in fünf Akten}|pw}-Sache\pwindex{Erklaerung [Schleier der Beatrice]@\emph{Erklärung [Schleier der Beatrice]}|pwv}}{\lemma{\textnormal{\emph{Bea.-Sache}}}\Cendnote{\textnormal{Schnitzler verhandelte
                  zu diesem Zeitpunkt mit mehreren Theatern über eine Inszenierung von \emph{Der Schleier der Beatrice}\pwindex{Schleier der Beatrice. Schauspiel in fuenf Akten@\emph{Der Schleier der Beatrice. Schauspiel in fünf Akten}|pwk},
                  siehe A. S.: \emph{Tagebuch}, 17. 7. 1902. Inwiefern hier Salten\pwindex{Salten, Felix 06.09.1869 – 08.10.1945@\textsc{Salten, Felix} (06.09.1869 – 08.10.1945), \emph{Schriftsteller/Schriftstellerin, Journalist/Journalistin, Chefredakteur/Chefredakteurin}|pwk} 
                  tätig war oder ob das überhaupt damit in Zusammenhang stand, konnte nicht geklärt werden.}}}\label{K_L02976-2} ka{\geminationn} ich wohl nach meiner Rückkehr noch ſehen,
               nicht wahr? Wie lange denken Sie in K.\oindex{Kaltenleutgeben@\textbf{Kaltenleutgeben}, \emph{P.PPLA3}|pw}{ }{\pb}zu bleiben?\pend
           
\pstart
           Ich grüße Sie herzlich {\\[\baselineskip]}Ihr {\\[\baselineskip]}\spacefill\mbox{A.}\pend
           \leftskip=0em{}\selectlanguage{ngerman}\endnumbering\briefempfaengerindex{Salten, Felix@\textsc{Salten, Felix}!zzzSchnitzler, Arthur@\emph{von Arthur Schnitzler}!1902-06-262@{{[}26. 6. 1902{]}}|)be}\mylabel{L02976h}  \normalsize

\doendnotes{C}
\bigskip
\vfill

\clearpage

\footnotesize

\lohead{\textsc{register}}

% Definiere theindex-Environment komplett neu ohne reledmac
\makeatletter
\renewenvironment{theindex}{%
  \section*{\indexname}%
  \setlength{\parindent}{0pt}%
  \setlength{\parskip}{0pt plus 0.3pt}%
  \let\item\@idxitem
}{%
  \clearpage
}
\makeatother

\IfFileExists{\jobname-pw.ind}{\input{\jobname-pw.ind}}{}

\end{document}

      