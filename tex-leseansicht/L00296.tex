%% latex-leseansicht-vorspann.tex
%% Vorspann für die Leseansicht.
%% Lädt die gemeinsame Datei latex-vorspann.tex mit nicht gesetztem Schalter.

\newif\ifkorrekturansicht
\korrekturansichtfalse

\input{../tex-inputs/latex-vorspann}


         
         \renewcommand{\erwaehntePersonen}{Personen: Henrik Ibsen}
         \renewcommand{\erwaehnteOrte}{Orte: Rájec-Jestřebí, Wien}
         \renewcommand{\erwaehnteWerke}{Werke: Alkandi’s Lied, Anatol, Das Märchen. Schauspiel in drei Aufzügen}
               \section[Ferdinand von Saar an Arthur Schnitzler, 5. 2. 1894]{ Ferdinand von Saar an Arthur Schnitzler, 5. 2. 1894}\nopagebreak\mylabel{v}\rehead{ }\begin{ledgroupsized}[t]{13cm}\normalsize\beginnumbering \toendnotes[C]{\smallbreak\pagebreak[2]} \Standort{DLA, A:Schnitzler, HS.NZ85.1.5739.}
\physDesc{1 Blatt, 3 Seiten, fotografische Vervielfältigung
\newline{}Schnitzler: mit rotem Buntstift (?) nummeriert: »2« }\toendnotes[C]{\smallbreak}\pstart
           \raggedleft{}{\pb}Raitz in Mähren\oindex{Rájec-Jestřebí@\textbf{Rájec-Jestřebí}|pw},
                            5 Februar 1894.\pend
           \pstart{}Sehr geehrter Herr Doctor!\pend\pstart
           Sie werden nicht am beſten von mir denken, weil ich Ihnen über die Werke\pwindex{Schnitzler, Arthur 15.05.1862 – 21.10.1931@\textsc{Schnitzler, Arthur} (15.05.1862 – 21.10.1931), \emph{Schriftsteller, Mediziner}!Anatol1892-10-29@\strich\emph{Anatol} {[}1892-10-29{]}|pwv}\pwindex{Schnitzler, Arthur 15.05.1862 – 21.10.1931@\textsc{Schnitzler, Arthur} (15.05.1862 – 21.10.1931), \emph{Schriftsteller, Mediziner}!Maerchen. Schauspiel in drei Aufzuegen1893-12-01@\strich\emph{Das Märchen. Schauspiel in drei Aufzügen} {[}1893-12-01{]}|pwv}\pwindex{Schnitzler, Arthur 15.05.1862 – 21.10.1931@\textsc{Schnitzler, Arthur} (15.05.1862 – 21.10.1931), \emph{Schriftsteller, Mediziner}!Alkandi s Lied15.8.1890 – 1.9.1890@\strich\emph{Alkandi’s Lied} {[}15.8.1890 – 1.9.1890{]}|pwv}, welche
                    Sie mir ſo überaus freundlich und anerkennend geſendet, noch immer kein Wort
                    geſchrieben hatte. Aber erſt hier, wohin ich mich aus dem hirn- und
                    nervenzerrüttenden Trubel des Wien\oindex{Wien@\textbf{Wien}|pw}er \damage{\textcolor{gray}{Lebe}}ns vor vier Wochen zurückgerettet, war es mir möglich, die Bücher mit
                    der nöthigen Sammlung vorzunehmen. Und da muß ich Ihnen dann gleich ſagen, daß
                    mir Ihr »Anatol\pwindex{Schnitzler, Arthur 15.05.1862 – 21.10.1931@\textsc{Schnitzler, Arthur} (15.05.1862 – 21.10.1931), \emph{Schriftsteller, Mediziner}!Anatol1892-10-29@\strich\emph{Anatol} {[}1892-10-29{]}|pw}« \uline{ungemein} gefallen hat. Das iſt ein hochintereſſantes, geiſtvolles
                    Buch, das von großer Welt- und Weiberkenntniß zeugt. Friſch und flott, wie es
                    geſchrieben iſt, gewährt es Einem beim Leſen großen Genuß. Das »Märchen\pwindex{Schnitzler, Arthur 15.05.1862 – 21.10.1931@\textsc{Schnitzler, Arthur} (15.05.1862 – 21.10.1931), \emph{Schriftsteller, Mediziner}!Maerchen. Schauspiel in drei Aufzuegen1893-12-01@\strich\emph{Das Märchen. Schauspiel in drei Aufzügen} {[}1893-12-01{]}|pw}« ist gewiſſermaßen eine concentrierte Vertiefung
                    der Anatol\pwindex{Schnitzler, Arthur 15.05.1862 – 21.10.1931@\textsc{Schnitzler, Arthur} (15.05.1862 – 21.10.1931), \emph{Schriftsteller, Mediziner}!Anatol1892-10-29@\strich\emph{Anatol} {[}1892-10-29{]}|pw}-Themen und hat, da ich ähnliche
                    Seelenqualen und Conflicte in meinem Leben oft genug durchgemacht, ſehr ſtark
                    auf mich gewirkt. Daß es ſich auf der Bühne nicht halten konnte, daran iſt,
                    meiner Meinung nach, nur der Umſtand ſchuld, daß Sie die Gestalt Fanny\pwindex{Schnitzler, Arthur 15.05.1862 – 21.10.1931@\textsc{Schnitzler, Arthur} (15.05.1862 – 21.10.1931), \emph{Schriftsteller, Mediziner}!Maerchen. Schauspiel in drei Aufzuegen1893-12-01@\strich\emph{Das Märchen. Schauspiel in drei Aufzügen} {[}1893-12-01{]}|pwv}s nicht genug verdichtet, nicht
                    genug herausgearbeitet haben. Ich glaube, die modernen jungen Dramatiker {\pb}ſchaden ſich ſehr, indem ſie gewiſſermaßen
                    unbedingt den Spuren Ibſen\pwindex{Ibsen, Henrik 20.03.1828 – 23.05.1906@\textsc{Ibsen, Henrik} (20.03.1828 – 23.05.1906), \emph{Schriftsteller}|pw}’s folgen. Dieſer
                    war es, der zuerſt den Monolog aus dem Drama hinausgedrängt hat. Ich aber
                    behaupte, daß der Monolog abſolut nothwendig iſt – und zwar als Moment – wenn
                    auch nicht der Selbſterkenntniß, ſo doch der \uline{Selbſtbeobachtung}, ohne welche kein Mensch (der dieſen Namen
                    beanſprucht) jemals ſein wird und ſein kann. Würde Fanny\pwindex{Schnitzler, Arthur 15.05.1862 – 21.10.1931@\textsc{Schnitzler, Arthur} (15.05.1862 – 21.10.1931), \emph{Schriftsteller, Mediziner}!Maerchen. Schauspiel in drei Aufzuegen1893-12-01@\strich\emph{Das Märchen. Schauspiel in drei Aufzügen} {[}1893-12-01{]}|pwv} nur ein einziges Mal ihre Stellung zu Denner\pwindex{Schnitzler, Arthur 15.05.1862 – 21.10.1931@\textsc{Schnitzler, Arthur} (15.05.1862 – 21.10.1931), \emph{Schriftsteller, Mediziner}!Maerchen. Schauspiel in drei Aufzuegen1893-12-01@\strich\emph{Das Märchen. Schauspiel in drei Aufzügen} {[}1893-12-01{]}|pwv} in ernſter
                    Selbſteinkehr überdacht, würde ſie ihr Geſicht geprüft – und dasſelbe \uline{wahr und echt} vor ihrem Gewiſſen \substVorne{}\textsuperscript{emp}\substDazwischen{}be\substHinten{}funden haben; dann wären auch wir \uline{überzeugt} und würden ihr Schickſal als ein tragiſches erkennen. So
                    müſſen wir, wie Denner\pwindex{Schnitzler, Arthur 15.05.1862 – 21.10.1931@\textsc{Schnitzler, Arthur} (15.05.1862 – 21.10.1931), \emph{Schriftsteller, Mediziner}!Maerchen. Schauspiel in drei Aufzuegen1893-12-01@\strich\emph{Das Märchen. Schauspiel in drei Aufzügen} {[}1893-12-01{]}|pwv}, an
                    Worte und Betheuerungen glauben – oder nicht, glauben, wie er ſelbſt. Die
                    anderen Figuren ſind ganz prächtig, und, wie geſagt, das Stück\pwindex{Schnitzler, Arthur 15.05.1862 – 21.10.1931@\textsc{Schnitzler, Arthur} (15.05.1862 – 21.10.1931), \emph{Schriftsteller, Mediziner}!Maerchen. Schauspiel in drei Aufzuegen1893-12-01@\strich\emph{Das Märchen. Schauspiel in drei Aufzügen} {[}1893-12-01{]}|pwv} hat mich, nicht blos ſtellenweiſe,
                    ſondern im Ganzen \uline{ergriffen}, wenn ich auch, was
                    die Durchführung betrifft, nicht immer mit dem Verfaſſer übereinſtimmen konnte.
                    Nach dieſen unter allen Umſtänden ſehr hervorragenden Leiſtungen erſchien mir
                        »Alkandis Lied\pwindex{Schnitzler, Arthur 15.05.1862 – 21.10.1931@\textsc{Schnitzler, Arthur} (15.05.1862 – 21.10.1931), \emph{Schriftsteller, Mediziner}!Alkandi s Lied15.8.1890 – 1.9.1890@\strich\emph{Alkandi’s Lied} {[}15.8.1890 – 1.9.1890{]}|pw}« weniger bedeutend, wiewohl
                    es als ganz hübſche Satire auf den Nachruhm gelten kann.\pend
           \pstart
           Verzeihen Sie mir mein »Geradezu« und die knappe Faſſung desſelben. Aber ich bin
                        {\pb}ein ſchlechter »Zerleger« – und überhaupt
                    ein mangelhafter Briefſchreiber. Aber was ich ſage, kommt mir vom Herzen, und in
                    dieſem Sinne drücke ich Ihnen mit aufrichtigen Glückwünſchen die Hand und bitte
                    Sie, überzeugt zu ſein, daß ich \introOben{}mit\introOben{}{ }\uline{wahrſter} Hochachtung bin\pend
           \pstart Ihr \spacefill\mbox{Ferdinand von Saar.}\pend{}
         
         \endnumbering\mylabel{h}\end{ledgroupsized}  \newcommand{\dateiname}{L00296}\newcommand{\titel}{Ferdinand von Saar an Arthur Schnitzler, 5. 2. 1894}\newcommand{\editorInnen}{Martin Anton Müller und Gerd-Hermann Susen}%% latex-leseansicht-abspann.tex
%% Abspann für die Leseansicht.
%% Der Schalter \ifkorrekturansicht ist bereits durch den Vorspann gesetzt.

%% latex-abspann.tex
%% Gemeinsamer Abspann für Korrekturansicht und Leseansicht.
%% Setzt den Schalter \ifkorrekturansicht voraus (gesetzt in den
%% einbindenden Dateien latex-korrekturansicht-abspann.tex bzw.
%% latex-leseansicht-abspann.tex).
%% ---------------------------------------------------------------

\normalsize

% Das esempio-Environment wird nur in der Leseansicht benötigt
\ifkorrekturansicht\else
\newenvironment{esempio}[3]%
{
    \vspace{1.5ex}
    \rlap{\underline{#1}}
    \par
    \setlength{\parindent}{0cm}
    \nopagebreak
    \leftskip=#2cm
    \rightskip=#3cm
}
{
    \par
}
\fi

\doendnotes{C}
\bigskip
\vfill

\clearpage

\footnotesize

\ifkorrekturansicht
  \lohead{\textsc{register}}
\fi

% theindex-Environment neu definieren ohne reledmac
\makeatletter
\renewenvironment{theindex}{%
  \ifkorrekturansicht
    \section*{\indexname}%
  \else
    \subsubsection*{Index der erwähnten Entitäten}%
  \fi
  \setlength{\parindent}{0pt}%
  \setlength{\parskip}{0pt plus 0.3pt}%
  \let\item\@idxitem
}{%
  \ifkorrekturansicht\clearpage\fi
}
\makeatother

\IfFileExists{\jobname-pw.ind}{\input{\jobname-pw.ind}}{}

% Quellenangabe nur in der Leseansicht
\ifkorrekturansicht\else
% Fallback-Definitionen, falls die .tex-Datei \titel etc. nicht gesetzt hat
\providecommand{\titel}{}
\providecommand{\editorInnen}{}
\providecommand{\dateiname}{\jobname}

\vspace{3cm}

\vfill

\footnotesize
\textsc{Quelle}: \titel. Herausgegeben von {\editorInnen}. In: \emph{Arthur Schnitzler: Briefwechsel mit Autorinnen und Autoren}.
 Digitale Edition, https://schnitzler-briefe.acdh.oeaw.ac.at/{\dateiname}.html (Stand \today)
\fi

\end{document}


      