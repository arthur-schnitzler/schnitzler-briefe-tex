%% latex-korrekturansicht-vorspann.tex
%% Vorspann für die Korrekturansicht.
%% Lädt die gemeinsame Datei latex-vorspann.tex mit gesetztem Schalter.

\newif\ifkorrekturansicht
\korrekturansichttrue

\input{../tex-inputs/latex-vorspann}


\section[Ferdinand von Saar an Arthur Schnitzler, 5. 2. 1894]{L00296 Ferdinand von Saar an Arthur Schnitzler, 5. 2. 1894}
\nopagebreak\mylabel{L00296v}
\rehead{ }\normalsize\beginnumbering\briefempfaengerindex{Schnitzler, Arthur@\textsc{Schnitzler, Arthur}!zzzSaar, Ferdinand von@\emph{von Ferdinand von Saar}!1894-02-051@{5. 2. 1894}|(be}
\toendnotes[C]{\smallbreak\pagebreak[2]}\Standort{DLA, A:Schnitzler, HS.NZ85.1.5739.}
\physDesc{Brief, fotografische Vervielfältigung2 Blätter, 3 Seiten, 2638 Zeichen
\newline{}Handschrift: schwarze Tinte, deutsche Kurrent
\newline{}Schnitzler: mit rotem Buntstift (?) nummeriert: »2« }\toendnotes[C]{\smallbreak}
\pstart
           \raggedleft{}{\pb}Raitz in Mähren\oindex{Rájec-Jestřebí@\textbf{Rájec-Jestřebí}, \emph{P.PPL}|pw},
                  5 Februar 1894.\pend
           
\pstart{}Sehr geehrter Herr Doctor!\pend\vspace{0.5em}
\pstart
           Sie werden nicht am beſten von mir denken, weil ich Ihnen über die Werke\pwindex{Anatol@\emph{Anatol}|pwv}\pwindex{Maerchen. Schauspiel in drei Aufzuegen@\emph{Das Märchen. Schauspiel in drei Aufzügen}|pwv}\pwindex{Alkandi s Lied@\emph{Alkandi’s Lied}|pwv}, welche Sie mir
               ſo überaus freundlich und anerkennend geſendet, noch immer kein Wort geſchrieben
               hatte. Aber erſt hier, wohin ich mich aus dem hirn- und nervenzerrüttenden Trubel des
                  Wien\oindex{Wien@\textbf{Wien}, \emph{A.ADM2}|pw}er \damage{\textcolor{gray}{Lebe}}ns vor vier Wochen zurückgerettet, war es mir möglich, die Bücher mit der
               nöthigen Sammlung vorzunehmen. Und da muß ich Ihnen dann gleich ſagen, daß mir Ihr
                  »Anatol\pwindex{Anatol@\emph{Anatol}|pw}« \uline{ungemein} gefallen hat. Das iſt ein hochintereſſantes, geiſtvolles Buch, das
               von großer Welt- und Weiberkenntniß zeugt. Friſch und flott, wie es geſchrieben iſt,
               gewährt es Einem beim Leſen großen Genuß. Das »Märchen\pwindex{Maerchen. Schauspiel in drei Aufzuegen@\emph{Das Märchen. Schauspiel in drei Aufzügen}|pw}« ist gewiſſermaßen eine concentrierte Vertiefung der Anatol\pwindex{Anatol@\emph{Anatol}|pw}-Themen und hat, da ich ähnliche Seelenqualen und
               Conflicte in meinem Leben oft genug durchgemacht, ſehr ſtark auf mich gewirkt. Daß es
               ſich auf der Bühne nicht halten konnte, daran iſt, meiner Meinung nach, nur der
               Umſtand ſchuld, daß Sie die Gestalt Fannys\pwindex{Maerchen. Schauspiel in drei Aufzuegen@\emph{Das Märchen. Schauspiel in drei Aufzügen}|pwv} nicht genug verdichtet, nicht genug herausgearbeitet haben. Ich
               glaube, die modernen jungen Dramatiker {\pb}ſchaden ſich
               ſehr, indem ſie gewiſſermaßen unbedingt den Spuren Ibſen\pwindex{Ibsen, Henrik 20.03.1828 – 23.05.1906@\textsc{Ibsen, Henrik} (20.03.1828 – 23.05.1906), \emph{Schriftsteller/Schriftstellerin}|pw}’s folgen. Dieſer war es, der zuerſt den Monolog aus dem Drama
               hinausgedrängt hat. Ich aber behaupte, daß der Monolog abſolut nothwendig iſt – und
               zwar als Moment – wenn auch nicht der Selbſterkenntniß, ſo doch der \uline{Selbſtbeobachtung}, ohne welche kein Mensch (der dieſen
               Namen beanſprucht) jemals ſein wird und ſein kann. Würde Fanny\pwindex{Maerchen. Schauspiel in drei Aufzuegen@\emph{Das Märchen. Schauspiel in drei Aufzügen}|pwv} nur ein einziges Mal ihre Stellung zu
                  Denner\pwindex{Maerchen. Schauspiel in drei Aufzuegen@\emph{Das Märchen. Schauspiel in drei Aufzügen}|pwv} in ernſter
               Selbſteinkehr überdacht, würde ſie ihr Geſicht geprüft – und dasſelbe \uline{wahr und echt} vor ihrem Gewiſſen \substVorne{}\textsuperscript{emp}\substDazwischen{}be\substHinten{}funden haben; dann wären auch wir \uline{überzeugt}
               und würden ihr Schickſal als ein tragiſches erkennen. So müſſen wir, wie Denner\pwindex{Maerchen. Schauspiel in drei Aufzuegen@\emph{Das Märchen. Schauspiel in drei Aufzügen}|pwv}, an Worte und
               Betheuerungen glauben – oder nicht, glauben, wie er ſelbſt. Die anderen Figuren ſind
               ganz prächtig, und, wie geſagt, das Stück\pwindex{Maerchen. Schauspiel in drei Aufzuegen@\emph{Das Märchen. Schauspiel in drei Aufzügen}|pwv} hat mich, nicht blos ſtellenweiſe, ſondern im Ganzen \uline{ergriffen}, wenn ich auch, was die Durchführung
               betrifft, nicht immer mit dem Verfaſſer übereinſtimmen konnte. Nach dieſen unter
               allen Umſtänden ſehr hervorragenden Leiſtungen erſchien mir »Alkandis Lied\pwindex{Alkandi s Lied@\emph{Alkandi’s Lied}|pw}« weniger bedeutend, wiewohl es als ganz hübſche
               Satire auf den Nachruhm gelten kann.\pend
           
\pstart
           Verzeihen Sie mir mein »Geradezu« und die knappe Faſſung desſelben. Aber ich bin {\pb}ein ſchlechter »Zerleger« – und überhaupt ein
               mangelhafter Briefſchreiber. Aber was ich ſage, kommt mir vom Herzen, und in dieſem
               Sinne drücke ich Ihnen mit aufrichtigen Glückwünſchen die Hand und bitte Sie,
               überzeugt zu ſein, daß ich \introOben{}mit\introOben{}{ }\uline{wahrſter} Hochachtung bin\pend
           \pstart Ihr \spacefill\mbox{Ferdinand von Saar.}\pend{}\selectlanguage{ngerman}\endnumbering\briefempfaengerindex{Schnitzler, Arthur@\textsc{Schnitzler, Arthur}!zzzSaar, Ferdinand von@\emph{von Ferdinand von Saar}!1894-02-051@{5. 2. 1894}|)be}\mylabel{L00296h}  \normalsize

\doendnotes{C}
\bigskip
\vfill

\clearpage

\footnotesize

\lohead{\textsc{register}}

% Definiere theindex-Environment komplett neu ohne reledmac
\makeatletter
\renewenvironment{theindex}{%
  \section*{\indexname}%
  \setlength{\parindent}{0pt}%
  \setlength{\parskip}{0pt plus 0.3pt}%
  \let\item\@idxitem
}{%
  \clearpage
}
\makeatother

\IfFileExists{\jobname-pw.ind}{\input{\jobname-pw.ind}}{}

\end{document}

      