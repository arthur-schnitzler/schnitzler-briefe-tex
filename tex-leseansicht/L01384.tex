%% latex-leseansicht-vorspann.tex
%% Vorspann für die Leseansicht.
%% Lädt die gemeinsame Datei latex-vorspann.tex mit nicht gesetztem Schalter.

\newif\ifkorrekturansicht
\korrekturansichtfalse

\input{../tex-inputs/latex-vorspann}


               \section[Hugo von Hofmannsthal an Arthur Schnitzler, {[}15. 3. 1904{]}]{ Hugo von Hofmannsthal an Arthur Schnitzler, {[}15. 3. 1904{]}}\nopagebreak\mylabel{v}\rehead{ }\begin{ledgroupsized}[t]{13cm}\normalsize\beginnumbering\briefempfaengerindex{Schnitzler, Arthur@\textsc{Schnitzler, Arthur}!zzzHofmannsthal, Hugo von@\emph{von Hugo von Hofmannsthal}!1904-03-152@{15. 3. 1904}|(be} \toendnotes[C]{\smallbreak\pagebreak[2]} \Standort{CUL, Schnitzler, B 43.}
\physDesc{Brief, 1 Blatt, 4 Seiten
\newline{}Handschrift: schwarze Tinte, deutsche Kurrent
\newline{}Schnitzler: mit Bleistift datiert: »15/3 904.« \newline{}Ordnung: 1) mit Bleistift von unbekannter Hand nummeriert: »\strikeout{241}« 2) mit Bleistift von unbekannter Hand nummeriert:
                                    »217«}\buchAbdrucke{\weitereDrucke{Hugo von Hofmannsthal, Arthur Schnitzler: \emph{Briefwechsel}. Hg. Therese Nickl und Heinrich Schnitzler. Frankfurt am Main: \emph{S. Fischer} 1964, S. 184.} }\toendnotes[C]{\smallbreak}\pstart
           \noindent{}{\pb}Mein lieber Arthur,\hspace*{1.5em}meiner Mama\pwindex{Hofmannsthal, Anna von 27.01.1849 – 22.03.1904@\textsc{Hofmannsthal, Anna von} (27.01.1849 – 22.03.1904)|pwv} Zuſtand iſt – wie ja nicht anders zu erwarten, – genau
               ſo elend wie vor ein paar Tagen. Geprüft durch jahrelangen Anblick eines ſolchen
               complicierten \label{K_L01384_1v}\edtext{pſychaſtheniſchen}{\lemma{\textnormal{\emph{pſychaſtheniſchen}}}\Cendnote{\textnormal{1903 von Pierre Janet\pwindex{Janet, Pierre 1859-05-30 – 1947-02-24@\textsc{Janet, Pierre} (1859-05-30 – 1947-02-24), \emph{Psychiater, Psychologe}|pwk}
                  eingeführter Ausdruck für jemanden, der aufgrund einer neurotischen Störung eine
                  nur geringe körperliche und psychische Belastbarkeit aufweist.}}}\label{K_L01384_1h} Leidens ſind
               wir ja auch nicht ungeduldig.\hspace*{1.5em}Nicht wahr aber, Sie
               ſind nicht bös, daſs das Leben es mit {\pb}ſich gebracht hat, daſs zwei ſo
               verſchiedene Dinge, wie Ihre zufällige Arzt-eigenſchaft und unſere Freundſchaft mich
               jetzt ermuthigen, Sie um Hilfe anzubetteln. Es erſcheint halt alles ringsum, alles
               was man verſuchen kann, alles was man herbeirufen kann, ſo erſchöpft.\pend
           \pstart
           Das iſt der Gegenſtand von meiner und meines Vaters\pwindex{Hofmannsthal, Hugo August von 21.12.1841 – 08.12.1915@\textsc{Hofmannsthal, Hugo August von} (21.12.1841 – 08.12.1915), \emph{Bankdirektor}|pwv} hauptſächlicher Bitte: daſs Sie {\pb}Ihr Verſtändnis der \uline{Geſamterſcheinung} dieſer kranken Frau\pwindex{Hofmannsthal, Anna von 27.01.1849 – 22.03.1904@\textsc{Hofmannsthal, Anna von} (27.01.1849 – 22.03.1904)|pwv} in einem Geſpräch Ihrem Bruder\pwindex{Schnitzler, Julius 13.07.1865 – 29.06.1939@\textsc{Schnitzler, Julius} (13.07.1865 – 29.06.1939), \emph{Chirurg}|pwv} nahebringen, ſo daſs er von ſeinem
               nächſten Beſuch an – und bei öfteren Beſuchen, die man erbitten wird – neben dem Hausarzt\pwindex{Schandlbauer, Hans 12.1.1844 – 1910-05-25@\textsc{Schandlbauer, Hans} (12.1.1844 – 1910-05-25), \emph{Mediziner}|pwuv} oder über
               dem Hausarzt\pwindex{Schandlbauer, Hans 12.1.1844 – 1910-05-25@\textsc{Schandlbauer, Hans} (12.1.1844 – 1910-05-25), \emph{Mediziner}|pwuv} der
               leitende Arzt im Ganzen wird, derjenige gute Arzt der die Einwirkungen {\pb}auf einen Theil (hier die
               Narbungen im Darm) ſo weit als möglich dem Einblick in das Ganze unterordnet.\pend
           \pstart
           Wir bilden uns nicht ein, daſs ein ſolcher Patient zu \uline{curieren} iſt. Aber von einer ſolchen Krise des Elends wieder in das relativ
               normale zurückzuführen iſt ſie doch vielleicht\textcolor{gray}{?} Sie werden mir
                  Freitag vielleicht ſagen, wann Sie mit Ihrem Bruder\pwindex{Schnitzler, Julius 13.07.1865 – 29.06.1939@\textsc{Schnitzler, Julius} (13.07.1865 – 29.06.1939), \emph{Chirurg}|pwv}{ }ſprechen können, nachher ruft man ihn dann wieder.
               Ihr \spacefill\mbox{Hugo}\pend
           \endnumbering\briefempfaengerindex{Schnitzler, Arthur@\textsc{Schnitzler, Arthur}!zzzHofmannsthal, Hugo von@\emph{von Hugo von Hofmannsthal}!1904-03-152@{15. 3. 1904}|)be}\mylabel{h}\end{ledgroupsized}  \newcommand{\dateiname}{L01384}\newcommand{\titel}{Hugo von Hofmannsthal an Arthur Schnitzler, [15. 3. 1904]}\newcommand{\editorInnen}{Martin Anton Müller und Gerd-Hermann Susen}
            \footnotesize
\begin{ledgroupsized}[t]{11.5cm}
\doendnotes{C}
\end{ledgroupsized}
         %% latex-leseansicht-abspann.tex
%% Abspann für die Leseansicht.
%% Der Schalter \ifkorrekturansicht ist bereits durch den Vorspann gesetzt.

%% latex-abspann.tex
%% Gemeinsamer Abspann für Korrekturansicht und Leseansicht.
%% Setzt den Schalter \ifkorrekturansicht voraus (gesetzt in den
%% einbindenden Dateien latex-korrekturansicht-abspann.tex bzw.
%% latex-leseansicht-abspann.tex).
%% ---------------------------------------------------------------

\normalsize

% Das esempio-Environment wird nur in der Leseansicht benötigt
\ifkorrekturansicht\else
\newenvironment{esempio}[3]%
{
    \vspace{1.5ex}
    \rlap{\underline{#1}}
    \par
    \setlength{\parindent}{0cm}
    \nopagebreak
    \leftskip=#2cm
    \rightskip=#3cm
}
{
    \par
}
\fi

\doendnotes{C}
\bigskip
\vfill

\clearpage

\footnotesize

\ifkorrekturansicht
  \lohead{\textsc{register}}
\fi

% theindex-Environment neu definieren ohne reledmac
\makeatletter
\renewenvironment{theindex}{%
  \ifkorrekturansicht
    \section*{\indexname}%
  \else
    \subsubsection*{Index der erwähnten Entitäten}%
  \fi
  \setlength{\parindent}{0pt}%
  \setlength{\parskip}{0pt plus 0.3pt}%
  \let\item\@idxitem
}{%
  \ifkorrekturansicht\clearpage\fi
}
\makeatother

\IfFileExists{\jobname-pw.ind}{\input{\jobname-pw.ind}}{}

% Quellenangabe nur in der Leseansicht
\ifkorrekturansicht\else
% Fallback-Definitionen, falls die .tex-Datei \titel etc. nicht gesetzt hat
\providecommand{\titel}{}
\providecommand{\editorInnen}{}
\providecommand{\dateiname}{\jobname}

\vspace{3cm}

\vfill

\footnotesize
\textsc{Quelle}: \titel. Herausgegeben von {\editorInnen}. In: \emph{Arthur Schnitzler: Briefwechsel mit Autorinnen und Autoren}.
 Digitale Edition, https://schnitzler-briefe.acdh.oeaw.ac.at/{\dateiname}.html (Stand \today)
\fi

\end{document}


      