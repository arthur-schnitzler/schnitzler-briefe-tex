%% latex-leseansicht-vorspann.tex
%% Vorspann für die Leseansicht.
%% Lädt die gemeinsame Datei latex-vorspann.tex mit nicht gesetztem Schalter.

\newif\ifkorrekturansicht
\korrekturansichtfalse

\input{../tex-inputs/latex-vorspann}


\section[ Paul Goldmann an Arthur Schnitzler, 14. 10. {[}1900{]}]{L02936 Paul Goldmann an Arthur Schnitzler,  14. 10. [1900]}
\nopagebreak\mylabel{L02936v}
\rehead{ }\normalsize\beginnumbering\briefempfaengerindex{Schnitzler, Arthur@\textsc{Schnitzler, Arthur}!zzzGoldmann, Paul@\emph{von Paul Goldmann}!1900-10-141@{14. 10. [1900]}|(be}
\toendnotes[C]{\smallbreak\pagebreak[2]}
\correspDesc{Versand  durch Paul Goldmann am 14. 10. [1900] in Berlin
\newline{}Erhalt  durch Arthur Schnitzler im Zeitraum [15. 10. 1900 – 19. 10. 1900?] in Baden bei Wien?}\toendnotes[C]{\smallbreak}
\Standort{DLA, A:Schnitzler, HS.NZ85.1.3170.}
\physDesc{Brief, 2 Blätter, 8 Seiten, 3316 Zeichen
\newline{}Handschrift: blaue Tinte, deutsche Kurrent
\newline{}Schnitzler: 1) mit Bleistift das Jahr »900« ergänzt und auf der ersten Seite des zweiten Blatts das
                                 vollständige Datum »14/10 900« vermerkt  2) mit rotem Buntstift acht Unterstreichungen und ein »X«}\toendnotes[C]{\smallbreak}
\pstart
           \raggedleft{}{\pb}Berlin\oindex{Berlin@\textbf{Berlin}, \emph{Hauptstadt}|pw}, 14. Oktober.\pend
           
\pstart\center{}Mein lieber Freund,\pend\vspace{0.5em}
\pstart
           Heut am Sonntag habe
               ich endlich ein paar Minuten frei zu einem Briefe an Dich.\pend
           
\pstart
           Die \label{K_L02936-1v}\edtext{»Fackel\pwindex{Fackel@\emph{Die Fackel}|pwv}\pwindex{Kraus, Karl 28.\,4.\,1874 Jičín – 12.\,6.\,1936 Wien@\textsc{Kraus, Karl} (28.\,4.\,1874 Jičín – 12.\,6.\,1936 Wien), \emph{Schriftsteller, Publizist, Schriftsteller}!Affaire Schlenther-Schnitzler]@\strich\emph{[Die Affaire Schlenther-Schnitzler]}|pwv}\pwindex{Kraus, Karl 28.\,4.\,1874 Jičín – 12.\,6.\,1936 Wien@\textsc{Kraus, Karl} (28.\,4.\,1874 Jičín – 12.\,6.\,1936 Wien), \emph{Schriftsteller, Publizist, Schriftsteller}!Antworten des Herausgebers. Habitué@\strich\emph{Antworten des Herausgebers. Habitué}|pwv}\pwindex{Fackel@\emph{Die Fackel}|pw}«}{\lemma{\textnormal{\emph{»Fackel«}}}\Cendnote{\textnormal{Bezugnahme auf Karl Kraus\pwindex{Kraus, Karl 28.\,4.\,1874 Jičín – 12.\,6.\,1936 Wien@\textsc{Kraus, Karl} (28.\,4.\,1874 Jičín – 12.\,6.\,1936 Wien), \emph{Schriftsteller, Publizist, Schriftsteller}|pwk}: \emph{[Die Affaire Schlenther-Schnitzler]}\pwindex{Kraus, Karl 28.\,4.\,1874 Jičín – 12.\,6.\,1936 Wien@\textsc{Kraus, Karl} (28.\,4.\,1874 Jičín – 12.\,6.\,1936 Wien), \emph{Schriftsteller, Publizist, Schriftsteller}!Affaire Schlenther-Schnitzler]@\strich\emph{[Die Affaire Schlenther-Schnitzler]}|pwk}. In: \emph{Die Fackel}\pwindex{Fackel@\emph{Die Fackel}|pwk}, Jg. 2, Nr. 53, Mitte September 1900, S. 1–6, und auf Karl Kraus\pwindex{Kraus, Karl 28.\,4.\,1874 Jičín – 12.\,6.\,1936 Wien@\textsc{Kraus, Karl} (28.\,4.\,1874 Jičín – 12.\,6.\,1936 Wien), \emph{Schriftsteller, Publizist, Schriftsteller}|pwk}: \emph{Antworten des Herausgebers. Habitué}\pwindex{Kraus, Karl 28.\,4.\,1874 Jičín – 12.\,6.\,1936 Wien@\textsc{Kraus, Karl} (28.\,4.\,1874 Jičín – 12.\,6.\,1936 Wien), \emph{Schriftsteller, Publizist, Schriftsteller}!Antworten des Herausgebers. Habitué@\strich\emph{Antworten des Herausgebers. Habitué}|pwk}. In: \emph{Die Fackel}\pwindex{Fackel@\emph{Die Fackel}|pwk}, Jg. 2, Nr. 54, Ende September 1900, S. 25–26. Siehe zum
                  Konflikt zwischen Schnitzler und Paul Schlenther\pwindex{Schlenther, Paul 20.\,8.\,1854 Chernyakhovsk – 30.\,4.\,1916 Berlin@\textsc{Schlenther, Paul} (20.\,8.\,1854 Chernyakhovsk – 30.\,4.\,1916 Berlin), \emph{Schriftsteller, Kritiker, Theaterleiter}|pwk} auch XXXX Auszeichnungsfehler: Dokument L02893 nicht gefunden.}}}\label{K_L02936-1}. Was willſt Du von dem
               Lausbuben? Offen geſtanden, ich hätte noch Schlimmeres erwartet. Im Übrigen hat
                  \label{K_L02936-2v}\edtext{\textsc{Burckhardt\pwindex{Burckhard, Max Eugen 14.\,7.\,1854 Korneuburg – 16.\,3.\,1912 Wien@\textsc{Burckhard, Max Eugen} (14.\,7.\,1854 Korneuburg – 16.\,3.\,1912 Wien), \emph{Schriftsteller, Rechtswissenschaftler, Theaterleiter}|pw}} in der »Zeit\pwindex{Zeit. Wiener Wochenschrift@\emph{Die Zeit. Wiener Wochenschrift}|pw}«}{\lemma{\textnormal{\emph{Burckhardt in der »Zeit«}}}\Cendnote{\textnormal{Max Burckhard\pwindex{Burckhard, Max Eugen 14.\,7.\,1854 Korneuburg – 16.\,3.\,1912 Wien@\textsc{Burckhard, Max Eugen} (14.\,7.\,1854 Korneuburg – 16.\,3.\,1912 Wien), \emph{Schriftsteller, Rechtswissenschaftler, Theaterleiter}|pwk}: \emph{Wienerinnen. Lustspiel in drei Aufzügen von Hermann Bahr.
                        Aufgeführt zum erstenmale im Deutschen Volkstheater am 3. October 1900}\pwindex{Burckhard, Max Eugen 14.\,7.\,1854 Korneuburg – 16.\,3.\,1912 Wien@\textsc{Burckhard, Max Eugen} (14.\,7.\,1854 Korneuburg – 16.\,3.\,1912 Wien), \emph{Schriftsteller, Rechtswissenschaftler, Theaterleiter}!Wienerinnen. Lustspiel in drei Aufzügen von Hermann Bahr. Aufgeführt zum erstenmale im Deutschen Volkstheater am 3. October 1900.@\strich\emph{Wienerinnen. Lustspiel in drei Aufzügen von Hermann Bahr. Aufgeführt zum erstenmale im Deutschen Volkstheater am 3. October 1900.}|pwk}.
                     In: \emph{Zeit}\pwindex{Zeit. Wiener Wochenschrift@\emph{Die Zeit. Wiener Wochenschrift}|pwk}, Bd. 25, Nr. 314, 6. 10. 1900, S. 10–11.}}}\label{K_L02936-2} das wahre Wort
                  geſchrieben\pwindex{Burckhard, Max Eugen 14.\,7.\,1854 Korneuburg – 16.\,3.\,1912 Wien@\textsc{Burckhard, Max Eugen} (14.\,7.\,1854 Korneuburg – 16.\,3.\,1912 Wien), \emph{Schriftsteller, Rechtswissenschaftler, Theaterleiter}!Wienerinnen. Lustspiel in drei Aufzügen von Hermann Bahr. Aufgeführt zum erstenmale im Deutschen Volkstheater am 3. October 1900.@\strich\emph{Wienerinnen. Lustspiel in drei Aufzügen von Hermann Bahr. Aufgeführt zum erstenmale im Deutschen Volkstheater am 3. October 1900.}|pwv}: die Leute
               rächen{ }ſich jetzt an Dir, weil{ }ſie Dir haben applaudiren müſſen. Auf das Geſindel im
               Allgemeinen war niemals zu rechnen. Ob die \label{K_L02936-3v}\edtext{Aktion\pwindex{\textcolor{red}{\textsuperscript{XXXX indx1}}!Erklärung [Schleier der Beatrice]@\strich\emph{Erklärung [Schleier der Beatrice]}|pwv}\pwindex{Salten, Felix 6.\,9.\,1869 Budapest – 8.\,10.\,1945 Zürich@\textsc{Salten, Felix} (6.\,9.\,1869 Budapest – 8.\,10.\,1945 Zürich), \emph{Schriftsteller, Journalist, Chefredakteur}!Erklärung [Schleier der Beatrice]@\strich\emph{Erklärung [Schleier der Beatrice]}|pwv}\pwindex{\textcolor{red}{\textsuperscript{XXXX indx1}}!Erklärung [Schleier der Beatrice]@\strich\emph{Erklärung [Schleier der Beatrice]}|pwv}\pwindex{\textcolor{red}{\textsuperscript{XXXX indx1}}!Erklärung [Schleier der Beatrice]@\strich\emph{Erklärung [Schleier der Beatrice]}|pwv}\pwindex{\textcolor{red}{\textsuperscript{XXXX indx1}}!Erklärung [Schleier der Beatrice]@\strich\emph{Erklärung [Schleier der Beatrice]}|pwv}\pwindex{\textcolor{red}{\textsuperscript{XXXX indx1}}!Erklärung [Schleier der Beatrice]@\strich\emph{Erklärung [Schleier der Beatrice]}|pwv}}{\lemma{\textnormal{\emph{Aktion}}}\Cendnote{\textnormal{Siehe Hermann Bahr, Arthur Schnitzler: \emph{Briefwechsel, Aufzeichnungen, Dokumente (1891–1931)}, Hermann Bahr, Julius Bauer, J. J. David, Robert Hirschfeld, Felix Salten, Ludwig Speidel: Erklärung, 14. 9. 1900. }}}\label{K_L02936-3}{ }ſonſt
               wirkungslos geblieben, wird{ }ſich zeigen. Welche Wirkung hätte {\pb}denn auch kommen{ }ſollen? Die Hauptſache war, daß der
               Herr \textsc{Schlenther\pwindex{Schlenther, Paul 20.\,8.\,1854 Chernyakhovsk – 30.\,4.\,1916 Berlin@\textsc{Schlenther, Paul} (20.\,8.\,1854 Chernyakhovsk – 30.\,4.\,1916 Berlin), \emph{Schriftsteller, Kritiker, Theaterleiter}|pw}} eine Antwort\pwindex{\textcolor{red}{\textsuperscript{XXXX indx1}}!Erklärung [Schleier der Beatrice]@\strich\emph{Erklärung [Schleier der Beatrice]}|pwv}\pwindex{Salten, Felix 6.\,9.\,1869 Budapest – 8.\,10.\,1945 Zürich@\textsc{Salten, Felix} (6.\,9.\,1869 Budapest – 8.\,10.\,1945 Zürich), \emph{Schriftsteller, Journalist, Chefredakteur}!Erklärung [Schleier der Beatrice]@\strich\emph{Erklärung [Schleier der Beatrice]}|pwv}\pwindex{\textcolor{red}{\textsuperscript{XXXX indx1}}!Erklärung [Schleier der Beatrice]@\strich\emph{Erklärung [Schleier der Beatrice]}|pwv}\pwindex{\textcolor{red}{\textsuperscript{XXXX indx1}}!Erklärung [Schleier der Beatrice]@\strich\emph{Erklärung [Schleier der Beatrice]}|pwv}\pwindex{\textcolor{red}{\textsuperscript{XXXX indx1}}!Erklärung [Schleier der Beatrice]@\strich\emph{Erklärung [Schleier der Beatrice]}|pwv}\pwindex{\textcolor{red}{\textsuperscript{XXXX indx1}}!Erklärung [Schleier der Beatrice]@\strich\emph{Erklärung [Schleier der Beatrice]}|pwv} auf{ }ſein
               unerhörtes Benehmen bekam. Und den{ }ſchlechten Ruf, den er ohnedies hat, hat dieſe
               Affaire nur noch vergrößert. Er hat’s geſpürt und wird’s noch weiter{ }ſpüren. Dieſe
               Affaire, mag man{ }ſagen, was man will, iſt ein Grund mehr für{ }ſeinen Weggang vom Burgtheater\orgindex{Burgtheater@Burgtheater|pw}. Selbſt hier\oindex{Berlin@\textbf{Berlin}, \emph{Hauptstadt}|pwv}, wo man ihn für einen Gott hält, hat{ }ſie
               ihm geſchadet{\dotsfive}\pend
           
\pstart
           Dein \label{K_L02936-4v}\edtext{»Ohrenleiden«}{\lemma{\textnormal{\emph{»Ohrenleiden«}}}\Cendnote{\textnormal{Schnitzler litt an Otosklerose
                  (Verknöcherung des Innenohrs mit zunehmender Schwerhörigkeit).}}}\label{K_L02936-4}. Darauf weiß
               ich nur \uline{eine} Antwort: Heirathen. Ich{ }ſchwöre Dir:
               wenn Du Frau {\pb}und Kinder haben wirſt, wirſt Du Dich
               weniger mit Deinem Ohre beſchäftigen; und wenn Du Dich weniger damit beſchäftigen
               wirſt, \strikeout{\textcolor{gray}{wi}} wirſt Du weniger darunter leiden.\pend
           
\pstart
           Mit \textsc{Lindau\pwindex{Lindau, Paul 3.\,6.\,1839 Magdeburg – 31.\,1.\,1919 Berlin@\textsc{Lindau, Paul} (3.\,6.\,1839 Magdeburg – 31.\,1.\,1919 Berlin), \emph{Schriftsteller, Kritiker, Theaterleiter}|pw}} werde ich bei nächſter Gelegenheit \label{K_L02936-5v}\edtext{wegen \textsc{Salten\pwindex{Salten, Felix 6.\,9.\,1869 Budapest – 8.\,10.\,1945 Zürich@\textsc{Salten, Felix} (6.\,9.\,1869 Budapest – 8.\,10.\,1945 Zürich), \emph{Schriftsteller, Journalist, Chefredakteur}|pw}}}{\lemma{\textnormal{\emph{wegen Salten}}}\Cendnote{\textnormal{Die Stelle ist unklar. Möglicherweise
                  ging es um eine etwaige Uraufführung von Saltens\pwindex{Salten, Felix 6.\,9.\,1869 Budapest – 8.\,10.\,1945 Zürich@\textsc{Salten, Felix} (6.\,9.\,1869 Budapest – 8.\,10.\,1945 Zürich), \emph{Schriftsteller, Journalist, Chefredakteur}|pwk} Dreiakter \emph{Der Gemeine}\pwindex{Salten, Felix 6.\,9.\,1869 Budapest – 8.\,10.\,1945 Zürich@\textsc{Salten, Felix} (6.\,9.\,1869 Budapest – 8.\,10.\,1945 Zürich), \emph{Schriftsteller, Journalist, Chefredakteur}!Gemeine. Schauspiel in drei Aufzügen@\strich\emph{Der Gemeine. Schauspiel in drei Aufzügen}|pwk}.
               }}}\label{K_L02936-5}{ }ſprechen.\pend
           
\pstart
           \textsc{Kerr\pwindex{Kerr, Alfred 25.\,12.\,1867 Breslau – 12.\,10.\,1948 Hamburg@\textsc{Kerr, Alfred} (25.\,12.\,1867 Breslau – 12.\,10.\,1948 Hamburg), \emph{Schriftsteller, Kritiker}|pw}}{ }ſehe ich{ }ſehr{ }ſelten. Wenn wir uns{ }ſehen,{ }ſprechen wir{ }ſehr freundſchaftlich
               miteinander. Er{ }ſteckt tief in{ }ſeinem \label{K_L02936-6v}\edtext{Liebeswonnen\pwindex{Wendt, Anna @\textsc{Wendt, Anna}|pwv}}{\lemma{\textnormal{\emph{Liebeswonnen}}}\Cendnote{\textnormal{Siehe XXXX Auszeichnungsfehler: Dokument L02911 nicht gefunden. }}}\label{K_L02936-6} und{ }ſtrebt
               der Erfüllung{ }ſeiner Wünſche zu, was mit großen {\pb}Kämpfen verbunden{ }ſcheint. Aber er\pwindex{Kerr, Alfred 25.\,12.\,1867 Breslau – 12.\,10.\,1948 Hamburg@\textsc{Kerr, Alfred} (25.\,12.\,1867 Breslau – 12.\,10.\,1948 Hamburg), \emph{Schriftsteller, Kritiker}|pwv} wird es{ }ſchon durchſetzen. Er und das Mädel\pwindex{Wendt, Anna @\textsc{Wendt, Anna}|pwv}{ }ſcheinen{ }ſich{ }ſehr zu lieben, und das iſt die
               Hauptſache.\pend
           
\pstart
           Ich bin mit dem Hauſe \textsc{M.-C.\pwindex{Meyer-Cohn, Helene 30.\,12.\,1859 Lviv – 9.\,11.\,1918 Berlin@\textsc{Meyer-Cohn, Helene} (30.\,12.\,1859 Lviv – 9.\,11.\,1918 Berlin), \emph{Übersetzerin}|pwv}\pwindex{Meyer-Cohn, Alexander 1.\,5.\,1853 Berlin – 11.\,8.\,1904 ebd.@\textsc{Meyer-Cohn, Alexander} (1.\,5.\,1853 Berlin – 11.\,8.\,1904 ebd.), \emph{Bankier}|pwv}} vollkommen auseinander. Dieſe ganze Geſchichte hat für mich mit einem großen
               Ekel geendet, – einem Ekel namentlich vor der »Geſellſchaft«, vor dieſen Leuten, die
               Einen nicht verſtehen und die Einen zur Tafel ziehen als Hanswurſt. Aber wehe, wenn
               man verſuchen will, auch einmal{ }ſein Leben zu leben! {\pb}Im Übrigen hat die Kleine\pwindex{Meyer-Cohn, Helene 30.\,12.\,1859 Lviv – 9.\,11.\,1918 Berlin@\textsc{Meyer-Cohn, Helene} (30.\,12.\,1859 Lviv – 9.\,11.\,1918 Berlin), \emph{Übersetzerin}|pwv}
               ja ganz recht gehabt, und ich bin fett und grotesk und nicht fähig, Liebe \strikeout{zu \textcolor{gray}{e}i} einzuflößen. Ich habe mich in
               die Arbeit geſtürzt, um das Alles zu vergeſſen.\pend
           
\pstart
           \textsc{Brandes\pwindex{Brandes, Georg 4.\,2.\,1842 Kopenhagen – 19.\,2.\,1927 ebd.@\textsc{Brandes, Georg} (4.\,2.\,1842 Kopenhagen – 19.\,2.\,1927 ebd.)|pw}} iſt hier\oindex{Berlin@\textbf{Berlin}, \emph{Hauptstadt}|pwv} und erzählt mir
               viel von{ }ſeinen \label{K_L02936-7v}\edtext{Liebesabenteuern}{\lemma{\textnormal{\emph{Liebesabenteuern}}}\Cendnote{\textnormal{Vgl. XXXX Auszeichnungsfehler: Dokument L02934 nicht gefunden. }}}\label{K_L02936-7}. Dieſer Tage
               kommt auch{ }ſeine Tochter\pwindex{Philipp, Edith 17.\,1.\,1879 Berlin – 16.\,2.\,1968 Kopenhagen@\textsc{Philipp, Edith} (17.\,1.\,1879 Berlin – 16.\,2.\,1968 Kopenhagen)|pwv}.\pend
           
\pstart
           Nach Breslau\oindex{Breslau@\textbf{Breslau}|pw} zur \label{K_L02936-8v}\edtext{Aufführung\eventindex{Lobe-Theater@\textbf{Lobe-Theater}!Uraufführung von Der Schleier der Beatrice, 1.12.1900@Uraufführung von Der Schleier der Beatrice, 1.12.1900|pwv} der »\textsc{Beatrice\pwindex{Schnitzler, Arthur 15.\,5.\,1862 Wien – 21.\,10.\,1931 ebd.@\textsc{Schnitzler, Arthur} (15.\,5.\,1862 Wien – 21.\,10.\,1931 ebd.), \emph{Schriftsteller, Mediziner}!Schleier der Beatrice. Schauspiel in fünf Akten@\strich\emph{Der Schleier der Beatrice. Schauspiel in fünf Akten}|pw}}«}{\lemma{\textnormal{\emph{Aufführung der »Beatrice«}}}\Cendnote{\textnormal{\emph{Der Schleier der Beatrice}\pwindex{Schnitzler, Arthur 15.\,5.\,1862 Wien – 21.\,10.\,1931 ebd.@\textsc{Schnitzler, Arthur} (15.\,5.\,1862 Wien – 21.\,10.\,1931 ebd.), \emph{Schriftsteller, Mediziner}!Schleier der Beatrice. Schauspiel in fünf Akten@\strich\emph{Der Schleier der Beatrice. Schauspiel in fünf Akten}|pwk} wurde am 1. 12. 1900 am Lobe-Theater\oindex{Lobe-Theater@\textbf{Lobe-Theater}, \emph{Theater}|pwk} in Breslau\oindex{Breslau@\textbf{Breslau}|pwk}{ }uraufgeführt\eventindex{Lobe-Theater@\textbf{Lobe-Theater}!Uraufführung von Der Schleier der Beatrice, 1.12.1900@Uraufführung von Der Schleier der Beatrice, 1.12.1900|pwkv}. Zu einem früheren
                  Zeitpunkt war der 17. 11. 1900 als Premierentermin
                  geplant.}}}\label{K_L02936-8} möchte ich unendlich gern fahren. Ich habe das hier mit meinem
               Collegen \textsc{Fuchs\pwindex{Fuchs, Isidor 25.\,9.\,1849 Lipnik Górny – um den 20.8.1920 Schruns@\textsc{Fuchs, Isidor} (25.\,9.\,1849 Lipnik Górny – um den 20.8.1920 Schruns), \emph{Schriftsteller, Journalist}|pwuv}} beſprochen, und {\pb}er{ }ſagte mir: »Ja, fahren Sie
               nur! Aber den Direktor \textsc{Löwe\pwindex{Loewe, Theodor 1.\,1.\,1855 Wien – 1935 Breslau@\textsc{Loewe, Theodor} (1.\,1.\,1855 Wien – 1935 Breslau), \emph{Theaterleiter}|pw}} dürfen Sie nicht tadeln; er iſt bei uns\orgindex{Neue Freie Presse@Neue Freie Presse|pwv}{ }\label{K_L02936-9v}\edtext{\textsc{persona gratissima}}{\lemma{\textnormal{\emph{persona gratissima}}}\Cendnote{\textnormal{lateinisch: willkommene Person, hier im
                  Sinne von ›immun‹}}}\label{K_L02936-9}.« Alſo, ich{ }ſetze den Fall, die Aufführung\eventindex{Lobe-Theater@\textbf{Lobe-Theater}!Uraufführung von Der Schleier der Beatrice, 1.12.1900@Uraufführung von Der Schleier der Beatrice, 1.12.1900|pwv} könnte den
               Aufgaben des Stück\pwindex{Schnitzler, Arthur 15.\,5.\,1862 Wien – 21.\,10.\,1931 ebd.@\textsc{Schnitzler, Arthur} (15.\,5.\,1862 Wien – 21.\,10.\,1931 ebd.), \emph{Schriftsteller, Mediziner}!Schleier der Beatrice. Schauspiel in fünf Akten@\strich\emph{Der Schleier der Beatrice. Schauspiel in fünf Akten}|pwv}es nicht
               gerecht werden (was ich befürchte),{ }ſo werde ich das nicht{ }ſagen dürfen, oder man
               wird es mir{ }ſtreichen. Unter dieſen Umſtänden iſt es wirklich beſſer, nicht
               hinzugehen und die Berichterſtattung dem Direktor \textsc{Löwe\pwindex{Loewe, Theodor 1.\,1.\,1855 Wien – 1935 Breslau@\textsc{Loewe, Theodor} (1.\,1.\,1855 Wien – 1935 Breslau), \emph{Theaterleiter}|pw}} zu überlaſſen, der{ }ſelbſt an die N. Fr. Pr.\orgindex{Neue Freie Presse@Neue Freie Presse|pw}{ }{\pb}zu telegraphiren pflegt und unter allen Umſtänden
                  \label{K_L02936-10v}\edtext{das Beſte{ }ſagen wird}{\lemma{\textnormal{\emph{das Beste sagen wird}}}\Cendnote{\textnormal{Siehe zur Berichterstattung über die Uraufführung von \emph{Der Schleier der Beatrice}\pwindex{Schnitzler, Arthur 15.\,5.\,1862 Wien – 21.\,10.\,1931 ebd.@\textsc{Schnitzler, Arthur} (15.\,5.\,1862 Wien – 21.\,10.\,1931 ebd.), \emph{Schriftsteller, Mediziner}!Schleier der Beatrice. Schauspiel in fünf Akten@\strich\emph{Der Schleier der Beatrice. Schauspiel in fünf Akten}|pwk}\eventindex{Lobe-Theater@\textbf{Lobe-Theater}!Uraufführung von Der Schleier der Beatrice, 1.12.1900@Uraufführung von Der Schleier der Beatrice, 1.12.1900|pwkv} in der \emph{Neuen Freien Presse}\pwindex{Neue Freie Presse@\emph{Neue Freie Presse}|pwk}{ }Goldmanns\pwindex{Goldmann, Paul 31.\,1.\,1865 Breslau – 25.\,9.\,1935 Wien@\textsc{Goldmann, Paul} (31.\,1.\,1865 Breslau – 25.\,9.\,1935 Wien), \emph{Schriftsteller, Journalist}|pwk} Briefe vom XXXX Auszeichnungsfehler: Dokument L02937 nicht gefunden und vom XXXX Auszeichnungsfehler: Dokument L02943 nicht gefunden.}}}\label{K_L02936-10}.\pend
           
\pstart
           Grüße mir die{ }ſtrebſamen \label{K_L02936-11v}\edtext{Fräulein\pwindex{Schnitzler, Olga 17.\,1.\,1882 Wien – 13.\,1.\,1970 Lugano@\textsc{Schnitzler, Olga} (17.\,1.\,1882 Wien – 13.\,1.\,1970 Lugano), \emph{Schauspielerin, Sängerin}|pwv}\pwindex{Steinrück, Elisabeth 19.\,11.\,1885 – 7.\,4.\,1920 Partenkirchen@\textsc{Steinrück, Elisabeth} (19.\,11.\,1885 – 7.\,4.\,1920 Partenkirchen)|pwv} aus der Rothen-Stern-Gaſſe\oindex{Wien@\textbf{Wien}!II., Leopoldstadt@\textbf{II., Leopoldstadt}!Rotensterngasse@\textbf{Rotensterngasse}, \emph{Straße}|pw}}{\lemma{\textnormal{\emph{Fräulein … Rothen-Stern-Gasse}}}\Cendnote{\textnormal{Siehe XXXX Auszeichnungsfehler: Dokument L02931 nicht gefunden. }}}\label{K_L02936-11} und theile
               mir deren genaue Adreſſe mit (Name und Hausnummer), damit ich ihnen mein \label{K_L02936-12v}\edtext{Buch\pwindex{Goldmann, Paul 31.\,1.\,1865 Breslau – 25.\,9.\,1935 Wien@\textsc{Goldmann, Paul} (31.\,1.\,1865 Breslau – 25.\,9.\,1935 Wien), \emph{Schriftsteller, Journalist}!Sommer in China. Reisebilder. Zweite, durchgesehene und vermehrte Auflage@\strich\emph{Ein Sommer in China. Reisebilder. Zweite, durchgesehene und vermehrte Auflage}|pwv}}{\lemma{\textnormal{\emph{Buch}}}\Cendnote{\textnormal{die zweite Auflage\pwindex{Goldmann, Paul 31.\,1.\,1865 Breslau – 25.\,9.\,1935 Wien@\textsc{Goldmann, Paul} (31.\,1.\,1865 Breslau – 25.\,9.\,1935 Wien), \emph{Schriftsteller, Journalist}!Sommer in China. Reisebilder. Zweite, durchgesehene und vermehrte Auflage@\strich\emph{Ein Sommer in China. Reisebilder. Zweite, durchgesehene und vermehrte Auflage}|pwkv} von \emph{Ein
                     Sommer in China}\pwindex{Goldmann, Paul 31.\,1.\,1865 Breslau – 25.\,9.\,1935 Wien@\textsc{Goldmann, Paul} (31.\,1.\,1865 Breslau – 25.\,9.\,1935 Wien), \emph{Schriftsteller, Journalist}!Sommer in China. Reisebilder@\strich\emph{Ein Sommer in China. Reisebilder}|pwk}, vgl. XXXX Auszeichnungsfehler: Dokument L02934 nicht gefunden.}}}\label{K_L02936-12}{ }ſchicken kann.\pend
           
\pstart
           Die \textsc{Glümerinnen\pwindex{Glümer, Marie 3.\,7.\,1867 Wien – 16.\,11.\,1925 München@\textsc{Glümer, Marie} (3.\,7.\,1867 Wien – 16.\,11.\,1925 München), \emph{Schauspielerin}|pwv}\pwindex{Glümer, Auguste 16.\,3.\,1862 Wien – 1956@\textsc{Glümer, Auguste} (16.\,3.\,1862 Wien – 1956), \emph{Lehrerin}|pwv}}{ }ſind wieder beieinander, und Frl. \textsc{Mizzi\pwindex{Glümer, Marie 3.\,7.\,1867 Wien – 16.\,11.\,1925 München@\textsc{Glümer, Marie} (3.\,7.\,1867 Wien – 16.\,11.\,1925 München), \emph{Schauspielerin}|pw}} hat neulich {\pb}einen{ }ſehr \strikeout{ſchöne\textcolor{gray}{nn}}{ }ſchönen und{ }ſehr verdienten \label{K_L02936-13v}\edtext{Erfolg}{\lemma{\textnormal{\emph{Erfolg}}}\Cendnote{\textnormal{als weibliche Hauptrolle\pwindex{Glümer, Marie 3.\,7.\,1867 Wien – 16.\,11.\,1925 München@\textsc{Glümer, Marie} (3.\,7.\,1867 Wien – 16.\,11.\,1925 München), \emph{Schauspielerin}|pwkv} der Berlin\oindex{Berlin@\textbf{Berlin}, \emph{Hauptstadt}|pwk}er \emph{Secessionsbühne}\orgindex{Secessionsbühne@Secessionsbühne|pwk} in \emph{Die Bildschnitzer}\pwindex{Schönherr, Karl 24.\,2.\,1867 Axams – 15.\,3.\,1943 Wien@\textsc{Schönherr, Karl} (24.\,2.\,1867 Axams – 15.\,3.\,1943 Wien), \emph{Schriftsteller, Mediziner}!Bildschnitzer@\strich\emph{Die Bildschnitzer}|pwk}
                  von Karl Schönherr\pwindex{Schönherr, Karl 24.\,2.\,1867 Axams – 15.\,3.\,1943 Wien@\textsc{Schönherr, Karl} (24.\,2.\,1867 Axams – 15.\,3.\,1943 Wien), \emph{Schriftsteller, Mediziner}|pwk} und in \emph{Der Bär}\pwindex{Čechov, Anton Pavlovič 17.\,1.\,1860 Taganrog – 15.\,7.\,1904 Badenweiler@\textsc{Čechov, Anton Pavlovič} (17.\,1.\,1860 Taganrog – 15.\,7.\,1904 Badenweiler), \emph{Schriftsteller}!Bär@\strich\emph{Der Bär}|pwk} von Anton
                     Čechov\pwindex{Čechov, Anton Pavlovič 17.\,1.\,1860 Taganrog – 15.\,7.\,1904 Badenweiler@\textsc{Čechov, Anton Pavlovič} (17.\,1.\,1860 Taganrog – 15.\,7.\,1904 Badenweiler), \emph{Schriftsteller}|pwk}}}}\label{K_L02936-13} gehabt bei Publikum und Kritik. Auch{ }ſie{ }ſehe ich{ }ſelten, und ich lebe,
               eingeſponnen in Arbeit, ein ödes und nutzloſes Leben.\pend
           
\pstart
           Was macht \textsc{Richard\pwindex{Beer-Hofmann, Richard 11.\,7.\,1866 Wien – 26.\,9.\,1945 New York City@\textsc{Beer-Hofmann, Richard} (11.\,7.\,1866 Wien – 26.\,9.\,1945 New York City), \emph{Schriftsteller}|pw}}? Keine Möglichkeit, von ihm eine Antwort zu bekommen.\pend
           
\pstart
           Schreib’ mir bald und \strikeout{ſei}{ }ſei von Herzen
               gegrüßt!{\\[\baselineskip]} Dein {\\[\baselineskip]}\spacefill\mbox{Paul Goldmnn}\pend
           \leftskip=0em{}\selectlanguage{ngerman}\endnumbering\briefempfaengerindex{Schnitzler, Arthur@\textsc{Schnitzler, Arthur}!zzzGoldmann, Paul@\emph{von Paul Goldmann}!1900-10-141@{14. 10. [1900]}|)be}\mylabel{L02936h}  \newcommand{\dateiname}{L02936}\newcommand{\titel}{Paul Goldmann an Arthur Schnitzler, 14. 10. [1900]}\newcommand{\editorInnen}{Martin Anton Müller und Laura Untner}%% latex-leseansicht-abspann.tex
%% Abspann für die Leseansicht.
%% Der Schalter \ifkorrekturansicht ist bereits durch den Vorspann gesetzt.

%% latex-abspann.tex
%% Gemeinsamer Abspann für Korrekturansicht und Leseansicht.
%% Setzt den Schalter \ifkorrekturansicht voraus (gesetzt in den
%% einbindenden Dateien latex-korrekturansicht-abspann.tex bzw.
%% latex-leseansicht-abspann.tex).
%% ---------------------------------------------------------------

\normalsize

% Das esempio-Environment wird nur in der Leseansicht benötigt
\ifkorrekturansicht\else
\newenvironment{esempio}[3]%
{
    \vspace{1.5ex}
    \rlap{\underline{#1}}
    \par
    \setlength{\parindent}{0cm}
    \nopagebreak
    \leftskip=#2cm
    \rightskip=#3cm
}
{
    \par
}
\fi

\doendnotes{C}
\bigskip
\vfill

\clearpage

\footnotesize

\ifkorrekturansicht
  \lohead{\textsc{register}}
\fi

% theindex-Environment neu definieren ohne reledmac
\makeatletter
\renewenvironment{theindex}{%
  \ifkorrekturansicht
    \section*{\indexname}%
  \else
    \subsubsection*{Index der erwähnten Entitäten}%
  \fi
  \setlength{\parindent}{0pt}%
  \setlength{\parskip}{0pt plus 0.3pt}%
  \let\item\@idxitem
}{%
  \ifkorrekturansicht\clearpage\fi
}
\makeatother

\IfFileExists{\jobname-pw.ind}{\input{\jobname-pw.ind}}{}

% Quellenangabe nur in der Leseansicht
\ifkorrekturansicht\else
% Fallback-Definitionen, falls die .tex-Datei \titel etc. nicht gesetzt hat
\providecommand{\titel}{}
\providecommand{\editorInnen}{}
\providecommand{\dateiname}{\jobname}

\vspace{3cm}

\vfill

\footnotesize
\textsc{Quelle}: \titel. Herausgegeben von {\editorInnen}. In: \emph{Arthur Schnitzler: Briefwechsel mit Autorinnen und Autoren}.
 Digitale Edition, https://schnitzler-briefe.acdh.oeaw.ac.at/{\dateiname}.html (Stand \today)
\fi

\end{document}


