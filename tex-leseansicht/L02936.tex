%% latex-leseansicht-vorspann.tex
%% Vorspann für die Leseansicht.
%% Lädt die gemeinsame Datei latex-vorspann.tex mit nicht gesetztem Schalter.

\newif\ifkorrekturansicht
\korrekturansichtfalse

\input{../tex-inputs/latex-vorspann}

\begin{center}
            \textcolor{red}{ENTWURF, NICHT FERTIG KORRIGIERT}
                      \end{center}
            
         
         \renewcommand{\erwaehntePersonen}{Personen: Hermann Bahr, Julius Bauer, Richard Beer-Hofmann, Georg Brandes, Max Eugen Burckhard, Auguste Chlum, Jakob Julius David, Isidor Fuchs, Marie Glümer, Robert Hirschfeld, Alfred Kerr, Karl Kraus, Paul Lindau, Theodor Loewe, Edith Philipp, Felix Salten, Paul Schlenther, Olga Schnitzler, Karl Schönherr, Ludwig Speidel, Elisabeth Steinrück, Anna Wendt, Anton Pavlovič Čechov}
         \renewcommand{\erwaehnteInstitutionen}{Institutionen: Burgtheater, Neue Freie Presse, Secessionsbühne}
         \renewcommand{\erwaehnteOrte}{Orte: Baden bei Wien, Berlin, Breslau, Lobe-Theater, Rotensterngasse, Wien}
         \renewcommand{\erwaehnteWerke}{Werke: Antworten des Herausgebers. Habitué, Der Bär, Der Gemeine. Schauspiel in drei Aufzügen, Der Schleier der Beatrice. Schauspiel in fünf Akten, Die Bildschnitzer, Die Fackel, Die Zeit. Wiener Wochenschrift, Ein Sommer in China. Reisebilder, Ein Sommer in China. Reisebilder. Zweite, durchgesehene und vermehrte Auflage, Erklärung [Schleier der Beatrice], Neue Freie Presse, [Burckhard über Schnitzler-Schlenther], [Die Affaire Schlenther-Schnitzler]}
               \section[ Paul Goldmann an Arthur Schnitzler, 14. 10. {[}1900{]}]{ Paul Goldmann an Arthur Schnitzler, 14. 10. {[}1900{]}}\nopagebreak\mylabel{v}\rehead{ }\begin{ledgroupsized}[t]{13cm}\normalsize\beginnumbering \toendnotes[C]{\smallbreak\pagebreak[2]} \Standort{DLA, A:Schnitzler, HS.NZ85.1.3170.}
\physDesc{Brief, 2 Blätter, 8 Seiten
\newline{}Handschrift: blaue Tinte, deutsche Kurrent
\newline{}Schnitzler: 1) mit Bleistift das Jahr »{[}1{]}900« vermerkt  2) mit rotem Buntstift acht Unterstreichungen und ein
                                    »X«}\toendnotes[C]{\smallbreak}\pstart
           \raggedleft{}{\pb}Berlin\oindex{Berlin@\textbf{Berlin}|pw}, 14. Oktober.\pend
           \pstart\center{}Mein lieber Freund,\pend\pstart
           Heut am Sonntag habe
               ich endlich ein paar Minuten frei zu einem Briefe an Dich.\pend
           \pstart
           Die \label{K_L02936-1v}\edtext{»Fackel\pwindex{Fackel1899-04 – 1936@\emph{Die Fackel} {[}1899-04 – 1936{]}|pwv}\pwindex{Kraus, Karl 28.04.1874 – 12.06.1936@\textsc{Kraus, Karl} (28.04.1874 – 12.06.1936), \emph{Schriftsteller, Publizist}!Affaire Schlenther-Schnitzler]1900-09-15@\strich\emph{[Die Affaire Schlenther-Schnitzler]} {[}1900-09-15{]}|pwv}\pwindex{Kraus, Karl 28.04.1874 – 12.06.1936@\textsc{Kraus, Karl} (28.04.1874 – 12.06.1936), \emph{Schriftsteller, Publizist}!Antworten des Herausgebers. Habitue1900-09-30@\strich\emph{Antworten des Herausgebers. Habitué} {[}1900-09-30{]}|pwv}\pwindex{Fackel1899-04 – 1936@\emph{Die Fackel} {[}1899-04 – 1936{]}|pw}«}{\lemma{\textnormal{\emph{»Fackel«}}}\Cendnote{\textnormal{Bezug auf Karl Kraus\pwindex{Kraus, Karl 28.04.1874 – 12.06.1936@\textsc{Kraus, Karl} (28.04.1874 – 12.06.1936), \emph{Schriftsteller, Publizist}|pwk}: \emph{[Die Affaire Schlenther-Schnitzler]}\pwindex{Kraus, Karl 28.04.1874 – 12.06.1936@\textsc{Kraus, Karl} (28.04.1874 – 12.06.1936), \emph{Schriftsteller, Publizist}!Affaire Schlenther-Schnitzler]1900-09-15@\strich\emph{[Die Affaire Schlenther-Schnitzler]} {[}1900-09-15{]}|pwk}. In: \emph{Die Fackel}\pwindex{Fackel1899-04 – 1936@\emph{Die Fackel} {[}1899-04 – 1936{]}|pwk}, Jg. 2, Nr. 53, Mitte September 1900, S. 1–6, möglicherweise
                  auch auf Karl Kraus\pwindex{Kraus, Karl 28.04.1874 – 12.06.1936@\textsc{Kraus, Karl} (28.04.1874 – 12.06.1936), \emph{Schriftsteller, Publizist}|pwk}: \emph{Antworten des Herausgebers. Habitué}\pwindex{Kraus, Karl 28.04.1874 – 12.06.1936@\textsc{Kraus, Karl} (28.04.1874 – 12.06.1936), \emph{Schriftsteller, Publizist}!Antworten des Herausgebers. Habitue1900-09-30@\strich\emph{Antworten des Herausgebers. Habitué} {[}1900-09-30{]}|pwk}. In: \emph{Die Fackel}\pwindex{Fackel1899-04 – 1936@\emph{Die Fackel} {[}1899-04 – 1936{]}|pwk}, Jg. 2, Nr. 54, Ende September 1900, S. 25–26. Siehe zum
                  Konflikt zwischen Schnitzler\pwindex{Schnitzler, Arthur 15.05.1862 – 21.10.1931@\textsc{Schnitzler, Arthur} (15.05.1862 – 21.10.1931), \emph{Schriftsteller, Mediziner}|pwk} und Paul Schlenther\pwindex{Schlenther, Paul 20.08.1854 – 30.04.1916@\textsc{Schlenther, Paul} (20.08.1854 – 30.04.1916), \emph{Schriftsteller, Kritiker, Theaterleiter}|pwk} auch 12. 11. [1899].}}}\label{K_L02936-1h}. Was
               willſt Du von dem Lausbuben\pwindex{Kraus, Karl 28.04.1874 – 12.06.1936@\textsc{Kraus, Karl} (28.04.1874 – 12.06.1936), \emph{Schriftsteller, Publizist}|pwv}?
               Offen geſtanden, ich hätte noch Schlimmeres erwartet. Im Übrigen hat \label{K_L02936-2v}\edtext{\textsc{Burckhardt\pwindex{Burckhard, Max Eugen 14.07.1854 – 16.03.1912@\textsc{Burckhard, Max Eugen} (14.07.1854 – 16.03.1912), \emph{Schriftsteller, Rechtswissenschaftler, Theaterleiter}|pw}} in der »Zeit\pwindex{Zeit. Wiener Wochenschrift1894 – 1904@\emph{Die Zeit. Wiener Wochenschrift} {[}1894 – 1904{]}|pw}«}{\lemma{\textnormal{\emph{Burckhardt in der »Zeit«}}}\Cendnote{\textnormal{XXXX bibl Burckhard in der Zeit über AS-Schlenther}}}\label{K_L02936-2h} das
               wahre Wort geſchrieben\pwindex{Burckhard, Max Eugen 14.07.1854 – 16.03.1912@\textsc{Burckhard, Max Eugen} (14.07.1854 – 16.03.1912), \emph{Schriftsteller, Rechtswissenschaftler, Theaterleiter}!Burckhard ueber Schnitzler-Schlenther]1900@\strich\emph{[Burckhard über Schnitzler-Schlenther]} {[}1900{]}|pwv}: die
               Leute rächen ſich jetzt an Dir, weil ſie Dir haben applaudiren müſſen. Auf das
               Geſindel im Allgemeinen war niemals zu rechnen. Ob die \label{K_L02936-3v}\edtext{Aktion\pwindex{Bahr, Hermann 19.07.1863 – 15.01.1934@\textsc{Bahr, Hermann} (19.07.1863 – 15.01.1934), \emph{Schriftsteller, Kritiker}!Erklaerung [Schleier der Beatrice]1900-09-14@\strich\emph{Erklärung [Schleier der Beatrice]} {[}1900-09-14{]}|pwv}\pwindex{Salten, Felix 06.09.1869 – 08.10.1945@\textsc{Salten, Felix} (06.09.1869 – 08.10.1945), \emph{Schriftsteller, Journalist}!Erklaerung [Schleier der Beatrice]1900-09-14@\strich\emph{Erklärung [Schleier der Beatrice]} {[}1900-09-14{]}|pwv}\pwindex{Bauer, Julius 15.10.1853 – 11.06.1941@\textsc{Bauer, Julius} (15.10.1853 – 11.06.1941), \emph{Schriftsteller, Journalist, Kritiker}!Erklaerung [Schleier der Beatrice]1900-09-14@\strich\emph{Erklärung [Schleier der Beatrice]} {[}1900-09-14{]}|pwv}\pwindex{Hirschfeld, Robert 17.09.1857 – 02.04.1914@\textsc{Hirschfeld, Robert} (17.09.1857 – 02.04.1914), \emph{Journalist, Musikkritiker}!Erklaerung [Schleier der Beatrice]1900-09-14@\strich\emph{Erklärung [Schleier der Beatrice]} {[}1900-09-14{]}|pwv}\pwindex{Speidel, Ludwig 1830-04-11 – 1906-02-03@\textsc{Speidel, Ludwig} (1830-04-11 – 1906-02-03), \emph{Journalist, Kritiker}!Erklaerung [Schleier der Beatrice]1900-09-14@\strich\emph{Erklärung [Schleier der Beatrice]} {[}1900-09-14{]}|pwv}\pwindex{David, Jakob Julius 1859-02-06 – 1906-11-20@\textsc{David, Jakob Julius} (1859-02-06 – 1906-11-20), \emph{Schriftsteller, Journalist}!Erklaerung [Schleier der Beatrice]1900-09-14@\strich\emph{Erklärung [Schleier der Beatrice]} {[}1900-09-14{]}|pwv}}{\lemma{\textnormal{\emph{Aktion}}}\Cendnote{\textnormal{Bezug auf das Protestschreiben\pwindex{Bahr, Hermann 19.07.1863 – 15.01.1934@\textsc{Bahr, Hermann} (19.07.1863 – 15.01.1934), \emph{Schriftsteller, Kritiker}!Erklaerung [Schleier der Beatrice]1900-09-14@\strich\emph{Erklärung [Schleier der Beatrice]} {[}1900-09-14{]}|pwkv}\pwindex{Salten, Felix 06.09.1869 – 08.10.1945@\textsc{Salten, Felix} (06.09.1869 – 08.10.1945), \emph{Schriftsteller, Journalist}!Erklaerung [Schleier der Beatrice]1900-09-14@\strich\emph{Erklärung [Schleier der Beatrice]} {[}1900-09-14{]}|pwkv}\pwindex{Bauer, Julius 15.10.1853 – 11.06.1941@\textsc{Bauer, Julius} (15.10.1853 – 11.06.1941), \emph{Schriftsteller, Journalist, Kritiker}!Erklaerung [Schleier der Beatrice]1900-09-14@\strich\emph{Erklärung [Schleier der Beatrice]} {[}1900-09-14{]}|pwkv}\pwindex{Hirschfeld, Robert 17.09.1857 – 02.04.1914@\textsc{Hirschfeld, Robert} (17.09.1857 – 02.04.1914), \emph{Journalist, Musikkritiker}!Erklaerung [Schleier der Beatrice]1900-09-14@\strich\emph{Erklärung [Schleier der Beatrice]} {[}1900-09-14{]}|pwkv}\pwindex{Speidel, Ludwig 1830-04-11 – 1906-02-03@\textsc{Speidel, Ludwig} (1830-04-11 – 1906-02-03), \emph{Journalist, Kritiker}!Erklaerung [Schleier der Beatrice]1900-09-14@\strich\emph{Erklärung [Schleier der Beatrice]} {[}1900-09-14{]}|pwkv}\pwindex{David, Jakob Julius 1859-02-06 – 1906-11-20@\textsc{David, Jakob Julius} (1859-02-06 – 1906-11-20), \emph{Schriftsteller, Journalist}!Erklaerung [Schleier der Beatrice]1900-09-14@\strich\emph{Erklärung [Schleier der Beatrice]} {[}1900-09-14{]}|pwkv} von Hermann Bahr\pwindex{Bahr, Hermann 19.07.1863 – 15.01.1934@\textsc{Bahr, Hermann} (19.07.1863 – 15.01.1934), \emph{Schriftsteller, Kritiker}|pwk}, Julius
                     Bauer\pwindex{Bauer, Julius 15.10.1853 – 11.06.1941@\textsc{Bauer, Julius} (15.10.1853 – 11.06.1941), \emph{Schriftsteller, Journalist, Kritiker}|pwk}, J. J. David\pwindex{David, Jakob Julius 1859-02-06 – 1906-11-20@\textsc{David, Jakob Julius} (1859-02-06 – 1906-11-20), \emph{Schriftsteller, Journalist}|pwk}, Robert Hirschfeld\pwindex{Hirschfeld, Robert 17.09.1857 – 02.04.1914@\textsc{Hirschfeld, Robert} (17.09.1857 – 02.04.1914), \emph{Journalist, Musikkritiker}|pwk}, Felix Salten\pwindex{Salten, Felix 06.09.1869 – 08.10.1945@\textsc{Salten, Felix} (06.09.1869 – 08.10.1945), \emph{Schriftsteller, Journalist}|pwk} und Ludwig
                     Speidel\pwindex{Speidel, Ludwig 1830-04-11 – 1906-02-03@\textsc{Speidel, Ludwig} (1830-04-11 – 1906-02-03), \emph{Journalist, Kritiker}|pwk}}}}\label{K_L02936-3h} ſonſt wirkungslos geblieben, wird ſich zeigen. Welche Wirkung hätte {\pb}denn auch kommen ſollen? Die Hauptſache war, daß der
               Herr \textsc{Schlenther\pwindex{Schlenther, Paul 20.08.1854 – 30.04.1916@\textsc{Schlenther, Paul} (20.08.1854 – 30.04.1916), \emph{Schriftsteller, Kritiker, Theaterleiter}|pw}} eine Antwort auf ſein unerhörtes Benehmen bekam. Und den ſchlechten Ruf, den er
               ohnedies hat, hat dieſe Affaire nur noch vergrößert. Er hat’s geſpürt und wirds noch
               weiter ſpüren. Dieſe Affaire, mag man ſagen, was man will, iſt ein Grund mehr für
               ſeinen Weggang vom Burgtheater\orgindex{Burgtheater@Burgtheater|pw}. Selbſt hier\oindex{Berlin@\textbf{Berlin}|pwv}, wo man ihn für einen Gott
               hält, hat ſie ihm geſchadet{\dotsfive}\pend
           \pstart
           Dein \label{K_L02936-4v}\edtext{»Ohrenleiden«}{\lemma{\textnormal{\emph{»Ohrenleiden«}}}\Cendnote{\textnormal{Bezug auf Schnitzler\pwindex{Schnitzler, Arthur 15.05.1862 – 21.10.1931@\textsc{Schnitzler, Arthur} (15.05.1862 – 21.10.1931), \emph{Schriftsteller, Mediziner}|pwk}s Otosklerose (Verknöcherung des Innenohrs mit zunehmender
                  Schwerhörigkeit)}}}\label{K_L02936-4h}: Darauf weiß ich nur eine Antwort: Heirathen. Ich ſchwöre
               Dir: wenn Du Frau {\pb}und Kinder haben wirſt, wirſt Du
               Dich weniger mit Deinem Ohre beſchäftigen; und wenn Du Dich weniger damit
               beſchäftigen wirſt, \strikeout{\textcolor{gray}{r}\textcolor{gray}{×}\-\textcolor{gray}{×}} wirſt Du weniger darunter leiden.\pend
           \pstart
           Mit \textsc{Lindau\pwindex{Lindau, Paul 03.06.1839 – 31.01.1919@\textsc{Lindau, Paul} (03.06.1839 – 31.01.1919), \emph{Schriftsteller, Kritiker, Theaterleiter}|pw}} werde ich bei nächſter Gelegenheit \label{K_L02936-5v}\edtext{wegen \textsc{Salten\pwindex{Salten, Felix 06.09.1869 – 08.10.1945@\textsc{Salten, Felix} (06.09.1869 – 08.10.1945), \emph{Schriftsteller, Journalist}|pw}}}{\lemma{\textnormal{\emph{wegen Salten}}}\Cendnote{\textnormal{Bezug unklar, womöglich ging es um die
                  Uraufführung von Salten\pwindex{Salten, Felix 06.09.1869 – 08.10.1945@\textsc{Salten, Felix} (06.09.1869 – 08.10.1945), \emph{Schriftsteller, Journalist}|pwk}s Dreiakter \emph{Der Gemeine}\pwindex{Salten, Felix 06.09.1869 – 08.10.1945@\textsc{Salten, Felix} (06.09.1869 – 08.10.1945), \emph{Schriftsteller, Journalist}!Gemeine. Schauspiel in drei Aufzuegen1901@\strich\emph{Der Gemeine. Schauspiel in drei Aufzügen} {[}1901{]}|pwk}, die in Wien\oindex{Wien@\textbf{Wien}|pwk} verboten wurde}}}\label{K_L02936-5h} ſprechen.\pend
           \pstart
           \textsc{Kerr\pwindex{Kerr, Alfred 25.12.1867 – 12.10.1948@\textsc{Kerr, Alfred} (25.12.1867 – 12.10.1948), \emph{Schriftsteller, Kritiker}|pw}} ſehe ich ſehr ſelten. Wenn wir uns ſehen, ſprechen wir ſehr freundſchaftlich
               miteinander. Er ſteckt tief in ſeinem \label{K_L02936-6v}\edtext{Liebes\textcolor{gray}{wonnen}}{\lemma{\textnormal{\emph{Liebeswonnen}}}\Cendnote{\textnormal{siehe Paul Goldmann an Arthur Schnitzler, 18. 4. [1900]}}}\label{K_L02936-6h} und ſtrebt der Erfüllung ſeiner Wünſche zu, was mit großen {\pb}Kämpfen verbunden ſcheint. Aber er\pwindex{Kerr, Alfred 25.12.1867 – 12.10.1948@\textsc{Kerr, Alfred} (25.12.1867 – 12.10.1948), \emph{Schriftsteller, Kritiker}|pwv} wird es ſchon durchſetzen. Er und das
                  Mädel\pwindex{Wendt, Anna @\textsc{Wendt, Anna}|pwv} ſcheinen ſich ſehr
               zu lieben, und das iſt die Hauptſache.\pend
           \pstart
           Ich bin mit dem Hauſe \textsc{M.-G.\pwindex{Gluemer, Marie 03.07.1867 – 16.11.1925@\textsc{Glümer, Marie} (03.07.1867 – 16.11.1925), \emph{Schauspielerin}|pwv}\pwindex{Chlum, Auguste 16.03.1862 – 1956@\textsc{Chlum, Auguste} (16.03.1862 – 1956)|pwv}} vollkommen auseinander. Dieſe ganze Geſchichte hat für mich mit einem großen
               Ekel geendet, einem Ekel namentlich vor der »Geſellſchaft«, vor dieſen Leuten, die
               Einen nicht verſtehen und die Einen zur Tafel ziehen als \label{K_L02936-7v}\edtext{Hanswurſt}{\lemma{\textnormal{\emph{Hanswurſt}}}\Cendnote{\textnormal{komische, lächerliche, derbe Person, als Figur aus der deutschsprachigen Komödie
                  seit dem 16. Jahrhundert bekannt}}}\label{K_L02936-7h}. Aber wehe, wenn man
               verſuchen will, auch einmal ſein Leben zu leben! {\pb}Im
               Übrigen hat die Kleine\pwindex{Gluemer, Marie 03.07.1867 – 16.11.1925@\textsc{Glümer, Marie} (03.07.1867 – 16.11.1925), \emph{Schauspielerin}|pwv} ja
               ganz recht gehabt, und ich bin fett und grotesk und nicht fähig, Liebe \strikeout{zu \textcolor{gray}{e}i} einzuflößen. Ich habe mich in
               die Arbeit geſtürzt, um das Alles zu vergeſſen.\pend
           \pstart
           \textsc{Brandes\pwindex{Brandes, Georg 04.02.1842 – 19.02.1927@\textsc{Brandes, Georg} (04.02.1842 – 19.02.1927)|pw}} iſt hier\oindex{Berlin@\textbf{Berlin}|pwv} und erzählt mir
               viel von ſeinen \label{K_L02936-9v}\edtext{Liebesabenteuern}{\lemma{\textnormal{\emph{Liebesabenteuern}}}\Cendnote{\textnormal{siehe Paul Goldmann an Arthur Schnitzler, 4. 10. [1900]}}}\label{K_L02936-9h}. Dieſer Tage kommt auch ſeine Tochter\pwindex{Philipp, Edith 17.01.1879 – 1968-02-16@\textsc{Philipp, Edith} (17.01.1879 – 1968-02-16)|pwv}.\pend
           \pstart
           Nach Breslau\oindex{Breslau@\textbf{Breslau}|pw} zur \label{K_L02936-11v}\edtext{Aufführung der »\textsc{Beatrice\pwindex{Schnitzler, Arthur 15.05.1862 – 21.10.1931@\textsc{Schnitzler, Arthur} (15.05.1862 – 21.10.1931), \emph{Schriftsteller, Mediziner}!Schleier der Beatrice. Schauspiel in fuenf Akten1900-12-01@\strich\emph{Der Schleier der Beatrice. Schauspiel in fünf Akten} {[}1900-12-01{]}|pw}}«}{\lemma{\textnormal{\emph{Aufführung der »Beatrice«}}}\Cendnote{\textnormal{\emph{Der Schleier der Beatrice}\pwindex{Schnitzler, Arthur 15.05.1862 – 21.10.1931@\textsc{Schnitzler, Arthur} (15.05.1862 – 21.10.1931), \emph{Schriftsteller, Mediziner}!Schleier der Beatrice. Schauspiel in fuenf Akten1900-12-01@\strich\emph{Der Schleier der Beatrice. Schauspiel in fünf Akten} {[}1900-12-01{]}|pwk} wurde am 1. 12. 1900 am Lobe-Theater\oindex{Lobe-Theater@\textbf{Lobe-Theater}|pwk} in Breslau\oindex{Breslau@\textbf{Breslau}|pwk} uraufgeführt.
                  Ursprünglich war der 17. 11. 1900 als
                  Premierentermin geplant.}}}\label{K_L02936-11h} möchte ich unendlich gern fahren. Ich habe das
               hier mit meinem Collegen \textsc{Fuchs\pwindex{Fuchs, Isidor 25.09.1849 – um den 20.8.1920@\textsc{Fuchs, Isidor} (25.09.1849 – um den 20.8.1920), \emph{Schriftsteller, Journalist}|pwuv}} beſprochen, und {\pb}er ſagte mir: »Ja, fahren Sie
               nur! Aber den Direktor \textsc{Löwe\pwindex{Loewe, Theodor 1855-01-01 – 1935@\textsc{Loewe, Theodor} (1855-01-01 – 1935), \emph{Theaterleiter}|pw}} dürfen Sie nicht tadeln; er iſt bei uns\orgindex{Neue Freie Presse@Neue Freie Presse|pwv}{ }\label{K_L02936-12v}\edtext{\textsc{persona gratissima}}{\lemma{\textnormal{\emph{persona gratissima}}}\Cendnote{\textnormal{lateinisch: willkommene Person, hier im
                  Sinne von ›immun‹}}}\label{K_L02936-12h}.« Alſo, ich ſetze den Fall, die Aufführung könnte den
               Aufgaben des Stück\pwindex{Schnitzler, Arthur 15.05.1862 – 21.10.1931@\textsc{Schnitzler, Arthur} (15.05.1862 – 21.10.1931), \emph{Schriftsteller, Mediziner}!Schleier der Beatrice. Schauspiel in fuenf Akten1900-12-01@\strich\emph{Der Schleier der Beatrice. Schauspiel in fünf Akten} {[}1900-12-01{]}|pwv}es nicht
               gerecht werden (was ich befürchte), ſo werde ich das nicht ſagen dürfen, oder man
               wird es mir ſtreichen. Unter dieſen Umſtänden iſt es wirklich beſſer, nicht
               hinzugehen und die Berichterſtattung dem Direktor \textsc{Löwe\pwindex{Loewe, Theodor 1855-01-01 – 1935@\textsc{Loewe, Theodor} (1855-01-01 – 1935), \emph{Theaterleiter}|pw}} zu überlaſſen, der ſelbſt an die N. Fr. Pr.\orgindex{Neue Freie Presse@Neue Freie Presse|pw}{ }{\pb}zu telegraphiren pflegt und unter allen Umſtänden
                  \label{K_L02936-13v}\edtext{das Beſte ſagen wird}{\lemma{\textnormal{\emph{das Beſte ſagen wird}}}\Cendnote{\textnormal{Siehe zur Berichterstattung zur
                  Uraufführung von \emph{Der Schleier der Beatrice}\pwindex{Schnitzler, Arthur 15.05.1862 – 21.10.1931@\textsc{Schnitzler, Arthur} (15.05.1862 – 21.10.1931), \emph{Schriftsteller, Mediziner}!Schleier der Beatrice. Schauspiel in fuenf Akten1900-12-01@\strich\emph{Der Schleier der Beatrice. Schauspiel in fünf Akten} {[}1900-12-01{]}|pwk} in
                  der \emph{Neuen Freien Presse}\pwindex{Neue Freie Presse1864 – 1939@\emph{Neue Freie Presse} {[}1864 – 1939{]}|pwk} auch 28. 2. [1898] und 3. 12. [1900].}}}\label{K_L02936-13h}.\pend
           \pstart
           Grüße mir die ſtrebſamen \label{K_L02936-17v}\edtext{Fräulein\pwindex{Schnitzler, Olga 17.01.1882 – 13.01.1970@\textsc{Schnitzler, Olga} (17.01.1882 – 13.01.1970), \emph{Schauspielerin, Sängerin}|pwv}\pwindex{Steinrueck, Elisabeth 19.11.1885 – 07.04.1920@\textsc{Steinrück, Elisabeth} (19.11.1885 – 07.04.1920)|pwv} aus der Rothen-Stern-Gaſſe\oindex{Rotensterngasse@\textbf{Rotensterngasse}|pw}}{\lemma{\textnormal{\emph{Fräulein … Rothen-Stern-Gaſſe}}}\Cendnote{\textnormal{siehe Paul Goldmann an Arthur Schnitzler, 19. 9. [1900]}}}\label{K_L02936-17h} und theile mir deren genaue Adreſſe mit (Name und Hausnummer), damit ich
               ihnen mein \label{K_L02936-14v}\edtext{Buch\pwindex{Goldmann, Paul 31.01.1865 – 25.09.1935@\textsc{Goldmann, Paul} (31.01.1865 – 25.09.1935), \emph{Schriftsteller, Journalist}!Sommer in China. Reisebilder. Zweite, durchgesehene und vermehrte Auflage1900-10-27@\strich\emph{Ein Sommer in China. Reisebilder. Zweite, durchgesehene und vermehrte Auflage} {[}1900-10-27{]}|pwv}}{\lemma{\textnormal{\emph{Buch}}}\Cendnote{\textnormal{die zweite Auflage\pwindex{Goldmann, Paul 31.01.1865 – 25.09.1935@\textsc{Goldmann, Paul} (31.01.1865 – 25.09.1935), \emph{Schriftsteller, Journalist}!Sommer in China. Reisebilder. Zweite, durchgesehene und vermehrte Auflage1900-10-27@\strich\emph{Ein Sommer in China. Reisebilder. Zweite, durchgesehene und vermehrte Auflage} {[}1900-10-27{]}|pwkv} von \emph{Ein
                     Sommer in China}\pwindex{Goldmann, Paul 31.01.1865 – 25.09.1935@\textsc{Goldmann, Paul} (31.01.1865 – 25.09.1935), \emph{Schriftsteller, Journalist}!Sommer in China. Reisebilder1899-05-02@\strich\emph{Ein Sommer in China. Reisebilder} {[}1899-05-02{]}|pwk}, siehe Paul Goldmann an Arthur Schnitzler, 4. 10. [1900]}}}\label{K_L02936-14h} ſchicken kann.\pend
           \pstart
           Die \textsc{Glümerinnen\pwindex{Gluemer, Marie 03.07.1867 – 16.11.1925@\textsc{Glümer, Marie} (03.07.1867 – 16.11.1925), \emph{Schauspielerin}|pwv}\pwindex{Chlum, Auguste 16.03.1862 – 1956@\textsc{Chlum, Auguste} (16.03.1862 – 1956)|pwv}} ſind wieder beieinander, und Frl. \textsc{Mizzi\pwindex{Gluemer, Marie 03.07.1867 – 16.11.1925@\textsc{Glümer, Marie} (03.07.1867 – 16.11.1925), \emph{Schauspielerin}|pw}} hat neulich {\pb}einen ſehr \strikeout{ſchöne\textcolor{gray}{nn}} ſchönen und ſehr verdienten \label{K_L02936-20v}\edtext{Erfolg}{\lemma{\textnormal{\emph{Erfolg}}}\Cendnote{\textnormal{als weibliche Hauptrolle\pwindex{Gluemer, Marie 03.07.1867 – 16.11.1925@\textsc{Glümer, Marie} (03.07.1867 – 16.11.1925), \emph{Schauspielerin}|pwkv} der Berlin\oindex{Berlin@\textbf{Berlin}|pwk}er \emph{Secessionsbühne}\orgindex{Secessionsbuehne@Secessionsbühne|pwk} in \emph{Die Bildschnitzer}\pwindex{Schoenherr, Karl 24.02.1867 – 15.03.1943@\textsc{Schönherr, Karl} (24.02.1867 – 15.03.1943), \emph{Schriftsteller, Mediziner}!Bildschnitzer1900@\strich\emph{Die Bildschnitzer} {[}1900{]}|pwk}
                     (Karl Schönherr\pwindex{Schoenherr, Karl 24.02.1867 – 15.03.1943@\textsc{Schönherr, Karl} (24.02.1867 – 15.03.1943), \emph{Schriftsteller, Mediziner}|pwk}) und in \emph{Der Bär}\pwindex{Cechov, Anton Pavlovic 1860-01-17 – 1904-07-15@\textsc{Čechov, Anton Pavlovič} (1860-01-17 – 1904-07-15), \emph{Schriftsteller}!Baer1888@\strich\emph{Der Bär} {[}1888{]}|pwk} (Anton
                     Čechov\pwindex{Cechov, Anton Pavlovic 1860-01-17 – 1904-07-15@\textsc{Čechov, Anton Pavlovič} (1860-01-17 – 1904-07-15), \emph{Schriftsteller}|pwk})}}}\label{K_L02936-20h} gehabt bei Publikum und Kritik. Auch ſie ſehe ich ſelten,
               und ich lebe, eingeſponnen in Arbeit, ein ödes und nutzloſes Leben.\pend
           \pstart
           Was macht \textsc{Richard\pwindex{Beer-Hofmann, Richard 1866-07-11 – 1945-09-26@\textsc{Beer-Hofmann, Richard} (1866-07-11 – 1945-09-26), \emph{Schriftsteller}|pw}}? Keine Möglichkeit, von ihm eine Antwort zu bekommen.\pend
           \pstart
           Schreib’ mir bald und {\\[\baselineskip]}\strikeout{ſei} ſei von Herzen {\\[\baselineskip]}gegrüßt! Dein {\\[\baselineskip]}\spacefill\mbox{Paul Goldmnn}\pend
           \leftskip=0em{}
         
         \endnumbering\mylabel{h}\end{ledgroupsized}\begin{anhang}\end{anhang}\newcommand{\dateiname}{L02936}\newcommand{\titel}{Paul Goldmann an Arthur Schnitzler, 14. 10. [1900]}\newcommand{\editorInnen}{Martin Anton Müller und Laura Untner}%% latex-leseansicht-abspann.tex
%% Abspann für die Leseansicht.
%% Der Schalter \ifkorrekturansicht ist bereits durch den Vorspann gesetzt.

%% latex-abspann.tex
%% Gemeinsamer Abspann für Korrekturansicht und Leseansicht.
%% Setzt den Schalter \ifkorrekturansicht voraus (gesetzt in den
%% einbindenden Dateien latex-korrekturansicht-abspann.tex bzw.
%% latex-leseansicht-abspann.tex).
%% ---------------------------------------------------------------

\normalsize

% Das esempio-Environment wird nur in der Leseansicht benötigt
\ifkorrekturansicht\else
\newenvironment{esempio}[3]%
{
    \vspace{1.5ex}
    \rlap{\underline{#1}}
    \par
    \setlength{\parindent}{0cm}
    \nopagebreak
    \leftskip=#2cm
    \rightskip=#3cm
}
{
    \par
}
\fi

\doendnotes{C}
\bigskip
\vfill

\clearpage

\footnotesize

\ifkorrekturansicht
  \lohead{\textsc{register}}
\fi

% theindex-Environment neu definieren ohne reledmac
\makeatletter
\renewenvironment{theindex}{%
  \ifkorrekturansicht
    \section*{\indexname}%
  \else
    \subsubsection*{Index der erwähnten Entitäten}%
  \fi
  \setlength{\parindent}{0pt}%
  \setlength{\parskip}{0pt plus 0.3pt}%
  \let\item\@idxitem
}{%
  \ifkorrekturansicht\clearpage\fi
}
\makeatother

\IfFileExists{\jobname-pw.ind}{\input{\jobname-pw.ind}}{}

% Quellenangabe nur in der Leseansicht
\ifkorrekturansicht\else
% Fallback-Definitionen, falls die .tex-Datei \titel etc. nicht gesetzt hat
\providecommand{\titel}{}
\providecommand{\editorInnen}{}
\providecommand{\dateiname}{\jobname}

\vspace{3cm}

\vfill

\footnotesize
\textsc{Quelle}: \titel. Herausgegeben von {\editorInnen}. In: \emph{Arthur Schnitzler: Briefwechsel mit Autorinnen und Autoren}.
 Digitale Edition, https://schnitzler-briefe.acdh.oeaw.ac.at/{\dateiname}.html (Stand \today)
\fi

\end{document}


      