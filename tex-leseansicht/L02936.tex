%% latex-korrekturansicht-vorspann.tex
%% Vorspann für die Korrekturansicht.
%% Lädt die gemeinsame Datei latex-vorspann.tex mit gesetztem Schalter.

\newif\ifkorrekturansicht
\korrekturansichttrue

\input{../tex-inputs/latex-vorspann}


\section[ Paul Goldmann an Arthur Schnitzler, 14. 10. {[}1900{]}]{L02936 Paul Goldmann an Arthur Schnitzler, 14. 10. {[}1900{]}}
\nopagebreak\mylabel{L02936v}
\rehead{ }\normalsize\beginnumbering\briefempfaengerindex{Schnitzler, Arthur@\textsc{Schnitzler, Arthur}!zzzGoldmann, Paul@\emph{von Paul Goldmann}!1900-10-141@{14. 10. {[}1900{]}}|(be}
\toendnotes[C]{\smallbreak\pagebreak[2]}\Standort{DLA, A:Schnitzler, HS.NZ85.1.3170.}
\physDesc{Brief, 2 Blätter, 8 Seiten, 3316 Zeichen
\newline{}Handschrift: blaue Tinte, deutsche Kurrent
\newline{}Schnitzler: 1) mit Bleistift das Jahr »900« ergänzt und auf der ersten Seite des zweiten Blatts das
                                 vollständige Datum »14/10 900« vermerkt  2) mit rotem Buntstift acht Unterstreichungen und ein
                                    »X«}\toendnotes[C]{\smallbreak}
\pstart
           \raggedleft{}{\pb}Berlin\oindex{Berlin@\textbf{Berlin}, \emph{P.PPLC}|pw}, 14. Oktober.\pend
           
\pstart\center{}Mein lieber Freund,\pend\vspace{0.5em}
\pstart
           Heut am Sonntag habe
               ich endlich ein paar Minuten frei zu einem Briefe an Dich.\pend
           
\pstart
           Die \label{K_L02936-1v}\edtext{»Fackel\pwindex{Fackel@\emph{Die Fackel}|pwv}\pwindex{Affaire Schlenther-Schnitzler]@\emph{[Die Affaire Schlenther-Schnitzler]}|pwv}\pwindex{Antworten des Herausgebers. Habitue@\emph{Antworten des Herausgebers. Habitué}|pwv}\pwindex{Fackel@\emph{Die Fackel}|pw}«}{\lemma{\textnormal{\emph{»Fackel«}}}\Cendnote{\textnormal{Bezugnahme auf Karl Kraus\pwindex{Kraus, Karl 28.04.1874 – 12.06.1936@\textsc{Kraus, Karl} (28.04.1874 – 12.06.1936), \emph{Schriftsteller/Schriftstellerin, Publizist/Publizistin, Schriftsteller/Schriftstellerin}|pwk}: \emph{[Die Affaire Schlenther-Schnitzler]}\pwindex{Affaire Schlenther-Schnitzler]@\emph{[Die Affaire Schlenther-Schnitzler]}|pwk}. In: \emph{Die Fackel}\pwindex{Fackel@\emph{Die Fackel}|pwk}, Jg. 2, Nr. 53, Mitte September 1900, S. 1–6, und auf Karl Kraus\pwindex{Kraus, Karl 28.04.1874 – 12.06.1936@\textsc{Kraus, Karl} (28.04.1874 – 12.06.1936), \emph{Schriftsteller/Schriftstellerin, Publizist/Publizistin, Schriftsteller/Schriftstellerin}|pwk}: \emph{Antworten des Herausgebers. Habitué}\pwindex{Antworten des Herausgebers. Habitue@\emph{Antworten des Herausgebers. Habitué}|pwk}. In: \emph{Die Fackel}\pwindex{Fackel@\emph{Die Fackel}|pwk}, Jg. 2, Nr. 54, Ende September 1900, S. 25–26. Siehe zum
                  Konflikt zwischen Schnitzler und Paul Schlenther\pwindex{Schlenther, Paul 20.08.1854 – 30.04.1916@\textsc{Schlenther, Paul} (20.08.1854 – 30.04.1916), \emph{Schriftsteller/Schriftstellerin, Kritiker/Kritikerin, Theaterleiter/Theaterleiterin}|pwk} auch Paul Goldmann an Arthur Schnitzler, 12. 11. [1899].}}}\label{K_L02936-1}. Was willſt Du von dem
               Lausbuben? Offen geſtanden, ich hätte noch Schlimmeres erwartet. Im Übrigen hat
                  \label{K_L02936-2v}\edtext{\textsc{Burckhardt\pwindex{Burckhard, Max Eugen 14.07.1854 – 16.03.1912@\textsc{Burckhard, Max Eugen} (14.07.1854 – 16.03.1912), \emph{Schriftsteller/Schriftstellerin, Rechtswissenschaftler/Rechtswissenschaftlerin, Theaterleiter/Theaterleiterin}|pw}} in der »Zeit\pwindex{Zeit. Wiener Wochenschrift@\emph{Die Zeit. Wiener Wochenschrift}|pw}«}{\lemma{\textnormal{\emph{Burckhardt in der »Zeit«}}}\Cendnote{\textnormal{Max Burckhard\pwindex{Burckhard, Max Eugen 14.07.1854 – 16.03.1912@\textsc{Burckhard, Max Eugen} (14.07.1854 – 16.03.1912), \emph{Schriftsteller/Schriftstellerin, Rechtswissenschaftler/Rechtswissenschaftlerin, Theaterleiter/Theaterleiterin}|pwk}: \emph{Wienerinnen. Lustspiel in drei Aufzügen von Hermann Bahr.
                        Aufgeführt zum erstenmale im Deutschen Volkstheater am 3. October
                     1900}\pwindex{Wienerinnen. Lustspiel in drei Aufzuegen von Hermann Bahr. Aufgefuehrt zum erstenmale im Deutschen Volkstheater am 3. October 1900.@\emph{Wienerinnen. Lustspiel in drei Aufzügen von Hermann Bahr. Aufgeführt zum erstenmale im Deutschen Volkstheater am 3. October 1900.}|pwk}. In: \emph{Zeit}\pwindex{Zeit. Wiener Wochenschrift@\emph{Die Zeit. Wiener Wochenschrift}|pwk}, Bd. 25, Nr. 314,
                        6. 10. 1900, S. 10–11.}}}\label{K_L02936-2} das
               wahre Wort geſchrieben\pwindex{Wienerinnen. Lustspiel in drei Aufzuegen von Hermann Bahr. Aufgefuehrt zum erstenmale im Deutschen Volkstheater am 3. October 1900.@\emph{Wienerinnen. Lustspiel in drei Aufzügen von Hermann Bahr. Aufgeführt zum erstenmale im Deutschen Volkstheater am 3. October 1900.}|pwv}: die
               Leute rächen ſich jetzt an Dir, weil ſie Dir haben applaudiren müſſen. Auf das
               Geſindel im Allgemeinen war niemals zu rechnen. Ob die \label{K_L02936-3v}\edtext{Aktion\pwindex{Erklaerung [Schleier der Beatrice]@\emph{Erklärung [Schleier der Beatrice]}|pwv}}{\lemma{\textnormal{\emph{Aktion}}}\Cendnote{\textnormal{Siehe Hermann Bahr, Arthur Schnitzler: \emph{Briefwechsel, Aufzeichnungen, Dokumente (1891–1931)}, Hermann Bahr, Julius Bauer, J. J. David, Robert Hirschfeld, Felix Salten, Ludwig Speidel: Erklärung, 14. 9. 1900.
               }}}\label{K_L02936-3} ſonſt wirkungslos geblieben, wird ſich zeigen. Welche Wirkung hätte {\pb}denn auch kommen ſollen? Die Hauptſache war, daß der
               Herr \textsc{Schlenther\pwindex{Schlenther, Paul 20.08.1854 – 30.04.1916@\textsc{Schlenther, Paul} (20.08.1854 – 30.04.1916), \emph{Schriftsteller/Schriftstellerin, Kritiker/Kritikerin, Theaterleiter/Theaterleiterin}|pw}} eine Antwort\pwindex{Erklaerung [Schleier der Beatrice]@\emph{Erklärung [Schleier der Beatrice]}|pwv} auf ſein
               unerhörtes Benehmen bekam. Und den ſchlechten Ruf, den er ohnedies hat, hat dieſe
               Affaire nur noch vergrößert. Er hat’s geſpürt und wird’s noch weiter ſpüren. Dieſe
               Affaire, mag man ſagen, was man will, iſt ein Grund mehr für ſeinen Weggang vom Burgtheater\orgindex{Burgtheater@Burgtheater|pw}. Selbſt hier\oindex{Berlin@\textbf{Berlin}, \emph{P.PPLC}|pwv}, wo man ihn für einen Gott hält, hat ſie
               ihm geſchadet{\dotsfive}\pend
           
\pstart
           Dein \label{K_L02936-4v}\edtext{»Ohrenleiden«}{\lemma{\textnormal{\emph{»Ohrenleiden«}}}\Cendnote{\textnormal{Schnitzler litt an Otosklerose
                  (Verknöcherung des Innenohrs mit zunehmender Schwerhörigkeit).}}}\label{K_L02936-4}. Darauf weiß
               ich nur \uline{eine} Antwort: Heirathen. Ich ſchwöre Dir:
               wenn Du Frau {\pb}und Kinder haben wirſt, wirſt Du Dich
               weniger mit Deinem Ohre beſchäftigen; und wenn Du Dich weniger damit beſchäftigen
               wirſt, \strikeout{\textcolor{gray}{wi}} wirſt Du weniger darunter leiden.\pend
           
\pstart
           Mit \textsc{Lindau\pwindex{Lindau, Paul 03.06.1839 – 31.01.1919@\textsc{Lindau, Paul} (03.06.1839 – 31.01.1919), \emph{Schriftsteller/Schriftstellerin, Kritiker/Kritikerin, Theaterleiter/Theaterleiterin}|pw}} werde ich bei nächſter Gelegenheit \label{K_L02936-5v}\edtext{wegen \textsc{Salten\pwindex{Salten, Felix 06.09.1869 – 08.10.1945@\textsc{Salten, Felix} (06.09.1869 – 08.10.1945), \emph{Schriftsteller/Schriftstellerin, Journalist/Journalistin, Chefredakteur/Chefredakteurin}|pw}}}{\lemma{\textnormal{\emph{wegen Salten}}}\Cendnote{\textnormal{Die Stelle ist unklar. Möglicherweise ging es um eine etwaige
                   Uraufführung von Saltens\pwindex{Salten, Felix 06.09.1869 – 08.10.1945@\textsc{Salten, Felix} (06.09.1869 – 08.10.1945), \emph{Schriftsteller/Schriftstellerin, Journalist/Journalistin, Chefredakteur/Chefredakteurin}|pwk} Dreiakter
                     \emph{Der Gemeine}\pwindex{Gemeine. Schauspiel in drei Aufzuegen@\emph{Der Gemeine. Schauspiel in drei Aufzügen}|pwk}.
               }}}\label{K_L02936-5} ſprechen.\pend
           
\pstart
           \textsc{Kerr\pwindex{Kerr, Alfred 25.12.1867 – 12.10.1948@\textsc{Kerr, Alfred} (25.12.1867 – 12.10.1948), \emph{Schriftsteller/Schriftstellerin, Kritiker/Kritikerin}|pw}} ſehe ich ſehr ſelten. Wenn wir uns ſehen, ſprechen wir ſehr freundſchaftlich
               miteinander. Er ſteckt tief in ſeinem \label{K_L02936-6v}\edtext{Liebeswonnen\pwindex{Wendt, Anna @\textsc{Wendt, Anna}|pwv}}{\lemma{\textnormal{\emph{Liebeswonnen}}}\Cendnote{\textnormal{Siehe Paul Goldmann an Arthur Schnitzler, 18. 4. [1900].
               }}}\label{K_L02936-6} und ſtrebt der Erfüllung ſeiner Wünſche zu, was mit großen {\pb}Kämpfen verbunden ſcheint. Aber er\pwindex{Kerr, Alfred 25.12.1867 – 12.10.1948@\textsc{Kerr, Alfred} (25.12.1867 – 12.10.1948), \emph{Schriftsteller/Schriftstellerin, Kritiker/Kritikerin}|pwv} wird es ſchon durchſetzen. Er und das
                  Mädel\pwindex{Wendt, Anna @\textsc{Wendt, Anna}|pwv} ſcheinen ſich ſehr
               zu lieben, und das iſt die Hauptſache.\pend
           
\pstart
           Ich bin mit dem Hauſe \textsc{M.-C.\pwindex{Meyer-Cohn, Helene 1859-12-30 – 1918-11-09@\textsc{Meyer-Cohn, Helene} (1859-12-30 – 1918-11-09), \emph{Übersetzer/Übersetzerin}|pwv}\pwindex{Meyer-Cohn, Alexander 01.05.1853 – 11.08.1904@\textsc{Meyer-Cohn, Alexander} (01.05.1853 – 11.08.1904), \emph{Bankier/Bankierin}|pwv}} vollkommen auseinander. Dieſe ganze Geſchichte hat für mich mit einem großen
               Ekel geendet, – einem Ekel namentlich vor der »Geſellſchaft«, vor dieſen Leuten, die
               Einen nicht verſtehen und die Einen zur Tafel ziehen als Hanswurſt. Aber wehe, wenn
               man verſuchen will, auch einmal ſein Leben zu leben! {\pb}Im Übrigen hat die Kleine\pwindex{Meyer-Cohn, Helene 1859-12-30 – 1918-11-09@\textsc{Meyer-Cohn, Helene} (1859-12-30 – 1918-11-09), \emph{Übersetzer/Übersetzerin}|pwv}
               ja ganz recht gehabt, und ich bin fett und grotesk und nicht fähig, Liebe \strikeout{zu \textcolor{gray}{e}i} einzuflößen. Ich habe mich in
               die Arbeit geſtürzt, um das Alles zu vergeſſen.\pend
           
\pstart
           \textsc{Brandes\pwindex{Brandes, Georg 04.02.1842 – 19.02.1927@\textsc{Brandes, Georg} (04.02.1842 – 19.02.1927)|pw}} iſt hier\oindex{Berlin@\textbf{Berlin}, \emph{P.PPLC}|pwv} und erzählt mir
               viel von ſeinen \label{K_L02936-7v}\edtext{Liebesabenteuern}{\lemma{\textnormal{\emph{Liebesabenteuern}}}\Cendnote{\textnormal{Vgl. Paul Goldmann an Arthur Schnitzler, 4. 10. [1900].
               }}}\label{K_L02936-7}. Dieſer Tage kommt auch ſeine Tochter\pwindex{Philipp, Edith 17.01.1879 – 1968-02-16@\textsc{Philipp, Edith} (17.01.1879 – 1968-02-16)|pwv}.\pend
           
\pstart
           Nach Breslau\oindex{Breslau@\textbf{Breslau}, \emph{P.PPLA}|pw} zur \label{K_L02936-8v}\edtext{Aufführung der »\textsc{Beatrice\pwindex{Schleier der Beatrice. Schauspiel in fuenf Akten@\emph{Der Schleier der Beatrice. Schauspiel in fünf Akten}|pw}}«}{\lemma{\textnormal{\emph{Aufführung der »Beatrice«}}}\Cendnote{\textnormal{\emph{Der Schleier der Beatrice}\pwindex{Schleier der Beatrice. Schauspiel in fuenf Akten@\emph{Der Schleier der Beatrice. Schauspiel in fünf Akten}|pwk} wurde am 1. 12. 1900 am Lobe-Theater\oindex{Lobe-Theater@\textbf{Lobe-Theater}, \emph{Theater (K.THE)}|pwk} in Breslau\oindex{Breslau@\textbf{Breslau}, \emph{P.PPLA}|pwk} uraufgeführt.
                  Zu einem früheren Zeitpunkt war der 17. 11. 1900
                  als Premierentermin geplant.}}}\label{K_L02936-8} möchte ich unendlich gern fahren. Ich habe das
               hier mit meinem Collegen \textsc{Fuchs\pwindex{Fuchs, Isidor 25.09.1849 – um den 20.8.1920@\textsc{Fuchs, Isidor} (25.09.1849 – um den 20.8.1920), \emph{Schriftsteller/Schriftstellerin, Journalist/Journalistin}|pwuv}} beſprochen, und {\pb}er ſagte mir: »Ja, fahren Sie
               nur! Aber den Direktor \textsc{Löwe\pwindex{Loewe, Theodor 1855-01-01 – 1935@\textsc{Loewe, Theodor} (1855-01-01 – 1935), \emph{Theaterleiter/Theaterleiterin}|pw}} dürfen Sie nicht tadeln; er iſt bei uns\orgindex{Neue Freie Presse@Neue Freie Presse|pwv}{ }\label{K_L02936-9v}\edtext{\textsc{persona gratissima}}{\lemma{\textnormal{\emph{persona gratissima}}}\Cendnote{\textnormal{lateinisch: willkommene Person, hier im
                  Sinne von ›immun‹}}}\label{K_L02936-9}.« Alſo, ich ſetze den Fall, die Aufführung könnte den
               Aufgaben des Stück\pwindex{Schleier der Beatrice. Schauspiel in fuenf Akten@\emph{Der Schleier der Beatrice. Schauspiel in fünf Akten}|pwv}es nicht
               gerecht werden (was ich befürchte), ſo werde ich das nicht ſagen dürfen, oder man
               wird es mir ſtreichen. Unter dieſen Umſtänden iſt es wirklich beſſer, nicht
               hinzugehen und die Berichterſtattung dem Direktor \textsc{Löwe\pwindex{Loewe, Theodor 1855-01-01 – 1935@\textsc{Loewe, Theodor} (1855-01-01 – 1935), \emph{Theaterleiter/Theaterleiterin}|pw}} zu überlaſſen, der ſelbſt an die N. Fr. Pr.\orgindex{Neue Freie Presse@Neue Freie Presse|pw}{ }{\pb}zu telegraphiren pflegt und unter allen Umſtänden
                  \label{K_L02936-10v}\edtext{das Beſte ſagen wird}{\lemma{\textnormal{\emph{das Beſte ſagen wird}}}\Cendnote{\textnormal{Siehe zur Berichterstattung über die
                  Uraufführung von \emph{Der Schleier der Beatrice}\pwindex{Schleier der Beatrice. Schauspiel in fuenf Akten@\emph{Der Schleier der Beatrice. Schauspiel in fünf Akten}|pwk} in
                  der \emph{Neuen Freien Presse}\pwindex{Neue Freie Presse@\emph{Neue Freie Presse}|pwk}{ }Goldmanns\pwindex{Goldmann, Paul 31.01.1865 – 25.09.1935@\textsc{Goldmann, Paul} (31.01.1865 – 25.09.1935), \emph{Schriftsteller/Schriftstellerin, Journalist/Journalistin}|pwk} Briefe vom 30. 10. [1900] und vom 3. 12. [1900].}}}\label{K_L02936-10}.\pend
           
\pstart
           Grüße mir die ſtrebſamen \label{K_L02936-11v}\edtext{Fräulein\pwindex{Schnitzler, Olga 17.01.1882 – 13.01.1970@\textsc{Schnitzler, Olga} (17.01.1882 – 13.01.1970), \emph{Schauspieler/Schauspielerin, Sänger/Sängerin}|pwv}\pwindex{Steinrueck, Elisabeth 19.11.1885 – 07.04.1920@\textsc{Steinrück, Elisabeth} (19.11.1885 – 07.04.1920)|pwv} aus der Rothen-Stern-Gaſſe\oindex{Rotensterngasse@\textbf{Rotensterngasse}, \emph{Straße (K.STR)}|pw}}{\lemma{\textnormal{\emph{Fräulein … Rothen-Stern-Gaſſe}}}\Cendnote{\textnormal{Siehe Paul Goldmann an Arthur Schnitzler, 19. 9. [1900].
               }}}\label{K_L02936-11} und theile mir deren genaue Adreſſe mit (Name und Hausnummer), damit ich
               ihnen mein \label{K_L02936-12v}\edtext{Buch\pwindex{Sommer in China. Reisebilder. Zweite, durchgesehene und vermehrte Auflage@\emph{Ein Sommer in China. Reisebilder. Zweite, durchgesehene und vermehrte Auflage}|pwv}}{\lemma{\textnormal{\emph{Buch}}}\Cendnote{\textnormal{die zweite Auflage\pwindex{Sommer in China. Reisebilder. Zweite, durchgesehene und vermehrte Auflage@\emph{Ein Sommer in China. Reisebilder. Zweite, durchgesehene und vermehrte Auflage}|pwkv} von \emph{Ein
                     Sommer in China}\pwindex{Sommer in China. Reisebilder@\emph{Ein Sommer in China. Reisebilder}|pwk}, vgl. Paul Goldmann an Arthur Schnitzler, 4. 10. [1900].}}}\label{K_L02936-12} ſchicken kann.\pend
           
\pstart
           Die \textsc{Glümerinnen\pwindex{Gluemer, Marie 03.07.1867 – 16.11.1925@\textsc{Glümer, Marie} (03.07.1867 – 16.11.1925), \emph{Schauspieler/Schauspielerin}|pwv}\pwindex{Gluemer, Auguste 1862-03-16 – 1956@\textsc{Glümer, Auguste} (1862-03-16 – 1956), \emph{Lehrer/Lehrerin}|pwv}} ſind wieder beieinander, und Frl. \textsc{Mizzi\pwindex{Gluemer, Marie 03.07.1867 – 16.11.1925@\textsc{Glümer, Marie} (03.07.1867 – 16.11.1925), \emph{Schauspieler/Schauspielerin}|pw}} hat neulich {\pb}einen ſehr \strikeout{ſchöne\textcolor{gray}{nn}} ſchönen und ſehr verdienten \label{K_L02936-13v}\edtext{Erfolg}{\lemma{\textnormal{\emph{Erfolg}}}\Cendnote{\textnormal{als weibliche Hauptrolle\pwindex{Gluemer, Marie 03.07.1867 – 16.11.1925@\textsc{Glümer, Marie} (03.07.1867 – 16.11.1925), \emph{Schauspieler/Schauspielerin}|pwkv} der Berlin\oindex{Berlin@\textbf{Berlin}, \emph{P.PPLC}|pwk}er \emph{Secessionsbühne}\orgindex{Secessionsbuehne@Secessionsbühne|pwk} in \emph{Die Bildschnitzer}\pwindex{Bildschnitzer@\emph{Die Bildschnitzer}|pwk}
                  von Karl Schönherr\pwindex{Schoenherr, Karl 24.02.1867 – 15.03.1943@\textsc{Schönherr, Karl} (24.02.1867 – 15.03.1943), \emph{Schriftsteller/Schriftstellerin, Mediziner/Medizinerin}|pwk} und in \emph{Der Bär}\pwindex{Baer@\emph{Der Bär}|pwk} von Anton
                     Čechov\pwindex{Cechov, Anton Pavlovic 1860-01-17 – 1904-07-15@\textsc{Čechov, Anton Pavlovič} (1860-01-17 – 1904-07-15), \emph{Schriftsteller/Schriftstellerin}|pwk}}}}\label{K_L02936-13} gehabt bei Publikum und Kritik. Auch ſie ſehe ich ſelten, und ich lebe,
               eingeſponnen in Arbeit, ein ödes und nutzloſes Leben.\pend
           
\pstart
           Was macht \textsc{Richard\pwindex{Beer-Hofmann, Richard 1866-07-11 – 1945-09-26@\textsc{Beer-Hofmann, Richard} (1866-07-11 – 1945-09-26), \emph{Schriftsteller/Schriftstellerin}|pw}}? Keine Möglichkeit, von ihm eine Antwort zu bekommen.\pend
           
\pstart
           Schreib’ mir bald und \strikeout{ſei} ſei von Herzen
               gegrüßt!{\\[\baselineskip]} Dein {\\[\baselineskip]}\spacefill\mbox{Paul Goldmnn}\pend
           \leftskip=0em{}\selectlanguage{ngerman}\endnumbering\briefempfaengerindex{Schnitzler, Arthur@\textsc{Schnitzler, Arthur}!zzzGoldmann, Paul@\emph{von Paul Goldmann}!1900-10-141@{14. 10. {[}1900{]}}|)be}\mylabel{L02936h}  \normalsize

\doendnotes{C}
\bigskip
\vfill

\clearpage

\footnotesize

\lohead{\textsc{register}}

% Definiere theindex-Environment komplett neu ohne reledmac
\makeatletter
\renewenvironment{theindex}{%
  \section*{\indexname}%
  \setlength{\parindent}{0pt}%
  \setlength{\parskip}{0pt plus 0.3pt}%
  \let\item\@idxitem
}{%
  \clearpage
}
\makeatother

\IfFileExists{\jobname-pw.ind}{\input{\jobname-pw.ind}}{}

\end{document}

      