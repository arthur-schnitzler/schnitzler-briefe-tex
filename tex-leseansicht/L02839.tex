%% latex-leseansicht-vorspann.tex
%% Vorspann für die Leseansicht.
%% Lädt die gemeinsame Datei latex-vorspann.tex mit nicht gesetztem Schalter.

\newif\ifkorrekturansicht
\korrekturansichtfalse

\input{../tex-inputs/latex-vorspann}


         
         \renewcommand{\erwaehntePersonen}{Personen: Richard Beer-Hofmann, Theodor Herzl, Marie Reinhard, Leopold Sonnemann, Émile Zola}
         \renewcommand{\erwaehnteInstitutionen}{Institutionen: Frankfurter Zeitung}
         \renewcommand{\erwaehnteOrte}{Orte: Griechenland, Paris, Salzburg, Wien, rue de la Bourse}
         \renewcommand{\erwaehnteWerke}{Werke: Feuilleton. Carl-Theater. (»Freiwild«, Schauspiel von Arthur Schnitzler.), Neue Freie Presse}
               \section[ Paul Goldmann an Arthur Schnitzler, 28. 2. {[}1898{]}]{ Paul Goldmann an Arthur Schnitzler, 28. 2. {[}1898{]}}\nopagebreak\mylabel{v}\rehead{ }\begin{ledgroupsized}[t]{13cm}\normalsize\beginnumbering \toendnotes[C]{\smallbreak\pagebreak[2]} \Standort{DLA, A:Schnitzler, HS.NZ85.1.3168.}
\physDesc{Brief, 1 Blatt, 4 Seiten, 2406 Zeichen
\newline{}Handschrift: blaue Tinte, deutsche Kurrent
\newline{}Schnitzler: 1) mit Bleistift das Jahr »98« vermerkt  2) mit rotem Buntstift eine Unterstreichung}\toendnotes[C]{\smallbreak}\pstart
           \noindent{}{\pb}\textcolor{gray}{\textbf{\textbf{Frankfurter Zeitung\orgindex{Frankfurter Zeitung@Frankfurter Zeitung|pw}}}}\pend
           \pstart
           \textcolor{gray}{\textbf{(\begin{otherlanguage}{french}Gazette de Francfort\end{otherlanguage}\orgindex{Frankfurter Zeitung@Frankfurter Zeitung|pw}).}}\pend
           \pstart
           \textcolor{gray}{\textbf{\textbf{\begin{otherlanguage}{french}Fondateur M.\end{otherlanguage}{ }L. Sonnemann\pwindex{Sonnemann, Leopold 1831-10-29 – 1909-10-30@\textsc{Sonnemann, Leopold} (1831-10-29 – 1909-10-30), \emph{Journalist, Herausgeber}|pw}.}}}\pend
           \pstart
           \begin{otherlanguage}{french}\textcolor{gray}{\textbf{Journal politique, financier,}}\end{otherlanguage}\hfill \textsc{Paris\oindex{Paris@\textbf{Paris}|pw}}, 28. Februar. \pend
           \pstart
           \begin{otherlanguage}{french}\textcolor{gray}{\textbf{commercial et littéraire.}}\end{otherlanguage}\pend
           \pstart
           \begin{otherlanguage}{french}\textcolor{gray}{\textbf{\textbf{Paraissant trois fois par jour.}}}\end{otherlanguage}\pend
           \pstart
           \begin{otherlanguage}{french}\textcolor{gray}{\textbf{\textbf{Bureau à Paris\oindex{Paris@\textbf{Paris}|pw}}}}\end{otherlanguage}\pend
           \pstart
           \begin{otherlanguage}{french}\textcolor{gray}{\textbf{\textbf{10 \so{Rue de la Bourse}\oindex{rue de la Bourse@\textbf{rue de la Bourse}|pw}.}}}\end{otherlanguage}\pend
           \pstart\center{}Mein lieber Freund,\pend\pstart
           Dieſe fürchterlichen drei Wochen \label{K_L02839-1v}\edtext{\textsc{Zola\pwindex{Zola, Emile 02.04.1840 – 29.09.1902@\textsc{Zola, Émile} (02.04.1840 – 29.09.1902), \emph{Schriftsteller, Journalist}|pw}}-Prozeß}{\lemma{\textnormal{\emph{Zola-Prozeß}}}\Cendnote{\textnormal{vgl. Paul Goldmann an Arthur Schnitzler, 6. 2. [1898]}}}\label{K_L02839-1h} ſind vorüber. Ich komme endlich wieder einmal zu mir und – zu Dir.\pend
           \pstart
           Sehr gefreut hat es mich, daß \label{K_L02839-2v}\edtext{Du und
                  \textsc{Richard\pwindex{Beer-Hofmann, Richard 1866-07-11 – 1945-09-26@\textsc{Beer-Hofmann, Richard} (1866-07-11 – 1945-09-26), \emph{Schriftsteller}|pw}} in Salzburg\oindex{Salzburg@\textbf{Salzburg}|pw}}{\lemma{\textnormal{\emph{Du … Salzburg}}}\Cendnote{\textnormal{Richard Beer-Hofmann\pwindex{Beer-Hofmann, Richard 1866-07-11 – 1945-09-26@\textsc{Beer-Hofmann, Richard} (1866-07-11 – 1945-09-26), \emph{Schriftsteller}|pwk} und Schnitzler\pwindex{Schnitzler, Arthur 15.05.1862 – 21.10.1931@\textsc{Schnitzler, Arthur} (15.05.1862 – 21.10.1931), \emph{Schriftsteller, Mediziner}|pwk} waren von 7. 2. 1898 bis 13. 2. 1898 gemeinsam in Salzburg\oindex{Salzburg@\textbf{Salzburg}|pwk}.}}}\label{K_L02839-2h} meiner gedacht habt. Ich danke Euch für Eure
               liebe Karte.\pend
           \pstart
           Dein lieber Brief war auch ſehr ſchön, aber er ſollte doch etwas heiterer ſein.
               Lieber Sohn, verbittere \strikeout{d\textcolor{gray}{oc}h} Dir
               doch nicht ſo Deines Lebens ſchönſte Zeit! Laß’ es in Deinem \label{K_L02839-3v}\edtext{Ohre klingen}{\lemma{\textnormal{\emph{Ohre klingen}}}\Cendnote{\textnormal{Bezug auf Schnitzler\pwindex{Schnitzler, Arthur 15.05.1862 – 21.10.1931@\textsc{Schnitzler, Arthur} (15.05.1862 – 21.10.1931), \emph{Schriftsteller, Mediziner}|pwk}s
                  Otosklerose – einer Verknöcherung des Innenohrs mit zunehmender Schwerhörigkeit –,
                  an der er seit Herbst 1896 litt.}}}\label{K_L02839-3h}, wenn es nun
               ſchon durchaus nicht anders will. Aber iſt denn das {\pb}etwas Ernſtes? \label{K_L02839-4v}\edtext{\begin{otherlanguage}{french}\textsc{C’est embêtant, voilà tout.}\end{otherlanguage}}{\lemma{\textnormal{\emph{C’est … tout.}}}\Cendnote{\textnormal{französisch: Es ist ärgerlich, das ist
                  alles.}}}\label{K_L02839-4h} Und Jeder hat ſein \label{K_L02839-8v}\edtext{\begin{otherlanguage}{french}\textsc{embêtement}\end{otherlanguage}}{\lemma{\textnormal{\emph{embêtement}}}\Cendnote{\textnormal{französisch: Unannehmlichkeit}}}\label{K_L02839-8h},
               und Du haſt abſolut kein Recht, ein Leben ohne \begin{otherlanguage}{french}\textsc{embêtement}\end{otherlanguage}{ }zu beanſpruchen. Sei froh, daß Du nichts
               Schlimmeres haſt. Hindert Dich das an irgend etwas Weſentlichem? Schaffen, Erleben,
                  \label{K_L02839-12v}\edtext{\begin{otherlanguage}{french}\textsc{faire l’amour}\end{otherlanguage}}{\lemma{\textnormal{\emph{faire l’amour}}}\Cendnote{\textnormal{französisch: Liebe machen}}}\label{K_L02839-12h}? Nein;
               alſo laß’ \strikeout{\textcolor{gray}{×}} es klingen! Und wenn Du meinſt, es mache Dir das Arbeiten unmöglich, ſo \strikeout{halte} halte ich das für einen Fehlſchluß, und ich
               glaube, Du ſchiebſt auf das Ohrenklingen nur \strikeout{e\textcolor{gray}{i}} den \strikeout{Mang} Mangel an Inſpiration, welcher daher
               kommt, daß Du zu feſt und zu warm ſitzeſt in Deinem \label{K_L02839-14v}\edtext{\textsc{Phaeaken}-Neſt}{\lemma{\textnormal{\emph{Phaeaken-Neſt}}}\Cendnote{\textnormal{Die Phaiaken sind ein Volk der griech\oindex{Griechenland@\textbf{Griechenland}|pwkv}ischen Mythologie. »Phaeaken-Neſt« meint im
                  übertragenen Sinne einen Ort, an dem Menschen faul im Luxus leben.}}}\label{K_L02839-14h}.\pend
           \pstart
           {\pb}Das \label{K_L02839-5v}\edtext{Feuilleton\pwindex{Feuilleton. Carl-Theater. (»Freiwild«, Schauspiel von Arthur
                  Schnitzler.)1898-02-13@\emph{Feuilleton. Carl-Theater. (»Freiwild«, Schauspiel von Arthur Schnitzler.)} {[}1898-02-13{]}|pwv} von \textsc{Herzl\pwindex{Herzl, Theodor 1860-05-02 – 1904-07-03@\textsc{Herzl, Theodor} (1860-05-02 – 1904-07-03), \emph{Schriftsteller, Journalist}|pw}}}{\lemma{\textnormal{\emph{Feuilleton von Herzl}}}\Cendnote{\textnormal{H.\pwindex{Herzl, Theodor 1860-05-02 – 1904-07-03@\textsc{Herzl, Theodor} (1860-05-02 – 1904-07-03), \emph{Schriftsteller, Journalist}|pwk} [ = Theodor Herzl\pwindex{Herzl, Theodor 1860-05-02 – 1904-07-03@\textsc{Herzl, Theodor} (1860-05-02 – 1904-07-03), \emph{Schriftsteller, Journalist}|pwk}]: \emph{Feuilleton.
                        Carl-Theater. (»Freiwild«, Schauspiel von Arthur Schnitzler.)}\pwindex{Feuilleton. Carl-Theater. (»Freiwild«, Schauspiel von Arthur
                  Schnitzler.)1898-02-13@\emph{Feuilleton. Carl-Theater. (»Freiwild«, Schauspiel von Arthur Schnitzler.)} {[}1898-02-13{]}|pwk}. In: \emph{Neue Freie Presse}\pwindex{Neue Freie Presse1864 – 1939@\emph{Neue Freie Presse} {[}1864 – 1939{]}|pwk}, Nr. 12024, 13. 2. 1898, S. 1–2. Vgl. A. S.: \emph{Tagebuch}, 13. 2. 1898.}}}\label{K_L02839-5h}, von
               welchem Du ſchreibſt, habe ich nicht geleſen. Könnteſt Du mir es nicht ſchicken?\pend
           \pstart
           Mach’ Dich mit der erſten warmen Frühlings-Sonne auf und fahre Deinen Hypochondrien
               davon, weit in die Welt hinaus. Wenn Du erſt einmal draußen biſt, wirſt Du ſelbſt
               erſtaunen, was für ein Kerl Du biſt!\pend
           \pstart
           Der \textsc{Zola\pwindex{Zola, Emile 02.04.1840 – 29.09.1902@\textsc{Zola, Émile} (02.04.1840 – 29.09.1902), \emph{Schriftsteller, Journalist}|pw}}-Prozeß hat Dir wohl auch bis zum Ende gut gefallen. Es iſt intereſſant, \strikeout{daß} wenn man plötzlich merkt, daß man wieder mitten im
               Mittelalter lebt. Aber es iſt auch gut ſo, daß \strikeout{\textcolor{gray}{w}} wir wieder die alten Feinde vor uns haben. \strikeout{W\textcolor{gray}{om}} Das gibt einen ſchönen Kampf, und {\pb}man weiß
               doch wenigſtens, \strikeout{\textcolor{gray}{e}} wozu man auf der Welt iſt und verliert ſich nicht mehr ins Bodenloſe, wie beim
                  \label{K_L02839-32v}\edtext{Aufſuchen der »neuen Künſte« und
               der »neuen Wahrheiten}{\lemma{\textnormal{\emph{Aufſuchen … Wahrheiten}}}\Cendnote{\textnormal{Anspielung auf
                  diverse Erneuerungsideen zur Zeit des \begin{otherlanguage}{french}Fin de
                     Siècle\end{otherlanguage}}}}\label{K_L02839-32h}«. Es gibt eben in Wirklichkeit nirgends \strikeout{d} und
               niemals etwas Neues, und das Einzige, wozu wir Menſchen fähig ſind, iſt, daß wir
               immer das Alte wiedererleben, als Individuen wie als Völker\textcolor{gray}{:} Wir
               leben ewig in der Vergangenheit, ein »Leben, wie es iſt«, und eine Sinnes-Täuſchung
               zeigt uns den Ausblick auf das »Leben, wie es ſein ſollte« (wie es aber niemals ſein
               wird), \strikeout{d\textcolor{gray}{a}} auf die Zukunft{\dotsfive}\pend
           \pstart
           Im Sommer? Wie gern möchte ich Dich wiederſehen! Aber ich weiß zur Stunde noch nicht,
               wie ſich gewiſſe Dinge geſtalten werden, welche meine Redaction\orgindex{Frankfurter Zeitung@Frankfurter Zeitung|pwv} projectirt. Sei von Herzen
               gegrüßt!\pend
           \pstart Dein treuer \spacefill\mbox{Paul Goldmn}\pend{}\pstart
           \noindent{}Viele Grüße an Deine Freundin\pwindex{Reinhard, Marie 1871-03-13 – 1899-03-18@\textsc{Reinhard, Marie} (1871-03-13 – 1899-03-18), \emph{Gesangspädagogin}|pwv}!\pend
           
         
         \endnumbering\mylabel{h}\end{ledgroupsized}  \newcommand{\dateiname}{L02839}\newcommand{\titel}{Paul Goldmann an Arthur Schnitzler, 28. 2. [1898]}\newcommand{\editorInnen}{Martin Anton Müller und Laura Untner}%% latex-leseansicht-abspann.tex
%% Abspann für die Leseansicht.
%% Der Schalter \ifkorrekturansicht ist bereits durch den Vorspann gesetzt.

%% latex-abspann.tex
%% Gemeinsamer Abspann für Korrekturansicht und Leseansicht.
%% Setzt den Schalter \ifkorrekturansicht voraus (gesetzt in den
%% einbindenden Dateien latex-korrekturansicht-abspann.tex bzw.
%% latex-leseansicht-abspann.tex).
%% ---------------------------------------------------------------

\normalsize

% Das esempio-Environment wird nur in der Leseansicht benötigt
\ifkorrekturansicht\else
\newenvironment{esempio}[3]%
{
    \vspace{1.5ex}
    \rlap{\underline{#1}}
    \par
    \setlength{\parindent}{0cm}
    \nopagebreak
    \leftskip=#2cm
    \rightskip=#3cm
}
{
    \par
}
\fi

\doendnotes{C}
\bigskip
\vfill

\clearpage

\footnotesize

\ifkorrekturansicht
  \lohead{\textsc{register}}
\fi

% theindex-Environment neu definieren ohne reledmac
\makeatletter
\renewenvironment{theindex}{%
  \ifkorrekturansicht
    \section*{\indexname}%
  \else
    \subsubsection*{Index der erwähnten Entitäten}%
  \fi
  \setlength{\parindent}{0pt}%
  \setlength{\parskip}{0pt plus 0.3pt}%
  \let\item\@idxitem
}{%
  \ifkorrekturansicht\clearpage\fi
}
\makeatother

\IfFileExists{\jobname-pw.ind}{\input{\jobname-pw.ind}}{}

% Quellenangabe nur in der Leseansicht
\ifkorrekturansicht\else
% Fallback-Definitionen, falls die .tex-Datei \titel etc. nicht gesetzt hat
\providecommand{\titel}{}
\providecommand{\editorInnen}{}
\providecommand{\dateiname}{\jobname}

\vspace{3cm}

\vfill

\footnotesize
\textsc{Quelle}: \titel. Herausgegeben von {\editorInnen}. In: \emph{Arthur Schnitzler: Briefwechsel mit Autorinnen und Autoren}.
 Digitale Edition, https://schnitzler-briefe.acdh.oeaw.ac.at/{\dateiname}.html (Stand \today)
\fi

\end{document}


      