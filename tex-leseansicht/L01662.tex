%% latex-leseansicht-vorspann.tex
%% Vorspann für die Leseansicht.
%% Lädt die gemeinsame Datei latex-vorspann.tex mit nicht gesetztem Schalter.

\newif\ifkorrekturansicht
\korrekturansichtfalse

\input{../tex-inputs/latex-vorspann}


\section[Arthur Schnitzler an Hermann Bahr, 11. 3. 1907]{L01662 Arthur Schnitzler an Hermann Bahr, 11. 3. 1907}
\nopagebreak\mylabel{L01662v}
\rehead{ }\normalsize\beginnumbering\briefempfaengerindex{Bahr, Hermann@\textsc{Bahr, Hermann}!zzzSchnitzler, Arthur@\emph{von Arthur Schnitzler}!1907-03-111@{11. 3. 1907}|(be}
\toendnotes[C]{\smallbreak\pagebreak[2]}
\correspDesc{Versand  durch Arthur Schnitzler am 11. 3. 1907 in Wien
\newline{}Erhalt  durch Hermann Bahr im Zeitraum [11. 3. 1907
                  – 15. 3. 1907?] \textbf{Ort fehlend} }\toendnotes[C]{\smallbreak}
\Standort{TMW, HS AM 23384 Ba.}
\physDesc{Brief, 1 Blatt, 1 Seite, 1230 Zeichen
\newline{}Schreibmaschine
\newline{}Handschrift: 1) blaue Tinte, deutsche Kurrent (\noindent{}Unterschrift und Nachschrift, Korrekturen)\hspace{1em}2) Bleistift, deutsche Kurrent (\noindent{}Unterschrift und Nachschrift, Korrekturen)\hspace{1em}}\Standort{DLA, A:Schnitzler, 85.1.294/1.}
\physDesc{Brief, Durchschlag, 1 Blatt, 1 Seite, 1230 Zeichen
\newline{}Schreibmaschine}
\buchAbdrucke{\weitereDrucke{1) \emph{11. 3. 1907.} In: Arthur Schnitzler: \emph{The Letters of Arthur Schnitzler to Hermann Bahr}. Edited, annotated, and with an introduction, by Donald G. Daviau. Chapel Hill: \emph{The University of North Carolina Press} 1978, S. 97 (University of North Carolina studies in the Germanic languages
                        and literatures, 89).} \weitereDrucke{2) Hermann Bahr, Arthur Schnitzler: \emph{Briefwechsel, Aufzeichnungen, Dokumente (1891–1931)}. Herausgegeben von Kurt Ifkovits und Martin Anton Müller. Göttingen: \emph{Wallstein} 2018, S. 390.} }\toendnotes[C]{\smallbreak}
\pstart
           \raggedleft{}{\pb}XVIII Spoettelgasse 7\oindex{Wien@\textbf{Wien}!XVIII., Währing@\textbf{XVIII., Währing}!Edmund-Weiß-Gasse 7@\textbf{Edmund-Weiß-Gasse 7}, \emph{Wohngebäude}|pw}\pend
           
\pstart
           \raggedleft{}Wien\oindex{Wien@\textbf{Wien}, \emph{Verwaltungsgebiet}|pw} am 11. März 07.\pend
           
\pstart{}Lieber Hermann,\pend\vspace{0.5em}
\pstart
           Da ich nichts weiter von Dir gehört habe scheint es, dass das Projekt der Kammer\oindex{Kammerspiele Berlin@\textbf{Kammerspiele Berlin}, \emph{Theater}|pw}liebelei\pwindex{Schnitzler, Arthur 15.\,5.\,1862 Wien – 21.\,10.\,1931 ebd.@\textsc{Schnitzler, Arthur} (15.\,5.\,1862 Wien – 21.\,10.\,1931 ebd.), \emph{Schriftsteller, Mediziner}!Liebelei. Schauspiel in drei Akten@\strich\emph{Liebelei. Schauspiel in drei Akten}|pw} vorläufig zurückgelegt worden ist. Nun
               fällt mir etwas ein, dass ich Dir zu gelegentlicher Ueberlegung mitteilen möchte.
                  \label{LL097-1v}Wie wärs, wenn die Kammerspiele\oindex{Kammerspiele Berlin@\textbf{Kammerspiele Berlin}, \emph{Theater}|pw} in der nächsten Saison einen Versuch mit dem
                     »Märchen\pwindex{Schnitzler, Arthur 15.\,5.\,1862 Wien – 21.\,10.\,1931 ebd.@\textsc{Schnitzler, Arthur} (15.\,5.\,1862 Wien – 21.\,10.\,1931 ebd.), \emph{Schriftsteller, Mediziner}!Märchen. Schauspiel in drei Aufzügen@\strich\emph{Das Märchen. Schauspiel in drei Aufzügen}|pw}« wagten. Du weisst, dass das
                  Stück über Wien\oindex{Wien@\textbf{Wien}, \emph{Verwaltungsgebiet}|pw} nie hinausgekommen ist, dass es
                  hingegen – in Russland\oindex{Russland@\textbf{Russland}|pw} – einen meiner
                  stärksten und dauerndsten Erfolge bedeutet hat.\label{LL097-1h} Es ist wirklich geradezu
               lächerlich, dass sich in Deutschland\oindex{Deutschland@\textbf{Deutschland}|pw} noch kein
               Theater an das Stück gewagt hat. Die Kammerspiele\oindex{Kammerspiele Berlin@\textbf{Kammerspiele Berlin}, \emph{Theater}|pw},
               die das \label{K_L01662-1v}\edtext{Friedensfest\pwindex{Hauptmann, Gerhart 15.\,11.\,1862 Szczawno-Zdrój – 6.\,6.\,1946 Jagniątków@\textsc{Hauptmann, Gerhart} (15.\,11.\,1862 Szczawno-Zdrój – 6.\,6.\,1946 Jagniątków), \emph{Schriftsteller}!Friedensfest. Eine Familienkatastrophe@\strich\emph{Das Friedensfest. Eine Familienkatastrophe}|pw} aufgeführt}{\lemma{\textnormal{\emph{Friedensfest aufgeführt}}}\Cendnote{\textnormal{Die Premiere
                  von
                  Hauptmanns\pwindex{Hauptmann, Gerhart 15.\,11.\,1862 Szczawno-Zdrój – 6.\,6.\,1946 Jagniątków@\textsc{Hauptmann, Gerhart} (15.\,11.\,1862 Szczawno-Zdrój – 6.\,6.\,1946 Jagniątków), \emph{Schriftsteller}|pwk}{ }\emph{Das Friedensfest}\pwindex{Hauptmann, Gerhart 15.\,11.\,1862 Szczawno-Zdrój – 6.\,6.\,1946 Jagniątków@\textsc{Hauptmann, Gerhart} (15.\,11.\,1862 Szczawno-Zdrój – 6.\,6.\,1946 Jagniątków), \emph{Schriftsteller}!Friedensfest. Eine Familienkatastrophe@\strich\emph{Das Friedensfest. Eine Familienkatastrophe}|pwk} fand am 7. 1. 1907 an den \emph{Kammerspielen}\orgindex{Kammerspiele Berlin@Kammerspiele Berlin|pwk} in 
                  Berlin\oindex{Berlin@\textbf{Berlin}, \emph{Hauptstadt}|pwk} statt.}}}\label{K_L01662-1} haben,
               wären vielleicht am ehesten dazu geeignet, eine Aufführung dieses Stücks | mit der
                  Höflich\pwindex{Höflich, Lucie 20.\,2.\,1883 Hannover – 8.\,10.\,1956 Schmargendorf@\textsc{Höflich, Lucie} (20.\,2.\,1883 Hannover – 8.\,10.\,1956 Schmargendorf), \emph{Schauspielerin}|pw} | zu versuchen, womit wenig
               riskiert und möglicherweise einiges zu gewinnen wäre. Dass der Schluss des dritten
               Aktes geändert ist dürfte Dir bekannt sein.\pend
           
\pstart
           Wenn Du glaubst, dass die Sache nicht ganz aus{[}s{]}ichts{\pb}los ist, so sprichst Du
               vielleicht bei irgend einer Gelegenheit in diesem Sinn mit Reinhart\pwindex{Reinhardt, Max 9.\,9.\,1873 Baden bei Wien – 30.\,10.\,1943 New York City@\textsc{Reinhardt, Max} (9.\,9.\,1873 Baden bei Wien – 30.\,10.\,1943 New York City), \emph{Theaterleiter, Regisseur, Schauspieler}|pw}.\pend
           
\pstart
           Sei herzlich gegrüsst und lass jedenfalls recht bald etwas von Dir hören. Wann kommst
               Du zurück? Du häl{[}t{]}st Dich doch vor \label{K_L01662-2v}\edtext{Ragusa\oindex{Dubrovnik@\textbf{Dubrovnik}|pw}}{\lemma{\textnormal{\emph{Ragusa}}}\Cendnote{\textnormal{\label{LKommKL097-1v}Vom 1. 5. 1907 bis zum
                        8. 5. 1907 urlaubte Bahr\pwindex{Bahr, Hermann 19.\,7.\,1863 Linz – 15.\,1.\,1934 München@\textsc{Bahr, Hermann} (19.\,7.\,1863 Linz – 15.\,1.\,1934 München), \emph{Schriftsteller, Kritiker}|pwk}
                     an der oberen Adria\oindex{Adriatisches Meer@\textbf{Adriatisches Meer}|pwk}, nach Dubrovnik\oindex{Dubrovnik@\textbf{Dubrovnik}|pwk} kam er nicht.\label{LKommKL097-1h}}}}\label{K_L01662-2} einige Zeit in Wien\oindex{Wien@\textbf{Wien}, \emph{Verwaltungsgebiet}|pw} auf?\pend
           
\pstart
           {[}hs.:{]} Dein{\\[\baselineskip]}\spacefill\mbox{Arthur}\pend
           \leftskip=0em{}
\pstart
           \noindent{}viele Grüße von meiner Frau\pwindex{Schnitzler, Olga 17.\,1.\,1882 Wien – 13.\,1.\,1970 Lugano@\textsc{Schnitzler, Olga} (17.\,1.\,1882 Wien – 13.\,1.\,1970 Lugano), \emph{Schauspielerin, Sängerin}|pwv}.\pend
           \selectlanguage{ngerman}\endnumbering\briefempfaengerindex{Bahr, Hermann@\textsc{Bahr, Hermann}!zzzSchnitzler, Arthur@\emph{von Arthur Schnitzler}!1907-03-111@{11. 3. 1907}|)be}\mylabel{L01662h}  \newcommand{\dateiname}{L01662}\newcommand{\titel}{Arthur Schnitzler an Hermann Bahr, 11. 3. 1907}\newcommand{\editorInnen}{Herausgegeben von Martin Anton Müller}%% latex-leseansicht-abspann.tex
%% Abspann für die Leseansicht.
%% Der Schalter \ifkorrekturansicht ist bereits durch den Vorspann gesetzt.

%% latex-abspann.tex
%% Gemeinsamer Abspann für Korrekturansicht und Leseansicht.
%% Setzt den Schalter \ifkorrekturansicht voraus (gesetzt in den
%% einbindenden Dateien latex-korrekturansicht-abspann.tex bzw.
%% latex-leseansicht-abspann.tex).
%% ---------------------------------------------------------------

\normalsize

% Das esempio-Environment wird nur in der Leseansicht benötigt
\ifkorrekturansicht\else
\newenvironment{esempio}[3]%
{
    \vspace{1.5ex}
    \rlap{\underline{#1}}
    \par
    \setlength{\parindent}{0cm}
    \nopagebreak
    \leftskip=#2cm
    \rightskip=#3cm
}
{
    \par
}
\fi

\doendnotes{C}
\bigskip
\vfill

\clearpage

\footnotesize

\ifkorrekturansicht
  \lohead{\textsc{register}}
\fi

% theindex-Environment neu definieren ohne reledmac
\makeatletter
\renewenvironment{theindex}{%
  \ifkorrekturansicht
    \section*{\indexname}%
  \else
    \subsubsection*{Index der erwähnten Entitäten}%
  \fi
  \setlength{\parindent}{0pt}%
  \setlength{\parskip}{0pt plus 0.3pt}%
  \let\item\@idxitem
}{%
  \ifkorrekturansicht\clearpage\fi
}
\makeatother

\IfFileExists{\jobname-pw.ind}{\input{\jobname-pw.ind}}{}

% Quellenangabe nur in der Leseansicht
\ifkorrekturansicht\else
% Fallback-Definitionen, falls die .tex-Datei \titel etc. nicht gesetzt hat
\providecommand{\titel}{}
\providecommand{\editorInnen}{}
\providecommand{\dateiname}{\jobname}

\vspace{3cm}

\vfill

\footnotesize
\textsc{Quelle}: \titel. Herausgegeben von {\editorInnen}. In: \emph{Arthur Schnitzler: Briefwechsel mit Autorinnen und Autoren}.
 Digitale Edition, https://schnitzler-briefe.acdh.oeaw.ac.at/{\dateiname}.html (Stand \today)
\fi

\end{document}


