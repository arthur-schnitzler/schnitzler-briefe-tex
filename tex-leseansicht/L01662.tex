%% latex-korrekturansicht-vorspann.tex
%% Vorspann für die Korrekturansicht.
%% Lädt die gemeinsame Datei latex-vorspann.tex mit gesetztem Schalter.

\newif\ifkorrekturansicht
\korrekturansichttrue

\input{../tex-inputs/latex-vorspann}


\section[Arthur Schnitzler an Hermann Bahr, 11. 3. 1907]{L01662 Arthur Schnitzler an Hermann Bahr, 11. 3. 1907}
\nopagebreak\mylabel{L01662v}
\rehead{ }\normalsize\beginnumbering\briefempfaengerindex{Bahr, Hermann@\textsc{Bahr, Hermann}!zzzSchnitzler, Arthur@\emph{von Arthur Schnitzler}!1907-03-111@{11. 3. 1907}|(be}
\toendnotes[C]{\smallbreak\pagebreak[2]}\Standort{TMW, HS AM 23384 Ba.}
\physDesc{Brief, 1 Blatt, 1 Seite, 1230 Zeichen
\newline{}Schreibmaschine
\newline{}Handschrift: 1) blaue Tinte, deutsche Kurrent (\noindent{}Unterschrift und Nachschrift, Korrekturen)\hspace{1em}2) Bleistift, deutsche Kurrent (\noindent{}Unterschrift und Nachschrift, Korrekturen)\hspace{1em}}\Standort{DLA, A:Schnitzler, 85.1.294/1.}
\physDesc{Brief, Durchschlag1 Blatt, 1 Seite, 1230 Zeichen
\newline{}Schreibmaschine}
\buchAbdrucke{\weitereDrucke{1) Arthur Schnitzler: \emph{The Letters of Arthur Schnitzler to Hermann Bahr}. Chapel Hill: \emph{The University of North Carolina Press} 1978, S. 97.} \weitereDrucke{2) Hermann Bahr, Arthur Schnitzler: \emph{Briefwechsel, Aufzeichnungen, Dokumente (1891–1931)}. Göttingen: \emph{Wallstein} 2018, S. 390.} }\toendnotes[C]{\smallbreak}
\pstart
           \raggedleft{}{\pb}XVIII Spoettelgasse 7\oindex{Edmund-Weiss-Gasse 7@\textbf{Edmund-Weiß-Gasse 7}, \emph{Wohngebäude (K.WHS)}|pw}\pend
           
\pstart
           \raggedleft{}Wien\oindex{Wien@\textbf{Wien}, \emph{A.ADM2}|pw} am 11. März 07.\pend
           
\pstart{}Lieber Hermann,\pend\vspace{0.5em}
\pstart
           Da ich nichts weiter von Dir gehört habe scheint es, dass das Projekt der Kammer\oindex{Kammerspiele Berlin@\textbf{Kammerspiele Berlin}, \emph{Theater (K.THE)}|pw}liebelei\pwindex{Liebelei. Schauspiel in drei Akten@\emph{Liebelei. Schauspiel in drei Akten}|pw} vorläufig zurückgelegt worden ist. Nun
               fällt mir etwas ein, dass ich Dir zu gelegentlicher Ueberlegung mitteilen möchte.
                  \label{LL097-1v}Wie wärs, wenn die Kammerspiele\oindex{Kammerspiele Berlin@\textbf{Kammerspiele Berlin}, \emph{Theater (K.THE)}|pw} in der nächsten Saison einen Versuch mit dem
                     »Märchen\pwindex{Maerchen. Schauspiel in drei Aufzuegen@\emph{Das Märchen. Schauspiel in drei Aufzügen}|pw}« wagten. Du weisst, dass das
                  Stück über Wien\oindex{Wien@\textbf{Wien}, \emph{A.ADM2}|pw} nie hinausgekommen ist, dass es
                  hingegen – in Russland\oindex{Russland@\textbf{Russland}, \emph{A.PCLI}|pw} – einen meiner
                  stärksten und dauerndsten Erfolge bedeutet hat.\label{LL097-1h} Es ist wirklich geradezu
               lächerlich, dass sich in Deutschland\oindex{Deutschland@\textbf{Deutschland}, \emph{A.PCLI}|pw} noch kein
               Theater an das Stück gewagt hat. Die Kammerspiele\oindex{Kammerspiele Berlin@\textbf{Kammerspiele Berlin}, \emph{Theater (K.THE)}|pw},
               die das \label{K_L01662-1v}\edtext{Friedensfest\pwindex{Friedensfest. Eine Familienkatastrophe@\emph{Das Friedensfest. Eine Familienkatastrophe}|pw} aufgeführt}{\lemma{\textnormal{\emph{Friedensfest aufgeführt}}}\Cendnote{\textnormal{Die Premiere
                  von
                  Hauptmanns\pwindex{Hauptmann, Gerhart 15.11.1862 – 06.06.1946@\textsc{Hauptmann, Gerhart} (15.11.1862 – 06.06.1946), \emph{Schriftsteller/Schriftstellerin}|pwk}{ }\emph{Das Friedensfest}\pwindex{Friedensfest. Eine Familienkatastrophe@\emph{Das Friedensfest. Eine Familienkatastrophe}|pwk} fand am 7. 1. 1907 an den \emph{Kammerspielen}\orgindex{Kammerspiele Berlin@Kammerspiele Berlin|pwk} in 
                  Berlin\oindex{Berlin@\textbf{Berlin}, \emph{P.PPLC}|pwk} statt.}}}\label{K_L01662-1} haben,
               wären vielleicht am ehesten dazu geeignet, eine Aufführung dieses Stücks | mit der
                  Höflich\pwindex{Hoeflich, Lucie 20.02.1883 – 08.10.1956@\textsc{Höflich, Lucie} (20.02.1883 – 08.10.1956), \emph{Schauspieler/Schauspielerin}|pw} | zu versuchen, womit wenig
               riskiert und möglicherweise einiges zu gewinnen wäre. Dass der Schluss des dritten
               Aktes geändert ist dürfte Dir bekannt sein.\pend
           
\pstart
           Wenn Du glaubst, dass die Sache nicht ganz aus{[}s{]}ichts{\pb}los ist, so sprichst Du
               vielleicht bei irgend einer Gelegenheit in diesem Sinn mit Reinhart\pwindex{Reinhardt, Max 09.09.1873 – 30.10.1943@\textsc{Reinhardt, Max} (09.09.1873 – 30.10.1943), \emph{Theaterleiter/Theaterleiterin, Regisseur/Regisseurin, Schauspieler/Schauspielerin}|pw}.\pend
           
\pstart
           Sei herzlich gegrüsst und lass jedenfalls recht bald etwas von Dir hören. Wann kommst
               Du zurück? Du häl{[}t{]}st Dich doch vor \label{K_L01662-2v}\edtext{Ragusa\oindex{Dubrovnik@\textbf{Dubrovnik}, \emph{P.PPLA}|pw}}{\lemma{\textnormal{\emph{Ragusa}}}\Cendnote{\textnormal{\label{LKommKL097-1v}Vom 1. 5. 1907 bis zum
                        8. 5. 1907 urlaubte Bahr\pwindex{Bahr, Hermann 19.07.1863 – 15.01.1934@\textsc{Bahr, Hermann} (19.07.1863 – 15.01.1934), \emph{Schriftsteller/Schriftstellerin, Kritiker/Kritikerin}|pwk}
                     an der oberen Adria\oindex{Adriatisches Meer@\textbf{Adriatisches Meer}, \emph{Meer (N.MER)}|pwk}, nach Dubrovnik\oindex{Dubrovnik@\textbf{Dubrovnik}, \emph{P.PPLA}|pwk} kam er nicht.\label{LKommKL097-1h}}}}\label{K_L01662-2} einige Zeit in Wien\oindex{Wien@\textbf{Wien}, \emph{A.ADM2}|pw} auf? \pend
           
\pstart
           {[}hs.:{]} Dein{\\[\baselineskip]}\spacefill\mbox{Arthur}\pend
           \leftskip=0em{}
\pstart
           \noindent{}viele Grüße von meiner Frau\pwindex{Schnitzler, Olga 17.01.1882 – 13.01.1970@\textsc{Schnitzler, Olga} (17.01.1882 – 13.01.1970), \emph{Schauspieler/Schauspielerin, Sänger/Sängerin}|pwv}.\pend
           \selectlanguage{ngerman}\endnumbering\briefempfaengerindex{Bahr, Hermann@\textsc{Bahr, Hermann}!zzzSchnitzler, Arthur@\emph{von Arthur Schnitzler}!1907-03-111@{11. 3. 1907}|)be}\mylabel{L01662h}  \normalsize

\doendnotes{C}
\bigskip
\vfill

\clearpage

\footnotesize

\lohead{\textsc{register}}

% Definiere theindex-Environment komplett neu ohne reledmac
\makeatletter
\renewenvironment{theindex}{%
  \section*{\indexname}%
  \setlength{\parindent}{0pt}%
  \setlength{\parskip}{0pt plus 0.3pt}%
  \let\item\@idxitem
}{%
  \clearpage
}
\makeatother

\IfFileExists{\jobname-pw.ind}{\input{\jobname-pw.ind}}{}

\end{document}

      