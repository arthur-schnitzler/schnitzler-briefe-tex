%% latex-leseansicht-vorspann.tex
%% Vorspann für die Leseansicht.
%% Lädt die gemeinsame Datei latex-vorspann.tex mit nicht gesetztem Schalter.

\newif\ifkorrekturansicht
\korrekturansichtfalse

\input{../tex-inputs/latex-vorspann}


         
         \renewcommand{\erwaehntePersonen}{Personen: Hermann Bahr, Gerhart Hauptmann, Lucie Höflich, Max Reinhardt, Olga Schnitzler}
         \renewcommand{\erwaehnteOrte}{Orte: Deutschland, Dubrovnik, Edmund-Weiß-Gasse 7, Kammerspiele Berlin, Russland, Wien}
         \renewcommand{\erwaehnteWerke}{Werke: Das Friedensfest, Das Märchen. Schauspiel in drei Aufzügen, Liebelei. Schauspiel in drei Akten}
               \section[Arthur Schnitzler an Hermann Bahr, 11. 3. 1907]{ Arthur Schnitzler an Hermann Bahr, 11. 3. 1907}\nopagebreak\mylabel{v}\rehead{ }\begin{ledgroupsized}[t]{13cm}\normalsize\beginnumbering \toendnotes[C]{\smallbreak\pagebreak[2]} \Standort{TMW, HS AM 23384 Ba.}
\physDesc{Brief, 1 Blatt, 1 Seite, 1230 Zeichen
\newline{}Schreibmaschine
\newline{}Handschrift: 1) blaue Tinte, deutsche Kurrent (\noindent{}Unterschrift und Nachschrift, Korrekturen)\hspace{1em}2) Bleistift, deutsche Kurrent (\noindent{}Unterschrift und Nachschrift, Korrekturen)\hspace{1em}}\Standort{DLA, A:Schnitzler, 85.1.294/1.}
\physDesc{Brief, Durchschlag, 1 Blatt, 1 Seite, 1230 Zeichen
\newline{}Schreibmaschine}\buchAbdrucke{\weitereDrucke{1) \emph{11. 3. 1907.} In: Arthur Schnitzler: \emph{The Letters of Arthur Schnitzler to Hermann Bahr}. Edited, annotated, and with an introduction, by Donald G.
                        Daviau. Chapel Hill: \emph{The University of North Carolina Press} 1978, S. 97 (University of North Carolina studies in the Germanic languages
                        and literatures, 89).} \weitereDrucke{2) Hermann Bahr, Arthur Schnitzler: \emph{Briefwechsel, Aufzeichnungen, Dokumente (1891–1931)}. Hg. Kurt Ifkovits und Martin Anton Müller. Göttingen: \emph{Wallstein} 2018, S. 390.} }\toendnotes[C]{\smallbreak}\pstart
           \noindent{}\raggedleft{}{\pb}XVIII Spoettelgasse 7\oindex{Edmund-Weiss-Gasse 7@\textbf{Edmund-Weiß-Gasse 7}|pw}\pend
           \pstart
           \raggedleft{}Wien\oindex{Wien@\textbf{Wien}|pw} am 11. März 07.\pend
           \pstart{}Lieber Hermann,\pend\pstart
           Da ich nichts weiter von Dir gehört habe scheint es, dass das Projekt der Kammer\oindex{Kammerspiele Berlin@\textbf{Kammerspiele Berlin}|pw}liebelei\pwindex{Schnitzler, Arthur 15.05.1862 – 21.10.1931@\textsc{Schnitzler, Arthur} (15.05.1862 – 21.10.1931), \emph{Schriftsteller, Mediziner}!Liebelei. Schauspiel in drei Akten1895-10-09@\strich\emph{Liebelei. Schauspiel in drei Akten} {[}1895-10-09{]}|pw} vorläufig zurückgelegt worden ist. Nun
               fällt mir etwas ein, dass ich Dir zu gelegentlicher Ueberlegung mitteilen möchte.
                  \label{LL097-1v}Wie wärs, wenn die Kammerspiele\oindex{Kammerspiele Berlin@\textbf{Kammerspiele Berlin}|pw} in der nächsten Saison einen Versuch mit dem
                     »Märchen\pwindex{Schnitzler, Arthur 15.05.1862 – 21.10.1931@\textsc{Schnitzler, Arthur} (15.05.1862 – 21.10.1931), \emph{Schriftsteller, Mediziner}!Maerchen. Schauspiel in drei Aufzuegen1893-12-01@\strich\emph{Das Märchen. Schauspiel in drei Aufzügen} {[}1893-12-01{]}|pw}« wagten. Du weisst, dass das
                  Stück über Wien\oindex{Wien@\textbf{Wien}|pw} nie hinausgekommen ist, dass es
                  hingegen – in Russland\oindex{Russland@\textbf{Russland}|pw} – einen meiner
                  stärksten und dauerndsten Erfolge bedeutet hat.\label{LL097-1h} Es ist wirklich geradezu
               lächerlich, dass sich in Deutschland\oindex{Deutschland@\textbf{Deutschland}|pw} noch kein
               Theater an das Stück gewagt hat. Die Kammerspiele\oindex{Kammerspiele Berlin@\textbf{Kammerspiele Berlin}|pw},
               die das \label{K_L01662-1v}\edtext{Friedensfest\pwindex{Hauptmann, Gerhart 15.11.1862 – 06.06.1946@\textsc{Hauptmann, Gerhart} (15.11.1862 – 06.06.1946), \emph{Schriftsteller}!Friedensfest1890@\strich\emph{Das Friedensfest} {[}1890{]}|pw} aufgeführt}{\lemma{\textnormal{\emph{Friedensfest aufgeführt}}}\Cendnote{\textnormal{Hauptmann\pwindex{Hauptmann, Gerhart 15.11.1862 – 06.06.1946@\textsc{Hauptmann, Gerhart} (15.11.1862 – 06.06.1946), \emph{Schriftsteller}|pwk}s \emph{Das Friedensfest}\pwindex{Hauptmann, Gerhart 15.11.1862 – 06.06.1946@\textsc{Hauptmann, Gerhart} (15.11.1862 – 06.06.1946), \emph{Schriftsteller}!Friedensfest1890@\strich\emph{Das Friedensfest} {[}1890{]}|pwk} hatte am 7. 1. 1907 Premiere.}}}\label{K_L01662-1h} haben,
               wären vielleicht am ehesten dazu geeignet, eine Aufführung dieses Stücks | mit der
                  Höflich\pwindex{Hoeflich, Lucie 20.02.1883 – 08.10.1956@\textsc{Höflich, Lucie} (20.02.1883 – 08.10.1956), \emph{Schauspielerin}|pw} | zu versuchen, womit wenig
               riskiert und möglicherweise einiges zu gewinnen wäre. Dass der Schluss des dritten
               Aktes geändert ist dürfte Dir bekannt sein.\pend
           \pstart
           Wenn Du glaubst, dass die Sache nicht ganz aus{[}s{]}ichts{\pb}los ist, so sprichst Du
               vielleicht bei irgend einer Gelegenheit in diesem Sinn mit Reinhart\pwindex{Reinhardt, Max 09.09.1873 – 30.10.1943@\textsc{Reinhardt, Max} (09.09.1873 – 30.10.1943), \emph{Theaterleiter, Regisseur, Schauspieler}|pw}.\pend
           \pstart
           Sei herzlich gegrüsst und lass jedenfalls recht bald etwas von Dir hören. Wann kommst
               Du zurück? Du häl{[}t{]}st Dich doch vor \label{K_L01662-2v}\edtext{Ragusa\oindex{Dubrovnik@\textbf{Dubrovnik}|pw}}{\lemma{\textnormal{\emph{Ragusa}}}\Cendnote{\textnormal{\label{LKommKL097-1v}Vom 1. bis zum
                        8. 5. 1907 urlaubt Bahr\pwindex{Bahr, Hermann 19.07.1863 – 15.01.1934@\textsc{Bahr, Hermann} (19.07.1863 – 15.01.1934), \emph{Schriftsteller, Kritiker}|pwk}
                     an der oberen Adria, nach Dubrovnik\oindex{Dubrovnik@\textbf{Dubrovnik}|pwk} kommt er nicht.\label{LKommKL097-1h}}}}\label{K_L01662-2h} einige Zeit in Wien\oindex{Wien@\textbf{Wien}|pw} auf? \pend
           \pstart
           {[}hs.:{]} Dein{\\[\baselineskip]}\spacefill\mbox{Arthur}\pend
           \leftskip=0em{}\pstart
           \noindent{}viele Grüße von meiner Frau\pwindex{Schnitzler, Olga 17.01.1882 – 13.01.1970@\textsc{Schnitzler, Olga} (17.01.1882 – 13.01.1970), \emph{Schauspielerin, Sängerin}|pwv}.\pend
           
         
         \endnumbering\mylabel{h}\end{ledgroupsized}  \newcommand{\dateiname}{L01662}\newcommand{\titel}{Arthur Schnitzler an Hermann Bahr, 11. 3. 1907}\newcommand{\editorInnen}{ Kurt Ifkovits,  Martin Anton Müller}%% latex-leseansicht-abspann.tex
%% Abspann für die Leseansicht.
%% Der Schalter \ifkorrekturansicht ist bereits durch den Vorspann gesetzt.

%% latex-abspann.tex
%% Gemeinsamer Abspann für Korrekturansicht und Leseansicht.
%% Setzt den Schalter \ifkorrekturansicht voraus (gesetzt in den
%% einbindenden Dateien latex-korrekturansicht-abspann.tex bzw.
%% latex-leseansicht-abspann.tex).
%% ---------------------------------------------------------------

\normalsize

% Das esempio-Environment wird nur in der Leseansicht benötigt
\ifkorrekturansicht\else
\newenvironment{esempio}[3]%
{
    \vspace{1.5ex}
    \rlap{\underline{#1}}
    \par
    \setlength{\parindent}{0cm}
    \nopagebreak
    \leftskip=#2cm
    \rightskip=#3cm
}
{
    \par
}
\fi

\doendnotes{C}
\bigskip
\vfill

\clearpage

\footnotesize

\ifkorrekturansicht
  \lohead{\textsc{register}}
\fi

% theindex-Environment neu definieren ohne reledmac
\makeatletter
\renewenvironment{theindex}{%
  \ifkorrekturansicht
    \section*{\indexname}%
  \else
    \subsubsection*{Index der erwähnten Entitäten}%
  \fi
  \setlength{\parindent}{0pt}%
  \setlength{\parskip}{0pt plus 0.3pt}%
  \let\item\@idxitem
}{%
  \ifkorrekturansicht\clearpage\fi
}
\makeatother

\IfFileExists{\jobname-pw.ind}{\input{\jobname-pw.ind}}{}

% Quellenangabe nur in der Leseansicht
\ifkorrekturansicht\else
% Fallback-Definitionen, falls die .tex-Datei \titel etc. nicht gesetzt hat
\providecommand{\titel}{}
\providecommand{\editorInnen}{}
\providecommand{\dateiname}{\jobname}

\vspace{3cm}

\vfill

\footnotesize
\textsc{Quelle}: \titel. Herausgegeben von {\editorInnen}. In: \emph{Arthur Schnitzler: Briefwechsel mit Autorinnen und Autoren}.
 Digitale Edition, https://schnitzler-briefe.acdh.oeaw.ac.at/{\dateiname}.html (Stand \today)
\fi

\end{document}


      