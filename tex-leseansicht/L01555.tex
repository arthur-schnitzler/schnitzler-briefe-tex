%% latex-korrekturansicht-vorspann.tex
%% Vorspann für die Korrekturansicht.
%% Lädt die gemeinsame Datei latex-vorspann.tex mit gesetztem Schalter.

\newif\ifkorrekturansicht
\korrekturansichttrue

\input{../tex-inputs/latex-vorspann}


\section[Arthur Schnitzler an Hugo von Hofmannsthal, 3. 10. 1905]{L01555 Arthur Schnitzler an Hugo von Hofmannsthal, 3. 10. 1905}
\nopagebreak\mylabel{L01555v}
\rehead{ }\normalsize\beginnumbering\briefempfaengerindex{Hofmannsthal, Hugo von@\textsc{Hofmannsthal, Hugo von}!zzzSchnitzler, Arthur@\emph{von Arthur Schnitzler}!1905-10-032@{3. 10. 1905}|(be}
\toendnotes[C]{\smallbreak\pagebreak[2]}\Standort{FDH, Hs-30885,123.}
\physDesc{Brief, 1 Blatt, 4 Seiten, 1071 Zeichen
\newline{}Handschrift: schwarze Tinte, deutsche Kurrent}
\buchAbdrucke{\weitereDrucke{Hugo von Hofmannsthal, Arthur Schnitzler: \emph{Briefwechsel}. Frankfurt am Main: \emph{S. Fischer} 1964, S. 216.} }\toendnotes[C]{\smallbreak}
\pstart
           \raggedleft{}{\pb}Wien\oindex{Wien@\textbf{Wien}, \emph{A.ADM2}|pw}{ }3/X 905\pend
           \vspace{0.5em}
\pstart
           lieber Hugo, den Ruf d. Lebens\pwindex{Ruf des Lebens. Schauspiel in drei Akten@\emph{Der Ruf des Lebens. Schauspiel in drei Akten}|pw}
               will ich jetzt gleich drucken laſſen und möchte Ihnen, zu erhöhter Bequemlichkeit der
               Lecture, die Correcturbogen zuſenden. Ich habe mich mit dem 3. Akt\pwindex{Ruf des Lebens. Schauspiel in drei Akten@\emph{Der Ruf des Lebens. Schauspiel in drei Akten}|pwv} nicht wenig geplagt, und bin eines
               Tags an den Punkt geko{\geminationm}en, wo ich nicht höher konnte.
               Mir iſt, als lägen gewiſſe Schwächen, die es wohl {\pb}auch
               jetzt noch darbietet, mehr im einakts-cycliſchen des Stoffs (worauf Sie ſelbſt ſchon
               hingewieſen haben) als in höchſt meiner Unfähigkeit begründet
               lägen. –\pend
           
\pstart
           Hätte ich bezüglich des Zwiſchenſpiels\pwindex{Zwischenspiel. Komoedie in drei Akten@\emph{Zwischenspiel. Komödie in drei Akten}|pw} auf
               andrer Beſetzung beſtanden, ſo wäre ein Aufſchub, wer weiſs auf wie lang,
               unvermeidlich geweſen. Freuen Sie ſich i{\geminationm}erhin auf Kainz\pwindex{Kainz, Josef 02.01.1858 – 20.09.1910@\textsc{Kainz, Josef} (02.01.1858 – 20.09.1910), \emph{Schauspieler/Schauspielerin}|pw}. Brahm\pwindex{Brahm, Otto 05.02.1856 – 28.11.1912@\textsc{Brahm, Otto} (05.02.1856 – 28.11.1912), \emph{Theaterleiter/Theaterleiterin, Regisseur/Regisseurin}|pw}{ }{\pb}ko{\geminationm}t wahrſcheinlich zur
                  \label{K_L01555-1v}\edtext{\textsc{Première}\pwindex{Zwischenspiel. Komoedie in drei Akten@\emph{Zwischenspiel. Komödie in drei Akten}|pwv}}{\lemma{\textnormal{\emph{Première}}}\Cendnote{\textnormal{am 12. 10. 1905}}}\label{K_L01555-1} her. –\pend
           
\pstart
           Ihre Karte deutet an, dſs man Sie vorläufig nicht ſehen ka{\geminationn}. Hoffentlich aber leſen Sie uns bälder vor. »Jederma{\geminationn}\pwindex{Jedermann. Das Spiel vom Sterben des reichen Mannes@\emph{Jedermann. Das Spiel vom Sterben des reichen Mannes}|pw}«?\strikeout{«}\pend
           
\pstart
           – Donnerſtag nächſter Woche iſt »Zwiſchenſpiel\pwindex{Zwischenspiel. Komoedie in drei Akten@\emph{Zwischenspiel. Komödie in drei Akten}|pw}«, Samſtag »Kakadu\pwindex{gruene Kakadu. Groteske in einem Akt@\emph{Der grüne Kakadu. Groteske in einem Akt}|pw}«. – \pend
           
\pstart
           Herzlichst Ihr{\\[\baselineskip]}\spacefill\mbox{A.}\pend
           \leftskip=0em{}
\pstart
           \noindent{}Grüßen Sie Gerty\pwindex{Hofmannsthal, Gertrude von 16.03.1880 – 09.11.1959@\textsc{Hofmannsthal, Gertrude von} (16.03.1880 – 09.11.1959)|pw}, und Richards\pwindex{Beer-Hofmann, Richard 1866-07-11 – 1945-09-26@\textsc{Beer-Hofmann, Richard} (1866-07-11 – 1945-09-26), \emph{Schriftsteller/Schriftstellerin}|pw}, die wohl ſchon daheim ſind. Schreiben Sie
                  gelegentlich ein {\pb}Wort, we{\geminationn} man ſchon nicht zuſa{\geminationm}enko{\geminationm}en kann. Ich hab natürlich jetzt täglich Proben.\pend
           \selectlanguage{ngerman}\endnumbering\briefempfaengerindex{Hofmannsthal, Hugo von@\textsc{Hofmannsthal, Hugo von}!zzzSchnitzler, Arthur@\emph{von Arthur Schnitzler}!1905-10-032@{3. 10. 1905}|)be}\mylabel{L01555h}  \normalsize

\doendnotes{C}
\bigskip
\vfill

\clearpage

\footnotesize

\lohead{\textsc{register}}

% Definiere theindex-Environment komplett neu ohne reledmac
\makeatletter
\renewenvironment{theindex}{%
  \section*{\indexname}%
  \setlength{\parindent}{0pt}%
  \setlength{\parskip}{0pt plus 0.3pt}%
  \let\item\@idxitem
}{%
  \clearpage
}
\makeatother

\IfFileExists{\jobname-pw.ind}{\input{\jobname-pw.ind}}{}

\end{document}

      