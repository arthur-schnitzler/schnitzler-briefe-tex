%% latex-leseansicht-vorspann.tex
%% Vorspann für die Leseansicht.
%% Lädt die gemeinsame Datei latex-vorspann.tex mit nicht gesetztem Schalter.

\newif\ifkorrekturansicht
\korrekturansichtfalse

\input{../tex-inputs/latex-vorspann}


\section[Hugo von Hofmannsthal an Arthur Schnitzler, 2. 7. {[}1898{]}]{L00810 Hugo von Hofmannsthal an Arthur Schnitzler, 2. 7. [1898]}
\nopagebreak\mylabel{L00810v}
\rehead{ }\normalsize\beginnumbering\briefempfaengerindex{Schnitzler, Arthur@\textsc{Schnitzler, Arthur}!zzzHofmannsthal, Hugo von@\emph{von Hugo von Hofmannsthal}!1898-07-021@{2. 7. [1898]}|(be}
\toendnotes[C]{\smallbreak\pagebreak[2]}
\correspDesc{Versand  durch Hugo von Hofmannsthal am 2. 7. [1898] in Tschortkiw
\newline{}Erhalt  durch Arthur Schnitzler im Zeitraum [3. 7. 1898
                  – 7. 7. 1898?] in Wien}\toendnotes[C]{\smallbreak}
\Standort{CUL, Schnitzler, B 43.}
\physDesc{Brief, 1 Blatt, 4 Seiten, 1164 Zeichen
\newline{}Handschrift: schwarze Tinte, deutsche Kurrent
\newline{}Schnitzler: mit Bleistift die Jahreszahl ergänzt: »98« 
\newline{}Ordnung: mit Bleistift von unbekannter Hand nummeriert:
                                    »116« }
\buchAbdrucke{\weitereDrucke{Hugo von Hofmannsthal, Arthur Schnitzler: \emph{Briefwechsel}. Herausgegeben von Therese Nickl und Heinrich Schnitzler. Frankfurt am Main: \emph{S. Fischer} 1964, S. 103–104.} }
\pstart
           
\pstart
           {\pb}\textsc{2\textsuperscript{ten} Juli}\pend
           
\pstart
           \raggedleft{}\textsc{Czortków\oindex{Tschortkiw@\textbf{Tschortkiw}, \emph{Hauptstadt}|pw}}\pend
           \pend
           
\pstart
           \raggedleft{}\textsc{Cavallerie Kaserne}{\\} (26 Stunden von Wien\oindex{Wien@\textbf{Wien}, \emph{Verwaltungsgebiet}|pw}!)\pend
           
\pstart{}mein lieber Arthur\pend\vspace{0.5em}
\pstart
           hier iſt mir{ }ſo zuwider zu Muth in dieſer troſtloſen niederſchlagenden Gegend, daſs
               ich mich immerfort mit dem Gedanken an das Spätere beſchäftige und damit es ja nicht
               an einer Art von Indolenz und Mangel an Verſtändigung{ }ſcheitert, will ich gleich
               etwas genaueres{ }ſagen. Es iſt {\pb}für
               mich aus Gründen die ich nicht alle aufzählen will, faſt nicht anders möglich als
               daſs wir unſere gemeinſame Fahrt zwiſchen dem 9\textsc{\textsuperscript{ten}} und \strikeout{A{ }}18\textsc{\textsuperscript{ten}} Auguſt machen. Ich weiß, daſs Sie ein paar Tage früher möchten, aber
               bitte geben Sie mir dieſmal {\pb}nach,{ }ſelbſt wenn Sie etwas anderes um 2–3 Tage hinausſchieben müſsten. Ich meine wir
               könnten uns etwa am 9\textsc{\textsuperscript{ten}} früh in Innsbruck\oindex{Innsbruck@\textbf{Innsbruck}, \emph{Verwaltungsgebiet}|pw} (?) treffen und dann den
               Weg fahren, den Sie wollen – Baſel\oindex{Basel@\textbf{Basel}|pw} etc. – und ich
               möchte{ }ſehr gern, daſs er in der Gegend von \textsc{Maloja}\oindex{Maloja@\textbf{Maloja}, \emph{Verwaltungsgebiet}|pw}{ }{\pb}oder{ }ſonſtwo in der{ }ſüdöſtlichen
                  Schweiz\oindex{Schweiz@\textbf{Schweiz}|pw} aufhörte. Bitte erkundigen Sie{ }ſich
                  we{\geminationn} es geht auch in Wien\oindex{Wien@\textbf{Wien}, \emph{Verwaltungsgebiet}|pw} – ich bin hier{ }ſo abgeſchnitten – wie es auf dieſer Strecke mit dem
               Gepäck geht – ich habe nicht{ }ſehr wenig mit. Bitte{ }ſchreiben Sie mir bald eine Zeile,
               und{ }ſagen mir, daſs es Ihnen recht iſt, ich freu mich{ }ſo darauf. Ihr
                  \spacefill\mbox{Hugo.}\pend
           \selectlanguage{ngerman}\endnumbering\briefempfaengerindex{Schnitzler, Arthur@\textsc{Schnitzler, Arthur}!zzzHofmannsthal, Hugo von@\emph{von Hugo von Hofmannsthal}!1898-07-021@{2. 7. [1898]}|)be}\mylabel{L00810h}  \newcommand{\dateiname}{L00810}\newcommand{\titel}{Hugo von Hofmannsthal an Arthur Schnitzler, 2. 7. [1898]}\newcommand{\editorInnen}{Martin Anton Müller und Gerd-Hermann Susen}%% latex-leseansicht-abspann.tex
%% Abspann für die Leseansicht.
%% Der Schalter \ifkorrekturansicht ist bereits durch den Vorspann gesetzt.

%% latex-abspann.tex
%% Gemeinsamer Abspann für Korrekturansicht und Leseansicht.
%% Setzt den Schalter \ifkorrekturansicht voraus (gesetzt in den
%% einbindenden Dateien latex-korrekturansicht-abspann.tex bzw.
%% latex-leseansicht-abspann.tex).
%% ---------------------------------------------------------------

\normalsize

% Das esempio-Environment wird nur in der Leseansicht benötigt
\ifkorrekturansicht\else
\newenvironment{esempio}[3]%
{
    \vspace{1.5ex}
    \rlap{\underline{#1}}
    \par
    \setlength{\parindent}{0cm}
    \nopagebreak
    \leftskip=#2cm
    \rightskip=#3cm
}
{
    \par
}
\fi

\doendnotes{C}
\bigskip
\vfill

\clearpage

\footnotesize

\ifkorrekturansicht
  \lohead{\textsc{register}}
\fi

% theindex-Environment neu definieren ohne reledmac
\makeatletter
\renewenvironment{theindex}{%
  \ifkorrekturansicht
    \section*{\indexname}%
  \else
    \subsubsection*{Index der erwähnten Entitäten}%
  \fi
  \setlength{\parindent}{0pt}%
  \setlength{\parskip}{0pt plus 0.3pt}%
  \let\item\@idxitem
}{%
  \ifkorrekturansicht\clearpage\fi
}
\makeatother

\IfFileExists{\jobname-pw.ind}{\input{\jobname-pw.ind}}{}

% Quellenangabe nur in der Leseansicht
\ifkorrekturansicht\else
% Fallback-Definitionen, falls die .tex-Datei \titel etc. nicht gesetzt hat
\providecommand{\titel}{}
\providecommand{\editorInnen}{}
\providecommand{\dateiname}{\jobname}

\vspace{3cm}

\vfill

\footnotesize
\textsc{Quelle}: \titel. Herausgegeben von {\editorInnen}. In: \emph{Arthur Schnitzler: Briefwechsel mit Autorinnen und Autoren}.
 Digitale Edition, https://schnitzler-briefe.acdh.oeaw.ac.at/{\dateiname}.html (Stand \today)
\fi

\end{document}


