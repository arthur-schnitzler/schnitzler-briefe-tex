%% latex-korrekturansicht-vorspann.tex
%% Vorspann für die Korrekturansicht.
%% Lädt die gemeinsame Datei latex-vorspann.tex mit gesetztem Schalter.

\newif\ifkorrekturansicht
\korrekturansichttrue

\input{../tex-inputs/latex-vorspann}


\section[Hugo von Hofmannsthal an Arthur Schnitzler, 2. 7. {[}1898{]}]{L00810 Hugo von Hofmannsthal an Arthur Schnitzler, 2. 7. {[}1898{]}}
\nopagebreak\mylabel{L00810v}
\rehead{ }\normalsize\beginnumbering\briefempfaengerindex{Schnitzler, Arthur@\textsc{Schnitzler, Arthur}!zzzHofmannsthal, Hugo von@\emph{von Hugo von Hofmannsthal}!1898-07-021@{2. 7. {[}1898{]}}|(be}
\toendnotes[C]{\smallbreak\pagebreak[2]}\Standort{CUL, Schnitzler, B 43.}
\physDesc{Brief, 1 Blatt, 4 Seiten, 1164 Zeichen
\newline{}Handschrift: schwarze Tinte, deutsche Kurrent
\newline{}Schnitzler: mit Bleistift die Jahreszahl ergänzt: »98« 
\newline{}Ordnung: mit Bleistift von unbekannter Hand nummeriert:
                                    »116« }
\buchAbdrucke{\weitereDrucke{Hugo von Hofmannsthal, Arthur Schnitzler: \emph{Briefwechsel}. Frankfurt am Main: \emph{S. Fischer} 1964, S. 103–104.} }
\pstart
           
\pstart
           {\pb}\textsc{2\textsuperscript{ten} Juli}\pend
           
\pstart
           \raggedleft{}\textsc{Czortków\oindex{Tschortkiw@\textbf{Tschortkiw}, \emph{P.PPLA2}|pw}}\pend
           \pend
           
\pstart
           \raggedleft{}\textsc{Cavallerie Kaserne}{\\} (26 Stunden von Wien\oindex{Wien@\textbf{Wien}, \emph{A.ADM2}|pw}!)\pend
           
\pstart{}mein lieber Arthur\pend\vspace{0.5em}
\pstart
           hier iſt mir ſo zuwider zu Muth in dieſer troſtloſen niederſchlagenden Gegend, daſs
               ich mich immerfort mit dem Gedanken an das Spätere beſchäftige und damit es ja nicht
               an einer Art von Indolenz und Mangel an Verſtändigung ſcheitert, will ich gleich
               etwas genaueres ſagen. Es iſt {\pb}für
               mich aus Gründen die ich nicht alle aufzählen will, faſt nicht anders möglich als
               daſs wir unſere gemeinſame Fahrt zwiſchen dem 9\textsc{\textsuperscript{ten}} und \strikeout{A}18\textsc{\textsuperscript{ten}} Auguſt machen. Ich weiß, daſs Sie ein paar Tage früher möchten, aber
               bitte geben Sie mir dieſmal {\pb}nach,
               ſelbſt wenn Sie etwas anderes um 2–3 Tage hinausſchieben müſsten. Ich meine wir
               könnten uns etwa am 9\textsc{\textsuperscript{ten}} früh in Innsbruck\oindex{Innsbruck@\textbf{Innsbruck}, \emph{A.ADM2}|pw} (?) treffen und dann den
               Weg fahren, den Sie wollen – Baſel\oindex{Basel@\textbf{Basel}, \emph{P.PPLA}|pw} etc. – und ich
               möchte ſehr gern, daſs er in der Gegend von \textsc{Maloja}\oindex{Maloja@\textbf{Maloja}, \emph{A.ADM2}|pw}{ }{\pb}oder ſonſtwo in der ſüdöſtlichen
                  Schweiz\oindex{Schweiz@\textbf{Schweiz}, \emph{A.PCLI}|pw} aufhörte. Bitte erkundigen Sie ſich
                  we{\geminationn} es geht auch in Wien\oindex{Wien@\textbf{Wien}, \emph{A.ADM2}|pw} – ich bin hier ſo abgeſchnitten – wie es auf dieſer Strecke mit dem
               Gepäck geht – ich habe nicht ſehr wenig mit. Bitte ſchreiben Sie mir bald eine Zeile,
               und ſagen mir, daſs es Ihnen recht iſt, ich freu mich ſo darauf. Ihr
                  \spacefill\mbox{Hugo.}\pend
           \selectlanguage{ngerman}\endnumbering\briefempfaengerindex{Schnitzler, Arthur@\textsc{Schnitzler, Arthur}!zzzHofmannsthal, Hugo von@\emph{von Hugo von Hofmannsthal}!1898-07-021@{2. 7. {[}1898{]}}|)be}\mylabel{L00810h}  \normalsize

\doendnotes{C}
\bigskip
\vfill

\clearpage

\footnotesize

\lohead{\textsc{register}}

% Definiere theindex-Environment komplett neu ohne reledmac
\makeatletter
\renewenvironment{theindex}{%
  \section*{\indexname}%
  \setlength{\parindent}{0pt}%
  \setlength{\parskip}{0pt plus 0.3pt}%
  \let\item\@idxitem
}{%
  \clearpage
}
\makeatother

\IfFileExists{\jobname-pw.ind}{\input{\jobname-pw.ind}}{}

\end{document}

      