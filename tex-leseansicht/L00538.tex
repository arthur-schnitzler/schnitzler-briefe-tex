%% latex-korrekturansicht-vorspann.tex
%% Vorspann für die Korrekturansicht.
%% Lädt die gemeinsame Datei latex-vorspann.tex mit gesetztem Schalter.

\newif\ifkorrekturansicht
\korrekturansichttrue

\input{../tex-inputs/latex-vorspann}


\section[Hugo von Hofmannsthal an Arthur Schnitzler, 13. 3. 1896]{L00538 Hugo von Hofmannsthal an Arthur Schnitzler, 13. 3. 1896}
\nopagebreak\mylabel{L00538v}
\rehead{ }\normalsize\beginnumbering\briefempfaengerindex{Schnitzler, Arthur@\textsc{Schnitzler, Arthur}!zzzHofmannsthal, Hugo von@\emph{von Hugo von Hofmannsthal}!1896-03-131@{13. 3. 1896}|(be}
\toendnotes[C]{\smallbreak\pagebreak[2]}\Standort{DLA, A:Schnitzler, 76.740.}
\physDesc{Brief, 1 Blatt, 1 Seite, 252 Zeichen
\newline{}Handschrift: schwarze Tinte, deutsche Kurrent
\newline{}Ordnung: Am 11. 5. 1937 schenkte Heinrich Schnitzler\pwindex{Schnitzler, Heinrich 09.08.1902 – 12.07.1982@\textsc{Schnitzler, Heinrich} (09.08.1902 – 12.07.1982), \emph{Regisseur/Regisseurin, Schauspieler/Schauspielerin}|pw} den Brief an
                                    »einen jungen Sammler«, vermutlich Hans Pietrkowski\pwindex{Pietrkowski, Hans 03.09.1906 – 14.6.1996@\textsc{Pietrkowski, Hans} (03.09.1906 – 14.6.1996)|pw} (Brief an
                                    Elise Pietrkowski\pwindex{Pietrkowski, Elise 28.06.1878 – 24.09.1967@\textsc{Pietrkowski, Elise} (28.06.1878 – 24.09.1967)|pw}, Theatermuseum\orgindex{Theatermuseum@Theatermuseum|pw},
                                    Schn 43/6). Am
                                    10. 5. 1976 wurde der Brief vom
                                 Antiquariat Hans Schneider in Tutzing an das Deutsche Literaturarchiv\orgindex{Deutsches Literaturarchiv@Deutsches Literaturarchiv|pw} verkauft. }
\buchAbdrucke{\weitereDrucke{1) Hugo von Hofmannsthal, Arthur Schnitzler: \emph{Briefwechsel}. Frankfurt am Main: \emph{S. Fischer} 1964, S. 65.} \weitereDrucke{2) Hans-Ulrich Lindken: \emph{Arthur Schnitzler. Aspekte und Akzente. Materialien zu Leben
                        und Werk}. Frankfurt am Main, Bern, Göttingen: \emph{Peter Lang} 1984, S. 173.} }\toendnotes[C]{\smallbreak}
\pstart
           {\pb}Freitag 13. 3. 96. \hfill \textcolor{gray}{\textbf{III Salesianergasse 12\oindex{Salesianergasse 12@\textbf{Salesianergasse 12}, \emph{Wohngebäude (K.WHS)}|pw}}}\pend
           
\pstart{}lieber Arthur\pend\vspace{0.5em}
\pstart
           Wir ſehen uns zu ſelten. Könnte ich nicht \label{K_L00538-1v}\edtext{Sonntag}{\lemma{\textnormal{\emph{Sonntag}}}\Cendnote{\textnormal{Vgl. A. S.: \emph{Tagebuch}, 15. 3. 1896. }}}\label{K_L00538-1} nach
                  7 zu Ihnen kommen. ich denke wir gehen dann zusammen um ½ 10
                  Uhr in dieſe Geſellſchaft. Oder ſtört es Sie beim Anziehen, wenn ich dabei
               bin?\pend
           
\pstart
           Bitte Antwort. Herzlich Ihr{\\[\baselineskip]}\spacefill\mbox{Hugo.}\pend
           \leftskip=0em{}\selectlanguage{ngerman}\endnumbering\briefempfaengerindex{Schnitzler, Arthur@\textsc{Schnitzler, Arthur}!zzzHofmannsthal, Hugo von@\emph{von Hugo von Hofmannsthal}!1896-03-131@{13. 3. 1896}|)be}\mylabel{L00538h}  \normalsize

\doendnotes{C}
\bigskip
\vfill

\clearpage

\footnotesize

\lohead{\textsc{register}}

% Definiere theindex-Environment komplett neu ohne reledmac
\makeatletter
\renewenvironment{theindex}{%
  \section*{\indexname}%
  \setlength{\parindent}{0pt}%
  \setlength{\parskip}{0pt plus 0.3pt}%
  \let\item\@idxitem
}{%
  \clearpage
}
\makeatother

\IfFileExists{\jobname-pw.ind}{\input{\jobname-pw.ind}}{}

\end{document}

      