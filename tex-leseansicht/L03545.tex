%% latex-korrekturansicht-vorspann.tex
%% Vorspann für die Korrekturansicht.
%% Lädt die gemeinsame Datei latex-vorspann.tex mit gesetztem Schalter.

\newif\ifkorrekturansicht
\korrekturansichttrue

\input{../tex-inputs/latex-vorspann}


\section[ Felix Salten an Arthur Schnitzler, 17. 1. 1910]{L03545 Felix Salten an Arthur Schnitzler, 17. 1. 1910}
\nopagebreak\mylabel{L03545v}
\rehead{ }\normalsize\beginnumbering\briefempfaengerindex{Schnitzler, Arthur@\textsc{Schnitzler, Arthur}!zzzSalten, Felix@\emph{von Felix Salten}!1910-01-171@{17. 1. 1910}|(be}
\toendnotes[C]{\smallbreak\pagebreak[2]}\Standort{CUL, Schnitzler, B 89, B 2.}
\physDesc{Bildpostkarte, 356 Zeichen
\newline{}Handschrift: schwarze Tinte, lateinische Kurrent
\newline{}Versand: Stempel: »\nobreak{}\oindex{Berlin@\textbf{Berlin}, \emph{P.PPLC}|pwk}Berlin W 9, 17. 1. 10, 8–9 N\nobreak{}«.  
\newline{}Ordnung: mit Bleistift von unbekannter Hand nummeriert: »260« }\toendnotes[C]{\smallbreak}\pstart{}{\pb}Herrn D\textsuperscript{r} Artur Schnitzler\pend{}\pstart{}Wien\oindex{Wien@\textbf{Wien}, \emph{A.ADM2}|pw}\pend{}\pstart{}XVIII. Spoettelgaße 7\oindex{Edmund-Weiss-Gasse 7@\textbf{Edmund-Weiß-Gasse 7}, \emph{Wohngebäude (K.WHS)}|pw}\pend{}{\bigskip}
\pstart
           {\pb}\textcolor{gray}{\textbf{\textbf{Berlin}.}}\oindex{Berlin@\textbf{Berlin}, \emph{P.PPLC}|pw}\hfill \textcolor{gray}{\textbf{Palais Kaiser Wilhelm des Grossen\oindex{Altes Palais@\textbf{Altes Palais}, \emph{Gebäude (K.GBD)}|pw} mit
                     dem historischen Eckfenster.}}\pend
           \vspace{1em}
\pstart
           \noindent{}{\pb}Lieber, wenn es etwas gibt, was noch unangenehmer ist, als Reinhardt\pwindex{Reinhardt, Max 09.09.1873 – 30.10.1943@\textsc{Reinhardt, Max} (09.09.1873 – 30.10.1943), \emph{Theaterleiter/Theaterleiterin, Regisseur/Regisseurin, Schauspieler/Schauspielerin}|pw} ein Stück einzureichen, dann ist es
               das: \label{K_L03545-1v}\edtext{bei Reinhardt\pwindex{Reinhardt, Max 09.09.1873 – 30.10.1943@\textsc{Reinhardt, Max} (09.09.1873 – 30.10.1943), \emph{Theaterleiter/Theaterleiterin, Regisseur/Regisseurin, Schauspieler/Schauspielerin}|pw} aufgeführt werden}{\lemma{\textnormal{\emph{bei … werden}}}\Cendnote{\textnormal{Zwei Tage später, am 19. 1. 1910, hatte das Lustspiel
                     \emph{Der gute König Dagobert}\pwindex{gute Koenig Dagobert. Lustspiel in vier Aufzuegen@\emph{Der gute König Dagobert. Lustspiel in vier Aufzügen}|pwk} von André Rivoire\pwindex{Rivoire, Andre 05.05.1872 – 19.08.1930@\textsc{Rivoire, André} (05.05.1872 – 19.08.1930), \emph{Schriftsteller/Schriftstellerin}|pwk}
                     am \emph{Deutschen Theater}\orgindex{Deutsches Theater Berlin@Deutsches Theater Berlin|pwk} in Berlin\oindex{Berlin@\textbf{Berlin}, \emph{P.PPLC}|pwk} Premiere. Die 
                     Übersetzung stammte von Salten\pwindex{Salten, Felix 06.09.1869 – 08.10.1945@\textsc{Salten, Felix} (06.09.1869 – 08.10.1945), \emph{Schriftsteller/Schriftstellerin, Journalist/Journalistin, Chefredakteur/Chefredakteurin}|pwk} (vgl. A. S.: \emph{Tagebuch}, 2. 1. 1910).
                     Auch Schnitzler hatte vornehmlich schlechte
                     Erfahrungen mit Max
                     Reinhardt\pwindex{Reinhardt, Max 09.09.1873 – 30.10.1943@\textsc{Reinhardt, Max} (09.09.1873 – 30.10.1943), \emph{Theaterleiter/Theaterleiterin, Regisseur/Regisseurin, Schauspieler/Schauspielerin}|pwk}, zuletzt rund um seine Einreichung von \emph{Der junge Medardus}\pwindex{junge Medardus. Dramatische Historie in einem Vorspiel und fuenf Aufzuegen@\emph{Der junge Medardus. Dramatische Historie in einem Vorspiel und fünf Aufzügen}|pwk} (vgl. \emph{Der Briefwechsel Arthur
                        Schnitzlers mit Max Reinhardt und dessen Mitarbeitern}. Herausgegeben
                     von Renate Wagner. Salzburg: \emph{Otto Müller
                        Verlag}{ }1971, S. 60–79).  
                     }}}\label{K_L03545-1}! Ich ärgere mich nicht mehr, aber
               ich habe eben eine Reise getan, und kann etwas erzählen!\pend
           
\pstart
           Hoffentlich bald! Herzliche Grüße von Haus zu Haus {\\[\baselineskip]}Ihr {\\[\baselineskip]}\spacefill\mbox{Felix Salten}\pend
           \leftskip=0em{}
\pstart
           Berlin\oindex{Berlin@\textbf{Berlin}, \emph{P.PPLC}|pw}{ }17. I. \textcolor{gray}{10}\pend
           \selectlanguage{ngerman}\endnumbering\briefempfaengerindex{Schnitzler, Arthur@\textsc{Schnitzler, Arthur}!zzzSalten, Felix@\emph{von Felix Salten}!1910-01-171@{17. 1. 1910}|)be}\mylabel{L03545h}  \normalsize

\doendnotes{C}
\bigskip
\vfill

\clearpage

\footnotesize

\lohead{\textsc{register}}

% Definiere theindex-Environment komplett neu ohne reledmac
\makeatletter
\renewenvironment{theindex}{%
  \section*{\indexname}%
  \setlength{\parindent}{0pt}%
  \setlength{\parskip}{0pt plus 0.3pt}%
  \let\item\@idxitem
}{%
  \clearpage
}
\makeatother

\IfFileExists{\jobname-pw.ind}{\input{\jobname-pw.ind}}{}

\end{document}

      