%% latex-leseansicht-vorspann.tex
%% Vorspann für die Leseansicht.
%% Lädt die gemeinsame Datei latex-vorspann.tex mit nicht gesetztem Schalter.

\newif\ifkorrekturansicht
\korrekturansichtfalse

\input{../tex-inputs/latex-vorspann}


\section[Arthur Schnitzler an Richard Beer-Hofmann, {[}30. 7.?{]} 1893]{L00247 Arthur Schnitzler an Richard Beer-Hofmann, [30. 7.?] 1893}
\nopagebreak\mylabel{L00247v}
\rehead{ }\normalsize\beginnumbering\briefempfaengerindex{Beer-Hofmann, Richard@\textsc{Beer-Hofmann, Richard}!zzzSchnitzler, Arthur@\emph{von Arthur Schnitzler}!1893-07-301@{[30. 7.?] 1893}|(be}
\toendnotes[C]{\smallbreak\pagebreak[2]}
\correspDesc{Versand  durch Arthur Schnitzler am [30. 7.?] 1893 in Wien
\newline{}Erhalt  durch Richard Beer-Hofmann am [30. 7.?] 1893 in Bad Ischl}\toendnotes[C]{\smallbreak}
\Standort{YCGL, MSS 31.}
\physDesc{Telegramm, 175 Zeichen
\newline{}maschinell
\newline{}Versand: mit Bleistift Eintragung am Vordruck: »\noindent{}\textcolor{gray}{\textbf{Aufgenommen von}} Wi06{ / }\textcolor{gray}{\textbf{auf Leitung Nr. ..........}}{ / }\textcolor{gray}{\textbf{am}}{ }\textcolor{gray}{30/7} 93{ }\textcolor{gray}{\textbf{um}}{ }21\textcolor{gray}{\textbf{Uhr}}{ }{\dots}\textcolor{gray}{\textbf{Min.}}{ }Vor\textcolor{gray}{\textbf{Mittag}}« }\toendnotes[C]{\smallbreak}\pstart{}{\pb}richard beer hofmann ischl\oindex{Bad Ischl@\textbf{Bad Ischl}|pw}\pend{}\pstart{}schulgasze 8\oindex{Schulgasse@\textbf{Schulgasse}, \emph{Straße}|pw}\pend{}{\bigskip}\vspace{1em}
\pstart
           {\pb}ischl\oindex{Bad Ischl@\textbf{Bad Ischl}|pw} fr wien\oindex{Wien@\textbf{Wien}, \emph{Verwaltungsgebiet}|pw}{ }10+1166{ }20{ }1+\pend
           \vspace{0.5em}
\pstart
           abschreiber\pwindex{?? [Schreibkraft für Arthur Schnitzler] @\textsc{?? [Schreibkraft für Arthur Schnitzler]}|pwv} brachte trotz
               wiederholten draengens die novelle\pwindex{Beer-Hofmann, Richard 11.\,7.\,1866 Wien – 26.\,9.\,1945 New York City@\textsc{Beer-Hofmann, Richard} (11.\,7.\,1866 Wien – 26.\,9.\,1945 New York City), \emph{Schriftsteller}!Kind@\strich\emph{Das Kind}|pwv} heute nicht, morgen sicher\pend
           \pstart herzlichen grusz \spacefill\mbox{= arthur =}\pend{}\selectlanguage{ngerman}\endnumbering\briefempfaengerindex{Beer-Hofmann, Richard@\textsc{Beer-Hofmann, Richard}!zzzSchnitzler, Arthur@\emph{von Arthur Schnitzler}!1893-07-301@{[30. 7.?] 1893}|)be}\mylabel{L00247h}  \newcommand{\dateiname}{L00247}\newcommand{\titel}{Arthur Schnitzler an Richard Beer-Hofmann, [30. 7.?] 1893}\newcommand{\editorInnen}{Martin Anton Müller und Gerd-Hermann Susen}%% latex-leseansicht-abspann.tex
%% Abspann für die Leseansicht.
%% Der Schalter \ifkorrekturansicht ist bereits durch den Vorspann gesetzt.

%% latex-abspann.tex
%% Gemeinsamer Abspann für Korrekturansicht und Leseansicht.
%% Setzt den Schalter \ifkorrekturansicht voraus (gesetzt in den
%% einbindenden Dateien latex-korrekturansicht-abspann.tex bzw.
%% latex-leseansicht-abspann.tex).
%% ---------------------------------------------------------------

\normalsize

% Das esempio-Environment wird nur in der Leseansicht benötigt
\ifkorrekturansicht\else
\newenvironment{esempio}[3]%
{
    \vspace{1.5ex}
    \rlap{\underline{#1}}
    \par
    \setlength{\parindent}{0cm}
    \nopagebreak
    \leftskip=#2cm
    \rightskip=#3cm
}
{
    \par
}
\fi

\doendnotes{C}
\bigskip
\vfill

\clearpage

\footnotesize

\ifkorrekturansicht
  \lohead{\textsc{register}}
\fi

% theindex-Environment neu definieren ohne reledmac
\makeatletter
\renewenvironment{theindex}{%
  \ifkorrekturansicht
    \section*{\indexname}%
  \else
    \subsubsection*{Index der erwähnten Entitäten}%
  \fi
  \setlength{\parindent}{0pt}%
  \setlength{\parskip}{0pt plus 0.3pt}%
  \let\item\@idxitem
}{%
  \ifkorrekturansicht\clearpage\fi
}
\makeatother

\IfFileExists{\jobname-pw.ind}{\input{\jobname-pw.ind}}{}

% Quellenangabe nur in der Leseansicht
\ifkorrekturansicht\else
% Fallback-Definitionen, falls die .tex-Datei \titel etc. nicht gesetzt hat
\providecommand{\titel}{}
\providecommand{\editorInnen}{}
\providecommand{\dateiname}{\jobname}

\vspace{3cm}

\vfill

\footnotesize
\textsc{Quelle}: \titel. Herausgegeben von {\editorInnen}. In: \emph{Arthur Schnitzler: Briefwechsel mit Autorinnen und Autoren}.
 Digitale Edition, https://schnitzler-briefe.acdh.oeaw.ac.at/{\dateiname}.html (Stand \today)
\fi

\end{document}


