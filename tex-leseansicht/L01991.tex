%% latex-leseansicht-vorspann.tex
%% Vorspann für die Leseansicht.
%% Lädt die gemeinsame Datei latex-vorspann.tex mit nicht gesetztem Schalter.

\newif\ifkorrekturansicht
\korrekturansichtfalse

\input{../tex-inputs/latex-vorspann}


\section[Georg Brandes an Arthur Schnitzler, 18. 12. 1910]{L01991 Georg Brandes an Arthur Schnitzler, 18. 12. 1910}
\nopagebreak\mylabel{L01991v}
\rehead{ }\normalsize\beginnumbering\briefempfaengerindex{Schnitzler, Arthur@\textsc{Schnitzler, Arthur}!zzzBrandes, Georg@\emph{von Georg Brandes}!1910-12-181@{18. 12. 1910}|(be}
\toendnotes[C]{\smallbreak\pagebreak[2]}
\correspDesc{Versand  durch Georg Brandes am 18. 12. 1910 in Kopenhagen
\newline{}Erhalt  durch Arthur Schnitzler im Zeitraum [19. 12. 1910 – 23. 12. 1910?] in Wien}\toendnotes[C]{\smallbreak}
\Standort{CUL, Schnitzler, B 17.}
\physDesc{Brief, 1 Blatt, 3 Seiten, 1524 Zeichen
\newline{}Handschrift: schwarze Tinte, lateinische Kurrent
\newline{}Schnitzler: mit Bleistift beschriftet: »\textsc{Brandes}« 
\newline{}Ordnung: mit Bleistift von unbekannter Hand nummeriert: »34« }
\buchAbdrucke{\weitereDrucke{Georg Brandes, Arthur Schnitzler: \emph{Ein Briefwechsel}. Herausgegeben von Kurt Bergel. Bern: \emph{Francke} 1956, S. 98.} }\toendnotes[C]{\smallbreak}
\pstart
           \raggedleft{}{\pb}\uline{Kopenhagen}\oindex{Kopenhagen@\textbf{Kopenhagen}, \emph{Hauptstadt}|pw}{\\}18. 12. 10\pend
           
\pstart{}Verehrter Freund\pend\vspace{0.5em}
\pstart
           Wenn ich Sie lese, thut es mir leid, dass ich so weit von Ihnen wohne und so selten
               Gelegenheit habe, mit Ihnen einige Worte zu wechseln.\pend
           
\pstart
           \uline{Medardus}\pwindex{Schnitzler, Arthur 15.\,5.\,1862 Wien – 21.\,10.\,1931 ebd.@\textsc{Schnitzler, Arthur} (15.\,5.\,1862 Wien – 21.\,10.\,1931 ebd.), \emph{Schriftsteller, Mediziner}!junge Medardus. Dramatische Historie in einem Vorspiel und fünf Aufzügen@\strich\emph{Der junge Medardus. Dramatische Historie in einem Vorspiel und fünf Aufzügen}|pw} habe ich sehr genau gelesen, laut vorgelesen, um es recht zu würdigen. Sie
               haben dort ein reiches Bild aufgerollt. Mit Ueberraschung und Freude erfuhr ich aus
               einer Zeitungs\label{K_L01991-1v}\edtext{notits}{\lemma{\textnormal{\emph{notits}}}\Cendnote{\textnormal{dänisch: Notiz}}}\label{K_L01991-1}, dass das Stück trotz
               seiner epischen Anlage erfolgreich aufgeführt worden ist. Die – im Goetheschen\pwindex{Goethe, Johann Wolfgang von 28.\,8.\,1749 Frankfurt am Main – 22.\,3.\,1832 Weimar@\textsc{Goethe, Johann Wolfgang von} (28.\,8.\,1749 Frankfurt am Main – 22.\,3.\,1832 Weimar), \emph{Schriftsteller}|pw} Sinn über Kleist\pwindex{Kleist, Heinrich von 18.\,10.\,1777 Frankfurt (Oder) – 21.\,11.\,1811 Kleiner Wannsee@\textsc{Kleist, Heinrich von} (18.\,10.\,1777 Frankfurt (Oder) – 21.\,11.\,1811 Kleiner Wannsee), \emph{Schriftsteller}|pw} – \strikeout{\textcolor{gray}{V}} fesselnde »\label{K_L01991-2v}\edtext{Verwirrung des
                  Gefühls}{\lemma{\textnormal{\emph{Verwirrung des
                  Gefühls}}}\Cendnote{\textnormal{Diese Formulierung findet sich
                  in Goethes\pwindex{Goethe, Johann Wolfgang von 28.\,8.\,1749 Frankfurt am Main – 22.\,3.\,1832 Weimar@\textsc{Goethe, Johann Wolfgang von} (28.\,8.\,1749 Frankfurt am Main – 22.\,3.\,1832 Weimar), \emph{Schriftsteller}|pwk}{ }Tagebuch\pwindex{Goethe, Johann Wolfgang von 28.\,8.\,1749 Frankfurt am Main – 22.\,3.\,1832 Weimar@\textsc{Goethe, Johann Wolfgang von} (28.\,8.\,1749 Frankfurt am Main – 22.\,3.\,1832 Weimar), \emph{Schriftsteller}!Tagebücher@\strich\emph{Tagebücher}|pwkv} am
                     13. 7. 1807. }}}\label{K_L01991-2}« in Medardus\pwindex{Schnitzler, Arthur 15.\,5.\,1862 Wien – 21.\,10.\,1931 ebd.@\textsc{Schnitzler, Arthur} (15.\,5.\,1862 Wien – 21.\,10.\,1931 ebd.), \emph{Schriftsteller, Mediziner}!junge Medardus. Dramatische Historie in einem Vorspiel und fünf Aufzügen@\strich\emph{Der junge Medardus. Dramatische Historie in einem Vorspiel und fünf Aufzügen}|pwv} ist so recht Ihre Domäne. {\pb}Sehr fein ist die schwache
               Andeutung \strikeout{der} einer geistigen Verwandtschaft zwischen
                  Helene\pwindex{Schnitzler, Arthur 15.\,5.\,1862 Wien – 21.\,10.\,1931 ebd.@\textsc{Schnitzler, Arthur} (15.\,5.\,1862 Wien – 21.\,10.\,1931 ebd.), \emph{Schriftsteller, Mediziner}!junge Medardus. Dramatische Historie in einem Vorspiel und fünf Aufzügen@\strich\emph{Der junge Medardus. Dramatische Historie in einem Vorspiel und fünf Aufzügen}|pwv} und Napoleon\pwindex{Schnitzler, Arthur 15.\,5.\,1862 Wien – 21.\,10.\,1931 ebd.@\textsc{Schnitzler, Arthur} (15.\,5.\,1862 Wien – 21.\,10.\,1931 ebd.), \emph{Schriftsteller, Mediziner}!junge Medardus. Dramatische Historie in einem Vorspiel und fünf Aufzügen@\strich\emph{Der junge Medardus. Dramatische Historie in einem Vorspiel und fünf Aufzügen}|pwv}.\pend
           
\pstart
           Die ganze Wien\oindex{Wien@\textbf{Wien}, \emph{Verwaltungsgebiet}|pw}eratmosphäre vor 100 Jahren haben
               Sie geben wollen. Und wenn ich nicht irre, lag es Ihnen besonders am Herzen, zu
               zeigen, auf welchem Hintergrund von Spiessbürgerlichkeit und lässiger Frivolität, die
               in Wien\oindex{Wien@\textbf{Wien}, \emph{Verwaltungsgebiet}|pw} zu Hause \substVorne{}\textsuperscript{sind}\substDazwischen{}waren\substHinten{}, und auf welchem Hintergrund von unnationalem Wesen und Gehorsam dem
               Eroberer gegenüber, die in Deutschland\oindex{Deutschland@\textbf{Deutschland}|pw}
               hervortraten, der Heroismus einiger Weniger sich geltend machte. Eine nachsichtige
               Menschenverachtung durchdringt das Schaupiel und findet u. a. in mir ein Echo.\pend
           
\pstart
           Ich möchte immer gerne wissen, wie es {\pb}Ihnen geht und wie es Beer-Hofmann\pwindex{Beer-Hofmann, Richard 11.\,7.\,1866 Wien – 26.\,9.\,1945 New York City@\textsc{Beer-Hofmann, Richard} (11.\,7.\,1866 Wien – 26.\,9.\,1945 New York City), \emph{Schriftsteller}|pw} geht, den ich (vor 16 Jahren,
               glaube ich) mit Ihnen kennen lernte.\pend
           
\pstart
           Ueber mich selbst ist nichts Interessantes, wenigstens nichts besonders Gutes zu
               melden. Ich bin nicht krank.\pend
           
\pstart
           Haben Sie für die Treue Dank, womit Sie bei jeder neuen Arbeit auch an mich
               denken.\pend
           
\pstart
           Ich bin Ihr unveränderlicher Freund{\\[\baselineskip]}\spacefill\mbox{Georg Brandes}\pend
           \leftskip=0em{}\selectlanguage{ngerman}\endnumbering\briefempfaengerindex{Schnitzler, Arthur@\textsc{Schnitzler, Arthur}!zzzBrandes, Georg@\emph{von Georg Brandes}!1910-12-181@{18. 12. 1910}|)be}\mylabel{L01991h}  \newcommand{\dateiname}{L01991}\newcommand{\titel}{Georg Brandes an Arthur Schnitzler, 18. 12. 1910}\newcommand{\editorInnen}{Martin Anton Müller und Gerd-Hermann Susen}%% latex-leseansicht-abspann.tex
%% Abspann für die Leseansicht.
%% Der Schalter \ifkorrekturansicht ist bereits durch den Vorspann gesetzt.

%% latex-abspann.tex
%% Gemeinsamer Abspann für Korrekturansicht und Leseansicht.
%% Setzt den Schalter \ifkorrekturansicht voraus (gesetzt in den
%% einbindenden Dateien latex-korrekturansicht-abspann.tex bzw.
%% latex-leseansicht-abspann.tex).
%% ---------------------------------------------------------------

\normalsize

% Das esempio-Environment wird nur in der Leseansicht benötigt
\ifkorrekturansicht\else
\newenvironment{esempio}[3]%
{
    \vspace{1.5ex}
    \rlap{\underline{#1}}
    \par
    \setlength{\parindent}{0cm}
    \nopagebreak
    \leftskip=#2cm
    \rightskip=#3cm
}
{
    \par
}
\fi

\doendnotes{C}
\bigskip
\vfill

\clearpage

\footnotesize

\ifkorrekturansicht
  \lohead{\textsc{register}}
\fi

% theindex-Environment neu definieren ohne reledmac
\makeatletter
\renewenvironment{theindex}{%
  \ifkorrekturansicht
    \section*{\indexname}%
  \else
    \subsubsection*{Index der erwähnten Entitäten}%
  \fi
  \setlength{\parindent}{0pt}%
  \setlength{\parskip}{0pt plus 0.3pt}%
  \let\item\@idxitem
}{%
  \ifkorrekturansicht\clearpage\fi
}
\makeatother

\IfFileExists{\jobname-pw.ind}{\input{\jobname-pw.ind}}{}

% Quellenangabe nur in der Leseansicht
\ifkorrekturansicht\else
% Fallback-Definitionen, falls die .tex-Datei \titel etc. nicht gesetzt hat
\providecommand{\titel}{}
\providecommand{\editorInnen}{}
\providecommand{\dateiname}{\jobname}

\vspace{3cm}

\vfill

\footnotesize
\textsc{Quelle}: \titel. Herausgegeben von {\editorInnen}. In: \emph{Arthur Schnitzler: Briefwechsel mit Autorinnen und Autoren}.
 Digitale Edition, https://schnitzler-briefe.acdh.oeaw.ac.at/{\dateiname}.html (Stand \today)
\fi

\end{document}


