%% latex-leseansicht-vorspann.tex
%% Vorspann für die Leseansicht.
%% Lädt die gemeinsame Datei latex-vorspann.tex mit nicht gesetztem Schalter.

\newif\ifkorrekturansicht
\korrekturansichtfalse

\input{../tex-inputs/latex-vorspann}


         
         \renewcommand{\erwaehntePersonen}{Personen: Richard Beer-Hofmann, Georg Brandes, Johann Wolfgang von Goethe, Heinrich von Kleist}
         \renewcommand{\erwaehnteOrte}{Orte: Deutschland, Kopenhagen, Wien}
         \renewcommand{\erwaehnteWerke}{Werke: Der junge Medardus. Dramatische Historie in einem Vorspiel und fünf Aufzügen, Tagebücher}
               \section[Georg Brandes an Arthur Schnitzler, 18. 12. 1910]{ Georg Brandes an Arthur Schnitzler, 18. 12. 1910}\nopagebreak\mylabel{v}\rehead{ }\begin{ledgroupsized}[t]{13cm}\normalsize\beginnumbering\briefempfaengerindex{Schnitzler, Arthur@\textsc{Schnitzler, Arthur}!zzzBrandes, Georg@\emph{von Georg Brandes}!1910-12-181@{18. 12. 1910}|(be} \toendnotes[C]{\smallbreak\pagebreak[2]} \Standort{CUL, Schnitzler, B 17.}
\physDesc{Brief, 1 Blatt, 3 Seiten, 1524 Zeichen
\newline{}Handschrift: schwarze Tinte, lateinische Kurrent
\newline{}Schnitzler: mit Bleistift beschriftet: »\textsc{Brandes}« 
\newline{}Ordnung: mit Bleistift von unbekannter Hand nummeriert:
                                    »34« }\buchAbdrucke{\weitereDrucke{Georg Brandes, Arthur Schnitzler: \emph{Ein Briefwechsel}. Hg. Kurt Bergel. Bern: \emph{Francke} 1956, S. 98.} }\toendnotes[C]{\smallbreak}\pstart
           \raggedleft{}{\pb}\uline{Kopenhagen}\oindex{Kopenhagen@\textbf{Kopenhagen}|pw}{\\}18. 12. 10\pend
           \pstart{}Verehrter Freund\pend\pstart
           Wenn ich Sie lese, thut es mir leid, dass ich so weit von Ihnen wohne und so selten
               Gelegenheit habe, mit Ihnen einige Worte zu wechseln.\pend
           \pstart
           \uline{Medardus}\pwindex{Schnitzler, Arthur 15.05.1862 – 21.10.1931@\textsc{Schnitzler, Arthur} (15.05.1862 – 21.10.1931), \emph{Schriftsteller, Mediziner}!junge Medardus. Dramatische Historie in einem Vorspiel und fuenf
                  Aufzuegen1910-10-26@\strich\emph{Der junge Medardus. Dramatische Historie in einem Vorspiel und fünf Aufzügen} {[}1910-10-26{]}|pw} habe ich sehr genau gelesen, laut vorgelesen, um es recht zu würdigen. Sie
               haben dort ein reiches Bild aufgerollt. Mit Ueberraschung und Freude erfuhr ich aus
               einer Zeitungs\label{K_L01991-1v}\edtext{notits}{\lemma{\textnormal{\emph{notits}}}\Cendnote{\textnormal{dänisch: Notiz}}}\label{K_L01991-1h}, dass das Stück trotz
               seiner epischen Anlage erfolgreich aufgeführt worden ist. Die – im Goethe\pwindex{Goethe, Johann Wolfgang von 1749-08-28 – 1832-03-22@\textsc{Goethe, Johann Wolfgang von} (1749-08-28 – 1832-03-22), \emph{Schriftsteller}|pw}schen Sinn über Kleist\pwindex{Kleist, Heinrich von 18.10.1777 – 21.11.1811@\textsc{Kleist, Heinrich von} (18.10.1777 – 21.11.1811), \emph{Schriftsteller}|pw} – \strikeout{\textcolor{gray}{V}} fesselnde »\label{K_L01991-2v}\edtext{Verwirrung des
                  Gefühls}{\lemma{\textnormal{\emph{Verwirrung des
                  Gefühls}}}\Cendnote{\textnormal{Äußerung Goethe\pwindex{Goethe, Johann Wolfgang von 1749-08-28 – 1832-03-22@\textsc{Goethe, Johann Wolfgang von} (1749-08-28 – 1832-03-22), \emph{Schriftsteller}|pwk}s in seinem Tagebuch\pwindex{Goethe, Johann Wolfgang von 1749-08-28 – 1832-03-22@\textsc{Goethe, Johann Wolfgang von} (1749-08-28 – 1832-03-22), \emph{Schriftsteller}!Tagebuecher1775 – 1832@\strich\emph{Tagebücher} {[}1775 – 1832{]}|pwkv}, 13. 7. 1807}}}\label{K_L01991-2h}« in Medardus\pwindex{Schnitzler, Arthur 15.05.1862 – 21.10.1931@\textsc{Schnitzler, Arthur} (15.05.1862 – 21.10.1931), \emph{Schriftsteller, Mediziner}!junge Medardus. Dramatische Historie in einem Vorspiel und fuenf
                  Aufzuegen1910-10-26@\strich\emph{Der junge Medardus. Dramatische Historie in einem Vorspiel und fünf Aufzügen} {[}1910-10-26{]}|pwv} ist so
               recht Ihre Domäne. {\pb}Sehr fein ist
               die schwache Andeutung \strikeout{der} einer geistigen
               Verwandtschaft zwischen Helene\pwindex{Schnitzler, Arthur 15.05.1862 – 21.10.1931@\textsc{Schnitzler, Arthur} (15.05.1862 – 21.10.1931), \emph{Schriftsteller, Mediziner}!junge Medardus. Dramatische Historie in einem Vorspiel und fuenf
                  Aufzuegen1910-10-26@\strich\emph{Der junge Medardus. Dramatische Historie in einem Vorspiel und fünf Aufzügen} {[}1910-10-26{]}|pwv}
               und Napoleon\pwindex{Schnitzler, Arthur 15.05.1862 – 21.10.1931@\textsc{Schnitzler, Arthur} (15.05.1862 – 21.10.1931), \emph{Schriftsteller, Mediziner}!junge Medardus. Dramatische Historie in einem Vorspiel und fuenf
                  Aufzuegen1910-10-26@\strich\emph{Der junge Medardus. Dramatische Historie in einem Vorspiel und fünf Aufzügen} {[}1910-10-26{]}|pwv}.\pend
           \pstart
           Die ganze Wien\oindex{Wien@\textbf{Wien}|pw}eratmosphäre vor 100 Jahren haben
               Sie geben wollen. Und wenn ich nicht irre, lag es Ihnen besonders am Herzen, zu
               zeigen, auf welchem Hintergrund von Spiessbürgerlichkeit und lässiger Frivolität, die
               in Wien\oindex{Wien@\textbf{Wien}|pw} zu Hause \substVorne{}\textsuperscript{sind}\substDazwischen{}waren\substHinten{}, und auf welchem Hintergrund von unnationalem Wesen und Gehorsam dem
               Eroberer gegenüber, die in Deutschland\oindex{Deutschland@\textbf{Deutschland}|pw}
               hervortraten, der Heroismus einiger Weniger sich geltend machte. Eine nachsichtige
               Menschenverachtung durchdringt das Schaupiel und findet u. a. in mir ein Echo.\pend
           \pstart
           Ich möchte immer gerne wissen, wie es {\pb}Ihnen geht und wie es Beer-Hofmann\pwindex{Beer-Hofmann, Richard 1866-07-11 – 1945-09-26@\textsc{Beer-Hofmann, Richard} (1866-07-11 – 1945-09-26), \emph{Schriftsteller}|pw} geht, den ich (vor 16 Jahren,
               glaube ich) mit Ihnen kennen lernte.\pend
           \pstart
           Ueber mich selbst ist nichts Interessantes, wenigstens nichts besonders Gutes zu
               melden. Ich bin nicht krank.\pend
           \pstart
           Haben Sie für die Treue Dank, womit Sie bei jeder neuen Arbeit auch an mich
               denken.\pend
           \pstart
           Ich bin Ihr unveränderlicher Freund{\\[\baselineskip]}\spacefill\mbox{Georg Brandes}\pend
           \leftskip=0em{}
         
         \endnumbering\mylabel{h}\end{ledgroupsized}  \newcommand{\dateiname}{L01991}\newcommand{\titel}{Georg Brandes an Arthur Schnitzler, 18. 12. 1910}\newcommand{\editorInnen}{Martin Anton Müller und Gerd-Hermann Susen}%% latex-leseansicht-abspann.tex
%% Abspann für die Leseansicht.
%% Der Schalter \ifkorrekturansicht ist bereits durch den Vorspann gesetzt.

%% latex-abspann.tex
%% Gemeinsamer Abspann für Korrekturansicht und Leseansicht.
%% Setzt den Schalter \ifkorrekturansicht voraus (gesetzt in den
%% einbindenden Dateien latex-korrekturansicht-abspann.tex bzw.
%% latex-leseansicht-abspann.tex).
%% ---------------------------------------------------------------

\normalsize

% Das esempio-Environment wird nur in der Leseansicht benötigt
\ifkorrekturansicht\else
\newenvironment{esempio}[3]%
{
    \vspace{1.5ex}
    \rlap{\underline{#1}}
    \par
    \setlength{\parindent}{0cm}
    \nopagebreak
    \leftskip=#2cm
    \rightskip=#3cm
}
{
    \par
}
\fi

\doendnotes{C}
\bigskip
\vfill

\clearpage

\footnotesize

\ifkorrekturansicht
  \lohead{\textsc{register}}
\fi

% theindex-Environment neu definieren ohne reledmac
\makeatletter
\renewenvironment{theindex}{%
  \ifkorrekturansicht
    \section*{\indexname}%
  \else
    \subsubsection*{Index der erwähnten Entitäten}%
  \fi
  \setlength{\parindent}{0pt}%
  \setlength{\parskip}{0pt plus 0.3pt}%
  \let\item\@idxitem
}{%
  \ifkorrekturansicht\clearpage\fi
}
\makeatother

\IfFileExists{\jobname-pw.ind}{\input{\jobname-pw.ind}}{}

% Quellenangabe nur in der Leseansicht
\ifkorrekturansicht\else
% Fallback-Definitionen, falls die .tex-Datei \titel etc. nicht gesetzt hat
\providecommand{\titel}{}
\providecommand{\editorInnen}{}
\providecommand{\dateiname}{\jobname}

\vspace{3cm}

\vfill

\footnotesize
\textsc{Quelle}: \titel. Herausgegeben von {\editorInnen}. In: \emph{Arthur Schnitzler: Briefwechsel mit Autorinnen und Autoren}.
 Digitale Edition, https://schnitzler-briefe.acdh.oeaw.ac.at/{\dateiname}.html (Stand \today)
\fi

\end{document}


      