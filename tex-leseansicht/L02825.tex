%% latex-leseansicht-vorspann.tex
%% Vorspann für die Leseansicht.
%% Lädt die gemeinsame Datei latex-vorspann.tex mit nicht gesetztem Schalter.

\newif\ifkorrekturansicht
\korrekturansichtfalse

\input{../tex-inputs/latex-vorspann}


\section[ Paul Goldmann an Arthur Schnitzler, {[}17./18.?{]} 9. 1897]{L02825 Paul Goldmann an Arthur Schnitzler,  [17./18.?] 9. 1897}
\nopagebreak\mylabel{L02825v}
\rehead{ }\normalsize\beginnumbering\briefempfaengerindex{Schnitzler, Arthur@\textsc{Schnitzler, Arthur}!zzzGoldmann, Paul@\emph{von Paul Goldmann}!1897-09-182@{[17./18.?] 9. 1897}|(be}
\toendnotes[C]{\smallbreak\pagebreak[2]}
\correspDesc{Versand  durch Paul Goldmann im Zeitraum [17./18.?] 9. 1897 in Frankfurt am Main
\newline{}Erhalt  durch Arthur Schnitzler im Zeitraum [18. 9. 1897
                  – 22. 9. 1897?] in Wien}\toendnotes[C]{\smallbreak}
\Standort{DLA, A:Schnitzler, HS.NZ85.1.3167.}
\physDesc{Brief, 1 Blatt, 3 Seiten, 1164 Zeichen
\newline{}Handschrift: blaue Tinte, deutsche Kurrent}\toendnotes[C]{\smallbreak}
\pstart
           {\pb}\textcolor{gray}{\textbf{Frankfurter Zeitung}}\orgindex{Frankfurter Zeitung@Frankfurter Zeitung|pw}\hfill \textcolor{gray}{\textbf{Frankfurt a. M.\oindex{Frankfurt am Main@\textbf{Frankfurt am Main}, \emph{Hauptstadt}|pw},}}{ }\label{K_L02825-1v}\edtext{17. Sept. \textcolor{gray}{\textbf{189}}7}{\lemma{\textnormal{\emph{17. Sept. 1897}}}\Cendnote{\textnormal{Dieser und der vorangegangene
                        Brief (XXXX Auszeichnungsfehler: Dokument L02824 nicht gefunden) sind auf den
                        gleichen Tag datiert, in diesem Brief wird aber auf den vorangegangenen als
                           »geſtrigen Brief« verwiesen, wodurch entweder der
                        vorliegende auf den 18. 9. 1897 oder
                        andernfalls der frühere auf den 16. 9. 1897
                        zu datieren wäre.}}}\label{K_L02825-1}.\pend
           
\pstart
           \textcolor{gray}{\textbf{und}}\pend
           
\pstart
           \textcolor{gray}{\textbf{Handelsblatt.}}\pend
           
\pstart
           \textcolor{gray}{\textbf{Redaktion\orgindex{Frankfurter Zeitung@Frankfurter Zeitung|pwv}.\footnote{\noindent{}\textcolor{gray}{\textbf{Für die Redaktion\orgindex{Frankfurter Zeitung@Frankfurter Zeitung|pwv} beſtimmte Briefe und Sendungen wolle man
                                 \so{nicht} an die Perſon eines Redakteurs,{ }ſondern{ }ſtets \textbf{an die Redaktion der Frankfurter Zeitung\orgindex{Frankfurter Zeitung@Frankfurter Zeitung|pw}} adreſſiren.}}}}}\pend
           
\pstart
           \textcolor{gray}{\textbf{Telegramm-Adreſſe:}}\pend
           
\pstart
           \textcolor{gray}{\textbf{Zeitung\orgindex{Frankfurter Zeitung@Frankfurter Zeitung|pwv}{ }Frankfurt Main\oindex{Frankfurt am Main@\textbf{Frankfurt am Main}, \emph{Hauptstadt}|pw}.}}\pend
           
\pstart\center{}Mein lieber Freund,\pend\vspace{0.5em}
\pstart
           Ich will Dir nur noch raſch für Deinen lieben Brief danken, den ich heut bekam.\pend
           
\pstart
           Sieh’ nicht{ }ſo trübſelig in die Zukunft und laß’ die Wolken machen, was{ }ſie wollen.
               Dein Lebensweg liegt klar und{ }ſchön vor meinen Blicken, und ich{ }ſehe beſſer, weil
               Deine augenblicklichen \label{K_L02825-2v}\edtext{Verſtimmungen}{\lemma{\textnormal{\emph{Verstimmungen}}}\Cendnote{\textnormal{wohl aufgrund von Schnitzlers Affäre mit der verheirateten Rosa Freudenthal\pwindex{Freudenthal, Rosa 1862 – 18.\,6.\,1905 Berlin@\textsc{Freudenthal, Rosa} (1862 – 18.\,6.\,1905 Berlin)|pwk} und der noch immer relativ
                  geheim gehaltenen Schwangerschaft Marie Reinhards\pwindex{Reinhard, Marie 13.\,3.\,1871 Wien – 18.\,3.\,1899 ebd.@\textsc{Reinhard, Marie} (13.\,3.\,1871 Wien – 18.\,3.\,1899 ebd.), \emph{Gesangspädagogin}|pwk}}}}\label{K_L02825-2} mir nicht die Ausſicht verdunkeln. Du wirſt wieder Ruhe
               bekommen, wirſt wieder arbeiten und dann wirſt Du{ }ſelbſt wieder {\pb}beſſer und heiterer geſtimmt{ }ſein. Ich meine, das
               Nöthigſte wäre für Dich, daß Du{ }ſo raſch als möglich die Arbeit wieder aufzunehmen{ }ſuchteſt.\pend
           
\pstart
           Mein Schwager\pwindex{Rosengart, Josef 8.\,2.\,1860 Laupheim – 4.\,8.\,1927 Frankfurt am Main@\textsc{Rosengart, Josef} (8.\,2.\,1860 Laupheim – 4.\,8.\,1927 Frankfurt am Main), \emph{Arzt}|pwv} hat{ }ſich über d\substVorne{}\textsuperscript{ie}\substDazwischen{}en\substHinten{} »\label{K_L02825-3v}\edtext{Bauernfänger}{\lemma{\textnormal{\emph{Bauernfänger}}}\Cendnote{\textnormal{Bezug unklar}}}\label{K_L02825-3}«{ }ſehr amüſirt, bleibt
               aber \label{K_L02825-4v}\edtext{betreffs des Ohrenklingens}{\lemma{\textnormal{\emph{betreffs des Ohrenklingens}}}\Cendnote{\textnormal{Siehe XXXX Auszeichnungsfehler: Dokument L02823 nicht gefunden. }}}\label{K_L02825-4}
               unerſchütterlich bei{ }ſeiner Anſicht.\pend
           
\pstart
           Wenn ich Deine Andeutungen bezüglich \label{K_L02825-5v}\edtext{Fräulein \textsc{G.\pwindex{Glümer, Marie 3.\,7.\,1867 Wien – 16.\,11.\,1925 München@\textsc{Glümer, Marie} (3.\,7.\,1867 Wien – 16.\,11.\,1925 München), \emph{Schauspielerin}|pw}}}{\lemma{\textnormal{\emph{Fräulein G.}}}\Cendnote{\textnormal{Nur in Andeutungen im \emph{Tagebuch}\pwindex{Schnitzler, Arthur 15.\,5.\,1862 Wien – 21.\,10.\,1931 ebd.@\textsc{Schnitzler, Arthur} (15.\,5.\,1862 Wien – 21.\,10.\,1931 ebd.), \emph{Schriftsteller, Mediziner}!Tagebuch@\strich\emph{Tagebuch}|pwk} werden Momente einer komischen Geschichte klar:
                  Einerseits hatte Schnitzler am 30. 8. 1897 eine Karte
                  von Marie Glümer\pwindex{Glümer, Marie 3.\,7.\,1867 Wien – 16.\,11.\,1925 München@\textsc{Glümer, Marie} (3.\,7.\,1867 Wien – 16.\,11.\,1925 München), \emph{Schauspielerin}|pwk} erhalten, die an einen
                  anderen Liebhaber gerichtet gewesen sein dürfte. Am 3. 9. 1897 trat sie
                  erstmals wieder in Wien\oindex{Wien@\textbf{Wien}, \emph{Verwaltungsgebiet}|pwk} auf und ihm war es ein
                  Anliegen, dass sie von seiner erfolgten Rückkehr nichts wusste.}}}\label{K_L02825-5} richtig
               verſtanden habe,{ }ſo iſt das eine vollendet komiſche Geſchichte.\pend
           
\pstart
           Die \label{K_L02825-6v}\edtext{nächſte Woche}{\lemma{\textnormal{\emph{nächste Woche}}}\Cendnote{\textnormal{Eine Woche später, am 24. 9. 1897, kam es
                  zur Totgeburt des Sohns\pwindex{?? [Totgeborener Sohn von Arthur Schnitzler und Marie Reinhard] 24.\,9.\,1897 Endresstraße 68 – 24.\,9.\,1897 ebd.@\textsc{?? [Totgeborener Sohn von Arthur Schnitzler und Marie Reinhard]} (24.\,9.\,1897 Endresstraße 68 – 24.\,9.\,1897 ebd.)|pwkv}
                  von Schnitzler und Marie Reinhard\pwindex{Reinhard, Marie 13.\,3.\,1871 Wien – 18.\,3.\,1899 ebd.@\textsc{Reinhard, Marie} (13.\,3.\,1871 Wien – 18.\,3.\,1899 ebd.), \emph{Gesangspädagogin}|pwk} in Mauer bei
                     Wien\oindex{Wien@\textbf{Wien}!XXIII., Liesing@\textbf{XXIII., Liesing}!Mauer@\textbf{Mauer}|pwk}.}}}\label{K_L02825-6} wird alſo, wie ich aus Deinem Briefe entnehme, wichtig und
               ereignißreich werden. Ich wünſche Dir und Deiner {\pb}Freundin\pwindex{Reinhard, Marie 13.\,3.\,1871 Wien – 18.\,3.\,1899 ebd.@\textsc{Reinhard, Marie} (13.\,3.\,1871 Wien – 18.\,3.\,1899 ebd.), \emph{Gesangspädagogin}|pwv} von Herzen allen
               guten Muth in den bevorſtehenden{ }ſchweren Stunden.\pend
           
\pstart
           Auf meinen geſtrigen Brief antworteſt Du wohl baldmöglichſt.\pend
           
\pstart
           Die Meinigen grüßen Dich.\pend
           
\pstart
           In Treue {\\[\baselineskip]}Dein {\\[\baselineskip]}\spacefill\mbox{Paul Goldm}\pend
           \leftskip=0em{}
\pstart
           \noindent{}Was machen \textsc{Richard\pwindex{Beer-Hofmann, Richard 11.\,7.\,1866 Wien – 26.\,9.\,1945 New York City@\textsc{Beer-Hofmann, Richard} (11.\,7.\,1866 Wien – 26.\,9.\,1945 New York City), \emph{Schriftsteller}|pw}} und \label{K_L02825-7v}\edtext{\textsc{Richards\pwindex{Beer-Hofmann, Richard 11.\,7.\,1866 Wien – 26.\,9.\,1945 New York City@\textsc{Beer-Hofmann, Richard} (11.\,7.\,1866 Wien – 26.\,9.\,1945 New York City), \emph{Schriftsteller}|pw}}{ }Tochter\pwindex{Beer-Hofmann, Mirjam 4.\,9.\,1897 Wien – 24.\,12.\,1984 New York City@\textsc{Beer-Hofmann, Mirjam} (4.\,9.\,1897 Wien – 24.\,12.\,1984 New York City)|pwv}}{\lemma{\textnormal{\emph{Richards Tochter}}}\Cendnote{\textnormal{Mirjam\pwindex{Beer-Hofmann, Mirjam 4.\,9.\,1897 Wien – 24.\,12.\,1984 New York City@\textsc{Beer-Hofmann, Mirjam} (4.\,9.\,1897 Wien – 24.\,12.\,1984 New York City)|pwk}, die Tochter von Richard\pwindex{Beer-Hofmann, Richard 11.\,7.\,1866 Wien – 26.\,9.\,1945 New York City@\textsc{Beer-Hofmann, Richard} (11.\,7.\,1866 Wien – 26.\,9.\,1945 New York City), \emph{Schriftsteller}|pwk} und Paula
                        Beer-Hofmann\pwindex{Beer-Hofmann, Paula 25.\,2.\,1879 Wien – 30.\,10.\,1939 Zürich@\textsc{Beer-Hofmann, Paula} (25.\,2.\,1879 Wien – 30.\,10.\,1939 Zürich)|pwk}, war am 4. 9. 1897 zur Welt gekommen.}}}\label{K_L02825-7}?\pend
           \selectlanguage{ngerman}\endnumbering\briefempfaengerindex{Schnitzler, Arthur@\textsc{Schnitzler, Arthur}!zzzGoldmann, Paul@\emph{von Paul Goldmann}!1897-09-172@{[17./18.?] 9. 1897}|)be}\mylabel{L02825h}  \newcommand{\dateiname}{L02825}\newcommand{\titel}{Paul Goldmann an Arthur Schnitzler, [17./18.?] 9. 1897}\newcommand{\editorInnen}{Martin Anton Müller und Laura Untner}%% latex-leseansicht-abspann.tex
%% Abspann für die Leseansicht.
%% Der Schalter \ifkorrekturansicht ist bereits durch den Vorspann gesetzt.

%% latex-abspann.tex
%% Gemeinsamer Abspann für Korrekturansicht und Leseansicht.
%% Setzt den Schalter \ifkorrekturansicht voraus (gesetzt in den
%% einbindenden Dateien latex-korrekturansicht-abspann.tex bzw.
%% latex-leseansicht-abspann.tex).
%% ---------------------------------------------------------------

\normalsize

% Das esempio-Environment wird nur in der Leseansicht benötigt
\ifkorrekturansicht\else
\newenvironment{esempio}[3]%
{
    \vspace{1.5ex}
    \rlap{\underline{#1}}
    \par
    \setlength{\parindent}{0cm}
    \nopagebreak
    \leftskip=#2cm
    \rightskip=#3cm
}
{
    \par
}
\fi

\doendnotes{C}
\bigskip
\vfill

\clearpage

\footnotesize

\ifkorrekturansicht
  \lohead{\textsc{register}}
\fi

% theindex-Environment neu definieren ohne reledmac
\makeatletter
\renewenvironment{theindex}{%
  \ifkorrekturansicht
    \section*{\indexname}%
  \else
    \subsubsection*{Index der erwähnten Entitäten}%
  \fi
  \setlength{\parindent}{0pt}%
  \setlength{\parskip}{0pt plus 0.3pt}%
  \let\item\@idxitem
}{%
  \ifkorrekturansicht\clearpage\fi
}
\makeatother

\IfFileExists{\jobname-pw.ind}{\input{\jobname-pw.ind}}{}

% Quellenangabe nur in der Leseansicht
\ifkorrekturansicht\else
% Fallback-Definitionen, falls die .tex-Datei \titel etc. nicht gesetzt hat
\providecommand{\titel}{}
\providecommand{\editorInnen}{}
\providecommand{\dateiname}{\jobname}

\vspace{3cm}

\vfill

\footnotesize
\textsc{Quelle}: \titel. Herausgegeben von {\editorInnen}. In: \emph{Arthur Schnitzler: Briefwechsel mit Autorinnen und Autoren}.
 Digitale Edition, https://schnitzler-briefe.acdh.oeaw.ac.at/{\dateiname}.html (Stand \today)
\fi

\end{document}


