%% latex-leseansicht-vorspann.tex
%% Vorspann für die Leseansicht.
%% Lädt die gemeinsame Datei latex-vorspann.tex mit nicht gesetztem Schalter.

\newif\ifkorrekturansicht
\korrekturansichtfalse

\input{../tex-inputs/latex-vorspann}


         
         \renewcommand{\erwaehntePersonen}{Personen:  ?? [Totgeborener Sohn von Arthur Schnitzler und Marie Reinhard], Richard Beer-Hofmann, Mirjam Beer-Hofmann, Paula Beer-Hofmann, Rosa Freudenthal, Marie Glümer, Paul Goldmann, Marie Reinhard, Josef Rosengart}
         \renewcommand{\erwaehnteInstitutionen}{Institutionen: Frankfurter Zeitung}
         \renewcommand{\erwaehnteOrte}{Orte: Frankfurt am Main, Mauer, Wien}
         \renewcommand{\erwaehnteWerke}{Werke: Tagebuch}
               \section[ Paul Goldmann an Arthur Schnitzler, {[}17./18.?{]} 9. 1897]{ Paul Goldmann an Arthur Schnitzler, {[}17./18.?{]} 9. 1897}\nopagebreak\mylabel{v}\rehead{ }\begin{ledgroupsized}[t]{13cm}\normalsize\beginnumbering \toendnotes[C]{\smallbreak\pagebreak[2]} \Standort{DLA, A:Schnitzler, HS.NZ85.1.3167.}
\physDesc{Brief, 1 Blatt, 3 Seiten, 1164 Zeichen
\newline{}Handschrift: blaue Tinte, deutsche Kurrent}\toendnotes[C]{\smallbreak}\pstart
           \noindent{}{\pb}\textcolor{gray}{\textbf{Frankfurter Zeitung}}\orgindex{Frankfurter Zeitung@Frankfurter Zeitung|pw}\hfill \textcolor{gray}{\textbf{Frankfurt a. M.\oindex{Frankfurt am Main@\textbf{Frankfurt am Main}|pw},}}{ }\label{K_L02825-1v}\edtext{17. Sept. \textcolor{gray}{\textbf{189}}7}{\lemma{\textnormal{\emph{17. Sept. 1897}}}\Cendnote{\textnormal{Dieser und der vorangegangene
                        Brief (Paul Goldmann an Arthur Schnitzler, [16./17.?] 9. [1897]) sind auf den
                        gleichen Tag datiert, in diesem Brief wird aber auf den vorangegangenen als
                           »geſtrigen Brief« verwiesen, wodurch entweder der
                        vorliegende auf den 18. 9. 1897 oder
                        andernfalls der frühere auf den 16. 9. 1897
                        zu datieren wäre.}}}\label{K_L02825-1h}.\pend
           \pstart
           \textcolor{gray}{\textbf{und}}\pend
           \pstart
           \textcolor{gray}{\textbf{Handelsblatt.}}\pend
           \pstart
           \textcolor{gray}{\textbf{Redaktion\orgindex{Frankfurter Zeitung@Frankfurter Zeitung|pwv}.\footnote{\noindent{}\textcolor{gray}{\textbf{Für die Redaktion\orgindex{Frankfurter Zeitung@Frankfurter Zeitung|pwv} beſtimmte Briefe und Sendungen wolle man
                                 \so{nicht} an die Perſon eines Redakteurs,
                              ſondern ſtets \textbf{an die Redaktion der Frankfurter Zeitung\orgindex{Frankfurter Zeitung@Frankfurter Zeitung|pw}} adreſſiren.}}}}}\pend
           \pstart
           \textcolor{gray}{\textbf{Telegramm-Adreſſe:}}\pend
           \pstart
           \textcolor{gray}{\textbf{Zeitung\orgindex{Frankfurter Zeitung@Frankfurter Zeitung|pwv}{ }Frankfurt Main\oindex{Frankfurt am Main@\textbf{Frankfurt am Main}|pw}. }}\pend
           \pstart\center{}Mein lieber Freund,\pend\pstart
           Ich will Dir nur noch raſch für Deinen lieben Brief danken, den ich heut bekam.\pend
           \pstart
           Sieh’ nicht ſo trübſelig in die Zukunft und laß’ die Wolken machen, was ſie wollen.
               Dein Lebensweg liegt klar und ſchön vor meinen Blicken, und ich ſehe beſſer, weil
               Deine augenblicklichen \label{K_L02825-2v}\edtext{Verſtimmungen}{\lemma{\textnormal{\emph{Verſtimmungen}}}\Cendnote{\textnormal{wohl aufgrund von Schnitzler\pwindex{Schnitzler, Arthur 15.05.1862 – 21.10.1931@\textsc{Schnitzler, Arthur} (15.05.1862 – 21.10.1931), \emph{Schriftsteller, Mediziner}|pwk}s Affäre mit der verheirateten Rosa Freudenthal\pwindex{Freudenthal, Rosa 1862 – 18.06.1905@\textsc{Freudenthal, Rosa} (1862 – 18.06.1905)|pwk} und der noch immer relativ
                  geheim gehaltenen Schwangerschaft Marie
                     Reinhard\pwindex{Reinhard, Marie 1871-03-13 – 1899-03-18@\textsc{Reinhard, Marie} (1871-03-13 – 1899-03-18), \emph{Gesangspädagogin}|pwk}s}}}\label{K_L02825-2h} mir nicht die Ausſicht verdunkeln. Du wirſt wieder Ruhe
               bekommen, wirſt wieder arbeiten und dann wirſt Du ſelbſt wieder {\pb}beſſer und heiterer geſtimmt ſein. Ich meine, das
               Nöthigſte wäre für Dich, daß Du ſo raſch als möglich die Arbeit wieder aufzunehmen
               ſuchteſt.\pend
           \pstart
           Mein Schwager\pwindex{Rosengart, Josef 1860-02-08 – 1927-08-04@\textsc{Rosengart, Josef} (1860-02-08 – 1927-08-04), \emph{Arzt}|pwv} hat ſich über d\substVorne{}\textsuperscript{ie}\substDazwischen{}en\substHinten{} »\label{K_L02825-3v}\edtext{Bauernfänger}{\lemma{\textnormal{\emph{Bauernfänger}}}\Cendnote{\textnormal{Bezug unklar}}}\label{K_L02825-3h}« ſehr amüſirt, bleibt
               aber \label{K_L02825-4v}\edtext{betreffs des Ohrenklingens}{\lemma{\textnormal{\emph{betreffs des Ohrenklingens}}}\Cendnote{\textnormal{siehe Paul Goldmann an Arthur Schnitzler, 13. 9. 1897}}}\label{K_L02825-4h} unerſchütterlich bei ſeiner Anſicht.\pend
           \pstart
           Wenn ich Deine Andeutungen bezüglich \label{K_L02825-5v}\edtext{Fräulein \textsc{G.\pwindex{Gluemer, Marie 03.07.1867 – 16.11.1925@\textsc{Glümer, Marie} (03.07.1867 – 16.11.1925), \emph{Schauspielerin}|pwv}}}{\lemma{\textnormal{\emph{Fräulein G.}}}\Cendnote{\textnormal{Nur in Andeutungen im \emph{Tagebuch}\pwindex{Schnitzler, Arthur 15.05.1862 – 21.10.1931@\textsc{Schnitzler, Arthur} (15.05.1862 – 21.10.1931), \emph{Schriftsteller, Mediziner}!Tagebuch1981 – 2000@\strich\emph{Tagebuch} {[}1981 – 2000{]}|pwk} werden Momente einer komischen Geschichte klar:
                  Einerseits erhielt Schnitzler\pwindex{Schnitzler, Arthur 15.05.1862 – 21.10.1931@\textsc{Schnitzler, Arthur} (15.05.1862 – 21.10.1931), \emph{Schriftsteller, Mediziner}|pwk} am 30. 8. 1897 eine Karte
                  von ihr, die an einen anderen Liebhaber gerichtet gewesen sein dürfte. Am 3. 9. 1897 debütierte
                  sie in Wien\oindex{Wien@\textbf{Wien}|pwk} und ihm war es ein Anliegen, dass
                  sie von seiner erfolgten Rückkehr nichts wusste.}}}\label{K_L02825-5h} richtig verſtanden habe,
               ſo iſt das eine vollendet komiſche Geſchichte.\pend
           \pstart
           Die \label{K_L02825-6v}\edtext{nächſte Woche}{\lemma{\textnormal{\emph{nächſte Woche}}}\Cendnote{\textnormal{Eine Woche später, am 24. 9. 1897, kam es
                  zur Totgeburt des Sohn\pwindex{?? [Totgeborener Sohn von Arthur Schnitzler und Marie Reinhard] 1897-09-24 – 1897-09-24@\textsc{?? [Totgeborener Sohn von Arthur Schnitzler und Marie Reinhard]} (1897-09-24 – 1897-09-24)|pwkv}s
                  von Schnitzler\pwindex{Schnitzler, Arthur 15.05.1862 – 21.10.1931@\textsc{Schnitzler, Arthur} (15.05.1862 – 21.10.1931), \emph{Schriftsteller, Mediziner}|pwk} und Marie Reinhard\pwindex{Reinhard, Marie 1871-03-13 – 1899-03-18@\textsc{Reinhard, Marie} (1871-03-13 – 1899-03-18), \emph{Gesangspädagogin}|pwk} in Mauer bei
                     Wien\oindex{Mauer@\textbf{Mauer}|pwk}.}}}\label{K_L02825-6h} wird alſo, wie ich aus Deinem Briefe entnehme, wichtig und
               ereignißreich werden. Ich wünſche Dir und Deiner {\pb}Freundin\pwindex{Reinhard, Marie 1871-03-13 – 1899-03-18@\textsc{Reinhard, Marie} (1871-03-13 – 1899-03-18), \emph{Gesangspädagogin}|pwv} von Herzen allen
               guten Muth in den bevorſtehenden ſchweren Stunden.\pend
           \pstart
           Auf meinen geſtrigen Brief antworteſt Du wohl baldmöglichſt.\pend
           \pstart
           Die Meinigen grüßen Dich.\pend
           \pstart
           In Treue {\\[\baselineskip]}Dein {\\[\baselineskip]}\spacefill\mbox{Paul Goldm}\pend
           \leftskip=0em{}\pstart
           \noindent{}Was machen \textsc{Richard\pwindex{Beer-Hofmann, Richard 1866-07-11 – 1945-09-26@\textsc{Beer-Hofmann, Richard} (1866-07-11 – 1945-09-26), \emph{Schriftsteller}|pw}} und \label{K_L02825-7v}\edtext{\textsc{Richard\pwindex{Beer-Hofmann, Richard 1866-07-11 – 1945-09-26@\textsc{Beer-Hofmann, Richard} (1866-07-11 – 1945-09-26), \emph{Schriftsteller}|pw}s}{ }Tochter\pwindex{Beer-Hofmann, Mirjam 04.09.1897 – 24.12.1984@\textsc{Beer-Hofmann, Mirjam} (04.09.1897 – 24.12.1984)|pwv}}{\lemma{\textnormal{\emph{Richards Tochter}}}\Cendnote{\textnormal{Mirjam\pwindex{Beer-Hofmann, Mirjam 04.09.1897 – 24.12.1984@\textsc{Beer-Hofmann, Mirjam} (04.09.1897 – 24.12.1984)|pwk}, die Tochter von Richard\pwindex{Beer-Hofmann, Richard 1866-07-11 – 1945-09-26@\textsc{Beer-Hofmann, Richard} (1866-07-11 – 1945-09-26), \emph{Schriftsteller}|pwk} und Paula
                        Beer-Hofmann\pwindex{Beer-Hofmann, Paula 25.02.1879 – 30.10.1939@\textsc{Beer-Hofmann, Paula} (25.02.1879 – 30.10.1939)|pwk}, kam am 4. 9. 1897 zur Welt.}}}\label{K_L02825-7h}?\pend
           
         
         \endnumbering\mylabel{h}\end{ledgroupsized}  \newcommand{\dateiname}{L02825}\newcommand{\titel}{Paul Goldmann an Arthur Schnitzler, [17./18.?] 9. 1897}\newcommand{\editorInnen}{Martin Anton Müller und Laura Untner}%% latex-leseansicht-abspann.tex
%% Abspann für die Leseansicht.
%% Der Schalter \ifkorrekturansicht ist bereits durch den Vorspann gesetzt.

%% latex-abspann.tex
%% Gemeinsamer Abspann für Korrekturansicht und Leseansicht.
%% Setzt den Schalter \ifkorrekturansicht voraus (gesetzt in den
%% einbindenden Dateien latex-korrekturansicht-abspann.tex bzw.
%% latex-leseansicht-abspann.tex).
%% ---------------------------------------------------------------

\normalsize

% Das esempio-Environment wird nur in der Leseansicht benötigt
\ifkorrekturansicht\else
\newenvironment{esempio}[3]%
{
    \vspace{1.5ex}
    \rlap{\underline{#1}}
    \par
    \setlength{\parindent}{0cm}
    \nopagebreak
    \leftskip=#2cm
    \rightskip=#3cm
}
{
    \par
}
\fi

\doendnotes{C}
\bigskip
\vfill

\clearpage

\footnotesize

\ifkorrekturansicht
  \lohead{\textsc{register}}
\fi

% theindex-Environment neu definieren ohne reledmac
\makeatletter
\renewenvironment{theindex}{%
  \ifkorrekturansicht
    \section*{\indexname}%
  \else
    \subsubsection*{Index der erwähnten Entitäten}%
  \fi
  \setlength{\parindent}{0pt}%
  \setlength{\parskip}{0pt plus 0.3pt}%
  \let\item\@idxitem
}{%
  \ifkorrekturansicht\clearpage\fi
}
\makeatother

\IfFileExists{\jobname-pw.ind}{\input{\jobname-pw.ind}}{}

% Quellenangabe nur in der Leseansicht
\ifkorrekturansicht\else
% Fallback-Definitionen, falls die .tex-Datei \titel etc. nicht gesetzt hat
\providecommand{\titel}{}
\providecommand{\editorInnen}{}
\providecommand{\dateiname}{\jobname}

\vspace{3cm}

\vfill

\footnotesize
\textsc{Quelle}: \titel. Herausgegeben von {\editorInnen}. In: \emph{Arthur Schnitzler: Briefwechsel mit Autorinnen und Autoren}.
 Digitale Edition, https://schnitzler-briefe.acdh.oeaw.ac.at/{\dateiname}.html (Stand \today)
\fi

\end{document}


      