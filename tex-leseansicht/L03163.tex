%% latex-leseansicht-vorspann.tex
%% Vorspann für die Leseansicht.
%% Lädt die gemeinsame Datei latex-vorspann.tex mit nicht gesetztem Schalter.

\newif\ifkorrekturansicht
\korrekturansichtfalse

\input{../tex-inputs/latex-vorspann}

\begin{center}
            \textcolor{red}{ENTWURF, NICHT FERTIG KORRIGIERT}
                      \end{center}
            
         \renewcommand{\erwaehnteOrte}{Orte: Bad Ischl, Hotel und Pension Rudolfshöhe (Leopold Petter), I., Innere Stadt, Wien}
         \renewcommand{\erwaehnteWerke}{}
               \section[Felix Salten an Arthur Schnitzler, {[}15. 8. 1895{]}]{ Felix Salten an Arthur Schnitzler, {[}15. 8. 1895{]}}\nopagebreak\mylabel{v}\rehead{ }\begin{ledgroupsized}[t]{13cm}\normalsize\beginnumbering \toendnotes[C]{\smallbreak\pagebreak[2]} \Standort{CUL, Schnitzler, B 89, A 1.}
\physDesc{Postkarte
\newline{}Handschrift: Bleistift, lateinische Kurrent\newline{}Versand: 1) Stempel: »\nobreak{}\oindex{I., Innere Stadt@\textbf{I., Innere Stadt}|pwk}Wien 1/1 1, 15 8 9\textcolor{gray}{5}, 8–9V\nobreak{}«.   2) Stempel: »\nobreak{}\oindex{Bad Ischl@\textbf{Bad Ischl}|pwk}Ischl, 15 8 9\textcolor{gray}{5}, 11A\nobreak{}«. 
\newline{}Schnitzler: mit Bleistift datiert:
                                 »15/8 95« }\toendnotes[C]{\smallbreak}\pstart{}{\pb}Herrn D\textsuperscript{r} Arthur Schnitzler \pend{}\pstart{}Ischl\oindex{Bad Ischl@\textbf{Bad Ischl}|pw}\pend{}\pstart{}Pension Leopold\oindex{Hotel und Pension Rudolfshoehe (Leopold Petter)@\textbf{Hotel und Pension Rudolfshöhe (Leopold Petter)}|pw}\pend{}{\bigskip}\pstart
           \noindent{}{\pb}Lieber Frd. Ich fahre Freitag{ }Nachmittag, bin aber
               \label{K_L03163-1v}\edtext{abends in Ischl\oindex{Bad Ischl@\textbf{Bad Ischl}|pw}}{\lemma{\textnormal{\emph{abends in Ischl}}}\Cendnote{\textnormal{siehe A. S.: \emph{Tagebuch}, 16. 8. 1895}}}\label{K_L03163-1h}, wenn Sie so gut sein wollen,
               nehmen Sie irgendwo ein billiges Zimmer. Ko{\geminationm}en Sie zur
               Bahn? Wenn ja, bitte mit Rad, damit ich nicht schieben muss. \pend
           \pstart
           Auf Wiedersehen {\\[\baselineskip]}Ihr \spacefill\mbox{Salten}\pend
           \leftskip=0em{}
         
         \endnumbering\mylabel{h}\end{ledgroupsized}\begin{anhang}\end{anhang}\newcommand{\dateiname}{L03163}\newcommand{\titel}{Felix Salten an Arthur Schnitzler, [15. 8. 1895]}\newcommand{\editorInnen}{Martin Anton Müller und Laura Untner}%% latex-leseansicht-abspann.tex
%% Abspann für die Leseansicht.
%% Der Schalter \ifkorrekturansicht ist bereits durch den Vorspann gesetzt.

%% latex-abspann.tex
%% Gemeinsamer Abspann für Korrekturansicht und Leseansicht.
%% Setzt den Schalter \ifkorrekturansicht voraus (gesetzt in den
%% einbindenden Dateien latex-korrekturansicht-abspann.tex bzw.
%% latex-leseansicht-abspann.tex).
%% ---------------------------------------------------------------

\normalsize

% Das esempio-Environment wird nur in der Leseansicht benötigt
\ifkorrekturansicht\else
\newenvironment{esempio}[3]%
{
    \vspace{1.5ex}
    \rlap{\underline{#1}}
    \par
    \setlength{\parindent}{0cm}
    \nopagebreak
    \leftskip=#2cm
    \rightskip=#3cm
}
{
    \par
}
\fi

\doendnotes{C}
\bigskip
\vfill

\clearpage

\footnotesize

\ifkorrekturansicht
  \lohead{\textsc{register}}
\fi

% theindex-Environment neu definieren ohne reledmac
\makeatletter
\renewenvironment{theindex}{%
  \ifkorrekturansicht
    \section*{\indexname}%
  \else
    \subsubsection*{Index der erwähnten Entitäten}%
  \fi
  \setlength{\parindent}{0pt}%
  \setlength{\parskip}{0pt plus 0.3pt}%
  \let\item\@idxitem
}{%
  \ifkorrekturansicht\clearpage\fi
}
\makeatother

\IfFileExists{\jobname-pw.ind}{\input{\jobname-pw.ind}}{}

% Quellenangabe nur in der Leseansicht
\ifkorrekturansicht\else
% Fallback-Definitionen, falls die .tex-Datei \titel etc. nicht gesetzt hat
\providecommand{\titel}{}
\providecommand{\editorInnen}{}
\providecommand{\dateiname}{\jobname}

\vspace{3cm}

\vfill

\footnotesize
\textsc{Quelle}: \titel. Herausgegeben von {\editorInnen}. In: \emph{Arthur Schnitzler: Briefwechsel mit Autorinnen und Autoren}.
 Digitale Edition, https://schnitzler-briefe.acdh.oeaw.ac.at/{\dateiname}.html (Stand \today)
\fi

\end{document}


      