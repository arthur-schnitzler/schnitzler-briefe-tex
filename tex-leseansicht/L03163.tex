%% latex-korrekturansicht-vorspann.tex
%% Vorspann für die Korrekturansicht.
%% Lädt die gemeinsame Datei latex-vorspann.tex mit gesetztem Schalter.

\newif\ifkorrekturansicht
\korrekturansichttrue

\input{../tex-inputs/latex-vorspann}


\section[ Felix Salten an Arthur Schnitzler, 15. 8. 1895]{L03163 Felix Salten an Arthur Schnitzler, 15. 8. 1895}
\nopagebreak\mylabel{L03163v}
\rehead{ }\normalsize\beginnumbering\briefempfaengerindex{Schnitzler, Arthur@\textsc{Schnitzler, Arthur}!zzzSalten, Felix@\emph{von Felix Salten}!1895-08-151@{15. 8. 1895}|(be}
\toendnotes[C]{\smallbreak\pagebreak[2]}\Standort{CUL, Schnitzler, B 89, A 1.}
\physDesc{Postkarte, 277 Zeichen
\newline{}Handschrift: Bleistift, lateinische Kurrent
\newline{}Versand: 1) Stempel: »\nobreak{}\oindex{I., Innere Stadt@\textbf{I., Innere Stadt}, \emph{A.ADM3}|pwk}Wien 1/1 1, 15 8 95, 8–9V\nobreak{}«.   2) Stempel: »\nobreak{}\oindex{Bad Ischl@\textbf{Bad Ischl}, \emph{P.PPL}|pwk}Ischl, 15 8 9\textcolor{gray}{5}, 11–A\nobreak{}«. 
\newline{}Schnitzler: mit Bleistift datiert: »15/8 95« 
\newline{}Ordnung: mit Bleistift von unbekannter Hand nummeriert: »63« }\toendnotes[C]{\smallbreak}\pstart{}{\pb}Herrn D\textsuperscript{r} Arthur Schnitzler \pend{}\pstart{}Ischl\oindex{Bad Ischl@\textbf{Bad Ischl}, \emph{P.PPL}|pw}\pend{}\pstart{}Pension Leopold\oindex{Hotel und Pension Rudolfshoehe (Leopold Petter)@\textbf{Hotel und Pension Rudolfshöhe (Leopold Petter)}, \emph{Hotel (K.HTL)}|pw}\pend{}{\bigskip}\vspace{1em}
\pstart
           \noindent{}{\pb}lieber Frd. Ich fahre Freitag{ }Nachmittag, bin also \label{K_L03163-1v}\edtext{Abends in Ischl\oindex{Bad Ischl@\textbf{Bad Ischl}, \emph{P.PPL}|pw}}{\lemma{\textnormal{\emph{Abends in Ischl}}}\Cendnote{\textnormal{Siehe A. S.: \emph{Tagebuch}, 16. 8. 1895.
               }}}\label{K_L03163-1}. Wenn Sie so gut sein wollen, nehmen Sie irgendwo ein \label{K_L03163-2v}\edtext{billiges Zimmer}{\lemma{\textnormal{\emph{billiges Zimmer}}}\Cendnote{\textnormal{Er wohnte im
                  Hôtel zum Erzherzog Franz Carl\oindex{Hôtel zum Erzherzog Franz Carl@\textbf{Hôtel zum Erzherzog Franz Carl}, \emph{Hotel (K.HTL)}|pwk}.}}}\label{K_L03163-2}. Ko{\geminationm}en Sie zur Bahn? Wenn ja, bitte mit Rad, damit ich
               nicht schieben muss. Auf Wiedersehen {\\}Ihr \spacefill\mbox{Salten.}\pend
           \selectlanguage{ngerman}\endnumbering\briefempfaengerindex{Schnitzler, Arthur@\textsc{Schnitzler, Arthur}!zzzSalten, Felix@\emph{von Felix Salten}!1895-08-151@{15. 8. 1895}|)be}\mylabel{L03163h}  \normalsize

\doendnotes{C}
\bigskip
\vfill

\clearpage

\footnotesize

\lohead{\textsc{register}}

% Definiere theindex-Environment komplett neu ohne reledmac
\makeatletter
\renewenvironment{theindex}{%
  \section*{\indexname}%
  \setlength{\parindent}{0pt}%
  \setlength{\parskip}{0pt plus 0.3pt}%
  \let\item\@idxitem
}{%
  \clearpage
}
\makeatother

\IfFileExists{\jobname-pw.ind}{\input{\jobname-pw.ind}}{}

\end{document}

      