%% latex-leseansicht-vorspann.tex
%% Vorspann für die Leseansicht.
%% Lädt die gemeinsame Datei latex-vorspann.tex mit nicht gesetztem Schalter.

\newif\ifkorrekturansicht
\korrekturansichtfalse

\input{../tex-inputs/latex-vorspann}


\section[Paul Goldmann an Arthur Schnitzler, 12. 12. [1891]]{L02674 Paul Goldmann an Arthur Schnitzler, 12. 12. [1891]}
\nopagebreak\mylabel{L02674v}
\rehead{ }\normalsize\beginnumbering\briefempfaengerindex{Schnitzler, Arthur@\textsc{Schnitzler, Arthur}!zzzGoldmann, Paul@\emph{von Paul Goldmann}!1891-12-122@{12. 12. [1891]}|(be}
\toendnotes[C]{\smallbreak\pagebreak[2]}
\correspDesc{Versand  durch Paul Goldmann am 12. 12. [1891] in Paris
\newline{}Erhalt  durch Arthur Schnitzler im Zeitraum [13. 12. 1891 – 17. 12. 1891?] in Wien}\toendnotes[C]{\smallbreak}
\Standort{DLA, A:Schnitzler, HS.NZ85.1.3162.}
\physDesc{Brief, 2 Blätter, 8 Seiten, 6330 Zeichen
\newline{}Handschrift: schwarze Tinte, deutsche Kurrent
\newline{}Schnitzler: 1) mit rotem Buntstift Vermerk »\textsc{(über \uline{Märchen\pwindex{Schnitzler, Arthur 15.\,5.\,1862 Wien – 21.\,10.\,1931 ebd.@\textsc{Schnitzler, Arthur} (15.\,5.\,1862 Wien – 21.\,10.\,1931 ebd.), \emph{Schriftsteller, Mediziner}!Märchen. Schauspiel in drei Aufzügen@\strich\emph{Das Märchen. Schauspiel in drei Aufzügen}|pw}}}«  2) mit Bleistift die Jahreszahl »91« ergänzt}\toendnotes[C]{\smallbreak}
\pstart
           \raggedleft{}{\pb}\textsc{Paris\oindex{Paris@\textbf{Paris}, \emph{Hauptstadt}|pw}}, 12. December.\pend
           
\pstart\center{}Mein lieber Arthur!\pend\vspace{0.5em}
\pstart
           Bei der ungeheuren Überbürdung, die gleich noch ehe ich den eigentlichen Dienſt
               übernommen, auf mich gefallen iſt, muß ich kurz{ }ſein und kann keine Form für meine
               Anſicht{ }ſuchen. Alſo folgendes: Der erſte Act\pwindex{Schnitzler, Arthur 15.\,5.\,1862 Wien – 21.\,10.\,1931 ebd.@\textsc{Schnitzler, Arthur} (15.\,5.\,1862 Wien – 21.\,10.\,1931 ebd.), \emph{Schriftsteller, Mediziner}!Märchen. Schauspiel in drei Aufzügen@\strich\emph{Das Märchen. Schauspiel in drei Aufzügen}|pwv} iſt{ }ſchlankweg entzückend, gehört zu den beſten erſten
               Acten, die ich kenne,{ }ſprüht von Geiſt und Leben, enthält prachtvolle dramatiſche
               Steigerungen und einen \strikeout{E} erbeben machenden Schluß,
               iſt meiſterhaft in der Bewältigung der Perſonenmehrheiten, vergnüglich in der
               Entwerfung der Phyſiognomien, edel und neu in den Gedanken. Ich{ }ſtelle ihn ruhig
               einem \textsc{Augier\pwindex{Augier, Émile 17.\,9.\,1820 Valence – 25.\,10.\,1889 Croissy-sur-Seine@\textsc{Augier, Émile} (17.\,9.\,1820 Valence – 25.\,10.\,1889 Croissy-sur-Seine), \emph{Schriftsteller}|pw}} zur Seite. Äußerlich habe ich einzuwenden, daß während der Hauptdialoge auf der
               Bühne Clavier geſpielt wird, was ich für einen Mangel an{ }ſceniſcher Geſchicklichkeit
               halte. Zweiter Act\pwindex{Schnitzler, Arthur 15.\,5.\,1862 Wien – 21.\,10.\,1931 ebd.@\textsc{Schnitzler, Arthur} (15.\,5.\,1862 Wien – 21.\,10.\,1931 ebd.), \emph{Schriftsteller, Mediziner}!Märchen. Schauspiel in drei Aufzügen@\strich\emph{Das Märchen. Schauspiel in drei Aufzügen}|pwv}: Beginn
               gut; erſtes Geſpräch zwiſchen Fedor\pwindex{Schnitzler, Arthur 15.\,5.\,1862 Wien – 21.\,10.\,1931 ebd.@\textsc{Schnitzler, Arthur} (15.\,5.\,1862 Wien – 21.\,10.\,1931 ebd.), \emph{Schriftsteller, Mediziner}!Märchen. Schauspiel in drei Aufzügen@\strich\emph{Das Märchen. Schauspiel in drei Aufzügen}|pwv} und Leo\pwindex{Schnitzler, Arthur 15.\,5.\,1862 Wien – 21.\,10.\,1931 ebd.@\textsc{Schnitzler, Arthur} (15.\,5.\,1862 Wien – 21.\,10.\,1931 ebd.), \emph{Schriftsteller, Mediziner}!Märchen. Schauspiel in drei Aufzügen@\strich\emph{Das Märchen. Schauspiel in drei Aufzügen}|pwv} gut,
               desgleichen erſtes Geſpräch zwiſchen Fedor\pwindex{Schnitzler, Arthur 15.\,5.\,1862 Wien – 21.\,10.\,1931 ebd.@\textsc{Schnitzler, Arthur} (15.\,5.\,1862 Wien – 21.\,10.\,1931 ebd.), \emph{Schriftsteller, Mediziner}!Märchen. Schauspiel in drei Aufzügen@\strich\emph{Das Märchen. Schauspiel in drei Aufzügen}|pwv} und Fanny\pwindex{Schnitzler, Arthur 15.\,5.\,1862 Wien – 21.\,10.\,1931 ebd.@\textsc{Schnitzler, Arthur} (15.\,5.\,1862 Wien – 21.\,10.\,1931 ebd.), \emph{Schriftsteller, Mediziner}!Märchen. Schauspiel in drei Aufzügen@\strich\emph{Das Märchen. Schauspiel in drei Aufzügen}|pwv}, {\pb}Auftreten \textsc{Fr.
                     Wittes\pwindex{Schnitzler, Arthur 15.\,5.\,1862 Wien – 21.\,10.\,1931 ebd.@\textsc{Schnitzler, Arthur} (15.\,5.\,1862 Wien – 21.\,10.\,1931 ebd.), \emph{Schriftsteller, Mediziner}!Märchen. Schauspiel in drei Aufzügen@\strich\emph{Das Märchen. Schauspiel in drei Aufzügen}|pwv}} guter dramatiſcher \label{K_L02674-1v}\edtext{\textsc{Truc}}{\lemma{\textnormal{\emph{Truc}}}\Cendnote{\textnormal{französisch: Kniff, Trick}}}\label{K_L02674-1}. \textsc{Fr. Witte\pwindex{Schnitzler, Arthur 15.\,5.\,1862 Wien – 21.\,10.\,1931 ebd.@\textsc{Schnitzler, Arthur} (15.\,5.\,1862 Wien – 21.\,10.\,1931 ebd.), \emph{Schriftsteller, Mediziner}!Märchen. Schauspiel in drei Aufzügen@\strich\emph{Das Märchen. Schauspiel in drei Aufzügen}|pwv}}{ }ſelbſt{[},{]} verſtändlich für Dich,
               mich und die gewiſſen drei oder vier Andern; für das große Publicum zu{ }ſehr im
               Viertelprofil; der Durchschnittszuſchauer weiß nicht, was er daraus \strikeout{\textcolor{gray}{×}\-\textcolor{gray}{×}} machen{ }ſoll. Aber bei den{ }ſchönen geiſtreichen Sachen, die der Dialog enthält,
               geht die Scene vielleicht durch; nur kommen mir die Pointen zu gehäuft vor. \textsc{Zola\pwindex{Zola, Émile 2.\,4.\,1840 Paris – 29.\,9.\,1902 ebd.@\textsc{Zola, Émile} (2.\,4.\,1840 Paris – 29.\,9.\,1902 ebd.), \emph{Schriftsteller, Journalist}|pw}}{ }ſprach mir in Brüſſel\oindex{Brüssel@\textbf{Brüssel}, \emph{Hauptstadt}|pw} von dieſen mit
               Pointen vollgeſtopften Scenen, deren dramatiſche Wirkung er bezweifelt: »\label{K_L02674-2v}\edtext{\textsc{\begin{otherlanguage}{french}On doit avoir le temps de se moucher\end{otherlanguage}}}{\lemma{\textnormal{\emph{On … moucher}}}\Cendnote{\textnormal{französisch: man muss Zeit haben, um
                  sich die Nase zu putzen}}}\label{K_L02674-2}«,{ }ſagt\textcolor{gray}{e} er. Letzte Scene
               zwiſchen Fedor\pwindex{Schnitzler, Arthur 15.\,5.\,1862 Wien – 21.\,10.\,1931 ebd.@\textsc{Schnitzler, Arthur} (15.\,5.\,1862 Wien – 21.\,10.\,1931 ebd.), \emph{Schriftsteller, Mediziner}!Märchen. Schauspiel in drei Aufzügen@\strich\emph{Das Märchen. Schauspiel in drei Aufzügen}|pwv} und Fanny\pwindex{Schnitzler, Arthur 15.\,5.\,1862 Wien – 21.\,10.\,1931 ebd.@\textsc{Schnitzler, Arthur} (15.\,5.\,1862 Wien – 21.\,10.\,1931 ebd.), \emph{Schriftsteller, Mediziner}!Märchen. Schauspiel in drei Aufzügen@\strich\emph{Das Märchen. Schauspiel in drei Aufzügen}|pwv}. Da beginnt das \label{K_L02674-3v}\edtext{\textsc{\begin{otherlanguage}{french}embrouillement\end{otherlanguage}}}{\lemma{\textnormal{\emph{embrouillement}}}\Cendnote{\textnormal{französisch: Verwirrung,
                  Verworrenheit}}}\label{K_L02674-3}. Der Zuſchauer kennt{ }ſich nicht mehr aus. Das Geſicht des Stückes\pwindex{Schnitzler, Arthur 15.\,5.\,1862 Wien – 21.\,10.\,1931 ebd.@\textsc{Schnitzler, Arthur} (15.\,5.\,1862 Wien – 21.\,10.\,1931 ebd.), \emph{Schriftsteller, Mediziner}!Märchen. Schauspiel in drei Aufzügen@\strich\emph{Das Märchen. Schauspiel in drei Aufzügen}|pwv} wechſelt plötzlich;{ }ſtatt der \label{K_L02674-4v}\edtext{Gefallenen}{\lemma{\textnormal{\emph{Gefallenen}}}\Cendnote{\textnormal{Gemeint ist damit die Figur der Fanny\pwindex{Schnitzler, Arthur 15.\,5.\,1862 Wien – 21.\,10.\,1931 ebd.@\textsc{Schnitzler, Arthur} (15.\,5.\,1862 Wien – 21.\,10.\,1931 ebd.), \emph{Schriftsteller, Mediziner}!Märchen. Schauspiel in drei Aufzügen@\strich\emph{Das Märchen. Schauspiel in drei Aufzügen}|pwkv}, die bereits vor ihrer
                  Beziehung zu Fedor\pwindex{Schnitzler, Arthur 15.\,5.\,1862 Wien – 21.\,10.\,1931 ebd.@\textsc{Schnitzler, Arthur} (15.\,5.\,1862 Wien – 21.\,10.\,1931 ebd.), \emph{Schriftsteller, Mediziner}!Märchen. Schauspiel in drei Aufzügen@\strich\emph{Das Märchen. Schauspiel in drei Aufzügen}|pwkv} sexuell
                  aktiv war.}}}\label{K_L02674-4} tritt auf einmal der \label{K_L02674-5v}\edtext{junge Mann, die Analyſe, die Seelenzerfleiſchung}{\lemma{\textnormal{\emph{junge … Seelenzerfleischung}}}\Cendnote{\textnormal{Fedor\pwindex{Schnitzler, Arthur 15.\,5.\,1862 Wien – 21.\,10.\,1931 ebd.@\textsc{Schnitzler, Arthur} (15.\,5.\,1862 Wien – 21.\,10.\,1931 ebd.), \emph{Schriftsteller, Mediziner}!Märchen. Schauspiel in drei Aufzügen@\strich\emph{Das Märchen. Schauspiel in drei Aufzügen}|pwkv} gelingt es nicht, das
                  sexuelle Vorleben von Fanny\pwindex{Schnitzler, Arthur 15.\,5.\,1862 Wien – 21.\,10.\,1931 ebd.@\textsc{Schnitzler, Arthur} (15.\,5.\,1862 Wien – 21.\,10.\,1931 ebd.), \emph{Schriftsteller, Mediziner}!Märchen. Schauspiel in drei Aufzügen@\strich\emph{Das Märchen. Schauspiel in drei Aufzügen}|pwkv}
                  zu akzeptieren, obwohl er mit dem Verstand die Idealisierung der
                  Jungfräulichkeit als »Märchen« abtut.}}}\label{K_L02674-5} in den {\pb}Vordergrund. Es kommen Motive in’s Spiel, mit einem
               Ruck, unvermittelt, welche zu fein und zu atomiſch zerfaſert{ }ſind, als daß das
               Publicum mit{ }ſeinen groben Werktagshänden ihnen nachtaſten könnte. Das iſt
               pſychologiſch, aber nicht mehr dramatiſch. Und wenn die Scene doch einen Erfolg hat,{ }ſo kann es nur dadurch geſchehen, daß Meiſter Publicus{ }ſich das auf{ }ſeine Weiſe
               zurechtlegt und, von all’ \strikeout{de\textcolor{gray}{m}} den pſychologiſchen \strikeout{h\textcolor{gray}{oc}hf\textcolor{gray}{×}\-\textcolor{gray}{×}\-\textcolor{gray}{×}\-\textcolor{gray}{×}\-\textcolor{gray}{×}} Tendenzen abſtrahierend, nur den rohen Kern herausnimmt, der darin{ }ſteckt: er
               will das Mädel nicht, aber das Mädel läßt nicht nach, und am End’ fallen{ }ſie{ }ſich
               doch in die Arme. Dritter Act\pwindex{Schnitzler, Arthur 15.\,5.\,1862 Wien – 21.\,10.\,1931 ebd.@\textsc{Schnitzler, Arthur} (15.\,5.\,1862 Wien – 21.\,10.\,1931 ebd.), \emph{Schriftsteller, Mediziner}!Märchen. Schauspiel in drei Aufzügen@\strich\emph{Das Märchen. Schauspiel in drei Aufzügen}|pwv}.
               Der hätte{ }ſein{ }ſollen wie der erſte: Perſonenmehrheiten, feſtes Zuſammenhalten der
               Handlung und Steigerung \strikeout{der H} auf einen Punkt hin, wo
               die Entladung mit mächtigem Ruck erfolgt; und dann Vorhang. Der \label{K_L02674-6v}\edtext{Contract}{\lemma{\textnormal{\emph{Contract}}}\Cendnote{\textnormal{Ein Arbeitsvertrag, der Fanny\pwindex{Schnitzler, Arthur 15.\,5.\,1862 Wien – 21.\,10.\,1931 ebd.@\textsc{Schnitzler, Arthur} (15.\,5.\,1862 Wien – 21.\,10.\,1931 ebd.), \emph{Schriftsteller, Mediziner}!Märchen. Schauspiel in drei Aufzügen@\strich\emph{Das Märchen. Schauspiel in drei Aufzügen}|pwkv}, wenn sie ihn unterzeichnet, an ein Theater in St. Petersburg\oindex{Sankt Petersburg@\textbf{Sankt Petersburg}|pwk} engagiert und damit auch einen
                  Ausweg aus der Beziehung zu Fedor\pwindex{Schnitzler, Arthur 15.\,5.\,1862 Wien – 21.\,10.\,1931 ebd.@\textsc{Schnitzler, Arthur} (15.\,5.\,1862 Wien – 21.\,10.\,1931 ebd.), \emph{Schriftsteller, Mediziner}!Märchen. Schauspiel in drei Aufzügen@\strich\emph{Das Märchen. Schauspiel in drei Aufzügen}|pwkv} ermöglicht.}}}\label{K_L02674-6}{ }{\pb}vortreffliche Idee. Aber am Schluß, nachdem man den
               ganzen Act\pwindex{Schnitzler, Arthur 15.\,5.\,1862 Wien – 21.\,10.\,1931 ebd.@\textsc{Schnitzler, Arthur} (15.\,5.\,1862 Wien – 21.\,10.\,1931 ebd.), \emph{Schriftsteller, Mediziner}!Märchen. Schauspiel in drei Aufzügen@\strich\emph{Das Märchen. Schauspiel in drei Aufzügen}|pwv} mit all’{ }ſeinen
               Fäden auf den Contract hat hinlaufen geſehen. Der Aufzug\pwindex{Schnitzler, Arthur 15.\,5.\,1862 Wien – 21.\,10.\,1931 ebd.@\textsc{Schnitzler, Arthur} (15.\,5.\,1862 Wien – 21.\,10.\,1931 ebd.), \emph{Schriftsteller, Mediziner}!Märchen. Schauspiel in drei Aufzügen@\strich\emph{Das Märchen. Schauspiel in drei Aufzügen}|pwv} fällt aber in lauter Dialoge auseinander, und die
               Handlungen{ }ſind{ }ſchichtenweis nebeneinander aufgeſtellt,{ }ſtatt in einem Körper
               zuſammengeſchmolzen zu{ }ſein. Dialog zwiſchen \textsc{Wandel\pwindex{Schnitzler, Arthur 15.\,5.\,1862 Wien – 21.\,10.\,1931 ebd.@\textsc{Schnitzler, Arthur} (15.\,5.\,1862 Wien – 21.\,10.\,1931 ebd.), \emph{Schriftsteller, Mediziner}!Märchen. Schauspiel in drei Aufzügen@\strich\emph{Das Märchen. Schauspiel in drei Aufzügen}|pwv}} und \textsc{Klara\pwindex{Schnitzler, Arthur 15.\,5.\,1862 Wien – 21.\,10.\,1931 ebd.@\textsc{Schnitzler, Arthur} (15.\,5.\,1862 Wien – 21.\,10.\,1931 ebd.), \emph{Schriftsteller, Mediziner}!Märchen. Schauspiel in drei Aufzügen@\strich\emph{Das Märchen. Schauspiel in drei Aufzügen}|pwv}} –{ }ſehr{ }ſchön an{ }ſich, aber bringt aus der Stimmung, iſt zu lang und verläuft,
               ohne in der Haupthandlung{ }ſeine Fortſetzung zu finden. Und{ }ſo weiter. Stell’ Dir das
               auf der Scene vor: einen Act, einen Hauptact eines Dramas, wo Alles Wichtige, was
               vorgeht, in lauter »Beiſeite«{ }ſtattfindet! Stell’ Dir vor, wie ein Act{ }ſich ausnimmt,
               wo \strikeout{i\textcolor{gray}{m}} die Haupt\strikeout{\textcolor{gray}{h}}zahl der Perſonen immer im{ }ſtummen Spiel im Hintergrunde oder auf der Seite{ }ſteht, während vorn immer zwei paarweis {\pb}die
               Handlung machen. Und welche Aufgabe für den Hauptdarſteller,{ }ſeine größten Scenen,{ }ſeine Leidenſchaftsausbrüche »gedämpft« vorzubringen! Welch’ ungünſtiger Abgang!
               Statt nach einer starken Scene mit einem{ }ſtarken Wort hinauszugehen,{ }ſchleicht er{ }ſich von hinnen, nachdem all’{ }ſeine dramatiſchen Feuer verloſchen! Starke und
               gewaltſame Mittel waren nöthig. Kein beiſeite, aus Furcht zu compromittiren,{ }ſondern
               eben dieſes Compromittiren{ }ſelbſt, ein wuchtiger Fauſtſchlag \strikeout{\textcolor{gray}{×}\-\textcolor{gray}{×}\-\textcolor{gray}{×}} in dieſes falſche \textsc{Milieu}, in dieſes Philiſtertum \textsc{à la Wandel\pwindex{Schnitzler, Arthur 15.\,5.\,1862 Wien – 21.\,10.\,1931 ebd.@\textsc{Schnitzler, Arthur} (15.\,5.\,1862 Wien – 21.\,10.\,1931 ebd.), \emph{Schriftsteller, Mediziner}!Märchen. Schauspiel in drei Aufzügen@\strich\emph{Das Märchen. Schauspiel in drei Aufzügen}|pwv}} hinein. Mit Aufſchrei muß die{ }ſchreckliche Wahrheit aus der Bruſt \strikeout{des}{ }Fedors\pwindex{Schnitzler, Arthur 15.\,5.\,1862 Wien – 21.\,10.\,1931 ebd.@\textsc{Schnitzler, Arthur} (15.\,5.\,1862 Wien – 21.\,10.\,1931 ebd.), \emph{Schriftsteller, Mediziner}!Märchen. Schauspiel in drei Aufzügen@\strich\emph{Das Märchen. Schauspiel in drei Aufzügen}|pwv} heraus, mit Aufſchrei
               muß das Mädchen die Vernichtung beantworten, Leidenſchaft gegen Leidenſchaft, zwei
               Flammen, die über dem Haupte des Stück\pwindex{Schnitzler, Arthur 15.\,5.\,1862 Wien – 21.\,10.\,1931 ebd.@\textsc{Schnitzler, Arthur} (15.\,5.\,1862 Wien – 21.\,10.\,1931 ebd.), \emph{Schriftsteller, Mediziner}!Märchen. Schauspiel in drei Aufzügen@\strich\emph{Das Märchen. Schauspiel in drei Aufzügen}|pwv}es zuſammenſchlagen. Schwung und Kunſt im dritten Acte\pwindex{Schnitzler, Arthur 15.\,5.\,1862 Wien – 21.\,10.\,1931 ebd.@\textsc{Schnitzler, Arthur} (15.\,5.\,1862 Wien – 21.\,10.\,1931 ebd.), \emph{Schriftsteller, Mediziner}!Märchen. Schauspiel in drei Aufzügen@\strich\emph{Das Märchen. Schauspiel in drei Aufzügen}|pwv}, aber {\pb}um Gotteswillen nur hier kein Grübeln, Quälen und Vertuſchen.\pend
           
\pstart
           Mit einem Wort: ein fertiges Stück\pwindex{Schnitzler, Arthur 15.\,5.\,1862 Wien – 21.\,10.\,1931 ebd.@\textsc{Schnitzler, Arthur} (15.\,5.\,1862 Wien – 21.\,10.\,1931 ebd.), \emph{Schriftsteller, Mediziner}!Märchen. Schauspiel in drei Aufzügen@\strich\emph{Das Märchen. Schauspiel in drei Aufzügen}|pwv} ist das nicht. Aber ich meine, Du haſt auch kein Recht, zu
               beanſpruchen, daß Dir ein fertiges Stück jetzt{ }ſchon gelingt. Als Weg zum Ziele iſt
               es jedoch ein gewaltiger Schritt, als Talentbeweis ein glänzendes Ergebniß. Wer
               dieſen erſten Act\pwindex{Schnitzler, Arthur 15.\,5.\,1862 Wien – 21.\,10.\,1931 ebd.@\textsc{Schnitzler, Arthur} (15.\,5.\,1862 Wien – 21.\,10.\,1931 ebd.), \emph{Schriftsteller, Mediziner}!Märchen. Schauspiel in drei Aufzügen@\strich\emph{Das Märchen. Schauspiel in drei Aufzügen}|pwv} geſchrieben,
               iſt ein Dramatiker von Gottes Gnaden; und wer \textsc{Robert\pwindex{Schnitzler, Arthur 15.\,5.\,1862 Wien – 21.\,10.\,1931 ebd.@\textsc{Schnitzler, Arthur} (15.\,5.\,1862 Wien – 21.\,10.\,1931 ebd.), \emph{Schriftsteller, Mediziner}!Märchen. Schauspiel in drei Aufzügen@\strich\emph{Das Märchen. Schauspiel in drei Aufzügen}|pwv}} und \textsc{Ninetten\pwindex{Schnitzler, Arthur 15.\,5.\,1862 Wien – 21.\,10.\,1931 ebd.@\textsc{Schnitzler, Arthur} (15.\,5.\,1862 Wien – 21.\,10.\,1931 ebd.), \emph{Schriftsteller, Mediziner}!Märchen. Schauspiel in drei Aufzügen@\strich\emph{Das Märchen. Schauspiel in drei Aufzügen}|pwv}} erdacht, iſt ein Dichter von goldenem Herzen. Als litterariſche Arbeit\pwindex{Schnitzler, Arthur 15.\,5.\,1862 Wien – 21.\,10.\,1931 ebd.@\textsc{Schnitzler, Arthur} (15.\,5.\,1862 Wien – 21.\,10.\,1931 ebd.), \emph{Schriftsteller, Mediziner}!Märchen. Schauspiel in drei Aufzügen@\strich\emph{Das Märchen. Schauspiel in drei Aufzügen}|pwv} iſt »Das Märchen\pwindex{Schnitzler, Arthur 15.\,5.\,1862 Wien – 21.\,10.\,1931 ebd.@\textsc{Schnitzler, Arthur} (15.\,5.\,1862 Wien – 21.\,10.\,1931 ebd.), \emph{Schriftsteller, Mediziner}!Märchen. Schauspiel in drei Aufzügen@\strich\emph{Das Märchen. Schauspiel in drei Aufzügen}|pw}« eine Erſcheinung\pwindex{Schnitzler, Arthur 15.\,5.\,1862 Wien – 21.\,10.\,1931 ebd.@\textsc{Schnitzler, Arthur} (15.\,5.\,1862 Wien – 21.\,10.\,1931 ebd.), \emph{Schriftsteller, Mediziner}!Märchen. Schauspiel in drei Aufzügen@\strich\emph{Das Märchen. Schauspiel in drei Aufzügen}|pwv}, wie{ }ſie in dem letzten Jahrzehnt in der deutſchen Litteratur{ }ſo
               bemerkenswerth kaum noch da war und iſt mit \textsc{Sudermann\pwindex{Sudermann, Hermann 30.\,9.\,1857 Macikai – 21.\,11.\,1928 Berlin@\textsc{Sudermann, Hermann} (30.\,9.\,1857 Macikai – 21.\,11.\,1928 Berlin), \emph{Schriftsteller}|pw}} und \textsc{Hauptmann\pwindex{Hauptmann, Gerhart 15.\,11.\,1862 Szczawno-Zdrój – 6.\,6.\,1946 Jagniątków@\textsc{Hauptmann, Gerhart} (15.\,11.\,1862 Szczawno-Zdrój – 6.\,6.\,1946 Jagniątków), \emph{Schriftsteller}|pw}} zu nennen. Dramatiſch, unter dem {\pb}Geſichtspunkte der Aufführbarkeit ein Unvollendetes\pwindex{Schnitzler, Arthur 15.\,5.\,1862 Wien – 21.\,10.\,1931 ebd.@\textsc{Schnitzler, Arthur} (15.\,5.\,1862 Wien – 21.\,10.\,1931 ebd.), \emph{Schriftsteller, Mediziner}!Märchen. Schauspiel in drei Aufzügen@\strich\emph{Das Märchen. Schauspiel in drei Aufzügen}|pwv}, das in Kürze Vollendetes verſpricht. Ich rathe
               Dir entſchieden ab, das »Märchen\pwindex{Schnitzler, Arthur 15.\,5.\,1862 Wien – 21.\,10.\,1931 ebd.@\textsc{Schnitzler, Arthur} (15.\,5.\,1862 Wien – 21.\,10.\,1931 ebd.), \emph{Schriftsteller, Mediziner}!Märchen. Schauspiel in drei Aufzügen@\strich\emph{Das Märchen. Schauspiel in drei Aufzügen}|pw}« \label{K_L02674-7v}\edtext{aufführen}{\lemma{\textnormal{\emph{aufführen}}}\Cendnote{\textnormal{\emph{Das Märchen}\pwindex{Schnitzler, Arthur 15.\,5.\,1862 Wien – 21.\,10.\,1931 ebd.@\textsc{Schnitzler, Arthur} (15.\,5.\,1862 Wien – 21.\,10.\,1931 ebd.), \emph{Schriftsteller, Mediziner}!Märchen. Schauspiel in drei Aufzügen@\strich\emph{Das Märchen. Schauspiel in drei Aufzügen}|pwk} wurde am 1. 12. 1893 am \emph{Deutschen Volkstheater}\orgindex{Volkstheater@Volkstheater|pwk} in Wien\oindex{Wien@\textbf{Wien}, \emph{Verwaltungsgebiet}|pwk} uraufgeführt, mit einem von Schnitzler modifizierten Schluss.}}}\label{K_L02674-7} zu laſſen; es gibt nur einen Weg
               für Dich: weiterſchreiben. Das thut weh; aber Du haſt noch keine Berechtigung, Dich
               auszuruhen; denke,{ }ſeit wie kurzer Zeit Du erſt auf dem Wege biſt. Und der Erfolg
               beſteht für Leute wie Dich, deren Berufung außer Zweifel{ }ſteht, nur in der Frage, ob{ }ſie nicht zu früh bequem werden. Ein neues Stück\pwindex{Schnitzler, Arthur 15.\,5.\,1862 Wien – 21.\,10.\,1931 ebd.@\textsc{Schnitzler, Arthur} (15.\,5.\,1862 Wien – 21.\,10.\,1931 ebd.), \emph{Schriftsteller, Mediziner}!Märchen. Schauspiel in drei Aufzügen@\strich\emph{Das Märchen. Schauspiel in drei Aufzügen}|pwv} alſo; in einem halben Jahre arbeiteſt Du vielleicht
               dann den dritten Akt\pwindex{Schnitzler, Arthur 15.\,5.\,1862 Wien – 21.\,10.\,1931 ebd.@\textsc{Schnitzler, Arthur} (15.\,5.\,1862 Wien – 21.\,10.\,1931 ebd.), \emph{Schriftsteller, Mediziner}!Märchen. Schauspiel in drei Aufzügen@\strich\emph{Das Märchen. Schauspiel in drei Aufzügen}|pwv} des »Märchens\pwindex{Schnitzler, Arthur 15.\,5.\,1862 Wien – 21.\,10.\,1931 ebd.@\textsc{Schnitzler, Arthur} (15.\,5.\,1862 Wien – 21.\,10.\,1931 ebd.), \emph{Schriftsteller, Mediziner}!Märchen. Schauspiel in drei Aufzügen@\strich\emph{Das Märchen. Schauspiel in drei Aufzügen}|pw}« um, und da haſt Du auch \strikeout{da \textcolor{gray}{ein}} damit einen dramatiſchen Erfolg \textsc{in petto}. Daß der
               Dialog von \textsc{A} bis \textsc{Z} voll iſt der
               entzückendſten Sachen habe ich \strikeout{\textcolor{gray}{×}} wohl{ }ſchon geſagt. Kein einziger unter den \label{K_L02674-8v}\edtext{Jungdeutschen}{\lemma{\textnormal{\emph{Jungdeutschen}}}\Cendnote{\textnormal{hier als Synonym für deutschsprachige Autorinnen und Autoren am Beginn ihrer
                  Karriere}}}\label{K_L02674-8} in Berlin\oindex{Berlin@\textbf{Berlin}, \emph{Hauptstadt}|pw} oder Wien\oindex{Wien@\textbf{Wien}, \emph{Verwaltungsgebiet}|pw} iſt Dir das {\pb}nachzuthun imſtande. Wie hoch{ }ſteht das »Märchen\pwindex{Schnitzler, Arthur 15.\,5.\,1862 Wien – 21.\,10.\,1931 ebd.@\textsc{Schnitzler, Arthur} (15.\,5.\,1862 Wien – 21.\,10.\,1931 ebd.), \emph{Schriftsteller, Mediziner}!Märchen. Schauspiel in drei Aufzügen@\strich\emph{Das Märchen. Schauspiel in drei Aufzügen}|pw}« mit allen{ }ſeinen Fehlern z. B. über \textsc{Herzl\pwindex{Herzl, Theodor 2.\,5.\,1860 Budapest – 3.\,7.\,1904 Edlach@\textsc{Herzl, Theodor} (2.\,5.\,1860 Budapest – 3.\,7.\,1904 Edlach), \emph{Schriftsteller, Journalist}|pw}}’s Sachen! {\dotsfour}\pend
           
\pstart
           Im Vertrauen auf Deine Freundſchaft, mein lieber Arthur, habe ich Dir geſagt, was ich
               denke, ohne ein \label{K_L02674-9v}\edtext{\textsc{Jota}}{\lemma{\textnormal{\emph{Jota}}}\Cendnote{\textnormal{Redewendung: Ohne die kleinste
                  Abänderung. (»Jota« bezeichnet den kleinsten Buchstaben im
                  griechischen Alphabet.)}}}\label{K_L02674-9} zu ändern. Es war unklug von mir, denn eine
               Bitterkeit wird bei Dir doch zurückbleiben. Ich habe Dir vielleicht noch nie{ }ſo weh
               gethan. Aber ich mußte wohl. Freundespflicht! Wenn \uline{ich} Dir nicht die Wahrheit{ }ſagen{ }ſollte – wer \strikeout{da\textcolor{gray}{n}} denn{ }ſonſt? Und{ }ſo bin ich wieder einmal das Opfer meiner Pflicht geworden,
               umſomehr als ich ja, wie Du weißt, nicht zu den Leuten gehöre, welche über allen
               Nachtheilen der Pflichterfüllung{ }ſich mit dem Bewußtſein begnügen, daß es eben doch
               die Pflicht war.\pend
           
\pstart
           Grüß’ Dich Gott!{\\[\baselineskip]}Dein{\\[\baselineskip]}\spacefill\mbox{Paul Goldmann}\pend
           \leftskip=0em{}
\pstart
           \noindent{}Bitte,{ }ſchick’ mir ein paar Empfehlungen für Paris\oindex{Paris@\textbf{Paris}, \emph{Hauptstadt}|pw}! – Grüße an \textsc{Richard\pwindex{Beer-Hofmann, Richard 11.\,7.\,1866 Wien – 26.\,9.\,1945 New York City@\textsc{Beer-Hofmann, Richard} (11.\,7.\,1866 Wien – 26.\,9.\,1945 New York City), \emph{Schriftsteller}|pw}}, \textsc{Loris\pwindex{Hofmannsthal, Hugo von 1.\,2.\,1874 Wien – 15.\,7.\,1929 Rodaun@\textsc{Hofmannsthal, Hugo von} (1.\,2.\,1874 Wien – 15.\,7.\,1929 Rodaun), \emph{Schriftsteller}|pw}} und \textsc{Kapper\pwindex{Kapper, Friedrich 21.\,4.\,1861 Wien – 22.\,7.\,1939 ebd.@\textsc{Kapper, Friedrich} (21.\,4.\,1861 Wien – 22.\,7.\,1939 ebd.), \emph{Mediziner}|pw}}.\pend
           \selectlanguage{ngerman}\endnumbering\briefempfaengerindex{Schnitzler, Arthur@\textsc{Schnitzler, Arthur}!zzzGoldmann, Paul@\emph{von Paul Goldmann}!1891-12-122@{12. 12. [1891]}|)be}\mylabel{L02674h}  \newcommand{\dateiname}{L02674}\newcommand{\titel}{Paul Goldmann an Arthur Schnitzler, 12. 12. [1891]}\newcommand{\editorInnen}{Martin Anton Müller und Laura Untner}%% latex-leseansicht-abspann.tex
%% Abspann für die Leseansicht.
%% Der Schalter \ifkorrekturansicht ist bereits durch den Vorspann gesetzt.

%% latex-abspann.tex
%% Gemeinsamer Abspann für Korrekturansicht und Leseansicht.
%% Setzt den Schalter \ifkorrekturansicht voraus (gesetzt in den
%% einbindenden Dateien latex-korrekturansicht-abspann.tex bzw.
%% latex-leseansicht-abspann.tex).
%% ---------------------------------------------------------------

\normalsize

% Das esempio-Environment wird nur in der Leseansicht benötigt
\ifkorrekturansicht\else
\newenvironment{esempio}[3]%
{
    \vspace{1.5ex}
    \rlap{\underline{#1}}
    \par
    \setlength{\parindent}{0cm}
    \nopagebreak
    \leftskip=#2cm
    \rightskip=#3cm
}
{
    \par
}
\fi

\doendnotes{C}
\bigskip
\vfill

\clearpage

\footnotesize

\ifkorrekturansicht
  \lohead{\textsc{register}}
\fi

% theindex-Environment neu definieren ohne reledmac
\makeatletter
\renewenvironment{theindex}{%
  \ifkorrekturansicht
    \section*{\indexname}%
  \else
    \subsubsection*{Index der erwähnten Entitäten}%
  \fi
  \setlength{\parindent}{0pt}%
  \setlength{\parskip}{0pt plus 0.3pt}%
  \let\item\@idxitem
}{%
  \ifkorrekturansicht\clearpage\fi
}
\makeatother

\IfFileExists{\jobname-pw.ind}{\input{\jobname-pw.ind}}{}

% Quellenangabe nur in der Leseansicht
\ifkorrekturansicht\else
% Fallback-Definitionen, falls die .tex-Datei \titel etc. nicht gesetzt hat
\providecommand{\titel}{}
\providecommand{\editorInnen}{}
\providecommand{\dateiname}{\jobname}

\vspace{3cm}

\vfill

\footnotesize
\textsc{Quelle}: \titel. Herausgegeben von {\editorInnen}. In: \emph{Arthur Schnitzler: Briefwechsel mit Autorinnen und Autoren}.
 Digitale Edition, https://schnitzler-briefe.acdh.oeaw.ac.at/{\dateiname}.html (Stand \today)
\fi

\end{document}


