%% latex-korrekturansicht-vorspann.tex
%% Vorspann für die Korrekturansicht.
%% Lädt die gemeinsame Datei latex-vorspann.tex mit gesetztem Schalter.

\newif\ifkorrekturansicht
\korrekturansichttrue

\input{../tex-inputs/latex-vorspann}


\section[Hugo von Hofmannsthal an Arthur Schnitzler, {[}5. 2. 1917{]}]{L02254 Hugo von Hofmannsthal an Arthur Schnitzler, {[}5. 2. 1917{]}}
\nopagebreak\mylabel{L02254v}
\rehead{ }\normalsize\beginnumbering\briefempfaengerindex{Schnitzler, Arthur@\textsc{Schnitzler, Arthur}!zzzHofmannsthal, Hugo von@\emph{von Hugo von Hofmannsthal}!1917-02-051@{{[}5. 2. 1917{]}}|(be}
\toendnotes[C]{\smallbreak\pagebreak[2]}\Standort{CUL, Schnitzler, B 43.}
\physDesc{Brief, 1 Blatt, 2 Seiten, 699 Zeichen
\newline{}Handschrift: schwarze Tinte, deutsche Kurrent
\newline{}Schnitzler: 1) mit Bleistift datiert: »5/2 917« und beschriftet: »\textsc{Hugo}«  2) mit rotem Buntstift eine Unterstreichung
\newline{}Ordnung: 1) mit Bleistift von unbekannter Hand nummeriert: »\strikeout{343}«  2) mit Bleistift von unbekannter Hand nummeriert:
                                    »356«}
\buchAbdrucke{\weitereDrucke{Hugo von Hofmannsthal, Arthur Schnitzler: \emph{Briefwechsel}. Frankfurt am Main: \emph{S. Fischer} 1964, S. 280.} }\toendnotes[C]{\smallbreak}
\pstart
           \raggedleft{}{\pb}Montag\pend
           
\pstart{}mein lieber Arthur\pend\vspace{0.5em}
\pstart
           heute abend iſt es leider nicht gegangen, weil Gerty\pwindex{Hofmannsthal, Gertrude von 16.03.1880 – 09.11.1959@\textsc{Hofmannsthal, Gertrude von} (16.03.1880 – 09.11.1959)|pw} mit den Kindern\pwindex{Zimmer, Christiane 14.05.1902 – 05.01.1987@\textsc{Zimmer, Christiane} (14.05.1902 – 05.01.1987)|pwv}\pwindex{Hofmannsthal, Raimund von 26.5.1906 – 20.03.1974@\textsc{Hofmannsthal, Raimund von} (26.5.1906 – 20.03.1974)|pwv}\pwindex{Hofmannsthal, Franz von 20.10.1903 – 13.07.1929@\textsc{Hofmannsthal, Franz von} (20.10.1903 – 13.07.1929)|pwv} zur Wieſenthal\pwindex{Wiesenthal, Grethe 09.12.1885 – 24.06.1970@\textsc{Wiesenthal, Grethe} (09.12.1885 – 24.06.1970), \emph{Tänzer/Tänzerin}|pw} geht und ich etwas mit Andrian\pwindex{Andrian-Werburg, Leopold von 09.05.1875 – 19.11.1951@\textsc{Andrian-Werburg, Leopold von} (09.05.1875 – 19.11.1951), \emph{Schriftsteller/Schriftstellerin, Diplomat/Diplomatin}|pw}{ }ſprechen muſs, der i{\geminationm}er erst von 9\textsuperscript{h} abends an frei iſt.\pend
           
\pstart
           Euer Herko{\geminationm}en Mittwoch iſt ein lieber
               Gedanke, aber ſo weit ſind wir noch nicht. Es iſt ja noch längſt \label{K_L02254-1v}\edtext{keine Wohnung\oindex{Stallburggasse@\textbf{Stallburggasse}, \emph{Straße (K.STR)}|pwv}}{\lemma{\textnormal{\emph{keine Wohnung}}}\Cendnote{\textnormal{Gemeint ist die Wohnung in der Stallburggasse 2\oindex{Stallburggasse@\textbf{Stallburggasse}, \emph{Straße (K.STR)}|pwk}, die sie sich
                  herrichteten.}}}\label{K_L02254-1}, die Handwerker liefern nichts, und ich habe auch, unter
               immer neuen Sorgen u. Verdüſterungen, gar nicht den Kopf, {\pb}die Leute zu drängen.\pend
           
\pstart
           Es ſcheint jetzt daſs ich erſt Ende der Woche abreiſen kann, ſo könnten wir \label{K_L02254-2v}\edtext{Mittwoch}{\lemma{\textnormal{\emph{Mittwoch}}}\Cendnote{\textnormal{Vgl. A. S.: \emph{Tagebuch}, 7. 2. 1917.
               }}}\label{K_L02254-2}{ }Abends zu Euch ko{\geminationm}en: Vorausſetzung ein
               wirklich der Situation gemäßes Nachtmahl, Brot bringen wir mit.\pend
           
\pstart
           Paſst es Euch nicht, bitten wir um Abſage morgen Dienstag{ }vormittags an \uline{229}.\pend
           
\pstart
           Ihr{\\[\baselineskip]}\spacefill\mbox{Hugo.}\pend
           \leftskip=0em{}\selectlanguage{ngerman}\endnumbering\briefempfaengerindex{Schnitzler, Arthur@\textsc{Schnitzler, Arthur}!zzzHofmannsthal, Hugo von@\emph{von Hugo von Hofmannsthal}!1917-02-051@{{[}5. 2. 1917{]}}|)be}\mylabel{L02254h}  \normalsize

\doendnotes{C}
\bigskip
\vfill

\clearpage

\footnotesize

\lohead{\textsc{register}}

% Definiere theindex-Environment komplett neu ohne reledmac
\makeatletter
\renewenvironment{theindex}{%
  \section*{\indexname}%
  \setlength{\parindent}{0pt}%
  \setlength{\parskip}{0pt plus 0.3pt}%
  \let\item\@idxitem
}{%
  \clearpage
}
\makeatother

\IfFileExists{\jobname-pw.ind}{\input{\jobname-pw.ind}}{}

\end{document}

      