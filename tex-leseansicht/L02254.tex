%% latex-leseansicht-vorspann.tex
%% Vorspann für die Leseansicht.
%% Lädt die gemeinsame Datei latex-vorspann.tex mit nicht gesetztem Schalter.

\newif\ifkorrekturansicht
\korrekturansichtfalse

\input{../tex-inputs/latex-vorspann}


               \section[Hugo von Hofmannsthal an Arthur Schnitzler, {[}5. 2. 1917{]}]{ Hugo von Hofmannsthal an Arthur Schnitzler, {[}5. 2. 1917{]}}\nopagebreak\mylabel{v}\rehead{ }\begin{ledgroupsized}[t]{13cm}\normalsize\beginnumbering\briefempfaengerindex{Schnitzler, Arthur@\textsc{Schnitzler, Arthur}!zzzHofmannsthal, Hugo von@\emph{von Hugo von Hofmannsthal}!1917-02-051@{{[}5. 2. 1917{]}}|(be} \toendnotes[C]{\smallbreak\pagebreak[2]} \Standort{CUL, Schnitzler, B 43.}
\physDesc{Brief, 1 Blatt, 2 Seiten
\newline{}Handschrift: schwarze Tinte, deutsche Kurrent
\newline{}Schnitzler: 1) mit Bleistift datiert: »5/2 917« und beschriftet: »\textsc{Hugo}« 2) mit rotem Buntstift eine Unterstreichung\newline{}Ordnung: 1) mit Bleistift von unbekannter Hand nummeriert: »\strikeout{343}« 2) mit Bleistift von unbekannter Hand nummeriert: »356«}\buchAbdrucke{\weitereDrucke{Hugo von Hofmannsthal, Arthur Schnitzler: \emph{Briefwechsel}. Hg. Therese Nickl und Heinrich Schnitzler. Frankfurt am Main: \emph{S. Fischer} 1964, S. 280.} }\toendnotes[C]{\smallbreak}\pstart
           \raggedleft{}{\pb}Montag\pend
           \pstart{}mein lieber Arthur\pend\pstart
           heute abend iſt es leider nicht gegangen, weil Gerty\pwindex{Hofmannsthal, Gertrude von 16.03.1880 – 09.11.1959@\textsc{Hofmannsthal, Gertrude von} (16.03.1880 – 09.11.1959)|pw} mit den Kindern\pwindex{Hofmannsthal, Christiane von 14.05.1902 – 05.01.1987@\textsc{Hofmannsthal, Christiane von} (14.05.1902 – 05.01.1987)|pwv}\pwindex{Hofmannsthal, Raimund von 26.5.1906 – 20.03.1974@\textsc{Hofmannsthal, Raimund von} (26.5.1906 – 20.03.1974)|pwv}\pwindex{Hofmannsthal, Franz von 20.10.1903 – 13.07.1929@\textsc{Hofmannsthal, Franz von} (20.10.1903 – 13.07.1929)|pwv} zur Wieſenthal\pwindex{Wiesenthal, Grethe 09.12.1885 – 24.06.1970@\textsc{Wiesenthal, Grethe} (09.12.1885 – 24.06.1970), \emph{Tänzerin}|pw} geht und ich
               etwas mit Andrian\pwindex{Andrian-Werburg, Leopold von 09.05.1875 – 19.11.1951@\textsc{Andrian-Werburg, Leopold von} (09.05.1875 – 19.11.1951), \emph{Schriftsteller, Diplomat}|pw}{ }ſprechen muſs, der i{\geminationm}er erst von 9\textsuperscript{h} abends an frei iſt.\pend
           \pstart
           Euer Herko{\geminationm}en Mittwoch iſt ein lieber
               Gedanke, aber ſo weit ſind wir noch nicht. Es iſt ja noch längſt \label{K_L02254_1v}\edtext{keine Wohnung\oindex{Stallburggasse@\textbf{Stallburggasse}|pwv}}{\lemma{\textnormal{\emph{keine Wohnung}}}\Cendnote{\textnormal{Gemeint ist die Wohnung in der Stallburggasse 2\oindex{Stallburggasse@\textbf{Stallburggasse}|pwk}, die sie sich
                  herrichteten.}}}\label{K_L02254_1h}, die Handwerker liefern nichts, und ich habe auch, unter
               immer neuen Sorgen u. Verdüſterungen, gar nicht den Kopf, {\pb}die Leute zu drängen.\pend
           \pstart
           Es ſcheint jetzt daſs ich erſt Ende der Woche abreiſen kann, ſo könnten wir \label{K_L02254_2v}\edtext{Mittwoch}{\lemma{\textnormal{\emph{Mittwoch}}}\Cendnote{\textnormal{vgl. A. S.: \emph{Tagebuch}, 7. 2. 1917}}}\label{K_L02254_2h}{ }Abends zu Euch ko{\geminationm}en: Vorausſetzung ein
               wirklich der Situation gemäßes Nachtmahl, Brot bringen wir mit.\pend
           \pstart
           Paſst es Euch nicht, bitten wir um Abſage morgen Dienstag{ }vormittags an \uline{229}.\pend
           \pstart
           Ihr{\\[\baselineskip]}\spacefill\mbox{Hugo.}\pend
           \leftskip=0em{}\endnumbering\briefempfaengerindex{Schnitzler, Arthur@\textsc{Schnitzler, Arthur}!zzzHofmannsthal, Hugo von@\emph{von Hugo von Hofmannsthal}!1917-02-051@{{[}5. 2. 1917{]}}|)be}\mylabel{h}\end{ledgroupsized}  \newcommand{\dateiname}{L02254}\newcommand{\titel}{Hugo von Hofmannsthal an Arthur Schnitzler, [5. 2. 1917]}\newcommand{\editorInnen}{Martin Anton Müller und Gerd-Hermann Susen}
            \footnotesize
\begin{ledgroupsized}[t]{11.5cm}
\doendnotes{C}
\end{ledgroupsized}
         %% latex-leseansicht-abspann.tex
%% Abspann für die Leseansicht.
%% Der Schalter \ifkorrekturansicht ist bereits durch den Vorspann gesetzt.

%% latex-abspann.tex
%% Gemeinsamer Abspann für Korrekturansicht und Leseansicht.
%% Setzt den Schalter \ifkorrekturansicht voraus (gesetzt in den
%% einbindenden Dateien latex-korrekturansicht-abspann.tex bzw.
%% latex-leseansicht-abspann.tex).
%% ---------------------------------------------------------------

\normalsize

% Das esempio-Environment wird nur in der Leseansicht benötigt
\ifkorrekturansicht\else
\newenvironment{esempio}[3]%
{
    \vspace{1.5ex}
    \rlap{\underline{#1}}
    \par
    \setlength{\parindent}{0cm}
    \nopagebreak
    \leftskip=#2cm
    \rightskip=#3cm
}
{
    \par
}
\fi

\doendnotes{C}
\bigskip
\vfill

\clearpage

\footnotesize

\ifkorrekturansicht
  \lohead{\textsc{register}}
\fi

% theindex-Environment neu definieren ohne reledmac
\makeatletter
\renewenvironment{theindex}{%
  \ifkorrekturansicht
    \section*{\indexname}%
  \else
    \subsubsection*{Index der erwähnten Entitäten}%
  \fi
  \setlength{\parindent}{0pt}%
  \setlength{\parskip}{0pt plus 0.3pt}%
  \let\item\@idxitem
}{%
  \ifkorrekturansicht\clearpage\fi
}
\makeatother

\IfFileExists{\jobname-pw.ind}{\input{\jobname-pw.ind}}{}

% Quellenangabe nur in der Leseansicht
\ifkorrekturansicht\else
% Fallback-Definitionen, falls die .tex-Datei \titel etc. nicht gesetzt hat
\providecommand{\titel}{}
\providecommand{\editorInnen}{}
\providecommand{\dateiname}{\jobname}

\vspace{3cm}

\vfill

\footnotesize
\textsc{Quelle}: \titel. Herausgegeben von {\editorInnen}. In: \emph{Arthur Schnitzler: Briefwechsel mit Autorinnen und Autoren}.
 Digitale Edition, https://schnitzler-briefe.acdh.oeaw.ac.at/{\dateiname}.html (Stand \today)
\fi

\end{document}


      