%% latex-leseansicht-vorspann.tex
%% Vorspann für die Leseansicht.
%% Lädt die gemeinsame Datei latex-vorspann.tex mit nicht gesetztem Schalter.

\newif\ifkorrekturansicht
\korrekturansichtfalse

\input{../tex-inputs/latex-vorspann}


         
         \newcommand{\erwaehntePersonen}{Personen:  ?? [politischer Korrespondent der Frankfurter Zeitung in Berlin 1897], Emma Altmann, Charlotte Bondy, Vít Šalomoun Bondy, Marie Reinhard, Louise Schnitzler, Leopold Sonnemann, Alice Ziegler}
         \newcommand{\erwaehnteInstitutionen}{Institutionen: Frankfurter Zeitung, Neue Freie Presse}
         \newcommand{\erwaehnteOrte}{Orte: Berlin, China, Lobkowitzplatz, Paris, Prag, Wien, rue de la Bourse}
         \newcommand{\erwaehnteWerke}{
               \section[ Paul Goldmann an Arthur Schnitzler, 30. 12. {[}1897{]}]{ Paul Goldmann an Arthur Schnitzler, 30. 12. {[}1897{]}}\nopagebreak\mylabel{v}\rehead{ }\begin{ledgroupsized}[t]{13cm}\normalsize\beginnumbering \toendnotes[C]{\smallbreak\pagebreak[2]} \Standort{DLA, A:Schnitzler, HS.NZ85.1.3167.}
\physDesc{Brief, 1 Blatt, 3 Seiten
\newline{}Handschrift: blaue Tinte, deutsche Kurrent
\newline{}Schnitzler: 1) Die obere und untere Seitenkante mutmaßlich beim Öffnen des
                                 Briefes mit Brieföffner abgeschnitten, was auf der zweiten Seite zu
                                 minimaler Textbeschädigung der letzten Zeile führte.  2) mit Bleistift das Jahr »97« vermerkt 3) mit rotem Buntstift eine Unterstreichung}\toendnotes[C]{\smallbreak}\pstart
           \noindent{}{\pb}\textcolor{gray}{\textbf{\textbf{Frankfurter Zeitung\orgindex{Frankfurter Zeitung@Frankfurter Zeitung|pw}}}}\pend
           \pstart
           \textcolor{gray}{\textbf{(\begin{otherlanguage}{french}Gazette de Francfort\end{otherlanguage}\orgindex{Frankfurter Zeitung@Frankfurter Zeitung|pw}).}}\pend
           \pstart
           \textcolor{gray}{\textbf{\textbf{\begin{otherlanguage}{french}Fondateur M.\end{otherlanguage}{ }L. Sonnemann\pwindex{Sonnemann, Leopold 1831-10-29 – 1909-10-30@\textsc{Sonnemann, Leopold} (1831-10-29 – 1909-10-30), \emph{Journalist, Herausgeber}|pw}.}}}\pend
           \pstart
           \begin{otherlanguage}{french}\textcolor{gray}{\textbf{Journal politique, financier,}}\end{otherlanguage}\pend
           \pstart
           \begin{otherlanguage}{french}\textcolor{gray}{\textbf{commercial et littéraire.}}\end{otherlanguage}\pend
           \pstart
           \begin{otherlanguage}{french}\textcolor{gray}{\textbf{\textbf{Paraissant trois fois par jour.}}}\end{otherlanguage}\pend
           \pstart
           \begin{otherlanguage}{french}\textcolor{gray}{\textbf{\textbf{Bureau à Paris\oindex{Paris@\textbf{Paris}|pw}}}}\end{otherlanguage}\hfill \textsc{Paris\oindex{Paris@\textbf{Paris}|pw}}, 30. December.\pend
           \pstart
           \begin{otherlanguage}{french}\textcolor{gray}{\textbf{\textbf{10 \so{Rue de la Bourse}\oindex{rue de la Bourse@\textbf{rue de la Bourse}|pw}.}}}\end{otherlanguage}\pend
           \pstart\center{}Mein lieber Freund,\pend\pstart
           Ich erwarte täglich einen Brief von Dir und bin ſehr traurig, daß er gar nicht kommt.
               Biſt Du unwohl? Oder was geht ſonſt vor? Ich bin recht ungeduldig, es zu wiſſen, denn
               Deine letzten Briefe waren nicht gerade erheiternd.\pend
           \pstart
           Ich will Dir heut nur ein recht glückliches neues Jahr
               wünſchen. Und das Gleiche Deiner Freundin\pwindex{Reinhard, Marie 1871-03-13 – 1899-03-18@\textsc{Reinhard, Marie} (1871-03-13 – 1899-03-18), \emph{Gesangspädagogin}|pwv}.\pend
           \pstart
           Die \label{K_L02835-77v}\edtext{Adreſſe der Frau \textsc{Altmann}\pwindex{Altmann, Emma 22.10.1849 – 31.12.1930@\textsc{Altmann, Emma} (22.10.1849 – 31.12.1930)|pw}}{\lemma{\textnormal{\emph{Adreſſe der Frau Altmann}}}\Cendnote{\textnormal{Sie wohnte am Lobkowitzplatz 1\oindex{Lobkowitzplatz@\textbf{Lobkowitzplatz}|pwk}. Ein Besuch Schnitzler\pwindex{Schnitzler, Arthur 15.05.1862 – 21.10.1931@\textsc{Schnitzler, Arthur} (15.05.1862 – 21.10.1931), \emph{Schriftsteller, Mediziner}|pwk}s bei ihr ist für die kommenden Tage nicht
                  belegt.}}}\label{K_L02835-77h} weiß ich nicht. Willſt Du ſo gut ſein, die \label{K_L02835-44v}\edtext{beiliegende Karte}{\lemma{\textnormal{\emph{beiliegende Karte}}}\Cendnote{\textnormal{Beilage nicht erhalten}}}\label{K_L02835-44h} an ſie zu befördern?\pend
           \pstart
           {\pb}In meiner Exiſtenz wird es wohl in
               einiger Zeit \strikeout{ei} eine Änderung geben. Ich bin mehr \textsc{Paris\oindex{Paris@\textbf{Paris}|pw}}\textcolor{gray}{-}müde als je. Ich habe meinem Chef\pwindex{Sonnemann, Leopold 1831-10-29 – 1909-10-30@\textsc{Sonnemann, Leopold} (1831-10-29 – 1909-10-30), \emph{Journalist, Herausgeber}|pwv} geſchrieben, daß ich nach \textsc{Berlin\oindex{Berlin@\textbf{Berlin}|pw}} will. Aber es ſcheint, daß das nicht geht, weil unſer \label{K_L02835-12v}\edtext{Berlin\oindex{Berlin@\textbf{Berlin}|pw}er politiſcher Correſpondent\pwindex{?? [politischer Korrespondent der Frankfurter Zeitung in Berlin 1897] @\textsc{?? [politischer Korrespondent der Frankfurter Zeitung in Berlin 1897]}|pwv}}{\lemma{\textnormal{\emph{Berliner … Correſpondent}}}\Cendnote{\textnormal{nicht ermittelt}}}\label{K_L02835-12h}, der meine
               Rivalität fürchtet, gegen mich hetzt. Zur Zeit beſteht das Project, mich auf ein Jahr
               nach \textsc{China\oindex{China@\textbf{China}|pw}} zu ſchicken. Auch von \textsc{Wien\oindex{Wien@\textbf{Wien}|pw}} war die Rede. Aber ſo froh ich wäre, in \textsc{Wien\oindex{Wien@\textbf{Wien}|pw}} mit Euch zu leben, ſo ſehe ich doch \strikeout{in
                     r\textcolor{gray}{a}} bei kühler Überlegung, daß ich dort keinerlei Zukunft habe. Es gibt dort nur
               die Neue Freie Preſſe\orgindex{Neue Freie Presse@Neue Freie Presse|pw}, und ich bin \strikeout{zu} doch zu gut, um bei \uline{den} Leuten Jahre lang zu \label{K_L02835-2v}\edtext{antichambriren}{\lemma{\textnormal{\emph{antichambriren}}}\Cendnote{\textnormal{sich dienstfertig im
                  Vorzimmer einer mächtigen Person aufhalten, um dadurch Gunst zu erlangen}}}\label{K_L02835-2h}.
               Auch würde meine Verſetzung nach \textsc{Wien\oindex{Wien@\textbf{Wien}|pw}} eine Gehalts-Reduction, beinahe um die Hälfte, bedeuten. Gott weiß, was bei
               alledem noch herauskommen wird! Bitte\label{T_L02835-1v}\edtext{\damage{, ſprich zu}}{\lemma{\textnormal{\emph{, ſprich zu}}}\Cendnote{\textnormal{am unteren Rand der beschädigten
                  Seite}}}\label{T_L02835-1h} keinem Menſchen darüber!\pend
           \pstart
           {\pb}Dabei wird es mit meinem \label{K_L02835-33v}\edtext{Auge}{\lemma{\textnormal{\emph{Auge}}}\Cendnote{\textnormal{siehe Paul Goldmann an Arthur Schnitzler, 2. [1.? 1897]}}}\label{K_L02835-33h} beinahe täglich ſchlechter.\pend
           \pstart
           Das kleine Fräulein\pwindex{Ziegler, Alice 1880-01-05 – Dezember 1943@\textsc{Ziegler, Alice} (1880-01-05 – Dezember 1943)|pwv} aus \textsc{Prag\oindex{Prag@\textbf{Prag}|pw}} hat mir ihre Photographie geſchickt. Was für ein liebes und ſüßes Geſicht!
               Glaubſt Du wirklich, ich ſollte nicht? Glaubſt Du ich \uline{dürſte} überhaupt? Haſt Du übrigens eine Ahnung, ob die Leute\pwindex{Bondy, Charlotte 25.03.1854 – 1914-03-07@\textsc{Bondy, Charlotte} (25.03.1854 – 1914-03-07), \emph{Schauspielerin}|pwv}\pwindex{Bondy, Vít Šalomoun 09.12.1831 – 31.10.1909@\textsc{Bondy, Vít Šalomoun} (09.12.1831 – 31.10.1909), \emph{Fabrikant}|pwv}\pwindex{Ziegler, Alice 1880-01-05 – Dezember 1943@\textsc{Ziegler, Alice} (1880-01-05 – Dezember 1943)|pwv} Geld haben?\pend
           \pstart
           Sei von Herzen gegrüßt, liebſter Freund, und ſchreib’ mir bald!\pend
           \pstart
           Dein treuer {\\[\baselineskip]}\spacefill\mbox{Paul Goldmann.}\pend
           \leftskip=0em{}\pstart
           \noindent{}Deiner Frau Mutter\pwindex{Schnitzler, Louise 1840-07-08 – 1911-09-09@\textsc{Schnitzler, Louise} (1840-07-08 – 1911-09-09)|pwv} bitte
                  ich meine ergebenen Neujahrs-Glückwünſche auszurichten.\pend
           
         
         \endnumbering\mylabel{h}\end{ledgroupsized}  \newcommand{\dateiname}{L02835}\newcommand{\titel}{Paul Goldmann an Arthur Schnitzler, 30. 12. [1897]}\newcommand{\editorInnen}{Martin Anton Müller und Laura Untner}%% latex-leseansicht-abspann.tex
%% Abspann für die Leseansicht.
%% Der Schalter \ifkorrekturansicht ist bereits durch den Vorspann gesetzt.

%% latex-abspann.tex
%% Gemeinsamer Abspann für Korrekturansicht und Leseansicht.
%% Setzt den Schalter \ifkorrekturansicht voraus (gesetzt in den
%% einbindenden Dateien latex-korrekturansicht-abspann.tex bzw.
%% latex-leseansicht-abspann.tex).
%% ---------------------------------------------------------------

\normalsize

% Das esempio-Environment wird nur in der Leseansicht benötigt
\ifkorrekturansicht\else
\newenvironment{esempio}[3]%
{
    \vspace{1.5ex}
    \rlap{\underline{#1}}
    \par
    \setlength{\parindent}{0cm}
    \nopagebreak
    \leftskip=#2cm
    \rightskip=#3cm
}
{
    \par
}
\fi

\doendnotes{C}
\bigskip
\vfill

\clearpage

\footnotesize

\ifkorrekturansicht
  \lohead{\textsc{register}}
\fi

% theindex-Environment neu definieren ohne reledmac
\makeatletter
\renewenvironment{theindex}{%
  \ifkorrekturansicht
    \section*{\indexname}%
  \else
    \subsubsection*{Index der erwähnten Entitäten}%
  \fi
  \setlength{\parindent}{0pt}%
  \setlength{\parskip}{0pt plus 0.3pt}%
  \let\item\@idxitem
}{%
  \ifkorrekturansicht\clearpage\fi
}
\makeatother

\IfFileExists{\jobname-pw.ind}{\input{\jobname-pw.ind}}{}

% Quellenangabe nur in der Leseansicht
\ifkorrekturansicht\else
% Fallback-Definitionen, falls die .tex-Datei \titel etc. nicht gesetzt hat
\providecommand{\titel}{}
\providecommand{\editorInnen}{}
\providecommand{\dateiname}{\jobname}

\vspace{3cm}

\vfill

\footnotesize
\textsc{Quelle}: \titel. Herausgegeben von {\editorInnen}. In: \emph{Arthur Schnitzler: Briefwechsel mit Autorinnen und Autoren}.
 Digitale Edition, https://schnitzler-briefe.acdh.oeaw.ac.at/{\dateiname}.html (Stand \today)
\fi

\end{document}


      