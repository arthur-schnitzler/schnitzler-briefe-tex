%% latex-leseansicht-vorspann.tex
%% Vorspann für die Leseansicht.
%% Lädt die gemeinsame Datei latex-vorspann.tex mit nicht gesetztem Schalter.

\newif\ifkorrekturansicht
\korrekturansichtfalse

\input{../tex-inputs/latex-vorspann}


\section[ Paul Goldmann an Arthur Schnitzler, 30. 12. [1897]]{L02835 Paul Goldmann an Arthur Schnitzler,  30. 12. [1897]}
\nopagebreak\mylabel{L02835v}
\rehead{ }\normalsize\beginnumbering\briefempfaengerindex{Schnitzler, Arthur@\textsc{Schnitzler, Arthur}!zzzGoldmann, Paul@\emph{von Paul Goldmann}!1897-12-303@{30. 12. [1897]}|(be}
\toendnotes[C]{\smallbreak\pagebreak[2]}
\correspDesc{Versand  durch Paul Goldmann am 30. 12. [1897] in Paris
\newline{}Erhalt  durch Arthur Schnitzler im Zeitraum [31. 12. 1897 – 4. 1. 1898?] in Wien}\toendnotes[C]{\smallbreak}
\Standort{DLA, A:Schnitzler, HS.NZ85.1.3167.}
\physDesc{Brief, 1 Blatt, 3 Seiten, 1710 Zeichen
\newline{}Handschrift: blaue Tinte, deutsche Kurrent
\newline{}Schnitzler: 1) Die obere und untere Seitenkante mutmaßlich beim Öffnen des
                                 Briefes mit Brieföffner abgeschnitten, was auf der zweiten Seite zu
                                 minimaler Textbeschädigung der letzten Zeile führte.  2) mit Bleistift das Jahr »97« vermerkt 3) mit rotem Buntstift eine Unterstreichung}\toendnotes[C]{\smallbreak}
\pstart
           {\pb}\textcolor{gray}{\textbf{\textbf{Frankfurter Zeitung\orgindex{Frankfurter Zeitung@Frankfurter Zeitung|pw}}}}\pend
           
\pstart
           \textcolor{gray}{\textbf{(\begin{otherlanguage}{french}Gazette de Francfort\end{otherlanguage}\orgindex{Frankfurter Zeitung@Frankfurter Zeitung|pw}).}}\pend
           
\pstart
           \textcolor{gray}{\textbf{\textbf{\begin{otherlanguage}{french}Fondateur M.\end{otherlanguage}{ }L. Sonnemann\pwindex{Sonnemann, Leopold 29.\,10.\,1831 Höchberg – 30.\,10.\,1909 Frankfurt am Main@\textsc{Sonnemann, Leopold} (29.\,10.\,1831 Höchberg – 30.\,10.\,1909 Frankfurt am Main), \emph{Journalist, Herausgeber}|pw}.}}}\pend
           
\pstart
           \begin{otherlanguage}{french}\textcolor{gray}{\textbf{Journal politique, financier,}}\end{otherlanguage}\pend
           
\pstart
           \begin{otherlanguage}{french}\textcolor{gray}{\textbf{commercial et littéraire.}}\end{otherlanguage}\pend
           
\pstart
           \begin{otherlanguage}{french}\textcolor{gray}{\textbf{\textbf{Paraissant trois fois par jour.}}}\end{otherlanguage}\pend
           
\pstart
           \begin{otherlanguage}{french}\textcolor{gray}{\textbf{\textbf{Bureau à Paris\oindex{Paris@\textbf{Paris}, \emph{Hauptstadt}|pw}}}}\end{otherlanguage}\hfill \textsc{Paris\oindex{Paris@\textbf{Paris}, \emph{Hauptstadt}|pw}}, 30. December.\pend
           
\pstart
           \begin{otherlanguage}{french}\textcolor{gray}{\textbf{\textbf{10 \so{Rue de la Bourse}\oindex{rue de la Bourse@\textbf{rue de la Bourse}, \emph{Straße}|pw}.}}}\end{otherlanguage}\pend
           
\pstart\center{}Mein lieber Freund,\pend\vspace{0.5em}
\pstart
           Ich erwarte täglich einen Brief von Dir und bin{ }ſehr traurig, daß er gar nicht kommt.
               Biſt Du unwohl? Oder was geht{ }ſonſt vor? Ich bin recht ungeduldig, es zu wiſſen, denn
               Deine letzten Briefe waren nicht gerade erheiternd.\pend
           
\pstart
           Ich will Dir heut nur ein recht glückliches neues Jahr
               wünſchen. Und das Gleiche Deiner Freundin\pwindex{Reinhard, Marie 13.\,3.\,1871 Wien – 18.\,3.\,1899 ebd.@\textsc{Reinhard, Marie} (13.\,3.\,1871 Wien – 18.\,3.\,1899 ebd.), \emph{Gesangspädagogin}|pwv}.\pend
           
\pstart
           Die \label{K_L02835-1v}\edtext{Adreſſe der Frau \textsc{Altmann}\pwindex{Altmann, Emma 22.\,10.\,1849 Budapest – 31.\,12.\,1930 Wien@\textsc{Altmann, Emma} (22.\,10.\,1849 Budapest – 31.\,12.\,1930 Wien)|pw}}{\lemma{\textnormal{\emph{Adresse der Frau Altmann}}}\Cendnote{\textnormal{Sie wohnte am Lobkowitzplatz 1\oindex{Wien@\textbf{Wien}!I., Innere Stadt@\textbf{I., Innere Stadt}!Lobkowitzplatz@\textbf{Lobkowitzplatz}, \emph{Platz}|pwk}. Ein Besuch Schnitzlers bei ihr ist für die kommenden Tage nicht
                  belegt.}}}\label{K_L02835-1} weiß ich nicht. Willſt Du{ }ſo gut{ }ſein, die \label{K_L02835-2v}\edtext{beiliegende Karte}{\lemma{\textnormal{\emph{beiliegende Karte}}}\Cendnote{\textnormal{Beilage nicht erhalten}}}\label{K_L02835-2} an{ }ſie zu befördern?\pend
           
\pstart
           {\pb}In meiner Exiſtenz wird es wohl in
               einiger Zeit \strikeout{ei} eine Änderung geben. Ich bin mehr \textsc{Paris\oindex{Paris@\textbf{Paris}, \emph{Hauptstadt}|pw}}\textcolor{gray}{-}müde als je. Ich habe meinem Chef\pwindex{Sonnemann, Leopold 29.\,10.\,1831 Höchberg – 30.\,10.\,1909 Frankfurt am Main@\textsc{Sonnemann, Leopold} (29.\,10.\,1831 Höchberg – 30.\,10.\,1909 Frankfurt am Main), \emph{Journalist, Herausgeber}|pwv} geſchrieben, daß ich nach \textsc{Berlin\oindex{Berlin@\textbf{Berlin}, \emph{Hauptstadt}|pw}} will. Aber es{ }ſcheint, daß das nicht geht, weil unſer \label{K_L02835-3v}\edtext{Berlin\oindex{Berlin@\textbf{Berlin}, \emph{Hauptstadt}|pw}er politiſcher Correſpondent\pwindex{?? [politischer Korrespondent der Frankfurter Zeitung in Berlin 1897] @\textsc{?? [politischer Korrespondent der Frankfurter Zeitung in Berlin 1897]}|pwv}}{\lemma{\textnormal{\emph{Berliner … Correspondent}}}\Cendnote{\textnormal{nicht ermittelt}}}\label{K_L02835-3}, der meine
               Rivalität fürchtet, gegen mich hetzt. Zur Zeit beſteht das Project, mich auf ein Jahr
               nach \textsc{China\oindex{China@\textbf{China}|pw}} zu{ }ſchicken. Auch von \textsc{Wien\oindex{Wien@\textbf{Wien}, \emph{Verwaltungsgebiet}|pw}} war die Rede. Aber{ }ſo froh ich wäre, in \textsc{Wien\oindex{Wien@\textbf{Wien}, \emph{Verwaltungsgebiet}|pw}} mit Euch zu leben,{ }ſo{ }ſehe ich doch \strikeout{in
                     r\textcolor{gray}{a}} bei kühler Überlegung, daß ich dort keinerlei Zukunft habe. Es gibt dort nur
               die Neue Freie Preſſe\orgindex{Neue Freie Presse@Neue Freie Presse|pw}, und ich bin \strikeout{zu} doch zu gut, um bei \uline{den} Leuten Jahre lang zu \label{K_L02835-4v}\edtext{antichambriren}{\lemma{\textnormal{\emph{antichambriren}}}\Cendnote{\textnormal{sich dienstfertig im
                  Vorzimmer einer mächtigen Person aufhalten, um dadurch Gunst zu erlangen}}}\label{K_L02835-4}.
               Auch würde meine Verſetzung nach \textsc{Wien\oindex{Wien@\textbf{Wien}, \emph{Verwaltungsgebiet}|pw}} eine Gehalts-Reduction, beinahe um die Hälfte, bedeuten. Gott weiß, was bei
               alledem noch herauskommen wird! Bitte\label{T_L02835-1v}\edtext{\damage{,{ }ſprich zu}}{\lemma{\textnormal{\emph{, sprich zu}}}\Cendnote{\textnormal{am unteren Rand der beschädigten
                  Seite}}}\label{T_L02835-1} keinem Menſchen darüber!\pend
           
\pstart
           {\pb}Dabei wird es mit meinem \label{K_L02835-5v}\edtext{Auge}{\lemma{\textnormal{\emph{Auge}}}\Cendnote{\textnormal{Siehe XXXX Auszeichnungsfehler: Dokument L02792 nicht gefunden.
               }}}\label{K_L02835-5} beinahe täglich{ }ſchlechter.\pend
           
\pstart
           Das kleine Fräulein\pwindex{Ziegler, Alice 5.\,1.\,1880 Prag – Dezember 1943 Konzentrationslager Auschwitz-Birkenau@\textsc{Ziegler, Alice} (5.\,1.\,1880 Prag – Dezember 1943 Konzentrationslager Auschwitz-Birkenau)|pwv} aus \textsc{Prag\oindex{Prag@\textbf{Prag}, \emph{Land}|pw}} hat mir ihre Photographie geſchickt. Was für ein liebes und{ }ſüßes Geſicht!
               Glaubſt Du wirklich, ich{ }ſollte nicht? Glaubſt Du ich \uline{dürfte} überhaupt? Haſt Du übrigens eine Ahnung, ob die Leute\pwindex{Bondy, Charlotte 25.\,3.\,1854 Bielsko-Biała – 7.\,3.\,1914 Prag@\textsc{Bondy, Charlotte} (25.\,3.\,1854 Bielsko-Biała – 7.\,3.\,1914 Prag), \emph{Schauspielerin}|pwv}\pwindex{Bondy, Vít Šalomoun 9.\,12.\,1831 Prag – 31.\,10.\,1909 ebd.@\textsc{Bondy, Vít Šalomoun} (9.\,12.\,1831 Prag – 31.\,10.\,1909 ebd.), \emph{Fabrikant}|pwv}\pwindex{Ziegler, Alice 5.\,1.\,1880 Prag – Dezember 1943 Konzentrationslager Auschwitz-Birkenau@\textsc{Ziegler, Alice} (5.\,1.\,1880 Prag – Dezember 1943 Konzentrationslager Auschwitz-Birkenau)|pwv} Geld haben?\pend
           
\pstart
           Sei von Herzen gegrüßt, liebſter Freund, und{ }ſchreib’ mir bald!\pend
           
\pstart
           Dein treuer {\\[\baselineskip]}\spacefill\mbox{Paul Goldmann.}\pend
           \leftskip=0em{}
\pstart
           \noindent{}Deiner Frau Mutter\pwindex{Schnitzler, Louise 8.\,7.\,1840 Kőszeg – 9.\,9.\,1911 Wien@\textsc{Schnitzler, Louise} (8.\,7.\,1840 Kőszeg – 9.\,9.\,1911 Wien)|pwv} bitte
                  ich meine ergebenen Neujahrs-Glückwünſche auszurichten.\pend
           \selectlanguage{ngerman}\endnumbering\briefempfaengerindex{Schnitzler, Arthur@\textsc{Schnitzler, Arthur}!zzzGoldmann, Paul@\emph{von Paul Goldmann}!1897-12-303@{30. 12. [1897]}|)be}\mylabel{L02835h}  \newcommand{\dateiname}{L02835}\newcommand{\titel}{Paul Goldmann an Arthur Schnitzler, 30. 12. [1897]}\newcommand{\editorInnen}{Martin Anton Müller und Laura Untner}%% latex-leseansicht-abspann.tex
%% Abspann für die Leseansicht.
%% Der Schalter \ifkorrekturansicht ist bereits durch den Vorspann gesetzt.

%% latex-abspann.tex
%% Gemeinsamer Abspann für Korrekturansicht und Leseansicht.
%% Setzt den Schalter \ifkorrekturansicht voraus (gesetzt in den
%% einbindenden Dateien latex-korrekturansicht-abspann.tex bzw.
%% latex-leseansicht-abspann.tex).
%% ---------------------------------------------------------------

\normalsize

% Das esempio-Environment wird nur in der Leseansicht benötigt
\ifkorrekturansicht\else
\newenvironment{esempio}[3]%
{
    \vspace{1.5ex}
    \rlap{\underline{#1}}
    \par
    \setlength{\parindent}{0cm}
    \nopagebreak
    \leftskip=#2cm
    \rightskip=#3cm
}
{
    \par
}
\fi

\doendnotes{C}
\bigskip
\vfill

\clearpage

\footnotesize

\ifkorrekturansicht
  \lohead{\textsc{register}}
\fi

% theindex-Environment neu definieren ohne reledmac
\makeatletter
\renewenvironment{theindex}{%
  \ifkorrekturansicht
    \section*{\indexname}%
  \else
    \subsubsection*{Index der erwähnten Entitäten}%
  \fi
  \setlength{\parindent}{0pt}%
  \setlength{\parskip}{0pt plus 0.3pt}%
  \let\item\@idxitem
}{%
  \ifkorrekturansicht\clearpage\fi
}
\makeatother

\IfFileExists{\jobname-pw.ind}{\input{\jobname-pw.ind}}{}

% Quellenangabe nur in der Leseansicht
\ifkorrekturansicht\else
% Fallback-Definitionen, falls die .tex-Datei \titel etc. nicht gesetzt hat
\providecommand{\titel}{}
\providecommand{\editorInnen}{}
\providecommand{\dateiname}{\jobname}

\vspace{3cm}

\vfill

\footnotesize
\textsc{Quelle}: \titel. Herausgegeben von {\editorInnen}. In: \emph{Arthur Schnitzler: Briefwechsel mit Autorinnen und Autoren}.
 Digitale Edition, https://schnitzler-briefe.acdh.oeaw.ac.at/{\dateiname}.html (Stand \today)
\fi

\end{document}


