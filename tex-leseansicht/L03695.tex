%% latex-korrekturansicht-vorspann.tex
%% Vorspann für die Korrekturansicht.
%% Lädt die gemeinsame Datei latex-vorspann.tex mit gesetztem Schalter.

\newif\ifkorrekturansicht
\korrekturansichttrue

\input{../tex-inputs/latex-vorspann}


\section[Elsa Plessner an Arthur Schnitzler, 15. 5. 1897]{L03695 Elsa Plessner an Arthur Schnitzler, 15. 5. 1897}
\nopagebreak\mylabel{L03695v}
\rehead{ }\normalsize\beginnumbering\briefempfaengerindex{Schnitzler, Arthur@\textsc{Schnitzler, Arthur}!zzzPlessner, Elsa@\emph{von Elsa Plessner}!1897-05-151@{15. 5. 1897}|(be}
\toendnotes[C]{\smallbreak\pagebreak[2]}\Standort{DLA, A:Schnitzler, 85.1.4198.}
\physDesc{Brief, 1 Blatt, 3 Seiten, 1941 Zeichen
\newline{}Handschrift: , lateinische Kurrent}
\buchAbdrucke{\weitereDrucke{Hermann Bahr, Arthur Schnitzler: \emph{Briefwechsel, Aufzeichnungen, Dokumente (1891–1931)}. Göttingen: \emph{Wallstein} 2018, S. 142–143.} }\toendnotes[C]{\smallbreak}
\pstart
           {\pb}Sievring, Fröschelgasse 6\oindex{Froeschelgasse 6@\textbf{Fröschelgasse 6}, \emph{Wohngebäude (K.WHS)}|pw}, den 15. V. 97.\pend
           
\pstart\center{}Verehrter Herr Doctor!\pend\vspace{0.5em}
\pstart
           Besten Dank für Ihre liebenswürdigen Zeilen aus Paris\oindex{Paris@\textbf{Paris}, \emph{P.PPLC}|pw}, von wo ich Sie wieder \label{K_L03695-1v}\edtext{zurückgekehrt}{\lemma{\textnormal{\emph{zurückgekehrt}}}\Cendnote{\textnormal{Sie irrt sich, Schnitzler war noch in Paris\oindex{Paris@\textbf{Paris}, \emph{P.PPLC}|pwk}. Von dort reiste er nach London\oindex{London@\textbf{London}, \emph{P.PPLC}|pwk} und kehrte erst am 2. 6. 1897 von seinem Auslandsaufenthalt zurück.
                  Aus dem Schreiben Elsa Plessner an Arthur Schnitzler, 30. 5. [1897] geht hervor, dass er ihr von unterwegs auf diesen Brief antwortete.}}}\label{K_L03695-1} glaube. Da
               Ihre dortige Adresse mir – zu Ihrem Besten – unbekannt war, sparte ich mir den
               allerherzlichsten Dank für Ihre gütige \label{K_L03695-88v}\edtext{Intervention}{\lemma{\textnormal{\emph{Intervention}}}\Cendnote{\textnormal{Arthur Schnitzler an Hermann Bahr, 22. 4. 1897.}}}\label{K_L03695-88} bei H. Bahr\pwindex{Bahr, Hermann 19.07.1863 – 15.01.1934@\textsc{Bahr, Hermann} (19.07.1863 – 15.01.1934), \emph{Schriftsteller/Schriftstellerin, Kritiker/Kritikerin}|pw} – bis jetzt auf. Es ist Ihnen sicherlich schon sehr
               langweilig, dass ich mich in jedem Brief an Sie bedanke – aber wenn Sie mir immer
               Grund dazu geben? – –\pend
           
\pstart
           Ich war in großer Angst und {\pb}Aufregung, als ich von der Pariser\oindex{Paris@\textbf{Paris}, \emph{P.PPLC}|pw}{ }\label{K_L03695-2v}\edtext{Unglücksgeschichte hörte, Sie in dem
               verbrannten Gebäude}{\lemma{\textnormal{\emph{Unglücksgeschichte … Gebäude}}}\Cendnote{\textnormal{Am 4. 5. 1897 brannte der \emph{Bazar de la Charité}\orgindex{Bazar de la Charite@Bazar de la Charité|pwk}, eine Wohltätigkeitseinrichtung, ab. Dabei kamen über
                  120 Menschen ums Leben.}}}\label{K_L03695-2} fürchtend – – – na – andere Leute, die mich
               interessieren kenne ich in Paris\oindex{Paris@\textbf{Paris}, \emph{P.PPLC}|pw} nicht.
               Hoffentlich sind Sie heil und wohl wieder hier eingetroffen – – – Ich sitze – – bei
                  \label{K_L03695-3v}\edtext{\textcolor{gray}{0}° R.}{\lemma{\textnormal{\emph{0° R.}}}\Cendnote{\textnormal{Wie die
                  Celsius-Skala setzt die Réaumur-Skala den Nullwert beim Taupunkt von
                  Wasser.}}}\label{K_L03695-3} und unendlichem Regen in der »Sommerfrische« – – alle gerechten
               Menschen seien davor behütet!! Bedauern Sie mich, verehrter Herr Doctor! Ich bin
               einmal ein unglückliches Geschöpf. Schreiben thue ich \uline{jetzt}{ }\uline{gar nichts!!} – Kann nicht!! – Malheur oder
               Glück!?\pend
           
\pstart
           {\pb}Beiliegend ein kleiner Einfall! – Habe mir Mühe gegeben, nicht
               »schlampig« zu arbeiten. Ich hoffe auf Ihren Beifall! – Bin neugierig, wann und ob
               ich einmal eine Arbeit zu Ihrer rückhaltlosen Anerkennung bringen werde. »\label{K_L03695-4v}\edtext{Meine Freundin Clotilde\pwindex{Meine Freundin Clotilde@\emph{Meine Freundin Clotilde}|pw}}{\lemma{\textnormal{\emph{Meine Freundin Clotilde}}}\Cendnote{\textnormal{Erstausgabe in: \emph{Der gläserne Käfig. Skizzen
                        und Novellen}\pwindex{glaeserne Kaefig. Skizzen und Novellen@\emph{Der gläserne Käfig. Skizzen und Novellen}|pwk}. Wien\oindex{Wien@\textbf{Wien}, \emph{A.ADM2}|pwk}, Leipzig\oindex{Leipzig@\textbf{Leipzig}, \emph{P.PPLA3}|pwk}: \emph{Leopold
                        Weiss}\orgindex{Leopold Weiss@Leopold Weiss|pwk}{ }1901.}}}\label{K_L03695-4}« vermeidet alle wissentliche Affectation – – – negativer Vorzug –
               Positiv? – Bilanz!? – – Ich bin jetzt furchtbar ängstlich in der Arbeit – darum
               geringe Lust dazu! Ist ja doch Stroh!! – Außer mir hat Keiner Freude davon und in
               fünfzig Jahren? – – –. Grau – grau – aber keine Theorie – leider die Praxis!– – –
               Doch Sie kommen aus Paris\oindex{Paris@\textbf{Paris}, \emph{P.PPLC}|pw}! und haben
               wahrscheinlich keine mitschwingende Saite für die Klage aus dem Sievringer\oindex{Sievering@\textbf{Sievering}, \emph{eingemeindeter Ort (A.VOO)}|pw} Wald. – Ich brauchte ein bisschen moralisches »Paris\oindex{Paris@\textbf{Paris}, \emph{P.PPLC}|pw}«–! d. h. um- und aufgekrempelt zu werden.
               Weinen Sie, wenn Sie wollen und lachen Sie, wenn Sie können über Ihre\pend
           \pstart \spacefill\mbox{ElsaPlessner}\pend{}
\pstart
           \noindent{}P. S. Causa H. Bahr\pwindex{Bahr, Hermann 19.07.1863 – 15.01.1934@\textsc{Bahr, Hermann} (19.07.1863 – 15.01.1934), \emph{Schriftsteller/Schriftstellerin, Kritiker/Kritikerin}|pw} ist noch nicht
                  erledigt. Vielmehr »gläserner Käfig\pwindex{glaeserne Kaefig. Eine Parabel@\emph{Der gläserne Käfig. Eine Parabel}|pw}«
                  hinzugekommen. Doppelt hält besser.\pend
           \selectlanguage{ngerman}\endnumbering\briefempfaengerindex{Schnitzler, Arthur@\textsc{Schnitzler, Arthur}!zzzPlessner, Elsa@\emph{von Elsa Plessner}!1897-05-151@{15. 5. 1897}|)be}\mylabel{L03695h}
\begin{anhang}
\end{anhang}\normalsize

\doendnotes{C}
\bigskip
\vfill

\clearpage

\footnotesize

\lohead{\textsc{register}}

% Definiere theindex-Environment komplett neu ohne reledmac
\makeatletter
\renewenvironment{theindex}{%
  \section*{\indexname}%
  \setlength{\parindent}{0pt}%
  \setlength{\parskip}{0pt plus 0.3pt}%
  \let\item\@idxitem
}{%
  \clearpage
}
\makeatother

\IfFileExists{\jobname-pw.ind}{\input{\jobname-pw.ind}}{}

\end{document}

      