%% latex-leseansicht-vorspann.tex
%% Vorspann für die Leseansicht.
%% Lädt die gemeinsame Datei latex-vorspann.tex mit nicht gesetztem Schalter.

\newif\ifkorrekturansicht
\korrekturansichtfalse

\input{../tex-inputs/latex-vorspann}


\section[Elsa Plessner an Arthur Schnitzler, 15.\,5.\,1897]{L03695 Elsa Plessner an Arthur Schnitzler, 15.\,5.\,1897}
\nopagebreak\mylabel{L03695v}
\rehead{ }\normalsize\beginnumbering\briefempfaengerindex{Schnitzler, Arthur@\textsc{Schnitzler, Arthur}!zzzPlessner, Elsa@\emph{von Elsa Plessner}!1897-05-151@{15.\,5.\,1897}|(be}
\toendnotes[C]{\smallbreak\pagebreak[2]}
\correspDesc{Versand  durch Elsa Plessner am 15. 5. 1897 in Wien
\newline{}Erhalt  durch Arthur Schnitzler im Zeitraum [16. 5. 1897
                  – 20. 5. 1897?] in Paris}\toendnotes[C]{\smallbreak}
\Standort{DLA, A:Schnitzler, 85.1.4198.}
\physDesc{Brief, 1 Blatt, 3 Seiten, 1936 Zeichen
\newline{}Handschrift: schwarze Tinte, lateinische Kurrent}
\buchAbdrucke{\weitereDrucke{Hermann Bahr, Arthur Schnitzler: \emph{Briefwechsel, Aufzeichnungen, Dokumente (1891–1931)}. Herausgegeben von Kurt Ifkovits und Martin Anton Müller. Göttingen: \emph{Wallstein} 2018, S. 142–143.} }\toendnotes[C]{\smallbreak}
\pstart
           {\pb}Sievring, Fröschelgasse 6\oindex{Wien@\textbf{Wien}!XIX., Döbling@\textbf{XIX., Döbling}!Fröschelgasse 6@\textbf{Fröschelgasse 6}, \emph{Wohngebäude}|pw}, den 15. V. 97.\pend
           
\pstart\center{}Verehrter Herr Doctor!\pend\vspace{0.5em}
\pstart
           Besten Dank für Ihre \label{K_L03695-1v}\edtext{liebenswürdigen Zeilen}{\lemma{\textnormal{\emph{liebenswürdigen Zeilen}}}\Cendnote{\textnormal{nicht überliefert}}}\label{K_L03695-1} aus Paris\oindex{Paris@\textbf{Paris}, \emph{Hauptstadt}|pw}, von wo ich Sie wieder \label{K_L03695-2v}\edtext{zurückgekehrt}{\lemma{\textnormal{\emph{zurückgekehrt}}}\Cendnote{\textnormal{Sie irrt sich, Schnitzler war noch in Paris\oindex{Paris@\textbf{Paris}, \emph{Hauptstadt}|pwk}. Von dort reiste er nach London\oindex{London@\textbf{London}, \emph{Hauptstadt}|pwk} und kehrte erst am 2. 6. 1897 von seinem Auslandsaufenthalt zurück.
                  Aus dem Schreiben XXXX Auszeichnungsfehler: Dokument L03696 nicht gefunden geht hervor,
                  dass er ihr von unterwegs auf diesen Brief antwortete.}}}\label{K_L03695-2} glaube. Da Ihre
               dortige Adresse\oindex{5, rue de Maubeuge@\textbf{5, rue de Maubeuge}, \emph{Wohngebäude}|pwv} mir – zu Ihrem Besten – unbekannt war, sparte ich mir den
               allerherzlichsten Dank für Ihre gütige \label{K_L03695-3v}\edtext{Intervention}{\lemma{\textnormal{\emph{Intervention}}}\Cendnote{\textnormal{XXXX Auszeichnungsfehler: Dokument L00668 nicht gefunden.}}}\label{K_L03695-3} bei H. Bahr\pwindex{Bahr, Hermann 19.\,7.\,1863 Linz – 15.\,1.\,1934 München@\textsc{Bahr, Hermann} (19.\,7.\,1863 Linz – 15.\,1.\,1934 München), \emph{Schriftsteller, Kritiker}|pw} – bis jetzt auf. Es ist Ihnen sicherlich schon sehr
               langweilig, dass ich mich in jedem Brief an Sie bedanke – aber wenn Sie mir immer
               Grund dazu geben? – –\pend
           
\pstart
           Ich war in großer Angst und {\pb}Aufregung, als ich von der Pariser\oindex{Paris@\textbf{Paris}, \emph{Hauptstadt}|pw}{ }\label{K_L03695-4v}\edtext{Unglücksgeschichte hörte, Sie in dem
               verbrannten Gebäude}{\lemma{\textnormal{\emph{Unglücksgeschichte … Gebäude}}}\Cendnote{\textnormal{Am 4. 5. 1897 brannte der \emph{Bazar de la Charité}\orgindex{Bazar de la Charité@Bazar de la Charité|pwk}, eine Wohltätigkeitseinrichtung, ab. Dabei kamen über
                  120 Menschen ums Leben.}}}\label{K_L03695-4} fürchtend – – – na – andere Leute, die mich
               interessieren, kenne ich in Paris\oindex{Paris@\textbf{Paris}, \emph{Hauptstadt}|pw} nicht.
               Hoffentlich sind Sie heil und wohl wieder hier eingetroffen – – – Ich sitze – – bei
                  \label{K_L03695-5v}\edtext{\textcolor{gray}{0}° R.}{\lemma{\textnormal{\emph{0° R.}}}\Cendnote{\textnormal{Wie die
                  Celsius-Skala setzt die Réaumur-Skala den Nullwert beim Taupunkt von
                  Wasser.}}}\label{K_L03695-5} und unendlichem Regen in der »Sommerfrische« – – alle gerechten
               Menschen seien davor behütet!! Bedauern Sie mich, verehrter Herr Doctor! Ich bin
               einmal ein unglückliches Geschöpf. Schreiben thue ich \uline{jetzt}{ }\uline{gar nichts!!} – Kann nicht!! – Malheur oder
               Glück!?\pend
           
\pstart
           {\pb}Beiliegend ein kleiner Einfall! – Habe mir Mühe gegeben, nicht
               »schlampig« zu arbeiten. Ich hoffe auf Ihren Beifall! – Bin neugierig, wann und ob
               ich einmal eine Arbeit zu Ihrer rückhaltlosen Anerkennung bringen werde. »\label{K_L03695-6v}\edtext{Meine Freundin Clotilde\pwindex{Plessner, Elsa 22.\,8.\,1875 Wien – 7.\,5.\,1932 Alicante@\textsc{Plessner, Elsa} (22.\,8.\,1875 Wien – 7.\,5.\,1932 Alicante), \emph{Schriftstellerin}!Meine Freundin Clotilde@\strich\emph{Meine Freundin Clotilde}|pw}}{\lemma{\textnormal{\emph{Meine Freundin Clotilde}}}\Cendnote{\textnormal{Erstausgabe in: \emph{Der gläserne Käfig. Skizzen
                        und Novellen}\pwindex{Plessner, Elsa 22.\,8.\,1875 Wien – 7.\,5.\,1932 Alicante@\textsc{Plessner, Elsa} (22.\,8.\,1875 Wien – 7.\,5.\,1932 Alicante), \emph{Schriftstellerin}!gläserne Käfig. Skizzen und Novellen@\strich\emph{Der gläserne Käfig. Skizzen und Novellen}|pwk}. Wien\oindex{Wien@\textbf{Wien}, \emph{Verwaltungsgebiet}|pwk}, Leipzig\oindex{Leipzig@\textbf{Leipzig}, \emph{Hauptstadt}|pwk}: \emph{Leopold
                        Weiss}\orgindex{Leopold Weiss@Leopold Weiss|pwk}{ }1901.}}}\label{K_L03695-6}« vermeidet alle wissentliche Affectation – – – negativer Vorzug –
               Positiv? – Bilanz!? – – Ich bin jetzt furchtbar ängstlich in der Arbeit – darum
               geringe Lust dazu! Ist ja doch Stroh!! – Außer mir hat Keiner Freude davon und in
               fünfzig Jahren? – – –. Grau – grau – aber keine Theorie – leider die Praxis! – –
               Doch Sie kommen aus Paris\oindex{Paris@\textbf{Paris}, \emph{Hauptstadt}|pw}! und haben
               wahrscheinlich keine mitschwingende Saite für die Klage aus dem Sievringer\oindex{Wien@\textbf{Wien}!XIX., Döbling@\textbf{XIX., Döbling}!Sievering@\textbf{Sievering}|pw} Wald. – Ich brauchte ein bisschen moralisches »Paris\oindex{Paris@\textbf{Paris}, \emph{Hauptstadt}|pw}«–! d. h. um- und aufgekrempelt zu werden.
               Weinen Sie, wenn Sie wollen und lachen Sie, wenn Sie können über Ihre\pend
           \pstart \spacefill\mbox{ElsaPlessner.}\pend{}
\pstart
           \noindent{}P. S. Causa H. Bahr\pwindex{Bahr, Hermann 19.\,7.\,1863 Linz – 15.\,1.\,1934 München@\textsc{Bahr, Hermann} (19.\,7.\,1863 Linz – 15.\,1.\,1934 München), \emph{Schriftsteller, Kritiker}|pw} ist noch nicht
                  erledigt. Vielmehr »gläserner Käfig\pwindex{Plessner, Elsa 22.\,8.\,1875 Wien – 7.\,5.\,1932 Alicante@\textsc{Plessner, Elsa} (22.\,8.\,1875 Wien – 7.\,5.\,1932 Alicante), \emph{Schriftstellerin}!gläserne Käfig. Eine Parabel@\strich\emph{Der gläserne Käfig. Eine Parabel}|pw}«
                  hinzugekommen. Doppelt hält besser.\pend
           \selectlanguage{ngerman}\endnumbering\briefempfaengerindex{Schnitzler, Arthur@\textsc{Schnitzler, Arthur}!zzzPlessner, Elsa@\emph{von Elsa Plessner}!1897-05-151@{15.\,5.\,1897}|)be}\mylabel{L03695h}  \newcommand{\dateiname}{L03695}\newcommand{\titel}{Elsa Plessner an Arthur Schnitzler, 15. 5. 1897}\newcommand{\editorInnen}{Kurt Ifkovits, Selma Jahnke und Martin Anton Müller}%% latex-leseansicht-abspann.tex
%% Abspann für die Leseansicht.
%% Der Schalter \ifkorrekturansicht ist bereits durch den Vorspann gesetzt.

%% latex-abspann.tex
%% Gemeinsamer Abspann für Korrekturansicht und Leseansicht.
%% Setzt den Schalter \ifkorrekturansicht voraus (gesetzt in den
%% einbindenden Dateien latex-korrekturansicht-abspann.tex bzw.
%% latex-leseansicht-abspann.tex).
%% ---------------------------------------------------------------

\normalsize

% Das esempio-Environment wird nur in der Leseansicht benötigt
\ifkorrekturansicht\else
\newenvironment{esempio}[3]%
{
    \vspace{1.5ex}
    \rlap{\underline{#1}}
    \par
    \setlength{\parindent}{0cm}
    \nopagebreak
    \leftskip=#2cm
    \rightskip=#3cm
}
{
    \par
}
\fi

\doendnotes{C}
\bigskip
\vfill

\clearpage

\footnotesize

\ifkorrekturansicht
  \lohead{\textsc{register}}
\fi

% theindex-Environment neu definieren ohne reledmac
\makeatletter
\renewenvironment{theindex}{%
  \ifkorrekturansicht
    \section*{\indexname}%
  \else
    \subsubsection*{Index der erwähnten Entitäten}%
  \fi
  \setlength{\parindent}{0pt}%
  \setlength{\parskip}{0pt plus 0.3pt}%
  \let\item\@idxitem
}{%
  \ifkorrekturansicht\clearpage\fi
}
\makeatother

\IfFileExists{\jobname-pw.ind}{\input{\jobname-pw.ind}}{}

% Quellenangabe nur in der Leseansicht
\ifkorrekturansicht\else
% Fallback-Definitionen, falls die .tex-Datei \titel etc. nicht gesetzt hat
\providecommand{\titel}{}
\providecommand{\editorInnen}{}
\providecommand{\dateiname}{\jobname}

\vspace{3cm}

\vfill

\footnotesize
\textsc{Quelle}: \titel. Herausgegeben von {\editorInnen}. In: \emph{Arthur Schnitzler: Briefwechsel mit Autorinnen und Autoren}.
 Digitale Edition, https://schnitzler-briefe.acdh.oeaw.ac.at/{\dateiname}.html (Stand \today)
\fi

\end{document}


