%% latex-leseansicht-vorspann.tex
%% Vorspann für die Leseansicht.
%% Lädt die gemeinsame Datei latex-vorspann.tex mit nicht gesetztem Schalter.

\newif\ifkorrekturansicht
\korrekturansichtfalse

\input{../tex-inputs/latex-vorspann}

\begin{center}
            \textcolor{red}{ENTWURF. ENTZIFFERUNG NOCH NICHT KORREKTURGELESEN}
                      \end{center}
            
               \section[Bertha von Suttner an Arthur und Olga Schnitzler, 30. 3. 1914]{ Bertha von Suttner an Arthur und Olga Schnitzler,
                    30. 3. 1914}\nopagebreak\mylabel{v}\rehead{ }\begin{ledgroupsized}[t]{13cm}\normalsize\beginnumbering\briefempfaengerindex{Schnitzler, Olga@\textsc{Schnitzler, Olga}!zzzSuttner, Bertha von@\emph{von Bertha von Suttner}!1914-03-302@{30. 3. 1914}|(be}\briefempfaengerindex{Schnitzler, Arthur@\textsc{Schnitzler, Arthur}!zzzSuttner, Bertha von@\emph{von Bertha von Suttner}!1914-03-302@{30. 3. 1914}|(be} \toendnotes[C]{\smallbreak\pagebreak[2]} \Standort{DLA, A:Schnitzler, HS.NZ66.198.}
\physDesc{Brief, 1 Blatt (mit Krone in Golddruck), 1 Seite
\newline{}Handschrift: schwarze Tinte, deutsche Kurrent
\newline{}Schnitzler: 1) mit Bleistift beschriftet: »\textsc{Suttner}« 2) mit rotem Buntstift eine Unterstreichung}\Standort{DLA, A:Schnitzler, HS.NZ85.1.4773.}
\physDesc{1 Blatt, 1 Seite, maschinelle Abschrift}\toendnotes[C]{\smallbreak}\pstart
           \raggedleft{}{\pb}30/III 1914\pend
           \pstart\center{}Geehrter Dichter und liebe Dichtersgattin\pend\pstart
           Das war mir u. noch jemand anders eine herbe Enttäuſchung geſtern: zuerſt zu- und
                    dann abgeſagt! Das müſſen Sie wieder gutmachen. Eine Dame kam \uline{nur}, weil ſie ſich ſo ſehr auf Ihr in Ausſicht
                    geſtelltes Erſcheinen \strikeout{ſo} freute. Und ſie nahm
                    mir das Verſprechen ab ſie bei der nächſten Gelegenheit wieder zu rufen. Es iſt
                    die \label{K_L02170_1v}\edtext{Pr.}{\lemma{\textnormal{\emph{Pr.}}}\Cendnote{\textnormal{Prinzessin}}}\label{K_L02170_1h}{ }\textsc{Lothar Metternich}\pwindex{Metternich-Winneburg, Karoline Franziska von 1846-11-08 – 1918-03-19@\textsc{Metternich-Winneburg, Karoline Franziska von} (1846-11-08 – 1918-03-19), \emph{Prinzessin}|pw}
                    (Schwägerin der Fürſtin \textsc{Pauline}\pwindex{Metternich-Sándor, Pauline von 26.02.1836 – 28.09.1921@\textsc{Metternich-Sándor, Pauline von} (26.02.1836 – 28.09.1921)|pw}). Die wäre glücklich, mit Ihnen zuſammenzukommen.
                    Alſo bitte: beſtimmen Sie einen der 3 Tage dieſer Woche: Donnerſtag, Freitag
                    oder Samſtag – und {\pb}ich arrangiere einen ganz
                    intimen kleinen Nachmittags-Gedankenaustauſch nur Sie beide, meine Freundin \textsc{Metternich}\pwindex{Metternich-Winneburg, Karoline Franziska von 1846-11-08 – 1918-03-19@\textsc{Metternich-Winneburg, Karoline Franziska von} (1846-11-08 – 1918-03-19), \emph{Prinzessin}|pw} und
                    höchſtens noch zwei drei Perſonen (5 Uhr)\pend
           \pstart
           Einer lieben Antwort gewertig{\\[\baselineskip]}\spacefill\mbox{Bertha Suttner}\pend
           \leftskip=0em{}\endnumbering\briefempfaengerindex{Schnitzler, Olga@\textsc{Schnitzler, Olga}!zzzSuttner, Bertha von@\emph{von Bertha von Suttner}!1914-03-302@{30. 3. 1914}|)be}\briefempfaengerindex{Schnitzler, Arthur@\textsc{Schnitzler, Arthur}!zzzSuttner, Bertha von@\emph{von Bertha von Suttner}!1914-03-302@{30. 3. 1914}|)be}\mylabel{h}\end{ledgroupsized}  \newcommand{\dateiname}{L02170}\newcommand{\titel}{Bertha von Suttner an Arthur und Olga Schnitzler, 30. 3. 1914}\newcommand{\editorInnen}{Martin Anton Müller und Gerd-Hermann Susen}%% latex-leseansicht-abspann.tex
%% Abspann für die Leseansicht.
%% Der Schalter \ifkorrekturansicht ist bereits durch den Vorspann gesetzt.

%% latex-abspann.tex
%% Gemeinsamer Abspann für Korrekturansicht und Leseansicht.
%% Setzt den Schalter \ifkorrekturansicht voraus (gesetzt in den
%% einbindenden Dateien latex-korrekturansicht-abspann.tex bzw.
%% latex-leseansicht-abspann.tex).
%% ---------------------------------------------------------------

\normalsize

% Das esempio-Environment wird nur in der Leseansicht benötigt
\ifkorrekturansicht\else
\newenvironment{esempio}[3]%
{
    \vspace{1.5ex}
    \rlap{\underline{#1}}
    \par
    \setlength{\parindent}{0cm}
    \nopagebreak
    \leftskip=#2cm
    \rightskip=#3cm
}
{
    \par
}
\fi

\doendnotes{C}
\bigskip
\vfill

\clearpage

\footnotesize

\ifkorrekturansicht
  \lohead{\textsc{register}}
\fi

% theindex-Environment neu definieren ohne reledmac
\makeatletter
\renewenvironment{theindex}{%
  \ifkorrekturansicht
    \section*{\indexname}%
  \else
    \subsubsection*{Index der erwähnten Entitäten}%
  \fi
  \setlength{\parindent}{0pt}%
  \setlength{\parskip}{0pt plus 0.3pt}%
  \let\item\@idxitem
}{%
  \ifkorrekturansicht\clearpage\fi
}
\makeatother

\IfFileExists{\jobname-pw.ind}{\input{\jobname-pw.ind}}{}

% Quellenangabe nur in der Leseansicht
\ifkorrekturansicht\else
% Fallback-Definitionen, falls die .tex-Datei \titel etc. nicht gesetzt hat
\providecommand{\titel}{}
\providecommand{\editorInnen}{}
\providecommand{\dateiname}{\jobname}

\vspace{3cm}

\vfill

\footnotesize
\textsc{Quelle}: \titel. Herausgegeben von {\editorInnen}. In: \emph{Arthur Schnitzler: Briefwechsel mit Autorinnen und Autoren}.
 Digitale Edition, https://schnitzler-briefe.acdh.oeaw.ac.at/{\dateiname}.html (Stand \today)
\fi

\end{document}


      