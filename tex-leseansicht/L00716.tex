\input{../tex-inputs/latex-pdf-vorspann}
\begin{center}
            \textcolor{red}{ENTWURF. ENTZIFFERUNG NOCH NICHT KORREKTURGELESEN}
                      \end{center}
            
               \section[Arthur Schnitzler an Richard Beer-Hofmann, 14. 8. 1897]{ Arthur Schnitzler an Richard Beer-Hofmann, 14. 8. 1897}\nopagebreak\mylabel{v}\rehead{ }\begin{ledgroupsized}[t]{13cm}\normalsize\beginnumbering\briefempfaengerindex{Beer-Hofmann, Richard@\textsc{Beer-Hofmann, Richard}!zzzSchnitzler, Arthur@\emph{von Arthur Schnitzler}!1897-08-142@{14. 8. 1897}|(be} \toendnotes[C]{\smallbreak\pagebreak[2]} \Standort{YCGL, MSS 31.}
\physDesc{Brief, 1 Blatt, 1 Seite, Umschlag
\newline{}Handschrift: Bleistift, deutsche Kurrent\newline{}Versand: ohne postalischen Übermittlungsvermerk }\toendnotes[C]{\smallbreak}\pstart{}{\pb}Herrn Doctor\pend{}\pstart{}Richard Beer-Hofmann\pend{}\pstart{}Wien\oindex{Wien@\textbf{Wien}|pw}\pend{}\pstart{}\textsc{VIII. Hotel Ha{\geminationm}erand\oindex{Hotel Hammerand@\textbf{Hotel Hammerand}|pw} (Schlösselgasse\oindex{Schloesselgasse@\textbf{Schlösselgasse}|pw})}\pend{}{\bigskip}\pstart
           {\pb}14/8 97\pend
           \pstart{}Lieber Richard,\pend\pstart
           Eben, 2 Uhr N. M. ko{\geminationm}t dieſes \label{K_L00716-1v}\edtext{Telegr.}{\lemma{\textnormal{\emph{Telegr.}}}\Cendnote{\textnormal{Paul Goldmann an Arthur Schnitzler, 14. 8. 1897}}}\label{K_L00716-1h} –\pend
           \pstart
           Auf Wiederſehn heut Abend. Bitte, nicht ſpät.\pend
           \pstart Herzlich Ihr \spacefill\mbox{Arth}\pend{}\endnumbering\briefempfaengerindex{Beer-Hofmann, Richard@\textsc{Beer-Hofmann, Richard}!zzzSchnitzler, Arthur@\emph{von Arthur Schnitzler}!1897-08-142@{14. 8. 1897}|)be}\mylabel{h}\end{ledgroupsized}  \newcommand{\dateiname}{L00716}\newcommand{\titel}{Arthur Schnitzler an Richard Beer-Hofmann, 14. 8. 1897}\newcommand{\editorInnen}{ Martin Anton Müller und Gerd-Hermann Susen}\input{../tex-inputs/latex-pdf-abspann}
      