%% latex-korrekturansicht-vorspann.tex
%% Vorspann für die Korrekturansicht.
%% Lädt die gemeinsame Datei latex-vorspann.tex mit gesetztem Schalter.

\newif\ifkorrekturansicht
\korrekturansichttrue

\input{../tex-inputs/latex-vorspann}


\section[Arthur Schnitzler an Richard Beer-Hofmann, 14. 8. 1897]{L00716 Arthur Schnitzler an Richard Beer-Hofmann, 14. 8. 1897}
\nopagebreak\mylabel{L00716v}
\rehead{ }\normalsize\beginnumbering\briefempfaengerindex{Beer-Hofmann, Richard@\textsc{Beer-Hofmann, Richard}!zzzSchnitzler, Arthur@\emph{von Arthur Schnitzler}!1897-08-142@{14. 8. 1897}|(be}
\toendnotes[C]{\smallbreak\pagebreak[2]}\Standort{YCGL, MSS 31.}
\physDesc{Brief, 1 Blatt, 1 Seite, Umschlag, 191 Zeichen
\newline{}Handschrift: Bleistift, deutsche Kurrent
\newline{}Versand: ohne postalischen Übermittlungsvermerk }\toendnotes[C]{\smallbreak}\pstart{}{\pb}Herrn Doctor\pend{}\pstart{}Richard Beer-Hofmann\pend{}\pstart{}Wien\oindex{Wien@\textbf{Wien}, \emph{A.ADM2}|pw}\pend{}\pstart{}\textsc{VIII. Hotel Ha{\geminationm}erand\oindex{Hotel Hammerand@\textbf{Hotel Hammerand}, \emph{Hotel (K.HTL)}|pw} (Schlösselgasse\oindex{Schloesselgasse@\textbf{Schlösselgasse}, \emph{Straße (K.STR)}|pw})}\pend{}{\bigskip}\vspace{1em}
\pstart
           {\pb}14/8 97\pend
           
\pstart{}Lieber Richard,\pend\vspace{0.5em}
\pstart
           Eben, 2 Uhr N. M. ko{\geminationm}t dieſes \label{K_L00716-1v}\edtext{Telegr.}{\lemma{\textnormal{\emph{Telegr.}}}\Cendnote{\textnormal{Paul Goldmann an Arthur Schnitzler, 14. 8. 1897.
               }}}\label{K_L00716-1} –\pend
           
\pstart
           Auf Wiederſehn heut Abend. Bitte, nicht ſpät.\pend
           \pstart Herzlich Ihr \spacefill\mbox{Arth}\pend{}\selectlanguage{ngerman}\endnumbering\briefempfaengerindex{Beer-Hofmann, Richard@\textsc{Beer-Hofmann, Richard}!zzzSchnitzler, Arthur@\emph{von Arthur Schnitzler}!1897-08-142@{14. 8. 1897}|)be}\mylabel{L00716h}  \normalsize

\doendnotes{C}
\bigskip
\vfill

\clearpage

\footnotesize

\lohead{\textsc{register}}

% Definiere theindex-Environment komplett neu ohne reledmac
\makeatletter
\renewenvironment{theindex}{%
  \section*{\indexname}%
  \setlength{\parindent}{0pt}%
  \setlength{\parskip}{0pt plus 0.3pt}%
  \let\item\@idxitem
}{%
  \clearpage
}
\makeatother

\IfFileExists{\jobname-pw.ind}{\input{\jobname-pw.ind}}{}

\end{document}

      