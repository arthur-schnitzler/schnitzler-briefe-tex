%% latex-korrekturansicht-vorspann.tex
%% Vorspann für die Korrekturansicht.
%% Lädt die gemeinsame Datei latex-vorspann.tex mit gesetztem Schalter.

\newif\ifkorrekturansicht
\korrekturansichttrue

\input{../tex-inputs/latex-vorspann}


\section[Arthur Schnitzler an Hermann Bahr, 7. 3. 1899]{L00899 Arthur Schnitzler an Hermann Bahr, 7. 3. 1899}
\nopagebreak\mylabel{L00899v}
\rehead{ }\normalsize\beginnumbering\briefempfaengerindex{Bahr, Hermann@\textsc{Bahr, Hermann}!zzzSchnitzler, Arthur@\emph{von Arthur Schnitzler}!1899-03-071@{7. 3. 1899}|(be}
\toendnotes[C]{\smallbreak\pagebreak[2]}\Standort{TMW, HS AM 23335 Ba.}
\physDesc{Brief, 1 Blatt, 2 Seiten, 611 Zeichen
\newline{}Handschrift: schwarze Tinte, deutsche Kurrent
\newline{}Ordnung: Lochung }
\buchAbdrucke{\weitereDrucke{1) Arthur Schnitzler: \emph{The Letters of Arthur Schnitzler to Hermann Bahr}. Chapel Hill: \emph{The University of North Carolina Press} 1978, S. 65–66.} \weitereDrucke{2) Hermann Bahr, Arthur Schnitzler: \emph{Briefwechsel, Aufzeichnungen, Dokumente (1891–1931)}. Göttingen: \emph{Wallstein} 2018, S. 169.} }\toendnotes[C]{\smallbreak}
\pstart{}{\pb}Lieber
                  Bahr,\pend\vspace{0.5em}
\pstart
           als meine 3
                  Einakter\pwindex{Paracelsus. Versspiel in einem Akt@\emph{Paracelsus. Versspiel in einem Akt}|pwv}\pwindex{Gefaehrtin. Schauspiel in einem Akt@\emph{Die Gefährtin. Schauspiel in einem Akt}|pwv}\pwindex{gruene Kakadu. Groteske in einem Akt@\emph{Der grüne Kakadu. Groteske in einem Akt}|pwv} angekündigt wurden wünſchteſt du einen davon. Ich \label{K_L00899-1v}\edtext{verſprach dir bald darauf die »Gefährtin\pwindex{Gefaehrtin. Schauspiel in einem Akt@\emph{Die Gefährtin. Schauspiel in einem Akt}|pw}«, du nahmſt an}{\lemma{\textnormal{\emph{verſprach … an}}}\Cendnote{\textnormal{Arthur Schnitzler an Hermann Bahr, 1. 12. 1898.
               }}}\label{K_L00899-1}. Du fragteſt wieder; ich ſagte dir das \textsc{Manuscript}
               nach der Aufführg zu. Damit band ich mich und beantwortete Aufforderungen von andrer
               Seite \label{K_L00899-2v}\edtext{abſchlägig}{\lemma{\textnormal{\emph{abſchlägig}}}\Cendnote{\textnormal{Es erschien, nach der Absage Bahrs\pwindex{Bahr, Hermann 19.07.1863 – 15.01.1934@\textsc{Bahr, Hermann} (19.07.1863 – 15.01.1934), \emph{Schriftsteller/Schriftstellerin, Kritiker/Kritikerin}|pwk}, in keinem anderen Organ.}}}\label{K_L00899-2}. Nun
               ſteckſt du plötzlich »ſo tief in alten Verpflichtungen«, daſs du das Stück {\pb}nicht bringen kannſt. –
                  Tr\damage{otz}dem Du durch den Aufſchub der Sobeïde\pwindex{Hochzeit der Sobeide@\emph{Die Hochzeit der Sobeide}|pw} 2
               oder 3 Nummern freibekommen haſt! – \pend
           
\pstart
           Dieſer Sachverhalt ſei hiemit conſtatirt. Jede weitere Discuſſion darüber lehne ich
               ab.\pend
           
\pstart
           Besten Gruſs. Dein ergebner{\\[\baselineskip]}\spacefill\mbox{Arthur Schnitzler}\pend
           \leftskip=0em{}
\pstart
           Wien\oindex{Wien@\textbf{Wien}, \emph{A.ADM2}|pw}{ }7. 3. 99.\pend
           \selectlanguage{ngerman}\endnumbering\briefempfaengerindex{Bahr, Hermann@\textsc{Bahr, Hermann}!zzzSchnitzler, Arthur@\emph{von Arthur Schnitzler}!1899-03-071@{7. 3. 1899}|)be}\mylabel{L00899h}  \normalsize

\doendnotes{C}
\bigskip
\vfill

\clearpage

\footnotesize

\lohead{\textsc{register}}

% Definiere theindex-Environment komplett neu ohne reledmac
\makeatletter
\renewenvironment{theindex}{%
  \section*{\indexname}%
  \setlength{\parindent}{0pt}%
  \setlength{\parskip}{0pt plus 0.3pt}%
  \let\item\@idxitem
}{%
  \clearpage
}
\makeatother

\IfFileExists{\jobname-pw.ind}{\input{\jobname-pw.ind}}{}

\end{document}

      