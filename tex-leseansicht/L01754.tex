%% latex-leseansicht-vorspann.tex
%% Vorspann für die Leseansicht.
%% Lädt die gemeinsame Datei latex-vorspann.tex mit nicht gesetztem Schalter.

\newif\ifkorrekturansicht
\korrekturansichtfalse

\input{../tex-inputs/latex-vorspann}


         
         \renewcommand{\erwaehntePersonen}{Personen: Kete Fox, Albert Claughten Fox}
         \renewcommand{\erwaehnteInstitutionen}{Institutionen: Franz-Grillparzer-Preis}
         \renewcommand{\erwaehnteOrte}{Orte: Addison Street, England, Kleines Theater, New Jersey, New York City, Schwindgasse, Vereinigte Staaten von Amerika (USA), Wien}
         \renewcommand{\erwaehnteWerke}{Werke: Der Schleier der Beatrice. Schauspiel in fünf Akten, Nachtasyl. Szenen aus der Tiefe in vier Aufzügen, Rausch, Salomé. Drame en une acte}
               \section[Karl Kraus an Arthur Schnitzler, 20. 1. 1908]{ Karl Kraus an Arthur Schnitzler, 20. 1. 1908}\nopagebreak\mylabel{v}\rehead{ }\begin{ledgroupsized}[t]{13cm}\normalsize\beginnumbering \toendnotes[C]{\smallbreak\pagebreak[2]} \Standort{DLA, A:Schnitzler, HS.NZ85.1.5731.}
\physDesc{Brief, 1 Blatt, 1 Seite, 635 Zeichen
\newline{}Handschrift: schwarze Tinte, deutsche Kurrent
\newline{}Schnitzler: mit Bleistift beschriftet: »\textsc{Carl Kraus}« und abgehakt, womöglich als Zeichen, dass es
                                 abgeschrieben wurde }\buchAbdrucke{\weitereDrucke{\emph{Karl Kraus und Arthur Schnitzler. Eine Dokumentation.} Hg. Reinhard Urbach. In: \emph{Literatur und Kritik}, Bd. 49, Oktober 1970, S. 522.} }\pstart
           \raggedleft{}{\pb}Wien\oindex{Wien@\textbf{Wien}|pw}{ }20. 1. 08\pend
           \pstart
           Eine in New York\oindex{New York City@\textbf{New York City}|pw} lebende Freundin, Mrs. Fox\pwindex{Fox, Kete 13.09.1880 – 21.06.1960@\textsc{Fox, Kete} (13.09.1880 – 21.06.1960), \emph{Schauspielerin}|pw} – die als Kete Parsenow\pwindex{Fox, Kete 13.09.1880 – 21.06.1960@\textsc{Fox, Kete} (13.09.1880 – 21.06.1960), \emph{Schauspielerin}|pw} vor einigen Jahren im Berliner
                  Kleinen Theater\oindex{Kleines Theater@\textbf{Kleines Theater}|pw}{ }Salome\pwindex{\textcolor{red}{\textsuperscript{XXXX1 indx}}!Salome. Drame en une acte1893@\strich\emph{Salomé. Drame en une acte} {[}1893{]}|pw}, in »Rausch\pwindex{\textcolor{red}{\textsuperscript{XXXX1 indx}}!Rausch1900@\strich\emph{Rausch} {[}1900{]}|pw}«, »Nachtasyl\pwindex{\textcolor{red}{\textsuperscript{XXXX1 indx}}!Nachtasyl. Szenen aus der Tiefe in vier Aufzuegen1902-12-31@\strich\emph{Nachtasyl. Szenen aus der Tiefe in vier Aufzügen} {[}1902-12-31{]}|pw}« etc. geſpielt
               hat –, erſucht mich Sie zu fragen, ob Sie geneigt wären, ihr das Recht der engliſchen\oindex{England@\textbf{England}|pw} Überſetzung und Aufführung Ihres »Schleiers der Beatrice\pwindex{Schnitzler, Arthur 15.05.1862 – 21.10.1931@\textsc{Schnitzler, Arthur} (15.05.1862 – 21.10.1931), \emph{Schriftsteller, Mediziner}!Schleier der Beatrice. Schauspiel in fuenf Akten1900-12-01@\strich\emph{Der Schleier der Beatrice. Schauspiel in fünf Akten} {[}1900-12-01{]}|pw}« zu erteilen. Für einen
               freundlichen Beſcheid an meine oder die Adreſſe: Mrs. A. C. Fox\pwindex{Fox, Albert Claughten 22.12.1879 – 20.02.1952@\textsc{Fox, Albert Claughten} (22.12.1879 – 20.02.1952), \emph{Händler}|pw}\pwindex{Fox, Kete 13.09.1880 – 21.06.1960@\textsc{Fox, Kete} (13.09.1880 – 21.06.1960), \emph{Schauspielerin}|pw}, New-Yersey\oindex{New Jersey@\textbf{New Jersey}|pw}{ }U.S.A.\oindex{Vereinigte Staaten von Amerika (USA)@\textbf{Vereinigte Staaten von Amerika (USA)}|pw}{ }Addison Street\oindex{Addison Street@\textbf{Addison Street}|pw}, wäre ich Ihnen sehr
               verbunden.\pend
           \pstart
           Ich geſtatte mir bei dieſer Gelegenheit, Sie zum Grillparzer-Preis\orgindex{Franz-Grillparzer-Preis@Franz-Grillparzer-Preis|pw} zu beglückwünſchen, und bin mit hochachtungsvollem
               Gruß\pend
           \pstart
           Ihr ganz ergebener{\\[\baselineskip]}\spacefill\mbox{Karl Kraus}\pend
           \leftskip=0em{}\pstart
           \noindent{}\raggedleft{}Wien IV. Schwindg. 3, Th. 3\oindex{Schwindgasse@\textbf{Schwindgasse}|pw}\pend
           
         
         \endnumbering\mylabel{h}\end{ledgroupsized}  \newcommand{\dateiname}{L01754}\newcommand{\titel}{Karl Kraus an Arthur Schnitzler, 20. 1. 1908}\newcommand{\editorInnen}{Martin Anton Müller und Gerd-Hermann Susen}%% latex-leseansicht-abspann.tex
%% Abspann für die Leseansicht.
%% Der Schalter \ifkorrekturansicht ist bereits durch den Vorspann gesetzt.

%% latex-abspann.tex
%% Gemeinsamer Abspann für Korrekturansicht und Leseansicht.
%% Setzt den Schalter \ifkorrekturansicht voraus (gesetzt in den
%% einbindenden Dateien latex-korrekturansicht-abspann.tex bzw.
%% latex-leseansicht-abspann.tex).
%% ---------------------------------------------------------------

\normalsize

% Das esempio-Environment wird nur in der Leseansicht benötigt
\ifkorrekturansicht\else
\newenvironment{esempio}[3]%
{
    \vspace{1.5ex}
    \rlap{\underline{#1}}
    \par
    \setlength{\parindent}{0cm}
    \nopagebreak
    \leftskip=#2cm
    \rightskip=#3cm
}
{
    \par
}
\fi

\doendnotes{C}
\bigskip
\vfill

\clearpage

\footnotesize

\ifkorrekturansicht
  \lohead{\textsc{register}}
\fi

% theindex-Environment neu definieren ohne reledmac
\makeatletter
\renewenvironment{theindex}{%
  \ifkorrekturansicht
    \section*{\indexname}%
  \else
    \subsubsection*{Index der erwähnten Entitäten}%
  \fi
  \setlength{\parindent}{0pt}%
  \setlength{\parskip}{0pt plus 0.3pt}%
  \let\item\@idxitem
}{%
  \ifkorrekturansicht\clearpage\fi
}
\makeatother

\IfFileExists{\jobname-pw.ind}{\input{\jobname-pw.ind}}{}

% Quellenangabe nur in der Leseansicht
\ifkorrekturansicht\else
% Fallback-Definitionen, falls die .tex-Datei \titel etc. nicht gesetzt hat
\providecommand{\titel}{}
\providecommand{\editorInnen}{}
\providecommand{\dateiname}{\jobname}

\vspace{3cm}

\vfill

\footnotesize
\textsc{Quelle}: \titel. Herausgegeben von {\editorInnen}. In: \emph{Arthur Schnitzler: Briefwechsel mit Autorinnen und Autoren}.
 Digitale Edition, https://schnitzler-briefe.acdh.oeaw.ac.at/{\dateiname}.html (Stand \today)
\fi

\end{document}


      