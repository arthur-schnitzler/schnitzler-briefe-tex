%% latex-korrekturansicht-vorspann.tex
%% Vorspann für die Korrekturansicht.
%% Lädt die gemeinsame Datei latex-vorspann.tex mit gesetztem Schalter.

\newif\ifkorrekturansicht
\korrekturansichttrue

\input{../tex-inputs/latex-vorspann}


\section[Karl Kraus an Arthur Schnitzler, 20. 1. 1908]{L01754 Karl Kraus an Arthur Schnitzler, 20. 1. 1908}
\nopagebreak\mylabel{L01754v}
\rehead{ }\normalsize\beginnumbering\briefempfaengerindex{Schnitzler, Arthur@\textsc{Schnitzler, Arthur}!zzzKraus, Karl@\emph{von Karl Kraus}!1908-01-201@{20. 1. 1908}|(be}
\toendnotes[C]{\smallbreak\pagebreak[2]}\Standort{DLA, A:Schnitzler, HS.NZ85.1.5731.}
\physDesc{Brief, 1 Blatt, 1 Seite, 635 Zeichen
\newline{}Handschrift: schwarze Tinte, deutsche Kurrent
\newline{}Schnitzler: mit Bleistift beschriftet: »\textsc{Carl Kraus}« und abgehakt, womöglich als Zeichen, dass es
                                 abgeschrieben wurde }
\buchAbdrucke{\weitereDrucke{\emph{Literatur und Kritik}, Bd. 49, Oktober 1970, S. 522.} }
\pstart
           \raggedleft{}{\pb}Wien\oindex{Wien@\textbf{Wien}, \emph{A.ADM2}|pw}{ }20. 1. 08\pend
           \vspace{0.5em}
\pstart
           Eine in New York\oindex{New York City@\textbf{New York City}, \emph{P.PPL}|pw} lebende Freundin, Mrs. Fox\pwindex{Fox, Kete 13.09.1880 – 21.06.1960@\textsc{Fox, Kete} (13.09.1880 – 21.06.1960), \emph{Schauspieler/Schauspielerin}|pw} – die als Kete Parsenow\pwindex{Fox, Kete 13.09.1880 – 21.06.1960@\textsc{Fox, Kete} (13.09.1880 – 21.06.1960), \emph{Schauspieler/Schauspielerin}|pw} vor einigen Jahren im Berliner
                  Kleinen Theater\oindex{Kleines Theater@\textbf{Kleines Theater}, \emph{Theater (K.THE)}|pw}{ }Salome\pwindex{Salome. Drame en une acte@\emph{Salomé. Drame en une acte}|pw}, in »Rausch\pwindex{Rausch@\emph{Rausch}|pw}«, »Nachtasyl\pwindex{Nachtasyl. Szenen aus der Tiefe in vier Aufzuegen@\emph{Nachtasyl. Szenen aus der Tiefe in vier Aufzügen}|pw}« etc. geſpielt
               hat –, erſucht mich Sie zu fragen, ob Sie geneigt wären, ihr das Recht der engliſchen\oindex{England@\textbf{England}, \emph{A.ADM1}|pw} Überſetzung und Aufführung Ihres »Schleiers der Beatrice\pwindex{Schleier der Beatrice. Schauspiel in fuenf Akten@\emph{Der Schleier der Beatrice. Schauspiel in fünf Akten}|pw}« zu erteilen. Für einen
               freundlichen Beſcheid an meine oder die Adreſſe: Mrs. A. C. Fox\pwindex{Fox, Albert Claughten 22.12.1879 – 20.02.1952@\textsc{Fox, Albert Claughten} (22.12.1879 – 20.02.1952), \emph{Importhändler/Importhändlerin}|pw}\pwindex{Fox, Kete 13.09.1880 – 21.06.1960@\textsc{Fox, Kete} (13.09.1880 – 21.06.1960), \emph{Schauspieler/Schauspielerin}|pw}, New-Yersey\oindex{New Jersey@\textbf{New Jersey}, \emph{A.ADM1}|pw}{ }U.S.A.\oindex{Vereinigte Staaten von Amerika [USA]@\textbf{Vereinigte Staaten von Amerika [USA]}, \emph{A.PCLI}|pw}{ }Addison Street\oindex{Addison Street@\textbf{Addison Street}, \emph{Straße (K.STR)}|pw}, wäre ich Ihnen sehr
               verbunden.\pend
           
\pstart
           Ich geſtatte mir bei dieſer Gelegenheit, Sie zum Grillparzer-Preis\orgindex{Franz-Grillparzer-Preis@Franz-Grillparzer-Preis|pw} zu beglückwünſchen, und bin mit hochachtungsvollem
               Gruß\pend
           
\pstart
           Ihr ganz ergebener{\\[\baselineskip]}\spacefill\mbox{Karl Kraus}\pend
           \leftskip=0em{}
\pstart
           \noindent{}\raggedleft{}Wien IV. Schwindg. 3, Th. 3\oindex{Schwindgasse@\textbf{Schwindgasse}, \emph{Straße (K.STR)}|pw}\pend
           \selectlanguage{ngerman}\endnumbering\briefempfaengerindex{Schnitzler, Arthur@\textsc{Schnitzler, Arthur}!zzzKraus, Karl@\emph{von Karl Kraus}!1908-01-201@{20. 1. 1908}|)be}\mylabel{L01754h}  \normalsize

\doendnotes{C}
\bigskip
\vfill

\clearpage

\footnotesize

\lohead{\textsc{register}}

% Definiere theindex-Environment komplett neu ohne reledmac
\makeatletter
\renewenvironment{theindex}{%
  \section*{\indexname}%
  \setlength{\parindent}{0pt}%
  \setlength{\parskip}{0pt plus 0.3pt}%
  \let\item\@idxitem
}{%
  \clearpage
}
\makeatother

\IfFileExists{\jobname-pw.ind}{\input{\jobname-pw.ind}}{}

\end{document}

      