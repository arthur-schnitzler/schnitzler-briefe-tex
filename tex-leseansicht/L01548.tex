%% latex-leseansicht-vorspann.tex
%% Vorspann für die Leseansicht.
%% Lädt die gemeinsame Datei latex-vorspann.tex mit nicht gesetztem Schalter.

\newif\ifkorrekturansicht
\korrekturansichtfalse

\input{../tex-inputs/latex-vorspann}


\section[Hermann Bahr an Arthur Schnitzler, 17. 9. 1905]{L01548 Hermann Bahr an Arthur Schnitzler, 17. 9. 1905}
\nopagebreak\mylabel{L01548v}
\rehead{ }\normalsize\beginnumbering\briefempfaengerindex{Schnitzler, Arthur@\textsc{Schnitzler, Arthur}!zzzBahr, Hermann@\emph{von Hermann Bahr}!1905-09-172@{17. 9. 1905}|(be}
\toendnotes[C]{\smallbreak\pagebreak[2]}
\correspDesc{Versand  durch Hermann Bahr am 17. 9. 1905 in Wien
\newline{}Erhalt  durch Arthur Schnitzler im Zeitraum [17. 9. 1905
                  – 21. 9. 1905?] in Wien}\toendnotes[C]{\smallbreak}
\Standort{TMW, HS AM 39978 Ba und AM 39979 Ba.}
\physDesc{Brief, maschinenschriftliche Abschrift, 1 Blatt, 1 Seite, 2948 Zeichen
\newline{}Schreibmaschine
\newline{}Ordnung: Original nicht nachweisbar; auf der Mappe in der Cambridge
                                 University Library hat Heinrich
                                    Schnitzler\pwindex{Schnitzler, Heinrich 9.\,8.\,1902 Hinterbrühl – 12.\,7.\,1982 Wien@\textsc{Schnitzler, Heinrich} (9.\,8.\,1902 Hinterbrühl – 12.\,7.\,1982 Wien), \emph{Regisseur, Schauspieler}|pw} vermerkt, dass Olga Schnitzler\pwindex{Schnitzler, Olga 17.\,1.\,1882 Wien – 13.\,1.\,1970 Lugano@\textsc{Schnitzler, Olga} (17.\,1.\,1882 Wien – 13.\,1.\,1970 Lugano), \emph{Schauspielerin, Sängerin}|pw} diesen Brief am 15. 8. 1936
                                 entnommen habe. }
\buchAbdrucke{\weitereDrucke{Hermann Bahr, Arthur Schnitzler: \emph{Briefwechsel, Aufzeichnungen, Dokumente (1891–1931)}. Herausgegeben von Kurt Ifkovits und Martin Anton Müller. Göttingen: \emph{Wallstein} 2018, S. 351–352.} }\toendnotes[C]{\smallbreak}
\pstart
           \raggedleft{}{\pb}17. 9. 1905\pend
           
\pstart{}Lieber Arthur!\pend\vspace{0.5em}
\pstart
           Ich war sehr verstimmt Dich heute verfehlt zu haben – ich bin sonst Vormittag fast
               immer zu Haus, nur heute musste ich ins Jubiläumstheater\oindex{Wien@\textbf{Wien}!IX., Alsergrund@\textbf{IX., Alsergrund}!Volksoper Wien@\textbf{Volksoper Wien}, \emph{Theater}|pw}, da dieses, seines Patriotismus wegen, ausersehen ist, den
                  »\label{K_L01548-1v}\edtext{Klub der Erlöser\pwindex{Bahr, Hermann 19.\,7.\,1863 Linz – 15.\,1.\,1934 München@\textsc{Bahr, Hermann} (19.\,7.\,1863 Linz – 15.\,1.\,1934 München), \emph{Schriftsteller, Kritiker}!Klub der Erlöser. Ein Akt@\strich\emph{Der Klub der Erlöser. Ein Akt}|pw}}{\lemma{\textnormal{\emph{Klub der Erlöser}}}\Cendnote{\textnormal{Das Schauspiel wurde Ende November von der Zensur nicht zur Aufführung zugelassen.}}}\label{K_L01548-1}« zu
               bringen, den ich Dir nächstens schicke, er ist eine Parallele zu »Unter sich\pwindex{Bahr, Hermann 19.\,7.\,1863 Linz – 15.\,1.\,1934 München@\textsc{Bahr, Hermann} (19.\,7.\,1863 Linz – 15.\,1.\,1934 München), \emph{Schriftsteller, Kritiker}!Unter sich. Ein Arme-Leut’-Stück@\strich\emph{Unter sich. Ein Arme-Leut’-Stück}|pw}«. Nun habe ich sogleich den »Ruf des Lebens\pwindex{Schnitzler, Arthur 15.\,5.\,1862 Wien – 21.\,10.\,1931 ebd.@\textsc{Schnitzler, Arthur} (15.\,5.\,1862 Wien – 21.\,10.\,1931 ebd.), \emph{Schriftsteller, Mediziner}!Ruf des Lebens. Schauspiel in drei Akten@\strich\emph{Der Ruf des Lebens. Schauspiel in drei Akten}|pw}« gelesen. Ich danke Dir herzlichst für die
               Absicht, ihn mir zu widmen, und Du machst mir eine sehr grosse Freude, wenn Du es
               wirklich tust. Seine »Gesinnung« (ich find im Augenblick nur dieses dumme Wort) hat
               mich sehr ergriffen und in \label{LL294-1v}dieser ungeheueren
                  Angst, die er ausdrückt und mitteilt, der Angst das Leben zu versäumen, das
                  einzige, das Höchste, geht er mir sehr nahe\label{LL294-1h}, ja ich glaube, dass Du noch
               nie so tief in das Gemüt unserer Generation und ihre letzte Sehnsucht eingedrungen
               bist. (An meinen »\label{K_L01548-2v}\edtext{armen Narren\pwindex{Bahr, Hermann 19.\,7.\,1863 Linz – 15.\,1.\,1934 München@\textsc{Bahr, Hermann} (19.\,7.\,1863 Linz – 15.\,1.\,1934 München), \emph{Schriftsteller, Kritiker}!arme Narr. Lustspiel in einem Akt@\strich\emph{Der arme Narr. Lustspiel in einem Akt}|pw}}{\lemma{\textnormal{\emph{armen Narren}}}\Cendnote{\textnormal{Hermann Bahr\pwindex{Bahr, Hermann 19.\,7.\,1863 Linz – 15.\,1.\,1934 München@\textsc{Bahr, Hermann} (19.\,7.\,1863 Linz – 15.\,1.\,1934 München), \emph{Schriftsteller, Kritiker}|pwk}: \emph{Der arme Narr. Schauspiel in einem Akt}\pwindex{Bahr, Hermann 19.\,7.\,1863 Linz – 15.\,1.\,1934 München@\textsc{Bahr, Hermann} (19.\,7.\,1863 Linz – 15.\,1.\,1934 München), \emph{Schriftsteller, Kritiker}!arme Narr. Lustspiel in einem Akt@\strich\emph{Der arme Narr. Lustspiel in einem Akt}|pwk}. In: \emph{Österreichische Rundschau}\pwindex{Österreichische Rundschau@\emph{Österreichische Rundschau}|pwk}, Jg. 4, H. 48,
                        28. 9. 1905, S. 396–407; H. 49, 5. 10. 1905,
                     S. 444–451; H. 50, 12. 10. 1905, S. 490–497.}}}\label{K_L01548-2}«, von
               dem ich nur noch kein Exemplar für Dich frei habe, und einem kleinen \label{K_L01548-3v}\edtext{Kainzbüchel\pwindex{Bahr, Hermann 19.\,7.\,1863 Linz – 15.\,1.\,1934 München@\textsc{Bahr, Hermann} (19.\,7.\,1863 Linz – 15.\,1.\,1934 München), \emph{Schriftsteller, Kritiker}!Josef Kainz@\strich\emph{Josef Kainz}|pwv}}{\lemma{\textnormal{\emph{Kainzbüchel}}}\Cendnote{\textnormal{Hermann Bahr: \emph{Josef Kainz}\pwindex{Bahr, Hermann 19.\,7.\,1863 Linz – 15.\,1.\,1934 München@\textsc{Bahr, Hermann} (19.\,7.\,1863 Linz – 15.\,1.\,1934 München), \emph{Schriftsteller, Kritiker}!Josef Kainz@\strich\emph{Josef Kainz}|pwk}. Wien,
                     Leipzig: \emph{Wiener Verlag}\orgindex{Wiener Verlag@Wiener Verlag|pwk}{ }1906.}}}\label{K_L01548-3}, das bei Freund\pwindex{Freund, Fritz 7.\,4.\,1879 Wien – 8.\,5.\,1950 ebd.@\textsc{Freund, Fritz} (7.\,4.\,1879 Wien – 8.\,5.\,1950 ebd.), \emph{Verleger}|pw} kommt,
               wirst \label{LL294-2v}Du sehen, dass mir dies, gerade dies und
                  eigentlich nur dies allein unser eigentliches Problem scheint, von dem mir alle
                  anderen unserer Forderungen oder Fragen nur Abwandlungen oder Variationen
                  scheinen\label{LL294-2h}). Was nun die Ausführung betrifft, einstweilen unter {\pb}dem ersten Eindruck nur folgendes: prachtvoll finde ich den Vater, von einer
               Plastik, die vielleicht noch nie eine Figur von Dir gehabt hat, ebenso stehen mir
               Marie und die gleich von mir geliebte Katharina wunderbar lebendig da, auch Dr.
               Schindler und Rainer sehe und höre ich, wenn schon ferner und stiller als jene.
               Dagegen (die Schuld mag an mir liegen, ich will Dir auch nur meinen ersten Eindruck
               sagen, wie sich ja schliesslich auch das Publikum immer nur an den unmittelbaren
               Eindruck hält) dagegen sehe ich den Obersten\pwindex{Schnitzler, Arthur 15.\,5.\,1862 Wien – 21.\,10.\,1931 ebd.@\textsc{Schnitzler, Arthur} (15.\,5.\,1862 Wien – 21.\,10.\,1931 ebd.), \emph{Schriftsteller, Mediziner}!Ruf des Lebens. Schauspiel in drei Akten@\strich\emph{Der Ruf des Lebens. Schauspiel in drei Akten}|pwv}, seine Frau und Max\pwindex{Schnitzler, Arthur 15.\,5.\,1862 Wien – 21.\,10.\,1931 ebd.@\textsc{Schnitzler, Arthur} (15.\,5.\,1862 Wien – 21.\,10.\,1931 ebd.), \emph{Schriftsteller, Mediziner}!Ruf des Lebens. Schauspiel in drei Akten@\strich\emph{Der Ruf des Lebens. Schauspiel in drei Akten}|pwv} gar nicht. Den Obersten \uline{kann} ich mir
               denken, und es reizt mich sehr, mir ihn zu denken, er geht mir nach, ich ihm, und ich
               dichte mir sein ganzes Leben hinzu, bald dieses, bald jenes, aber dies bleibt meiner
               Willkür frei, ich \uline{muss} nicht, denn es ist doch zu
               wenig von seiner Vergangenheit da, und nichts, das mich zwingen würde, daraus sein
               ganzes Wesen zu erkennen. Was noch mehr für seine Frau und vom L{[}i{]}eutenant\pwindex{Schnitzler, Arthur 15.\,5.\,1862 Wien – 21.\,10.\,1931 ebd.@\textsc{Schnitzler, Arthur} (15.\,5.\,1862 Wien – 21.\,10.\,1931 ebd.), \emph{Schriftsteller, Mediziner}!Ruf des Lebens. Schauspiel in drei Akten@\strich\emph{Der Ruf des Lebens. Schauspiel in drei Akten}|pwv} gilt. Die
               Kritik wird deshalb den zweiten Akt zu stark an Handlung und melodramatisch oder
               boulevarddramatisch oder dergleichen finden. Er ist es nicht, gewiss nicht, nur
               scheint mir der Ausgleich zwischen der auf die Handlung verteilten Kraft und der in
               die Figuren gelegten nicht völlig getroffen. Woher auch wohl das Gefühl stammt, das
               ich sehr lebhaft hatte, der Akt sei viel zu kurz, als ob alles nur angedeutet wäre,
               besonders an der sehr ruhig breiten Ausführung im ersten und dann wieder im dritten
               Akt gemessen. Doch über all das mündlich, sehr bald, wir müssen uns endlich einmal
               gründlich sehen. Grüss Deine liebe Frau\pwindex{Schnitzler, Olga 17.\,1.\,1882 Wien – 13.\,1.\,1970 Lugano@\textsc{Schnitzler, Olga} (17.\,1.\,1882 Wien – 13.\,1.\,1970 Lugano), \emph{Schauspielerin, Sängerin}|pwv} bestens und sei herzlichst gegrüsst von Deinem\pend
           \pstart \spacefill\mbox{H.}\pend{}
\pstart
           \noindent{}Brauchst Du das Manuscript zurück? »Zwischenspiel\pwindex{Schnitzler, Arthur 15.\,5.\,1862 Wien – 21.\,10.\,1931 ebd.@\textsc{Schnitzler, Arthur} (15.\,5.\,1862 Wien – 21.\,10.\,1931 ebd.), \emph{Schriftsteller, Mediziner}!Zwischenspiel. Komödie in drei Akten@\strich\emph{Zwischenspiel. Komödie in drei Akten}|pw}« les ich morgen.\pend
           \selectlanguage{ngerman}\endnumbering\briefempfaengerindex{Schnitzler, Arthur@\textsc{Schnitzler, Arthur}!zzzBahr, Hermann@\emph{von Hermann Bahr}!1905-09-172@{17. 9. 1905}|)be}\mylabel{L01548h}  \newcommand{\dateiname}{L01548}\newcommand{\titel}{Hermann Bahr an Arthur Schnitzler, 17. 9. 1905}\newcommand{\editorInnen}{Herausgegeben von Martin Anton Müller}%% latex-leseansicht-abspann.tex
%% Abspann für die Leseansicht.
%% Der Schalter \ifkorrekturansicht ist bereits durch den Vorspann gesetzt.

%% latex-abspann.tex
%% Gemeinsamer Abspann für Korrekturansicht und Leseansicht.
%% Setzt den Schalter \ifkorrekturansicht voraus (gesetzt in den
%% einbindenden Dateien latex-korrekturansicht-abspann.tex bzw.
%% latex-leseansicht-abspann.tex).
%% ---------------------------------------------------------------

\normalsize

% Das esempio-Environment wird nur in der Leseansicht benötigt
\ifkorrekturansicht\else
\newenvironment{esempio}[3]%
{
    \vspace{1.5ex}
    \rlap{\underline{#1}}
    \par
    \setlength{\parindent}{0cm}
    \nopagebreak
    \leftskip=#2cm
    \rightskip=#3cm
}
{
    \par
}
\fi

\doendnotes{C}
\bigskip
\vfill

\clearpage

\footnotesize

\ifkorrekturansicht
  \lohead{\textsc{register}}
\fi

% theindex-Environment neu definieren ohne reledmac
\makeatletter
\renewenvironment{theindex}{%
  \ifkorrekturansicht
    \section*{\indexname}%
  \else
    \subsubsection*{Index der erwähnten Entitäten}%
  \fi
  \setlength{\parindent}{0pt}%
  \setlength{\parskip}{0pt plus 0.3pt}%
  \let\item\@idxitem
}{%
  \ifkorrekturansicht\clearpage\fi
}
\makeatother

\IfFileExists{\jobname-pw.ind}{\input{\jobname-pw.ind}}{}

% Quellenangabe nur in der Leseansicht
\ifkorrekturansicht\else
% Fallback-Definitionen, falls die .tex-Datei \titel etc. nicht gesetzt hat
\providecommand{\titel}{}
\providecommand{\editorInnen}{}
\providecommand{\dateiname}{\jobname}

\vspace{3cm}

\vfill

\footnotesize
\textsc{Quelle}: \titel. Herausgegeben von {\editorInnen}. In: \emph{Arthur Schnitzler: Briefwechsel mit Autorinnen und Autoren}.
 Digitale Edition, https://schnitzler-briefe.acdh.oeaw.ac.at/{\dateiname}.html (Stand \today)
\fi

\end{document}


