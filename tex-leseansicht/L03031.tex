%% latex-korrekturansicht-vorspann.tex
%% Vorspann für die Korrekturansicht.
%% Lädt die gemeinsame Datei latex-vorspann.tex mit gesetztem Schalter.

\newif\ifkorrekturansicht
\korrekturansichttrue

\input{../tex-inputs/latex-vorspann}


\section[Arthur Schnitzler an Felix Salten, {[}7. 5. 1892?{]}]{L03031 Arthur Schnitzler an Felix Salten, {[}7. 5. 1892?{]}}
\nopagebreak\mylabel{L03031v}
\rehead{ }\normalsize\beginnumbering\briefempfaengerindex{Salten, Felix@\textsc{Salten, Felix}!zzzSchnitzler, Arthur@\emph{von Arthur Schnitzler}!1892-05-072@{{[}7. 5. 1892?{]}}|(be}
\toendnotes[C]{\smallbreak\pagebreak[2]}\Standort{Wienbibliothek im Rathaus, ZPH 1681, 2.1.516.}
\physDesc{Briefkarte, 284 Zeichen
\newline{}Handschrift: Bleistift, deutsche Kurrent
\newline{}Ordnung: mit Bleistift von unbekannter Hand nummeriert: »37« }\toendnotes[C]{\smallbreak}
\pstart
           \noindent{}{\pb}Lieber Freund, ich ko{\geminationn}te
                  geſtern nicht ko{\geminationm}en u nicht abſagen –
               Pardon! – Heute hab ich Sitze für Sie, d h für uns beide geno{\geminationm}en, bitte ſehr, erwarten Sie mich {\pb}4 Uhr in meiner \label{K_L03031-1v}\edtext{Wohnung
                  \textsc{Giselastraße\oindex{Ordination Arthur Schnitzler [Boesendorferstrasse 11]@\textbf{Ordination Arthur Schnitzler [Bösendorferstraße 11]}, \emph{Ordination}|pw}}}{\lemma{\textnormal{\emph{Wohnung
                  Giselastraße}}}\Cendnote{\textnormal{Nach hinten kann das undatierte
                  Korrespondenzstück durch den Zeitraum eingegrenzt werden, in dem Schnitzler an dieser Adresse gewohnt hat (14. 10. 1892). Im Zuge
                  der \emph{Wiener Musik- und Theaterausstellung 1892}\orgindex{Internationale Ausstellung fuer Musik und Theaterwesen@Internationale Ausstellung für Musik und Theaterwesen|pwk}
                  sind häufige gemeinsame Theaterbesuche nachgewiesen. Der erste Tag der Ausstellung\orgindex{Internationale Ausstellung fuer Musik und Theaterwesen@Internationale Ausstellung für Musik und Theaterwesen|pwkv}, der 7. 5. 1892, dürfte
                  auch der Versandtag dieses Schreibens sein, da Schnitzler am [7. 5. 1892] seinen erkrankten Vater\pwindex{Schnitzler, Johann 10.04.1835 – 02.05.1893@\textsc{Schnitzler, Johann} (10.04.1835 – 02.05.1893), \emph{Laryngologe/Laryngologin}|pwkv} in der Ordination\oindex{Wohnung und Ordination Johann Schnitzler Burgring 1@\textbf{Wohnung und Ordination Johann Schnitzler Burgring 1}, \emph{Ordination}|pwkv} am Burgring 1\oindex{Burgring@\textbf{Burgring}, \emph{Straße (K.STR)}|pwk} vertreten hat.}}}\label{K_L03031-1} – we{\geminationn} Sie nicht
               eventuell ſchon früher Burgring\oindex{Wohnung und Ordination Johann Schnitzler Burgring 1@\textbf{Wohnung und Ordination Johann Schnitzler Burgring 1}, \emph{Ordination}|pw} ko{\geminationm}en können. Aber treffen müſſen wir uns.\pend
           \pstart Ihr \spacefill\mbox{Arth Sch}\pend{}\selectlanguage{ngerman}\endnumbering\briefempfaengerindex{Salten, Felix@\textsc{Salten, Felix}!zzzSchnitzler, Arthur@\emph{von Arthur Schnitzler}!1892-05-072@{{[}7. 5. 1892?{]}}|)be}\mylabel{L03031h}  \normalsize

\doendnotes{C}
\bigskip
\vfill

\clearpage

\footnotesize

\lohead{\textsc{register}}

% Definiere theindex-Environment komplett neu ohne reledmac
\makeatletter
\renewenvironment{theindex}{%
  \section*{\indexname}%
  \setlength{\parindent}{0pt}%
  \setlength{\parskip}{0pt plus 0.3pt}%
  \let\item\@idxitem
}{%
  \clearpage
}
\makeatother

\IfFileExists{\jobname-pw.ind}{\input{\jobname-pw.ind}}{}

\end{document}

      