%% latex-leseansicht-vorspann.tex
%% Vorspann für die Leseansicht.
%% Lädt die gemeinsame Datei latex-vorspann.tex mit nicht gesetztem Schalter.

\newif\ifkorrekturansicht
\korrekturansichtfalse

\input{../tex-inputs/latex-vorspann}


         
         \renewcommand{\erwaehntePersonen}{Personen: Felix Salten, Johann Schnitzler}
         \renewcommand{\erwaehnteInstitutionen}{Institutionen: Internationale Ausstellung für Musik und Theaterwesen}
         \renewcommand{\erwaehnteOrte}{Orte: Burgring, Ordination Dr. Arthur Schnitzler Giselastraße 11, Wien, Wohnung und Ordination Johann Schnitzler Burgring 1}
         \renewcommand{\erwaehnteWerke}{}
               \section[Arthur Schnitzler an Felix Salten, {[}7. 5. 1892?{]}]{ Arthur Schnitzler an Felix Salten, {[}7. 5. 1892?{]}}\nopagebreak\mylabel{v}\rehead{ }\begin{ledgroupsized}[t]{13cm}\normalsize\beginnumbering\briefempfaengerindex{Salten, Felix@\textsc{Salten, Felix}!zzzSchnitzler, Arthur@\emph{von Arthur Schnitzler}!1892-05-072@{{[}7. 5. 1892?{]}}|(be} \toendnotes[C]{\smallbreak\pagebreak[2]} \Standort{Wienbibliothek im Rathaus, ZPH 1681, 2.1.516.}
\physDesc{Briefkarte, 284 Zeichen
\newline{}Handschrift: Bleistift, deutsche Kurrent
\newline{}Ordnung: mit Bleistift von unbekannter Hand nummeriert: »37« }\toendnotes[C]{\smallbreak}\pstart
           \noindent{}{\pb}Lieber Freund, ich ko{\geminationn}te
                  geſtern nicht ko{\geminationm}en u nicht abſagen –
               Pardon! – Heute hab ich Sitze für Sie, d h für uns beide geno{\geminationm}en, bitte ſehr, erwarten Sie mich {\pb}4 Uhr in meiner \label{K_L03031-1v}\edtext{Wohnung
                  \textsc{Giselastraße\oindex{Ordination Dr. Arthur Schnitzler Giselastrasse 11@\textbf{Ordination Dr. Arthur Schnitzler Giselastraße 11}|pw}}}{\lemma{\textnormal{\emph{Wohnung
                  Giselastraße}}}\Cendnote{\textnormal{Nach hinten kann das undatierte
                  Korrespondenzstück durch den Zeitraum eingegrenzt werden, in dem Schnitzler\pwindex{Schnitzler, Arthur 15.05.1862 – 21.10.1931@\textsc{Schnitzler, Arthur} (15.05.1862 – 21.10.1931), \emph{Schriftsteller, Mediziner}|pwk} an dieser Adresse gewohnt hat (14. 10. 1892). Im Zuge
                  der \emph{Wiener Musik- und Theaterausstellung 1892}\orgindex{Internationale Ausstellung fuer Musik und Theaterwesen@Internationale Ausstellung für Musik und Theaterwesen|pwk}
                  sind häufige gemeinsame Theaterbesuche nachgewiesen. Der erste Tag der Ausstellung\orgindex{Internationale Ausstellung fuer Musik und Theaterwesen@Internationale Ausstellung für Musik und Theaterwesen|pwkv}, der 7. 5. 1892, dürfte
                  auch der Versandtag dieses Schreibens sein, da Schnitzler\pwindex{Schnitzler, Arthur 15.05.1862 – 21.10.1931@\textsc{Schnitzler, Arthur} (15.05.1862 – 21.10.1931), \emph{Schriftsteller, Mediziner}|pwk} am [7. 5. 1892] seinen erkrankten Vater\pwindex{Schnitzler, Johann 10.04.1835 – 02.05.1893@\textsc{Schnitzler, Johann} (10.04.1835 – 02.05.1893), \emph{Laryngologe}|pwkv} in der Ordination\oindex{Wohnung und Ordination Johann Schnitzler Burgring 1@\textbf{Wohnung und Ordination Johann Schnitzler Burgring 1}|pwkv} am Burgring 1\oindex{Burgring@\textbf{Burgring}|pwk} vertreten hat.}}}\label{K_L03031-1h} – we{\geminationn} Sie nicht
               eventuell ſchon früher Burgring\oindex{Wohnung und Ordination Johann Schnitzler Burgring 1@\textbf{Wohnung und Ordination Johann Schnitzler Burgring 1}|pw} ko{\geminationm}en können. Aber treffen müſſen wir uns.\pend
           \pstart Ihr \spacefill\mbox{Arth Sch}\pend{}
         
         \endnumbering\mylabel{h}\end{ledgroupsized}  \newcommand{\dateiname}{L03031}\newcommand{\titel}{Arthur Schnitzler an Felix Salten, [7. 5. 1892?]}\newcommand{\editorInnen}{Martin Anton Müller und Laura Untner}%% latex-leseansicht-abspann.tex
%% Abspann für die Leseansicht.
%% Der Schalter \ifkorrekturansicht ist bereits durch den Vorspann gesetzt.

%% latex-abspann.tex
%% Gemeinsamer Abspann für Korrekturansicht und Leseansicht.
%% Setzt den Schalter \ifkorrekturansicht voraus (gesetzt in den
%% einbindenden Dateien latex-korrekturansicht-abspann.tex bzw.
%% latex-leseansicht-abspann.tex).
%% ---------------------------------------------------------------

\normalsize

% Das esempio-Environment wird nur in der Leseansicht benötigt
\ifkorrekturansicht\else
\newenvironment{esempio}[3]%
{
    \vspace{1.5ex}
    \rlap{\underline{#1}}
    \par
    \setlength{\parindent}{0cm}
    \nopagebreak
    \leftskip=#2cm
    \rightskip=#3cm
}
{
    \par
}
\fi

\doendnotes{C}
\bigskip
\vfill

\clearpage

\footnotesize

\ifkorrekturansicht
  \lohead{\textsc{register}}
\fi

% theindex-Environment neu definieren ohne reledmac
\makeatletter
\renewenvironment{theindex}{%
  \ifkorrekturansicht
    \section*{\indexname}%
  \else
    \subsubsection*{Index der erwähnten Entitäten}%
  \fi
  \setlength{\parindent}{0pt}%
  \setlength{\parskip}{0pt plus 0.3pt}%
  \let\item\@idxitem
}{%
  \ifkorrekturansicht\clearpage\fi
}
\makeatother

\IfFileExists{\jobname-pw.ind}{\input{\jobname-pw.ind}}{}

% Quellenangabe nur in der Leseansicht
\ifkorrekturansicht\else
% Fallback-Definitionen, falls die .tex-Datei \titel etc. nicht gesetzt hat
\providecommand{\titel}{}
\providecommand{\editorInnen}{}
\providecommand{\dateiname}{\jobname}

\vspace{3cm}

\vfill

\footnotesize
\textsc{Quelle}: \titel. Herausgegeben von {\editorInnen}. In: \emph{Arthur Schnitzler: Briefwechsel mit Autorinnen und Autoren}.
 Digitale Edition, https://schnitzler-briefe.acdh.oeaw.ac.at/{\dateiname}.html (Stand \today)
\fi

\end{document}


      