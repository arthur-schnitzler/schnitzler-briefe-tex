%% latex-leseansicht-vorspann.tex
%% Vorspann für die Leseansicht.
%% Lädt die gemeinsame Datei latex-vorspann.tex mit nicht gesetztem Schalter.

\newif\ifkorrekturansicht
\korrekturansichtfalse

\input{../tex-inputs/latex-vorspann}


         
         \newcommand{\erwaehntePersonen}{Personen: Adalbert Franz Seligmann, Richard Specht}
         \newcommand{\erwaehnteInstitutionen}{Institutionen: Wiener Verlag}
         \newcommand{\erwaehnteOrte}{Orte: Wien, XIX., Döbling}
         \newcommand{\erwaehnteWerke}{Werke: Liebelei. Erstes Bild, Widmungen zur Feier des siebzigsten Geburtstages Ferdinand von Saar’s.}
               \section[Ferdinand von Saar an Arthur Schnitzler, 25. 10. 1902]{ Ferdinand von Saar an Arthur Schnitzler, 25. 10. 1902}\nopagebreak\mylabel{v}\rehead{ }\begin{ledgroupsized}[t]{13cm}\normalsize\beginnumbering \toendnotes[C]{\smallbreak\pagebreak[2]} \Standort{CUL, Schnitzler, B 88.}
\physDesc{Briefkarte
\newline{}Handschrift: schwarze Tinte, deutsche Kurrent
\newline{}Schnitzler: mit Bleistift nummeriert: »6« }\toendnotes[C]{\smallbreak}\pstart
           \noindent{}\textcolor{gray}{\textbf{{\pb}Bei meinem Eintritt in das
                        70. Lebensjahr sind mir so zahlreiche Beweise der Anerkennung und Zuneigung
                        geworden, dass ich nur in dieser Weise meinen wärmsten Dank darbringen kann.
                        Mögen Alle, die mich am späten Abend meines Lebens durch Ehrungen
                        ausgezeichnet, mir Liebes und Gutes gesagt oder bezeigt, die Versicherung
                        entgegen nehmen, dass ich mich durch all diese Kundgebungen im tiefsten
                        beglückt fühle. Bin ich doch jetzt von dem erhebenden Bewusstsein
                        durchdrungen, den Besten meiner Zeit genug gethan zu haben.}}\pend
           \pstart
           \textcolor{gray}{\textbf{Wien-Döbling\oindex{XIX., Doebling@\textbf{XIX., Döbling}|pw}.}}{ }25/10. 1902\pend
           \pstart
           mit herzlichem Dichtergruß{\\[\baselineskip]}und beſonderem Danke{\\[\baselineskip]}für die collegial
                    anerkennende »\label{K_L01245_1v}\edtext{Widmung.\pwindex{Schnitzler, Arthur 15.05.1862 – 21.10.1931@\textsc{Schnitzler, Arthur} (15.05.1862 – 21.10.1931), \emph{Schriftsteller, Mediziner}!Liebelei. Erstes Bild1902-11-14@\strich\emph{Liebelei. Erstes Bild} {[}1902-11-14{]}|pwv}}{\lemma{\textnormal{\emph{Widmung.}}}\Cendnote{\textnormal{Gemeint ist Schnitzler\pwindex{Schnitzler, Arthur 15.05.1862 – 21.10.1931@\textsc{Schnitzler, Arthur} (15.05.1862 – 21.10.1931), \emph{Schriftsteller, Mediziner}|pwk}s Beitrag für eine Festschrift: \emph{Liebelei. Erstes Bild}\pwindex{Schnitzler, Arthur 15.05.1862 – 21.10.1931@\textsc{Schnitzler, Arthur} (15.05.1862 – 21.10.1931), \emph{Schriftsteller, Mediziner}!Liebelei. Erstes Bild1902-11-14@\strich\emph{Liebelei. Erstes Bild} {[}1902-11-14{]}|pwk}. In: \emph{Widmungen zur Feier des siebzigsten
                                Geburtstages Ferdinand von Saar’s}\pwindex{Widmungen zur Feier des siebzigsten Geburtstages Ferdinand von Saar s.1902-11-14@\emph{Widmungen zur Feier des siebzigsten Geburtstages Ferdinand von Saar’s.} {[}1902-11-14{]}|pwk}. Hg. v. Richard Specht\pwindex{Specht, Richard 07.12.1870 – 18.03.1932@\textsc{Specht, Richard} (07.12.1870 – 18.03.1932), \emph{Schriftsteller, Journalist, Kritiker}|pwk}. Buchschmuck v. A. F. Seligmann\pwindex{Seligmann, Adalbert Franz 02.04.1862 – 13.12.1945@\textsc{Seligmann, Adalbert Franz} (02.04.1862 – 13.12.1945), \emph{Maler, Publizist}|pwk}. Wien: \emph{Wiener Verlag}\orgindex{Wiener Verlag@Wiener Verlag|pwk}{ }1903 (vordatiert von 14. 11. 1902),
                            S. 175–196.}}}\label{K_L01245_1h}«{\\[\baselineskip]}\spacefill\mbox{Ferdinand von Saar.}\pend
           \leftskip=0em{}
         
         \endnumbering\mylabel{h}\end{ledgroupsized}  \newcommand{\dateiname}{L01245}\newcommand{\titel}{Ferdinand von Saar an Arthur Schnitzler, 25. 10. 1902}\newcommand{\editorInnen}{Martin Anton Müller und Gerd-Hermann Susen}%% latex-leseansicht-abspann.tex
%% Abspann für die Leseansicht.
%% Der Schalter \ifkorrekturansicht ist bereits durch den Vorspann gesetzt.

%% latex-abspann.tex
%% Gemeinsamer Abspann für Korrekturansicht und Leseansicht.
%% Setzt den Schalter \ifkorrekturansicht voraus (gesetzt in den
%% einbindenden Dateien latex-korrekturansicht-abspann.tex bzw.
%% latex-leseansicht-abspann.tex).
%% ---------------------------------------------------------------

\normalsize

% Das esempio-Environment wird nur in der Leseansicht benötigt
\ifkorrekturansicht\else
\newenvironment{esempio}[3]%
{
    \vspace{1.5ex}
    \rlap{\underline{#1}}
    \par
    \setlength{\parindent}{0cm}
    \nopagebreak
    \leftskip=#2cm
    \rightskip=#3cm
}
{
    \par
}
\fi

\doendnotes{C}
\bigskip
\vfill

\clearpage

\footnotesize

\ifkorrekturansicht
  \lohead{\textsc{register}}
\fi

% theindex-Environment neu definieren ohne reledmac
\makeatletter
\renewenvironment{theindex}{%
  \ifkorrekturansicht
    \section*{\indexname}%
  \else
    \subsubsection*{Index der erwähnten Entitäten}%
  \fi
  \setlength{\parindent}{0pt}%
  \setlength{\parskip}{0pt plus 0.3pt}%
  \let\item\@idxitem
}{%
  \ifkorrekturansicht\clearpage\fi
}
\makeatother

\IfFileExists{\jobname-pw.ind}{\input{\jobname-pw.ind}}{}

% Quellenangabe nur in der Leseansicht
\ifkorrekturansicht\else
% Fallback-Definitionen, falls die .tex-Datei \titel etc. nicht gesetzt hat
\providecommand{\titel}{}
\providecommand{\editorInnen}{}
\providecommand{\dateiname}{\jobname}

\vspace{3cm}

\vfill

\footnotesize
\textsc{Quelle}: \titel. Herausgegeben von {\editorInnen}. In: \emph{Arthur Schnitzler: Briefwechsel mit Autorinnen und Autoren}.
 Digitale Edition, https://schnitzler-briefe.acdh.oeaw.ac.at/{\dateiname}.html (Stand \today)
\fi

\end{document}


      