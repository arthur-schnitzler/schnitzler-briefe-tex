%% latex-korrekturansicht-vorspann.tex
%% Vorspann für die Korrekturansicht.
%% Lädt die gemeinsame Datei latex-vorspann.tex mit gesetztem Schalter.

\newif\ifkorrekturansicht
\korrekturansichttrue

\input{../tex-inputs/latex-vorspann}


\section[Ferdinand von Saar an Arthur Schnitzler, 25. 10. 1902]{L01245 Ferdinand von Saar an Arthur Schnitzler, 25. 10. 1902}
\nopagebreak\mylabel{L01245v}
\rehead{ }\normalsize\beginnumbering\briefempfaengerindex{Schnitzler, Arthur@\textsc{Schnitzler, Arthur}!zzzSaar, Ferdinand von@\emph{von Ferdinand von Saar}!1902-10-251@{25. 10. 1902}|(be}
\toendnotes[C]{\smallbreak\pagebreak[2]}\Standort{CUL, Schnitzler, B 88.}
\physDesc{Briefkarte, 118 Zeichen
\newline{}Handschrift: schwarze Tinte, deutsche Kurrent
\newline{}Schnitzler: mit Bleistift nummeriert: »6« }\toendnotes[C]{\smallbreak}
\pstart
           \noindent{}\textcolor{gray}{\textbf{{\pb}Bei meinem Eintritt in das
                  70. Lebensjahr sind mir so zahlreiche Beweise der Anerkennung und Zuneigung
                  geworden, dass ich nur in dieser Weise meinen wärmsten Dank darbringen kann. Mögen
                  Alle, die mich am späten Abend meines Lebens durch Ehrungen ausgezeichnet, mir
                  Liebes und Gutes gesagt oder bezeigt, die Versicherung entgegen nehmen, dass ich
                  mich durch all diese Kundgebungen im tiefsten beglückt fühle. Bin ich doch jetzt
                  von dem erhebenden Bewusstsein durchdrungen, den Besten meiner Zeit genug gethan
                  zu haben.}}\pend
           
\pstart
           \textcolor{gray}{\textbf{Wien-Döbling\oindex{XIX., Doebling@\textbf{XIX., Döbling}, \emph{A.ADM3}|pw}.}}{ }25/10. 1902\pend
           
\pstart
           mit herzlichem Dichtergruß{\\[\baselineskip]}und beſonderem Danke{\\[\baselineskip]}für die collegial
               anerkennende »\label{K_L01245-1v}\edtext{Widmung.\pwindex{Liebelei. Erstes Bild@\emph{Liebelei. Erstes Bild}|pwv}}{\lemma{\textnormal{\emph{Widmung.}}}\Cendnote{\textnormal{Gemeint ist Schnitzlers Beitrag für eine Festschrift: \emph{Liebelei. Erstes Bild}\pwindex{Liebelei. Erstes Bild@\emph{Liebelei. Erstes Bild}|pwk}. In: \emph{Widmungen zur Feier des siebzigsten Geburtstages Ferdinand
                        von Saar’s}\pwindex{Widmungen zur Feier des siebzigsten Geburtstages Ferdinand von Saar s.@\emph{Widmungen zur Feier des siebzigsten Geburtstages Ferdinand von Saar’s.}|pwk}. Herausgegeben von Richard
                     Specht\pwindex{Specht, Richard 07.12.1870 – 18.03.1932@\textsc{Specht, Richard} (07.12.1870 – 18.03.1932), \emph{Schriftsteller/Schriftstellerin, Journalist/Journalistin, Kritiker/Kritikerin}|pwk}. Buchschmuck von A. F.
                        Seligmann\pwindex{Seligmann, Adalbert Franz 02.04.1862 – 13.12.1945@\textsc{Seligmann, Adalbert Franz} (02.04.1862 – 13.12.1945), \emph{Maler/Malerin, Publizist/Publizistin}|pwk}. Wien: \emph{Wiener Verlag}\orgindex{Wiener Verlag@Wiener Verlag|pwk}{ }1903 [vordatiert, erschienen am 14. 11. 1902],
                  S. 175–196.}}}\label{K_L01245-1}«{\\[\baselineskip]}\spacefill\mbox{Ferdinand von Saar.}\pend
           \leftskip=0em{}\selectlanguage{ngerman}\endnumbering\briefempfaengerindex{Schnitzler, Arthur@\textsc{Schnitzler, Arthur}!zzzSaar, Ferdinand von@\emph{von Ferdinand von Saar}!1902-10-251@{25. 10. 1902}|)be}\mylabel{L01245h}  \normalsize

\doendnotes{C}
\bigskip
\vfill

\clearpage

\footnotesize

\lohead{\textsc{register}}

% Definiere theindex-Environment komplett neu ohne reledmac
\makeatletter
\renewenvironment{theindex}{%
  \section*{\indexname}%
  \setlength{\parindent}{0pt}%
  \setlength{\parskip}{0pt plus 0.3pt}%
  \let\item\@idxitem
}{%
  \clearpage
}
\makeatother

\IfFileExists{\jobname-pw.ind}{\input{\jobname-pw.ind}}{}

\end{document}

      