%% latex-leseansicht-vorspann.tex
%% Vorspann für die Leseansicht.
%% Lädt die gemeinsame Datei latex-vorspann.tex mit nicht gesetztem Schalter.

\newif\ifkorrekturansicht
\korrekturansichtfalse

\input{../tex-inputs/latex-vorspann}


\section[Richard Beer-Hofmann an Arthur Schnitzler, {[}2. 10. 1894{]}]{L00375 Richard Beer-Hofmann an Arthur Schnitzler, {[}2. 10. 1894{]}}
\nopagebreak\mylabel{L00375v}
\rehead{ }\normalsize\beginnumbering\briefempfaengerindex{Schnitzler, Arthur@\textsc{Schnitzler, Arthur}!zzzBeer-Hofmann, Richard@\emph{von Richard Beer-Hofmann}!1894-10-021@{{[}2. 10. 1894{]}}|(be}
\toendnotes[C]{\smallbreak\pagebreak[2]}
\correspDesc{Versand  durch Richard Beer-Hofmann am [2. 10. 1894] in Florenz
\newline{}Erhalt  durch Arthur Schnitzler im Zeitraum [3. 10. 1894
                  – 7. 10. 1894?] in Wien}\toendnotes[C]{\smallbreak}
\Standort{CUL, Schnitzler, B 8.}
\physDesc{Brief, 1 Blatt, 4 Seiten, 1404 Zeichen
\newline{}Handschrift: Bleistift, lateinische Kurrent
\newline{}Schnitzler: mit Bleistift beschriftet »Florenz\oindex{Florenz@\textbf{Florenz}|pw},
                                       2/10 94« und nummeriert: »48« }
\buchAbdrucke{\weitereDrucke{1) Arthur Schnitzler, Richard Beer-Hofmann: \emph{Briefwechsel 1891–1931}. Herausgegeben von Konstanze Fliedl. Wien, Zürich: \emph{Europaverlag} 1992, S. 61–62.} \weitereDrucke{2) Hermann Bahr, Arthur Schnitzler: \emph{Briefwechsel, Aufzeichnungen, Dokumente (1891–1931)}. Herausgegeben von Kurt Ifkovits und Martin Anton Müller. Göttingen: \emph{Wallstein} 2018.} }\toendnotes[C]{\smallbreak}
\pstart
           \noindent{}{\pb}Lieber Arthur! Mit Ihrem Brief hab ich mich sehr gefreut. Wenn man
               Tagelang stu{\geminationm} unter schönen Sachen herum geht freut
               einen eine – na wie soll ich sagen, – na eine \uline{bekannte} sti{\geminationm}e wieder –\pend
           
\pstart
           Ich bin von den Uffizien\oindex{Uffizien@\textbf{Uffizien}, \emph{Museum}|pw} geko{\geminationm}en u. habe auf dem Wege ins Restaurant {\pb}Ihren Brief von der Post geholt und
               ihn dann mit Behagen während des Speisens gelesen. Ich habe Aufsehen erregt weil ich
               fortwährend, auch nachher geschmunzelt habe, schließlich hat der Kellner auch
               geschmunzelt und mich für eine heitere joviale Natur gehalten.\pend
           
\pstart
           Sie schreiben i{\geminationm}er schlechter; d. h. ich kann sehr
               schwer {\pb}Ihre Zeilen entziffern,
               höchstens die Unterschrift, und die heisst dann »Richard«. Wenn Sie mich nach der
               »Madonna«
               fragen, und noch dazu so nebenher im Postscriptum ({\{}2, 4, 6, 8 – – – – ∞?{\}}gradig?) so beweist dies nur daß »sie« Ihre
               sexuelle Phantasie stark erregt. Bitte. – Bitte tun Sie wie wenn ich nicht zu Hause
               wäre. – Sie können auch nach meiner Adresse fragen, – mehrmals – {\pb}und dabei findet sich
               Gelegenheit.\pend
           
\pstart
           Bitte: Bahr\pwindex{Bahr, Hermann 19.\,7.\,1863 Linz – 15.\,1.\,1934 München@\textsc{Bahr, Hermann} (19.\,7.\,1863 Linz – 15.\,1.\,1934 München), \emph{Schriftsteller, Kritiker}|pw} soll die »Zeit\orgindex{Zeit. Wiener Wochenschrift@Die Zeit. Wiener Wochenschrift|pw}« (die erste Nu{\geminationm}er) \uline{a posta ferma}{ }\uline{Rom\oindex{Rom@\textbf{Rom}, \emph{Hauptstadt}|pw}} senden – ja? Von Donnerstag an, bitte adressiren Sie auch die
               Briefe u. Karten an mich, dorthin. Und schreiben Sie mir öfters: Ich werde jeden Tag
               vor Tisch mir etwas von Ihnen abholen gehen. Ihr »Guercino\pwindex{Guercino 8.\,2.\,1591 Cento – 22.\,12.\,1666 Bologna@\textsc{Guercino} (8.\,2.\,1591 Cento – 22.\,12.\,1666 Bologna), \emph{Maler}|pw}\pwindex{Guercino 8.\,2.\,1591 Cento – 22.\,12.\,1666 Bologna@\textsc{Guercino} (8.\,2.\,1591 Cento – 22.\,12.\,1666 Bologna), \emph{Maler}!Verstoßung der Hagar@\strich\emph{Die Verstoßung der Hagar}|pwv}« hängt in Mailand\oindex{Mailand@\textbf{Mailand}|pw}. Grüße bitte richten Sie
               ein für allemal \uline{à discretion} aus, wissen Sie, so als
               Belohnung. Herzlichst Ihr –\pend
           \pstart \spacefill\mbox{Richard}\pend{}
\pstart
           Dienstag{ }\introOben{}(½ 11)\introOben{}{ }früh,! Florenz\oindex{Florenz@\textbf{Florenz}|pw}\pend
           \selectlanguage{ngerman}\endnumbering\briefempfaengerindex{Schnitzler, Arthur@\textsc{Schnitzler, Arthur}!zzzBeer-Hofmann, Richard@\emph{von Richard Beer-Hofmann}!1894-10-021@{{[}2. 10. 1894{]}}|)be}\mylabel{L00375h}  \newcommand{\dateiname}{L00375}\newcommand{\titel}{Richard Beer-Hofmann an Arthur Schnitzler, [2. 10. 1894]}\newcommand{\editorInnen}{Herausgegeben von Martin Anton Müller}%% latex-leseansicht-abspann.tex
%% Abspann für die Leseansicht.
%% Der Schalter \ifkorrekturansicht ist bereits durch den Vorspann gesetzt.

%% latex-abspann.tex
%% Gemeinsamer Abspann für Korrekturansicht und Leseansicht.
%% Setzt den Schalter \ifkorrekturansicht voraus (gesetzt in den
%% einbindenden Dateien latex-korrekturansicht-abspann.tex bzw.
%% latex-leseansicht-abspann.tex).
%% ---------------------------------------------------------------

\normalsize

% Das esempio-Environment wird nur in der Leseansicht benötigt
\ifkorrekturansicht\else
\newenvironment{esempio}[3]%
{
    \vspace{1.5ex}
    \rlap{\underline{#1}}
    \par
    \setlength{\parindent}{0cm}
    \nopagebreak
    \leftskip=#2cm
    \rightskip=#3cm
}
{
    \par
}
\fi

\doendnotes{C}
\bigskip
\vfill

\clearpage

\footnotesize

\ifkorrekturansicht
  \lohead{\textsc{register}}
\fi

% theindex-Environment neu definieren ohne reledmac
\makeatletter
\renewenvironment{theindex}{%
  \ifkorrekturansicht
    \section*{\indexname}%
  \else
    \subsubsection*{Index der erwähnten Entitäten}%
  \fi
  \setlength{\parindent}{0pt}%
  \setlength{\parskip}{0pt plus 0.3pt}%
  \let\item\@idxitem
}{%
  \ifkorrekturansicht\clearpage\fi
}
\makeatother

\IfFileExists{\jobname-pw.ind}{\input{\jobname-pw.ind}}{}

% Quellenangabe nur in der Leseansicht
\ifkorrekturansicht\else
% Fallback-Definitionen, falls die .tex-Datei \titel etc. nicht gesetzt hat
\providecommand{\titel}{}
\providecommand{\editorInnen}{}
\providecommand{\dateiname}{\jobname}

\vspace{3cm}

\vfill

\footnotesize
\textsc{Quelle}: \titel. Herausgegeben von {\editorInnen}. In: \emph{Arthur Schnitzler: Briefwechsel mit Autorinnen und Autoren}.
 Digitale Edition, https://schnitzler-briefe.acdh.oeaw.ac.at/{\dateiname}.html (Stand \today)
\fi

\end{document}


