%% latex-korrekturansicht-vorspann.tex
%% Vorspann für die Korrekturansicht.
%% Lädt die gemeinsame Datei latex-vorspann.tex mit gesetztem Schalter.

\newif\ifkorrekturansicht
\korrekturansichttrue

\input{../tex-inputs/latex-vorspann}


\section[Arthur und Olga Schnitzler an Richard Beer-Hofmann, 13. 8. 1907]{L01699 Arthur und Olga Schnitzler an Richard Beer-Hofmann, 13. 8. 1907}
\nopagebreak\mylabel{L01699v}
\rehead{ }\normalsize\beginnumbering\briefempfaengerindex{Beer-Hofmann, Richard@\textsc{Beer-Hofmann, Richard}!zzzSchnitzler, Olga@\emph{von Olga Schnitzler}!1907-08-131@{13. 8. 1907}|(be}\briefempfaengerindex{Beer-Hofmann, Richard@\textsc{Beer-Hofmann, Richard}!zzzSchnitzler, Arthur@\emph{von Arthur Schnitzler}!1907-08-131@{13. 8. 1907}|(be}
\toendnotes[C]{\smallbreak\pagebreak[2]}\Standort{YCGL, MSS 31.}
\physDesc{Bildpostkarte, 194 Zeichen
\newline{}Handschrift Arthur Schnitzler: Bleistift, deutsche Kurrent
\newline{}Handschrift Olga Schnitzler: Bleistift, lateinische Kurrent
\newline{}Versand: Stempel: »\nobreak{}\oindex{Misurina@\textbf{Misurina}, \emph{P.PPL}|pwk}Misurina, 13. 8. 1907\nobreak{}«.  
\newline{}Ordnung: mit Bleistift von unbekannter Hand datiert: »12. 8.« }\toendnotes[C]{\smallbreak}\pstart{}{\pb}\textsc{Dr. Richard Beerhofmann}\pend{}\pstart{}Wien\oindex{Wien@\textbf{Wien}, \emph{A.ADM2}|pw}\pend{}\pstart{}\textsc{Hasenauerstr 59\oindex{Hasenauerstrasse 59@\textbf{Hasenauerstraße 59}, \emph{Wohngebäude (K.WHS)}|pw}.}\pend{}\pstart{}\textsc{Austria\oindex{Oesterreich@\textbf{Österreich}, \emph{A.PCLI}|pw}}\pend{}{\bigskip}
\pstart
           \noindent{}\centering{}{\pb}\textcolor{gray}{\textbf{Lago di Misurina\oindex{Misurinasee@\textbf{Misurinasee}, \emph{See (N.SEE)}|pw} (1755 m), Grand Hôtel\oindex{Grand Hotel Misurina@\textbf{Grand Hotel Misurina}, \emph{Hotel (K.HTL)}|pwv}.}}\pend
           \vspace{1em}
\pstart
           \raggedleft{}{\pb}\label{K_L01699-1v}\edtext{12. 8. 907}{\lemma{\textnormal{\emph{12. 8. 907}}}\Cendnote{\textnormal{Im \emph{Tagebuch}\pwindex{Tagebuch@\emph{Tagebuch}|pwk} wird der Ausflug erst für den 13. 8. 1907 erwähnt, weswegen dieses Datum
                        falsch sein dürfte.}}}\label{K_L01699-1}\pend
           \vspace{0.5em}
\pstart
           In \label{K_L01699-2v}\edtext{Erinnerung}{\lemma{\textnormal{\emph{Erinnerung}}}\Cendnote{\textnormal{Siehe A. S.: \emph{Tagebuch}, 6. 8. 1898.
               }}}\label{K_L01699-2} und herzlichſt,\pend
           
\pstart
           Ihr{\\[\baselineskip]}\spacefill\mbox{Arthur}\pend
           \leftskip=0em{}\selectlanguage{ngerman}\vspace{1em}
\pstart
           \noindent{}{[}hs. :{]} Die herzlichsten Grüsse!\pend
           \pstart \spacefill\mbox{Olga.}\pend{}\selectlanguage{ngerman}\vspace{1em}
\pstart
           \noindent{}{[}hs. :{]} Auf einer Fußwanderung \textsc{Schluderbach}\oindex{Carbonin@\textbf{Carbonin}, \emph{P.PPL}|pw}{ }\textsc{Cortina}\oindex{Cortina DAmpezzo@\textbf{Cortina d’Ampezzo}, \emph{P.PPLA3}|pw}\pend
           
\pstart
           Morgen wieder in \textsc{Welsberg}\oindex{Welsberg-Taisten@\textbf{Welsberg-Taisten}, \emph{A.ADM3}|pw}. –\pend
           \selectlanguage{ngerman}\endnumbering\briefempfaengerindex{Beer-Hofmann, Richard@\textsc{Beer-Hofmann, Richard}!zzzSchnitzler, Olga@\emph{von Olga Schnitzler}!1907-08-131@{13. 8. 1907}|)be}\briefempfaengerindex{Beer-Hofmann, Richard@\textsc{Beer-Hofmann, Richard}!zzzSchnitzler, Arthur@\emph{von Arthur Schnitzler}!1907-08-131@{13. 8. 1907}|)be}\mylabel{L01699h}  \normalsize

\doendnotes{C}
\bigskip
\vfill

\clearpage

\footnotesize

\lohead{\textsc{register}}

% Definiere theindex-Environment komplett neu ohne reledmac
\makeatletter
\renewenvironment{theindex}{%
  \section*{\indexname}%
  \setlength{\parindent}{0pt}%
  \setlength{\parskip}{0pt plus 0.3pt}%
  \let\item\@idxitem
}{%
  \clearpage
}
\makeatother

\IfFileExists{\jobname-pw.ind}{\input{\jobname-pw.ind}}{}

\end{document}

      