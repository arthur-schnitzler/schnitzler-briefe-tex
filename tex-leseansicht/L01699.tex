%% latex-leseansicht-vorspann.tex
%% Vorspann für die Leseansicht.
%% Lädt die gemeinsame Datei latex-vorspann.tex mit nicht gesetztem Schalter.

\newif\ifkorrekturansicht
\korrekturansichtfalse

\input{../tex-inputs/latex-vorspann}


               \section[Arthur und Olga Schnitzler an Richard Beer-Hofmann, 13. 8. 1907]{ Arthur und Olga Schnitzler an Richard Beer-Hofmann, 13. 8. 1907}\nopagebreak\mylabel{v}\rehead{ }\begin{ledgroupsized}[t]{13cm}\normalsize\beginnumbering\briefempfaengerindex{Beer-Hofmann, Richard@\textsc{Beer-Hofmann, Richard}!zzzSchnitzler, Olga@\emph{von Olga Schnitzler}!1907-08-131@{13. 8. 1907}|(be}\briefempfaengerindex{Beer-Hofmann, Richard@\textsc{Beer-Hofmann, Richard}!zzzSchnitzler, Arthur@\emph{von Arthur Schnitzler}!1907-08-131@{13. 8. 1907}|(be} \toendnotes[C]{\smallbreak\pagebreak[2]} \Standort{YCGL, MSS 31.}
\physDesc{Bildpostkarte
\newline{}Handschrift Arthur Schnitzler: Bleistift, deutsche Kurrent\newline{}Handschrift Olga Schnitzler: Bleistift, lateinische Kurrent\newline{}Versand: Stempel: »\nobreak{}\oindex{Misurina@\textbf{Misurina}|pwk}Misurina, 13. 8. 1907\nobreak{}«.  \newline{}Ordnung: mit Bleistift von unbekannter Hand datiert: »12. 8.« }\toendnotes[C]{\smallbreak}\pstart{}{\pb}\textsc{Dr. Richard Beerhofmann}\pend{}\pstart{}Wien\oindex{Wien@\textbf{Wien}|pw}\pend{}\pstart{}\textsc{Hasenauerstr 59\oindex{Hasenauerstrasse@\textbf{Hasenauerstraße}|pw}.}\pend{}\pstart{}\textsc{Austria\oindex{Oesterreich@\textbf{Österreich}|pw}}\pend{}{\bigskip}\pstart
           \noindent{}\centering{}{\pb}\textcolor{gray}{\textbf{Lago di Misurina\oindex{Misurinasee@\textbf{Misurinasee}|pw} (1755 m), Grand Hôtel\oindex{Grand Hotel Misurina@\textbf{Grand Hotel Misurina}|pwv}.}}\pend
           \pstart
           \raggedleft{}{\pb}\label{K_L01699_1v}\edtext{12. 8. 907}{\lemma{\textnormal{\emph{12. 8. 907}}}\Cendnote{\textnormal{Das \emph{Tagebuch}\pwindex{Schnitzler, Arthur 15.05.1862 – 21.10.1931@\textsc{Schnitzler, Arthur} (15.05.1862 – 21.10.1931), \emph{Schriftsteller, Mediziner}!Tagebuch1981 – 2000@\strich\emph{Tagebuch} {[}1981 – 2000{]}|pwk} erwähnt den Ausflug erst für den 13. 8. 1907, weswegen dieses Datum
                        falsch sein dürfte.}}}\label{K_L01699_1h}\pend
           \pstart
           In \label{K_L01699_2v}\edtext{Erinnerung}{\lemma{\textnormal{\emph{Erinnerung}}}\Cendnote{\textnormal{siehe A. S.: \emph{Tagebuch}, 6. 8. 1898}}}\label{K_L01699_2h} und herzlichſt,\pend
           \pstart
           Ihr{\\[\baselineskip]}\spacefill\mbox{Arthur}\pend
           \leftskip=0em{}\pstart
           \noindent{}{[}hs. O. Schnitzler:{]} Die herzlichsten Grüsse!\pend
           \pstart \spacefill\mbox{Olga.}\pend{}\pstart
           \noindent{}{[}hs. Schnitzler:{]} Auf einer Fußwanderung \textsc{Schluderbach}\oindex{Carbonin@\textbf{Carbonin}|pw}{ }\textsc{Cortina}\oindex{Cortina d'Ampezzo@\textbf{Cortina d'Ampezzo}|pw}\pend
           \pstart
           Morgen wieder in \textsc{Welsberg}\oindex{Welsberg-Taisten@\textbf{Welsberg-Taisten}|pw}. –\pend
           \endnumbering\briefempfaengerindex{Beer-Hofmann, Richard@\textsc{Beer-Hofmann, Richard}!zzzSchnitzler, Olga@\emph{von Olga Schnitzler}!1907-08-131@{13. 8. 1907}|)be}\briefempfaengerindex{Beer-Hofmann, Richard@\textsc{Beer-Hofmann, Richard}!zzzSchnitzler, Arthur@\emph{von Arthur Schnitzler}!1907-08-131@{13. 8. 1907}|)be}\mylabel{h}\end{ledgroupsized}  \newcommand{\dateiname}{L01699}\newcommand{\titel}{Arthur und Olga Schnitzler an Richard Beer-Hofmann, 13. 8. 1907}\newcommand{\editorInnen}{Martin Anton Müller und Gerd-Hermann Susen}%% latex-leseansicht-abspann.tex
%% Abspann für die Leseansicht.
%% Der Schalter \ifkorrekturansicht ist bereits durch den Vorspann gesetzt.

%% latex-abspann.tex
%% Gemeinsamer Abspann für Korrekturansicht und Leseansicht.
%% Setzt den Schalter \ifkorrekturansicht voraus (gesetzt in den
%% einbindenden Dateien latex-korrekturansicht-abspann.tex bzw.
%% latex-leseansicht-abspann.tex).
%% ---------------------------------------------------------------

\normalsize

% Das esempio-Environment wird nur in der Leseansicht benötigt
\ifkorrekturansicht\else
\newenvironment{esempio}[3]%
{
    \vspace{1.5ex}
    \rlap{\underline{#1}}
    \par
    \setlength{\parindent}{0cm}
    \nopagebreak
    \leftskip=#2cm
    \rightskip=#3cm
}
{
    \par
}
\fi

\doendnotes{C}
\bigskip
\vfill

\clearpage

\footnotesize

\ifkorrekturansicht
  \lohead{\textsc{register}}
\fi

% theindex-Environment neu definieren ohne reledmac
\makeatletter
\renewenvironment{theindex}{%
  \ifkorrekturansicht
    \section*{\indexname}%
  \else
    \subsubsection*{Index der erwähnten Entitäten}%
  \fi
  \setlength{\parindent}{0pt}%
  \setlength{\parskip}{0pt plus 0.3pt}%
  \let\item\@idxitem
}{%
  \ifkorrekturansicht\clearpage\fi
}
\makeatother

\IfFileExists{\jobname-pw.ind}{\input{\jobname-pw.ind}}{}

% Quellenangabe nur in der Leseansicht
\ifkorrekturansicht\else
% Fallback-Definitionen, falls die .tex-Datei \titel etc. nicht gesetzt hat
\providecommand{\titel}{}
\providecommand{\editorInnen}{}
\providecommand{\dateiname}{\jobname}

\vspace{3cm}

\vfill

\footnotesize
\textsc{Quelle}: \titel. Herausgegeben von {\editorInnen}. In: \emph{Arthur Schnitzler: Briefwechsel mit Autorinnen und Autoren}.
 Digitale Edition, https://schnitzler-briefe.acdh.oeaw.ac.at/{\dateiname}.html (Stand \today)
\fi

\end{document}


      