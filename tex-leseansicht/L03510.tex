%% latex-korrekturansicht-vorspann.tex
%% Vorspann für die Korrekturansicht.
%% Lädt die gemeinsame Datei latex-vorspann.tex mit gesetztem Schalter.

\newif\ifkorrekturansicht
\korrekturansichttrue

\input{../tex-inputs/latex-vorspann}


\section[ Felix Salten an Arthur Schnitzler, 15. 8. 1907]{L03510 Felix Salten an Arthur Schnitzler, 15. 8. 1907}
\nopagebreak\mylabel{L03510v}
\rehead{ }\normalsize\beginnumbering\briefempfaengerindex{Schnitzler, Arthur@\textsc{Schnitzler, Arthur}!zzzSalten, Felix@\emph{von Felix Salten}!1907-08-151@{15. 8. 1907}|(be}
\toendnotes[C]{\smallbreak\pagebreak[2]}\Standort{CUL, Schnitzler, B 89, B 1.}
\physDesc{Brief, 1 Blatt, 3 Seiten, 4005 Zeichen
\newline{}Handschrift: schwarze Tinte, lateinische Kurrent
\newline{}Ordnung: mit Bleistift von unbekannter Hand nummeriert: »233« }\toendnotes[C]{\smallbreak}
\pstart
           \raggedleft{}{\pb}Marienbad\oindex{Marienbad@\textbf{Marienbad}, \emph{P.PPL}|pw}, 15. August 07\pend
           
\pstart
           \raggedleft{}Haus Quisisana\oindex{Hotel Quisisana@\textbf{Hotel Quisisana}, \emph{Hotel (K.HTL)}|pw}.\pend
           \vspace{0.5em}
\pstart
           Lieber, wir sind jetzt bald eine Woche da. Otti\pwindex{Salten, Ottilie 07.03.1868 – 22.06.1942@\textsc{Salten, Ottilie} (07.03.1868 – 22.06.1942), \emph{Schauspieler/Schauspielerin}|pw} braucht die Kur. Kreuzbrunnen\oindex{Kreuzbrunnen@\textbf{Kreuzbrunnen}, \emph{Monument (K.MON)}|pw} und Ferdinandsquelle\oindex{Ferdinandquelle@\textbf{Ferdinandquelle}, \emph{Monument (K.MON)}|pw},
               Moorbäder und Kohlensäure; sie befindet sich dabei sehr wol, und ihre Genesung macht
               sichtlich Fortschritte. Ich habe auch mit einer Kur begonnen, aber nur einen Tag
               ausgehalten. Um 5Uhr aufstehen und um neun erst frühstücken
               könnte ich nur dann vertragen, wenn ich von hier aus erst noch auf vier Wochen
               anderswohin zu Erholung ginge. Da ich mich aber ausruhen muss, hat es keinen Sinn,
               wenn ich mich jetzt quäle, und dann vielleicht noch matter und noch nervöser nach Wien\oindex{Wien@\textbf{Wien}, \emph{A.ADM2}|pw} zurückkomme. Den Kindern\pwindex{Salten, Paul 11.08.1903 – 08.05.1937@\textsc{Salten, Paul} (11.08.1903 – 08.05.1937), \emph{Filmcutter/Filmcutterin}|pwv}\pwindex{Rehmann, Anna Katharina 18.08.1904 – 27.03.1977@\textsc{Rehmann, Anna Katharina} (18.08.1904 – 27.03.1977), \emph{Schauspieler/Schauspielerin, Übersetzer/Übersetzerin}|pwv} tut Mbd.\oindex{Marienbad@\textbf{Marienbad}, \emph{P.PPL}|pw} unglaublich gut. Sie essen hier, dass wir eine Freude
               haben. Und sie lernen endlich weite Spaziergänge machen, was man an der See weniger
               übt, und wozu sie – durch unseren Garten – in Wien\oindex{Wien@\textbf{Wien}, \emph{A.ADM2}|pw}
               nie gelangt sind. Hier sind die Wälder herrlich, und die vielen Jausenorte, die
               überall auf den kleinen Berggipfeln und Hochplateaus liegen, sind wirklich famos. Wir
               wohnen ganz ausserhalb von Marienbad\oindex{Marienbad@\textbf{Marienbad}, \emph{P.PPL}|pw} in einer
               Straße, die nur auf der einen Seite Häuser, auf der anderen den Wald hat, zahlen für
               zwei hübsche Zimmer 25fl die Woche, was sehr billig ist, haben das Mittagessen – und
               was für ein Mittagessen! – für 60 Kreuzer die Person auf dem Zimmer. Das Frühstück
               macht das 
               Fräulein\pwindex{?? [Kinderbetreuerin bei Salten] @\textsc{?? [Kinderbetreuerin bei Salten]}|pw}, gejaust wird
               irgendwo auf einem Berg. (Rübezahl\oindex{Hotel Ruebezahl@\textbf{Hotel Rübezahl}, \emph{Hotel (K.HTL)}|pw}{[},{]}Forstwarte\oindex{Cafe Forstwarte@\textbf{Café Forstwarte}, \emph{Kaffeehaus (K.KAF)}|pw}, Nimrod\oindex{Cafe Nimrod@\textbf{Café Nimrod}, \emph{Kaffeehaus (K.KAF)}|pw}, Egerländer\oindex{Cafe Egerlaender@\textbf{Café Egerländer}, \emph{Kaffeehaus (K.KAF)}|pw} u. s. w.) Und
               Nachtmahl holt man sich in der 
               Delikatessenhandlung\oindex{?? [Delikatessenhandlung in Marienbad]@\textbf{?? [Delikatessenhandlung in Marienbad]}, \emph{Geschäft (K.GES)}|pw}, die hier alle
               Begriffe, die man sich in einer Delikatessenhandlung macht hoch
               übertrifft. Ich verstehe, warum Elias\pwindex{Elias, Julius 12.07.1861 – 02.07.1927@\textsc{Elias, Julius} (12.07.1861 – 02.07.1927), \emph{Übersetzer/Übersetzerin, Publizist/Publizistin}|pw} von Marienbad\oindex{Marienbad@\textbf{Marienbad}, \emph{P.PPL}|pw} so begeistert {\pb}ist. Die Tennisplätze sind die
               schönsten, die ich kenne. Man spielt eine halbe Stunde nach dem Regen. Wir haben eine
               ganz gute Partie, ein taubstummes junges 
               Mädchen\pwindex{?? [Maedchen, das Tennis spielt] @\textsc{?? [Mädchen, das Tennis spielt]}|pwv}, die sehr nett
               ist und sehr scharf spielt. Morgen{ }früh kommt Siegfried Jacobsohn\pwindex{Jacobsohn, Siegfried 28.01.1881 – 03.12.1926@\textsc{Jacobsohn, Siegfried} (28.01.1881 – 03.12.1926), \emph{Journalist/Journalistin, Kritiker/Kritikerin, Publizist/Publizistin}|pw}{ }hier\oindex{Marienbad@\textbf{Marienbad}, \emph{P.PPL}|pwv} an, von den Kindern\pwindex{Salten, Paul 11.08.1903 – 08.05.1937@\textsc{Salten, Paul} (11.08.1903 – 08.05.1937), \emph{Filmcutter/Filmcutterin}|pwv}\pwindex{Rehmann, Anna Katharina 18.08.1904 – 27.03.1977@\textsc{Rehmann, Anna Katharina} (18.08.1904 – 27.03.1977), \emph{Schauspieler/Schauspielerin, Übersetzer/Übersetzerin}|pwv} Onkel Japottsohn\pwindex{Jacobsohn, Siegfried 28.01.1881 – 03.12.1926@\textsc{Jacobsohn, Siegfried} (28.01.1881 – 03.12.1926), \emph{Journalist/Journalistin, Kritiker/Kritikerin, Publizist/Publizistin}|pw} genannt. Er bleibt bis Mittwoch und geht dann nach Wien\oindex{Wien@\textbf{Wien}, \emph{A.ADM2}|pw}. Hier sind natürlich eine Menge Menschen, denen man nicht immer
               ausweichen kann. Wir waren denn auch die ersten Tage in einem Gebrodel von Berlin\oindex{Berlin@\textbf{Berlin}, \emph{P.PPLC}|pw}er, Lemberg\oindex{Lviv@\textbf{Lviv}, \emph{P.PPLA}|pw}er, Wien\oindex{Wien@\textbf{Wien}, \emph{A.ADM2}|pw}er, München\oindex{Muenchen@\textbf{München}, \emph{P.PPLA}|pw}er und Mannheim\oindex{Mannheim@\textbf{Mannheim}, \emph{P.PPLA3}|pw}er
               Leuten, von Wagenfahrten, Automobilpartien, u. s. w. Aber wir haben schnell gebremst
               und leben jetzt ruhig. Wenn Otti\pwindex{Salten, Ottilie 07.03.1868 – 22.06.1942@\textsc{Salten, Ottilie} (07.03.1868 – 22.06.1942), \emph{Schauspieler/Schauspielerin}|pw} nicht früh
               und Abend zum Brunnen müßte, würden wir noch weniger Verkehr haben. Die Kinder\pwindex{Salten, Paul 11.08.1903 – 08.05.1937@\textsc{Salten, Paul} (11.08.1903 – 08.05.1937), \emph{Filmcutter/Filmcutterin}|pwv}\pwindex{Rehmann, Anna Katharina 18.08.1904 – 27.03.1977@\textsc{Rehmann, Anna Katharina} (18.08.1904 – 27.03.1977), \emph{Schauspieler/Schauspielerin, Übersetzer/Übersetzerin}|pwv} trinken Ambrosiusquelle\oindex{Ambrosiusquelle@\textbf{Ambrosiusquelle}, \emph{Monument (K.MON)}|pw} (Eisen){[},{]}
               was immer ein großer Spass ist. Dann fahren sie Eselwagen, und da sie jetzt \label{K_L03510-1v}\edtext{nacheinander Geburtstag}{\lemma{\textnormal{\emph{nacheinander Geburtstag}}}\Cendnote{\textnormal{Paul\pwindex{Salten, Paul 11.08.1903 – 08.05.1937@\textsc{Salten, Paul} (11.08.1903 – 08.05.1937), \emph{Filmcutter/Filmcutterin}|pwk} war am 11. 8. 1907 vier Jahre alt geworden. Annerls\pwindex{Rehmann, Anna Katharina 18.08.1904 – 27.03.1977@\textsc{Rehmann, Anna Katharina} (18.08.1904 – 27.03.1977), \emph{Schauspieler/Schauspielerin, Übersetzer/Übersetzerin}|pwk} dritter Geburtstag stand am 18. 8. 1907 bevor.}}}\label{K_L03510-1} feiern, ist ihr Jubel groß. Annerl\pwindex{Rehmann, Anna Katharina 18.08.1904 – 27.03.1977@\textsc{Rehmann, Anna Katharina} (18.08.1904 – 27.03.1977), \emph{Schauspieler/Schauspielerin, Übersetzer/Übersetzerin}|pw} hat fabelhafte Erfolge, während die tieferen Naturen
                  Pauli\pwindex{Salten, Paul 11.08.1903 – 08.05.1937@\textsc{Salten, Paul} (11.08.1903 – 08.05.1937), \emph{Filmcutter/Filmcutterin}|pw} schätzen. Neulich haben die Kinder\pwindex{Salten, Paul 11.08.1903 – 08.05.1937@\textsc{Salten, Paul} (11.08.1903 – 08.05.1937), \emph{Filmcutter/Filmcutterin}|pwv}\pwindex{Rehmann, Anna Katharina 18.08.1904 – 27.03.1977@\textsc{Rehmann, Anna Katharina} (18.08.1904 – 27.03.1977), \emph{Schauspieler/Schauspielerin, Übersetzer/Übersetzerin}|pwv} im Wald
               Theater gespielt und Rothkäppchen\pwindex{Rotkaeppchen@\emph{Rotkäppchen}|pw} aufgeführt.
               Sie waren förmlich betrunken davon, dass da ein wirklicher Wald war, und man kann
               sagen, dass es auch sonst eine vortreffliche Aufführung gewesen ist. – Wir haben
               manchmal auch schon Schlenther\pwindex{Schlenther, Paul 20.08.1854 – 30.04.1916@\textsc{Schlenther, Paul} (20.08.1854 – 30.04.1916), \emph{Schriftsteller/Schriftstellerin, Kritiker/Kritikerin, Theaterleiter/Theaterleiterin}|pw} gesehen. Er
               sieht aus, als ob er heimliche Balggeschwülste und Drüsen hätte.\pend
           
\pstart
           Hier\oindex{Marienbad@\textbf{Marienbad}, \emph{P.PPL}|pwv} arbeite ich nur
               Kleinigkeiten, die von der Redaction\orgindex{Zeit@Die Zeit|pwv} verlangt werden, sonst nichts. Ich habe in Wien\oindex{Wien@\textbf{Wien}, \emph{A.ADM2}|pw} allerlei gemacht. Darunter die \label{K_L03510-2v}\edtext{drei kleinen Stücke\pwindex{Vom andern Ufer. Einakter@\emph{Vom andern Ufer. Einakter}|pwv}\pwindex{Auferstehung. Komoedie in einem Akt@\emph{Auferstehung. Komödie in einem Akt}|pwv}\pwindex{Graf. Komoedie in einem Akt@\emph{Der Graf. Komödie in einem Akt}|pwv}\pwindex{Ernst des Lebens. Schauspiel in einem Akt@\emph{Der Ernst des Lebens. Schauspiel in einem Akt}|pwv}}{\lemma{\textnormal{\emph{drei kleinen Stücke}}}\Cendnote{\textnormal{\emph{Auferstehung}\pwindex{Auferstehung. Komoedie in einem Akt@\emph{Auferstehung. Komödie in einem Akt}|pwk}, \emph{Der Graf}\pwindex{Graf. Komoedie in einem Akt@\emph{Der Graf. Komödie in einem Akt}|pwk} und \emph{Ernst des
                     Lebens}\pwindex{Ernst des Lebens. Schauspiel in einem Akt@\emph{Der Ernst des Lebens. Schauspiel in einem Akt}|pwk}, versammelt in \emph{Vom andern Ufer}\pwindex{Vom andern Ufer. Einakter@\emph{Vom andern Ufer. Einakter}|pwk}}}}\label{K_L03510-2}, die nun in Maschinschrift vorliegen. Wenn ich sie im Herbst noch erträglich
               finde, \label{K_L03510-3v}\edtext{les’ ich sie vielleicht {\pb}vor}{\lemma{\textnormal{\emph{les’ … vor}}}\Cendnote{\textnormal{Schnitzler bekam sie nicht vorgelesen, sondern las sie am 5. 10. 1907.}}}\label{K_L03510-3}. Im September schreibe ich den »Hund v. Florenz\pwindex{Hund von Florenz@\emph{Der Hund von Florenz}|pw}«. Er ist jetzt ganz fertig dazu
               und vielfach verändert. Könnte ich die Zeitung los sein, wäre ich froh und vermöchte
               vielleicht einiges Gute zustande zu bringen. Mir wird die Zeitungschreiberei immer
               leerer und leerer. Bin ich wirklich im September mit dem
                  »Hund\pwindex{Hund von Florenz@\emph{Der Hund von Florenz}|pw}« fertig, dann mache ich die Seereise.
               Der Gardasee\oindex{Lago di Garda@\textbf{Lago di Garda}, \emph{See (N.SEE)}|pw} genügt mir davor wirklich nicht.
               Im Übrigen wissen Sie ja, wie es mit meinen Plänen geht. Von zwanzig projektirten
               Reisen werden zwei verwirklicht. Am 1. Septbr. bin ich
               jedenfalls in Wien\oindex{Wien@\textbf{Wien}, \emph{A.ADM2}|pw}. Vorher zwei, drei, Tage Semmering\oindex{Semmering@\textbf{Semmering}, \emph{A.ADM3}|pw} oder Schneeberg\oindex{Schneeberg@\textbf{Schneeberg}, \emph{Berg (N.BRG)}|pw}.\pend
           
\pstart
           Auf Wiedersehen, und viele herzliche Grüße von uns\pwindex{Salten, Ottilie 07.03.1868 – 22.06.1942@\textsc{Salten, Ottilie} (07.03.1868 – 22.06.1942), \emph{Schauspieler/Schauspielerin}|pwv} zu Ihnen. Schreiben Sie mir bald
               wieder. {\\[\baselineskip]}Aufrichtig {\\[\baselineskip]}Ihr {\\[\baselineskip]}\spacefill\mbox{Salten}\pend
           \leftskip=0em{}
\pstart
           \noindent{}Hier das \label{K_L03510-4v}\edtext{Feuill.\pwindex{Wiener Korrespondent@\emph{Der Wiener Korrespondent}|pwv}}{\lemma{\textnormal{\emph{Feuill.}}}\Cendnote{\textnormal{Felix Salten\pwindex{Salten, Felix 06.09.1869 – 08.10.1945@\textsc{Salten, Felix} (06.09.1869 – 08.10.1945), \emph{Schriftsteller/Schriftstellerin, Journalist/Journalistin, Chefredakteur/Chefredakteurin}|pwk}: \emph{Der Wiener Korrespondent}\pwindex{Wiener Korrespondent@\emph{Der Wiener Korrespondent}|pwk}. In: \emph{Der Morgen}\pwindex{Morgen. Wochenschrift fuer deutsche Kultur@\emph{Morgen. Wochenschrift für deutsche Kultur}|pwk}, Jg. 1, H. 4, 5. 7. 1907, S. 113–116. Vgl. Arthur Schnitzler an Felix Salten, 5. 8. 1907.}}}\label{K_L03510-4} aus dem »Morgen\pwindex{Morgen. Wochenschrift fuer deutsche Kultur@\emph{Morgen. Wochenschrift für deutsche Kultur}|pw}« das Sie wünschten. Die \label{K_L03510-5v}\edtext{»engl. Reise\pwindex{?? [Englische Reise]@\emph{?? [Englische Reise]}|pw}«}{\lemma{\textnormal{\emph{»engl. Reise«}}}\Cendnote{\textnormal{Nicht ermittelt; womöglich handelt es sich um einen Teil der in Schnitzlers Brief vom 5. 8. 1907 erwähnten
                     Feuilletonsammlung?}}}\label{K_L03510-5} suche ich selbst schon seit
                  Monaten vergebens. Sonst hätten Sie sie schon. Pötzl\pwindex{gelobte Wien@\emph{Das gelobte Wien}|pwv}\pwindex{Poetzl, Eduard 17.03.1851 – 20.08.1914@\textsc{Pötzl, Eduard} (17.03.1851 – 20.08.1914), \emph{Schriftsteller/Schriftstellerin, Journalist/Journalistin}|pw} habe ich nicht zur Hand.\pend
           \selectlanguage{ngerman}\endnumbering\briefempfaengerindex{Schnitzler, Arthur@\textsc{Schnitzler, Arthur}!zzzSalten, Felix@\emph{von Felix Salten}!1907-08-151@{15. 8. 1907}|)be}\mylabel{L03510h}  \normalsize

\doendnotes{C}
\bigskip
\vfill

\clearpage

\footnotesize

\lohead{\textsc{register}}

% Definiere theindex-Environment komplett neu ohne reledmac
\makeatletter
\renewenvironment{theindex}{%
  \section*{\indexname}%
  \setlength{\parindent}{0pt}%
  \setlength{\parskip}{0pt plus 0.3pt}%
  \let\item\@idxitem
}{%
  \clearpage
}
\makeatother

\IfFileExists{\jobname-pw.ind}{\input{\jobname-pw.ind}}{}

\end{document}

      