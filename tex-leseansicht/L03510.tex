%% latex-leseansicht-vorspann.tex
%% Vorspann für die Leseansicht.
%% Lädt die gemeinsame Datei latex-vorspann.tex mit nicht gesetztem Schalter.

\newif\ifkorrekturansicht
\korrekturansichtfalse

\input{../tex-inputs/latex-vorspann}


\section[ Felix Salten an Arthur Schnitzler, 15. 8. 1907]{L03510 Felix Salten an Arthur Schnitzler,  15. 8. 1907}
\nopagebreak\mylabel{L03510v}
\rehead{ }\normalsize\beginnumbering\briefempfaengerindex{Schnitzler, Arthur@\textsc{Schnitzler, Arthur}!zzzSalten, Felix@\emph{von Felix Salten}!1907-08-151@{15. 8. 1907}|(be}
\toendnotes[C]{\smallbreak\pagebreak[2]}
\correspDesc{Versand  durch Felix Salten am 15. 8. 1907 in Marienbad
\newline{}Erhalt  durch Arthur Schnitzler im Zeitraum [16. 8. 1907
                  – 20. 8. 1907?] in Welsberg-Taisten}\toendnotes[C]{\smallbreak}
\Standort{CUL, Schnitzler, B 89, B 1.}
\physDesc{Brief, 1 Blatt, 3 Seiten, 4005 Zeichen
\newline{}Handschrift: schwarze Tinte, lateinische Kurrent
\newline{}Ordnung: mit Bleistift von unbekannter Hand nummeriert: »233« }\toendnotes[C]{\smallbreak}
\pstart
           \raggedleft{}{\pb}Marienbad\oindex{Marienbad@\textbf{Marienbad}|pw}, 15. August 07\pend
           
\pstart
           \raggedleft{}Haus Quisisana\oindex{Hotel Quisisana@\textbf{Hotel Quisisana}, \emph{Hotel}|pw}.\pend
           \vspace{0.5em}
\pstart
           Lieber, wir sind jetzt bald eine Woche da. Otti\pwindex{Salten, Ottilie 7.\,3.\,1868 Prag – 22.\,6.\,1942 Zürich@\textsc{Salten, Ottilie} (7.\,3.\,1868 Prag – 22.\,6.\,1942 Zürich), \emph{Schauspielerin}|pw} braucht die Kur. Kreuzbrunnen\oindex{Kreuzbrunnen@\textbf{Kreuzbrunnen}, \emph{Monument}|pw} und Ferdinandsquelle\oindex{Ferdinandquelle@\textbf{Ferdinandquelle}, \emph{Monument}|pw},
               Moorbäder und Kohlensäure; sie befindet sich dabei sehr wol, und ihre Genesung macht
               sichtlich Fortschritte. Ich habe auch mit einer Kur begonnen, aber nur einen Tag
               ausgehalten. Um 5Uhr aufstehen und um neun erst frühstücken
               könnte ich nur dann vertragen, wenn ich von hier aus erst noch auf vier Wochen
               anderswohin zu Erholung ginge. Da ich mich aber ausruhen muss, hat es keinen Sinn,
               wenn ich mich jetzt quäle, und dann vielleicht noch matter und noch nervöser nach Wien\oindex{Wien@\textbf{Wien}, \emph{Verwaltungsgebiet}|pw} zurückkomme. Den Kindern\pwindex{Salten, Paul 11.\,8.\,1903 Wien – 8.\,5.\,1937 ebd.@\textsc{Salten, Paul} (11.\,8.\,1903 Wien – 8.\,5.\,1937 ebd.), \emph{Filmcutter}|pwv}\pwindex{Rehmann, Anna Katharina 18.\,8.\,1904 Wien – 27.\,3.\,1977 Zürich@\textsc{Rehmann, Anna Katharina} (18.\,8.\,1904 Wien – 27.\,3.\,1977 Zürich), \emph{Schauspielerin, Übersetzerin}|pwv} tut Mbd.\oindex{Marienbad@\textbf{Marienbad}|pw} unglaublich gut. Sie essen hier, dass wir eine Freude
               haben. Und sie lernen endlich weite Spaziergänge machen, was man an der See weniger
               übt, und wozu sie – durch unseren Garten – in Wien\oindex{Wien@\textbf{Wien}, \emph{Verwaltungsgebiet}|pw}
               nie gelangt sind. Hier sind die Wälder herrlich, und die vielen Jausenorte, die
               überall auf den kleinen Berggipfeln und Hochplateaus liegen, sind wirklich famos. Wir
               wohnen ganz ausserhalb von Marienbad\oindex{Marienbad@\textbf{Marienbad}|pw} in einer
               Straße, die nur auf der einen Seite Häuser, auf der anderen den Wald hat, zahlen für
               zwei hübsche Zimmer 25fl die Woche, was sehr billig ist, haben das Mittagessen – und
               was für ein Mittagessen! – für 60 Kreuzer die Person auf dem Zimmer. Das Frühstück
               macht das 
               Fräulein\pwindex{?? [Kinderbetreuerin bei Salten] @\textsc{?? [Kinderbetreuerin bei Salten]}|pw}, gejaust wird
               irgendwo auf einem Berg. (Rübezahl\oindex{Hotel Rübezahl@\textbf{Hotel Rübezahl}, \emph{Hotel}|pw}{[},{]}{ }Forstwarte\oindex{Café Forstwarte@\textbf{Café Forstwarte}, \emph{Kaffeehaus}|pw}, Nimrod\oindex{Café Nimrod@\textbf{Café Nimrod}, \emph{Kaffeehaus}|pw}, Egerländer\oindex{Café Egerländer@\textbf{Café Egerländer}, \emph{Kaffeehaus}|pw} u. s. w.) Und
               Nachtmahl holt man sich in der 
               Delikatessenhandlung\oindex{?? [Delikatessenhandlung in Marienbad]@\textbf{?? [Delikatessenhandlung in Marienbad]}, \emph{Geschäft}|pw}, die hier alle
               Begriffe, die man sich in einer Delikatessenhandlung macht hoch
               übertrifft. Ich verstehe, warum Elias\pwindex{Elias, Julius 12.\,7.\,1861 Hoya – 2.\,7.\,1927 Berlin@\textsc{Elias, Julius} (12.\,7.\,1861 Hoya – 2.\,7.\,1927 Berlin), \emph{Übersetzer, Publizist}|pw} von Marienbad\oindex{Marienbad@\textbf{Marienbad}|pw} so begeistert {\pb}ist. Die Tennisplätze sind die
               schönsten, die ich kenne. Man spielt eine halbe Stunde nach dem Regen. Wir haben eine
               ganz gute Partie, ein taubstummes junges 
               Mädchen\pwindex{?? [Mädchen, das Tennis spielt] @\textsc{?? [Mädchen, das Tennis spielt]}|pwv}, die sehr nett
               ist und sehr scharf spielt. Morgen{ }früh kommt Siegfried Jacobsohn\pwindex{Jacobsohn, Siegfried 28.\,1.\,1881 Berlin – 3.\,12.\,1926 ebd.@\textsc{Jacobsohn, Siegfried} (28.\,1.\,1881 Berlin – 3.\,12.\,1926 ebd.), \emph{Journalist, Kritiker, Publizist}|pw}{ }hier\oindex{Marienbad@\textbf{Marienbad}|pwv} an, von den Kindern\pwindex{Salten, Paul 11.\,8.\,1903 Wien – 8.\,5.\,1937 ebd.@\textsc{Salten, Paul} (11.\,8.\,1903 Wien – 8.\,5.\,1937 ebd.), \emph{Filmcutter}|pwv}\pwindex{Rehmann, Anna Katharina 18.\,8.\,1904 Wien – 27.\,3.\,1977 Zürich@\textsc{Rehmann, Anna Katharina} (18.\,8.\,1904 Wien – 27.\,3.\,1977 Zürich), \emph{Schauspielerin, Übersetzerin}|pwv} Onkel Japottsohn\pwindex{Jacobsohn, Siegfried 28.\,1.\,1881 Berlin – 3.\,12.\,1926 ebd.@\textsc{Jacobsohn, Siegfried} (28.\,1.\,1881 Berlin – 3.\,12.\,1926 ebd.), \emph{Journalist, Kritiker, Publizist}|pw} genannt. Er bleibt bis Mittwoch und geht dann nach Wien\oindex{Wien@\textbf{Wien}, \emph{Verwaltungsgebiet}|pw}. Hier sind natürlich eine Menge Menschen, denen man nicht immer
               ausweichen kann. Wir waren denn auch die ersten Tage in einem Gebrodel von Berlin\oindex{Berlin@\textbf{Berlin}, \emph{Hauptstadt}|pw}er, Lemberg\oindex{Lviv@\textbf{Lviv}|pw}er, Wien\oindex{Wien@\textbf{Wien}, \emph{Verwaltungsgebiet}|pw}er, München\oindex{München@\textbf{München}|pw}er und Mannheim\oindex{Mannheim@\textbf{Mannheim}, \emph{Hauptstadt}|pw}er
               Leuten, von Wagenfahrten, Automobilpartien, u. s. w. Aber wir haben schnell gebremst
               und leben jetzt ruhig. Wenn Otti\pwindex{Salten, Ottilie 7.\,3.\,1868 Prag – 22.\,6.\,1942 Zürich@\textsc{Salten, Ottilie} (7.\,3.\,1868 Prag – 22.\,6.\,1942 Zürich), \emph{Schauspielerin}|pw} nicht früh
               und Abend zum Brunnen müßte, würden wir noch weniger Verkehr haben. Die Kinder\pwindex{Salten, Paul 11.\,8.\,1903 Wien – 8.\,5.\,1937 ebd.@\textsc{Salten, Paul} (11.\,8.\,1903 Wien – 8.\,5.\,1937 ebd.), \emph{Filmcutter}|pwv}\pwindex{Rehmann, Anna Katharina 18.\,8.\,1904 Wien – 27.\,3.\,1977 Zürich@\textsc{Rehmann, Anna Katharina} (18.\,8.\,1904 Wien – 27.\,3.\,1977 Zürich), \emph{Schauspielerin, Übersetzerin}|pwv} trinken Ambrosiusquelle\oindex{Ambrosiusquelle@\textbf{Ambrosiusquelle}, \emph{Monument}|pw} (Eisen){[},{]}
               was immer ein großer Spass ist. Dann fahren sie Eselwagen, und da sie jetzt \label{K_L03510-1v}\edtext{nacheinander Geburtstag}{\lemma{\textnormal{\emph{nacheinander Geburtstag}}}\Cendnote{\textnormal{Paul\pwindex{Salten, Paul 11.\,8.\,1903 Wien – 8.\,5.\,1937 ebd.@\textsc{Salten, Paul} (11.\,8.\,1903 Wien – 8.\,5.\,1937 ebd.), \emph{Filmcutter}|pwk} war am 11. 8. 1907 vier Jahre alt geworden. Annerls\pwindex{Rehmann, Anna Katharina 18.\,8.\,1904 Wien – 27.\,3.\,1977 Zürich@\textsc{Rehmann, Anna Katharina} (18.\,8.\,1904 Wien – 27.\,3.\,1977 Zürich), \emph{Schauspielerin, Übersetzerin}|pwk} dritter Geburtstag stand am 18. 8. 1907 bevor.}}}\label{K_L03510-1} feiern, ist ihr Jubel groß. Annerl\pwindex{Rehmann, Anna Katharina 18.\,8.\,1904 Wien – 27.\,3.\,1977 Zürich@\textsc{Rehmann, Anna Katharina} (18.\,8.\,1904 Wien – 27.\,3.\,1977 Zürich), \emph{Schauspielerin, Übersetzerin}|pw} hat fabelhafte Erfolge, während die tieferen Naturen
                  Pauli\pwindex{Salten, Paul 11.\,8.\,1903 Wien – 8.\,5.\,1937 ebd.@\textsc{Salten, Paul} (11.\,8.\,1903 Wien – 8.\,5.\,1937 ebd.), \emph{Filmcutter}|pw} schätzen. Neulich haben die Kinder\pwindex{Salten, Paul 11.\,8.\,1903 Wien – 8.\,5.\,1937 ebd.@\textsc{Salten, Paul} (11.\,8.\,1903 Wien – 8.\,5.\,1937 ebd.), \emph{Filmcutter}|pwv}\pwindex{Rehmann, Anna Katharina 18.\,8.\,1904 Wien – 27.\,3.\,1977 Zürich@\textsc{Rehmann, Anna Katharina} (18.\,8.\,1904 Wien – 27.\,3.\,1977 Zürich), \emph{Schauspielerin, Übersetzerin}|pwv} im Wald
               Theater gespielt und Rothkäppchen\pwindex{Rotkäppchen@\emph{Rotkäppchen}|pw} aufgeführt.
               Sie waren förmlich betrunken davon, dass da ein wirklicher Wald war, und man kann
               sagen, dass es auch sonst eine vortreffliche Aufführung gewesen ist. – Wir haben
               manchmal auch schon Schlenther\pwindex{Schlenther, Paul 20.\,8.\,1854 Chernyakhovsk – 30.\,4.\,1916 Berlin@\textsc{Schlenther, Paul} (20.\,8.\,1854 Chernyakhovsk – 30.\,4.\,1916 Berlin), \emph{Schriftsteller, Kritiker, Theaterleiter}|pw} gesehen. Er
               sieht aus, als ob er heimliche Balggeschwülste und Drüsen hätte.\pend
           
\pstart
           Hier\oindex{Marienbad@\textbf{Marienbad}|pwv} arbeite ich nur
               Kleinigkeiten, die von der Redaction\orgindex{Zeit@Die Zeit|pwv} verlangt werden, sonst nichts. Ich habe in Wien\oindex{Wien@\textbf{Wien}, \emph{Verwaltungsgebiet}|pw} allerlei gemacht. Darunter die \label{K_L03510-2v}\edtext{drei kleinen Stücke\pwindex{Salten, Felix 6.\,9.\,1869 Budapest – 8.\,10.\,1945 Zürich@\textsc{Salten, Felix} (6.\,9.\,1869 Budapest – 8.\,10.\,1945 Zürich), \emph{Schriftsteller, Journalist, Chefredakteur}!Vom andern Ufer. Einakter@\strich\emph{Vom andern Ufer. Einakter}|pwv}\pwindex{Salten, Felix 6.\,9.\,1869 Budapest – 8.\,10.\,1945 Zürich@\textsc{Salten, Felix} (6.\,9.\,1869 Budapest – 8.\,10.\,1945 Zürich), \emph{Schriftsteller, Journalist, Chefredakteur}!Auferstehung. Komödie in einem Akt@\strich\emph{Auferstehung. Komödie in einem Akt}|pwv}\pwindex{Salten, Felix 6.\,9.\,1869 Budapest – 8.\,10.\,1945 Zürich@\textsc{Salten, Felix} (6.\,9.\,1869 Budapest – 8.\,10.\,1945 Zürich), \emph{Schriftsteller, Journalist, Chefredakteur}!Graf. Komödie in einem Akt@\strich\emph{Der Graf. Komödie in einem Akt}|pwv}\pwindex{Salten, Felix 6.\,9.\,1869 Budapest – 8.\,10.\,1945 Zürich@\textsc{Salten, Felix} (6.\,9.\,1869 Budapest – 8.\,10.\,1945 Zürich), \emph{Schriftsteller, Journalist, Chefredakteur}!Ernst des Lebens. Schauspiel in einem Akt@\strich\emph{Der Ernst des Lebens. Schauspiel in einem Akt}|pwv}}{\lemma{\textnormal{\emph{drei kleinen Stücke}}}\Cendnote{\textnormal{\emph{Auferstehung}\pwindex{Salten, Felix 6.\,9.\,1869 Budapest – 8.\,10.\,1945 Zürich@\textsc{Salten, Felix} (6.\,9.\,1869 Budapest – 8.\,10.\,1945 Zürich), \emph{Schriftsteller, Journalist, Chefredakteur}!Auferstehung. Komödie in einem Akt@\strich\emph{Auferstehung. Komödie in einem Akt}|pwk}, \emph{Der Graf}\pwindex{Salten, Felix 6.\,9.\,1869 Budapest – 8.\,10.\,1945 Zürich@\textsc{Salten, Felix} (6.\,9.\,1869 Budapest – 8.\,10.\,1945 Zürich), \emph{Schriftsteller, Journalist, Chefredakteur}!Graf. Komödie in einem Akt@\strich\emph{Der Graf. Komödie in einem Akt}|pwk} und \emph{Ernst des
                     Lebens}\pwindex{Salten, Felix 6.\,9.\,1869 Budapest – 8.\,10.\,1945 Zürich@\textsc{Salten, Felix} (6.\,9.\,1869 Budapest – 8.\,10.\,1945 Zürich), \emph{Schriftsteller, Journalist, Chefredakteur}!Ernst des Lebens. Schauspiel in einem Akt@\strich\emph{Der Ernst des Lebens. Schauspiel in einem Akt}|pwk}, versammelt in \emph{Vom andern Ufer}\pwindex{Salten, Felix 6.\,9.\,1869 Budapest – 8.\,10.\,1945 Zürich@\textsc{Salten, Felix} (6.\,9.\,1869 Budapest – 8.\,10.\,1945 Zürich), \emph{Schriftsteller, Journalist, Chefredakteur}!Vom andern Ufer. Einakter@\strich\emph{Vom andern Ufer. Einakter}|pwk}}}}\label{K_L03510-2}, die nun in Maschinschrift vorliegen. Wenn ich sie im Herbst noch erträglich
               finde, \label{K_L03510-3v}\edtext{les’ ich sie vielleicht {\pb}vor}{\lemma{\textnormal{\emph{les’ … vor}}}\Cendnote{\textnormal{Schnitzler bekam sie nicht vorgelesen, sondern las sie am 5. 10. 1907.}}}\label{K_L03510-3}. Im September schreibe ich den »Hund v. Florenz\pwindex{Salten, Felix 6.\,9.\,1869 Budapest – 8.\,10.\,1945 Zürich@\textsc{Salten, Felix} (6.\,9.\,1869 Budapest – 8.\,10.\,1945 Zürich), \emph{Schriftsteller, Journalist, Chefredakteur}!Hund von Florenz@\strich\emph{Der Hund von Florenz}|pw}«. Er ist jetzt ganz fertig dazu
               und vielfach verändert. Könnte ich die Zeitung los sein, wäre ich froh und vermöchte
               vielleicht einiges Gute zustande zu bringen. Mir wird die Zeitungschreiberei immer
               leerer und leerer. Bin ich wirklich im September mit dem
                  »Hund\pwindex{Salten, Felix 6.\,9.\,1869 Budapest – 8.\,10.\,1945 Zürich@\textsc{Salten, Felix} (6.\,9.\,1869 Budapest – 8.\,10.\,1945 Zürich), \emph{Schriftsteller, Journalist, Chefredakteur}!Hund von Florenz@\strich\emph{Der Hund von Florenz}|pw}« fertig, dann mache ich die Seereise.
               Der Gardasee\oindex{Lago di Garda@\textbf{Lago di Garda}, \emph{See}|pw} genügt mir davor wirklich nicht.
               Im Übrigen wissen Sie ja, wie es mit meinen Plänen geht. Von zwanzig projektirten
               Reisen werden zwei verwirklicht. Am 1. Septbr. bin ich
               jedenfalls in Wien\oindex{Wien@\textbf{Wien}, \emph{Verwaltungsgebiet}|pw}. Vorher zwei, drei, Tage Semmering\oindex{Semmering@\textbf{Semmering}, \emph{Verwaltungsgebiet}|pw} oder Schneeberg\oindex{Schneeberg@\textbf{Schneeberg}, \emph{Berg}|pw}.\pend
           
\pstart
           Auf Wiedersehen, und viele herzliche Grüße von uns\pwindex{Salten, Ottilie 7.\,3.\,1868 Prag – 22.\,6.\,1942 Zürich@\textsc{Salten, Ottilie} (7.\,3.\,1868 Prag – 22.\,6.\,1942 Zürich), \emph{Schauspielerin}|pwv} zu Ihnen. Schreiben Sie mir bald
               wieder. {\\[\baselineskip]}Aufrichtig {\\[\baselineskip]}Ihr {\\[\baselineskip]}\spacefill\mbox{Salten}\pend
           \leftskip=0em{}
\pstart
           \noindent{}Hier das \label{K_L03510-4v}\edtext{Feuill.\pwindex{Salten, Felix 6.\,9.\,1869 Budapest – 8.\,10.\,1945 Zürich@\textsc{Salten, Felix} (6.\,9.\,1869 Budapest – 8.\,10.\,1945 Zürich), \emph{Schriftsteller, Journalist, Chefredakteur}!Wiener Korrespondent@\strich\emph{Der Wiener Korrespondent}|pwv}}{\lemma{\textnormal{\emph{Feuill.}}}\Cendnote{\textnormal{Felix Salten\pwindex{Salten, Felix 6.\,9.\,1869 Budapest – 8.\,10.\,1945 Zürich@\textsc{Salten, Felix} (6.\,9.\,1869 Budapest – 8.\,10.\,1945 Zürich), \emph{Schriftsteller, Journalist, Chefredakteur}|pwk}: \emph{Der Wiener Korrespondent}\pwindex{Salten, Felix 6.\,9.\,1869 Budapest – 8.\,10.\,1945 Zürich@\textsc{Salten, Felix} (6.\,9.\,1869 Budapest – 8.\,10.\,1945 Zürich), \emph{Schriftsteller, Journalist, Chefredakteur}!Wiener Korrespondent@\strich\emph{Der Wiener Korrespondent}|pwk}. In: \emph{Der Morgen}\pwindex{Morgen. Wochenschrift für deutsche Kultur@\emph{Morgen. Wochenschrift für deutsche Kultur}|pwk}, Jg. 1, H. 4, 5. 7. 1907, S. 113–116. Vgl. XXXX Auszeichnungsfehler: Dokument L03009 nicht gefunden.}}}\label{K_L03510-4} aus dem »Morgen\pwindex{Morgen. Wochenschrift für deutsche Kultur@\emph{Morgen. Wochenschrift für deutsche Kultur}|pw}« das Sie wünschten. Die \label{K_L03510-5v}\edtext{»engl. Reise\pwindex{Salten, Felix 6.\,9.\,1869 Budapest – 8.\,10.\,1945 Zürich@\textsc{Salten, Felix} (6.\,9.\,1869 Budapest – 8.\,10.\,1945 Zürich), \emph{Schriftsteller, Journalist, Chefredakteur}!?? [Englische Reise]@\strich\emph{?? [Englische Reise]}|pw}«}{\lemma{\textnormal{\emph{»engl. Reise«}}}\Cendnote{\textnormal{Nicht ermittelt; womöglich handelt es sich um einen Teil der in Schnitzlers Brief vom XXXX Auszeichnungsfehler: Dokument L03009 nicht gefunden erwähnten
                     Feuilletonsammlung?}}}\label{K_L03510-5} suche ich selbst schon seit
                  Monaten vergebens. Sonst hätten Sie sie schon. Pötzl\pwindex{Pötzl, Eduard 17.\,3.\,1851 Wien – 20.\,8.\,1914 Mödling@\textsc{Pötzl, Eduard} (17.\,3.\,1851 Wien – 20.\,8.\,1914 Mödling), \emph{Schriftsteller, Journalist}!gelobte Wien@\strich\emph{Das gelobte Wien}|pwv}\pwindex{Pötzl, Eduard 17.\,3.\,1851 Wien – 20.\,8.\,1914 Mödling@\textsc{Pötzl, Eduard} (17.\,3.\,1851 Wien – 20.\,8.\,1914 Mödling), \emph{Schriftsteller, Journalist}|pw} habe ich nicht zur Hand.\pend
           \selectlanguage{ngerman}\endnumbering\briefempfaengerindex{Schnitzler, Arthur@\textsc{Schnitzler, Arthur}!zzzSalten, Felix@\emph{von Felix Salten}!1907-08-151@{15. 8. 1907}|)be}\mylabel{L03510h}  \newcommand{\dateiname}{L03510}\newcommand{\titel}{Felix Salten an Arthur Schnitzler, 15. 8. 1907}\newcommand{\editorInnen}{Martin Anton Müller und Laura Untner}%% latex-leseansicht-abspann.tex
%% Abspann für die Leseansicht.
%% Der Schalter \ifkorrekturansicht ist bereits durch den Vorspann gesetzt.

%% latex-abspann.tex
%% Gemeinsamer Abspann für Korrekturansicht und Leseansicht.
%% Setzt den Schalter \ifkorrekturansicht voraus (gesetzt in den
%% einbindenden Dateien latex-korrekturansicht-abspann.tex bzw.
%% latex-leseansicht-abspann.tex).
%% ---------------------------------------------------------------

\normalsize

% Das esempio-Environment wird nur in der Leseansicht benötigt
\ifkorrekturansicht\else
\newenvironment{esempio}[3]%
{
    \vspace{1.5ex}
    \rlap{\underline{#1}}
    \par
    \setlength{\parindent}{0cm}
    \nopagebreak
    \leftskip=#2cm
    \rightskip=#3cm
}
{
    \par
}
\fi

\doendnotes{C}
\bigskip
\vfill

\clearpage

\footnotesize

\ifkorrekturansicht
  \lohead{\textsc{register}}
\fi

% theindex-Environment neu definieren ohne reledmac
\makeatletter
\renewenvironment{theindex}{%
  \ifkorrekturansicht
    \section*{\indexname}%
  \else
    \subsubsection*{Index der erwähnten Entitäten}%
  \fi
  \setlength{\parindent}{0pt}%
  \setlength{\parskip}{0pt plus 0.3pt}%
  \let\item\@idxitem
}{%
  \ifkorrekturansicht\clearpage\fi
}
\makeatother

\IfFileExists{\jobname-pw.ind}{\input{\jobname-pw.ind}}{}

% Quellenangabe nur in der Leseansicht
\ifkorrekturansicht\else
% Fallback-Definitionen, falls die .tex-Datei \titel etc. nicht gesetzt hat
\providecommand{\titel}{}
\providecommand{\editorInnen}{}
\providecommand{\dateiname}{\jobname}

\vspace{3cm}

\vfill

\footnotesize
\textsc{Quelle}: \titel. Herausgegeben von {\editorInnen}. In: \emph{Arthur Schnitzler: Briefwechsel mit Autorinnen und Autoren}.
 Digitale Edition, https://schnitzler-briefe.acdh.oeaw.ac.at/{\dateiname}.html (Stand \today)
\fi

\end{document}


