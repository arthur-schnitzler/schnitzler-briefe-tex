%% latex-leseansicht-vorspann.tex
%% Vorspann für die Leseansicht.
%% Lädt die gemeinsame Datei latex-vorspann.tex mit nicht gesetztem Schalter.

\newif\ifkorrekturansicht
\korrekturansichtfalse

\input{../tex-inputs/latex-vorspann}

\begin{center}
            \textcolor{red}{ENTWURF, NICHT FERTIG KORRIGIERT}
                      \end{center}
            
         
         \renewcommand{\erwaehntePersonen}{Personen: Felix Salten, Ottilie Salten}
         \renewcommand{\erwaehnteOrte}{Orte: Cottagegasse, Wien}
         \renewcommand{\erwaehnteWerke}{}
               \section[Ottilie und Felix Salten an Arthur Schnitzler, {[}Anfang April 1927?{]}]{ Ottilie und Felix Salten an Arthur Schnitzler, {[}Anfang
               April 1927?{]}}\nopagebreak\mylabel{v}\rehead{ }\begin{ledgroupsized}[t]{13cm}\normalsize\beginnumbering \toendnotes[C]{\smallbreak\pagebreak[2]} \Standort{CUL, Schnitzler, B 89, B 2.}
\physDesc{Briefkarte, 213 Zeichen
\newline{}Handschrift Ottilie Salten: schwarze Tinte, lateinische Kurrent\newline{}Handschrift Felix Salten: schwarze Tinte, lateinische Kurrent}\toendnotes[C]{\smallbreak}\pstart
           \noindent{}\centering{}{\pb}\textcolor{gray}{\textbf{HERR UND FRAU FELIX
                        SALTEN}}\pend
           \pstart
           \noindent{}\textcolor{gray}{\textbf{BITTEN}} Herrn D\textsuperscript{r}\pend
           \pstart
           Arthur Schnitzler\pend
           \pstart
           \textcolor{gray}{\textbf{für}}{ }Mittwoch \textcolor{gray}{\textbf{DEN}} 13. April\textcolor{gray}{\textbf{192}}\pend
           \pstart
           zum Thee\pend
           \pstart
           5–8 \textcolor{gray}{\textbf{UHR}}.\pend
           \pstart
           \textcolor{gray}{\textbf{XVIII., COTTAGEGASSE 37\oindex{Cottagegasse@\textbf{Cottagegasse}|pw}}}\hfill \textcolor{gray}{\textbf{\label{K_L03581-1v}\edtext{U.A.W.G.}{\lemma{\textnormal{\emph{u.A.w.g.}}}\Cendnote{\textnormal{um Antwort wird gebeten}}}\label{K_L03581-1h}}}\pend
           \pstart
           {[}hs. Felix Salten:{]} Lieber, wir werden uns freuen, Sie
                  \label{K_L03581-2v}\edtext{an diesem Tag bei uns zu haben}{\lemma{\textnormal{\emph{an … haben}}}\Cendnote{\textnormal{siehe A. S.: \emph{Tagebuch}, 13. 4. 1927}}}\label{K_L03581-2h} – vielleicht, wenn Sie das vorziehen, kommen Sie gegen 7\textsuperscript{h} und bleiben zum Nachtmahl!!\pend
           \pstart
           Herzlich {\\[\baselineskip]}\spacefill\mbox{F. S.}\pend
           \leftskip=0em{}
         
         \endnumbering\mylabel{h}\end{ledgroupsized}  \newcommand{\dateiname}{L03581}\newcommand{\titel}{Ottilie und Felix Salten an Arthur Schnitzler, [Anfang April 1927?]}\newcommand{\editorInnen}{Martin Anton Müller und Laura Untner}%% latex-leseansicht-abspann.tex
%% Abspann für die Leseansicht.
%% Der Schalter \ifkorrekturansicht ist bereits durch den Vorspann gesetzt.

%% latex-abspann.tex
%% Gemeinsamer Abspann für Korrekturansicht und Leseansicht.
%% Setzt den Schalter \ifkorrekturansicht voraus (gesetzt in den
%% einbindenden Dateien latex-korrekturansicht-abspann.tex bzw.
%% latex-leseansicht-abspann.tex).
%% ---------------------------------------------------------------

\normalsize

% Das esempio-Environment wird nur in der Leseansicht benötigt
\ifkorrekturansicht\else
\newenvironment{esempio}[3]%
{
    \vspace{1.5ex}
    \rlap{\underline{#1}}
    \par
    \setlength{\parindent}{0cm}
    \nopagebreak
    \leftskip=#2cm
    \rightskip=#3cm
}
{
    \par
}
\fi

\doendnotes{C}
\bigskip
\vfill

\clearpage

\footnotesize

\ifkorrekturansicht
  \lohead{\textsc{register}}
\fi

% theindex-Environment neu definieren ohne reledmac
\makeatletter
\renewenvironment{theindex}{%
  \ifkorrekturansicht
    \section*{\indexname}%
  \else
    \subsubsection*{Index der erwähnten Entitäten}%
  \fi
  \setlength{\parindent}{0pt}%
  \setlength{\parskip}{0pt plus 0.3pt}%
  \let\item\@idxitem
}{%
  \ifkorrekturansicht\clearpage\fi
}
\makeatother

\IfFileExists{\jobname-pw.ind}{\input{\jobname-pw.ind}}{}

% Quellenangabe nur in der Leseansicht
\ifkorrekturansicht\else
% Fallback-Definitionen, falls die .tex-Datei \titel etc. nicht gesetzt hat
\providecommand{\titel}{}
\providecommand{\editorInnen}{}
\providecommand{\dateiname}{\jobname}

\vspace{3cm}

\vfill

\footnotesize
\textsc{Quelle}: \titel. Herausgegeben von {\editorInnen}. In: \emph{Arthur Schnitzler: Briefwechsel mit Autorinnen und Autoren}.
 Digitale Edition, https://schnitzler-briefe.acdh.oeaw.ac.at/{\dateiname}.html (Stand \today)
\fi

\end{document}


      