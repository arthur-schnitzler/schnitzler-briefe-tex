%% latex-leseansicht-vorspann.tex
%% Vorspann für die Leseansicht.
%% Lädt die gemeinsame Datei latex-vorspann.tex mit nicht gesetztem Schalter.

\newif\ifkorrekturansicht
\korrekturansichtfalse

\input{../tex-inputs/latex-vorspann}


\section[Ottilie und Felix Salten an Arthur Schnitzler, {{[}}Anfang April 1927?{{]}}]{L03581 Ottilie und Felix Salten an Arthur Schnitzler, {[}Anfang April 1927?{]}}
\nopagebreak\mylabel{L03581v}
\rehead{ }\normalsize\beginnumbering\briefempfaengerindex{Schnitzler, Arthur@\textsc{Schnitzler, Arthur}!zzzSalten, Felix@\emph{von Felix Salten}!1927-04-101@{{[}Anfang April 1927?{]}}|(be}\briefempfaengerindex{Schnitzler, Arthur@\textsc{Schnitzler, Arthur}!zzzSalten, Ottilie@\emph{von Ottilie Salten}!1927-04-101@{{[}Anfang April 1927?{]}}|(be}
\toendnotes[C]{\smallbreak\pagebreak[2]}
\correspDesc{Versand  durch Ottilie Salten, Felix Salten im Zeitraum [Anfang April 1927?] in Wien
\newline{}Erhalt  durch Arthur Schnitzler im Zeitraum [Anfang April 1927?] in Wien}\toendnotes[C]{\smallbreak}
\Standort{CUL, Schnitzler, B 89, B 2.}
\physDesc{Briefkarte, 215 Zeichen
\newline{}Handschrift Ottilie Salten: schwarze Tinte, lateinische Kurrent
\newline{}Handschrift Felix Salten: schwarze Tinte, lateinische Kurrent}\toendnotes[C]{\smallbreak}
\pstart
           \noindent{}\centering{}{\pb}\textcolor{gray}{\textbf{HERR UND FRAU FELIX
                        SALTEN}}\pend
           
\pstart
           \centering{}\textcolor{gray}{\textbf{BITTEN}} Herrn D\textsuperscript{r}\pend
           
\pstart
           \centering{}Arthur Schnitzler\pend
           
\pstart
           \centering{}\textcolor{gray}{\textbf{FÜR}}{ }Mittwoch \textcolor{gray}{\textbf{DEN}} 13. April{ }\textcolor{gray}{\textbf{192}}\pend
           
\pstart
           \centering{}zum Thee\pend
           
\pstart
           5–8 \textcolor{gray}{\textbf{UHR}}.\pend
           
\pstart
           
\pstart
           \textcolor{gray}{\textbf{XVIII., COTTAGEGASSE 37\oindex{Wien@\textbf{Wien}!XVIII., Währing@\textbf{XVIII., Währing}!Cottagegasse@\textbf{Cottagegasse}, \emph{Straße}|pw}\oindex{Wien@\textbf{Wien}!XIX., Döbling@\textbf{XIX., Döbling}!Cottagegasse@\textbf{Cottagegasse}, \emph{Straße}|pw}}}\pend
           
\pstart
           \raggedleft{}\textcolor{gray}{\textbf{\label{K_L03581-1v}\edtext{U. A. W. G.}{\lemma{\textnormal{\emph{u. A. w. g.}}}\Cendnote{\textnormal{um Antwort wird gebeten}}}\label{K_L03581-1}}}\pend
           \pend
           
\pstart
           {[}hs. Salten:{]} Lieber, wir werden uns freuen, Sie
               an \label{K_L03581-2v}\edtext{diesem Tag}{\lemma{\textnormal{\emph{diesem Tag}}}\Cendnote{\textnormal{Es handelte
                  sich um die Feier des 
                  25. Hochzeitstags von Ottilie\pwindex{Salten, Ottilie 7.\,3.\,1868 Prag – 22.\,6.\,1942 Zürich@\textsc{Salten, Ottilie} (7.\,3.\,1868 Prag – 22.\,6.\,1942 Zürich), \emph{Schauspielerin}|pwk} und Felix Salten\pwindex{Salten, Felix 6.\,9.\,1869 Budapest – 8.\,10.\,1945 Zürich@\textsc{Salten, Felix} (6.\,9.\,1869 Budapest – 8.\,10.\,1945 Zürich), \emph{Schriftsteller, Journalist, Chefredakteur}|pwk}. Schnitzler
                  nahm teil, 
                  vgl. A. S.: \emph{Tagebuch}, 13. 4. 1927.
               }}}\label{K_L03581-2} bei uns zu haben – vielleicht, wenn Sie das vorziehen, kommen Sie gegen 7\textsuperscript{h} und bleiben zum Nachtmahl!!\pend
           
\pstart
           Herzlich {\\[\baselineskip]}\spacefill\mbox{F. S.}\pend
           \leftskip=0em{}\selectlanguage{ngerman}\endnumbering\briefempfaengerindex{Schnitzler, Arthur@\textsc{Schnitzler, Arthur}!zzzSalten, Felix@\emph{von Felix Salten}!1927-04-011@{{[}Anfang April 1927?{]}}|)be}\briefempfaengerindex{Schnitzler, Arthur@\textsc{Schnitzler, Arthur}!zzzSalten, Ottilie@\emph{von Ottilie Salten}!1927-04-011@{{[}Anfang April 1927?{]}}|)be}\mylabel{L03581h}  \newcommand{\dateiname}{L03581}\newcommand{\titel}{Ottilie und Felix Salten an Arthur Schnitzler, [Anfang April 1927?]}\newcommand{\editorInnen}{Martin Anton Müller und Laura Untner}%% latex-leseansicht-abspann.tex
%% Abspann für die Leseansicht.
%% Der Schalter \ifkorrekturansicht ist bereits durch den Vorspann gesetzt.

%% latex-abspann.tex
%% Gemeinsamer Abspann für Korrekturansicht und Leseansicht.
%% Setzt den Schalter \ifkorrekturansicht voraus (gesetzt in den
%% einbindenden Dateien latex-korrekturansicht-abspann.tex bzw.
%% latex-leseansicht-abspann.tex).
%% ---------------------------------------------------------------

\normalsize

% Das esempio-Environment wird nur in der Leseansicht benötigt
\ifkorrekturansicht\else
\newenvironment{esempio}[3]%
{
    \vspace{1.5ex}
    \rlap{\underline{#1}}
    \par
    \setlength{\parindent}{0cm}
    \nopagebreak
    \leftskip=#2cm
    \rightskip=#3cm
}
{
    \par
}
\fi

\doendnotes{C}
\bigskip
\vfill

\clearpage

\footnotesize

\ifkorrekturansicht
  \lohead{\textsc{register}}
\fi

% theindex-Environment neu definieren ohne reledmac
\makeatletter
\renewenvironment{theindex}{%
  \ifkorrekturansicht
    \section*{\indexname}%
  \else
    \subsubsection*{Index der erwähnten Entitäten}%
  \fi
  \setlength{\parindent}{0pt}%
  \setlength{\parskip}{0pt plus 0.3pt}%
  \let\item\@idxitem
}{%
  \ifkorrekturansicht\clearpage\fi
}
\makeatother

\IfFileExists{\jobname-pw.ind}{\input{\jobname-pw.ind}}{}

% Quellenangabe nur in der Leseansicht
\ifkorrekturansicht\else
% Fallback-Definitionen, falls die .tex-Datei \titel etc. nicht gesetzt hat
\providecommand{\titel}{}
\providecommand{\editorInnen}{}
\providecommand{\dateiname}{\jobname}

\vspace{3cm}

\vfill

\footnotesize
\textsc{Quelle}: \titel. Herausgegeben von {\editorInnen}. In: \emph{Arthur Schnitzler: Briefwechsel mit Autorinnen und Autoren}.
 Digitale Edition, https://schnitzler-briefe.acdh.oeaw.ac.at/{\dateiname}.html (Stand \today)
\fi

\end{document}


