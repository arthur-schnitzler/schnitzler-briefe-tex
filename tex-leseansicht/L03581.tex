%% latex-korrekturansicht-vorspann.tex
%% Vorspann für die Korrekturansicht.
%% Lädt die gemeinsame Datei latex-vorspann.tex mit gesetztem Schalter.

\newif\ifkorrekturansicht
\korrekturansichttrue

\input{../tex-inputs/latex-vorspann}


\section[Ottilie und Felix Salten an Arthur Schnitzler, {[}Anfang April 1927?{]}]{L03581 Ottilie und Felix Salten an Arthur Schnitzler, {[}Anfang
               April 1927?{]}}
\nopagebreak\mylabel{L03581v}
\rehead{ }\normalsize\beginnumbering\briefempfaengerindex{Schnitzler, Arthur@\textsc{Schnitzler, Arthur}!zzzSalten, Felix@\emph{von Felix Salten}!1927-04-101@{{[}Anfang April 1927?{]}}|(be}\briefempfaengerindex{Schnitzler, Arthur@\textsc{Schnitzler, Arthur}!zzzSalten, Ottilie@\emph{von Ottilie Salten}!1927-04-101@{{[}Anfang April 1927?{]}}|(be}
\toendnotes[C]{\smallbreak\pagebreak[2]}\Standort{CUL, Schnitzler, B 89, B 2.}
\physDesc{Briefkarte, 215 Zeichen
\newline{}Handschrift Ottilie Salten: schwarze Tinte, lateinische Kurrent
\newline{}Handschrift Felix Salten: schwarze Tinte, lateinische Kurrent}\toendnotes[C]{\smallbreak}
\pstart
           \noindent{}\centering{}{\pb}\textcolor{gray}{\textbf{HERR UND FRAU FELIX
                        SALTEN}}\pend
           
\pstart
           \centering{}\textcolor{gray}{\textbf{BITTEN}} Herrn D\textsuperscript{r}\pend
           
\pstart
           \centering{}Arthur Schnitzler\pend
           
\pstart
           \centering{}\textcolor{gray}{\textbf{FÜR}}{ }Mittwoch \textcolor{gray}{\textbf{DEN}} 13. April{ }\textcolor{gray}{\textbf{192}}\pend
           
\pstart
           \centering{}zum Thee\pend
           
\pstart
           5–8 \textcolor{gray}{\textbf{UHR}}.\pend
           
\pstart
           
\pstart
           \textcolor{gray}{\textbf{XVIII., COTTAGEGASSE 37\oindex{Cottagegasse@\textbf{Cottagegasse}, \emph{Straße (K.STR)}|pw}}}\pend
           
\pstart
           \raggedleft{}\textcolor{gray}{\textbf{\label{K_L03581-1v}\edtext{U. A. W. G.}{\lemma{\textnormal{\emph{u. A. w. g.}}}\Cendnote{\textnormal{um Antwort wird gebeten}}}\label{K_L03581-1}}}\pend
           \pend
           
\pstart
           {[}hs. :{]} Lieber, wir werden uns freuen, Sie
               an \label{K_L03581-2v}\edtext{diesem Tag}{\lemma{\textnormal{\emph{diesem Tag}}}\Cendnote{\textnormal{Es handelte
                  sich um die Feier des 
                  25. Hochzeitstags von Ottilie\pwindex{Salten, Ottilie 07.03.1868 – 22.06.1942@\textsc{Salten, Ottilie} (07.03.1868 – 22.06.1942), \emph{Schauspieler/Schauspielerin}|pwk} und Felix Salten\pwindex{Salten, Felix 06.09.1869 – 08.10.1945@\textsc{Salten, Felix} (06.09.1869 – 08.10.1945), \emph{Schriftsteller/Schriftstellerin, Journalist/Journalistin, Chefredakteur/Chefredakteurin}|pwk}. Schnitzler
                  nahm teil, 
                  vgl. A. S.: \emph{Tagebuch}, 13. 4. 1927.
               }}}\label{K_L03581-2} bei uns zu haben – vielleicht, wenn Sie das vorziehen, kommen Sie gegen 7\textsuperscript{h} und bleiben zum Nachtmahl!!\pend
           
\pstart
           Herzlich {\\[\baselineskip]}\spacefill\mbox{F. S.}\pend
           \leftskip=0em{}\selectlanguage{ngerman}\endnumbering\briefempfaengerindex{Schnitzler, Arthur@\textsc{Schnitzler, Arthur}!zzzSalten, Felix@\emph{von Felix Salten}!1927-04-011@{{[}Anfang April 1927?{]}}|)be}\briefempfaengerindex{Schnitzler, Arthur@\textsc{Schnitzler, Arthur}!zzzSalten, Ottilie@\emph{von Ottilie Salten}!1927-04-011@{{[}Anfang April 1927?{]}}|)be}\mylabel{L03581h}  \normalsize

\doendnotes{C}
\bigskip
\vfill

\clearpage

\footnotesize

\lohead{\textsc{register}}

% Definiere theindex-Environment komplett neu ohne reledmac
\makeatletter
\renewenvironment{theindex}{%
  \section*{\indexname}%
  \setlength{\parindent}{0pt}%
  \setlength{\parskip}{0pt plus 0.3pt}%
  \let\item\@idxitem
}{%
  \clearpage
}
\makeatother

\IfFileExists{\jobname-pw.ind}{\input{\jobname-pw.ind}}{}

\end{document}

      