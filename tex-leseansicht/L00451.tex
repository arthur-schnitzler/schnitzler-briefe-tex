\input{../tex-inputs/latex-pdf-vorspann}
\begin{center}
            \textcolor{red}{ENTWURF. ENTZIFFERUNG NOCH NICHT KORREKTURGELESEN}
                      \end{center}
            
               \section[Richard Beer-Hofmann an Arthur Schnitzler, 11. 6. 1895]{ Richard Beer-Hofmann an Arthur Schnitzler,
               11. 6. 1895}\nopagebreak\mylabel{v}\rehead{ }\begin{ledgroupsized}[t]{13cm}\normalsize\beginnumbering\briefempfaengerindex{Schnitzler, Arthur@\textsc{Schnitzler, Arthur}!zzzBeer-Hofmann, Richard@\emph{von Richard Beer-Hofmann}!1895-06-111@{11. 6. 1895}|(be} \toendnotes[C]{\smallbreak\pagebreak[2]} \Standort{CUL, Schnitzler, B 8.}
\physDesc{Briefkarte
\newline{}Handschrift: Bleistift, lateinische Kurrent
\newline{}Schnitzler: mit Bleistift nummeriert: »61.« }\buchAbdrucke{\weitereDrucke{Arthur Schnitzler, Richard Beer-Hofmann: \emph{Briefwechsel 1891–1931}. Hg. Konstanze Fliedl. Wien, Zürich: \emph{Europaverlag} 1992, S. 73.} }\pstart
           {\pb}Caslau\oindex{Caslau@\textbf{Caslau}|pw}{ }11/VI 95\pend
           \pstart
           Lieber Arthur! Kann Ihnen nur wenig schreiben. Wir werden
               entsetzlich geschunden. Vor ½ 7 Abends
                sind wir bisher noch nicht
               eingerückt. Dies soll nur ein Lebenszeichen sein. »Ist denn {\pb}das e Leben?« Ihr Brief hat mich
               natürlich doch beunruhigt. Vielleicht kommt das »Ausschlagen« des Pferdes noch. Bitte
               um viel Brief. Herzlichst\pend
           \pstart
           Ihr{\\[\baselineskip]}\spacefill\mbox{Richard}\pend
           \leftskip=0em{}\pstart
           Grüße an Salten\pwindex{Salten, Felix 06.09.1869 – 08.10.1945@\textsc{Salten, Felix} (06.09.1869 – 08.10.1945), \emph{Schriftsteller, Journalist}|pw}{ }Schwarzkopf\pwindex{Schwarzkopf, Gustav 07.11.1853 – 13.11.1939@\textsc{Schwarzkopf, Gustav} (07.11.1853 – 13.11.1939), \emph{Schriftsteller}|pw} u. à discretion\pend
           \endnumbering\briefempfaengerindex{Schnitzler, Arthur@\textsc{Schnitzler, Arthur}!zzzBeer-Hofmann, Richard@\emph{von Richard Beer-Hofmann}!1895-06-111@{11. 6. 1895}|)be}\mylabel{h}\end{ledgroupsized}  \newcommand{\dateiname}{L00451}\newcommand{\titel}{Richard Beer-Hofmann an Arthur Schnitzler, 11. 6. 1895}\newcommand{\editorInnen}{Martin Anton Müller und Gerd-Hermann Susen}\input{../tex-inputs/latex-pdf-abspann}
      