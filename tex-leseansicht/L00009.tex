%% latex-leseansicht-vorspann.tex
%% Vorspann für die Leseansicht.
%% Lädt die gemeinsame Datei latex-vorspann.tex mit nicht gesetztem Schalter.

\newif\ifkorrekturansicht
\korrekturansichtfalse

\input{../tex-inputs/latex-vorspann}


         
         \renewcommand{\erwaehntePersonen}{Personen: Michael Georg Conrad, Karl von Perfall}
         \renewcommand{\erwaehnteInstitutionen}{Institutionen: Königliche Hof- und Nationaltheater München}
         \renewcommand{\erwaehnteOrte}{Orte: Wien}
         \renewcommand{\erwaehnteWerke}{Werke: Alkandi’s Lied}
               \section[Arthur Schnitzler an Michael Georg Conrad, 11. 3. 1891]{ Arthur Schnitzler an Michael Georg Conrad, 11. 3. 1891}\nopagebreak\mylabel{v}\rehead{ }\begin{ledgroupsized}[t]{13cm}\normalsize\beginnumbering \toendnotes[C]{\smallbreak\pagebreak[2]} \Standort{München, Monacensia, Schnitzler, Arthur A I/1.}
\physDesc{Brief, 1 Blatt, 2 Seiten
\newline{}Handschrift: schwarze Tinte, deutsche Kurrent}\toendnotes[C]{\smallbreak}\pstart
           \raggedleft{}{\pb}Wien\oindex{Wien@\textbf{Wien}|pw}, 11. März 1891\pend
           \pstart
           Erlauben Sie mir, ſehr verehrter Herr, Ihnen hiemit \label{K_L00009_1v}\edtext{Alkandi’s Lied\pwindex{Schnitzler, Arthur 15.05.1862 – 21.10.1931@\textsc{Schnitzler, Arthur} (15.05.1862 – 21.10.1931), \emph{Schriftsteller, Mediziner}!Alkandi s Lied15.8.1890 – 1.9.1890@\strich\emph{Alkandi’s Lied} {[}15.8.1890 – 1.9.1890{]}|pw}}{\lemma{\textnormal{\emph{Alkandi’s Lied}}}\Cendnote{\textnormal{Schnitzler hatte das Stück\pwindex{Schnitzler, Arthur 15.05.1862 – 21.10.1931@\textsc{Schnitzler, Arthur} (15.05.1862 – 21.10.1931), \emph{Schriftsteller, Mediziner}!Alkandi s Lied15.8.1890 – 1.9.1890@\strich\emph{Alkandi’s Lied} {[}15.8.1890 – 1.9.1890{]}|pwkv} bereits im Herbst 1889
                  vollendet, vgl. A. S.: \emph{Tagebuch}, 15. 11. 1889}}}\label{K_L00009_1h},
               ein dramatiſches Gedicht zu überſenden. Vielleicht haben Sie einmal eine halbe
               Stunde, es durchzuleſen. Ihr Urtheil wäre mir ſehr werthvoll. Halten Sie das Stück\pwindex{Schnitzler, Arthur 15.05.1862 – 21.10.1931@\textsc{Schnitzler, Arthur} (15.05.1862 – 21.10.1931), \emph{Schriftsteller, Mediziner}!Alkandi s Lied15.8.1890 – 1.9.1890@\strich\emph{Alkandi’s Lied} {[}15.8.1890 – 1.9.1890{]}|pwv} für aufführbar? Kö{\geminationn}ten Sie mir rathen, es der \label{K_L00009_2v}\edtext{Münchner Bühne\orgindex{Koenigliche Hof- und Nationaltheater Muenchen@Königliche Hof- und Nationaltheater München|pw}}{\lemma{\textnormal{\emph{Münchner Bühne}}}\Cendnote{\textnormal{Schnitzler bezieht sich auf das \emph{Kgl. Hof- und Nationaltheater und das Kgl.
                     Residenz-Theater}\orgindex{Koenigliche Hof- und Nationaltheater Muenchen@Königliche Hof- und Nationaltheater München|pwk}; General-Intendant war Karl Freiherr von Perfall\pwindex{Perfall, Karl von 24.03.1851 – 31.08.1924@\textsc{Perfall, Karl von} (24.03.1851 – 31.08.1924), \emph{Schriftsteller}|pwk}; zur Beziehung Conrad\pwindex{Conrad, Michael Georg 05.04.1846 – 20.12.1927@\textsc{Conrad, Michael Georg} (05.04.1846 – 20.12.1927), \emph{Schriftsteller, Kritiker}|pwk}s zu den Königlichen Bühnen vgl. Michael Georg Conrad an Arthur Schnitzler, 28. 3. 1893}}}\label{K_L00009_2h} einzuſenden? Wie ſehr möchte ich Ihnen für eine kurze
               Beantwortung dieſer {\pb}Fragen danken!\pend
           \pstart
           In aufrichtiger Verehrung{\\[\baselineskip]}Ihr ſehr ergebener{\\[\baselineskip]}\spacefill\mbox{Dr. Arthur Schnitzler}\pend
           \leftskip=0em{}\pstart
           \noindent{}\textsc{Wien, I. Giselastraße 11\oindex{Wien@\textbf{Wien}|pw}.}\pend
           
         
         \endnumbering\mylabel{h}\end{ledgroupsized}  \newcommand{\dateiname}{L00009}\newcommand{\titel}{Arthur Schnitzler an Michael Georg Conrad, 11. 3. 1891}\newcommand{\editorInnen}{Martin Anton Müller und Gerd-Hermann Susen}%% latex-leseansicht-abspann.tex
%% Abspann für die Leseansicht.
%% Der Schalter \ifkorrekturansicht ist bereits durch den Vorspann gesetzt.

%% latex-abspann.tex
%% Gemeinsamer Abspann für Korrekturansicht und Leseansicht.
%% Setzt den Schalter \ifkorrekturansicht voraus (gesetzt in den
%% einbindenden Dateien latex-korrekturansicht-abspann.tex bzw.
%% latex-leseansicht-abspann.tex).
%% ---------------------------------------------------------------

\normalsize

% Das esempio-Environment wird nur in der Leseansicht benötigt
\ifkorrekturansicht\else
\newenvironment{esempio}[3]%
{
    \vspace{1.5ex}
    \rlap{\underline{#1}}
    \par
    \setlength{\parindent}{0cm}
    \nopagebreak
    \leftskip=#2cm
    \rightskip=#3cm
}
{
    \par
}
\fi

\doendnotes{C}
\bigskip
\vfill

\clearpage

\footnotesize

\ifkorrekturansicht
  \lohead{\textsc{register}}
\fi

% theindex-Environment neu definieren ohne reledmac
\makeatletter
\renewenvironment{theindex}{%
  \ifkorrekturansicht
    \section*{\indexname}%
  \else
    \subsubsection*{Index der erwähnten Entitäten}%
  \fi
  \setlength{\parindent}{0pt}%
  \setlength{\parskip}{0pt plus 0.3pt}%
  \let\item\@idxitem
}{%
  \ifkorrekturansicht\clearpage\fi
}
\makeatother

\IfFileExists{\jobname-pw.ind}{\input{\jobname-pw.ind}}{}

% Quellenangabe nur in der Leseansicht
\ifkorrekturansicht\else
% Fallback-Definitionen, falls die .tex-Datei \titel etc. nicht gesetzt hat
\providecommand{\titel}{}
\providecommand{\editorInnen}{}
\providecommand{\dateiname}{\jobname}

\vspace{3cm}

\vfill

\footnotesize
\textsc{Quelle}: \titel. Herausgegeben von {\editorInnen}. In: \emph{Arthur Schnitzler: Briefwechsel mit Autorinnen und Autoren}.
 Digitale Edition, https://schnitzler-briefe.acdh.oeaw.ac.at/{\dateiname}.html (Stand \today)
\fi

\end{document}


      