%% latex-leseansicht-vorspann.tex
%% Vorspann für die Leseansicht.
%% Lädt die gemeinsame Datei latex-vorspann.tex mit nicht gesetztem Schalter.

\newif\ifkorrekturansicht
\korrekturansichtfalse

\input{../tex-inputs/latex-vorspann}


\section[Arthur Schnitzler an Michael Georg Conrad, 11. 3. 1891]{L00009 Arthur Schnitzler an Michael Georg Conrad, 11. 3. 1891}
\nopagebreak\mylabel{L00009v}
\rehead{ }\normalsize\beginnumbering\briefempfaengerindex{Conrad, Michael Georg@\textsc{Conrad, Michael Georg}!zzzSchnitzler, Arthur@\emph{von Arthur Schnitzler}!1891-03-111@{11. 3. 1891}|(be}
\toendnotes[C]{\smallbreak\pagebreak[2]}
\correspDesc{Versand  durch Arthur Schnitzler am 11. 3. 1891 in Wien
\newline{}Erhalt  durch Michael Georg Conrad im Zeitraum [11. 3. 1891
                  – 15. 3. 1891?] \textbf{Ort fehlend} }\toendnotes[C]{\smallbreak}
\Standort{München, Monacensia, Schnitzler, Arthur A I/1.}
\physDesc{Brief, 1 Blatt, 2 Seiten, 479 Zeichen
\newline{}Handschrift: schwarze Tinte, deutsche Kurrent}\toendnotes[C]{\smallbreak}
\pstart
           \raggedleft{}{\pb}Wien\oindex{Wien@\textbf{Wien}, \emph{Verwaltungsgebiet}|pw}, 11. März 1891\pend
           \vspace{0.5em}
\pstart
           Erlauben Sie mir,{ }ſehr verehrter Herr, Ihnen hiemit \label{K_L00009-1v}\edtext{Alkandi’s Lied\pwindex{Schnitzler, Arthur 15.\,5.\,1862 Wien – 21.\,10.\,1931 ebd.@\textsc{Schnitzler, Arthur} (15.\,5.\,1862 Wien – 21.\,10.\,1931 ebd.), \emph{Schriftsteller, Mediziner}!Alkandi’s Lied@\strich\emph{Alkandi’s Lied}|pw}}{\lemma{\textnormal{\emph{Alkandi’s Lied}}}\Cendnote{\textnormal{Schnitzler hatte das Stück\pwindex{Schnitzler, Arthur 15.\,5.\,1862 Wien – 21.\,10.\,1931 ebd.@\textsc{Schnitzler, Arthur} (15.\,5.\,1862 Wien – 21.\,10.\,1931 ebd.), \emph{Schriftsteller, Mediziner}!Alkandi’s Lied@\strich\emph{Alkandi’s Lied}|pwkv} bereits im Herbst 1889 vollendet, vgl. A. S.: \emph{Tagebuch}, 15. 11. 1889.
               }}}\label{K_L00009-1}, ein dramatiſches Gedicht zu überſenden. Vielleicht haben Sie einmal eine
               halbe Stunde, es durchzuleſen. Ihr Urtheil wäre mir{ }ſehr werthvoll. Halten Sie das
                  Stück\pwindex{Schnitzler, Arthur 15.\,5.\,1862 Wien – 21.\,10.\,1931 ebd.@\textsc{Schnitzler, Arthur} (15.\,5.\,1862 Wien – 21.\,10.\,1931 ebd.), \emph{Schriftsteller, Mediziner}!Alkandi’s Lied@\strich\emph{Alkandi’s Lied}|pwv} für aufführbar? Kö{\geminationn}ten Sie mir rathen, es der \label{K_L00009-2v}\edtext{Münchner Bühne\orgindex{Nationaltheater München@Nationaltheater München|pw}}{\lemma{\textnormal{\emph{Münchner Bühne}}}\Cendnote{\textnormal{Schnitzler bezieht sich auf das \emph{Kgl. Hof- und Nationaltheater und das Kgl.
                     Residenz-Theater}\orgindex{Nationaltheater München@Nationaltheater München|pwk}; General-Intendant war Karl Freiherr von Perfall\pwindex{Perfall, Karl von 24.\,3.\,1851 Landsberg am Lech – 31.\,8.\,1924 Düsseldorf@\textsc{Perfall, Karl von} (24.\,3.\,1851 Landsberg am Lech – 31.\,8.\,1924 Düsseldorf), \emph{Schriftsteller, Herausgeber}|pwk}. Zur Beziehung Conrads\pwindex{Conrad, Michael Georg 5.\,4.\,1846 Gnodstadt – 20.\,12.\,1927 München@\textsc{Conrad, Michael Georg} (5.\,4.\,1846 Gnodstadt – 20.\,12.\,1927 München), \emph{Schriftsteller, Kritiker}|pwk} zu den Königlichen Bühnen vgl. XXXX Auszeichnungsfehler: Dokument L00194 nicht gefunden.
               }}}\label{K_L00009-2} einzuſenden? Wie{ }ſehr möchte ich Ihnen für eine kurze Beantwortung dieſer {\pb}Fragen danken!\pend
           
\pstart
           In aufrichtiger Verehrung{\\[\baselineskip]}Ihr{ }ſehr ergebener{\\[\baselineskip]}\spacefill\mbox{Dr. Arthur Schnitzler}\pend
           \leftskip=0em{}
\pstart
           \noindent{}\textsc{Wien, I. Giselastraße 11\oindex{Wien@\textbf{Wien}!I., Innere Stadt@\textbf{I., Innere Stadt}!Ordination Arthur Schnitzler [Bösendorferstraße 11]@\textbf{Ordination Arthur Schnitzler [Bösendorferstraße 11]}, \emph{Ordination}|pw}.}\pend
           \selectlanguage{ngerman}\endnumbering\briefempfaengerindex{Conrad, Michael Georg@\textsc{Conrad, Michael Georg}!zzzSchnitzler, Arthur@\emph{von Arthur Schnitzler}!1891-03-111@{11. 3. 1891}|)be}\mylabel{L00009h}  \newcommand{\dateiname}{L00009}\newcommand{\titel}{Arthur Schnitzler an Michael Georg Conrad, 11. 3. 1891}\newcommand{\editorInnen}{Martin Anton Müller und Gerd-Hermann Susen}%% latex-leseansicht-abspann.tex
%% Abspann für die Leseansicht.
%% Der Schalter \ifkorrekturansicht ist bereits durch den Vorspann gesetzt.

%% latex-abspann.tex
%% Gemeinsamer Abspann für Korrekturansicht und Leseansicht.
%% Setzt den Schalter \ifkorrekturansicht voraus (gesetzt in den
%% einbindenden Dateien latex-korrekturansicht-abspann.tex bzw.
%% latex-leseansicht-abspann.tex).
%% ---------------------------------------------------------------

\normalsize

% Das esempio-Environment wird nur in der Leseansicht benötigt
\ifkorrekturansicht\else
\newenvironment{esempio}[3]%
{
    \vspace{1.5ex}
    \rlap{\underline{#1}}
    \par
    \setlength{\parindent}{0cm}
    \nopagebreak
    \leftskip=#2cm
    \rightskip=#3cm
}
{
    \par
}
\fi

\doendnotes{C}
\bigskip
\vfill

\clearpage

\footnotesize

\ifkorrekturansicht
  \lohead{\textsc{register}}
\fi

% theindex-Environment neu definieren ohne reledmac
\makeatletter
\renewenvironment{theindex}{%
  \ifkorrekturansicht
    \section*{\indexname}%
  \else
    \subsubsection*{Index der erwähnten Entitäten}%
  \fi
  \setlength{\parindent}{0pt}%
  \setlength{\parskip}{0pt plus 0.3pt}%
  \let\item\@idxitem
}{%
  \ifkorrekturansicht\clearpage\fi
}
\makeatother

\IfFileExists{\jobname-pw.ind}{\input{\jobname-pw.ind}}{}

% Quellenangabe nur in der Leseansicht
\ifkorrekturansicht\else
% Fallback-Definitionen, falls die .tex-Datei \titel etc. nicht gesetzt hat
\providecommand{\titel}{}
\providecommand{\editorInnen}{}
\providecommand{\dateiname}{\jobname}

\vspace{3cm}

\vfill

\footnotesize
\textsc{Quelle}: \titel. Herausgegeben von {\editorInnen}. In: \emph{Arthur Schnitzler: Briefwechsel mit Autorinnen und Autoren}.
 Digitale Edition, https://schnitzler-briefe.acdh.oeaw.ac.at/{\dateiname}.html (Stand \today)
\fi

\end{document}


