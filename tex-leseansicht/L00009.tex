%% latex-korrekturansicht-vorspann.tex
%% Vorspann für die Korrekturansicht.
%% Lädt die gemeinsame Datei latex-vorspann.tex mit gesetztem Schalter.

\newif\ifkorrekturansicht
\korrekturansichttrue

\input{../tex-inputs/latex-vorspann}


\section[Arthur Schnitzler an Michael Georg Conrad, 11. 3. 1891]{L00009 Arthur Schnitzler an Michael Georg Conrad, 11. 3. 1891}
\nopagebreak\mylabel{L00009v}
\rehead{ }\normalsize\beginnumbering\briefempfaengerindex{Conrad, Michael Georg@\textsc{Conrad, Michael Georg}!zzzSchnitzler, Arthur@\emph{von Arthur Schnitzler}!1891-03-111@{11. 3. 1891}|(be}
\toendnotes[C]{\smallbreak\pagebreak[2]}\Standort{München, Monacensia, Schnitzler, Arthur A I/1.}
\physDesc{Brief, 1 Blatt, 2 Seiten, 479 Zeichen
\newline{}Handschrift: schwarze Tinte, deutsche Kurrent}\toendnotes[C]{\smallbreak}
\pstart
           \raggedleft{}{\pb}Wien\oindex{Wien@\textbf{Wien}, \emph{A.ADM2}|pw}, 11. März 1891\pend
           \vspace{0.5em}
\pstart
           Erlauben Sie mir, ſehr verehrter Herr, Ihnen hiemit \label{K_L00009-1v}\edtext{Alkandi’s Lied\pwindex{Alkandi s Lied@\emph{Alkandi’s Lied}|pw}}{\lemma{\textnormal{\emph{Alkandi’s Lied}}}\Cendnote{\textnormal{Schnitzler hatte das Stück\pwindex{Alkandi s Lied@\emph{Alkandi’s Lied}|pwkv} bereits im Herbst
                     1889 vollendet, vgl. A. S.: \emph{Tagebuch}, 15. 11. 1889.
               }}}\label{K_L00009-1}, ein dramatiſches Gedicht zu überſenden. Vielleicht haben Sie einmal eine
               halbe Stunde, es durchzuleſen. Ihr Urtheil wäre mir ſehr werthvoll. Halten Sie das
                  Stück\pwindex{Alkandi s Lied@\emph{Alkandi’s Lied}|pwv} für aufführbar? Kö{\geminationn}ten Sie mir rathen, es der \label{K_L00009-2v}\edtext{Münchner Bühne\orgindex{Nationaltheater Muenchen@Nationaltheater München|pw}}{\lemma{\textnormal{\emph{Münchner Bühne}}}\Cendnote{\textnormal{Schnitzler bezieht sich auf das \emph{Kgl. Hof- und Nationaltheater und das Kgl.
                     Residenz-Theater}\orgindex{Nationaltheater Muenchen@Nationaltheater München|pwk}; General-Intendant war Karl Freiherr von Perfall\pwindex{Perfall, Karl von 24.03.1851 – 31.08.1924@\textsc{Perfall, Karl von} (24.03.1851 – 31.08.1924), \emph{Schriftsteller/Schriftstellerin, Herausgeber/Herausgeberin}|pwk}. Zur Beziehung Conrads\pwindex{Conrad, Michael Georg 05.04.1846 – 20.12.1927@\textsc{Conrad, Michael Georg} (05.04.1846 – 20.12.1927), \emph{Schriftsteller/Schriftstellerin, Kritiker/Kritikerin}|pwk} zu den Königlichen Bühnen vgl. Michael Georg Conrad an Arthur Schnitzler, 28. 3. 1893.
               }}}\label{K_L00009-2} einzuſenden? Wie ſehr möchte ich Ihnen für eine kurze Beantwortung dieſer {\pb}Fragen danken!\pend
           
\pstart
           In aufrichtiger Verehrung{\\[\baselineskip]}Ihr ſehr ergebener{\\[\baselineskip]}\spacefill\mbox{Dr. Arthur Schnitzler}\pend
           \leftskip=0em{}
\pstart
           \noindent{}\textsc{Wien, I. Giselastraße 11\oindex{Ordination Arthur Schnitzler [Boesendorferstrasse 11]@\textbf{Ordination Arthur Schnitzler [Bösendorferstraße 11]}, \emph{Ordination}|pw}.}\pend
           \selectlanguage{ngerman}\endnumbering\briefempfaengerindex{Conrad, Michael Georg@\textsc{Conrad, Michael Georg}!zzzSchnitzler, Arthur@\emph{von Arthur Schnitzler}!1891-03-111@{11. 3. 1891}|)be}\mylabel{L00009h}  \normalsize

\doendnotes{C}
\bigskip
\vfill

\clearpage

\footnotesize

\lohead{\textsc{register}}

% Definiere theindex-Environment komplett neu ohne reledmac
\makeatletter
\renewenvironment{theindex}{%
  \section*{\indexname}%
  \setlength{\parindent}{0pt}%
  \setlength{\parskip}{0pt plus 0.3pt}%
  \let\item\@idxitem
}{%
  \clearpage
}
\makeatother

\IfFileExists{\jobname-pw.ind}{\input{\jobname-pw.ind}}{}

\end{document}

      