%% latex-leseansicht-vorspann.tex
%% Vorspann für die Leseansicht.
%% Lädt die gemeinsame Datei latex-vorspann.tex mit nicht gesetztem Schalter.

\newif\ifkorrekturansicht
\korrekturansichtfalse

\input{../tex-inputs/latex-vorspann}


\section[Arthur Schnitzler an Gustav Schwarzkopf, 19. 7. 1895]{L04109 Arthur Schnitzler an Gustav Schwarzkopf, 19. 7. 1895}
\nopagebreak\mylabel{L04109v}
\rehead{ }\normalsize\beginnumbering\briefempfaengerindex{Schwarzkopf, Gustav@\textsc{Schwarzkopf, Gustav}!zzzSchnitzler, Arthur@\emph{von Arthur Schnitzler}!1895-07-191@{19. 7. 1895}|(be}
\toendnotes[C]{\smallbreak\pagebreak[2]}
\correspDesc{Versand  durch Arthur Schnitzler am 19. 7. 1895 in Wien
\newline{}Erhalt  durch Gustav Schwarzkopf im Zeitraum [19. 7. 1895 – 22. 7. 1895?] in Wien}\toendnotes[C]{\smallbreak}
\Standort{CUL, Schnitzler, B 96.}
\physDesc{Brief, 1 Blatt, 4 Seiten, 1237 Zeichen
\newline{}Handschrift: schwarze Tinte, deutsche Kurrent}
\buchAbdrucke{\weitereDrucke{Arthur Schnitzler: \emph{Briefe 1875–1912}. Herausgegeben von Therese Nickl und Heinrich Schnitzler. Frankfurt am Main: \emph{S. Fischer} 1981, S. 268–269.} }\toendnotes[C]{\smallbreak}
\pstart{}{\pb}Lieber Freund,\pend\vspace{0.5em}
\pstart
           nach kurzen Wanderung durch böhmiſche\oindex{Böhmen@\textbf{Böhmen}, \emph{Region}|pw} Wälder,
               Bäder, Ausſtellungen und Theater bin ich vor ein paar Tagen beim »\textsc{Leopold\oindex{Hotel und Pension Rudolfshöhe (Leopold Petter)@\textbf{Hotel und Pension Rudolfshöhe (Leopold Petter)}, \emph{Hotel}|pw}}« gelandet und will Sie von hier aus vielmals herzlich grüßen. Außerdem aber
               wiederhole ich meinen wohlgemeinten Rath: ko{\geminationm}en Sie auch
               hieher. Von der Geſellſchaft will ich gar nicht reden – aber Natur und Küche ſind
               noch {\pb}beſſer als voriges Jahr.
               Verſäumt haben Sie ſchon genug: z. B. die Erſtaufführung\eventindex{Lehártheater@\textbf{Lehártheater}!Aufführung von Jugendsünden. Schwank in einem Aufzuge, Cavalleria rusticana, 16.7.1895@Aufführung von Jugendsünden. Schwank in einem Aufzuge, Cavalleria rusticana, 16.7.1895|pwv} der »Jugendsünde\pwindex{Pserhofer, Arthur 28.\,10.\,1873 Wien – 13.\,1.\,1907 Berlin@\textsc{Pserhofer, Arthur} (28.\,10.\,1873 Wien – 13.\,1.\,1907 Berlin), \emph{Schriftsteller, Theaterleiter}!Jugendsünden. Schwank in einem Aufzuge@\strich\emph{Jugendsünden. Schwank in einem Aufzuge}|pw}« von \textsc{Pserhofer\pwindex{Pserhofer, Arthur 28.\,10.\,1873 Wien – 13.\,1.\,1907 Berlin@\textsc{Pserhofer, Arthur} (28.\,10.\,1873 Wien – 13.\,1.\,1907 Berlin), \emph{Schriftsteller, Theaterleiter}|pw}}, einen blauen Ka{\geminationm}garnanzug von \textsc{Beer-Hofmann\pwindex{Beer-Hofmann, Richard 11.\,7.\,1866 Wien – 26.\,9.\,1945 New York City@\textsc{Beer-Hofmann, Richard} (11.\,7.\,1866 Wien – 26.\,9.\,1945 New York City), \emph{Schriftsteller}|pw}} und drei Witze von \textsc{Paul von Schönthan\pwindex{Schönthan-Pernwald, Paul von 19.\,3.\,1853 Wien – 4.\,8.\,1905 ebd.@\textsc{Schönthan-Pernwald, Paul von} (19.\,3.\,1853 Wien – 4.\,8.\,1905 ebd.), \emph{Schriftsteller, Journalist}|pw}}. Geben Sie Acht, daſs Sie nicht gar zu viel zu bereuen haben und begeben Sie{ }ſich eilends her. Sonſt reiſt am Ende noch \textsc{Lautenburg\pwindex{Lautenburg, Sigmund 11.\,9.\,1851 Budapest – 21.\,7.\,1918 Marienbad@\textsc{Lautenburg, Sigmund} (11.\,9.\,1851 Budapest – 21.\,7.\,1918 Marienbad), \emph{Theaterleiter, Schauspieler}|pw}} ab, und es bleibt von Berlin\oindex{Berlin@\textbf{Berlin}, \emph{Hauptstadt}|pw}er Direktoren
               nur Blumenthal\pwindex{Blumenthal, Oskar 13.\,3.\,1852 Berlin – 24.\,4.\,1917 ebd.@\textsc{Blumenthal, Oskar} (13.\,3.\,1852 Berlin – 24.\,4.\,1917 ebd.), \emph{Schriftsteller, Journalist, Theaterleiter}|pw} übrig – das iſt für Sie, {\pb}mit Ihren ewigen Verſuchen,
               vortheilhafte Verbindungen anzuknüpfen, doch zu wenig. – Weiters ſoll Ihnen
               nicht verhehlt werden, daſs man an ſchönen Tagen Frau Odilon\pwindex{Odilon, Helene 31.\,7.\,1863 Dresden – 9.\,2.\,1939 Baden bei Wien@\textsc{Odilon, Helene} (31.\,7.\,1863 Dresden – 9.\,2.\,1939 Baden bei Wien), \emph{Schauspielerin}|pw} ihr Rad bu{\geminationm}eln sehen
               kann, was Ihrer Lüſternheit einen jähen und gefliſſentlichen Reiz bedeuten dürfte.
               Auf die \textsc{Neue Revue\pwindex{Neue Revue. Wiener Literatur-Zeitung@\emph{Neue Revue. Wiener Literatur-Zeitung}|pw}} u \textsc{Zeit\pwindex{Zeit. Wiener Wochenschrift@\emph{Die Zeit. Wiener Wochenschrift}|pw}} bin ich abonnirt – Ihr Bildungstrieb braucht alſo keine He{\geminationm}ung zu erleiden.\pend
           
\pstart
           {\pb}Zu weiteren Auskünften bin in gerne
               bereit – am liebſten mündlich.\pend
           
\pstart
           Mit herzlichem Gruß Ihr{\\[\baselineskip]}\spacefill\mbox{ArthurSch}\pend
           \leftskip=0em{}
\pstart
           19. 7. 95\pend
           
\pstart
           \textsc{Ischl Rudolfshöhe\oindex{Hotel und Pension Rudolfshöhe (Leopold Petter)@\textbf{Hotel und Pension Rudolfshöhe (Leopold Petter)}, \emph{Hotel}|pw}}.\pend
           \selectlanguage{ngerman}\endnumbering\briefempfaengerindex{Schwarzkopf, Gustav@\textsc{Schwarzkopf, Gustav}!zzzSchnitzler, Arthur@\emph{von Arthur Schnitzler}!1895-07-191@{19. 7. 1895}|)be}\mylabel{L04109h}
\begin{anhang}
\end{anhang}\newcommand{\dateiname}{L04109}\newcommand{\titel}{Arthur Schnitzler an Gustav Schwarzkopf, 19. 7. 1895}\newcommand{\editorInnen}{Herausgegeben von Jahnke, SelmaMüller, Martin Anton}%% latex-leseansicht-abspann.tex
%% Abspann für die Leseansicht.
%% Der Schalter \ifkorrekturansicht ist bereits durch den Vorspann gesetzt.

%% latex-abspann.tex
%% Gemeinsamer Abspann für Korrekturansicht und Leseansicht.
%% Setzt den Schalter \ifkorrekturansicht voraus (gesetzt in den
%% einbindenden Dateien latex-korrekturansicht-abspann.tex bzw.
%% latex-leseansicht-abspann.tex).
%% ---------------------------------------------------------------

\normalsize

% Das esempio-Environment wird nur in der Leseansicht benötigt
\ifkorrekturansicht\else
\newenvironment{esempio}[3]%
{
    \vspace{1.5ex}
    \rlap{\underline{#1}}
    \par
    \setlength{\parindent}{0cm}
    \nopagebreak
    \leftskip=#2cm
    \rightskip=#3cm
}
{
    \par
}
\fi

\doendnotes{C}
\bigskip
\vfill

\clearpage

\footnotesize

\ifkorrekturansicht
  \lohead{\textsc{register}}
\fi

% theindex-Environment neu definieren ohne reledmac
\makeatletter
\renewenvironment{theindex}{%
  \ifkorrekturansicht
    \section*{\indexname}%
  \else
    \subsubsection*{Index der erwähnten Entitäten}%
  \fi
  \setlength{\parindent}{0pt}%
  \setlength{\parskip}{0pt plus 0.3pt}%
  \let\item\@idxitem
}{%
  \ifkorrekturansicht\clearpage\fi
}
\makeatother

\IfFileExists{\jobname-pw.ind}{\input{\jobname-pw.ind}}{}

% Quellenangabe nur in der Leseansicht
\ifkorrekturansicht\else
% Fallback-Definitionen, falls die .tex-Datei \titel etc. nicht gesetzt hat
\providecommand{\titel}{}
\providecommand{\editorInnen}{}
\providecommand{\dateiname}{\jobname}

\vspace{3cm}

\vfill

\footnotesize
\textsc{Quelle}: \titel. Herausgegeben von {\editorInnen}. In: \emph{Arthur Schnitzler: Briefwechsel mit Autorinnen und Autoren}.
 Digitale Edition, https://schnitzler-briefe.acdh.oeaw.ac.at/{\dateiname}.html (Stand \today)
\fi

\end{document}


