\input{../tex-inputs/latex-pdf-vorspann}
\begin{center}
            \textcolor{red}{ENTWURF. ENTZIFFERUNG NOCH NICHT KORREKTURGELESEN}
                      \end{center}
            
               \section[Hugo von Hofmannsthal an Arthur Schnitzler, {[}8. 9. 1899{]}]{ Hugo von Hofmannsthal an Arthur Schnitzler, {[}8. 9. 1899{]}}\nopagebreak\mylabel{v}\rehead{ }\begin{ledgroupsized}[t]{13cm}\normalsize\beginnumbering\briefempfaengerindex{Schnitzler, Arthur@\textsc{Schnitzler, Arthur}!zzzHofmannsthal, Hugo von@\emph{von Hugo von Hofmannsthal}!1899-09-081@{{[}8. 9. 1899{]}}|(be} \toendnotes[C]{\smallbreak\pagebreak[2]} \Standort{CUL, Schnitzler, B 43.}
\physDesc{Brief, 1 Blatt, 4 Seiten
\newline{}Handschrift: schwarze Tinte, deutsche Kurrent
\newline{}Schnitzler: mit Bleistift datiert: »7/9. 99.« \newline{}Ordnung: 1) mit Bleistift von unbekannter Hand nummeriert: »\strikeout{160}« 2) mit Bleistift von unbekannter Hand nummeriert:
                                    »157«}\buchAbdrucke{\weitereDrucke{Hugo von Hofmannsthal, Arthur Schnitzler: \emph{Briefwechsel}. Hg. Therese Nickl und Heinrich Schnitzler. Frankfurt am Main: \emph{S. Fischer} 1964, S. 129–130.} }\toendnotes[C]{\smallbreak}\pstart{}{\pb}mein lieber Arthur\pend\pstart
           ſeien Sie nicht bös ich hab in meinen Kopfſchmerzen \label{K_L00969_1v}\edtext{geſtern}{\lemma{\textnormal{\emph{geſtern}}}\Cendnote{\textnormal{In Schnitzler\pwindex{Schnitzler, Arthur 15.05.1862 – 21.10.1931@\textsc{Schnitzler, Arthur} (15.05.1862 – 21.10.1931), \emph{Schriftsteller, Mediziner}|pwk}s \emph{Tagebuch}\pwindex{Schnitzler, Arthur 15.05.1862 – 21.10.1931@\textsc{Schnitzler, Arthur} (15.05.1862 – 21.10.1931), \emph{Schriftsteller, Mediziner}!Tagebuch1981 – 2000@\strich\emph{Tagebuch} {[}1981 – 2000{]}|pwk} ist die Abreise am 7. 9. 1899 vermerkt. Entsprechend ist dieses
                  Korrespondenzstück auf den Folgetag zu datieren.}}}\label{K_L00969_1h} verſchiedenes in Iſchl\oindex{Bad Ischl@\textbf{Bad Ischl}|pw} liegen laſſen. Bitte ſeien Sie ſo lieb und
               verſchaffen mirs wieder. Erſtens hab ich in meinem Bett mein Nachthemd liegen laſſen.
               Bitte vielmals laſſens {\pb}Sie mirs
               durch den \textsc{Petter}\pwindex{Petter, Leopold 17.11.1850 – 03.07.1917@\textsc{Petter, Leopold} (17.11.1850 – 03.07.1917), \emph{Hotelier}|pw}{ }ſchicken, als Poſtpacket. Das zweite tut mir aber
               noch viel mehr leid. Ich hab auf der Bahn durch Schlamperei des Trägers (\uuline{\textsc{N\textsuperscript{o}} 1}) mein von Ihnen bewundertes dunkles Schirmfutteral mit einem {\pb}ſchönen Schirm von Rodeck\orgindex{Gebrueder Rodeck@Gebrüder Rodeck|pw} und grauem Naturſtock vergeſſen. Bitte
               vielmals gehen Sie zum Stationschef\pwindex{Miliczek, Ferdinand @\textsc{Miliczek, Ferdinand}, \emph{Stationsvorsteher}|pwv} und Sie werdens gewiſs beko{\geminationm}en.
               Bitte vielmals ſchicken Sie mir dann das Packet (das ist das wenigſt mühſame für Sie)
                  {\pb}in die große \textsc{Gassner-Villa}\oindex{Villa Gassner@\textbf{Villa Gassner}|pw} mit der Weiſung, Gehört Hofmannsthal, ſoll liegen bleiben.\pend
           \pstart Nicht bös ſein. Ihr \spacefill\mbox{Hugo.}\pend{}\endnumbering\briefempfaengerindex{Schnitzler, Arthur@\textsc{Schnitzler, Arthur}!zzzHofmannsthal, Hugo von@\emph{von Hugo von Hofmannsthal}!1899-09-081@{{[}8. 9. 1899{]}}|)be}\mylabel{h}\end{ledgroupsized}  \newcommand{\dateiname}{L00969}\newcommand{\titel}{Hugo von Hofmannsthal an Arthur Schnitzler, [8. 9. 1899]}\newcommand{\editorInnen}{Martin Anton Müller und Gerd-Hermann Susen}\input{../tex-inputs/latex-pdf-abspann}
      