%% latex-leseansicht-vorspann.tex
%% Vorspann für die Leseansicht.
%% Lädt die gemeinsame Datei latex-vorspann.tex mit nicht gesetztem Schalter.

\newif\ifkorrekturansicht
\korrekturansichtfalse

\input{../tex-inputs/latex-vorspann}


\section[Hugo von Hofmannsthal an Arthur Schnitzler, {{[}}8. 9. 1899{{]}}]{L00969 Hugo von Hofmannsthal an Arthur Schnitzler, {[}8. 9. 1899{]}}
\nopagebreak\mylabel{L00969v}
\rehead{ }\normalsize\beginnumbering\briefempfaengerindex{Schnitzler, Arthur@\textsc{Schnitzler, Arthur}!zzzHofmannsthal, Hugo von@\emph{von Hugo von Hofmannsthal}!1899-09-081@{{[}8. 9. 1899{]}}|(be}
\toendnotes[C]{\smallbreak\pagebreak[2]}
\correspDesc{Versand  durch Hugo von Hofmannsthal am [8. 9. 1899] in Altaussee
\newline{}Erhalt  durch Arthur Schnitzler im Zeitraum [9. 9. 1899
                  – 13. 9. 1899?] in Bad Ischl}\toendnotes[C]{\smallbreak}
\Standort{CUL, Schnitzler, B 43.}
\physDesc{Brief, 1 Blatt, 4 Seiten, 782 Zeichen
\newline{}Handschrift: schwarze Tinte, deutsche Kurrent
\newline{}Schnitzler: mit Bleistift datiert: »7/9. 99.« 
\newline{}Ordnung: 1) mit Bleistift von unbekannter Hand nummeriert: »\strikeout{160}«  2) mit Bleistift von unbekannter Hand nummeriert:
                                    »157«}
\buchAbdrucke{\weitereDrucke{Hugo von Hofmannsthal, Arthur Schnitzler: \emph{Briefwechsel}. Herausgegeben von Therese Nickl und Heinrich Schnitzler. Frankfurt am Main: \emph{S. Fischer} 1964, S. 129–130.} }\toendnotes[C]{\smallbreak}
\pstart{}{\pb}mein lieber Arthur\pend\vspace{0.5em}
\pstart
           ſeien Sie nicht bös ich hab in meinen Kopfſchmerzen \label{K_L00969-1v}\edtext{geſtern}{\lemma{\textnormal{\emph{gestern}}}\Cendnote{\textnormal{In Schnitzlers{ }\emph{Tagebuch}\pwindex{Schnitzler, Arthur 15.\,5.\,1862 Wien – 21.\,10.\,1931 ebd.@\textsc{Schnitzler, Arthur} (15.\,5.\,1862 Wien – 21.\,10.\,1931 ebd.), \emph{Schriftsteller, Mediziner}!Tagebuch@\strich\emph{Tagebuch}|pwk} ist die Abreise am 7. 9. 1899 vermerkt. Entsprechend ist
                  dieses Korrespondenzstück auf den Folgetag zu datieren.}}}\label{K_L00969-1} verſchiedenes in
                  Iſchl\oindex{Bad Ischl@\textbf{Bad Ischl}|pw} liegen laſſen. Bitte{ }ſeien Sie{ }ſo lieb
               und verſchaffen mirs wieder. Erſtens hab ich in meinem Bett mein Nachthemd liegen
               laſſen. Bitte vielmals laſſens {\pb}Sie mirs durch den \textsc{Petter}\pwindex{Petter, Leopold 17.\,11.\,1850 Bad Ischl – 3.\,7.\,1917 ebd.@\textsc{Petter, Leopold} (17.\,11.\,1850 Bad Ischl – 3.\,7.\,1917 ebd.), \emph{Hotelier}|pw}{ }ſchicken, als Poſtpacket. Das zweite tut mir aber
               noch viel mehr leid. Ich hab auf der Bahn durch Schlamperei des Trägers (\uuline{\textsc{N\textsuperscript{o}} 1}) mein von Ihnen bewundertes dunkles Schirmfutteral mit einem {\pb}ſchönen Schirm von Rodeck\orgindex{Gebrüder Rodeck@Gebrüder Rodeck|pw} und grauem Naturſtock vergeſſen. Bitte
               vielmals gehen Sie zum Stationschef\pwindex{Miliczek, Ferdinand @\textsc{Miliczek, Ferdinand}, \emph{Stationsvorsteher}|pwv} und Sie werdens gewiſs beko{\geminationm}en.
               Bitte vielmals{ }ſchicken Sie mir dann das Packet (das ist das wenigſt mühſame für Sie)
                  {\pb}in die große \textsc{Gassner-Villa}\oindex{Villa Gassner@\textbf{Villa Gassner}, \emph{Gebäude}|pw} mit der Weiſung, Gehört Hofmannsthal,{ }ſoll liegen bleiben.\pend
           \pstart Nicht bös{ }ſein. Ihr \spacefill\mbox{Hugo.}\pend{}\selectlanguage{ngerman}\endnumbering\briefempfaengerindex{Schnitzler, Arthur@\textsc{Schnitzler, Arthur}!zzzHofmannsthal, Hugo von@\emph{von Hugo von Hofmannsthal}!1899-09-081@{{[}8. 9. 1899{]}}|)be}\mylabel{L00969h}  \newcommand{\dateiname}{L00969}\newcommand{\titel}{Hugo von Hofmannsthal an Arthur Schnitzler, [8. 9. 1899]}\newcommand{\editorInnen}{Martin Anton Müller und Gerd-Hermann Susen}%% latex-leseansicht-abspann.tex
%% Abspann für die Leseansicht.
%% Der Schalter \ifkorrekturansicht ist bereits durch den Vorspann gesetzt.

%% latex-abspann.tex
%% Gemeinsamer Abspann für Korrekturansicht und Leseansicht.
%% Setzt den Schalter \ifkorrekturansicht voraus (gesetzt in den
%% einbindenden Dateien latex-korrekturansicht-abspann.tex bzw.
%% latex-leseansicht-abspann.tex).
%% ---------------------------------------------------------------

\normalsize

% Das esempio-Environment wird nur in der Leseansicht benötigt
\ifkorrekturansicht\else
\newenvironment{esempio}[3]%
{
    \vspace{1.5ex}
    \rlap{\underline{#1}}
    \par
    \setlength{\parindent}{0cm}
    \nopagebreak
    \leftskip=#2cm
    \rightskip=#3cm
}
{
    \par
}
\fi

\doendnotes{C}
\bigskip
\vfill

\clearpage

\footnotesize

\ifkorrekturansicht
  \lohead{\textsc{register}}
\fi

% theindex-Environment neu definieren ohne reledmac
\makeatletter
\renewenvironment{theindex}{%
  \ifkorrekturansicht
    \section*{\indexname}%
  \else
    \subsubsection*{Index der erwähnten Entitäten}%
  \fi
  \setlength{\parindent}{0pt}%
  \setlength{\parskip}{0pt plus 0.3pt}%
  \let\item\@idxitem
}{%
  \ifkorrekturansicht\clearpage\fi
}
\makeatother

\IfFileExists{\jobname-pw.ind}{\input{\jobname-pw.ind}}{}

% Quellenangabe nur in der Leseansicht
\ifkorrekturansicht\else
% Fallback-Definitionen, falls die .tex-Datei \titel etc. nicht gesetzt hat
\providecommand{\titel}{}
\providecommand{\editorInnen}{}
\providecommand{\dateiname}{\jobname}

\vspace{3cm}

\vfill

\footnotesize
\textsc{Quelle}: \titel. Herausgegeben von {\editorInnen}. In: \emph{Arthur Schnitzler: Briefwechsel mit Autorinnen und Autoren}.
 Digitale Edition, https://schnitzler-briefe.acdh.oeaw.ac.at/{\dateiname}.html (Stand \today)
\fi

\end{document}


