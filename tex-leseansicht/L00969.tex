%% latex-korrekturansicht-vorspann.tex
%% Vorspann für die Korrekturansicht.
%% Lädt die gemeinsame Datei latex-vorspann.tex mit gesetztem Schalter.

\newif\ifkorrekturansicht
\korrekturansichttrue

\input{../tex-inputs/latex-vorspann}


\section[Hugo von Hofmannsthal an Arthur Schnitzler, {[}8. 9. 1899{]}]{L00969 Hugo von Hofmannsthal an Arthur Schnitzler, {[}8. 9. 1899{]}}
\nopagebreak\mylabel{L00969v}
\rehead{ }\normalsize\beginnumbering\briefempfaengerindex{Schnitzler, Arthur@\textsc{Schnitzler, Arthur}!zzzHofmannsthal, Hugo von@\emph{von Hugo von Hofmannsthal}!1899-09-081@{{[}8. 9. 1899{]}}|(be}
\toendnotes[C]{\smallbreak\pagebreak[2]}\Standort{CUL, Schnitzler, B 43.}
\physDesc{Brief, 1 Blatt, 4 Seiten, 782 Zeichen
\newline{}Handschrift: schwarze Tinte, deutsche Kurrent
\newline{}Schnitzler: mit Bleistift datiert: »7/9. 99.« 
\newline{}Ordnung: 1) mit Bleistift von unbekannter Hand nummeriert: »\strikeout{160}«  2) mit Bleistift von unbekannter Hand nummeriert:
                                    »157«}
\buchAbdrucke{\weitereDrucke{Hugo von Hofmannsthal, Arthur Schnitzler: \emph{Briefwechsel}. Frankfurt am Main: \emph{S. Fischer} 1964, S. 129–130.} }\toendnotes[C]{\smallbreak}
\pstart{}{\pb}mein lieber Arthur\pend\vspace{0.5em}
\pstart
           ſeien Sie nicht bös ich hab in meinen Kopfſchmerzen \label{K_L00969-1v}\edtext{geſtern}{\lemma{\textnormal{\emph{geſtern}}}\Cendnote{\textnormal{In Schnitzlers{ }\emph{Tagebuch}\pwindex{Tagebuch@\emph{Tagebuch}|pwk} ist die Abreise am 7. 9. 1899 vermerkt. Entsprechend ist
                  dieses Korrespondenzstück auf den Folgetag zu datieren.}}}\label{K_L00969-1} verſchiedenes in
                  Iſchl\oindex{Bad Ischl@\textbf{Bad Ischl}, \emph{P.PPL}|pw} liegen laſſen. Bitte ſeien Sie ſo lieb
               und verſchaffen mirs wieder. Erſtens hab ich in meinem Bett mein Nachthemd liegen
               laſſen. Bitte vielmals laſſens {\pb}Sie mirs durch den \textsc{Petter}\pwindex{Petter, Leopold 17.11.1850 – 03.07.1917@\textsc{Petter, Leopold} (17.11.1850 – 03.07.1917), \emph{Hotelier/Hotelière}|pw}{ }ſchicken, als Poſtpacket. Das zweite tut mir aber
               noch viel mehr leid. Ich hab auf der Bahn durch Schlamperei des Trägers (\uuline{\textsc{N\textsuperscript{o}} 1}) mein von Ihnen bewundertes dunkles Schirmfutteral mit einem {\pb}ſchönen Schirm von Rodeck\orgindex{Gebrueder Rodeck@Gebrüder Rodeck|pw} und grauem Naturſtock vergeſſen. Bitte
               vielmals gehen Sie zum Stationschef\pwindex{Miliczek, Ferdinand @\textsc{Miliczek, Ferdinand}, \emph{Stationsvorsteher/Stationsvorsteherin}|pwv} und Sie werdens gewiſs beko{\geminationm}en.
               Bitte vielmals ſchicken Sie mir dann das Packet (das ist das wenigſt mühſame für Sie)
                  {\pb}in die große \textsc{Gassner-Villa}\oindex{Villa Gassner@\textbf{Villa Gassner}, \emph{Gebäude (K.GBD)}|pw} mit der Weiſung, Gehört Hofmannsthal, ſoll liegen bleiben.\pend
           \pstart Nicht bös ſein. Ihr \spacefill\mbox{Hugo.}\pend{}\selectlanguage{ngerman}\endnumbering\briefempfaengerindex{Schnitzler, Arthur@\textsc{Schnitzler, Arthur}!zzzHofmannsthal, Hugo von@\emph{von Hugo von Hofmannsthal}!1899-09-081@{{[}8. 9. 1899{]}}|)be}\mylabel{L00969h}  \normalsize

\doendnotes{C}
\bigskip
\vfill

\clearpage

\footnotesize

\lohead{\textsc{register}}

% Definiere theindex-Environment komplett neu ohne reledmac
\makeatletter
\renewenvironment{theindex}{%
  \section*{\indexname}%
  \setlength{\parindent}{0pt}%
  \setlength{\parskip}{0pt plus 0.3pt}%
  \let\item\@idxitem
}{%
  \clearpage
}
\makeatother

\IfFileExists{\jobname-pw.ind}{\input{\jobname-pw.ind}}{}

\end{document}

      