%% latex-leseansicht-vorspann.tex
%% Vorspann für die Leseansicht.
%% Lädt die gemeinsame Datei latex-vorspann.tex mit nicht gesetztem Schalter.

\newif\ifkorrekturansicht
\korrekturansichtfalse

\input{../tex-inputs/latex-vorspann}


\section[Richard Beer-Hofmann an Arthur Schnitzler, 10. 3. 1892]{L00078 Richard Beer-Hofmann an Arthur Schnitzler, 10. 3. 1892}
\nopagebreak\mylabel{L00078v}
\rehead{ }\normalsize\beginnumbering\briefempfaengerindex{Schnitzler, Arthur@\textsc{Schnitzler, Arthur}!zzzBeer-Hofmann, Richard@\emph{von Richard Beer-Hofmann}!1892-03-101@{10. 3. 1892}|(be}
\toendnotes[C]{\smallbreak\pagebreak[2]}
\correspDesc{Versand  durch Richard Beer-Hofmann am 10. 3. 1892 in Opatija
\newline{}Erhalt  durch Arthur Schnitzler im Zeitraum [11. 3. 1892
                  – 15. 3. 1892?] in Wien}\toendnotes[C]{\smallbreak}
\Standort{CUL, Schnitzler, B 8.}
\physDesc{Brief, 1 Blatt, 4 Seiten, 802 Zeichen (Briefpapier mit Trauerrand)
\newline{}Handschrift: blauer Buntstift, lateinische Kurrent
\newline{}Schnitzler: mit Bleistift nummeriert: »8« }
\buchAbdrucke{\weitereDrucke{Arthur Schnitzler, Richard Beer-Hofmann: \emph{Briefwechsel 1891–1931}. Herausgegeben von Konstanze Fliedl. Wien, Zürich: \emph{Europaverlag} 1992, S. 33.} }\toendnotes[C]{\smallbreak}
\pstart
           {\pb}\textcolor{gray}{\textbf{RB}}\pend
           
\pstart{}Lieber Arthur!\pend\vspace{0.5em}
\pstart
           Ich wohne \uline{Pension Quisisana\oindex{Pension Quisisana@\textbf{Pension Quisisana}, \emph{Hotel}|pw}}; was machen Sie, Loris\pwindex{Hofmannsthal, Hugo von 1.\,2.\,1874 Wien – 15.\,7.\,1929 Rodaun@\textsc{Hofmannsthal, Hugo von} (1.\,2.\,1874 Wien – 15.\,7.\,1929 Rodaun), \emph{Schriftsteller}|pw}, Salten\pwindex{Salten, Felix 6.\,9.\,1869 Budapest – 8.\,10.\,1945 Zürich@\textsc{Salten, Felix} (6.\,9.\,1869 Budapest – 8.\,10.\,1945 Zürich), \emph{Schriftsteller, Journalist, Chefredakteur}|pw}?\pend
           
\pstart
           Wird etwas aus der Vorstellung, hat Kaffka\pwindex{Kafka, Eduard Michael 11.\,3.\,1869 Wien – 6.\,8.\,1893 Brünn@\textsc{Kafka, Eduard Michael} (11.\,3.\,1869 Wien – 6.\,8.\,1893 Brünn), \emph{Redakteur}|pw}
               Nachrichten von der »freien Bühne\orgindex{»Freie Bühne« Verein für moderne Literatur@»Freie Bühne« Verein für moderne Literatur|pw}« wegen »Camelias\pwindex{Beer-Hofmann, Richard 11.\,7.\,1866 Wien – 26.\,9.\,1945 New York City@\textsc{Beer-Hofmann, Richard} (11.\,7.\,1866 Wien – 26.\,9.\,1945 New York City), \emph{Schriftsteller}!Camelias@\strich\emph{Camelias}|pw}«?\pend
           
\pstart
           {\pb}Ich faullenze und langweile mich;
               keine gesunde erquiquende ruhige Langeweile, sondern eine pretentiöse, lärmende mit
               Gesprächen, und Gesellschaft; ausserdem regnet es heute auch noch. Ist \label{K_L00078-1v}\edtext{mein Artikel}{\lemma{\textnormal{\emph{mein Artikel}}}\Cendnote{\textnormal{über Maximilian
                  Harden\pwindex{Harden, Maximilian 20.\,10.\,1861 Berlin – 30.\,10.\,1927 Montana@\textsc{Harden, Maximilian} (20.\,10.\,1861 Berlin – 30.\,10.\,1927 Montana), \emph{Schriftsteller, Publizist}|pwk}: 
                  Richard Beer-Hofmann\pwindex{Beer-Hofmann, Richard 11.\,7.\,1866 Wien – 26.\,9.\,1945 New York City@\textsc{Beer-Hofmann, Richard} (11.\,7.\,1866 Wien – 26.\,9.\,1945 New York City), \emph{Schriftsteller}|pwk}:
                     \emph{Maximilian Harden}\pwindex{Beer-Hofmann, Richard 11.\,7.\,1866 Wien – 26.\,9.\,1945 New York City@\textsc{Beer-Hofmann, Richard} (11.\,7.\,1866 Wien – 26.\,9.\,1945 New York City), \emph{Schriftsteller}!Maximilian Harden@\strich\emph{Maximilian Harden}|pwk}. In: \emph{Wiener Allgemeinen Zeitung}\pwindex{Wiener Allgemeine Zeitung@\emph{Wiener Allgemeine Zeitung}|pwk}, Nr. 4213,
                   30. 4. 1892, S. 7–8.}}}\label{K_L00078-1} in der »Frankfurter\pwindex{Frankfurter Zeitung@\emph{Frankfurter Zeitung}|pw}« erschienen? {\pb}Ich glaube nicht; schon wegen der
                  \introOben{}letzten\introOben{}{ }\label{K_L00078-2v}\edtext{Confiscation}{\lemma{\textnormal{\emph{Confiscation}}}\Cendnote{\textnormal{ Die Morgenausgabe der \emph{Frankfurter Zeitung}\pwindex{Frankfurter Zeitung@\emph{Frankfurter Zeitung}|pwk} vom 1. 3. 1893 war wegen eines Beitrags von Maximilian Harden\pwindex{Harden, Maximilian 20.\,10.\,1861 Berlin – 30.\,10.\,1927 Montana@\textsc{Harden, Maximilian} (20.\,10.\,1861 Berlin – 30.\,10.\,1927 Montana), \emph{Schriftsteller, Publizist}|pwk}
                     – \emph{Gekrönte Worte}\pwindex{Harden, Maximilian 20.\,10.\,1861 Berlin – 30.\,10.\,1927 Montana@\textsc{Harden, Maximilian} (20.\,10.\,1861 Berlin – 30.\,10.\,1927 Montana), \emph{Schriftsteller, Publizist}!Gekrönte Worte@\strich\emph{Gekrönte Worte}|pwk} – beschlagnahmt worden. Dieser hatte sich
                  darin abfällig über eine Rede des deutschen Kaisers Wilhelm II.\pwindex{Wilhelm II. von Preußen 27.\,1.\,1859 Potsdam – 4.\,6.\,1941 Gemeente Utrechtse Heuvelrug@\textsc{Wilhelm II. von Preußen} (27.\,1.\,1859 Potsdam – 4.\,6.\,1941 Gemeente Utrechtse Heuvelrug), \emph{Kaiser}|pwk} geäußert.}}}\label{K_L00078-2}{ }Hardens\pwindex{Harden, Maximilian 20.\,10.\,1861 Berlin – 30.\,10.\,1927 Montana@\textsc{Harden, Maximilian} (20.\,10.\,1861 Berlin – 30.\,10.\,1927 Montana), \emph{Schriftsteller, Publizist}|pw} nicht!\pend
           
\pstart
           Julius Bauer\pwindex{Bauer, Julius 15.\,10.\,1853 Szigetvár – 11.\,6.\,1941 Wien@\textsc{Bauer, Julius} (15.\,10.\,1853 Szigetvár – 11.\,6.\,1941 Wien), \emph{Schriftsteller, Journalist, Kritiker}|pw} ist seit 3 Tagen hier; und spielt
               Piquet. Wir bleiben mindestens eine Woche noch hier, dann vielleicht Venedig\oindex{Venedig@\textbf{Venedig}|pw}. Bitte schreiben Sie mir \uline{recht viel}; wissen Sie: »Glühende Kohlen«.\pend
           
\pstart
           {\pb}ich selbst bin hier mehr als je
               der launeverderbende »Miesmacher{[}«,{]} würde Hermann Cagliostro\pwindex{Cagliostro, Alessandro von 8.\,6.\,1743 Palermo – 26.\,8.\,1795 San Leo@\textsc{Cagliostro, Alessandro von} (8.\,6.\,1743 Palermo – 26.\,8.\,1795 San Leo), \emph{Abenteurer, Hochstapler}|pwv} (Bahr\pwindex{Bahr, Hermann 19.\,7.\,1863 Linz – 15.\,1.\,1934 München@\textsc{Bahr, Hermann} (19.\,7.\,1863 Linz – 15.\,1.\,1934 München), \emph{Schriftsteller, Kritiker}|pw}) sagen.\pend
           
\pstart
           Ich grüße Sie von Herzen.
               {\\[\baselineskip]}\spacefill\mbox{Richard}\pend
           \leftskip=0em{}
\pstart
           10/III 92{ }Abbazia\oindex{Opatija@\textbf{Opatija}, \emph{Hauptstadt}|pw}\pend
           \selectlanguage{ngerman}\endnumbering\briefempfaengerindex{Schnitzler, Arthur@\textsc{Schnitzler, Arthur}!zzzBeer-Hofmann, Richard@\emph{von Richard Beer-Hofmann}!1892-03-101@{10. 3. 1892}|)be}\mylabel{L00078h}  \newcommand{\dateiname}{L00078}\newcommand{\titel}{Richard Beer-Hofmann an Arthur Schnitzler, 10. 3. 1892}\newcommand{\editorInnen}{Gerd-Hermann Susen und Martin Anton Müller}%% latex-leseansicht-abspann.tex
%% Abspann für die Leseansicht.
%% Der Schalter \ifkorrekturansicht ist bereits durch den Vorspann gesetzt.

%% latex-abspann.tex
%% Gemeinsamer Abspann für Korrekturansicht und Leseansicht.
%% Setzt den Schalter \ifkorrekturansicht voraus (gesetzt in den
%% einbindenden Dateien latex-korrekturansicht-abspann.tex bzw.
%% latex-leseansicht-abspann.tex).
%% ---------------------------------------------------------------

\normalsize

% Das esempio-Environment wird nur in der Leseansicht benötigt
\ifkorrekturansicht\else
\newenvironment{esempio}[3]%
{
    \vspace{1.5ex}
    \rlap{\underline{#1}}
    \par
    \setlength{\parindent}{0cm}
    \nopagebreak
    \leftskip=#2cm
    \rightskip=#3cm
}
{
    \par
}
\fi

\doendnotes{C}
\bigskip
\vfill

\clearpage

\footnotesize

\ifkorrekturansicht
  \lohead{\textsc{register}}
\fi

% theindex-Environment neu definieren ohne reledmac
\makeatletter
\renewenvironment{theindex}{%
  \ifkorrekturansicht
    \section*{\indexname}%
  \else
    \subsubsection*{Index der erwähnten Entitäten}%
  \fi
  \setlength{\parindent}{0pt}%
  \setlength{\parskip}{0pt plus 0.3pt}%
  \let\item\@idxitem
}{%
  \ifkorrekturansicht\clearpage\fi
}
\makeatother

\IfFileExists{\jobname-pw.ind}{\input{\jobname-pw.ind}}{}

% Quellenangabe nur in der Leseansicht
\ifkorrekturansicht\else
% Fallback-Definitionen, falls die .tex-Datei \titel etc. nicht gesetzt hat
\providecommand{\titel}{}
\providecommand{\editorInnen}{}
\providecommand{\dateiname}{\jobname}

\vspace{3cm}

\vfill

\footnotesize
\textsc{Quelle}: \titel. Herausgegeben von {\editorInnen}. In: \emph{Arthur Schnitzler: Briefwechsel mit Autorinnen und Autoren}.
 Digitale Edition, https://schnitzler-briefe.acdh.oeaw.ac.at/{\dateiname}.html (Stand \today)
\fi

\end{document}


