%% latex-korrekturansicht-vorspann.tex
%% Vorspann für die Korrekturansicht.
%% Lädt die gemeinsame Datei latex-vorspann.tex mit gesetztem Schalter.

\newif\ifkorrekturansicht
\korrekturansichttrue

\input{../tex-inputs/latex-vorspann}


\section[Richard Beer-Hofmann an Arthur Schnitzler, 10. 3. 1892]{L00078 Richard Beer-Hofmann an Arthur Schnitzler, 10. 3. 1892}
\nopagebreak\mylabel{L00078v}
\rehead{ }\normalsize\beginnumbering\briefempfaengerindex{Schnitzler, Arthur@\textsc{Schnitzler, Arthur}!zzzBeer-Hofmann, Richard@\emph{von Richard Beer-Hofmann}!1892-03-101@{10. 3. 1892}|(be}
\toendnotes[C]{\smallbreak\pagebreak[2]}\Standort{CUL, Schnitzler, B 8.}
\physDesc{Brief, 1 Blatt, 4 Seiten, 802 Zeichen (Briefpapier mit Trauerrand)
\newline{}Handschrift: blauer Buntstift, lateinische Kurrent
\newline{}Schnitzler: mit Bleistift nummeriert: »8« }
\buchAbdrucke{\weitereDrucke{Arthur Schnitzler, Richard Beer-Hofmann: \emph{Briefwechsel 1891–1931}. Wien, Zürich: \emph{Europaverlag} 1992, S. 33.} }\toendnotes[C]{\smallbreak}
\pstart
           {\pb}\textcolor{gray}{\textbf{RB}}\pend
           
\pstart{}Lieber Arthur!\pend\vspace{0.5em}
\pstart
           Ich wohne \uline{Pension Quisisana\oindex{Pension Quisisana@\textbf{Pension Quisisana}, \emph{Hotel (K.HTL)}|pw}}; was machen Sie, Loris\pwindex{Hofmannsthal, Hugo von 1874-02-01 – 1929-07-15@\textsc{Hofmannsthal, Hugo von} (1874-02-01 – 1929-07-15), \emph{Schriftsteller/Schriftstellerin}|pw}, Salten\pwindex{Salten, Felix 06.09.1869 – 08.10.1945@\textsc{Salten, Felix} (06.09.1869 – 08.10.1945), \emph{Schriftsteller/Schriftstellerin, Journalist/Journalistin, Chefredakteur/Chefredakteurin}|pw}?\pend
           
\pstart
           Wird etwas aus der Vorstellung, hat Kaffka\pwindex{Kafka, Eduard Michael 11.03.1869 – 06.08.1893@\textsc{Kafka, Eduard Michael} (11.03.1869 – 06.08.1893), \emph{Redakteur/Redakteurin}|pw}
               Nachrichten von der »freien Bühne\orgindex{»Freie Buehne« Verein fuer moderne Literatur@»Freie Bühne« Verein für moderne Literatur|pw}« wegen »Camelias\pwindex{Camelias@\emph{Camelias}|pw}«?\pend
           
\pstart
           {\pb}Ich faullenze und langweile mich;
               keine gesunde erquiquende ruhige Langeweile, sondern eine pretentiöse, lärmende mit
               Gesprächen, und Gesellschaft; ausserdem regnet es heute auch noch. Ist \label{K_L00078-1v}\edtext{mein Artikel}{\lemma{\textnormal{\emph{mein Artikel}}}\Cendnote{\textnormal{über Maximilian
                  Harden\pwindex{Harden, Maximilian 20.10.1861 – 30.10.1927@\textsc{Harden, Maximilian} (20.10.1861 – 30.10.1927), \emph{Schriftsteller/Schriftstellerin, Publizist/Publizistin}|pwk}: 
                  Richard Beer-Hofmann\pwindex{Beer-Hofmann, Richard 1866-07-11 – 1945-09-26@\textsc{Beer-Hofmann, Richard} (1866-07-11 – 1945-09-26), \emph{Schriftsteller/Schriftstellerin}|pwk}:
                     \emph{Maximilian Harden}\pwindex{Maximilian Harden@\emph{Maximilian Harden}|pwk}. In: \emph{Wiener Allgemeinen Zeitung}\pwindex{Wiener Allgemeine Zeitung@\emph{Wiener Allgemeine Zeitung}|pwk}, Nr. 4213,
                   30. 4. 1892, S. 7–8.}}}\label{K_L00078-1} in der »Frankfurter\pwindex{Frankfurter Zeitung@\emph{Frankfurter Zeitung}|pw}« erschienen? {\pb}Ich glaube nicht; schon wegen der
                  \introOben{}letzten\introOben{}{ }\label{K_L00078-2v}\edtext{Confiscation}{\lemma{\textnormal{\emph{Confiscation}}}\Cendnote{\textnormal{ Die Morgenausgabe der \emph{Frankfurter Zeitung}\pwindex{Frankfurter Zeitung@\emph{Frankfurter Zeitung}|pwk} vom 1. 3. 1893 war wegen eines Beitrags von Maximilian Harden\pwindex{Harden, Maximilian 20.10.1861 – 30.10.1927@\textsc{Harden, Maximilian} (20.10.1861 – 30.10.1927), \emph{Schriftsteller/Schriftstellerin, Publizist/Publizistin}|pwk}
                     – \emph{Gekrönte Worte}\pwindex{Gekroente Worte@\emph{Gekrönte Worte}|pwk} – beschlagnahmt worden. Dieser hatte sich
                  darin abfällig über eine Rede des deutschen Kaisers Wilhelm II.\pwindex{Wilhelm II. von Preussen 27.1.1859 – 4.6.1941@\textsc{Wilhelm II. von Preußen} (27.1.1859 – 4.6.1941), \emph{Kaiser/Kaiserin}|pwk} geäußert.}}}\label{K_L00078-2}{ }Hardens\pwindex{Harden, Maximilian 20.10.1861 – 30.10.1927@\textsc{Harden, Maximilian} (20.10.1861 – 30.10.1927), \emph{Schriftsteller/Schriftstellerin, Publizist/Publizistin}|pw} nicht!\pend
           
\pstart
           Julius Bauer\pwindex{Bauer, Julius 15.10.1853 – 11.06.1941@\textsc{Bauer, Julius} (15.10.1853 – 11.06.1941), \emph{Schriftsteller/Schriftstellerin, Journalist/Journalistin, Kritiker/Kritikerin}|pw} ist seit 3 Tagen hier; und spielt
               Piquet. Wir bleiben mindestens eine Woche noch hier, dann vielleicht Venedig\oindex{Venedig@\textbf{Venedig}, \emph{P.PPLA}|pw}. Bitte schreiben Sie mir \uline{recht viel}; wissen Sie: »Glühende Kohlen«.\pend
           
\pstart
           {\pb}ich selbst bin hier mehr als je
               der launeverderbende »Miesmacher{[}«,{]} würde Hermann Cagliostro\pwindex{Cagliostro, Alessandro von 1743-06-08 – 1795-08-26@\textsc{Cagliostro, Alessandro von} (1743-06-08 – 1795-08-26), \emph{Abenteurer/Abenteurerin, Hochstapler/Hochstaplerin}|pwv} (Bahr\pwindex{Bahr, Hermann 19.07.1863 – 15.01.1934@\textsc{Bahr, Hermann} (19.07.1863 – 15.01.1934), \emph{Schriftsteller/Schriftstellerin, Kritiker/Kritikerin}|pw}) sagen.\pend
           
\pstart
           Ich grüße Sie von Herzen.
               {\\[\baselineskip]}\spacefill\mbox{Richard}\pend
           \leftskip=0em{}
\pstart
           10/III 92{ }Abbazia\oindex{Opatija@\textbf{Opatija}, \emph{P.PPLA2}|pw}\pend
           \selectlanguage{ngerman}\endnumbering\briefempfaengerindex{Schnitzler, Arthur@\textsc{Schnitzler, Arthur}!zzzBeer-Hofmann, Richard@\emph{von Richard Beer-Hofmann}!1892-03-101@{10. 3. 1892}|)be}\mylabel{L00078h}  \normalsize

\doendnotes{C}
\bigskip
\vfill

\clearpage

\footnotesize

\lohead{\textsc{register}}

% Definiere theindex-Environment komplett neu ohne reledmac
\makeatletter
\renewenvironment{theindex}{%
  \section*{\indexname}%
  \setlength{\parindent}{0pt}%
  \setlength{\parskip}{0pt plus 0.3pt}%
  \let\item\@idxitem
}{%
  \clearpage
}
\makeatother

\IfFileExists{\jobname-pw.ind}{\input{\jobname-pw.ind}}{}

\end{document}

      