%% latex-leseansicht-vorspann.tex
%% Vorspann für die Leseansicht.
%% Lädt die gemeinsame Datei latex-vorspann.tex mit nicht gesetztem Schalter.

\newif\ifkorrekturansicht
\korrekturansichtfalse

\input{../tex-inputs/latex-vorspann}


         
         \renewcommand{\erwaehntePersonen}{Personen: Hermann Bahr, Julius Bauer, Richard Beer-Hofmann, Alessandro von Cagliostro, Maximilian Harden, Hugo von Hofmannsthal, Eduard Michael Kafka, Felix Salten,  Wilhelm II. von Preußen}
         \renewcommand{\erwaehnteInstitutionen}{Institutionen: »Freie Bühne« Verein für moderne Literatur}
         \renewcommand{\erwaehnteOrte}{Orte: Opatija, Pension Quisisana, Venedig, Wien}
         \renewcommand{\erwaehnteWerke}{Werke: Camelias, Frankfurter Zeitung, Gekrönte Worte, Maximilian Harden, Wiener Allgemeine Zeitung}
               \section[Richard Beer-Hofmann an Arthur Schnitzler, 10. 3. 1892]{ Richard Beer-Hofmann an Arthur Schnitzler, 10. 3. 1892}\nopagebreak\mylabel{v}\rehead{ }\begin{ledgroupsized}[t]{13cm}\normalsize\beginnumbering\briefempfaengerindex{Schnitzler, Arthur@\textsc{Schnitzler, Arthur}!zzzBeer-Hofmann, Richard@\emph{von Richard Beer-Hofmann}!1892-03-101@{10. 3. 1892}|(be} \toendnotes[C]{\smallbreak\pagebreak[2]} \Standort{CUL, Schnitzler, B 8.}
\physDesc{Brief, 1 Blatt, 4 Seiten, 802 Zeichen (Briefpapier mit Trauerrand)
\newline{}Handschrift: blauer Buntstift, lateinische Kurrent
\newline{}Schnitzler: mit Bleistift nummeriert: »8« }\buchAbdrucke{\weitereDrucke{Arthur Schnitzler, Richard Beer-Hofmann: \emph{Briefwechsel 1891–1931}. Hg. Konstanze Fliedl. Wien, Zürich: \emph{Europaverlag} 1992, S. 33.} }\toendnotes[C]{\smallbreak}\pstart
           \noindent{}{\pb}\textcolor{gray}{\textbf{RB}}\pend
           \pstart{}Lieber Arthur!\pend\pstart
           Ich wohne \uline{Pension Quisisana\oindex{Pension Quisisana@\textbf{Pension Quisisana}|pw}}; was machen Sie, Loris\pwindex{Hofmannsthal, Hugo von 1874-02-01 – 1929-07-15@\textsc{Hofmannsthal, Hugo von} (1874-02-01 – 1929-07-15), \emph{Schriftsteller}|pw}, Salten\pwindex{Salten, Felix 06.09.1869 – 08.10.1945@\textsc{Salten, Felix} (06.09.1869 – 08.10.1945), \emph{Schriftsteller, Journalist, Chefredakteur}|pw}?\pend
           \pstart
           Wird etwas aus der Vorstellung, hat Kaffka\pwindex{Kafka, Eduard Michael 11.03.1869 – 06.08.1893@\textsc{Kafka, Eduard Michael} (11.03.1869 – 06.08.1893), \emph{Redakteur}|pw}
               Nachrichten von der »freien Bühne\orgindex{»Freie Buehne« Verein fuer moderne Literatur@»Freie Bühne« Verein für moderne Literatur|pw}« wegen »Camelias\pwindex{Beer-Hofmann, Richard 1866-07-11 – 1945-09-26@\textsc{Beer-Hofmann, Richard} (1866-07-11 – 1945-09-26), \emph{Schriftsteller}!Camelias1893@\strich\emph{Camelias} {[}1893{]}|pw}«?\pend
           \pstart
           {\pb}Ich faullenze und langweile mich;
               keine gesunde erquiquende ruhige Langeweile, sondern eine pretentiöse, lärmende mit
               Gesprächen, und Gesellschaft; ausserdem regnet es heute auch noch. Ist \label{K_L00078-1v}\edtext{mein Artikel}{\lemma{\textnormal{\emph{mein Artikel}}}\Cendnote{\textnormal{über Maximilian
                  Harden\pwindex{Harden, Maximilian 20.10.1861 – 30.10.1927@\textsc{Harden, Maximilian} (20.10.1861 – 30.10.1927), \emph{Schriftsteller, Publizist}|pwk}: 
                  Richard Beer-Hofmann\pwindex{Beer-Hofmann, Richard 1866-07-11 – 1945-09-26@\textsc{Beer-Hofmann, Richard} (1866-07-11 – 1945-09-26), \emph{Schriftsteller}|pwk}:
                     \emph{Maximilian Harden}\pwindex{Beer-Hofmann, Richard 1866-07-11 – 1945-09-26@\textsc{Beer-Hofmann, Richard} (1866-07-11 – 1945-09-26), \emph{Schriftsteller}!Maximilian Harden30. 4. 1892@\strich\emph{Maximilian Harden} {[}30. 4. 1892{]}|pwk}. In: \emph{Wiener Allgemeinen Zeitung}\pwindex{Wiener Allgemeine Zeitung1.3.1880 – 11.2.1934@\emph{Wiener Allgemeine Zeitung} {[}1.3.1880 – 11.2.1934{]}|pwk}, Nr. 4213,
                   30. 4. 1892, S. 7–8.}}}\label{K_L00078-1h} in der »Frankfurter\pwindex{?? Werk@Nicht ermittelte Verfasserinnen und Verfasser!Frankfurter Zeitung1856 – 1943@\emph{Frankfurter Zeitung} {[}1856 – 1943{]}|pw}« erschienen? {\pb}Ich glaube nicht; schon wegen der
                  \introOben{}letzten\introOben{}{ }\label{K_L00078-2v}\edtext{Confiscation}{\lemma{\textnormal{\emph{Confiscation}}}\Cendnote{\textnormal{ Die Morgenausgabe der \emph{Frankfurter Zeitung}\pwindex{?? Werk@Nicht ermittelte Verfasserinnen und Verfasser!Frankfurter Zeitung1856 – 1943@\emph{Frankfurter Zeitung} {[}1856 – 1943{]}|pwk} vom 1. 3. 1893 war wegen eines Beitrags von Maximilian Harden\pwindex{Harden, Maximilian 20.10.1861 – 30.10.1927@\textsc{Harden, Maximilian} (20.10.1861 – 30.10.1927), \emph{Schriftsteller, Publizist}|pwk}
                     – \emph{Gekrönte Worte}\pwindex{Harden, Maximilian 20.10.1861 – 30.10.1927@\textsc{Harden, Maximilian} (20.10.1861 – 30.10.1927), \emph{Schriftsteller, Publizist}!Gekroente Worte1. 3. 1892@\strich\emph{Gekrönte Worte} {[}1. 3. 1892{]}|pwk} – beschlagnahmt worden. Dieser hatte sich
                  darin abfällig über eine Rede des deutschen Kaisers Wilhelm II.\pwindex{Wilhelm II. von Preussen 27.1.1859 – 4.6.1941@\textsc{Wilhelm II. von Preußen} (27.1.1859 – 4.6.1941), \emph{Kaiser}|pwk} geäußert.}}}\label{K_L00078-2h}{ }Hardens\pwindex{Harden, Maximilian 20.10.1861 – 30.10.1927@\textsc{Harden, Maximilian} (20.10.1861 – 30.10.1927), \emph{Schriftsteller, Publizist}|pw} nicht!\pend
           \pstart
           Julius Bauer\pwindex{Bauer, Julius 15.10.1853 – 11.06.1941@\textsc{Bauer, Julius} (15.10.1853 – 11.06.1941), \emph{Schriftsteller, Journalist, Kritiker}|pw} ist seit 3 Tagen hier; und spielt
               Piquet. Wir bleiben mindestens eine Woche noch hier, dann vielleicht Venedig\oindex{Venedig@\textbf{Venedig}|pw}. Bitte schreiben Sie mir \uline{recht viel}; wissen Sie: »Glühende Kohlen«.\pend
           \pstart
           {\pb}ich selbst bin hier mehr als je
               der launeverderbende »Miesmacher{[}«,{]} würde Hermann Cagliostro\pwindex{Cagliostro, Alessandro von 1743-06-08 – 1795-08-26@\textsc{Cagliostro, Alessandro von} (1743-06-08 – 1795-08-26), \emph{Abenteurer, Hochstapler}|pwv} (Bahr\pwindex{Bahr, Hermann 19.07.1863 – 15.01.1934@\textsc{Bahr, Hermann} (19.07.1863 – 15.01.1934), \emph{Schriftsteller, Kritiker}|pw}) sagen.\pend
           \pstart
           Ich grüße Sie von Herzen.{\\[\baselineskip]}\spacefill\mbox{Richard}\pend
           \leftskip=0em{}\pstart
           10/III 92{ }Abbazia\oindex{Opatija@\textbf{Opatija}|pw}\pend
           
         
         \endnumbering\mylabel{h}\end{ledgroupsized}  \newcommand{\dateiname}{L00078}\newcommand{\titel}{Richard Beer-Hofmann an Arthur Schnitzler, 10. 3. 1892}\newcommand{\editorInnen}{ Martin Anton Müller und Gerd-Hermann Susen}%% latex-leseansicht-abspann.tex
%% Abspann für die Leseansicht.
%% Der Schalter \ifkorrekturansicht ist bereits durch den Vorspann gesetzt.

%% latex-abspann.tex
%% Gemeinsamer Abspann für Korrekturansicht und Leseansicht.
%% Setzt den Schalter \ifkorrekturansicht voraus (gesetzt in den
%% einbindenden Dateien latex-korrekturansicht-abspann.tex bzw.
%% latex-leseansicht-abspann.tex).
%% ---------------------------------------------------------------

\normalsize

% Das esempio-Environment wird nur in der Leseansicht benötigt
\ifkorrekturansicht\else
\newenvironment{esempio}[3]%
{
    \vspace{1.5ex}
    \rlap{\underline{#1}}
    \par
    \setlength{\parindent}{0cm}
    \nopagebreak
    \leftskip=#2cm
    \rightskip=#3cm
}
{
    \par
}
\fi

\doendnotes{C}
\bigskip
\vfill

\clearpage

\footnotesize

\ifkorrekturansicht
  \lohead{\textsc{register}}
\fi

% theindex-Environment neu definieren ohne reledmac
\makeatletter
\renewenvironment{theindex}{%
  \ifkorrekturansicht
    \section*{\indexname}%
  \else
    \subsubsection*{Index der erwähnten Entitäten}%
  \fi
  \setlength{\parindent}{0pt}%
  \setlength{\parskip}{0pt plus 0.3pt}%
  \let\item\@idxitem
}{%
  \ifkorrekturansicht\clearpage\fi
}
\makeatother

\IfFileExists{\jobname-pw.ind}{\input{\jobname-pw.ind}}{}

% Quellenangabe nur in der Leseansicht
\ifkorrekturansicht\else
% Fallback-Definitionen, falls die .tex-Datei \titel etc. nicht gesetzt hat
\providecommand{\titel}{}
\providecommand{\editorInnen}{}
\providecommand{\dateiname}{\jobname}

\vspace{3cm}

\vfill

\footnotesize
\textsc{Quelle}: \titel. Herausgegeben von {\editorInnen}. In: \emph{Arthur Schnitzler: Briefwechsel mit Autorinnen und Autoren}.
 Digitale Edition, https://schnitzler-briefe.acdh.oeaw.ac.at/{\dateiname}.html (Stand \today)
\fi

\end{document}


      