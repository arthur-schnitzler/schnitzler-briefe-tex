\input{../tex-inputs/latex-pdf-vorspann}
\begin{center}
            \textcolor{red}{ENTWURF. ENTZIFFERUNG NOCH NICHT KORREKTURGELESEN}
                      \end{center}
            
               \section[Arthur Schnitzler an Richard Beer-Hofmann, 20. 5. 1897]{ Arthur Schnitzler an Richard Beer-Hofmann, 20. 5. 1897}\nopagebreak\mylabel{v}\rehead{ }\begin{ledgroupsized}[t]{13cm}\normalsize\beginnumbering\briefempfaengerindex{Beer-Hofmann, Richard@\textsc{Beer-Hofmann, Richard}!zzzSchnitzler, Arthur@\emph{von Arthur Schnitzler}!1897-05-201@{20. 5. 1897}|(be} \toendnotes[C]{\smallbreak\pagebreak[2]} \Standort{YCGL, MSS 31.}
\physDesc{Brief, 2 Blätter, 8 Seiten, Umschlag
\newline{}Handschrift: schwarze Tinte, deutsche Kurrent\newline{}Versand: 1) Stempel: »\nobreak{}\oindex{rue Milton@\textbf{rue Milton}|pwk}Paris 2 B. Milton, 20 Mai 97, 7\textsuperscript{E}\nobreak{}«.  2) Stempel: »\nobreak{}\oindex{I., Innere Stadt@\textbf{I., Innere Stadt}|pwk}Wien 1/1, 22 5. 97, 9–10½V., Bestellt\nobreak{}«. }\buchAbdrucke{\weitereDrucke{1) Arthur Schnitzler: \emph{Briefe 1875–1912}. Hg. Therese Nickl und Heinrich Schnitzler. Frankfurt am Main: \emph{S. Fischer} 1981, S. 322–323.} \weitereDrucke{2) Arthur Schnitzler, Richard Beer-Hofmann: \emph{Briefwechsel 1891–1931}. Hg. Konstanze Fliedl. Wien, Zürich: \emph{Europaverlag} 1992, S. 104–105.} \weitereDrucke{3) Hermann Bahr, Arthur Schnitzler: \emph{Briefwechsel, Aufzeichnungen, Dokumente (1891–1931)}. Hg. Kurt Ifkovits und Martin Anton Müller. Göttingen: \emph{Wallstein} 2018.} }\toendnotes[C]{\smallbreak}\pstart{}{\pb}\textsc{Mr Dr Richard Beer-Hofmann}\pend{}\pstart{}\textsc{Wien}\oindex{Wien@\textbf{Wien}|pw}\pend{}\pstart{}\textsc{I. Wollzeile 15}\oindex{Wollzeile@\textbf{Wollzeile}|pw}\pend{}\pstart{}\textsc{Autriche}\oindex{Oesterreich@\textbf{Österreich}|pw}\pend{}{\bigskip}\pstart
           \raggedleft{}{\pb}20. 5. 97{\\}\textsc{Paris}\oindex{Paris@\textbf{Paris}|pw}.\pend
           \pstart
           Lieber Richard, die Pariſer\oindex{Paris@\textbf{Paris}|pw} Tage
               – ſie werden wahrſcheinlich bald »ſehr ſchön geweſen« ſein – nahen ihrem Ende; Montag
               fahre ich nach London\oindex{London@\textbf{London}|pw} und bin in den ersten
               Junitagen in Wien\oindex{Wien@\textbf{Wien}|pw}. Sie aber fahren bereits in den
               ſelben erſten Junitagen nach Iſchl\oindex{Bad Ischl@\textbf{Bad Ischl}|pw}?\pend
           \pstart
           Ich werde Sie doch hoffentlich noch in Wien\oindex{Wien@\textbf{Wien}|pw}
               finden? Beruhigen {\pb}Sie mich darüber, indem Sie mir
               eine Zeile nach London\oindex{London@\textbf{London}|pw}{ }ſchreiben. Meine Adreſſe ist sehr complicirt: bei
                  \textsc{Felix Markbreiter\pwindex{Markbreiter, Felix 20.11.1855 – 15.09.1914@\textsc{Markbreiter, Felix} (20.11.1855 – 15.09.1914), \emph{Kaufmann}|pw}{ }London S E. Honor Oak, Woodville Hall\oindex{Honor Oak@\textbf{Honor Oak}|pw}}. –\pend
           \pstart
           Paul\pwindex{Goldmann, Paul 31.01.1865 – 25.09.1935@\textsc{Goldmann, Paul} (31.01.1865 – 25.09.1935), \emph{Schriftsteller, Journalist}|pw} behauptet, ſo oft ich irgend ein
               Entzücken oder eine Befriedigung über irgend was hier äußere – und es wi{\geminationm}elt von ſolchen Gelegenheiten, dſs Sie einmal ge{\pb}äußert, Paris\oindex{Paris@\textbf{Paris}|pw}
               hätte Ihnen nichts zu ſagen. Sie werden das einmal beſchämt zurücknehmen. Sie ahnen
               nicht, was Ihnen Paris\oindex{Paris@\textbf{Paris}|pw} alles zu ſagen hätte und
               wie viel Sie gerne antworten möchten. Dieſe Stadt dampft von Cultur, und ich hab mich
               kaum über einen Menſchen ärgern kö{\geminationn}en, der mir zufällig
               heute grad ſagte, er ſei in Wien\oindex{Wien@\textbf{Wien}|pw} geweſen, {\pb}denke gern dran zurück: \textsc{c’est une
                  gentille petite ville}. Man ſpürt auch etwas wahres in dieſer Phraſe: dſs
               eigentlich die ganze Welt in Paris\oindex{Paris@\textbf{Paris}|pw} enthalten ſei;
               man hat eine Ahnung von Unendlichkeit, in der man beinah so einſam ſein könnte wie in
               der Wüſte. Wiſſen Sie, was mir eine große Freude ſein würde? einmal mit Ihnen hieher
               zu kommen – nicht {\pb}ohne Ihnen das Verſprechen abgeno{\geminationm}en zu haben, nicht bei jeder Auslage stehn zu bleiben.
               Ich würde Sie aber nie an die Seine führen, wo an den Quais auf den Steinbrüſtungen
               Millionen Bücher liegen – Sie würden dazu allein zwanzig Jahre brauchen. Dort findet
               man, wie Sie gleich ſehen werden, alle Bücher der Welt; {\pb}um mir eine Emotion zu verſchaffen, hab ich mit einer
               Verkäuferin um ein Exemplar von »\textsc{Mourir}\pwindex{Schnitzler, Arthur 15.05.1862 – 21.10.1931@\textsc{Schnitzler, Arthur} (15.05.1862 – 21.10.1931), \emph{Schriftsteller, Mediziner}!Sterben. Novelle1.10.1894 – 1.12.1894@\strich\emph{Sterben. Novelle} {[}1.10.1894 – 1.12.1894{]}|pw}« »gefeilſcht« – das Luder hat’s mir für 60 \textsc{centimes}
               gelaſſen – unaufgeſchnitten! (das Buch mein ich.)\pend
           \pstart
           – Mit Ihr\pwindex{Reinhard, Marie 13.03.1871 – 18.03.1899@\textsc{Reinhard, Marie} (13.03.1871 – 18.03.1899), \emph{Gesangspädagogin}|pwv} bin ich ſehr
               zufrieden; ſanft, lieb, ein bischen rührend. Ich hab ſie wahrſcheinlich viel lieber,
               als wenn ich ſie lieb hätte. – Wir {\dots} na, wir reden ja in
                  Wien\oindex{Wien@\textbf{Wien}|pw} darüber. –\pend
           \pstart
           {\pb}Der \label{K_L00678_1v}\edtext{Graf\pwindex{Graf, Max 01.10.1873 – 24.06.1958@\textsc{Graf, Max} (01.10.1873 – 24.06.1958), \emph{Kritiker}|pw}}{\lemma{\textnormal{\emph{Graf}}}\Cendnote{\textnormal{Max Graf\pwindex{Graf, Max 01.10.1873 – 24.06.1958@\textsc{Graf, Max} (01.10.1873 – 24.06.1958), \emph{Kritiker}|pwk}}}}\label{K_L00678_1h}, dem Sie die Empfehlung an \strikeout{Richard}{ }Paul\pwindex{Goldmann, Paul 31.01.1865 – 25.09.1935@\textsc{Goldmann, Paul} (31.01.1865 – 25.09.1935), \emph{Schriftsteller, Journalist}|pw} mitgegeben, iſt, losgelöſt von den
               Leuten, unter denen er noch einer der anſtändigſten iſt, ein ganz widerliches
               Subjekt; verlogen und verlottert. Moralſchule Altenberg\pwindex{Altenberg, Peter 09.03.1859 – 08.01.1919@\textsc{Altenberg, Peter} (09.03.1859 – 08.01.1919), \emph{Schriftsteller}|pw}, Beobachtungsſchule Bahr\pwindex{Bahr, Hermann 19.07.1863 – 15.01.1934@\textsc{Bahr, Hermann} (19.07.1863 – 15.01.1934), \emph{Schriftsteller, Kritiker}|pw}.\pend
           \pstart
           Sie\pwindex{Reinhard, Marie 13.03.1871 – 18.03.1899@\textsc{Reinhard, Marie} (13.03.1871 – 18.03.1899), \emph{Gesangspädagogin}|pwv}{ }ſitzt, während ich Ihnen ſchreibe, im Nebenzimmer
               und lieſt eben die Scene\pwindex{Schnitzler, Arthur 15.05.1862 – 21.10.1931@\textsc{Schnitzler, Arthur} (15.05.1862 – 21.10.1931), \emph{Schriftsteller, Mediziner}!Reigen. Zehn Dialoge1900@\strich\emph{Reigen. Zehn Dialoge} {[}1900{]}|pwv}
               zwiſchen dem {\pb}Dichter (Biebitz\pwindex{Schnitzler, Arthur 15.05.1862 – 21.10.1931@\textsc{Schnitzler, Arthur} (15.05.1862 – 21.10.1931), \emph{Schriftsteller, Mediziner}!Reigen. Zehn Dialoge1900@\strich\emph{Reigen. Zehn Dialoge} {[}1900{]}|pwv}) und der Schauſpielerin, die ich übrigens geändert
               habe, ſo dſs man ſagen kann: Biebitz\pwindex{Schnitzler, Arthur 15.05.1862 – 21.10.1931@\textsc{Schnitzler, Arthur} (15.05.1862 – 21.10.1931), \emph{Schriftsteller, Mediziner}!Reigen. Zehn Dialoge1900@\strich\emph{Reigen. Zehn Dialoge} {[}1900{]}|pwv} bleibt Biebitz\pwindex{Schnitzler, Arthur 15.05.1862 – 21.10.1931@\textsc{Schnitzler, Arthur} (15.05.1862 – 21.10.1931), \emph{Schriftsteller, Mediziner}!Reigen. Zehn Dialoge1900@\strich\emph{Reigen. Zehn Dialoge} {[}1900{]}|pwv}! – Aber ſonſt haben Sie hoffentlich mehr gearbeitet als ich. Nach
               dieſen zwei Dingen ſehn ich mich unbeſchreiblich: nach dem Schreiben und nach dem \textsc{Bicycle}! – Kö{\geminationn}en Sie’s endlich?
               (Bicycle natürlich. –)\pend
           \pstart
           Seien Sie herzlich gegrüßt. Ihr{\\[\baselineskip]}\spacefill\mbox{Arthur.}\pend
           \leftskip=0em{}\endnumbering\briefempfaengerindex{Beer-Hofmann, Richard@\textsc{Beer-Hofmann, Richard}!zzzSchnitzler, Arthur@\emph{von Arthur Schnitzler}!1897-05-201@{20. 5. 1897}|)be}\mylabel{h}\end{ledgroupsized}  \newcommand{\dateiname}{L00678}\newcommand{\titel}{Arthur Schnitzler an Richard Beer-Hofmann, 20. 5. 1897}\newcommand{\editorInnen}{ Martin Anton Müller und Gerd-Hermann Susen}\input{../tex-inputs/latex-pdf-abspann}
      