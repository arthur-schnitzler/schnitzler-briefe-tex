%% latex-leseansicht-vorspann.tex
%% Vorspann für die Leseansicht.
%% Lädt die gemeinsame Datei latex-vorspann.tex mit nicht gesetztem Schalter.

\newif\ifkorrekturansicht
\korrekturansichtfalse

\input{../tex-inputs/latex-vorspann}


\section[Arthur Schnitzler an Richard Beer-Hofmann, 20. 5. 1897]{L00678 Arthur Schnitzler an Richard Beer-Hofmann, 20. 5. 1897}
\nopagebreak\mylabel{L00678v}
\rehead{ }\normalsize\beginnumbering\briefempfaengerindex{Beer-Hofmann, Richard@\textsc{Beer-Hofmann, Richard}!zzzSchnitzler, Arthur@\emph{von Arthur Schnitzler}!1897-05-201@{20. 5. 1897}|(be}
\toendnotes[C]{\smallbreak\pagebreak[2]}
\correspDesc{Versand  durch Arthur Schnitzler am 20. 5. 1897 in Paris
\newline{}Erhalt  durch Richard Beer-Hofmann am 22. 5. 1897 in Wien}\toendnotes[C]{\smallbreak}
\Standort{YCGL, MSS 31.}
\physDesc{Brief, 2 Blätter, 8 Seiten, Kuvert, 2717 Zeichen
\newline{}Handschrift: schwarze Tinte, deutsche Kurrent
\newline{}Versand: 1) Stempel: »\nobreak{}\oindex{rue Milton@\textbf{rue Milton}, \emph{Straße}|pwk}Paris 2 B. Milton, 20 Mai 97, 7\textsuperscript{E}\nobreak{}«.   2) Stempel: »\nobreak{}\oindex{I., Innere Stadt@\textbf{I., Innere Stadt}, \emph{Verwaltungsgebiet}|pwk}Wien 1/1, 22 5. 97, 9–10½V., Bestellt\nobreak{}«. }
\buchAbdrucke{\weitereDrucke{1) Arthur Schnitzler: \emph{Briefe 1875–1912}. Herausgegeben von Therese Nickl und Heinrich Schnitzler. Frankfurt am Main: \emph{S. Fischer} 1981, S. 322–323.} \weitereDrucke{2) Arthur Schnitzler, Richard Beer-Hofmann: \emph{Briefwechsel 1891–1931}. Herausgegeben von Konstanze Fliedl. Wien, Zürich: \emph{Europaverlag} 1992, S. 104–105.} \weitereDrucke{3) Hermann Bahr, Arthur Schnitzler: \emph{Briefwechsel, Aufzeichnungen, Dokumente (1891–1931)}. Herausgegeben von Kurt Ifkovits und Martin Anton Müller. Göttingen: \emph{Wallstein} 2018.} }\toendnotes[C]{\smallbreak}\pstart{}\textsc{{\pb}Mr Dr Richard Beer-Hofmann}\pend{}\pstart{}\textsc{Wien\oindex{Wien@\textbf{Wien}, \emph{Verwaltungsgebiet}|pw}}\pend{}\pstart{}\textsc{I. Wollzeile 15\oindex{Wien@\textbf{Wien}!I., Innere Stadt@\textbf{I., Innere Stadt}!Wollzeile 15 (»Berthahof«)@\textbf{Wollzeile 15 (»Berthahof«)}, \emph{Wohngebäude}|pw}}\pend{}\pstart{}\textsc{Autriche\oindex{Österreich@\textbf{Österreich}|pw}}\pend{}{\bigskip}\vspace{1em}
\pstart
           \raggedleft{}{\pb}20. 5. 97{\\}\textsc{Paris}\oindex{Paris@\textbf{Paris}, \emph{Hauptstadt}|pw}.\pend
           \vspace{0.5em}
\pstart
           Lieber Richard, die Pariſer\oindex{Paris@\textbf{Paris}, \emph{Hauptstadt}|pw} Tage
               –{ }ſie werden wahrſcheinlich bald »ſehr{ }ſchön geweſen«{ }ſein – nahen ihrem Ende; Montag
               fahre ich nach London\oindex{London@\textbf{London}, \emph{Hauptstadt}|pw} und bin in den ersten
               Junitagen in Wien\oindex{Wien@\textbf{Wien}, \emph{Verwaltungsgebiet}|pw}. Sie aber fahren bereits in den{ }ſelben erſten Junitagen nach Iſchl\oindex{Bad Ischl@\textbf{Bad Ischl}|pw}?\pend
           
\pstart
           Ich werde Sie doch hoffentlich noch in Wien\oindex{Wien@\textbf{Wien}, \emph{Verwaltungsgebiet}|pw}
               finden? Beruhigen {\pb}Sie mich darüber, indem Sie mir
               eine Zeile nach London\oindex{London@\textbf{London}, \emph{Hauptstadt}|pw}{ }ſchreiben. Meine Adreſſe ist sehr complicirt: bei
                  \textsc{Felix Markbreiter\pwindex{Markbreiter, Felix 20.\,11.\,1855 Wien – 15.\,9.\,1914 London@\textsc{Markbreiter, Felix} (20.\,11.\,1855 Wien – 15.\,9.\,1914 London), \emph{Kaufmann}|pw}{ }London S E. Honor Oak, Woodville Hall\oindex{Honor Oak@\textbf{Honor Oak}, \emph{Teil eines besiedelten Ortes}|pw}}. –\pend
           
\pstart
           Paul\pwindex{Goldmann, Paul 31.\,1.\,1865 Breslau – 25.\,9.\,1935 Wien@\textsc{Goldmann, Paul} (31.\,1.\,1865 Breslau – 25.\,9.\,1935 Wien), \emph{Schriftsteller, Journalist}|pw} behauptet,{ }ſo oft ich irgend ein
               Entzücken oder eine Befriedigung über irgend was hier äußere – und es wi{\geminationm}elt von{ }ſolchen Gelegenheiten, dſs Sie einmal ge{\pb}äußert, Paris\oindex{Paris@\textbf{Paris}, \emph{Hauptstadt}|pw}
               hätte Ihnen nichts zu{ }ſagen. Sie werden das einmal beſchämt zurücknehmen. Sie ahnen
               nicht, was Ihnen Paris\oindex{Paris@\textbf{Paris}, \emph{Hauptstadt}|pw} alles zu{ }ſagen hätte und
               wie viel Sie gerne antworten möchten. Dieſe Stadt dampft von Cultur, und ich hab mich
               kaum über einen Menſchen ärgern kö{\geminationn}en, der mir zufällig
               heute grad{ }ſagte, er{ }ſei in Wien\oindex{Wien@\textbf{Wien}, \emph{Verwaltungsgebiet}|pw} geweſen, {\pb}denke gern dran zurück: \textsc{c’est une
                  gentille petite ville}. Man{ }ſpürt auch etwas wahres in dieſer Phraſe: dſs
               eigentlich die ganze Welt in Paris\oindex{Paris@\textbf{Paris}, \emph{Hauptstadt}|pw} enthalten{ }ſei;
               man hat eine Ahnung von Unendlichkeit, in der man beinah so einſam{ }ſein könnte wie in
               der Wüſte. Wiſſen Sie, was mir eine große Freude{ }ſein würde? einmal mit Ihnen hieher
               zu kommen – nicht {\pb}ohne Ihnen das Verſprechen abgeno{\geminationm}en zu haben, nicht bei jeder Auslage stehn zu bleiben.
               Ich würde Sie aber nie an die Seine führen, wo an den Quais auf den Steinbrüſtungen
               Millionen Bücher liegen – Sie würden dazu allein zwanzig Jahre brauchen. Dort findet
               man, wie Sie gleich{ }ſehen werden, alle Bücher der Welt; {\pb}um mir eine Emotion zu verſchaffen, hab ich mit einer
               Verkäuferin um ein Exemplar von »\textsc{Mourir}\pwindex{Schnitzler, Arthur 15.\,5.\,1862 Wien – 21.\,10.\,1931 ebd.@\textsc{Schnitzler, Arthur} (15.\,5.\,1862 Wien – 21.\,10.\,1931 ebd.), \emph{Schriftsteller, Mediziner}!Sterben. Novelle@\strich\emph{Sterben. Novelle}|pw}« »gefeilſcht« – das Luder hat’s mir für 60 \textsc{centimes}
               gelaſſen – unaufgeſchnitten! (das Buch mein ich.)\pend
           
\pstart
           – Mit Ihr\pwindex{Reinhard, Marie 13.\,3.\,1871 Wien – 18.\,3.\,1899 ebd.@\textsc{Reinhard, Marie} (13.\,3.\,1871 Wien – 18.\,3.\,1899 ebd.), \emph{Gesangspädagogin}|pwv} bin ich{ }ſehr
               zufrieden;{ }ſanft, lieb, ein bischen rührend. Ich hab{ }ſie wahrſcheinlich viel lieber,
               als wenn ich{ }ſie lieb hätte. – Wir {\dots} na, wir reden ja in
                  Wien\oindex{Wien@\textbf{Wien}, \emph{Verwaltungsgebiet}|pw} darüber. –\pend
           
\pstart
           {\pb}Der \label{K_L00678-1v}\edtext{Graf\pwindex{Graf, Max 1.\,10.\,1873 Wien – 24.\,6.\,1958 ebd.@\textsc{Graf, Max} (1.\,10.\,1873 Wien – 24.\,6.\,1958 ebd.), \emph{Kritiker}|pw}}{\lemma{\textnormal{\emph{Graf}}}\Cendnote{\textnormal{Max Graf\pwindex{Graf, Max 1.\,10.\,1873 Wien – 24.\,6.\,1958 ebd.@\textsc{Graf, Max} (1.\,10.\,1873 Wien – 24.\,6.\,1958 ebd.), \emph{Kritiker}|pwk}}}}\label{K_L00678-1}, dem Sie die Empfehlung an \strikeout{Richard}{ }Paul\pwindex{Goldmann, Paul 31.\,1.\,1865 Breslau – 25.\,9.\,1935 Wien@\textsc{Goldmann, Paul} (31.\,1.\,1865 Breslau – 25.\,9.\,1935 Wien), \emph{Schriftsteller, Journalist}|pw} mitgegeben, iſt, losgelöſt von den
               Leuten, unter denen er noch einer der anſtändigſten iſt, ein ganz widerliches
               Subjekt; verlogen und verlottert. Moralſchule Altenberg\pwindex{Altenberg, Peter 9.\,3.\,1859 Wien – 8.\,1.\,1919 ebd.@\textsc{Altenberg, Peter} (9.\,3.\,1859 Wien – 8.\,1.\,1919 ebd.), \emph{Schriftsteller}|pw}, Beobachtungsſchule Bahr\pwindex{Bahr, Hermann 19.\,7.\,1863 Linz – 15.\,1.\,1934 München@\textsc{Bahr, Hermann} (19.\,7.\,1863 Linz – 15.\,1.\,1934 München), \emph{Schriftsteller, Kritiker}|pw}.\pend
           
\pstart
           \uline{Sie\pwindex{Reinhard, Marie 13.\,3.\,1871 Wien – 18.\,3.\,1899 ebd.@\textsc{Reinhard, Marie} (13.\,3.\,1871 Wien – 18.\,3.\,1899 ebd.), \emph{Gesangspädagogin}|pwv}}{ }ſitzt, während ich Ihnen{ }ſchreibe, im Nebenzimmer
               und lieſt eben die Scene\pwindex{Schnitzler, Arthur 15.\,5.\,1862 Wien – 21.\,10.\,1931 ebd.@\textsc{Schnitzler, Arthur} (15.\,5.\,1862 Wien – 21.\,10.\,1931 ebd.), \emph{Schriftsteller, Mediziner}!Reigen. Zehn Dialoge@\strich\emph{Reigen. Zehn Dialoge}|pwv}
               zwiſchen dem {\pb}Dichter (Biebitz\pwindex{Schnitzler, Arthur 15.\,5.\,1862 Wien – 21.\,10.\,1931 ebd.@\textsc{Schnitzler, Arthur} (15.\,5.\,1862 Wien – 21.\,10.\,1931 ebd.), \emph{Schriftsteller, Mediziner}!Reigen. Zehn Dialoge@\strich\emph{Reigen. Zehn Dialoge}|pwv}) und der Schauſpielerin, die ich übrigens geändert
               habe,{ }ſo dſs man{ }ſagen kann: Biebitz\pwindex{Schnitzler, Arthur 15.\,5.\,1862 Wien – 21.\,10.\,1931 ebd.@\textsc{Schnitzler, Arthur} (15.\,5.\,1862 Wien – 21.\,10.\,1931 ebd.), \emph{Schriftsteller, Mediziner}!Reigen. Zehn Dialoge@\strich\emph{Reigen. Zehn Dialoge}|pwv} bleibt Biebitz\pwindex{Schnitzler, Arthur 15.\,5.\,1862 Wien – 21.\,10.\,1931 ebd.@\textsc{Schnitzler, Arthur} (15.\,5.\,1862 Wien – 21.\,10.\,1931 ebd.), \emph{Schriftsteller, Mediziner}!Reigen. Zehn Dialoge@\strich\emph{Reigen. Zehn Dialoge}|pwv}! – Aber{ }ſonſt haben Sie hoffentlich mehr gearbeitet als ich. Nach
               dieſen zwei Dingen{ }ſehn ich mich unbeſchreiblich: nach dem Schreiben und nach dem \textsc{Bicycle}! – Kö{\geminationn}en Sie’s endlich?
               (Bicycle natürlich. –)\pend
           
\pstart
           Seien Sie herzlich gegrüßt. Ihr{\\[\baselineskip]}\spacefill\mbox{Arthur.}\pend
           \leftskip=0em{}\selectlanguage{ngerman}\endnumbering\briefempfaengerindex{Beer-Hofmann, Richard@\textsc{Beer-Hofmann, Richard}!zzzSchnitzler, Arthur@\emph{von Arthur Schnitzler}!1897-05-201@{20. 5. 1897}|)be}\mylabel{L00678h}  \newcommand{\dateiname}{L00678}\newcommand{\titel}{Arthur Schnitzler an Richard Beer-Hofmann, 20. 5. 1897}\newcommand{\editorInnen}{Herausgegeben von Martin Anton Müller}%% latex-leseansicht-abspann.tex
%% Abspann für die Leseansicht.
%% Der Schalter \ifkorrekturansicht ist bereits durch den Vorspann gesetzt.

%% latex-abspann.tex
%% Gemeinsamer Abspann für Korrekturansicht und Leseansicht.
%% Setzt den Schalter \ifkorrekturansicht voraus (gesetzt in den
%% einbindenden Dateien latex-korrekturansicht-abspann.tex bzw.
%% latex-leseansicht-abspann.tex).
%% ---------------------------------------------------------------

\normalsize

% Das esempio-Environment wird nur in der Leseansicht benötigt
\ifkorrekturansicht\else
\newenvironment{esempio}[3]%
{
    \vspace{1.5ex}
    \rlap{\underline{#1}}
    \par
    \setlength{\parindent}{0cm}
    \nopagebreak
    \leftskip=#2cm
    \rightskip=#3cm
}
{
    \par
}
\fi

\doendnotes{C}
\bigskip
\vfill

\clearpage

\footnotesize

\ifkorrekturansicht
  \lohead{\textsc{register}}
\fi

% theindex-Environment neu definieren ohne reledmac
\makeatletter
\renewenvironment{theindex}{%
  \ifkorrekturansicht
    \section*{\indexname}%
  \else
    \subsubsection*{Index der erwähnten Entitäten}%
  \fi
  \setlength{\parindent}{0pt}%
  \setlength{\parskip}{0pt plus 0.3pt}%
  \let\item\@idxitem
}{%
  \ifkorrekturansicht\clearpage\fi
}
\makeatother

\IfFileExists{\jobname-pw.ind}{\input{\jobname-pw.ind}}{}

% Quellenangabe nur in der Leseansicht
\ifkorrekturansicht\else
% Fallback-Definitionen, falls die .tex-Datei \titel etc. nicht gesetzt hat
\providecommand{\titel}{}
\providecommand{\editorInnen}{}
\providecommand{\dateiname}{\jobname}

\vspace{3cm}

\vfill

\footnotesize
\textsc{Quelle}: \titel. Herausgegeben von {\editorInnen}. In: \emph{Arthur Schnitzler: Briefwechsel mit Autorinnen und Autoren}.
 Digitale Edition, https://schnitzler-briefe.acdh.oeaw.ac.at/{\dateiname}.html (Stand \today)
\fi

\end{document}


