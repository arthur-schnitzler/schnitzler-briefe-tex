%% latex-leseansicht-vorspann.tex
%% Vorspann für die Leseansicht.
%% Lädt die gemeinsame Datei latex-vorspann.tex mit nicht gesetztem Schalter.

\newif\ifkorrekturansicht
\korrekturansichtfalse

\input{../tex-inputs/latex-vorspann}


\section[ Paul Goldmann an Arthur Schnitzler, 2. 9. 1900]{L02929 Paul Goldmann an Arthur Schnitzler,  2. 9. 1900}
\nopagebreak\mylabel{L02929v}
\rehead{ }\normalsize\beginnumbering\briefempfaengerindex{Schnitzler, Arthur@\textsc{Schnitzler, Arthur}!zzzGoldmann, Paul@\emph{von Paul Goldmann}!1900-09-021@{2. 9. 1900}|(be}
\toendnotes[C]{\smallbreak\pagebreak[2]}
\correspDesc{Versand  durch Paul Goldmann am 2. 9. 1900 in Trafoi
\newline{}Erhalt  durch Arthur Schnitzler am 4. 9. 1900 in Wien}\toendnotes[C]{\smallbreak}
\Standort{DLA, A:Schnitzler, HS.NZ85.1.3170.}
\physDesc{Bildpostkarte, 278 Zeichen
\newline{}Handschrift: schwarze Tinte, deutsche Kurrent
\newline{}Versand: 1) Stempel: »\nobreak{}\oindex{Trafoi@\textbf{Trafoi}|pwk}Trafoi, 2. 9. 00\nobreak{}«.   2) Stempel: »\nobreak{}\oindex{IX., Alsergrund@\textbf{IX., Alsergrund}, \emph{Verwaltungsgebiet}|pwk}Wien 9/3 72, 4. 9. 00, 8. V, Bestellt\nobreak{}«. 
\newline{}Schnitzler: mit Bleistift das Jahr »900.« vermerkt }\toendnotes[C]{\smallbreak}\pstart{}\textsc{{\pb}Herrn}\pend{}\pstart{}\textsc{Dr. Arthur Schnitzler}\pend{}\pstart{}\textsc{Wien\oindex{Wien@\textbf{Wien}, \emph{Verwaltungsgebiet}|pw}}\pend{}\pstart{}\textsc{IX. Frankgaſse 1\oindex{Wien@\textbf{Wien}!IX., Alsergrund@\textbf{IX., Alsergrund}!Frankgasse 1@\textbf{Frankgasse 1}, \emph{Wohngebäude}|pw}.}\pend{}{\bigskip}
\pstart
           \noindent{}\centering{}{\pb}\textcolor{gray}{\textbf{Der Weisse Knott\oindex{Zum Weissen Knott@\textbf{Zum Weissen Knott}, \emph{Gastgewerbegebäude}|pw}}}\pend
           \vspace{1em}
\pstart
           {\pb}\textsc{Trafoi\oindex{Trafoi@\textbf{Trafoi}|pw}}, 2. September.\pend
           
\pstart{}Mein lieber Freund,\pend\vspace{0.5em}
\pstart
           Ich danke für Dein Telegramm. Komme vorausſichtlich \label{K_L02929-1v}\edtext{Dienſtag oder Mittwoch
               nach Wien\oindex{Wien@\textbf{Wien}, \emph{Verwaltungsgebiet}|pw}}{\lemma{\textnormal{\emph{Dienstag … Wien}}}\Cendnote{\textnormal{Goldmann\pwindex{Goldmann, Paul 31.\,1.\,1865 Breslau – 25.\,9.\,1935 Wien@\textsc{Goldmann, Paul} (31.\,1.\,1865 Breslau – 25.\,9.\,1935 Wien), \emph{Schriftsteller, Journalist}|pwk} hielt sich jedenfalls vom 6. 9. 1900 bis zum XXXX Auszeichnungsfehler: Dokument L02930 nicht gefunden in Wien\oindex{Wien@\textbf{Wien}, \emph{Verwaltungsgebiet}|pwk} auf.}}}\label{K_L02929-1} und wäre Dir{ }ſehr dankbar, wenn
               Du mir im \textsc{Hotel Hammerand\oindex{Wien@\textbf{Wien}!VIII., Josefstadt@\textbf{VIII., Josefstadt}!Hotel Hammerand@\textbf{Hotel Hammerand}, \emph{Hotel}|pw}} ein Zimmer beſtellen wollteſt. Herzl. Gruß. {\\}\spacefill\mbox{Paul Goldmann.}\pend
           \selectlanguage{ngerman}\endnumbering\briefempfaengerindex{Schnitzler, Arthur@\textsc{Schnitzler, Arthur}!zzzGoldmann, Paul@\emph{von Paul Goldmann}!1900-09-021@{2. 9. 1900}|)be}\mylabel{L02929h}  \newcommand{\dateiname}{L02929}\newcommand{\titel}{Paul Goldmann an Arthur Schnitzler, 2. 9. 1900}\newcommand{\editorInnen}{Martin Anton Müller und Laura Untner}%% latex-leseansicht-abspann.tex
%% Abspann für die Leseansicht.
%% Der Schalter \ifkorrekturansicht ist bereits durch den Vorspann gesetzt.

%% latex-abspann.tex
%% Gemeinsamer Abspann für Korrekturansicht und Leseansicht.
%% Setzt den Schalter \ifkorrekturansicht voraus (gesetzt in den
%% einbindenden Dateien latex-korrekturansicht-abspann.tex bzw.
%% latex-leseansicht-abspann.tex).
%% ---------------------------------------------------------------

\normalsize

% Das esempio-Environment wird nur in der Leseansicht benötigt
\ifkorrekturansicht\else
\newenvironment{esempio}[3]%
{
    \vspace{1.5ex}
    \rlap{\underline{#1}}
    \par
    \setlength{\parindent}{0cm}
    \nopagebreak
    \leftskip=#2cm
    \rightskip=#3cm
}
{
    \par
}
\fi

\doendnotes{C}
\bigskip
\vfill

\clearpage

\footnotesize

\ifkorrekturansicht
  \lohead{\textsc{register}}
\fi

% theindex-Environment neu definieren ohne reledmac
\makeatletter
\renewenvironment{theindex}{%
  \ifkorrekturansicht
    \section*{\indexname}%
  \else
    \subsubsection*{Index der erwähnten Entitäten}%
  \fi
  \setlength{\parindent}{0pt}%
  \setlength{\parskip}{0pt plus 0.3pt}%
  \let\item\@idxitem
}{%
  \ifkorrekturansicht\clearpage\fi
}
\makeatother

\IfFileExists{\jobname-pw.ind}{\input{\jobname-pw.ind}}{}

% Quellenangabe nur in der Leseansicht
\ifkorrekturansicht\else
% Fallback-Definitionen, falls die .tex-Datei \titel etc. nicht gesetzt hat
\providecommand{\titel}{}
\providecommand{\editorInnen}{}
\providecommand{\dateiname}{\jobname}

\vspace{3cm}

\vfill

\footnotesize
\textsc{Quelle}: \titel. Herausgegeben von {\editorInnen}. In: \emph{Arthur Schnitzler: Briefwechsel mit Autorinnen und Autoren}.
 Digitale Edition, https://schnitzler-briefe.acdh.oeaw.ac.at/{\dateiname}.html (Stand \today)
\fi

\end{document}


