%% latex-korrekturansicht-vorspann.tex
%% Vorspann für die Korrekturansicht.
%% Lädt die gemeinsame Datei latex-vorspann.tex mit gesetztem Schalter.

\newif\ifkorrekturansicht
\korrekturansichttrue

\input{../tex-inputs/latex-vorspann}


\section[ Paul Goldmann an Arthur Schnitzler, 2. 9. 1900]{L02929 Paul Goldmann an Arthur Schnitzler, 2. 9. 1900}
\nopagebreak\mylabel{L02929v}
\rehead{ }\normalsize\beginnumbering\briefempfaengerindex{Schnitzler, Arthur@\textsc{Schnitzler, Arthur}!zzzGoldmann, Paul@\emph{von Paul Goldmann}!1900-09-021@{2. 9. 1900}|(be}
\toendnotes[C]{\smallbreak\pagebreak[2]}\Standort{DLA, A:Schnitzler, HS.NZ85.1.3170.}
\physDesc{Bildpostkarte, 278 Zeichen
\newline{}Handschrift: 1) schwarze Tinte, deutsche Kurrent\hspace{1em}2) schwarze Tinte, lateinische Kurrent (\noindent{}Adresse)\hspace{1em}
\newline{}Versand: 1) Stempel: »\nobreak{}\oindex{Trafoi@\textbf{Trafoi}, \emph{P.PPL}|pwk}Trafoi, 2. 9. 00\nobreak{}«.   2) Stempel: »\nobreak{}\oindex{IX., Alsergrund@\textbf{IX., Alsergrund}, \emph{A.ADM3}|pwk}Wien 9/3 72, 4. 9. 00, 8. V, Bestellt\nobreak{}«. 
\newline{}Schnitzler: mit Bleistift das Jahr »900.« vermerkt }\toendnotes[C]{\smallbreak}\pstart{}{\pb}Herrn\pend{}\pstart{}Dr. Arthur Schnitzler\pend{}\pstart{}Wien\oindex{Wien@\textbf{Wien}, \emph{A.ADM2}|pw}\pend{}\pstart{}IX. Frankgaſse 1\oindex{Frankgasse 1@\textbf{Frankgasse 1}, \emph{Wohngebäude (K.WHS)}|pw}.\pend{}{\bigskip}
\pstart
           \noindent{}\centering{}{\pb}\textcolor{gray}{\textbf{Der Weisse Knott\oindex{Zum Weissen Knott@\textbf{Zum Weissen Knott}, \emph{Gastgewerbegebäude (K.GGW)}|pw}}}\pend
           \vspace{1em}
\pstart
           {\pb}\textsc{Trafoi\oindex{Trafoi@\textbf{Trafoi}, \emph{P.PPL}|pw}}, 2. September.\pend
           
\pstart{}Mein lieber Freund,\pend\vspace{0.5em}
\pstart
           Ich danke für Dein Telegramm. Komme vorausſichtlich \label{K_L02929-1v}\edtext{Dienſtag oder Mittwoch
               nach Wien\oindex{Wien@\textbf{Wien}, \emph{A.ADM2}|pw}}{\lemma{\textnormal{\emph{Dienſtag … Wien}}}\Cendnote{\textnormal{Goldmann\pwindex{Goldmann, Paul 31.01.1865 – 25.09.1935@\textsc{Goldmann, Paul} (31.01.1865 – 25.09.1935), \emph{Schriftsteller/Schriftstellerin, Journalist/Journalistin}|pwk} hielt sich jedenfalls vom 6. 9. 1900 bis zum 16. 9. [1900] in Wien\oindex{Wien@\textbf{Wien}, \emph{A.ADM2}|pwk} auf.}}}\label{K_L02929-1} und wäre Dir ſehr dankbar, wenn
               Du mir im \textsc{Hotel Hammerand\oindex{Hotel Hammerand@\textbf{Hotel Hammerand}, \emph{Hotel (K.HTL)}|pw}} ein Zimmer beſtellen wollteſt. Herzl. Gruß. {\\}\spacefill\mbox{Paul Goldmann.}\pend
           \selectlanguage{ngerman}\endnumbering\briefempfaengerindex{Schnitzler, Arthur@\textsc{Schnitzler, Arthur}!zzzGoldmann, Paul@\emph{von Paul Goldmann}!1900-09-021@{2. 9. 1900}|)be}\mylabel{L02929h}  \normalsize

\doendnotes{C}
\bigskip
\vfill

\clearpage

\footnotesize

\lohead{\textsc{register}}

% Definiere theindex-Environment komplett neu ohne reledmac
\makeatletter
\renewenvironment{theindex}{%
  \section*{\indexname}%
  \setlength{\parindent}{0pt}%
  \setlength{\parskip}{0pt plus 0.3pt}%
  \let\item\@idxitem
}{%
  \clearpage
}
\makeatother

\IfFileExists{\jobname-pw.ind}{\input{\jobname-pw.ind}}{}

\end{document}

      