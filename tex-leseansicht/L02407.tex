%% latex-korrekturansicht-vorspann.tex
%% Vorspann für die Korrekturansicht.
%% Lädt die gemeinsame Datei latex-vorspann.tex mit gesetztem Schalter.

\newif\ifkorrekturansicht
\korrekturansichttrue

\input{../tex-inputs/latex-vorspann}


\section[Arthur Schnitzler an Georg Brandes, 4. 1. 1924]{L02407 Arthur Schnitzler an Georg Brandes, 4. 1. 1924}
\nopagebreak\mylabel{L02407v}
\rehead{ }\normalsize\beginnumbering\briefempfaengerindex{Brandes, Georg@\textsc{Brandes, Georg}!zzzSchnitzler, Arthur@\emph{von Arthur Schnitzler}!1924-01-041@{4. 1. 1924}|(be}
\toendnotes[C]{\smallbreak\pagebreak[2]}\Standort{Kopenhagen, Det Kongelige Bibliotek, Georg Brandes Arkiv, box 125.}
\physDesc{Postkarte, 794 Zeichen
\newline{}Handschrift: schwarze Tinte, lateinische Kurrent
\newline{}Versand: Stempel: »\nobreak{}\oindex{XVIII., Waehring@\textbf{XVIII., Währing}, \emph{A.ADM3}|pwk}18/\textcolor{gray}{×}{ }{[}Wien{]}, \textcolor{gray}{5}. 1. 2\textcolor{gray}{4}, 16\nobreak{}«.  
\newline{}Ordnung: mit Bleistift von unbekannter Hand nummeriert:
                                    »Schnitzler 47.« }
\buchAbdrucke{\weitereDrucke{Georg Brandes, Arthur Schnitzler: \emph{Ein Briefwechsel}. Bern: \emph{Francke} 1956, S. 139.} }\toendnotes[C]{\smallbreak}\pstart{}{\pb}\label{T_L02407-1v}\edtext{\textcolor{gray}{\textbf{A. S.}}}{\lemma{\textnormal{\emph{A. S.}}}\Cendnote{\textnormal{ovaler Absenderkleber}}}\label{T_L02407-1}\pend{}\pstart{}\textcolor{gray}{\textbf{WIEN, XVIII.}}\oindex{XVIII., Waehring@\textbf{XVIII., Währing}, \emph{A.ADM3}|pw}\pend{}\pstart{}\textcolor{gray}{\textbf{STERNWARTESTR. 71}}\oindex{Sternwartestrasse 71@\textbf{Sternwartestraße 71}, \emph{Wohngebäude (K.WHS)}|pw}\pend{}{\bigskip}\pstart{}An\pend{}\pstart{}Georg Brandes\pend{}\pstart{}Kopenhagen\oindex{Kopenhagen@\textbf{Kopenhagen}, \emph{P.PPLC}|pw}. \pend{}{\bigskip}\vspace{1em}
\pstart
           \raggedleft{}{\pb}Wien\oindex{Wien@\textbf{Wien}, \emph{A.ADM2}|pw}{ }4. 1. 24.\pend
           \vspace{0.5em}
\pstart
           Mein verehrter und lieber Freund, nach dem wunderbaren Voltaire\pwindex{Voltaire und sein Jahrhundert@\emph{Voltaire und sein Jahrhundert}|pw} ist nun, zu Weihnachten, Ihr Michel Angelo\pwindex{Michelangelo Buonarotti@\emph{Michelangelo Buonarotti}|pw} bei mir eingetroffen und ich ka{\geminationn} nur mit stolzer Freude für das neue prächtige Geschenk
               danken. Ich will heute nur meine herzlichen Neujahrsgrüße hinzufügen und Sie bitten,
               mir gelegentlich wieder ein Wort über Ihr Befinden zu schreiben. Daß Sie vor nicht
               langer Zeit in Paris\oindex{Paris@\textbf{Paris}, \emph{P.PPLC}|pw} waren, hab ich gelesen und
               gehört; – ich habe mich in diesem Winter bisher daheim gehalten, u. {\pb}führe ein ziemlich zurückgezogenes Leben, sehe
               aber dabei nicht wenig Menschen, viele aus dem Ausland, meistens Amerikaner\oindex{Amerika@\textbf{Amerika}, \emph{kein passender Code gefunden}|pw}. Und nächstens werd ich Ihnen wohl wieder ein neues
               Stück zusenden können. –\pend
           
\pstart
           Seien Sie tausendmal gegrüßt in alter Bewunderung und Liebe von Ihrem{\\[\baselineskip]}\spacefill\mbox{Arthur Schnitzler}\pend
           \leftskip=0em{}\selectlanguage{ngerman}\endnumbering\briefempfaengerindex{Brandes, Georg@\textsc{Brandes, Georg}!zzzSchnitzler, Arthur@\emph{von Arthur Schnitzler}!1924-01-041@{4. 1. 1924}|)be}\mylabel{L02407h}  \normalsize

\doendnotes{C}
\bigskip
\vfill

\clearpage

\footnotesize

\lohead{\textsc{register}}

% Definiere theindex-Environment komplett neu ohne reledmac
\makeatletter
\renewenvironment{theindex}{%
  \section*{\indexname}%
  \setlength{\parindent}{0pt}%
  \setlength{\parskip}{0pt plus 0.3pt}%
  \let\item\@idxitem
}{%
  \clearpage
}
\makeatother

\IfFileExists{\jobname-pw.ind}{\input{\jobname-pw.ind}}{}

\end{document}

      