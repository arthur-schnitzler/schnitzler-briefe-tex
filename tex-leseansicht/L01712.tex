%% latex-leseansicht-vorspann.tex
%% Vorspann für die Leseansicht.
%% Lädt die gemeinsame Datei latex-vorspann.tex mit nicht gesetztem Schalter.

\newif\ifkorrekturansicht
\korrekturansichtfalse

\input{../tex-inputs/latex-vorspann}


\section[Arthur Schnitzler an Stefan Großmann, 28. 9. 1907]{L01712 Arthur Schnitzler an Stefan Großmann, 28. 9. 1907}
\nopagebreak\mylabel{L01712v}
\rehead{ }\normalsize\beginnumbering\briefempfaengerindex{Großmann, Stefan@\textsc{Großmann, Stefan}!zzzSchnitzler, Arthur@\emph{von Arthur Schnitzler}!1907-09-281@{28. 9. 1907}|(be}
\toendnotes[C]{\smallbreak\pagebreak[2]}
\correspDesc{Versand  durch Arthur Schnitzler am 28. 9. 1907 in Wien
\newline{}Erhalt  durch Stefan Großmann im Zeitraum [28. 9. 1907
                  – 2. 10. 1907?] in Wien}\toendnotes[C]{\smallbreak}
\Standort{DLA, A:Schnitzler, HS.NZ85.1.896.}
\physDesc{Brief, Durchschlag, 1 Blatt, 1 Seite, 704 Zeichen
\newline{}Schreibmaschine
\newline{}Handschrift: roter Buntstift, deutsche Kurrent (\noindent{}Korrektur eines Satzzeichens, eine Unterstreichung)}\toendnotes[C]{\smallbreak}
\pstart
           \raggedleft{}{\pb}28. Sept. 07.\pend
           
\pstart{}Sehr geehrter Herr Grossmann,\pend\vspace{0.5em}
\pstart
           Ihre freundliche Einladun\textcolor{gray}{g} an einem Abend vor Mitgliedern der freien Volksbühne\orgindex{Wiener Freie Volksbühne@Wiener Freie Volksbühne|pw} zu lesen nehme ich gern an. Nur
               bitte ich Sie einen kleinen Saal zu wählen, von einem Fassungsraum für höchstens
               fünf- bis sechshundert Personen, da meine Stimme in ei\textcolor{gray}{nem} grössern
               Saale nicht weit genug trägt. Auch glaub ich nicht, dass ich mit meinen Stimmmitteln
               einen Abend allein bestreiten kann, wenigstens einen, der länger \label{T_L01712-1v}\edtext{währte}{\lemma{\textnormal{\emph{währte}}}\Cendnote{\textnormal{geschrieben: »wehrte«}}}\label{T_L01712-1}, als eine Stunde.
               Vielleicht arrangieren Sie es so, dass noch ein zweiter Autor am gleichen Abend liest\substVorne{}\textsuperscript{.}\substDazwischen{}?\substHinten{} Wollen Sie mir nicht auch einen Vorschlag hinsichtlich des Programms
               machen?\pend
           
\pstart
           Mit vorzüglicher Hochachtung{\\[\baselineskip]}Ihr ergebener\pend
           \leftskip=0em{}{\vspace{1\baselineskip}}
\pstart
           Herrn Stefan Grossmann, Wien\oindex{Wien@\textbf{Wien}, \emph{Verwaltungsgebiet}|pw}\pend
           \selectlanguage{ngerman}\endnumbering\briefempfaengerindex{Großmann, Stefan@\textsc{Großmann, Stefan}!zzzSchnitzler, Arthur@\emph{von Arthur Schnitzler}!1907-09-281@{28. 9. 1907}|)be}\mylabel{L01712h}  \newcommand{\dateiname}{L01712}\newcommand{\titel}{Arthur Schnitzler an Stefan Großmann, 28. 9. 1907}\newcommand{\editorInnen}{Martin Anton Müller und Gerd-Hermann Susen}%% latex-leseansicht-abspann.tex
%% Abspann für die Leseansicht.
%% Der Schalter \ifkorrekturansicht ist bereits durch den Vorspann gesetzt.

%% latex-abspann.tex
%% Gemeinsamer Abspann für Korrekturansicht und Leseansicht.
%% Setzt den Schalter \ifkorrekturansicht voraus (gesetzt in den
%% einbindenden Dateien latex-korrekturansicht-abspann.tex bzw.
%% latex-leseansicht-abspann.tex).
%% ---------------------------------------------------------------

\normalsize

% Das esempio-Environment wird nur in der Leseansicht benötigt
\ifkorrekturansicht\else
\newenvironment{esempio}[3]%
{
    \vspace{1.5ex}
    \rlap{\underline{#1}}
    \par
    \setlength{\parindent}{0cm}
    \nopagebreak
    \leftskip=#2cm
    \rightskip=#3cm
}
{
    \par
}
\fi

\doendnotes{C}
\bigskip
\vfill

\clearpage

\footnotesize

\ifkorrekturansicht
  \lohead{\textsc{register}}
\fi

% theindex-Environment neu definieren ohne reledmac
\makeatletter
\renewenvironment{theindex}{%
  \ifkorrekturansicht
    \section*{\indexname}%
  \else
    \subsubsection*{Index der erwähnten Entitäten}%
  \fi
  \setlength{\parindent}{0pt}%
  \setlength{\parskip}{0pt plus 0.3pt}%
  \let\item\@idxitem
}{%
  \ifkorrekturansicht\clearpage\fi
}
\makeatother

\IfFileExists{\jobname-pw.ind}{\input{\jobname-pw.ind}}{}

% Quellenangabe nur in der Leseansicht
\ifkorrekturansicht\else
% Fallback-Definitionen, falls die .tex-Datei \titel etc. nicht gesetzt hat
\providecommand{\titel}{}
\providecommand{\editorInnen}{}
\providecommand{\dateiname}{\jobname}

\vspace{3cm}

\vfill

\footnotesize
\textsc{Quelle}: \titel. Herausgegeben von {\editorInnen}. In: \emph{Arthur Schnitzler: Briefwechsel mit Autorinnen und Autoren}.
 Digitale Edition, https://schnitzler-briefe.acdh.oeaw.ac.at/{\dateiname}.html (Stand \today)
\fi

\end{document}


