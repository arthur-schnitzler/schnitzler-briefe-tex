%% latex-leseansicht-vorspann.tex
%% Vorspann für die Leseansicht.
%% Lädt die gemeinsame Datei latex-vorspann.tex mit nicht gesetztem Schalter.

\newif\ifkorrekturansicht
\korrekturansichtfalse

\input{../tex-inputs/latex-vorspann}


\section[Felix Salten an Arthur Schnitzler, 23. 8. 1892]{L03113 Felix Salten an Arthur Schnitzler, 23. 8. 1892}
\nopagebreak\mylabel{L03113v}
\rehead{ }\normalsize\beginnumbering\briefempfaengerindex{Schnitzler, Arthur@\textsc{Schnitzler, Arthur}!zzzSalten, Felix@\emph{von Felix Salten}!1892-08-232@{23. 8. 1892}|(be}
\toendnotes[C]{\smallbreak\pagebreak[2]}
\correspDesc{Versand  durch Felix Salten am 23. 8. 1892 in Unterach am Attersee
\newline{}Erhalt  durch Arthur Schnitzler im Zeitraum [24. 8. 1892
                  – 28. 8. 1892?] in Wien}\toendnotes[C]{\smallbreak}
\Standort{CUL, Schnitzler, B 89, A 1.}
\physDesc{Brief, 1 Blatt, 4 Seiten, 1384 Zeichen
\newline{}Handschrift: schwarze Tinte, lateinische Kurrent
\newline{}Ordnung: mit Bleistift von unbekannter Hand nummeriert: »17« }\toendnotes[C]{\smallbreak}
\pstart
           \raggedleft{}{\pb}Unterach\oindex{Bahnhof@\textbf{Bahnhof}, \emph{Bahnhofsgebäude}|pw}{ }23. August 1892\pend
           \vspace{0.5em}
\pstart
           Verehrter Freund! Dass die \label{K_L03113-1v}\edtext{Lösung nicht von mir
               ausging}{\lemma{\textnormal{\emph{Lösung … ausging}}}\Cendnote{\textnormal{Vgl. A. S.: \emph{Tagebuch}, 19. 8. 1892: »Von S.\pwindex{Salten, Felix 6.\,9.\,1869 Budapest – 8.\,10.\,1945 Zürich@\textsc{Salten, Felix} (6.\,9.\,1869 Budapest – 8.\,10.\,1945 Zürich), \emph{Schriftsteller, Journalist, Chefredakteur}|pw} zerknirschter Brief, allerdings erst auf dringende
                        Aufforderung.«}}}\label{K_L03113-1} liegt nur daran, dass \uline{Sie} mir zuvorgekommen
               sind. Seien Sie überzeugt, dass ich entsetzlich unter diesen \label{K_L03113-2v}\edtext{Erbärmlichkeiten}{\lemma{\textnormal{\emph{Erbärmlichkeiten}}}\Cendnote{\textnormal{Siehe XXXX Auszeichnungsfehler: Dokument L03186 nicht gefunden.
               }}}\label{K_L03113-2} gelitten habe u. noch unsagbar leide, u. dass ich sofort mit der Wahrheit vor
               Sie hin getreten wäre, im Augenblicke in dem ich alles wieder hätte gut gemacht. Dass
               es überhaupt möglich war, läßt sich allerdings nicht aus der Welt schaffen, u. wenn
               auch Sie möglicher{\pb}weise darüber hinwegkommen, ich werde
               es kaum imstande sein. Ich bin vollständig niedergebrochen u. habe auch
               den Rest von Elasticität verloren, den ich noch hatte, und wie mein äußeres
               Leben unter dem Zeichen dieser fruchtlosen
               Reue u. Selbstpeinigung steht, so kann ich Ihnen auch von dem inneren
               künstlerischen nichts berichten. Es kann ja doch jetzt von irgend einer 
               Arbeit nicht die Rede bei mir sein.\pend
           
\pstart
           Ich werde hier von Liebe und {\pb}Güte erdrückt u. habe doch
               Beides nie so wenig ertragen, als gerade jetzt, ich wäre auch längst
               fort von hier, wo ich den Leuten durch meine consequente Verstimmung
               auffalle, wenn nicht der Gedanke an Wien\oindex{Wien@\textbf{Wien}, \emph{Verwaltungsgebiet}|pw} noch so
               schrecklich für mich wäre.\pend
           
\pstart
           Auch zu Richard\pwindex{Beer-Hofmann, Richard 11.\,7.\,1866 Wien – 26.\,9.\,1945 New York City@\textsc{Beer-Hofmann, Richard} (11.\,7.\,1866 Wien – 26.\,9.\,1945 New York City), \emph{Schriftsteller}|pw} u. Loris\pwindex{Hofmannsthal, Hugo von 1.\,2.\,1874 Wien – 15.\,7.\,1929 Rodaun@\textsc{Hofmannsthal, Hugo von} (1.\,2.\,1874 Wien – 15.\,7.\,1929 Rodaun), \emph{Schriftsteller}|pw} wäre ich
               längst gefahren, aber wie soll ich jetzt mit ihnen reden? Übrigens wäre ja wahrscheinlich
               Alles ebenso gewesen, wenn die Lösung auch nicht erfolgt wäre.\pend
           
\pstart
           Ich sage Ihnen herzlichen Dank für Ihren Brief, u. {\pb}bin
               immer\pend
           
\pstart
           Ihr {\\[\baselineskip]}\spacefill\mbox{FSalten}\pend
           \leftskip=0em{}\selectlanguage{ngerman}\endnumbering\briefempfaengerindex{Schnitzler, Arthur@\textsc{Schnitzler, Arthur}!zzzSalten, Felix@\emph{von Felix Salten}!1892-08-232@{23. 8. 1892}|)be}\mylabel{L03113h}  \newcommand{\dateiname}{L03113}\newcommand{\titel}{Felix Salten an Arthur Schnitzler, 23. 8. 1892}\newcommand{\editorInnen}{Martin Anton Müller und Laura Untner}%% latex-leseansicht-abspann.tex
%% Abspann für die Leseansicht.
%% Der Schalter \ifkorrekturansicht ist bereits durch den Vorspann gesetzt.

%% latex-abspann.tex
%% Gemeinsamer Abspann für Korrekturansicht und Leseansicht.
%% Setzt den Schalter \ifkorrekturansicht voraus (gesetzt in den
%% einbindenden Dateien latex-korrekturansicht-abspann.tex bzw.
%% latex-leseansicht-abspann.tex).
%% ---------------------------------------------------------------

\normalsize

% Das esempio-Environment wird nur in der Leseansicht benötigt
\ifkorrekturansicht\else
\newenvironment{esempio}[3]%
{
    \vspace{1.5ex}
    \rlap{\underline{#1}}
    \par
    \setlength{\parindent}{0cm}
    \nopagebreak
    \leftskip=#2cm
    \rightskip=#3cm
}
{
    \par
}
\fi

\doendnotes{C}
\bigskip
\vfill

\clearpage

\footnotesize

\ifkorrekturansicht
  \lohead{\textsc{register}}
\fi

% theindex-Environment neu definieren ohne reledmac
\makeatletter
\renewenvironment{theindex}{%
  \ifkorrekturansicht
    \section*{\indexname}%
  \else
    \subsubsection*{Index der erwähnten Entitäten}%
  \fi
  \setlength{\parindent}{0pt}%
  \setlength{\parskip}{0pt plus 0.3pt}%
  \let\item\@idxitem
}{%
  \ifkorrekturansicht\clearpage\fi
}
\makeatother

\IfFileExists{\jobname-pw.ind}{\input{\jobname-pw.ind}}{}

% Quellenangabe nur in der Leseansicht
\ifkorrekturansicht\else
% Fallback-Definitionen, falls die .tex-Datei \titel etc. nicht gesetzt hat
\providecommand{\titel}{}
\providecommand{\editorInnen}{}
\providecommand{\dateiname}{\jobname}

\vspace{3cm}

\vfill

\footnotesize
\textsc{Quelle}: \titel. Herausgegeben von {\editorInnen}. In: \emph{Arthur Schnitzler: Briefwechsel mit Autorinnen und Autoren}.
 Digitale Edition, https://schnitzler-briefe.acdh.oeaw.ac.at/{\dateiname}.html (Stand \today)
\fi

\end{document}


