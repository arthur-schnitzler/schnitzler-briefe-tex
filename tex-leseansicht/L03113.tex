%% latex-korrekturansicht-vorspann.tex
%% Vorspann für die Korrekturansicht.
%% Lädt die gemeinsame Datei latex-vorspann.tex mit gesetztem Schalter.

\newif\ifkorrekturansicht
\korrekturansichttrue

\input{../tex-inputs/latex-vorspann}


\section[Felix Salten an Arthur Schnitzler, 23. 8. 1892]{L03113 Felix Salten an Arthur Schnitzler, 23. 8. 1892}
\nopagebreak\mylabel{L03113v}
\rehead{ }\normalsize\beginnumbering\briefempfaengerindex{Schnitzler, Arthur@\textsc{Schnitzler, Arthur}!zzzSalten, Felix@\emph{von Felix Salten}!1892-08-232@{23. 8. 1892}|(be}
\toendnotes[C]{\smallbreak\pagebreak[2]}\Standort{CUL, Schnitzler, B 89, A 1.}
\physDesc{Brief, 1 Blatt, 4 Seiten, 1384 Zeichen
\newline{}Handschrift: schwarze Tinte, lateinische Kurrent
\newline{}Ordnung: mit Bleistift von unbekannter Hand nummeriert: »17« }\toendnotes[C]{\smallbreak}
\pstart
           \raggedleft{}{\pb}Unterach\oindex{Bahnhof@\textbf{Bahnhof}, \emph{Bahnhofsgebäude (K.BHF)}|pw}{ }23. August 1892\pend
           \vspace{0.5em}
\pstart
           Verehrter Freund! Dass die \label{K_L03113-1v}\edtext{Lösung nicht von mir
               ausging}{\lemma{\textnormal{\emph{Lösung … ausging}}}\Cendnote{\textnormal{Vgl. A. S.: \emph{Tagebuch}, 19. 8. 1892: »Von S.\pwindex{Salten, Felix 06.09.1869 – 08.10.1945@\textsc{Salten, Felix} (06.09.1869 – 08.10.1945), \emph{Schriftsteller/Schriftstellerin, Journalist/Journalistin, Chefredakteur/Chefredakteurin}|pw} zerknirschter Brief, allerdings erst auf dringende
                        Aufforderung.«}}}\label{K_L03113-1} liegt nur daran, dass \uline{Sie} mir zuvorgekommen
               sind. Seien Sie überzeugt, dass ich entsetzlich unter diesen \label{K_L03113-2v}\edtext{Erbärmlichkeiten}{\lemma{\textnormal{\emph{Erbärmlichkeiten}}}\Cendnote{\textnormal{Siehe Felix Salten an Arthur Schnitzler, 10. 8. 1892.
               }}}\label{K_L03113-2} gelitten habe u. noch unsagbar leide, u. dass ich sofort mit der Wahrheit vor
               Sie hin getreten wäre, im Augenblicke in dem ich alles wieder hätte gut gemacht. Dass
               es überhaupt möglich war, läßt sich allerdings nicht aus der Welt schaffen, u. wenn
               auch Sie möglicher{\pb}weise darüber hinwegkommen, ich werde
               es kaum imstande sein. Ich bin vollständig niedergebrochen u. habe auch
               den Rest von Elasticität verloren, den ich noch hatte, und wie mein äußeres
               Leben unter dem Zeichen dieser fruchtlosen
               Reue u. Selbstpeinigung steht, so kann ich Ihnen auch von dem inneren
               künstlerischen nichts berichten. Es kann ja doch jetzt von irgend einer 
               Arbeit nicht die Rede bei mir sein.\pend
           
\pstart
           Ich werde hier von Liebe und {\pb}Güte erdrückt u. habe doch
               Beides nie so wenig ertragen, als gerade jetzt, ich wäre auch längst
               fort von hier, wo ich den Leuten durch meine consequente Verstimmung
               auffalle, wenn nicht der Gedanke an Wien\oindex{Wien@\textbf{Wien}, \emph{A.ADM2}|pw} noch so
               schrecklich für mich wäre.\pend
           
\pstart
           Auch zu Richard\pwindex{Beer-Hofmann, Richard 1866-07-11 – 1945-09-26@\textsc{Beer-Hofmann, Richard} (1866-07-11 – 1945-09-26), \emph{Schriftsteller/Schriftstellerin}|pw} u. Loris\pwindex{Hofmannsthal, Hugo von 1874-02-01 – 1929-07-15@\textsc{Hofmannsthal, Hugo von} (1874-02-01 – 1929-07-15), \emph{Schriftsteller/Schriftstellerin}|pw} wäre ich
               längst gefahren, aber wie soll ich jetzt mit ihnen reden? Übrigens wäre ja wahrscheinlich
               Alles ebenso gewesen, wenn die Lösung auch nicht erfolgt wäre.\pend
           
\pstart
           Ich sage Ihnen herzlichen Dank für Ihren Brief, u. {\pb}bin
               immer\pend
           
\pstart
           Ihr {\\[\baselineskip]}\spacefill\mbox{FSalten}\pend
           \leftskip=0em{}\selectlanguage{ngerman}\endnumbering\briefempfaengerindex{Schnitzler, Arthur@\textsc{Schnitzler, Arthur}!zzzSalten, Felix@\emph{von Felix Salten}!1892-08-232@{23. 8. 1892}|)be}\mylabel{L03113h}  \normalsize

\doendnotes{C}
\bigskip
\vfill

\clearpage

\footnotesize

\lohead{\textsc{register}}

% Definiere theindex-Environment komplett neu ohne reledmac
\makeatletter
\renewenvironment{theindex}{%
  \section*{\indexname}%
  \setlength{\parindent}{0pt}%
  \setlength{\parskip}{0pt plus 0.3pt}%
  \let\item\@idxitem
}{%
  \clearpage
}
\makeatother

\IfFileExists{\jobname-pw.ind}{\input{\jobname-pw.ind}}{}

\end{document}

      