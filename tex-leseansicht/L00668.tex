%% latex-leseansicht-vorspann.tex
%% Vorspann für die Leseansicht.
%% Lädt die gemeinsame Datei latex-vorspann.tex mit nicht gesetztem Schalter.

\newif\ifkorrekturansicht
\korrekturansichtfalse

\input{../tex-inputs/latex-vorspann}


\section[Arthur Schnitzler an Hermann Bahr, 22. 4. 1897]{L00668 Arthur Schnitzler an Hermann Bahr, 22. 4. 1897}
\nopagebreak\mylabel{L00668v}
\rehead{ }\normalsize\beginnumbering\briefempfaengerindex{Bahr, Hermann@\textsc{Bahr, Hermann}!zzzSchnitzler, Arthur@\emph{von Arthur Schnitzler}!1897-04-221@{22. 4. 1897}|(be}
\toendnotes[C]{\smallbreak\pagebreak[2]}
\correspDesc{Versand  durch Arthur Schnitzler am 22. 4. 1897 in Paris
\newline{}Erhalt  durch Hermann Bahr im Zeitraum [23. 4. 1897
                  – 27. 4. 1897?] in Wien}\toendnotes[C]{\smallbreak}
\Standort{TMW, HS AM 23330 Ba.}
\physDesc{Brief, 1 Blatt, 4 Seiten, 883 Zeichen
\newline{}Handschrift: schwarze Tinte, deutsche Kurrent
\newline{}Ordnung: 1) Lochung  2) mit Bleistift von unbekannter Hand datiert:
                                    »22. 4. 97«}
\buchAbdrucke{\weitereDrucke{1) \emph{22. 4. 1897.} In: Arthur Schnitzler: \emph{The Letters of Arthur Schnitzler to Hermann Bahr}. Edited, annotated, and with an introduction, by Donald G. Daviau. Chapel Hill: \emph{The University of North Carolina Press} 1978, S. 61 (University of North Carolina studies in the Germanic languages
                        and literatures, 89).} \weitereDrucke{2) Hermann Bahr, Arthur Schnitzler: \emph{Briefwechsel, Aufzeichnungen, Dokumente (1891–1931)}. Herausgegeben von Kurt Ifkovits und Martin Anton Müller. Göttingen: \emph{Wallstein} 2018, S. 141–142.} }\toendnotes[C]{\smallbreak}
\pstart{}{\pb}Lieber
                  Hermann,\pend\vspace{0.5em}
\pstart
           ich bekomme eben einen \label{K_L00668-44v}\edtext{Brief}{\lemma{\textnormal{\emph{Brief}}}\Cendnote{\textnormal{XXXX Auszeichnungsfehler: Dokument L03694 nicht gefunden.}}}\label{K_L00668-44} von dem dir bekannten Frl. \textsc{Elsa Plessner}\pwindex{Plessner, Elsa 22.\,8.\,1875 Wien – 7.\,5.\,1932 Alicante@\textsc{Plessner, Elsa} (22.\,8.\,1875 Wien – 7.\,5.\,1932 Alicante), \emph{Schriftstellerin}|pw}, die dir eine \label{K_L00668-1v}\edtext{Novelle\pwindex{Plessner, Elsa 22.\,8.\,1875 Wien – 7.\,5.\,1932 Alicante@\textsc{Plessner, Elsa} (22.\,8.\,1875 Wien – 7.\,5.\,1932 Alicante), \emph{Schriftstellerin}!Warten. Novelle@\strich\emph{Warten. Novelle}|pwv}}{\lemma{\textnormal{\emph{Novelle}}}\Cendnote{\textnormal{E. Pleßner\pwindex{Plessner, Elsa 22.\,8.\,1875 Wien – 7.\,5.\,1932 Alicante@\textsc{Plessner, Elsa} (22.\,8.\,1875 Wien – 7.\,5.\,1932 Alicante), \emph{Schriftstellerin}|pwk}: 
                  \emph{Warten}\pwindex{Plessner, Elsa 22.\,8.\,1875 Wien – 7.\,5.\,1932 Alicante@\textsc{Plessner, Elsa} (22.\,8.\,1875 Wien – 7.\,5.\,1932 Alicante), \emph{Schriftstellerin}!Warten. Novelle@\strich\emph{Warten. Novelle}|pwk}. In: \emph{Magazin für Litteratur}\pwindex{Magazin für die Literatur des Auslandes@\emph{Magazin für die Literatur des Auslandes}|pwk}, Jg. 66, Nr. 29,
                        24. 7. 1897, Sp. 867–875.}}}\label{K_L00668-1} eingereicht hat. Ich
               glaube mich zu \label{K_L00668-2v}\edtext{erinnern}{\lemma{\textnormal{\emph{erinnern}}}\Cendnote{\textnormal{Vgl. A. S.: \emph{Tagebuch}, 19. 9. 1896.
               }}}\label{K_L00668-2}, daſs{ }ſie, die Novelle, als ich{ }ſie{ }ſ. Z. im \textsc{Mscrpt}
               las, mir nicht {\pb}misfiel, am Ende{ }ſogar gefiel – ich weiſs nicht mehr genau. Meiner Anſicht nach iſt
               eben benannte Elſa\pwindex{Plessner, Elsa 22.\,8.\,1875 Wien – 7.\,5.\,1932 Alicante@\textsc{Plessner, Elsa} (22.\,8.\,1875 Wien – 7.\,5.\,1932 Alicante), \emph{Schriftstellerin}|pw} von einer unerträglichen
               Schlamperei in Stil und Arbeit; hat aber zuweilen Einfälle, die mit Sicherheit auf
               Talent{ }ſchließen laſſen. Wie weit es geht und ob{ }ſie es nicht eher {\pb}\introOben{}zu\introOben{} ruiniren als weiter zu entwickeln gedenkt, kann ich
               nicht beſti{\geminationm}en. Aber es wäre vielleicht möglich{ }ſie auf
               einen guten Weg zu bringen. – Womit ich dir das Fräulein beſtens empfohlen zu haben
               wünſche. –\pend
           
\pstart
           Ich hoffe es geht dir gut; \damage{von}{ }Pariſer\oindex{Paris@\textbf{Paris}, \emph{Hauptstadt}|pw} Kunſt {\pb}werd ich dir manches
               erzählen können, we{\geminationn} ich \label{K_L00668-3v}\edtext{zurückkomme}{\lemma{\textnormal{\emph{zurückkomme}}}\Cendnote{\textnormal{Schnitzler war am 2. 6. 1897 wieder in Wien\oindex{Wien@\textbf{Wien}, \emph{Verwaltungsgebiet}|pwk}.}}}\label{K_L00668-3}. Aber verlange keine Artikel von mir!\pend
           
\pstart
           Herzlich grüßt dich dein{\\[\baselineskip]}\spacefill\mbox{Arthur Schnitzler}\pend
           \leftskip=0em{}
\pstart
           \textsc{Paris\oindex{Paris@\textbf{Paris}, \emph{Hauptstadt}|pw}}{ }22. 4. 97.\pend
           \selectlanguage{ngerman}\endnumbering\briefempfaengerindex{Bahr, Hermann@\textsc{Bahr, Hermann}!zzzSchnitzler, Arthur@\emph{von Arthur Schnitzler}!1897-04-221@{22. 4. 1897}|)be}\mylabel{L00668h}  \newcommand{\dateiname}{L00668}\newcommand{\titel}{Arthur Schnitzler an Hermann Bahr, 22. 4. 1897}\newcommand{\editorInnen}{Herausgegeben von Martin Anton Müller}%% latex-leseansicht-abspann.tex
%% Abspann für die Leseansicht.
%% Der Schalter \ifkorrekturansicht ist bereits durch den Vorspann gesetzt.

%% latex-abspann.tex
%% Gemeinsamer Abspann für Korrekturansicht und Leseansicht.
%% Setzt den Schalter \ifkorrekturansicht voraus (gesetzt in den
%% einbindenden Dateien latex-korrekturansicht-abspann.tex bzw.
%% latex-leseansicht-abspann.tex).
%% ---------------------------------------------------------------

\normalsize

% Das esempio-Environment wird nur in der Leseansicht benötigt
\ifkorrekturansicht\else
\newenvironment{esempio}[3]%
{
    \vspace{1.5ex}
    \rlap{\underline{#1}}
    \par
    \setlength{\parindent}{0cm}
    \nopagebreak
    \leftskip=#2cm
    \rightskip=#3cm
}
{
    \par
}
\fi

\doendnotes{C}
\bigskip
\vfill

\clearpage

\footnotesize

\ifkorrekturansicht
  \lohead{\textsc{register}}
\fi

% theindex-Environment neu definieren ohne reledmac
\makeatletter
\renewenvironment{theindex}{%
  \ifkorrekturansicht
    \section*{\indexname}%
  \else
    \subsubsection*{Index der erwähnten Entitäten}%
  \fi
  \setlength{\parindent}{0pt}%
  \setlength{\parskip}{0pt plus 0.3pt}%
  \let\item\@idxitem
}{%
  \ifkorrekturansicht\clearpage\fi
}
\makeatother

\IfFileExists{\jobname-pw.ind}{\input{\jobname-pw.ind}}{}

% Quellenangabe nur in der Leseansicht
\ifkorrekturansicht\else
% Fallback-Definitionen, falls die .tex-Datei \titel etc. nicht gesetzt hat
\providecommand{\titel}{}
\providecommand{\editorInnen}{}
\providecommand{\dateiname}{\jobname}

\vspace{3cm}

\vfill

\footnotesize
\textsc{Quelle}: \titel. Herausgegeben von {\editorInnen}. In: \emph{Arthur Schnitzler: Briefwechsel mit Autorinnen und Autoren}.
 Digitale Edition, https://schnitzler-briefe.acdh.oeaw.ac.at/{\dateiname}.html (Stand \today)
\fi

\end{document}


