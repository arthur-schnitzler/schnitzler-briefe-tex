%% latex-korrekturansicht-vorspann.tex
%% Vorspann für die Korrekturansicht.
%% Lädt die gemeinsame Datei latex-vorspann.tex mit gesetztem Schalter.

\newif\ifkorrekturansicht
\korrekturansichttrue

\input{../tex-inputs/latex-vorspann}


\section[Arthur Schnitzler an Hermann Bahr, 22. 4. 1897]{L00668 Arthur Schnitzler an Hermann Bahr, 22. 4. 1897}
\nopagebreak\mylabel{L00668v}
\rehead{ }\normalsize\beginnumbering\briefempfaengerindex{Bahr, Hermann@\textsc{Bahr, Hermann}!zzzSchnitzler, Arthur@\emph{von Arthur Schnitzler}!1897-04-221@{22. 4. 1897}|(be}
\toendnotes[C]{\smallbreak\pagebreak[2]}\Standort{TMW, HS AM 23330 Ba.}
\physDesc{Brief, 1 Blatt, 4 Seiten, 883 Zeichen
\newline{}Handschrift: schwarze Tinte, deutsche Kurrent
\newline{}Ordnung: 1) Lochung  2) mit Bleistift von unbekannter Hand datiert:
                                    »22. 4. 97«}
\buchAbdrucke{\weitereDrucke{1) Arthur Schnitzler: \emph{The Letters of Arthur Schnitzler to Hermann Bahr}. Chapel Hill: \emph{The University of North Carolina Press} 1978, S. 61.} \weitereDrucke{2) Hermann Bahr, Arthur Schnitzler: \emph{Briefwechsel, Aufzeichnungen, Dokumente (1891–1931)}. Göttingen: \emph{Wallstein} 2018, S. 141–142.} }\toendnotes[C]{\smallbreak}
\pstart{}{\pb}Lieber
                  Hermann,\pend\vspace{0.5em}
\pstart
           ich bekomme eben einen \label{K_L00668-44v}\edtext{Brief}{\lemma{\textnormal{\emph{Brief}}}\Cendnote{\textnormal{Elsa Plessner an Arthur Schnitzler, [Mitte April 1897].}}}\label{K_L00668-44} von dem dir bekannten Frl. \textsc{Elsa Plessner}\pwindex{Plessner, Elsa 22.08.1875 – 01.05.1932@\textsc{Plessner, Elsa} (22.08.1875 – 01.05.1932), \emph{Schriftsteller/Schriftstellerin}|pw}, die dir eine \label{K_L00668-1v}\edtext{Novelle\pwindex{Warten@\emph{Warten}|pwv}}{\lemma{\textnormal{\emph{Novelle}}}\Cendnote{\textnormal{E. Pleßner\pwindex{Plessner, Elsa 22.08.1875 – 01.05.1932@\textsc{Plessner, Elsa} (22.08.1875 – 01.05.1932), \emph{Schriftsteller/Schriftstellerin}|pwk}: 
                  \emph{Warten}\pwindex{Warten@\emph{Warten}|pwk}. In: \emph{Magazin für Litteratur}\pwindex{Magazin fuer die Literatur des Auslandes@\emph{Magazin für die Literatur des Auslandes}|pwk}, Jg. 66, Nr. 29,
                        24. 7. 1897, Sp. 867–875.}}}\label{K_L00668-1} eingereicht hat. Ich
               glaube mich zu \label{K_L00668-2v}\edtext{erinnern}{\lemma{\textnormal{\emph{erinnern}}}\Cendnote{\textnormal{Vgl. A. S.: \emph{Tagebuch}, 19. 9. 1896.
               }}}\label{K_L00668-2}, daſs ſie, die Novelle, als ich ſie ſ. Z. im \textsc{Mscrpt}
               las, mir nicht {\pb}misfiel, am Ende ſogar gefiel – ich weiſs nicht mehr genau. Meiner Anſicht nach iſt
               eben benannte Elſa\pwindex{Plessner, Elsa 22.08.1875 – 01.05.1932@\textsc{Plessner, Elsa} (22.08.1875 – 01.05.1932), \emph{Schriftsteller/Schriftstellerin}|pw} von einer unerträglichen
               Schlamperei in Stil und Arbeit; hat aber zuweilen Einfälle, die mit Sicherheit auf
               Talent ſchließen laſſen. Wie weit es geht und ob ſie es nicht eher {\pb}\introOben{}zu\introOben{} ruiniren als weiter zu entwickeln gedenkt, kann ich
               nicht beſti{\geminationm}en. Aber es wäre vielleicht möglich ſie auf
               einen guten Weg zu bringen. – Womit ich dir das Fräulein beſtens empfohlen zu haben
               wünſche. –\pend
           
\pstart
           Ich hoffe es geht dir gut; \damage{von}{ }Pariſer\oindex{Paris@\textbf{Paris}, \emph{P.PPLC}|pw} Kunſt {\pb}werd ich dir manches
               erzählen können, we{\geminationn} ich \label{K_L00668-3v}\edtext{zurückkomme}{\lemma{\textnormal{\emph{zurückkomme}}}\Cendnote{\textnormal{Schnitzler war am 2. 6. 1897 wieder in Wien\oindex{Wien@\textbf{Wien}, \emph{A.ADM2}|pwk}.}}}\label{K_L00668-3}. Aber verlange keine Artikel von mir! \pend
           
\pstart
           Herzlich grüßt dich dein{\\[\baselineskip]}\spacefill\mbox{Arthur Schnitzler}\pend
           \leftskip=0em{}
\pstart
           \textsc{Paris\oindex{Paris@\textbf{Paris}, \emph{P.PPLC}|pw}}{ }22. 4. 97.\pend
           \selectlanguage{ngerman}\endnumbering\briefempfaengerindex{Bahr, Hermann@\textsc{Bahr, Hermann}!zzzSchnitzler, Arthur@\emph{von Arthur Schnitzler}!1897-04-221@{22. 4. 1897}|)be}\mylabel{L00668h}  \normalsize

\doendnotes{C}
\bigskip
\vfill

\clearpage

\footnotesize

\lohead{\textsc{register}}

% Definiere theindex-Environment komplett neu ohne reledmac
\makeatletter
\renewenvironment{theindex}{%
  \section*{\indexname}%
  \setlength{\parindent}{0pt}%
  \setlength{\parskip}{0pt plus 0.3pt}%
  \let\item\@idxitem
}{%
  \clearpage
}
\makeatother

\IfFileExists{\jobname-pw.ind}{\input{\jobname-pw.ind}}{}

\end{document}

      