%% latex-korrekturansicht-vorspann.tex
%% Vorspann für die Korrekturansicht.
%% Lädt die gemeinsame Datei latex-vorspann.tex mit gesetztem Schalter.

\newif\ifkorrekturansicht
\korrekturansichttrue

\input{../tex-inputs/latex-vorspann}


\section[Arthur Schnitzler an Gerhart Hauptmann, 6. 6. 1922]{L02386 Arthur Schnitzler an Gerhart Hauptmann, 6. 6. 1922}
\nopagebreak\mylabel{L02386v}
\rehead{ }\normalsize\beginnumbering\briefempfaengerindex{Hauptmann, Gerhart@\textsc{Hauptmann, Gerhart}!zzzSchnitzler, Arthur@\emph{von Arthur Schnitzler}!1922-06-062@{6. 6. 1922}|(be}
\toendnotes[C]{\smallbreak\pagebreak[2]}\Standort{Staatsbibliothek Berlin – Preußischer Kulturbesitz, GHBrBl A:Schnitzler (14).}
\physDesc{Postkarte, 243 Zeichen
\newline{}Handschrift: schwarze Tinte, deutsche Kurrent
\newline{}Versand: Stempel: »\nobreak{}\oindex{XVIII., Waehring@\textbf{XVIII., Währing}, \emph{A.ADM3}|pwk}1\textcolor{gray}{8}/1 Wien
                                       110, 6. VI. 22, XII\nobreak{}«.  }\toendnotes[C]{\smallbreak}\pstart{}{\pb}Herrn \textsc{Gerhart
                     Hauptmann}\pend{}\pstart{}\textsc{Agnetendorf\oindex{Jagniątków@\textbf{Jagniątków}, \emph{P.PPL}|pw}}\pend{}\pstart{}\textsc{in Schlesien\oindex{Schlesien@\textbf{Schlesien}, \emph{L.RGN}|pw}}\pend{}{\bigskip}\vspace{1em}
\pstart
           \raggedleft{}{\pb}Wien\oindex{Wien@\textbf{Wien}, \emph{A.ADM2}|pw}. 6. 6. 22\pend
           \vspace{0.5em}
\pstart
           mein verehrter und lieber Herr Gerhart Hauptmann, ſeien Sie und Ihre
               verehrte Gattin\pwindex{Hauptmann, Margarete 07.01.1875 – 17.01.1957@\textsc{Hauptmann, Margarete} (07.01.1875 – 17.01.1957)|pwv} für das
               herzliche Glückwunſchtelegramm aufs allerwärmſte bedankt\pend
           \pstart Ihr bewundernd\textcolor{gray}{-}getreuer \spacefill\mbox{Arthur Schnitzler}\pend{}\selectlanguage{ngerman}\endnumbering\briefempfaengerindex{Hauptmann, Gerhart@\textsc{Hauptmann, Gerhart}!zzzSchnitzler, Arthur@\emph{von Arthur Schnitzler}!1922-06-062@{6. 6. 1922}|)be}\mylabel{L02386h}  \normalsize

\doendnotes{C}
\bigskip
\vfill

\clearpage

\footnotesize

\lohead{\textsc{register}}

% Definiere theindex-Environment komplett neu ohne reledmac
\makeatletter
\renewenvironment{theindex}{%
  \section*{\indexname}%
  \setlength{\parindent}{0pt}%
  \setlength{\parskip}{0pt plus 0.3pt}%
  \let\item\@idxitem
}{%
  \clearpage
}
\makeatother

\IfFileExists{\jobname-pw.ind}{\input{\jobname-pw.ind}}{}

\end{document}

      