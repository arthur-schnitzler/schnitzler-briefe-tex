%% latex-korrekturansicht-vorspann.tex
%% Vorspann für die Korrekturansicht.
%% Lädt die gemeinsame Datei latex-vorspann.tex mit gesetztem Schalter.

\newif\ifkorrekturansicht
\korrekturansichttrue

\input{../tex-inputs/latex-vorspann}


\section[ Arthur Schnitzler an Felix Salten, {[}14. 9. 1901?{]}]{L02968 Arthur Schnitzler an Felix Salten, {[}14. 9. 1901?{]}}
\nopagebreak\mylabel{L02968v}
\rehead{ }\normalsize\beginnumbering\briefempfaengerindex{Salten, Felix@\textsc{Salten, Felix}!zzzSchnitzler, Arthur@\emph{von Arthur Schnitzler}!1901-09-141@{{[}14. 9. 1901?{]}}|(be}
\toendnotes[C]{\smallbreak\pagebreak[2]}\Standort{Wienbibliothek im Rathaus, ZPH 1681, 2.1.516.}
\physDesc{Brief, 1 Blatt, 1 Seite, 174 Zeichen
\newline{}Handschrift: Bleistift, deutsche Kurrent
\newline{}Ordnung: mit Bleistift von unbekannter Hand nummeriert: »7« }\toendnotes[C]{\smallbreak}
\pstart
           \raggedleft{}{\pb}\label{K_L02968-1v}\edtext{Samſtag}{\lemma{\textnormal{\emph{Samſtag}}}\Cendnote{\textnormal{Die Datierung folgt der Annahme, dass sich Schnitzlers Brief an Salten\pwindex{Salten, Felix 06.09.1869 – 08.10.1945@\textsc{Salten, Felix} (06.09.1869 – 08.10.1945), \emph{Schriftsteller/Schriftstellerin, Journalist/Journalistin, Chefredakteur/Chefredakteurin}|pwk} vom 16. 9. 1901 auf die in diesem
                        Korrespondenzstück angesprochenen Vorgänge bezieht.}}}\label{K_L02968-1}.\pend
           \vspace{0.5em}
\pstart
           lieber Freund, ich werde heut{ }Abend (ohne damit einen \label{K_L02968-2v}\edtext{Eintritt}{\lemma{\textnormal{\emph{Eintritt}}}\Cendnote{\textnormal{Siehe Arthur Schnitzler an Richard Beer-Hofmann, 6. 9. 1901; vgl. Arthur Schnitzler an Richard Beer-Hofmann, 21. 9. 1901. }}}\label{K_L02968-2} zu
               praejudiziren) im Club\orgindex{?? [Wiener Club September 1901]@?? [Wiener Club September 1901]|pwv} ſein,
               hoffentlich ſeh ich Sie bei dieſer Gelegenheit einmal wieder.\pend
           
\pstart
           Herzlichſt {\\[\baselineskip]}Ihr {\\[\baselineskip]}\spacefill\mbox{Arth}\pend
           \leftskip=0em{}\selectlanguage{ngerman}\endnumbering\briefempfaengerindex{Salten, Felix@\textsc{Salten, Felix}!zzzSchnitzler, Arthur@\emph{von Arthur Schnitzler}!1901-09-141@{{[}14. 9. 1901?{]}}|)be}\mylabel{L02968h}  \normalsize

\doendnotes{C}
\bigskip
\vfill

\clearpage

\footnotesize

\lohead{\textsc{register}}

% Definiere theindex-Environment komplett neu ohne reledmac
\makeatletter
\renewenvironment{theindex}{%
  \section*{\indexname}%
  \setlength{\parindent}{0pt}%
  \setlength{\parskip}{0pt plus 0.3pt}%
  \let\item\@idxitem
}{%
  \clearpage
}
\makeatother

\IfFileExists{\jobname-pw.ind}{\input{\jobname-pw.ind}}{}

\end{document}

      