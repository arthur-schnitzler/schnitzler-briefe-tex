\input{../tex-inputs/latex-pdf-vorspann}
\begin{center}
            \textcolor{red}{ENTWURF. ENTZIFFERUNG NOCH NICHT KORREKTURGELESEN}
                      \end{center}
            
               \section[Hermann Bahr an Arthur Schnitzler, 25. 4. {[}1901{]}]{ Hermann Bahr an Arthur Schnitzler, 25. 4. {[}1901{]}}\nopagebreak\mylabel{v}\rehead{ }\begin{ledgroupsized}[t]{13cm}\normalsize\beginnumbering\briefempfaengerindex{Schnitzler, Arthur@\textsc{Schnitzler, Arthur}!zzzBahr, Hermann@\emph{von Hermann Bahr}!1901-04-252@{25. 4. 1901}|(be} \toendnotes[C]{\smallbreak\pagebreak[2]} \Standort{CUL, Schnitzler, B 5b.}
\physDesc{Brief, 1 Blatt, 1 Seite
\newline{}Handschrift: schwarze Tinte, deutsche Kurrent
\newline{}Schnitzler: mit Bleistift die Jahreszahl »901« ergänzt \newline{}Ordnung: mit Bleistift von unbekannter Hand nummeriert: »76« }\buchAbdrucke{\weitereDrucke{Hermann Bahr, Arthur Schnitzler: \emph{Briefwechsel, Aufzeichnungen, Dokumente (1891–1931)}. Hg. Kurt Ifkovits und Martin Anton Müller. Göttingen: \emph{Wallstein} 2018, S. 203.} }\toendnotes[C]{\smallbreak}\pstart
           \noindent{}\centering{}{\pb}\textcolor{gray}{\textbf{Redaktion des Neuen Wiener
                           Tagblatt\orgindex{Neues Wiener Tagblatt@Neues Wiener Tagblatt|pw}}}\pend
           \pstart
           \noindent{}\centering{}\textcolor{gray}{\textbf{\textsc{Wien, I., Rothenturmstrasse,
                        Steyrerhof\oindex{Steyrerhof@\textbf{Steyrerhof}|pw}.}}}\pend
           \pstart
           \noindent{}\centering{}\textcolor{gray}{\textbf{Telegramm-Adresse: Tagblatt\orgindex{Neues Wiener Tagblatt@Neues Wiener Tagblatt|pw}, Steyrerhof, Wien\oindex{Steyrerhof@\textbf{Steyrerhof}|pw}. –
                     Telephon Nr. 384. Staats-Telephon Nr. 36.}}\pend
           \pstart
           25/4\pend
           \pstart\center{}Lieber Freund!\pend\pstart
           Danke ſehr für die Zuſendung Deines Romanes\pwindex{Schnitzler, Arthur 15.05.1862 – 21.10.1931@\textsc{Schnitzler, Arthur} (15.05.1862 – 21.10.1931), \emph{Schriftsteller, Mediziner}!Frau Bertha Garlan. Roman15.1.1901 – 15.3.1901@\strich\emph{Frau Bertha Garlan. Roman} {[}15.1.1901 – 15.3.1901{]}|pwv} und die \label{K_L01115_1v}\edtext{römiſche\oindex{Rom@\textbf{Rom}|pw} Karte}{\lemma{\textnormal{\emph{römiſche Karte}}}\Cendnote{\textnormal{In Rom\oindex{Rom@\textbf{Rom}|pwk} urlaubte Schnitzler vom
                     31. 3. bis zum 17. 4. 1901.}}}\label{K_L01115_1h},
               die mich ſehr neidiſch gemacht hat.\pend
           \pstart
           Sonntag gehe ich zu jener Vorſtellung, habe aber den Namen Deines Schützlings\pwindex{Schnitzler, Olga 17.01.1882 – 13.01.1970@\textsc{Schnitzler, Olga} (17.01.1882 – 13.01.1970), \emph{Schauspielerin, Sängerin}|pwv} vergeſſen und bitte Dich, ihn
               mir per Poſtkarte mitzutheilen.\pend
           \pstart
           Herzlichſt{\\[\baselineskip]}Dein{\\[\baselineskip]}\spacefill\mbox{Hermann}\pend
           \leftskip=0em{}\endnumbering\briefempfaengerindex{Schnitzler, Arthur@\textsc{Schnitzler, Arthur}!zzzBahr, Hermann@\emph{von Hermann Bahr}!1901-04-252@{25. 4. 1901}|)be}\mylabel{h}\end{ledgroupsized}  \newcommand{\dateiname}{L01115}\newcommand{\titel}{Hermann Bahr an Arthur Schnitzler, 25. 4. [1901]}\newcommand{\editorInnen}{ Kurt Ifkovits,  Martin Anton Müller}\input{../tex-inputs/latex-pdf-abspann}
      