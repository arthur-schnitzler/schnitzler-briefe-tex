%% latex-korrekturansicht-vorspann.tex
%% Vorspann für die Korrekturansicht.
%% Lädt die gemeinsame Datei latex-vorspann.tex mit gesetztem Schalter.

\newif\ifkorrekturansicht
\korrekturansichttrue

\input{../tex-inputs/latex-vorspann}


\section[Hermann Bahr an Arthur Schnitzler, 25. 4. {[}1901{]}]{L01115 Hermann Bahr an Arthur Schnitzler, 25. 4. {[}1901{]}}
\nopagebreak\mylabel{L01115v}
\rehead{ }\normalsize\beginnumbering\briefempfaengerindex{Schnitzler, Arthur@\textsc{Schnitzler, Arthur}!zzzBahr, Hermann@\emph{von Hermann Bahr}!1901-04-252@{25. 4. 1901}|(be}
\toendnotes[C]{\smallbreak\pagebreak[2]}\Standort{CUL, Schnitzler, B 5b.}
\physDesc{Brief, 1 Blatt, 1 Seite, 275 Zeichen
\newline{}Handschrift: schwarze Tinte, deutsche Kurrent
\newline{}Schnitzler: mit Bleistift die Jahreszahl »901« ergänzt 
\newline{}Ordnung: mit Bleistift von unbekannter Hand nummeriert:
                                    »76« }
\buchAbdrucke{\weitereDrucke{Hermann Bahr, Arthur Schnitzler: \emph{Briefwechsel, Aufzeichnungen, Dokumente (1891–1931)}. Göttingen: \emph{Wallstein} 2018, S. 203.} }\toendnotes[C]{\smallbreak}
\pstart
           \centering{}{\pb}\textcolor{gray}{\textbf{Redaktion des Neuen Wiener Tagblatt\orgindex{Neues Wiener Tagblatt@Neues Wiener Tagblatt|pw}}}\pend
           
\pstart
           \centering{}\textcolor{gray}{\textbf{\textsc{Wien, I., Rothenturmstrasse,
                        Steyrerhof\oindex{Steyrerhof@\textbf{Steyrerhof}, \emph{Gebäude (K.GBD)}|pw}.}}}\pend
           
\pstart
           \centering{}\textcolor{gray}{\textbf{Telegramm-Adresse: Tagblatt\orgindex{Neues Wiener Tagblatt@Neues Wiener Tagblatt|pw}, Steyrerhof, Wien\oindex{Steyrerhof@\textbf{Steyrerhof}, \emph{Gebäude (K.GBD)}|pw}. –
                     Telephon Nr. 384. Staats-Telephon Nr. 36.}}\pend
           
\pstart
           25/4\pend
           
\pstart\center{}Lieber Freund!\pend\vspace{0.5em}
\pstart
           Danke ſehr für die Zuſendung Deines Romanes\pwindex{Frau Bertha Garlan. Roman@\emph{Frau Bertha Garlan. Roman}|pwv} und die \label{K_L01115-1v}\edtext{römiſche\oindex{Rom@\textbf{Rom}, \emph{P.PPLC}|pw} Karte}{\lemma{\textnormal{\emph{römiſche Karte}}}\Cendnote{\textnormal{In Rom\oindex{Rom@\textbf{Rom}, \emph{P.PPLC}|pwk} urlaubte Schnitzler
                  vom 31. 3. bis zum 17. 4. 1901.}}}\label{K_L01115-1}, die mich ſehr
               neidiſch gemacht hat.\pend
           
\pstart
           Sonntag gehe ich zu jener Vorſtellung, habe aber den Namen Deines Schützlings\pwindex{Schnitzler, Olga 17.01.1882 – 13.01.1970@\textsc{Schnitzler, Olga} (17.01.1882 – 13.01.1970), \emph{Schauspieler/Schauspielerin, Sänger/Sängerin}|pwv} vergeſſen und bitte Dich, ihn
               mir per Poſtkarte mitzutheilen.\pend
           
\pstart
           Herzlichſt{\\[\baselineskip]}Dein{\\[\baselineskip]}\spacefill\mbox{Hermann}\pend
           \leftskip=0em{}\selectlanguage{ngerman}\endnumbering\briefempfaengerindex{Schnitzler, Arthur@\textsc{Schnitzler, Arthur}!zzzBahr, Hermann@\emph{von Hermann Bahr}!1901-04-252@{25. 4. 1901}|)be}\mylabel{L01115h}  \normalsize

\doendnotes{C}
\bigskip
\vfill

\clearpage

\footnotesize

\lohead{\textsc{register}}

% Definiere theindex-Environment komplett neu ohne reledmac
\makeatletter
\renewenvironment{theindex}{%
  \section*{\indexname}%
  \setlength{\parindent}{0pt}%
  \setlength{\parskip}{0pt plus 0.3pt}%
  \let\item\@idxitem
}{%
  \clearpage
}
\makeatother

\IfFileExists{\jobname-pw.ind}{\input{\jobname-pw.ind}}{}

\end{document}

      