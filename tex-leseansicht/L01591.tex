%% latex-korrekturansicht-vorspann.tex
%% Vorspann für die Korrekturansicht.
%% Lädt die gemeinsame Datei latex-vorspann.tex mit gesetztem Schalter.

\newif\ifkorrekturansicht
\korrekturansichttrue

\input{../tex-inputs/latex-vorspann}


\section[Charlotte Ehrenstein an Arthur Schnitzler, {[}16. 3.? 1906{]}]{L01591 Charlotte Ehrenstein an Arthur Schnitzler, {[}16. 3.? 1906{]}}
\nopagebreak\mylabel{L01591v}
\rehead{ }\normalsize\beginnumbering\briefempfaengerindex{Schnitzler, Arthur@\textsc{Schnitzler, Arthur}!zzzEhrenstein, Charlotte@\emph{von Charlotte Ehrenstein}!1906-03-161@{{[}16. 3.? 1906{]}}|(be}
\toendnotes[C]{\smallbreak\pagebreak[2]}\Standort{DLA, A:Schnitzler, HS.NZ85.1.2837,1.}
\physDesc{Brief, 1 Blatt, 2 Seiten, 795 Zeichen
\newline{}Handschrift: Bleistift, deutsche Kurrent
\newline{}Schnitzler: mit Bleistift beschriftet: »\textsc{Ehrenstein}« und datiert: »ca 16/3 906« }\toendnotes[C]{\smallbreak}
\pstart
           {\pb}\textsc{Hochwohlgeb. Herrn Dr. Arthur Schnitzler}! \pend
           
\pstart\center{}\textsc{Sehr geehrter Herr Doctor!}\pend\vspace{0.5em}
\pstart
           Um in dieſen vielbeſchäftigten Tagen nicht zu beläſtigen, und da Dr Kornfeld\pwindex{Kornfeld, Sigmund 21.04.1859 – 15.04.1927@\textsc{Kornfeld, Sigmund} (21.04.1859 – 15.04.1927), \emph{Psychiater/Psychiaterin}|pw} nach vielwöchentlicher Pauſe, meinen l.
                  Albert\pwindex{Ehrenstein, Albert 23.12.1886 – 08.04.1950@\textsc{Ehrenstein, Albert} (23.12.1886 – 08.04.1950), \emph{Schriftsteller/Schriftstellerin}|pw} vor einigen Tagen beſuchte, geſtatte
               ich mir heute über Alberts\pwindex{Ehrenstein, Albert 23.12.1886 – 08.04.1950@\textsc{Ehrenstein, Albert} (23.12.1886 – 08.04.1950), \emph{Schriftsteller/Schriftstellerin}|pw} Befinden zu
               berichten. Dr. Kornfeld\pwindex{Kornfeld, Sigmund 21.04.1859 – 15.04.1927@\textsc{Kornfeld, Sigmund} (21.04.1859 – 15.04.1927), \emph{Psychiater/Psychiaterin}|pw} iſt mit der Beſſerung
               in Alberts\pwindex{Ehrenstein, Albert 23.12.1886 – 08.04.1950@\textsc{Ehrenstein, Albert} (23.12.1886 – 08.04.1950), \emph{Schriftsteller/Schriftstellerin}|pw} Zuſtand in jeder Hinſicht
               zufrieden, er beſchäftigt ſich fleißig mit Büchern ernſten Inhalts, Eſſays und
               Lebensbeſchreibungen. Bei weiterer ernſter Beherrſchung dürfte Albert\pwindex{Ehrenstein, Albert 23.12.1886 – 08.04.1950@\textsc{Ehrenstein, Albert} (23.12.1886 – 08.04.1950), \emph{Schriftsteller/Schriftstellerin}|pw}, ſo meint Dr. Kornfeld\pwindex{Kornfeld, Sigmund 21.04.1859 – 15.04.1927@\textsc{Kornfeld, Sigmund} (21.04.1859 – 15.04.1927), \emph{Psychiater/Psychiaterin}|pw}, im zweiten {\pb}Semeſter noch die
               Univerſität beſuchen. Einſtweilen muſs er aber, obwohl er recht gut ausſieht noch
               immer im Bette bleiben. Mich Ihrer verehrten Frau Gemahlin\pwindex{Schnitzler, Olga 17.01.1882 – 13.01.1970@\textsc{Schnitzler, Olga} (17.01.1882 – 13.01.1970), \emph{Schauspieler/Schauspielerin, Sänger/Sängerin}|pwv} empfehlend, von meinem Mann\pwindex{Ehrenstein, Alexander 29.03.1857 – 29.05.1925@\textsc{Ehrenstein, Alexander} (29.03.1857 – 29.05.1925), \emph{Kassier/Kassierin}|pwv} u. Albert\pwindex{Ehrenstein, Albert 23.12.1886 – 08.04.1950@\textsc{Ehrenstein, Albert} (23.12.1886 – 08.04.1950), \emph{Schriftsteller/Schriftstellerin}|pw} die höflichſten Grüße, bin ich Ihre\pend
           
\pstart
           dankbare u. Sie verehrende{\\[\baselineskip]}\spacefill\mbox{Charlotte Ehrenſtein}\pend
           \leftskip=0em{}\selectlanguage{ngerman}\endnumbering\briefempfaengerindex{Schnitzler, Arthur@\textsc{Schnitzler, Arthur}!zzzEhrenstein, Charlotte@\emph{von Charlotte Ehrenstein}!1906-03-161@{{[}16. 3.? 1906{]}}|)be}\mylabel{L01591h}  \normalsize

\doendnotes{C}
\bigskip
\vfill

\clearpage

\footnotesize

\lohead{\textsc{register}}

% Definiere theindex-Environment komplett neu ohne reledmac
\makeatletter
\renewenvironment{theindex}{%
  \section*{\indexname}%
  \setlength{\parindent}{0pt}%
  \setlength{\parskip}{0pt plus 0.3pt}%
  \let\item\@idxitem
}{%
  \clearpage
}
\makeatother

\IfFileExists{\jobname-pw.ind}{\input{\jobname-pw.ind}}{}

\end{document}

      