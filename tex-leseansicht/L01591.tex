%% latex-leseansicht-vorspann.tex
%% Vorspann für die Leseansicht.
%% Lädt die gemeinsame Datei latex-vorspann.tex mit nicht gesetztem Schalter.

\newif\ifkorrekturansicht
\korrekturansichtfalse

\input{../tex-inputs/latex-vorspann}


               \section[Charlotte Ehrenstein an Arthur Schnitzler, {[}16. 3.? 1906{]}]{ Charlotte Ehrenstein an Arthur Schnitzler, {[}16. 3.? 1906{]}}\nopagebreak\mylabel{v}\rehead{ }\begin{ledgroupsized}[t]{13cm}\normalsize\beginnumbering\briefempfaengerindex{Schnitzler, Arthur@\textsc{Schnitzler, Arthur}!zzzEhrenstein, Charlotte@\emph{von Charlotte Ehrenstein}!1906-03-161@{{[}16. 3.? 1906{]}}|(be} \toendnotes[C]{\smallbreak\pagebreak[2]} \Standort{DLA, A:Schnitzler, HS.NZ85.1.2837,1.}
\physDesc{Brief, 1 Blatt, 2 Seiten
\newline{}Handschrift: Bleistift, deutsche Kurrent
\newline{}Schnitzler: mit Bleistift beschriftet: »\textsc{Ehrenstein}« und datiert: »ca 16/3 906« }\toendnotes[C]{\smallbreak}\pstart
           \noindent{}{\pb}\textsc{Hochwohlgeb. Herrn Dr. Arthur Schnitzler}! \pend
           \pstart\center{}\textsc{Sehr geehrter Herr Doctor!}\pend\pstart
           Um in dieſen vielbeſchäftigten Tagen nicht zu beläſtigen, und da Dr Kornfeld\pwindex{Kornfeld, Sigmund 21.04.1859 – 15.04.1927@\textsc{Kornfeld, Sigmund} (21.04.1859 – 15.04.1927), \emph{Psychiater}|pw} nach vielwöchentlicher Pauſe, meinen
                    l. Albert\pwindex{Ehrenstein, Albert 23.12.1886 – 08.04.1950@\textsc{Ehrenstein, Albert} (23.12.1886 – 08.04.1950), \emph{Schriftsteller}|pw} vor einigen Tagen beſuchte,
                    geſtatte ich mir heute über Albert\pwindex{Ehrenstein, Albert 23.12.1886 – 08.04.1950@\textsc{Ehrenstein, Albert} (23.12.1886 – 08.04.1950), \emph{Schriftsteller}|pw}s Befinden
                    zu berichten. Dr. Kornfeld\pwindex{Kornfeld, Sigmund 21.04.1859 – 15.04.1927@\textsc{Kornfeld, Sigmund} (21.04.1859 – 15.04.1927), \emph{Psychiater}|pw} iſt mit der
                    Beſſerung in Albert\pwindex{Ehrenstein, Albert 23.12.1886 – 08.04.1950@\textsc{Ehrenstein, Albert} (23.12.1886 – 08.04.1950), \emph{Schriftsteller}|pw}s Zuſtand in jeder
                    Hinſicht zufrieden, er beſchäftigt ſich fleißig mit Büchern ernſten Inhalts,
                    Eſſays und Lebensbeſchreibungen. Bei weiterer ernſter Beherrſchung dürfte Albert\pwindex{Ehrenstein, Albert 23.12.1886 – 08.04.1950@\textsc{Ehrenstein, Albert} (23.12.1886 – 08.04.1950), \emph{Schriftsteller}|pw}, ſo meint Dr. Kornfeld\pwindex{Kornfeld, Sigmund 21.04.1859 – 15.04.1927@\textsc{Kornfeld, Sigmund} (21.04.1859 – 15.04.1927), \emph{Psychiater}|pw}, im zweiten {\pb}Semeſter noch die Univerſität beſuchen. Einſtweilen muſs er aber, obwohl er
                    recht gut ausſieht noch immer im Bette bleiben. Mich Ihrer verehrten Frau Gemahlin\pwindex{Schnitzler, Olga 17.01.1882 – 13.01.1970@\textsc{Schnitzler, Olga} (17.01.1882 – 13.01.1970), \emph{Schauspielerin, Sängerin}|pwv} empfehlend, von
                    meinem Mann\pwindex{Ehrenstein, Alexander 29.03.1857 – 29.05.1925@\textsc{Ehrenstein, Alexander} (29.03.1857 – 29.05.1925), \emph{Kassier}|pwv} u. Albert\pwindex{Ehrenstein, Albert 23.12.1886 – 08.04.1950@\textsc{Ehrenstein, Albert} (23.12.1886 – 08.04.1950), \emph{Schriftsteller}|pw} die höflichſten Grüße, bin ich
                    Ihre\pend
           \pstart
           dankbare u. Sie verehrende{\\[\baselineskip]}\spacefill\mbox{Charlotte Ehrenſtein}\pend
           \leftskip=0em{}\endnumbering\briefempfaengerindex{Schnitzler, Arthur@\textsc{Schnitzler, Arthur}!zzzEhrenstein, Charlotte@\emph{von Charlotte Ehrenstein}!1906-03-161@{{[}16. 3.? 1906{]}}|)be}\mylabel{h}\end{ledgroupsized}  \newcommand{\dateiname}{L01591}\newcommand{\titel}{Charlotte Ehrenstein an Arthur Schnitzler, [16. 3.? 1906]}\newcommand{\editorInnen}{Martin Anton Müller und Gerd-Hermann Susen}%% latex-leseansicht-abspann.tex
%% Abspann für die Leseansicht.
%% Der Schalter \ifkorrekturansicht ist bereits durch den Vorspann gesetzt.

%% latex-abspann.tex
%% Gemeinsamer Abspann für Korrekturansicht und Leseansicht.
%% Setzt den Schalter \ifkorrekturansicht voraus (gesetzt in den
%% einbindenden Dateien latex-korrekturansicht-abspann.tex bzw.
%% latex-leseansicht-abspann.tex).
%% ---------------------------------------------------------------

\normalsize

% Das esempio-Environment wird nur in der Leseansicht benötigt
\ifkorrekturansicht\else
\newenvironment{esempio}[3]%
{
    \vspace{1.5ex}
    \rlap{\underline{#1}}
    \par
    \setlength{\parindent}{0cm}
    \nopagebreak
    \leftskip=#2cm
    \rightskip=#3cm
}
{
    \par
}
\fi

\doendnotes{C}
\bigskip
\vfill

\clearpage

\footnotesize

\ifkorrekturansicht
  \lohead{\textsc{register}}
\fi

% theindex-Environment neu definieren ohne reledmac
\makeatletter
\renewenvironment{theindex}{%
  \ifkorrekturansicht
    \section*{\indexname}%
  \else
    \subsubsection*{Index der erwähnten Entitäten}%
  \fi
  \setlength{\parindent}{0pt}%
  \setlength{\parskip}{0pt plus 0.3pt}%
  \let\item\@idxitem
}{%
  \ifkorrekturansicht\clearpage\fi
}
\makeatother

\IfFileExists{\jobname-pw.ind}{\input{\jobname-pw.ind}}{}

% Quellenangabe nur in der Leseansicht
\ifkorrekturansicht\else
% Fallback-Definitionen, falls die .tex-Datei \titel etc. nicht gesetzt hat
\providecommand{\titel}{}
\providecommand{\editorInnen}{}
\providecommand{\dateiname}{\jobname}

\vspace{3cm}

\vfill

\footnotesize
\textsc{Quelle}: \titel. Herausgegeben von {\editorInnen}. In: \emph{Arthur Schnitzler: Briefwechsel mit Autorinnen und Autoren}.
 Digitale Edition, https://schnitzler-briefe.acdh.oeaw.ac.at/{\dateiname}.html (Stand \today)
\fi

\end{document}


      