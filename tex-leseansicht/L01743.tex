%% latex-korrekturansicht-vorspann.tex
%% Vorspann für die Korrekturansicht.
%% Lädt die gemeinsame Datei latex-vorspann.tex mit gesetztem Schalter.

\newif\ifkorrekturansicht
\korrekturansichttrue

\input{../tex-inputs/latex-vorspann}


\section[Arthur Schnitzler an Hermann Bahr, 20. 12. 1907]{L01743 Arthur Schnitzler an Hermann Bahr, 20. 12. 1907}
\nopagebreak\mylabel{L01743v}
\rehead{ }\normalsize\beginnumbering\briefempfaengerindex{Bahr, Hermann@\textsc{Bahr, Hermann}!zzzSchnitzler, Arthur@\emph{von Arthur Schnitzler}!1907-12-201@{20. 12. 1907}|(be}
\toendnotes[C]{\smallbreak\pagebreak[2]}\Standort{TMW, HS AM 23389 Ba.}
\physDesc{Brief, 1 Blatt, 3 Seiten, 1027 Zeichen
\newline{}Handschrift: schwarze Tinte, deutsche Kurrent
\newline{}Ordnung: Lochung }
\buchAbdrucke{\weitereDrucke{1) Arthur Schnitzler: \emph{The Letters of Arthur Schnitzler to Hermann Bahr}. Chapel Hill: \emph{The University of North Carolina Press} 1978, S. 100–101.} \weitereDrucke{2) Hermann Bahr, Arthur Schnitzler: \emph{Briefwechsel, Aufzeichnungen, Dokumente (1891–1931)}. Göttingen: \emph{Wallstein} 2018, S. 399–400.} }\toendnotes[C]{\smallbreak}
\pstart
           {\pb}\textcolor{gray}{\textbf{Dr. Arthur Schnitzler}}\hfill 20. 12. 907\pend
           
\pstart
           \textcolor{gray}{\textbf{Wien XVIII. Spoettelgasse 7\oindex{Edmund-Weiss-Gasse 7@\textbf{Edmund-Weiß-Gasse 7}, \emph{Wohngebäude (K.WHS)}|pw}.}}\pend
           
\pstart{}lieber Hermann,\pend\vspace{0.5em}
\pstart
           ich danke dir herzlich. So ungefähr hab ich mir \textsc{Reinh}\pwindex{Reinhardt, Max 09.09.1873 – 30.10.1943@\textsc{Reinhardt, Max} (09.09.1873 – 30.10.1943), \emph{Theaterleiter/Theaterleiterin, Regisseur/Regisseurin, Schauspieler/Schauspielerin}|pw}.s Verhältnis zur \textsc{Beatrice\pwindex{Schleier der Beatrice. Schauspiel in fuenf Akten@\emph{Der Schleier der Beatrice. Schauspiel in fünf Akten}|pw}} (u Verfaſſer) vorgeſtellt. Ich werde also mit \label{LL102-1v}\substVorne{}\textsuperscript{V}\substDazwischen{}H\substHinten{}ebbel\oindex{Hebbel-Theater@\textbf{Hebbel-Theater}, \emph{Theater (K.THE)}|pw} abſchließen\label{LL102-1h} – und darf wohl
               ausſprechen, daſs der Gedanke du und die \textsc{Mildenburg}\pwindex{Bahr-Mildenburg, Anna 29.11.1872 – 27.01.1947@\textsc{Bahr-Mildenburg, Anna} (29.11.1872 – 27.01.1947), \emph{Sänger/Sängerin}|pw} wollten ſich der \textsc{Ritscher}\pwindex{Ritscher, Helene 02.06.1888 – 1964-11-27@\textsc{Ritscher, Helene} (02.06.1888 – 1964-11-27), \emph{Schauspieler/Schauspielerin}|pw} und der \textsc{Beatrice}\pwindex{Schleier der Beatrice. Schauspiel in fuenf Akten@\emph{Der Schleier der Beatrice. Schauspiel in fünf Akten}|pw} annehmen, mich höchſt wohlthuend berührt. In den Delirien meiner {\pb}Frau\pwindex{Schnitzler, Olga 17.01.1882 – 13.01.1970@\textsc{Schnitzler, Olga} (17.01.1882 – 13.01.1970), \emph{Schauspieler/Schauspielerin, Sänger/Sängerin}|pwv} kam es übrigens öfters
               vor, daſs du und die \textsc{Mildenburg}\pwindex{Bahr-Mildenburg, Anna 29.11.1872 – 27.01.1947@\textsc{Bahr-Mildenburg, Anna} (29.11.1872 – 27.01.1947), \emph{Sänger/Sängerin}|pw} oben auf dem Kaſten ſaßen. Dieser Pl\damage{atz} war Euch reſervirt; die übrigen Geſtalten trieben ſich in tieferen Regionen
               herum. Jetzt ſcherzt man darüber! So gut es Olga\pwindex{Schnitzler, Olga 17.01.1882 – 13.01.1970@\textsc{Schnitzler, Olga} (17.01.1882 – 13.01.1970), \emph{Schauspieler/Schauspielerin, Sänger/Sängerin}|pw} im ganzen ſchon geht – wir müſſen noch längere Zeit \label{K_L01743-1v}\edtext{contumazirt}{\lemma{\textnormal{\emph{contumazirt}}}\Cendnote{\textnormal{in Quarantäne}}}\label{K_L01743-1} bleiben. (Unser Bub\pwindex{Schnitzler, Heinrich 09.08.1902 – 12.07.1982@\textsc{Schnitzler, Heinrich} (09.08.1902 – 12.07.1982), \emph{Regisseur/Regisseurin, Schauspieler/Schauspielerin}|pwv} wohnt ſeit 14 Tagen bei ſeiner Großmama\pwindex{Schnitzler, Louise 1840-07-08 – 1911-09-09@\textsc{Schnitzler, Louise} (1840-07-08 – 1911-09-09)|pwv}). Alſo ob ich dich
               noch vor Deiner Abreiſe ſehen werde? Mir wärs natürlich ſehr lieb. (für alle Fälle
               ſei’s geſagt: ich bin ſorg{\pb}fältig desinfizirt eh
               ich Briefe ſchreibe)\pend
           
\pstart
           Vielleicht haſt du Zeit mir, wenigſtens in ein paar Zeilen etwas über dich zu ſagen;
               ich weiſs so gut wie nichts von dir. – \pend
           
\pstart
           Herzlichſt grüßt dich (u meine Frau\pwindex{Schnitzler, Olga 17.01.1882 – 13.01.1970@\textsc{Schnitzler, Olga} (17.01.1882 – 13.01.1970), \emph{Schauspieler/Schauspielerin, Sänger/Sängerin}|pwv} thut desgleichen){\\[\baselineskip]}dein{\\[\baselineskip]}\spacefill\mbox{Arthur}\pend
           \leftskip=0em{}\selectlanguage{ngerman}\endnumbering\briefempfaengerindex{Bahr, Hermann@\textsc{Bahr, Hermann}!zzzSchnitzler, Arthur@\emph{von Arthur Schnitzler}!1907-12-201@{20. 12. 1907}|)be}\mylabel{L01743h}  \normalsize

\doendnotes{C}
\bigskip
\vfill

\clearpage

\footnotesize

\lohead{\textsc{register}}

% Definiere theindex-Environment komplett neu ohne reledmac
\makeatletter
\renewenvironment{theindex}{%
  \section*{\indexname}%
  \setlength{\parindent}{0pt}%
  \setlength{\parskip}{0pt plus 0.3pt}%
  \let\item\@idxitem
}{%
  \clearpage
}
\makeatother

\IfFileExists{\jobname-pw.ind}{\input{\jobname-pw.ind}}{}

\end{document}

      