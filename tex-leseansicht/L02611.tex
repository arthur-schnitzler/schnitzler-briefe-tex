%% latex-korrekturansicht-vorspann.tex
%% Vorspann für die Korrekturansicht.
%% Lädt die gemeinsame Datei latex-vorspann.tex mit gesetztem Schalter.

\newif\ifkorrekturansicht
\korrekturansichttrue

\input{../tex-inputs/latex-vorspann}


\section[ Paul Goldmann an Arthur Schnitzler, 28. 2. {[}1894{]}]{L02611 Paul Goldmann an Arthur Schnitzler, 28. 2. {[}1894{]}}
\nopagebreak\mylabel{L02611v}
\rehead{ }\normalsize\beginnumbering\briefempfaengerindex{Schnitzler, Arthur@\textsc{Schnitzler, Arthur}!zzzGoldmann, Paul@\emph{von Paul Goldmann}!1894-02-281@{28. 2. {[}1894{]}}|(be}
\toendnotes[C]{\smallbreak\pagebreak[2]}\Standort{DLA, A:Schnitzler, HS.NZ85.1.3164.}
\physDesc{Brief, 1 Blatt, 3 Seiten, 916 Zeichen
\newline{}Handschrift: schwarze Tinte, deutsche Kurrent
\newline{}Schnitzler: 1) mit Bleistift auf dem ersten Blatt die Jahreszahl »94« vermerkt  2) mit rotem Buntstift zwei Unterstreichungen}\toendnotes[C]{\smallbreak}
\pstart
           \raggedleft{}{\pb}\textsc{Paris\oindex{Paris@\textbf{Paris}, \emph{P.PPLC}|pw}}, 28. Februar.\pend
           
\pstart\center{}Mein lieber Arthur,\pend\vspace{0.5em}
\pstart
           Anbei erhälſt Du den »\textsc{Mercure de France\pwindex{Mercure de France@\emph{Mercure de France}|pw}}«, die bedeutendſte unter den Pariſ\oindex{Paris@\textbf{Paris}, \emph{P.PPLC}|pw}er jungen
                  \textsc{Revuen}. \textsc{Henri Albert\pwindex{Albert, Henri 1869-11-16 – 1921-08-03@\textsc{Albert, Henri} (1869-11-16 – 1921-08-03), \emph{Journalist/Journalistin, Kritiker/Kritikerin, Übersetzer/Übersetzerin}|pw}}, von dem ich Dir \label{K_L02611-1v}\edtext{neulich}{\lemma{\textnormal{\emph{neulich}}}\Cendnote{\textnormal{Siehe Paul Goldmann an Arthur Schnitzler, 17. 2. [1894].
               }}}\label{K_L02611-1} ſchrieb, hat \label{K_L02611-2v}\edtext{Dir und \textsc{Loris\pwindex{Hofmannsthal, Hugo von 1874-02-01 – 1929-07-15@\textsc{Hofmannsthal, Hugo von} (1874-02-01 – 1929-07-15), \emph{Schriftsteller/Schriftstellerin}|pw}} darin ein paar Worte\pwindex{Le nouvel almanach de M. Bierbaum@\emph{Le nouvel almanach de M. Bierbaum}|pwv}
                  gewidmet}{\lemma{\textnormal{\emph{Dir … gewidmet}}}\Cendnote{\textnormal{Henri Albert\pwindex{Albert, Henri 1869-11-16 – 1921-08-03@\textsc{Albert, Henri} (1869-11-16 – 1921-08-03), \emph{Journalist/Journalistin, Kritiker/Kritikerin, Übersetzer/Übersetzerin}|pwk}: \emph{Le nouvel almanach de M. Bierbaum}\pwindex{Le nouvel almanach de M. Bierbaum@\emph{Le nouvel almanach de M. Bierbaum}|pwk}. In: \emph{Mercure de France}\pwindex{Mercure de France@\emph{Mercure de France}|pwk}, Jg. 10, Nr. 51, März 1894, S. 243–246, hier: S. 244–245.}}}\label{K_L02611-2} (S. 244).
               Noch ſteht mein \strikeout{Urt} Urtheil nicht ganz feſt, aber ich
               glaube, der Mann\pwindex{Albert, Henri 1869-11-16 – 1921-08-03@\textsc{Albert, Henri} (1869-11-16 – 1921-08-03), \emph{Journalist/Journalistin, Kritiker/Kritikerin, Übersetzer/Übersetzerin}|pwv} gehört zu
               uns.\pend
           
\pstart
           Wenn Du willſt, ſo {\pb}\label{K_L02611-3v}\edtext{ſchreib’ ihm direct}{\lemma{\textnormal{\emph{ſchreib’ ihm direct}}}\Cendnote{\textnormal{Schnitzler paraphrasierte diese Stelle in
                  seinem Brief an Hofmannsthal\pwindex{Hofmannsthal, Hugo von 1874-02-01 – 1929-07-15@\textsc{Hofmannsthal, Hugo von} (1874-02-01 – 1929-07-15), \emph{Schriftsteller/Schriftstellerin}|pwk} vom [9. 3. 1894].}}}\label{K_L02611-3} ein paar
                  \label{K_L02611-4v}\edtext{Worte}{\lemma{\textnormal{\emph{Worte}}}\Cendnote{\textnormal{Schnitzler dürfte Albert\pwindex{Albert, Henri 1869-11-16 – 1921-08-03@\textsc{Albert, Henri} (1869-11-16 – 1921-08-03), \emph{Journalist/Journalistin, Kritiker/Kritikerin, Übersetzer/Übersetzerin}|pwk} geschrieben haben, denn diese Stelle in dessen
                  Antwortschreiben vom 9. 4. 1894 scheint hierauf
                  Bezug zu nehmen: »Meine kleine Besprechung\pwindex{Le nouvel almanach de M. Bierbaum@\emph{Le nouvel almanach de M. Bierbaum}|pwv} wurde abgefasst, als ich unseren lieben Freund
                        Paul Goldmann\pwindex{Goldmann, Paul 31.01.1865 – 25.09.1935@\textsc{Goldmann, Paul} (31.01.1865 – 25.09.1935), \emph{Schriftsteller/Schriftstellerin, Journalist/Journalistin}|pw} erst sehr oberflächlich
                     kannte. Sie blieb zwei Monate lang auf der Redaction\orgindex{Mercure de France@Mercure de France|pwv} liegen – Diese Freundschaft hat aber in
                     keiner Weise mein Urtheil beeinflusst.« (\emph{DLA}, HS.1985.1.2331,1.)}}}\label{K_L02611-4}. Das wird ihn
               freuen (\textsc{M. Henri Albert\pwindex{Albert, Henri 1869-11-16 – 1921-08-03@\textsc{Albert, Henri} (1869-11-16 – 1921-08-03), \emph{Journalist/Journalistin, Kritiker/Kritikerin, Übersetzer/Übersetzerin}|pw}}, \textsc{25. Rue Jacob, Paris\oindex{rue Jacob@\textbf{rue Jacob}, \emph{Straße (K.STR)}|pw}}.). Natürlich deutſch. Auch \label{K_L02611-5v}\edtext{»\textsc{\begin{otherlanguage}{french}le génial\end{otherlanguage}{ }Loris\pwindex{Hofmannsthal, Hugo von 1874-02-01 – 1929-07-15@\textsc{Hofmannsthal, Hugo von} (1874-02-01 – 1929-07-15), \emph{Schriftsteller/Schriftstellerin}|pw}}«}{\lemma{\textnormal{\emph{»le génial Loris«}}}\Cendnote{\textnormal{Zitat aus der angeführten \emph{Besprechung}\pwindex{Le nouvel almanach de M. Bierbaum@\emph{Le nouvel almanach de M. Bierbaum}|pwk}{ }Alberts\pwindex{Albert, Henri 1869-11-16 – 1921-08-03@\textsc{Albert, Henri} (1869-11-16 – 1921-08-03), \emph{Journalist/Journalistin, Kritiker/Kritikerin, Übersetzer/Übersetzerin}|pwk}, S. 245.}}}\label{K_L02611-5} ſoll ihm
               ſchreiben und vielleicht für mich einen Gruß zufügen, damit ich wieder einmal
               wenigſtens etwas Indirectes von ihm höre. Willſt Du glauben, daß ich nichts weiß, was
               er ſchreibt? Daß er mir nicht einmal »\label{K_L02611-6v}\edtext{Der Thor und der Tod\pwindex{Thor und der Tod@\emph{Der Thor und der Tod}|pw}}{\lemma{\textnormal{\emph{Der Thor und der Tod}}}\Cendnote{\textnormal{\emph{Der Thor und der Tod}\pwindex{Thor und der Tod@\emph{Der Thor und der Tod}|pwk} ist im \emph{Modernen Musen-Almanach auf das Jahr 1894}\pwindex{Moderner Musen-Almanach auf das Jahr 1894. Ein Jahrbuch deutscher Kunst@\emph{Moderner Musen-Almanach auf das Jahr 1894. Ein Jahrbuch deutscher Kunst}|pwk} enthalten, den
                     Henri Albert\pwindex{Albert, Henri 1869-11-16 – 1921-08-03@\textsc{Albert, Henri} (1869-11-16 – 1921-08-03), \emph{Journalist/Journalistin, Kritiker/Kritikerin, Übersetzer/Übersetzerin}|pwk} besprochen hat.}}}\label{K_L02611-6}« geſchickt
               hat? Ich kenne alles das nur aus Deinen Briefen. Und was das {\pb}heißt, eine Sache aus Deinen Briefen kennen, darüber
               machſt Du Dir wohl ſelbſt keine Illuſionen.\pend
           
\pstart
           Schreibſt Du mir bald wieder einmal?\pend
           
\pstart
           In Treue {\\[\baselineskip]}Dein{\\[\baselineskip]}\spacefill\mbox{Paul Goldmann}\pend
           \leftskip=0em{}\selectlanguage{ngerman}\endnumbering\briefempfaengerindex{Schnitzler, Arthur@\textsc{Schnitzler, Arthur}!zzzGoldmann, Paul@\emph{von Paul Goldmann}!1894-02-281@{28. 2. {[}1894{]}}|)be}\mylabel{L02611h}  \normalsize

\doendnotes{C}
\bigskip
\vfill

\clearpage

\footnotesize

\lohead{\textsc{register}}

% Definiere theindex-Environment komplett neu ohne reledmac
\makeatletter
\renewenvironment{theindex}{%
  \section*{\indexname}%
  \setlength{\parindent}{0pt}%
  \setlength{\parskip}{0pt plus 0.3pt}%
  \let\item\@idxitem
}{%
  \clearpage
}
\makeatother

\IfFileExists{\jobname-pw.ind}{\input{\jobname-pw.ind}}{}

\end{document}

      