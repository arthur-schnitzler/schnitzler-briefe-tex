%% latex-leseansicht-vorspann.tex
%% Vorspann für die Leseansicht.
%% Lädt die gemeinsame Datei latex-vorspann.tex mit nicht gesetztem Schalter.

\newif\ifkorrekturansicht
\korrekturansichtfalse

\input{../tex-inputs/latex-vorspann}


         
         \renewcommand{\erwaehntePersonen}{Personen: Henri Albert, Hugo von Hofmannsthal}
         \renewcommand{\erwaehnteInstitutionen}{Institutionen: Mercure de France}
         \renewcommand{\erwaehnteOrte}{Orte: Paris, Wien, rue Jacob}
         \renewcommand{\erwaehnteWerke}{Werke: Der Thor und der Tod, Le nouvel almanach de M. Bierbaum, Mercure de France, Moderner Musen-Almanach auf das Jahr 1894. Ein Jahrbuch deutscher Kunst}
               \section[ Paul Goldmann an Arthur Schnitzler, 28. 2. {[}1894{]}]{ Paul Goldmann an Arthur Schnitzler, 28. 2. {[}1894{]}}\nopagebreak\mylabel{v}\rehead{ }\begin{ledgroupsized}[t]{13cm}\normalsize\beginnumbering \toendnotes[C]{\smallbreak\pagebreak[2]} \Standort{DLA, A:Schnitzler, HS.NZ85.1.3164.}
\physDesc{Brief, 1 Blatt, 3 Seiten
\newline{}Handschrift: schwarze Tinte, deutsche Kurrent
\newline{}Schnitzler: 1) mit Bleistift auf dem ersten Blatt die Jahreszahl »94« vermerkt  2) mit rotem Buntstift zwei Unterstreichungen}\toendnotes[C]{\smallbreak}\pstart
           \raggedleft{}{\pb}\textsc{Paris\oindex{Paris@\textbf{Paris}|pw}}, 28. Februar.\pend
           \pstart\center{}Mein lieber Arthur,\pend\pstart
           Anbei erhälſt Du den »\textsc{Mercure de France\pwindex{?? Werk@Nicht ermittelte Verfasserinnen und Verfasser!Mercure de France1890 – 1965@\emph{Mercure de France} {[}1890 – 1965{]}|pw}}«, die bedeutendſte unter den Pariſ\oindex{Paris@\textbf{Paris}|pw}er jungen
                  \textsc{Revuen}. \textsc{Henri Albert\pwindex{Albert, Henri 1869-11-16 – 1921-08-03@\textsc{Albert, Henri} (1869-11-16 – 1921-08-03), \emph{Journalist, Kritiker, Übersetzer}|pw}}, von dem ich Dir \label{K_L02611-1v}\edtext{neulich}{\lemma{\textnormal{\emph{neulich}}}\Cendnote{\textnormal{siehe Paul Goldmann an Arthur Schnitzler, 17. 2. [1894]}}}\label{K_L02611-1h} ſchrieb, hat \label{K_L02611-2v}\edtext{Dir und \textsc{Loris\pwindex{Hofmannsthal, Hugo von 1874-02-01 – 1929-07-15@\textsc{Hofmannsthal, Hugo von} (1874-02-01 – 1929-07-15), \emph{Schriftsteller}|pw}} darin ein paar Worte\pwindex{Albert, Henri 1869-11-16 – 1921-08-03@\textsc{Albert, Henri} (1869-11-16 – 1921-08-03), \emph{Journalist, Kritiker, Übersetzer}!Le nouvel almanach de M. Bierbaum1. 3. 1894@\strich\emph{Le nouvel almanach de M. Bierbaum} {[}1. 3. 1894{]}|pwv}
                  gewidmet}{\lemma{\textnormal{\emph{Dir … gewidmet}}}\Cendnote{\textnormal{Henri Albert\pwindex{Albert, Henri 1869-11-16 – 1921-08-03@\textsc{Albert, Henri} (1869-11-16 – 1921-08-03), \emph{Journalist, Kritiker, Übersetzer}|pwk}: \emph{Le nouvel almanach de M. Bierbaum}\pwindex{Albert, Henri 1869-11-16 – 1921-08-03@\textsc{Albert, Henri} (1869-11-16 – 1921-08-03), \emph{Journalist, Kritiker, Übersetzer}!Le nouvel almanach de M. Bierbaum1. 3. 1894@\strich\emph{Le nouvel almanach de M. Bierbaum} {[}1. 3. 1894{]}|pwk}. In: \emph{Mercure de France}\pwindex{?? Werk@Nicht ermittelte Verfasserinnen und Verfasser!Mercure de France1890 – 1965@\emph{Mercure de France} {[}1890 – 1965{]}|pwk}, Jg. 10, Nr. 51, März 1894, S. 243–246, hier: S. 244–245.}}}\label{K_L02611-2h} (S. 244).
               Noch ſteht mein \strikeout{Urt} Urtheil nicht ganz feſt, aber ich
               glaube, der Mann\pwindex{Albert, Henri 1869-11-16 – 1921-08-03@\textsc{Albert, Henri} (1869-11-16 – 1921-08-03), \emph{Journalist, Kritiker, Übersetzer}|pwv} gehört zu
               uns.\pend
           \pstart
           Wenn Du willſt, ſo {\pb}\label{K_L02611-33v}\edtext{ſchreib’ ihm direct}{\lemma{\textnormal{\emph{ſchreib’ ihm direct}}}\Cendnote{\textnormal{Schnitzler\pwindex{Schnitzler, Arthur 15.05.1862 – 21.10.1931@\textsc{Schnitzler, Arthur} (15.05.1862 – 21.10.1931), \emph{Schriftsteller, Mediziner}|pwk} paraphrasierte diese Stelle in
                  seinem Brief an Hofmannsthal\pwindex{Hofmannsthal, Hugo von 1874-02-01 – 1929-07-15@\textsc{Hofmannsthal, Hugo von} (1874-02-01 – 1929-07-15), \emph{Schriftsteller}|pwk} vom [9. 3. 1894].}}}\label{K_L02611-33h} ein paar \label{K_L02611-3v}\edtext{Worte}{\lemma{\textnormal{\emph{Worte}}}\Cendnote{\textnormal{Schnitzler\pwindex{Schnitzler, Arthur 15.05.1862 – 21.10.1931@\textsc{Schnitzler, Arthur} (15.05.1862 – 21.10.1931), \emph{Schriftsteller, Mediziner}|pwk} dürfte Albert\pwindex{Albert, Henri 1869-11-16 – 1921-08-03@\textsc{Albert, Henri} (1869-11-16 – 1921-08-03), \emph{Journalist, Kritiker, Übersetzer}|pwk} geschrieben haben, denn diese Stelle in dessen
                  Antwortschreiben vom 9. 4. 1894 scheint hierauf
                  Bezug zu nehmen: »Meine kleine Besprechung\pwindex{Albert, Henri 1869-11-16 – 1921-08-03@\textsc{Albert, Henri} (1869-11-16 – 1921-08-03), \emph{Journalist, Kritiker, Übersetzer}!Le nouvel almanach de M. Bierbaum1. 3. 1894@\strich\emph{Le nouvel almanach de M. Bierbaum} {[}1. 3. 1894{]}|pwv} wurde abgefasst, als ich unseren lieben Freund
                        Paul Goldmann\pwindex{Goldmann, Paul 31.01.1865 – 25.09.1935@\textsc{Goldmann, Paul} (31.01.1865 – 25.09.1935), \emph{Schriftsteller, Journalist}|pw} erst sehr oberflächlich
                     kannte. Sie blieb zwei Monate lang auf der Redaction\orgindex{Mercure de France@Mercure de France|pwv} liegen – Diese Freundschaft hat aber in
                     keiner Weise mein Urtheil beeinflusst.« (\emph{DLA}, HS.1985.1.2331,1.)}}}\label{K_L02611-3h}. Das wird ihn
               freuen (\textsc{M. Henri Albert\pwindex{Albert, Henri 1869-11-16 – 1921-08-03@\textsc{Albert, Henri} (1869-11-16 – 1921-08-03), \emph{Journalist, Kritiker, Übersetzer}|pw}}, \textsc{25. Rue Jacob, Paris\oindex{rue Jacob@\textbf{rue Jacob}|pw}}.). Natürlich deutſch. Auch \label{K_L02611-4v}\edtext{»\textsc{\begin{otherlanguage}{french}le génial\end{otherlanguage}{ }Loris\pwindex{Hofmannsthal, Hugo von 1874-02-01 – 1929-07-15@\textsc{Hofmannsthal, Hugo von} (1874-02-01 – 1929-07-15), \emph{Schriftsteller}|pw}}«}{\lemma{\textnormal{\emph{»le génial Loris«}}}\Cendnote{\textnormal{Zitat aus der angeführten \emph{Besprechung}\pwindex{Albert, Henri 1869-11-16 – 1921-08-03@\textsc{Albert, Henri} (1869-11-16 – 1921-08-03), \emph{Journalist, Kritiker, Übersetzer}!Le nouvel almanach de M. Bierbaum1. 3. 1894@\strich\emph{Le nouvel almanach de M. Bierbaum} {[}1. 3. 1894{]}|pwk}{ }Albert\pwindex{Albert, Henri 1869-11-16 – 1921-08-03@\textsc{Albert, Henri} (1869-11-16 – 1921-08-03), \emph{Journalist, Kritiker, Übersetzer}|pwk}s, S. 245.}}}\label{K_L02611-4h} ſoll ihm
               ſchreiben und vielleicht für mich einen Gruß zufügen, damit ich wieder einmal
               wenigſtens etwas Indirectes von ihm höre. Willſt Du glauben, daß ich nichts weiß, was
               er ſchreibt? Daß er mir nicht einmal »\label{K_L02611-55v}\edtext{Der Thor und der Tod\pwindex{Hofmannsthal, Hugo von 1874-02-01 – 1929-07-15@\textsc{Hofmannsthal, Hugo von} (1874-02-01 – 1929-07-15), \emph{Schriftsteller}!Thor und der Tod1893@\strich\emph{Der Thor und der Tod} {[}1893{]}|pw}}{\lemma{\textnormal{\emph{Der Thor und der Tod}}}\Cendnote{\textnormal{\emph{Der Thor und der Tod}\pwindex{Hofmannsthal, Hugo von 1874-02-01 – 1929-07-15@\textsc{Hofmannsthal, Hugo von} (1874-02-01 – 1929-07-15), \emph{Schriftsteller}!Thor und der Tod1893@\strich\emph{Der Thor und der Tod} {[}1893{]}|pwk} ist im \emph{Modernen Musen-Almanach auf das Jahr 1894}\pwindex{Moderner Musen-Almanach auf das Jahr 1894. Ein Jahrbuch deutscher Kunst1893@\emph{Moderner Musen-Almanach auf das Jahr 1894. Ein Jahrbuch deutscher Kunst} {[}1893{]}|pwk} enthalten, den
                     Henri Albert\pwindex{Albert, Henri 1869-11-16 – 1921-08-03@\textsc{Albert, Henri} (1869-11-16 – 1921-08-03), \emph{Journalist, Kritiker, Übersetzer}|pwk} bespricht.}}}\label{K_L02611-55h}« geſchickt
               hat? Ich kenne alles das nur aus Deinen Briefen. Und was das {\pb}heißt, eine Sache aus Deinen Briefen kennen, darüber
               machſt Du Dir wohl ſelbſt keine Illuſionen.\pend
           \pstart
           Schreibſt Du mir bald wieder einmal?\pend
           \pstart
           In Treue {\\[\baselineskip]}Dein{\\[\baselineskip]}\spacefill\mbox{Paul Goldmann}\pend
           \leftskip=0em{}
         
         \endnumbering\mylabel{h}\end{ledgroupsized}  \newcommand{\dateiname}{L02611}\newcommand{\titel}{Paul Goldmann an Arthur Schnitzler, 28. 2. [1894]}\newcommand{\editorInnen}{Martin Anton Müller und Laura Untner}%% latex-leseansicht-abspann.tex
%% Abspann für die Leseansicht.
%% Der Schalter \ifkorrekturansicht ist bereits durch den Vorspann gesetzt.

%% latex-abspann.tex
%% Gemeinsamer Abspann für Korrekturansicht und Leseansicht.
%% Setzt den Schalter \ifkorrekturansicht voraus (gesetzt in den
%% einbindenden Dateien latex-korrekturansicht-abspann.tex bzw.
%% latex-leseansicht-abspann.tex).
%% ---------------------------------------------------------------

\normalsize

% Das esempio-Environment wird nur in der Leseansicht benötigt
\ifkorrekturansicht\else
\newenvironment{esempio}[3]%
{
    \vspace{1.5ex}
    \rlap{\underline{#1}}
    \par
    \setlength{\parindent}{0cm}
    \nopagebreak
    \leftskip=#2cm
    \rightskip=#3cm
}
{
    \par
}
\fi

\doendnotes{C}
\bigskip
\vfill

\clearpage

\footnotesize

\ifkorrekturansicht
  \lohead{\textsc{register}}
\fi

% theindex-Environment neu definieren ohne reledmac
\makeatletter
\renewenvironment{theindex}{%
  \ifkorrekturansicht
    \section*{\indexname}%
  \else
    \subsubsection*{Index der erwähnten Entitäten}%
  \fi
  \setlength{\parindent}{0pt}%
  \setlength{\parskip}{0pt plus 0.3pt}%
  \let\item\@idxitem
}{%
  \ifkorrekturansicht\clearpage\fi
}
\makeatother

\IfFileExists{\jobname-pw.ind}{\input{\jobname-pw.ind}}{}

% Quellenangabe nur in der Leseansicht
\ifkorrekturansicht\else
% Fallback-Definitionen, falls die .tex-Datei \titel etc. nicht gesetzt hat
\providecommand{\titel}{}
\providecommand{\editorInnen}{}
\providecommand{\dateiname}{\jobname}

\vspace{3cm}

\vfill

\footnotesize
\textsc{Quelle}: \titel. Herausgegeben von {\editorInnen}. In: \emph{Arthur Schnitzler: Briefwechsel mit Autorinnen und Autoren}.
 Digitale Edition, https://schnitzler-briefe.acdh.oeaw.ac.at/{\dateiname}.html (Stand \today)
\fi

\end{document}


      