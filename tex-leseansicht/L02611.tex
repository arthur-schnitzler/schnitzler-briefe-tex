%% latex-leseansicht-vorspann.tex
%% Vorspann für die Leseansicht.
%% Lädt die gemeinsame Datei latex-vorspann.tex mit nicht gesetztem Schalter.

\newif\ifkorrekturansicht
\korrekturansichtfalse

\input{../tex-inputs/latex-vorspann}


\section[ Paul Goldmann an Arthur Schnitzler, 28. 2. {[}1894{]}]{L02611 Paul Goldmann an Arthur Schnitzler,  28. 2. [1894]}
\nopagebreak\mylabel{L02611v}
\rehead{ }\normalsize\beginnumbering\briefempfaengerindex{Schnitzler, Arthur@\textsc{Schnitzler, Arthur}!zzzGoldmann, Paul@\emph{von Paul Goldmann}!1894-02-281@{28. 2. [1894]}|(be}
\toendnotes[C]{\smallbreak\pagebreak[2]}
\correspDesc{Versand  durch Paul Goldmann am 28. 2. [1894] in Paris
\newline{}Erhalt  durch Arthur Schnitzler im Zeitraum [1. 3. 1894
                  – 5. 3. 1894?] in Wien}\toendnotes[C]{\smallbreak}
\Standort{DLA, A:Schnitzler, HS.NZ85.1.3164.}
\physDesc{Brief, 1 Blatt, 3 Seiten, 916 Zeichen
\newline{}Handschrift: schwarze Tinte, deutsche Kurrent
\newline{}Schnitzler: 1) mit Bleistift auf dem ersten Blatt die Jahreszahl »94« vermerkt  2) mit rotem Buntstift zwei Unterstreichungen}\toendnotes[C]{\smallbreak}
\pstart
           \raggedleft{}{\pb}\textsc{Paris\oindex{Paris@\textbf{Paris}, \emph{Hauptstadt}|pw}}, 28. Februar.\pend
           
\pstart\center{}Mein lieber Arthur,\pend\vspace{0.5em}
\pstart
           Anbei erhälſt Du den »\textsc{Mercure de France\pwindex{Mercure de France@\emph{Mercure de France}|pw}}«, die bedeutendſte unter den Pariſ\oindex{Paris@\textbf{Paris}, \emph{Hauptstadt}|pw}er jungen
                  \textsc{Revuen}. \textsc{Henri Albert\pwindex{Albert, Henri 16.\,11.\,1869 Niederbronn-les-Bains – 3.\,8.\,1921 Straßburg@\textsc{Albert, Henri} (16.\,11.\,1869 Niederbronn-les-Bains – 3.\,8.\,1921 Straßburg), \emph{Journalist, Kritiker, Übersetzer}|pw}}, von dem ich Dir \label{K_L02611-1v}\edtext{neulich}{\lemma{\textnormal{\emph{neulich}}}\Cendnote{\textnormal{Siehe XXXX Auszeichnungsfehler: Dokument L02609 nicht gefunden.
               }}}\label{K_L02611-1}{ }ſchrieb, hat \label{K_L02611-2v}\edtext{Dir und \textsc{Loris\pwindex{Hofmannsthal, Hugo von 1.\,2.\,1874 Wien – 15.\,7.\,1929 Rodaun@\textsc{Hofmannsthal, Hugo von} (1.\,2.\,1874 Wien – 15.\,7.\,1929 Rodaun), \emph{Schriftsteller}|pw}} darin ein paar Worte\pwindex{Albert, Henri 16.\,11.\,1869 Niederbronn-les-Bains – 3.\,8.\,1921 Straßburg@\textsc{Albert, Henri} (16.\,11.\,1869 Niederbronn-les-Bains – 3.\,8.\,1921 Straßburg), \emph{Journalist, Kritiker, Übersetzer}!Le nouvel almanach de M. Bierbaum@\strich\emph{Le nouvel almanach de M. Bierbaum}|pwv}
                  gewidmet}{\lemma{\textnormal{\emph{Dir … gewidmet}}}\Cendnote{\textnormal{Henri Albert\pwindex{Albert, Henri 16.\,11.\,1869 Niederbronn-les-Bains – 3.\,8.\,1921 Straßburg@\textsc{Albert, Henri} (16.\,11.\,1869 Niederbronn-les-Bains – 3.\,8.\,1921 Straßburg), \emph{Journalist, Kritiker, Übersetzer}|pwk}: \emph{Le nouvel almanach de M. Bierbaum}\pwindex{Albert, Henri 16.\,11.\,1869 Niederbronn-les-Bains – 3.\,8.\,1921 Straßburg@\textsc{Albert, Henri} (16.\,11.\,1869 Niederbronn-les-Bains – 3.\,8.\,1921 Straßburg), \emph{Journalist, Kritiker, Übersetzer}!Le nouvel almanach de M. Bierbaum@\strich\emph{Le nouvel almanach de M. Bierbaum}|pwk}. In: \emph{Mercure de France}\pwindex{Mercure de France@\emph{Mercure de France}|pwk}, Jg. 10, Nr. 51, März 1894, S. 243–246, hier: S. 244–245.}}}\label{K_L02611-2} (S. 244).
               Noch{ }ſteht mein \strikeout{Urt} Urtheil nicht ganz feſt, aber ich
               glaube, der Mann\pwindex{Albert, Henri 16.\,11.\,1869 Niederbronn-les-Bains – 3.\,8.\,1921 Straßburg@\textsc{Albert, Henri} (16.\,11.\,1869 Niederbronn-les-Bains – 3.\,8.\,1921 Straßburg), \emph{Journalist, Kritiker, Übersetzer}|pwv} gehört zu
               uns.\pend
           
\pstart
           Wenn Du willſt,{ }ſo {\pb}\label{K_L02611-3v}\edtext{ſchreib’ ihm direct}{\lemma{\textnormal{\emph{schreib’ ihm direct}}}\Cendnote{\textnormal{Schnitzler paraphrasierte diese Stelle in
                  seinem Brief an Hofmannsthal\pwindex{Hofmannsthal, Hugo von 1.\,2.\,1874 Wien – 15.\,7.\,1929 Rodaun@\textsc{Hofmannsthal, Hugo von} (1.\,2.\,1874 Wien – 15.\,7.\,1929 Rodaun), \emph{Schriftsteller}|pwk} vom XXXX Auszeichnungsfehler: Dokument L00305 nicht gefunden.}}}\label{K_L02611-3} ein paar
                  \label{K_L02611-4v}\edtext{Worte}{\lemma{\textnormal{\emph{Worte}}}\Cendnote{\textnormal{Schnitzler dürfte Albert\pwindex{Albert, Henri 16.\,11.\,1869 Niederbronn-les-Bains – 3.\,8.\,1921 Straßburg@\textsc{Albert, Henri} (16.\,11.\,1869 Niederbronn-les-Bains – 3.\,8.\,1921 Straßburg), \emph{Journalist, Kritiker, Übersetzer}|pwk} geschrieben haben, denn diese Stelle in dessen
                  Antwortschreiben vom 9. 4. 1894 scheint hierauf
                  Bezug zu nehmen: »Meine kleine Besprechung\pwindex{Albert, Henri 16.\,11.\,1869 Niederbronn-les-Bains – 3.\,8.\,1921 Straßburg@\textsc{Albert, Henri} (16.\,11.\,1869 Niederbronn-les-Bains – 3.\,8.\,1921 Straßburg), \emph{Journalist, Kritiker, Übersetzer}!Le nouvel almanach de M. Bierbaum@\strich\emph{Le nouvel almanach de M. Bierbaum}|pwv} wurde abgefasst, als ich unseren lieben Freund
                        Paul Goldmann\pwindex{Goldmann, Paul 31.\,1.\,1865 Breslau – 25.\,9.\,1935 Wien@\textsc{Goldmann, Paul} (31.\,1.\,1865 Breslau – 25.\,9.\,1935 Wien), \emph{Schriftsteller, Journalist}|pw} erst sehr oberflächlich
                     kannte. Sie blieb zwei Monate lang auf der Redaction\orgindex{Mercure de France@Mercure de France|pwv} liegen – Diese Freundschaft hat aber in
                     keiner Weise mein Urtheil beeinflusst.« (\emph{DLA}, HS.1985.1.2331,1.)}}}\label{K_L02611-4}. Das wird ihn
               freuen (\textsc{M. Henri Albert\pwindex{Albert, Henri 16.\,11.\,1869 Niederbronn-les-Bains – 3.\,8.\,1921 Straßburg@\textsc{Albert, Henri} (16.\,11.\,1869 Niederbronn-les-Bains – 3.\,8.\,1921 Straßburg), \emph{Journalist, Kritiker, Übersetzer}|pw}}, \textsc{25. Rue Jacob, Paris\oindex{rue Jacob@\textbf{rue Jacob}, \emph{Straße}|pw}}.). Natürlich deutſch. Auch \label{K_L02611-5v}\edtext{»\textsc{\begin{otherlanguage}{french}le génial\end{otherlanguage}{ }Loris\pwindex{Hofmannsthal, Hugo von 1.\,2.\,1874 Wien – 15.\,7.\,1929 Rodaun@\textsc{Hofmannsthal, Hugo von} (1.\,2.\,1874 Wien – 15.\,7.\,1929 Rodaun), \emph{Schriftsteller}|pw}}«}{\lemma{\textnormal{\emph{»le génial Loris«}}}\Cendnote{\textnormal{Zitat aus der angeführten \emph{Besprechung}\pwindex{Albert, Henri 16.\,11.\,1869 Niederbronn-les-Bains – 3.\,8.\,1921 Straßburg@\textsc{Albert, Henri} (16.\,11.\,1869 Niederbronn-les-Bains – 3.\,8.\,1921 Straßburg), \emph{Journalist, Kritiker, Übersetzer}!Le nouvel almanach de M. Bierbaum@\strich\emph{Le nouvel almanach de M. Bierbaum}|pwk}{ }Alberts\pwindex{Albert, Henri 16.\,11.\,1869 Niederbronn-les-Bains – 3.\,8.\,1921 Straßburg@\textsc{Albert, Henri} (16.\,11.\,1869 Niederbronn-les-Bains – 3.\,8.\,1921 Straßburg), \emph{Journalist, Kritiker, Übersetzer}|pwk}, S. 245.}}}\label{K_L02611-5}{ }ſoll ihm{ }ſchreiben und vielleicht für mich einen Gruß zufügen, damit ich wieder einmal
               wenigſtens etwas Indirectes von ihm höre. Willſt Du glauben, daß ich nichts weiß, was
               er{ }ſchreibt? Daß er mir nicht einmal »\label{K_L02611-6v}\edtext{Der Thor und der Tod\pwindex{Hofmannsthal, Hugo von 1.\,2.\,1874 Wien – 15.\,7.\,1929 Rodaun@\textsc{Hofmannsthal, Hugo von} (1.\,2.\,1874 Wien – 15.\,7.\,1929 Rodaun), \emph{Schriftsteller}!Thor und der Tod@\strich\emph{Der Thor und der Tod}|pw}}{\lemma{\textnormal{\emph{Der Thor und der Tod}}}\Cendnote{\textnormal{\emph{Der Thor und der Tod}\pwindex{Hofmannsthal, Hugo von 1.\,2.\,1874 Wien – 15.\,7.\,1929 Rodaun@\textsc{Hofmannsthal, Hugo von} (1.\,2.\,1874 Wien – 15.\,7.\,1929 Rodaun), \emph{Schriftsteller}!Thor und der Tod@\strich\emph{Der Thor und der Tod}|pwk} ist im \emph{Modernen Musen-Almanach auf das Jahr 1894}\pwindex{Moderner Musen-Almanach auf das Jahr 1894. Ein Jahrbuch deutscher Kunst@\emph{Moderner Musen-Almanach auf das Jahr 1894. Ein Jahrbuch deutscher Kunst}|pwk} enthalten, den
                     Henri Albert\pwindex{Albert, Henri 16.\,11.\,1869 Niederbronn-les-Bains – 3.\,8.\,1921 Straßburg@\textsc{Albert, Henri} (16.\,11.\,1869 Niederbronn-les-Bains – 3.\,8.\,1921 Straßburg), \emph{Journalist, Kritiker, Übersetzer}|pwk} besprochen hat.}}}\label{K_L02611-6}« geſchickt
               hat? Ich kenne alles das nur aus Deinen Briefen. Und was das {\pb}heißt, eine Sache aus Deinen Briefen kennen, darüber
               machſt Du Dir wohl{ }ſelbſt keine Illuſionen.\pend
           
\pstart
           Schreibſt Du mir bald wieder einmal?\pend
           
\pstart
           In Treue {\\[\baselineskip]}Dein{\\[\baselineskip]}\spacefill\mbox{Paul Goldmann}\pend
           \leftskip=0em{}\selectlanguage{ngerman}\endnumbering\briefempfaengerindex{Schnitzler, Arthur@\textsc{Schnitzler, Arthur}!zzzGoldmann, Paul@\emph{von Paul Goldmann}!1894-02-281@{28. 2. [1894]}|)be}\mylabel{L02611h}  \newcommand{\dateiname}{L02611}\newcommand{\titel}{Paul Goldmann an Arthur Schnitzler, 28. 2. [1894]}\newcommand{\editorInnen}{Martin Anton Müller und Laura Untner}%% latex-leseansicht-abspann.tex
%% Abspann für die Leseansicht.
%% Der Schalter \ifkorrekturansicht ist bereits durch den Vorspann gesetzt.

%% latex-abspann.tex
%% Gemeinsamer Abspann für Korrekturansicht und Leseansicht.
%% Setzt den Schalter \ifkorrekturansicht voraus (gesetzt in den
%% einbindenden Dateien latex-korrekturansicht-abspann.tex bzw.
%% latex-leseansicht-abspann.tex).
%% ---------------------------------------------------------------

\normalsize

% Das esempio-Environment wird nur in der Leseansicht benötigt
\ifkorrekturansicht\else
\newenvironment{esempio}[3]%
{
    \vspace{1.5ex}
    \rlap{\underline{#1}}
    \par
    \setlength{\parindent}{0cm}
    \nopagebreak
    \leftskip=#2cm
    \rightskip=#3cm
}
{
    \par
}
\fi

\doendnotes{C}
\bigskip
\vfill

\clearpage

\footnotesize

\ifkorrekturansicht
  \lohead{\textsc{register}}
\fi

% theindex-Environment neu definieren ohne reledmac
\makeatletter
\renewenvironment{theindex}{%
  \ifkorrekturansicht
    \section*{\indexname}%
  \else
    \subsubsection*{Index der erwähnten Entitäten}%
  \fi
  \setlength{\parindent}{0pt}%
  \setlength{\parskip}{0pt plus 0.3pt}%
  \let\item\@idxitem
}{%
  \ifkorrekturansicht\clearpage\fi
}
\makeatother

\IfFileExists{\jobname-pw.ind}{\input{\jobname-pw.ind}}{}

% Quellenangabe nur in der Leseansicht
\ifkorrekturansicht\else
% Fallback-Definitionen, falls die .tex-Datei \titel etc. nicht gesetzt hat
\providecommand{\titel}{}
\providecommand{\editorInnen}{}
\providecommand{\dateiname}{\jobname}

\vspace{3cm}

\vfill

\footnotesize
\textsc{Quelle}: \titel. Herausgegeben von {\editorInnen}. In: \emph{Arthur Schnitzler: Briefwechsel mit Autorinnen und Autoren}.
 Digitale Edition, https://schnitzler-briefe.acdh.oeaw.ac.at/{\dateiname}.html (Stand \today)
\fi

\end{document}


