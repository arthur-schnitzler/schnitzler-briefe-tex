%% latex-leseansicht-vorspann.tex
%% Vorspann für die Leseansicht.
%% Lädt die gemeinsame Datei latex-vorspann.tex mit nicht gesetztem Schalter.

\newif\ifkorrekturansicht
\korrekturansichtfalse

\input{../tex-inputs/latex-vorspann}


\section[Arthur Schnitzler an Adalbert Seligmann, 15. 6. 1897]{L00688 Arthur Schnitzler an Adalbert Seligmann, 15. 6. 1897}
\nopagebreak\mylabel{L00688v}
\rehead{ }\normalsize\beginnumbering\briefempfaengerindex{Seligmann, Adalbert Franz@\textsc{Seligmann, Adalbert Franz}!zzzSchnitzler, Arthur@\emph{von Arthur Schnitzler}!1897-06-151@{15. 6. 1897}|(be}
\toendnotes[C]{\smallbreak\pagebreak[2]}
\correspDesc{Versand  durch Arthur Schnitzler am 15. 6. 1897 in Wien
\newline{}Erhalt  durch Adalbert Seligmann im Zeitraum [15. 6. 1897
                  – 19. 6. 1897?] in Wien}\toendnotes[C]{\smallbreak}
\Standort{Wienbibliothek im Rathaus, H.I.N.-96445.}
\physDesc{Visitenkarte, 392 Zeichen
\newline{}Handschrift: schwarze Tinte, deutsche Kurrent}\toendnotes[C]{\smallbreak}
\pstart
           \noindent{}{\pb}Herzlichſten Dank! Wirklich köſtlich. Eine
               Bemerkung geſtatten Sie mir. So wunderbar der \textsc{Burckhard}\pwindex{Burckhard, Max Eugen 14.\,7.\,1854 Korneuburg – 16.\,3.\,1912 Wien@\textsc{Burckhard, Max Eugen} (14.\,7.\,1854 Korneuburg – 16.\,3.\,1912 Wien), \emph{Schriftsteller, Rechtswissenschaftler, Theaterleiter}|pw}ſche Stil getroffen; die Satire\pwindex{Seligmann, Adalbert Franz 2.\,4.\,1862 Wien – 13.\,12.\,1945 ebd.@\textsc{Seligmann, Adalbert Franz} (2.\,4.\,1862 Wien – 13.\,12.\,1945 ebd.), \emph{Maler, Publizist}!Timon Sums, Bekenntnisse einer schönen Seele. (3798. Fortsetzung und Schluss.)@\strich\emph{Timon Sums, Bekenntnisse einer schönen Seele. (3798. Fortsetzung und Schluss.)}|pwv} auf{ }ſein \uline{Weſen} geht manchmal{ }ſehr
               daneben. Sie haben eine Seite von ihm als das ganze genommen und ihm dadurch,{ }ſcheint
               mir, in gewiſſem Sinn Unrecht gethan. {\pb}Ich{ }ſage Ihnen das, weil ich das Buch\pwindex{Hinter dem Leben@\emph{Hinter dem Leben}|pwv}{ }ſonſt{ }ſo wunderbar finde.\pend
           \pstart Herzlichen Gruß Ihr{ }ſehr ergebener\pend{}
\pstart
           \centering{}\textcolor{gray}{\textbf{D\textsuperscript{r} Arthur Schnitzler}}\pend
           
\pstart
           Wien\oindex{Wien@\textbf{Wien}, \emph{Verwaltungsgebiet}|pw}{ }15. 6. 97.\pend
           \selectlanguage{ngerman}\endnumbering\briefempfaengerindex{Seligmann, Adalbert Franz@\textsc{Seligmann, Adalbert Franz}!zzzSchnitzler, Arthur@\emph{von Arthur Schnitzler}!1897-06-151@{15. 6. 1897}|)be}\mylabel{L00688h}  \newcommand{\dateiname}{L00688}\newcommand{\titel}{Arthur Schnitzler an Adalbert Seligmann, 15. 6. 1897}\newcommand{\editorInnen}{Martin Anton Müller und Gerd-Hermann Susen}%% latex-leseansicht-abspann.tex
%% Abspann für die Leseansicht.
%% Der Schalter \ifkorrekturansicht ist bereits durch den Vorspann gesetzt.

%% latex-abspann.tex
%% Gemeinsamer Abspann für Korrekturansicht und Leseansicht.
%% Setzt den Schalter \ifkorrekturansicht voraus (gesetzt in den
%% einbindenden Dateien latex-korrekturansicht-abspann.tex bzw.
%% latex-leseansicht-abspann.tex).
%% ---------------------------------------------------------------

\normalsize

% Das esempio-Environment wird nur in der Leseansicht benötigt
\ifkorrekturansicht\else
\newenvironment{esempio}[3]%
{
    \vspace{1.5ex}
    \rlap{\underline{#1}}
    \par
    \setlength{\parindent}{0cm}
    \nopagebreak
    \leftskip=#2cm
    \rightskip=#3cm
}
{
    \par
}
\fi

\doendnotes{C}
\bigskip
\vfill

\clearpage

\footnotesize

\ifkorrekturansicht
  \lohead{\textsc{register}}
\fi

% theindex-Environment neu definieren ohne reledmac
\makeatletter
\renewenvironment{theindex}{%
  \ifkorrekturansicht
    \section*{\indexname}%
  \else
    \subsubsection*{Index der erwähnten Entitäten}%
  \fi
  \setlength{\parindent}{0pt}%
  \setlength{\parskip}{0pt plus 0.3pt}%
  \let\item\@idxitem
}{%
  \ifkorrekturansicht\clearpage\fi
}
\makeatother

\IfFileExists{\jobname-pw.ind}{\input{\jobname-pw.ind}}{}

% Quellenangabe nur in der Leseansicht
\ifkorrekturansicht\else
% Fallback-Definitionen, falls die .tex-Datei \titel etc. nicht gesetzt hat
\providecommand{\titel}{}
\providecommand{\editorInnen}{}
\providecommand{\dateiname}{\jobname}

\vspace{3cm}

\vfill

\footnotesize
\textsc{Quelle}: \titel. Herausgegeben von {\editorInnen}. In: \emph{Arthur Schnitzler: Briefwechsel mit Autorinnen und Autoren}.
 Digitale Edition, https://schnitzler-briefe.acdh.oeaw.ac.at/{\dateiname}.html (Stand \today)
\fi

\end{document}


