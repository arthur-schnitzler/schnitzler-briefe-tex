%% latex-leseansicht-vorspann.tex
%% Vorspann für die Leseansicht.
%% Lädt die gemeinsame Datei latex-vorspann.tex mit nicht gesetztem Schalter.

\newif\ifkorrekturansicht
\korrekturansichtfalse

\input{../tex-inputs/latex-vorspann}


         
         \renewcommand{\erwaehntePersonen}{Personen: Otto Brahm, Georg Hirschfeld}
         \renewcommand{\erwaehnteOrte}{Orte: Dornbach, London, Neuwaldegg, Paris, Wien}
         \renewcommand{\erwaehnteWerke}{
               \section[Hugo von Hofmannsthal an Arthur Schnitzler, 17. 5. {[}1897{]}]{ Hugo von Hofmannsthal an Arthur Schnitzler, 17. 5. {[}1897{]}}\nopagebreak\mylabel{v}\rehead{ }\begin{ledgroupsized}[t]{13cm}\normalsize\beginnumbering \toendnotes[C]{\smallbreak\pagebreak[2]} \Standort{CUL, Schnitzler, B 43.}
\physDesc{Brief, 1 Blatt, 4 Seiten
\newline{}Handschrift: schwarze Tinte, deutsche Kurrent
\newline{}Schnitzler: mit Bleistift die Jahreszahl ergänzt: »97« \newline{}Ordnung: mit Bleistift von unbekannter Hand nummeriert:
                                        »90a« }\buchAbdrucke{\weitereDrucke{Hugo von Hofmannsthal, Arthur Schnitzler: \emph{Briefwechsel}. Hg. Therese Nickl und Heinrich Schnitzler. Frankfurt am Main: \emph{S. Fischer} 1964, S. 86.} }\pstart
           \raggedleft{}{\pb}Wien\oindex{Wien@\textbf{Wien}|pw}{ }17\textsuperscript{ten} Mai.\pend
           \pstart{}Mein lieber Arthur\pend\pstart
           ich höre mit großer Freude von verſchiedenen, daſs es Ihnen ſehr gut \strikeout{G} geht und hoffe, dieſer Brief trifft Sie noch
                    vor der Abreiſe nach London\oindex{London@\textbf{London}|pw}. Mir ginge es auch
                    recht gut (beſſer als lange) wenn nicht dieſes unglaubliche Wetter wäre. Man muß
                    das Wetter erwähnen, es iſt {\pb}zu wichtig. Seit den erſten Tagen Mai iſt ein finſterer Himmel
                    wie im Februar, ſtundenlange Regengüſſe, 3–5 Grad, manchmal in einer Woche kein
                    Stück blauer Himmel. Und da ſchon vorher ein paar ſehr ſchöne Tage waren, ſo
                    ſehnt man ſich umſomehr, wie nach einem unterbrochenen Traum. Ich war die ganze
                    Zeit faſt nur zuhaus und habe meine Grammatika gelernt {\pb}und alte Texte geleſen. Ich
                    freue mich mehr als ich ſagen kann, darauf wieder aufs Land zu können, das
                    drängt alles andere zurück.\pend
           \pstart
           Vom Sommer weiß ich noch nicht viel beſtimmtes. Jedenfalls bin ich bis zum
                            20\textsuperscript{ten} Juni in Wien\oindex{Wien@\textbf{Wien}|pw}. Einen Abend, dann noch einen und einen kalten
                    unfreundlichen Tag am Land (Dornbach\oindex{Dornbach@\textbf{Dornbach}|pw}, Neuwaldegg\oindex{Neuwaldegg@\textbf{Neuwaldegg}|pw}) hab ich mit Brahm\pwindex{Brahm, Otto 05.02.1856 – 28.11.1912@\textsc{Brahm, Otto} (05.02.1856 – 28.11.1912), \emph{Theaterleiter, Regisseur}|pw} verbracht, jedesmal {\pb}nur mit ihm und Hirſchfeld\pwindex{Hirschfeld, Georg 11.02.1873 – 17.01.1942@\textsc{Hirschfeld, Georg} (11.02.1873 – 17.01.1942), \emph{Schriftsteller}|pw}. Brahm\pwindex{Brahm, Otto 05.02.1856 – 28.11.1912@\textsc{Brahm, Otto} (05.02.1856 – 28.11.1912), \emph{Theaterleiter, Regisseur}|pw} iſt ein überaus guter und angenehmer Menſch; es muſs von ſolchen
                    Menſchen wohl gar nicht ſo wenige geben und wir ſind manchmal zu ſehr geneigt,
                    diejenigen, die wir zufällig nicht kennen, abzuleugnen. Wir ſind überhaupt ſehr
                    vorlaut. Wir haben aber vielleicht doch ein bischen Talent.\pend
           \pstart
           Leben Sie weiter wohl und erfreuen uns bald durch merkwürdige Erzählungen.\pend
           \pstart Ihr\spacefill\mbox{Hugo.}\pend{}
         
         \endnumbering\mylabel{h}\end{ledgroupsized}  \newcommand{\dateiname}{L00677}\newcommand{\titel}{Hugo von Hofmannsthal an Arthur Schnitzler, 17. 5. [1897]}\newcommand{\editorInnen}{Martin Anton Müller und Gerd-Hermann Susen}%% latex-leseansicht-abspann.tex
%% Abspann für die Leseansicht.
%% Der Schalter \ifkorrekturansicht ist bereits durch den Vorspann gesetzt.

%% latex-abspann.tex
%% Gemeinsamer Abspann für Korrekturansicht und Leseansicht.
%% Setzt den Schalter \ifkorrekturansicht voraus (gesetzt in den
%% einbindenden Dateien latex-korrekturansicht-abspann.tex bzw.
%% latex-leseansicht-abspann.tex).
%% ---------------------------------------------------------------

\normalsize

% Das esempio-Environment wird nur in der Leseansicht benötigt
\ifkorrekturansicht\else
\newenvironment{esempio}[3]%
{
    \vspace{1.5ex}
    \rlap{\underline{#1}}
    \par
    \setlength{\parindent}{0cm}
    \nopagebreak
    \leftskip=#2cm
    \rightskip=#3cm
}
{
    \par
}
\fi

\doendnotes{C}
\bigskip
\vfill

\clearpage

\footnotesize

\ifkorrekturansicht
  \lohead{\textsc{register}}
\fi

% theindex-Environment neu definieren ohne reledmac
\makeatletter
\renewenvironment{theindex}{%
  \ifkorrekturansicht
    \section*{\indexname}%
  \else
    \subsubsection*{Index der erwähnten Entitäten}%
  \fi
  \setlength{\parindent}{0pt}%
  \setlength{\parskip}{0pt plus 0.3pt}%
  \let\item\@idxitem
}{%
  \ifkorrekturansicht\clearpage\fi
}
\makeatother

\IfFileExists{\jobname-pw.ind}{\input{\jobname-pw.ind}}{}

% Quellenangabe nur in der Leseansicht
\ifkorrekturansicht\else
% Fallback-Definitionen, falls die .tex-Datei \titel etc. nicht gesetzt hat
\providecommand{\titel}{}
\providecommand{\editorInnen}{}
\providecommand{\dateiname}{\jobname}

\vspace{3cm}

\vfill

\footnotesize
\textsc{Quelle}: \titel. Herausgegeben von {\editorInnen}. In: \emph{Arthur Schnitzler: Briefwechsel mit Autorinnen und Autoren}.
 Digitale Edition, https://schnitzler-briefe.acdh.oeaw.ac.at/{\dateiname}.html (Stand \today)
\fi

\end{document}


      