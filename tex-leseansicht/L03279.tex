%% latex-korrekturansicht-vorspann.tex
%% Vorspann für die Korrekturansicht.
%% Lädt die gemeinsame Datei latex-vorspann.tex mit gesetztem Schalter.

\newif\ifkorrekturansicht
\korrekturansichttrue

\input{../tex-inputs/latex-vorspann}


\section[ Felix Salten an Arthur Schnitzler, {[}10. 7. 1898{]}]{L03279 Felix Salten an Arthur Schnitzler, {[}10. 7. 1898{]}}
\nopagebreak\mylabel{L03279v}
\rehead{ }\normalsize\beginnumbering\briefempfaengerindex{Schnitzler, Arthur@\textsc{Schnitzler, Arthur}!zzzSalten, Felix@\emph{von Felix Salten}!1898-07-103@{{[}10. 7. 1898{]}}|(be}
\toendnotes[C]{\smallbreak\pagebreak[2]}\Standort{CUL, Schnitzler, B 89, A 2.}
\physDesc{Brief, 1 Blatt, 1 Seite, 286 Zeichen
\newline{}Handschrift: schwarze Tinte, lateinische Kurrent
\newline{}Schnitzler: mit Bleistift datiert: »10/7 98« 
\newline{}Ordnung: mit Bleistift von unbekannter Hand nummeriert: »103« }\toendnotes[C]{\smallbreak}
\pstart
           \raggedleft{}{\pb}Sonntag{\\}Mittag.\pend
           \vspace{0.5em}
\pstart
           Lieber Arthur, soeben erhalte ich die Nachricht, dass der \label{K_L03279-1v}\edtext{Erzh.\pwindex{Woelfling, Leopold Ferdinand Salvator 1868-12-02 – 1935-07-04@\textsc{Wölfling, Leopold Ferdinand Salvator} (1868-12-02 – 1935-07-04), \emph{Erzherzog/Erzherzogin}|pwuv}{ }morgen{ }Abend eintrifft}{\lemma{\textnormal{\emph{Erzh. … eintrifft}}}\Cendnote{\textnormal{vermutlich
                     Leopold
                        Ferdinand von Österreich-Toskana\pwindex{Woelfling, Leopold Ferdinand Salvator 1868-12-02 – 1935-07-04@\textsc{Wölfling, Leopold Ferdinand Salvator} (1868-12-02 – 1935-07-04), \emph{Erzherzog/Erzherzogin}|pwuvk}, der in Schnitzlers{ }\emph{Tagebuch}\pwindex{Tagebuch@\emph{Tagebuch}|pwk} mit Bezug zu
                     Salten\pwindex{Salten, Felix 06.09.1869 – 08.10.1945@\textsc{Salten, Felix} (06.09.1869 – 08.10.1945), \emph{Schriftsteller/Schriftstellerin, Journalist/Journalistin, Chefredakteur/Chefredakteurin}|pwk} häufig nur
                     »Erzherzog« genannt wird, vgl. A. S.: \emph{Tagebuch}, 22. 6. 1898.}}}\label{K_L03279-1} – also nichts mit \label{K_L03279-2v}\edtext{Graz\oindex{Graz@\textbf{Graz}, \emph{A.ADM2}|pw}}{\lemma{\textnormal{\emph{Graz}}}\Cendnote{\textnormal{Siehe A. S.: \emph{Tagebuch}, 11. 7. 1898.
               }}}\label{K_L03279-2}, was uns\pwindex{Salten, Ottilie 07.03.1868 – 22.06.1942@\textsc{Salten, Ottilie} (07.03.1868 – 22.06.1942), \emph{Schauspieler/Schauspielerin}|pwv} sehr leid
               thut. Leben Sie wol und verbringen einen angenehmen Sommer. Briefe in die \label{K_L03279-3v}\edtext{Sensengasse\oindex{Sensengasse@\textbf{Sensengasse}, \emph{Straße (K.STR)}|pw}}{\lemma{\textnormal{\emph{Sensengasse}}}\Cendnote{\textnormal{In den »Veränderungen während des
                  Druckes« wird in \emph{Lehmann’s allgemeiner
                     Wohnungs-Anzeiger}\pwindex{Lehmann s Allgemeiner Wohnungs-Anzeiger@\emph{Lehmann’s Allgemeiner Wohnungs-Anzeiger}|pwk} für das Jahr 1898{ }Saltens\pwindex{Salten, Felix 06.09.1869 – 08.10.1945@\textsc{Salten, Felix} (06.09.1869 – 08.10.1945), \emph{Schriftsteller/Schriftstellerin, Journalist/Journalistin, Chefredakteur/Chefredakteurin}|pwk} neue Adresse mit
                     Sensengasse 5\oindex{Sensengasse@\textbf{Sensengasse}, \emph{Straße (K.STR)}|pwk} angegeben. Daraus ergibt sich, dass
                  er im Herbst 1897 hierhin übergesiedelt war. Ab
                     1. 8. 1898 wohnte er in der Wattmanngasse 11\oindex{Wattmanngasse@\textbf{Wattmanngasse}, \emph{Straße (K.STR)}|pwk}, siehe Felix Salten an Arthur Schnitzler, 30. 7. 1898.}}}\label{K_L03279-3}
               adressirt, erreichen mich immer.\pend
           
\pstart
           Auf Wiedersehen {\\[\baselineskip]}herzlichst {\\[\baselineskip]}Ihr {\\[\baselineskip]}\spacefill\mbox{Salten}\pend
           \leftskip=0em{}\selectlanguage{ngerman}\endnumbering\briefempfaengerindex{Schnitzler, Arthur@\textsc{Schnitzler, Arthur}!zzzSalten, Felix@\emph{von Felix Salten}!1898-07-103@{{[}10. 7. 1898{]}}|)be}\mylabel{L03279h}  \normalsize

\doendnotes{C}
\bigskip
\vfill

\clearpage

\footnotesize

\lohead{\textsc{register}}

% Definiere theindex-Environment komplett neu ohne reledmac
\makeatletter
\renewenvironment{theindex}{%
  \section*{\indexname}%
  \setlength{\parindent}{0pt}%
  \setlength{\parskip}{0pt plus 0.3pt}%
  \let\item\@idxitem
}{%
  \clearpage
}
\makeatother

\IfFileExists{\jobname-pw.ind}{\input{\jobname-pw.ind}}{}

\end{document}

      