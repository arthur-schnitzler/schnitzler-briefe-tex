%% latex-leseansicht-vorspann.tex
%% Vorspann für die Leseansicht.
%% Lädt die gemeinsame Datei latex-vorspann.tex mit nicht gesetztem Schalter.

\newif\ifkorrekturansicht
\korrekturansichtfalse

\input{../tex-inputs/latex-vorspann}


         
         \renewcommand{\erwaehntePersonen}{Personen: Ottilie Salten, Leopold Ferdinand Salvator Wölfling}
         \renewcommand{\erwaehnteOrte}{Orte: Graz, Sensengasse, Wien}
         \renewcommand{\erwaehnteWerke}{Werke: Tagebuch}
               \section[ Felix Salten an Arthur Schnitzler, {[}10. 7. 1898{]}]{ Felix Salten an Arthur Schnitzler, {[}10. 7. 1898{]}}\nopagebreak\mylabel{v}\rehead{ }\begin{ledgroupsized}[t]{13cm}\normalsize\beginnumbering \toendnotes[C]{\smallbreak\pagebreak[2]} \Standort{CUL, Schnitzler, B 89, A 2.}
\physDesc{Brief, 1 Blatt, 1 Seite, 286 Zeichen
\newline{}Handschrift: schwarze Tinte, lateinische Kurrent
\newline{}Schnitzler: mit Bleistift datiert: »10/7 98« 
\newline{}Ordnung: mit Bleistift von unbekannter Hand nummeriert: »103« }\toendnotes[C]{\smallbreak}\pstart
           \raggedleft{}{\pb}Sonntag{\\}Mittag.\pend
           \pstart
           Lieber Arthur, soeben erhalte ich die Nachricht, dass der \label{K_L03279-1v}\edtext{Erzh.\pwindex{Woelfling, Leopold Ferdinand Salvator 1868-12-02 – 1935-07-04@\textsc{Wölfling, Leopold Ferdinand Salvator} (1868-12-02 – 1935-07-04), \emph{Erzherzog}|pwuv}{ }morgen{ }Abend eintrifft}{\lemma{\textnormal{\emph{Erzh. … eintrifft}}}\Cendnote{\textnormal{vermutlich
                     Leopold
                     Ferdinand von Österreich-Toskana\pwindex{Woelfling, Leopold Ferdinand Salvator 1868-12-02 – 1935-07-04@\textsc{Wölfling, Leopold Ferdinand Salvator} (1868-12-02 – 1935-07-04), \emph{Erzherzog}|pwuvk}, der in Schnitzler\pwindex{Schnitzler, Arthur 15.05.1862 – 21.10.1931@\textsc{Schnitzler, Arthur} (15.05.1862 – 21.10.1931), \emph{Schriftsteller, Mediziner}|pwk}s \emph{Tagebuch}\pwindex{\textcolor{red}{\textsuperscript{XXXX1 indx}}!Tagebuch1981 – 2000@\strich\emph{Tagebuch} {[}Hrsg., 1981 – 2000{]}|pwk} mit Bezug zu
                     Salten\pwindex{Salten, Felix 06.09.1869 – 08.10.1945@\textsc{Salten, Felix} (06.09.1869 – 08.10.1945), \emph{Schriftsteller, Journalist}|pwk} häufig nur
                     »Erzherzog« genannt wird, vgl. A. S.: \emph{Tagebuch}, 22. 6. 1898}}}\label{K_L03279-1h} – also nichts mit \label{K_L03279-2v}\edtext{Graz\oindex{Graz@\textbf{Graz}|pw}}{\lemma{\textnormal{\emph{Graz}}}\Cendnote{\textnormal{siehe A. S.: \emph{Tagebuch}, 11. 7. 1898}}}\label{K_L03279-2h}, was uns\pwindex{Salten, Ottilie 07.03.1868 – 22.06.1942@\textsc{Salten, Ottilie} (07.03.1868 – 22.06.1942), \emph{Schauspielerin}|pwv} sehr leid
               thut. Leben Sie wol und verbringen einen angenehmen Sommer. Briefe in die \label{K_L03279-3v}\edtext{Sensengasse\oindex{Sensengasse@\textbf{Sensengasse}|pw}}{\lemma{\textnormal{\emph{Sensengasse}}}\Cendnote{\textnormal{In den »Veränderungen während des
                  Druckes« wird in \emph{Lehmann’s allgemeiner
                     Wohnungs-Anzeiger}\textcolor{red}{\textsuperscript{XXXX indx}} für das Jahr 1898{ }Salten\pwindex{Salten, Felix 06.09.1869 – 08.10.1945@\textsc{Salten, Felix} (06.09.1869 – 08.10.1945), \emph{Schriftsteller, Journalist}|pwk}s neue Adresse mit
                     Sensengasse 5\oindex{Sensengasse@\textbf{Sensengasse}|pwk} angegeben. Daraus ergibt sich, dass
                  er im Herbst 1897 hierhin übersiedelte. Ab
                     1. 8. 1898 wohnte er in der Wattmanngasse 11\oindex{XXXX Ortsangabe fehlt|pwk}, siehe Felix Salten an Arthur Schnitzler, 30. 7. 1898.}}}\label{K_L03279-3h}
               adressirt, erreichen mich immer.\pend
           \pstart
           Auf Wiedersehen {\\[\baselineskip]}herzlichst {\\[\baselineskip]}Ihr {\\[\baselineskip]}\spacefill\mbox{Salten}\pend
           \leftskip=0em{}
         
         \endnumbering\mylabel{h}\end{ledgroupsized}  \newcommand{\dateiname}{L03279}\newcommand{\titel}{Felix Salten an Arthur Schnitzler, [10. 7. 1898]}\newcommand{\editorInnen}{Martin Anton Müller und Laura Untner}%% latex-leseansicht-abspann.tex
%% Abspann für die Leseansicht.
%% Der Schalter \ifkorrekturansicht ist bereits durch den Vorspann gesetzt.

%% latex-abspann.tex
%% Gemeinsamer Abspann für Korrekturansicht und Leseansicht.
%% Setzt den Schalter \ifkorrekturansicht voraus (gesetzt in den
%% einbindenden Dateien latex-korrekturansicht-abspann.tex bzw.
%% latex-leseansicht-abspann.tex).
%% ---------------------------------------------------------------

\normalsize

% Das esempio-Environment wird nur in der Leseansicht benötigt
\ifkorrekturansicht\else
\newenvironment{esempio}[3]%
{
    \vspace{1.5ex}
    \rlap{\underline{#1}}
    \par
    \setlength{\parindent}{0cm}
    \nopagebreak
    \leftskip=#2cm
    \rightskip=#3cm
}
{
    \par
}
\fi

\doendnotes{C}
\bigskip
\vfill

\clearpage

\footnotesize

\ifkorrekturansicht
  \lohead{\textsc{register}}
\fi

% theindex-Environment neu definieren ohne reledmac
\makeatletter
\renewenvironment{theindex}{%
  \ifkorrekturansicht
    \section*{\indexname}%
  \else
    \subsubsection*{Index der erwähnten Entitäten}%
  \fi
  \setlength{\parindent}{0pt}%
  \setlength{\parskip}{0pt plus 0.3pt}%
  \let\item\@idxitem
}{%
  \ifkorrekturansicht\clearpage\fi
}
\makeatother

\IfFileExists{\jobname-pw.ind}{\input{\jobname-pw.ind}}{}

% Quellenangabe nur in der Leseansicht
\ifkorrekturansicht\else
% Fallback-Definitionen, falls die .tex-Datei \titel etc. nicht gesetzt hat
\providecommand{\titel}{}
\providecommand{\editorInnen}{}
\providecommand{\dateiname}{\jobname}

\vspace{3cm}

\vfill

\footnotesize
\textsc{Quelle}: \titel. Herausgegeben von {\editorInnen}. In: \emph{Arthur Schnitzler: Briefwechsel mit Autorinnen und Autoren}.
 Digitale Edition, https://schnitzler-briefe.acdh.oeaw.ac.at/{\dateiname}.html (Stand \today)
\fi

\end{document}


      