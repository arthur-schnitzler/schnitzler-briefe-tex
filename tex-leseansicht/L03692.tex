%% latex-leseansicht-vorspann.tex
%% Vorspann für die Leseansicht.
%% Lädt die gemeinsame Datei latex-vorspann.tex mit nicht gesetztem Schalter.

\newif\ifkorrekturansicht
\korrekturansichtfalse

\input{../tex-inputs/latex-vorspann}


\section[Stefan Zweig an Arthur Schnitzler, 6. 11. 1929]{L03692 Stefan Zweig an Arthur Schnitzler, 6. 11. 1929}
\nopagebreak\mylabel{L03692v}
\rehead{ }\normalsize\beginnumbering\briefempfaengerindex{Schnitzler, Arthur@\textsc{Schnitzler, Arthur}!zzzZweig, Stefan@\emph{von Stefan Zweig}!1929-11-062@{6. 11. 1929}|(be}
\toendnotes[C]{\smallbreak\pagebreak[2]}
\correspDesc{Versand  durch Stefan Zweig am 6. 11. 1929 in Salzburg
\newline{}Erhalt  durch Arthur Schnitzler im Zeitraum [7. 11. 1929
                  – 11. 11. 1929?] in Wien}\toendnotes[C]{\smallbreak}
\Standort{CUL, Schnitzler, B 118.}
\physDesc{Brief, 1 Blatt, 1 Seite, 1207 Zeichen
\newline{}Schreibmaschine
\newline{}Handschrift: blauer Buntstift, lateinische Kurrent (\noindent{}Unterschrift)
\newline{}Schnitzler: 1) mit rotem Buntstift zehn Unterstreichungen  2) mit rotem Buntstift eine seitliche Anstreichung}
\buchAbdrucke{\weitereDrucke{Stefan Zweig: \emph{Briefwechsel mit Hermann Bahr, Sigmund Freud, Rainer Maria
                        Rilke und Arthur Schnitzler}. Herausgegeben von Jeffrey B. Berlin, Hans-Ulrich Lindken und Donald A. Prater. Frankfurt am Main: \emph{S. Fischer} 1987, S. 447–448.} }\toendnotes[C]{\smallbreak}
\pstart
           {\pb}\textcolor{gray}{\textbf{SZ}}\hfill \textcolor{gray}{\textbf{SALZBURG\oindex{Salzburg@\textbf{Salzburg}, \emph{Verwaltungsgebiet}|pw}}}\pend
           
\pstart
           \raggedleft{}\textcolor{gray}{\textbf{KAPUZINERBERG 5\oindex{Paschinger Schlössl@\textbf{Paschinger Schlössl}, \emph{Wohngebäude}|pw}}}\pend
           
\pstart
           \raggedleft{}6. November 1929.\pend
           
\pstart\center{}Lieber, verehrter Herr Doktor!\pend\vspace{0.5em}
\pstart
           Ueber meine Vereinbarungen mit Spanien\oindex{Spanien@\textbf{Spanien}|pw} kann ich
               Sie genau informieren: ich habe meinen »Fouché\pwindex{Zweig, Stefan 28.\,11.\,1881 Wien – 23.\,2.\,1942 Petrópolis@\textsc{Zweig, Stefan} (28.\,11.\,1881 Wien – 23.\,2.\,1942 Petrópolis), \emph{Schriftsteller}!Fouché. Retrato di un Político@\strich\emph{Fouché. Retrato di un Político}|pw}«
               an A. del Vayos\pwindex{Álvarez del Vayo, Julio 9.\,2.\,1891 Villaviciosa de Odón – 3.\,5.\,1975 Genf@\textsc{Álvarez del Vayo, Julio} (9.\,2.\,1891 Villaviciosa de Odón – 3.\,5.\,1975 Genf), \emph{Schriftsteller, Politiker, Journalist}|pwu}{ }Verlag\orgindex{Espasa-Calpe@Espasa-Calpe|pwuv} zu 7 {\%} vergeben mit einem \label{K_L03692-1v}\edtext{\begin{otherlanguage}{french}à valoir\end{otherlanguage}}{\lemma{\textnormal{\emph{à valoir}}}\Cendnote{\textnormal{französisch: Vorschuss}}}\label{K_L03692-1} von 1000
                  frz\oindex{Frankreich@\textbf{Frankreich}|pw}. Frs., die sie sofort ausbezahlten, und
               Sie werden sicherlich zumindest dieselben Bedingungen bekommen.\pend
           
\pstart
           Dass man in Paris\oindex{Paris@\textbf{Paris}, \emph{Hauptstadt}|pw}{ }\label{K_L03692-2v}\edtext{im Kino eine Novelle\pwindex{\textcolor{red}{\textsuperscript{XXXX indx1}}!Angst@\strich\emph{Angst}|pwuv}}{\lemma{\textnormal{\emph{im Kino eine Novelle}}}\Cendnote{\textnormal{1928 wurde Zweigs\pwindex{Zweig, Stefan 28.\,11.\,1881 Wien – 23.\,2.\,1942 Petrópolis@\textsc{Zweig, Stefan} (28.\,11.\,1881 Wien – 23.\,2.\,1942 Petrópolis), \emph{Schriftsteller}|pwk} Novelle \emph{Angst}\pwindex{Zweig, Stefan 28.\,11.\,1881 Wien – 23.\,2.\,1942 Petrópolis@\textsc{Zweig, Stefan} (28.\,11.\,1881 Wien – 23.\,2.\,1942 Petrópolis), \emph{Schriftsteller}!Angst@\strich\emph{Angst}|pwk} verfilmt. Vermutlich ist dieser \emph{Film}\pwindex{\textcolor{red}{\textsuperscript{XXXX indx1}}!Angst@\strich\emph{Angst}|pwk} gemeint, vgl. XXXX Auszeichnungsfehler: Dokument L03734 nicht gefunden. 1929 erschien außerdem die
                  Verfilmung von \emph{Brief einer Unbekannten}\pwindex{Zweig, Stefan 28.\,11.\,1881 Wien – 23.\,2.\,1942 Petrópolis@\textsc{Zweig, Stefan} (28.\,11.\,1881 Wien – 23.\,2.\,1942 Petrópolis), \emph{Schriftsteller}!Brief einer Unbekannten@\strich\emph{Brief einer Unbekannten}|pwk} unter
                  dem Titel \emph{Narkose}\pwindex{Zweig, Stefan 28.\,11.\,1881 Wien – 23.\,2.\,1942 Petrópolis@\textsc{Zweig, Stefan} (28.\,11.\,1881 Wien – 23.\,2.\,1942 Petrópolis), \emph{Schriftsteller}!Narkose@\strich\emph{Narkose}|pwk}\pwindex{\textcolor{red}{\textsuperscript{XXXX indx1}}!Narkose@\strich\emph{Narkose}|pwk}\pwindex{\textcolor{red}{\textsuperscript{XXXX indx1}}!Narkose@\strich\emph{Narkose}|pwk}.}}}\label{K_L03692-2} von mir Ihnen
               zugeschrieben hat, betrachte ich als eine hohe Ehre. Die Leute werfen dort alles auf
               das rührendste durcheinander. Uebrigens ist »Fräulein Else\pwindex{Schnitzler, Arthur 15.\,5.\,1862 Wien – 21.\,10.\,1931 ebd.@\textsc{Schnitzler, Arthur} (15.\,5.\,1862 Wien – 21.\,10.\,1931 ebd.), \emph{Schriftsteller, Mediziner}!Madmoiselle Else@\strich\emph{Madmoiselle Else}|pw}« dort ein grosser Erfolg, Stock\orgindex{Éditions Stock@Éditions Stock|pw} bringt, wie ich höre, eine neue Auflage und erwartet sich sehr viel,
               wenn der Film\pwindex{\textcolor{red}{\textsuperscript{XXXX indx1}}!Fräulein Else@\strich\emph{Fräulein Else}|pw} abrollt. Wichtig ist nur, einmal
               in Paris\oindex{Paris@\textbf{Paris}, \emph{Hauptstadt}|pw} ein Theaterstück durchzusetzen. Man ist
               jetzt in Frankreich\oindex{Frankreich@\textbf{Frankreich}|pw} dem Ausländer viel offener
               als vordem und, während Oesterreich\oindex{Österreich-Ungarn@\textbf{Österreich-Ungarn}|pw} herrlich in
               die Alpenländerei hineinmarschiert, beginnt dort der europäische\oindex{Europa@\textbf{Europa}|pw} Gedanke immer selbstverständlicher zu werden. Ich habe mich in
                  Paris\oindex{Paris@\textbf{Paris}, \emph{Hauptstadt}|pw} ungemein wohl gefühlt und wundere mich
               eigentlich, dass Sie sich niemals entschlossen haben, einmal dort einen Wintermonat
               zu verbringen. Viele Freunde Ihrer Bücher erwarten Sie und besonders Frédéric Lefèvre\pwindex{Lefèvre, Frédéric 7.\,5.\,1889 Izé – 11.\,9.\,1949 Paris@\textsc{Lefèvre, Frédéric} (7.\,5.\,1889 Izé – 11.\,9.\,1949 Paris), \emph{Schriftsteller, Journalist, Literaturkritiker}|pw} mit seinen \label{K_L03692-3v}\edtext{»heures avec {\dotsfour}«}{\lemma{\textnormal{\emph{»heures avec «}}}\Cendnote{\textnormal{Der Literaturkritiker Frédéric Lefèvre\pwindex{Lefèvre, Frédéric 7.\,5.\,1889 Izé – 11.\,9.\,1949 Paris@\textsc{Lefèvre, Frédéric} (7.\,5.\,1889 Izé – 11.\,9.\,1949 Paris), \emph{Schriftsteller, Journalist, Literaturkritiker}|pwk} begründete
                     1922 in der Zeitschrift \emph{Les
                     nouvelles littéraires}\pwindex{Nouvelles littéraires@\emph{Les Nouvelles littéraires}|pwk} mit der Serie »Une Heure avec {\dots}« ein neuartiges literaturkritisches Interviewformat, das er bis
                     1938 fortsetzte. Erst 1932{ }befragte\pwindex{Lefèvre, Frédéric 7.\,5.\,1889 Izé – 11.\,9.\,1949 Paris@\textsc{Lefèvre, Frédéric} (7.\,5.\,1889 Izé – 11.\,9.\,1949 Paris), \emph{Schriftsteller, Journalist, Literaturkritiker}!Une heure avec Stefan Zweig. Le rôle de l’intellectuel dans la crise actuelle@\strich\emph{Une heure avec Stefan Zweig. Le rôle de l’intellectuel dans la crise actuelle}|pwkv}
                  er Zweig\pwindex{Zweig, Stefan 28.\,11.\,1881 Wien – 23.\,2.\,1942 Petrópolis@\textsc{Zweig, Stefan} (28.\,11.\,1881 Wien – 23.\,2.\,1942 Petrópolis), \emph{Schriftsteller}|pwk}.}}}\label{K_L03692-3}\pend
           
\pstart
           Getreulichst{\\[\baselineskip]} Ihr {\\[\baselineskip]}\spacefill\mbox{{[}hs.:{]} Stefan Zweig}\pend
           \leftskip=0em{}
\pstart
           \noindent{}{[}ms.:{]} Herrn Dr. Arthur Schnitzler\pend
           
\pstart
           \uline{\so{Wien}}\pend
           \selectlanguage{ngerman}\endnumbering\briefempfaengerindex{Schnitzler, Arthur@\textsc{Schnitzler, Arthur}!zzzZweig, Stefan@\emph{von Stefan Zweig}!1929-11-062@{6. 11. 1929}|)be}\mylabel{L03692h}  \newcommand{\dateiname}{L03692}\newcommand{\titel}{Stefan Zweig an Arthur Schnitzler, 6. 11. 1929}\newcommand{\editorInnen}{Selma Jahnke und Martin Anton Müller}%% latex-leseansicht-abspann.tex
%% Abspann für die Leseansicht.
%% Der Schalter \ifkorrekturansicht ist bereits durch den Vorspann gesetzt.

%% latex-abspann.tex
%% Gemeinsamer Abspann für Korrekturansicht und Leseansicht.
%% Setzt den Schalter \ifkorrekturansicht voraus (gesetzt in den
%% einbindenden Dateien latex-korrekturansicht-abspann.tex bzw.
%% latex-leseansicht-abspann.tex).
%% ---------------------------------------------------------------

\normalsize

% Das esempio-Environment wird nur in der Leseansicht benötigt
\ifkorrekturansicht\else
\newenvironment{esempio}[3]%
{
    \vspace{1.5ex}
    \rlap{\underline{#1}}
    \par
    \setlength{\parindent}{0cm}
    \nopagebreak
    \leftskip=#2cm
    \rightskip=#3cm
}
{
    \par
}
\fi

\doendnotes{C}
\bigskip
\vfill

\clearpage

\footnotesize

\ifkorrekturansicht
  \lohead{\textsc{register}}
\fi

% theindex-Environment neu definieren ohne reledmac
\makeatletter
\renewenvironment{theindex}{%
  \ifkorrekturansicht
    \section*{\indexname}%
  \else
    \subsubsection*{Index der erwähnten Entitäten}%
  \fi
  \setlength{\parindent}{0pt}%
  \setlength{\parskip}{0pt plus 0.3pt}%
  \let\item\@idxitem
}{%
  \ifkorrekturansicht\clearpage\fi
}
\makeatother

\IfFileExists{\jobname-pw.ind}{\input{\jobname-pw.ind}}{}

% Quellenangabe nur in der Leseansicht
\ifkorrekturansicht\else
% Fallback-Definitionen, falls die .tex-Datei \titel etc. nicht gesetzt hat
\providecommand{\titel}{}
\providecommand{\editorInnen}{}
\providecommand{\dateiname}{\jobname}

\vspace{3cm}

\vfill

\footnotesize
\textsc{Quelle}: \titel. Herausgegeben von {\editorInnen}. In: \emph{Arthur Schnitzler: Briefwechsel mit Autorinnen und Autoren}.
 Digitale Edition, https://schnitzler-briefe.acdh.oeaw.ac.at/{\dateiname}.html (Stand \today)
\fi

\end{document}


