%% latex-korrekturansicht-vorspann.tex
%% Vorspann für die Korrekturansicht.
%% Lädt die gemeinsame Datei latex-vorspann.tex mit gesetztem Schalter.

\newif\ifkorrekturansicht
\korrekturansichttrue

\input{../tex-inputs/latex-vorspann}


\section[Stefan Zweig an Arthur Schnitzler, 6. 11. 1929]{L03692 Stefan Zweig an Arthur Schnitzler, 6. 11. 1929}
\nopagebreak\mylabel{L03692v}
\rehead{ }\normalsize\beginnumbering\briefempfaengerindex{Schnitzler, Arthur@\textsc{Schnitzler, Arthur}!zzzZweig, Stefan@\emph{von Stefan Zweig}!1929-11-062@{6. 11. 1929}|(be}
\toendnotes[C]{\smallbreak\pagebreak[2]}\Standort{CUL, Schnitzler, B 118.}
\physDesc{Brief, 1 Blatt, 1 Seite, 1207 Zeichen
\newline{}Schreibmaschine
\newline{}Handschrift: blauer Buntstift, lateinische Kurrent (\noindent{}Unterschrift)
\newline{}Schnitzler: mit rotem Buntstift zehn Unterstreichungen, eine Anstreichung }
\buchAbdrucke{\weitereDrucke{Stefan Zweig: \emph{Briefwechsel mit Hermann Bahr, Sigmund Freud, Rainer Maria
                        Rilke und Arthur Schnitzler}. Frankfurt am Main: \emph{S. Fischer} 1987, S. 447–448.} }\toendnotes[C]{\smallbreak}
\pstart
           {\pb}\textcolor{gray}{\textbf{SZ}}\hfill \textcolor{gray}{\textbf{SALZBURG\oindex{Salzburg@\textbf{Salzburg}, \emph{A.ADM2}|pw}}}\pend
           
\pstart
           \raggedleft{}\textcolor{gray}{\textbf{KAPUZINERBERG 5\oindex{Paschinger Schloessl@\textbf{Paschinger Schlössl}, \emph{Wohngebäude (K.WHS)}|pw}}}\pend
           
\pstart
           \raggedleft{}6. November 1929.\pend
           
\pstart{}Lieber, verehrter Herr Doktor!\pend\vspace{0.5em}
\pstart
           Ueber meine Vereinbarungen mit Spanien\oindex{Spanien@\textbf{Spanien}, \emph{A.PCLI}|pw} kann ich
               Sie genau informieren: ich habe meinen »Fouché\pwindex{Fouche. Retrato di un Político@\emph{Fouché. Retrato di un Político}|pw}«
               an A. del Vayos\pwindex{Álvarez del Vayo, Julio 1891-02-09 – 1975-05-03@\textsc{Álvarez del Vayo, Julio} (1891-02-09 – 1975-05-03), \emph{Schriftsteller/Schriftstellerin, Politiker/Politikerin, Journalist/Journalistin}|pwu}{ }Verlag\orgindex{Espasa-Calpe@Espasa-Calpe|pwuv} zu 7 {\%} vergeben mit einem \label{K_L03692-1v}\edtext{\begin{otherlanguage}{french}à valoir\end{otherlanguage}}{\lemma{\textnormal{\emph{à valoir}}}\Cendnote{\textnormal{französisch: Vorschuss}}}\label{K_L03692-1} von 1000
                  frz\oindex{Frankreich@\textbf{Frankreich}, \emph{A.PCLI}|pw}. Frs., die Sie sofort ausbezahlten, und
               Sie werden sicherlich zumindest dieselben Bedingungen bekommen. \pend
           
\pstart
           Dass man in Paris\oindex{Paris@\textbf{Paris}, \emph{P.PPLC}|pw}{ }\label{K_L03692-2v}\edtext{im Kino eine Novelle\pwindex{Angst@\emph{Angst}|pwuv}}{\lemma{\textnormal{\emph{im Kino eine Novelle}}}\Cendnote{\textnormal{1928 wurde Zweigs\pwindex{Zweig, Stefan 28.11.1881 – 23.02.1942@\textsc{Zweig, Stefan} (28.11.1881 – 23.02.1942), \emph{Schriftsteller/Schriftstellerin}|pwk} Novelle \emph{Angst}\pwindex{Angst@\emph{Angst}|pwk} verfilmt. Vermutlich ist dieser \emph{Film}\pwindex{Angst@\emph{Angst}|pwk} gemeint, vgl. Arthur Schnitzler an Stefan Zweig, 4. 11. 1929. 1929
                  erschien außerdem die Verfilmung von \emph{Brief einer
                     Unbekannten}\pwindex{Brief einer Unbekannten@\emph{Brief einer Unbekannten}|pwk} unter dem Titel \emph{Narkose}\pwindex{Narkose@\emph{Narkose}|pwk}.}}}\label{K_L03692-2} von mir Ihnen zugeschrieben hat, betrachte ich als eine hohe
               Ehre. Die Leute werfen dort alles auf das rührendste durcheinander. Uebrigens ist
                  »Fräulein Else\pwindex{Madmoiselle Else@\emph{Madmoiselle Else}|pw}« dort ein grosser Erfolg, Stock\orgindex{Stock@Stock|pw} bringt, wie ich höre, eine neue Auflage und
               erwartet sich sehr viel, wenn der Film\pwindex{Fraeulein Else@\emph{Fräulein Else}|pw} abrollt.
               Wichtig ist nur, einmal in Paris\oindex{Paris@\textbf{Paris}, \emph{P.PPLC}|pw} ein Theaterstück
               durchzusetzen. Man ist jetzt in Frankreich\oindex{Frankreich@\textbf{Frankreich}, \emph{A.PCLI}|pw} dem
               Ausländer viel offener als vordem und, während Oesterreich\oindex{Oesterreich-Ungarn@\textbf{Österreich-Ungarn}, \emph{Land (A.LND)}|pw} herrlich in die Alpenländerei hineinmarschiert, beginnt dort der
                  europäische\oindex{Europa@\textbf{Europa}, \emph{Kontinent (A.KNT)}|pw} Gedanke immer selbstverständlicher
               zu werden. Ich habe mich in Paris\oindex{Paris@\textbf{Paris}, \emph{P.PPLC}|pw} ungemein wohl
               gefühlt und wundere mich eigentlich, dass Sie sich niemals entschlossen haben, einmal
               dort einen Wintermonat zu verbringen. Viele Freunde Ihrer Bücher erwarten Sie und
               besonders Frédéric Lefèvre\pwindex{Lefevre, Frederic 1889-05-07 – 1949-09-11@\textsc{Lefèvre, Frédéric} (1889-05-07 – 1949-09-11), \emph{Schriftsteller/Schriftstellerin, Journalist/Journalistin, Literaturkritiker/Literaturkritikerin}|pw} mit seinen \label{K_L03692-3v}\edtext{»heures avec ....«}{\lemma{\textnormal{\emph{»heures avec ....«}}}\Cendnote{\textnormal{Der Literaturkritiker Frédéric Lefèvre\pwindex{Lefevre, Frederic 1889-05-07 – 1949-09-11@\textsc{Lefèvre, Frédéric} (1889-05-07 – 1949-09-11), \emph{Schriftsteller/Schriftstellerin, Journalist/Journalistin, Literaturkritiker/Literaturkritikerin}|pwk} begründete 1922 in der Zeitschrift \emph{Les nouvelles littéraires}\pwindex{nouvelles litteraires@\emph{Les nouvelles littéraires}|pwk} mit der Serie »Une
                  Heure avec ....« ein neuartiges literaturkritisches Interviewformat, das er bis
                     1938 fortsetzte.}}}\label{K_L03692-3}\pend
           
\pstart
           Getreulichst{\\[\baselineskip]} Ihr {\\[\baselineskip]}\spacefill\mbox{{[}hs.:{]} Stefan Zweig}\pend
           \leftskip=0em{}
\pstart
           \noindent{}Herrn Dr. Arthur Schnitzler{\\}\uline{Wien}\pend
           \selectlanguage{ngerman}\endnumbering\briefempfaengerindex{Schnitzler, Arthur@\textsc{Schnitzler, Arthur}!zzzZweig, Stefan@\emph{von Stefan Zweig}!1929-11-062@{6. 11. 1929}|)be}\mylabel{L03692h}
\begin{anhang}
\end{anhang}\normalsize

\doendnotes{C}
\bigskip
\vfill

\clearpage

\footnotesize

\lohead{\textsc{register}}

% Definiere theindex-Environment komplett neu ohne reledmac
\makeatletter
\renewenvironment{theindex}{%
  \section*{\indexname}%
  \setlength{\parindent}{0pt}%
  \setlength{\parskip}{0pt plus 0.3pt}%
  \let\item\@idxitem
}{%
  \clearpage
}
\makeatother

\IfFileExists{\jobname-pw.ind}{\input{\jobname-pw.ind}}{}

\end{document}

      