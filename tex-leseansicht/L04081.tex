%% latex-leseansicht-vorspann.tex
%% Vorspann für die Leseansicht.
%% Lädt die gemeinsame Datei latex-vorspann.tex mit nicht gesetztem Schalter.

\newif\ifkorrekturansicht
\korrekturansichtfalse

\input{../tex-inputs/latex-vorspann}


\section[Arthur Schnitzler an Gustav Schwarzkopf, 30. 7. 1900]{L04081 Arthur Schnitzler an Gustav Schwarzkopf, 30. 7. 1900}
\nopagebreak\mylabel{L04081v}
\rehead{ }\normalsize\beginnumbering\briefempfaengerindex{Schwarzkopf, Gustav@\textsc{Schwarzkopf, Gustav}!zzzSchnitzler, Arthur@\emph{von Arthur Schnitzler}!1900-07-302@{30. 7. 1900}|(be}
\toendnotes[C]{\smallbreak\pagebreak[2]}
\correspDesc{Versand  durch Arthur Schnitzler am 30. 7. 1900 in Bad Aussee
\newline{}Erhalt  durch Gustav Schwarzkopf im Zeitraum [31. 7. 1900 – 4. 8. 1900?] in Wien}\toendnotes[C]{\smallbreak}
\Standort{CUL, Schnitzler, B 96.}
\physDesc{Brief, 1 Blatt, 3 Seiten, 735 Zeichen
\newline{}Handschrift: Bleistift, deutsche Kurrent}
\pstart
           \raggedleft{}{\pb}\textsc{Aussee\oindex{Bad Aussee@\textbf{Bad Aussee}, \emph{Hauptstadt}|pw}}{ }30/7 900\pend
           \vspace{0.5em}
\pstart
           lieber Guſtav, heute wurde folgender Plan feſtgeſtellt.\pend
           
\pstart
           12. August
             (So{\geminationn}tag) Salzburg\oindex{Salzburg@\textbf{Salzburg}, \emph{Verwaltungsgebiet}|pw}\pend
           
\pstart
           13.{ }Salzburg\oindex{Salzburg@\textbf{Salzburg}, \emph{Verwaltungsgebiet}|pw}.\pend
           
\pstart
           14.{ }Salzburg\oindex{Salzburg@\textbf{Salzburg}, \emph{Verwaltungsgebiet}|pw} – Zell
                  am See\oindex{Zell am See@\textbf{Zell am See}, \emph{Hauptstadt}|pw}{ }\introOben{}(Bahn.)\introOben{}\pend
           
\pstart
           Ausflug auf den Moserboden\oindex{Berghaus Moserboden@\textbf{Berghaus Moserboden}, \emph{Pension}|pw}.\pend
           
\pstart
           15.{ }Moserboden\oindex{Berghaus Moserboden@\textbf{Berghaus Moserboden}, \emph{Pension}|pw}{ }Zell am See\oindex{Zell am See@\textbf{Zell am See}, \emph{Hauptstadt}|pw} – Innsbruck\oindex{Innsbruck@\textbf{Innsbruck}, \emph{Verwaltungsgebiet}|pw} (Bahn.)\pend
           
\pstart
           16. Innsbruck\oindex{Innsbruck@\textbf{Innsbruck}, \emph{Verwaltungsgebiet}|pw} – Bludenz\oindex{Bludenz@\textbf{Bludenz}, \emph{Hauptstadt}|pw} (Bahn) – Schruns\oindex{Schruns@\textbf{Schruns}, \emph{Verwaltungsgebiet}|pw} (Poſt)\pend
           
\pstart
           {\pb}17.{ }18. Aufenthalt in Schruns\oindex{Schruns@\textbf{Schruns}, \emph{Verwaltungsgebiet}|pw}. (event. zu verlängern)\pend
           
\pstart
           19.{ }Beginn den Fußtour, die etwa
                  22.{ }Pontreſina\oindex{Pontresina@\textbf{Pontresina}|pw}{ }\textcolor{gray}{recht}, von Pontreſina\oindex{Pontresina@\textbf{Pontresina}|pw}
               entweder nach Bormio\oindex{Bormio@\textbf{Bormio}, \emph{Hauptstadt}|pw}, dann Trafoi\oindex{Trafoi@\textbf{Trafoi}|pw}, Sulden\oindex{Solda@\textbf{Solda}|pw}, oder Bormio\oindex{Bormio@\textbf{Bormio}, \emph{Hauptstadt}|pw} – nach Italien\oindex{Italien@\textbf{Italien}|pw} (\textsc{Lago d’Iseo\oindex{Iseosee@\textbf{Iseosee}, \emph{See}|pw}}) oder von Pontreſina\oindex{Pontresina@\textbf{Pontresina}|pw} in die Schweiz\oindex{Schweiz@\textbf{Schweiz}|pw}. Ich hoffe, daſs Sie mindeſtens vom
                  12.bis 19. mitthun. {\pb}Aber überlegen Sie, ob Sie ſich nicht
               doch zu der Fußpartie entſchließen. Schreiben Sie mir bitte nach Iſchl\oindex{Bad Ischl@\textbf{Bad Ischl}|pw}, \textsc{Rudolfshöhe\oindex{Hotel und Pension Rudolfshöhe (Leopold Petter)@\textbf{Hotel und Pension Rudolfshöhe (Leopold Petter)}, \emph{Hotel}|pw}}, wohin ich morgen abſegle.\pend
           
\pstart
           Herzlichſt Ihr{\\[\baselineskip]}\spacefill\mbox{Arthur.}\pend
           \leftskip=0em{}
\pstart
           \noindent{}Könnten Sie nicht auch nach Iſchl\oindex{Bad Ischl@\textbf{Bad Ischl}|pw} kommen?\pend
           \selectlanguage{ngerman}\endnumbering\briefempfaengerindex{Schwarzkopf, Gustav@\textsc{Schwarzkopf, Gustav}!zzzSchnitzler, Arthur@\emph{von Arthur Schnitzler}!1900-07-302@{30. 7. 1900}|)be}\mylabel{L04081h}
\begin{anhang}
\end{anhang}\newcommand{\dateiname}{L04081}\newcommand{\titel}{Arthur Schnitzler an Gustav Schwarzkopf, 30. 7. 1900}\newcommand{\editorInnen}{Herausgegeben von Jahnke, SelmaMüller, Martin Anton}%% latex-leseansicht-abspann.tex
%% Abspann für die Leseansicht.
%% Der Schalter \ifkorrekturansicht ist bereits durch den Vorspann gesetzt.

%% latex-abspann.tex
%% Gemeinsamer Abspann für Korrekturansicht und Leseansicht.
%% Setzt den Schalter \ifkorrekturansicht voraus (gesetzt in den
%% einbindenden Dateien latex-korrekturansicht-abspann.tex bzw.
%% latex-leseansicht-abspann.tex).
%% ---------------------------------------------------------------

\normalsize

% Das esempio-Environment wird nur in der Leseansicht benötigt
\ifkorrekturansicht\else
\newenvironment{esempio}[3]%
{
    \vspace{1.5ex}
    \rlap{\underline{#1}}
    \par
    \setlength{\parindent}{0cm}
    \nopagebreak
    \leftskip=#2cm
    \rightskip=#3cm
}
{
    \par
}
\fi

\doendnotes{C}
\bigskip
\vfill

\clearpage

\footnotesize

\ifkorrekturansicht
  \lohead{\textsc{register}}
\fi

% theindex-Environment neu definieren ohne reledmac
\makeatletter
\renewenvironment{theindex}{%
  \ifkorrekturansicht
    \section*{\indexname}%
  \else
    \subsubsection*{Index der erwähnten Entitäten}%
  \fi
  \setlength{\parindent}{0pt}%
  \setlength{\parskip}{0pt plus 0.3pt}%
  \let\item\@idxitem
}{%
  \ifkorrekturansicht\clearpage\fi
}
\makeatother

\IfFileExists{\jobname-pw.ind}{\input{\jobname-pw.ind}}{}

% Quellenangabe nur in der Leseansicht
\ifkorrekturansicht\else
% Fallback-Definitionen, falls die .tex-Datei \titel etc. nicht gesetzt hat
\providecommand{\titel}{}
\providecommand{\editorInnen}{}
\providecommand{\dateiname}{\jobname}

\vspace{3cm}

\vfill

\footnotesize
\textsc{Quelle}: \titel. Herausgegeben von {\editorInnen}. In: \emph{Arthur Schnitzler: Briefwechsel mit Autorinnen und Autoren}.
 Digitale Edition, https://schnitzler-briefe.acdh.oeaw.ac.at/{\dateiname}.html (Stand \today)
\fi

\end{document}


