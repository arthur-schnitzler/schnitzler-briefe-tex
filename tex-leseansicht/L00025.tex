%% latex-korrekturansicht-vorspann.tex
%% Vorspann für die Korrekturansicht.
%% Lädt die gemeinsame Datei latex-vorspann.tex mit gesetztem Schalter.

\newif\ifkorrekturansicht
\korrekturansichttrue

\input{../tex-inputs/latex-vorspann}


\section[Arthur Schnitzler an Hugo von Hofmannsthal, 27. 7. 1891]{L00025 Arthur Schnitzler an Hugo von Hofmannsthal, 27. 7. 1891}
\nopagebreak\mylabel{L00025v}
\rehead{ }\normalsize\beginnumbering\briefempfaengerindex{Hofmannsthal, Hugo von@\textsc{Hofmannsthal, Hugo von}!zzzSchnitzler, Arthur@\emph{von Arthur Schnitzler}!1891-07-271@{27. 7. 1891}|(be}
\toendnotes[C]{\smallbreak\pagebreak[2]}\Standort{FDH, Hs-30885,9.}
\physDesc{Brief, 2 Blätter, 6 Seiten, 2301 Zeichen
\newline{}Handschrift: schwarze Tinte, deutsche Kurrent}
\buchAbdrucke{\weitereDrucke{1) Hugo von Hofmannsthal, Arthur Schnitzler: \emph{Briefwechsel}. Frankfurt am Main: \emph{S. Fischer} 1964, S. 9–10.} \weitereDrucke{2) Arthur Schnitzler: \emph{Briefe 1875–1912}. Frankfurt am Main: \emph{S. Fischer} 1981, S. 119–120.} }\toendnotes[C]{\smallbreak}
\pstart
           \raggedleft{}{\pb}Wien\oindex{Wien@\textbf{Wien}, \emph{A.ADM2}|pw}, 27. Juli
                  1891.\pend
           \vspace{0.5em}
\pstart
           Verehrter Freund, eine \label{K_L00025-1v}\edtext{Karte}{\lemma{\textnormal{\emph{Karte}}}\Cendnote{\textnormal{Siehe Paul Goldmann an Arthur Schnitzler, 25. 7. 1891.
               }}}\label{K_L00025-1}, die ich eben von Paul Goldma{\geminationn}\pwindex{Goldmann, Paul 31.01.1865 – 25.09.1935@\textsc{Goldmann, Paul} (31.01.1865 – 25.09.1935), \emph{Schriftsteller/Schriftstellerin, Journalist/Journalistin}|pw} beko{\geminationm}e, eri{\geminationn}ert mich,
               wie üblich es iſt, Briefe zu beantworten, und wie ich Ihnen ſchon längſt hätte
               ſchreiben ſollen, ja, wie ich Ihnen ſogar hätte ſchreiben wollen, we{\geminationn} mein Gehirn nicht die ganze letzte Zeit über todte
               Stellen hätte hinwegko{\geminationm}en müſſen. In zweierlei Perioden
               bietet einem das Leben was, in der der Anfänge, wo tauſenderlei über einen ko{\geminationm}t, und man {\pb}jeden Tag ein
               neues Blatt herzunehmen hat und nur drauflos zu begi{\geminationn}en.
                  Da{\geminationn} die andre Periode, wo man das Bedürfnis des
               Abſchließens hat – wo man die alten Blätter ni{\geminationm}t und
               einem alle möglichen Worte, Punkte u Gedankenſtriche einfallen, – die man verg\substVorne{}\textsuperscript{eſſen}\substDazwischen{}aß\substHinten{}{ }\strikeout{hat}. Die erſte Periode: wo man ſich an ſich
               berauſcht, die zweite: wo man ſich an ſich beruhigt. Ich bin jetzt in keiner von
               beiden, alſo arm und blöd. Nervös, ſehr. Beer-{\pb}Hofma{\geminationn}\pwindex{Beer-Hofmann, Richard 1866-07-11 – 1945-09-26@\textsc{Beer-Hofmann, Richard} (1866-07-11 – 1945-09-26), \emph{Schriftsteller/Schriftstellerin}|pw} iſt auch ſchon weg, das wiſſen Sie ja. – In die \textsc{Kugel}\oindex{Cafe Kugel@\textbf{Café Kugel}, \emph{Kaffeehaus (K.KAF)}|pw} ko{\geminationm} ich ſelten, es waren ſchon ein paar Ausſchuſsſitzungen\orgindex{»Freie Buehne« Verein fuer moderne Literatur@»Freie Bühne« Verein für moderne Literatur|pwv};
                  Specialcomités{ }ſind gew\textcolor{gray}{ä}hlt worden; ich ſitze im Theatercomité
               zuſammen mit \textsc{Pernerstorfer}\pwindex{Pernerstorfer, Engelbert 27.04.1850 – 06.01.1918@\textsc{Pernerstorfer, Engelbert} (27.04.1850 – 06.01.1918), \emph{Politiker/Politikerin, Journalist/Journalistin}|pw}, \textsc{Wengraf}\pwindex{Wengraf, Edmund 09.01.1860 – 08.12.1933@\textsc{Wengraf, Edmund} (09.01.1860 – 08.12.1933), \emph{Schriftsteller/Schriftstellerin, Journalist/Journalistin, Kaufmann/Kauffrau}|pw}, \textsc{Osten}\pwindex{Osten, Heinrich 16.08.1855 – 01.08.1931@\textsc{Osten, Heinrich} (16.08.1855 – 01.08.1931), \emph{Schriftsteller/Schriftstellerin, Journalist/Journalistin}|pw}, \textsc{Kafka}\pwindex{Kafka, Eduard Michael 11.03.1869 – 06.08.1893@\textsc{Kafka, Eduard Michael} (11.03.1869 – 06.08.1893), \emph{Redakteur/Redakteurin}|pw}, \textsc{Kulka}\pwindex{Kulka, Julius 25.09.1865 – 22.09.1893@\textsc{Kulka, Julius} (25.09.1865 – 22.09.1893), \emph{Rechtsanwalt/Rechtsanwältin}|pw}. –\hspace*{2.5em}Bis jetzt iſt noch nicht viel geſcheidtes
                  herausgeko{\geminationm}en. – Mit \textsc{Salten}\pwindex{Salten, Felix 06.09.1869 – 08.10.1945@\textsc{Salten, Felix} (06.09.1869 – 08.10.1945), \emph{Schriftsteller/Schriftstellerin, Journalist/Journalistin, Chefredakteur/Chefredakteurin}|pw} bin ich viel zuſa{\geminationm}en, auch auf dem »Land« des
               Abends. \textsc{Burckhard}\pwindex{Burckhard, Max Eugen 14.07.1854 – 16.03.1912@\textsc{Burckhard, Max Eugen} (14.07.1854 – 16.03.1912), \emph{Schriftsteller/Schriftstellerin, Rechtswissenschaftler/Rechtswissenschaftlerin, Theaterleiter/Theaterleiterin}|pw} hat mir den Alkandi\pwindex{Alkandi s Lied@\emph{Alkandi’s Lied}|pw} mit einigen
               ſchmeichelhaften Worten {\pb}zurückgeſandt – ich hab’ ihn
                  angeno{\geminationm}en. Mein Stück\pwindex{Maerchen. Schauspiel in drei Aufzuegen@\emph{Das Märchen. Schauspiel in drei Aufzügen}|pwv} ruht und iſt mir zuwider. – Wie geht es Ihrem hi{\geminationm}elblauen Einakter\pwindex{Gestern. Dramatische Studie in einem Akt in Versen@\emph{Gestern. Dramatische Studie in einem Akt in Versen}|pwv}? Und wollen Sie mir nichts von Ihren Sachen ſchicken? Sie
               würden mir eine wirkliche Freude machen, ſeien Sie erſter oder ſiebenter Grad! –
               Geleſen wird mancherlei \textsc{Burckhardt}\pwindex{Burckhardt, Jacob 25.05.1818 – 08.08.1897@\textsc{Burckhardt, Jacob} (25.05.1818 – 08.08.1897), \emph{Historiker/Historikerin, Kunsthistoriker/Kunsthistorikerin}|pw}, Cultur der Renaiſſance\pwindex{Cultur der Renaissance in Italien. Ein Versuch@\emph{Die Cultur der Renaissance in Italien. Ein Versuch}|pw}, \textsc{Goethe}\pwindex{Goethe, Johann Wolfgang von 1749-08-28 – 1832-03-22@\textsc{Goethe, Johann Wolfgang von} (1749-08-28 – 1832-03-22), \emph{Schriftsteller/Schriftstellerin}|pw}, Annalen\pwindex{Tag- und Jahreshefte@\emph{Tag- und Jahreshefte}|pw}, \textsc{Lessings}\pwindex{Lessing, Gotthold Ephraim 22.01.1729 – 15.02.1781@\textsc{Lessing, Gotthold Ephraim} (22.01.1729 – 15.02.1781), \emph{Schriftsteller/Schriftstellerin, Kritiker/Kritikerin, Philosoph/Philosophin}|pw}{ }Drama\strikeout{turgie}
                  Entwürfe\pwindex{Vierundfunfzig zum Theil noch ungedruckte Dramatische Entwuerfe und Plaene Gotthold Ephraim Lessings@\emph{Vierundfunfzig zum Theil noch ungedruckte Dramatische Entwürfe und Pläne Gotthold Ephraim Lessings}|pw}, \textsc{Jonas Lie}\pwindex{Lie, Jonas 06.11.1833 – 05.07.1908@\textsc{Lie, Jonas} (06.11.1833 – 05.07.1908), \emph{Schriftsteller/Schriftstellerin}|pw}{ }\textsc{etc.} Beſonders \textsc{Nietz}’ſche\pwindex{Nietzsche, Friedrich 15.10.1844 – 25.08.1900@\textsc{Nietzsche, Friedrich} (15.10.1844 – 25.08.1900), \emph{Schriftsteller/Schriftstellerin, Philosoph/Philosophin}|pw} – zuletzt {\pb}hat mich ſein Schluſscapitel und das Schlußgedicht\pwindex{Nachgesang. Aus den hohen Bergen@\emph{Nachgesang. Aus den hohen Bergen}|pwv} zu \textsc{Jenseits von Gut u Böse}\pwindex{Jenseits von Gut und Boese@\emph{Jenseits von Gut und Böse}|pw} ergriffen. – Eri{\geminationn}ern Sie ſich? \textsc{Nietz}’ſche\pwindex{Nietzsche, Friedrich 15.10.1844 – 25.08.1900@\textsc{Nietzsche, Friedrich} (15.10.1844 – 25.08.1900), \emph{Schriftsteller/Schriftstellerin, Philosoph/Philosophin}|pw}{ }Sentimentalität! – Weinender Marmor! Stellen, die
               ſogar auf Weiber wirken, ohne daß man den Stellen oder den Weibern bös werden
               müßte. – Werden Sie mir bald wieder ſchreiben? Arbeiten Sie viel? Erleben {\pb}Sie was? Spielen Sie aber lieber \textsc{lawn-tennis}, ſtatt ſich zu verlieben, oder nehmen Sie wenigſtens, we{\geminationn} beides über Sie geko{\geminationm}en,
               das erſtere ernſter.\pend
           
\pstart
           Herzlichen Gruſs. Den Ihrigen meine Empfehlungen. Iſt \textsc{Schwarzkopf}\pwindex{Schwarzkopf, Gustav 07.11.1853 – 13.11.1939@\textsc{Schwarzkopf, Gustav} (07.11.1853 – 13.11.1939), \emph{Schriftsteller/Schriftstellerin}|pw}{ }ſchon bei Ihnen? Ich ſah ihn ſchon Wochen lang
               nicht. –\pend
           
\pstart
           Alſo nochmals, viele Grüße{\\[\baselineskip]}Ihr \spacefill\mbox{Arthur Sch}\pend
           \leftskip=0em{}\selectlanguage{ngerman}\endnumbering\briefempfaengerindex{Hofmannsthal, Hugo von@\textsc{Hofmannsthal, Hugo von}!zzzSchnitzler, Arthur@\emph{von Arthur Schnitzler}!1891-07-271@{27. 7. 1891}|)be}\mylabel{L00025h}  \normalsize

\doendnotes{C}
\bigskip
\vfill

\clearpage

\footnotesize

\lohead{\textsc{register}}

% Definiere theindex-Environment komplett neu ohne reledmac
\makeatletter
\renewenvironment{theindex}{%
  \section*{\indexname}%
  \setlength{\parindent}{0pt}%
  \setlength{\parskip}{0pt plus 0.3pt}%
  \let\item\@idxitem
}{%
  \clearpage
}
\makeatother

\IfFileExists{\jobname-pw.ind}{\input{\jobname-pw.ind}}{}

\end{document}

      