%% latex-leseansicht-vorspann.tex
%% Vorspann für die Leseansicht.
%% Lädt die gemeinsame Datei latex-vorspann.tex mit nicht gesetztem Schalter.

\newif\ifkorrekturansicht
\korrekturansichtfalse

\input{../tex-inputs/latex-vorspann}


         
         \newcommand{\erwaehntePersonen}{Personen: Richard Beer-Hofmann, Max Eugen Burckhard, Jacob Burckhardt, Johann Wolfgang von Goethe, Paul Goldmann, Hugo von Hofmannsthal, Eduard Michael Kafka, Julius Kulka, Gotthold Ephraim Lessing, Jonas Lie, Friedrich Nietzsche, Heinrich Osten, Engelbert Pernerstorfer, Felix Salten, Gustav Schwarzkopf, Edmund Wengraf}
         \newcommand{\erwaehnteInstitutionen}{Institutionen: »Freie Bühne« Verein für moderne Literatur}
         \newcommand{\erwaehnteOrte}{Orte: Café Kugel, Wien}
         \newcommand{\erwaehnteWerke}{Werke: Alkandi’s Lied, Das Märchen. Schauspiel in drei Aufzügen, Die Cultur der Renaissance in Italien. Ein Versuch, Dramatische Entwürfe und Pläne, Gestern. Dramatische Studie in einem Akt in Versen, Jenseits von Gut und Böse, Nachgesang. Aus den hohen Bergen, Tag- und Jahreshefte}
               \section[Arthur Schnitzler an Hugo von Hofmannsthal, 27. 7. 1891]{ Arthur Schnitzler an Hugo von Hofmannsthal, 27. 7. 1891}\nopagebreak\mylabel{v}\rehead{ }\begin{ledgroupsized}[t]{13cm}\normalsize\beginnumbering \toendnotes[C]{\smallbreak\pagebreak[2]} \Standort{FDH, Hs-30885,9.}
\physDesc{Brief, 2 Blätter, 6 Seiten
\newline{}Handschrift: schwarze Tinte, deutsche Kurrent}\buchAbdrucke{\weitereDrucke{1) Hugo von Hofmannsthal, Arthur Schnitzler: \emph{Briefwechsel}. Hg. Therese Nickl und Heinrich Schnitzler. Frankfurt am Main: \emph{S. Fischer} 1964, S. 9–10.} \weitereDrucke{2) Arthur Schnitzler: \emph{Briefe 1875–1912}. Hg. Therese Nickl und Heinrich Schnitzler. Frankfurt am Main: \emph{S. Fischer} 1981, S. 119–120.} }\toendnotes[C]{\smallbreak}\pstart
           \raggedleft{}{\pb}Wien\oindex{Wien@\textbf{Wien}|pw}, 27. Juli
                  1891.\pend
           \pstart
           Verehrter Freund, eine \label{K_L00025-3v}\edtext{Karte}{\lemma{\textnormal{\emph{Karte}}}\Cendnote{\textnormal{siehe Paul Goldmann an Arthur Schnitzler, 25. 7. 1891}}}\label{K_L00025-3h}, die ich eben von Paul Goldma{\geminationn}\pwindex{Goldmann, Paul 31.01.1865 – 25.09.1935@\textsc{Goldmann, Paul} (31.01.1865 – 25.09.1935), \emph{Schriftsteller, Journalist}|pw} beko{\geminationm}e, eri{\geminationn}ert mich,
               wie üblich es iſt, Briefe zu beantworten, und wie ich Ihnen ſchon längſt hätte
               ſchreiben ſollen, ja, wie ich Ihnen ſogar hätte ſchreiben wollen, we{\geminationn} mein Gehirn nicht die ganze letzte Zeit über todte
               Stellen hätte hinwegko{\geminationm}en müſſen. In zweierlei Perioden
               bietet einem das Leben was, in der der Anfänge, wo tauſenderlei über einen ko{\geminationm}t, und man {\pb}jeden Tag ein
               neues Blatt herzunehmen hat und nur drauflos zu begi{\geminationn}en.
                  Da{\geminationn} die andre Periode, wo man das Bedürfnis des
               Abſchließens hat – wo man die alten Blätter ni{\geminationm}t und
               einem alle möglichen Worte, Punkte u Gedankenſtriche einfallen, – die man verg\substVorne{}\textsuperscript{eſſen}\substDazwischen{}aß\substHinten{}{ }\strikeout{hat}. Die erſte Periode: wo man ſich an ſich
               berauſcht, die zweite: wo man ſich an ſich beruhigt. Ich bin jetzt in keiner von
               beiden, alſo arm und blöd. Nervös, ſehr. Beer-{\pb}Hofma{\geminationn}\pwindex{Beer-Hofmann, Richard 1866-07-11 – 1945-09-26@\textsc{Beer-Hofmann, Richard} (1866-07-11 – 1945-09-26), \emph{Schriftsteller}|pw} iſt auch ſchon weg, das wiſſen Sie ja. – In die \textsc{Kugel}\oindex{Cafe Kugel@\textbf{Café Kugel}|pw} ko{\geminationm} ich ſelten, es waren ſchon ein paar Ausſchuſsſitzungen\orgindex{»Freie Buehne« Verein fuer moderne Literatur@»Freie Bühne« Verein für moderne Literatur|pwv};
                  Specialcomités{ }ſind gew\textcolor{gray}{ä}hlt worden; ich ſitze im Theatercomité
               zuſammen mit \textsc{Pernerstorfer}\pwindex{Pernerstorfer, Engelbert 27.04.1850 – 06.01.1918@\textsc{Pernerstorfer, Engelbert} (27.04.1850 – 06.01.1918), \emph{Politiker, Journalist}|pw}, \textsc{Wengraf}\pwindex{Wengraf, Edmund 09.01.1860 – 08.12.1933@\textsc{Wengraf, Edmund} (09.01.1860 – 08.12.1933), \emph{Journalist}|pw}, \textsc{Osten}\pwindex{Osten, Heinrich 16.08.1855 – 01.08.1931@\textsc{Osten, Heinrich} (16.08.1855 – 01.08.1931), \emph{Schriftsteller, Journalist}|pw}, \textsc{Kafka}\pwindex{Kafka, Eduard Michael 11.03.1869 – 06.08.1893@\textsc{Kafka, Eduard Michael} (11.03.1869 – 06.08.1893), \emph{Redakteur}|pw}, \textsc{Kulka}\pwindex{Kulka, Julius 25.09.1865 – 22.09.1893@\textsc{Kulka, Julius} (25.09.1865 – 22.09.1893), \emph{Rechtsanwalt}|pw}. –\hspace*{2.5em}Bis jetzt iſt noch nicht viel geſcheidtes
                  herausgeko{\geminationm}en. – Mit \textsc{Salten}\pwindex{Salten, Felix 06.09.1869 – 08.10.1945@\textsc{Salten, Felix} (06.09.1869 – 08.10.1945), \emph{Schriftsteller, Journalist}|pw} bin ich viel zuſa{\geminationm}en, auch auf dem »Land« des
               Abends. \textsc{Burckhard}\pwindex{Burckhard, Max Eugen 14.07.1854 – 16.03.1912@\textsc{Burckhard, Max Eugen} (14.07.1854 – 16.03.1912), \emph{Schriftsteller, Rechtswissenschaftler, Theaterleiter}|pw} hat mir den Alkandi\pwindex{Schnitzler, Arthur 15.05.1862 – 21.10.1931@\textsc{Schnitzler, Arthur} (15.05.1862 – 21.10.1931), \emph{Schriftsteller, Mediziner}!Alkandi s Lied15.8.1890 – 1.9.1890@\strich\emph{Alkandi’s Lied} {[}15.8.1890 – 1.9.1890{]}|pw} mit einigen
               ſchmeichelhaften Worten {\pb}zurückgeſandt – ich hab’ ihn
                  angeno{\geminationm}en. Mein Stück\pwindex{Schnitzler, Arthur 15.05.1862 – 21.10.1931@\textsc{Schnitzler, Arthur} (15.05.1862 – 21.10.1931), \emph{Schriftsteller, Mediziner}!Maerchen. Schauspiel in drei Aufzuegen1893-12-01@\strich\emph{Das Märchen. Schauspiel in drei Aufzügen} {[}1893-12-01{]}|pwv} ruht und iſt mir zuwider. – Wie geht es Ihrem hi{\geminationm}elblauen Einakter\pwindex{Gestern. Dramatische Studie in einem Akt in Versen15. 10. 1891@\emph{Gestern. Dramatische Studie in einem Akt in Versen} {[}15. 10. 1891{]}|pwv}? Und wollen Sie mir nichts von Ihren Sachen ſchicken? Sie
               würden mir eine wirkliche Freude machen, ſeien Sie erſter oder ſiebenter Grad! –
               Geleſen wird mancherlei \textsc{Burckhardt}\pwindex{Burckhardt, Jacob 25.05.1818 – 08.08.1897@\textsc{Burckhardt, Jacob} (25.05.1818 – 08.08.1897), \emph{Historiker}|pw}, Cultur der Renaiſſance\pwindex{Burckhardt, Jacob 25.05.1818 – 08.08.1897@\textsc{Burckhardt, Jacob} (25.05.1818 – 08.08.1897), \emph{Historiker}!Cultur der Renaissance in Italien. Ein Versuch1860@\strich\emph{Die Cultur der Renaissance in Italien. Ein Versuch} {[}1860{]}|pw}, \textsc{Goethe}\pwindex{Goethe, Johann Wolfgang von 1749-08-28 – 1832-03-22@\textsc{Goethe, Johann Wolfgang von} (1749-08-28 – 1832-03-22), \emph{Schriftsteller}|pw}, Annalen\pwindex{Goethe, Johann Wolfgang von 1749-08-28 – 1832-03-22@\textsc{Goethe, Johann Wolfgang von} (1749-08-28 – 1832-03-22), \emph{Schriftsteller}!Tag- und Jahreshefte1830@\strich\emph{Tag- und Jahreshefte} {[}1830{]}|pw}, \textsc{Lessing}\pwindex{Lessing, Gotthold Ephraim 22.01.1729 – 15.02.1781@\textsc{Lessing, Gotthold Ephraim} (22.01.1729 – 15.02.1781), \emph{Schriftsteller, Bibliothekar}|pw}s Drama\strikeout{turgie}
                  Entwürfe\pwindex{Lessing, Gotthold Ephraim 22.01.1729 – 15.02.1781@\textsc{Lessing, Gotthold Ephraim} (22.01.1729 – 15.02.1781), \emph{Schriftsteller, Bibliothekar}!Dramatische Entwuerfe und PlaeneNone@\strich\emph{Dramatische Entwürfe und Pläne} {[}None{]}|pw}, \textsc{Jonas Lie}\pwindex{Lie, Jonas 06.11.1833 – 05.07.1908@\textsc{Lie, Jonas} (06.11.1833 – 05.07.1908), \emph{Schriftsteller}|pw}{ }\textsc{etc.} Beſonders \textsc{Nietz}’ſche\pwindex{Nietzsche, Friedrich 15.10.1844 – 25.08.1900@\textsc{Nietzsche, Friedrich} (15.10.1844 – 25.08.1900), \emph{Schriftsteller, Philosoph}|pw} – zuletzt {\pb}hat mich ſein Schluſscapitel und das Schlußgedicht\pwindex{Nietzsche, Friedrich 15.10.1844 – 25.08.1900@\textsc{Nietzsche, Friedrich} (15.10.1844 – 25.08.1900), \emph{Schriftsteller, Philosoph}!Nachgesang. Aus den hohen Bergen1886@\strich\emph{Nachgesang. Aus den hohen Bergen} {[}1886{]}|pwv} zu \textsc{Jenseits von Gut u Böse}\pwindex{Nietzsche, Friedrich 15.10.1844 – 25.08.1900@\textsc{Nietzsche, Friedrich} (15.10.1844 – 25.08.1900), \emph{Schriftsteller, Philosoph}!Jenseits von Gut und Boese1886@\strich\emph{Jenseits von Gut und Böse} {[}1886{]}|pw} ergriffen. – Eri{\geminationn}ern Sie ſich? \textsc{Nietz}’ſche\pwindex{Nietzsche, Friedrich 15.10.1844 – 25.08.1900@\textsc{Nietzsche, Friedrich} (15.10.1844 – 25.08.1900), \emph{Schriftsteller, Philosoph}|pw}{ }Sentimentalität! – Weinender Marmor! Stellen, die
               ſogar auf Weiber wirken, ohne daß man den Stellen oder den Weibern bös werden
               müßte. – Werden Sie mir bald wieder ſchreiben? Arbeiten Sie viel? Erleben {\pb}Sie was? Spielen Sie aber lieber \textsc{lawn-tennis}, ſtatt ſich zu verlieben, oder nehmen Sie wenigſtens, we{\geminationn} beides über Sie geko{\geminationm}en,
               das erſtere ernſter.\pend
           \pstart
           Herzlichen Gruſs. Den Ihrigen meine Empfehlungen. Iſt \textsc{Schwarzkopf}\pwindex{Schwarzkopf, Gustav 07.11.1853 – 13.11.1939@\textsc{Schwarzkopf, Gustav} (07.11.1853 – 13.11.1939), \emph{Schriftsteller}|pw}{ }ſchon bei Ihnen? Ich ſah ihn ſchon Wochen lang
               nicht. –\pend
           \pstart
           Alſo nochmals, viele Grüße{\\[\baselineskip]}Ihr \spacefill\mbox{Arthur Sch}\pend
           \leftskip=0em{}
         
         \endnumbering\mylabel{h}\end{ledgroupsized}  \newcommand{\dateiname}{L00025}\newcommand{\titel}{Arthur Schnitzler an Hugo von Hofmannsthal, 27. 7. 1891}\newcommand{\editorInnen}{Martin Anton Müller und Gerd-Hermann Susen}%% latex-leseansicht-abspann.tex
%% Abspann für die Leseansicht.
%% Der Schalter \ifkorrekturansicht ist bereits durch den Vorspann gesetzt.

%% latex-abspann.tex
%% Gemeinsamer Abspann für Korrekturansicht und Leseansicht.
%% Setzt den Schalter \ifkorrekturansicht voraus (gesetzt in den
%% einbindenden Dateien latex-korrekturansicht-abspann.tex bzw.
%% latex-leseansicht-abspann.tex).
%% ---------------------------------------------------------------

\normalsize

% Das esempio-Environment wird nur in der Leseansicht benötigt
\ifkorrekturansicht\else
\newenvironment{esempio}[3]%
{
    \vspace{1.5ex}
    \rlap{\underline{#1}}
    \par
    \setlength{\parindent}{0cm}
    \nopagebreak
    \leftskip=#2cm
    \rightskip=#3cm
}
{
    \par
}
\fi

\doendnotes{C}
\bigskip
\vfill

\clearpage

\footnotesize

\ifkorrekturansicht
  \lohead{\textsc{register}}
\fi

% theindex-Environment neu definieren ohne reledmac
\makeatletter
\renewenvironment{theindex}{%
  \ifkorrekturansicht
    \section*{\indexname}%
  \else
    \subsubsection*{Index der erwähnten Entitäten}%
  \fi
  \setlength{\parindent}{0pt}%
  \setlength{\parskip}{0pt plus 0.3pt}%
  \let\item\@idxitem
}{%
  \ifkorrekturansicht\clearpage\fi
}
\makeatother

\IfFileExists{\jobname-pw.ind}{\input{\jobname-pw.ind}}{}

% Quellenangabe nur in der Leseansicht
\ifkorrekturansicht\else
% Fallback-Definitionen, falls die .tex-Datei \titel etc. nicht gesetzt hat
\providecommand{\titel}{}
\providecommand{\editorInnen}{}
\providecommand{\dateiname}{\jobname}

\vspace{3cm}

\vfill

\footnotesize
\textsc{Quelle}: \titel. Herausgegeben von {\editorInnen}. In: \emph{Arthur Schnitzler: Briefwechsel mit Autorinnen und Autoren}.
 Digitale Edition, https://schnitzler-briefe.acdh.oeaw.ac.at/{\dateiname}.html (Stand \today)
\fi

\end{document}


      