%% latex-korrekturansicht-vorspann.tex
%% Vorspann für die Korrekturansicht.
%% Lädt die gemeinsame Datei latex-vorspann.tex mit gesetztem Schalter.

\newif\ifkorrekturansicht
\korrekturansichttrue

\input{../tex-inputs/latex-vorspann}


\section[Arthur Schnitzler an Hugo von Hofmannsthal, 27. 7. 1891]{L00025 Arthur Schnitzler an Hugo von Hofmannsthal,27. 7. 1891}
\nopagebreak\mylabel{L00025v}
\rehead{ }\normalsize\beginnumbering\briefempfaengerindex{Hofmannsthal, Hugo von@\textsc{Hofmannsthal, Hugo von}!zzzSchnitzler, Arthur@\emph{von Arthur Schnitzler}!1891-07-271@{27. 7. 1891}|(be}
\toendnotes[C]{\smallbreak\pagebreak[2]}\Standort{FDH, Hs-30885,9.}
\physDesc{Brief, 2 Blätter, 6 Seiten, 2301 Zeichen
\newline{}Handschrift: schwarze Tinte, deutsche Kurrent}
\buchAbdrucke{\weitereDrucke{1) Hugo von Hofmannsthal, Arthur Schnitzler: \emph{Briefwechsel}. Frankfurt am Main: \emph{S. Fischer} 1964, S. 9–10.} \weitereDrucke{2) Arthur Schnitzler: \emph{Briefe 1875–1912}. Frankfurt am Main: \emph{S. Fischer} 1981, S. 119–120.} }\toendnotes[C]{\smallbreak}
\pstart
           \raggedleft{}{\pb}Wien\oindex{Wien@\textbf{Wien}|pw}, 27. Juli 1891.\pend
           \vspace{0.5em}
\pstart
           Verehrter Freund, eine \label{K_L00025-1v}\edtext{Karte}{\lemma{\textnormal{\emph{Karte}}}\Cendnote{\textnormal{Siehe XXXX Auszeichnungsfehler: Dokument L02667 nicht gefunden.
               }}}\label{K_L00025-1}, die ich eben von Paul Goldma{\geminationn}\pwindex{Goldmann, Paul 31.\,1.\,1865 Breslau – 25.\,9.\,1935 Wien@\textsc{Goldmann, Paul} (31.\,1.\,1865 Breslau – 25.\,9.\,1935 Wien), \emph{Schriftsteller, Journalist}|pw} beko{\geminationm}e, eri{\geminationn}ert mich,
               wie üblich es iſt, Briefe zu beantworten, und wie ich Ihnen{ }ſchon längſt hätte{ }ſchreiben{ }ſollen, ja, wie ich Ihnen{ }ſogar hätte{ }ſchreiben wollen, we{\geminationn} mein Gehirn nicht die ganze letzte Zeit über todte
               Stellen hätte hinwegko{\geminationm}en müſſen. In zweierlei Perioden
               bietet einem das Leben was, in der der Anfänge, wo tauſenderlei über einen ko{\geminationm}t, und man {\pb}jeden Tag ein
               neues Blatt herzunehmen hat und nur drauflos zu begi{\geminationn}en.
                  Da{\geminationn} die andre Periode, wo man das Bedürfnis des
               Abſchließens hat – wo man die alten Blätter ni{\geminationm}t und
               einem alle möglichen Worte, Punkte u Gedankenſtriche einfallen, – die man verg\substVorne{}\textsuperscript{eſſen}\substDazwischen{}aß\substHinten{}{ }\strikeout{hat}. Die erſte Periode: wo man{ }ſich an{ }ſich
               berauſcht, die zweite: wo man{ }ſich an{ }ſich beruhigt. Ich bin jetzt in keiner von
               beiden, alſo arm und blöd. Nervös,{ }ſehr. Beer-{\pb}Hofma{\geminationn}\pwindex{Beer-Hofmann, Richard 11.\,7.\,1866 Wien – 26.\,9.\,1945 New York City@\textsc{Beer-Hofmann, Richard} (11.\,7.\,1866 Wien – 26.\,9.\,1945 New York City), \emph{Schriftsteller}|pw} iſt auch{ }ſchon weg, das wiſſen Sie ja. – In die \textsc{Kugel}\oindex{Cafe Kugel@\textbf{Café Kugel}|pw} ko{\geminationm} ich{ }ſelten, es waren{ }ſchon ein paar Ausſchuſsſitzungen\orgindex{»Freie Buehne« Verein fuer moderne Literatur@»Freie Bühne« Verein für moderne Literatur|pwv};
                  Specialcomités{ }ſind gew\textcolor{gray}{ä}hlt worden; ich{ }ſitze im Theatercomité
               zuſammen mit \textsc{Pernerstorfer}\pwindex{Pernerstorfer, Engelbert 27.\,4.\,1850 Wien – 6.\,1.\,1918 ebd.@\textsc{Pernerstorfer, Engelbert} (27.\,4.\,1850 Wien – 6.\,1.\,1918 ebd.), \emph{Politiker, Journalist}|pw}, \textsc{Wengraf}\pwindex{Wengraf, Edmund 9.\,1.\,1860 Mikulov – 8.\,12.\,1933 Wien@\textsc{Wengraf, Edmund} (9.\,1.\,1860 Mikulov – 8.\,12.\,1933 Wien), \emph{Schriftsteller, Journalist, Kaufmann}|pw}, \textsc{Osten}\pwindex{Osten, Heinrich 16.\,8.\,1855 Brody [Ukraine] – 1.\,8.\,1931 Wien@\textsc{Osten, Heinrich} (16.\,8.\,1855 Brody [Ukraine] – 1.\,8.\,1931 Wien), \emph{Schriftsteller, Journalist}|pw}, \textsc{Kafka}\pwindex{Kafka, Eduard Michael 11.\,3.\,1869 Wien – 6.\,8.\,1893 Bruenn@\textsc{Kafka, Eduard Michael} (11.\,3.\,1869 Wien – 6.\,8.\,1893 Brünn), \emph{Redakteur}|pw}, \textsc{Kulka}\pwindex{Kulka, Julius 25.\,9.\,1865 Lipník nad Becvou – 22.\,9.\,1893 Wien@\textsc{Kulka, Julius} (25.\,9.\,1865 Lipník nad Bečvou – 22.\,9.\,1893 Wien), \emph{Rechtsanwalt}|pw}. –\hspace*{2.5em}Bis jetzt iſt noch nicht viel geſcheidtes
                  herausgeko{\geminationm}en. – Mit \textsc{Salten}\pwindex{Salten, Felix 6.\,9.\,1869 Budapest – 8.\,10.\,1945 Zuerich@\textsc{Salten, Felix} (6.\,9.\,1869 Budapest – 8.\,10.\,1945 Zürich), \emph{Schriftsteller, Journalist, Chefredakteur}|pw} bin ich viel zuſa{\geminationm}en, auch auf dem »Land« des
               Abends. \textsc{Burckhard}\pwindex{Burckhard, Max Eugen 14.\,7.\,1854 Korneuburg – 16.\,3.\,1912 Wien@\textsc{Burckhard, Max Eugen} (14.\,7.\,1854 Korneuburg – 16.\,3.\,1912 Wien), \emph{Schriftsteller, Rechtswissenschaftler, Theaterleiter}|pw} hat mir den AlkandiSEXref\pwindex{Schnitzler, Arthur 15.\,5.\,1862 Wien – 21.\,10.\,1931 ebd.@\textsc{Schnitzler, Arthur} (15.\,5.\,1862 Wien – 21.\,10.\,1931 ebd.), \emph{Schriftsteller*in, Mediziner*in}!Alkandi s Lied@\strich\emph{Alkandi’s Lied}|pw} mit einigen{ }ſchmeichelhaften Worten {\pb}zurückgeſandt – ich hab’ ihn
                  angeno{\geminationm}en. Mein StückSEXref\pwindex{Schnitzler, Arthur 15.\,5.\,1862 Wien – 21.\,10.\,1931 ebd.@\textsc{Schnitzler, Arthur} (15.\,5.\,1862 Wien – 21.\,10.\,1931 ebd.), \emph{Schriftsteller*in, Mediziner*in}!Maerchen. Schauspiel in drei Aufzuegen@\strich\emph{Das Märchen. Schauspiel in drei Aufzügen}|pwv} ruht und iſt mir zuwider. – Wie geht es Ihrem hi{\geminationm}elblauen EinakterSEXref\pwindex{Hofmannsthal, Hugo von 1.\,2.\,1874 Wien – 15.\,7.\,1929 Rodaun@\textsc{Hofmannsthal, Hugo von} (1.\,2.\,1874 Wien – 15.\,7.\,1929 Rodaun), \emph{Schriftsteller}!Gestern. Dramatische Studie in einem Akt in Versen@\strich\emph{Gestern. Dramatische Studie in einem Akt in Versen}|pwv}? Und wollen Sie mir nichts von Ihren Sachen{ }ſchicken? Sie
               würden mir eine wirkliche Freude machen,{ }ſeien Sie erſter oder{ }ſiebenter Grad! –
               Geleſen wird mancherlei \textsc{Burckhardt}\pwindex{Burckhardt, Jacob 25.\,5.\,1818 Basel – 8.\,8.\,1897 ebd.@\textsc{Burckhardt, Jacob} (25.\,5.\,1818 Basel – 8.\,8.\,1897 ebd.), \emph{Historiker, Kunsthistoriker}|pw}, Cultur der RenaiſſanceSEXref\pwindex{Burckhardt, Jacob 25.\,5.\,1818 Basel – 8.\,8.\,1897 ebd.@\textsc{Burckhardt, Jacob} (25.\,5.\,1818 Basel – 8.\,8.\,1897 ebd.), \emph{Historiker, Kunsthistoriker}!Cultur der Renaissance in Italien. Ein Versuch@\strich\emph{Die Cultur der Renaissance in Italien. Ein Versuch}|pw}, \textsc{Goethe}\pwindex{Goethe, Johann Wolfgang von 28.\,8.\,1749 Frankfurt am Main – 22.\,3.\,1832 Weimar@\textsc{Goethe, Johann Wolfgang von} (28.\,8.\,1749 Frankfurt am Main – 22.\,3.\,1832 Weimar), \emph{Schriftsteller}|pw}, AnnalenSEXref\pwindex{Goethe, Johann Wolfgang von 28.\,8.\,1749 Frankfurt am Main – 22.\,3.\,1832 Weimar@\textsc{Goethe, Johann Wolfgang von} (28.\,8.\,1749 Frankfurt am Main – 22.\,3.\,1832 Weimar), \emph{Schriftsteller}!Tag- und Jahreshefte@\strich\emph{Tag- und Jahreshefte}|pw}, \textsc{Lessings}\pwindex{Lessing, Gotthold Ephraim 22.\,1.\,1729 Kamenz – 15.\,2.\,1781 Braunschweig@\textsc{Lessing, Gotthold Ephraim} (22.\,1.\,1729 Kamenz – 15.\,2.\,1781 Braunschweig), \emph{Schriftsteller, Kritiker, Philosoph}|pw}{ }Drama\strikeout{turgie}
                  EntwürfeSEXref\pwindex{\textcolor{red}{\textsuperscript{XXXX indx}}!Vierundfunfzig zum Theil noch ungedruckte Dramatische Entwuerfe und Plaene Gotthold Ephraim Lessings@\strich\emph{Vierundfunfzig zum Theil noch ungedruckte Dramatische Entwürfe und Pläne Gotthold Ephraim Lessings}|pw}, \textsc{Jonas Lie}\pwindex{Lie, Jonas 6.\,11.\,1833 Hokksund – 5.\,7.\,1908 Stavern@\textsc{Lie, Jonas} (6.\,11.\,1833 Hokksund – 5.\,7.\,1908 Stavern), \emph{Schriftsteller}|pw}{ }\textsc{etc.} Beſonders \textsc{Nietz}’ſche\pwindex{Nietzsche, Friedrich 15.\,10.\,1844 Roecken – 25.\,8.\,1900 Weimar@\textsc{Nietzsche, Friedrich} (15.\,10.\,1844 Röcken – 25.\,8.\,1900 Weimar), \emph{Schriftsteller, Philosoph}|pw} – zuletzt {\pb}hat mich{ }ſein Schluſscapitel und das SchlußgedichtSEXref\pwindex{Nietzsche, Friedrich 15.\,10.\,1844 Roecken – 25.\,8.\,1900 Weimar@\textsc{Nietzsche, Friedrich} (15.\,10.\,1844 Röcken – 25.\,8.\,1900 Weimar), \emph{Schriftsteller, Philosoph}!Nachgesang. Aus den hohen Bergen@\strich\emph{Nachgesang. Aus den hohen Bergen}|pwv} zu \textsc{Jenseits von Gut u Böse}SEXref\pwindex{Nietzsche, Friedrich 15.\,10.\,1844 Roecken – 25.\,8.\,1900 Weimar@\textsc{Nietzsche, Friedrich} (15.\,10.\,1844 Röcken – 25.\,8.\,1900 Weimar), \emph{Schriftsteller, Philosoph}!Jenseits von Gut und Boese@\strich\emph{Jenseits von Gut und Böse}|pw} ergriffen. – Eri{\geminationn}ern Sie{ }ſich? \textsc{Nietz}’ſche\pwindex{Nietzsche, Friedrich 15.\,10.\,1844 Roecken – 25.\,8.\,1900 Weimar@\textsc{Nietzsche, Friedrich} (15.\,10.\,1844 Röcken – 25.\,8.\,1900 Weimar), \emph{Schriftsteller, Philosoph}|pw}{ }Sentimentalität! – Weinender Marmor! Stellen, die{ }ſogar auf Weiber wirken, ohne daß man den Stellen oder den Weibern bös werden
               müßte. – Werden Sie mir bald wieder{ }ſchreiben? Arbeiten Sie viel? Erleben {\pb}Sie was? Spielen Sie aber lieber \textsc{lawn-tennis},{ }ſtatt{ }ſich zu verlieben, oder nehmen Sie wenigſtens, we{\geminationn} beides über Sie geko{\geminationm}en,
               das erſtere ernſter.\pend
           
\pstart
           Herzlichen Gruſs. Den Ihrigen meine Empfehlungen. Iſt \textsc{Schwarzkopf}\pwindex{Schwarzkopf, Gustav 7.\,11.\,1853 Wien – 13.\,11.\,1939 ebd.@\textsc{Schwarzkopf, Gustav} (7.\,11.\,1853 Wien – 13.\,11.\,1939 ebd.), \emph{Schriftsteller}|pw}{ }ſchon bei Ihnen? Ich{ }ſah ihn{ }ſchon Wochen lang
               nicht. –\pend
           
\pstart
           Alſo nochmals, viele Grüße{\\[\baselineskip]}Ihr \spacefill\mbox{Arthur Sch}\pend
           \leftskip=0em{}\selectlanguage{ngerman}\endnumbering\briefempfaengerindex{Hofmannsthal, Hugo von@\textsc{Hofmannsthal, Hugo von}!zzzSchnitzler, Arthur@\emph{von Arthur Schnitzler}!1891-07-271@{27. 7. 1891}|)be}\mylabel{L00025h}  \normalsize

\doendnotes{C}
\bigskip
\vfill

\clearpage

\footnotesize

\lohead{\textsc{register}}

% Definiere theindex-Environment komplett neu ohne reledmac
\makeatletter
\renewenvironment{theindex}{%
  \section*{\indexname}%
  \setlength{\parindent}{0pt}%
  \setlength{\parskip}{0pt plus 0.3pt}%
  \let\item\@idxitem
}{%
  \clearpage
}
\makeatother

\IfFileExists{\jobname-pw.ind}{\input{\jobname-pw.ind}}{}

\end{document}

      