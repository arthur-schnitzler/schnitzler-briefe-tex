%% latex-korrekturansicht-vorspann.tex
%% Vorspann für die Korrekturansicht.
%% Lädt die gemeinsame Datei latex-vorspann.tex mit gesetztem Schalter.

\newif\ifkorrekturansicht
\korrekturansichttrue

\input{../tex-inputs/latex-vorspann}


\section[ Arthur Schnitzler an Felix Salten, 10. 1. 1905]{L02995 Arthur Schnitzler an Felix Salten, 10. 1. 1905}
\nopagebreak\mylabel{L02995v}
\rehead{ }\normalsize\beginnumbering\briefempfaengerindex{Salten, Felix@\textsc{Salten, Felix}!zzzSchnitzler, Arthur@\emph{von Arthur Schnitzler}!1905-01-101@{10. 1. 1905}|(be}
\toendnotes[C]{\smallbreak\pagebreak[2]}\Standort{Wienbibliothek im Rathaus, ZPH 1681, 2.1.516.}
\physDesc{Brief, 1 Blatt, 4 Seiten, 1275 Zeichen
\newline{}Handschrift: schwarze Tinte, deutsche Kurrent
\newline{}Ordnung: mit Bleistift von unbekannter Hand Nummerierung der Doppelseiten des Konvoluts:
                                    »59«–»60« }
\buchAbdrucke{\weitereDrucke{1) Arthur Schnitzler: \emph{Briefe 1875–1912}. Frankfurt am Main: \emph{S. Fischer} 1981, S. 510.} \weitereDrucke{2) Hermann Bahr, Arthur Schnitzler: \emph{Briefwechsel, Aufzeichnungen, Dokumente (1891–1931)}. Göttingen: \emph{Wallstein} 2018, S. 338–339.} }\toendnotes[C]{\smallbreak}
\pstart
           \raggedleft{}{\pb}Wien\oindex{Wien@\textbf{Wien}, \emph{A.ADM2}|pw}, 10. 1. 905.\pend
           \vspace{0.5em}
\pstart
           lieber, die Sandrock\pwindex{Sandrock, Adele 1863-08-19 – 1937-08-30@\textsc{Sandrock, Adele} (1863-08-19 – 1937-08-30), \emph{Schauspieler/Schauspielerin}|pw} war
               wegen der \label{K_L02995-1v}\edtext{\textsc{Hervay\pwindex{Hervay von Kirchberg, Elvira Leontine 18.07.1860 – nach 1929@\textsc{Hervay von Kirchberg, Elvira Leontine} (18.07.1860 – nach 1929)|pw}}-Vorleſung}{\lemma{\textnormal{\emph{Hervay-Vorleſung}}}\Cendnote{\textnormal{Diese fand am 2. 2. 1905 statt. Hintergrund bildete ein
                  viel beachteter Prozess, bei dem Tamara von
                     Hervay\pwindex{Hervay von Kirchberg, Elvira Leontine 18.07.1860 – nach 1929@\textsc{Hervay von Kirchberg, Elvira Leontine} (18.07.1860 – nach 1929)|pwk} als Bigamistin verurteilt worden war. Bahr\pwindex{Bahr, Hermann 19.07.1863 – 15.01.1934@\textsc{Bahr, Hermann} (19.07.1863 – 15.01.1934), \emph{Schriftsteller/Schriftstellerin, Kritiker/Kritikerin}|pwk} ließ sich von den Ereignissen zum Roman \emph{Drut}\pwindex{Drut. Roman@\emph{Drut. Roman}|pwk} (1909)
                  inspirieren.}}}\label{K_L02995-1}{ }\label{K_L02995-2v}\edtext{bei mir}{\lemma{\textnormal{\emph{bei mir}}}\Cendnote{\textnormal{»Traf Sandrock\pwindex{Sandrock, Adele 1863-08-19 – 1937-08-30@\textsc{Sandrock, Adele} (1863-08-19 – 1937-08-30), \emph{Schauspieler/Schauspielerin}|pw}, die eben zu
                     mir wollte; sie forderte mich zur Mitwirkung an einer Vorlesung für die Hervay\pwindex{Hervay von Kirchberg, Elvira Leontine 18.07.1860 – nach 1929@\textsc{Hervay von Kirchberg, Elvira Leontine} (18.07.1860 – nach 1929)|pw} auf, ich
                     sagte halb zu, schrieb aber Nachm. an Salten\pwindex{Salten, Felix 06.09.1869 – 08.10.1945@\textsc{Salten, Felix} (06.09.1869 – 08.10.1945), \emph{Schriftsteller/Schriftstellerin, Journalist/Journalistin, Chefredakteur/Chefredakteurin}|pw} ab.« A. S.: \emph{Tagebuch}, 10. 1. 1905.
               }}}\label{K_L02995-2}; da ich heuer
               ſowie voriges Jahr{ }\uline{abſolut immer} abgelehnt \strikeout{habe,} und in Wien\oindex{Wien@\textbf{Wien}, \emph{A.ADM2}|pw} (von jener \label{K_L02995-3v}\edtext{\textsc{Karlweis\pwindex{Karlweis, Carl 23.11.1850 – 27.10.1901@\textsc{Karlweis, Carl} (23.11.1850 – 27.10.1901), \emph{Schriftsteller/Schriftstellerin}|pw}}-Sache im Jahre 97}{\lemma{\textnormal{\emph{Karlweis-Sache … 97}}}\Cendnote{\textnormal{Siehe A. S.: \emph{Tagebuch}, 28. 3. 1897.
               }}}\label{K_L02995-3} abgeſehen) überhaupt nur ein paar Mal in Arbeitervereinen geleſen habe, mir
               das Vorleſen vor der Wien\oindex{Wien@\textbf{Wien}, \emph{A.ADM2}|pw}er Bürgerſchaft ſo
               widerwärtig wie möglich iſt und ich nebſtbei {\pb}alle die Leute, denen ich bisher Refus gegeben, nicht\strikeout{,} ohne tiefe innere Nöthigung zu verletzen Luſt habe; – widerſtrebt es mir
               ſehr, in dieſem Fall eine Ausnahme zu machen, und ich ſchreibe Ihnen das, weil die
                  \textsc{S.\pwindex{Sandrock, Adele 1863-08-19 – 1937-08-30@\textsc{Sandrock, Adele} (1863-08-19 – 1937-08-30), \emph{Schauspieler/Schauspielerin}|pw}} natürlich gegen alle dieſe Gründe taub war, und ich annehme, daſs es Ihnen ganz
               leicht ſein wird, ihr meine Mitwirkung auszureden. Bahr\pwindex{Bahr, Hermann 19.07.1863 – 15.01.1934@\textsc{Bahr, Hermann} (19.07.1863 – 15.01.1934), \emph{Schriftsteller/Schriftstellerin, Kritiker/Kritikerin}|pw} hat tele{\pb}grafiſch zugeſagt (ich
               verſprach der \textsc{S.\pwindex{Sandrock, Adele 1863-08-19 – 1937-08-30@\textsc{Sandrock, Adele} (1863-08-19 – 1937-08-30), \emph{Schauspieler/Schauspielerin}|pw}} Ihnen das gleich zu ſchreiben) der Abend ſelbſt iſt durch Sie, \textsc{Bahr\pwindex{Bahr, Hermann 19.07.1863 – 15.01.1934@\textsc{Bahr, Hermann} (19.07.1863 – 15.01.1934), \emph{Schriftsteller/Schriftstellerin, Kritiker/Kritikerin}|pw}}; \textsc{Sandrock\pwindex{Sandrock, Adele 1863-08-19 – 1937-08-30@\textsc{Sandrock, Adele} (1863-08-19 – 1937-08-30), \emph{Schauspieler/Schauspielerin}|pw}} zugkräftig \textcolor{gray}{–} geſichert genug; und ich hoffe überzeugt ſein
               zu dürfen, daſs Ihnen meine Vorleſerei an dieſem Abend nicht fehlen
               wird. (Den wohltätigen Zweck ka{\geminationn} ich ja, hab ich ſchon,
               in beſcheidener Weiſe gefördert, indem ich mich an der \textsc{Sandrock\pwindex{Sandrock, Adele 1863-08-19 – 1937-08-30@\textsc{Sandrock, Adele} (1863-08-19 – 1937-08-30), \emph{Schauspieler/Schauspielerin}|pw}} Sa{\geminationm}lung betheilige{\dotstwo}).
               Ich beläſtige Sie {\pb}mit dieſem Brief, weil Sie
               ja die \textsc{Sandrock\pwindex{Sandrock, Adele 1863-08-19 – 1937-08-30@\textsc{Sandrock, Adele} (1863-08-19 – 1937-08-30), \emph{Schauspieler/Schauspielerin}|pw}} gewiſs in dieſer Angelegenheit bald ſprechen – u weil es wohl ja nichts hilft,
                  we{\geminationn} ich ihr ſelbſt diese Sachen ſchreibe.\pend
           
\pstart
           Seien Sie herzlich gegrüßt {\\[\baselineskip]}Ihr {\\[\baselineskip]}\spacefill\mbox{Arth}\pend
           \leftskip=0em{}\selectlanguage{ngerman}\endnumbering\briefempfaengerindex{Salten, Felix@\textsc{Salten, Felix}!zzzSchnitzler, Arthur@\emph{von Arthur Schnitzler}!1905-01-101@{10. 1. 1905}|)be}\mylabel{L02995h}  \normalsize

\doendnotes{C}
\bigskip
\vfill

\clearpage

\footnotesize

\lohead{\textsc{register}}

% Definiere theindex-Environment komplett neu ohne reledmac
\makeatletter
\renewenvironment{theindex}{%
  \section*{\indexname}%
  \setlength{\parindent}{0pt}%
  \setlength{\parskip}{0pt plus 0.3pt}%
  \let\item\@idxitem
}{%
  \clearpage
}
\makeatother

\IfFileExists{\jobname-pw.ind}{\input{\jobname-pw.ind}}{}

\end{document}

      