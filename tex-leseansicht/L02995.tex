%% latex-leseansicht-vorspann.tex
%% Vorspann für die Leseansicht.
%% Lädt die gemeinsame Datei latex-vorspann.tex mit nicht gesetztem Schalter.

\newif\ifkorrekturansicht
\korrekturansichtfalse

\input{../tex-inputs/latex-vorspann}


\section[ Arthur Schnitzler an Felix Salten, 10. 1. 1905]{L02995 Arthur Schnitzler an Felix Salten,  10. 1. 1905}
\nopagebreak\mylabel{L02995v}
\rehead{ }\normalsize\beginnumbering\briefempfaengerindex{Salten, Felix@\textsc{Salten, Felix}!zzzSchnitzler, Arthur@\emph{von Arthur Schnitzler}!1905-01-101@{10. 1. 1905}|(be}
\toendnotes[C]{\smallbreak\pagebreak[2]}
\correspDesc{Versand  durch Arthur Schnitzler am 10. 1. 1905 in Wien
\newline{}Erhalt  durch Felix Salten im Zeitraum [10. 1. 1905 –
                  11. 1. 1905] in Wien}\toendnotes[C]{\smallbreak}
\Standort{Wienbibliothek im Rathaus, ZPH 1681, 2.1.516.}
\physDesc{Brief, 1 Blatt, 4 Seiten, 1275 Zeichen
\newline{}Handschrift: schwarze Tinte, deutsche Kurrent
\newline{}Ordnung: mit Bleistift von unbekannter Hand Nummerierung der Doppelseiten des Konvoluts:
                                    »59«–»60« }
\buchAbdrucke{\weitereDrucke{1) Arthur Schnitzler: \emph{Briefe 1875–1912}. Herausgegeben von Therese Nickl und Heinrich Schnitzler. Frankfurt am Main: \emph{S. Fischer} 1981, S. 510.} \weitereDrucke{2) Hermann Bahr, Arthur Schnitzler: \emph{Briefwechsel, Aufzeichnungen, Dokumente (1891–1931)}. Herausgegeben von Kurt Ifkovits und Martin Anton Müller. Göttingen: \emph{Wallstein} 2018, S. 338–339.} }\toendnotes[C]{\smallbreak}
\pstart
           \raggedleft{}{\pb}Wien\oindex{Wien@\textbf{Wien}, \emph{Verwaltungsgebiet}|pw}, 10. 1. 905.\pend
           \vspace{0.5em}
\pstart
           lieber, die Sandrock\pwindex{Sandrock, Adele 19.\,8.\,1863 Rotterdam – 30.\,8.\,1937 Berlin@\textsc{Sandrock, Adele} (19.\,8.\,1863 Rotterdam – 30.\,8.\,1937 Berlin), \emph{Schauspielerin}|pw} war
               wegen der \label{K_L02995-1v}\edtext{\textsc{Hervay\pwindex{Hervay von Kirchberg, Elvira Leontine 18.\,7.\,1860 Poznan – nach 1929@\textsc{Hervay von Kirchberg, Elvira Leontine} (18.\,7.\,1860 Poznan – nach 1929)|pw}}-Vorleſung}{\lemma{\textnormal{\emph{Hervay-Vorlesung}}}\Cendnote{\textnormal{Diese fand am 2. 2. 1905 statt. Hintergrund bildete ein
                  viel beachteter Prozess, bei dem Tamara von
                     Hervay\pwindex{Hervay von Kirchberg, Elvira Leontine 18.\,7.\,1860 Poznan – nach 1929@\textsc{Hervay von Kirchberg, Elvira Leontine} (18.\,7.\,1860 Poznan – nach 1929)|pwk} als Bigamistin verurteilt worden war. Bahr\pwindex{Bahr, Hermann 19.\,7.\,1863 Linz – 15.\,1.\,1934 München@\textsc{Bahr, Hermann} (19.\,7.\,1863 Linz – 15.\,1.\,1934 München), \emph{Schriftsteller, Kritiker}|pwk} ließ sich von den Ereignissen zum Roman \emph{Drut}\pwindex{Bahr, Hermann 19.\,7.\,1863 Linz – 15.\,1.\,1934 München@\textsc{Bahr, Hermann} (19.\,7.\,1863 Linz – 15.\,1.\,1934 München), \emph{Schriftsteller, Kritiker}!Drut. Roman@\strich\emph{Drut. Roman}|pwk} (1909)
                  inspirieren.}}}\label{K_L02995-1}{ }\label{K_L02995-2v}\edtext{bei mir}{\lemma{\textnormal{\emph{bei mir}}}\Cendnote{\textnormal{»Traf Sandrock\pwindex{Sandrock, Adele 19.\,8.\,1863 Rotterdam – 30.\,8.\,1937 Berlin@\textsc{Sandrock, Adele} (19.\,8.\,1863 Rotterdam – 30.\,8.\,1937 Berlin), \emph{Schauspielerin}|pw}, die eben zu
                     mir wollte; sie forderte mich zur Mitwirkung an einer Vorlesung für die Hervay\pwindex{Hervay von Kirchberg, Elvira Leontine 18.\,7.\,1860 Poznan – nach 1929@\textsc{Hervay von Kirchberg, Elvira Leontine} (18.\,7.\,1860 Poznan – nach 1929)|pw} auf, ich
                     sagte halb zu, schrieb aber Nachm. an Salten\pwindex{Salten, Felix 6.\,9.\,1869 Budapest – 8.\,10.\,1945 Zürich@\textsc{Salten, Felix} (6.\,9.\,1869 Budapest – 8.\,10.\,1945 Zürich), \emph{Schriftsteller, Journalist, Chefredakteur}|pw} ab.« A. S.: \emph{Tagebuch}, 10. 1. 1905.
               }}}\label{K_L02995-2}; da ich heuer{ }ſowie voriges Jahr{ }\uline{abſolut immer} abgelehnt \strikeout{habe,} und in Wien\oindex{Wien@\textbf{Wien}, \emph{Verwaltungsgebiet}|pw} (von jener \label{K_L02995-3v}\edtext{\textsc{Karlweis\pwindex{Karlweis, Carl 23.\,11.\,1850 Wien – 27.\,10.\,1901 ebd.@\textsc{Karlweis, Carl} (23.\,11.\,1850 Wien – 27.\,10.\,1901 ebd.), \emph{Schriftsteller}|pw}}-Sache im Jahre 97}{\lemma{\textnormal{\emph{Karlweis-Sache … 97}}}\Cendnote{\textnormal{Siehe A. S.: \emph{Tagebuch}, 28. 3. 1897.
               }}}\label{K_L02995-3} abgeſehen) überhaupt nur ein paar Mal in Arbeitervereinen geleſen habe, mir
               das Vorleſen vor der Wien\oindex{Wien@\textbf{Wien}, \emph{Verwaltungsgebiet}|pw}er Bürgerſchaft{ }ſo
               widerwärtig wie möglich iſt und ich nebſtbei {\pb}alle die Leute, denen ich bisher Refus gegeben, nicht\strikeout{,} ohne tiefe innere Nöthigung zu verletzen Luſt habe; – widerſtrebt es mir{ }ſehr, in dieſem Fall eine Ausnahme zu machen, und ich{ }ſchreibe Ihnen das, weil die
                  \textsc{S.\pwindex{Sandrock, Adele 19.\,8.\,1863 Rotterdam – 30.\,8.\,1937 Berlin@\textsc{Sandrock, Adele} (19.\,8.\,1863 Rotterdam – 30.\,8.\,1937 Berlin), \emph{Schauspielerin}|pw}} natürlich gegen alle dieſe Gründe taub war, und ich annehme, daſs es Ihnen ganz
               leicht{ }ſein wird, ihr meine Mitwirkung auszureden. Bahr\pwindex{Bahr, Hermann 19.\,7.\,1863 Linz – 15.\,1.\,1934 München@\textsc{Bahr, Hermann} (19.\,7.\,1863 Linz – 15.\,1.\,1934 München), \emph{Schriftsteller, Kritiker}|pw} hat tele{\pb}grafiſch zugeſagt (ich
               verſprach der \textsc{S.\pwindex{Sandrock, Adele 19.\,8.\,1863 Rotterdam – 30.\,8.\,1937 Berlin@\textsc{Sandrock, Adele} (19.\,8.\,1863 Rotterdam – 30.\,8.\,1937 Berlin), \emph{Schauspielerin}|pw}} Ihnen das gleich zu{ }ſchreiben) der Abend{ }ſelbſt iſt durch Sie, \textsc{Bahr\pwindex{Bahr, Hermann 19.\,7.\,1863 Linz – 15.\,1.\,1934 München@\textsc{Bahr, Hermann} (19.\,7.\,1863 Linz – 15.\,1.\,1934 München), \emph{Schriftsteller, Kritiker}|pw}}; \textsc{Sandrock\pwindex{Sandrock, Adele 19.\,8.\,1863 Rotterdam – 30.\,8.\,1937 Berlin@\textsc{Sandrock, Adele} (19.\,8.\,1863 Rotterdam – 30.\,8.\,1937 Berlin), \emph{Schauspielerin}|pw}} zugkräftig \textcolor{gray}{–} geſichert genug; und ich hoffe überzeugt{ }ſein
               zu dürfen, daſs Ihnen meine Vorleſerei an dieſem Abend nicht fehlen
               wird. (Den wohltätigen Zweck ka{\geminationn} ich ja, hab ich{ }ſchon,
               in beſcheidener Weiſe gefördert, indem ich mich an der \textsc{Sandrock\pwindex{Sandrock, Adele 19.\,8.\,1863 Rotterdam – 30.\,8.\,1937 Berlin@\textsc{Sandrock, Adele} (19.\,8.\,1863 Rotterdam – 30.\,8.\,1937 Berlin), \emph{Schauspielerin}|pw}} Sa{\geminationm}lung betheilige{\dotstwo}).
               Ich beläſtige Sie {\pb}mit dieſem Brief, weil Sie
               ja die \textsc{Sandrock\pwindex{Sandrock, Adele 19.\,8.\,1863 Rotterdam – 30.\,8.\,1937 Berlin@\textsc{Sandrock, Adele} (19.\,8.\,1863 Rotterdam – 30.\,8.\,1937 Berlin), \emph{Schauspielerin}|pw}} gewiſs in dieſer Angelegenheit bald{ }ſprechen – u weil es wohl ja nichts hilft,
                  we{\geminationn} ich ihr{ }ſelbſt diese Sachen{ }ſchreibe.\pend
           
\pstart
           Seien Sie herzlich gegrüßt {\\[\baselineskip]}Ihr {\\[\baselineskip]}\spacefill\mbox{Arth}\pend
           \leftskip=0em{}\selectlanguage{ngerman}\endnumbering\briefempfaengerindex{Salten, Felix@\textsc{Salten, Felix}!zzzSchnitzler, Arthur@\emph{von Arthur Schnitzler}!1905-01-101@{10. 1. 1905}|)be}\mylabel{L02995h}  \newcommand{\dateiname}{L02995}\newcommand{\titel}{Arthur Schnitzler an Felix Salten, 10. 1. 1905}\newcommand{\editorInnen}{Martin Anton Müller und Laura Untner}%% latex-leseansicht-abspann.tex
%% Abspann für die Leseansicht.
%% Der Schalter \ifkorrekturansicht ist bereits durch den Vorspann gesetzt.

%% latex-abspann.tex
%% Gemeinsamer Abspann für Korrekturansicht und Leseansicht.
%% Setzt den Schalter \ifkorrekturansicht voraus (gesetzt in den
%% einbindenden Dateien latex-korrekturansicht-abspann.tex bzw.
%% latex-leseansicht-abspann.tex).
%% ---------------------------------------------------------------

\normalsize

% Das esempio-Environment wird nur in der Leseansicht benötigt
\ifkorrekturansicht\else
\newenvironment{esempio}[3]%
{
    \vspace{1.5ex}
    \rlap{\underline{#1}}
    \par
    \setlength{\parindent}{0cm}
    \nopagebreak
    \leftskip=#2cm
    \rightskip=#3cm
}
{
    \par
}
\fi

\doendnotes{C}
\bigskip
\vfill

\clearpage

\footnotesize

\ifkorrekturansicht
  \lohead{\textsc{register}}
\fi

% theindex-Environment neu definieren ohne reledmac
\makeatletter
\renewenvironment{theindex}{%
  \ifkorrekturansicht
    \section*{\indexname}%
  \else
    \subsubsection*{Index der erwähnten Entitäten}%
  \fi
  \setlength{\parindent}{0pt}%
  \setlength{\parskip}{0pt plus 0.3pt}%
  \let\item\@idxitem
}{%
  \ifkorrekturansicht\clearpage\fi
}
\makeatother

\IfFileExists{\jobname-pw.ind}{\input{\jobname-pw.ind}}{}

% Quellenangabe nur in der Leseansicht
\ifkorrekturansicht\else
% Fallback-Definitionen, falls die .tex-Datei \titel etc. nicht gesetzt hat
\providecommand{\titel}{}
\providecommand{\editorInnen}{}
\providecommand{\dateiname}{\jobname}

\vspace{3cm}

\vfill

\footnotesize
\textsc{Quelle}: \titel. Herausgegeben von {\editorInnen}. In: \emph{Arthur Schnitzler: Briefwechsel mit Autorinnen und Autoren}.
 Digitale Edition, https://schnitzler-briefe.acdh.oeaw.ac.at/{\dateiname}.html (Stand \today)
\fi

\end{document}


