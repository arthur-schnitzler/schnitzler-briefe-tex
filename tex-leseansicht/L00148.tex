%% latex-leseansicht-vorspann.tex
%% Vorspann für die Leseansicht.
%% Lädt die gemeinsame Datei latex-vorspann.tex mit nicht gesetztem Schalter.

\newif\ifkorrekturansicht
\korrekturansichtfalse

\input{../tex-inputs/latex-vorspann}


               \section[Richard Beer-Hofmann an Arthur Schnitzler, {[}28. 12. 1892?{]}]{ Richard Beer-Hofmann an Arthur Schnitzler, {[}28. 12. 1892?{]}}\nopagebreak\mylabel{v}\rehead{ }\begin{ledgroupsized}[t]{13cm}\normalsize\beginnumbering\briefempfaengerindex{Schnitzler, Arthur@\textsc{Schnitzler, Arthur}!zzzBeer-Hofmann, Richard@\emph{von Richard Beer-Hofmann}!1892-12-281@{{[}28. 12. 1892?{]}}|(be} \toendnotes[C]{\smallbreak\pagebreak[2]} \Standort{CUL, Schnitzler, B 8.}
\physDesc{Brief, 1 Blatt, 2 Seiten
\newline{}Handschrift: blauer Buntstift, lateinische Kurrent
\newline{}Schnitzler: mit Bleistift nummeriert: »15« }\toendnotes[C]{\smallbreak}\pstart{}{\pb}Lieber Arthur!\pend\pstart
           Frau Flegmann\pwindex{Flegmann, Bertha 27.05.1852 – 24.6.1933@\textsc{Flegmann, Bertha} (27.05.1852 – 24.6.1933), \emph{Salonnière}|pw} hat \uline{uns} für nächsten \label{K_L00148_1v}\edtext{Freitag}{\lemma{\textnormal{\emph{Freitag}}}\Cendnote{\textnormal{Der Brief ist undatiert. Am
                     23. 12. 1892 wird im Brief Hofmannsthal\pwindex{Hofmannsthal, Hugo von 01.02.1874 – 15.07.1929@\textsc{Hofmannsthal, Hugo von} (01.02.1874 – 15.07.1929), \emph{Schriftsteller}|pwk}s an Schnitzler\pwindex{Schnitzler, Arthur 15.05.1862 – 21.10.1931@\textsc{Schnitzler, Arthur} (15.05.1862 – 21.10.1931), \emph{Schriftsteller, Mediziner}|pwk}{ }\emph{Aspasia}\pwindex{\textcolor{red}{\textsuperscript{XXXX1 indx}}!Aspasia3.3.1892 – 3.3.1892@\strich\emph{Aspasia} {[}3.3.1892 – 3.3.1892{]}|pwk} erwähnt. Es dürfte sich um
                  Vorbereitungen zu einer Privataufführung bei Bertha
                     Flegmann\pwindex{Flegmann, Bertha 27.05.1852 – 24.6.1933@\textsc{Flegmann, Bertha} (27.05.1852 – 24.6.1933), \emph{Salonnière}|pwk} gehandelt haben. Da Schnitzler\pwindex{Schnitzler, Arthur 15.05.1862 – 21.10.1931@\textsc{Schnitzler, Arthur} (15.05.1862 – 21.10.1931), \emph{Schriftsteller, Mediziner}|pwk} am Donnerstag, dem 29. 12. 1892 für den
                  morgigen Tag ein Treffen bei Flegmann\pwindex{Flegmann, Bertha 27.05.1852 – 24.6.1933@\textsc{Flegmann, Bertha} (27.05.1852 – 24.6.1933), \emph{Salonnière}|pwk} absagt,
                  scheint dieses Korrespondenzstück der wahrscheinliche Vorgänger desselben zu sein.}}}\label{K_L00148_1h}
               eingeladen (Aspasia\pwindex{\textcolor{red}{\textsuperscript{XXXX1 indx}}!Aspasia3.3.1892 – 3.3.1892@\strich\emph{Aspasia} {[}3.3.1892 – 3.3.1892{]}|pw}) ich refusire daher Singer\pwindex{Singer, Alexander 16.11.1841 – 30.11.1906@\textsc{Singer, Alexander} (16.11.1841 – 30.11.1906), \emph{Herausgeber, Administrator}|pw}. {\pb}Sie hoffentlich auch.\pend
           \pstart
           Herzlichst{\\[\baselineskip]}\spacefill\mbox{Richard}\pend
           \leftskip=0em{}\endnumbering\briefempfaengerindex{Schnitzler, Arthur@\textsc{Schnitzler, Arthur}!zzzBeer-Hofmann, Richard@\emph{von Richard Beer-Hofmann}!1892-12-281@{{[}28. 12. 1892?{]}}|)be}\mylabel{h}\end{ledgroupsized}  \newcommand{\dateiname}{L00148}\newcommand{\titel}{Richard Beer-Hofmann an Arthur Schnitzler, [28. 12. 1892?]}\newcommand{\editorInnen}{Martin Anton Müller und Gerd-Hermann Susen}
            \footnotesize
\begin{ledgroupsized}[t]{11.5cm}
\doendnotes{C}
\end{ledgroupsized}
         %% latex-leseansicht-abspann.tex
%% Abspann für die Leseansicht.
%% Der Schalter \ifkorrekturansicht ist bereits durch den Vorspann gesetzt.

%% latex-abspann.tex
%% Gemeinsamer Abspann für Korrekturansicht und Leseansicht.
%% Setzt den Schalter \ifkorrekturansicht voraus (gesetzt in den
%% einbindenden Dateien latex-korrekturansicht-abspann.tex bzw.
%% latex-leseansicht-abspann.tex).
%% ---------------------------------------------------------------

\normalsize

% Das esempio-Environment wird nur in der Leseansicht benötigt
\ifkorrekturansicht\else
\newenvironment{esempio}[3]%
{
    \vspace{1.5ex}
    \rlap{\underline{#1}}
    \par
    \setlength{\parindent}{0cm}
    \nopagebreak
    \leftskip=#2cm
    \rightskip=#3cm
}
{
    \par
}
\fi

\doendnotes{C}
\bigskip
\vfill

\clearpage

\footnotesize

\ifkorrekturansicht
  \lohead{\textsc{register}}
\fi

% theindex-Environment neu definieren ohne reledmac
\makeatletter
\renewenvironment{theindex}{%
  \ifkorrekturansicht
    \section*{\indexname}%
  \else
    \subsubsection*{Index der erwähnten Entitäten}%
  \fi
  \setlength{\parindent}{0pt}%
  \setlength{\parskip}{0pt plus 0.3pt}%
  \let\item\@idxitem
}{%
  \ifkorrekturansicht\clearpage\fi
}
\makeatother

\IfFileExists{\jobname-pw.ind}{\input{\jobname-pw.ind}}{}

% Quellenangabe nur in der Leseansicht
\ifkorrekturansicht\else
% Fallback-Definitionen, falls die .tex-Datei \titel etc. nicht gesetzt hat
\providecommand{\titel}{}
\providecommand{\editorInnen}{}
\providecommand{\dateiname}{\jobname}

\vspace{3cm}

\vfill

\footnotesize
\textsc{Quelle}: \titel. Herausgegeben von {\editorInnen}. In: \emph{Arthur Schnitzler: Briefwechsel mit Autorinnen und Autoren}.
 Digitale Edition, https://schnitzler-briefe.acdh.oeaw.ac.at/{\dateiname}.html (Stand \today)
\fi

\end{document}


      