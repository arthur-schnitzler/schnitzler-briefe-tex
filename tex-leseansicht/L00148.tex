%% latex-korrekturansicht-vorspann.tex
%% Vorspann für die Korrekturansicht.
%% Lädt die gemeinsame Datei latex-vorspann.tex mit gesetztem Schalter.

\newif\ifkorrekturansicht
\korrekturansichttrue

\input{../tex-inputs/latex-vorspann}


\section[Richard Beer-Hofmann an Arthur Schnitzler, {[}28. 12. 1892?{]}]{L00148 Richard Beer-Hofmann an Arthur Schnitzler, {[}28. 12. 1892?{]}}
\nopagebreak\mylabel{L00148v}
\rehead{ }\normalsize\beginnumbering\briefempfaengerindex{Schnitzler, Arthur@\textsc{Schnitzler, Arthur}!zzzBeer-Hofmann, Richard@\emph{von Richard Beer-Hofmann}!1892-12-281@{{[}28. 12. 1892?{]}}|(be}
\toendnotes[C]{\smallbreak\pagebreak[2]}\Standort{CUL, Schnitzler, B 8.}
\physDesc{Brief, 1 Blatt, 2 Seiten, 135 Zeichen
\newline{}Handschrift: blauer Buntstift, lateinische Kurrent
\newline{}Schnitzler: mit Bleistift nummeriert: »15« }\toendnotes[C]{\smallbreak}
\pstart{}{\pb}Lieber Arthur!\pend\vspace{0.5em}
\pstart
           Frau Flegmann\pwindex{Flegmann, Bertha 27.05.1852 – 24.6.1933@\textsc{Flegmann, Bertha} (27.05.1852 – 24.6.1933), \emph{männliche Salonnière/Salonnière}|pw} hat \uline{uns} für nächsten \label{K_L00148-1v}\edtext{Freitag}{\lemma{\textnormal{\emph{Freitag}}}\Cendnote{\textnormal{Der Brief ist undatiert. Am
                  23. 12. [1892] wird im Brief Hofmannsthals\pwindex{Hofmannsthal, Hugo von 1874-02-01 – 1929-07-15@\textsc{Hofmannsthal, Hugo von} (1874-02-01 – 1929-07-15), \emph{Schriftsteller/Schriftstellerin}|pwk} an Schnitzler{ }\emph{Aspasia}\pwindex{Aspasia@\emph{Aspasia}|pwk} erwähnt. Es dürfte sich um
                  Vorbereitungen zu einer Privataufführung bei Bertha Flegmann\pwindex{Flegmann, Bertha 27.05.1852 – 24.6.1933@\textsc{Flegmann, Bertha} (27.05.1852 – 24.6.1933), \emph{männliche Salonnière/Salonnière}|pwk} gehandelt haben. Da Schnitzler am Donnerstag, dem 29. 12. 1892 für den folgenden
                  Tag ein Treffen bei Flegmann\pwindex{Flegmann, Bertha 27.05.1852 – 24.6.1933@\textsc{Flegmann, Bertha} (27.05.1852 – 24.6.1933), \emph{männliche Salonnière/Salonnière}|pwk} absagt,
                  scheint dieses Korrespondenzstück der wahrscheinliche Vorgänger desselben zu
                  sein.}}}\label{K_L00148-1} eingeladen (Aspasia\pwindex{Aspasia@\emph{Aspasia}|pw}) ich
               refusire daher Singer\pwindex{Singer, Alexander 16.11.1841 – 30.11.1906@\textsc{Singer, Alexander} (16.11.1841 – 30.11.1906), \emph{Herausgeber/Herausgeberin, Administrator/Administratorin}|pw}. {\pb}Sie hoffentlich auch.\pend
           
\pstart
           Herzlichst{\\[\baselineskip]}\spacefill\mbox{Richard}\pend
           \leftskip=0em{}\selectlanguage{ngerman}\endnumbering\briefempfaengerindex{Schnitzler, Arthur@\textsc{Schnitzler, Arthur}!zzzBeer-Hofmann, Richard@\emph{von Richard Beer-Hofmann}!1892-12-281@{{[}28. 12. 1892?{]}}|)be}\mylabel{L00148h}  \normalsize

\doendnotes{C}
\bigskip
\vfill

\clearpage

\footnotesize

\lohead{\textsc{register}}

% Definiere theindex-Environment komplett neu ohne reledmac
\makeatletter
\renewenvironment{theindex}{%
  \section*{\indexname}%
  \setlength{\parindent}{0pt}%
  \setlength{\parskip}{0pt plus 0.3pt}%
  \let\item\@idxitem
}{%
  \clearpage
}
\makeatother

\IfFileExists{\jobname-pw.ind}{\input{\jobname-pw.ind}}{}

\end{document}

      