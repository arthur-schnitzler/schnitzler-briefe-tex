%% latex-leseansicht-vorspann.tex
%% Vorspann für die Leseansicht.
%% Lädt die gemeinsame Datei latex-vorspann.tex mit nicht gesetztem Schalter.

\newif\ifkorrekturansicht
\korrekturansichtfalse

\input{../tex-inputs/latex-vorspann}


\section[Arthur Schnitzler an Stefan Zweig, 4. 11. 1929]{L03734 Arthur Schnitzler an Stefan Zweig, 4. 11. 1929}
\nopagebreak\mylabel{L03734v}
\rehead{ }\normalsize\beginnumbering\briefempfaengerindex{Zweig, Stefan@\textsc{Zweig, Stefan}!zzzSchnitzler, Arthur@\emph{von Arthur Schnitzler}!1929-11-041@{4. 11. 1929}|(be}
\toendnotes[C]{\smallbreak\pagebreak[2]}
\correspDesc{Versand  durch Arthur Schnitzler am 4. 11. 1929 in Wien
\newline{}Erhalt  durch Stefan Zweig im Zeitraum [5. 11. 1929
                  – 6. 11. 1929?] in Salzburg}\toendnotes[C]{\smallbreak}
\Standort{Jerusalem, National Library of Israel, ARC. Ms. Var. 305 1 58 Stefan Zweig Collection.}
\physDesc{Brief, 1 Blatt, 1 Seite, 1322 Zeichen
\newline{}Schreibmaschine
\newline{}Handschrift: Bleistift (\noindent{}Unterschrift, Ergänzung eines Buchstabens und eine Streichung)}\toendnotes[C]{\smallbreak}
\pstart
           {\pb}\textcolor{gray}{\textbf{D\textsuperscript{R} ARTHUR SCHNITZLER}}\hfill 4. 11. 1929.\pend
           
\pstart
           \textcolor{gray}{\textbf{WIEN, XVIII.
                        STERNWARTESTRASSE 71\oindex{Wien@\textbf{Wien}!XVIII., Währing@\textbf{XVIII., Währing}!Sternwartestraße 71@\textbf{Sternwartestraße 71}, \emph{Wohngebäude}|pw}.}}\pend
           
\pstart{}Lieber und verehrter Stefan Zweig.\pend\vspace{0.5em}
\pstart
           Besten Dank für Ihre Mitteilung Herrn A. del
                  Vayo\pwindex{Álvarez del Vayo, Julio 9.\,2.\,1891 Villaviciosa de Odón – 3.\,5.\,1975 Genf@\textsc{Álvarez del Vayo, Julio} (9.\,2.\,1891 Villaviciosa de Odón – 3.\,5.\,1975 Genf), \emph{Schriftsteller, Politiker, Journalist}|pw} betreffend. Er möge sich direkt an mich wenden. Können Sie mir
               vielleicht sagen, was für Honorare er zahlt? Bei Fischer\pwindex{Fischer, Samuel 24.\,12.\,1859 Liptovský Mikuláš – 15.\,10.\,1934 Berlin@\textsc{Fischer, Samuel} (24.\,12.\,1859 Liptovský Mikuláš – 15.\,10.\,1934 Berlin), \emph{Verleger}|pw} werde ich reklamieren. In Spanien\oindex{Spanien@\textbf{Spanien}|pw} ist ja verhältnismässig recht wenig von mir erschienen – so weit ich
               darüber informiert bin.\pend
           
\pstart
           Ich freue mich auf das versprochene neue Buch\pwindex{Zweig, Stefan 28.\,11.\,1881 Wien – 23.\,2.\,1942 Petrópolis@\textsc{Zweig, Stefan} (28.\,11.\,1881 Wien – 23.\,2.\,1942 Petrópolis), \emph{Schriftsteller}!Kleine Chronik@\strich\emph{Kleine Chronik}|pwv} und beglückwünsche Sie noch einmal zu dem
               ausserordentlichen »Fouché\pwindex{Zweig, Stefan 28.\,11.\,1881 Wien – 23.\,2.\,1942 Petrópolis@\textsc{Zweig, Stefan} (28.\,11.\,1881 Wien – 23.\,2.\,1942 Petrópolis), \emph{Schriftsteller}!Joseph Fouché. Bildnis eines politischen Menschen@\strich\emph{Joseph Fouché. Bildnis eines politischen Menschen}|pw}«, dessen Erfolg sich,
               wie ich mit Vergnügen höre und lese, in Nähe und Ferne immer glänzender bestätigt.\pend
           
\pstart
           Neulich hat man mir aus Paris\oindex{Paris@\textbf{Paris}, \emph{Hauptstadt}|pw} einen \label{K_L03734-1v}\edtext{Ausschnitt\pwindex{?? [Peur, Film von Arthur Schnitzler]@\emph{?? [Peur, Film von Arthur Schnitzler]}|pwv}}{\lemma{\textnormal{\emph{Ausschnitt}}}\Cendnote{\textnormal{Obzwar Schnitzler in Folge die Zeitschrift, in der die Notiz\pwindex{?? [Peur, Film von Arthur Schnitzler]@\emph{?? [Peur, Film von Arthur Schnitzler]}|pwkv} stand, als \emph{Gringoire}\pwindex{Gringoire@\emph{Gringoire}|pwk} spezifiziert, konnte die betreffende Stelle
                   nicht nachgewiesen werden.}}}\label{K_L03734-1} geschickt, in dem eine Kinovorstellung
               besprochen war »Peur\pwindex{\textcolor{red}{\textsuperscript{XXXX indx1}}!Angst@\strich\emph{Angst}|pw}« d’apres la nouvelle\pwindex{Zweig, Stefan 28.\,11.\,1881 Wien – 23.\,2.\,1942 Petrópolis@\textsc{Zweig, Stefan} (28.\,11.\,1881 Wien – 23.\,2.\,1942 Petrópolis), \emph{Schriftsteller}!Angst@\strich\emph{Angst}|pwv} de M. Arthur Schnitzler. Nach dem
               Inhalt muss es sich um die »Angst\pwindex{\textcolor{red}{\textsuperscript{XXXX indx1}}!Angst@\strich\emph{Angst}|pw}« gehandelt
               haben, die ich selbst hier in einem \label{K_L03734-2v}\edtext{Kino\oindex{Wien@\textbf{Wien}!I., Innere Stadt@\textbf{I., Innere Stadt}!Gartenbaukino@\textbf{Gartenbaukino}, \emph{Kino}|pwuv}\oindex{Wien@\textbf{Wien}!I., Innere Stadt@\textbf{I., Innere Stadt}!Imperialkino@\textbf{Imperialkino}, \emph{Kino}|pwuv}}{\lemma{\textnormal{\emph{Kino}}}\Cendnote{\textnormal{Schnitzler und Clara Katharina Pollaczek\pwindex{Pollaczek, Clara Katharina 15.\,1.\,1875 Wien – 22.\,7.\,1951 ebd.@\textsc{Pollaczek, Clara Katharina} (15.\,1.\,1875 Wien – 22.\,7.\,1951 ebd.), \emph{Schriftstellerin}|pwk} sahen den Film\pwindex{\textcolor{red}{\textsuperscript{XXXX indx1}}!Angst@\strich\emph{Angst}|pwkv} am 14. 6. 1929 entweder im Imperialkino\oindex{Wien@\textbf{Wien}!I., Innere Stadt@\textbf{I., Innere Stadt}!Imperialkino@\textbf{Imperialkino}, \emph{Kino}|pwk} oder im Gartenbaukino\oindex{Wien@\textbf{Wien}!I., Innere Stadt@\textbf{I., Innere Stadt}!Gartenbaukino@\textbf{Gartenbaukino}, \emph{Kino}|pwk}.}}}\label{K_L03734-2} gesehen habe. Die Notiz\pwindex{?? [Peur, Film von Arthur Schnitzler]@\emph{?? [Peur, Film von Arthur Schnitzler]}|pwv} stand im »Gringoir{[}e{]}\pwindex{Gringoire@\emph{Gringoire}|pw}«; meine weiteren Recherchen sind noch ohne Erfolg geblieben.\pend
           
\pstart
           Es wäre schön, wenn ich Sie wieder einmal sprechen könnte. Dass Sie das letzte Mal in
                  Wien\oindex{Wien@\textbf{Wien}, \emph{Verwaltungsgebiet}|pw} keine Zeit hatten ist \strikeout{ja} natürlich und Sie, lieber Stefan Zweig, haben mir
               sicher verziehen, dass ich bei der \label{K_L03734-3v}\edtext{Trauerfeier für Hofmannsthal\pwindex{Hofmannsthal, Hugo von 1.\,2.\,1874 Wien – 15.\,7.\,1929 Rodaun@\textsc{Hofmannsthal, Hugo von} (1.\,2.\,1874 Wien – 15.\,7.\,1929 Rodaun), \emph{Schriftsteller}|pw}\eventindex{Burgtheater@\textbf{Burgtheater}!Gedenkfeier für Hugo von Hofmannsthal, 13.10.1929@Gedenkfeier für Hugo von Hofmannsthal, 13.10.1929|pw}}{\lemma{\textnormal{\emph{Trauerfeier für Hofmannsthal}}}\Cendnote{\textnormal{Am 13. 10. 1929 fand im Burgtheater\oindex{Wien@\textbf{Wien}!I., Innere Stadt@\textbf{I., Innere Stadt}!Burgtheater@\textbf{Burgtheater}, \emph{Theater}|pwk}
                  eine Gedenkfeier\eventindex{Burgtheater@\textbf{Burgtheater}!Gedenkfeier für Hugo von Hofmannsthal, 13.10.1929@Gedenkfeier für Hugo von Hofmannsthal, 13.10.1929|pwk} für Hugo von Hofmannsthal\pwindex{Hofmannsthal, Hugo von 1.\,2.\,1874 Wien – 15.\,7.\,1929 Rodaun@\textsc{Hofmannsthal, Hugo von} (1.\,2.\,1874 Wien – 15.\,7.\,1929 Rodaun), \emph{Schriftsteller}|pwk} statt, bei der \emph{Der Thor und der Tod}\pwindex{Hofmannsthal, Hugo von 1.\,2.\,1874 Wien – 15.\,7.\,1929 Rodaun@\textsc{Hofmannsthal, Hugo von} (1.\,2.\,1874 Wien – 15.\,7.\,1929 Rodaun), \emph{Schriftsteller}!Thor und der Tod@\strich\emph{Der Thor und der Tod}|pwk} gespielt und von Stefan Zweig\pwindex{Zweig, Stefan 28.\,11.\,1881 Wien – 23.\,2.\,1942 Petrópolis@\textsc{Zweig, Stefan} (28.\,11.\,1881 Wien – 23.\,2.\,1942 Petrópolis), \emph{Schriftsteller}|pwk} eine \emph{Gedächtnisrede}\pwindex{Zweig, Stefan 28.\,11.\,1881 Wien – 23.\,2.\,1942 Petrópolis@\textsc{Zweig, Stefan} (28.\,11.\,1881 Wien – 23.\,2.\,1942 Petrópolis), \emph{Schriftsteller}!Hugo von Hofmannsthal. Gedächtnisrede zur Trauerfeier im Wiener Burgtheater@\strich\emph{Hugo von Hofmannsthal. Gedächtnisrede zur Trauerfeier im Wiener Burgtheater}|pwk} gehalten wurde.}}}\label{K_L03734-3} nicht im Theater\oindex{Wien@\textbf{Wien}!I., Innere Stadt@\textbf{I., Innere Stadt}!Burgtheater@\textbf{Burgtheater}, \emph{Theater}|pwv} war und so Ihre Rede\pwindex{Zweig, Stefan 28.\,11.\,1881 Wien – 23.\,2.\,1942 Petrópolis@\textsc{Zweig, Stefan} (28.\,11.\,1881 Wien – 23.\,2.\,1942 Petrópolis), \emph{Schriftsteller}!Hugo von Hofmannsthal. Gedächtnisrede zur Trauerfeier im Wiener Burgtheater@\strich\emph{Hugo von Hofmannsthal. Gedächtnisrede zur Trauerfeier im Wiener Burgtheater}|pwv} nicht gehört habe. Man hat mir
               erzählt, wie schön Sie gesprochen haben.\pend
           
\pstart
           Mit den herzlichsten Grüssen{\\[\baselineskip]}Ihr freundschaftlich ergebener{\\[\baselineskip]}\spacefill\mbox{{[}hs.:{]} ArthSchnitzler}\pend
           \leftskip=0em{}
\pstart
           \noindent{}{[}ms.:{]} Herrn Stefan Zweig\pend
           
\pstart
           \noindent{}Salzburg\oindex{Salzburg@\textbf{Salzburg}, \emph{Verwaltungsgebiet}|pw}.\pend
           \selectlanguage{ngerman}\endnumbering\briefempfaengerindex{Zweig, Stefan@\textsc{Zweig, Stefan}!zzzSchnitzler, Arthur@\emph{von Arthur Schnitzler}!1929-11-041@{4. 11. 1929}|)be}\mylabel{L03734h}  \newcommand{\dateiname}{L03734}\newcommand{\titel}{Arthur Schnitzler an Stefan Zweig, 4. 11. 1929}\newcommand{\editorInnen}{Selma Jahnke und Martin Anton Müller}%% latex-leseansicht-abspann.tex
%% Abspann für die Leseansicht.
%% Der Schalter \ifkorrekturansicht ist bereits durch den Vorspann gesetzt.

%% latex-abspann.tex
%% Gemeinsamer Abspann für Korrekturansicht und Leseansicht.
%% Setzt den Schalter \ifkorrekturansicht voraus (gesetzt in den
%% einbindenden Dateien latex-korrekturansicht-abspann.tex bzw.
%% latex-leseansicht-abspann.tex).
%% ---------------------------------------------------------------

\normalsize

% Das esempio-Environment wird nur in der Leseansicht benötigt
\ifkorrekturansicht\else
\newenvironment{esempio}[3]%
{
    \vspace{1.5ex}
    \rlap{\underline{#1}}
    \par
    \setlength{\parindent}{0cm}
    \nopagebreak
    \leftskip=#2cm
    \rightskip=#3cm
}
{
    \par
}
\fi

\doendnotes{C}
\bigskip
\vfill

\clearpage

\footnotesize

\ifkorrekturansicht
  \lohead{\textsc{register}}
\fi

% theindex-Environment neu definieren ohne reledmac
\makeatletter
\renewenvironment{theindex}{%
  \ifkorrekturansicht
    \section*{\indexname}%
  \else
    \subsubsection*{Index der erwähnten Entitäten}%
  \fi
  \setlength{\parindent}{0pt}%
  \setlength{\parskip}{0pt plus 0.3pt}%
  \let\item\@idxitem
}{%
  \ifkorrekturansicht\clearpage\fi
}
\makeatother

\IfFileExists{\jobname-pw.ind}{\input{\jobname-pw.ind}}{}

% Quellenangabe nur in der Leseansicht
\ifkorrekturansicht\else
% Fallback-Definitionen, falls die .tex-Datei \titel etc. nicht gesetzt hat
\providecommand{\titel}{}
\providecommand{\editorInnen}{}
\providecommand{\dateiname}{\jobname}

\vspace{3cm}

\vfill

\footnotesize
\textsc{Quelle}: \titel. Herausgegeben von {\editorInnen}. In: \emph{Arthur Schnitzler: Briefwechsel mit Autorinnen und Autoren}.
 Digitale Edition, https://schnitzler-briefe.acdh.oeaw.ac.at/{\dateiname}.html (Stand \today)
\fi

\end{document}


