%% latex-korrekturansicht-vorspann.tex
%% Vorspann für die Korrekturansicht.
%% Lädt die gemeinsame Datei latex-vorspann.tex mit gesetztem Schalter.

\newif\ifkorrekturansicht
\korrekturansichttrue

\input{../tex-inputs/latex-vorspann}


\section[Arthur Schnitzler an Stefan Zweig, 4. 11. 1929]{L03734 Arthur Schnitzler an Stefan Zweig, 4. 11. 1929}
\nopagebreak\mylabel{L03734v}
\rehead{ }\normalsize\beginnumbering\briefempfaengerindex{Zweig, Stefan@\textsc{Zweig, Stefan}!zzzSchnitzler, Arthur@\emph{von Arthur Schnitzler}!1929-11-041@{4. 11. 1929}|(be}
\toendnotes[C]{\smallbreak\pagebreak[2]}\Standort{Jerusalem, National Library of Israel, ARC. Ms. Var. 305 1 58 Stefan Zweig Collection.}
\physDesc{Brief, 1 Blatt, 1 Seite, 1321 Zeichen
\newline{}Schreibmaschine
\newline{}Handschrift: Bleistift (\noindent{}Unterschrift, Ergänzung eines Buchstabens und eine Streichung)}\toendnotes[C]{\smallbreak}
\pstart
           {\pb}\textcolor{gray}{\textbf{D\textsuperscript{R} ARTHUR SCHNITZLER}}\hfill 4. 11. 1929.\pend
           
\pstart
           \textcolor{gray}{\textbf{WIEN, XVIII. STERNWARTESTRASSE 71\oindex{Sternwartestrasse 71@\textbf{Sternwartestraße 71}, \emph{Wohngebäude (K.WHS)}|pw}.}}\pend
           
\pstart{}Lieber und verehrter Stefan Zweig.\pend\vspace{0.5em}
\pstart
           Besten Dank für Ihre Mitteilung Herrn A. del Vayo\pwindex{Álvarez del Vayo, Julio 1891-02-09 – 1975-05-03@\textsc{Álvarez del Vayo, Julio} (1891-02-09 – 1975-05-03), \emph{Schriftsteller/Schriftstellerin, Politiker/Politikerin, Journalist/Journalistin}|pw}
          betreffend. Er möge sich direkt an mich wenden. Können Sie mir vielleicht sagen, was für
          Honorare er zahlt? Bei Fischer\pwindex{Fischer, Samuel 24.12.1859 – 15.10.1934@\textsc{Fischer, Samuel} (24.12.1859 – 15.10.1934), \emph{Verleger/Verlegerin}|pw} werde ich
          reklamieren. In Spanien\oindex{Spanien@\textbf{Spanien}, \emph{A.PCLI}|pw} ist ja verhältnismässig recht
          wenig von mir erschienen – so weit ich darüber informiert bin. \pend
           
\pstart
           Ich freue mich auf das versprochene neue Buch\pwindex{Kleine Chronik@\emph{Kleine Chronik}|pwv} und beglückwünsche Sie noch einmal zu dem ausserordentlichen
            »Fouché\pwindex{Joseph Fouche. Bildnis eines politischen Menschen@\emph{Joseph Fouché. Bildnis eines politischen Menschen}|pw}«, dessen Erfolg sich, wie ich mit Vergnügen
          höre und lese, in Nähe und Ferne immer glänzender bestätigt. \pend
           
\pstart
           Neulich hat man mir aus Paris\oindex{Paris@\textbf{Paris}, \emph{P.PPLC}|pw} einen \label{K_L03734-33v}\edtext{Ausschnitt\pwindex{?? [Peur, Film von Arthur Schnitzler]@\emph{?? [Peur, Film von Arthur Schnitzler]}|pwv}}{\lemma{\textnormal{\emph{Ausschnitt}}}\Cendnote{\textnormal{Obzwar Schnitzler in Folge die 
              Zeitschrift, in der die Notiz\pwindex{?? [Peur, Film von Arthur Schnitzler]@\emph{?? [Peur, Film von Arthur Schnitzler]}|pwkv} stand, als \emph{Gringoire}\pwindex{Gringoire@\emph{Gringoire}|pwk} spezifiziert, konnte die betreffende Stelle bislang nicht nachgewiesen werden.}}}\label{K_L03734-33} geschickt, in dem eine
          Kinovorstellung besprochen war »Peur\pwindex{Angst@\emph{Angst}|pw}« d’apres la nouvelle\pwindex{Angst@\emph{Angst}|pwv} de M. Arthur Schnitzler.
          Nach dem Inhalt muss es sich um die »Angst\pwindex{Angst@\emph{Angst}|pw}« gehandelt
          haben, die ich selbst hier in einem \label{K_L03734-1v}\edtext{Kino\oindex{Gartenbaukino@\textbf{Gartenbaukino}, \emph{Kino (K.KNO)}|pwuv}\oindex{Imperialkino@\textbf{Imperialkino}, \emph{Kino (K.KNO)}|pwuv}}{\lemma{\textnormal{\emph{Kino}}}\Cendnote{\textnormal{Schnitzler und Clara Katharina Pollaczek\pwindex{Pollaczek, Clara Katharina 15.01.1875 – 22.07.1951@\textsc{Pollaczek, Clara Katharina} (15.01.1875 – 22.07.1951), \emph{Schriftsteller/Schriftstellerin}|pwk} sahen den Film\pwindex{Angst@\emph{Angst}|pwkv} am 14. 6. 1929 entweder im Imperialkino\oindex{Imperialkino@\textbf{Imperialkino}, \emph{Kino (K.KNO)}|pwk}
            oder im Gartenbaukino\oindex{Gartenbaukino@\textbf{Gartenbaukino}, \emph{Kino (K.KNO)}|pwk}.}}}\label{K_L03734-1} gesehen habe. Die
            Notiz\pwindex{?? [Peur, Film von Arthur Schnitzler]@\emph{?? [Peur, Film von Arthur Schnitzler]}|pwv} stand im »Gringoir{[}e{]}\pwindex{Gringoire@\emph{Gringoire}|pw}«; meine weiteren Recherchen sind
          noch ohne Erfolg geblieben.\pend
           
\pstart
           Es wäre schön, wenn ich Sie wieder einmal sprechen könnte. Dass Sie das letzte Mal in Wien\oindex{Wien@\textbf{Wien}, \emph{A.ADM2}|pw} keine Zeit hatten ist \strikeout{ja} natürlich und Sie, lieber Stefan Zweig, haben mir sicher verziehen, dass ich
          bei der \label{K_L03734-2v}\edtext{Trauerfeier für Hofmannsthal\pwindex{Hofmannsthal, Hugo von 1874-02-01 – 1929-07-15@\textsc{Hofmannsthal, Hugo von} (1874-02-01 – 1929-07-15), \emph{Schriftsteller/Schriftstellerin}|pw}\eventindex{Burgtheater@\textbf{Burgtheater}!Gedenkfeier fuer Hugo von Hofmannsthal, 13.10.1929@Gedenkfeier für Hugo von Hofmannsthal, 13.10.1929|pw}}{\lemma{\textnormal{\emph{Trauerfeier für Hofmannsthal}}}\Cendnote{\textnormal{Am 13. 10. 1929 fand im Burgtheater\oindex{Burgtheater@\textbf{Burgtheater}, \emph{S.THTR}|pwk} eine \emph{Gedenkfeier}\eventindex{Burgtheater@\textbf{Burgtheater}!Gedenkfeier fuer Hugo von Hofmannsthal, 13.10.1929@Gedenkfeier für Hugo von Hofmannsthal, 13.10.1929|pwk} für Hugo von Hofmannsthal\pwindex{Hofmannsthal, Hugo von 1874-02-01 – 1929-07-15@\textsc{Hofmannsthal, Hugo von} (1874-02-01 – 1929-07-15), \emph{Schriftsteller/Schriftstellerin}|pwk} statt, bei der \emph{Der Thor
              und der Tod}\pwindex{Thor und der Tod@\emph{Der Thor und der Tod}|pwk} gespielt und von Stefan Zweig\pwindex{Zweig, Stefan 28.11.1881 – 23.02.1942@\textsc{Zweig, Stefan} (28.11.1881 – 23.02.1942), \emph{Schriftsteller/Schriftstellerin}|pwk}
            eine \emph{Gedächtnisrede}\pwindex{Hugo von Hofmannsthal. Gedaechtnisrede zur Trauerfeier im Wiener Burgtheater@\emph{Hugo von Hofmannsthal. Gedächtnisrede zur Trauerfeier im Wiener Burgtheater}|pwk} gehalten wurde.}}}\label{K_L03734-2} nicht
          im Theater\oindex{Burgtheater@\textbf{Burgtheater}, \emph{S.THTR}|pwv} war und so Ihre Rede\pwindex{Hugo von Hofmannsthal. Gedaechtnisrede zur Trauerfeier im Wiener Burgtheater@\emph{Hugo von Hofmannsthal. Gedächtnisrede zur Trauerfeier im Wiener Burgtheater}|pwv} nicht gehört habe. Man hat mir
          erzählt, wie schön Sie gesprochen haben. \pend
           
\pstart
           Mit den herzlichsten Grüssen{\\[\baselineskip]}Ihr freundschaftlich ergebener{\\[\baselineskip]}\spacefill\mbox{{[}hs.:{]} ArthSchnitzler}\pend
           \leftskip=0em{}
\pstart
           \noindent{}Herrn Stefan Zweig\pend
           
\pstart
           \noindent{}Salzburg\oindex{Salzburg@\textbf{Salzburg}, \emph{A.ADM2}|pw}.\pend
           \selectlanguage{ngerman}\endnumbering\briefempfaengerindex{Zweig, Stefan@\textsc{Zweig, Stefan}!zzzSchnitzler, Arthur@\emph{von Arthur Schnitzler}!1929-11-041@{4. 11. 1929}|)be}\mylabel{L03734h}
\begin{anhang}
\end{anhang}\normalsize

\doendnotes{C}
\bigskip
\vfill

\clearpage

\footnotesize

\lohead{\textsc{register}}

% Definiere theindex-Environment komplett neu ohne reledmac
\makeatletter
\renewenvironment{theindex}{%
  \section*{\indexname}%
  \setlength{\parindent}{0pt}%
  \setlength{\parskip}{0pt plus 0.3pt}%
  \let\item\@idxitem
}{%
  \clearpage
}
\makeatother

\IfFileExists{\jobname-pw.ind}{\input{\jobname-pw.ind}}{}

\end{document}

      