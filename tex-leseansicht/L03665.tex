%% latex-leseansicht-vorspann.tex
%% Vorspann für die Leseansicht.
%% Lädt die gemeinsame Datei latex-vorspann.tex mit nicht gesetztem Schalter.

\newif\ifkorrekturansicht
\korrekturansichtfalse

\input{../tex-inputs/latex-vorspann}


\section[Stefan Zweig an Arthur Schnitzler, 13. 5. 1923]{L03665 Stefan Zweig an Arthur Schnitzler, 13. 5. 1923}
\nopagebreak\mylabel{L03665v}
\rehead{ }\normalsize\beginnumbering\briefempfaengerindex{Schnitzler, Arthur@\textsc{Schnitzler, Arthur}!zzzZweig, Stefan@\emph{von Stefan Zweig}!1923-05-131@{13. 5. 1923}|(be}
\toendnotes[C]{\smallbreak\pagebreak[2]}
\correspDesc{Versand  durch Stefan Zweig am 13. 5. 1923 in Salzburg
\newline{}Zustellung  im Zeitraum [14. 5. 1923
                  – 18. 5. 1923?] in Wien
\newline{}Erhalt  durch Arthur Schnitzler am 27. 3. 1929 in Wien}\toendnotes[C]{\smallbreak}
\Standort{CUL, Schnitzler, B 118.}
\physDesc{Brief, 1 Blatt, 1 Seite, 613 Zeichen
\newline{}Handschrift: blaue Tinte, lateinische Kurrent
\newline{}Beilage: \emph{Deutschen Literaturarchiv Marbach},
                                    HS.NZ85.1.4978: Ausschnitt mit den Seiten 60 und 61 aus
                                 dem Antiquariatskatalog von Emil
                                    Hirsch\orgindex{Antiquariat Emil Hirsch@Antiquariat Emil Hirsch|pw}, 1 Blatt, 2 Seiten. Die Angabe der Lot-Nummer und
                                 die Adresse des Antiquariats mit Bleistift unterstrichen. Der
                                 Hinweis auf Schnitzler in der Beschreibung mit blauem Buntstift
                                 (von Zweig?) unterstrichen. Auf der ersten Seite mit rotem
                                 Buntstift Vermerk von Schnitzler: »\textsc{Krell}\pwindex{Krell, Max 24.\,9.\,1887 Hubertusburg – 11.\,6.\,1962 Florenz@\textsc{Krell, Max} (24.\,9.\,1887 Hubertusburg – 11.\,6.\,1962 Florenz), \emph{Schriftsteller, Verlagslektor}|pw}«. 
\newline{}Schnitzler: mit rotem Buntstift eine Unterstreichung }
\buchAbdrucke{\weitereDrucke{Stefan Zweig: \emph{Briefwechsel mit Hermann Bahr, Sigmund Freud, Rainer Maria
                        Rilke und Arthur Schnitzler}. Herausgegeben von Jeffrey B. Berlin, Hans-Ulrich Lindken und Donald A. Prater. Frankfurt am Main: \emph{S. Fischer} 1987, S. 414.} }\toendnotes[C]{\smallbreak}
\pstart
           {\pb}\uline{Salzburg}, Kapuzinerberg 5\oindex{Paschinger Schlössl@\textbf{Paschinger Schlössl}, \emph{Wohngebäude}|pw}\hfill 13. Mai 1923\pend
           \vspace{0.5em}
\pstart
           Lieber verehrter Herr Doktor, in einem Versteigerungskatalog
               entdecke ich eben dieses \label{K_L03665-1v}\edtext{Buch\pwindex{Wassermann, Jakob 10.\,3.\,1873 Fürth – 1.\,1.\,1934 Altaussee@\textsc{Wassermann, Jakob} (10.\,3.\,1873 Fürth – 1.\,1.\,1934 Altaussee), \emph{Schriftsteller}!Gänsemännchen. Roman@\strich\emph{Das Gänsemännchen. Roman}|pwv}}{\lemma{\textnormal{\emph{Buch}}}\Cendnote{\textnormal{Am 29. 5. 1923 schrieb Schnitzler an den Antiquar Emil Hirsch\pwindex{Hirsch, Emil 14.\,3.\,1866 Bad Mergentheim – 27.\,7.\,1954 New York City@\textsc{Hirsch, Emil} (14.\,3.\,1866 Bad Mergentheim – 27.\,7.\,1954 New York City), \emph{Verleger, Antiquar}|pwk}: »29. 5. 1923.{ / }Sehr geehrter Herr.{ / }In ihrem Versteigerungskatalog, der mir von befreundeter Seite zugesandt
                        wird, finde ich auf Seite 60, Nr. 784, Wassermann Jakob\pwindex{Wassermann, Jakob 10.\,3.\,1873 Fürth – 1.\,1.\,1934 Altaussee@\textsc{Wassermann, Jakob} (10.\,3.\,1873 Fürth – 1.\,1.\,1934 Altaussee), \emph{Schriftsteller}|pw}, Das
                           Gänsemännchen\pwindex{Wassermann, Jakob 10.\,3.\,1873 Fürth – 1.\,1.\,1934 Altaussee@\textsc{Wassermann, Jakob} (10.\,3.\,1873 Fürth – 1.\,1.\,1934 Altaussee), \emph{Schriftsteller}!Gänsemännchen. Roman@\strich\emph{Das Gänsemännchen. Roman}|pw} mit handschriftlicher Widmung an Arthur und Olga
                           Schnitzler\pwindex{Schnitzler, Olga 17.\,1.\,1882 Wien – 13.\,1.\,1970 Lugano@\textsc{Schnitzler, Olga} (17.\,1.\,1882 Wien – 13.\,1.\,1970 Lugano), \emph{Schauspielerin, Sängerin}|pw}. Ich ersuche hiemit die Versteigerung dieses Buches, das
                        auf eine mir vorläufig unbegreifliche Weise aus meinem Besitz verschwunden
                        ist, zu unterlassen und das mir gehörige Exemplar an meine Adresse
                        freundlichst rücksenden zu wollen.{ / }Mit vorzüglicher Hochachtung{ / }{[}Raum für Unterschrift{]}{ / }Herrn Emil Hirsch\pwindex{Hirsch, Emil 14.\,3.\,1866 Bad Mergentheim – 27.\,7.\,1954 New York City@\textsc{Hirsch, Emil} (14.\,3.\,1866 Bad Mergentheim – 27.\,7.\,1954 New York City), \emph{Verleger, Antiquar}|pw}, Verleger,{ / }München\oindex{München@\textbf{München}|pw}.« (\emph{Deutsches Literaturarchiv Marbach}, HS.1985.1.1016). Aus den zwei weiteren
                  Schreiben Schnitzlers an Hirsch\pwindex{Hirsch, Emil 14.\,3.\,1866 Bad Mergentheim – 27.\,7.\,1954 New York City@\textsc{Hirsch, Emil} (14.\,3.\,1866 Bad Mergentheim – 27.\,7.\,1954 New York City), \emph{Verleger, Antiquar}|pwk} geht hervor, dass das Exemplar von Max Krell\pwindex{Krell, Max 24.\,9.\,1887 Hubertusburg – 11.\,6.\,1962 Florenz@\textsc{Krell, Max} (24.\,9.\,1887 Hubertusburg – 11.\,6.\,1962 Florenz), \emph{Schriftsteller, Verlagslektor}|pwk} zum Verkauf freigegeben wurde –
                  der es wiederum von Schnitzlers Schwägerin
                     Elisabeth Steinrück\pwindex{Steinrück, Elisabeth 19.\,11.\,1885 – 7.\,4.\,1920 Partenkirchen@\textsc{Steinrück, Elisabeth} (19.\,11.\,1885 – 7.\,4.\,1920 Partenkirchen)|pwk} bezogen hatte. Ob nun
                  diese oder Krell\pwindex{Krell, Max 24.\,9.\,1887 Hubertusburg – 11.\,6.\,1962 Florenz@\textsc{Krell, Max} (24.\,9.\,1887 Hubertusburg – 11.\,6.\,1962 Florenz), \emph{Schriftsteller, Verlagslektor}|pwk} das Buch sich zu Unrecht
                  angeeignet hat, lässt sich nicht mehr bestimmen. Die erhaltene Korrespondenz
                  zwischen Schnitzler und Krell\pwindex{Krell, Max 24.\,9.\,1887 Hubertusburg – 11.\,6.\,1962 Florenz@\textsc{Krell, Max} (24.\,9.\,1887 Hubertusburg – 11.\,6.\,1962 Florenz), \emph{Schriftsteller, Verlagslektor}|pwk} ist im betreffenden Zeitraum ausgesetzt. Das Buch
                  dürfte letztlich an Schnitzler retourniert
                  worden sein, wohingegen er dem Verleger ein von ihm gewidmetes Exemplar der
                  Erstausgabe von \emph{Das Märchen}\pwindex{Schnitzler, Arthur 15.\,5.\,1862 Wien – 21.\,10.\,1931 ebd.@\textsc{Schnitzler, Arthur} (15.\,5.\,1862 Wien – 21.\,10.\,1931 ebd.), \emph{Schriftsteller, Mediziner}!Märchen. Schauspiel in drei Aufzügen@\strich\emph{Das Märchen. Schauspiel in drei Aufzügen}|pwk} zukommen ließ.
               }}}\label{K_L03665-1}. Da ich nicht annehme, dass Sie die Exemplare Ihrer gewidmeten Bücher
               verkaufen (vielleicht werden wir bald so weit sein) so handelt es sich offenbar um
               ein entwendetes Exemplar und Sie haben wohl das Recht es zurückzufordern. Ich glaubte
               Sie aufmerksam machen zu müssen, weil ich selbst jüngst ähnlich einem entwendeten
                  \label{K_L03665-2v}\edtext{Buch\pwindex{?? [Widmungsexemplar eines unbekannten Buchs an Stefan Zweig, 1923]@\emph{?? [Widmungsexemplar eines unbekannten Buchs an Stefan Zweig, 1923]}|pwv}}{\lemma{\textnormal{\emph{Buch}}}\Cendnote{\textnormal{nicht identifiziert}}}\label{K_L03665-2} auf die Spur
               kam – und dann freue ich mich jeder Gelegenheit, Ihnen meine herzliche Verehrung
               aussprechen zu können.\pend
           
\pstart
           Ihr getreuer{\\[\baselineskip]}\spacefill\mbox{Stefan Zweig}\pend
           \leftskip=0em{}\selectlanguage{ngerman}\vspace{1em}
\pstart
           \noindent{}{[}\ldots{]}\pend
           
\pstart
           {\pb}\textcolor{gray}{\textbf{784}}\hspace*{1.5em}\textcolor{gray}{\textbf{\textbf{WASSERMANN}, Jakob\pwindex{Wassermann, Jakob 10.\,3.\,1873 Fürth – 1.\,1.\,1934 Altaussee@\textsc{Wassermann, Jakob} (10.\,3.\,1873 Fürth – 1.\,1.\,1934 Altaussee), \emph{Schriftsteller}|pw}. Das Gänsemännchen.
                     Roman\pwindex{Wassermann, Jakob 10.\,3.\,1873 Fürth – 1.\,1.\,1934 Altaussee@\textsc{Wassermann, Jakob} (10.\,3.\,1873 Fürth – 1.\,1.\,1934 Altaussee), \emph{Schriftsteller}!Gänsemännchen. Roman@\strich\emph{Das Gänsemännchen. Roman}|pw}. Berlin\oindex{Berlin@\textbf{Berlin}, \emph{Hauptstadt}|pw}, S. Fischer\orgindex{S. Fischer Verlag@S. Fischer Verlag|pw}, 1915. 8. Origlwd.}}\pend
           
\pstart
           \textcolor{gray}{\textbf{Erste Ausgabe.{ }}}{ }\textcolor{gray}{\textbf{\so{Mit handschriftl. Widmung}}}{ }\textcolor{gray}{\textbf{{ }des Verf. an Arthur u. Olga Schnitzler.}}\pend
           
\pstart
           {[}\ldots{]}\pend
           
\pstart
           {\pb}{[}\ldots{]}\pend
           
\pstart
           \centering{}\textcolor{gray}{\textbf{\textbf{Emil Hirsch\orgindex{Antiquariat Emil Hirsch@Antiquariat Emil Hirsch|pw}, Karlstr. 10, München\oindex{Karlstraße 10@\textbf{Karlstraße 10}, \emph{Gebäude}|pw}.}}}\pend
           
\pstart
           \raggedleft{}Versteigerung 4. Juni\pend
           \selectlanguage{ngerman}\endnumbering\briefempfaengerindex{Schnitzler, Arthur@\textsc{Schnitzler, Arthur}!zzzZweig, Stefan@\emph{von Stefan Zweig}!1923-05-131@{13. 5. 1923}|)be}\mylabel{L03665h}  \newcommand{\dateiname}{L03665}\newcommand{\titel}{Stefan Zweig an Arthur Schnitzler, 13. 5. 1923}\newcommand{\editorInnen}{Selma Jahnke und Martin Anton Müller}%% latex-leseansicht-abspann.tex
%% Abspann für die Leseansicht.
%% Der Schalter \ifkorrekturansicht ist bereits durch den Vorspann gesetzt.

%% latex-abspann.tex
%% Gemeinsamer Abspann für Korrekturansicht und Leseansicht.
%% Setzt den Schalter \ifkorrekturansicht voraus (gesetzt in den
%% einbindenden Dateien latex-korrekturansicht-abspann.tex bzw.
%% latex-leseansicht-abspann.tex).
%% ---------------------------------------------------------------

\normalsize

% Das esempio-Environment wird nur in der Leseansicht benötigt
\ifkorrekturansicht\else
\newenvironment{esempio}[3]%
{
    \vspace{1.5ex}
    \rlap{\underline{#1}}
    \par
    \setlength{\parindent}{0cm}
    \nopagebreak
    \leftskip=#2cm
    \rightskip=#3cm
}
{
    \par
}
\fi

\doendnotes{C}
\bigskip
\vfill

\clearpage

\footnotesize

\ifkorrekturansicht
  \lohead{\textsc{register}}
\fi

% theindex-Environment neu definieren ohne reledmac
\makeatletter
\renewenvironment{theindex}{%
  \ifkorrekturansicht
    \section*{\indexname}%
  \else
    \subsubsection*{Index der erwähnten Entitäten}%
  \fi
  \setlength{\parindent}{0pt}%
  \setlength{\parskip}{0pt plus 0.3pt}%
  \let\item\@idxitem
}{%
  \ifkorrekturansicht\clearpage\fi
}
\makeatother

\IfFileExists{\jobname-pw.ind}{\input{\jobname-pw.ind}}{}

% Quellenangabe nur in der Leseansicht
\ifkorrekturansicht\else
% Fallback-Definitionen, falls die .tex-Datei \titel etc. nicht gesetzt hat
\providecommand{\titel}{}
\providecommand{\editorInnen}{}
\providecommand{\dateiname}{\jobname}

\vspace{3cm}

\vfill

\footnotesize
\textsc{Quelle}: \titel. Herausgegeben von {\editorInnen}. In: \emph{Arthur Schnitzler: Briefwechsel mit Autorinnen und Autoren}.
 Digitale Edition, https://schnitzler-briefe.acdh.oeaw.ac.at/{\dateiname}.html (Stand \today)
\fi

\end{document}


