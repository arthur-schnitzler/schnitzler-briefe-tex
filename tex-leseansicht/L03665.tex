%% latex-korrekturansicht-vorspann.tex
%% Vorspann für die Korrekturansicht.
%% Lädt die gemeinsame Datei latex-vorspann.tex mit gesetztem Schalter.

\newif\ifkorrekturansicht
\korrekturansichttrue

\input{../tex-inputs/latex-vorspann}


\section[Stefan Zweig an Arthur Schnitzler, 13. 5. 1923]{L03665 Stefan Zweig an Arthur Schnitzler, 13. 5. 1923}
\nopagebreak\mylabel{L03665v}
\rehead{ }\normalsize\beginnumbering\briefempfaengerindex{Schnitzler, Arthur@\textsc{Schnitzler, Arthur}!zzzZweig, Stefan@\emph{von Stefan Zweig}!1923-05-131@{13. 5. 1923}|(be}
\toendnotes[C]{\smallbreak\pagebreak[2]}\Standort{CUL, Schnitzler, B 118.}
\physDesc{Brief, 1 Blatt, 1 Seite, 590 Zeichen
\newline{}Handschrift: blaue Tinte, lateinische Kurrent}
\buchAbdrucke{\weitereDrucke{Stefan Zweig: \emph{Briefwechsel mit Hermann Bahr, Sigmund Freud, Rainer Maria
                        Rilke und Arthur Schnitzler}. Frankfurt am Main: \emph{S. Fischer} 1987, S. 414.} }\toendnotes[C]{\smallbreak}
\pstart
           {\pb}\uline{Salzburg}, Kapuzinerberg 5\oindex{Paschinger Schloessl@\textbf{Paschinger Schlössl}, \emph{Wohngebäude (K.WHS)}|pw}\hfill 13. Mai 1923\pend
           \vspace{0.5em}
\pstart
           Lieber verehrter Herr Doktor, in einem Versteigerungskatalog
               entdecke ich eben dieses \label{K_L03665-1v}\edtext{Buch\pwindex{Gaensemaennchen. Roman@\emph{Das Gänsemännchen. Roman}|pwv}}{\lemma{\textnormal{\emph{Buch}}}\Cendnote{\textnormal{Das Einzelblatt, das die Beilage zu
                  diesem Brief gewesen sein dürfte, wird heute im Nachlass Schnitzlers im \emph{Deutschen Literaturarchiv
                        Marbach} aufbewahrt (HS.NZ85.1.4978). Am
                     29. 5. 1923 schrieb Schnitzler an den Antiquar Emil
                     Hirsch\pwindex{Hirsch, Emil 1866-03-14 – 1954-07-27@\textsc{Hirsch, Emil} (1866-03-14 – 1954-07-27), \emph{Verleger/Verlegerin, Antiquar/Antiquarin}|pwk}: »29. 5. 1923.{ / }Sehr geehrter Herr.{ / }In ihrem Versteigerungskatalog, der mir von
                        befreundeter Seite zugesandt wird, finde ich auf Seite 60,
                        Nr. 784, Wassermann Jakon\pwindex{Wassermann, Jakob 10.03.1873 – 01.01.1934@\textsc{Wassermann, Jakob} (10.03.1873 – 01.01.1934), \emph{Schriftsteller/Schriftstellerin}|pw}, Das Gänsemännchen\pwindex{Gaensemaennchen. Roman@\emph{Das Gänsemännchen. Roman}|pw} mit handschriftlicher
                        Widmung an Arthur und Olga Schnitzler\pwindex{Schnitzler, Olga 17.01.1882 – 13.01.1970@\textsc{Schnitzler, Olga} (17.01.1882 – 13.01.1970), \emph{Schauspieler/Schauspielerin, Sänger/Sängerin}|pw}. Ich ersuche hiemit die
                        Versteigerung dieses Buches, das auf eine mir vorläufig unbegreifliche Weise
                        aus meinem Besitz verschwunden ist, zu unterlassen und das mir gehörige
                        Exemplar an meine Adresse freundlichst rücksenden zu wollen.{ / }Mit vorzüglicher Hochachtung{ / }\hspace*{2.5em}{ / }Herrn Emil Hirsch\pwindex{Hirsch, Emil 1866-03-14 – 1954-07-27@\textsc{Hirsch, Emil} (1866-03-14 – 1954-07-27), \emph{Verleger/Verlegerin, Antiquar/Antiquarin}|pw},
                        Verleger,{ / }München\oindex{Muenchen@\textbf{München}, \emph{P.PPLA}|pw}.«
                        (\emph{DLA}, HS.1985.1.1016). Aus den zwei
                  weiteren Schreiben Schnitzlers an Hirsch\pwindex{Hirsch, Emil 1866-03-14 – 1954-07-27@\textsc{Hirsch, Emil} (1866-03-14 – 1954-07-27), \emph{Verleger/Verlegerin, Antiquar/Antiquarin}|pwk} geht hervor, dass das Exemplar von
                     Max Krell\pwindex{Krell, Max 24.09.1887 – 11.06.1962@\textsc{Krell, Max} (24.09.1887 – 11.06.1962), \emph{Schriftsteller/Schriftstellerin, Verlagslektor/Verlagslektorin}|pwk} zum Verkauf freigegeben wurde
                  – der es wiederum von Schnitzlers Schwägerin
                     Elisabeth Steinrück\pwindex{Steinrueck, Elisabeth 19.11.1885 – 07.04.1920@\textsc{Steinrück, Elisabeth} (19.11.1885 – 07.04.1920)|pwk} bezogen hatte. Ob nun
                  diese oder Krell\pwindex{Krell, Max 24.09.1887 – 11.06.1962@\textsc{Krell, Max} (24.09.1887 – 11.06.1962), \emph{Schriftsteller/Schriftstellerin, Verlagslektor/Verlagslektorin}|pwk} das Buch sich zu Unrecht
                  angeeignet hat, lässt sich nicht mehr bestimmen. Die erhaltene Korrespondenz
                  zwischen Schnitzler und Krell\pwindex{Krell, Max 24.09.1887 – 11.06.1962@\textsc{Krell, Max} (24.09.1887 – 11.06.1962), \emph{Schriftsteller/Schriftstellerin, Verlagslektor/Verlagslektorin}|pwk} ist im betreffenden Zeitraum ausgesetzt. Das Buch
                  dürfte letztlich an Schnitzler retourniert
                  worden sein, wohingegen er dem Verleger ein von ihm gewidmetes Exemplar der
                  Erstausgabe von \emph{Das Märchen}\pwindex{Maerchen. Schauspiel in drei Aufzuegen@\emph{Das Märchen. Schauspiel in drei Aufzügen}|pwk} zukommen ließ.
               }}}\label{K_L03665-1}. Da ich nicht annehme, dass Sie die Exemplare Ihrer gewidmeten Bücher
               verkaufen (vielleicht werden wir bald so weit sein) so handelt es sich offenbar um
               ein entwendetes Exemplar und Sie haben wohl das Recht es zurückzufordern. Ich glaubte
               Sie aufmerksam machen zu müssen, weil ich selbst jüngst ähnlich einem entwendeten
                  \label{K_L03665-2v}\edtext{Buch\pwindex{?? [Widmungsexemplar eines unbekannten Buchs an Stefan Zweig, 1923]@\emph{?? [Widmungsexemplar eines unbekannten Buchs an Stefan Zweig, 1923]}|pwv}}{\lemma{\textnormal{\emph{Buch}}}\Cendnote{\textnormal{nicht identifiziert}}}\label{K_L03665-2} auf die Spur
               kam – und dann freue ich mich jeder Gelegenheit, Ihnen meine herzliche Verehrung
               aussprechen zu können. \pend
           
\pstart
           Ihr getreuer{\\[\baselineskip]}\spacefill\mbox{Stefan Zweig}\pend
           \leftskip=0em{}\selectlanguage{ngerman}\endnumbering\briefempfaengerindex{Schnitzler, Arthur@\textsc{Schnitzler, Arthur}!zzzZweig, Stefan@\emph{von Stefan Zweig}!1923-05-131@{13. 5. 1923}|)be}\mylabel{L03665h}
\begin{anhang}
\end{anhang}\normalsize

\doendnotes{C}
\bigskip
\vfill

\clearpage

\footnotesize

\lohead{\textsc{register}}

% Definiere theindex-Environment komplett neu ohne reledmac
\makeatletter
\renewenvironment{theindex}{%
  \section*{\indexname}%
  \setlength{\parindent}{0pt}%
  \setlength{\parskip}{0pt plus 0.3pt}%
  \let\item\@idxitem
}{%
  \clearpage
}
\makeatother

\IfFileExists{\jobname-pw.ind}{\input{\jobname-pw.ind}}{}

\end{document}

      