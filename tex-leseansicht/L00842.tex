%% latex-leseansicht-vorspann.tex
%% Vorspann für die Leseansicht.
%% Lädt die gemeinsame Datei latex-vorspann.tex mit nicht gesetztem Schalter.

\newif\ifkorrekturansicht
\korrekturansichtfalse

\input{../tex-inputs/latex-vorspann}


         
         \newcommand{\erwaehntePersonen}{Personen: }
         \newcommand{\erwaehnteInstitutionen}{}
         \newcommand{\erwaehnteOrte}{}
         \newcommand{\erwaehnteWerke}{
               \section[Arthur Schnitzler an Hugo von Hofmannsthal, 31. 8. 1898]{ Arthur Schnitzler an Hugo von Hofmannsthal, 31. 8. 1898}\nopagebreak\mylabel{v}\rehead{ }\begin{ledgroupsized}[t]{13cm}\normalsize\beginnumbering \toendnotes[C]{\smallbreak\pagebreak[2]} \Standort{FDH, Hs-30885,76.}
\physDesc{Bildpostkarte
\newline{}Handschrift: Bleistift, deutsche Kurrent\newline{}Versand: Stempel: »\nobreak{}\oindex{XXXX Ortsangabe fehlt|pwk}Lugano Lettere, 1. IX. 98\nobreak{}«.  }\buchAbdrucke{\weitereDrucke{Hugo von Hofmannsthal, Arthur Schnitzler: \emph{Briefwechsel}. Hg. Therese Nickl und Heinrich Schnitzler. Frankfurt am Main: \emph{S. Fischer} 1964, S. 111–112.} }\toendnotes[C]{\smallbreak}\pstart{}{\pb}Hrn \textsc{Hugo v
                            Hofmannsthal}\pend{}\pstart{}\textsc{Lugano\oindex{XXXX Ortsangabe fehlt|pw}}\pend{}\pstart{}\textsc{Hotel du parc\oindex{XXXX Ortsangabe fehlt|pw}}.\pend{}\pstart{}\textsc{Svizzera}\oindex{XXXX Ortsangabe fehlt|pw}\pend{}{\bigskip}\pstart
           \noindent{}\centering{}\textcolor{gray}{\textbf{{\pb}Bologna\oindex{XXXX Ortsangabe fehlt|pw}. Le Torri Carisenda e Asinelli\oindex{XXXX Ortsangabe fehlt|pw}.}}\pend
           \pstart
           \raggedleft{}31. 8. 98.\pend
           \pstart{}Mein Lieber Hugo, \pend\pstart
           Ich freue mich ſehr, meinem Einfall nachgegeben zu haben und ein paar ital.\oindex{XXXX Ortsangabe fehlt|pw}{ }Städte zu ſehen. Wär’s mir doch bald möglich,
                    weiter und auf längre Zeit, und, ich glaub das zu wünſchen, nicht allein. – Hier
                    ſende ich Ihnen die zwei ſchiefen Türme\oindex{XXXX Ortsangabe fehlt|pw}; der
                    eine gehört dem Richard\pwindex{\textcolor{red}{\textsuperscript{XXXX1 indx}}|pw}\footnote{\noindent{}er kann wählen}, ebenſo wie Ihnen beiden meine herzlichſten Grüße. Schreiben Sie mir
                    nach Wien\oindex{XXXX Ortsangabe fehlt|pw}; ich bin wahrſcheinlich So{\geminationn}tag zu Hauſe.\pend
           \pstart Ihr \spacefill\mbox{Arthur}\pend{}\pstart
           \noindent{}\label{T_L00842_1v}\edtext{Was für Proſa ſchreiben Sie?}{\lemma{\textnormal{\emph{Was … Sie?}}}\Cendnote{\textnormal{in der linken oberen Ecke auf dem
                            Kopf}}}\label{T_L00842_1h}\pend
           
         
         \endnumbering\mylabel{h}\end{ledgroupsized}  \newcommand{\dateiname}{L00842}\newcommand{\titel}{Arthur Schnitzler an Hugo von Hofmannsthal, 31. 8. 1898}\newcommand{\editorInnen}{Martin Anton Müller und Gerd-Hermann Susen}%% latex-leseansicht-abspann.tex
%% Abspann für die Leseansicht.
%% Der Schalter \ifkorrekturansicht ist bereits durch den Vorspann gesetzt.

%% latex-abspann.tex
%% Gemeinsamer Abspann für Korrekturansicht und Leseansicht.
%% Setzt den Schalter \ifkorrekturansicht voraus (gesetzt in den
%% einbindenden Dateien latex-korrekturansicht-abspann.tex bzw.
%% latex-leseansicht-abspann.tex).
%% ---------------------------------------------------------------

\normalsize

% Das esempio-Environment wird nur in der Leseansicht benötigt
\ifkorrekturansicht\else
\newenvironment{esempio}[3]%
{
    \vspace{1.5ex}
    \rlap{\underline{#1}}
    \par
    \setlength{\parindent}{0cm}
    \nopagebreak
    \leftskip=#2cm
    \rightskip=#3cm
}
{
    \par
}
\fi

\doendnotes{C}
\bigskip
\vfill

\clearpage

\footnotesize

\ifkorrekturansicht
  \lohead{\textsc{register}}
\fi

% theindex-Environment neu definieren ohne reledmac
\makeatletter
\renewenvironment{theindex}{%
  \ifkorrekturansicht
    \section*{\indexname}%
  \else
    \subsubsection*{Index der erwähnten Entitäten}%
  \fi
  \setlength{\parindent}{0pt}%
  \setlength{\parskip}{0pt plus 0.3pt}%
  \let\item\@idxitem
}{%
  \ifkorrekturansicht\clearpage\fi
}
\makeatother

\IfFileExists{\jobname-pw.ind}{\input{\jobname-pw.ind}}{}

% Quellenangabe nur in der Leseansicht
\ifkorrekturansicht\else
% Fallback-Definitionen, falls die .tex-Datei \titel etc. nicht gesetzt hat
\providecommand{\titel}{}
\providecommand{\editorInnen}{}
\providecommand{\dateiname}{\jobname}

\vspace{3cm}

\vfill

\footnotesize
\textsc{Quelle}: \titel. Herausgegeben von {\editorInnen}. In: \emph{Arthur Schnitzler: Briefwechsel mit Autorinnen und Autoren}.
 Digitale Edition, https://schnitzler-briefe.acdh.oeaw.ac.at/{\dateiname}.html (Stand \today)
\fi

\end{document}


      