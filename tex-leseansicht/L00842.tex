%% latex-korrekturansicht-vorspann.tex
%% Vorspann für die Korrekturansicht.
%% Lädt die gemeinsame Datei latex-vorspann.tex mit gesetztem Schalter.

\newif\ifkorrekturansicht
\korrekturansichttrue

\input{../tex-inputs/latex-vorspann}


\section[Arthur Schnitzler an Hugo von Hofmannsthal, 31. 8. 1898]{L00842 Arthur Schnitzler an Hugo von Hofmannsthal, 31. 8. 1898}
\nopagebreak\mylabel{L00842v}
\rehead{ }\normalsize\beginnumbering\briefempfaengerindex{Hofmannsthal, Hugo von@\textsc{Hofmannsthal, Hugo von}!zzzSchnitzler, Arthur@\emph{von Arthur Schnitzler}!1898-08-311@{31. 8. 1898}|(be}
\toendnotes[C]{\smallbreak\pagebreak[2]}\Standort{FDH, Hs-30885,76.}
\physDesc{Bildpostkarte, 499 Zeichen
\newline{}Handschrift: Bleistift, deutsche Kurrent
\newline{}Versand: Stempel: »\nobreak{}\oindex{Lugano@\textbf{Lugano}, \emph{P.PPLA2}|pwk}Lugano Lettere, 1. IX. 98\nobreak{}«.  }
\buchAbdrucke{\weitereDrucke{Hugo von Hofmannsthal, Arthur Schnitzler: \emph{Briefwechsel}. Frankfurt am Main: \emph{S. Fischer} 1964, S. 111–112.} }\toendnotes[C]{\smallbreak}\pstart{}{\pb}Hrn \textsc{Hugo v Hofmannsthal}\pend{}\pstart{}\textsc{Lugano\oindex{Lugano@\textbf{Lugano}, \emph{P.PPLA2}|pw}}\pend{}\pstart{}\textsc{Hotel du parc\oindex{Hôtel du Parc@\textbf{Hôtel du Parc}, \emph{Hotel (K.HTL)}|pw}}.\pend{}\pstart{}\textsc{Svizzera}\oindex{Schweiz@\textbf{Schweiz}, \emph{A.PCLI}|pw}\pend{}{\bigskip}
\pstart
           \noindent{}\centering{}{\pb}\textcolor{gray}{\textbf{Bologna\oindex{Bologna@\textbf{Bologna}, \emph{P.PPLA}|pw}. Le
                     Torri Carisenda e Asinelli\oindex{Le due Torri: Garisenda e degli Asinelli@\textbf{Le due Torri: Garisenda e degli Asinelli}, \emph{Monument (K.MON)}|pw}.}}\pend
           \vspace{1em}
\pstart
           \raggedleft{}{\pb}31. 8. 98.\pend
           
\pstart{}Mein Lieber Hugo, \pend\vspace{0.5em}
\pstart
           Ich freue mich ſehr, meinem Einfall nachgegeben zu haben und ein paar ital.\oindex{Italien@\textbf{Italien}, \emph{A.PCLI}|pw}{ }Städte zu ſehen. Wär’s mir doch bald möglich,
               weiter und auf längre Zeit, und, ich glaub das zu wünſchen, nicht allein. – Hier
               ſende ich Ihnen die zwei ſchiefen Türme\oindex{Le due Torri: Garisenda e degli Asinelli@\textbf{Le due Torri: Garisenda e degli Asinelli}, \emph{Monument (K.MON)}|pw}; der
               eine gehört dem Richard\pwindex{Beer-Hofmann, Richard 1866-07-11 – 1945-09-26@\textsc{Beer-Hofmann, Richard} (1866-07-11 – 1945-09-26), \emph{Schriftsteller/Schriftstellerin}|pw}\noindent{}er kann wählen, ebenſo wie Ihnen beiden meine herzlichſten Grüße. Schreiben Sie mir nach Wien\oindex{Wien@\textbf{Wien}, \emph{A.ADM2}|pw}; ich bin wahrſcheinlich So{\geminationn}tag zu Hauſe.\pend
           \pstart Ihr \spacefill\mbox{Arthur}\pend{}
\pstart
           \noindent{}\label{T_L00842-1v}\edtext{Was für Proſa ſchreiben Sie?}{\lemma{\textnormal{\emph{Was … Sie?}}}\Cendnote{\textnormal{in der linken oberen Ecke auf dem
                     Kopf}}}\label{T_L00842-1}\pend
           \selectlanguage{ngerman}\endnumbering\briefempfaengerindex{Hofmannsthal, Hugo von@\textsc{Hofmannsthal, Hugo von}!zzzSchnitzler, Arthur@\emph{von Arthur Schnitzler}!1898-08-311@{31. 8. 1898}|)be}\mylabel{L00842h}  \normalsize

\doendnotes{C}
\bigskip
\vfill

\clearpage

\footnotesize

\lohead{\textsc{register}}

% Definiere theindex-Environment komplett neu ohne reledmac
\makeatletter
\renewenvironment{theindex}{%
  \section*{\indexname}%
  \setlength{\parindent}{0pt}%
  \setlength{\parskip}{0pt plus 0.3pt}%
  \let\item\@idxitem
}{%
  \clearpage
}
\makeatother

\IfFileExists{\jobname-pw.ind}{\input{\jobname-pw.ind}}{}

\end{document}

      