%% latex-leseansicht-vorspann.tex
%% Vorspann für die Leseansicht.
%% Lädt die gemeinsame Datei latex-vorspann.tex mit nicht gesetztem Schalter.

\newif\ifkorrekturansicht
\korrekturansichtfalse

\input{../tex-inputs/latex-vorspann}

\begin{center}
            \textcolor{red}{ENTWURF, NICHT FERTIG KORRIGIERT}
                      \end{center}
            
         
         \renewcommand{\erwaehntePersonen}{Personen: Felix Salten}
         \renewcommand{\erwaehnteOrte}{Orte: Wien}
         \renewcommand{\erwaehnteWerke}{}
               \section[Arthur Schnitzler an Felix Salten, 10. 11. 1903]{ Arthur Schnitzler an Felix Salten, 10. 11. 1903}\nopagebreak\mylabel{v}\rehead{ }\begin{ledgroupsized}[t]{13cm}\normalsize\beginnumbering \toendnotes[C]{\smallbreak\pagebreak[2]} \Standort{Wienbibliothek im Rathaus, ZPH 1681, 2.1.516.}
\physDesc{
\newline{}Handschrift: , deutsche Kurrent}\toendnotes[C]{\smallbreak}\pstart
           \raggedleft{}{\pb}10/11 903.\pend
           \pstart
           lieber Freund, ich frage mich nun wieder einmal, ob es nicht beſſer
               wäre alles, was man gegen jemanden, der einem nahe ſteht auf dem Herzen hat, zu
               verſchweigen, um ein Verhältnis, wie auch nicht in der Höhe abſoluter Ehrlichkeit,
               doch wenigſtens auf dem Niveau angenehmer Unterhaltung {\pb}und gelegentlicher intellectueller
               Ausſprache weiterzurführen{\dotstwo} Ich habe Ihnen \strikeout{nur \textcolor{gray}{XXXX }}einfach geſchrieben, nicht ohne Erregung, vielleicht nicht ganz ohne
               Ungerechtigkeit, was mich in Ihrem Feu{[}i{]}lleton\textcolor{red}{\textsuperscript{\textbf{KEY}}} befremdet, durch welche Bemerkg ich mich
               am Ende ſogar unangenehm berührt fühlen durfte. Gut. Darauf ſchreiben Sie mir einen
               ſehr {\pb}ſchönen Brief, in dem Sie
               mich allerdings nicht vollko{\geminationm}en überzeugen, der mir aber
               als ganzes wohlgethan – und der jedenfalls alle Rese von Bitterkeit (oder halten Sie
               mich für nachträgeriſch?) weggewaſchen hat. Und nun ko{\geminationm}t, da ich eben bereit bin, die Sache als erledigt zu betrachten, und nach der
                  Ausſprach\textcolor{gray}{e} von beiden Seiten {\pb}Ihnen wie ſonſt die Hand zu
               drücken, da ko{\geminationm}t dieſer ärgerliche, \textsc{enervante} Schluſs – in dem Sie ſich von der Vorleſund zu abſentiren
               wünſchen, zu der ich Sie als einen Freund und als einen Menſchen, deſſen Urtheil mir
               aufs höchſte werth war u iſt (auch we{\geminationn} er ſich nur wie
               wir alle {\pb}gelegentlich irrt oder,
               wie alle einmal misverſtändlich ausdrückt) eingeladen habe – ko{\geminationm}t die unglaubliche Bemerkung: »Ich überlege mir – ob es
               einen Werth für Sie haben kann, we{\geminationn} ich jetzt Ihrer
               Vorleſung beiwohne.« – Nicht als ob mein Urtheil über Sie befangen oder ſchwankend
               gemacht werden könnte – aber \strikeout{ic }wie ich Ihnen nun meine {\pb}Meinung formuliren ſoll – u wie Sie ſie aufnehmen werden {\dots} lieber Freund, hier verſagt mir die Antwort. Soweit ich mich erinnere, haben wir
               einander in mündlichem Verkehr wenigſtens bisher nicht misverſtanden. \strikeout{Durch} Nichts gibt Ihnen das entfernteſte
                  vermuthen Recht zu \strikeout{bezweifeln}, daſs ich
               Sie aus einem andern {\pb}Grunde zu mir
               bitte, als weil ich Werth auf Ihre Zuhören und auf Ihr Urtheil wie auf Ihr Eingreifen
               in die Discuſſion lege. Ich darf von Ihnen verlangen, daſs Sie mir und der
               Aufrichtigkeit und Unbeeinflußtheit meiner Motive glauben wenn ich zu
               Ihnen rede. Empfindlichkeiten, Nervoſitäten, Befan{\pb}genheiten, Unklarheiten ſtören
               unſere Beziehungen ſeit Jahren. Das Mistrauen aber wäre einfach die Todeskrankheit.
               Und an dem, wenigſtens an dem, bin \uline{ich} völlig
               unſchuldig. Ja können wir de{\geminationn} wirklich nicht ſo zu
               einander ſtehen – wie Menſchen, die in klaren Worten zu einander ſprechen? {\pb}müſſen Meinungsverſchiedenheiten
               immer wie Nebel ſein, die unſre Phyſiognomien vor ein ander verbergen ſtatt Blitze,
               die ſie erleuchten? – Es iſt nichts »vorgefallen«; für mich nichts. Ich habe mich
               geärgert. Ja. Ich ärgere mich ſogar noch. – Sie auch. Nun ja. We{\geminationn} aber ein Anlaſs \substVorne{}\textsuperscript{dſs}\substDazwischen{}ſein ſoll\substHinten{}, ſich von einander abzu{\pb}wenden – ſo komme dieſe Schuld auf
               Sie allein. Ich vermag es nicht, – dergleichen \introOben{}dauernd\introOben{}\strikeout{\textcolor{gray}{X}} ſchwer zu nehmen – und we{\geminationn} ich auch \strikeout{\textcolor{gray}{XXXXXX}}\strikeout{\textcolor{gray}{X}} eine Stunde lang oder eine Nacht lang gekränkt oder erbittert war. Sich
               ausſprechen iſt alles. Aber es darf einem nicht {\pb}zu ſchwer gemacht werden \pend
           \pstart
           Ihr {\\[\baselineskip]}\spacefill\mbox{A. S.}\pend
           \leftskip=0em{}
         
         \endnumbering\mylabel{h}\end{ledgroupsized}\begin{anhang}\end{anhang}\newcommand{\dateiname}{L02989}\newcommand{\titel}{Arthur Schnitzler an Felix Salten, 10. 11. 1903}\newcommand{\editorInnen}{Martin Anton Müller und Laura Untner}%% latex-leseansicht-abspann.tex
%% Abspann für die Leseansicht.
%% Der Schalter \ifkorrekturansicht ist bereits durch den Vorspann gesetzt.

%% latex-abspann.tex
%% Gemeinsamer Abspann für Korrekturansicht und Leseansicht.
%% Setzt den Schalter \ifkorrekturansicht voraus (gesetzt in den
%% einbindenden Dateien latex-korrekturansicht-abspann.tex bzw.
%% latex-leseansicht-abspann.tex).
%% ---------------------------------------------------------------

\normalsize

% Das esempio-Environment wird nur in der Leseansicht benötigt
\ifkorrekturansicht\else
\newenvironment{esempio}[3]%
{
    \vspace{1.5ex}
    \rlap{\underline{#1}}
    \par
    \setlength{\parindent}{0cm}
    \nopagebreak
    \leftskip=#2cm
    \rightskip=#3cm
}
{
    \par
}
\fi

\doendnotes{C}
\bigskip
\vfill

\clearpage

\footnotesize

\ifkorrekturansicht
  \lohead{\textsc{register}}
\fi

% theindex-Environment neu definieren ohne reledmac
\makeatletter
\renewenvironment{theindex}{%
  \ifkorrekturansicht
    \section*{\indexname}%
  \else
    \subsubsection*{Index der erwähnten Entitäten}%
  \fi
  \setlength{\parindent}{0pt}%
  \setlength{\parskip}{0pt plus 0.3pt}%
  \let\item\@idxitem
}{%
  \ifkorrekturansicht\clearpage\fi
}
\makeatother

\IfFileExists{\jobname-pw.ind}{\input{\jobname-pw.ind}}{}

% Quellenangabe nur in der Leseansicht
\ifkorrekturansicht\else
% Fallback-Definitionen, falls die .tex-Datei \titel etc. nicht gesetzt hat
\providecommand{\titel}{}
\providecommand{\editorInnen}{}
\providecommand{\dateiname}{\jobname}

\vspace{3cm}

\vfill

\footnotesize
\textsc{Quelle}: \titel. Herausgegeben von {\editorInnen}. In: \emph{Arthur Schnitzler: Briefwechsel mit Autorinnen und Autoren}.
 Digitale Edition, https://schnitzler-briefe.acdh.oeaw.ac.at/{\dateiname}.html (Stand \today)
\fi

\end{document}


      