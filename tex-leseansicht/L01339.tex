%% latex-korrekturansicht-vorspann.tex
%% Vorspann für die Korrekturansicht.
%% Lädt die gemeinsame Datei latex-vorspann.tex mit gesetztem Schalter.

\newif\ifkorrekturansicht
\korrekturansichttrue

\input{../tex-inputs/latex-vorspann}


\section[Arthur Schnitzler an Hermann Bahr, 10. 11. {[}1903{]}]{L01339 Arthur Schnitzler an Hermann Bahr, 10. 11. {[}1903{]}}
\nopagebreak\mylabel{L01339v}
\rehead{ }\normalsize\beginnumbering\briefempfaengerindex{Bahr, Hermann@\textsc{Bahr, Hermann}!zzzSchnitzler, Arthur@\emph{von Arthur Schnitzler}!1903-11-102@{10. 11. {[}1903{]}}|(be}
\toendnotes[C]{\smallbreak\pagebreak[2]}\Standort{TMW, HS AM 60156 Ba.}
\physDesc{Postkarte, 279 Zeichen
\newline{}Handschrift: Bleistift, deutsche Kurrent
\newline{}Versand: 1) Stempel: »\nobreak{}Wien, 10–11 V\nobreak{}«.   2) Stempel: »\nobreak{}\oindex{XIII., Hietzing@\textbf{XIII., Hietzing}, \emph{A.ADM3}|pwk}Wien 13/7, 10. 11. 0{[}3{]}\nobreak{}«. 
\newline{}Ordnung: Lochung }
\buchAbdrucke{\weitereDrucke{1) Arthur Schnitzler: \emph{The Letters of Arthur Schnitzler to Hermann Bahr}. Chapel Hill: \emph{The University of North Carolina Press} 1978, S. 77.} \weitereDrucke{2) Hermann Bahr, Arthur Schnitzler: \emph{Briefwechsel, Aufzeichnungen, Dokumente (1891–1931)}. Göttingen: \emph{Wallstein} 2018, S. 279.} }\toendnotes[C]{\smallbreak}\pstart{}{\pb}\textsc{Herrn Hermann Bahr}\pend{}\pstart{}Wien-Ob. St. Veit\oindex{Ober Sankt Veit@\textbf{Ober Sankt Veit}, \emph{P.PPLX}|pw}\pend{}\pstart{}Veitliſſengaſſe 15\oindex{Veitlissengasse@\textbf{Veitlissengasse}, \emph{Straße (K.STR)}|pw}\pend{}{\bigskip}\vspace{1em}
\pstart
           \noindent{}{\pb}lieber Hermann,du haſt dich für den \label{K_L01339-1v}\edtext{\uline{Goethe Zelter} Briefwechſel\pwindex{Briefwechsel zwischen Goethe und Zelter@\emph{Briefwechsel zwischen Goethe und Zelter}|pw}}{\lemma{\textnormal{\emph{Goethe … Briefwechſel}}}\Cendnote{\textnormal{nicht unter den aus Bahrs\pwindex{Bahr, Hermann 19.07.1863 – 15.01.1934@\textsc{Bahr, Hermann} (19.07.1863 – 15.01.1934), \emph{Schriftsteller/Schriftstellerin, Kritiker/Kritikerin}|pwk} Besitz überlieferten Büchern}}}\label{K_L01339-1} intereſſirt. Im
                  \textsc{Catalog XXXIV}{ }ſüddeutſches Antiquariat\orgindex{Sueddeutsches Antiquariat@Süddeutsches Antiquariat|pw}, \uline{München} Gallerieſtraße 20\oindex{Galeriestrasse@\textbf{Galeriestraße}, \emph{Straße (K.STR)}|pw}, unter Nr 699 iſt
               ein Exemplar, 6 Bände um 40 Mark angekündigt. Herzlichſt dein{\\}\spacefill\mbox{Arth Sch}\pend
           \selectlanguage{ngerman}\endnumbering\briefempfaengerindex{Bahr, Hermann@\textsc{Bahr, Hermann}!zzzSchnitzler, Arthur@\emph{von Arthur Schnitzler}!1903-11-102@{10. 11. {[}1903{]}}|)be}\mylabel{L01339h}  \normalsize

\doendnotes{C}
\bigskip
\vfill

\clearpage

\footnotesize

\lohead{\textsc{register}}

% Definiere theindex-Environment komplett neu ohne reledmac
\makeatletter
\renewenvironment{theindex}{%
  \section*{\indexname}%
  \setlength{\parindent}{0pt}%
  \setlength{\parskip}{0pt plus 0.3pt}%
  \let\item\@idxitem
}{%
  \clearpage
}
\makeatother

\IfFileExists{\jobname-pw.ind}{\input{\jobname-pw.ind}}{}

\end{document}

      