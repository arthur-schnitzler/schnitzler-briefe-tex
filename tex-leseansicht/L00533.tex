%% latex-leseansicht-vorspann.tex
%% Vorspann für die Leseansicht.
%% Lädt die gemeinsame Datei latex-vorspann.tex mit nicht gesetztem Schalter.

\newif\ifkorrekturansicht
\korrekturansichtfalse

\input{../tex-inputs/latex-vorspann}


\section[Arthur Schnitzler an Hermann Bahr, 7. 2. 1896]{L00533 Arthur Schnitzler an Hermann Bahr, 7. 2. 1896}
\nopagebreak\mylabel{L00533v}
\rehead{ }\normalsize\beginnumbering\briefempfaengerindex{Bahr, Hermann@\textsc{Bahr, Hermann}!zzzSchnitzler, Arthur@\emph{von Arthur Schnitzler}!1896-02-071@{7. 2. 1896}|(be}
\toendnotes[C]{\smallbreak\pagebreak[2]}
\correspDesc{Versand  durch Arthur Schnitzler am 7. 2. 1896 in Berlin
\newline{}Erhalt  durch Hermann Bahr im Zeitraum [8. 2. 1896
                  – 12. 2. 1896?] in Wien}\toendnotes[C]{\smallbreak}
\Standort{TMW, HS AM 23325 Ba.}
\physDesc{Brief, 1 Blatt, 3 Seiten, 817 Zeichen
\newline{}Handschrift: schwarze Tinte, deutsche Kurrent
\newline{}Ordnung: Lochung }
\buchAbdrucke{\weitereDrucke{1) \emph{7. 2. 1896.} In: Arthur Schnitzler: \emph{The Letters of Arthur Schnitzler to Hermann Bahr}. Edited, annotated, and with an introduction, by Donald G. Daviau. Chapel Hill: \emph{The University of North Carolina Press} 1978, S. 58–59 (University of North Carolina studies in the Germanic languages
                        and literatures, 89).} \weitereDrucke{2) Hermann Bahr, Arthur Schnitzler: \emph{Briefwechsel, Aufzeichnungen, Dokumente (1891–1931)}. Herausgegeben von Kurt Ifkovits und Martin Anton Müller. Göttingen: \emph{Wallstein} 2018, S. 117.} }\toendnotes[C]{\smallbreak}
\pstart{}{\pb}Lieber
                  Hermann,\pend\vspace{0.5em}
\pstart
           herzlichen Dank für deine freundlichen Glückwünſche.\pend
           
\pstart
           Was dich intereſſieren wird: \label{K_L00533-1v}\edtext{verriſſsen\pwindex{Peschkau, Emil 19.\,2.\,1856 Wien – vermutlich 1929 Berlin@\textsc{Peschkau, Emil} (19.\,2.\,1856 Wien – vermutlich 1929 Berlin), \emph{Schriftsteller, Journalist}!Deutsches Theater@\strich\emph{Deutsches Theater}|pwv} hat mich nur einer,
               nemlich Herr Peſchkau\pwindex{Peschkau, Emil 19.\,2.\,1856 Wien – vermutlich 1929 Berlin@\textsc{Peschkau, Emil} (19.\,2.\,1856 Wien – vermutlich 1929 Berlin), \emph{Schriftsteller, Journalist}|pw} in den Berl. Neueſten Nachrichten\orgindex{Berliner Neueste Nachrichten@Berliner Neueste Nachrichten|pw}}{\lemma{\textnormal{\emph{verrisssen … Nachrichten}}}\Cendnote{\textnormal{»›Man dramatisirt Zustände,
                        indem man Menschen in sie bringt, die sich ihnen widersetzen; dort, wo sich
                        die Menschen mit den Dingen entzweien, fängt das Drama erst an. Aber seine
                        Menschen, die nichts wollen, sitzen unbeweglich in ihren Zuständen drin, wie
                        Chamäleons, die immer die Farbe ihrer Umgebung haben\pwindex{Bahr, Hermann 19.\,7.\,1863 Linz – 15.\,1.\,1934 München@\textsc{Bahr, Hermann} (19.\,7.\,1863 Linz – 15.\,1.\,1934 München), \emph{Schriftsteller, Kritiker}!Burgtheater (Liebelei, Schauspiel in drei Acten von Arthur Schnitzler. Rechte der Seele, Schauspiel in einem Act von Guiseppe Giacosa. Zum ersten Mal aufgeführt am 9. October)@\strich\emph{Burgtheater (Liebelei, Schauspiel in drei Acten von Arthur Schnitzler. Rechte der Seele, Schauspiel in einem Act von Guiseppe Giacosa. Zum ersten Mal aufgeführt am 9. October)}|pwv}‹« (E. Peschkau\pwindex{Peschkau, Emil 19.\,2.\,1856 Wien – vermutlich 1929 Berlin@\textsc{Peschkau, Emil} (19.\,2.\,1856 Wien – vermutlich 1929 Berlin), \emph{Schriftsteller, Journalist}|pwk}: \emph{Deutsches Theater}\pwindex{Peschkau, Emil 19.\,2.\,1856 Wien – vermutlich 1929 Berlin@\textsc{Peschkau, Emil} (19.\,2.\,1856 Wien – vermutlich 1929 Berlin), \emph{Schriftsteller, Journalist}!Deutsches Theater@\strich\emph{Deutsches Theater}|pwk}. In: \emph{Berliner Neueste Nachrichten}\pwindex{Berliner Neueste Nachrichten@\emph{Berliner Neueste Nachrichten}|pwk}, Jg. 16, Nr. 59, 5. 2. 1896,
                     S. 2–3, hier: S. 3).}}}\label{K_L00533-1}, u weißt du, was er zu dieſem Behufe gethan
               hat? einfach \uline{wörtlich} citirt (mit Anführung der
               Quelle), was du über mich{ }ſagſt und daraus zwingend bewieſen, daſs ich weder {\pb}ein Dramatiker noch ein
               Dichter bin, sondern daſs mir{ }ſelbſt die Elementarkenntniſſe zu dieſen beiden{ }ſchönen
               Stellungen fehlen. –\pend
           
\pstart
           Sehr erfreulich waren mir Deine Mittheilungen über das Märchen\pwindex{Schnitzler, Arthur 15.\,5.\,1862 Wien – 21.\,10.\,1931 ebd.@\textsc{Schnitzler, Arthur} (15.\,5.\,1862 Wien – 21.\,10.\,1931 ebd.), \emph{Schriftsteller, Mediziner}!Märchen. Schauspiel in drei Aufzügen@\strich\emph{Das Märchen. Schauspiel in drei Aufzügen}|pw} und Langka{\geminationm}ers\pwindex{Langkammer, Karl 4.\,8.\,1854 Wien – 18.\,5.\,1936 ebd.@\textsc{Langkammer, Karl} (4.\,8.\,1854 Wien – 18.\,5.\,1936 ebd.), \emph{Theaterleiter, Regisseur, Schauspieler}|pw} Urtheil. Aber ich habe wieder{ }ſehr lebhafte
               Bedenken betreffs einer eventuellen Aufführung bekommen. Ich werde ja wohl bald
               Gelegenheit {[}haben{]},{ }ſowohl mit dir als mit Langkammer\pwindex{Langkammer, Karl 4.\,8.\,1854 Wien – 18.\,5.\,1936 ebd.@\textsc{Langkammer, Karl} (4.\,8.\,1854 Wien – 18.\,5.\,1936 ebd.), \emph{Theaterleiter, Regisseur, Schauspieler}|pw}{ }{\pb}darüber zu reden. Bis
               dahin beſte Grüße und nochmals vielen Dank.\pend
           \pstart Dein \spacefill\mbox{ArthSchn}\pend{}
\pstart
           \textsc{Berlin\oindex{Berlin@\textbf{Berlin}, \emph{Hauptstadt}|pw}}{ }\substVorne{}\textsuperscript{6}\substDazwischen{}7\substHinten{}. 2. 96.\pend
           \selectlanguage{ngerman}\endnumbering\briefempfaengerindex{Bahr, Hermann@\textsc{Bahr, Hermann}!zzzSchnitzler, Arthur@\emph{von Arthur Schnitzler}!1896-02-071@{7. 2. 1896}|)be}\mylabel{L00533h}  \newcommand{\dateiname}{L00533}\newcommand{\titel}{Arthur Schnitzler an Hermann Bahr, 7. 2. 1896}\newcommand{\editorInnen}{Herausgegeben von Martin Anton Müller}%% latex-leseansicht-abspann.tex
%% Abspann für die Leseansicht.
%% Der Schalter \ifkorrekturansicht ist bereits durch den Vorspann gesetzt.

%% latex-abspann.tex
%% Gemeinsamer Abspann für Korrekturansicht und Leseansicht.
%% Setzt den Schalter \ifkorrekturansicht voraus (gesetzt in den
%% einbindenden Dateien latex-korrekturansicht-abspann.tex bzw.
%% latex-leseansicht-abspann.tex).
%% ---------------------------------------------------------------

\normalsize

% Das esempio-Environment wird nur in der Leseansicht benötigt
\ifkorrekturansicht\else
\newenvironment{esempio}[3]%
{
    \vspace{1.5ex}
    \rlap{\underline{#1}}
    \par
    \setlength{\parindent}{0cm}
    \nopagebreak
    \leftskip=#2cm
    \rightskip=#3cm
}
{
    \par
}
\fi

\doendnotes{C}
\bigskip
\vfill

\clearpage

\footnotesize

\ifkorrekturansicht
  \lohead{\textsc{register}}
\fi

% theindex-Environment neu definieren ohne reledmac
\makeatletter
\renewenvironment{theindex}{%
  \ifkorrekturansicht
    \section*{\indexname}%
  \else
    \subsubsection*{Index der erwähnten Entitäten}%
  \fi
  \setlength{\parindent}{0pt}%
  \setlength{\parskip}{0pt plus 0.3pt}%
  \let\item\@idxitem
}{%
  \ifkorrekturansicht\clearpage\fi
}
\makeatother

\IfFileExists{\jobname-pw.ind}{\input{\jobname-pw.ind}}{}

% Quellenangabe nur in der Leseansicht
\ifkorrekturansicht\else
% Fallback-Definitionen, falls die .tex-Datei \titel etc. nicht gesetzt hat
\providecommand{\titel}{}
\providecommand{\editorInnen}{}
\providecommand{\dateiname}{\jobname}

\vspace{3cm}

\vfill

\footnotesize
\textsc{Quelle}: \titel. Herausgegeben von {\editorInnen}. In: \emph{Arthur Schnitzler: Briefwechsel mit Autorinnen und Autoren}.
 Digitale Edition, https://schnitzler-briefe.acdh.oeaw.ac.at/{\dateiname}.html (Stand \today)
\fi

\end{document}


