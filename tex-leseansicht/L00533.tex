%% latex-korrekturansicht-vorspann.tex
%% Vorspann für die Korrekturansicht.
%% Lädt die gemeinsame Datei latex-vorspann.tex mit gesetztem Schalter.

\newif\ifkorrekturansicht
\korrekturansichttrue

\input{../tex-inputs/latex-vorspann}


\section[Arthur Schnitzler an Hermann Bahr, 7. 2. 1896]{L00533 Arthur Schnitzler an Hermann Bahr, 7. 2. 1896}
\nopagebreak\mylabel{L00533v}
\rehead{ }\normalsize\beginnumbering\briefempfaengerindex{Bahr, Hermann@\textsc{Bahr, Hermann}!zzzSchnitzler, Arthur@\emph{von Arthur Schnitzler}!1896-02-071@{7. 2. 1896}|(be}
\toendnotes[C]{\smallbreak\pagebreak[2]}\Standort{TMW, HS AM 23325 Ba.}
\physDesc{Brief, 1 Blatt, 3 Seiten, 817 Zeichen
\newline{}Handschrift: schwarze Tinte, deutsche Kurrent
\newline{}Ordnung: Lochung }
\buchAbdrucke{\weitereDrucke{1) Arthur Schnitzler: \emph{The Letters of Arthur Schnitzler to Hermann Bahr}. Chapel Hill: \emph{The University of North Carolina Press} 1978, S. 58–59.} \weitereDrucke{2) Hermann Bahr, Arthur Schnitzler: \emph{Briefwechsel, Aufzeichnungen, Dokumente (1891–1931)}. Göttingen: \emph{Wallstein} 2018, S. 117.} }\toendnotes[C]{\smallbreak}
\pstart{}{\pb}Lieber
                  Hermann,\pend\vspace{0.5em}
\pstart
           herzlichen Dank für deine freundlichen Glückwünſche.\pend
           
\pstart
           Was dich intereſſieren wird: \label{K_L00533-1v}\edtext{verriſſsen\pwindex{Deutsches Theater@\emph{Deutsches Theater}|pwv} hat mich nur einer,
               nemlich Herr Peſchkau\pwindex{Peschkau, Emil 1856-02-19 – vermutlich 1929@\textsc{Peschkau, Emil} (1856-02-19 – vermutlich 1929), \emph{Schriftsteller/Schriftstellerin, Journalist/Journalistin}|pw} in den Berl. Neueſten Nachrichten\orgindex{Berliner Neueste Nachrichten@Berliner Neueste Nachrichten|pw}}{\lemma{\textnormal{\emph{verriſſsen … Nachrichten}}}\Cendnote{\textnormal{»›Man dramatisirt Zustände,
                        indem man Menschen in sie bringt, die sich ihnen widersetzen; dort, wo sich
                        die Menschen mit den Dingen entzweien, fängt das Drama erst an. Aber seine
                        Menschen, die nichts wollen, sitzen unbeweglich in ihren Zuständen drin, wie
                        Chamäleons, die immer die Farbe ihrer Umgebung haben\pwindex{Burgtheater (Liebelei, Schauspiel in drei Acten von Arthur Schnitzler. Rechte der Seele, Schauspiel in einem Act von Guiseppe Giacosa. Zum ersten Mal aufgefuehrt am 9. October)@\emph{Burgtheater (Liebelei, Schauspiel in drei Acten von Arthur Schnitzler. Rechte der Seele, Schauspiel in einem Act von Guiseppe Giacosa. Zum ersten Mal aufgeführt am 9. October)}|pwv}‹« (E. Peschkau\pwindex{Peschkau, Emil 1856-02-19 – vermutlich 1929@\textsc{Peschkau, Emil} (1856-02-19 – vermutlich 1929), \emph{Schriftsteller/Schriftstellerin, Journalist/Journalistin}|pwk}: \emph{Deutsches Theater}\pwindex{Deutsches Theater@\emph{Deutsches Theater}|pwk}. In: \emph{Berliner Neueste Nachrichten}\pwindex{Berliner Neueste Nachrichten@\emph{Berliner Neueste Nachrichten}|pwk}, Jg. 16, Nr. 59, 5. 2. 1896,
                     S. 2–3, hier: S. 3).}}}\label{K_L00533-1}, u weißt du, was er zu dieſem Behufe gethan
               hat? einfach \uline{wörtlich} citirt (mit Anführung der
               Quelle), was du über mich ſagſt und daraus zwingend bewieſen, daſs ich weder {\pb}ein Dramatiker noch ein
               Dichter bin, sondern daſs mir ſelbſt die Elementarkenntniſſe zu dieſen beiden ſchönen
               Stellungen fehlen. – \pend
           
\pstart
           Sehr erfreulich waren mir Deine Mittheilungen über das Märchen\pwindex{Maerchen. Schauspiel in drei Aufzuegen@\emph{Das Märchen. Schauspiel in drei Aufzügen}|pw} und Langka{\geminationm}ers\pwindex{Langkammer, Karl 04.08.1854 – 18.05.1936@\textsc{Langkammer, Karl} (04.08.1854 – 18.05.1936), \emph{Theaterleiter/Theaterleiterin, Regisseur/Regisseurin, Schauspieler/Schauspielerin}|pw} Urtheil. Aber ich habe wieder ſehr lebhafte
               Bedenken betreffs einer eventuellen Aufführung bekommen. Ich werde ja wohl bald
               Gelegenheit {[}haben{]}, ſowohl mit dir als mit Langkammer\pwindex{Langkammer, Karl 04.08.1854 – 18.05.1936@\textsc{Langkammer, Karl} (04.08.1854 – 18.05.1936), \emph{Theaterleiter/Theaterleiterin, Regisseur/Regisseurin, Schauspieler/Schauspielerin}|pw}{ }{\pb}darüber zu reden. Bis
               dahin beſte Grüße und nochmals vielen Dank.\pend
           \pstart Dein \spacefill\mbox{ArthSchn}\pend{}
\pstart
           \textsc{Berlin\oindex{Berlin@\textbf{Berlin}, \emph{P.PPLC}|pw}}{ }\substVorne{}\textsuperscript{6}\substDazwischen{}7\substHinten{}. 2. 96.\pend
           \selectlanguage{ngerman}\endnumbering\briefempfaengerindex{Bahr, Hermann@\textsc{Bahr, Hermann}!zzzSchnitzler, Arthur@\emph{von Arthur Schnitzler}!1896-02-071@{7. 2. 1896}|)be}\mylabel{L00533h}  \normalsize

\doendnotes{C}
\bigskip
\vfill

\clearpage

\footnotesize

\lohead{\textsc{register}}

% Definiere theindex-Environment komplett neu ohne reledmac
\makeatletter
\renewenvironment{theindex}{%
  \section*{\indexname}%
  \setlength{\parindent}{0pt}%
  \setlength{\parskip}{0pt plus 0.3pt}%
  \let\item\@idxitem
}{%
  \clearpage
}
\makeatother

\IfFileExists{\jobname-pw.ind}{\input{\jobname-pw.ind}}{}

\end{document}

      