%% latex-korrekturansicht-vorspann.tex
%% Vorspann für die Korrekturansicht.
%% Lädt die gemeinsame Datei latex-vorspann.tex mit gesetztem Schalter.

\newif\ifkorrekturansicht
\korrekturansichttrue

\input{../tex-inputs/latex-vorspann}


\section[Arthur Schnitzler an Richard Beer-Hofmann, {[}zwischen 21. 1. und 3. 12. 1897?{]}]{L00642 Arthur Schnitzler an Richard Beer-Hofmann, {[}zwischen 21. 1. und
               3. 12. 1897?{]}}
\nopagebreak\mylabel{L00642v}
\rehead{ }\normalsize\beginnumbering\briefempfaengerindex{Beer-Hofmann, Richard@\textsc{Beer-Hofmann, Richard}!zzzSchnitzler, Arthur@\emph{von Arthur Schnitzler}!1897-12-031@{{[}zwischen 21. 1. und
                  3. 12. 1897?{]}}|(be}
\toendnotes[C]{\smallbreak\pagebreak[2]}\Standort{YCGL, MSS 31.}
\physDesc{Brief, 1 Blatt, 2 Seiten, Umschlag, 225 Zeichen
\newline{}Handschrift: Bleistift, deutsche Kurrent
\newline{}Versand: ohne postalischen Übermittlungsvermerk }\toendnotes[C]{\smallbreak}\pstart{}{\pb}Herrn \textsc{Dr. Rich. Beer
                     Hofmann}\pend{}\pstart{}Wien\oindex{Wien@\textbf{Wien}, \emph{A.ADM2}|pw}\pend{}\pstart{}\textsc{I. Wollzeile 15\oindex{Wollzeile@\textbf{Wollzeile}, \emph{Straße (K.STR)}|pw}}.\pend{}{\bigskip}\vspace{1em}
\pstart{}{\pb}Lieber Richard,\pend\vspace{0.5em}
\pstart
           bitte ko{\geminationm}en Sie \label{K_L00642-1v}\edtext{heut Abd}{\lemma{\textnormal{\emph{heut Abd}}}\Cendnote{\textnormal{Das
                  Korrespondenzstück ist undatiert, aber in der Überlieferung dem Jahr
                     1897 zugeordnet. In diesem Jahr besuchte Schnitzler an folgenden Tagen Aufführungen im Carl-Theater\oindex{Carl-Theater@\textbf{Carl-Theater}, \emph{Theater (K.THE)}|pwk}: 21. 1. 1897, 27. 2. 1897, 27. 2. 1897, 13. 3. 1897, 12. 9. 1897, 26. 9. 1897, 28. 9. 1897, 30. 9. 1897, 27. 10. 1897, 30. 10. 1897, 3. 11. 1897 und 4. 12. 1897.}}}\label{K_L00642-1}{ }\textsc{Carl}theater\oindex{Carl-Theater@\textbf{Carl-Theater}, \emph{Theater (K.THE)}|pw} in die Loge \textsc{Parterre} rechts 2, {\pb}es wäre mir \uline{ſehr} lieb.\pend
           
\pstart
           Jedenfalls benachricht Sie mich, pneumatiſch oder teleph.\pend
           
\pstart
           Herzlichſt{\\[\baselineskip]}Ihr \spacefill\mbox{\substVorne{}\textsuperscript{Rich}\substDazwischen{}Arthur\substHinten{}}\pend
           \leftskip=0em{}\selectlanguage{ngerman}\endnumbering\briefempfaengerindex{Beer-Hofmann, Richard@\textsc{Beer-Hofmann, Richard}!zzzSchnitzler, Arthur@\emph{von Arthur Schnitzler}!1897-01-211@{{[}zwischen 21. 1. und
                  3. 12. 1897?{]}}|)be}\mylabel{L00642h}  \normalsize

\doendnotes{C}
\bigskip
\vfill

\clearpage

\footnotesize

\lohead{\textsc{register}}

% Definiere theindex-Environment komplett neu ohne reledmac
\makeatletter
\renewenvironment{theindex}{%
  \section*{\indexname}%
  \setlength{\parindent}{0pt}%
  \setlength{\parskip}{0pt plus 0.3pt}%
  \let\item\@idxitem
}{%
  \clearpage
}
\makeatother

\IfFileExists{\jobname-pw.ind}{\input{\jobname-pw.ind}}{}

\end{document}

      