%% latex-leseansicht-vorspann.tex
%% Vorspann für die Leseansicht.
%% Lädt die gemeinsame Datei latex-vorspann.tex mit nicht gesetztem Schalter.

\newif\ifkorrekturansicht
\korrekturansichtfalse

\input{../tex-inputs/latex-vorspann}


\section[Arthur Schnitzler an Richard Beer-Hofmann, 9. 9. 1900]{L01072 Arthur Schnitzler an Richard Beer-Hofmann, 9. 9. 1900}
\nopagebreak\mylabel{L01072v}
\rehead{ }\normalsize\beginnumbering\briefempfaengerindex{Beer-Hofmann, Richard@\textsc{Beer-Hofmann, Richard}!zzzSchnitzler, Arthur@\emph{von Arthur Schnitzler}!1900-09-091@{9. 9. 1900}|(be}
\toendnotes[C]{\smallbreak\pagebreak[2]}
\correspDesc{Versand  durch Arthur Schnitzler am 9. 9. 1900 in Wien
\newline{}Erhalt  durch Richard Beer-Hofmann im Zeitraum [10. 9. 1900
                  – 14. 9. 1900?] in Altaussee}\toendnotes[C]{\smallbreak}
\Standort{YCGL, MSS 31.}
\physDesc{Postkarte, 266 Zeichen
\newline{}Handschrift: Bleistift, deutsche Kurrent
\newline{}Versand: 1) Stempel: »\nobreak{}\oindex{I., Innere Stadt@\textbf{I., Innere Stadt}, \emph{Verwaltungsgebiet}|pwk}Wien 1/1 1, 9. 9. 00, 8–9V\nobreak{}«.   2) Stempel: »\nobreak{}\oindex{Altaussee@\textbf{Altaussee}, \emph{Verwaltungsgebiet}|pwk}Alt-Aussee\nobreak{}«. }\toendnotes[C]{\smallbreak}\pstart{}\textsc{{\pb}Herrn Dr. Rich.}\pend{}\pstart{}\textsc{Beer-Hofmann}\pend{}\pstart{}\textsc{Altaussee\oindex{Altaussee@\textbf{Altaussee}, \emph{Verwaltungsgebiet}|pw}}\pend{}{\bigskip}\vspace{1em}
\pstart
           \noindent{}{\pb}Paul\pwindex{Goldmann, Paul 31.\,1.\,1865 Breslau – 25.\,9.\,1935 Wien@\textsc{Goldmann, Paul} (31.\,1.\,1865 Breslau – 25.\,9.\,1935 Wien), \emph{Schriftsteller, Journalist}|pw} bleibt, we{\geminationn}s
               gut geht bis Ende der nächſten Woche, das wäre \uline{15}. We{\geminationn} Sie alſo erſt 18. kommen, finden
               Sie ihn nicht mehr. Im übrigen hoff ich, Sie{ }ſchreiben mir nächſtens nicht »\label{K_L01072-1v}\edtext{in Eile}{\lemma{\textnormal{\emph{in Eile}}}\Cendnote{\textnormal{Siehe XXXX Auszeichnungsfehler: Dokument L01071 nicht gefunden.
               }}}\label{K_L01072-1}«, und ich höre nur gutes.\pend
           
\pstart
           Herzlichſt Ihr{\\[\baselineskip]}\spacefill\mbox{Arthur}\pend
           \leftskip=0em{}\selectlanguage{ngerman}\endnumbering\briefempfaengerindex{Beer-Hofmann, Richard@\textsc{Beer-Hofmann, Richard}!zzzSchnitzler, Arthur@\emph{von Arthur Schnitzler}!1900-09-091@{9. 9. 1900}|)be}\mylabel{L01072h}  \newcommand{\dateiname}{L01072}\newcommand{\titel}{Arthur Schnitzler an Richard Beer-Hofmann, 9. 9. 1900}\newcommand{\editorInnen}{Martin Anton Müller und Gerd-Hermann Susen}%% latex-leseansicht-abspann.tex
%% Abspann für die Leseansicht.
%% Der Schalter \ifkorrekturansicht ist bereits durch den Vorspann gesetzt.

%% latex-abspann.tex
%% Gemeinsamer Abspann für Korrekturansicht und Leseansicht.
%% Setzt den Schalter \ifkorrekturansicht voraus (gesetzt in den
%% einbindenden Dateien latex-korrekturansicht-abspann.tex bzw.
%% latex-leseansicht-abspann.tex).
%% ---------------------------------------------------------------

\normalsize

% Das esempio-Environment wird nur in der Leseansicht benötigt
\ifkorrekturansicht\else
\newenvironment{esempio}[3]%
{
    \vspace{1.5ex}
    \rlap{\underline{#1}}
    \par
    \setlength{\parindent}{0cm}
    \nopagebreak
    \leftskip=#2cm
    \rightskip=#3cm
}
{
    \par
}
\fi

\doendnotes{C}
\bigskip
\vfill

\clearpage

\footnotesize

\ifkorrekturansicht
  \lohead{\textsc{register}}
\fi

% theindex-Environment neu definieren ohne reledmac
\makeatletter
\renewenvironment{theindex}{%
  \ifkorrekturansicht
    \section*{\indexname}%
  \else
    \subsubsection*{Index der erwähnten Entitäten}%
  \fi
  \setlength{\parindent}{0pt}%
  \setlength{\parskip}{0pt plus 0.3pt}%
  \let\item\@idxitem
}{%
  \ifkorrekturansicht\clearpage\fi
}
\makeatother

\IfFileExists{\jobname-pw.ind}{\input{\jobname-pw.ind}}{}

% Quellenangabe nur in der Leseansicht
\ifkorrekturansicht\else
% Fallback-Definitionen, falls die .tex-Datei \titel etc. nicht gesetzt hat
\providecommand{\titel}{}
\providecommand{\editorInnen}{}
\providecommand{\dateiname}{\jobname}

\vspace{3cm}

\vfill

\footnotesize
\textsc{Quelle}: \titel. Herausgegeben von {\editorInnen}. In: \emph{Arthur Schnitzler: Briefwechsel mit Autorinnen und Autoren}.
 Digitale Edition, https://schnitzler-briefe.acdh.oeaw.ac.at/{\dateiname}.html (Stand \today)
\fi

\end{document}


