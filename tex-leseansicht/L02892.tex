%% latex-korrekturansicht-vorspann.tex
%% Vorspann für die Korrekturansicht.
%% Lädt die gemeinsame Datei latex-vorspann.tex mit gesetztem Schalter.

\newif\ifkorrekturansicht
\korrekturansichttrue

\input{../tex-inputs/latex-vorspann}


\section[ Paul Goldmann an Arthur Schnitzler, 26. 10. 1899]{L02892 Paul Goldmann an Arthur Schnitzler, 26. 10. 1899}
\nopagebreak\mylabel{L02892v}
\rehead{ }\normalsize\beginnumbering\briefempfaengerindex{Schnitzler, Arthur@\textsc{Schnitzler, Arthur}!zzzGoldmann, Paul@\emph{von Paul Goldmann}!1899-10-261@{26. 10. 1899}|(be}
\toendnotes[C]{\smallbreak\pagebreak[2]}\Standort{DLA, A:Schnitzler, HS.NZ85.1.3169.}
\physDesc{Brief, 1 Blatt, 4 Seiten, 1918 Zeichen
\newline{}Handschrift: schwarze Tinte, deutsche Kurrent
\newline{}Beilage: ein beschnittener Zeitungsausschnitt auf der letzten
                                 Seite 
\newline{}Schnitzler: mit rotem Buntstift eine Unterstreichung }\toendnotes[C]{\smallbreak}
\pstart
           {\pb}\textcolor{gray}{\textbf{\textbf{Frankfurter Zeitung}}}\orgindex{Frankfurter Zeitung@Frankfurter Zeitung|pw}\hfill \textcolor{gray}{\textbf{\textbf{Frankfurt a. M.\oindex{Frankfurt am Main@\textbf{Frankfurt am Main}, \emph{P.PPLA3}|pw},}}}{ }26. Oktober \textcolor{gray}{\textbf{189}}9.\pend
           
\pstart
           \textcolor{gray}{\textbf{und}}\pend
           
\pstart
           \textcolor{gray}{\textbf{Handelsblatt.}}\pend
           
\pstart
           \textcolor{gray}{\textbf{\textbf{Redaktion\orgindex{Frankfurter Zeitung@Frankfurter Zeitung|pwv}.}\noindent{}\textcolor{gray}{\textbf{Für die Redaktion\orgindex{Frankfurter Zeitung@Frankfurter Zeitung|pwv} beſtimmte Briefe und Sendungen wolle man
                                 \so{nicht} an die Perſon eines Redakteurs,
                              ſondern ſtets \textbf{an die Redaktion der Frankfurter Zeitung\orgindex{Frankfurter Zeitung@Frankfurter Zeitung|pw}} adreſſiren. }}}}\pend
           
\pstart
           \textcolor{gray}{\textbf{Telegramm-Adreſſe:}}\pend
           
\pstart
           \textcolor{gray}{\textbf{\textbf{Zeitung\orgindex{Frankfurter Zeitung@Frankfurter Zeitung|pwv}{ }Frankfurt Main\oindex{Frankfurt am Main@\textbf{Frankfurt am Main}, \emph{P.PPLA3}|pw}.}}}\pend
           
\pstart\center{}Mein lieber Freund,\pend\vspace{0.5em}
\pstart
           Mit der »Neuen Freien Preſſe\orgindex{Neue Freie Presse@Neue Freie Presse|pw}« iſt es alſo \label{K_L02892-1v}\edtext{auch diesmal}{\lemma{\textnormal{\emph{auch diesmal}}}\Cendnote{\textnormal{Erst ab 1900 war Goldmann\pwindex{Goldmann, Paul 31.01.1865 – 25.09.1935@\textsc{Goldmann, Paul} (31.01.1865 – 25.09.1935), \emph{Schriftsteller/Schriftstellerin, Journalist/Journalistin}|pwk} Theaterkorrespondent der \emph{Neuen Freien Presse}\orgindex{Neue Freie Presse@Neue Freie Presse|pwk} in Berlin\oindex{Berlin@\textbf{Berlin}, \emph{P.PPLC}|pwk} (siehe Paul Goldmann an Arthur Schnitzler, 5. 3. [1899] und
                     Paul Goldmann an Arthur Schnitzler, 4. 12. [1899]).}}}\label{K_L02892-1} nichts. Nachdem die
                  Herausgeber\pwindex{Bacher, Eduard 07.03.1846 – 16.01.1908@\textsc{Bacher, Eduard} (07.03.1846 – 16.01.1908), \emph{Journalist/Journalistin, Herausgeber/Herausgeberin}|pwv}\pwindex{Benedikt, Moriz 27.05.1849 – 18.03.1920@\textsc{Benedikt, Moriz} (27.05.1849 – 18.03.1920), \emph{Journalist/Journalistin, Herausgeber/Herausgeberin}|pwv}
               mich ſo furchtbar gedrängt, telegraphirte ich ſofort nach meinem Eintreffen in Frankfurt\oindex{Frankfurt am Main@\textbf{Frankfurt am Main}, \emph{P.PPLA3}|pw}, ich ſei bereit, am 1. Jänner in Berlin\oindex{Berlin@\textbf{Berlin}, \emph{P.PPLC}|pw}
               anzutreten. Zugleich ſetzte ich brieflich meine materiellen Bedingungen auseinander.
                  Geſtern erhielt ich nun ein Telegramm der Herausgeber\pwindex{Bacher, Eduard 07.03.1846 – 16.01.1908@\textsc{Bacher, Eduard} (07.03.1846 – 16.01.1908), \emph{Journalist/Journalistin, Herausgeber/Herausgeberin}|pwv}\pwindex{Benedikt, Moriz 27.05.1849 – 18.03.1920@\textsc{Benedikt, Moriz} (27.05.1849 – 18.03.1920), \emph{Journalist/Journalistin, Herausgeber/Herausgeberin}|pwv} der N. Fr. Pr.\orgindex{Neue Freie Presse@Neue Freie Presse|pw}, worin ſie mir mittheilten, daß ſie
               meine materiellen Bedingungen wohl acceptiren würden, daß aber die Nachrichten
               inbezug auf \label{K_L02892-2v}\edtext{\textsc{Frischauers\pwindex{Frischauer, Berthold 1851-09-09 – 1924-02-04@\textsc{Frischauer, Berthold} (1851-09-09 – 1924-02-04), \emph{Journalist/Journalistin}|pw}} Rückkehr nach \textsc{Paris\oindex{Paris@\textbf{Paris}, \emph{P.PPLC}|pw}}}{\lemma{\textnormal{\emph{Frischauers … Paris}}}\Cendnote{\textnormal{Berthold Frischauer\pwindex{Frischauer, Berthold 1851-09-09 – 1924-02-04@\textsc{Frischauer, Berthold} (1851-09-09 – 1924-02-04), \emph{Journalist/Journalistin}|pwk} war seit 1895 der Nachfolger Theodor Herzls\pwindex{Herzl, Theodor 1860-05-02 – 1904-07-03@\textsc{Herzl, Theodor} (1860-05-02 – 1904-07-03), \emph{Schriftsteller/Schriftstellerin, Journalist/Journalistin}|pwk} als Korrespondent der \emph{Neuen
                     Freien Presse}\orgindex{Neue Freie Presse@Neue Freie Presse|pwk} in Paris\oindex{Paris@\textbf{Paris}, \emph{P.PPLC}|pwk}. Am 16. 2. 1899 war er wegen Ehrenbeleidigung der \emph{französischen Armee}\orgindex{Franzoesische Streitkraefte@Französische Streitkräfte|pwk} im Rahmen seiner
                  Berichterstattung zur Dreyfus\pwindex{Dreyfus, Alfred 1859-10-09 – 1935-07-12@\textsc{Dreyfus, Alfred} (1859-10-09 – 1935-07-12), \emph{Militär/Militärin}|pwk}-Affäre aus Frankreich\oindex{Frankreich@\textbf{Frankreich}, \emph{A.PCLI}|pwk} ausgewiesen worden. Anfang Dezember 1899 wurde ihm die Einreise wieder gestattet und
                  er kehrte zurück. In der Zwischenzeit dürfte er in Berlin\oindex{Berlin@\textbf{Berlin}, \emph{P.PPLC}|pwk} eingesetzt gewesen sein.}}}\label{K_L02892-2} jetzt wieder ſehr ungünſtig
               lauteten. Zugleich wurde mir vorgeſchlagen, für die N.
                  Fr. Pr.\orgindex{Neue Freie Presse@Neue Freie Presse|pw}{ }{\pb}nach \textsc{Paris\oindex{Paris@\textbf{Paris}, \emph{P.PPLC}|pw}} zu gehen. Dieſen Vorſchlag habe ich ſelbſtverſtändlich abgelehnt, und ſo
               bleibt’s beim Alten. Glücklicher Weiſe \strikeout{war} bin ich
               vorſichtig genug geweſen, hier\orgindex{Frankfurter Zeitung@Frankfurter Zeitung|pwuv} meine Beziehungen noch nicht abzubrechen. Sonſt wäre ich
               jetzt ohne Stellung. Hoffentlich erfährt man auch in Frankfurt\oindex{Frankfurt am Main@\textbf{Frankfurt am Main}, \emph{P.PPLA3}|pw} nichts von den geführten Verhandlungen, und ich bitte Dich, die
               ganze Angelegenheit \strikeout{\textcolor{gray}{d}} diskret zu behandeln. Aber was ſagſt Du zu dieſen Zeitungs-\textsc{Paschahs}\pwindex{Bacher, Eduard 07.03.1846 – 16.01.1908@\textsc{Bacher, Eduard} (07.03.1846 – 16.01.1908), \emph{Journalist/Journalistin, Herausgeber/Herausgeberin}|pwv}\pwindex{Benedikt, Moriz 27.05.1849 – 18.03.1920@\textsc{Benedikt, Moriz} (27.05.1849 – 18.03.1920), \emph{Journalist/Journalistin, Herausgeber/Herausgeberin}|pwv}, die Einen über Hals und Kopf \strikeout{für} in eine
               Stellung hineinhetzen und erſt hinterher merken, daß die Stellung noch gar nicht frei
               iſt?\pend
           
\pstart
           Ich ſende Dir anbei Dein \label{K_L02892-3v}\edtext{Burgtheater-Referat\pwindex{Wiener Burgtheater. (»Agnes Jordan« von Georg Hirschfeld.)@\emph{Wiener Burgtheater. (»Agnes Jordan« von Georg Hirschfeld.)}|pwv}}{\lemma{\textnormal{\emph{Burgtheater-Referat}}}\Cendnote{\textnormal{Beilage nicht erhalten. –rm– [ = Arthur Schnitzler]: \emph{Wiener Burgtheater. (»Agnes Jordan« von Georg
                        Hirschfeld)}\pwindex{Wiener Burgtheater. (»Agnes Jordan« von Georg Hirschfeld.)@\emph{Wiener Burgtheater. (»Agnes Jordan« von Georg Hirschfeld.)}|pwk}. In: \emph{Frankfurter
                        Zeitung}\pwindex{Frankfurter Zeitung@\emph{Frankfurter Zeitung}|pwk}, Jg. 44, Nr. 296, 25. 10. 1899, Zweites Morgenblatt, S. 1. Siehe Paul Goldmann an Arthur Schnitzler, 23. 10. [1899].}}}\label{K_L02892-3}. Selbſt ich
               habe nicht alle Worte der Handſchrift entziffern können, und mein {\pb}Onkel\pwindex{Mamroth, Fedor 21.02.1851 – 25.06.1907@\textsc{Mamroth, Fedor} (21.02.1851 – 25.06.1907), \emph{Journalist/Journalistin, Kritiker/Kritikerin}|pwv} hat ſich leider für
               verpflichtet gehalten, zwei Stellen\pwindex{Wiener Burgtheater. (»Agnes Jordan« von Georg Hirschfeld.)@\emph{Wiener Burgtheater. (»Agnes Jordan« von Georg Hirschfeld.)}|pwv}, für die er nicht die Verantwortung übernehmen wollte,
               herauszuſtreichen. Ich \strikeout{ko\textcolor{gray}{n}} konnte da nichts hindern. In redaktion\orgindex{Frankfurter Zeitung@Frankfurter Zeitung|pwv}ellen Angelegenheiten iſt mein Onkel\pwindex{Mamroth, Fedor 21.02.1851 – 25.06.1907@\textsc{Mamroth, Fedor} (21.02.1851 – 25.06.1907), \emph{Journalist/Journalistin, Kritiker/Kritikerin}|pwv} unumſchränkter Gebieter.\pend
           
\pstart
           Gegen \textsc{Wassermann\pwindex{Wassermann, Jakob 10.03.1873 – 01.01.1934@\textsc{Wassermann, Jakob} (10.03.1873 – 01.01.1934), \emph{Schriftsteller/Schriftstellerin}|pw}} iſt die Stimmmung in der Redaktion\orgindex{Frankfurter Zeitung@Frankfurter Zeitung|pwv}{ }\strikeout{u\textcolor{gray}{×}\textcolor{gray}{f}} ſchlechter als je, und ich bin überzeugt, daß er bei der nächſten Gelegenheit
                  \label{K_L02892-4v}\edtext{hinausfliegt}{\lemma{\textnormal{\emph{hinausfliegt}}}\Cendnote{\textnormal{Jakob Wassermann\pwindex{Wassermann, Jakob 10.03.1873 – 01.01.1934@\textsc{Wassermann, Jakob} (10.03.1873 – 01.01.1934), \emph{Schriftsteller/Schriftstellerin}|pwk} verlor seine Stelle als
                     Wien\oindex{Wien@\textbf{Wien}, \emph{A.ADM2}|pwk}er Theaterkorrespondent der \emph{Frankfurter Zeitung}\orgindex{Frankfurter Zeitung@Frankfurter Zeitung|pwk} mit dem 1. 1. 1900, vgl. Paul Goldmann an Arthur Schnitzler, 6. 12. [1899].}}}\label{K_L02892-4}.\pend
           {\vspace{1\baselineskip}}
\pstart
           Wie Du aus dem \substVorne{}\textsuperscript{beiliegenden}\substDazwischen{}nachfolgenden\substHinten{} kl. \label{K_L02892-5v}\edtext{Referat\pwindex{Kleines Feuilleton. [Kleine Mittheilungen.]@\emph{Kleines Feuilleton. [Kleine Mittheilungen.]}|pwv}}{\lemma{\textnormal{\emph{Referat}}}\Cendnote{\textnormal{[O. V.]: \emph{Kleines Feuilleton. [Kleine
                        Mittheilungen]}\pwindex{Kleines Feuilleton. [Kleine Mittheilungen.]@\emph{Kleines Feuilleton. [Kleine Mittheilungen.]}|pwk}. In: \emph{Frankfurter
                        Zeitung}\pwindex{Frankfurter Zeitung@\emph{Frankfurter Zeitung}|pwk}, Jg. 44, Nr. 297, 26. 10. 1899,
                     Abendblatt, S. 2.}}}\label{K_L02892-5} erſiehſt, ſind Deine \label{K_L02892-6v}\edtext{drei Einakter\pwindex{gruene Kakadu – Paracelsus – Die Gefaehrtin. Drei Einakter@\emph{Der grüne Kakadu – Paracelsus – Die Gefährtin. Drei Einakter}|pwv} am Darmſtädter Hoftheater\oindex{Staatstheater Darmstadt@\textbf{Staatstheater Darmstadt}, \emph{Theater (K.THE)}|pw} geſpielt}{\lemma{\textnormal{\emph{drei … geſpielt}}}\Cendnote{\textnormal{Die Einakter\pwindex{gruene Kakadu – Paracelsus – Die Gefaehrtin. Drei Einakter@\emph{Der grüne Kakadu – Paracelsus – Die Gefährtin. Drei Einakter}|pwkv}{ }\emph{Paracelsus}\pwindex{Paracelsus. Versspiel in einem Akt@\emph{Paracelsus. Versspiel in einem Akt}|pwk}, \emph{Die Gefährtin}\pwindex{Gefaehrtin. Schauspiel in einem Akt@\emph{Die Gefährtin. Schauspiel in einem Akt}|pwk} und \emph{Der grüne Kakadu}\pwindex{gruene Kakadu. Groteske in einem Akt@\emph{Der grüne Kakadu. Groteske in einem Akt}|pwk}
                  wurden am 24. 10. 1899 sowie am 3. 11. 1899 im Darmstädter Hoftheater\oindex{Staatstheater Darmstadt@\textbf{Staatstheater Darmstadt}, \emph{Theater (K.THE)}|pwk} aufgeführt.}}}\label{K_L02892-6} worden.\pend
           
\pstart
           Bitte, ſchreib’ mir bald, wie es Dir geht (Stimmung und Geſundheit).\pend
           
\pstart
           Viele treue Grüße! {\\[\baselineskip]}Dein {\\[\baselineskip]}\spacefill\mbox{Paul Goldmann.}\pend
           \leftskip=0em{}{\vspace{1\baselineskip}}
\pstart
           {\pb}\textcolor{gray}{\textbf{– Man berichtet uns aus \so{Darmſtadt}\oindex{Darmstadt@\textbf{Darmstadt}, \emph{P.PPLA2}|pw} v. 25. ds.: Zu Ehren des Dichter-Komponiſten
                     Peter \so{Cornelius}\pwindex{Cornelius, Peter 24.12.1824 – 26.10.1874@\textsc{Cornelius, Peter} (24.12.1824 – 26.10.1874), \emph{Komponist/Komponistin}|pw} veranſtaltete am Montag der \so{Richard Wagner-Verein}\orgindex{Richard-Wagner-Verein@Richard-Wagner-Verein|pw} einen Concertabend, an welchem, mit einer Ausnahme, lediglich Kompoſitionen
                  von Cornelius\pwindex{Cornelius, Peter 24.12.1824 – 26.10.1874@\textsc{Cornelius, Peter} (24.12.1824 – 26.10.1874), \emph{Komponist/Komponistin}|pw} zum Vortrag gelangten. Die
                  Chöre ſtellte der Mozart-Verein\orgindex{Mozartverein Darmstadt@Mozartverein Darmstadt|pw}, als Soliſten
                  traten auf Frl. \so{Zinkeiſen}\pwindex{Zinkeisen, Anna 1866-12-17 – nach 1930@\textsc{Zinkeisen, Anna} (1866-12-17 – nach 1930), \emph{Pianist/Pianistin}|pw} aus Frankfurt a. M.\oindex{Frankfurt am Main@\textbf{Frankfurt am Main}, \emph{P.PPLA3}|pw}, Frau \so{Senff}\pwindex{Senff @\textsc{Senff}, \emph{Musiker/Musikerin}|pw}–Darmſtadt\oindex{Darmstadt@\textbf{Darmstadt}, \emph{P.PPLA2}|pw} und Herr \so{Joachim}\pwindex{Joachim, Bruno @\textsc{Joachim, Bruno}, \emph{Sänger/Sängerin, Bariton/}|pw}–Darmſtadt\oindex{Darmstadt@\textbf{Darmstadt}, \emph{P.PPLA2}|pw}. Das zahlreich erſchienene
                  Publikum dankte ſehr lebhaft für das Gebotene. Im \so{Hoftheater}\oindex{Staatstheater Darmstadt@\textbf{Staatstheater Darmstadt}, \emph{Theater (K.THE)}|pw} kamen geſtern{ }Abend{ }\so{Schnitzler’s}{ }Einakter\pwindex{gruene Kakadu – Paracelsus – Die Gefaehrtin. Drei Einakter@\emph{Der grüne Kakadu – Paracelsus – Die Gefährtin. Drei Einakter}|pwv} »Paracelſus\pwindex{Paracelsus. Versspiel in einem Akt@\emph{Paracelsus. Versspiel in einem Akt}|pw}«, »Die
                     Gefährtin\pwindex{Gefaehrtin. Schauspiel in einem Akt@\emph{Die Gefährtin. Schauspiel in einem Akt}|pw}« und »Der grüne Kakadu\pwindex{gruene Kakadu. Groteske in einem Akt@\emph{Der grüne Kakadu. Groteske in einem Akt}|pw}« zur
                  erſten Aufführung. Die Aufnahme war eine recht freundliche, wennſchon »Der grüne Kakadu\pwindex{gruene Kakadu. Groteske in einem Akt@\emph{Der grüne Kakadu. Groteske in einem Akt}|pw}« einigen Widerſpruch erregte.
                  Geſpielt wurde namentlich von Herrn \so{Hacker}\pwindex{Hacker, Georg Heinrich 1856-01-16 – 1922-08-15@\textsc{Hacker, Georg Heinrich} (1856-01-16 – 1922-08-15), \emph{Schauspieler/Schauspielerin}|pw} (Paracelſus\pwindex{Paracelsus. Versspiel in einem Akt@\emph{Paracelsus. Versspiel in einem Akt}|pwv}, Pilgram\pwindex{Gefaehrtin. Schauspiel in einem Akt@\emph{Die Gefährtin. Schauspiel in einem Akt}|pw} und Cardignan\pwindex{gruene Kakadu. Groteske in einem Akt@\emph{Der grüne Kakadu. Groteske in einem Akt}|pwv}) und Herrn \so{Löhr}\pwindex{Loehr, Willy 1872-12-21 – 1940-06-27@\textsc{Loehr, Willy} (1872-12-21 – 1940-06-27), \emph{Regisseur/Regisseurin, Schauspieler/Schauspielerin, Intendant/Intendantin}|pw} (Hausmann\pwindex{Gefaehrtin. Schauspiel in einem Akt@\emph{Die Gefährtin. Schauspiel in einem Akt}|pwv} und Henri\pwindex{gruene Kakadu. Groteske in einem Akt@\emph{Der grüne Kakadu. Groteske in einem Akt}|pwv}) recht gut. Herr \so{Conradi}\pwindex{Conradi, Gustav 1850-08-15 – 1926-03-13@\textsc{Conradi, Gustav} (1850-08-15 – 1926-03-13), \emph{Schauspieler/Schauspielerin}|pw} konnte als Strolch Grain\pwindex{gruene Kakadu. Groteske in einem Akt@\emph{Der grüne Kakadu. Groteske in einem Akt}|pwv} einen ſtarken Heiterkeitserfolg verzeichnen. \so{Schiller}\pwindex{Schiller, Friedrich von 10.11.1759 – 09.05.1805@\textsc{Schiller, Friedrich von} (10.11.1759 – 09.05.1805), \emph{Schriftsteller/Schriftstellerin, Historiker/Historikerin, Philosoph/Philosophin}|pw}\so{s}{ }Geburtstag wird hier durch Aufführungen der »Wallenſtein«-Trilogie\pwindex{Wallenstein@\emph{Wallenstein}|pw} und der »Jungfrau von Orleans\pwindex{Jungfrau von Orleans@\emph{Die Jungfrau von Orleans}|pw}« gefeiert werden. –}}\pend
           \selectlanguage{ngerman}\endnumbering\briefempfaengerindex{Schnitzler, Arthur@\textsc{Schnitzler, Arthur}!zzzGoldmann, Paul@\emph{von Paul Goldmann}!1899-10-261@{26. 10. 1899}|)be}\mylabel{L02892h}  \normalsize

\doendnotes{C}
\bigskip
\vfill

\clearpage

\footnotesize

\lohead{\textsc{register}}

% Definiere theindex-Environment komplett neu ohne reledmac
\makeatletter
\renewenvironment{theindex}{%
  \section*{\indexname}%
  \setlength{\parindent}{0pt}%
  \setlength{\parskip}{0pt plus 0.3pt}%
  \let\item\@idxitem
}{%
  \clearpage
}
\makeatother

\IfFileExists{\jobname-pw.ind}{\input{\jobname-pw.ind}}{}

\end{document}

      