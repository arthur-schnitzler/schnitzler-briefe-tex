%% latex-leseansicht-vorspann.tex
%% Vorspann für die Leseansicht.
%% Lädt die gemeinsame Datei latex-vorspann.tex mit nicht gesetztem Schalter.

\newif\ifkorrekturansicht
\korrekturansichtfalse

\input{../tex-inputs/latex-vorspann}


\section[ Paul Goldmann an Arthur Schnitzler, 26. 10. 1899]{L02892 Paul Goldmann an Arthur Schnitzler,  26. 10. 1899}
\nopagebreak\mylabel{L02892v}
\rehead{ }\normalsize\beginnumbering\briefempfaengerindex{Schnitzler, Arthur@\textsc{Schnitzler, Arthur}!zzzGoldmann, Paul@\emph{von Paul Goldmann}!1899-10-261@{26. 10. 1899}|(be}
\toendnotes[C]{\smallbreak\pagebreak[2]}
\correspDesc{Versand  durch Paul Goldmann am 26. 10. 1899 in Frankfurt am Main
\newline{}Erhalt  durch Arthur Schnitzler im Zeitraum [27. 10. 1899 – 31. 10. 1899?] in Wien}\toendnotes[C]{\smallbreak}
\Standort{DLA, A:Schnitzler, HS.NZ85.1.3169.}
\physDesc{Brief, 1 Blatt, 4 Seiten, 1918 Zeichen
\newline{}Handschrift: schwarze Tinte, deutsche Kurrent
\newline{}Beilage: ein beschnittener Zeitungsausschnitt auf der letzten
                                 Seite 
\newline{}Schnitzler: mit rotem Buntstift eine Unterstreichung }\toendnotes[C]{\smallbreak}
\pstart
           {\pb}\textcolor{gray}{\textbf{\textbf{Frankfurter Zeitung}}}\orgindex{Frankfurter Zeitung@Frankfurter Zeitung|pw}\hfill \textcolor{gray}{\textbf{\textbf{Frankfurt a. M.\oindex{Frankfurt am Main@\textbf{Frankfurt am Main}, \emph{Hauptstadt}|pw},}}}{ }26. Oktober \textcolor{gray}{\textbf{189}}9.\pend
           
\pstart
           \textcolor{gray}{\textbf{und}}\pend
           
\pstart
           \textcolor{gray}{\textbf{Handelsblatt.}}\pend
           
\pstart
           \textcolor{gray}{\textbf{\textbf{Redaktion\orgindex{Frankfurter Zeitung@Frankfurter Zeitung|pwv}.}\footnote{\noindent{}\textcolor{gray}{\textbf{Für die Redaktion\orgindex{Frankfurter Zeitung@Frankfurter Zeitung|pwv} beſtimmte Briefe und Sendungen wolle man
                                 \so{nicht} an die Perſon eines Redakteurs,{ }ſondern{ }ſtets \textbf{an die Redaktion der Frankfurter Zeitung\orgindex{Frankfurter Zeitung@Frankfurter Zeitung|pw}} adreſſiren.}}}}}\pend
           
\pstart
           \textcolor{gray}{\textbf{Telegramm-Adreſſe:}}\pend
           
\pstart
           \textcolor{gray}{\textbf{\textbf{Zeitung\orgindex{Frankfurter Zeitung@Frankfurter Zeitung|pwv}{ }Frankfurt Main\oindex{Frankfurt am Main@\textbf{Frankfurt am Main}, \emph{Hauptstadt}|pw}.}}}\pend
           
\pstart\center{}Mein lieber Freund,\pend\vspace{0.5em}
\pstart
           Mit der »Neuen Freien Preſſe\orgindex{Neue Freie Presse@Neue Freie Presse|pw}« iſt es alſo \label{K_L02892-1v}\edtext{auch diesmal}{\lemma{\textnormal{\emph{auch diesmal}}}\Cendnote{\textnormal{Erst ab 1900 war Goldmann\pwindex{Goldmann, Paul 31.\,1.\,1865 Breslau – 25.\,9.\,1935 Wien@\textsc{Goldmann, Paul} (31.\,1.\,1865 Breslau – 25.\,9.\,1935 Wien), \emph{Schriftsteller, Journalist}|pwk} Theaterkorrespondent der \emph{Neuen Freien Presse}\orgindex{Neue Freie Presse@Neue Freie Presse|pwk} in Berlin\oindex{Berlin@\textbf{Berlin}, \emph{Hauptstadt}|pwk} (siehe XXXX Auszeichnungsfehler: Dokument L02868 nicht gefunden und
                     XXXX Auszeichnungsfehler: Dokument L02896 nicht gefunden).}}}\label{K_L02892-1} nichts. Nachdem die
                  Herausgeber\pwindex{Bacher, Eduard 7.\,3.\,1846 Postoloprty – 16.\,1.\,1908 Wien@\textsc{Bacher, Eduard} (7.\,3.\,1846 Postoloprty – 16.\,1.\,1908 Wien), \emph{Journalist, Herausgeber}|pwv}\pwindex{Benedikt, Moriz 27.\,5.\,1849 Kvačice – 18.\,3.\,1920 Wien@\textsc{Benedikt, Moriz} (27.\,5.\,1849 Kvačice – 18.\,3.\,1920 Wien), \emph{Journalist, Herausgeber}|pwv}
               mich{ }ſo furchtbar gedrängt, telegraphirte ich{ }ſofort nach meinem Eintreffen in Frankfurt\oindex{Frankfurt am Main@\textbf{Frankfurt am Main}, \emph{Hauptstadt}|pw}, ich{ }ſei bereit, am 1. Jänner in Berlin\oindex{Berlin@\textbf{Berlin}, \emph{Hauptstadt}|pw}
               anzutreten. Zugleich{ }ſetzte ich brieflich meine materiellen Bedingungen auseinander.
                  Geſtern erhielt ich nun ein Telegramm der Herausgeber\pwindex{Bacher, Eduard 7.\,3.\,1846 Postoloprty – 16.\,1.\,1908 Wien@\textsc{Bacher, Eduard} (7.\,3.\,1846 Postoloprty – 16.\,1.\,1908 Wien), \emph{Journalist, Herausgeber}|pwv}\pwindex{Benedikt, Moriz 27.\,5.\,1849 Kvačice – 18.\,3.\,1920 Wien@\textsc{Benedikt, Moriz} (27.\,5.\,1849 Kvačice – 18.\,3.\,1920 Wien), \emph{Journalist, Herausgeber}|pwv} der N. Fr. Pr.\orgindex{Neue Freie Presse@Neue Freie Presse|pw}, worin{ }ſie mir mittheilten, daß{ }ſie
               meine materiellen Bedingungen wohl acceptiren würden, daß aber die Nachrichten
               inbezug auf \label{K_L02892-2v}\edtext{\textsc{Frischauers\pwindex{Frischauer, Berthold 9.\,9.\,1851 Brünn – 4.\,2.\,1924 Wien@\textsc{Frischauer, Berthold} (9.\,9.\,1851 Brünn – 4.\,2.\,1924 Wien), \emph{Journalist}|pw}} Rückkehr nach \textsc{Paris\oindex{Paris@\textbf{Paris}, \emph{Hauptstadt}|pw}}}{\lemma{\textnormal{\emph{Frischauers … Paris}}}\Cendnote{\textnormal{Berthold Frischauer\pwindex{Frischauer, Berthold 9.\,9.\,1851 Brünn – 4.\,2.\,1924 Wien@\textsc{Frischauer, Berthold} (9.\,9.\,1851 Brünn – 4.\,2.\,1924 Wien), \emph{Journalist}|pwk} war seit 1895 der Nachfolger Theodor Herzls\pwindex{Herzl, Theodor 2.\,5.\,1860 Budapest – 3.\,7.\,1904 Edlach@\textsc{Herzl, Theodor} (2.\,5.\,1860 Budapest – 3.\,7.\,1904 Edlach), \emph{Schriftsteller, Journalist}|pwk} als Korrespondent der \emph{Neuen
                     Freien Presse}\orgindex{Neue Freie Presse@Neue Freie Presse|pwk} in Paris\oindex{Paris@\textbf{Paris}, \emph{Hauptstadt}|pwk}. Am 16. 2. 1899 war er wegen Ehrenbeleidigung der \emph{französischen Armee}\orgindex{Französische Streitkräfte@Französische Streitkräfte|pwk} im Rahmen seiner
                  Berichterstattung zur Dreyfus\pwindex{Dreyfus, Alfred 9.\,10.\,1859 Mulhouse – 12.\,7.\,1935 Paris@\textsc{Dreyfus, Alfred} (9.\,10.\,1859 Mulhouse – 12.\,7.\,1935 Paris), \emph{Militär}|pwk}-Affäre aus Frankreich\oindex{Frankreich@\textbf{Frankreich}|pwk} ausgewiesen worden. Anfang Dezember 1899 wurde ihm die Einreise wieder gestattet und
                  er kehrte zurück. In der Zwischenzeit dürfte er in Berlin\oindex{Berlin@\textbf{Berlin}, \emph{Hauptstadt}|pwk} eingesetzt gewesen sein.}}}\label{K_L02892-2} jetzt wieder{ }ſehr ungünſtig
               lauteten. Zugleich wurde mir vorgeſchlagen, für die N.
                  Fr. Pr.\orgindex{Neue Freie Presse@Neue Freie Presse|pw}{ }{\pb}nach \textsc{Paris\oindex{Paris@\textbf{Paris}, \emph{Hauptstadt}|pw}} zu gehen. Dieſen Vorſchlag habe ich{ }ſelbſtverſtändlich abgelehnt, und{ }ſo
               bleibt’s beim Alten. Glücklicher Weiſe \strikeout{war} bin ich
               vorſichtig genug geweſen, hier\orgindex{Frankfurter Zeitung@Frankfurter Zeitung|pwuv} meine Beziehungen noch nicht abzubrechen. Sonſt wäre ich
               jetzt ohne Stellung. Hoffentlich erfährt man auch in Frankfurt\oindex{Frankfurt am Main@\textbf{Frankfurt am Main}, \emph{Hauptstadt}|pw} nichts von den geführten Verhandlungen, und ich bitte Dich, die
               ganze Angelegenheit \strikeout{\textcolor{gray}{d}} diskret zu behandeln. Aber was{ }ſagſt Du zu dieſen Zeitungs-\textsc{Paschahs}\pwindex{Bacher, Eduard 7.\,3.\,1846 Postoloprty – 16.\,1.\,1908 Wien@\textsc{Bacher, Eduard} (7.\,3.\,1846 Postoloprty – 16.\,1.\,1908 Wien), \emph{Journalist, Herausgeber}|pwv}\pwindex{Benedikt, Moriz 27.\,5.\,1849 Kvačice – 18.\,3.\,1920 Wien@\textsc{Benedikt, Moriz} (27.\,5.\,1849 Kvačice – 18.\,3.\,1920 Wien), \emph{Journalist, Herausgeber}|pwv}, die Einen über Hals und Kopf \strikeout{für} in eine
               Stellung hineinhetzen und erſt hinterher merken, daß die Stellung noch gar nicht frei
               iſt?\pend
           
\pstart
           Ich{ }ſende Dir anbei Dein \label{K_L02892-3v}\edtext{Burgtheater-Referat\pwindex{Schnitzler, Arthur 15.\,5.\,1862 Wien – 21.\,10.\,1931 ebd.@\textsc{Schnitzler, Arthur} (15.\,5.\,1862 Wien – 21.\,10.\,1931 ebd.), \emph{Schriftsteller, Mediziner}!Wiener Burgtheater. (»Agnes Jordan« von Georg Hirschfeld.)@\strich\emph{Wiener Burgtheater. (»Agnes Jordan« von Georg Hirschfeld.)}|pwv}}{\lemma{\textnormal{\emph{Burgtheater-Referat}}}\Cendnote{\textnormal{Beilage nicht erhalten. –rm– [ = Arthur Schnitzler]: \emph{Wiener Burgtheater. (»Agnes Jordan« von Georg
                        Hirschfeld)}\pwindex{Schnitzler, Arthur 15.\,5.\,1862 Wien – 21.\,10.\,1931 ebd.@\textsc{Schnitzler, Arthur} (15.\,5.\,1862 Wien – 21.\,10.\,1931 ebd.), \emph{Schriftsteller, Mediziner}!Wiener Burgtheater. (»Agnes Jordan« von Georg Hirschfeld.)@\strich\emph{Wiener Burgtheater. (»Agnes Jordan« von Georg Hirschfeld.)}|pwk}. In: \emph{Frankfurter
                        Zeitung}\pwindex{Frankfurter Zeitung@\emph{Frankfurter Zeitung}|pwk}, Jg. 44, Nr. 296, 25. 10. 1899, Zweites Morgenblatt, S. 1. Siehe XXXX Auszeichnungsfehler: Dokument L02891 nicht gefunden.}}}\label{K_L02892-3}. Selbſt ich
               habe nicht alle Worte der Handſchrift entziffern können, und mein {\pb}Onkel\pwindex{Mamroth, Fedor 21.\,2.\,1851 Breslau – 25.\,6.\,1907 Frankfurt am Main@\textsc{Mamroth, Fedor} (21.\,2.\,1851 Breslau – 25.\,6.\,1907 Frankfurt am Main), \emph{Journalist, Kritiker}|pwv} hat{ }ſich leider für
               verpflichtet gehalten, zwei Stellen\pwindex{Schnitzler, Arthur 15.\,5.\,1862 Wien – 21.\,10.\,1931 ebd.@\textsc{Schnitzler, Arthur} (15.\,5.\,1862 Wien – 21.\,10.\,1931 ebd.), \emph{Schriftsteller, Mediziner}!Wiener Burgtheater. (»Agnes Jordan« von Georg Hirschfeld.)@\strich\emph{Wiener Burgtheater. (»Agnes Jordan« von Georg Hirschfeld.)}|pwv}, für die er nicht die Verantwortung übernehmen wollte,
               herauszuſtreichen. Ich \strikeout{ko\textcolor{gray}{n}} konnte da nichts hindern. In redaktion\orgindex{Frankfurter Zeitung@Frankfurter Zeitung|pwv}ellen Angelegenheiten iſt mein Onkel\pwindex{Mamroth, Fedor 21.\,2.\,1851 Breslau – 25.\,6.\,1907 Frankfurt am Main@\textsc{Mamroth, Fedor} (21.\,2.\,1851 Breslau – 25.\,6.\,1907 Frankfurt am Main), \emph{Journalist, Kritiker}|pwv} unumſchränkter Gebieter.\pend
           
\pstart
           Gegen \textsc{Wassermann\pwindex{Wassermann, Jakob 10.\,3.\,1873 Fürth – 1.\,1.\,1934 Altaussee@\textsc{Wassermann, Jakob} (10.\,3.\,1873 Fürth – 1.\,1.\,1934 Altaussee), \emph{Schriftsteller}|pw}} iſt die Stimmmung in der Redaktion\orgindex{Frankfurter Zeitung@Frankfurter Zeitung|pwv}{ }\strikeout{u\textcolor{gray}{×}\textcolor{gray}{f}}{ }ſchlechter als je, und ich bin überzeugt, daß er bei der nächſten Gelegenheit
                  \label{K_L02892-4v}\edtext{hinausfliegt}{\lemma{\textnormal{\emph{hinausfliegt}}}\Cendnote{\textnormal{Jakob Wassermann\pwindex{Wassermann, Jakob 10.\,3.\,1873 Fürth – 1.\,1.\,1934 Altaussee@\textsc{Wassermann, Jakob} (10.\,3.\,1873 Fürth – 1.\,1.\,1934 Altaussee), \emph{Schriftsteller}|pwk} verlor seine Stelle als
                     Wien\oindex{Wien@\textbf{Wien}, \emph{Verwaltungsgebiet}|pwk}er Theaterkorrespondent der \emph{Frankfurter Zeitung}\orgindex{Frankfurter Zeitung@Frankfurter Zeitung|pwk} mit dem 1. 1. 1900, vgl. XXXX Auszeichnungsfehler: Dokument L02897 nicht gefunden.}}}\label{K_L02892-4}.\pend
           {\vspace{1\baselineskip}}
\pstart
           Wie Du aus dem \substVorne{}\textsuperscript{beiliegenden}\substDazwischen{}nachfolgenden\substHinten{} kl. \label{K_L02892-5v}\edtext{Referat\pwindex{Kleines Feuilleton. [Kleine Mittheilungen.]@\emph{Kleines Feuilleton. [Kleine Mittheilungen.]}|pwv}}{\lemma{\textnormal{\emph{Referat}}}\Cendnote{\textnormal{[O. V.]: \emph{Kleines Feuilleton. [Kleine
                        Mittheilungen]}\pwindex{Kleines Feuilleton. [Kleine Mittheilungen.]@\emph{Kleines Feuilleton. [Kleine Mittheilungen.]}|pwk}. In: \emph{Frankfurter
                        Zeitung}\pwindex{Frankfurter Zeitung@\emph{Frankfurter Zeitung}|pwk}, Jg. 44, Nr. 297, 26. 10. 1899,
                     Abendblatt, S. 2.}}}\label{K_L02892-5} erſiehſt,{ }ſind Deine \label{K_L02892-6v}\edtext{drei Einakter\pwindex{Schnitzler, Arthur 15.\,5.\,1862 Wien – 21.\,10.\,1931 ebd.@\textsc{Schnitzler, Arthur} (15.\,5.\,1862 Wien – 21.\,10.\,1931 ebd.), \emph{Schriftsteller, Mediziner}!grüne Kakadu – Paracelsus – Die Gefährtin. Drei Einakter@\strich\emph{Der grüne Kakadu – Paracelsus – Die Gefährtin. Drei Einakter}|pwv} am Darmſtädter Hoftheater\oindex{Staatstheater Darmstadt@\textbf{Staatstheater Darmstadt}, \emph{Theater}|pw} geſpielt}{\lemma{\textnormal{\emph{drei … gespielt}}}\Cendnote{\textnormal{Die Einakter\pwindex{Schnitzler, Arthur 15.\,5.\,1862 Wien – 21.\,10.\,1931 ebd.@\textsc{Schnitzler, Arthur} (15.\,5.\,1862 Wien – 21.\,10.\,1931 ebd.), \emph{Schriftsteller, Mediziner}!grüne Kakadu – Paracelsus – Die Gefährtin. Drei Einakter@\strich\emph{Der grüne Kakadu – Paracelsus – Die Gefährtin. Drei Einakter}|pwkv}{ }\emph{Paracelsus}\pwindex{Schnitzler, Arthur 15.\,5.\,1862 Wien – 21.\,10.\,1931 ebd.@\textsc{Schnitzler, Arthur} (15.\,5.\,1862 Wien – 21.\,10.\,1931 ebd.), \emph{Schriftsteller, Mediziner}!Paracelsus. Versspiel in einem Akt@\strich\emph{Paracelsus. Versspiel in einem Akt}|pwk}, \emph{Die Gefährtin}\pwindex{Schnitzler, Arthur 15.\,5.\,1862 Wien – 21.\,10.\,1931 ebd.@\textsc{Schnitzler, Arthur} (15.\,5.\,1862 Wien – 21.\,10.\,1931 ebd.), \emph{Schriftsteller, Mediziner}!Gefährtin. Schauspiel in einem Akt@\strich\emph{Die Gefährtin. Schauspiel in einem Akt}|pwk} und \emph{Der grüne Kakadu}\pwindex{Schnitzler, Arthur 15.\,5.\,1862 Wien – 21.\,10.\,1931 ebd.@\textsc{Schnitzler, Arthur} (15.\,5.\,1862 Wien – 21.\,10.\,1931 ebd.), \emph{Schriftsteller, Mediziner}!grüne Kakadu. Groteske in einem Akt@\strich\emph{Der grüne Kakadu. Groteske in einem Akt}|pwk}
                  wurden am 24. 10. 1899 sowie am 3. 11. 1899 im Darmstädter Hoftheater\oindex{Staatstheater Darmstadt@\textbf{Staatstheater Darmstadt}, \emph{Theater}|pwk} aufgeführt.}}}\label{K_L02892-6} worden.\pend
           
\pstart
           Bitte,{ }ſchreib’ mir bald, wie es Dir geht (Stimmung und Geſundheit).\pend
           
\pstart
           Viele treue Grüße! {\\[\baselineskip]}Dein {\\[\baselineskip]}\spacefill\mbox{Paul Goldmann.}\pend
           \leftskip=0em{}{\vspace{1\baselineskip}}
\pstart
           {\pb}\textcolor{gray}{\textbf{– Man berichtet uns aus \so{Darmſtadt}\oindex{Darmstadt@\textbf{Darmstadt}, \emph{Hauptstadt}|pw} v. 25. ds.: Zu Ehren des Dichter-Komponiſten
                     Peter \so{Cornelius}\pwindex{Cornelius, Peter 24.\,12.\,1824 Mainz – 26.\,10.\,1874 ebd.@\textsc{Cornelius, Peter} (24.\,12.\,1824 Mainz – 26.\,10.\,1874 ebd.), \emph{Komponist}|pw} veranſtaltete am Montag der \so{Richard Wagner-Verein}\orgindex{Richard-Wagner-Verein@Richard-Wagner-Verein|pw} einen Concertabend, an welchem, mit einer Ausnahme, lediglich Kompoſitionen
                  von Cornelius\pwindex{Cornelius, Peter 24.\,12.\,1824 Mainz – 26.\,10.\,1874 ebd.@\textsc{Cornelius, Peter} (24.\,12.\,1824 Mainz – 26.\,10.\,1874 ebd.), \emph{Komponist}|pw} zum Vortrag gelangten. Die
                  Chöre{ }ſtellte der Mozart-Verein\orgindex{Mozartverein Darmstadt@Mozartverein Darmstadt|pw}, als Soliſten
                  traten auf Frl. \so{Zinkeiſen}\pwindex{Zinkeisen, Anna 17.\,12.\,1866 Köln – nach 1930@\textsc{Zinkeisen, Anna} (17.\,12.\,1866 Köln – nach 1930), \emph{Pianistin}|pw} aus Frankfurt a. M.\oindex{Frankfurt am Main@\textbf{Frankfurt am Main}, \emph{Hauptstadt}|pw}, Frau \so{Senff}\pwindex{Senff @\textsc{Senff}, \emph{Musikerin}|pw}–Darmſtadt\oindex{Darmstadt@\textbf{Darmstadt}, \emph{Hauptstadt}|pw} und Herr \so{Joachim}\pwindex{Joachim, Bruno @\textsc{Joachim, Bruno}, \emph{Sänger, Bariton}|pw}–Darmſtadt\oindex{Darmstadt@\textbf{Darmstadt}, \emph{Hauptstadt}|pw}. Das zahlreich erſchienene
                  Publikum dankte{ }ſehr lebhaft für das Gebotene. Im \so{Hoftheater}\oindex{Staatstheater Darmstadt@\textbf{Staatstheater Darmstadt}, \emph{Theater}|pw} kamen geſtern{ }Abend{ }\so{Schnitzler’s}{ }Einakter\pwindex{Schnitzler, Arthur 15.\,5.\,1862 Wien – 21.\,10.\,1931 ebd.@\textsc{Schnitzler, Arthur} (15.\,5.\,1862 Wien – 21.\,10.\,1931 ebd.), \emph{Schriftsteller, Mediziner}!grüne Kakadu – Paracelsus – Die Gefährtin. Drei Einakter@\strich\emph{Der grüne Kakadu – Paracelsus – Die Gefährtin. Drei Einakter}|pwv} »Paracelſus\pwindex{Schnitzler, Arthur 15.\,5.\,1862 Wien – 21.\,10.\,1931 ebd.@\textsc{Schnitzler, Arthur} (15.\,5.\,1862 Wien – 21.\,10.\,1931 ebd.), \emph{Schriftsteller, Mediziner}!Paracelsus. Versspiel in einem Akt@\strich\emph{Paracelsus. Versspiel in einem Akt}|pw}«, »Die
                     Gefährtin\pwindex{Schnitzler, Arthur 15.\,5.\,1862 Wien – 21.\,10.\,1931 ebd.@\textsc{Schnitzler, Arthur} (15.\,5.\,1862 Wien – 21.\,10.\,1931 ebd.), \emph{Schriftsteller, Mediziner}!Gefährtin. Schauspiel in einem Akt@\strich\emph{Die Gefährtin. Schauspiel in einem Akt}|pw}« und »Der grüne Kakadu\pwindex{Schnitzler, Arthur 15.\,5.\,1862 Wien – 21.\,10.\,1931 ebd.@\textsc{Schnitzler, Arthur} (15.\,5.\,1862 Wien – 21.\,10.\,1931 ebd.), \emph{Schriftsteller, Mediziner}!grüne Kakadu. Groteske in einem Akt@\strich\emph{Der grüne Kakadu. Groteske in einem Akt}|pw}« zur
                  erſten Aufführung. Die Aufnahme war eine recht freundliche, wennſchon »Der grüne Kakadu\pwindex{Schnitzler, Arthur 15.\,5.\,1862 Wien – 21.\,10.\,1931 ebd.@\textsc{Schnitzler, Arthur} (15.\,5.\,1862 Wien – 21.\,10.\,1931 ebd.), \emph{Schriftsteller, Mediziner}!grüne Kakadu. Groteske in einem Akt@\strich\emph{Der grüne Kakadu. Groteske in einem Akt}|pw}« einigen Widerſpruch erregte.
                  Geſpielt wurde namentlich von Herrn \so{Hacker}\pwindex{Hacker, Georg Heinrich 16.\,1.\,1856 Mainz – 15.\,8.\,1922 Darmstadt@\textsc{Hacker, Georg Heinrich} (16.\,1.\,1856 Mainz – 15.\,8.\,1922 Darmstadt), \emph{Schauspieler}|pw} (Paracelſus\pwindex{Schnitzler, Arthur 15.\,5.\,1862 Wien – 21.\,10.\,1931 ebd.@\textsc{Schnitzler, Arthur} (15.\,5.\,1862 Wien – 21.\,10.\,1931 ebd.), \emph{Schriftsteller, Mediziner}!Paracelsus. Versspiel in einem Akt@\strich\emph{Paracelsus. Versspiel in einem Akt}|pwv}, Pilgram\pwindex{Schnitzler, Arthur 15.\,5.\,1862 Wien – 21.\,10.\,1931 ebd.@\textsc{Schnitzler, Arthur} (15.\,5.\,1862 Wien – 21.\,10.\,1931 ebd.), \emph{Schriftsteller, Mediziner}!Gefährtin. Schauspiel in einem Akt@\strich\emph{Die Gefährtin. Schauspiel in einem Akt}|pw} und Cardignan\pwindex{Schnitzler, Arthur 15.\,5.\,1862 Wien – 21.\,10.\,1931 ebd.@\textsc{Schnitzler, Arthur} (15.\,5.\,1862 Wien – 21.\,10.\,1931 ebd.), \emph{Schriftsteller, Mediziner}!grüne Kakadu. Groteske in einem Akt@\strich\emph{Der grüne Kakadu. Groteske in einem Akt}|pwv}) und Herrn \so{Löhr}\pwindex{Loehr, Willy 21.\,12.\,1872 Berlin – 27.\,6.\,1940 Karlsbad@\textsc{Loehr, Willy} (21.\,12.\,1872 Berlin – 27.\,6.\,1940 Karlsbad), \emph{Regisseur, Schauspieler, Intendant}|pw} (Hausmann\pwindex{Schnitzler, Arthur 15.\,5.\,1862 Wien – 21.\,10.\,1931 ebd.@\textsc{Schnitzler, Arthur} (15.\,5.\,1862 Wien – 21.\,10.\,1931 ebd.), \emph{Schriftsteller, Mediziner}!Gefährtin. Schauspiel in einem Akt@\strich\emph{Die Gefährtin. Schauspiel in einem Akt}|pwv} und Henri\pwindex{Schnitzler, Arthur 15.\,5.\,1862 Wien – 21.\,10.\,1931 ebd.@\textsc{Schnitzler, Arthur} (15.\,5.\,1862 Wien – 21.\,10.\,1931 ebd.), \emph{Schriftsteller, Mediziner}!grüne Kakadu. Groteske in einem Akt@\strich\emph{Der grüne Kakadu. Groteske in einem Akt}|pwv}) recht gut. Herr \so{Conradi}\pwindex{Conradi, Gustav 15.\,8.\,1850 Altdamm – 13.\,3.\,1926 Wilmersdorf@\textsc{Conradi, Gustav} (15.\,8.\,1850 Altdamm – 13.\,3.\,1926 Wilmersdorf), \emph{Schauspieler}|pw} konnte als Strolch Grain\pwindex{Schnitzler, Arthur 15.\,5.\,1862 Wien – 21.\,10.\,1931 ebd.@\textsc{Schnitzler, Arthur} (15.\,5.\,1862 Wien – 21.\,10.\,1931 ebd.), \emph{Schriftsteller, Mediziner}!grüne Kakadu. Groteske in einem Akt@\strich\emph{Der grüne Kakadu. Groteske in einem Akt}|pwv} einen{ }ſtarken Heiterkeitserfolg verzeichnen. \so{Schiller}\pwindex{Schiller, Friedrich von 10.\,11.\,1759 Marbach am Neckar – 9.\,5.\,1805 Weimar@\textsc{Schiller, Friedrich von} (10.\,11.\,1759 Marbach am Neckar – 9.\,5.\,1805 Weimar), \emph{Schriftsteller, Historiker, Philosoph}|pw}\so{s}{ }Geburtstag wird hier durch Aufführungen der »Wallenſtein«-Trilogie\pwindex{Schiller, Friedrich von 10.\,11.\,1759 Marbach am Neckar – 9.\,5.\,1805 Weimar@\textsc{Schiller, Friedrich von} (10.\,11.\,1759 Marbach am Neckar – 9.\,5.\,1805 Weimar), \emph{Schriftsteller, Historiker, Philosoph}!Wallenstein@\strich\emph{Wallenstein}|pw} und der »Jungfrau von Orleans\pwindex{Schiller, Friedrich von 10.\,11.\,1759 Marbach am Neckar – 9.\,5.\,1805 Weimar@\textsc{Schiller, Friedrich von} (10.\,11.\,1759 Marbach am Neckar – 9.\,5.\,1805 Weimar), \emph{Schriftsteller, Historiker, Philosoph}!Jungfrau von Orleans@\strich\emph{Die Jungfrau von Orleans}|pw}« gefeiert werden. –}}\pend
           \selectlanguage{ngerman}\endnumbering\briefempfaengerindex{Schnitzler, Arthur@\textsc{Schnitzler, Arthur}!zzzGoldmann, Paul@\emph{von Paul Goldmann}!1899-10-261@{26. 10. 1899}|)be}\mylabel{L02892h}  \newcommand{\dateiname}{L02892}\newcommand{\titel}{Paul Goldmann an Arthur Schnitzler, 26. 10. 1899}\newcommand{\editorInnen}{Martin Anton Müller und Laura Untner}%% latex-leseansicht-abspann.tex
%% Abspann für die Leseansicht.
%% Der Schalter \ifkorrekturansicht ist bereits durch den Vorspann gesetzt.

%% latex-abspann.tex
%% Gemeinsamer Abspann für Korrekturansicht und Leseansicht.
%% Setzt den Schalter \ifkorrekturansicht voraus (gesetzt in den
%% einbindenden Dateien latex-korrekturansicht-abspann.tex bzw.
%% latex-leseansicht-abspann.tex).
%% ---------------------------------------------------------------

\normalsize

% Das esempio-Environment wird nur in der Leseansicht benötigt
\ifkorrekturansicht\else
\newenvironment{esempio}[3]%
{
    \vspace{1.5ex}
    \rlap{\underline{#1}}
    \par
    \setlength{\parindent}{0cm}
    \nopagebreak
    \leftskip=#2cm
    \rightskip=#3cm
}
{
    \par
}
\fi

\doendnotes{C}
\bigskip
\vfill

\clearpage

\footnotesize

\ifkorrekturansicht
  \lohead{\textsc{register}}
\fi

% theindex-Environment neu definieren ohne reledmac
\makeatletter
\renewenvironment{theindex}{%
  \ifkorrekturansicht
    \section*{\indexname}%
  \else
    \subsubsection*{Index der erwähnten Entitäten}%
  \fi
  \setlength{\parindent}{0pt}%
  \setlength{\parskip}{0pt plus 0.3pt}%
  \let\item\@idxitem
}{%
  \ifkorrekturansicht\clearpage\fi
}
\makeatother

\IfFileExists{\jobname-pw.ind}{\input{\jobname-pw.ind}}{}

% Quellenangabe nur in der Leseansicht
\ifkorrekturansicht\else
% Fallback-Definitionen, falls die .tex-Datei \titel etc. nicht gesetzt hat
\providecommand{\titel}{}
\providecommand{\editorInnen}{}
\providecommand{\dateiname}{\jobname}

\vspace{3cm}

\vfill

\footnotesize
\textsc{Quelle}: \titel. Herausgegeben von {\editorInnen}. In: \emph{Arthur Schnitzler: Briefwechsel mit Autorinnen und Autoren}.
 Digitale Edition, https://schnitzler-briefe.acdh.oeaw.ac.at/{\dateiname}.html (Stand \today)
\fi

\end{document}


