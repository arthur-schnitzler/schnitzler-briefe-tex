%% latex-leseansicht-vorspann.tex
%% Vorspann für die Leseansicht.
%% Lädt die gemeinsame Datei latex-vorspann.tex mit nicht gesetztem Schalter.

\newif\ifkorrekturansicht
\korrekturansichtfalse

\input{../tex-inputs/latex-vorspann}


         
         \renewcommand{\erwaehntePersonen}{Personen: Eduard Bacher, Moriz Benedikt, Gustav Conradi, Peter Cornelius, Alfred Dreyfus, Berthold Frischauer, Georg Heinrich Hacker, Theodor Herzl, Bruno Joachim, Willy Loehr, Fedor Mamroth, Friedrich von Schiller,  Senff, Jakob Wassermann, Anna Zinkeisen}
         \renewcommand{\erwaehnteInstitutionen}{Institutionen: Frankfurter Zeitung, Französische Streitkräfte, Mozartverein Darmstadt, Neue Freie Presse, Richard-Wagner-Verein}
         \renewcommand{\erwaehnteOrte}{Orte: Berlin, Darmstadt, Frankfurt am Main, Frankreich, Paris, Staatstheater Darmstadt, Wien}
         \renewcommand{\erwaehnteWerke}{Werke: Der grüne Kakadu – Paracelsus – Die Gefährtin. Drei Einakter, Der grüne Kakadu. Groteske in einem Akt, Die Gefährtin. Schauspiel in einem Akt, Die Jungfrau von Orleans, Frankfurter Zeitung, Kleines Feuilleton. [Kleine Mittheilungen.], Paracelsus. Versspiel in einem Akt, Wallenstein, Wiener Burgtheater. (»Agnes Jordan« von Georg Hirschfeld.)}
               \section[ Paul Goldmann an Arthur Schnitzler, 26. 10. 1899]{ Paul Goldmann an Arthur Schnitzler, 26. 10. 1899}\nopagebreak\mylabel{v}\rehead{ }\begin{ledgroupsized}[t]{13cm}\normalsize\beginnumbering \toendnotes[C]{\smallbreak\pagebreak[2]} \Standort{DLA, A:Schnitzler, HS.NZ85.1.3169.}
\physDesc{Brief, 1 Blatt, 4 Seiten
\newline{}Handschrift: schwarze Tinte, deutsche Kurrent\newline{}Beilage: ein beschnittener Zeitungsausschnitt auf der letzten
                                 Seite 
\newline{}Schnitzler: mit rotem Buntstift eine Unterstreichung }\toendnotes[C]{\smallbreak}\pstart
           \noindent{}{\pb}\textcolor{gray}{\textbf{\textbf{Frankfurter Zeitung}}}\orgindex{Frankfurter Zeitung@Frankfurter Zeitung|pw}\hfill \textcolor{gray}{\textbf{\textbf{Frankfurt a.
                        M.\oindex{Frankfurt am Main@\textbf{Frankfurt am Main}|pw},}}}{ }26. Oktober \textcolor{gray}{\textbf{189}}9.\pend
           \pstart
           \textcolor{gray}{\textbf{und}}\pend
           \pstart
           \textcolor{gray}{\textbf{Handelsblatt.}}\pend
           \pstart
           \textcolor{gray}{\textbf{\textbf{Redaktion\orgindex{Frankfurter Zeitung@Frankfurter Zeitung|pwv}.}\footnote{\noindent{}\textcolor{gray}{\textbf{Für die Redaktion\orgindex{Frankfurter Zeitung@Frankfurter Zeitung|pwv} beſtimmte Briefe und Sendungen wolle man
                                 \so{nicht} an die Perſon eines Redakteurs,
                              ſondern ſtets \textbf{an die Redaktion der Frankfurter Zeitung\orgindex{Frankfurter Zeitung@Frankfurter Zeitung|pw}} adreſſiren. }}}}}\pend
           \pstart
           \textcolor{gray}{\textbf{Telegramm-Adreſſe:}}\pend
           \pstart
           \textcolor{gray}{\textbf{\textbf{Zeitung\orgindex{Frankfurter Zeitung@Frankfurter Zeitung|pwv}{ }Frankfurt Main\oindex{Frankfurt am Main@\textbf{Frankfurt am Main}|pw}.}}}\pend
           \pstart\center{}Mein lieber Freund,\pend\pstart
           Mit der »Neuen Freien Preſſe\orgindex{Neue Freie Presse@Neue Freie Presse|pw}« iſt es alſo \label{K_L02892-1v}\edtext{auch diesmal}{\lemma{\textnormal{\emph{auch diesmal}}}\Cendnote{\textnormal{Erst ab 1900 war Goldmann\pwindex{Goldmann, Paul 31.01.1865 – 25.09.1935@\textsc{Goldmann, Paul} (31.01.1865 – 25.09.1935), \emph{Schriftsteller, Journalist}|pwk} Theaterkorrespondent der \emph{Neuen Freien Presse}\orgindex{Neue Freie Presse@Neue Freie Presse|pwk} in Berlin\oindex{Berlin@\textbf{Berlin}|pwk} (siehe Paul Goldmann an Arthur Schnitzler, 5. 3. [1899] und
                     Paul Goldmann an Arthur Schnitzler, 4. 12. [1899]).}}}\label{K_L02892-1h} nichts. Nachdem die
                  Herausgeber\pwindex{Bacher, Eduard 07.03.1846 – 16.01.1908@\textsc{Bacher, Eduard} (07.03.1846 – 16.01.1908), \emph{Journalist, Herausgeber}|pwv}\pwindex{Benedikt, Moriz 27.05.1849 – 18.03.1920@\textsc{Benedikt, Moriz} (27.05.1849 – 18.03.1920), \emph{Journalist, Herausgeber}|pwv}
               mich ſo furchtbar gedrängt, telegraphirte ich ſofort nach meinem Eintreffen in Frankfurt\oindex{Frankfurt am Main@\textbf{Frankfurt am Main}|pw}, ich ſei bereit, am 1. Jänner in Berlin\oindex{Berlin@\textbf{Berlin}|pw}
               anzutreten. Zugleich ſetzte ich brieflich meine materiellen Bedingungen auseinander.
                  Geſtern erhielt ich nun ein Telegramm der Herausgeber\pwindex{Bacher, Eduard 07.03.1846 – 16.01.1908@\textsc{Bacher, Eduard} (07.03.1846 – 16.01.1908), \emph{Journalist, Herausgeber}|pwv}\pwindex{Benedikt, Moriz 27.05.1849 – 18.03.1920@\textsc{Benedikt, Moriz} (27.05.1849 – 18.03.1920), \emph{Journalist, Herausgeber}|pwv} der N. Fr. Pr.\orgindex{Neue Freie Presse@Neue Freie Presse|pw}, worin ſie mir mittheilten, daß ſie
               meine materiellen Bedingungen wohl acceptiren würden, daß aber die Nachrichten
               inbezug auf \label{K_L02892-5v}\edtext{\textsc{Frischauer\pwindex{Frischauer, Berthold 1851-09-09 – 1924-02-04@\textsc{Frischauer, Berthold} (1851-09-09 – 1924-02-04), \emph{Journalist}|pw}s} Rückkehr nach \textsc{Paris\oindex{Paris@\textbf{Paris}|pw}}}{\lemma{\textnormal{\emph{Frischauers … Paris}}}\Cendnote{\textnormal{Berthold Frischauer\pwindex{Frischauer, Berthold 1851-09-09 – 1924-02-04@\textsc{Frischauer, Berthold} (1851-09-09 – 1924-02-04), \emph{Journalist}|pwk} war seit 1895 der Nachfolger Theodor Herzl\pwindex{Herzl, Theodor 1860-05-02 – 1904-07-03@\textsc{Herzl, Theodor} (1860-05-02 – 1904-07-03), \emph{Schriftsteller, Journalist}|pwk}s als Korrespondent der \emph{Neuen
                     Freien Presse}\orgindex{Neue Freie Presse@Neue Freie Presse|pwk} in Paris\oindex{Paris@\textbf{Paris}|pwk}. Am 16. 2. 1899 war er wegen Ehrenbeleidigung der \emph{französischen Armee}\orgindex{Franzoesische Streitkraefte@Französische Streitkräfte|pwk} im Rahmen seiner
                  Berichterstattung zur Dreyfus\pwindex{Dreyfus, Alfred 1859-10-09 – 1935-07-12@\textsc{Dreyfus, Alfred} (1859-10-09 – 1935-07-12), \emph{Militär}|pwk}-Affäre aus Frankreich\oindex{Frankreich@\textbf{Frankreich}|pwk} ausgewiesen worden. Anfang Dezember 1899 wurde ihm die Einreise wieder gestattet und
                  er kehrte zurück. In der Zwischenzeit dürfte er in Berlin\oindex{Berlin@\textbf{Berlin}|pwk} eingesetzt gewesen sein.}}}\label{K_L02892-5h} jetzt wieder ſehr ungünſtig
               lauteten. Zugleich wurde mir vorgeſchlagen, für die N.
                  Fr. Pr.\orgindex{Neue Freie Presse@Neue Freie Presse|pw}{ }{\pb}nach \textsc{Paris\oindex{Paris@\textbf{Paris}|pw}} zu gehen. Dieſen Vorſchlag habe ich ſelbſtverſtändlich abgelehnt, und ſo
               bleibt’s beim Alten. Glücklicher Weiſe \strikeout{war} bin ich
               vorſichtig genug geweſen, hier\orgindex{Frankfurter Zeitung@Frankfurter Zeitung|pwuv} meine Beziehungen noch nicht abzubrechen. Sonſt wäre ich
               jetzt ohne Stellung. Hoffentlich erfährt man auch in Frankfurt\oindex{Frankfurt am Main@\textbf{Frankfurt am Main}|pw} nichts von den geführten Verhandlungen, und ich bitte Dich, die
               ganze Angelegenheit \strikeout{\textcolor{gray}{d}} diskret zu
               behandeln. Aber was ſagſt Du zu dieſen Zeitungs-\textsc{Paschahs}\pwindex{Bacher, Eduard 07.03.1846 – 16.01.1908@\textsc{Bacher, Eduard} (07.03.1846 – 16.01.1908), \emph{Journalist, Herausgeber}|pwv}\pwindex{Benedikt, Moriz 27.05.1849 – 18.03.1920@\textsc{Benedikt, Moriz} (27.05.1849 – 18.03.1920), \emph{Journalist, Herausgeber}|pwv}, die
               Einen über Hals und Kopf \strikeout{für} in eine Stellung
               hineinhetzen und erſt hinterher merken, daß die Stellung noch gar nicht frei iſt?\pend
           \pstart
           Ich ſende Dir anbei Dein \label{K_L02892-9v}\edtext{Burgtheater-Referat\pwindex{Wiener Burgtheater. (»Agnes Jordan« von Georg Hirschfeld.)1899-10-25@\emph{Wiener Burgtheater. (»Agnes Jordan« von Georg Hirschfeld.)} {[}1899-10-25{]}|pwv}}{\lemma{\textnormal{\emph{Burgtheater-Referat}}}\Cendnote{\textnormal{Beilage nicht erhalten. –rm–\pwindex{Schnitzler, Arthur 15.05.1862 – 21.10.1931@\textsc{Schnitzler, Arthur} (15.05.1862 – 21.10.1931), \emph{Schriftsteller, Mediziner}|pwkv} [=Arthur Schnitzler\pwindex{Schnitzler, Arthur 15.05.1862 – 21.10.1931@\textsc{Schnitzler, Arthur} (15.05.1862 – 21.10.1931), \emph{Schriftsteller, Mediziner}|pwk}]: \emph{Wiener Burgtheater. (»Agnes Jordan« von Georg
                        Hirschfeld.)}\pwindex{Wiener Burgtheater. (»Agnes Jordan« von Georg Hirschfeld.)1899-10-25@\emph{Wiener Burgtheater. (»Agnes Jordan« von Georg Hirschfeld.)} {[}1899-10-25{]}|pwk}. In: \emph{Frankfurter
                        Zeitung}\pwindex{?? Werk@Nicht ermittelte Verfasserinnen und Verfasser!Frankfurter Zeitung1856 – 1943@\emph{Frankfurter Zeitung} {[}1856 – 1943{]}|pwk}, Jg. 44, Nr. 296, 25. 10. 1899, Zweites Morgenblatt, S. 1. Siehe Paul Goldmann an Arthur Schnitzler, 23. 10. [1899].}}}\label{K_L02892-9h}. Selbſt ich
               habe nicht alle Worte der Handſchrift entziffern können, und mein {\pb}Onkel\pwindex{Mamroth, Fedor 21.02.1851 – 25.06.1907@\textsc{Mamroth, Fedor} (21.02.1851 – 25.06.1907), \emph{Journalist, Kritiker}|pwv} hat ſich leider für
               verpflichtet gehalten, zwei Stellen\pwindex{Wiener Burgtheater. (»Agnes Jordan« von Georg Hirschfeld.)1899-10-25@\emph{Wiener Burgtheater. (»Agnes Jordan« von Georg Hirschfeld.)} {[}1899-10-25{]}|pwv}, für die er nicht die Verantwortung übernehmen wollte,
               herauszuſtreichen. Ich \strikeout{ko\textcolor{gray}{n}} konnte da nichts hindern. In redaktion\orgindex{Frankfurter Zeitung@Frankfurter Zeitung|pwv}ellen Angelegenheiten iſt mein Onkel\pwindex{Mamroth, Fedor 21.02.1851 – 25.06.1907@\textsc{Mamroth, Fedor} (21.02.1851 – 25.06.1907), \emph{Journalist, Kritiker}|pwv} unumſchränkter Gebieter.\pend
           \pstart
           Gegen \textsc{Wassermann\pwindex{Wassermann, Jakob 10.03.1873 – 01.01.1934@\textsc{Wassermann, Jakob} (10.03.1873 – 01.01.1934), \emph{Schriftsteller}|pw}} iſt die Stimmmung in der Redaktion\orgindex{Frankfurter Zeitung@Frankfurter Zeitung|pwv}{ }\strikeout{u\textcolor{gray}{×}\textcolor{gray}{f}} ſchlechter als je, und ich bin überzeugt, daß er
               bei der nächſten Gelegenheit \label{K_L02892-11v}\edtext{hinausfliegt}{\lemma{\textnormal{\emph{hinausfliegt}}}\Cendnote{\textnormal{Jakob Wassermann\pwindex{Wassermann, Jakob 10.03.1873 – 01.01.1934@\textsc{Wassermann, Jakob} (10.03.1873 – 01.01.1934), \emph{Schriftsteller}|pwk} verlor seine Stelle als
                     Wien\oindex{Wien@\textbf{Wien}|pwk}er Theaterkorrespondent der \emph{Frankfurter Zeitung}\orgindex{Frankfurter Zeitung@Frankfurter Zeitung|pwk} mit dem 1. 1. 1900, vgl. Paul Goldmann an Arthur Schnitzler, 6. 12. [1899].}}}\label{K_L02892-11h}.\pend
           {\bigskip}\pstart
           \noindent{}Wie Du aus dem \substVorne{}\textsuperscript{beiliegenden}{\allowbreak}\substDazwischen{}nachfolgenden\substHinten{} kl. \label{K_L02892-87v}\edtext{Referat\pwindex{?? Werk@Nicht ermittelte Verfasserinnen und Verfasser!Kleines Feuilleton. [Kleine Mittheilungen.]1899-10-26@\emph{Kleines Feuilleton. [Kleine Mittheilungen.]} {[}1899-10-26{]}|pwv}}{\lemma{\textnormal{\emph{Referat}}}\Cendnote{\textnormal{[O. V.]: \emph{Kleines Feuilleton. [Kleine
                        Mittheilungen.]}\pwindex{?? Werk@Nicht ermittelte Verfasserinnen und Verfasser!Kleines Feuilleton. [Kleine Mittheilungen.]1899-10-26@\emph{Kleines Feuilleton. [Kleine Mittheilungen.]} {[}1899-10-26{]}|pwk}. In: \emph{Frankfurter
                        Zeitung}\pwindex{?? Werk@Nicht ermittelte Verfasserinnen und Verfasser!Frankfurter Zeitung1856 – 1943@\emph{Frankfurter Zeitung} {[}1856 – 1943{]}|pwk}, Jg. 44, Nr. 297, 26. 10. 1899,
                     Abendblatt, S. 2.}}}\label{K_L02892-87h} erſiehſt, ſind Deine \label{K_L02892-12v}\edtext{drei Einakter\pwindex{Schnitzler, Arthur 15.05.1862 – 21.10.1931@\textsc{Schnitzler, Arthur} (15.05.1862 – 21.10.1931), \emph{Schriftsteller, Mediziner}!gruene Kakadu – Paracelsus – Die Gefaehrtin. Drei Einakter1898 – 1899@\strich\emph{Der grüne Kakadu – Paracelsus – Die Gefährtin. Drei Einakter} {[}1898 – 1899{]}|pwv} am Darmſtädter Hoftheater\oindex{Staatstheater Darmstadt@\textbf{Staatstheater Darmstadt}|pw} geſpielt}{\lemma{\textnormal{\emph{drei … geſpielt}}}\Cendnote{\textnormal{Die Einakter\pwindex{Schnitzler, Arthur 15.05.1862 – 21.10.1931@\textsc{Schnitzler, Arthur} (15.05.1862 – 21.10.1931), \emph{Schriftsteller, Mediziner}!gruene Kakadu – Paracelsus – Die Gefaehrtin. Drei Einakter1898 – 1899@\strich\emph{Der grüne Kakadu – Paracelsus – Die Gefährtin. Drei Einakter} {[}1898 – 1899{]}|pwkv}{ }\emph{Paracelsus}\pwindex{Schnitzler, Arthur 15.05.1862 – 21.10.1931@\textsc{Schnitzler, Arthur} (15.05.1862 – 21.10.1931), \emph{Schriftsteller, Mediziner}!Paracelsus. Versspiel in einem Akt01. 11. 1898@\strich\emph{Paracelsus. Versspiel in einem Akt} {[}01. 11. 1898{]}|pwk}, \emph{Die Gefährtin}\pwindex{Schnitzler, Arthur 15.05.1862 – 21.10.1931@\textsc{Schnitzler, Arthur} (15.05.1862 – 21.10.1931), \emph{Schriftsteller, Mediziner}!Gefaehrtin. Schauspiel in einem Akt1899-03-01@\strich\emph{Die Gefährtin. Schauspiel in einem Akt} {[}1899-03-01{]}|pwk} und \emph{Der grüne Kakadu}\pwindex{Schnitzler, Arthur 15.05.1862 – 21.10.1931@\textsc{Schnitzler, Arthur} (15.05.1862 – 21.10.1931), \emph{Schriftsteller, Mediziner}!gruene Kakadu. Groteske in einem Akt1. 3. 1899@\strich\emph{Der grüne Kakadu. Groteske in einem Akt} {[}1. 3. 1899{]}|pwk}
                  wurden am 24. 10. 1899 sowie am 3. 11. 1899 im Darmstädter Hoftheater\oindex{Staatstheater Darmstadt@\textbf{Staatstheater Darmstadt}|pwk} aufgeführt.}}}\label{K_L02892-12h} worden.\pend
           \pstart
           Bitte, ſchreib’ mir bald, wie es Dir geht (Stimmung und Geſundheit).\pend
           \pstart
           Viele treue Grüße! {\\[\baselineskip]}Dein {\\[\baselineskip]}\spacefill\mbox{Paul Goldmann.}\pend
           \leftskip=0em{}{\bigskip}\pstart
           \noindent{}{\pb}\textcolor{gray}{\textbf{– Man berichtet uns aus \so{Darmſtadt}\oindex{Darmstadt@\textbf{Darmstadt}|pw} v. 25. ds.: Zu Ehren des Dichter-Komponiſten
                     Peter \so{Cornelius}\pwindex{Cornelius, Peter 24.12.1824 – 26.10.1874@\textsc{Cornelius, Peter} (24.12.1824 – 26.10.1874), \emph{Komponist}|pw} veranſtaltete am Montag der \so{Richard Wagner-Verein}\orgindex{Richard-Wagner-Verein@Richard-Wagner-Verein|pw} einen Concertabend, an welchem, mit einer Ausnahme, lediglich Kompoſitionen
                  von Cornelius\pwindex{Cornelius, Peter 24.12.1824 – 26.10.1874@\textsc{Cornelius, Peter} (24.12.1824 – 26.10.1874), \emph{Komponist}|pw} zum Vortrag gelangten. Die
                  Chöre ſtellte der Mozart-Verein\orgindex{Mozartverein Darmstadt@Mozartverein Darmstadt|pw}, als Soliſten
                  traten auf Frl. \so{Zinkeiſen}\pwindex{Zinkeisen, Anna 1866-12-17 – nach 1930@\textsc{Zinkeisen, Anna} (1866-12-17 – nach 1930), \emph{Pianistin}|pw} aus Frankfurt a. M.\oindex{Frankfurt am Main@\textbf{Frankfurt am Main}|pw}, Frau \so{Senff}\pwindex{Senff @\textsc{Senff}, \emph{Musikerin}|pw}—Darmſtadt\oindex{Darmstadt@\textbf{Darmstadt}|pw} und Herr \so{Joachim}\pwindex{Joachim, Bruno @\textsc{Joachim, Bruno}, \emph{Sänger, Bariton <Sänger>}|pw}—Darmſtadt\oindex{Darmstadt@\textbf{Darmstadt}|pw}. Das zahlreich erſchienene
                  Publikum dankte ſehr lebhaft für das Gebotene. Im \so{Hoftheater}\oindex{Staatstheater Darmstadt@\textbf{Staatstheater Darmstadt}|pw} kamen geſtern{ }Abend{ }\so{Schnitzler’s}{ }Einakter\pwindex{Schnitzler, Arthur 15.05.1862 – 21.10.1931@\textsc{Schnitzler, Arthur} (15.05.1862 – 21.10.1931), \emph{Schriftsteller, Mediziner}!gruene Kakadu – Paracelsus – Die Gefaehrtin. Drei Einakter1898 – 1899@\strich\emph{Der grüne Kakadu – Paracelsus – Die Gefährtin. Drei Einakter} {[}1898 – 1899{]}|pwv} »Paracelſus\pwindex{Schnitzler, Arthur 15.05.1862 – 21.10.1931@\textsc{Schnitzler, Arthur} (15.05.1862 – 21.10.1931), \emph{Schriftsteller, Mediziner}!Paracelsus. Versspiel in einem Akt01. 11. 1898@\strich\emph{Paracelsus. Versspiel in einem Akt} {[}01. 11. 1898{]}|pw}«, »Die
                     Gefährtin\pwindex{Schnitzler, Arthur 15.05.1862 – 21.10.1931@\textsc{Schnitzler, Arthur} (15.05.1862 – 21.10.1931), \emph{Schriftsteller, Mediziner}!Gefaehrtin. Schauspiel in einem Akt1899-03-01@\strich\emph{Die Gefährtin. Schauspiel in einem Akt} {[}1899-03-01{]}|pw}« und »Der grüne Kakadu\pwindex{Schnitzler, Arthur 15.05.1862 – 21.10.1931@\textsc{Schnitzler, Arthur} (15.05.1862 – 21.10.1931), \emph{Schriftsteller, Mediziner}!gruene Kakadu. Groteske in einem Akt1. 3. 1899@\strich\emph{Der grüne Kakadu. Groteske in einem Akt} {[}1. 3. 1899{]}|pw}« zur
                  erſten Aufführung. Die Aufnahme war eine recht freundliche, wennſchon »Der grüne Kakadu\pwindex{Schnitzler, Arthur 15.05.1862 – 21.10.1931@\textsc{Schnitzler, Arthur} (15.05.1862 – 21.10.1931), \emph{Schriftsteller, Mediziner}!gruene Kakadu. Groteske in einem Akt1. 3. 1899@\strich\emph{Der grüne Kakadu. Groteske in einem Akt} {[}1. 3. 1899{]}|pw}« einigen Widerſpruch erregte.
                  Geſpielt wurde namentlich von Herrn \so{Hacker}\pwindex{Hacker, Georg Heinrich 1856-01-16 – 1922-08-15@\textsc{Hacker, Georg Heinrich} (1856-01-16 – 1922-08-15), \emph{Schauspieler}|pw} (Paracelſus\pwindex{Schnitzler, Arthur 15.05.1862 – 21.10.1931@\textsc{Schnitzler, Arthur} (15.05.1862 – 21.10.1931), \emph{Schriftsteller, Mediziner}!Paracelsus. Versspiel in einem Akt01. 11. 1898@\strich\emph{Paracelsus. Versspiel in einem Akt} {[}01. 11. 1898{]}|pwv}, Pilgram\pwindex{Schnitzler, Arthur 15.05.1862 – 21.10.1931@\textsc{Schnitzler, Arthur} (15.05.1862 – 21.10.1931), \emph{Schriftsteller, Mediziner}!Gefaehrtin. Schauspiel in einem Akt1899-03-01@\strich\emph{Die Gefährtin. Schauspiel in einem Akt} {[}1899-03-01{]}|pw} und Cardignan\pwindex{Schnitzler, Arthur 15.05.1862 – 21.10.1931@\textsc{Schnitzler, Arthur} (15.05.1862 – 21.10.1931), \emph{Schriftsteller, Mediziner}!gruene Kakadu. Groteske in einem Akt1. 3. 1899@\strich\emph{Der grüne Kakadu. Groteske in einem Akt} {[}1. 3. 1899{]}|pwv}) und Herrn \so{Löhr}\pwindex{Loehr, Willy 1872-12-21 – 1940-06-27@\textsc{Loehr, Willy} (1872-12-21 – 1940-06-27), \emph{Regisseur/Regisseurin, Schauspieler/Schauspielerin, Intendant/Intendantin}|pw} (Hausmann\pwindex{Schnitzler, Arthur 15.05.1862 – 21.10.1931@\textsc{Schnitzler, Arthur} (15.05.1862 – 21.10.1931), \emph{Schriftsteller, Mediziner}!Gefaehrtin. Schauspiel in einem Akt1899-03-01@\strich\emph{Die Gefährtin. Schauspiel in einem Akt} {[}1899-03-01{]}|pwv} und Henri\pwindex{Schnitzler, Arthur 15.05.1862 – 21.10.1931@\textsc{Schnitzler, Arthur} (15.05.1862 – 21.10.1931), \emph{Schriftsteller, Mediziner}!gruene Kakadu. Groteske in einem Akt1. 3. 1899@\strich\emph{Der grüne Kakadu. Groteske in einem Akt} {[}1. 3. 1899{]}|pwv}) recht gut. Herr \so{Conradi}\pwindex{Conradi, Gustav 1850-08-15 – 1926-03-13@\textsc{Conradi, Gustav} (1850-08-15 – 1926-03-13), \emph{Schauspieler}|pw} konnte als Strolch Grain\pwindex{Schnitzler, Arthur 15.05.1862 – 21.10.1931@\textsc{Schnitzler, Arthur} (15.05.1862 – 21.10.1931), \emph{Schriftsteller, Mediziner}!gruene Kakadu. Groteske in einem Akt1. 3. 1899@\strich\emph{Der grüne Kakadu. Groteske in einem Akt} {[}1. 3. 1899{]}|pwv} einen ſtarken Heiterkeitserfolg verzeichnen. \so{Schiller}\pwindex{Schiller, Friedrich von 10.11.1759 – 09.05.1805@\textsc{Schiller, Friedrich von} (10.11.1759 – 09.05.1805), \emph{Schriftsteller, Historiker, Philosoph}|pw}\so{s}{ }Geburtstag wird hier durch Aufführungen der »Wallenſtein«-Trilogie\pwindex{Schiller, Friedrich von 10.11.1759 – 09.05.1805@\textsc{Schiller, Friedrich von} (10.11.1759 – 09.05.1805), \emph{Schriftsteller, Historiker, Philosoph}!Wallenstein1796 – 1800@\strich\emph{Wallenstein} {[}1796 – 1800{]}|pw} und der »Jungfrau von Orleans\pwindex{Schiller, Friedrich von 10.11.1759 – 09.05.1805@\textsc{Schiller, Friedrich von} (10.11.1759 – 09.05.1805), \emph{Schriftsteller, Historiker, Philosoph}!Jungfrau von Orleans1801@\strich\emph{Die Jungfrau von Orleans} {[}1801{]}|pw}« gefeiert werden. –}}\pend
           
         
         \endnumbering\mylabel{h}\end{ledgroupsized}  \newcommand{\dateiname}{L02892}\newcommand{\titel}{Paul Goldmann an Arthur Schnitzler, 26. 10. 1899}\newcommand{\editorInnen}{Martin Anton Müller und Laura Untner}%% latex-leseansicht-abspann.tex
%% Abspann für die Leseansicht.
%% Der Schalter \ifkorrekturansicht ist bereits durch den Vorspann gesetzt.

%% latex-abspann.tex
%% Gemeinsamer Abspann für Korrekturansicht und Leseansicht.
%% Setzt den Schalter \ifkorrekturansicht voraus (gesetzt in den
%% einbindenden Dateien latex-korrekturansicht-abspann.tex bzw.
%% latex-leseansicht-abspann.tex).
%% ---------------------------------------------------------------

\normalsize

% Das esempio-Environment wird nur in der Leseansicht benötigt
\ifkorrekturansicht\else
\newenvironment{esempio}[3]%
{
    \vspace{1.5ex}
    \rlap{\underline{#1}}
    \par
    \setlength{\parindent}{0cm}
    \nopagebreak
    \leftskip=#2cm
    \rightskip=#3cm
}
{
    \par
}
\fi

\doendnotes{C}
\bigskip
\vfill

\clearpage

\footnotesize

\ifkorrekturansicht
  \lohead{\textsc{register}}
\fi

% theindex-Environment neu definieren ohne reledmac
\makeatletter
\renewenvironment{theindex}{%
  \ifkorrekturansicht
    \section*{\indexname}%
  \else
    \subsubsection*{Index der erwähnten Entitäten}%
  \fi
  \setlength{\parindent}{0pt}%
  \setlength{\parskip}{0pt plus 0.3pt}%
  \let\item\@idxitem
}{%
  \ifkorrekturansicht\clearpage\fi
}
\makeatother

\IfFileExists{\jobname-pw.ind}{\input{\jobname-pw.ind}}{}

% Quellenangabe nur in der Leseansicht
\ifkorrekturansicht\else
% Fallback-Definitionen, falls die .tex-Datei \titel etc. nicht gesetzt hat
\providecommand{\titel}{}
\providecommand{\editorInnen}{}
\providecommand{\dateiname}{\jobname}

\vspace{3cm}

\vfill

\footnotesize
\textsc{Quelle}: \titel. Herausgegeben von {\editorInnen}. In: \emph{Arthur Schnitzler: Briefwechsel mit Autorinnen und Autoren}.
 Digitale Edition, https://schnitzler-briefe.acdh.oeaw.ac.at/{\dateiname}.html (Stand \today)
\fi

\end{document}


      