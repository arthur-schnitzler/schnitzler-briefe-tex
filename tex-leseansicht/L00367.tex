%% latex-korrekturansicht-vorspann.tex
%% Vorspann für die Korrekturansicht.
%% Lädt die gemeinsame Datei latex-vorspann.tex mit gesetztem Schalter.

\newif\ifkorrekturansicht
\korrekturansichttrue

\input{../tex-inputs/latex-vorspann}


\section[Richard Beer-Hofmann an Arthur Schnitzler, 7. 9. 1894]{L00367 Richard Beer-Hofmann an Arthur Schnitzler, 7. 9. 1894}
\nopagebreak\mylabel{L00367v}
\rehead{ }\normalsize\beginnumbering\briefempfaengerindex{Schnitzler, Arthur@\textsc{Schnitzler, Arthur}!zzzBeer-Hofmann, Richard@\emph{von Richard Beer-Hofmann}!1894-09-071@{7. 9. 1894}|(be}
\toendnotes[C]{\smallbreak\pagebreak[2]}\Standort{CUL, Schnitzler, B 8.}
\physDesc{Brief, 1 Blatt, 2 Seiten, 963 Zeichen
\newline{}Handschrift: Bleistift, lateinische Kurrent
\newline{}Schnitzler: mit Bleistift nummeriert: »32« }
\buchAbdrucke{\weitereDrucke{1) Arthur Schnitzler, Richard Beer-Hofmann: \emph{Briefwechsel 1891–1931}. Wien, Zürich: \emph{Europaverlag} 1992, S. 58–59.} \weitereDrucke{2) Hermann Bahr, Arthur Schnitzler: \emph{Briefwechsel, Aufzeichnungen, Dokumente (1891–1931)}. Göttingen: \emph{Wallstein} 2018.} }\toendnotes[C]{\smallbreak}
\pstart
           \noindent{}{\pb}Lieber Arthur! Ich
               habe eine Menge Bitten an Sie.\pend
           
\pstart
           I. Senden Sie mir unter Kreuzband den Bolgar\pwindex{Regeln des Duells@\emph{Die Regeln des Duells}|pwv}\pwindex{Bolgár, Franz von 03.01.1851 – 23.05.1923@\textsc{Bolgár, Franz von} (03.01.1851 – 23.05.1923), \emph{Politiker/Politikerin, Publizist/Publizistin}|pw}, ich nehme ihn auf die Reise mit.\pend
           
\pstart
           II. Fragen Sie telefonisch bei Paul Horn\pwindex{Horn, Paul 13.02.1867 – 18.01.1936@\textsc{Horn, Paul} (13.02.1867 – 18.01.1936), \emph{Fabrikant/Fabrikantin}|pw} an ob
               es geht daß ich \strikeout{Dinge an} falls ich zollpflichtige
               Sachen \strikeout{an} von Italien\oindex{Italien@\textbf{Italien}, \emph{A.PCLI}|pw} herübersenden sollte ich sie adressiren kann an Herrn \uline{Paul Horn\pwindex{Horn, Paul 13.02.1867 – 18.01.1936@\textsc{Horn, Paul} (13.02.1867 – 18.01.1936), \emph{Fabrikant/Fabrikantin}|pw}} p. Adr. \uline{Schenker u.} Co\orgindex{Schenker und Co.@Schenker {\kaufmannsund}  Co.|pw} und ob dann Schenkers\orgindex{Schenker und Co.@Schenker {\kaufmannsund}  Co.|pw} die Verzollung\introOben{}sarbeiten\introOben{} etc. \strikeout{er} übernehmen. Weil ich
               nicht wegen meines Papa\pwindex{Beer, Hermann 10.8.1835 – 03.10.1902@\textsc{Beer, Hermann} (10.8.1835 – 03.10.1902), \emph{Rechtsanwalt/Rechtsanwältin}|pwv}’s
               die Sachen (Moritz gehste herunter vom Bock) an mich adressiren kann, und ich denke
               daß es ihm \introOben{}Paul Horn\pwindex{Horn, Paul 13.02.1867 – 18.01.1936@\textsc{Horn, Paul} (13.02.1867 – 18.01.1936), \emph{Fabrikant/Fabrikantin}|pw} od Schenker\pwindex{Schenker, Gottfried 14.2.1842 – 26.11.1901@\textsc{Schenker, Gottfried} (14.2.1842 – 26.11.1901), \emph{Spediteur/Spediteurin}|pw}\introOben{} eben weniger Scherereien macht. Wie ist die Adresse von Paul Horn\pwindex{Horn, Paul 13.02.1867 – 18.01.1936@\textsc{Horn, Paul} (13.02.1867 – 18.01.1936), \emph{Fabrikant/Fabrikantin}|pw} und wie die der \uline{Firma}{ }Schenker\orgindex{Schenker und Co.@Schenker {\kaufmannsund}  Co.|pw}? –\pend
           
\pstart
           {\pb}III. Grüße à Discretion.\pend
           
\pstart
           IV. Bitten Sie Bahr\pwindex{Bahr, Hermann 19.07.1863 – 15.01.1934@\textsc{Bahr, Hermann} (19.07.1863 – 15.01.1934), \emph{Schriftsteller/Schriftstellerin, Kritiker/Kritikerin}|pw} er möchte die Nummern der
                  »Zeit\pwindex{Zeit. Wiener Wochenschrift@\emph{Die Zeit. Wiener Wochenschrift}|pw}« mir nachsenden ich werde meine Adresse
               ihm bekannt geben. Ich abonnire natürlich.\pend
           
\pstart
           V. Danke ich für alle Scherereien die Sie mit mir haben.\pend
           
\pstart
           Genaue Route, Tag der Abreise gebe ich Ihnen noch bekannt.\pend
           
\pstart
           Herzlichst Ihr{\\[\baselineskip]}\spacefill\mbox{Richard}\pend
           \leftskip=0em{}
\pstart
           7 Sept 94{ }Ischl\oindex{Bad Ischl@\textbf{Bad Ischl}, \emph{P.PPL}|pw}\pend
           
\pstart
           Wie ist die \uline{Adresse} der \introOben{}Adele\introOben{} Sandrock\pwindex{Sandrock, Adele 1863-08-19 – 1937-08-30@\textsc{Sandrock, Adele} (1863-08-19 – 1937-08-30), \emph{Schauspieler/Schauspielerin}|pw}? \pend
           \selectlanguage{ngerman}\endnumbering\briefempfaengerindex{Schnitzler, Arthur@\textsc{Schnitzler, Arthur}!zzzBeer-Hofmann, Richard@\emph{von Richard Beer-Hofmann}!1894-09-071@{7. 9. 1894}|)be}\mylabel{L00367h}  \normalsize

\doendnotes{C}
\bigskip
\vfill

\clearpage

\footnotesize

\lohead{\textsc{register}}

% Definiere theindex-Environment komplett neu ohne reledmac
\makeatletter
\renewenvironment{theindex}{%
  \section*{\indexname}%
  \setlength{\parindent}{0pt}%
  \setlength{\parskip}{0pt plus 0.3pt}%
  \let\item\@idxitem
}{%
  \clearpage
}
\makeatother

\IfFileExists{\jobname-pw.ind}{\input{\jobname-pw.ind}}{}

\end{document}

      