%% latex-leseansicht-vorspann.tex
%% Vorspann für die Leseansicht.
%% Lädt die gemeinsame Datei latex-vorspann.tex mit nicht gesetztem Schalter.

\newif\ifkorrekturansicht
\korrekturansichtfalse

\input{../tex-inputs/latex-vorspann}


\section[Richard Beer-Hofmann an Arthur Schnitzler, 7. 9. 1894]{L00367 Richard Beer-Hofmann an Arthur Schnitzler, 7. 9. 1894}
\nopagebreak\mylabel{L00367v}
\rehead{ }\normalsize\beginnumbering\briefempfaengerindex{Schnitzler, Arthur@\textsc{Schnitzler, Arthur}!zzzBeer-Hofmann, Richard@\emph{von Richard Beer-Hofmann}!1894-09-071@{7. 9. 1894}|(be}
\toendnotes[C]{\smallbreak\pagebreak[2]}
\correspDesc{Versand  durch Richard Beer-Hofmann am 7. 9. 1894 in Bad Ischl
\newline{}Erhalt  durch Arthur Schnitzler im Zeitraum [8. 9. 1894
                  – 12. 9. 1894?] in Wien}\toendnotes[C]{\smallbreak}
\Standort{CUL, Schnitzler, B 8.}
\physDesc{Brief, 1 Blatt, 2 Seiten, 963 Zeichen
\newline{}Handschrift: Bleistift, lateinische Kurrent
\newline{}Schnitzler: mit Bleistift nummeriert: »32« }
\buchAbdrucke{\weitereDrucke{1) Arthur Schnitzler, Richard Beer-Hofmann: \emph{Briefwechsel 1891–1931}. Herausgegeben von Konstanze Fliedl. Wien, Zürich: \emph{Europaverlag} 1992, S. 58–59.} \weitereDrucke{2) Hermann Bahr, Arthur Schnitzler: \emph{Briefwechsel, Aufzeichnungen, Dokumente (1891–1931)}. Herausgegeben von Kurt Ifkovits und Martin Anton Müller. Göttingen: \emph{Wallstein} 2018.} }\toendnotes[C]{\smallbreak}
\pstart
           \noindent{}{\pb}Lieber Arthur! Ich
               habe eine Menge Bitten an Sie.\pend
           
\pstart
           I. Senden Sie mir unter Kreuzband den Bolgar\pwindex{Bolgár, Franz von 3.\,1.\,1851 Sighetul Marmaţiei – 23.\,5.\,1923 Budapest@\textsc{Bolgár, Franz von} (3.\,1.\,1851 Sighetul Marmaţiei – 23.\,5.\,1923 Budapest), \emph{Politiker, Publizist}!Regeln des Duells@\strich\emph{Die Regeln des Duells}|pwv}\pwindex{Bolgár, Franz von 3.\,1.\,1851 Sighetul Marmaţiei – 23.\,5.\,1923 Budapest@\textsc{Bolgár, Franz von} (3.\,1.\,1851 Sighetul Marmaţiei – 23.\,5.\,1923 Budapest), \emph{Politiker, Publizist}|pw}, ich nehme ihn auf die Reise mit.\pend
           
\pstart
           II. Fragen Sie telefonisch bei Paul Horn\pwindex{Horn, Paul 13.\,2.\,1867 Wien – 18.\,1.\,1936 Menton@\textsc{Horn, Paul} (13.\,2.\,1867 Wien – 18.\,1.\,1936 Menton), \emph{Fabrikant}|pw} an ob
               es geht daß ich \strikeout{Dinge an} falls ich zollpflichtige
               Sachen \strikeout{an} von Italien\oindex{Italien@\textbf{Italien}|pw} herübersenden sollte ich sie adressiren kann an Herrn \uline{Paul Horn\pwindex{Horn, Paul 13.\,2.\,1867 Wien – 18.\,1.\,1936 Menton@\textsc{Horn, Paul} (13.\,2.\,1867 Wien – 18.\,1.\,1936 Menton), \emph{Fabrikant}|pw}} p. Adr. \uline{Schenker u.} Co\orgindex{Schenker und Co.@Schenker {\kaufmannsund}  Co.|pw} und ob dann Schenkers\orgindex{Schenker und Co.@Schenker {\kaufmannsund}  Co.|pw} die Verzollung\introOben{}sarbeiten\introOben{} etc. \strikeout{er} übernehmen. Weil ich
               nicht wegen meines Papa\pwindex{Beer, Hermann 10.\,8.\,1835 Radiměř – 3.\,10.\,1902 Wien@\textsc{Beer, Hermann} (10.\,8.\,1835 Radiměř – 3.\,10.\,1902 Wien), \emph{Rechtsanwalt}|pwv}’s
               die Sachen (Moritz gehste herunter vom Bock) an mich adressiren kann, und ich denke
               daß es ihm \introOben{}Paul Horn\pwindex{Horn, Paul 13.\,2.\,1867 Wien – 18.\,1.\,1936 Menton@\textsc{Horn, Paul} (13.\,2.\,1867 Wien – 18.\,1.\,1936 Menton), \emph{Fabrikant}|pw} od Schenker\pwindex{Schenker, Gottfried 14.\,2.\,1842 – 26.\,11.\,1901@\textsc{Schenker, Gottfried} (14.\,2.\,1842 – 26.\,11.\,1901), \emph{Spediteur}|pw}\introOben{} eben weniger Scherereien macht. Wie ist die Adresse von Paul Horn\pwindex{Horn, Paul 13.\,2.\,1867 Wien – 18.\,1.\,1936 Menton@\textsc{Horn, Paul} (13.\,2.\,1867 Wien – 18.\,1.\,1936 Menton), \emph{Fabrikant}|pw} und wie die der \uline{Firma}{ }Schenker\orgindex{Schenker und Co.@Schenker {\kaufmannsund}  Co.|pw}? –\pend
           
\pstart
           {\pb}III. Grüße à Discretion.\pend
           
\pstart
           IV. Bitten Sie Bahr\pwindex{Bahr, Hermann 19.\,7.\,1863 Linz – 15.\,1.\,1934 München@\textsc{Bahr, Hermann} (19.\,7.\,1863 Linz – 15.\,1.\,1934 München), \emph{Schriftsteller, Kritiker}|pw} er möchte die Nummern der
                  »Zeit\pwindex{Zeit. Wiener Wochenschrift@\emph{Die Zeit. Wiener Wochenschrift}|pw}« mir nachsenden ich werde meine Adresse
               ihm bekannt geben. Ich abonnire natürlich.\pend
           
\pstart
           V. Danke ich für alle Scherereien die Sie mit mir haben.\pend
           
\pstart
           Genaue Route, Tag der Abreise gebe ich Ihnen noch bekannt.\pend
           
\pstart
           Herzlichst Ihr{\\[\baselineskip]}\spacefill\mbox{Richard}\pend
           \leftskip=0em{}
\pstart
           7 Sept 94{ }Ischl\oindex{Bad Ischl@\textbf{Bad Ischl}|pw}\pend
           
\pstart
           Wie ist die \uline{Adresse} der \introOben{}Adele\introOben{} Sandrock\pwindex{Sandrock, Adele 19.\,8.\,1863 Rotterdam – 30.\,8.\,1937 Berlin@\textsc{Sandrock, Adele} (19.\,8.\,1863 Rotterdam – 30.\,8.\,1937 Berlin), \emph{Schauspielerin}|pw}?\pend
           \selectlanguage{ngerman}\endnumbering\briefempfaengerindex{Schnitzler, Arthur@\textsc{Schnitzler, Arthur}!zzzBeer-Hofmann, Richard@\emph{von Richard Beer-Hofmann}!1894-09-071@{7. 9. 1894}|)be}\mylabel{L00367h}  \newcommand{\dateiname}{L00367}\newcommand{\titel}{Richard Beer-Hofmann an Arthur Schnitzler, 7. 9. 1894}\newcommand{\editorInnen}{Herausgegeben von Martin Anton Müller}%% latex-leseansicht-abspann.tex
%% Abspann für die Leseansicht.
%% Der Schalter \ifkorrekturansicht ist bereits durch den Vorspann gesetzt.

%% latex-abspann.tex
%% Gemeinsamer Abspann für Korrekturansicht und Leseansicht.
%% Setzt den Schalter \ifkorrekturansicht voraus (gesetzt in den
%% einbindenden Dateien latex-korrekturansicht-abspann.tex bzw.
%% latex-leseansicht-abspann.tex).
%% ---------------------------------------------------------------

\normalsize

% Das esempio-Environment wird nur in der Leseansicht benötigt
\ifkorrekturansicht\else
\newenvironment{esempio}[3]%
{
    \vspace{1.5ex}
    \rlap{\underline{#1}}
    \par
    \setlength{\parindent}{0cm}
    \nopagebreak
    \leftskip=#2cm
    \rightskip=#3cm
}
{
    \par
}
\fi

\doendnotes{C}
\bigskip
\vfill

\clearpage

\footnotesize

\ifkorrekturansicht
  \lohead{\textsc{register}}
\fi

% theindex-Environment neu definieren ohne reledmac
\makeatletter
\renewenvironment{theindex}{%
  \ifkorrekturansicht
    \section*{\indexname}%
  \else
    \subsubsection*{Index der erwähnten Entitäten}%
  \fi
  \setlength{\parindent}{0pt}%
  \setlength{\parskip}{0pt plus 0.3pt}%
  \let\item\@idxitem
}{%
  \ifkorrekturansicht\clearpage\fi
}
\makeatother

\IfFileExists{\jobname-pw.ind}{\input{\jobname-pw.ind}}{}

% Quellenangabe nur in der Leseansicht
\ifkorrekturansicht\else
% Fallback-Definitionen, falls die .tex-Datei \titel etc. nicht gesetzt hat
\providecommand{\titel}{}
\providecommand{\editorInnen}{}
\providecommand{\dateiname}{\jobname}

\vspace{3cm}

\vfill

\footnotesize
\textsc{Quelle}: \titel. Herausgegeben von {\editorInnen}. In: \emph{Arthur Schnitzler: Briefwechsel mit Autorinnen und Autoren}.
 Digitale Edition, https://schnitzler-briefe.acdh.oeaw.ac.at/{\dateiname}.html (Stand \today)
\fi

\end{document}


