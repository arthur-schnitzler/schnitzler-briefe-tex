%% latex-leseansicht-vorspann.tex
%% Vorspann für die Leseansicht.
%% Lädt die gemeinsame Datei latex-vorspann.tex mit nicht gesetztem Schalter.

\newif\ifkorrekturansicht
\korrekturansichtfalse

\input{../tex-inputs/latex-vorspann}


         
         \renewcommand{\erwaehntePersonen}{Personen: Hermann Bahr, Hermann Beer, Franz von Bolgár, Paul Horn, Adele Sandrock, Gottfried Schenker}
         \renewcommand{\erwaehnteInstitutionen}{Institutionen: Schenker {\kaufmannsund} Co.}
         \renewcommand{\erwaehnteOrte}{Orte: Bad Ischl, Italien, Wien}
         \renewcommand{\erwaehnteWerke}{Werke: Die Regeln des Duells, Die Zeit. Wiener Wochenschrift}
               \section[Richard Beer-Hofmann an Arthur Schnitzler, 7. 9. 1894]{ Richard Beer-Hofmann an Arthur Schnitzler, 7. 9. 1894}\nopagebreak\mylabel{v}\rehead{ }\begin{ledgroupsized}[t]{13cm}\normalsize\beginnumbering \toendnotes[C]{\smallbreak\pagebreak[2]} \Standort{CUL, Schnitzler, B 8.}
\physDesc{Brief, 1 Blatt, 2 Seiten
\newline{}Handschrift: Bleistift, lateinische Kurrent
\newline{}Schnitzler: mit Bleistift nummeriert: »32« }\buchAbdrucke{\weitereDrucke{1) Arthur Schnitzler, Richard Beer-Hofmann: \emph{Briefwechsel 1891–1931}. Hg. Konstanze Fliedl. Wien, Zürich: \emph{Europaverlag} 1992, S. 58–59.} \weitereDrucke{2) Hermann Bahr, Arthur Schnitzler: \emph{Briefwechsel, Aufzeichnungen, Dokumente
                                (1891–1931)}. Hg. Kurt Ifkovits und Martin Anton Müller. Göttingen: \emph{Wallstein} 2018.} }\toendnotes[C]{\smallbreak}\pstart
           \noindent{}{\pb}Lieber Arthur!
                    Ich habe eine Menge Bitten an Sie.\pend
           \pstart
           I. Senden Sie mir unter Kreuzband den Bolgar\pwindex{Bolgár, Franz von 03.01.1851 – 23.05.1923@\textsc{Bolgár, Franz von} (03.01.1851 – 23.05.1923), \emph{Politiker, Publizist}!Regeln des Duells1880@\strich\emph{Die Regeln des Duells} {[}1880{]}|pwv}\pwindex{Bolgár, Franz von 03.01.1851 – 23.05.1923@\textsc{Bolgár, Franz von} (03.01.1851 – 23.05.1923), \emph{Politiker, Publizist}|pw}, ich nehme ihn auf die Reise mit.\pend
           \pstart
           II. Fragen Sie telefonisch bei Paul Horn\pwindex{Horn, Paul 13.02.1867 – 18.01.1936@\textsc{Horn, Paul} (13.02.1867 – 18.01.1936), \emph{Fabrikant}|pw} an
                    ob es geht daß ich \strikeout{Dinge an} falls ich
                    zollpflichtige Sachen \strikeout{an} von Italien\oindex{Italien@\textbf{Italien}|pw} herübersenden sollte ich sie adressiren
                    kann an Herrn \uline{Paul
                            Horn\pwindex{Horn, Paul 13.02.1867 – 18.01.1936@\textsc{Horn, Paul} (13.02.1867 – 18.01.1936), \emph{Fabrikant}|pw}} p. Adr. \uline{Schenker u.} Co\orgindex{Schenker und Co.@Schenker {\kaufmannsund}  Co.|pw} und ob dann Schenkers\orgindex{Schenker und Co.@Schenker {\kaufmannsund}  Co.|pw} die Verzollung\introOben{}sarbeiten\introOben{} etc. \strikeout{er} übernehmen. Weil
                    ich nicht wegen meines Papa\pwindex{Beer, Hermann 10.8.1835 – 03.10.1902@\textsc{Beer, Hermann} (10.8.1835 – 03.10.1902), \emph{Rechtsanwalt}|pwv}’s die Sachen
                    (Moritz gehste herunter vom Bock) an mich adressiren kann, und ich denke daß es
                    ihm \introOben{}Paul Horn\pwindex{Horn, Paul 13.02.1867 – 18.01.1936@\textsc{Horn, Paul} (13.02.1867 – 18.01.1936), \emph{Fabrikant}|pw} od
                            Schenker\pwindex{Schenker, Gottfried 14.2.1842 – 26.11.1901@\textsc{Schenker, Gottfried} (14.2.1842 – 26.11.1901), \emph{Spediteur}|pw}\introOben{} eben weniger
                    Scherereien macht. Wie ist die Adresse von Paul
                        Horn\pwindex{Horn, Paul 13.02.1867 – 18.01.1936@\textsc{Horn, Paul} (13.02.1867 – 18.01.1936), \emph{Fabrikant}|pw} und wie die der \uline{Firma}{ }Schenker\orgindex{Schenker und Co.@Schenker {\kaufmannsund}  Co.|pw}? –\pend
           \pstart
           {\pb}III. Grüße à
                    Discretion.\pend
           \pstart
           IV. Bitten Sie Bahr\pwindex{Bahr, Hermann 19.07.1863 – 15.01.1934@\textsc{Bahr, Hermann} (19.07.1863 – 15.01.1934), \emph{Schriftsteller, Kritiker}|pw} er möchte die Nummern der
                    »Zeit\pwindex{Zeit. Wiener Wochenschrift1894 – 1904@\emph{Die Zeit. Wiener Wochenschrift} {[}1894 – 1904{]}|pw}« mir nachsenden ich werde meine
                    Adresse ihm bekannt geben. Ich abonnire natürlich.\pend
           \pstart
           V. Danke ich für alle Scherereien die Sie mit mir haben.\pend
           \pstart
           Genaue Route, Tag der Abreise gebe ich Ihnen noch bekannt.\pend
           \pstart
           Herzlichst Ihr{\\[\baselineskip]}\spacefill\mbox{Richard}\pend
           \leftskip=0em{}\pstart
           7 Sept 94{ }Ischl\oindex{Bad Ischl@\textbf{Bad Ischl}|pw}\pend
           \pstart
           Wie ist die \uline{Adresse} der \introOben{}Adele\introOben{}
                     Sandrock\pwindex{Sandrock, Adele 1863-08-19 – 1937-08-30@\textsc{Sandrock, Adele} (1863-08-19 – 1937-08-30), \emph{Schauspielerin}|pw}? \pend
           
         
         \endnumbering\mylabel{h}\end{ledgroupsized}  \newcommand{\dateiname}{L00367}\newcommand{\titel}{Richard Beer-Hofmann an Arthur Schnitzler, 7. 9. 1894}\newcommand{\editorInnen}{ Martin Anton Müller und Gerd-Hermann Susen}%% latex-leseansicht-abspann.tex
%% Abspann für die Leseansicht.
%% Der Schalter \ifkorrekturansicht ist bereits durch den Vorspann gesetzt.

%% latex-abspann.tex
%% Gemeinsamer Abspann für Korrekturansicht und Leseansicht.
%% Setzt den Schalter \ifkorrekturansicht voraus (gesetzt in den
%% einbindenden Dateien latex-korrekturansicht-abspann.tex bzw.
%% latex-leseansicht-abspann.tex).
%% ---------------------------------------------------------------

\normalsize

% Das esempio-Environment wird nur in der Leseansicht benötigt
\ifkorrekturansicht\else
\newenvironment{esempio}[3]%
{
    \vspace{1.5ex}
    \rlap{\underline{#1}}
    \par
    \setlength{\parindent}{0cm}
    \nopagebreak
    \leftskip=#2cm
    \rightskip=#3cm
}
{
    \par
}
\fi

\doendnotes{C}
\bigskip
\vfill

\clearpage

\footnotesize

\ifkorrekturansicht
  \lohead{\textsc{register}}
\fi

% theindex-Environment neu definieren ohne reledmac
\makeatletter
\renewenvironment{theindex}{%
  \ifkorrekturansicht
    \section*{\indexname}%
  \else
    \subsubsection*{Index der erwähnten Entitäten}%
  \fi
  \setlength{\parindent}{0pt}%
  \setlength{\parskip}{0pt plus 0.3pt}%
  \let\item\@idxitem
}{%
  \ifkorrekturansicht\clearpage\fi
}
\makeatother

\IfFileExists{\jobname-pw.ind}{\input{\jobname-pw.ind}}{}

% Quellenangabe nur in der Leseansicht
\ifkorrekturansicht\else
% Fallback-Definitionen, falls die .tex-Datei \titel etc. nicht gesetzt hat
\providecommand{\titel}{}
\providecommand{\editorInnen}{}
\providecommand{\dateiname}{\jobname}

\vspace{3cm}

\vfill

\footnotesize
\textsc{Quelle}: \titel. Herausgegeben von {\editorInnen}. In: \emph{Arthur Schnitzler: Briefwechsel mit Autorinnen und Autoren}.
 Digitale Edition, https://schnitzler-briefe.acdh.oeaw.ac.at/{\dateiname}.html (Stand \today)
\fi

\end{document}


      