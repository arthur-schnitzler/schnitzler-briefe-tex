%% latex-leseansicht-vorspann.tex
%% Vorspann für die Leseansicht.
%% Lädt die gemeinsame Datei latex-vorspann.tex mit nicht gesetztem Schalter.

\newif\ifkorrekturansicht
\korrekturansichtfalse

\input{../tex-inputs/latex-vorspann}

\begin{center}
            \textcolor{red}{ENTWURF, NICHT FERTIG KORRIGIERT}
                      \end{center}
            
         
         \newcommand{\erwaehntePersonen}{Personen: Hermann Bahr, Gustav Kadelburg, Felix Salten, Ottilie Salten, Richard Skowronnek}
         \newcommand{\erwaehnteInstitutionen}{}
         \newcommand{\erwaehnteOrte}{Orte: Armbrustergasse, Heiligenstadt, Wien, Österreich}
         \newcommand{\erwaehnteWerke}{Werke: Burgtheater »Husarenfieber.« Schwank in vier Akten von Gustav Kadelburg und Richard Skowronnek. – Zum erstenmal: am 17. Januar 1907, Die Zeit, Husarenfieber}
               \section[Arthur Schnitzler an Felix Salten, 18. 1. 1907]{ Arthur Schnitzler an Felix Salten, 18. 1. 1907}\nopagebreak\mylabel{v}\rehead{ }\begin{ledgroupsized}[t]{13cm}\normalsize\beginnumbering \toendnotes[C]{\smallbreak\pagebreak[2]} \Standort{Wienbibliothek im Rathaus, ZPH 1681, 2.1.516.}
\physDesc{
\newline{}Handschrift: , deutsche Kurrent}\buchAbdrucke{\weitereDrucke{Hermann Bahr, Arthur Schnitzler: \emph{
                        Briefwechsel, Aufzeichnungen, Dokumente (1891–1931)
                     }. Hg. Kurt Ifkovits und Martin Anton Müller. Göttingen: \emph{Wallstein} 2018, S. 388.} }\toendnotes[C]{\smallbreak}\pstart{}{\pb}Herrn Felix Salten\pend{}\pstart{}Wien Heiligenstadt\oindex{Heiligenstadt@\textbf{Heiligenstadt}|pw}\pend{}\pstart{}Armbrusterstr. 6\oindex{Armbrustergasse@\textbf{Armbrustergasse}|pw}\pend{}{\bigskip}\pstart
           \raggedleft{}{\pb}18/1 907\pend
           \pstart
           lieber, Bahr\pwindex{Bahr, Hermann 19.07.1863 – 15.01.1934@\textsc{Bahr, Hermann} (19.07.1863 – 15.01.1934), \emph{Schriftsteller, Kritiker}|pw} ko{\geminationm}t erſt
                  \label{K_L03007-2v}\edtext{½ 2}{\lemma{\textnormal{\emph{½ 2}}}\Cendnote{\textnormal{13 Uhr 30}}}\label{K_L03007-2h}, wir ſpeiſen alſo erſt \label{K_L03007-3v}\edtext{¾ 2}{\lemma{\textnormal{\emph{¾ 2}}}\Cendnote{\textnormal{13 Uhr 45}}}\label{K_L03007-3h}, was ich zu Ordnung eventueller
               Hungerangelegenheiten gebührend mittheile. Aber ko{\geminationm}en
               Sie u Otti\pwindex{Salten, Ottilie 07.03.1868 – 22.06.1942@\textsc{Salten, Ottilie} (07.03.1868 – 22.06.1942), \emph{Schauspielerin}|pw} deswegen nicht ſpäter. \pend
           \pstart
           Herzlich {\\[\baselineskip]}\spacefill\mbox{A.}\pend
           \leftskip=0em{}\pstart
           \noindent{}Ihr \label{K_L03007-1v}\edtext{Huſarenfieberfeu{[}i{]}ll\pwindex{Salten, Felix 06.09.1869 – 08.10.1945@\textsc{Salten, Felix} (06.09.1869 – 08.10.1945), \emph{Schriftsteller, Journalist}!Burgtheater »Husarenfieber.« Schwank in vier Akten von Gustav Kadelburg und Richard Skowronnek. – Zum erstenmal: am 17. Januar 190718. 01. 1907@\strich\emph{Burgtheater »Husarenfieber.« Schwank in vier Akten von Gustav Kadelburg und Richard Skowronnek. – Zum erstenmal: am 17. Januar 1907} {[}18. 01. 1907{]}|pw}}{\lemma{\textnormal{\emph{Huſarenfieberfeuill}}}\Cendnote{\textnormal{Felix Salten\pwindex{Salten, Felix 06.09.1869 – 08.10.1945@\textsc{Salten, Felix} (06.09.1869 – 08.10.1945), \emph{Schriftsteller, Journalist}|pwk}: \emph{Burgtheater »\emph{Husarenfieber}\pwindex{Skowronnek, Richard 1862-03-12 – 1932-10-17@\textsc{Skowronnek, Richard} (1862-03-12 – 1932-10-17), \emph{Schriftsteller, Journalist, Dramatiker}!Husarenfieber1906@\strich\emph{Husarenfieber} {[}1906{]}|pwk}\pwindex{Kadelburg, Gustav 26.07.1851 – 11.09.1925@\textsc{Kadelburg, Gustav} (26.07.1851 – 11.09.1925), \emph{Schriftsteller, Schauspieler}!Husarenfieber1906@\strich\emph{Husarenfieber} {[}1906{]}|pwk}.« Schwank in vier Akten von Gustav Kadelburg\pwindex{Kadelburg, Gustav 26.07.1851 – 11.09.1925@\textsc{Kadelburg, Gustav} (26.07.1851 – 11.09.1925), \emph{Schriftsteller, Schauspieler}|pwk} und Richard Skowronnek\pwindex{Skowronnek, Richard 1862-03-12 – 1932-10-17@\textsc{Skowronnek, Richard} (1862-03-12 – 1932-10-17), \emph{Schriftsteller, Journalist, Dramatiker}|pwk}. – Zum erstenmal: am 17.
                              Januar 1907}\pwindex{Salten, Felix 06.09.1869 – 08.10.1945@\textsc{Salten, Felix} (06.09.1869 – 08.10.1945), \emph{Schriftsteller, Journalist}!Burgtheater »Husarenfieber.« Schwank in vier Akten von Gustav Kadelburg und Richard Skowronnek. – Zum erstenmal: am 17. Januar 190718. 01. 1907@\strich\emph{Burgtheater »Husarenfieber.« Schwank in vier Akten von Gustav Kadelburg und Richard Skowronnek. – Zum erstenmal: am 17. Januar 1907} {[}18. 01. 1907{]}|pwk}. In: \emph{Die Zeit}\pwindex{Zeit1902 – 1919@\emph{Die Zeit} {[}1902 – 1919{]}|pwk}, Jg. 6,
                        Nr. 1.552, 18. 1. 1907 , S. 1–2 .}}}\label{K_L03007-1h}
                  erſter Rang. Was hilft’s? Oeſterreich\oindex{Oesterreich@\textbf{Österreich}|pw} iſt
                  das Land des Verhallens. \pend
           
         
         \endnumbering\mylabel{h}\end{ledgroupsized}\begin{anhang}\end{anhang}\newcommand{\dateiname}{L03007}\newcommand{\titel}{Arthur Schnitzler an Felix Salten, 18. 1. 1907}\newcommand{\editorInnen}{Martin Anton Müller und Laura Untner}%% latex-leseansicht-abspann.tex
%% Abspann für die Leseansicht.
%% Der Schalter \ifkorrekturansicht ist bereits durch den Vorspann gesetzt.

%% latex-abspann.tex
%% Gemeinsamer Abspann für Korrekturansicht und Leseansicht.
%% Setzt den Schalter \ifkorrekturansicht voraus (gesetzt in den
%% einbindenden Dateien latex-korrekturansicht-abspann.tex bzw.
%% latex-leseansicht-abspann.tex).
%% ---------------------------------------------------------------

\normalsize

% Das esempio-Environment wird nur in der Leseansicht benötigt
\ifkorrekturansicht\else
\newenvironment{esempio}[3]%
{
    \vspace{1.5ex}
    \rlap{\underline{#1}}
    \par
    \setlength{\parindent}{0cm}
    \nopagebreak
    \leftskip=#2cm
    \rightskip=#3cm
}
{
    \par
}
\fi

\doendnotes{C}
\bigskip
\vfill

\clearpage

\footnotesize

\ifkorrekturansicht
  \lohead{\textsc{register}}
\fi

% theindex-Environment neu definieren ohne reledmac
\makeatletter
\renewenvironment{theindex}{%
  \ifkorrekturansicht
    \section*{\indexname}%
  \else
    \subsubsection*{Index der erwähnten Entitäten}%
  \fi
  \setlength{\parindent}{0pt}%
  \setlength{\parskip}{0pt plus 0.3pt}%
  \let\item\@idxitem
}{%
  \ifkorrekturansicht\clearpage\fi
}
\makeatother

\IfFileExists{\jobname-pw.ind}{\input{\jobname-pw.ind}}{}

% Quellenangabe nur in der Leseansicht
\ifkorrekturansicht\else
% Fallback-Definitionen, falls die .tex-Datei \titel etc. nicht gesetzt hat
\providecommand{\titel}{}
\providecommand{\editorInnen}{}
\providecommand{\dateiname}{\jobname}

\vspace{3cm}

\vfill

\footnotesize
\textsc{Quelle}: \titel. Herausgegeben von {\editorInnen}. In: \emph{Arthur Schnitzler: Briefwechsel mit Autorinnen und Autoren}.
 Digitale Edition, https://schnitzler-briefe.acdh.oeaw.ac.at/{\dateiname}.html (Stand \today)
\fi

\end{document}


      