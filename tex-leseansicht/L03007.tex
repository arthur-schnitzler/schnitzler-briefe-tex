%% latex-korrekturansicht-vorspann.tex
%% Vorspann für die Korrekturansicht.
%% Lädt die gemeinsame Datei latex-vorspann.tex mit gesetztem Schalter.

\newif\ifkorrekturansicht
\korrekturansichttrue

\input{../tex-inputs/latex-vorspann}


\section[ Arthur Schnitzler an Felix Salten, 18. 1. 1907]{L03007 Arthur Schnitzler an Felix Salten, 18. 1. 1907}
\nopagebreak\mylabel{L03007v}
\rehead{ }\normalsize\beginnumbering\briefempfaengerindex{Salten, Felix@\textsc{Salten, Felix}!zzzSchnitzler, Arthur@\emph{von Arthur Schnitzler}!1907-01-181@{18. 1. 1907}|(be}
\toendnotes[C]{\smallbreak\pagebreak[2]}\Standort{Wienbibliothek im Rathaus, ZPH 1681, 2.1.516.}
\physDesc{Postkarte, 325 Zeichen
\newline{}Handschrift: 1) schwarze Tinte, deutsche Kurrent\hspace{1em}2) schwarze Tinte, lateinische Kurrent (\noindent{}Adresse)\hspace{1em}
\newline{}Versand: Stempel: »\nobreak{}\textcolor{gray}{Wien}, 18. 1. 07, 9\nobreak{}«. Stempel: »\nobreak{}Wien \textcolor{gray}{118}, 19. 1. 07, 8. V, Bestellt\nobreak{}«.  
\newline{}Ordnung: mit Bleistift von unbekannter Hand nummeriert: »13« }
\buchAbdrucke{\weitereDrucke{1) Arthur Schnitzler: \emph{Briefe 1875–1912}. Frankfurt am Main: \emph{S. Fischer} 1981, S. 550.} \weitereDrucke{2) Hermann Bahr, Arthur Schnitzler: \emph{Briefwechsel, Aufzeichnungen, Dokumente (1891–1931)}. Göttingen: \emph{Wallstein} 2018, S. 388.} }\toendnotes[C]{\smallbreak}\pstart{}{\pb}Herrn Felix Salten\pend{}\pstart{}Wien Heiligenstadt\oindex{Heiligenstadt@\textbf{Heiligenstadt}, \emph{P.PPL}|pw}\pend{}\pstart{}Armbrusterstr. 6\oindex{Armbrustergasse@\textbf{Armbrustergasse}, \emph{R.ST}|pw}.\pend{}{\bigskip}\vspace{1em}
\pstart
           {\pb}\textcolor{gray}{\textbf{Dr. Arthur Schnitzler}}\hfill 18/1 907\pend
           
\pstart
           \textcolor{gray}{\textbf{Wien, XVIII. Spoettelgasse 7\oindex{Edmund-Weiss-Gasse 7@\textbf{Edmund-Weiß-Gasse 7}, \emph{Wohngebäude (K.WHS)}|pw}.}}\pend
           \vspace{0.5em}
\pstart
           lieber,{ }Bahr\pwindex{Bahr, Hermann 19.07.1863 – 15.01.1934@\textsc{Bahr, Hermann} (19.07.1863 – 15.01.1934), \emph{Schriftsteller/Schriftstellerin, Kritiker/Kritikerin}|pw} ko{\geminationm}t erſt
                  \label{K_L03007-1v}\edtext{½ 2}{\lemma{\textnormal{\emph{½ 2}}}\Cendnote{\textnormal{13 Uhr 30}}}\label{K_L03007-1}, wir ſpeiſen alſo erſt
                  \label{K_L03007-2v}\edtext{¾ 2}{\lemma{\textnormal{\emph{¾ 2}}}\Cendnote{\textnormal{13 Uhr 45}}}\label{K_L03007-2}, was ich zu Ordnung
               eventueller Hungerangelegenheiten gebührend mittheile. Aber \label{K_L03007-3v}\edtext{ko{\geminationm}en Sie u Otti\pwindex{Salten, Ottilie 07.03.1868 – 22.06.1942@\textsc{Salten, Ottilie} (07.03.1868 – 22.06.1942), \emph{Schauspieler/Schauspielerin}|pw}}{\lemma{\textnormal{\emph{kommen Sie u Otti}}}\Cendnote{\textnormal{Siehe A. S.: \emph{Tagebuch}, 19. 1. 1907.
               }}}\label{K_L03007-3} deswegen nicht ſpäter.\pend
           
\pstart
           herzlich {\\[\baselineskip]}\spacefill\mbox{A.}\pend
           \leftskip=0em{}
\pstart
           \noindent{}Ihr \label{K_L03007-4v}\edtext{Huſarenfieberfeu{[}i{]}ll\pwindex{Burgtheater. »Husarenfieber.« Schwank in vier Akten von Gustav Kadelburg und Richard Skowronnek. – Zum erstenmal: am 17. Januar 1907@\emph{Burgtheater. »Husarenfieber.« Schwank in vier Akten von Gustav Kadelburg und Richard Skowronnek. – Zum erstenmal: am 17. Januar 1907}|pw}.}{\lemma{\textnormal{\emph{Huſarenfieberfeuill.}}}\Cendnote{\textnormal{Felix Salten\pwindex{Salten, Felix 06.09.1869 – 08.10.1945@\textsc{Salten, Felix} (06.09.1869 – 08.10.1945), \emph{Schriftsteller/Schriftstellerin, Journalist/Journalistin, Chefredakteur/Chefredakteurin}|pwk}: \emph{Burgtheater. »Husarenfieber.« Schwank in vier Akten von
                           Gustav Kadelburg und Richard Skowronnek. – Zum erstenmal: am 17. Januar
                           1907}\pwindex{Burgtheater. »Husarenfieber.« Schwank in vier Akten von Gustav Kadelburg und Richard Skowronnek. – Zum erstenmal: am 17. Januar 1907@\emph{Burgtheater. »Husarenfieber.« Schwank in vier Akten von Gustav Kadelburg und Richard Skowronnek. – Zum erstenmal: am 17. Januar 1907}|pwk}. In: \emph{Die Zeit}\pwindex{Zeit@\emph{Die Zeit}|pwk}, Jg. 6,
                        Nr. 1552, 18. 1. 1907, S. 1–2.}}}\label{K_L03007-4}
                  erſter Rang. Was hilft’s? Oeſterreich\oindex{Oesterreich@\textbf{Österreich}, \emph{A.PCLI}|pw} iſt
                  das Land des Verhallens.\pend
           \selectlanguage{ngerman}\endnumbering\briefempfaengerindex{Salten, Felix@\textsc{Salten, Felix}!zzzSchnitzler, Arthur@\emph{von Arthur Schnitzler}!1907-01-181@{18. 1. 1907}|)be}\mylabel{L03007h}  \normalsize

\doendnotes{C}
\bigskip
\vfill

\clearpage

\footnotesize

\lohead{\textsc{register}}

% Definiere theindex-Environment komplett neu ohne reledmac
\makeatletter
\renewenvironment{theindex}{%
  \section*{\indexname}%
  \setlength{\parindent}{0pt}%
  \setlength{\parskip}{0pt plus 0.3pt}%
  \let\item\@idxitem
}{%
  \clearpage
}
\makeatother

\IfFileExists{\jobname-pw.ind}{\input{\jobname-pw.ind}}{}

\end{document}

      