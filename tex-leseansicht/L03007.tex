%% latex-leseansicht-vorspann.tex
%% Vorspann für die Leseansicht.
%% Lädt die gemeinsame Datei latex-vorspann.tex mit nicht gesetztem Schalter.

\newif\ifkorrekturansicht
\korrekturansichtfalse

\input{../tex-inputs/latex-vorspann}


\section[ Arthur Schnitzler an Felix Salten, 18. 1. 1907]{L03007 Arthur Schnitzler an Felix Salten,  18. 1. 1907}
\nopagebreak\mylabel{L03007v}
\rehead{ }\normalsize\beginnumbering\briefempfaengerindex{Salten, Felix@\textsc{Salten, Felix}!zzzSchnitzler, Arthur@\emph{von Arthur Schnitzler}!1907-01-181@{18. 1. 1907}|(be}
\toendnotes[C]{\smallbreak\pagebreak[2]}
\correspDesc{Versand  durch Arthur Schnitzler am 18. 1. 1907 in Wien
\newline{}Erhalt  durch Felix Salten am 19. 1. 1907 in Wien}\toendnotes[C]{\smallbreak}
\Standort{Wienbibliothek im Rathaus, ZPH 1681, 2.1.516.}
\physDesc{Postkarte, 325 Zeichen
\newline{}Handschrift: schwarze Tinte, deutsche Kurrent
\newline{}Versand: Stempel: »\nobreak{}\oindex{Wien@\textbf{Wien}, \emph{Verwaltungsgebiet}|pwk}\textcolor{gray}{Wien}, 18. 1. 07, 9\nobreak{}«. Stempel: »\nobreak{}\oindex{Wien@\textbf{Wien}, \emph{Verwaltungsgebiet}|pwk}Wien \textcolor{gray}{118}, 19. 1. 07, 8. V, Bestellt\nobreak{}«.  
\newline{}Ordnung: mit Bleistift von unbekannter Hand nummeriert: »13« }
\buchAbdrucke{\weitereDrucke{1) Arthur Schnitzler: \emph{Briefe 1875–1912}. Herausgegeben von Therese Nickl und Heinrich Schnitzler. Frankfurt am Main: \emph{S. Fischer} 1981, S. 550.} \weitereDrucke{2) Hermann Bahr, Arthur Schnitzler: \emph{Briefwechsel, Aufzeichnungen, Dokumente (1891–1931)}. Herausgegeben von Kurt Ifkovits und Martin Anton Müller. Göttingen: \emph{Wallstein} 2018, S. 388.} }\toendnotes[C]{\smallbreak}\pstart{}\textsc{{\pb}Herrn Felix Salten}\pend{}\pstart{}\textsc{Wien Heiligenstadt\oindex{Wien@\textbf{Wien}!XIX., Döbling@\textbf{XIX., Döbling}!Heiligenstadt@\textbf{Heiligenstadt}|pw}}\pend{}\pstart{}\textsc{Armbrusterstr. 6\oindex{Wien@\textbf{Wien}!XIX., Döbling@\textbf{XIX., Döbling}!Armbrustergasse@\textbf{Armbrustergasse}, \emph{Straße}|pw}.}\pend{}{\bigskip}\vspace{1em}
\pstart
           {\pb}\textcolor{gray}{\textbf{Dr. Arthur Schnitzler}}\hfill 18/1 907\pend
           
\pstart
           \textcolor{gray}{\textbf{Wien, XVIII. Spoettelgasse 7\oindex{Wien@\textbf{Wien}!XVIII., Währing@\textbf{XVIII., Währing}!Edmund-Weiß-Gasse 7@\textbf{Edmund-Weiß-Gasse 7}, \emph{Wohngebäude}|pw}.}}\pend
           \vspace{0.5em}
\pstart
           lieber,{ }Bahr\pwindex{Bahr, Hermann 19.\,7.\,1863 Linz – 15.\,1.\,1934 München@\textsc{Bahr, Hermann} (19.\,7.\,1863 Linz – 15.\,1.\,1934 München), \emph{Schriftsteller, Kritiker}|pw} ko{\geminationm}t erſt
                  \label{K_L03007-1v}\edtext{½ 2}{\lemma{\textnormal{\emph{½ 2}}}\Cendnote{\textnormal{13 Uhr 30}}}\label{K_L03007-1}, wir{ }ſpeiſen alſo erſt
                  \label{K_L03007-2v}\edtext{¾ 2}{\lemma{\textnormal{\emph{¾ 2}}}\Cendnote{\textnormal{13 Uhr 45}}}\label{K_L03007-2}, was ich zu Ordnung
               eventueller Hungerangelegenheiten gebührend mittheile. Aber \label{K_L03007-3v}\edtext{ko{\geminationm}en Sie u Otti\pwindex{Salten, Ottilie 7.\,3.\,1868 Prag – 22.\,6.\,1942 Zürich@\textsc{Salten, Ottilie} (7.\,3.\,1868 Prag – 22.\,6.\,1942 Zürich), \emph{Schauspielerin}|pw}}{\lemma{\textnormal{\emph{kommen Sie u Otti}}}\Cendnote{\textnormal{Siehe A. S.: \emph{Tagebuch}, 19. 1. 1907.
               }}}\label{K_L03007-3} deswegen nicht{ }ſpäter.\pend
           
\pstart
           herzlich {\\[\baselineskip]}\spacefill\mbox{A.}\pend
           \leftskip=0em{}
\pstart
           \noindent{}Ihr \label{K_L03007-4v}\edtext{Huſarenfieberfeu{[}i{]}ll\pwindex{Salten, Felix 6.\,9.\,1869 Budapest – 8.\,10.\,1945 Zürich@\textsc{Salten, Felix} (6.\,9.\,1869 Budapest – 8.\,10.\,1945 Zürich), \emph{Schriftsteller, Journalist, Chefredakteur}!Burgtheater. »Husarenfieber.« Schwank in vier Akten von Gustav Kadelburg und Richard Skowronnek. – Zum erstenmal: am 17. Januar 1907@\strich\emph{Burgtheater. »Husarenfieber.« Schwank in vier Akten von Gustav Kadelburg und Richard Skowronnek. – Zum erstenmal: am 17. Januar 1907}|pw}.}{\lemma{\textnormal{\emph{Husarenfieberfeuill.}}}\Cendnote{\textnormal{Felix Salten\pwindex{Salten, Felix 6.\,9.\,1869 Budapest – 8.\,10.\,1945 Zürich@\textsc{Salten, Felix} (6.\,9.\,1869 Budapest – 8.\,10.\,1945 Zürich), \emph{Schriftsteller, Journalist, Chefredakteur}|pwk}: \emph{Burgtheater. »Husarenfieber.« Schwank in vier Akten von
                           Gustav Kadelburg und Richard Skowronnek. – Zum erstenmal: am 17. Januar
                           1907}\pwindex{Salten, Felix 6.\,9.\,1869 Budapest – 8.\,10.\,1945 Zürich@\textsc{Salten, Felix} (6.\,9.\,1869 Budapest – 8.\,10.\,1945 Zürich), \emph{Schriftsteller, Journalist, Chefredakteur}!Burgtheater. »Husarenfieber.« Schwank in vier Akten von Gustav Kadelburg und Richard Skowronnek. – Zum erstenmal: am 17. Januar 1907@\strich\emph{Burgtheater. »Husarenfieber.« Schwank in vier Akten von Gustav Kadelburg und Richard Skowronnek. – Zum erstenmal: am 17. Januar 1907}|pwk}. In: \emph{Die Zeit}\pwindex{Zeit@\emph{Die Zeit}|pwk}, Jg. 6,
                        Nr. 1552, 18. 1. 1907, S. 1–2.}}}\label{K_L03007-4}
                  erſter Rang. Was hilft’s? Oeſterreich\oindex{Österreich@\textbf{Österreich}|pw} iſt
                  das Land des Verhallens.\pend
           \selectlanguage{ngerman}\endnumbering\briefempfaengerindex{Salten, Felix@\textsc{Salten, Felix}!zzzSchnitzler, Arthur@\emph{von Arthur Schnitzler}!1907-01-181@{18. 1. 1907}|)be}\mylabel{L03007h}  \newcommand{\dateiname}{L03007}\newcommand{\titel}{Arthur Schnitzler an Felix Salten, 18. 1. 1907}\newcommand{\editorInnen}{Martin Anton Müller und Laura Untner}%% latex-leseansicht-abspann.tex
%% Abspann für die Leseansicht.
%% Der Schalter \ifkorrekturansicht ist bereits durch den Vorspann gesetzt.

%% latex-abspann.tex
%% Gemeinsamer Abspann für Korrekturansicht und Leseansicht.
%% Setzt den Schalter \ifkorrekturansicht voraus (gesetzt in den
%% einbindenden Dateien latex-korrekturansicht-abspann.tex bzw.
%% latex-leseansicht-abspann.tex).
%% ---------------------------------------------------------------

\normalsize

% Das esempio-Environment wird nur in der Leseansicht benötigt
\ifkorrekturansicht\else
\newenvironment{esempio}[3]%
{
    \vspace{1.5ex}
    \rlap{\underline{#1}}
    \par
    \setlength{\parindent}{0cm}
    \nopagebreak
    \leftskip=#2cm
    \rightskip=#3cm
}
{
    \par
}
\fi

\doendnotes{C}
\bigskip
\vfill

\clearpage

\footnotesize

\ifkorrekturansicht
  \lohead{\textsc{register}}
\fi

% theindex-Environment neu definieren ohne reledmac
\makeatletter
\renewenvironment{theindex}{%
  \ifkorrekturansicht
    \section*{\indexname}%
  \else
    \subsubsection*{Index der erwähnten Entitäten}%
  \fi
  \setlength{\parindent}{0pt}%
  \setlength{\parskip}{0pt plus 0.3pt}%
  \let\item\@idxitem
}{%
  \ifkorrekturansicht\clearpage\fi
}
\makeatother

\IfFileExists{\jobname-pw.ind}{\input{\jobname-pw.ind}}{}

% Quellenangabe nur in der Leseansicht
\ifkorrekturansicht\else
% Fallback-Definitionen, falls die .tex-Datei \titel etc. nicht gesetzt hat
\providecommand{\titel}{}
\providecommand{\editorInnen}{}
\providecommand{\dateiname}{\jobname}

\vspace{3cm}

\vfill

\footnotesize
\textsc{Quelle}: \titel. Herausgegeben von {\editorInnen}. In: \emph{Arthur Schnitzler: Briefwechsel mit Autorinnen und Autoren}.
 Digitale Edition, https://schnitzler-briefe.acdh.oeaw.ac.at/{\dateiname}.html (Stand \today)
\fi

\end{document}


