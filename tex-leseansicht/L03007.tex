%% latex-leseansicht-vorspann.tex
%% Vorspann für die Leseansicht.
%% Lädt die gemeinsame Datei latex-vorspann.tex mit nicht gesetztem Schalter.

\newif\ifkorrekturansicht
\korrekturansichtfalse

\input{../tex-inputs/latex-vorspann}

\begin{center}
            \textcolor{red}{ENTWURF, NICHT FERTIG KORRIGIERT}
                      \end{center}
            
         
         \renewcommand{\erwaehntePersonen}{Personen: Hermann Bahr, Felix Salten, Ottilie Salten}
         \renewcommand{\erwaehnteOrte}{Orte: Armbrustergasse, Edmund-Weiß-Gasse 7, Heiligenstadt, Wien, Österreich}
         \renewcommand{\erwaehnteWerke}{Werke: Burgtheater. »Husarenfieber.« Schwank in vier Akten von Gustav Kadelburg und Richard Skowronnek. – Zum erstenmal: am 17. Januar 1907, Die Zeit}
               \section[ Arthur Schnitzler an Felix Salten, 18. 1. 1907]{ Arthur Schnitzler an Felix Salten, 18. 1. 1907}\nopagebreak\mylabel{v}\rehead{ }\begin{ledgroupsized}[t]{13cm}\normalsize\beginnumbering \toendnotes[C]{\smallbreak\pagebreak[2]} \Standort{Wienbibliothek im Rathaus, ZPH 1681, 2.1.516.}
\physDesc{Postkarte, 324 Zeichen
\newline{}Handschrift: 1) schwarze Tinte, deutsche Kurrent\hspace{1em}2) schwarze Tinte, lateinische Kurrent (\noindent{}Adresse)\hspace{1em}
\newline{}Versand: Stempel: »\nobreak{}\textcolor{gray}{Wien} 3a, 18. 1. 07, 9\nobreak{}«. Stempel: »\nobreak{}Wien \textcolor{gray}{118}, 19. 1. 07, 8. V, Bestellt\nobreak{}«.  
\newline{}Ordnung: mit Bleistift von unbekannter Hand Nummerierung der Blätter des Konvoluts:
                                    »13« }\buchAbdrucke{\weitereDrucke{1) Arthur Schnitzler: \emph{Briefe 1875–1912}. Hg. Therese Nickl und Heinrich Schnitzler. Frankfurt am Main: \emph{S. Fischer} 1981, S. 550.} \weitereDrucke{2) Hermann Bahr, Arthur Schnitzler: \emph{Briefwechsel, Aufzeichnungen, Dokumente (1891–1931)}. Hg. Kurt Ifkovits und Martin Anton Müller. Göttingen: \emph{Wallstein} 2018, S. 388.} }\toendnotes[C]{\smallbreak}\pstart{}{\pb}Herrn Felix Salten\pend{}\pstart{}Wien Heiligenstadt\oindex{Heiligenstadt@\textbf{Heiligenstadt}|pw}\pend{}\pstart{}Armbrusterstr. 6\oindex{Armbrustergasse@\textbf{Armbrustergasse}|pw}.\pend{}{\bigskip}\pstart
           \noindent{}{\pb}\textcolor{gray}{\textbf{Dr. Arthur Schnitzler}}\hfill 18/1 907\pend
           \pstart
           \textcolor{gray}{\textbf{Wien, XVIII. Spoettelgasse 7\oindex{Edmund-Weiss-Gasse 7@\textbf{Edmund-Weiß-Gasse 7}|pw}.}}\pend
           \pstart
           lieber,{ }Bahr\pwindex{Bahr, Hermann 19.07.1863 – 15.01.1934@\textsc{Bahr, Hermann} (19.07.1863 – 15.01.1934), \emph{Schriftsteller, Kritiker}|pw} ko{\geminationm}t erſt
                  \label{K_L03007-1v}\edtext{½ 2}{\lemma{\textnormal{\emph{½ 2}}}\Cendnote{\textnormal{13 Uhr 30}}}\label{K_L03007-1h}, wir ſpeiſen alſo erſt
                  \label{K_L03007-2v}\edtext{¾ 2}{\lemma{\textnormal{\emph{¾ 2}}}\Cendnote{\textnormal{13 Uhr 45}}}\label{K_L03007-2h}, was ich zu Ordnung
               eventueller Hungerangelegenheiten gebührend mittheile. Aber \label{K_L03007-3v}\edtext{ko{\geminationm}en Sie u Otti\pwindex{Salten, Ottilie 07.03.1868 – 22.06.1942@\textsc{Salten, Ottilie} (07.03.1868 – 22.06.1942), \emph{Schauspielerin}|pw}}{\lemma{\textnormal{\emph{kommen Sie u Otti}}}\Cendnote{\textnormal{siehe A. S.: \emph{Tagebuch}, 19. 1. 1907}}}\label{K_L03007-3h} deswegen nicht ſpäter.\pend
           \pstart
           herzlich {\\[\baselineskip]}\spacefill\mbox{A.}\pend
           \leftskip=0em{}\pstart
           \noindent{}Ihr \label{K_L03007-4v}\edtext{Huſarenfieberfeu{[}i{]}ll\pwindex{Salten, Felix 06.09.1869 – 08.10.1945@\textsc{Salten, Felix} (06.09.1869 – 08.10.1945), \emph{Schriftsteller, Journalist}!Burgtheater. »Husarenfieber.« Schwank in vier Akten von Gustav Kadelburg und
                  Richard Skowronnek. – Zum erstenmal: am 17. Januar 19071907-01-18@\strich\emph{Burgtheater. »Husarenfieber.« Schwank in vier Akten von Gustav Kadelburg und Richard Skowronnek. – Zum erstenmal: am 17. Januar 1907} {[}1907-01-18{]}|pw}}{\lemma{\textnormal{\emph{Huſarenfieberfeuill}}}\Cendnote{\textnormal{Felix Salten\pwindex{Salten, Felix 06.09.1869 – 08.10.1945@\textsc{Salten, Felix} (06.09.1869 – 08.10.1945), \emph{Schriftsteller, Journalist}|pwk}: \emph{Burgtheater. »Husarenfieber.« Schwank in vier Akten von
                           Gustav Kadelburg und Richard Skowronnek. – Zum erstenmal: am 17. Januar
                           1907}\pwindex{Salten, Felix 06.09.1869 – 08.10.1945@\textsc{Salten, Felix} (06.09.1869 – 08.10.1945), \emph{Schriftsteller, Journalist}!Burgtheater. »Husarenfieber.« Schwank in vier Akten von Gustav Kadelburg und
                  Richard Skowronnek. – Zum erstenmal: am 17. Januar 19071907-01-18@\strich\emph{Burgtheater. »Husarenfieber.« Schwank in vier Akten von Gustav Kadelburg und Richard Skowronnek. – Zum erstenmal: am 17. Januar 1907} {[}1907-01-18{]}|pwk}. In: \emph{Die Zeit}\pwindex{Zeit1902-09-27 – 1919@\emph{Die Zeit} {[}1902-09-27 – 1919{]}|pwk}, Jg. 6,
                        Nr. 1.552, 18. 1. 1907, S. 1–2.}}}\label{K_L03007-4h}
                  erſter Rang. Was hilft’s? Oeſterreich\oindex{Oesterreich@\textbf{Österreich}|pw} iſt
                  das Land des Verhallens.\pend
           
         
         \endnumbering\mylabel{h}\end{ledgroupsized}  \newcommand{\dateiname}{L03007}\newcommand{\titel}{Arthur Schnitzler an Felix Salten, 18. 1. 1907}\newcommand{\editorInnen}{Martin Anton Müller und Laura Untner}%% latex-leseansicht-abspann.tex
%% Abspann für die Leseansicht.
%% Der Schalter \ifkorrekturansicht ist bereits durch den Vorspann gesetzt.

%% latex-abspann.tex
%% Gemeinsamer Abspann für Korrekturansicht und Leseansicht.
%% Setzt den Schalter \ifkorrekturansicht voraus (gesetzt in den
%% einbindenden Dateien latex-korrekturansicht-abspann.tex bzw.
%% latex-leseansicht-abspann.tex).
%% ---------------------------------------------------------------

\normalsize

% Das esempio-Environment wird nur in der Leseansicht benötigt
\ifkorrekturansicht\else
\newenvironment{esempio}[3]%
{
    \vspace{1.5ex}
    \rlap{\underline{#1}}
    \par
    \setlength{\parindent}{0cm}
    \nopagebreak
    \leftskip=#2cm
    \rightskip=#3cm
}
{
    \par
}
\fi

\doendnotes{C}
\bigskip
\vfill

\clearpage

\footnotesize

\ifkorrekturansicht
  \lohead{\textsc{register}}
\fi

% theindex-Environment neu definieren ohne reledmac
\makeatletter
\renewenvironment{theindex}{%
  \ifkorrekturansicht
    \section*{\indexname}%
  \else
    \subsubsection*{Index der erwähnten Entitäten}%
  \fi
  \setlength{\parindent}{0pt}%
  \setlength{\parskip}{0pt plus 0.3pt}%
  \let\item\@idxitem
}{%
  \ifkorrekturansicht\clearpage\fi
}
\makeatother

\IfFileExists{\jobname-pw.ind}{\input{\jobname-pw.ind}}{}

% Quellenangabe nur in der Leseansicht
\ifkorrekturansicht\else
% Fallback-Definitionen, falls die .tex-Datei \titel etc. nicht gesetzt hat
\providecommand{\titel}{}
\providecommand{\editorInnen}{}
\providecommand{\dateiname}{\jobname}

\vspace{3cm}

\vfill

\footnotesize
\textsc{Quelle}: \titel. Herausgegeben von {\editorInnen}. In: \emph{Arthur Schnitzler: Briefwechsel mit Autorinnen und Autoren}.
 Digitale Edition, https://schnitzler-briefe.acdh.oeaw.ac.at/{\dateiname}.html (Stand \today)
\fi

\end{document}


      