%% latex-leseansicht-vorspann.tex
%% Vorspann für die Leseansicht.
%% Lädt die gemeinsame Datei latex-vorspann.tex mit nicht gesetztem Schalter.

\newif\ifkorrekturansicht
\korrekturansichtfalse

\input{../tex-inputs/latex-vorspann}


               \section[Arthur Schnitzler an Hugo von Hofmannsthal, 18. 2. 1893]{ Arthur Schnitzler an Hugo von Hofmannsthal, 18. 2. 1893}\nopagebreak\mylabel{v}\rehead{ }\begin{ledgroupsized}[t]{13cm}\normalsize\beginnumbering\briefempfaengerindex{Hofmannsthal, Hugo von@\textsc{Hofmannsthal, Hugo von}!zzzSchnitzler, Arthur@\emph{von Arthur Schnitzler}!1893-02-181@{18. 2. 1893}|(be} \toendnotes[C]{\smallbreak\pagebreak[2]} \Standort{FDH, Hs-30885,34.}
\physDesc{Brief, 1 Blatt, 4 Seiten
\newline{}Handschrift: Bleistift, deutsche Kurrent\newline{}Ordnung: von Schnitzler mutmaßlich während der Durchsicht der Briefe 1929 mit Bleistift am oberen
                                    Blattrand zusätzlich datiert: »18/2 93« }\buchAbdrucke{\weitereDrucke{1) Hugo von Hofmannsthal, Arthur Schnitzler: \emph{Briefwechsel}. Hg. Therese Nickl und Heinrich Schnitzler. Frankfurt am Main: \emph{S. Fischer} 1964, S. 36.} \weitereDrucke{2) Hermann Bahr, Arthur Schnitzler: \emph{Briefwechsel, Aufzeichnungen, Dokumente
                                (1891–1931)}. Hg. Kurt Ifkovits und Martin Anton Müller. Göttingen: \emph{Wallstein} 2018.} }\toendnotes[C]{\smallbreak}\pstart{}{\pb}Lieber Hugo,\pend\pstart
           bitte leſen Sie \label{K_L00178_1v}\edtext{beiliegenden
                        Brief}{\lemma{\textnormal{\emph{beiliegenden
                        Brief}}}\Cendnote{\textnormal{Zwei Briefe Fels\pwindex{Fels, Friedrich Michael *~1864@\textsc{Fels, Friedrich Michael} (*~1864), \emph{Journalist}|pwk}’ aus dem Hotel
                            Erzherzog Rainer in Meran-Obermais\oindex{Erzherzog Rainer@\textbf{Erzherzog Rainer}|pwk} (\emph{Deutsches Literaturarchiv}, A:Schnitzler, 85.1.2956) sind mit
                            18. 2. 1893 datiert, wobei sich erschließen lässt, dass
                        einer am Tag vor dem anderen verfasst ist. Mit Bleistift wurde zum ersten
                        Datum »16«, zum zweiten »17« geschrieben. Schnitzler\pwindex{Schnitzler, Arthur 15.05.1862 – 21.10.1931@\textsc{Schnitzler, Arthur} (15.05.1862 – 21.10.1931), \emph{Schriftsteller, Mediziner}|pwk} dürfte Hofmannsthal\pwindex{Hofmannsthal, Hugo von 01.02.1874 – 15.07.1929@\textsc{Hofmannsthal, Hugo von} (01.02.1874 – 15.07.1929), \emph{Schriftsteller}|pwk}
                        den ersten mitteilen, der die Ankunft in Meran\oindex{Meran@\textbf{Meran}|pwk} schildert. Für die Rekonvaleszenz sind drei Monate
                        angesetzt, weswegen Fels\pwindex{Fels, Friedrich Michael *~1864@\textsc{Fels, Friedrich Michael} (*~1864), \emph{Journalist}|pwk} fürchtet, keine
                        Stelle bei der \emph{Deutschen Zeitung}\orgindex{Deutsche Zeitung@Deutsche Zeitung|pwk} zu
                        bekommen.}}}\label{K_L00178_1h}. Und dann fragen Sie gütigſt Bahr\pwindex{Bahr, Hermann 19.07.1863 – 15.01.1934@\textsc{Bahr, Hermann} (19.07.1863 – 15.01.1934), \emph{Schriftsteller, Kritiker}|pw}, wie die Ausſichten des Dr. \textsc{Fels}\pwindex{Fels, Friedrich Michael *~1864@\textsc{Fels, Friedrich Michael} (*~1864), \emph{Journalist}|pw} bei der Dtſch Ztg\orgindex{Deutsche Zeitung@Deutsche Zeitung|pw}{ }ſtehn, und wann er eintreffen müſſte. Es wäre
                    mir höchſt erwünſcht, darüber vollko{\geminationm}ene Klarheit
                    zu haben. Sie erſehen auch {\pb}weiters aus dem Brief,
                    daſs auf Ihre liebenswürdige Zuſage, eine neuerliche Sa{\geminationm}lg zu veranſtalten, reflectirt wird. Je früher
                    mir Ihre Reſultate in jeder Richtung bekannt werden, umſo dankbarer bin ich
                    Ihnen im Namen unſres Kranken.\pend
           \pstart
           – Wa{\geminationn} werden wir wieder einmal geſcheidte Dinge {\pb}miteinander ſprechen? Was machen Sie? Ich wäre ſehr
                    erfreut, wieder einmal mit Ihnen zusa{\geminationm}en zu ſein.
                    Ich bin jeden Abend nach 10 im Central\oindex{Cafe Central@\textbf{Café Central}|pw},
                    Dienſtag, Donnerſtag, Samſtag ſicher. Den beigelegten Brief bitte mir mit Ihrer
                    frdl Antwort gef rückzuſenden.\pend
           \pstart
           {\pb}Herzlich der Ihre{\\[\baselineskip]}\spacefill\mbox{Arthur.}\pend
           \leftskip=0em{}\pstart
           \raggedleft{}18. 2. 93\pend
           \endnumbering\briefempfaengerindex{Hofmannsthal, Hugo von@\textsc{Hofmannsthal, Hugo von}!zzzSchnitzler, Arthur@\emph{von Arthur Schnitzler}!1893-02-181@{18. 2. 1893}|)be}\mylabel{h}\end{ledgroupsized}  \newcommand{\dateiname}{L00178}\newcommand{\titel}{Arthur Schnitzler an Hugo von Hofmannsthal, 18. 2. 1893}\newcommand{\editorInnen}{ Martin Anton Müller und Gerd-Hermann Susen}
            \footnotesize
\begin{ledgroupsized}[t]{11.5cm}
\doendnotes{C}
\end{ledgroupsized}
         %% latex-leseansicht-abspann.tex
%% Abspann für die Leseansicht.
%% Der Schalter \ifkorrekturansicht ist bereits durch den Vorspann gesetzt.

%% latex-abspann.tex
%% Gemeinsamer Abspann für Korrekturansicht und Leseansicht.
%% Setzt den Schalter \ifkorrekturansicht voraus (gesetzt in den
%% einbindenden Dateien latex-korrekturansicht-abspann.tex bzw.
%% latex-leseansicht-abspann.tex).
%% ---------------------------------------------------------------

\normalsize

% Das esempio-Environment wird nur in der Leseansicht benötigt
\ifkorrekturansicht\else
\newenvironment{esempio}[3]%
{
    \vspace{1.5ex}
    \rlap{\underline{#1}}
    \par
    \setlength{\parindent}{0cm}
    \nopagebreak
    \leftskip=#2cm
    \rightskip=#3cm
}
{
    \par
}
\fi

\doendnotes{C}
\bigskip
\vfill

\clearpage

\footnotesize

\ifkorrekturansicht
  \lohead{\textsc{register}}
\fi

% theindex-Environment neu definieren ohne reledmac
\makeatletter
\renewenvironment{theindex}{%
  \ifkorrekturansicht
    \section*{\indexname}%
  \else
    \subsubsection*{Index der erwähnten Entitäten}%
  \fi
  \setlength{\parindent}{0pt}%
  \setlength{\parskip}{0pt plus 0.3pt}%
  \let\item\@idxitem
}{%
  \ifkorrekturansicht\clearpage\fi
}
\makeatother

\IfFileExists{\jobname-pw.ind}{\input{\jobname-pw.ind}}{}

% Quellenangabe nur in der Leseansicht
\ifkorrekturansicht\else
% Fallback-Definitionen, falls die .tex-Datei \titel etc. nicht gesetzt hat
\providecommand{\titel}{}
\providecommand{\editorInnen}{}
\providecommand{\dateiname}{\jobname}

\vspace{3cm}

\vfill

\footnotesize
\textsc{Quelle}: \titel. Herausgegeben von {\editorInnen}. In: \emph{Arthur Schnitzler: Briefwechsel mit Autorinnen und Autoren}.
 Digitale Edition, https://schnitzler-briefe.acdh.oeaw.ac.at/{\dateiname}.html (Stand \today)
\fi

\end{document}


      