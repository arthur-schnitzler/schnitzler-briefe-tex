%% latex-leseansicht-vorspann.tex
%% Vorspann für die Leseansicht.
%% Lädt die gemeinsame Datei latex-vorspann.tex mit nicht gesetztem Schalter.

\newif\ifkorrekturansicht
\korrekturansichtfalse

\input{../tex-inputs/latex-vorspann}


         
         \renewcommand{\erwaehntePersonen}{Personen:  ?? [Mitarbeiterin von La Vie Parisienne], Hermann Bahr, Richard Beer-Hofmann, Paul Goldmann, Hugo von Hofmannsthal, Carl Pietzner, Felix Salten, Gustav Schwarzkopf, Jean Thorel, Auguste Vaillant}
         \renewcommand{\erwaehnteInstitutionen}{Institutionen: Deutsche Zeitung, Französische Abgeordnetenkammer, Französische Nationalversammlung, La Vie Parisienne}
         \renewcommand{\erwaehnteOrte}{Orte: Paris, Wien}
         \renewcommand{\erwaehnteWerke}{Werke: Anatol, Arthur Schnitzler (1893), Deutsche Zeitung, Liebelei. Schauspiel in drei Akten, Sterben. Novelle}
               \section[Paul Goldmann an Arthur Schnitzler, 23. 12. {[}1893{]}]{ Paul Goldmann an Arthur Schnitzler, 23. 12. {[}1893{]}}\nopagebreak\mylabel{v}\rehead{ }\begin{ledgroupsized}[t]{13cm}\normalsize\beginnumbering\briefempfaengerindex{Schnitzler, Arthur@\textsc{Schnitzler, Arthur}!zzzGoldmann, Paul@\emph{von Paul Goldmann}!1893-12-231@{23. 12. {[}1893{]}}|(be} \toendnotes[C]{\smallbreak\pagebreak[2]} \Standort{DLA, A:Schnitzler, HS.NZ85.1.3163.}
\physDesc{Brief, 1 Blatt, 4 Seiten, 2062 Zeichen
\newline{}Handschrift: schwarze Tinte, deutsche Kurrent
\newline{}Schnitzler: 1) mit Bleistift das Jahr »93« vermerkt  2) mit rotem Buntstift zwei Unterstreichungen}\toendnotes[C]{\smallbreak}\pstart
           \raggedleft{}{\pb}\textsc{Paris\oindex{Paris@\textbf{Paris}|pw}}, 23. December.\pend
           \pstart\center{}Mein lieber Freund!\pend\pstart
           Dein letzter Brief und die ſich daran ſchließenden Zeilen der \label{K_L02724-1v}\edtext{Freunde\pwindex{Hofmannsthal, Hugo von 1874-02-01 – 1929-07-15@\textsc{Hofmannsthal, Hugo von} (1874-02-01 – 1929-07-15), \emph{Schriftsteller}|pwuv}\pwindex{Beer-Hofmann, Richard 1866-07-11 – 1945-09-26@\textsc{Beer-Hofmann, Richard} (1866-07-11 – 1945-09-26), \emph{Schriftsteller}|pwuv}\pwindex{Salten, Felix 06.09.1869 – 08.10.1945@\textsc{Salten, Felix} (06.09.1869 – 08.10.1945), \emph{Schriftsteller, Journalist}|pwuv}\pwindex{Schwarzkopf, Gustav 07.11.1853 – 13.11.1939@\textsc{Schwarzkopf, Gustav} (07.11.1853 – 13.11.1939), \emph{Schriftsteller}|pwuv}}{\lemma{\textnormal{\emph{Freunde}}}\Cendnote{\textnormal{Schnitzler\pwindex{Schnitzler, Arthur 15.05.1862 – 21.10.1931@\textsc{Schnitzler, Arthur} (15.05.1862 – 21.10.1931), \emph{Schriftsteller, Mediziner}|pwk}s Brief könnte am 10. 12. 1893 abgefasst
                  worden sein, als er die mit Goldmann\pwindex{Goldmann, Paul 31.01.1865 – 25.09.1935@\textsc{Goldmann, Paul} (31.01.1865 – 25.09.1935), \emph{Schriftsteller, Journalist}|pwk}
                  bekannten Freunde Hugo von Hofmannsthal\pwindex{Hofmannsthal, Hugo von 1874-02-01 – 1929-07-15@\textsc{Hofmannsthal, Hugo von} (1874-02-01 – 1929-07-15), \emph{Schriftsteller}|pwk},
                     Richard Beer-Hofmann\pwindex{Beer-Hofmann, Richard 1866-07-11 – 1945-09-26@\textsc{Beer-Hofmann, Richard} (1866-07-11 – 1945-09-26), \emph{Schriftsteller}|pwk}, Felix Salten\pwindex{Salten, Felix 06.09.1869 – 08.10.1945@\textsc{Salten, Felix} (06.09.1869 – 08.10.1945), \emph{Schriftsteller, Journalist}|pwk} und Gustav
                     Schwarzkopf\pwindex{Schwarzkopf, Gustav 07.11.1853 – 13.11.1939@\textsc{Schwarzkopf, Gustav} (07.11.1853 – 13.11.1939), \emph{Schriftsteller}|pwk} traf.}}}\label{K_L02724-1h} haben mir eine unendliche Freude bereitet. Mir
               ſind die Thränen in die Augen gekommen, als ich all’ das las. Und ich war einen
               ganzen Tag lang glücklich, ſo viel Freundſchaft und Treue verdient zu haben. Gern
               hätte ich Dir, dem lieben Anſtifter der Freudengabe, und allen Betheiligten\pwindex{Hofmannsthal, Hugo von 1874-02-01 – 1929-07-15@\textsc{Hofmannsthal, Hugo von} (1874-02-01 – 1929-07-15), \emph{Schriftsteller}|pwuv}\pwindex{Beer-Hofmann, Richard 1866-07-11 – 1945-09-26@\textsc{Beer-Hofmann, Richard} (1866-07-11 – 1945-09-26), \emph{Schriftsteller}|pwuv}\pwindex{Salten, Felix 06.09.1869 – 08.10.1945@\textsc{Salten, Felix} (06.09.1869 – 08.10.1945), \emph{Schriftsteller, Journalist}|pwuv}\pwindex{Schwarzkopf, Gustav 07.11.1853 – 13.11.1939@\textsc{Schwarzkopf, Gustav} (07.11.1853 – 13.11.1939), \emph{Schriftsteller}|pwuv} ſofort gedankt. Da kam die \label{K_L02724-2v}\edtext{Bombe in der Kammer\orgindex{Franzoesische Abgeordnetenkammer@Französische Abgeordnetenkammer|pw}}{\lemma{\textnormal{\emph{Bombe in der Kammer}}}\Cendnote{\textnormal{Am 9. 12. 1893 verübte der
                  Anarchist Auguste Vaillant\pwindex{Vaillant, Auguste 1861-12-27 – 1894-02-05@\textsc{Vaillant, Auguste} (1861-12-27 – 1894-02-05), \emph{Anarchist, Attentäter}|pwk} ein
                  Bombenattentat auf die \emph{Französische
                     Nationalversammlung}\orgindex{Franzoesische Nationalversammlung@Französische Nationalversammlung|pwk}, bei dem um die 50 Personen verletzt wurden.}}}\label{K_L02724-2h}
               und ſonſt Allerlei und warf mich weit ab von Euch\pwindex{Hofmannsthal, Hugo von 1874-02-01 – 1929-07-15@\textsc{Hofmannsthal, Hugo von} (1874-02-01 – 1929-07-15), \emph{Schriftsteller}|pwuv}\pwindex{Beer-Hofmann, Richard 1866-07-11 – 1945-09-26@\textsc{Beer-Hofmann, Richard} (1866-07-11 – 1945-09-26), \emph{Schriftsteller}|pwuv}\pwindex{Salten, Felix 06.09.1869 – 08.10.1945@\textsc{Salten, Felix} (06.09.1869 – 08.10.1945), \emph{Schriftsteller, Journalist}|pwuv}\pwindex{Schwarzkopf, Gustav 07.11.1853 – 13.11.1939@\textsc{Schwarzkopf, Gustav} (07.11.1853 – 13.11.1939), \emph{Schriftsteller}|pwuv} und all’ den
               frohen Gedanken. {\pb}Inzwiſchen kam auch Dein liebes
                  \label{K_L02724-3v}\edtext{Bild\pwindex{Pietzner, Carl 1853-04-09 – 1927-11-25@\textsc{Pietzner, Carl} (1853-04-09 – 1927-11-25), \emph{Bildender Künstler}!Arthur Schnitzler (1893)1893@\strich\emph{Arthur Schnitzler (1893)} {[}1893{]}|pwv}}{\lemma{\textnormal{\emph{Bild}}}\Cendnote{\textnormal{Wohl das von Carl Pietzner\pwindex{Pietzner, Carl 1853-04-09 – 1927-11-25@\textsc{Pietzner, Carl} (1853-04-09 – 1927-11-25), \emph{Bildender Künstler}|pwk} erstellte Porträtfoto\pwindex{Pietzner, Carl 1853-04-09 – 1927-11-25@\textsc{Pietzner, Carl} (1853-04-09 – 1927-11-25), \emph{Bildender Künstler}!Arthur Schnitzler (1893)1893@\strich\emph{Arthur Schnitzler (1893)} {[}1893{]}|pwkv} von Schnitzler\pwindex{Schnitzler, Arthur 15.05.1862 – 21.10.1931@\textsc{Schnitzler, Arthur} (15.05.1862 – 21.10.1931), \emph{Schriftsteller, Mediziner}|pwk}, vgl. Arthur Schnitzler an Hermann Bahr, 2. 12. 1893.}}}\label{K_L02724-3h}. Dank, innigen Dank für die Sendung. Ich habe es auf meinem
               Schreibtiſch aufgeſtellt und tauſche mit Dir manch’ einen Blick und verſinke in
               manch’ eine Träumerei während irgend eines politiſchen Artikels. Es iſt eine
               vorzügliche Aufnahme\pwindex{Pietzner, Carl 1853-04-09 – 1927-11-25@\textsc{Pietzner, Carl} (1853-04-09 – 1927-11-25), \emph{Bildender Künstler}!Arthur Schnitzler (1893)1893@\strich\emph{Arthur Schnitzler (1893)} {[}1893{]}|pwv} –
               wenngleich Du freilich in Wirklichkeit nie ſo hübſch geweſen. Auch zeige ich Dich
               Allen, die mich beſuchen kommen, und Du haſt viel Erfolg. Neulich war \textsc{Jean Thorel\pwindex{Thorel, Jean 1859-09-11 – 1916-08-20@\textsc{Thorel, Jean} (1859-09-11 – 1916-08-20), \emph{Übersetzer, Schriftsteller}|pw}} bei mir und ſagte: \label{K_L02724-4v}\edtext{»\textsc{\begin{otherlanguage}{french}Je jurerais, que c’est un monsieur, qui écrit des
                     comédies.\end{otherlanguage}}«}{\lemma{\textnormal{\emph{»Je … comédies.«}}}\Cendnote{\textnormal{französisch: Ich könnte schwören,
                  dass das ein Herr\pwindex{Schnitzler, Arthur 15.05.1862 – 21.10.1931@\textsc{Schnitzler, Arthur} (15.05.1862 – 21.10.1931), \emph{Schriftsteller, Mediziner}|pwkv} ist, der
                  Lustspiele schreibt.}}}\label{K_L02724-4h} Wenn Du jetzt {\pb}\strikeout{\textcolor{gray}{noc}} noch \label{K_L02724-5v}\edtext{keine Luſtſpiele ſchreiben
                  willſt}{\lemma{\textnormal{\emph{keine … willſt}}}\Cendnote{\textnormal{vgl. Paul Goldmann an Arthur Schnitzler, 8. 12. [1893]}}}\label{K_L02724-5h}{\dotsfour}!\pend
           \pstart
           Bitte liebſter Freund, ſchreib’ mir ein ausführlicheres Wort über Deine Pläne. Die
               Idee mit dem \label{K_L02724-6v}\edtext{ſüßen Wien\oindex{Wien@\textbf{Wien}|pw}er Stück\pwindex{Schnitzler, Arthur 15.05.1862 – 21.10.1931@\textsc{Schnitzler, Arthur} (15.05.1862 – 21.10.1931), \emph{Schriftsteller, Mediziner}!Liebelei. Schauspiel in drei Akten1895-10-09@\strich\emph{Liebelei. Schauspiel in drei Akten} {[}1895-10-09{]}|pwv}}{\lemma{\textnormal{\emph{ſüßen Wiener Stück}}}\Cendnote{\textnormal{\emph{Liebelei}\pwindex{Schnitzler, Arthur 15.05.1862 – 21.10.1931@\textsc{Schnitzler, Arthur} (15.05.1862 – 21.10.1931), \emph{Schriftsteller, Mediziner}!Liebelei. Schauspiel in drei Akten1895-10-09@\strich\emph{Liebelei. Schauspiel in drei Akten} {[}1895-10-09{]}|pwk}, das unter dem Titel »Armes Mädl\pwindex{Schnitzler, Arthur 15.05.1862 – 21.10.1931@\textsc{Schnitzler, Arthur} (15.05.1862 – 21.10.1931), \emph{Schriftsteller, Mediziner}!Liebelei. Schauspiel in drei Akten1895-10-09@\strich\emph{Liebelei. Schauspiel in drei Akten} {[}1895-10-09{]}|pwkv}« als Volksstück
                  geplant war. Das »süß« dürfte sich auf das »süße Mädl« beziehen, das schon früher
                  in den Briefen Goldmann\pwindex{Goldmann, Paul 31.01.1865 – 25.09.1935@\textsc{Goldmann, Paul} (31.01.1865 – 25.09.1935), \emph{Schriftsteller, Journalist}|pwk}s Thema war (vgl. Paul Goldmann an Arthur Schnitzler, 18. 8. 1890). Die Popularisierung
                  des Begriffs wird häufig Schnitzler\pwindex{Schnitzler, Arthur 15.05.1862 – 21.10.1931@\textsc{Schnitzler, Arthur} (15.05.1862 – 21.10.1931), \emph{Schriftsteller, Mediziner}|pwk}
                  zugeschrieben und der Erfolg der \emph{Liebelei}\pwindex{Schnitzler, Arthur 15.05.1862 – 21.10.1931@\textsc{Schnitzler, Arthur} (15.05.1862 – 21.10.1931), \emph{Schriftsteller, Mediziner}!Liebelei. Schauspiel in drei Akten1895-10-09@\strich\emph{Liebelei. Schauspiel in drei Akten} {[}1895-10-09{]}|pwk}
                  spielt dabei eine zentrale Rolle. Diese Briefstelle legt nahe, dass schon bei der
                  Konzeption der \emph{Liebelei}\pwindex{Schnitzler, Arthur 15.05.1862 – 21.10.1931@\textsc{Schnitzler, Arthur} (15.05.1862 – 21.10.1931), \emph{Schriftsteller, Mediziner}!Liebelei. Schauspiel in drei Akten1895-10-09@\strich\emph{Liebelei. Schauspiel in drei Akten} {[}1895-10-09{]}|pwk} das Vorhaben eine
                  zentrale Rolle spielte, den Typus »unkomplizierte Frau für eine sexuelle Beziehung
                  ohne längerfristige Bindung« auf die Bühne zu bringen.}}}\label{K_L02724-6h} gefällt mir ſehr.
               Das müßte Dir ganz ausnehmend liegen. Und ſchreib’ vor allen Dingen ein Stück ohne
               Dich. Was macht dein Roman\pwindex{Schnitzler, Arthur 15.05.1862 – 21.10.1931@\textsc{Schnitzler, Arthur} (15.05.1862 – 21.10.1931), \emph{Schriftsteller, Mediziner}!Sterben. Novelle1894-10-01 – 1894-12-01@\strich\emph{Sterben. Novelle} {[}1894-10-01 – 1894-12-01{]}|pwv}?
               Brinſgt Du ihn nirgends an? Sende mir auch, wenn möglich, ein oder zwei Exemplare \textsc{Anatol\pwindex{Schnitzler, Arthur 15.05.1862 – 21.10.1931@\textsc{Schnitzler, Arthur} (15.05.1862 – 21.10.1931), \emph{Schriftsteller, Mediziner}!Anatol1892-10-29@\strich\emph{Anatol} {[}1892-10-29{]}|pw}} zu Progaganda-Zwecken. In Paris\oindex{Paris@\textbf{Paris}|pw} bekommt man
               nämlich nie ein Buch wieder, wenn man es wegborgt. Ich hoffe doch {\pb}noch etwas für Dich hier durchzuſetzen. Die Übergabe
               Deiner \label{K_L02724-7v}\edtext{Novellen}{\lemma{\textnormal{\emph{Novellen}}}\Cendnote{\textnormal{Von Schnitzler\pwindex{Schnitzler, Arthur 15.05.1862 – 21.10.1931@\textsc{Schnitzler, Arthur} (15.05.1862 – 21.10.1931), \emph{Schriftsteller, Mediziner}|pwk} existierte zu dieser Zeit keine Buchausgabe seiner Novellen.
                  Welche hier für die Vermittlung vorgesehen waren, lässt sich nicht
                  bestimmen.}}}\label{K_L02724-7h} an eine \label{K_L02724-8v}\edtext{Mitarbeiterin\pwindex{?? [Mitarbeiterin von La Vie Parisienne] @\textsc{?? [Mitarbeiterin von La Vie Parisienne]}|pwv}}{\lemma{\textnormal{\emph{Mitarbeiterin}}}\Cendnote{\textnormal{nicht identifiziert}}}\label{K_L02724-8h} der \textsc{Vie Parisienne\orgindex{Vie Parisienne@La Vie Parisienne|pw}} habe ich doch nicht in’s Werk ſetzen wollen. Gewiſſe \label{K_L02724-9v}\edtext{Erfahrungen der letzten Zeit haben mich gelehrt, daß
               möglicher Weiſe Deine Novelle Aufnahme gefunden hätte, aber nicht unter Deinem
                  Namen}{\lemma{\textnormal{\emph{Erfahrungen … Namen}}}\Cendnote{\textnormal{Auf welchen Plagiatsvorwurf Goldmann\pwindex{Goldmann, Paul 31.01.1865 – 25.09.1935@\textsc{Goldmann, Paul} (31.01.1865 – 25.09.1935), \emph{Schriftsteller, Journalist}|pwk} anspielt, ist unklar.}}}\label{K_L02724-9h}, – Du
               verſtehſt? \pend
           \pstart
           Schreib’ mir auch, \label{K_L02724-10v}\edtext{was mit \textsc{Bahr\pwindex{Bahr, Hermann 19.07.1863 – 15.01.1934@\textsc{Bahr, Hermann} (19.07.1863 – 15.01.1934), \emph{Schriftsteller, Kritiker}|pw}} vorgegangen iſt? Warum der Austritt aus der »Deutſchen Ztg\orgindex{Deutsche Zeitung@Deutsche Zeitung|pw}}{\lemma{\textnormal{\emph{was … Ztg}}}\Cendnote{\textnormal{Am 21. 12. 1893 stand in
                  der \emph{Deutschen Zeitung}\pwindex{?? Werk@Nicht ermittelte Verfasserinnen und Verfasser!Deutsche Zeitung1871 – 1907@\emph{Deutsche Zeitung} {[}1871 – 1907{]}|pwk}, dass Bahr\pwindex{Bahr, Hermann 19.07.1863 – 15.01.1934@\textsc{Bahr, Hermann} (19.07.1863 – 15.01.1934), \emph{Schriftsteller, Kritiker}|pwk} die Redaktion des Blattes\orgindex{Deutsche Zeitung@Deutsche Zeitung|pwkv} verlassen habe (Nr. 7898, S. 5).
                  Offizielle Begründung gab es keine. Bahr\pwindex{Bahr, Hermann 19.07.1863 – 15.01.1934@\textsc{Bahr, Hermann} (19.07.1863 – 15.01.1934), \emph{Schriftsteller, Kritiker}|pwk}
                  betreute seit September 1892 die Theaterkritik und kündigte, nachdem
                  zwei Mal in Kritiken von ihm eingegriffen worden war: einmal um ein Lob, einmal um
                  eine kritische Äußerung zu streichen.}}}\label{K_L02724-10h}«? Wird das Blatt\orgindex{Deutsche Zeitung@Deutsche Zeitung|pwv} eingehen?\pend
           \pstart
           Fröhliche Feiertage\textcolor{gray}{;} mein lieber Freund, und nochmals Dank Dir und
               den Andern\pwindex{Hofmannsthal, Hugo von 1874-02-01 – 1929-07-15@\textsc{Hofmannsthal, Hugo von} (1874-02-01 – 1929-07-15), \emph{Schriftsteller}|pwuv}\pwindex{Beer-Hofmann, Richard 1866-07-11 – 1945-09-26@\textsc{Beer-Hofmann, Richard} (1866-07-11 – 1945-09-26), \emph{Schriftsteller}|pwuv}\pwindex{Salten, Felix 06.09.1869 – 08.10.1945@\textsc{Salten, Felix} (06.09.1869 – 08.10.1945), \emph{Schriftsteller, Journalist}|pwuv}\pwindex{Schwarzkopf, Gustav 07.11.1853 – 13.11.1939@\textsc{Schwarzkopf, Gustav} (07.11.1853 – 13.11.1939), \emph{Schriftsteller}|pwuv} und viele treue Grüße an
               Euch Alle.\pend
           \pstart Dein \spacefill\mbox{Paul Goldm}\pend{}
         
         \endnumbering\mylabel{h}\end{ledgroupsized}  \newcommand{\dateiname}{L02724}\newcommand{\titel}{Paul Goldmann an Arthur Schnitzler, 23. 12. [1893]}\newcommand{\editorInnen}{Martin Anton Müller und Laura Untner}%% latex-leseansicht-abspann.tex
%% Abspann für die Leseansicht.
%% Der Schalter \ifkorrekturansicht ist bereits durch den Vorspann gesetzt.

%% latex-abspann.tex
%% Gemeinsamer Abspann für Korrekturansicht und Leseansicht.
%% Setzt den Schalter \ifkorrekturansicht voraus (gesetzt in den
%% einbindenden Dateien latex-korrekturansicht-abspann.tex bzw.
%% latex-leseansicht-abspann.tex).
%% ---------------------------------------------------------------

\normalsize

% Das esempio-Environment wird nur in der Leseansicht benötigt
\ifkorrekturansicht\else
\newenvironment{esempio}[3]%
{
    \vspace{1.5ex}
    \rlap{\underline{#1}}
    \par
    \setlength{\parindent}{0cm}
    \nopagebreak
    \leftskip=#2cm
    \rightskip=#3cm
}
{
    \par
}
\fi

\doendnotes{C}
\bigskip
\vfill

\clearpage

\footnotesize

\ifkorrekturansicht
  \lohead{\textsc{register}}
\fi

% theindex-Environment neu definieren ohne reledmac
\makeatletter
\renewenvironment{theindex}{%
  \ifkorrekturansicht
    \section*{\indexname}%
  \else
    \subsubsection*{Index der erwähnten Entitäten}%
  \fi
  \setlength{\parindent}{0pt}%
  \setlength{\parskip}{0pt plus 0.3pt}%
  \let\item\@idxitem
}{%
  \ifkorrekturansicht\clearpage\fi
}
\makeatother

\IfFileExists{\jobname-pw.ind}{\input{\jobname-pw.ind}}{}

% Quellenangabe nur in der Leseansicht
\ifkorrekturansicht\else
% Fallback-Definitionen, falls die .tex-Datei \titel etc. nicht gesetzt hat
\providecommand{\titel}{}
\providecommand{\editorInnen}{}
\providecommand{\dateiname}{\jobname}

\vspace{3cm}

\vfill

\footnotesize
\textsc{Quelle}: \titel. Herausgegeben von {\editorInnen}. In: \emph{Arthur Schnitzler: Briefwechsel mit Autorinnen und Autoren}.
 Digitale Edition, https://schnitzler-briefe.acdh.oeaw.ac.at/{\dateiname}.html (Stand \today)
\fi

\end{document}


      