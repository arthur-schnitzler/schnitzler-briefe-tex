%% latex-leseansicht-vorspann.tex
%% Vorspann für die Leseansicht.
%% Lädt die gemeinsame Datei latex-vorspann.tex mit nicht gesetztem Schalter.

\newif\ifkorrekturansicht
\korrekturansichtfalse

\input{../tex-inputs/latex-vorspann}


\section[Paul Goldmann an Arthur Schnitzler, 23. 12. {[}1893{]}]{L02724 Paul Goldmann an Arthur Schnitzler, 23. 12. [1893]}
\nopagebreak\mylabel{L02724v}
\rehead{ }\normalsize\beginnumbering\briefempfaengerindex{Schnitzler, Arthur@\textsc{Schnitzler, Arthur}!zzzGoldmann, Paul@\emph{von Paul Goldmann}!1893-12-231@{23. 12. [1893]}|(be}
\toendnotes[C]{\smallbreak\pagebreak[2]}
\correspDesc{Versand  durch Paul Goldmann am 23. 12. [1893] in Paris
\newline{}Erhalt  durch Arthur Schnitzler im Zeitraum [24. 12. 1893 – 28. 12. 1893?] in Wien}\toendnotes[C]{\smallbreak}
\Standort{DLA, A:Schnitzler, HS.NZ85.1.3163.}
\physDesc{Brief, 1 Blatt, 4 Seiten, 2062 Zeichen
\newline{}Handschrift: schwarze Tinte, deutsche Kurrent
\newline{}Schnitzler: 1) mit Bleistift das Jahr »93« vermerkt  2) mit rotem Buntstift zwei Unterstreichungen}\toendnotes[C]{\smallbreak}
\pstart
           \raggedleft{}{\pb}\textsc{Paris\oindex{Paris@\textbf{Paris}, \emph{Hauptstadt}|pw}}, 23. December.\pend
           
\pstart\center{}Mein lieber Freund!\pend\vspace{0.5em}
\pstart
           Dein letzter Brief und die{ }ſich daran{ }ſchließenden Zeilen der \label{K_L02724-1v}\edtext{Freunde\pwindex{Hofmannsthal, Hugo von 1.\,2.\,1874 Wien – 15.\,7.\,1929 Rodaun@\textsc{Hofmannsthal, Hugo von} (1.\,2.\,1874 Wien – 15.\,7.\,1929 Rodaun), \emph{Schriftsteller}|pwuv}\pwindex{Beer-Hofmann, Richard 11.\,7.\,1866 Wien – 26.\,9.\,1945 New York City@\textsc{Beer-Hofmann, Richard} (11.\,7.\,1866 Wien – 26.\,9.\,1945 New York City), \emph{Schriftsteller}|pwuv}\pwindex{Salten, Felix 6.\,9.\,1869 Budapest – 8.\,10.\,1945 Zürich@\textsc{Salten, Felix} (6.\,9.\,1869 Budapest – 8.\,10.\,1945 Zürich), \emph{Schriftsteller, Journalist, Chefredakteur}|pwuv}\pwindex{Schwarzkopf, Gustav 7.\,11.\,1853 Wien – 13.\,11.\,1939 ebd.@\textsc{Schwarzkopf, Gustav} (7.\,11.\,1853 Wien – 13.\,11.\,1939 ebd.), \emph{Schriftsteller}|pwuv}}{\lemma{\textnormal{\emph{Freunde}}}\Cendnote{\textnormal{Schnitzlers Brief könnte am 10. 12. 1893 abgefasst
                  worden sein, als er die mit Goldmann\pwindex{Goldmann, Paul 31.\,1.\,1865 Breslau – 25.\,9.\,1935 Wien@\textsc{Goldmann, Paul} (31.\,1.\,1865 Breslau – 25.\,9.\,1935 Wien), \emph{Schriftsteller, Journalist}|pwk}
                  bekannten Freunde Hugo von Hofmannsthal\pwindex{Hofmannsthal, Hugo von 1.\,2.\,1874 Wien – 15.\,7.\,1929 Rodaun@\textsc{Hofmannsthal, Hugo von} (1.\,2.\,1874 Wien – 15.\,7.\,1929 Rodaun), \emph{Schriftsteller}|pwk},
                     Richard Beer-Hofmann\pwindex{Beer-Hofmann, Richard 11.\,7.\,1866 Wien – 26.\,9.\,1945 New York City@\textsc{Beer-Hofmann, Richard} (11.\,7.\,1866 Wien – 26.\,9.\,1945 New York City), \emph{Schriftsteller}|pwk}, Felix Salten\pwindex{Salten, Felix 6.\,9.\,1869 Budapest – 8.\,10.\,1945 Zürich@\textsc{Salten, Felix} (6.\,9.\,1869 Budapest – 8.\,10.\,1945 Zürich), \emph{Schriftsteller, Journalist, Chefredakteur}|pwk} und Gustav
                     Schwarzkopf\pwindex{Schwarzkopf, Gustav 7.\,11.\,1853 Wien – 13.\,11.\,1939 ebd.@\textsc{Schwarzkopf, Gustav} (7.\,11.\,1853 Wien – 13.\,11.\,1939 ebd.), \emph{Schriftsteller}|pwk} zusammengetroffen war.}}}\label{K_L02724-1} haben mir eine unendliche Freude bereitet. Mir{ }ſind die Thränen in die Augen gekommen, als ich all’ das las. Und ich war einen
               ganzen Tag lang glücklich,{ }ſo viel Freundſchaft und Treue verdient zu haben. Gern
               hätte ich Dir, dem lieben Anſtifter der Freudengabe, und allen Betheiligten\pwindex{Hofmannsthal, Hugo von 1.\,2.\,1874 Wien – 15.\,7.\,1929 Rodaun@\textsc{Hofmannsthal, Hugo von} (1.\,2.\,1874 Wien – 15.\,7.\,1929 Rodaun), \emph{Schriftsteller}|pwuv}\pwindex{Beer-Hofmann, Richard 11.\,7.\,1866 Wien – 26.\,9.\,1945 New York City@\textsc{Beer-Hofmann, Richard} (11.\,7.\,1866 Wien – 26.\,9.\,1945 New York City), \emph{Schriftsteller}|pwuv}\pwindex{Salten, Felix 6.\,9.\,1869 Budapest – 8.\,10.\,1945 Zürich@\textsc{Salten, Felix} (6.\,9.\,1869 Budapest – 8.\,10.\,1945 Zürich), \emph{Schriftsteller, Journalist, Chefredakteur}|pwuv}\pwindex{Schwarzkopf, Gustav 7.\,11.\,1853 Wien – 13.\,11.\,1939 ebd.@\textsc{Schwarzkopf, Gustav} (7.\,11.\,1853 Wien – 13.\,11.\,1939 ebd.), \emph{Schriftsteller}|pwuv}{ }ſofort gedankt. Da kam die \label{K_L02724-2v}\edtext{Bombe in der Kammer\orgindex{Französische Abgeordnetenkammer@Französische Abgeordnetenkammer|pw}}{\lemma{\textnormal{\emph{Bombe in der Kammer}}}\Cendnote{\textnormal{Am 9. 12. 1893 hatte der
                  Anarchist Auguste Vaillant\pwindex{Vaillant, Auguste 27.\,12.\,1861 Charleville-Mézières – 5.\,2.\,1894 Paris@\textsc{Vaillant, Auguste} (27.\,12.\,1861 Charleville-Mézières – 5.\,2.\,1894 Paris), \emph{Anarchist, Attentäter}|pwk} ein
                  Bombenattentat auf die \emph{Französische
                     Nationalversammlung}\orgindex{Französische Nationalversammlung@Französische Nationalversammlung|pwk} verübt, bei dem um die 50 Personen verletzt wurden.}}}\label{K_L02724-2}
               und{ }ſonſt Allerlei und warf mich weit ab von Euch\pwindex{Hofmannsthal, Hugo von 1.\,2.\,1874 Wien – 15.\,7.\,1929 Rodaun@\textsc{Hofmannsthal, Hugo von} (1.\,2.\,1874 Wien – 15.\,7.\,1929 Rodaun), \emph{Schriftsteller}|pwuv}\pwindex{Beer-Hofmann, Richard 11.\,7.\,1866 Wien – 26.\,9.\,1945 New York City@\textsc{Beer-Hofmann, Richard} (11.\,7.\,1866 Wien – 26.\,9.\,1945 New York City), \emph{Schriftsteller}|pwuv}\pwindex{Salten, Felix 6.\,9.\,1869 Budapest – 8.\,10.\,1945 Zürich@\textsc{Salten, Felix} (6.\,9.\,1869 Budapest – 8.\,10.\,1945 Zürich), \emph{Schriftsteller, Journalist, Chefredakteur}|pwuv}\pwindex{Schwarzkopf, Gustav 7.\,11.\,1853 Wien – 13.\,11.\,1939 ebd.@\textsc{Schwarzkopf, Gustav} (7.\,11.\,1853 Wien – 13.\,11.\,1939 ebd.), \emph{Schriftsteller}|pwuv} und all’ den
               frohen Gedanken. {\pb}Inzwiſchen kam auch Dein liebes
                  \label{K_L02724-3v}\edtext{Bild\pwindex{Pietzner, Carl 9.\,4.\,1853 Wriezen – 25.\,11.\,1927 Wien@\textsc{Pietzner, Carl} (9.\,4.\,1853 Wriezen – 25.\,11.\,1927 Wien), \emph{Fotograf}!Arthur Schnitzler (1893)@\strich\emph{Arthur Schnitzler (1893)}|pwv}}{\lemma{\textnormal{\emph{Bild}}}\Cendnote{\textnormal{Wohl das von Carl Pietzner\pwindex{Pietzner, Carl 9.\,4.\,1853 Wriezen – 25.\,11.\,1927 Wien@\textsc{Pietzner, Carl} (9.\,4.\,1853 Wriezen – 25.\,11.\,1927 Wien), \emph{Fotograf}|pwk} erstellte Porträtfoto\pwindex{Pietzner, Carl 9.\,4.\,1853 Wriezen – 25.\,11.\,1927 Wien@\textsc{Pietzner, Carl} (9.\,4.\,1853 Wriezen – 25.\,11.\,1927 Wien), \emph{Fotograf}!Arthur Schnitzler (1893)@\strich\emph{Arthur Schnitzler (1893)}|pwkv} von Schnitzler, vgl. XXXX Auszeichnungsfehler: Dokument L00285 nicht gefunden.}}}\label{K_L02724-3}. Dank, innigen Dank für die Sendung. Ich habe es auf meinem
               Schreibtiſch aufgeſtellt und tauſche mit Dir manch’ einen Blick und verſinke in
               manch’ eine Träumerei während irgend eines politiſchen Artikels. Es iſt eine
               vorzügliche Aufnahme\pwindex{Pietzner, Carl 9.\,4.\,1853 Wriezen – 25.\,11.\,1927 Wien@\textsc{Pietzner, Carl} (9.\,4.\,1853 Wriezen – 25.\,11.\,1927 Wien), \emph{Fotograf}!Arthur Schnitzler (1893)@\strich\emph{Arthur Schnitzler (1893)}|pwv} –
               wenngleich Du freilich in Wirklichkeit nie{ }ſo hübſch geweſen. Auch zeige ich Dich
               Allen, die mich beſuchen kommen, und Du haſt viel Erfolg. Neulich war \textsc{Jean Thorel\pwindex{Thorel, Jean 11.\,9.\,1859 Éragny – 20.\,8.\,1916 Enghien-les-Bains@\textsc{Thorel, Jean} (11.\,9.\,1859 Éragny – 20.\,8.\,1916 Enghien-les-Bains), \emph{Übersetzer, Dramatiker}|pw}} bei mir und{ }ſagte: \label{K_L02724-4v}\edtext{»\textsc{\begin{otherlanguage}{french}Je jurerais, que c’est un monsieur, qui écrit des
                     comédies.\end{otherlanguage}}«}{\lemma{\textnormal{\emph{»Je … comédies.«}}}\Cendnote{\textnormal{französisch: Ich könnte schwören,
                  dass das ein Herr ist, der
                  Lustspiele schreibt.}}}\label{K_L02724-4} Wenn Du jetzt {\pb}\strikeout{\textcolor{gray}{noc}} noch \label{K_L02724-5v}\edtext{keine Luſtſpiele{ }ſchreiben
                  willſt}{\lemma{\textnormal{\emph{keine … willst}}}\Cendnote{\textnormal{Vgl. XXXX Auszeichnungsfehler: Dokument L02723 nicht gefunden.
               }}}\label{K_L02724-5}{\dotsfour}!\pend
           
\pstart
           Bitte liebſter Freund,{ }ſchreib’ mir ein ausführlicheres Wort über Deine Pläne. Die
               Idee mit dem \label{K_L02724-6v}\edtext{ſüßen Wien\oindex{Wien@\textbf{Wien}, \emph{Verwaltungsgebiet}|pw}er Stück\pwindex{Schnitzler, Arthur 15.\,5.\,1862 Wien – 21.\,10.\,1931 ebd.@\textsc{Schnitzler, Arthur} (15.\,5.\,1862 Wien – 21.\,10.\,1931 ebd.), \emph{Schriftsteller, Mediziner}!Liebelei. Schauspiel in drei Akten@\strich\emph{Liebelei. Schauspiel in drei Akten}|pwv}}{\lemma{\textnormal{\emph{süßen Wiener Stück}}}\Cendnote{\textnormal{\emph{Liebelei}\pwindex{Schnitzler, Arthur 15.\,5.\,1862 Wien – 21.\,10.\,1931 ebd.@\textsc{Schnitzler, Arthur} (15.\,5.\,1862 Wien – 21.\,10.\,1931 ebd.), \emph{Schriftsteller, Mediziner}!Liebelei. Schauspiel in drei Akten@\strich\emph{Liebelei. Schauspiel in drei Akten}|pwk}, das unter dem Titel »Armes Mädl\pwindex{Schnitzler, Arthur 15.\,5.\,1862 Wien – 21.\,10.\,1931 ebd.@\textsc{Schnitzler, Arthur} (15.\,5.\,1862 Wien – 21.\,10.\,1931 ebd.), \emph{Schriftsteller, Mediziner}!Liebelei. Schauspiel in drei Akten@\strich\emph{Liebelei. Schauspiel in drei Akten}|pwkv}« als Volksstück
                  geplant war. Das »süß« dürfte sich auf das »süße Mädl« beziehen, das schon früher
                  in den Briefen Goldmanns\pwindex{Goldmann, Paul 31.\,1.\,1865 Breslau – 25.\,9.\,1935 Wien@\textsc{Goldmann, Paul} (31.\,1.\,1865 Breslau – 25.\,9.\,1935 Wien), \emph{Schriftsteller, Journalist}|pwk} Thema war (vgl. XXXX Auszeichnungsfehler: Dokument L02649 nicht gefunden). Die Popularisierung
                  des Begriffs wird häufig Schnitzler
                  zugeschrieben und der Erfolg von \emph{Liebelei}\pwindex{Schnitzler, Arthur 15.\,5.\,1862 Wien – 21.\,10.\,1931 ebd.@\textsc{Schnitzler, Arthur} (15.\,5.\,1862 Wien – 21.\,10.\,1931 ebd.), \emph{Schriftsteller, Mediziner}!Liebelei. Schauspiel in drei Akten@\strich\emph{Liebelei. Schauspiel in drei Akten}|pwk}
                  spielt dabei eine zentrale Rolle. Diese Briefstelle legt nahe, dass schon bei der
                  Konzeption von \emph{Liebelei}\pwindex{Schnitzler, Arthur 15.\,5.\,1862 Wien – 21.\,10.\,1931 ebd.@\textsc{Schnitzler, Arthur} (15.\,5.\,1862 Wien – 21.\,10.\,1931 ebd.), \emph{Schriftsteller, Mediziner}!Liebelei. Schauspiel in drei Akten@\strich\emph{Liebelei. Schauspiel in drei Akten}|pwk} das Vorhaben eine
                  zentrale Rolle spielte, den Typus »unkomplizierte Frau für eine sexuelle Beziehung
                  ohne längerfristige Bindung« auf die Bühne zu bringen.}}}\label{K_L02724-6} gefällt mir{ }ſehr.
               Das müßte Dir ganz ausnehmend liegen. Und{ }ſchreib’ vor allen Dingen ein Stück ohne
               Dich. Was macht dein Roman\pwindex{Schnitzler, Arthur 15.\,5.\,1862 Wien – 21.\,10.\,1931 ebd.@\textsc{Schnitzler, Arthur} (15.\,5.\,1862 Wien – 21.\,10.\,1931 ebd.), \emph{Schriftsteller, Mediziner}!Sterben. Novelle@\strich\emph{Sterben. Novelle}|pwv}?
               Brinſgt Du ihn nirgends an? Sende mir auch, wenn möglich, ein oder zwei Exemplare \textsc{Anatol\pwindex{Schnitzler, Arthur 15.\,5.\,1862 Wien – 21.\,10.\,1931 ebd.@\textsc{Schnitzler, Arthur} (15.\,5.\,1862 Wien – 21.\,10.\,1931 ebd.), \emph{Schriftsteller, Mediziner}!Anatol@\strich\emph{Anatol}|pw}} zu Progaganda-Zwecken. In Paris\oindex{Paris@\textbf{Paris}, \emph{Hauptstadt}|pw} bekommt man
               nämlich nie ein Buch wieder, wenn man es wegborgt. Ich hoffe doch {\pb}noch etwas für Dich hier durchzuſetzen. Die Übergabe
               Deiner \label{K_L02724-7v}\edtext{Novellen}{\lemma{\textnormal{\emph{Novellen}}}\Cendnote{\textnormal{Es existierte zu dieser Zeit keine Buchausgabe von Schnitzlers Novellen.
                  Welche hier für die Vermittlung vorgesehen waren, lässt sich nicht
                  bestimmen.}}}\label{K_L02724-7} an eine \label{K_L02724-8v}\edtext{Mitarbeiterin\pwindex{?? [Mitarbeiterin von La Vie Parisienne] @\textsc{?? [Mitarbeiterin von La Vie Parisienne]}|pwv}}{\lemma{\textnormal{\emph{Mitarbeiterin}}}\Cendnote{\textnormal{nicht identifiziert}}}\label{K_L02724-8} der \textsc{Vie Parisienne\orgindex{Vie Parisienne@La Vie Parisienne|pw}} habe ich doch nicht in’s Werk{ }ſetzen wollen. Gewiſſe \label{K_L02724-9v}\edtext{Erfahrungen der letzten Zeit haben mich gelehrt, daß
               möglicher Weiſe Deine Novelle Aufnahme gefunden hätte, aber nicht unter Deinem
                  Namen}{\lemma{\textnormal{\emph{Erfahrungen … Namen}}}\Cendnote{\textnormal{Auf welchen Plagiatsvorwurf Goldmann\pwindex{Goldmann, Paul 31.\,1.\,1865 Breslau – 25.\,9.\,1935 Wien@\textsc{Goldmann, Paul} (31.\,1.\,1865 Breslau – 25.\,9.\,1935 Wien), \emph{Schriftsteller, Journalist}|pwk} anspielt, ist unklar.}}}\label{K_L02724-9}, – Du
               verſtehſt?\pend
           
\pstart
           Schreib’ mir auch, \label{K_L02724-10v}\edtext{was mit \textsc{Bahr\pwindex{Bahr, Hermann 19.\,7.\,1863 Linz – 15.\,1.\,1934 München@\textsc{Bahr, Hermann} (19.\,7.\,1863 Linz – 15.\,1.\,1934 München), \emph{Schriftsteller, Kritiker}|pw}} vorgegangen iſt? Warum der Austritt aus der »Deutſchen Ztg\orgindex{Deutsche Zeitung@Deutsche Zeitung|pw}}{\lemma{\textnormal{\emph{was … Ztg}}}\Cendnote{\textnormal{Am 21. 12. 1893 stand in
                  der \emph{Deutschen Zeitung}\pwindex{Deutsche Zeitung@\emph{Deutsche Zeitung}|pwk}, dass Bahr\pwindex{Bahr, Hermann 19.\,7.\,1863 Linz – 15.\,1.\,1934 München@\textsc{Bahr, Hermann} (19.\,7.\,1863 Linz – 15.\,1.\,1934 München), \emph{Schriftsteller, Kritiker}|pwk} die Redaktion des Blattes\orgindex{Deutsche Zeitung@Deutsche Zeitung|pwkv} verlassen habe (Nr. 7898, S. 5).
                  Offizielle Begründung gab es keine. Bahr\pwindex{Bahr, Hermann 19.\,7.\,1863 Linz – 15.\,1.\,1934 München@\textsc{Bahr, Hermann} (19.\,7.\,1863 Linz – 15.\,1.\,1934 München), \emph{Schriftsteller, Kritiker}|pwk}
                  betreute seit September 1892 die Theaterkritik und kündigte, nachdem
                  zweimal in Kritiken von ihm eingegriffen worden war: einmal um ein Lob, einmal um
                  eine kritische Äußerung zu streichen.}}}\label{K_L02724-10}«? Wird das Blatt\orgindex{Deutsche Zeitung@Deutsche Zeitung|pwv} eingehen?\pend
           
\pstart
           Fröhliche Feiertage\textcolor{gray}{;} mein lieber Freund, und nochmals Dank Dir und
               den Andern\pwindex{Hofmannsthal, Hugo von 1.\,2.\,1874 Wien – 15.\,7.\,1929 Rodaun@\textsc{Hofmannsthal, Hugo von} (1.\,2.\,1874 Wien – 15.\,7.\,1929 Rodaun), \emph{Schriftsteller}|pwuv}\pwindex{Beer-Hofmann, Richard 11.\,7.\,1866 Wien – 26.\,9.\,1945 New York City@\textsc{Beer-Hofmann, Richard} (11.\,7.\,1866 Wien – 26.\,9.\,1945 New York City), \emph{Schriftsteller}|pwuv}\pwindex{Salten, Felix 6.\,9.\,1869 Budapest – 8.\,10.\,1945 Zürich@\textsc{Salten, Felix} (6.\,9.\,1869 Budapest – 8.\,10.\,1945 Zürich), \emph{Schriftsteller, Journalist, Chefredakteur}|pwuv}\pwindex{Schwarzkopf, Gustav 7.\,11.\,1853 Wien – 13.\,11.\,1939 ebd.@\textsc{Schwarzkopf, Gustav} (7.\,11.\,1853 Wien – 13.\,11.\,1939 ebd.), \emph{Schriftsteller}|pwuv} und viele treue Grüße an
               Euch Alle.\pend
           \pstart Dein \spacefill\mbox{Paul Goldm}\pend{}\selectlanguage{ngerman}\endnumbering\briefempfaengerindex{Schnitzler, Arthur@\textsc{Schnitzler, Arthur}!zzzGoldmann, Paul@\emph{von Paul Goldmann}!1893-12-231@{23. 12. [1893]}|)be}\mylabel{L02724h}  \newcommand{\dateiname}{L02724}\newcommand{\titel}{Paul Goldmann an Arthur Schnitzler, 23. 12. [1893]}\newcommand{\editorInnen}{Martin Anton Müller und Laura Untner}%% latex-leseansicht-abspann.tex
%% Abspann für die Leseansicht.
%% Der Schalter \ifkorrekturansicht ist bereits durch den Vorspann gesetzt.

%% latex-abspann.tex
%% Gemeinsamer Abspann für Korrekturansicht und Leseansicht.
%% Setzt den Schalter \ifkorrekturansicht voraus (gesetzt in den
%% einbindenden Dateien latex-korrekturansicht-abspann.tex bzw.
%% latex-leseansicht-abspann.tex).
%% ---------------------------------------------------------------

\normalsize

% Das esempio-Environment wird nur in der Leseansicht benötigt
\ifkorrekturansicht\else
\newenvironment{esempio}[3]%
{
    \vspace{1.5ex}
    \rlap{\underline{#1}}
    \par
    \setlength{\parindent}{0cm}
    \nopagebreak
    \leftskip=#2cm
    \rightskip=#3cm
}
{
    \par
}
\fi

\doendnotes{C}
\bigskip
\vfill

\clearpage

\footnotesize

\ifkorrekturansicht
  \lohead{\textsc{register}}
\fi

% theindex-Environment neu definieren ohne reledmac
\makeatletter
\renewenvironment{theindex}{%
  \ifkorrekturansicht
    \section*{\indexname}%
  \else
    \subsubsection*{Index der erwähnten Entitäten}%
  \fi
  \setlength{\parindent}{0pt}%
  \setlength{\parskip}{0pt plus 0.3pt}%
  \let\item\@idxitem
}{%
  \ifkorrekturansicht\clearpage\fi
}
\makeatother

\IfFileExists{\jobname-pw.ind}{\input{\jobname-pw.ind}}{}

% Quellenangabe nur in der Leseansicht
\ifkorrekturansicht\else
% Fallback-Definitionen, falls die .tex-Datei \titel etc. nicht gesetzt hat
\providecommand{\titel}{}
\providecommand{\editorInnen}{}
\providecommand{\dateiname}{\jobname}

\vspace{3cm}

\vfill

\footnotesize
\textsc{Quelle}: \titel. Herausgegeben von {\editorInnen}. In: \emph{Arthur Schnitzler: Briefwechsel mit Autorinnen und Autoren}.
 Digitale Edition, https://schnitzler-briefe.acdh.oeaw.ac.at/{\dateiname}.html (Stand \today)
\fi

\end{document}


