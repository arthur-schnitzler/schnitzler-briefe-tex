%% latex-korrekturansicht-vorspann.tex
%% Vorspann für die Korrekturansicht.
%% Lädt die gemeinsame Datei latex-vorspann.tex mit gesetztem Schalter.

\newif\ifkorrekturansicht
\korrekturansichttrue

\input{../tex-inputs/latex-vorspann}


\section[Hugo von Hofmannsthal an Arthur Schnitzler, 15. 6. 1907]{L01684 Hugo von Hofmannsthal an Arthur Schnitzler, 15. 6. 1907}
\nopagebreak\mylabel{L01684v}
\rehead{ }\normalsize\beginnumbering\briefempfaengerindex{Schnitzler, Arthur@\textsc{Schnitzler, Arthur}!zzzHofmannsthal, Hugo von@\emph{von Hugo von Hofmannsthal}!1907-06-151@{15. 6. 1907}|(be}
\toendnotes[C]{\smallbreak\pagebreak[2]}\Standort{CUL, Schnitzler, B 43.}
\physDesc{Bildpostkarte, 169 Zeichen
\newline{}Handschrift: schwarze Tinte, deutsche Kurrent
\newline{}Versand: Stempel: »\nobreak{}\oindex{Venedig@\textbf{Venedig}, \emph{P.PPLA}|pwk}Sezioni Riunite Venezia, 15 6–07, 6S\nobreak{}«.  
\newline{}Schnitzler: mit Bleistift datiert: »Jun 907« 
\newline{}Ordnung: 1) mit Bleistift von unbekannter Hand nummeriert:
                                    »279«  2) mit Bleistift von unbekannter Hand nummeriert:
                                    »280«}
\buchAbdrucke{\weitereDrucke{Hugo von Hofmannsthal, Arthur Schnitzler: \emph{Briefwechsel}. Frankfurt am Main: \emph{S. Fischer} 1964, S. 229.} }\toendnotes[C]{\smallbreak}\pstart{}{\pb}\textsc{Herrn D\textsuperscript{r} Arthur Schnitzler}\pend{}\pstart{}\textsc{Wien}\oindex{Wien@\textbf{Wien}, \emph{A.ADM2}|pw}\pend{}\pstart{}\textsc{XVII. Spöttelgasse 7}.\oindex{Edmund-Weiss-Gasse 7@\textbf{Edmund-Weiß-Gasse 7}, \emph{Wohngebäude (K.WHS)}|pw}\pend{}{\bigskip}
\pstart
           \noindent{}\centering{}{\pb}\textcolor{gray}{\textbf{Duino\oindex{Duino@\textbf{Duino}, \emph{P.PPL}|pw} bei Sistiana\oindex{Sistiana@\textbf{Sistiana}, \emph{P.PPL}|pw}}}. \pend
           \vspace{1em}
\pstart
           \noindent{}{\pb}Hier haben wir ein paar ſehr
               angenehme Tage verbracht. Am Lido\oindex{Lido@\textbf{Lido}, \emph{P.PPL}|pw} hoffe ich
               manches zu arbeiten.\pend
           
\pstart
           Herzlich{\\[\baselineskip]}\spacefill\mbox{Hugo.}\pend
           \leftskip=0em{}
\pstart
           \noindent{}\label{T_L01684-1v}\edtext{+ unſere Zi{\geminationm}er}{\lemma{\textnormal{\emph{+ unſere Zimmer}}}\Cendnote{\textnormal{Verweis auf eine ebenfalls mit »+« markierte Stelle auf dem
                     Kartenmotiv.}}}\label{T_L01684-1}\pend
           \selectlanguage{ngerman}\endnumbering\briefempfaengerindex{Schnitzler, Arthur@\textsc{Schnitzler, Arthur}!zzzHofmannsthal, Hugo von@\emph{von Hugo von Hofmannsthal}!1907-06-151@{15. 6. 1907}|)be}\mylabel{L01684h}  \normalsize

\doendnotes{C}
\bigskip
\vfill

\clearpage

\footnotesize

\lohead{\textsc{register}}

% Definiere theindex-Environment komplett neu ohne reledmac
\makeatletter
\renewenvironment{theindex}{%
  \section*{\indexname}%
  \setlength{\parindent}{0pt}%
  \setlength{\parskip}{0pt plus 0.3pt}%
  \let\item\@idxitem
}{%
  \clearpage
}
\makeatother

\IfFileExists{\jobname-pw.ind}{\input{\jobname-pw.ind}}{}

\end{document}

      