%% latex-korrekturansicht-vorspann.tex
%% Vorspann für die Korrekturansicht.
%% Lädt die gemeinsame Datei latex-vorspann.tex mit gesetztem Schalter.

\newif\ifkorrekturansicht
\korrekturansichttrue

\input{../tex-inputs/latex-vorspann}


\section[Hugo von Hofmannsthal an Arthur Schnitzler, 30. 8. 1898]{L00841 Hugo von Hofmannsthal an Arthur Schnitzler, 30. 8. 1898}
\nopagebreak\mylabel{L00841v}
\rehead{ }\normalsize\beginnumbering\briefempfaengerindex{Schnitzler, Arthur@\textsc{Schnitzler, Arthur}!zzzHofmannsthal, Hugo von@\emph{von Hugo von Hofmannsthal}!1898-08-301@{30. 8. 1898}|(be}
\toendnotes[C]{\smallbreak\pagebreak[2]}\Standort{CUL, Schnitzler, B 43.}
\physDesc{Postkarte, 288 Zeichen
\newline{}Handschrift: 1) Bleistift, deutsche Kurrent\hspace{1em}2) Bleistift, lateinische Kurrent (\noindent{}Adresse)\hspace{1em}
\newline{}Versand: 1) Stempel: »\nobreak{}\oindex{Lugano@\textbf{Lugano}, \emph{P.PPLA2}|pwk}Lugano, 30. VIII. 98, 8\nobreak{}«.   2) Stempel: »\nobreak{}\oindex{Bologna@\textbf{Bologna}, \emph{P.PPLA}|pwk}Bologna, 30. VIII. 98, 8\nobreak{}«.  3) Stempel: »\nobreak{}\oindex{Bologna@\textbf{Bologna}, \emph{P.PPLA}|pwk}{[}Bo{]}logna, {[}31.{]}  8. 1898, 8H\nobreak{}«. 
\newline{}Ordnung: 1) mit Bleistift von unbekannter Hand nummeriert: »\strikeout{134}«  2) mit Bleistift von unbekannter Hand nummeriert:
                                    »123«}
\buchAbdrucke{\weitereDrucke{Hugo von Hofmannsthal, Arthur Schnitzler: \emph{Briefwechsel}. Frankfurt am Main: \emph{S. Fischer} 1964, S. 111.} }\pstart{}{\pb}Herrn D\textsuperscript{r} Arthur Schnitzler\pend{}\pstart{}Italia\oindex{Italien@\textbf{Italien}, \emph{A.PCLI}|pw}\pend{}\pstart{}Bologna\oindex{Bologna@\textbf{Bologna}, \emph{P.PPLA}|pw}\pend{}\pstart{}ferma in posta\pend{}{\bigskip}\vspace{1em}
\pstart
           \raggedleft{}{\pb}Lugano\oindex{Hôtel du Parc@\textbf{Hôtel du Parc}, \emph{Hotel (K.HTL)}|pw}{ }30. XIII.\pend
           \vspace{0.5em}
\pstart
           lieber, ich lebe nun ganz ruhig und zufrieden, ſchreibe etwas Proſa,
               erwarte Richard\pwindex{Beer-Hofmann, Richard 1866-07-11 – 1945-09-26@\textsc{Beer-Hofmann, Richard} (1866-07-11 – 1945-09-26), \emph{Schriftsteller/Schriftstellerin}|pw} und genieße die nun ſehr
               ſchöngefärbte reine Luft.\pend
           
\pstart
           Mit Briefen oder Karten machen Sie mir eine große Freude,{\\}und hierher!\pend
           
\pstart
           Von Herzen Ihr{\\[\baselineskip]}\spacefill\mbox{Hugo.}\pend
           \leftskip=0em{}\selectlanguage{ngerman}\endnumbering\briefempfaengerindex{Schnitzler, Arthur@\textsc{Schnitzler, Arthur}!zzzHofmannsthal, Hugo von@\emph{von Hugo von Hofmannsthal}!1898-08-301@{30. 8. 1898}|)be}\mylabel{L00841h}  \normalsize

\doendnotes{C}
\bigskip
\vfill

\clearpage

\footnotesize

\lohead{\textsc{register}}

% Definiere theindex-Environment komplett neu ohne reledmac
\makeatletter
\renewenvironment{theindex}{%
  \section*{\indexname}%
  \setlength{\parindent}{0pt}%
  \setlength{\parskip}{0pt plus 0.3pt}%
  \let\item\@idxitem
}{%
  \clearpage
}
\makeatother

\IfFileExists{\jobname-pw.ind}{\input{\jobname-pw.ind}}{}

\end{document}

      