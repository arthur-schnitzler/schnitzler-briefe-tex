%% latex-leseansicht-vorspann.tex
%% Vorspann für die Leseansicht.
%% Lädt die gemeinsame Datei latex-vorspann.tex mit nicht gesetztem Schalter.

\newif\ifkorrekturansicht
\korrekturansichtfalse

\input{../tex-inputs/latex-vorspann}


\section[Arthur Schnitzler an Theodor Herzl, 1. 5. 1895]{L03929 Arthur Schnitzler an Theodor Herzl, 1. 5. 1895}
\nopagebreak\mylabel{L03929v}
\rehead{ }\normalsize\beginnumbering\briefempfaengerindex{Herzl, Theodor@\textsc{Herzl, Theodor}!zzzSchnitzler, Arthur@\emph{von Arthur Schnitzler}!1895-05-011@{1. 5. 1895}|(be}
\toendnotes[C]{\smallbreak\pagebreak[2]}
\correspDesc{Versand  durch Arthur Schnitzler am 1. 5. 1895 in Wien
\newline{}Erhalt  durch Theodor Herzl in Wien}\toendnotes[C]{\smallbreak}
\Standort{Jerusalem, Central Zionist Archives, H1:1925-14.}
\physDesc{,  Blätter,  Seiten
\newline{}Handschrift: , deutsche Kurrent}
\buchAbdrucke{\weitereDrucke{Arthur Schnitzler: \emph{Briefe 1875–1912}. Herausgegeben von Therese Nickl und Heinrich Schnitzler. Frankfurt am Main: \emph{S. Fischer} 1981, S. 256.} }\toendnotes[C]{\smallbreak}
\pstart{}{\pb}Lieber Freund,\pend\vspace{0.5em}
\pstart
           ich ſende Ihnen den \label{K_L03929-1v}\edtext{Brief}{\lemma{\textnormal{\emph{Brief}}}\Cendnote{\textnormal{Dass es sich um das Schreiben von Adam Müller-Guttenbrunn\pwindex{Müller-Guttenbrunn, Adam 22.\,10.\,1852 Zăbrani – 5.\,1.\,1923 Wien@\textsc{Müller-Guttenbrunn, Adam} (22.\,10.\,1852 Zăbrani – 5.\,1.\,1923 Wien), \emph{Schriftsteller, Theaterleiter, Beamter}|pwk} vom
                     9. 4. 1895 handelt, ergibt sich aus einer inhaltlichen
                     Übereinstimmung zwischen dem hier im Anhang wiedergegebenen Brief (»Ich
                        überlaſſe die ganze Sache dem Taktgefühl des Herr Verfaſſers.«) und Herzls\pwindex{Herzl, Theodor 2.\,5.\,1860 Budapest – 3.\,7.\,1904 Edlach@\textsc{Herzl, Theodor} (2.\,5.\,1860 Budapest – 3.\,7.\,1904 Edlach), \emph{Schriftsteller, Journalist}|pwk} Reaktion darauf vom XXXX Auszeichnungsfehler: Dokument L03860 nicht gefunden (»Müller G.\pwindex{Müller-Guttenbrunn, Adam 22.\,10.\,1852 Zăbrani – 5.\,1.\,1923 Wien@\textsc{Müller-Guttenbrunn, Adam} (22.\,10.\,1852 Zăbrani – 5.\,1.\,1923 Wien), \emph{Schriftsteller, Theaterleiter, Beamter}|pw} hat den richtigen Einfall
                           gehabt, als er an meinen Tact appellirte.«).}}}\label{K_L03929-1} des M. G.\pwindex{Müller-Guttenbrunn, Adam 22.\,10.\,1852 Zăbrani – 5.\,1.\,1923 Wien@\textsc{Müller-Guttenbrunn, Adam} (22.\,10.\,1852 Zăbrani – 5.\,1.\,1923 Wien), \emph{Schriftsteller, Theaterleiter, Beamter}|pw} ein
               und wäre der Anſicht, daß Sie \strikeout{Ihm} ihm auch denſelben
               vielleicht perſönlich antworteten. Glauben Sie nicht?– Daß \textsc{Blumenthal\pwindex{Blumenthal, Oskar 13.\,3.\,1852 Berlin – 24.\,4.\,1917 ebd.@\textsc{Blumenthal, Oskar} (13.\,3.\,1852 Berlin – 24.\,4.\,1917 ebd.), \emph{Schriftsteller, Journalist, Theaterleiter}|pw}} einfach auf die neuerliche Anfrage, ob man ihm das Stück\pwindex{Herzl, Theodor 2.\,5.\,1860 Budapest – 3.\,7.\,1904 Edlach@\textsc{Herzl, Theodor} (2.\,5.\,1860 Budapest – 3.\,7.\,1904 Edlach), \emph{Schriftsteller, Journalist}!neue Ghetto. Schauspiel in vier Acten@\strich\emph{Das neue Ghetto. Schauspiel in vier Acten}|pwv} noch einmal zuſenden ſolle, nicht
               geantwortet, wiſſen Sie ſchon – jetzt läßt dieſer unleid{\pb}liche Herr Fischer\pwindex{Fischer, Samuel 24.\,12.\,1859 Liptovský Mikuláš – 15.\,10.\,1934 Berlin@\textsc{Fischer, Samuel} (24.\,12.\,1859 Liptovský Mikuláš – 15.\,10.\,1934 Berlin), \emph{Verleger}|pw} weiter auf Antwort warten.
               Vertreter in der \textsc{Ghetto\pwindex{Herzl, Theodor 2.\,5.\,1860 Budapest – 3.\,7.\,1904 Edlach@\textsc{Herzl, Theodor} (2.\,5.\,1860 Budapest – 3.\,7.\,1904 Edlach), \emph{Schriftsteller, Journalist}!neue Ghetto. Schauspiel in vier Acten@\strich\emph{Das neue Ghetto. Schauspiel in vier Acten}|pw}}\textsc{affaire} iſt jetzt nicht mehr Herr \textsc{Schick\pwindex{Schik, Friedrich *~6.\,9.\,1857 Wien@\textsc{Schik, Friedrich} (*~6.\,9.\,1857 Wien), \emph{Notar, Journalist, Dramaturg}|pw}}, der im So{\geminationm}er oft abweſend iſt ſondern der Hof u. Gerichtsadvokat Dr. \textsc{Julius Baumgarten\pwindex{Baumgarten, Julius 26.\,5.\,1860 Wien – 28.\,8.\,1934 ebd.@\textsc{Baumgarten, Julius} (26.\,5.\,1860 Wien – 28.\,8.\,1934 ebd.), \emph{Anwalt}|pw}}, der natürlich auch von Ihrer Autorſchaft keine Ahnung und das \textsc{Mscrpt\pwindex{Herzl, Theodor 2.\,5.\,1860 Budapest – 3.\,7.\,1904 Edlach@\textsc{Herzl, Theodor} (2.\,5.\,1860 Budapest – 3.\,7.\,1904 Edlach), \emph{Schriftsteller, Journalist}!neue Ghetto. Schauspiel in vier Acten@\strich\emph{Das neue Ghetto. Schauspiel in vier Acten}|pw}}, da ich es perſönlich befördere, gar nicht geſehen hat. – Mir thuts {\pb}leid, daß ich morgen der Tabarin\pwindex{Herzl, Theodor 2.\,5.\,1860 Budapest – 3.\,7.\,1904 Edlach@\textsc{Herzl, Theodor} (2.\,5.\,1860 Budapest – 3.\,7.\,1904 Edlach), \emph{Schriftsteller, Journalist}!Tabarin. Schauspiel in einem Act. Frei nach Catulle Mendès@\strich\emph{Tabarin. Schauspiel in einem Act. Frei nach Catulle Mendès}|pw} Premiere\eventindex{Burgtheater@\textbf{Burgtheater}!Premiere von Tabarin und Verbotene Früchte, 2.5.1895@Premiere von Tabarin und Verbotene Früchte, 2.5.1895|pw} nicht beiwohnen
               kann,{ }ſie fällt gerade auf den 2. Mai den Todestag meines Vaters\pwindex{Schnitzler, Johann 10.\,4.\,1835 Nagykanizsa – 2.\,5.\,1893 Wien@\textsc{Schnitzler, Johann} (10.\,4.\,1835 Nagykanizsa – 2.\,5.\,1893 Wien), \emph{Laryngologe}|pwv}. Was aber Ihre
               Misſti{\geminationm}ung über dieſe verſpätete Première\eventindex{Burgtheater@\textbf{Burgtheater}!Premiere von Tabarin und Verbotene Früchte, 2.5.1895@Premiere von Tabarin und Verbotene Früchte, 2.5.1895|pwv} anbelangt, ſo wünſche ich herzlich, daß Sie nie ernſteren Grund
               haben ſollen, misgeſti{\geminationm}t ſein. Ich bin überzeugt, daſs der Ausfall Sie wieder in
               beſſere Laune bringen wird.– Ich ko{\geminationm} wohl {\pb}erſt in der
               nächſten Saiſon dran\eventindex{Burgtheater@\textbf{Burgtheater}!Uraufführung von Liebelei, Premiere von Rechte der Seele, 9.10.1895@Uraufführung von Liebelei, Premiere von Rechte der Seele, 9.10.1895|pwv}. Zu
               neuen Arbeiten bin ich durch einige wenige äußere und zahlreiche innere Umſtände gar
               nicht gekommen.\pend
           
\pstart
           Seien die vielmals herzlich gegrüßt{\\[\baselineskip]}Ihr ergebener \spacefill\mbox{ArthSch}\pend
           \leftskip=0em{}
\pstart
           1. Mai 95.\pend
           
\pstart
           \textsc{Duncker u Humblot\orgindex{Duncker und Humblot@Duncker {\kaufmannsund}  Humblot|pw}} iſt ausgezeichnet. Wann{ }ſoll das Buch\pwindex{Herzl, Theodor 2.\,5.\,1860 Budapest – 3.\,7.\,1904 Edlach@\textsc{Herzl, Theodor} (2.\,5.\,1860 Budapest – 3.\,7.\,1904 Edlach), \emph{Schriftsteller, Journalist}!Palais Bourbon. Bilder aus dem französischen Parlamentsleben@\strich\emph{Das Palais Bourbon. Bilder aus dem französischen Parlamentsleben}|pwv} heraus?\pend
           \selectlanguage{ngerman}\vspace{1em}{\vspace{1\baselineskip}}
\pstart
           \centering{}{\pb}\textcolor{gray}{\textbf{Raimund Theater\orgindex{Raimund-Theater@Raimund-Theater|pw}.}}\pend
           
\pstart
           \centering{}\textcolor{gray}{\textbf{Direction: A.
                        Müller-Guttenbrunn\pwindex{Müller-Guttenbrunn, Adam 22.\,10.\,1852 Zăbrani – 5.\,1.\,1923 Wien@\textsc{Müller-Guttenbrunn, Adam} (22.\,10.\,1852 Zăbrani – 5.\,1.\,1923 Wien), \emph{Schriftsteller, Theaterleiter, Beamter}|pw}.}}\pend
           
\pstart
           Hr D\textsuperscript{r} Arthur Schnitzler\pend
           
\pstart
           \textcolor{gray}{\textbf{Wien\oindex{Wien@\textbf{Wien}, \emph{Verwaltungsgebiet}|pw}, am}}{ }9. 4. \textcolor{gray}{\textbf{189}}5\pend
           
\pstart
           I. Frankgaſſe 1\oindex{Wien@\textbf{Wien}!IX., Alsergrund@\textbf{IX., Alsergrund}!Frankgasse 1@\textbf{Frankgasse 1}, \emph{Wohngebäude}|pw}\pend
           
\pstart{}Verehrter Herr Doctor!\pend\vspace{0.5em}
\pstart
           Ich betrachte es als ganz ſelbſtverſtändlich, daß ich nicht das Recht habe, das
               Pſeudonym \textsc{Albert Schnabel} zu lüften u. es wird \strikeout{es} alſo niemals Jemand von mir den Namen des Autors des
               »Ghetto\pwindex{Herzl, Theodor 2.\,5.\,1860 Budapest – 3.\,7.\,1904 Edlach@\textsc{Herzl, Theodor} (2.\,5.\,1860 Budapest – 3.\,7.\,1904 Edlach), \emph{Schriftsteller, Journalist}!neue Ghetto. Schauspiel in vier Acten@\strich\emph{Das neue Ghetto. Schauspiel in vier Acten}|pw}« erfahren.\pend
           
\pstart
           Ihr Anfrage bezüglich meines Briefes überraſcht mich. Derſelbe wurde in der Eile
               hingeworfen u. ich kann nur nicht denken, {\pb}daß deſſen
               Veröffentlichung dem Stücke\pwindex{Herzl, Theodor 2.\,5.\,1860 Budapest – 3.\,7.\,1904 Edlach@\textsc{Herzl, Theodor} (2.\,5.\,1860 Budapest – 3.\,7.\,1904 Edlach), \emph{Schriftsteller, Journalist}!neue Ghetto. Schauspiel in vier Acten@\strich\emph{Das neue Ghetto. Schauspiel in vier Acten}|pwv}
               nützen oder deſſen Schickſal auch nur beeinfluſſen kann. Jedenfalls hätte ich meine
               Ablehnung eingehender u. literariſcher begründet, we{\geminationn} ich an eine Veröffentlichung
               des Briefes gedacht hätte. Nach meinem Gefühl könnte mein letzter Bf. nur benützt
               werden, wenn auch der \uline{erſte} benützt wurde, denn in
               meinem \strikeout{Abſch.} Ablehnungsſchreiben iſt mein eigenes
               Urtheil über das Stück\pwindex{Herzl, Theodor 2.\,5.\,1860 Budapest – 3.\,7.\,1904 Edlach@\textsc{Herzl, Theodor} (2.\,5.\,1860 Budapest – 3.\,7.\,1904 Edlach), \emph{Schriftsteller, Journalist}!neue Ghetto. Schauspiel in vier Acten@\strich\emph{Das neue Ghetto. Schauspiel in vier Acten}|pwv} nicht
               enthalten. Ich überlaſſe die ganze Sache dem Taktgefühl des Herr Verfaſſers. Er wird
               am beſen wiſſen, was er, thun darf. Was ich geſchrieben vertrete ich unter allen
               Umſtänden.\pend
           
\pstart
           Ihr Hochachtungsvoll{\\[\baselineskip]}ergebner \spacefill\mbox{\textcolor{gray}{MGuttenbrunn}}\pend
           \leftskip=0em{}\selectlanguage{ngerman}\endnumbering\briefempfaengerindex{Herzl, Theodor@\textsc{Herzl, Theodor}!zzzSchnitzler, Arthur@\emph{von Arthur Schnitzler}!1895-05-011@{1. 5. 1895}|)be}\mylabel{L03929h}
\begin{anhang}
\end{anhang}\newcommand{\dateiname}{L03929}\newcommand{\titel}{Arthur Schnitzler an Theodor Herzl, 1. 5. 1895}\newcommand{\editorInnen}{Herausgegeben von Jahnke, SelmaMüller, Martin Anton}%% latex-leseansicht-abspann.tex
%% Abspann für die Leseansicht.
%% Der Schalter \ifkorrekturansicht ist bereits durch den Vorspann gesetzt.

%% latex-abspann.tex
%% Gemeinsamer Abspann für Korrekturansicht und Leseansicht.
%% Setzt den Schalter \ifkorrekturansicht voraus (gesetzt in den
%% einbindenden Dateien latex-korrekturansicht-abspann.tex bzw.
%% latex-leseansicht-abspann.tex).
%% ---------------------------------------------------------------

\normalsize

% Das esempio-Environment wird nur in der Leseansicht benötigt
\ifkorrekturansicht\else
\newenvironment{esempio}[3]%
{
    \vspace{1.5ex}
    \rlap{\underline{#1}}
    \par
    \setlength{\parindent}{0cm}
    \nopagebreak
    \leftskip=#2cm
    \rightskip=#3cm
}
{
    \par
}
\fi

\doendnotes{C}
\bigskip
\vfill

\clearpage

\footnotesize

\ifkorrekturansicht
  \lohead{\textsc{register}}
\fi

% theindex-Environment neu definieren ohne reledmac
\makeatletter
\renewenvironment{theindex}{%
  \ifkorrekturansicht
    \section*{\indexname}%
  \else
    \subsubsection*{Index der erwähnten Entitäten}%
  \fi
  \setlength{\parindent}{0pt}%
  \setlength{\parskip}{0pt plus 0.3pt}%
  \let\item\@idxitem
}{%
  \ifkorrekturansicht\clearpage\fi
}
\makeatother

\IfFileExists{\jobname-pw.ind}{\input{\jobname-pw.ind}}{}

% Quellenangabe nur in der Leseansicht
\ifkorrekturansicht\else
% Fallback-Definitionen, falls die .tex-Datei \titel etc. nicht gesetzt hat
\providecommand{\titel}{}
\providecommand{\editorInnen}{}
\providecommand{\dateiname}{\jobname}

\vspace{3cm}

\vfill

\footnotesize
\textsc{Quelle}: \titel. Herausgegeben von {\editorInnen}. In: \emph{Arthur Schnitzler: Briefwechsel mit Autorinnen und Autoren}.
 Digitale Edition, https://schnitzler-briefe.acdh.oeaw.ac.at/{\dateiname}.html (Stand \today)
\fi

\end{document}


