%% latex-korrekturansicht-vorspann.tex
%% Vorspann für die Korrekturansicht.
%% Lädt die gemeinsame Datei latex-vorspann.tex mit gesetztem Schalter.

\newif\ifkorrekturansicht
\korrekturansichttrue

\input{../tex-inputs/latex-vorspann}


\section[Theodor Herzl an Arthur Schnitzler, 19. 5. 1893]{L03829 Theodor Herzl an Arthur Schnitzler, 19. 5. 1893}
\nopagebreak\mylabel{L03829v}
\rehead{ }\normalsize\beginnumbering\briefempfaengerindex{Schnitzler, Arthur@\textsc{Schnitzler, Arthur}!zzzHerzl, Theodor@\emph{von Theodor Herzl}!1893-05-191@{19. 5. 1893}|(be}
\toendnotes[C]{\smallbreak\pagebreak[2]}\Standort{CUL, Schnitzler, B 39.}
\physDesc{Brief, 1 Blatt, 2 Seiten, 908 Zeichen
\newline{}Handschrift: schwarze Tinte, lateinische Kurrent
\newline{}Ordnung: mit Bleistift von unbekannter Hand nummeriert: »9« }\toendnotes[C]{\smallbreak}
\pstart
           {\pb}\textcolor{gray}{\textbf{NOUVELLE PRESSE LIBRE }}\orgindex{Neue Freie Presse@Neue Freie Presse|pw}\hfill \textcolor{gray}{\textbf{8, Rue de Monceau }}\oindex{8, Rue de Monceau@\textbf{8, Rue de Monceau}, \emph{Wohngebäude (K.WHS)}|pw}\pend
           
\pstart
           \textcolor{gray}{\textbf{D\textsuperscript{R} TH. HERZL}}\hfill 19. V. 93\pend
           
\pstart{}Mein lieber Freund!\pend\vspace{0.5em}
\pstart
           Verzeihen Sie, dass ich erst \label{K_L03829-1v}\edtext{heute antworte}{\lemma{\textnormal{\emph{heute antworte}}}\Cendnote{\textnormal{Herzl\pwindex{Herzl, Theodor 1860-05-02 – 1904-07-03@\textsc{Herzl, Theodor} (1860-05-02 – 1904-07-03), \emph{Schriftsteller/Schriftstellerin, Journalist/Journalistin}|pwk} hatte bereits sechs Tage zuvor am
                     13. 5. 1893 eine Antwort verfasst, diese aber nicht abgesandt. Der
                  nicht verschickte Brief (1 Blatt, 3 Seiten, schwarze Tinte, lateinische Kurrent;
                  mit Bleistift beschriftet: »Brief an Schnitzler«, mit Bleistift von Leon Kellner\pwindex{Kellner, Leon 1859-04-17 – 1928-12-05@\textsc{Kellner, Leon} (1859-04-17 – 1928-12-05), \emph{Zionist/Zionistin, Literaturhistoriker/Literaturhistorikerin, Anglist/Anglistin}|pwk} Markierung von Stellen für die
                  Publikation) befindet sich in Herzls\pwindex{Herzl, Theodor 1860-05-02 – 1904-07-03@\textsc{Herzl, Theodor} (1860-05-02 – 1904-07-03), \emph{Schriftsteller/Schriftstellerin, Journalist/Journalistin}|pwk}
                  Nachlass in Jerusalem\oindex{Jerusalem@\textbf{Jerusalem}, \emph{P.PPLA}|pwk}. Der Brief enthält, anders als der später versendete, eine Reflektion über
                     Herzls\pwindex{Herzl, Theodor 1860-05-02 – 1904-07-03@\textsc{Herzl, Theodor} (1860-05-02 – 1904-07-03), \emph{Schriftsteller/Schriftstellerin, Journalist/Journalistin}|pwk} Selbstverständnis als
                  Theaterautor und Journalist: »\textcolor{gray}{\textbf{NOUVELLE PRESSE LIBRE }}\orgindex{Neue Freie Presse@Neue Freie Presse|pw}{ / }\raggedleft{}\textcolor{gray}{\textbf{8, Rue de Monceau }}\oindex{8, Rue de Monceau@\textbf{8, Rue de Monceau}, \emph{Wohngebäude (K.WHS)}|pw}{ / }\textcolor{gray}{\textbf{D\textsuperscript{R} TH. HERZL}}{ / }Mein lieber Freund!{ / }Wie ernst muss es mir mit meinem Entschlusse sein, meine Theaterstücke
                        begraben sein zu lassen, wenn ich sie selbst auf Ihre liebe und unter
                        solchen Umständen wiederholte Aufforderung nicht hervorhole. { / }Verzeihen Sie es mir, aber ich will nichts mehr von mir wissen. Ich bin nur
                        mehr Journalist. Ich gehe als Comfortabelpferd in der Gabel, und nur wenn
                        eine Militärmusik vorüberspielt, mache ich einige komisch aussehende
                        Tanzschritte. { / }Ich glaube, Ihnen das schon einmal
                        auseinandergesetzt zu
                        haben. Es ist weniger Verdruss über meine Misserfolge, über die wegwerfende
                        Behandlung, die mir von der Kritik zu theil wurde – denn was sie loben macht
                        ihren Tadel werthlos – als Reue über meine frühere leichtsinnige
                        unkünstlerische und erfolghascherische Production. Zur Strafe habe ich mich
                        eingemauert und begraben. Aber wäre ich frei, hoffnungsvoll wie in meiner
                        Jugend, könnte ich dichtend in irgend einer angenehmen Landschaft
                        herumwandeln – ich glaube, ich schriebe doch nichts mehr fürs Theater. Ich
                        glaube, ich würde still in mich hineinraisonniren und lächeln und empfände
                        nicht das Bedürfniss dem Premièrenpublicum von Wien\oindex{Wien@\textbf{Wien}, \emph{A.ADM2}|pw} oder Berlin\oindex{Berlin@\textbf{Berlin}, \emph{P.PPLC}|pw} oder irgend
                        einer anderen Stadt sein Händeklätschen herauszulocken.{ / }Ich glaube es am 13 mai 893 wie nun schon
                        ununterbrochen. Die
                        Stimmung ist so dauerhaft, dass sie wol schon die definitive ist. { / }Sie aber sollen schreiben. Jetzt auch, weil es beitragen wird, Sie zu
                        trösten. Was haben Sie in der Arbeit?{ / }Im Sommer, lieber Freund, komme ich auf ein paar Wochen nach Oestreich\oindex{Oesterreich-Ungarn@\textbf{Österreich-Ungarn}, \emph{Land (A.LND)}|pw}, nach Baden\oindex{Baden bei Wien@\textbf{Baden bei Wien}, \emph{P.PPLA3}|pw} bei Wien\oindex{Wien@\textbf{Wien}, \emph{A.ADM2}|pw}: Wir
                        erwarten die Entbindung meiner Frau\pwindex{Herzl, Julie 01.02.1868 – 10.11.1907@\textsc{Herzl, Julie} (01.02.1868 – 10.11.1907)|pwv} von Stunde zu Stunde. Sobald sie reisefähig
                        sein wird begleite ich sie mit meinen drei Kindern\pwindex{Hueft, Pauline 1890-03-29 – 1930-09-08@\textsc{Hüft, Pauline} (1890-03-29 – 1930-09-08)|pwv}\pwindex{Herzl, Hans 10.06.1891 – 14.09.1930@\textsc{Herzl, Hans} (10.06.1891 – 14.09.1930)|pwv}\pwindex{Neumann, Margarethe 20.05.1893 – 15.03.1943@\textsc{Neumann, Margarethe} (20.05.1893 – 15.03.1943)|pwv} nach
                           Baden\oindex{Baden bei Wien@\textbf{Baden bei Wien}, \emph{P.PPLA3}|pw}. Kinder sind noch das beste
                        Mittel, uns zu perpetuiren. { / }Ich würde mich, wie Sie sich denken können sehr freuen, Sie zu sehen, wenn
                        ich nach Oestreich\oindex{Oesterreich-Ungarn@\textbf{Österreich-Ungarn}, \emph{Land (A.LND)}|pw} komme, und mit einem
                        Freund zu plaudern, den ich erst gewann, als wir nicht die Möglichkeit
                        hatten, miteinander zu verkehren. { / }Wer weiss übrigens? Das ist vielleicht die beste Grundlage einer
                        Freundschaft{ / }Ich grüsse Sie herzlich ihr aufrichtiger{ / }Th Herzl{ / }13/5 893«}}}\label{K_L03829-1}, und dass ich Ihren freundlichen Wunsch nicht erfülle.\pend
           
\pstart
           Wie ernst muss es mir mit meinem Entschlusse sein, meine Theaterstücke begraben sein
               zu lassen, wenn ich sie selbst auf Ihre liebe und unter solchen Umständen wiederholte
               Aufforderung nicht hervorhole. Ich will wirklich nichts mehr von mir wissen. {\pb}Wo sind Sie im Sommer? Ich
               werde meine wenigen Urlaubswochen heuer in Oestreich\oindex{Oesterreich-Ungarn@\textbf{Österreich-Ungarn}, \emph{Land (A.LND)}|pw} zubringen. Ich erwarte \label{K_L03829-2v}\edtext{die Entbindung}{\lemma{\textnormal{\emph{die Entbindung}}}\Cendnote{\textnormal{Am
                     20. 5. 1893 kam die dritte Tochter Margarethe\pwindex{Neumann, Margarethe 20.05.1893 – 15.03.1943@\textsc{Neumann, Margarethe} (20.05.1893 – 15.03.1943)|pwk}, genannt Trude, zur Welt.}}}\label{K_L03829-2} meiner Frau\pwindex{Herzl, Julie 01.02.1868 – 10.11.1907@\textsc{Herzl, Julie} (01.02.1868 – 10.11.1907)|pwv}. Sobald sie wieder
               reisefähig ist bringe ich sie mit meinen drei Kindern\pwindex{Hueft, Pauline 1890-03-29 – 1930-09-08@\textsc{Hüft, Pauline} (1890-03-29 – 1930-09-08)|pwv}\pwindex{Herzl, Hans 10.06.1891 – 14.09.1930@\textsc{Herzl, Hans} (10.06.1891 – 14.09.1930)|pwv}\pwindex{Neumann, Margarethe 20.05.1893 – 15.03.1943@\textsc{Neumann, Margarethe} (20.05.1893 – 15.03.1943)|pwv} nach Baden\oindex{Baden bei Wien@\textbf{Baden bei Wien}, \emph{P.PPLA3}|pw} bei Wien\oindex{Wien@\textbf{Wien}, \emph{A.ADM2}|pw}
               zum Sommeraufenthalt. Ich könnte für meine müden Nerven zwar Gebirgsluft brauchen
               muss aber das Opfer bringen nach Baden\oindex{Baden bei Wien@\textbf{Baden bei Wien}, \emph{P.PPLA3}|pw} zu gehen,
               da meine Frau\pwindex{Herzl, Julie 01.02.1868 – 10.11.1907@\textsc{Herzl, Julie} (01.02.1868 – 10.11.1907)|pwv} das Bedürfnis
               hat mit ihrer Familie\pwindex{Naschauer, Jacob 1837-05-20 – 1894-01-01@\textsc{Naschauer, Jacob} (1837-05-20 – 1894-01-01), \emph{Industrieller/Industrielle}|pwv}\pwindex{Naschauer, Jenny 1843-07-13 – 1900-11-03@\textsc{Naschauer, Jenny} (1843-07-13 – 1900-11-03)|pwv} zusammen zukommen. Wenn ich weiss wo Sie sind drücke ich Ihnen im
               Vorbeigehen die Hand.\pend
           
\pstart
           Leben Sie wohl und schreiben Sie!{\\[\baselineskip]} Ihr herzlich ergebener {\\[\baselineskip]}\spacefill\mbox{Th. Herzl}\pend
           \leftskip=0em{}\selectlanguage{ngerman}\endnumbering\briefempfaengerindex{Schnitzler, Arthur@\textsc{Schnitzler, Arthur}!zzzHerzl, Theodor@\emph{von Theodor Herzl}!1893-05-191@{19. 5. 1893}|)be}\mylabel{L03829h}
\begin{anhang}
\end{anhang}\normalsize

\doendnotes{C}
\bigskip
\vfill

\clearpage

\footnotesize

\lohead{\textsc{register}}

% Definiere theindex-Environment komplett neu ohne reledmac
\makeatletter
\renewenvironment{theindex}{%
  \section*{\indexname}%
  \setlength{\parindent}{0pt}%
  \setlength{\parskip}{0pt plus 0.3pt}%
  \let\item\@idxitem
}{%
  \clearpage
}
\makeatother

\IfFileExists{\jobname-pw.ind}{\input{\jobname-pw.ind}}{}

\end{document}

      