%% latex-korrekturansicht-vorspann.tex
%% Vorspann für die Korrekturansicht.
%% Lädt die gemeinsame Datei latex-vorspann.tex mit gesetztem Schalter.

\newif\ifkorrekturansicht
\korrekturansichttrue

\input{../tex-inputs/latex-vorspann}


\section[Hugo von Hofmannsthal an Arthur Schnitzler, 3. 10. 1908]{L01791 Hugo von Hofmannsthal an Arthur Schnitzler, 3. 10. 1908}
\nopagebreak\mylabel{L01791v}
\rehead{ }\normalsize\beginnumbering\briefempfaengerindex{Schnitzler, Arthur@\textsc{Schnitzler, Arthur}!zzzHofmannsthal, Hugo von@\emph{von Hugo von Hofmannsthal}!1908-10-031@{3. 10. 1908}|(be}
\toendnotes[C]{\smallbreak\pagebreak[2]}\Standort{CUL, Schnitzler, B 43.}
\physDesc{Postkarte, 456 Zeichen
\newline{}Handschrift: 1) schwarze Tinte, deutsche Kurrent\hspace{1em}2) schwarze Tinte, lateinische Kurrent (\noindent{}Adresse)\hspace{1em}
\newline{}Versand: Stempel: »\nobreak{}\oindex{Semmering@\textbf{Semmering}, \emph{A.ADM3}|pwk}Semmering 1, 3. X 08, 3\nobreak{}«.  
\newline{}Schnitzler: mit Bleistift datiert: »3. X 08« und beschriftet: »Hofmannsthal« 
\newline{}Ordnung: 1) mit Bleistift von unbekannter Hand nummeriert: »\strikeout{297}«  2) mit Bleistift von unbekannter Hand nummeriert:
                                    »301«}
\buchAbdrucke{\weitereDrucke{Hugo von Hofmannsthal, Arthur Schnitzler: \emph{Briefwechsel}. Frankfurt am Main: \emph{S. Fischer} 1964, S. 241.} }\toendnotes[C]{\smallbreak}\pstart{}{\pb}Herrn D\textsuperscript{r} Arthur Schnitzler\pend{}\pstart{}Wien\oindex{Wien@\textbf{Wien}, \emph{A.ADM2}|pw}\pend{}\pstart{}XVIII Spöttelgasse 7\oindex{Edmund-Weiss-Gasse 7@\textbf{Edmund-Weiß-Gasse 7}, \emph{Wohngebäude (K.WHS)}|pw}\pend{}{\bigskip}\vspace{1em}
\pstart
           \raggedleft{}\textsc{Se{\geminationm}ering}\hspace*{1.5em}3 X.\pend
           \vspace{0.5em}
\pstart
           mein lieber, ich bin hier für unbeſti{\geminationm}te Dauer um meinen 4\textsuperscript{ten} Act\pwindex{Cristinas Heimreise. Komoedie@\emph{Cristinas Heimreise. Komödie}|pwv} zu machen – und den Anfang {\pb}vom erſten\pwindex{Cristinas Heimreise. Komoedie@\emph{Cristinas Heimreise. Komödie}|pwv}, und ein Stückel vom dritten\pwindex{Cristinas Heimreise. Komoedie@\emph{Cristinas Heimreise. Komödie}|pwv}.\hspace*{1.5em}Ko{\geminationm}en Sie nicht mit Ihrem Arbeiterl ein biſſerl
               herauf? wie nett wäre das. Es iſt ſo ein ſchöner Moment in der Landſchaft.\pend
           
\pstart
           Ihr{\\[\baselineskip]}\spacefill\mbox{Hugo}\pend
           \leftskip=0em{}
\pstart
           \noindent{}\label{K_L01791-1v}\edtext{\textsc{L’arbre des roses\pwindex{Rostler, Karl 20.04.1872 – 31.01.1940@\textsc{Rostler, Karl} (20.04.1872 – 31.01.1940), \emph{Hotelportier/Hotelportierin}|pw}, assis dans sa loge,
                     lit toujours avec une mine transfigurée »le
                        chemin à la liberté!\pwindex{Weg ins Freie. Roman@\emph{Der Weg ins Freie. Roman}|pw}« C’est absolument touchant à voir.}}{\lemma{\textnormal{\emph{L’arbre … voir.}}}\Cendnote{\textnormal{»Rosenbaum\pwindex{Rostler, Karl 20.04.1872 – 31.01.1940@\textsc{Rostler, Karl} (20.04.1872 – 31.01.1940), \emph{Hotelportier/Hotelportierin}|pwk}, in seiner Loge sitzend, liest immer mit verklärter Mine
                        ›\emph{Der Weg ins Freie}\pwindex{Weg ins Freie. Roman@\emph{Der Weg ins Freie. Roman}|pwk}‹. Es ist zutiefst
                     rührend anzusehen.« Das Postskript wohl französisch, weil die Karte an besagten
                     Hotelportier Rosenbaum/Rostler\pwindex{Rostler, Karl 20.04.1872 – 31.01.1940@\textsc{Rostler, Karl} (20.04.1872 – 31.01.1940), \emph{Hotelportier/Hotelportierin}|pwk} zur
                     Weiterleitung übermittelt wurde.}}}\label{K_L01791-1}\pend
           \selectlanguage{ngerman}\endnumbering\briefempfaengerindex{Schnitzler, Arthur@\textsc{Schnitzler, Arthur}!zzzHofmannsthal, Hugo von@\emph{von Hugo von Hofmannsthal}!1908-10-031@{3. 10. 1908}|)be}\mylabel{L01791h}  \normalsize

\doendnotes{C}
\bigskip
\vfill

\clearpage

\footnotesize

\lohead{\textsc{register}}

% Definiere theindex-Environment komplett neu ohne reledmac
\makeatletter
\renewenvironment{theindex}{%
  \section*{\indexname}%
  \setlength{\parindent}{0pt}%
  \setlength{\parskip}{0pt plus 0.3pt}%
  \let\item\@idxitem
}{%
  \clearpage
}
\makeatother

\IfFileExists{\jobname-pw.ind}{\input{\jobname-pw.ind}}{}

\end{document}

      