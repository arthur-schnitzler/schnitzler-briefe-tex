%% latex-leseansicht-vorspann.tex
%% Vorspann für die Leseansicht.
%% Lädt die gemeinsame Datei latex-vorspann.tex mit nicht gesetztem Schalter.

\newif\ifkorrekturansicht
\korrekturansichtfalse

\input{../tex-inputs/latex-vorspann}


         
         \renewcommand{\erwaehntePersonen}{Personen: Karl Rostler}
         \renewcommand{\erwaehnteOrte}{Orte: Edmund-Weiß-Gasse, Semmering, Wien}
         \renewcommand{\erwaehnteWerke}{Werke: Cristinas Heimreise. Komödie, Der Weg ins Freie. Roman}
               \section[Hugo von Hofmannsthal an Arthur Schnitzler, 3. 10. 1908]{ Hugo von Hofmannsthal an Arthur Schnitzler, 3. 10. 1908}\nopagebreak\mylabel{v}\rehead{ }\begin{ledgroupsized}[t]{13cm}\normalsize\beginnumbering \toendnotes[C]{\smallbreak\pagebreak[2]} \Standort{CUL, Schnitzler, B 43.}
\physDesc{Postkarte
\newline{}Handschrift: 1) schwarze Tinte, deutsche Kurrent\hspace{1em}2) schwarze Tinte, lateinische Kurrent (\noindent{}Adresse)\hspace{1em}\newline{}Versand: Stempel: »\nobreak{}\oindex{Semmering@\textbf{Semmering}|pwk}Semmering 1, 3. X 08, 3\nobreak{}«.  
\newline{}Schnitzler: mit Bleistift datiert: »3. X 08« und beschriftet: »Hofmannsthal« \newline{}Ordnung: 1) mit Bleistift von unbekannter Hand nummeriert: »\strikeout{297}«  2) mit Bleistift von unbekannter Hand nummeriert:
                                    »301«}\buchAbdrucke{\weitereDrucke{Hugo von Hofmannsthal, Arthur Schnitzler: \emph{Briefwechsel}. Hg. Therese Nickl und Heinrich Schnitzler. Frankfurt am Main: \emph{S. Fischer} 1964, S. 241.} }\toendnotes[C]{\smallbreak}\pstart{}{\pb}Herrn D\textsuperscript{r} Arthur Schnitzler\pend{}\pstart{}Wien\oindex{Wien@\textbf{Wien}|pw}\pend{}\pstart{}XVIII Spöttelgasse 7\oindex{Edmund-Weiss-Gasse@\textbf{Edmund-Weiß-Gasse}|pw}\pend{}{\bigskip}\pstart
           \raggedleft{}\textsc{Se{\geminationm}ering}\hspace*{1.5em}3 X.\pend
           \pstart
           mein lieber, ich bin hier für unbeſti{\geminationm}te Dauer um meinen 4\textsuperscript{ten} Act\pwindex{Hofmannsthal, Hugo von 1874-02-01 – 1929-07-15@\textsc{Hofmannsthal, Hugo von} (1874-02-01 – 1929-07-15), \emph{Schriftsteller}!Cristinas Heimreise. Komoedie11. 2. 1910@\strich\emph{Cristinas Heimreise. Komödie} {[}11. 2. 1910{]}|pwv} zu machen – und den Anfang {\pb}vom erſten\pwindex{Hofmannsthal, Hugo von 1874-02-01 – 1929-07-15@\textsc{Hofmannsthal, Hugo von} (1874-02-01 – 1929-07-15), \emph{Schriftsteller}!Cristinas Heimreise. Komoedie11. 2. 1910@\strich\emph{Cristinas Heimreise. Komödie} {[}11. 2. 1910{]}|pwv}, und ein Stückel vom dritten\pwindex{Hofmannsthal, Hugo von 1874-02-01 – 1929-07-15@\textsc{Hofmannsthal, Hugo von} (1874-02-01 – 1929-07-15), \emph{Schriftsteller}!Cristinas Heimreise. Komoedie11. 2. 1910@\strich\emph{Cristinas Heimreise. Komödie} {[}11. 2. 1910{]}|pwv}.\hspace*{1.5em}Ko{\geminationm}en Sie nicht mit Ihrem Arbeiterl ein biſſerl
               herauf? wie nett wäre das. Es iſt ſo ein ſchöner Moment in der Landſchaft.\pend
           \pstart
           Ihr{\\[\baselineskip]}\spacefill\mbox{Hugo}\pend
           \leftskip=0em{}\pstart
           \noindent{}\label{K_L01791_1v}\edtext{\textsc{L’arbre des roses\pwindex{Rostler, Karl 20.04.1872 – 31.01.1940@\textsc{Rostler, Karl} (20.04.1872 – 31.01.1940), \emph{Hotelportier}|pw}, assis dans sa loge,
                     lit toujours avec une mine transfigurée »le
                        chemin à la liberté!\pwindex{Schnitzler, Arthur 15.05.1862 – 21.10.1931@\textsc{Schnitzler, Arthur} (15.05.1862 – 21.10.1931), \emph{Schriftsteller, Mediziner}!Weg ins Freie. Roman1.1.1908 – 1.6.1908@\strich\emph{Der Weg ins Freie. Roman} {[}1.1.1908 – 1.6.1908{]}|pw}« C’est absolument touchant à voir.}}{\lemma{\textnormal{\emph{L’arbre … voir.}}}\Cendnote{\textnormal{»Rosenbaum\pwindex{Rostler, Karl 20.04.1872 – 31.01.1940@\textsc{Rostler, Karl} (20.04.1872 – 31.01.1940), \emph{Hotelportier}|pwk}, in seiner Loge sitzend, liest immer mit verklärter Mine
                        ›\emph{Der Weg ins Freie}\pwindex{Schnitzler, Arthur 15.05.1862 – 21.10.1931@\textsc{Schnitzler, Arthur} (15.05.1862 – 21.10.1931), \emph{Schriftsteller, Mediziner}!Weg ins Freie. Roman1.1.1908 – 1.6.1908@\strich\emph{Der Weg ins Freie. Roman} {[}1.1.1908 – 1.6.1908{]}|pwk}‹. Es ist zutiefst
                     rührend anzusehen.« Das Postskript wohl französisch, weil die Karte an besagten
                     Hotelportier Rosenbaum/Rostler\pwindex{Rostler, Karl 20.04.1872 – 31.01.1940@\textsc{Rostler, Karl} (20.04.1872 – 31.01.1940), \emph{Hotelportier}|pwk} zur
                     Weiterleitung übermittelt wurde.}}}\label{K_L01791_1h}\pend
           
         
         \endnumbering\mylabel{h}\end{ledgroupsized}  \newcommand{\dateiname}{L01791}\newcommand{\titel}{Hugo von Hofmannsthal an Arthur Schnitzler, 3. 10. 1908}\newcommand{\editorInnen}{Martin Anton Müller und Gerd-Hermann Susen}%% latex-leseansicht-abspann.tex
%% Abspann für die Leseansicht.
%% Der Schalter \ifkorrekturansicht ist bereits durch den Vorspann gesetzt.

%% latex-abspann.tex
%% Gemeinsamer Abspann für Korrekturansicht und Leseansicht.
%% Setzt den Schalter \ifkorrekturansicht voraus (gesetzt in den
%% einbindenden Dateien latex-korrekturansicht-abspann.tex bzw.
%% latex-leseansicht-abspann.tex).
%% ---------------------------------------------------------------

\normalsize

% Das esempio-Environment wird nur in der Leseansicht benötigt
\ifkorrekturansicht\else
\newenvironment{esempio}[3]%
{
    \vspace{1.5ex}
    \rlap{\underline{#1}}
    \par
    \setlength{\parindent}{0cm}
    \nopagebreak
    \leftskip=#2cm
    \rightskip=#3cm
}
{
    \par
}
\fi

\doendnotes{C}
\bigskip
\vfill

\clearpage

\footnotesize

\ifkorrekturansicht
  \lohead{\textsc{register}}
\fi

% theindex-Environment neu definieren ohne reledmac
\makeatletter
\renewenvironment{theindex}{%
  \ifkorrekturansicht
    \section*{\indexname}%
  \else
    \subsubsection*{Index der erwähnten Entitäten}%
  \fi
  \setlength{\parindent}{0pt}%
  \setlength{\parskip}{0pt plus 0.3pt}%
  \let\item\@idxitem
}{%
  \ifkorrekturansicht\clearpage\fi
}
\makeatother

\IfFileExists{\jobname-pw.ind}{\input{\jobname-pw.ind}}{}

% Quellenangabe nur in der Leseansicht
\ifkorrekturansicht\else
% Fallback-Definitionen, falls die .tex-Datei \titel etc. nicht gesetzt hat
\providecommand{\titel}{}
\providecommand{\editorInnen}{}
\providecommand{\dateiname}{\jobname}

\vspace{3cm}

\vfill

\footnotesize
\textsc{Quelle}: \titel. Herausgegeben von {\editorInnen}. In: \emph{Arthur Schnitzler: Briefwechsel mit Autorinnen und Autoren}.
 Digitale Edition, https://schnitzler-briefe.acdh.oeaw.ac.at/{\dateiname}.html (Stand \today)
\fi

\end{document}


      