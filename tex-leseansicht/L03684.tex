%% latex-korrekturansicht-vorspann.tex
%% Vorspann für die Korrekturansicht.
%% Lädt die gemeinsame Datei latex-vorspann.tex mit gesetztem Schalter.

\newif\ifkorrekturansicht
\korrekturansichttrue

\input{../tex-inputs/latex-vorspann}


\section[Stefan Zweig an Arthur Schnitzler, 14. 8. 1920]{L03684 Stefan Zweig an Arthur Schnitzler, 14. 8. 1920}
\nopagebreak\mylabel{L03684v}
\rehead{ }\normalsize\beginnumbering\briefempfaengerindex{Schnitzler, Arthur@\textsc{Schnitzler, Arthur}!zzzZweig, Stefan@\emph{von Stefan Zweig}!1920-08-141@{14. 8. 1920}|(be}
\toendnotes[C]{\smallbreak\pagebreak[2]}\Standort{CUL, Schnitzler, B 118.}
\physDesc{Brief, 1 Blatt, 1 Seite, 649 Zeichen
\newline{}Schreibmaschine
\newline{}Handschrift: blaue Tinte, lateinische Kurrent (\noindent{}Unterschrift)
\newline{}Schnitzler: mit Bleistift beschriftet: »\textsc{Zweig}« }
\buchAbdrucke{\weitereDrucke{Stefan Zweig: \emph{Briefwechsel mit Hermann Bahr, Sigmund Freud, Rainer Maria
                        Rilke und Arthur Schnitzler}. Frankfurt am Main: \emph{S. Fischer} 1987, S. 410–411.} }\toendnotes[C]{\smallbreak}
\pstart
           \raggedleft{}{\pb}Salzburg\oindex{Salzburg@\textbf{Salzburg}, \emph{A.ADM2}|pw} , am 14. August 1920\pend
           
\pstart{}Lieber verehrter Herr Doktor!\pend\vspace{0.5em}
\pstart
           Ich erhielt heute beifolgendes Telegramm von dem New-Yorker\oindex{New York City@\textbf{New York City}, \emph{P.PPL}|pw} Verleger Thomas Seltzer\pwindex{Seltzer, Thomas 22.2.1875 – 11.09.1943@\textsc{Seltzer, Thomas} (22.2.1875 – 11.09.1943), \emph{Übersetzer/Übersetzerin, Verleger/Verlegerin}|pw},
                  5, West Fif\substVorne{}\textsuperscript{tie}\substDazwischen{}th\substHinten{} Street\oindex{5 west 50th Street@\textbf{5 west 50th Street}, \emph{Bürogebäude (K.BUR)}|pw}, New-York\oindex{New York City@\textbf{New York City}, \emph{P.PPL}|pw}, der auch von
               mir \label{K_L03684-1v}\edtext{einige Bücher\pwindex{Romain Rolland@\emph{Romain Rolland}|pwv}\pwindex{Jeremias. Ein dramatische Dichtung in neun Bildern@\emph{Jeremias. Ein dramatische Dichtung in neun Bildern}|pwv}}{\lemma{\textnormal{\emph{einige Bücher}}}\Cendnote{\textnormal{1921 erschien Zweigs\pwindex{Zweig, Stefan 28.11.1881 – 23.02.1942@\textsc{Zweig, Stefan} (28.11.1881 – 23.02.1942), \emph{Schriftsteller/Schriftstellerin}|pwk}{ }Rolland-Biografie\pwindex{Romain Rolland@\emph{Romain Rolland}|pwkv} in
                  englischer Übersetzung bei \emph{Seltzer}\orgindex{Thomas Seltzer, Inc.@Thomas Seltzer, Inc.|pwk} (Stefan Zweig\pwindex{Zweig, Stefan 28.11.1881 – 23.02.1942@\textsc{Zweig, Stefan} (28.11.1881 – 23.02.1942), \emph{Schriftsteller/Schriftstellerin}|pwk}: \emph{Romain Rolland. The man and his work}\pwindex{Romain Rolland@\emph{Romain Rolland}|pwk}.
                     New York: \emph{Seltzer}\orgindex{Thomas Seltzer, Inc.@Thomas Seltzer, Inc.|pwk}{ }1921.), 1922 seine dramatische Dichtung \emph{Jeremias}\pwindex{Jeremias. Ein dramatische Dichtung in neun Bildern@\emph{Jeremias. Ein dramatische Dichtung in neun Bildern}|pwk} (Stefan Zweig\pwindex{Zweig, Stefan 28.11.1881 – 23.02.1942@\textsc{Zweig, Stefan} (28.11.1881 – 23.02.1942), \emph{Schriftsteller/Schriftstellerin}|pwk}: \emph{Jeremiah. A drama in Nine Scenes}\pwindex{Jeremias. Ein dramatische Dichtung in neun Bildern@\emph{Jeremias. Ein dramatische Dichtung in neun Bildern}|pwk}.
                     New York: \emph{T. Seltzer}\orgindex{Thomas Seltzer, Inc.@Thomas Seltzer, Inc.|pwk}{ }1922.)}}}\label{K_L03684-1} bringt. Er hatte ursprünglich ein Buch\pwindex{Brennendes Geheimnis@\emph{Brennendes Geheimnis}|pwv} widerrechtlich von mir
               gebracht, sogar \label{K_L03684-2v}\edtext{unter falschem
                  Namen}{\lemma{\textnormal{\emph{unter falschem
                  Namen}}}\Cendnote{\textnormal{1919 erschien die Novelle \emph{Brennendes
                     Geheimnis}\pwindex{Brennendes Geheimnis@\emph{Brennendes Geheimnis}|pwk} von Stefan Zweig\pwindex{Zweig, Stefan 28.11.1881 – 23.02.1942@\textsc{Zweig, Stefan} (28.11.1881 – 23.02.1942), \emph{Schriftsteller/Schriftstellerin}|pwk} nicht
                     autorisiert unter ebenfalls nicht autorisiertem, ins Englische\oindex{England@\textbf{England}, \emph{A.ADM1}|pwk} übersetzten
                  Autornamen Stephen Branch\pwindex{Zweig, Stefan 28.11.1881 – 23.02.1942@\textsc{Zweig, Stefan} (28.11.1881 – 23.02.1942), \emph{Schriftsteller/Schriftstellerin}|pwk}: \emph{The burning secret}\pwindex{Romain Rolland@\emph{Romain Rolland}|pwk}. New York: \emph{Seltzer and Scott}\orgindex{Thomas Seltzer, Inc.@Thomas Seltzer, Inc.|pwk}{ }1919.}}}\label{K_L03684-2}, hat aber dann die Sache anständig beigelegt und gilt als einer der
               tatkräftigsten Unternehmer. Ich würde Ihnen immerhin raten ihm ein Angebot zu machen,
               das jedenfalls durch den Unterschied der Valuta schon erfreulich wird.\pend
           
\pstart
            Ich nutze den guten Anlass um mich lhnen in Erinnerung zu bringen und bleibe
               mit vielen Grüssen Ihr{\\[\baselineskip]}aufrichtig ergebener{\\[\baselineskip]}\spacefill\mbox{{[}hs.:{]} Stefan Zweig}\pend
           \leftskip=0em{}\selectlanguage{ngerman}\endnumbering\briefempfaengerindex{Schnitzler, Arthur@\textsc{Schnitzler, Arthur}!zzzZweig, Stefan@\emph{von Stefan Zweig}!1920-08-141@{14. 8. 1920}|)be}\mylabel{L03684h}
\begin{anhang}
\end{anhang}\normalsize

\doendnotes{C}
\bigskip
\vfill

\clearpage

\footnotesize

\lohead{\textsc{register}}

% Definiere theindex-Environment komplett neu ohne reledmac
\makeatletter
\renewenvironment{theindex}{%
  \section*{\indexname}%
  \setlength{\parindent}{0pt}%
  \setlength{\parskip}{0pt plus 0.3pt}%
  \let\item\@idxitem
}{%
  \clearpage
}
\makeatother

\IfFileExists{\jobname-pw.ind}{\input{\jobname-pw.ind}}{}

\end{document}

      