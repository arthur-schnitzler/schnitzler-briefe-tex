%% latex-leseansicht-vorspann.tex
%% Vorspann für die Leseansicht.
%% Lädt die gemeinsame Datei latex-vorspann.tex mit nicht gesetztem Schalter.

\newif\ifkorrekturansicht
\korrekturansichtfalse

\input{../tex-inputs/latex-vorspann}


\section[Stefan Zweig an Arthur Schnitzler, 14. 8. 1920]{L03684 Stefan Zweig an Arthur Schnitzler, 14. 8. 1920}
\nopagebreak\mylabel{L03684v}
\rehead{ }\normalsize\beginnumbering\briefempfaengerindex{Schnitzler, Arthur@\textsc{Schnitzler, Arthur}!zzzZweig, Stefan@\emph{von Stefan Zweig}!1920-08-141@{14. 8. 1920}|(be}
\toendnotes[C]{\smallbreak\pagebreak[2]}
\correspDesc{Versand  durch Stefan Zweig am 14. 8. 1920 in Salzburg
\newline{}Erhalt  durch Arthur Schnitzler im Zeitraum [15. 8. 1920
                  – 18. 8. 1920?] in Wien}\toendnotes[C]{\smallbreak}
\Standort{CUL, Schnitzler, B 118.}
\physDesc{Brief, 1 Blatt, 1 Seite, 643 Zeichen
\newline{}Schreibmaschine
\newline{}Handschrift: blaue Tinte, lateinische Kurrent (\noindent{}Unterschrift)
\newline{}Schnitzler: mit Bleistift beschriftet: »\textsc{Zweig}« }
\buchAbdrucke{\weitereDrucke{Stefan Zweig: \emph{Briefwechsel mit Hermann Bahr, Sigmund Freud, Rainer Maria
                        Rilke und Arthur Schnitzler}. Herausgegeben von Jeffrey B. Berlin, Hans-Ulrich Lindken und Donald A. Prater. Frankfurt am Main: \emph{S. Fischer} 1987, S. 410–411.} }\toendnotes[C]{\smallbreak}
\pstart
           \raggedleft{}{\pb}Salzburg\oindex{Salzburg@\textbf{Salzburg}, \emph{Verwaltungsgebiet}|pw}, am 14. August 1920\pend
           
\pstart\center{}Lieber verehrter Herr Doktor!\pend\vspace{0.5em}
\pstart
           Ich erhielt heute \label{K_L03684-1v}\edtext{beifolgendes Telegramm}{\lemma{\textnormal{\emph{beifolgendes Telegramm}}}\Cendnote{\textnormal{Beilage nicht erhalten. Es handelte sich um die Erlaubnis,
                   eine amerikanische Ausgabe von \emph{Casanova’s Heimfahrt}\pwindex{Schnitzler, Arthur 15. 5. 1862 Wien – 21. 10. 1931 ebd.@\textsc{Schnitzler, Arthur} (15. 5. 1862 Wien – 21. 10. 1931 ebd.), \emph{Schriftsteller, Mediziner}!Casanovas Heimfahrt@\strich\emph{Casanovas Heimfahrt}|pwk} drucken zu dürfen, siehe XXXX Auszeichnungsfehler: Dokument L03763 nicht gefunden.}}}\label{K_L03684-1} von dem New-Yorker\oindex{New York City@\textbf{New York City}|pw} Verleger Thomas Seltzer\pwindex{Seltzer, Thomas 22.\,2.\,1875 Poltava – 11.\,9.\,1943 New York City@\textsc{Seltzer, Thomas} (22.\,2.\,1875 Poltava – 11.\,9.\,1943 New York City), \emph{Übersetzer, Verleger}|pw},
                  5, West Fifth Street\oindex{5 west 50th Street@\textbf{5 west 50th Street}, \emph{Bürogebäude}|pw}, New-York\oindex{New York City@\textbf{New York City}|pw}, der auch von
               mir \label{K_L03684-2v}\edtext{einige Bücher\pwindex{Zweig, Stefan 28.\,11.\,1881 Wien – 23.\,2.\,1942 Petrópolis@\textsc{Zweig, Stefan} (28.\,11.\,1881 Wien – 23.\,2.\,1942 Petrópolis), \emph{Schriftsteller}!Romain Rolland. Der Mann und das Werk.@\strich\emph{Romain Rolland. Der Mann und das Werk.}|pwv}\pwindex{Zweig, Stefan 28.\,11.\,1881 Wien – 23.\,2.\,1942 Petrópolis@\textsc{Zweig, Stefan} (28.\,11.\,1881 Wien – 23.\,2.\,1942 Petrópolis), \emph{Schriftsteller}!Jeremias. Eine dramatische Dichtung in neun Bildern@\strich\emph{Jeremias. Eine dramatische Dichtung in neun Bildern}|pwv}}{\lemma{\textnormal{\emph{einige Bücher}}}\Cendnote{\textnormal{Bis auf den Raubdruck \emph{The Burning Secret}\pwindex{Zweig, Stefan 28.\,11.\,1881 Wien – 23.\,2.\,1942 Petrópolis@\textsc{Zweig, Stefan} (28.\,11.\,1881 Wien – 23.\,2.\,1942 Petrópolis), \emph{Schriftsteller}!Romain Rolland. Der Mann und das Werk.@\strich\emph{Romain Rolland. Der Mann und das Werk.}|pwk} (siehe unten) hatte
                  das Verlagshaus \emph{Thomas Seltzer}\orgindex{Thomas Seltzer, Inc.@Thomas Seltzer, Inc.|pwk} zu diesem Zeitpunkt noch nichts von Zweig\pwindex{Zweig, Stefan 28.\,11.\,1881 Wien – 23.\,2.\,1942 Petrópolis@\textsc{Zweig, Stefan} (28.\,11.\,1881 Wien – 23.\,2.\,1942 Petrópolis), \emph{Schriftsteller}|pwk}
                  publiziert. 1921 erschien Zweigs\pwindex{Zweig, Stefan 28.\,11.\,1881 Wien – 23.\,2.\,1942 Petrópolis@\textsc{Zweig, Stefan} (28.\,11.\,1881 Wien – 23.\,2.\,1942 Petrópolis), \emph{Schriftsteller}|pwk}{ }Rolland-Biografie\pwindex{Zweig, Stefan 28.\,11.\,1881 Wien – 23.\,2.\,1942 Petrópolis@\textsc{Zweig, Stefan} (28.\,11.\,1881 Wien – 23.\,2.\,1942 Petrópolis), \emph{Schriftsteller}!Romain Rolland. Der Mann und das Werk.@\strich\emph{Romain Rolland. Der Mann und das Werk.}|pwkv} (Stefan Zweig\pwindex{Zweig, Stefan 28.\,11.\,1881 Wien – 23.\,2.\,1942 Petrópolis@\textsc{Zweig, Stefan} (28.\,11.\,1881 Wien – 23.\,2.\,1942 Petrópolis), \emph{Schriftsteller}|pwk}: \emph{Romain Rolland. The man and his work}\pwindex{Zweig, Stefan 28.\,11.\,1881 Wien – 23.\,2.\,1942 Petrópolis@\textsc{Zweig, Stefan} (28.\,11.\,1881 Wien – 23.\,2.\,1942 Petrópolis), \emph{Schriftsteller}!Romain Rolland. The man and his work@\strich\emph{Romain Rolland. The man and his work}|pwk}. Translated
                     from the original manuscript by Eden\pwindex{Paul, Eden 1.\,11.\,1865 Sturminster Marshall – 1.\,12.\,1944 Salisbury@\textsc{Paul, Eden} (1.\,11.\,1865 Sturminster Marshall – 1.\,12.\,1944 Salisbury), \emph{Schriftsteller, Mediziner}|pwk} and Cedar Paul\pwindex{Paul, Cedar 25.\,12.\,1879 Hampstead – 18.\,3.\,1972 South West Surrey@\textsc{Paul, Cedar} (25.\,12.\,1879 Hampstead – 18.\,3.\,1972 South West Surrey), \emph{Schriftstellerin}|pwk}.
                     New York: \emph{Seltzer}\orgindex{Thomas Seltzer, Inc.@Thomas Seltzer, Inc.|pwk}{ }1921), 1922 seine dramatische Dichtung \emph{Jeremias}\pwindex{Zweig, Stefan 28.\,11.\,1881 Wien – 23.\,2.\,1942 Petrópolis@\textsc{Zweig, Stefan} (28.\,11.\,1881 Wien – 23.\,2.\,1942 Petrópolis), \emph{Schriftsteller}!Jeremias. Eine dramatische Dichtung in neun Bildern@\strich\emph{Jeremias. Eine dramatische Dichtung in neun Bildern}|pwk} (Stefan Zweig\pwindex{Zweig, Stefan 28.\,11.\,1881 Wien – 23.\,2.\,1942 Petrópolis@\textsc{Zweig, Stefan} (28.\,11.\,1881 Wien – 23.\,2.\,1942 Petrópolis), \emph{Schriftsteller}|pwk}: \emph{Jeremiah. A Drama in Nine Scenes}\pwindex{Zweig, Stefan 28.\,11.\,1881 Wien – 23.\,2.\,1942 Petrópolis@\textsc{Zweig, Stefan} (28.\,11.\,1881 Wien – 23.\,2.\,1942 Petrópolis), \emph{Schriftsteller}!Jeremiah. A Drama in Nine Scenes@\strich\emph{Jeremiah. A Drama in Nine Scenes}|pwk}. Translated
                        from the author’s revised German text by Eden\pwindex{Paul, Eden 1.\,11.\,1865 Sturminster Marshall – 1.\,12.\,1944 Salisbury@\textsc{Paul, Eden} (1.\,11.\,1865 Sturminster Marshall – 1.\,12.\,1944 Salisbury), \emph{Schriftsteller, Mediziner}|pwk} and Cedar Paul\pwindex{Paul, Cedar 25.\,12.\,1879 Hampstead – 18.\,3.\,1972 South West Surrey@\textsc{Paul, Cedar} (25.\,12.\,1879 Hampstead – 18.\,3.\,1972 South West Surrey), \emph{Schriftstellerin}|pwk}.
                     New York: \emph{T. Seltzer}\orgindex{Thomas Seltzer, Inc.@Thomas Seltzer, Inc.|pwk}{ }1922).}}}\label{K_L03684-2} bringt. Er hatte ursprünglich ein Buch\pwindex{Zweig, Stefan 28.\,11.\,1881 Wien – 23.\,2.\,1942 Petrópolis@\textsc{Zweig, Stefan} (28.\,11.\,1881 Wien – 23.\,2.\,1942 Petrópolis), \emph{Schriftsteller}!Brennendes Geheimnis@\strich\emph{Brennendes Geheimnis}|pwv} widerrechtlich von mir
               gebracht\pwindex{Zweig, Stefan 28.\,11.\,1881 Wien – 23.\,2.\,1942 Petrópolis@\textsc{Zweig, Stefan} (28.\,11.\,1881 Wien – 23.\,2.\,1942 Petrópolis), \emph{Schriftsteller}!Burning Secret@\strich\emph{The Burning Secret}|pwv}, sogar \label{K_L03684-3v}\edtext{unter falschem
                  Namen}{\lemma{\textnormal{\emph{unter falschem
                  Namen}}}\Cendnote{\textnormal{1919 erschien die Novelle \emph{Brennendes
                     Geheimnis}\pwindex{Zweig, Stefan 28.\,11.\,1881 Wien – 23.\,2.\,1942 Petrópolis@\textsc{Zweig, Stefan} (28.\,11.\,1881 Wien – 23.\,2.\,1942 Petrópolis), \emph{Schriftsteller}!Brennendes Geheimnis@\strich\emph{Brennendes Geheimnis}|pwk} von Stefan Zweig\pwindex{Zweig, Stefan 28.\,11.\,1881 Wien – 23.\,2.\,1942 Petrópolis@\textsc{Zweig, Stefan} (28.\,11.\,1881 Wien – 23.\,2.\,1942 Petrópolis), \emph{Schriftsteller}|pwk} nicht
                     autorisiert unter ebenfalls nicht autorisiertem, ins Englische\oindex{England@\textbf{England}, \emph{Land}|pwk} übersetzten
                  Autornamen Stephen Branch\pwindex{Zweig, Stefan 28.\,11.\,1881 Wien – 23.\,2.\,1942 Petrópolis@\textsc{Zweig, Stefan} (28.\,11.\,1881 Wien – 23.\,2.\,1942 Petrópolis), \emph{Schriftsteller}|pwk} [ = Stefan Zweig\pwindex{Zweig, Stefan 28.\,11.\,1881 Wien – 23.\,2.\,1942 Petrópolis@\textsc{Zweig, Stefan} (28.\,11.\,1881 Wien – 23.\,2.\,1942 Petrópolis), \emph{Schriftsteller}|pwk}]: \emph{The Burning Secret}\pwindex{Zweig, Stefan 28.\,11.\,1881 Wien – 23.\,2.\,1942 Petrópolis@\textsc{Zweig, Stefan} (28.\,11.\,1881 Wien – 23.\,2.\,1942 Petrópolis), \emph{Schriftsteller}!Burning Secret@\strich\emph{The Burning Secret}|pwk}. New York: \emph{Seltzer and Scott}\orgindex{Thomas Seltzer, Inc.@Thomas Seltzer, Inc.|pwk}{ }1919.}}}\label{K_L03684-3}, hat aber dann die Sache anständig beigelegt und gilt als einer der
               tatkräftigsten Unternehmer. Ich würde Ihnen immerhin raten ihm ein Angebot zu machen,
               das jedenfalls durch den Unterschied der Valuta schon erfreulich wird.\pend
           
\pstart
           Ich nutze den guten Anlass um mich Ihnen in Erinnerung zu bringen und bleibe
               mit vielen Grüssen Ihr{\\[\baselineskip]}aufrichtig ergebener{\\[\baselineskip]}\spacefill\mbox{{[}hs.:{]} Stefan Zweig}\pend
           \leftskip=0em{}\selectlanguage{ngerman}\endnumbering\briefempfaengerindex{Schnitzler, Arthur@\textsc{Schnitzler, Arthur}!zzzZweig, Stefan@\emph{von Stefan Zweig}!1920-08-141@{14. 8. 1920}|)be}\mylabel{L03684h}  \newcommand{\dateiname}{L03684}\newcommand{\titel}{Stefan Zweig an Arthur Schnitzler, 14. 8. 1920}\newcommand{\editorInnen}{Selma Jahnke und Martin Anton Müller}%% latex-leseansicht-abspann.tex
%% Abspann für die Leseansicht.
%% Der Schalter \ifkorrekturansicht ist bereits durch den Vorspann gesetzt.

%% latex-abspann.tex
%% Gemeinsamer Abspann für Korrekturansicht und Leseansicht.
%% Setzt den Schalter \ifkorrekturansicht voraus (gesetzt in den
%% einbindenden Dateien latex-korrekturansicht-abspann.tex bzw.
%% latex-leseansicht-abspann.tex).
%% ---------------------------------------------------------------

\normalsize

% Das esempio-Environment wird nur in der Leseansicht benötigt
\ifkorrekturansicht\else
\newenvironment{esempio}[3]%
{
    \vspace{1.5ex}
    \rlap{\underline{#1}}
    \par
    \setlength{\parindent}{0cm}
    \nopagebreak
    \leftskip=#2cm
    \rightskip=#3cm
}
{
    \par
}
\fi

\doendnotes{C}
\bigskip
\vfill

\clearpage

\footnotesize

\ifkorrekturansicht
  \lohead{\textsc{register}}
\fi

% theindex-Environment neu definieren ohne reledmac
\makeatletter
\renewenvironment{theindex}{%
  \ifkorrekturansicht
    \section*{\indexname}%
  \else
    \subsubsection*{Index der erwähnten Entitäten}%
  \fi
  \setlength{\parindent}{0pt}%
  \setlength{\parskip}{0pt plus 0.3pt}%
  \let\item\@idxitem
}{%
  \ifkorrekturansicht\clearpage\fi
}
\makeatother

\IfFileExists{\jobname-pw.ind}{\input{\jobname-pw.ind}}{}

% Quellenangabe nur in der Leseansicht
\ifkorrekturansicht\else
% Fallback-Definitionen, falls die .tex-Datei \titel etc. nicht gesetzt hat
\providecommand{\titel}{}
\providecommand{\editorInnen}{}
\providecommand{\dateiname}{\jobname}

\vspace{3cm}

\vfill

\footnotesize
\textsc{Quelle}: \titel. Herausgegeben von {\editorInnen}. In: \emph{Arthur Schnitzler: Briefwechsel mit Autorinnen und Autoren}.
 Digitale Edition, https://schnitzler-briefe.acdh.oeaw.ac.at/{\dateiname}.html (Stand \today)
\fi

\end{document}


