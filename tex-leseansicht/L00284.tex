%% latex-leseansicht-vorspann.tex
%% Vorspann für die Leseansicht.
%% Lädt die gemeinsame Datei latex-vorspann.tex mit nicht gesetztem Schalter.

\newif\ifkorrekturansicht
\korrekturansichtfalse

\input{../tex-inputs/latex-vorspann}


\section[Wilhelm Bölsche an Arthur Schnitzler, 16. 11. 1893]{L00284 Wilhelm Bölsche an Arthur Schnitzler, 16. 11. 1893}
\nopagebreak\mylabel{L00284v}
\rehead{ }\normalsize\beginnumbering\briefempfaengerindex{Schnitzler, Arthur@\textsc{Schnitzler, Arthur}!zzzBölsche, Wilhelm@\emph{von Wilhelm Bölsche}!1893-11-161@{16. 11. 1893}|(be}
\toendnotes[C]{\smallbreak\pagebreak[2]}
\correspDesc{Versand  durch Wilhelm Bölsche am 16. 11. 1893 in Zürich
\newline{}Erhalt  durch Arthur Schnitzler am 18 11 93 in Wien}\toendnotes[C]{\smallbreak}
\Standort{DLA, A:Schnitzler, HS.NZ85.1.2577,9.}
\physDesc{Postkarte, 377 Zeichen
\newline{}Handschrift: schwarze Tinte, deutsche Kurrent
\newline{}Versand: 1) Stempel: »\nobreak{}\oindex{Enge [Zürich]@\textbf{Enge [Zürich]}, \emph{Teil eines besiedelten Ortes}|pwk}Zürich 7 Enge, 16. XI. 93., 6\nobreak{}«.   2) Stempel: »\nobreak{}\oindex{IX., Alsergrund@\textbf{IX., Alsergrund}, \emph{Verwaltungsgebiet}|pwk}Wien 9/3 72, 18. 11. 93, 8.V, Bestellt\nobreak{}«. 
\newline{}Schnitzler: mit rotem Buntstift nummeriert: »10« }
\buchAbdrucke{\weitereDrucke{Wilhelm Bölsche: \emph{Briefwechsel. Mit Autoren der Freien Bühne}. Herausgegeben von Gerd-Hermann Susen. Berlin: \emph{Weidler} 2010, S. 695 (Werke und Briefe. Wissenschaftliche Ausgabe, Briefe I).} }\pstart{}{\pb}Herrn Dr. Schnitzler\pend{}\pstart{}Wien IX\oindex{IX., Alsergrund@\textbf{IX., Alsergrund}, \emph{Verwaltungsgebiet}|pw}\pend{}\pstart{}Frankgaſſe 1\oindex{Wien@\textbf{Wien}!IX., Alsergrund@\textbf{IX., Alsergrund}!Frankgasse 1@\textbf{Frankgasse 1}, \emph{Wohngebäude}|pw}. \pend{}{\bigskip}\vspace{1em}
\pstart{}{\pb}Hochgeehrter Herr Dr.!\pend\vspace{0.5em}
\pstart
           Die Redaktion der »Freien Bühne\pwindex{Freie Bühne für den Entwickelungskampf der Zeit@\emph{Freie Bühne für den Entwickelungskampf der Zeit}|pw}« hat Hr. Otto Julius Bierbaum\pwindex{Bierbaum, Otto Julius 28.\,6.\,1865 Zielona Góra – 1.\,2.\,1910 Dresden@\textsc{Bierbaum, Otto Julius} (28.\,6.\,1865 Zielona Góra – 1.\,2.\,1910 Dresden)|pw}, Berlin, Köthener Str. 44\oindex{Köthenerstraße@\textbf{Köthenerstraße}, \emph{Straße}|pw} übernommen, ich bitte Sie, bei dieſem
               nachzufragen. Ich bin{ }ſeit 1. Okt. zurückgetreten, – in einer
               allgemeinen »Redaktionsmüdigkeit,« die Sie vielleicht verſtehen werden.\pend
           
\pstart
           Mit herzlichem Gruß{\\[\baselineskip]}Ihr\spacefill\mbox{W. Bölsche}\pend
           \leftskip=0em{}
\pstart
           \noindent{}Zürich-Enge\oindex{Enge [Zürich]@\textbf{Enge [Zürich]}, \emph{Teil eines besiedelten Ortes}|pw}.{\\}Seewartstr. 12\textsubscript{I}\oindex{Seewartstraße@\textbf{Seewartstraße}, \emph{Straße}|pw}.\pend
           \selectlanguage{ngerman}\endnumbering\briefempfaengerindex{Schnitzler, Arthur@\textsc{Schnitzler, Arthur}!zzzBölsche, Wilhelm@\emph{von Wilhelm Bölsche}!1893-11-161@{16. 11. 1893}|)be}\mylabel{L00284h}  \newcommand{\dateiname}{L00284}\newcommand{\titel}{Wilhelm Bölsche an Arthur Schnitzler, 16. 11. 1893}\newcommand{\editorInnen}{Martin Anton Müller und Gerd-Hermann Susen}%% latex-leseansicht-abspann.tex
%% Abspann für die Leseansicht.
%% Der Schalter \ifkorrekturansicht ist bereits durch den Vorspann gesetzt.

%% latex-abspann.tex
%% Gemeinsamer Abspann für Korrekturansicht und Leseansicht.
%% Setzt den Schalter \ifkorrekturansicht voraus (gesetzt in den
%% einbindenden Dateien latex-korrekturansicht-abspann.tex bzw.
%% latex-leseansicht-abspann.tex).
%% ---------------------------------------------------------------

\normalsize

% Das esempio-Environment wird nur in der Leseansicht benötigt
\ifkorrekturansicht\else
\newenvironment{esempio}[3]%
{
    \vspace{1.5ex}
    \rlap{\underline{#1}}
    \par
    \setlength{\parindent}{0cm}
    \nopagebreak
    \leftskip=#2cm
    \rightskip=#3cm
}
{
    \par
}
\fi

\doendnotes{C}
\bigskip
\vfill

\clearpage

\footnotesize

\ifkorrekturansicht
  \lohead{\textsc{register}}
\fi

% theindex-Environment neu definieren ohne reledmac
\makeatletter
\renewenvironment{theindex}{%
  \ifkorrekturansicht
    \section*{\indexname}%
  \else
    \subsubsection*{Index der erwähnten Entitäten}%
  \fi
  \setlength{\parindent}{0pt}%
  \setlength{\parskip}{0pt plus 0.3pt}%
  \let\item\@idxitem
}{%
  \ifkorrekturansicht\clearpage\fi
}
\makeatother

\IfFileExists{\jobname-pw.ind}{\input{\jobname-pw.ind}}{}

% Quellenangabe nur in der Leseansicht
\ifkorrekturansicht\else
% Fallback-Definitionen, falls die .tex-Datei \titel etc. nicht gesetzt hat
\providecommand{\titel}{}
\providecommand{\editorInnen}{}
\providecommand{\dateiname}{\jobname}

\vspace{3cm}

\vfill

\footnotesize
\textsc{Quelle}: \titel. Herausgegeben von {\editorInnen}. In: \emph{Arthur Schnitzler: Briefwechsel mit Autorinnen und Autoren}.
 Digitale Edition, https://schnitzler-briefe.acdh.oeaw.ac.at/{\dateiname}.html (Stand \today)
\fi

\end{document}


