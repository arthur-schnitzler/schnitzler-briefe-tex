%% latex-korrekturansicht-vorspann.tex
%% Vorspann für die Korrekturansicht.
%% Lädt die gemeinsame Datei latex-vorspann.tex mit gesetztem Schalter.

\newif\ifkorrekturansicht
\korrekturansichttrue

\input{../tex-inputs/latex-vorspann}


\section[Arthur Schnitzler an Georg Brandes, 3. 2. 1897]{L00643 Arthur Schnitzler an Georg Brandes, 3. 2. 1897}
\nopagebreak\mylabel{L00643v}
\rehead{ }\normalsize\beginnumbering\briefempfaengerindex{Brandes, Georg@\textsc{Brandes, Georg}!zzzSchnitzler, Arthur@\emph{von Arthur Schnitzler}!1897-02-031@{3. 2. 1897}|(be}
\toendnotes[C]{\smallbreak\pagebreak[2]}\Standort{Kopenhagen, Det Kongelige Bibliotek, Georg Brandes Arkiv, box 125.}
\physDesc{Brief, 2 Blätter, 8 Seiten, 3738 Zeichen
\newline{}Handschrift: schwarze Tinte, deutsche Kurrent
\newline{}Ordnung: mit Bleistift von unbekannter Hand nummeriert mit »7.
                                    Schnitzler « und das zweite Blatt datiert mit »3/2 97« }
\buchAbdrucke{\weitereDrucke{1) Georg Brandes, Arthur Schnitzler: \emph{Ein Briefwechsel}. Bern: \emph{Francke} 1956, S. 61–63.} \weitereDrucke{2) Arthur Schnitzler: \emph{Briefe 1875–1912}. Frankfurt am Main: \emph{S. Fischer} 1981, S. 312–313.} }\toendnotes[C]{\smallbreak}
\pstart
           \raggedleft{}{\pb}Wien\oindex{Wien@\textbf{Wien}, \emph{A.ADM2}|pw}, 3. Feber 1897.\pend
           
\pstart{}Verehrteſter Herr Brandes,\pend\vspace{0.5em}
\pstart
           Sie haben mir einen ſo herzlichen Brief geſchrieben, das freut mich ſehr. Es gehört
               wohl zu den angenehmſten Erfahrungen, einen Menſchen, der einem längſt viel bedeutet
               hat, ſich auch menſchlich nah zu fühlen. Laſſen Sie mich das weiter glauben.\pend
           
\pstart
           Die Milde, mit der Sie mein Stück\pwindex{Freiwild. Schauspiel in 3 Akten@\emph{Freiwild. Schauspiel in 3 Akten}|pwv} beurtheilen ko{\geminationm}t zum großen Theil wohl
               daher, dſs Sie merken, ich ſelbſt ſchätze es richtig. {\pb}Ich meine, man ſchätzt ſich und, was man macht
               beinah i{\geminationm}er richtig, we{\geminationn}
               man nur überhaupt auf einem gewiſſen Niveau ſteht (Wo iſt nur dieſes Niveau? Da
               ſteckt die Schwierigkeit!) Man kennt ſich ſelbſt, und das Streben, nur halb unbewußt,
               geht dahin, ſich \introOben{}ſelbſt\introOben{} miszuverſtehn, was ja freilich nicht
               angenehmer iſt als ſich zu kennen. Das Leben will im allgemeinen doch, daſs wir zur
               Klarheit über uns gelangen.\pend
           
\pstart
           Wie ko{\geminationm}t es nur, dſs Sie mich nach dem Anatol\pwindex{Anatol@\emph{Anatol}|pw}{ }{\pb}für leichtſi{\geminationn}ig
               hielten, jetzt für ernſt? Und doch iſt vielleicht beides richtig. Ich bin
               leichtſinnig in der Art wie ich in Erlebniſſe ſtürze un\textcolor{gray}{d}{ }\label{T_L00643-1v}\edtext{ſchwerlebig}{\lemma{\textnormal{\emph{ſchwerlebig}}}\Cendnote{\textnormal{Darüber findet sich in Bleistift eine lateinische Entzifferung: »schwer\textcolor{gray}{blick}ig«.}}}\label{T_L00643-1} durch
               die Art, wie ſie ſich da{\geminationn} meiner bemächtigen. Ich
               glaube, jeder Menſch hat einen großen Lebensfehler, der ihn abhält, ſein Weſen zur
               möglichen Vollendung zu bringen; meine Sünde mag ſein, dſs ich nicht verſtehe, was zu
               Ende zu leben. Daher befinde ich mich meiſt in einem Zuſtand beträchtlicher innerer
               Schlamperei; Dinge, in denen ich eben ſtehe, ſind in Wirklichkeit {\pb}vorbei; andre, die lang zu Ende gelebt ſind, haben
               ihren Duft zurückgelaſſen – und der Duft von todten Sachen iſt nie ſchön, die Blumen
               auf den Gräbern ſind eine traurige Ausflucht. Ich glaube mit dieſer unreinlichen ja
               faſt unmoraliſchen Art inneren Lebens hängt es auch zuſa\textcolor{gray}{{\geminationm}}en, daſs ich beinah in jedem Einzelfall gedanklich mit allen Möglichkeiten
               einer Weiterentwicklung fertig bin – und daſs ich den Ereigniſſen ſelbſt meiſtens als
               ein verblüffter gegenüberſteh.\pend
           
\pstart
           {\pb}Jetzt eben hab ich manche Verdrießlichkeiten
               durchzumachen, die mich im Arbeiten ja ſogar im ordentlichen Leſen ſtören. Aber bis
               zum Frühjahr muſs manches in Ordnung kommen, und ich will ein bischen fortreiſen. Da
               nehme ich mir Ihren »Shakespeare\pwindex{William Shakespeare@\emph{William Shakespeare}|pw}« mit worauf man
               ſich freut, das ſoll man in Ruhe zu durchleben ſuchen; auch Bücher. Wenn mir was
               einfällt während der Lecture, werde ichs Ihnen ſagen, da Sie mir das ſo freundlich
               erlauben. Daſs {\pb}mir Ihr Buch gefallen wird, iſt
               ſicher; nicht einfach deshalb weil ich weiſs, dſs alles was Sie ſchreiben ſchön iſt
               ſondern weil alles was Sie ſchreiben, \uline{Sie}{ }ſind. Und das iſt viel, das iſt alles beinah. Sie
               ſelbſt haben das heuer in einer dieſer wunderbaren Kopenhagn\oindex{Kopenhagen@\textbf{Kopenhagen}, \emph{P.PPLC}|pw}er Stunden ſo einfach geſagt: »Was einer ſchreibt und ob er
               ſchreibt, iſt eigentlich gleichgiltig, es ko{\geminationm}t drauf an,
               wer ſchreibt –« Sie ſagten es anders, beſſer, aber der Sinn {\pb}war es.\pend
           
\pstart
           Ihre Briefe haben faſt alle etwas Wehmuth; Sehnſucht nach Einſamkeit
                  un\textcolor{gray}{d}{ }Schmerz über Einſamkeit liegt darin, beides. Im
               übrigen gibts de{\geminationn} etwas, was traurig macht oder luſtig
               macht? Ich meine, was die tiefere Trauer un\textcolor{gray}{d} die echte Heiterkeit
               gibt? Wir ſind wie wir ſind und das Leben hat faſt ſo wenig Macht über uns wie wir
               über das Leben – Nun aber fange ich an das Gegentheil von dem zu behaupten, {\pb}was am Anfang dieſes Briefes ſteht. Das läßt einen
               Verdacht gegen mich ſelbſt in mir neu erwachen; daſs ich nemlich nicht klug, ſondern
               »geiſtreich« bin. Es ſind wohl nur Anfälle.\pend
           
\pstart
           Richard Beer-Hofmann\pwindex{Beer-Hofmann, Richard 1866-07-11 – 1945-09-26@\textsc{Beer-Hofmann, Richard} (1866-07-11 – 1945-09-26), \emph{Schriftsteller/Schriftstellerin}|pw} bittet mich, Sie
               herzlichſt zu grüßen.\pend
           
\pstart
           Was ich zunächſt ſchreiben möchte, iſt eine Komödie, \label{K_L00643-1v}\edtext{ſehr geſund, ſehr frech, und wo einer ſiegt}{\lemma{\textnormal{\emph{ſehr … ſiegt}}}\Cendnote{\textnormal{Zu dieser Zeit war Schnitzler mit der Abfassung des
                     \emph{Reigen}\pwindex{Reigen. Zehn Dialoge@\emph{Reigen. Zehn Dialoge}|pwk} beschäftigt, doch es dürfte sich
                  eher um den Stoff der »Entrüsteten« handeln, aus dem sich im Laufe der Zeit \emph{Der Weg ins Freie}\pwindex{Weg ins Freie. Roman@\emph{Der Weg ins Freie. Roman}|pwk} herausschälte. Vgl. den Brief
                  an Otto Brahm\pwindex{Brahm, Otto 05.02.1856 – 28.11.1912@\textsc{Brahm, Otto} (05.02.1856 – 28.11.1912), \emph{Theaterleiter/Theaterleiterin, Regisseur/Regisseurin}|pwk} vom
                  13. 5. 1897.}}}\label{K_L00643-1}. Denn bis jetzt ſind meine Leute immer recht
               ſchäbig zu Grunde gegangen – und ſelten war es ein ſchöner Kampf.\pend
           \pstart – Für heute, mein verehrter Herr Brandes, ſag ich Ihnen einen herzlichen \label{T_L00643-2v}\edtext{Gruſs, vielen innigen Dank und bin Ihr treu
               ergebener \spacefill\mbox{Arthur Schnitzler}}{\lemma{\textnormal{\emph{Gruſs, … Schnitzler}}}\Cendnote{\textnormal{den restlichen Teil der Grußformel und
                  die Unterschrift am unteren Ende der fünften Seite geschrieben}}}\label{T_L00643-2}\pend{}\selectlanguage{ngerman}\endnumbering\briefempfaengerindex{Brandes, Georg@\textsc{Brandes, Georg}!zzzSchnitzler, Arthur@\emph{von Arthur Schnitzler}!1897-02-031@{3. 2. 1897}|)be}\mylabel{L00643h}  \normalsize

\doendnotes{C}
\bigskip
\vfill

\clearpage

\footnotesize

\lohead{\textsc{register}}

% Definiere theindex-Environment komplett neu ohne reledmac
\makeatletter
\renewenvironment{theindex}{%
  \section*{\indexname}%
  \setlength{\parindent}{0pt}%
  \setlength{\parskip}{0pt plus 0.3pt}%
  \let\item\@idxitem
}{%
  \clearpage
}
\makeatother

\IfFileExists{\jobname-pw.ind}{\input{\jobname-pw.ind}}{}

\end{document}

      