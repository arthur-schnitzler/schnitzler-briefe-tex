%% latex-korrekturansicht-vorspann.tex
%% Vorspann für die Korrekturansicht.
%% Lädt die gemeinsame Datei latex-vorspann.tex mit gesetztem Schalter.

\newif\ifkorrekturansicht
\korrekturansichttrue

\input{../tex-inputs/latex-vorspann}


\section[Arthur Schnitzler: Widmungsexemplar Buch der Sprüche und Bedenken für Hugo Hofmannsthal, 16. 12. 1927]{L02493 Arthur Schnitzler: Widmungsexemplar Buch der Sprüche und Bedenken für
               Hugo Hofmannsthal, 16. 12. 1927}
\nopagebreak\mylabel{L02493v}
\rehead{ }\normalsize\beginnumbering\briefempfaengerindex{Hofmannsthal, Hugo von@\textsc{Hofmannsthal, Hugo von}!zzzSchnitzler, Arthur@\emph{von Arthur Schnitzler}!1927-12-161@{16. 12. 1927}|(be}
\toendnotes[C]{\smallbreak\pagebreak[2]}\Standort{FDH, FDH 3221.}
\physDesc{Widmung am Vorsatzblatt, 63 Zeichen
\newline{}Handschrift: schwarze Tinte, lateinische Kurrent}
\buchAbdrucke{\weitereDrucke{Hugo von Hofmannsthal: \emph{Bibliothek}. Frankfurt am Main: \emph{S. Fischer} 2011, S. 603.} }
\pstart
           \noindent{}{\pb}Meinem lieben Hugo
                  Hofma{\geminationn}sthal\pend
           
\pstart
           herzlichſt{\\[\baselineskip]}\spacefill\mbox{ArthSch}\pend
           \leftskip=0em{}
\pstart
           Wien\oindex{Wien@\textbf{Wien}, \emph{A.ADM2}|pw}{ }16/12 927\pend
           \selectlanguage{ngerman}\vspace{1em}{\vspace{1\baselineskip}}
\pstart
           \centering{}{\pb}\textcolor{gray}{\textbf{\so{ARTHUR SCHNITZLER}}}\pend
           
\pstart
           \centering{}\textcolor{gray}{\textbf{BUCH DER SPRÜCHE UND BEDENKEN\pwindex{Buch der Sprueche und Bedenken@\emph{Buch der Sprüche und Bedenken}|pw}}}\pend
           
\pstart
           \centering{}\textcolor{gray}{\textbf{\so{APHORISMEN UND FRAGMENTE}}}\pend
           {\vspace{1\baselineskip}}
\pstart
           \centering{}\textcolor{gray}{\textbf{IM PHAIDON-VERLAG\orgindex{Phaidon-Verlag@Phaidon-Verlag|pw} ⋅ WIEN\oindex{Wien@\textbf{Wien}, \emph{A.ADM2}|pw} ⋅ MCMXXVII}}\pend
           \selectlanguage{ngerman}\endnumbering\briefempfaengerindex{Hofmannsthal, Hugo von@\textsc{Hofmannsthal, Hugo von}!zzzSchnitzler, Arthur@\emph{von Arthur Schnitzler}!1927-12-161@{16. 12. 1927}|)be}\mylabel{L02493h}  \normalsize

\doendnotes{C}
\bigskip
\vfill

\clearpage

\footnotesize

\lohead{\textsc{register}}

% Definiere theindex-Environment komplett neu ohne reledmac
\makeatletter
\renewenvironment{theindex}{%
  \section*{\indexname}%
  \setlength{\parindent}{0pt}%
  \setlength{\parskip}{0pt plus 0.3pt}%
  \let\item\@idxitem
}{%
  \clearpage
}
\makeatother

\IfFileExists{\jobname-pw.ind}{\input{\jobname-pw.ind}}{}

\end{document}

      