%% latex-leseansicht-vorspann.tex
%% Vorspann für die Leseansicht.
%% Lädt die gemeinsame Datei latex-vorspann.tex mit nicht gesetztem Schalter.

\newif\ifkorrekturansicht
\korrekturansichtfalse

\input{../tex-inputs/latex-vorspann}


\section[Arthur Schnitzler an Hermann Bahr, 18. 4. 1913]{L02122 Arthur Schnitzler an Hermann Bahr, 18. 4. 1913}
\nopagebreak\mylabel{L02122v}
\rehead{ }\normalsize\beginnumbering\briefempfaengerindex{Bahr, Hermann@\textsc{Bahr, Hermann}!zzzSchnitzler, Arthur@\emph{von Arthur Schnitzler}!1913-04-181@{18. 4. 1913}|(be}
\toendnotes[C]{\smallbreak\pagebreak[2]}
\correspDesc{Versand  durch Arthur Schnitzler am 18. 4. 1913 in Wien
\newline{}Erhalt  durch Hermann Bahr im Zeitraum [19. 4. 1913
                  – 23. 4. 1913?] in Salzburg}\toendnotes[C]{\smallbreak}
\Standort{TMW, HS AM 60160 Ba.}
\physDesc{Briefkarte, 1060 Zeichen
\newline{}Schreibmaschine
\newline{}Handschrift: schwarze Tinte, deutsche Kurrent (\noindent{}Grußformel und Unterschrift)
\newline{}Ordnung: Lochung }
\buchAbdrucke{\weitereDrucke{1) \emph{18. 4. 1913, Abschrift.} In: Arthur Schnitzler: \emph{The Letters of Arthur Schnitzler to Hermann Bahr}. Edited, annotated, and with an introduction, by Donald G. Daviau. Chapel Hill: \emph{The University of North Carolina Press} 1978, S. 110 (University of North Carolina studies in the Germanic languages
                        and literatures, 89).} \weitereDrucke{2) Hermann Bahr, Arthur Schnitzler: \emph{Briefwechsel, Aufzeichnungen, Dokumente (1891–1931)}. Herausgegeben von Kurt Ifkovits und Martin Anton Müller. Göttingen: \emph{Wallstein} 2018, S. 482.} }\toendnotes[C]{\smallbreak}
\pstart
           {\pb}\textcolor{gray}{\textbf{Dr. Arthur Schnitzler}}\hfill 18. 4. 1913.\pend
           
\pstart
           \textcolor{gray}{\textbf{Wien XVIII. Sternwartestrasse 71\oindex{Wien@\textbf{Wien}!XVIII., Währing@\textbf{XVIII., Währing}!Sternwartestraße 71@\textbf{Sternwartestraße 71}, \emph{Wohngebäude}|pw}}}\pend
           
\pstart{}Lieber Hermann.\pend\vspace{0.5em}
\pstart
           Auch ich habe einen Brief von Altenberg\pwindex{Altenberg, Peter 9.\,3.\,1859 Wien – 8.\,1.\,1919 ebd.@\textsc{Altenberg, Peter} (9.\,3.\,1859 Wien – 8.\,1.\,1919 ebd.), \emph{Schriftsteller}|pw}{ }\introOben{}(\introOben{}offenbar ähnlichen Inhalts wie der an Dich\introOben{})\introOben{} erhalten; sein Bruder\pwindex{Engländer, Georg 3.\,4.\,1862 Wien – 10.\,4.\,1927 ebd.@\textsc{Engländer, Georg} (3.\,4.\,1862 Wien – 10.\,4.\,1927 ebd.), \emph{Privatbeamter}|pwv} hat ihn mir überschickt. Diesem habe ich nun
               geantwortet, er möge mir sagen, was ich seiner Ansicht nach in der Angelegenheit tun
               könne; ich sei natürlich gerne bereit in die Anstalt zu gehen und dort mit dem
               behandelnden Arzt\pwindex{Richter, Karl 9.\,3.\,1862 Bruntál – 25.\,6.\,1937 Wien@\textsc{Richter, Karl} (9.\,3.\,1862 Bruntál – 25.\,6.\,1937 Wien), \emph{Mediziner, Sanatoriumsleiter}|pwv}
               Rücksprache zu nehmen. Ich selbst habe Altenberg\pwindex{Altenberg, Peter 9.\,3.\,1859 Wien – 8.\,1.\,1919 ebd.@\textsc{Altenberg, Peter} (9.\,3.\,1859 Wien – 8.\,1.\,1919 ebd.), \emph{Schriftsteller}|pw} schon über ein Jahr nicht gesehen und stehe trotz allem, was mir
               selbst von ärztlicher Seite berichtet wird, der absoluten Echtheit von P. A.\pwindex{Altenberg, Peter 9.\,3.\,1859 Wien – 8.\,1.\,1919 ebd.@\textsc{Altenberg, Peter} (9.\,3.\,1859 Wien – 8.\,1.\,1919 ebd.), \emph{Schriftsteller}|pw}’s Irr{\pb}sinn – es ist ja
               vielleicht dumm – mit einer seit fast drei Jahrzehnten bewährten Skepsis gegenüber.
               Dass an P. A.\pwindex{Altenberg, Peter 9.\,3.\,1859 Wien – 8.\,1.\,1919 ebd.@\textsc{Altenberg, Peter} (9.\,3.\,1859 Wien – 8.\,1.\,1919 ebd.), \emph{Schriftsteller}|pw}’s Einschliessung nicht etwa böser
               Wille schuld sein kann ist selbstverständlich. Also, wenn eine Entlassung überhaupt
               möglich (was ich aus vielen Gründen für höchst wahrscheinlich halte) wird dazu weder
               Skandal noch Entführung notwendig sein. Du hörst bald mehr von mir. Wann kommst Du
               nach Wien\oindex{Wien@\textbf{Wien}, \emph{Verwaltungsgebiet}|pw}? Man sieht Dich nun doch nicht trotzdem
               Du in Salzburg\oindex{Salzburg@\textbf{Salzburg}, \emph{Verwaltungsgebiet}|pw} wohnst.\pend
           
\pstart
           Herzliche Grüsse von Haus zu Haus{\\[\baselineskip]}Dein{\\[\baselineskip]}\spacefill\mbox{{[}hs.:{]} Arthur}\pend
           \leftskip=0em{}\selectlanguage{ngerman}\endnumbering\briefempfaengerindex{Bahr, Hermann@\textsc{Bahr, Hermann}!zzzSchnitzler, Arthur@\emph{von Arthur Schnitzler}!1913-04-181@{18. 4. 1913}|)be}\mylabel{L02122h}  \newcommand{\dateiname}{L02122}\newcommand{\titel}{Arthur Schnitzler an Hermann Bahr, 18. 4. 1913}\newcommand{\editorInnen}{Herausgegeben von Martin Anton Müller}%% latex-leseansicht-abspann.tex
%% Abspann für die Leseansicht.
%% Der Schalter \ifkorrekturansicht ist bereits durch den Vorspann gesetzt.

%% latex-abspann.tex
%% Gemeinsamer Abspann für Korrekturansicht und Leseansicht.
%% Setzt den Schalter \ifkorrekturansicht voraus (gesetzt in den
%% einbindenden Dateien latex-korrekturansicht-abspann.tex bzw.
%% latex-leseansicht-abspann.tex).
%% ---------------------------------------------------------------

\normalsize

% Das esempio-Environment wird nur in der Leseansicht benötigt
\ifkorrekturansicht\else
\newenvironment{esempio}[3]%
{
    \vspace{1.5ex}
    \rlap{\underline{#1}}
    \par
    \setlength{\parindent}{0cm}
    \nopagebreak
    \leftskip=#2cm
    \rightskip=#3cm
}
{
    \par
}
\fi

\doendnotes{C}
\bigskip
\vfill

\clearpage

\footnotesize

\ifkorrekturansicht
  \lohead{\textsc{register}}
\fi

% theindex-Environment neu definieren ohne reledmac
\makeatletter
\renewenvironment{theindex}{%
  \ifkorrekturansicht
    \section*{\indexname}%
  \else
    \subsubsection*{Index der erwähnten Entitäten}%
  \fi
  \setlength{\parindent}{0pt}%
  \setlength{\parskip}{0pt plus 0.3pt}%
  \let\item\@idxitem
}{%
  \ifkorrekturansicht\clearpage\fi
}
\makeatother

\IfFileExists{\jobname-pw.ind}{\input{\jobname-pw.ind}}{}

% Quellenangabe nur in der Leseansicht
\ifkorrekturansicht\else
% Fallback-Definitionen, falls die .tex-Datei \titel etc. nicht gesetzt hat
\providecommand{\titel}{}
\providecommand{\editorInnen}{}
\providecommand{\dateiname}{\jobname}

\vspace{3cm}

\vfill

\footnotesize
\textsc{Quelle}: \titel. Herausgegeben von {\editorInnen}. In: \emph{Arthur Schnitzler: Briefwechsel mit Autorinnen und Autoren}.
 Digitale Edition, https://schnitzler-briefe.acdh.oeaw.ac.at/{\dateiname}.html (Stand \today)
\fi

\end{document}


