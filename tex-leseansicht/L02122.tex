%% latex-korrekturansicht-vorspann.tex
%% Vorspann für die Korrekturansicht.
%% Lädt die gemeinsame Datei latex-vorspann.tex mit gesetztem Schalter.

\newif\ifkorrekturansicht
\korrekturansichttrue

\input{../tex-inputs/latex-vorspann}


\section[Arthur Schnitzler an Hermann Bahr, 18. 4. 1913]{L02122 Arthur Schnitzler an Hermann Bahr, 18. 4. 1913}
\nopagebreak\mylabel{L02122v}
\rehead{ }\normalsize\beginnumbering\briefempfaengerindex{Bahr, Hermann@\textsc{Bahr, Hermann}!zzzSchnitzler, Arthur@\emph{von Arthur Schnitzler}!1913-04-181@{18. 4. 1913}|(be}
\toendnotes[C]{\smallbreak\pagebreak[2]}\Standort{TMW, HS AM 60160 Ba.}
\physDesc{Briefkarte, 1060 Zeichen
\newline{}Schreibmaschine
\newline{}Handschrift: schwarze Tinte, deutsche Kurrent (\noindent{}Grußformel und Unterschrift)
\newline{}Ordnung: Lochung }
\buchAbdrucke{\weitereDrucke{1) Arthur Schnitzler: \emph{The Letters of Arthur Schnitzler to Hermann Bahr}. Chapel Hill: \emph{The University of North Carolina Press} 1978, S. 110.} \weitereDrucke{2) Hermann Bahr, Arthur Schnitzler: \emph{Briefwechsel, Aufzeichnungen, Dokumente (1891–1931)}. Göttingen: \emph{Wallstein} 2018, S. 482.} }\toendnotes[C]{\smallbreak}
\pstart
           {\pb}\textcolor{gray}{\textbf{Dr. Arthur Schnitzler}}\hfill 18. 4. 1913. \pend
           
\pstart
           \textcolor{gray}{\textbf{Wien XVIII. Sternwartestrasse 71\oindex{Sternwartestrasse 71@\textbf{Sternwartestraße 71}, \emph{Wohngebäude (K.WHS)}|pw}}}\pend
           
\pstart{}Lieber Hermann.\pend\vspace{0.5em}
\pstart
           Auch ich habe einen Brief von Altenberg\pwindex{Altenberg, Peter 09.03.1859 – 08.01.1919@\textsc{Altenberg, Peter} (09.03.1859 – 08.01.1919), \emph{Schriftsteller/Schriftstellerin}|pw}{ }\introOben{}(\introOben{}offenbar ähnlichen Inhalts wie der an Dich\introOben{})\introOben{} erhalten; sein Bruder\pwindex{Englaender, Georg 03.04.1862 – 10.04.1927@\textsc{Engländer, Georg} (03.04.1862 – 10.04.1927), \emph{Privatbeamter/Privatbeamtin}|pwv} hat ihn mir überschickt. Diesem habe ich nun
               geantwortet, er möge mir sagen, was ich seiner Ansicht nach in der Angelegenheit tun
               könne; ich sei natürlich gerne bereit in die Anstalt zu gehen und dort mit dem
               behandelnden Arzt\pwindex{Richter, Karl 09.03.1862 – 25.06.1937@\textsc{Richter, Karl} (09.03.1862 – 25.06.1937), \emph{Mediziner/Medizinerin, Sanatoriumsleiter/Sanatoriumsleiterin}|pwv}
               Rücksprache zu nehmen. Ich selbst habe Altenberg\pwindex{Altenberg, Peter 09.03.1859 – 08.01.1919@\textsc{Altenberg, Peter} (09.03.1859 – 08.01.1919), \emph{Schriftsteller/Schriftstellerin}|pw} schon über ein Jahr nicht gesehen und stehe trotz allem, was mir
               selbst von ärztlicher Seite berichtet wird, der absoluten Echtheit von P. A.\pwindex{Altenberg, Peter 09.03.1859 – 08.01.1919@\textsc{Altenberg, Peter} (09.03.1859 – 08.01.1919), \emph{Schriftsteller/Schriftstellerin}|pw}’s Irr{\pb}sinn – es ist ja
               vielleicht dumm – mit einer seit fast drei Jahrzehnten bewährten Skepsis gegenüber.
               Dass an P. A.\pwindex{Altenberg, Peter 09.03.1859 – 08.01.1919@\textsc{Altenberg, Peter} (09.03.1859 – 08.01.1919), \emph{Schriftsteller/Schriftstellerin}|pw}’s Einschliessung nicht etwa böser
               Wille schuld sein kann ist selbstverständlich. Also, wenn eine Entlassung überhaupt
               möglich (was ich aus vielen Gründen für höchst wahrscheinlich halte) wird dazu weder
               Skandal noch Entführung notwendig sein. Du hörst bald mehr von mir. Wann kommst Du
               nach Wien\oindex{Wien@\textbf{Wien}, \emph{A.ADM2}|pw}? Man sieht Dich nun doch nicht trotzdem
               Du in Salzburg\oindex{Salzburg@\textbf{Salzburg}, \emph{A.ADM2}|pw} wohnst.\pend
           
\pstart
           Herzliche Grüsse von Haus zu Haus{\\[\baselineskip]}Dein{\\[\baselineskip]}\spacefill\mbox{{[}hs.:{]} Arthur}\pend
           \leftskip=0em{}\selectlanguage{ngerman}\endnumbering\briefempfaengerindex{Bahr, Hermann@\textsc{Bahr, Hermann}!zzzSchnitzler, Arthur@\emph{von Arthur Schnitzler}!1913-04-181@{18. 4. 1913}|)be}\mylabel{L02122h}  \normalsize

\doendnotes{C}
\bigskip
\vfill

\clearpage

\footnotesize

\lohead{\textsc{register}}

% Definiere theindex-Environment komplett neu ohne reledmac
\makeatletter
\renewenvironment{theindex}{%
  \section*{\indexname}%
  \setlength{\parindent}{0pt}%
  \setlength{\parskip}{0pt plus 0.3pt}%
  \let\item\@idxitem
}{%
  \clearpage
}
\makeatother

\IfFileExists{\jobname-pw.ind}{\input{\jobname-pw.ind}}{}

\end{document}

      