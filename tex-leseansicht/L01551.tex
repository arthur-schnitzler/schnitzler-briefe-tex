%% latex-korrekturansicht-vorspann.tex
%% Vorspann für die Korrekturansicht.
%% Lädt die gemeinsame Datei latex-vorspann.tex mit gesetztem Schalter.

\newif\ifkorrekturansicht
\korrekturansichttrue

\input{../tex-inputs/latex-vorspann}


\section[Arthur Schnitzler an Hermann Bahr, 21. 9. 1905]{L01551 Arthur Schnitzler an Hermann Bahr, 21. 9. 1905}
\nopagebreak\mylabel{L01551v}
\rehead{ }\normalsize\beginnumbering\briefempfaengerindex{Bahr, Hermann@\textsc{Bahr, Hermann}!zzzSchnitzler, Arthur@\emph{von Arthur Schnitzler}!1905-09-211@{21. 9. 1905}|(be}
\toendnotes[C]{\smallbreak\pagebreak[2]}\Standort{TMW, HS AM 23372 Ba.}
\physDesc{Brief, 1 Blatt, 3 Seiten, 990 Zeichen
\newline{}Handschrift: schwarze Tinte, deutsche Kurrent
\newline{}Ordnung: Lochung }
\buchAbdrucke{\weitereDrucke{1) Arthur Schnitzler: \emph{The Letters of Arthur Schnitzler to Hermann Bahr}. Chapel Hill: \emph{The University of North Carolina Press} 1978, S. 91–92.} \weitereDrucke{2) Hermann Bahr, Arthur Schnitzler: \emph{Briefwechsel, Aufzeichnungen, Dokumente (1891–1931)}. Göttingen: \emph{Wallstein} 2018, S. 354–355.} }\toendnotes[C]{\smallbreak}
\pstart
           \raggedleft{}{\pb}21. 9. 905\pend
           
\pstart{}lieber Hermann, \pend\vspace{0.5em}
\pstart
           alles zugegeben, und das \label{K_L01551-1v}\edtext{\textsc{Epitheton}}{\lemma{\textnormal{\emph{Epitheton}}}\Cendnote{\textnormal{schmückendes Beiwort; hier ist es auf »reizend«
                  gemünzt.}}}\label{K_L01551-1} reizend als allzu freundlich empfunden: nur den Fürſten\pwindex{Zwischenspiel. Komoedie in drei Akten@\emph{Zwischenspiel. Komödie in drei Akten}|pwv} geb ich dir nicht ſo ohne weiteres
               preis. Ich weiſs \textcolor{gray}{zu} gut, dſs dieſe Art, von der ich einen zu
               ſchildern verſucht, nicht die Regel iſt – aber gerade dſs er eine Ausnahme unter
               denen ſeines Standes iſt, bildet für {\pb}\textsc{Caecilie\pwindex{Zwischenspiel. Komoedie in drei Akten@\emph{Zwischenspiel. Komödie in drei Akten}|pwv}} wahrſcheinlich einen Charme mehr. Ich hatte früher ein paar Stellen im Dialog,
               die ich als überdeutlich eliminirte, und in denen auf den tiefen Weſensunterſchied
               zwiſchen Menſchen à la \textsc{Amadeus\pwindex{Zwischenspiel. Komoedie in drei Akten@\emph{Zwischenspiel. Komödie in drei Akten}|pwv}} und ſolchen à la \textsc{Sigismund\pwindex{Zwischenspiel. Komoedie in drei Akten@\emph{Zwischenspiel. Komödie in drei Akten}|pwv}} eingegangen \strikeout{wird} und dieſes »Andersſein« \introOben{}des \textsc{Sigism.}\introOben{} als Motiv für \textsc{Caeciliens\pwindex{Zwischenspiel. Komoedie in drei Akten@\emph{Zwischenspiel. Komödie in drei Akten}|pwv}} Hinüberſchwanken \substVorne{}\textsuperscript{verwendet}\substDazwischen{}ausgeſprochen\substHinten{} wurde. –\pend
           
\pstart
           – Morgen fahren wir\pwindex{Schnitzler, Olga 17.01.1882 – 13.01.1970@\textsc{Schnitzler, Olga} (17.01.1882 – 13.01.1970), \emph{Schauspieler/Schauspielerin, Sänger/Sängerin}|pwv} auf ein
               paar Tage fort (Semmering\oindex{Semmering@\textbf{Semmering}, \emph{A.ADM3}|pw}, {\pb}ev. weiter) – ſobald
               ich zurück \substVorne{}\textsuperscript{komme}\substDazwischen{}bin\substHinten{}, mußt du zu \textcolor{gray}{uns} ko{\geminationm}en. Wärs
               dir nicht am bequemſten, bei uns\pwindex{Schnitzler, Olga 17.01.1882 – 13.01.1970@\textsc{Schnitzler, Olga} (17.01.1882 – 13.01.1970), \emph{Schauspieler/Schauspielerin, Sänger/Sängerin}|pwv} zu Mittag zu eſſen? Etwa 11–12 zu ko{\geminationm}en und
               dann zu bleiben, ſo lang du eben ka{\geminationn}ſt? Jedenfalls muſs
               etwas gefunden werden, damit man einander \substVorne{}\textsuperscript{mehr}\substDazwischen{}oefter\substHinten{}{ }ſieht. –\pend
           
\pstart
           Von Herzen dein{\\[\baselineskip]}\spacefill\mbox{A.}\pend
           \leftskip=0em{}\selectlanguage{ngerman}\endnumbering\briefempfaengerindex{Bahr, Hermann@\textsc{Bahr, Hermann}!zzzSchnitzler, Arthur@\emph{von Arthur Schnitzler}!1905-09-211@{21. 9. 1905}|)be}\mylabel{L01551h}  \normalsize

\doendnotes{C}
\bigskip
\vfill

\clearpage

\footnotesize

\lohead{\textsc{register}}

% Definiere theindex-Environment komplett neu ohne reledmac
\makeatletter
\renewenvironment{theindex}{%
  \section*{\indexname}%
  \setlength{\parindent}{0pt}%
  \setlength{\parskip}{0pt plus 0.3pt}%
  \let\item\@idxitem
}{%
  \clearpage
}
\makeatother

\IfFileExists{\jobname-pw.ind}{\input{\jobname-pw.ind}}{}

\end{document}

      