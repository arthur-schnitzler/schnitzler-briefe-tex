%% latex-korrekturansicht-vorspann.tex
%% Vorspann für die Korrekturansicht.
%% Lädt die gemeinsame Datei latex-vorspann.tex mit gesetztem Schalter.

\newif\ifkorrekturansicht
\korrekturansichttrue

\input{../tex-inputs/latex-vorspann}


\section[Hermann Bahr an Arthur Schnitzler, 10. 11. 1903]{L01340 Hermann Bahr an Arthur Schnitzler, 10. 11. 1903}
\nopagebreak\mylabel{L01340v}
\rehead{ }\normalsize\beginnumbering\briefempfaengerindex{Schnitzler, Arthur@\textsc{Schnitzler, Arthur}!zzzBahr, Hermann@\emph{von Hermann Bahr}!1903-11-104@{10. 11. 1903}|(be}
\toendnotes[C]{\smallbreak\pagebreak[2]}\Standort{CUL, Schnitzler, B 5b.}
\physDesc{Brief, 1 Blatt, 1 Seite, 569 Zeichen
\newline{}Handschrift: schwarze Tinte, deutsche Kurrent
\newline{}Ordnung: mit Bleistift von unbekannter Hand nummeriert:
                                    »102a« }
\buchAbdrucke{\weitereDrucke{Hermann Bahr, Arthur Schnitzler: \emph{Briefwechsel, Aufzeichnungen, Dokumente (1891–1931)}. Göttingen: \emph{Wallstein} 2018, S. 279.} }\toendnotes[C]{\smallbreak}
\pstart
           \raggedleft{}{\pb}10. 11. 03\pend
           
\pstart\center{}Lieber Arthur!\pend\vspace{0.5em}
\pstart
           Kannſt Du mir, auf einer Correspondenz Karte, Auskunft geben, ob der Titel »\label{K_L01340-1v}\edtext{Primarius}{\lemma{\textnormal{\emph{Primarius}}}\Cendnote{\textnormal{Hintergrund der Anfrage Brahms\pwindex{Brahm, Otto 05.02.1856 – 28.11.1912@\textsc{Brahm, Otto} (05.02.1856 – 28.11.1912), \emph{Theaterleiter/Theaterleiterin, Regisseur/Regisseurin}|pwk} bildet die Premierenvorbereitung von \emph{Der Meister}\pwindex{Meister. Komoedie in drei Akten@\emph{Der Meister. Komödie in drei Akten}|pwk}.}}}\label{K_L01340-1}« in Süddeutſchland\oindex{Deutschland@\textbf{Deutschland}, \emph{A.PCLI}|pw} üblich iſt und wie jemand, der bei uns Primarius heißt, in
                  Norddeutſchland\oindex{Deutschland@\textbf{Deutschland}, \emph{A.PCLI}|pw} genannt wird? Brahm\pwindex{Brahm, Otto 05.02.1856 – 28.11.1912@\textsc{Brahm, Otto} (05.02.1856 – 28.11.1912), \emph{Theaterleiter/Theaterleiterin, Regisseur/Regisseurin}|pw} weiß es nicht und gibt \substVorne{}\textsuperscript{vor}\substDazwischen{}an\substHinten{}, den Titel überhaupt nie gehört zu haben.\pend
           
\pstart
           Brahm\pwindex{Brahm, Otto 05.02.1856 – 28.11.1912@\textsc{Brahm, Otto} (05.02.1856 – 28.11.1912), \emph{Theaterleiter/Theaterleiterin, Regisseur/Regisseurin}|pw} telegrafiert mir eben um die Änderungen,
               die ich in meinem Stück\pwindex{Meister. Komoedie in drei Akten@\emph{Der Meister. Komödie in drei Akten}|pwv} noch
               machen will. Was geht da vor? Ich denke doch, daß Du zunächſt daran kommst. Es wäre
               mir wichtig, das Datum Deiner \label{K_L01340-2v}\edtext{Première\pwindex{einsame Weg. Schauspiel in fuenf Akten@\emph{Der einsame Weg. Schauspiel in fünf Akten}|pwv}}{\lemma{\textnormal{\emph{Première}}}\Cendnote{\textnormal{von \emph{Der
                     einsame Weg}\pwindex{einsame Weg. Schauspiel in fuenf Akten@\emph{Der einsame Weg. Schauspiel in fünf Akten}|pwk}}}}\label{K_L01340-2} zu erfahren, ſo bald Du es weißt.\pend
           
\pstart
           Verzeih die Haſt dieſer Zeilen{\\[\baselineskip]}Deinem abgehetzten{\\[\baselineskip]}\spacefill\mbox{Hermann}\pend
           \leftskip=0em{}\selectlanguage{ngerman}\endnumbering\briefempfaengerindex{Schnitzler, Arthur@\textsc{Schnitzler, Arthur}!zzzBahr, Hermann@\emph{von Hermann Bahr}!1903-11-104@{10. 11. 1903}|)be}\mylabel{L01340h}  \normalsize

\doendnotes{C}
\bigskip
\vfill

\clearpage

\footnotesize

\lohead{\textsc{register}}

% Definiere theindex-Environment komplett neu ohne reledmac
\makeatletter
\renewenvironment{theindex}{%
  \section*{\indexname}%
  \setlength{\parindent}{0pt}%
  \setlength{\parskip}{0pt plus 0.3pt}%
  \let\item\@idxitem
}{%
  \clearpage
}
\makeatother

\IfFileExists{\jobname-pw.ind}{\input{\jobname-pw.ind}}{}

\end{document}

      