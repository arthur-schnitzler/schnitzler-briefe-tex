\input{../tex-inputs/latex-pdf-vorspann}
\begin{center}
            \textcolor{red}{ENTWURF. ENTZIFFERUNG NOCH NICHT KORREKTURGELESEN}
                      \end{center}
            
               \section[Jakob Julius David an Arthur Schnitzler, 4. 11. 1895]{ Jakob Julius David an Arthur Schnitzler, 4. 11. 1895}\nopagebreak\mylabel{v}\rehead{ }\begin{ledgroupsized}[t]{13cm}\normalsize\beginnumbering\briefempfaengerindex{Schnitzler, Arthur@\textsc{Schnitzler, Arthur}!zzzDavid, Jakob Julius@\emph{von Jakob Julius David}!1895-11-041@{4. 11. 1895}|(be} \toendnotes[C]{\smallbreak\pagebreak[2]} \Standort{CUL, Schnitzler, B 25.}
\physDesc{Postkarte
\newline{}Handschrift: 1) schwarze Tinte, lateinische Kurrent\hspace{1em}2) Bleistift, lateinische Kurrent (\noindent{}Vermerk der Absenderadresse)\hspace{1em}\newline{}Versand: 1) Stempel: »\nobreak{}\oindex{III., Landstrasse@\textbf{III., Landstraße}|pwk}Wien 3/1, 4. 11. 95\nobreak{}«.  2) Stempel: »\nobreak{}\oindex{IX., Alsergrund@\textbf{IX., Alsergrund}|pwk}Wien 9/2, 4. 11. 95, 5–6N, Bestellt\nobreak{}«. 3) Stempel: »\nobreak{}\oindex{IX., Alsergrund@\textbf{IX., Alsergrund}|pwk}Wien 9/3, 5. 11. 95, 8.V, Bestellt\nobreak{}«. 
\newline{}Schnitzler: mit Bleistift nummeriert: »3.« \newline{}Ordnung: mit Bleistift von unbekannter Hand nummeriert:
                                 »4« }\pstart{}{\pb}Herrn D\textsuperscript{r}
                  Arthur Schnitzler\pend{}\pstart{}IX.\oindex{IX., Alsergrund@\textbf{IX., Alsergrund}|pw}{ }Franckgaße N\textsuperscript{o} 3\oindex{Frankgasse@\textbf{Frankgasse}|pw}.
               \pend{}{\bigskip}\pstart{}{\pb}Werther Herr Doctor!\pend\pstart
           Könnten Sie mir nicht zur nächsten Vorstellung von »Liebelei\pwindex{Schnitzler, Arthur 15.05.1862 – 21.10.1931@\textsc{Schnitzler, Arthur} (15.05.1862 – 21.10.1931), \emph{Schriftsteller, Mediziner}!Liebelei. Schauspiel in drei Akten9. 10. 1895@\strich\emph{Liebelei. Schauspiel in drei Akten} {[}9. 10. 1895{]}|pw}« zwei Karten geben?\pend
           \pstart
           Ihr{\\[\baselineskip]}\spacefill\mbox{David}\pend
           \leftskip=0em{}\pstart
           \noindent{}\introOben{}II. Ob Donaustr. 59\textsuperscript{IV}\oindex{Obere Donaustrasse@\textbf{Obere Donaustraße}|pw}.\introOben{}\pend
           \endnumbering\briefempfaengerindex{Schnitzler, Arthur@\textsc{Schnitzler, Arthur}!zzzDavid, Jakob Julius@\emph{von Jakob Julius David}!1895-11-041@{4. 11. 1895}|)be}\mylabel{h}\end{ledgroupsized}  \newcommand{\dateiname}{L00511}\newcommand{\titel}{Jakob Julius David an Arthur Schnitzler, 4. 11. 1895}\newcommand{\editorInnen}{Martin Anton Müller und Gerd-Hermann Susen}\input{../tex-inputs/latex-pdf-abspann}
      