%% latex-korrekturansicht-vorspann.tex
%% Vorspann für die Korrekturansicht.
%% Lädt die gemeinsame Datei latex-vorspann.tex mit gesetztem Schalter.

\newif\ifkorrekturansicht
\korrekturansichttrue

\input{../tex-inputs/latex-vorspann}


\section[Jakob Julius David an Arthur Schnitzler, 4. 11. 1895]{L00511 Jakob Julius David an Arthur Schnitzler, 4. 11. 1895}
\nopagebreak\mylabel{L00511v}
\rehead{ }\normalsize\beginnumbering\briefempfaengerindex{Schnitzler, Arthur@\textsc{Schnitzler, Arthur}!zzzDavid, Jakob Julius@\emph{von Jakob Julius David}!1895-11-041@{4. 11. 1895}|(be}
\toendnotes[C]{\smallbreak\pagebreak[2]}\Standort{CUL, Schnitzler, B 25.}
\physDesc{Postkarte, 175 Zeichen
\newline{}Handschrift: 1) schwarze Tinte, lateinische Kurrent\hspace{1em}2) Bleistift, lateinische Kurrent (\noindent{}Vermerk der Absenderadresse)\hspace{1em}
\newline{}Versand: 1) Stempel: »\nobreak{}\oindex{III., Landstrasse@\textbf{III., Landstraße}, \emph{A.ADM3}|pwk}Wien 3/1, 4. 11. 95\nobreak{}«.   2) Stempel: »\nobreak{}\oindex{IX., Alsergrund@\textbf{IX., Alsergrund}, \emph{A.ADM3}|pwk}Wien 9/2, 4. 11. 95, 5–6N, Bestellt\nobreak{}«.  3) Stempel: »\nobreak{}\oindex{IX., Alsergrund@\textbf{IX., Alsergrund}, \emph{A.ADM3}|pwk}Wien 9/3, 5. 11. 95, 8.V, Bestellt\nobreak{}«. 
\newline{}Schnitzler: mit Bleistift nummeriert: »3.« 
\newline{}Ordnung: mit Bleistift von unbekannter Hand nummeriert:
                                 »4« }\pstart{}{\pb}Herrn D\textsuperscript{r}
                  Arthur Schnitzler\pend{}\pstart{}IX.\oindex{IX., Alsergrund@\textbf{IX., Alsergrund}, \emph{A.ADM3}|pw}{ }Franckgaße N\textsuperscript{o} 3\oindex{Frankgasse 1@\textbf{Frankgasse 1}, \emph{Wohngebäude (K.WHS)}|pw}.
               \pend{}{\bigskip}\vspace{1em}
\pstart{}{\pb}Werther Herr Doctor!\pend\vspace{0.5em}
\pstart
           Könnten Sie mir nicht zur nächsten Vorstellung von »Liebelei\pwindex{Liebelei. Schauspiel in drei Akten@\emph{Liebelei. Schauspiel in drei Akten}|pw}« zwei Karten geben?\pend
           
\pstart
           Ihr{\\[\baselineskip]}\spacefill\mbox{David}\pend
           \leftskip=0em{}
\pstart
           \noindent{}\introOben{}II. Ob Donaustr. 59\textsuperscript{IV}\oindex{Obere Donaustrasse@\textbf{Obere Donaustraße}, \emph{Straße (K.STR)}|pw}.\introOben{}\pend
           \selectlanguage{ngerman}\endnumbering\briefempfaengerindex{Schnitzler, Arthur@\textsc{Schnitzler, Arthur}!zzzDavid, Jakob Julius@\emph{von Jakob Julius David}!1895-11-041@{4. 11. 1895}|)be}\mylabel{L00511h}  \normalsize

\doendnotes{C}
\bigskip
\vfill

\clearpage

\footnotesize

\lohead{\textsc{register}}

% Definiere theindex-Environment komplett neu ohne reledmac
\makeatletter
\renewenvironment{theindex}{%
  \section*{\indexname}%
  \setlength{\parindent}{0pt}%
  \setlength{\parskip}{0pt plus 0.3pt}%
  \let\item\@idxitem
}{%
  \clearpage
}
\makeatother

\IfFileExists{\jobname-pw.ind}{\input{\jobname-pw.ind}}{}

\end{document}

      