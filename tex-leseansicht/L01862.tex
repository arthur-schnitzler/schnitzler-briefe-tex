%% latex-korrekturansicht-vorspann.tex
%% Vorspann für die Korrekturansicht.
%% Lädt die gemeinsame Datei latex-vorspann.tex mit gesetztem Schalter.

\newif\ifkorrekturansicht
\korrekturansichttrue

\input{../tex-inputs/latex-vorspann}


\section[Arthur Schnitzler an Richard Beer-Hofmann, 31. 7. 1909]{L01862 Arthur Schnitzler an Richard Beer-Hofmann, 31. 7. 1909}
\nopagebreak\mylabel{L01862v}
\rehead{ }\normalsize\beginnumbering\briefempfaengerindex{Beer-Hofmann, Richard@\textsc{Beer-Hofmann, Richard}!zzzSchnitzler, Arthur@\emph{von Arthur Schnitzler}!1909-07-311@{31. 7. 1909}|(be}
\toendnotes[C]{\smallbreak\pagebreak[2]}\Standort{YCGL, MSS 31.}
\physDesc{Brief, 1 Blatt, 4 Seiten, Umschlag, 891 Zeichen
\newline{}Handschrift: schwarze Tinte, lateinische Kurrent
\newline{}Versand: Stempel: »\nobreak{}\oindex{Edlach@\textbf{Edlach}, \emph{P.PPL}|pwk}Edlach \textcolor{gray}{bei Reichenau
                                          N.Ö.}, XII\nobreak{}«.  
\newline{}Beer-Hofmann: mit rotem Buntstift mit dem Datum der Beantwortung beschriftet:
                                    »B 4/VIII 09« }
\buchAbdrucke{\weitereDrucke{Arthur Schnitzler, Richard Beer-Hofmann: \emph{Briefwechsel 1891–1931}. Wien, Zürich: \emph{Europaverlag} 1992, S. 194.} }\toendnotes[C]{\smallbreak}\pstart{}{\pb}\textcolor{gray}{\textbf{Dr. Arthur Schnitzler}}\pend{}\pstart{}\textcolor{gray}{\textbf{Wien XVIII. Spoettelgasse 7\oindex{Edmund-Weiss-Gasse 7@\textbf{Edmund-Weiß-Gasse 7}, \emph{Wohngebäude (K.WHS)}|pw}.}}\pend{}{\bigskip}\pstart{}{\pb}Herrn Dr Richard Beer Hofmann\pend{}\pstart{}Wien XVIII\oindex{XVIII., Waehring@\textbf{XVIII., Währing}, \emph{A.ADM3}|pw}\pend{}\pstart{}Hasenauerstr. 59\oindex{Hasenauerstrasse 59@\textbf{Hasenauerstraße 59}, \emph{Wohngebäude (K.WHS)}|pw}.\pend{}{\bigskip}\vspace{1em}
\pstart
           {\pb}\textcolor{gray}{\textbf{Dr. Arthur Schnitzler}}\hfill Edlach\oindex{Edlach@\textbf{Edlach}, \emph{P.PPL}|pw}, Edlacher Hof\oindex{Hotel Edlacherhof@\textbf{Hotel Edlacherhof}, \emph{Hotel (K.HTL)}|pw}\pend
           
\pstart
           \textcolor{gray}{\textbf{Wien XVIII. Spoettelgasse 7\oindex{Edmund-Weiss-Gasse 7@\textbf{Edmund-Weiß-Gasse 7}, \emph{Wohngebäude (K.WHS)}|pw}.}}\hfill 31. 7. 09.\pend
           \vspace{0.5em}
\pstart
           lieber Richard, Ihnen allen innig theilnahmsvollen Gruß und
               Händedruck, auch von Olga\pwindex{Schnitzler, Olga 17.01.1882 – 13.01.1970@\textsc{Schnitzler, Olga} (17.01.1882 – 13.01.1970), \emph{Schauspieler/Schauspielerin, Sänger/Sängerin}|pw}. Wir wissen, wie gern
               Sie diese \label{K_L01862-1v}\edtext{Frau\pwindex{Beer, Agnes 1833-02-12 – 27.7.1909@\textsc{Beer, Agnes} (1833-02-12 – 27.7.1909)|pwv}}{\lemma{\textnormal{\emph{Frau}}}\Cendnote{\textnormal{Am 27. 7. 1909 war seine
                  Tante Agnes Beer\pwindex{Beer, Agnes 1833-02-12 – 27.7.1909@\textsc{Beer, Agnes} (1833-02-12 – 27.7.1909)|pwk} in ihrer Wohnung in Wien\oindex{Wien@\textbf{Wien}, \emph{A.ADM2}|pwk} gestorben.}}}\label{K_L01862-1} gehabt haben; es müssen traurige
               Tage für Sie sein. Schreiben Sie mir doch bald ein Wort, {\pb}wie lange Sie in Wien\oindex{Wien@\textbf{Wien}, \emph{A.ADM2}|pw} bleiben werden. Möchten Sie sich nicht doch entschliessen hieher zu
                  ko{\geminationm}en? Wir würden uns so sehr freuen und ich glaube,
               für Sie alle wäre die Luft hier, trotz gelegentlicher Mittagsschwüle (Abends immer
               kühl) sehr angenehm. Die Spaziergänge charmant, vielfältig, jeder {\pb}Art von Ansprüchen gemäß. –\pend
           
\pstart
           – Wir denken bis Ende August zu bleiben, doch wäre es sehr möglich, daß
               ich in der zweiten August Hälfte auf ca 8 Tage nach München\oindex{Muenchen@\textbf{München}, \emph{P.PPLA}|pw} gehe (aus praktischen Reinhardt\pwindex{Reinhardt, Max 09.09.1873 – 30.10.1943@\textsc{Reinhardt, Max} (09.09.1873 – 30.10.1943), \emph{Theaterleiter/Theaterleiterin, Regisseur/Regisseurin, Schauspieler/Schauspielerin}|pw} Gründen.)\pend
           
\pstart
           Lassen Sie doch recht bald hören, wie’s Ihnen Allen geht. Bei uns gut; der Bub\pwindex{Schnitzler, Heinrich 09.08.1902 – 12.07.1982@\textsc{Schnitzler, Heinrich} (09.08.1902 – 12.07.1982), \emph{Regisseur/Regisseurin, Schauspieler/Schauspielerin}|pwv} schon {\pb}ganz gesund.\pend
           
\pstart
           Herzlichst Ihr{\\[\baselineskip]}\spacefill\mbox{Arthur.}\pend
           \leftskip=0em{}\selectlanguage{ngerman}\endnumbering\briefempfaengerindex{Beer-Hofmann, Richard@\textsc{Beer-Hofmann, Richard}!zzzSchnitzler, Arthur@\emph{von Arthur Schnitzler}!1909-07-311@{31. 7. 1909}|)be}\mylabel{L01862h}  \normalsize

\doendnotes{C}
\bigskip
\vfill

\clearpage

\footnotesize

\lohead{\textsc{register}}

% Definiere theindex-Environment komplett neu ohne reledmac
\makeatletter
\renewenvironment{theindex}{%
  \section*{\indexname}%
  \setlength{\parindent}{0pt}%
  \setlength{\parskip}{0pt plus 0.3pt}%
  \let\item\@idxitem
}{%
  \clearpage
}
\makeatother

\IfFileExists{\jobname-pw.ind}{\input{\jobname-pw.ind}}{}

\end{document}

      