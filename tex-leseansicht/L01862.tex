%% latex-leseansicht-vorspann.tex
%% Vorspann für die Leseansicht.
%% Lädt die gemeinsame Datei latex-vorspann.tex mit nicht gesetztem Schalter.

\newif\ifkorrekturansicht
\korrekturansichtfalse

\input{../tex-inputs/latex-vorspann}


\section[Arthur Schnitzler an Richard Beer-Hofmann, 31. 7. 1909]{L01862 Arthur Schnitzler an Richard Beer-Hofmann, 31. 7. 1909}
\nopagebreak\mylabel{L01862v}
\rehead{ }\normalsize\beginnumbering\briefempfaengerindex{Beer-Hofmann, Richard@\textsc{Beer-Hofmann, Richard}!zzzSchnitzler, Arthur@\emph{von Arthur Schnitzler}!1909-07-311@{31. 7. 1909}|(be}
\toendnotes[C]{\smallbreak\pagebreak[2]}
\correspDesc{Versand  durch Arthur Schnitzler am 31. 7. 1909 in Edlach
\newline{}Erhalt  durch Richard Beer-Hofmann im Zeitraum [1. 8. 1909
                  – 5. 8. 1909?] in Wien}\toendnotes[C]{\smallbreak}
\Standort{YCGL, MSS 31.}
\physDesc{Brief, 1 Blatt, 4 Seiten, Kuvert, 891 Zeichen
\newline{}Handschrift: schwarze Tinte, lateinische Kurrent
\newline{}Versand: Stempel: »\nobreak{}\oindex{Edlach@\textbf{Edlach}|pwk}Edlach \textcolor{gray}{bei Reichenau
                                          N.Ö.}, XII\nobreak{}«.  
\newline{}Beer-Hofmann: mit rotem Buntstift mit dem Datum der Beantwortung beschriftet:
                                    »B 4/VIII 09« }
\buchAbdrucke{\weitereDrucke{Arthur Schnitzler, Richard Beer-Hofmann: \emph{Briefwechsel 1891–1931}. Herausgegeben von Konstanze Fliedl. Wien, Zürich: \emph{Europaverlag} 1992, S. 194.} }\toendnotes[C]{\smallbreak}\pstart{}{\pb}\textcolor{gray}{\textbf{Dr. Arthur Schnitzler}}\pend{}\pstart{}\textcolor{gray}{\textbf{Wien XVIII. Spoettelgasse 7\oindex{Wien@\textbf{Wien}!XVIII., Währing@\textbf{XVIII., Währing}!Edmund-Weiß-Gasse 7@\textbf{Edmund-Weiß-Gasse 7}, \emph{Wohngebäude}|pw}.}}\pend{}{\bigskip}\pstart{}{\pb}Herrn Dr Richard Beer Hofmann\pend{}\pstart{}Wien XVIII\oindex{XVIII., Währing@\textbf{XVIII., Währing}, \emph{Verwaltungsgebiet}|pw}\pend{}\pstart{}Hasenauerstr. 59\oindex{Wien@\textbf{Wien}!XVIII., Währing@\textbf{XVIII., Währing}!Hasenauerstraße 59@\textbf{Hasenauerstraße 59}, \emph{Wohngebäude}|pw}.\pend{}{\bigskip}\vspace{1em}
\pstart
           {\pb}\textcolor{gray}{\textbf{Dr. Arthur Schnitzler}}\hfill Edlach\oindex{Edlach@\textbf{Edlach}|pw}, Edlacher Hof\oindex{Hotel Edlacherhof@\textbf{Hotel Edlacherhof}, \emph{Hotel}|pw}\pend
           
\pstart
           \textcolor{gray}{\textbf{Wien XVIII. Spoettelgasse 7\oindex{Wien@\textbf{Wien}!XVIII., Währing@\textbf{XVIII., Währing}!Edmund-Weiß-Gasse 7@\textbf{Edmund-Weiß-Gasse 7}, \emph{Wohngebäude}|pw}.}}\hfill 31. 7. 09.\pend
           \vspace{0.5em}
\pstart
           lieber Richard, Ihnen allen innig theilnahmsvollen Gruß und
               Händedruck, auch von Olga\pwindex{Schnitzler, Olga 17.\,1.\,1882 Wien – 13.\,1.\,1970 Lugano@\textsc{Schnitzler, Olga} (17.\,1.\,1882 Wien – 13.\,1.\,1970 Lugano), \emph{Schauspielerin, Sängerin}|pw}. Wir wissen, wie gern
               Sie diese \label{K_L01862-1v}\edtext{Frau\pwindex{Beer, Agnes 12.\,2.\,1833 – 27.\,7.\,1909 Wien@\textsc{Beer, Agnes} (12.\,2.\,1833 – 27.\,7.\,1909 Wien)|pwv}}{\lemma{\textnormal{\emph{Frau}}}\Cendnote{\textnormal{Am 27. 7. 1909 war seine
                  Tante Agnes Beer\pwindex{Beer, Agnes 12.\,2.\,1833 – 27.\,7.\,1909 Wien@\textsc{Beer, Agnes} (12.\,2.\,1833 – 27.\,7.\,1909 Wien)|pwk} in ihrer Wohnung in Wien\oindex{Wien@\textbf{Wien}, \emph{Verwaltungsgebiet}|pwk} gestorben.}}}\label{K_L01862-1} gehabt haben; es müssen traurige
               Tage für Sie sein. Schreiben Sie mir doch bald ein Wort, {\pb}wie lange Sie in Wien\oindex{Wien@\textbf{Wien}, \emph{Verwaltungsgebiet}|pw} bleiben werden. Möchten Sie sich nicht doch entschliessen hieher zu
                  ko{\geminationm}en? Wir würden uns so sehr freuen und ich glaube,
               für Sie alle wäre die Luft hier, trotz gelegentlicher Mittagsschwüle (Abends immer
               kühl) sehr angenehm. Die Spaziergänge charmant, vielfältig, jeder {\pb}Art von Ansprüchen gemäß. –\pend
           
\pstart
           – Wir denken bis Ende August zu bleiben, doch wäre es sehr möglich, daß
               ich in der zweiten August Hälfte auf ca 8 Tage nach München\oindex{München@\textbf{München}|pw} gehe (aus praktischen Reinhardt\pwindex{Reinhardt, Max 9.\,9.\,1873 Baden bei Wien – 30.\,10.\,1943 New York City@\textsc{Reinhardt, Max} (9.\,9.\,1873 Baden bei Wien – 30.\,10.\,1943 New York City), \emph{Theaterleiter, Regisseur, Schauspieler}|pw} Gründen.)\pend
           
\pstart
           Lassen Sie doch recht bald hören, wie’s Ihnen Allen geht. Bei uns gut; der Bub\pwindex{Schnitzler, Heinrich 9.\,8.\,1902 Hinterbrühl – 12.\,7.\,1982 Wien@\textsc{Schnitzler, Heinrich} (9.\,8.\,1902 Hinterbrühl – 12.\,7.\,1982 Wien), \emph{Regisseur, Schauspieler}|pwv} schon {\pb}ganz gesund.\pend
           
\pstart
           Herzlichst Ihr{\\[\baselineskip]}\spacefill\mbox{Arthur.}\pend
           \leftskip=0em{}\selectlanguage{ngerman}\endnumbering\briefempfaengerindex{Beer-Hofmann, Richard@\textsc{Beer-Hofmann, Richard}!zzzSchnitzler, Arthur@\emph{von Arthur Schnitzler}!1909-07-311@{31. 7. 1909}|)be}\mylabel{L01862h}  \newcommand{\dateiname}{L01862}\newcommand{\titel}{Arthur Schnitzler an Richard Beer-Hofmann, 31. 7. 1909}\newcommand{\editorInnen}{Martin Anton Müller und Gerd-Hermann Susen}%% latex-leseansicht-abspann.tex
%% Abspann für die Leseansicht.
%% Der Schalter \ifkorrekturansicht ist bereits durch den Vorspann gesetzt.

%% latex-abspann.tex
%% Gemeinsamer Abspann für Korrekturansicht und Leseansicht.
%% Setzt den Schalter \ifkorrekturansicht voraus (gesetzt in den
%% einbindenden Dateien latex-korrekturansicht-abspann.tex bzw.
%% latex-leseansicht-abspann.tex).
%% ---------------------------------------------------------------

\normalsize

% Das esempio-Environment wird nur in der Leseansicht benötigt
\ifkorrekturansicht\else
\newenvironment{esempio}[3]%
{
    \vspace{1.5ex}
    \rlap{\underline{#1}}
    \par
    \setlength{\parindent}{0cm}
    \nopagebreak
    \leftskip=#2cm
    \rightskip=#3cm
}
{
    \par
}
\fi

\doendnotes{C}
\bigskip
\vfill

\clearpage

\footnotesize

\ifkorrekturansicht
  \lohead{\textsc{register}}
\fi

% theindex-Environment neu definieren ohne reledmac
\makeatletter
\renewenvironment{theindex}{%
  \ifkorrekturansicht
    \section*{\indexname}%
  \else
    \subsubsection*{Index der erwähnten Entitäten}%
  \fi
  \setlength{\parindent}{0pt}%
  \setlength{\parskip}{0pt plus 0.3pt}%
  \let\item\@idxitem
}{%
  \ifkorrekturansicht\clearpage\fi
}
\makeatother

\IfFileExists{\jobname-pw.ind}{\input{\jobname-pw.ind}}{}

% Quellenangabe nur in der Leseansicht
\ifkorrekturansicht\else
% Fallback-Definitionen, falls die .tex-Datei \titel etc. nicht gesetzt hat
\providecommand{\titel}{}
\providecommand{\editorInnen}{}
\providecommand{\dateiname}{\jobname}

\vspace{3cm}

\vfill

\footnotesize
\textsc{Quelle}: \titel. Herausgegeben von {\editorInnen}. In: \emph{Arthur Schnitzler: Briefwechsel mit Autorinnen und Autoren}.
 Digitale Edition, https://schnitzler-briefe.acdh.oeaw.ac.at/{\dateiname}.html (Stand \today)
\fi

\end{document}


