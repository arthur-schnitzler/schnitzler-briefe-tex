%% latex-leseansicht-vorspann.tex
%% Vorspann für die Leseansicht.
%% Lädt die gemeinsame Datei latex-vorspann.tex mit nicht gesetztem Schalter.

\newif\ifkorrekturansicht
\korrekturansichtfalse

\input{../tex-inputs/latex-vorspann}


\section[ Paul Goldmann an Arthur Schnitzler, 9. 2. {[}1897{]}]{L02802 Paul Goldmann an Arthur Schnitzler,  9. 2. [1897]}
\nopagebreak\mylabel{L02802v}
\rehead{ }\normalsize\beginnumbering\briefempfaengerindex{Schnitzler, Arthur@\textsc{Schnitzler, Arthur}!zzzGoldmann, Paul@\emph{von Paul Goldmann}!1897-02-093@{9. 2. [1897]}|(be}
\toendnotes[C]{\smallbreak\pagebreak[2]}
\correspDesc{Versand  durch Paul Goldmann am 9. 2. [1897] in Paris
\newline{}Erhalt  durch Arthur Schnitzler im Zeitraum [10. 2. 1897
                  – 14. 2. 1897?] in Wien}\toendnotes[C]{\smallbreak}
\Standort{DLA, A:Schnitzler, HS.NZ85.1.3167.}
\physDesc{Brief, 1 Blatt, 4 Seiten, 2788 Zeichen
\newline{}Handschrift: blaue Tinte, deutsche Kurrent
\newline{}Beilage: handschriftlicher Brief: 1 Blatt, 2 Seiten, schwarze Tinte,
                                 Lateinschrift 
\newline{}Schnitzler: 1) mit Bleistift das Jahr »97« vermerkt  2) mit rotem Buntstift eine Unterstreichung}\toendnotes[C]{\smallbreak}
\pstart
           {\pb}\textcolor{gray}{\textbf{\textbf{Frankfurter Zeitung\orgindex{Frankfurter Zeitung@Frankfurter Zeitung|pw}}}}\pend
           
\pstart
           \textcolor{gray}{\textbf{(\begin{otherlanguage}{french}Gazette de Francfort\end{otherlanguage}\orgindex{Frankfurter Zeitung@Frankfurter Zeitung|pw}).}}\pend
           
\pstart
           \textcolor{gray}{\textbf{\textbf{\begin{otherlanguage}{french}Fondateur M.\end{otherlanguage}{ }L. Sonnemann\pwindex{Sonnemann, Leopold 29.\,10.\,1831 Höchberg – 30.\,10.\,1909 Frankfurt am Main@\textsc{Sonnemann, Leopold} (29.\,10.\,1831 Höchberg – 30.\,10.\,1909 Frankfurt am Main), \emph{Journalist, Herausgeber}|pw}.}}}\pend
           
\pstart
           \begin{otherlanguage}{french}\textcolor{gray}{\textbf{Journal politique, financier,}}\end{otherlanguage}\pend
           
\pstart
           \begin{otherlanguage}{french}\textcolor{gray}{\textbf{commercial et littéraire.}}\end{otherlanguage}\pend
           
\pstart
           \begin{otherlanguage}{french}\textcolor{gray}{\textbf{\textbf{Paraissant trois fois par jour.}}}\end{otherlanguage}\hfill \textsc{Paris\oindex{Paris@\textbf{Paris}, \emph{Hauptstadt}|pw}}, 9. Februar.\pend
           
\pstart
           \begin{otherlanguage}{french}\textcolor{gray}{\textbf{\textbf{Bureau à Paris\oindex{Paris@\textbf{Paris}, \emph{Hauptstadt}|pw}}}}\end{otherlanguage}\pend
           
\pstart
           \begin{otherlanguage}{french}\textcolor{gray}{\textbf{\textbf{24. Rue Feydeau\oindex{rue Feydeau@\textbf{rue Feydeau}, \emph{Straße}|pw}.}}}\end{otherlanguage}\pend
           
\pstart\center{}Mein lieber Freund,\pend\vspace{0.5em}
\pstart
           Dein lieber Brief, den ich mit Ungeduld \strikeout{\textcolor{gray}{e}\textcolor{gray}{×}} erwartet habe, hat mich ein wenig erregt und beunruhigt. In einem Augenblick,
               wo{ }ſo wichtige Dinge in Deinem Leben vorgehen, biſt Du gar wortkarg; und Du ahnſt
               nicht, wie{ }ſehr dieſe allgemeinen Andeutungen, die man zu errathen verſuchen muß,
               denjenigen quälen können, der in der Ferne liebevollen Antheil an Dir nimmt und nicht
               weiß, was vorgeht. Was gibts eigentlich? Sags doch heraus mit drei klaren Worten!
               Worin liegt vor allen Dingen der »Ernſt« {\pb}der
               Verhältniſſe, von dem Du{ }ſprichſt? Biſt Du bedroht in irgend einer Weise? Du wirſt
               Dich doch nicht etwa mit Jemandem{ }ſchlagen müſſen? Dann{ }ſetze \uline{ich} mich in den Zug und komme nach Wien\oindex{Wien@\textbf{Wien}, \emph{Verwaltungsgebiet}|pw}. Und was{ }ſoll dieſe »\label{K_L02802-1v}\edtext{Flucht}{\lemma{\textnormal{\emph{Flucht}}}\Cendnote{\textnormal{Bereits am 21. 1. 1897 hatte Schnitzler im \emph{Tagebuch}\pwindex{Schnitzler, Arthur 15.\,5.\,1862 Wien – 21.\,10.\,1931 ebd.@\textsc{Schnitzler, Arthur} (15.\,5.\,1862 Wien – 21.\,10.\,1931 ebd.), \emph{Schriftsteller, Mediziner}!Tagebuch@\strich\emph{Tagebuch}|pwk} von einem »Reiseplan« geschrieben, den er für
                  eine geheime Entbindung »[g]anz ernstlich mit Mz. Rh.\pwindex{Reinhard, Marie 13.\,3.\,1871 Wien – 18.\,3.\,1899 ebd.@\textsc{Reinhard, Marie} (13.\,3.\,1871 Wien – 18.\,3.\,1899 ebd.), \emph{Gesangspädagogin}|pw}« erwog. Es wurden verschiedene Orte in Betracht gezogen. Spätestens am
                     6. 8. 1897
                  siedelte Marie Reinhard\pwindex{Reinhard, Marie 13.\,3.\,1871 Wien – 18.\,3.\,1899 ebd.@\textsc{Reinhard, Marie} (13.\,3.\,1871 Wien – 18.\,3.\,1899 ebd.), \emph{Gesangspädagogin}|pwk} in den Wien\oindex{Wien@\textbf{Wien}, \emph{Verwaltungsgebiet}|pwk}er Vorort Mauer\oindex{Wien@\textbf{Wien}!XXIII., Liesing@\textbf{XXIII., Liesing}!Mauer@\textbf{Mauer}|pwk}, wo das gemeinsame Kind\pwindex{?? [Totgeborener Sohn von Arthur Schnitzler und Marie Reinhard] 24.\,9.\,1897 Endresstraße 68 – 24.\,9.\,1897 ebd.@\textsc{?? [Totgeborener Sohn von Arthur Schnitzler und Marie Reinhard]} (24.\,9.\,1897 Endresstraße 68 – 24.\,9.\,1897 ebd.)|pwkv} tot auf die Welt kam.}}}\label{K_L02802-1}«? Wohin willſt Du gehen?
                  Komm\textcolor{gray}{’} wenigſtens \label{K_L02802-2v}\edtext{nach \textsc{Paris\oindex{Paris@\textbf{Paris}, \emph{Hauptstadt}|pw}}}{\lemma{\textnormal{\emph{nach Paris}}}\Cendnote{\textnormal{Schnitzler reiste im Frühjahr 1897 gemeinsam mit Marie
                     Reinhard\pwindex{Reinhard, Marie 13.\,3.\,1871 Wien – 18.\,3.\,1899 ebd.@\textsc{Reinhard, Marie} (13.\,3.\,1871 Wien – 18.\,3.\,1899 ebd.), \emph{Gesangspädagogin}|pwk} nach Paris\oindex{Paris@\textbf{Paris}, \emph{Hauptstadt}|pwk}. Am 12. 4. 1897 kamen sie
                  dort an. Schnitzler blieb bis zum 24. 5. 1897 und reiste
                  dann weiter nach London\oindex{London@\textbf{London}, \emph{Hauptstadt}|pwk}.}}}\label{K_L02802-2}, Liebſter, –
               hier kannſt Du in irgend einem Vorort wunderſchön und billig wohnen, ohne daß ein
               Menſch von Deiner Anweſenheit etwas zu ahnen braucht. Und wir{ }ſollen uns im \label{K_L02802-3v}\edtext{Sommer}{\lemma{\textnormal{\emph{Sommer}}}\Cendnote{\textnormal{Zwischen 19. 8. 1897 und 30. 8. 1897 sahen sich Schnitzler und Goldmann\pwindex{Goldmann, Paul 31.\,1.\,1865 Breslau – 25.\,9.\,1935 Wien@\textsc{Goldmann, Paul} (31.\,1.\,1865 Breslau – 25.\,9.\,1935 Wien), \emph{Schriftsteller, Journalist}|pwk} mehrmals
                  in Bad Ischl\oindex{Bad Ischl@\textbf{Bad Ischl}|pwk}.}}}\label{K_L02802-3} nicht wiederſehen? Ja,
               liebes Kind, willſt Du denn nach Auſtralien\oindex{Australien@\textbf{Australien}|pw}
               gehen? Und Du glaubſt, daß ich {\pb}nach{ }ſolchen
               Vorgängen auf eine Ausſprache mit Dir verzichten werde, nachdem ich Dich bisher in
               jedem gleichgiltigen Sommer anzutreffen geſucht? Wo immer und mit wem immer Du biſt,
               – ich komme hin. Und wenn Du mir dieſes Freundſchafts-Recht verſagen wollteſt, würde
               ich das{ }ſehr bitter empfinden. Und die äußeren Unannehmlichkeiten, von denen Du{ }ſprichſt, – kann ich Dir da nicht wenigſtens etwas tragen helfen? Kannſt Du nicht
               irgend etwas auf mich{ }ſchieben? Ich habe einen breiten Rücken.\pend
           
\pstart
           {\pb}Den Anlaß zu allen dieſen Vorgängen verſtehe ich
               natürlich; von dem Übrigen habe ich keine Ahnung, da ich die Verhältniſſe nicht
               kenne. Ich bitte dringend um zwei Zeilen Aufklärung.\pend
           
\pstart
           Ich{ }ſende Dir anbei einen Brief von \textsc{Thorel\pwindex{Thorel, Jean 11.\,9.\,1859 Éragny – 20.\,8.\,1916 Enghien-les-Bains@\textsc{Thorel, Jean} (11.\,9.\,1859 Éragny – 20.\,8.\,1916 Enghien-les-Bains), \emph{Übersetzer, Dramatiker}|pw}}, den ich auf eine Anfrage bei dieſem bekam.\pend
           
\pstart
           Haſt Du noch ein Exemplar von »\textsc{Mourir\pwindex{Schnitzler, Arthur 15.\,5.\,1862 Wien – 21.\,10.\,1931 ebd.@\textsc{Schnitzler, Arthur} (15.\,5.\,1862 Wien – 21.\,10.\,1931 ebd.), \emph{Schriftsteller, Mediziner}!Mourir. Roman@\strich\emph{Mourir. Roman}|pw}}«? Bitte,{ }ſende es\strikeout{,}{ }\strikeout{mit} an \textsc{Madame J. Marnière\pwindex{Marni, Jeanne 31.\,1.\,1854 Toulouse – 6.\,1.\,1910 Cannes@\textsc{Marni, Jeanne} (31.\,1.\,1854 Toulouse – 6.\,1.\,1910 Cannes), \emph{Schriftstellerin}|pw}}, \textsc{68. rue Jouffroy, Paris\oindex{Rue Jouffroy d'Abbans@\textbf{Rue Jouffroy d'Abbans}, \emph{Straße}|pw}}. Schreibe hinein: \label{K_L02802-4v}\edtext{\begin{otherlanguage}{french}\textsc{À Madame J. \strikeout{Mar}\pwindex{Marni, Jeanne 31.\,1.\,1854 Toulouse – 6.\,1.\,1910 Cannes@\textsc{Marni, Jeanne} (31.\,1.\,1854 Toulouse – 6.\,1.\,1910 Cannes), \emph{Schriftstellerin}|pwv}{ }Marni\pwindex{Marni, Jeanne 31.\,1.\,1854 Toulouse – 6.\,1.\,1910 Cannes@\textsc{Marni, Jeanne} (31.\,1.\,1854 Toulouse – 6.\,1.\,1910 Cannes), \emph{Schriftstellerin}|pw}, hommage respectueux}\end{otherlanguage}}{\lemma{\textnormal{\emph{À … respectueux}}}\Cendnote{\textnormal{An Frau J. Marni\pwindex{Marni, Jeanne 31.\,1.\,1854 Toulouse – 6.\,1.\,1910 Cannes@\textsc{Marni, Jeanne} (31.\,1.\,1854 Toulouse – 6.\,1.\,1910 Cannes), \emph{Schriftstellerin}|pwk}, respektvolle Anerkennung}}}\label{K_L02802-4}, und Deinen Namen. Es iſt eine
               geiſtvolle und liebenswürdige \textsc{\label{K_L02802-5v}\edtext{\begin{otherlanguage}{french}femme de lettres\end{otherlanguage}}{\lemma{\textnormal{\emph{femme de lettres}}}\Cendnote{\textnormal{französisch: Literatin}}}\label{K_L02802-5}{ } (\label{K_L02802-6v}\edtext{E. Voilà\pwindex{Marni, Jeanne 31.\,1.\,1854 Toulouse – 6.\,1.\,1910 Cannes@\textsc{Marni, Jeanne} (31.\,1.\,1854 Toulouse – 6.\,1.\,1910 Cannes), \emph{Schriftstellerin}|pw}}{\lemma{\textnormal{\emph{E. Voilà}}}\Cendnote{\textnormal{Pseudonym}}}\label{K_L02802-6}} der »\textsc{Vie Parisienne\orgindex{Vie Parisienne@La Vie Parisienne|pw}}«), der ich von Dir geſprochen habe.\pend
           
\pstart
           Tauſend Grüße! Dein {\\[\baselineskip]}\spacefill\mbox{Paul Goldm}\pend
           \leftskip=0em{}\selectlanguage{ngerman}\vspace{1em}{\vspace{1\baselineskip}}
\pstart
           \raggedleft{}{\pb}{[}hs. Thorel:{]} 12 rue de Milan\oindex{Rue de Milan@\textbf{Rue de Milan}, \emph{Straße}|pw}\pend
           
\pstart\center{}Cher monsieur Goldmann.\pend\vspace{0.5em}
\pstart
           \label{K_L02802-7v}\edtext{Non, rien de nouveau. Il fallait
               laiſser à Carré\pwindex{Carré, Albert 22.\,6.\,1852 Straßburg – 11.\,12.\,1938 Paris@\textsc{Carré, Albert} (22.\,6.\,1852 Straßburg – 11.\,12.\,1938 Paris), \emph{Schriftsteller, Theaterleiter, Schauspieler}|pw} quelques
                  semain\textcolor{gray}{e}s. Je les lui ai laiſsées. Maintenant, je vais le
               relancer aſsez souvent. J’ai commencé vendredi dernier. Et je
               continuerai, en rapprochant de plus en plus les distances. Il faut traquer les
               directeurs de théâtre, comme on traque les cerfs à la chasse.}{\lemma{\textnormal{\emph{Non, … chasse.}}}\Cendnote{\textnormal{französisch: Nein, nichts Neues. Es war nötig, Carré\pwindex{Carré, Albert 22.\,6.\,1852 Straßburg – 11.\,12.\,1938 Paris@\textsc{Carré, Albert} (22.\,6.\,1852 Straßburg – 11.\,12.\,1938 Paris), \emph{Schriftsteller, Theaterleiter, Schauspieler}|pwk} ein paar Wochen Zeit zu geben. Ich
                  habe sie ihm gegeben. Jetzt werde ich es immer wieder ansprechen. Ich habe
                     letzten Freitag damit angefangen. Und ich werde weitermachen,
                  indem ich die Abstände immer kleiner lassen werde. Man muss Theaterdirektoren
                  aufspüren, wie man Hirsche auf der Jagd aufspürt. }}}\label{K_L02802-7}\pend
           
\pstart
           \label{K_L02802-8v}\edtext{Signalez donc à Schnitzler, \label{K_L02802-9v}\edtext{l’article\pwindex{Wyzewa, Théodore de 12.\,9.\,1862 Kalush – 7.\,4.\,1917 Paris@\textsc{Wyzewa, Théodore de} (12.\,9.\,1862 Kalush – 7.\,4.\,1917 Paris), \emph{Schriftsteller, Journalist}!Un vaudevilliste viennois@\strich\emph{Un vaudevilliste viennois}|pwv} de Wyzewa\pwindex{Wyzewa, Théodore de 12.\,9.\,1862 Kalush – 7.\,4.\,1917 Paris@\textsc{Wyzewa, Théodore de} (12.\,9.\,1862 Kalush – 7.\,4.\,1917 Paris), \emph{Schriftsteller, Journalist}|pw}}{\lemma{\textnormal{\emph{l’article de Wyzewa}}}\Cendnote{\textnormal{Théodore de Wyzewa\pwindex{Wyzewa, Théodore de 12.\,9.\,1862 Kalush – 7.\,4.\,1917 Paris@\textsc{Wyzewa, Théodore de} (12.\,9.\,1862 Kalush – 7.\,4.\,1917 Paris), \emph{Schriftsteller, Journalist}|pwk}: \emph{Un vaudevilliste viennois}\pwindex{Wyzewa, Théodore de 12.\,9.\,1862 Kalush – 7.\,4.\,1917 Paris@\textsc{Wyzewa, Théodore de} (12.\,9.\,1862 Kalush – 7.\,4.\,1917 Paris), \emph{Schriftsteller, Journalist}!Un vaudevilliste viennois@\strich\emph{Un vaudevilliste viennois}|pwk}. In: \emph{Le Temps}\pwindex{Le Temps@\emph{Le Temps}|pwk}, Jg. 37, Nr. 13.023, 27. 1. 1897, S. 2.}}}\label{K_L02802-9} dans le \uline{Temps\pwindex{Le Temps@\emph{Le Temps}|pw}} du 27 janvier{[}.{]} J’avais dit à Wyzewa\pwindex{Wyzewa, Théodore de 12.\,9.\,1862 Kalush – 7.\,4.\,1917 Paris@\textsc{Wyzewa, Théodore de} (12.\,9.\,1862 Kalush – 7.\,4.\,1917 Paris), \emph{Schriftsteller, Journalist}|pw}
               que je traduisais\pwindex{Schnitzler, Arthur 15.\,5.\,1862 Wien – 21.\,10.\,1931 ebd.@\textsc{Schnitzler, Arthur} (15.\,5.\,1862 Wien – 21.\,10.\,1931 ebd.), \emph{Schriftsteller, Mediziner}!Amourette. Pièce en trois actes. Adaptée de Arthur Schnitzler@\strich\emph{Amourette. Pièce en trois actes. Adaptée de Arthur Schnitzler}|pwv} du
               Schnitzler, et il a ainsi cherché {\pb}à me rendre service
               par les quelques lignes\pwindex{Wyzewa, Théodore de 12.\,9.\,1862 Kalush – 7.\,4.\,1917 Paris@\textsc{Wyzewa, Théodore de} (12.\,9.\,1862 Kalush – 7.\,4.\,1917 Paris), \emph{Schriftsteller, Journalist}!Un vaudevilliste viennois@\strich\emph{Un vaudevilliste viennois}|pwv}
               extrêmement flatteuses, qu’il a consacrées à Schnitzler –}{\lemma{\textnormal{\emph{Signalez … –}}}\Cendnote{\textnormal{französisch: Bitte weisen Sie Schnitzler auf Wyzewas\pwindex{Wyzewa, Théodore de 12.\,9.\,1862 Kalush – 7.\,4.\,1917 Paris@\textsc{Wyzewa, Théodore de} (12.\,9.\,1862 Kalush – 7.\,4.\,1917 Paris), \emph{Schriftsteller, Journalist}|pwk}{ }Artikel\pwindex{Wyzewa, Théodore de 12.\,9.\,1862 Kalush – 7.\,4.\,1917 Paris@\textsc{Wyzewa, Théodore de} (12.\,9.\,1862 Kalush – 7.\,4.\,1917 Paris), \emph{Schriftsteller, Journalist}!Un vaudevilliste viennois@\strich\emph{Un vaudevilliste viennois}|pwkv} in \emph{Le Temps}\pwindex{Le Temps@\emph{Le Temps}|pwk} vom 27. Januar
                  hin. Ich hatte Wyzewa\pwindex{Wyzewa, Théodore de 12.\,9.\,1862 Kalush – 7.\,4.\,1917 Paris@\textsc{Wyzewa, Théodore de} (12.\,9.\,1862 Kalush – 7.\,4.\,1917 Paris), \emph{Schriftsteller, Journalist}|pwk} gesagt, dass ich Schnitzler{ }übersetze\pwindex{Schnitzler, Arthur 15.\,5.\,1862 Wien – 21.\,10.\,1931 ebd.@\textsc{Schnitzler, Arthur} (15.\,5.\,1862 Wien – 21.\,10.\,1931 ebd.), \emph{Schriftsteller, Mediziner}!Amourette. Pièce en trois actes. Adaptée de Arthur Schnitzler@\strich\emph{Amourette. Pièce en trois actes. Adaptée de Arthur Schnitzler}|pwkv}, und so versuchte
                  er, mir mit ein paar äußerst schmeichelhaften Zeilen\pwindex{Wyzewa, Théodore de 12.\,9.\,1862 Kalush – 7.\,4.\,1917 Paris@\textsc{Wyzewa, Théodore de} (12.\,9.\,1862 Kalush – 7.\,4.\,1917 Paris), \emph{Schriftsteller, Journalist}!Un vaudevilliste viennois@\strich\emph{Un vaudevilliste viennois}|pwkv} über Schnitzler einen Gefallen zu tun.}}}\label{K_L02802-8}\pend
           
\pstart
           \label{K_L02802-10v}\edtext{Je vous tiendrai au courant.}{\lemma{\textnormal{\emph{Je … courant.}}}\Cendnote{\textnormal{französisch: Ich werde Sie auf dem
                  Laufenden halten.}}}\label{K_L02802-10}\pend
           
\pstart
           Votre bien devoué {\\[\baselineskip]}\spacefill\mbox{Jean Thorel\pwindex{Thorel, Jean 11.\,9.\,1859 Éragny – 20.\,8.\,1916 Enghien-les-Bains@\textsc{Thorel, Jean} (11.\,9.\,1859 Éragny – 20.\,8.\,1916 Enghien-les-Bains), \emph{Übersetzer, Dramatiker}|pw}}\pend
           \leftskip=0em{}\selectlanguage{ngerman}\endnumbering\briefempfaengerindex{Schnitzler, Arthur@\textsc{Schnitzler, Arthur}!zzzGoldmann, Paul@\emph{von Paul Goldmann}!1897-02-093@{9. 2. [1897]}|)be}\mylabel{L02802h}  \newcommand{\dateiname}{L02802}\newcommand{\titel}{Paul Goldmann an Arthur Schnitzler, 9. 2. [1897]}\newcommand{\editorInnen}{Martin Anton Müller und Laura Untner}%% latex-leseansicht-abspann.tex
%% Abspann für die Leseansicht.
%% Der Schalter \ifkorrekturansicht ist bereits durch den Vorspann gesetzt.

%% latex-abspann.tex
%% Gemeinsamer Abspann für Korrekturansicht und Leseansicht.
%% Setzt den Schalter \ifkorrekturansicht voraus (gesetzt in den
%% einbindenden Dateien latex-korrekturansicht-abspann.tex bzw.
%% latex-leseansicht-abspann.tex).
%% ---------------------------------------------------------------

\normalsize

% Das esempio-Environment wird nur in der Leseansicht benötigt
\ifkorrekturansicht\else
\newenvironment{esempio}[3]%
{
    \vspace{1.5ex}
    \rlap{\underline{#1}}
    \par
    \setlength{\parindent}{0cm}
    \nopagebreak
    \leftskip=#2cm
    \rightskip=#3cm
}
{
    \par
}
\fi

\doendnotes{C}
\bigskip
\vfill

\clearpage

\footnotesize

\ifkorrekturansicht
  \lohead{\textsc{register}}
\fi

% theindex-Environment neu definieren ohne reledmac
\makeatletter
\renewenvironment{theindex}{%
  \ifkorrekturansicht
    \section*{\indexname}%
  \else
    \subsubsection*{Index der erwähnten Entitäten}%
  \fi
  \setlength{\parindent}{0pt}%
  \setlength{\parskip}{0pt plus 0.3pt}%
  \let\item\@idxitem
}{%
  \ifkorrekturansicht\clearpage\fi
}
\makeatother

\IfFileExists{\jobname-pw.ind}{\input{\jobname-pw.ind}}{}

% Quellenangabe nur in der Leseansicht
\ifkorrekturansicht\else
% Fallback-Definitionen, falls die .tex-Datei \titel etc. nicht gesetzt hat
\providecommand{\titel}{}
\providecommand{\editorInnen}{}
\providecommand{\dateiname}{\jobname}

\vspace{3cm}

\vfill

\footnotesize
\textsc{Quelle}: \titel. Herausgegeben von {\editorInnen}. In: \emph{Arthur Schnitzler: Briefwechsel mit Autorinnen und Autoren}.
 Digitale Edition, https://schnitzler-briefe.acdh.oeaw.ac.at/{\dateiname}.html (Stand \today)
\fi

\end{document}


