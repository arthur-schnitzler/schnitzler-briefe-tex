%% latex-korrekturansicht-vorspann.tex
%% Vorspann für die Korrekturansicht.
%% Lädt die gemeinsame Datei latex-vorspann.tex mit gesetztem Schalter.

\newif\ifkorrekturansicht
\korrekturansichttrue

\input{../tex-inputs/latex-vorspann}


\section[ Paul Goldmann an Arthur Schnitzler, 9. 2. {[}1897{]}]{L02802 Paul Goldmann an Arthur Schnitzler, 9. 2. {[}1897{]}}
\nopagebreak\mylabel{L02802v}
\rehead{ }\normalsize\beginnumbering\briefempfaengerindex{Schnitzler, Arthur@\textsc{Schnitzler, Arthur}!zzzGoldmann, Paul@\emph{von Paul Goldmann}!1897-02-093@{9. 2. {[}1897{]}}|(be}
\toendnotes[C]{\smallbreak\pagebreak[2]}\Standort{DLA, A:Schnitzler, HS.NZ85.1.3167.}
\physDesc{Brief, 1 Blatt, 4 Seiten, 2788 Zeichen
\newline{}Handschrift: blaue Tinte, deutsche Kurrent
\newline{}Beilage: handschriftlicher Brief: 1 Blatt, 2 Seiten, schwarze Tinte,
                                 Lateinschrift 
\newline{}Schnitzler: 1) mit Bleistift das Jahr »97« vermerkt  2) mit rotem Buntstift eine Unterstreichung}\toendnotes[C]{\smallbreak}
\pstart
           {\pb}\textcolor{gray}{\textbf{\textbf{Frankfurter Zeitung\orgindex{Frankfurter Zeitung@Frankfurter Zeitung|pw}}}}\pend
           
\pstart
           \textcolor{gray}{\textbf{(\begin{otherlanguage}{french}Gazette de Francfort\end{otherlanguage}\orgindex{Frankfurter Zeitung@Frankfurter Zeitung|pw}).}}\pend
           
\pstart
           \textcolor{gray}{\textbf{\textbf{\begin{otherlanguage}{french}Fondateur M.\end{otherlanguage}{ }L. Sonnemann\pwindex{Sonnemann, Leopold 1831-10-29 – 1909-10-30@\textsc{Sonnemann, Leopold} (1831-10-29 – 1909-10-30), \emph{Journalist/Journalistin, Herausgeber/Herausgeberin}|pw}.}}}\pend
           
\pstart
           \begin{otherlanguage}{french}\textcolor{gray}{\textbf{Journal politique, financier,}}\end{otherlanguage}\pend
           
\pstart
           \begin{otherlanguage}{french}\textcolor{gray}{\textbf{commercial et littéraire.}}\end{otherlanguage}\pend
           
\pstart
           \begin{otherlanguage}{french}\textcolor{gray}{\textbf{\textbf{Paraissant trois fois par jour.}}}\end{otherlanguage}\hfill \textsc{Paris\oindex{Paris@\textbf{Paris}, \emph{P.PPLC}|pw}}, 9. Februar.\pend
           
\pstart
           \begin{otherlanguage}{french}\textcolor{gray}{\textbf{\textbf{Bureau à Paris\oindex{Paris@\textbf{Paris}, \emph{P.PPLC}|pw}}}}\end{otherlanguage}\pend
           
\pstart
           \begin{otherlanguage}{french}\textcolor{gray}{\textbf{\textbf{24. Rue Feydeau\oindex{rue Feydeau@\textbf{rue Feydeau}, \emph{Straße (K.STR)}|pw}.}}}\end{otherlanguage}\pend
           
\pstart\center{}Mein lieber Freund,\pend\vspace{0.5em}
\pstart
           Dein lieber Brief, den ich mit Ungeduld \strikeout{\textcolor{gray}{e}\textcolor{gray}{×}} erwartet habe, hat mich ein wenig erregt und beunruhigt. In einem Augenblick,
               wo ſo wichtige Dinge in Deinem Leben vorgehen, biſt Du gar wortkarg; und Du ahnſt
               nicht, wie ſehr dieſe allgemeinen Andeutungen, die man zu errathen verſuchen muß,
               denjenigen quälen können, der in der Ferne liebevollen Antheil an Dir nimmt und nicht
               weiß, was vorgeht. Was gibts eigentlich? Sags doch heraus mit drei klaren Worten!
               Worin liegt vor allen Dingen der »Ernſt« {\pb}der
               Verhältniſſe, von dem Du ſprichſt? Biſt Du bedroht in irgend einer Weise? Du wirſt
               Dich doch nicht etwa mit Jemandem ſchlagen müſſen? Dann ſetze \uline{ich} mich in den Zug und komme nach Wien\oindex{Wien@\textbf{Wien}, \emph{A.ADM2}|pw}. Und was ſoll dieſe »\label{K_L02802-1v}\edtext{Flucht}{\lemma{\textnormal{\emph{Flucht}}}\Cendnote{\textnormal{Bereits am 21. 1. 1897 hatte Schnitzler im \emph{Tagebuch}\pwindex{Tagebuch@\emph{Tagebuch}|pwk} von einem »Reiseplan« geschrieben, den er für
                  eine geheime Entbindung »[g]anz ernstlich mit Mz. Rh.\pwindex{Reinhard, Marie 1871-03-13 – 1899-03-18@\textsc{Reinhard, Marie} (1871-03-13 – 1899-03-18), \emph{Gesangspädagoge/Gesangspädagogin}|pw}« erwog. Es wurden verschiedene Orte in Betracht gezogen. Spätestens am
                     6. 8. 1897
                  siedelte Marie Reinhard\pwindex{Reinhard, Marie 1871-03-13 – 1899-03-18@\textsc{Reinhard, Marie} (1871-03-13 – 1899-03-18), \emph{Gesangspädagoge/Gesangspädagogin}|pwk} in den Wien\oindex{Wien@\textbf{Wien}, \emph{A.ADM2}|pwk}er Vorort Mauer\oindex{Mauer@\textbf{Mauer}, \emph{eingemeindeter Ort (A.VOO)}|pwk}, wo das gemeinsame Kind\pwindex{?? [Totgeborener Sohn von Arthur Schnitzler und Marie Reinhard] 1897-09-24 – 1897-09-24@\textsc{?? [Totgeborener Sohn von Arthur Schnitzler und Marie Reinhard]} (1897-09-24 – 1897-09-24)|pwkv} tot auf die Welt kam.}}}\label{K_L02802-1}«? Wohin willſt Du gehen?
                  Komm\textcolor{gray}{’} wenigſtens \label{K_L02802-2v}\edtext{nach \textsc{Paris\oindex{Paris@\textbf{Paris}, \emph{P.PPLC}|pw}}}{\lemma{\textnormal{\emph{nach Paris}}}\Cendnote{\textnormal{Schnitzler reiste im Frühjahr 1897 gemeinsam mit Marie
                     Reinhard\pwindex{Reinhard, Marie 1871-03-13 – 1899-03-18@\textsc{Reinhard, Marie} (1871-03-13 – 1899-03-18), \emph{Gesangspädagoge/Gesangspädagogin}|pwk} nach Paris\oindex{Paris@\textbf{Paris}, \emph{P.PPLC}|pwk}. Am 12. 4. 1897 kamen sie
                  dort an. Schnitzler blieb bis zum 24. 5. 1897 und reiste
                  dann weiter nach London\oindex{London@\textbf{London}, \emph{P.PPLC}|pwk}.}}}\label{K_L02802-2}, Liebſter, –
               hier kannſt Du in irgend einem Vorort wunderſchön und billig wohnen, ohne daß ein
               Menſch von Deiner Anweſenheit etwas zu ahnen braucht. Und wir ſollen uns im \label{K_L02802-3v}\edtext{Sommer}{\lemma{\textnormal{\emph{Sommer}}}\Cendnote{\textnormal{Zwischen 19. 8. 1897 und 30. 8. 1897 sahen sich Schnitzler und Goldmann\pwindex{Goldmann, Paul 31.01.1865 – 25.09.1935@\textsc{Goldmann, Paul} (31.01.1865 – 25.09.1935), \emph{Schriftsteller/Schriftstellerin, Journalist/Journalistin}|pwk} mehrmals
                  in Bad Ischl\oindex{Bad Ischl@\textbf{Bad Ischl}, \emph{P.PPL}|pwk}.}}}\label{K_L02802-3} nicht wiederſehen? Ja,
               liebes Kind, willſt Du denn nach Auſtralien\oindex{Australien@\textbf{Australien}, \emph{A.PCLI}|pw}
               gehen? Und Du glaubſt, daß ich {\pb}nach ſolchen
               Vorgängen auf eine Ausſprache mit Dir verzichten werde, nachdem ich Dich bisher in
               jedem gleichgiltigen Sommer anzutreffen geſucht? Wo immer und mit wem immer Du biſt,
               – ich komme hin. Und wenn Du mir dieſes Freundſchafts-Recht verſagen wollteſt, würde
               ich das ſehr bitter empfinden. Und die äußeren Unannehmlichkeiten, von denen Du
               ſprichſt, – kann ich Dir da nicht wenigſtens etwas tragen helfen? Kannſt Du nicht
               irgend etwas auf mich ſchieben? Ich habe einen breiten Rücken.\pend
           
\pstart
           {\pb}Den Anlaß zu allen dieſen Vorgängen verſtehe ich
               natürlich; von dem Übrigen habe ich keine Ahnung, da ich die Verhältniſſe nicht
               kenne. Ich bitte dringend um zwei Zeilen Aufklärung.\pend
           
\pstart
           Ich ſende Dir anbei einen Brief von \textsc{Thorel\pwindex{Thorel, Jean 1859-09-11 – 1916-08-20@\textsc{Thorel, Jean} (1859-09-11 – 1916-08-20), \emph{Übersetzer/Übersetzerin, Dramatiker/Dramatikerin}|pw}}, den ich auf eine Anfrage bei dieſem bekam.\pend
           
\pstart
           Haſt Du noch ein Exemplar von »\textsc{Mourir\pwindex{Mourir. Roman@\emph{Mourir. Roman}|pw}}«? Bitte, ſende es\strikeout{,}{ }\strikeout{mit} an \textsc{Madame J. Marnière\pwindex{Marni, Jeanne 1854-01-31 – 1910-01-06@\textsc{Marni, Jeanne} (1854-01-31 – 1910-01-06), \emph{Schriftsteller/Schriftstellerin}|pw}}, \textsc{68. rue Jouffroy, Paris\oindex{Rue Jouffroy d'Abbans@\textbf{Rue Jouffroy d'Abbans}, \emph{Straße (K.STR)}|pw}}. Schreibe hinein: \label{K_L02802-4v}\edtext{\begin{otherlanguage}{french}\textsc{À Madame J. \strikeout{Mar}\pwindex{Marni, Jeanne 1854-01-31 – 1910-01-06@\textsc{Marni, Jeanne} (1854-01-31 – 1910-01-06), \emph{Schriftsteller/Schriftstellerin}|pwv}{ }Marni\pwindex{Marni, Jeanne 1854-01-31 – 1910-01-06@\textsc{Marni, Jeanne} (1854-01-31 – 1910-01-06), \emph{Schriftsteller/Schriftstellerin}|pw}, hommage respectueux}\end{otherlanguage}}{\lemma{\textnormal{\emph{À … respectueux}}}\Cendnote{\textnormal{An Frau J. Marni\pwindex{Marni, Jeanne 1854-01-31 – 1910-01-06@\textsc{Marni, Jeanne} (1854-01-31 – 1910-01-06), \emph{Schriftsteller/Schriftstellerin}|pwk}, respektvolle Anerkennung}}}\label{K_L02802-4}, und Deinen Namen. Es iſt eine
               geiſtvolle und liebenswürdige \textsc{\label{K_L02802-5v}\edtext{\begin{otherlanguage}{french}femme de lettres\end{otherlanguage}}{\lemma{\textnormal{\emph{femme de lettres}}}\Cendnote{\textnormal{französisch: Literatin}}}\label{K_L02802-5}{ } (\label{K_L02802-6v}\edtext{E. Voilà\pwindex{Marni, Jeanne 1854-01-31 – 1910-01-06@\textsc{Marni, Jeanne} (1854-01-31 – 1910-01-06), \emph{Schriftsteller/Schriftstellerin}|pw}}{\lemma{\textnormal{\emph{E. Voilà}}}\Cendnote{\textnormal{Pseudonym}}}\label{K_L02802-6}} der »\textsc{Vie Parisienne\orgindex{Vie Parisienne@La Vie Parisienne|pw}}«), der ich von Dir geſprochen habe.\pend
           
\pstart
           Tauſend Grüße! Dein {\\[\baselineskip]}\spacefill\mbox{Paul Goldm}\pend
           \leftskip=0em{}\selectlanguage{ngerman}\vspace{1em}{\vspace{1\baselineskip}}
\pstart
           \raggedleft{}{\pb}{[}hs. :{]} 12 rue de Milan\oindex{Rue de Milan@\textbf{Rue de Milan}, \emph{Straße (K.STR)}|pw}\pend
           
\pstart\center{}Cher monsieur Goldmann.\pend\vspace{0.5em}
\pstart
           \label{K_L02802-7v}\edtext{Non, rien de nouveau. Il fallait
               laiſser à Carré\pwindex{Carre, Albert 22.06.1852 – 11.12.1938@\textsc{Carré, Albert} (22.06.1852 – 11.12.1938), \emph{Schriftsteller/Schriftstellerin, Theaterleiter/Theaterleiterin, Schauspieler/Schauspielerin}|pw} quelques
                  semain\textcolor{gray}{e}s. Je les lui ai laiſsées. Maintenant, je vais le
               relancer aſsez souvent. J’ai commencé vendredi dernier. Et je
               continuerai, en rapprochant de plus en plus les distances. Il faut traquer les
               directeurs de théâtre, comme on traque les cerfs à la chasse.}{\lemma{\textnormal{\emph{Non, … chasse.}}}\Cendnote{\textnormal{französisch: Nein, nichts Neues. Es war nötig, Carré\pwindex{Carre, Albert 22.06.1852 – 11.12.1938@\textsc{Carré, Albert} (22.06.1852 – 11.12.1938), \emph{Schriftsteller/Schriftstellerin, Theaterleiter/Theaterleiterin, Schauspieler/Schauspielerin}|pwk} ein paar Wochen Zeit zu geben. Ich
                  habe sie ihm gegeben. Jetzt werde ich es immer wieder ansprechen. Ich habe
                     letzten Freitag damit angefangen. Und ich werde weitermachen,
                  indem ich die Abstände immer kleiner lassen werde. Man muss Theaterdirektoren
                  aufspüren, wie man Hirsche auf der Jagd aufspürt. }}}\label{K_L02802-7}\pend
           
\pstart
           \label{K_L02802-8v}\edtext{Signalez donc à Schnitzler, \label{K_L02802-9v}\edtext{l’article\pwindex{Un vaudevilliste viennois@\emph{Un vaudevilliste viennois}|pwv} de Wyzewa\pwindex{Wyzewa, Theodore de 1862-09-12 – 1917-04-07@\textsc{Wyzewa, Théodore de} (1862-09-12 – 1917-04-07), \emph{Schriftsteller/Schriftstellerin, Journalist/Journalistin}|pw}}{\lemma{\textnormal{\emph{l’article de Wyzewa}}}\Cendnote{\textnormal{Théodore de Wyzewa\pwindex{Wyzewa, Theodore de 1862-09-12 – 1917-04-07@\textsc{Wyzewa, Théodore de} (1862-09-12 – 1917-04-07), \emph{Schriftsteller/Schriftstellerin, Journalist/Journalistin}|pwk}: \emph{Un vaudevilliste viennois}\pwindex{Un vaudevilliste viennois@\emph{Un vaudevilliste viennois}|pwk}. In: \emph{Le Temps}\pwindex{Le Temps@\emph{Le Temps}|pwk}, Jg. 37, Nr. 13.023, 27. 1. 1897, S. 2.}}}\label{K_L02802-9} dans le \uline{Temps\pwindex{Le Temps@\emph{Le Temps}|pw}} du 27 janvier{[}.{]} J’avais dit à Wyzewa\pwindex{Wyzewa, Theodore de 1862-09-12 – 1917-04-07@\textsc{Wyzewa, Théodore de} (1862-09-12 – 1917-04-07), \emph{Schriftsteller/Schriftstellerin, Journalist/Journalistin}|pw}
               que je traduisais\pwindex{Amourette. Piece en trois actes. Adaptee de Arthur Schnitzler@\emph{Amourette. Pièce en trois actes. Adaptée de Arthur Schnitzler}|pwv} du
               Schnitzler, et il a ainsi cherché {\pb}à me rendre service
               par les quelques lignes\pwindex{Un vaudevilliste viennois@\emph{Un vaudevilliste viennois}|pwv}
               extrêmement flatteuses, qu’il a consacrées à Schnitzler –}{\lemma{\textnormal{\emph{Signalez … –}}}\Cendnote{\textnormal{französisch: Bitte weisen Sie Schnitzler auf Wyzewas\pwindex{Wyzewa, Theodore de 1862-09-12 – 1917-04-07@\textsc{Wyzewa, Théodore de} (1862-09-12 – 1917-04-07), \emph{Schriftsteller/Schriftstellerin, Journalist/Journalistin}|pwk}{ }Artikel\pwindex{Un vaudevilliste viennois@\emph{Un vaudevilliste viennois}|pwkv} in \emph{Le Temps}\pwindex{Le Temps@\emph{Le Temps}|pwk} vom 27. Januar
                  hin. Ich hatte Wyzewa\pwindex{Wyzewa, Theodore de 1862-09-12 – 1917-04-07@\textsc{Wyzewa, Théodore de} (1862-09-12 – 1917-04-07), \emph{Schriftsteller/Schriftstellerin, Journalist/Journalistin}|pwk} gesagt, dass ich Schnitzler{ }übersetze\pwindex{Amourette. Piece en trois actes. Adaptee de Arthur Schnitzler@\emph{Amourette. Pièce en trois actes. Adaptée de Arthur Schnitzler}|pwkv}, und so versuchte
                  er, mir mit ein paar äußerst schmeichelhaften Zeilen\pwindex{Un vaudevilliste viennois@\emph{Un vaudevilliste viennois}|pwkv} über Schnitzler einen Gefallen zu tun.}}}\label{K_L02802-8}\pend
           
\pstart
           \label{K_L02802-10v}\edtext{Je vous tiendrai au courant.}{\lemma{\textnormal{\emph{Je … courant.}}}\Cendnote{\textnormal{französisch: Ich werde Sie auf dem
                  Laufenden halten.}}}\label{K_L02802-10}\pend
           
\pstart
           Votre bien devoué {\\[\baselineskip]}\spacefill\mbox{Jean Thorel\pwindex{Thorel, Jean 1859-09-11 – 1916-08-20@\textsc{Thorel, Jean} (1859-09-11 – 1916-08-20), \emph{Übersetzer/Übersetzerin, Dramatiker/Dramatikerin}|pw}}\pend
           \leftskip=0em{}\selectlanguage{ngerman}\endnumbering\briefempfaengerindex{Schnitzler, Arthur@\textsc{Schnitzler, Arthur}!zzzGoldmann, Paul@\emph{von Paul Goldmann}!1897-02-093@{9. 2. {[}1897{]}}|)be}\mylabel{L02802h}  \normalsize

\doendnotes{C}
\bigskip
\vfill

\clearpage

\footnotesize

\lohead{\textsc{register}}

% Definiere theindex-Environment komplett neu ohne reledmac
\makeatletter
\renewenvironment{theindex}{%
  \section*{\indexname}%
  \setlength{\parindent}{0pt}%
  \setlength{\parskip}{0pt plus 0.3pt}%
  \let\item\@idxitem
}{%
  \clearpage
}
\makeatother

\IfFileExists{\jobname-pw.ind}{\input{\jobname-pw.ind}}{}

\end{document}

      