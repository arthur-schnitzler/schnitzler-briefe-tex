%% latex-leseansicht-vorspann.tex
%% Vorspann für die Leseansicht.
%% Lädt die gemeinsame Datei latex-vorspann.tex mit nicht gesetztem Schalter.

\newif\ifkorrekturansicht
\korrekturansichtfalse

\input{../tex-inputs/latex-vorspann}


         
         \renewcommand{\erwaehntePersonen}{Personen: Richard Beer-Hofmann, Paula Beer-Hofmann, Sandro Botticelli, Olga Schnitzler}
         \renewcommand{\erwaehnteOrte}{Orte: Florenz, Hasenauerstraße, Santa Maria Novella, Uffizien, Wien, XVIII., Währing}
         \renewcommand{\erwaehnteWerke}{Werke: Die Anbetung der heiligen drei Könige}
               \section[Arthur und Olga Schnitzler an Richard und Paula Beer-Hofmann, 5. 5. 1914]{ Arthur und Olga Schnitzler an Richard und Paula Beer-Hofmann,
               5. 5. 1914}\nopagebreak\mylabel{v}\rehead{ }\begin{ledgroupsized}[t]{13cm}\normalsize\beginnumbering\briefempfaengerindex{Beer-Hofmann, Paula@\textsc{Beer-Hofmann, Paula}!zzzSchnitzler, Olga@\emph{von Olga Schnitzler}!1914-05-051@{5. 5. 1914}|(be}\briefempfaengerindex{Beer-Hofmann, Paula@\textsc{Beer-Hofmann, Paula}!zzzSchnitzler, Arthur@\emph{von Arthur Schnitzler}!1914-05-051@{5. 5. 1914}|(be}\briefempfaengerindex{Beer-Hofmann, Richard@\textsc{Beer-Hofmann, Richard}!zzzSchnitzler, Olga@\emph{von Olga Schnitzler}!1914-05-051@{5. 5. 1914}|(be}\briefempfaengerindex{Beer-Hofmann, Richard@\textsc{Beer-Hofmann, Richard}!zzzSchnitzler, Arthur@\emph{von Arthur Schnitzler}!1914-05-051@{5. 5. 1914}|(be} \toendnotes[C]{\smallbreak\pagebreak[2]} \Standort{YCGL, MSS 31.}
\physDesc{Bildpostkarte, 159 Zeichen
\newline{}Handschrift Arthur Schnitzler: Bleistift, deutsche Kurrent\newline{}Handschrift Olga Schnitzler: Bleistift
\newline{}Versand: Stempel: »\nobreak{}\oindex{Santa Maria Novella@\textbf{Santa Maria Novella}|pwk}Firenze Ferrovia, 5. V 1914, 20–21\nobreak{}«.  }\pstart{}{\pb}Hrn \textsc{Dr. Richard BeerHofma{\geminationn}}\pend{}\pstart{}u Frau\pend{}\pstart{}\textsc{Wien XVIII\oindex{XVIII., Waehring@\textbf{XVIII., Währing}|pw}}\pend{}\pstart{}\textsc{Hasenauerstr. 59}\oindex{Hasenauerstrasse@\textbf{Hasenauerstraße}|pw}\pend{}{\bigskip}\pstart
           \noindent{}\centering{}{\pb}\textcolor{gray}{\textbf{Firenze – R. Galleria Uffizi\oindex{Uffizien@\textbf{Uffizien}|pw}}}\pend
           \pstart
           \noindent{}\centering{}\textcolor{gray}{\textbf{L’adorazione dei Re Magi\pwindex{Botticelli, Sandro 1445-03-01 – 1510-05-17@\textsc{Botticelli, Sandro} (1445-03-01 – 1510-05-17), \emph{Maler}!Anbetung der heiligen drei Koenigeum 1476@\strich\emph{Die Anbetung der heiligen drei Könige} {[}um 1476{]}|pw} (Botticelli\pwindex{Botticelli, Sandro 1445-03-01 – 1510-05-17@\textsc{Botticelli, Sandro} (1445-03-01 – 1510-05-17), \emph{Maler}|pw})}}\pend
           \pstart
           \centering{}{\pb}Florenz\oindex{Florenz@\textbf{Florenz}|pw}{ }5. 5. 914\pend
           \pstart
           Herzliche Grüße!\pend
           \pstart
           Wir haben das ſchönſte Wetter und befinden uns ſehr wohl.\pend
           \pstart
           \spacefill\mbox{Arthur.}{\\[\baselineskip]}\spacefill\mbox{{[}hs. Olga Schnitzler:{]} Olga.}\pend
           \leftskip=0em{}
         
         \endnumbering\mylabel{h}\end{ledgroupsized}  \newcommand{\dateiname}{L02177}\newcommand{\titel}{Arthur und Olga Schnitzler an Richard und Paula Beer-Hofmann, 5. 5. 1914}\newcommand{\editorInnen}{Martin Anton Müller und Gerd-Hermann Susen}%% latex-leseansicht-abspann.tex
%% Abspann für die Leseansicht.
%% Der Schalter \ifkorrekturansicht ist bereits durch den Vorspann gesetzt.

%% latex-abspann.tex
%% Gemeinsamer Abspann für Korrekturansicht und Leseansicht.
%% Setzt den Schalter \ifkorrekturansicht voraus (gesetzt in den
%% einbindenden Dateien latex-korrekturansicht-abspann.tex bzw.
%% latex-leseansicht-abspann.tex).
%% ---------------------------------------------------------------

\normalsize

% Das esempio-Environment wird nur in der Leseansicht benötigt
\ifkorrekturansicht\else
\newenvironment{esempio}[3]%
{
    \vspace{1.5ex}
    \rlap{\underline{#1}}
    \par
    \setlength{\parindent}{0cm}
    \nopagebreak
    \leftskip=#2cm
    \rightskip=#3cm
}
{
    \par
}
\fi

\doendnotes{C}
\bigskip
\vfill

\clearpage

\footnotesize

\ifkorrekturansicht
  \lohead{\textsc{register}}
\fi

% theindex-Environment neu definieren ohne reledmac
\makeatletter
\renewenvironment{theindex}{%
  \ifkorrekturansicht
    \section*{\indexname}%
  \else
    \subsubsection*{Index der erwähnten Entitäten}%
  \fi
  \setlength{\parindent}{0pt}%
  \setlength{\parskip}{0pt plus 0.3pt}%
  \let\item\@idxitem
}{%
  \ifkorrekturansicht\clearpage\fi
}
\makeatother

\IfFileExists{\jobname-pw.ind}{\input{\jobname-pw.ind}}{}

% Quellenangabe nur in der Leseansicht
\ifkorrekturansicht\else
% Fallback-Definitionen, falls die .tex-Datei \titel etc. nicht gesetzt hat
\providecommand{\titel}{}
\providecommand{\editorInnen}{}
\providecommand{\dateiname}{\jobname}

\vspace{3cm}

\vfill

\footnotesize
\textsc{Quelle}: \titel. Herausgegeben von {\editorInnen}. In: \emph{Arthur Schnitzler: Briefwechsel mit Autorinnen und Autoren}.
 Digitale Edition, https://schnitzler-briefe.acdh.oeaw.ac.at/{\dateiname}.html (Stand \today)
\fi

\end{document}


      