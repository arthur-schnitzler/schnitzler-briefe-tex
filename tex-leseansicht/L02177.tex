%% latex-korrekturansicht-vorspann.tex
%% Vorspann für die Korrekturansicht.
%% Lädt die gemeinsame Datei latex-vorspann.tex mit gesetztem Schalter.

\newif\ifkorrekturansicht
\korrekturansichttrue

\input{../tex-inputs/latex-vorspann}


\section[Arthur und Olga Schnitzler an Richard und Paula Beer-Hofmann, 5. 5. 1914]{L02177 Arthur und Olga Schnitzler an Richard und Paula Beer-Hofmann,
               5. 5. 1914}
\nopagebreak\mylabel{L02177v}
\rehead{ }\normalsize\beginnumbering\briefempfaengerindex{Beer-Hofmann, Paula@\textsc{Beer-Hofmann, Paula}!zzzSchnitzler, Olga@\emph{von Olga Schnitzler}!1914-05-051@{5. 5. 1914}|(be}\briefempfaengerindex{Beer-Hofmann, Paula@\textsc{Beer-Hofmann, Paula}!zzzSchnitzler, Arthur@\emph{von Arthur Schnitzler}!1914-05-051@{5. 5. 1914}|(be}\briefempfaengerindex{Beer-Hofmann, Richard@\textsc{Beer-Hofmann, Richard}!zzzSchnitzler, Olga@\emph{von Olga Schnitzler}!1914-05-051@{5. 5. 1914}|(be}\briefempfaengerindex{Beer-Hofmann, Richard@\textsc{Beer-Hofmann, Richard}!zzzSchnitzler, Arthur@\emph{von Arthur Schnitzler}!1914-05-051@{5. 5. 1914}|(be}
\toendnotes[C]{\smallbreak\pagebreak[2]}\Standort{YCGL, MSS 31.}
\physDesc{Bildpostkarte, 159 Zeichen
\newline{}Handschrift Arthur Schnitzler: Bleistift, deutsche Kurrent
\newline{}Handschrift Olga Schnitzler: Bleistift
\newline{}Versand: Stempel: »\nobreak{}\oindex{Bahnhof Firenze Santa Maria Novella@\textbf{Bahnhof Firenze Santa Maria Novella}, \emph{Bahnhofsgebäude (K.BHF)}|pwk}Firenze Ferrovia, 5. V 1914, 20–21\nobreak{}«.  }\pstart{}{\pb}Hrn \textsc{Dr. Richard BeerHofma{\geminationn}}\pend{}\pstart{}u Frau\pend{}\pstart{}\textsc{Wien XVIII\oindex{XVIII., Waehring@\textbf{XVIII., Währing}, \emph{A.ADM3}|pw}}\pend{}\pstart{}\textsc{Hasenauerstr. 59}\oindex{Hasenauerstrasse 59@\textbf{Hasenauerstraße 59}, \emph{Wohngebäude (K.WHS)}|pw}\pend{}{\bigskip}
\pstart
           \noindent{}\centering{}{\pb}\textcolor{gray}{\textbf{Firenze – R. Galleria Uffizi\oindex{Uffizien@\textbf{Uffizien}, \emph{Museum (K.MUS)}|pw}}}\pend
           
\pstart
           \centering{}\textcolor{gray}{\textbf{L’adorazione dei Re Magi\pwindex{Anbetung der heiligen drei Koenige@\emph{Die Anbetung der heiligen drei Könige}|pw} (Botticelli\pwindex{Botticelli, Sandro 1445-03-01 – 1510-05-17@\textsc{Botticelli, Sandro} (1445-03-01 – 1510-05-17), \emph{Maler/Malerin}|pw})}}\pend
           \vspace{1em}
\pstart
           \centering{}{\pb}Florenz\oindex{Florenz@\textbf{Florenz}, \emph{P.PPLA}|pw}{ }5. 5. 914\pend
           \vspace{0.5em}
\pstart
           Herzliche Grüße!\pend
           
\pstart
           Wir haben das ſchönſte Wetter und befinden uns ſehr wohl.\pend
           
\pstart
           \spacefill\mbox{Arthur.}{\\[\baselineskip]}\spacefill\mbox{{[}hs. :{]} Olga.}\pend
           \leftskip=0em{}\selectlanguage{ngerman}\endnumbering\briefempfaengerindex{Beer-Hofmann, Paula@\textsc{Beer-Hofmann, Paula}!zzzSchnitzler, Olga@\emph{von Olga Schnitzler}!1914-05-051@{5. 5. 1914}|)be}\briefempfaengerindex{Beer-Hofmann, Paula@\textsc{Beer-Hofmann, Paula}!zzzSchnitzler, Arthur@\emph{von Arthur Schnitzler}!1914-05-051@{5. 5. 1914}|)be}\briefempfaengerindex{Beer-Hofmann, Richard@\textsc{Beer-Hofmann, Richard}!zzzSchnitzler, Olga@\emph{von Olga Schnitzler}!1914-05-051@{5. 5. 1914}|)be}\briefempfaengerindex{Beer-Hofmann, Richard@\textsc{Beer-Hofmann, Richard}!zzzSchnitzler, Arthur@\emph{von Arthur Schnitzler}!1914-05-051@{5. 5. 1914}|)be}\mylabel{L02177h}  \normalsize

\doendnotes{C}
\bigskip
\vfill

\clearpage

\footnotesize

\lohead{\textsc{register}}

% Definiere theindex-Environment komplett neu ohne reledmac
\makeatletter
\renewenvironment{theindex}{%
  \section*{\indexname}%
  \setlength{\parindent}{0pt}%
  \setlength{\parskip}{0pt plus 0.3pt}%
  \let\item\@idxitem
}{%
  \clearpage
}
\makeatother

\IfFileExists{\jobname-pw.ind}{\input{\jobname-pw.ind}}{}

\end{document}

      