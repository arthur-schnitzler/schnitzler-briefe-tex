%% latex-korrekturansicht-vorspann.tex
%% Vorspann für die Korrekturansicht.
%% Lädt die gemeinsame Datei latex-vorspann.tex mit gesetztem Schalter.

\newif\ifkorrekturansicht
\korrekturansichttrue

\input{../tex-inputs/latex-vorspann}


\section[Arthur Schnitzler an Richard Beer-Hofmann, 2. 6. 1898]{L00798 Arthur Schnitzler an Richard Beer-Hofmann, 2. 6. 1898}
\nopagebreak\mylabel{L00798v}
\rehead{ }\normalsize\beginnumbering\briefempfaengerindex{Beer-Hofmann, Richard@\textsc{Beer-Hofmann, Richard}!zzzSchnitzler, Arthur@\emph{von Arthur Schnitzler}!1898-06-021@{2. 6. 1898}|(be}
\toendnotes[C]{\smallbreak\pagebreak[2]}\Standort{YCGL, MSS 31.}
\physDesc{Brief, 1 Blatt, 3 Seiten, Umschlag, 558 Zeichen
\newline{}Handschrift: Bleistift, deutsche Kurrent
\newline{}Versand: 1) Stempel: »\nobreak{}\oindex{IX., Alsergrund@\textbf{IX., Alsergrund}, \emph{A.ADM3}|pwk}Wien 9/3, 2. 6. 98, 1–2N\nobreak{}«.   2) Stempel: »\nobreak{}\oindex{Steindorf am Ossiacher See@\textbf{Steindorf am Ossiacher See}, \emph{A.ADM3}|pwk}Steindorf am Ossiacher
                                       See, 3 6 98\nobreak{}«. }
\buchAbdrucke{\weitereDrucke{Arthur Schnitzler, Richard Beer-Hofmann: \emph{Briefwechsel 1891–1931}. Wien, Zürich: \emph{Europaverlag} 1992, S. 116.} }\pstart{}{\pb}Herrn \textsc{Dr. Richard
                     Beer-Hofmann}\pend{}\pstart{}\textsc{Steindorf\oindex{Steindorf am Ossiacher See@\textbf{Steindorf am Ossiacher See}, \emph{A.ADM3}|pw}}\pend{}\pstart{}\textsc{am}{ }\textsc{Ossiacher}{ }See\oindex{Ossiacher See@\textbf{Ossiacher See}, \emph{See (N.SEE)}|pw}\pend{}\pstart{}in \textsc{Kärnthen}\oindex{Kaernten@\textbf{Kärnten}, \emph{A.ADM1}|pw}\pend{}{\bigskip}\vspace{1em}
\pstart{}{\pb}Lieber Richard\pend\vspace{0.5em}
\pstart
           ich habe ganz den Eindruck, als ob ich So{\geminationn}tag{ }früh von hier wegfahren und vielleicht Dinſtag am \textsc{Ossiacher}{ }See\oindex{Ossiacher See@\textbf{Ossiacher See}, \emph{See (N.SEE)}|pw} eintreffen würde. {\pb}Weiteres und näheres, was ganz dasſelbe iſt,
               merkwürdigerweiſe, weiſs ich noch nicht. Doch ſcheints mir, daſs ich ein paar Tage im
                  Annenheim\oindex{Annenheim@\textbf{Annenheim}, \emph{P.PPL}|pw} wohnen werde; eventuell radle ich
               aber mit \textsc{Kramer}\pwindex{Kramer, Leopold 29.09.1869 – 29.10.1942@\textsc{Kramer, Leopold} (29.09.1869 – 29.10.1942), \emph{Theaterleiter/Theaterleiterin, Schauspieler/Schauspielerin}|pw} weiter ins \textsc{Lavantthal}\oindex{Lavanttal@\textbf{Lavanttal}, \emph{Tal (N.TAL)}|pw}. {\pb}Vom \textsc{Semmering}\oindex{Semmering@\textbf{Semmering}, \emph{A.ADM3}|pw} aus will ich die ganze Tour per Rad machen.\pend
           
\pstart
           Ich freue mich Sie bald zu ſehen.\pend
           
\pstart
           Grüßen Sie Paula\pwindex{Beer-Hofmann, Paula 25.02.1879 – 30.10.1939@\textsc{Beer-Hofmann, Paula} (25.02.1879 – 30.10.1939)|pw}, Mirjam\pwindex{Beer-Hofmann, Mirjam 04.09.1897 – 24.12.1984@\textsc{Beer-Hofmann, Mirjam} (04.09.1897 – 24.12.1984)|pw} und ſich ſelbſt.\pend
           
\pstart
           Herzlichſt der Ihre{\\[\baselineskip]}\spacefill\mbox{Arthur}\pend
           \leftskip=0em{}\selectlanguage{ngerman}\endnumbering\briefempfaengerindex{Beer-Hofmann, Richard@\textsc{Beer-Hofmann, Richard}!zzzSchnitzler, Arthur@\emph{von Arthur Schnitzler}!1898-06-021@{2. 6. 1898}|)be}\mylabel{L00798h}  \normalsize

\doendnotes{C}
\bigskip
\vfill

\clearpage

\footnotesize

\lohead{\textsc{register}}

% Definiere theindex-Environment komplett neu ohne reledmac
\makeatletter
\renewenvironment{theindex}{%
  \section*{\indexname}%
  \setlength{\parindent}{0pt}%
  \setlength{\parskip}{0pt plus 0.3pt}%
  \let\item\@idxitem
}{%
  \clearpage
}
\makeatother

\IfFileExists{\jobname-pw.ind}{\input{\jobname-pw.ind}}{}

\end{document}

      