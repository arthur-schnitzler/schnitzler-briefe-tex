%% latex-leseansicht-vorspann.tex
%% Vorspann für die Leseansicht.
%% Lädt die gemeinsame Datei latex-vorspann.tex mit nicht gesetztem Schalter.

\newif\ifkorrekturansicht
\korrekturansichtfalse

\input{../tex-inputs/latex-vorspann}


\section[Arthur Schnitzler an Richard Beer-Hofmann, 2. 6. 1898]{L00798 Arthur Schnitzler an Richard Beer-Hofmann, 2. 6. 1898}
\nopagebreak\mylabel{L00798v}
\rehead{ }\normalsize\beginnumbering\briefempfaengerindex{Beer-Hofmann, Richard@\textsc{Beer-Hofmann, Richard}!zzzSchnitzler, Arthur@\emph{von Arthur Schnitzler}!1898-06-021@{2. 6. 1898}|(be}
\toendnotes[C]{\smallbreak\pagebreak[2]}
\correspDesc{Versand  durch Arthur Schnitzler am 2. 6. 1898 in Wien
\newline{}Erhalt  durch Richard Beer-Hofmann am 3. 6. 1898 in Steindorf am Ossiacher See}\toendnotes[C]{\smallbreak}
\Standort{YCGL, MSS 31.}
\physDesc{Brief, 1 Blatt, 3 Seiten, Kuvert, 558 Zeichen
\newline{}Handschrift: Bleistift, deutsche Kurrent
\newline{}Versand: 1) Stempel: »\nobreak{}\oindex{IX., Alsergrund@\textbf{IX., Alsergrund}, \emph{Verwaltungsgebiet}|pwk}Wien 9/3, 2. 6. 98, 1–2N\nobreak{}«.   2) Stempel: »\nobreak{}\oindex{Steindorf am Ossiacher See@\textbf{Steindorf am Ossiacher See}, \emph{Verwaltungsgebiet}|pwk}Steindorf am Ossiacher
                                       See, 3 6 98\nobreak{}«. }
\buchAbdrucke{\weitereDrucke{Arthur Schnitzler, Richard Beer-Hofmann: \emph{Briefwechsel 1891–1931}. Herausgegeben von Konstanze Fliedl. Wien, Zürich: \emph{Europaverlag} 1992, S. 116.} }\pstart{}{\pb}Herrn \textsc{Dr. Richard
                     Beer-Hofmann}\pend{}\pstart{}\textsc{Steindorf\oindex{Steindorf am Ossiacher See@\textbf{Steindorf am Ossiacher See}, \emph{Verwaltungsgebiet}|pw}}\pend{}\pstart{}\textsc{am}{ }\textsc{Ossiacher}{ }See\oindex{Ossiacher See@\textbf{Ossiacher See}, \emph{See}|pw}\pend{}\pstart{}in \textsc{Kärnthen}\oindex{Kärnten@\textbf{Kärnten}, \emph{Land}|pw}\pend{}{\bigskip}\vspace{1em}
\pstart{}{\pb}Lieber Richard\pend\vspace{0.5em}
\pstart
           ich habe ganz den Eindruck, als ob ich So{\geminationn}tag{ }früh von hier wegfahren und vielleicht Dinſtag am \textsc{Ossiacher}{ }See\oindex{Ossiacher See@\textbf{Ossiacher See}, \emph{See}|pw} eintreffen würde. {\pb}Weiteres und näheres, was ganz dasſelbe iſt,
               merkwürdigerweiſe, weiſs ich noch nicht. Doch{ }ſcheints mir, daſs ich ein paar Tage im
                  Annenheim\oindex{Annenheim@\textbf{Annenheim}|pw} wohnen werde; eventuell radle ich
               aber mit \textsc{Kramer}\pwindex{Kramer, Leopold 29.\,9.\,1869 Prag – 29.\,10.\,1942 Wien@\textsc{Kramer, Leopold} (29.\,9.\,1869 Prag – 29.\,10.\,1942 Wien), \emph{Theaterleiter, Schauspieler}|pw} weiter ins \textsc{Lavantthal}\oindex{Lavanttal@\textbf{Lavanttal}, \emph{Tal}|pw}. {\pb}Vom \textsc{Semmering}\oindex{Semmering@\textbf{Semmering}, \emph{Verwaltungsgebiet}|pw} aus will ich die ganze Tour per Rad machen.\pend
           
\pstart
           Ich freue mich Sie bald zu{ }ſehen.\pend
           
\pstart
           Grüßen Sie Paula\pwindex{Beer-Hofmann, Paula 25.\,2.\,1879 Wien – 30.\,10.\,1939 Zürich@\textsc{Beer-Hofmann, Paula} (25.\,2.\,1879 Wien – 30.\,10.\,1939 Zürich)|pw}, Mirjam\pwindex{Beer-Hofmann, Mirjam 4.\,9.\,1897 Wien – 24.\,12.\,1984 New York City@\textsc{Beer-Hofmann, Mirjam} (4.\,9.\,1897 Wien – 24.\,12.\,1984 New York City)|pw} und{ }ſich{ }ſelbſt.\pend
           
\pstart
           Herzlichſt der Ihre{\\[\baselineskip]}\spacefill\mbox{Arthur}\pend
           \leftskip=0em{}\selectlanguage{ngerman}\endnumbering\briefempfaengerindex{Beer-Hofmann, Richard@\textsc{Beer-Hofmann, Richard}!zzzSchnitzler, Arthur@\emph{von Arthur Schnitzler}!1898-06-021@{2. 6. 1898}|)be}\mylabel{L00798h}  \newcommand{\dateiname}{L00798}\newcommand{\titel}{Arthur Schnitzler an Richard Beer-Hofmann, 2. 6. 1898}\newcommand{\editorInnen}{Martin Anton Müller und Gerd-Hermann Susen}%% latex-leseansicht-abspann.tex
%% Abspann für die Leseansicht.
%% Der Schalter \ifkorrekturansicht ist bereits durch den Vorspann gesetzt.

%% latex-abspann.tex
%% Gemeinsamer Abspann für Korrekturansicht und Leseansicht.
%% Setzt den Schalter \ifkorrekturansicht voraus (gesetzt in den
%% einbindenden Dateien latex-korrekturansicht-abspann.tex bzw.
%% latex-leseansicht-abspann.tex).
%% ---------------------------------------------------------------

\normalsize

% Das esempio-Environment wird nur in der Leseansicht benötigt
\ifkorrekturansicht\else
\newenvironment{esempio}[3]%
{
    \vspace{1.5ex}
    \rlap{\underline{#1}}
    \par
    \setlength{\parindent}{0cm}
    \nopagebreak
    \leftskip=#2cm
    \rightskip=#3cm
}
{
    \par
}
\fi

\doendnotes{C}
\bigskip
\vfill

\clearpage

\footnotesize

\ifkorrekturansicht
  \lohead{\textsc{register}}
\fi

% theindex-Environment neu definieren ohne reledmac
\makeatletter
\renewenvironment{theindex}{%
  \ifkorrekturansicht
    \section*{\indexname}%
  \else
    \subsubsection*{Index der erwähnten Entitäten}%
  \fi
  \setlength{\parindent}{0pt}%
  \setlength{\parskip}{0pt plus 0.3pt}%
  \let\item\@idxitem
}{%
  \ifkorrekturansicht\clearpage\fi
}
\makeatother

\IfFileExists{\jobname-pw.ind}{\input{\jobname-pw.ind}}{}

% Quellenangabe nur in der Leseansicht
\ifkorrekturansicht\else
% Fallback-Definitionen, falls die .tex-Datei \titel etc. nicht gesetzt hat
\providecommand{\titel}{}
\providecommand{\editorInnen}{}
\providecommand{\dateiname}{\jobname}

\vspace{3cm}

\vfill

\footnotesize
\textsc{Quelle}: \titel. Herausgegeben von {\editorInnen}. In: \emph{Arthur Schnitzler: Briefwechsel mit Autorinnen und Autoren}.
 Digitale Edition, https://schnitzler-briefe.acdh.oeaw.ac.at/{\dateiname}.html (Stand \today)
\fi

\end{document}


