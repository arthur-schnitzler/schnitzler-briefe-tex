%% latex-leseansicht-vorspann.tex
%% Vorspann für die Leseansicht.
%% Lädt die gemeinsame Datei latex-vorspann.tex mit nicht gesetztem Schalter.

\newif\ifkorrekturansicht
\korrekturansichtfalse

\input{../tex-inputs/latex-vorspann}


         
         \renewcommand{\erwaehntePersonen}{Personen: Albert Ehrenstein, Franz Grillparzer, Karl Gutzkow, Friedrich Halm,  Walther von der Vogelweide}
         \renewcommand{\erwaehnteInstitutionen}{Institutionen: Burgtheater, Franz-Grillparzer-Preis}
         \renewcommand{\erwaehnteOrte}{Orte: Ottakringerstraße, Wien, Österreich}
         \renewcommand{\erwaehnteWerke}{Werke: Der Schleier der Beatrice. Schauspiel in fünf Akten, Der Weg ins Freie. Roman, Die neue Rundschau, Zwischenspiel. Komödie in drei Akten}
               \section[Albert Ehrenstein an Arthur Schnitzler, 16. 1. 1908]{ Albert Ehrenstein an Arthur Schnitzler, 16. 1. 1908}\nopagebreak\mylabel{v}\rehead{ }\begin{ledgroupsized}[t]{13cm}\normalsize\beginnumbering \toendnotes[C]{\smallbreak\pagebreak[2]} \Standort{CUL, Schnitzler, B 30.}
\physDesc{Brief, 1 Blatt, 4 Seiten, 1742 Zeichen
\newline{}Handschrift: schwarze Tinte, deutsche Kurrent
\newline{}Schnitzler: mit Bleistift Vermerk: »\textsc{A. Ehrenstein}« und neben das Datum die richtige Jahreszahl »08« geschrieben }\buchAbdrucke{\weitereDrucke{Albert Ehrenstein: \emph{Briefe}. Hg. Hanni Mittelmann. München: \emph{Boer} 1989, S. 21 (Werke, 1).} }\toendnotes[C]{\smallbreak}\pstart
           \raggedleft{}{\pb}\textsc{Wien, XVI. Ottakringerstr 114\oindex{Ottakringerstrasse@\textbf{Ottakringerstraße}|pw}}\pend
           \pstart
           \textsc{16. Januar \label{T_L01751-1v}\edtext{07}{\lemma{\textnormal{\emph{07}}}\Cendnote{\textnormal{Schreibirrtum}}}\label{T_L01751-1h}}\pend
           \pstart\center{}\textsc{Sehr geehrter Herr Doktor!}\pend\pstart
           Zu den vielen \label{K_L01751-1v}\edtext{Glückwünſchen}{\lemma{\textnormal{\emph{Glückwünſchen}}}\Cendnote{\textnormal{Das Auswahlkomitee hatte am 15. 1. 1908
                  entschieden, dass Schnitzler\pwindex{Schnitzler, Arthur 15.05.1862 – 21.10.1931@\textsc{Schnitzler, Arthur} (15.05.1862 – 21.10.1931), \emph{Schriftsteller, Mediziner}|pwk} für seine
                  Komödie \emph{Zwischenspiel}\pwindex{Schnitzler, Arthur 15.05.1862 – 21.10.1931@\textsc{Schnitzler, Arthur} (15.05.1862 – 21.10.1931), \emph{Schriftsteller, Mediziner}!Zwischenspiel. Komoedie in drei Akten1905-10-12@\strich\emph{Zwischenspiel. Komödie in drei Akten} {[}1905-10-12{]}|pwk} der mit
                  5.000 Kronen dotierte \emph{Grillparzer-Preis}\orgindex{Franz-Grillparzer-Preis@Franz-Grillparzer-Preis|pwk}
                  verliehen würde. In den Jahren zuvor war er zwar immer wieder als Favorit
                  gehandelt worden, doch stellte das Zerwürfnis mit dem \emph{Burgtheater}\orgindex{Burgtheater@Burgtheater|pwk} in Folge der Rückgabe von \emph{Der Schleier der Beatrice}\pwindex{Schnitzler, Arthur 15.05.1862 – 21.10.1931@\textsc{Schnitzler, Arthur} (15.05.1862 – 21.10.1931), \emph{Schriftsteller, Mediziner}!Schleier der Beatrice. Schauspiel in fuenf Akten1900-12-01@\strich\emph{Der Schleier der Beatrice. Schauspiel in fünf Akten} {[}1900-12-01{]}|pwk} (1901) ein
                  Hindernis dar. Seit Sommer 1905 war der Konflikt behoben und Schnitzler\pwindex{Schnitzler, Arthur 15.05.1862 – 21.10.1931@\textsc{Schnitzler, Arthur} (15.05.1862 – 21.10.1931), \emph{Schriftsteller, Mediziner}|pwk} konnte wieder bei der Preisvergabe\orgindex{Franz-Grillparzer-Preis@Franz-Grillparzer-Preis|pwkv} berücksichtigt
                  werden. }}}\label{K_L01751-1h}, die Sie, ſehr verehrter Herr Doktor, in dieſen Tagen überfliegen
               werden, auch meine beſcheidene Gratulation.\pend
           \pstart
           Dürfte doch dieſe öſterreichiſch\oindex{Oesterreich@\textbf{Österreich}|pw}{ }ſo unverzeihlich lang hinausgezögerte Ehrung, die
               nun, ſchwer vermeidbar geworden, nicht einmal auf deren Urheber zurückfällt,
               geſchweige denn ihren Zweck erreicht, manchen, und unter ihren auch mich,
               möglicherweiſe mehr und inniger gefreut haben als den Geehrten ſelbſt, dem die jetzt
               mit üblicher Rückſichtsloſigkeit hereinbrechende Briefflut vielleicht beſchwerlich
               fällt {\pb}und die Freude verkümmert. Aber
               auch ſo muß man einigermaßen froh ſein, daß ſich die Dinge etwas gebeſſert haben,
               indem ſich auch bei uns ſogar akademiſche Preisrichter dem längſt feſtſtehenden
               Urteil der Verſtändigen bequemten. Denn gewiß: hätte es zu Grillparzer\pwindex{Grillparzer, Franz 15.01.1791 – 21.01.1872@\textsc{Grillparzer, Franz} (15.01.1791 – 21.01.1872), \emph{Schriftsteller, Beamter}|pw}s Zeiten etwa einen Walther von der Vogelweide\pwindex{Walther von der Vogelweide um 1170 – um 1230@\textsc{Walther von der Vogelweide} (um 1170 – um 1230), \emph{Dichter}|pw}-Preis gegeben, alle möglichen Halme\pwindex{Halm, Friedrich 02.04.1806 – 22.05.1871@\textsc{Halm, Friedrich} (02.04.1806 – 22.05.1871), \emph{Schriftsteller}|pw} und Gutzkows\pwindex{Gutzkow, Karl 17.03.1811 – 16.12.1878@\textsc{Gutzkow, Karl} (17.03.1811 – 16.12.1878), \emph{Schriftsteller}|pw}
               hätten ihn erbuckelt, nur nicht den Wien\oindex{Wien@\textbf{Wien}|pw}er Dichter\pwindex{Grillparzer, Franz 15.01.1791 – 21.01.1872@\textsc{Grillparzer, Franz} (15.01.1791 – 21.01.1872), \emph{Schriftsteller, Beamter}|pwv} hätte man durch ihn
               zu neuem Leben aufgerufen.\pend
           \pstart
           Jedenfalls, der Wunſch, ſolche und ähnliche Auszeichnung durch wiederholte {\pb}Verleihung an den ihrer Würdigſten ebenſo
               lächerlicher als trauriger Parteilichkeit entzogen zu ſehen, kommt mir aus dem
               Herzen. Habe ich doch Ihnen, ſehr geehrter Herr Doktor, nichts Kleines zu danken:
               Troſt in der Krankheit, Ermunterung im Trübſinn, Anregung aus Ihren Werken –
               namentlich dem prämierten Stücke\pwindex{Schnitzler, Arthur 15.05.1862 – 21.10.1931@\textsc{Schnitzler, Arthur} (15.05.1862 – 21.10.1931), \emph{Schriftsteller, Mediziner}!Zwischenspiel. Komoedie in drei Akten1905-10-12@\strich\emph{Zwischenspiel. Komödie in drei Akten} {[}1905-10-12{]}|pwv}. Und wenn es mir gegönnt war, bloß den \label{K_L01751-2v}\edtext{Anfang}{\lemma{\textnormal{\emph{Anfang}}}\Cendnote{\textnormal{Der erste
                  von sechs Teilen des Vorabdrucks von \emph{Der Weg ins
                     Freie}\pwindex{Schnitzler, Arthur 15.05.1862 – 21.10.1931@\textsc{Schnitzler, Arthur} (15.05.1862 – 21.10.1931), \emph{Schriftsteller, Mediziner}!Weg ins Freie. Roman1.1.1908 – 1.6.1908@\strich\emph{Der Weg ins Freie. Roman} {[}1.1.1908 – 1.6.1908{]}|pwk} wurde im Anfang des Monats ausgegebenen Januar-Heft der
                     \emph{Neuen Rundschau}\pwindex{?? Werk@Nicht ermittelte Verfasserinnen und Verfasser!neue Rundschau1904@\emph{Die neue Rundschau} {[}1904{]}|pwk} (S. 31–71)
                  gedruckt.}}}\label{K_L01751-2h} Ihres neuen Romans\pwindex{Schnitzler, Arthur 15.05.1862 – 21.10.1931@\textsc{Schnitzler, Arthur} (15.05.1862 – 21.10.1931), \emph{Schriftsteller, Mediziner}!Weg ins Freie. Roman1.1.1908 – 1.6.1908@\strich\emph{Der Weg ins Freie. Roman} {[}1.1.1908 – 1.6.1908{]}|pwv} mehrmals mit ſtets erneutem Entzücken zu leſen, haben Sie, ſehr
               geehrter Herr Doktor, daran keinen geringen Anteil.\pend
           \pstart
           {\pb}Indem ich noch für dieſe Beläſtigung um
               Entſchuldigung bitte, verbleibe ich\pend
           \pstart
           Hochachtungsvoll{\\[\baselineskip]}Ihr Ergebenſter{\\[\baselineskip]}\spacefill\mbox{Albert Ehrenstein}\pend
           \leftskip=0em{}
         
         \endnumbering\mylabel{h}\end{ledgroupsized}  \newcommand{\dateiname}{L01751}\newcommand{\titel}{Albert Ehrenstein an Arthur Schnitzler, 16. 1. 1908}\newcommand{\editorInnen}{Martin Anton Müller und Gerd-Hermann Susen}%% latex-leseansicht-abspann.tex
%% Abspann für die Leseansicht.
%% Der Schalter \ifkorrekturansicht ist bereits durch den Vorspann gesetzt.

%% latex-abspann.tex
%% Gemeinsamer Abspann für Korrekturansicht und Leseansicht.
%% Setzt den Schalter \ifkorrekturansicht voraus (gesetzt in den
%% einbindenden Dateien latex-korrekturansicht-abspann.tex bzw.
%% latex-leseansicht-abspann.tex).
%% ---------------------------------------------------------------

\normalsize

% Das esempio-Environment wird nur in der Leseansicht benötigt
\ifkorrekturansicht\else
\newenvironment{esempio}[3]%
{
    \vspace{1.5ex}
    \rlap{\underline{#1}}
    \par
    \setlength{\parindent}{0cm}
    \nopagebreak
    \leftskip=#2cm
    \rightskip=#3cm
}
{
    \par
}
\fi

\doendnotes{C}
\bigskip
\vfill

\clearpage

\footnotesize

\ifkorrekturansicht
  \lohead{\textsc{register}}
\fi

% theindex-Environment neu definieren ohne reledmac
\makeatletter
\renewenvironment{theindex}{%
  \ifkorrekturansicht
    \section*{\indexname}%
  \else
    \subsubsection*{Index der erwähnten Entitäten}%
  \fi
  \setlength{\parindent}{0pt}%
  \setlength{\parskip}{0pt plus 0.3pt}%
  \let\item\@idxitem
}{%
  \ifkorrekturansicht\clearpage\fi
}
\makeatother

\IfFileExists{\jobname-pw.ind}{\input{\jobname-pw.ind}}{}

% Quellenangabe nur in der Leseansicht
\ifkorrekturansicht\else
% Fallback-Definitionen, falls die .tex-Datei \titel etc. nicht gesetzt hat
\providecommand{\titel}{}
\providecommand{\editorInnen}{}
\providecommand{\dateiname}{\jobname}

\vspace{3cm}

\vfill

\footnotesize
\textsc{Quelle}: \titel. Herausgegeben von {\editorInnen}. In: \emph{Arthur Schnitzler: Briefwechsel mit Autorinnen und Autoren}.
 Digitale Edition, https://schnitzler-briefe.acdh.oeaw.ac.at/{\dateiname}.html (Stand \today)
\fi

\end{document}


      