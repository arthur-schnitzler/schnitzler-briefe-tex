%% latex-korrekturansicht-vorspann.tex
%% Vorspann für die Korrekturansicht.
%% Lädt die gemeinsame Datei latex-vorspann.tex mit gesetztem Schalter.

\newif\ifkorrekturansicht
\korrekturansichttrue

\input{../tex-inputs/latex-vorspann}


\section[Albert Ehrenstein an Arthur Schnitzler, 16. 1. 1908]{L01751 Albert Ehrenstein an Arthur Schnitzler, 16. 1. 1908}
\nopagebreak\mylabel{L01751v}
\rehead{ }\normalsize\beginnumbering\briefempfaengerindex{Schnitzler, Arthur@\textsc{Schnitzler, Arthur}!zzzEhrenstein, Albert@\emph{von Albert Ehrenstein}!1908-01-161@{16. 1. 1908}|(be}
\toendnotes[C]{\smallbreak\pagebreak[2]}\Standort{CUL, Schnitzler, B 30.}
\physDesc{Brief, 1 Blatt, 4 Seiten, 1742 Zeichen
\newline{}Handschrift: schwarze Tinte, deutsche Kurrent
\newline{}Schnitzler: mit Bleistift Vermerk: »\textsc{A. Ehrenstein}« und neben das Datum die richtige Jahreszahl »08« geschrieben }
\buchAbdrucke{\weitereDrucke{Albert Ehrenstein: \emph{Briefe}. München: \emph{Boer} 1989, S. 21.} }\toendnotes[C]{\smallbreak}
\pstart
           \raggedleft{}{\pb}\textsc{Wien, XVI. Ottakringerstr 114\oindex{Ottakringer Strasse@\textbf{Ottakringer Straße}, \emph{Straße (K.STR)}|pw}}\pend
           
\pstart
           \textsc{16. Januar \label{T_L01751-1v}\edtext{07}{\lemma{\textnormal{\emph{07}}}\Cendnote{\textnormal{Schreibirrtum}}}\label{T_L01751-1}}\pend
           
\pstart\center{}\textsc{Sehr geehrter Herr Doktor!}\pend\vspace{0.5em}
\pstart
           Zu den vielen \label{K_L01751-1v}\edtext{Glückwünſchen}{\lemma{\textnormal{\emph{Glückwünſchen}}}\Cendnote{\textnormal{Das Auswahlkomitee hatte am 15. 1. 1908
               entschieden, Schnitzler für seine
               Komödie \emph{Zwischenspiel}\pwindex{Zwischenspiel. Komoedie in drei Akten@\emph{Zwischenspiel. Komödie in drei Akten}|pwk} den mit 5000 Kronen
               dotierten \emph{Grillparzer-Preis}\orgindex{Franz-Grillparzer-Preis@Franz-Grillparzer-Preis|pwk} zu verleihen. In
               den Jahren zuvor war er zwar immer wieder als Favorit gehandelt worden, doch
               stellte das Zerwürfnis mit dem \emph{Burgtheater}\orgindex{Burgtheater@Burgtheater|pwk} in
               Folge der Rückgabe von \emph{Der Schleier der
                  Beatrice}\pwindex{Schleier der Beatrice. Schauspiel in fuenf Akten@\emph{Der Schleier der Beatrice. Schauspiel in fünf Akten}|pwk} (1901) ein Hindernis dar. Seit Sommer 1905 war der Konflikt behoben und Schnitzler konnte wieder bei der Preisvergabe\orgindex{Franz-Grillparzer-Preis@Franz-Grillparzer-Preis|pwkv} berücksichtigt
               werden.}}}\label{K_L01751-1}, die Sie, ſehr verehrter Herr Doktor, in dieſen Tagen überfliegen
               werden, auch meine beſcheidene Gratulation.\pend
           
\pstart
           Dürfte doch dieſe öſterreichiſch\oindex{Oesterreich@\textbf{Österreich}, \emph{A.PCLI}|pw}{ }ſo unverzeihlich lang hinausgezögerte Ehrung, die
               nun, ſchwer vermeidbar geworden, nicht einmal auf deren Urheber zurückfällt,
               geſchweige denn ihren Zweck erreicht, manchen, und unter ihren auch mich,
               möglicherweiſe mehr und inniger gefreut haben als den Geehrten ſelbſt, dem die jetzt
               mit üblicher Rückſichtsloſigkeit hereinbrechende Briefflut vielleicht beſchwerlich
               fällt {\pb}und die Freude verkümmert. Aber
               auch ſo muß man einigermaßen froh ſein, daß ſich die Dinge etwas gebeſſert haben,
               indem ſich auch bei uns ſogar akademiſche Preisrichter dem längſt feſtſtehenden
               Urteil der Verſtändigen bequemten. Denn gewiß: hätte es zu Grillparzers\pwindex{Grillparzer, Franz 15.01.1791 – 21.01.1872@\textsc{Grillparzer, Franz} (15.01.1791 – 21.01.1872), \emph{Schriftsteller/Schriftstellerin, Beamter/Beamte}|pw} Zeiten etwa einen Walther von der Vogelweide\pwindex{Walther von der Vogelweide um 1170 – um 1230@\textsc{Walther von der Vogelweide} (um 1170 – um 1230), \emph{Dichter/Dichterin}|pw}-Preis gegeben, alle möglichen Halme\pwindex{Halm, Friedrich 02.04.1806 – 22.05.1871@\textsc{Halm, Friedrich} (02.04.1806 – 22.05.1871), \emph{Schriftsteller/Schriftstellerin}|pw} und Gutzkows\pwindex{Gutzkow, Karl 17.03.1811 – 16.12.1878@\textsc{Gutzkow, Karl} (17.03.1811 – 16.12.1878), \emph{Schriftsteller/Schriftstellerin}|pw}
               hätten ihn erbuckelt, nur nicht den Wien\oindex{Wien@\textbf{Wien}, \emph{A.ADM2}|pw}er Dichter\pwindex{Grillparzer, Franz 15.01.1791 – 21.01.1872@\textsc{Grillparzer, Franz} (15.01.1791 – 21.01.1872), \emph{Schriftsteller/Schriftstellerin, Beamter/Beamte}|pwv} hätte man durch ihn
               zu neuem Leben aufgerufen.\pend
           
\pstart
           Jedenfalls, der Wunſch, ſolche und ähnliche Auszeichnung durch wiederholte {\pb}Verleihung an den ihrer Würdigſten ebenſo
               lächerlicher als trauriger Parteilichkeit entzogen zu ſehen, kommt mir aus dem
               Herzen. Habe ich doch Ihnen, ſehr geehrter Herr Doktor, nichts Kleines zu danken:
               Troſt in der Krankheit, Ermunterung im Trübſinn, Anregung aus Ihren Werken –
               namentlich dem prämierten Stücke\pwindex{Zwischenspiel. Komoedie in drei Akten@\emph{Zwischenspiel. Komödie in drei Akten}|pwv}. Und wenn es mir gegönnt war, bloß den \label{K_L01751-2v}\edtext{Anfang}{\lemma{\textnormal{\emph{Anfang}}}\Cendnote{\textnormal{Der erste
                  von sechs Teilen des Vorabdrucks von \emph{Der Weg ins
                     Freie}\pwindex{Weg ins Freie. Roman@\emph{Der Weg ins Freie. Roman}|pwk} war im Anfang des Monats ausgegebenen Januar-Heft der
                     \emph{Neuen Rundschau}\pwindex{neue Rundschau@\emph{Die neue Rundschau}|pwk} (Jg. 19, H. 1, S. 31–71)
                  gedruckt.}}}\label{K_L01751-2} Ihres neuen Romans\pwindex{Weg ins Freie. Roman@\emph{Der Weg ins Freie. Roman}|pwv} mehrmals mit ſtets erneutem Entzücken zu leſen, haben Sie, ſehr
               geehrter Herr Doktor, daran keinen geringen Anteil.\pend
           
\pstart
           {\pb}Indem ich noch für dieſe Beläſtigung um
               Entſchuldigung bitte, verbleibe ich\pend
           
\pstart
           Hochachtungsvoll{\\[\baselineskip]}Ihr Ergebenſter{\\[\baselineskip]}\spacefill\mbox{Albert Ehrenstein}\pend
           \leftskip=0em{}\selectlanguage{ngerman}\endnumbering\briefempfaengerindex{Schnitzler, Arthur@\textsc{Schnitzler, Arthur}!zzzEhrenstein, Albert@\emph{von Albert Ehrenstein}!1908-01-161@{16. 1. 1908}|)be}\mylabel{L01751h}  \normalsize

\doendnotes{C}
\bigskip
\vfill

\clearpage

\footnotesize

\lohead{\textsc{register}}

% Definiere theindex-Environment komplett neu ohne reledmac
\makeatletter
\renewenvironment{theindex}{%
  \section*{\indexname}%
  \setlength{\parindent}{0pt}%
  \setlength{\parskip}{0pt plus 0.3pt}%
  \let\item\@idxitem
}{%
  \clearpage
}
\makeatother

\IfFileExists{\jobname-pw.ind}{\input{\jobname-pw.ind}}{}

\end{document}

      