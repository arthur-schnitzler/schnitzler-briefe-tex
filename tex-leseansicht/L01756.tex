%% latex-korrekturansicht-vorspann.tex
%% Vorspann für die Korrekturansicht.
%% Lädt die gemeinsame Datei latex-vorspann.tex mit gesetztem Schalter.

\newif\ifkorrekturansicht
\korrekturansichttrue

\input{../tex-inputs/latex-vorspann}


\section[Arthur Schnitzler an Hermann Bahr, 24. 1. 1908]{L01756 Arthur Schnitzler an Hermann Bahr, 24. 1. 1908}
\nopagebreak\mylabel{L01756v}
\rehead{ }\normalsize\beginnumbering\briefempfaengerindex{Bahr, Hermann@\textsc{Bahr, Hermann}!zzzSchnitzler, Arthur@\emph{von Arthur Schnitzler}!1908-01-291@{24. 1. 1908}|(be}
\toendnotes[C]{\smallbreak\pagebreak[2]}\Standort{TMW, HS AM 23380 Ba.}
\physDesc{Brief, 1 Blatt, 1 Seite, 126 Zeichen
\newline{}Handschrift: blaue Tinte, deutsche Kurrent
\newline{}Ordnung: Lochung }
\buchAbdrucke{\weitereDrucke{1) Arthur Schnitzler: \emph{The Letters of Arthur Schnitzler to Hermann Bahr}. Chapel Hill: \emph{The University of North Carolina Press} 1978, S. 101.} \weitereDrucke{2) Hermann Bahr, Arthur Schnitzler: \emph{Briefwechsel, Aufzeichnungen, Dokumente (1891–1931)}. Göttingen: \emph{Wallstein} 2018, S. 401.} }\toendnotes[C]{\smallbreak}
\pstart
           {\pb}\textcolor{gray}{\textbf{Dr. Arthur Schnitzler}}\hfill 24. 1. 908.\pend
           
\pstart
           \textcolor{gray}{\textbf{Wien XVIII. Spoettelgasse 7\oindex{Edmund-Weiss-Gasse 7@\textbf{Edmund-Weiß-Gasse 7}, \emph{Wohngebäude (K.WHS)}|pw}.}}\pend
           
\pstart{}lieber Hermann, \pend\vspace{0.5em}
\pstart
           ich danke dir herzlichſt für deinen freundlichen \label{K_L01756-1v}\edtext{Glückwunſch}{\lemma{\textnormal{\emph{Glückwunſch}}}\Cendnote{\textnormal{Der Glückwunsch,
               mit dem Bahr\pwindex{Bahr, Hermann 19.07.1863 – 15.01.1934@\textsc{Bahr, Hermann} (19.07.1863 – 15.01.1934), \emph{Schriftsteller/Schriftstellerin, Kritiker/Kritikerin}|pwk} zur Zuerkennung des \emph{Grillparzer-Preises}\orgindex{Franz-Grillparzer-Preis@Franz-Grillparzer-Preis|pwk} für \emph{Zwischenspiel}\pwindex{Unter sich. Ein Arme-Leut -Stueck@\emph{Unter sich. Ein Arme-Leut’-Stück}|pwk} gratulierte,
               ist nicht überliefert. Die Jurysitzung hatte am 15. 1. 1908 stattgefunden, am Folgetag hatten die Zeitungen
                  darüber berichtet.}}}\label{K_L01756-1}! Wie lang biſt du noch in Wien\oindex{Wien@\textbf{Wien}, \emph{A.ADM2}|pw}?\pend
           
\pstart
           Dein{\\[\baselineskip]}\spacefill\mbox{Arthur}\pend
           \leftskip=0em{}\selectlanguage{ngerman}\endnumbering\briefempfaengerindex{Bahr, Hermann@\textsc{Bahr, Hermann}!zzzSchnitzler, Arthur@\emph{von Arthur Schnitzler}!1908-01-291@{24. 1. 1908}|)be}\mylabel{L01756h}  \normalsize

\doendnotes{C}
\bigskip
\vfill

\clearpage

\footnotesize

\lohead{\textsc{register}}

% Definiere theindex-Environment komplett neu ohne reledmac
\makeatletter
\renewenvironment{theindex}{%
  \section*{\indexname}%
  \setlength{\parindent}{0pt}%
  \setlength{\parskip}{0pt plus 0.3pt}%
  \let\item\@idxitem
}{%
  \clearpage
}
\makeatother

\IfFileExists{\jobname-pw.ind}{\input{\jobname-pw.ind}}{}

\end{document}

      