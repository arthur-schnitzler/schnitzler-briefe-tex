%% latex-leseansicht-vorspann.tex
%% Vorspann für die Leseansicht.
%% Lädt die gemeinsame Datei latex-vorspann.tex mit nicht gesetztem Schalter.

\newif\ifkorrekturansicht
\korrekturansichtfalse

\input{../tex-inputs/latex-vorspann}


\section[Hugo von Hofmannsthal und Richard Beer-Hofmann an Arthur Schnitzler, {[}15.? 2. 1903{]}]{L01269 Hugo von Hofmannsthal und Richard Beer-Hofmann an Arthur Schnitzler, {[}15.? 2. 1903{]}}
\nopagebreak\mylabel{L01269v}
\rehead{ }\normalsize\beginnumbering\briefempfaengerindex{Schnitzler, Arthur@\textsc{Schnitzler, Arthur}!zzzBeer-Hofmann, Richard@\emph{von Richard Beer-Hofmann}!1903-02-152@{{[}15. 2. 1903{]}}|(be}\briefempfaengerindex{Schnitzler, Arthur@\textsc{Schnitzler, Arthur}!zzzHofmannsthal, Hugo von@\emph{von Hugo von Hofmannsthal}!1903-02-152@{{[}15. 2. 1903{]}}|(be}
\toendnotes[C]{\smallbreak\pagebreak[2]}
\correspDesc{Versand  durch Hugo von Hofmannsthal, Richard Beer-Hofmann am [15. 2. 1903] \textbf{Ort fehlend} 
\newline{}Erhalt  durch Arthur Schnitzler im Zeitraum [15. 2. 1903
                  – 19. 2. 1903?] in Wien}\toendnotes[C]{\smallbreak}
\Standort{CUL, Schnitzler, B 43.}
\physDesc{Brief, 1 Blatt, 4 Seiten, 1610 Zeichen
\newline{}Handschrift Richard Beer-Hofmann: schwarze Tinte, lateinische Kurrent
\newline{}Handschrift Hugo von Hofmannsthal: schwarze Tinte, deutsche Kurrent
\newline{}Ordnung: 1) mit Bleistift von unbekannter Hand datiert: »15/2 903.«  2) mit Bleistift von unbekannter Hand nummeriert: »\strikeout{213}« 3) mit Bleistift von unbekannter Hand nummeriert:
                                    »194«}
\buchAbdrucke{\weitereDrucke{1) Hugo von Hofmannsthal, Arthur Schnitzler: \emph{Briefwechsel}. Herausgegeben von Therese Nickl und Heinrich Schnitzler. Frankfurt am Main: \emph{S. Fischer} 1964, S. 167–168.} \weitereDrucke{2) Arthur Schnitzler, Richard Beer-Hofmann: \emph{Briefwechsel 1891–1931}. Herausgegeben von Konstanze Fliedl. Wien, Zürich: \emph{Europaverlag} 1992, S. 160–161.} }\toendnotes[C]{\smallbreak}
\pstart{}{\pb}lieber Pornograph\pend\vspace{0.5em}
\pstart
           wir denken es käme darauf an was für ein Verlag Ihr Schmutzwerk\pwindex{Schnitzler, Arthur 15.\,5.\,1862 Wien – 21.\,10.\,1931 ebd.@\textsc{Schnitzler, Arthur} (15.\,5.\,1862 Wien – 21.\,10.\,1931 ebd.), \emph{Schriftsteller, Mediziner}!Reigen. Zehn Dialoge@\strich\emph{Reigen. Zehn Dialoge}|pwv} herausgibt. Iſt es etwa \textsc{Grimm}\orgindex{Gustav Grimm Verlag@Gustav Grimm Verlag|pw} in \textsc{Búda-Pest}\oindex{Budapest@\textbf{Budapest}, \emph{Hauptstadt}|pw}? Dazu würden wir nicht rathen. Iſt es aber ein ernſter Verlag, die Ausſtattung{ }ſehr ernſthaft und anſtändig (Illuſtrationen \textsc{à la}{ }\textsc{Coschelle}\pwindex{Coschell, Moritz 18.\,9.\,1872 Wien – 11.\,7.\,1943 ebd.@\textsc{Coschell, Moritz} (18.\,9.\,1872 Wien – 11.\,7.\,1943 ebd.), \emph{Maler}|pw} würden dieſe \label{K_L01269-1v}\edtext{\textsc{Cochonnerie}}{\lemma{\textnormal{\emph{Cochonnerie}}}\Cendnote{\textnormal{französisch: Ferkelei}}}\label{K_L01269-1} zum
               Gelächter Europas\oindex{Europa@\textbf{Europa}|pw} machen) dann geht es immerhin.
               Denn{ }ſchließlich {\pb}iſt es ja Ihr
               beſtes Buch\pwindex{Schnitzler, Arthur 15.\,5.\,1862 Wien – 21.\,10.\,1931 ebd.@\textsc{Schnitzler, Arthur} (15.\,5.\,1862 Wien – 21.\,10.\,1931 ebd.), \emph{Schriftsteller, Mediziner}!Reigen. Zehn Dialoge@\strich\emph{Reigen. Zehn Dialoge}|pwv}, Sie Schmutzfink.
               Weder iſt es{ }ſo confus wie das Vermächtnis\pwindex{Schnitzler, Arthur 15.\,5.\,1862 Wien – 21.\,10.\,1931 ebd.@\textsc{Schnitzler, Arthur} (15.\,5.\,1862 Wien – 21.\,10.\,1931 ebd.), \emph{Schriftsteller, Mediziner}!Vermächtnis. Schauspiel in drei Akten@\strich\emph{Das Vermächtnis. Schauspiel in drei Akten}|pw}, noch{ }ſo glatt wie die Liebelei\pwindex{Schnitzler, Arthur 15.\,5.\,1862 Wien – 21.\,10.\,1931 ebd.@\textsc{Schnitzler, Arthur} (15.\,5.\,1862 Wien – 21.\,10.\,1931 ebd.), \emph{Schriftsteller, Mediziner}!Liebelei. Schauspiel in drei Akten@\strich\emph{Liebelei. Schauspiel in drei Akten}|pw}, noch{ }ſo \textsc{snobish} wie die \textsc{Beatrice}\pwindex{Schnitzler, Arthur 15.\,5.\,1862 Wien – 21.\,10.\,1931 ebd.@\textsc{Schnitzler, Arthur} (15.\,5.\,1862 Wien – 21.\,10.\,1931 ebd.), \emph{Schriftsteller, Mediziner}!Schleier der Beatrice. Schauspiel in fünf Akten@\strich\emph{Der Schleier der Beatrice. Schauspiel in fünf Akten}|pw}, noch{ }ſo unsäglich langweilig wie Ihre läppiſchen Novellen, kurz, natürlich{ }ſollen Sie es herausgeben, unter dem \textsc{Pseudonym}{ }\textsc{Ludassy}\pwindex{Gans-Ludassy, Julius von 13.\,4.\,1858 Wien – 30.\,9.\,1922 ebd.@\textsc{Gans-Ludassy, Julius von} (13.\,4.\,1858 Wien – 30.\,9.\,1922 ebd.), \emph{Schriftsteller, Journalist, Herausgeber}|pw} oder auch unter Ihrem eigenen Namen. Aber in einer {\pb}anständigen Form. Das iſt unſere
               Anſicht.\pend
           
\pstart
           {[}hs. Beer-Hofmann:{]} Sie müssen soviel Geld dafür beko{\geminationm}en (im \uuline{Vorhinein}, de{\geminationn} im Nachhinein wird es confiscirt) daß Sie Sich
               jedenfalls darüber mehr freuen, als Sie Sich später über das Schwätzen der Leute
               ärgern. Viele Leute werden es als Ihr erectiefstes Werk bezeichnen. Ob \uline{ich} es an Ihrer Stelle herausgeben würde weiß ich nicht;
               jedenfalls würde ich \uline{Sie} um Rath gefragt haben; geben
               Sie ihn mir also!\pend
           
\pstart
           {[}hs. Hofmannsthal:{]} Ob ich es an Ihrer Stelle herausgegeben hätte?
               Unbedingt, gegen einen beträchtlichen Vorſchuſs und unter Ihrem Namen. (Der Vorſchuſs
               natürlich unter meinem Namen zahlbar.)\pend
           
\pstart
           Verſtehen Sie alſo, was wir Ihnen gerathen haben?\pend
           
\pstart
           {[}hs. Beer-Hofmann:{]} Ernstlich:\pend
           \settowidth{\longeste}{3) Ausstattung}\settowidth{\longestz}{entscheiden}\settowidth{\longestd}{}\settowidth{\longestv}{}\settowidth{\longestf}{}\addtolength\longeste{1em}
        \addtolength\longestz{1em}
      \pstart\noindent\makebox[\the\longeste][l]{1) Summe}\makebox[\the\longestz][l]{}
                  \pend\pstart\noindent\makebox[\the\longeste][l]{2.) Verlag}\makebox[\the\longestz][l]{entscheiden}
                  \pend\pstart\noindent\makebox[\the\longeste][l]{3) Ausstattung}\makebox[\the\longestz][l]{}
                  \pend
\pstart
           1.) Sehr groß, 2.) Sehr ernst (die war’s nicht, der’s geschah) 3.) Würdig, d. h.
               Papier stark – wie Ihr Talent Format einfach, und eher groß, ja nicht Taschenformat
               oder zierlich.\pend
           
\pstart
           {[}hs. Hofmannsthal:{]} Genug. \spacefill\mbox{Hugo}\pend
           
\pstart
           {[}hs. Beer-Hofmann:{]} Ja! \spacefill\mbox{Richard}\pend
           
\pstart
           \noindent{}\label{T_L01269-1v}\edtext{Dieser Brief kann als Vorrede
                  abgedruckt werden!}{\lemma{\textnormal{\emph{Dieser … werden!}}}\Cendnote{\textnormal{quer am linken Rand
                     der letzten Seite}}}\label{T_L01269-1}\pend
           \selectlanguage{ngerman}\endnumbering\briefempfaengerindex{Schnitzler, Arthur@\textsc{Schnitzler, Arthur}!zzzBeer-Hofmann, Richard@\emph{von Richard Beer-Hofmann}!1903-02-152@{{[}15. 2. 1903{]}}|)be}\briefempfaengerindex{Schnitzler, Arthur@\textsc{Schnitzler, Arthur}!zzzHofmannsthal, Hugo von@\emph{von Hugo von Hofmannsthal}!1903-02-152@{{[}15. 2. 1903{]}}|)be}\mylabel{L01269h}  \newcommand{\dateiname}{L01269}\newcommand{\titel}{Hugo von Hofmannsthal und Richard Beer-Hofmann an Arthur Schnitzler, [15.? 2. 1903]}\newcommand{\editorInnen}{Martin Anton Müller und Gerd-Hermann Susen}%% latex-leseansicht-abspann.tex
%% Abspann für die Leseansicht.
%% Der Schalter \ifkorrekturansicht ist bereits durch den Vorspann gesetzt.

%% latex-abspann.tex
%% Gemeinsamer Abspann für Korrekturansicht und Leseansicht.
%% Setzt den Schalter \ifkorrekturansicht voraus (gesetzt in den
%% einbindenden Dateien latex-korrekturansicht-abspann.tex bzw.
%% latex-leseansicht-abspann.tex).
%% ---------------------------------------------------------------

\normalsize

% Das esempio-Environment wird nur in der Leseansicht benötigt
\ifkorrekturansicht\else
\newenvironment{esempio}[3]%
{
    \vspace{1.5ex}
    \rlap{\underline{#1}}
    \par
    \setlength{\parindent}{0cm}
    \nopagebreak
    \leftskip=#2cm
    \rightskip=#3cm
}
{
    \par
}
\fi

\doendnotes{C}
\bigskip
\vfill

\clearpage

\footnotesize

\ifkorrekturansicht
  \lohead{\textsc{register}}
\fi

% theindex-Environment neu definieren ohne reledmac
\makeatletter
\renewenvironment{theindex}{%
  \ifkorrekturansicht
    \section*{\indexname}%
  \else
    \subsubsection*{Index der erwähnten Entitäten}%
  \fi
  \setlength{\parindent}{0pt}%
  \setlength{\parskip}{0pt plus 0.3pt}%
  \let\item\@idxitem
}{%
  \ifkorrekturansicht\clearpage\fi
}
\makeatother

\IfFileExists{\jobname-pw.ind}{\input{\jobname-pw.ind}}{}

% Quellenangabe nur in der Leseansicht
\ifkorrekturansicht\else
% Fallback-Definitionen, falls die .tex-Datei \titel etc. nicht gesetzt hat
\providecommand{\titel}{}
\providecommand{\editorInnen}{}
\providecommand{\dateiname}{\jobname}

\vspace{3cm}

\vfill

\footnotesize
\textsc{Quelle}: \titel. Herausgegeben von {\editorInnen}. In: \emph{Arthur Schnitzler: Briefwechsel mit Autorinnen und Autoren}.
 Digitale Edition, https://schnitzler-briefe.acdh.oeaw.ac.at/{\dateiname}.html (Stand \today)
\fi

\end{document}


