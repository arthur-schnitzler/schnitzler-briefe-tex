%% latex-leseansicht-vorspann.tex
%% Vorspann für die Leseansicht.
%% Lädt die gemeinsame Datei latex-vorspann.tex mit nicht gesetztem Schalter.

\newif\ifkorrekturansicht
\korrekturansichtfalse

\input{../tex-inputs/latex-vorspann}


         
         \newcommand{\erwaehntePersonen}{Personen: }
         \newcommand{\erwaehnteInstitutionen}{}
         \newcommand{\erwaehnteOrte}{}
         \newcommand{\erwaehnteWerke}{
               \section[Arthur Schnitzler an Hugo von Hofmannsthal, 5. 7. 1898]{ Arthur Schnitzler an Hugo von Hofmannsthal, 5. 7. 1898}\nopagebreak\mylabel{v}\rehead{ }\begin{ledgroupsized}[t]{13cm}\normalsize\beginnumbering \toendnotes[C]{\smallbreak\pagebreak[2]} \Standort{FDH, Hs-30885,68.}
\physDesc{Brief, 1 Blatt, 3 Seiten
\newline{}Handschrift: schwarze Tinte, deutsche Kurrent}\buchAbdrucke{\weitereDrucke{Hugo von Hofmannsthal, Arthur Schnitzler: \emph{Briefwechsel}. Hg. Therese Nickl und Heinrich Schnitzler. Frankfurt am Main: \emph{S. Fischer} 1964, S. 104.} }\toendnotes[C]{\smallbreak}\pstart
           \raggedleft{}{\pb}Wien\oindex{XXXX Ortsangabe fehlt|pw}, 5. Juli 98.\pend
           \pstart
           mein lieber Hugo, das ka{\geminationn} ich ganz
                    gut ſo einrichten, daſs wir uns etwa am 9. Auguſt treffen – ob Innsbruck\oindex{XXXX Ortsangabe fehlt|pw} oder vielleicht München\oindex{XXXX Ortsangabe fehlt|pw}, das wollen wir noch ſehn; ich dürfte ja vom
                        1. bis 9. Auguſt unter ſolchen Umſtänden (we{\geminationn} nicht meine Mama\pwindex{\textcolor{red}{\textsuperscript{XXXX1 indx}}|pwv} doch noch auf mich Anſprüche macht) in Tegernſee\oindex{XXXX Ortsangabe fehlt|pw}{ }ſein. Hoffentlich wird Ihre Sti{\geminationm}ung {\pb}noch in Galizien\oindex{XXXX Ortsangabe fehlt|pw} beſſer. Haben Sie viel zu thun?\pend
           \pstart
           Ich werde wahrſcheinlich Montag abreiſen; eine Reihe von Tagen in
                        Graz\oindex{XXXX Ortsangabe fehlt|pw} bleiben. Sie werden i{\geminationm}er wiſſen, wo ich bin. Wie wird das nur mit Richard\pwindex{\textcolor{red}{\textsuperscript{XXXX1 indx}}|pw}{ }ſein, we{\geminationn} unſer Rendezvous ſo weit hinaus
                    geſchoben iſt? Ich erwarte heute einen Brief von ihm, der telegrafiſch aviſirt
                    iſt.\pend
           \pstart
           Ich ſchreibe an dem Stück, das vorläufig »\textsc{Shawl}\textcolor{red}{\textsuperscript{XXXX indx}}« heißen ſoll; bin im 2. Akt, {\pb}der mir
                    aber bisher im Ton durchaus nicht gelingen will.\pend
           \pstart
           Im übrigen bin ich recht gequält. –\pend
           \pstart
           Schauen wir nur, daſs dieſes Zuſa{\geminationm}enſein im
                        Auguſt zuſtande kommt.\pend
           \pstart Von Herzen Ihr \spacefill\mbox{Arthur.}\pend{}
         
         \endnumbering\mylabel{h}\end{ledgroupsized}  \newcommand{\dateiname}{L00813}\newcommand{\titel}{Arthur Schnitzler an Hugo von Hofmannsthal, 5. 7. 1898}\newcommand{\editorInnen}{Martin Anton Müller und Gerd-Hermann Susen}%% latex-leseansicht-abspann.tex
%% Abspann für die Leseansicht.
%% Der Schalter \ifkorrekturansicht ist bereits durch den Vorspann gesetzt.

%% latex-abspann.tex
%% Gemeinsamer Abspann für Korrekturansicht und Leseansicht.
%% Setzt den Schalter \ifkorrekturansicht voraus (gesetzt in den
%% einbindenden Dateien latex-korrekturansicht-abspann.tex bzw.
%% latex-leseansicht-abspann.tex).
%% ---------------------------------------------------------------

\normalsize

% Das esempio-Environment wird nur in der Leseansicht benötigt
\ifkorrekturansicht\else
\newenvironment{esempio}[3]%
{
    \vspace{1.5ex}
    \rlap{\underline{#1}}
    \par
    \setlength{\parindent}{0cm}
    \nopagebreak
    \leftskip=#2cm
    \rightskip=#3cm
}
{
    \par
}
\fi

\doendnotes{C}
\bigskip
\vfill

\clearpage

\footnotesize

\ifkorrekturansicht
  \lohead{\textsc{register}}
\fi

% theindex-Environment neu definieren ohne reledmac
\makeatletter
\renewenvironment{theindex}{%
  \ifkorrekturansicht
    \section*{\indexname}%
  \else
    \subsubsection*{Index der erwähnten Entitäten}%
  \fi
  \setlength{\parindent}{0pt}%
  \setlength{\parskip}{0pt plus 0.3pt}%
  \let\item\@idxitem
}{%
  \ifkorrekturansicht\clearpage\fi
}
\makeatother

\IfFileExists{\jobname-pw.ind}{\input{\jobname-pw.ind}}{}

% Quellenangabe nur in der Leseansicht
\ifkorrekturansicht\else
% Fallback-Definitionen, falls die .tex-Datei \titel etc. nicht gesetzt hat
\providecommand{\titel}{}
\providecommand{\editorInnen}{}
\providecommand{\dateiname}{\jobname}

\vspace{3cm}

\vfill

\footnotesize
\textsc{Quelle}: \titel. Herausgegeben von {\editorInnen}. In: \emph{Arthur Schnitzler: Briefwechsel mit Autorinnen und Autoren}.
 Digitale Edition, https://schnitzler-briefe.acdh.oeaw.ac.at/{\dateiname}.html (Stand \today)
\fi

\end{document}


      