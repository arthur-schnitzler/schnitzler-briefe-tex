%% latex-leseansicht-vorspann.tex
%% Vorspann für die Leseansicht.
%% Lädt die gemeinsame Datei latex-vorspann.tex mit nicht gesetztem Schalter.

\newif\ifkorrekturansicht
\korrekturansichtfalse

\input{../tex-inputs/latex-vorspann}


\section[Arthur Schnitzler an Hugo von Hofmannsthal, 5. 7. 1898]{L00813 Arthur Schnitzler an Hugo von Hofmannsthal, 5. 7. 1898}
\nopagebreak\mylabel{L00813v}
\rehead{ }\normalsize\beginnumbering\briefempfaengerindex{Hofmannsthal, Hugo von@\textsc{Hofmannsthal, Hugo von}!zzzSchnitzler, Arthur@\emph{von Arthur Schnitzler}!1898-07-051@{5. 7. 1898}|(be}
\toendnotes[C]{\smallbreak\pagebreak[2]}
\correspDesc{Versand  durch Arthur Schnitzler am 5. 7. 1898 in Wien
\newline{}Erhalt  durch Hugo von Hofmannsthal im Zeitraum [6. 7. 1898
                  – 10. 7. 1898?] in Tschortkiw}\toendnotes[C]{\smallbreak}
\Standort{FDH, Hs-30885,68.}
\physDesc{Brief, 1 Blatt, 3 Seiten, 887 Zeichen
\newline{}Handschrift: schwarze Tinte, deutsche Kurrent}
\buchAbdrucke{\weitereDrucke{Hugo von Hofmannsthal, Arthur Schnitzler: \emph{Briefwechsel}. Herausgegeben von Therese Nickl und Heinrich Schnitzler. Frankfurt am Main: \emph{S. Fischer} 1964, S. 104.} }\toendnotes[C]{\smallbreak}
\pstart
           \raggedleft{}{\pb}Wien\oindex{Wien@\textbf{Wien}, \emph{Verwaltungsgebiet}|pw}, 5. Juli 98.\pend
           \vspace{0.5em}
\pstart
           mein lieber Hugo, das ka{\geminationn} ich ganz gut{ }ſo einrichten, daſs wir uns etwa am 9. Auguſt treffen – ob Innsbruck\oindex{Innsbruck@\textbf{Innsbruck}, \emph{Verwaltungsgebiet}|pw} oder vielleicht München\oindex{München@\textbf{München}|pw}, das wollen wir noch{ }ſehn; ich dürfte ja vom
                  1. bis 9. Auguſt unter{ }ſolchen Umſtänden (we{\geminationn} nicht meine Mama\pwindex{Schnitzler, Louise 8.\,7.\,1840 Kőszeg – 9.\,9.\,1911 Wien@\textsc{Schnitzler, Louise} (8.\,7.\,1840 Kőszeg – 9.\,9.\,1911 Wien)|pwv} doch noch auf mich Anſprüche macht) in Tegernſee\oindex{Tegernsee@\textbf{Tegernsee}|pw}{ }ſein. Hoffentlich wird Ihre Sti{\geminationm}ung {\pb}noch in Galizien\oindex{Galizien@\textbf{Galizien}|pw} beſſer. Haben Sie viel zu thun?\pend
           
\pstart
           Ich werde wahrſcheinlich Montag abreiſen; eine Reihe von Tagen in Graz\oindex{Graz@\textbf{Graz}, \emph{Verwaltungsgebiet}|pw} bleiben. Sie werden i{\geminationm}er wiſſen, wo ich bin. Wie wird das nur mit Richard\pwindex{Beer-Hofmann, Richard 11.\,7.\,1866 Wien – 26.\,9.\,1945 New York City@\textsc{Beer-Hofmann, Richard} (11.\,7.\,1866 Wien – 26.\,9.\,1945 New York City), \emph{Schriftsteller}|pw}{ }ſein, we{\geminationn} unſer
               Rendezvous{ }ſo weit hinaus geſchoben iſt? Ich erwarte heute einen Brief von ihm, der
               telegrafiſch aviſirt iſt.\pend
           
\pstart
           Ich{ }ſchreibe an dem Stück, das vorläufig »\textsc{Shawl}\pwindex{Schnitzler, Arthur 15.\,5.\,1862 Wien – 21.\,10.\,1931 ebd.@\textsc{Schnitzler, Arthur} (15.\,5.\,1862 Wien – 21.\,10.\,1931 ebd.), \emph{Schriftsteller, Mediziner}!Schleier der Beatrice. Schauspiel in fünf Akten@\strich\emph{Der Schleier der Beatrice. Schauspiel in fünf Akten}|pw}« heißen{ }ſoll; bin im 2. Akt, {\pb}der mir aber bisher
               im Ton durchaus nicht gelingen will.\pend
           
\pstart
           Im übrigen bin ich recht gequält. –\pend
           
\pstart
           Schauen wir nur, daſs dieſes Zuſa{\geminationm}enſein im
                  Auguſt zuſtande kommt.\pend
           \pstart Von Herzen Ihr \spacefill\mbox{Arthur.}\pend{}\selectlanguage{ngerman}\endnumbering\briefempfaengerindex{Hofmannsthal, Hugo von@\textsc{Hofmannsthal, Hugo von}!zzzSchnitzler, Arthur@\emph{von Arthur Schnitzler}!1898-07-051@{5. 7. 1898}|)be}\mylabel{L00813h}  \newcommand{\dateiname}{L00813}\newcommand{\titel}{Arthur Schnitzler an Hugo von Hofmannsthal, 5. 7. 1898}\newcommand{\editorInnen}{Martin Anton Müller und Gerd-Hermann Susen}%% latex-leseansicht-abspann.tex
%% Abspann für die Leseansicht.
%% Der Schalter \ifkorrekturansicht ist bereits durch den Vorspann gesetzt.

%% latex-abspann.tex
%% Gemeinsamer Abspann für Korrekturansicht und Leseansicht.
%% Setzt den Schalter \ifkorrekturansicht voraus (gesetzt in den
%% einbindenden Dateien latex-korrekturansicht-abspann.tex bzw.
%% latex-leseansicht-abspann.tex).
%% ---------------------------------------------------------------

\normalsize

% Das esempio-Environment wird nur in der Leseansicht benötigt
\ifkorrekturansicht\else
\newenvironment{esempio}[3]%
{
    \vspace{1.5ex}
    \rlap{\underline{#1}}
    \par
    \setlength{\parindent}{0cm}
    \nopagebreak
    \leftskip=#2cm
    \rightskip=#3cm
}
{
    \par
}
\fi

\doendnotes{C}
\bigskip
\vfill

\clearpage

\footnotesize

\ifkorrekturansicht
  \lohead{\textsc{register}}
\fi

% theindex-Environment neu definieren ohne reledmac
\makeatletter
\renewenvironment{theindex}{%
  \ifkorrekturansicht
    \section*{\indexname}%
  \else
    \subsubsection*{Index der erwähnten Entitäten}%
  \fi
  \setlength{\parindent}{0pt}%
  \setlength{\parskip}{0pt plus 0.3pt}%
  \let\item\@idxitem
}{%
  \ifkorrekturansicht\clearpage\fi
}
\makeatother

\IfFileExists{\jobname-pw.ind}{\input{\jobname-pw.ind}}{}

% Quellenangabe nur in der Leseansicht
\ifkorrekturansicht\else
% Fallback-Definitionen, falls die .tex-Datei \titel etc. nicht gesetzt hat
\providecommand{\titel}{}
\providecommand{\editorInnen}{}
\providecommand{\dateiname}{\jobname}

\vspace{3cm}

\vfill

\footnotesize
\textsc{Quelle}: \titel. Herausgegeben von {\editorInnen}. In: \emph{Arthur Schnitzler: Briefwechsel mit Autorinnen und Autoren}.
 Digitale Edition, https://schnitzler-briefe.acdh.oeaw.ac.at/{\dateiname}.html (Stand \today)
\fi

\end{document}


