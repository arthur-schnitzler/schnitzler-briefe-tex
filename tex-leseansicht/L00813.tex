%% latex-korrekturansicht-vorspann.tex
%% Vorspann für die Korrekturansicht.
%% Lädt die gemeinsame Datei latex-vorspann.tex mit gesetztem Schalter.

\newif\ifkorrekturansicht
\korrekturansichttrue

\input{../tex-inputs/latex-vorspann}


\section[Arthur Schnitzler an Hugo von Hofmannsthal, 5. 7. 1898]{L00813 Arthur Schnitzler an Hugo von Hofmannsthal, 5. 7. 1898}
\nopagebreak\mylabel{L00813v}
\rehead{ }\normalsize\beginnumbering\briefempfaengerindex{Hofmannsthal, Hugo von@\textsc{Hofmannsthal, Hugo von}!zzzSchnitzler, Arthur@\emph{von Arthur Schnitzler}!1898-07-051@{5. 7. 1898}|(be}
\toendnotes[C]{\smallbreak\pagebreak[2]}\Standort{FDH, Hs-30885,68.}
\physDesc{Brief, 1 Blatt, 3 Seiten, 887 Zeichen
\newline{}Handschrift: schwarze Tinte, deutsche Kurrent}
\buchAbdrucke{\weitereDrucke{Hugo von Hofmannsthal, Arthur Schnitzler: \emph{Briefwechsel}. Frankfurt am Main: \emph{S. Fischer} 1964, S. 104.} }\toendnotes[C]{\smallbreak}
\pstart
           \raggedleft{}{\pb}Wien\oindex{Wien@\textbf{Wien}, \emph{A.ADM2}|pw}, 5. Juli 98.\pend
           \vspace{0.5em}
\pstart
           mein lieber Hugo, das ka{\geminationn} ich ganz gut
               ſo einrichten, daſs wir uns etwa am 9. Auguſt treffen – ob Innsbruck\oindex{Innsbruck@\textbf{Innsbruck}, \emph{A.ADM2}|pw} oder vielleicht München\oindex{Muenchen@\textbf{München}, \emph{P.PPLA}|pw}, das wollen wir noch ſehn; ich dürfte ja vom
                  1. bis 9. Auguſt unter ſolchen Umſtänden (we{\geminationn} nicht meine Mama\pwindex{Schnitzler, Louise 1840-07-08 – 1911-09-09@\textsc{Schnitzler, Louise} (1840-07-08 – 1911-09-09)|pwv} doch noch auf mich Anſprüche macht) in Tegernſee\oindex{Tegernsee@\textbf{Tegernsee}, \emph{P.PPL}|pw}{ }ſein. Hoffentlich wird Ihre Sti{\geminationm}ung {\pb}noch in Galizien\oindex{Galizien@\textbf{Galizien}, \emph{Region}|pw} beſſer. Haben Sie viel zu thun?\pend
           
\pstart
           Ich werde wahrſcheinlich Montag abreiſen; eine Reihe von Tagen in Graz\oindex{Graz@\textbf{Graz}, \emph{A.ADM2}|pw} bleiben. Sie werden i{\geminationm}er wiſſen, wo ich bin. Wie wird das nur mit Richard\pwindex{Beer-Hofmann, Richard 1866-07-11 – 1945-09-26@\textsc{Beer-Hofmann, Richard} (1866-07-11 – 1945-09-26), \emph{Schriftsteller/Schriftstellerin}|pw}{ }ſein, we{\geminationn} unſer
               Rendezvous ſo weit hinaus geſchoben iſt? Ich erwarte heute einen Brief von ihm, der
               telegrafiſch aviſirt iſt.\pend
           
\pstart
           Ich ſchreibe an dem Stück, das vorläufig »\textsc{Shawl}\pwindex{Schleier der Beatrice. Schauspiel in fuenf Akten@\emph{Der Schleier der Beatrice. Schauspiel in fünf Akten}|pw}« heißen ſoll; bin im 2. Akt, {\pb}der mir aber bisher
               im Ton durchaus nicht gelingen will.\pend
           
\pstart
           Im übrigen bin ich recht gequält. –\pend
           
\pstart
           Schauen wir nur, daſs dieſes Zuſa{\geminationm}enſein im
                  Auguſt zuſtande kommt.\pend
           \pstart Von Herzen Ihr \spacefill\mbox{Arthur.}\pend{}\selectlanguage{ngerman}\endnumbering\briefempfaengerindex{Hofmannsthal, Hugo von@\textsc{Hofmannsthal, Hugo von}!zzzSchnitzler, Arthur@\emph{von Arthur Schnitzler}!1898-07-051@{5. 7. 1898}|)be}\mylabel{L00813h}  \normalsize

\doendnotes{C}
\bigskip
\vfill

\clearpage

\footnotesize

\lohead{\textsc{register}}

% Definiere theindex-Environment komplett neu ohne reledmac
\makeatletter
\renewenvironment{theindex}{%
  \section*{\indexname}%
  \setlength{\parindent}{0pt}%
  \setlength{\parskip}{0pt plus 0.3pt}%
  \let\item\@idxitem
}{%
  \clearpage
}
\makeatother

\IfFileExists{\jobname-pw.ind}{\input{\jobname-pw.ind}}{}

\end{document}

      