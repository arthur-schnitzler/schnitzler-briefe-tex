%% latex-leseansicht-vorspann.tex
%% Vorspann für die Leseansicht.
%% Lädt die gemeinsame Datei latex-vorspann.tex mit nicht gesetztem Schalter.

\newif\ifkorrekturansicht
\korrekturansichtfalse

\input{../tex-inputs/latex-vorspann}


               \section[Arthur Schnitzler an Georg Brandes, 2. 8. 1918]{ Arthur Schnitzler an Georg Brandes, 2. 8. 1918}\nopagebreak\mylabel{v}\rehead{ }\begin{ledgroupsized}[t]{13cm}\normalsize\beginnumbering\briefempfaengerindex{Brandes, Georg@\textsc{Brandes, Georg}!zzzSchnitzler, Arthur@\emph{von Arthur Schnitzler}!1918-08-021@{2. 8. 1918}|(be} \toendnotes[C]{\smallbreak\pagebreak[2]} \Standort{Kopenhagen, Det Kongelige Bibliotek, Georg Brandes Arkiv, box 125.}
\physDesc{Brief, 2 Blätter, 4 Seiten
\newline{}Handschrift: schwarze Tinte, lateinische Kurrent\newline{}Ordnung: mit Bleistift von unbekannter Hand nummeriert: »40.«
                                    und mit »Schnitzler« beschriftet }\buchAbdrucke{\weitereDrucke{1) Georg Brandes, Arthur Schnitzler: \emph{Ein Briefwechsel}. Hg. Kurt Bergel. Bern: \emph{Francke} 1956, S. 122–123.} \weitereDrucke{2) Arthur Schnitzler: \emph{Briefe 1913–1931}. Hg. Peter Michael Braunwarth, Richard Miklin, Susanne Pertlik und Heinrich Schnitzler. Frankfurt am Main: \emph{S. Fischer} 1984, S. 165–166.} }\toendnotes[C]{\smallbreak}\pstart
           \raggedleft{}{\pb}2. 8 1918{\\}Wien XVII. Sternwartestr. 71\oindex{Sternwartestrasse@\textbf{Sternwartestraße}|pw}\pend
           \pstart{}mein lieber und verehrter Herr Brandes,\pend\pstart
           ich lese vom Tode Peter Nansen\pwindex{Nansen, Peter 20.01.1861 – 31.07.1918@\textsc{Nansen, Peter} (20.01.1861 – 31.07.1918), \emph{Schriftsteller, Journalist, Verleger}|pw}s, und habe das
                    Bedürfnis irgend jemandem zu sagen, wie tief mich das Hinscheiden dieses
                    liebenswerthen Menschen bewegt, den ich zuletzt kurz vor Ausbruch des Kriegs bei
                    mir in Wien\oindex{Wien@\textbf{Wien}|pw} gesehen habe – schon recht verändert, ja irgendwie gezeichnet – aber
                    doch noch von dem ganzen Zauber seines Wesens umwittert, den ich, fast mehr als
                    aus seinen reizvollen Büchern, aus seinem Gehaben, seiner Art zu sprechen,
                    seinem Schweigen, seinen Blicken zu spüren vermeinte. Nun fügt es der Zufall,
                    daß ich mir gerade in der letzten Zeit Ihre Briefe, lieber und verehrter Freund
                    abschreiben ließ – einige, mit Bleistift geschrieben, waren fast unlesbar
                    geworden, – und nun, da ich sie, \uline{vom ersten} bis
                    zum letzten, \strikeout{alle} – mit welchem Vergnügen! –
                    wieder durchnahm, fand ich öfters Peter
                        Nansen\pwindex{Nansen, Peter 20.01.1861 – 31.07.1918@\textsc{Nansen, Peter} (20.01.1861 – 31.07.1918), \emph{Schriftsteller, Journalist, Verleger}|pw}s Namen wiederkehren; auch von seinem Kranksein ist die Rede
                    darin, und da liegt es nahe mich mit meinem Beileid, – meinem Leid an Sie zu
                    wenden, der Nansen\pwindex{Nansen, Peter 20.01.1861 – 31.07.1918@\textsc{Nansen, Peter} (20.01.1861 – 31.07.1918), \emph{Schriftsteller, Journalist, Verleger}|pw}s Freund war und für mich
                    zugleich, und für die meisten Mitlebenden, {\pb}der repraesentative Mann Daenemarks\oindex{Daenemark@\textbf{Dänemark}|pw} ist. Und
                    ich benutze die Gelegenheit Ihnen wieder einmal, über diese zerrissene und
                    stöhnende Welt hin\substVorne{}\textsuperscript{über}\substDazwischen{}weg\substHinten{}, die Hand zu drücken um Ihnen zu sagen, mit welcher Sympathie, ja darf
                    ich es etwas sentimental ausdrücken –: mit welcher Sehnsucht ich Ihrer gedenke!
                    Von Ihren letzten Büchern haben Sie mir geschrieben;– vom Goethe\pwindex{Goethe, Johann Wolfgang von 28.08.1749 – 22.03.1832@\textsc{Goethe, Johann Wolfgang von} (28.08.1749 – 22.03.1832), \emph{Schriftsteller}|pw}\pwindex{Brandes, Georg 04.02.1842 – 19.02.1927@\textsc{Brandes, Georg} (04.02.1842 – 19.02.1927)!Wolfgang Goethe1915@\strich\emph{Wolfgang Goethe} {[}1915{]}|pwv} und Voltaire\pwindex{Voltaire 21.11.1694 – 30.05.1778@\textsc{Voltaire} (21.11.1694 – 30.05.1778), \emph{Schriftsteller, Philosoph}|pw}\pwindex{Brandes, Georg 04.02.1842 – 19.02.1927@\textsc{Brandes, Georg} (04.02.1842 – 19.02.1927)!Voltaire und sein Jahrhundert1916 – 1917@\strich\emph{Voltaire und sein Jahrhundert} {[}1916 – 1917{]}|pwv};– sie existiren noch nicht in deutscher Sprache, – und nun werden Sie wohl
                    auch Ihren Julius Caesar\pwindex{Caesar, Gaius Iulius 13.7.100? v. Chr. – 15.3.44 v. Chr.@\textsc{Caesar, Gaius Iulius} (13.7.100? v. Chr. – 15.3.44 v. Chr.), \emph{Politiker, Kaiser, Heerführer}|pw}\pwindex{Brandes, Georg 04.02.1842 – 19.02.1927@\textsc{Brandes, Georg} (04.02.1842 – 19.02.1927)!Gaius Julius Cæsar1918@\strich\emph{Gaius Julius Cæsar} {[}1918{]}|pwv} bald abschliessen. Aber wa{\geminationn} werd ich
                    Ignorant, der nicht daenisch versteht, sie endlich lesen dürfen? – Auch ich hab
                    allerlei gemacht – nicht so bedeutungsvolles! – und nach meiner alten
                    zudringlichen Gewohnheit werd ich Ihnen ein Stück\pwindex{Schnitzler, Arthur 15.05.1862 – 21.10.1931@\textsc{Schnitzler, Arthur} (15.05.1862 – 21.10.1931), \emph{Schriftsteller, Mediziner}!Schwestern oder Casanova in Spa. Lustspiel in Versen01. 10. 1919@\strich\emph{Die Schwestern oder Casanova in Spa. Lustspiel in Versen} {[}01. 10. 1919{]}|pwv} und eine Novelle\pwindex{Schnitzler, Arthur 15.05.1862 – 21.10.1931@\textsc{Schnitzler, Arthur} (15.05.1862 – 21.10.1931), \emph{Schriftsteller, Mediziner}!Casanovas Heimfahrt1.7.1918 – 1.9.1918@\strich\emph{Casanovas Heimfahrt} {[}1.7.1918 – 1.9.1918{]}|pwv} zusenden, sobald sie gedruckt sind. – Aber
                    wann werden wir einander wiedersehen? Lassen Sie mich doch bald wieder – und
                    wärs nur mit einem Wort, wissen, daß Sie sich wohl befinden und Ihre edle Stirn
                    über den Dunst und Dampf dieser Jammerwelt in {\pb}reinere Lüfte emporzurecken vermögen. Ihnen im neutralen Land\oindex{Daenemark@\textbf{Dänemark}|pwv} ist es doch immerhin leichter als
                    uns. In meiner Familie geht es ganz leidlich; mein Bub\pwindex{Schnitzler, Heinrich 09.08.1902 – 12.07.1982@\textsc{Schnitzler, Heinrich} (09.08.1902 – 12.07.1982), \emph{Regisseur, Schauspieler}|pwv} (wird 16) meine Tochter\pwindex{Schnitzler, Lili 13.09.1909 – 26.07.1928@\textsc{Schnitzler, Lili} (13.09.1909 – 26.07.1928)|pwv} (wird 9) entwickeln sich in jeder
                    Hinsicht gut; meine Frau\pwindex{Schnitzler, Olga 17.01.1882 – 13.01.1970@\textsc{Schnitzler, Olga} (17.01.1882 – 13.01.1970), \emph{Schauspielerin, Sängerin}|pwv}
                    hat wohl unter den häuslichen Kriegswirtschaftssorgen wie jede u jeder etwas
                    gelitten, trotzdem aber ihre Kunst nicht vernachlässigt, ihre Stimme entwickelt
                    sich aufs schönste. Nun ist sie bei ihrer Schwester\pwindex{Steinrueck, Elisabeth 19.11.1885 – 07.04.1920@\textsc{Steinrück, Elisabeth} (19.11.1885 – 07.04.1920)|pwv} in Bayern\oindex{Bayern@\textbf{Bayern}|pw}
                        (Partenkirchen\oindex{Garmisch-Partenkirchen@\textbf{Garmisch-Partenkirchen}|pw}) wohin ich Mitte dieses
                    Monats auch zu fahren gedenke. Über politisches ka{\geminationn}
                    ich mich in einem Brief nicht so ausführlich äußern als ich möchte – wie
                    complicirt gerade bei uns all diese Probleme sind, ersehen Sie aus jeder
                    Zeitung, selbst aus dem censurirtesten Wien\oindex{Wien@\textbf{Wien}|pw}er
                    Blatt. Und trotz aller Schwierigkeiten – Misslichkeiten – Unsicherheiten: wie
                    viel Auftrieb, Sti{\geminationm}ungskraft, Talent – welche
                    positive Möglichkeiten in diesem Land\oindex{Oesterreich@\textbf{Österreich}|pwv}, das vielleicht nicht {\pb}alle
                    seine Bewohner als »Vaterland« aber jeder als »Heimat« liebt. Ich muß hier
                    innehalten – trotzdem ich daran bin, viel freundlicheres über Oesterreich\oindex{Oesterreich@\textbf{Österreich}|pw} zu sagen, als es \introOben{}selbst\introOben{} unsere officiösen Zeitungen zu thun pflegen.\pend
           \pstart
           Bitte bestätigen Sie mir bald den Empfang dieses Briefes und erhalten Sie mir und
                    den Meinen Ihre Freundschaft.\pend
           \pstart
           Von Herzen{\\[\baselineskip]}Ihr{\\[\baselineskip]}\spacefill\mbox{Arthur Schnitzler}\pend
           \leftskip=0em{}\endnumbering\briefempfaengerindex{Brandes, Georg@\textsc{Brandes, Georg}!zzzSchnitzler, Arthur@\emph{von Arthur Schnitzler}!1918-08-021@{2. 8. 1918}|)be}\mylabel{h}\end{ledgroupsized}  \newcommand{\dateiname}{L02292}\newcommand{\titel}{Arthur Schnitzler an Georg Brandes, 2. 8. 1918}\newcommand{\editorInnen}{Martin Anton Müller und Gerd-Hermann Susen}%% latex-leseansicht-abspann.tex
%% Abspann für die Leseansicht.
%% Der Schalter \ifkorrekturansicht ist bereits durch den Vorspann gesetzt.

%% latex-abspann.tex
%% Gemeinsamer Abspann für Korrekturansicht und Leseansicht.
%% Setzt den Schalter \ifkorrekturansicht voraus (gesetzt in den
%% einbindenden Dateien latex-korrekturansicht-abspann.tex bzw.
%% latex-leseansicht-abspann.tex).
%% ---------------------------------------------------------------

\normalsize

% Das esempio-Environment wird nur in der Leseansicht benötigt
\ifkorrekturansicht\else
\newenvironment{esempio}[3]%
{
    \vspace{1.5ex}
    \rlap{\underline{#1}}
    \par
    \setlength{\parindent}{0cm}
    \nopagebreak
    \leftskip=#2cm
    \rightskip=#3cm
}
{
    \par
}
\fi

\doendnotes{C}
\bigskip
\vfill

\clearpage

\footnotesize

\ifkorrekturansicht
  \lohead{\textsc{register}}
\fi

% theindex-Environment neu definieren ohne reledmac
\makeatletter
\renewenvironment{theindex}{%
  \ifkorrekturansicht
    \section*{\indexname}%
  \else
    \subsubsection*{Index der erwähnten Entitäten}%
  \fi
  \setlength{\parindent}{0pt}%
  \setlength{\parskip}{0pt plus 0.3pt}%
  \let\item\@idxitem
}{%
  \ifkorrekturansicht\clearpage\fi
}
\makeatother

\IfFileExists{\jobname-pw.ind}{\input{\jobname-pw.ind}}{}

% Quellenangabe nur in der Leseansicht
\ifkorrekturansicht\else
% Fallback-Definitionen, falls die .tex-Datei \titel etc. nicht gesetzt hat
\providecommand{\titel}{}
\providecommand{\editorInnen}{}
\providecommand{\dateiname}{\jobname}

\vspace{3cm}

\vfill

\footnotesize
\textsc{Quelle}: \titel. Herausgegeben von {\editorInnen}. In: \emph{Arthur Schnitzler: Briefwechsel mit Autorinnen und Autoren}.
 Digitale Edition, https://schnitzler-briefe.acdh.oeaw.ac.at/{\dateiname}.html (Stand \today)
\fi

\end{document}


      