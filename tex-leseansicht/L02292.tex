%% latex-korrekturansicht-vorspann.tex
%% Vorspann für die Korrekturansicht.
%% Lädt die gemeinsame Datei latex-vorspann.tex mit gesetztem Schalter.

\newif\ifkorrekturansicht
\korrekturansichttrue

\input{../tex-inputs/latex-vorspann}


\section[Arthur Schnitzler an Georg Brandes, 2. 8. 1918]{L02292 Arthur Schnitzler an Georg Brandes, 2. 8. 1918}
\nopagebreak\mylabel{L02292v}
\rehead{ }\normalsize\beginnumbering\briefempfaengerindex{Brandes, Georg@\textsc{Brandes, Georg}!zzzSchnitzler, Arthur@\emph{von Arthur Schnitzler}!1918-08-021@{2. 8. 1918}|(be}
\toendnotes[C]{\smallbreak\pagebreak[2]}\Standort{Kopenhagen, Det Kongelige Bibliotek, Georg Brandes Arkiv, box 125.}
\physDesc{Brief, 2 Blätter, 4 Seiten, 3323 Zeichen
\newline{}Handschrift: schwarze Tinte, lateinische Kurrent
\newline{}Ordnung: mit Bleistift von unbekannter Hand
                                    nummeriert: »40.« und mit
                                    »Schnitzler« beschriftet }
\buchAbdrucke{\weitereDrucke{1) Georg Brandes, Arthur Schnitzler: \emph{Ein Briefwechsel}. Bern: \emph{Francke} 1956, S. 122–123.} \weitereDrucke{2) Arthur Schnitzler: \emph{Briefe 1913–1931}. Frankfurt am Main: \emph{S. Fischer} 1984, S. 165–166.} }\toendnotes[C]{\smallbreak}
\pstart
           \raggedleft{}{\pb}2. 8 1918{\\}Wien XVII. Sternwartestr. 71\oindex{Sternwartestrasse 71@\textbf{Sternwartestraße 71}, \emph{Wohngebäude (K.WHS)}|pw}\pend
           
\pstart{}mein lieber und verehrter Herr Brandes,\pend\vspace{0.5em}
\pstart
           ich lese vom Tode Peter Nansens\pwindex{Nansen, Peter 20.01.1861 – 31.07.1918@\textsc{Nansen, Peter} (20.01.1861 – 31.07.1918), \emph{Schriftsteller/Schriftstellerin, Journalist/Journalistin, Verleger/Verlegerin}|pw}, und habe das
               Bedürfnis irgend jemandem zu sagen, wie tief mich das Hinscheiden dieses
               liebenswerthen Menschen bewegt, den ich zuletzt kurz vor Ausbruch des Kriegs bei mir
               in Wien\oindex{Wien@\textbf{Wien}, \emph{A.ADM2}|pw} gesehen habe – schon recht verändert, ja
               irgendwie gezeichnet – aber doch noch von dem ganzen Zauber seines Wesens umwittert,
               den ich, fast mehr als aus seinen reizvollen Büchern, aus seinem Gehaben, seiner Art
               zu sprechen, seinem Schweigen, seinen Blicken zu spüren vermeinte. Nun fügt es der
               Zufall, daß ich mir gerade in der letzten Zeit Ihre Briefe, lieber und verehrter
               Freund abschreiben ließ – einige, mit Bleistift geschrieben, waren fast unlesbar
               geworden, – und nun, da ich sie, \uline{vom ersten} bis zum
               letzten, \strikeout{alle} – mit welchem Vergnügen! – wieder
               durchnahm, fand ich öfters Peter Nansens\pwindex{Nansen, Peter 20.01.1861 – 31.07.1918@\textsc{Nansen, Peter} (20.01.1861 – 31.07.1918), \emph{Schriftsteller/Schriftstellerin, Journalist/Journalistin, Verleger/Verlegerin}|pw} Namen
               wiederkehren; auch von seinem Kranksein ist die Rede darin, und da liegt es nahe mich
               mit meinem Beileid, – meinem Leid an Sie zu wenden, der Nansens\pwindex{Nansen, Peter 20.01.1861 – 31.07.1918@\textsc{Nansen, Peter} (20.01.1861 – 31.07.1918), \emph{Schriftsteller/Schriftstellerin, Journalist/Journalistin, Verleger/Verlegerin}|pw} Freund war und für mich zugleich, und für die meisten
               Mitlebenden, {\pb}der repraesentative Mann Daenemarks\oindex{Daenemark@\textbf{Dänemark}, \emph{A.PCLI}|pw} ist. Und ich benutze die Gelegenheit
               Ihnen wieder einmal, über diese zerrissene und stöhnende Welt hin\substVorne{}\textsuperscript{über}\substDazwischen{}weg\substHinten{}, die Hand zu drücken um Ihnen zu sagen, mit welcher Sympathie, ja darf ich
               es etwas sentimental ausdrücken –: mit welcher Sehnsucht ich Ihrer gedenke! Von Ihren
               letzten Büchern haben Sie mir geschrieben;– vom Goethe\pwindex{Goethe, Johann Wolfgang von 1749-08-28 – 1832-03-22@\textsc{Goethe, Johann Wolfgang von} (1749-08-28 – 1832-03-22), \emph{Schriftsteller/Schriftstellerin}|pw}\pwindex{Wolfgang Goethe@\emph{Wolfgang Goethe}|pwv} und Voltaire\pwindex{Voltaire 21.11.1694 – 30.05.1778@\textsc{Voltaire} (21.11.1694 – 30.05.1778), \emph{Schriftsteller/Schriftstellerin, Philosoph/Philosophin}|pw}\pwindex{Voltaire und sein Jahrhundert@\emph{Voltaire und sein Jahrhundert}|pwv};– sie existiren noch nicht in deutscher Sprache, – und nun werden Sie wohl auch
               Ihren Julius Caesar\pwindex{Caesar, Gaius Iulius 13.7.100? v. Chr. – 15.3.44 v. Chr.@\textsc{Caesar, Gaius Iulius} (13.7.100? v. Chr. – 15.3.44 v. Chr.), \emph{Politiker/Politikerin, Kaiser/Kaiserin, Heerführer/Heerführerin}|pw}\pwindex{Gaius Julius Cæsar@\emph{Gaius Julius Cæsar}|pwv} bald abschliessen. Aber wa{\geminationn} werd ich Ignorant, der
               nicht daenisch versteht, sie endlich lesen dürfen? – Auch ich hab allerlei gemacht –
               nicht so bedeutungsvolles! – und nach meiner alten zudringlichen Gewohnheit werd ich
               Ihnen ein Stück\pwindex{Schwestern oder Casanova in Spa. Lustspiel in Versen@\emph{Die Schwestern oder Casanova in Spa. Lustspiel in Versen}|pwv} und eine Novelle\pwindex{Casanovas Heimfahrt@\emph{Casanovas Heimfahrt}|pwv} zusenden, sobald sie
               gedruckt sind. – Aber wann werden wir einander wiedersehen? Lassen Sie mich doch bald
               wieder – und wärs nur mit einem Wort, wissen, daß Sie sich wohl befinden und Ihre
               edle Stirn über den Dunst und Dampf dieser Jammerwelt in {\pb}reinere Lüfte emporzurecken vermögen. Ihnen im
               neutralen Land\oindex{Daenemark@\textbf{Dänemark}, \emph{A.PCLI}|pwv} ist es doch
               immerhin leichter als uns. In meiner Familie geht es ganz leidlich; mein Bub\pwindex{Schnitzler, Heinrich 09.08.1902 – 12.07.1982@\textsc{Schnitzler, Heinrich} (09.08.1902 – 12.07.1982), \emph{Regisseur/Regisseurin, Schauspieler/Schauspielerin}|pwv} (wird 16) meine Tochter\pwindex{Cappellini, Lili 13.09.1909 – 26.07.1928@\textsc{Cappellini, Lili} (13.09.1909 – 26.07.1928)|pwv} (wird 9) entwickeln
               sich in jeder Hinsicht gut; meine Frau\pwindex{Schnitzler, Olga 17.01.1882 – 13.01.1970@\textsc{Schnitzler, Olga} (17.01.1882 – 13.01.1970), \emph{Schauspieler/Schauspielerin, Sänger/Sängerin}|pwv} hat wohl unter den häuslichen Kriegswirtschaftssorgen wie jede u jeder
               etwas gelitten, trotzdem aber ihre Kunst nicht vernachlässigt, ihre Stimme entwickelt
               sich aufs schönste. Nun ist sie bei ihrer Schwester\pwindex{Steinrueck, Elisabeth 19.11.1885 – 07.04.1920@\textsc{Steinrück, Elisabeth} (19.11.1885 – 07.04.1920)|pwv} in Bayern\oindex{Bayern@\textbf{Bayern}, \emph{A.ADM1}|pw}
                  (Partenkirchen\oindex{Garmisch-Partenkirchen@\textbf{Garmisch-Partenkirchen}, \emph{P.PPLA3}|pw}) wohin ich Mitte dieses Monats
               auch zu fahren gedenke. Über politisches ka{\geminationn} ich mich in
               einem Brief nicht so ausführlich äußern als ich möchte – wie complicirt gerade bei
               uns all diese Probleme sind, ersehen Sie aus jeder Zeitung, selbst aus dem
               censurirtesten Wien\oindex{Wien@\textbf{Wien}, \emph{A.ADM2}|pw}er Blatt. Und trotz aller
               Schwierigkeiten – Misslichkeiten – Unsicherheiten: wie viel Auftrieb, Sti{\geminationm}ungskraft, Talent – welche positive Möglichkeiten in
               diesem Land\oindex{Oesterreich@\textbf{Österreich}, \emph{A.PCLI}|pwv}, das vielleicht
               nicht {\pb}alle seine Bewohner als »Vaterland« aber
               jeder als »Heimat« liebt. Ich muß hier innehalten – trotzdem ich daran bin, viel
               freundlicheres über Oesterreich\oindex{Oesterreich@\textbf{Österreich}, \emph{A.PCLI}|pw} zu sagen, als
               es \introOben{}selbst\introOben{} unsere officiösen Zeitungen zu thun pflegen.\pend
           
\pstart
           Bitte bestätigen Sie mir bald den Empfang dieses Briefes und erhalten Sie mir und den
               Meinen Ihre Freundschaft.\pend
           
\pstart
           Von Herzen{\\[\baselineskip]}Ihr{\\[\baselineskip]}\spacefill\mbox{Arthur Schnitzler}\pend
           \leftskip=0em{}\selectlanguage{ngerman}\endnumbering\briefempfaengerindex{Brandes, Georg@\textsc{Brandes, Georg}!zzzSchnitzler, Arthur@\emph{von Arthur Schnitzler}!1918-08-021@{2. 8. 1918}|)be}\mylabel{L02292h}  \normalsize

\doendnotes{C}
\bigskip
\vfill

\clearpage

\footnotesize

\lohead{\textsc{register}}

% Definiere theindex-Environment komplett neu ohne reledmac
\makeatletter
\renewenvironment{theindex}{%
  \section*{\indexname}%
  \setlength{\parindent}{0pt}%
  \setlength{\parskip}{0pt plus 0.3pt}%
  \let\item\@idxitem
}{%
  \clearpage
}
\makeatother

\IfFileExists{\jobname-pw.ind}{\input{\jobname-pw.ind}}{}

\end{document}

      