%% latex-korrekturansicht-vorspann.tex
%% Vorspann für die Korrekturansicht.
%% Lädt die gemeinsame Datei latex-vorspann.tex mit gesetztem Schalter.

\newif\ifkorrekturansicht
\korrekturansichttrue

\input{../tex-inputs/latex-vorspann}


\section[Gerhart Hauptmann an Arthur Schnitzler, {[}25.?{]} 7. 1899]{L00950 Gerhart Hauptmann an Arthur Schnitzler, {[}25.?{]} 7. 1899}
\nopagebreak\mylabel{L00950v}
\rehead{ }\normalsize\beginnumbering\briefempfaengerindex{Schnitzler, Arthur@\textsc{Schnitzler, Arthur}!zzzHauptmann, Gerhart@\emph{von Gerhart Hauptmann}!1899-07-251@{{[}25.?{]} 7. 1899}|(be}
\toendnotes[C]{\smallbreak\pagebreak[2]}\Standort{DLA, A:Schnitzler, 66.206.}
\physDesc{Brief, 1 Blatt, 1 Seite, 619 Zeichen
\newline{}Handschrift: schwarze Tinte, lateinische Kurrent
\newline{}Schnitzler: mit Bleistift datiert: »Juli 99« 
\newline{}Ordnung: mit Bleistift von unbekannter Hand seitlich am Blatt: »{\char`~} e\textcolor{gray}{v.}« }\toendnotes[C]{\smallbreak}
\pstart{}{\pb}Lieber Herr Schnitzler.\pend\vspace{0.5em}
\pstart
           ich empfing erst \label{K_L00950-1v}\edtext{hier\oindex{Szklarska Poręba@\textbf{Szklarska Poręba}, \emph{P.PPL}|pw}}{\lemma{\textnormal{\emph{hier}}}\Cendnote{\textnormal{Hauptmann\pwindex{Hauptmann, Gerhart 15.11.1862 – 06.06.1946@\textsc{Hauptmann, Gerhart} (15.11.1862 – 06.06.1946), \emph{Schriftsteller/Schriftstellerin}|pwk} kam am 24. 7. 1899
                  nach Schreiberhau\oindex{Szklarska Poręba@\textbf{Szklarska Poręba}, \emph{P.PPL}|pwk}, wo er das
                  Korrespondenzstück vorfand. Er dürfte es an einem der darauffolgenden Tage
                  beantwortet haben.}}}\label{K_L00950-1} Ihren Brief. Sie sind so liebenswürdig und es ist mir so
               schwer, Ihnen etwas abzuschlagen. Aber das kann ich ja gar nicht thun, was Sie
               wünschen. Wäre ich in Wien\oindex{Wien@\textbf{Wien}, \emph{A.ADM2}|pw}! Allein ich bin ja
               meistens weit weg und fühle zu genau, dass es über meine Kräfte geht, in der Weise
               mitzuwirken, wie es sein müsste, wenn ich meinen Namen auf dem Blatttitel
               rechtfertigen sollte.\pend
           
\pstart
           Seien Sie mir gegrüsst. Ich denke oft an unsern \label{K_L00950-2v}\edtext{Spaziergang}{\lemma{\textnormal{\emph{Spaziergang}}}\Cendnote{\textnormal{Vgl. A. S.: \emph{Tagebuch}, 22. 1. 1899.
               }}}\label{K_L00950-2} auf dem Semmering\oindex{Semmering@\textbf{Semmering}, \emph{A.ADM3}|pw} und hoffe herzlich, Sie
               bald einmal, und am liebsten ausserhalb der Stadtmauern, wiederzusehen\pend
           \pstart Viele Grüsse von Ihrem ergebenen \spacefill\mbox{Gerhart Hauptmann}\pend{}\selectlanguage{ngerman}\endnumbering\briefempfaengerindex{Schnitzler, Arthur@\textsc{Schnitzler, Arthur}!zzzHauptmann, Gerhart@\emph{von Gerhart Hauptmann}!1899-07-251@{{[}25.?{]} 7. 1899}|)be}\mylabel{L00950h}  \normalsize

\doendnotes{C}
\bigskip
\vfill

\clearpage

\footnotesize

\lohead{\textsc{register}}

% Definiere theindex-Environment komplett neu ohne reledmac
\makeatletter
\renewenvironment{theindex}{%
  \section*{\indexname}%
  \setlength{\parindent}{0pt}%
  \setlength{\parskip}{0pt plus 0.3pt}%
  \let\item\@idxitem
}{%
  \clearpage
}
\makeatother

\IfFileExists{\jobname-pw.ind}{\input{\jobname-pw.ind}}{}

\end{document}

      