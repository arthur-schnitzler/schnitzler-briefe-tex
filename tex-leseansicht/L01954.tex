%% latex-korrekturansicht-vorspann.tex
%% Vorspann für die Korrekturansicht.
%% Lädt die gemeinsame Datei latex-vorspann.tex mit gesetztem Schalter.

\newif\ifkorrekturansicht
\korrekturansichttrue

\input{../tex-inputs/latex-vorspann}


\section[Arthur und Olga Schnitzler an Richard Beer-Hofmann, 23. 8. 1910]{L01954 Arthur und Olga Schnitzler an Richard Beer-Hofmann, 23. 8. 1910}
\nopagebreak\mylabel{L01954v}
\rehead{ }\normalsize\beginnumbering\briefempfaengerindex{Beer-Hofmann, Richard@\textsc{Beer-Hofmann, Richard}!zzzSchnitzler, Olga@\emph{von Olga Schnitzler}!1910-08-231@{23. 8. 1910}|(be}\briefempfaengerindex{Beer-Hofmann, Richard@\textsc{Beer-Hofmann, Richard}!zzzSchnitzler, Arthur@\emph{von Arthur Schnitzler}!1910-08-231@{23. 8. 1910}|(be}
\toendnotes[C]{\smallbreak\pagebreak[2]}\Standort{YCGL, MSS 31.}
\physDesc{Bildpostkarte, 346 Zeichen
\newline{}Handschrift Arthur Schnitzler: Bleistift, deutsche Kurrent
\newline{}Handschrift Olga Schnitzler: Bleistift, lateinische Kurrent
\newline{}Versand: Stempel: »\nobreak{}\oindex{Partenkirchen@\textbf{Partenkirchen}, \emph{Teil eines besiedelten Ortes (A.BSOX)}|pwk}Parten\textcolor{gray}{kirchen}, {[}23. 8.{]} 10, 6–7\nobreak{}«.  }
\buchAbdrucke{\weitereDrucke{Arthur Schnitzler, Richard Beer-Hofmann: \emph{Briefwechsel 1891–1931}. Wien, Zürich: \emph{Europaverlag} 1992, S. 212 .} }\toendnotes[C]{\smallbreak}\pstart{}{\pb}Hrn Dr. \textsc{Richard Beer
                     Hofmann}\pend{}\pstart{}\textsc{Ischl\oindex{Bad Ischl@\textbf{Bad Ischl}, \emph{P.PPL}|pw}}\pend{}\pstart{}\textsc{Steinfeld\strikeout{\textcolor{gray}{gass}} 6\oindex{Steinfeld@\textbf{Steinfeld}, \emph{P.PPL}|pw}}.\pend{}{\bigskip}
\pstart
           \noindent{}\centering{}{\pb}\textcolor{gray}{\textbf{Partenkirchen\oindex{Partenkirchen@\textbf{Partenkirchen}, \emph{Teil eines besiedelten Ortes (A.BSOX)}|pw}. St. Anton\oindex{St. Anton [Partenkirchen]@\textbf{St. Anton [Partenkirchen]}, \emph{Kirche (K.KRC)}|pw} mit Blick auf Dreithorspitze\oindex{Dreitorspitze@\textbf{Dreitorspitze}, \emph{Berg (N.BRG)}|pw}.}}\pend
           \vspace{1em}
\pstart
           \centering{}{\pb}23. 8. 1910\pend
           \vspace{0.5em}
\pstart
           Herzliche Grüße!{\\[\baselineskip]}\spacefill\mbox{A.}\pend
           \leftskip=0em{}\selectlanguage{ngerman}\vspace{1em}
\pstart
           \noindent{}{[}hs. :{]} Heute wurde am Krankenbett meiner Schwester\pwindex{Steinrueck, Elisabeth 19.11.1885 – 07.04.1920@\textsc{Steinrück, Elisabeth} (19.11.1885 – 07.04.1920)|pwv} viel von Ihnen gesprochen. Sie
               sagt immer wieder: »B.-H. ist von Euch allen der merscht Begabte!«\pend
           
\pstart
           Von hier fahren wir \uline{nicht} nach Ischl\oindex{Bad Ischl@\textbf{Bad Ischl}, \emph{P.PPL}|pw}, sondern Frankfurt\oindex{Frankfurt am Main@\textbf{Frankfurt am Main}, \emph{P.PPLA3}|pw}{ }\label{K_L01954-1v}\edtext{Liebelei Opern-Première\pwindex{Liebelei. Oper in drei Akten@\emph{Liebelei. Oper in drei Akten}|pw}}{\lemma{\textnormal{\emph{Liebelei Opern-Première}}}\Cendnote{\textnormal{Siehe A. S.: \emph{Tagebuch}, 18. 9. 1910.
               }}}\label{K_L01954-1}, vorher Heidelberg\oindex{Heidelberg@\textbf{Heidelberg}, \emph{P.PPLA3}|pw}.\pend
           \pstart Herzlichste Grüsse Ihnen Allen! \spacefill\mbox{O. S.}\pend{}\selectlanguage{ngerman}\endnumbering\briefempfaengerindex{Beer-Hofmann, Richard@\textsc{Beer-Hofmann, Richard}!zzzSchnitzler, Olga@\emph{von Olga Schnitzler}!1910-08-231@{23. 8. 1910}|)be}\briefempfaengerindex{Beer-Hofmann, Richard@\textsc{Beer-Hofmann, Richard}!zzzSchnitzler, Arthur@\emph{von Arthur Schnitzler}!1910-08-231@{23. 8. 1910}|)be}\mylabel{L01954h}  \normalsize

\doendnotes{C}
\bigskip
\vfill

\clearpage

\footnotesize

\lohead{\textsc{register}}

% Definiere theindex-Environment komplett neu ohne reledmac
\makeatletter
\renewenvironment{theindex}{%
  \section*{\indexname}%
  \setlength{\parindent}{0pt}%
  \setlength{\parskip}{0pt plus 0.3pt}%
  \let\item\@idxitem
}{%
  \clearpage
}
\makeatother

\IfFileExists{\jobname-pw.ind}{\input{\jobname-pw.ind}}{}

\end{document}

      