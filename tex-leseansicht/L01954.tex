%% latex-leseansicht-vorspann.tex
%% Vorspann für die Leseansicht.
%% Lädt die gemeinsame Datei latex-vorspann.tex mit nicht gesetztem Schalter.

\newif\ifkorrekturansicht
\korrekturansichtfalse

\input{../tex-inputs/latex-vorspann}


         
         \newcommand{\erwaehntePersonen}{Personen: Richard Beer-Hofmann, Elisabeth Steinrück}
         \newcommand{\erwaehnteOrte}{Orte: Bad Ischl, Dreitorspitze, Frankfurt am Main, Garmisch-Partenkirchen, Heidelberg, Partenkirchen, St. Anton, Steinfeld}
         \newcommand{\erwaehnteWerke}{Werke: Liebelei. Oper in drei Akten}
               \section[Arthur und Olga Schnitzler an Richard Beer-Hofmann, 23. 8. 1910]{ Arthur und Olga Schnitzler an Richard Beer-Hofmann, 23. 8. 1910}\nopagebreak\mylabel{v}\rehead{ }\begin{ledgroupsized}[t]{13cm}\normalsize\beginnumbering \toendnotes[C]{\smallbreak\pagebreak[2]} \Standort{YCGL, MSS 31.}
\physDesc{Bildpostkarte
\newline{}Handschrift Arthur Schnitzler: Bleistift, deutsche Kurrent\newline{}Handschrift Olga Schnitzler: Bleistift, lateinische Kurrent\newline{}Versand: Stempel: »\nobreak{}\oindex{Partenkirchen@\textbf{Partenkirchen}|pwk}Parten\textcolor{gray}{kirchen}, {[}23. 8.{]} 10, 6–7\nobreak{}«.  }\buchAbdrucke{\weitereDrucke{Arthur Schnitzler, Richard Beer-Hofmann: \emph{Briefwechsel 1891–1931}. Hg. Konstanze Fliedl. Wien, Zürich: \emph{Europaverlag} 1992, S. 212 .} }\toendnotes[C]{\smallbreak}\pstart{}{\pb}Hrn Dr. \textsc{Richard Beer
                     Hofmann}\pend{}\pstart{}\textsc{Ischl\oindex{Bad Ischl@\textbf{Bad Ischl}|pw}}\pend{}\pstart{}\textsc{Steinfeld\strikeout{\textcolor{gray}{gass}} 6\oindex{Steinfeld@\textbf{Steinfeld}|pw}}.\pend{}{\bigskip}\pstart
           \noindent{}\centering{}{\pb}\textcolor{gray}{\textbf{Partenkirchen\oindex{Partenkirchen@\textbf{Partenkirchen}|pw}. St. Anton\oindex{St. Anton@\textbf{St. Anton}|pw} mit Blick auf Dreithorspitze\oindex{Dreitorspitze@\textbf{Dreitorspitze}|pw}.}}\pend
           \pstart
           \centering{}{\pb}23. 8. 1910\pend
           \pstart
           Herzliche Grüße!{\\[\baselineskip]}\spacefill\mbox{A.}\pend
           \leftskip=0em{}\pstart
           \noindent{}{[}hs. Olga Schnitzler:{]} Heute wurde am Krankenbett meiner Schwester\pwindex{Steinrueck, Elisabeth 19.11.1885 – 07.04.1920@\textsc{Steinrück, Elisabeth} (19.11.1885 – 07.04.1920)|pwv} viel von Ihnen gesprochen. Sie sagt
               immer wieder: »B.-H. ist von Euch allen der merscht Begabte!«\pend
           \pstart
           Von hier fahren wir \uline{nicht} nach Ischl\oindex{Bad Ischl@\textbf{Bad Ischl}|pw}, sondern Frankfurt\oindex{Frankfurt am Main@\textbf{Frankfurt am Main}|pw}{ }\label{K_L01954_1v}\edtext{Liebelei Opern-Première\pwindex{Schnitzler, Arthur 15.05.1862 – 21.10.1931@\textsc{Schnitzler, Arthur} (15.05.1862 – 21.10.1931), \emph{Schriftsteller, Mediziner}!Liebelei. Oper in drei Akten1909-09-18@\strich\emph{Liebelei. Oper in drei Akten} {[}1909-09-18{]}|pw}}{\lemma{\textnormal{\emph{Liebelei Opern-Première}}}\Cendnote{\textnormal{siehe A. S.: \emph{Tagebuch}, 18. 9. 1910}}}\label{K_L01954_1h}, vorher Heidelberg\oindex{Heidelberg@\textbf{Heidelberg}|pw}.\pend
           \pstart Herzlichste Grüsse Ihnen Allen! \spacefill\mbox{O. S.}\pend{}
         
         \endnumbering\mylabel{h}\end{ledgroupsized}  \newcommand{\dateiname}{L01954}\newcommand{\titel}{Arthur und Olga Schnitzler an Richard Beer-Hofmann, 23. 8. 1910}\newcommand{\editorInnen}{Martin Anton Müller und Gerd-Hermann Susen}%% latex-leseansicht-abspann.tex
%% Abspann für die Leseansicht.
%% Der Schalter \ifkorrekturansicht ist bereits durch den Vorspann gesetzt.

%% latex-abspann.tex
%% Gemeinsamer Abspann für Korrekturansicht und Leseansicht.
%% Setzt den Schalter \ifkorrekturansicht voraus (gesetzt in den
%% einbindenden Dateien latex-korrekturansicht-abspann.tex bzw.
%% latex-leseansicht-abspann.tex).
%% ---------------------------------------------------------------

\normalsize

% Das esempio-Environment wird nur in der Leseansicht benötigt
\ifkorrekturansicht\else
\newenvironment{esempio}[3]%
{
    \vspace{1.5ex}
    \rlap{\underline{#1}}
    \par
    \setlength{\parindent}{0cm}
    \nopagebreak
    \leftskip=#2cm
    \rightskip=#3cm
}
{
    \par
}
\fi

\doendnotes{C}
\bigskip
\vfill

\clearpage

\footnotesize

\ifkorrekturansicht
  \lohead{\textsc{register}}
\fi

% theindex-Environment neu definieren ohne reledmac
\makeatletter
\renewenvironment{theindex}{%
  \ifkorrekturansicht
    \section*{\indexname}%
  \else
    \subsubsection*{Index der erwähnten Entitäten}%
  \fi
  \setlength{\parindent}{0pt}%
  \setlength{\parskip}{0pt plus 0.3pt}%
  \let\item\@idxitem
}{%
  \ifkorrekturansicht\clearpage\fi
}
\makeatother

\IfFileExists{\jobname-pw.ind}{\input{\jobname-pw.ind}}{}

% Quellenangabe nur in der Leseansicht
\ifkorrekturansicht\else
% Fallback-Definitionen, falls die .tex-Datei \titel etc. nicht gesetzt hat
\providecommand{\titel}{}
\providecommand{\editorInnen}{}
\providecommand{\dateiname}{\jobname}

\vspace{3cm}

\vfill

\footnotesize
\textsc{Quelle}: \titel. Herausgegeben von {\editorInnen}. In: \emph{Arthur Schnitzler: Briefwechsel mit Autorinnen und Autoren}.
 Digitale Edition, https://schnitzler-briefe.acdh.oeaw.ac.at/{\dateiname}.html (Stand \today)
\fi

\end{document}


      