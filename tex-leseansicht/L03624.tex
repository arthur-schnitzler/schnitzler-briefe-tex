%% latex-korrekturansicht-vorspann.tex
%% Vorspann für die Korrekturansicht.
%% Lädt die gemeinsame Datei latex-vorspann.tex mit gesetztem Schalter.

\newif\ifkorrekturansicht
\korrekturansichttrue

\input{../tex-inputs/latex-vorspann}


\section[Stefan Zweig an Arthur Schnitzler, 13. 12. 1909]{L03624 Stefan Zweig an Arthur Schnitzler, 13. 12. 1909}
\nopagebreak\mylabel{L03624v}
\rehead{ }\normalsize\beginnumbering\briefempfaengerindex{Schnitzler, Arthur@\textsc{Schnitzler, Arthur}!zzzZweig, Stefan@\emph{von Stefan Zweig}!1909-12-133@{13. 12. 1909}|(be}
\toendnotes[C]{\smallbreak\pagebreak[2]}\Standort{CUL, Schnitzler, B 118.}
\physDesc{Brief, 1 Blatt, 3 Seiten, 1909 Zeichen
\newline{}Handschrift: lila Tinte, lateinische Kurrent
\newline{}Schnitzler: mit Bleistift »\textsc{Zweig}« }
\buchAbdrucke{\weitereDrucke{1) Stefan Zweig: \emph{Briefwechsel mit Hermann Bahr, Sigmund Freud, Rainer Maria
                        Rilke und Arthur Schnitzler}. Frankfurt am Main: \emph{S. Fischer} 1987, S. 357–358.} \weitereDrucke{2) Stefan Zweig: \emph{Briefe. Bd. I: 1897–1914}. Frankfurt am Main: \emph{S. Fischer} 1995, S. 201.} }\toendnotes[C]{\smallbreak}
\pstart
           {\pb}\textcolor{gray}{\textbf{SZ}}\hfill \textcolor{gray}{\textbf{VIII. KOCHGASSE 8\oindex{Kochgasse 8@\textbf{Kochgasse 8}, \emph{Wohngebäude (K.WHS)}|pw}}}\pend
           
\pstart
           \raggedleft{}\textcolor{gray}{\textbf{WIEN\oindex{Wien@\textbf{Wien}, \emph{A.ADM2}|pw},}}{ }13. Dez 09\pend
           {\vspace{1\baselineskip}}
\pstart{}Sehr verehrter Herr Doktor,\pend\vspace{0.5em}
\pstart
           ich hatte \label{K_L03624-1v}\edtext{gestern}{\lemma{\textnormal{\emph{gestern}}}\Cendnote{\textnormal{\emph{Der Ruf des Lebens}\pwindex{Ruf des Lebens. Schauspiel in drei Akten@\emph{Der Ruf des Lebens. Schauspiel in drei Akten}|pwk} von Schnitzler erlebte am 11. 12. 1909 am Deutschen Volkstheater\oindex{Volkstheater@\textbf{Volkstheater}, \emph{Theater (K.THE)}|pwk} seine Wiener\oindex{Wien@\textbf{Wien}, \emph{A.ADM2}|pwk}
                  Erstaufführung. Am 12. 12. 1909 fand die zweite Vorstellung statt. Schnitzler wohnte beiden Aufführungen bei, vgl. A. S.: \emph{Tagebuch}, 11. 12. 1909 und 12. 12. 1909.}}}\label{K_L03624-1}
               die Freude, der erfolgreichen Aufführung Ihres »Ruf
                  des Lebens\pwindex{Ruf des Lebens. Schauspiel in drei Akten@\emph{Der Ruf des Lebens. Schauspiel in drei Akten}|pw}« beizuwohnen. Es wäre ungeziemend wollte ich mir eine Bemerkung
               über das Wesen und den Wert des Stückes\pwindex{Ruf des Lebens. Schauspiel in drei Akten@\emph{Der Ruf des Lebens. Schauspiel in drei Akten}|pwv}{ }\introOben{}zu\introOben{} Ihnen ungefragt gestatten, aber das darf ich Ihnen wohl
               sagen, dass ich vielleicht niemals von einem Ihrer Werke im Theater einen so
               gewaltigen und wirklich die letzten Erschütterungen aufwühlenden Eindruck empfunden
               habe. Sie bedürfen heute längst nicht mehr einer Zustimmung – am wenigsten von uns,
               die wir alle an Ihnen zu lernen haben – aber eben, weil diesem Stück\pwindex{Ruf des Lebens. Schauspiel in drei Akten@\emph{Der Ruf des Lebens. Schauspiel in drei Akten}|pwv}{ }\label{K_L03624-2v}\edtext{soviel Missverständnis}{\lemma{\textnormal{\emph{soviel Missverständnis}}}\Cendnote{\textnormal{Schnitzler vermerkt im \emph{Tagebuch}\pwindex{Tagebuch@\emph{Tagebuch}|pwk} am 17. 12. 1909 nach einem Gespräch mit Stefan Zweig\pwindex{Zweig, Stefan 28.11.1881 – 23.02.1942@\textsc{Zweig, Stefan} (28.11.1881 – 23.02.1942), \emph{Schriftsteller/Schriftstellerin}|pwk}, dass dieser »mit einem
                     Vorurtheil nach den Berliner\oindex{Berlin@\textbf{Berlin}, \emph{P.PPLC}|pw} Kritiken
                     gekommen und ganz gewonnen« worden sei. Die Berliner\oindex{Berlin@\textbf{Berlin}, \emph{P.PPLC}|pwk} Premiere am 24. 2. 1906 war ambivalent
                  besprochen worden, vgl. etwa: 
                     Rudolf Herzog\pwindex{Herzog, Rudolf 06.12.1869 – 03.03.1943@\textsc{Herzog, Rudolf} (06.12.1869 – 03.03.1943), \emph{Schriftsteller/Schriftstellerin}|pwk}: \emph{Lessing Theater. Zum ersten Male: »Der Ruf des Lebens«,
                        Schauspiel in drei Akten von Arthur Schnitzler}\pwindex{Lessing Theater. Zum ersten Male: »Der Ruf des Lebens«, Schauspiel in drei Akten von Arthur Schnitzler@\emph{Lessing Theater. Zum ersten Male: »Der Ruf des Lebens«, Schauspiel in drei Akten von Arthur Schnitzler}|pwk}. In: \emph{Berliner Neueste Nachrichten}\pwindex{Berliner Neueste Nachrichten@\emph{Berliner Neueste Nachrichten}|pwk}, Jg. 26, Nr. 94,
                        25. 2. 1906, S. 3. M. J. [=Max Jordan]\pwindex{Jordan, Max 1837-06-19 – 1906-11-11@\textsc{Jordan, Max} (1837-06-19 – 1906-11-11), \emph{Museumsdirektor/Museumsdirektorin, Kunsthistoriker/Kunsthistorikerin, Kritiker/Kritikerin}|pwk}: \emph{Lessing Theater. Zum ersten Mal: »Der Ruf des Lebens«,
                        Schauspiel in drei Akten von Arthur Schnitzler}\pwindex{Lessing Theater. Zum ersten Mal: »Der Ruf des Lebens«, Schauspiel in drei Akten von Arthur Schnitzler@\emph{Lessing Theater. Zum ersten Mal: »Der Ruf des Lebens«, Schauspiel in drei Akten von Arthur Schnitzler}|pwk}. In: \emph{Berliner Tageblatt}\pwindex{Berliner Tageblatt@\emph{Berliner Tageblatt}|pwk}, Jg. 35, Nr. 102,
                        25. 2. 1906, S. 2–3. Alfred Kerr\pwindex{Kerr, Alfred 25.12.1867 – 12.10.1948@\textsc{Kerr, Alfred} (25.12.1867 – 12.10.1948), \emph{Schriftsteller/Schriftstellerin, Kritiker/Kritikerin}|pwk}: \emph{Ödipus und der Ruf des Lebens}\pwindex{Oedipus und der Ruf des Lebens@\emph{Ödipus und der Ruf des Lebens}|pwk}, in: \emph{Neue Rundschau}\pwindex{neue Rundschau@\emph{Die neue Rundschau}|pwk}, Jg. 17, H. 5, Mai 1906,
                     S. 492-498. [Siegfried Jacobsohn]\pwindex{Jacobsohn, Siegfried 28.01.1881 – 03.12.1926@\textsc{Jacobsohn, Siegfried} (28.01.1881 – 03.12.1926), \emph{Journalist/Journalistin, Kritiker/Kritikerin, Publizist/Publizistin}|pwk}: \emph{Der Ruf des Lebens}\pwindex{Ruf des Lebens@\emph{Der Ruf des Lebens}|pwk}. In: \emph{Die Schaubühne}\pwindex{Schaubuehne@\emph{Die Schaubühne}|pwk}, Jg. 2, Nr. 9, März 1906,
                     S. 246–250. Auch Schnitzler selbst war, besonders 
               vom zweiten Akt, nicht überzeugt und versuchte zeitlebens immer wieder, die Schwächen des Stückes zu beheben,
               jedoch ohne eine neue Fassung fertigzustellen.}}}\label{K_L03624-2} – feind{\pb}lich oder auch freundlich – gegenüber stand, möchte ich Ihnen sagen, dass ich das
               Gefühl gänzlichen Einverständnis hatte. Ich habe wie selten hier die Gefühle in einer
                  nah\strikeout{t}en und doch nicht schamlosen menschlichen
               Körperlichkeit gefühlt und den ungeheuren Raum wirklich mit einem süssen und
               bezwingenden Schrecken aufgerissen gesehen, der zwischen dem intensivesten Leben und
               dem Nichts plötzlich aufspringen kann. Nie, soweit ich Ihr Werk überschaue, haben Sie
               eine ähnliche Gewalt über das Schicksal gezeigt und ich wäre froh, wenn Sie \label{T_L03624-1v}\edtext{sich}{\lemma{\textnormal{\emph{sich}}}\Cendnote{\textnormal{Er schreibt: »Sich«.}}}\label{T_L03624-1}
               dieses Stück\pwindex{Ruf des Lebens. Schauspiel in drei Akten@\emph{Der Ruf des Lebens. Schauspiel in drei Akten}|pwv} nicht um ein paar
               theatralischer Dinge willen jemals verärgern oder minder lieb haben liessen. Ich
               werde Ihnen immer dafür dankbar sein und ich glaube, immer mehr werden sich finden,
               die es so fühlen werden: nicht um des Gesagten willen, der Worte und der Menschen
               sosehr, sondern um der ungeheuren Vitalität willen, die aus jedem \substVorne{}\textsuperscript{A}\substDazwischen{}W\substHinten{}esen darin atmet.
               Diese feindliche Um{\pb}schlingung von Leben
               und Tod, die feurige Secunde ihres Einswerdens in der Leidenschaft wird mir
               unvergesslich eine der schönsten Erinnerungen an d\substVorne{}\textsuperscript{ie}\substDazwischen{}en\substHinten{} Abend sein.\pend
           
\pstart
           Nehmen Sie also innigen Dank für dieses Werk\pwindex{Ruf des Lebens. Schauspiel in drei Akten@\emph{Der Ruf des Lebens. Schauspiel in drei Akten}|pwv}, das alte Liebe und Verehrung bei mir nur vermehrt,
               bekräftigt und vertieft hat. Wie freue ich mich Ihrem nächsten entgegen!\pend
           
\pstart
           In herzlicher Ergebenheit{\\[\baselineskip]}\spacefill\mbox{Stefan Zweig}\pend
           \leftskip=0em{}\selectlanguage{ngerman}\endnumbering\briefempfaengerindex{Schnitzler, Arthur@\textsc{Schnitzler, Arthur}!zzzZweig, Stefan@\emph{von Stefan Zweig}!1909-12-133@{13. 12. 1909}|)be}\mylabel{L03624h}  \normalsize

\doendnotes{C}
\bigskip
\vfill

\clearpage

\footnotesize

\lohead{\textsc{register}}

% Definiere theindex-Environment komplett neu ohne reledmac
\makeatletter
\renewenvironment{theindex}{%
  \section*{\indexname}%
  \setlength{\parindent}{0pt}%
  \setlength{\parskip}{0pt plus 0.3pt}%
  \let\item\@idxitem
}{%
  \clearpage
}
\makeatother

\IfFileExists{\jobname-pw.ind}{\input{\jobname-pw.ind}}{}

\end{document}

      