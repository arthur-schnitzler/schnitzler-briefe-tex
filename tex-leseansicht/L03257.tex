%% latex-korrekturansicht-vorspann.tex
%% Vorspann für die Korrekturansicht.
%% Lädt die gemeinsame Datei latex-vorspann.tex mit gesetztem Schalter.

\newif\ifkorrekturansicht
\korrekturansichttrue

\input{../tex-inputs/latex-vorspann}


\section[ Paul Goldmann an Arthur Schnitzler, 18. 8. 1907]{L03257 Paul Goldmann an Arthur Schnitzler, 18. 8. 1907}
\nopagebreak\mylabel{L03257v}
\rehead{ }\normalsize\beginnumbering\briefempfaengerindex{Schnitzler, Arthur@\textsc{Schnitzler, Arthur}!zzzGoldmann, Paul@\emph{von Paul Goldmann}!1907-08-181@{18. 8. 1907}|(be}
\toendnotes[C]{\smallbreak\pagebreak[2]}\Standort{DLA, A:Schnitzler, HS.NZ85.1.3175.}
\physDesc{Postkarte, 373 Zeichen
\newline{}Handschrift: 1) blaue Tinte, deutsche Kurrent\hspace{1em}2) blaue Tinte, lateinische Kurrent (\noindent{}Adresse)\hspace{1em}
\newline{}Versand: 1) Stempel: »\nobreak{}\oindex{Berlin@\textbf{Berlin}, \emph{P.PPLC}|pwk}Berlin,
                                       W\textcolor{gray}{.} 9, 18. 8. 07, 6–7N\nobreak{}«.   2) Stempel: »\nobreak{}\oindex{Welsberg-Taisten@\textbf{Welsberg-Taisten}, \emph{A.ADM3}|pwk}We\textcolor{gray}{ls}{[}berg{]}, 1\textcolor{gray}{×}. 8. \textcolor{gray}{×}\textcolor{gray}{7}\nobreak{}«. }\toendnotes[C]{\smallbreak}\pstart{}{\pb}Herrn\pend{}\pstart{}Dr. Arthur Schnitzler\pend{}\pstart{}Welsberg im Pustertal\oindex{Welsberg-Taisten@\textbf{Welsberg-Taisten}, \emph{A.ADM3}|pw}\pend{}\pstart{}Wildbad Waldbrunn\oindex{Wildbad Waldbrunn@\textbf{Wildbad Waldbrunn}, \emph{S.SPA}|pw}\pend{}\pstart{}Tirol\oindex{Tirol@\textbf{Tirol}, \emph{A.ADM1}|pw}.\pend{}{\bigskip}\vspace{1em}
\pstart
           \noindent{}{\pb}18. 8. 07. Lieber Freund, Da ich mit meiner Mutter\pwindex{Goldmann, Clementine 1842-05-15 – 1924-02-24@\textsc{Goldmann, Clementine} (1842-05-15 – 1924-02-24)|pwv} in möglichſt kleinen Etappen reiſe, komme ich dieſe Woche wohl noch
               nicht nach \textsc{Welsberg\oindex{Welsberg-Taisten@\textbf{Welsberg-Taisten}, \emph{A.ADM3}|pw}}, ſondern bleibe erſt ein paar Tage in \textsc{Gossensass\oindex{Gossensass@\textbf{Gossensaß}, \emph{P.PPLA3}|pw}}, \textsc{Hotel Gröbner\oindex{Grandhotel Groebner@\textbf{Grandhotel Gröbner}, \emph{Hotel (K.HTL)}|pw}}. Schreib’ mir bitte dorthin, wohin Du \label{K_L03257-1v}\edtext{von \textsc{Welsberg\oindex{Welsberg-Taisten@\textbf{Welsberg-Taisten}, \emph{A.ADM3}|pw}} aus}{\lemma{\textnormal{\emph{von Welsberg aus}}}\Cendnote{\textnormal{Schnitzler hielt sich noch bis 26. 8. 1907 im Wildbad Waldbrunn\oindex{Wildbad Waldbrunn@\textbf{Wildbad Waldbrunn}, \emph{S.SPA}|pwk} auf. Zu einem Zusammentreffen
                  mit Goldmann\pwindex{Goldmann, Paul 31.01.1865 – 25.09.1935@\textsc{Goldmann, Paul} (31.01.1865 – 25.09.1935), \emph{Schriftsteller/Schriftstellerin, Journalist/Journalistin}|pwk} kam es nicht, vgl. Arthur Schnitzler an Richard Beer-Hofmann, 25. 8. 1907.}}}\label{K_L03257-1} gehſt.
               Herzliche Grüße Dir u. Deiner Frau\pwindex{Schnitzler, Olga 17.01.1882 – 13.01.1970@\textsc{Schnitzler, Olga} (17.01.1882 – 13.01.1970), \emph{Schauspieler/Schauspielerin, Sänger/Sängerin}|pwv}!\pend
           
\pstart
           Dein {\\[\baselineskip]}\spacefill\mbox{Paul Goldmn}\pend
           \leftskip=0em{}\selectlanguage{ngerman}\endnumbering\briefempfaengerindex{Schnitzler, Arthur@\textsc{Schnitzler, Arthur}!zzzGoldmann, Paul@\emph{von Paul Goldmann}!1907-08-181@{18. 8. 1907}|)be}\mylabel{L03257h}  \normalsize

\doendnotes{C}
\bigskip
\vfill

\clearpage

\footnotesize

\lohead{\textsc{register}}

% Definiere theindex-Environment komplett neu ohne reledmac
\makeatletter
\renewenvironment{theindex}{%
  \section*{\indexname}%
  \setlength{\parindent}{0pt}%
  \setlength{\parskip}{0pt plus 0.3pt}%
  \let\item\@idxitem
}{%
  \clearpage
}
\makeatother

\IfFileExists{\jobname-pw.ind}{\input{\jobname-pw.ind}}{}

\end{document}

      