%% latex-leseansicht-vorspann.tex
%% Vorspann für die Leseansicht.
%% Lädt die gemeinsame Datei latex-vorspann.tex mit nicht gesetztem Schalter.

\newif\ifkorrekturansicht
\korrekturansichtfalse

\input{../tex-inputs/latex-vorspann}


         
         \newcommand{\erwaehntePersonen}{Personen: }
         \newcommand{\erwaehnteInstitutionen}{}
         \newcommand{\erwaehnteOrte}{Orte: Wien, Zistersdorf}
         \newcommand{\erwaehnteWerke}{Werke: Die Fackel, Gesellschaft [Eine Gaunerkomödie], Rechtsphilosophie, Sprüche}
               \section[Robert Adam an Arthur Schnitzler, 16. 7. 1915]{ Robert Adam an Arthur Schnitzler, 16. 7. 1915}\nopagebreak\mylabel{v}\rehead{ }\begin{ledgroupsized}[t]{13cm}\normalsize\beginnumbering \toendnotes[C]{\smallbreak\pagebreak[2]} \Standort{DLA, A:Schnitzler, HS.NZ85.1.4230,10.}
\physDesc{Brief, 1 Blatt, 3 Seiten
\newline{}Handschrift: schwarze Tinte, deutsche Kurrent
\newline{}Schnitzler: 1) mit Bleistift beschriftet: »\textsc{Adam}«  2) mit rotem Buntstift eine Unterstreichung}\toendnotes[C]{\smallbreak}\pstart
           \raggedleft{}{\pb}Ziſtersdorf\oindex{Zistersdorf@\textbf{Zistersdorf}|pw}, 16. Juli 1915\pend
           \pstart{}Hochverehrter Herr Doktor!\pend\pstart
           Ich danke Ihnen herzlich für Ihren Brief und beſtätige die Rückſendung des Manuſkripts\pwindex{Adam, Robert 20.04.1877 – 16.10.1961@\textsc{Adam, Robert} (20.04.1877 – 16.10.1961), \emph{Schriftsteller, Richter}!Gesellschaft [Eine Gaunerkomoedie]None@\strich\emph{Gesellschaft [Eine Gaunerkomödie]} {[}None{]}|pwv}.\pend
           \pstart
           Das Urteil, das Sie über meine Gaunerkomödie\pwindex{Adam, Robert 20.04.1877 – 16.10.1961@\textsc{Adam, Robert} (20.04.1877 – 16.10.1961), \emph{Schriftsteller, Richter}!Gesellschaft [Eine Gaunerkomoedie]None@\strich\emph{Gesellschaft [Eine Gaunerkomödie]} {[}None{]}|pwv} gefällt haben, hat mich einigermaßen betrübt, weil ich
                    an dieſer Arbeit, weshalb weiß ich eigentlich ſelbſt nicht mehr, immer mit einer
                    gewiſſen Affenliebe hing. Beruhte ſie im Grunde vielleicht auf Schadenfreude
                    darüber, daß jene Kumpane, die mir manche ſaure Arbeitsſtunde und viel bitteren
                    Ärger gekoſtet haben, ſich meiner Laune fügen mußten? oder bloß \label{T_L02215_1v}\edtext{aus}{\lemma{\textnormal{\emph{aus}}}\Cendnote{\textnormal{Er schreibt: »auf«.}}}\label{T_L02215_1h} Luſt daran,
                    daß ich die Erinnerung an alle dieſe Quälgeiſter durch ihre Verarbeitung
                    losgeworden bin?\pend
           \pstart
           Sie ſehen, daß es gewiß keine künſtleriſchen Gründe ſind, die ich zur Erklärung
                    meiner Vorliebe heranziehe; und ſo muß ich auch, wenn {\pb}ich mich – gewiß etwas verſpätet – zu objektiver
                    Selbſtkritik aufraffe, ganz einfach offen zugeben, daß ich gegen Ihren
                    Urteilsſpruch keine rechten Berufungsgründe aufzutreiben weiß. Daß ich mir mit
                    dieſer Komödie nicht die Tiefe Berührendes, ſondern wohl nur Ärger von der Seele
                    geſchrieben habe, habe ich bereits angedeutet, und zum Schreiben ſelbſt zwang
                    mich nicht, wie bei andern Arbeiten, die ich ernſt nahm, die Macht einer Idee,
                    die Ausdruck finden will und muß, ſondern lockte mich die Durchführung einer
                    Pointe. Der Pointe geſellte ſich allerdings eine kleine Idee, aber beide waren
                    ſich fremd, und ſo kam es zwiſchen ihnen zu einer mißhelligen Ehe.\pend
           \pstart
           Und jetzt erſt, da mir Ihre Kritik die Komödie\pwindex{Adam, Robert 20.04.1877 – 16.10.1961@\textsc{Adam, Robert} (20.04.1877 – 16.10.1961), \emph{Schriftsteller, Richter}!Gesellschaft [Eine Gaunerkomoedie]None@\strich\emph{Gesellschaft [Eine Gaunerkomödie]} {[}None{]}|pwv}{ }ſo gezeigt hat, wie ſie ſich, ohne meine
                    Vorliebe für sie geſehen, darstellt\strikeout{e}, weiß ich
                    wieder etwas, was mich die – wie geſagt, ſchwer zu begründende – Freude über die
                    vollendete Arbeit vergeſſen ließ: Daß die Hauptveranlaſſung zur Niederſchrift
                    der Komödie eigentlich die ſehr lebhafte Sehnſucht war, endlich einmal etwas zu
                    ſchreiben, was theatermöglich wäre und das große Publikum anzöge. Ich hielt mich
                    einmal an den zweiten Teil meines Wahlſpruchs (der zu den wenigen meiner \label{K_L02215_1v}\edtext{gedruckten \textsc{opera}}{\lemma{\textnormal{\emph{gedruckten opera}}}\Cendnote{\textnormal{Robert Adam\pwindex{Adam, Robert 20.04.1877 – 16.10.1961@\textsc{Adam, Robert} (20.04.1877 – 16.10.1961), \emph{Schriftsteller, Richter}|pwk}: \emph{Sprüche}\pwindex{Adam, Robert 20.04.1877 – 16.10.1961@\textsc{Adam, Robert} (20.04.1877 – 16.10.1961), \emph{Schriftsteller, Richter}!Sprueche12. 03. 1908@\strich\emph{Sprüche} {[}12. 03. 1908{]}|pwk}. In: \emph{Die
                                Fackel}\pwindex{Fackel1899-04 – 1936@\emph{Die Fackel} {[}1899-04 – 1936{]}|pwk}, Jg. 9, H. 246/247, 12. 3. 1908, S. 25–26,
                            hier: S. 26.}}}\label{K_L02215_1h} gehört):\pend
           \stanza{}{\pb}Wie auch dein Sinn nach Ehre
                            ſehnt und ſüchtet\pwindex{Adam, Robert 20.04.1877 – 16.10.1961@\textsc{Adam, Robert} (20.04.1877 – 16.10.1961), \emph{Schriftsteller, Richter}!Sprueche12. 03. 1908@\strich\emph{Sprüche} {[}12. 03. 1908{]}|pwv}\newverse{}nichts, was dir ſelber innig
                            nicht entſtammt, gedichtet\pwindex{Adam, Robert 20.04.1877 – 16.10.1961@\textsc{Adam, Robert} (20.04.1877 – 16.10.1961), \emph{Schriftsteller, Richter}!Sprueche12. 03. 1908@\strich\emph{Sprüche} {[}12. 03. 1908{]}|pwv}\newverse{}(Schließlich kannſt du aber
                            auch der Welt\pwindex{Adam, Robert 20.04.1877 – 16.10.1961@\textsc{Adam, Robert} (20.04.1877 – 16.10.1961), \emph{Schriftsteller, Richter}!Sprueche12. 03. 1908@\strich\emph{Sprüche} {[}12. 03. 1908{]}|pwv}\newverse{}von Zeit zu Zeit was
                            hinſchmeißen, was ihr gefällt).\pwindex{Adam, Robert 20.04.1877 – 16.10.1961@\textsc{Adam, Robert} (20.04.1877 – 16.10.1961), \emph{Schriftsteller, Richter}!Sprueche12. 03. 1908@\strich\emph{Sprüche} {[}12. 03. 1908{]}|pwv}\stanzaend{}\pstart
           Aber ich geſtehe ein, daß mir jetzt, da mir etwas urſprünglich »Hingeſchmiſſenes«
                    ſelbſt den guten richtigen Geſchmack verderben und meine – nicht immer
                    verſagende – Fähigkeit der Selbſtkritik geſchmälert hat, die Gefährlichkeit
                    dieſer zweiten Wahlſpruchhälfte ſehr klar geworden iſt. –\pend
           \pstart
           Möge dieſe reumütige Beichte Ihnen genügen, hochverehrter Herr Doktor! –\pend
           \pstart
           Ich habe mich nun wieder in meine »Rechtsphiloſophie\pwindex{Adam, Robert 20.04.1877 – 16.10.1961@\textsc{Adam, Robert} (20.04.1877 – 16.10.1961), \emph{Schriftsteller, Richter}!RechtsphilosophieNone@\strich\emph{Rechtsphilosophie} {[}None{]}|pw}« eingeſponnen, deren erſter Teil – es wird ein Buch
                    von über 200 Seiten werden – endlich der Fertigſtellung entgegengeht. Bin ich
                    erſt dieſe Laſt halbwegs los, dann will ich mich an die Ausführung eines
                    Komödienplanes machen, und ich hoffe, daß ich damit ſeinerzeit die von der »Geſellſchaft\pwindex{Adam, Robert 20.04.1877 – 16.10.1961@\textsc{Adam, Robert} (20.04.1877 – 16.10.1961), \emph{Schriftsteller, Richter}!Gesellschaft [Eine Gaunerkomoedie]None@\strich\emph{Gesellschaft [Eine Gaunerkomödie]} {[}None{]}|pw}« geſchlagene Scharte auswetzen
                    kann.\pend
           \pstart
           Mit den ergebenſten Grüßen Ihr\pend
           \pstart
           dankbarer{\\[\baselineskip]}\spacefill\mbox{Robert Adam}\pend
           \leftskip=0em{}
         
         \endnumbering\mylabel{h}\end{ledgroupsized}  \newcommand{\dateiname}{L02215}\newcommand{\titel}{Robert Adam an Arthur Schnitzler, 16. 7. 1915}\newcommand{\editorInnen}{Martin Anton Müller und Gerd-Hermann Susen}%% latex-leseansicht-abspann.tex
%% Abspann für die Leseansicht.
%% Der Schalter \ifkorrekturansicht ist bereits durch den Vorspann gesetzt.

%% latex-abspann.tex
%% Gemeinsamer Abspann für Korrekturansicht und Leseansicht.
%% Setzt den Schalter \ifkorrekturansicht voraus (gesetzt in den
%% einbindenden Dateien latex-korrekturansicht-abspann.tex bzw.
%% latex-leseansicht-abspann.tex).
%% ---------------------------------------------------------------

\normalsize

% Das esempio-Environment wird nur in der Leseansicht benötigt
\ifkorrekturansicht\else
\newenvironment{esempio}[3]%
{
    \vspace{1.5ex}
    \rlap{\underline{#1}}
    \par
    \setlength{\parindent}{0cm}
    \nopagebreak
    \leftskip=#2cm
    \rightskip=#3cm
}
{
    \par
}
\fi

\doendnotes{C}
\bigskip
\vfill

\clearpage

\footnotesize

\ifkorrekturansicht
  \lohead{\textsc{register}}
\fi

% theindex-Environment neu definieren ohne reledmac
\makeatletter
\renewenvironment{theindex}{%
  \ifkorrekturansicht
    \section*{\indexname}%
  \else
    \subsubsection*{Index der erwähnten Entitäten}%
  \fi
  \setlength{\parindent}{0pt}%
  \setlength{\parskip}{0pt plus 0.3pt}%
  \let\item\@idxitem
}{%
  \ifkorrekturansicht\clearpage\fi
}
\makeatother

\IfFileExists{\jobname-pw.ind}{\input{\jobname-pw.ind}}{}

% Quellenangabe nur in der Leseansicht
\ifkorrekturansicht\else
% Fallback-Definitionen, falls die .tex-Datei \titel etc. nicht gesetzt hat
\providecommand{\titel}{}
\providecommand{\editorInnen}{}
\providecommand{\dateiname}{\jobname}

\vspace{3cm}

\vfill

\footnotesize
\textsc{Quelle}: \titel. Herausgegeben von {\editorInnen}. In: \emph{Arthur Schnitzler: Briefwechsel mit Autorinnen und Autoren}.
 Digitale Edition, https://schnitzler-briefe.acdh.oeaw.ac.at/{\dateiname}.html (Stand \today)
\fi

\end{document}


      