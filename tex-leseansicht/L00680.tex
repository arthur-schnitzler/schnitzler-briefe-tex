%% latex-leseansicht-vorspann.tex
%% Vorspann für die Leseansicht.
%% Lädt die gemeinsame Datei latex-vorspann.tex mit nicht gesetztem Schalter.

\newif\ifkorrekturansicht
\korrekturansichtfalse

\input{../tex-inputs/latex-vorspann}


\section[Richard Beer-Hofmann an Arthur Schnitzler, 20. 5. 1897]{L00680 Richard Beer-Hofmann an Arthur Schnitzler, 20. 5. 1897}
\nopagebreak\mylabel{L00680v}
\rehead{ }\normalsize\beginnumbering\briefempfaengerindex{Schnitzler, Arthur@\textsc{Schnitzler, Arthur}!zzzBeer-Hofmann, Richard@\emph{von Richard Beer-Hofmann}!1897-05-203@{20. 5. 1897}|(be}
\toendnotes[C]{\smallbreak\pagebreak[2]}
\correspDesc{Versand  durch Richard Beer-Hofmann am 20. 5. 1897 in Wien
\newline{}Erhalt  durch Arthur Schnitzler im Zeitraum [21. 5. 1897
                  – 25. 5. 1897?] in London}\toendnotes[C]{\smallbreak}
\Standort{CUL, Schnitzler, B 8.}
\physDesc{Brief, 3 Blätter, 9 Seiten, 1553 Zeichen
\newline{}Handschrift: blauer Buntstift, lateinische Kurrent
\newline{}Ordnung: mit Bleistift von unbekannter Hand nummeriert:
                                    »96« }
\buchAbdrucke{\weitereDrucke{Arthur Schnitzler, Richard Beer-Hofmann: \emph{Briefwechsel 1891–1931}. Herausgegeben von Konstanze Fliedl. Wien, Zürich: \emph{Europaverlag} 1992, S. 105–106.} }\toendnotes[C]{\smallbreak}
\pstart
           \raggedleft{}{\pb}20/V 97{ }Wien\oindex{Wien@\textbf{Wien}, \emph{Verwaltungsgebiet}|pw}\pend
           \vspace{0.5em}
\pstart
           Lieber Arthur, ich hab Ihren Brief vor einer Viertelstunde erhalten
               und antworte schon damit Sie bei Ihrer Ankunft in London\oindex{London@\textbf{London}, \emph{Hauptstadt}|pw} ihn vorfinden. Ich reise am 3. Juni Früh nach Ischl\oindex{Bad Ischl@\textbf{Bad Ischl}|pw}. Länger kann ich nicht hier bleiben. Ich bin
                  {\pb}recht verdrießlich: Mein
               Husten, kein Geld, Wohnung in Ordnung bringen – ich beko{\geminationm}e Wutanfälle wenn ich hausfrauliche Pflichten erfüllen soll. Ko{\geminationm}en Sie nicht {\pb}im Juni mit Ihrer Mama\pwindex{Schnitzler, Louise 8.\,7.\,1840 Kőszeg – 9.\,9.\,1911 Wien@\textsc{Schnitzler, Louise} (8.\,7.\,1840 Kőszeg – 9.\,9.\,1911 Wien)|pwv} nach Ischl\oindex{Bad Ischl@\textbf{Bad Ischl}|pw}? Wien\oindex{Wien@\textbf{Wien}, \emph{Verwaltungsgebiet}|pw} dürfte Ihnen ja
               unerträglich sein.\pend
           
\pstart
           Dem Paul\pwindex{Goldmann, Paul 31.\,1.\,1865 Breslau – 25.\,9.\,1935 Wien@\textsc{Goldmann, Paul} (31.\,1.\,1865 Breslau – 25.\,9.\,1935 Wien), \emph{Schriftsteller, Journalist}|pw} sagen Sie: »Ein \uline{guter} Mensch in seinem – – – –« und betonen Sie das »gut«. Er ha\substVorne{}\textsuperscript{tt}\substDazwischen{}t\substHinten{}{ }{\pb}tausendmal recht gehabt mit Allem
               was er von der Verlogenheit und Niedrigkeit dieses Packs sagte.\pend
           
\pstart
           Altenberg\pwindex{Altenberg, Peter 9.\,3.\,1859 Wien – 8.\,1.\,1919 ebd.@\textsc{Altenberg, Peter} (9.\,3.\,1859 Wien – 8.\,1.\,1919 ebd.), \emph{Schriftsteller}|pw} hat mir – ich bat ihn nicht darum –
                  {\pb}im Tiergarten\oindex{Wien@\textbf{Wien}!XIII., Hietzing@\textbf{XIII., Hietzing}!Tiergarten Schönbrunn@\textbf{Tiergarten Schönbrunn}, \emph{Zoo}|pw} durch einige Stunden Gesellschaft
                  geleistet{[}.{]} Von dem plumpem Comödiespielen dieses armseeligen
               Schmierencomödianten können Sie sich kaum einen Begriff machen. {\pb}Er lehnt verzückt an irgend einer
               Umfriedung und starrt auf irgend einen Schwarzen oder Schwarze und wartet daß ihn ein
               zufällig Vorübergehender (– er ist natürlich nur am Nachmittag in den Besuchsstunden
               dort wo er gesehen wird –) {\pb}aus
               seiner Verzückung reiße. Dabei ist er blind für den wirklichen Reiz dieser dunkeln
               Menschen\pend
           
\pstart
           Er kann \uline{nur} lügen.\pend
           
\pstart
           Von Bahr\pwindex{Bahr, Hermann 19.\,7.\,1863 Linz – 15.\,1.\,1934 München@\textsc{Bahr, Hermann} (19.\,7.\,1863 Linz – 15.\,1.\,1934 München), \emph{Schriftsteller, Kritiker}|pw} mag ich {\pb}nicht mehr reden. Er »sinkt« i{\geminationm}er tiefer würde ich sagen, wenn er jemals hoch
               gestanden wäre. –\pend
           
\pstart
           P.\pwindex{Beer-Hofmann, Paula 25.\,2.\,1879 Wien – 30.\,10.\,1939 Zürich@\textsc{Beer-Hofmann, Paula} (25.\,2.\,1879 Wien – 30.\,10.\,1939 Zürich)|pw} schreibt mir täglich und ist geduldig und
               brav. Da fällt {\pb}mir ein daß Sie ja
               – da ich nach London\oindex{London@\textbf{London}, \emph{Hauptstadt}|pw} adressire – Paul\pwindex{Goldmann, Paul 31.\,1.\,1865 Breslau – 25.\,9.\,1935 Wien@\textsc{Goldmann, Paul} (31.\,1.\,1865 Breslau – 25.\,9.\,1935 Wien), \emph{Schriftsteller, Journalist}|pw} nicht mehr sprechen; also schreiben Sie
               ihm viel Herzliches von mir, und seine neue Adresse möcht ich wissen. Bicycle? Noch
               nicht!\pend
           \pstart Ihr \spacefill\mbox{Richard}\pend{}\selectlanguage{ngerman}\endnumbering\briefempfaengerindex{Schnitzler, Arthur@\textsc{Schnitzler, Arthur}!zzzBeer-Hofmann, Richard@\emph{von Richard Beer-Hofmann}!1897-05-203@{20. 5. 1897}|)be}\mylabel{L00680h}  \newcommand{\dateiname}{L00680}\newcommand{\titel}{Richard Beer-Hofmann an Arthur Schnitzler, 20. 5. 1897}\newcommand{\editorInnen}{Herausgegeben von Martin Anton Müller}%% latex-leseansicht-abspann.tex
%% Abspann für die Leseansicht.
%% Der Schalter \ifkorrekturansicht ist bereits durch den Vorspann gesetzt.

%% latex-abspann.tex
%% Gemeinsamer Abspann für Korrekturansicht und Leseansicht.
%% Setzt den Schalter \ifkorrekturansicht voraus (gesetzt in den
%% einbindenden Dateien latex-korrekturansicht-abspann.tex bzw.
%% latex-leseansicht-abspann.tex).
%% ---------------------------------------------------------------

\normalsize

% Das esempio-Environment wird nur in der Leseansicht benötigt
\ifkorrekturansicht\else
\newenvironment{esempio}[3]%
{
    \vspace{1.5ex}
    \rlap{\underline{#1}}
    \par
    \setlength{\parindent}{0cm}
    \nopagebreak
    \leftskip=#2cm
    \rightskip=#3cm
}
{
    \par
}
\fi

\doendnotes{C}
\bigskip
\vfill

\clearpage

\footnotesize

\ifkorrekturansicht
  \lohead{\textsc{register}}
\fi

% theindex-Environment neu definieren ohne reledmac
\makeatletter
\renewenvironment{theindex}{%
  \ifkorrekturansicht
    \section*{\indexname}%
  \else
    \subsubsection*{Index der erwähnten Entitäten}%
  \fi
  \setlength{\parindent}{0pt}%
  \setlength{\parskip}{0pt plus 0.3pt}%
  \let\item\@idxitem
}{%
  \ifkorrekturansicht\clearpage\fi
}
\makeatother

\IfFileExists{\jobname-pw.ind}{\input{\jobname-pw.ind}}{}

% Quellenangabe nur in der Leseansicht
\ifkorrekturansicht\else
% Fallback-Definitionen, falls die .tex-Datei \titel etc. nicht gesetzt hat
\providecommand{\titel}{}
\providecommand{\editorInnen}{}
\providecommand{\dateiname}{\jobname}

\vspace{3cm}

\vfill

\footnotesize
\textsc{Quelle}: \titel. Herausgegeben von {\editorInnen}. In: \emph{Arthur Schnitzler: Briefwechsel mit Autorinnen und Autoren}.
 Digitale Edition, https://schnitzler-briefe.acdh.oeaw.ac.at/{\dateiname}.html (Stand \today)
\fi

\end{document}


