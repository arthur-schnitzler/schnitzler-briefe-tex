%% latex-leseansicht-vorspann.tex
%% Vorspann für die Leseansicht.
%% Lädt die gemeinsame Datei latex-vorspann.tex mit nicht gesetztem Schalter.

\newif\ifkorrekturansicht
\korrekturansichtfalse

\input{../tex-inputs/latex-vorspann}


         
         \renewcommand{\erwaehntePersonen}{Personen: Frieda Pollak, Felix Salten}
         \renewcommand{\erwaehnteOrte}{Orte: San Marco, Stazione di Venezia Santa Lucia, Sternwartestraße 71, Torre dell’Orologio, Venedig, Wien, Österreich}
         \renewcommand{\erwaehnteWerke}{}
               \section[ Felix Salten an Arthur Schnitzler, 29. 3. 1922]{ Felix Salten an Arthur Schnitzler, 29. 3. 1922}\nopagebreak\mylabel{v}\rehead{ }\begin{ledgroupsized}[t]{13cm}\normalsize\beginnumbering\briefempfaengerindex{Schnitzler, Arthur@\textsc{Schnitzler, Arthur}!zzzSalten, Felix@\emph{von Felix Salten}!1922-03-291@{29. 3. 1922}|(be} \toendnotes[C]{\smallbreak\pagebreak[2]} \Standort{CUL, Schnitzler, B 89, B 2.}
\physDesc{Bildpostkarte, 400 Zeichen
\newline{}Handschrift: schwarze Tinte, lateinische Kurrent
\newline{}Versand: 1) Stempel: »\nobreak{}Pregate i vostri corrispondenti di aggiungere
                                       all’indirizzo il numero del quartiere postale\nobreak{}«.   2) Stempel: »\nobreak{}\oindex{Stazione di Venezia Santa Lucia@\textbf{Stazione di Venezia Santa Lucia}|pwk}Venezia Ferrovia, 29. III 1922, 23–24\nobreak{}«. 
\newline{}Ordnung: 1) mit Bleistift von Frieda Pollak\pwindex{Pollak, Frieda 08.12.1881 – 13.07.1937@\textsc{Pollak, Frieda} (08.12.1881 – 13.07.1937), \emph{Sekretärin}|pw} (?) mit
                                 dem Buchstaben »A« (Abgeschrieben/Abschrift)
                                 gekennzeichnet  2) mit Bleistift von unbekannter Hand nummeriert: »289«}\toendnotes[C]{\smallbreak}\pstart{}{\pb}Aust\textcolor{gray}{ria}\oindex{Oesterreich@\textbf{Österreich}|pw}\pend{}\pstart{}Herrn D\textsuperscript{r} Arthur Schnitzler\pend{}\pstart{}XVIII. Sternwartestrasse 71\oindex{Sternwartestrasse 71@\textbf{Sternwartestraße 71}|pw}\pend{}\pstart{}Wien\oindex{Wien@\textbf{Wien}|pw}\pend{}{\bigskip}\pstart
           \noindent{}\centering{}{\pb}\textcolor{gray}{\textbf{VENEZIA\oindex{Venedig@\textbf{Venedig}|pw} – Chiesa S. Marco\oindex{San Marco@\textbf{San Marco}|pw} e Torre dell’ Orologio\oindex{Torre dell Orologio@\textbf{Torre dell’Orologio}|pw}}}\pend
           \pstart
           \noindent{}{\pb}Lieber, es ist schon sehr schön, wieder hier\oindex{Venedig@\textbf{Venedig}|pwv} zu sein. Bin heute vier Stunden spazieren gegangen. Die Leute sind so freundlich, als
               wären auch sie des Wiedersehens froh. Es sind fast gar keine Fremden da. Ich glaube,
               man kann hier mit 50–60 Lire im Tag gut auskommen. Das ist, an unseren Preisen
               gemessen, nicht viel.\pend
           \pstart Herzlichst Ihr \spacefill\mbox{Felix Salten}\pend{}
         
         \endnumbering\mylabel{h}\end{ledgroupsized}  \newcommand{\dateiname}{L03576}\newcommand{\titel}{Felix Salten an Arthur Schnitzler, 29. 3. 1922}\newcommand{\editorInnen}{Martin Anton Müller und Laura Untner}%% latex-leseansicht-abspann.tex
%% Abspann für die Leseansicht.
%% Der Schalter \ifkorrekturansicht ist bereits durch den Vorspann gesetzt.

%% latex-abspann.tex
%% Gemeinsamer Abspann für Korrekturansicht und Leseansicht.
%% Setzt den Schalter \ifkorrekturansicht voraus (gesetzt in den
%% einbindenden Dateien latex-korrekturansicht-abspann.tex bzw.
%% latex-leseansicht-abspann.tex).
%% ---------------------------------------------------------------

\normalsize

% Das esempio-Environment wird nur in der Leseansicht benötigt
\ifkorrekturansicht\else
\newenvironment{esempio}[3]%
{
    \vspace{1.5ex}
    \rlap{\underline{#1}}
    \par
    \setlength{\parindent}{0cm}
    \nopagebreak
    \leftskip=#2cm
    \rightskip=#3cm
}
{
    \par
}
\fi

\doendnotes{C}
\bigskip
\vfill

\clearpage

\footnotesize

\ifkorrekturansicht
  \lohead{\textsc{register}}
\fi

% theindex-Environment neu definieren ohne reledmac
\makeatletter
\renewenvironment{theindex}{%
  \ifkorrekturansicht
    \section*{\indexname}%
  \else
    \subsubsection*{Index der erwähnten Entitäten}%
  \fi
  \setlength{\parindent}{0pt}%
  \setlength{\parskip}{0pt plus 0.3pt}%
  \let\item\@idxitem
}{%
  \ifkorrekturansicht\clearpage\fi
}
\makeatother

\IfFileExists{\jobname-pw.ind}{\input{\jobname-pw.ind}}{}

% Quellenangabe nur in der Leseansicht
\ifkorrekturansicht\else
% Fallback-Definitionen, falls die .tex-Datei \titel etc. nicht gesetzt hat
\providecommand{\titel}{}
\providecommand{\editorInnen}{}
\providecommand{\dateiname}{\jobname}

\vspace{3cm}

\vfill

\footnotesize
\textsc{Quelle}: \titel. Herausgegeben von {\editorInnen}. In: \emph{Arthur Schnitzler: Briefwechsel mit Autorinnen und Autoren}.
 Digitale Edition, https://schnitzler-briefe.acdh.oeaw.ac.at/{\dateiname}.html (Stand \today)
\fi

\end{document}


      