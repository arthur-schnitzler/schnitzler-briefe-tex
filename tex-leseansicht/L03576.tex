%% latex-korrekturansicht-vorspann.tex
%% Vorspann für die Korrekturansicht.
%% Lädt die gemeinsame Datei latex-vorspann.tex mit gesetztem Schalter.

\newif\ifkorrekturansicht
\korrekturansichttrue

\input{../tex-inputs/latex-vorspann}


\section[ Felix Salten an Arthur Schnitzler, 29. 3. 1922]{L03576 Felix Salten an Arthur Schnitzler, 29. 3. 1922}
\nopagebreak\mylabel{L03576v}
\rehead{ }\normalsize\beginnumbering\briefempfaengerindex{Schnitzler, Arthur@\textsc{Schnitzler, Arthur}!zzzSalten, Felix@\emph{von Felix Salten}!1922-03-291@{29. 3. 1922}|(be}
\toendnotes[C]{\smallbreak\pagebreak[2]}\Standort{CUL, Schnitzler, B 89, B 2.}
\physDesc{Bildpostkarte, 400 Zeichen
\newline{}Handschrift: schwarze Tinte, lateinische Kurrent
\newline{}Versand: 1) Stempel: »\nobreak{}Pregate i vostri corrispondenti di aggiungere
                                       all’indirizzo il numero del quartiere postale\nobreak{}«.   2) Stempel: »\nobreak{}\oindex{Stazione di Venezia Santa Lucia@\textbf{Stazione di Venezia Santa Lucia}, \emph{Bahnhofsgebäude (K.BHF)}|pwk}Venezia Ferrovia, 29. III 1922, 23–24\nobreak{}«. 
\newline{}Ordnung: 1) mit Bleistift von Frieda Pollak\pwindex{Pollak, Frieda 08.12.1881 – 13.07.1937@\textsc{Pollak, Frieda} (08.12.1881 – 13.07.1937), \emph{Sekretär/Sekretärin}|pw} (?) mit
                                 dem Buchstaben »A« (Abgeschrieben/Abschrift)
                                 gekennzeichnet  2) mit Bleistift von unbekannter Hand nummeriert: »289«}\toendnotes[C]{\smallbreak}\pstart{}{\pb}Aust\textcolor{gray}{ria}\oindex{Oesterreich@\textbf{Österreich}, \emph{A.PCLI}|pw}\pend{}\pstart{}Herrn D\textsuperscript{r} Arthur Schnitzler\pend{}\pstart{}XVIII. Sternwartestrasse 71\oindex{Sternwartestrasse 71@\textbf{Sternwartestraße 71}, \emph{Wohngebäude (K.WHS)}|pw}\pend{}\pstart{}Wien\oindex{Wien@\textbf{Wien}, \emph{A.ADM2}|pw}\pend{}{\bigskip}
\pstart
           \noindent{}\centering{}{\pb}\textcolor{gray}{\textbf{VENEZIA\oindex{Venedig@\textbf{Venedig}, \emph{P.PPLA}|pw} – Chiesa S. Marco\oindex{Piazza San Marco@\textbf{Piazza San Marco}, \emph{S.SQR}|pw} e Torre dell’ Orologio\oindex{Torre dell Orologio@\textbf{Torre dell’Orologio}, \emph{S.BLDG}|pw}}}\pend
           \vspace{1em}
\pstart
           \noindent{}{\pb}Lieber, es ist schon sehr schön, wieder hier\oindex{Venedig@\textbf{Venedig}, \emph{P.PPLA}|pwv} zu sein. Bin heute vier Stunden spazieren gegangen. Die Leute sind so freundlich, als
               wären auch sie des Wiedersehens froh. Es sind fast gar keine Fremden da. Ich glaube,
               man kann hier mit 50–60 Lire im Tag gut auskommen. Das ist, an unseren Preisen
               gemessen, nicht viel.\pend
           \pstart Herzlichst Ihr \spacefill\mbox{Felix Salten}\pend{}\selectlanguage{ngerman}\endnumbering\briefempfaengerindex{Schnitzler, Arthur@\textsc{Schnitzler, Arthur}!zzzSalten, Felix@\emph{von Felix Salten}!1922-03-291@{29. 3. 1922}|)be}\mylabel{L03576h}  \normalsize

\doendnotes{C}
\bigskip
\vfill

\clearpage

\footnotesize

\lohead{\textsc{register}}

% Definiere theindex-Environment komplett neu ohne reledmac
\makeatletter
\renewenvironment{theindex}{%
  \section*{\indexname}%
  \setlength{\parindent}{0pt}%
  \setlength{\parskip}{0pt plus 0.3pt}%
  \let\item\@idxitem
}{%
  \clearpage
}
\makeatother

\IfFileExists{\jobname-pw.ind}{\input{\jobname-pw.ind}}{}

\end{document}

      