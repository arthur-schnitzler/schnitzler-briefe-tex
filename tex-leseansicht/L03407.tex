%% latex-korrekturansicht-vorspann.tex
%% Vorspann für die Korrekturansicht.
%% Lädt die gemeinsame Datei latex-vorspann.tex mit gesetztem Schalter.

\newif\ifkorrekturansicht
\korrekturansichttrue

\input{../tex-inputs/latex-vorspann}


\section[ Felix Salten an Arthur Schnitzler, 11. 4. 1905]{L03407 Felix Salten an Arthur Schnitzler, 11. 4. 1905}
\nopagebreak\mylabel{L03407v}
\rehead{ }\normalsize\beginnumbering\briefempfaengerindex{Schnitzler, Arthur@\textsc{Schnitzler, Arthur}!zzzSalten, Felix@\emph{von Felix Salten}!1905-04-112@{11. 4. 1905}|(be}
\toendnotes[C]{\smallbreak\pagebreak[2]}\Standort{CUL, Schnitzler, B 89, B 1.}
\physDesc{Briefkarte, 903 Zeichen
\newline{}Handschrift: schwarze Tinte, lateinische Kurrent
\newline{}Ordnung: mit Bleistift von unbekannter Hand nummeriert: »199« }\toendnotes[C]{\smallbreak}
\pstart
           {\pb}\textcolor{gray}{\textbf{DIE}}\pend
           
\pstart
           \textcolor{gray}{\textbf{ZEIT\orgindex{Zeit@Die Zeit|pw}}}\hfill \textcolor{gray}{\textbf{\emph{WIEN\oindex{Wien@\textbf{Wien}, \emph{A.ADM2}|pw}}}}{ }11. IV. 05\pend
           
\pstart
           \textcolor{gray}{\textbf{Wien\oindex{Wien@\textbf{Wien}, \emph{A.ADM2}|pw}er Tageszeitung}}\hfill \textcolor{gray}{\textbf{\emph{I. Wipplingerstrasse 38\oindex{Wipplingerstrasse@\textbf{Wipplingerstraße}, \emph{Straße (K.STR)}|pw}}}}\pend
           
\pstart
           \textcolor{gray}{\textbf{Herausgeber:}}\pend
           
\pstart
           \textcolor{gray}{\textbf{\textbf{Prof. Dr. I. Singer\pwindex{Singer, Isidor 16.01.1857 – 08.12.1927@\textsc{Singer, Isidor} (16.01.1857 – 08.12.1927), \emph{Journalist/Journalistin, Herausgeber/Herausgeberin, Soziologe/Soziologin}|pw}}}}\pend
           
\pstart
           \textcolor{gray}{\textbf{\textbf{Dr. Heinrich Kanner\pwindex{Kanner, Heinrich 09.11.1864 – 15.02.1930@\textsc{Kanner, Heinrich} (09.11.1864 – 15.02.1930), \emph{Herausgeber/Herausgeberin, Publizist/Publizistin}|pw}}}}\pend
           
\pstart
           \textcolor{gray}{\textbf{\textbf{Feuilleton-Redaktion}}}\pend
           \vspace{0.5em}
\pstart
           Lieber, vielen Dank für den Beitrag\pwindex{Schiller-Feier@\emph{Schiller-Feier}|pwv} zur Schiller-Nu{\geminationm}er\pwindex{Schiller-Zeit 1805 * 1905@\emph{Schiller-Zeit 1805 * 1905}|pwv}. \label{K_L03407-1v}\edtext{Den großen
                  Wurstel\pwindex{Zum grossen Wurstel. Burleske in einem Akt@\emph{Zum großen Wurstel. Burleske in einem Akt}|pw}}{\lemma{\textnormal{\emph{Den großen
                  Wurstel}}}\Cendnote{\textnormal{Siehe Arthur Schnitzler an Felix Salten, 8. 2. 1905.
               }}}\label{K_L03407-1} wollen wir noch für die Osternummer\pwindex{Schiller-Zeit 1805 * 1905@\emph{Schiller-Zeit 1805 * 1905}|pwv} bringen, und schlage ich Ihnen Frl. Berta Czegka\pwindex{Czegka, Berta 30.07.1880 – 04.11.1954@\textsc{Czegka, Berta} (30.07.1880 – 04.11.1954), \emph{Maler/Malerin}|pw} als Zeichnerin vor, die ich für sehr begabt
               halte. Ich hatte sie schon in der vergangenen Woche zu mir bitten wollen, konnte aber
               mit Niemandem ordentlich sprechen, und war nur immer sehr flüchtig in der Redaction\oindex{Wipplingerstrasse@\textbf{Wipplingerstraße}, \emph{Straße (K.STR)}|pwv}. Nun kommt sie wegen
               des großen Wurstel\pwindex{Zum grossen Wurstel. Burleske in einem Akt@\emph{Zum großen Wurstel. Burleske in einem Akt}|pw}{ }morgen gegen 2 – od. ½ 3 zu mir, und ich will
               sie bitten, \label{K_L03407-2v}\edtext{am Donnerstag um 4\textsuperscript{h}–5\textsuperscript{h.} bei Ihnen}{\lemma{\textnormal{\emph{am … Ihnen}}}\Cendnote{\textnormal{Dazu kam es am
                  Dienstag, dem 13. 4. 1905.}}}\label{K_L03407-2} zu sein. Sie arbeitet sehr flink; aber man muß
               ihr alles genau erklären. Wie Sie mir \label{K_L03407-3v}\edtext{s. Z.}{\lemma{\textnormal{\emph{s. Z.}}}\Cendnote{\textnormal{seiner Zeit}}}\label{K_L03407-3}{ }\label{K_L03407-4v}\edtext{schrieben}{\lemma{\textnormal{\emph{schrieben}}}\Cendnote{\textnormal{Siehe Arthur Schnitzler an Felix Salten, 8. 2. 1905.
               }}}\label{K_L03407-4}, verlangen Sie 600 Kronen für den Abdruck; und wird das Honorar am 1. Mai an Sie gesendet. Selbstverständlich erhalten Sie
               von \uline{beiden}{ }{\pb}Manuscripten\pwindex{Schiller-Feier@\emph{Schiller-Feier}|pwv}\pwindex{Zum grossen Wurstel. Burleske in einem Akt@\emph{Zum großen Wurstel. Burleske in einem Akt}|pwv} Autoren
               Correctur.\pend
           
\pstart
           Ich habe sehr bedauert, Sie Beide\pwindex{Schnitzler, Olga 17.01.1882 – 13.01.1970@\textsc{Schnitzler, Olga} (17.01.1882 – 13.01.1970), \emph{Schauspieler/Schauspielerin, Sänger/Sängerin}|pwv}{ }\label{K_L03407-5v}\edtext{neulich}{\lemma{\textnormal{\emph{neulich}}}\Cendnote{\textnormal{Siehe A. S.: \emph{Tagebuch}, 7. 4. 1905.
               }}}\label{K_L03407-5} verfehlt zu haben u. danke Ihnen noch nachträglich für Ihren Besuch.
               Hoffentlich auf bald.\pend
           
\pstart
           Herzlichst {\\[\baselineskip]}Ihr {\\[\baselineskip]}\spacefill\mbox{Salten}\pend
           \leftskip=0em{}\selectlanguage{ngerman}\endnumbering\briefempfaengerindex{Schnitzler, Arthur@\textsc{Schnitzler, Arthur}!zzzSalten, Felix@\emph{von Felix Salten}!1905-04-112@{11. 4. 1905}|)be}\mylabel{L03407h}  \normalsize

\doendnotes{C}
\bigskip
\vfill

\clearpage

\footnotesize

\lohead{\textsc{register}}

% Definiere theindex-Environment komplett neu ohne reledmac
\makeatletter
\renewenvironment{theindex}{%
  \section*{\indexname}%
  \setlength{\parindent}{0pt}%
  \setlength{\parskip}{0pt plus 0.3pt}%
  \let\item\@idxitem
}{%
  \clearpage
}
\makeatother

\IfFileExists{\jobname-pw.ind}{\input{\jobname-pw.ind}}{}

\end{document}

      