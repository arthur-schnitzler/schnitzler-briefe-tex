%% latex-leseansicht-vorspann.tex
%% Vorspann für die Leseansicht.
%% Lädt die gemeinsame Datei latex-vorspann.tex mit nicht gesetztem Schalter.

\newif\ifkorrekturansicht
\korrekturansichtfalse

\input{../tex-inputs/latex-vorspann}

\begin{center}
            \textcolor{red}{ENTWURF, NICHT FERTIG KORRIGIERT}
                      \end{center}
            
         
         \renewcommand{\erwaehntePersonen}{Personen: Berta Czegka, Heinrich Kanner, Felix Salten, Olga Schnitzler, Isidor Singer}
         \renewcommand{\erwaehnteInstitutionen}{Institutionen: Die Zeit}
         \renewcommand{\erwaehnteOrte}{Orte: Wien, Wipplingerstraße}
         \renewcommand{\erwaehnteWerke}{Werke: Schiller-Feier, Schiller-Zeit 1805 * 1905, Zum großen Wurstel. Burleske in einem Akt}
               \section[ Felix Salten an Arthur Schnitzler, 11. 4. 1905]{ Felix Salten an Arthur Schnitzler, 11. 4. 1905}\nopagebreak\mylabel{v}\rehead{ }\begin{ledgroupsized}[t]{13cm}\normalsize\beginnumbering \toendnotes[C]{\smallbreak\pagebreak[2]} \Standort{CUL, Schnitzler, B 89, B 1.}
\physDesc{Briefkarte, 903 Zeichen
\newline{}Handschrift: schwarze Tinte, lateinische Kurrent
\newline{}Ordnung: mit Bleistift von unbekannter Hand nummeriert: »199« }\toendnotes[C]{\smallbreak}\pstart
           \noindent{}{\pb}\textcolor{gray}{\textbf{DIE}}\pend
           \pstart
           \textcolor{gray}{\textbf{ZEIT\orgindex{Zeit@Die Zeit|pw}}}\hfill \textcolor{gray}{\textbf{\emph{WIEN\oindex{Wien@\textbf{Wien}|pw}}}}{ }11. IV. 05\pend
           \pstart
           \textcolor{gray}{\textbf{Wien\oindex{Wien@\textbf{Wien}|pw}er Tageszeitung}}\hfill \textcolor{gray}{\textbf{\emph{I. Wipplingerstrasse 38\oindex{Wipplingerstrasse@\textbf{Wipplingerstraße}|pw}}}}\pend
           \pstart
           \textcolor{gray}{\textbf{Herausgeber:}}\pend
           \pstart
           \textcolor{gray}{\textbf{\textbf{Prof. Dr. I. Singer\pwindex{Singer, Isidor 16.01.1857 – 08.12.1927@\textsc{Singer, Isidor} (16.01.1857 – 08.12.1927), \emph{Journalist, Herausgeber, Soziologe}|pw}}}}\pend
           \pstart
           \textcolor{gray}{\textbf{\textbf{Dr. Heinrich Kanner\pwindex{Kanner, Heinrich 09.11.1864 – 15.02.1930@\textsc{Kanner, Heinrich} (09.11.1864 – 15.02.1930), \emph{Herausgeber, Publizist}|pw}}}}\pend
           \pstart
           \textcolor{gray}{\textbf{\textbf{Feuilleton-Redaktion}}}\pend
           \pstart
           Lieber, vielen Dank für den Beitrag\pwindex{Schnitzler, Arthur 15.05.1862 – 21.10.1931@\textsc{Schnitzler, Arthur} (15.05.1862 – 21.10.1931), \emph{Schriftsteller, Mediziner}!Schiller-Feier23. 4. 1905@\strich\emph{Schiller-Feier} {[}23. 4. 1905{]}|pwv} zur Schiller-Nu{\geminationm}er\pwindex{Schiller-Zeit 1805 * 19051905-04-23@\emph{Schiller-Zeit 1805 * 1905} {[}1905-04-23{]}|pwv}. \label{K_L03407-1v}\edtext{Den großen
                  Wurstel\pwindex{Schnitzler, Arthur 15.05.1862 – 21.10.1931@\textsc{Schnitzler, Arthur} (15.05.1862 – 21.10.1931), \emph{Schriftsteller, Mediziner}!Zum grossen Wurstel. Burleske in einem Akt08. 03. 1901@\strich\emph{Zum großen Wurstel. Burleske in einem Akt} {[}08. 03. 1901{]}|pw}}{\lemma{\textnormal{\emph{Den großen
                  Wurstel}}}\Cendnote{\textnormal{siehe Arthur Schnitzler an Felix Salten, 8. 2. 1905}}}\label{K_L03407-1h} wollen wir noch für die Osternummer\pwindex{Schiller-Zeit 1805 * 19051905-04-23@\emph{Schiller-Zeit 1805 * 1905} {[}1905-04-23{]}|pwv} bringen, und schlage ich Ihnen Frl. Berta Czegka\pwindex{Czegka, Berta 30.07.1880 – 04.11.1954@\textsc{Czegka, Berta} (30.07.1880 – 04.11.1954), \emph{Malerin}|pw} als Zeichnerin vor, die ich für sehr begabt
               halte. Ich hatte sie schon in der vergangenen Woche zu mir bitten wollen, konnte aber
               mit Niemandem ordentlich sprechen, und war nur immer sehr flüchtig in der Redaction\oindex{Wipplingerstrasse@\textbf{Wipplingerstraße}|pwv}. Nun kommt sie wegen
               des großen Wurstel\pwindex{Schnitzler, Arthur 15.05.1862 – 21.10.1931@\textsc{Schnitzler, Arthur} (15.05.1862 – 21.10.1931), \emph{Schriftsteller, Mediziner}!Zum grossen Wurstel. Burleske in einem Akt08. 03. 1901@\strich\emph{Zum großen Wurstel. Burleske in einem Akt} {[}08. 03. 1901{]}|pw}{ }morgen gegen 2 – od. ½ 3 zu mir, und ich will
               sie bitten, \label{K_L03407-2v}\edtext{am Donnerstag um 4\textsuperscript{h}–5\textsuperscript{h.} bei Ihnen}{\lemma{\textnormal{\emph{am … Ihnen}}}\Cendnote{\textnormal{Dazu kam es am
                  Dienstag, dem 13. 4. 1905.}}}\label{K_L03407-2h} zu sein. Sie arbeitet sehr flink; aber man muß
               ihr alles genau erklären. Wie Sie mir \label{K_L03407-3v}\edtext{s. Z.}{\lemma{\textnormal{\emph{s. Z.}}}\Cendnote{\textnormal{seiner Zeit}}}\label{K_L03407-3h}{ }\label{K_L03407-4v}\edtext{schrieben}{\lemma{\textnormal{\emph{schrieben}}}\Cendnote{\textnormal{siehe Arthur Schnitzler an Felix Salten, 8. 2. 1905}}}\label{K_L03407-4h}, verlangen Sie 600 Kronen für den Abdruck; und wird das Honorar am 1. Mai an Sie gesendet. Selbstverständlich erhalten Sie
               von \uline{beiden}{ }{\pb}Manuscripten\pwindex{Schnitzler, Arthur 15.05.1862 – 21.10.1931@\textsc{Schnitzler, Arthur} (15.05.1862 – 21.10.1931), \emph{Schriftsteller, Mediziner}!Schiller-Feier23. 4. 1905@\strich\emph{Schiller-Feier} {[}23. 4. 1905{]}|pwv}\pwindex{Schnitzler, Arthur 15.05.1862 – 21.10.1931@\textsc{Schnitzler, Arthur} (15.05.1862 – 21.10.1931), \emph{Schriftsteller, Mediziner}!Zum grossen Wurstel. Burleske in einem Akt08. 03. 1901@\strich\emph{Zum großen Wurstel. Burleske in einem Akt} {[}08. 03. 1901{]}|pwv} Autoren
               Correctur.\pend
           \pstart
           Ich habe sehr bedauert, Sie Beide\pwindex{Schnitzler, Olga 17.01.1882 – 13.01.1970@\textsc{Schnitzler, Olga} (17.01.1882 – 13.01.1970), \emph{Schauspielerin, Sängerin}|pwv}{ }\label{K_L03407-5v}\edtext{neulich}{\lemma{\textnormal{\emph{neulich}}}\Cendnote{\textnormal{siehe A. S.: \emph{Tagebuch}, 7. 4. 1905}}}\label{K_L03407-5h} verfehlt zu haben u. danke Ihnen noch nachträglich für Ihren Besuch.
               Hoffentlich auf bald.\pend
           \pstart
           Herzlichst {\\[\baselineskip]}Ihr {\\[\baselineskip]}\spacefill\mbox{Salten}\pend
           \leftskip=0em{}
         
         \endnumbering\mylabel{h}\end{ledgroupsized}  \newcommand{\dateiname}{L03407}\newcommand{\titel}{Felix Salten an Arthur Schnitzler, 11. 4. 1905}\newcommand{\editorInnen}{Martin Anton Müller und Laura Untner}%% latex-leseansicht-abspann.tex
%% Abspann für die Leseansicht.
%% Der Schalter \ifkorrekturansicht ist bereits durch den Vorspann gesetzt.

%% latex-abspann.tex
%% Gemeinsamer Abspann für Korrekturansicht und Leseansicht.
%% Setzt den Schalter \ifkorrekturansicht voraus (gesetzt in den
%% einbindenden Dateien latex-korrekturansicht-abspann.tex bzw.
%% latex-leseansicht-abspann.tex).
%% ---------------------------------------------------------------

\normalsize

% Das esempio-Environment wird nur in der Leseansicht benötigt
\ifkorrekturansicht\else
\newenvironment{esempio}[3]%
{
    \vspace{1.5ex}
    \rlap{\underline{#1}}
    \par
    \setlength{\parindent}{0cm}
    \nopagebreak
    \leftskip=#2cm
    \rightskip=#3cm
}
{
    \par
}
\fi

\doendnotes{C}
\bigskip
\vfill

\clearpage

\footnotesize

\ifkorrekturansicht
  \lohead{\textsc{register}}
\fi

% theindex-Environment neu definieren ohne reledmac
\makeatletter
\renewenvironment{theindex}{%
  \ifkorrekturansicht
    \section*{\indexname}%
  \else
    \subsubsection*{Index der erwähnten Entitäten}%
  \fi
  \setlength{\parindent}{0pt}%
  \setlength{\parskip}{0pt plus 0.3pt}%
  \let\item\@idxitem
}{%
  \ifkorrekturansicht\clearpage\fi
}
\makeatother

\IfFileExists{\jobname-pw.ind}{\input{\jobname-pw.ind}}{}

% Quellenangabe nur in der Leseansicht
\ifkorrekturansicht\else
% Fallback-Definitionen, falls die .tex-Datei \titel etc. nicht gesetzt hat
\providecommand{\titel}{}
\providecommand{\editorInnen}{}
\providecommand{\dateiname}{\jobname}

\vspace{3cm}

\vfill

\footnotesize
\textsc{Quelle}: \titel. Herausgegeben von {\editorInnen}. In: \emph{Arthur Schnitzler: Briefwechsel mit Autorinnen und Autoren}.
 Digitale Edition, https://schnitzler-briefe.acdh.oeaw.ac.at/{\dateiname}.html (Stand \today)
\fi

\end{document}


      