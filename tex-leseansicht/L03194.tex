%% latex-leseansicht-vorspann.tex
%% Vorspann für die Leseansicht.
%% Lädt die gemeinsame Datei latex-vorspann.tex mit nicht gesetztem Schalter.

\newif\ifkorrekturansicht
\korrekturansichtfalse

\input{../tex-inputs/latex-vorspann}


\section[ Paul Goldmann an Arthur Schnitzler, 20. 1. [1902]]{L03194 Paul Goldmann an Arthur Schnitzler,  20. 1. [1902]}
\nopagebreak\mylabel{L03194v}
\rehead{ }\normalsize\beginnumbering\briefempfaengerindex{Schnitzler, Arthur@\textsc{Schnitzler, Arthur}!zzzGoldmann, Paul@\emph{von Paul Goldmann}!1902-01-202@{20. 1. [1902]}|(be}
\toendnotes[C]{\smallbreak\pagebreak[2]}
\correspDesc{Versand  durch Paul Goldmann am 20. 1. [1902] in Berlin
\newline{}Erhalt  durch Arthur Schnitzler im Zeitraum [21. 1. 1902
                  – 25. 1. 1902?] in Wien}\toendnotes[C]{\smallbreak}
\Standort{DLA, A:Schnitzler, HS.NZ85.1.3172.}
\physDesc{Brief, 1 Blatt, 4 Seiten, 3657 Zeichen
\newline{}Handschrift: blaue Tinte, deutsche Kurrent
\newline{}Schnitzler: mit Bleistift das Jahr »902« vermerkt }\toendnotes[C]{\smallbreak}
\pstart
           {\pb}Berlin\oindex{Berlin@\textbf{Berlin}, \emph{Hauptstadt}|pw}, 20. Januar.\hfill \textcolor{gray}{\textbf{DESSAUERSTRASSE 19}}\oindex{Dessauer Straße@\textbf{Dessauer Straße}, \emph{Straße}|pw}\pend
           \vspace{0.5em}
\pstart
           Mein lieber Freund, daß Du Dir keiner \label{K_L03194-1v}\edtext{Schuld}{\lemma{\textnormal{\emph{Schuld}}}\Cendnote{\textnormal{Siehe XXXX Auszeichnungsfehler: Dokument L03193 nicht gefunden.
               }}}\label{K_L03194-1} bewußt biſt, iſt zweifellos, – ebenſo, daß Du mir \substVorne{}\textsuperscript{nie}\substDazwischen{}nie\substHinten{} mit Abſicht wehgethan haſt. Dazu biſt Du viel zu gut und mir viel zu gut.
                  \strikeout{Deine Schuld liegt d\textcolor{gray}{ar}} Trotzdem haſt Du eine Schuld, und{ }ſie liegt darin (Dir unbewußt, mir{ }ſeit
               Jahren bewußt und recht{ }ſchmerzlich bewußt), daß in unſerer Freundſchaft Du mir
               längſt nicht mehr das Gleiche wiedergibſt, \substVorne{}\textsuperscript{was}\substDazwischen{}das\substHinten{} ich Dir gebe, – daß Du \strikeout{es Dir fe} es Dir, von
               Dir erfüllt,{ }ſeit Langem abgewöhnt haſt, \strikeout{\textcolor{gray}{r}} gründlich auch auf mich einzugehen. Ich lebe mit Dir viel mehr, als Du mit mir
               lebſt. Und ich habe{ }ſeit Langem den Eindruck, daß ich (ich muß das Wort wieder
               gebrauchen, \strikeout{ob\textcolor{gray}{w}} obwohl es \substVorne{}\textsuperscript{\textcolor{gray}{An}}\substDazwischen{}Dir\substHinten{} mißfällt) nicht viel mehr bin, als eine Bequemlichkeit in Deinem Leben. Die
               Beweiſe? So etwas kann man nur fühlen, aber nicht beweiſen. Aber wenn Du Beweiſe
               willſt,{ }ſo denke an unſeren Briefwechſel, all’ die Jahre hindurch. Denke {\pb}daran, wie viel von Dir darin{ }ſteht und wie wenig
               von mir. Oh, es hat an Anfragen nach meinen Erlebniſſen von Deiner Seite nicht
               gefehlt. Aber Du haſt Dich{ }ſtets leicht dabei beruhigt, wenn ich mich, wie es zumeiſt
               geſchah, nicht habe entſchließen können,{ }ſie zu beantworten. Nun weiß ich ja, daß in
               keinem menſchlichen Verhältniß, in der Liebe ebenſowenig wie in der Freundſchaft,
               Gleiches für Gleiches gegeben wird. Und ich verlange auch nicht mehr, da es in Deiner
               Natur liegt,{ }ſo zu{ }ſein, da ich Dich{ }ſehr lieb habe und da es mir eben darum Freude
               macht, an Deinem Leben theilzunehmen, wenn Du Dich auch an dem meinen{ }ſo wenig
               betheiligſt. Aber da Du Dir in Deinem letzten Brief keinen Zwang auferlegt und der
               Verſtimmung, in die ein Brief von mir Dich verſetzt, rückhaltslos Ausdruck gegeben
               haſt,{ }ſo{ }ſehe ich nicht ein, warum ich nicht auch einmal Dir{ }ſagen{ }ſoll, wie bitter
               und{ }ſchmerzlich \strikeout{\textcolor{gray}{h}} ich in den letzten Jahren oft \strikeout{das} empfunden
               habe, daß {\pb}ich bei Dir die Stärkung und Aufrichtung,
               die ich von Deiner Freundſchaft erwartet hatte, nicht habe finden können und daß ich
               vom \label{K_L03194-2v}\edtext{Beiſammenſein}{\lemma{\textnormal{\emph{Beisammensein}}}\Cendnote{\textnormal{Zuletzt hatten sie sich in Wien\oindex{Wien@\textbf{Wien}, \emph{Verwaltungsgebiet}|pwk}{ }Ende August / Anfang September 1901 und in Berlin\oindex{Berlin@\textbf{Berlin}, \emph{Hauptstadt}|pwk} Anfang
                     Januar 1902 gesehen.}}}\label{K_L03194-2} mit Dir nur noch
               verſtimmter und gedrückter heimgekehrt bin. Und das muß umſo mehr geſagt werden, als
               es in der letzten Zeit mehrfach dahin gekommen iſt, daß Du, weil Du eben nicht
               gründlich genug auf mich eingehſt, mich \strikeout{\textcolor{gray}{×}\-\textcolor{gray}{×}} nicht verſtanden und mich darum verletzt haſt. Du haſt, wenn ich mich darüber
               erregt habe, darin nichts geſehen, als eine koloſſale Empfindlichkeit. Ich will Dir
               nur{ }ſagen, daß die Gründe dieſer koloſſalen Empfindlichkeit tiefer liegen und daß
               unſere Differenzen nicht blos daher gekommen{ }ſind, weil Du ein \label{K_L03194-3v}\edtext{Feuilleton\pwindex{Goldmann, Paul 31.\,1.\,1865 Breslau – 25.\,9.\,1935 Wien@\textsc{Goldmann, Paul} (31.\,1.\,1865 Breslau – 25.\,9.\,1935 Wien), \emph{Schriftsteller, Journalist}!Berliner Theater. »Einsame Menschen« im Deutschen Theater@\strich\emph{Berliner Theater. »Einsame Menschen« im Deutschen Theater}|pwv}}{\lemma{\textnormal{\emph{Feuilleton}}}\Cendnote{\textnormal{Bezug auf die Auseinandersetzung, die Ende
                     1901 rund um ein Feuilleton\pwindex{Goldmann, Paul 31.\,1.\,1865 Breslau – 25.\,9.\,1935 Wien@\textsc{Goldmann, Paul} (31.\,1.\,1865 Breslau – 25.\,9.\,1935 Wien), \emph{Schriftsteller, Journalist}!Berliner Theater. »Einsame Menschen« im Deutschen Theater@\strich\emph{Berliner Theater. »Einsame Menschen« im Deutschen Theater}|pwkv}{ }Goldmanns\pwindex{Goldmann, Paul 31.\,1.\,1865 Breslau – 25.\,9.\,1935 Wien@\textsc{Goldmann, Paul} (31.\,1.\,1865 Breslau – 25.\,9.\,1935 Wien), \emph{Schriftsteller, Journalist}|pwk} über Gerhart Hauptmann\pwindex{Hauptmann, Gerhart 15.\,11.\,1862 Szczawno-Zdrój – 6.\,6.\,1946 Jagniątków@\textsc{Hauptmann, Gerhart} (15.\,11.\,1862 Szczawno-Zdrój – 6.\,6.\,1946 Jagniątków), \emph{Schriftsteller}|pwk} stattgefunden hatte, siehe XXXX Auszeichnungsfehler: Dokument L03090 nicht gefunden.}}}\label{K_L03194-3} von mir ungünſtig beurtheilt \strikeout{haſt und
                     \textcolor{gray}{weil}} oder weil Du mir eine \label{K_L03194-4v}\edtext{»Nachricht«}{\lemma{\textnormal{\emph{»Nachricht«}}}\Cendnote{\textnormal{Siehe XXXX Auszeichnungsfehler: Dokument L03192 nicht gefunden.
               }}}\label{K_L03194-4} gegeben haſt.\pend
           
\pstart
           Zweck hat es nicht viel, das Alles zu{ }ſagen. Ändern wird{ }ſich dadurch nichts. Unſer
               Verhältniß hat die Geſtalt angenommen, {\pb}die es
               nothwendiger Weiſe annehmen mußte in Folge der Verſchiedenheit der Lebensſtellungen
               und der Charaktere. In{ }ſolchen Verhältniſſen entſcheiden ja{ }ſchließlich auch nicht
               Raiſonnements{ }ſondern Empfindungen. Und über meine Empfindungen Dir gegenüber brauche
               ich wohl nicht erſt zu{ }ſprechen, ebenſo wie ich an Deinen aufrichtig\strikeout{\textcolor{gray}{e}} freundſchaftlichen Empfindungen
                  \strikeout{zu} mir gegenüber nicht den mindeſten Zweifel habe.
               Aber ich meine, die »Mißverſtändniſſe« (wie Du es nennſt), die in letzter Zeit
               zwiſchen uns vorgekommen{ }ſind,{ }ſollten in Zukunft unterbleiben. Gewiß, wir{ }ſollen
               nicht als Diplomaten,{ }ſondern als Freunde verkehren. Aber der Freund\strikeout{, kann nicht mit \textcolor{gray}{dem} Freund verkehren,
                  ohne{ }ſich} iſt im Verkehr mit dem Freunde erſt recht nicht der Verpflichtung
               enthoben,{ }ſich zu vergegenwärtigen, was eigentlich in deſſen Seele vorgeht.\pend
           
\pstart
           Und nun gib’ mir Deine Hand und{ }ſei \strikeout{von} vielmals
               und von Herzen gegrüßt! {\\[\baselineskip]}Dein \spacefill\mbox{Paul Goldmnn}\pend
           \leftskip=0em{}\selectlanguage{ngerman}\endnumbering\briefempfaengerindex{Schnitzler, Arthur@\textsc{Schnitzler, Arthur}!zzzGoldmann, Paul@\emph{von Paul Goldmann}!1902-01-202@{20. 1. [1902]}|)be}\mylabel{L03194h}  \newcommand{\dateiname}{L03194}\newcommand{\titel}{Paul Goldmann an Arthur Schnitzler, 20. 1. [1902]}\newcommand{\editorInnen}{Martin Anton Müller und Laura Untner}%% latex-leseansicht-abspann.tex
%% Abspann für die Leseansicht.
%% Der Schalter \ifkorrekturansicht ist bereits durch den Vorspann gesetzt.

%% latex-abspann.tex
%% Gemeinsamer Abspann für Korrekturansicht und Leseansicht.
%% Setzt den Schalter \ifkorrekturansicht voraus (gesetzt in den
%% einbindenden Dateien latex-korrekturansicht-abspann.tex bzw.
%% latex-leseansicht-abspann.tex).
%% ---------------------------------------------------------------

\normalsize

% Das esempio-Environment wird nur in der Leseansicht benötigt
\ifkorrekturansicht\else
\newenvironment{esempio}[3]%
{
    \vspace{1.5ex}
    \rlap{\underline{#1}}
    \par
    \setlength{\parindent}{0cm}
    \nopagebreak
    \leftskip=#2cm
    \rightskip=#3cm
}
{
    \par
}
\fi

\doendnotes{C}
\bigskip
\vfill

\clearpage

\footnotesize

\ifkorrekturansicht
  \lohead{\textsc{register}}
\fi

% theindex-Environment neu definieren ohne reledmac
\makeatletter
\renewenvironment{theindex}{%
  \ifkorrekturansicht
    \section*{\indexname}%
  \else
    \subsubsection*{Index der erwähnten Entitäten}%
  \fi
  \setlength{\parindent}{0pt}%
  \setlength{\parskip}{0pt plus 0.3pt}%
  \let\item\@idxitem
}{%
  \ifkorrekturansicht\clearpage\fi
}
\makeatother

\IfFileExists{\jobname-pw.ind}{\input{\jobname-pw.ind}}{}

% Quellenangabe nur in der Leseansicht
\ifkorrekturansicht\else
% Fallback-Definitionen, falls die .tex-Datei \titel etc. nicht gesetzt hat
\providecommand{\titel}{}
\providecommand{\editorInnen}{}
\providecommand{\dateiname}{\jobname}

\vspace{3cm}

\vfill

\footnotesize
\textsc{Quelle}: \titel. Herausgegeben von {\editorInnen}. In: \emph{Arthur Schnitzler: Briefwechsel mit Autorinnen und Autoren}.
 Digitale Edition, https://schnitzler-briefe.acdh.oeaw.ac.at/{\dateiname}.html (Stand \today)
\fi

\end{document}


