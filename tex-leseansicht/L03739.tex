%% latex-korrekturansicht-vorspann.tex
%% Vorspann für die Korrekturansicht.
%% Lädt die gemeinsame Datei latex-vorspann.tex mit gesetztem Schalter.

\newif\ifkorrekturansicht
\korrekturansichttrue

\input{../tex-inputs/latex-vorspann}


\section[Arthur Schnitzler an Stefan Zweig, 18. 8. {[}1920{]}]{L03739 Arthur Schnitzler an Stefan Zweig, 18. 8. {[}1920{]}}
\nopagebreak\mylabel{L03739v}
\rehead{ }\normalsize\beginnumbering\briefempfaengerindex{Zweig, Stefan@\textsc{Zweig, Stefan}!zzzSchnitzler, Arthur@\emph{von Arthur Schnitzler}!1920-08-181@{18. 8. {[}1920{]}}|(be}
\toendnotes[C]{\smallbreak\pagebreak[2]}\Standort{Jerusalem, National Library of Israel, ARC. Ms. Var. 305 1 58 Stefan Zweig Collection.}
\physDesc{Telegramm, 1 Blatt, 1 Seite, 164 Zeichen
\newline{}maschinell
\newline{}Versand: Stempel: »\nobreak{}18 Aug, 207\nobreak{}«.  }\pstart{}{\pb}stefan zweig salzburg kapuzinerberg\oindex{Kapuzinerberg@\textbf{Kapuzinerberg}, \emph{T.MT}|pw} =\pend{}{\bigskip}\vspace{1em}
\pstart
           \centering{}wien\oindex{Wien@\textbf{Wien}, \emph{A.ADM2}|pw} 111+ 346 15 18/8 10 m\pend
           \vspace{0.5em}
\pstart
           erbitte unverbindlichen ratschlag
               hinsichtlich bedingungen seltzer\orgindex{Thomas Seltzer, Inc.@Thomas Seltzer, Inc.|pw} herzlichen
      dank und gruss = \spacefill\mbox{schnitzler ’”}\pend
           \selectlanguage{ngerman}\endnumbering\briefempfaengerindex{Zweig, Stefan@\textsc{Zweig, Stefan}!zzzSchnitzler, Arthur@\emph{von Arthur Schnitzler}!1920-08-181@{18. 8. {[}1920{]}}|)be}\mylabel{L03739h}
\begin{anhang}
\end{anhang}\normalsize

\doendnotes{C}
\bigskip
\vfill

\clearpage

\footnotesize

\lohead{\textsc{register}}

% Definiere theindex-Environment komplett neu ohne reledmac
\makeatletter
\renewenvironment{theindex}{%
  \section*{\indexname}%
  \setlength{\parindent}{0pt}%
  \setlength{\parskip}{0pt plus 0.3pt}%
  \let\item\@idxitem
}{%
  \clearpage
}
\makeatother

\IfFileExists{\jobname-pw.ind}{\input{\jobname-pw.ind}}{}

\end{document}

      