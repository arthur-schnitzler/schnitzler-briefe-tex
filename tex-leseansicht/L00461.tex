%% latex-korrekturansicht-vorspann.tex
%% Vorspann für die Korrekturansicht.
%% Lädt die gemeinsame Datei latex-vorspann.tex mit gesetztem Schalter.

\newif\ifkorrekturansicht
\korrekturansichttrue

\input{../tex-inputs/latex-vorspann}


\section[Arthur Schnitzler an Richard Beer-Hofmann, {[}10. 7. 1895?{]}]{L00461 Arthur Schnitzler an Richard Beer-Hofmann, {[}10. 7. 1895?{]}}
\nopagebreak\mylabel{L00461v}
\rehead{ }\normalsize\beginnumbering\briefempfaengerindex{Beer-Hofmann, Richard@\textsc{Beer-Hofmann, Richard}!zzzSchnitzler, Arthur@\emph{von Arthur Schnitzler}!1895-07-101@{{[}10. 7. 1895?{]}}|(be}
\toendnotes[C]{\smallbreak\pagebreak[2]}\Standort{YCGL, MSS 31.}
\physDesc{Briefkarte, 439 Zeichen
\newline{}Handschrift: Bleistift, deutsche Kurrent}
\buchAbdrucke{\weitereDrucke{Arthur Schnitzler, Richard Beer-Hofmann: \emph{Briefwechsel 1891–1931}. Wien, Zürich: \emph{Europaverlag} 1992, S. 78.} }
\pstart
           \noindent{}{\pb}Lieber Richard, ich bin noch in \textsc{Marienbad}\oindex{Marienbad@\textbf{Marienbad}, \emph{P.PPL}|pw}. Vielleicht ko{\geminationm} ich So{\geminationn}tag nach Iſchl\oindex{Bad Ischl@\textbf{Bad Ischl}, \emph{P.PPL}|pw}. Jedenfalls erhalten Sie früher Nachricht, damit Sie nicht
               erſchrecken. In Prag\oindex{Prag@\textbf{Prag}, \emph{A.ADM1}|pw}, \textsc{Karlsbad}\oindex{Karlsbad@\textbf{Karlsbad}, \emph{P.PPLA}|pw} bin ich geweſen. Wenn Sie mir {\pb}noch heute
               ſchreiben, d. h. nach Erhalten dieſes hier, oder auch morgen, ſo beko{\geminationm} ich Ihren Brief noch da; – was mich herzlich freuen
               würde. Ich hoffe Sie ſind tief im Liebling\pwindex{Tod Georgs@\emph{Der Tod Georgs}|pw} und
               befinden ſich ſo wohl als ichs Ihnen wünſche.\pend
           \pstart Viele herzl. Grüße Ihr \spacefill\mbox{Arth}\pend{}\selectlanguage{ngerman}\endnumbering\briefempfaengerindex{Beer-Hofmann, Richard@\textsc{Beer-Hofmann, Richard}!zzzSchnitzler, Arthur@\emph{von Arthur Schnitzler}!1895-07-101@{{[}10. 7. 1895?{]}}|)be}\mylabel{L00461h}  \normalsize

\doendnotes{C}
\bigskip
\vfill

\clearpage

\footnotesize

\lohead{\textsc{register}}

% Definiere theindex-Environment komplett neu ohne reledmac
\makeatletter
\renewenvironment{theindex}{%
  \section*{\indexname}%
  \setlength{\parindent}{0pt}%
  \setlength{\parskip}{0pt plus 0.3pt}%
  \let\item\@idxitem
}{%
  \clearpage
}
\makeatother

\IfFileExists{\jobname-pw.ind}{\input{\jobname-pw.ind}}{}

\end{document}

      