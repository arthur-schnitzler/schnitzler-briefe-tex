%% latex-leseansicht-vorspann.tex
%% Vorspann für die Leseansicht.
%% Lädt die gemeinsame Datei latex-vorspann.tex mit nicht gesetztem Schalter.

\newif\ifkorrekturansicht
\korrekturansichtfalse

\input{../tex-inputs/latex-vorspann}

\begin{center}
            \textcolor{red}{ENTWURF, NICHT FERTIG KORRIGIERT}
                      \end{center}
            
         
         \renewcommand{\erwaehntePersonen}{Personen: Georg Hirschfeld}
         \renewcommand{\erwaehnteOrte}{Orte: Berlin, Wien, Wiesbaden}
         \renewcommand{\erwaehnteWerke}{}
               \section[Felix Salten an Arthur Schnitzler, 30. 10. 189{[}9?{]}]{ Felix Salten an Arthur Schnitzler, 30. 10. 189{[}9?{]}}\nopagebreak\mylabel{v}\rehead{ }\begin{ledgroupsized}[t]{13cm}\normalsize\beginnumbering \toendnotes[C]{\smallbreak\pagebreak[2]} \Standort{CUL, Schnitzler, B 89, A 2.}
\physDesc{Brief, 1 Blatt, 1 Seite
\newline{}Handschrift: Bleistift, lateinische Kurrent\newline{}Ordnung: mit Bleistift von unbekannter Hand nummeriert:
                                    »124« }\toendnotes[C]{\smallbreak}\pstart
           \raggedleft{}{\pb}3/10 \label{K_L03300-1v}\edtext{9\textcolor{gray}{9}}{\lemma{\textnormal{\emph{99}}}\Cendnote{\textnormal{Obzwar
                        die Jahresziffer nicht mit letzter Sicherheit zu lesen ist, lässt sich nur
                        in diesem Jahr eine zeitliche Nähe zwischen Schnitzler\pwindex{Schnitzler, Arthur 15.05.1862 – 21.10.1931@\textsc{Schnitzler, Arthur} (15.05.1862 – 21.10.1931), \emph{Schriftsteller, Mediziner}|pwk}s Aufenthalt in Wiesbaden\oindex{Wiesbaden@\textbf{Wiesbaden}|pwk} und dem 3. Oktober feststellen.}}}\label{K_L03300-1h}\pend
           \pstart
           Lieber, bitte theilen Sie mir mit, wie lange Sie wegbleiben, und
               wohin Sie von \label{K_L03300-2v}\edtext{Wiesbaden\oindex{Wiesbaden@\textbf{Wiesbaden}|pw} aus reisen}{\lemma{\textnormal{\emph{Wiesbaden aus reisen}}}\Cendnote{\textnormal{Schnitzler\pwindex{Schnitzler, Arthur 15.05.1862 – 21.10.1931@\textsc{Schnitzler, Arthur} (15.05.1862 – 21.10.1931), \emph{Schriftsteller, Mediziner}|pwk} hatte Wiesbaden\oindex{Wiesbaden@\textbf{Wiesbaden}|pwk} am 3. 10. 1899 verlassen und war nach Berlin\oindex{Berlin@\textbf{Berlin}|pwk} gereist. Am 11. 10. 1899 nahm er
                  Abends den Nachtzug nach Wien\oindex{Wien@\textbf{Wien}|pwk}. }}}\label{K_L03300-2h}. \pend
           \pstart
           Ich arbeite und lebe mühsam, das ist der Auszug meiner Tage. Mehr hab ich wirklich
               nicht zu sagen, wenigstens im Augenblick nicht. \pend
           \pstart
           Georg\pwindex{Hirschfeld, Georg 11.02.1873 – 17.01.1942@\textsc{Hirschfeld, Georg} (11.02.1873 – 17.01.1942), \emph{Schriftsteller}|pw} ist da. Schönsten Dank für Ihre Karten. Schreiben
               Sie bald.\pend
           \pstart Herzl Grüße Ihr \spacefill\mbox{F. S.}\pend{}
         
         \endnumbering\mylabel{h}\end{ledgroupsized}\begin{anhang}\end{anhang}\newcommand{\dateiname}{L03300}\newcommand{\titel}{Felix Salten an Arthur Schnitzler, 30. 10. 189[9?]}\newcommand{\editorInnen}{Martin Anton Müller und Laura Untner}%% latex-leseansicht-abspann.tex
%% Abspann für die Leseansicht.
%% Der Schalter \ifkorrekturansicht ist bereits durch den Vorspann gesetzt.

%% latex-abspann.tex
%% Gemeinsamer Abspann für Korrekturansicht und Leseansicht.
%% Setzt den Schalter \ifkorrekturansicht voraus (gesetzt in den
%% einbindenden Dateien latex-korrekturansicht-abspann.tex bzw.
%% latex-leseansicht-abspann.tex).
%% ---------------------------------------------------------------

\normalsize

% Das esempio-Environment wird nur in der Leseansicht benötigt
\ifkorrekturansicht\else
\newenvironment{esempio}[3]%
{
    \vspace{1.5ex}
    \rlap{\underline{#1}}
    \par
    \setlength{\parindent}{0cm}
    \nopagebreak
    \leftskip=#2cm
    \rightskip=#3cm
}
{
    \par
}
\fi

\doendnotes{C}
\bigskip
\vfill

\clearpage

\footnotesize

\ifkorrekturansicht
  \lohead{\textsc{register}}
\fi

% theindex-Environment neu definieren ohne reledmac
\makeatletter
\renewenvironment{theindex}{%
  \ifkorrekturansicht
    \section*{\indexname}%
  \else
    \subsubsection*{Index der erwähnten Entitäten}%
  \fi
  \setlength{\parindent}{0pt}%
  \setlength{\parskip}{0pt plus 0.3pt}%
  \let\item\@idxitem
}{%
  \ifkorrekturansicht\clearpage\fi
}
\makeatother

\IfFileExists{\jobname-pw.ind}{\input{\jobname-pw.ind}}{}

% Quellenangabe nur in der Leseansicht
\ifkorrekturansicht\else
% Fallback-Definitionen, falls die .tex-Datei \titel etc. nicht gesetzt hat
\providecommand{\titel}{}
\providecommand{\editorInnen}{}
\providecommand{\dateiname}{\jobname}

\vspace{3cm}

\vfill

\footnotesize
\textsc{Quelle}: \titel. Herausgegeben von {\editorInnen}. In: \emph{Arthur Schnitzler: Briefwechsel mit Autorinnen und Autoren}.
 Digitale Edition, https://schnitzler-briefe.acdh.oeaw.ac.at/{\dateiname}.html (Stand \today)
\fi

\end{document}


      