%% latex-leseansicht-vorspann.tex
%% Vorspann für die Leseansicht.
%% Lädt die gemeinsame Datei latex-vorspann.tex mit nicht gesetztem Schalter.

\newif\ifkorrekturansicht
\korrekturansichtfalse

\input{../tex-inputs/latex-vorspann}


\section[ Felix Salten an Arthur Schnitzler, 3. 10. 189[9?]]{L03300 Felix Salten an Arthur Schnitzler,  3. 10. 189[9?]}
\nopagebreak\mylabel{L03300v}
\rehead{ }\normalsize\beginnumbering\briefempfaengerindex{Schnitzler, Arthur@\textsc{Schnitzler, Arthur}!zzzSalten, Felix@\emph{von Felix Salten}!1899-10-033@{3. 10. 189[9?]}|(be}
\toendnotes[C]{\smallbreak\pagebreak[2]}
\correspDesc{Versand  durch Felix Salten am 3. 10. 189[9?] in Wien
\newline{}Umleitung  im Zeitraum [4. 10. 1899
                  – 8. 10. 1899?] in Wiesbaden
\newline{}Erhalt  durch Arthur Schnitzler im Zeitraum [5. 10. 1899
                  – 9. 10. 1899?] in Berlin}\toendnotes[C]{\smallbreak}
\Standort{CUL, Schnitzler, B 89, A 2.}
\physDesc{Brief, 1 Blatt, 1 Seite, 317 Zeichen
\newline{}Handschrift: Bleistift, lateinische Kurrent
\newline{}Ordnung: mit Bleistift von unbekannter Hand nummeriert: »124« }\toendnotes[C]{\smallbreak}
\pstart
           \raggedleft{}{\pb}3/10 \label{K_L03300-1v}\edtext{9\textcolor{gray}{9}}{\lemma{\textnormal{\emph{99}}}\Cendnote{\textnormal{Obzwar die Jahresziffer nicht mit
                        letzter Sicherheit zu lesen ist, lässt sich nur für das Jahr 1899 eine zeitliche Nähe zwischen Schnitzlers Aufenthalt in Wiesbaden\oindex{Wiesbaden@\textbf{Wiesbaden}|pwk} und dem 3. Oktober feststellen – Schnitzler war dort zwischen 24. 9. 1899 und
                           3. 10. 1899.}}}\label{K_L03300-1}\pend
           \vspace{0.5em}
\pstart
           Lieber, bitte theilen Sie mir mit, wie lange Sie wegbleiben, und
               wohin Sie von \label{K_L03300-2v}\edtext{Wiesbaden\oindex{Wiesbaden@\textbf{Wiesbaden}|pw} aus reisen}{\lemma{\textnormal{\emph{Wiesbaden aus reisen}}}\Cendnote{\textnormal{Schnitzler hatte Wiesbaden\oindex{Wiesbaden@\textbf{Wiesbaden}|pwk} am 3. 10. 1899 verlassen und war nach Berlin\oindex{Berlin@\textbf{Berlin}, \emph{Hauptstadt}|pwk} gereist. Am 11. 10. 1899 nahm er
                     Abends den Nachtzug nach Wien\oindex{Wien@\textbf{Wien}, \emph{Verwaltungsgebiet}|pwk}.}}}\label{K_L03300-2}.\pend
           
\pstart
           Ich arbeite und lebe mühsam, das ist der Auszug meiner Tage. Mehr hab ich wirklich
               nicht zu sagen, wenigstens im Augenblick nicht.\pend
           
\pstart
           Georg\pwindex{Hirschfeld, Georg 11.\,2.\,1873 Berlin – 17.\,1.\,1942 München@\textsc{Hirschfeld, Georg} (11.\,2.\,1873 Berlin – 17.\,1.\,1942 München), \emph{Schriftsteller}|pw} ist da.\pend
           
\pstart
           Schönsten Dank für Ihre Karten. Schreiben Sie bald.\pend
           \pstart Herzl Grüße Ihr \spacefill\mbox{F. S.}\pend{}\selectlanguage{ngerman}\endnumbering\briefempfaengerindex{Schnitzler, Arthur@\textsc{Schnitzler, Arthur}!zzzSalten, Felix@\emph{von Felix Salten}!1899-10-033@{3. 10. 189[9?]}|)be}\mylabel{L03300h}  \newcommand{\dateiname}{L03300}\newcommand{\titel}{Felix Salten an Arthur Schnitzler, 3. 10. 189[9?]}\newcommand{\editorInnen}{Martin Anton Müller und Laura Untner}%% latex-leseansicht-abspann.tex
%% Abspann für die Leseansicht.
%% Der Schalter \ifkorrekturansicht ist bereits durch den Vorspann gesetzt.

%% latex-abspann.tex
%% Gemeinsamer Abspann für Korrekturansicht und Leseansicht.
%% Setzt den Schalter \ifkorrekturansicht voraus (gesetzt in den
%% einbindenden Dateien latex-korrekturansicht-abspann.tex bzw.
%% latex-leseansicht-abspann.tex).
%% ---------------------------------------------------------------

\normalsize

% Das esempio-Environment wird nur in der Leseansicht benötigt
\ifkorrekturansicht\else
\newenvironment{esempio}[3]%
{
    \vspace{1.5ex}
    \rlap{\underline{#1}}
    \par
    \setlength{\parindent}{0cm}
    \nopagebreak
    \leftskip=#2cm
    \rightskip=#3cm
}
{
    \par
}
\fi

\doendnotes{C}
\bigskip
\vfill

\clearpage

\footnotesize

\ifkorrekturansicht
  \lohead{\textsc{register}}
\fi

% theindex-Environment neu definieren ohne reledmac
\makeatletter
\renewenvironment{theindex}{%
  \ifkorrekturansicht
    \section*{\indexname}%
  \else
    \subsubsection*{Index der erwähnten Entitäten}%
  \fi
  \setlength{\parindent}{0pt}%
  \setlength{\parskip}{0pt plus 0.3pt}%
  \let\item\@idxitem
}{%
  \ifkorrekturansicht\clearpage\fi
}
\makeatother

\IfFileExists{\jobname-pw.ind}{\input{\jobname-pw.ind}}{}

% Quellenangabe nur in der Leseansicht
\ifkorrekturansicht\else
% Fallback-Definitionen, falls die .tex-Datei \titel etc. nicht gesetzt hat
\providecommand{\titel}{}
\providecommand{\editorInnen}{}
\providecommand{\dateiname}{\jobname}

\vspace{3cm}

\vfill

\footnotesize
\textsc{Quelle}: \titel. Herausgegeben von {\editorInnen}. In: \emph{Arthur Schnitzler: Briefwechsel mit Autorinnen und Autoren}.
 Digitale Edition, https://schnitzler-briefe.acdh.oeaw.ac.at/{\dateiname}.html (Stand \today)
\fi

\end{document}


