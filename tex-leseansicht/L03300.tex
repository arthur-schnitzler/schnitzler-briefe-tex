%% latex-korrekturansicht-vorspann.tex
%% Vorspann für die Korrekturansicht.
%% Lädt die gemeinsame Datei latex-vorspann.tex mit gesetztem Schalter.

\newif\ifkorrekturansicht
\korrekturansichttrue

\input{../tex-inputs/latex-vorspann}


\section[ Felix Salten an Arthur Schnitzler, 3. 10. 189{[}9?{]}]{L03300 Felix Salten an Arthur Schnitzler, 3. 10. 189{[}9?{]}}
\nopagebreak\mylabel{L03300v}
\rehead{ }\normalsize\beginnumbering\briefempfaengerindex{Schnitzler, Arthur@\textsc{Schnitzler, Arthur}!zzzSalten, Felix@\emph{von Felix Salten}!1899-10-033@{3. 10. 189{[}9?{]}}|(be}
\toendnotes[C]{\smallbreak\pagebreak[2]}\Standort{CUL, Schnitzler, B 89, A 2.}
\physDesc{Brief, 1 Blatt, 1 Seite, 317 Zeichen
\newline{}Handschrift: Bleistift, lateinische Kurrent
\newline{}Ordnung: mit Bleistift von unbekannter Hand nummeriert: »124« }\toendnotes[C]{\smallbreak}
\pstart
           \raggedleft{}{\pb}3/10 \label{K_L03300-1v}\edtext{9\textcolor{gray}{9}}{\lemma{\textnormal{\emph{99}}}\Cendnote{\textnormal{Obzwar die Jahresziffer nicht mit
                        letzter Sicherheit zu lesen ist, lässt sich nur für das Jahr 1899 eine zeitliche Nähe zwischen Schnitzlers Aufenthalt in Wiesbaden\oindex{Wiesbaden@\textbf{Wiesbaden}, \emph{P.PPLA}|pwk} und dem 3. Oktober feststellen – Schnitzler war dort zwischen 24. 9. 1899 und
                           3. 10. 1899.}}}\label{K_L03300-1}\pend
           \vspace{0.5em}
\pstart
           Lieber, bitte theilen Sie mir mit, wie lange Sie wegbleiben, und
               wohin Sie von \label{K_L03300-2v}\edtext{Wiesbaden\oindex{Wiesbaden@\textbf{Wiesbaden}, \emph{P.PPLA}|pw} aus reisen}{\lemma{\textnormal{\emph{Wiesbaden aus reisen}}}\Cendnote{\textnormal{Schnitzler hatte Wiesbaden\oindex{Wiesbaden@\textbf{Wiesbaden}, \emph{P.PPLA}|pwk} am 3. 10. 1899 verlassen und war nach Berlin\oindex{Berlin@\textbf{Berlin}, \emph{P.PPLC}|pwk} gereist. Am 11. 10. 1899 nahm er
                     Abends den Nachtzug nach Wien\oindex{Wien@\textbf{Wien}, \emph{A.ADM2}|pwk}.}}}\label{K_L03300-2}.\pend
           
\pstart
           Ich arbeite und lebe mühsam, das ist der Auszug meiner Tage. Mehr hab ich wirklich
               nicht zu sagen, wenigstens im Augenblick nicht.\pend
           
\pstart
           Georg\pwindex{Hirschfeld, Georg 11.02.1873 – 17.01.1942@\textsc{Hirschfeld, Georg} (11.02.1873 – 17.01.1942), \emph{Schriftsteller/Schriftstellerin}|pw} ist da.\pend
           
\pstart
           Schönsten Dank für Ihre Karten. Schreiben Sie bald.\pend
           \pstart Herzl Grüße Ihr \spacefill\mbox{F. S.}\pend{}\selectlanguage{ngerman}\endnumbering\briefempfaengerindex{Schnitzler, Arthur@\textsc{Schnitzler, Arthur}!zzzSalten, Felix@\emph{von Felix Salten}!1899-10-033@{3. 10. 189{[}9?{]}}|)be}\mylabel{L03300h}  \normalsize

\doendnotes{C}
\bigskip
\vfill

\clearpage

\footnotesize

\lohead{\textsc{register}}

% Definiere theindex-Environment komplett neu ohne reledmac
\makeatletter
\renewenvironment{theindex}{%
  \section*{\indexname}%
  \setlength{\parindent}{0pt}%
  \setlength{\parskip}{0pt plus 0.3pt}%
  \let\item\@idxitem
}{%
  \clearpage
}
\makeatother

\IfFileExists{\jobname-pw.ind}{\input{\jobname-pw.ind}}{}

\end{document}

      