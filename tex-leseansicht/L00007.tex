%% latex-korrekturansicht-vorspann.tex
%% Vorspann für die Korrekturansicht.
%% Lädt die gemeinsame Datei latex-vorspann.tex mit gesetztem Schalter.

\newif\ifkorrekturansicht
\korrekturansichttrue

\input{../tex-inputs/latex-vorspann}


\section[Wilhelm Bölsche an Arthur Schnitzler, 25. 10. 1890]{L00007 Wilhelm Bölsche an Arthur Schnitzler, 25. 10. 1890}
\nopagebreak\mylabel{L00007v}
\rehead{ }\normalsize\beginnumbering\briefempfaengerindex{Schnitzler, Arthur@\textsc{Schnitzler, Arthur}!zzzBoelsche, Wilhelm@\emph{von Wilhelm Bölsche}!1890-10-251@{25. 10. 1890}|(be}
\toendnotes[C]{\smallbreak\pagebreak[2]}\Standort{DLA, A:Schnitzler, HS.NZ85.1.2577,1.}
\physDesc{Brief, 1 Blatt, 1 Seite, 280 Zeichen
\newline{}Handschrift: schwarze Tinte, deutsche Kurrent
\newline{}Schnitzler: mit rotem Buntstift nummeriert: »3« }
\buchAbdrucke{\weitereDrucke{Wilhelm Bölsche: \emph{Briefwechsel. Mit Autoren der Freien Bühne}. Berlin: \emph{Weidler} 2010, S. 669.} }\toendnotes[C]{\smallbreak}
\pstart
           \raggedleft{}{\pb}25. X. 90.\pend
           
\pstart\center{}Verehrter Herr Doktor!\pend\vspace{0.5em}
\pstart
           Leider haben wir »Gedichten« bei der »Freien
                  Bühne\pwindex{Freie Buehne fuer modernes Leben@\emph{Freie Bühne für modernes Leben}|pw}« jetzt \label{K_L00007-1v}\edtext{ganz
                  abgeſchworen}{\lemma{\textnormal{\emph{ganz
                  abgeſchworen}}}\Cendnote{\textnormal{Das letzte Gedicht war
                  knapp vier Monate zuvor in der \emph{Freien Bühne}\pwindex{Freie Buehne fuer modernes Leben@\emph{Freie Bühne für modernes Leben}|pwk}
                  in Heft 22 vom 2. 7. 1890 erschienen.}}}\label{K_L00007-1} und bringen \uline{nur} Proſa. So muß ich alſo Ihr Gedicht\pwindex{Morgenandacht@\emph{Morgenandacht}|pwv} auch ablehnen, das übrigens (bei
               etwas ſtarker Länge) ſeines Reizes nicht entbehrt.\pend
           
\pstart
           Mit vorzüglicher Hochachtung{\\[\baselineskip]}\spacefill\mbox{Wilhelm Bölsche.}\pend
           \leftskip=0em{}\selectlanguage{ngerman}\endnumbering\briefempfaengerindex{Schnitzler, Arthur@\textsc{Schnitzler, Arthur}!zzzBoelsche, Wilhelm@\emph{von Wilhelm Bölsche}!1890-10-251@{25. 10. 1890}|)be}\mylabel{L00007h}  \normalsize

\doendnotes{C}
\bigskip
\vfill

\clearpage

\footnotesize

\lohead{\textsc{register}}

% Definiere theindex-Environment komplett neu ohne reledmac
\makeatletter
\renewenvironment{theindex}{%
  \section*{\indexname}%
  \setlength{\parindent}{0pt}%
  \setlength{\parskip}{0pt plus 0.3pt}%
  \let\item\@idxitem
}{%
  \clearpage
}
\makeatother

\IfFileExists{\jobname-pw.ind}{\input{\jobname-pw.ind}}{}

\end{document}

      