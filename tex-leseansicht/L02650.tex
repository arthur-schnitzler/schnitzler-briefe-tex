%% latex-leseansicht-vorspann.tex
%% Vorspann für die Leseansicht.
%% Lädt die gemeinsame Datei latex-vorspann.tex mit nicht gesetztem Schalter.

\newif\ifkorrekturansicht
\korrekturansichtfalse

\input{../tex-inputs/latex-vorspann}


         
         \renewcommand{\erwaehntePersonen}{Personen: Marie Glümer, Fedor Mamroth}
         \renewcommand{\erwaehnteInstitutionen}{Institutionen: An der schönen blauen Donau, Josef Eberle  Stein-, Buch und Musikaliendruckerei}
         \renewcommand{\erwaehnteOrte}{Orte: Berggasse, Salzburg, Seidengasse, Wien, Österreichischer Hof}
         \renewcommand{\erwaehnteWerke}{
               \section[Paul Goldmann an Arthur Schnitzler, 25. 9. 1890]{ Paul Goldmann an Arthur Schnitzler, 25. 9. 1890}\nopagebreak\mylabel{v}\rehead{ }\begin{ledgroupsized}[t]{13cm}\normalsize\beginnumbering \toendnotes[C]{\smallbreak\pagebreak[2]} \Standort{DLA, A:Schnitzler, HS.NZ85.1.3162.}
\physDesc{Brief, 1 Blatt, 2 Seiten
\newline{}Handschrift: blaue Tinte, deutsche Kurrent}\toendnotes[C]{\smallbreak}\pstart
           \noindent{}\centering{}{\pb}\textcolor{gray}{\textbf{\textbf{Adminiſtration: VII.
                           Seidengaſſe 7\oindex{Seidengasse@\textbf{Seidengasse}|pw}} (Jos. Eberle {\kaufmannsund} Co.\orgindex{Josef Eberle Stein-, Buch und Musikaliendruckerei@Josef Eberle Stein-, Buch und Musikaliendruckerei|pw})}}\pend
           \pstart
           \noindent{}\centering{}\textcolor{gray}{\textbf{An der Schönen Blauen Donau\orgindex{der schoenen blauen Donau@An der schönen blauen Donau|pw}}}\pend
           \pstart
           \noindent{}\centering{}\textcolor{gray}{\textbf{Chef-Redacteur: Dr. F.
                        Mamroth\pwindex{Mamroth, Fedor 21.02.1851 – 25.06.1907@\textsc{Mamroth, Fedor} (21.02.1851 – 25.06.1907), \emph{Journalist, Kritiker}|pw}. – Redaction: IX.,
                        Berggaſſe 31\oindex{Berggasse@\textbf{Berggasse}|pw}.}}\pend
           \pstart
           \raggedleft{}\textcolor{gray}{\textbf{Wien\oindex{Wien@\textbf{Wien}|pw}, den}}{ }25. September \textcolor{gray}{\textbf{18}}90.\pend
           \pstart\center{}Mein lieber Arthur!\pend\pstart
           Es hat ſich ſo getroffen, daß ich erſt heut nach Salzburg\oindex{Salzburg@\textbf{Salzburg}|pw} fahre. Ich \label{K_L02650-1v}\edtext{ſuche Dich in den nächſten Tagen auf}{\lemma{\textnormal{\emph{ſuche … auf}}}\Cendnote{\textnormal{Schnitzler\pwindex{Schnitzler, Arthur 15.05.1862 – 21.10.1931@\textsc{Schnitzler, Arthur} (15.05.1862 – 21.10.1931), \emph{Schriftsteller, Mediziner}|pwk} hielt sich vom 18. 9. 1890 bis zum
                     4. 10. 1890 in
                     Salzburg\oindex{Salzburg@\textbf{Salzburg}|pwk} auf, um hier ein paar Tage mit Marie Glümer\pwindex{Gluemer, Marie 03.07.1867 – 16.11.1925@\textsc{Glümer, Marie} (03.07.1867 – 16.11.1925), \emph{Schauspielerin}|pwk} verbringen zu können.}}}\label{K_L02650-1h}
               und bitte Dich, täglich im Hotel\oindex{Oesterreichischer Hof@\textbf{Österreichischer Hof}|pwv} eine Notiz zu hinterlaſſen, wo Du \label{K_L02650-12v}\edtext{zu finden biſt}{\lemma{\textnormal{\emph{zu finden biſt}}}\Cendnote{\textnormal{Sie trafen sich am 27. 9. 1890, 28. 9. 1890 und 29. 9. 1890.}}}\label{K_L02650-12h}\substVorne{}\textsuperscript{.}\substDazwischen{},\substHinten{} das heißt wenigſtens zu gewiſſen Hauptzeiten des Tages, zum Mittag- und
               Nachtmahl. Erſt muß ich nämlich mit meinem \label{K_L02650-2v}\edtext{Onkel\pwindex{Mamroth, Fedor 21.02.1851 – 25.06.1907@\textsc{Mamroth, Fedor} (21.02.1851 – 25.06.1907), \emph{Journalist, Kritiker}|pwv}}{\lemma{\textnormal{\emph{Onkel}}}\Cendnote{\textnormal{Auch Fedor Mamroth\pwindex{Mamroth, Fedor 21.02.1851 – 25.06.1907@\textsc{Mamroth, Fedor} (21.02.1851 – 25.06.1907), \emph{Journalist, Kritiker}|pwk} reiste mit nach Salzburg\oindex{Salzburg@\textbf{Salzburg}|pwk}.}}}\label{K_L02650-2h} das Viele, was vorliegt, beſprechen, und dann kann ich erſt
               zu Dir.\pend
           \pstart
           {\pb}Da ich die wenigen Stunden vor meiner Abreiſe alle
               Hände voll zu thun habe, kann ich \label{K_L02650-11v}\edtext{Deinen lieben Brief}{\lemma{\textnormal{\emph{Deinen lieben Brief}}}\Cendnote{\textnormal{Der Inhalt des
                  Briefes ist unklar. Aus der verspäteten Antwort, die Goldmann\pwindex{Goldmann, Paul 31.01.1865 – 25.09.1935@\textsc{Goldmann, Paul} (31.01.1865 – 25.09.1935), \emph{Schriftsteller, Journalist}|pwk} hier rechtfertigt, geht zumindest hervor, dass er
                     Schnitzler\pwindex{Schnitzler, Arthur 15.05.1862 – 21.10.1931@\textsc{Schnitzler, Arthur} (15.05.1862 – 21.10.1931), \emph{Schriftsteller, Mediziner}|pwk} ins Vertrauen über eine
                  Krankheit gesetzt habe, an der er leide. Genaueres lässt sich nicht bestimmen,
                  doch dürfte es sich eher um eine psychische Disposition als um etwas Behandelbares
                  gehandelt haben (vgl. Paul Goldmann an Arthur Schnitzler, 1. 10. 1890).}}}\label{K_L02650-11h}
               nicht beantworten, ſo ſehr \strikeout{ich} es mich dazu drängt.
               Mündlich läßt ſich das aber nicht ſagen, wie Du mit feinem Tact herausgefühlt. Ich
               denke alſo, wir betrachten ihn für die Stunden unſeres jetzigen Zuſammenſeins als
               nicht geſchrieben und reden nicht davon. Willſt Du aber doch davon reden, ſo fang’ Du
               an. Sonſt ſchreibe ich Dir all’ das Viele, was ich darauf zu bemerken habe, nach
               meiner Rückkehr. Einſtweilen danke ich Dir für die männliche und offene Rede!\pend
           \pstart
           Gott zum Gruß! Auf Wiederſehen! {\\[\baselineskip]} Dein {\\[\baselineskip]}\spacefill\mbox{Paul Goldmann.}\pend
           \leftskip=0em{}
         
         \endnumbering\mylabel{h}\end{ledgroupsized}  \newcommand{\dateiname}{L02650}\newcommand{\titel}{Paul Goldmann an Arthur Schnitzler, 25. 9. 1890}\newcommand{\editorInnen}{Martin Anton Müller und Laura Untner}%% latex-leseansicht-abspann.tex
%% Abspann für die Leseansicht.
%% Der Schalter \ifkorrekturansicht ist bereits durch den Vorspann gesetzt.

%% latex-abspann.tex
%% Gemeinsamer Abspann für Korrekturansicht und Leseansicht.
%% Setzt den Schalter \ifkorrekturansicht voraus (gesetzt in den
%% einbindenden Dateien latex-korrekturansicht-abspann.tex bzw.
%% latex-leseansicht-abspann.tex).
%% ---------------------------------------------------------------

\normalsize

% Das esempio-Environment wird nur in der Leseansicht benötigt
\ifkorrekturansicht\else
\newenvironment{esempio}[3]%
{
    \vspace{1.5ex}
    \rlap{\underline{#1}}
    \par
    \setlength{\parindent}{0cm}
    \nopagebreak
    \leftskip=#2cm
    \rightskip=#3cm
}
{
    \par
}
\fi

\doendnotes{C}
\bigskip
\vfill

\clearpage

\footnotesize

\ifkorrekturansicht
  \lohead{\textsc{register}}
\fi

% theindex-Environment neu definieren ohne reledmac
\makeatletter
\renewenvironment{theindex}{%
  \ifkorrekturansicht
    \section*{\indexname}%
  \else
    \subsubsection*{Index der erwähnten Entitäten}%
  \fi
  \setlength{\parindent}{0pt}%
  \setlength{\parskip}{0pt plus 0.3pt}%
  \let\item\@idxitem
}{%
  \ifkorrekturansicht\clearpage\fi
}
\makeatother

\IfFileExists{\jobname-pw.ind}{\input{\jobname-pw.ind}}{}

% Quellenangabe nur in der Leseansicht
\ifkorrekturansicht\else
% Fallback-Definitionen, falls die .tex-Datei \titel etc. nicht gesetzt hat
\providecommand{\titel}{}
\providecommand{\editorInnen}{}
\providecommand{\dateiname}{\jobname}

\vspace{3cm}

\vfill

\footnotesize
\textsc{Quelle}: \titel. Herausgegeben von {\editorInnen}. In: \emph{Arthur Schnitzler: Briefwechsel mit Autorinnen und Autoren}.
 Digitale Edition, https://schnitzler-briefe.acdh.oeaw.ac.at/{\dateiname}.html (Stand \today)
\fi

\end{document}


      