%% latex-korrekturansicht-vorspann.tex
%% Vorspann für die Korrekturansicht.
%% Lädt die gemeinsame Datei latex-vorspann.tex mit gesetztem Schalter.

\newif\ifkorrekturansicht
\korrekturansichttrue

\input{../tex-inputs/latex-vorspann}


\section[Paul Goldmann an Arthur Schnitzler, 25. 9. 1890]{L02650 Paul Goldmann an Arthur Schnitzler, 25. 9. 1890}
\nopagebreak\mylabel{L02650v}
\rehead{ }\normalsize\beginnumbering\briefempfaengerindex{Schnitzler, Arthur@\textsc{Schnitzler, Arthur}!zzzGoldmann, Paul@\emph{von Paul Goldmann}!1890-09-251@{25. 9. 1890}|(be}
\toendnotes[C]{\smallbreak\pagebreak[2]}\Standort{DLA, A:Schnitzler, HS.NZ85.1.3162.}
\physDesc{Brief, 1 Blatt, 2 Seiten, 1013 Zeichen
\newline{}Handschrift: blaue Tinte, deutsche Kurrent}\toendnotes[C]{\smallbreak}
\pstart
           \centering{}{\pb}\textcolor{gray}{\textbf{\textbf{Adminiſtration: VII.
                           Seidengaſſe 7\oindex{Seidengasse@\textbf{Seidengasse}, \emph{Straße (K.STR)}|pw}} (Jos. Eberle {\kaufmannsund} Co.\orgindex{Josef Eberle Stein-, Buch und Musikaliendruckerei@Josef Eberle Stein-, Buch und Musikaliendruckerei|pw})}}\pend
           
\pstart
           \centering{}\textcolor{gray}{\textbf{An der Schönen Blauen Donau\orgindex{der schoenen blauen Donau@An der schönen blauen Donau|pw}}}\pend
           
\pstart
           \centering{}\textcolor{gray}{\textbf{Chef-Redacteur: Dr. F.
                        Mamroth\pwindex{Mamroth, Fedor 21.02.1851 – 25.06.1907@\textsc{Mamroth, Fedor} (21.02.1851 – 25.06.1907), \emph{Journalist/Journalistin, Kritiker/Kritikerin}|pw}. – Redaction: IX.,
                        Berggaſſe 31\oindex{Berggasse@\textbf{Berggasse}, \emph{Straße (K.STR)}|pw}.}}\pend
           
\pstart
           \raggedleft{}\textcolor{gray}{\textbf{Wien\oindex{Wien@\textbf{Wien}, \emph{A.ADM2}|pw}, den}}{ }25. September \textcolor{gray}{\textbf{18}}90.\pend
           
\pstart\center{}Mein lieber Arthur!\pend\vspace{0.5em}
\pstart
           Es hat ſich ſo getroffen, daß ich erſt heut nach Salzburg\oindex{Salzburg@\textbf{Salzburg}, \emph{A.ADM2}|pw} fahre. Ich \label{K_L02650-1v}\edtext{ſuche Dich in den nächſten Tagen auf}{\lemma{\textnormal{\emph{ſuche … auf}}}\Cendnote{\textnormal{Schnitzler hielt sich vom 18. 9. 1890 bis zum
                     4. 10. 1890 in
                     Salzburg\oindex{Salzburg@\textbf{Salzburg}, \emph{A.ADM2}|pwk} auf, um ein paar Tage mit Marie Glümer\pwindex{Gluemer, Marie 03.07.1867 – 16.11.1925@\textsc{Glümer, Marie} (03.07.1867 – 16.11.1925), \emph{Schauspieler/Schauspielerin}|pwk} verbringen zu können.}}}\label{K_L02650-1}
               und bitte Dich, täglich im Hotel\oindex{Oesterreichischer Hof@\textbf{Österreichischer Hof}, \emph{Hotel (K.HTL)}|pwv} eine Notiz zu hinterlaſſen, wo Du \label{K_L02650-2v}\edtext{zu finden biſt}{\lemma{\textnormal{\emph{zu finden biſt}}}\Cendnote{\textnormal{Sie trafen sich am 27. 9. 1890, 28. 9. 1890 und 29. 9. 1890.}}}\label{K_L02650-2}\substVorne{}\textsuperscript{.}\substDazwischen{},\substHinten{} das heißt wenigſtens zu gewiſſen Hauptzeiten des Tages, zum Mittag- und
               Nachtmahl. Erſt muß ich nämlich mit meinem \label{K_L02650-3v}\edtext{Onkel\pwindex{Mamroth, Fedor 21.02.1851 – 25.06.1907@\textsc{Mamroth, Fedor} (21.02.1851 – 25.06.1907), \emph{Journalist/Journalistin, Kritiker/Kritikerin}|pwv}}{\lemma{\textnormal{\emph{Onkel}}}\Cendnote{\textnormal{Auch Fedor Mamroth\pwindex{Mamroth, Fedor 21.02.1851 – 25.06.1907@\textsc{Mamroth, Fedor} (21.02.1851 – 25.06.1907), \emph{Journalist/Journalistin, Kritiker/Kritikerin}|pwk} reiste mit nach Salzburg\oindex{Salzburg@\textbf{Salzburg}, \emph{A.ADM2}|pwk}.}}}\label{K_L02650-3} das Viele, was vorliegt, beſprechen, und dann kann ich erſt
               zu Dir.\pend
           
\pstart
           {\pb}Da ich die wenigen Stunden vor meiner Abreiſe alle
               Hände voll zu thun habe, kann ich \label{K_L02650-4v}\edtext{Deinen lieben Brief}{\lemma{\textnormal{\emph{Deinen lieben Brief}}}\Cendnote{\textnormal{Der Inhalt des
                  Briefes ist unklar. Aus der verspäteten Antwort, die Goldmann\pwindex{Goldmann, Paul 31.01.1865 – 25.09.1935@\textsc{Goldmann, Paul} (31.01.1865 – 25.09.1935), \emph{Schriftsteller/Schriftstellerin, Journalist/Journalistin}|pwk} hier rechtfertigt, geht zumindest hervor, dass er
                     Schnitzler ins Vertrauen über eine
                  Erkrankung gesetzt hatte. Genaueres lässt sich nicht bestimmen,
                  doch dürfte es sich um eine psychische Disposition 
                  gehandelt haben (vgl. Paul Goldmann an Arthur Schnitzler, 1. 10. 1890).}}}\label{K_L02650-4}
               nicht beantworten, ſo ſehr \strikeout{ich} es mich dazu drängt.
               Mündlich läßt ſich das aber nicht ſagen, wie Du mit feinem Tact herausgefühlt. Ich
               denke alſo, wir betrachten ihn für die Stunden unſeres jetzigen Zuſammenſeins als
               nicht geſchrieben und reden nicht davon. Willſt Du aber doch davon reden, ſo fang’ Du
               an. Sonſt ſchreibe ich Dir all’ das Viele, was ich darauf zu bemerken habe, nach
               meiner Rückkehr. Einſtweilen danke ich Dir für die männliche und offene Rede!\pend
           
\pstart
           Gott zum Gruß! Auf Wiederſehen! {\\[\baselineskip]} Dein {\\[\baselineskip]}\spacefill\mbox{Paul Goldmann.}\pend
           \leftskip=0em{}\selectlanguage{ngerman}\endnumbering\briefempfaengerindex{Schnitzler, Arthur@\textsc{Schnitzler, Arthur}!zzzGoldmann, Paul@\emph{von Paul Goldmann}!1890-09-251@{25. 9. 1890}|)be}\mylabel{L02650h}  \normalsize

\doendnotes{C}
\bigskip
\vfill

\clearpage

\footnotesize

\lohead{\textsc{register}}

% Definiere theindex-Environment komplett neu ohne reledmac
\makeatletter
\renewenvironment{theindex}{%
  \section*{\indexname}%
  \setlength{\parindent}{0pt}%
  \setlength{\parskip}{0pt plus 0.3pt}%
  \let\item\@idxitem
}{%
  \clearpage
}
\makeatother

\IfFileExists{\jobname-pw.ind}{\input{\jobname-pw.ind}}{}

\end{document}

      