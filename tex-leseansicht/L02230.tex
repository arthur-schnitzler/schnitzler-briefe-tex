%% latex-leseansicht-vorspann.tex
%% Vorspann für die Leseansicht.
%% Lädt die gemeinsame Datei latex-vorspann.tex mit nicht gesetztem Schalter.

\newif\ifkorrekturansicht
\korrekturansichtfalse

\input{../tex-inputs/latex-vorspann}


\section[Arthur Schnitzler an Robert Adam, 10. 7. 1916]{L02230 Arthur Schnitzler an Robert Adam, 10. 7. 1916}
\nopagebreak\mylabel{L02230v}
\rehead{ }\normalsize\beginnumbering\briefempfaengerindex{Adam, Robert@\textsc{Adam, Robert}!zzzSchnitzler, Arthur@\emph{von Arthur Schnitzler}!1916-07-101@{10. 7. 1916}|(be}
\toendnotes[C]{\smallbreak\pagebreak[2]}
\correspDesc{Versand  durch Arthur Schnitzler am 10. 7. 1916 in Altaussee
\newline{}Erhalt  durch Robert Adam im Zeitraum [10. 7. 1916
                  – 14. 7. 1916?] in Altaussee}\toendnotes[C]{\smallbreak}
\Standort{DLA, 96.34.1/18.}
\physDesc{Postkarte, 449 Zeichen
\newline{}Handschrift: Bleistift, lateinische Kurrent
\newline{}Versand: Stempel: »\nobreak{}10. VII. 16\nobreak{}«.  }\toendnotes[C]{\smallbreak}\pstart{}{\pb}Schnitzler\pend{}\pstart{}Altaussee\oindex{Altaussee@\textbf{Altaussee}, \emph{Verwaltungsgebiet}|pw}\pend{}\pstart{}Fischerndorf 79\oindex{Fischerndorf@\textbf{Fischerndorf}|pw}\pend{}{\bigskip}\pstart{}Hrn Dr. Rob. Adam Pollak, Bezirksrichter aus Wien\oindex{Wien@\textbf{Wien}, \emph{Verwaltungsgebiet}|pw}\pend{}\pstart{}d. Z. Markt Aussee\oindex{Bad Aussee@\textbf{Bad Aussee}, \emph{Hauptstadt}|pw}\pend{}\pstart{}bei Althauser\pwindex{Althauser @\textsc{Althauser}, \emph{Vermieter/Vermieterin}|pw}\pend{}{\bigskip}\vspace{1em}
\pstart
           \noindent{}{\pb}verehrter Herr Doktor, ich habe sehr bedauert Ihren Besuch versäumt
               zu haben. Würde es Ihnen etwa passen, am \label{K_L02230-1v}\edtext{Mittwoch}{\lemma{\textnormal{\emph{Mittwoch}}}\Cendnote{\textnormal{12. 7. 1916}}}\label{K_L02230-1} oder Donnerstag um 6 Uhr Nm mich zu einem
               Spaziergang abzuholen? Oder ist Ihnen der Vormittg lieber? Da könnten Sie da{\geminationn} (wie wir es auch thun) beim Seewirth\oindex{Hotel am See@\textbf{Hotel am See}, \emph{Hotel}|pw} speisen? Ich \textcolor{gray}{hoffe in }{ }\textcolor{gray}{×}\-\textcolor{gray}{×}\-\textcolor{gray}{×}\-\textcolor{gray}{×}\-\textcolor{gray}{×} u. Sie zu sehen. Herzlichst
                  grüßen{[}d{]} Ihr ergebn\textcolor{gray}{er}\spacefill\mbox{A. S.}\pend
           \selectlanguage{ngerman}\endnumbering\briefempfaengerindex{Adam, Robert@\textsc{Adam, Robert}!zzzSchnitzler, Arthur@\emph{von Arthur Schnitzler}!1916-07-101@{10. 7. 1916}|)be}\mylabel{L02230h}  \newcommand{\dateiname}{L02230}\newcommand{\titel}{Arthur Schnitzler an Robert Adam, 10. 7. 1916}\newcommand{\editorInnen}{Martin Anton Müller und Gerd-Hermann Susen}%% latex-leseansicht-abspann.tex
%% Abspann für die Leseansicht.
%% Der Schalter \ifkorrekturansicht ist bereits durch den Vorspann gesetzt.

%% latex-abspann.tex
%% Gemeinsamer Abspann für Korrekturansicht und Leseansicht.
%% Setzt den Schalter \ifkorrekturansicht voraus (gesetzt in den
%% einbindenden Dateien latex-korrekturansicht-abspann.tex bzw.
%% latex-leseansicht-abspann.tex).
%% ---------------------------------------------------------------

\normalsize

% Das esempio-Environment wird nur in der Leseansicht benötigt
\ifkorrekturansicht\else
\newenvironment{esempio}[3]%
{
    \vspace{1.5ex}
    \rlap{\underline{#1}}
    \par
    \setlength{\parindent}{0cm}
    \nopagebreak
    \leftskip=#2cm
    \rightskip=#3cm
}
{
    \par
}
\fi

\doendnotes{C}
\bigskip
\vfill

\clearpage

\footnotesize

\ifkorrekturansicht
  \lohead{\textsc{register}}
\fi

% theindex-Environment neu definieren ohne reledmac
\makeatletter
\renewenvironment{theindex}{%
  \ifkorrekturansicht
    \section*{\indexname}%
  \else
    \subsubsection*{Index der erwähnten Entitäten}%
  \fi
  \setlength{\parindent}{0pt}%
  \setlength{\parskip}{0pt plus 0.3pt}%
  \let\item\@idxitem
}{%
  \ifkorrekturansicht\clearpage\fi
}
\makeatother

\IfFileExists{\jobname-pw.ind}{\input{\jobname-pw.ind}}{}

% Quellenangabe nur in der Leseansicht
\ifkorrekturansicht\else
% Fallback-Definitionen, falls die .tex-Datei \titel etc. nicht gesetzt hat
\providecommand{\titel}{}
\providecommand{\editorInnen}{}
\providecommand{\dateiname}{\jobname}

\vspace{3cm}

\vfill

\footnotesize
\textsc{Quelle}: \titel. Herausgegeben von {\editorInnen}. In: \emph{Arthur Schnitzler: Briefwechsel mit Autorinnen und Autoren}.
 Digitale Edition, https://schnitzler-briefe.acdh.oeaw.ac.at/{\dateiname}.html (Stand \today)
\fi

\end{document}


