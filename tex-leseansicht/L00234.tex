%% latex-korrekturansicht-vorspann.tex
%% Vorspann für die Korrekturansicht.
%% Lädt die gemeinsame Datei latex-vorspann.tex mit gesetztem Schalter.

\newif\ifkorrekturansicht
\korrekturansichttrue

\input{../tex-inputs/latex-vorspann}


\section[Arthur Schnitzler an Wilhelm Bölsche, 8. 7. 1893]{L00234 Arthur Schnitzler an Wilhelm Bölsche, 8. 7. 1893}
\nopagebreak\mylabel{L00234v}
\rehead{ }\normalsize\beginnumbering\briefempfaengerindex{Boelsche, Wilhelm@\textsc{Bölsche, Wilhelm}!zzzSchnitzler, Arthur@\emph{von Arthur Schnitzler}!1893-07-081@{8. 7. 1893}|(be}
\toendnotes[C]{\smallbreak\pagebreak[2]}\Standort{Wrocław, Biblioteka Uniwersytecka, Böl.Pis 1770.}
\physDesc{Brief, 1 Blatt, 3 Seiten, 528 Zeichen (Briefpapier mit Trauerrand)
\newline{}Handschrift: schwarze Tinte, deutsche Kurrent
\newline{}Bölsche: als »Erl{[}edigt{]}« gezeichnet }
\buchAbdrucke{\weitereDrucke{1) \emph{Germanica Wratislaviensia} (1987) Nr. 77, S. 463–464.} \weitereDrucke{2) Wilhelm Bölsche: \emph{Briefwechsel. Mit Autoren der Freien Bühne}. Berlin: \emph{Weidler} 2010, S. 692.} }\toendnotes[C]{\smallbreak}
\pstart{}{\pb}Sehr geehrter Herr Doktor,\pend\vspace{0.5em}
\pstart
           erlauben Sie mir nunmehr die folgende Frage: Kö{\geminationn}ten Sie
                  Das Märchen\pwindex{Maerchen. Schauspiel in drei Aufzuegen@\emph{Das Märchen. Schauspiel in drei Aufzügen}|pw} nach \textsc{Halbe}\pwindex{Halbe, Max 04.10.1865 – 30.11.1944@\textsc{Halbe, Max} (04.10.1865 – 30.11.1944), \emph{Schriftsteller/Schriftstellerin}|pw}’s neuem Stück\pwindex{Amerikafahrer@\emph{Der Amerikafahrer}|pwv}, alſo etwa
               im Oktober oder November bringen, \textsc{resp.} kö{\geminationn}te ich darauf rechnen? – {\pb}Ich glaube annehmen zu können, dß es im \textsc{Lessingtheater}\orgindex{Lessing-Theater@Lessing-Theater|pw} im Oktober dranko{\geminationm}t. Falls Sie mein
               Ihnen gewidmetes Exemplar\pwindex{Maerchen. Schauspiel in drei Aufzuegen@\emph{Das Märchen. Schauspiel in drei Aufzügen}|pwv}
               verlegt haben, will ich Ihnen zur Durchſicht gern ein andres ſchicken. Daſs es ſich
               für Ihr Blatt\pwindex{Freie Buehne fuer den Entwickelungskampf der Zeit@\emph{Freie Bühne für den Entwickelungskampf der Zeit}|pwv}{ }{\pb}eignet, iſt kaum zu bezweifeln. –\pend
           
\pstart
           Hochachtungsvoll{\\[\baselineskip]}\spacefill\mbox{Dr. Arthur Schnitzler}\pend
           \leftskip=0em{}
\pstart
           \textsc{Ischl}\oindex{Bad Ischl@\textbf{Bad Ischl}, \emph{P.PPL}|pw}, 8. 7. 93.\pend
           
\pstart
           (Adreſſe nach wie vor \textsc{Wien I Grillparzerstr 7}\oindex{Grillparzerstrasse@\textbf{Grillparzerstraße}, \emph{R.ST}|pw}.)\pend
           
\pstart
           \raggedleft{}\spacefill\mbox{Sch}\pend
           \selectlanguage{ngerman}\endnumbering\briefempfaengerindex{Boelsche, Wilhelm@\textsc{Bölsche, Wilhelm}!zzzSchnitzler, Arthur@\emph{von Arthur Schnitzler}!1893-07-081@{8. 7. 1893}|)be}\mylabel{L00234h}  \normalsize

\doendnotes{C}
\bigskip
\vfill

\clearpage

\footnotesize

\lohead{\textsc{register}}

% Definiere theindex-Environment komplett neu ohne reledmac
\makeatletter
\renewenvironment{theindex}{%
  \section*{\indexname}%
  \setlength{\parindent}{0pt}%
  \setlength{\parskip}{0pt plus 0.3pt}%
  \let\item\@idxitem
}{%
  \clearpage
}
\makeatother

\IfFileExists{\jobname-pw.ind}{\input{\jobname-pw.ind}}{}

\end{document}

      