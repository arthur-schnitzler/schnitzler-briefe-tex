%% latex-korrekturansicht-vorspann.tex
%% Vorspann für die Korrekturansicht.
%% Lädt die gemeinsame Datei latex-vorspann.tex mit gesetztem Schalter.

\newif\ifkorrekturansicht
\korrekturansichttrue

\input{../tex-inputs/latex-vorspann}


\section[Max Burckhard: Widmungsexemplar Der Richter für Arthur Schnitzler, 16. 4. 1909]{L01838 Max Burckhard: Widmungsexemplar Der Richter für Arthur Schnitzler,
               16. 4. 1909}
\nopagebreak\mylabel{L01838v}
\rehead{ }\normalsize\beginnumbering\briefempfaengerindex{Schnitzler, Arthur@\textsc{Schnitzler, Arthur}!zzzBurckhard, Max Eugen@\emph{von Max Eugen Burckhard}!1909-04-161@{16. 4. 1909}|(be}
\toendnotes[C]{\smallbreak\pagebreak[2]}\Standort{DLA, G:Schnitzler, Arthur (Sammlung Heinrich Schnitzler).}
\physDesc{Widmung am Schmutztitel, 70 Zeichen
\newline{}Handschrift: schwarze Tinte, deutsche Kurrent
\newline{}Ordnung: bei der Enteignung des Exemplars 1938 mit Bleistift
                                 von unbekannter Hand als Dublette markiert: »\noindent{}= 467.288–B{ / }vol. IV.« }
\pstart
           \noindent{}{\pb}Arthur Schnitzler\pend
           
\pstart
           mit herzlicher Verehrung{\\[\baselineskip]}\spacefill\mbox{D\textsuperscript{r}Burckhard}\pend
           \leftskip=0em{}
\pstart
           München\oindex{Muenchen@\textbf{München}, \emph{P.PPLA}|pw}{ }16/4 09\pend
           {\vspace{1\baselineskip}}
\pstart
           \centering{}\textcolor{gray}{\textbf{Übersetzungsrecht, sowie alle anderen Rechte vorbehalten.}}\pend
           \selectlanguage{ngerman}\vspace{1em}{\vspace{1\baselineskip}}
\pstart
           \centering{}{\pb}\textcolor{gray}{\textbf{DER RICHTER\pwindex{Richter@\emph{Der Richter}|pw}}}\pend
           
\pstart
           \centering{}\textcolor{gray}{\textbf{VON}}\pend
           
\pstart
           \centering{}\textcolor{gray}{\textbf{\textsc{Dr.}{ }MAX BURCKHARD}}\pend
           {\vspace{1\baselineskip}}
\pstart
           \centering{}\textcolor{gray}{\textbf{BERLIN\oindex{Berlin@\textbf{Berlin}, \emph{P.PPLC}|pw}{ }1909}}\pend
           
\pstart
           \centering{}\textcolor{gray}{\textbf{PUTTKAMMER {\kaufmannsund}
                        MÜHLBRECHT\orgindex{Puttkammer und Muehlbrecht, Buchhandlung fuer Staats- und Rechtswissenschaft@Puttkammer {\kaufmannsund}  Mühlbrecht, Buchhandlung für Staats- und Rechtswissenschaft|pw}}}\pend
           
\pstart
           \centering{}\textcolor{gray}{\textbf{Buchhandlung für Staats- u. Rechtswissenschaft.}}\pend
           \selectlanguage{ngerman}\endnumbering\briefempfaengerindex{Schnitzler, Arthur@\textsc{Schnitzler, Arthur}!zzzBurckhard, Max Eugen@\emph{von Max Eugen Burckhard}!1909-04-161@{16. 4. 1909}|)be}\mylabel{L01838h}  \normalsize

\doendnotes{C}
\bigskip
\vfill

\clearpage

\footnotesize

\lohead{\textsc{register}}

% Definiere theindex-Environment komplett neu ohne reledmac
\makeatletter
\renewenvironment{theindex}{%
  \section*{\indexname}%
  \setlength{\parindent}{0pt}%
  \setlength{\parskip}{0pt plus 0.3pt}%
  \let\item\@idxitem
}{%
  \clearpage
}
\makeatother

\IfFileExists{\jobname-pw.ind}{\input{\jobname-pw.ind}}{}

\end{document}

      