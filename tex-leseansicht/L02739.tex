%% latex-leseansicht-vorspann.tex
%% Vorspann für die Leseansicht.
%% Lädt die gemeinsame Datei latex-vorspann.tex mit nicht gesetztem Schalter.

\newif\ifkorrekturansicht
\korrekturansichtfalse

\input{../tex-inputs/latex-vorspann}


\section[Paul Goldmann an Arthur Schnitzler, 6. 7. {[}1895{]}]{L02739 Paul Goldmann an Arthur Schnitzler, 6. 7. [1895]}
\nopagebreak\mylabel{L02739v}
\rehead{ }\normalsize\beginnumbering\briefempfaengerindex{Schnitzler, Arthur@\textsc{Schnitzler, Arthur}!zzzGoldmann, Paul@\emph{von Paul Goldmann}!1895-07-061@{6. 7. [1895]}|(be}
\toendnotes[C]{\smallbreak\pagebreak[2]}
\correspDesc{Versand  durch Paul Goldmann am 6. 7. [1895] in Paris
\newline{}Erhalt  durch Arthur Schnitzler im Zeitraum [7. 7. 1895
                  – 11. 7. 1895?] in Wien}\toendnotes[C]{\smallbreak}
\Standort{DLA, A:Schnitzler, HS.NZ85.1.3165.}
\physDesc{Brief, 2 Blätter, 8 Seiten, 2560 Zeichen
\newline{}Handschrift: schwarze Tinte, deutsche Kurrent
\newline{}Schnitzler: 1) mit Bleistift das Jahr »95« vermerkt  2) mit rotem Buntstift eine Unterstreichung}\toendnotes[C]{\smallbreak}
\pstart
           {\pb}\textcolor{gray}{\textbf{\textbf{Frankfurter Zeitung\orgindex{Frankfurter Zeitung@Frankfurter Zeitung|pw}}}}\pend
           
\pstart
           \textcolor{gray}{\textbf{(\begin{otherlanguage}{french}Gazette de Francfort\end{otherlanguage}\orgindex{Frankfurter Zeitung@Frankfurter Zeitung|pw}).}}\pend
           
\pstart
           \textcolor{gray}{\textbf{\textbf{\begin{otherlanguage}{french}Fondateur M. L.
                              Sonnemann\pwindex{Sonnemann, Leopold 29.\,10.\,1831 Höchberg – 30.\,10.\,1909 Frankfurt am Main@\textsc{Sonnemann, Leopold} (29.\,10.\,1831 Höchberg – 30.\,10.\,1909 Frankfurt am Main), \emph{Journalist, Herausgeber}|pw}\end{otherlanguage}.}}}\pend
           
\pstart
           \begin{otherlanguage}{french}\textcolor{gray}{\textbf{Journal politique, financier,}}\end{otherlanguage}\pend
           
\pstart
           \begin{otherlanguage}{french}\textcolor{gray}{\textbf{commercial et littéraire.}}\end{otherlanguage}\pend
           
\pstart
           \begin{otherlanguage}{french}\textcolor{gray}{\textbf{\textbf{Paraissant trois fois par jour.}}}\end{otherlanguage}\hfill \textsc{Paris\oindex{Paris@\textbf{Paris}, \emph{Hauptstadt}|pw}}, 6. Juli.\pend
           
\pstart
           \begin{otherlanguage}{french}\textcolor{gray}{\textbf{\textbf{Bureau à Paris\oindex{Paris@\textbf{Paris}, \emph{Hauptstadt}|pw}}}}\end{otherlanguage}\pend
           
\pstart
           \begin{otherlanguage}{french}\textcolor{gray}{\textbf{\textbf{24. Rue Feydeau\oindex{rue Feydeau@\textbf{rue Feydeau}, \emph{Straße}|pw}.}}}\end{otherlanguage}\pend
           
\pstart\center{}Mein lieber Freund,\pend\vspace{0.5em}
\pstart
           Ich habe Dir nichts Neues zu{ }ſagen, aber ich{ }ſchreib’ Dir, um Dir zu{ }ſagen, daß ich
               mich von Herzen freue, Dich \label{K_L02739-1v}\edtext{unterwegs}{\lemma{\textnormal{\emph{unterwegs}}}\Cendnote{\textnormal{Am 3. 7. 1895 trat Schnitzler seinen Sommerurlaub an, der ihn
                  zuerst für vier Tage nach Prag\oindex{Prag@\textbf{Prag}, \emph{Land}|pwk} führte. Es
                  folgten Karlsbad\oindex{Karlsbad@\textbf{Karlsbad}|pwk}, Marienbad\oindex{Marienbad@\textbf{Marienbad}|pwk}, Franzensbad\oindex{Franzensbad@\textbf{Franzensbad}|pwk}
                  und Nürnberg\oindex{Nürnberg@\textbf{Nürnberg}|pwk}. Ab 15. 7. 1895 war er bis
                     10. 8. 1895 in
                     Bad Ischl\oindex{Bad Ischl@\textbf{Bad Ischl}|pwk}.}}}\label{K_L02739-1} zu wiſſen, und daß ich
               Dich mit meinen beſten Wünſchen begleite.\pend
           
\pstart
           Prag\oindex{Prag@\textbf{Prag}, \emph{Land}|pw} muß{ }ſchön{ }ſein. Viel alte Steine und blonde
               junge Mädchen dazwiſchen {\pb}und ein rauſchender Fluß\oindex{Moldau@\textbf{Moldau}, \emph{Fluss}|pwv} und der dreißigjährige
               Krieg. So{ }ſtell’ ich mirs vor. Was Du von dem alten \label{K_L02739-2v}\edtext{Friedhof\oindex{Alter Jüdischer Friedhof@\textbf{Alter Jüdischer Friedhof}, \emph{Friedhof}|pwv}}{\lemma{\textnormal{\emph{Friedhof}}}\Cendnote{\textnormal{Am 5. 7. 1895 besuchte Schnitzler mit Marie
                     Reinhard\pwindex{Reinhard, Marie 13.\,3.\,1871 Wien – 18.\,3.\,1899 ebd.@\textsc{Reinhard, Marie} (13.\,3.\,1871 Wien – 18.\,3.\,1899 ebd.), \emph{Gesangspädagogin}|pwk} den jüdischen Friedhof\oindex{Alter Jüdischer Friedhof@\textbf{Alter Jüdischer Friedhof}, \emph{Friedhof}|pwk}, der
                  seit ein paar Jahren nicht mehr in aktiver Verwendung war.}}}\label{K_L02739-2}{ }ſchreibſt, hat
               mir beinahe Heimweh danach gemacht. So iſt der Tod anheimelnd, – wenn man nämlich
               oben{ }ſteht zwiſchen den verſinkenden Steinen und dem grünen Gras, in Sommerluft und
               Frieden. Nur iſt das nicht der eigentliche Friedhof. Der eigentliche {\pb}Friedhof – das wäre, wenn man ihn von unten anſieht.
               Da muß er{ }ſchauderhaft{ }ſein, aber das \strikeout{i\textcolor{gray}{n}} iſt auch des Todes wahres Geſicht. Hierher gehört ein Capitel über die
               Oberflächlichkeit der menſchlichen Todes-Anſchauung, welche die Friedhöfe von oben
               betrachtet{ }ſtatt von unten\substVorne{}\textsuperscript{.}\substDazwischen{},\substHinten{} welche{ }ſich unter die \strikeout{g\textcolor{gray}{er}} rauſchenden Bäume der Friedhöfe{ }ſtellt und{ }ſagt: {\pb}»Welch’{ }ſanfte Ruhe!« Nein, es iſt nicht die Ruhe,
               es iſt das Vermodern.\pend
           
\pstart
           Dabei vergeſſe ich, daß ich zum Autor von »Sterben\pwindex{Schnitzler, Arthur 15.\,5.\,1862 Wien – 21.\,10.\,1931 ebd.@\textsc{Schnitzler, Arthur} (15.\,5.\,1862 Wien – 21.\,10.\,1931 ebd.), \emph{Schriftsteller, Mediziner}!Sterben. Novelle@\strich\emph{Sterben. Novelle}|pw}«{ }ſpreche.\pend
           
\pstart
           \strikeout{Ac\textcolor{gray}{h}} Oh, ich möchte gern \strikeout{hinunte\textcolor{gray}{r}} hinunter, unter die Erde. Ich kan\substVorne{}\textsuperscript{s}\substDazwischen{}ns\substHinten{} wirklich nicht mehr. Seit einigen Tagen{ }ſehe ich wieder mit erbarmungsloſer
               Klarheit, was ich Alles verfehlt, was {\pb}ich nie
               erreichen werde. Ich{ }ſehe mich mit energieloſem Schritte durch die Straßen gehen, und
               aus den Spiegeln der Läden blickt mir mein Geſicht entgegen und ruft: »\label{K_L02739-3v}\edtext{\begin{otherlanguage}{french}\textsc{Un rat\textcolor{gray}{e}}\end{otherlanguage}!}{\lemma{\textnormal{\emph{Un rate!}}}\Cendnote{\textnormal{raté, französisch: Versager. Der Strich hinter dem
                  »e« dürfte kein \begin{otherlanguage}{french}accent aigu\end{otherlanguage}, sondern Teil eines
                  Rufezeichens sein. Dafür spricht die Ähnlichkeit des Strichs zu den
                  darunterstehenden und dass Goldmann\pwindex{Goldmann, Paul 31.\,1.\,1865 Breslau – 25.\,9.\,1935 Wien@\textsc{Goldmann, Paul} (31.\,1.\,1865 Breslau – 25.\,9.\,1935 Wien), \emph{Schriftsteller, Journalist}|pwk}
                     »ruft« schrieb.}}}\label{K_L02739-3}.« Haha, mit 30 Jahren!\pend
           
\pstart
           Sterben, oh ja! Aber glücklich leben wäre doch noch viel{ }ſchöner, und {\pb}ich glaube immer noch daran, obwohl ich es mit
               unbeweisbarer Logik darthun kann, daß ich zu{ }ſchwach bin, mir irgend eines der großen
               Lebensgüter zu erkämpfen.\pend
           
\pstart
           Das iſt{ }ſo ehrlich, was ich Dir da{ }ſchreibe,{ }ſo ohne Poſe, weiß Gott!\pend
           
\pstart
           \textsc{Becque\pwindex{Becque, Henry 9.\,4.\,1837 Paris – 12.\,5.\,1899 Neuilly-sur-Seine@\textsc{Becque, Henry} (9.\,4.\,1837 Paris – 12.\,5.\,1899 Neuilly-sur-Seine), \emph{Schriftsteller, Dramatiker}|pw}} hat mir verſprochen, er will »\textsc{Mourir\pwindex{Schnitzler, Arthur 15.\,5.\,1862 Wien – 21.\,10.\,1931 ebd.@\textsc{Schnitzler, Arthur} (15.\,5.\,1862 Wien – 21.\,10.\,1931 ebd.), \emph{Schriftsteller, Mediziner}!Mourir. Roman@\strich\emph{Mourir. Roman}|pwv}}« leſen. Wird ers thun? {\dots} Ich {\pb}ſchicks ihm Montag.
               Dann könnte man mit ihm die Verleger-Frage berathen.\pend
           
\pstart
           Wer die betreffende \label{K_L02739-4v}\edtext{Frau\pwindex{Candiani, Regina @\textsc{Candiani, Regina}, \emph{Schriftstellerin, Übersetzerin}|pwv}}{\lemma{\textnormal{\emph{Frau}}}\Cendnote{\textnormal{Am 2. 7. 1895 notierte Schnitzler im \emph{Tagebuch}\pwindex{Schnitzler, Arthur 15.\,5.\,1862 Wien – 21.\,10.\,1931 ebd.@\textsc{Schnitzler, Arthur} (15.\,5.\,1862 Wien – 21.\,10.\,1931 ebd.), \emph{Schriftsteller, Mediziner}!Tagebuch@\strich\emph{Tagebuch}|pwk}: »Uebersetzungsantrag Sterben\pwindex{Schnitzler, Arthur 15.\,5.\,1862 Wien – 21.\,10.\,1931 ebd.@\textsc{Schnitzler, Arthur} (15.\,5.\,1862 Wien – 21.\,10.\,1931 ebd.), \emph{Schriftsteller, Mediziner}!Sterben. Novelle@\strich\emph{Sterben. Novelle}|pw} und andre Frau Candiani\pwindex{Candiani, Regina @\textsc{Candiani, Regina}, \emph{Schriftstellerin, Übersetzerin}|pw} –«. Regine
                     Candiani\pwindex{Candiani, Regina @\textsc{Candiani, Regina}, \emph{Schriftstellerin, Übersetzerin}|pwk} war eine russland\oindex{Russland@\textbf{Russland}|pwk}stämmige
                  Übersetzerin, die seit 1875 in Frankreich\oindex{Frankreich@\textbf{Frankreich}|pwk} lebte und Tolstoi\pwindex{Tolstoi, Lew Nikolajewitsch 9.\,9.\,1828 Yasnaya Polyana – 20.\,11.\,1910 Lev Tolstoy@\textsc{Tolstoi, Lew Nikolajewitsch} (9.\,9.\,1828 Yasnaya Polyana – 20.\,11.\,1910 Lev Tolstoy), \emph{Schriftsteller}|pwk} und
                     Turgenjew\pwindex{Turgenev, Ivan Sergeevič 9.\,11.\,1818 Orjol – 3.\,9.\,1883 Bougival@\textsc{Turgenev, Ivan Sergeevič} (9.\,11.\,1818 Orjol – 3.\,9.\,1883 Bougival), \emph{Schriftsteller}|pwk} übersetzte. Übersetzungen von
                     Schnitzler sind nicht nachgewiesen. Für
                  die Zeit zwischen 1902 und 1903
                  liegen Durchschläge von vier Korrespondenzstücken von Schnitzler an sie  im \emph{Deutschen Literaturarchiv Marbach},
                  HS.19851.507.}}}\label{K_L02739-4} iſt, möchte ich Dir gern{ }ſagen\substVorne{}\textsuperscript{.}\substDazwischen{},\substHinten{} könnt’ ich nur ihren Namen leſen. Bitte{ }ſchreib’ mir ihn noch einmal recht
               deutlich auf. Von was iſt{ }ſie Sekretär\pwindex{Candiani, Regina @\textsc{Candiani, Regina}, \emph{Schriftstellerin, Übersetzerin}|pwv}? In welcher Stadt lebt{ }ſie? Daß Du Dich zu nichts verpflichtet
               haſt, iſt gut. Unter keinen Umſtänden {\pb}darfſt Du
               Deine übrigen Werke vergeben. Sieh’ Dir auch an, ob die Überſetzungen ’was taugen
               oder{ }ſchick’{ }ſie mir. Die Frauenzimmer thun in der Regel das Übersetzen ab, wie das
               Strümpfeflicken.\pend
           
\pstart
           Grüß’ Dich Gott, mein lieber Freund. Mit wem immer Du biſt, grüß’ ihn von mir. Ich
               wünſche Dir von Herzen Glück und Sonnenſchein auf den Weg.\pend
           
\pstart
           Dein treuer {\\[\baselineskip]}\spacefill\mbox{Paul Goldmann}\pend
           \leftskip=0em{}\selectlanguage{ngerman}\endnumbering\briefempfaengerindex{Schnitzler, Arthur@\textsc{Schnitzler, Arthur}!zzzGoldmann, Paul@\emph{von Paul Goldmann}!1895-07-061@{6. 7. [1895]}|)be}\mylabel{L02739h}  \newcommand{\dateiname}{L02739}\newcommand{\titel}{Paul Goldmann an Arthur Schnitzler, 6. 7. [1895]}\newcommand{\editorInnen}{Martin Anton Müller und Laura Untner}%% latex-leseansicht-abspann.tex
%% Abspann für die Leseansicht.
%% Der Schalter \ifkorrekturansicht ist bereits durch den Vorspann gesetzt.

%% latex-abspann.tex
%% Gemeinsamer Abspann für Korrekturansicht und Leseansicht.
%% Setzt den Schalter \ifkorrekturansicht voraus (gesetzt in den
%% einbindenden Dateien latex-korrekturansicht-abspann.tex bzw.
%% latex-leseansicht-abspann.tex).
%% ---------------------------------------------------------------

\normalsize

% Das esempio-Environment wird nur in der Leseansicht benötigt
\ifkorrekturansicht\else
\newenvironment{esempio}[3]%
{
    \vspace{1.5ex}
    \rlap{\underline{#1}}
    \par
    \setlength{\parindent}{0cm}
    \nopagebreak
    \leftskip=#2cm
    \rightskip=#3cm
}
{
    \par
}
\fi

\doendnotes{C}
\bigskip
\vfill

\clearpage

\footnotesize

\ifkorrekturansicht
  \lohead{\textsc{register}}
\fi

% theindex-Environment neu definieren ohne reledmac
\makeatletter
\renewenvironment{theindex}{%
  \ifkorrekturansicht
    \section*{\indexname}%
  \else
    \subsubsection*{Index der erwähnten Entitäten}%
  \fi
  \setlength{\parindent}{0pt}%
  \setlength{\parskip}{0pt plus 0.3pt}%
  \let\item\@idxitem
}{%
  \ifkorrekturansicht\clearpage\fi
}
\makeatother

\IfFileExists{\jobname-pw.ind}{\input{\jobname-pw.ind}}{}

% Quellenangabe nur in der Leseansicht
\ifkorrekturansicht\else
% Fallback-Definitionen, falls die .tex-Datei \titel etc. nicht gesetzt hat
\providecommand{\titel}{}
\providecommand{\editorInnen}{}
\providecommand{\dateiname}{\jobname}

\vspace{3cm}

\vfill

\footnotesize
\textsc{Quelle}: \titel. Herausgegeben von {\editorInnen}. In: \emph{Arthur Schnitzler: Briefwechsel mit Autorinnen und Autoren}.
 Digitale Edition, https://schnitzler-briefe.acdh.oeaw.ac.at/{\dateiname}.html (Stand \today)
\fi

\end{document}


