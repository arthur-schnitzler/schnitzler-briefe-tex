%% latex-korrekturansicht-vorspann.tex
%% Vorspann für die Korrekturansicht.
%% Lädt die gemeinsame Datei latex-vorspann.tex mit gesetztem Schalter.

\newif\ifkorrekturansicht
\korrekturansichttrue

\input{../tex-inputs/latex-vorspann}


\section[Paul Goldmann an Arthur Schnitzler, 6. 7. {[}1895{]}]{L02739 Paul Goldmann an Arthur Schnitzler, 6. 7. {[}1895{]}}
\nopagebreak\mylabel{L02739v}
\rehead{ }\normalsize\beginnumbering\briefempfaengerindex{Schnitzler, Arthur@\textsc{Schnitzler, Arthur}!zzzGoldmann, Paul@\emph{von Paul Goldmann}!1895-07-061@{6. 7. {[}1895{]}}|(be}
\toendnotes[C]{\smallbreak\pagebreak[2]}\Standort{DLA, A:Schnitzler, HS.NZ85.1.3165.}
\physDesc{Brief, 2 Blätter, 8 Seiten, 2560 Zeichen
\newline{}Handschrift: schwarze Tinte, deutsche Kurrent
\newline{}Schnitzler: 1) mit Bleistift das Jahr »95« vermerkt  2) mit rotem Buntstift eine Unterstreichung}\toendnotes[C]{\smallbreak}
\pstart
           {\pb}\textcolor{gray}{\textbf{\textbf{Frankfurter Zeitung\orgindex{Frankfurter Zeitung@Frankfurter Zeitung|pw}}}}\pend
           
\pstart
           \textcolor{gray}{\textbf{(\begin{otherlanguage}{french}Gazette de Francfort\end{otherlanguage}\orgindex{Frankfurter Zeitung@Frankfurter Zeitung|pw}). }}\pend
           
\pstart
           \textcolor{gray}{\textbf{\textbf{\begin{otherlanguage}{french}Fondateur M. L.
                              Sonnemann\pwindex{Sonnemann, Leopold 1831-10-29 – 1909-10-30@\textsc{Sonnemann, Leopold} (1831-10-29 – 1909-10-30), \emph{Journalist/Journalistin, Herausgeber/Herausgeberin}|pw}\end{otherlanguage}.}}}\pend
           
\pstart
           \begin{otherlanguage}{french}\textcolor{gray}{\textbf{Journal politique, financier,}}\end{otherlanguage}\pend
           
\pstart
           \begin{otherlanguage}{french}\textcolor{gray}{\textbf{commercial et littéraire.}}\end{otherlanguage}\pend
           
\pstart
           \begin{otherlanguage}{french}\textcolor{gray}{\textbf{\textbf{Paraissant trois fois par jour.}}}\end{otherlanguage}\hfill \textsc{Paris\oindex{Paris@\textbf{Paris}, \emph{P.PPLC}|pw}}, 6. Juli.\pend
           
\pstart
           \begin{otherlanguage}{french}\textcolor{gray}{\textbf{\textbf{Bureau à Paris\oindex{Paris@\textbf{Paris}, \emph{P.PPLC}|pw}}}}\end{otherlanguage}\pend
           
\pstart
           \begin{otherlanguage}{french}\textcolor{gray}{\textbf{\textbf{24. Rue Feydeau\oindex{rue Feydeau@\textbf{rue Feydeau}, \emph{Straße (K.STR)}|pw}.}}}\end{otherlanguage}\pend
           
\pstart\center{}Mein lieber Freund,\pend\vspace{0.5em}
\pstart
           Ich habe Dir nichts Neues zu ſagen, aber ich ſchreib’ Dir, um Dir zu ſagen, daß ich
               mich von Herzen freue, Dich \label{K_L02739-1v}\edtext{unterwegs}{\lemma{\textnormal{\emph{unterwegs}}}\Cendnote{\textnormal{Am 3. 7. 1895 trat Schnitzler seinen Sommerurlaub an, der ihn
                  zuerst für vier Tage nach Prag\oindex{Prag@\textbf{Prag}, \emph{A.ADM1}|pwk} führte. Es
                  folgten Karlsbad\oindex{Karlsbad@\textbf{Karlsbad}, \emph{P.PPLA}|pwk}, Marienbad\oindex{Marienbad@\textbf{Marienbad}, \emph{P.PPL}|pwk}, Franzensbad\oindex{Franzensbad@\textbf{Franzensbad}, \emph{P.PPL}|pwk}
                  und Nürnberg\oindex{Nuernberg@\textbf{Nürnberg}, \emph{P.PPL}|pwk}. Ab 15. 7. 1895 war er bis
                     10. 8. 1895 in
                     Bad Ischl\oindex{Bad Ischl@\textbf{Bad Ischl}, \emph{P.PPL}|pwk}.}}}\label{K_L02739-1} zu wiſſen, und daß ich
               Dich mit meinen beſten Wünſchen begleite.\pend
           
\pstart
           Prag\oindex{Prag@\textbf{Prag}, \emph{A.ADM1}|pw} muß ſchön ſein. Viel alte Steine und blonde
               junge Mädchen dazwiſchen {\pb}und ein rauſchender Fluß\oindex{Moldau@\textbf{Moldau}, \emph{Fluss (N.FLS)}|pwv} und der dreißigjährige
               Krieg. So ſtell’ ich mirs vor. Was Du von dem alten \label{K_L02739-2v}\edtext{Friedhof\oindex{Alter Juedischer Friedhof@\textbf{Alter Jüdischer Friedhof}, \emph{Friedhof (K.FRH)}|pwv}}{\lemma{\textnormal{\emph{Friedhof}}}\Cendnote{\textnormal{Am 5. 7. 1895 besuchte Schnitzler mit Marie
                     Reinhard\pwindex{Reinhard, Marie 1871-03-13 – 1899-03-18@\textsc{Reinhard, Marie} (1871-03-13 – 1899-03-18), \emph{Gesangspädagoge/Gesangspädagogin}|pwk} den jüdischen Friedhof\oindex{Alter Juedischer Friedhof@\textbf{Alter Jüdischer Friedhof}, \emph{Friedhof (K.FRH)}|pwk}, der
                  seit ein paar Jahren nicht mehr in aktiver Verwendung war.}}}\label{K_L02739-2} ſchreibſt, hat
               mir beinahe Heimweh danach gemacht. So iſt der Tod anheimelnd, – wenn man nämlich
               oben ſteht zwiſchen den verſinkenden Steinen und dem grünen Gras, in Sommerluft und
               Frieden. Nur iſt das nicht der eigentliche Friedhof. Der eigentliche {\pb}Friedhof – das wäre, wenn man ihn von unten anſieht.
               Da muß er ſchauderhaft ſein, aber das \strikeout{i\textcolor{gray}{n}} iſt auch des Todes wahres Geſicht. Hierher gehört ein Capitel über die
               Oberflächlichkeit der menſchlichen Todes-Anſchauung, welche die Friedhöfe von oben
               betrachtet ſtatt von unten\substVorne{}\textsuperscript{.}\substDazwischen{},\substHinten{} welche ſich unter die \strikeout{g\textcolor{gray}{er}} rauſchenden Bäume der Friedhöfe ſtellt und ſagt: {\pb}»Welch’ ſanfte Ruhe!« Nein, es iſt nicht die Ruhe,
               es iſt das Vermodern. \pend
           
\pstart
           Dabei vergeſſe ich, daß ich zum Autor von »Sterben\pwindex{Sterben. Novelle@\emph{Sterben. Novelle}|pw}« ſpreche.\pend
           
\pstart
           \strikeout{Ac\textcolor{gray}{h}} Oh, ich möchte gern \strikeout{hinunte\textcolor{gray}{r}} hinunter, unter die Erde. Ich kan\substVorne{}\textsuperscript{s}\substDazwischen{}ns\substHinten{} wirklich nicht mehr. Seit einigen Tagen ſehe ich wieder mit erbarmungsloſer
               Klarheit, was ich Alles verfehlt, was {\pb}ich nie
               erreichen werde. Ich ſehe mich mit energieloſem Schritte durch die Straßen gehen, und
               aus den Spiegeln der Läden blickt mir mein Geſicht entgegen und ruft: »\label{K_L02739-3v}\edtext{\begin{otherlanguage}{french}\textsc{Un rat\textcolor{gray}{e}}\end{otherlanguage}!}{\lemma{\textnormal{\emph{Un rate!}}}\Cendnote{\textnormal{raté, französisch: Versager. Der Strich hinter dem
                  »e« dürfte kein \begin{otherlanguage}{french}accent aigu\end{otherlanguage}, sondern Teil eines
                  Rufezeichens sein. Dafür spricht die Ähnlichkeit des Strichs zu den
                  darunterstehenden und dass Goldmann\pwindex{Goldmann, Paul 31.01.1865 – 25.09.1935@\textsc{Goldmann, Paul} (31.01.1865 – 25.09.1935), \emph{Schriftsteller/Schriftstellerin, Journalist/Journalistin}|pwk}
                     »ruft« schrieb.}}}\label{K_L02739-3}.« Haha, mit 30 Jahren!\pend
           
\pstart
           Sterben, oh ja! Aber glücklich leben wäre doch noch viel ſchöner, und {\pb}ich glaube immer noch daran, obwohl ich es mit
               unbeweisbarer Logik darthun kann, daß ich zu ſchwach bin, mir irgend eines der großen
               Lebensgüter zu erkämpfen.\pend
           
\pstart
           Das iſt ſo ehrlich, was ich Dir da ſchreibe, ſo ohne Poſe, weiß Gott!\pend
           
\pstart
           \textsc{Becque\pwindex{Becque, Henry 1837-04-09 – 1899-05-12@\textsc{Becque, Henry} (1837-04-09 – 1899-05-12), \emph{Schriftsteller/Schriftstellerin, Dramatiker/Dramatikerin}|pw}} hat mir verſprochen, er will »\textsc{Mourir\pwindex{Mourir. Roman@\emph{Mourir. Roman}|pwv}}« leſen. Wird ers thun? {\dots} Ich {\pb}ſchicks ihm Montag.
               Dann könnte man mit ihm die Verleger-Frage berathen.\pend
           
\pstart
           Wer die betreffende \label{K_L02739-4v}\edtext{Frau\pwindex{Candiani, Regina @\textsc{Candiani, Regina}, \emph{Schriftsteller/Schriftstellerin, Übersetzer/Übersetzerin}|pwv}}{\lemma{\textnormal{\emph{Frau}}}\Cendnote{\textnormal{Am 2. 7. 1895 notierte Schnitzler im \emph{Tagebuch}\pwindex{Tagebuch@\emph{Tagebuch}|pwk}: »Uebersetzungsantrag Sterben\pwindex{Sterben. Novelle@\emph{Sterben. Novelle}|pw} und andre Frau Candiani\pwindex{Candiani, Regina @\textsc{Candiani, Regina}, \emph{Schriftsteller/Schriftstellerin, Übersetzer/Übersetzerin}|pw} –«. Regine
                     Candiani\pwindex{Candiani, Regina @\textsc{Candiani, Regina}, \emph{Schriftsteller/Schriftstellerin, Übersetzer/Übersetzerin}|pwk} war eine russland\oindex{Russland@\textbf{Russland}, \emph{A.PCLI}|pwk}stämmige
                  Übersetzerin, die seit 1875 in Frankreich\oindex{Frankreich@\textbf{Frankreich}, \emph{A.PCLI}|pwk} lebte und Tolstoi\pwindex{Tolstoi, Leo N. von 09.09.1828 – 20.11.1910@\textsc{Tolstoi, Leo N. von} (09.09.1828 – 20.11.1910), \emph{Schriftsteller/Schriftstellerin, Schriftsteller/Schriftstellerin, Krimiautor/Krimiautorin}|pwk} und
                     Turgenjew\pwindex{Turgenev, Ivan Sergeevic 09.11.1818 – 03.09.1883@\textsc{Turgenev, Ivan Sergeevič} (09.11.1818 – 03.09.1883), \emph{Schriftsteller/Schriftstellerin}|pwk} übersetzte. Übersetzungen von
                     Schnitzler sind nicht nachgewiesen. Für
                  die Zeit zwischen 1902 und 1903
                  liegen Durchschläge von vier Korrespondenzstücken von Schnitzler an sie  im \emph{Deutschen Literaturarchiv Marbach},
                  HS.19851.507.}}}\label{K_L02739-4} iſt, möchte ich Dir gern ſagen\substVorne{}\textsuperscript{.}\substDazwischen{},\substHinten{} könnt’ ich nur ihren Namen leſen. Bitte ſchreib’ mir ihn noch einmal recht
               deutlich auf. Von was iſt ſie Sekretär\pwindex{Candiani, Regina @\textsc{Candiani, Regina}, \emph{Schriftsteller/Schriftstellerin, Übersetzer/Übersetzerin}|pwv}? In welcher Stadt lebt ſie? Daß Du Dich zu nichts verpflichtet
               haſt, iſt gut. Unter keinen Umſtänden {\pb}darfſt Du
               Deine übrigen Werke vergeben. Sieh’ Dir auch an, ob die Überſetzungen ’was taugen
               oder ſchick’ ſie mir. Die Frauenzimmer thun in der Regel das Übersetzen ab, wie das
               Strümpfeflicken.\pend
           
\pstart
           Grüß’ Dich Gott, mein lieber Freund. Mit wem immer Du biſt, grüß’ ihn von mir. Ich
               wünſche Dir von Herzen Glück und Sonnenſchein auf den Weg.\pend
           
\pstart
           Dein treuer {\\[\baselineskip]}\spacefill\mbox{Paul Goldmann}\pend
           \leftskip=0em{}\selectlanguage{ngerman}\endnumbering\briefempfaengerindex{Schnitzler, Arthur@\textsc{Schnitzler, Arthur}!zzzGoldmann, Paul@\emph{von Paul Goldmann}!1895-07-061@{6. 7. {[}1895{]}}|)be}\mylabel{L02739h}  \normalsize

\doendnotes{C}
\bigskip
\vfill

\clearpage

\footnotesize

\lohead{\textsc{register}}

% Definiere theindex-Environment komplett neu ohne reledmac
\makeatletter
\renewenvironment{theindex}{%
  \section*{\indexname}%
  \setlength{\parindent}{0pt}%
  \setlength{\parskip}{0pt plus 0.3pt}%
  \let\item\@idxitem
}{%
  \clearpage
}
\makeatother

\IfFileExists{\jobname-pw.ind}{\input{\jobname-pw.ind}}{}

\end{document}

      