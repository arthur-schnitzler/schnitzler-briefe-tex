%% latex-leseansicht-vorspann.tex
%% Vorspann für die Leseansicht.
%% Lädt die gemeinsame Datei latex-vorspann.tex mit nicht gesetztem Schalter.

\newif\ifkorrekturansicht
\korrekturansichtfalse

\input{../tex-inputs/latex-vorspann}


         
         \renewcommand{\erwaehntePersonen}{Personen: Paul Goldmann, Theodor Herzl, Julie Herzl, Robert Hirschfeld, Friedrich Kapper, Fedor Mamroth, Rudolf Pick, Gustav Pick, Alfred Pick, Franz Rettinger, Olga Waissnix, Carl Waissnix}
         \renewcommand{\erwaehnteInstitutionen}{Institutionen: An der schönen blauen Donau, Josef Eberle Stein-, Buch und Musikaliendruckerei}
         \renewcommand{\erwaehnteOrte}{Orte: Berggasse, Donau, Donaukanal, Pörtschach, Reichenau an der Rax, Seidengasse, Thalhof, Wien}
         \renewcommand{\erwaehnteWerke}{Werke: Jugend in Wien, Tagebuch}
               \section[Paul Goldmann an Arthur Schnitzler, 11. 8. 1890]{ Paul Goldmann an Arthur Schnitzler, 11. 8. 1890}\nopagebreak\mylabel{v}\rehead{ }\begin{ledgroupsized}[t]{13cm}\normalsize\beginnumbering\briefempfaengerindex{Schnitzler, Arthur@\textsc{Schnitzler, Arthur}!zzzGoldmann, Paul@\emph{von Paul Goldmann}!1890-08-111@{11. 8. 1890}|(be} \toendnotes[C]{\smallbreak\pagebreak[2]} \Standort{DLA, A:Schnitzler, HS.NZ85.1.3162.}
\physDesc{Brief, 2 Blätter, 7 Seiten, 3565 Zeichen
\newline{}Handschrift: schwarze Tinte, deutsche Kurrent
\newline{}Schnitzler: mit rotem Buntstift eine Unterstreichung }\toendnotes[C]{\smallbreak}\pstart
           \noindent{}\centering{}{\pb}\textcolor{gray}{\textbf{\textbf{Adminiſtration: VII.
                           Seidengaſſe 7\oindex{Seidengasse@\textbf{Seidengasse}|pw}} (Jos. Eberle {\kaufmannsund} Co.\orgindex{Josef Eberle Stein-, Buch und Musikaliendruckerei@Josef Eberle Stein-, Buch und Musikaliendruckerei|pw})}}\pend
           \pstart
           \noindent{}\centering{}\textcolor{gray}{\textbf{An der Schönen Blauen Donau\orgindex{der schoenen blauen Donau@An der schönen blauen Donau|pw}}}\pend
           \pstart
           \noindent{}\centering{}\textcolor{gray}{\textbf{Chef-Redacteur: Dr. F.
                        Mamroth\pwindex{Mamroth, Fedor 21.02.1851 – 25.06.1907@\textsc{Mamroth, Fedor} (21.02.1851 – 25.06.1907), \emph{Journalist, Kritiker}|pw}. – Redaction: IX.,
                        Berggaſſe 31\oindex{Berggasse@\textbf{Berggasse}|pw}.}}\pend
           \pstart
           \raggedleft{}\textsc{Pörtschach\oindex{Poertschach@\textbf{Pörtschach}|pw}}{ }\textcolor{gray}{\textbf{\strikeout{Wien\oindex{Wien@\textbf{Wien}|pw}}, den}}{ }11. August \textcolor{gray}{\textbf{18}}90.\pend
           \pstart\center{}Lieber Arthur!\pend\pstart
           Du haſt Recht gehabt: ich bin von dieſer \label{K_L02648-1v}\edtext{Frau\pwindex{Waissnix, Olga 03.11.1862 – 04.11.1897@\textsc{Waissnix, Olga} (03.11.1862 – 04.11.1897), \emph{Hotelière}|pwv}}{\lemma{\textnormal{\emph{Frau}}}\Cendnote{\textnormal{Mit Olga Waissnix\pwindex{Waissnix, Olga 03.11.1862 – 04.11.1897@\textsc{Waissnix, Olga} (03.11.1862 – 04.11.1897), \emph{Hotelière}|pwk} verband Schnitzler\pwindex{Schnitzler, Arthur 15.05.1862 – 21.10.1931@\textsc{Schnitzler, Arthur} (15.05.1862 – 21.10.1931), \emph{Schriftsteller, Mediziner}|pwk} in
                  den Jahren nach 1886 eine für ihn bedeutsame Liebesbeziehung. Sie war
                  die Wirtin\pwindex{Waissnix, Olga 03.11.1862 – 04.11.1897@\textsc{Waissnix, Olga} (03.11.1862 – 04.11.1897), \emph{Hotelière}|pwkv} des Thalhof\oindex{Thalhof@\textbf{Thalhof}|pwk}es in Reichenau\oindex{Reichenau an der Rax@\textbf{Reichenau an der Rax}|pwk}. Ihr Ehemann Carl
                     Waissnix\pwindex{Waissnix, Carl 09.12.1851 – 25.04.1943@\textsc{Waissnix, Carl} (09.12.1851 – 25.04.1943), \emph{Hotelier}|pwk} wird zugleich als gutmütig und eifersüchtig beschrieben. Schnitzler\pwindex{Schnitzler, Arthur 15.05.1862 – 21.10.1931@\textsc{Schnitzler, Arthur} (15.05.1862 – 21.10.1931), \emph{Schriftsteller, Mediziner}|pwk} und Goldmann\pwindex{Goldmann, Paul 31.01.1865 – 25.09.1935@\textsc{Goldmann, Paul} (31.01.1865 – 25.09.1935), \emph{Schriftsteller, Journalist}|pwk} hatten sich am 7. 8. 1890 zuletzt gesehen, sodass der
                  zweitägige Besuch im Thalhof\oindex{Thalhof@\textbf{Thalhof}|pwk} auf dem Weg nach
                     Pörtschach\oindex{Poertschach@\textbf{Pörtschach}|pwk} stattfand und zeitlich
                  weitgehend genau eingegrenzt werden kann.}}}\label{K_L02648-1h} mit einer Empfindung warmer und
               aufrichtiger Sympathie weggegangen. Viele Fehler wohl, \strikeout{aber} die typischen Fehler der ſchönen Frau: eitel, \label{K_L02648-2v}\edtext{\begin{otherlanguage}{french}\textsc{poseure}\end{otherlanguage}}{\lemma{\textnormal{\emph{poseure}}}\Cendnote{\textnormal{französisch: wichtigtuerisch}}}\label{K_L02648-2h},
                  \begin{otherlanguage}{french}coquett\end{otherlanguage}; aber wenn man auf den Grund kommt,
               findet man einen Schatz von Ehrlichkeit und Natürlichkeit. Ich bin der Frau\pwindex{Waissnix, Olga 03.11.1862 – 04.11.1897@\textsc{Waissnix, Olga} (03.11.1862 – 04.11.1897), \emph{Hotelière}|pwv} mit allen möglichen
               Vorurtheilen {\pb}entgegengekommen; aber als wir am
               letzten Tag allein im Walde ſaßen und die gewiſſen tieferen Sachen
                  beſpr\textcolor{gray}{a}chen, da kam ein ſo heißer Glückshunger, ein ſo rechtes
               Streben nach dem Beſſeren zutage, daß ich dabei etwas empfand, das ich nicht anders,
               als Rührung nennen kann\textcolor{gray}{.} Ich bin der Frau \textsc{Olga\pwindex{Waissnix, Olga 03.11.1862 – 04.11.1897@\textsc{Waissnix, Olga} (03.11.1862 – 04.11.1897), \emph{Hotelière}|pw}} ein wahrer Freund geworden; und in dieser Eigenschaft muß ich Dir Eines sagen:
               Du darfſt dieſe Frau\pwindex{Waissnix, Olga 03.11.1862 – 04.11.1897@\textsc{Waissnix, Olga} (03.11.1862 – 04.11.1897), \emph{Hotelière}|pwv} unter
               keinen Umſtänden \label{K_L02648-3v}\edtext{betrügen}{\lemma{\textnormal{\emph{betrügen}}}\Cendnote{\textnormal{Die Beziehung zwischen Olga Waissnix\pwindex{Waissnix, Olga 03.11.1862 – 04.11.1897@\textsc{Waissnix, Olga} (03.11.1862 – 04.11.1897), \emph{Hotelière}|pwk} und Schnitzler\pwindex{Schnitzler, Arthur 15.05.1862 – 21.10.1931@\textsc{Schnitzler, Arthur} (15.05.1862 – 21.10.1931), \emph{Schriftsteller, Mediziner}|pwk} war weitgehend platonisch, doch wie dieser Brief, oder auch
                  die im \emph{Tagebuch}\pwindex{Schnitzler, Arthur 15.05.1862 – 21.10.1931@\textsc{Schnitzler, Arthur} (15.05.1862 – 21.10.1931), \emph{Schriftsteller, Mediziner}!Tagebuch1981 – 2000@\strich\emph{Tagebuch} {[}1981 – 2000{]}|pwk} festgehaltenen Küsse
                  beweisen, waren sie sich zu diesem Zeitpunkt der Beziehung unsicher, ob das so
                  bleiben sollte.}}}\label{K_L02648-3h}. Sie iſt auf Alles vorbereitet: daß das Liebesglück, das
               ſie ſucht, kurz dauern, daß es mit Qualen verbunden sein und mit Enttäuſchungen enden
               kann. Aber in einer Beziehung glaubt ſie an Dich – meine Vermuthung; {\pb}Confidencen hat’s nicht gegeben – daß Du ſie nur
               dann zur Deinigen machen wirſt, wenn du ſie liebſt. Ich habe mit Erſtaunen geſehn,
               daß dieſe Frau\pwindex{Waissnix, Olga 03.11.1862 – 04.11.1897@\textsc{Waissnix, Olga} (03.11.1862 – 04.11.1897), \emph{Hotelière}|pwv} wirklich und
               ehrlich kämpft und daß es \substVorne{}\textsuperscript{\textcolor{gray}{ihr}}\substDazwischen{}ſie\substHinten{} einen großen Entschluß koſtet, über ſo und ſoviel Pflichten hinweg dahin zu
               gehen, wo ſie ihr Glück vermuthet. Aber eben darum hat ſie doppelt das Recht, nicht
               getäuſcht zu werden. Wenn ſie wieder zu Dir kommt – und ſie wird wieder kommen, ich
               glaube das iſt das Facit unſerer Geſpräche, ich habe mich bemüht ihr Muth zum Glück
               zu machen – ſo ſage ihr, wie es mit Dir ſteht. Will ſie dann immer noch, ſo brauchſt
               Du keine Scrupeln {\pb}mehr zu haben. Aber dieſe Frau\pwindex{Waissnix, Olga 03.11.1862 – 04.11.1897@\textsc{Waissnix, Olga} (03.11.1862 – 04.11.1897), \emph{Hotelière}|pwv} aus bloßer Sinnenluſt zu
               genießen, mit einer Lüge auf der Zunge, wäre ein Verrath an Allem, was gut und edel
               iſt auf der Welt{\dotsfour}\pend
           \pstart
           Dies, \label{K_L02648-4v}\edtext{\textsc{ut animam meam salvarem}}{\lemma{\textnormal{\emph{ut animam meam salvarem}}}\Cendnote{\textnormal{lateinisch: um meine Seele zu
                  retten}}}\label{K_L02648-4h}. Im Übrigen haben wir, wie geſagt, viel von Dir geſprochen, direct
               und indirect, und ich habe es als meine Aufgabe betrachtet, die Frau\pwindex{Waissnix, Olga 03.11.1862 – 04.11.1897@\textsc{Waissnix, Olga} (03.11.1862 – 04.11.1897), \emph{Hotelière}|pwv} in der Liebe zu Dir zu bestärken, um ſo
               mehr, als ich diese Liebe auch – trotz Allem und Allen – als ein großes Glück für
               Dich erkannt habe. Ich habe natürlich die größte Vorſicht angewendet, und ich glaube
               nicht, daß Frau \textsc{Olga\pwindex{Waissnix, Olga 03.11.1862 – 04.11.1897@\textsc{Waissnix, Olga} (03.11.1862 – 04.11.1897), \emph{Hotelière}|pw}} eine Ahnung hat, daß ich Mit{\pb}wiſſer bin. In
               dieſem Punkte kannſt Du alſo vollauf beruhigt ſein. Im Übrigen hat ſie mir
               außerordentlich viel auch von den \label{K_L02648-5v}\edtext{\textsc{Pick\pwindex{Pick, Rudolf 14.12.1865 – 12.12.1915@\textsc{Pick, Rudolf} (14.12.1865 – 12.12.1915), \emph{Bildender Künstler}|pwv}\pwindex{Pick, Gustav 10.12.1832 – 29.04.1921@\textsc{Pick, Gustav} (10.12.1832 – 29.04.1921), \emph{Komponist}|pwv}\pwindex{Pick, Alfred 15.01.1864 – 23.10.1937@\textsc{Pick, Alfred} (15.01.1864 – 23.10.1937), \emph{Richter}|pwv}}’s}{\lemma{\textnormal{\emph{Pick’s}}}\Cendnote{\textnormal{Schnitzler\pwindex{Schnitzler, Arthur 15.05.1862 – 21.10.1931@\textsc{Schnitzler, Arthur} (15.05.1862 – 21.10.1931), \emph{Schriftsteller, Mediziner}|pwk}s Verwandte Gustav Pick\pwindex{Pick, Gustav 10.12.1832 – 29.04.1921@\textsc{Pick, Gustav} (10.12.1832 – 29.04.1921), \emph{Komponist}|pwk} und dessen Söhne Rudolf\pwindex{Pick, Rudolf 14.12.1865 – 12.12.1915@\textsc{Pick, Rudolf} (14.12.1865 – 12.12.1915), \emph{Bildender Künstler}|pwk} und Alfred\pwindex{Pick, Alfred 15.01.1864 – 23.10.1937@\textsc{Pick, Alfred} (15.01.1864 – 23.10.1937), \emph{Richter}|pwk}.}}}\label{K_L02648-5h} erzählt, offenbar, damit ich es wiedererzähle, was ich auch
               hiermit thue. Ich ſelbſt bin größtentheils von einer neuen mentalen Blödheit geweſen.
               Und ich werde ſie stark enttäuſcht haben. Wenn Du mir einen großen Freundesdienſt
               thun willſt – ich bitte Dich recht ſehr darum – ſo ſchreib’ {\pb}mir, \label{K_L02648-6v}\edtext{was ſie Dir über mich}{\lemma{\textnormal{\emph{was ſie Dir über mich}}}\Cendnote{\textnormal{Sie schrieb Schnitzler\pwindex{Schnitzler, Arthur 15.05.1862 – 21.10.1931@\textsc{Schnitzler, Arthur} (15.05.1862 – 21.10.1931), \emph{Schriftsteller, Mediziner}|pwk}: »Dr. Goldmann\pwindex{Goldmann, Paul 31.01.1865 – 25.09.1935@\textsc{Goldmann, Paul} (31.01.1865 – 25.09.1935), \emph{Schriftsteller, Journalist}|pw} ist schon abgereist, er schrieb mir aus Pörtschach\oindex{Poertschach@\textbf{Pörtschach}|pw}. Wir haben in den 2 Tagen viel
                     mit einander geplaudert, vieles auch über Sie. Ausgefragt hab’ ich ihn nicht,
                     erstens weil es mir zu gemein schien u. zweitens weil ich ja doch weiß, er sagt
                     mir nichts. Übrigens, ich bin sage \begin{otherlanguage}{french}comme une
                        image\end{otherlanguage} u. will gar nichts wissen.« (Arthur Schnitzler\pwindex{Schnitzler, Arthur 15.05.1862 – 21.10.1931@\textsc{Schnitzler, Arthur} (15.05.1862 – 21.10.1931), \emph{Schriftsteller, Mediziner}|pwk}, Olga Waissnix\pwindex{Waissnix, Olga 03.11.1862 – 04.11.1897@\textsc{Waissnix, Olga} (03.11.1862 – 04.11.1897), \emph{Hotelière}|pwk}: \emph{Liebe, die starb vor der
                        Zeit. Ein Briefwechsel}. Mit einem Vorwort von Hans Weigel. Hg. von
                     Therese Nickl und Heinrich Schnitzler. Wien, München, Zürich: \emph{Fritz
                        Molden}{ }1970, S. 216) }}}\label{K_L02648-6h} geſchrieben hat. Verliebt habe ich
               mich \uuline{\edtext{nicht}{\Cendnote{vierfach unterstrichen}}}; ſinnlich läßt mich die Frau\pwindex{Waissnix, Olga 03.11.1862 – 04.11.1897@\textsc{Waissnix, Olga} (03.11.1862 – 04.11.1897), \emph{Hotelière}|pwv} kalt.\pend
           \pstart
           Thatſächliches von meinem Aufenthalte iſt, daß ich bei meiner Ankunft ein Zimmer
               reſervirt fand (das vom vorigem Jahr); daß \uline{er}\pwindex{Waissnix, Carl 09.12.1851 – 25.04.1943@\textsc{Waissnix, Carl} (09.12.1851 – 25.04.1943), \emph{Hotelier}|pwv} um mich herum gegangen \strikeout{hat} iſt, als wollte er\pwindex{Waissnix, Carl 09.12.1851 – 25.04.1943@\textsc{Waissnix, Carl} (09.12.1851 – 25.04.1943), \emph{Hotelier}|pwv} mich freſſen, zuletzt aber
               recht zuthunlich und geſprächig geworden; daß ich \textsc{Herzl\pwindex{Herzl, Theodor 1860-05-02 – 1904-07-03@\textsc{Herzl, Theodor} (1860-05-02 – 1904-07-03), \emph{Schriftsteller, Journalist}|pw}} und Frau\pwindex{Herzl, Julie 01.02.1868 – 10.11.1907@\textsc{Herzl, Julie} (01.02.1868 – 10.11.1907)|pwv} dort
               geſprochen und meine Antipathie gegen Beide\pwindex{Herzl, Theodor 1860-05-02 – 1904-07-03@\textsc{Herzl, Theodor} (1860-05-02 – 1904-07-03), \emph{Schriftsteller, Journalist}|pwv}\pwindex{Herzl, Julie 01.02.1868 – 10.11.1907@\textsc{Herzl, Julie} (01.02.1868 – 10.11.1907)|pwv} recht grämlich verſtärkt habe; daß ich bei
               meiner Abreiſe, als ich die Zimmerrechnung verlangte, den Beſcheid erhielt: der
               gnädigen Frau\pwindex{Waissnix, Olga 03.11.1862 – 04.11.1897@\textsc{Waissnix, Olga} (03.11.1862 – 04.11.1897), \emph{Hotelière}|pwv} war es ein
                  Vergnügen\strikeout{,} – was mir unendlich peinlich war; daß
               ſie mir, in Gegenwart von {\pb}Fremden beim Abschied
               ſagte: »Wenn Sie nach Wien\oindex{Wien@\textbf{Wien}|pw} Briefe ſenden, ſo ſagen
               Sie viele Grüße von mir«; daß \textsc{\label{K_L02648-7v}\edtext{Rettinger\pwindex{Rettinger, Franz 1841 – 27.04.1901@\textsc{Rettinger, Franz} (1841 – 27.04.1901), \emph{Hotelsekretär}|pw}}{\lemma{\textnormal{\emph{Rettinger}}}\Cendnote{\textnormal{In \emph{Jugend in Wien}\pwindex{Schnitzler, Arthur 15.05.1862 – 21.10.1931@\textsc{Schnitzler, Arthur} (15.05.1862 – 21.10.1931), \emph{Schriftsteller, Mediziner}!Jugend in Wien1968@\strich\emph{Jugend in Wien} {[}1968{]}|pwk} wird er von Schnitzler\pwindex{Schnitzler, Arthur 15.05.1862 – 21.10.1931@\textsc{Schnitzler, Arthur} (15.05.1862 – 21.10.1931), \emph{Schriftsteller, Mediziner}|pwk} folgendermaßen beschrieben: »Das war der Buchhalter\pwindex{Rettinger, Franz 1841 – 27.04.1901@\textsc{Rettinger, Franz} (1841 – 27.04.1901), \emph{Hotelsekretär}|pwv}, Geschäftsführer\pwindex{Rettinger, Franz 1841 – 27.04.1901@\textsc{Rettinger, Franz} (1841 – 27.04.1901), \emph{Hotelsekretär}|pwv}, Vizedirektor\pwindex{Rettinger, Franz 1841 – 27.04.1901@\textsc{Rettinger, Franz} (1841 – 27.04.1901), \emph{Hotelsekretär}|pwv} des Thalhof\oindex{Thalhof@\textbf{Thalhof}|pw}s; ein kleiner, dicker,
                        beweglicher Mann\pwindex{Rettinger, Franz 1841 – 27.04.1901@\textsc{Rettinger, Franz} (1841 – 27.04.1901), \emph{Hotelsekretär}|pwv} in
                        den Dreißigern, meist städtisch gekleidet oder mit einem grünen Jagdrock
                        angetan, aber jederzeit ohne Kragen und Halsbinde. Er hatte eine spaßige,
                        geschwinde Art zu reden, war das Faktotum\pwindex{Rettinger, Franz 1841 – 27.04.1901@\textsc{Rettinger, Franz} (1841 – 27.04.1901), \emph{Hotelsekretär}|pwv}, der Vertraute\pwindex{Rettinger, Franz 1841 – 27.04.1901@\textsc{Rettinger, Franz} (1841 – 27.04.1901), \emph{Hotelsekretär}|pwv} und mehr oder weniger auch der Spion\pwindex{Rettinger, Franz 1841 – 27.04.1901@\textsc{Rettinger, Franz} (1841 – 27.04.1901), \emph{Hotelsekretär}|pwv} des Gatten\pwindex{Waissnix, Carl 09.12.1851 – 25.04.1943@\textsc{Waissnix, Carl} (09.12.1851 – 25.04.1943), \emph{Hotelier}|pwv}, was ihn
                        nicht hinderte oder vielleicht erst recht dazu veranlaßte, mit Frau Olga\pwindex{Waissnix, Olga 03.11.1862 – 04.11.1897@\textsc{Waissnix, Olga} (03.11.1862 – 04.11.1897), \emph{Hotelière}|pw} auf freundschaftlichem Fuß zu
                        stehen, die ihm keineswegs traute, aber eine gewisse Sympathie für ihn
                        hegte.« (S. 243)}}}\label{K_L02648-7h}} im Herbſt nach Wien\oindex{Wien@\textbf{Wien}|pw} kommt.\pend
           \pstart
           Alle Details mündlich.\pend
           \pstart
           Bitte, ſchreib’ mir genau, wie es Dir geht! Adreſſe: \textsc{Pörtschach\oindex{Poertschach@\textbf{Pörtschach}|pw}, Poste restante}.\pend
           \pstart
           Viele Grüße! {\\[\baselineskip]}Dein {\\[\baselineskip]}\spacefill\mbox{Paul Goldmann}\pend
           \leftskip=0em{}\pstart
           \noindent{}\label{K_L02648-8v}\edtext{Strombad}{\lemma{\textnormal{\emph{Strombad}}}\Cendnote{\textnormal{Wien\oindex{Wien@\textbf{Wien}|pwk} verfügte über mehrere Badeschiffe, die
                     sowohl am Ufer des Donaukanal\oindex{Donaukanal@\textbf{Donaukanal}|pwk}s wie der
                        Donau\oindex{Donau@\textbf{Donau}|pwk} Anker setzten. Geschwommen wurde
                     nicht direkt im Fluss, sondern in Becken innerhalb des Schiffes, die vom Fluss
                     gespeist wurden.}}}\label{K_L02648-8h}?? Biſt Du viel mit \textsc{Hirschfeld\pwindex{Hirschfeld, Robert 17.09.1857 – 02.04.1914@\textsc{Hirschfeld, Robert} (17.09.1857 – 02.04.1914), \emph{Journalist, Kritiker}|pw}} zuſammen? Grüße an \textsc{Kapper\pwindex{Kapper, Friedrich 21.04.1861 – 22.07.1939@\textsc{Kapper, Friedrich} (21.04.1861 – 22.07.1939), \emph{Mediziner}|pw}}!\pend
           
         
         \endnumbering\mylabel{h}\end{ledgroupsized}  \newcommand{\dateiname}{L02648}\newcommand{\titel}{Paul Goldmann an Arthur Schnitzler, 11. 8. 1890}\newcommand{\editorInnen}{Martin Anton Müller und Laura Untner}%% latex-leseansicht-abspann.tex
%% Abspann für die Leseansicht.
%% Der Schalter \ifkorrekturansicht ist bereits durch den Vorspann gesetzt.

%% latex-abspann.tex
%% Gemeinsamer Abspann für Korrekturansicht und Leseansicht.
%% Setzt den Schalter \ifkorrekturansicht voraus (gesetzt in den
%% einbindenden Dateien latex-korrekturansicht-abspann.tex bzw.
%% latex-leseansicht-abspann.tex).
%% ---------------------------------------------------------------

\normalsize

% Das esempio-Environment wird nur in der Leseansicht benötigt
\ifkorrekturansicht\else
\newenvironment{esempio}[3]%
{
    \vspace{1.5ex}
    \rlap{\underline{#1}}
    \par
    \setlength{\parindent}{0cm}
    \nopagebreak
    \leftskip=#2cm
    \rightskip=#3cm
}
{
    \par
}
\fi

\doendnotes{C}
\bigskip
\vfill

\clearpage

\footnotesize

\ifkorrekturansicht
  \lohead{\textsc{register}}
\fi

% theindex-Environment neu definieren ohne reledmac
\makeatletter
\renewenvironment{theindex}{%
  \ifkorrekturansicht
    \section*{\indexname}%
  \else
    \subsubsection*{Index der erwähnten Entitäten}%
  \fi
  \setlength{\parindent}{0pt}%
  \setlength{\parskip}{0pt plus 0.3pt}%
  \let\item\@idxitem
}{%
  \ifkorrekturansicht\clearpage\fi
}
\makeatother

\IfFileExists{\jobname-pw.ind}{\input{\jobname-pw.ind}}{}

% Quellenangabe nur in der Leseansicht
\ifkorrekturansicht\else
% Fallback-Definitionen, falls die .tex-Datei \titel etc. nicht gesetzt hat
\providecommand{\titel}{}
\providecommand{\editorInnen}{}
\providecommand{\dateiname}{\jobname}

\vspace{3cm}

\vfill

\footnotesize
\textsc{Quelle}: \titel. Herausgegeben von {\editorInnen}. In: \emph{Arthur Schnitzler: Briefwechsel mit Autorinnen und Autoren}.
 Digitale Edition, https://schnitzler-briefe.acdh.oeaw.ac.at/{\dateiname}.html (Stand \today)
\fi

\end{document}


      