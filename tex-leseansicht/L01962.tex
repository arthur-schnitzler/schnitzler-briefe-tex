%% latex-leseansicht-vorspann.tex
%% Vorspann für die Leseansicht.
%% Lädt die gemeinsame Datei latex-vorspann.tex mit nicht gesetztem Schalter.

\newif\ifkorrekturansicht
\korrekturansichtfalse

\input{../tex-inputs/latex-vorspann}


               \section[Arthur Schnitzler an Hugo von Hofmannsthal, 4. 10. 1910]{ Arthur Schnitzler an Hugo von Hofmannsthal, 4. 10. 1910}\nopagebreak\mylabel{v}\rehead{ }\begin{ledgroupsized}[t]{13cm}\normalsize\beginnumbering\briefempfaengerindex{Hofmannsthal, Hugo von@\textsc{Hofmannsthal, Hugo von}!zzzSchnitzler, Arthur@\emph{von Arthur Schnitzler}!1910-10-041@{4. 10. 1910}|(be} \toendnotes[C]{\smallbreak\pagebreak[2]} \Standort{FDH, Hs-30885,139.}
\physDesc{Brief, 1 Blatt, 2 Seiten
\newline{}Handschrift: schwarze Tinte, deutsche Kurrent (\noindent{}Schlußformel und
                                        Unterschrift)}\buchAbdrucke{\weitereDrucke{Hugo von Hofmannsthal, Arthur Schnitzler: \emph{Briefwechsel}. Hg. Therese Nickl und Heinrich Schnitzler. Frankfurt am Main: \emph{S. Fischer} 1964, S. 253–254.} }\toendnotes[C]{\smallbreak}\pstart
           {\pb}\textcolor{gray}{\textbf{Dr. Arthur Schnitzler}}\hfill 4. 10. 1910.\pend
           \pstart
           \textcolor{gray}{\textbf{Wien XVIII. Sternwartestrasse 71\oindex{Sternwartestrasse@\textbf{Sternwartestraße}|pw}}}\pend
           \pstart\center{}Mein lieber Hugo.\pend\pstart
           Mein Telegramm hat Sie hoffentlich noch in München\oindex{Muenchen@\textbf{München}|pw} erreicht. Es war mir nicht möglich eine telephonische
                    Verbindung mit Rosenbaum\pwindex{Rosenbaum, Richard 04.11.1867 – 25.06.1942@\textsc{Rosenbaum, Richard} (04.11.1867 – 25.06.1942), \emph{Dramaturg, Verleger}|pw} zu bekommen. Bald
                    war er auf der Probe, bald hat sich überhaupt niemand gemeldet. Berger\pwindex{Berger, Alfred von 30.04.1853 – 24.08.1912@\textsc{Berger, Alfred von} (30.04.1853 – 24.08.1912), \emph{Schriftsteller, Journalist, Theaterleiter}|pw} selbst war verreist und bis gestern noch nicht
                    zurückgekehrt. So habe ich also Ihren Besetzungsvorschlag an die Direktion\oindex{Burgtheater@\textbf{Burgtheater}|pwv} schriftlich
                    mitgeteilt und mich zugleich damit sehr einverstanden erklärt. Im übrigen lag
                    Ihrem Brief kein Besetzungsvorschlag des Burgtheater\oindex{Burgtheater@\textbf{Burgtheater}|pw}s bei; Sie schreiben von Tressler\pwindex{Tressler, Otto 13.04.1871 – 27.04.1965@\textsc{Tressler, Otto} (13.04.1871 – 27.04.1965), \emph{Schauspieler, Bildhauer}|pw} für den Claudio\pwindex{Hofmannsthal, Hugo von 01.02.1874 – 15.07.1929@\textsc{Hofmannsthal, Hugo von} (01.02.1874 – 15.07.1929), \emph{Schriftsteller}!Thor und der Tod1893@\strich\emph{Der Thor und der Tod} {[}1893{]}|pwv}, was wirklich lächerlich wäre. Wie sonst die Rollen hätten
                    verteilt werden sollen, weiss ich nicht, nur dass die Bleibtreu\pwindex{Bleibtreu, Hedwig 23.12.1868 – 24.01.1958@\textsc{Bleibtreu, Hedwig} (23.12.1868 – 24.01.1958), \emph{Schauspielerin}|pw} für den Tod\pwindex{Hofmannsthal, Hugo von 01.02.1874 – 15.07.1929@\textsc{Hofmannsthal, Hugo von} (01.02.1874 – 15.07.1929), \emph{Schriftsteller}!Thor und der Tod1893@\strich\emph{Der Thor und der Tod} {[}1893{]}|pwv} in Aussicht genommen war, hatte ich schon früher
                    gehört, ohne für diese Idee sehr eingenommen zu sein. Ich hoffe übrigens Sie
                    haben sich auch persönlich an die Direktion gewandt, was ich doch jedenfalls
                    viel wirksamer fände als meine Intervention, so gern ich immer dazu
                        {[}bereit{]}{ }{\pb}war und bin. An dem \label{K_L01962_1v}\edtext{Oedipus\pwindex{\textcolor{red}{\textsuperscript{XXXX1 indx}}!Koenig Oedipus. Uebersetzt und fuer die neuere Buehne eingerichtet1910@\strich\emph{König Ödipus. Übersetzt und für die neuere Bühne eingerichtet} {[}1910{]}|pw} haben Sie hoffentlich in München\oindex{Muenchen@\textbf{München}|pw}}{\lemma{\textnormal{\emph{Oedipus … München}}}\Cendnote{\textnormal{Die Premiere von \emph{König Ödipus}\pwindex{\textcolor{red}{\textsuperscript{XXXX1 indx}}!Koenig Oedipus. Uebersetzt und fuer die neuere Buehne eingerichtet1910@\strich\emph{König Ödipus. Übersetzt und für die neuere Bühne eingerichtet} {[}1910{]}|pwk} (Regie: Max
                            Reinhardt\pwindex{Reinhardt, Max 09.09.1873 – 30.10.1943@\textsc{Reinhardt, Max} (09.09.1873 – 30.10.1943), \emph{Theaterleiter, Regisseur, Schauspieler}|pwk}) in der Übersetzung von Hofmannsthal\pwindex{Hofmannsthal, Hugo von 01.02.1874 – 15.07.1929@\textsc{Hofmannsthal, Hugo von} (01.02.1874 – 15.07.1929), \emph{Schriftsteller}|pwk} hatte am 25. 9. 1910 in der Neuen Musik-Festhalle\oindex{Neue Musik-Festhalle@\textbf{Neue Musik-Festhalle}|pwk}
                        stattgefunden.}}}\label{K_L01962_1h} viel Freude gehabt. Hier schicke ich Ihnen also »das weite Land\pwindex{\textcolor{red}{\textsuperscript{XXXX1 indx}}!weite Land«. (Tragikomoedie in fuenf Akten von Artur Schnitzler. Zum erstenmal aufgefuehrt am 14. Oktober 1911)15. 10. 1911@\strich\emph{»Das weite Land«. (Tragikomödie in fünf Akten von Artur Schnitzler. Zum erstenmal aufgeführt am 14. Oktober 1911)} {[}15. 10. 1911{]}|pw}«, das ich bitte noch durchaus
                    als Manuscript zu behandeln.\pend
           \pstart
           {[}hs.:{]} Auf baldiges Wiederſehen.{\\[\baselineskip]}Herzlichſt Ihr{\\[\baselineskip]}\spacefill\mbox{Arthur}\pend
           \leftskip=0em{}\endnumbering\briefempfaengerindex{Hofmannsthal, Hugo von@\textsc{Hofmannsthal, Hugo von}!zzzSchnitzler, Arthur@\emph{von Arthur Schnitzler}!1910-10-041@{4. 10. 1910}|)be}\mylabel{h}\end{ledgroupsized}  \newcommand{\dateiname}{L01962}\newcommand{\titel}{Arthur Schnitzler an Hugo von Hofmannsthal, 4. 10. 1910}\newcommand{\editorInnen}{Martin Anton Müller und Gerd-Hermann Susen}
            \footnotesize
\begin{ledgroupsized}[t]{11.5cm}
\doendnotes{C}
\end{ledgroupsized}
         %% latex-leseansicht-abspann.tex
%% Abspann für die Leseansicht.
%% Der Schalter \ifkorrekturansicht ist bereits durch den Vorspann gesetzt.

%% latex-abspann.tex
%% Gemeinsamer Abspann für Korrekturansicht und Leseansicht.
%% Setzt den Schalter \ifkorrekturansicht voraus (gesetzt in den
%% einbindenden Dateien latex-korrekturansicht-abspann.tex bzw.
%% latex-leseansicht-abspann.tex).
%% ---------------------------------------------------------------

\normalsize

% Das esempio-Environment wird nur in der Leseansicht benötigt
\ifkorrekturansicht\else
\newenvironment{esempio}[3]%
{
    \vspace{1.5ex}
    \rlap{\underline{#1}}
    \par
    \setlength{\parindent}{0cm}
    \nopagebreak
    \leftskip=#2cm
    \rightskip=#3cm
}
{
    \par
}
\fi

\doendnotes{C}
\bigskip
\vfill

\clearpage

\footnotesize

\ifkorrekturansicht
  \lohead{\textsc{register}}
\fi

% theindex-Environment neu definieren ohne reledmac
\makeatletter
\renewenvironment{theindex}{%
  \ifkorrekturansicht
    \section*{\indexname}%
  \else
    \subsubsection*{Index der erwähnten Entitäten}%
  \fi
  \setlength{\parindent}{0pt}%
  \setlength{\parskip}{0pt plus 0.3pt}%
  \let\item\@idxitem
}{%
  \ifkorrekturansicht\clearpage\fi
}
\makeatother

\IfFileExists{\jobname-pw.ind}{\input{\jobname-pw.ind}}{}

% Quellenangabe nur in der Leseansicht
\ifkorrekturansicht\else
% Fallback-Definitionen, falls die .tex-Datei \titel etc. nicht gesetzt hat
\providecommand{\titel}{}
\providecommand{\editorInnen}{}
\providecommand{\dateiname}{\jobname}

\vspace{3cm}

\vfill

\footnotesize
\textsc{Quelle}: \titel. Herausgegeben von {\editorInnen}. In: \emph{Arthur Schnitzler: Briefwechsel mit Autorinnen und Autoren}.
 Digitale Edition, https://schnitzler-briefe.acdh.oeaw.ac.at/{\dateiname}.html (Stand \today)
\fi

\end{document}


      