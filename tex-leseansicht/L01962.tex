%% latex-korrekturansicht-vorspann.tex
%% Vorspann für die Korrekturansicht.
%% Lädt die gemeinsame Datei latex-vorspann.tex mit gesetztem Schalter.

\newif\ifkorrekturansicht
\korrekturansichttrue

\input{../tex-inputs/latex-vorspann}


\section[Arthur Schnitzler an Hugo von Hofmannsthal, 4. 10. 1910]{L01962 Arthur Schnitzler an Hugo von Hofmannsthal, 4. 10. 1910}
\nopagebreak\mylabel{L01962v}
\rehead{ }\normalsize\beginnumbering\briefempfaengerindex{Hofmannsthal, Hugo von@\textsc{Hofmannsthal, Hugo von}!zzzSchnitzler, Arthur@\emph{von Arthur Schnitzler}!1910-10-041@{4. 10. 1910}|(be}
\toendnotes[C]{\smallbreak\pagebreak[2]}\Standort{FDH, Hs-30885,139.}
\physDesc{Brief, 1 Blatt, 2 Seiten, 1182 Zeichen
\newline{}Handschrift: schwarze Tinte, deutsche Kurrent (\noindent{}Schlußformel und Unterschrift)}
\buchAbdrucke{\weitereDrucke{Hugo von Hofmannsthal, Arthur Schnitzler: \emph{Briefwechsel}. Frankfurt am Main: \emph{S. Fischer} 1964, S. 253–254.} }\toendnotes[C]{\smallbreak}
\pstart
           
\pstart
           {\pb}\textcolor{gray}{\textbf{Dr. Arthur Schnitzler}}\pend
           
\pstart
           \raggedleft{}4. 10. 1910.\pend
           \pend
           
\pstart
           \textcolor{gray}{\textbf{Wien XVIII. Sternwartestrasse 71\oindex{Sternwartestrasse 71@\textbf{Sternwartestraße 71}, \emph{Wohngebäude (K.WHS)}|pw}}}\pend
           
\pstart\center{}Mein lieber Hugo.\pend\vspace{0.5em}
\pstart
           Mein Telegramm hat Sie hoffentlich noch in München\oindex{Muenchen@\textbf{München}, \emph{P.PPLA}|pw} erreicht. Es war mir nicht möglich eine telephonische Verbindung mit
                  Rosenbaum\pwindex{Rosenbaum, Richard 04.11.1867 – 25.06.1942@\textsc{Rosenbaum, Richard} (04.11.1867 – 25.06.1942), \emph{Dramaturg/Dramaturgin, Verleger/Verlegerin}|pw} zu bekommen. Bald war er auf der
               Probe, bald hat sich überhaupt niemand gemeldet. Berger\pwindex{Berger, Alfred von 30.04.1853 – 24.08.1912@\textsc{Berger, Alfred von} (30.04.1853 – 24.08.1912), \emph{Schriftsteller/Schriftstellerin, Journalist/Journalistin, Theaterleiter/Theaterleiterin}|pw} selbst war verreist und bis gestern noch nicht zurückgekehrt. So habe
               ich also Ihren Besetzungsvorschlag an die Direktion\oindex{Burgtheater@\textbf{Burgtheater}, \emph{S.THTR}|pwv} schriftlich mitgeteilt und mich zugleich damit
               sehr einverstanden erklärt. Im übrigen lag Ihrem Brief kein Besetzungsvorschlag des
                  Burgtheaters\oindex{Burgtheater@\textbf{Burgtheater}, \emph{S.THTR}|pw} bei; Sie schreiben von Tressler\pwindex{Tressler, Otto 13.04.1871 – 27.04.1965@\textsc{Tressler, Otto} (13.04.1871 – 27.04.1965), \emph{Schauspieler/Schauspielerin, Bildhauer/Bildhauerin}|pw} für den Claudio\pwindex{Thor und der Tod@\emph{Der Thor und der Tod}|pwv}, was wirklich lächerlich wäre. Wie
               sonst die Rollen hätten verteilt werden sollen, weiss ich nicht, nur dass die Bleibtreu\pwindex{Bleibtreu, Hedwig 23.12.1868 – 24.01.1958@\textsc{Bleibtreu, Hedwig} (23.12.1868 – 24.01.1958), \emph{Schauspieler/Schauspielerin}|pw} für den Tod\pwindex{Thor und der Tod@\emph{Der Thor und der Tod}|pwv} in Aussicht genommen war, hatte ich
               schon früher gehört, ohne für diese Idee sehr eingenommen zu sein. Ich hoffe übrigens
               Sie haben sich auch persönlich an die Direktion gewandt, was ich doch jedenfalls viel
               wirksamer fände als meine Intervention, so gern ich immer dazu
                  {[}bereit{]}{ }{\pb}war und bin. An dem \label{K_L01962-1v}\edtext{Oedipus\pwindex{Koenig Oedipus. Uebersetzt und fuer die neuere Buehne eingerichtet@\emph{König Ödipus. Übersetzt und für die neuere Bühne eingerichtet}|pw} haben Sie hoffentlich in München\oindex{Muenchen@\textbf{München}, \emph{P.PPLA}|pw}}{\lemma{\textnormal{\emph{Oedipus … München}}}\Cendnote{\textnormal{Die Premiere von \emph{König Ödipus}\pwindex{Koenig Oedipus. Uebersetzt und fuer die neuere Buehne eingerichtet@\emph{König Ödipus. Übersetzt und für die neuere Bühne eingerichtet}|pwk}  in der Übersetzung von Hofmannsthal\pwindex{Hofmannsthal, Hugo von 1874-02-01 – 1929-07-15@\textsc{Hofmannsthal, Hugo von} (1874-02-01 – 1929-07-15), \emph{Schriftsteller/Schriftstellerin}|pwk} hatte am 25. 9. 1910 in der Neuen Musik-Festhalle\oindex{Neue Musik-Festhalle@\textbf{Neue Musik-Festhalle}, \emph{Gebäude (K.GBD)}|pwk} stattgefunden. Die Regie führte Max
                  Reinhardt\pwindex{Reinhardt, Max 09.09.1873 – 30.10.1943@\textsc{Reinhardt, Max} (09.09.1873 – 30.10.1943), \emph{Theaterleiter/Theaterleiterin, Regisseur/Regisseurin, Schauspieler/Schauspielerin}|pwk}.}}}\label{K_L01962-1} viel Freude
               gehabt. Hier schicke ich Ihnen also »das weite
                  Land\pwindex{weite Land«. (Tragikomoedie in fuenf Akten von Artur Schnitzler. Zum erstenmal aufgefuehrt am 14. Oktober 1911)@\emph{»Das weite Land«. (Tragikomödie in fünf Akten von Artur Schnitzler. Zum erstenmal aufgeführt am 14. Oktober 1911)}|pw}«, das ich bitte noch durchaus als Manuscript zu behandeln.\pend
           
\pstart
           {[}hs.:{]} Auf baldiges Wiederſehen.{\\[\baselineskip]}Herzlichſt Ihr{\\[\baselineskip]}\spacefill\mbox{Arthur}\pend
           \leftskip=0em{}\selectlanguage{ngerman}\endnumbering\briefempfaengerindex{Hofmannsthal, Hugo von@\textsc{Hofmannsthal, Hugo von}!zzzSchnitzler, Arthur@\emph{von Arthur Schnitzler}!1910-10-041@{4. 10. 1910}|)be}\mylabel{L01962h}  \normalsize

\doendnotes{C}
\bigskip
\vfill

\clearpage

\footnotesize

\lohead{\textsc{register}}

% Definiere theindex-Environment komplett neu ohne reledmac
\makeatletter
\renewenvironment{theindex}{%
  \section*{\indexname}%
  \setlength{\parindent}{0pt}%
  \setlength{\parskip}{0pt plus 0.3pt}%
  \let\item\@idxitem
}{%
  \clearpage
}
\makeatother

\IfFileExists{\jobname-pw.ind}{\input{\jobname-pw.ind}}{}

\end{document}

      