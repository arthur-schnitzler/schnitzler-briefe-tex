%% latex-leseansicht-vorspann.tex
%% Vorspann für die Leseansicht.
%% Lädt die gemeinsame Datei latex-vorspann.tex mit nicht gesetztem Schalter.

\newif\ifkorrekturansicht
\korrekturansichtfalse

\input{../tex-inputs/latex-vorspann}


               \section[Friedrich M. Fels an Arthur Schnitzler, 4. 10. 1895]{ Friedrich M. Fels an Arthur Schnitzler, 4. 10. 1895}\nopagebreak\mylabel{v}\rehead{ }\begin{ledgroupsized}[t]{13cm}\normalsize\beginnumbering\briefempfaengerindex{Schnitzler, Arthur@\textsc{Schnitzler, Arthur}!zzzFels, Friedrich Michael@\emph{von Friedrich Michael Fels}!1895-10-043@{4. 10. 1895}|(be} \toendnotes[C]{\smallbreak\pagebreak[2]} \Standort{DLA, A:Schnitzler, HS.NZ85.1.2956.}
\physDesc{Brief, 1 Blatt, 3 Seiten
\newline{}Handschrift: schwarze Tinte, lateinische Kurrent
\newline{}Schnitzler: mit Bleistift nummeriert: »26« }\toendnotes[C]{\smallbreak}\pstart
           \raggedleft{}{\pb}Zürich I, Schifflände 30\oindex{Schifflaende@\textbf{Schifflände}|pw},
                        III. Stock{\\}am 4. Oktober 1895\pend
           \pstart\center{}Lieber Doktor Schnitzler!\pend\pstart
           Wie Sie aus der Datierung ersehen, bin ich, dank Ihrer und Beer-Hofma{\geminationn}\pwindex{Beer-Hofmann, Richard 11.07.1866 – 26.09.1945@\textsc{Beer-Hofmann, Richard} (11.07.1866 – 26.09.1945), \emph{Schriftsteller}|pw}s Hilfe, wieder im Besitze einer eigenen Wohnung. Ich danke Ihnen herzlich.
                    Ich wohne jetzt bei einer beka{\geminationn}ten Familie\pwindex{Ott, Julius *~1850@\textsc{Ott, Julius} (*~1850), \emph{Bankangestellter}|pwv}\pwindex{Ott, Anna Elisabetha *~1861@\textsc{Ott, Anna Elisabetha} (*~1861)|pwv}, zusa{\geminationm}en mit einem Freunde\pwindex{Meichl @\textsc{Meichl}, \emph{Industrieller, Handlungsreisender}|pwv}, einem alten Herrn, Wien\oindex{Wien@\textbf{Wien}|pw}er, Schwager von Dreher\pwindex{Dreher, Carl Anton 21.3.1849 – 7.8.1921@\textsc{Dreher, Carl Anton} (21.3.1849 – 7.8.1921), \emph{Bierbrauer}|pw} in Schwechat\oindex{Schwechat@\textbf{Schwechat}|pw}, der früher
                    lange Jahre in Amerika\oindex{Amerika@\textbf{Amerika}|pw} und Deutschland\oindex{Deutschland@\textbf{Deutschland}|pw} ein groſser Fabrikant war, da{\geminationn} fallierte und nun in seinen alten Tagen als
                    Reisender eines Papiergeschäfts mühsam sein Leben fristet. Wir haben zusa{\geminationm}en ein groſses Wohnzi{\geminationm}er, ein Kabinet und einen Alkoven, wofür wir 50 francs zahlen – gewiſs billig.
                    Na, der Teufel wird schon weiterhelfen.\pend
           \pstart
           Ich hätte noch eine Bitte. Wären Sie so freundlich, bei Beer-Hofma{\geminationn}\pwindex{Beer-Hofmann, Richard 11.07.1866 – 26.09.1945@\textsc{Beer-Hofmann, Richard} (11.07.1866 – 26.09.1945), \emph{Schriftsteller}|pw} nachzufragen, ob er vielleicht wieder einen {\pb}alten Anzug hat; das Porto ka{\geminationn} ja nicht viel kosten. Und ich bin absolut
                    auſserstande, mir selbst einen beizubringen. Seien Sie nicht böse, und besten
                    Dank im vorhinein.\pend
           \pstart
           Ich schreibe wirklich einen Aufsatz\pwindex{Fels, Friedrich Michael *~1864@\textsc{Fels, Friedrich Michael} (*~1864), \emph{Journalist}!Volkslieder der Bulgaren11.12.1895 – 11.12.1895@\strich\emph{Die Volkslieder der Bulgaren} {[}11.12.1895 – 11.12.1895{]}|pwv} für Wengraf\pwindex{Wengraf, Edmund 09.01.1860 – 08.12.1933@\textsc{Wengraf, Edmund} (09.01.1860 – 08.12.1933), \emph{Journalist}|pw} und Osten\pwindex{Osten, Heinrich 16.08.1855 – 01.08.1931@\textsc{Osten, Heinrich} (16.08.1855 – 01.08.1931), \emph{Schriftsteller, Journalist}|pw} und werde da{\geminationn}{ }\label{K_L00499_1v}\edtext{einen}{\lemma{\textnormal{\emph{einen}}}\Cendnote{\textnormal{nicht nachgewiesen}}}\label{K_L00499_1h} für die Preſse\pwindex{Presse3. 7. 1848@\emph{Die Presse}|pw}{ }ſchreiben. Apropos Preſse\orgindex{Presse@Die Presse|pw}: Dr. Hirschfeld\pwindex{Hirschfeld, Robert 17.09.1857 – 02.04.1914@\textsc{Hirschfeld, Robert} (17.09.1857 – 02.04.1914), \emph{Journalist, Musikkritiker}|pw} muſs ja
                    jetzt wieder in Wien\oindex{Wien@\textbf{Wien}|pw}{ }ſein, und Sie kö{\geminationn}ten vielleicht bei Gelegenheit mit ihm sprechen, ob es sich nicht machen
                    lieſse, daſs ich für das Blatt die Schweiz\oindex{Schweiz@\textbf{Schweiz}|pw}er
                    Korrespondenz, auch über Politik und Volkswirtschaft, übernähme. Ich haben
                        bego{\geminationn}en, mich in die Verhältniſse einzuleben,
                    und glaube, daſs ich genügen würde.\pend
           \pstart
           Daſs Mackay\pwindex{Mackay, John Henry 06.02.1864 – 21.05.1933@\textsc{Mackay, John Henry} (06.02.1864 – 21.05.1933), \emph{Schriftsteller}|pw} Ihnen gefallen hat, freut mich.
                    Auch ich habe ihn gern. Er hat, bei viel Schlauheit und einiger Reserviertheit,
                    viele liebenswürdige Seiten, vor allem eine sehr angenehme Naivetät. Naiv ist
                    zwar auch Henckell\pwindex{Henckell, Karl Friedrich 17.04.1864 – 30.07.1929@\textsc{Henckell, Karl Friedrich} (17.04.1864 – 30.07.1929), \emph{Schriftsteller}|pw}, dabei aber entsetzlich
                    langweilig und geistlos. Sie haben mich einen Antisemiten gena{\geminationn}t, aber – mit Ariern verkehrt es sich wirklich zu
                    schwer.\pend
           \pstart
           {\pb}Nehmen Sie mir meine neue Bitte nicht
                    übel, grüßen Sie Beer-Hofma{\geminationn}\pwindex{Beer-Hofmann, Richard 11.07.1866 – 26.09.1945@\textsc{Beer-Hofmann, Richard} (11.07.1866 – 26.09.1945), \emph{Schriftsteller}|pw}, Loris\pwindex{Hofmannsthal, Hugo von 01.02.1874 – 15.07.1929@\textsc{Hofmannsthal, Hugo von} (01.02.1874 – 15.07.1929), \emph{Schriftsteller}|pw}, Hirschfeld\pwindex{Hirschfeld, Robert 17.09.1857 – 02.04.1914@\textsc{Hirschfeld, Robert} (17.09.1857 – 02.04.1914), \emph{Journalist, Musikkritiker}|pw} etc von mir und seien Sie selbst herzlichst gegrüßt\pend
           \pstart
           von{\\[\baselineskip]}Ihrem{\\[\baselineskip]}\spacefill\mbox{Fels}\pend
           \leftskip=0em{}\pstart
           \noindent{}Was sagen Sie zu Mackay\pwindex{Mackay, John Henry 06.02.1864 – 21.05.1933@\textsc{Mackay, John Henry} (06.02.1864 – 21.05.1933), \emph{Schriftsteller}|pw}s neuestem Buch\pwindex{Mackay, John Henry 06.02.1864 – 21.05.1933@\textsc{Mackay, John Henry} (06.02.1864 – 21.05.1933), \emph{Schriftsteller}!Albert Schnell s Untergang. Eine Geschichte ohne Handlung1895 – 1895@\strich\emph{Albert Schnell’s Untergang. Eine Geschichte ohne Handlung} {[}1895 – 1895{]}|pwv}? Erscheint bald
                        wieder etwas von Ihnen? Wie stehts mit der Aufführung\pwindex{Schnitzler, Arthur 15.05.1862 – 21.10.1931@\textsc{Schnitzler, Arthur} (15.05.1862 – 21.10.1931), \emph{Schriftsteller, Mediziner}!Liebelei. Schauspiel in drei Akten9. 10. 1895@\strich\emph{Liebelei. Schauspiel in drei Akten} {[}9. 10. 1895{]}|pwv}? David\pwindex{David, Jakob Julius 06.02.1859 – 20.11.1906@\textsc{David, Jakob Julius} (06.02.1859 – 20.11.1906), \emph{Schriftsteller, Journalist}|pw}\pwindex{David, Jakob Julius 06.02.1859 – 20.11.1906@\textsc{David, Jakob Julius} (06.02.1859 – 20.11.1906), \emph{Schriftsteller, Journalist}!Regentag12.10.1895 – 12.10.1895@\strich\emph{Ein Regentag} {[}12.10.1895 – 12.10.1895{]}|pwv} ko{\geminationm}t also am 12. daran; ich
                        bin begierig.\pend
           \endnumbering\briefempfaengerindex{Schnitzler, Arthur@\textsc{Schnitzler, Arthur}!zzzFels, Friedrich Michael@\emph{von Friedrich Michael Fels}!1895-10-043@{4. 10. 1895}|)be}\mylabel{h}\end{ledgroupsized}  \newcommand{\dateiname}{L00499}\newcommand{\titel}{Friedrich M. Fels an Arthur Schnitzler, 4. 10. 1895}\newcommand{\editorInnen}{Martin Anton Müller und Gerd-Hermann Susen}
            \footnotesize
\begin{ledgroupsized}[t]{11.5cm}
\doendnotes{C}
\end{ledgroupsized}
         %% latex-leseansicht-abspann.tex
%% Abspann für die Leseansicht.
%% Der Schalter \ifkorrekturansicht ist bereits durch den Vorspann gesetzt.

%% latex-abspann.tex
%% Gemeinsamer Abspann für Korrekturansicht und Leseansicht.
%% Setzt den Schalter \ifkorrekturansicht voraus (gesetzt in den
%% einbindenden Dateien latex-korrekturansicht-abspann.tex bzw.
%% latex-leseansicht-abspann.tex).
%% ---------------------------------------------------------------

\normalsize

% Das esempio-Environment wird nur in der Leseansicht benötigt
\ifkorrekturansicht\else
\newenvironment{esempio}[3]%
{
    \vspace{1.5ex}
    \rlap{\underline{#1}}
    \par
    \setlength{\parindent}{0cm}
    \nopagebreak
    \leftskip=#2cm
    \rightskip=#3cm
}
{
    \par
}
\fi

\doendnotes{C}
\bigskip
\vfill

\clearpage

\footnotesize

\ifkorrekturansicht
  \lohead{\textsc{register}}
\fi

% theindex-Environment neu definieren ohne reledmac
\makeatletter
\renewenvironment{theindex}{%
  \ifkorrekturansicht
    \section*{\indexname}%
  \else
    \subsubsection*{Index der erwähnten Entitäten}%
  \fi
  \setlength{\parindent}{0pt}%
  \setlength{\parskip}{0pt plus 0.3pt}%
  \let\item\@idxitem
}{%
  \ifkorrekturansicht\clearpage\fi
}
\makeatother

\IfFileExists{\jobname-pw.ind}{\input{\jobname-pw.ind}}{}

% Quellenangabe nur in der Leseansicht
\ifkorrekturansicht\else
% Fallback-Definitionen, falls die .tex-Datei \titel etc. nicht gesetzt hat
\providecommand{\titel}{}
\providecommand{\editorInnen}{}
\providecommand{\dateiname}{\jobname}

\vspace{3cm}

\vfill

\footnotesize
\textsc{Quelle}: \titel. Herausgegeben von {\editorInnen}. In: \emph{Arthur Schnitzler: Briefwechsel mit Autorinnen und Autoren}.
 Digitale Edition, https://schnitzler-briefe.acdh.oeaw.ac.at/{\dateiname}.html (Stand \today)
\fi

\end{document}


      