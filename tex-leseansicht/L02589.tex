\input{../tex-inputs/latex-pdf-vorspann}
\begin{center}
            \textcolor{red}{ENTWURF. ENTZIFFERUNG NOCH NICHT KORREKTURGELESEN}
                      \end{center}
            
               \section[Marie Herzfeld an Arthur Schnitzler, 5. 3. 1931]{ Marie Herzfeld an Arthur Schnitzler, 5. 3. 1931}\nopagebreak\mylabel{v}\rehead{ }\begin{ledgroupsized}[t]{13cm}\normalsize\beginnumbering\briefempfaengerindex{Schnitzler, Arthur@\textsc{Schnitzler, Arthur}!zzzHerzfeld, Marie@\emph{von Marie Herzfeld}!1931-03-052@{5. 3. 1931}|(be} \toendnotes[C]{\smallbreak\pagebreak[2]} \Standort{Privatbesitz, Reinhard Urbach, \emph{ohne Signatur}.}
\physDesc{Brief, 1 Blatt, 4 Seiten, fotografische Vervielfältigung
\newline{}Handschrift: schwarze Tinte, lateinische Kurrent
\newline{}Schnitzler: mutmaßlich mit rotem Buntstift drei Unterstreichungen \newline{}Zusatz: Das Original des Briefes ist verschollen. Eine Kopie des Briefes
                                 wurde am 20. 10. 1972 von Heinrich Schnitzler\pwindex{Schnitzler, Heinrich 09.08.1902 – 12.07.1982@\textsc{Schnitzler, Heinrich} (09.08.1902 – 12.07.1982), \emph{Regisseur, Schauspieler}|pw} an Reinhard Urbach\pwindex{Urbach, Reinhard *~1939-11-12@\textsc{Urbach, Reinhard} (*~1939-11-12), \emph{Theaterleiter, Literaturwissenschaftler}|pw}
                                 übermittelt. }\toendnotes[C]{\smallbreak}\pstart
           \noindent{}\centering{}{\pb}Wien III/\textsubscript{3}, Oetzeltgasse 1\oindex{Oelzeltgasse@\textbf{Ölzeltgasse}|pw}\pend
           \pstart
           \raggedleft{}den 5. März 1931\pend
           \pstart
           Sehr geehrter Herr Doktor, verzeihen Sie, wenn ich Ihre Muße – Arbeitsmuße – störe und mit einer Frage in Ihre
               Einsamkeit breche. Auf Wunsch der Zeitschrift »Corona\pwindex{Corona. Zweimonatsschrift1930 – 1944@\emph{Corona. Zweimonatsschrift}|pw}« habe ich aus meinen Loris\pwindex{Hofmannsthal, Hugo von 01.02.1874 – 15.07.1929@\textsc{Hofmannsthal, Hugo von} (01.02.1874 – 15.07.1929), \emph{Schriftsteller}|pw}-Erinnerungen und Loris\pwindex{Hofmannsthal, Hugo von 01.02.1874 – 15.07.1929@\textsc{Hofmannsthal, Hugo von} (01.02.1874 – 15.07.1929), \emph{Schriftsteller}|pw}-Briefen einen
                  \label{K_L02589-1v}\edtext{Aufsatz\pwindex{Herzfeld, Marie 20.03.1855 – 22.09.1940@\textsc{Herzfeld, Marie} (20.03.1855 – 22.09.1940), \emph{Schriftstellerin, Übersetzerin}!Loris. Blaetter der Erinnerung1932-05 – 1932-05@\strich\emph{Loris. Blätter der Erinnerung} {[}1932-05 – 1932-05{]}|pw}}{\lemma{\textnormal{\emph{Aufsatz}}}\Cendnote{\textnormal{Trotz der im Brief
                  vorgebrachten Eile verzögerte sich die Publikation: Marie Herzfeld\pwindex{Herzfeld, Marie 20.03.1855 – 22.09.1940@\textsc{Herzfeld, Marie} (20.03.1855 – 22.09.1940), \emph{Schriftstellerin, Übersetzerin}|pwk}: \emph{Loris.
                        Blätter der Erinnerungen}\pwindex{Herzfeld, Marie 20.03.1855 – 22.09.1940@\textsc{Herzfeld, Marie} (20.03.1855 – 22.09.1940), \emph{Schriftstellerin, Übersetzerin}!Loris. Blaetter der Erinnerung1932-05 – 1932-05@\strich\emph{Loris. Blätter der Erinnerung} {[}1932-05 – 1932-05{]}|pwk}. In: \emph{Corona.
                        Zweimonatsschrift}\pwindex{Corona. Zweimonatsschrift1930 – 1944@\emph{Corona. Zweimonatsschrift}|pwk}, Jg. 2, Nr. 6, Mai 1932,
                     S. 715–732.}}}\label{K_L02589-1h} zusammengestellt, {\pb}in dem ich auch aus den schönen
                  \label{K_L02589-2v}\edtext{Briefen\pwindex{Hofmannsthal, Hugo von 01.02.1874 – 15.07.1929@\textsc{Hofmannsthal, Hugo von} (01.02.1874 – 15.07.1929), \emph{Schriftsteller}!Briefe an Freunde1930-04-01@\strich\emph{Briefe an Freunde} {[}1930-04-01{]}|pw} schöpfe, die Sie im Aprilheft der N. R.\pwindex{neue Rundschau1904@\emph{Die neue Rundschau}|pw} v. 1930}{\lemma{\textnormal{\emph{Briefen … 1930}}}\Cendnote{\textnormal{Hugo von Hofmannsthal\pwindex{Hofmannsthal, Hugo von 01.02.1874 – 15.07.1929@\textsc{Hofmannsthal, Hugo von} (01.02.1874 – 15.07.1929), \emph{Schriftsteller}|pwk}: \emph{Briefe an Freunde}\pwindex{Hofmannsthal, Hugo von 01.02.1874 – 15.07.1929@\textsc{Hofmannsthal, Hugo von} (01.02.1874 – 15.07.1929), \emph{Schriftsteller}!Briefe an Freunde1930-04-01@\strich\emph{Briefe an Freunde} {[}1930-04-01{]}|pwk}. In: \emph{Die
                        neue Rundschau}\pwindex{neue Rundschau1904@\emph{Die neue Rundschau}|pwk}, Jg. 41, Nr. 4,
                        1. 4. 1930, S. 512–519. Siehe Hugo von Hofmannsthal an Arthur Schnitzler, 19. 7. [1892]}}}\label{K_L02589-2h} hatten. Am
                  19. Juli 92 spricht Hofmannsthal\pwindex{Hofmannsthal, Hugo von 01.02.1874 – 15.07.1929@\textsc{Hofmannsthal, Hugo von} (01.02.1874 – 15.07.1929), \emph{Schriftsteller}|pw} von dem \label{K_L02589-3v}\edtext{Renaissancedrama\pwindex{Hofmannsthal, Hugo von 01.02.1874 – 15.07.1929@\textsc{Hofmannsthal, Hugo von} (01.02.1874 – 15.07.1929), \emph{Schriftsteller}!Ascanio und Gioconda1979@\strich\emph{Ascanio und Gioconda} {[}1979{]}|pwv}, an dem er
                  arbeite}{\lemma{\textnormal{\emph{Renaissancedrama, … arbeite}}}\Cendnote{\textnormal{Ascanio und Gioconda\pwindex{Hofmannsthal, Hugo von 01.02.1874 – 15.07.1929@\textsc{Hofmannsthal, Hugo von} (01.02.1874 – 15.07.1929), \emph{Schriftsteller}!Ascanio und Gioconda1979@\strich\emph{Ascanio und Gioconda} {[}1979{]}|pwkv} blieb zu Lebzeiten
                  unveröffentlicht, heute in \emph{Sämtliche Werke. Kritische
                        Ausgabe}, Bd. 18.}}}\label{K_L02589-3h}: mir erzählte er davon nichts,
               obwohl er um diese Zeit mit mir lebhaft korrespon{\pb}dierte, und ich wagte, trotz einiger
               innerer Einwände, die Hypothese, dass es sich um eine Beschäftigung mit d. geretteten Venedig\pwindex{\textcolor{red}{\textsuperscript{XXXX1 indx}}!gerettete Venedig9.2.1682 – 9.2.1682@\strich\emph{Das gerettete Venedig} {[}9.2.1682 – 9.2.1682{]}|pw} handelte, die er dann später, wie
               Sie wissen, \uline{mehrmals} neu aufnahm und erst \label{K_L02589-4v}\edtext{nach Jahren zu Ende brachte.}{\lemma{\textnormal{\emph{nach … brachte.}}}\Cendnote{\textnormal{Hofmannsthal\pwindex{Hofmannsthal, Hugo von 01.02.1874 – 15.07.1929@\textsc{Hofmannsthal, Hugo von} (01.02.1874 – 15.07.1929), \emph{Schriftsteller}|pwk} arbeitete von August 1902 bis
                     Juli 1904 an seinem Trauerspiel \emph{Das gerettete Venedig}\pwindex{Hofmannsthal, Hugo von 01.02.1874 – 15.07.1929@\textsc{Hofmannsthal, Hugo von} (01.02.1874 – 15.07.1929), \emph{Schriftsteller}!gerettete Venedig. Trauerspiel in fuenf Aufzuegen1905@\strich\emph{Das gerettete Venedig. Trauerspiel in fünf Aufzügen} {[}1905{]}|pwk}, das am
                     21. 1. 1905 in Berlin\oindex{Berlin@\textbf{Berlin}|pwk}
                  uraufgeführt wurde und im gleichen Jahr gedruckt erschien.}}}\label{K_L02589-4h}{ }{\pb}Wollen Sie, aus Ihrem besseren Wissen,
               mich aufklären? Ich wäre Ihnen sehr dankbar! Aber die Sache drängt! In großer
               Schätzung,\pend
           \pstart \spacefill\mbox{Marie Herzfeld}\pend{}\endnumbering\briefempfaengerindex{Schnitzler, Arthur@\textsc{Schnitzler, Arthur}!zzzHerzfeld, Marie@\emph{von Marie Herzfeld}!1931-03-052@{5. 3. 1931}|)be}\mylabel{h}\end{ledgroupsized}  \newcommand{\dateiname}{L02589}\newcommand{\titel}{Marie Herzfeld an Arthur Schnitzler, 5. 3. 1931}\newcommand{\editorInnen}{Martin Anton Müller und Laura Untner}\input{../tex-inputs/latex-pdf-abspann}
      