%% latex-leseansicht-vorspann.tex
%% Vorspann für die Leseansicht.
%% Lädt die gemeinsame Datei latex-vorspann.tex mit nicht gesetztem Schalter.

\newif\ifkorrekturansicht
\korrekturansichtfalse

\input{../tex-inputs/latex-vorspann}


\section[Marie Herzfeld an Arthur Schnitzler, 5.\,3.\,1931]{L02589 Marie Herzfeld an Arthur Schnitzler, 5.\,3.\,1931}
\nopagebreak\mylabel{L02589v}
\rehead{ }\normalsize\beginnumbering\briefempfaengerindex{Schnitzler, Arthur@\textsc{Schnitzler, Arthur}!zzzHerzfeld, Marie@\emph{von Marie Herzfeld}!1931-03-052@{5.\,3.\,1931}|(be}
\toendnotes[C]{\smallbreak\pagebreak[2]}
\correspDesc{Versand  durch Marie Herzfeld am 5. 3. 1931 in Wien
\newline{}Erhalt  durch Arthur Schnitzler im Zeitraum [5. 3. 1931
                  – 9. 3. 1931?] in Wien}\toendnotes[C]{\smallbreak}
\Standort{Privatbesitz, Reinhard Urbach, \emph{ohne Signatur}.}
\physDesc{Brief, fotografische Vervielfältigung, 1 Blatt, 4 Seiten, 907 Zeichen
\newline{}Handschrift: schwarze Tinte, lateinische Kurrent
\newline{}Schnitzler: mutmaßlich mit rotem Buntstift drei Unterstreichungen 
\newline{}Zusatz: Das Original des Briefes ist verschollen. Eine Kopie des Briefes
                                 wurde am 20.\,10.\,1972 von Heinrich Schnitzler\pwindex{Schnitzler, Heinrich 9.\,8.\,1902 Hinterbrühl – 12.\,7.\,1982 Wien@\textsc{Schnitzler, Heinrich} (9.\,8.\,1902 Hinterbrühl – 12.\,7.\,1982 Wien), \emph{Regisseur, Schauspieler}|pw} an Reinhard Urbach\pwindex{Urbach, Reinhard *~12.\,11.\,1939 Weimar@\textsc{Urbach, Reinhard} (*~12.\,11.\,1939 Weimar), \emph{Theaterleiter, Literaturwissenschaftler}|pw} übermittelt. }\toendnotes[C]{\smallbreak}
\pstart
           \centering{}{\pb}Wien III/\textsubscript{3},
                     Oetzeltgasse 1\oindex{Ölzeltgasse@\textbf{Ölzeltgasse}, \emph{Straße}|pw}\pend
           
\pstart
           \raggedleft{}den 5. März 1931\pend
           \vspace{0.5em}
\pstart
           Sehr geehrter Herr Doktor, verzeihen Sie, wenn ich Ihre Muße –
               Arbeitsmuße – störe und mit einer Frage in Ihre Einsamkeit breche. Auf Wunsch der
               Zeitschrift »Corona\pwindex{Corona. Zweimonatsschrift@\emph{Corona. Zweimonatsschrift}|pw}« habe ich aus meinen Loris\pwindex{Hofmannsthal, Hugo von 1.\,2.\,1874 Wien – 15.\,7.\,1929 Rodaun@\textsc{Hofmannsthal, Hugo von} (1.\,2.\,1874 Wien – 15.\,7.\,1929 Rodaun), \emph{Schriftsteller}|pw}-Erinnerungen und Loris\pwindex{Hofmannsthal, Hugo von 1.\,2.\,1874 Wien – 15.\,7.\,1929 Rodaun@\textsc{Hofmannsthal, Hugo von} (1.\,2.\,1874 Wien – 15.\,7.\,1929 Rodaun), \emph{Schriftsteller}|pw}-Briefen einen \label{K_L02589-1v}\edtext{Aufsatz\pwindex{Herzfeld, Marie 20.\,3.\,1855 Kőszeg – 22.\,9.\,1940 Mining@\textsc{Herzfeld, Marie} (20.\,3.\,1855 Kőszeg – 22.\,9.\,1940 Mining), \emph{Schriftstellerin, Übersetzerin}!Loris. Blätter der Erinnerung@\strich\emph{Loris. Blätter der Erinnerung}|pw}}{\lemma{\textnormal{\emph{Aufsatz}}}\Cendnote{\textnormal{Trotz der im Brief vorgebrachten Eile
                  verzögerte sich die Publikation: Marie Herzfeld\pwindex{Herzfeld, Marie 20.\,3.\,1855 Kőszeg – 22.\,9.\,1940 Mining@\textsc{Herzfeld, Marie} (20.\,3.\,1855 Kőszeg – 22.\,9.\,1940 Mining), \emph{Schriftstellerin, Übersetzerin}|pwk}: \emph{Loris. Blätter der Erinnerungen}\pwindex{Herzfeld, Marie 20.\,3.\,1855 Kőszeg – 22.\,9.\,1940 Mining@\textsc{Herzfeld, Marie} (20.\,3.\,1855 Kőszeg – 22.\,9.\,1940 Mining), \emph{Schriftstellerin, Übersetzerin}!Loris. Blätter der Erinnerung@\strich\emph{Loris. Blätter der Erinnerung}|pwk}. In: \emph{Corona. Zweimonatsschrift}\pwindex{Corona. Zweimonatsschrift@\emph{Corona. Zweimonatsschrift}|pwk}, Jg. 2, Nr. 6,
                        Mai 1932, S. 715–732.}}}\label{K_L02589-1} zusammengestellt, {\pb}in dem ich auch aus den schönen
                  \label{K_L02589-2v}\edtext{Briefen\pwindex{Hofmannsthal, Hugo von 1.\,2.\,1874 Wien – 15.\,7.\,1929 Rodaun@\textsc{Hofmannsthal, Hugo von} (1.\,2.\,1874 Wien – 15.\,7.\,1929 Rodaun), \emph{Schriftsteller}!Briefe an Freunde@\strich\emph{Briefe an Freunde}|pw} schöpfe, die Sie im Aprilheft der N. R.\pwindex{neue Rundschau@\emph{Die neue Rundschau}|pw} v. 1930}{\lemma{\textnormal{\emph{Briefen … 1930}}}\Cendnote{\textnormal{Hugo von Hofmannsthal\pwindex{Hofmannsthal, Hugo von 1.\,2.\,1874 Wien – 15.\,7.\,1929 Rodaun@\textsc{Hofmannsthal, Hugo von} (1.\,2.\,1874 Wien – 15.\,7.\,1929 Rodaun), \emph{Schriftsteller}|pwk}: \emph{Briefe an Freunde}\pwindex{Hofmannsthal, Hugo von 1.\,2.\,1874 Wien – 15.\,7.\,1929 Rodaun@\textsc{Hofmannsthal, Hugo von} (1.\,2.\,1874 Wien – 15.\,7.\,1929 Rodaun), \emph{Schriftsteller}!Briefe an Freunde@\strich\emph{Briefe an Freunde}|pwk}. In: \emph{Die neue Rundschau}\pwindex{neue Rundschau@\emph{Die neue Rundschau}|pwk}, Jg. 41, Nr. 4, April 1930,
                     S. 512–519. Vgl. XXXX Auszeichnungsfehler: Dokument L00105 nicht gefunden.}}}\label{K_L02589-2} hatten. Am 19. Juli 92 spricht Hofmannsthal\pwindex{Hofmannsthal, Hugo von 1.\,2.\,1874 Wien – 15.\,7.\,1929 Rodaun@\textsc{Hofmannsthal, Hugo von} (1.\,2.\,1874 Wien – 15.\,7.\,1929 Rodaun), \emph{Schriftsteller}|pw} von dem \label{K_L02589-3v}\edtext{Renaissancedrama\pwindex{Hofmannsthal, Hugo von 1.\,2.\,1874 Wien – 15.\,7.\,1929 Rodaun@\textsc{Hofmannsthal, Hugo von} (1.\,2.\,1874 Wien – 15.\,7.\,1929 Rodaun), \emph{Schriftsteller}!Ascanio und Gioconda@\strich\emph{Ascanio und Gioconda}|pwv}, an dem er
                  arbeite}{\lemma{\textnormal{\emph{Renaissancedrama, … arbeite}}}\Cendnote{\textnormal{\emph{Ascanio und Gioconda}\pwindex{Hofmannsthal, Hugo von 1.\,2.\,1874 Wien – 15.\,7.\,1929 Rodaun@\textsc{Hofmannsthal, Hugo von} (1.\,2.\,1874 Wien – 15.\,7.\,1929 Rodaun), \emph{Schriftsteller}!Ascanio und Gioconda@\strich\emph{Ascanio und Gioconda}|pwk} blieb
                  zu Lebzeiten unveröffentlicht, heute in \emph{Sämtliche Werke. Kritische Ausgabe},
                  Bd. 18.}}}\label{K_L02589-3}: mir erzählte er davon nichts, obwohl er um diese Zeit mit
               mir lebhaft korrespon{\pb}dierte, und ich
               wagte, trotz einiger innerer Einwände, die Hypothese, dass es sich um eine
               Beschäftigung mit d. geretteten Venedig\pwindex{\textcolor{red}{\textsuperscript{XXXX indx1}}!Venice Preserv'd@\strich\emph{Venice Preserv'd}|pw}
               handelte, die er dann später, wie Sie wissen, \uline{mehrmals} neu aufnahm und erst \label{K_L02589-4v}\edtext{nach Jahren zu Ende brachte.}{\lemma{\textnormal{\emph{nach … brachte.}}}\Cendnote{\textnormal{Hofmannsthal\pwindex{Hofmannsthal, Hugo von 1.\,2.\,1874 Wien – 15.\,7.\,1929 Rodaun@\textsc{Hofmannsthal, Hugo von} (1.\,2.\,1874 Wien – 15.\,7.\,1929 Rodaun), \emph{Schriftsteller}|pwk} arbeitete von
                     August 1902 bis Juli 1904 an seinem Trauerspiel \emph{Das gerettete Venedig}\pwindex{Hofmannsthal, Hugo von 1.\,2.\,1874 Wien – 15.\,7.\,1929 Rodaun@\textsc{Hofmannsthal, Hugo von} (1.\,2.\,1874 Wien – 15.\,7.\,1929 Rodaun), \emph{Schriftsteller}!gerettete Venedig. Trauerspiel in fünf Aufzügen@\strich\emph{Das gerettete Venedig. Trauerspiel in fünf Aufzügen}|pwk}, das am
                     21. 1. 1905 in Berlin\oindex{Berlin@\textbf{Berlin}, \emph{Hauptstadt}|pwk}
                  uraufgeführt wurde und im gleichen Jahr gedruckt erschien.}}}\label{K_L02589-4}{ }{\pb}Wollen Sie, aus Ihrem besseren
               Wissen, mich aufklären? Ich wäre Ihnen sehr dankbar! Aber die Sache drängt! In großer
               Schätzung,\pend
           \pstart \spacefill\mbox{Marie Herzfeld}\pend{}\selectlanguage{ngerman}\endnumbering\briefempfaengerindex{Schnitzler, Arthur@\textsc{Schnitzler, Arthur}!zzzHerzfeld, Marie@\emph{von Marie Herzfeld}!1931-03-052@{5.\,3.\,1931}|)be}\mylabel{L02589h}  \newcommand{\dateiname}{L02589}\newcommand{\titel}{Marie Herzfeld an Arthur Schnitzler, 5. 3. 1931}\newcommand{\editorInnen}{Martin Anton Müller und Laura Untner}%% latex-leseansicht-abspann.tex
%% Abspann für die Leseansicht.
%% Der Schalter \ifkorrekturansicht ist bereits durch den Vorspann gesetzt.

%% latex-abspann.tex
%% Gemeinsamer Abspann für Korrekturansicht und Leseansicht.
%% Setzt den Schalter \ifkorrekturansicht voraus (gesetzt in den
%% einbindenden Dateien latex-korrekturansicht-abspann.tex bzw.
%% latex-leseansicht-abspann.tex).
%% ---------------------------------------------------------------

\normalsize

% Das esempio-Environment wird nur in der Leseansicht benötigt
\ifkorrekturansicht\else
\newenvironment{esempio}[3]%
{
    \vspace{1.5ex}
    \rlap{\underline{#1}}
    \par
    \setlength{\parindent}{0cm}
    \nopagebreak
    \leftskip=#2cm
    \rightskip=#3cm
}
{
    \par
}
\fi

\doendnotes{C}
\bigskip
\vfill

\clearpage

\footnotesize

\ifkorrekturansicht
  \lohead{\textsc{register}}
\fi

% theindex-Environment neu definieren ohne reledmac
\makeatletter
\renewenvironment{theindex}{%
  \ifkorrekturansicht
    \section*{\indexname}%
  \else
    \subsubsection*{Index der erwähnten Entitäten}%
  \fi
  \setlength{\parindent}{0pt}%
  \setlength{\parskip}{0pt plus 0.3pt}%
  \let\item\@idxitem
}{%
  \ifkorrekturansicht\clearpage\fi
}
\makeatother

\IfFileExists{\jobname-pw.ind}{\input{\jobname-pw.ind}}{}

% Quellenangabe nur in der Leseansicht
\ifkorrekturansicht\else
% Fallback-Definitionen, falls die .tex-Datei \titel etc. nicht gesetzt hat
\providecommand{\titel}{}
\providecommand{\editorInnen}{}
\providecommand{\dateiname}{\jobname}

\vspace{3cm}

\vfill

\footnotesize
\textsc{Quelle}: \titel. Herausgegeben von {\editorInnen}. In: \emph{Arthur Schnitzler: Briefwechsel mit Autorinnen und Autoren}.
 Digitale Edition, https://schnitzler-briefe.acdh.oeaw.ac.at/{\dateiname}.html (Stand \today)
\fi

\end{document}


