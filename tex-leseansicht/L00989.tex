%% latex-leseansicht-vorspann.tex
%% Vorspann für die Leseansicht.
%% Lädt die gemeinsame Datei latex-vorspann.tex mit nicht gesetztem Schalter.

\newif\ifkorrekturansicht
\korrekturansichtfalse

\input{../tex-inputs/latex-vorspann}


         
         \renewcommand{\erwaehntePersonen}{Personen: Richard Beer-Hofmann, Otto Brahm, Leo Feld, Paul Goldmann, Alfred Kerr, Olga Wohlbrück}
         \renewcommand{\erwaehnteOrte}{Orte: Berlin, Sankt Michael, Tirol, Wien}
         \renewcommand{\erwaehnteWerke}{Werke: Der Graf von Charolais. Ein Trauerspiel, Der Schleier der Beatrice. Schauspiel in fünf Akten}
               \section[Arthur Schnitzler an Richard Beer-Hofmann, 8. 10. 1899]{ Arthur Schnitzler an Richard Beer-Hofmann, 8. 10. 1899}\nopagebreak\mylabel{v}\rehead{ }\begin{ledgroupsized}[t]{13cm}\normalsize\beginnumbering \toendnotes[C]{\smallbreak\pagebreak[2]} \Standort{YCGL, MSS 31.}
\physDesc{Brief, 1 Blatt (Briefpapier mit Trauerrand), 3 Seiten, Umschlag
\newline{}Handschrift: 1) schwarze Tinte, deutsche Kurrent\hspace{1em}2) schwarze Tinte, lateinische Kurrent (\noindent{}Adresse)\hspace{1em}\newline{}Versand: 1) Stempel: »\nobreak{}\oindex{Berlin@\textbf{Berlin}|pwk}Berlin, 8. 10. 99, 5–6N\nobreak{}«.   2) Stempel: »\nobreak{}\oindex{Sankt Michael@\textbf{Sankt Michael}|pwk}St. Mich{[}ae{]}l in
                                       Eppan, 10 10 99\nobreak{}«. }\buchAbdrucke{\weitereDrucke{Arthur Schnitzler, Richard Beer-Hofmann: \emph{Briefwechsel 1891–1931}. Hg. Konstanze Fliedl. Wien, Zürich: \emph{Europaverlag} 1992, S. 139.} }\toendnotes[C]{\smallbreak}\pstart{}{\pb}Herrn Dr. Richard Beer-Hofmann\pend{}\pstart{}St. Michael im Eppan\oindex{Sankt Michael@\textbf{Sankt Michael}|pw}\pend{}\pstart{}Tirol\oindex{Tirol@\textbf{Tirol}|pw}\pend{}{\bigskip}\pstart
           \raggedleft{}{\pb}\textsc{Berlin}\oindex{Berlin@\textbf{Berlin}|pw}{ } 8. X. 99.\pend
           \pstart
           mein lieber Richard, das iſt entſetzlich, was dieſer Leo\pwindex{Feld, Leo 14.02.1869 – 05.09.1924@\textsc{Feld, Leo} (14.02.1869 – 05.09.1924), \emph{Schriftsteller, Übersetzer, Dirigent}|pw} wieder \label{K_L00989_1v}\edtext{durchmachen}{\lemma{\textnormal{\emph{durchmachen}}}\Cendnote{\textnormal{Er
                  hatte sich mit Olga Wohlbrück\pwindex{Wohlbrueck, Olga 05.07.1865 – 22.07.1943@\textsc{Wohlbrück, Olga} (05.07.1865 – 22.07.1943), \emph{Schriftstellerin, Schauspielerin}|pwk} verlobt, die
                  beiden heirateten im März 1900 in Berlin\oindex{Berlin@\textbf{Berlin}|pwk}.}}}\label{K_L00989_1h} muſs! Da kommen einem immer wieder dieſe alten Phraſen in
               den Mund, aber ich will ſie unterdrücken. Wa{\geminationn} kommen Sie
               nach Wien\oindex{Wien@\textbf{Wien}|pw}? Paul
                  Goldmann\pwindex{Goldmann, Paul 31.01.1865 – 25.09.1935@\textsc{Goldmann, Paul} (31.01.1865 – 25.09.1935), \emph{Schriftsteller, Journalist}|pw} ko{\geminationm}t, ebenſo wie ich, Do{\geminationn}erſtg oder Freitag in Wien\oindex{Wien@\textbf{Wien}|pw} an – pardon – \uline{will} ankommen – ebenſo wie \uline{ich will}; er wird
               etwa 8 Tage bei mir wohnen. Ich denke, Sie {\pb}werden
               auch nicht mehr lang da unten oder da oben bleiben? Nun jedenfalls richten Sie ſichs
               wohl ſo ein, dſs Sie \strikeout{Rich}{ }Paul\pwindex{Goldmann, Paul 31.01.1865 – 25.09.1935@\textsc{Goldmann, Paul} (31.01.1865 – 25.09.1935), \emph{Schriftsteller, Journalist}|pw} noch in Wien\oindex{Wien@\textbf{Wien}|pw} antreffen –?\pend
           \pstart
           Ich habe geſtern dem Brahm\pwindex{Brahm, Otto 05.02.1856 – 28.11.1912@\textsc{Brahm, Otto} (05.02.1856 – 28.11.1912), \emph{Theaterleiter, Regisseur}|pw} die \textsc{Beatrice}\pwindex{Schnitzler, Arthur 15.05.1862 – 21.10.1931@\textsc{Schnitzler, Arthur} (15.05.1862 – 21.10.1931), \emph{Schriftsteller, Mediziner}!Schleier der Beatrice. Schauspiel in fuenf Akten1900-12-01@\strich\emph{Der Schleier der Beatrice. Schauspiel in fünf Akten} {[}1900-12-01{]}|pw}, mit guter Wirkung, glaub ich, vorgeleſen. Er hat kaum gemerkt, wie viel ich
               noch dran zu machen habe. Die ungeſtrichene Aufführg würde fünf Stunden {\pb}dauern.\pend
           \pstart
           Ihre Ermahnung kam zu ſpät – ich hatte Brahm\pwindex{Brahm, Otto 05.02.1856 – 28.11.1912@\textsc{Brahm, Otto} (05.02.1856 – 28.11.1912), \emph{Theaterleiter, Regisseur}|pw}{ }ſchon eine »beſſere Meinung« beigebracht. So grüßt
               er Sie alſo weiter, \textsc{Kerr}\pwindex{Kerr, Alfred 25.12.1867 – 12.10.1948@\textsc{Kerr, Alfred} (25.12.1867 – 12.10.1948), \emph{Schriftsteller, Kritiker}|pw} desgleichen.\pend
           \pstart
           – Hier friert man bereits un\textcolor{gray}{d} heizt ein und friert trotzdem.\pend
           \pstart
           Leben Sie wohl und erlauben Sie mir mich auf die unſelige Mitgift\pwindex{Beer-Hofmann, Richard 1866-07-11 – 1945-09-26@\textsc{Beer-Hofmann, Richard} (1866-07-11 – 1945-09-26), \emph{Schriftsteller}!Graf von Charolais. Ein Trauerspiel1904-12-23@\strich\emph{Der Graf von Charolais. Ein Trauerspiel} {[}1904-12-23{]}|pw} zu freuen.\pend
           \pstart
           Herzlichſt Ihr{\\[\baselineskip]}\spacefill\mbox{Arthur}\pend
           \leftskip=0em{}
         
         \endnumbering\mylabel{h}\end{ledgroupsized}  \newcommand{\dateiname}{L00989}\newcommand{\titel}{Arthur Schnitzler an Richard Beer-Hofmann, 8. 10. 1899}\newcommand{\editorInnen}{Martin Anton Müller und Gerd-Hermann Susen}%% latex-leseansicht-abspann.tex
%% Abspann für die Leseansicht.
%% Der Schalter \ifkorrekturansicht ist bereits durch den Vorspann gesetzt.

%% latex-abspann.tex
%% Gemeinsamer Abspann für Korrekturansicht und Leseansicht.
%% Setzt den Schalter \ifkorrekturansicht voraus (gesetzt in den
%% einbindenden Dateien latex-korrekturansicht-abspann.tex bzw.
%% latex-leseansicht-abspann.tex).
%% ---------------------------------------------------------------

\normalsize

% Das esempio-Environment wird nur in der Leseansicht benötigt
\ifkorrekturansicht\else
\newenvironment{esempio}[3]%
{
    \vspace{1.5ex}
    \rlap{\underline{#1}}
    \par
    \setlength{\parindent}{0cm}
    \nopagebreak
    \leftskip=#2cm
    \rightskip=#3cm
}
{
    \par
}
\fi

\doendnotes{C}
\bigskip
\vfill

\clearpage

\footnotesize

\ifkorrekturansicht
  \lohead{\textsc{register}}
\fi

% theindex-Environment neu definieren ohne reledmac
\makeatletter
\renewenvironment{theindex}{%
  \ifkorrekturansicht
    \section*{\indexname}%
  \else
    \subsubsection*{Index der erwähnten Entitäten}%
  \fi
  \setlength{\parindent}{0pt}%
  \setlength{\parskip}{0pt plus 0.3pt}%
  \let\item\@idxitem
}{%
  \ifkorrekturansicht\clearpage\fi
}
\makeatother

\IfFileExists{\jobname-pw.ind}{\input{\jobname-pw.ind}}{}

% Quellenangabe nur in der Leseansicht
\ifkorrekturansicht\else
% Fallback-Definitionen, falls die .tex-Datei \titel etc. nicht gesetzt hat
\providecommand{\titel}{}
\providecommand{\editorInnen}{}
\providecommand{\dateiname}{\jobname}

\vspace{3cm}

\vfill

\footnotesize
\textsc{Quelle}: \titel. Herausgegeben von {\editorInnen}. In: \emph{Arthur Schnitzler: Briefwechsel mit Autorinnen und Autoren}.
 Digitale Edition, https://schnitzler-briefe.acdh.oeaw.ac.at/{\dateiname}.html (Stand \today)
\fi

\end{document}


      