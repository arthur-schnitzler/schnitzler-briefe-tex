%% latex-leseansicht-vorspann.tex
%% Vorspann für die Leseansicht.
%% Lädt die gemeinsame Datei latex-vorspann.tex mit nicht gesetztem Schalter.

\newif\ifkorrekturansicht
\korrekturansichtfalse

\input{../tex-inputs/latex-vorspann}


\section[Arthur Schnitzler an Richard Beer-Hofmann, 8. 10. 1899]{L00989 Arthur Schnitzler an Richard Beer-Hofmann, 8. 10. 1899}
\nopagebreak\mylabel{L00989v}
\rehead{ }\normalsize\beginnumbering\briefempfaengerindex{Beer-Hofmann, Richard@\textsc{Beer-Hofmann, Richard}!zzzSchnitzler, Arthur@\emph{von Arthur Schnitzler}!1899-10-081@{8. 10. 1899}|(be}
\toendnotes[C]{\smallbreak\pagebreak[2]}
\correspDesc{Versand  durch Arthur Schnitzler am 8. 10. 1899 in Berlin
\newline{}Erhalt  durch Richard Beer-Hofmann am 10. 10. 1899 in Sankt Michael}\toendnotes[C]{\smallbreak}
\Standort{YCGL, MSS 31.}
\physDesc{Brief, 1 Blatt, 3 Seiten, Kuvert, 1041 Zeichen
\newline{}Handschrift: schwarze Tinte, deutsche Kurrent
\newline{}Versand: 1) Stempel: »\nobreak{}\oindex{Berlin@\textbf{Berlin}, \emph{Hauptstadt}|pwk}Berlin, 8. 10. 99, 5–6N\nobreak{}«.   2) Stempel: »\nobreak{}\oindex{Sankt Michael@\textbf{Sankt Michael}, \emph{Bezirk}|pwk}St. Mich{[}ae{]}l in
                                       Eppan, 10 10 99\nobreak{}«. }
\buchAbdrucke{\weitereDrucke{Arthur Schnitzler, Richard Beer-Hofmann: \emph{Briefwechsel 1891–1931}. Herausgegeben von Konstanze Fliedl. Wien, Zürich: \emph{Europaverlag} 1992, S. 139.} }\toendnotes[C]{\smallbreak}\pstart{}\textsc{{\pb}Herrn Dr. Richard Beer-Hofmann}\pend{}\pstart{}\textsc{St. Michael im Eppan\oindex{Sankt Michael@\textbf{Sankt Michael}, \emph{Bezirk}|pw}}\pend{}\pstart{}\textsc{Tirol\oindex{Tirol@\textbf{Tirol}, \emph{Land}|pw}}\pend{}{\bigskip}\vspace{1em}
\pstart
           \raggedleft{}{\pb}\textsc{Berlin}\oindex{Berlin@\textbf{Berlin}, \emph{Hauptstadt}|pw}{ }8. X. 99.\pend
           \vspace{0.5em}
\pstart
           mein lieber Richard, das iſt entſetzlich, was dieſer Leo\pwindex{Feld, Leo 14.\,2.\,1869 Augsburg – 5.\,9.\,1924 Florenz@\textsc{Feld, Leo} (14.\,2.\,1869 Augsburg – 5.\,9.\,1924 Florenz), \emph{Schriftsteller, Übersetzer, Dirigent}|pw} wieder \label{K_L00989-1v}\edtext{durchmachen}{\lemma{\textnormal{\emph{durchmachen}}}\Cendnote{\textnormal{Er
                  hatte sich mit Olga Wohlbrück\pwindex{Wohlbrück, Olga 5.\,7.\,1865 Gainfarn – 22.\,7.\,1943 Berlin@\textsc{Wohlbrück, Olga} (5.\,7.\,1865 Gainfarn – 22.\,7.\,1943 Berlin), \emph{Schriftstellerin, Schauspielerin}|pwk} verlobt, die
                  beiden heirateten im März 1900 in Berlin\oindex{Berlin@\textbf{Berlin}, \emph{Hauptstadt}|pwk}.}}}\label{K_L00989-1} muſs! Da kommen einem immer wieder dieſe alten Phraſen in
               den Mund, aber ich will{ }ſie unterdrücken. Wa{\geminationn} kommen Sie
               nach Wien\oindex{Wien@\textbf{Wien}, \emph{Verwaltungsgebiet}|pw}? Paul
                  Goldmann\pwindex{Goldmann, Paul 31.\,1.\,1865 Breslau – 25.\,9.\,1935 Wien@\textsc{Goldmann, Paul} (31.\,1.\,1865 Breslau – 25.\,9.\,1935 Wien), \emph{Schriftsteller, Journalist}|pw} ko{\geminationm}t, ebenſo wie ich, Do{\geminationn}erſtg oder Freitag in Wien\oindex{Wien@\textbf{Wien}, \emph{Verwaltungsgebiet}|pw} an – pardon – \uline{will} ankommen – ebenſo wie \uline{ich will}; er wird
               etwa 8 Tage bei mir wohnen. Ich denke, Sie {\pb}werden
               auch nicht mehr lang da unten oder da oben bleiben? Nun jedenfalls richten Sie{ }ſichs
               wohl{ }ſo ein, dſs Sie \strikeout{Rich}{ }Paul\pwindex{Goldmann, Paul 31.\,1.\,1865 Breslau – 25.\,9.\,1935 Wien@\textsc{Goldmann, Paul} (31.\,1.\,1865 Breslau – 25.\,9.\,1935 Wien), \emph{Schriftsteller, Journalist}|pw} noch in Wien\oindex{Wien@\textbf{Wien}, \emph{Verwaltungsgebiet}|pw} antreffen –?\pend
           
\pstart
           Ich habe geſtern dem Brahm\pwindex{Brahm, Otto 5.\,2.\,1856 Hamburg – 28.\,11.\,1912 Berlin@\textsc{Brahm, Otto} (5.\,2.\,1856 Hamburg – 28.\,11.\,1912 Berlin), \emph{Theaterleiter, Regisseur}|pw} die \textsc{Beatrice}\pwindex{Schnitzler, Arthur 15.\,5.\,1862 Wien – 21.\,10.\,1931 ebd.@\textsc{Schnitzler, Arthur} (15.\,5.\,1862 Wien – 21.\,10.\,1931 ebd.), \emph{Schriftsteller, Mediziner}!Schleier der Beatrice. Schauspiel in fünf Akten@\strich\emph{Der Schleier der Beatrice. Schauspiel in fünf Akten}|pw}, mit guter Wirkung, glaub ich, vorgeleſen. Er hat kaum gemerkt, wie viel ich
               noch dran zu machen habe. Die ungeſtrichene Aufführg würde fünf Stunden {\pb}dauern.\pend
           
\pstart
           Ihre Ermahnung kam zu{ }ſpät – ich hatte Brahm\pwindex{Brahm, Otto 5.\,2.\,1856 Hamburg – 28.\,11.\,1912 Berlin@\textsc{Brahm, Otto} (5.\,2.\,1856 Hamburg – 28.\,11.\,1912 Berlin), \emph{Theaterleiter, Regisseur}|pw}{ }ſchon eine »beſſere Meinung« beigebracht. So grüßt
               er Sie alſo weiter, \textsc{Kerr}\pwindex{Kerr, Alfred 25.\,12.\,1867 Breslau – 12.\,10.\,1948 Hamburg@\textsc{Kerr, Alfred} (25.\,12.\,1867 Breslau – 12.\,10.\,1948 Hamburg), \emph{Schriftsteller, Kritiker}|pw} desgleichen.\pend
           
\pstart
           – Hier friert man bereits un\textcolor{gray}{d} heizt ein und friert trotzdem.\pend
           
\pstart
           Leben Sie wohl und erlauben Sie mir mich auf die unſelige Mitgift\pwindex{Beer-Hofmann, Richard 11.\,7.\,1866 Wien – 26.\,9.\,1945 New York City@\textsc{Beer-Hofmann, Richard} (11.\,7.\,1866 Wien – 26.\,9.\,1945 New York City), \emph{Schriftsteller}!Graf von Charolais. Ein Trauerspiel@\strich\emph{Der Graf von Charolais. Ein Trauerspiel}|pw} zu freuen.\pend
           
\pstart
           Herzlichſt Ihr{\\[\baselineskip]}\spacefill\mbox{Arthur}\pend
           \leftskip=0em{}\selectlanguage{ngerman}\endnumbering\briefempfaengerindex{Beer-Hofmann, Richard@\textsc{Beer-Hofmann, Richard}!zzzSchnitzler, Arthur@\emph{von Arthur Schnitzler}!1899-10-081@{8. 10. 1899}|)be}\mylabel{L00989h}  \newcommand{\dateiname}{L00989}\newcommand{\titel}{Arthur Schnitzler an Richard Beer-Hofmann, 8. 10. 1899}\newcommand{\editorInnen}{Martin Anton Müller und Gerd-Hermann Susen}%% latex-leseansicht-abspann.tex
%% Abspann für die Leseansicht.
%% Der Schalter \ifkorrekturansicht ist bereits durch den Vorspann gesetzt.

%% latex-abspann.tex
%% Gemeinsamer Abspann für Korrekturansicht und Leseansicht.
%% Setzt den Schalter \ifkorrekturansicht voraus (gesetzt in den
%% einbindenden Dateien latex-korrekturansicht-abspann.tex bzw.
%% latex-leseansicht-abspann.tex).
%% ---------------------------------------------------------------

\normalsize

% Das esempio-Environment wird nur in der Leseansicht benötigt
\ifkorrekturansicht\else
\newenvironment{esempio}[3]%
{
    \vspace{1.5ex}
    \rlap{\underline{#1}}
    \par
    \setlength{\parindent}{0cm}
    \nopagebreak
    \leftskip=#2cm
    \rightskip=#3cm
}
{
    \par
}
\fi

\doendnotes{C}
\bigskip
\vfill

\clearpage

\footnotesize

\ifkorrekturansicht
  \lohead{\textsc{register}}
\fi

% theindex-Environment neu definieren ohne reledmac
\makeatletter
\renewenvironment{theindex}{%
  \ifkorrekturansicht
    \section*{\indexname}%
  \else
    \subsubsection*{Index der erwähnten Entitäten}%
  \fi
  \setlength{\parindent}{0pt}%
  \setlength{\parskip}{0pt plus 0.3pt}%
  \let\item\@idxitem
}{%
  \ifkorrekturansicht\clearpage\fi
}
\makeatother

\IfFileExists{\jobname-pw.ind}{\input{\jobname-pw.ind}}{}

% Quellenangabe nur in der Leseansicht
\ifkorrekturansicht\else
% Fallback-Definitionen, falls die .tex-Datei \titel etc. nicht gesetzt hat
\providecommand{\titel}{}
\providecommand{\editorInnen}{}
\providecommand{\dateiname}{\jobname}

\vspace{3cm}

\vfill

\footnotesize
\textsc{Quelle}: \titel. Herausgegeben von {\editorInnen}. In: \emph{Arthur Schnitzler: Briefwechsel mit Autorinnen und Autoren}.
 Digitale Edition, https://schnitzler-briefe.acdh.oeaw.ac.at/{\dateiname}.html (Stand \today)
\fi

\end{document}


