%% latex-korrekturansicht-vorspann.tex
%% Vorspann für die Korrekturansicht.
%% Lädt die gemeinsame Datei latex-vorspann.tex mit gesetztem Schalter.

\newif\ifkorrekturansicht
\korrekturansichttrue

\input{../tex-inputs/latex-vorspann}


\section[Arthur Schnitzler an Richard Beer-Hofmann, 8. 10. 1899]{L00989 Arthur Schnitzler an Richard Beer-Hofmann, 8. 10. 1899}
\nopagebreak\mylabel{L00989v}
\rehead{ }\normalsize\beginnumbering\briefempfaengerindex{Beer-Hofmann, Richard@\textsc{Beer-Hofmann, Richard}!zzzSchnitzler, Arthur@\emph{von Arthur Schnitzler}!1899-10-081@{8. 10. 1899}|(be}
\toendnotes[C]{\smallbreak\pagebreak[2]}\Standort{YCGL, MSS 31.}
\physDesc{Brief, 1 Blatt, 3 Seiten, Umschlag, 1041 Zeichen
\newline{}Handschrift: 1) schwarze Tinte, deutsche Kurrent\hspace{1em}2) schwarze Tinte, lateinische Kurrent (\noindent{}Adresse)\hspace{1em}
\newline{}Versand: 1) Stempel: »\nobreak{}\oindex{Berlin@\textbf{Berlin}, \emph{P.PPLC}|pwk}Berlin, 8. 10. 99, 5–6N\nobreak{}«.   2) Stempel: »\nobreak{}\oindex{Sankt Michael@\textbf{Sankt Michael}, \emph{Bezirk (A.BZK)}|pwk}St. Mich{[}ae{]}l in
                                       Eppan, 10 10 99\nobreak{}«. }
\buchAbdrucke{\weitereDrucke{Arthur Schnitzler, Richard Beer-Hofmann: \emph{Briefwechsel 1891–1931}. Wien, Zürich: \emph{Europaverlag} 1992, S. 139.} }\toendnotes[C]{\smallbreak}\pstart{}{\pb}Herrn Dr. Richard Beer-Hofmann\pend{}\pstart{}St. Michael im Eppan\oindex{Sankt Michael@\textbf{Sankt Michael}, \emph{Bezirk (A.BZK)}|pw}\pend{}\pstart{}Tirol\oindex{Tirol@\textbf{Tirol}, \emph{A.ADM1}|pw}\pend{}{\bigskip}\vspace{1em}
\pstart
           \raggedleft{}{\pb}\textsc{Berlin}\oindex{Berlin@\textbf{Berlin}, \emph{P.PPLC}|pw}{ }8. X. 99.\pend
           \vspace{0.5em}
\pstart
           mein lieber Richard, das iſt entſetzlich, was dieſer Leo\pwindex{Feld, Leo 14.02.1869 – 05.09.1924@\textsc{Feld, Leo} (14.02.1869 – 05.09.1924), \emph{Schriftsteller/Schriftstellerin, Übersetzer/Übersetzerin, Dirigent/Dirigentin}|pw} wieder \label{K_L00989-1v}\edtext{durchmachen}{\lemma{\textnormal{\emph{durchmachen}}}\Cendnote{\textnormal{Er
                  hatte sich mit Olga Wohlbrück\pwindex{Wohlbrueck, Olga 05.07.1865 – 22.07.1943@\textsc{Wohlbrück, Olga} (05.07.1865 – 22.07.1943), \emph{Schriftsteller/Schriftstellerin, Schauspieler/Schauspielerin}|pwk} verlobt, die
                  beiden heirateten im März 1900 in Berlin\oindex{Berlin@\textbf{Berlin}, \emph{P.PPLC}|pwk}.}}}\label{K_L00989-1} muſs! Da kommen einem immer wieder dieſe alten Phraſen in
               den Mund, aber ich will ſie unterdrücken. Wa{\geminationn} kommen Sie
               nach Wien\oindex{Wien@\textbf{Wien}, \emph{A.ADM2}|pw}? Paul
                  Goldmann\pwindex{Goldmann, Paul 31.01.1865 – 25.09.1935@\textsc{Goldmann, Paul} (31.01.1865 – 25.09.1935), \emph{Schriftsteller/Schriftstellerin, Journalist/Journalistin}|pw} ko{\geminationm}t, ebenſo wie ich, Do{\geminationn}erſtg oder Freitag in Wien\oindex{Wien@\textbf{Wien}, \emph{A.ADM2}|pw} an – pardon – \uline{will} ankommen – ebenſo wie \uline{ich will}; er wird
               etwa 8 Tage bei mir wohnen. Ich denke, Sie {\pb}werden
               auch nicht mehr lang da unten oder da oben bleiben? Nun jedenfalls richten Sie ſichs
               wohl ſo ein, dſs Sie \strikeout{Rich}{ }Paul\pwindex{Goldmann, Paul 31.01.1865 – 25.09.1935@\textsc{Goldmann, Paul} (31.01.1865 – 25.09.1935), \emph{Schriftsteller/Schriftstellerin, Journalist/Journalistin}|pw} noch in Wien\oindex{Wien@\textbf{Wien}, \emph{A.ADM2}|pw} antreffen –?\pend
           
\pstart
           Ich habe geſtern dem Brahm\pwindex{Brahm, Otto 05.02.1856 – 28.11.1912@\textsc{Brahm, Otto} (05.02.1856 – 28.11.1912), \emph{Theaterleiter/Theaterleiterin, Regisseur/Regisseurin}|pw} die \textsc{Beatrice}\pwindex{Schleier der Beatrice. Schauspiel in fuenf Akten@\emph{Der Schleier der Beatrice. Schauspiel in fünf Akten}|pw}, mit guter Wirkung, glaub ich, vorgeleſen. Er hat kaum gemerkt, wie viel ich
               noch dran zu machen habe. Die ungeſtrichene Aufführg würde fünf Stunden {\pb}dauern.\pend
           
\pstart
           Ihre Ermahnung kam zu ſpät – ich hatte Brahm\pwindex{Brahm, Otto 05.02.1856 – 28.11.1912@\textsc{Brahm, Otto} (05.02.1856 – 28.11.1912), \emph{Theaterleiter/Theaterleiterin, Regisseur/Regisseurin}|pw}{ }ſchon eine »beſſere Meinung« beigebracht. So grüßt
               er Sie alſo weiter, \textsc{Kerr}\pwindex{Kerr, Alfred 25.12.1867 – 12.10.1948@\textsc{Kerr, Alfred} (25.12.1867 – 12.10.1948), \emph{Schriftsteller/Schriftstellerin, Kritiker/Kritikerin}|pw} desgleichen.\pend
           
\pstart
           – Hier friert man bereits un\textcolor{gray}{d} heizt ein und friert trotzdem.\pend
           
\pstart
           Leben Sie wohl und erlauben Sie mir mich auf die unſelige Mitgift\pwindex{Graf von Charolais. Ein Trauerspiel@\emph{Der Graf von Charolais. Ein Trauerspiel}|pw} zu freuen.\pend
           
\pstart
           Herzlichſt Ihr{\\[\baselineskip]}\spacefill\mbox{Arthur}\pend
           \leftskip=0em{}\selectlanguage{ngerman}\endnumbering\briefempfaengerindex{Beer-Hofmann, Richard@\textsc{Beer-Hofmann, Richard}!zzzSchnitzler, Arthur@\emph{von Arthur Schnitzler}!1899-10-081@{8. 10. 1899}|)be}\mylabel{L00989h}  \normalsize

\doendnotes{C}
\bigskip
\vfill

\clearpage

\footnotesize

\lohead{\textsc{register}}

% Definiere theindex-Environment komplett neu ohne reledmac
\makeatletter
\renewenvironment{theindex}{%
  \section*{\indexname}%
  \setlength{\parindent}{0pt}%
  \setlength{\parskip}{0pt plus 0.3pt}%
  \let\item\@idxitem
}{%
  \clearpage
}
\makeatother

\IfFileExists{\jobname-pw.ind}{\input{\jobname-pw.ind}}{}

\end{document}

      