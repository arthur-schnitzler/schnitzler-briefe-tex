%% latex-korrekturansicht-vorspann.tex
%% Vorspann für die Korrekturansicht.
%% Lädt die gemeinsame Datei latex-vorspann.tex mit gesetztem Schalter.

\newif\ifkorrekturansicht
\korrekturansichttrue

\input{../tex-inputs/latex-vorspann}


\section[Olga Schnitzler an Richard Beer-Hofmann, {[}19. 10. 1907{]}]{L01725 Olga Schnitzler an Richard Beer-Hofmann, {[}19. 10. 1907{]}}
\nopagebreak\mylabel{L01725v}
\rehead{ }\normalsize\beginnumbering\briefempfaengerindex{Beer-Hofmann, Richard@\textsc{Beer-Hofmann, Richard}!zzzSchnitzler, Olga@\emph{von Olga Schnitzler}!1907-10-192@{{[}19. 10. 1907{]}}|(be}
\toendnotes[C]{\smallbreak\pagebreak[2]}\Standort{YCGL, MSS 31.}
\physDesc{Briefkarte, , Umschlag, 221 Zeichen
\newline{}Handschrift: Bleistift, lateinische Kurrent
\newline{}Versand: ohne postalischen Übermittlungsvermerk }\toendnotes[C]{\smallbreak}\pstart{}{\pb}\textcolor{gray}{\textbf{O. S.}}\pend{}{\bigskip}\pstart{}{\pb}Herrn D\textsuperscript{r} Beer-Hofmann
               \pend{}{\bigskip}\vspace{1em}
\pstart
           {\pb}\textcolor{gray}{\textbf{O. S.}}\pend
           \vspace{0.5em}
\pstart
           Lieber Herr Doctor, wie schade! Aber wir haben Speidels\pwindex{Speidel, Felix 02.07.1875 – 1952-10-03@\textsc{Speidel, Felix} (02.07.1875 – 1952-10-03), \emph{Schriftsteller/Schriftstellerin, Verleger/Verlegerin}|pw}\pwindex{Speidel-Haeberle, Else 11.07.1877 – 21.07.1937@\textsc{Speidel-Haeberle, Else} (11.07.1877 – 21.07.1937), \emph{Schauspieler/Schauspielerin}|pw} getroffen, die kommen zu uns, nach dem
               Nachtmal liest er\pwindex{Speidel, Felix 02.07.1875 – 1952-10-03@\textsc{Speidel, Felix} (02.07.1875 – 1952-10-03), \emph{Schriftsteller/Schriftstellerin, Verleger/Verlegerin}|pwv} uns sein
                  {\pb}neues Stück\pwindex{Foehn@\emph{Föhn}|pwv} vor.\pend
           
\pstart
           Pech, Pech, Pech!\pend
           
\pstart
           Arth. dictiert, lässt Alle herzlichst grüssen.
               Ebenso\pend
           \pstart \spacefill\mbox{O.}\pend{}\selectlanguage{ngerman}\endnumbering\briefempfaengerindex{Beer-Hofmann, Richard@\textsc{Beer-Hofmann, Richard}!zzzSchnitzler, Olga@\emph{von Olga Schnitzler}!1907-10-192@{{[}19. 10. 1907{]}}|)be}\mylabel{L01725h}  \normalsize

\doendnotes{C}
\bigskip
\vfill

\clearpage

\footnotesize

\lohead{\textsc{register}}

% Definiere theindex-Environment komplett neu ohne reledmac
\makeatletter
\renewenvironment{theindex}{%
  \section*{\indexname}%
  \setlength{\parindent}{0pt}%
  \setlength{\parskip}{0pt plus 0.3pt}%
  \let\item\@idxitem
}{%
  \clearpage
}
\makeatother

\IfFileExists{\jobname-pw.ind}{\input{\jobname-pw.ind}}{}

\end{document}

      