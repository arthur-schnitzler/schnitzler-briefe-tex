%% latex-leseansicht-vorspann.tex
%% Vorspann für die Leseansicht.
%% Lädt die gemeinsame Datei latex-vorspann.tex mit nicht gesetztem Schalter.

\newif\ifkorrekturansicht
\korrekturansichtfalse

\input{../tex-inputs/latex-vorspann}


         
         \newcommand{\erwaehntePersonen}{Personen: Richard Beer-Hofmann, Felix Speidel, Else Speidel-Haeberle}
         \newcommand{\erwaehnteOrte}{Orte: Wien}
         \newcommand{\erwaehnteWerke}{Werke: Föhn}
               \section[Olga Schnitzler an Richard Beer-Hofmann, {[}19. 10. 1907{]}]{ Olga Schnitzler an Richard Beer-Hofmann,
               {[}19. 10. 1907{]}}\nopagebreak\mylabel{v}\rehead{ }\begin{ledgroupsized}[t]{13cm}\normalsize\beginnumbering \toendnotes[C]{\smallbreak\pagebreak[2]} \Standort{YCGL, MSS 31.}
\physDesc{Briefkarte, Umschlag
\newline{}Handschrift: Bleistift, lateinische Kurrent\newline{}Versand: ohne postalischen Übermittlungsvermerk }\toendnotes[C]{\smallbreak}\pstart{}{\pb}\textcolor{gray}{\textbf{O. S.}}\pend{}{\bigskip}\pstart{}{\pb}Herrn D\textsuperscript{r} Beer-Hofmann
               \pend{}{\bigskip}\pstart
           \noindent{}{\pb}\textcolor{gray}{\textbf{O. S.}}\pend
           \pstart
           Lieber Herr Doctor, wie schade! Aber wir haben Speidels\pwindex{Speidel, Felix 02.07.1875 – 1952-10-03@\textsc{Speidel, Felix} (02.07.1875 – 1952-10-03), \emph{Schriftsteller, Verleger}|pw}\pwindex{Speidel-Haeberle, Else 11.07.1877 – 21.07.1937@\textsc{Speidel-Haeberle, Else} (11.07.1877 – 21.07.1937), \emph{Schauspielerin}|pw} getroffen, die kommen zu uns, nach dem Nachtmal
               liest er\pwindex{Speidel, Felix 02.07.1875 – 1952-10-03@\textsc{Speidel, Felix} (02.07.1875 – 1952-10-03), \emph{Schriftsteller, Verleger}|pwv} uns sein {\pb}neues Stück\pwindex{Speidel, Felix 02.07.1875 – 1952-10-03@\textsc{Speidel, Felix} (02.07.1875 – 1952-10-03), \emph{Schriftsteller, Verleger}!FoehnNone@\strich\emph{Föhn} {[}None{]}|pwv} vor.\pend
           \pstart
           Pech, Pech, Pech!\pend
           \pstart
           Arth.\pwindex{Schnitzler, Arthur 15.05.1862 – 21.10.1931@\textsc{Schnitzler, Arthur} (15.05.1862 – 21.10.1931), \emph{Schriftsteller, Mediziner}|pw} dictiert, lässt Alle herzlichst grüssen.
               Ebenso\pend
           \pstart \spacefill\mbox{O.}\pend{}
         
         \endnumbering\mylabel{h}\end{ledgroupsized}  \newcommand{\dateiname}{L01725}\newcommand{\titel}{Olga Schnitzler an Richard Beer-Hofmann, [19. 10. 1907]}\newcommand{\editorInnen}{Martin Anton Müller und Gerd-Hermann Susen}%% latex-leseansicht-abspann.tex
%% Abspann für die Leseansicht.
%% Der Schalter \ifkorrekturansicht ist bereits durch den Vorspann gesetzt.

%% latex-abspann.tex
%% Gemeinsamer Abspann für Korrekturansicht und Leseansicht.
%% Setzt den Schalter \ifkorrekturansicht voraus (gesetzt in den
%% einbindenden Dateien latex-korrekturansicht-abspann.tex bzw.
%% latex-leseansicht-abspann.tex).
%% ---------------------------------------------------------------

\normalsize

% Das esempio-Environment wird nur in der Leseansicht benötigt
\ifkorrekturansicht\else
\newenvironment{esempio}[3]%
{
    \vspace{1.5ex}
    \rlap{\underline{#1}}
    \par
    \setlength{\parindent}{0cm}
    \nopagebreak
    \leftskip=#2cm
    \rightskip=#3cm
}
{
    \par
}
\fi

\doendnotes{C}
\bigskip
\vfill

\clearpage

\footnotesize

\ifkorrekturansicht
  \lohead{\textsc{register}}
\fi

% theindex-Environment neu definieren ohne reledmac
\makeatletter
\renewenvironment{theindex}{%
  \ifkorrekturansicht
    \section*{\indexname}%
  \else
    \subsubsection*{Index der erwähnten Entitäten}%
  \fi
  \setlength{\parindent}{0pt}%
  \setlength{\parskip}{0pt plus 0.3pt}%
  \let\item\@idxitem
}{%
  \ifkorrekturansicht\clearpage\fi
}
\makeatother

\IfFileExists{\jobname-pw.ind}{\input{\jobname-pw.ind}}{}

% Quellenangabe nur in der Leseansicht
\ifkorrekturansicht\else
% Fallback-Definitionen, falls die .tex-Datei \titel etc. nicht gesetzt hat
\providecommand{\titel}{}
\providecommand{\editorInnen}{}
\providecommand{\dateiname}{\jobname}

\vspace{3cm}

\vfill

\footnotesize
\textsc{Quelle}: \titel. Herausgegeben von {\editorInnen}. In: \emph{Arthur Schnitzler: Briefwechsel mit Autorinnen und Autoren}.
 Digitale Edition, https://schnitzler-briefe.acdh.oeaw.ac.at/{\dateiname}.html (Stand \today)
\fi

\end{document}


      