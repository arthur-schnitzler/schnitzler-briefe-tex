%% latex-korrekturansicht-vorspann.tex
%% Vorspann für die Korrekturansicht.
%% Lädt die gemeinsame Datei latex-vorspann.tex mit gesetztem Schalter.

\newif\ifkorrekturansicht
\korrekturansichttrue

\input{../tex-inputs/latex-vorspann}


\section[Hermann Bahr an Arthur Schnitzler, 22. 11. 1910]{L01985 Hermann Bahr an Arthur Schnitzler, 22. 11. 1910}
\nopagebreak\mylabel{L01985v}
\rehead{ }\normalsize\beginnumbering\briefempfaengerindex{Schnitzler, Arthur@\textsc{Schnitzler, Arthur}!zzzBahr, Hermann@\emph{von Hermann Bahr}!1910-11-222@{22. 11. 1910}|(be}
\toendnotes[C]{\smallbreak\pagebreak[2]}\Standort{CUL, Schnitzler, B 5b.}
\physDesc{Bildpostkarte, 152 Zeichen
\newline{}Handschrift: 1) schwarze Tinte, deutsche Kurrent\hspace{1em}2) schwarze Tinte, lateinische Kurrent (\noindent{}Adresse)\hspace{1em}
\newline{}Versand: 1) Stempel: »\nobreak{}\oindex{Frankfurt am Main@\textbf{Frankfurt am Main}, \emph{P.PPLA3}|pwk}Cöln – Frankfurt (M.)
                                       Bahnpost, 22. 11. 10\nobreak{}«.   2) mit Bleistift von unbekannter Hand die Straße der Anschrift
                                 gestrichen
\newline{}Ordnung: mit Bleistift von unbekannter Hand nummeriert:
                                    »170« }
\buchAbdrucke{\weitereDrucke{Hermann Bahr, Arthur Schnitzler: \emph{Briefwechsel, Aufzeichnungen, Dokumente (1891–1931)}. Göttingen: \emph{Wallstein} 2018, S. 446.} }\toendnotes[C]{\smallbreak}\pstart{}{\pb}Arthur Schnitzler\pend{}\pstart{}Wien XIII\oindex{XIII., Hietzing@\textbf{XIII., Hietzing}, \emph{A.ADM3}|pw}\pend{}\pstart{}Spöttelgasse 7\oindex{Edmund-Weiss-Gasse 7@\textbf{Edmund-Weiß-Gasse 7}, \emph{Wohngebäude (K.WHS)}|pw}\pend{}{\bigskip}
\pstart
           \noindent{}\centering{}{\pb}\textcolor{gray}{\textbf{Blick in das Beethoven\pwindex{Beethoven, Ludwig van 17.12.1770 – 26.03.1827@\textsc{Beethoven, Ludwig van} (17.12.1770 – 26.03.1827), \emph{Komponist/Komponistin}|pw}museum\oindex{Beethovenmuseum@\textbf{Beethovenmuseum}, \emph{Museum (K.MUS)}|pw} im 2. Stockwerk
                  mit dem Flügel und den Streichinstrumenten des Meisters.}}\pend
           \vspace{1em}
\pstart
           \noindent{}{\pb}Schönſten Dank für Deinen lieben Brief u. die
               herzlichſten Grüße an Dich u Deine Frau\pwindex{Schnitzler, Olga 17.01.1882 – 13.01.1970@\textsc{Schnitzler, Olga} (17.01.1882 – 13.01.1970), \emph{Schauspieler/Schauspielerin, Sänger/Sängerin}|pwv}\textcolor{gray}{!}\pend
           \pstart Herzlichſt \spacefill\mbox{Hermann}\pend{}
\pstart
           22. 11. 10\pend
           \selectlanguage{ngerman}\endnumbering\briefempfaengerindex{Schnitzler, Arthur@\textsc{Schnitzler, Arthur}!zzzBahr, Hermann@\emph{von Hermann Bahr}!1910-11-222@{22. 11. 1910}|)be}\mylabel{L01985h}  \normalsize

\doendnotes{C}
\bigskip
\vfill

\clearpage

\footnotesize

\lohead{\textsc{register}}

% Definiere theindex-Environment komplett neu ohne reledmac
\makeatletter
\renewenvironment{theindex}{%
  \section*{\indexname}%
  \setlength{\parindent}{0pt}%
  \setlength{\parskip}{0pt plus 0.3pt}%
  \let\item\@idxitem
}{%
  \clearpage
}
\makeatother

\IfFileExists{\jobname-pw.ind}{\input{\jobname-pw.ind}}{}

\end{document}

      