%% latex-leseansicht-vorspann.tex
%% Vorspann für die Leseansicht.
%% Lädt die gemeinsame Datei latex-vorspann.tex mit nicht gesetztem Schalter.

\newif\ifkorrekturansicht
\korrekturansichtfalse

\input{../tex-inputs/latex-vorspann}


\section[Paul Goldmann an Arthur Schnitzler, Olga und Elisabeth Gussmann, 27. 9. {[}1901{]}]{L03086 Paul Goldmann an Arthur Schnitzler, Olga und Elisabeth
               Gussmann,  27. 9. [1901]}
\nopagebreak\mylabel{L03086v}
\rehead{ }\normalsize\beginnumbering\briefempfaengerindex{Steinrück, Elisabeth@\textsc{Steinrück, Elisabeth}!zzzGoldmann, Paul@\emph{von Paul Goldmann}!1901-09-272@{27. 9. [1901]}|(be}\briefempfaengerindex{Schnitzler, Olga@\textsc{Schnitzler, Olga}!zzzGoldmann, Paul@\emph{von Paul Goldmann}!1901-09-272@{27. 9. [1901]}|(be}\briefempfaengerindex{Schnitzler, Arthur@\textsc{Schnitzler, Arthur}!zzzGoldmann, Paul@\emph{von Paul Goldmann}!1901-09-272@{27. 9. [1901]}|(be}
\toendnotes[C]{\smallbreak\pagebreak[2]}
\correspDesc{Versand  durch Paul Goldmann am 27. 9. [1901] in Berlin
\newline{}Erhalt  durch Arthur Schnitzler, Olga Gussmann, Elisabeth Gussmann im Zeitraum [28. 9. 1901
                  – 2. 10. 1901?] in Wien}\toendnotes[C]{\smallbreak}
\Standort{DLA, A:Schnitzler, HS.NZ85.1.3171.}
\physDesc{Brief, 2 Blätter, 5 Seiten, 2431 Zeichen
\newline{}Handschrift: blaue Tinte, deutsche Kurrent
\newline{}Schnitzler: mit Bleistift das Jahr »901« vermerkt }\toendnotes[C]{\smallbreak}
\pstart
           \centering{}{\pb}Berlin\oindex{Berlin@\textbf{Berlin}, \emph{Hauptstadt}|pw}, 27. September.\pend
           
\pstart{}Mein lieber Freund,\pend\vspace{0.5em}
\pstart
           Bitte übermittle dieſen Brief an{ }ſeine Adreſſe, da Du mir auf meine \label{K_L03086-1v}\edtext{Frage}{\lemma{\textnormal{\emph{Frage}}}\Cendnote{\textnormal{Siehe XXXX Auszeichnungsfehler: Dokument L03085 nicht gefunden.
               }}}\label{K_L03086-1}, wo die Mädeln jetzt wohnen, noch nicht geantwortet haſt.\pend
           
\pstart
           Herzlichſt Dein {\\[\baselineskip]}\spacefill\mbox{Paul Goldmnn}\pend
           \leftskip=0em{}
\pstart
           \noindent{}Iſt \label{K_L03086-2v}\edtext{\textsc{Richard\pwindex{Beer-Hofmann, Richard 11.\,7.\,1866 Wien – 26.\,9.\,1945 New York City@\textsc{Beer-Hofmann, Richard} (11.\,7.\,1866 Wien – 26.\,9.\,1945 New York City), \emph{Schriftsteller}|pw}}}{\lemma{\textnormal{\emph{Richard}}}\Cendnote{\textnormal{vermutlich gemeint: aus dessen
                     Sommerquartier in Pörtschach\oindex{Pörtschach am Wörthersee@\textbf{Pörtschach am Wörthersee}|pwk}, von wo er
                        Mitte September nach Wien\oindex{Wien@\textbf{Wien}, \emph{Verwaltungsgebiet}|pwk}
                     zurückgekehrt war}}}\label{K_L03086-2}{ }ſchon in Wien\oindex{Wien@\textbf{Wien}, \emph{Verwaltungsgebiet}|pw}?\pend
           \selectlanguage{ngerman}\vspace{1em}
\pstart
           \raggedleft{}{\pb}\textcolor{gray}{\textbf{DESSAUERSTRASSE 19}}\oindex{Dessauer Straße@\textbf{Dessauer Straße}, \emph{Straße}|pw}\pend
           
\pstart
           Berlin\oindex{Berlin@\textbf{Berlin}, \emph{Hauptstadt}|pw}, 27. September.\pend
           
\pstart\center{}Liebes Fräulein Olga,\pend\vspace{0.5em}
\pstart
           Endlich eine freie halbe Stunde! Gleich hole ich mir Ihren Brief heraus aus
                  de\textcolor{gray}{m} Paket, das auf meinen Pult{ }ſich aufgehäuft hat. Sie haben
               mir{ }ſo lieb geſchrieben und haben mir damit eine{ }ſo große Freude gemacht! (Das »Sie«
               iſt immer Mehrzahl und bedeutet hier \textsc{Olga} und \textsc{Liesl}). Ich habe \textsc{Arthur} bereits
               gebeten, Ihnen zu danken. Da er dies, wie ich vorausſetze, vergeſſen hat,{ }ſo danke
               ich Ihnen hier noch einmal.\pend
           
\pstart
           Liebes Fräulein \textsc{Olga} (jetzt in der Einzahl): Daß Sie keine
               Berichte über \label{K_L03086-3v}\edtext{Kaiſer\pwindex{Wilhelm II. von Preußen 27.\,1.\,1859 Potsdam – 4.\,6.\,1941 Gemeente Utrechtse Heuvelrug@\textsc{Wilhelm II. von Preußen} (27.\,1.\,1859 Potsdam – 4.\,6.\,1941 Gemeente Utrechtse Heuvelrug), \emph{Kaiser}|pwv}\pwindex{Nikolaus II. von Russland 6.\,5.\,1868 Pushkin – 17.\,7.\,1918 Jekaterinburg@\textsc{Nikolaus II. von Russland} (6.\,5.\,1868 Pushkin – 17.\,7.\,1918 Jekaterinburg), \emph{Zar}|pwv}zuſammenkünfte}{\lemma{\textnormal{\emph{Kaiserzusammenkünfte}}}\Cendnote{\textnormal{Bezug auf
                  die Zusammenkunft von Wilhelm II.\pwindex{Wilhelm II. von Preußen 27.\,1.\,1859 Potsdam – 4.\,6.\,1941 Gemeente Utrechtse Heuvelrug@\textsc{Wilhelm II. von Preußen} (27.\,1.\,1859 Potsdam – 4.\,6.\,1941 Gemeente Utrechtse Heuvelrug), \emph{Kaiser}|pwk} und Nikolaus II.\pwindex{Nikolaus II. von Russland 6.\,5.\,1868 Pushkin – 17.\,7.\,1918 Jekaterinburg@\textsc{Nikolaus II. von Russland} (6.\,5.\,1868 Pushkin – 17.\,7.\,1918 Jekaterinburg), \emph{Zar}|pwk} am 13. 9. 1901 in Danzig\oindex{Danzig@\textbf{Danzig}, \emph{Verwaltungsgebiet}|pwk}}}}\label{K_L03086-3}
               leſen, thut mir leid. Erſtens war mein \label{K_L03086-4v}\edtext{Bericht\pwindex{Kaiser Wilhelm in Danzig@\emph{Kaiser Wilhelm in Danzig}|pwv}}{\lemma{\textnormal{\emph{Bericht}}}\Cendnote{\textnormal{Vermutlich [Paul Goldmann\pwindex{Goldmann, Paul 31.\,1.\,1865 Breslau – 25.\,9.\,1935 Wien@\textsc{Goldmann, Paul} (31.\,1.\,1865 Breslau – 25.\,9.\,1935 Wien), \emph{Schriftsteller, Journalist}|pwk}]: \emph{Kaiser
                        Wilhelm in Danzig}\pwindex{Kaiser Wilhelm in Danzig@\emph{Kaiser Wilhelm in Danzig}|pwk}. In: \emph{Neue Freie
                        Presse}\pwindex{Neue Freie Presse@\emph{Neue Freie Presse}|pwk}, Nr. 13.311, 15. 9. 1901,
                     Morgenblatt, S. 10.
               }}}\label{K_L03086-4} hübſch. Zweitens iſt die Nichtachtung {\pb}der
               Politik, wie{ }ſie unter unſeren Wien\oindex{Wien@\textbf{Wien}, \emph{Verwaltungsgebiet}|pw}er Freunden
               beſteht, ein Fehler. Alles Menſchliche iſt intereſſant; und \strikeout{\textcolor{gray}{poli}} eine Kaiſerzuſammenkunft bietet nicht weniger
               menſchliches Intereſſe als das erſte Auftreten von Fräulein \textsc{Medelsky}\pwindex{Medelsky, Lotte 18.\,5.\,1880 Wien – 4.\,12.\,1960 Nussdorf am Attersee@\textsc{Medelsky, Lotte} (18.\,5.\,1880 Wien – 4.\,12.\,1960 Nussdorf am Attersee), \emph{Schauspielerin}|pw} im Volkstheater\orgindex{Volkstheater@Volkstheater|pw}. Geſchichte betreiben
               unſere Freunde. Aber was iſt Politik Anderes, als Geſchichte, die wir miterleben? Die
               großen Frauen der Renaiſſance und in Frankreich\oindex{Frankreich@\textbf{Frankreich}|pw}
               haben{ }ſich mit Politik immer viel beſchäftigt und haben viel davon \strikeout{verſtanden} verſtanden.\pend
           
\pstart
           Das \label{K_L03086-5v}\edtext{Feuilleton\pwindex{Lesser, Max *~12.\,8.\,1850@\textsc{Lesser, Max} (*~12.\,8.\,1850), \emph{Journalist}!Götterdämmerung@\strich\emph{Götterdämmerung}|pwv} von \textsc{Lesser}\pwindex{Lesser, Max *~12.\,8.\,1850@\textsc{Lesser, Max} (*~12.\,8.\,1850), \emph{Journalist}|pw}}{\lemma{\textnormal{\emph{Feuilleton von Lesser}}}\Cendnote{\textnormal{Max Lesser\pwindex{Lesser, Max *~12.\,8.\,1850@\textsc{Lesser, Max} (*~12.\,8.\,1850), \emph{Journalist}|pwk}: \emph{Götterdämmerung}\pwindex{Lesser, Max *~12.\,8.\,1850@\textsc{Lesser, Max} (*~12.\,8.\,1850), \emph{Journalist}!Götterdämmerung@\strich\emph{Götterdämmerung}|pwk}. In: \emph{Neues Wiener Tagblatt}\pwindex{Neues Wiener Tagblatt@\emph{Neues Wiener Tagblatt}|pwk}, Jg. 35, Nr. 243, 5. 9. 1901,
                     S. 1–3. Darin heißt es: »Wir haben Gerhart Hauptmann\pwindex{Hauptmann, Gerhart 15.\,11.\,1862 Szczawno-Zdrój – 6.\,6.\,1946 Jagniątków@\textsc{Hauptmann, Gerhart} (15.\,11.\,1862 Szczawno-Zdrój – 6.\,6.\,1946 Jagniątków), \emph{Schriftsteller}|pw} und Helene Böhlau\pwindex{Böhlau, Helene 22.\,11.\,1856 Weimar – 26.\,3.\,1940 Augsburg@\textsc{Böhlau, Helene} (22.\,11.\,1856 Weimar – 26.\,3.\,1940 Augsburg), \emph{Schriftstellerin}|pw}, wir haben Hugo v.
                        Hofmannsthal\pwindex{Hofmannsthal, Hugo von 1.\,2.\,1874 Wien – 15.\,7.\,1929 Rodaun@\textsc{Hofmannsthal, Hugo von} (1.\,2.\,1874 Wien – 15.\,7.\,1929 Rodaun), \emph{Schriftsteller}|pw} und Peter
                     Altenberg\pwindex{Altenberg, Peter 9.\,3.\,1859 Wien – 8.\,1.\,1919 ebd.@\textsc{Altenberg, Peter} (9.\,3.\,1859 Wien – 8.\,1.\,1919 ebd.), \emph{Schriftsteller}|pw}.« Das wurde von der Redaktion mit einer Fußnote ergänzt:
                     »Der Verfaſſer gibt mit dieſer Auswahl einer perſönlichen Vorliebe
                     Ausdruck, die zum Theil wohl auf{ }ſtarken Widerſpruch{ }ſtoßen
                  wird.«}}}\label{K_L03086-5} habe ich nicht geleſen. Er iſt perſönlich ein braver Menſch.
               Meinetwegen alſo{ }ſoll er für \textsc{Altenberg}\pwindex{Altenberg, Peter 9.\,3.\,1859 Wien – 8.\,1.\,1919 ebd.@\textsc{Altenberg, Peter} (9.\,3.\,1859 Wien – 8.\,1.\,1919 ebd.), \emph{Schriftsteller}|pw}{ }ſchwärmen und{ }ſogar für \textsc{Hoffmannsthal}\pwindex{Hofmannsthal, Hugo von 1.\,2.\,1874 Wien – 15.\,7.\,1929 Rodaun@\textsc{Hofmannsthal, Hugo von} (1.\,2.\,1874 Wien – 15.\,7.\,1929 Rodaun), \emph{Schriftsteller}|pw}. Von Letzterem werden wir im \label{K_L03086-6v}\edtext{»Deutſchen Theater\orgindex{Deutsches Theater Berlin@Deutsches Theater Berlin|pw}« ein Versdrama\pwindex{Hofmannsthal, Hugo von 1.\,2.\,1874 Wien – 15.\,7.\,1929 Rodaun@\textsc{Hofmannsthal, Hugo von} (1.\,2.\,1874 Wien – 15.\,7.\,1929 Rodaun), \emph{Schriftsteller}!Turm. Ein Trauerspiel@\strich\emph{Der Turm. Ein Trauerspiel}|pwv}}{\lemma{\textnormal{\emph{»Deutschen … Versdrama}}}\Cendnote{\textnormal{Dazu kam es nicht. Gemeint sein dürfte
                  die Bearbeitung von Calderón de la Barcas\pwindex{Calderón de la Barca, Pedro 17.\,1.\,1600 Madrid – 25.\,5.\,1681 ebd.@\textsc{Calderón de la Barca, Pedro} (17.\,1.\,1600 Madrid – 25.\,5.\,1681 ebd.), \emph{Schriftsteller}|pwk}{ }\emph{Das Leben ein Traum}\pwindex{Calderón de la Barca, Pedro 17.\,1.\,1600 Madrid – 25.\,5.\,1681 ebd.@\textsc{Calderón de la Barca, Pedro} (17.\,1.\,1600 Madrid – 25.\,5.\,1681 ebd.), \emph{Schriftsteller}!vida es sueño@\strich\emph{La vida es sueño}|pwk}, die Jahrzehnte später
                  zu \emph{Der Turm}\pwindex{Hofmannsthal, Hugo von 1.\,2.\,1874 Wien – 15.\,7.\,1929 Rodaun@\textsc{Hofmannsthal, Hugo von} (1.\,2.\,1874 Wien – 15.\,7.\,1929 Rodaun), \emph{Schriftsteller}!Turm. Ein Trauerspiel@\strich\emph{Der Turm. Ein Trauerspiel}|pwk} werden sollte.}}}\label{K_L03086-6} zu{ }ſehen
               bekommen. Ich freue mich{ }ſchon rieſig.\pend
           
\pstart
           {\pb}Daß Ihr \label{K_L03086-7v}\edtext{Vater\pwindex{Gussmann, Rudolf 5.\,3.\,1842 Veprovac – 24.\,1.\,1921 Wien@\textsc{Gussmann, Rudolf} (5.\,3.\,1842 Veprovac – 24.\,1.\,1921 Wien), \emph{Handelsagent}|pwv}{ }ſich{ }ſo abſcheulich
                  benimmt}{\lemma{\textnormal{\emph{Vater … benimmt}}}\Cendnote{\textnormal{Siehe A. S.: \emph{Tagebuch}, 6. 9. 1901.
               }}}\label{K_L03086-7}, thut mir unendlich leid. Kann man da gar nichts machen? \textsc{Arthur}{ }ſoll den Prozeß nur jedenfalls einleiten. Ich bedaure namentlich, daß
                  \uline{ich} in der Angelegenheit{ }ſo gar nicht zu Hilfe
               kommen kann. Zum Beiſpiel, wenn ich eine Million hätte, wäre das{ }ſehr einfach. Bitte,
               warum hab’ ich keine Million?\pend
           
\pstart
           Diese neuen Kleider müſſen herrlich{ }ſein. Beſonders, wenn ich den himmelblauen
               Gürtel{ }ſehen könnte, es thäte meinem Herzen wohl!\pend
           
\pstart
           Ich denke oft und herzlich an Sie (wieder Mehrzahl). \label{K_L03086-8v}\edtext{\textsc{Welsberg\oindex{Welsberg-Taisten@\textbf{Welsberg-Taisten}, \emph{Verwaltungsgebiet}|pw}}}{\lemma{\textnormal{\emph{Welsberg}}}\Cendnote{\textnormal{Siehe XXXX Auszeichnungsfehler: Dokument L03064 nicht gefunden.
               }}}\label{K_L03086-8} liegt fern. Ich lebe wieder mein elendes Leben und bin unbeſchreiblich einſam
               in dieſer kalten Stadt\oindex{Berlin@\textbf{Berlin}, \emph{Hauptstadt}|pwv}, in der
               Niemand mich mag, Keiner und Keine.\pend
           
\pstart
           {\pb}Liebes Fräulein \textsc{Olga},
               kommen Sie bald mit dem \textsc{Arthur} nach Berlin\oindex{Berlin@\textbf{Berlin}, \emph{Hauptstadt}|pw},{ }ſchreiben Sie mir bis dahin noch manchen lieben Brief
               und{ }ſeien Sie herzlich gegrüßt von\pend
           
\pstart
           Ihrem ergebenen{\\[\baselineskip]}\spacefill\mbox{Dr. Paul Goldmann.}\pend
           \leftskip=0em{}\selectlanguage{ngerman}\endnumbering\briefempfaengerindex{Steinrück, Elisabeth@\textsc{Steinrück, Elisabeth}!zzzGoldmann, Paul@\emph{von Paul Goldmann}!1901-09-272@{27. 9. [1901]}|)be}\briefempfaengerindex{Schnitzler, Olga@\textsc{Schnitzler, Olga}!zzzGoldmann, Paul@\emph{von Paul Goldmann}!1901-09-272@{27. 9. [1901]}|)be}\briefempfaengerindex{Schnitzler, Arthur@\textsc{Schnitzler, Arthur}!zzzGoldmann, Paul@\emph{von Paul Goldmann}!1901-09-272@{27. 9. [1901]}|)be}\mylabel{L03086h}  \newcommand{\dateiname}{L03086}\newcommand{\titel}{Paul Goldmann an Arthur Schnitzler, Olga und Elisabeth Gussmann, 27. 9. [1901]}\newcommand{\editorInnen}{Martin Anton Müller und Laura Untner}%% latex-leseansicht-abspann.tex
%% Abspann für die Leseansicht.
%% Der Schalter \ifkorrekturansicht ist bereits durch den Vorspann gesetzt.

%% latex-abspann.tex
%% Gemeinsamer Abspann für Korrekturansicht und Leseansicht.
%% Setzt den Schalter \ifkorrekturansicht voraus (gesetzt in den
%% einbindenden Dateien latex-korrekturansicht-abspann.tex bzw.
%% latex-leseansicht-abspann.tex).
%% ---------------------------------------------------------------

\normalsize

% Das esempio-Environment wird nur in der Leseansicht benötigt
\ifkorrekturansicht\else
\newenvironment{esempio}[3]%
{
    \vspace{1.5ex}
    \rlap{\underline{#1}}
    \par
    \setlength{\parindent}{0cm}
    \nopagebreak
    \leftskip=#2cm
    \rightskip=#3cm
}
{
    \par
}
\fi

\doendnotes{C}
\bigskip
\vfill

\clearpage

\footnotesize

\ifkorrekturansicht
  \lohead{\textsc{register}}
\fi

% theindex-Environment neu definieren ohne reledmac
\makeatletter
\renewenvironment{theindex}{%
  \ifkorrekturansicht
    \section*{\indexname}%
  \else
    \subsubsection*{Index der erwähnten Entitäten}%
  \fi
  \setlength{\parindent}{0pt}%
  \setlength{\parskip}{0pt plus 0.3pt}%
  \let\item\@idxitem
}{%
  \ifkorrekturansicht\clearpage\fi
}
\makeatother

\IfFileExists{\jobname-pw.ind}{\input{\jobname-pw.ind}}{}

% Quellenangabe nur in der Leseansicht
\ifkorrekturansicht\else
% Fallback-Definitionen, falls die .tex-Datei \titel etc. nicht gesetzt hat
\providecommand{\titel}{}
\providecommand{\editorInnen}{}
\providecommand{\dateiname}{\jobname}

\vspace{3cm}

\vfill

\footnotesize
\textsc{Quelle}: \titel. Herausgegeben von {\editorInnen}. In: \emph{Arthur Schnitzler: Briefwechsel mit Autorinnen und Autoren}.
 Digitale Edition, https://schnitzler-briefe.acdh.oeaw.ac.at/{\dateiname}.html (Stand \today)
\fi

\end{document}


