%% latex-leseansicht-vorspann.tex
%% Vorspann für die Leseansicht.
%% Lädt die gemeinsame Datei latex-vorspann.tex mit nicht gesetztem Schalter.

\newif\ifkorrekturansicht
\korrekturansichtfalse

\input{../tex-inputs/latex-vorspann}


         
         \renewcommand{\erwaehntePersonen}{Personen: Peter Altenberg, Richard Beer-Hofmann, Helene Böhlau, Pedro Calderón de la Barca, Rudolf Gussmann, Gerhart Hauptmann, Hugo von Hofmannsthal, Max Lesser, Lotte Medelsky,  Nikolaus II. von Russland, Olga Schnitzler, Elisabeth Steinrück,  Wilhelm II. von Preußen}
         \renewcommand{\erwaehnteInstitutionen}{Institutionen: Deutsches Theater Berlin, Volkstheater}
         \renewcommand{\erwaehnteOrte}{Orte: Berlin, Danzig, Dessauer Straße, Frankreich, Pörtschach, Welsberg-Taisten, Wien}
         \renewcommand{\erwaehnteWerke}{Werke: Das Leben ein Traum, Der Turm. Ein Trauerspiel, Götterdämmerung, Kaiser Wilhelm in Danzig, Neue Freie Presse, Neues Wiener Tagblatt}
               \section[Paul Goldmann an Arthur Schnitzler, Olga und Elisabeth Gussmann, 27. 9. {[}1901{]}]{ Paul Goldmann an Arthur Schnitzler, Olga und Elisabeth
               Gussmann, 27. 9. {[}1901{]}}\nopagebreak\mylabel{v}\rehead{ }\begin{ledgroupsized}[t]{13cm}\normalsize\beginnumbering \toendnotes[C]{\smallbreak\pagebreak[2]} \Standort{DLA, A:Schnitzler, HS.NZ85.1.3171.}
\physDesc{Brief, 2 Blätter, 5 Seiten, 2431 Zeichen
\newline{}Handschrift: blaue Tinte, deutsche Kurrent
\newline{}Schnitzler: mit Bleistift das Jahr »{[}1{]}901« vermerkt }\toendnotes[C]{\smallbreak}\pstart
           \centering{}{\pb}Berlin\oindex{Berlin@\textbf{Berlin}|pw}, 27. September.\pend
           \pstart{}Mein lieber Freund,\pend\pstart
           Bitte übermittle dieſen Brief an ſeine Adreſſe, da Du mir auf meine \label{K_L03086-3v}\edtext{Frage}{\lemma{\textnormal{\emph{Frage}}}\Cendnote{\textnormal{siehe Paul Goldmann an Arthur Schnitzler, 23. 9. [1901]}}}\label{K_L03086-3h}, wo die Mädeln jetzt wohnen, noch nicht geantwortet haſt.\pend
           \pstart
           Herzlichſt Dein {\\[\baselineskip]}\spacefill\mbox{Paul Goldmnn}\pend
           \leftskip=0em{}\pstart
           \noindent{}Iſt \label{K_L03086-2v}\edtext{\textsc{Richard\pwindex{Beer-Hofmann, Richard 1866-07-11 – 1945-09-26@\textsc{Beer-Hofmann, Richard} (1866-07-11 – 1945-09-26), \emph{Schriftsteller}|pw}}}{\lemma{\textnormal{\emph{Richard}}}\Cendnote{\textnormal{vermutlich gemeint: aus dessen
                     Sommerquartier in Pörtschach\oindex{Poertschach@\textbf{Pörtschach}|pwk}, von wo er
                        Mitte September nach Wien\oindex{Wien@\textbf{Wien}|pwk}
                     zurückgekehrt war}}}\label{K_L03086-2h} ſchon in Wien\oindex{Wien@\textbf{Wien}|pw}? \pend
           \pstart
           \noindent{}\raggedleft{}{\pb}\textcolor{gray}{\textbf{DESSAUERSTRASSE 19}}\oindex{Dessauer Strasse@\textbf{Dessauer Straße}|pw}\pend
           \pstart
           Berlin\oindex{Berlin@\textbf{Berlin}|pw}, 27. September.\pend
           \pstart\center{}Liebes Fräulein Olga,\pend\pstart
           Endlich eine freie halbe Stunde! Gleich hole ich mir Ihren Brief heraus aus
                  de\textcolor{gray}{m} Paket, das auf meinen Pult ſich aufgehäuft hat. Sie haben
               mir ſo lieb geſchrieben und haben mir damit eine ſo große Freude gemacht! (Das »Sie«
               iſt immer Mehrzahl und bedeutet hier \textsc{Olga} und \textsc{Liesl}). Ich habe \textsc{Arthur} bereits
               gebeten, Ihnen zu danken. Da er dies, wie ich vorausſetze, vergeſſen hat, ſo danke
               ich Ihnen hier noch einmal.\pend
           \pstart
           Liebes Fräulein \textsc{Olga} (jetzt in der Einzahl): Daß Sie keine
               Berichte über \label{K_L03086-141v}\edtext{Kaiſer\pwindex{Wilhelm II. von Preussen 27.1.1859 – 4.6.1941@\textsc{Wilhelm II. von Preußen} (27.1.1859 – 4.6.1941), \emph{Kaiser}|pwv}\pwindex{Nikolaus II. von Russland 1868-05-06 – 1918-07-17@\textsc{Nikolaus II. von Russland} (1868-05-06 – 1918-07-17), \emph{Zar}|pwv}zuſammenkünfte}{\lemma{\textnormal{\emph{Kaiſerzuſammenkünfte}}}\Cendnote{\textnormal{Bezug auf
                  die Zusammenkunft von Wilhelm II.\pwindex{Wilhelm II. von Preussen 27.1.1859 – 4.6.1941@\textsc{Wilhelm II. von Preußen} (27.1.1859 – 4.6.1941), \emph{Kaiser}|pwk} und Nikolaus II.\pwindex{Nikolaus II. von Russland 1868-05-06 – 1918-07-17@\textsc{Nikolaus II. von Russland} (1868-05-06 – 1918-07-17), \emph{Zar}|pwk} am 13. 9. 1901 in Danzig\oindex{Danzig@\textbf{Danzig}|pwk}}}}\label{K_L03086-141h}
               leſen, thut mir leid. Erſtens war mein \label{K_L03086-222v}\edtext{Bericht\pwindex{Kaiser Wilhelm in Danzig1901-09-15@\emph{Kaiser Wilhelm in Danzig} {[}1901-09-15{]}|pwv}}{\lemma{\textnormal{\emph{Bericht}}}\Cendnote{\textnormal{vermutlich [Paul Goldmann\pwindex{Goldmann, Paul 31.01.1865 – 25.09.1935@\textsc{Goldmann, Paul} (31.01.1865 – 25.09.1935), \emph{Schriftsteller, Journalist}|pwk}]: \emph{Kaiser
                        Wilhelm in Danzig}\pwindex{Kaiser Wilhelm in Danzig1901-09-15@\emph{Kaiser Wilhelm in Danzig} {[}1901-09-15{]}|pwk}. In: \emph{Neue Freie
                        Presse}\pwindex{Neue Freie Presse1864 – 1939@\emph{Neue Freie Presse} {[}1864 – 1939{]}|pwk}, Nr. 13311, 15. 9. 1901,
                     Morgenblatt, S. 10}}}\label{K_L03086-222h} hübſch. Zweitens iſt die Nichtachtung {\pb}der
               Politik, wie ſie unter unſeren Wien\oindex{Wien@\textbf{Wien}|pw}er Freunden
               beſteht, ein Fehler. Alles Menſchliche iſt intereſſant; und \strikeout{\textcolor{gray}{poli}} eine Kaiſerzuſammenkunft bietet nicht weniger
               menſchliches Intereſſe als das erſte Auftreten von Fräulein \textsc{Medelsky}\pwindex{Medelsky, Lotte 18.05.1880 – 1960-12-04@\textsc{Medelsky, Lotte} (18.05.1880 – 1960-12-04), \emph{Schauspielerin}|pw} im Volkstheater\orgindex{Volkstheater@Volkstheater|pw}. Geſchichte betreiben
               unſere Freunde. Aber was iſt Politik Anderes, als Geſchichte, die wir miterleben? Die
               großen Frauen der Renaiſſance und in Frankreich\oindex{Frankreich@\textbf{Frankreich}|pw}
               haben ſich mit Politik immer viel beſchäftigt und haben viel davon \strikeout{verſtanden} verſtanden.\pend
           \pstart
           Das \label{K_L03086-1v}\edtext{Feuilleton\pwindex{Lesser, Max *~1850-08-12@\textsc{Lesser, Max} (*~1850-08-12), \emph{Journalist}!Goetterdaemmerung1901-09-05@\strich\emph{Götterdämmerung} {[}1901-09-05{]}|pwv} von \textsc{Lesser}\pwindex{Lesser, Max *~1850-08-12@\textsc{Lesser, Max} (*~1850-08-12), \emph{Journalist}|pw}}{\lemma{\textnormal{\emph{Feuilleton von Lesser}}}\Cendnote{\textnormal{Max Lesser\pwindex{Lesser, Max *~1850-08-12@\textsc{Lesser, Max} (*~1850-08-12), \emph{Journalist}|pwk}: \emph{Götterdämmerung}\pwindex{Lesser, Max *~1850-08-12@\textsc{Lesser, Max} (*~1850-08-12), \emph{Journalist}!Goetterdaemmerung1901-09-05@\strich\emph{Götterdämmerung} {[}1901-09-05{]}|pwk}. In: \emph{Neues Wiener Tagblatt}\pwindex{?? Werk@Nicht ermittelte Verfasserinnen und Verfasser!Neues Wiener Tagblatt1867 – 1945@\emph{Neues Wiener Tagblatt} {[}1867 – 1945{]}|pwk}, Jg. 35, Nr. 243, 5. 9. 1901,
                     S. 1–3. Darin heißt es: »Wir haben Gerhart Hauptmann\pwindex{Hauptmann, Gerhart 15.11.1862 – 06.06.1946@\textsc{Hauptmann, Gerhart} (15.11.1862 – 06.06.1946), \emph{Schriftsteller}|pw} und Helene Böhlau\pwindex{Boehlau, Helene 22.11.1856 – 26.03.1940@\textsc{Böhlau, Helene} (22.11.1856 – 26.03.1940), \emph{Schriftstellerin}|pw}, wir haben Hugo v.
                        Hofmannsthal\pwindex{Hofmannsthal, Hugo von 1874-02-01 – 1929-07-15@\textsc{Hofmannsthal, Hugo von} (1874-02-01 – 1929-07-15), \emph{Schriftsteller}|pw} und Peter
                     Altenberg\pwindex{Altenberg, Peter 09.03.1859 – 08.01.1919@\textsc{Altenberg, Peter} (09.03.1859 – 08.01.1919), \emph{Schriftsteller}|pw}.« Das wurde von der Redaktion mit einer Fußnote ergänzt:
                     »Der Verfaſſer gibt mit dieſer Auswahl einer perſönlichen Vorliebe
                     Ausdruck, die zum Theil wohl auf ſtarken Widerſpruch ſtoßen
                  wird.«}}}\label{K_L03086-1h} habe ich nicht geleſen. Er iſt perſönlich ein braver Menſch.
               Meinetwegen alſo ſoll er für \textsc{Altenberg}\pwindex{Altenberg, Peter 09.03.1859 – 08.01.1919@\textsc{Altenberg, Peter} (09.03.1859 – 08.01.1919), \emph{Schriftsteller}|pw} ſchwärmen und ſogar für \textsc{Hoffmannsthal}\pwindex{Hofmannsthal, Hugo von 1874-02-01 – 1929-07-15@\textsc{Hofmannsthal, Hugo von} (1874-02-01 – 1929-07-15), \emph{Schriftsteller}|pw}. Von Letzterem werden wir im \label{K_L03086-123v}\edtext{»Deutſchen Theater\orgindex{Deutsches Theater Berlin@Deutsches Theater Berlin|pw}« ein Versdrama\pwindex{Hofmannsthal, Hugo von 1874-02-01 – 1929-07-15@\textsc{Hofmannsthal, Hugo von} (1874-02-01 – 1929-07-15), \emph{Schriftsteller}!Turm. Ein Trauerspiel1925@\strich\emph{Der Turm. Ein Trauerspiel} {[}1925{]}|pwv}}{\lemma{\textnormal{\emph{»Deutſchen … Versdrama}}}\Cendnote{\textnormal{Dazu kam es nicht. Gemeint sein dürfte
                  die Bearbeitung von Calderón de la Barca\pwindex{Calderón de la Barca, Pedro 17.01.1600 – 25.05.1681@\textsc{Calderón de la Barca, Pedro} (17.01.1600 – 25.05.1681), \emph{Schriftsteller}|pwk}s
                     \emph{Das Leben ein Traum}\pwindex{Calderón de la Barca, Pedro 17.01.1600 – 25.05.1681@\textsc{Calderón de la Barca, Pedro} (17.01.1600 – 25.05.1681), \emph{Schriftsteller}!Leben ein Traum1635@\strich\emph{Das Leben ein Traum} {[}1635{]}|pwk}, die Jahrzehnte später
                  zu \emph{Der Turm}\pwindex{Hofmannsthal, Hugo von 1874-02-01 – 1929-07-15@\textsc{Hofmannsthal, Hugo von} (1874-02-01 – 1929-07-15), \emph{Schriftsteller}!Turm. Ein Trauerspiel1925@\strich\emph{Der Turm. Ein Trauerspiel} {[}1925{]}|pwk} werden sollte.}}}\label{K_L03086-123h} zu ſehen
               bekommen. Ich freue mich ſchon rieſig.\pend
           \pstart
           {\pb}Daß Ihr \label{K_L03086-111v}\edtext{Vater\pwindex{Gussmann, Rudolf 05.03.1842 – 24.01.1921@\textsc{Gussmann, Rudolf} (05.03.1842 – 24.01.1921), \emph{Handelsagent}|pwv} ſich ſo abſcheulich
                  benimmt}{\lemma{\textnormal{\emph{Vater … benimmt}}}\Cendnote{\textnormal{siehe A. S.: \emph{Tagebuch}, 6. 9. 1901}}}\label{K_L03086-111h}, thut mir unendlich leid. Kann man da gar nichts machen? \textsc{Arthur} ſoll den Prozeß nur jedenfalls einleiten. Ich bedaure namentlich, daß
                  \uline{ich} in der Angelegenheit ſo gar nicht zu Hilfe
               kommen kann. Zum Beiſpiel, wenn ich eine Million hätte, wäre das ſehr einfach. Bitte,
               warum hab’ ich keine Million?\pend
           \pstart
           Diese neuen Kleider müſſen herrlich ſein. Beſonders, wenn ich den himmel¬ blauen
               Gürtel ſehen könnte, es thäte meinem Herzen wohl!\pend
           \pstart
           Ich denke oft und herzlich an Sie (wieder Mehrzahl). \label{K_L03086-177v}\edtext{\textsc{Welsberg\oindex{Welsberg-Taisten@\textbf{Welsberg-Taisten}|pw}}}{\lemma{\textnormal{\emph{Welsberg}}}\Cendnote{\textnormal{siehe Paul Goldmann an Arthur Schnitzler, 26. 4. [1901]}}}\label{K_L03086-177h} liegt fern. Ich lebe wieder mein elendes Leben und bin unbeſchreiblich einſam
               in dieſer kalten Stadt\oindex{Berlin@\textbf{Berlin}|pwv}, in der
               Niemand mich mag, Keiner und Keine.\pend
           \pstart
           {\pb}Liebes Fräulein \textsc{Olga},
               kommen Sie bald mit dem \textsc{Arthur} nach Berlin\oindex{Berlin@\textbf{Berlin}|pw}, ſchreiben Sie mir bis dahin noch manchen lieben Brief
               und ſeien Sie herzlich gegrüßt von\pend
           \pstart
            Ihrem ergebenen{\\[\baselineskip]}\spacefill\mbox{Dr. Paul Goldmann.}\pend
           \leftskip=0em{}
         
         \endnumbering\mylabel{h}\end{ledgroupsized}  \newcommand{\dateiname}{L03086}\newcommand{\titel}{Paul Goldmann an Arthur Schnitzler, Olga und Elisabeth Gussmann, 27. 9. [1901]}\newcommand{\editorInnen}{Martin Anton Müller und Laura Untner}%% latex-leseansicht-abspann.tex
%% Abspann für die Leseansicht.
%% Der Schalter \ifkorrekturansicht ist bereits durch den Vorspann gesetzt.

%% latex-abspann.tex
%% Gemeinsamer Abspann für Korrekturansicht und Leseansicht.
%% Setzt den Schalter \ifkorrekturansicht voraus (gesetzt in den
%% einbindenden Dateien latex-korrekturansicht-abspann.tex bzw.
%% latex-leseansicht-abspann.tex).
%% ---------------------------------------------------------------

\normalsize

% Das esempio-Environment wird nur in der Leseansicht benötigt
\ifkorrekturansicht\else
\newenvironment{esempio}[3]%
{
    \vspace{1.5ex}
    \rlap{\underline{#1}}
    \par
    \setlength{\parindent}{0cm}
    \nopagebreak
    \leftskip=#2cm
    \rightskip=#3cm
}
{
    \par
}
\fi

\doendnotes{C}
\bigskip
\vfill

\clearpage

\footnotesize

\ifkorrekturansicht
  \lohead{\textsc{register}}
\fi

% theindex-Environment neu definieren ohne reledmac
\makeatletter
\renewenvironment{theindex}{%
  \ifkorrekturansicht
    \section*{\indexname}%
  \else
    \subsubsection*{Index der erwähnten Entitäten}%
  \fi
  \setlength{\parindent}{0pt}%
  \setlength{\parskip}{0pt plus 0.3pt}%
  \let\item\@idxitem
}{%
  \ifkorrekturansicht\clearpage\fi
}
\makeatother

\IfFileExists{\jobname-pw.ind}{\input{\jobname-pw.ind}}{}

% Quellenangabe nur in der Leseansicht
\ifkorrekturansicht\else
% Fallback-Definitionen, falls die .tex-Datei \titel etc. nicht gesetzt hat
\providecommand{\titel}{}
\providecommand{\editorInnen}{}
\providecommand{\dateiname}{\jobname}

\vspace{3cm}

\vfill

\footnotesize
\textsc{Quelle}: \titel. Herausgegeben von {\editorInnen}. In: \emph{Arthur Schnitzler: Briefwechsel mit Autorinnen und Autoren}.
 Digitale Edition, https://schnitzler-briefe.acdh.oeaw.ac.at/{\dateiname}.html (Stand \today)
\fi

\end{document}


      