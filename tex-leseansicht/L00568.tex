%% latex-korrekturansicht-vorspann.tex
%% Vorspann für die Korrekturansicht.
%% Lädt die gemeinsame Datei latex-vorspann.tex mit gesetztem Schalter.

\newif\ifkorrekturansicht
\korrekturansichttrue

\input{../tex-inputs/latex-vorspann}


\section[Richard Beer-Hofmann an Arthur Schnitzler, 26. 7. 1896]{L00568 Richard Beer-Hofmann an Arthur Schnitzler, 26. 7. 1896}
\nopagebreak\mylabel{L00568v}
\rehead{ }\normalsize\beginnumbering\briefempfaengerindex{Schnitzler, Arthur@\textsc{Schnitzler, Arthur}!zzzBeer-Hofmann, Richard@\emph{von Richard Beer-Hofmann}!1896-07-261@{26. 7. 1896}|(be}
\toendnotes[C]{\smallbreak\pagebreak[2]}\Standort{CUL, Schnitzler, B 8.}
\physDesc{Postkarte, 258 Zeichen
\newline{}Handschrift: Bleistift, lateinische Kurrent
\newline{}Versand: 1) Stempel: »\nobreak{}\oindex{Kopenhagen@\textbf{Kopenhagen}, \emph{P.PPLC}|pwk}Kjøbenhavn, 26. 7. 96, 5–6 E\nobreak{}«.   2) Stempel: »\nobreak{}\oindex{Stockholm@\textbf{Stockholm}, \emph{P.PPLC}|pwk}Stockholm, 27 7 96, 18\nobreak{}«. 
\newline{}Ordnung: mit Bleistift von unbekannter Hand nummeriert:
                                    »77« }\pstart{}{\pb}Herrn D\textsuperscript{r} Arthur Schnitzler\pend{}\pstart{}Stockholm\oindex{Stockholm@\textbf{Stockholm}, \emph{P.PPLC}|pw}\pend{}\pstart{}Poste restante\pend{}{\bigskip}\vspace{1em}
\pstart
           \raggedleft{}{\pb}26/VII 96{ }\textsc{Kopenhagn}\oindex{Kopenhagen@\textbf{Kopenhagen}, \emph{P.PPLC}|pw}\pend
           \vspace{0.5em}
\pstart
           Lieber Arthur! Soeben Ihre\introOben{}n\introOben{} Brief u.
               Karten erhalten. Weiß noch nicht da erst seit gestern hier, wo ich wohnen werde;
               werde aber morgen oder übermorgen Stockholm\oindex{Stockholm@\textbf{Stockholm}, \emph{P.PPLC}|pw}
               telegrafiren.\pend
           
\pstart
           Herzlichst{\\[\baselineskip]}Ihr{\\[\baselineskip]}\spacefill\mbox{Richard}\pend
           \leftskip=0em{}\selectlanguage{ngerman}\endnumbering\briefempfaengerindex{Schnitzler, Arthur@\textsc{Schnitzler, Arthur}!zzzBeer-Hofmann, Richard@\emph{von Richard Beer-Hofmann}!1896-07-261@{26. 7. 1896}|)be}\mylabel{L00568h}  \normalsize

\doendnotes{C}
\bigskip
\vfill

\clearpage

\footnotesize

\lohead{\textsc{register}}

% Definiere theindex-Environment komplett neu ohne reledmac
\makeatletter
\renewenvironment{theindex}{%
  \section*{\indexname}%
  \setlength{\parindent}{0pt}%
  \setlength{\parskip}{0pt plus 0.3pt}%
  \let\item\@idxitem
}{%
  \clearpage
}
\makeatother

\IfFileExists{\jobname-pw.ind}{\input{\jobname-pw.ind}}{}

\end{document}

      