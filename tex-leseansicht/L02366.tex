%% latex-leseansicht-vorspann.tex
%% Vorspann für die Leseansicht.
%% Lädt die gemeinsame Datei latex-vorspann.tex mit nicht gesetztem Schalter.

\newif\ifkorrekturansicht
\korrekturansichtfalse

\input{../tex-inputs/latex-vorspann}


\section[Arthur Schnitzler an Richard Beer-Hofmann, 8. 5. 1921]{L02366 Arthur Schnitzler an Richard Beer-Hofmann, 8. 5. 1921}
\nopagebreak\mylabel{L02366v}
\rehead{ }\normalsize\beginnumbering\briefempfaengerindex{Beer-Hofmann, Richard@\textsc{Beer-Hofmann, Richard}!zzzSchnitzler, Arthur@\emph{von Arthur Schnitzler}!1921-05-081@{8. 5. 1921}|(be}
\toendnotes[C]{\smallbreak\pagebreak[2]}
\correspDesc{Versand  durch Arthur Schnitzler am 8. 5. 1921 in München
\newline{}Erhalt  durch Richard Beer-Hofmann im Zeitraum [9. 5. 1921
                  – 13. 5. 1921?] in Wien}\toendnotes[C]{\smallbreak}
\Standort{YCGL, MSS 31.}
\physDesc{Bildpostkarte, 145 Zeichen
\newline{}Handschrift: Bleistift, deutsche Kurrent
\newline{}Versand: 1) Stempel: »\nobreak{}\oindex{München@\textbf{München}|pwk}München, 8. V. 21, 1–2N\nobreak{}«.   2) mit blauem Buntstift von unbekannter Hand bei der Ortsangabe der
                                 Adresse die Bezirksnummer ergänzt: »XVIII«}\toendnotes[C]{\smallbreak}\pstart{}{\pb}Hrn \textsc{Richard Beer Hofmann}\pend{}\pstart{}Wien\oindex{Wien@\textbf{Wien}, \emph{Verwaltungsgebiet}|pw}\pend{}\pstart{}\textsc{Hasenauerstr 59\oindex{Wien@\textbf{Wien}!XVIII., Währing@\textbf{XVIII., Währing}!Hasenauerstraße 59@\textbf{Hasenauerstraße 59}, \emph{Wohngebäude}|pw}}.\pend{}{\bigskip}
\pstart
           \noindent{}\centering{}{\pb}\textcolor{gray}{\textbf{\textbf{München}\oindex{München@\textbf{München}|pw}\hspace*{1.5em}Partie an der Isar\oindex{Isar@\textbf{Isar}, \emph{Fluss}|pw} – Blick gegen Süden}}\pend
           \vspace{1em}
\pstart
           \noindent{}{\pb}Herzliche Grüße\pend
           
\pstart
           Ihr{\\[\baselineskip]}\spacefill\mbox{Arthur}\pend
           \leftskip=0em{}
\pstart
           \raggedleft{}\label{T_L02366-1v}\edtext{8. 5. 1921}{\lemma{\textnormal{\emph{8. 5. 1921}}}\Cendnote{\textnormal{ab hier seitlich zum Text}}}\label{T_L02366-1}\pend
           
\pstart
           \label{K_L02366-1v}\edtext{Geſtern}{\lemma{\textnormal{\emph{Gestern}}}\Cendnote{\textnormal{Siehe A. S.: \emph{Tagebuch}, 7. 5. 1921.
                  }}}\label{K_L02366-1}, in Harlaching\oindex{Harlaching@\textbf{Harlaching}, \emph{Ehemaliger Ort}|pw}, hab ich viel Ihrer,
                  und schöner Tage gedacht.\pend
           \selectlanguage{ngerman}\endnumbering\briefempfaengerindex{Beer-Hofmann, Richard@\textsc{Beer-Hofmann, Richard}!zzzSchnitzler, Arthur@\emph{von Arthur Schnitzler}!1921-05-081@{8. 5. 1921}|)be}\mylabel{L02366h}  \newcommand{\dateiname}{L02366}\newcommand{\titel}{Arthur Schnitzler an Richard Beer-Hofmann, 8. 5. 1921}\newcommand{\editorInnen}{Martin Anton Müller und Gerd-Hermann Susen}%% latex-leseansicht-abspann.tex
%% Abspann für die Leseansicht.
%% Der Schalter \ifkorrekturansicht ist bereits durch den Vorspann gesetzt.

%% latex-abspann.tex
%% Gemeinsamer Abspann für Korrekturansicht und Leseansicht.
%% Setzt den Schalter \ifkorrekturansicht voraus (gesetzt in den
%% einbindenden Dateien latex-korrekturansicht-abspann.tex bzw.
%% latex-leseansicht-abspann.tex).
%% ---------------------------------------------------------------

\normalsize

% Das esempio-Environment wird nur in der Leseansicht benötigt
\ifkorrekturansicht\else
\newenvironment{esempio}[3]%
{
    \vspace{1.5ex}
    \rlap{\underline{#1}}
    \par
    \setlength{\parindent}{0cm}
    \nopagebreak
    \leftskip=#2cm
    \rightskip=#3cm
}
{
    \par
}
\fi

\doendnotes{C}
\bigskip
\vfill

\clearpage

\footnotesize

\ifkorrekturansicht
  \lohead{\textsc{register}}
\fi

% theindex-Environment neu definieren ohne reledmac
\makeatletter
\renewenvironment{theindex}{%
  \ifkorrekturansicht
    \section*{\indexname}%
  \else
    \subsubsection*{Index der erwähnten Entitäten}%
  \fi
  \setlength{\parindent}{0pt}%
  \setlength{\parskip}{0pt plus 0.3pt}%
  \let\item\@idxitem
}{%
  \ifkorrekturansicht\clearpage\fi
}
\makeatother

\IfFileExists{\jobname-pw.ind}{\input{\jobname-pw.ind}}{}

% Quellenangabe nur in der Leseansicht
\ifkorrekturansicht\else
% Fallback-Definitionen, falls die .tex-Datei \titel etc. nicht gesetzt hat
\providecommand{\titel}{}
\providecommand{\editorInnen}{}
\providecommand{\dateiname}{\jobname}

\vspace{3cm}

\vfill

\footnotesize
\textsc{Quelle}: \titel. Herausgegeben von {\editorInnen}. In: \emph{Arthur Schnitzler: Briefwechsel mit Autorinnen und Autoren}.
 Digitale Edition, https://schnitzler-briefe.acdh.oeaw.ac.at/{\dateiname}.html (Stand \today)
\fi

\end{document}


