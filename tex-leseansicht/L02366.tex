%% latex-korrekturansicht-vorspann.tex
%% Vorspann für die Korrekturansicht.
%% Lädt die gemeinsame Datei latex-vorspann.tex mit gesetztem Schalter.

\newif\ifkorrekturansicht
\korrekturansichttrue

\input{../tex-inputs/latex-vorspann}


\section[Arthur Schnitzler an Richard Beer-Hofmann, 8. 5. 1921]{L02366 Arthur Schnitzler an Richard Beer-Hofmann, 8. 5. 1921}
\nopagebreak\mylabel{L02366v}
\rehead{ }\normalsize\beginnumbering\briefempfaengerindex{Beer-Hofmann, Richard@\textsc{Beer-Hofmann, Richard}!zzzSchnitzler, Arthur@\emph{von Arthur Schnitzler}!1921-05-081@{8. 5. 1921}|(be}
\toendnotes[C]{\smallbreak\pagebreak[2]}\Standort{YCGL, MSS 31.}
\physDesc{Bildpostkarte, 145 Zeichen
\newline{}Handschrift: Bleistift, deutsche Kurrent
\newline{}Versand: 1) Stempel: »\nobreak{}\oindex{Muenchen@\textbf{München}, \emph{P.PPLA}|pwk}München, 8. V. 21, 1–2N\nobreak{}«.   2) mit blauem Buntstift von unbekannter Hand bei der Ortsangabe der
                                 Adresse die Bezirksnummer ergänzt: »XVIII«}\toendnotes[C]{\smallbreak}\pstart{}{\pb}Hrn \textsc{Richard Beer Hofmann}\pend{}\pstart{}Wien\oindex{Wien@\textbf{Wien}, \emph{A.ADM2}|pw}\pend{}\pstart{}\textsc{Hasenauerstr 59\oindex{Hasenauerstrasse 59@\textbf{Hasenauerstraße 59}, \emph{Wohngebäude (K.WHS)}|pw}}.\pend{}{\bigskip}
\pstart
           \noindent{}\centering{}{\pb}\textcolor{gray}{\textbf{\textbf{München}\oindex{Muenchen@\textbf{München}, \emph{P.PPLA}|pw}\hspace*{1.5em}Partie an der Isar\oindex{Isar@\textbf{Isar}, \emph{Fluss (N.FLS)}|pw} – Blick gegen Süden}}\pend
           \vspace{1em}
\pstart
           \noindent{}{\pb}Herzliche Grüße\pend
           
\pstart
           Ihr{\\[\baselineskip]}\spacefill\mbox{Arthur}\pend
           \leftskip=0em{}
\pstart
           \raggedleft{}\label{T_L02366-1v}\edtext{8. 5. 1921}{\lemma{\textnormal{\emph{8. 5. 1921}}}\Cendnote{\textnormal{ab hier seitlich zum Text}}}\label{T_L02366-1}\pend
           
\pstart
           \label{K_L02366-1v}\edtext{Geſtern}{\lemma{\textnormal{\emph{Geſtern}}}\Cendnote{\textnormal{Siehe A. S.: \emph{Tagebuch}, 7. 5. 1921.
                  }}}\label{K_L02366-1}, in Harlaching\oindex{Harlaching@\textbf{Harlaching}, \emph{P.PPLX}|pw}, hab ich viel Ihrer,
                  und schöner Tage gedacht.\pend
           \selectlanguage{ngerman}\endnumbering\briefempfaengerindex{Beer-Hofmann, Richard@\textsc{Beer-Hofmann, Richard}!zzzSchnitzler, Arthur@\emph{von Arthur Schnitzler}!1921-05-081@{8. 5. 1921}|)be}\mylabel{L02366h}  \normalsize

\doendnotes{C}
\bigskip
\vfill

\clearpage

\footnotesize

\lohead{\textsc{register}}

% Definiere theindex-Environment komplett neu ohne reledmac
\makeatletter
\renewenvironment{theindex}{%
  \section*{\indexname}%
  \setlength{\parindent}{0pt}%
  \setlength{\parskip}{0pt plus 0.3pt}%
  \let\item\@idxitem
}{%
  \clearpage
}
\makeatother

\IfFileExists{\jobname-pw.ind}{\input{\jobname-pw.ind}}{}

\end{document}

      