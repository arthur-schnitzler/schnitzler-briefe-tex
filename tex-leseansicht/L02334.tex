%% latex-leseansicht-vorspann.tex
%% Vorspann für die Leseansicht.
%% Lädt die gemeinsame Datei latex-vorspann.tex mit nicht gesetztem Schalter.

\newif\ifkorrekturansicht
\korrekturansichtfalse

\input{../tex-inputs/latex-vorspann}


\section[Hugo Hofmannsthal an Arthur Schnitzler, 23. 1. 1920]{L02334 Hugo Hofmannsthal an Arthur Schnitzler, 23. 1. 1920}
\nopagebreak\mylabel{L02334v}
\rehead{ }\normalsize\beginnumbering\briefempfaengerindex{Schnitzler, Arthur@\textsc{Schnitzler, Arthur}!zzzHofmannsthal, Hugo von@\emph{von Hugo von Hofmannsthal}!1920-01-231@{23. 1. 1920}|(be}
\toendnotes[C]{\smallbreak\pagebreak[2]}
\correspDesc{Versand  durch Hugo von Hofmannsthal am 23. 1. 1920 \textbf{Ort fehlend} 
\newline{}Erhalt  durch Arthur Schnitzler im Zeitraum [23. 1. 1920
                  – 27. 1. 1920?] in Wien}\toendnotes[C]{\smallbreak}
\Standort{CUL, Schnitzler, B 43.}
\physDesc{Brief, 1 Blatt, 3 Seiten, 936 Zeichen
\newline{}Handschrift: schwarze Tinte, deutsche Kurrent
\newline{}Ordnung: 1) mit Bleistift von Frieda
                                    Pollak\pwindex{Pollak, Frieda 8.\,12.\,1881 Wien – 13.\,7.\,1937 ebd.@\textsc{Pollak, Frieda} (8.\,12.\,1881 Wien – 13.\,7.\,1937 ebd.), \emph{Sekretärin}|pw} (?) mit dem Buchstaben »A«
                                 (Abgeschrieben/Abschrift) gekennzeichnet  2) mit Bleistift von unbekannter Hand nummeriert: »\strikeout{264}« 3) mit Bleistift von unbekannter Hand nummeriert:
                                    »362«}
\buchAbdrucke{\weitereDrucke{Hugo von Hofmannsthal, Arthur Schnitzler: \emph{Briefwechsel}. Herausgegeben von Therese Nickl und Heinrich Schnitzler. Frankfurt am Main: \emph{S. Fischer} 1964, S. 290.} }\toendnotes[C]{\smallbreak}
\pstart
           \raggedleft{}{\pb}Freitag 23 I 20.\pend
           
\pstart{}mein lieber Arthur\pend\vspace{0.5em}
\pstart
           \label{K_L02334-1v}\edtext{neulich}{\lemma{\textnormal{\emph{neulich}}}\Cendnote{\textnormal{Siehe A. S.: \emph{Tagebuch}, 14. 1. 1920.
               }}}\label{K_L02334-1}, in einer ängſtlichen Stunde, war mir{ }ſo{ }ſehr woltuend, Ihre Sti{\geminationm}e zu hören und Ihren Rat zu empfangen.\hspace*{1.5em}Die vieljährige Zuſa{\geminationm}engehörigkeit iſt doch ein{ }ſo großes Wirkliches.\hspace*{1.5em}–
               Wie nahe war mir in dieſem Augenblick der Tag vor 20 Jahren, das \label{K_L02334-2v}\edtext{Unglück}{\lemma{\textnormal{\emph{Unglück}}}\Cendnote{\textnormal{Am 18. 3. 1899 starb Marie
                     Reinhard\pwindex{Reinhard, Marie 13.\,3.\,1871 Wien – 18.\,3.\,1899 ebd.@\textsc{Reinhard, Marie} (13.\,3.\,1871 Wien – 18.\,3.\,1899 ebd.), \emph{Gesangspädagogin}|pwk}; am gleichen Tag hatte \emph{Die
                         Hochzeit der Sobeide}\pwindex{Hofmannsthal, Hugo von 1.\,2.\,1874 Wien – 15.\,7.\,1929 Rodaun@\textsc{Hofmannsthal, Hugo von} (1.\,2.\,1874 Wien – 15.\,7.\,1929 Rodaun), \emph{Schriftsteller}!Hochzeit der Sobeide@\strich\emph{Die Hochzeit der Sobeide}|pwk}{ }Uraufführung\eventindex{Deutsches Theater Berlin@\textbf{Deutsches Theater Berlin}!Berliner Uraufführung von Der Abenteurer und die Sängerin und Die Hochzeit der Sobeide, 18.3.1899@Berliner Uraufführung von Der Abenteurer und die Sängerin und Die Hochzeit der Sobeide, 18.3.1899|pwkv}.}}}\label{K_L02334-2}, wodurch die erſte Aufführung\eventindex{Deutsches Theater Berlin@\textbf{Deutsches Theater Berlin}!Berliner Uraufführung von Der Abenteurer und die Sängerin und Die Hochzeit der Sobeide, 18.3.1899@Berliner Uraufführung von Der Abenteurer und die Sängerin und Die Hochzeit der Sobeide, 18.3.1899|pwv}
               meiner Stücke {\pb}mir für immer
               beſchattet wurde – auch das Weſen\pwindex{Reinhard, Marie 13.\,3.\,1871 Wien – 18.\,3.\,1899 ebd.@\textsc{Reinhard, Marie} (13.\,3.\,1871 Wien – 18.\,3.\,1899 ebd.), \emph{Gesangspädagogin}|pwv}, das ich nie geſehen u. von dem ich doch ein unverlöſchliches
               Phantaſiebild in mir trage.\pend
           
\pstart
           Lieber Arthur, ich ko{\geminationm}e demnächst vormittags zu Ihnen,
               melde mich vorher.\pend
           
\pstart
           Bitte blättern Sie die Stelle im Märchen\pwindex{Hofmannsthal, Hugo von 1.\,2.\,1874 Wien – 15.\,7.\,1929 Rodaun@\textsc{Hofmannsthal, Hugo von} (1.\,2.\,1874 Wien – 15.\,7.\,1929 Rodaun), \emph{Schriftsteller}!Frau ohne Schatten. Erzählung@\strich\emph{Die Frau ohne Schatten. Erzählung}|pwv} auf und{ }ſchreiben Sie mir, wodurch Ihr Eindruck von \textsc{Baraks}\pwindex{Hofmannsthal, Hugo von 1.\,2.\,1874 Wien – 15.\,7.\,1929 Rodaun@\textsc{Hofmannsthal, Hugo von} (1.\,2.\,1874 Wien – 15.\,7.\,1929 Rodaun), \emph{Schriftsteller}!Frau ohne Schatten. Erzählung@\strich\emph{Die Frau ohne Schatten. Erzählung}|pwv} phyſiſcher {\pb}Erſcheinung als
               einer widerwärtigen{ }ſich{ }ſo fixiert hat.\hspace*{1.5em}Ich überlas
               die Stelle, die mir vorſchwebte, fand{ }ſie relativ harmlos, in groben epiſch
               primitiven Zügen: ein Maul wie ein Spalt – das heißt aber doch nicht: eine geſpaltene
               Lippe.\pend
           
\pstart
           Ich würde es gerne retouchieren.\pend
           
\pstart
           Von Herzen Ihr{\\[\baselineskip]}\spacefill\mbox{Hugo.}\pend
           \leftskip=0em{}\selectlanguage{ngerman}\endnumbering\briefempfaengerindex{Schnitzler, Arthur@\textsc{Schnitzler, Arthur}!zzzHofmannsthal, Hugo von@\emph{von Hugo von Hofmannsthal}!1920-01-231@{23. 1. 1920}|)be}\mylabel{L02334h}  \newcommand{\dateiname}{L02334}\newcommand{\titel}{Hugo Hofmannsthal an Arthur Schnitzler, 23. 1. 1920}\newcommand{\editorInnen}{Martin Anton Müller und Gerd-Hermann Susen}%% latex-leseansicht-abspann.tex
%% Abspann für die Leseansicht.
%% Der Schalter \ifkorrekturansicht ist bereits durch den Vorspann gesetzt.

%% latex-abspann.tex
%% Gemeinsamer Abspann für Korrekturansicht und Leseansicht.
%% Setzt den Schalter \ifkorrekturansicht voraus (gesetzt in den
%% einbindenden Dateien latex-korrekturansicht-abspann.tex bzw.
%% latex-leseansicht-abspann.tex).
%% ---------------------------------------------------------------

\normalsize

% Das esempio-Environment wird nur in der Leseansicht benötigt
\ifkorrekturansicht\else
\newenvironment{esempio}[3]%
{
    \vspace{1.5ex}
    \rlap{\underline{#1}}
    \par
    \setlength{\parindent}{0cm}
    \nopagebreak
    \leftskip=#2cm
    \rightskip=#3cm
}
{
    \par
}
\fi

\doendnotes{C}
\bigskip
\vfill

\clearpage

\footnotesize

\ifkorrekturansicht
  \lohead{\textsc{register}}
\fi

% theindex-Environment neu definieren ohne reledmac
\makeatletter
\renewenvironment{theindex}{%
  \ifkorrekturansicht
    \section*{\indexname}%
  \else
    \subsubsection*{Index der erwähnten Entitäten}%
  \fi
  \setlength{\parindent}{0pt}%
  \setlength{\parskip}{0pt plus 0.3pt}%
  \let\item\@idxitem
}{%
  \ifkorrekturansicht\clearpage\fi
}
\makeatother

\IfFileExists{\jobname-pw.ind}{\input{\jobname-pw.ind}}{}

% Quellenangabe nur in der Leseansicht
\ifkorrekturansicht\else
% Fallback-Definitionen, falls die .tex-Datei \titel etc. nicht gesetzt hat
\providecommand{\titel}{}
\providecommand{\editorInnen}{}
\providecommand{\dateiname}{\jobname}

\vspace{3cm}

\vfill

\footnotesize
\textsc{Quelle}: \titel. Herausgegeben von {\editorInnen}. In: \emph{Arthur Schnitzler: Briefwechsel mit Autorinnen und Autoren}.
 Digitale Edition, https://schnitzler-briefe.acdh.oeaw.ac.at/{\dateiname}.html (Stand \today)
\fi

\end{document}


