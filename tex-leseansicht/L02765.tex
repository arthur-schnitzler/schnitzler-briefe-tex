%% latex-leseansicht-vorspann.tex
%% Vorspann für die Leseansicht.
%% Lädt die gemeinsame Datei latex-vorspann.tex mit nicht gesetztem Schalter.

\newif\ifkorrekturansicht
\korrekturansichtfalse

\input{../tex-inputs/latex-vorspann}


\section[ Paul Goldmann an Arthur Schnitzler, 23. 1. [1896]]{L02765 Paul Goldmann an Arthur Schnitzler,  23. 1. [1896]}
\nopagebreak\mylabel{L02765v}
\rehead{ }\normalsize\beginnumbering\briefempfaengerindex{Schnitzler, Arthur@\textsc{Schnitzler, Arthur}!zzzGoldmann, Paul@\emph{von Paul Goldmann}!1896-01-231@{23. 1. [1896]}|(be}
\toendnotes[C]{\smallbreak\pagebreak[2]}
\correspDesc{Versand  durch Paul Goldmann am 23. 1. [1896] in Paris
\newline{}Erhalt  durch Arthur Schnitzler im Zeitraum [24. 1. 1896
                  – 28. 1. 1896?] in Wien}\toendnotes[C]{\smallbreak}
\Standort{DLA, A:Schnitzler, HS.NZ85.1.3166.}
\physDesc{Brief, 3 Blätter, 11 Seiten, 4373 Zeichen
\newline{}Handschrift: blaue Tinte, deutsche Kurrent
\newline{}Schnitzler: 1) mit Bleistift das Jahr »96« vermerkt  2) mit rotem Buntstift vier Unterstreichungen}\toendnotes[C]{\smallbreak}
\pstart
           {\pb}\textcolor{gray}{\textbf{\textbf{Frankfurter Zeitung\orgindex{Frankfurter Zeitung@Frankfurter Zeitung|pw}}}}\pend
           
\pstart
           \textcolor{gray}{\textbf{(\begin{otherlanguage}{french}Gazette de Francfort\end{otherlanguage}\orgindex{Frankfurter Zeitung@Frankfurter Zeitung|pw}).}}\pend
           
\pstart
           \textcolor{gray}{\textbf{\textbf{\begin{otherlanguage}{french}Fondateur M.\end{otherlanguage}{ }L. Sonnemann\pwindex{Sonnemann, Leopold 29.\,10.\,1831 Höchberg – 30.\,10.\,1909 Frankfurt am Main@\textsc{Sonnemann, Leopold} (29.\,10.\,1831 Höchberg – 30.\,10.\,1909 Frankfurt am Main), \emph{Journalist, Herausgeber}|pw}.}}}\pend
           
\pstart
           \begin{otherlanguage}{french}\textcolor{gray}{\textbf{Journal\pwindex{Frankfurter Zeitung@\emph{Frankfurter Zeitung}|pwv} politique,
                        financier,}}\end{otherlanguage}\pend
           
\pstart
           \begin{otherlanguage}{french}\textcolor{gray}{\textbf{commercial et littéraire.}}\end{otherlanguage}\pend
           
\pstart
           \begin{otherlanguage}{french}\textcolor{gray}{\textbf{\textbf{Paraissant trois fois par jour.}}}\end{otherlanguage}\pend
           
\pstart
           \begin{otherlanguage}{french}\textcolor{gray}{\textbf{\textbf{Bureau à Paris\oindex{Paris@\textbf{Paris}, \emph{Hauptstadt}|pw}:}}}\end{otherlanguage}\pend
           
\pstart
           \begin{otherlanguage}{french}\textcolor{gray}{\textbf{\textbf{24. Rue Feydeau\oindex{rue Feydeau@\textbf{rue Feydeau}, \emph{Straße}|pw}.}}}\end{otherlanguage}\hfill \textsc{Paris\oindex{Paris@\textbf{Paris}, \emph{Hauptstadt}|pw}}, 23. Januar.\pend
           
\pstart\center{}Mein lieber Freund,\pend\vspace{0.5em}
\pstart
           Wann iſt alſo die \label{K_L02765-1v}\edtext{Berlin\oindex{Berlin@\textbf{Berlin}, \emph{Hauptstadt}|pw}er Aufführung\pwindex{Schnitzler, Arthur 15.\,5.\,1862 Wien – 21.\,10.\,1931 ebd.@\textsc{Schnitzler, Arthur} (15.\,5.\,1862 Wien – 21.\,10.\,1931 ebd.), \emph{Schriftsteller, Mediziner}!Liebelei. Schauspiel in drei Akten@\strich\emph{Liebelei. Schauspiel in drei Akten}|pwv}}{\lemma{\textnormal{\emph{Berliner Aufführung}}}\Cendnote{\textnormal{Die Premiere\eventindex{Deutsches Theater Berlin@\textbf{Deutsches Theater Berlin}!Premiere von Liebelei, Der zerbrochene Krug, 4.2.1896@Premiere von Liebelei, Der zerbrochene Krug, 4.2.1896|pwkv} von \emph{Liebelei}\pwindex{Schnitzler, Arthur 15.\,5.\,1862 Wien – 21.\,10.\,1931 ebd.@\textsc{Schnitzler, Arthur} (15.\,5.\,1862 Wien – 21.\,10.\,1931 ebd.), \emph{Schriftsteller, Mediziner}!Liebelei. Schauspiel in drei Akten@\strich\emph{Liebelei. Schauspiel in drei Akten}|pwk} am \emph{Deutschen Theater
                     Berlin}\orgindex{Deutsches Theater Berlin@Deutsches Theater Berlin|pwk} fand am 4. 2. 1896 in Anwesenheit Schnitzlers statt.}}}\label{K_L02765-1}? Ich{ }ſehe mit Vergnügen, wie ein Stück nach dem
               andern dort durchfällt: \textsc{\label{K_L02765-2v}\edtext{Hauptmann\pwindex{Hauptmann, Gerhart 15.\,11.\,1862 Szczawno-Zdrój – 6.\,6.\,1946 Jagniątków@\textsc{Hauptmann, Gerhart} (15.\,11.\,1862 Szczawno-Zdrój – 6.\,6.\,1946 Jagniątków), \emph{Schriftsteller}!Florian Geyer. Die Tragödie des Bauernkrieges@\strich\emph{Florian Geyer. Die Tragödie des Bauernkrieges}|pwv}\pwindex{Hauptmann, Gerhart 15.\,11.\,1862 Szczawno-Zdrój – 6.\,6.\,1946 Jagniątków@\textsc{Hauptmann, Gerhart} (15.\,11.\,1862 Szczawno-Zdrój – 6.\,6.\,1946 Jagniątków), \emph{Schriftsteller}|pw}}{\lemma{\textnormal{\emph{Hauptmann}}}\Cendnote{\textnormal{Gerhart Hauptmann\pwindex{Hauptmann, Gerhart 15.\,11.\,1862 Szczawno-Zdrój – 6.\,6.\,1946 Jagniątków@\textsc{Hauptmann, Gerhart} (15.\,11.\,1862 Szczawno-Zdrój – 6.\,6.\,1946 Jagniątków), \emph{Schriftsteller}|pwk}: \emph{Florian Geyer. Die Tragödie des Bauernkrieges}\pwindex{Hauptmann, Gerhart 15.\,11.\,1862 Szczawno-Zdrój – 6.\,6.\,1946 Jagniątków@\textsc{Hauptmann, Gerhart} (15.\,11.\,1862 Szczawno-Zdrój – 6.\,6.\,1946 Jagniątków), \emph{Schriftsteller}!Florian Geyer. Die Tragödie des Bauernkrieges@\strich\emph{Florian Geyer. Die Tragödie des Bauernkrieges}|pwk} hatte am
                        4. 1. 1896 am \emph{Deutschen
                        Theater}\orgindex{Deutsches Theater Berlin@Deutsches Theater Berlin|pwk} in Berlin\oindex{Berlin@\textbf{Berlin}, \emph{Hauptstadt}|pwk} die
                     Uraufführung\eventindex{Deutsches Theater Berlin@\textbf{Deutsches Theater Berlin}!Uraufführung von Florian Geyer, 4.1.1896@Uraufführung von Florian Geyer, 4.1.1896|pwkv} erlebt.}}}\label{K_L02765-2}}, \textsc{\label{K_L02765-3v}\edtext{Halbe\pwindex{Halbe, Max 4.\,10.\,1865 Gmina Suchy Dąb – 30.\,11.\,1944 Neuötting@\textsc{Halbe, Max} (4.\,10.\,1865 Gmina Suchy Dąb – 30.\,11.\,1944 Neuötting), \emph{Schriftsteller}!Lebenswende. Tragikomödie in 5 Akten@\strich\emph{Lebenswende. Tragikomödie in 5 Akten}|pwv}\pwindex{Halbe, Max 4.\,10.\,1865 Gmina Suchy Dąb – 30.\,11.\,1944 Neuötting@\textsc{Halbe, Max} (4.\,10.\,1865 Gmina Suchy Dąb – 30.\,11.\,1944 Neuötting), \emph{Schriftsteller}|pw}}{\lemma{\textnormal{\emph{Halbe}}}\Cendnote{\textnormal{Max Halbe\pwindex{Halbe, Max 4.\,10.\,1865 Gmina Suchy Dąb – 30.\,11.\,1944 Neuötting@\textsc{Halbe, Max} (4.\,10.\,1865 Gmina Suchy Dąb – 30.\,11.\,1944 Neuötting), \emph{Schriftsteller}|pwk}: \emph{Lebenswende. Tragikomödie in 5 Akten}\pwindex{Halbe, Max 4.\,10.\,1865 Gmina Suchy Dąb – 30.\,11.\,1944 Neuötting@\textsc{Halbe, Max} (4.\,10.\,1865 Gmina Suchy Dąb – 30.\,11.\,1944 Neuötting), \emph{Schriftsteller}!Lebenswende. Tragikomödie in 5 Akten@\strich\emph{Lebenswende. Tragikomödie in 5 Akten}|pwk} war am
                        21. 1. 1896 am \emph{Deutschen
                           Theater}\orgindex{Deutsches Theater Berlin@Deutsches Theater Berlin|pwk} in Berlin\oindex{Berlin@\textbf{Berlin}, \emph{Hauptstadt}|pwk}{ }uraufgeführt\eventindex{Deutsches Theater Berlin@\textbf{Deutsches Theater Berlin}!Uraufführung von Lebenswende, 21.1.1896@Uraufführung von Lebenswende, 21.1.1896|pwkv} worden.
                  }}}\label{K_L02765-3}}{ }\textsc{etc}. Das iſt vom Schickſal glänzend arrangirt, um Deinen
               Erfolg \strikeout{ins rech} das nöthige Relief zu geben. Mein
               College \textsc{Wolff\pwindex{Wolff, Theodor 2.\,8.\,1868 Berlin – 23.\,9.\,1943 ebd.@\textsc{Wolff, Theodor} (2.\,8.\,1868 Berlin – 23.\,9.\,1943 ebd.), \emph{Schriftsteller, Journalist}|pw}} vom »Berl. Tageblatt\orgindex{Berliner Tageblatt@Berliner Tageblatt|pw}«, der Dir zu Deinem
                  Frankfurt\oindex{Frankfurt am Main@\textbf{Frankfurt am Main}, \emph{Hauptstadt}|pw}er Erfolge\pwindex{Schnitzler, Arthur 15.\,5.\,1862 Wien – 21.\,10.\,1931 ebd.@\textsc{Schnitzler, Arthur} (15.\,5.\,1862 Wien – 21.\,10.\,1931 ebd.), \emph{Schriftsteller, Mediziner}!Liebelei. Schauspiel in drei Akten@\strich\emph{Liebelei. Schauspiel in drei Akten}|pwv} gratuliren läßt, läßt Dich auch fragen, ob er Dir in
                  Berlin\oindex{Berlin@\textbf{Berlin}, \emph{Hauptstadt}|pw} irgendwie mit Einführungen dienen kann?
               Er kennt dort natürlich {\pb}die ganze Welt. Ich glaube,
               die beſte Einführung iſt Dein Stück\pwindex{Schnitzler, Arthur 15.\,5.\,1862 Wien – 21.\,10.\,1931 ebd.@\textsc{Schnitzler, Arthur} (15.\,5.\,1862 Wien – 21.\,10.\,1931 ebd.), \emph{Schriftsteller, Mediziner}!Liebelei. Schauspiel in drei Akten@\strich\emph{Liebelei. Schauspiel in drei Akten}|pwv} und Deine Perſon. Immerhin wollte ich Dir doch das Anerbieten
               übermitteln.\pend
           
\pstart
           \textsc{Thorel\pwindex{Thorel, Jean 11.\,9.\,1859 Éragny – 20.\,8.\,1916 Enghien-les-Bains@\textsc{Thorel, Jean} (11.\,9.\,1859 Éragny – 20.\,8.\,1916 Enghien-les-Bains), \emph{Übersetzer, Dramatiker}|pw}} habe ich lange nicht geſehen; aber{ }ſobald ich Zeit habe,{ }ſuche ich ihn auf.\pend
           
\pstart
           Daß Dir das \label{K_L02765-4v}\edtext{Opernglas}{\lemma{\textnormal{\emph{Opernglas}}}\Cendnote{\textnormal{Siehe XXXX Auszeichnungsfehler: Dokument L02762 nicht gefunden.
               }}}\label{K_L02765-4} gefällt, erſtaunt mich. Mir gefällt es nicht. Aber im Theater hat es{ }ſich
               wohl bewährt? Ja? Was{ }ſoll ich mit den 5 \textsc{Frcs} 40 machen,
               die mir von der Kaufſumme übrig bleiben?\pend
           
\pstart
           \textsc{Bahrs\pwindex{Bahr, Hermann 19.\,7.\,1863 Linz – 15.\,1.\,1934 München@\textsc{Bahr, Hermann} (19.\,7.\,1863 Linz – 15.\,1.\,1934 München), \emph{Schriftsteller, Kritiker}|pw}} kleine \label{K_L02765-5v}\edtext{Erbärmlichkeiten}{\lemma{\textnormal{\emph{Erbärmlichkeiten}}}\Cendnote{\textnormal{Am 21. 1. 1896 kam
                  es zu einer Aussprache zwischen Schnitzler
                  und Bahr\pwindex{Bahr, Hermann 19.\,7.\,1863 Linz – 15.\,1.\,1934 München@\textsc{Bahr, Hermann} (19.\,7.\,1863 Linz – 15.\,1.\,1934 München), \emph{Schriftsteller, Kritiker}|pwk}, die sowohl den Freundeskreis
                  betraf als auch die Reaktion Bahrs\pwindex{Bahr, Hermann 19.\,7.\,1863 Linz – 15.\,1.\,1934 München@\textsc{Bahr, Hermann} (19.\,7.\,1863 Linz – 15.\,1.\,1934 München), \emph{Schriftsteller, Kritiker}|pwk} auf den
                  Erfolg von \emph{Liebelei}\pwindex{Schnitzler, Arthur 15.\,5.\,1862 Wien – 21.\,10.\,1931 ebd.@\textsc{Schnitzler, Arthur} (15.\,5.\,1862 Wien – 21.\,10.\,1931 ebd.), \emph{Schriftsteller, Mediziner}!Liebelei. Schauspiel in drei Akten@\strich\emph{Liebelei. Schauspiel in drei Akten}|pwk}.}}}\label{K_L02765-5}{ }ſind recht
               heiter; {\pb}es werden{ }ſchon größere nachfolgen,{ }ſei
               beruhigt! Die »Zeit\pwindex{Zeit. Wiener Wochenschrift@\emph{Die Zeit. Wiener Wochenschrift}|pw}« leſe ich kaum mehr;{ }ſie iſt
               gar zu{ }ſchlecht geworden. Höchſtens hier und da ein Artikel von \textsc{Loris\pwindex{Hofmannsthal, Hugo von 1.\,2.\,1874 Wien – 15.\,7.\,1929 Rodaun@\textsc{Hofmannsthal, Hugo von} (1.\,2.\,1874 Wien – 15.\,7.\,1929 Rodaun), \emph{Schriftsteller}|pw}}, und auch an dem habe ich wenig Freude. Ich wende mich immer mehr von ihm ab,
               und vor Allem werde ich ihm nie verzeihen, daß er nicht in entſchiedener Weiſe
               zwiſchen Dir und \textsc{Bahr\pwindex{Bahr, Hermann 19.\,7.\,1863 Linz – 15.\,1.\,1934 München@\textsc{Bahr, Hermann} (19.\,7.\,1863 Linz – 15.\,1.\,1934 München), \emph{Schriftsteller, Kritiker}|pw}} gewählt hat. Lieſt Du \label{K_L02765-6v}\edtext{\textsc{Kanners\pwindex{Kanner, Heinrich 9.\,11.\,1864 Galați – 15.\,2.\,1930 Wien@\textsc{Kanner, Heinrich} (9.\,11.\,1864 Galați – 15.\,2.\,1930 Wien), \emph{Herausgeber, Publizist}|pw}}{ }Feuilletons\pwindex{Kanner, Heinrich 9.\,11.\,1864 Galați – 15.\,2.\,1930 Wien@\textsc{Kanner, Heinrich} (9.\,11.\,1864 Galați – 15.\,2.\,1930 Wien), \emph{Herausgeber, Publizist}!Ostwärts um die Welt@\strich\emph{Ostwärts um die Welt}|pwv} aus China\oindex{China@\textbf{China}|pw}}{\lemma{\textnormal{\emph{Kanners … China}}}\Cendnote{\textnormal{Heinrich Kanner\pwindex{Kanner, Heinrich 9.\,11.\,1864 Galați – 15.\,2.\,1930 Wien@\textsc{Kanner, Heinrich} (9.\,11.\,1864 Galați – 15.\,2.\,1930 Wien), \emph{Herausgeber, Publizist}|pwk} war im Auftrag der \emph{Frankfurter Zeitung}\orgindex{Frankfurter Zeitung@Frankfurter Zeitung|pwk} nach China\oindex{China@\textbf{China}|pwk} gereist und publizierte seine Reiseeindrücke in
                  dieser Zeitung\pwindex{Frankfurter Zeitung@\emph{Frankfurter Zeitung}|pwkv}. Das erste
                  Feuilleton – \emph{Östlich um die Welt. I}\pwindex{Kanner, Heinrich 9.\,11.\,1864 Galați – 15.\,2.\,1930 Wien@\textsc{Kanner, Heinrich} (9.\,11.\,1864 Galați – 15.\,2.\,1930 Wien), \emph{Herausgeber, Publizist}!Östlich um die Welt I.@\strich\emph{Östlich um die Welt I.}|pwk} –
                  erschien am 26. 11. 1895 (Jg. 40, Nr. 328,
                     Erstes Morgenblatt, S. 1–3). Weitere Einträge – zumeist unter dem Titel
                  »Ostwärts um die Welt« – folgten am 2. 12. 1895,
                     12. 12. 1895, 24. 12. 1895, 15. 1. 1896, 22. 1. 1896, 26. 1. 1896, 8. 2. 1896 und am 29. 2. 1896. Teilweise wurden sie auch in der
                  Wochenschrift \emph{Die Zeit}\pwindex{Zeit. Wiener Wochenschrift@\emph{Die Zeit. Wiener Wochenschrift}|pwk} nachgedruckt.}}}\label{K_L02765-6}?
               Sie{ }ſind erbärmlich. Der Mann\pwindex{Kanner, Heinrich 9.\,11.\,1864 Galați – 15.\,2.\,1930 Wien@\textsc{Kanner, Heinrich} (9.\,11.\,1864 Galați – 15.\,2.\,1930 Wien), \emph{Herausgeber, Publizist}|pwv} hat keine Augen und{ }ſieht nichts.\pend
           
\pstart
           {\pb}Natürlich waren meine Leute in Frankfurt\oindex{Frankfurt am Main@\textbf{Frankfurt am Main}, \emph{Hauptstadt}|pw} von Dir entzückt, beſonders meine Mutter\pwindex{Goldmann, Clementine 15.\,5.\,1842 Breslau – 24.\,2.\,1924 Frankfurt am Main@\textsc{Goldmann, Clementine} (15.\,5.\,1842 Breslau – 24.\,2.\,1924 Frankfurt am Main)|pwv}. Mein Schwager\pwindex{Rosengart, Josef 8.\,2.\,1860 Laupheim – 4.\,8.\,1927 Frankfurt am Main@\textsc{Rosengart, Josef} (8.\,2.\,1860 Laupheim – 4.\,8.\,1927 Frankfurt am Main), \emph{Arzt}|pwv} findet, Du hätteſt Ähnlichkeit mit
               mir. Bedank’ Dich bei ihm für das Compliment.\pend
           
\pstart
           Deine \label{K_L02765-7v}\edtext{Zweifel, Melancholien und
                  Hypochondrien}{\lemma{\textnormal{\emph{Zweifel, … Hypochondrien}}}\Cendnote{\textnormal{Siehe A. S.: \emph{Tagebuch}, 27. 1. 1896 und 29. 1. 1896.
               }}}\label{K_L02765-7} nehme ich recht gleichmüthig auf. Das heißt, es thut mir innig leid, daß Du
               von alledem gequält wirſt. Aber da man auf \strikeout{Erden{ }ſchon} Erden{ }ſchon einmal gequält werden muß,{ }ſo iſt es beſſer, daß das Leid
               in dieſer Form an Dich \strikeout{heran}{ }{\pb}herantritt, als in einer andern. In dem, was Du{ }ſchreibſt, iſt nichts, was nicht normal wäre. Du biſt ein großes Talent, und Du mußt
               infolgedeſſen naturnothwendig zu Zeiten glauben, daß Du es \uline{nicht} biſt. All’ das, was Du von Deinen Verſtimmungen{ }ſchilderſt, – das iſt
               der \strikeout{Ne} Nebel, der im Grunde jeder Künſtlerſeele
               braut, \strikeout{und} – der Schöpfungsnebel, aus dem die
               Kunſtwerke erſtehen. Und{ }ſo iſt des Künſtlers Erdenwallen: durch Verſtimmungen zur
               Stimmung! {\dots} Daß Dir {\pb}die
               Vergänglichkeit des Lebens wehthut, iſt traurig. Aber ich kann Dir darauf nur immer
               antworten: Wenn Du, wie jemand Anderer, den ich kenne, bereits immer am 15. jedes
               Monats mit Deinem Gehalt fertig wäreſt und nicht wüßteſt, woher Du Geld nehmen{ }ſollſt, um weiter zu leben und Schulden zu zahlen –{ }ſo hätteſt Du keine Zeit, Dich um
               die Vergänglichkeit des Lebens zu{ }ſorgen. Und – ganz im Ernſt geſprochen – es iſt
               beſſer, vor dem Tode zu zittern, als vor {\pb}dem
               Schneider, der die unbezahlte Rechnung präſentiren kommt. Du haſt die edleren
               Schmerzen, mein lieber Freund – und{ }ſelbſt hier biſt Du ein »Sonnt\substVorne{}\textsuperscript{g}\substDazwischen{}a\substHinten{}gskind«. Und wenn ich Deinen Kummer leſe,{ }ſo ruft das in mir nur ein Gefühl
               des – Neides wach. Oh wenn ich auch{ }ſo \strikeout{leid} leiden
               könnte, wie dieſer glückliche junge Mann! Und dann: Du erlebſt nichts zu Ende. Aber
               wenigſtens erlebſt Du etwas. Aber ich kenne {\pb}Leute,
               bei \strikeout{dene} denen es im ganzen Leben nie auch nur zum
               Anfang kommt. Das iſt das Entſetzliche, wenn man{ }ſieht, wie das Leben vorüberraſt –
               wenn man mitleben möchte und nicht die Kraft dazu hat – wenn man eines{ }ſchönen Tages
                  \strikeout{en} entdeckt, daß die Jugend vorbei iſt, ohne daß
               man jemals jung war – und wenn man genau weiß, daß das immer{ }ſo{ }ſein wird und daß man
               eines \strikeout{Ta} anderen{ }ſchönen Tages auf das {\pb}ganze Leben zurückblicken wird mit dem Bewußtſein,
               mit der zehrenden Reue, daß man nie gelebt hat! Du hingegen – Du lebſt! Kein
               glühendes Gefühl des Daſeins – meinetwegen! Aber wo iſt es, dieſes glühende Gefühl,
               als bei den ganz Animaliſchen? Und auch bei denen, glaube ich, iſt es nicht{ }ſo
               glühend. Ich meine, auch das iſt ein Ideal, das nicht exiſtirt. Alles Menſchliche iſt
                  \strikeout{unv} unvollkommen, und ich glaube, nicht einmal {\pb}\uline{leben} können wir ordentlich. Nicht Du allein –
               Keiner! Es gibt keine ganzen, keine glühenden Gefühle. Oder doch\strikeout{,} ein einziges: die \uline{Sehnſucht}. Was wir nicht haben – oh ja, in dem iſt Gluth, Schönheit und
               Vollendung. Aber in dem, was wir haben, – in dem, was wir leben, – da iſt Alles halb,
               jämmerlich und ungefähr.\pend
           
\pstart
           {\pb}Schreib’ weiter an Deinem Stücke\pwindex{Schnitzler, Arthur 15.\,5.\,1862 Wien – 21.\,10.\,1931 ebd.@\textsc{Schnitzler, Arthur} (15.\,5.\,1862 Wien – 21.\,10.\,1931 ebd.), \emph{Schriftsteller, Mediziner}!Freiwild. Schauspiel in 3 Akten@\strich\emph{Freiwild. Schauspiel in 3 Akten}|pwv}, mein theurer Freund,
               und{ }ſei guter Dinge!\pend
           
\pstart
           In Treue {\\[\baselineskip]}Dein {\\[\baselineskip]}\spacefill\mbox{Paul Goldmann}\pend
           \leftskip=0em{}
\pstart
           \noindent{}Und grüß’ mir meinen lieben \textsc{Richard\pwindex{Beer-Hofmann, Richard 11.\,7.\,1866 Wien – 26.\,9.\,1945 New York City@\textsc{Beer-Hofmann, Richard} (11.\,7.\,1866 Wien – 26.\,9.\,1945 New York City), \emph{Schriftsteller}|pw}}!\pend
           \selectlanguage{ngerman}\endnumbering\briefempfaengerindex{Schnitzler, Arthur@\textsc{Schnitzler, Arthur}!zzzGoldmann, Paul@\emph{von Paul Goldmann}!1896-01-231@{23. 1. [1896]}|)be}\mylabel{L02765h}  \newcommand{\dateiname}{L02765}\newcommand{\titel}{Paul Goldmann an Arthur Schnitzler, 23. 1. [1896]}\newcommand{\editorInnen}{Martin Anton Müller und Laura Untner}%% latex-leseansicht-abspann.tex
%% Abspann für die Leseansicht.
%% Der Schalter \ifkorrekturansicht ist bereits durch den Vorspann gesetzt.

%% latex-abspann.tex
%% Gemeinsamer Abspann für Korrekturansicht und Leseansicht.
%% Setzt den Schalter \ifkorrekturansicht voraus (gesetzt in den
%% einbindenden Dateien latex-korrekturansicht-abspann.tex bzw.
%% latex-leseansicht-abspann.tex).
%% ---------------------------------------------------------------

\normalsize

% Das esempio-Environment wird nur in der Leseansicht benötigt
\ifkorrekturansicht\else
\newenvironment{esempio}[3]%
{
    \vspace{1.5ex}
    \rlap{\underline{#1}}
    \par
    \setlength{\parindent}{0cm}
    \nopagebreak
    \leftskip=#2cm
    \rightskip=#3cm
}
{
    \par
}
\fi

\doendnotes{C}
\bigskip
\vfill

\clearpage

\footnotesize

\ifkorrekturansicht
  \lohead{\textsc{register}}
\fi

% theindex-Environment neu definieren ohne reledmac
\makeatletter
\renewenvironment{theindex}{%
  \ifkorrekturansicht
    \section*{\indexname}%
  \else
    \subsubsection*{Index der erwähnten Entitäten}%
  \fi
  \setlength{\parindent}{0pt}%
  \setlength{\parskip}{0pt plus 0.3pt}%
  \let\item\@idxitem
}{%
  \ifkorrekturansicht\clearpage\fi
}
\makeatother

\IfFileExists{\jobname-pw.ind}{\input{\jobname-pw.ind}}{}

% Quellenangabe nur in der Leseansicht
\ifkorrekturansicht\else
% Fallback-Definitionen, falls die .tex-Datei \titel etc. nicht gesetzt hat
\providecommand{\titel}{}
\providecommand{\editorInnen}{}
\providecommand{\dateiname}{\jobname}

\vspace{3cm}

\vfill

\footnotesize
\textsc{Quelle}: \titel. Herausgegeben von {\editorInnen}. In: \emph{Arthur Schnitzler: Briefwechsel mit Autorinnen und Autoren}.
 Digitale Edition, https://schnitzler-briefe.acdh.oeaw.ac.at/{\dateiname}.html (Stand \today)
\fi

\end{document}


