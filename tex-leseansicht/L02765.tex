%% latex-leseansicht-vorspann.tex
%% Vorspann für die Leseansicht.
%% Lädt die gemeinsame Datei latex-vorspann.tex mit nicht gesetztem Schalter.

\newif\ifkorrekturansicht
\korrekturansichtfalse

\input{../tex-inputs/latex-vorspann}


         
         \renewcommand{\erwaehntePersonen}{Personen: Hermann Bahr, Richard Beer-Hofmann, Clementine Goldmann, Max Halbe, Gerhart Hauptmann, Hugo von Hofmannsthal, Heinrich Kanner, Josef Rosengart, Leopold Sonnemann, Jean Thorel, Theodor Wolff}
         \renewcommand{\erwaehnteInstitutionen}{Institutionen: Berliner Tageblatt, Deutsches Theater Berlin, Frankfurter Zeitung}
         \renewcommand{\erwaehnteOrte}{Orte: Berlin, China, Deutsches Theater Berlin, Frankfurt am Main, Paris, Wien, rue Feydeau}
         \renewcommand{\erwaehnteWerke}{Werke: Die Zeit. Wiener Wochenschrift, Florian Geyer. Die Tragödie des Bauernkrieges, Frankfurter Zeitung, Freiwild. Schauspiel in 3 Akten, Lebenswende. Tragikomödie in 5 Akten, Liebelei. Schauspiel in drei Akten, Ostwärts um die Welt, Östlich um die Welt I.}
               \section[ Paul Goldmann an Arthur Schnitzler, 23. 1. {[}1896{]}]{ Paul Goldmann an Arthur Schnitzler, 23. 1. {[}1896{]}}\nopagebreak\mylabel{v}\rehead{ }\begin{ledgroupsized}[t]{13cm}\normalsize\beginnumbering \toendnotes[C]{\smallbreak\pagebreak[2]} \Standort{DLA, A:Schnitzler, HS.NZ85.1.3166.}
\physDesc{Brief, 3 Blätter, 11 Seiten
\newline{}Handschrift: blaue Tinte, deutsche Kurrent
\newline{}Schnitzler: 1) mit Bleistift das Jahr »96« vermerkt  2) mit rotem Buntstift vier Unterstreichungen}\toendnotes[C]{\smallbreak}\pstart
           \noindent{}{\pb}\textcolor{gray}{\textbf{\textbf{Frankfurter Zeitung\orgindex{Frankfurter Zeitung@Frankfurter Zeitung|pw}}}}\pend
           \pstart
           \textcolor{gray}{\textbf{(\begin{otherlanguage}{french}Gazette de Francfort\end{otherlanguage}\orgindex{Frankfurter Zeitung@Frankfurter Zeitung|pw}).}}\pend
           \pstart
           \textcolor{gray}{\textbf{\textbf{\begin{otherlanguage}{french}Fondateur M.\end{otherlanguage}{ }L. Sonnemann\pwindex{Sonnemann, Leopold 1831-10-29 – 1909-10-30@\textsc{Sonnemann, Leopold} (1831-10-29 – 1909-10-30), \emph{Journalist, Herausgeber}|pw}.}}}\pend
           \pstart
           \begin{otherlanguage}{french}\textcolor{gray}{\textbf{Journal\pwindex{?? Werk@Nicht ermittelte Verfasserinnen und Verfasser!Frankfurter Zeitung1856 – 1943@\emph{Frankfurter Zeitung} {[}1856 – 1943{]}|pwv} politique,
                        financier,}}\end{otherlanguage}\pend
           \pstart
           \begin{otherlanguage}{french}\textcolor{gray}{\textbf{commercial et littéraire.}}\end{otherlanguage}\pend
           \pstart
           \begin{otherlanguage}{french}\textcolor{gray}{\textbf{\textbf{Paraissant trois fois par jour.}}}\end{otherlanguage}\pend
           \pstart
           \begin{otherlanguage}{french}\textcolor{gray}{\textbf{\textbf{Bureau à Paris\oindex{Paris@\textbf{Paris}|pw}:}}}\end{otherlanguage}\pend
           \pstart
           \begin{otherlanguage}{french}\textcolor{gray}{\textbf{\textbf{24. Rue Feydeau\oindex{rue Feydeau@\textbf{rue Feydeau}|pw}.}}}\end{otherlanguage}\hfill \textsc{Paris\oindex{Paris@\textbf{Paris}|pw}}, 23. Januar.\pend
           \pstart\center{}Mein lieber Freund,\pend\pstart
           Wann iſt alſo die \label{K_L02765-1v}\edtext{Berlin\oindex{Berlin@\textbf{Berlin}|pw}er Aufführung\pwindex{Schnitzler, Arthur 15.05.1862 – 21.10.1931@\textsc{Schnitzler, Arthur} (15.05.1862 – 21.10.1931), \emph{Schriftsteller, Mediziner}!Liebelei. Schauspiel in drei Akten1895-10-09@\strich\emph{Liebelei. Schauspiel in drei Akten} {[}1895-10-09{]}|pwv}}{\lemma{\textnormal{\emph{Berliner Aufführung}}}\Cendnote{\textnormal{Die Premiere der \emph{Liebelei}\pwindex{Schnitzler, Arthur 15.05.1862 – 21.10.1931@\textsc{Schnitzler, Arthur} (15.05.1862 – 21.10.1931), \emph{Schriftsteller, Mediziner}!Liebelei. Schauspiel in drei Akten1895-10-09@\strich\emph{Liebelei. Schauspiel in drei Akten} {[}1895-10-09{]}|pwk} am Deutschen Theater
                     Berlin\oindex{Deutsches Theater Berlin@\textbf{Deutsches Theater Berlin}|pwk} fand am 4. 2. 1896 in Anwesenheit Schnitzler\pwindex{Schnitzler, Arthur 15.05.1862 – 21.10.1931@\textsc{Schnitzler, Arthur} (15.05.1862 – 21.10.1931), \emph{Schriftsteller, Mediziner}|pwk}s statt.}}}\label{K_L02765-1h}? Ich ſehe mit Vergnügen, wie ein Stück nach dem
               andern dort durchfällt: \textsc{\label{K_L02765-88v}\edtext{Hauptmann\pwindex{Hauptmann, Gerhart 15.11.1862 – 06.06.1946@\textsc{Hauptmann, Gerhart} (15.11.1862 – 06.06.1946), \emph{Schriftsteller}!Florian Geyer. Die Tragoedie des Bauernkrieges1896@\strich\emph{Florian Geyer. Die Tragödie des Bauernkrieges} {[}1896{]}|pwv}\pwindex{Hauptmann, Gerhart 15.11.1862 – 06.06.1946@\textsc{Hauptmann, Gerhart} (15.11.1862 – 06.06.1946), \emph{Schriftsteller}|pw}}{\lemma{\textnormal{\emph{Hauptmann}}}\Cendnote{\textnormal{Gerhart Hauptmann\pwindex{Hauptmann, Gerhart 15.11.1862 – 06.06.1946@\textsc{Hauptmann, Gerhart} (15.11.1862 – 06.06.1946), \emph{Schriftsteller}|pwk}: \emph{Florian Geyer. Die Tragödie des Bauernkrieges}\pwindex{Hauptmann, Gerhart 15.11.1862 – 06.06.1946@\textsc{Hauptmann, Gerhart} (15.11.1862 – 06.06.1946), \emph{Schriftsteller}!Florian Geyer. Die Tragoedie des Bauernkrieges1896@\strich\emph{Florian Geyer. Die Tragödie des Bauernkrieges} {[}1896{]}|pwk} hatte am
                        4. 1. 1896 am \emph{Deutschen
                        Theater}\orgindex{Deutsches Theater Berlin@Deutsches Theater Berlin|pwk} in Berlin\oindex{Berlin@\textbf{Berlin}|pwk} die
                     Uraufführung.}}}\label{K_L02765-88h}}, \textsc{\label{K_L02765-77v}\edtext{Halbe\pwindex{Halbe, Max 04.10.1865 – 30.11.1944@\textsc{Halbe, Max} (04.10.1865 – 30.11.1944), \emph{Schriftsteller}!Lebenswende. Tragikomoedie in 5 Akten1896-01-21@\strich\emph{Lebenswende. Tragikomödie in 5 Akten} {[}1896-01-21{]}|pwv}\pwindex{Halbe, Max 04.10.1865 – 30.11.1944@\textsc{Halbe, Max} (04.10.1865 – 30.11.1944), \emph{Schriftsteller}|pw}}{\lemma{\textnormal{\emph{Halbe}}}\Cendnote{\textnormal{Max Halbe\pwindex{Halbe, Max 04.10.1865 – 30.11.1944@\textsc{Halbe, Max} (04.10.1865 – 30.11.1944), \emph{Schriftsteller}|pwk}: \emph{Lebenswende. Tragikomödie in 5 Akten}\pwindex{Halbe, Max 04.10.1865 – 30.11.1944@\textsc{Halbe, Max} (04.10.1865 – 30.11.1944), \emph{Schriftsteller}!Lebenswende. Tragikomoedie in 5 Akten1896-01-21@\strich\emph{Lebenswende. Tragikomödie in 5 Akten} {[}1896-01-21{]}|pwk} hatte am
                        21. 1. 1896 am \emph{Deutschen
                        Theater}\orgindex{Deutsches Theater Berlin@Deutsches Theater Berlin|pwk} in Berlin\oindex{Berlin@\textbf{Berlin}|pwk} die Uraufführung.
                  }}}\label{K_L02765-77h}}{ }\textsc{etc}. Das iſt vom Schickſal glänzend arrangirt, um Deinen
               Erfolg \strikeout{ins rech} das nöthige Relief zu geben. Mein
               College \textsc{Wolff\pwindex{Wolff, Theodor 1868-08-02 – 1943-09-23@\textsc{Wolff, Theodor} (1868-08-02 – 1943-09-23), \emph{Schriftsteller, Journalist}|pw}} vom »Berl. Tageblatt\orgindex{Berliner Tageblatt@Berliner Tageblatt|pw}«, der Dir zu Deinem
                  Frankfurt\oindex{Frankfurt am Main@\textbf{Frankfurt am Main}|pw}er Erfolge\pwindex{Schnitzler, Arthur 15.05.1862 – 21.10.1931@\textsc{Schnitzler, Arthur} (15.05.1862 – 21.10.1931), \emph{Schriftsteller, Mediziner}!Liebelei. Schauspiel in drei Akten1895-10-09@\strich\emph{Liebelei. Schauspiel in drei Akten} {[}1895-10-09{]}|pwv} gratuliren läßt, läßt Dich auch fragen, ob er Dir in
                  Berlin\oindex{Berlin@\textbf{Berlin}|pw} irgendwie mit Einführungen dienen kann?
               Er kennt dort natürlich {\pb}die ganze Welt. Ich glaube,
               die beſte Einführung iſt Dein Stück\pwindex{Schnitzler, Arthur 15.05.1862 – 21.10.1931@\textsc{Schnitzler, Arthur} (15.05.1862 – 21.10.1931), \emph{Schriftsteller, Mediziner}!Liebelei. Schauspiel in drei Akten1895-10-09@\strich\emph{Liebelei. Schauspiel in drei Akten} {[}1895-10-09{]}|pwv} und Deine Perſon. Immerhin wollte ich Dir doch das Anerbieten
               übermitteln.\pend
           \pstart
           \textsc{Thorel\pwindex{Thorel, Jean 1859-09-11 – 1916-08-20@\textsc{Thorel, Jean} (1859-09-11 – 1916-08-20), \emph{Übersetzer, Dramatiker}|pw}} habe ich lange nicht geſehen; aber ſobald ich Zeit habe, ſuche ich ihn auf.\pend
           \pstart
           Daß Dir das \label{K_L02765-2v}\edtext{Opernglas}{\lemma{\textnormal{\emph{Opernglas}}}\Cendnote{\textnormal{siehe Paul Goldmann an Arthur Schnitzler, 11. 1. [1896]}}}\label{K_L02765-2h} gefällt, erſtaunt mich. Mir gefällt es nicht. Aber im Theater hat es ſich
               wohl bewährt? Ja? Was ſoll ich mit den 5 \textsc{Frcs} 40 machen,
               die mir von der Kaufſumme übrig bleiben?\pend
           \pstart
           \textsc{Bahr\pwindex{Bahr, Hermann 19.07.1863 – 15.01.1934@\textsc{Bahr, Hermann} (19.07.1863 – 15.01.1934), \emph{Schriftsteller, Kritiker}|pw}s} kleine \label{K_L02765-5v}\edtext{Erbärmlichkeiten}{\lemma{\textnormal{\emph{Erbärmlichkeiten}}}\Cendnote{\textnormal{Am 21. 1. 1896 kam
                  es zu einer Aussprache zwischen Schnitzler\pwindex{Schnitzler, Arthur 15.05.1862 – 21.10.1931@\textsc{Schnitzler, Arthur} (15.05.1862 – 21.10.1931), \emph{Schriftsteller, Mediziner}|pwk}
                  und Bahr\pwindex{Bahr, Hermann 19.07.1863 – 15.01.1934@\textsc{Bahr, Hermann} (19.07.1863 – 15.01.1934), \emph{Schriftsteller, Kritiker}|pwk}, die sowohl den Freundeskreis
                  betraf als auch die Reaktion Bahr\pwindex{Bahr, Hermann 19.07.1863 – 15.01.1934@\textsc{Bahr, Hermann} (19.07.1863 – 15.01.1934), \emph{Schriftsteller, Kritiker}|pwk}s auf den
                  Erfolg der \emph{Liebelei}\pwindex{Schnitzler, Arthur 15.05.1862 – 21.10.1931@\textsc{Schnitzler, Arthur} (15.05.1862 – 21.10.1931), \emph{Schriftsteller, Mediziner}!Liebelei. Schauspiel in drei Akten1895-10-09@\strich\emph{Liebelei. Schauspiel in drei Akten} {[}1895-10-09{]}|pwk}.}}}\label{K_L02765-5h} ſind recht
               heiter; {\pb}es werden ſchon größere nachfolgen, ſei
               beruhigt! Die »Zeit\pwindex{Zeit. Wiener Wochenschrift1894 – 1904@\emph{Die Zeit. Wiener Wochenschrift} {[}1894 – 1904{]}|pw}« leſe ich kaum mehr; ſie iſt
               gar zu ſchlecht geworden. Höchſtens hier und da ein Artikel von \textsc{Loris\pwindex{Hofmannsthal, Hugo von 1874-02-01 – 1929-07-15@\textsc{Hofmannsthal, Hugo von} (1874-02-01 – 1929-07-15), \emph{Schriftsteller}|pw}}, und auch an dem habe ich wenig Freude. Ich wende mich immer mehr von ihm ab,
               und vor Allem werde ich ihm nie verzeihen, daß er nicht in entſchiedener Weiſe
               zwiſchen Dir und \textsc{Bahr\pwindex{Bahr, Hermann 19.07.1863 – 15.01.1934@\textsc{Bahr, Hermann} (19.07.1863 – 15.01.1934), \emph{Schriftsteller, Kritiker}|pw}} gewählt hat. Lieſt Du \label{K_L02765-3v}\edtext{\textsc{Kanner\pwindex{Kanner, Heinrich 09.11.1864 – 15.02.1930@\textsc{Kanner, Heinrich} (09.11.1864 – 15.02.1930), \emph{Herausgeber, Publizist}|pw}s}{ }Feuilletons\pwindex{Kanner, Heinrich 09.11.1864 – 15.02.1930@\textsc{Kanner, Heinrich} (09.11.1864 – 15.02.1930), \emph{Herausgeber, Publizist}!Ostwaerts um die Welt1895-11-26 – 1896-02-29@\strich\emph{Ostwärts um die Welt} {[}1895-11-26 – 1896-02-29{]}|pwv} aus China\oindex{China@\textbf{China}|pw}}{\lemma{\textnormal{\emph{Kanners … China}}}\Cendnote{\textnormal{Heinrich Kanner\pwindex{Kanner, Heinrich 09.11.1864 – 15.02.1930@\textsc{Kanner, Heinrich} (09.11.1864 – 15.02.1930), \emph{Herausgeber, Publizist}|pwk} war im Auftrag der \emph{Frankfurter Zeitung}\orgindex{Frankfurter Zeitung@Frankfurter Zeitung|pwk} nach China\oindex{China@\textbf{China}|pwk} gereist und publizierte seine Reiseeindrücke in
                  dieser Zeitung\pwindex{?? Werk@Nicht ermittelte Verfasserinnen und Verfasser!Frankfurter Zeitung1856 – 1943@\emph{Frankfurter Zeitung} {[}1856 – 1943{]}|pwkv}. Das erste
                  Feuilleton – \emph{Östlich um die Welt. I}\pwindex{Kanner, Heinrich 09.11.1864 – 15.02.1930@\textsc{Kanner, Heinrich} (09.11.1864 – 15.02.1930), \emph{Herausgeber, Publizist}!Oestlich um die Welt I.1895-11-26@\strich\emph{Östlich um die Welt I.} {[}1895-11-26{]}|pwk} –
                  erschien am 26. 11. 1895 (Jg. 40, Nr. 328,
                     Erstes Morgenblatt, S. 1–3). Weitere Einträge – zumeist unter dem
                  Titel »Ostwärts um die Welt« – folgten am 2. 12. 1895,
                     12. 12. 1895, 24. 12. 1895,
                     15. 1. 1896, 22. 1. 1896,
                     26. 1. 1896, 8. 2. 1896 und am 
                     29. 2. 1896. Teilweise wurden sie auch in der
                  Wochenschrift \emph{Die Zeit}\pwindex{Zeit. Wiener Wochenschrift1894 – 1904@\emph{Die Zeit. Wiener Wochenschrift} {[}1894 – 1904{]}|pwk} nachgedruckt.}}}\label{K_L02765-3h}?
               Sie ſind erbärmlich. Der Mann\pwindex{Kanner, Heinrich 09.11.1864 – 15.02.1930@\textsc{Kanner, Heinrich} (09.11.1864 – 15.02.1930), \emph{Herausgeber, Publizist}|pwv} hat keine Augen und ſieht nichts.\pend
           \pstart
           {\pb}Natürlich waren meine Leute in Frankfurt\oindex{Frankfurt am Main@\textbf{Frankfurt am Main}|pw} von Dir entzückt, beſonders meine Mutter\pwindex{Goldmann, Clementine 1842-05-15 – 1924-02-24@\textsc{Goldmann, Clementine} (1842-05-15 – 1924-02-24)|pwv}. Mein Schwager\pwindex{Rosengart, Josef 1860-02-08 – 1927-08-04@\textsc{Rosengart, Josef} (1860-02-08 – 1927-08-04), \emph{Arzt}|pwv} findet, Du hätteſt Ähnlichkeit mit
               mir. Bedank’ Dich bei ihm für das Compliment.\pend
           \pstart
           Deine \label{K_L02765-4v}\edtext{Zweifel, Melancholien und
                  Hypochondrien}{\lemma{\textnormal{\emph{Zweifel, … Hypochondrien}}}\Cendnote{\textnormal{siehe A. S.: \emph{Tagebuch}, 27. 1. 1896 und 29. 1. 1896}}}\label{K_L02765-4h} nehme ich recht gleichmüthig auf. Das heißt, es thut mir innig leid, daß Du
               von alledem gequält wirſt. Aber da man auf \strikeout{Erden
                  ſchon} Erden ſchon einmal gequält werden muß, ſo iſt es beſſer, daß das Leid
               in dieſer Form an Dich \strikeout{heran}{ }{\pb}herantritt, als in einer andern. In dem, was Du
               ſchreibſt, iſt nichts, was nicht normal wäre. Du biſt ein großes Talent, und Du mußt
               infolgedeſſen naturnothwendig zu Zeiten glauben, daß Du es \uline{nicht} biſt. All’ das, was Du von Deinen Verſtimmungen ſchilderſt, – das iſt
               der \strikeout{Ne} Nebel, der im Grunde jeder Künſtlerſeele
               braut, \strikeout{und} – der Schöpfungsnebel, aus dem die
               Kunſtwerke erſtehen. Und ſo iſt des Künſtlers Erdenwallen: durch Verſtimmungen zur
               Stimmung! {\dots} Daß Dir {\pb}die
               Vergänglichkeit des Lebens wehthut, iſt traurig. Aber ich kann Dir darauf nur immer
               antworten: Wenn Du, wie jemand Anderer, den ich kenne, bereits immer am 15. jedes
               Monats mit Deinem Gehalt fertig wäreſt und nicht wüßteſt, woher Du Geld nehmen
               ſollſt, um weiter zu leben und Schulden zu zahlen – ſo hätteſt Du keine Zeit, Dich um
               die Vergänglichkeit des Lebens zu ſorgen. Und – ganz im Ernſt geſprochen – es iſt
               beſſer, vor dem Tode zu zittern, als vor {\pb}dem
               Schneider, der die unbezahlte Rechnung präſentiren kommt. Du haſt die edleren
               Schmerzen, mein lieber Freund – und ſelbſt hier biſt Du ein »Sonnt\substVorne{}\textsuperscript{g}\substDazwischen{}a\substHinten{}gskind«. Und wenn ich Deinen Kummer leſe, ſo ruft das in mir nur ein Gefühl
               des – Neides wach. Oh wenn ich auch ſo \strikeout{leid} leiden
               könnte, wie dieſer glückliche junge Mann! Und dann: Du erlebſt nichts zu Ende. Aber
               wenigſtens erlebſt Du etwas. Aber ich kenne {\pb}Leute,
               bei \strikeout{dene} denen es im ganzen Leben nie auch nur zum
               Anfang kommt. Das iſt das Entſetzliche, wenn man ſieht, wie das Leben vorüberraſt –
               wenn man mitleben möchte und nicht die Kraft dazu hat – wenn man eines ſchönen Tages
                  \strikeout{en} entdeckt, daß die Jugend vorbei iſt, ohne daß
               man jemals jung war – und wenn man genau weiß, daß das immer ſo ſein wird und daß man
               eines \strikeout{Ta} anderen ſchönen Tages auf das {\pb}ganze Leben zurückblicken wird mit dem Bewußtſein,
               mit der zehrenden Reue, daß man nie gelebt hat! Du hingegen – Du lebſt! Kein
               glühendes Gefühl des Daſeins – meinetwegen! Aber wo iſt es, dieſes glühende Gefühl,
               als bei den ganz Animaliſchen? Und auch bei denen, glaube ich, iſt es nicht ſo
               glühend. Ich meine, auch das iſt ein Ideal, das nicht exiſtirt. Alles Menſchliche iſt
                  \strikeout{unv} unvollkommen, und ich glaube, nicht einmal {\pb}\uline{leben} können wir ordentlich. Nicht Du allein –
               Keiner! Es gibt keine ganzen, keine glühenden Gefühle. Oder doch\strikeout{,} ein einziges: die \uline{Sehnſucht}. Was wir nicht haben – oh ja, in dem iſt Gluth, Schönheit und
               Vollendung. Aber in dem, was wir haben, – in dem, was wir leben, – da iſt Alles halb,
               jämmerlich und ungefähr.\pend
           \pstart
           {\pb}Schreib’ weiter an Deinem Stücke\pwindex{Schnitzler, Arthur 15.05.1862 – 21.10.1931@\textsc{Schnitzler, Arthur} (15.05.1862 – 21.10.1931), \emph{Schriftsteller, Mediziner}!Freiwild. Schauspiel in 3 Akten1896@\strich\emph{Freiwild. Schauspiel in 3 Akten} {[}1896{]}|pwv}, mein theurer Freund,
               und ſei guter Dinge!\pend
           \pstart
           In Treue {\\[\baselineskip]}Dein {\\[\baselineskip]}\spacefill\mbox{Paul Goldmann}\pend
           \leftskip=0em{}\pstart
           \noindent{}Und grüß’ mir meinen lieben \textsc{Richard\pwindex{Beer-Hofmann, Richard 1866-07-11 – 1945-09-26@\textsc{Beer-Hofmann, Richard} (1866-07-11 – 1945-09-26), \emph{Schriftsteller}|pw}}!\pend
           
         
         \endnumbering\mylabel{h}\end{ledgroupsized}  \newcommand{\dateiname}{L02765}\newcommand{\titel}{Paul Goldmann an Arthur Schnitzler, 23. 1. [1896]}\newcommand{\editorInnen}{Martin Anton Müller und Laura Untner}%% latex-leseansicht-abspann.tex
%% Abspann für die Leseansicht.
%% Der Schalter \ifkorrekturansicht ist bereits durch den Vorspann gesetzt.

%% latex-abspann.tex
%% Gemeinsamer Abspann für Korrekturansicht und Leseansicht.
%% Setzt den Schalter \ifkorrekturansicht voraus (gesetzt in den
%% einbindenden Dateien latex-korrekturansicht-abspann.tex bzw.
%% latex-leseansicht-abspann.tex).
%% ---------------------------------------------------------------

\normalsize

% Das esempio-Environment wird nur in der Leseansicht benötigt
\ifkorrekturansicht\else
\newenvironment{esempio}[3]%
{
    \vspace{1.5ex}
    \rlap{\underline{#1}}
    \par
    \setlength{\parindent}{0cm}
    \nopagebreak
    \leftskip=#2cm
    \rightskip=#3cm
}
{
    \par
}
\fi

\doendnotes{C}
\bigskip
\vfill

\clearpage

\footnotesize

\ifkorrekturansicht
  \lohead{\textsc{register}}
\fi

% theindex-Environment neu definieren ohne reledmac
\makeatletter
\renewenvironment{theindex}{%
  \ifkorrekturansicht
    \section*{\indexname}%
  \else
    \subsubsection*{Index der erwähnten Entitäten}%
  \fi
  \setlength{\parindent}{0pt}%
  \setlength{\parskip}{0pt plus 0.3pt}%
  \let\item\@idxitem
}{%
  \ifkorrekturansicht\clearpage\fi
}
\makeatother

\IfFileExists{\jobname-pw.ind}{\input{\jobname-pw.ind}}{}

% Quellenangabe nur in der Leseansicht
\ifkorrekturansicht\else
% Fallback-Definitionen, falls die .tex-Datei \titel etc. nicht gesetzt hat
\providecommand{\titel}{}
\providecommand{\editorInnen}{}
\providecommand{\dateiname}{\jobname}

\vspace{3cm}

\vfill

\footnotesize
\textsc{Quelle}: \titel. Herausgegeben von {\editorInnen}. In: \emph{Arthur Schnitzler: Briefwechsel mit Autorinnen und Autoren}.
 Digitale Edition, https://schnitzler-briefe.acdh.oeaw.ac.at/{\dateiname}.html (Stand \today)
\fi

\end{document}


      