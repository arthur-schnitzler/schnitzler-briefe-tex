%% latex-korrekturansicht-vorspann.tex
%% Vorspann für die Korrekturansicht.
%% Lädt die gemeinsame Datei latex-vorspann.tex mit gesetztem Schalter.

\newif\ifkorrekturansicht
\korrekturansichttrue

\input{../tex-inputs/latex-vorspann}


\section[Jakob Julius David an Arthur Schnitzler, 13. 5. 1901]{L01119 Jakob Julius David an Arthur Schnitzler, 13. 5. 1901}
\nopagebreak\mylabel{L01119v}
\rehead{ }\normalsize\beginnumbering\briefempfaengerindex{Schnitzler, Arthur@\textsc{Schnitzler, Arthur}!zzzDavid, Jakob Julius@\emph{von Jakob Julius David}!1901-05-132@{13. 5. 1901}|(be}
\toendnotes[C]{\smallbreak\pagebreak[2]}\Standort{CUL, Schnitzler, B 25.}
\physDesc{Visitenkarte, 182 Zeichen
\newline{}Handschrift: schwarze Tinte, lateinische Kurrent
\newline{}Ordnung: mit Bleistift von unbekannter Hand nummeriert:
                                 »6« }
\pstart
           \noindent{}\centering{}{\pb}\textcolor{gray}{\textbf{D\textsuperscript{r.} J. J. David}}\pend
           
\pstart
           \raggedleft{}\textcolor{gray}{\textbf{IX., Glasergasse 4 \textsc{a}\oindex{Glasergasse@\textbf{Glasergasse}, \emph{Straße (K.STR)}|pw}}}\pend
           \selectlanguage{ngerman}\vspace{1em}
\pstart
           \raggedleft{}{\pb}13/5 01\pend
           
\pstart\center{}Lieber D\textsuperscript{r} Schnitzler!\pend\vspace{0.5em}
\pstart
           Ich bin nun stark in Ihrer Buchschuld. Laſsen Sie mir Zeit. Alles soll gelesen und
               bedacht sein, sowie ich freieren Kopf bekomme.\pend
           
\pstart
           Bestens wie immer Ihr{\\[\baselineskip]}\spacefill\mbox{David}\pend
           \leftskip=0em{}\selectlanguage{ngerman}\endnumbering\briefempfaengerindex{Schnitzler, Arthur@\textsc{Schnitzler, Arthur}!zzzDavid, Jakob Julius@\emph{von Jakob Julius David}!1901-05-132@{13. 5. 1901}|)be}\mylabel{L01119h}  \normalsize

\doendnotes{C}
\bigskip
\vfill

\clearpage

\footnotesize

\lohead{\textsc{register}}

% Definiere theindex-Environment komplett neu ohne reledmac
\makeatletter
\renewenvironment{theindex}{%
  \section*{\indexname}%
  \setlength{\parindent}{0pt}%
  \setlength{\parskip}{0pt plus 0.3pt}%
  \let\item\@idxitem
}{%
  \clearpage
}
\makeatother

\IfFileExists{\jobname-pw.ind}{\input{\jobname-pw.ind}}{}

\end{document}

      