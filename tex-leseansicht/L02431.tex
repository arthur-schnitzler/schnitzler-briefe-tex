%% latex-leseansicht-vorspann.tex
%% Vorspann für die Leseansicht.
%% Lädt die gemeinsame Datei latex-vorspann.tex mit nicht gesetztem Schalter.

\newif\ifkorrekturansicht
\korrekturansichtfalse

\input{../tex-inputs/latex-vorspann}


         
         \newcommand{\erwaehntePersonen}{Personen: Richard Beer-Hofmann, Hans Bodmer, Herbert Steiner}
         \newcommand{\erwaehnteInstitutionen}{}
         \newcommand{\erwaehnteOrte}{Orte: Hasenauerstraße, Sankt Moritz, St. Gallen, Wien, XVIII., Währing, Zürich}
         \newcommand{\erwaehnteWerke}{
               \section[Arthur Schnitzler an Richard Beer-Hofmann, 23. 1. 1925]{ Arthur Schnitzler an Richard Beer-Hofmann, 23. 1. 1925}\nopagebreak\mylabel{v}\rehead{ }\begin{ledgroupsized}[t]{13cm}\normalsize\beginnumbering \toendnotes[C]{\smallbreak\pagebreak[2]} \Standort{YCGL, MSS 31.}
\physDesc{Bildpostkarte
\newline{}Handschrift: Bleistift, lateinische Kurrent\newline{}Versand: Stempel: »\nobreak{}\oindex{Sankt Moritz@\textbf{Sankt Moritz}|pwk}{[}St. Moritz{]}-Dorf, 23. 1. 25, 21\nobreak{}«.  }\buchAbdrucke{\weitereDrucke{Arthur Schnitzler, Richard Beer-Hofmann: \emph{Briefwechsel 1891–1931}. Hg. Konstanze Fliedl. Wien, Zürich: \emph{Europaverlag} 1992, S. 229.} }\toendnotes[C]{\smallbreak}\pstart{}{\pb}Hrn\pend{}\pstart{}Dr Richard Beerhofma{\geminationn}\pend{}\pstart{}Wien XVIII\oindex{XVIII., Waehring@\textbf{XVIII., Währing}|pw}\pend{}\pstart{}Hasenauerstr 59\oindex{Hasenauerstrasse@\textbf{Hasenauerstraße}|pw}.\pend{}{\bigskip}\pstart
           \noindent{}\centering{}{\pb}\textcolor{gray}{\textbf{St. Moritz-Dorf\oindex{Sankt Moritz@\textbf{Sankt Moritz}|pw} im
                     Winter}}\pend
           \pstart
           \raggedleft{}{\pb}23. 1. 25\pend
           \pstart
           Herzlichste Grüße. Klangs Ihnen neulich nicht im Ohr – wir sprachen in Zürich\oindex{Zuerich@\textbf{Zürich}|pw} so viel von Ihnen, der steinerne Herbert\pwindex{Steiner, Herbert 15.08.1892 – 09.02.1966@\textsc{Steiner, Herbert} (15.08.1892 – 09.02.1966)|pw} und Bodmer\pwindex{Bodmer, Hans 15.09.1863 – 18.09.1948@\textsc{Bodmer, Hans} (15.09.1863 – 18.09.1948), \emph{Veranstaltungsorganisator, Germanist}|pw}. –\pend
           \pstart
           Es war ganz schön – gegen Schluſs, – 6. u 7. Vorlesung {\pb}\label{T_L02431_1v}\edtext{Gallen\oindex{St. Gallen@\textbf{St. Gallen}|pw}, u Zürich\oindex{Zuerich@\textbf{Zürich}|pw},
               gings nicht ganz ohne}{\lemma{\textnormal{\emph{Gallen, … ohne}}}\Cendnote{\textnormal{über dem Bild}}}\label{T_L02431_1h}{ }\label{T_L02431_2v}\edtext{Erkältung
                  ab.}{\lemma{\textnormal{\emph{Erkältung
                  ab.}}}\Cendnote{\textnormal{weiter
                  unter dem Bild}}}\label{T_L02431_2h} – (Kurort, im Winter). \pend
           \pstart Ihr \spacefill\mbox{A.}\pend{}
         
         \endnumbering\mylabel{h}\end{ledgroupsized}  \newcommand{\dateiname}{L02431}\newcommand{\titel}{Arthur Schnitzler an Richard Beer-Hofmann, 23. 1. 1925}\newcommand{\editorInnen}{Martin Anton Müller und Gerd-Hermann Susen}%% latex-leseansicht-abspann.tex
%% Abspann für die Leseansicht.
%% Der Schalter \ifkorrekturansicht ist bereits durch den Vorspann gesetzt.

%% latex-abspann.tex
%% Gemeinsamer Abspann für Korrekturansicht und Leseansicht.
%% Setzt den Schalter \ifkorrekturansicht voraus (gesetzt in den
%% einbindenden Dateien latex-korrekturansicht-abspann.tex bzw.
%% latex-leseansicht-abspann.tex).
%% ---------------------------------------------------------------

\normalsize

% Das esempio-Environment wird nur in der Leseansicht benötigt
\ifkorrekturansicht\else
\newenvironment{esempio}[3]%
{
    \vspace{1.5ex}
    \rlap{\underline{#1}}
    \par
    \setlength{\parindent}{0cm}
    \nopagebreak
    \leftskip=#2cm
    \rightskip=#3cm
}
{
    \par
}
\fi

\doendnotes{C}
\bigskip
\vfill

\clearpage

\footnotesize

\ifkorrekturansicht
  \lohead{\textsc{register}}
\fi

% theindex-Environment neu definieren ohne reledmac
\makeatletter
\renewenvironment{theindex}{%
  \ifkorrekturansicht
    \section*{\indexname}%
  \else
    \subsubsection*{Index der erwähnten Entitäten}%
  \fi
  \setlength{\parindent}{0pt}%
  \setlength{\parskip}{0pt plus 0.3pt}%
  \let\item\@idxitem
}{%
  \ifkorrekturansicht\clearpage\fi
}
\makeatother

\IfFileExists{\jobname-pw.ind}{\input{\jobname-pw.ind}}{}

% Quellenangabe nur in der Leseansicht
\ifkorrekturansicht\else
% Fallback-Definitionen, falls die .tex-Datei \titel etc. nicht gesetzt hat
\providecommand{\titel}{}
\providecommand{\editorInnen}{}
\providecommand{\dateiname}{\jobname}

\vspace{3cm}

\vfill

\footnotesize
\textsc{Quelle}: \titel. Herausgegeben von {\editorInnen}. In: \emph{Arthur Schnitzler: Briefwechsel mit Autorinnen und Autoren}.
 Digitale Edition, https://schnitzler-briefe.acdh.oeaw.ac.at/{\dateiname}.html (Stand \today)
\fi

\end{document}


      