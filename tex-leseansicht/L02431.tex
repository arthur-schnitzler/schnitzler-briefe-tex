%% latex-leseansicht-vorspann.tex
%% Vorspann für die Leseansicht.
%% Lädt die gemeinsame Datei latex-vorspann.tex mit nicht gesetztem Schalter.

\newif\ifkorrekturansicht
\korrekturansichtfalse

\input{../tex-inputs/latex-vorspann}


\section[Arthur Schnitzler an Richard Beer-Hofmann, 23. 1. 1925]{L02431 Arthur Schnitzler an Richard Beer-Hofmann, 23. 1. 1925}
\nopagebreak\mylabel{L02431v}
\rehead{ }\normalsize\beginnumbering\briefempfaengerindex{Beer-Hofmann, Richard@\textsc{Beer-Hofmann, Richard}!zzzSchnitzler, Arthur@\emph{von Arthur Schnitzler}!1925-01-231@{23. 1. 1925}|(be}
\toendnotes[C]{\smallbreak\pagebreak[2]}
\correspDesc{Versand  durch Arthur Schnitzler am 23. 1. 1925 in Sankt Moritz
\newline{}Erhalt  durch Richard Beer-Hofmann im Zeitraum [24. 1. 1925
                  – 28. 1. 1925?] in Wien}\toendnotes[C]{\smallbreak}
\Standort{YCGL, MSS 31.}
\physDesc{Bildpostkarte, 316 Zeichen
\newline{}Handschrift: Bleistift, lateinische Kurrent
\newline{}Versand: Stempel: »\nobreak{}\oindex{St. Moritz@\textbf{St. Moritz}|pwk}{[}St. Moritz{]}-Dorf, 23. 1. 25, 21\nobreak{}«.  }
\buchAbdrucke{\weitereDrucke{Arthur Schnitzler, Richard Beer-Hofmann: \emph{Briefwechsel 1891–1931}. Herausgegeben von Konstanze Fliedl. Wien, Zürich: \emph{Europaverlag} 1992, S. 229.} }\toendnotes[C]{\smallbreak}\pstart{}{\pb}Hrn\pend{}\pstart{}Dr Richard Beerhofma{\geminationn}\pend{}\pstart{}Wien XVIII\oindex{XVIII., Währing@\textbf{XVIII., Währing}, \emph{Verwaltungsgebiet}|pw}\pend{}\pstart{}Hasenauerstr 59\oindex{Wien@\textbf{Wien}!XVIII., Währing@\textbf{XVIII., Währing}!Hasenauerstraße 59@\textbf{Hasenauerstraße 59}, \emph{Wohngebäude}|pw}.\pend{}{\bigskip}
\pstart
           \noindent{}\centering{}{\pb}\textcolor{gray}{\textbf{St. Moritz-Dorf\oindex{St. Moritz@\textbf{St. Moritz}|pw} im Winter}}\pend
           \vspace{1em}
\pstart
           \raggedleft{}{\pb}23. 1. 25\pend
           \vspace{0.5em}
\pstart
           Herzlichste Grüße. Klangs Ihnen neulich nicht im Ohr – wir sprachen in Zürich\oindex{Zürich@\textbf{Zürich}|pw} so viel von Ihnen, der steinerne Herbert\pwindex{Steiner, Herbert 15.\,8.\,1892 Wien – 9.\,2.\,1966 Genf@\textsc{Steiner, Herbert} (15.\,8.\,1892 Wien – 9.\,2.\,1966 Genf)|pw} und Bodmer\pwindex{Bodmer, Hans 15.\,9.\,1863 Zürich – 18.\,9.\,1948 ebd.@\textsc{Bodmer, Hans} (15.\,9.\,1863 Zürich – 18.\,9.\,1948 ebd.), \emph{Veranstaltungsorganisator, Germanist}|pw}. –\pend
           
\pstart
           Es war ganz schön – gegen Schluſs, – 6. u 7. Vorlesung {\pb}\label{T_L02431-1v}\edtext{Gallen\oindex{St. Gallen@\textbf{St. Gallen}|pw}, u Zürich\oindex{Zürich@\textbf{Zürich}|pw}, gings nicht ganz ohne}{\lemma{\textnormal{\emph{Gallen, … ohne}}}\Cendnote{\textnormal{über dem Bild}}}\label{T_L02431-1}{ }\label{T_L02431-2v}\edtext{Erkältung ab.}{\lemma{\textnormal{\emph{Erkältung ab.}}}\Cendnote{\textnormal{weiter unter dem Bild}}}\label{T_L02431-2} – (Kurort, im Winter).\pend
           \pstart Ihr \spacefill\mbox{A.}\pend{}\selectlanguage{ngerman}\endnumbering\briefempfaengerindex{Beer-Hofmann, Richard@\textsc{Beer-Hofmann, Richard}!zzzSchnitzler, Arthur@\emph{von Arthur Schnitzler}!1925-01-231@{23. 1. 1925}|)be}\mylabel{L02431h}  \newcommand{\dateiname}{L02431}\newcommand{\titel}{Arthur Schnitzler an Richard Beer-Hofmann, 23. 1. 1925}\newcommand{\editorInnen}{Martin Anton Müller und Gerd-Hermann Susen}%% latex-leseansicht-abspann.tex
%% Abspann für die Leseansicht.
%% Der Schalter \ifkorrekturansicht ist bereits durch den Vorspann gesetzt.

%% latex-abspann.tex
%% Gemeinsamer Abspann für Korrekturansicht und Leseansicht.
%% Setzt den Schalter \ifkorrekturansicht voraus (gesetzt in den
%% einbindenden Dateien latex-korrekturansicht-abspann.tex bzw.
%% latex-leseansicht-abspann.tex).
%% ---------------------------------------------------------------

\normalsize

% Das esempio-Environment wird nur in der Leseansicht benötigt
\ifkorrekturansicht\else
\newenvironment{esempio}[3]%
{
    \vspace{1.5ex}
    \rlap{\underline{#1}}
    \par
    \setlength{\parindent}{0cm}
    \nopagebreak
    \leftskip=#2cm
    \rightskip=#3cm
}
{
    \par
}
\fi

\doendnotes{C}
\bigskip
\vfill

\clearpage

\footnotesize

\ifkorrekturansicht
  \lohead{\textsc{register}}
\fi

% theindex-Environment neu definieren ohne reledmac
\makeatletter
\renewenvironment{theindex}{%
  \ifkorrekturansicht
    \section*{\indexname}%
  \else
    \subsubsection*{Index der erwähnten Entitäten}%
  \fi
  \setlength{\parindent}{0pt}%
  \setlength{\parskip}{0pt plus 0.3pt}%
  \let\item\@idxitem
}{%
  \ifkorrekturansicht\clearpage\fi
}
\makeatother

\IfFileExists{\jobname-pw.ind}{\input{\jobname-pw.ind}}{}

% Quellenangabe nur in der Leseansicht
\ifkorrekturansicht\else
% Fallback-Definitionen, falls die .tex-Datei \titel etc. nicht gesetzt hat
\providecommand{\titel}{}
\providecommand{\editorInnen}{}
\providecommand{\dateiname}{\jobname}

\vspace{3cm}

\vfill

\footnotesize
\textsc{Quelle}: \titel. Herausgegeben von {\editorInnen}. In: \emph{Arthur Schnitzler: Briefwechsel mit Autorinnen und Autoren}.
 Digitale Edition, https://schnitzler-briefe.acdh.oeaw.ac.at/{\dateiname}.html (Stand \today)
\fi

\end{document}


