%% latex-leseansicht-vorspann.tex
%% Vorspann für die Leseansicht.
%% Lädt die gemeinsame Datei latex-vorspann.tex mit nicht gesetztem Schalter.

\newif\ifkorrekturansicht
\korrekturansichtfalse

\input{../tex-inputs/latex-vorspann}


\section[Hermann Bahr an Arthur Schnitzler, 5. 7. 1901]{L01143 Hermann Bahr an Arthur Schnitzler, 5. 7. 1901}
\nopagebreak\mylabel{L01143v}
\rehead{ }\normalsize\beginnumbering\briefempfaengerindex{Schnitzler, Arthur@\textsc{Schnitzler, Arthur}!zzzBahr, Hermann@\emph{von Hermann Bahr}!1901-07-051@{5. 7. 1901}|(be}
\toendnotes[C]{\smallbreak\pagebreak[2]}
\correspDesc{Versand  durch Hermann Bahr am 5. 7. 1901 in Wien
\newline{}Erhalt  durch Arthur Schnitzler im Zeitraum [5. 7. 1901
                  – 9. 7. 1901?] in Wien}\toendnotes[C]{\smallbreak}
\Standort{CUL, Schnitzler, B 5b.}
\physDesc{Brief, 1 Blatt, 4 Seiten, 1373 Zeichen
\newline{}Handschrift: schwarze Tinte, deutsche Kurrent
\newline{}Schnitzler: mit Bleistift die Jahreszahl »901« ergänzt 
\newline{}Ordnung: mit Bleistift von unbekannter Hand nummeriert:
                                    »78« }
\buchAbdrucke{\weitereDrucke{Hermann Bahr, Arthur Schnitzler: \emph{Briefwechsel, Aufzeichnungen, Dokumente (1891–1931)}. Herausgegeben von Kurt Ifkovits und Martin Anton Müller. Göttingen: \emph{Wallstein} 2018, S. 212–213.} }\toendnotes[C]{\smallbreak}
\pstart
           \raggedleft{}{\pb}\substVorne{}\textsuperscript{7}\substDazwischen{}5\substHinten{}/7\pend
           
\pstart\center{}Lieber Arthur!\pend\vspace{0.5em}
\pstart
           Ich danke Dir herzlich für Deinen lieben Brief. Ich habe neulich mit Hugo\pwindex{Hofmannsthal, Hugo von 1.\,2.\,1874 Wien – 15.\,7.\,1929 Rodaun@\textsc{Hofmannsthal, Hugo von} (1.\,2.\,1874 Wien – 15.\,7.\,1929 Rodaun), \emph{Schriftsteller}|pw} davon geſprochen, wie es mich freut, zu
               Dir endlich ein aufrichtiges und gutes Verhältnis gefunden zu haben und zu empfinden,
               daß \introOben{}es\introOben{} wohl nicht mehr geſtört werden kann, mögen unſere
               Meinungen immerhin auch künftig noch manchmal auseinandergehen.\pend
           
\pstart
           {\pb}Hugo\pwindex{Hofmannsthal, Hugo von 1.\,2.\,1874 Wien – 15.\,7.\,1929 Rodaun@\textsc{Hofmannsthal, Hugo von} (1.\,2.\,1874 Wien – 15.\,7.\,1929 Rodaun), \emph{Schriftsteller}|pw} iſt{ }ſehr{ }ſtolz, weil er das Gefühl hat,
               in dieſer Sache von jeher geſcheiter geweſen zu{ }ſein, als wir es Jahre lang
               waren.\pend
           
\pstart
           Für Pötzl\pwindex{Pötzl, Eduard 17.\,3.\,1851 Wien – 20.\,8.\,1914 Mödling@\textsc{Pötzl, Eduard} (17.\,3.\,1851 Wien – 20.\,8.\,1914 Mödling), \emph{Schriftsteller, Journalist}|pw} kann ich,{ }ſo unerfreulich er{ }ſich
               gegen mich, mit anonymen Briefen und auf Hintertreppen operierend, fortgeſetzt
               benimmt, ein{[}e{]}{ }ſtille Bewunderung nicht los werden, weil er doch
               das vollendet{\pb}ſte Exemplar des biederen Wieners\oindex{Wien@\textbf{Wien}, \emph{Verwaltungsgebiet}|pw} iſt, und mir immer nur leid thut, daß ihn
                  Flaubert\pwindex{Flaubert, Gustave 12.\,12.\,1821 Rouen – 8.\,5.\,1880 Canteleu@\textsc{Flaubert, Gustave} (12.\,12.\,1821 Rouen – 8.\,5.\,1880 Canteleu), \emph{Schriftsteller}|pw} nicht gekannt hat, der ein wahres
               Freudengeheul über ihn ausgestoßen hätte. »\label{K_L01143-1v}\edtext{Den Arier}{\lemma{\textnormal{\emph{Den Arier}}}\Cendnote{\textnormal{Pötzl\pwindex{Pötzl, Eduard 17.\,3.\,1851 Wien – 20.\,8.\,1914 Mödling@\textsc{Pötzl, Eduard} (17.\,3.\,1851 Wien – 20.\,8.\,1914 Mödling), \emph{Schriftsteller, Journalist}|pwk} behandelte in seinen Texten häufig
                     Wien\oindex{Wien@\textbf{Wien}, \emph{Verwaltungsgebiet}|pwk}er Typen.}}}\label{K_L01143-1}« müßte einmal Jemand{ }ſchildern und müßte einmal die andere Seite der »armen Spielleute« zeigen, den
               gemütlichen \label{K_L01143-2v}\edtext{Naderer}{\lemma{\textnormal{\emph{Naderer}}}\Cendnote{\textnormal{österreichisch: Verräter, Petze}}}\label{K_L01143-2}, der
               eigentlich der Grundtypus des Öſtreichers\oindex{Österreich@\textbf{Österreich}|pw} zu{ }ſein{ }ſcheint, was irgendwie {\pb}ſehr tief mit dem
               Katholicismus zuſammen\introOben{}zu\introOben{}hängen{ }ſcheint – worüber Poldi\pwindex{Andrian-Werburg, Leopold von 9.\,5.\,1875 Berlin – 19.\,11.\,1951 Fribourg@\textsc{Andrian-Werburg, Leopold von} (9.\,5.\,1875 Berlin – 19.\,11.\,1951 Fribourg), \emph{Schriftsteller, Diplomat}|pw} und Hugo\pwindex{Hofmannsthal, Hugo von 1.\,2.\,1874 Wien – 15.\,7.\,1929 Rodaun@\textsc{Hofmannsthal, Hugo von} (1.\,2.\,1874 Wien – 15.\,7.\,1929 Rodaun), \emph{Schriftsteller}|pw} freilich Zeter und Mordio{ }ſchreien würden. Pötzl\pwindex{Pötzl, Eduard 17.\,3.\,1851 Wien – 20.\,8.\,1914 Mödling@\textsc{Pötzl, Eduard} (17.\,3.\,1851 Wien – 20.\,8.\,1914 Mödling), \emph{Schriftsteller, Journalist}|pw} oder der Herr \label{LL238-1v}Davis\pwindex{Davis, Gustav 3.\,3.\,1856 Bratislava – 21.\,8.\,1951 Hollenstein an der Ybbs@\textsc{Davis, Gustav} (3.\,3.\,1856 Bratislava – 21.\,8.\,1951 Hollenstein an der Ybbs), \emph{Journalist, Herausgeber}|pw} von der »Reichswehr\orgindex{Reichswehr@Reichswehr|pw}«\label{LL238-1h} oder der Ton des \label{K_L01143-3v}\edtext{Kikeriki\orgindex{Kikeriki@Kikeriki|pw}}{\lemma{\textnormal{\emph{Kikeriki}}}\Cendnote{\textnormal{antisemitische Satirezeitschrift}}}\label{K_L01143-3}
               – das{ }ſind lauter Sachen, die an den Hof Philipps\pwindex{Phillipp II. von Spanien 21.\,5.\,1527 Valladolid – 13.\,9.\,1598 El Escorial@\textsc{Phillipp II. von Spanien} (21.\,5.\,1527 Valladolid – 13.\,9.\,1598 El Escorial), \emph{König}|pw} gehören und die ich mir großartig von \textsc{Velasquez}\pwindex{Velázquez, Diego Rodríguez de Silva y 6.\,6.\,1599 Sevilla – 6.\,8.\,1660 Madrid@\textsc{Velázquez, Diego Rodríguez de Silva y} (6.\,6.\,1599 Sevilla – 6.\,8.\,1660 Madrid), \emph{Maler}|pw} gemalt denken könnte.\pend
           
\pstart
           Einen guten Sommer wünſcht Dir{\\[\baselineskip]}herzlichſt{\\[\baselineskip]}Dein{\\[\baselineskip]}\spacefill\mbox{Hermann}\pend
           \leftskip=0em{}\selectlanguage{ngerman}\endnumbering\briefempfaengerindex{Schnitzler, Arthur@\textsc{Schnitzler, Arthur}!zzzBahr, Hermann@\emph{von Hermann Bahr}!1901-07-051@{5. 7. 1901}|)be}\mylabel{L01143h}  \newcommand{\dateiname}{L01143}\newcommand{\titel}{Hermann Bahr an Arthur Schnitzler, 5. 7. 1901}\newcommand{\editorInnen}{Herausgegeben von Martin Anton Müller}%% latex-leseansicht-abspann.tex
%% Abspann für die Leseansicht.
%% Der Schalter \ifkorrekturansicht ist bereits durch den Vorspann gesetzt.

%% latex-abspann.tex
%% Gemeinsamer Abspann für Korrekturansicht und Leseansicht.
%% Setzt den Schalter \ifkorrekturansicht voraus (gesetzt in den
%% einbindenden Dateien latex-korrekturansicht-abspann.tex bzw.
%% latex-leseansicht-abspann.tex).
%% ---------------------------------------------------------------

\normalsize

% Das esempio-Environment wird nur in der Leseansicht benötigt
\ifkorrekturansicht\else
\newenvironment{esempio}[3]%
{
    \vspace{1.5ex}
    \rlap{\underline{#1}}
    \par
    \setlength{\parindent}{0cm}
    \nopagebreak
    \leftskip=#2cm
    \rightskip=#3cm
}
{
    \par
}
\fi

\doendnotes{C}
\bigskip
\vfill

\clearpage

\footnotesize

\ifkorrekturansicht
  \lohead{\textsc{register}}
\fi

% theindex-Environment neu definieren ohne reledmac
\makeatletter
\renewenvironment{theindex}{%
  \ifkorrekturansicht
    \section*{\indexname}%
  \else
    \subsubsection*{Index der erwähnten Entitäten}%
  \fi
  \setlength{\parindent}{0pt}%
  \setlength{\parskip}{0pt plus 0.3pt}%
  \let\item\@idxitem
}{%
  \ifkorrekturansicht\clearpage\fi
}
\makeatother

\IfFileExists{\jobname-pw.ind}{\input{\jobname-pw.ind}}{}

% Quellenangabe nur in der Leseansicht
\ifkorrekturansicht\else
% Fallback-Definitionen, falls die .tex-Datei \titel etc. nicht gesetzt hat
\providecommand{\titel}{}
\providecommand{\editorInnen}{}
\providecommand{\dateiname}{\jobname}

\vspace{3cm}

\vfill

\footnotesize
\textsc{Quelle}: \titel. Herausgegeben von {\editorInnen}. In: \emph{Arthur Schnitzler: Briefwechsel mit Autorinnen und Autoren}.
 Digitale Edition, https://schnitzler-briefe.acdh.oeaw.ac.at/{\dateiname}.html (Stand \today)
\fi

\end{document}


