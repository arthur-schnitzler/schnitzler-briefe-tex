%% latex-korrekturansicht-vorspann.tex
%% Vorspann für die Korrekturansicht.
%% Lädt die gemeinsame Datei latex-vorspann.tex mit gesetztem Schalter.

\newif\ifkorrekturansicht
\korrekturansichttrue

\input{../tex-inputs/latex-vorspann}


\section[Hugo von Hofmannsthal und Louise Schnitzler an Arthur Schnitzler, 5. 9. 1907]{L01705 Hugo von Hofmannsthal und Louise Schnitzler an Arthur Schnitzler,
               5. 9. 1907}
\nopagebreak\mylabel{L01705v}
\rehead{ }\normalsize\beginnumbering\briefempfaengerindex{Schnitzler, Arthur@\textsc{Schnitzler, Arthur}!zzzSchnitzler, Louise@\emph{von Louise Schnitzler}!1907-09-051@{5. 9. 1907}|(be}\briefempfaengerindex{Schnitzler, Arthur@\textsc{Schnitzler, Arthur}!zzzHofmannsthal, Hugo von@\emph{von Hugo von Hofmannsthal}!1907-09-051@{5. 9. 1907}|(be}
\toendnotes[C]{\smallbreak\pagebreak[2]}\Standort{CUL, Schnitzler, B 43.}
\physDesc{Bildpostkarte, 446 Zeichen
\newline{}Handschrift: 1) Bleistift, deutsche Kurrent\hspace{1em}2) Bleistift, lateinische Kurrent (\noindent{}Adresse)\hspace{1em}
\newline{}Versand: Stempel: »\nobreak{}\oindex{Semmering@\textbf{Semmering}, \emph{A.ADM3}|pwk}Semmering 1, 5. IX. 07, 6\nobreak{}«.  
\newline{}Ordnung: 1) mit Bleistift von unbekannter Hand nummeriert: »\strikeout{302}«  2) mit Bleistift von unbekannter Hand nummeriert:
                                    »284«}
\buchAbdrucke{\weitereDrucke{Hugo von Hofmannsthal, Arthur Schnitzler: \emph{Briefwechsel}. Frankfurt am Main: \emph{S. Fischer} 1964, S. 230.} }\toendnotes[C]{\smallbreak}\pstart{}{\pb}Herrn D\textsuperscript{r}\pend{}\pstart{}Arthur Schnitzler\pend{}\pstart{}Meran\oindex{Meran@\textbf{Meran}, \emph{P.PPLA3}|pw}\pend{}\pstart{}Palast Hôtel\oindex{Palasthotel Meran@\textbf{Palasthotel Meran}, \emph{Hotel (K.HTL)}|pw}\pend{}{\bigskip}
\pstart
           \noindent{}\centering{}{\pb}\textcolor{gray}{\textbf{Semmering\oindex{Semmering@\textbf{Semmering}, \emph{A.ADM3}|pw}bahn. Poleruswand\oindex{Kleiner Krausel-Tunnel@\textbf{Kleiner Krausel-Tunnel}, \emph{Infrastrukturgebäude (K.INF)}|pw} mit Adlitzgraben\oindex{Adlitzgraben@\textbf{Adlitzgraben}, \emph{Tal (N.TAL)}|pw}.}}\pend
           \vspace{1em}
\pstart
           {\pb}5 IX.\pend
           \vspace{0.5em}
\pstart
           Schreibe bei Ihrer Mama\pwindex{Schnitzler, Louise 1840-07-08 – 1911-09-09@\textsc{Schnitzler, Louise} (1840-07-08 – 1911-09-09)|pwv} –
               d. h. dieſe Karte nicht mein Stück\pwindex{Silvia im »Stern«@\emph{Silvia im »Stern«}|pwv}. Letzteres ſchreib ich in meinem Zimmer. Es iſt ſehr ſchön. Ich finde
               es iſt wie von Neſtroy\pwindex{Nestroy, Johann Nepomuk 07.12.1801 – 25.05.1862@\textsc{Nestroy, Johann Nepomuk} (07.12.1801 – 25.05.1862), \emph{Schauspieler/Schauspielerin, Sänger/Sängerin, Dramatiker/Dramatikerin}|pw}, wenn er \strikeout{viel} Schnitzler geleſen hätte und Goldoni\pwindex{Goldoni, Carlo 25.02.1707 – 06.02.1793@\textsc{Goldoni, Carlo} (25.02.1707 – 06.02.1793), \emph{Schriftsteller/Schriftstellerin}|pw} copieren wollen hätte. Nun im Ernſt, es iſt viel beſſer
               wie Nestroy\pwindex{Nestroy, Johann Nepomuk 07.12.1801 – 25.05.1862@\textsc{Nestroy, Johann Nepomuk} (07.12.1801 – 25.05.1862), \emph{Schauspieler/Schauspielerin, Sänger/Sängerin, Dramatiker/Dramatikerin}|pw}{ }Goldoni\pwindex{Goldoni, Carlo 25.02.1707 – 06.02.1793@\textsc{Goldoni, Carlo} (25.02.1707 – 06.02.1793), \emph{Schriftsteller/Schriftstellerin}|pw} und (natürlich) Schnitzler und
               ſchlechter wie nichts.\pend
           
\pstart
           Ende Septemb. bin ich in Wien\oindex{Wien@\textbf{Wien}, \emph{A.ADM2}|pw}.\pend
           \pstart \spacefill\mbox{Hugo.}\pend{}
\pstart
           \noindent{}{[}hs. :{]} Einen ſchönen Gruß und viele Küſſe von{\\}\spacefill\mbox{Mama}\pend
           \selectlanguage{ngerman}\endnumbering\briefempfaengerindex{Schnitzler, Arthur@\textsc{Schnitzler, Arthur}!zzzSchnitzler, Louise@\emph{von Louise Schnitzler}!1907-09-051@{5. 9. 1907}|)be}\briefempfaengerindex{Schnitzler, Arthur@\textsc{Schnitzler, Arthur}!zzzHofmannsthal, Hugo von@\emph{von Hugo von Hofmannsthal}!1907-09-051@{5. 9. 1907}|)be}\mylabel{L01705h}  \normalsize

\doendnotes{C}
\bigskip
\vfill

\clearpage

\footnotesize

\lohead{\textsc{register}}

% Definiere theindex-Environment komplett neu ohne reledmac
\makeatletter
\renewenvironment{theindex}{%
  \section*{\indexname}%
  \setlength{\parindent}{0pt}%
  \setlength{\parskip}{0pt plus 0.3pt}%
  \let\item\@idxitem
}{%
  \clearpage
}
\makeatother

\IfFileExists{\jobname-pw.ind}{\input{\jobname-pw.ind}}{}

\end{document}

      