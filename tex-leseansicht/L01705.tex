%% latex-leseansicht-vorspann.tex
%% Vorspann für die Leseansicht.
%% Lädt die gemeinsame Datei latex-vorspann.tex mit nicht gesetztem Schalter.

\newif\ifkorrekturansicht
\korrekturansichtfalse

\input{../tex-inputs/latex-vorspann}


         
         \renewcommand{\erwaehntePersonen}{Personen: Carlo Goldoni, Hugo von Hofmannsthal, Johann Nepomuk Nestroy, Louise Schnitzler}
         \renewcommand{\erwaehnteOrte}{Orte: Adlitzgraben, Kleiner Krausel-Tunnel, Meran, Palasthotel Meran, Semmering, Wien}
         \renewcommand{\erwaehnteWerke}{Werke: Silvia im »Stern«}
               \section[Hugo von Hofmannsthal und Louise Schnitzler an Arthur Schnitzler, 5. 9. 1907]{ Hugo von Hofmannsthal und Louise Schnitzler an Arthur Schnitzler,
               5. 9. 1907}\nopagebreak\mylabel{v}\rehead{ }\begin{ledgroupsized}[t]{13cm}\normalsize\beginnumbering\briefempfaengerindex{Schnitzler, Arthur@\textsc{Schnitzler, Arthur}!zzzSchnitzler, Louise@\emph{von Louise Schnitzler}!1907-09-051@{5. 9. 1907}|(be}\briefempfaengerindex{Schnitzler, Arthur@\textsc{Schnitzler, Arthur}!zzzHofmannsthal, Hugo von@\emph{von Hugo von Hofmannsthal}!1907-09-051@{5. 9. 1907}|(be} \toendnotes[C]{\smallbreak\pagebreak[2]} \Standort{CUL, Schnitzler, B 43.}
\physDesc{Bildpostkarte, 446 Zeichen
\newline{}Handschrift: 1) Bleistift, deutsche Kurrent\hspace{1em}2) Bleistift, lateinische Kurrent (\noindent{}Adresse)\hspace{1em}
\newline{}Versand: Stempel: »\nobreak{}\oindex{Semmering@\textbf{Semmering}|pwk}Semmering 1, 5. IX. 07, 6\nobreak{}«.  
\newline{}Ordnung: 1) mit Bleistift von unbekannter Hand nummeriert: »\strikeout{302}«  2) mit Bleistift von unbekannter Hand nummeriert:
                                    »284«}\buchAbdrucke{\weitereDrucke{Hugo von Hofmannsthal, Arthur Schnitzler: \emph{Briefwechsel}. Hg. Therese Nickl und Heinrich Schnitzler. Frankfurt am Main: \emph{S. Fischer} 1964, S. 230.} }\toendnotes[C]{\smallbreak}\pstart{}{\pb}Herrn D\textsuperscript{r}\pend{}\pstart{}Arthur Schnitzler\pend{}\pstart{}Meran\oindex{Meran@\textbf{Meran}|pw}\pend{}\pstart{}Palast Hôtel\oindex{Palasthotel Meran@\textbf{Palasthotel Meran}|pw}\pend{}{\bigskip}\pstart
           \noindent{}\centering{}\textcolor{gray}{\textbf{{\pb}Semmering\oindex{Semmering@\textbf{Semmering}|pw}bahn. Poleruswand\oindex{Kleiner Krausel-Tunnel@\textbf{Kleiner Krausel-Tunnel}|pw} mit Adlitzgraben\oindex{Adlitzgraben@\textbf{Adlitzgraben}|pw}.}}\pend
           \pstart
           {\pb}5 IX.\pend
           \pstart
           Schreibe bei Ihrer Mama\pwindex{Schnitzler, Louise 1840-07-08 – 1911-09-09@\textsc{Schnitzler, Louise} (1840-07-08 – 1911-09-09)|pwv} –
               d. h. dieſe Karte nicht mein Stück\pwindex{Hofmannsthal, Hugo von 1874-02-01 – 1929-07-15@\textsc{Hofmannsthal, Hugo von} (1874-02-01 – 1929-07-15), \emph{Schriftsteller}!Silvia im »Stern«1909@\strich\emph{Silvia im »Stern«} {[}1909{]}|pwv}. Letzteres ſchreib ich in meinem Zimmer. Es iſt ſehr ſchön. Ich finde
               es iſt wie von Neſtroy\pwindex{Nestroy, Johann Nepomuk 07.12.1801 – 25.05.1862@\textsc{Nestroy, Johann Nepomuk} (07.12.1801 – 25.05.1862), \emph{Schauspieler, Sänger, Dramatiker}|pw}, wenn er \strikeout{viel} Schnitzler geleſen hätte und Goldoni\pwindex{Goldoni, Carlo 25.02.1707 – 06.02.1793@\textsc{Goldoni, Carlo} (25.02.1707 – 06.02.1793), \emph{Schriftsteller}|pw} copieren wollen hätte. Nun im Ernſt, es iſt viel beſſer
               wie Nestroy\pwindex{Nestroy, Johann Nepomuk 07.12.1801 – 25.05.1862@\textsc{Nestroy, Johann Nepomuk} (07.12.1801 – 25.05.1862), \emph{Schauspieler, Sänger, Dramatiker}|pw}{ }Goldoni\pwindex{Goldoni, Carlo 25.02.1707 – 06.02.1793@\textsc{Goldoni, Carlo} (25.02.1707 – 06.02.1793), \emph{Schriftsteller}|pw} und (natürlich) Schnitzler und
               ſchlechter wie nichts.\pend
           \pstart
           Ende Septemb. bin ich in Wien\oindex{Wien@\textbf{Wien}|pw}.\pend
           \pstart \spacefill\mbox{Hugo.}\pend{}\pstart
           \noindent{}{[}hs. Schnitzler:{]} Einen ſchönen Gruß und viele Küſſe von{\\}\spacefill\mbox{Mama}\pend
           
         
         \endnumbering\mylabel{h}\end{ledgroupsized}  \newcommand{\dateiname}{L01705}\newcommand{\titel}{Hugo von Hofmannsthal und Louise Schnitzler an Arthur Schnitzler, 5. 9. 1907}\newcommand{\editorInnen}{Martin Anton Müller und Gerd-Hermann Susen}%% latex-leseansicht-abspann.tex
%% Abspann für die Leseansicht.
%% Der Schalter \ifkorrekturansicht ist bereits durch den Vorspann gesetzt.

%% latex-abspann.tex
%% Gemeinsamer Abspann für Korrekturansicht und Leseansicht.
%% Setzt den Schalter \ifkorrekturansicht voraus (gesetzt in den
%% einbindenden Dateien latex-korrekturansicht-abspann.tex bzw.
%% latex-leseansicht-abspann.tex).
%% ---------------------------------------------------------------

\normalsize

% Das esempio-Environment wird nur in der Leseansicht benötigt
\ifkorrekturansicht\else
\newenvironment{esempio}[3]%
{
    \vspace{1.5ex}
    \rlap{\underline{#1}}
    \par
    \setlength{\parindent}{0cm}
    \nopagebreak
    \leftskip=#2cm
    \rightskip=#3cm
}
{
    \par
}
\fi

\doendnotes{C}
\bigskip
\vfill

\clearpage

\footnotesize

\ifkorrekturansicht
  \lohead{\textsc{register}}
\fi

% theindex-Environment neu definieren ohne reledmac
\makeatletter
\renewenvironment{theindex}{%
  \ifkorrekturansicht
    \section*{\indexname}%
  \else
    \subsubsection*{Index der erwähnten Entitäten}%
  \fi
  \setlength{\parindent}{0pt}%
  \setlength{\parskip}{0pt plus 0.3pt}%
  \let\item\@idxitem
}{%
  \ifkorrekturansicht\clearpage\fi
}
\makeatother

\IfFileExists{\jobname-pw.ind}{\input{\jobname-pw.ind}}{}

% Quellenangabe nur in der Leseansicht
\ifkorrekturansicht\else
% Fallback-Definitionen, falls die .tex-Datei \titel etc. nicht gesetzt hat
\providecommand{\titel}{}
\providecommand{\editorInnen}{}
\providecommand{\dateiname}{\jobname}

\vspace{3cm}

\vfill

\footnotesize
\textsc{Quelle}: \titel. Herausgegeben von {\editorInnen}. In: \emph{Arthur Schnitzler: Briefwechsel mit Autorinnen und Autoren}.
 Digitale Edition, https://schnitzler-briefe.acdh.oeaw.ac.at/{\dateiname}.html (Stand \today)
\fi

\end{document}


      