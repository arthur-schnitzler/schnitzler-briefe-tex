%% latex-leseansicht-vorspann.tex
%% Vorspann für die Leseansicht.
%% Lädt die gemeinsame Datei latex-vorspann.tex mit nicht gesetztem Schalter.

\newif\ifkorrekturansicht
\korrekturansichtfalse

\input{../tex-inputs/latex-vorspann}


         
         \renewcommand{\erwaehntePersonen}{Personen:  ?? [Chinesischer Fotograf], Richard Beer-Hofmann, Louis Philipp Friedmann, Rose Friedmann, Paul Goldmann, Hugo von Hofmannsthal, Marie Reinhard, Arthur Rosthorn, Leo Van-Jung, Alice Ziegler}
         \renewcommand{\erwaehnteInstitutionen}{Institutionen: Pun lun}
         \renewcommand{\erwaehnteOrte}{Orte: Bad Gastein, China, Deutsches Postamt in Shanghai, Deutschland, Europa, Graz, Guangzhou, Hong Kong, Hongkong Hotel, Italien, München, Peking, Prag, Salzburg, Schweiz, Shanghai, Tegernsee, Wien}
         \renewcommand{\erwaehnteWerke}{Werke: Paul Goldmann, Tagebuch}
               \section[ Paul Goldmann an Arthur Schnitzler, 16. 5. 1898]{ Paul Goldmann an Arthur Schnitzler, 16. 5. 1898}\nopagebreak\mylabel{v}\rehead{ }\begin{ledgroupsized}[t]{13cm}\normalsize\beginnumbering \toendnotes[C]{\smallbreak\pagebreak[2]} \Standort{DLA, A:Schnitzler, HS.NZ85.1.3168.}
\physDesc{Brief, 2 Blätter, 6 Seiten, 1644 Zeichen
\newline{}Handschrift: schwarze Tinte, deutsche Kurrent
\newline{}Beilage: Fotografie, DLA, B 1989.Q 0431 
\newline{}Schnitzler: mit rotem Buntstift eine Unterstreichung }\toendnotes[C]{\smallbreak}\begin{figure}[H]\centering\includegraphics[width=10cm]{../tex-inputs/img/img6272-46.jpg}\end{figure}\pstart
           \noindent{}\centering{}{\pb}\textcolor{gray}{\textbf{HONG KONG HOTEL\oindex{Hongkong Hotel@\textbf{Hongkong Hotel}|pw}}}\pend
           \pstart
           \noindent{}\textcolor{gray}{\textbf{\textbf{A.B.C.CODE.}}}\pend
           \pstart
           \textcolor{gray}{\textbf{\emph{Telegraphic Address,}}}\pend
           \pstart
           \textcolor{gray}{\textbf{\textbf{»KREMLIN.«}}}\pend
           \pstart
           \raggedleft{}\textcolor{gray}{\textbf{Hong Kong}}\oindex{Hong Kong@\textbf{Hong Kong}|pw},{ }16. Mai \textcolor{gray}{\textbf{189}}8.\pend
           \pstart\center{}Mein lieber Freund,\pend\pstart
           Deinen erſten Brief nach \textsc{Shanghai\oindex{Shanghai@\textbf{Shanghai}|pw}} habe ich ſchon hier\oindex{Hong Kong@\textbf{Hong Kong}|pwv}
               erhalten, und er iſt das erſte Wort, das ich hier in der Ferne von zu Hauſe u. von
               lieben Menſchen höre. Herzlichſten Dank dafür, ſowie für die beigelegte \label{K_L02845-1v}\edtext{Empfehlung}{\lemma{\textnormal{\emph{Empfehlung}}}\Cendnote{\textnormal{Siehe Paul Goldmann an Arthur Schnitzler, 10. 3. [1898] und 16. 10. [1898]. Eine
                  nachweisbare Verbindung Schnitzler\pwindex{Schnitzler, Arthur 15.05.1862 – 21.10.1931@\textsc{Schnitzler, Arthur} (15.05.1862 – 21.10.1931), \emph{Schriftsteller, Mediziner}|pwk}s nach China\oindex{China@\textbf{China}|pwk} verläuft über seinen Klassenkameraden
                     Louis Friedmann\pwindex{Friedmann, Louis Philipp 29.06.1861 – 01.04.1939@\textsc{Friedmann, Louis Philipp} (29.06.1861 – 01.04.1939), \emph{Industrieller, Bergsteiger}|pwk}, der mit Rose Rosthorn\pwindex{Friedmann, Rose 12.02.1864 – 14.01.1919@\textsc{Friedmann, Rose} (12.02.1864 – 14.01.1919)|pwk} verheiratet war. Ihr Bruder
                     Arthur Rosthorn\pwindex{Rosthorn, Arthur 1862-04-14 – 1945-12-17@\textsc{Rosthorn, Arthur} (1862-04-14 – 1945-12-17), \emph{Diplomat, Schriftsteller, Sinologe}|pwk} leitete zwischen
                     1895 und 1898 die österreichische Gesandtschaft in
                     Peking\oindex{Peking@\textbf{Peking}|pwk}.}}}\label{K_L02845-1h}!\pend
           \pstart
           Ich habe in der letzten {\pb}Zeit viel merkwürdige Dinge
               geſehen, namentlich \textsc{Canton\oindex{Guangzhou@\textbf{Guangzhou}|pw}}, das einfach aller Beſchreibung ſpottet.\pend
           \pstart
           Aber Alles in Allem wünſchte ich, ich wäre ſchon wieder zu Haufe. Das Reiſen hier iſt
               mit unſäglichen Strapazen und Entbehrungen verknüpft. Eſſen u. Wohnen ſind ſchlecht,
               die Hitze iſt \strikeout{\textcolor{gray}{e}\textcolor{gray}{×}} unmenſchlich, hält auch in der Nacht an, macht infolgedeſſen das Schlafen
               unmöglich. Die Deutſch\oindex{Deutschland@\textbf{Deutschland}|pwv}en hier
               ſind von einer {\pb}Gaſtfreundſchaft, die man zu Hauſe
               kaum ahnt; und doch ſind es nicht Leute \uline{unſerer} Art,
               und überhaupt liegt Alles, was uns betrifft u. unſer Leben ausmacht, in Europa\oindex{Europa@\textbf{Europa}|pw}. \strikeout{Ma} Man
               kann nicht Monate lang allein vom \textsc{Pittoresken} leben. Das
               iſt zu dünne Nahrung. Das Alles hier geſehen zu \uline{haben}, iſt ſchön; \strikeout{ab\textcolor{gray}{er}} aber es zu ſehen, erfordert mehr \strikeout{Selb}{ }{\pb}Selbſtüberwindung, Energie u. Entſagung, als man
               glauben möchte.\pend
           \pstart
           Ich ſende Dir anbei meine \label{K_L02845-2v}\edtext{Photographie\pwindex{?? [Chinesischer Fotograf] @\textsc{?? [Chinesischer Fotograf]}!Paul GoldmannMai 1898@\strich\emph{Paul Goldmann} {[}Mai 1898{]}|pwv}}{\lemma{\textnormal{\emph{Photographie}}}\Cendnote{\textnormal{Das Foto\pwindex{?? [Chinesischer Fotograf] @\textsc{?? [Chinesischer Fotograf]}!Paul GoldmannMai 1898@\strich\emph{Paul Goldmann} {[}Mai 1898{]}|pwkv}, hergestellt vom Fotoatelier \emph{Pun lun}\orgindex{Pun lun@Pun lun|pwk}, findet sich heute im Deutschen Literaturarchiv Marbach,
                  B 1989.Q 0431.}}}\label{K_L02845-2h} als Erforſcher \strikeout{fr\textcolor{gray}{e}} fremder Welttheile, gemacht vom \label{K_L02845-3v}\edtext{chin\oindex{China@\textbf{China}|pwv}eſiſchen Photographen\pwindex{?? [Chinesischer Fotograf] @\textsc{?? [Chinesischer Fotograf]}|pw}}{\lemma{\textnormal{\emph{chineſiſchen Photographen}}}\Cendnote{\textnormal{nicht ermittelt}}}\label{K_L02845-3h}. Ich hoffe,
               baldigſt wieder von Dir zu hören, (Adreſſe bleibt: \textsc{Shanghai}\oindex{Shanghai@\textbf{Shanghai}|pw}, \textsc{Kais. Deutsches Postamt\oindex{Deutsches Postamt in Shanghai@\textbf{Deutsches Postamt in Shanghai}|pwv}}), wünſche Dir von Herzen Glück auf die \label{K_L02845-4v}\edtext{Sommer-Reiſe}{\lemma{\textnormal{\emph{Sommer-Reiſe}}}\Cendnote{\textnormal{Am
                     11. 7. 1898
                  begann Schnitzler\pwindex{Schnitzler, Arthur 15.05.1862 – 21.10.1931@\textsc{Schnitzler, Arthur} (15.05.1862 – 21.10.1931), \emph{Schriftsteller, Mediziner}|pwk}s große
                     »Sommer-Reiſe«. Zuerst fuhr er mit Marie Reinhard\pwindex{Reinhard, Marie 1871-03-13 – 1899-03-18@\textsc{Reinhard, Marie} (1871-03-13 – 1899-03-18), \emph{Gesangspädagogin}|pwk} nach Graz\oindex{Graz@\textbf{Graz}|pwk}, machte in der Umgebung Radausflüge und kam am 20. 7. 1898 in Bad Gastein\oindex{Bad Gastein@\textbf{Bad Gastein}|pwk} an. Am 26. 7. 1898 ging es
                  für ihn weiter nach Salzburg\oindex{Salzburg@\textbf{Salzburg}|pwk} und am 31. 7. 1898 über München\oindex{Muenchen@\textbf{München}|pwk} nach Tegernsee\oindex{Tegernsee@\textbf{Tegernsee}|pwk}. Wieder über München\oindex{Muenchen@\textbf{München}|pwk} fuhr
                  er am 9. 8. 1898
                  weiter in die Schweiz\oindex{Schweiz@\textbf{Schweiz}|pwk}, wo er u. a. mit Hugo von Hofmannsthal\pwindex{Hofmannsthal, Hugo von 1874-02-01 – 1929-07-15@\textsc{Hofmannsthal, Hugo von} (1874-02-01 – 1929-07-15), \emph{Schriftsteller}|pwk} Rad fuhr. Am 28. 8. 1898 reiste Schnitzler\pwindex{Schnitzler, Arthur 15.05.1862 – 21.10.1931@\textsc{Schnitzler, Arthur} (15.05.1862 – 21.10.1931), \emph{Schriftsteller, Mediziner}|pwk} weiter nach Italien\oindex{Italien@\textbf{Italien}|pwk}, am 3. 9. 1898 kehrte er nach Wien\oindex{Wien@\textbf{Wien}|pwk} zurück.}}}\label{K_L02845-4h}, {\pb}\strikeout{g} gute Stimmung (warum ſo \label{K_L02845-5v}\edtext{düſter}{\lemma{\textnormal{\emph{düſter}}}\Cendnote{\textnormal{Verstimmungen sind dem \emph{Tagebuch}\pwindex{Schnitzler, Arthur 15.05.1862 – 21.10.1931@\textsc{Schnitzler, Arthur} (15.05.1862 – 21.10.1931), \emph{Schriftsteller, Mediziner}!Tagebuch1981 – 2000@\strich\emph{Tagebuch} {[}1981 – 2000{]}|pwk} in dieser
                  Zeit (Goldmann\pwindex{Goldmann, Paul 31.01.1865 – 25.09.1935@\textsc{Goldmann, Paul} (31.01.1865 – 25.09.1935), \emph{Schriftsteller, Journalist}|pwk} bezog sich wohl auf einen
                  Brief Schnitzler\pwindex{Schnitzler, Arthur 15.05.1862 – 21.10.1931@\textsc{Schnitzler, Arthur} (15.05.1862 – 21.10.1931), \emph{Schriftsteller, Mediziner}|pwk}s von vor einigen Wochen)
                  häufig zu entnehmen, siehe z. B. A. S.: \emph{Tagebuch}, 13. 4. 1898.}}}\label{K_L02845-5h}, liebes Kind? warum Dich ſo unnütz quälen?) und
               frohe Erlebniſſe, bitte Dich, Deine Freundin\pwindex{Reinhard, Marie 1871-03-13 – 1899-03-18@\textsc{Reinhard, Marie} (1871-03-13 – 1899-03-18), \emph{Gesangspädagogin}|pwv} recht herzlich zu grüßen, mich den Deinen zu
               empfehlen u. bin in Treue\pend
           \pstart
           Dein {\\[\baselineskip]}\spacefill\mbox{Paul Goldmann}\pend
           \leftskip=0em{}\pstart
           \noindent{}Viele Grüße an \textsc{Richard\pwindex{Beer-Hofmann, Richard 1866-07-11 – 1945-09-26@\textsc{Beer-Hofmann, Richard} (1866-07-11 – 1945-09-26), \emph{Schriftsteller}|pw}} und \textsc{Leo\pwindex{Van-Jung, Leo 15.10.1866 – 02.07.1939@\textsc{Van-Jung, Leo} (15.10.1866 – 02.07.1939), \emph{Gesangspädagoge, Mathematiker}|pw}}! \pend
           \pstart
           \textsc{\uuline{\label{K_L02845-6v}\edtext{verte}{\lemma{\textnormal{\emph{verte}}}\Cendnote{\textnormal{lateinisch: (Blatt) wenden}}}\label{K_L02845-6h}}}\pend
           \pstart
           {\pb}Hörſt Du irgend etwas von dem kleinen \label{K_L02845-7v}\edtext{Mädchen\pwindex{Ziegler, Alice 1880-01-05 – Dezember 1943@\textsc{Ziegler, Alice} (1880-01-05 – Dezember 1943)|pwv} aus \textsc{Prag\oindex{Prag@\textbf{Prag}|pw}}}{\lemma{\textnormal{\emph{Mädchen aus Prag}}}\Cendnote{\textnormal{siehe Paul Goldmann an Arthur Schnitzler, 19. 11. [1897]}}}\label{K_L02845-7h}? \strikeout{Gl\textcolor{gray}{au}} Wirſt Du ſie \label{K_L02845-8v}\edtext{dieſen Sommer
                     ſehen}{\lemma{\textnormal{\emph{dieſen Sommer
                     ſehen}}}\Cendnote{\textnormal{Dazu kam es
                  nicht.}}}\label{K_L02845-8h}?\pend
           
         
         \endnumbering\mylabel{h}\end{ledgroupsized}  \newcommand{\dateiname}{L02845}\newcommand{\titel}{Paul Goldmann an Arthur Schnitzler, 16. 5. 1898}\newcommand{\editorInnen}{Martin Anton Müller und Laura Untner}%% latex-leseansicht-abspann.tex
%% Abspann für die Leseansicht.
%% Der Schalter \ifkorrekturansicht ist bereits durch den Vorspann gesetzt.

%% latex-abspann.tex
%% Gemeinsamer Abspann für Korrekturansicht und Leseansicht.
%% Setzt den Schalter \ifkorrekturansicht voraus (gesetzt in den
%% einbindenden Dateien latex-korrekturansicht-abspann.tex bzw.
%% latex-leseansicht-abspann.tex).
%% ---------------------------------------------------------------

\normalsize

% Das esempio-Environment wird nur in der Leseansicht benötigt
\ifkorrekturansicht\else
\newenvironment{esempio}[3]%
{
    \vspace{1.5ex}
    \rlap{\underline{#1}}
    \par
    \setlength{\parindent}{0cm}
    \nopagebreak
    \leftskip=#2cm
    \rightskip=#3cm
}
{
    \par
}
\fi

\doendnotes{C}
\bigskip
\vfill

\clearpage

\footnotesize

\ifkorrekturansicht
  \lohead{\textsc{register}}
\fi

% theindex-Environment neu definieren ohne reledmac
\makeatletter
\renewenvironment{theindex}{%
  \ifkorrekturansicht
    \section*{\indexname}%
  \else
    \subsubsection*{Index der erwähnten Entitäten}%
  \fi
  \setlength{\parindent}{0pt}%
  \setlength{\parskip}{0pt plus 0.3pt}%
  \let\item\@idxitem
}{%
  \ifkorrekturansicht\clearpage\fi
}
\makeatother

\IfFileExists{\jobname-pw.ind}{\input{\jobname-pw.ind}}{}

% Quellenangabe nur in der Leseansicht
\ifkorrekturansicht\else
% Fallback-Definitionen, falls die .tex-Datei \titel etc. nicht gesetzt hat
\providecommand{\titel}{}
\providecommand{\editorInnen}{}
\providecommand{\dateiname}{\jobname}

\vspace{3cm}

\vfill

\footnotesize
\textsc{Quelle}: \titel. Herausgegeben von {\editorInnen}. In: \emph{Arthur Schnitzler: Briefwechsel mit Autorinnen und Autoren}.
 Digitale Edition, https://schnitzler-briefe.acdh.oeaw.ac.at/{\dateiname}.html (Stand \today)
\fi

\end{document}


      