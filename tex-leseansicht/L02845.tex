%% latex-leseansicht-vorspann.tex
%% Vorspann für die Leseansicht.
%% Lädt die gemeinsame Datei latex-vorspann.tex mit nicht gesetztem Schalter.

\newif\ifkorrekturansicht
\korrekturansichtfalse

\input{../tex-inputs/latex-vorspann}


\section[ Paul Goldmann an Arthur Schnitzler, 16. 5. 1898]{L02845 Paul Goldmann an Arthur Schnitzler,  16. 5. 1898}
\nopagebreak\mylabel{L02845v}
\rehead{ }\normalsize\beginnumbering\briefempfaengerindex{Schnitzler, Arthur@\textsc{Schnitzler, Arthur}!zzzGoldmann, Paul@\emph{von Paul Goldmann}!1898-05-162@{16. 5. 1898}|(be}
\toendnotes[C]{\smallbreak\pagebreak[2]}
\correspDesc{Versand  durch Paul Goldmann am 16. 5. 1898 in Hong Kong
\newline{}Erhalt  durch Arthur Schnitzler im Zeitraum [1. 6. 1898
                  – 30. 6. 1898?] in Wien}\toendnotes[C]{\smallbreak}
\Standort{DLA, A:Schnitzler, HS.NZ85.1.3168.}
\physDesc{Brief, 2 Blätter, 6 Seiten, 1644 Zeichen
\newline{}Handschrift: schwarze Tinte, deutsche Kurrent
\newline{}Beilage: Fotografie, DLA, B 1989.Q 0431 
\newline{}Schnitzler: mit rotem Buntstift eine Unterstreichung }\toendnotes[C]{\smallbreak}\begin{figure}[H]\centering\includegraphics[width=10cm]{../tex-inputs/img/img6272-46.jpg}\end{figure}\vspace{1em}
\pstart
           \centering{}{\pb}\textcolor{gray}{\textbf{HONG KONG HOTEL\oindex{Hongkong Hotel@\textbf{Hongkong Hotel}, \emph{Hotel}|pw}}}\pend
           
\pstart
           \textcolor{gray}{\textbf{\textbf{A.B.C.CODE.}}}\pend
           
\pstart
           \textcolor{gray}{\textbf{\emph{Telegraphic Address,}}}\pend
           
\pstart
           \textcolor{gray}{\textbf{\textbf{»KREMLIN.«}}}\pend
           
\pstart
           \raggedleft{}\textcolor{gray}{\textbf{Hong Kong}}\oindex{Hong Kong@\textbf{Hong Kong}, \emph{Hauptstadt}|pw},{ }16. Mai \textcolor{gray}{\textbf{189}}8.\pend
           
\pstart\center{}Mein lieber Freund,\pend\vspace{0.5em}
\pstart
           Deinen erſten Brief nach \textsc{Shanghai\oindex{Shanghai@\textbf{Shanghai}|pw}} habe ich{ }ſchon hier\oindex{Hong Kong@\textbf{Hong Kong}, \emph{Hauptstadt}|pwv}
               erhalten, und er iſt das erſte Wort, das ich hier in der Ferne von zu Hauſe u. von
               lieben Menſchen höre. Herzlichſten Dank dafür,{ }ſowie für die beigelegte \label{K_L02845-1v}\edtext{Empfehlung}{\lemma{\textnormal{\emph{Empfehlung}}}\Cendnote{\textnormal{Siehe XXXX Auszeichnungsfehler: Dokument L02842 nicht gefunden und XXXX Auszeichnungsfehler: Dokument L02861 nicht gefunden. Eine
                  nachweisbare Verbindung Schnitzlers nach China\oindex{China@\textbf{China}|pwk} verläuft über seinen Klassenkameraden
                     Louis Friedmann\pwindex{Friedmann, Louis Philipp 29.\,6.\,1861 Paris – 1.\,4.\,1939 Wien@\textsc{Friedmann, Louis Philipp} (29.\,6.\,1861 Paris – 1.\,4.\,1939 Wien), \emph{Industrieller, Bergsteiger}|pwk}, der mit Rose Rosthorn\pwindex{Friedmann, Rose 12.\,2.\,1864 – 14.\,1.\,1919 Baden bei Wien@\textsc{Friedmann, Rose} (12.\,2.\,1864 – 14.\,1.\,1919 Baden bei Wien)|pwk} verheiratet war. Ihr Bruder
                     Arthur Rosthorn\pwindex{Rosthorn, Arthur 14.\,4.\,1862 Wien – 17.\,12.\,1945 Oed@\textsc{Rosthorn, Arthur} (14.\,4.\,1862 Wien – 17.\,12.\,1945 Oed), \emph{Schriftsteller, Diplomat, Sinologe}|pwk} leitete zwischen
                     1895 und 1898 die österreichische Gesandtschaft in
                     Peking\oindex{Peking@\textbf{Peking}, \emph{Hauptstadt}|pwk}.}}}\label{K_L02845-1}!\pend
           
\pstart
           Ich habe in der letzten {\pb}Zeit viel merkwürdige Dinge
               geſehen, namentlich \textsc{Canton\oindex{Guangzhou@\textbf{Guangzhou}|pw}}, das einfach aller Beſchreibung{ }ſpottet.\pend
           
\pstart
           Aber Alles in Allem wünſchte ich, ich wäre{ }ſchon wieder zu Haufe. Das Reiſen hier iſt
               mit unſäglichen Strapazen und Entbehrungen verknüpft. Eſſen u. Wohnen{ }ſind{ }ſchlecht,
               die Hitze iſt \strikeout{\textcolor{gray}{e}\textcolor{gray}{×}} unmenſchlich, hält auch in der Nacht an, macht infolgedeſſen das Schlafen
               unmöglich. Die Deutſch\oindex{Deutschland@\textbf{Deutschland}|pwv}en hier{ }ſind von einer {\pb}Gaſtfreundſchaft, die man zu Hauſe
               kaum ahnt; und doch{ }ſind es nicht Leute \uline{unſerer} Art,
               und überhaupt liegt Alles, was uns betrifft u. unſer Leben ausmacht, in Europa\oindex{Europa@\textbf{Europa}|pw}. \strikeout{Ma} Man
               kann nicht Monate lang allein vom \textsc{Pittoresken} leben. Das
               iſt zu dünne Nahrung. Das Alles hier geſehen zu \uline{haben}, iſt{ }ſchön; \strikeout{ab\textcolor{gray}{er}} aber es zu{ }ſehen, erfordert mehr \strikeout{Selb}{ }{\pb}Selbſtüberwindung, Energie u. Entſagung, als man
               glauben möchte.\pend
           
\pstart
           Ich{ }ſende Dir anbei meine \label{K_L02845-2v}\edtext{Photographie\pwindex{?? [Chinesischer Fotograf] @\textsc{?? [Chinesischer Fotograf]}!Paul Goldmann@\strich\emph{Paul Goldmann}|pwv}}{\lemma{\textnormal{\emph{Photographie}}}\Cendnote{\textnormal{Das Foto\pwindex{?? [Chinesischer Fotograf] @\textsc{?? [Chinesischer Fotograf]}!Paul Goldmann@\strich\emph{Paul Goldmann}|pwkv}, hergestellt vom Fotoatelier \emph{Pun lun}\orgindex{Pun lun@Pun lun|pwk}, findet sich heute im Deutschen Literaturarchiv Marbach,
                  B 1989.Q 0431.}}}\label{K_L02845-2} als Erforſcher \strikeout{fr\textcolor{gray}{e}} fremder Welttheile, gemacht vom \label{K_L02845-3v}\edtext{chin\oindex{China@\textbf{China}|pwv}eſiſchen Photographen\pwindex{?? [Chinesischer Fotograf] @\textsc{?? [Chinesischer Fotograf]}|pw}}{\lemma{\textnormal{\emph{chinesischen Photographen}}}\Cendnote{\textnormal{nicht ermittelt}}}\label{K_L02845-3}. Ich hoffe,
               baldigſt wieder von Dir zu hören, (Adreſſe bleibt: \textsc{Shanghai}\oindex{Shanghai@\textbf{Shanghai}|pw}, \textsc{Kais. Deutsches Postamt\oindex{Deutsches Postamt in Shanghai@\textbf{Deutsches Postamt in Shanghai}, \emph{Bürogebäude}|pwv}}), wünſche Dir von Herzen Glück auf die \label{K_L02845-4v}\edtext{Sommer-Reiſe}{\lemma{\textnormal{\emph{Sommer-Reise}}}\Cendnote{\textnormal{Am
                     11. 7. 1898
                  begann Schnitzlers große
                     »Sommer-Reiſe«. Zuerst fuhr er mit Marie Reinhard\pwindex{Reinhard, Marie 13.\,3.\,1871 Wien – 18.\,3.\,1899 ebd.@\textsc{Reinhard, Marie} (13.\,3.\,1871 Wien – 18.\,3.\,1899 ebd.), \emph{Gesangspädagogin}|pwk} nach Graz\oindex{Graz@\textbf{Graz}, \emph{Verwaltungsgebiet}|pwk}, machte in der Umgebung Radausflüge und kam am 20. 7. 1898 in Bad Gastein\oindex{Bad Gastein@\textbf{Bad Gastein}, \emph{Hauptstadt}|pwk} an. Am 26. 7. 1898 ging es
                  für ihn weiter nach Salzburg\oindex{Salzburg@\textbf{Salzburg}, \emph{Verwaltungsgebiet}|pwk} und am 31. 7. 1898 über München\oindex{München@\textbf{München}|pwk} nach Tegernsee\oindex{Tegernsee@\textbf{Tegernsee}|pwk}. Wieder über München\oindex{München@\textbf{München}|pwk} fuhr
                  er am 9. 8. 1898
                  weiter in die Schweiz\oindex{Schweiz@\textbf{Schweiz}|pwk}, wo er u. a. mit Hugo von Hofmannsthal\pwindex{Hofmannsthal, Hugo von 1.\,2.\,1874 Wien – 15.\,7.\,1929 Rodaun@\textsc{Hofmannsthal, Hugo von} (1.\,2.\,1874 Wien – 15.\,7.\,1929 Rodaun), \emph{Schriftsteller}|pwk} Rad fuhr. Am 28. 8. 1898 reiste Schnitzler weiter nach Italien\oindex{Italien@\textbf{Italien}|pwk}, am 3. 9. 1898 kehrte er nach Wien\oindex{Wien@\textbf{Wien}, \emph{Verwaltungsgebiet}|pwk} zurück.}}}\label{K_L02845-4}, {\pb}\strikeout{g} gute Stimmung (warum{ }ſo \label{K_L02845-5v}\edtext{düſter}{\lemma{\textnormal{\emph{düster}}}\Cendnote{\textnormal{Verstimmungen sind dem \emph{Tagebuch}\pwindex{Schnitzler, Arthur 15.\,5.\,1862 Wien – 21.\,10.\,1931 ebd.@\textsc{Schnitzler, Arthur} (15.\,5.\,1862 Wien – 21.\,10.\,1931 ebd.), \emph{Schriftsteller, Mediziner}!Tagebuch@\strich\emph{Tagebuch}|pwk} in dieser
                  Zeit (Goldmann\pwindex{Goldmann, Paul 31.\,1.\,1865 Breslau – 25.\,9.\,1935 Wien@\textsc{Goldmann, Paul} (31.\,1.\,1865 Breslau – 25.\,9.\,1935 Wien), \emph{Schriftsteller, Journalist}|pwk} bezog sich wohl auf einen
                  Brief Schnitzlers von vor einigen Wochen)
                  häufig zu entnehmen, siehe z. B. A. S.: \emph{Tagebuch}, 13. 4. 1898.}}}\label{K_L02845-5}, liebes Kind? warum Dich{ }ſo unnütz quälen?) und
               frohe Erlebniſſe, bitte Dich, Deine Freundin\pwindex{Reinhard, Marie 13.\,3.\,1871 Wien – 18.\,3.\,1899 ebd.@\textsc{Reinhard, Marie} (13.\,3.\,1871 Wien – 18.\,3.\,1899 ebd.), \emph{Gesangspädagogin}|pwv} recht herzlich zu grüßen, mich den Deinen zu
               empfehlen u. bin in Treue\pend
           
\pstart
           Dein {\\[\baselineskip]}\spacefill\mbox{Paul Goldmann}\pend
           \leftskip=0em{}
\pstart
           \noindent{}Viele Grüße an \textsc{Richard\pwindex{Beer-Hofmann, Richard 11.\,7.\,1866 Wien – 26.\,9.\,1945 New York City@\textsc{Beer-Hofmann, Richard} (11.\,7.\,1866 Wien – 26.\,9.\,1945 New York City), \emph{Schriftsteller}|pw}} und \textsc{Leo\pwindex{Van-Jung, Leo 15.\,10.\,1866 Odessa – 2.\,7.\,1939 Riga@\textsc{Van-Jung, Leo} (15.\,10.\,1866 Odessa – 2.\,7.\,1939 Riga), \emph{Gesangspädagoge, Mathematiker}|pw}}!\pend
           
\pstart
           \textsc{\uuline{\label{K_L02845-6v}\edtext{verte}{\lemma{\textnormal{\emph{verte}}}\Cendnote{\textnormal{lateinisch: (Blatt) wenden}}}\label{K_L02845-6}}}\pend
           
\pstart
           {\pb}Hörſt Du irgend etwas von dem kleinen \label{K_L02845-7v}\edtext{Mädchen\pwindex{Ziegler, Alice 5.\,1.\,1880 Prag – Dezember 1943 Konzentrationslager Auschwitz-Birkenau@\textsc{Ziegler, Alice} (5.\,1.\,1880 Prag – Dezember 1943 Konzentrationslager Auschwitz-Birkenau)|pwv} aus \textsc{Prag\oindex{Prag@\textbf{Prag}, \emph{Land}|pw}}}{\lemma{\textnormal{\emph{Mädchen aus Prag}}}\Cendnote{\textnormal{Siehe XXXX Auszeichnungsfehler: Dokument L02831 nicht gefunden.
                  }}}\label{K_L02845-7}? \strikeout{Gl\textcolor{gray}{au}} Wirſt Du{ }ſie \label{K_L02845-8v}\edtext{dieſen Sommer{ }ſehen}{\lemma{\textnormal{\emph{diesen Sommer sehen}}}\Cendnote{\textnormal{Dazu kam es
                  nicht.}}}\label{K_L02845-8}?\pend
           \selectlanguage{ngerman}\endnumbering\briefempfaengerindex{Schnitzler, Arthur@\textsc{Schnitzler, Arthur}!zzzGoldmann, Paul@\emph{von Paul Goldmann}!1898-05-162@{16. 5. 1898}|)be}\mylabel{L02845h}  \newcommand{\dateiname}{L02845}\newcommand{\titel}{Paul Goldmann an Arthur Schnitzler, 16. 5. 1898}\newcommand{\editorInnen}{Martin Anton Müller und Laura Untner}%% latex-leseansicht-abspann.tex
%% Abspann für die Leseansicht.
%% Der Schalter \ifkorrekturansicht ist bereits durch den Vorspann gesetzt.

%% latex-abspann.tex
%% Gemeinsamer Abspann für Korrekturansicht und Leseansicht.
%% Setzt den Schalter \ifkorrekturansicht voraus (gesetzt in den
%% einbindenden Dateien latex-korrekturansicht-abspann.tex bzw.
%% latex-leseansicht-abspann.tex).
%% ---------------------------------------------------------------

\normalsize

% Das esempio-Environment wird nur in der Leseansicht benötigt
\ifkorrekturansicht\else
\newenvironment{esempio}[3]%
{
    \vspace{1.5ex}
    \rlap{\underline{#1}}
    \par
    \setlength{\parindent}{0cm}
    \nopagebreak
    \leftskip=#2cm
    \rightskip=#3cm
}
{
    \par
}
\fi

\doendnotes{C}
\bigskip
\vfill

\clearpage

\footnotesize

\ifkorrekturansicht
  \lohead{\textsc{register}}
\fi

% theindex-Environment neu definieren ohne reledmac
\makeatletter
\renewenvironment{theindex}{%
  \ifkorrekturansicht
    \section*{\indexname}%
  \else
    \subsubsection*{Index der erwähnten Entitäten}%
  \fi
  \setlength{\parindent}{0pt}%
  \setlength{\parskip}{0pt plus 0.3pt}%
  \let\item\@idxitem
}{%
  \ifkorrekturansicht\clearpage\fi
}
\makeatother

\IfFileExists{\jobname-pw.ind}{\input{\jobname-pw.ind}}{}

% Quellenangabe nur in der Leseansicht
\ifkorrekturansicht\else
% Fallback-Definitionen, falls die .tex-Datei \titel etc. nicht gesetzt hat
\providecommand{\titel}{}
\providecommand{\editorInnen}{}
\providecommand{\dateiname}{\jobname}

\vspace{3cm}

\vfill

\footnotesize
\textsc{Quelle}: \titel. Herausgegeben von {\editorInnen}. In: \emph{Arthur Schnitzler: Briefwechsel mit Autorinnen und Autoren}.
 Digitale Edition, https://schnitzler-briefe.acdh.oeaw.ac.at/{\dateiname}.html (Stand \today)
\fi

\end{document}


