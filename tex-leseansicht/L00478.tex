%% latex-korrekturansicht-vorspann.tex
%% Vorspann für die Korrekturansicht.
%% Lädt die gemeinsame Datei latex-vorspann.tex mit gesetztem Schalter.

\newif\ifkorrekturansicht
\korrekturansichttrue

\input{../tex-inputs/latex-vorspann}


\section[Arthur Schnitzler an Richard Beer-Hofmann, 27. 8. 1895]{L00478 Arthur Schnitzler an Richard Beer-Hofmann, 27. 8. 1895}
\nopagebreak\mylabel{L00478v}
\rehead{ }\normalsize\beginnumbering\briefempfaengerindex{Beer-Hofmann, Richard@\textsc{Beer-Hofmann, Richard}!zzzSchnitzler, Arthur@\emph{von Arthur Schnitzler}!1895-08-271@{27. 8. 1895}|(be}
\toendnotes[C]{\smallbreak\pagebreak[2]}\Standort{YCGL, MSS 31.}
\physDesc{Telegramm, 111 Zeichen
\newline{}Handschrift einer Schreibkraft: blaue Tinte, deutsche Kurrent
\newline{}Versand: »\noindent{}\textcolor{gray}{\textbf{Aufgegeben am {\dots} 18{\dots} um}}{ }4 \textcolor{gray}{\textbf{Uhr}}{ }45 \textcolor{gray}{\textbf{Min.}} N\textcolor{gray}{\textbf{Mittag}}{ / }\textcolor{gray}{\textbf{Eingelangt von}} S \textcolor{gray}{\textbf{auf Leitung Nr.}} 1050 \textcolor{gray}{\textbf{am}}{ }27/8\textcolor{gray}{\textbf{189}}5{ }\textcolor{gray}{\textbf{um}}{ }5 \textcolor{gray}{\textbf{Uhr}} 50 \textcolor{gray}{\textbf{Min. {\dots} Mittag}}{ / }\textcolor{gray}{\textbf{Aufgenommen durch}}{ }\textcolor{gray}{JF.}{ / }\textcolor{gray}{\textbf{Von}}{ }München\oindex{Muenchen@\textbf{München}, \emph{P.PPLA}|pw}{ }\textcolor{gray}{\textbf{mit}} 7.232p{ }\textcolor{gray}{\textbf{Taxworten (}}17 \textcolor{gray}{\textbf{Worten {\dots} Chiffern)}}« }\toendnotes[C]{\smallbreak}\pstart{}{\pb}Richard Beer\pend{}\pstart{}Hoffmann\pend{}\pstart{}Egelmos 22\oindex{Eglmoosgasse@\textbf{Eglmoosgasse}, \emph{Bezirk (A.BZK)}|pw}\pend{}\pstart{}\textcolor{gray}{\textbf{\textit{Ischl}}}\oindex{Bad Ischl@\textbf{Bad Ischl}, \emph{P.PPL}|pw}\pend{}{\bigskip}\vspace{1em}
\pstart
           \noindent{}{\pb}Wohne ſchön Hotel
                  Continental\oindex{Hotel Continental [Muenchen]@\textbf{Hotel Continental [München]}, \emph{Hotel (K.HTL)}|pw}{ }\label{K_L00478-1v}\edtext{ſitze beſorgt}{\lemma{\textnormal{\emph{ſitze beſorgt}}}\Cendnote{\textnormal{Möglicherweise ist dieses Telegramm der
                  Ursprung eines beliebten Witzes, den Zeitungen mehrfach abdruckten und der zumeist
                     Hofmannsthal\pwindex{Hofmannsthal, Gertrude von 16.03.1880 – 09.11.1959@\textsc{Hofmannsthal, Gertrude von} (16.03.1880 – 09.11.1959)|pwk} und Schnitzler als Protagonisten hat: »In Wien\oindex{Wien@\textbf{Wien}, \emph{A.ADM2}|pw}er Literatenkreisen wird über folgende
                     angeblich wahre Geschichte herzlich gelacht: Artur Schnitzler ersuchte in Aussee\oindex{Bad Aussee@\textbf{Bad Aussee}, \emph{P.PPLA3}|pw}{ }seinen Freund Hugo Hoffmannsthal\pwindex{Hofmannsthal, Gertrude von 16.03.1880 – 09.11.1959@\textsc{Hofmannsthal, Gertrude von} (16.03.1880 – 09.11.1959)|pw}, er möge ihm, wenn er nach Salzburg\oindex{Salzburg@\textbf{Salzburg}, \emph{A.ADM2}|pw} fahre, Karten für die Jedermann\pwindex{Jedermann. Das Spiel vom Sterben des reichen Mannes@\emph{Jedermann. Das Spiel vom Sterben des reichen Mannes}|pw}-Aufführung besorgen. Nach einigen
                     Wochen, als Schnitzler längst diese Bitte
                     vergessen hatte, erhielt er aus Salzburg\oindex{Salzburg@\textbf{Salzburg}, \emph{A.ADM2}|pw}
                     folgendes Telegramm: \so{Sitze besorgt{ }}\so{Hotel Europe}\oindex{Grand Hotel de L Europe, G. Jung@\textbf{Grand Hotel de L’Europe, G. Jung}, \emph{Hotel (K.HTL)}|pw}. Hoffmannsthal\pwindex{Hofmannsthal, Gertrude von 16.03.1880 – 09.11.1959@\textsc{Hofmannsthal, Gertrude von} (16.03.1880 – 09.11.1959)|pw}. Worauf Schnitzler bestürzt zurückdrahtete: \so{Warum sitzt du besorgt im{ }}\so{Hotel Europe}\oindex{Grand Hotel de L Europe, G. Jung@\textbf{Grand Hotel de L’Europe, G. Jung}, \emph{Hotel (K.HTL)}|pw}\so{?{ }}\so{Schnitzler}\so{.}« (\emph{Der Morgen}\pwindex{Morgen. Wiener Montagsblatt@\emph{Der Morgen. Wiener Montagsblatt}|pwk}, Jg. 12, Nr. 42,
                        17. 10. 1921, S. 8.) Vgl. Arthur Schnitzler an Richard Beer-Hofmann, 5. 8. 1912, 28. 7. 1922.
               }}}\label{K_L00478-1}{ }Paul\pwindex{Goldmann, Paul 31.01.1865 – 25.09.1935@\textsc{Goldmann, Paul} (31.01.1865 – 25.09.1935), \emph{Schriftsteller/Schriftstellerin, Journalist/Journalistin}|pw} kommt morgen herzlichſt\pend
           \pstart \spacefill\mbox{Arthur}\pend{}\selectlanguage{ngerman}\endnumbering\briefempfaengerindex{Beer-Hofmann, Richard@\textsc{Beer-Hofmann, Richard}!zzzSchnitzler, Arthur@\emph{von Arthur Schnitzler}!1895-08-271@{27. 8. 1895}|)be}\mylabel{L00478h}  \normalsize

\doendnotes{C}
\bigskip
\vfill

\clearpage

\footnotesize

\lohead{\textsc{register}}

% Definiere theindex-Environment komplett neu ohne reledmac
\makeatletter
\renewenvironment{theindex}{%
  \section*{\indexname}%
  \setlength{\parindent}{0pt}%
  \setlength{\parskip}{0pt plus 0.3pt}%
  \let\item\@idxitem
}{%
  \clearpage
}
\makeatother

\IfFileExists{\jobname-pw.ind}{\input{\jobname-pw.ind}}{}

\end{document}

      