%% latex-leseansicht-vorspann.tex
%% Vorspann für die Leseansicht.
%% Lädt die gemeinsame Datei latex-vorspann.tex mit nicht gesetztem Schalter.

\newif\ifkorrekturansicht
\korrekturansichtfalse

\input{../tex-inputs/latex-vorspann}


         
         \renewcommand{\erwaehntePersonen}{Personen: Richard Beer-Hofmann, Paul Goldmann, Gertrude von Hofmannsthal}
         \renewcommand{\erwaehnteOrte}{Orte: Bad Aussee, Bad Ischl, Eglmoosgasse, Grand Hotel de L’Europe, G. Jung, Hotel Continental (München), München, Salzburg, Wien}
         \renewcommand{\erwaehnteWerke}{Werke: Der Morgen. Wiener Montagsblatt, Jedermann. Das Spiel vom Sterben des reichen Mannes}
               \section[Arthur Schnitzler an Richard Beer-Hofmann, 27. 8. 1895]{ Arthur Schnitzler an Richard Beer-Hofmann, 27. 8. 1895}\nopagebreak\mylabel{v}\rehead{ }\begin{ledgroupsized}[t]{13cm}\normalsize\beginnumbering \toendnotes[C]{\smallbreak\pagebreak[2]} \Standort{YCGL, MSS 31.}
\physDesc{Telegramm, 111 Zeichen
\newline{}Handschrift einer Schreibkraft: blaue Tinte, deutsche Kurrent
\newline{}Versand: »\noindent{}\textcolor{gray}{\textbf{Aufgegeben am {\dots} 18{\dots} um}}{ }4 \textcolor{gray}{\textbf{Uhr}}{ }45 \textcolor{gray}{\textbf{Min.}} N\textcolor{gray}{\textbf{Mittag}}{ / }\textcolor{gray}{\textbf{Eingelangt von}} S \textcolor{gray}{\textbf{auf Leitung Nr.}} 1050 \textcolor{gray}{\textbf{am}}{ }27/8\textcolor{gray}{\textbf{189}}5{ }\textcolor{gray}{\textbf{um}}{ }5 \textcolor{gray}{\textbf{Uhr}} 50 \textcolor{gray}{\textbf{Min. {\dots} Mittag}}{ / }\textcolor{gray}{\textbf{Aufgenommen durch}}{ }\textcolor{gray}{JF.}{ / }\textcolor{gray}{\textbf{Von}}{ }München\oindex{Muenchen@\textbf{München}|pw}{ }\textcolor{gray}{\textbf{mit}} 7.232p{ }\textcolor{gray}{\textbf{Taxworten (}}17 \textcolor{gray}{\textbf{Worten {\dots} Chiffern)}}« }\toendnotes[C]{\smallbreak}\pstart{}{\pb}Richard Beer\pend{}\pstart{}Hoffmann\pend{}\pstart{}Egelmos 22\oindex{Eglmoosgasse@\textbf{Eglmoosgasse}|pw}\pend{}\pstart{}\textcolor{gray}{\textbf{\textit{Ischl}}}\oindex{Bad Ischl@\textbf{Bad Ischl}|pw}\pend{}{\bigskip}\pstart
           \noindent{}{\pb}Wohne ſchön Hotel
                  Continental\oindex{Hotel Continental (Muenchen)@\textbf{Hotel Continental (München)}|pw}{ }\label{K_L00478_1v}\edtext{ſitze beſorgt}{\lemma{\textnormal{\emph{ſitze beſorgt}}}\Cendnote{\textnormal{Möglicherweise ist dieses Telegramm der
                  Ursprung eines beliebten Witzes, den Zeitungen mehrfach abdrucken und der zumeist
                     Hofmannsthal\pwindex{Hofmannsthal, Gertrude von 16.03.1880 – 09.11.1959@\textsc{Hofmannsthal, Gertrude von} (16.03.1880 – 09.11.1959)|pwk} und Schnitzler\pwindex{Schnitzler, Arthur 15.05.1862 – 21.10.1931@\textsc{Schnitzler, Arthur} (15.05.1862 – 21.10.1931), \emph{Schriftsteller, Mediziner}|pwk} als Protagonisten hat: »In Wien\oindex{Wien@\textbf{Wien}|pw}er Literatenkreisen wird über folgende
                     angeblich wahre Geschichte herzlich gelacht: Artur Schnitzler\pwindex{Schnitzler, Arthur 15.05.1862 – 21.10.1931@\textsc{Schnitzler, Arthur} (15.05.1862 – 21.10.1931), \emph{Schriftsteller, Mediziner}|pw} ersuchte in Aussee\oindex{Bad Aussee@\textbf{Bad Aussee}|pw}{ }seinen Freund Hugo Hoffmannsthal\pwindex{Hofmannsthal, Gertrude von 16.03.1880 – 09.11.1959@\textsc{Hofmannsthal, Gertrude von} (16.03.1880 – 09.11.1959)|pw}, er möge ihm, wenn er nach Salzburg\oindex{Salzburg@\textbf{Salzburg}|pw} fahre, Karten für die Jedermann\pwindex{\textcolor{red}{\textsuperscript{XXXX1 indx}}!Jedermann. Das Spiel vom Sterben des reichen Mannes1911@\strich\emph{Jedermann. Das Spiel vom Sterben des reichen Mannes} {[}1911{]}|pw}-Aufführung besorgen. Nach einigen
                     Wochen, als Schnitzler\pwindex{Schnitzler, Arthur 15.05.1862 – 21.10.1931@\textsc{Schnitzler, Arthur} (15.05.1862 – 21.10.1931), \emph{Schriftsteller, Mediziner}|pw} längst diese Bitte
                     vergessen hatte, erhielt er aus Salzburg\oindex{Salzburg@\textbf{Salzburg}|pw}
                     folgendes Telegramm: \so{Sitze besorgt }\so{Hotel Europe}\oindex{Grand Hotel de L Europe, G. Jung@\textbf{Grand Hotel de L’Europe, G. Jung}|pw}. Hoffmannsthal\pwindex{Hofmannsthal, Gertrude von 16.03.1880 – 09.11.1959@\textsc{Hofmannsthal, Gertrude von} (16.03.1880 – 09.11.1959)|pw}. Worauf Schnitzler\pwindex{Schnitzler, Arthur 15.05.1862 – 21.10.1931@\textsc{Schnitzler, Arthur} (15.05.1862 – 21.10.1931), \emph{Schriftsteller, Mediziner}|pw} bestürzt zurückdrahtete: \so{Warum sitzt du besorgt im }\so{Hotel Europe}\oindex{Grand Hotel de L Europe, G. Jung@\textbf{Grand Hotel de L’Europe, G. Jung}|pw}\so{? }\so{Schnitzler}\pwindex{Schnitzler, Arthur 15.05.1862 – 21.10.1931@\textsc{Schnitzler, Arthur} (15.05.1862 – 21.10.1931), \emph{Schriftsteller, Mediziner}|pw}\so{.}« (\emph{Der Morgen}\pwindex{?? Werk@Nicht ermittelte Verfasserinnen und Verfasser!Morgen. Wiener Montagsblatt1910 – 1938@\emph{Der Morgen. Wiener Montagsblatt} {[}1910 – 1938{]}|pwk}, Jg. 12, Nr. 42,
                        17. 10. 1921, S. 8.) Vgl. Arthur Schnitzler an Richard Beer-Hofmann, 5. 8. 1912, 28. 7. 1922}}}\label{K_L00478_1h}{ }Paul\pwindex{Goldmann, Paul 31.01.1865 – 25.09.1935@\textsc{Goldmann, Paul} (31.01.1865 – 25.09.1935), \emph{Schriftsteller, Journalist}|pw} kommt morgen herzlichſt\pend
           \pstart \spacefill\mbox{Arthur}\pend{}
         
         \endnumbering\mylabel{h}\end{ledgroupsized}  \newcommand{\dateiname}{L00478}\newcommand{\titel}{Arthur Schnitzler an Richard Beer-Hofmann, 27. 8. 1895}\newcommand{\editorInnen}{Martin Anton Müller und Gerd-Hermann Susen}%% latex-leseansicht-abspann.tex
%% Abspann für die Leseansicht.
%% Der Schalter \ifkorrekturansicht ist bereits durch den Vorspann gesetzt.

%% latex-abspann.tex
%% Gemeinsamer Abspann für Korrekturansicht und Leseansicht.
%% Setzt den Schalter \ifkorrekturansicht voraus (gesetzt in den
%% einbindenden Dateien latex-korrekturansicht-abspann.tex bzw.
%% latex-leseansicht-abspann.tex).
%% ---------------------------------------------------------------

\normalsize

% Das esempio-Environment wird nur in der Leseansicht benötigt
\ifkorrekturansicht\else
\newenvironment{esempio}[3]%
{
    \vspace{1.5ex}
    \rlap{\underline{#1}}
    \par
    \setlength{\parindent}{0cm}
    \nopagebreak
    \leftskip=#2cm
    \rightskip=#3cm
}
{
    \par
}
\fi

\doendnotes{C}
\bigskip
\vfill

\clearpage

\footnotesize

\ifkorrekturansicht
  \lohead{\textsc{register}}
\fi

% theindex-Environment neu definieren ohne reledmac
\makeatletter
\renewenvironment{theindex}{%
  \ifkorrekturansicht
    \section*{\indexname}%
  \else
    \subsubsection*{Index der erwähnten Entitäten}%
  \fi
  \setlength{\parindent}{0pt}%
  \setlength{\parskip}{0pt plus 0.3pt}%
  \let\item\@idxitem
}{%
  \ifkorrekturansicht\clearpage\fi
}
\makeatother

\IfFileExists{\jobname-pw.ind}{\input{\jobname-pw.ind}}{}

% Quellenangabe nur in der Leseansicht
\ifkorrekturansicht\else
% Fallback-Definitionen, falls die .tex-Datei \titel etc. nicht gesetzt hat
\providecommand{\titel}{}
\providecommand{\editorInnen}{}
\providecommand{\dateiname}{\jobname}

\vspace{3cm}

\vfill

\footnotesize
\textsc{Quelle}: \titel. Herausgegeben von {\editorInnen}. In: \emph{Arthur Schnitzler: Briefwechsel mit Autorinnen und Autoren}.
 Digitale Edition, https://schnitzler-briefe.acdh.oeaw.ac.at/{\dateiname}.html (Stand \today)
\fi

\end{document}


      