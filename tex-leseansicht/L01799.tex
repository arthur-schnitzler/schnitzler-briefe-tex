%% latex-leseansicht-vorspann.tex
%% Vorspann für die Leseansicht.
%% Lädt die gemeinsame Datei latex-vorspann.tex mit nicht gesetztem Schalter.

\newif\ifkorrekturansicht
\korrekturansichtfalse

\input{../tex-inputs/latex-vorspann}


\section[Hermann Bodmer und andere an Arthur Schnitzler, 9. 11. 1908]{L01799 Hermann Bodmer und andere an Arthur Schnitzler, 9. 11. 1908}
\nopagebreak\mylabel{L01799v}
\rehead{ }\normalsize\beginnumbering\briefempfaengerindex{Schnitzler, Arthur@\textsc{Schnitzler, Arthur}!zzzBahr, Hermann@\emph{von Hermann Bahr}!1908-11-092@{9. 11. 1908}|(be}\briefempfaengerindex{Schnitzler, Arthur@\textsc{Schnitzler, Arthur}!zzzBodmer, Hans@\emph{von Hans Bodmer}!1908-11-092@{9. 11. 1908}|(be}\briefempfaengerindex{Schnitzler, Arthur@\textsc{Schnitzler, Arthur}!zzzSartoris, M.@\emph{von M. Sartoris}!1908-11-092@{9. 11. 1908}|(be}\briefempfaengerindex{Schnitzler, Arthur@\textsc{Schnitzler, Arthur}!zzzSartoris, Spyridon Demetrius@\emph{von Spyridon Demetrius Sartoris}!1908-11-092@{9. 11. 1908}|(be}\briefempfaengerindex{Schnitzler, Arthur@\textsc{Schnitzler, Arthur}!zzzBodman, Emanuel von@\emph{von Emanuel von Bodman}!1908-11-092@{9. 11. 1908}|(be}\briefempfaengerindex{Schnitzler, Arthur@\textsc{Schnitzler, Arthur}!zzzBodmer, Hermann@\emph{von Hermann Bodmer}!1908-11-092@{9. 11. 1908}|(be}\briefempfaengerindex{Schnitzler, Arthur@\textsc{Schnitzler, Arthur}!zzzBodmer, Mathilde@\emph{von Mathilde Bodmer}!1908-11-092@{9. 11. 1908}|(be}\briefempfaengerindex{Schnitzler, Arthur@\textsc{Schnitzler, Arthur}!zzzKesser, Hermann@\emph{von Hermann Kesser}!1908-11-092@{9. 11. 1908}|(be}\briefempfaengerindex{Schnitzler, Arthur@\textsc{Schnitzler, Arthur}!zzzStaehelin-Baechtold, Gertrud@\emph{von Gertrud Staehelin-Baechtold}!1908-11-092@{9. 11. 1908}|(be}\briefempfaengerindex{Schnitzler, Arthur@\textsc{Schnitzler, Arthur}!zzzHuber, R. W.@\emph{von R. W. Huber}!1908-11-092@{9. 11. 1908}|(be}\briefempfaengerindex{Schnitzler, Arthur@\textsc{Schnitzler, Arthur}!zzzStaehelin-Baechtold, Sepp@\emph{von Sepp Staehelin-Baechtold}!1908-11-092@{9. 11. 1908}|(be}\briefempfaengerindex{Schnitzler, Arthur@\textsc{Schnitzler, Arthur}!zzzProbst, R.@\emph{von R. Probst}!1908-11-092@{9. 11. 1908}|(be}\briefempfaengerindex{Schnitzler, Arthur@\textsc{Schnitzler, Arthur}!zzzProbst, E.@\emph{von E. Probst}!1908-11-092@{9. 11. 1908}|(be}
\toendnotes[C]{\smallbreak\pagebreak[2]}
\correspDesc{Versand  durch E. Probst, R. Probst, Josef Staehelin-Baechtold, R. W. Huber, Gertrud Staehelin-Baechtold, Hermann Kesser, Mathilde Bodmer, Hermann Bodmer, Emanuel von Bodman, Spyridon Sartoris, W. Sartoris, Hans Bodmer, Hermann Bahr am 9. 11. 1908 in Zürich
\newline{}Erhalt  durch Arthur Schnitzler am 10. XI. 08 in Wien}\toendnotes[C]{\smallbreak}
\Standort{DLA, A:Schnitzler, HS.NZ85.1.2568.}
\physDesc{Bildpostkarte, 626 Zeichen
\newline{}Handschrift E. Probst: Bleistift
\newline{}Handschrift R. Probst: Bleistift
\newline{}Handschrift Sepp Staehelin-Baechtold: Bleistift
\newline{}Handschrift R. W. Huber: Bleistift
\newline{}Handschrift Gertrud Staehelin-Baechtold: Bleistift
\newline{}Handschrift Hermann Kesser: Bleistift
\newline{}Handschrift Mathilde Bodmer: Bleistift
\newline{}Handschrift Hermann Bodmer: Bleistift, lateinische Kurrent
\newline{}Handschrift Emanuel von Bodman: Bleistift
\newline{}Handschrift Spyridon Demetrius Sartoris: Bleistift
\newline{}Handschrift M. Sartoris: Bleistift
\newline{}Handschrift Hans Bodmer: Bleistift, lateinische Kurrent
\newline{}Handschrift Hermann Bahr: Bleistift, deutsche Kurrent
\newline{}Versand: Stempel: »\nobreak{}\oindex{Neumünster@\textbf{Neumünster}, \emph{Bezirk}|pwk}Zürich 12 Neumünster, 10. XI. 08, 2\nobreak{}«.  }\toendnotes[C]{\smallbreak}\pstart{}{\pb}D\textsuperscript{r} Artur
                  Schnitzler\pend{}\pstart{}Wien XVIII\oindex{XVIII., Währing@\textbf{XVIII., Währing}, \emph{Verwaltungsgebiet}|pw}\pend{}\pstart{}Spöttelgasse 7\oindex{Wien@\textbf{Wien}!XVIII., Währing@\textbf{XVIII., Währing}!Edmund-Weiß-Gasse 7@\textbf{Edmund-Weiß-Gasse 7}, \emph{Wohngebäude}|pw}\pend{}{\bigskip}
\pstart
           \noindent{}\centering{}{\pb}\textcolor{gray}{\textbf{Zürich – Rotes Schloss\oindex{Rotes Schloss@\textbf{Rotes Schloss}, \emph{Schloss}|pw} und Tonhalle\oindex{Kongresshaus Zürich@\textbf{Kongresshaus Zürich}, \emph{Konzertsaal}|pw}}}\pend
           \vspace{1em}\stanza{}{\pb}{[}hs. Bodmer:{]} Von Bahr vernahmen das Dilemma\newverse{}Wir eben zwischen Franz\pwindex{Schnitzler, Arthur 15.\,5.\,1862 Wien – 21.\,10.\,1931 ebd.@\textsc{Schnitzler, Arthur} (15.\,5.\,1862 Wien – 21.\,10.\,1931 ebd.), \emph{Schriftsteller, Mediziner}!Toten schweigen@\strich\emph{Die Toten schweigen}|pwv} u
                     Emma\pwindex{Schnitzler, Arthur 15.\,5.\,1862 Wien – 21.\,10.\,1931 ebd.@\textsc{Schnitzler, Arthur} (15.\,5.\,1862 Wien – 21.\,10.\,1931 ebd.), \emph{Schriftsteller, Mediziner}!Toten schweigen@\strich\emph{Die Toten schweigen}|pwv}\newverse{}Ja es ist bitter, wenn die »Toten
                  schweigen\pwindex{Schnitzler, Arthur 15.\,5.\,1862 Wien – 21.\,10.\,1931 ebd.@\textsc{Schnitzler, Arthur} (15.\,5.\,1862 Wien – 21.\,10.\,1931 ebd.), \emph{Schriftsteller, Mediziner}!Toten schweigen@\strich\emph{Die Toten schweigen}|pw}«\newverse{}Doch jetzt, in frohem, lust’gem »Reigen\pwindex{Schnitzler, Arthur 15.\,5.\,1862 Wien – 21.\,10.\,1931 ebd.@\textsc{Schnitzler, Arthur} (15.\,5.\,1862 Wien – 21.\,10.\,1931 ebd.), \emph{Schriftsteller, Mediziner}!Reigen. Zehn Dialoge@\strich\emph{Reigen. Zehn Dialoge}|pwv}« \newverse{}Von guter Speis’ u Trank ganz voll\newverse{}Gedenken wir des »Anatol\pwindex{Schnitzler, Arthur 15.\,5.\,1862 Wien – 21.\,10.\,1931 ebd.@\textsc{Schnitzler, Arthur} (15.\,5.\,1862 Wien – 21.\,10.\,1931 ebd.), \emph{Schriftsteller, Mediziner}!Anatol@\strich\emph{Anatol}|pw}«\newverse{}Und wünschen sehnlich ihn herbei\newverse{}Das gäb’ ne nette »Liebelei\pwindex{Schnitzler, Arthur 15.\,5.\,1862 Wien – 21.\,10.\,1931 ebd.@\textsc{Schnitzler, Arthur} (15.\,5.\,1862 Wien – 21.\,10.\,1931 ebd.), \emph{Schriftsteller, Mediziner}!Liebelei. Schauspiel in drei Akten@\strich\emph{Liebelei. Schauspiel in drei Akten}|pw}«\newverse{}Und mit dem »grünen Kakadu\pwindex{Schnitzler, Arthur 15.\,5.\,1862 Wien – 21.\,10.\,1931 ebd.@\textsc{Schnitzler, Arthur} (15.\,5.\,1862 Wien – 21.\,10.\,1931 ebd.), \emph{Schriftsteller, Mediziner}!grüne Kakadu. Groteske in einem Akt@\strich\emph{Der grüne Kakadu. Groteske in einem Akt}|pw}«\newverse{}Wär’n wir gar balde Du und Du!!\stanzaend{}\selectlanguage{ngerman}\vspace{1em}
\pstart
           \noindent{}{[}hs. Bahr:{]} \label{T_L01799-1v}\edtext{Herzlich \spacefill\mbox{Hermann}}{\lemma{\textnormal{\emph{Herzlich Hermann}}}\Cendnote{\textnormal{in der oberen linken Ecke, verkehrt zum
                  Text}}}\label{T_L01799-1}\pend
           \selectlanguage{ngerman}\vspace{1em}
\pstart
           \noindent{}{\pb}{[}hs. Bodmer:{]} Vom Schnitzler-Abend, den uns Herma{\geminationn} Bahr bot, senden wir Ihnen herzliche Grüsse:\pend
           
\pstart
           \spacefill\mbox{Hans Bodmer}\pend
           
\pstart
           \spacefill\mbox{{[}hs. Probst:{]} EProbst}\pend
           
\pstart
           \spacefill\mbox{{[}hs. Probst:{]} R. Probst}\pend
           
\pstart
           \spacefill\mbox{{[}hs. Staehelin-Baechtold:{]} Sepp Staehelin}\pend
           
\pstart
           \spacefill\mbox{{[}hs. Huber:{]} R. W. Huber.}\pend
           
\pstart
           \spacefill\mbox{{[}hs. Staehelin-Baechtold:{]} Gertrud Staehelin-Baechtold.}\pend
           
\pstart
           \spacefill\mbox{{[}hs. Kesser:{]} Hermann Kesser}\pend
           
\pstart
           \spacefill\mbox{{[}hs. Bodmer:{]} Mathilde Bodmer}\pend
           
\pstart
           \spacefill\mbox{{[}hs. Bodmer:{]} Hermann Bodmer}\pend
           
\pstart
           \spacefill\mbox{{[}hs. Bodman:{]} Emanuel von Bodman}\pend
           
\pstart
           \spacefill\mbox{{[}hs. Sartoris:{]} Spyridon Sartoris}\pend
           
\pstart
           \spacefill\mbox{{[}hs. Sartoris:{]} M. Sartoris.}\pend
           \selectlanguage{ngerman}\endnumbering\briefempfaengerindex{Schnitzler, Arthur@\textsc{Schnitzler, Arthur}!zzzBahr, Hermann@\emph{von Hermann Bahr}!1908-11-092@{9. 11. 1908}|)be}\briefempfaengerindex{Schnitzler, Arthur@\textsc{Schnitzler, Arthur}!zzzBodmer, Hans@\emph{von Hans Bodmer}!1908-11-092@{9. 11. 1908}|)be}\briefempfaengerindex{Schnitzler, Arthur@\textsc{Schnitzler, Arthur}!zzzSartoris, M.@\emph{von M. Sartoris}!1908-11-092@{9. 11. 1908}|)be}\briefempfaengerindex{Schnitzler, Arthur@\textsc{Schnitzler, Arthur}!zzzSartoris, Spyridon Demetrius@\emph{von Spyridon Demetrius Sartoris}!1908-11-092@{9. 11. 1908}|)be}\briefempfaengerindex{Schnitzler, Arthur@\textsc{Schnitzler, Arthur}!zzzBodman, Emanuel von@\emph{von Emanuel von Bodman}!1908-11-092@{9. 11. 1908}|)be}\briefempfaengerindex{Schnitzler, Arthur@\textsc{Schnitzler, Arthur}!zzzBodmer, Hermann@\emph{von Hermann Bodmer}!1908-11-092@{9. 11. 1908}|)be}\briefempfaengerindex{Schnitzler, Arthur@\textsc{Schnitzler, Arthur}!zzzBodmer, Mathilde@\emph{von Mathilde Bodmer}!1908-11-092@{9. 11. 1908}|)be}\briefempfaengerindex{Schnitzler, Arthur@\textsc{Schnitzler, Arthur}!zzzKesser, Hermann@\emph{von Hermann Kesser}!1908-11-092@{9. 11. 1908}|)be}\briefempfaengerindex{Schnitzler, Arthur@\textsc{Schnitzler, Arthur}!zzzStaehelin-Baechtold, Gertrud@\emph{von Gertrud Staehelin-Baechtold}!1908-11-092@{9. 11. 1908}|)be}\briefempfaengerindex{Schnitzler, Arthur@\textsc{Schnitzler, Arthur}!zzzHuber, R. W.@\emph{von R. W. Huber}!1908-11-092@{9. 11. 1908}|)be}\briefempfaengerindex{Schnitzler, Arthur@\textsc{Schnitzler, Arthur}!zzzStaehelin-Baechtold, Sepp@\emph{von Sepp Staehelin-Baechtold}!1908-11-092@{9. 11. 1908}|)be}\briefempfaengerindex{Schnitzler, Arthur@\textsc{Schnitzler, Arthur}!zzzProbst, R.@\emph{von R. Probst}!1908-11-092@{9. 11. 1908}|)be}\briefempfaengerindex{Schnitzler, Arthur@\textsc{Schnitzler, Arthur}!zzzProbst, E.@\emph{von E. Probst}!1908-11-092@{9. 11. 1908}|)be}\mylabel{L01799h}  \newcommand{\dateiname}{L01799}\newcommand{\titel}{Hermann Bodmer und andere an Arthur Schnitzler, 9. 11. 1908}\newcommand{\editorInnen}{Martin Anton Müller und Gerd-Hermann Susen}%% latex-leseansicht-abspann.tex
%% Abspann für die Leseansicht.
%% Der Schalter \ifkorrekturansicht ist bereits durch den Vorspann gesetzt.

%% latex-abspann.tex
%% Gemeinsamer Abspann für Korrekturansicht und Leseansicht.
%% Setzt den Schalter \ifkorrekturansicht voraus (gesetzt in den
%% einbindenden Dateien latex-korrekturansicht-abspann.tex bzw.
%% latex-leseansicht-abspann.tex).
%% ---------------------------------------------------------------

\normalsize

% Das esempio-Environment wird nur in der Leseansicht benötigt
\ifkorrekturansicht\else
\newenvironment{esempio}[3]%
{
    \vspace{1.5ex}
    \rlap{\underline{#1}}
    \par
    \setlength{\parindent}{0cm}
    \nopagebreak
    \leftskip=#2cm
    \rightskip=#3cm
}
{
    \par
}
\fi

\doendnotes{C}
\bigskip
\vfill

\clearpage

\footnotesize

\ifkorrekturansicht
  \lohead{\textsc{register}}
\fi

% theindex-Environment neu definieren ohne reledmac
\makeatletter
\renewenvironment{theindex}{%
  \ifkorrekturansicht
    \section*{\indexname}%
  \else
    \subsubsection*{Index der erwähnten Entitäten}%
  \fi
  \setlength{\parindent}{0pt}%
  \setlength{\parskip}{0pt plus 0.3pt}%
  \let\item\@idxitem
}{%
  \ifkorrekturansicht\clearpage\fi
}
\makeatother

\IfFileExists{\jobname-pw.ind}{\input{\jobname-pw.ind}}{}

% Quellenangabe nur in der Leseansicht
\ifkorrekturansicht\else
% Fallback-Definitionen, falls die .tex-Datei \titel etc. nicht gesetzt hat
\providecommand{\titel}{}
\providecommand{\editorInnen}{}
\providecommand{\dateiname}{\jobname}

\vspace{3cm}

\vfill

\footnotesize
\textsc{Quelle}: \titel. Herausgegeben von {\editorInnen}. In: \emph{Arthur Schnitzler: Briefwechsel mit Autorinnen und Autoren}.
 Digitale Edition, https://schnitzler-briefe.acdh.oeaw.ac.at/{\dateiname}.html (Stand \today)
\fi

\end{document}


