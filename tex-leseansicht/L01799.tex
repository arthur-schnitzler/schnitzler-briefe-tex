%% latex-korrekturansicht-vorspann.tex
%% Vorspann für die Korrekturansicht.
%% Lädt die gemeinsame Datei latex-vorspann.tex mit gesetztem Schalter.

\newif\ifkorrekturansicht
\korrekturansichttrue

\input{../tex-inputs/latex-vorspann}


\section[Hermann Bodmer und andere an Arthur Schnitzler, 9. 11. 1908]{L01799 Hermann Bodmer und andere an Arthur Schnitzler, 9. 11. 1908}
\nopagebreak\mylabel{L01799v}
\rehead{ }\normalsize\beginnumbering\briefempfaengerindex{Schnitzler, Arthur@\textsc{Schnitzler, Arthur}!zzzBahr, Hermann@\emph{von Hermann Bahr}!1908-11-092@{9. 11. 1908}|(be}\briefempfaengerindex{Schnitzler, Arthur@\textsc{Schnitzler, Arthur}!zzzBodmer, Hans@\emph{von Hans Bodmer}!1908-11-092@{9. 11. 1908}|(be}\briefempfaengerindex{Schnitzler, Arthur@\textsc{Schnitzler, Arthur}!zzzSartoris, M.@\emph{von M. Sartoris}!1908-11-092@{9. 11. 1908}|(be}\briefempfaengerindex{Schnitzler, Arthur@\textsc{Schnitzler, Arthur}!zzzSartoris, Spyridon Demetrius@\emph{von Spyridon Demetrius Sartoris}!1908-11-092@{9. 11. 1908}|(be}\briefempfaengerindex{Schnitzler, Arthur@\textsc{Schnitzler, Arthur}!zzzBodman, Emanuel von@\emph{von Emanuel von Bodman}!1908-11-092@{9. 11. 1908}|(be}\briefempfaengerindex{Schnitzler, Arthur@\textsc{Schnitzler, Arthur}!zzzBodmer, Hermann@\emph{von Hermann Bodmer}!1908-11-092@{9. 11. 1908}|(be}\briefempfaengerindex{Schnitzler, Arthur@\textsc{Schnitzler, Arthur}!zzzBodmer, Mathilde@\emph{von Mathilde Bodmer}!1908-11-092@{9. 11. 1908}|(be}\briefempfaengerindex{Schnitzler, Arthur@\textsc{Schnitzler, Arthur}!zzzKesser, Hermann@\emph{von Hermann Kesser}!1908-11-092@{9. 11. 1908}|(be}\briefempfaengerindex{Schnitzler, Arthur@\textsc{Schnitzler, Arthur}!zzzStaehelin-Baechtold, Gertrud@\emph{von Gertrud Staehelin-Baechtold}!1908-11-092@{9. 11. 1908}|(be}\briefempfaengerindex{Schnitzler, Arthur@\textsc{Schnitzler, Arthur}!zzzHuber, R. W.@\emph{von R. W. Huber}!1908-11-092@{9. 11. 1908}|(be}\briefempfaengerindex{Schnitzler, Arthur@\textsc{Schnitzler, Arthur}!zzzStaehelin-Baechtold, Sepp@\emph{von Sepp Staehelin-Baechtold}!1908-11-092@{9. 11. 1908}|(be}\briefempfaengerindex{Schnitzler, Arthur@\textsc{Schnitzler, Arthur}!zzzProbst, R.@\emph{von R. Probst}!1908-11-092@{9. 11. 1908}|(be}\briefempfaengerindex{Schnitzler, Arthur@\textsc{Schnitzler, Arthur}!zzzProbst, E.@\emph{von E. Probst}!1908-11-092@{9. 11. 1908}|(be}
\toendnotes[C]{\smallbreak\pagebreak[2]}\Standort{DLA, A:Schnitzler, HS.NZ85.1.2568.}
\physDesc{Bildpostkarte, 626 Zeichen
\newline{}Handschrift E. Probst: Bleistift
\newline{}Handschrift R. Probst: Bleistift
\newline{}Handschrift Sepp Staehelin-Baechtold: Bleistift
\newline{}Handschrift R. W. Huber: Bleistift
\newline{}Handschrift Gertrud Staehelin-Baechtold: Bleistift
\newline{}Handschrift Hermann Kesser: Bleistift
\newline{}Handschrift Mathilde Bodmer: Bleistift
\newline{}Handschrift Hermann Bodmer: Bleistift, lateinische Kurrent
\newline{}Handschrift Emanuel von Bodman: Bleistift
\newline{}Handschrift Spyridon Demetrius Sartoris: Bleistift
\newline{}Handschrift M. Sartoris: Bleistift
\newline{}Handschrift Hans Bodmer: Bleistift, lateinische Kurrent
\newline{}Handschrift Hermann Bahr: Bleistift, deutsche Kurrent
\newline{}Versand: Stempel: »\nobreak{}\oindex{Neumuenster@\textbf{Neumünster}, \emph{Bezirk (A.BZK)}|pwk}Zürich 12 Neumünster, 10. XI. 08, 2\nobreak{}«.  }\toendnotes[C]{\smallbreak}\pstart{}{\pb}D\textsuperscript{r} Artur
                  Schnitzler\pend{}\pstart{}Wien XVIII\oindex{XVIII., Waehring@\textbf{XVIII., Währing}, \emph{A.ADM3}|pw}\pend{}\pstart{}Spöttelgasse 7\oindex{Edmund-Weiss-Gasse 7@\textbf{Edmund-Weiß-Gasse 7}, \emph{Wohngebäude (K.WHS)}|pw}\pend{}{\bigskip}
\pstart
           \noindent{}\centering{}{\pb}\textcolor{gray}{\textbf{Zürich – Rotes Schloss\oindex{Rotes Schloss@\textbf{Rotes Schloss}, \emph{Schloss (K.SLS)}|pw} und Tonhalle\oindex{Kongresshaus Zuerich@\textbf{Kongresshaus Zürich}, \emph{Konzertsaal (K.KNZ)}|pw}}}\pend
           \vspace{1em}\stanza{}{\pb}{[}hs. :{]} Von Bahr vernahmen das DilemmaWir eben zwischen Franz\pwindex{Toten schweigen@\emph{Die Toten schweigen}|pwv} u
                     Emma\pwindex{Toten schweigen@\emph{Die Toten schweigen}|pwv}Ja es ist bitter, wenn die »Toten
                  schweigen\pwindex{Toten schweigen@\emph{Die Toten schweigen}|pw}«Doch jetzt, in frohem, lust’gem »Reigen\pwindex{Reigen. Zehn Dialoge@\emph{Reigen. Zehn Dialoge}|pwv}« Von guter Speis’ u Trank ganz vollGedenken wir des »Anatol\pwindex{Anatol@\emph{Anatol}|pw}«Und wünschen sehnlich ihn herbeiDas gäb’ ne nette »Liebelei\pwindex{Liebelei. Schauspiel in drei Akten@\emph{Liebelei. Schauspiel in drei Akten}|pw}«Und mit dem »grünen Kakadu\pwindex{gruene Kakadu. Groteske in einem Akt@\emph{Der grüne Kakadu. Groteske in einem Akt}|pw}«Wär’n wir gar balde Du und Du!!\stanzaend{}\selectlanguage{ngerman}\vspace{1em}
\pstart
           \noindent{}{[}hs. :{]} \label{T_L01799-1v}\edtext{Herzlich \spacefill\mbox{Hermann}}{\lemma{\textnormal{\emph{Herzlich Hermann}}}\Cendnote{\textnormal{in der oberen linken Ecke, verkehrt zum
                  Text}}}\label{T_L01799-1}\pend
           \selectlanguage{ngerman}\vspace{1em}
\pstart
           \noindent{}{\pb}{[}hs. :{]} Vom Schnitzler-Abend, den uns Herma{\geminationn} Bahr bot, senden wir Ihnen herzliche Grüsse:\pend
           
\pstart
           \spacefill\mbox{Hans Bodmer}\pend
           
\pstart
           \spacefill\mbox{{[}hs. :{]} EProbst}\pend
           
\pstart
           \spacefill\mbox{{[}hs. :{]} R. Probst}\pend
           
\pstart
           \spacefill\mbox{{[}hs. :{]} Sepp Staehelin}\pend
           
\pstart
           \spacefill\mbox{{[}hs. :{]} R. W. Huber.}\pend
           
\pstart
           \spacefill\mbox{{[}hs. :{]} Gertrud Staehelin-Baechtold.}\pend
           
\pstart
           \spacefill\mbox{{[}hs. :{]} Hermann Kesser}\pend
           
\pstart
           \spacefill\mbox{{[}hs. :{]} Mathilde Bodmer}\pend
           
\pstart
           \spacefill\mbox{{[}hs. :{]} Hermann Bodmer}\pend
           
\pstart
           \spacefill\mbox{{[}hs. :{]} Emanuel von Bodman}\pend
           
\pstart
           \spacefill\mbox{{[}hs. :{]} Spyridon Sartoris}\pend
           
\pstart
           \spacefill\mbox{{[}hs. :{]} M. Sartoris.}\pend
           \selectlanguage{ngerman}\endnumbering\briefempfaengerindex{Schnitzler, Arthur@\textsc{Schnitzler, Arthur}!zzzBahr, Hermann@\emph{von Hermann Bahr}!1908-11-092@{9. 11. 1908}|)be}\briefempfaengerindex{Schnitzler, Arthur@\textsc{Schnitzler, Arthur}!zzzBodmer, Hans@\emph{von Hans Bodmer}!1908-11-092@{9. 11. 1908}|)be}\briefempfaengerindex{Schnitzler, Arthur@\textsc{Schnitzler, Arthur}!zzzSartoris, M.@\emph{von M. Sartoris}!1908-11-092@{9. 11. 1908}|)be}\briefempfaengerindex{Schnitzler, Arthur@\textsc{Schnitzler, Arthur}!zzzSartoris, Spyridon Demetrius@\emph{von Spyridon Demetrius Sartoris}!1908-11-092@{9. 11. 1908}|)be}\briefempfaengerindex{Schnitzler, Arthur@\textsc{Schnitzler, Arthur}!zzzBodman, Emanuel von@\emph{von Emanuel von Bodman}!1908-11-092@{9. 11. 1908}|)be}\briefempfaengerindex{Schnitzler, Arthur@\textsc{Schnitzler, Arthur}!zzzBodmer, Hermann@\emph{von Hermann Bodmer}!1908-11-092@{9. 11. 1908}|)be}\briefempfaengerindex{Schnitzler, Arthur@\textsc{Schnitzler, Arthur}!zzzBodmer, Mathilde@\emph{von Mathilde Bodmer}!1908-11-092@{9. 11. 1908}|)be}\briefempfaengerindex{Schnitzler, Arthur@\textsc{Schnitzler, Arthur}!zzzKesser, Hermann@\emph{von Hermann Kesser}!1908-11-092@{9. 11. 1908}|)be}\briefempfaengerindex{Schnitzler, Arthur@\textsc{Schnitzler, Arthur}!zzzStaehelin-Baechtold, Gertrud@\emph{von Gertrud Staehelin-Baechtold}!1908-11-092@{9. 11. 1908}|)be}\briefempfaengerindex{Schnitzler, Arthur@\textsc{Schnitzler, Arthur}!zzzHuber, R. W.@\emph{von R. W. Huber}!1908-11-092@{9. 11. 1908}|)be}\briefempfaengerindex{Schnitzler, Arthur@\textsc{Schnitzler, Arthur}!zzzStaehelin-Baechtold, Sepp@\emph{von Sepp Staehelin-Baechtold}!1908-11-092@{9. 11. 1908}|)be}\briefempfaengerindex{Schnitzler, Arthur@\textsc{Schnitzler, Arthur}!zzzProbst, R.@\emph{von R. Probst}!1908-11-092@{9. 11. 1908}|)be}\briefempfaengerindex{Schnitzler, Arthur@\textsc{Schnitzler, Arthur}!zzzProbst, E.@\emph{von E. Probst}!1908-11-092@{9. 11. 1908}|)be}\mylabel{L01799h}  \normalsize

\doendnotes{C}
\bigskip
\vfill

\clearpage

\footnotesize

\lohead{\textsc{register}}

% Definiere theindex-Environment komplett neu ohne reledmac
\makeatletter
\renewenvironment{theindex}{%
  \section*{\indexname}%
  \setlength{\parindent}{0pt}%
  \setlength{\parskip}{0pt plus 0.3pt}%
  \let\item\@idxitem
}{%
  \clearpage
}
\makeatother

\IfFileExists{\jobname-pw.ind}{\input{\jobname-pw.ind}}{}

\end{document}

      