%% latex-leseansicht-vorspann.tex
%% Vorspann für die Leseansicht.
%% Lädt die gemeinsame Datei latex-vorspann.tex mit nicht gesetztem Schalter.

\newif\ifkorrekturansicht
\korrekturansichtfalse

\input{../tex-inputs/latex-vorspann}


         
         \renewcommand{\erwaehntePersonen}{Personen: Hermann Bahr, Emanuel von Bodman, Mathilde Bodmer, Hermann Bodmer, Hans Bodmer, Spyridon Demetrius Sartoris, M. Sartoris}
         \renewcommand{\erwaehnteOrte}{Orte: Edmund-Weiß-Gasse, Kongresshaus Zürich, Neumünster, Rotes Schloss, Wien, XVIII., Währing, Zürich}
         \renewcommand{\erwaehnteWerke}{Werke: Anatol, Der grüne Kakadu. Groteske in einem Akt, Die Toten schweigen, Liebelei. Schauspiel in drei Akten, Reigen. Zehn Dialoge}
               \section[Hermann Bodmer und andere an Arthur Schnitzler, 9. 11. 1908]{ Hermann Bodmer und andere an Arthur Schnitzler, 9. 11. 1908}\nopagebreak\mylabel{v}\rehead{ }\begin{ledgroupsized}[t]{13cm}\normalsize\beginnumbering \toendnotes[C]{\smallbreak\pagebreak[2]} \Standort{DLA, A:Schnitzler, HS.NZ85.1.2568.}
\physDesc{Bildpostkarte, 626 Zeichen
\newline{}Handschrift E. Probst: Bleistift\newline{}Handschrift R. Probst: Bleistift\newline{}Handschrift Sepp Staehelin-Baechtold: Bleistift\newline{}Handschrift R. W. Huber: Bleistift\newline{}Handschrift Gertrud Staehelin-Baechtold: Bleistift\newline{}Handschrift Hermann Kesser: Bleistift\newline{}Handschrift Mathilde Bodmer: Bleistift\newline{}Handschrift Hermann Bodmer: Bleistift, lateinische Kurrent\newline{}Handschrift Emanuel von Bodman: Bleistift\newline{}Handschrift Spyridon Demetrius Sartoris: Bleistift\newline{}Handschrift M. Sartoris: Bleistift\newline{}Handschrift Hans Bodmer: Bleistift, lateinische Kurrent\newline{}Handschrift Hermann Bahr: Bleistift, deutsche Kurrent
\newline{}Versand: Stempel: »\nobreak{}\oindex{Neumuenster@\textbf{Neumünster}|pwk}Zürich 12 Neumünster, 10. XI. 08, 2\nobreak{}«.  }\toendnotes[C]{\smallbreak}\pstart{}{\pb}D\textsuperscript{r} Artur
                  Schnitzler\pend{}\pstart{}Wien XVIII\oindex{XVIII., Waehring@\textbf{XVIII., Währing}|pw}\pend{}\pstart{}Spöttelgasse 7\oindex{XXXX Ortsangabe fehlt|pw}\pend{}{\bigskip}\pstart
           \noindent{}\centering{}{\pb}\textcolor{gray}{\textbf{Zürich – Rotes Schloss\oindex{Rotes Schloss@\textbf{Rotes Schloss}|pw} und Tonhalle\oindex{Kongresshaus Zuerich@\textbf{Kongresshaus Zürich}|pw}}}\pend
           \stanza{}{\pb}{[}hs. Hermann Bodmer:{]} Von Bahr vernahmen das Dilemma\newverse{}Wir eben zwischen Franz\pwindex{Schnitzler, Arthur 15.05.1862 – 21.10.1931@\textsc{Schnitzler, Arthur} (15.05.1862 – 21.10.1931), \emph{Schriftsteller, Mediziner}!Toten schweigen01. 10. 1897@\strich\emph{Die Toten schweigen} {[}01. 10. 1897{]}|pwv} u
                     Emma\pwindex{Schnitzler, Arthur 15.05.1862 – 21.10.1931@\textsc{Schnitzler, Arthur} (15.05.1862 – 21.10.1931), \emph{Schriftsteller, Mediziner}!Toten schweigen01. 10. 1897@\strich\emph{Die Toten schweigen} {[}01. 10. 1897{]}|pwv}\newverse{}Ja es ist bitter, wenn die »Toten
                  schweigen\pwindex{Schnitzler, Arthur 15.05.1862 – 21.10.1931@\textsc{Schnitzler, Arthur} (15.05.1862 – 21.10.1931), \emph{Schriftsteller, Mediziner}!Toten schweigen01. 10. 1897@\strich\emph{Die Toten schweigen} {[}01. 10. 1897{]}|pw}«\newverse{}Doch jetzt, in frohem, lust’gem »Reigen\pwindex{Schnitzler, Arthur 15.05.1862 – 21.10.1931@\textsc{Schnitzler, Arthur} (15.05.1862 – 21.10.1931), \emph{Schriftsteller, Mediziner}!Reigen. Zehn Dialoge1900@\strich\emph{Reigen. Zehn Dialoge} {[}1900{]}|pwv}« \newverse{}Von guter Speis’ u Trank ganz voll\newverse{}Gedenken wir des »Anatol\pwindex{Schnitzler, Arthur 15.05.1862 – 21.10.1931@\textsc{Schnitzler, Arthur} (15.05.1862 – 21.10.1931), \emph{Schriftsteller, Mediziner}!Anatol1892-10-29@\strich\emph{Anatol} {[}1892-10-29{]}|pw}«\newverse{}Und wünschen sehnlich ihn herbei\newverse{}Das gäb’ ne nette »Liebelei\pwindex{Schnitzler, Arthur 15.05.1862 – 21.10.1931@\textsc{Schnitzler, Arthur} (15.05.1862 – 21.10.1931), \emph{Schriftsteller, Mediziner}!Liebelei. Schauspiel in drei Akten1895-10-09@\strich\emph{Liebelei. Schauspiel in drei Akten} {[}1895-10-09{]}|pw}«\newverse{}Und mit dem »grünen Kakadu\pwindex{Schnitzler, Arthur 15.05.1862 – 21.10.1931@\textsc{Schnitzler, Arthur} (15.05.1862 – 21.10.1931), \emph{Schriftsteller, Mediziner}!gruene Kakadu. Groteske in einem Akt1. 3. 1899@\strich\emph{Der grüne Kakadu. Groteske in einem Akt} {[}1. 3. 1899{]}|pw}«\newverse{}Wär’n wir gar balde Du und Du!!\stanzaend{}\pstart
           \noindent{}{[}hs. Bahr:{]} \label{T_L01799-1v}\edtext{Herzlich \spacefill\mbox{Hermann}}{\lemma{\textnormal{\emph{Herzlich Hermann}}}\Cendnote{\textnormal{in der oberen linken Ecke, verkehrt zum
                  Text}}}\label{T_L01799-1h}\pend
           \pstart
           \noindent{}{\pb}{[}hs. Hans Bodmer:{]} Vom Schnitzler-Abend, den uns Herma{\geminationn} Bahr bot, senden wir Ihnen herzliche Grüsse:\pend
           \pstart
           \spacefill\mbox{Hans Bodmer}\pend
           \pstart
           \spacefill\mbox{{[}hs. E. Probst:{]} EProbst}\pend
           \pstart
           \spacefill\mbox{{[}hs. R. Probst:{]} R. Probst}\pend
           \pstart
           \spacefill\mbox{{[}hs. Sepp Staehelin-Baechtold:{]} Sepp Staehelin}\pend
           \pstart
           \spacefill\mbox{{[}hs. Huber:{]} R. W. Huber.}\pend
           \pstart
           \spacefill\mbox{{[}hs. Gertrud Staehelin-Baechtold:{]} Gertrud Staehelin-Baechtold.}\pend
           \pstart
           \spacefill\mbox{{[}hs. Kesser:{]} Hermann Kesser}\pend
           \pstart
           \spacefill\mbox{{[}hs. Mathilde Bodmer:{]} Mathilde Bodmer}\pend
           \pstart
           \spacefill\mbox{{[}hs. Hermann Bodmer:{]} Hermann Bodmer}\pend
           \pstart
           \spacefill\mbox{{[}hs. Bodman:{]} Emanuel von Bodman}\pend
           \pstart
           \spacefill\mbox{{[}hs. Spyridon Demetrius Sartoris:{]} Spyridon Sartoris}\pend
           \pstart
           \spacefill\mbox{{[}hs. M. Sartoris:{]} M. Sartoris.}\pend
           
         
         \endnumbering\mylabel{h}\end{ledgroupsized}  \newcommand{\dateiname}{L01799}\newcommand{\titel}{Hermann Bodmer und andere an Arthur Schnitzler, 9. 11. 1908}\newcommand{\editorInnen}{Martin Anton Müller und Gerd-Hermann Susen}%% latex-leseansicht-abspann.tex
%% Abspann für die Leseansicht.
%% Der Schalter \ifkorrekturansicht ist bereits durch den Vorspann gesetzt.

%% latex-abspann.tex
%% Gemeinsamer Abspann für Korrekturansicht und Leseansicht.
%% Setzt den Schalter \ifkorrekturansicht voraus (gesetzt in den
%% einbindenden Dateien latex-korrekturansicht-abspann.tex bzw.
%% latex-leseansicht-abspann.tex).
%% ---------------------------------------------------------------

\normalsize

% Das esempio-Environment wird nur in der Leseansicht benötigt
\ifkorrekturansicht\else
\newenvironment{esempio}[3]%
{
    \vspace{1.5ex}
    \rlap{\underline{#1}}
    \par
    \setlength{\parindent}{0cm}
    \nopagebreak
    \leftskip=#2cm
    \rightskip=#3cm
}
{
    \par
}
\fi

\doendnotes{C}
\bigskip
\vfill

\clearpage

\footnotesize

\ifkorrekturansicht
  \lohead{\textsc{register}}
\fi

% theindex-Environment neu definieren ohne reledmac
\makeatletter
\renewenvironment{theindex}{%
  \ifkorrekturansicht
    \section*{\indexname}%
  \else
    \subsubsection*{Index der erwähnten Entitäten}%
  \fi
  \setlength{\parindent}{0pt}%
  \setlength{\parskip}{0pt plus 0.3pt}%
  \let\item\@idxitem
}{%
  \ifkorrekturansicht\clearpage\fi
}
\makeatother

\IfFileExists{\jobname-pw.ind}{\input{\jobname-pw.ind}}{}

% Quellenangabe nur in der Leseansicht
\ifkorrekturansicht\else
% Fallback-Definitionen, falls die .tex-Datei \titel etc. nicht gesetzt hat
\providecommand{\titel}{}
\providecommand{\editorInnen}{}
\providecommand{\dateiname}{\jobname}

\vspace{3cm}

\vfill

\footnotesize
\textsc{Quelle}: \titel. Herausgegeben von {\editorInnen}. In: \emph{Arthur Schnitzler: Briefwechsel mit Autorinnen und Autoren}.
 Digitale Edition, https://schnitzler-briefe.acdh.oeaw.ac.at/{\dateiname}.html (Stand \today)
\fi

\end{document}


      