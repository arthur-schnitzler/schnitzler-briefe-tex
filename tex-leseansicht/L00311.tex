%% latex-leseansicht-vorspann.tex
%% Vorspann für die Leseansicht.
%% Lädt die gemeinsame Datei latex-vorspann.tex mit nicht gesetztem Schalter.

\newif\ifkorrekturansicht
\korrekturansichtfalse

\input{../tex-inputs/latex-vorspann}


         
         \newcommand{\erwaehntePersonen}{Personen: Marcel Prévost}
         \newcommand{\erwaehnteInstitutionen}{Institutionen: Frankfurter Zeitung}
         \newcommand{\erwaehnteOrte}{Orte: Frankfurt am Main, Wien}
         \newcommand{\erwaehnteWerke}{Werke: Blumen, Kraft}
               \section[Fedor Mamroth an Arthur Schnitzler, 4. 4. 1894]{ Fedor Mamroth an Arthur Schnitzler, 4. 4. 1894}\nopagebreak\mylabel{v}\rehead{ }\begin{ledgroupsized}[t]{13cm}\normalsize\beginnumbering \toendnotes[C]{\smallbreak\pagebreak[2]} \Standort{CUL, Schnitzler, B 68.}
\physDesc{Brief, 1 Blatt, 1 Seite
\newline{}Handschrift einer Schreibkraft: schwarze Tinte, deutsche Kurrent
\newline{}Schnitzler: 1) mit Bleistift nummeriert: »6« und   2) mit rotem Buntstift
            beschriftet: »\textsc{Mam}« und zwei Unterstreichungen}\toendnotes[C]{\smallbreak}\pstart
           \noindent{}{\pb}\textcolor{gray}{\textbf{Frankfurter Zeitung\orgindex{Frankfurter Zeitung@Frankfurter Zeitung|pw}}}\orgindex{Frankfurter Zeitung@Frankfurter Zeitung|pw}\hfill \textcolor{gray}{\textbf{Frankfurt a. M.\oindex{Frankfurt am Main@\textbf{Frankfurt am Main}|pw},}}{ }4/4 \textcolor{gray}{\textbf{189}}4.\pend
           \pstart
           \textcolor{gray}{\textbf{und}}{\\}\textcolor{gray}{\textbf{Handelsblatt.}}\pend
           \pstart
           \textcolor{gray}{\textbf{Redaction.\footnote{\noindent{}\textcolor{gray}{\textbf{Für die Redaktion beſtimmte Briefe und
                                        Sendungen wolle man \so{nicht} an die
                                        Perſon eines Redakteurs, ſondern ſtets \textbf{an
                                            die Redaktion der Frankfurter Zeitung}
                                        adreſſiren}}.}}}\pend
           \pstart
           \textcolor{gray}{\textbf{Telegramm-Adreſſe:}}\pend
           \pstart
           \textcolor{gray}{\textbf{Zeitung Frankfurt Main\orgindex{Frankfurter Zeitung@Frankfurter Zeitung|pw}.}}\pend
           \pstart{}Hochgeehrter Herr Doktor.\pend\pstart
           Ich veröffentliche gegenwärtig einen großen Roman\pwindex{\textcolor{red}{\textsuperscript{XXXX1 indx}}!Kraft1894@\strich\emph{Kraft} {[}1894{]}|pwv}, dem ſich unmittelbar ein \label{K_L00311_1v}\edtext{anderer}{\lemma{\textnormal{\emph{anderer}}}\Cendnote{\textnormal{Das war dann nicht der Fall,
                    in Folge erschienen Novellen und Erzählungen verschiedener Autoren.}}}\label{K_L00311_1h} von \textsc{M. Prevost}\pwindex{Prevost, Marcel 01.05.1862 – 08.04.1941@\textsc{Prévost, Marcel} (01.05.1862 – 08.04.1941), \emph{Schriftsteller}|pw} anreihen wird. Ich bin deshalb auf lange Zeit hinaus außer ſtande,
                    für kleine novelliſtiſche Arbeiten Raum zu finden u. muß Ihnen deßhalb Ihr ſehr
                    ſchönes \textsc{Pastell}\pwindex{Schnitzler, Arthur 15.05.1862 – 21.10.1931@\textsc{Schnitzler, Arthur} (15.05.1862 – 21.10.1931), \emph{Schriftsteller, Mediziner}!Blumen01. 08. 1894@\strich\emph{Blumen} {[}01. 08. 1894{]}|pwv} zu meinem lebhaften Bedauern retournieren. Ich empfehle mich mit
                    herzlichem Gruß.\pend
           \pstart
           Hochachtungsvoll{\\[\baselineskip]}Ihr ergebener{\\[\baselineskip]}per{\\[\baselineskip]}\spacefill\mbox{D\textsuperscript{r.} F. Mamroth}\pend
           \leftskip=0em{}
         
         \endnumbering\mylabel{h}\end{ledgroupsized}  \newcommand{\dateiname}{L00311}\newcommand{\titel}{Fedor Mamroth an Arthur Schnitzler, 4. 4. 1894}\newcommand{\editorInnen}{Martin Anton Müller und Gerd-Hermann Susen}%% latex-leseansicht-abspann.tex
%% Abspann für die Leseansicht.
%% Der Schalter \ifkorrekturansicht ist bereits durch den Vorspann gesetzt.

%% latex-abspann.tex
%% Gemeinsamer Abspann für Korrekturansicht und Leseansicht.
%% Setzt den Schalter \ifkorrekturansicht voraus (gesetzt in den
%% einbindenden Dateien latex-korrekturansicht-abspann.tex bzw.
%% latex-leseansicht-abspann.tex).
%% ---------------------------------------------------------------

\normalsize

% Das esempio-Environment wird nur in der Leseansicht benötigt
\ifkorrekturansicht\else
\newenvironment{esempio}[3]%
{
    \vspace{1.5ex}
    \rlap{\underline{#1}}
    \par
    \setlength{\parindent}{0cm}
    \nopagebreak
    \leftskip=#2cm
    \rightskip=#3cm
}
{
    \par
}
\fi

\doendnotes{C}
\bigskip
\vfill

\clearpage

\footnotesize

\ifkorrekturansicht
  \lohead{\textsc{register}}
\fi

% theindex-Environment neu definieren ohne reledmac
\makeatletter
\renewenvironment{theindex}{%
  \ifkorrekturansicht
    \section*{\indexname}%
  \else
    \subsubsection*{Index der erwähnten Entitäten}%
  \fi
  \setlength{\parindent}{0pt}%
  \setlength{\parskip}{0pt plus 0.3pt}%
  \let\item\@idxitem
}{%
  \ifkorrekturansicht\clearpage\fi
}
\makeatother

\IfFileExists{\jobname-pw.ind}{\input{\jobname-pw.ind}}{}

% Quellenangabe nur in der Leseansicht
\ifkorrekturansicht\else
% Fallback-Definitionen, falls die .tex-Datei \titel etc. nicht gesetzt hat
\providecommand{\titel}{}
\providecommand{\editorInnen}{}
\providecommand{\dateiname}{\jobname}

\vspace{3cm}

\vfill

\footnotesize
\textsc{Quelle}: \titel. Herausgegeben von {\editorInnen}. In: \emph{Arthur Schnitzler: Briefwechsel mit Autorinnen und Autoren}.
 Digitale Edition, https://schnitzler-briefe.acdh.oeaw.ac.at/{\dateiname}.html (Stand \today)
\fi

\end{document}


      