%% latex-leseansicht-vorspann.tex
%% Vorspann für die Leseansicht.
%% Lädt die gemeinsame Datei latex-vorspann.tex mit nicht gesetztem Schalter.

\newif\ifkorrekturansicht
\korrekturansichtfalse

\input{../tex-inputs/latex-vorspann}


\section[Hugo von Hofmannsthal an Arthur Schnitzler, 9. 3. 1894]{L00304 Hugo von Hofmannsthal an Arthur Schnitzler, 9. 3. 1894}
\nopagebreak\mylabel{L00304v}
\rehead{ }\normalsize\beginnumbering\briefempfaengerindex{Schnitzler, Arthur@\textsc{Schnitzler, Arthur}!zzzHofmannsthal, Hugo von@\emph{von Hugo von Hofmannsthal}!1894-03-091@{9. 3. 1894}|(be}
\toendnotes[C]{\smallbreak\pagebreak[2]}
\correspDesc{Versand  durch Hugo von Hofmannsthal am 9. 3. 1894 in Wien
\newline{}Erhalt  durch Arthur Schnitzler im Zeitraum [9. 3. 1894
                  – 13. 3. 1894?] in Wien}\toendnotes[C]{\smallbreak}
\Standort{CUL, Schnitzler, B 43.}
\physDesc{Briefkarte, 408 Zeichen (aufgeprägtes Wappen )
\newline{}Handschrift: Bleistift, deutsche Kurrent
\newline{}Schnitzler: mit Bleistift nummeriert: »63« }
\buchAbdrucke{\weitereDrucke{Hugo von Hofmannsthal, Arthur Schnitzler: \emph{Briefwechsel}. Herausgegeben von Therese Nickl und Heinrich Schnitzler. Frankfurt am Main: \emph{S. Fischer} 1964, S. 50.} }
\pstart
           \raggedleft{}{\pb}9. III. 94.\pend
           
\pstart{}lieber Arthur!\pend\vspace{0.5em}
\pstart
           Ich möchte mit Ihnen 1.) ins Arſenal\oindex{Wien@\textbf{Wien}!III., Landstraße@\textbf{III., Landstraße}!Arsenal@\textbf{Arsenal}, \emph{Gebäude}|pw} 2.) auf den
                  Stephansthurm\oindex{Wien@\textbf{Wien}!I., Innere Stadt@\textbf{I., Innere Stadt}!Stephansdom@\textbf{Stephansdom}, \emph{Kirche}|pw} gehen.\pend
           
\pstart
           Bitte erkundigen Sie{ }ſich um die möglichen Stunden, wählen Sie dann ein paar Stunden
               und Tage, die Ihnen paſſen und{ }ſchreiben Sie mirs{ }ſogleich. Ich werde{ }ſofort
               antworten {\pb}und{ }ſo wirds
               hoffentlich zuſammengehen.\pend
           
\pstart
           Sonntag gehe ich wahrſcheinlich zu den »Nibelungen\pwindex{\textcolor{red}{\textsuperscript{XXXX indx1}}!Nibelungen@\strich\emph{Die Nibelungen}|pw}« (Loge) dann gewiſs zu Ihnen.\pend
           
\pstart
           Oder Nicht?\pend
           
\pstart
           von Herzen{\\[\baselineskip]}Ihr{\\[\baselineskip]}\spacefill\mbox{Hugo.}\pend
           \leftskip=0em{}\selectlanguage{ngerman}\endnumbering\briefempfaengerindex{Schnitzler, Arthur@\textsc{Schnitzler, Arthur}!zzzHofmannsthal, Hugo von@\emph{von Hugo von Hofmannsthal}!1894-03-091@{9. 3. 1894}|)be}\mylabel{L00304h}  \newcommand{\dateiname}{L00304}\newcommand{\titel}{Hugo von Hofmannsthal an Arthur Schnitzler, 9. 3. 1894}\newcommand{\editorInnen}{Martin Anton Müller und Gerd-Hermann Susen}%% latex-leseansicht-abspann.tex
%% Abspann für die Leseansicht.
%% Der Schalter \ifkorrekturansicht ist bereits durch den Vorspann gesetzt.

%% latex-abspann.tex
%% Gemeinsamer Abspann für Korrekturansicht und Leseansicht.
%% Setzt den Schalter \ifkorrekturansicht voraus (gesetzt in den
%% einbindenden Dateien latex-korrekturansicht-abspann.tex bzw.
%% latex-leseansicht-abspann.tex).
%% ---------------------------------------------------------------

\normalsize

% Das esempio-Environment wird nur in der Leseansicht benötigt
\ifkorrekturansicht\else
\newenvironment{esempio}[3]%
{
    \vspace{1.5ex}
    \rlap{\underline{#1}}
    \par
    \setlength{\parindent}{0cm}
    \nopagebreak
    \leftskip=#2cm
    \rightskip=#3cm
}
{
    \par
}
\fi

\doendnotes{C}
\bigskip
\vfill

\clearpage

\footnotesize

\ifkorrekturansicht
  \lohead{\textsc{register}}
\fi

% theindex-Environment neu definieren ohne reledmac
\makeatletter
\renewenvironment{theindex}{%
  \ifkorrekturansicht
    \section*{\indexname}%
  \else
    \subsubsection*{Index der erwähnten Entitäten}%
  \fi
  \setlength{\parindent}{0pt}%
  \setlength{\parskip}{0pt plus 0.3pt}%
  \let\item\@idxitem
}{%
  \ifkorrekturansicht\clearpage\fi
}
\makeatother

\IfFileExists{\jobname-pw.ind}{\input{\jobname-pw.ind}}{}

% Quellenangabe nur in der Leseansicht
\ifkorrekturansicht\else
% Fallback-Definitionen, falls die .tex-Datei \titel etc. nicht gesetzt hat
\providecommand{\titel}{}
\providecommand{\editorInnen}{}
\providecommand{\dateiname}{\jobname}

\vspace{3cm}

\vfill

\footnotesize
\textsc{Quelle}: \titel. Herausgegeben von {\editorInnen}. In: \emph{Arthur Schnitzler: Briefwechsel mit Autorinnen und Autoren}.
 Digitale Edition, https://schnitzler-briefe.acdh.oeaw.ac.at/{\dateiname}.html (Stand \today)
\fi

\end{document}


