%% latex-leseansicht-vorspann.tex
%% Vorspann für die Leseansicht.
%% Lädt die gemeinsame Datei latex-vorspann.tex mit nicht gesetztem Schalter.

\newif\ifkorrekturansicht
\korrekturansichtfalse

\input{../tex-inputs/latex-vorspann}


\section[Arthur Schnitzler an Richard Beer-Hofmann, 29. 7. 1907]{L01695 Arthur Schnitzler an Richard Beer-Hofmann, 29. 7. 1907}
\nopagebreak\mylabel{L01695v}
\rehead{ }\normalsize\beginnumbering\briefempfaengerindex{Beer-Hofmann, Richard@\textsc{Beer-Hofmann, Richard}!zzzSchnitzler, Arthur@\emph{von Arthur Schnitzler}!1907-07-291@{29. 7. 1907}|(be}
\toendnotes[C]{\smallbreak\pagebreak[2]}
\correspDesc{Versand  durch Arthur Schnitzler am 29. 7. 1907 in Welsberg-Taisten
\newline{}Erhalt  durch Richard Beer-Hofmann am 30. 7. 1907 in Maria Schutz}\toendnotes[C]{\smallbreak}
\Standort{YCGL, MSS 31.}
\physDesc{Brief, 1 Blatt, 2 Seiten, Kuvert, 1570 Zeichen (Umschlag (auf der Rückseite Fotografien von »Hôtel {\kaufmannsund} Pension Bavaria\oindex{Hotel {\kaufmannsund} Pension Bavaria@\textbf{Hotel {\kaufmannsund} Pension Bavaria}, \emph{Hotel}|pw}{ }Meran-Obermais\oindex{Obermais@\textbf{Obermais}, \emph{Bezirk}|pw}« und »Hôtel {\kaufmannsund} Pension
                                       Wildbad-Waldbrunn\oindex{Wildbad Waldbrunn@\textbf{Wildbad Waldbrunn}, \emph{Spa}|pw}, Pustertal (Tirol)\oindex{Pustertal@\textbf{Pustertal}, \emph{Tal}|pw}«.))
\newline{}Handschrift: schwarze Tinte, lateinische Kurrent
\newline{}Versand: 1) Stempel: »\nobreak{}\oindex{Wildbad Waldbrunn@\textbf{Wildbad Waldbrunn}, \emph{Spa}|pwk}Wildbad Waldbrunn, 29. Jul. 1907\nobreak{}«.   2) Stempel: »\nobreak{}\oindex{Welsberg-Taisten@\textbf{Welsberg-Taisten}, \emph{Verwaltungsgebiet}|pwk}{[}Wel{]}sb{[}erg{]}, 29. 7. 07\nobreak{}«.  3) Stempel: »\nobreak{}\oindex{Maria Schutz@\textbf{Maria Schutz}|pwk}{\pb}Maria
                                          {[}Schu{]}tz, 0\textcolor{gray}{3.} 7. 07, 8–9N\nobreak{}«. 
\newline{}Beer-Hofmann: mit blauem Buntstift den Zeitpunkt der Beantwortung
                                 festgehalten: »B. 30/VII 07« }
\buchAbdrucke{\weitereDrucke{Arthur Schnitzler, Richard Beer-Hofmann: \emph{Briefwechsel 1891–1931}. Herausgegeben von Konstanze Fliedl. Wien, Zürich: \emph{Europaverlag} 1992, S. 181–182.} }\toendnotes[C]{\smallbreak}\pstart{}{\pb}\textcolor{gray}{\textbf{HÔTEL {\kaufmannsund}
                           PENSION BAVARIA\oindex{Hotel {\kaufmannsund} Pension Bavaria@\textbf{Hotel {\kaufmannsund} Pension Bavaria}, \emph{Hotel}|pw}{ }MERAN-OBERMAIS\oindex{Obermais@\textbf{Obermais}, \emph{Bezirk}|pw}}}\pend{}\pstart{}\textcolor{gray}{\textbf{HÔTEL {\kaufmannsund} PENSION WILDBAD WALDBRUNN\oindex{Wildbad Waldbrunn@\textbf{Wildbad Waldbrunn}, \emph{Spa}|pw}, PUSTERTAL\oindex{Pustertal@\textbf{Pustertal}, \emph{Tal}|pw}}}\pend{}\pstart{}\textcolor{gray}{\textbf{BESITZER: JOS. BÖHM\pwindex{Böhm, Josef @\textsc{Böhm, Josef}, \emph{Hotelbesitzer}|pw}}}\pend{}{\bigskip}\pstart{}Dr. Richard Beer-Hofmann\pend{}\pstart{}Maria Schutz\oindex{Maria Schutz@\textbf{Maria Schutz}|pw}\pend{}\pstart{}am Semmering\oindex{Semmering@\textbf{Semmering}, \emph{Verwaltungsgebiet}|pw}\pend{}{\bigskip}\vspace{1em}
\pstart
           {\pb}\textcolor{gray}{\textbf{Telegramm-Adresse: Böhm – Welsberg}}\pend
           
\pstart
           \textcolor{gray}{\textbf{Hôtel {\kaufmannsund} Pension Wildbad Waldbrunn\oindex{Wildbad Waldbrunn@\textbf{Wildbad Waldbrunn}, \emph{Spa}|pw}}}\pend
           
\pstart
           \textcolor{gray}{\textbf{bei Welsberg\oindex{Welsberg-Taisten@\textbf{Welsberg-Taisten}, \emph{Verwaltungsgebiet}|pw}
                     (Eilzughaltestelle)}}\pend
           
\pstart
           \textcolor{gray}{\textbf{1150 M. \textsuperscript{ü}/Meer.\hspace*{1.5em}Hochpusterthal\oindex{Pustertal@\textbf{Pustertal}, \emph{Tal}|pw} (Tirol\oindex{Tirol@\textbf{Tirol}, \emph{Land}|pw})}}\pend
           
\pstart
           \textcolor{gray}{\textbf{Heilkräftiges altbekanntes Bad in prachtvoller Lage.}}\pend
           
\pstart
           \textcolor{gray}{\textbf{Ausgezeichnete Trinkquelle.}}\pend
           
\pstart
           \textcolor{gray}{\textbf{70 mit allem Comfort eingerichtete Zimmer.}}\pend
           
\pstart
           \raggedleft{}\textcolor{gray}{\textbf{Waldbrunn\oindex{Welsberg-Taisten@\textbf{Welsberg-Taisten}, \emph{Verwaltungsgebiet}|pw}, den}}{ }29. 7. \textcolor{gray}{\textbf{190}}7\pend
           
\pstart{}lieber Richard,\pend\vspace{0.5em}
\pstart
           im Lidohotel\oindex{Palast Hotel Lido@\textbf{Palast Hotel Lido}, \emph{Hotel}|pw}, Riva\oindex{Riva del Garda@\textbf{Riva del Garda}, \emph{Hauptstadt}|pw}, war ich ein oder \label{K_L01695-1v}\edtext{zwei
                  Tage}{\lemma{\textnormal{\emph{zwei
                  Tage}}}\Cendnote{\textnormal{Siehe A. S.: \emph{Tagebuch}, 18. 8. 1903.
               }}}\label{K_L01695-1}, vor 4 Jahren, Ende August, mit Goldmann\pwindex{Goldmann, Paul 31.\,1.\,1865 Breslau – 25.\,9.\,1935 Wien@\textsc{Goldmann, Paul} (31.\,1.\,1865 Breslau – 25.\,9.\,1935 Wien), \emph{Schriftsteller, Journalist}|pw}; es war ein comfortables, sogar elegantes Hotel, das
               Essen damals mäßig; eine Badeanstalt we{\geminationn} ich mich recht
               erinnere im Park. Würde wohl wieder dort absteigen. Gerühmt wurde mir s. Z. (weiß
               nicht mehr von wem) »Hotel Riva\oindex{Grand Hotel Riva@\textbf{Grand Hotel Riva}, \emph{Hotel}|pw}«. Hotel du lac\oindex{Hotel du Lac@\textbf{Hotel du Lac}, \emph{Hotel}|pw} ke{\geminationn}
               ich nicht. Torbole\oindex{Torbole sul Garda@\textbf{Torbole sul Garda}, \emph{Hauptstadt}|pw} kenn ich nur ausflugs- nicht
               aufenthaltsweise.\pend
           
\pstart
           Auch wir haben Dolomitenstraßenpläne\oindex{Große Dolomitenstraße@\textbf{Große Dolomitenstraße}, \emph{Straße}|pw}, Pordoi\oindex{Pordoijoch@\textbf{Pordoijoch}, \emph{Berg}|pw}, Karersee\oindex{Karersee@\textbf{Karersee}, \emph{See}|pw} – wollen in Meran\oindex{Meran@\textbf{Meran}, \emph{Hauptstadt}|pw} einige Zeit
               verweilen, vielleicht auch am Gardasee\oindex{Lago di Garda@\textbf{Lago di Garda}, \emph{See}|pw}.
               Vielleicht fügt es sich, dass wir zusa{\geminationm}en paßwandern?
               Wir sind hier (denk ich) bis nach dem 20–25. etwa – wenn
               Sie von Ihrem {[}n{]}\oindex{Kärnten@\textbf{Kärnten}, \emph{Land}|pw}see
               kommen, könnten Sie hier, in Welsberg\oindex{Welsberg-Taisten@\textbf{Welsberg-Taisten}, \emph{Verwaltungsgebiet}|pw} ein paar
               Tage {\pb}Station machen; vielleicht daß wir gemeinsam
               aufbrechen, südwärts –?\pend
           
\pstart
           Hugo\pwindex{Hofmannsthal, Hugo von 1.\,2.\,1874 Wien – 15.\,7.\,1929 Rodaun@\textsc{Hofmannsthal, Hugo von} (1.\,2.\,1874 Wien – 15.\,7.\,1929 Rodaun), \emph{Schriftsteller}|pw}\pwindex{Hofmannsthal, Gertrude von 16.\,3.\,1880 Wien – 9.\,11.\,1959 Paddington@\textsc{Hofmannsthal, Gertrude von} (16.\,3.\,1880 Wien – 9.\,11.\,1959 Paddington)|pw}’s waren 10 Tage da, wie Sie
               schon wissen; es sind wirklich schattenlos angenehme gewesen. – Wir fühlen uns alle
               hier recht wohl; ich arbeite nicht unfleißig und hoffe mit dem Roman\pwindex{Schnitzler, Arthur 15.\,5.\,1862 Wien – 21.\,10.\,1931 ebd.@\textsc{Schnitzler, Arthur} (15.\,5.\,1862 Wien – 21.\,10.\,1931 ebd.), \emph{Schriftsteller, Mediziner}!Weg ins Freie. Roman@\strich\emph{Der Weg ins Freie. Roman}|pwv}, eh wir von hier abfahren, recht,
               sehr weit zu gedeihen. Waldspaziergänge lassen sich täglich neue entdecken; das Essen
               ist gut, mein Zimmer ein von mir längst gesuchtes Ideal; Balkon, Blick ins freie, vom
               Hotel nichts zu sehn (so vorgebaut); die Gesellschaft indifferent, da man sich
               absolut nicht umeinander kümmert.\pend
           
\pstart
           Lassen Sie sehr bald von sich hören; dass wir Hoffnung haben, Sie beide unter
                  Reisehi{\geminationm}el, und wenn sich alles gut fügt, für nicht
               gar zu kurze Zeit wiederzusehen, freut uns riesig.\pend
           
\pstart
           Von Herzen, mit Grüßen von Höhe zu Höhe, Haus zu Haus\pend
           
\pstart
           Ihr{\\[\baselineskip]}\spacefill\mbox{Arthur}\pend
           \leftskip=0em{}\selectlanguage{ngerman}\endnumbering\briefempfaengerindex{Beer-Hofmann, Richard@\textsc{Beer-Hofmann, Richard}!zzzSchnitzler, Arthur@\emph{von Arthur Schnitzler}!1907-07-291@{29. 7. 1907}|)be}\mylabel{L01695h}  \newcommand{\dateiname}{L01695}\newcommand{\titel}{Arthur Schnitzler an Richard Beer-Hofmann, 29. 7. 1907}\newcommand{\editorInnen}{Martin Anton Müller und Gerd-Hermann Susen}%% latex-leseansicht-abspann.tex
%% Abspann für die Leseansicht.
%% Der Schalter \ifkorrekturansicht ist bereits durch den Vorspann gesetzt.

%% latex-abspann.tex
%% Gemeinsamer Abspann für Korrekturansicht und Leseansicht.
%% Setzt den Schalter \ifkorrekturansicht voraus (gesetzt in den
%% einbindenden Dateien latex-korrekturansicht-abspann.tex bzw.
%% latex-leseansicht-abspann.tex).
%% ---------------------------------------------------------------

\normalsize

% Das esempio-Environment wird nur in der Leseansicht benötigt
\ifkorrekturansicht\else
\newenvironment{esempio}[3]%
{
    \vspace{1.5ex}
    \rlap{\underline{#1}}
    \par
    \setlength{\parindent}{0cm}
    \nopagebreak
    \leftskip=#2cm
    \rightskip=#3cm
}
{
    \par
}
\fi

\doendnotes{C}
\bigskip
\vfill

\clearpage

\footnotesize

\ifkorrekturansicht
  \lohead{\textsc{register}}
\fi

% theindex-Environment neu definieren ohne reledmac
\makeatletter
\renewenvironment{theindex}{%
  \ifkorrekturansicht
    \section*{\indexname}%
  \else
    \subsubsection*{Index der erwähnten Entitäten}%
  \fi
  \setlength{\parindent}{0pt}%
  \setlength{\parskip}{0pt plus 0.3pt}%
  \let\item\@idxitem
}{%
  \ifkorrekturansicht\clearpage\fi
}
\makeatother

\IfFileExists{\jobname-pw.ind}{\input{\jobname-pw.ind}}{}

% Quellenangabe nur in der Leseansicht
\ifkorrekturansicht\else
% Fallback-Definitionen, falls die .tex-Datei \titel etc. nicht gesetzt hat
\providecommand{\titel}{}
\providecommand{\editorInnen}{}
\providecommand{\dateiname}{\jobname}

\vspace{3cm}

\vfill

\footnotesize
\textsc{Quelle}: \titel. Herausgegeben von {\editorInnen}. In: \emph{Arthur Schnitzler: Briefwechsel mit Autorinnen und Autoren}.
 Digitale Edition, https://schnitzler-briefe.acdh.oeaw.ac.at/{\dateiname}.html (Stand \today)
\fi

\end{document}


