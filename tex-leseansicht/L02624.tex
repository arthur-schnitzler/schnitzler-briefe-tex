%% latex-korrekturansicht-vorspann.tex
%% Vorspann für die Korrekturansicht.
%% Lädt die gemeinsame Datei latex-vorspann.tex mit gesetztem Schalter.

\newif\ifkorrekturansicht
\korrekturansichttrue

\input{../tex-inputs/latex-vorspann}


\section[Paul Goldmann an Arthur Schnitzler, 4. 12. 1894]{L02624 Paul Goldmann an Arthur Schnitzler, 4. 12. 1894}
\nopagebreak\mylabel{L02624v}
\rehead{ }\normalsize\beginnumbering\briefempfaengerindex{Schnitzler, Arthur@\textsc{Schnitzler, Arthur}!zzzGoldmann, Paul@\emph{von Paul Goldmann}!1894-12-041@{04. 12. 1894}|(be}
\toendnotes[C]{\smallbreak\pagebreak[2]}\Standort{DLA, A:Schnitzler, HS.NZ85.1.3164.}
\physDesc{Brief, 1 Blatt, 1 Seite, 323 Zeichen
\newline{}Handschrift: schwarze Tinte, deutsche Kurrent
\newline{}Schnitzler: mit Bleistift die Jahreszahl »94« vermerkt }\toendnotes[C]{\smallbreak}
\pstart
           \raggedleft{}{\pb}Paris\oindex{Paris@\textbf{Paris}, \emph{P.PPLC}|pw},
                  4. December.\pend
           
\pstart\center{}Mein lieber Freund,\pend\vspace{0.5em}
\pstart
           Die »Frkf. Ztg.\pwindex{Frankfurter Zeitung@\emph{Frankfurter Zeitung}|pw}« worin Dein \label{K_L02624-1v}\edtext{Buch\pwindex{Sterben. Novelle@\emph{Sterben. Novelle}|pwv}{ }beſprochen\pwindex{Belletristische Rundschau@\emph{Belletristische Rundschau}|pwv}}{\lemma{\textnormal{\emph{Buch beſprochen}}}\Cendnote{\textnormal{J. Schwarz\pwindex{Schwarz, J. @\textsc{Schwarz, J.}, \emph{Journalist/Journalistin}|pwk}: \emph{Belletristische Rundschau}\pwindex{Belletristische Rundschau@\emph{Belletristische Rundschau}|pwk}. In: \emph{Frankfurter Zeitung}\pwindex{Frankfurter Zeitung@\emph{Frankfurter Zeitung}|pwk}, Nr. 336, 4. 12. 1894,
                     S. 1–3.}}}\label{K_L02624-1} worden, haſt Du gewiß ſchon geſehen. Der Sicherheit
               halber \label{K_L02624-2v}\edtext{ſchicke ich ſie Dir zu}{\lemma{\textnormal{\emph{ſchicke ich ſie Dir zu}}}\Cendnote{\textnormal{Beilage nicht erhalten}}}\label{K_L02624-2}. Schreib’, bitte, eine \label{K_L02624-3v}\edtext{Zeile an meinen Onkel\pwindex{Mamroth, Fedor 21.02.1851 – 25.06.1907@\textsc{Mamroth, Fedor} (21.02.1851 – 25.06.1907), \emph{Journalist/Journalistin, Kritiker/Kritikerin}|pwv}}{\lemma{\textnormal{\emph{Zeile an meinen Onkel}}}\Cendnote{\textnormal{Siehe Arthur Schnitzler an Fedor Mamroth, 7. 12. 1894.
               }}}\label{K_L02624-3}, der diesmal beſonders brav geweſen iſt.\pend
           
\pstart
           Wie gehts Dir? Und wann höre ich wieder etwas von Dir?\pend
           
\pstart
           In Treue{\\[\baselineskip]}Dein{\\[\baselineskip]}\spacefill\mbox{Paul Goldmann\textcolor{gray}{.}}\pend
           \leftskip=0em{}\selectlanguage{ngerman}\endnumbering\briefempfaengerindex{Schnitzler, Arthur@\textsc{Schnitzler, Arthur}!zzzGoldmann, Paul@\emph{von Paul Goldmann}!1894-12-041@{04. 12. 1894}|)be}\mylabel{L02624h}  \normalsize

\doendnotes{C}
\bigskip
\vfill

\clearpage

\footnotesize

\lohead{\textsc{register}}

% Definiere theindex-Environment komplett neu ohne reledmac
\makeatletter
\renewenvironment{theindex}{%
  \section*{\indexname}%
  \setlength{\parindent}{0pt}%
  \setlength{\parskip}{0pt plus 0.3pt}%
  \let\item\@idxitem
}{%
  \clearpage
}
\makeatother

\IfFileExists{\jobname-pw.ind}{\input{\jobname-pw.ind}}{}

\end{document}

      