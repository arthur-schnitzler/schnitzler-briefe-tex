%% latex-leseansicht-vorspann.tex
%% Vorspann für die Leseansicht.
%% Lädt die gemeinsame Datei latex-vorspann.tex mit nicht gesetztem Schalter.

\newif\ifkorrekturansicht
\korrekturansichtfalse

\input{../tex-inputs/latex-vorspann}

\begin{center}
            \textcolor{red}{ENTWURF, NICHT FERTIG KORRIGIERT}
                      \end{center}
            
               \section[Paul Goldmann an Arthur Schnitzler, 3. 11. 1894]{ Paul Goldmann an Arthur Schnitzler, 3. 11. 1894}\nopagebreak\mylabel{v}\rehead{ }\begin{ledgroupsized}[t]{13cm}\normalsize\beginnumbering\briefempfaengerindex{Schnitzler, Arthur@\textsc{Schnitzler, Arthur}!zzzGoldmann, Paul@\emph{von Paul Goldmann}!1894-11-281@{28. 11. 1894}|(be} \toendnotes[C]{\smallbreak\pagebreak[2]} \Standort{DLA, A:Schnitzler, HS.NZ85.1.3164.}
\physDesc{Brief, 1 Blatt, 1 Seite
\newline{}Handschrift: schwarze Tinte, deutsche Kurrent
\newline{}Schnitzler: mit Bleistift die Jahreszahl »94«
                                 vermerkt }\toendnotes[C]{\smallbreak}\pstart
           \raggedleft{}4. December.\pend
           \pstart\center{}Mein lieber Freund,\pend\pstart
           die »Frkf. Ztg.\orgindex{Frankfurter Zeitung@Frankfurter Zeitung|pw}« worin Dein \label{K_L02624-2v}\edtext{Buch\pwindex{Schnitzler, Arthur 15.05.1862 – 21.10.1931@\textsc{Schnitzler, Arthur} (15.05.1862 – 21.10.1931), \emph{Schriftsteller, Mediziner}!Sterben. Novelle1.10.1894 – 1.12.1894@\strich\emph{Sterben. Novelle} {[}1.10.1894 – 1.12.1894{]}|pwv}{ }beſprochen\pwindex{Schwarz, J. @\textsc{Schwarz, J.}, \emph{Journalist/Journalistin}!Belletristische Rundschau4.12.1894 – 4.12.1894@\strich\emph{Belletristische Rundschau} {[}4.12.1894 – 4.12.1894{]}|pwv}}{\lemma{\textnormal{\emph{Buch beſprochen}}}\Cendnote{\textnormal{J. Schwarz\pwindex{Schwarz, J. @\textsc{Schwarz, J.}, \emph{Journalist/Journalistin}|pwk}: \emph{Belletristische Rundschau}\pwindex{Schwarz, J. @\textsc{Schwarz, J.}, \emph{Journalist/Journalistin}!Belletristische Rundschau4.12.1894 – 4.12.1894@\strich\emph{Belletristische Rundschau} {[}4.12.1894 – 4.12.1894{]}|pwk}. In: \emph{Frankfurter Zeitung}\pwindex{Frankfurter Zeitung1856 – 1943@\emph{Frankfurter Zeitung}|pwk}, Nr. 336, 4. 12. 1894,
                     S. 1–3.}}}\label{K_L02624-2h} worden, haſt Du gewiß ſchon geſehen. Der Sicherheit
               halber ſchicke ich ſie Dir zu. Schreib’, bitte, eine \label{K_L02624-1v}\edtext{Zeile an meinen Onkel\pwindex{Mamroth, Fedor 21.02.1851 – 25.06.1907@\textsc{Mamroth, Fedor} (21.02.1851 – 25.06.1907), \emph{Journalist, Kritiker}|pwv}}{\lemma{\textnormal{\emph{Zeile an meinen Onkel}}}\Cendnote{\textnormal{siehe Arthur Schnitzler an Fedor Mamroth, 7. 12. 1894}}}\label{K_L02624-1h}, der diesmal beſonders
               brav geweſen iſt.\pend
           \pstart
           Wie gehts Dir? Und wann höre ich wieder etwas von Dir?\pend
           \pstart
           In Treue{\\[\baselineskip]}Dein{\\[\baselineskip]}\spacefill\mbox{Paul Goldmann}\pend
           \leftskip=0em{}          \endnumbering\briefempfaengerindex{Schnitzler, Arthur@\textsc{Schnitzler, Arthur}!zzzGoldmann, Paul@\emph{von Paul Goldmann}!1894-11-281@{28. 11. 1894}|)be}\mylabel{h}\end{ledgroupsized}  \newcommand{\dateiname}{L02624}\newcommand{\titel}{Paul Goldmann an Arthur Schnitzler, 3. 11. 1894}\newcommand{\editorInnen}{Martin Anton Müller und Laura Untner}
            \footnotesize
\begin{ledgroupsized}[t]{11.5cm}
\doendnotes{C}
\end{ledgroupsized}
         %% latex-leseansicht-abspann.tex
%% Abspann für die Leseansicht.
%% Der Schalter \ifkorrekturansicht ist bereits durch den Vorspann gesetzt.

%% latex-abspann.tex
%% Gemeinsamer Abspann für Korrekturansicht und Leseansicht.
%% Setzt den Schalter \ifkorrekturansicht voraus (gesetzt in den
%% einbindenden Dateien latex-korrekturansicht-abspann.tex bzw.
%% latex-leseansicht-abspann.tex).
%% ---------------------------------------------------------------

\normalsize

% Das esempio-Environment wird nur in der Leseansicht benötigt
\ifkorrekturansicht\else
\newenvironment{esempio}[3]%
{
    \vspace{1.5ex}
    \rlap{\underline{#1}}
    \par
    \setlength{\parindent}{0cm}
    \nopagebreak
    \leftskip=#2cm
    \rightskip=#3cm
}
{
    \par
}
\fi

\doendnotes{C}
\bigskip
\vfill

\clearpage

\footnotesize

\ifkorrekturansicht
  \lohead{\textsc{register}}
\fi

% theindex-Environment neu definieren ohne reledmac
\makeatletter
\renewenvironment{theindex}{%
  \ifkorrekturansicht
    \section*{\indexname}%
  \else
    \subsubsection*{Index der erwähnten Entitäten}%
  \fi
  \setlength{\parindent}{0pt}%
  \setlength{\parskip}{0pt plus 0.3pt}%
  \let\item\@idxitem
}{%
  \ifkorrekturansicht\clearpage\fi
}
\makeatother

\IfFileExists{\jobname-pw.ind}{\input{\jobname-pw.ind}}{}

% Quellenangabe nur in der Leseansicht
\ifkorrekturansicht\else
% Fallback-Definitionen, falls die .tex-Datei \titel etc. nicht gesetzt hat
\providecommand{\titel}{}
\providecommand{\editorInnen}{}
\providecommand{\dateiname}{\jobname}

\vspace{3cm}

\vfill

\footnotesize
\textsc{Quelle}: \titel. Herausgegeben von {\editorInnen}. In: \emph{Arthur Schnitzler: Briefwechsel mit Autorinnen und Autoren}.
 Digitale Edition, https://schnitzler-briefe.acdh.oeaw.ac.at/{\dateiname}.html (Stand \today)
\fi

\end{document}


      