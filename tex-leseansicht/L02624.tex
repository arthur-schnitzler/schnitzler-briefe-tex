%% latex-leseansicht-vorspann.tex
%% Vorspann für die Leseansicht.
%% Lädt die gemeinsame Datei latex-vorspann.tex mit nicht gesetztem Schalter.

\newif\ifkorrekturansicht
\korrekturansichtfalse

\input{../tex-inputs/latex-vorspann}


\section[Paul Goldmann an Arthur Schnitzler, 4. 12. 1894]{L02624 Paul Goldmann an Arthur Schnitzler, 4. 12. 1894}
\nopagebreak\mylabel{L02624v}
\rehead{ }\normalsize\beginnumbering\briefempfaengerindex{Schnitzler, Arthur@\textsc{Schnitzler, Arthur}!zzzGoldmann, Paul@\emph{von Paul Goldmann}!1894-12-041@{04. 12. 1894}|(be}
\toendnotes[C]{\smallbreak\pagebreak[2]}
\correspDesc{Versand  durch Paul Goldmann am 04. 12. 1894 in Paris
\newline{}Erhalt  durch Arthur Schnitzler im Zeitraum [5. 12. 1894
                  – 9. 12. 1894?] in Wien}\toendnotes[C]{\smallbreak}
\Standort{DLA, A:Schnitzler, HS.NZ85.1.3164.}
\physDesc{Brief, 1 Blatt, 1 Seite, 323 Zeichen
\newline{}Handschrift: schwarze Tinte, deutsche Kurrent
\newline{}Schnitzler: mit Bleistift die Jahreszahl »94« vermerkt }\toendnotes[C]{\smallbreak}
\pstart
           \raggedleft{}{\pb}Paris\oindex{Paris@\textbf{Paris}, \emph{Hauptstadt}|pw},
                  4. December.\pend
           
\pstart\center{}Mein lieber Freund,\pend\vspace{0.5em}
\pstart
           Die »Frkf. Ztg.\pwindex{Frankfurter Zeitung@\emph{Frankfurter Zeitung}|pw}« worin Dein \label{K_L02624-1v}\edtext{Buch\pwindex{Schnitzler, Arthur 15.\,5.\,1862 Wien – 21.\,10.\,1931 ebd.@\textsc{Schnitzler, Arthur} (15.\,5.\,1862 Wien – 21.\,10.\,1931 ebd.), \emph{Schriftsteller, Mediziner}!Sterben. Novelle@\strich\emph{Sterben. Novelle}|pwv}{ }beſprochen\pwindex{Schwarz, J. @\textsc{Schwarz, J.}, \emph{Journalist/Journalistin}!Belletristische Rundschau@\strich\emph{Belletristische Rundschau}|pwv}}{\lemma{\textnormal{\emph{Buch besprochen}}}\Cendnote{\textnormal{J. Schwarz\pwindex{Schwarz, J. @\textsc{Schwarz, J.}, \emph{Journalist/Journalistin}|pwk}: \emph{Belletristische Rundschau}\pwindex{Schwarz, J. @\textsc{Schwarz, J.}, \emph{Journalist/Journalistin}!Belletristische Rundschau@\strich\emph{Belletristische Rundschau}|pwk}. In: \emph{Frankfurter Zeitung}\pwindex{Frankfurter Zeitung@\emph{Frankfurter Zeitung}|pwk}, Nr. 336, 4. 12. 1894,
                     S. 1–3.}}}\label{K_L02624-1} worden, haſt Du gewiß{ }ſchon geſehen. Der Sicherheit
               halber \label{K_L02624-2v}\edtext{ſchicke ich{ }ſie Dir zu}{\lemma{\textnormal{\emph{schicke ich sie Dir zu}}}\Cendnote{\textnormal{Beilage nicht erhalten}}}\label{K_L02624-2}. Schreib’, bitte, eine \label{K_L02624-3v}\edtext{Zeile an meinen Onkel\pwindex{Mamroth, Fedor 21.\,2.\,1851 Breslau – 25.\,6.\,1907 Frankfurt am Main@\textsc{Mamroth, Fedor} (21.\,2.\,1851 Breslau – 25.\,6.\,1907 Frankfurt am Main), \emph{Journalist, Kritiker}|pwv}}{\lemma{\textnormal{\emph{Zeile an meinen Onkel}}}\Cendnote{\textnormal{Siehe XXXX Auszeichnungsfehler: Dokument L00409 nicht gefunden.
               }}}\label{K_L02624-3}, der diesmal beſonders brav geweſen iſt.\pend
           
\pstart
           Wie gehts Dir? Und wann höre ich wieder etwas von Dir?\pend
           
\pstart
           In Treue{\\[\baselineskip]}Dein{\\[\baselineskip]}\spacefill\mbox{Paul Goldmann\textcolor{gray}{.}}\pend
           \leftskip=0em{}\selectlanguage{ngerman}\endnumbering\briefempfaengerindex{Schnitzler, Arthur@\textsc{Schnitzler, Arthur}!zzzGoldmann, Paul@\emph{von Paul Goldmann}!1894-12-041@{04. 12. 1894}|)be}\mylabel{L02624h}  \newcommand{\dateiname}{L02624}\newcommand{\titel}{Paul Goldmann an Arthur Schnitzler, 4. 12. 1894}\newcommand{\editorInnen}{Martin Anton Müller und Laura Untner}%% latex-leseansicht-abspann.tex
%% Abspann für die Leseansicht.
%% Der Schalter \ifkorrekturansicht ist bereits durch den Vorspann gesetzt.

%% latex-abspann.tex
%% Gemeinsamer Abspann für Korrekturansicht und Leseansicht.
%% Setzt den Schalter \ifkorrekturansicht voraus (gesetzt in den
%% einbindenden Dateien latex-korrekturansicht-abspann.tex bzw.
%% latex-leseansicht-abspann.tex).
%% ---------------------------------------------------------------

\normalsize

% Das esempio-Environment wird nur in der Leseansicht benötigt
\ifkorrekturansicht\else
\newenvironment{esempio}[3]%
{
    \vspace{1.5ex}
    \rlap{\underline{#1}}
    \par
    \setlength{\parindent}{0cm}
    \nopagebreak
    \leftskip=#2cm
    \rightskip=#3cm
}
{
    \par
}
\fi

\doendnotes{C}
\bigskip
\vfill

\clearpage

\footnotesize

\ifkorrekturansicht
  \lohead{\textsc{register}}
\fi

% theindex-Environment neu definieren ohne reledmac
\makeatletter
\renewenvironment{theindex}{%
  \ifkorrekturansicht
    \section*{\indexname}%
  \else
    \subsubsection*{Index der erwähnten Entitäten}%
  \fi
  \setlength{\parindent}{0pt}%
  \setlength{\parskip}{0pt plus 0.3pt}%
  \let\item\@idxitem
}{%
  \ifkorrekturansicht\clearpage\fi
}
\makeatother

\IfFileExists{\jobname-pw.ind}{\input{\jobname-pw.ind}}{}

% Quellenangabe nur in der Leseansicht
\ifkorrekturansicht\else
% Fallback-Definitionen, falls die .tex-Datei \titel etc. nicht gesetzt hat
\providecommand{\titel}{}
\providecommand{\editorInnen}{}
\providecommand{\dateiname}{\jobname}

\vspace{3cm}

\vfill

\footnotesize
\textsc{Quelle}: \titel. Herausgegeben von {\editorInnen}. In: \emph{Arthur Schnitzler: Briefwechsel mit Autorinnen und Autoren}.
 Digitale Edition, https://schnitzler-briefe.acdh.oeaw.ac.at/{\dateiname}.html (Stand \today)
\fi

\end{document}


