%% latex-leseansicht-vorspann.tex
%% Vorspann für die Leseansicht.
%% Lädt die gemeinsame Datei latex-vorspann.tex mit nicht gesetztem Schalter.

\newif\ifkorrekturansicht
\korrekturansichtfalse

\input{../tex-inputs/latex-vorspann}


         
         \renewcommand{\erwaehntePersonen}{Personen: Ellen Delp}
         \renewcommand{\erwaehnteOrte}{Orte: Wien}
         \renewcommand{\erwaehnteWerke}{}
               \section[Lou Andreas-Salomé an Arthur Schnitzler, {[}vor dem 15. 11. 1912?{]}]{ Lou Andreas-Salomé an Arthur Schnitzler, {[}vor dem
               15. 11. 1912?{]}}\nopagebreak\mylabel{v}\rehead{ }\begin{ledgroupsized}[t]{13cm}\normalsize\beginnumbering \toendnotes[C]{\smallbreak\pagebreak[2]} \Standort{CUL, Schnitzler, B 3.}
\physDesc{Briefkarte, 313 Zeichen
\newline{}Handschrift: schwarze Tinte, deutsche Kurrent
\newline{}Schnitzler: 1) mit Bleistift beschriftet: »\substVorne{}\textsuperscript{A}\substDazwischen{}L\substHinten{}ou Salom\textcolor{gray}{e} Andr.«  2) mit rotem Buntstift eine Unterstreichung}\toendnotes[C]{\smallbreak}\pstart{}{\pb}Lieber Herr Doktor,\pend\pstart
           ich bin mit einem jungen \label{K_L02097_1v}\edtext{Mädchen\pwindex{Delp, Ellen 09.02.1890 – 25.02.1990@\textsc{Delp, Ellen} (09.02.1890 – 25.02.1990), \emph{Schriftstellerin}|pwv}}{\lemma{\textnormal{\emph{Mädchen}}}\Cendnote{\textnormal{Die Datierung basiert auf der Annahme,
                  dass es sich bei ihr um Ellen Delp\pwindex{Delp, Ellen 09.02.1890 – 25.02.1990@\textsc{Delp, Ellen} (09.02.1890 – 25.02.1990), \emph{Schriftstellerin}|pwk} handelt.
                     Vgl. A. S.: \emph{Tagebuch}, 15. 11. 1912.}}}\label{K_L02097_1h} hier
               in Wien\oindex{Wien@\textbf{Wien}|pw}, und würde mich freuen, wenn wir Ihnen
               einen Beſuch abſtatten dürften, falls dies keine Störung für Sie bedeutet?
                  Dienstag oder Donnerstag oder Freitag
               brauchten Sie nur die Stunde zu beſtimmen, wir halten uns frei.\pend
           \pstart
           Mit herzlichem Gruß Ihre{\\[\baselineskip]}\spacefill\mbox{Lou Andreas-Salomé}\pend
           \leftskip=0em{}
         
         \endnumbering\mylabel{h}\end{ledgroupsized}  \newcommand{\dateiname}{L02097}\newcommand{\titel}{Lou Andreas-Salomé an Arthur Schnitzler, [vor dem 15. 11. 1912?]}\newcommand{\editorInnen}{Martin Anton Müller und Gerd-Hermann Susen}%% latex-leseansicht-abspann.tex
%% Abspann für die Leseansicht.
%% Der Schalter \ifkorrekturansicht ist bereits durch den Vorspann gesetzt.

%% latex-abspann.tex
%% Gemeinsamer Abspann für Korrekturansicht und Leseansicht.
%% Setzt den Schalter \ifkorrekturansicht voraus (gesetzt in den
%% einbindenden Dateien latex-korrekturansicht-abspann.tex bzw.
%% latex-leseansicht-abspann.tex).
%% ---------------------------------------------------------------

\normalsize

% Das esempio-Environment wird nur in der Leseansicht benötigt
\ifkorrekturansicht\else
\newenvironment{esempio}[3]%
{
    \vspace{1.5ex}
    \rlap{\underline{#1}}
    \par
    \setlength{\parindent}{0cm}
    \nopagebreak
    \leftskip=#2cm
    \rightskip=#3cm
}
{
    \par
}
\fi

\doendnotes{C}
\bigskip
\vfill

\clearpage

\footnotesize

\ifkorrekturansicht
  \lohead{\textsc{register}}
\fi

% theindex-Environment neu definieren ohne reledmac
\makeatletter
\renewenvironment{theindex}{%
  \ifkorrekturansicht
    \section*{\indexname}%
  \else
    \subsubsection*{Index der erwähnten Entitäten}%
  \fi
  \setlength{\parindent}{0pt}%
  \setlength{\parskip}{0pt plus 0.3pt}%
  \let\item\@idxitem
}{%
  \ifkorrekturansicht\clearpage\fi
}
\makeatother

\IfFileExists{\jobname-pw.ind}{\input{\jobname-pw.ind}}{}

% Quellenangabe nur in der Leseansicht
\ifkorrekturansicht\else
% Fallback-Definitionen, falls die .tex-Datei \titel etc. nicht gesetzt hat
\providecommand{\titel}{}
\providecommand{\editorInnen}{}
\providecommand{\dateiname}{\jobname}

\vspace{3cm}

\vfill

\footnotesize
\textsc{Quelle}: \titel. Herausgegeben von {\editorInnen}. In: \emph{Arthur Schnitzler: Briefwechsel mit Autorinnen und Autoren}.
 Digitale Edition, https://schnitzler-briefe.acdh.oeaw.ac.at/{\dateiname}.html (Stand \today)
\fi

\end{document}


      