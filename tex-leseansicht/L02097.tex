%% latex-korrekturansicht-vorspann.tex
%% Vorspann für die Korrekturansicht.
%% Lädt die gemeinsame Datei latex-vorspann.tex mit gesetztem Schalter.

\newif\ifkorrekturansicht
\korrekturansichttrue

\input{../tex-inputs/latex-vorspann}


\section[Lou Andreas-Salomé an Arthur Schnitzler, {[}vor dem 15. 11. 1912?{]}]{L02097 Lou Andreas-Salomé an Arthur Schnitzler, {[}vor dem
               15. 11. 1912?{]}}
\nopagebreak\mylabel{L02097v}
\rehead{ }\normalsize\beginnumbering\briefempfaengerindex{Schnitzler, Arthur@\textsc{Schnitzler, Arthur}!zzzAndreas-Salome, Lou@\emph{von Lou Andreas-Salomé}!1912-11-142@{{[}vor dem 15. 11. 1912?{]}}|(be}
\toendnotes[C]{\smallbreak\pagebreak[2]}\Standort{CUL, Schnitzler, B 3.}
\physDesc{Briefkarte, 313 Zeichen
\newline{}Handschrift: schwarze Tinte, deutsche Kurrent
\newline{}Schnitzler: 1) mit Bleistift beschriftet: »\substVorne{}\textsuperscript{A}\substDazwischen{}L\substHinten{}ou Salom\textcolor{gray}{e} Andr.«  2) mit rotem Buntstift eine Unterstreichung}\toendnotes[C]{\smallbreak}
\pstart{}{\pb}Lieber Herr Doktor,\pend\vspace{0.5em}
\pstart
           ich bin mit einem jungen \label{K_L02097-1v}\edtext{Mädchen\pwindex{Delp, Ellen 09.02.1890 – 25.02.1990@\textsc{Delp, Ellen} (09.02.1890 – 25.02.1990), \emph{Schriftsteller/Schriftstellerin}|pwv}}{\lemma{\textnormal{\emph{Mädchen}}}\Cendnote{\textnormal{Die Datierung basiert auf der Annahme,
                  dass es sich bei der Begleitung um Ellen Delp\pwindex{Delp, Ellen 09.02.1890 – 25.02.1990@\textsc{Delp, Ellen} (09.02.1890 – 25.02.1990), \emph{Schriftsteller/Schriftstellerin}|pwk} handelt,
                     vgl. A. S.: \emph{Tagebuch}, 15. 11. 1912.}}}\label{K_L02097-1} hier
               in Wien\oindex{Wien@\textbf{Wien}, \emph{A.ADM2}|pw}, und würde mich freuen, wenn wir Ihnen
               einen Beſuch abſtatten dürften, falls dies keine Störung für Sie bedeutet?
                  Dienstag oder Donnerstag oder Freitag
               brauchten Sie nur die Stunde zu beſtimmen, wir halten uns frei.\pend
           
\pstart
           Mit herzlichem Gruß Ihre{\\[\baselineskip]}\spacefill\mbox{Lou Andreas-Salomé}\pend
           \leftskip=0em{}\selectlanguage{ngerman}\endnumbering\briefempfaengerindex{Schnitzler, Arthur@\textsc{Schnitzler, Arthur}!zzzAndreas-Salome, Lou@\emph{von Lou Andreas-Salomé}!1912-11-102@{{[}vor dem 15. 11. 1912?{]}}|)be}\mylabel{L02097h}  \normalsize

\doendnotes{C}
\bigskip
\vfill

\clearpage

\footnotesize

\lohead{\textsc{register}}

% Definiere theindex-Environment komplett neu ohne reledmac
\makeatletter
\renewenvironment{theindex}{%
  \section*{\indexname}%
  \setlength{\parindent}{0pt}%
  \setlength{\parskip}{0pt plus 0.3pt}%
  \let\item\@idxitem
}{%
  \clearpage
}
\makeatother

\IfFileExists{\jobname-pw.ind}{\input{\jobname-pw.ind}}{}

\end{document}

      