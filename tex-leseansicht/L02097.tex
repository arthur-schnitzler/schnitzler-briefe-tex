%% latex-leseansicht-vorspann.tex
%% Vorspann für die Leseansicht.
%% Lädt die gemeinsame Datei latex-vorspann.tex mit nicht gesetztem Schalter.

\newif\ifkorrekturansicht
\korrekturansichtfalse

\input{../tex-inputs/latex-vorspann}


\section[Lou Andreas-Salomé an Arthur Schnitzler, {{[}}vor dem 15. 11. 1912?{{]}}]{L02097 Lou Andreas-Salomé an Arthur Schnitzler, {[}vor dem 15. 11. 1912?{]}}
\nopagebreak\mylabel{L02097v}
\rehead{ }\normalsize\beginnumbering\briefempfaengerindex{Schnitzler, Arthur@\textsc{Schnitzler, Arthur}!zzzAndreas-Salomé, Lou@\emph{von Lou Andreas-Salomé}!1912-11-142@{{[}vor dem 15. 11. 1912?{]}}|(be}
\toendnotes[C]{\smallbreak\pagebreak[2]}
\correspDesc{Versand  durch Lou Andreas-Salomé im Zeitraum [vor dem 15. 11. 1912?] in Wien
\newline{}Erhalt  durch Arthur Schnitzler im Zeitraum [10. 11. 1912 – 15. 11. 1912?] in Wien}\toendnotes[C]{\smallbreak}
\Standort{CUL, Schnitzler, B 3.}
\physDesc{Briefkarte, 313 Zeichen
\newline{}Handschrift: schwarze Tinte, deutsche Kurrent
\newline{}Schnitzler: 1) mit Bleistift beschriftet: »\substVorne{}\textsuperscript{A}\substDazwischen{}L\substHinten{}ou Salom\textcolor{gray}{e} Andr.«  2) mit rotem Buntstift eine Unterstreichung}\toendnotes[C]{\smallbreak}
\pstart{}{\pb}Lieber Herr Doktor,\pend\vspace{0.5em}
\pstart
           ich bin mit einem jungen \label{K_L02097-1v}\edtext{Mädchen\pwindex{Delp, Ellen 9.\,2.\,1890 Leipzig – 25.\,2.\,1990 Insel Reichenau@\textsc{Delp, Ellen} (9.\,2.\,1890 Leipzig – 25.\,2.\,1990 Insel Reichenau), \emph{Schriftstellerin}|pwv}}{\lemma{\textnormal{\emph{Mädchen}}}\Cendnote{\textnormal{Die Datierung basiert auf der Annahme,
                  dass es sich bei der Begleitung um Ellen Delp\pwindex{Delp, Ellen 9.\,2.\,1890 Leipzig – 25.\,2.\,1990 Insel Reichenau@\textsc{Delp, Ellen} (9.\,2.\,1890 Leipzig – 25.\,2.\,1990 Insel Reichenau), \emph{Schriftstellerin}|pwk} handelt,
                     vgl. A. S.: \emph{Tagebuch}, 15. 11. 1912.}}}\label{K_L02097-1} hier
               in Wien\oindex{Wien@\textbf{Wien}, \emph{Verwaltungsgebiet}|pw}, und würde mich freuen, wenn wir Ihnen
               einen Beſuch abſtatten dürften, falls dies keine Störung für Sie bedeutet?
                  Dienstag oder Donnerstag oder Freitag
               brauchten Sie nur die Stunde zu beſtimmen, wir halten uns frei.\pend
           
\pstart
           Mit herzlichem Gruß Ihre{\\[\baselineskip]}\spacefill\mbox{Lou Andreas-Salomé}\pend
           \leftskip=0em{}\selectlanguage{ngerman}\endnumbering\briefempfaengerindex{Schnitzler, Arthur@\textsc{Schnitzler, Arthur}!zzzAndreas-Salomé, Lou@\emph{von Lou Andreas-Salomé}!1912-11-102@{{[}vor dem 15. 11. 1912?{]}}|)be}\mylabel{L02097h}  \newcommand{\dateiname}{L02097}\newcommand{\titel}{Lou Andreas-Salomé an Arthur Schnitzler, [vor dem 15. 11. 1912?]}\newcommand{\editorInnen}{Martin Anton Müller und Gerd-Hermann Susen}%% latex-leseansicht-abspann.tex
%% Abspann für die Leseansicht.
%% Der Schalter \ifkorrekturansicht ist bereits durch den Vorspann gesetzt.

%% latex-abspann.tex
%% Gemeinsamer Abspann für Korrekturansicht und Leseansicht.
%% Setzt den Schalter \ifkorrekturansicht voraus (gesetzt in den
%% einbindenden Dateien latex-korrekturansicht-abspann.tex bzw.
%% latex-leseansicht-abspann.tex).
%% ---------------------------------------------------------------

\normalsize

% Das esempio-Environment wird nur in der Leseansicht benötigt
\ifkorrekturansicht\else
\newenvironment{esempio}[3]%
{
    \vspace{1.5ex}
    \rlap{\underline{#1}}
    \par
    \setlength{\parindent}{0cm}
    \nopagebreak
    \leftskip=#2cm
    \rightskip=#3cm
}
{
    \par
}
\fi

\doendnotes{C}
\bigskip
\vfill

\clearpage

\footnotesize

\ifkorrekturansicht
  \lohead{\textsc{register}}
\fi

% theindex-Environment neu definieren ohne reledmac
\makeatletter
\renewenvironment{theindex}{%
  \ifkorrekturansicht
    \section*{\indexname}%
  \else
    \subsubsection*{Index der erwähnten Entitäten}%
  \fi
  \setlength{\parindent}{0pt}%
  \setlength{\parskip}{0pt plus 0.3pt}%
  \let\item\@idxitem
}{%
  \ifkorrekturansicht\clearpage\fi
}
\makeatother

\IfFileExists{\jobname-pw.ind}{\input{\jobname-pw.ind}}{}

% Quellenangabe nur in der Leseansicht
\ifkorrekturansicht\else
% Fallback-Definitionen, falls die .tex-Datei \titel etc. nicht gesetzt hat
\providecommand{\titel}{}
\providecommand{\editorInnen}{}
\providecommand{\dateiname}{\jobname}

\vspace{3cm}

\vfill

\footnotesize
\textsc{Quelle}: \titel. Herausgegeben von {\editorInnen}. In: \emph{Arthur Schnitzler: Briefwechsel mit Autorinnen und Autoren}.
 Digitale Edition, https://schnitzler-briefe.acdh.oeaw.ac.at/{\dateiname}.html (Stand \today)
\fi

\end{document}


