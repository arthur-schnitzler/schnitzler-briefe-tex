%% latex-korrekturansicht-vorspann.tex
%% Vorspann für die Korrekturansicht.
%% Lädt die gemeinsame Datei latex-vorspann.tex mit gesetztem Schalter.

\newif\ifkorrekturansicht
\korrekturansichttrue

\input{../tex-inputs/latex-vorspann}


\section[Arthur Schnitzler an Wilhelm Bölsche, 24. 2. 1892]{L00076 Arthur Schnitzler an Wilhelm Bölsche, 24. 2. 1892}
\nopagebreak\mylabel{L00076v}
\rehead{ }\normalsize\beginnumbering\briefempfaengerindex{Boelsche, Wilhelm@\textsc{Bölsche, Wilhelm}!zzzSchnitzler, Arthur@\emph{von Arthur Schnitzler}!1892-02-241@{24. 2. 1892}|(be}
\toendnotes[C]{\smallbreak\pagebreak[2]}\Standort{Wrocław, Biblioteka Uniwersytecka, Böl.Pis 1762.}
\physDesc{Brief, 1 Blatt, 2 Seiten, 493 Zeichen
\newline{}Handschrift: schwarze Tinte, deutsche Kurrent}
\buchAbdrucke{\weitereDrucke{1) \emph{Germanica Wratislaviensia} (1987) Nr. 77, S. 459.} \weitereDrucke{2) Wilhelm Bölsche: \emph{Briefwechsel. Mit Autoren der Freien Bühne}. Berlin: \emph{Weidler} 2010, S. 676.} }
\pstart
           {\pb}\textsc{Wien I Giselastraße 11\oindex{Ordination Arthur Schnitzler [Boesendorferstrasse 11]@\textbf{Ordination Arthur Schnitzler [Bösendorferstraße 11]}, \emph{Ordination}|pw}}\pend
           
\pstart
           \raggedleft{}24/2 92.\pend
           
\pstart{}Verehrteſter Herr,\pend\vspace{0.5em}
\pstart
           erlauben Sie mir, zwei Fragen an Sie zu richten, für deren Beantwortung ich Ihnen
               ſehr dankbar wäre.\pend
           
\pstart
           1.) Wa{\geminationn} gedenken Sie meine »\textsc{Elixire}\pwindex{drei Elixire@\emph{Die drei Elixire}|pw}« in der Freien Bühne\pwindex{Freie Buehne fuer den Entwickelungskampf der Zeit@\emph{Freie Bühne für den Entwickelungskampf der Zeit}|pw} zum Abdruck zu
               bringen?\pend
           
\pstart
           2) Veröffentlichen Sie in den nächſten Heften vielleicht auch Gedichte? Ich möchte
                  {\pb}Ihnen für dieſen Fall ſehr gern welche ſenden.\pend
           
\pstart
           Entſchuldigen Sie, verehrteſter Herr, die verurſachte Mühe und ſeien Sie meiner
               ausgezeichneten Hochachtung verſichert.{\\[\baselineskip]}\spacefill\mbox{Dr Arthur Schnitzler.}\pend
           \leftskip=0em{}\selectlanguage{ngerman}\endnumbering\briefempfaengerindex{Boelsche, Wilhelm@\textsc{Bölsche, Wilhelm}!zzzSchnitzler, Arthur@\emph{von Arthur Schnitzler}!1892-02-241@{24. 2. 1892}|)be}\mylabel{L00076h}  \normalsize

\doendnotes{C}
\bigskip
\vfill

\clearpage

\footnotesize

\lohead{\textsc{register}}

% Definiere theindex-Environment komplett neu ohne reledmac
\makeatletter
\renewenvironment{theindex}{%
  \section*{\indexname}%
  \setlength{\parindent}{0pt}%
  \setlength{\parskip}{0pt plus 0.3pt}%
  \let\item\@idxitem
}{%
  \clearpage
}
\makeatother

\IfFileExists{\jobname-pw.ind}{\input{\jobname-pw.ind}}{}

\end{document}

      