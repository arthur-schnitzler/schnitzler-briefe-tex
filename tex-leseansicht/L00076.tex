%% latex-leseansicht-vorspann.tex
%% Vorspann für die Leseansicht.
%% Lädt die gemeinsame Datei latex-vorspann.tex mit nicht gesetztem Schalter.

\newif\ifkorrekturansicht
\korrekturansichtfalse

\input{../tex-inputs/latex-vorspann}


         
         \renewcommand{\erwaehntePersonen}{Personen: Wilhelm Bölsche}
         \renewcommand{\erwaehnteOrte}{Orte: Berlin, Bösendorferstraße, Wien}
         \renewcommand{\erwaehnteWerke}{Werke: Die drei Elixire, Freie Bühne für den Entwickelungskampf der Zeit}
               \section[Arthur Schnitzler an Wilhelm Bölsche, 24. 2. 1892]{ Arthur Schnitzler an Wilhelm Bölsche, 24. 2. 1892}\nopagebreak\mylabel{v}\rehead{ }\begin{ledgroupsized}[t]{13cm}\normalsize\beginnumbering \toendnotes[C]{\smallbreak\pagebreak[2]} \Standort{Wrocław, Biblioteka Uniwersytecka, Böl.Pis 1762.}
\physDesc{Brief, 1 Blatt, 2 Seiten, 493 Zeichen
\newline{}Handschrift: schwarze Tinte, deutsche Kurrent}\buchAbdrucke{\weitereDrucke{1) Alois Woldan: \emph{Arthur Schnitzler – Briefe an Wilhelm Bölsche.} In: \emph{Germanica Wratislaviensia} (1987) Nr. 77, S. 459.} \weitereDrucke{2) Wilhelm Bölsche: \emph{Briefwechsel. Mit Autoren der Freien Bühne}. Hg. Gerd-Hermann Susen. Berlin: \emph{Weidler} 2010, S. 676 (Werke und Briefe. Wissenschaftliche Ausgabe, Briefe I).} }\pstart
           \noindent{}{\pb}\textsc{Wien I Giselastraße 11\oindex{Boesendorferstrasse@\textbf{Bösendorferstraße}|pw}}\pend
           \pstart
           \raggedleft{}24/2 92.\pend
           \pstart{}Verehrteſter Herr,\pend\pstart
           erlauben Sie mir, zwei Fragen an Sie zu richten, für deren Beantwortung ich Ihnen
               ſehr dankbar wäre.\pend
           \pstart
           1.) Wa{\geminationn} gedenken Sie meine »\textsc{Elixire}\pwindex{Schnitzler, Arthur 15.05.1862 – 21.10.1931@\textsc{Schnitzler, Arthur} (15.05.1862 – 21.10.1931), \emph{Schriftsteller, Mediziner}!drei Elixire1893@\strich\emph{Die drei Elixire} {[}1893{]}|pw}« in der Freien Bühne\pwindex{Freie Buehne fuer den Entwickelungskampf der Zeit1892 – 1893@\emph{Freie Bühne für den Entwickelungskampf der Zeit} {[}1892 – 1893{]}|pw} zum Abdruck zu
               bringen?\pend
           \pstart
           2) Veröffentlichen Sie in den nächſten Heften vielleicht auch Gedichte? Ich möchte
                  {\pb}Ihnen für dieſen Fall ſehr gern welche ſenden.\pend
           \pstart
           Entſchuldigen Sie, verehrteſter Herr, die verurſachte Mühe und ſeien Sie meiner
               ausgezeichneten Hochachtung verſichert.{\\[\baselineskip]}\spacefill\mbox{Dr Arthur Schnitzler.}\pend
           \leftskip=0em{}
         
         \endnumbering\mylabel{h}\end{ledgroupsized}  \newcommand{\dateiname}{L00076}\newcommand{\titel}{Arthur Schnitzler an Wilhelm Bölsche, 24. 2. 1892}\newcommand{\editorInnen}{Martin Anton Müller und Gerd-Hermann Susen}%% latex-leseansicht-abspann.tex
%% Abspann für die Leseansicht.
%% Der Schalter \ifkorrekturansicht ist bereits durch den Vorspann gesetzt.

%% latex-abspann.tex
%% Gemeinsamer Abspann für Korrekturansicht und Leseansicht.
%% Setzt den Schalter \ifkorrekturansicht voraus (gesetzt in den
%% einbindenden Dateien latex-korrekturansicht-abspann.tex bzw.
%% latex-leseansicht-abspann.tex).
%% ---------------------------------------------------------------

\normalsize

% Das esempio-Environment wird nur in der Leseansicht benötigt
\ifkorrekturansicht\else
\newenvironment{esempio}[3]%
{
    \vspace{1.5ex}
    \rlap{\underline{#1}}
    \par
    \setlength{\parindent}{0cm}
    \nopagebreak
    \leftskip=#2cm
    \rightskip=#3cm
}
{
    \par
}
\fi

\doendnotes{C}
\bigskip
\vfill

\clearpage

\footnotesize

\ifkorrekturansicht
  \lohead{\textsc{register}}
\fi

% theindex-Environment neu definieren ohne reledmac
\makeatletter
\renewenvironment{theindex}{%
  \ifkorrekturansicht
    \section*{\indexname}%
  \else
    \subsubsection*{Index der erwähnten Entitäten}%
  \fi
  \setlength{\parindent}{0pt}%
  \setlength{\parskip}{0pt plus 0.3pt}%
  \let\item\@idxitem
}{%
  \ifkorrekturansicht\clearpage\fi
}
\makeatother

\IfFileExists{\jobname-pw.ind}{\input{\jobname-pw.ind}}{}

% Quellenangabe nur in der Leseansicht
\ifkorrekturansicht\else
% Fallback-Definitionen, falls die .tex-Datei \titel etc. nicht gesetzt hat
\providecommand{\titel}{}
\providecommand{\editorInnen}{}
\providecommand{\dateiname}{\jobname}

\vspace{3cm}

\vfill

\footnotesize
\textsc{Quelle}: \titel. Herausgegeben von {\editorInnen}. In: \emph{Arthur Schnitzler: Briefwechsel mit Autorinnen und Autoren}.
 Digitale Edition, https://schnitzler-briefe.acdh.oeaw.ac.at/{\dateiname}.html (Stand \today)
\fi

\end{document}


      