%% latex-leseansicht-vorspann.tex
%% Vorspann für die Leseansicht.
%% Lädt die gemeinsame Datei latex-vorspann.tex mit nicht gesetztem Schalter.

\newif\ifkorrekturansicht
\korrekturansichtfalse

\input{../tex-inputs/latex-vorspann}


         
         \renewcommand{\erwaehntePersonen}{Personen: Hugo von Hofmannsthal, Gertrude von Hofmannsthal, Hugo August von Hofmannsthal, Olga Schnitzler}
         \renewcommand{\erwaehnteOrte}{Orte: Bad Fusch, Hotel Alpenrose, Schweiz, Seis am Schlern, Sils im Engadin}
         \renewcommand{\erwaehnteWerke}{Werke: Der Mann von fünfzig Jahren, Der Rosenkavalier, Der Weg ins Freie. Roman, Der einsame Weg. Schauspiel in fünf Akten, Liebelei. Schauspiel in drei Akten}
               \section[Hugo von Hofmannsthal an Arthur Schnitzler, 24. 7. {[}1908{]}]{ Hugo von Hofmannsthal an Arthur Schnitzler, 24. 7. {[}1908{]}}\nopagebreak\mylabel{v}\rehead{ }\begin{ledgroupsized}[t]{13cm}\normalsize\beginnumbering\briefempfaengerindex{Schnitzler, Arthur@\textsc{Schnitzler, Arthur}!zzzHofmannsthal, Hugo von@\emph{von Hugo von Hofmannsthal}!1908-07-241@{24. 7. {[}1908{]}}|(be} \toendnotes[C]{\smallbreak\pagebreak[2]} \Standort{CUL, Schnitzler, B 43.}
\physDesc{Brief, 2 Blätter, 8 Seiten, 2092 Zeichen (Das zweite Blatt mit »2« beschriftet)
\newline{}Handschrift: schwarze Tinte, deutsche Kurrent
\newline{}Schnitzler: mit Bleistift die Jahreszahl ergänzt: »08« und beschriftet: »Hugo v H.« 
\newline{}Ordnung: 1) mit Bleistift von unbekannter Hand nummeriert: »\strikeout{294}«  2) mit Bleistift von unbekannter Hand nummeriert:
                                    »299«}\buchAbdrucke{\weitereDrucke{Hugo von Hofmannsthal, Arthur Schnitzler: \emph{Briefwechsel}. Hg. Therese Nickl und Heinrich Schnitzler. Frankfurt am Main: \emph{S. Fischer} 1964, S. 237.} }\toendnotes[C]{\smallbreak}\pstart
           \raggedleft{}{\pb}Bad Fuſch\oindex{Bad Fusch@\textbf{Bad Fusch}|pw}{ }24\textsuperscript{ten} VII.\pend
           \pstart{}mein lieber Arthur\pend\pstart
           ich habe dieſe 14 Tage hier ſo viel gearbeitet, gedacht, notiert daſs ich wirklich
               außer kleinen Karten an Gerty\pwindex{Hofmannsthal, Gertrude von 16.03.1880 – 09.11.1959@\textsc{Hofmannsthal, Gertrude von} (16.03.1880 – 09.11.1959)|pw} und meinen Vater\pwindex{Hofmannsthal, Hugo August von 21.12.1841 – 08.12.1915@\textsc{Hofmannsthal, Hugo August von} (21.12.1841 – 08.12.1915), \emph{Bankdirektor}|pwv} nichts Briefartiges
               habe ſchreiben können und wollen, ſchon aus Angſt vor einem Überſpannen und
                  Nicht-ſchlafen.\hspace*{1.5em}Übermorgen kommt Gerty\pwindex{Hofmannsthal, Gertrude von 16.03.1880 – 09.11.1959@\textsc{Hofmannsthal, Gertrude von} (16.03.1880 – 09.11.1959)|pw} mir nach und {\pb}wir fahren nach \textsc{Sils}\oindex{Sils im Engadin@\textbf{Sils im Engadin}|pw}. Dort hoffe ich nicht nur mit dieſer Comödie\pwindex{Hofmannsthal, Hugo von 1874-02-01 – 1929-07-15@\textsc{Hofmannsthal, Hugo von} (1874-02-01 – 1929-07-15), \emph{Schriftsteller}!Rosenkavalier1911@\strich\emph{Der Rosenkavalier} {[}1911{]}|pwv} fertig zu werden, ſondern auch ein anderes, kurzes
                  Stück\pwindex{Hofmannsthal, Hugo von 1874-02-01 – 1929-07-15@\textsc{Hofmannsthal, Hugo von} (1874-02-01 – 1929-07-15), \emph{Schriftsteller}!Mann von fuenfzig Jahren@\strich\emph{Der Mann von fünfzig Jahren}|pwuv}, das mir
               mit zudringlicher Lebhaftigkeit vorſchwebt, zum mindeſten anzufangen. \hspace*{1.5em}Statt nach \textsc{Sils}\oindex{Sils im Engadin@\textbf{Sils im Engadin}|pw} könnten wir doch ganz wohl auch dorthin kommen wo {\pb}Ihr ſeid – ich meine: »hätten wir
               können.« Es iſt eine Schrulle von mir daß wenn jemand wie Sie nach dem ich mich gerne
               richte, einen Plan ausſpricht, wie Sie im Winter den, in die Schweiz\oindex{Schweiz@\textbf{Schweiz}|pw} zu gehen – ich mich ſo daran halte als ob es etwas ganz
               Feſtes wäre. Auf dieſe Weiſe habe ich in \textsc{Sils}\oindex{Sils im Engadin@\textbf{Sils im Engadin}|pw} gemiethet – um eine Begegnung mit Euch {\pb}bequem zu haben. Dann im
                  Mai wäre dieſe Sache wohl noch rückgängig zu machen geweſen, da hat
               aber meinen Willen und meine Luſt etwas anderes gelähmt: ich meine mein gar nicht
               glückliches Verhältnis zu Ihrem Roman\pwindex{Schnitzler, Arthur 15.05.1862 – 21.10.1931@\textsc{Schnitzler, Arthur} (15.05.1862 – 21.10.1931), \emph{Schriftsteller, Mediziner}!Weg ins Freie. Roman1.1.1908 – 1.6.1908@\strich\emph{Der Weg ins Freie. Roman} {[}1.1.1908 – 1.6.1908{]}|pwv}. Da ich Sie eben ſehr gerne habe, und zwiſchen Ihnen und Ihren
               Arbeiten natürlich keine Grenze ziehen kann, ſo hat mich dies {\pb}durch einige Wochen ſehr verſtört.
               Es wäre mir ebenſo qualvoll geweſen, darüber reden zu müſſen, als es mir peinlich
               war, \strikeout{darüber} zu ſchweigen.\pend
           \pstart
           Jetzt bin ich darüber ruhiger geworden, und ich erwähne es jetzt abſichtlich, weil
               Ihnen ja doch mein Schweigen aufgefallen ſein muſs.\pend
           \pstart
           Jetzt macht es mir gar nichts, {\pb}entweder niemals darüber zu reden oder doch zu reden, wenn es sich einmal
               ergibt.\pend
           \pstart
           \numberlinefalse{}–\numberlinetrue{}\pend
           \pstart
           Ich bin ſo begierig was Sie machen.\hspace*{1.5em}Bitte ſchreiben
               Sie mir ein paar Zeilen, oder es ſchreibt vielleicht Olga\pwindex{Schnitzler, Olga 17.01.1882 – 13.01.1970@\textsc{Schnitzler, Olga} (17.01.1882 – 13.01.1970), \emph{Schauspielerin, Sängerin}|pw} an Gerty\pwindex{Hofmannsthal, Gertrude von 16.03.1880 – 09.11.1959@\textsc{Hofmannsthal, Gertrude von} (16.03.1880 – 09.11.1959)|pw}.\pend
           \pstart
           Von Herzen Ihr{\\[\baselineskip]}\spacefill\mbox{Hugo}\pend
           \leftskip=0em{}\pstart
           \noindent{}PS. Habe, um unter vielen Büchern auch etwas von Ihnen mitzuhaben, den »einſamen Weg\pwindex{Schnitzler, Arthur 15.05.1862 – 21.10.1931@\textsc{Schnitzler, Arthur} (15.05.1862 – 21.10.1931), \emph{Schriftsteller, Mediziner}!einsame Weg. Schauspiel in fuenf Akten1904@\strich\emph{Der einsame Weg. Schauspiel in fünf Akten} {[}1904{]}|pw}« mitgenommen und ihn auf einem
                  Spaziergang mit großer Freude vom Anfang zum Ende \substVorne{}\textsuperscript{geſehen.}{\allowbreak}\substDazwischen{}geleſen.\substHinten{}\hspace*{1.5em}Es iſt doch für eine zweite Periode ihres
                  Schaffens ebenſo ſchön und bedeutend, als {\pb}»Liebelei\pwindex{Schnitzler, Arthur 15.05.1862 – 21.10.1931@\textsc{Schnitzler, Arthur} (15.05.1862 – 21.10.1931), \emph{Schriftsteller, Mediziner}!Liebelei. Schauspiel in drei Akten1895-10-09@\strich\emph{Liebelei. Schauspiel in drei Akten} {[}1895-10-09{]}|pw}« für eine erſte.\pend
           \pstart
           Unſere Adreſſe:\pend
           \leftskip=3em{}\pstart
           \noindent{}\textsc{Sils Maria im Engadin}\oindex{Sils im Engadin@\textbf{Sils im Engadin}|pw}\pend
           \leftskip=0em{}\leftskip=3em{}\pstart
           \textsc{\uline{Hôtel Alpenrose}}\oindex{Hotel Alpenrose@\textbf{Hotel Alpenrose}|pw}.\pend
           \leftskip=0em{}
         
         \endnumbering\mylabel{h}\end{ledgroupsized}  \newcommand{\dateiname}{L01785}\newcommand{\titel}{Hugo von Hofmannsthal an Arthur Schnitzler, 24. 7. [1908]}\newcommand{\editorInnen}{Martin Anton Müller und Gerd-Hermann Susen}%% latex-leseansicht-abspann.tex
%% Abspann für die Leseansicht.
%% Der Schalter \ifkorrekturansicht ist bereits durch den Vorspann gesetzt.

%% latex-abspann.tex
%% Gemeinsamer Abspann für Korrekturansicht und Leseansicht.
%% Setzt den Schalter \ifkorrekturansicht voraus (gesetzt in den
%% einbindenden Dateien latex-korrekturansicht-abspann.tex bzw.
%% latex-leseansicht-abspann.tex).
%% ---------------------------------------------------------------

\normalsize

% Das esempio-Environment wird nur in der Leseansicht benötigt
\ifkorrekturansicht\else
\newenvironment{esempio}[3]%
{
    \vspace{1.5ex}
    \rlap{\underline{#1}}
    \par
    \setlength{\parindent}{0cm}
    \nopagebreak
    \leftskip=#2cm
    \rightskip=#3cm
}
{
    \par
}
\fi

\doendnotes{C}
\bigskip
\vfill

\clearpage

\footnotesize

\ifkorrekturansicht
  \lohead{\textsc{register}}
\fi

% theindex-Environment neu definieren ohne reledmac
\makeatletter
\renewenvironment{theindex}{%
  \ifkorrekturansicht
    \section*{\indexname}%
  \else
    \subsubsection*{Index der erwähnten Entitäten}%
  \fi
  \setlength{\parindent}{0pt}%
  \setlength{\parskip}{0pt plus 0.3pt}%
  \let\item\@idxitem
}{%
  \ifkorrekturansicht\clearpage\fi
}
\makeatother

\IfFileExists{\jobname-pw.ind}{\input{\jobname-pw.ind}}{}

% Quellenangabe nur in der Leseansicht
\ifkorrekturansicht\else
% Fallback-Definitionen, falls die .tex-Datei \titel etc. nicht gesetzt hat
\providecommand{\titel}{}
\providecommand{\editorInnen}{}
\providecommand{\dateiname}{\jobname}

\vspace{3cm}

\vfill

\footnotesize
\textsc{Quelle}: \titel. Herausgegeben von {\editorInnen}. In: \emph{Arthur Schnitzler: Briefwechsel mit Autorinnen und Autoren}.
 Digitale Edition, https://schnitzler-briefe.acdh.oeaw.ac.at/{\dateiname}.html (Stand \today)
\fi

\end{document}


      