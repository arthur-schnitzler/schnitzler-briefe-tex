%% latex-korrekturansicht-vorspann.tex
%% Vorspann für die Korrekturansicht.
%% Lädt die gemeinsame Datei latex-vorspann.tex mit gesetztem Schalter.

\newif\ifkorrekturansicht
\korrekturansichttrue

\input{../tex-inputs/latex-vorspann}


\section[Arthur Schnitzler an Richard Beer-Hofmann, 17. 5. 1895]{L00442 Arthur Schnitzler an Richard Beer-Hofmann, 17. 5. 1895}
\nopagebreak\mylabel{L00442v}
\rehead{ }\normalsize\beginnumbering\briefempfaengerindex{Beer-Hofmann, Richard@\textsc{Beer-Hofmann, Richard}!zzzSchnitzler, Arthur@\emph{von Arthur Schnitzler}!1895-05-171@{17. 5. 1895}|(be}
\toendnotes[C]{\smallbreak\pagebreak[2]}\Standort{YCGL, MSS 31.}
\physDesc{Postkarte, 324 Zeichen
\newline{}Handschrift: Bleistift, deutsche Kurrent
\newline{}Versand: 1) Rohrpost  2) Stempel: »\nobreak{}\oindex{I., Innere Stadt@\textbf{I., Innere Stadt}, \emph{A.ADM3}|pwk}Wien 1/1, 17 V 95, 1 20N\nobreak{}«.  3) Stempel: »\nobreak{}\oindex{I., Innere Stadt@\textbf{I., Innere Stadt}, \emph{A.ADM3}|pwk}Wien 1/1, 17 V 95, 1 30N\nobreak{}«. }
\buchAbdrucke{\weitereDrucke{Arthur Schnitzler, Richard Beer-Hofmann: \emph{Briefwechsel 1891–1931}. Wien, Zürich: \emph{Europaverlag} 1992, S. 72.} }\toendnotes[C]{\smallbreak}\pstart{}{\pb}\textsc{Dr. Richard Beer-Hofmann}\pend{}\pstart{}Wien\oindex{Wien@\textbf{Wien}, \emph{A.ADM2}|pw}\pend{}\pstart{}\textsc{I Wollzeile 15\oindex{Wollzeile@\textbf{Wollzeile}, \emph{Straße (K.STR)}|pw}}\pend{}\pstart{}4. Stock.\pend{}{\bigskip}\vspace{1em}
\pstart
           \noindent{}{\pb}Lieber Richard, das Waſſer fällt in die Donau\oindex{Donau@\textbf{Donau}, \emph{Fluss (N.FLS)}|pw} alſo fällt die \label{K_L00442-1v}\edtext{Donau\oindex{Donau@\textbf{Donau}, \emph{Fluss (N.FLS)}|pw} ins Waſſer}{\lemma{\textnormal{\emph{Donau ins Waſſer}}}\Cendnote{\textnormal{mutmaßlich der Ausflug zusammen mit Lou Andreas Salomé\pwindex{Andreas-Salome, Lou 12.02.1861 – 05.02.1937@\textsc{Andreas-Salomé, Lou} (12.02.1861 – 05.02.1937), \emph{Schriftsteller/Schriftstellerin}|pwk} in die Wachau\oindex{Wachau@\textbf{Wachau}, \emph{L.RGN}|pwk}, der dann am 20. 5. 1895 stattfand}}}\label{K_L00442-1}. Sollte es daher \substVorne{}\textsuperscript{nicht}\substDazwischen{}um\substHinten{}{ }3 nicht herrlich ſchön ſein, ſo ko{\geminationm} ich
               erſt gegen ½ 5 zu Ihnen. Sollten Sie früher weggehen, bitte um
               zurückgelaſſene Poſt. –\pend
           
\pstart
           Herzlich Ihr \spacefill\mbox{Arthur.}\pend
           
\pstart
           \noindent{}Aber Sie gehen ja nicht früher weg.\pend
           \selectlanguage{ngerman}\endnumbering\briefempfaengerindex{Beer-Hofmann, Richard@\textsc{Beer-Hofmann, Richard}!zzzSchnitzler, Arthur@\emph{von Arthur Schnitzler}!1895-05-171@{17. 5. 1895}|)be}\mylabel{L00442h}  \normalsize

\doendnotes{C}
\bigskip
\vfill

\clearpage

\footnotesize

\lohead{\textsc{register}}

% Definiere theindex-Environment komplett neu ohne reledmac
\makeatletter
\renewenvironment{theindex}{%
  \section*{\indexname}%
  \setlength{\parindent}{0pt}%
  \setlength{\parskip}{0pt plus 0.3pt}%
  \let\item\@idxitem
}{%
  \clearpage
}
\makeatother

\IfFileExists{\jobname-pw.ind}{\input{\jobname-pw.ind}}{}

\end{document}

      