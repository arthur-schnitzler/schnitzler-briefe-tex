%% latex-korrekturansicht-vorspann.tex
%% Vorspann für die Korrekturansicht.
%% Lädt die gemeinsame Datei latex-vorspann.tex mit gesetztem Schalter.

\newif\ifkorrekturansicht
\korrekturansichttrue

\input{../tex-inputs/latex-vorspann}


\section[ Paul Goldmann an Arthur Schnitzler, 31. 7. 1902]{L03216 Paul Goldmann an Arthur Schnitzler, 31. 7. 1902}
\nopagebreak\mylabel{L03216v}
\rehead{ }\normalsize\beginnumbering\briefempfaengerindex{Schnitzler, Arthur@\textsc{Schnitzler, Arthur}!zzzGoldmann, Paul@\emph{von Paul Goldmann}!1902-07-311@{31. 7. 1902}|(be}
\toendnotes[C]{\smallbreak\pagebreak[2]}\Standort{DLA, A:Schnitzler, HS.NZ85.1.3172.}
\physDesc{Postkarte, 278 Zeichen
\newline{}Handschrift: 1) Bleistift, deutsche Kurrent\hspace{1em}2) Bleistift, lateinische Kurrent (\noindent{}Adresse)\hspace{1em}
\newline{}Versand: 1) Stempel: »\nobreak{}\oindex{Basel@\textbf{Basel}, \emph{P.PPLA}|pwk}Basel 1 Fil. S. B., 31. VII. 02, 9\nobreak{}«.   2) Stempel: »\nobreak{}\oindex{IX., Alsergrund@\textbf{IX., Alsergrund}, \emph{A.ADM3}|pwk}9/3 {[}Wien{]} 72, 2. 8. 02, 8. V, Bes{[}tellt{]}\nobreak{}«. 
\newline{}Schnitzler: mit Bleistift das Jahr »902« vermerkt }\toendnotes[C]{\smallbreak}\pstart{}{\pb}Herrn\pend{}\pstart{}Dr. Arthur Schnitzler\pend{}\pstart{}Wien\oindex{Wien@\textbf{Wien}, \emph{A.ADM2}|pw}\pend{}\pstart{}IX. Frankgaſse 1\oindex{Frankgasse 1@\textbf{Frankgasse 1}, \emph{Wohngebäude (K.WHS)}|pw}.\pend{}{\bigskip}\vspace{1em}
\pstart
           \centering{}{\pb}Baſel\oindex{Basel@\textbf{Basel}, \emph{P.PPLA}|pw}{ }31. Juli\pend
           \vspace{0.5em}
\pstart
           Mein lieber Freund, Kurz vor der Abreiſe nach der Schweiz\oindex{Schweiz@\textbf{Schweiz}, \emph{A.PCLI}|pw} erhielt ich Deine \label{K_L03216-1v}\edtext{l.}{\lemma{\textnormal{\emph{l.}}}\Cendnote{\textnormal{liebe}}}\label{K_L03216-1}
               Karte. Da iſt ſchwer zu rathen. Aber ich meine doch, das \label{K_L03216-2v}\edtext{D.th\orgindex{Deutsches Theater Berlin@Deutsches Theater Berlin|pw}}{\lemma{\textnormal{\emph{D.th}}}\Cendnote{\textnormal{\emph{Deutsches Theater}\orgindex{Deutsches Theater Berlin@Deutsches Theater Berlin|pwk}; Bezug auf die Berlin\oindex{Berlin@\textbf{Berlin}, \emph{P.PPLC}|pwk}er Premiere von \emph{Der Schleier der Beatrice}\pwindex{Schleier der Beatrice. Schauspiel in fuenf Akten@\emph{Der Schleier der Beatrice. Schauspiel in fünf Akten}|pwk}, siehe Paul Goldmann an Arthur Schnitzler, 14. 7. [1902].}}}\label{K_L03216-2}, ſelbſt \label{K_L03216-3v}\edtext{\uline{nach}{ }\textsc{Monna Vanna\pwindex{Monna Vanna. Schauspiel in drei Akten@\emph{Monna Vanna. Schauspiel in drei Akten}|pw}}}{\lemma{\textnormal{\emph{nach Monna Vanna}}}\Cendnote{\textnormal{Siehe Paul Goldmann an Arthur Schnitzler, 16. 6. [1902].
               }}}\label{K_L03216-3}, iſt beſſer als das Schillertheater\orgindex{Schiller-Theater@Schiller-Theater|pw}.\pend
           
\pstart
           Viele Grüße {\\[\baselineskip]}Dein {\\[\baselineskip]}\spacefill\mbox{P. Goldm}\pend
           \leftskip=0em{}\selectlanguage{ngerman}\endnumbering\briefempfaengerindex{Schnitzler, Arthur@\textsc{Schnitzler, Arthur}!zzzGoldmann, Paul@\emph{von Paul Goldmann}!1902-07-311@{31. 7. 1902}|)be}\mylabel{L03216h}  \normalsize

\doendnotes{C}
\bigskip
\vfill

\clearpage

\footnotesize

\lohead{\textsc{register}}

% Definiere theindex-Environment komplett neu ohne reledmac
\makeatletter
\renewenvironment{theindex}{%
  \section*{\indexname}%
  \setlength{\parindent}{0pt}%
  \setlength{\parskip}{0pt plus 0.3pt}%
  \let\item\@idxitem
}{%
  \clearpage
}
\makeatother

\IfFileExists{\jobname-pw.ind}{\input{\jobname-pw.ind}}{}

\end{document}

      