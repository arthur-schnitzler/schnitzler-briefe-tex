%% latex-leseansicht-vorspann.tex
%% Vorspann für die Leseansicht.
%% Lädt die gemeinsame Datei latex-vorspann.tex mit nicht gesetztem Schalter.

\newif\ifkorrekturansicht
\korrekturansichtfalse

\input{../tex-inputs/latex-vorspann}


         
         \renewcommand{\erwaehntePersonen}{Personen: Maurice Maeterlinck}
         \renewcommand{\erwaehnteInstitutionen}{Institutionen: Deutsches Theater Berlin, Schiller-Theater}
         \renewcommand{\erwaehnteOrte}{Orte: Basel, Berlin, Frankgasse, Schweiz, Wien}
         \renewcommand{\erwaehnteWerke}{Werke: Der Schleier der Beatrice. Schauspiel in fünf Akten, Monna Vanna. Schauspiel in drei Akten}
               \section[ Paul Goldmann an Arthur Schnitzler, 31. 7. 1902]{ Paul Goldmann an Arthur Schnitzler, 31. 7. 1902}\nopagebreak\mylabel{v}\rehead{ }\begin{ledgroupsized}[t]{13cm}\normalsize\beginnumbering \toendnotes[C]{\smallbreak\pagebreak[2]} \Standort{DLA, A:Schnitzler, HS.NZ85.1.3172.}
\physDesc{Postkarte, 278 Zeichen
\newline{}Handschrift: 1) Bleistift, deutsche Kurrent\hspace{1em}2) Bleistift, lateinische Kurrent (\noindent{}Adresse)\hspace{1em}
\newline{}Versand: 1) Stempel: »\nobreak{}\oindex{Basel@\textbf{Basel}|pwk}Basel 1 Fil. S. B., 31. VII. 02, 9\nobreak{}«.   2) Stempel: »\nobreak{}9/3 {[}Wien{]} 72, 2. 8. 02, 8. V, Bes{[}tellt{]}\nobreak{}«. 
\newline{}Schnitzler: mit Bleistift das Jahr »{[}1{]}902« vermerkt }\toendnotes[C]{\smallbreak}\pstart{}{\pb}Herrn\pend{}\pstart{}Dr. Arthur Schnitzler\pend{}\pstart{}Wien\oindex{Wien@\textbf{Wien}|pw}\pend{}\pstart{}IX. Frankgaſse 1\oindex{Frankgasse@\textbf{Frankgasse}|pw}.\pend{}{\bigskip}\pstart
           \centering{}{\pb}Baſel\oindex{Basel@\textbf{Basel}|pw}{ }31. Juli\pend
           \pstart
           Mein lieber Freund, Kurz vor der Abreiſe nach der Schweiz\oindex{Schweiz@\textbf{Schweiz}|pw} erhielt ich Deine \label{K_L03216-1v}\edtext{l.}{\lemma{\textnormal{\emph{l.}}}\Cendnote{\textnormal{liebe}}}\label{K_L03216-1h} Karte. Da iſt ſchwer zu rathen. Aber ich meine
               doch, das \label{K_L03216-2v}\edtext{D.th\orgindex{Deutsches Theater Berlin@Deutsches Theater Berlin|pw}}{\lemma{\textnormal{\emph{D.th}}}\Cendnote{\textnormal{\emph{Deutsches Theater}\orgindex{Deutsches Theater Berlin@Deutsches Theater Berlin|pwk}; Bezug auf die Berlin\oindex{Berlin@\textbf{Berlin}|pwk}er Premiere von \emph{Der Schleier der Beatrice}\pwindex{Schnitzler, Arthur 15.05.1862 – 21.10.1931@\textsc{Schnitzler, Arthur} (15.05.1862 – 21.10.1931), \emph{Schriftsteller, Mediziner}!Schleier der Beatrice. Schauspiel in fuenf Akten1900-12-01@\strich\emph{Der Schleier der Beatrice. Schauspiel in fünf Akten} {[}1900-12-01{]}|pwk}, siehe Paul Goldmann an Arthur Schnitzler, 14. 7. [1902]}}}\label{K_L03216-2h}, ſelbſt \label{K_L03216-22v}\edtext{\uline{nach}{ }\textsc{Monna Vanna\pwindex{Maeterlinck, Maurice 29.08.1862 – 06.05.1949@\textsc{Maeterlinck, Maurice} (29.08.1862 – 06.05.1949), \emph{Schriftsteller}!Monna Vanna. Schauspiel in drei Akten1903@\strich\emph{Monna Vanna. Schauspiel in drei Akten} {[}1903{]}|pw}}}{\lemma{\textnormal{\emph{nach Monna Vanna}}}\Cendnote{\textnormal{\emph{Der Schleier der Beatrice}\pwindex{Schnitzler, Arthur 15.05.1862 – 21.10.1931@\textsc{Schnitzler, Arthur} (15.05.1862 – 21.10.1931), \emph{Schriftsteller, Mediziner}!Schleier der Beatrice. Schauspiel in fuenf Akten1900-12-01@\strich\emph{Der Schleier der Beatrice. Schauspiel in fünf Akten} {[}1900-12-01{]}|pwk} und \emph{Monna Vanna}\pwindex{Maeterlinck, Maurice 29.08.1862 – 06.05.1949@\textsc{Maeterlinck, Maurice} (29.08.1862 – 06.05.1949), \emph{Schriftsteller}!Monna Vanna. Schauspiel in drei Akten1903@\strich\emph{Monna Vanna. Schauspiel in drei Akten} {[}1903{]}|pwk} haben offensichtliche Parallelen, vor allem
                  im Ort der Handlung und in der zentralen Figur einer Frau zwischen zwei Männern.
                  Obzwar Schnitzler\pwindex{Schnitzler, Arthur 15.05.1862 – 21.10.1931@\textsc{Schnitzler, Arthur} (15.05.1862 – 21.10.1931), \emph{Schriftsteller, Mediziner}|pwk}s Stück\pwindex{Schnitzler, Arthur 15.05.1862 – 21.10.1931@\textsc{Schnitzler, Arthur} (15.05.1862 – 21.10.1931), \emph{Schriftsteller, Mediziner}!Schleier der Beatrice. Schauspiel in fuenf Akten1900-12-01@\strich\emph{Der Schleier der Beatrice. Schauspiel in fünf Akten} {[}1900-12-01{]}|pwkv} früher erschienen war, war es
                  offensichtlich eine schwierige Entscheidung, ob es auch am \emph{Deutschen Theater}\orgindex{Deutsches Theater Berlin@Deutsches Theater Berlin|pwk} gegeben werden sollte, nachdem dort Maeterlinck\pwindex{Maeterlinck, Maurice 29.08.1862 – 06.05.1949@\textsc{Maeterlinck, Maurice} (29.08.1862 – 06.05.1949), \emph{Schriftsteller}|pwk}s{ }Stück\pwindex{Maeterlinck, Maurice 29.08.1862 – 06.05.1949@\textsc{Maeterlinck, Maurice} (29.08.1862 – 06.05.1949), \emph{Schriftsteller}!Monna Vanna. Schauspiel in drei Akten1903@\strich\emph{Monna Vanna. Schauspiel in drei Akten} {[}1903{]}|pwkv} am Spielplan
                  gestanden hatte.}}}\label{K_L03216-22h}, iſt beſſer als das Schillertheater\orgindex{Schiller-Theater@Schiller-Theater|pw}.\pend
           \pstart
           Viele Grüße {\\[\baselineskip]}Dein {\\[\baselineskip]}\spacefill\mbox{P. Goldm}\pend
           \leftskip=0em{}
         
         \endnumbering\mylabel{h}\end{ledgroupsized}  \newcommand{\dateiname}{L03216}\newcommand{\titel}{Paul Goldmann an Arthur Schnitzler, 31. 7. 1902}\newcommand{\editorInnen}{Martin Anton Müller und Laura Untner}%% latex-leseansicht-abspann.tex
%% Abspann für die Leseansicht.
%% Der Schalter \ifkorrekturansicht ist bereits durch den Vorspann gesetzt.

%% latex-abspann.tex
%% Gemeinsamer Abspann für Korrekturansicht und Leseansicht.
%% Setzt den Schalter \ifkorrekturansicht voraus (gesetzt in den
%% einbindenden Dateien latex-korrekturansicht-abspann.tex bzw.
%% latex-leseansicht-abspann.tex).
%% ---------------------------------------------------------------

\normalsize

% Das esempio-Environment wird nur in der Leseansicht benötigt
\ifkorrekturansicht\else
\newenvironment{esempio}[3]%
{
    \vspace{1.5ex}
    \rlap{\underline{#1}}
    \par
    \setlength{\parindent}{0cm}
    \nopagebreak
    \leftskip=#2cm
    \rightskip=#3cm
}
{
    \par
}
\fi

\doendnotes{C}
\bigskip
\vfill

\clearpage

\footnotesize

\ifkorrekturansicht
  \lohead{\textsc{register}}
\fi

% theindex-Environment neu definieren ohne reledmac
\makeatletter
\renewenvironment{theindex}{%
  \ifkorrekturansicht
    \section*{\indexname}%
  \else
    \subsubsection*{Index der erwähnten Entitäten}%
  \fi
  \setlength{\parindent}{0pt}%
  \setlength{\parskip}{0pt plus 0.3pt}%
  \let\item\@idxitem
}{%
  \ifkorrekturansicht\clearpage\fi
}
\makeatother

\IfFileExists{\jobname-pw.ind}{\input{\jobname-pw.ind}}{}

% Quellenangabe nur in der Leseansicht
\ifkorrekturansicht\else
% Fallback-Definitionen, falls die .tex-Datei \titel etc. nicht gesetzt hat
\providecommand{\titel}{}
\providecommand{\editorInnen}{}
\providecommand{\dateiname}{\jobname}

\vspace{3cm}

\vfill

\footnotesize
\textsc{Quelle}: \titel. Herausgegeben von {\editorInnen}. In: \emph{Arthur Schnitzler: Briefwechsel mit Autorinnen und Autoren}.
 Digitale Edition, https://schnitzler-briefe.acdh.oeaw.ac.at/{\dateiname}.html (Stand \today)
\fi

\end{document}


      