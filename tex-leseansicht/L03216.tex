%% latex-leseansicht-vorspann.tex
%% Vorspann für die Leseansicht.
%% Lädt die gemeinsame Datei latex-vorspann.tex mit nicht gesetztem Schalter.

\newif\ifkorrekturansicht
\korrekturansichtfalse

\input{../tex-inputs/latex-vorspann}


\section[ Paul Goldmann an Arthur Schnitzler, 31. 7. 1902]{L03216 Paul Goldmann an Arthur Schnitzler,  31. 7. 1902}
\nopagebreak\mylabel{L03216v}
\rehead{ }\normalsize\beginnumbering\briefempfaengerindex{Schnitzler, Arthur@\textsc{Schnitzler, Arthur}!zzzGoldmann, Paul@\emph{von Paul Goldmann}!1902-07-311@{31. 7. 1902}|(be}
\toendnotes[C]{\smallbreak\pagebreak[2]}
\correspDesc{Versand  durch Paul Goldmann am 31. 7. 1902 in Basel
\newline{}Erhalt  durch Arthur Schnitzler am 2. 8. 1902 in Wien}\toendnotes[C]{\smallbreak}
\Standort{DLA, A:Schnitzler, HS.NZ85.1.3172.}
\physDesc{Postkarte, 278 Zeichen
\newline{}Handschrift: Bleistift, deutsche Kurrent
\newline{}Versand: 1) Stempel: »\nobreak{}\oindex{Basel@\textbf{Basel}|pwk}Basel 1 Fil. S. B., 31. VII. 02, 9\nobreak{}«.   2) Stempel: »\nobreak{}\oindex{IX., Alsergrund@\textbf{IX., Alsergrund}, \emph{Verwaltungsgebiet}|pwk}9/3 {[}Wien{]} 72, 2. 8. 02, 8. V, Bes{[}tellt{]}\nobreak{}«. 
\newline{}Schnitzler: mit Bleistift das Jahr »902« vermerkt }\toendnotes[C]{\smallbreak}\pstart{}\textsc{{\pb}Herrn}\pend{}\pstart{}\textsc{Dr. Arthur Schnitzler}\pend{}\pstart{}\textsc{Wien\oindex{Wien@\textbf{Wien}, \emph{Verwaltungsgebiet}|pw}}\pend{}\pstart{}\textsc{IX. Frankgaſse 1\oindex{Wien@\textbf{Wien}!IX., Alsergrund@\textbf{IX., Alsergrund}!Frankgasse 1@\textbf{Frankgasse 1}, \emph{Wohngebäude}|pw}.}\pend{}{\bigskip}\vspace{1em}
\pstart
           \centering{}{\pb}Baſel\oindex{Basel@\textbf{Basel}|pw}{ }31. Juli\pend
           \vspace{0.5em}
\pstart
           Mein lieber Freund, Kurz vor der Abreiſe nach der Schweiz\oindex{Schweiz@\textbf{Schweiz}|pw} erhielt ich Deine \label{K_L03216-1v}\edtext{l.}{\lemma{\textnormal{\emph{l.}}}\Cendnote{\textnormal{liebe}}}\label{K_L03216-1}
               Karte. Da iſt{ }ſchwer zu rathen. Aber ich meine doch, das \label{K_L03216-2v}\edtext{D.th\orgindex{Deutsches Theater Berlin@Deutsches Theater Berlin|pw}}{\lemma{\textnormal{\emph{D.th}}}\Cendnote{\textnormal{\emph{Deutsches Theater}\orgindex{Deutsches Theater Berlin@Deutsches Theater Berlin|pwk}; Bezug auf die Berlin\oindex{Berlin@\textbf{Berlin}, \emph{Hauptstadt}|pwk}er Premiere von \emph{Der Schleier der Beatrice}\pwindex{Schnitzler, Arthur 15.\,5.\,1862 Wien – 21.\,10.\,1931 ebd.@\textsc{Schnitzler, Arthur} (15.\,5.\,1862 Wien – 21.\,10.\,1931 ebd.), \emph{Schriftsteller, Mediziner}!Schleier der Beatrice. Schauspiel in fünf Akten@\strich\emph{Der Schleier der Beatrice. Schauspiel in fünf Akten}|pwk}, siehe XXXX Auszeichnungsfehler: Dokument L03213 nicht gefunden.}}}\label{K_L03216-2},{ }ſelbſt \label{K_L03216-3v}\edtext{\uline{nach}{ }\textsc{Monna Vanna\pwindex{\textcolor{red}{\textsuperscript{XXXX indx1}}!Monna Vanna. Schauspiel in drei Akten@\strich\emph{Monna Vanna. Schauspiel in drei Akten}|pw}}}{\lemma{\textnormal{\emph{nach Monna Vanna}}}\Cendnote{\textnormal{Siehe XXXX Auszeichnungsfehler: Dokument L03211 nicht gefunden.
               }}}\label{K_L03216-3}, iſt beſſer als das Schillertheater\orgindex{Schiller-Theater@Schiller-Theater|pw}.\pend
           
\pstart
           Viele Grüße {\\[\baselineskip]}Dein {\\[\baselineskip]}\spacefill\mbox{P. Goldm}\pend
           \leftskip=0em{}\selectlanguage{ngerman}\endnumbering\briefempfaengerindex{Schnitzler, Arthur@\textsc{Schnitzler, Arthur}!zzzGoldmann, Paul@\emph{von Paul Goldmann}!1902-07-311@{31. 7. 1902}|)be}\mylabel{L03216h}  \newcommand{\dateiname}{L03216}\newcommand{\titel}{Paul Goldmann an Arthur Schnitzler, 31. 7. 1902}\newcommand{\editorInnen}{Martin Anton Müller und Laura Untner}%% latex-leseansicht-abspann.tex
%% Abspann für die Leseansicht.
%% Der Schalter \ifkorrekturansicht ist bereits durch den Vorspann gesetzt.

%% latex-abspann.tex
%% Gemeinsamer Abspann für Korrekturansicht und Leseansicht.
%% Setzt den Schalter \ifkorrekturansicht voraus (gesetzt in den
%% einbindenden Dateien latex-korrekturansicht-abspann.tex bzw.
%% latex-leseansicht-abspann.tex).
%% ---------------------------------------------------------------

\normalsize

% Das esempio-Environment wird nur in der Leseansicht benötigt
\ifkorrekturansicht\else
\newenvironment{esempio}[3]%
{
    \vspace{1.5ex}
    \rlap{\underline{#1}}
    \par
    \setlength{\parindent}{0cm}
    \nopagebreak
    \leftskip=#2cm
    \rightskip=#3cm
}
{
    \par
}
\fi

\doendnotes{C}
\bigskip
\vfill

\clearpage

\footnotesize

\ifkorrekturansicht
  \lohead{\textsc{register}}
\fi

% theindex-Environment neu definieren ohne reledmac
\makeatletter
\renewenvironment{theindex}{%
  \ifkorrekturansicht
    \section*{\indexname}%
  \else
    \subsubsection*{Index der erwähnten Entitäten}%
  \fi
  \setlength{\parindent}{0pt}%
  \setlength{\parskip}{0pt plus 0.3pt}%
  \let\item\@idxitem
}{%
  \ifkorrekturansicht\clearpage\fi
}
\makeatother

\IfFileExists{\jobname-pw.ind}{\input{\jobname-pw.ind}}{}

% Quellenangabe nur in der Leseansicht
\ifkorrekturansicht\else
% Fallback-Definitionen, falls die .tex-Datei \titel etc. nicht gesetzt hat
\providecommand{\titel}{}
\providecommand{\editorInnen}{}
\providecommand{\dateiname}{\jobname}

\vspace{3cm}

\vfill

\footnotesize
\textsc{Quelle}: \titel. Herausgegeben von {\editorInnen}. In: \emph{Arthur Schnitzler: Briefwechsel mit Autorinnen und Autoren}.
 Digitale Edition, https://schnitzler-briefe.acdh.oeaw.ac.at/{\dateiname}.html (Stand \today)
\fi

\end{document}


