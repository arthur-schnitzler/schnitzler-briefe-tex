%% latex-leseansicht-vorspann.tex
%% Vorspann für die Leseansicht.
%% Lädt die gemeinsame Datei latex-vorspann.tex mit nicht gesetztem Schalter.

\newif\ifkorrekturansicht
\korrekturansichtfalse

\input{../tex-inputs/latex-vorspann}


\section[Arthur Schnitzler an Auguste Hauschner, 29.\,6.\,1908]{L02600 Arthur Schnitzler an Auguste Hauschner, 29.\,6.\,1908}
\nopagebreak\mylabel{L02600v}
\rehead{ }\normalsize\beginnumbering\briefempfaengerindex{Hauschner, Auguste@\textsc{Hauschner, Auguste}!zzzSchnitzler, Arthur@\emph{von Arthur Schnitzler}!1908-06-291@{29.\,6.\,1908}|(be}
\toendnotes[C]{\smallbreak\pagebreak[2]}
\correspDesc{Versand  durch Arthur Schnitzler am 29. 6. 1908 in Seis am Schlern
\newline{}Erhalt  durch Auguste Hauschner im Zeitraum [30. 6. 1908
                  – 4. 7. 1908?] in Berlin}\toendnotes[C]{\smallbreak}
\Standort{Staatsbibliothek Berlin – Preußischer Kulturbesitz, Handschriftenabteilung, Nachlass Auguste Hauschner.}
\physDesc{Briefkarte, 459 Zeichen
\newline{}Handschrift: schwarze Tinte, lateinische Kurrent}\toendnotes[C]{\smallbreak}
\pstart
           {\pb}\textcolor{gray}{\textbf{Dr. Arthur Schnitzler}}\hfill Seis am Schlern\oindex{Seis am Schlern@\textbf{Seis am Schlern}|pw},\pend
           
\pstart
           
\pstart
           \textcolor{gray}{\textbf{Wien, XVIII. Spoettelgasse 7\oindex{Wien@\textbf{Wien}!XVIII., Währing@\textbf{XVIII., Währing}!Edmund-Weiß-Gasse 7@\textbf{Edmund-Weiß-Gasse 7}, \emph{Wohngebäude}|pw}.}}\pend
           
\pstart
           \raggedleft{}29. 6. 08\pend
           \pend
           \vspace{0.5em}
\pstart
           verehrte Frau, Ihr Brief ist mir hieher nachgereist – dass er mich
               sehr gefreut hat, kö{\geminationn}en Sie sich wohl denken. Nun hab
               ich mir auch Ihr Buch\pwindex{Hauschner, Auguste 12.\,2.\,1850 Prag – 10.\,4.\,1924 Berlin@\textsc{Hauschner, Auguste} (12.\,2.\,1850 Prag – 10.\,4.\,1924 Berlin), \emph{Schriftstellerin}!Familie Lowositz. Roman@\strich\emph{Die Familie Lowositz. Roman}|pwv} aus Wien\oindex{Wien@\textbf{Wien}, \emph{Verwaltungsgebiet}|pw} herschicken lassen und bin sehr begierig Ihre
                  Beka{\geminationn}tschaft zu machen. De{\geminationn} ich kenne noch gar nichts von Ihnen – zu meinen
               Vorsätzen {\pb}gehört schon lange Zeit
                  »Kunst\pwindex{Hauschner, Auguste 12.\,2.\,1850 Prag – 10.\,4.\,1924 Berlin@\textsc{Hauschner, Auguste} (12.\,2.\,1850 Prag – 10.\,4.\,1924 Berlin), \emph{Schriftstellerin}!Kunst. Roman@\strich\emph{Kunst. Roman}|pw}« – über das mir kluge Leute das beste
               zu sagen wußten. Seien Sie herzlichst bedankt und gegrüßt!\pend
           
\pstart
           Ihr ergebener{\\[\baselineskip]}\spacefill\mbox{Arthur Schnitzler}\pend
           \leftskip=0em{}\selectlanguage{ngerman}\endnumbering\briefempfaengerindex{Hauschner, Auguste@\textsc{Hauschner, Auguste}!zzzSchnitzler, Arthur@\emph{von Arthur Schnitzler}!1908-06-291@{29.\,6.\,1908}|)be}\mylabel{L02600h}  \newcommand{\dateiname}{L02600}\newcommand{\titel}{Arthur Schnitzler an Auguste Hauschner, 29. 6. 1908}\newcommand{\editorInnen}{Martin Anton Müller und Laura Untner}%% latex-leseansicht-abspann.tex
%% Abspann für die Leseansicht.
%% Der Schalter \ifkorrekturansicht ist bereits durch den Vorspann gesetzt.

%% latex-abspann.tex
%% Gemeinsamer Abspann für Korrekturansicht und Leseansicht.
%% Setzt den Schalter \ifkorrekturansicht voraus (gesetzt in den
%% einbindenden Dateien latex-korrekturansicht-abspann.tex bzw.
%% latex-leseansicht-abspann.tex).
%% ---------------------------------------------------------------

\normalsize

% Das esempio-Environment wird nur in der Leseansicht benötigt
\ifkorrekturansicht\else
\newenvironment{esempio}[3]%
{
    \vspace{1.5ex}
    \rlap{\underline{#1}}
    \par
    \setlength{\parindent}{0cm}
    \nopagebreak
    \leftskip=#2cm
    \rightskip=#3cm
}
{
    \par
}
\fi

\doendnotes{C}
\bigskip
\vfill

\clearpage

\footnotesize

\ifkorrekturansicht
  \lohead{\textsc{register}}
\fi

% theindex-Environment neu definieren ohne reledmac
\makeatletter
\renewenvironment{theindex}{%
  \ifkorrekturansicht
    \section*{\indexname}%
  \else
    \subsubsection*{Index der erwähnten Entitäten}%
  \fi
  \setlength{\parindent}{0pt}%
  \setlength{\parskip}{0pt plus 0.3pt}%
  \let\item\@idxitem
}{%
  \ifkorrekturansicht\clearpage\fi
}
\makeatother

\IfFileExists{\jobname-pw.ind}{\input{\jobname-pw.ind}}{}

% Quellenangabe nur in der Leseansicht
\ifkorrekturansicht\else
% Fallback-Definitionen, falls die .tex-Datei \titel etc. nicht gesetzt hat
\providecommand{\titel}{}
\providecommand{\editorInnen}{}
\providecommand{\dateiname}{\jobname}

\vspace{3cm}

\vfill

\footnotesize
\textsc{Quelle}: \titel. Herausgegeben von {\editorInnen}. In: \emph{Arthur Schnitzler: Briefwechsel mit Autorinnen und Autoren}.
 Digitale Edition, https://schnitzler-briefe.acdh.oeaw.ac.at/{\dateiname}.html (Stand \today)
\fi

\end{document}


