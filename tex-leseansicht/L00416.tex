%% latex-leseansicht-vorspann.tex
%% Vorspann für die Leseansicht.
%% Lädt die gemeinsame Datei latex-vorspann.tex mit nicht gesetztem Schalter.

\newif\ifkorrekturansicht
\korrekturansichtfalse

\input{../tex-inputs/latex-vorspann}


         
         \renewcommand{\erwaehntePersonen}{Personen: Karl Costa, Max Weinzierl}
         \renewcommand{\erwaehnteOrte}{Orte: Frankgasse 1, I., Innere Stadt, IX., Alsergrund, Raimund-Theater, Wien}
         \renewcommand{\erwaehnteWerke}{Werke: Bruder Martin}
               \section[Richard Beer-Hofmann an Arthur Schnitzler, 28. 1. 1895]{ Richard Beer-Hofmann an Arthur Schnitzler, 28. 1. 1895}\nopagebreak\mylabel{v}\rehead{ }\begin{ledgroupsized}[t]{13cm}\normalsize\beginnumbering \toendnotes[C]{\smallbreak\pagebreak[2]} \Standort{CUL, Schnitzler, B 8.}
\physDesc{Postkarte, 212 Zeichen
\newline{}Handschrift: Bleistift, lateinische Kurrent
\newline{}Versand: 1) Rohrpost  2) Stempel: »\nobreak{}\oindex{I., Innere Stadt@\textbf{I., Innere Stadt}|pwk}Wien 1/1, 28 I 95, 3 40 N\nobreak{}«.  3) Stempel: »\nobreak{}\oindex{IX., Alsergrund@\textbf{IX., Alsergrund}|pwk}Wien 9/2, 28 I 95, 4 10 N\nobreak{}«. 
\newline{}Schnitzler: mit Bleistift nummeriert: »53« }\buchAbdrucke{\weitereDrucke{Arthur Schnitzler, Richard Beer-Hofmann: \emph{Briefwechsel 1891–1931}. Hg. Konstanze Fliedl. Wien, Zürich: \emph{Europaverlag} 1992, S. 70.} }\toendnotes[C]{\smallbreak}\pstart{}{\pb}Herrn\pend{}\pstart{}D\textsuperscript{r} Arthur Schnitzler\pend{}\pstart{}IX Frankgasse\oindex{Frankgasse 1@\textbf{Frankgasse 1}|pw} 1\pend{}{\bigskip}\pstart
           \noindent{}{\pb}Lieber Arthur! Wo haben Sie Ihren schwarzen So{\geminationm}erstrohhut gekauft? Morgen ist nämlich \label{K_L00416-1v}\edtext{Raimundtheater\oindex{Raimund-Theater@\textbf{Raimund-Theater}|pw}abend}{\lemma{\textnormal{\emph{Raimundtheaterabend}}}\Cendnote{\textnormal{Am 29. 1. 1895 wurde im Raimund-Theater\oindex{Raimund-Theater@\textbf{Raimund-Theater}|pwk} das Volksstück \emph{Bruder
                     Martin}\pwindex{Bruder Martin1892@\emph{Bruder Martin} {[}1892{]}|pwk} von Karl Costa\pwindex{Costa, Karl 02.02.1832 – 11.10.1907@\textsc{Costa, Karl} (02.02.1832 – 11.10.1907), \emph{Schriftsteller, Journalist, Theaterdirektor}|pwk} mit der Musik
                  von Max von Weinzierl\pwindex{Weinzierl, Max 1841-09-16 – 1898-07-10@\textsc{Weinzierl, Max} (1841-09-16 – 1898-07-10), \emph{Komponist}|pwk}
               gegeben.}}}\label{K_L00416-1h}. –\pend
           \pstart
           Ich gehe vielleicht, – fast sicher wenn Sie gehen. Herzlichst\pend
           \pstart \spacefill\mbox{Richard}\pend{}
         
         \endnumbering\mylabel{h}\end{ledgroupsized}  \newcommand{\dateiname}{L00416}\newcommand{\titel}{Richard Beer-Hofmann an Arthur Schnitzler, 28. 1. 1895}\newcommand{\editorInnen}{Martin Anton Müller und Gerd-Hermann Susen}%% latex-leseansicht-abspann.tex
%% Abspann für die Leseansicht.
%% Der Schalter \ifkorrekturansicht ist bereits durch den Vorspann gesetzt.

%% latex-abspann.tex
%% Gemeinsamer Abspann für Korrekturansicht und Leseansicht.
%% Setzt den Schalter \ifkorrekturansicht voraus (gesetzt in den
%% einbindenden Dateien latex-korrekturansicht-abspann.tex bzw.
%% latex-leseansicht-abspann.tex).
%% ---------------------------------------------------------------

\normalsize

% Das esempio-Environment wird nur in der Leseansicht benötigt
\ifkorrekturansicht\else
\newenvironment{esempio}[3]%
{
    \vspace{1.5ex}
    \rlap{\underline{#1}}
    \par
    \setlength{\parindent}{0cm}
    \nopagebreak
    \leftskip=#2cm
    \rightskip=#3cm
}
{
    \par
}
\fi

\doendnotes{C}
\bigskip
\vfill

\clearpage

\footnotesize

\ifkorrekturansicht
  \lohead{\textsc{register}}
\fi

% theindex-Environment neu definieren ohne reledmac
\makeatletter
\renewenvironment{theindex}{%
  \ifkorrekturansicht
    \section*{\indexname}%
  \else
    \subsubsection*{Index der erwähnten Entitäten}%
  \fi
  \setlength{\parindent}{0pt}%
  \setlength{\parskip}{0pt plus 0.3pt}%
  \let\item\@idxitem
}{%
  \ifkorrekturansicht\clearpage\fi
}
\makeatother

\IfFileExists{\jobname-pw.ind}{\input{\jobname-pw.ind}}{}

% Quellenangabe nur in der Leseansicht
\ifkorrekturansicht\else
% Fallback-Definitionen, falls die .tex-Datei \titel etc. nicht gesetzt hat
\providecommand{\titel}{}
\providecommand{\editorInnen}{}
\providecommand{\dateiname}{\jobname}

\vspace{3cm}

\vfill

\footnotesize
\textsc{Quelle}: \titel. Herausgegeben von {\editorInnen}. In: \emph{Arthur Schnitzler: Briefwechsel mit Autorinnen und Autoren}.
 Digitale Edition, https://schnitzler-briefe.acdh.oeaw.ac.at/{\dateiname}.html (Stand \today)
\fi

\end{document}


      