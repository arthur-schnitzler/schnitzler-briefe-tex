%% latex-korrekturansicht-vorspann.tex
%% Vorspann für die Korrekturansicht.
%% Lädt die gemeinsame Datei latex-vorspann.tex mit gesetztem Schalter.

\newif\ifkorrekturansicht
\korrekturansichttrue

\input{../tex-inputs/latex-vorspann}


\section[Richard Beer-Hofmann an Arthur Schnitzler, 28. 1. 1895]{L00416 Richard Beer-Hofmann an Arthur Schnitzler, 28. 1. 1895}
\nopagebreak\mylabel{L00416v}
\rehead{ }\normalsize\beginnumbering\briefempfaengerindex{Schnitzler, Arthur@\textsc{Schnitzler, Arthur}!zzzBeer-Hofmann, Richard@\emph{von Richard Beer-Hofmann}!1895-01-281@{28. 1. 1895}|(be}
\toendnotes[C]{\smallbreak\pagebreak[2]}\Standort{CUL, Schnitzler, B 8.}
\physDesc{Postkarte, 212 Zeichen
\newline{}Handschrift: Bleistift, lateinische Kurrent
\newline{}Versand: 1) Rohrpost  2) Stempel: »\nobreak{}\oindex{I., Innere Stadt@\textbf{I., Innere Stadt}, \emph{A.ADM3}|pwk}Wien 1/1, 28 I 95, 3 40 N\nobreak{}«.  3) Stempel: »\nobreak{}\oindex{IX., Alsergrund@\textbf{IX., Alsergrund}, \emph{A.ADM3}|pwk}Wien 9/2, 28 I 95, 4 10 N\nobreak{}«. 
\newline{}Schnitzler: mit Bleistift nummeriert: »53« }
\buchAbdrucke{\weitereDrucke{Arthur Schnitzler, Richard Beer-Hofmann: \emph{Briefwechsel 1891–1931}. Wien, Zürich: \emph{Europaverlag} 1992, S. 70.} }\toendnotes[C]{\smallbreak}\pstart{}{\pb}Herrn\pend{}\pstart{}D\textsuperscript{r} Arthur Schnitzler\pend{}\pstart{}IX Frankgasse\oindex{Frankgasse 1@\textbf{Frankgasse 1}, \emph{Wohngebäude (K.WHS)}|pw} 1\pend{}{\bigskip}\vspace{1em}
\pstart
           \noindent{}{\pb}Lieber Arthur! Wo haben Sie Ihren schwarzen So{\geminationm}erstrohhut gekauft? Morgen ist nämlich \label{K_L00416-1v}\edtext{Raimundtheater\oindex{Raimund-Theater@\textbf{Raimund-Theater}, \emph{Theater (K.THE)}|pw}abend}{\lemma{\textnormal{\emph{Raimundtheaterabend}}}\Cendnote{\textnormal{Am 29. 1. 1895 wurde im Raimund-Theater\oindex{Raimund-Theater@\textbf{Raimund-Theater}, \emph{Theater (K.THE)}|pwk} das Volksstück \emph{Bruder
                     Martin}\pwindex{Bruder Martin@\emph{Bruder Martin}|pwk} von Karl Costa\pwindex{Costa, Karl 02.02.1832 – 11.10.1907@\textsc{Costa, Karl} (02.02.1832 – 11.10.1907), \emph{Schriftsteller/Schriftstellerin, Journalist/Journalistin, Theaterdirektor/Theaterdirektorin}|pwk} mit der Musik
                  von Max von Weinzierl\pwindex{Weinzierl, Max 1841-09-16 – 1898-07-10@\textsc{Weinzierl, Max} (1841-09-16 – 1898-07-10), \emph{Komponist/Komponistin}|pwk}
               gegeben.}}}\label{K_L00416-1}. –\pend
           
\pstart
           Ich gehe vielleicht, – fast sicher wenn Sie gehen. Herzlichst\pend
           \pstart \spacefill\mbox{Richard}\pend{}\selectlanguage{ngerman}\endnumbering\briefempfaengerindex{Schnitzler, Arthur@\textsc{Schnitzler, Arthur}!zzzBeer-Hofmann, Richard@\emph{von Richard Beer-Hofmann}!1895-01-281@{28. 1. 1895}|)be}\mylabel{L00416h}  \normalsize

\doendnotes{C}
\bigskip
\vfill

\clearpage

\footnotesize

\lohead{\textsc{register}}

% Definiere theindex-Environment komplett neu ohne reledmac
\makeatletter
\renewenvironment{theindex}{%
  \section*{\indexname}%
  \setlength{\parindent}{0pt}%
  \setlength{\parskip}{0pt plus 0.3pt}%
  \let\item\@idxitem
}{%
  \clearpage
}
\makeatother

\IfFileExists{\jobname-pw.ind}{\input{\jobname-pw.ind}}{}

\end{document}

      