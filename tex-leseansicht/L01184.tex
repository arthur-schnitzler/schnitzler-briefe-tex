%% latex-korrekturansicht-vorspann.tex
%% Vorspann für die Korrekturansicht.
%% Lädt die gemeinsame Datei latex-vorspann.tex mit gesetztem Schalter.

\newif\ifkorrekturansicht
\korrekturansichttrue

\input{../tex-inputs/latex-vorspann}


\section[Hermann Bahr an Arthur Schnitzler, 27. 10. {[}1901{]}]{L01184 Hermann Bahr an Arthur Schnitzler, 27. 10. {[}1901{]}}
\nopagebreak\mylabel{L01184v}
\rehead{ }\normalsize\beginnumbering\briefempfaengerindex{Schnitzler, Arthur@\textsc{Schnitzler, Arthur}!zzzBahr, Hermann@\emph{von Hermann Bahr}!1901-10-271@{27. 10. 1901}|(be}
\toendnotes[C]{\smallbreak\pagebreak[2]}\Standort{CUL, Schnitzler, B 5b.}
\physDesc{Brief, 1 Blatt, 3 Seiten, 723 Zeichen
\newline{}Handschrift: blaue Tinte, deutsche Kurrent
\newline{}Schnitzler: mit Bleistift die Jahreszahl »901« ergänzt 
\newline{}Ordnung: mit Bleistift von unbekannter Hand nummeriert:
                                    »82« }
\buchAbdrucke{\weitereDrucke{Hermann Bahr, Arthur Schnitzler: \emph{Briefwechsel, Aufzeichnungen, Dokumente (1891–1931)}. Göttingen: \emph{Wallstein} 2018, S. 216–217.} }\toendnotes[C]{\smallbreak}
\pstart
           \raggedleft{}{\pb}27. 10.\pend
           
\pstart\center{}Lieber Arthur!\pend\vspace{0.5em}
\pstart
           Für Deinen lieben Brief danke ich Dir ſehr. – Die Pantomime\pwindex{Pantomime vom braven Manne@\emph{Die Pantomime vom braven Manne}|pwv} finde ich ſehr, ſehr ſchlecht; ich habe ſie nur
               abgedruckt, um den Berlin\oindex{Berlin@\textbf{Berlin}, \emph{P.PPLC}|pw}ern mitzutheilen, daß
               ich ſchon 1892{ }\textsc{en plein naturalisme} Pantomimen gemacht habe (wie übrigens
               Du und Hugo\pwindex{Hofmannsthal, Hugo von 1874-02-01 – 1929-07-15@\textsc{Hofmannsthal, Hugo von} (1874-02-01 – 1929-07-15), \emph{Schriftsteller/Schriftstellerin}|pw} und Richard\pwindex{Beer-Hofmann, Richard 1866-07-11 – 1945-09-26@\textsc{Beer-Hofmann, Richard} (1866-07-11 – 1945-09-26), \emph{Schriftsteller/Schriftstellerin}|pw} auch).\pend
           
\pstart
           {\pb}Mit Baron \textsc{Berger}\pwindex{Berger, Alfred von 30.04.1853 – 24.08.1912@\textsc{Berger, Alfred von} (30.04.1853 – 24.08.1912), \emph{Schriftsteller/Schriftstellerin, Journalist/Journalistin, Theaterleiter/Theaterleiterin}|pw} habe ich lange über Deine Stücke geſprochen: er hält die »letzten Maſken\pwindex{letzten Masken@\emph{Die letzten Masken}|pw}« und »Literatur\pwindex{Literatur@\emph{Literatur}|pw}« für »Meiſterwerke erſten Ranges«, während er für das Sceniſche
               der »Frau mit dem Dolch\pwindex{Frau mit dem Dolche@\emph{Die Frau mit dem Dolche}|pw}« Angſt zu haben
               ſcheint.\pend
           
\pstart
           Wenn Du mit \textsc{Bukovics}\pwindex{Bukovics, Emerich von 28.02.1844 – 04.07.1905@\textsc{Bukovics, Emerich von} (28.02.1844 – 04.07.1905), \emph{Journalist/Journalistin, Theaterleiter/Theaterleiterin}|pw} nicht energiſcher biſt, ſage ich Dir {\pb}voraus,
               daß Du in dieser Saiſon nicht mehr dran kommſt.\pend
           
\pstart
           \label{K_L01184-1v}\edtext{Raſend}{\lemma{\textnormal{\emph{Raſend}}}\Cendnote{\textnormal{In seiner Besprechung der Inszenierung von Gerhart Hauptmanns\pwindex{Hauptmann, Gerhart 15.11.1862 – 06.06.1946@\textsc{Hauptmann, Gerhart} (15.11.1862 – 06.06.1946), \emph{Schriftsteller/Schriftstellerin}|pwk}{ }Stück\pwindex{Einsame Menschen. Drama@\emph{Einsame Menschen. Drama}|pwkv}, \emph{Berliner Theater.
                     »Einsame Menschen« im Deutschen Theater}\pwindex{Berliner Theater. »Einsame Menschen« im Deutschen Theater@\emph{Berliner Theater. »Einsame Menschen« im Deutschen Theater}|pwk} (\emph{Neue Freie Presse}\pwindex{Neue Freie Presse@\emph{Neue Freie Presse}|pwk}, Nr. 13.345,
                        19. 10. 1901, S. 1–3), nennt Goldmann\pwindex{Goldmann, Paul 31.01.1865 – 25.09.1935@\textsc{Goldmann, Paul} (31.01.1865 – 25.09.1935), \emph{Schriftsteller/Schriftstellerin, Journalist/Journalistin}|pwk} die jüngeren Bühnenschriftsteller unfähig zum
                  Dramatischem; diese hätten ihre Schwäche zum Ideal erhoben und dabei das Theater
                  langweilig gemacht.}}}\label{K_L01184-1} war ich über Goldmanns\pwindex{Goldmann, Paul 31.01.1865 – 25.09.1935@\textsc{Goldmann, Paul} (31.01.1865 – 25.09.1935), \emph{Schriftsteller/Schriftstellerin, Journalist/Journalistin}|pw}{ }Feuilleton »Einſame Menſchen\pwindex{Einsame Menschen. Drama@\emph{Einsame Menschen. Drama}|pw}«\pwindex{Berliner Theater. »Einsame Menschen« im Deutschen Theater@\emph{Berliner Theater. »Einsame Menschen« im Deutschen Theater}|pwv}. Das ſollte wirklich polizeilich
               verboten ſein.\pend
           
\pstart
           Herzlichſt{\\[\baselineskip]}Dein{\\[\baselineskip]}\spacefill\mbox{Hermann}\pend
           \leftskip=0em{}\selectlanguage{ngerman}\endnumbering\briefempfaengerindex{Schnitzler, Arthur@\textsc{Schnitzler, Arthur}!zzzBahr, Hermann@\emph{von Hermann Bahr}!1901-10-271@{27. 10. 1901}|)be}\mylabel{L01184h}  \normalsize

\doendnotes{C}
\bigskip
\vfill

\clearpage

\footnotesize

\lohead{\textsc{register}}

% Definiere theindex-Environment komplett neu ohne reledmac
\makeatletter
\renewenvironment{theindex}{%
  \section*{\indexname}%
  \setlength{\parindent}{0pt}%
  \setlength{\parskip}{0pt plus 0.3pt}%
  \let\item\@idxitem
}{%
  \clearpage
}
\makeatother

\IfFileExists{\jobname-pw.ind}{\input{\jobname-pw.ind}}{}

\end{document}

      