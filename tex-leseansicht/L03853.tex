%% latex-leseansicht-vorspann.tex
%% Vorspann für die Leseansicht.
%% Lädt die gemeinsame Datei latex-vorspann.tex mit nicht gesetztem Schalter.

\newif\ifkorrekturansicht
\korrekturansichtfalse

\input{../tex-inputs/latex-vorspann}


\section[Theodor Herzl an Arthur Schnitzler, {[}zwischen 8. und 17. 3. 1895?{]}]{L03853 Theodor Herzl an Arthur Schnitzler, {[}zwischen 8. und 17. 3. 1895?{]}}
\nopagebreak\mylabel{L03853v}
\rehead{ }\normalsize\beginnumbering\briefempfaengerindex{Schnitzler, Arthur@\textsc{Schnitzler, Arthur}!zzzHerzl, Theodor@\emph{von Theodor Herzl}!1895-03-171@{{[}zwischen 8. und 17. 3. 1895?{]}}|(be}
\toendnotes[C]{\smallbreak\pagebreak[2]}
\correspDesc{Versand  durch Theodor Herzl im Zeitraum [zwischen 8. und
                  17. 3. 1895?] in Paris
\newline{}Erhalt  durch Arthur Schnitzler in Wien}\toendnotes[C]{\smallbreak}
\Standort{CUL, Schnitzler, B 39.}
\physDesc{Brief, 1 Blatt, 3 Seiten, 1228 Zeichen, Fragment (Textverlust durch Ausriss der oberen Blattecke)
\newline{}Handschrift: schwarze Tinte, lateinische Kurrent
\newline{}Schnitzler: mit Bleistift datiert: »Mitt März 95« 
\newline{}Ordnung: mit Bleistift von unbekannter Hand nummeriert: »32« }
\buchAbdrucke{\weitereDrucke{Theodor Herzl: \emph{Briefe und autobiographische Notizen 1866–1895}. Bearbeitet von Johannes Wachten in Zusammenarbeit mit Chaya Harel, Daisy Tycho und Manfred Winkler. Berlin, Frankfurt am Main, Wien: \emph{Propyläen} 1983, S. 579 (Briefe und Tagebücher. Herausgegeben von Alex Bein, Hermann Greive, Moshe Schaerf, Julius H. Schoeps und Johannes Wachten, 1).} }\toendnotes[C]{\smallbreak}
\pstart{}{\pb}Mein lieber \damage{Freund,}\pend\vspace{0.5em}
\pstart
           \label{K_L03853-1v}\edtext{Nächster Tage}{\lemma{\textnormal{\emph{Nächster Tage}}}\Cendnote{\textnormal{Die Datierung des Briefes befand sich
                  auf der ausgerissenen Ecke. Herzl\pwindex{Herzl, Theodor 2.\,5.\,1860 Budapest – 3.\,7.\,1904 Edlach@\textsc{Herzl, Theodor} (2.\,5.\,1860 Budapest – 3.\,7.\,1904 Edlach), \emph{Schriftsteller, Journalist}|pwk} fordert
                     Schnitzler auf, ihm bis auf weiteres
                  nicht nach Paris\oindex{Paris@\textbf{Paris}, \emph{Hauptstadt}|pwk} zu schreiben. Also lag die
                  Abfassung des Briefes nach Absenden von Schnitzlers vorangegangenem Brief vom refXXXX8.3.1895. Die angekündigte
                  Abreise nach Wien\oindex{Paris@\textbf{Paris}, \emph{Hauptstadt}|pwk} terminiert Herzl\pwindex{Herzl, Theodor 2.\,5.\,1860 Budapest – 3.\,7.\,1904 Edlach@\textsc{Herzl, Theodor} (2.\,5.\,1860 Budapest – 3.\,7.\,1904 Edlach), \emph{Schriftsteller, Journalist}|pwk} auf frühestens »Mitte der nächsten Woche«.
                  Tatsächlich reiste er am 24. 3. 1895 ab, wie aus seinem Brief vom
                  XXXX Auszeichnungsfehler: Dokument L03854 nicht gefunden hervorgeht. Daraus ergibt sich der 17. 3. 1895 als
                  letzten Tag der Vorwoche vor der Abreise als hinteres Datum der Zeitspanne,
                  innerhalb derer der Brief abgefasst wurde.}}}\label{K_L03853-1} we\damage{rde ich mich} wahrscheinlich entschl\damage{ießen n}ach Wien\oindex{Wien@\textbf{Wien}, \emph{Verwaltungsgebiet}|pw} zu fahren, um m\damage{ei}ne Eltern\pwindex{Herzl, Jakob 14.\,3.\,1837 Zemun – 9.\,6.\,1902 Wien@\textsc{Herzl, Jakob} (14.\,3.\,1837 Zemun – 9.\,6.\,1902 Wien), \emph{Bankdirektor, Großkaufmann}|pwv}\pwindex{Herzl, Jeanette 28.\,7.\,1836 Budapest – 20.\,2.\,1911 Wien@\textsc{Herzl, Jeanette} (28.\,7.\,1836 Budapest – 20.\,2.\,1911 Wien)|pwv} zu besuchen, die ich jetzt schon über ein halbes Jahr nicht gesehen
               habe.\pend
           
\pstart
           Wann ich fahre, weiss ich noch nicht. Aber da es ebensowol Mitte der nächsten Woche als knapp vor \label{K_L03853-2v}\edtext{Ostern}{\lemma{\textnormal{\emph{Ostern}}}\Cendnote{\textnormal{1895 fiel der Ostersonntag auf den 14. 4. 1895.}}}\label{K_L03853-2} sein kann, bitte ich Sie \label{K_L03853-3v}\edtext{mir und Albert}{\lemma{\textnormal{\emph{mir und Albert}}}\Cendnote{\textnormal{Zu Herzls\pwindex{Herzl, Theodor 2.\,5.\,1860 Budapest – 3.\,7.\,1904 Edlach@\textsc{Herzl, Theodor} (2.\,5.\,1860 Budapest – 3.\,7.\,1904 Edlach), \emph{Schriftsteller, Journalist}|pwk} Anweisungen
                  bezüglich der klandestinen Kommunikation über sein unter dem Pseudonym Albert
                  Schnabel eingereichtes Schauspiel \emph{Das neue
                     Ghetto}\pwindex{Herzl, Theodor 2.\,5.\,1860 Budapest – 3.\,7.\,1904 Edlach@\textsc{Herzl, Theodor} (2.\,5.\,1860 Budapest – 3.\,7.\,1904 Edlach), \emph{Schriftsteller, Journalist}!neue Ghetto. Schauspiel in vier Acten@\strich\emph{Das neue Ghetto. Schauspiel in vier Acten}|pwk}{ }vgl. XXXX Auszeichnungsfehler: Dokument L03836 nicht gefunden und XXXX Auszeichnungsfehler: Dokument L03844 nicht gefunden.}}}\label{K_L03853-3} nicht mehr zu schreiben, auch nicht zu telegraphiren. Brief
               oder Depesche könnten die Geschichte ablüften, da in meiner Abwesenheit mein Sekretär\pwindex{?? [Sekretär von Theodor Herzl in Paris] @\textsc{?? [Sekretär von Theodor Herzl in Paris]}|pw} mich vertritt, u. Ihre Depesche würde
               ihn {\pb}\damage{\textcolor{gray}{×}\-\textcolor{gray}{×}\-\textcolor{gray}{×}\-\textcolor{gray}{×}{ }verwi}rren.\pend
           
\pstart
           \damage{\textcolor{gray}{×}\-\textcolor{gray}{×}\-\textcolor{gray}{×}\-\textcolor{gray}{×}\-\textcolor{gray}{×}\-\textcolor{gray}{×}\-\textcolor{gray}{×}\-\textcolor{gray}{×}\-\textcolor{gray}{×}\-\textcolor{gray}{×}\-\textcolor{gray}{×}\-\textcolor{gray}{×}}ch wirklich \textcolor{gray}{×}\-\textcolor{gray}{×}\-\textcolor{gray}{×}\-\textcolor{gray}{×}\-\textcolor{gray}{×}\-\textcolor{gray}{×}\-\textcolor{gray}{×}\-\textcolor{gray}{×}\-\textcolor{gray}{×}\-\textcolor{gray}{×} Ich
               glaube, \textcolor{gray}{×}\-\textcolor{gray}{×}\-\textcolor{gray}{×}\-\textcolor{gray}{×}\-\textcolor{gray}{×}\-\textcolor{gray}{×}\-\textcolor{gray}{×}\-\textcolor{gray}{×}\-\textcolor{gray}{×}\-\textcolor{gray}{×}ch schon todt, \textcolor{gray}{×}\-\textcolor{gray}{×}\-\textcolor{gray}{×}\-\textcolor{gray}{×}\-\textcolor{gray}{×}\-\textcolor{gray}{×} ergebener Diener s\damage{ein}. Mir scheint sogar, Sie verschweigen mir Müllers\pwindex{Müller-Guttenbrunn, Adam 22.\,10.\,1852 Zăbrani – 5.\,1.\,1923 Wien@\textsc{Müller-Guttenbrunn, Adam} (22.\,10.\,1852 Zăbrani – 5.\,1.\,1923 Wien), \emph{Schriftsteller, Theaterleiter, Beamter}|pw} abschlägigen Bescheid, um mich nicht zu kränken. Es wäre lieb, es
               ist nicht nöthig.\pend
           
\pstart
           Wenn Sie wüssten, wieviel ich an Misserfolgen u. verschiedenem Missgeschick schon
               verwunden habe! Nicht leicht, aber hinter einer standhaften und hochmüthigen Maske.
               So wirds auch damit gehen. Uebrigens plaudern wir uns darüber gut aus, wenn ich in
                  Wien\oindex{Wien@\textbf{Wien}, \emph{Verwaltungsgebiet}|pw} bin. Ich werde \strikeout{übrigens} nur ein paar Minuten dort {\pb}bleiben können u. bei meinen Eltern\pwindex{Herzl, Jakob 14.\,3.\,1837 Zemun – 9.\,6.\,1902 Wien@\textsc{Herzl, Jakob} (14.\,3.\,1837 Zemun – 9.\,6.\,1902 Wien), \emph{Bankdirektor, Großkaufmann}|pwv}\pwindex{Herzl, Jeanette 28.\,7.\,1836 Budapest – 20.\,2.\,1911 Wien@\textsc{Herzl, Jeanette} (28.\,7.\,1836 Budapest – 20.\,2.\,1911 Wien)|pwv} wohnen.\pend
           
\pstart
           Erwarten Sie mich zwischen dem 21 März u. 21 Juni.\pend
           
\pstart
           Ich meine: im Frühling.\pend
           
\pstart
           Herzlich Ihr Freund{\\[\baselineskip]}\spacefill\mbox{Th. H.}\pend
           \leftskip=0em{}
\pstart
           \noindent{}Wenn mir etwas dazwischen kommen sollte, melde ich mich Anfang der nächsten Woche bei Ihnen ab.\pend
           \selectlanguage{ngerman}\endnumbering\briefempfaengerindex{Schnitzler, Arthur@\textsc{Schnitzler, Arthur}!zzzHerzl, Theodor@\emph{von Theodor Herzl}!1895-03-081@{{[}zwischen 8. und 17. 3. 1895?{]}}|)be}\mylabel{L03853h}
\begin{anhang}
\end{anhang}\newcommand{\dateiname}{L03853}\newcommand{\titel}{Theodor Herzl an Arthur Schnitzler, [zwischen 8. und 17. 3. 1895?]}\newcommand{\editorInnen}{Herausgegeben von Jahnke, SelmaMüller, Martin Anton}%% latex-leseansicht-abspann.tex
%% Abspann für die Leseansicht.
%% Der Schalter \ifkorrekturansicht ist bereits durch den Vorspann gesetzt.

%% latex-abspann.tex
%% Gemeinsamer Abspann für Korrekturansicht und Leseansicht.
%% Setzt den Schalter \ifkorrekturansicht voraus (gesetzt in den
%% einbindenden Dateien latex-korrekturansicht-abspann.tex bzw.
%% latex-leseansicht-abspann.tex).
%% ---------------------------------------------------------------

\normalsize

% Das esempio-Environment wird nur in der Leseansicht benötigt
\ifkorrekturansicht\else
\newenvironment{esempio}[3]%
{
    \vspace{1.5ex}
    \rlap{\underline{#1}}
    \par
    \setlength{\parindent}{0cm}
    \nopagebreak
    \leftskip=#2cm
    \rightskip=#3cm
}
{
    \par
}
\fi

\doendnotes{C}
\bigskip
\vfill

\clearpage

\footnotesize

\ifkorrekturansicht
  \lohead{\textsc{register}}
\fi

% theindex-Environment neu definieren ohne reledmac
\makeatletter
\renewenvironment{theindex}{%
  \ifkorrekturansicht
    \section*{\indexname}%
  \else
    \subsubsection*{Index der erwähnten Entitäten}%
  \fi
  \setlength{\parindent}{0pt}%
  \setlength{\parskip}{0pt plus 0.3pt}%
  \let\item\@idxitem
}{%
  \ifkorrekturansicht\clearpage\fi
}
\makeatother

\IfFileExists{\jobname-pw.ind}{\input{\jobname-pw.ind}}{}

% Quellenangabe nur in der Leseansicht
\ifkorrekturansicht\else
% Fallback-Definitionen, falls die .tex-Datei \titel etc. nicht gesetzt hat
\providecommand{\titel}{}
\providecommand{\editorInnen}{}
\providecommand{\dateiname}{\jobname}

\vspace{3cm}

\vfill

\footnotesize
\textsc{Quelle}: \titel. Herausgegeben von {\editorInnen}. In: \emph{Arthur Schnitzler: Briefwechsel mit Autorinnen und Autoren}.
 Digitale Edition, https://schnitzler-briefe.acdh.oeaw.ac.at/{\dateiname}.html (Stand \today)
\fi

\end{document}


