%% latex-korrekturansicht-vorspann.tex
%% Vorspann für die Korrekturansicht.
%% Lädt die gemeinsame Datei latex-vorspann.tex mit gesetztem Schalter.

\newif\ifkorrekturansicht
\korrekturansichttrue

\input{../tex-inputs/latex-vorspann}


\section[Hugo von Hofmannsthal an Arthur Schnitzler, 3. 8. {[}1898{]}]{L00829 Hugo von Hofmannsthal an Arthur Schnitzler, 3. 8. {[}1898{]}}
\nopagebreak\mylabel{L00829v}
\rehead{ }\normalsize\beginnumbering\briefempfaengerindex{Schnitzler, Arthur@\textsc{Schnitzler, Arthur}!zzzHofmannsthal, Hugo von@\emph{von Hugo von Hofmannsthal}!1898-08-031@{3. 8. {[}1898{]}}|(be}
\toendnotes[C]{\smallbreak\pagebreak[2]}\Standort{CUL, Schnitzler, B 43.}
\physDesc{Brief, 1 Blatt, 4 Seiten, 923 Zeichen
\newline{}Handschrift: schwarze Tinte, deutsche Kurrent
\newline{}Schnitzler: mit Bleistift die Jahreszahl ergänzt: »98« 
\newline{}Ordnung: mit Bleistift von unbekannter Hand nummeriert:
                                    »119« }
\buchAbdrucke{\weitereDrucke{Hugo von Hofmannsthal, Arthur Schnitzler: \emph{Briefwechsel}. Frankfurt am Main: \emph{S. Fischer} 1964, S. 108.} }\toendnotes[C]{\smallbreak}
\pstart
           \raggedleft{}{\pb}Hinterbrühl\oindex{Hinterbruehl@\textbf{Hinterbrühl}, \emph{P.PPLA3}|pw}{\\}3 VIII.\pend
           
\pstart{}mein lieber Arthur\pend\vspace{0.5em}
\pstart
           ich bin ſehr froh, ſchreiben zu können, daſs es ja nun faſt ſicher zu dem ko{\geminationm}en wird, was wir uns beide gewünſcht haben und woran
               ich noch in \textsc{Czortków}\oindex{Tschortkiw@\textbf{Tschortkiw}, \emph{P.PPLA2}|pw} nicht ſehr feſt geglaubt habe.\pend
           
\pstart
           Bitte ſchreiben Sie mir jetzt {\pb}aber gleich hierher welchen Weg durch die Schweiz\oindex{Schweiz@\textbf{Schweiz}, \emph{A.PCLI}|pw} wir eigentlich vorhaben, damit ichs meinen Eltern\pwindex{Hofmannsthal, Hugo August von 21.12.1841 – 08.12.1915@\textsc{Hofmannsthal, Hugo August von} (21.12.1841 – 08.12.1915), \emph{Bankdirektor/Bankdirektorin}|pwv}\pwindex{Hofmannsthal, Anna von 27.01.1849 – 22.03.1904@\textsc{Hofmannsthal, Anna von} (27.01.1849 – 22.03.1904)|pwv}{ }ſagen kann. Ich hab gar keinen Wunſch als daſs die
               Tour ungefähr am 20\textsuperscript{\textsc{ten}} in der Gegend von Chur\oindex{Chur@\textbf{Chur}, \emph{P.PPLA}|pw} aufhören ſoll von
               wo man dann leicht über \textsc{Maloja}\oindex{Maloja@\textbf{Maloja}, \emph{A.ADM2}|pw} oder anders {\pb}in meine oberitalieniſche\oindex{Italien@\textbf{Italien}, \emph{A.PCLI}|pw}{ }Seengegend kommt. Dort möchte ich 14–20 Tage an
               einem Ort ruhig bleiben. Wunderſchön wäre es natürlich wenn Sie mit mir bleiben
               könnten, wir die Mahlzeiten und Abende und hie und da einen Unterbrechungstag {\pb}zuſa{\geminationm}en verbrächten.\pend
           
\pstart
           Ich denke am vormittag des 11\textsuperscript{\textsc{ten}} in Innsbruck\oindex{Innsbruck@\textbf{Innsbruck}, \emph{A.ADM2}|pw} zu ſein, höchſtens etwa um
                  \uline{einen} Tag ſpäter. Bitte antworten Sie auf dieſen
               Brief recht ſchnell, ob Ihnen alles recht iſt.\pend
           
\pstart
           Von Herzen Ihr{\\[\baselineskip]}\spacefill\mbox{Hugo.}\pend
           \leftskip=0em{}\selectlanguage{ngerman}\endnumbering\briefempfaengerindex{Schnitzler, Arthur@\textsc{Schnitzler, Arthur}!zzzHofmannsthal, Hugo von@\emph{von Hugo von Hofmannsthal}!1898-08-031@{3. 8. {[}1898{]}}|)be}\mylabel{L00829h}  \normalsize

\doendnotes{C}
\bigskip
\vfill

\clearpage

\footnotesize

\lohead{\textsc{register}}

% Definiere theindex-Environment komplett neu ohne reledmac
\makeatletter
\renewenvironment{theindex}{%
  \section*{\indexname}%
  \setlength{\parindent}{0pt}%
  \setlength{\parskip}{0pt plus 0.3pt}%
  \let\item\@idxitem
}{%
  \clearpage
}
\makeatother

\IfFileExists{\jobname-pw.ind}{\input{\jobname-pw.ind}}{}

\end{document}

      