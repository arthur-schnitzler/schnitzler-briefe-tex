%% latex-leseansicht-vorspann.tex
%% Vorspann für die Leseansicht.
%% Lädt die gemeinsame Datei latex-vorspann.tex mit nicht gesetztem Schalter.

\newif\ifkorrekturansicht
\korrekturansichtfalse

\input{../tex-inputs/latex-vorspann}


         
         \newcommand{\erwaehntePersonen}{Personen: }
         \newcommand{\erwaehnteInstitutionen}{}
         \newcommand{\erwaehnteOrte}{}
         \newcommand{\erwaehnteWerke}{
               \section[Hugo von Hofmannsthal an Arthur Schnitzler, 3. 8. {[}1898{]}]{ Hugo von Hofmannsthal an Arthur Schnitzler, 3. 8. {[}1898{]}}\nopagebreak\mylabel{v}\rehead{ }\begin{ledgroupsized}[t]{13cm}\normalsize\beginnumbering \toendnotes[C]{\smallbreak\pagebreak[2]} \Standort{CUL, Schnitzler, B 43.}
\physDesc{Brief, 1 Blatt, 4 Seiten
\newline{}Handschrift: schwarze Tinte, deutsche Kurrent
\newline{}Schnitzler: mit Bleistift die Jahreszahl ergänzt: »98« \newline{}Ordnung: mit Bleistift von unbekannter Hand nummeriert:
                                    »119« }\buchAbdrucke{\weitereDrucke{Hugo von Hofmannsthal, Arthur Schnitzler: \emph{Briefwechsel}. Hg. Therese Nickl und Heinrich Schnitzler. Frankfurt am Main: \emph{S. Fischer} 1964, S. 108.} }\toendnotes[C]{\smallbreak}\pstart
           \raggedleft{}{\pb}Hinterbrühl\oindex{XXXX Ortsangabe fehlt|pw}{\\}3 VIII.\pend
           \pstart{}mein lieber Arthur\pend\pstart
           ich bin ſehr froh, ſchreiben zu können, daſs es ja nun faſt ſicher zu dem ko{\geminationm}en wird, was wir uns beide gewünſcht haben und
                    woran ich noch in \textsc{Czortków}\oindex{XXXX Ortsangabe fehlt|pw} nicht ſehr feſt geglaubt habe.\pend
           \pstart
           Bitte ſchreiben Sie mir jetzt {\pb}aber gleich hierher welchen Weg durch die Schweiz\oindex{XXXX Ortsangabe fehlt|pw} wir eigentlich vorhaben, damit ichs meinen Eltern\pwindex{\textcolor{red}{\textsuperscript{XXXX1 indx}}|pwv}\pwindex{\textcolor{red}{\textsuperscript{XXXX1 indx}}|pwv}{ }ſagen kann. Ich hab gar keinen Wunſch als daſs die Tour
                    ungefähr am 20\textsuperscript{\textsc{ten}} in der Gegend von Chur\oindex{XXXX Ortsangabe fehlt|pw} aufhören ſoll
                    von wo man dann leicht über \textsc{Maloja}\oindex{XXXX Ortsangabe fehlt|pw} oder anders {\pb}in meine oberitalieniſche\oindex{XXXX Ortsangabe fehlt|pw}{ }Seengegend kommt. Dort möchte ich 14–20 Tage an einem Ort ruhig bleiben.
                    Wunderſchön wäre es natürlich wenn Sie mit mir bleiben könnten, wir die
                    Mahlzeiten und Abende und hie und da einen Unterbrechungstag {\pb}zuſa{\geminationm}en verbrächten.\pend
           \pstart
           Ich denke am vormittag des 11\textsuperscript{\textsc{ten}} in Innsbruck\oindex{XXXX Ortsangabe fehlt|pw} zu ſein, höchſtens etwa um
                        \uline{einen} Tag ſpäter. Bitte antworten Sie auf
                    dieſen Brief recht ſchnell, ob Ihnen alles recht iſt.\pend
           \pstart
           Von Herzen Ihr{\\[\baselineskip]}\spacefill\mbox{Hugo.}\pend
           \leftskip=0em{}
         
         \endnumbering\mylabel{h}\end{ledgroupsized}  \newcommand{\dateiname}{L00829}\newcommand{\titel}{Hugo von Hofmannsthal an Arthur Schnitzler, 3. 8. [1898]}\newcommand{\editorInnen}{Martin Anton Müller und Gerd-Hermann Susen}%% latex-leseansicht-abspann.tex
%% Abspann für die Leseansicht.
%% Der Schalter \ifkorrekturansicht ist bereits durch den Vorspann gesetzt.

%% latex-abspann.tex
%% Gemeinsamer Abspann für Korrekturansicht und Leseansicht.
%% Setzt den Schalter \ifkorrekturansicht voraus (gesetzt in den
%% einbindenden Dateien latex-korrekturansicht-abspann.tex bzw.
%% latex-leseansicht-abspann.tex).
%% ---------------------------------------------------------------

\normalsize

% Das esempio-Environment wird nur in der Leseansicht benötigt
\ifkorrekturansicht\else
\newenvironment{esempio}[3]%
{
    \vspace{1.5ex}
    \rlap{\underline{#1}}
    \par
    \setlength{\parindent}{0cm}
    \nopagebreak
    \leftskip=#2cm
    \rightskip=#3cm
}
{
    \par
}
\fi

\doendnotes{C}
\bigskip
\vfill

\clearpage

\footnotesize

\ifkorrekturansicht
  \lohead{\textsc{register}}
\fi

% theindex-Environment neu definieren ohne reledmac
\makeatletter
\renewenvironment{theindex}{%
  \ifkorrekturansicht
    \section*{\indexname}%
  \else
    \subsubsection*{Index der erwähnten Entitäten}%
  \fi
  \setlength{\parindent}{0pt}%
  \setlength{\parskip}{0pt plus 0.3pt}%
  \let\item\@idxitem
}{%
  \ifkorrekturansicht\clearpage\fi
}
\makeatother

\IfFileExists{\jobname-pw.ind}{\input{\jobname-pw.ind}}{}

% Quellenangabe nur in der Leseansicht
\ifkorrekturansicht\else
% Fallback-Definitionen, falls die .tex-Datei \titel etc. nicht gesetzt hat
\providecommand{\titel}{}
\providecommand{\editorInnen}{}
\providecommand{\dateiname}{\jobname}

\vspace{3cm}

\vfill

\footnotesize
\textsc{Quelle}: \titel. Herausgegeben von {\editorInnen}. In: \emph{Arthur Schnitzler: Briefwechsel mit Autorinnen und Autoren}.
 Digitale Edition, https://schnitzler-briefe.acdh.oeaw.ac.at/{\dateiname}.html (Stand \today)
\fi

\end{document}


      