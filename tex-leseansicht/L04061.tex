%% latex-leseansicht-vorspann.tex
%% Vorspann für die Leseansicht.
%% Lädt die gemeinsame Datei latex-vorspann.tex mit nicht gesetztem Schalter.

\newif\ifkorrekturansicht
\korrekturansichtfalse

\input{../tex-inputs/latex-vorspann}


\section[Arthur Schnitzler an Gustav Schwarzkopf, 14. 3. 1903]{L04061 Arthur Schnitzler an Gustav Schwarzkopf, 14. 3. 1903}
\nopagebreak\mylabel{L04061v}
\rehead{ }\normalsize\beginnumbering\briefempfaengerindex{Schwarzkopf, Gustav@\textsc{Schwarzkopf, Gustav}!zzzSchnitzler, Arthur@\emph{von Arthur Schnitzler}!1903-03-141@{14. 3. 1903}|(be}
\toendnotes[C]{\smallbreak\pagebreak[2]}
\correspDesc{Versand  durch Arthur Schnitzler am 14. 3. 1903 in Wien
\newline{}Erhalt  durch Gustav Schwarzkopf im Zeitraum [14. 3. 1903 – 17. 3. 1903?] in Wien}\toendnotes[C]{\smallbreak}
\Standort{CUL, Schnitzler, B 96.}
\physDesc{Brief, 1 Blatt, 3 Seiten, 611 Zeichen
\newline{}Handschrift: schwarze Tinte, deutsche Kurrent}\toendnotes[C]{\smallbreak}
\pstart
           \raggedleft{}{\pb}14. 3. 903.\pend
           \vspace{0.5em}
\pstart
           lieber Guſtav; beifolgd 2 Sitze für morgen\eventindex{Volkstheater@\textbf{Volkstheater}!2. Aufführung von Lebendige Stunden, 15.3.1903@2. Aufführung von Lebendige Stunden, 15.3.1903|pwv}. Ich verſichre Sie, daſs das deutſche Theater\orgindex{Deutsches Theater Berlin@Deutsches Theater Berlin|pw} gegen das Volkstheater\orgindex{Volkstheater@Volkstheater|pw} ein Burgtheater\orgindex{Burgtheater@Burgtheater|pw} ist. Die \uline{heute} fertig zu \textcolor{gray}{erwarten} Gallerie iſt – gegen die Gallerie des
                  Dtſch Theaters\orgindex{Deutsches Theater Berlin@Deutsches Theater Berlin|pw} gehalten Iglau\oindex{Jihlava@\textbf{Jihlava}|pw}: das Portrait\pwindex{Schnitzler, Arthur 15. 5. 1862 Wien – 21. 10. 1931 ebd.@\textsc{Schnitzler, Arthur} (15. 5. 1862 Wien – 21. 10. 1931 ebd.), \emph{Schriftsteller, Mediziner}!Lebendige Stunden. Vier Einakter@\strich\emph{Lebendige Stunden. Vier Einakter}|pwv} der Sandrock\pwindex{Sandrock, Adele 19.\,8.\,1863 Rotterdam – 30.\,8.\,1937 Berlin@\textsc{Sandrock, Adele} (19.\,8.\,1863 Rotterdam – 30.\,8.\,1937 Berlin), \emph{Schauspielerin}|pw}{ }ſo lächerlich,
               daſs es heut Abend\eventindex{Volkstheater@\textbf{Volkstheater}!Premiere von Lebendige Stunden, 14.3.1903@Premiere von Lebendige Stunden, 14.3.1903|pwv} in meinem
               Beiſein im Theater geändert werden muſs – das \label{K_L04061-1v}\edtext{zweite Portrait\pwindex{Schnitzler, Arthur 15. 5. 1862 Wien – 21. 10. 1931 ebd.@\textsc{Schnitzler, Arthur} (15. 5. 1862 Wien – 21. 10. 1931 ebd.), \emph{Schriftsteller, Mediziner}!Lebendige Stunden. Vier Einakter@\strich\emph{Lebendige Stunden. Vier Einakter}|pwv}}{\lemma{\textnormal{\emph{zweite Portrait}}}\Cendnote{\textnormal{Er spricht von
                  der Ausstattung der Kunstgalerie in \emph{Die Frau mit dem Dolche}\textcolor{red}{\textsuperscript{XXXX indx2}}.}}}\label{K_L04061-1} (im Atelier\pwindex{Schnitzler, Arthur 15. 5. 1862 Wien – 21. 10. 1931 ebd.@\textsc{Schnitzler, Arthur} (15. 5. 1862 Wien – 21. 10. 1931 ebd.), \emph{Schriftsteller, Mediziner}!Lebendige Stunden. Vier Einakter@\strich\emph{Lebendige Stunden. Vier Einakter}|pwv}) noch – nicht fertig. – U. ſ. w. »Was {\pb}ſoll ich Ihnen viel erzählen? – Sie
               werden ja ſelbſt ſehn.« Und ich glaube wirklich, daſs Herr Weisse\pwindex{Weisse, Adolf 4.\,4.\,1855 Tauţ – 17.\,7.\,1933 Wien@\textsc{Weisse, Adolf} (4.\,4.\,1855 Tauţ – 17.\,7.\,1933 Wien), \emph{Theaterleiter, Schauspieler}|pw} das widerlichſte Exemplar von Direktor iſt, der mir
               noch untergekommen iſt.\pend
           
\pstart
           leben Sie wohl. Herzlich grüßend{\\[\baselineskip]} Ihr{\\[\baselineskip]}\spacefill\mbox{A.}\pend
           \leftskip=0em{}\selectlanguage{ngerman}\endnumbering\briefempfaengerindex{Schwarzkopf, Gustav@\textsc{Schwarzkopf, Gustav}!zzzSchnitzler, Arthur@\emph{von Arthur Schnitzler}!1903-03-141@{14. 3. 1903}|)be}\mylabel{L04061h}
\begin{anhang}
\end{anhang}\newcommand{\dateiname}{L04061}\newcommand{\titel}{Arthur Schnitzler an Gustav Schwarzkopf, 14. 3. 1903}\newcommand{\editorInnen}{Herausgegeben von Jahnke, SelmaMüller, Martin Anton}%% latex-leseansicht-abspann.tex
%% Abspann für die Leseansicht.
%% Der Schalter \ifkorrekturansicht ist bereits durch den Vorspann gesetzt.

%% latex-abspann.tex
%% Gemeinsamer Abspann für Korrekturansicht und Leseansicht.
%% Setzt den Schalter \ifkorrekturansicht voraus (gesetzt in den
%% einbindenden Dateien latex-korrekturansicht-abspann.tex bzw.
%% latex-leseansicht-abspann.tex).
%% ---------------------------------------------------------------

\normalsize

% Das esempio-Environment wird nur in der Leseansicht benötigt
\ifkorrekturansicht\else
\newenvironment{esempio}[3]%
{
    \vspace{1.5ex}
    \rlap{\underline{#1}}
    \par
    \setlength{\parindent}{0cm}
    \nopagebreak
    \leftskip=#2cm
    \rightskip=#3cm
}
{
    \par
}
\fi

\doendnotes{C}
\bigskip
\vfill

\clearpage

\footnotesize

\ifkorrekturansicht
  \lohead{\textsc{register}}
\fi

% theindex-Environment neu definieren ohne reledmac
\makeatletter
\renewenvironment{theindex}{%
  \ifkorrekturansicht
    \section*{\indexname}%
  \else
    \subsubsection*{Index der erwähnten Entitäten}%
  \fi
  \setlength{\parindent}{0pt}%
  \setlength{\parskip}{0pt plus 0.3pt}%
  \let\item\@idxitem
}{%
  \ifkorrekturansicht\clearpage\fi
}
\makeatother

\IfFileExists{\jobname-pw.ind}{\input{\jobname-pw.ind}}{}

% Quellenangabe nur in der Leseansicht
\ifkorrekturansicht\else
% Fallback-Definitionen, falls die .tex-Datei \titel etc. nicht gesetzt hat
\providecommand{\titel}{}
\providecommand{\editorInnen}{}
\providecommand{\dateiname}{\jobname}

\vspace{3cm}

\vfill

\footnotesize
\textsc{Quelle}: \titel. Herausgegeben von {\editorInnen}. In: \emph{Arthur Schnitzler: Briefwechsel mit Autorinnen und Autoren}.
 Digitale Edition, https://schnitzler-briefe.acdh.oeaw.ac.at/{\dateiname}.html (Stand \today)
\fi

\end{document}


