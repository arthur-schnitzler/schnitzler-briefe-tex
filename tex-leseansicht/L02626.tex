%% latex-leseansicht-vorspann.tex
%% Vorspann für die Leseansicht.
%% Lädt die gemeinsame Datei latex-vorspann.tex mit nicht gesetztem Schalter.

\newif\ifkorrekturansicht
\korrekturansichtfalse

\input{../tex-inputs/latex-vorspann}


\section[Paul Goldmann an Arthur Schnitzler, 6. 2. [1893]]{L02626 Paul Goldmann an Arthur Schnitzler, 6. 2. [1893]}
\nopagebreak\mylabel{L02626v}
\rehead{ }\normalsize\beginnumbering\briefempfaengerindex{Schnitzler, Arthur@\textsc{Schnitzler, Arthur}!zzzGoldmann, Paul@\emph{von Paul Goldmann}!1893-02-061@{6. 2. [1893]}|(be}
\toendnotes[C]{\smallbreak\pagebreak[2]}
\correspDesc{Versand  durch Paul Goldmann am 6. 2. [1893] in Paris
\newline{}Erhalt  durch Arthur Schnitzler im Zeitraum [7. 2. 1893
                  – 11. 2. 1893?] in Wien}\toendnotes[C]{\smallbreak}
\Standort{DLA, A:Schnitzler, HS.NZ85.1.3164.}
\physDesc{Brief, 1 Blatt, 2 Seiten, 820 Zeichen
\newline{}Handschrift: schwarze Tinte, deutsche Kurrent
\newline{}Schnitzler: mit Bleistift die Jahreszahl »93« vermerkt }\toendnotes[C]{\smallbreak}
\pstart
           {\pb}\textcolor{gray}{\textbf{Frankfurter Zeitung\orgindex{Frankfurter Zeitung@Frankfurter Zeitung|pw}.}}\pend
           
\pstart
           \textcolor{gray}{\textbf{(Gazette de
                     Francfort\orgindex{Frankfurter Zeitung@Frankfurter Zeitung|pw}.)}}\pend
           
\pstart
           \textcolor{gray}{\textbf{Directeur \textbf{M. L. Sonnemann\pwindex{Sonnemann, Leopold 29.\,10.\,1831 Höchberg – 30.\,10.\,1909 Frankfurt am Main@\textsc{Sonnemann, Leopold} (29.\,10.\,1831 Höchberg – 30.\,10.\,1909 Frankfurt am Main), \emph{Journalist, Herausgeber}|pw}}.}}\hfill \textsc{Paris\oindex{Paris@\textbf{Paris}, \emph{Hauptstadt}|pw}}, 6. Februar.\pend
           
\pstart
           \textcolor{gray}{\textbf{\begin{otherlanguage}{french}Journal politique, financier,\end{otherlanguage}}}\pend
           
\pstart
           \textcolor{gray}{\textbf{\begin{otherlanguage}{french}commercial et litteraire.\end{otherlanguage}}}\pend
           
\pstart
           \textcolor{gray}{\textbf{\begin{otherlanguage}{french}\textbf{Paraissant trois fois par jour}\end{otherlanguage}}}\pend
           
\pstart
           \textcolor{gray}{\textbf{\begin{otherlanguage}{french}\textbf{Bureaux à Paris\oindex{Paris@\textbf{Paris}, \emph{Hauptstadt}|pw}:}\end{otherlanguage}}}\pend
           
\pstart
           \textcolor{gray}{\textbf{\begin{otherlanguage}{french}rue Richelieu 75\oindex{rue Richelieu@\textbf{rue Richelieu}, \emph{Straße}|pw}.\end{otherlanguage}}}\pend
           
\pstart\center{}Mein theurer Freund!\pend\vspace{0.5em}
\pstart
           Ich{ }ſage Dir von ganzem Herzen Dank für Deine lieben \label{K_L02626-1v}\edtext{Glückwünſche}{\lemma{\textnormal{\emph{Glückwünsche}}}\Cendnote{\textnormal{Goldmann hatte am 31. 1. 1893 seinen 28.
                  Geburtstag. In Schnitzlers Nachlass im \emph{Deutschen Literaturarchiv Marbach} wurde dieser Brief fälschlicherweise ins Jahr 1894 eingeordnet.}}}\label{K_L02626-1}.\pend
           
\pstart
           Du haſt Recht: das müßte für mich eine hohe Freude, eine Erleichterung und Befreiung{ }ſein. Müßte! Aber das Geſchick \strikeout{ni} nimmt{ }ſeine{ }ſchwere
               Hand nicht von mir. Kaum will ich aufathmen und etwas freier in die Zukunft blicken,{ }ſo geſchieht mir etwas, was mir dieſe Zukunft wohl auf immer verſchließt. Das \label{K_L02626-2v}\edtext{Fürchterlichſte, mein lieber Freund, was
               einem jungen Manne überhaupt paſſiren kann}{\lemma{\textnormal{\emph{Fürchterlichste, … kann}}}\Cendnote{\textnormal{wahrscheinlich Syphilis}}}\label{K_L02626-2}, – das, wovor {\pb}ich jahrelang gezittert. Du verſtehſt mich, nicht
               wahr? Und Du biſt der Einzige, dem ich es{ }ſage \substVorne{}\textsuperscript{.}\substDazwischen{}–\substHinten{} außer dem Arzte, der mich behandelt. Du wirſt es ja nicht
               weitertragen. Und ich bin es Dir{ }ſchuldig, Dir dieſe Mittheilung zu machen.\pend
           
\pstart
           Gott behüte Dich mein theurer Freund, – beſſer, als er es mit mir gethan.\pend
           
\pstart
           Dein{\\[\baselineskip]}\spacefill\mbox{Paul Goldmann.}\pend
           \leftskip=0em{}\selectlanguage{ngerman}\endnumbering\briefempfaengerindex{Schnitzler, Arthur@\textsc{Schnitzler, Arthur}!zzzGoldmann, Paul@\emph{von Paul Goldmann}!1893-02-061@{6. 2. [1893]}|)be}\mylabel{L02626h}  \newcommand{\dateiname}{L02626}\newcommand{\titel}{Paul Goldmann an Arthur Schnitzler, 6. 2. [1893]}\newcommand{\editorInnen}{Martin Anton Müller und Laura Untner}%% latex-leseansicht-abspann.tex
%% Abspann für die Leseansicht.
%% Der Schalter \ifkorrekturansicht ist bereits durch den Vorspann gesetzt.

%% latex-abspann.tex
%% Gemeinsamer Abspann für Korrekturansicht und Leseansicht.
%% Setzt den Schalter \ifkorrekturansicht voraus (gesetzt in den
%% einbindenden Dateien latex-korrekturansicht-abspann.tex bzw.
%% latex-leseansicht-abspann.tex).
%% ---------------------------------------------------------------

\normalsize

% Das esempio-Environment wird nur in der Leseansicht benötigt
\ifkorrekturansicht\else
\newenvironment{esempio}[3]%
{
    \vspace{1.5ex}
    \rlap{\underline{#1}}
    \par
    \setlength{\parindent}{0cm}
    \nopagebreak
    \leftskip=#2cm
    \rightskip=#3cm
}
{
    \par
}
\fi

\doendnotes{C}
\bigskip
\vfill

\clearpage

\footnotesize

\ifkorrekturansicht
  \lohead{\textsc{register}}
\fi

% theindex-Environment neu definieren ohne reledmac
\makeatletter
\renewenvironment{theindex}{%
  \ifkorrekturansicht
    \section*{\indexname}%
  \else
    \subsubsection*{Index der erwähnten Entitäten}%
  \fi
  \setlength{\parindent}{0pt}%
  \setlength{\parskip}{0pt plus 0.3pt}%
  \let\item\@idxitem
}{%
  \ifkorrekturansicht\clearpage\fi
}
\makeatother

\IfFileExists{\jobname-pw.ind}{\input{\jobname-pw.ind}}{}

% Quellenangabe nur in der Leseansicht
\ifkorrekturansicht\else
% Fallback-Definitionen, falls die .tex-Datei \titel etc. nicht gesetzt hat
\providecommand{\titel}{}
\providecommand{\editorInnen}{}
\providecommand{\dateiname}{\jobname}

\vspace{3cm}

\vfill

\footnotesize
\textsc{Quelle}: \titel. Herausgegeben von {\editorInnen}. In: \emph{Arthur Schnitzler: Briefwechsel mit Autorinnen und Autoren}.
 Digitale Edition, https://schnitzler-briefe.acdh.oeaw.ac.at/{\dateiname}.html (Stand \today)
\fi

\end{document}


