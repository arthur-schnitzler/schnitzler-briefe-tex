%% latex-korrekturansicht-vorspann.tex
%% Vorspann für die Korrekturansicht.
%% Lädt die gemeinsame Datei latex-vorspann.tex mit gesetztem Schalter.

\newif\ifkorrekturansicht
\korrekturansichttrue

\input{../tex-inputs/latex-vorspann}


\section[Paul Goldmann an Arthur Schnitzler, 6. 2. {[}1893{]}]{L02626 Paul Goldmann an Arthur Schnitzler, 6. 2. {[}1893{]}}
\nopagebreak\mylabel{L02626v}
\rehead{ }\normalsize\beginnumbering\briefempfaengerindex{Schnitzler, Arthur@\textsc{Schnitzler, Arthur}!zzzGoldmann, Paul@\emph{von Paul Goldmann}!1893-02-061@{6. 2. {[}1893{]}}|(be}
\toendnotes[C]{\smallbreak\pagebreak[2]}\Standort{DLA, A:Schnitzler, HS.NZ85.1.3164.}
\physDesc{Brief, 1 Blatt, 2 Seiten, 820 Zeichen
\newline{}Handschrift: schwarze Tinte, deutsche Kurrent
\newline{}Schnitzler: mit Bleistift die Jahreszahl »93« vermerkt }\toendnotes[C]{\smallbreak}
\pstart
           {\pb}\textcolor{gray}{\textbf{Frankfurter Zeitung\orgindex{Frankfurter Zeitung@Frankfurter Zeitung|pw}.}}\pend
           
\pstart
           \textcolor{gray}{\textbf{(Gazette de
                     Francfort\orgindex{Frankfurter Zeitung@Frankfurter Zeitung|pw}.)}}\pend
           
\pstart
           \textcolor{gray}{\textbf{Directeur \textbf{M. L. Sonnemann\pwindex{Sonnemann, Leopold 1831-10-29 – 1909-10-30@\textsc{Sonnemann, Leopold} (1831-10-29 – 1909-10-30), \emph{Journalist/Journalistin, Herausgeber/Herausgeberin}|pw}}.}}\hfill \textsc{Paris\oindex{Paris@\textbf{Paris}, \emph{P.PPLC}|pw}}, 6. Februar.\pend
           
\pstart
           \textcolor{gray}{\textbf{\begin{otherlanguage}{french}Journal politique, financier,\end{otherlanguage}}}\pend
           
\pstart
           \textcolor{gray}{\textbf{\begin{otherlanguage}{french}commercial et litteraire.\end{otherlanguage}}}\pend
           
\pstart
           \textcolor{gray}{\textbf{\begin{otherlanguage}{french}\textbf{Paraissant trois fois par jour}\end{otherlanguage}}}\pend
           
\pstart
           \textcolor{gray}{\textbf{\begin{otherlanguage}{french}\textbf{Bureaux à Paris\oindex{Paris@\textbf{Paris}, \emph{P.PPLC}|pw}:}\end{otherlanguage}}}\pend
           
\pstart
           \textcolor{gray}{\textbf{\begin{otherlanguage}{french}rue Richelieu 75\oindex{rue Richelieu@\textbf{rue Richelieu}, \emph{Straße (K.STR)}|pw}.\end{otherlanguage}}}\pend
           
\pstart\center{}Mein theurer Freund!\pend\vspace{0.5em}
\pstart
           Ich ſage Dir von ganzem Herzen Dank für Deine lieben \label{K_L02626-1v}\edtext{Glückwünſche}{\lemma{\textnormal{\emph{Glückwünſche}}}\Cendnote{\textnormal{Goldmann hatte am 31. 1. 1893 seinen 28.
                  Geburtstag. In Schnitzlers Nachlass im \emph{Deutschen Literaturarchiv Marbach} wurde dieser Brief fälschlicherweise ins Jahr 1894 eingeordnet.}}}\label{K_L02626-1}.\pend
           
\pstart
           Du haſt Recht: das müßte für mich eine hohe Freude, eine Erleichterung und Befreiung
               ſein. Müßte! Aber das Geſchick \strikeout{ni} nimmt ſeine ſchwere
               Hand nicht von mir. Kaum will ich aufathmen und etwas freier in die Zukunft blicken,
               ſo geſchieht mir etwas, was mir dieſe Zukunft wohl auf immer verſchließt. Das \label{K_L02626-2v}\edtext{Fürchterlichſte, mein lieber Freund, was
               einem jungen Manne überhaupt paſſiren kann}{\lemma{\textnormal{\emph{Fürchterlichſte, … kann}}}\Cendnote{\textnormal{wahrscheinlich Syphilis}}}\label{K_L02626-2}, – das, wovor {\pb}ich jahrelang gezittert. Du verſtehſt mich, nicht
               wahr? Und Du biſt der Einzige, dem ich es ſage \substVorne{}\textsuperscript{.}\substDazwischen{}–\substHinten{} außer dem Arzte, der mich behandelt. Du wirſt es ja nicht
               weitertragen. Und ich bin es Dir ſchuldig, Dir dieſe Mittheilung zu machen.\pend
           
\pstart
           Gott behüte Dich mein theurer Freund, – beſſer, als er es mit mir gethan.\pend
           
\pstart
           Dein{\\[\baselineskip]}\spacefill\mbox{Paul Goldmann.}\pend
           \leftskip=0em{}\selectlanguage{ngerman}\endnumbering\briefempfaengerindex{Schnitzler, Arthur@\textsc{Schnitzler, Arthur}!zzzGoldmann, Paul@\emph{von Paul Goldmann}!1893-02-061@{6. 2. {[}1893{]}}|)be}\mylabel{L02626h}  \normalsize

\doendnotes{C}
\bigskip
\vfill

\clearpage

\footnotesize

\lohead{\textsc{register}}

% Definiere theindex-Environment komplett neu ohne reledmac
\makeatletter
\renewenvironment{theindex}{%
  \section*{\indexname}%
  \setlength{\parindent}{0pt}%
  \setlength{\parskip}{0pt plus 0.3pt}%
  \let\item\@idxitem
}{%
  \clearpage
}
\makeatother

\IfFileExists{\jobname-pw.ind}{\input{\jobname-pw.ind}}{}

\end{document}

      