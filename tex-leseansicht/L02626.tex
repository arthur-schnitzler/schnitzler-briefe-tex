\input{../tex-inputs/latex-pdf-vorspann}
\begin{center}
            \textcolor{red}{ENTWURF. ENTZIFFERUNG NOCH NICHT KORREKTURGELESEN}
                      \end{center}
            
               \section[Paul Goldmann an Arthur Schnitzler, 6. 2. {[}1893{]}]{ Paul Goldmann an Arthur Schnitzler, 6. 2. {[}1893{]}}\nopagebreak\mylabel{v}\rehead{ }\begin{ledgroupsized}[t]{13cm}\normalsize\beginnumbering\briefempfaengerindex{Schnitzler, Arthur@\textsc{Schnitzler, Arthur}!zzzGoldmann, Paul@\emph{von Paul Goldmann}!1893-02-061@{6. 2. {[}1893{]}}|(be} \toendnotes[C]{\smallbreak\pagebreak[2]} \Standort{DLA, A:Schnitzler, HS.NZ85.1.3164.}
\physDesc{Brief, 1 Blatt, 2 Seiten
\newline{}Handschrift: schwarze Tinte, deutsche Kurrent
\newline{}Schnitzler: mit Bleistift die Jahreszahl »93«
                                 vermerkt }\toendnotes[C]{\smallbreak}\pstart
           \noindent{}{\pb}\textcolor{gray}{\textbf{Frankfurter Zeitung\orgindex{Frankfurter Zeitung@Frankfurter Zeitung|pw}.}}\hfill \textsc{Paris\oindex{Paris@\textbf{Paris}|pw}},
                        6. Februar.\pend
           \pstart
           \textcolor{gray}{\textbf{(Gazette de
                  Francfort\orgindex{Frankfurter Zeitung@Frankfurter Zeitung|pw}.)}}\pend
           \pstart
           \textcolor{gray}{\textbf{Directeur \textbf{M. L. Sonnemann\pwindex{Sonnemann, Leopold 1831-10-29 – 1909-10-30@\textsc{Sonnemann, Leopold} (1831-10-29 – 1909-10-30), \emph{Journalist, Herausgeber}|pw}}.}}\pend
           \pstart
           \textcolor{gray}{\textbf{\begin{otherlanguage}{french}Journal politique,
                        financier,\end{otherlanguage}}}\pend
           \pstart
           \textcolor{gray}{\textbf{\begin{otherlanguage}{french}commercial et
                     litteraire.\end{otherlanguage}}}\pend
           \pstart
           \textcolor{gray}{\textbf{\begin{otherlanguage}{french}\textbf{Paraissant trois fois
                           par jour}\end{otherlanguage}}}\pend
           \pstart
           \textcolor{gray}{\textbf{–}}\pend
           \pstart
           \textcolor{gray}{\textbf{\begin{otherlanguage}{french}\textbf{Bureaux à Paris\oindex{Paris@\textbf{Paris}|pw}:}\end{otherlanguage}}}\pend
           \pstart
           \textcolor{gray}{\textbf{\begin{otherlanguage}{french}rue
                           Richelieu 75\oindex{rue Richelieu@\textbf{rue Richelieu}|pw}.\end{otherlanguage}}}\pend
           \pstart\center{}Mein theurer Freund!\pend\pstart
           Ich ſage Dir von ganzem Herzen Dank für Deine lieben \label{K_L02626-1v}\edtext{Glückwünſche}{\lemma{\textnormal{\emph{Glückwünſche}}}\Cendnote{\textnormal{Goldmann hatte am 31. 1. 1893 seinen 28.
                  Geburtstag.}}}\label{K_L02626-1h}.\pend
           \pstart
           Du haſt Recht: das müßte für mich eine hohe Freude, eine Erleichterung und Befreiung
               ſein. Müſſte! Aber das Geſchick \strikeout{ni} nimmt ſeine
               ſchwere Hand nicht von mir. Kaum will ich aufathmen und etwas freier in die Zukunft
               blicken, ſo geſchieht mir etwas, was mir dieſe Zukunft wohl auf immer verſchließt.
               Das Fürchterlichſte, mein lieber Freund, was einem jungen Manne überhaupt paſſiren
               kann, – das, wovor {\pb}ich jahrelang \label{K_L02626-2v}\edtext{gezittert}{\lemma{\textnormal{\emph{gezittert}}}\Cendnote{\textnormal{wahrscheinlich eine Geschlechtskrankheit}}}\label{K_L02626-2h}. Du verſtehſt
               mich, nicht wahr? Und die biſt der Einzige, dem ich es ſage – außer dem Arzte, der
               mich behandelt\textcolor{gray}{.} Du wirſt es ja nicht weitertragen. Und ich bin es
               Dir ſchuldig, Dir dieſe Mittheilung zu machen.\pend
           \pstart
           Gott behüte Dich mein theurer Freund, – beſſer, als er es mit mir gethan.\pend
           \pstart
           Dein{\\[\baselineskip]}\spacefill\mbox{Paul Goldmann.}\pend
           \leftskip=0em{}\endnumbering\briefempfaengerindex{Schnitzler, Arthur@\textsc{Schnitzler, Arthur}!zzzGoldmann, Paul@\emph{von Paul Goldmann}!1893-02-061@{6. 2. {[}1893{]}}|)be}\mylabel{h}\end{ledgroupsized}  \newcommand{\dateiname}{L02626}\newcommand{\titel}{Paul Goldmann an Arthur Schnitzler, 6. 2. [1893]}\newcommand{\editorInnen}{Martin Anton Müller und Laura Untner}\input{../tex-inputs/latex-pdf-abspann}
      