%% latex-leseansicht-vorspann.tex
%% Vorspann für die Leseansicht.
%% Lädt die gemeinsame Datei latex-vorspann.tex mit nicht gesetztem Schalter.

\newif\ifkorrekturansicht
\korrekturansichtfalse

\input{../tex-inputs/latex-vorspann}


\section[Arthur Schnitzler und Olga Gussmann an Gustav Schwarzkopf, 16. 2. 1903]{L04219 Arthur Schnitzler und Olga Gussmann an Gustav Schwarzkopf, 16. 2. 1903}
\nopagebreak\mylabel{L04219v}
\rehead{ }\normalsize\beginnumbering\briefempfaengerindex{Schwarzkopf, Gustav@\textsc{Schwarzkopf, Gustav}!zzzSchnitzler, Olga@\emph{von Olga Schnitzler}!1903-02-161@{16. 2. 1903}|(be}\briefempfaengerindex{Schwarzkopf, Gustav@\textsc{Schwarzkopf, Gustav}!zzzSchnitzler, Arthur@\emph{von Arthur Schnitzler}!1903-02-161@{16. 2. 1903}|(be}
\toendnotes[C]{\smallbreak\pagebreak[2]}
\correspDesc{Versand  durch Arthur Schnitzler, Olga Schnitzler am 16. 2. 1903 in Wien
\newline{}Erhalt  durch Gustav Schwarzkopf im Zeitraum [16. 2. 1903
                  – 19. 2. 1903?] in Wien}\toendnotes[C]{\smallbreak}
\Standort{CUL, Schnitzler, B 96.}
\physDesc{Brief, 1 Blatt, 4 Seiten, 730 Zeichen
\newline{}Handschrift Arthur Schnitzler: Bleistift, deutsche Kurrent
\newline{}Handschrift Olga Schnitzler: Bleistift, lateinische Kurrent}\toendnotes[C]{\smallbreak}
\pstart
           \raggedleft{}{\pb}\uline{Montag.}{\\}16. 2. 903.\pend
           
\pstart{}lieber Guſtav,\pend\vspace{0.5em}
\pstart
           we{\geminationn} Sie \uline{morgen} ko{\geminationm}en bitte holen Sie mich vielleicht Frankgaſſe\oindex{Wien@\textbf{Wien}!IX., Alsergrund@\textbf{IX., Alsergrund}!Frankgasse 1@\textbf{Frankgasse 1}, \emph{Wohngebäude}|pw} um 6 – ½ 7 ab. – \substVorne{}\textsuperscript{Kon }\substDazwischen{}Wen\substHinten{}n Ihnen aber die beiden Schübe lächeln,{ }ſo bitte ich Sie \label{K_L04219-5v}\edtext{Donnerſtag}{\lemma{\textnormal{\emph{Donnerstag}}}\Cendnote{\textnormal{Das von Schnitzler gewünschte Treffen fand am Donnerstag, dem 19. 2. 1903 in der
                  Unterkunft von Olga Gussmann\pwindex{Schnitzler, Olga 17.\,1.\,1882 Wien – 13.\,1.\,1970 Lugano@\textsc{Schnitzler, Olga} (17.\,1.\,1882 Wien – 13.\,1.\,1970 Lugano), \emph{Schauspielerin, Sängerin}|pwk} in der Gentzgasse 110\oindex{Wien@\textbf{Wien}!XVIII., Währing@\textbf{XVIII., Währing}!Gentzgasse 110@\textbf{Gentzgasse 110}, \emph{Wohngebäude}|pwk} statt. Schnitzler las den beiden \emph{Der einsame Weg vor}\pwindex{Schnitzler, Arthur 15. 5. 1862 Wien – 21. 10. 1931 ebd.@\textsc{Schnitzler, Arthur} (15. 5. 1862 Wien – 21. 10. 1931 ebd.), \emph{Schriftsteller, Mediziner}!einsame Weg. Schauspiel in fünf Akten@\strich\emph{Der einsame Weg. Schauspiel in fünf Akten}|pwk}, siehe A. S.: \emph{Kulturveranstaltungen}, 19. 2. 1903. }}}\label{K_L04219-5}{ }{\pb}Gentzgaſſe\oindex{Wien@\textbf{Wien}!XVIII., Währing@\textbf{XVIII., Währing}!Gentzgasse 110@\textbf{Gentzgasse 110}, \emph{Wohngebäude}|pw} zu nachtmahlen. (zu welchem Behufe
               Sie mich auch zur ſelben Zeit abholen mögen, we{\geminationn} es
               Ihnen bequem iſt.)\pend
           
\pstart
           Es handelt ſich nur mehr darum, ob das Stück\pwindex{Schnitzler, Arthur 15. 5. 1862 Wien – 21. 10. 1931 ebd.@\textsc{Schnitzler, Arthur} (15. 5. 1862 Wien – 21. 10. 1931 ebd.), \emph{Schriftsteller, Mediziner}!einsame Weg. Schauspiel in fünf Akten@\strich\emph{Der einsame Weg. Schauspiel in fünf Akten}|pwv} als {\pb}ganzes zu vernichten{ }ſein wird oder ob \substVorne{}\textsuperscript{e }\substDazwischen{}E\substHinten{}inzel\substVorne{}\textsuperscript{n }\substDazwischen{}h\substHinten{}eiten zu retten wären. Angenehme Ausſichten; aber es gibt auch gefülltes
               Kraut (Dinſtag) oder{ }ſerbiſches
               Reisfleiſch (Donnerſtag)\pend
           \pstart Herzlichſt Ihr \spacefill\mbox{A.}\pend{}\selectlanguage{ngerman}\vspace{1em}
\pstart
           \noindent{}{\pb}{[}hs. Schnitzler:{]} Die »angenehmen Ausſichten« wurden mir ſoeben erſt
               mitgetheilt, nachdem \introOben{}{[}hs. Schnitzler:{]} nein\introOben{}{ }{[}hs. Schnitzler:{]} ich Ihnen erzählt habe. Das Stück\pwindex{Schnitzler, Arthur 15. 5. 1862 Wien – 21. 10. 1931 ebd.@\textsc{Schnitzler, Arthur} (15. 5. 1862 Wien – 21. 10. 1931 ebd.), \emph{Schriftsteller, Mediziner}!einsame Weg. Schauspiel in fünf Akten@\strich\emph{Der einsame Weg. Schauspiel in fünf Akten}|pwv} vor wenigen Tagen »ganz gut«
                  war.\footnote{\noindent{}{[}hs. Schnitzler:{]} Sie träumt.{\\}A.}\pend
           
\pstart
           Wir werden ja ſehen, und Strenge richten.\pend
           
\pstart
           Beſten Gruß{\\[\baselineskip]}\spacefill\mbox{OGussmann}\pend
           \leftskip=0em{}\selectlanguage{ngerman}\endnumbering\briefempfaengerindex{Schwarzkopf, Gustav@\textsc{Schwarzkopf, Gustav}!zzzSchnitzler, Olga@\emph{von Olga Schnitzler}!1903-02-161@{16. 2. 1903}|)be}\briefempfaengerindex{Schwarzkopf, Gustav@\textsc{Schwarzkopf, Gustav}!zzzSchnitzler, Arthur@\emph{von Arthur Schnitzler}!1903-02-161@{16. 2. 1903}|)be}\mylabel{L04219h}
\begin{anhang}
\end{anhang}\newcommand{\dateiname}{L04219}\newcommand{\titel}{Arthur Schnitzler und Olga Gussmann an Gustav Schwarzkopf, 16. 2. 1903}\newcommand{\editorInnen}{Herausgegeben von Jahnke, SelmaMüller, Martin Anton}%% latex-leseansicht-abspann.tex
%% Abspann für die Leseansicht.
%% Der Schalter \ifkorrekturansicht ist bereits durch den Vorspann gesetzt.

%% latex-abspann.tex
%% Gemeinsamer Abspann für Korrekturansicht und Leseansicht.
%% Setzt den Schalter \ifkorrekturansicht voraus (gesetzt in den
%% einbindenden Dateien latex-korrekturansicht-abspann.tex bzw.
%% latex-leseansicht-abspann.tex).
%% ---------------------------------------------------------------

\normalsize

% Das esempio-Environment wird nur in der Leseansicht benötigt
\ifkorrekturansicht\else
\newenvironment{esempio}[3]%
{
    \vspace{1.5ex}
    \rlap{\underline{#1}}
    \par
    \setlength{\parindent}{0cm}
    \nopagebreak
    \leftskip=#2cm
    \rightskip=#3cm
}
{
    \par
}
\fi

\doendnotes{C}
\bigskip
\vfill

\clearpage

\footnotesize

\ifkorrekturansicht
  \lohead{\textsc{register}}
\fi

% theindex-Environment neu definieren ohne reledmac
\makeatletter
\renewenvironment{theindex}{%
  \ifkorrekturansicht
    \section*{\indexname}%
  \else
    \subsubsection*{Index der erwähnten Entitäten}%
  \fi
  \setlength{\parindent}{0pt}%
  \setlength{\parskip}{0pt plus 0.3pt}%
  \let\item\@idxitem
}{%
  \ifkorrekturansicht\clearpage\fi
}
\makeatother

\IfFileExists{\jobname-pw.ind}{\input{\jobname-pw.ind}}{}

% Quellenangabe nur in der Leseansicht
\ifkorrekturansicht\else
% Fallback-Definitionen, falls die .tex-Datei \titel etc. nicht gesetzt hat
\providecommand{\titel}{}
\providecommand{\editorInnen}{}
\providecommand{\dateiname}{\jobname}

\vspace{3cm}

\vfill

\footnotesize
\textsc{Quelle}: \titel. Herausgegeben von {\editorInnen}. In: \emph{Arthur Schnitzler: Briefwechsel mit Autorinnen und Autoren}.
 Digitale Edition, https://schnitzler-briefe.acdh.oeaw.ac.at/{\dateiname}.html (Stand \today)
\fi

\end{document}


