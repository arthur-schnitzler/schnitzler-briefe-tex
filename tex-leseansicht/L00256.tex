%% latex-korrekturansicht-vorspann.tex
%% Vorspann für die Korrekturansicht.
%% Lädt die gemeinsame Datei latex-vorspann.tex mit gesetztem Schalter.

\newif\ifkorrekturansicht
\korrekturansichttrue

\input{../tex-inputs/latex-vorspann}


\section[Arthur Schnitzler an Richard Beer-Hofmann, 18. 8. 1893]{L00256 Arthur Schnitzler an Richard Beer-Hofmann, 18. 8. 1893}
\nopagebreak\mylabel{L00256v}
\rehead{ }\normalsize\beginnumbering\briefempfaengerindex{Beer-Hofmann, Richard@\textsc{Beer-Hofmann, Richard}!zzzSchnitzler, Arthur@\emph{von Arthur Schnitzler}!1893-08-181@{18. 8. 1893}|(be}
\toendnotes[C]{\smallbreak\pagebreak[2]}\Standort{YCGL, MSS 31.}
\physDesc{Brief, 1 Blatt, 3 Seiten, Umschlag, 329 Zeichen (Briefpapier und Umschlag mit Trauerrand)
\newline{}Handschrift: Bleistift, deutsche Kurrent
\newline{}Versand: 1) Stempel: »\nobreak{}Wien 1/1, 18. 8. 93, 7 N\nobreak{}«.   2) Stempel: »\nobreak{}\oindex{Bad Ischl@\textbf{Bad Ischl}, \emph{P.PPL}|pwk}Ischl, 19 8 93, 7 F\nobreak{}«. 
\newline{}Ordnung: mit rotem Buntstift von unbekannter Hand in der linken oberen Ecke mit
                                 einem »X« versehen }
\buchAbdrucke{\weitereDrucke{Arthur Schnitzler, Richard Beer-Hofmann: \emph{Briefwechsel 1891–1931}. Wien, Zürich: \emph{Europaverlag} 1992, S. 51.} }\pstart{}{\pb}Hrn \textsc{Dr. Rich.
                     Beer-Hofmann}\pend{}\pstart{}\textsc{Ischl\oindex{Bad Ischl@\textbf{Bad Ischl}, \emph{P.PPL}|pw}}\pend{}\pstart{}\textsc{Schulgasse 8}\oindex{Schulgasse@\textbf{Schulgasse}, \emph{Straße (K.STR)}|pw}\pend{}{\bigskip}\vspace{1em}
\pstart{}{\pb}Lieber Richard –\pend\vspace{0.5em}
\pstart
           Ich verreiſe Montag oder Dinſtag. Schreiben Sie mir vorher
               2 Zeilen. Ko{\geminationm}en Sie vor der Waffenübg nach Wien\oindex{Wien@\textbf{Wien}, \emph{A.ADM2}|pw}? –\pend
           
\pstart
           Haben Sie was über {\pb}\textsc{Freund}\pwindex{Freund, Carl @\textsc{Freund, Carl}, \emph{Verleger/Verlegerin}|pw} erfahren? –\pend
           
\pstart
           – Ich treffe in \textsc{Lienz}\oindex{Lienz@\textbf{Lienz}, \emph{P.PPLA3}|pw} oder ſo wo mit \textsc{Salten}\pwindex{Salten, Felix 06.09.1869 – 08.10.1945@\textsc{Salten, Felix} (06.09.1869 – 08.10.1945), \emph{Schriftsteller/Schriftstellerin, Journalist/Journalistin, Chefredakteur/Chefredakteurin}|pw} zusa{\geminationm}en. –\pend
           
\pstart
           Es ist unglaublich, dß Sie gar nicht ſchreiben.\pend
           
\pstart
           Herzlichen Gruſs{\\[\baselineskip]}\spacefill\mbox{{\pb}Arthur}\pend
           \leftskip=0em{}
\pstart
           Wien\oindex{Wien@\textbf{Wien}, \emph{A.ADM2}|pw}{ }18/8 93\pend
           \selectlanguage{ngerman}\endnumbering\briefempfaengerindex{Beer-Hofmann, Richard@\textsc{Beer-Hofmann, Richard}!zzzSchnitzler, Arthur@\emph{von Arthur Schnitzler}!1893-08-181@{18. 8. 1893}|)be}\mylabel{L00256h}  \normalsize

\doendnotes{C}
\bigskip
\vfill

\clearpage

\footnotesize

\lohead{\textsc{register}}

% Definiere theindex-Environment komplett neu ohne reledmac
\makeatletter
\renewenvironment{theindex}{%
  \section*{\indexname}%
  \setlength{\parindent}{0pt}%
  \setlength{\parskip}{0pt plus 0.3pt}%
  \let\item\@idxitem
}{%
  \clearpage
}
\makeatother

\IfFileExists{\jobname-pw.ind}{\input{\jobname-pw.ind}}{}

\end{document}

      