%% latex-leseansicht-vorspann.tex
%% Vorspann für die Leseansicht.
%% Lädt die gemeinsame Datei latex-vorspann.tex mit nicht gesetztem Schalter.

\newif\ifkorrekturansicht
\korrekturansichtfalse

\input{../tex-inputs/latex-vorspann}

\begin{center}
            \textcolor{red}{ENTWURF, NICHT FERTIG KORRIGIERT}
                      \end{center}
            
               \section[Paul Goldmann an Arthur Schnitzler, 3. 11. {[}1894{]}]{ Paul Goldmann an Arthur Schnitzler, 3. 11. {[}1894{]}}\nopagebreak\mylabel{v}\rehead{ }\begin{ledgroupsized}[t]{13cm}\normalsize\beginnumbering\briefempfaengerindex{Schnitzler, Arthur@\textsc{Schnitzler, Arthur}!zzzGoldmann, Paul@\emph{von Paul Goldmann}!1894-11-031@{3. 11. {[}1894{]}}|(be} \toendnotes[C]{\smallbreak\pagebreak[2]} \Standort{DLA, A:Schnitzler, HS.NZ85.1.3164.}
\physDesc{Brief, 1 Blatt, 4 Seiten
\newline{}Handschrift: schwarze Tinte, deutsche Kurrent
\newline{}Schnitzler: 1) mit Bleistift auf dem ersten Blatt die Jahreszahl
                                       »94« vermerkt 2) mit rotem Buntstift eine Unterstreichung}\toendnotes[C]{\smallbreak}\pstart
           \noindent{}{\pb}\textcolor{gray}{\textbf{Frankfurter Zeitung\orgindex{Frankfurter Zeitung@Frankfurter Zeitung|pw}.}}\hfill \textsc{Paris\oindex{Paris@\textbf{Paris}|pw}}, 3. November.\pend
           \pstart
           \textcolor{gray}{\textbf{(Gazette de
                  Francfort\orgindex{Frankfurter Zeitung@Frankfurter Zeitung|pw}.)}}\pend
           \pstart
           \textcolor{gray}{\textbf{\begin{otherlanguage}{french}Fondateur\end{otherlanguage}{ }\textbf{M. L.
                  Sonnemann\pwindex{Sonnemann, Leopold 1831-10-29 – 1909-10-30@\textsc{Sonnemann, Leopold} (1831-10-29 – 1909-10-30), \emph{Journalist, Herausgeber}|pw}}.}}\pend
           \pstart
           \textcolor{gray}{\textbf{\begin{otherlanguage}{french}Journal politique,
                        financier,\end{otherlanguage}}}\pend
           \pstart
           \textcolor{gray}{\textbf{\begin{otherlanguage}{french}commercial et
                     littéraire.\end{otherlanguage}}}\pend
           \pstart
           \textcolor{gray}{\textbf{\begin{otherlanguage}{french}\textbf{Paraissant trois fois
                           par jour}\end{otherlanguage}}}.\pend
           \pstart
           \textcolor{gray}{\textbf{–}}\pend
           \pstart
           \textcolor{gray}{\textbf{\begin{otherlanguage}{french}\textbf{Bureaux à Paris\oindex{Paris@\textbf{Paris}|pw}:}\end{otherlanguage}}}\pend
           \pstart
           \textcolor{gray}{\textbf{\begin{otherlanguage}{french}\textbf{24. Rue Feydeau}\oindex{rue Feydeau@\textbf{rue Feydeau}|pw}.\end{otherlanguage}}}\pend
           \pstart\center{}Mein lieber Freund,\pend\pstart
           Wir ſind mitten im \label{K_L02618-1v}\edtext{Ruſſen\oindex{Russland@\textbf{Russland}|pw}fieber}{\lemma{\textnormal{\emph{Ruſſenfieber}}}\Cendnote{\textnormal{Die
                  politische Annäherung zwischen Russland\oindex{Russland@\textbf{Russland}|pwk} und Frankreich\oindex{Frankreich@\textbf{Frankreich}|pwk} führte zu einer Begeisterungswelle, die
                  durch öffentliche »Freundschaftsfeste« weiter gefördert wurden.}}}\label{K_L02618-1h} und ich
               finde gerade Zeit, Dir raſch beide Hände zu drücken, mit einem innigen \label{K_L02618-2v}\edtext{Glückwunſch}{\lemma{\textnormal{\emph{Glückwunſch}}}\Cendnote{\textnormal{siehe Max Burckhard an Arthur Schnitzler, [31. 10. 1894]}}}\label{K_L02618-2h}.
               So ſcheint alſo der liebſte Wunſch, den ich für Dich gehegt, wahr werden zu wollen.
               Ich habe mir heut Früh’, als ich Deinen lieben \label{K_L02618-22v}\edtext{Brief}{\lemma{\textnormal{\emph{Brief}}}\Cendnote{\textnormal{vgl. A. S.: \emph{Tagebuch}, 31. 10. 1894}}}\label{K_L02618-22h} erhielt, die
               Zukunft ausgemalt und habe mich an all’ dem Licht und der Freude ergötzt, die ich
               darin für Dich fand. Ich bin ſicher: Du wirſt {\pb}aufgeführt werden; ich bin ſicher: Du wirſt Erfolg haben, – ſo ſicher, daß mir iſt,
               als ſei das Alles ſchon geſchehen. B.\pwindex{Burckhard, Max Eugen 14.07.1854 – 16.03.1912@\textsc{Burckhard, Max Eugen} (14.07.1854 – 16.03.1912), \emph{Schriftsteller, Rechtswissenschaftler, Theaterleiter}|pw}’s Telegramm
               bedeutet ſicher die Annahme, und der Director\pwindex{Burckhard, Max Eugen 14.07.1854 – 16.03.1912@\textsc{Burckhard, Max Eugen} (14.07.1854 – 16.03.1912), \emph{Schriftsteller, Rechtswissenschaftler, Theaterleiter}|pwv} gefällt mir ſehr, der in dieſer Form anzunehmen verſteht. Bitte,
               ſchreib’ mir ſofort, \strikeout{daß} wie die Unterredung mit B.\pwindex{Burckhard, Max Eugen 14.07.1854 – 16.03.1912@\textsc{Burckhard, Max Eugen} (14.07.1854 – 16.03.1912), \emph{Schriftsteller, Rechtswissenschaftler, Theaterleiter}|pw} ausgefallen. Im Übrigen will ich gar nicht
               länger darüber reden, aus Aberglauben – denn es iſt gar zu ſchön. Und den Namen des
                  Theaters\orgindex{Burgtheater@Burgtheater|pwv} nenne ich erſt gar
               nicht, auch aus Aberglauben. Aber froh bin ich; und {\pb}ich fühle die glückliche Wendung und denke, daß Niemand in der Welt ſie mehr
               verdient hat, als Du, mein lieber Freund.\pend
           \pstart
           Ich \strikeout{\textcolor{gray}{be}} möchte gern das Alles
               beſſer ſagen. Aber es iſt ſo ſchwer, über die guten Dinge zu
                  ſchreiben{[}.{]} Überdies empfing ich heut mein Feuilleton\pwindex{Goldmann, Paul 31.01.1865 – 25.09.1935@\textsc{Goldmann, Paul} (31.01.1865 – 25.09.1935), \emph{Schriftsteller, Journalist}!?? [Rezension der Urauffuehrung von Gismonda]1.11.1894?@\strich\emph{?? [Rezension der Uraufführung von Gismonda]} {[}1.11.1894?{]}|pwv}
               über »\label{K_L02618-111v}\edtext{\textsc{Gismonda\pwindex{Sardou, Victorien 07.09.1831 – 08.11.1908@\textsc{Sardou, Victorien} (07.09.1831 – 08.11.1908), \emph{Schriftsteller}!Gismonda. Piece en 4 actes et 5 tableaux1894-10-31@\strich\emph{Gismonda. Pièce en 4 actes et 5 tableaux} {[}1894-10-31{]}|pw}}}{\lemma{\textnormal{\emph{Gismonda}}}\Cendnote{\textnormal{\emph{Gismonda. Pièce en 4 actes et
                     5 tableaux}\pwindex{Sardou, Victorien 07.09.1831 – 08.11.1908@\textsc{Sardou, Victorien} (07.09.1831 – 08.11.1908), \emph{Schriftsteller}!Gismonda. Piece en 4 actes et 5 tableaux1894-10-31@\strich\emph{Gismonda. Pièce en 4 actes et 5 tableaux} {[}1894-10-31{]}|pwk}, von Victorien Sardou\pwindex{Sardou, Victorien 07.09.1831 – 08.11.1908@\textsc{Sardou, Victorien} (07.09.1831 – 08.11.1908), \emph{Schriftsteller}|pwk} für
                     Sarah Bernhardt\pwindex{Bernhardt, Sarah 22.10.1844 – 26.03.1923@\textsc{Bernhardt, Sarah} (22.10.1844 – 26.03.1923), \emph{Schauspielerin}|pwk} geschrieben, erlebte seine
                  Uraufführung am 31. 10. 1894 am \emph{Théâtre de la Renaissance}\orgindex{Theâtre de la Renaissance@Théâtre de la Renaissance|pwk}.}}}\label{K_L02618-111h}«, das mein Onkel\pwindex{Mamroth, Fedor 21.02.1851 – 25.06.1907@\textsc{Mamroth, Fedor} (21.02.1851 – 25.06.1907), \emph{Journalist, Kritiker}|pwv} in einer irrſinnigen Weiſe
               zuſammengeſtrichen hat. Das iſt ein Lähmungsſchlag ins Gehirn.\pend
           \pstart
           Ich danke Dir von ganzem Herzen für den Freundſchafts-Beweis, den Du mir
               gegeben, indem Du mir ſofort die {\pb}Nachricht
               mitgetheilt; und ich begrüße Dich vielmals und in Treue {\\[\baselineskip]}Dein \spacefill\mbox{Paul
                  Goldmann}\pend
           \leftskip=0em{}          \endnumbering\briefempfaengerindex{Schnitzler, Arthur@\textsc{Schnitzler, Arthur}!zzzGoldmann, Paul@\emph{von Paul Goldmann}!1894-11-031@{3. 11. {[}1894{]}}|)be}\mylabel{h}\end{ledgroupsized}\begin{anhang}\end{anhang}\newcommand{\dateiname}{L02618}\newcommand{\titel}{Paul Goldmann an Arthur Schnitzler, 3. 11. [1894]}\newcommand{\editorInnen}{Martin Anton Müller und Laura Untner}
            \footnotesize
\begin{ledgroupsized}[t]{11.5cm}
\doendnotes{C}
\end{ledgroupsized}
         %% latex-leseansicht-abspann.tex
%% Abspann für die Leseansicht.
%% Der Schalter \ifkorrekturansicht ist bereits durch den Vorspann gesetzt.

%% latex-abspann.tex
%% Gemeinsamer Abspann für Korrekturansicht und Leseansicht.
%% Setzt den Schalter \ifkorrekturansicht voraus (gesetzt in den
%% einbindenden Dateien latex-korrekturansicht-abspann.tex bzw.
%% latex-leseansicht-abspann.tex).
%% ---------------------------------------------------------------

\normalsize

% Das esempio-Environment wird nur in der Leseansicht benötigt
\ifkorrekturansicht\else
\newenvironment{esempio}[3]%
{
    \vspace{1.5ex}
    \rlap{\underline{#1}}
    \par
    \setlength{\parindent}{0cm}
    \nopagebreak
    \leftskip=#2cm
    \rightskip=#3cm
}
{
    \par
}
\fi

\doendnotes{C}
\bigskip
\vfill

\clearpage

\footnotesize

\ifkorrekturansicht
  \lohead{\textsc{register}}
\fi

% theindex-Environment neu definieren ohne reledmac
\makeatletter
\renewenvironment{theindex}{%
  \ifkorrekturansicht
    \section*{\indexname}%
  \else
    \subsubsection*{Index der erwähnten Entitäten}%
  \fi
  \setlength{\parindent}{0pt}%
  \setlength{\parskip}{0pt plus 0.3pt}%
  \let\item\@idxitem
}{%
  \ifkorrekturansicht\clearpage\fi
}
\makeatother

\IfFileExists{\jobname-pw.ind}{\input{\jobname-pw.ind}}{}

% Quellenangabe nur in der Leseansicht
\ifkorrekturansicht\else
% Fallback-Definitionen, falls die .tex-Datei \titel etc. nicht gesetzt hat
\providecommand{\titel}{}
\providecommand{\editorInnen}{}
\providecommand{\dateiname}{\jobname}

\vspace{3cm}

\vfill

\footnotesize
\textsc{Quelle}: \titel. Herausgegeben von {\editorInnen}. In: \emph{Arthur Schnitzler: Briefwechsel mit Autorinnen und Autoren}.
 Digitale Edition, https://schnitzler-briefe.acdh.oeaw.ac.at/{\dateiname}.html (Stand \today)
\fi

\end{document}


      