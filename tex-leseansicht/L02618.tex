%% latex-leseansicht-vorspann.tex
%% Vorspann für die Leseansicht.
%% Lädt die gemeinsame Datei latex-vorspann.tex mit nicht gesetztem Schalter.

\newif\ifkorrekturansicht
\korrekturansichtfalse

\input{../tex-inputs/latex-vorspann}


\section[Paul Goldmann an Arthur Schnitzler, 3. 11. {[}1894{]}]{L02618 Paul Goldmann an Arthur Schnitzler, 3. 11. [1894]}
\nopagebreak\mylabel{L02618v}
\rehead{ }\normalsize\beginnumbering\briefempfaengerindex{Schnitzler, Arthur@\textsc{Schnitzler, Arthur}!zzzGoldmann, Paul@\emph{von Paul Goldmann}!1894-11-031@{3. 11. [1894]}|(be}
\toendnotes[C]{\smallbreak\pagebreak[2]}
\correspDesc{Versand  durch Paul Goldmann am 3. 11. [1894] in Paris
\newline{}Erhalt  durch Arthur Schnitzler im Zeitraum [4. 11. 1894
                  – 8. 11. 1894?] in Wien}\toendnotes[C]{\smallbreak}
\Standort{DLA, A:Schnitzler, HS.NZ85.1.3164.}
\physDesc{Brief, 1 Blatt, 4 Seiten, 1488 Zeichen
\newline{}Handschrift: schwarze Tinte, deutsche Kurrent
\newline{}Schnitzler: 1) mit Bleistift auf dem ersten Blatt die Jahreszahl »94« vermerkt  2) mit rotem Buntstift eine Unterstreichung}\toendnotes[C]{\smallbreak}
\pstart
           {\pb}\textcolor{gray}{\textbf{Frankfurter Zeitung\orgindex{Frankfurter Zeitung@Frankfurter Zeitung|pw}}}\hfill \textsc{Paris\oindex{Paris@\textbf{Paris}, \emph{Hauptstadt}|pw}}, 3. November.\pend
           
\pstart
           \textcolor{gray}{\textbf{(Gazette de
                     Francfort\orgindex{Frankfurter Zeitung@Frankfurter Zeitung|pw}).}}\pend
           
\pstart
           \textcolor{gray}{\textbf{\begin{otherlanguage}{french}Fondateur\end{otherlanguage}{ }\textbf{M. L. Sonnemann\pwindex{Sonnemann, Leopold 29.\,10.\,1831 Höchberg – 30.\,10.\,1909 Frankfurt am Main@\textsc{Sonnemann, Leopold} (29.\,10.\,1831 Höchberg – 30.\,10.\,1909 Frankfurt am Main), \emph{Journalist, Herausgeber}|pw}}.}}\pend
           
\pstart
           \textcolor{gray}{\textbf{\begin{otherlanguage}{french}Journal politique, financier,\end{otherlanguage}}}\pend
           
\pstart
           \textcolor{gray}{\textbf{\begin{otherlanguage}{french}commercial et littéraire.\end{otherlanguage}}}\pend
           
\pstart
           \textcolor{gray}{\textbf{\begin{otherlanguage}{french}\textbf{Paraissant trois fois par jour}\end{otherlanguage}}}.\pend
           
\pstart
           \textcolor{gray}{\textbf{\begin{otherlanguage}{french}\textbf{Bureaux à Paris\oindex{Paris@\textbf{Paris}, \emph{Hauptstadt}|pw}:}\end{otherlanguage}}}\pend
           
\pstart
           \textcolor{gray}{\textbf{\begin{otherlanguage}{french}\textbf{24. Rue Feydeau}\oindex{rue Feydeau@\textbf{rue Feydeau}, \emph{Straße}|pw}.\end{otherlanguage}}}\pend
           
\pstart\center{}Mein lieber Freund,\pend\vspace{0.5em}
\pstart
           Wir{ }ſind mitten im \label{K_L02618-1v}\edtext{Ruſſ\oindex{Russland@\textbf{Russland}|pwv}enfieber}{\lemma{\textnormal{\emph{Russenfieber}}}\Cendnote{\textnormal{Die politische Annäherung zwischen Russland\oindex{Russland@\textbf{Russland}|pwk} und Frankreich\oindex{Frankreich@\textbf{Frankreich}|pwk} führte zu einer Begeisterungswelle, die durch öffentliche
                  »Freundschaftsfeste« weiter gefördert wurde.}}}\label{K_L02618-1} und ich finde gerade Zeit, Dir
               raſch beide Hände zu drücken, mit einem innigen \label{K_L02618-2v}\edtext{Glückwunſch}{\lemma{\textnormal{\emph{Glückwunsch}}}\Cendnote{\textnormal{Siehe XXXX Auszeichnungsfehler: Dokument L00395 nicht gefunden.
               }}}\label{K_L02618-2}. So{ }ſcheint alſo der liebſte Wunſch, den ich für Dich gehegt, wahr werden zu
               wollen. Ich habe mir heut{ }Früh’, als ich Deinen lieben \label{K_L02618-3v}\edtext{Brief}{\lemma{\textnormal{\emph{Brief}}}\Cendnote{\textnormal{Vgl. A. S.: \emph{Tagebuch}, 31. 10. 1894.
               }}}\label{K_L02618-3} erhielt, die Zukunft ausgemalt und habe mich an all’ dem Licht und der Freude
               ergötzt, die ich darin für Dich fand. Ich bin{ }ſicher: Du wirſt {\pb}aufgeführt werden; ich bin{ }ſicher: Du wirſt Erfolg haben\substVorne{}\textsuperscript{.}\substDazwischen{},\substHinten{} –{ }ſo{ }ſicher, daß mir iſt, als{ }ſei das Alles{ }ſchon geſchehen. \textsc{B.\pwindex{Burckhard, Max Eugen 14.\,7.\,1854 Korneuburg – 16.\,3.\,1912 Wien@\textsc{Burckhard, Max Eugen} (14.\,7.\,1854 Korneuburg – 16.\,3.\,1912 Wien), \emph{Schriftsteller, Rechtswissenschaftler, Theaterleiter}|pw}}’s Telegramm bedeutet{ }ſicher die Annahme, und der Director\pwindex{Burckhard, Max Eugen 14.\,7.\,1854 Korneuburg – 16.\,3.\,1912 Wien@\textsc{Burckhard, Max Eugen} (14.\,7.\,1854 Korneuburg – 16.\,3.\,1912 Wien), \emph{Schriftsteller, Rechtswissenschaftler, Theaterleiter}|pwv} gefällt mir{ }ſehr, der in dieſer
               Form anzunehmen verſteht. Bitte,{ }ſchreib’ mir{ }ſofort, \strikeout{daß} wie die Unterredung mit \textsc{B.\pwindex{Burckhard, Max Eugen 14.\,7.\,1854 Korneuburg – 16.\,3.\,1912 Wien@\textsc{Burckhard, Max Eugen} (14.\,7.\,1854 Korneuburg – 16.\,3.\,1912 Wien), \emph{Schriftsteller, Rechtswissenschaftler, Theaterleiter}|pw}} ausgefallen. Im Übrigen will ich gar nicht länger darüber reden, aus
               Aberglauben – denn es iſt gar zu{ }ſchön. Und den Namen des Theaters\orgindex{Burgtheater@Burgtheater|pwv} nenne ich erſt gar nicht, auch aus
               Aberglauben. Aber froh bin ich; und {\pb}ich fühle die
               glückliche Wendung und denke, daß Niemand in der Welt{ }ſie mehr verdient hat, als Du,
               mein lieber Freund.\pend
           
\pstart
           Ich \strikeout{\textcolor{gray}{be}} möchte gern das Alles beſſer{ }ſagen. Aber es iſt{ }ſo{ }ſchwer, über die guten
               Dinge zu{ }ſchreiben{[}.{]} Überdies empfing ich heut mein \label{K_L02618-4v}\edtext{Feuilleton\pwindex{Goldmann, Paul 31.\,1.\,1865 Breslau – 25.\,9.\,1935 Wien@\textsc{Goldmann, Paul} (31.\,1.\,1865 Breslau – 25.\,9.\,1935 Wien), \emph{Schriftsteller, Journalist}!Gismonda«@\strich\emph{»Gismonda«}|pwv}}{\lemma{\textnormal{\emph{Feuilleton}}}\Cendnote{\textnormal{G.\pwindex{Goldmann, Paul 31.\,1.\,1865 Breslau – 25.\,9.\,1935 Wien@\textsc{Goldmann, Paul} (31.\,1.\,1865 Breslau – 25.\,9.\,1935 Wien), \emph{Schriftsteller, Journalist}|pwk} [ = Paul Goldmann\pwindex{Goldmann, Paul 31.\,1.\,1865 Breslau – 25.\,9.\,1935 Wien@\textsc{Goldmann, Paul} (31.\,1.\,1865 Breslau – 25.\,9.\,1935 Wien), \emph{Schriftsteller, Journalist}|pwk}]: \emph{»Gismonda«}\pwindex{Goldmann, Paul 31.\,1.\,1865 Breslau – 25.\,9.\,1935 Wien@\textsc{Goldmann, Paul} (31.\,1.\,1865 Breslau – 25.\,9.\,1935 Wien), \emph{Schriftsteller, Journalist}!Gismonda«@\strich\emph{»Gismonda«}|pwk}. In:
                        \emph{Frankfurter Zeitung}\pwindex{Frankfurter Zeitung@\emph{Frankfurter Zeitung}|pwk}, Jg. 39, Nr. 305,
                        3. 11. 1894, Erstes Morgenblatt, S. 1–2.}}}\label{K_L02618-4} über
                  »\label{K_L02618-5v}\edtext{\textsc{Gismonda\pwindex{Sardou, Victorien 7.\,9.\,1831 Paris – 8.\,11.\,1908 ebd.@\textsc{Sardou, Victorien} (7.\,9.\,1831 Paris – 8.\,11.\,1908 ebd.), \emph{Schriftsteller}!Gismonda. Pièce en 4 actes et 5 tableaux@\strich\emph{Gismonda. Pièce en 4 actes et 5 tableaux}|pw}}}{\lemma{\textnormal{\emph{Gismonda}}}\Cendnote{\textnormal{\emph{Gismonda. Pièce en 4 actes et 5 tableaux}\pwindex{Sardou, Victorien 7.\,9.\,1831 Paris – 8.\,11.\,1908 ebd.@\textsc{Sardou, Victorien} (7.\,9.\,1831 Paris – 8.\,11.\,1908 ebd.), \emph{Schriftsteller}!Gismonda. Pièce en 4 actes et 5 tableaux@\strich\emph{Gismonda. Pièce en 4 actes et 5 tableaux}|pwk}, von
                   Victorien Sardou\pwindex{Sardou, Victorien 7.\,9.\,1831 Paris – 8.\,11.\,1908 ebd.@\textsc{Sardou, Victorien} (7.\,9.\,1831 Paris – 8.\,11.\,1908 ebd.), \emph{Schriftsteller}|pwk} für Sarah Bernhardt\pwindex{Bernhardt, Sarah 22.\,10.\,1844 Paris – 26.\,3.\,1923 ebd.@\textsc{Bernhardt, Sarah} (22.\,10.\,1844 Paris – 26.\,3.\,1923 ebd.), \emph{Schauspielerin}|pwk} geschrieben, erlebte seine Uraufführung\eventindex{Paris@\textbf{Paris}!Uraufführung von Gismonda, 31.10.1894@Uraufführung von Gismonda, 31.10.1894|pwkv} am
                     31. 10. 1894 am \emph{Théâtre de la
                     Renaissance}\orgindex{Théâtre de la Renaissance@Théâtre de la Renaissance|pwk}.}}}\label{K_L02618-5}«, das mein Onkel\pwindex{Mamroth, Fedor 21.\,2.\,1851 Breslau – 25.\,6.\,1907 Frankfurt am Main@\textsc{Mamroth, Fedor} (21.\,2.\,1851 Breslau – 25.\,6.\,1907 Frankfurt am Main), \emph{Journalist, Kritiker}|pwv} in einer irrſinnigen Weiſe zuſammengeſtrichen hat.
               Das iſt ein Lähmungsſchlag ins Gehirn.\pend
           
\pstart
           Ich danke Dir von ganzem Herzen für den Freundſchafts-Beweis, den Du mir gegeben,
               indem Du mir{ }ſofort die {\pb}Nachricht mitgetheilt; und
               ich begrüße Dich vielmals und in Treue\pend
           
\pstart
           Dein {\\[\baselineskip]}\spacefill\mbox{Paul Goldmann}\pend
           \leftskip=0em{}\selectlanguage{ngerman}\endnumbering\briefempfaengerindex{Schnitzler, Arthur@\textsc{Schnitzler, Arthur}!zzzGoldmann, Paul@\emph{von Paul Goldmann}!1894-11-031@{3. 11. [1894]}|)be}\mylabel{L02618h}  \newcommand{\dateiname}{L02618}\newcommand{\titel}{Paul Goldmann an Arthur Schnitzler, 3. 11. [1894]}\newcommand{\editorInnen}{Martin Anton Müller und Laura Untner}%% latex-leseansicht-abspann.tex
%% Abspann für die Leseansicht.
%% Der Schalter \ifkorrekturansicht ist bereits durch den Vorspann gesetzt.

%% latex-abspann.tex
%% Gemeinsamer Abspann für Korrekturansicht und Leseansicht.
%% Setzt den Schalter \ifkorrekturansicht voraus (gesetzt in den
%% einbindenden Dateien latex-korrekturansicht-abspann.tex bzw.
%% latex-leseansicht-abspann.tex).
%% ---------------------------------------------------------------

\normalsize

% Das esempio-Environment wird nur in der Leseansicht benötigt
\ifkorrekturansicht\else
\newenvironment{esempio}[3]%
{
    \vspace{1.5ex}
    \rlap{\underline{#1}}
    \par
    \setlength{\parindent}{0cm}
    \nopagebreak
    \leftskip=#2cm
    \rightskip=#3cm
}
{
    \par
}
\fi

\doendnotes{C}
\bigskip
\vfill

\clearpage

\footnotesize

\ifkorrekturansicht
  \lohead{\textsc{register}}
\fi

% theindex-Environment neu definieren ohne reledmac
\makeatletter
\renewenvironment{theindex}{%
  \ifkorrekturansicht
    \section*{\indexname}%
  \else
    \subsubsection*{Index der erwähnten Entitäten}%
  \fi
  \setlength{\parindent}{0pt}%
  \setlength{\parskip}{0pt plus 0.3pt}%
  \let\item\@idxitem
}{%
  \ifkorrekturansicht\clearpage\fi
}
\makeatother

\IfFileExists{\jobname-pw.ind}{\input{\jobname-pw.ind}}{}

% Quellenangabe nur in der Leseansicht
\ifkorrekturansicht\else
% Fallback-Definitionen, falls die .tex-Datei \titel etc. nicht gesetzt hat
\providecommand{\titel}{}
\providecommand{\editorInnen}{}
\providecommand{\dateiname}{\jobname}

\vspace{3cm}

\vfill

\footnotesize
\textsc{Quelle}: \titel. Herausgegeben von {\editorInnen}. In: \emph{Arthur Schnitzler: Briefwechsel mit Autorinnen und Autoren}.
 Digitale Edition, https://schnitzler-briefe.acdh.oeaw.ac.at/{\dateiname}.html (Stand \today)
\fi

\end{document}


