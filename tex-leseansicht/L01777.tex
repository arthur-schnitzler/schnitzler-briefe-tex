%% latex-korrekturansicht-vorspann.tex
%% Vorspann für die Korrekturansicht.
%% Lädt die gemeinsame Datei latex-vorspann.tex mit gesetztem Schalter.

\newif\ifkorrekturansicht
\korrekturansichttrue

\input{../tex-inputs/latex-vorspann}


\section[Georg Brandes an Arthur Schnitzler, {[}25. 6. 1908{]}]{L01777 Georg Brandes an Arthur Schnitzler, {[}25. 6. 1908{]}}
\nopagebreak\mylabel{L01777v}
\rehead{ }\normalsize\beginnumbering\briefempfaengerindex{Schnitzler, Arthur@\textsc{Schnitzler, Arthur}!zzzBrandes, Georg@\emph{von Georg Brandes}!1908-06-251@{{[}25. 6. 1908{]}}|(be}
\toendnotes[C]{\smallbreak\pagebreak[2]}\Standort{CUL, Schnitzler, B 17.}
\physDesc{Brief, 1 Blatt, 4 Seiten, 2043 Zeichen
\newline{}Handschrift: schwarze Tinte, lateinische Kurrent
\newline{}Schnitzler: mit Bleistift beschriftet: »\textsc{Brandes}« und datiert: »Ende Juni 908« 
\newline{}Ordnung: mit Bleistift von unbekannter Hand nummeriert:
                                    »33« }
\buchAbdrucke{\weitereDrucke{Georg Brandes, Arthur Schnitzler: \emph{Ein Briefwechsel}. Bern: \emph{Francke} 1956, S. 95.} }\toendnotes[C]{\smallbreak}
\pstart
           {\pb}\textcolor{gray}{\textbf{Villa San Remigio}}\oindex{Villa San Remigio@\textbf{Villa San Remigio}, \emph{Wohngebäude (K.WHS)}|pw}, {\\}\textcolor{gray}{\textbf{Pallanza\oindex{Pallanza@\textbf{Pallanza}, \emph{P.PPL}|pw},}}{\\}\textcolor{gray}{\textbf{Lago Maggiore\oindex{Lago Maggiore@\textbf{Lago Maggiore}, \emph{H.LK}|pw}.}}\pend
           
\pstart{}Verehrter Freund\pend\vspace{0.5em}
\pstart
           Seien Sie bedankt, dass Sie, obwohl wir uns so selten sehen, sich immer meiner
               erinnern und mir die Freude bereiten, jedes neues Buch, dass Sie hervorbringen, aus
               Ihren eigenen Händen zu erhalten. Es ist mir, der ich so viele Bücher bekomme, immer
               ein \uline{Fest}, wenn eines von Ihnen anlangt.\pend
           
\pstart
           Ich habe Ihr Buch\pwindex{Weg ins Freie. Roman@\emph{Der Weg ins Freie. Roman}|pwv} auf einer
               Reise gelesen, langsam und sorgfältig und mit so grossem Interesse, dass jede
               Unterbrechung mir unlieb war.\pend
           
\pstart
           {\pb}Ich bin traurig, dass ich Ihnen
               nie ein Buch von ähnlichem Interesse von mir hätte schicken können. Und meine Sachen
               in deutscher Uebersetzung sind mir ein solches Greuel, dass ich sie nicht ansehen
               kann.\pend
           
\pstart
           Leider kenne ich nicht Oesterreich\oindex{Oesterreich@\textbf{Österreich}, \emph{A.PCLI}|pw} oder Wien\oindex{Wien@\textbf{Wien}, \emph{A.ADM2}|pw} gut genug, um im Stande zu sein, eine Ansicht
               darüber zu haben, wie ähnlich das Bild ist, das Sie geben. Es scheint ähnlich. Aber
               haben Sie nicht zwei Bücher geschrieben? Das Verhältnis des jungen Barons\pwindex{Weg ins Freie. Roman@\emph{Der Weg ins Freie. Roman}|pwv} zu seiner Geliebten ist Eine Sache,
               und die {\pb}neue Lage der jüdischen
               Bevölkerung in Wien\oindex{Wien@\textbf{Wien}, \emph{A.ADM2}|pw} durch den Antisemitismus eine
               andere, die mit der ersteren, scheint mir, in nicht notwendiger Verbindung steht. Die
               Geliebte ist nicht Jüdin.\pend
           
\pstart
           Das Thema: die Zärtlichkeit gegen das weibliche Wesen, mit Angst vor der Ehe
               versetzt, und die Collisionen, die diese Combination veranlasst, \strikeout{ist} macht vielleicht ein Buch für sich. Die
               Zerrissenheit einiger Juden, die unruhigen Begierden einiger junger Jüdinnen, der
               Snobismus eines jüdischen Jünglings, der {\pb}Mut und die Innigkeit eines
               anderen, die Keckheit, der Leichtsinn und der Ernst der Therese\pwindex{Weg ins Freie. Roman@\emph{Der Weg ins Freie. Roman}|pwv} bilden aber zusammen den Kern des
               Buches, nicht wahr? Ich freue mich über den inneren Reichthum des Werkes und sehe ja
               sehr gut die vielen Zusammenhänge (z. B. dass das Wesen der Juden dem Baron
               unverständlich und doch verständlich ist) aber nicht den strengen nothwendigen
               Zusammenhang. – Ihre Gestalten sind fesselnd. Ich kenne nicht eben solche Menschen,
               aber glaube an ihre Wahrheit.\pend
           
\pstart
           Wenige Bücher fesseln mich wie die ihrigen. Ich glaube immer etwas Verwandtes zu
               spüren.\pend
           
\pstart
           Ich habe Sie kurz gesagt ausserordentlich lieb.\pend
           \pstart Ihr \spacefill\mbox{Georg Brandes}\pend{}\selectlanguage{ngerman}\endnumbering\briefempfaengerindex{Schnitzler, Arthur@\textsc{Schnitzler, Arthur}!zzzBrandes, Georg@\emph{von Georg Brandes}!1908-06-251@{{[}25. 6. 1908{]}}|)be}\mylabel{L01777h}  \normalsize

\doendnotes{C}
\bigskip
\vfill

\clearpage

\footnotesize

\lohead{\textsc{register}}

% Definiere theindex-Environment komplett neu ohne reledmac
\makeatletter
\renewenvironment{theindex}{%
  \section*{\indexname}%
  \setlength{\parindent}{0pt}%
  \setlength{\parskip}{0pt plus 0.3pt}%
  \let\item\@idxitem
}{%
  \clearpage
}
\makeatother

\IfFileExists{\jobname-pw.ind}{\input{\jobname-pw.ind}}{}

\end{document}

      