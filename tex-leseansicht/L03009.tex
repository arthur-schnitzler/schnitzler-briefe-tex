%% latex-leseansicht-vorspann.tex
%% Vorspann für die Leseansicht.
%% Lädt die gemeinsame Datei latex-vorspann.tex mit nicht gesetztem Schalter.

\newif\ifkorrekturansicht
\korrekturansichtfalse

\input{../tex-inputs/latex-vorspann}


\section[ Arthur Schnitzler an Felix Salten, 5. 8. 1907]{L03009 Arthur Schnitzler an Felix Salten,  5. 8. 1907}
\nopagebreak\mylabel{L03009v}
\rehead{ }\normalsize\beginnumbering\briefempfaengerindex{Salten, Felix@\textsc{Salten, Felix}!zzzSchnitzler, Arthur@\emph{von Arthur Schnitzler}!1907-08-051@{5. 8. 1907}|(be}
\toendnotes[C]{\smallbreak\pagebreak[2]}
\correspDesc{Versand  durch Arthur Schnitzler am 5. 8. 1907 in Welsberg-Taisten
\newline{}Erhalt  durch Felix Salten im Zeitraum [6. 8. 1907
                  – 10. 8. 1907?] in Wien}\toendnotes[C]{\smallbreak}
\Standort{Wienbibliothek im Rathaus, ZPH 1681, 2.1.516.}
\physDesc{Brief, 3 Blätter, 6 Seiten, 2900 Zeichen
\newline{}Handschrift: Bleistift, deutsche Kurrent (\noindent{}Text und Nummerierung der Blätter: »1«–»3«)
\newline{}Ordnung: mit Bleistift von unbekannter Hand Nummerierung der Doppelseiten des
                                 Konvoluts: »8«–»10« }
\buchAbdrucke{\weitereDrucke{1) Arthur Schnitzler: \emph{Briefe 1875–1912}. Herausgegeben von Therese Nickl und Heinrich Schnitzler. Frankfurt am Main: \emph{S. Fischer} 1981, S. 560–561.} \weitereDrucke{2) Hermann Bahr, Arthur Schnitzler: \emph{Briefwechsel, Aufzeichnungen, Dokumente (1891–1931)}. Herausgegeben von Kurt Ifkovits und Martin Anton Müller. Göttingen: \emph{Wallstein} 2018, S. 395.} }\toendnotes[C]{\smallbreak}
\pstart
           {\pb}\textcolor{gray}{\textbf{Telegramm-Adresse: \textbf{Böhm\pwindex{Böhm, Josef @\textsc{Böhm, Josef}, \emph{Hotelbesitzer}|pw} – Welsberg\oindex{Welsberg-Taisten@\textbf{Welsberg-Taisten}, \emph{Verwaltungsgebiet}|pw}.}}}\pend
           
\pstart
           \textcolor{gray}{\textbf{\textbf{Hôtel {\kaufmannsund} Pension Wildbad Waldbrunn\oindex{Wildbad Waldbrunn@\textbf{Wildbad Waldbrunn}, \emph{Spa}|pw}}}}\pend
           
\pstart
           \textcolor{gray}{\textbf{bei \textbf{Welsberg\oindex{Welsberg-Taisten@\textbf{Welsberg-Taisten}, \emph{Verwaltungsgebiet}|pw}} (Eilzughaltestelle)}}\pend
           
\pstart
           \textcolor{gray}{\textbf{1150 M. \textsuperscript{ü}/Meer. \hspace*{1.5em}Hochpusterthal\oindex{Pustertal@\textbf{Pustertal}, \emph{Tal}|pw} (Tirol\oindex{Tirol@\textbf{Tirol}, \emph{Land}|pw})}}\pend
           
\pstart
           \textcolor{gray}{\textbf{Heilkräftiges altbekanntes Bad in prachtvoller Lage.}}\pend
           
\pstart
           \textcolor{gray}{\textbf{\textbf{Ausgezeichnete Trinkquelle}.}}\pend
           
\pstart
           \textcolor{gray}{\textbf{70 mit allem Comfort eingerichtete Zimmer.}}\pend
           
\pstart
           \raggedleft{}\textcolor{gray}{\textbf{\emph{Waldbrunn\oindex{Welsberg-Taisten@\textbf{Welsberg-Taisten}, \emph{Verwaltungsgebiet}|pw}, den}}}{ }5. 8. \textcolor{gray}{\textbf{\emph{190}}}7\pend
           \vspace{0.5em}
\pstart
           lieber, ich danke Ihnen für Ihre Nachrichten, laſſen Sie uns jetzt
               nur bald hören, dſs Ihre Frau\pwindex{Salten, Ottilie 7.\,3.\,1868 Prag – 22.\,6.\,1942 Zürich@\textsc{Salten, Ottilie} (7.\,3.\,1868 Prag – 22.\,6.\,1942 Zürich), \emph{Schauspielerin}|pwv}{ }ſich vollko{\geminationm}en erholt hat. Dem Buben\pwindex{Salten, Paul 11.\,8.\,1903 Wien – 8.\,5.\,1937 ebd.@\textsc{Salten, Paul} (11.\,8.\,1903 Wien – 8.\,5.\,1937 ebd.), \emph{Filmcutter}|pwv} geht’s wohl{ }ſchon wieder
               ganz gut? Wir{ }ſind nun einen vollen Monat da\oindex{Wildbad Waldbrunn@\textbf{Wildbad Waldbrunn}, \emph{Spa}|pwv} und werden wahrſcheinlich \label{K_L03009-1v}\edtext{bis nach dem 20. bleiben}{\lemma{\textnormal{\emph{bis nach dem 20. bleiben}}}\Cendnote{\textnormal{Sie blieben bis zum 26. 8. 1907.}}}\label{K_L03009-1}.
                  \label{K_L03009-2v}\edtext{Heute ko{\geminationm}t meine Mama\pwindex{Schnitzler, Louise 8.\,7.\,1840 Kőszeg – 9.\,9.\,1911 Wien@\textsc{Schnitzler, Louise} (8.\,7.\,1840 Kőszeg – 9.\,9.\,1911 Wien)|pwv} an, vielleicht ni{\geminationm}t{ }ſie Heini\pwindex{Schnitzler, Heinrich 9.\,8.\,1902 Hinterbrühl – 12.\,7.\,1982 Wien@\textsc{Schnitzler, Heinrich} (9.\,8.\,1902 Hinterbrühl – 12.\,7.\,1982 Wien), \emph{Regisseur, Schauspieler}|pw} mit
               nach Wien\oindex{Wien@\textbf{Wien}, \emph{Verwaltungsgebiet}|pw}}{\lemma{\textnormal{\emph{Heute … Wien}}}\Cendnote{\textnormal{Louise Schnitzler\pwindex{Schnitzler, Louise 8.\,7.\,1840 Kőszeg – 9.\,9.\,1911 Wien@\textsc{Schnitzler, Louise} (8.\,7.\,1840 Kőszeg – 9.\,9.\,1911 Wien)|pwk} war zwischen 5. 8. 1907 und 24. 8. 1907 in Welsberg\oindex{Welsberg-Taisten@\textbf{Welsberg-Taisten}, \emph{Verwaltungsgebiet}|pwk}. Heinrich Schnitzler\pwindex{Schnitzler, Heinrich 9.\,8.\,1902 Hinterbrühl – 12.\,7.\,1982 Wien@\textsc{Schnitzler, Heinrich} (9.\,8.\,1902 Hinterbrühl – 12.\,7.\,1982 Wien), \emph{Regisseur, Schauspieler}|pwk} reiste erst am 26. 8. 1907 ab.}}}\label{K_L03009-2}; da{\geminationn} wollen wir, Olga\pwindex{Schnitzler, Olga 17.\,1.\,1882 Wien – 13.\,1.\,1970 Lugano@\textsc{Schnitzler, Olga} (17.\,1.\,1882 Wien – 13.\,1.\,1970 Lugano), \emph{Schauspielerin, Sängerin}|pw}
               u ich{[},{]} noch \label{K_L03009-3v}\edtext{ſüdlicher}{\lemma{\textnormal{\emph{südlicher}}}\Cendnote{\textnormal{Siehe XXXX Auszeichnungsfehler: Dokument L03489 nicht gefunden. }}}\label{K_L03009-3},
               vielleicht, u theilweiſe zu Fuſs, über die neue Dolomitenstraße\oindex{Große Dolomitenstraße@\textbf{Große Dolomitenstraße}, \emph{Straße}|pw}; nach Bozen\oindex{Bozen@\textbf{Bozen}, \emph{Hauptstadt}|pw}. In Meran\oindex{Meran@\textbf{Meran}, \emph{Hauptstadt}|pw} oder am Gardaſee\oindex{Lago di Garda@\textbf{Lago di Garda}, \emph{See}|pw} denken wir eine Woche zu raſten und da{\geminationn}, in den erſten Septembertagen, in Wien\oindex{Wien@\textbf{Wien}, \emph{Verwaltungsgebiet}|pw} einzutreffen. Möglich, daſs wir irgendwo \label{K_L03009-4v}\edtext{mit {\pb}Richard\pwindex{Beer-Hofmann, Richard 11.\,7.\,1866 Wien – 26.\,9.\,1945 New York City@\textsc{Beer-Hofmann, Richard} (11.\,7.\,1866 Wien – 26.\,9.\,1945 New York City), \emph{Schriftsteller}|pw} u Paula\pwindex{Beer-Hofmann, Paula 25.\,2.\,1879 Wien – 30.\,10.\,1939 Zürich@\textsc{Beer-Hofmann, Paula} (25.\,2.\,1879 Wien – 30.\,10.\,1939 Zürich)|pw} zuſa{\geminationm}entreffen}{\lemma{\textnormal{\emph{mit … zusammentreffen}}}\Cendnote{\textnormal{Dazu kam es nicht, vgl. XXXX Auszeichnungsfehler: Dokument L01703 nicht gefunden und XXXX Auszeichnungsfehler: Dokument L01706 nicht gefunden. }}}\label{K_L03009-4}. Sie wollen im September
               eine Meerfahrt unternehmen? Thäts der Gardaſee\oindex{Lago di Garda@\textbf{Lago di Garda}, \emph{See}|pw}
               nicht auch? Mein Rad hab ich nicht mit, bedaure es auch nicht{ }ſehr, da meine Zeit
               reichlich ausgefüllt iſt. Vormittg Waldwanderungen, allein, oder mit Olga\pwindex{Schnitzler, Olga 17.\,1.\,1882 Wien – 13.\,1.\,1970 Lugano@\textsc{Schnitzler, Olga} (17.\,1.\,1882 Wien – 13.\,1.\,1970 Lugano), \emph{Schauspielerin, Sängerin}|pw}; Nachmittg 2–6 etwa arbeit
               ich; da{\geminationn}{ }ſpaziren; da{\geminationn}
               Nachtmahl und \label{K_L03009-5v}\edtext{Platformwandelei}{\lemma{\textnormal{\emph{Platformwandelei}}}\Cendnote{\textnormal{Die Schreibweise deutet auf eine
                  englischsprachige Aussprache durch Schnitzler hin.}}}\label{K_L03009-5}. Tennis haben wir erſt einmal geſpielt – der Platz
               lächerlich; unſre Partnerin ware eine{ }ſehr charmante junge Frau \textsc{Epstein\pwindex{Epstein, Marie 6.\,4.\,1880 Wien – 3.\,1.\,1953 ebd.@\textsc{Epstein, Marie} (6.\,4.\,1880 Wien – 3.\,1.\,1953 ebd.)|pw}} (geboren \textsc{Miss Hudetz\pwindex{Epstein, Marie 6.\,4.\,1880 Wien – 3.\,1.\,1953 ebd.@\textsc{Epstein, Marie} (6.\,4.\,1880 Wien – 3.\,1.\,1953 ebd.)|pw}}), Schwägerin der \textsc{Anna – Epstein Loeb\pwindex{Epstein, Anna 6.\,3.\,1877 Wien – 16.\,3.\,1943 Konzentrationslager Theresienstadt@\textsc{Epstein, Anna} (6.\,3.\,1877 Wien – 16.\,3.\,1943 Konzentrationslager Theresienstadt)|pw}}. Ferner befinden{ }ſich hier die Schweſtern\pwindex{Guttmann, Leonie @\textsc{Guttmann, Leonie}, \emph{Übersetzerin}|pwv}\pwindex{Frankfurter, Ella 2.\,2.\,1873 Budapest – 5.\,10.\,1957 Wien@\textsc{Frankfurter, Ella} (2.\,2.\,1873 Budapest – 5.\,10.\,1957 Wien), \emph{Malerin}|pwv} der Frau \textsc{Auernheimer\pwindex{Auernheimer, Irene 6.\,3.\,1880 Budapest – 30.\,10.\,1967 Oakland@\textsc{Auernheimer, Irene} (6.\,3.\,1880 Budapest – 30.\,10.\,1967 Oakland)|pw}}, und allerlei \label{K_L03009-6v}\edtext{\textsc{Ascenden\textcolor{gray}{z}} u \textsc{Descendenz}}{\lemma{\textnormal{\emph{Ascendenz u Descendenz}}}\Cendnote{\textnormal{Auf- und Absteigendes}}}\label{K_L03009-6}; zum Theil
               gutes u. vorzügliches Menſchenmaterial. Der Mann der verheirateten Schweſter\pwindex{Frankfurter, Ella 2.\,2.\,1873 Budapest – 5.\,10.\,1957 Wien@\textsc{Frankfurter, Ella} (2.\,2.\,1873 Budapest – 5.\,10.\,1957 Wien), \emph{Malerin}|pwv}, Frankfurter\pwindex{Frankfurter, Albert 27.\,3.\,1868 Cincinnati – 29.\,3.\,1952@\textsc{Frankfurter, Albert} (27.\,3.\,1868 Cincinnati – 29.\,3.\,1952), \emph{Generaldirektor}|pw} mit Namen, Direktor {\pb}des oeſterr.
                  Lloyd\orgindex{Österreichischer Lloyd@Österreichischer Lloyd|pw},{ }ſcheint was nicht gewöhnliches zu{ }ſein. – Daſs Bahr\pwindex{Bahr, Hermann 19.\,7.\,1863 Linz – 15.\,1.\,1934 München@\textsc{Bahr, Hermann} (19.\,7.\,1863 Linz – 15.\,1.\,1934 München), \emph{Schriftsteller, Kritiker}|pw} Sie gegen Pötzl\pwindex{Pötzl, Eduard 17.\,3.\,1851 Wien – 20.\,8.\,1914 Mödling@\textsc{Pötzl, Eduard} (17.\,3.\,1851 Wien – 20.\,8.\,1914 Mödling), \emph{Schriftsteller, Journalist}|pw} –
               wie{ }ſoll man da{ }ſagen – in Schmutz nehmen? – mußte, hat uns{ }ſehr amusirt. We{\geminationn} ich{ }ſowohl Ihren Morgen\pwindex{Morgen. Wochenschrift für deutsche Kultur@\emph{Morgen. Wochenschrift für deutsche Kultur}|pw}ruf\pwindex{Salten, Felix 6.\,9.\,1869 Budapest – 8.\,10.\,1945 Zürich@\textsc{Salten, Felix} (6.\,9.\,1869 Budapest – 8.\,10.\,1945 Zürich), \emph{Schriftsteller, Journalist, Chefredakteur}!Wiener Korrespondent@\strich\emph{Der Wiener Korrespondent}|pwv} als Pötzl\pwindex{Pötzl, Eduard 17.\,3.\,1851 Wien – 20.\,8.\,1914 Mödling@\textsc{Pötzl, Eduard} (17.\,3.\,1851 Wien – 20.\,8.\,1914 Mödling), \emph{Schriftsteller, Journalist}|pw}’s Lobeshymne\pwindex{Pötzl, Eduard 17.\,3.\,1851 Wien – 20.\,8.\,1914 Mödling@\textsc{Pötzl, Eduard} (17.\,3.\,1851 Wien – 20.\,8.\,1914 Mödling), \emph{Schriftsteller, Journalist}!gelobte Wien@\strich\emph{Das gelobte Wien}|pwv} zu leſen beko{\geminationm}en könnte, wär ich
               Ihnen herzlich verbunden. (Daſs Sie mir die berühmte \label{K_L03009-7v}\edtext{Sa{\geminationm}lung der 12 Berl\oindex{Berlin@\textbf{Berlin}, \emph{Hauptstadt}|pw}. Feu{[}i{]}lletons}{\lemma{\textnormal{\emph{Sammlung … Feuilletons}}}\Cendnote{\textnormal{Es dürfte sich um Saltens\pwindex{Salten, Felix 6.\,9.\,1869 Budapest – 8.\,10.\,1945 Zürich@\textsc{Salten, Felix} (6.\,9.\,1869 Budapest – 8.\,10.\,1945 Zürich), \emph{Schriftsteller, Journalist, Chefredakteur}|pwk} Beiträge für die \emph{B. Z. am
                     Mittag}\pwindex{B.Z. am Mittag@\emph{B.Z. am Mittag}|pwk} handeln. Abgesehen von einer Ausnahme fehlen diese vollständig in
                     Saltens\pwindex{Salten, Felix 6.\,9.\,1869 Budapest – 8.\,10.\,1945 Zürich@\textsc{Salten, Felix} (6.\,9.\,1869 Budapest – 8.\,10.\,1945 Zürich), \emph{Schriftsteller, Journalist, Chefredakteur}|pwk} Zusammenstellungen seiner
                  journalistischen Arbeiten in seinem Nachlass. Das kann als Indiz genommen
                  werden, dass Salten\pwindex{Salten, Felix 6.\,9.\,1869 Budapest – 8.\,10.\,1945 Zürich@\textsc{Salten, Felix} (6.\,9.\,1869 Budapest – 8.\,10.\,1945 Zürich), \emph{Schriftsteller, Journalist, Chefredakteur}|pwk} mit den Texten eine
                  Publikation plante oder sie zumindest als zusammengehörig betrachtete. Saltens\pwindex{Salten, Felix 6.\,9.\,1869 Budapest – 8.\,10.\,1945 Zürich@\textsc{Salten, Felix} (6.\,9.\,1869 Budapest – 8.\,10.\,1945 Zürich), \emph{Schriftsteller, Journalist, Chefredakteur}|pwk} Brief vom XXXX Auszeichnungsfehler: Dokument L03510 nicht gefunden lässt zudem
                  vermuten, dass es sich um Beiträge zu seiner England\oindex{England@\textbf{England}, \emph{Land}|pwk}-Reise im Juni 1906 handelte, vgl. XXXX Auszeichnungsfehler: Dokument L03427 nicht gefunden.}}}\label{K_L03009-7} noch immer
               nicht gegeben haben, nur nebenbei.) Wie{ }ſtehts im übrigen mit Ihren Arbeiten? In
               welcher{ }ſtecken Sie am liebſten? – Ich{ }ſchreibe hier nur an dem Roman\pwindex{Schnitzler, Arthur 15.\,5.\,1862 Wien – 21.\,10.\,1931 ebd.@\textsc{Schnitzler, Arthur} (15.\,5.\,1862 Wien – 21.\,10.\,1931 ebd.), \emph{Schriftsteller, Mediziner}!Weg ins Freie. Roman@\strich\emph{Der Weg ins Freie. Roman}|pwv}; letzte, zum Theil wohl {\pb}vorletzte Feile; habe ein wunderſchönes
               Zimmer, in das vom Hoteltrubel nichts dringt, mit einem guten Blick über Wieſen und
               Wald ins Thal; vorgebauter Balkon; oberſter Stock. – (Das idealſte Arbeitszimmer –
               ohne dieſes, glaub ich, hielt es mich doch nicht{ }ſo lang hier). An \label{K_L03009-8v}\edtext{Lienz\oindex{Lienz@\textbf{Lienz}, \emph{Hauptstadt}|pw} vorüberfahrend und an \textsc{Dölsach\oindex{Dölsach@\textbf{Dölsach}, \emph{Verwaltungsgebiet}|pw}}}{\lemma{\textnormal{\emph{Lienz … Dölsach}}}\Cendnote{\textnormal{Vgl. XXXX Auszeichnungsfehler: Dokument L03127 nicht gefunden. }}}\label{K_L03009-8} (ſo heißts
               doch) blieb ich nicht ungerührt – – \label{K_L03009-9v}\edtext{»wie war ich jung\pwindex{Schnitzler, Arthur 15.\,5.\,1862 Wien – 21.\,10.\,1931 ebd.@\textsc{Schnitzler, Arthur} (15.\,5.\,1862 Wien – 21.\,10.\,1931 ebd.), \emph{Schriftsteller, Mediziner}!Ruf des Lebens. Schauspiel in drei Akten@\strich\emph{Der Ruf des Lebens. Schauspiel in drei Akten}|pwv}« heißt
               es in der{ }ſchönſten Scene\pwindex{Schnitzler, Arthur 15.\,5.\,1862 Wien – 21.\,10.\,1931 ebd.@\textsc{Schnitzler, Arthur} (15.\,5.\,1862 Wien – 21.\,10.\,1931 ebd.), \emph{Schriftsteller, Mediziner}!Ruf des Lebens. Schauspiel in drei Akten@\strich\emph{Der Ruf des Lebens. Schauspiel in drei Akten}|pw}}{\lemma{\textnormal{\emph{»wie … Scene}}}\Cendnote{\textnormal{\emph{Der Ruf des Lebens}\pwindex{Schnitzler, Arthur 15.\,5.\,1862 Wien – 21.\,10.\,1931 ebd.@\textsc{Schnitzler, Arthur} (15.\,5.\,1862 Wien – 21.\,10.\,1931 ebd.), \emph{Schriftsteller, Mediziner}!Ruf des Lebens. Schauspiel in drei Akten@\strich\emph{Der Ruf des Lebens. Schauspiel in drei Akten}|pwk}, 1. Akt, 7. Szene}}}\label{K_L03009-9}
               die ich je geſchrieben habe (aber es{ }ſtehen auch originellere Sachen drin.) – Leſe
               hauptſächlich \textsc{Bülow (Hans v.\pwindex{Bülow, Hans von 8.\,1.\,1830 Dresden – 12.\,2.\,1894 Kairo@\textsc{Bülow, Hans von} (8.\,1.\,1830 Dresden – 12.\,2.\,1894 Kairo), \emph{Dirigent}|pw}}) Briefe\pwindex{Bülow, Hans von 8.\,1.\,1830 Dresden – 12.\,2.\,1894 Kairo@\textsc{Bülow, Hans von} (8.\,1.\,1830 Dresden – 12.\,2.\,1894 Kairo), \emph{Dirigent}!Briefe und Schriften@\strich\emph{Briefe und Schriften}|pw}, jetzt den letzten, 5. Band. Die
                  \textsc{Mann\pwindex{Mann, Heinrich 27.\,3.\,1871 Lübeck – 11.\,3.\,1950 Santa Monica@\textsc{Mann, Heinrich} (27.\,3.\,1871 Lübeck – 11.\,3.\,1950 Santa Monica), \emph{Schriftsteller}|pw}}ſchen Zwei Racen\pwindex{Mann, Heinrich 27.\,3.\,1871 Lübeck – 11.\,3.\,1950 Santa Monica@\textsc{Mann, Heinrich} (27.\,3.\,1871 Lübeck – 11.\,3.\,1950 Santa Monica), \emph{Schriftsteller}!Zwischen den Rassen@\strich\emph{Zwischen den Rassen}|pwv} mit
               Bewunderung und mit \strikeout{allerlei} leiſem Widerſtand gegen
               allerlei menſchliches in \textsc{Heinrichs\pwindex{Mann, Heinrich 27.\,3.\,1871 Lübeck – 11.\,3.\,1950 Santa Monica@\textsc{Mann, Heinrich} (27.\,3.\,1871 Lübeck – 11.\,3.\,1950 Santa Monica), \emph{Schriftsteller}|pw}} Seele\pend
           
\pstart
           {\pb}Es wäre lieb von Ihnen, we{\geminationn} Sie nächſtens etwas mehr von{ }ſich vernehmen ließen;
               insbeſonders wünſcht’ ich zu wiſſen, welchen Ihrer Stoffe{ }ſie jetzt am{ }ſtärkſten
               bewegt und welchen Sie »zunächſt« (ein{ }ſcheußliches Berlin\oindex{Berlin@\textbf{Berlin}, \emph{Hauptstadt}|pw}er Wort) in Bewegung zu{ }ſetzen gedenken. Da{\geminationn} Ihr Befinden, kurz u gut, was Sie mir \introOben{}zu\introOben{}{ }ſagen haben\textcolor{gray}{.} Schöner wärs natürlich,
                  we{\geminationn}{ }{\pb}man an irgd einem Ufer gemeinſam wandelte,
               wo{ }ſich »denn« u. ſ. w.\pend
           
\pstart
           Wir\pwindex{Schnitzler, Olga 17.\,1.\,1882 Wien – 13.\,1.\,1970 Lugano@\textsc{Schnitzler, Olga} (17.\,1.\,1882 Wien – 13.\,1.\,1970 Lugano), \emph{Schauspielerin, Sängerin}|pwv} grüßen Sie vielmals
               {\\[\baselineskip]}Von Herzen {\\[\baselineskip]}Ihr {\\[\baselineskip]}\spacefill\mbox{Arthur}\pend
           \leftskip=0em{}\selectlanguage{ngerman}\endnumbering\briefempfaengerindex{Salten, Felix@\textsc{Salten, Felix}!zzzSchnitzler, Arthur@\emph{von Arthur Schnitzler}!1907-08-051@{5. 8. 1907}|)be}\mylabel{L03009h}  \newcommand{\dateiname}{L03009}\newcommand{\titel}{Arthur Schnitzler an Felix Salten, 5. 8. 1907}\newcommand{\editorInnen}{Martin Anton Müller und Laura Untner}%% latex-leseansicht-abspann.tex
%% Abspann für die Leseansicht.
%% Der Schalter \ifkorrekturansicht ist bereits durch den Vorspann gesetzt.

%% latex-abspann.tex
%% Gemeinsamer Abspann für Korrekturansicht und Leseansicht.
%% Setzt den Schalter \ifkorrekturansicht voraus (gesetzt in den
%% einbindenden Dateien latex-korrekturansicht-abspann.tex bzw.
%% latex-leseansicht-abspann.tex).
%% ---------------------------------------------------------------

\normalsize

% Das esempio-Environment wird nur in der Leseansicht benötigt
\ifkorrekturansicht\else
\newenvironment{esempio}[3]%
{
    \vspace{1.5ex}
    \rlap{\underline{#1}}
    \par
    \setlength{\parindent}{0cm}
    \nopagebreak
    \leftskip=#2cm
    \rightskip=#3cm
}
{
    \par
}
\fi

\doendnotes{C}
\bigskip
\vfill

\clearpage

\footnotesize

\ifkorrekturansicht
  \lohead{\textsc{register}}
\fi

% theindex-Environment neu definieren ohne reledmac
\makeatletter
\renewenvironment{theindex}{%
  \ifkorrekturansicht
    \section*{\indexname}%
  \else
    \subsubsection*{Index der erwähnten Entitäten}%
  \fi
  \setlength{\parindent}{0pt}%
  \setlength{\parskip}{0pt plus 0.3pt}%
  \let\item\@idxitem
}{%
  \ifkorrekturansicht\clearpage\fi
}
\makeatother

\IfFileExists{\jobname-pw.ind}{\input{\jobname-pw.ind}}{}

% Quellenangabe nur in der Leseansicht
\ifkorrekturansicht\else
% Fallback-Definitionen, falls die .tex-Datei \titel etc. nicht gesetzt hat
\providecommand{\titel}{}
\providecommand{\editorInnen}{}
\providecommand{\dateiname}{\jobname}

\vspace{3cm}

\vfill

\footnotesize
\textsc{Quelle}: \titel. Herausgegeben von {\editorInnen}. In: \emph{Arthur Schnitzler: Briefwechsel mit Autorinnen und Autoren}.
 Digitale Edition, https://schnitzler-briefe.acdh.oeaw.ac.at/{\dateiname}.html (Stand \today)
\fi

\end{document}


