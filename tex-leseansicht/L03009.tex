%% latex-korrekturansicht-vorspann.tex
%% Vorspann für die Korrekturansicht.
%% Lädt die gemeinsame Datei latex-vorspann.tex mit gesetztem Schalter.

\newif\ifkorrekturansicht
\korrekturansichttrue

\input{../tex-inputs/latex-vorspann}


\section[ Arthur Schnitzler an Felix Salten, 5. 8. 1907]{L03009 Arthur Schnitzler an Felix Salten, 5. 8. 1907}
\nopagebreak\mylabel{L03009v}
\rehead{ }\normalsize\beginnumbering\briefempfaengerindex{Salten, Felix@\textsc{Salten, Felix}!zzzSchnitzler, Arthur@\emph{von Arthur Schnitzler}!1907-08-051@{5. 8. 1907}|(be}
\toendnotes[C]{\smallbreak\pagebreak[2]}\Standort{Wienbibliothek im Rathaus, ZPH 1681, 2.1.516.}
\physDesc{Brief, 3 Blätter, 6 Seiten, 2900 Zeichen
\newline{}Handschrift: Bleistift, deutsche Kurrent (\noindent{}Text und Nummerierung der Blätter: »1«–»3«)
\newline{}Ordnung: mit Bleistift von unbekannter Hand Nummerierung der Doppelseiten des
                                 Konvoluts: »8«–»10« }
\buchAbdrucke{\weitereDrucke{1) Arthur Schnitzler: \emph{Briefe 1875–1912}. Frankfurt am Main: \emph{S. Fischer} 1981, S. 560–561.} \weitereDrucke{2) Hermann Bahr, Arthur Schnitzler: \emph{Briefwechsel, Aufzeichnungen, Dokumente (1891–1931)}. Göttingen: \emph{Wallstein} 2018, S. 395.} }\toendnotes[C]{\smallbreak}
\pstart
           {\pb}\textcolor{gray}{\textbf{Telegramm-Adresse: \textbf{Böhm\pwindex{Boehm, Josef @\textsc{Böhm, Josef}, \emph{Hotelbesitzer/Hotelbesitzerin}|pw} – Welsberg\oindex{Welsberg-Taisten@\textbf{Welsberg-Taisten}, \emph{A.ADM3}|pw}.}}}\pend
           
\pstart
           \textcolor{gray}{\textbf{\textbf{Hôtel {\kaufmannsund} Pension Wildbad Waldbrunn\oindex{Wildbad Waldbrunn@\textbf{Wildbad Waldbrunn}, \emph{S.SPA}|pw}}}}\pend
           
\pstart
           \textcolor{gray}{\textbf{bei \textbf{Welsberg\oindex{Welsberg-Taisten@\textbf{Welsberg-Taisten}, \emph{A.ADM3}|pw}} (Eilzughaltestelle) }}\pend
           
\pstart
           \textcolor{gray}{\textbf{1150 M. \textsuperscript{ü}/Meer. \hspace*{1.5em}Hochpusterthal\oindex{Pustertal@\textbf{Pustertal}, \emph{T.VAL}|pw} (Tirol\oindex{Tirol@\textbf{Tirol}, \emph{A.ADM1}|pw})}}\pend
           
\pstart
           \textcolor{gray}{\textbf{Heilkräftiges altbekanntes Bad in prachtvoller Lage.}}\pend
           
\pstart
           \textcolor{gray}{\textbf{\textbf{Ausgezeichnete Trinkquelle}.}}\pend
           
\pstart
           \textcolor{gray}{\textbf{70 mit allem Comfort eingerichtete Zimmer.}}\pend
           
\pstart
           \raggedleft{}\textcolor{gray}{\textbf{\emph{Waldbrunn\oindex{Welsberg-Taisten@\textbf{Welsberg-Taisten}, \emph{A.ADM3}|pw}, den}}}{ }5. 8. \textcolor{gray}{\textbf{\emph{190}}}7\pend
           \vspace{0.5em}
\pstart
           lieber, ich danke Ihnen für Ihre Nachrichten, laſſen Sie uns jetzt
               nur bald hören, dſs Ihre Frau\pwindex{Salten, Ottilie 07.03.1868 – 22.06.1942@\textsc{Salten, Ottilie} (07.03.1868 – 22.06.1942), \emph{Schauspieler/Schauspielerin}|pwv} ſich vollko{\geminationm}en erholt hat. Dem Buben\pwindex{Salten, Paul 11.08.1903 – 08.05.1937@\textsc{Salten, Paul} (11.08.1903 – 08.05.1937), \emph{Filmcutter/Filmcutterin}|pwv} geht’s wohl ſchon wieder
               ganz gut? Wir ſind nun einen vollen Monat da\oindex{Wildbad Waldbrunn@\textbf{Wildbad Waldbrunn}, \emph{S.SPA}|pwv} und werden wahrſcheinlich \label{K_L03009-1v}\edtext{bis nach dem 20. bleiben}{\lemma{\textnormal{\emph{bis nach dem 20. bleiben}}}\Cendnote{\textnormal{Sie blieben bis zum 26. 8. 1907.}}}\label{K_L03009-1}.
                  \label{K_L03009-2v}\edtext{Heute ko{\geminationm}t meine Mama\pwindex{Schnitzler, Louise 1840-07-08 – 1911-09-09@\textsc{Schnitzler, Louise} (1840-07-08 – 1911-09-09)|pwv} an, vielleicht ni{\geminationm}t ſie Heini\pwindex{Schnitzler, Heinrich 09.08.1902 – 12.07.1982@\textsc{Schnitzler, Heinrich} (09.08.1902 – 12.07.1982), \emph{Regisseur/Regisseurin, Schauspieler/Schauspielerin}|pw} mit
               nach Wien\oindex{Wien@\textbf{Wien}, \emph{A.ADM2}|pw}}{\lemma{\textnormal{\emph{Heute … Wien}}}\Cendnote{\textnormal{Louise Schnitzler\pwindex{Schnitzler, Louise 1840-07-08 – 1911-09-09@\textsc{Schnitzler, Louise} (1840-07-08 – 1911-09-09)|pwk} war zwischen 5. 8. 1907 und 24. 8. 1907 in Welsberg\oindex{Welsberg-Taisten@\textbf{Welsberg-Taisten}, \emph{A.ADM3}|pwk}. Heinrich Schnitzler\pwindex{Schnitzler, Heinrich 09.08.1902 – 12.07.1982@\textsc{Schnitzler, Heinrich} (09.08.1902 – 12.07.1982), \emph{Regisseur/Regisseurin, Schauspieler/Schauspielerin}|pwk} reiste erst am 26. 8. 1907 ab.}}}\label{K_L03009-2}; da{\geminationn} wollen wir, Olga\pwindex{Schnitzler, Olga 17.01.1882 – 13.01.1970@\textsc{Schnitzler, Olga} (17.01.1882 – 13.01.1970), \emph{Schauspieler/Schauspielerin, Sänger/Sängerin}|pw}
               u ich{[},{]} noch \label{K_L03009-3v}\edtext{ſüdlicher}{\lemma{\textnormal{\emph{ſüdlicher}}}\Cendnote{\textnormal{Siehe Felix und Ottilie Salten an Arthur Schnitzler, 3. 8. 1907. }}}\label{K_L03009-3},
               vielleicht, u theilweiſe zu Fuſs, über die neue Dolomitenstraße\oindex{Grosse Dolomitenstrasse@\textbf{Große Dolomitenstraße}, \emph{Straße (K.STR)}|pw}; nach Bozen\oindex{Bozen@\textbf{Bozen}, \emph{P.PPLA2}|pw}. In Meran\oindex{Meran@\textbf{Meran}, \emph{P.PPLA3}|pw} oder am Gardaſee\oindex{Lago di Garda@\textbf{Lago di Garda}, \emph{See (N.SEE)}|pw} denken wir eine Woche zu raſten und da{\geminationn}, in den erſten Septembertagen, in Wien\oindex{Wien@\textbf{Wien}, \emph{A.ADM2}|pw} einzutreffen. Möglich, daſs wir irgendwo \label{K_L03009-4v}\edtext{mit {\pb}Richard\pwindex{Beer-Hofmann, Richard 1866-07-11 – 1945-09-26@\textsc{Beer-Hofmann, Richard} (1866-07-11 – 1945-09-26), \emph{Schriftsteller/Schriftstellerin}|pw} u Paula\pwindex{Beer-Hofmann, Paula 25.02.1879 – 30.10.1939@\textsc{Beer-Hofmann, Paula} (25.02.1879 – 30.10.1939)|pw} zuſa{\geminationm}entreffen}{\lemma{\textnormal{\emph{mit … zuſammentreffen}}}\Cendnote{\textnormal{Dazu kam es nicht, vgl. Richard Beer-Hofmann an Arthur Schnitzler, 29. 8. 1907 und Arthur Schnitzler an Richard Beer-Hofmann, 9. 9. 1907. }}}\label{K_L03009-4}. Sie wollen im September
               eine Meerfahrt unternehmen? Thäts der Gardaſee\oindex{Lago di Garda@\textbf{Lago di Garda}, \emph{See (N.SEE)}|pw}
               nicht auch? Mein Rad hab ich nicht mit, bedaure es auch nicht ſehr, da meine Zeit
               reichlich ausgefüllt iſt. Vormittg Waldwanderungen, allein, oder mit Olga\pwindex{Schnitzler, Olga 17.01.1882 – 13.01.1970@\textsc{Schnitzler, Olga} (17.01.1882 – 13.01.1970), \emph{Schauspieler/Schauspielerin, Sänger/Sängerin}|pw}; Nachmittg 2–6 etwa arbeit
               ich; da{\geminationn} ſpaziren; da{\geminationn}
               Nachtmahl und \label{K_L03009-5v}\edtext{Platformwandelei}{\lemma{\textnormal{\emph{Platformwandelei}}}\Cendnote{\textnormal{Die Schreibweise deutet auf eine
                  englischsprachige Aussprache durch Schnitzler hin.}}}\label{K_L03009-5}. Tennis haben wir erſt einmal geſpielt – der Platz
               lächerlich; unſre Partnerin ware eine ſehr charmante junge Frau \textsc{Epstein\pwindex{Epstein, Marie 06.04.1880 – 03.01.1953@\textsc{Epstein, Marie} (06.04.1880 – 03.01.1953)|pw}} (geboren \textsc{Miss Hudetz\pwindex{Epstein, Marie 06.04.1880 – 03.01.1953@\textsc{Epstein, Marie} (06.04.1880 – 03.01.1953)|pw}}), Schwägerin der \textsc{Anna – Epstein Loeb\pwindex{Epstein, Anna 6.3.1877 – 16.3.1943@\textsc{Epstein, Anna} (6.3.1877 – 16.3.1943)|pw}}. Ferner befinden ſich hier die Schweſtern\pwindex{Guttmann, Leonie @\textsc{Guttmann, Leonie}, \emph{Übersetzer/Übersetzerin}|pwv}\pwindex{Frankfurter, Ella 02.02.1873 – 05.10.1957@\textsc{Frankfurter, Ella} (02.02.1873 – 05.10.1957), \emph{Maler/Malerin}|pwv} der Frau \textsc{Auernheimer\pwindex{Auernheimer, Irene 6.3.1880 – 1967-10-30@\textsc{Auernheimer, Irene} (6.3.1880 – 1967-10-30)|pw}}, und allerlei \label{K_L03009-6v}\edtext{\textsc{Ascenden\textcolor{gray}{z}} u \textsc{Descendenz}}{\lemma{\textnormal{\emph{Ascendenz u Descendenz}}}\Cendnote{\textnormal{Auf- und Absteigendes}}}\label{K_L03009-6}; zum Theil
               gutes u. vorzügliches Menſchenmaterial. Der Mann der verheirateten Schweſter\pwindex{Frankfurter, Ella 02.02.1873 – 05.10.1957@\textsc{Frankfurter, Ella} (02.02.1873 – 05.10.1957), \emph{Maler/Malerin}|pwv}, Frankfurter\pwindex{Frankfurter, Albert 27.03.1868 – 29.03.1952@\textsc{Frankfurter, Albert} (27.03.1868 – 29.03.1952), \emph{Generaldirektor/Generaldirektorin}|pw} mit Namen, Direktor {\pb}des oeſterr.
                  Lloyd\orgindex{Oesterreichischer Lloyd@Österreichischer Lloyd|pw}, ſcheint was nicht gewöhnliches zu ſein. – Daſs Bahr\pwindex{Bahr, Hermann 19.07.1863 – 15.01.1934@\textsc{Bahr, Hermann} (19.07.1863 – 15.01.1934), \emph{Schriftsteller/Schriftstellerin, Kritiker/Kritikerin}|pw} Sie gegen Pötzl\pwindex{Poetzl, Eduard 17.03.1851 – 20.08.1914@\textsc{Pötzl, Eduard} (17.03.1851 – 20.08.1914), \emph{Schriftsteller/Schriftstellerin, Journalist/Journalistin}|pw} –
               wie ſoll man da ſagen – in Schmutz nehmen? – mußte, hat uns ſehr amusirt. We{\geminationn} ich ſowohl Ihren Morgen\pwindex{Morgen. Wochenschrift fuer deutsche Kultur@\emph{Morgen. Wochenschrift für deutsche Kultur}|pw}ruf\pwindex{Wiener Korrespondent@\emph{Der Wiener Korrespondent}|pwv} als Pötzl\pwindex{Poetzl, Eduard 17.03.1851 – 20.08.1914@\textsc{Pötzl, Eduard} (17.03.1851 – 20.08.1914), \emph{Schriftsteller/Schriftstellerin, Journalist/Journalistin}|pw}’s Lobeshymne\pwindex{gelobte Wien@\emph{Das gelobte Wien}|pwv} zu leſen beko{\geminationm}en könnte, wär ich
               Ihnen herzlich verbunden. (Daſs Sie mir die berühmte \label{K_L03009-7v}\edtext{Sa{\geminationm}lung der 12 Berl\oindex{Berlin@\textbf{Berlin}, \emph{P.PPLC}|pw}. Feu{[}i{]}lletons}{\lemma{\textnormal{\emph{Sammlung … Feuilletons}}}\Cendnote{\textnormal{Es dürfte sich um Saltens\pwindex{Salten, Felix 06.09.1869 – 08.10.1945@\textsc{Salten, Felix} (06.09.1869 – 08.10.1945), \emph{Schriftsteller/Schriftstellerin, Journalist/Journalistin, Chefredakteur/Chefredakteurin}|pwk} Beiträge für die \emph{B. Z. am
                     Mittag}\pwindex{B.Z. am Mittag@\emph{B.Z. am Mittag}|pwk} handeln. Abgesehen von einer Ausnahme fehlen diese vollständig in
                     Saltens\pwindex{Salten, Felix 06.09.1869 – 08.10.1945@\textsc{Salten, Felix} (06.09.1869 – 08.10.1945), \emph{Schriftsteller/Schriftstellerin, Journalist/Journalistin, Chefredakteur/Chefredakteurin}|pwk} Zusammenstellungen seiner
                  journalistischen Arbeiten in seinem Nachlass. Das kann als Indiz genommen
                  werden, dass Salten\pwindex{Salten, Felix 06.09.1869 – 08.10.1945@\textsc{Salten, Felix} (06.09.1869 – 08.10.1945), \emph{Schriftsteller/Schriftstellerin, Journalist/Journalistin, Chefredakteur/Chefredakteurin}|pwk} mit den Texten eine
                  Publikation plante oder sie zumindest als zusammengehörig betrachtete. Saltens\pwindex{Salten, Felix 06.09.1869 – 08.10.1945@\textsc{Salten, Felix} (06.09.1869 – 08.10.1945), \emph{Schriftsteller/Schriftstellerin, Journalist/Journalistin, Chefredakteur/Chefredakteurin}|pwk} Brief vom 15. 8. 1907 lässt zudem
                  vermuten, dass es sich um Beiträge zu seiner England\oindex{England@\textbf{England}, \emph{A.ADM1}|pwk}-Reise im Juni 1906 handelte, vgl. Felix Salten an Arthur Schnitzler, 19. 6. 1906.}}}\label{K_L03009-7} noch immer
               nicht gegeben haben, nur nebenbei.) Wie ſtehts im übrigen mit Ihren Arbeiten? In
               welcher ſtecken Sie am liebſten? – Ich ſchreibe hier nur an dem Roman\pwindex{Weg ins Freie. Roman@\emph{Der Weg ins Freie. Roman}|pwv}; letzte, zum Theil wohl {\pb}vorletzte Feile; habe ein wunderſchönes
               Zimmer, in das vom Hoteltrubel nichts dringt, mit einem guten Blick über Wieſen und
               Wald ins Thal; vorgebauter Balkon; oberſter Stock. – (Das idealſte Arbeitszimmer –
               ohne dieſes, glaub ich, hielt es mich doch nicht ſo lang hier). An \label{K_L03009-8v}\edtext{Lienz\oindex{Lienz@\textbf{Lienz}, \emph{P.PPLA3}|pw} vorüberfahrend und an \textsc{Dölsach\oindex{Doelsach@\textbf{Dölsach}, \emph{A.ADM3}|pw}}}{\lemma{\textnormal{\emph{Lienz … Dölsach}}}\Cendnote{\textnormal{Vgl. Felix Salten an Arthur Schnitzler, 14. 8. 1893. }}}\label{K_L03009-8} (ſo heißts
               doch) blieb ich nicht ungerührt – – \label{K_L03009-9v}\edtext{»wie war ich jung\pwindex{Ruf des Lebens. Schauspiel in drei Akten@\emph{Der Ruf des Lebens. Schauspiel in drei Akten}|pwv}« heißt
               es in der ſchönſten Scene\pwindex{Ruf des Lebens. Schauspiel in drei Akten@\emph{Der Ruf des Lebens. Schauspiel in drei Akten}|pw}}{\lemma{\textnormal{\emph{»wie … Scene}}}\Cendnote{\textnormal{\emph{Der Ruf des Lebens}\pwindex{Ruf des Lebens. Schauspiel in drei Akten@\emph{Der Ruf des Lebens. Schauspiel in drei Akten}|pwk}, 1. Akt, 7. Szene}}}\label{K_L03009-9}
               die ich je geſchrieben habe (aber es ſtehen auch originellere Sachen drin.) – Leſe
               hauptſächlich \textsc{Bülow (Hans v.\pwindex{Buelow, Hans von 08.01.1830 – 12.02.1894@\textsc{Bülow, Hans von} (08.01.1830 – 12.02.1894), \emph{Dirigent/Dirigentin}|pw}}) Briefe\pwindex{Briefe und Schriften@\emph{Briefe und Schriften}|pw}, jetzt den letzten, 5. Band. Die
                  \textsc{Mann\pwindex{Mann, Heinrich 27.03.1871 – 11.03.1950@\textsc{Mann, Heinrich} (27.03.1871 – 11.03.1950), \emph{Schriftsteller/Schriftstellerin}|pw}}ſchen Zwei Racen\pwindex{Zwischen den Rassen@\emph{Zwischen den Rassen}|pwv} mit
               Bewunderung und mit \strikeout{allerlei} leiſem Widerſtand gegen
               allerlei menſchliches in \textsc{Heinrichs\pwindex{Mann, Heinrich 27.03.1871 – 11.03.1950@\textsc{Mann, Heinrich} (27.03.1871 – 11.03.1950), \emph{Schriftsteller/Schriftstellerin}|pw}} Seele\pend
           
\pstart
           {\pb}Es wäre lieb von Ihnen, we{\geminationn} Sie nächſtens etwas mehr von ſich vernehmen ließen;
               insbeſonders wünſcht’ ich zu wiſſen, welchen Ihrer Stoffe ſie jetzt am ſtärkſten
               bewegt und welchen Sie »zunächſt« (ein ſcheußliches Berlin\oindex{Berlin@\textbf{Berlin}, \emph{P.PPLC}|pw}er Wort) in Bewegung zu ſetzen gedenken. Da{\geminationn} Ihr Befinden, kurz u gut, was Sie mir \introOben{}zu\introOben{} ſagen haben\textcolor{gray}{.} Schöner wärs natürlich,
                  we{\geminationn}{ }{\pb}man an irgd einem Ufer gemeinſam wandelte,
               wo ſich »denn« u. ſ. w.\pend
           
\pstart
           Wir\pwindex{Schnitzler, Olga 17.01.1882 – 13.01.1970@\textsc{Schnitzler, Olga} (17.01.1882 – 13.01.1970), \emph{Schauspieler/Schauspielerin, Sänger/Sängerin}|pwv} grüßen Sie vielmals
               {\\[\baselineskip]}Von Herzen {\\[\baselineskip]}Ihr {\\[\baselineskip]}\spacefill\mbox{Arthur}\pend
           \leftskip=0em{}\selectlanguage{ngerman}\endnumbering\briefempfaengerindex{Salten, Felix@\textsc{Salten, Felix}!zzzSchnitzler, Arthur@\emph{von Arthur Schnitzler}!1907-08-051@{5. 8. 1907}|)be}\mylabel{L03009h}  \normalsize

\doendnotes{C}
\bigskip
\vfill

\clearpage

\footnotesize

\lohead{\textsc{register}}

% Definiere theindex-Environment komplett neu ohne reledmac
\makeatletter
\renewenvironment{theindex}{%
  \section*{\indexname}%
  \setlength{\parindent}{0pt}%
  \setlength{\parskip}{0pt plus 0.3pt}%
  \let\item\@idxitem
}{%
  \clearpage
}
\makeatother

\IfFileExists{\jobname-pw.ind}{\input{\jobname-pw.ind}}{}

\end{document}

      