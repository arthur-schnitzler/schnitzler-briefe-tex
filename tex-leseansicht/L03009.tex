%% latex-leseansicht-vorspann.tex
%% Vorspann für die Leseansicht.
%% Lädt die gemeinsame Datei latex-vorspann.tex mit nicht gesetztem Schalter.

\newif\ifkorrekturansicht
\korrekturansichtfalse

\input{../tex-inputs/latex-vorspann}

\begin{center}
            \textcolor{red}{ENTWURF, NICHT FERTIG KORRIGIERT}
                      \end{center}
            
         
         \newcommand{\erwaehntePersonen}{Personen: Irene Auernheimer, Hermann Bahr, Richard Beer-Hofmann, Hans von Bülow, Marie Epstein, Anna Epstein, Ella Frankfurter, Albert Frankfurter, Leonie Guttmann, Heinrich Mann, Eduard Pötzl, Felix Salten, Heinrich Schnitzler, Olga Schnitzler}
         \newcommand{\erwaehnteInstitutionen}{Institutionen: Österreichischer Lloyd}
         \newcommand{\erwaehnteOrte}{Orte: Berlin, Bozen, Dölsach, Große Dolomitenstraße, Lago di Garda, Lienz, Meran, Wien}
         \newcommand{\erwaehnteWerke}{Werke: Briefe und Schriften, Der Weg ins Freie. Roman, Zwischen den Rassen}
               \section[Arthur Schnitzler an Felix Salten, 5. 8. 1907]{ Arthur Schnitzler an Felix Salten, 5. 8. 1907}\nopagebreak\mylabel{v}\rehead{ }\begin{ledgroupsized}[t]{13cm}\normalsize\beginnumbering \toendnotes[C]{\smallbreak\pagebreak[2]} \Standort{Wienbibliothek im Rathaus, ZPH 1681, 2.1.516.}
\physDesc{
\newline{}Handschrift: , deutsche Kurrent}\buchAbdrucke{\weitereDrucke{Hermann Bahr, Arthur Schnitzler: \emph{
                        Briefwechsel, Aufzeichnungen, Dokumente (1891–1931)
                     }. Hg. Kurt Ifkovits und Martin Anton Müller. Göttingen: \emph{Wallstein} 2018, S. 395.} }\toendnotes[C]{\smallbreak}\pstart
           \noindent{}{\pb}\textcolor{gray}{\textbf{Telegramm-Adresse: Böhm – Welsberg}}\pend
           \pstart
           \textcolor{gray}{\textbf{
                     Hôtel
                     {\kaufmannsund}
                     Pension
                     Wildbad Waldbrunn\oindex{XXXX Ortsangabe fehlt|pw}}}\pend
           \pstart
           \textcolor{gray}{\textbf{
                     bei
                     Welsberg\oindex{XXXX Ortsangabe fehlt|pw}
                     (Eilzughaltestelle)
                  }}\pend
           \pstart
           \textcolor{gray}{\textbf{
                     1150 M.
                     \textsuperscript{ü}/Meer.
                     \hspace*{1.5em}Hochpusterthal\oindex{XXXX Ortsangabe fehlt|pw}
                     (Tirol\oindex{XXXX Ortsangabe fehlt|pw})
                  }}\pend
           \pstart
           \textcolor{gray}{\textbf{
                     Heilkräftiges altbekanntes Bad in prachtvoller Lage.
                  }}\pend
           \pstart
           \textcolor{gray}{\textbf{Ausgezeichnete Trinkquelle.}}\pend
           \pstart
           \textcolor{gray}{\textbf{70 mit allem Comfort eingerichtete Zimmer.}}\pend
           \pstart
           \raggedleft{}\textcolor{gray}{\textbf{Waldbrunn\oindex{XXXX Ortsangabe fehlt|pw}, den
                  }}{ }5. 8. \textcolor{gray}{\textbf{190}}7\pend
           \pstart
           lieber, ich danke Ihnen für Ihre Nachrichten, laſſen Sie uns jetzt
               nur bald hören, dſs Ihre Frau\textcolor{red}{\textsuperscript{\textbf{KEY}}} ſich
                  vollko{\geminationm}en erholt hat. \textcolor{gray}{Dem}Buben\pwindex{Schnitzler, Heinrich 09.08.1902 – 12.07.1982@\textsc{Schnitzler, Heinrich} (09.08.1902 – 12.07.1982), \emph{Regisseur, Schauspieler}|pwv} geht’s wohl ſchon
               wieder ganz gut? Wir ſind nun einen vollen Monat da und werden wahrſcheinlich bis
               nach dem 20. bleiben. Heute ko{\geminationm}t meine Mama\textcolor{red}{\textsuperscript{\textbf{KEY}}} an, vielleicht ni{\geminationm}t ſie Heini\textcolor{red}{\textsuperscript{\textbf{KEY}}} mit nach Wien\oindex{Wien@\textbf{Wien}|pw}; da{\geminationn} wollen wir,
                  Olga\pwindex{Schnitzler, Olga 17.01.1882 – 13.01.1970@\textsc{Schnitzler, Olga} (17.01.1882 – 13.01.1970), \emph{Schauspielerin, Sängerin}|pw} u ich noch ſüdlicher, vielleicht, u
               theilweiſe zu Fuſs, über die neue Dolomitenstraße\oindex{Grosse Dolomitenstrasse@\textbf{Große Dolomitenstraße}|pw}; nach Bozen\oindex{Bozen@\textbf{Bozen}|pw}. In Meran\oindex{Meran@\textbf{Meran}|pw} oder am Gardaſee\oindex{Lago di Garda@\textbf{Lago di Garda}|pw} denken wir eine Woche zu raſten und da{\geminationn}, in den erſten Septembertagen, in Wien\oindex{Wien@\textbf{Wien}|pw}
               einzutreffen. Möglich, daſs wir irgendwo mit {\pb}Richard\pwindex{Beer-Hofmann, Richard 1866-07-11 – 1945-09-26@\textsc{Beer-Hofmann, Richard} (1866-07-11 – 1945-09-26), \emph{Schriftsteller}|pw} u Paula\textcolor{red}{\textsuperscript{\textbf{KEY}}}
                  zuſa{\geminationm}entreffen. Sie wollen im September eine
               Meerfahrt unternehmen? Thäts der Gardaſee\oindex{Lago di Garda@\textbf{Lago di Garda}|pw} nicht
               auch? Mein Rad hab ich nicht mit, bedaure es auch nicht ſehr, da meine Zeit reichlich
               ausgefüllt iſt. Vormittag Waldwanderungen, allein, oder mit Olga\pwindex{Schnitzler, Olga 17.01.1882 – 13.01.1970@\textsc{Schnitzler, Olga} (17.01.1882 – 13.01.1970), \emph{Schauspielerin, Sängerin}|pw}; Nachmittg 2–6 etwa arbeit ich; dan ſpaziren; da{\geminationn} Nachtmahl und Platformwandelei. Tennis haben wir erſt
               einmal geſpielt – der Platz lächerlich; unſre Partnerin ware eine ſehr charmante
               junge Frau \textsc{Epstein\pwindex{Epstein, Marie 06.04.1880 – 03.01.1953@\textsc{Epstein, Marie} (06.04.1880 – 03.01.1953)|pw}} (geboren \textsc{Miss Hudetz\pwindex{Epstein, Marie 06.04.1880 – 03.01.1953@\textsc{Epstein, Marie} (06.04.1880 – 03.01.1953)|pw}}), Schwägerin der \textsc{Anna – Epstein Loeb\pwindex{Epstein, Anna 6.3.1877 – 16.3.1943@\textsc{Epstein, Anna} (6.3.1877 – 16.3.1943)|pw}}. Ferner befinden ſich hier die Schweſtern\pwindex{Guttmann, Leonie @\textsc{Guttmann, Leonie}, \emph{Übersetzerin}|pwv}\pwindex{Frankfurter, Ella 02.02.1873 – 05.10.1957@\textsc{Frankfurter, Ella} (02.02.1873 – 05.10.1957), \emph{Bildende Künstlerin >> Maler}|pwv} der Frau \textsc{Auernheimer\pwindex{Auernheimer, Irene 6.3.1880 – 1967-10-30@\textsc{Auernheimer, Irene} (6.3.1880 – 1967-10-30)|pw}}, und allerelei \textsc{Ascendte} und \textsc{Descendenz}; zum Theil gutes u. vorzügliches Menſchenmaterial. Der Mann der
               verheirateten Schweſter\pwindex{Frankfurter, Ella 02.02.1873 – 05.10.1957@\textsc{Frankfurter, Ella} (02.02.1873 – 05.10.1957), \emph{Bildende Künstlerin >> Maler}|pwv}, Frankfurter\pwindex{Frankfurter, Albert 27.03.1868 – 29.03.1952@\textsc{Frankfurter, Albert} (27.03.1868 – 29.03.1952), \emph{Generaldirektor}|pw} mit Namen, Direktor {\pb}des oeſterr. Lloyd\orgindex{Oesterreichischer Lloyd@Österreichischer Lloyd|pw}, ſcheint was nicht gewöhnliches zu ſein.– Daſs Bahr\pwindex{Bahr, Hermann 19.07.1863 – 15.01.1934@\textsc{Bahr, Hermann} (19.07.1863 – 15.01.1934), \emph{Schriftsteller, Kritiker}|pw} Sie gegen Pötzl\textcolor{red}{\textsuperscript{\textbf{KEY}}} – wie ſoll man ſagen – in Schmutz nehmen? – mußte, hat uns ſehr
               amusirt. We{\geminationn} ich ſowohl Ihren Morgenruf\textcolor{red}{\textsuperscript{\textbf{KEY}}} als Pötzl\pwindex{Poetzl, Eduard 17.03.1851 – 20.08.1914@\textsc{Pötzl, Eduard} (17.03.1851 – 20.08.1914), \emph{Schriftsteller, Journalist}|pw}’s Lobeshymne\textcolor{red}{\textsuperscript{\textbf{KEY}}} zu leſen beko{\geminationm}en könnte, wär ich Ihnen herzlich verbunden. (Daſs Sie mir die berühmte Sa{\geminationm}lung der 12
                     Berl\oindex{Berlin@\textbf{Berlin}|pw}. Feu{[}i{]}lletons\textcolor{red}{\textsuperscript{\textbf{KEY}}}
               noch immer nicht gegeben habem nur nebenbei.) Wie ſtehts im übrigen mit Ihren
               Arbeiten? In welcher ſtecken Sie am liebſten?– Ich ſchreibe hier nur an dem Roman\pwindex{Schnitzler, Arthur 15.05.1862 – 21.10.1931@\textsc{Schnitzler, Arthur} (15.05.1862 – 21.10.1931), \emph{Schriftsteller, Mediziner}!Weg ins Freie. Roman1.1.1908 – 1.6.1908@\strich\emph{Der Weg ins Freie. Roman} {[}1.1.1908 – 1.6.1908{]}|pwv}; letzte, zum Theil wohl
               {\pb}vorletzte Feile; habe ein
               wunderſchönes Zimmer, in das vom Hoteltrubel nichts dringt, mit einem guten Blick
               über Wieſen und Wald ins Thal; vorgebauter Balkon; oberſter Stock.– (Das idealſte
               Arbeitszimmer – ohne dieſes, glaub ich, hielt es mich doch nicht ſo lang hier). An
                  Lienz\oindex{Lienz@\textbf{Lienz}|pw} vorüberfahrend und an \textsc{Dölsach\oindex{Doelsach@\textbf{Dölsach}|pw}} (ſo heißts doch) blieb ich nicht ungerührt – – »wie war ich jung« heißt es in
               der ſchönſten Scene die ich je geſchrieben habe (aber es ſtehen auch originellere
               Sachen drin.) – Leſe hauptſächlich \textsc{Bülow (Hans v.\pwindex{Buelow, Hans von 08.01.1830 – 12.02.1894@\textsc{Bülow, Hans von} (08.01.1830 – 12.02.1894), \emph{Dirigent}|pw}}) Briefe\pwindex{Buelow, Hans von 08.01.1830 – 12.02.1894@\textsc{Bülow, Hans von} (08.01.1830 – 12.02.1894), \emph{Dirigent}!Briefe und Schriften1895 – 1911@\strich\emph{Briefe und Schriften} {[}1895 – 1911{]}|pw}, jetzt den letzten, 5. Band. Die
                  \textsc{Mann\pwindex{Mann, Heinrich 27.03.1871 – 11.03.1950@\textsc{Mann, Heinrich} (27.03.1871 – 11.03.1950), \emph{Schriftsteller}|pw}}ſchen Zwei Racen\pwindex{Mann, Heinrich 27.03.1871 – 11.03.1950@\textsc{Mann, Heinrich} (27.03.1871 – 11.03.1950), \emph{Schriftsteller}!Zwischen den Rassen1907@\strich\emph{Zwischen den Rassen} {[}1907{]}|pwv} mit
               Bewunderung und mit \strikeout{allerlei} leiſem Widerſtand gegen
               allerlei menſchliches in\textsc{Heinrich\pwindex{Mann, Heinrich 27.03.1871 – 11.03.1950@\textsc{Mann, Heinrich} (27.03.1871 – 11.03.1950), \emph{Schriftsteller}|pw}s} Seele {\pb}Es wäre lieb von Ihnen, we{\geminationn} Sie nächſtens etwas mehr von ſich vernehmen ließen;
               ins beſonders wünſcht’ ich zu wiſſen, welchen Ihrer Stoffe ſie jetzt am ſtärkſten
               bewegt und welchen Sie »zunächſt« (ein ſcheußliches Berlin\oindex{Berlin@\textbf{Berlin}|pw}er Wort) in Bewegung zu ſetzen gedenken. Da{\geminationn} Ihr Befinden, kurz u gut, was Sie mir zu
               ſagen haben. Schöner wärs natürlich, we{\geminationn}{\pb}man an irgd einem Ufer gemeinſam
               wandelte, wo ſich »denn« u. ſ. w. \pend
           \pstart
           Wir grüßen Sie vielmals {\\[\baselineskip]}Von Herzen {\\[\baselineskip]}Ihr {\\[\baselineskip]}\spacefill\mbox{Arthur}\pend
           \leftskip=0em{}
         
         \endnumbering\mylabel{h}\end{ledgroupsized}\begin{anhang}\end{anhang}\newcommand{\dateiname}{L03009}\newcommand{\titel}{Arthur Schnitzler an Felix Salten, 5. 8. 1907}\newcommand{\editorInnen}{Martin Anton Müller und Laura Untner}%% latex-leseansicht-abspann.tex
%% Abspann für die Leseansicht.
%% Der Schalter \ifkorrekturansicht ist bereits durch den Vorspann gesetzt.

%% latex-abspann.tex
%% Gemeinsamer Abspann für Korrekturansicht und Leseansicht.
%% Setzt den Schalter \ifkorrekturansicht voraus (gesetzt in den
%% einbindenden Dateien latex-korrekturansicht-abspann.tex bzw.
%% latex-leseansicht-abspann.tex).
%% ---------------------------------------------------------------

\normalsize

% Das esempio-Environment wird nur in der Leseansicht benötigt
\ifkorrekturansicht\else
\newenvironment{esempio}[3]%
{
    \vspace{1.5ex}
    \rlap{\underline{#1}}
    \par
    \setlength{\parindent}{0cm}
    \nopagebreak
    \leftskip=#2cm
    \rightskip=#3cm
}
{
    \par
}
\fi

\doendnotes{C}
\bigskip
\vfill

\clearpage

\footnotesize

\ifkorrekturansicht
  \lohead{\textsc{register}}
\fi

% theindex-Environment neu definieren ohne reledmac
\makeatletter
\renewenvironment{theindex}{%
  \ifkorrekturansicht
    \section*{\indexname}%
  \else
    \subsubsection*{Index der erwähnten Entitäten}%
  \fi
  \setlength{\parindent}{0pt}%
  \setlength{\parskip}{0pt plus 0.3pt}%
  \let\item\@idxitem
}{%
  \ifkorrekturansicht\clearpage\fi
}
\makeatother

\IfFileExists{\jobname-pw.ind}{\input{\jobname-pw.ind}}{}

% Quellenangabe nur in der Leseansicht
\ifkorrekturansicht\else
% Fallback-Definitionen, falls die .tex-Datei \titel etc. nicht gesetzt hat
\providecommand{\titel}{}
\providecommand{\editorInnen}{}
\providecommand{\dateiname}{\jobname}

\vspace{3cm}

\vfill

\footnotesize
\textsc{Quelle}: \titel. Herausgegeben von {\editorInnen}. In: \emph{Arthur Schnitzler: Briefwechsel mit Autorinnen und Autoren}.
 Digitale Edition, https://schnitzler-briefe.acdh.oeaw.ac.at/{\dateiname}.html (Stand \today)
\fi

\end{document}


      