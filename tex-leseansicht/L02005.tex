%% latex-korrekturansicht-vorspann.tex
%% Vorspann für die Korrekturansicht.
%% Lädt die gemeinsame Datei latex-vorspann.tex mit gesetztem Schalter.

\newif\ifkorrekturansicht
\korrekturansichttrue

\input{../tex-inputs/latex-vorspann}


\section[Stefan Großmann an Arthur Schnitzler, {[}7.{]} 2. 1911]{L02005 Stefan Großmann an Arthur Schnitzler, {[}7.{]} 2. 1911}
\nopagebreak\mylabel{L02005v}
\rehead{ }\normalsize\beginnumbering\briefempfaengerindex{Schnitzler, Arthur@\textsc{Schnitzler, Arthur}!zzzGrossmann, Stefan@\emph{von Stefan Großmann}!1911-02-072@{{[}7.{]} 2. 1911}|(be}
\toendnotes[C]{\smallbreak\pagebreak[2]}\Standort{CUL, Schnitzler, B 34.}
\physDesc{Brief, 1 Blatt, 1 Seite, 808 Zeichen
\newline{}Handschrift: schwarze Tinte, deutsche Kurrent
\newline{}Schnitzler: 1) Datum mit Bleistift geändert zu »7.«  2) mit rotem Buntstift zwei Unterstreichungen
\newline{}Ordnung: mit Bleistift von unbekannter Hand nummeriert:
                                 »9« }
\pstart
           
\pstart
           {\pb}\textcolor{gray}{\textbf{STEFAN GROHSMANN}}\pend
           
\pstart
           \raggedleft{}\textcolor{gray}{\textbf{WIEN\oindex{Wien@\textbf{Wien}, \emph{A.ADM2}|pw},}}{ }11. Februar 1911\pend
           \pend
           
\pstart
           \textcolor{gray}{\textbf{LEITER DER FREIEN
                           VOLKSBÜHNE\orgindex{Wiener Freie Volksbuehne@Wiener Freie Volksbühne|pw}}}\hfill \textcolor{gray}{\textbf{VI. UFERGASSE 18\oindex{Linke Wienzeile@\textbf{Linke Wienzeile}, \emph{Straße (K.STR)}|pw}.}}\pend
           
\pstart\center{}Sehr verehrter Herr.\pend\vspace{0.5em}
\pstart
           Verzeihen Sie, daſs ich Ihre werthvolle Zeit für zwei Minuten mit einer
               Klatſchgeſchichte \strikeout{b} in Anſpruch nehmen muſs.\pend
           
\pstart
           Ein junger Literat (von Talent) Herr \uline{\textsc{Ehrenstein}\pwindex{Ehrenstein, Albert 23.12.1886 – 08.04.1950@\textsc{Ehrenstein, Albert} (23.12.1886 – 08.04.1950), \emph{Schriftsteller/Schriftstellerin}|pw}} erzählt verſchiedenen Leuten, u. A. auch dem Fackel\orgindex{Fackel@Die Fackel|pw}kraus\pwindex{Kraus, Karl 28.04.1874 – 12.06.1936@\textsc{Kraus, Karl} (28.04.1874 – 12.06.1936), \emph{Schriftsteller/Schriftstellerin, Publizist/Publizistin, Schriftsteller/Schriftstellerin}|pw}, Sie hätten ihm »beſtätigt«,
               daſs ich meine Macht als Kritiker zu erotiſchen Erpreſſungen an Schauſpielerinnen
               ausgenutzt hätte.\pend
           
\pstart
           Ich weiß wohl, daſs derlei Klatſchgeſchichten zu dem Koth gehören, der jeden
               Schnell-Schreibenden befleckt, aber ich bitte Sie doch um eine Silbe darüber, daſs
               Sie eine ſolche »Beſtätigung« nicht gaben, wie Sie ſie ja auch nicht geben
               konnten.\pend
           
\pstart
           Verzeihen Sie die lästige Behelligung!! Wäre Ihr Name in der dummen Geſchichte nicht
               eitel genannt worden, hätte ich sie nicht beachtet.\pend
           
\pstart
           Mit aufrichtigſter Hochſchätzung:{\\[\baselineskip]}\spacefill\mbox{Stefan Großmann}\pend
           \leftskip=0em{}\selectlanguage{ngerman}\endnumbering\briefempfaengerindex{Schnitzler, Arthur@\textsc{Schnitzler, Arthur}!zzzGrossmann, Stefan@\emph{von Stefan Großmann}!1911-02-072@{{[}7.{]} 2. 1911}|)be}\mylabel{L02005h}  \normalsize

\doendnotes{C}
\bigskip
\vfill

\clearpage

\footnotesize

\lohead{\textsc{register}}

% Definiere theindex-Environment komplett neu ohne reledmac
\makeatletter
\renewenvironment{theindex}{%
  \section*{\indexname}%
  \setlength{\parindent}{0pt}%
  \setlength{\parskip}{0pt plus 0.3pt}%
  \let\item\@idxitem
}{%
  \clearpage
}
\makeatother

\IfFileExists{\jobname-pw.ind}{\input{\jobname-pw.ind}}{}

\end{document}

      