%% latex-leseansicht-vorspann.tex
%% Vorspann für die Leseansicht.
%% Lädt die gemeinsame Datei latex-vorspann.tex mit nicht gesetztem Schalter.

\newif\ifkorrekturansicht
\korrekturansichtfalse

\input{../tex-inputs/latex-vorspann}


         
         \renewcommand{\erwaehntePersonen}{Personen: Albert Ehrenstein, Karl Kraus}
         \renewcommand{\erwaehnteInstitutionen}{Institutionen: Die Fackel, Wiener Freie Volksbühne}
         \renewcommand{\erwaehnteOrte}{Orte: Linke Wienzeile, Wien}
         \renewcommand{\erwaehnteWerke}{}
               \section[Stefan Großmann an Arthur Schnitzler, {[}7.{]} 2. 1911]{ Stefan Großmann an Arthur Schnitzler, {[}7.{]} 2. 1911}\nopagebreak\mylabel{v}\rehead{ }\begin{ledgroupsized}[t]{13cm}\normalsize\beginnumbering \toendnotes[C]{\smallbreak\pagebreak[2]} \Standort{CUL, Schnitzler, B 34.}
\physDesc{Brief, 1 Blatt, 1 Seite, 808 Zeichen
\newline{}Handschrift: schwarze Tinte, deutsche Kurrent
\newline{}Schnitzler: 1) Datum mit Bleistift geändert zu »7.«  2) mit rotem Buntstift zwei Unterstreichungen
\newline{}Ordnung: mit Bleistift von unbekannter Hand nummeriert:
                                 »9« }\pstart
           {\pb}\textcolor{gray}{\textbf{STEFAN GROHSMANN}}\hfill \textcolor{gray}{\textbf{WIEN\oindex{Wien@\textbf{Wien}|pw},}}{ }11. Februar 1911\pend
           \pstart
           \textcolor{gray}{\textbf{LEITER DER FREIEN
                           VOLKSBÜHNE\orgindex{Wiener Freie Volksbuehne@Wiener Freie Volksbühne|pw}}}\hfill \textcolor{gray}{\textbf{VI. UFERGASSE 18\oindex{Linke Wienzeile@\textbf{Linke Wienzeile}|pw}.}}\pend
           \pstart\center{}Sehr verehrter Herr.\pend\pstart
           Verzeihen Sie, daſs ich Ihre werthvolle Zeit für zwei Minuten mit einer
               Klatſchgeſchichte \strikeout{b} in Anſpruch nehmen muſs.\pend
           \pstart
           Ein junger Literat (von Talent) Herr \uline{\textsc{Ehrenstein}\pwindex{Ehrenstein, Albert 23.12.1886 – 08.04.1950@\textsc{Ehrenstein, Albert} (23.12.1886 – 08.04.1950), \emph{Schriftsteller}|pw}} erzählt verſchiedenen Leuten, u. A. auch dem Fackel\orgindex{Fackel@Die Fackel|pw}kraus\pwindex{Kraus, Karl 28.04.1874 – 12.06.1936@\textsc{Kraus, Karl} (28.04.1874 – 12.06.1936), \emph{Schriftsteller, Publizist}|pw}, Sie hätten ihm »beſtätigt«,
               daſs ich meine Macht als Kritiker zu erotiſchen Erpreſſungen an Schauſpielerinnen
               ausgenutzt hätte.\pend
           \pstart
           Ich weiß wohl, daſs derlei Klatſchgeſchichten zu dem Koth gehören, der jeden
               Schnell-Schreibenden befleckt, aber ich bitte Sie doch um eine Silbe darüber, daſs
               Sie eine ſolche »Beſtätigung« nicht gaben, wie Sie ſie ja auch nicht geben
               konnten.\pend
           \pstart
           Verzeihen Sie die lästige Behelligung!! Wäre Ihr Name in der dummen Geſchichte nicht
               eitel genannt worden, hätte ich sie nicht beachtet.\pend
           \pstart
           Mit aufrichtigſter Hochſchätzung:{\\[\baselineskip]}\spacefill\mbox{Stefan Großmann}\pend
           \leftskip=0em{}
         
         \endnumbering\mylabel{h}\end{ledgroupsized}  \newcommand{\dateiname}{L02005}\newcommand{\titel}{Stefan Großmann an Arthur Schnitzler, [7.] 2. 1911}\newcommand{\editorInnen}{Martin Anton Müller und Gerd-Hermann Susen}%% latex-leseansicht-abspann.tex
%% Abspann für die Leseansicht.
%% Der Schalter \ifkorrekturansicht ist bereits durch den Vorspann gesetzt.

%% latex-abspann.tex
%% Gemeinsamer Abspann für Korrekturansicht und Leseansicht.
%% Setzt den Schalter \ifkorrekturansicht voraus (gesetzt in den
%% einbindenden Dateien latex-korrekturansicht-abspann.tex bzw.
%% latex-leseansicht-abspann.tex).
%% ---------------------------------------------------------------

\normalsize

% Das esempio-Environment wird nur in der Leseansicht benötigt
\ifkorrekturansicht\else
\newenvironment{esempio}[3]%
{
    \vspace{1.5ex}
    \rlap{\underline{#1}}
    \par
    \setlength{\parindent}{0cm}
    \nopagebreak
    \leftskip=#2cm
    \rightskip=#3cm
}
{
    \par
}
\fi

\doendnotes{C}
\bigskip
\vfill

\clearpage

\footnotesize

\ifkorrekturansicht
  \lohead{\textsc{register}}
\fi

% theindex-Environment neu definieren ohne reledmac
\makeatletter
\renewenvironment{theindex}{%
  \ifkorrekturansicht
    \section*{\indexname}%
  \else
    \subsubsection*{Index der erwähnten Entitäten}%
  \fi
  \setlength{\parindent}{0pt}%
  \setlength{\parskip}{0pt plus 0.3pt}%
  \let\item\@idxitem
}{%
  \ifkorrekturansicht\clearpage\fi
}
\makeatother

\IfFileExists{\jobname-pw.ind}{\input{\jobname-pw.ind}}{}

% Quellenangabe nur in der Leseansicht
\ifkorrekturansicht\else
% Fallback-Definitionen, falls die .tex-Datei \titel etc. nicht gesetzt hat
\providecommand{\titel}{}
\providecommand{\editorInnen}{}
\providecommand{\dateiname}{\jobname}

\vspace{3cm}

\vfill

\footnotesize
\textsc{Quelle}: \titel. Herausgegeben von {\editorInnen}. In: \emph{Arthur Schnitzler: Briefwechsel mit Autorinnen und Autoren}.
 Digitale Edition, https://schnitzler-briefe.acdh.oeaw.ac.at/{\dateiname}.html (Stand \today)
\fi

\end{document}


      