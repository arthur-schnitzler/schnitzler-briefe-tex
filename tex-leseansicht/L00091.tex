%% latex-leseansicht-vorspann.tex
%% Vorspann für die Leseansicht.
%% Lädt die gemeinsame Datei latex-vorspann.tex mit nicht gesetztem Schalter.

\newif\ifkorrekturansicht
\korrekturansichtfalse

\input{../tex-inputs/latex-vorspann}


         
         \renewcommand{\erwaehntePersonen}{Personen: Wilhelm Bölsche}
         \renewcommand{\erwaehnteOrte}{Orte: Berlin, Friedrichshagen, Wien}
         \renewcommand{\erwaehnteWerke}{Werke: Das Himmelbett, Freie Bühne für den Entwickelungskampf der Zeit}
               \section[Wilhelm Bölsche an Arthur Schnitzler, {[}Anfang April 1892{]}]{ Wilhelm Bölsche an Arthur Schnitzler, {[}Anfang April 1892{]}}\nopagebreak\mylabel{v}\rehead{ }\begin{ledgroupsized}[t]{13cm}\normalsize\beginnumbering\briefempfaengerindex{Schnitzler, Arthur@\textsc{Schnitzler, Arthur}!zzzBoelsche, Wilhelm@\emph{von Wilhelm Bölsche}!1892-04-011@{{[}Anfang April 1892{]}}|(be} \toendnotes[C]{\smallbreak\pagebreak[2]} \Standort{DLA, A:Schnitzler, HS.NZ85.1.2577,5.}
\physDesc{Brief, 1 Blatt, 1 Seite, 199 Zeichen
\newline{}Handschrift: schwarze Tinte, deutsche Kurrent
\newline{}Schnitzler: 1) mit Bleistift datiert: »April 92«  2) mit rotem Buntstift nummeriert: »6«}\buchAbdrucke{\weitereDrucke{Wilhelm Bölsche: \emph{Briefwechsel. Mit Autoren der Freien Bühne}. Hg. Gerd-Hermann Susen. Berlin: \emph{Weidler} 2010, S. 679 (Werke und Briefe. Wissenschaftliche Ausgabe, Briefe I).} }\toendnotes[C]{\smallbreak}\pstart
           \raggedleft{}{\pb}Friedrichshagen\oindex{Friedrichshagen@\textbf{Friedrichshagen}|pw}{\\} Wilhelmſtr. 72.\pend
           \pstart{}Hochgeehrter Herr Doktor!\pend\pstart
           Bitte ſenden Sie\pwindex{Schnitzler, Arthur 15.05.1862 – 21.10.1931@\textsc{Schnitzler, Arthur} (15.05.1862 – 21.10.1931), \emph{Schriftsteller, Mediziner}!Himmelbett1977@\strich\emph{Das Himmelbett} {[}1977{]}|pwv} möglichſt
               bald, – doch weiß ich nicht, ob ich noch etwas in’s Maiheft\pwindex{Freie Buehne fuer den Entwickelungskampf der Zeit1892-01-01 – 1893-12-31@\emph{Freie Bühne für den Entwickelungskampf der Zeit} {[}1892-01-01 – 1893-12-31{]}|pwv}{ }ſtopfen kann, das ganz voll iſt.\pend
           \pstart
           Mit beſtem Gruß{\\[\baselineskip]} Ihr{\\[\baselineskip]}\spacefill\mbox{W. Bölsche}\pend
           \leftskip=0em{}
         
         \endnumbering\mylabel{h}\end{ledgroupsized}  \newcommand{\dateiname}{L00091}\newcommand{\titel}{Wilhelm Bölsche an Arthur Schnitzler, [Anfang April 1892]}\newcommand{\editorInnen}{Martin Anton Müller und Gerd-Hermann Susen}%% latex-leseansicht-abspann.tex
%% Abspann für die Leseansicht.
%% Der Schalter \ifkorrekturansicht ist bereits durch den Vorspann gesetzt.

%% latex-abspann.tex
%% Gemeinsamer Abspann für Korrekturansicht und Leseansicht.
%% Setzt den Schalter \ifkorrekturansicht voraus (gesetzt in den
%% einbindenden Dateien latex-korrekturansicht-abspann.tex bzw.
%% latex-leseansicht-abspann.tex).
%% ---------------------------------------------------------------

\normalsize

% Das esempio-Environment wird nur in der Leseansicht benötigt
\ifkorrekturansicht\else
\newenvironment{esempio}[3]%
{
    \vspace{1.5ex}
    \rlap{\underline{#1}}
    \par
    \setlength{\parindent}{0cm}
    \nopagebreak
    \leftskip=#2cm
    \rightskip=#3cm
}
{
    \par
}
\fi

\doendnotes{C}
\bigskip
\vfill

\clearpage

\footnotesize

\ifkorrekturansicht
  \lohead{\textsc{register}}
\fi

% theindex-Environment neu definieren ohne reledmac
\makeatletter
\renewenvironment{theindex}{%
  \ifkorrekturansicht
    \section*{\indexname}%
  \else
    \subsubsection*{Index der erwähnten Entitäten}%
  \fi
  \setlength{\parindent}{0pt}%
  \setlength{\parskip}{0pt plus 0.3pt}%
  \let\item\@idxitem
}{%
  \ifkorrekturansicht\clearpage\fi
}
\makeatother

\IfFileExists{\jobname-pw.ind}{\input{\jobname-pw.ind}}{}

% Quellenangabe nur in der Leseansicht
\ifkorrekturansicht\else
% Fallback-Definitionen, falls die .tex-Datei \titel etc. nicht gesetzt hat
\providecommand{\titel}{}
\providecommand{\editorInnen}{}
\providecommand{\dateiname}{\jobname}

\vspace{3cm}

\vfill

\footnotesize
\textsc{Quelle}: \titel. Herausgegeben von {\editorInnen}. In: \emph{Arthur Schnitzler: Briefwechsel mit Autorinnen und Autoren}.
 Digitale Edition, https://schnitzler-briefe.acdh.oeaw.ac.at/{\dateiname}.html (Stand \today)
\fi

\end{document}


      