%% latex-leseansicht-vorspann.tex
%% Vorspann für die Leseansicht.
%% Lädt die gemeinsame Datei latex-vorspann.tex mit nicht gesetztem Schalter.

\newif\ifkorrekturansicht
\korrekturansichtfalse

\input{../tex-inputs/latex-vorspann}


\section[ Paul Goldmann an Arthur Schnitzler, 26. 6. [1898]]{L02848 Paul Goldmann an Arthur Schnitzler,  26. 6. [1898]}
\nopagebreak\mylabel{L02848v}
\rehead{ }\normalsize\beginnumbering\briefempfaengerindex{Schnitzler, Arthur@\textsc{Schnitzler, Arthur}!zzzGoldmann, Paul@\emph{von Paul Goldmann}!1898-06-261@{26. 6. [1898]}|(be}
\toendnotes[C]{\smallbreak\pagebreak[2]}
\correspDesc{Versand  durch Paul Goldmann am 26. 6. [1898] in Shanghai
\newline{}Erhalt  durch Arthur Schnitzler im Zeitraum [27. 6. 1898
                  – 1. 7. 1898?] in Wien}\toendnotes[C]{\smallbreak}
\Standort{DLA, A:Schnitzler, HS.NZ85.1.3168.}
\physDesc{Brief, 3 Blätter, 12 Seiten, 4412 Zeichen
\newline{}Handschrift: blaue Tinte, deutsche Kurrent
\newline{}Schnitzler: mit Bleistift das Jahr »98« vermerkt }\toendnotes[C]{\smallbreak}
\pstart
           \raggedleft{}{\pb}\textsc{Shanghai\oindex{Shanghai@\textbf{Shanghai}|pw}}, 26. Juni.\pend
           
\pstart\center{}Mein lieber Freund,\pend\vspace{0.5em}
\pstart
           Ich danke Dir für Deinen lieben Brief (vom 17. Mai)
               und alle die Nachrichten, die er enthält. \label{K_L02848-1v}\edtext{\textsc{Richards\pwindex{Beer-Hofmann, Richard 11.\,7.\,1866 Wien – 26.\,9.\,1945 New York City@\textsc{Beer-Hofmann, Richard} (11.\,7.\,1866 Wien – 26.\,9.\,1945 New York City), \emph{Schriftsteller}|pw}} Verheirathung}{\lemma{\textnormal{\emph{Richards Verheirathung}}}\Cendnote{\textnormal{Richard Beer-Hofmann\pwindex{Beer-Hofmann, Richard 11.\,7.\,1866 Wien – 26.\,9.\,1945 New York City@\textsc{Beer-Hofmann, Richard} (11.\,7.\,1866 Wien – 26.\,9.\,1945 New York City), \emph{Schriftsteller}|pwk} und Paula Lissy\pwindex{Beer-Hofmann, Paula 25.\,2.\,1879 Wien – 30.\,10.\,1939 Zürich@\textsc{Beer-Hofmann, Paula} (25.\,2.\,1879 Wien – 30.\,10.\,1939 Zürich)|pwk} hatten am 14. 5. 1898 geheiratet. Schnitzler war Trauzeuge.}}}\label{K_L02848-1} hat mich nicht wenig überraſcht. Ich
               denke auch, er wird{ }ſein Glück \strikeout{da\textcolor{gray}{nn}} dabei finden, und das iſt ja der einzige Geſichtspunkt, unter dem wir die
               Sache zu beurtheilen haben.\pend
           
\pstart
           Aus Deinen letzten Briefen, liebſter Freund,{ }ſehe ich nicht ohne Sorge, wie {\pb}unruhig und \label{K_L02848-2v}\edtext{verdüſtert Deine Gemüths-\substVorne{}\textsuperscript{St\textcolor{gray}{ü}}\substDazwischen{}Stimmung\substHinten{} iſt und wie Du, weil es Dir im Ohre klingt}{\lemma{\textnormal{\emph{verdüstert … klingt}}}\Cendnote{\textnormal{Vgl. A. S.: \emph{Tagebuch}, 15. 5. 1898: »zu
                     Hause in tiefer Verstimmung; stets mit dem Gedanken an mein Ohr
                     beschäftigt« }}}\label{K_L02848-2}, all’ das Herrliche mißachteſt, was{ }ſonſt Dein
               Leben bietet. Es iſt unerhört, wenn ein Menſch, wie Du, in der Blüthe des Daſeins,
               auf der Höhe des Lebens, das Wort »verzweifelt« ausſpricht. Ich kann mir vorſtellen,
               wie läſtig die Symptome{ }ſein mögen, die Du{ }ſchilderſt. Bedenklich{ }ſind{ }ſie in keiner Weiſe\substVorne{}\textsuperscript{.}\substDazwischen{},\substHinten{} das weiß ich aus einer beſſeren Quelle, als von Dir {\pb}(\strikeout{ni} nimm’ mir das
               nicht übel!). Ich finde, Du biſt zu nachgiebig gegen Deine Hypochondrie. Krankheit!
               Aber um des Himmels Willen, wer iſt nicht krank? Die körperlichen Übel{ }ſind eine
               Lebens-Erſcheinung, wie alle anderen, und da{ }ſie nicht zu vermeiden{ }ſind, handelt es{ }ſich nur darum, ihnen nicht zu erlauben, daß{ }ſie gar zu viel Macht \strikeout{zu} über uns gewinnen. Ich verſichere Dich, daß man mit
               alledem fertig werden kann. Du müßteſt {\pb}Deine
               Lebensweiſe ändern, müßteſt nicht zu viel allein{ }ſein, und vor allen Dingen, das kann
               ich Dir nicht oft genug{ }ſagen, müßteſt Du aus Deinem Wien\oindex{Wien@\textbf{Wien}, \emph{Verwaltungsgebiet}|pw}er Trübſals-Winkel hinaus in die helle und große Welt. Ich hoffe, die
                  \label{K_L02848-3v}\edtext{Sommer-Reiſe}{\lemma{\textnormal{\emph{Sommer-Reise}}}\Cendnote{\textnormal{Siehe XXXX Auszeichnungsfehler: Dokument L02845 nicht gefunden.
               }}}\label{K_L02848-3} wird Dir gut thun; und der Sommer-Reiſe müßte eine Winter-Reiſe folgen; und
               dann, hoffe ich, werde ich Dich wieder einmal{ }ſehen {\pb}und Dich recht tüchtig auslachen, daß Du{ }ſo \strikeout{d} dumm
               biſt, Dein Leben Dir zu vergrämen, während Du doch, den Thatſachen nach, der Froheſte
               und Ruhigſte von uns Allen{ }ſein könnteſt und müßteſt{\dotsseven}\pend
           
\pstart
           Am \strikeout{\textcolor{gray}{A}}{ }\label{K_L02848-4v}\edtext{15. Mai}{\lemma{\textnormal{\emph{15. Mai}}}\Cendnote{\textnormal{Schnitzlers 36. Geburtstag}}}\label{K_L02848-4} habe auch
               ich in Freundſchaft Deiner gedacht. Aber war es wirklich{ }ſo{ }ſchön \label{K_L02848-5v}\edtext{vor einem Jahre}{\lemma{\textnormal{\emph{vor einem Jahre}}}\Cendnote{\textnormal{Den 35. Geburtstag hatten sie gemeinsam in Paris\oindex{Paris@\textbf{Paris}, \emph{Hauptstadt}|pwk} verbracht.}}}\label{K_L02848-5}? Ich glaube, Du hatteſt an jenem{ }Abend{ }\label{K_L02848-6v}\edtext{Kopfſchmerzen}{\lemma{\textnormal{\emph{Kopfschmerzen}}}\Cendnote{\textnormal{Das \emph{Tagebuch}\pwindex{Schnitzler, Arthur 15.\,5.\,1862 Wien – 21.\,10.\,1931 ebd.@\textsc{Schnitzler, Arthur} (15.\,5.\,1862 Wien – 21.\,10.\,1931 ebd.), \emph{Schriftsteller, Mediziner}!Tagebuch@\strich\emph{Tagebuch}|pwk} vermerkt sowohl die Kopfschmerzen als auch, dass es ein perfekter
                  Geburtstag war
                     (vgl. A. S.: \emph{Tagebuch}, 15. 5. 1897).}}}\label{K_L02848-6}
               und warſt verſtimmt. Das haſt Du{ }ſchon wieder {\pb}vergeſſen, und{ }ſo wirſt Du wahrſcheinlich auch in einem Jahre wieder vergeſſen
               haben, was Dich jetzt quält.\pend
           
\pstart
           Dein \label{K_L02848-7v}\edtext{Buch\pwindex{Schnitzler, Arthur 15.\,5.\,1862 Wien – 21.\,10.\,1931 ebd.@\textsc{Schnitzler, Arthur} (15.\,5.\,1862 Wien – 21.\,10.\,1931 ebd.), \emph{Schriftsteller, Mediziner}!Frau des Weisen. Novelletten@\strich\emph{Die Frau des Weisen. Novelletten}|pwv}}{\lemma{\textnormal{\emph{Buch}}}\Cendnote{\textnormal{Schnitzlers erste Sammlung von Prosatexten,
                     \emph{Die Frau des Weisen. Novelletten}\pwindex{Schnitzler, Arthur 15.\,5.\,1862 Wien – 21.\,10.\,1931 ebd.@\textsc{Schnitzler, Arthur} (15.\,5.\,1862 Wien – 21.\,10.\,1931 ebd.), \emph{Schriftsteller, Mediziner}!Frau des Weisen. Novelletten@\strich\emph{Die Frau des Weisen. Novelletten}|pwk}, war
                  am 3. 5. 1898 erschienen.}}}\label{K_L02848-7} habe ich geleſen. Es{ }ſind herrliche Seiten
               darin. Der »Ehrentag\pwindex{Schnitzler, Arthur 15.\,5.\,1862 Wien – 21.\,10.\,1931 ebd.@\textsc{Schnitzler, Arthur} (15.\,5.\,1862 Wien – 21.\,10.\,1931 ebd.), \emph{Schriftsteller, Mediziner}!Ehrentag@\strich\emph{Der Ehrentag}|pw}« iſt mir das Liebſte
               daraus. Aber wenn man{ }ſchon einmal im Stande iſt, dieſe erſchütternde Figur des
                  \label{K_L02848-8v}\edtext{\begin{otherlanguage}{french}\textsc{raté}\end{otherlanguage}}{\lemma{\textnormal{\emph{raté}}}\Cendnote{\textnormal{französisch: Versager; gemeint war die
                  Figur des August von Witte\pwindex{Schnitzler, Arthur 15.\,5.\,1862 Wien – 21.\,10.\,1931 ebd.@\textsc{Schnitzler, Arthur} (15.\,5.\,1862 Wien – 21.\,10.\,1931 ebd.), \emph{Schriftsteller, Mediziner}!Ehrentag@\strich\emph{Der Ehrentag}|pwkv}}}}\label{K_L02848-8} zu zeichnen, warum das Alles nur gleichſam als Epiſode hineinzwängen in eine
                  Liebesgeſchichte\pwindex{Schnitzler, Arthur 15.\,5.\,1862 Wien – 21.\,10.\,1931 ebd.@\textsc{Schnitzler, Arthur} (15.\,5.\,1862 Wien – 21.\,10.\,1931 ebd.), \emph{Schriftsteller, Mediziner}!Ehrentag@\strich\emph{Der Ehrentag}|pwv} zwiſchen
               einem Theater-{\pb}Menſch und \strikeout{\textcolor{gray}{ei}} einem düſteren \label{K_L02848-9v}\edtext{\begin{otherlanguage}{french}\textsc{Poseur}\end{otherlanguage}}{\lemma{\textnormal{\emph{Poseur}}}\Cendnote{\textnormal{französisch: Angeber}}}\label{K_L02848-9} von \textsc{August\pwindex{Schnitzler, Arthur 15.\,5.\,1862 Wien – 21.\,10.\,1931 ebd.@\textsc{Schnitzler, Arthur} (15.\,5.\,1862 Wien – 21.\,10.\,1931 ebd.), \emph{Schriftsteller, Mediziner}!Ehrentag@\strich\emph{Der Ehrentag}|pwv}}? Warum hat nicht die Rohheit des Directors den »Ehrentag\pwindex{Schnitzler, Arthur 15.\,5.\,1862 Wien – 21.\,10.\,1931 ebd.@\textsc{Schnitzler, Arthur} (15.\,5.\,1862 Wien – 21.\,10.\,1931 ebd.), \emph{Schriftsteller, Mediziner}!Ehrentag@\strich\emph{Der Ehrentag}|pw}« angeſtiftet,{ }ſtatt der Eiferſucht eines Liebhabers? Ich glaube,
               das würde die Geſchichte\pwindex{Schnitzler, Arthur 15.\,5.\,1862 Wien – 21.\,10.\,1931 ebd.@\textsc{Schnitzler, Arthur} (15.\,5.\,1862 Wien – 21.\,10.\,1931 ebd.), \emph{Schriftsteller, Mediziner}!Ehrentag@\strich\emph{Der Ehrentag}|pwv} noch
               mehr vertieft und vermenſchlicht haben. Ich meine auch, Du{ }ſollteſt Dich jetzt eine
               Zeit lang zwingen, \uline{keine Liebesgeſchichten mehr} zu{ }ſchreiben. Tief ergeifend iſt auch der {\pb}»Abſchied\pwindex{Schnitzler, Arthur 15.\,5.\,1862 Wien – 21.\,10.\,1931 ebd.@\textsc{Schnitzler, Arthur} (15.\,5.\,1862 Wien – 21.\,10.\,1931 ebd.), \emph{Schriftsteller, Mediziner}!Abschied@\strich\emph{Ein Abschied}|pw}«. Nur die letzten zwanzig Zeilen{ }ſtimmen
               mir nicht recht zum Ganzen\textcolor{gray}{,} ich weiß nicht warum? Die »Frau des Weiſen\pwindex{Schnitzler, Arthur 15.\,5.\,1862 Wien – 21.\,10.\,1931 ebd.@\textsc{Schnitzler, Arthur} (15.\,5.\,1862 Wien – 21.\,10.\,1931 ebd.), \emph{Schriftsteller, Mediziner}!Frau des Weisen. Erzählung@\strich\emph{Die Frau des Weisen. Erzählung}|pw}« mag ich nicht, die letzte Geſchichte\pwindex{Schnitzler, Arthur 15.\,5.\,1862 Wien – 21.\,10.\,1931 ebd.@\textsc{Schnitzler, Arthur} (15.\,5.\,1862 Wien – 21.\,10.\,1931 ebd.), \emph{Schriftsteller, Mediziner}!Toten schweigen@\strich\emph{Die Toten schweigen}|pwv} auch nicht{ }ſehr,
               trotz der meiſterhaften Darſtellung (\label{K_L02848-10v}\edtext{ſie iſt doch eine dumme Gans, daß{ }ſie dem Manne Alles{ }ſagt}{\lemma{\textnormal{\emph{sie … sagt}}}\Cendnote{\textnormal{\emph{Die Toten schweigen}\pwindex{Schnitzler, Arthur 15.\,5.\,1862 Wien – 21.\,10.\,1931 ebd.@\textsc{Schnitzler, Arthur} (15.\,5.\,1862 Wien – 21.\,10.\,1931 ebd.), \emph{Schriftsteller, Mediziner}!Toten schweigen@\strich\emph{Die Toten schweigen}|pwk} endet damit, dass die
                  Frau zu einem Zeitpunkt, an dem ihre außereheliche Affäre nicht mehr entdeckt
                  werden kann, beschließt, ihrem Ehemann die Wahrheit zu sagen.}}}\label{K_L02848-10}!). Der Erfolg
               Deines Buch\pwindex{Schnitzler, Arthur 15.\,5.\,1862 Wien – 21.\,10.\,1931 ebd.@\textsc{Schnitzler, Arthur} (15.\,5.\,1862 Wien – 21.\,10.\,1931 ebd.), \emph{Schriftsteller, Mediziner}!Frau des Weisen. Novelletten@\strich\emph{Die Frau des Weisen. Novelletten}|pwv}es freut mich von
               Herzen. Er iſt redlich verdient, {\pb}denn ich glaube
               nicht, daß{ }ſeit Langem in Deutſchland\oindex{Deutschland@\textbf{Deutschland}|pw} eine Sammlung\pwindex{Schnitzler, Arthur 15.\,5.\,1862 Wien – 21.\,10.\,1931 ebd.@\textsc{Schnitzler, Arthur} (15.\,5.\,1862 Wien – 21.\,10.\,1931 ebd.), \emph{Schriftsteller, Mediziner}!Frau des Weisen. Novelletten@\strich\emph{Die Frau des Weisen. Novelletten}|pwv}{ }ſo guter Novellen
               erſchienen iſt. Du biſt ein beneidenswerther Menſch, daß Du zu{ }ſolchen Leiſtungen
               fähig biſt. Aber nein, ich vergaß, Du haſt Ohrenklingen, Du biſt der Unglücklichſte
               der Unglücklichen!\pend
           
\pstart
           Mach’ Dich darauf gefaßt, daß meine theure Tante\pwindex{Mamroth, Jenny *~26.\,9.\,1846@\textsc{Mamroth, Jenny} (*~26.\,9.\,1846), \emph{Journalistin}|pwv} in der Frankfurter
                  Ztg.\pwindex{Frankfurter Zeitung@\emph{Frankfurter Zeitung}|pw} auf Dein Buch\pwindex{Schnitzler, Arthur 15.\,5.\,1862 Wien – 21.\,10.\,1931 ebd.@\textsc{Schnitzler, Arthur} (15.\,5.\,1862 Wien – 21.\,10.\,1931 ebd.), \emph{Schriftsteller, Mediziner}!Frau des Weisen. Novelletten@\strich\emph{Die Frau des Weisen. Novelletten}|pwv}{ }\label{K_L02848-11v}\edtext{ſchimpft}{\lemma{\textnormal{\emph{schimpft}}}\Cendnote{\textnormal{Eine Rezension in der \emph{Frankfurter Zeitung}\pwindex{Frankfurter Zeitung@\emph{Frankfurter Zeitung}|pwk} ist nicht belegt.}}}\label{K_L02848-11}.\pend
           
\pstart
           Welches iſt das \label{K_L02848-12v}\edtext{Stück\pwindex{Schnitzler, Arthur 15.\,5.\,1862 Wien – 21.\,10.\,1931 ebd.@\textsc{Schnitzler, Arthur} (15.\,5.\,1862 Wien – 21.\,10.\,1931 ebd.), \emph{Schriftsteller, Mediziner}!Vermächtnis. Schauspiel in drei Akten@\strich\emph{Das Vermächtnis. Schauspiel in drei Akten}|pwv}}{\lemma{\textnormal{\emph{Stück}}}\Cendnote{\textnormal{\emph{Das Vermächtnis}\pwindex{Schnitzler, Arthur 15.\,5.\,1862 Wien – 21.\,10.\,1931 ebd.@\textsc{Schnitzler, Arthur} (15.\,5.\,1862 Wien – 21.\,10.\,1931 ebd.), \emph{Schriftsteller, Mediziner}!Vermächtnis. Schauspiel in drei Akten@\strich\emph{Das Vermächtnis. Schauspiel in drei Akten}|pwk} wurde am 8. 10. 1898 am Deutschen Theater\oindex{Deutsches Theater Berlin@\textbf{Deutsches Theater Berlin}, \emph{Theater}|pwk} in Berlin\oindex{Berlin@\textbf{Berlin}, \emph{Hauptstadt}|pwk} uraufgeführt.}}}\label{K_L02848-12}, {\pb}das im Herbſt das »Deutſche Theater\orgindex{Deutsches Theater Berlin@Deutsches Theater Berlin|pw}« herausbringen{ }ſoll? Sehr traurig oder ein wenig luſtig?
               Viel Handlung? Viel Perſonen? Viel Pſychologie? Bitte,{ }ſchreib’ mir ein Wort darüber.
               Ich weiß gar nichts davon.\pend
           
\pstart
           Ich{ }ſehe viel Seltſames, aber die Schönheit fehlt in dieſem Lande\oindex{China@\textbf{China}|pwv}. Ich{ }ſehne mich unendlich nach ein
               paar Wochen Italien\oindex{Italien@\textbf{Italien}|pw}, nach Paläſten und alten
               Bildern! {\pb}Die Reiſe zieht{ }ſich{ }ſehr in die Länge.
               Ich arbeite{ }ſchwer, leide unſäglich unter meiner Impotenz\strikeout{,} dieſer neuen Welt gegenüber, habe Wochen lang Kopfſchmerzen, bin nervöſer
               als je und fühle mich\strikeout{,} mehr noch als früher aus dem
               Geleiſe geworfen. Heut{ }Abend fahre ich den \textsc{Yang-tse\oindex{Jangtsekiang@\textbf{Jangtsekiang}, \emph{Fluss}|pw}} hinauf (100 Grad \textsc{Fahrenheit} im Schatten). Meine
               Adreſſe bleibt \textsc{Shanghai\oindex{Shanghai@\textbf{Shanghai}|pw}}, {\pb}deutſches Poſtamt\oindex{Deutsches Postamt in Shanghai@\textbf{Deutsches Postamt in Shanghai}, \emph{Bürogebäude}|pwv}. Bitte,{ }ſag’ dem \textsc{Richard\pwindex{Beer-Hofmann, Richard 11.\,7.\,1866 Wien – 26.\,9.\,1945 New York City@\textsc{Beer-Hofmann, Richard} (11.\,7.\,1866 Wien – 26.\,9.\,1945 New York City), \emph{Schriftsteller}|pw}}, daß ich ihm nach \textsc{Wollzeile} 15\oindex{XXXX Ortsangabe fehlt|pw} einen Brief und ein Paket geſandt
               habe.\pend
           
\pstart
           \strikeout{B\textcolor{gray}{it}} Grüße mir Deine Freundin\pwindex{Reinhard, Marie 13.\,3.\,1871 Wien – 18.\,3.\,1899 ebd.@\textsc{Reinhard, Marie} (13.\,3.\,1871 Wien – 18.\,3.\,1899 ebd.), \emph{Gesangspädagogin}|pwv} recht herzlich und{ }ſei{ }ſelbſt tauſend Mal gegrüßt {\\[\baselineskip]}von Deinem treuen{\\[\baselineskip]}\spacefill\mbox{Paul Goldmnn}\pend
           \leftskip=0em{}\selectlanguage{ngerman}\endnumbering\briefempfaengerindex{Schnitzler, Arthur@\textsc{Schnitzler, Arthur}!zzzGoldmann, Paul@\emph{von Paul Goldmann}!1898-06-261@{26. 6. [1898]}|)be}\mylabel{L02848h}  \newcommand{\dateiname}{L02848}\newcommand{\titel}{Paul Goldmann an Arthur Schnitzler, 26. 6. [1898]}\newcommand{\editorInnen}{Martin Anton Müller und Laura Untner}%% latex-leseansicht-abspann.tex
%% Abspann für die Leseansicht.
%% Der Schalter \ifkorrekturansicht ist bereits durch den Vorspann gesetzt.

%% latex-abspann.tex
%% Gemeinsamer Abspann für Korrekturansicht und Leseansicht.
%% Setzt den Schalter \ifkorrekturansicht voraus (gesetzt in den
%% einbindenden Dateien latex-korrekturansicht-abspann.tex bzw.
%% latex-leseansicht-abspann.tex).
%% ---------------------------------------------------------------

\normalsize

% Das esempio-Environment wird nur in der Leseansicht benötigt
\ifkorrekturansicht\else
\newenvironment{esempio}[3]%
{
    \vspace{1.5ex}
    \rlap{\underline{#1}}
    \par
    \setlength{\parindent}{0cm}
    \nopagebreak
    \leftskip=#2cm
    \rightskip=#3cm
}
{
    \par
}
\fi

\doendnotes{C}
\bigskip
\vfill

\clearpage

\footnotesize

\ifkorrekturansicht
  \lohead{\textsc{register}}
\fi

% theindex-Environment neu definieren ohne reledmac
\makeatletter
\renewenvironment{theindex}{%
  \ifkorrekturansicht
    \section*{\indexname}%
  \else
    \subsubsection*{Index der erwähnten Entitäten}%
  \fi
  \setlength{\parindent}{0pt}%
  \setlength{\parskip}{0pt plus 0.3pt}%
  \let\item\@idxitem
}{%
  \ifkorrekturansicht\clearpage\fi
}
\makeatother

\IfFileExists{\jobname-pw.ind}{\input{\jobname-pw.ind}}{}

% Quellenangabe nur in der Leseansicht
\ifkorrekturansicht\else
% Fallback-Definitionen, falls die .tex-Datei \titel etc. nicht gesetzt hat
\providecommand{\titel}{}
\providecommand{\editorInnen}{}
\providecommand{\dateiname}{\jobname}

\vspace{3cm}

\vfill

\footnotesize
\textsc{Quelle}: \titel. Herausgegeben von {\editorInnen}. In: \emph{Arthur Schnitzler: Briefwechsel mit Autorinnen und Autoren}.
 Digitale Edition, https://schnitzler-briefe.acdh.oeaw.ac.at/{\dateiname}.html (Stand \today)
\fi

\end{document}


