%% latex-korrekturansicht-vorspann.tex
%% Vorspann für die Korrekturansicht.
%% Lädt die gemeinsame Datei latex-vorspann.tex mit gesetztem Schalter.

\newif\ifkorrekturansicht
\korrekturansichttrue

\input{../tex-inputs/latex-vorspann}


\section[ Felix Salten an Arthur Schnitzler, {[}11. 9. 1894{]}]{L03145 Felix Salten an Arthur Schnitzler, {[}11. 9. 1894{]}}
\nopagebreak\mylabel{L03145v}
\rehead{ }\normalsize\beginnumbering\briefempfaengerindex{Schnitzler, Arthur@\textsc{Schnitzler, Arthur}!zzzSalten, Felix@\emph{von Felix Salten}!1894-09-111@{{[}11. 9. 1894{]}}|(be}
\toendnotes[C]{\smallbreak\pagebreak[2]}\Standort{CUL, Schnitzler, B 89, A 1.}
\physDesc{Visitenkarte, 168 Zeichen
\newline{}Handschrift: Bleistift, lateinische Kurrent
\newline{}Schnitzler: mit Bleistift datiert: »11/9 94« 
\newline{}Ordnung: mit Bleistift von unbekannter Hand nummeriert: »46« }
\buchAbdrucke{\weitereDrucke{Hermann Bahr, Arthur Schnitzler: \emph{Briefwechsel, Aufzeichnungen, Dokumente (1891–1931)}. Göttingen: \emph{Wallstein} 2018, S. 80.} }\toendnotes[C]{\smallbreak}
\pstart
           \centering{}{\pb}\textcolor{gray}{\textbf{FELIX SALTEN}}\pend
           
\pstart
           \textcolor{gray}{\textbf{WIEN\oindex{Wien@\textbf{Wien}, \emph{A.ADM2}|pw},}}\hfill \textcolor{gray}{\textbf{»Berliner Neueste
                           Nachrichten\orgindex{Berliner Neueste Nachrichten@Berliner Neueste Nachrichten|pw}.«}}\pend
           
\pstart
           \textcolor{gray}{\textbf{IX., Hörlgasse 16\oindex{Hoerlgasse 16@\textbf{Hörlgasse 16}, \emph{Wohngebäude (K.WHS)}|pw}.}}\hfill \textcolor{gray}{\textbf{»Münchener
                           General-Anzeiger\orgindex{Muenchener General-Anzeiger@Münchener General-Anzeiger|pw}.«}}\pend
           \vspace{0.5em}
\pstart
           {\pb}Dear Sir,\pend
           
\pstart
           To-day, I cannot \label{K_L03145-1v}\edtext{glide with you}{\lemma{\textnormal{\emph{glide with you}}}\Cendnote{\textnormal{Salten\pwindex{Salten, Felix 06.09.1869 – 08.10.1945@\textsc{Salten, Felix} (06.09.1869 – 08.10.1945), \emph{Schriftsteller/Schriftstellerin, Journalist/Journalistin, Chefredakteur/Chefredakteurin}|pwk}
                  dürfte das deutsche Wort »gleiten« übersetzen; hier wohl als Ausdruck für
                  ›Radfahren‹ verwendet.}}}\label{K_L03145-1} because I must visit the p\substVorne{}\textsuperscript{\textcolor{gray}{oore}}\substDazwischen{}oor\substHinten{} little girl\pwindex{Pohl-Glas, Charlotte 1873-01-01 – 1944-02-15@\textsc{Pohl-Glas, Charlotte} (1873-01-01 – 1944-02-15), \emph{Schriftsteller/Schriftstellerin, Politiker/Politikerin, Sozialist/Sozialistin}|pwv} in the
                  \label{K_L03145-2v}\edtext{prison}{\lemma{\textnormal{\emph{prison}}}\Cendnote{\textnormal{Siehe Felix Salten an Arthur Schnitzler, 7. 8. 1894.
               }}}\label{K_L03145-2}, You must excuse me.\pend
           
\pstart
           Perhaps you can sent the \label{K_L03145-3v}\edtext{\substVorne{}\textsuperscript{Bo}\substDazwischen{}bo\substHinten{}ok\pwindex{Studien zur Kritik der Moderne@\emph{Studien zur Kritik der Moderne}|pwuv}}{\lemma{\textnormal{\emph{book}}}\Cendnote{\textnormal{Eventuell meint er 
                  \emph{Studien zur Kritik der Moderne}\pwindex{Studien zur Kritik der Moderne@\emph{Studien zur Kritik der Moderne}|pwk}.}}}\label{K_L03145-3} from
                  H. Bahr\pwindex{Bahr, Hermann 19.07.1863 – 15.01.1934@\textsc{Bahr, Hermann} (19.07.1863 – 15.01.1934), \emph{Schriftsteller/Schriftstellerin, Kritiker/Kritikerin}|pw}?\pend
           
\pstart
           Yours {\\[\baselineskip]}\spacefill\mbox{Salten}\pend
           \leftskip=0em{}\selectlanguage{ngerman}\endnumbering\briefempfaengerindex{Schnitzler, Arthur@\textsc{Schnitzler, Arthur}!zzzSalten, Felix@\emph{von Felix Salten}!1894-09-111@{{[}11. 9. 1894{]}}|)be}\mylabel{L03145h}  \normalsize

\doendnotes{C}
\bigskip
\vfill

\clearpage

\footnotesize

\lohead{\textsc{register}}

% Definiere theindex-Environment komplett neu ohne reledmac
\makeatletter
\renewenvironment{theindex}{%
  \section*{\indexname}%
  \setlength{\parindent}{0pt}%
  \setlength{\parskip}{0pt plus 0.3pt}%
  \let\item\@idxitem
}{%
  \clearpage
}
\makeatother

\IfFileExists{\jobname-pw.ind}{\input{\jobname-pw.ind}}{}

\end{document}

      