%% latex-leseansicht-vorspann.tex
%% Vorspann für die Leseansicht.
%% Lädt die gemeinsame Datei latex-vorspann.tex mit nicht gesetztem Schalter.

\newif\ifkorrekturansicht
\korrekturansichtfalse

\input{../tex-inputs/latex-vorspann}

\begin{center}
            \textcolor{red}{ENTWURF, NICHT FERTIG KORRIGIERT}
                      \end{center}
            
         
         \renewcommand{\erwaehntePersonen}{Personen: Hermann Bahr, Charlotte Pohl-Glas}
         \renewcommand{\erwaehnteInstitutionen}{Institutionen: Berliner Neueste Nachrichten, Münchener General-Anzeiger}
         \renewcommand{\erwaehnteOrte}{Orte: Hörlgasse, Wien}
         \renewcommand{\erwaehnteWerke}{Werke: Studien zur Kritik der Moderne}
               \section[Felix Salten an Arthur Schnitzler, {[}11. 9. 1894{]}]{ Felix Salten an Arthur Schnitzler, {[}11. 9. 1894{]}}\nopagebreak\mylabel{v}\rehead{ }\begin{ledgroupsized}[t]{13cm}\normalsize\beginnumbering \toendnotes[C]{\smallbreak\pagebreak[2]} \Standort{CUL, Schnitzler, B 89, A 1.}
\physDesc{Visitenkarte, 166 Zeichen
\newline{}Handschrift: Bleistift, lateinische Kurrent
\newline{}Schnitzler: mit Bleistift datiert: »11/9 94« 
\newline{}Ordnung: mit Bleistift von unbekannter Hand nummeriert:
                                    »46« }\buchAbdrucke{\weitereDrucke{Hermann Bahr, Arthur Schnitzler: \emph{Briefwechsel, Aufzeichnungen, Dokumente (1891–1931)}. Hg. Kurt Ifkovits und Martin Anton Müller. Göttingen: \emph{Wallstein} 2018, S. 80.} }\toendnotes[C]{\smallbreak}\pstart
           \noindent{}\centering{}{\pb}\textcolor{gray}{\textbf{FELIX SALTEN}}\pend
           \pstart
           \noindent{}\textcolor{gray}{\textbf{WIEN\oindex{Wien@\textbf{Wien}|pw},}}\hfill \textcolor{gray}{\textbf{»Berliner Neueste
                        Nachrichten\orgindex{Berliner Neueste Nachrichten@Berliner Neueste Nachrichten|pw}.«}}\pend
           \pstart
           \textcolor{gray}{\textbf{IX., Hörlgasse 16\oindex{Hoerlgasse@\textbf{Hörlgasse}|pw}.}}\hfill \textcolor{gray}{\textbf{»Münchener
                        General-Anzeiger\orgindex{Muenchener General-Anzeiger@Münchener General-Anzeiger|pw}.«}}\pend
           \pstart
           {\pb}Dear Sir,\pend
           \pstart
           To-day, I cannot glide with you because I must visit the p\substVorne{}\textsuperscript{\textcolor{gray}{×}\-\textcolor{gray}{×}\-\textcolor{gray}{×}\-\textcolor{gray}{×}}\substDazwischen{}oor\substHinten{} little girl\pwindex{Pohl-Glas, Charlotte 1873-01-01 – 1944-02-15@\textsc{Pohl-Glas, Charlotte} (1873-01-01 – 1944-02-15), \emph{Schriftstellerin, Politikerin, Sozialistin}|pwv} in the
                  \label{K_L03145-1v}\edtext{prison}{\lemma{\textnormal{\emph{prison}}}\Cendnote{\textnormal{vgl. Felix Salten an Arthur Schnitzler, 7. 8. 1894}}}\label{K_L03145-1h}. You must excuse me. \pend
           \pstart
           Perhaps you can sent the \label{K_L03145-2v}\edtext{\substVorne{}\textsuperscript{Bo}\substDazwischen{}bo\substHinten{}ok \pwindex{Bahr, Hermann 19.07.1863 – 15.01.1934@\textsc{Bahr, Hermann} (19.07.1863 – 15.01.1934), \emph{Schriftsteller, Kritiker}!Studien zur Kritik der Moderne1894@\strich\emph{Studien zur Kritik der Moderne} {[}1894{]}|pwuv}}{\lemma{\textnormal{\emph{Bobook }}}\Cendnote{\textnormal{\emph{Studien zur Kritik der Moderne}\pwindex{Bahr, Hermann 19.07.1863 – 15.01.1934@\textsc{Bahr, Hermann} (19.07.1863 – 15.01.1934), \emph{Schriftsteller, Kritiker}!Studien zur Kritik der Moderne1894@\strich\emph{Studien zur Kritik der Moderne} {[}1894{]}|pwk}?}}}\label{K_L03145-2h} from
                  H. Bahr\pwindex{Bahr, Hermann 19.07.1863 – 15.01.1934@\textsc{Bahr, Hermann} (19.07.1863 – 15.01.1934), \emph{Schriftsteller, Kritiker}|pw}? \pend
           \pstart
           Yours {\\[\baselineskip]}\spacefill\mbox{Salten}\pend
           \leftskip=0em{}
         
         \endnumbering\mylabel{h}\end{ledgroupsized}\begin{anhang}\end{anhang}\newcommand{\dateiname}{L03145}\newcommand{\titel}{Felix Salten an Arthur Schnitzler, [11. 9. 1894]}\newcommand{\editorInnen}{Martin Anton Müller und Laura Untner}%% latex-leseansicht-abspann.tex
%% Abspann für die Leseansicht.
%% Der Schalter \ifkorrekturansicht ist bereits durch den Vorspann gesetzt.

%% latex-abspann.tex
%% Gemeinsamer Abspann für Korrekturansicht und Leseansicht.
%% Setzt den Schalter \ifkorrekturansicht voraus (gesetzt in den
%% einbindenden Dateien latex-korrekturansicht-abspann.tex bzw.
%% latex-leseansicht-abspann.tex).
%% ---------------------------------------------------------------

\normalsize

% Das esempio-Environment wird nur in der Leseansicht benötigt
\ifkorrekturansicht\else
\newenvironment{esempio}[3]%
{
    \vspace{1.5ex}
    \rlap{\underline{#1}}
    \par
    \setlength{\parindent}{0cm}
    \nopagebreak
    \leftskip=#2cm
    \rightskip=#3cm
}
{
    \par
}
\fi

\doendnotes{C}
\bigskip
\vfill

\clearpage

\footnotesize

\ifkorrekturansicht
  \lohead{\textsc{register}}
\fi

% theindex-Environment neu definieren ohne reledmac
\makeatletter
\renewenvironment{theindex}{%
  \ifkorrekturansicht
    \section*{\indexname}%
  \else
    \subsubsection*{Index der erwähnten Entitäten}%
  \fi
  \setlength{\parindent}{0pt}%
  \setlength{\parskip}{0pt plus 0.3pt}%
  \let\item\@idxitem
}{%
  \ifkorrekturansicht\clearpage\fi
}
\makeatother

\IfFileExists{\jobname-pw.ind}{\input{\jobname-pw.ind}}{}

% Quellenangabe nur in der Leseansicht
\ifkorrekturansicht\else
% Fallback-Definitionen, falls die .tex-Datei \titel etc. nicht gesetzt hat
\providecommand{\titel}{}
\providecommand{\editorInnen}{}
\providecommand{\dateiname}{\jobname}

\vspace{3cm}

\vfill

\footnotesize
\textsc{Quelle}: \titel. Herausgegeben von {\editorInnen}. In: \emph{Arthur Schnitzler: Briefwechsel mit Autorinnen und Autoren}.
 Digitale Edition, https://schnitzler-briefe.acdh.oeaw.ac.at/{\dateiname}.html (Stand \today)
\fi

\end{document}


      