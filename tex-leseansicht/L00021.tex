%% latex-korrekturansicht-vorspann.tex
%% Vorspann für die Korrekturansicht.
%% Lädt die gemeinsame Datei latex-vorspann.tex mit gesetztem Schalter.

\newif\ifkorrekturansicht
\korrekturansichttrue

\input{../tex-inputs/latex-vorspann}


\section[Arthur Schnitzler an Richard Beer-Hofmann, 24. 6. 1891]{L00021 Arthur Schnitzler an Richard Beer-Hofmann, 24. 6. 1891}
\nopagebreak\mylabel{L00021v}
\rehead{ }\normalsize\beginnumbering\briefempfaengerindex{Beer-Hofmann, Richard@\textsc{Beer-Hofmann, Richard}!zzzSchnitzler, Arthur@\emph{von Arthur Schnitzler}!1891-06-241@{24. 6. 1891}|(be}
\toendnotes[C]{\smallbreak\pagebreak[2]}\Standort{YCGL, MSS 31.}
\physDesc{Postkarte, 333 Zeichen
\newline{}Handschrift: Bleistift, deutsche Kurrent
\newline{}Versand: 1) Rohrpost  2) Stempel: »\nobreak{}Wien Kärntnerring, 24 6 91, 11\nobreak{}«.  3) Stempel: »\nobreak{}Wien Landstr\textcolor{gray}{.}
                                       Hauptstr., 24/6 91, 1–2 N\nobreak{}«. }\toendnotes[C]{\smallbreak}\pstart{}{\pb}\textsc{Herrn Dr R. Beer Hofmann}\pend{}\pstart{}\textsc{Wien\oindex{Wien@\textbf{Wien}, \emph{A.ADM2}|pw}}\pend{}\pstart{}\textsc{III. Seidlgasse 30\oindex{Seidlgasse@\textbf{Seidlgasse}, \emph{Straße (K.STR)}|pw}}. \pend{}{\bigskip}\vspace{1em}
\pstart
           \noindent{}{\pb}Lieber Richard, ich habe einen völlig freien Abend vor mir, we{\geminationn} es Ihnen alſo recht iſt, treffen wir uns. Haben Sie
               die Abſicht, \label{K_L00021-1v}\edtext{eventuell aufs Land}{\lemma{\textnormal{\emph{eventuell aufs Land}}}\Cendnote{\textnormal{Schnitzler fuhr mit Beer-Hofmann\pwindex{Beer-Hofmann, Richard 1866-07-11 – 1945-09-26@\textsc{Beer-Hofmann, Richard} (1866-07-11 – 1945-09-26), \emph{Schriftsteller/Schriftstellerin}|pwk} und Salten\pwindex{Salten, Felix 06.09.1869 – 08.10.1945@\textsc{Salten, Felix} (06.09.1869 – 08.10.1945), \emph{Schriftsteller/Schriftstellerin, Journalist/Journalistin, Chefredakteur/Chefredakteurin}|pwk} in den Türkenschanzpark\oindex{Tuerkenschanzpark@\textbf{Türkenschanzpark}, \emph{Park (K.PRK)}|pwk}.}}}\label{K_L00021-1}, ſo holen Sie mich vielleicht zwiſchen
                  5 u ½ 6 ab – Erſcheinen Sie nicht, ſo werd ich \uline{\textsc{ca}}{ }6, 7 im \textsc{Griensteidl}\oindex{Cafe Griensteidl@\textbf{Café Griensteidl}, \emph{Kaffeehaus (K.KAF)}|pw}{ }ſein.\pend
           \pstart Herzlich grüß\textcolor{gray}{end} Ihr\spacefill\mbox{Arth Schnitzler}\pend{}\selectlanguage{ngerman}\endnumbering\briefempfaengerindex{Beer-Hofmann, Richard@\textsc{Beer-Hofmann, Richard}!zzzSchnitzler, Arthur@\emph{von Arthur Schnitzler}!1891-06-241@{24. 6. 1891}|)be}\mylabel{L00021h}  \normalsize

\doendnotes{C}
\bigskip
\vfill

\clearpage

\footnotesize

\lohead{\textsc{register}}

% Definiere theindex-Environment komplett neu ohne reledmac
\makeatletter
\renewenvironment{theindex}{%
  \section*{\indexname}%
  \setlength{\parindent}{0pt}%
  \setlength{\parskip}{0pt plus 0.3pt}%
  \let\item\@idxitem
}{%
  \clearpage
}
\makeatother

\IfFileExists{\jobname-pw.ind}{\input{\jobname-pw.ind}}{}

\end{document}

      