%% latex-korrekturansicht-vorspann.tex
%% Vorspann für die Korrekturansicht.
%% Lädt die gemeinsame Datei latex-vorspann.tex mit gesetztem Schalter.

\newif\ifkorrekturansicht
\korrekturansichttrue

\input{../tex-inputs/latex-vorspann}


\section[Arthur Schnitzler an Hermann Bahr, 15. 10. 1902]{L01241 Arthur Schnitzler an Hermann Bahr, 15. 10. 1902}
\nopagebreak\mylabel{L01241v}
\rehead{ }\normalsize\beginnumbering\briefempfaengerindex{Bahr, Hermann@\textsc{Bahr, Hermann}!zzzSchnitzler, Arthur@\emph{von Arthur Schnitzler}!1902-10-151@{15. 10. 1902}|(be}
\toendnotes[C]{\smallbreak\pagebreak[2]}\Standort{TMW, HS AM 60183 Ba.}
\physDesc{Postkarte, 94 Zeichen
\newline{}Handschrift: Bleistift, deutsche Kurrent
\newline{}Versand: 1) Stempel: »\nobreak{}\oindex{Berlin@\textbf{Berlin}, \emph{P.PPLC}|pwk}Berlin, W. 64, 15. 10. 02., 11–12 V\nobreak{}«.   2) Stempel: »\nobreak{}\oindex{Berlin@\textbf{Berlin}, \emph{P.PPLC}|pwk}Berlin, W. 64, 15. 10. 02., 11\textsuperscript{10} V\nobreak{}«.  3) Stempel: »\nobreak{}\oindex{Berlin@\textbf{Berlin}, \emph{P.PPLC}|pwk}Berlin, W. / H. T. A., 15. 10. 02., 11\textsuperscript{15} V\nobreak{}«.  4) Stempel: »\nobreak{}Ausgefolgt, 15 Oct., 11\nobreak{}«. 
\newline{}Ordnung: Lochung }
\buchAbdrucke{\weitereDrucke{1) Arthur Schnitzler: \emph{The Letters of Arthur Schnitzler to Hermann Bahr}. Chapel Hill: \emph{The University of North Carolina Press} 1978, S. 76.} \weitereDrucke{2) Hermann Bahr, Arthur Schnitzler: \emph{Briefwechsel, Aufzeichnungen, Dokumente (1891–1931)}. Göttingen: \emph{Wallstein} 2018, S. 244.} }\toendnotes[C]{\smallbreak}\pstart{}{\pb}Rohrpost\pend{}\pstart{}Herrn Hermann Bahr\pend{}\pstart{}Berlin\oindex{Berlin@\textbf{Berlin}, \emph{P.PPLC}|pw}\pend{}\pstart{}\textsc{Hotel Rom\oindex{Hotel de Rome@\textbf{Hotel de Rome}, \emph{Hotel (K.HTL)}|pw}}\pend{}\pstart{}Linden 39\oindex{Unter den Linden@\textbf{Unter den Linden}, \emph{P.PPLX}|pw}\pend{}{\bigskip}\vspace{1em}
\pstart
           \noindent{}{\pb}Herzlichen \label{K_L01241-1v}\edtext{Glückwunſch}{\lemma{\textnormal{\emph{Glückwunſch}}}\Cendnote{\textnormal{zur Premiere von \emph{Wienerinnen}\pwindex{Wienerinnen. Lustspiel in drei Akten@\emph{Wienerinnen. Lustspiel in drei Akten}|pwk} am 14. 10. 1902 im Berliner Theater\oindex{Berliner Theater@\textbf{Berliner Theater}, \emph{Theater (K.THE)}|pwk}}}}\label{K_L01241-1}!\pend
           
\pstart
           Dein{\\[\baselineskip]}\spacefill\mbox{ArthSch}\pend
           \leftskip=0em{}
\pstart
           15. 10. 902\pend
           \selectlanguage{ngerman}\endnumbering\briefempfaengerindex{Bahr, Hermann@\textsc{Bahr, Hermann}!zzzSchnitzler, Arthur@\emph{von Arthur Schnitzler}!1902-10-151@{15. 10. 1902}|)be}\mylabel{L01241h}  \normalsize

\doendnotes{C}
\bigskip
\vfill

\clearpage

\footnotesize

\lohead{\textsc{register}}

% Definiere theindex-Environment komplett neu ohne reledmac
\makeatletter
\renewenvironment{theindex}{%
  \section*{\indexname}%
  \setlength{\parindent}{0pt}%
  \setlength{\parskip}{0pt plus 0.3pt}%
  \let\item\@idxitem
}{%
  \clearpage
}
\makeatother

\IfFileExists{\jobname-pw.ind}{\input{\jobname-pw.ind}}{}

\end{document}

      