%% latex-leseansicht-vorspann.tex
%% Vorspann für die Leseansicht.
%% Lädt die gemeinsame Datei latex-vorspann.tex mit nicht gesetztem Schalter.

\newif\ifkorrekturansicht
\korrekturansichtfalse

\input{../tex-inputs/latex-vorspann}


\section[Arthur Schnitzler an Robert Adam, 27. 8. 1917]{L02269 Arthur Schnitzler an Robert Adam, 27. 8. 1917}
\nopagebreak\mylabel{L02269v}
\rehead{ }\normalsize\beginnumbering\briefempfaengerindex{Adam, Robert@\textsc{Adam, Robert}!zzzSchnitzler, Arthur@\emph{von Arthur Schnitzler}!1917-08-271@{27. 8. 1917}|(be}
\toendnotes[C]{\smallbreak\pagebreak[2]}
\correspDesc{Versand  durch Arthur Schnitzler am 27. 8. 1917 in Wien
\newline{}Erhalt  durch Robert Adam im Zeitraum [27. 8. 1917
                  – 31. 8. 1917?] in Wien}\toendnotes[C]{\smallbreak}
\Standort{DLA, 96.34.2/4.}
\physDesc{Postkarte, 285 Zeichen
\newline{}Handschrift: Bleistift, lateinische Kurrent
\newline{}Versand: Stempel: »\nobreak{}\oindex{XVIII., Währing@\textbf{XVIII., Währing}, \emph{Verwaltungsgebiet}|pwk}18/1 Wien, 27. VIII. 17\nobreak{}«.  }\toendnotes[C]{\smallbreak}\pstart{}{\pb}Schnitzler Wien XVIII\oindex{VIII., Josefstadt@\textbf{VIII., Josefstadt}, \emph{Verwaltungsgebiet}|pw}\pend{}\pstart{}Sternwartestr 71\oindex{Wien@\textbf{Wien}!XVIII., Währing@\textbf{XVIII., Währing}!Sternwartestraße 71@\textbf{Sternwartestraße 71}, \emph{Wohngebäude}|pw}\pend{}{\bigskip}\pstart{}Herrn Dr. Robert Adam\pend{}\pstart{}Pollak,\pend{}\pstart{}Wien XII\oindex{XII., Meidling@\textbf{XII., Meidling}, \emph{Verwaltungsgebiet}|pw}.\pend{}\pstart{}Meidlinger Hptstr 52 od. 56\oindex{Wien@\textbf{Wien}!XII., Meidling@\textbf{XII., Meidling}!Meidlinger Hauptstraße@\textbf{Meidlinger Hauptstraße}, \emph{Straße}|pw}.\pend{}{\bigskip}\vspace{1em}
\pstart
           \noindent{}{\pb}Verehrter Herr Doctor, vielen Dank für Ihren lieben Brief. Sollten
               Sie \label{K_L02269-1v}\edtext{Dinstag}{\lemma{\textnormal{\emph{Dinstag}}}\Cendnote{\textnormal{Es kam zum gewünschten Treffen am
                  Folgetag, dem 28. 8. 1917.}}}\label{K_L02269-1} gegen 7 nichts bessres
               vorhaben, so wird es mich freuen, Sie bei mir zu sehen.\pend
           
\pstart
           Herzlichst grüßt Sie{\\[\baselineskip]}Ihr erg{\\[\baselineskip]}\spacefill\mbox{A. S.}\pend
           \leftskip=0em{}\selectlanguage{ngerman}\endnumbering\briefempfaengerindex{Adam, Robert@\textsc{Adam, Robert}!zzzSchnitzler, Arthur@\emph{von Arthur Schnitzler}!1917-08-271@{27. 8. 1917}|)be}\mylabel{L02269h}  \newcommand{\dateiname}{L02269}\newcommand{\titel}{Arthur Schnitzler an Robert Adam, 27. 8. 1917}\newcommand{\editorInnen}{Martin Anton Müller und Gerd-Hermann Susen}%% latex-leseansicht-abspann.tex
%% Abspann für die Leseansicht.
%% Der Schalter \ifkorrekturansicht ist bereits durch den Vorspann gesetzt.

%% latex-abspann.tex
%% Gemeinsamer Abspann für Korrekturansicht und Leseansicht.
%% Setzt den Schalter \ifkorrekturansicht voraus (gesetzt in den
%% einbindenden Dateien latex-korrekturansicht-abspann.tex bzw.
%% latex-leseansicht-abspann.tex).
%% ---------------------------------------------------------------

\normalsize

% Das esempio-Environment wird nur in der Leseansicht benötigt
\ifkorrekturansicht\else
\newenvironment{esempio}[3]%
{
    \vspace{1.5ex}
    \rlap{\underline{#1}}
    \par
    \setlength{\parindent}{0cm}
    \nopagebreak
    \leftskip=#2cm
    \rightskip=#3cm
}
{
    \par
}
\fi

\doendnotes{C}
\bigskip
\vfill

\clearpage

\footnotesize

\ifkorrekturansicht
  \lohead{\textsc{register}}
\fi

% theindex-Environment neu definieren ohne reledmac
\makeatletter
\renewenvironment{theindex}{%
  \ifkorrekturansicht
    \section*{\indexname}%
  \else
    \subsubsection*{Index der erwähnten Entitäten}%
  \fi
  \setlength{\parindent}{0pt}%
  \setlength{\parskip}{0pt plus 0.3pt}%
  \let\item\@idxitem
}{%
  \ifkorrekturansicht\clearpage\fi
}
\makeatother

\IfFileExists{\jobname-pw.ind}{\input{\jobname-pw.ind}}{}

% Quellenangabe nur in der Leseansicht
\ifkorrekturansicht\else
% Fallback-Definitionen, falls die .tex-Datei \titel etc. nicht gesetzt hat
\providecommand{\titel}{}
\providecommand{\editorInnen}{}
\providecommand{\dateiname}{\jobname}

\vspace{3cm}

\vfill

\footnotesize
\textsc{Quelle}: \titel. Herausgegeben von {\editorInnen}. In: \emph{Arthur Schnitzler: Briefwechsel mit Autorinnen und Autoren}.
 Digitale Edition, https://schnitzler-briefe.acdh.oeaw.ac.at/{\dateiname}.html (Stand \today)
\fi

\end{document}


