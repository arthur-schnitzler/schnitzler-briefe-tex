%% latex-korrekturansicht-vorspann.tex
%% Vorspann für die Korrekturansicht.
%% Lädt die gemeinsame Datei latex-vorspann.tex mit gesetztem Schalter.

\newif\ifkorrekturansicht
\korrekturansichttrue

\input{../tex-inputs/latex-vorspann}


\section[Arthur Schnitzler an Hermann Bahr, 28. 8. 1909]{L01869 Arthur Schnitzler an Hermann Bahr, 28. 8. 1909}
\nopagebreak\mylabel{L01869v}
\rehead{ }\normalsize\beginnumbering\briefempfaengerindex{Bahr, Hermann@\textsc{Bahr, Hermann}!zzzSchnitzler, Arthur@\emph{von Arthur Schnitzler}!1909-08-281@{28. 8. 1909}|(be}
\toendnotes[C]{\smallbreak\pagebreak[2]}\Standort{TMW, HS AM 60170 Ba.}
\physDesc{Bildpostkarte, 168 Zeichen
\newline{}Handschrift: Bleistift, deutsche Kurrent
\newline{}Versand: Stempel: »\nobreak{}\oindex{Muenchen@\textbf{München}, \emph{P.PPLA}|pwk}München, 28 Aug 09, 3–4 N\nobreak{}«.  
\newline{}Ordnung: Lochung 
\newline{}Zusatz: Postkartenmotiv von Heinrich
                                    Kley.\pwindex{Kley, Heinrich 1863-04-15 – 1945-02-08@\textsc{Kley, Heinrich} (1863-04-15 – 1945-02-08), \emph{Maler/Malerin, Grafiker/Grafikerin}|pw} }
\buchAbdrucke{\weitereDrucke{1) Arthur Schnitzler: \emph{The Letters of Arthur Schnitzler to Hermann Bahr}. Chapel Hill: \emph{The University of North Carolina Press} 1978, S. 104.} \weitereDrucke{2) Hermann Bahr, Arthur Schnitzler: \emph{Briefwechsel, Aufzeichnungen, Dokumente (1891–1931)}. Göttingen: \emph{Wallstein} 2018, S. 424.} }\toendnotes[C]{\smallbreak}\pstart{}{\pb}\textsc{Abs. Schnitzler Wien\oindex{Wien@\textbf{Wien}, \emph{A.ADM2}|pw}}\pend{}\pstart{}\textsc{XVIII Spoettelg. 7.\oindex{Edmund-Weiss-Gasse 7@\textbf{Edmund-Weiß-Gasse 7}, \emph{Wohngebäude (K.WHS)}|pw}}\pend{}{\bigskip}\pstart{}\textsc{Herrn Hermann Bahr}\pend{}\pstart{}aus Wien\oindex{Wien@\textbf{Wien}, \emph{A.ADM2}|pw} d. Z.\pend{}\pstart{}\textsc{Zell im Zillerthal\oindex{Zell am Ziller@\textbf{Zell am Ziller}, \emph{A.ADM3}|pw}}\pend{}\pstart{}\textsc{Tirol\oindex{Tirol@\textbf{Tirol}, \emph{A.ADM1}|pw}}\pend{}{\bigskip}
\pstart
           \noindent{}\centering{}{\pb}\textcolor{gray}{\textbf{Alte Mariensäule\oindex{Alte Mariensaeule@\textbf{Alte Mariensäule}, \emph{Monument (K.MON)}|pw}}}\pend
           \vspace{1em}
\pstart
           {\pb}München\oindex{Muenchen@\textbf{München}, \emph{P.PPLA}|pw}{\\}28. 8. 09.\pend
           \vspace{0.5em}
\pstart
           Herzlichen Gruſs, auch deiner verehrten Gattin\pwindex{Bahr-Mildenburg, Anna 29.11.1872 – 27.01.1947@\textsc{Bahr-Mildenburg, Anna} (29.11.1872 – 27.01.1947), \emph{Sänger/Sängerin}|pwv}.\pend
           \pstart I{\geminationm}er dein \spacefill\mbox{Arthur}\pend{}\selectlanguage{ngerman}\endnumbering\briefempfaengerindex{Bahr, Hermann@\textsc{Bahr, Hermann}!zzzSchnitzler, Arthur@\emph{von Arthur Schnitzler}!1909-08-281@{28. 8. 1909}|)be}\mylabel{L01869h}  \normalsize

\doendnotes{C}
\bigskip
\vfill

\clearpage

\footnotesize

\lohead{\textsc{register}}

% Definiere theindex-Environment komplett neu ohne reledmac
\makeatletter
\renewenvironment{theindex}{%
  \section*{\indexname}%
  \setlength{\parindent}{0pt}%
  \setlength{\parskip}{0pt plus 0.3pt}%
  \let\item\@idxitem
}{%
  \clearpage
}
\makeatother

\IfFileExists{\jobname-pw.ind}{\input{\jobname-pw.ind}}{}

\end{document}

      