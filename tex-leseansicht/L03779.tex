%% latex-leseansicht-vorspann.tex
%% Vorspann für die Leseansicht.
%% Lädt die gemeinsame Datei latex-vorspann.tex mit nicht gesetztem Schalter.

\newif\ifkorrekturansicht
\korrekturansichtfalse

\input{../tex-inputs/latex-vorspann}


\section[Arthur Schnitzler an Stefan Zweig, 2. 12. 1914]{L03779 Arthur Schnitzler an Stefan Zweig, 2. 12. 1914}
\nopagebreak\mylabel{L03779v}
\rehead{ }\normalsize\beginnumbering\briefempfaengerindex{Zweig, Stefan@\textsc{Zweig, Stefan}!zzzSchnitzler, Arthur@\emph{von Arthur Schnitzler}!1914-12-021@{2. 12. 1914}|(be}
\toendnotes[C]{\smallbreak\pagebreak[2]}
\correspDesc{Versand  durch Arthur Schnitzler am 2. 12. 1914 in Wien
\newline{}Erhalt  durch Stefan Zweig im Zeitraum [2. 12. 1914
                  – 5. 12. 1914?] in Wien}\toendnotes[C]{\smallbreak}
\Standort{Jerusalem, National Library of Israel, ARC. Ms. Var. 305 1 58 Stefan Zweig Collection.}
\physDesc{Brief, 1 Blatt, 2 Seiten, 1646 Zeichen
\newline{}Schreibmaschine
\newline{}Handschrift: schwarze Tinte, lateinische Kurrent (\noindent{}Korrekturen, Ergänzungen, Unterschrift)}
\buchAbdrucke{\weitereDrucke{Arthur Schnitzler: \emph{Briefe 1913–1931}. Herausgegeben von Peter Michael Braunwarth, Richard Miklin, Susanne Pertlik und Heinrich Schnitzler. Frankfurt am Main: \emph{S. Fischer} 1984, S. 59–62.} }\toendnotes[C]{\smallbreak}
\pstart
           {\pb}\textcolor{gray}{\textbf{Dr. Arthur Schnitzler}}\hfill 2. 12. 1914.\pend
           
\pstart
           \textcolor{gray}{\textbf{Wien XVIII. Sternwartestrasse 71\oindex{Wien@\textbf{Wien}!XVIII., Währing@\textbf{XVIII., Währing}!Sternwartestraße 71@\textbf{Sternwartestraße 71}, \emph{Wohngebäude}|pw}}}\pend
           
\pstart\center{}Lieber Herr Doktor.\pend\vspace{0.5em}
\pstart
           Hier beigeschlossen ein Exemplar der Erklärung\pwindex{Schnitzler, Arthur 15.\,5.\,1862 Wien – 21.\,10.\,1931 ebd.@\textsc{Schnitzler, Arthur} (15.\,5.\,1862 Wien – 21.\,10.\,1931 ebd.), \emph{Schriftsteller, Mediziner}!Brief Artur Schnitzlers@\strich\emph{Ein Brief Artur Schnitzlers}|pwv} mit den besprochenen Aenderungen. Einen andern,
               einen wahrhaft bekennerischen Ton, vermöchte ich kaum zu finden. Je mehr man über die
               Sache nachdenkt, umso dümmer kommt sie einem vor. Ich wollte Sie noch fragen: Was\introOben{},\introOben{} denken Sie, soll nun \label{K_L03779-1v}\edtext{Rolland\pwindex{Rolland, Romain 29.\,1.\,1866 Clamecy – 30.\,12.\,1944 Vézelay@\textsc{Rolland, Romain} (29.\,1.\,1866 Clamecy – 30.\,12.\,1944 Vézelay), \emph{Schriftsteller}|pw}}{\lemma{\textnormal{\emph{Rolland}}}\Cendnote{\textnormal{Zweig\pwindex{Zweig, Stefan 28.\,11.\,1881 Wien – 23.\,2.\,1942 Petrópolis@\textsc{Zweig, Stefan} (28.\,11.\,1881 Wien – 23.\,2.\,1942 Petrópolis), \emph{Schriftsteller}|pwk} schrieb am 5. 12. 1914
                  an Rolland\pwindex{Rolland, Romain 29.\,1.\,1866 Clamecy – 30.\,12.\,1944 Vézelay@\textsc{Rolland, Romain} (29.\,1.\,1866 Clamecy – 30.\,12.\,1944 Vézelay), \emph{Schriftsteller}|pwk}: »Arthur Schnitzler sendet Ihnen bei
                        diesem Anlass seine respectvollen Grüße (er wohnt, wenn Sie sie erwidern
                        wollen, Wien XVIII,
                           Sternwartestraße 71\oindex{Wien@\textbf{Wien}!XVIII., Währing@\textbf{XVIII., Währing}!Sternwartestraße 71@\textbf{Sternwartestraße 71}, \emph{Wohngebäude}|pw}). Ich freue mich, dass nun ein neuer Beweis in
                        Ihren Händen ist, wie sehr unsere Besten sich bemühen, gerecht zu bleiben.
                        Lassen Sie sich durch einzelne Manifestationen des Hasses nicht verstimmen:
                        gerade extreme Naturen verlieren in solchen Zeiten am leichtesten das innere
                        Gleichgewicht. Und es bedarf einer großen moralischen Stabilität, um
                        aufrecht zu bleiben in diesen Stürmen!{ / }{[}\ldots{]}{ / }PS: Das Original\pwindex{Schnitzler, Arthur 15.\,5.\,1862 Wien – 21.\,10.\,1931 ebd.@\textsc{Schnitzler, Arthur} (15.\,5.\,1862 Wien – 21.\,10.\,1931 ebd.), \emph{Schriftsteller, Mediziner}!Brief Artur Schnitzlers@\strich\emph{Ein Brief Artur Schnitzlers}|pwv}{ }Schnitzlers könnte auch in einer deutschen Schweizer
                           Zeitung\pwindex{Neue Zürcher Zeitung@\emph{Neue Zürcher Zeitung}|pwv} erscheinen! Bitte dann um ein Exemplar!« Romain Rolland\pwindex{Rolland, Romain 29.\,1.\,1866 Clamecy – 30.\,12.\,1944 Vézelay@\textsc{Rolland, Romain} (29.\,1.\,1866 Clamecy – 30.\,12.\,1944 Vézelay), \emph{Schriftsteller}|pwk}, Stefan Zweig\pwindex{Zweig, Stefan 28.\,11.\,1881 Wien – 23.\,2.\,1942 Petrópolis@\textsc{Zweig, Stefan} (28.\,11.\,1881 Wien – 23.\,2.\,1942 Petrópolis), \emph{Schriftsteller}|pwk}: \emph{Von Welt zu Welt. Briefe
                        einer Freundschaft 1914–1918}. Mit einem Begleitwort von Peter
                     Handke. Aus dem Französischen von Eva und Gerhard Schwewe (Briefe Rollands) und
                     Christel Gersch (Briefe Zweigs). Berlin: \emph{Aufbau
                        Verlag}{ }2014. }}}\label{K_L03779-1} mit unseren Erklärungen\pwindex{Schnitzler, Arthur 15.\,5.\,1862 Wien – 21.\,10.\,1931 ebd.@\textsc{Schnitzler, Arthur} (15.\,5.\,1862 Wien – 21.\,10.\,1931 ebd.), \emph{Schriftsteller, Mediziner}!Brief Artur Schnitzlers@\strich\emph{Ein Brief Artur Schnitzlers}|pwv} tun? Sie ins Französi\introOben{}s\introOben{}che\oindex{Frankreich@\textbf{Frankreich}|pw} übersetzen und eventuell nicht nur an das Journal de Gen\substVorne{}\textsuperscript{é}\substDazwischen{}è\substHinten{}ve\orgindex{Journal de Genève@Journal de Genève|pw}, sondern sie auch an französische\oindex{Frankreich@\textbf{Frankreich}|pw} Journale weitergeben? Könnte er es auch übernehmen den
               Erklärungen in ein deutsches \label{K_L03779-2v}\edtext{schweizer\oindex{Schweiz@\textbf{Schweiz}|pw} Journal}{\lemma{\textnormal{\emph{schweizer Journal}}}\Cendnote{\textnormal{\emph{Ein Brief Artur Schnitzlers}\pwindex{Schnitzler, Arthur 15.\,5.\,1862 Wien – 21.\,10.\,1931 ebd.@\textsc{Schnitzler, Arthur} (15.\,5.\,1862 Wien – 21.\,10.\,1931 ebd.), \emph{Schriftsteller, Mediziner}!Brief Artur Schnitzlers@\strich\emph{Ein Brief Artur Schnitzlers}|pwk}. In: \emph{Neue Zürcher Zeitung}\pwindex{Neue Zürcher Zeitung@\emph{Neue Zürcher Zeitung}|pwk}, Jg. 135, Nr. 1700,
                        22. 12. 1914, 2. Mittagsblatt, S. 2.}}}\label{K_L03779-2} Aufnahme zu
               verschaffen? Mir fällt eben ein, dass wir neulich über Regierungsrat Winternitz\pwindex{Winternitz, Jakob von 3.\,3.\,1843 Horažďovice – 26.\,1.\,1921 Wien@\textsc{Winternitz, Jakob von} (3.\,3.\,1843 Horažďovice – 26.\,1.\,1921 Wien), \emph{Ministerialbeamter}|pw} nicht gesprochen haben. Bitte um
               eine Zeile, wann ich Sie anrufen dürfte. Den \label{K_L03779-3v}\edtext{Appell an die Blätter}{\lemma{\textnormal{\emph{Appell an die Blätter}}}\Cendnote{\textnormal{XXXX Auszeichnungsfehler: Dokument L03774 nicht gefunden.}}}\label{K_L03779-3}, mit dem meine vorige Erklärung\pwindex{Schnitzler, Arthur 15.\,5.\,1862 Wien – 21.\,10.\,1931 ebd.@\textsc{Schnitzler, Arthur} (15.\,5.\,1862 Wien – 21.\,10.\,1931 ebd.), \emph{Schriftsteller, Mediziner}!Brief Artur Schnitzlers@\strich\emph{Ein Brief Artur Schnitzlers}|pwv} schloss, {\pb}(\label{K_L03779-4v}\edtext{bitte \introOben{}die\introOben{} beide\introOben{}n\introOben{} Exemplare zu
                  vernichten}{\lemma{\textnormal{\emph{bitte … vernichten}}}\Cendnote{\textnormal{Zweig\pwindex{Zweig, Stefan 28.\,11.\,1881 Wien – 23.\,2.\,1942 Petrópolis@\textsc{Zweig, Stefan} (28.\,11.\,1881 Wien – 23.\,2.\,1942 Petrópolis), \emph{Schriftsteller}|pwk} kam der Bitte nicht nach, die erste
                  Fassung ist als Beilage von XXXX Auszeichnungsfehler: Dokument L03774 nicht gefunden
                  überliefert.}}}\label{K_L03779-4}) habe ich diesmal weggelassen. Ich glaube, man bedarf ihrer
               nicht.\pend
           
\pstart
           Ich hatte heute den sonderbaren \label{K_L03779-5v}\edtext{Traum}{\lemma{\textnormal{\emph{Traum}}}\Cendnote{\textnormal{Vgl. A. S.: \emph{Tagebuch}, 2. 12. 1914.}}}\label{K_L03779-5}, dass
               ich mit Ihnen in einem offenen Fiaker auf erhöhter Strasse durch eine irgendwie
               orientalische Stadt fuhr; \substVorne{}\textsuperscript{s }\substDazwischen{}S\substHinten{}ie transportierten mich nämlich nach Sibirien\oindex{Sibirien@\textbf{Sibirien}, \emph{Region}|pw}, was ein wenig dadurch gemildert war, dass der Weg zuerst durchs
                  Helenenthal\oindex{Helenental@\textbf{Helenental}, \emph{Tal}|pw} führen sollte. Ich war nur auf
               sechs Monate verbannt, hatte aber den leisen Verdacht gegen Sie, dass Sie mich für
               immer dort lassen wollten. Im übrigen sahen Sie, was eine allgemein bekannte Tatsache
               war, einem Grafen Schönstein wie einem Zwillingsbruder ähnlich. Dieser Graf wurde
               auch irgendwie sichtbar, sah Ihnen natürlich gar nicht ähnlich, hatte einen offenen
               Ueberzieher mit Pelz, trug einen Zwicker und sah verdrossen drein. Nun deuten Sie\substVorne{}\textsuperscript{.}\substDazwischen{}!\substHinten{}\pend
           
\pstart
           Herzlichst grüssend{\\[\baselineskip]}Ihr{\\[\baselineskip]}\spacefill\mbox{{[}hs.:{]} Arthur Schnitzler}\pend
           \leftskip=0em{}\selectlanguage{ngerman}\endnumbering\briefempfaengerindex{Zweig, Stefan@\textsc{Zweig, Stefan}!zzzSchnitzler, Arthur@\emph{von Arthur Schnitzler}!1914-12-021@{2. 12. 1914}|)be}\mylabel{L03779h}  \newcommand{\dateiname}{L03779}\newcommand{\titel}{Arthur Schnitzler an Stefan Zweig, 2. 12. 1914}\newcommand{\editorInnen}{Selma Jahnke und Martin Anton Müller}%% latex-leseansicht-abspann.tex
%% Abspann für die Leseansicht.
%% Der Schalter \ifkorrekturansicht ist bereits durch den Vorspann gesetzt.

%% latex-abspann.tex
%% Gemeinsamer Abspann für Korrekturansicht und Leseansicht.
%% Setzt den Schalter \ifkorrekturansicht voraus (gesetzt in den
%% einbindenden Dateien latex-korrekturansicht-abspann.tex bzw.
%% latex-leseansicht-abspann.tex).
%% ---------------------------------------------------------------

\normalsize

% Das esempio-Environment wird nur in der Leseansicht benötigt
\ifkorrekturansicht\else
\newenvironment{esempio}[3]%
{
    \vspace{1.5ex}
    \rlap{\underline{#1}}
    \par
    \setlength{\parindent}{0cm}
    \nopagebreak
    \leftskip=#2cm
    \rightskip=#3cm
}
{
    \par
}
\fi

\doendnotes{C}
\bigskip
\vfill

\clearpage

\footnotesize

\ifkorrekturansicht
  \lohead{\textsc{register}}
\fi

% theindex-Environment neu definieren ohne reledmac
\makeatletter
\renewenvironment{theindex}{%
  \ifkorrekturansicht
    \section*{\indexname}%
  \else
    \subsubsection*{Index der erwähnten Entitäten}%
  \fi
  \setlength{\parindent}{0pt}%
  \setlength{\parskip}{0pt plus 0.3pt}%
  \let\item\@idxitem
}{%
  \ifkorrekturansicht\clearpage\fi
}
\makeatother

\IfFileExists{\jobname-pw.ind}{\input{\jobname-pw.ind}}{}

% Quellenangabe nur in der Leseansicht
\ifkorrekturansicht\else
% Fallback-Definitionen, falls die .tex-Datei \titel etc. nicht gesetzt hat
\providecommand{\titel}{}
\providecommand{\editorInnen}{}
\providecommand{\dateiname}{\jobname}

\vspace{3cm}

\vfill

\footnotesize
\textsc{Quelle}: \titel. Herausgegeben von {\editorInnen}. In: \emph{Arthur Schnitzler: Briefwechsel mit Autorinnen und Autoren}.
 Digitale Edition, https://schnitzler-briefe.acdh.oeaw.ac.at/{\dateiname}.html (Stand \today)
\fi

\end{document}


