%% latex-korrekturansicht-vorspann.tex
%% Vorspann für die Korrekturansicht.
%% Lädt die gemeinsame Datei latex-vorspann.tex mit gesetztem Schalter.

\newif\ifkorrekturansicht
\korrekturansichttrue

\input{../tex-inputs/latex-vorspann}


\section[Arthur Schnitzler an Stefan Zweig, 2. 12. 1914]{L03779 Arthur Schnitzler an Stefan Zweig, 2. 12. 1914}
\nopagebreak\mylabel{L03779v}
\rehead{ }\normalsize\beginnumbering\briefempfaengerindex{Zweig, Stefan@\textsc{Zweig, Stefan}!zzzSchnitzler, Arthur@\emph{von Arthur Schnitzler}!1914-12-021@{2. 12. 1914}|(be}
\toendnotes[C]{\smallbreak\pagebreak[2]}\Standort{Jerusalem, National Library of Israel, ARC. Ms. Var. 305 1 58 Stefan Zweig Collection.}
\physDesc{Brief, 1 Blatt, 2 Seiten, 1646 Zeichen
\newline{}Handschrift: schwarze Tinte, lateinische Kurrent (\noindent{}Korrekturen, Ergänzungen, Unterschrift)}
\buchAbdrucke{\weitereDrucke{Arthur Schnitzler: \emph{Briefe 1913–1931}. Frankfurt am Main: \emph{S. Fischer} 1984, S. 59–62.} }\toendnotes[C]{\smallbreak}
\pstart
           {\pb}\textcolor{gray}{\textbf{Dr. Arthur Schnitzler}}\hfill 2. 12. 1914. \pend
           
\pstart
           \textcolor{gray}{\textbf{Wien XVIII.
                        Sternwartestrasse 71\oindex{Sternwartestrasse 71@\textbf{Sternwartestraße 71}, \emph{Wohngebäude (K.WHS)}|pw}}}\pend
           
\pstart\center{}Lieber Herr Doktor.\pend\vspace{0.5em}
\pstart
           Hier beigeschlossen ein Exemplar der Erklärung\pwindex{Brief Artur Schnitzlers@\emph{Ein Brief Artur Schnitzlers}|pwv} mit den besprochenen Aenderungen. Einen andern,
               einen wahrhaft bekennerischen Ton, vermöchte ich kaum zu finden. Je mehr man über die
               Sache nachdenkt, umso dümmer kommt sie einem vor. Ich wollte Sie noch fragen: Was\introOben{},\introOben{} denken sie, soll nun Rolland\pwindex{Rolland, Romain 29.01.1866 – 30.12.1944@\textsc{Rolland, Romain} (29.01.1866 – 30.12.1944), \emph{Schriftsteller/Schriftstellerin}|pw} mit unseren Erklärungen\pwindex{Brief Artur Schnitzlers@\emph{Ein Brief Artur Schnitzlers}|pwv} tun? Sie ins Französische\oindex{Frankreich@\textbf{Frankreich}, \emph{A.PCLI}|pw}
               übersetzen und eventuell nicht nur an das Journal de
                  Gen\substVorne{}\textsuperscript{é}\substDazwischen{}è\substHinten{}ve\orgindex{Journal de Geneve@Journal de Genève|pw}, sondern sie auch an französische\oindex{Frankreich@\textbf{Frankreich}, \emph{A.PCLI}|pw}
               Journale weitergeben? Könnte er es auch übernehmen den Erklärungen in ein deutsches
                  \label{K_L03779-1v}\edtext{schweizer\oindex{Schweiz@\textbf{Schweiz}, \emph{A.PCLI}|pw} Journal}{\lemma{\textnormal{\emph{schweizer Journal}}}\Cendnote{\textnormal{\emph{Ein Brief Artur Schnitzlers}\pwindex{Brief Artur Schnitzlers@\emph{Ein Brief Artur Schnitzlers}|pwk}. In: \emph{Neue Zürcher Zeitung}\pwindex{Neue Zuercher Zeitung@\emph{Neue Zürcher Zeitung}|pwk}, Jg. 135, Nr. 1700,
                        22. 12. 1914, 2. Mittagsblatt, S. 2.}}}\label{K_L03779-1} Aufnahme zu
               verschaffen? Mir fällt eben ein, dass wir neulich über Regierungsrat Winternitz\pwindex{Winternitz, Jakob von 03.03.1843 – 26.01.1921@\textsc{Winternitz, Jakob von} (03.03.1843 – 26.01.1921), \emph{Ministerialbeamter/Ministerialbeamte}|pw} nicht gesprochen haben. Bitte um
               eine Zeile, wann ich Sie anrufen dürfte. Den \label{K_L03779-2v}\edtext{Appell an die Blätter}{\lemma{\textnormal{\emph{Appell an die Blätter}}}\Cendnote{\textnormal{Arthur Schnitzler an Stefan Zweig, 27. 11. 1914.}}}\label{K_L03779-2}, mit dem meine vorige Erklärung\pwindex{Brief Artur Schnitzlers@\emph{Ein Brief Artur Schnitzlers}|pwv} schloss, {\pb}(\label{K_L03779-3v}\edtext{bitte \introOben{}die\introOben{} beide\introOben{}n\introOben{} Exemplare zu
                  vernichten}{\lemma{\textnormal{\emph{bitte … vernichten}}}\Cendnote{\textnormal{Zweig\pwindex{Zweig, Stefan 28.11.1881 – 23.02.1942@\textsc{Zweig, Stefan} (28.11.1881 – 23.02.1942), \emph{Schriftsteller/Schriftstellerin}|pwk} kam der Bitte nicht nach, er behielt
                  sich ein Exemplar.}}}\label{K_L03779-3}) habe ich diesmal weggelassen. Ich glaube, man bedarf
               ihrer nicht. Ich hatte heute den sonderbaren \label{K_L03779-4v}\edtext{Traum}{\lemma{\textnormal{\emph{Traum}}}\Cendnote{\textnormal{Vgl. A. S.: \emph{Tagebuch}, 2. 12. 1914.}}}\label{K_L03779-4}, dass
               ich mit Ihnen in einem offenen Fiaker auf erhöhter Strasse durch eine irgendwie
               orientalische Stadt fuhr; \substVorne{}\textsuperscript{s }\substDazwischen{}S\substHinten{}ie transportierten mich nämlich nach Sibirien\oindex{Sibirien@\textbf{Sibirien}, \emph{L.RGN}|pw}, was ein wenig dadurch gemildert war, dass der Weg zuerst durchs
                  Helenenthal\oindex{Helenental@\textbf{Helenental}, \emph{Tal (N.TAL)}|pw} führen sollte. Ich war nur auf
               sechs Monate verbannt, hatte aber den leisen Verdacht gegen Sie, dass Sie mich für
               immer dort lassen wollten. Im übrigen sahen Sie, was eine allgemein bekannte Tatsache
               war, einem Grafen Schönstein wie einem Zwillingsbruder ähnlich. Dieser Graf wurde
               auch irgendwie sichtbar, sah Ihnen natürlich gar nicht ähnlich, hatte einen offenen
               Ueberzieher mit Pelz, trug einen Zwicker und sah verdrossen drein. Nun deuten Sie\substVorne{}\textsuperscript{.}\substDazwischen{}!\substHinten{}\pend
           
\pstart
           Herzlichst grüssend{\\[\baselineskip]}Ihr{\\[\baselineskip]}\spacefill\mbox{{[}hs.:{]} Arthur Schnitzler}\pend
           \leftskip=0em{}\selectlanguage{ngerman}\endnumbering\briefempfaengerindex{Zweig, Stefan@\textsc{Zweig, Stefan}!zzzSchnitzler, Arthur@\emph{von Arthur Schnitzler}!1914-12-021@{2. 12. 1914}|)be}\mylabel{L03779h}
\begin{anhang}
\end{anhang}\normalsize

\doendnotes{C}
\bigskip
\vfill

\clearpage

\footnotesize

\lohead{\textsc{register}}

% Definiere theindex-Environment komplett neu ohne reledmac
\makeatletter
\renewenvironment{theindex}{%
  \section*{\indexname}%
  \setlength{\parindent}{0pt}%
  \setlength{\parskip}{0pt plus 0.3pt}%
  \let\item\@idxitem
}{%
  \clearpage
}
\makeatother

\IfFileExists{\jobname-pw.ind}{\input{\jobname-pw.ind}}{}

\end{document}

      