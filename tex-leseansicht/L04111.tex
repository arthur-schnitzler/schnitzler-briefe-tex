%% latex-leseansicht-vorspann.tex
%% Vorspann für die Leseansicht.
%% Lädt die gemeinsame Datei latex-vorspann.tex mit nicht gesetztem Schalter.

\newif\ifkorrekturansicht
\korrekturansichtfalse

\input{../tex-inputs/latex-vorspann}


\section[Arthur Schnitzler an Gustav Schwarzkopf, 9. 5. 1897]{L04111 Arthur Schnitzler an Gustav Schwarzkopf, 9. 5. 1897}
\nopagebreak\mylabel{L04111v}
\rehead{ }\normalsize\beginnumbering\briefempfaengerindex{Schwarzkopf, Gustav@\textsc{Schwarzkopf, Gustav}!zzzSchnitzler, Arthur@\emph{von Arthur Schnitzler}!1897-05-091@{9. 5. 1897}|(be}
\toendnotes[C]{\smallbreak\pagebreak[2]}
\correspDesc{Versand  durch Arthur Schnitzler am 9. 5. 1897 in Paris
\newline{}Erhalt  durch Gustav Schwarzkopf im Zeitraum [10. 5. 1897 – 14. 5. 1897?] in Wien}\toendnotes[C]{\smallbreak}
\Standort{CUL, Schnitzler, B 96.}
\physDesc{Brief, 2 Blätter, 7 Seiten, 3197 Zeichen
\newline{}Handschrift: schwarze Tinte, deutsche Kurrent}
\buchAbdrucke{\weitereDrucke{Arthur Schnitzler: \emph{Briefe 1875–1912}. Herausgegeben von Therese Nickl und Heinrich Schnitzler. Frankfurt am Main: \emph{S. Fischer} 1981, S. 320–322.} }\toendnotes[C]{\smallbreak}
\pstart
           \raggedleft{}{\pb}\textsc{Paris\oindex{Paris@\textbf{Paris}, \emph{Hauptstadt}|pw}}{ }9. 5. 97\pend
           
\pstart
           \raggedleft{}rue de Maubeuge\oindex{5, rue de Maubeuge@\textbf{5, rue de Maubeuge}, \emph{Wohngebäude}|pw}\pend
           \vspace{0.5em}
\pstart
           Lieber Guſtav. Ich rufe 9. Sie rufen 9 u. 18 (Zeilen.) Ich halte ſie
               und rufe noch drüber – als \textsc{Poker}-Kiebitz verſtehen Sie ja
               das. Den Empfang Ihres herzlichen Neides beſtätige ich mit herzlichem Dank; ich
               wollte von andren{ }ſo liebenswürdig geſchätzt als von Ihnen beneidet{ }ſein. Im übrigen
               liegt weniger Anlaſs vor, mich zu beneiden; – wenigſtens für die abgelaufenen vier
               Wochen; da ich nichts verſchreien will. Sie ſind ein fleißiger Leſer der Zeit\pwindex{Zeit. Wiener Wochenschrift@\emph{Die Zeit. Wiener Wochenschrift}|pw} und haben wohl die \textsc{Entrefilets\pwindex{Graf, Max 1.\,10.\,1873 Wien – 24.\,6.\,1958 ebd.@\textsc{Graf, Max} (1.\,10.\,1873 Wien – 24.\,6.\,1958 ebd.), \emph{Kritiker}!Pariser Köpfe@\strich\emph{Pariser Köpfe}|pwv}\pwindex{Graf, Max 1.\,10.\,1873 Wien – 24.\,6.\,1958 ebd.@\textsc{Graf, Max} (1.\,10.\,1873 Wien – 24.\,6.\,1958 ebd.), \emph{Kritiker}!Pariser Köpfe [II]@\strich\emph{Pariser Köpfe [II]}|pwv}\pwindex{Graf, Max 1.\,10.\,1873 Wien – 24.\,6.\,1958 ebd.@\textsc{Graf, Max} (1.\,10.\,1873 Wien – 24.\,6.\,1958 ebd.), \emph{Kritiker}!Pariser Köpfe [III]@\strich\emph{Pariser Köpfe [III]}|pwv}} geleſen, die in den letzten Nu{\geminationm}ern über Paris\oindex{Paris@\textbf{Paris}, \emph{Hauptstadt}|pw} darin {\pb}zu leſen waren; sie kommen aus der
               Feder des Herrn Graf\pwindex{Graf, Max 1.\,10.\,1873 Wien – 24.\,6.\,1958 ebd.@\textsc{Graf, Max} (1.\,10.\,1873 Wien – 24.\,6.\,1958 ebd.), \emph{Kritiker}|pw} und ich weiß nicht, ob
               Sie von Wien\oindex{Wien@\textbf{Wien}, \emph{Verwaltungsgebiet}|pw} aus die ganze Läpperei dieser Notizen\pwindex{Graf, Max 1.\,10.\,1873 Wien – 24.\,6.\,1958 ebd.@\textsc{Graf, Max} (1.\,10.\,1873 Wien – 24.\,6.\,1958 ebd.), \emph{Kritiker}!Pariser Köpfe@\strich\emph{Pariser Köpfe}|pwv}\pwindex{Graf, Max 1.\,10.\,1873 Wien – 24.\,6.\,1958 ebd.@\textsc{Graf, Max} (1.\,10.\,1873 Wien – 24.\,6.\,1958 ebd.), \emph{Kritiker}!Pariser Köpfe [II]@\strich\emph{Pariser Köpfe [II]}|pwv}\pwindex{Graf, Max 1.\,10.\,1873 Wien – 24.\,6.\,1958 ebd.@\textsc{Graf, Max} (1.\,10.\,1873 Wien – 24.\,6.\,1958 ebd.), \emph{Kritiker}!Pariser Köpfe [III]@\strich\emph{Pariser Köpfe [III]}|pwv}
               beurtheilen können. Allerdings iſt es auch der Mühe werth hieherzuko{\geminationm}en um Paris\oindex{Paris@\textbf{Paris}, \emph{Hauptstadt}|pw} nur
               miszuverſtehen; und auch hier gibt es eine hoffnungsvolle Jugend, welche
               bemüht ſcheint, das Weſen ihrer Heimat zu fälſchen und das Leben zu misdeuten. Man
               hat ſie mir neulich bei Gelegenheit einer Theateraufführg\eventindex{Théâtre de l’Œuvre@\textbf{Théâtre de l’Œuvre}!Aufführung von Ton sang, 7.5.1897@Aufführung von Ton sang, 7.5.1897|pwv}{ }\substVorne{}\textsuperscript{d}\substDazwischen{}im\substHinten{} »\textsc{Oeuvre\oindex{Théâtre de l’Œuvre@\textbf{Théâtre de l’Œuvre}, \emph{Theater}|pw}}« gezeigt. Ich habe Herrn {\pb}\textsc{Mauclair\pwindex{Mauclair, Camille 29.\,11.\,1872 Paris – 23.\,4.\,1945 ebd.@\textsc{Mauclair, Camille} (29.\,11.\,1872 Paris – 23.\,4.\,1945 ebd.), \emph{Schriftsteller}|pw}} geſehn; zugleich Herrn \textsc{La Jeunesse\pwindex{La Jeunesse, Ernest 1874 Paris – 1917 ebd.@\textsc{La Jeunesse, Ernest} (1874 Paris – 1917 ebd.), \emph{Schriftsteller}|pw}}, der den erſtgenannten bei der vorletzten \textsc{œuvre\orgindex{Théâtre de l’Œuvre@Théâtre de l’Œuvre|pw}}-Vorstellg geohrfeigt hat und, wie man ſich erzählt, darauf hinarbeitet, Kaiſer
               von Frankreich\oindex{Frankreich@\textbf{Frankreich}|pw} zu werden. Er beginnt damit,
                  Feu{[}i{]}lletons zu{ }ſchreiben und mythiſche Medaillen zu
               vertheilen. Ich habe zahlreiche andere Jünglinge mit praeraphaelitiſchen Fräuleins
               gesehn, die in den \begin{otherlanguage}{french}\textsc{Couloirs}\end{otherlanguage} herumgeſpenſterten. Leider hab ich auch ein Stück\pwindex{Bataille, Henri 4.\,4.\,1872 Nîmes – 2.\,3.\,1922 Rueil-Malmaison@\textsc{Bataille, Henri} (4.\,4.\,1872 Nîmes – 2.\,3.\,1922 Rueil-Malmaison), \emph{Schriftsteller}!Ton sang@\strich\emph{Ton sang}|pwv}\eventindex{Théâtre de l’Œuvre@\textbf{Théâtre de l’Œuvre}!Aufführung von Ton sang, 7.5.1897@Aufführung von Ton sang, 7.5.1897|pwv} geſehen, war aber nur aber zwei erſten Akten gewachſen. Im erſten ja{\geminationm}ert {\pb}ein
               Schwindſüchtiger, daſs er ſchwindſüchtig und complicirt iſt (\label{K_L04111-1v}\edtext{\textsc{\begin{otherlanguage}{french}Oh ma mère que je suis compliqué\end{otherlanguage}}}{\lemma{\textnormal{\emph{Oh … compliqué}}}\Cendnote{\textnormal{französisch: Oh meine Mutter, wie kompliziert ich bin}}}\label{K_L04111-1})
               und eine Blinde, daſs ſie blind iſt; im zweiten kommen die Blinde und der
               Schwindſüchtige mit verbundenen Handgelenken herein; es iſt eine Transfuſion gemacht
               worden und der Schwindſüchtige wird geſund. Und die Blinde, welche noch im erſten Akt
               die Geliebte des Bruders des Schwindſüchtigen war, wird die Frau des
               Schwindſüchtigen. Man glaubt eben nicht, was die {\pb}Transfusion für ein Wundermittel iſt!
               Dann ko{\geminationm}en noch zwei Akte, die ich nicht mehr geſehn
               habe und das ganze heißt: »Ton sang\pwindex{Bataille, Henri 4.\,4.\,1872 Nîmes – 2.\,3.\,1922 Rueil-Malmaison@\textsc{Bataille, Henri} (4.\,4.\,1872 Nîmes – 2.\,3.\,1922 Rueil-Malmaison), \emph{Schriftsteller}!Ton sang@\strich\emph{Ton sang}|pw}«. –\pend
           
\pstart
           – Sehr intereſſant waren mir die drei Haupterfolge der Saiſon, \textsc{Douloureuse\pwindex{Donnay, Maurice 12.\,10.\,1859 Paris – 31.\,3.\,1945 ebd.@\textsc{Donnay, Maurice} (12.\,10.\,1859 Paris – 31.\,3.\,1945 ebd.), \emph{Schriftsteller}!Douloureuse@\strich\emph{La Douloureuse}|pw}}\eventindex{Théâtre du Vaudeville@\textbf{Théâtre du Vaudeville}!Aufführung von Elle et lui, La Douloureuse, 22.4.1897@Aufführung von Elle et lui, La Douloureuse, 22.4.1897|pwv}, , \textsc{Carriére\pwindex{Hermant, Abel 3.\,2.\,1862 Paris – 28.\,9.\,1950@\textsc{Hermant, Abel} (3.\,2.\,1862 Paris – 28.\,9.\,1950), \emph{Schriftsteller}!Carrière@\strich\emph{La Carrière}|pw}}\eventindex{Théâtre du Gymnase Marie Bell@\textbf{Théâtre du Gymnase Marie Bell}!Aufführung von La Carrière, 17.4.1897@Aufführung von La Carrière, 17.4.1897|pwv}, \textsc{Snob\pwindex{Guiches, Gustave 18.\,6.\,1860 – 3.\,8.\,1935 Paris@\textsc{Guiches, Gustave} (18.\,6.\,1860 – 3.\,8.\,1935 Paris), \emph{Schriftsteller}!Snob@\strich\emph{Snob}|pw}}\eventindex{Théâtre de la Renaissance@\textbf{Théâtre de la Renaissance}!Aufführung von Snob, 23.4.1897@Aufführung von Snob, 23.4.1897|pwv} – hauptsächlich wegen – hauptsächlich wegen der Familienähnlichkeit der drei
                  Stücke\pwindex{Donnay, Maurice 12.\,10.\,1859 Paris – 31.\,3.\,1945 ebd.@\textsc{Donnay, Maurice} (12.\,10.\,1859 Paris – 31.\,3.\,1945 ebd.), \emph{Schriftsteller}!Douloureuse@\strich\emph{La Douloureuse}|pwv}\pwindex{Hermant, Abel 3.\,2.\,1862 Paris – 28.\,9.\,1950@\textsc{Hermant, Abel} (3.\,2.\,1862 Paris – 28.\,9.\,1950), \emph{Schriftsteller}!Carrière@\strich\emph{La Carrière}|pwv}\pwindex{Guiches, Gustave 18.\,6.\,1860 – 3.\,8.\,1935 Paris@\textsc{Guiches, Gustave} (18.\,6.\,1860 – 3.\,8.\,1935 Paris), \emph{Schriftsteller}!Snob@\strich\emph{Snob}|pwv}.
               In allen dreien könnten vor allem die Titel gewechſelt werden, ohne dſs es ein Menſch
               merkt; ja ich hatte{ }ſogar den Eindruck, ſie würden da{\geminationn}{ }{\pb}beſſer zu den Stücken paſſen. Alle
                  drei\pwindex{Donnay, Maurice 12.\,10.\,1859 Paris – 31.\,3.\,1945 ebd.@\textsc{Donnay, Maurice} (12.\,10.\,1859 Paris – 31.\,3.\,1945 ebd.), \emph{Schriftsteller}!Douloureuse@\strich\emph{La Douloureuse}|pwv}\pwindex{Hermant, Abel 3.\,2.\,1862 Paris – 28.\,9.\,1950@\textsc{Hermant, Abel} (3.\,2.\,1862 Paris – 28.\,9.\,1950), \emph{Schriftsteller}!Carrière@\strich\emph{La Carrière}|pwv}\pwindex{Guiches, Gustave 18.\,6.\,1860 – 3.\,8.\,1935 Paris@\textsc{Guiches, Gustave} (18.\,6.\,1860 – 3.\,8.\,1935 Paris), \emph{Schriftsteller}!Snob@\strich\emph{Snob}|pwv} ſind keine Stücke; in allen dreien\pwindex{Donnay, Maurice 12.\,10.\,1859 Paris – 31.\,3.\,1945 ebd.@\textsc{Donnay, Maurice} (12.\,10.\,1859 Paris – 31.\,3.\,1945 ebd.), \emph{Schriftsteller}!Douloureuse@\strich\emph{La Douloureuse}|pwv}\pwindex{Hermant, Abel 3.\,2.\,1862 Paris – 28.\,9.\,1950@\textsc{Hermant, Abel} (3.\,2.\,1862 Paris – 28.\,9.\,1950), \emph{Schriftsteller}!Carrière@\strich\emph{La Carrière}|pwv}\pwindex{Guiches, Gustave 18.\,6.\,1860 – 3.\,8.\,1935 Paris@\textsc{Guiches, Gustave} (18.\,6.\,1860 – 3.\,8.\,1935 Paris), \emph{Schriftsteller}!Snob@\strich\emph{Snob}|pwv} ereignet{ }ſich
               das wichtige \uline{zwiſchen} dem vorletzten u. letzten Akt –
               u. wir müſſen einfach dran glauben; in allen dreien\pwindex{Donnay, Maurice 12.\,10.\,1859 Paris – 31.\,3.\,1945 ebd.@\textsc{Donnay, Maurice} (12.\,10.\,1859 Paris – 31.\,3.\,1945 ebd.), \emph{Schriftsteller}!Douloureuse@\strich\emph{La Douloureuse}|pwv}\pwindex{Hermant, Abel 3.\,2.\,1862 Paris – 28.\,9.\,1950@\textsc{Hermant, Abel} (3.\,2.\,1862 Paris – 28.\,9.\,1950), \emph{Schriftsteller}!Carrière@\strich\emph{La Carrière}|pwv}\pwindex{Guiches, Gustave 18.\,6.\,1860 – 3.\,8.\,1935 Paris@\textsc{Guiches, Gustave} (18.\,6.\,1860 – 3.\,8.\,1935 Paris), \emph{Schriftsteller}!Snob@\strich\emph{Snob}|pwv} iſt der letzte
               Akt eigentlich nichts als ein ſentimentaler Dialog der zwei Hauptperſonen, welche
               eingeſehen haben daſs u.ſ.w. – Aber geſpielt wird – zum Entzücken. – Ihre
               Anſicht über Reicher\pwindex{Reicher, Emanuel 18.\,6.\,1849 Bochnia – 15.\,5.\,1924 Berlin@\textsc{Reicher, Emanuel} (18.\,6.\,1849 Bochnia – 15.\,5.\,1924 Berlin), \emph{Schauspieler}|pw}{ }ſcheint {\pb}mir die richtige zu ſein; ich habe{ }ſeine Größe nie begriffen, obwohl er, wie Sie wohl in der Zeit\pwindex{Zeit. Wiener Wochenschrift@\emph{Die Zeit. Wiener Wochenschrift}|pw}{ }geleſen\pwindex{Bahr, Hermann 19.\,7.\,1863 Linz – 15.\,1.\,1934 München@\textsc{Bahr, Hermann} (19.\,7.\,1863 Linz – 15.\,1.\,1934 München), \emph{Schriftsteller, Kritiker}!Emanuel Reicher (Als Gast im Carltheater vom 28. April bis zum 3. Mai 1897)@\strich\emph{Emanuel Reicher (Als Gast im Carltheater vom 28. April bis zum 3. Mai 1897)}|pwv} haben, »\label{K_L04111-2v}\edtext{für mich eingeſtanden \substVorne{}\textsuperscript{hat}\substDazwischen{}iſt\substHinten{}, als die guten Wien\oindex{Wien@\textbf{Wien}, \emph{Verwaltungsgebiet}|pw}er noch über mich
                  lächelten.\pwindex{Bahr, Hermann 19.\,7.\,1863 Linz – 15.\,1.\,1934 München@\textsc{Bahr, Hermann} (19.\,7.\,1863 Linz – 15.\,1.\,1934 München), \emph{Schriftsteller, Kritiker}!Emanuel Reicher (Als Gast im Carltheater vom 28. April bis zum 3. Mai 1897)@\strich\emph{Emanuel Reicher (Als Gast im Carltheater vom 28. April bis zum 3. Mai 1897)}|pwv}}{\lemma{\textnormal{\emph{für … lächelten.}}}\Cendnote{\textnormal{Die Stelle in Hermann Bahrs\pwindex{Bahr, Hermann 19.\,7.\,1863 Linz – 15.\,1.\,1934 München@\textsc{Bahr, Hermann} (19.\,7.\,1863 Linz – 15.\,1.\,1934 München), \emph{Schriftsteller, Kritiker}|pwk}{ }\emph{Emanuel Reicher}\pwindex{Bahr, Hermann 19.\,7.\,1863 Linz – 15.\,1.\,1934 München@\textsc{Bahr, Hermann} (19.\,7.\,1863 Linz – 15.\,1.\,1934 München), \emph{Schriftsteller, Kritiker}!Emanuel Reicher (Als Gast im Carltheater vom 28. April bis zum 3. Mai 1897)@\strich\emph{Emanuel Reicher (Als Gast im Carltheater vom 28. April bis zum 3. Mai 1897)}|pwk} (\emph{Die Zeit}\pwindex{Zeit. Wiener Wochenschrift@\emph{Die Zeit. Wiener Wochenschrift}|pwk}, Bd. 11, H. 135, 1.\,5.\,1897, S. 75–76, hier: S. 75) lautet: »Er ist für
                     unseren Schnitzler eingestanden, als die
                     guten Wien\oindex{Wien@\textbf{Wien}, \emph{Verwaltungsgebiet}|pw} er noch vornehm über ihn
                     lächelten;«. }}}\label{K_L04111-2}«\pend
           
\pstart
           Ich bleibe noch etwa 14 Tagen hier, dann geh ich nach London\oindex{London@\textbf{London}, \emph{Hauptstadt}|pw}, und bin wohl in den letzten Maitagen in Wien\oindex{Wien@\textbf{Wien}, \emph{Verwaltungsgebiet}|pw}. Sollte ich nicht auch Sie als Radfahrer
               wiederfinden? –\pend
           
\pstart
           Leben Sie wohl und ſeien Sie herzlich{\\[\baselineskip]} gegrüßt! Ihr \spacefill\mbox{Arth
                  Schn}\pend
           \leftskip=0em{}\selectlanguage{ngerman}\endnumbering\briefempfaengerindex{Schwarzkopf, Gustav@\textsc{Schwarzkopf, Gustav}!zzzSchnitzler, Arthur@\emph{von Arthur Schnitzler}!1897-05-091@{9. 5. 1897}|)be}\mylabel{L04111h}
\begin{anhang}
\end{anhang}\newcommand{\dateiname}{L04111}\newcommand{\titel}{Arthur Schnitzler an Gustav Schwarzkopf, 9. 5. 1897}\newcommand{\editorInnen}{Herausgegeben von Jahnke, SelmaMüller, Martin Anton}%% latex-leseansicht-abspann.tex
%% Abspann für die Leseansicht.
%% Der Schalter \ifkorrekturansicht ist bereits durch den Vorspann gesetzt.

%% latex-abspann.tex
%% Gemeinsamer Abspann für Korrekturansicht und Leseansicht.
%% Setzt den Schalter \ifkorrekturansicht voraus (gesetzt in den
%% einbindenden Dateien latex-korrekturansicht-abspann.tex bzw.
%% latex-leseansicht-abspann.tex).
%% ---------------------------------------------------------------

\normalsize

% Das esempio-Environment wird nur in der Leseansicht benötigt
\ifkorrekturansicht\else
\newenvironment{esempio}[3]%
{
    \vspace{1.5ex}
    \rlap{\underline{#1}}
    \par
    \setlength{\parindent}{0cm}
    \nopagebreak
    \leftskip=#2cm
    \rightskip=#3cm
}
{
    \par
}
\fi

\doendnotes{C}
\bigskip
\vfill

\clearpage

\footnotesize

\ifkorrekturansicht
  \lohead{\textsc{register}}
\fi

% theindex-Environment neu definieren ohne reledmac
\makeatletter
\renewenvironment{theindex}{%
  \ifkorrekturansicht
    \section*{\indexname}%
  \else
    \subsubsection*{Index der erwähnten Entitäten}%
  \fi
  \setlength{\parindent}{0pt}%
  \setlength{\parskip}{0pt plus 0.3pt}%
  \let\item\@idxitem
}{%
  \ifkorrekturansicht\clearpage\fi
}
\makeatother

\IfFileExists{\jobname-pw.ind}{\input{\jobname-pw.ind}}{}

% Quellenangabe nur in der Leseansicht
\ifkorrekturansicht\else
% Fallback-Definitionen, falls die .tex-Datei \titel etc. nicht gesetzt hat
\providecommand{\titel}{}
\providecommand{\editorInnen}{}
\providecommand{\dateiname}{\jobname}

\vspace{3cm}

\vfill

\footnotesize
\textsc{Quelle}: \titel. Herausgegeben von {\editorInnen}. In: \emph{Arthur Schnitzler: Briefwechsel mit Autorinnen und Autoren}.
 Digitale Edition, https://schnitzler-briefe.acdh.oeaw.ac.at/{\dateiname}.html (Stand \today)
\fi

\end{document}


