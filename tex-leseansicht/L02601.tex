%% latex-leseansicht-vorspann.tex
%% Vorspann für die Leseansicht.
%% Lädt die gemeinsame Datei latex-vorspann.tex mit nicht gesetztem Schalter.

\newif\ifkorrekturansicht
\korrekturansichtfalse

\input{../tex-inputs/latex-vorspann}


         
         \renewcommand{\erwaehntePersonen}{Personen: Georg Brandes, Auguste Hauschner}
         \renewcommand{\erwaehnteOrte}{Orte: Salzburg, Sternwartestraße, Südtirol, Wien, XVIII., Währing, Österreichischer Hof}
         \renewcommand{\erwaehnteWerke}{}
               \section[Arthur Schnitzler an Gertrud Rung, 2. 6. 1925]{ Arthur Schnitzler an Gertrud Rung, 2. 6. 1925}\nopagebreak\mylabel{v}\rehead{ }\begin{ledgroupsized}[t]{13cm}\normalsize\beginnumbering \toendnotes[C]{\smallbreak\pagebreak[2]} \Standort{Kopenhagen, Det Kongelige Bibliotek, NKS 3756 4°.}
\physDesc{Postkarte
\newline{}Handschrift: schwarze Tinte, lateinische Kurrent\newline{}Versand: Stempel: »\nobreak{}\oindex{XVIII., Waehring@\textbf{XVIII., Währing}|pwk}18/\textsubscript{1} Wien  \textcolor{gray}{110}, 2. 6. 2\textcolor{gray}{5}, 19\nobreak{}«.  \newline{}Ordnung: mit Bleistift von unbekannter Hand vermerkt: »\textsc{Schnitzler}« }\buchAbdrucke{\weitereDrucke{Arthur Schnitzler: \emph{Arthur Schnitzlers Briefe nach Dänemark}. Hg. Ernst-Ulrich Pinkert. Roskilde: \emph{Center for Østrigsk-Nordiske Kulturstudier} 2006, S. 20.} }\toendnotes[C]{\smallbreak}\pstart{}{\pb}\label{T_L02601-1v}\edtext{\textcolor{gray}{\textbf{A. S.}}}{\lemma{\textnormal{\emph{A. S.}}}\Cendnote{\textnormal{ovaler Absenderkleber}}}\label{T_L02601-1h}\pend{}\pstart{}\textcolor{gray}{\textbf{WIEN, XVIII.}}\oindex{XVIII., Waehring@\textbf{XVIII., Währing}|pw}\pend{}\pstart{}\textcolor{gray}{\textbf{STERNWARTESTR. 71}}\oindex{Sternwartestrasse@\textbf{Sternwartestraße}|pw}\pend{}{\bigskip}\pstart{}{\pb}Frau \pend{}\pstart{}Gertrud Rung,\pend{}\pstart{}Oesterr. Hof –\oindex{Oesterreichischer Hof@\textbf{Österreichischer Hof}|pw}\pend{}\pstart{}Salzburg.\oindex{Salzburg@\textbf{Salzburg}|pw}\pend{}{\bigskip}\pstart
           \raggedleft{}{\pb}Wien\oindex{Wien@\textbf{Wien}|pw},
                        2. 6. 2\textcolor{gray}{5}\pend
           \pstart
           Verehrte Frau Rung, danke sehr für Ihre lieben und
               erfreulichen Nachrichten! Wie lange sind Sie noch in Salzburg\oindex{Salzburg@\textbf{Salzburg}|pw}? Ich ko{\geminationm}e vielleicht mit der Rückreise
               aus \label{K_L02601-1v}\edtext{Südtirol\oindex{Suedtirol@\textbf{Südtirol}|pw} (wohin ich etwa am 17. d.
               abreise) gegen Ende Juni nach Salzburg\oindex{Salzburg@\textbf{Salzburg}|pw}}{\lemma{\textnormal{\emph{Südtirol … Salzburg}}}\Cendnote{\textnormal{Schnitzler\pwindex{Schnitzler, Arthur 15.05.1862 – 21.10.1931@\textsc{Schnitzler, Arthur} (15.05.1862 – 21.10.1931), \emph{Schriftsteller, Mediziner}|pwk} war von 23. 6. 1925 bis 3. 7. 1925 in Südtirol\oindex{Suedtirol@\textbf{Südtirol}|pwk} und reiste ohne Unterbrechung nach Wien\oindex{Wien@\textbf{Wien}|pwk} durch.}}}\label{K_L02601-1h} – treff ich Sie und Brandes\pwindex{Brandes, Georg 04.02.1842 – 19.02.1927@\textsc{Brandes, Georg} (04.02.1842 – 19.02.1927)|pw} noch an – ? Grüßen Sie den von mir verehrten u geliebten Freund
               viele Male. Alles herzliche Ihnen.\pend
           \pstart
           {\pb}Auf Wiedersehen
               {\\[\baselineskip]}Ihr \spacefill\mbox{Arthur Schnitzler}\pend
           \leftskip=0em{}
         
         \endnumbering\mylabel{h}\end{ledgroupsized}  \newcommand{\dateiname}{L02601}\newcommand{\titel}{Arthur Schnitzler an Gertrud Rung, 2. 6. 1925}\newcommand{\editorInnen}{Martin Anton Müller und Laura Untner}%% latex-leseansicht-abspann.tex
%% Abspann für die Leseansicht.
%% Der Schalter \ifkorrekturansicht ist bereits durch den Vorspann gesetzt.

%% latex-abspann.tex
%% Gemeinsamer Abspann für Korrekturansicht und Leseansicht.
%% Setzt den Schalter \ifkorrekturansicht voraus (gesetzt in den
%% einbindenden Dateien latex-korrekturansicht-abspann.tex bzw.
%% latex-leseansicht-abspann.tex).
%% ---------------------------------------------------------------

\normalsize

% Das esempio-Environment wird nur in der Leseansicht benötigt
\ifkorrekturansicht\else
\newenvironment{esempio}[3]%
{
    \vspace{1.5ex}
    \rlap{\underline{#1}}
    \par
    \setlength{\parindent}{0cm}
    \nopagebreak
    \leftskip=#2cm
    \rightskip=#3cm
}
{
    \par
}
\fi

\doendnotes{C}
\bigskip
\vfill

\clearpage

\footnotesize

\ifkorrekturansicht
  \lohead{\textsc{register}}
\fi

% theindex-Environment neu definieren ohne reledmac
\makeatletter
\renewenvironment{theindex}{%
  \ifkorrekturansicht
    \section*{\indexname}%
  \else
    \subsubsection*{Index der erwähnten Entitäten}%
  \fi
  \setlength{\parindent}{0pt}%
  \setlength{\parskip}{0pt plus 0.3pt}%
  \let\item\@idxitem
}{%
  \ifkorrekturansicht\clearpage\fi
}
\makeatother

\IfFileExists{\jobname-pw.ind}{\input{\jobname-pw.ind}}{}

% Quellenangabe nur in der Leseansicht
\ifkorrekturansicht\else
% Fallback-Definitionen, falls die .tex-Datei \titel etc. nicht gesetzt hat
\providecommand{\titel}{}
\providecommand{\editorInnen}{}
\providecommand{\dateiname}{\jobname}

\vspace{3cm}

\vfill

\footnotesize
\textsc{Quelle}: \titel. Herausgegeben von {\editorInnen}. In: \emph{Arthur Schnitzler: Briefwechsel mit Autorinnen und Autoren}.
 Digitale Edition, https://schnitzler-briefe.acdh.oeaw.ac.at/{\dateiname}.html (Stand \today)
\fi

\end{document}


      