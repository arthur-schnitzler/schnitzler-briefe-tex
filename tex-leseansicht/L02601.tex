%% latex-korrekturansicht-vorspann.tex
%% Vorspann für die Korrekturansicht.
%% Lädt die gemeinsame Datei latex-vorspann.tex mit gesetztem Schalter.

\newif\ifkorrekturansicht
\korrekturansichttrue

\input{../tex-inputs/latex-vorspann}


\section[Arthur Schnitzler an Gertrud Rung, 2. 6. 1925]{L02601 Arthur Schnitzler an Gertrud Rung, 2. 6. 1925}
\nopagebreak\mylabel{L02601v}
\rehead{ }\normalsize\beginnumbering\briefempfaengerindex{Rung, Gertrud@\textsc{Rung, Gertrud}!zzzSchnitzler, Arthur@\emph{von Arthur Schnitzler}!1925-06-021@{2. 6. 1925}|(be}
\toendnotes[C]{\smallbreak\pagebreak[2]}\Standort{Kopenhagen, Det Kongelige Bibliotek, NKS 3756 4°.}
\physDesc{Postkarte, 432 Zeichen
\newline{}Handschrift: schwarze Tinte, lateinische Kurrent
\newline{}Versand: Stempel: »\nobreak{}\oindex{XVIII., Waehring@\textbf{XVIII., Währing}, \emph{A.ADM3}|pwk}18/\textsubscript{1} Wien 
                                          \textcolor{gray}{110}, 2. 6. 2\textcolor{gray}{5}, 19\nobreak{}«.  
\newline{}Ordnung: mit Bleistift von unbekannter Hand vermerkt: »\textsc{Schnitzler}« }
\buchAbdrucke{\weitereDrucke{Arthur Schnitzler: \emph{Arthur Schnitzlers Briefe nach Dänemark}. Roskilde: \emph{Center for Østrigsk-Nordiske Kulturstudier} 2006, S. 20.} }\toendnotes[C]{\smallbreak}\pstart{}{\pb}\label{T_L02601-1v}\edtext{\textcolor{gray}{\textbf{A. S.}}}{\lemma{\textnormal{\emph{A. S.}}}\Cendnote{\textnormal{ovaler Absenderkleber}}}\label{T_L02601-1}\pend{}\pstart{}\textcolor{gray}{\textbf{WIEN, XVIII.}}\oindex{XVIII., Waehring@\textbf{XVIII., Währing}, \emph{A.ADM3}|pw}\pend{}\pstart{}\textcolor{gray}{\textbf{STERNWARTESTR. 71}}\oindex{Sternwartestrasse 71@\textbf{Sternwartestraße 71}, \emph{Wohngebäude (K.WHS)}|pw}\pend{}{\bigskip}\pstart{}{\pb}Frau \pend{}\pstart{}Gertrud Rung,\pend{}\pstart{}Oesterr. Hof –\oindex{Oesterreichischer Hof@\textbf{Österreichischer Hof}, \emph{Hotel (K.HTL)}|pw}\pend{}\pstart{}Salzburg.\oindex{Salzburg@\textbf{Salzburg}, \emph{A.ADM2}|pw}\pend{}{\bigskip}\vspace{1em}
\pstart
           \raggedleft{}{\pb}Wien\oindex{Wien@\textbf{Wien}, \emph{A.ADM2}|pw}, 2. 6. 2\textcolor{gray}{5}\pend
           \vspace{0.5em}
\pstart
           Verehrte Frau Rung, danke sehr für Ihre lieben und
               erfreulichen Nachrichten! Wie lange sind Sie noch in Salzburg\oindex{Salzburg@\textbf{Salzburg}, \emph{A.ADM2}|pw}? Ich ko{\geminationm}e vielleicht mit der Rückreise
               aus \label{K_L02601-1v}\edtext{Südtirol\oindex{Suedtirol@\textbf{Südtirol}, \emph{A.ADM2}|pw} (wohin ich etwa am 17. d. abreise) gegen Ende Juni nach Salzburg\oindex{Salzburg@\textbf{Salzburg}, \emph{A.ADM2}|pw}}{\lemma{\textnormal{\emph{Südtirol … Salzburg}}}\Cendnote{\textnormal{Schnitzler war vom 23. 6. 1925 bis zum 3. 7. 1925 in Südtirol\oindex{Suedtirol@\textbf{Südtirol}, \emph{A.ADM2}|pwk} und reiste ohne Unterbrechung nach Wien\oindex{Wien@\textbf{Wien}, \emph{A.ADM2}|pwk} durch.}}}\label{K_L02601-1} – treff ich Sie und Brandes\pwindex{Brandes, Georg 04.02.1842 – 19.02.1927@\textsc{Brandes, Georg} (04.02.1842 – 19.02.1927)|pw} noch an – ? Grüßen Sie den von mir verehrten u
               geliebten Freund viele Male. Alles herzliche Ihnen.\pend
           
\pstart
           {\pb}Auf Wiedersehen {\\[\baselineskip]}Ihr
                  \spacefill\mbox{Arthur Schnitzler}\pend
           \leftskip=0em{}\selectlanguage{ngerman}\endnumbering\briefempfaengerindex{Rung, Gertrud@\textsc{Rung, Gertrud}!zzzSchnitzler, Arthur@\emph{von Arthur Schnitzler}!1925-06-021@{2. 6. 1925}|)be}\mylabel{L02601h}  \normalsize

\doendnotes{C}
\bigskip
\vfill

\clearpage

\footnotesize

\lohead{\textsc{register}}

% Definiere theindex-Environment komplett neu ohne reledmac
\makeatletter
\renewenvironment{theindex}{%
  \section*{\indexname}%
  \setlength{\parindent}{0pt}%
  \setlength{\parskip}{0pt plus 0.3pt}%
  \let\item\@idxitem
}{%
  \clearpage
}
\makeatother

\IfFileExists{\jobname-pw.ind}{\input{\jobname-pw.ind}}{}

\end{document}

      