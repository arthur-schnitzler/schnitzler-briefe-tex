%% latex-leseansicht-vorspann.tex
%% Vorspann für die Leseansicht.
%% Lädt die gemeinsame Datei latex-vorspann.tex mit nicht gesetztem Schalter.

\newif\ifkorrekturansicht
\korrekturansichtfalse

\input{../tex-inputs/latex-vorspann}


         
         \renewcommand{\erwaehntePersonen}{Personen: Hermann Bahr, Otto Brahm, Rudolf Rittner, Olga Schnitzler}
         \renewcommand{\erwaehnteOrte}{Orte: Berlin, Deutsches Theater Berlin, Hotel Continental (Berlin), Konstanz, Konzilgebäude, Schumannstraße}
         \renewcommand{\erwaehnteWerke}{}
               \section[Hermann Bahr an Arthur Schnitzler, 12. 2. 1904]{ Hermann Bahr an Arthur Schnitzler, 12. 2. 1904}\nopagebreak\mylabel{v}\rehead{ }\begin{ledgroupsized}[t]{13cm}\normalsize\beginnumbering \toendnotes[C]{\smallbreak\pagebreak[2]} \Standort{CUL, Schnitzler, B 5b.}
\physDesc{Bildpostkarte, 197 Zeichen
\newline{}Handschrift: Bleistift, deutsche Kurrent
\newline{}Versand: 1) Stempel: »\nobreak{}\oindex{Konstanz@\textbf{Konstanz}|pwk}Konstanz–Basel Bahnpost, Zug 1627, 13/2 04\nobreak{}«.   2) Stempel: »\nobreak{}14/2 {[}04{]}\nobreak{}«.  3) Stempel: »\nobreak{}\textcolor{gray}{×}\-\textcolor{gray}{×}{[}/2{]} 04, 7–8N, Bestellt vom Postamte 7\nobreak{}«.  4) von unbekannten Händen Adresse teilweise gestrichen und ergänzt:
                                    »7«, »\textsc{NW 7 Continentalhotel\oindex{Hotel Continental (Berlin)@\textbf{Hotel Continental (Berlin)}|pw}}« sowie »\textcolor{gray}{×}\-\textcolor{gray}{×}\-\textcolor{gray}{×}\-\textcolor{gray}{×}\-\textcolor{gray}{×}\-\textcolor{gray}{×}\-\textcolor{gray}{×}\-\textcolor{gray}{×}\-\textcolor{gray}{×}\-\textcolor{gray}{×}\-\textcolor{gray}{×}\-\textcolor{gray}{×}
                                    8/6.«}\buchAbdrucke{\weitereDrucke{Hermann Bahr, Arthur Schnitzler: \emph{Briefwechsel, Aufzeichnungen, Dokumente (1891–1931)}. Hg. Kurt Ifkovits und Martin Anton Müller. Göttingen: \emph{Wallstein} 2018, S. 300.} }\toendnotes[C]{\smallbreak}\pstart{}{\pb}Herrn \textsc{D\textsuperscript{r} Arthur Schnitzler}\pend{}\pstart{}\textsc{Deutsches Theater}\oindex{Deutsches Theater Berlin@\textbf{Deutsches Theater Berlin}|pw}\pend{}\pstart{}\textsc{Berlin NW}\oindex{Berlin@\textbf{Berlin}|pw}\pend{}\pstart{}\textsc{Schumannstr.}\oindex{Schumannstrasse@\textbf{Schumannstraße}|pw}\pend{}{\bigskip}\pstart
           \noindent{}\centering{}\textcolor{gray}{\textbf{{\pb}Konstanz Conciliumsgebäude\oindex{Konzilgebaeude@\textbf{Konzilgebäude}|pw}}}\pend
           \pstart
           12/2\pend
           \pstart
           Ich denk viel an morgen Abend.\pend
           \pstart
           Das Beſte an Deine Frau\pwindex{Schnitzler, Olga 17.01.1882 – 13.01.1970@\textsc{Schnitzler, Olga} (17.01.1882 – 13.01.1970), \emph{Schauspielerin, Sängerin}|pwv}, an
                  Brahm\pwindex{Brahm, Otto 05.02.1856 – 28.11.1912@\textsc{Brahm, Otto} (05.02.1856 – 28.11.1912), \emph{Theaterleiter, Regisseur}|pw} u. an Rittner\pwindex{Rittner, Rudolf 30.06.1869 – 04.02.1943@\textsc{Rittner, Rudolf} (30.06.1869 – 04.02.1943), \emph{Theaterleiter, Schauspieler}|pw}.\pend
           \pstart Herzlichſt grüßt Dein \spacefill\mbox{Herma{\geminationn}}\pend{}\pstart
           \noindent{}Auf der Flucht nach dem Süden!\pend
           
         
         \endnumbering\mylabel{h}\end{ledgroupsized}  \newcommand{\dateiname}{L01372}\newcommand{\titel}{Hermann Bahr an Arthur Schnitzler, 12. 2. 1904}\newcommand{\editorInnen}{ Kurt Ifkovits,  Martin Anton Müller}%% latex-leseansicht-abspann.tex
%% Abspann für die Leseansicht.
%% Der Schalter \ifkorrekturansicht ist bereits durch den Vorspann gesetzt.

%% latex-abspann.tex
%% Gemeinsamer Abspann für Korrekturansicht und Leseansicht.
%% Setzt den Schalter \ifkorrekturansicht voraus (gesetzt in den
%% einbindenden Dateien latex-korrekturansicht-abspann.tex bzw.
%% latex-leseansicht-abspann.tex).
%% ---------------------------------------------------------------

\normalsize

% Das esempio-Environment wird nur in der Leseansicht benötigt
\ifkorrekturansicht\else
\newenvironment{esempio}[3]%
{
    \vspace{1.5ex}
    \rlap{\underline{#1}}
    \par
    \setlength{\parindent}{0cm}
    \nopagebreak
    \leftskip=#2cm
    \rightskip=#3cm
}
{
    \par
}
\fi

\doendnotes{C}
\bigskip
\vfill

\clearpage

\footnotesize

\ifkorrekturansicht
  \lohead{\textsc{register}}
\fi

% theindex-Environment neu definieren ohne reledmac
\makeatletter
\renewenvironment{theindex}{%
  \ifkorrekturansicht
    \section*{\indexname}%
  \else
    \subsubsection*{Index der erwähnten Entitäten}%
  \fi
  \setlength{\parindent}{0pt}%
  \setlength{\parskip}{0pt plus 0.3pt}%
  \let\item\@idxitem
}{%
  \ifkorrekturansicht\clearpage\fi
}
\makeatother

\IfFileExists{\jobname-pw.ind}{\input{\jobname-pw.ind}}{}

% Quellenangabe nur in der Leseansicht
\ifkorrekturansicht\else
% Fallback-Definitionen, falls die .tex-Datei \titel etc. nicht gesetzt hat
\providecommand{\titel}{}
\providecommand{\editorInnen}{}
\providecommand{\dateiname}{\jobname}

\vspace{3cm}

\vfill

\footnotesize
\textsc{Quelle}: \titel. Herausgegeben von {\editorInnen}. In: \emph{Arthur Schnitzler: Briefwechsel mit Autorinnen und Autoren}.
 Digitale Edition, https://schnitzler-briefe.acdh.oeaw.ac.at/{\dateiname}.html (Stand \today)
\fi

\end{document}


      