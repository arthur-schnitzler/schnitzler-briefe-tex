%% latex-leseansicht-vorspann.tex
%% Vorspann für die Leseansicht.
%% Lädt die gemeinsame Datei latex-vorspann.tex mit nicht gesetztem Schalter.

\newif\ifkorrekturansicht
\korrekturansichtfalse

\input{../tex-inputs/latex-vorspann}


\section[Arthur Schnitzler an Richard Beer-Hofmann, 13. 5. 1912]{L02063 Arthur Schnitzler an Richard Beer-Hofmann, 13. 5. 1912}
\nopagebreak\mylabel{L02063v}
\rehead{ }\normalsize\beginnumbering\briefempfaengerindex{Beer-Hofmann, Richard@\textsc{Beer-Hofmann, Richard}!zzzSchnitzler, Arthur@\emph{von Arthur Schnitzler}!1912-05-131@{13. 5. 1912}|(be}
\toendnotes[C]{\smallbreak\pagebreak[2]}
\correspDesc{Versand  durch Arthur Schnitzler am 13. 5. 1912 in Brijuni
\newline{}Erhalt  durch Richard Beer-Hofmann im Zeitraum [14. 5. 1912
                  – 18. 5. 1912?] in Venedig}\toendnotes[C]{\smallbreak}
\Standort{YCGL, MSS 31.}
\physDesc{Bildpostkarte, 302 Zeichen
\newline{}Handschrift: Bleistift, deutsche Kurrent
\newline{}Versand: 1) Stempel: »\nobreak{}\oindex{Brijuni@\textbf{Brijuni}|pwk}Insel Brioni i. d. Adria, Alle Rechte vorbehalten.\nobreak{}«.   2) Stempel: »\nobreak{}\oindex{Brijuni@\textbf{Brijuni}|pwk}Brioni, 13. 5. 12\nobreak{}«. }\toendnotes[C]{\smallbreak}\pstart{}{\pb}\textsc{Dr. Richard BeerHofmann aus Wien\oindex{Wien@\textbf{Wien}, \emph{Verwaltungsgebiet}|pw}}\pend{}\pstart{}\textsc{Hotel Bauer u Grünwald\oindex{Grand Hotel Bauer-Grünwald@\textbf{Grand Hotel Bauer-Grünwald}, \emph{Hotel}|pw}}\pend{}\pstart{}\textsc{Venedig\oindex{Venedig@\textbf{Venedig}|pw}}\pend{}\pstart{}nachſenden: \textsc{(nach Wien
                        XVIII Hasenauerstr 59\oindex{Wien@\textbf{Wien}!XVIII., Währing@\textbf{XVIII., Währing}!Hasenauerstraße 59@\textbf{Hasenauerstraße 59}, \emph{Wohngebäude}|pw})}\pend{}{\bigskip}
\pstart
           \noindent{}\centering{}{\pb}\textcolor{gray}{\textbf{{[}Römische Ausgrabungen{]}}}\pend
           \vspace{1em}
\pstart
           \centering{}{\pb}13. 5. 912.\pend
           \vspace{0.5em}
\pstart
           {\pb}Brioni\oindex{Brijuni@\textbf{Brijuni}|pw} iſt \uline{bezaubernd}. Hotel vorzüglich. Wir haben gemiethet. Rathen Ihnen von Herzen
               das gleiche! Meer, Wald, Wieſen, Vergangenheit.\pend
           \pstart Herzlichſt Ihr \spacefill\mbox{A.}\pend{}
\pstart
           \noindent{}Hoffen \uline{\label{K_L02063-1v}\edtext{Mittwoch}{\lemma{\textnormal{\emph{Mittwoch}}}\Cendnote{\textnormal{Siehe A. S.: \emph{Tagebuch}, 15. 5. 1912.
                     }}}\label{K_L02063-1} Abend} in Vened\oindex{Venedig@\textbf{Venedig}|pw}{ }\label{T_L02063-1v}\edtext{\textsc{Europe}\oindex{Hotel de l’Europe [Venedig]@\textbf{Hotel de l’Europe [Venedig]}, \emph{Hotel}|pw} zu{ }ſein.}{\lemma{\textnormal{\emph{Europe zu sein.}}}\Cendnote{\textnormal{quer zum Text entlang
                     des oberen Kartenrandes}}}\label{T_L02063-1}\pend
           \selectlanguage{ngerman}\endnumbering\briefempfaengerindex{Beer-Hofmann, Richard@\textsc{Beer-Hofmann, Richard}!zzzSchnitzler, Arthur@\emph{von Arthur Schnitzler}!1912-05-131@{13. 5. 1912}|)be}\mylabel{L02063h}  \newcommand{\dateiname}{L02063}\newcommand{\titel}{Arthur Schnitzler an Richard Beer-Hofmann, 13. 5. 1912}\newcommand{\editorInnen}{Martin Anton Müller und Gerd-Hermann Susen}%% latex-leseansicht-abspann.tex
%% Abspann für die Leseansicht.
%% Der Schalter \ifkorrekturansicht ist bereits durch den Vorspann gesetzt.

%% latex-abspann.tex
%% Gemeinsamer Abspann für Korrekturansicht und Leseansicht.
%% Setzt den Schalter \ifkorrekturansicht voraus (gesetzt in den
%% einbindenden Dateien latex-korrekturansicht-abspann.tex bzw.
%% latex-leseansicht-abspann.tex).
%% ---------------------------------------------------------------

\normalsize

% Das esempio-Environment wird nur in der Leseansicht benötigt
\ifkorrekturansicht\else
\newenvironment{esempio}[3]%
{
    \vspace{1.5ex}
    \rlap{\underline{#1}}
    \par
    \setlength{\parindent}{0cm}
    \nopagebreak
    \leftskip=#2cm
    \rightskip=#3cm
}
{
    \par
}
\fi

\doendnotes{C}
\bigskip
\vfill

\clearpage

\footnotesize

\ifkorrekturansicht
  \lohead{\textsc{register}}
\fi

% theindex-Environment neu definieren ohne reledmac
\makeatletter
\renewenvironment{theindex}{%
  \ifkorrekturansicht
    \section*{\indexname}%
  \else
    \subsubsection*{Index der erwähnten Entitäten}%
  \fi
  \setlength{\parindent}{0pt}%
  \setlength{\parskip}{0pt plus 0.3pt}%
  \let\item\@idxitem
}{%
  \ifkorrekturansicht\clearpage\fi
}
\makeatother

\IfFileExists{\jobname-pw.ind}{\input{\jobname-pw.ind}}{}

% Quellenangabe nur in der Leseansicht
\ifkorrekturansicht\else
% Fallback-Definitionen, falls die .tex-Datei \titel etc. nicht gesetzt hat
\providecommand{\titel}{}
\providecommand{\editorInnen}{}
\providecommand{\dateiname}{\jobname}

\vspace{3cm}

\vfill

\footnotesize
\textsc{Quelle}: \titel. Herausgegeben von {\editorInnen}. In: \emph{Arthur Schnitzler: Briefwechsel mit Autorinnen und Autoren}.
 Digitale Edition, https://schnitzler-briefe.acdh.oeaw.ac.at/{\dateiname}.html (Stand \today)
\fi

\end{document}


