%% latex-leseansicht-vorspann.tex
%% Vorspann für die Leseansicht.
%% Lädt die gemeinsame Datei latex-vorspann.tex mit nicht gesetztem Schalter.

\newif\ifkorrekturansicht
\korrekturansichtfalse

\input{../tex-inputs/latex-vorspann}


\section[Elsa Plessner an Arthur Schnitzler, 23. 12. 1896]{L03709 Elsa Plessner an Arthur Schnitzler, 23. 12. 1896}
\nopagebreak\mylabel{L03709v}
\rehead{ }\normalsize\beginnumbering\briefempfaengerindex{Schnitzler, Arthur@\textsc{Schnitzler, Arthur}!zzzPlessner, Elsa@\emph{von Elsa Plessner}!1896-12-232@{23. 12. 1896}|(be}
\toendnotes[C]{\smallbreak\pagebreak[2]}
\correspDesc{Versand  durch Elsa Plessner am 23. 12. 1896 in Meran
\newline{}Erhalt  durch Arthur Schnitzler im Zeitraum [24. 12. 1896 – 28. 12. 1896?] in Wien}\toendnotes[C]{\smallbreak}
\Standort{DLA, A:Schnitzler, HS.1985.1.419.}
\physDesc{Brief, 1 Blatt, 3 Seiten, 2173 Zeichen
\newline{}Handschrift: schwarze Tinte, lateinische Kurrent
\newline{}Schnitzler: mit rotem Buntstift eine Unterstreichung }\toendnotes[C]{\smallbreak}
\pstart
           \raggedleft{}{\pb}Meran, Pension Wolf\oindex{Hotel Meranerhof@\textbf{Hotel Meranerhof}, \emph{Hotel}|pw}, den 23. Dez. 1896.\pend
           
\pstart
           \raggedleft{}½ 12 Uhr Nachts \pend
           
\pstart\center{}Verehrter Herr Doctor!! – –\pend\vspace{0.5em}
\pstart
           Hallelujah!! – Mit demselben Tintentropfen, mit welchem ich das Wort »Ende« unter
               mein neues Stück\pwindex{Plessner, Elsa 22.\,8.\,1875 Wien – 7.\,5.\,1932 Alicante@\textsc{Plessner, Elsa} (22.\,8.\,1875 Wien – 7.\,5.\,1932 Alicante), \emph{Schriftstellerin}!Orchideen [Schauspiel in drei Akten]@\strich\emph{Orchideen [Schauspiel in drei Akten]}|pwv}{ }\uline{soeben} gesetzt habe – erhalten Sie diese Zeilen
               geschmiert – was Sie mir mit Rücksicht auf diese, Ihnen bekannte Stimmung verzeihen
               werden –. (Einen Styl –{ }\introOben{}was?\introOben{}– !!?) Aber das macht nichts!! – Ich freue mich – denn
                  »Orchideen\pwindex{Plessner, Elsa 22.\,8.\,1875 Wien – 7.\,5.\,1932 Alicante@\textsc{Plessner, Elsa} (22.\,8.\,1875 Wien – 7.\,5.\,1932 Alicante), \emph{Schriftstellerin}!Orchideen [Schauspiel in drei Akten]@\strich\emph{Orchideen [Schauspiel in drei Akten]}|pw}« Schauspiel in 3 Acten, ist mir
               gelungen – oder ich heiße Eugenie Marlitt\pwindex{Marlitt, E. 5.\,12.\,1825 Arnstadt – 22.\,6.\,1887 ebd.@\textsc{Marlitt, E.} (5.\,12.\,1825 Arnstadt – 22.\,6.\,1887 ebd.), \emph{Schriftstellerin, Sängerin}|pw}!! –
               Sie erhalten es, sobald Feile und Abschrift {\pb}hinter mir,
               zur gütigen Durchsicht! – Es ist ein unerbittliches Stück\pwindex{Plessner, Elsa 22.\,8.\,1875 Wien – 7.\,5.\,1932 Alicante@\textsc{Plessner, Elsa} (22.\,8.\,1875 Wien – 7.\,5.\,1932 Alicante), \emph{Schriftstellerin}!Orchideen [Schauspiel in drei Akten]@\strich\emph{Orchideen [Schauspiel in drei Akten]}|pwv}, von dramatischer Wucht (das ist Thatsache – lachen
               Sie nicht – bitte) und wie ich glaube \uline{echter} Tragik!
               – Thatsache – blos – ich habe \uuline{alles} zusammengekratzt,
               was ich an Können und künstlerischem Wollen besitze – und auch die negativen
               Erfahrungen des »Heimweh\pwindex{Plessner, Elsa 22.\,8.\,1875 Wien – 7.\,5.\,1932 Alicante@\textsc{Plessner, Elsa} (22.\,8.\,1875 Wien – 7.\,5.\,1932 Alicante), \emph{Schriftstellerin}!Heimweh [dreiaktige Tragikomödie]@\strich\emph{Heimweh [dreiaktige Tragikomödie]}|pw}« haben mir genützt –
               und mein zweites Stück\pwindex{Plessner, Elsa 22.\,8.\,1875 Wien – 7.\,5.\,1932 Alicante@\textsc{Plessner, Elsa} (22.\,8.\,1875 Wien – 7.\,5.\,1932 Alicante), \emph{Schriftstellerin}!Orchideen [Schauspiel in drei Akten]@\strich\emph{Orchideen [Schauspiel in drei Akten]}|pwv}, fast
                  \uline{2 Jahre} nach dem ersten\pwindex{Plessner, Elsa 22.\,8.\,1875 Wien – 7.\,5.\,1932 Alicante@\textsc{Plessner, Elsa} (22.\,8.\,1875 Wien – 7.\,5.\,1932 Alicante), \emph{Schriftstellerin}!Heimweh [dreiaktige Tragikomödie]@\strich\emph{Heimweh [dreiaktige Tragikomödie]}|pwv} entstanden{[},{]}{ }\uline{muß} aufführbar sein – sonst kann ich die Kratzerei an
               den Nagel hängen!! – Wenn Alles was ich besitze nicht genug ist – – – ! – Tausend
               herzlichen Dank für \label{K_L03709-1v}\edtext{Ihre reizenden
                  Zeilen}{\lemma{\textnormal{\emph{Ihre reizenden
                  Zeilen}}}\Cendnote{\textnormal{nicht überliefert}}}\label{K_L03709-1}, die
               mir mitten in meiner Arbeit ein lieber, anfeuernder Gruß {\pb}erschienen! – – – Das Scenarium und die Disposition habe – 5 mal geschmissen und
               von Grund wieder aufgebaut – na – wie steh ich da? – Freilich – wenn es Glück haben
               sollte – und warum soll eine blinde Henne wie ich, nicht einmal ein Körnchen finden –
               würde das Publikum, sagen »Arche (arge) Ideen« hat E. P. – (»Witze thu ich auch
               machen«!!) – – Aber gearbeitet habe ich – wie ein Holzknecht!! – Auch \label{K_L03709-2v}\edtext{\begin{otherlanguage}{french}à la\end{otherlanguage}{ }Penelope\pwindex{\textcolor{red}{\textsuperscript{XXXX indx1}}!Odyssee@\strich\emph{Odyssee}|pw}}{\lemma{\textnormal{\emph{à la Penelope}}}\Cendnote{\textnormal{Während Penelope im Epos der \emph{Odyssee}\pwindex{\textcolor{red}{\textsuperscript{XXXX indx1}}!Odyssee@\strich\emph{Odyssee}|pwk} auf die Rückkehr ihres Gatten Odysseus
                  von Kriegs- und Irrfahrten wartete, trennte sie nachts das Tuch auf, das sie
                  tagsüber webte, um die Freier hinzuhalten, die sie zu einer neuen Hochzeit drängen
                  wollten.}}}\label{K_L03709-2}, denn sehr oft Morgens verbrannt, was Abends geschrieben!! – Wenn
               das meine Ärzte wüssten, die meine »Nerven« nach Meran\oindex{Meran@\textbf{Meran}, \emph{Hauptstadt}|pw} geschickt haben – – 
               \label{K_L03709-3v}\edtext{\begin{otherlanguage}{french}Entre nous\end{otherlanguage}}{\lemma{\textnormal{\emph{Entre nous}}}\Cendnote{\textnormal{französisch: unter uns}}}\label{K_L03709-3}! – Besser sind freilich die hohen Herschaften dadurch nicht geworden – –
               Aber dafür hole ich es jetzt nach und lege mir ein paar Kurkilogramme
               zu! – Aber der Schnee! – Und \uline{die}!! – Hundekälte –!
               Auf meinem Südbalcon kann ich Schlittschuh laufen!! – – – – \begin{otherlanguage}{english}\label{K_L03709-4v}\edtext{Merry Christmas and new years (100)
                  and all the holidays}{\lemma{\textnormal{\emph{Merry … holidays}}}\Cendnote{\textnormal{englisch: frohe
                     Weihnachten und neue Jahre (100) und all die Ferien}}}\label{K_L03709-4}{ }\end{otherlanguage}!!! – \label{K_L03709-5v}\edtext{Gratulire »Freiwild\pwindex{Schnitzler, Arthur 15.\,5.\,1862 Wien – 21.\,10.\,1931 ebd.@\textsc{Schnitzler, Arthur} (15.\,5.\,1862 Wien – 21.\,10.\,1931 ebd.), \emph{Schriftsteller, Mediziner}!Freiwild. Schauspiel in 3 Akten@\strich\emph{Freiwild. Schauspiel in 3 Akten}|pw}\eventindex{Deutsches Theater Berlin@\textbf{Deutsches Theater Berlin}!Uraufführung von Freiwild, 3.11.1896@Uraufführung von Freiwild, 3.11.1896|pwv}}{\lemma{\textnormal{\emph{Gratulire »Freiwild}}}\Cendnote{\textnormal{Auch Schnitzler verbucht die Berliner\oindex{Berlin@\textbf{Berlin}, \emph{Hauptstadt}|pwk}{ }Erstaufführung von \emph{Freiwild}\pwindex{Schnitzler, Arthur 15.\,5.\,1862 Wien – 21.\,10.\,1931 ebd.@\textsc{Schnitzler, Arthur} (15.\,5.\,1862 Wien – 21.\,10.\,1931 ebd.), \emph{Schriftsteller, Mediziner}!Freiwild. Schauspiel in 3 Akten@\strich\emph{Freiwild. Schauspiel in 3 Akten}|pwk}\eventindex{Deutsches Theater Berlin@\textbf{Deutsches Theater Berlin}!Uraufführung von Freiwild, 3.11.1896@Uraufführung von Freiwild, 3.11.1896|pwk} als Erfolg, vgl. A. S.: \emph{Tagebuch}, 3. 11. 1896.}}}\label{K_L03709-5}« – \label{K_L03709-6v}\edtext{Breslau\oindex{Breslau@\textbf{Breslau}|pw}}{\lemma{\textnormal{\emph{Breslau}}}\Cendnote{\textnormal{Schnitzler war am 26. 10. 1896 über Breslau\oindex{Breslau@\textbf{Breslau}|pwk} nach Berlin\oindex{Berlin@\textbf{Berlin}, \emph{Hauptstadt}|pwk} gefahren.}}}\label{K_L03709-6}. Fräulein \label{K_L03709-7v}\edtext{Jurberg\pwindex{Jurberg, Gisela 15.\,7.\,1874 Wien – 20.\,7.\,1942 Hannover@\textsc{Jurberg, Gisela} (15.\,7.\,1874 Wien – 20.\,7.\,1942 Hannover), \emph{Schauspielerin}|pw}}{\lemma{\textnormal{\emph{Jurberg}}}\Cendnote{\textnormal{Die aus Wien\oindex{Wien@\textbf{Wien}, \emph{Verwaltungsgebiet}|pwk} stammende Gisela Jurberg\pwindex{Jurberg, Gisela 15.\,7.\,1874 Wien – 20.\,7.\,1942 Hannover@\textsc{Jurberg, Gisela} (15.\,7.\,1874 Wien – 20.\,7.\,1942 Hannover), \emph{Schauspielerin}|pwk}
                  spielte am \emph{Lobe-Theater}\orgindex{Lobe-Theater@Lobe-Theater|pwk} in Breslau\oindex{Breslau@\textbf{Breslau}|pwk} die Hauptrolle in \emph{Liebelei}\pwindex{Schnitzler, Arthur 15.\,5.\,1862 Wien – 21.\,10.\,1931 ebd.@\textsc{Schnitzler, Arthur} (15.\,5.\,1862 Wien – 21.\,10.\,1931 ebd.), \emph{Schriftsteller, Mediziner}!Liebelei. Schauspiel in drei Akten@\strich\emph{Liebelei. Schauspiel in drei Akten}|pwk} (Premiere\eventindex{Lobe-Theater@\textbf{Lobe-Theater}!Premiere von Liebelei, 11.2.1896@Premiere von Liebelei, 11.2.1896|pwk} am 11. 2. 1896). Schnitzler sah die Aufführung nicht.}}}\label{K_L03709-7} gesehen? – »\label{K_L03709-8v}\edtext{Süsses Mädel}{\lemma{\textnormal{\emph{Süsses Mädel}}}\Cendnote{\textnormal{Eine Wortprägung, die auf Schnitzler zurückgeht und die junge Frauen aus einfachen Verhältnissen
                     bezeichnet, die von wohlsituierten Männern sexuell begehrt werden, aber niemals für eine Ehe in Betracht gezogen würden.}}}\label{K_L03709-8}«\pend
           
\pstart
           Hochachtungsvolle Grüße{\\[\baselineskip]}your{\\[\baselineskip]}\spacefill\mbox{Elsa Plessner}\pend
           \leftskip=0em{}
\pstart
           \noindent{}\raggedleft{}(\begin{otherlanguage}{english}\label{K_L03709-9v}\edtext{a little foolish}{\lemma{\textnormal{\emph{a little foolish}}}\Cendnote{\textnormal{englisch: ein bisschen töricht}}}\label{K_L03709-9}\end{otherlanguage})\pend
           \selectlanguage{ngerman}\endnumbering\briefempfaengerindex{Schnitzler, Arthur@\textsc{Schnitzler, Arthur}!zzzPlessner, Elsa@\emph{von Elsa Plessner}!1896-12-232@{23. 12. 1896}|)be}\mylabel{L03709h}  \newcommand{\dateiname}{L03709}\newcommand{\titel}{Elsa Plessner an Arthur Schnitzler, 23. 12. 1896}\newcommand{\editorInnen}{Selma Jahnke und Martin Anton Müller}%% latex-leseansicht-abspann.tex
%% Abspann für die Leseansicht.
%% Der Schalter \ifkorrekturansicht ist bereits durch den Vorspann gesetzt.

%% latex-abspann.tex
%% Gemeinsamer Abspann für Korrekturansicht und Leseansicht.
%% Setzt den Schalter \ifkorrekturansicht voraus (gesetzt in den
%% einbindenden Dateien latex-korrekturansicht-abspann.tex bzw.
%% latex-leseansicht-abspann.tex).
%% ---------------------------------------------------------------

\normalsize

% Das esempio-Environment wird nur in der Leseansicht benötigt
\ifkorrekturansicht\else
\newenvironment{esempio}[3]%
{
    \vspace{1.5ex}
    \rlap{\underline{#1}}
    \par
    \setlength{\parindent}{0cm}
    \nopagebreak
    \leftskip=#2cm
    \rightskip=#3cm
}
{
    \par
}
\fi

\doendnotes{C}
\bigskip
\vfill

\clearpage

\footnotesize

\ifkorrekturansicht
  \lohead{\textsc{register}}
\fi

% theindex-Environment neu definieren ohne reledmac
\makeatletter
\renewenvironment{theindex}{%
  \ifkorrekturansicht
    \section*{\indexname}%
  \else
    \subsubsection*{Index der erwähnten Entitäten}%
  \fi
  \setlength{\parindent}{0pt}%
  \setlength{\parskip}{0pt plus 0.3pt}%
  \let\item\@idxitem
}{%
  \ifkorrekturansicht\clearpage\fi
}
\makeatother

\IfFileExists{\jobname-pw.ind}{\input{\jobname-pw.ind}}{}

% Quellenangabe nur in der Leseansicht
\ifkorrekturansicht\else
% Fallback-Definitionen, falls die .tex-Datei \titel etc. nicht gesetzt hat
\providecommand{\titel}{}
\providecommand{\editorInnen}{}
\providecommand{\dateiname}{\jobname}

\vspace{3cm}

\vfill

\footnotesize
\textsc{Quelle}: \titel. Herausgegeben von {\editorInnen}. In: \emph{Arthur Schnitzler: Briefwechsel mit Autorinnen und Autoren}.
 Digitale Edition, https://schnitzler-briefe.acdh.oeaw.ac.at/{\dateiname}.html (Stand \today)
\fi

\end{document}


