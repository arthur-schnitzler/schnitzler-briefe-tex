%% latex-korrekturansicht-vorspann.tex
%% Vorspann für die Korrekturansicht.
%% Lädt die gemeinsame Datei latex-vorspann.tex mit gesetztem Schalter.

\newif\ifkorrekturansicht
\korrekturansichttrue

\input{../tex-inputs/latex-vorspann}


\section[ Felix Salten an Arthur Schnitzler, {[}9. 6. 1895{]}]{L03156 Felix Salten an Arthur Schnitzler, {[}9. 6. 1895{]}}
\nopagebreak\mylabel{L03156v}
\rehead{ }\normalsize\beginnumbering\briefempfaengerindex{Schnitzler, Arthur@\textsc{Schnitzler, Arthur}!zzzSalten, Felix@\emph{von Felix Salten}!1895-06-091@{{[}9. 6. 1895{]}}|(be}
\toendnotes[C]{\smallbreak\pagebreak[2]}\Standort{CUL, Schnitzler, B 89, A 1.}
\physDesc{Brief, 1 Blatt, 2 Seiten, 520 Zeichen
\newline{}Handschrift: Bleistift, lateinische Kurrent
\newline{}Schnitzler: mit Bleistift datiert: »9/6 95« 
\newline{}Ordnung: mit Bleistift von unbekannter Hand nummeriert: »55« }\toendnotes[C]{\smallbreak}
\pstart
           \noindent{}{\pb}Lieber Freund, Sie sind nicht böse, dass ich nochmals
               zu Ihnen komme, ehe ich Ihnen das Erste zurückgegeben. Aber ich muss Sie jetzt
               bitten, mir noch einmal mit 10 fl zu helfen. Die 
               Kostfrau\pwindex{?? [Kostfrau von Charlotte Lamberg] @\textsc{?? [Kostfrau von Charlotte Lamberg]}|pwv}
                des \label{K_L03156-1v}\edtext{Kindes\pwindex{Lamberg, Maria Charlotte 1895-03-24 – 1895-07-27@\textsc{Lamberg, Maria Charlotte} (1895-03-24 – 1895-07-27)|pwv}}{\lemma{\textnormal{\emph{Kindes}}}\Cendnote{\textnormal{Maria
                     Charlotte Lamberg\pwindex{Lamberg, Maria Charlotte 1895-03-24 – 1895-07-27@\textsc{Lamberg, Maria Charlotte} (1895-03-24 – 1895-07-27)|pwk}, das am 24. 3. 1895 geborene Kind
                  von Charlotte Glas\pwindex{Pohl-Glas, Charlotte 1873-01-01 – 1944-02-15@\textsc{Pohl-Glas, Charlotte} (1873-01-01 – 1944-02-15), \emph{Schriftsteller/Schriftstellerin, Politiker/Politikerin, Sozialist/Sozialistin}|pwk} und Salten\pwindex{Salten, Felix 06.09.1869 – 08.10.1945@\textsc{Salten, Felix} (06.09.1869 – 08.10.1945), \emph{Schriftsteller/Schriftstellerin, Journalist/Journalistin, Chefredakteur/Chefredakteurin}|pwk}, war bei einer nicht namentlich bekannten Kostfrau\pwindex{?? [Kostfrau von Charlotte Lamberg] @\textsc{?? [Kostfrau von Charlotte Lamberg]}|pwkv} in Gerasdorf\oindex{Gerasdorf bei Wien@\textbf{Gerasdorf bei Wien}, \emph{A.ADM3}|pwk} nördlich von Wien\oindex{Wien@\textbf{Wien}, \emph{A.ADM2}|pwk} untergebracht. Die Sterblichkeitsrate unter solchen zur
                  Pflege aufs Land gegebenen Kindern war sehr hoch; auch Maria Charlotte\pwindex{Lamberg, Maria Charlotte 1895-03-24 – 1895-07-27@\textsc{Lamberg, Maria Charlotte} (1895-03-24 – 1895-07-27)|pwk} starb im Alter von vier Monaten am 27. 7. 1895.}}}\label{K_L03156-1} ist vom Land hereingekommen: Das
                  \uline{K.\pwindex{Lamberg, Maria Charlotte 1895-03-24 – 1895-07-27@\textsc{Lamberg, Maria Charlotte} (1895-03-24 – 1895-07-27)|pwv}} sei krank und sie brauche das Geld für das und für jenes. Ich kann sie nicht
               fortschicken ohne G. Bitte, senden Sie mir noch einmal 10 fl, ich {\pb}werde Ihnen diese \uline{20 fl.} bis Dienstag{ }Vormittag{ }\uuline{ganz positiv} zurückgeben. Sie können sich vollständig
               darauf verlaßen. Ich danke Ihnen \pend
           
\pstart
           herzlich {\\[\baselineskip]}Ihr {\\[\baselineskip]}\spacefill\mbox{Salten}\pend
           \leftskip=0em{}\selectlanguage{ngerman}\endnumbering\briefempfaengerindex{Schnitzler, Arthur@\textsc{Schnitzler, Arthur}!zzzSalten, Felix@\emph{von Felix Salten}!1895-06-091@{{[}9. 6. 1895{]}}|)be}\mylabel{L03156h}  \normalsize

\doendnotes{C}
\bigskip
\vfill

\clearpage

\footnotesize

\lohead{\textsc{register}}

% Definiere theindex-Environment komplett neu ohne reledmac
\makeatletter
\renewenvironment{theindex}{%
  \section*{\indexname}%
  \setlength{\parindent}{0pt}%
  \setlength{\parskip}{0pt plus 0.3pt}%
  \let\item\@idxitem
}{%
  \clearpage
}
\makeatother

\IfFileExists{\jobname-pw.ind}{\input{\jobname-pw.ind}}{}

\end{document}

      