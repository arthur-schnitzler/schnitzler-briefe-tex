%% latex-leseansicht-vorspann.tex
%% Vorspann für die Leseansicht.
%% Lädt die gemeinsame Datei latex-vorspann.tex mit nicht gesetztem Schalter.

\newif\ifkorrekturansicht
\korrekturansichtfalse

\input{../tex-inputs/latex-vorspann}


\section[ Felix Salten an Arthur Schnitzler, {[}9. 6. 1895{]}]{L03156 Felix Salten an Arthur Schnitzler,  [9. 6. 1895]}
\nopagebreak\mylabel{L03156v}
\rehead{ }\normalsize\beginnumbering\briefempfaengerindex{Schnitzler, Arthur@\textsc{Schnitzler, Arthur}!zzzSalten, Felix@\emph{von Felix Salten}!1895-06-091@{{[}9. 6. 1895{]}}|(be}
\toendnotes[C]{\smallbreak\pagebreak[2]}
\correspDesc{Versand  durch Felix Salten am [9. 6. 1895] in Wien
\newline{}Erhalt  durch Arthur Schnitzler am [9. 6. 1895] in Wien}\toendnotes[C]{\smallbreak}
\Standort{CUL, Schnitzler, B 89, A 1.}
\physDesc{Brief, 1 Blatt, 2 Seiten, 520 Zeichen
\newline{}Handschrift: Bleistift, lateinische Kurrent
\newline{}Schnitzler: mit Bleistift datiert: »9/6 95« 
\newline{}Ordnung: mit Bleistift von unbekannter Hand nummeriert: »55« }\toendnotes[C]{\smallbreak}
\pstart
           \noindent{}{\pb}Lieber Freund, Sie sind nicht böse, dass ich nochmals
               zu Ihnen komme, ehe ich Ihnen das Erste zurückgegeben. Aber ich muss Sie jetzt
               bitten, mir noch einmal mit 10 fl zu helfen. Die 
               Kostfrau\pwindex{?? [Kostfrau von Charlotte Lamberg] @\textsc{?? [Kostfrau von Charlotte Lamberg]}|pwv}
                des \label{K_L03156-1v}\edtext{Kindes\pwindex{Lamberg, Maria Charlotte 24.\,3.\,1895 Wien – 27.\,7.\,1895 Gerasdorf bei Wien@\textsc{Lamberg, Maria Charlotte} (24.\,3.\,1895 Wien – 27.\,7.\,1895 Gerasdorf bei Wien)|pwv}}{\lemma{\textnormal{\emph{Kindes}}}\Cendnote{\textnormal{Maria
                     Charlotte Lamberg\pwindex{Lamberg, Maria Charlotte 24.\,3.\,1895 Wien – 27.\,7.\,1895 Gerasdorf bei Wien@\textsc{Lamberg, Maria Charlotte} (24.\,3.\,1895 Wien – 27.\,7.\,1895 Gerasdorf bei Wien)|pwk}, das am 24. 3. 1895 geborene Kind
                  von Charlotte Glas\pwindex{Pohl-Glas, Charlotte 1.\,1.\,1873 Wien – 15.\,2.\,1944 Zürich@\textsc{Pohl-Glas, Charlotte} (1.\,1.\,1873 Wien – 15.\,2.\,1944 Zürich), \emph{Schriftstellerin, Politikerin, Sozialistin}|pwk} und Salten\pwindex{Salten, Felix 6.\,9.\,1869 Budapest – 8.\,10.\,1945 Zürich@\textsc{Salten, Felix} (6.\,9.\,1869 Budapest – 8.\,10.\,1945 Zürich), \emph{Schriftsteller, Journalist, Chefredakteur}|pwk}, war bei einer nicht namentlich bekannten Kostfrau\pwindex{?? [Kostfrau von Charlotte Lamberg] @\textsc{?? [Kostfrau von Charlotte Lamberg]}|pwkv} in Gerasdorf\oindex{Gerasdorf bei Wien@\textbf{Gerasdorf bei Wien}, \emph{Verwaltungsgebiet}|pwk} nördlich von Wien\oindex{Wien@\textbf{Wien}, \emph{Verwaltungsgebiet}|pwk} untergebracht. Die Sterblichkeitsrate unter solchen zur
                  Pflege aufs Land gegebenen Kindern war sehr hoch; auch Maria Charlotte\pwindex{Lamberg, Maria Charlotte 24.\,3.\,1895 Wien – 27.\,7.\,1895 Gerasdorf bei Wien@\textsc{Lamberg, Maria Charlotte} (24.\,3.\,1895 Wien – 27.\,7.\,1895 Gerasdorf bei Wien)|pwk} starb im Alter von vier Monaten am 27. 7. 1895.}}}\label{K_L03156-1} ist vom Land hereingekommen: Das
                  \uline{K.\pwindex{Lamberg, Maria Charlotte 24.\,3.\,1895 Wien – 27.\,7.\,1895 Gerasdorf bei Wien@\textsc{Lamberg, Maria Charlotte} (24.\,3.\,1895 Wien – 27.\,7.\,1895 Gerasdorf bei Wien)|pwv}} sei krank und sie brauche das Geld für das und für jenes. Ich kann sie nicht
               fortschicken ohne G. Bitte, senden Sie mir noch einmal 10 fl, ich {\pb}werde Ihnen diese \uline{20 fl.} bis Dienstag{ }Vormittag{ }\uuline{ganz positiv} zurückgeben. Sie können sich vollständig
               darauf verlaßen. Ich danke Ihnen\pend
           
\pstart
           herzlich {\\[\baselineskip]}Ihr {\\[\baselineskip]}\spacefill\mbox{Salten}\pend
           \leftskip=0em{}\selectlanguage{ngerman}\endnumbering\briefempfaengerindex{Schnitzler, Arthur@\textsc{Schnitzler, Arthur}!zzzSalten, Felix@\emph{von Felix Salten}!1895-06-091@{{[}9. 6. 1895{]}}|)be}\mylabel{L03156h}  \newcommand{\dateiname}{L03156}\newcommand{\titel}{Felix Salten an Arthur Schnitzler, [9. 6. 1895]}\newcommand{\editorInnen}{Martin Anton Müller und Laura Untner}%% latex-leseansicht-abspann.tex
%% Abspann für die Leseansicht.
%% Der Schalter \ifkorrekturansicht ist bereits durch den Vorspann gesetzt.

%% latex-abspann.tex
%% Gemeinsamer Abspann für Korrekturansicht und Leseansicht.
%% Setzt den Schalter \ifkorrekturansicht voraus (gesetzt in den
%% einbindenden Dateien latex-korrekturansicht-abspann.tex bzw.
%% latex-leseansicht-abspann.tex).
%% ---------------------------------------------------------------

\normalsize

% Das esempio-Environment wird nur in der Leseansicht benötigt
\ifkorrekturansicht\else
\newenvironment{esempio}[3]%
{
    \vspace{1.5ex}
    \rlap{\underline{#1}}
    \par
    \setlength{\parindent}{0cm}
    \nopagebreak
    \leftskip=#2cm
    \rightskip=#3cm
}
{
    \par
}
\fi

\doendnotes{C}
\bigskip
\vfill

\clearpage

\footnotesize

\ifkorrekturansicht
  \lohead{\textsc{register}}
\fi

% theindex-Environment neu definieren ohne reledmac
\makeatletter
\renewenvironment{theindex}{%
  \ifkorrekturansicht
    \section*{\indexname}%
  \else
    \subsubsection*{Index der erwähnten Entitäten}%
  \fi
  \setlength{\parindent}{0pt}%
  \setlength{\parskip}{0pt plus 0.3pt}%
  \let\item\@idxitem
}{%
  \ifkorrekturansicht\clearpage\fi
}
\makeatother

\IfFileExists{\jobname-pw.ind}{\input{\jobname-pw.ind}}{}

% Quellenangabe nur in der Leseansicht
\ifkorrekturansicht\else
% Fallback-Definitionen, falls die .tex-Datei \titel etc. nicht gesetzt hat
\providecommand{\titel}{}
\providecommand{\editorInnen}{}
\providecommand{\dateiname}{\jobname}

\vspace{3cm}

\vfill

\footnotesize
\textsc{Quelle}: \titel. Herausgegeben von {\editorInnen}. In: \emph{Arthur Schnitzler: Briefwechsel mit Autorinnen und Autoren}.
 Digitale Edition, https://schnitzler-briefe.acdh.oeaw.ac.at/{\dateiname}.html (Stand \today)
\fi

\end{document}


