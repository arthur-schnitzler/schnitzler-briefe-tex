%% latex-leseansicht-vorspann.tex
%% Vorspann für die Leseansicht.
%% Lädt die gemeinsame Datei latex-vorspann.tex mit nicht gesetztem Schalter.

\newif\ifkorrekturansicht
\korrekturansichtfalse

\input{../tex-inputs/latex-vorspann}


\section[Arno Holz an Arthur Schnitzler, 26. 4. 1913]{L02134 Arno Holz an Arthur Schnitzler, 26. 4. 1913}
\nopagebreak\mylabel{L02134v}
\rehead{ }\normalsize\beginnumbering\briefempfaengerindex{Schnitzler, Arthur@\textsc{Schnitzler, Arthur}!zzzHolz, Arno@\emph{von Arno Holz}!1913-04-262@{26. 4. 1913}|(be}
\toendnotes[C]{\smallbreak\pagebreak[2]}
\correspDesc{Versand  durch Arno Holz am 26. 4. 1913 in Berlin
\newline{}Erhalt  durch Arthur Schnitzler im Zeitraum [27. 4. 1913
                  – 1. 5. 1913?] in Wien}\toendnotes[C]{\smallbreak}
\Standort{CUL, Schnitzler, B 44.}
\physDesc{Brief, 1 Blatt, 2 Seiten, 71 Zeichen (Faltblatt mit einem Porträt von Holz und einer faksimilierten
                                 Unterschrift links, rechts einem gedruckten Aufruf )
\newline{}Handschrift: schwarze Tinte, lateinische Kurrent}
\pstart
           \noindent{}{\pb}Herrn\pend
           
\pstart
           \uline{Dr.}{ }\uline{Arthur}{ }\uline{Schnitzler}\pend
           
\pstart
           mit aufrichtigſt herzlichem Dank\pend
           
\pstart
           \centering{}\textcolor{gray}{\textbf{ArnoHolz}}\pend
           
\pstart
           \centering{}26. IV. 1913.\pend
           
\pstart
           \centering{}{\pb}\textcolor{gray}{\textbf{Aufruf für Arno Holz!}}\pend
           
\pstart
           \textcolor{gray}{\textbf{Arno Holz, der am 26. d. Mts. 50 Jahre alt wird,
                  erzählt im Vorwort zu seiner letzten, eben erschienenen Tragödie »Ignorabimus\pwindex{Holz, Arno 26.\,4.\,1863 Kętrzyn – 26.\,10.\,1929 Berlin@\textsc{Holz, Arno} (26.\,4.\,1863 Kętrzyn – 26.\,10.\,1929 Berlin), \emph{Schriftsteller}!Ignoramibus@\strich\emph{Ignoramibus}|pw}«, dass er »heute noch immer
                  buchstäblich in einer Dachbude hockt«. Es ergreift und beschämt uns, einen um das
                  deutsche Schrifttum der letzten dreissig Jahre im höchsten Grade verdienten, schon
                  um der Reinheit seines nur der Kunst ergebenen Strebens willen
                  bewunderungswürdigen Dichter in Not zu wissen. Wir fordern die Nation auf, diese
                  Schuld durch eine Ehrenspende zu tilgen, und eröffnen hiermit die Sammlung.
                  Beiträge wolle man an die Leitung des »Kunstwarts\orgindex{Kunstwart@Der Kunstwart|pw}«, Dresden Blasewitz\oindex{Blasewitz@\textbf{Blasewitz}, \emph{Ehemaliger Ort}|pw},
                  richten.}}\pend
           
\pstart
           \centering{}\textcolor{gray}{\textbf{\textbf{Hermann Bahr\pwindex{Bahr, Hermann 19.\,7.\,1863 Linz – 15.\,1.\,1934 München@\textsc{Bahr, Hermann} (19.\,7.\,1863 Linz – 15.\,1.\,1934 München), \emph{Schriftsteller, Kritiker}|pw}. Hans Baluschek\pwindex{Baluschek, Hans 9.\,5.\,1870 Breslau – 27.\,9.\,1935 Berlin@\textsc{Baluschek, Hans} (9.\,5.\,1870 Breslau – 27.\,9.\,1935 Berlin), \emph{Schriftsteller, Maler}|pw}. Prof. Peter Behrens\pwindex{Behrens, Peter 14.\,4.\,1868 Hamburg – 27.\,2.\,1940 Berlin@\textsc{Behrens, Peter} (14.\,4.\,1868 Hamburg – 27.\,2.\,1940 Berlin), \emph{Architekt}|pw}.}{ }\textbf{Dr. Georg Brandes\pwindex{Brandes, Georg 4.\,2.\,1842 Kopenhagen – 19.\,2.\,1927 ebd.@\textsc{Brandes, Georg} (4.\,2.\,1842 Kopenhagen – 19.\,2.\,1927 ebd.)|pw}.
                        Prof. Dr. Collin\pwindex{Collin, Josef 2.\,2.\,1864 Mainz – 1942@\textsc{Collin, Josef} (2.\,2.\,1864 Mainz – 1942), \emph{Literaturwissenschaftler}|pw}} (Giessen\oindex{Gießen@\textbf{Gießen}, \emph{Hauptstadt}|pw})\textbf{.}{ }\textbf{Dr. Richard Dehmel\pwindex{Dehmel, Richard 18.\,11.\,1863 Hermsdorf – 8.\,2.\,1920 Blankenese@\textsc{Dehmel, Richard} (18.\,11.\,1863 Hermsdorf – 8.\,2.\,1920 Blankenese), \emph{Schriftsteller, Schriftsteller, Krimiautor}|pw}. Dr. Ludwig Fulda\pwindex{Fulda, Ludwig 15.\,7.\,1862 Frankfurt am Main – 30.\,3.\,1939 Berlin@\textsc{Fulda, Ludwig} (15.\,7.\,1862 Frankfurt am Main – 30.\,3.\,1939 Berlin), \emph{Schriftsteller, Übersetzer}|pw}.}{ }Geh. Hofrat \textbf{Prof. Dr. Ing. h. c. Cornelius Gurlitt\pwindex{Gurlitt, Cornelius 1.\,1.\,1850 Thallwitz – 25.\,3.\,1939 Dresden@\textsc{Gurlitt, Cornelius} (1.\,1.\,1850 Thallwitz – 25.\,3.\,1939 Dresden)|pw}. Maximilian Harden\pwindex{Harden, Maximilian 20.\,10.\,1861 Berlin – 30.\,10.\,1927 Montana@\textsc{Harden, Maximilian} (20.\,10.\,1861 Berlin – 30.\,10.\,1927 Montana), \emph{Schriftsteller, Publizist}|pw}. Dr. Georg Hirth\pwindex{Hirth, Georg 13.\,7.\,1841 Gräfentonna – 28.\,3.\,1916 Tegernsee@\textsc{Hirth, Georg} (13.\,7.\,1841 Gräfentonna – 28.\,3.\,1916 Tegernsee), \emph{Verleger}|pw}.} General-Intendant \textbf{Graf von Hülsen-Haeseler\pwindex{Hülsen-Haeseler, Georg von 15.\,7.\,1858 Berlin – 21.\,6.\,1922 ebd.@\textsc{Hülsen-Haeseler, Georg von} (15.\,7.\,1858 Berlin – 21.\,6.\,1922 ebd.), \emph{Theaterleiter}|pw},} Exz\textbf{.}{ }\textbf{Dr. O. E. Lessing\pwindex{Lessing, Otto Eduard 28.\,9.\,1875 – 1942 Illinois@\textsc{Lessing, Otto Eduard} (28.\,9.\,1875 – 1942 Illinois), \emph{Literaturwissenschaftler}|pw}, Prof.} an der Universität Illinois\oindex{University of Illinois@\textbf{University of Illinois}, \emph{Universität}|pw}\textbf{. Prof. Dr. Alfred Lichtwark\pwindex{Lichtwark, Alfred 14.\,11.\,1852 Hamburg – 13.\,1.\,1914 ebd.@\textsc{Lichtwark, Alfred} (14.\,11.\,1852 Hamburg – 13.\,1.\,1914 ebd.), \emph{Museumsleiter}|pw}. Prof. Dr. h. c. Max Liebermann\pwindex{Liebermann, Max 20.\,7.\,1847 Berlin – 8.\,2.\,1935 ebd.@\textsc{Liebermann, Max} (20.\,7.\,1847 Berlin – 8.\,2.\,1935 ebd.), \emph{Maler, Maler, Maler}|pw}. Dr. Paul Lindau\pwindex{Lindau, Paul 3.\,6.\,1839 Magdeburg – 31.\,1.\,1919 Berlin@\textsc{Lindau, Paul} (3.\,6.\,1839 Magdeburg – 31.\,1.\,1919 Berlin), \emph{Schriftsteller, Kritiker, Theaterleiter}|pw}. Prof. Dr. Ernst Mach\pwindex{Mach, Ernst 18.\,2.\,1838 Tuřany – 19.\,2.\,1916 Vaterstetten@\textsc{Mach, Ernst} (18.\,2.\,1838 Tuřany – 19.\,2.\,1916 Vaterstetten), \emph{Philosoph, Philosophiehistoriker, Physiker}|pw}} (Wien\oindex{Wien@\textbf{Wien}, \emph{Verwaltungsgebiet}|pw})\textbf{. Heinrich Mann\pwindex{Mann, Heinrich 27.\,3.\,1871 Lübeck – 11.\,3.\,1950 Santa Monica@\textsc{Mann, Heinrich} (27.\,3.\,1871 Lübeck – 11.\,3.\,1950 Santa Monica), \emph{Schriftsteller}|pw}.}{ }\textbf{Thomas Mann\pwindex{Mann, Thomas 6.\,6.\,1875 Lübeck – 12.\,8.\,1955 Zürich@\textsc{Mann, Thomas} (6.\,6.\,1875 Lübeck – 12.\,8.\,1955 Zürich), \emph{Schriftsteller}|pw}.} Intendant \textbf{Kurt von Mutzenbecher\pwindex{Mutzenbecher, Kurt von 18.\,11.\,1866 Hamburg – 7.\,10.\,1938 Berlin@\textsc{Mutzenbecher, Kurt von} (18.\,11.\,1866 Hamburg – 7.\,10.\,1938 Berlin), \emph{Theaterleiter, Beamter}|pw}, }Kgl. Kammerherr\textbf{.
                        Prof. Dr. Franz Muncker\pwindex{Muncker, Franz 4.\,12.\,1855 Bayreuth – 7.\,9.\,1926 München@\textsc{Muncker, Franz} (4.\,12.\,1855 Bayreuth – 7.\,9.\,1926 München), \emph{Literaturwissenschaftler}|pw}} (München\oindex{München@\textbf{München}|pw})\textbf{.
                        Dr. ing. G. Reg.-R. Hermann Muthesius\pwindex{Muthesius, Hermann 20.\,4.\,1861 Großneuhausen – 26.\,10.\,1927 Berlin@\textsc{Muthesius, Hermann} (20.\,4.\,1861 Großneuhausen – 26.\,10.\,1927 Berlin), \emph{Architekt}|pw}.} Geh. Hofrat \textbf{Prof. Dr. Wilhelm Ostwald\pwindex{Ostwald, Wilhelm 2.\,9.\,1853 Riga – 4.\,4.\,1932 Leipzig@\textsc{Ostwald, Wilhelm} (2.\,9.\,1853 Riga – 4.\,4.\,1932 Leipzig), \emph{Philosoph, Chemiker}|pw}.}
                     General-Intendant \textbf{Baron von Putlitz\pwindex{Gans-Putlitz, Joachim von 7.\,5.\,1860 Retzin – 9.\,3.\,1922 Stuttgart@\textsc{Gans-Putlitz, Joachim von} (7.\,5.\,1860 Retzin – 9.\,3.\,1922 Stuttgart), \emph{Theaterleiter, Vereinspräsident, Theaterintendant}|pw},} Exz\textbf{.} Bürgermeister \textbf{Dr. Georg Reicke\pwindex{Reicke, Georg 26.\,11.\,1863 Kaliningrad – 7.\,4.\,1923 Berlin@\textsc{Reicke, Georg} (26.\,11.\,1863 Kaliningrad – 7.\,4.\,1923 Berlin), \emph{Politiker}|pw}.
                     Dr. Arthur Schnitzler. Dr. Franz Servaes\pwindex{Servaes, Franz 17.\,6.\,1862 Köln – 14.\,7.\,1947 Wien@\textsc{Servaes, Franz} (17.\,6.\,1862 Köln – 14.\,7.\,1947 Wien), \emph{Journalist, Kritiker}|pw}. Hermann Sudermann\pwindex{Sudermann, Hermann 30.\,9.\,1857 Macikai – 21.\,11.\,1928 Berlin@\textsc{Sudermann, Hermann} (30.\,9.\,1857 Macikai – 21.\,11.\,1928 Berlin), \emph{Schriftsteller}|pw}.} Geh. Rat \textbf{Prof. Dr. Henry Thode\pwindex{Thode, Henry 13.\,1.\,1857 Dresden – 19.\,11.\,1920 Kopenhagen@\textsc{Thode, Henry} (13.\,1.\,1857 Dresden – 19.\,11.\,1920 Kopenhagen)|pw}. Prof. Dr. h. c. Hans Thoma\pwindex{Vogeler, Hans 2.\,10.\,1839 Bernau – 7.\,11.\,1924 Karlsruhe@\textsc{Vogeler, Hans} (2.\,10.\,1839 Bernau – 7.\,11.\,1924 Karlsruhe), \emph{Maler}|pw}. Prof. Dr. Ferdinand Vetter\pwindex{Vetter, Ferdinand 3.\,2.\,1847 Bad Osterfingen – 6.\,8.\,1924 Stein am Rhein@\textsc{Vetter, Ferdinand} (3.\,2.\,1847 Bad Osterfingen – 6.\,8.\,1924 Stein am Rhein), \emph{Literaturwissenschaftler}|pw}} (Bern\oindex{Bern@\textbf{Bern}, \emph{Hauptstadt}|pw})\textbf{. Siegfried Wagner\pwindex{Wagner, Siegfried 6.\,6.\,1869 Tribschen – 4.\,8.\,1930 Bayreuth@\textsc{Wagner, Siegfried} (6.\,6.\,1869 Tribschen – 4.\,8.\,1930 Bayreuth), \emph{Komponist}|pw}. Prof. Dr. Eugen Wolff\pwindex{Wolff, Eugen 28.\,9.\,1863 Frankfurt (Oder) – 25.\,2.\,1929 Neubabelsberg@\textsc{Wolff, Eugen} (28.\,9.\,1863 Frankfurt (Oder) – 25.\,2.\,1929 Neubabelsberg), \emph{Literaturwissenschaftler}|pw}} (Kiel\oindex{Kiel@\textbf{Kiel}|pw}.)}}\pend
           \selectlanguage{ngerman}\endnumbering\briefempfaengerindex{Schnitzler, Arthur@\textsc{Schnitzler, Arthur}!zzzHolz, Arno@\emph{von Arno Holz}!1913-04-262@{26. 4. 1913}|)be}\mylabel{L02134h}  \newcommand{\dateiname}{L02134}\newcommand{\titel}{Arno Holz an Arthur Schnitzler, 26. 4. 1913}\newcommand{\editorInnen}{Martin Anton Müller und Gerd-Hermann Susen}%% latex-leseansicht-abspann.tex
%% Abspann für die Leseansicht.
%% Der Schalter \ifkorrekturansicht ist bereits durch den Vorspann gesetzt.

%% latex-abspann.tex
%% Gemeinsamer Abspann für Korrekturansicht und Leseansicht.
%% Setzt den Schalter \ifkorrekturansicht voraus (gesetzt in den
%% einbindenden Dateien latex-korrekturansicht-abspann.tex bzw.
%% latex-leseansicht-abspann.tex).
%% ---------------------------------------------------------------

\normalsize

% Das esempio-Environment wird nur in der Leseansicht benötigt
\ifkorrekturansicht\else
\newenvironment{esempio}[3]%
{
    \vspace{1.5ex}
    \rlap{\underline{#1}}
    \par
    \setlength{\parindent}{0cm}
    \nopagebreak
    \leftskip=#2cm
    \rightskip=#3cm
}
{
    \par
}
\fi

\doendnotes{C}
\bigskip
\vfill

\clearpage

\footnotesize

\ifkorrekturansicht
  \lohead{\textsc{register}}
\fi

% theindex-Environment neu definieren ohne reledmac
\makeatletter
\renewenvironment{theindex}{%
  \ifkorrekturansicht
    \section*{\indexname}%
  \else
    \subsubsection*{Index der erwähnten Entitäten}%
  \fi
  \setlength{\parindent}{0pt}%
  \setlength{\parskip}{0pt plus 0.3pt}%
  \let\item\@idxitem
}{%
  \ifkorrekturansicht\clearpage\fi
}
\makeatother

\IfFileExists{\jobname-pw.ind}{\input{\jobname-pw.ind}}{}

% Quellenangabe nur in der Leseansicht
\ifkorrekturansicht\else
% Fallback-Definitionen, falls die .tex-Datei \titel etc. nicht gesetzt hat
\providecommand{\titel}{}
\providecommand{\editorInnen}{}
\providecommand{\dateiname}{\jobname}

\vspace{3cm}

\vfill

\footnotesize
\textsc{Quelle}: \titel. Herausgegeben von {\editorInnen}. In: \emph{Arthur Schnitzler: Briefwechsel mit Autorinnen und Autoren}.
 Digitale Edition, https://schnitzler-briefe.acdh.oeaw.ac.at/{\dateiname}.html (Stand \today)
\fi

\end{document}


