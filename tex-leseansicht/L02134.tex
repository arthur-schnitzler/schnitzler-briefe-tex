%% latex-leseansicht-vorspann.tex
%% Vorspann für die Leseansicht.
%% Lädt die gemeinsame Datei latex-vorspann.tex mit nicht gesetztem Schalter.

\newif\ifkorrekturansicht
\korrekturansichtfalse

\input{../tex-inputs/latex-vorspann}

\begin{center}
            \textcolor{red}{ENTWURF. ENTZIFFERUNG NOCH NICHT KORREKTURGELESEN}
                      \end{center}
            
               \section[Arno Holz an Arthur Schnitzler, 26. 4. 1913]{ Arno Holz an Arthur Schnitzler, 26. 4. 1913}\nopagebreak\mylabel{v}\rehead{ }\begin{ledgroupsized}[t]{13cm}\normalsize\beginnumbering\briefempfaengerindex{Schnitzler, Arthur@\textsc{Schnitzler, Arthur}!zzzHolz, Arno@\emph{von Arno Holz}!1913-04-262@{26. 4. 1913}|(be} \toendnotes[C]{\smallbreak\pagebreak[2]} \Standort{CUL, Schnitzler, B 44.}
\physDesc{Faltblatt mit einem Porträt von Holz und einer faksimilierten Unterschrift links, rechts einem gedruckten Aufruf, 1 Blatt, 2 Seiten
\newline{}Handschrift: schwarze Tinte, lateinische Kurrent}\pstart
           \noindent{}{\pb}Herrn\pend
           \pstart
           \uline{Dr.}{ }\uline{Arthur}{ }\uline{Schnitzler}\pend
           \pstart
           mit aufrichtigſt herzlichem Dank\pend
           \pstart
           \centering{}\textcolor{gray}{\textbf{ArnoHolz}}\pend
           \pstart
           \noindent{}\centering{}26. IV. 1913.\pend
           \pstart
           \noindent{}\centering{}{\pb}\textcolor{gray}{\textbf{Aufruf für Arno Holz!}}\pend
           \pstart
           \noindent{}\textcolor{gray}{\textbf{Arno Holz, der am 26. d. Mts. 50 Jahre alt
                        wird, erzählt im Vorwort zu seiner letzten, eben erschienenen Tragödie »Ignorabimus\pwindex{Holz, Arno 26.04.1863 – 26.10.1929@\textsc{Holz, Arno} (26.04.1863 – 26.10.1929), \emph{Schriftsteller}!Ignoramibus1913@\strich\emph{Ignoramibus} {[}1913{]}|pw}«, dass er »heute noch immer
                        buchstäblich in einer Dachbude hockt«. Es ergreift und beschämt uns, einen
                        um das deutsche Schrifttum der letzten dreissig Jahre im höchsten Grade
                        verdienten, schon um der Reinheit seines nur der Kunst ergebenen Strebens
                        willen bewunderungswürdigen Dichter in Not zu wissen. Wir fordern die Nation
                        auf, diese Schuld durch eine Ehrenspende zu tilgen, und eröffnen hiermit die
                        Sammlung. Beiträge wolle man an die Leitung des »Kunstwarts\orgindex{Kunstwart@Der Kunstwart|pw}«, Dresden Blasewitz\oindex{Blasewitz@\textbf{Blasewitz}|pw},
                        richten.}}\pend
           \pstart
           \centering{}\textcolor{gray}{\textbf{\textbf{Hermann Bahr\pwindex{Bahr, Hermann 19.07.1863 – 15.01.1934@\textsc{Bahr, Hermann} (19.07.1863 – 15.01.1934), \emph{Schriftsteller, Kritiker}|pw}. Hans Baluschek\pwindex{Baluschek, Hans 09.05.1870 – 27.09.1935@\textsc{Baluschek, Hans} (09.05.1870 – 27.09.1935), \emph{Schriftsteller, Maler}|pw}. Prof. Peter Behrens\pwindex{Behrens, Peter 14.04.1868 – 27.02.1940@\textsc{Behrens, Peter} (14.04.1868 – 27.02.1940), \emph{Architekt}|pw}.}{ }\textbf{Dr. Georg Brandes\pwindex{Brandes, Georg 04.02.1842 – 19.02.1927@\textsc{Brandes, Georg} (04.02.1842 – 19.02.1927)|pw}.
                                Prof. Dr. Collin\pwindex{Collin, Josef 02.02.1864 – 1942@\textsc{Collin, Josef} (02.02.1864 – 1942), \emph{Literaturwissenschaftler}|pw}} (Giessen\oindex{Giessen@\textbf{Gießen}|pw})\textbf{.}{ }\textbf{Dr. Richard Dehmel\pwindex{Dehmel, Richard 18.11.1863 – 08.02.1920@\textsc{Dehmel, Richard} (18.11.1863 – 08.02.1920), \emph{Schriftsteller}|pw}.
                                Dr. Ludwig Fulda\pwindex{Fulda, Ludwig 15.07.1862 – 30.03.1939@\textsc{Fulda, Ludwig} (15.07.1862 – 30.03.1939), \emph{Schriftsteller, Übersetzer}|pw}.}{ }Geh. Hofrat \textbf{Prof. Dr. Ing. h. c. Cornelius Gurlitt\pwindex{Gurlitt, Cornelius 01.01.1850 – 25.03.1939@\textsc{Gurlitt, Cornelius} (01.01.1850 – 25.03.1939)|pw}. Maximilian Harden\pwindex{Harden, Maximilian 20.10.1861 – 30.10.1927@\textsc{Harden, Maximilian} (20.10.1861 – 30.10.1927), \emph{Schriftsteller, Publizist}|pw}. Dr. Georg Hirth\pwindex{Hirth, Georg 13.07.1841 – 28.03.1916@\textsc{Hirth, Georg} (13.07.1841 – 28.03.1916), \emph{Verleger/Verlegerin}|pw}.} General-Intendant \textbf{Graf von Hülsen-Haeseler\pwindex{Huelsen-Haeseler, Georg von 15.07.1858 – 21.06.1922@\textsc{Hülsen-Haeseler, Georg von} (15.07.1858 – 21.06.1922), \emph{Theaterleiter}|pw},} Exz\textbf{.}{ }\textbf{Dr. O. E. Lessing\pwindex{Lessing, Otto Eduard 28.09.1875 – 1942@\textsc{Lessing, Otto Eduard} (28.09.1875 – 1942), \emph{Literaturwissenschaftler}|pw}, Prof.} an der Universität Illinois\oindex{University of Illinois@\textbf{University of Illinois}|pw}\textbf{. Prof. Dr. Alfred Lichtwark\pwindex{Lichtwark, Alfred 1852-11-14 – 1914-01-13@\textsc{Lichtwark, Alfred} (1852-11-14 – 1914-01-13), \emph{Museumsleiter}|pw}. Prof. Dr. h. c. Max Liebermann\pwindex{Liebermann, Max 20.07.1847 – 08.02.1935@\textsc{Liebermann, Max} (20.07.1847 – 08.02.1935), \emph{Maler}|pw}. Dr. Paul Lindau\pwindex{Lindau, Paul 03.06.1839 – 31.01.1919@\textsc{Lindau, Paul} (03.06.1839 – 31.01.1919), \emph{Schriftsteller, Kritiker, Theaterleiter}|pw}. Prof. Dr. Ernst Mach\pwindex{Mach, Ernst 18.02.1838 – 19.02.1916@\textsc{Mach, Ernst} (18.02.1838 – 19.02.1916), \emph{Philosoph, Physiker}|pw}} (Wien\oindex{Wien@\textbf{Wien}|pw})\textbf{. Heinrich Mann\pwindex{Mann, Heinrich 27.03.1871 – 11.03.1950@\textsc{Mann, Heinrich} (27.03.1871 – 11.03.1950), \emph{Schriftsteller}|pw}.}{ }\textbf{Thomas Mann\pwindex{Mann, Thomas 06.06.1875 – 12.08.1955@\textsc{Mann, Thomas} (06.06.1875 – 12.08.1955), \emph{Schriftsteller}|pw}.} Intendant \textbf{Kurt von Mutzenbecher\pwindex{Mutzenbecher, Kurt von 18.11.1866 – 07.10.1938@\textsc{Mutzenbecher, Kurt von} (18.11.1866 – 07.10.1938), \emph{Theaterleiter, Beamter}|pw}, }Kgl. Kammerherr\textbf{.
                                Prof. Dr. Franz Muncker\pwindex{Muncker, Franz 04.12.1855 – 07.09.1926@\textsc{Muncker, Franz} (04.12.1855 – 07.09.1926), \emph{Literaturwissenschaftler}|pw}} (München\oindex{Muenchen@\textbf{München}|pw})\textbf{.
                                Dr. ing. G. Reg.-R. Hermann Muthesius\pwindex{Muthesius, Hermann 20.04.1861 – 26.10.1927@\textsc{Muthesius, Hermann} (20.04.1861 – 26.10.1927), \emph{Architekt/Architektin}|pw}.} Geh. Hofrat \textbf{Prof. Dr. Wilhelm Ostwald\pwindex{Ostwald, Wilhelm 1853-09-02 – 1932-04-04@\textsc{Ostwald, Wilhelm} (1853-09-02 – 1932-04-04), \emph{Philosoph, Chemiker}|pw}.}
                            General-Intendant \textbf{Baron von Putlitz\pwindex{Putlitz, Joachim Gans zu 1861-05-07 – 1922-03-09@\textsc{Putlitz, Joachim Gans zu} (1861-05-07 – 1922-03-09), \emph{Theaterintendant}|pw},} Exz\textbf{.} Bürgermeister \textbf{Dr. Georg Reicke\pwindex{Reicke, Georg 26.11.1863 – 07.04.1923@\textsc{Reicke, Georg} (26.11.1863 – 07.04.1923), \emph{Politiker}|pw}.
                            Dr. Arthur Schnitzler. Dr. Franz Servaes\pwindex{Servaes, Franz 17.06.1862 – 14.07.1947@\textsc{Servaes, Franz} (17.06.1862 – 14.07.1947), \emph{Journalist/Journalistin, Kritiker/Kritikerin}|pw}. Hermann Sudermann\pwindex{Sudermann, Hermann 30.09.1857 – 21.11.1928@\textsc{Sudermann, Hermann} (30.09.1857 – 21.11.1928), \emph{Schriftsteller}|pw}.} Geh. Rat \textbf{Prof. Dr. Henry Thode\pwindex{Thode, Henry 13.01.1857 – 19.11.1920@\textsc{Thode, Henry} (13.01.1857 – 19.11.1920)|pw}. Prof. Dr. h. c. Hans Thoma\pwindex{Thoma, Hans 1839 – 1924@\textsc{Thoma, Hans} (1839 – 1924), \emph{Maler}|pw}. Prof. Dr. Ferdinand Vetter\pwindex{Vetter, Ferdinand 03.02.1847 – 06.08.1924@\textsc{Vetter, Ferdinand} (03.02.1847 – 06.08.1924), \emph{Literaturwissenschaftler}|pw}} (Bern\oindex{Bern@\textbf{Bern}|pw})\textbf{. Siegfried Wagner\pwindex{Wagner, Siegfried 06.06.1869 – 04.08.1930@\textsc{Wagner, Siegfried} (06.06.1869 – 04.08.1930), \emph{Komponist/Komponistin}|pw}. Prof. Dr. Eugen Wolff\pwindex{Wolff, Eugen 28.09.1863 – 25.02.1929@\textsc{Wolff, Eugen} (28.09.1863 – 25.02.1929), \emph{Literaturwissenschaftler}|pw}} (Kiel\oindex{Kiel@\textbf{Kiel}|pw}.)}}\pend
           \endnumbering\briefempfaengerindex{Schnitzler, Arthur@\textsc{Schnitzler, Arthur}!zzzHolz, Arno@\emph{von Arno Holz}!1913-04-262@{26. 4. 1913}|)be}\mylabel{h}\end{ledgroupsized}  \newcommand{\dateiname}{L02134}\newcommand{\titel}{Arno Holz an Arthur Schnitzler, 26. 4. 1913}\newcommand{\editorInnen}{Martin Anton Müller und Gerd-Hermann Susen}%% latex-leseansicht-abspann.tex
%% Abspann für die Leseansicht.
%% Der Schalter \ifkorrekturansicht ist bereits durch den Vorspann gesetzt.

%% latex-abspann.tex
%% Gemeinsamer Abspann für Korrekturansicht und Leseansicht.
%% Setzt den Schalter \ifkorrekturansicht voraus (gesetzt in den
%% einbindenden Dateien latex-korrekturansicht-abspann.tex bzw.
%% latex-leseansicht-abspann.tex).
%% ---------------------------------------------------------------

\normalsize

% Das esempio-Environment wird nur in der Leseansicht benötigt
\ifkorrekturansicht\else
\newenvironment{esempio}[3]%
{
    \vspace{1.5ex}
    \rlap{\underline{#1}}
    \par
    \setlength{\parindent}{0cm}
    \nopagebreak
    \leftskip=#2cm
    \rightskip=#3cm
}
{
    \par
}
\fi

\doendnotes{C}
\bigskip
\vfill

\clearpage

\footnotesize

\ifkorrekturansicht
  \lohead{\textsc{register}}
\fi

% theindex-Environment neu definieren ohne reledmac
\makeatletter
\renewenvironment{theindex}{%
  \ifkorrekturansicht
    \section*{\indexname}%
  \else
    \subsubsection*{Index der erwähnten Entitäten}%
  \fi
  \setlength{\parindent}{0pt}%
  \setlength{\parskip}{0pt plus 0.3pt}%
  \let\item\@idxitem
}{%
  \ifkorrekturansicht\clearpage\fi
}
\makeatother

\IfFileExists{\jobname-pw.ind}{\input{\jobname-pw.ind}}{}

% Quellenangabe nur in der Leseansicht
\ifkorrekturansicht\else
% Fallback-Definitionen, falls die .tex-Datei \titel etc. nicht gesetzt hat
\providecommand{\titel}{}
\providecommand{\editorInnen}{}
\providecommand{\dateiname}{\jobname}

\vspace{3cm}

\vfill

\footnotesize
\textsc{Quelle}: \titel. Herausgegeben von {\editorInnen}. In: \emph{Arthur Schnitzler: Briefwechsel mit Autorinnen und Autoren}.
 Digitale Edition, https://schnitzler-briefe.acdh.oeaw.ac.at/{\dateiname}.html (Stand \today)
\fi

\end{document}


      