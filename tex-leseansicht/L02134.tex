%% latex-korrekturansicht-vorspann.tex
%% Vorspann für die Korrekturansicht.
%% Lädt die gemeinsame Datei latex-vorspann.tex mit gesetztem Schalter.

\newif\ifkorrekturansicht
\korrekturansichttrue

\input{../tex-inputs/latex-vorspann}


\section[Arno Holz an Arthur Schnitzler, 26. 4. 1913]{L02134 Arno Holz an Arthur Schnitzler, 26. 4. 1913}
\nopagebreak\mylabel{L02134v}
\rehead{ }\normalsize\beginnumbering\briefempfaengerindex{Schnitzler, Arthur@\textsc{Schnitzler, Arthur}!zzzHolz, Arno@\emph{von Arno Holz}!1913-04-262@{26. 4. 1913}|(be}
\toendnotes[C]{\smallbreak\pagebreak[2]}\Standort{CUL, Schnitzler, B 44.}
\physDesc{Brief, 1 Blatt, 2 Seiten, 71 Zeichen (Faltblatt mit einem Porträt von Holz und einer faksimilierten
                                 Unterschrift links, rechts einem gedruckten Aufruf )
\newline{}Handschrift: schwarze Tinte, lateinische Kurrent}
\pstart
           \noindent{}{\pb}Herrn\pend
           
\pstart
           \uline{Dr.}{ }\uline{Arthur}{ }\uline{Schnitzler}\pend
           
\pstart
           mit aufrichtigſt herzlichem Dank\pend
           
\pstart
           \centering{}\textcolor{gray}{\textbf{ArnoHolz}}\pend
           
\pstart
           \centering{}26. IV. 1913.\pend
           
\pstart
           \centering{}{\pb}\textcolor{gray}{\textbf{Aufruf für Arno Holz!}}\pend
           
\pstart
           \textcolor{gray}{\textbf{Arno Holz, der am 26. d. Mts. 50 Jahre alt wird,
                  erzählt im Vorwort zu seiner letzten, eben erschienenen Tragödie »Ignorabimus\pwindex{Ignoramibus@\emph{Ignoramibus}|pw}«, dass er »heute noch immer
                  buchstäblich in einer Dachbude hockt«. Es ergreift und beschämt uns, einen um das
                  deutsche Schrifttum der letzten dreissig Jahre im höchsten Grade verdienten, schon
                  um der Reinheit seines nur der Kunst ergebenen Strebens willen
                  bewunderungswürdigen Dichter in Not zu wissen. Wir fordern die Nation auf, diese
                  Schuld durch eine Ehrenspende zu tilgen, und eröffnen hiermit die Sammlung.
                  Beiträge wolle man an die Leitung des »Kunstwarts\orgindex{Kunstwart@Der Kunstwart|pw}«, Dresden Blasewitz\oindex{Blasewitz@\textbf{Blasewitz}, \emph{P.PPLX}|pw},
                  richten.}}\pend
           
\pstart
           \centering{}\textcolor{gray}{\textbf{\textbf{Hermann Bahr\pwindex{Bahr, Hermann 19.07.1863 – 15.01.1934@\textsc{Bahr, Hermann} (19.07.1863 – 15.01.1934), \emph{Schriftsteller/Schriftstellerin, Kritiker/Kritikerin}|pw}. Hans Baluschek\pwindex{Baluschek, Hans 09.05.1870 – 27.09.1935@\textsc{Baluschek, Hans} (09.05.1870 – 27.09.1935), \emph{Schriftsteller/Schriftstellerin, Maler/Malerin}|pw}. Prof. Peter Behrens\pwindex{Behrens, Peter 14.04.1868 – 27.02.1940@\textsc{Behrens, Peter} (14.04.1868 – 27.02.1940), \emph{Architekt/Architektin}|pw}.}{ }\textbf{Dr. Georg Brandes\pwindex{Brandes, Georg 04.02.1842 – 19.02.1927@\textsc{Brandes, Georg} (04.02.1842 – 19.02.1927)|pw}.
                        Prof. Dr. Collin\pwindex{Collin, Josef 1864-02-02 – 1942@\textsc{Collin, Josef} (1864-02-02 – 1942), \emph{Literaturwissenschaftler/Literaturwissenschaftlerin}|pw}} (Giessen\oindex{Giessen@\textbf{Gießen}, \emph{P.PPLA2}|pw})\textbf{.}{ }\textbf{Dr. Richard Dehmel\pwindex{Dehmel, Richard 18.11.1863 – 08.02.1920@\textsc{Dehmel, Richard} (18.11.1863 – 08.02.1920), \emph{Schriftsteller/Schriftstellerin, Schriftsteller/Schriftstellerin, Krimiautor/Krimiautorin}|pw}. Dr. Ludwig Fulda\pwindex{Fulda, Ludwig 15.07.1862 – 30.03.1939@\textsc{Fulda, Ludwig} (15.07.1862 – 30.03.1939), \emph{Schriftsteller/Schriftstellerin, Übersetzer/Übersetzerin}|pw}.}{ }Geh. Hofrat \textbf{Prof. Dr. Ing. h. c. Cornelius Gurlitt\pwindex{Gurlitt, Cornelius 01.01.1850 – 25.03.1939@\textsc{Gurlitt, Cornelius} (01.01.1850 – 25.03.1939)|pw}. Maximilian Harden\pwindex{Harden, Maximilian 20.10.1861 – 30.10.1927@\textsc{Harden, Maximilian} (20.10.1861 – 30.10.1927), \emph{Schriftsteller/Schriftstellerin, Publizist/Publizistin}|pw}. Dr. Georg Hirth\pwindex{Hirth, Georg 13.07.1841 – 28.03.1916@\textsc{Hirth, Georg} (13.07.1841 – 28.03.1916), \emph{Verleger/Verlegerin}|pw}.} General-Intendant \textbf{Graf von Hülsen-Haeseler\pwindex{Huelsen-Haeseler, Georg von 15.07.1858 – 21.06.1922@\textsc{Hülsen-Haeseler, Georg von} (15.07.1858 – 21.06.1922), \emph{Theaterleiter/Theaterleiterin}|pw},} Exz\textbf{.}{ }\textbf{Dr. O. E. Lessing\pwindex{Lessing, Otto Eduard 28.09.1875 – 1942@\textsc{Lessing, Otto Eduard} (28.09.1875 – 1942), \emph{Literaturwissenschaftler/Literaturwissenschaftlerin}|pw}, Prof.} an der Universität Illinois\oindex{University of Illinois@\textbf{University of Illinois}, \emph{Universität (K.UNI)}|pw}\textbf{. Prof. Dr. Alfred Lichtwark\pwindex{Lichtwark, Alfred 1852-11-14 – 1914-01-13@\textsc{Lichtwark, Alfred} (1852-11-14 – 1914-01-13), \emph{Museumsleiter/Museumsleiterin}|pw}. Prof. Dr. h. c. Max Liebermann\pwindex{Liebermann, Max 20.07.1847 – 08.02.1935@\textsc{Liebermann, Max} (20.07.1847 – 08.02.1935), \emph{Maler/Malerin, Maler/Malerin, Maler/Malerin}|pw}. Dr. Paul Lindau\pwindex{Lindau, Paul 03.06.1839 – 31.01.1919@\textsc{Lindau, Paul} (03.06.1839 – 31.01.1919), \emph{Schriftsteller/Schriftstellerin, Kritiker/Kritikerin, Theaterleiter/Theaterleiterin}|pw}. Prof. Dr. Ernst Mach\pwindex{Mach, Ernst 18.02.1838 – 19.02.1916@\textsc{Mach, Ernst} (18.02.1838 – 19.02.1916), \emph{Philosoph/Philosophin, Philosophiehistoriker/Philosophiehistorikerin, Physiker/Physikerin}|pw}} (Wien\oindex{Wien@\textbf{Wien}, \emph{A.ADM2}|pw})\textbf{. Heinrich Mann\pwindex{Mann, Heinrich 27.03.1871 – 11.03.1950@\textsc{Mann, Heinrich} (27.03.1871 – 11.03.1950), \emph{Schriftsteller/Schriftstellerin}|pw}.}{ }\textbf{Thomas Mann\pwindex{Mann, Thomas 06.06.1875 – 12.08.1955@\textsc{Mann, Thomas} (06.06.1875 – 12.08.1955), \emph{Schriftsteller/Schriftstellerin}|pw}.} Intendant \textbf{Kurt von Mutzenbecher\pwindex{Mutzenbecher, Kurt von 18.11.1866 – 07.10.1938@\textsc{Mutzenbecher, Kurt von} (18.11.1866 – 07.10.1938), \emph{Theaterleiter/Theaterleiterin, Beamter/Beamte}|pw}, }Kgl. Kammerherr\textbf{.
                        Prof. Dr. Franz Muncker\pwindex{Muncker, Franz 04.12.1855 – 07.09.1926@\textsc{Muncker, Franz} (04.12.1855 – 07.09.1926), \emph{Literaturwissenschaftler/Literaturwissenschaftlerin}|pw}} (München\oindex{Muenchen@\textbf{München}, \emph{P.PPLA}|pw})\textbf{.
                        Dr. ing. G. Reg.-R. Hermann Muthesius\pwindex{Muthesius, Hermann 20.04.1861 – 26.10.1927@\textsc{Muthesius, Hermann} (20.04.1861 – 26.10.1927), \emph{Architekt/Architektin}|pw}.} Geh. Hofrat \textbf{Prof. Dr. Wilhelm Ostwald\pwindex{Ostwald, Wilhelm 1853-09-02 – 1932-04-04@\textsc{Ostwald, Wilhelm} (1853-09-02 – 1932-04-04), \emph{Philosoph/Philosophin, Chemiker/Chemikerin}|pw}.}
                     General-Intendant \textbf{Baron von Putlitz\pwindex{Gans-Putlitz, Joachim von 07.05.1860 – 09.03.1922@\textsc{Gans-Putlitz, Joachim von} (07.05.1860 – 09.03.1922), \emph{Theaterleiter/Theaterleiterin, Vereinspräsident/Vereinspräsidentin, Theaterintendant/Theaterintendantin}|pw},} Exz\textbf{.} Bürgermeister \textbf{Dr. Georg Reicke\pwindex{Reicke, Georg 26.11.1863 – 07.04.1923@\textsc{Reicke, Georg} (26.11.1863 – 07.04.1923), \emph{Politiker/Politikerin}|pw}.
                     Dr. Arthur Schnitzler. Dr. Franz Servaes\pwindex{Servaes, Franz 17.06.1862 – 14.07.1947@\textsc{Servaes, Franz} (17.06.1862 – 14.07.1947), \emph{Journalist/Journalistin, Kritiker/Kritikerin}|pw}. Hermann Sudermann\pwindex{Sudermann, Hermann 30.09.1857 – 21.11.1928@\textsc{Sudermann, Hermann} (30.09.1857 – 21.11.1928), \emph{Schriftsteller/Schriftstellerin}|pw}.} Geh. Rat \textbf{Prof. Dr. Henry Thode\pwindex{Thode, Henry 13.01.1857 – 19.11.1920@\textsc{Thode, Henry} (13.01.1857 – 19.11.1920)|pw}. Prof. Dr. h. c. Hans Thoma\pwindex{Vogeler, Hans 1839-10-02 – 1924-11-07@\textsc{Vogeler, Hans} (1839-10-02 – 1924-11-07), \emph{Maler/Malerin}|pw}. Prof. Dr. Ferdinand Vetter\pwindex{Vetter, Ferdinand 03.02.1847 – 06.08.1924@\textsc{Vetter, Ferdinand} (03.02.1847 – 06.08.1924), \emph{Literaturwissenschaftler/Literaturwissenschaftlerin}|pw}} (Bern\oindex{Bern@\textbf{Bern}, \emph{P.PPLC}|pw})\textbf{. Siegfried Wagner\pwindex{Wagner, Siegfried 06.06.1869 – 04.08.1930@\textsc{Wagner, Siegfried} (06.06.1869 – 04.08.1930), \emph{Komponist/Komponistin}|pw}. Prof. Dr. Eugen Wolff\pwindex{Wolff, Eugen 28.09.1863 – 25.02.1929@\textsc{Wolff, Eugen} (28.09.1863 – 25.02.1929), \emph{Literaturwissenschaftler/Literaturwissenschaftlerin}|pw}} (Kiel\oindex{Kiel@\textbf{Kiel}, \emph{P.PPLA}|pw}.)}}\pend
           \selectlanguage{ngerman}\endnumbering\briefempfaengerindex{Schnitzler, Arthur@\textsc{Schnitzler, Arthur}!zzzHolz, Arno@\emph{von Arno Holz}!1913-04-262@{26. 4. 1913}|)be}\mylabel{L02134h}  \normalsize

\doendnotes{C}
\bigskip
\vfill

\clearpage

\footnotesize

\lohead{\textsc{register}}

% Definiere theindex-Environment komplett neu ohne reledmac
\makeatletter
\renewenvironment{theindex}{%
  \section*{\indexname}%
  \setlength{\parindent}{0pt}%
  \setlength{\parskip}{0pt plus 0.3pt}%
  \let\item\@idxitem
}{%
  \clearpage
}
\makeatother

\IfFileExists{\jobname-pw.ind}{\input{\jobname-pw.ind}}{}

\end{document}

      