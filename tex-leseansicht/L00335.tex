%% latex-leseansicht-vorspann.tex
%% Vorspann für die Leseansicht.
%% Lädt die gemeinsame Datei latex-vorspann.tex mit nicht gesetztem Schalter.

\newif\ifkorrekturansicht
\korrekturansichtfalse

\input{../tex-inputs/latex-vorspann}


\section[Richard Beer-Hofmann und Arthur Schnitzler an Hugo von Hofmannsthal, {[}5. 6. 1894{]}]{L00335 Richard Beer-Hofmann und Arthur Schnitzler an Hugo von Hofmannsthal, {[}5. 6. 1894{]}}
\nopagebreak\mylabel{L00335v}
\rehead{ }\normalsize\beginnumbering\briefempfaengerindex{Hofmannsthal, Hugo von@\textsc{Hofmannsthal, Hugo von}!zzzSchnitzler, Arthur@\emph{von Arthur Schnitzler}!1894-06-051@{{[}5. 6. 1894{]}}|(be}\briefempfaengerindex{Hofmannsthal, Hugo von@\textsc{Hofmannsthal, Hugo von}!zzzBeer-Hofmann, Richard@\emph{von Richard Beer-Hofmann}!1894-06-051@{{[}5. 6. 1894{]}}|(be}
\toendnotes[C]{\smallbreak\pagebreak[2]}
\correspDesc{Versand  durch Richard Beer-Hofmann, Arthur Schnitzler am [5. 6. 1894] in München
\newline{}Erhalt  durch Hugo von Hofmannsthal im Zeitraum [6. 6. 1894
                  – 10. 6. 1894?] in Wien}\toendnotes[C]{\smallbreak}
\buchAlsQuelle{Hugo von Hofmannsthal, Richard Beer-Hofmann: \emph{Briefwechsel}. Herausgegeben von Eugene Weber. Frankfurt am Main: \emph{S. Fischer} 1972, S. 32.}
\buchAbdrucke{\weitereDrucke{Hermann Bahr, Arthur Schnitzler: \emph{Briefwechsel, Aufzeichnungen, Dokumente (1891–1931)}. Herausgegeben von Kurt Ifkovits und Martin Anton Müller. Göttingen: \emph{Wallstein} 2018.} }
\pstart
           \raggedleft{}{\pb}{[}München\oindex{München@\textbf{München}|pw}{]}{ }Dienstag{ }Mittag{ }{[}5. Juni 1894{]}\pend
           \vspace{0.5em}
\pstart
           Gestern Ihren Brief erhalten. Bahr\pwindex{Bahr, Hermann 19.\,7.\,1863 Linz – 15.\,1.\,1934 München@\textsc{Bahr, Hermann} (19.\,7.\,1863 Linz – 15.\,1.\,1934 München), \emph{Schriftsteller, Kritiker}|pw} erst heute
               früh angekommen. Weiß noch nicht, wie’s, mit C.\pwindex{Cantacuzène, Elsa 23.\,2.\,1865 Gmunden – 7.\,6.\,1946 Garmisch-Partenkirchen@\textsc{Cantacuzène, Elsa} (23.\,2.\,1865 Gmunden – 7.\,6.\,1946 Garmisch-Partenkirchen)|pw}
               und Bahr\pwindex{Bahr, Hermann 19.\,7.\,1863 Linz – 15.\,1.\,1934 München@\textsc{Bahr, Hermann} (19.\,7.\,1863 Linz – 15.\,1.\,1934 München), \emph{Schriftsteller, Kritiker}|pw} und mir sein wird, ob Zeit vorhanden.
                  Muther\pwindex{Muther, Richard 25.\,2.\,1860 Ohrdruf – 28.\,6.\,1909 Międzygórze@\textsc{Muther, Richard} (25.\,2.\,1860 Ohrdruf – 28.\,6.\,1909 Międzygórze), \emph{Kunsthistoriker}|pw} habe ich gestern gesprochen.\pend
           
\pstart
           Herzlichst{\\[\baselineskip]}\spacefill\mbox{Richard}\pend
           \leftskip=0em{}\selectlanguage{ngerman}\vspace{1em}
\pstart
           \noindent{}{[}hs. Schnitzler:{]} Herzliche Grüße\pend
           \pstart \spacefill\mbox{Arthur.}\pend{}\selectlanguage{ngerman}\endnumbering\briefempfaengerindex{Hofmannsthal, Hugo von@\textsc{Hofmannsthal, Hugo von}!zzzSchnitzler, Arthur@\emph{von Arthur Schnitzler}!1894-06-051@{{[}5. 6. 1894{]}}|)be}\briefempfaengerindex{Hofmannsthal, Hugo von@\textsc{Hofmannsthal, Hugo von}!zzzBeer-Hofmann, Richard@\emph{von Richard Beer-Hofmann}!1894-06-051@{{[}5. 6. 1894{]}}|)be}\mylabel{L00335h}  \newcommand{\dateiname}{L00335}\newcommand{\titel}{Richard Beer-Hofmann und Arthur Schnitzler an Hugo von Hofmannsthal, [5. 6. 1894]}\newcommand{\editorInnen}{Herausgegeben von Martin Anton Müller}%% latex-leseansicht-abspann.tex
%% Abspann für die Leseansicht.
%% Der Schalter \ifkorrekturansicht ist bereits durch den Vorspann gesetzt.

%% latex-abspann.tex
%% Gemeinsamer Abspann für Korrekturansicht und Leseansicht.
%% Setzt den Schalter \ifkorrekturansicht voraus (gesetzt in den
%% einbindenden Dateien latex-korrekturansicht-abspann.tex bzw.
%% latex-leseansicht-abspann.tex).
%% ---------------------------------------------------------------

\normalsize

% Das esempio-Environment wird nur in der Leseansicht benötigt
\ifkorrekturansicht\else
\newenvironment{esempio}[3]%
{
    \vspace{1.5ex}
    \rlap{\underline{#1}}
    \par
    \setlength{\parindent}{0cm}
    \nopagebreak
    \leftskip=#2cm
    \rightskip=#3cm
}
{
    \par
}
\fi

\doendnotes{C}
\bigskip
\vfill

\clearpage

\footnotesize

\ifkorrekturansicht
  \lohead{\textsc{register}}
\fi

% theindex-Environment neu definieren ohne reledmac
\makeatletter
\renewenvironment{theindex}{%
  \ifkorrekturansicht
    \section*{\indexname}%
  \else
    \subsubsection*{Index der erwähnten Entitäten}%
  \fi
  \setlength{\parindent}{0pt}%
  \setlength{\parskip}{0pt plus 0.3pt}%
  \let\item\@idxitem
}{%
  \ifkorrekturansicht\clearpage\fi
}
\makeatother

\IfFileExists{\jobname-pw.ind}{\input{\jobname-pw.ind}}{}

% Quellenangabe nur in der Leseansicht
\ifkorrekturansicht\else
% Fallback-Definitionen, falls die .tex-Datei \titel etc. nicht gesetzt hat
\providecommand{\titel}{}
\providecommand{\editorInnen}{}
\providecommand{\dateiname}{\jobname}

\vspace{3cm}

\vfill

\footnotesize
\textsc{Quelle}: \titel. Herausgegeben von {\editorInnen}. In: \emph{Arthur Schnitzler: Briefwechsel mit Autorinnen und Autoren}.
 Digitale Edition, https://schnitzler-briefe.acdh.oeaw.ac.at/{\dateiname}.html (Stand \today)
\fi

\end{document}


