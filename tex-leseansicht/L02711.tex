%% latex-leseansicht-vorspann.tex
%% Vorspann für die Leseansicht.
%% Lädt die gemeinsame Datei latex-vorspann.tex mit nicht gesetztem Schalter.

\newif\ifkorrekturansicht
\korrekturansichtfalse

\input{../tex-inputs/latex-vorspann}

\begin{center}
            \textcolor{red}{ENTWURF, NICHT FERTIG KORRIGIERT}
                      \end{center}
            
               \section[Paul Goldmann an Arthur Schnitzler, 8. 8. 1893]{ Paul Goldmann an Arthur Schnitzler, 8. 8. 1893}\nopagebreak\mylabel{v}\rehead{ }\begin{ledgroupsized}[t]{13cm}\normalsize\beginnumbering\briefempfaengerindex{Schnitzler, Arthur@\textsc{Schnitzler, Arthur}!zzzGoldmann, Paul@\emph{von Paul Goldmann}!1893-08-081@{8. 8. 1893}|(be} \toendnotes[C]{\smallbreak\pagebreak[2]} \Standort{DLA, A:Schnitzler, HS.NZ85.1.3163.}
\physDesc{Brief, 2 Blätter, 8 Seiten
\newline{}Handschrift: schwarze Tinte, deutsche Kurrent
\newline{}Schnitzler: mit rotem Buntstift eine Unterstreichung }\toendnotes[C]{\smallbreak}\pstart
           \noindent{}\textcolor{gray}{\textbf{{\pb}\textbf{Frankfurter Zeitung\orgindex{Frankfurter Zeitung@Frankfurter Zeitung|pw}.}}}\hfill \textsc{Paris\oindex{Paris@\textbf{Paris}|pw}}, 8. August.\pend
           \pstart
           \textcolor{gray}{\textbf{\textbf{(\begin{otherlanguage}{french}Gazette de Francfort\end{otherlanguage}\orgindex{Frankfurter Zeitung@Frankfurter Zeitung|pw}.)}}}\hfill 93.\pend
           \pstart
           \textcolor{gray}{\textbf{\begin{otherlanguage}{french}Directeur\pwindex{Sonnemann, Leopold 1831-10-29 – 1909-10-30@\textsc{Sonnemann, Leopold} (1831-10-29 – 1909-10-30), \emph{Journalist, Herausgeber}|pwv}\end{otherlanguage}{ }\textbf{M. L. Sonnemann\pwindex{Sonnemann, Leopold 1831-10-29 – 1909-10-30@\textsc{Sonnemann, Leopold} (1831-10-29 – 1909-10-30), \emph{Journalist, Herausgeber}|pw}.}}}\pend
           \pstart
           \begin{otherlanguage}{french}\textcolor{gray}{\textbf{Journal\pwindex{Frankfurter Zeitung1856 – 1943@\emph{Frankfurter Zeitung}|pw} politique, financier,}}\end{otherlanguage}\pend
           \pstart
           \begin{otherlanguage}{french}\textcolor{gray}{\textbf{commercial et litteraire.}}\end{otherlanguage}\pend
           \pstart
           \begin{otherlanguage}{french}\textcolor{gray}{\textbf{\textbf{Paraissant trois fois par jour}}}\end{otherlanguage}\pend
           \pstart
           \begin{otherlanguage}{french}\textcolor{gray}{\textbf{\textbf{Bureaux à Paris\oindex{Paris@\textbf{Paris}|pw}:}}}\end{otherlanguage}\pend
           \pstart
           \begin{otherlanguage}{french}\textcolor{gray}{\textbf{\textbf{rue Richelieu 75\oindex{rue Richelieu@\textbf{rue Richelieu}|pw}.}}}\end{otherlanguage}\pend
           \pstart\center{}Mein lieber Arthur!\pend\pstart
           Nicht ohne Bangen habe ich diesmal Deinen lieben Brief eröffnet. Ich war mir einer
               großen Schuld bewußt, und fürchtete Vorwürf\textcolor{gray}{e}. Die bekam ich nun
               nicht direct – ich kenne Deine Güte und Nachſicht – wohl gibt es aber da ein Wort,
               das ich nicht verſtehe\textcolor{gray}{:} »Mißtrauen«. Wirklich, ich habe keine
               Ahnung, worauf Du damit anſpielſt, und befürchte irgend eine verleumderiſche
               Klatſcherei. Mißtrauen? Aber wenn es irgend einen Menſchen gibt, den ich mit ruhigem
               Herzen bis in den letzten Winkel meines Weſens hineinſehen laße, ſo {\pb}biſt Du es, und das weißt Du ſehr wohl. Ich traue
               Dir ebenſo wie mir ſelbſt – nicht ideal, ſchwärmeriſch, penſionsmädchenhaft, ſondern
               auf Grund kühler Manneserfahrung, mit der ich Dich als den Beſten und Treueſten
               erprobt habe. Was willſt Du alſo mit dem kurioſen Wort? Es klingt wie eine falſche
               Note und zeigt mir, daß Zeit und Entfernung auch zwiſchen uns die übliche Arbeit
               gethan.\pend
           \pstart
           Ich habe mich mit Deinem letzten Briefe unendlich gefreut, wochenlang! Und doch habe
               ich Dir nicht geantwortet. Warum? Weil ich gelähmt bin – moraliſch und geiſtig, weil
                  dieſe\strikeout{s} grauenhafte \label{K_L02711-1v}\edtext{Krankheit}{\lemma{\textnormal{\emph{Krankheit}}}\Cendnote{\textnormal{Siehe Paul Goldmann an Arthur Schnitzler, 6. 2. [1893]}}}\label{K_L02711-1h} mein ganzes Sein in einen Nebel hüllt, weil ich am Leben und an der Zukunft
               verzweifle, weil mein Leben {\pb}in zwei Abſchnitte
               zerfällt, die geſunde und die kranke Zeit, weil ich an die geſunde Zeit kein Anrecht
               mehr habe und weil Alles, was mir daher kommt, Alles Liebe und Hoffnungsreiche, mir
               als verloren erſcheint. Mir kommt es vor, als hätte ich kein Recht mehr, mitzuleben.
               Darum konnt’ ich den alten Ton nicht finden, nicht einmal die Energie, eine Feder in
               die Hand zu nehmen, und darum habe ich Dir nicht geantwortet. Mir geht es
               gottsſchlecht trotz aller Kuren. Das Übel greift um ſich, und ich weiß nicht, was aus
               mir wird. Da klammere ich mich denn an die Arbeit und pflüge jeden Tag mein
               abgeſtecktes Stück Feld ab. Bin ich aber fertig, ſo kommen alle Geſpenſter {\pb}wieder. Sehr ſtark bin ich nie geweſen, nun bin ich
               weinerlich wie eine alte Frau, und kaum ein Abend vergeht ohne Thränen. Dabei glaubt
               man nun doch nicht und hat nicht einmal den Troſt, daß Einem Gott das zur Prüfung
               geſchickt hat. Man weiß nur, daß man ein ſchädliches Exemplar der Race geworden,
               deſſen Mitthunwollen ein Verſtoß gegen alle Geſetze der Hygiene iſt. Dann kommt
               natürlich der gute Selbſtmord. Aber es iſt unmöglich, das Leben zu verlaſſen, das man
               jetzt erſt zu verſtehen beginnt, das ſo mannigfaltig und ſo farbig iſt. So bleibt
               Einem nichts als Händeringen und Haarausraufen.\pend
           \pstart
           Ich habe bisher nicht einmal den Entschluß faſſen können, auf Urlaub zu gehen. {\pb}Ich fürchte mich vor der arbeitsloſen Zeit. Von
               Hauſe drängen ſie mich aber. Mein Onkel\pwindex{Mamroth, Fedor 21.02.1851 – 25.06.1907@\textsc{Mamroth, Fedor} (21.02.1851 – 25.06.1907), \emph{Journalist, Kritiker}|pwv} iſt im September in \textsc{Salzburg\oindex{Salzburg@\textbf{Salzburg}|pw}}, und ich ſoll durchaus \label{K_L02711-2v}\edtext{hinkommen}{\lemma{\textnormal{\emph{hinkommen}}}\Cendnote{\textnormal{Goldmann\pwindex{Goldmann, Paul 31.01.1865 – 25.09.1935@\textsc{Goldmann, Paul} (31.01.1865 – 25.09.1935), \emph{Schriftsteller, Journalist}|pwk} reiste tatsächlich im September 1893 nach Salzburg\oindex{Salzburg@\textbf{Salzburg}|pwk}. Vom 17. 9. 1893 ist ein gemeinsamer Abend in Hellbrunn\oindex{Hellbrunn@\textbf{Hellbrunn}|pwk} mit Schnitzler\pwindex{Schnitzler, Arthur 15.05.1862 – 21.10.1931@\textsc{Schnitzler, Arthur} (15.05.1862 – 21.10.1931), \emph{Schriftsteller, Mediziner}|pwk} und Fedor Mamroth\pwindex{Mamroth, Fedor 21.02.1851 – 25.06.1907@\textsc{Mamroth, Fedor} (21.02.1851 – 25.06.1907), \emph{Journalist, Kritiker}|pwk}, vom
                     18. 9. 1893 ein
                  Konzertbesuch mit Schnitzler\pwindex{Schnitzler, Arthur 15.05.1862 – 21.10.1931@\textsc{Schnitzler, Arthur} (15.05.1862 – 21.10.1931), \emph{Schriftsteller, Mediziner}|pwk}
               bekannt.}}}\label{K_L02711-2h}. Er malt mir all’ die Herrlichkeiten von \textsc{Salzburg}\oindex{Salzburg@\textbf{Salzburg}|pw} aus, wie man einem ſtörriſchen Kinde zuredet. Da iſt beſonders eine Verheißung:
                  \textsc{Arthur Schnitzler}. Ach, ich habe ein ſolches Heimweh
               nach Dir, mein theurer Freund. Vielleicht reiße ich mich doch heraus und komme. Thu’
               mir jedenfalls die Liebe und halte Dir im September ein
               paar Tage für mich frei. Wenn ich reiſen ſollte, verſtändige ich Dich {\pb}in den letzten Tagen des Auguſt. Schreib’ mir, ob Dich um dieſe Zeit eine Nachricht in Wien\oindex{Wien@\textbf{Wien}|pw} trifft. Aber bereite Dich vor, mich ſehr zum
               Nachtheil verändert zu finden, und geh’ nicht zu ſtreng mit mir in’s Gericht.\pend
           \pstart
           Dann ſprechen wir auch über alles Übrige. Ich halte zum Beiſpiel eine \label{K_L02711-3v}\edtext{Reiſe nach Berlin\oindex{Berlin@\textbf{Berlin}|pw} zur Betreibung Deiner dramatiſchen Angelegenheiten}{\lemma{\textnormal{\emph{Reiſe … Angelegenheiten}}}\Cendnote{\textnormal{nicht erfolgt}}}\label{K_L02711-3h} für unerläßlich.
               Ebenſo ließe ſich vielleicht hier etwas mit \textsc{Antoine\pwindex{Antoine, Andre 31.01.1858 – 23.10.1943@\textsc{Antoine, André} (31.01.1858 – 23.10.1943), \emph{Theaterleiter, Schauspieler}|pw}} machen, wenn Du eines der \textsc{Anatol\pwindex{Schnitzler, Arthur 15.05.1862 – 21.10.1931@\textsc{Schnitzler, Arthur} (15.05.1862 – 21.10.1931), \emph{Schriftsteller, Mediziner}!Anatol1892-10-29@\strich\emph{Anatol} {[}1892-10-29{]}|pw}}-Stücke ins Franzöſiſche überſetzen könnteſt und ſelbſt hierherkämeſt, um die
               Sache zu betreiben. Seit dem \label{K_L02711-4v}\edtext{Erfolge
                  \textsc{Gerhart Hauptmann\pwindex{Hauptmann, Gerhart 15.11.1862 – 06.06.1946@\textsc{Hauptmann, Gerhart} (15.11.1862 – 06.06.1946), \emph{Schriftsteller}|pw}s}}{\lemma{\textnormal{\emph{Erfolge … Hauptmanns}}}\Cendnote{\textnormal{Gerhart Hauptmann\pwindex{Hauptmann, Gerhart 15.11.1862 – 06.06.1946@\textsc{Hauptmann, Gerhart} (15.11.1862 – 06.06.1946), \emph{Schriftsteller}|pwk}s \emph{Die Weber}\pwindex{Hauptmann, Gerhart 15.11.1862 – 06.06.1946@\textsc{Hauptmann, Gerhart} (15.11.1862 – 06.06.1946), \emph{Schriftsteller}!Weber. Schauspiel aus den vierziger Jahren1892@\strich\emph{Die Weber. Schauspiel aus den vierziger Jahren} {[}1892{]}|pwk}
                  feierte als \emph{Les Tisserands}\pwindex{Hauptmann, Gerhart 15.11.1862 – 06.06.1946@\textsc{Hauptmann, Gerhart} (15.11.1862 – 06.06.1946), \emph{Schriftsteller}!Tisserands. Drame en cinq actes, en prose1893-05-29@\strich\emph{Les Tisserands. Drame en cinq actes, en prose} {[}1893-05-29{]}|pwk} am 29. 5. 1893 am \emph{Théâtre Libre}\orgindex{Theâtre Libre@Théâtre Libre|pwk}
                  Premiere. Wegen des Erfolgs fand am 1. 2. 1894 die nächste Premiere in Anwesenheit des
                  Autors statt: \emph{L'Assomption de Hannele Mattern. Drame de rêve en deux parties}\pwindex{Hauptmann, Gerhart 15.11.1862 – 06.06.1946@\textsc{Hauptmann, Gerhart} (15.11.1862 – 06.06.1946), \emph{Schriftsteller}!L'Assomption de Hannele Mattern. Drame de rêve en deux parties1894-02-01@\strich\emph{L'Assomption de Hannele Mattern. Drame de rêve en deux parties} {[}1894-02-01{]}|pwk}, neuerlich
                  am \emph{Théâtre Libre}\orgindex{Theâtre Libre@Théâtre Libre|pwk}.}}}\label{K_L02711-4h} ſind ſie dort wie ich höre nicht unzugänglich {\pb}für Deutſch\oindex{Deutschland@\textbf{Deutschland}|pw}es
               und Öſterreich\oindex{Oesterreich@\textbf{Österreich}|pw}iſches. Mit dem, was Trottel in
               Saublättern \label{K_L02711-5v}\edtext{über Dich ſchreiben}{\lemma{\textnormal{\emph{über Dich ſchreiben}}}\Cendnote{\textnormal{Am 3. 8. 1893 war ein von Florentine
                     Galliny\pwindex{Galliny, Florentine 24.06.1845 – 19.07.1913@\textsc{Galliny, Florentine} (24.06.1845 – 19.07.1913), \emph{Schriftstellerin, Journalistin}|pwk} unter dem Pseudonym Bruno Walden\pwindex{Galliny, Florentine 24.06.1845 – 19.07.1913@\textsc{Galliny, Florentine} (24.06.1845 – 19.07.1913), \emph{Schriftstellerin, Journalistin}|pwkv} verfasster Verriss des \emph{Anatol-Zyklus}\pwindex{Schnitzler, Arthur 15.05.1862 – 21.10.1931@\textsc{Schnitzler, Arthur} (15.05.1862 – 21.10.1931), \emph{Schriftsteller, Mediziner}!Anatol1892-10-29@\strich\emph{Anatol} {[}1892-10-29{]}|pwk} erschienen: Bruno Walden\pwindex{Galliny, Florentine 24.06.1845 – 19.07.1913@\textsc{Galliny, Florentine} (24.06.1845 – 19.07.1913), \emph{Schriftstellerin, Journalistin}|pwk} [= Florentine Galliny\pwindex{Galliny, Florentine 24.06.1845 – 19.07.1913@\textsc{Galliny, Florentine} (24.06.1845 – 19.07.1913), \emph{Schriftstellerin, Journalistin}|pwk}]: \emph{Feuilleton. Literatur}\pwindex{Feuilleton. Literatur [Anatol]1893-08-03@\emph{Feuilleton. Literatur [Anatol]} {[}1893-08-03{]}|pwk}. In: \emph{Wiener
                        Abendpost}\pwindex{Wiener Abendpost1.7.1863 – 31.12.1921@\emph{Wiener Abendpost}|pwk}, Jg. 190, Nr. 176, 3. 8. 1893,
                     S. 1–2. Sie schreibt: »Bei Arthur Schnitzler\pwindex{Schnitzler, Arthur 15.05.1862 – 21.10.1931@\textsc{Schnitzler, Arthur} (15.05.1862 – 21.10.1931), \emph{Schriftsteller, Mediziner}|pw}s ›Anatol\pwindex{Schnitzler, Arthur 15.05.1862 – 21.10.1931@\textsc{Schnitzler, Arthur} (15.05.1862 – 21.10.1931), \emph{Schriftsteller, Mediziner}!Anatol1892-10-29@\strich\emph{Anatol} {[}1892-10-29{]}|pw}‹ hat
                     ganz und gar die ›\textsc{Vie Paris\oindex{Paris@\textbf{Paris}|pw}ienne}‹ Pathin gestanden, und hier tritt das
                     Nachtreterthum noch viel unangenehmer und plumper zu Tage {[}{\dots}{]} Was dem Paris\oindex{Paris@\textbf{Paris}|pw}er Blatte
                     petillante Frivolität, iſt hier crüder Cynismus, der sich in der Schlußszene
                     zum Höhenpunkte des Anwidernden potencirt.« Über Hugo von Hofmannsthal\pwindex{Hofmannsthal, Hugo von 01.02.1874 – 15.07.1929@\textsc{Hofmannsthal, Hugo von} (01.02.1874 – 15.07.1929), \emph{Schriftsteller}|pwk}s einleitende Verse\pwindex{Schnitzler, Arthur 15.05.1862 – 21.10.1931@\textsc{Schnitzler, Arthur} (15.05.1862 – 21.10.1931), \emph{Schriftsteller, Mediziner}!Anatol1892-10-29@\strich\emph{Anatol} {[}1892-10-29{]}|pwkv} schreibt Walden\pwindex{Galliny, Florentine 24.06.1845 – 19.07.1913@\textsc{Galliny, Florentine} (24.06.1845 – 19.07.1913), \emph{Schriftstellerin, Journalistin}|pwkv} außerdem: »Die
                     Leichtbeschwingheit dieser Verse\pwindex{Schnitzler, Arthur 15.05.1862 – 21.10.1931@\textsc{Schnitzler, Arthur} (15.05.1862 – 21.10.1931), \emph{Schriftsteller, Mediziner}!Anatol1892-10-29@\strich\emph{Anatol} {[}1892-10-29{]}|pwv} gebricht der vorgeführten Scenenreihe\pwindex{Schnitzler, Arthur 15.05.1862 – 21.10.1931@\textsc{Schnitzler, Arthur} (15.05.1862 – 21.10.1931), \emph{Schriftsteller, Mediziner}!Anatol1892-10-29@\strich\emph{Anatol} {[}1892-10-29{]}|pwv}, und damit entfällt die ›hübsche Formel
                     böſer Dinge‹, deren Abstoßendes in Folge dessen ungemildert bleibt, was, wenn
                     auch ethisch ganz nützlich, doch kaum beabsichtigt gewesen sein dürfte. Die
                     introspectiven Grübeleien – ein echt deutscher Zug – dieses Anatol, der sich so
                     ver – – zweifelt interessant vorkommt, sind es, die einer Leichtfertigkeit,
                     welche einzig in unbewußter Lebensüberschäumung eine \textsc{\begin{otherlanguage}{french}Raison d’être\end{otherlanguage}} aufzuweisen vermag, einen so anwidernd perversen Zug aufdrücken. Das
                     entrüstete Freundeswort seines so langmüthig verständnißvollen Vertrauten in
                     der Schlußscene\pwindex{Schnitzler, Arthur 15.05.1862 – 21.10.1931@\textsc{Schnitzler, Arthur} (15.05.1862 – 21.10.1931), \emph{Schriftsteller, Mediziner}!Anatols Hochzeitsmorgen01. 07. 1890@\strich\emph{Anatols Hochzeitsmorgen} {[}01. 07. 1890{]}|pwv} »Anatols Hochzeitstag\pwindex{Schnitzler, Arthur 15.05.1862 – 21.10.1931@\textsc{Schnitzler, Arthur} (15.05.1862 – 21.10.1931), \emph{Schriftsteller, Mediziner}!Anatols Hochzeitsmorgen01. 07. 1890@\strich\emph{Anatols Hochzeitsmorgen} {[}01. 07. 1890{]}|pwv}«:
                     ›So was thut man nicht!‹ läßt sich für dieselbe dahin variiren: So was schreibt
                     man nicht.« (S. 2) Am 4. 8. 1893 notierte sich Schnitzler\pwindex{Schnitzler, Arthur 15.05.1862 – 21.10.1931@\textsc{Schnitzler, Arthur} (15.05.1862 – 21.10.1931), \emph{Schriftsteller, Mediziner}|pwk} im \emph{Tagebuch}\pwindex{Schnitzler, Arthur 15.05.1862 – 21.10.1931@\textsc{Schnitzler, Arthur} (15.05.1862 – 21.10.1931), \emph{Schriftsteller, Mediziner}!Tagebuch1981 – 2000@\strich\emph{Tagebuch} {[}1981 – 2000{]}|pwk}: »In
                     der Abendpost\pwindex{Wiener Abendpost1.7.1863 – 31.12.1921@\emph{Wiener Abendpost}|pw} von Bruno Walden\pwindex{Galliny, Florentine 24.06.1845 – 19.07.1913@\textsc{Galliny, Florentine} (24.06.1845 – 19.07.1913), \emph{Schriftstellerin, Journalistin}|pwv} eine alberne und
                     niederträchtige Kritik\pwindex{Feuilleton. Literatur [Anatol]1893-08-03@\emph{Feuilleton. Literatur [Anatol]} {[}1893-08-03{]}|pwv}
                     über Anatol\pwindex{Schnitzler, Arthur 15.05.1862 – 21.10.1931@\textsc{Schnitzler, Arthur} (15.05.1862 – 21.10.1931), \emph{Schriftsteller, Mediziner}!Anatol1892-10-29@\strich\emph{Anatol} {[}1892-10-29{]}|pw}, die mich
                  verstimmte.«}}}\label{K_L02711-5h}, ſollſt Du Dir dein \textsc{cabinet}
               tapezieren und ruhig weiterſchaffen, auch von vorübergehenden Muthloſigkeiten
               unbeirrt, wie ſie die alltäglichen Erſcheinungsformen aller \strikeout{p\textcolor{gray}{rh}} producirenden Thätigkeit ſind, wenn etwas zuviel Gehirnſchmalz verbraucht iſt.
               Das dumme Gethue, das Dir heute in die Beine kläfft, wird Dir morgen die Hand
               ſchlecken, wenn erſt der \uline{Erfolg} da ſein wird, das
               einzige Beweisſtück in den Augen des Geſindels. Den aber wirſt Du haben, aus dem
               einfachen Grunde, weil Du von de\substVorne{}\textsuperscript{\textcolor{gray}{n}}\substDazwischen{}r\substHinten{} jungen ſchreibenden {\pb}Generation eines der
               größten und glänzendſten Talente biſt. Du biſt viel mehr als \textsc{Herzl\pwindex{Herzl, Theodor 02.05.1860 – 03.07.1904@\textsc{Herzl, Theodor} (02.05.1860 – 03.07.1904), \emph{Schriftsteller, Journalist}|pw}}, denn dieſer iſt – ſo erſtaunlich Dir das klingen mag – ein enger Geiſt, kein
               Dichter, und nur eine Formbegabung. Ich kenne nur Einen, mit dem ich Dich ernſtlich
               vergleiche, das iſt \textsc{Gerhart Hauptmann\pwindex{Hauptmann, Gerhart 15.11.1862 – 06.06.1946@\textsc{Hauptmann, Gerhart} (15.11.1862 – 06.06.1946), \emph{Schriftsteller}|pw}}. Du biſt im Weichen das, was er im Starken iſt – ich urtheile nach den »Webern\pwindex{Hauptmann, Gerhart 15.11.1862 – 06.06.1946@\textsc{Hauptmann, Gerhart} (15.11.1862 – 06.06.1946), \emph{Schriftsteller}!Weber@\strich\emph{Die Weber}|pw}« – und dieſe Überzeugung werden mir alle
               kritiſirenden Pinſel nicht erſchüttern. Deine letzten Werke kenn ich nicht. Mein Onkel\pwindex{Mamroth, Fedor 21.02.1851 – 25.06.1907@\textsc{Mamroth, Fedor} (21.02.1851 – 25.06.1907), \emph{Journalist, Kritiker}|pwv} nennt Deinen Roman\pwindex{Schnitzler, Arthur 15.05.1862 – 21.10.1931@\textsc{Schnitzler, Arthur} (15.05.1862 – 21.10.1931), \emph{Schriftsteller, Mediziner}!Sterben. Novelle1.10.1894 – 1.12.1894@\strich\emph{Sterben. Novelle} {[}1.10.1894 – 1.12.1894{]}|pwv} »bedeutend«. Das iſt ein
                  \label{K_L02711-6v}\edtext{\textsc{Epitheton}}{\lemma{\textnormal{\emph{Epitheton}}}\Cendnote{\textnormal{sprachlicher Zusatz in der Form eines
                  Attributs}}}\label{K_L02711-6h}, das ich von ihm nur auf die bewunderten Meiſter bisher anwenden
               gehört und ich nehme es als erfreuliches Zeugniß.\pend
           \pstart
           Sei von Herzen gegrüßt, mein lieber Arthur!{\\[\baselineskip]}Dein \spacefill\mbox{Paul Goldm}\pend
           \leftskip=0em{}
         
         \endnumbering\briefempfaengerindex{Schnitzler, Arthur@\textsc{Schnitzler, Arthur}!zzzGoldmann, Paul@\emph{von Paul Goldmann}!1893-08-081@{8. 8. 1893}|)be}\mylabel{h}\end{ledgroupsized}  \newcommand{\dateiname}{L02711}\newcommand{\titel}{Paul Goldmann an Arthur Schnitzler, 8. 8. 1893}\newcommand{\editorInnen}{Martin Anton Müller und Laura Untner}
            \footnotesize
\begin{ledgroupsized}[t]{11.5cm}
\doendnotes{C}
\end{ledgroupsized}
         %% latex-leseansicht-abspann.tex
%% Abspann für die Leseansicht.
%% Der Schalter \ifkorrekturansicht ist bereits durch den Vorspann gesetzt.

%% latex-abspann.tex
%% Gemeinsamer Abspann für Korrekturansicht und Leseansicht.
%% Setzt den Schalter \ifkorrekturansicht voraus (gesetzt in den
%% einbindenden Dateien latex-korrekturansicht-abspann.tex bzw.
%% latex-leseansicht-abspann.tex).
%% ---------------------------------------------------------------

\normalsize

% Das esempio-Environment wird nur in der Leseansicht benötigt
\ifkorrekturansicht\else
\newenvironment{esempio}[3]%
{
    \vspace{1.5ex}
    \rlap{\underline{#1}}
    \par
    \setlength{\parindent}{0cm}
    \nopagebreak
    \leftskip=#2cm
    \rightskip=#3cm
}
{
    \par
}
\fi

\doendnotes{C}
\bigskip
\vfill

\clearpage

\footnotesize

\ifkorrekturansicht
  \lohead{\textsc{register}}
\fi

% theindex-Environment neu definieren ohne reledmac
\makeatletter
\renewenvironment{theindex}{%
  \ifkorrekturansicht
    \section*{\indexname}%
  \else
    \subsubsection*{Index der erwähnten Entitäten}%
  \fi
  \setlength{\parindent}{0pt}%
  \setlength{\parskip}{0pt plus 0.3pt}%
  \let\item\@idxitem
}{%
  \ifkorrekturansicht\clearpage\fi
}
\makeatother

\IfFileExists{\jobname-pw.ind}{\input{\jobname-pw.ind}}{}

% Quellenangabe nur in der Leseansicht
\ifkorrekturansicht\else
% Fallback-Definitionen, falls die .tex-Datei \titel etc. nicht gesetzt hat
\providecommand{\titel}{}
\providecommand{\editorInnen}{}
\providecommand{\dateiname}{\jobname}

\vspace{3cm}

\vfill

\footnotesize
\textsc{Quelle}: \titel. Herausgegeben von {\editorInnen}. In: \emph{Arthur Schnitzler: Briefwechsel mit Autorinnen und Autoren}.
 Digitale Edition, https://schnitzler-briefe.acdh.oeaw.ac.at/{\dateiname}.html (Stand \today)
\fi

\end{document}


      