%% latex-korrekturansicht-vorspann.tex
%% Vorspann für die Korrekturansicht.
%% Lädt die gemeinsame Datei latex-vorspann.tex mit gesetztem Schalter.

\newif\ifkorrekturansicht
\korrekturansichttrue

\input{../tex-inputs/latex-vorspann}


\section[Paul Goldmann an Arthur Schnitzler, 8. 8. 1893]{L02711 Paul Goldmann an Arthur Schnitzler, 8. 8. 1893}
\nopagebreak\mylabel{L02711v}
\rehead{ }\normalsize\beginnumbering\briefempfaengerindex{Schnitzler, Arthur@\textsc{Schnitzler, Arthur}!zzzGoldmann, Paul@\emph{von Paul Goldmann}!1893-08-081@{8. 8. 1893}|(be}
\toendnotes[C]{\smallbreak\pagebreak[2]}\Standort{DLA, A:Schnitzler, HS.NZ85.1.3163.}
\physDesc{Brief, 2 Blätter, 8 Seiten, 5049 Zeichen
\newline{}Handschrift: schwarze Tinte, deutsche Kurrent
\newline{}Schnitzler: mit rotem Buntstift zwei Unterstreichungen }\toendnotes[C]{\smallbreak}
\pstart
           \textcolor{gray}{\textbf{{\pb}\textbf{Frankfurter Zeitung\orgindex{Frankfurter Zeitung@Frankfurter Zeitung|pw}.}}}\hfill \textsc{Paris\oindex{Paris@\textbf{Paris}, \emph{P.PPLC}|pw}}, 8. August.\pend
           
\pstart
           \textcolor{gray}{\textbf{\textbf{(\begin{otherlanguage}{french}Gazette de Francfort\end{otherlanguage}\orgindex{Frankfurter Zeitung@Frankfurter Zeitung|pw}.)}}}\hfill 93.\pend
           
\pstart
           \textcolor{gray}{\textbf{\begin{otherlanguage}{french}Directeur\end{otherlanguage}{ }\textbf{M. L. Sonnemann\pwindex{Sonnemann, Leopold 1831-10-29 – 1909-10-30@\textsc{Sonnemann, Leopold} (1831-10-29 – 1909-10-30), \emph{Journalist/Journalistin, Herausgeber/Herausgeberin}|pw}.}}}\pend
           
\pstart
           \begin{otherlanguage}{french}\textcolor{gray}{\textbf{Journal politique, financier,}}\end{otherlanguage}\pend
           
\pstart
           \begin{otherlanguage}{french}\textcolor{gray}{\textbf{commercial et litteraire.}}\end{otherlanguage}\pend
           
\pstart
           \begin{otherlanguage}{french}\textcolor{gray}{\textbf{\textbf{Paraissant trois fois par jour}}}\end{otherlanguage}\pend
           
\pstart
           \begin{otherlanguage}{french}\textcolor{gray}{\textbf{\textbf{Bureaux à Paris\oindex{Paris@\textbf{Paris}, \emph{P.PPLC}|pw}:}}}\end{otherlanguage}\pend
           
\pstart
           \begin{otherlanguage}{french}\textcolor{gray}{\textbf{\textbf{rue Richelieu 75\oindex{rue Richelieu@\textbf{rue Richelieu}, \emph{Straße (K.STR)}|pw}.}}}\end{otherlanguage}\pend
           
\pstart\center{}Mein lieber Arthur!\pend\vspace{0.5em}
\pstart
           Nicht ohne Bangen habe ich diesmal Deinen lieben Brief eröffnet. Ich war mir einer
               großen Schuld bewußt, und fürchtete Vorwürfe. Die bekam ich nun nicht direct – ich
               kenne Deine Güte und Nachſicht – wohl gibt es aber da ein Wort, das ich nicht
                  verſtehe\textcolor{gray}{.} »Mißtrauen«. Wirklich, ich habe keine Ahnung, worauf
               Du damit anſpielſt, und befürchte irgend eine verleumderiſche Klatſcherei. Mißtrauen?
               Aber wenn es irgend einen Menſchen gibt, den ich mit ruhigem Herzen bis in den
               letzten Winkel meines Weſens hineinſehen l\substVorne{}\textsuperscript{a}\substDazwischen{}ie\substHinten{}ße, ſo {\pb}biſt Du es, und das weißt Du ſehr
               wohl. Ich traue Dir ebenſo wie mir ſelbſt – nicht ideal, ſchwärmeriſch,
               penſionsmädchenhaft, ſondern auf Grund kühler Manneserfahrung, mit der ich Dich als
               den Beſten und Treueſten erprobt habe. Was willſt Du alſo mit dem kurioſen Wort? Es
               klingt wie eine falſche Note und zeigt mir, daß Zeit und Entfernung auch zwiſchen uns
               die übliche Arbeit gethan.\pend
           
\pstart
           Ich habe mich mit Deinem letzten Briefe unendlich gefreut, wochenlang! Und doch habe
               ich Dir nicht geantwortet. Warum? Weil ich gelähmt bin – moraliſch und geiſtig, weil
                  dieſe\strikeout{s} grauenhafte \label{K_L02711-1v}\edtext{Krankheit}{\lemma{\textnormal{\emph{Krankheit}}}\Cendnote{\textnormal{Siehe Paul Goldmann an Arthur Schnitzler, 6. 2. [1893].
               }}}\label{K_L02711-1} mein ganzes Sein in einen Nebel hüllt, weil ich am Leben und an der Zukunft
               verzweifle, weil mein Leben {\pb}in zwei Abſchnitte
               zerfällt, die geſunde und die kranke Zeit, weil ich an die geſunde Zeit kein Anrecht
               mehr habe und weil Alles, was mir daher kommt, Alles Liebe und Hoffnungsreiche, mir
               als verloren erſcheint. Mir kommt es vor, als hätte ich kein Recht mehr, mitzuleben.
               Darum konnt’ ich den alten Ton nicht finden, nicht einmal die Energie, eine Feder in
               die Hand zu nehmen, und darum habe ich Dir nicht geantwortet\textcolor{gray}{.} Mir
               geht es gottsſchlecht trotz aller Kuren. Das Übel greift um ſich, und ich weiß
                  nicht\textcolor{gray}{,} was aus mir wird. Da klammere ich mich denn an die
               Arbeit und pflüge jeden Tag mein abgeſtecktes Stück Feld ab. Bin ich aber fertig, ſo
               kommen alle Geſpenſter {\pb}wieder. Sehr ſtark bin ich
               nie geweſen, nun bin ich weinerlich wie eine alte Frau, und kaum ein Abend vergeht
               ohne Thränen. Dabei glaubt man nun doch nicht und hat nicht einmal den Troſt, daß
               Einem Gott das zur Prüfung geſchickt hat. Man weiß nur, daß man ein ſchädliches
               Exemplar der Race geworden, deſſen Mitthunwollen ein Verſtoß gegen alle Geſetze der
               Hygiene iſt. Dann kommt natürlich der gute Selbſtmord. Aber es iſt unmöglich, das
               Leben zu verlaſſen, das man jetzt erſt zu verſtehen beginnt, das ſo mannigfaltig und
               ſo farbig iſt. So bleibt Einem nichts als Händeringen und Haarausraufen.\pend
           
\pstart
           Ich habe bisher nicht einmal d\substVorne{}\textsuperscript{\textcolor{gray}{ie}}\substDazwischen{}en\substHinten{} Entschluß faſſen können, auf Urlaub zu gehen. {\pb}Ich fürchte mich vor der arbeitsloſen Zeit. Von
               Hauſe drängen ſie mich aber. Mein Onkel\pwindex{Mamroth, Fedor 21.02.1851 – 25.06.1907@\textsc{Mamroth, Fedor} (21.02.1851 – 25.06.1907), \emph{Journalist/Journalistin, Kritiker/Kritikerin}|pwv} iſt im September in \textsc{Salzburg\oindex{Salzburg@\textbf{Salzburg}, \emph{A.ADM2}|pw}}, und ich ſoll durchaus \label{K_L02711-2v}\edtext{hinkommen}{\lemma{\textnormal{\emph{hinkommen}}}\Cendnote{\textnormal{Goldmann\pwindex{Goldmann, Paul 31.01.1865 – 25.09.1935@\textsc{Goldmann, Paul} (31.01.1865 – 25.09.1935), \emph{Schriftsteller/Schriftstellerin, Journalist/Journalistin}|pwk} reiste im September 1893 tatsächlich nach Salzburg\oindex{Salzburg@\textbf{Salzburg}, \emph{A.ADM2}|pwk}. Vom 17. 9. 1893 ist ein gemeinsamer Abend in Hellbrunn\oindex{Hellbrunn@\textbf{Hellbrunn}, \emph{P.PPL}|pwk} mit Schnitzler und Fedor Mamroth\pwindex{Mamroth, Fedor 21.02.1851 – 25.06.1907@\textsc{Mamroth, Fedor} (21.02.1851 – 25.06.1907), \emph{Journalist/Journalistin, Kritiker/Kritikerin}|pwk}, vom
                     18. 9. 1893 ein
                  Konzertbesuch mit Schnitzler
               bekannt.}}}\label{K_L02711-2}. Er malt mir all’ die Herrlichkeiten von \textsc{Salzburg}\oindex{Salzburg@\textbf{Salzburg}, \emph{A.ADM2}|pw} aus, wie man einem ſtörriſchen Kinde zuredet. Da iſt beſonders eine Verheißung:
                  \textsc{Arthur Schnitzler}. Ach, ich habe ein ſolches Heimweh
               nach Dir, mein theurer Freund\substVorne{}\textsuperscript{. V}\substDazwischen{}, v\substHinten{}ielleicht reiße ich mich doch heraus und komme. Thu’ mir jedenfalls die Liebe
               und halte Dir im September ein paar Tage für mich frei.
               Wenn ich reiſen ſollte, verſtändige ich Dich {\pb}in den
               letzten Tagen des Auguſt. Schreib’ mir, ob Dich um dieſe
               Zeit eine Nachricht in Wien\oindex{Wien@\textbf{Wien}, \emph{A.ADM2}|pw} trifft. Aber bereite
               Dich vor, mich ſehr zum Nachtheil verändert zu finden, und geh’ nicht zu ſtreng mit
               mir in’s Gericht.\pend
           
\pstart
           Dann ſprechen wir auch über alles Übrige. Ich halte zum Beiſpiel eine \label{K_L02711-3v}\edtext{Reiſe nach Berlin\oindex{Berlin@\textbf{Berlin}, \emph{P.PPLC}|pw}\textcolor{gray}{,} zur Betreibung Deiner dramatiſchen Angelegenheiten}{\lemma{\textnormal{\emph{Reiſe … Angelegenheiten}}}\Cendnote{\textnormal{nicht erfolgt}}}\label{K_L02711-3} für unerläßlich.
               Ebenſo ließe ſich vielleicht hier etwas mit \textsc{Antoine\pwindex{Antoine, Andre 1858-01-31 – 1943-10-23@\textsc{Antoine, André} (1858-01-31 – 1943-10-23), \emph{Theaterleiter/Theaterleiterin, Schauspieler/Schauspielerin}|pw}} machen, wenn Du eines der \textsc{Anatol\pwindex{Anatol@\emph{Anatol}|pw}}-Stücke ins Franzöſiſche überſetzen könnteſt und ſelbſt hierherkämeſt, um die
               Sache zu betreiben. Seit dem \label{K_L02711-4v}\edtext{Erfolge
                  \textsc{Gerhart Hauptmanns\pwindex{Hauptmann, Gerhart 15.11.1862 – 06.06.1946@\textsc{Hauptmann, Gerhart} (15.11.1862 – 06.06.1946), \emph{Schriftsteller/Schriftstellerin}|pw}}}{\lemma{\textnormal{\emph{Erfolge … Hauptmanns}}}\Cendnote{\textnormal{Gerhart Hauptmanns\pwindex{Hauptmann, Gerhart 15.11.1862 – 06.06.1946@\textsc{Hauptmann, Gerhart} (15.11.1862 – 06.06.1946), \emph{Schriftsteller/Schriftstellerin}|pwk}{ }\emph{Die Weber}\pwindex{Weber. Schauspiel aus den vierziger Jahren@\emph{Die Weber. Schauspiel aus den vierziger Jahren}|pwk} feierte als \emph{Les
                     Tisserands}\pwindex{Tisserands. Drame en cinq actes, en prose@\emph{Les Tisserands. Drame en cinq actes, en prose}|pwk} am 29. 5. 1893 am \emph{Théâtre Libre}\orgindex{Theâtre Libre@Théâtre Libre|pwk} Premiere. Wegen des Erfolgs fand am
                     1. 2. 1894 die nächste Premiere in Anwesenheit des Autors\pwindex{Hauptmann, Gerhart 15.11.1862 – 06.06.1946@\textsc{Hauptmann, Gerhart} (15.11.1862 – 06.06.1946), \emph{Schriftsteller/Schriftstellerin}|pwkv} statt: \emph{L’Assomption de Hannele Mattern. Drame de rêve en
                     deux parties}\pwindex{L'Assomption de Hannele Mattern. Drame de rêve en deux parties@\emph{L'Assomption de Hannele Mattern. Drame de rêve en deux parties}|pwk}, neuerlich am \emph{Théâtre
                     Libre}\orgindex{Theâtre Libre@Théâtre Libre|pwk}.}}}\label{K_L02711-4} ſind ſie dort wie ich höre nicht unzugänglich {\pb}für Deutſch\oindex{Deutschland@\textbf{Deutschland}, \emph{A.PCLI}|pw}es
               und Öſterreich\oindex{Oesterreich@\textbf{Österreich}, \emph{A.PCLI}|pw}iſches\textcolor{gray}{.} Mit
               dem, was Trottel\pwindex{Galliny, Florentine 24.06.1845 – 19.07.1913@\textsc{Galliny, Florentine} (24.06.1845 – 19.07.1913), \emph{Schriftsteller/Schriftstellerin, Journalist/Journalistin}|pwv} in Saublättern\pwindex{Wiener Abendpost@\emph{Wiener Abendpost}|pwv}{ }\label{K_L02711-5v}\edtext{über Dich ſchreiben}{\lemma{\textnormal{\emph{über Dich ſchreiben}}}\Cendnote{\textnormal{Am 3. 8. 1893 war ein von Florentine
                  Galliny\pwindex{Galliny, Florentine 24.06.1845 – 19.07.1913@\textsc{Galliny, Florentine} (24.06.1845 – 19.07.1913), \emph{Schriftsteller/Schriftstellerin, Journalist/Journalistin}|pwk} unter dem Pseudonym Bruno Walden\pwindex{Galliny, Florentine 24.06.1845 – 19.07.1913@\textsc{Galliny, Florentine} (24.06.1845 – 19.07.1913), \emph{Schriftsteller/Schriftstellerin, Journalist/Journalistin}|pwkv} verfasster Verriss\pwindex{Feuilleton. Literatur [Anatol]@\emph{Feuilleton. Literatur [Anatol]}|pwkv} des \emph{Anatol}\pwindex{Anatol@\emph{Anatol}|pwk}-Zyklus erschienen: Bruno Walden\pwindex{Galliny, Florentine 24.06.1845 – 19.07.1913@\textsc{Galliny, Florentine} (24.06.1845 – 19.07.1913), \emph{Schriftsteller/Schriftstellerin, Journalist/Journalistin}|pwk} [ = Florentine Galliny\pwindex{Galliny, Florentine 24.06.1845 – 19.07.1913@\textsc{Galliny, Florentine} (24.06.1845 – 19.07.1913), \emph{Schriftsteller/Schriftstellerin, Journalist/Journalistin}|pwk}]: \emph{Feuilleton. Literatur}\pwindex{Feuilleton. Literatur [Anatol]@\emph{Feuilleton. Literatur [Anatol]}|pwk}. In: \emph{Wiener
                        Abendpost}\pwindex{Wiener Abendpost@\emph{Wiener Abendpost}|pwk}, Jg. 190, Nr. 176, 3. 8. 1893,
                     S. 1–2. Sie schrieb: »Bei Arthur
                        Schnitzlers ›Anatol\pwindex{Anatol@\emph{Anatol}|pw}‹ hat ganz und
                     gar die ›\textsc{Vie Paris\oindex{Paris@\textbf{Paris}, \emph{P.PPLC}|pw}ienne}‹ Pathin gestanden, und hier tritt das Nachtreterthum
                     noch viel unangenehmer und plumper zu Tage {[}{\dots}{]}. Was dem Paris\oindex{Paris@\textbf{Paris}, \emph{P.PPLC}|pw}er Blatte
                     petillante Frivolität, iſt hier crüder Cynismus, der sich in der Schlußszene\pwindex{Anatol@\emph{Anatol}|pwv} zum
                     Höhenpunkte des Anwidernden potencirt.« Über Hugo von Hofmannsthals\pwindex{Hofmannsthal, Hugo von 1874-02-01 – 1929-07-15@\textsc{Hofmannsthal, Hugo von} (1874-02-01 – 1929-07-15), \emph{Schriftsteller/Schriftstellerin}|pwk} einleitende Verse\pwindex{Anatol@\emph{Anatol}|pwkv} steht außerdem geschrieben:
                     »Die Leichtbeschwingtheit dieser Verse\pwindex{Anatol@\emph{Anatol}|pwv} gebricht der vorgeführten Scenenreihe\pwindex{Anatol@\emph{Anatol}|pwv}, und damit entfällt die
                     ›hübsche Formel böſer Dinge‹, deren Abstoßendes in Folge dessen ungemildert
                     bleibt, was, wenn auch ethisch ganz nützlich, doch kaum beabsichtigt gewesen
                     sein dürfte. Die introspectiven Grübeleien – ein echt deutscher Zug – dieses
                        Anatol\pwindex{Anatol@\emph{Anatol}|pwv}, der sich so
                     ver – – zweifelt interessant vorkommt, sind es, die einer Leichtfertigkeit,
                     welche einzig in unbewußter Lebensüberschäumung eine \textsc{\begin{otherlanguage}{french}Raison d’être\end{otherlanguage}} aufzuweisen vermag, einen so anwidernd perversen Zug aufdrücken. Das
                     entrüstete Freundeswort seines so langmüthig verständnißvollen Vertrauten in
                     der Schlußscene ›Anatols
                        Hochzeitstag\pwindex{Anatols Hochzeitsmorgen@\emph{Anatols Hochzeitsmorgen}|pwv}‹: ›So was thut man nicht!‹ läßt sich für dieselbe dahin
                     variiren: So was schreibt man nicht.« (S. 2) Am 4. 8. 1893 notierte
                   Schnitzler im \emph{Tagebuch}\pwindex{Tagebuch@\emph{Tagebuch}|pwk}: »In der Abendpost\pwindex{Wiener Abendpost@\emph{Wiener Abendpost}|pw} von Bruno
                        Walden\pwindex{Galliny, Florentine 24.06.1845 – 19.07.1913@\textsc{Galliny, Florentine} (24.06.1845 – 19.07.1913), \emph{Schriftsteller/Schriftstellerin, Journalist/Journalistin}|pwv} eine alberne und niederträchtige Kritik\pwindex{Feuilleton. Literatur [Anatol]@\emph{Feuilleton. Literatur [Anatol]}|pwv} über Anatol\pwindex{Anatol@\emph{Anatol}|pw}, die mich verstimmte.«}}}\label{K_L02711-5}, ſollſt Du Dir Dein \textsc{cabinet} tapezieren und ruhig weiterſchaffen, auch von
               vorübergehenden Muthloſigkeiten unbeirrt, wie ſie die alltäglichen Erſcheinungsformen
               aller \strikeout{p\textcolor{gray}{rh}} producirenden Thätigkeit ſind, wenn etwas zuviel Gehirnſchmalz verbraucht iſt.
               Das dumme Geth\textcolor{gray}{ier}, das Dir heute in die Beine kläfft, wird Dir
               morgen die Hand ſchlecken, wenn erſt der \uline{Erfolg} da
               ſein wird, das einzige Beweisſtück in den Augen des Geſindels. Den aber wirſt Du
               haben, aus dem einfachen Grunde, weil Du von de\substVorne{}\textsuperscript{\textcolor{gray}{m}}\substDazwischen{}r\substHinten{} jungen ſchreibenden {\pb}Generation eines der
               größten und glänzendſten Talente biſt. Du biſt viel mehr als \textsc{Herzl\pwindex{Herzl, Theodor 1860-05-02 – 1904-07-03@\textsc{Herzl, Theodor} (1860-05-02 – 1904-07-03), \emph{Schriftsteller/Schriftstellerin, Journalist/Journalistin}|pw}}, denn dieſer iſt – ſo erſtaunlich Dir das klingen mag – ein enger Geiſt, kein
               Dichter, und nur eine Formbegabung. Ich kenne nur Einen, mit dem ich Dich ernſtlich
               vergleiche, das iſt \textsc{Gerhart Hauptmann\pwindex{Hauptmann, Gerhart 15.11.1862 – 06.06.1946@\textsc{Hauptmann, Gerhart} (15.11.1862 – 06.06.1946), \emph{Schriftsteller/Schriftstellerin}|pw}}. Du biſt im Weichen das, was er im Starken iſt – ich urtheile nach den »Webern\pwindex{Weber@\emph{Die Weber}|pw}« – und dieſe Überzeugung werden mir alle
               kritiſirenden Pinſel nicht erſchüttern. Deine letzten Werke kenne ich nicht. Mein Onkel\pwindex{Mamroth, Fedor 21.02.1851 – 25.06.1907@\textsc{Mamroth, Fedor} (21.02.1851 – 25.06.1907), \emph{Journalist/Journalistin, Kritiker/Kritikerin}|pwv} nennt Deinen Roman\pwindex{Sterben. Novelle@\emph{Sterben. Novelle}|pwv} »bedeutend«. Das iſt ein
                  \label{K_L02711-6v}\edtext{\textsc{Epitheton}}{\lemma{\textnormal{\emph{Epitheton}}}\Cendnote{\textnormal{sprachlicher Zusatz in der Form eines
                  Attributs}}}\label{K_L02711-6}, das ich von ihm nur auf die bewunderten Meiſter bisher anwenden
               gehört und ich nehme es als erfreuliches Zeugniß.\pend
           
\pstart
           Sei von Herzen gegrüßt, mein lieber Arthur!\pend
           \pstart Dein \spacefill\mbox{Paul Goldmnn}\pend{}\selectlanguage{ngerman}\endnumbering\briefempfaengerindex{Schnitzler, Arthur@\textsc{Schnitzler, Arthur}!zzzGoldmann, Paul@\emph{von Paul Goldmann}!1893-08-081@{8. 8. 1893}|)be}\mylabel{L02711h}  \normalsize

\doendnotes{C}
\bigskip
\vfill

\clearpage

\footnotesize

\lohead{\textsc{register}}

% Definiere theindex-Environment komplett neu ohne reledmac
\makeatletter
\renewenvironment{theindex}{%
  \section*{\indexname}%
  \setlength{\parindent}{0pt}%
  \setlength{\parskip}{0pt plus 0.3pt}%
  \let\item\@idxitem
}{%
  \clearpage
}
\makeatother

\IfFileExists{\jobname-pw.ind}{\input{\jobname-pw.ind}}{}

\end{document}

      