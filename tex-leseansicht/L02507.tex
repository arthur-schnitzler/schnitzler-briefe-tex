%% latex-leseansicht-vorspann.tex
%% Vorspann für die Leseansicht.
%% Lädt die gemeinsame Datei latex-vorspann.tex mit nicht gesetztem Schalter.

\newif\ifkorrekturansicht
\korrekturansichtfalse

\input{../tex-inputs/latex-vorspann}


\section[Arthur Schnitzler an Thomas Mann, 30. 1. 1929]{L02507 Arthur Schnitzler an Thomas Mann, 30. 1. 1929}
\nopagebreak\mylabel{L02507v}
\rehead{ }\normalsize\beginnumbering\briefempfaengerindex{Mann, Thomas@\textsc{Mann, Thomas}!zzzSchnitzler, Arthur@\emph{von Arthur Schnitzler}!1929-01-301@{30. 1. 1929}|(be}
\toendnotes[C]{\smallbreak\pagebreak[2]}
\correspDesc{Versand  durch Arthur Schnitzler am 30. 1. 1929 in Wien
\newline{}Erhalt  durch Thomas Mann im Zeitraum [31. 1. 1929
                  – 4. 2. 1929?] in München}\toendnotes[C]{\smallbreak}
\Standort{DLA, A:Schnitzler, 85.1.1371,3.}
\physDesc{Brief, Durchschlag, 1 Blatt, 2 Seiten, 1618 Zeichen
\newline{}Schreibmaschine
\newline{}Handschrift: 1) Bleistift (\noindent{}zusätzliche Kommas, Streichung eines Halbsatzes durch
                                 Einklammerung)\hspace{1em}2) roter Buntstift, lateinische Kurrent (\noindent{}Unterstreichungen. Beschriftung: »Mann Thomas«, »Book of M Club\orgindex{Book of The Month Club@Book of The Month Club|pw}«, »München\oindex{München@\textbf{München}|pw}«)\hspace{1em}}
\buchAbdrucke{\weitereDrucke{Arthur Schnitzler: \emph{Briefe 1913–1931}. Herausgegeben von Peter Michael Braunwarth, Richard Miklin, Susanne Pertlik und Heinrich Schnitzler. Frankfurt am Main: \emph{S. Fischer} 1984, S. 585–586.} }\toendnotes[C]{\smallbreak}
\pstart
           \raggedleft{}{\pb}30. 1. 1929.\pend
           
\pstart{}Lieber und verehrter Herr Thomas Mann.\pend\vspace{0.5em}
\pstart
           Sie haben jedenfalls vom »Book-of-the-Month Club\orgindex{Book of The Month Club@Book of The Month Club|pw}«
               ein ähnliches Schreiben erhalten wie ich und man hat auch Ihnen einen gewissen
               Betrag, sozusagen einen jährlichen Gehalt, offeriert, um als Mitglied eines \label{T_L02507-1v}\edtext{Advisory Committee}{\lemma{\textnormal{\emph{Advisory Committee}}}\Cendnote{\textnormal{In der Vorlage steht »Advisotory Comittee«.}}}\label{T_L02507-1}
               zu fungieren, das seine Meinung über die Empfehlungswürdigkeit der in Deutschland\oindex{Deutschland@\textbf{Deutschland}|pw} erscheinenden Bücher zum Zwecke einer
               Auswahl für die Liste des »Book of the Month
               Club\orgindex{Book of The Month Club@Book of The Month Club|pw}« auszusprechen hätte. Mir ist es nun nicht ganz klar, ob es eigentlich
               angeht, für eine solche Ehrenstelle, die mit nennenswerten Verpflichtungen kaum
               verbunden ist, einen Gehalt zu beziehen; andererseits aber weiss ich nicht recht, von
               wem und in welcher Weise die Annahme einer solchen Stelle oder eines solchen
               Ehrenamtes falsch aufgefasst werden könnte. Trotzdem werden meine freilich sehr vagen
               Bedenken erst beschwichtigt sein, wenn ich erfahre, dass Sie den Vorschlag des »Book of the Month Club\orgindex{Book of The Month Club@Book of The Month Club|pw}« zu akzeptieren gesonnen
               sind. Vorläufig habe ich dem Klub als prinzipiell nicht abgeneigt, doch aufschiebend
               geantwortet. Durch eine baldige Rückäusserung, mein verehrter Herr Thomas Mann,
               würden Sie mich sehr verbinden.\pend
           
\pstart
           Ich habe sehr bedauert\introOben{},\introOben{} Ihnen anlässlich Ihres letzten
               Besuchs in Wien\oindex{Wien@\textbf{Wien}, \emph{Verwaltungsgebiet}|pw} nicht begegnet zu sein. Ein
               nächstes Mal, sei es hier, sei es anderswo, hoffe ich nicht nur Gelegenheit \strikeout{sondern auch die erwünschte innere Verfassung)}{ }{\pb}zu finden, sondern auch \strikeout{in} wieder in der inneren Verfassung zu sein\introOben{},\introOben{}
               Ihnen die Hand zu drücken und mich an vernünftigen und fruchtbaren Gesprächen zu
               beteiligen.\pend
           
\pstart
           Mit den herzlichsten Grüssen{\\[\baselineskip]}Ihr wärmstens ergebener\pend
           \leftskip=0em{}{\vspace{1\baselineskip}}
\pstart
           \noindent{}Herrn Thomas Mann,{\\}München\oindex{München@\textbf{München}|pw}.\pend
           \selectlanguage{ngerman}\endnumbering\briefempfaengerindex{Mann, Thomas@\textsc{Mann, Thomas}!zzzSchnitzler, Arthur@\emph{von Arthur Schnitzler}!1929-01-301@{30. 1. 1929}|)be}\mylabel{L02507h}  \newcommand{\dateiname}{L02507}\newcommand{\titel}{Arthur Schnitzler an Thomas Mann, 30. 1. 1929}\newcommand{\editorInnen}{Martin Anton Müller und Gerd-Hermann Susen}%% latex-leseansicht-abspann.tex
%% Abspann für die Leseansicht.
%% Der Schalter \ifkorrekturansicht ist bereits durch den Vorspann gesetzt.

%% latex-abspann.tex
%% Gemeinsamer Abspann für Korrekturansicht und Leseansicht.
%% Setzt den Schalter \ifkorrekturansicht voraus (gesetzt in den
%% einbindenden Dateien latex-korrekturansicht-abspann.tex bzw.
%% latex-leseansicht-abspann.tex).
%% ---------------------------------------------------------------

\normalsize

% Das esempio-Environment wird nur in der Leseansicht benötigt
\ifkorrekturansicht\else
\newenvironment{esempio}[3]%
{
    \vspace{1.5ex}
    \rlap{\underline{#1}}
    \par
    \setlength{\parindent}{0cm}
    \nopagebreak
    \leftskip=#2cm
    \rightskip=#3cm
}
{
    \par
}
\fi

\doendnotes{C}
\bigskip
\vfill

\clearpage

\footnotesize

\ifkorrekturansicht
  \lohead{\textsc{register}}
\fi

% theindex-Environment neu definieren ohne reledmac
\makeatletter
\renewenvironment{theindex}{%
  \ifkorrekturansicht
    \section*{\indexname}%
  \else
    \subsubsection*{Index der erwähnten Entitäten}%
  \fi
  \setlength{\parindent}{0pt}%
  \setlength{\parskip}{0pt plus 0.3pt}%
  \let\item\@idxitem
}{%
  \ifkorrekturansicht\clearpage\fi
}
\makeatother

\IfFileExists{\jobname-pw.ind}{\input{\jobname-pw.ind}}{}

% Quellenangabe nur in der Leseansicht
\ifkorrekturansicht\else
% Fallback-Definitionen, falls die .tex-Datei \titel etc. nicht gesetzt hat
\providecommand{\titel}{}
\providecommand{\editorInnen}{}
\providecommand{\dateiname}{\jobname}

\vspace{3cm}

\vfill

\footnotesize
\textsc{Quelle}: \titel. Herausgegeben von {\editorInnen}. In: \emph{Arthur Schnitzler: Briefwechsel mit Autorinnen und Autoren}.
 Digitale Edition, https://schnitzler-briefe.acdh.oeaw.ac.at/{\dateiname}.html (Stand \today)
\fi

\end{document}


