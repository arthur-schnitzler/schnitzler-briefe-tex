\input{../tex-inputs/latex-pdf-vorspann}
\begin{center}
            \textcolor{red}{ENTWURF. ENTZIFFERUNG NOCH NICHT KORREKTURGELESEN}
                      \end{center}
            
               \section[Hermann Bahr an Arthur Schnitzler, 15. 3. 1903]{ Hermann Bahr an Arthur Schnitzler, 15. 3. 1903}\nopagebreak\mylabel{v}\rehead{ }\begin{ledgroupsized}[t]{13cm}\normalsize\beginnumbering\briefempfaengerindex{Schnitzler, Arthur@\textsc{Schnitzler, Arthur}!zzzBahr, Hermann@\emph{von Hermann Bahr}!1903-03-151@{15. 3. 1903}|(be} \toendnotes[C]{\smallbreak\pagebreak[2]} \Standort{CUL, Schnitzler, B 5b.}
\physDesc{Brief, 1 Blatt, 1 Seite
\newline{}Handschrift: schwarze Tinte, deutsche Kurrent
\newline{}Schnitzler: mit Bleistift beschriftet: »Bahr« und die Jahreszahl »903« ergänzt \newline{}Ordnung: mit Bleistift von unbekannter Hand nummeriert:
                              »94« }\buchAbdrucke{\weitereDrucke{Hermann Bahr, Arthur Schnitzler: \emph{Briefwechsel, Aufzeichnungen, Dokumente (1891–1931)}. Hg. Kurt Ifkovits und Martin Anton Müller. Göttingen: \emph{Wallstein} 2018, S. 253–254.} }\toendnotes[C]{\smallbreak}\pstart
           \raggedleft{}{\pb}15. 3\pend
           \pstart\center{}Lieber Arthur,\pend\pstart
           ich kann aus unſerer \label{K_L01276_1v}\edtext{Depeſche}{\lemma{\textnormal{\emph{Depeſche}}}\Cendnote{\textnormal{Vgl. \emph{Neues Wiener Tagblatt}\orgindex{Neues Wiener Tagblatt@Neues Wiener Tagblatt|pwk},
                     Jg. 37, Nr. 66, 8. 3. 1903,
                     S. 11: »Aus \so{Berlin}\oindex{Berlin@\textbf{Berlin}|pw} wird uns telegraphiert: Im \so{Deutschen Theater}\oindex{Deutsches Theater Berlin@\textbf{Deutsches Theater Berlin}|pw} fand \so{Schnitzlers}\pwindex{Schnitzler, Arthur 15.05.1862 – 21.10.1931@\textsc{Schnitzler, Arthur} (15.05.1862 – 21.10.1931), \emph{Schriftsteller, Mediziner}|pw} ›\so{Schleier der Beatrice}\pwindex{Schnitzler, Arthur 15.05.1862 – 21.10.1931@\textsc{Schnitzler, Arthur} (15.05.1862 – 21.10.1931), \emph{Schriftsteller, Mediziner}!Schleier der Beatrice. Schauspiel in fuenf Akten1900-12-01 – 1900-12-01@\strich\emph{Der Schleier der Beatrice. Schauspiel in fünf Akten} {[}1900-12-01 – 1900-12-01{]}|pw}‹ bei vortrefflicher Darstellung eine geteilte Aufnahme. Das starke und
                     tiefsinnige Stück interessierte ersichtlich, aber man fand, daß es zu sehr mit
                     konventionellen Theatermitteln arbeite. Nach jedem Aktschlusse kämpften Beifall
                     und Zischen ungemein lebhaft. Der Dichter konnte wiederholt
                  erscheinen.«}}}\label{K_L01276_1h} nicht recht erkennen, wies eigentlich ergangen iſt,
               freue mich aber ſehr, daß die Leute Dein Schmerzenskind\pwindex{Schnitzler, Arthur 15.05.1862 – 21.10.1931@\textsc{Schnitzler, Arthur} (15.05.1862 – 21.10.1931), \emph{Schriftsteller, Mediziner}!Schleier der Beatrice. Schauspiel in fuenf Akten1900-12-01 – 1900-12-01@\strich\emph{Der Schleier der Beatrice. Schauspiel in fünf Akten} {[}1900-12-01 – 1900-12-01{]}|pwv} wenigſtens endlich einmal geſehen haben, und
               hoffe für Berlin\oindex{Berlin@\textbf{Berlin}|pw}, daß sich doch ein paar Kritiker
               finden werden, die ſeine Schönheit merken.\pend
           \pstart
           Ich liege seit vierzehn Tagen wieder, eine Ligatur eitert.\pend
           \pstart
           Herzlichſt{\\[\baselineskip]}Dein{\\[\baselineskip]}\spacefill\mbox{Hermann}\pend
           \leftskip=0em{}\endnumbering\briefempfaengerindex{Schnitzler, Arthur@\textsc{Schnitzler, Arthur}!zzzBahr, Hermann@\emph{von Hermann Bahr}!1903-03-151@{15. 3. 1903}|)be}\mylabel{h}\end{ledgroupsized}  \newcommand{\dateiname}{L01276}\newcommand{\titel}{Hermann Bahr an Arthur Schnitzler, 15. 3. 1903}\newcommand{\editorInnen}{ Kurt Ifkovits,  Martin Anton Müller}\input{../tex-inputs/latex-pdf-abspann}
      