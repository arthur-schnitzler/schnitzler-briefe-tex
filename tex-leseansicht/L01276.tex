%% latex-korrekturansicht-vorspann.tex
%% Vorspann für die Korrekturansicht.
%% Lädt die gemeinsame Datei latex-vorspann.tex mit gesetztem Schalter.

\newif\ifkorrekturansicht
\korrekturansichttrue

\input{../tex-inputs/latex-vorspann}


\section[Hermann Bahr an Arthur Schnitzler, 15. 3. 1903]{L01276 Hermann Bahr an Arthur Schnitzler, 15. 3. 1903}
\nopagebreak\mylabel{L01276v}
\rehead{ }\normalsize\beginnumbering\briefempfaengerindex{Schnitzler, Arthur@\textsc{Schnitzler, Arthur}!zzzBahr, Hermann@\emph{von Hermann Bahr}!1903-03-151@{15. 3. 1903}|(be}
\toendnotes[C]{\smallbreak\pagebreak[2]}\Standort{CUL, Schnitzler, B 5b.}
\physDesc{Brief, 1 Blatt, 1 Seite, 368 Zeichen
\newline{}Handschrift: schwarze Tinte, deutsche Kurrent
\newline{}Schnitzler: mit Bleistift beschriftet: »Bahr« und die
                                 Jahreszahl »903« ergänzt 
\newline{}Ordnung: mit Bleistift von unbekannter Hand nummeriert:
                                    »94« }
\buchAbdrucke{\weitereDrucke{Hermann Bahr, Arthur Schnitzler: \emph{Briefwechsel, Aufzeichnungen, Dokumente (1891–1931)}. Göttingen: \emph{Wallstein} 2018, S. 253–254.} }\toendnotes[C]{\smallbreak}
\pstart
           \raggedleft{}{\pb}15. 3\pend
           
\pstart\center{}Lieber Arthur,\pend\vspace{0.5em}
\pstart
           ich kann aus unſerer \label{K_L01276-1v}\edtext{Depeſche}{\lemma{\textnormal{\emph{Depeſche}}}\Cendnote{\textnormal{Vgl. \emph{Neues Wiener Tagblatt}\orgindex{Neues Wiener Tagblatt@Neues Wiener Tagblatt|pwk}, Jg. 37,
                     Nr. 66, 8. 3. 1903, S. 11: »Aus
                        \so{Berlin}\oindex{Berlin@\textbf{Berlin}, \emph{P.PPLC}|pw} wird uns telegraphiert: Im \so{Deutschen Theater}\oindex{Deutsches Theater Berlin@\textbf{Deutsches Theater Berlin}, \emph{Theater (K.THE)}|pw} fand \so{Schnitzlers} ›\so{Schleier der Beatrice}\pwindex{Schleier der Beatrice. Schauspiel in fuenf Akten@\emph{Der Schleier der Beatrice. Schauspiel in fünf Akten}|pw}‹ bei vortrefflicher Darstellung eine geteilte Aufnahme. Das starke und
                     tiefsinnige Stück interessierte ersichtlich, aber man fand, daß es zu sehr mit
                     konventionellen Theatermitteln arbeite. Nach jedem Aktschlusse kämpften Beifall
                     und Zischen ungemein lebhaft. Der Dichter konnte wiederholt
                  erscheinen.«}}}\label{K_L01276-1} nicht recht erkennen, wies eigentlich ergangen iſt,
               freue mich aber ſehr, daß die Leute Dein Schmerzenskind\pwindex{Schleier der Beatrice. Schauspiel in fuenf Akten@\emph{Der Schleier der Beatrice. Schauspiel in fünf Akten}|pwv} wenigſtens endlich einmal geſehen haben, und
               hoffe für Berlin\oindex{Berlin@\textbf{Berlin}, \emph{P.PPLC}|pw}, daß sich doch ein paar Kritiker
               finden werden, die ſeine Schönheit merken.\pend
           
\pstart
           Ich liege seit vierzehn Tagen wieder, eine Ligatur eitert.\pend
           
\pstart
           Herzlichſt{\\[\baselineskip]}Dein{\\[\baselineskip]}\spacefill\mbox{Hermann}\pend
           \leftskip=0em{}\selectlanguage{ngerman}\endnumbering\briefempfaengerindex{Schnitzler, Arthur@\textsc{Schnitzler, Arthur}!zzzBahr, Hermann@\emph{von Hermann Bahr}!1903-03-151@{15. 3. 1903}|)be}\mylabel{L01276h}  \normalsize

\doendnotes{C}
\bigskip
\vfill

\clearpage

\footnotesize

\lohead{\textsc{register}}

% Definiere theindex-Environment komplett neu ohne reledmac
\makeatletter
\renewenvironment{theindex}{%
  \section*{\indexname}%
  \setlength{\parindent}{0pt}%
  \setlength{\parskip}{0pt plus 0.3pt}%
  \let\item\@idxitem
}{%
  \clearpage
}
\makeatother

\IfFileExists{\jobname-pw.ind}{\input{\jobname-pw.ind}}{}

\end{document}

      