%% latex-leseansicht-vorspann.tex
%% Vorspann für die Leseansicht.
%% Lädt die gemeinsame Datei latex-vorspann.tex mit nicht gesetztem Schalter.

\newif\ifkorrekturansicht
\korrekturansichtfalse

\input{../tex-inputs/latex-vorspann}


\section[Arthur Schnitzler an Stefan Zweig, 14. 11. 1912]{L03782 Arthur Schnitzler an Stefan Zweig, 14. 11. 1912}
\nopagebreak\mylabel{L03782v}
\rehead{ }\normalsize\beginnumbering\briefempfaengerindex{Zweig, Stefan@\textsc{Zweig, Stefan}!zzzSchnitzler, Arthur@\emph{von Arthur Schnitzler}!1912-11-141@{14. 11. 1912}|(be}
\toendnotes[C]{\smallbreak\pagebreak[2]}
\correspDesc{Versand  durch Arthur Schnitzler am 14. 11. 1912 in Wien
\newline{}Erhalt  durch Stefan Zweig im Zeitraum [14. 11. 1912 – 17. 11. 1912?] in Wien}\toendnotes[C]{\smallbreak}
\Standort{Jerusalem, National Library of Israel, ARC. Ms. Var. 305 1 58 Stefan Zweig Collection.}
\physDesc{Briefkarte, 520 Zeichen
\newline{}Handschrift: schwarze Tinte, deutsche Kurrent}\toendnotes[C]{\smallbreak}
\pstart
           {\pb}\textcolor{gray}{\textbf{Dr. Arthur Schnitzler}}\hfill am 14. Nov 1912\pend
           
\pstart
           \textcolor{gray}{\textbf{Wien XVIII. Sternwartestrasse 71\oindex{Wien@\textbf{Wien}!XVIII., Währing@\textbf{XVIII., Währing}!Sternwartestraße 71@\textbf{Sternwartestraße 71}, \emph{Wohngebäude}|pw}}}\pend
           \vspace{0.5em}
\pstart
           lieber Herr Doktor, eben von den \label{K_L03782-1v}\edtext{Berliner\oindex{Berlin@\textbf{Berlin}, \emph{Hauptstadt}|pw}{ }Arrangirproben\eventindex{Kleines Theater@\textbf{Kleines Theater}!Arrangierprobe von Professor Bernhardi, 12.11.1912@Arrangierprobe von Professor Bernhardi, 12.11.1912|pwv}\eventindex{Kleines Theater@\textbf{Kleines Theater}!Arrangierprobe von Professor Bernhardi, 13.11.1912@Arrangierprobe von Professor Bernhardi, 13.11.1912|pwv}}{\lemma{\textnormal{\emph{Berliner Arrangirproben}}}\Cendnote{\textnormal{Diese hatten am 12. 11. 1912 und am 13. 11. 1912 stattgefunden. }}}\label{K_L03782-1}{ }\label{K_L03782-2v}\edtext{zurück}{\lemma{\textnormal{\emph{zurück}}}\Cendnote{\textnormal{Er war tatsächlich am gleichen Tag in der Früh
                  angekommen.}}}\label{K_L03782-2}, find ich Ihren{ }ſchönen und erfreuenden \label{K_L03782-3v}\edtext{Brief}{\lemma{\textnormal{\emph{Brief}}}\Cendnote{\textnormal{XXXX Auszeichnungsfehler: Dokument L03639 nicht gefunden.}}}\label{K_L03782-3} über den \textsc{Bernhardi\pwindex{Schnitzler, Arthur 15.\,5.\,1862 Wien – 21.\,10.\,1931 ebd.@\textsc{Schnitzler, Arthur} (15.\,5.\,1862 Wien – 21.\,10.\,1931 ebd.), \emph{Schriftsteller, Mediziner}!Professor Bernhardi. Komödie in fünf Akten@\strich\emph{Professor Bernhardi. Komödie in fünf Akten}|pw}} vor. Laſſen Sie{ }ſich für Ihr freundſchaftliches, künſtleriſch-menſchliches
               Verſtehn herzlichſt danken. Die \label{K_L03782-4v}\edtext{\textsc{Première}\eventindex{Kleines Theater@\textbf{Kleines Theater}!Uraufführung von Professor Bernhardi, 28.11.1912@Uraufführung von Professor Bernhardi, 28.11.1912|pw}}{\lemma{\textnormal{\emph{Première}}}\Cendnote{\textnormal{Wie hier noch nicht mit letzter
                  Sicherheit konstatiert, fand die Uraufführung von
                        \emph{Professor Bernhardi}\pwindex{Schnitzler, Arthur 15.\,5.\,1862 Wien – 21.\,10.\,1931 ebd.@\textsc{Schnitzler, Arthur} (15.\,5.\,1862 Wien – 21.\,10.\,1931 ebd.), \emph{Schriftsteller, Mediziner}!Professor Bernhardi. Komödie in fünf Akten@\strich\emph{Professor Bernhardi. Komödie in fünf Akten}|pwk}\eventindex{Kleines Theater@\textbf{Kleines Theater}!Uraufführung von Professor Bernhardi, 28.11.1912@Uraufführung von Professor Bernhardi, 28.11.1912|pwk} wirklich am 28. 11. 1912 am \emph{Kleinen Theater}\orgindex{Kleines Theater@Kleines Theater|pwk} in Berlin\oindex{Berlin@\textbf{Berlin}, \emph{Hauptstadt}|pwk} statt. }}}\label{K_L03782-4}{ }ſoll am 28. Nov. in Berlin\oindex{Berlin@\textbf{Berlin}, \emph{Hauptstadt}|pw}{ }ſtattfinden. We{\geminationn} Sie Ihre
               liebenswürdige Abſicht verwirklichen könnten,{ }ſo wär es mir {\pb}eine beſondre Freude, Sie an jenem Abend in Berlin\oindex{Berlin@\textbf{Berlin}, \emph{Hauptstadt}|pw} zu wiſſen. Ich hör wohl noch von Ihnen, ob
               Sie Zeit für die Reiſe haben.\pend
           
\pstart
           Mit herzlichen Grüßen{\\[\baselineskip]}Ihr{\\[\baselineskip]}\spacefill\mbox{Arthur Schnitzler}\pend
           \leftskip=0em{}\selectlanguage{ngerman}\endnumbering\briefempfaengerindex{Zweig, Stefan@\textsc{Zweig, Stefan}!zzzSchnitzler, Arthur@\emph{von Arthur Schnitzler}!1912-11-141@{14. 11. 1912}|)be}\mylabel{L03782h}  \newcommand{\dateiname}{L03782}\newcommand{\titel}{Arthur Schnitzler an Stefan Zweig, 14. 11. 1912}\newcommand{\editorInnen}{Selma Jahnke und Martin Anton Müller}%% latex-leseansicht-abspann.tex
%% Abspann für die Leseansicht.
%% Der Schalter \ifkorrekturansicht ist bereits durch den Vorspann gesetzt.

%% latex-abspann.tex
%% Gemeinsamer Abspann für Korrekturansicht und Leseansicht.
%% Setzt den Schalter \ifkorrekturansicht voraus (gesetzt in den
%% einbindenden Dateien latex-korrekturansicht-abspann.tex bzw.
%% latex-leseansicht-abspann.tex).
%% ---------------------------------------------------------------

\normalsize

% Das esempio-Environment wird nur in der Leseansicht benötigt
\ifkorrekturansicht\else
\newenvironment{esempio}[3]%
{
    \vspace{1.5ex}
    \rlap{\underline{#1}}
    \par
    \setlength{\parindent}{0cm}
    \nopagebreak
    \leftskip=#2cm
    \rightskip=#3cm
}
{
    \par
}
\fi

\doendnotes{C}
\bigskip
\vfill

\clearpage

\footnotesize

\ifkorrekturansicht
  \lohead{\textsc{register}}
\fi

% theindex-Environment neu definieren ohne reledmac
\makeatletter
\renewenvironment{theindex}{%
  \ifkorrekturansicht
    \section*{\indexname}%
  \else
    \subsubsection*{Index der erwähnten Entitäten}%
  \fi
  \setlength{\parindent}{0pt}%
  \setlength{\parskip}{0pt plus 0.3pt}%
  \let\item\@idxitem
}{%
  \ifkorrekturansicht\clearpage\fi
}
\makeatother

\IfFileExists{\jobname-pw.ind}{\input{\jobname-pw.ind}}{}

% Quellenangabe nur in der Leseansicht
\ifkorrekturansicht\else
% Fallback-Definitionen, falls die .tex-Datei \titel etc. nicht gesetzt hat
\providecommand{\titel}{}
\providecommand{\editorInnen}{}
\providecommand{\dateiname}{\jobname}

\vspace{3cm}

\vfill

\footnotesize
\textsc{Quelle}: \titel. Herausgegeben von {\editorInnen}. In: \emph{Arthur Schnitzler: Briefwechsel mit Autorinnen und Autoren}.
 Digitale Edition, https://schnitzler-briefe.acdh.oeaw.ac.at/{\dateiname}.html (Stand \today)
\fi

\end{document}


