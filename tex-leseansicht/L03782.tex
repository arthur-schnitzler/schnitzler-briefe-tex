%% latex-korrekturansicht-vorspann.tex
%% Vorspann für die Korrekturansicht.
%% Lädt die gemeinsame Datei latex-vorspann.tex mit gesetztem Schalter.

\newif\ifkorrekturansicht
\korrekturansichttrue

\input{../tex-inputs/latex-vorspann}


\section[Arthur Schnitzler an Stefan Zweig, 14. 11. 1912]{L03782 Arthur Schnitzler an Stefan Zweig, 14. 11. 1912}
\nopagebreak\mylabel{L03782v}
\rehead{ }\normalsize\beginnumbering\briefempfaengerindex{Zweig, Stefan@\textsc{Zweig, Stefan}!zzzSchnitzler, Arthur@\emph{von Arthur Schnitzler}!1912-11-141@{14. 11. 1912}|(be}
\toendnotes[C]{\smallbreak\pagebreak[2]}\Standort{Jerusalem, National Library of Israel, ARC. Ms. Var. 305 1 58 Stefan Zweig Collection.}
\physDesc{Briefkarte, 1 Blatt, 2 Seiten, 521 Zeichen
\newline{}Handschrift: schwarze Tinte, deutsche Kurrent}\toendnotes[C]{\smallbreak}
\pstart
           {\pb}\textcolor{gray}{\textbf{Dr. Arthur Schnitzler}}\hfill am 14. Nov 1912\pend
           
\pstart
           \textcolor{gray}{\textbf{Wien XVIII. Sternwartestrasse 71\oindex{Sternwartestrasse 71@\textbf{Sternwartestraße 71}, \emph{Wohngebäude (K.WHS)}|pw}}}\pend
           \vspace{0.5em}
\pstart
           lieber Herr Doktor, eben von den Berliner\oindex{Berlin@\textbf{Berlin}, \emph{P.PPLC}|pw} Arrangirproben \label{K_L03782-1v}\edtext{zurück}{\lemma{\textnormal{\emph{zurück}}}\Cendnote{\textnormal{Er war tatsächlich am
                  gleichen Tag in der Früh angekommen.}}}\label{K_L03782-1}, find ich Ihren ſchönen und
               erfreuenden \label{K_L03782-2v}\edtext{Brief}{\lemma{\textnormal{\emph{Brief}}}\Cendnote{\textnormal{Stefan Zweig an Arthur Schnitzler, 12. 11. 1912.}}}\label{K_L03782-2} über den Bernhardi\pwindex{Professor Bernhardi. Komoedie in fuenf Akten@\emph{Professor Bernhardi. Komödie in fünf Akten}|pw} vor. Lassen Sie ſich für Ihr freundſchaftliches,
               künſtleriſch-menſchliches Verſtehn herzlichſt danken. Die \label{K_L03782-3v}\edtext{\textsc{Première}\eventindex{Kleines Theater@\textbf{Kleines Theater}!Urauffuehrung von Professor Bernhardi, 28.11.1912@Uraufführung von Professor Bernhardi, 28.11.1912|pw}}{\lemma{\textnormal{\emph{Première}}}\Cendnote{\textnormal{Wie hier noch nicht mit letzter Sicherheit konstatiert, fand die \emph{Uraufführung von \emph{Professor
                        Bernhardi}\pwindex{Professor Bernhardi. Komoedie in fuenf Akten@\emph{Professor Bernhardi. Komödie in fünf Akten}|pwk}}\eventindex{Kleines Theater@\textbf{Kleines Theater}!Urauffuehrung von Professor Bernhardi, 28.11.1912@Uraufführung von Professor Bernhardi, 28.11.1912|pwk} wirklich am 28. 11. 1912 am \emph{Kleinen Theater}\orgindex{Kleines Theater@Kleines Theater|pwk} in
                     Berlin\oindex{Berlin@\textbf{Berlin}, \emph{P.PPLC}|pwk} statt. }}}\label{K_L03782-3} ſoll am
                  28. Nov. in Berlin\oindex{Berlin@\textbf{Berlin}, \emph{P.PPLC}|pw} ſtattfinden.
               Wenn Sie Ihre liebenswürdige Abſicht verwirklichen könnten, ſo wär es mir {\pb}eine beſondere Freude, Sie an jenem Abend in Berlin\oindex{Berlin@\textbf{Berlin}, \emph{P.PPLC}|pw} zu wiſſen. Ich hör wohl noch von Ihren, ob
               Sie Zeit für die Reiſe haben.\pend
           
\pstart
           Mit herzlichen Grüßen{\\[\baselineskip]}Ihr{\\[\baselineskip]}\spacefill\mbox{Arthur Schnitzler}\pend
           \leftskip=0em{}\selectlanguage{ngerman}\endnumbering\briefempfaengerindex{Zweig, Stefan@\textsc{Zweig, Stefan}!zzzSchnitzler, Arthur@\emph{von Arthur Schnitzler}!1912-11-141@{14. 11. 1912}|)be}\mylabel{L03782h}
\begin{anhang}
\end{anhang}\normalsize

\doendnotes{C}
\bigskip
\vfill

\clearpage

\footnotesize

\lohead{\textsc{register}}

% Definiere theindex-Environment komplett neu ohne reledmac
\makeatletter
\renewenvironment{theindex}{%
  \section*{\indexname}%
  \setlength{\parindent}{0pt}%
  \setlength{\parskip}{0pt plus 0.3pt}%
  \let\item\@idxitem
}{%
  \clearpage
}
\makeatother

\IfFileExists{\jobname-pw.ind}{\input{\jobname-pw.ind}}{}

\end{document}

      