%% latex-korrekturansicht-vorspann.tex
%% Vorspann für die Korrekturansicht.
%% Lädt die gemeinsame Datei latex-vorspann.tex mit gesetztem Schalter.

\newif\ifkorrekturansicht
\korrekturansichttrue

\input{../tex-inputs/latex-vorspann}


\section[Thomas Mann an Arthur Schnitzler, 28. 5. 1928]{L02501 Thomas Mann an Arthur Schnitzler, 28. 5. 1928}
\nopagebreak\mylabel{L02501v}
\rehead{ }\normalsize\beginnumbering\briefempfaengerindex{Schnitzler, Arthur@\textsc{Schnitzler, Arthur}!zzzMann, Thomas@\emph{von Thomas Mann}!1928-05-281@{28. 5. 1928}|(be}
\toendnotes[C]{\smallbreak\pagebreak[2]}\Standort{CUL, Schnitzler, B 67.}
\physDesc{Briefkarte, 943 Zeichen
\newline{}Handschrift: schwarze Tinte, deutsche Kurrent
\newline{}Schnitzler: mit rotem Buntstift beschrieben: »\textsc{Therese\pwindex{Therese. Chronik eines Frauenlebens@\emph{Therese. Chronik eines Frauenlebens}|pw}}« }
\buchAbdrucke{\weitereDrucke{\emph{Modern Austrian Literature}, Jg. 7 (1974) Nr. 1/2, S. 25.} }\toendnotes[C]{\smallbreak}
\pstart
           {\pb}\textcolor{gray}{\textbf{DR. THOMAS MANN}}\hfill \textcolor{gray}{\textbf{MÜNCHEN\oindex{Muenchen@\textbf{München}, \emph{P.PPLA}|pw} den}}{ }28. V. 28.\pend
           
\pstart
           \raggedleft{}\textcolor{gray}{\textbf{POSCHINGERSTR. 1\oindex{Poschingerstrasse@\textbf{Poschingerstraße}, \emph{Straße (K.STR)}|pw}}}\pend
           
\pstart{}Lieber, verehrter Arthur Schnitzler,\pend\vspace{0.5em}
\pstart
           ich muß Ihnen ſagen, wie ſehr ich Ihre »Thereſe\pwindex{Therese. Chronik eines Frauenlebens@\emph{Therese. Chronik eines Frauenlebens}|pw}«
               liebe, dieſen Roman, der, wie alle Guten und Wichtigen heute, keiner mehr iſt, und in
               den ich in langſamer, inniger Lektüre in mich aufgenommen habe. Was ich ſo bewundere,
               iſt die Conception des Buches, das Große, Einfache, Wahre, durchaus Lebensgemäße, die
               dauernde ſtille und tiefe Erſchütterung durch das {\pb}Menſchliche, ohne Aufwand, ohne Spannung,
               Konflikte, »Knotenſchürzung«, »Erfindung«, – lauter Dinge, die als läppiſch zu
               empfinden dies Buch\pwindex{Therese. Chronik eines Frauenlebens@\emph{Therese. Chronik eines Frauenlebens}|pwv} wie kein
               anderes zu lehren geeignet iſt. Und Sie haben dem Menſchenleben, wie es iſt, wie es
               meiſtens iſt, eine Sprache zu finden gewußt, ſchlicht und rein und wahr wiederum,
               wahr, treffend und ſcheinbar unbewegt, aber von ſo zwingender Melodik dabei, daß man
               nach den erſten paar Sätzen weiß: Das leſe ich mit Luſt zu Ende. Haben Sie vielen
               Dank und aufrichtigen Glückwunſch!\pend
           
\pstart
           Ihr ergebener{\\[\baselineskip]}\spacefill\mbox{Thomas Mann.}\pend
           \leftskip=0em{}\selectlanguage{ngerman}\endnumbering\briefempfaengerindex{Schnitzler, Arthur@\textsc{Schnitzler, Arthur}!zzzMann, Thomas@\emph{von Thomas Mann}!1928-05-281@{28. 5. 1928}|)be}\mylabel{L02501h}  \normalsize

\doendnotes{C}
\bigskip
\vfill

\clearpage

\footnotesize

\lohead{\textsc{register}}

% Definiere theindex-Environment komplett neu ohne reledmac
\makeatletter
\renewenvironment{theindex}{%
  \section*{\indexname}%
  \setlength{\parindent}{0pt}%
  \setlength{\parskip}{0pt plus 0.3pt}%
  \let\item\@idxitem
}{%
  \clearpage
}
\makeatother

\IfFileExists{\jobname-pw.ind}{\input{\jobname-pw.ind}}{}

\end{document}

      