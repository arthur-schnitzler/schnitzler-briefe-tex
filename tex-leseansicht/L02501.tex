%% latex-leseansicht-vorspann.tex
%% Vorspann für die Leseansicht.
%% Lädt die gemeinsame Datei latex-vorspann.tex mit nicht gesetztem Schalter.

\newif\ifkorrekturansicht
\korrekturansichtfalse

\input{../tex-inputs/latex-vorspann}


\section[Thomas Mann an Arthur Schnitzler, 28. 5. 1928]{L02501 Thomas Mann an Arthur Schnitzler, 28. 5. 1928}
\nopagebreak\mylabel{L02501v}
\rehead{ }\normalsize\beginnumbering\briefempfaengerindex{Schnitzler, Arthur@\textsc{Schnitzler, Arthur}!zzzMann, Thomas@\emph{von Thomas Mann}!1928-05-281@{28. 5. 1928}|(be}
\toendnotes[C]{\smallbreak\pagebreak[2]}
\correspDesc{Versand  durch Thomas Mann am 28. 5. 1928 in München
\newline{}Erhalt  durch Arthur Schnitzler im Zeitraum [29. 5. 1928
                  – 2. 6. 1928?] in Wien}\toendnotes[C]{\smallbreak}
\Standort{CUL, Schnitzler, B 67.}
\physDesc{Briefkarte, 943 Zeichen
\newline{}Handschrift: schwarze Tinte, deutsche Kurrent
\newline{}Schnitzler: mit rotem Buntstift beschrieben: »\textsc{Therese\pwindex{Schnitzler, Arthur 15.\,5.\,1862 Wien – 21.\,10.\,1931 ebd.@\textsc{Schnitzler, Arthur} (15.\,5.\,1862 Wien – 21.\,10.\,1931 ebd.), \emph{Schriftsteller, Mediziner}!Therese. Chronik eines Frauenlebens@\strich\emph{Therese. Chronik eines Frauenlebens}|pw}}« }
\buchAbdrucke{\weitereDrucke{Hertha Krotkoff: \emph{Arthur Schnitzler – Thomas Mann: Briefe.} In: \emph{Modern Austrian Literature}, Jg. 7 (1974) Nr. 1/2, S. 25.} }\toendnotes[C]{\smallbreak}
\pstart
           {\pb}\textcolor{gray}{\textbf{DR. THOMAS MANN}}\hfill \textcolor{gray}{\textbf{MÜNCHEN\oindex{München@\textbf{München}|pw} den}}{ }28. V. 28.\pend
           
\pstart
           \raggedleft{}\textcolor{gray}{\textbf{POSCHINGERSTR. 1\oindex{Poschingerstraße@\textbf{Poschingerstraße}, \emph{Straße}|pw}}}\pend
           
\pstart{}Lieber, verehrter Arthur Schnitzler,\pend\vspace{0.5em}
\pstart
           ich muß Ihnen{ }ſagen, wie{ }ſehr ich Ihre »Thereſe\pwindex{Schnitzler, Arthur 15.\,5.\,1862 Wien – 21.\,10.\,1931 ebd.@\textsc{Schnitzler, Arthur} (15.\,5.\,1862 Wien – 21.\,10.\,1931 ebd.), \emph{Schriftsteller, Mediziner}!Therese. Chronik eines Frauenlebens@\strich\emph{Therese. Chronik eines Frauenlebens}|pw}«
               liebe, dieſen Roman, der, wie alle Guten und Wichtigen heute, keiner mehr iſt, und in
               den ich in langſamer, inniger Lektüre in mich aufgenommen habe. Was ich{ }ſo bewundere,
               iſt die Conception des Buches, das Große, Einfache, Wahre, durchaus Lebensgemäße, die
               dauernde{ }ſtille und tiefe Erſchütterung durch das {\pb}Menſchliche, ohne Aufwand, ohne Spannung,
               Konflikte, »Knotenſchürzung«, »Erfindung«, – lauter Dinge, die als läppiſch zu
               empfinden dies Buch\pwindex{Schnitzler, Arthur 15.\,5.\,1862 Wien – 21.\,10.\,1931 ebd.@\textsc{Schnitzler, Arthur} (15.\,5.\,1862 Wien – 21.\,10.\,1931 ebd.), \emph{Schriftsteller, Mediziner}!Therese. Chronik eines Frauenlebens@\strich\emph{Therese. Chronik eines Frauenlebens}|pwv} wie kein
               anderes zu lehren geeignet iſt. Und Sie haben dem Menſchenleben, wie es iſt, wie es
               meiſtens iſt, eine Sprache zu finden gewußt,{ }ſchlicht und rein und wahr wiederum,
               wahr, treffend und{ }ſcheinbar unbewegt, aber von{ }ſo zwingender Melodik dabei, daß man
               nach den erſten paar Sätzen weiß: Das leſe ich mit Luſt zu Ende. Haben Sie vielen
               Dank und aufrichtigen Glückwunſch!\pend
           
\pstart
           Ihr ergebener{\\[\baselineskip]}\spacefill\mbox{Thomas Mann.}\pend
           \leftskip=0em{}\selectlanguage{ngerman}\endnumbering\briefempfaengerindex{Schnitzler, Arthur@\textsc{Schnitzler, Arthur}!zzzMann, Thomas@\emph{von Thomas Mann}!1928-05-281@{28. 5. 1928}|)be}\mylabel{L02501h}  \newcommand{\dateiname}{L02501}\newcommand{\titel}{Thomas Mann an Arthur Schnitzler, 28. 5. 1928}\newcommand{\editorInnen}{Martin Anton Müller und Gerd-Hermann Susen}%% latex-leseansicht-abspann.tex
%% Abspann für die Leseansicht.
%% Der Schalter \ifkorrekturansicht ist bereits durch den Vorspann gesetzt.

%% latex-abspann.tex
%% Gemeinsamer Abspann für Korrekturansicht und Leseansicht.
%% Setzt den Schalter \ifkorrekturansicht voraus (gesetzt in den
%% einbindenden Dateien latex-korrekturansicht-abspann.tex bzw.
%% latex-leseansicht-abspann.tex).
%% ---------------------------------------------------------------

\normalsize

% Das esempio-Environment wird nur in der Leseansicht benötigt
\ifkorrekturansicht\else
\newenvironment{esempio}[3]%
{
    \vspace{1.5ex}
    \rlap{\underline{#1}}
    \par
    \setlength{\parindent}{0cm}
    \nopagebreak
    \leftskip=#2cm
    \rightskip=#3cm
}
{
    \par
}
\fi

\doendnotes{C}
\bigskip
\vfill

\clearpage

\footnotesize

\ifkorrekturansicht
  \lohead{\textsc{register}}
\fi

% theindex-Environment neu definieren ohne reledmac
\makeatletter
\renewenvironment{theindex}{%
  \ifkorrekturansicht
    \section*{\indexname}%
  \else
    \subsubsection*{Index der erwähnten Entitäten}%
  \fi
  \setlength{\parindent}{0pt}%
  \setlength{\parskip}{0pt plus 0.3pt}%
  \let\item\@idxitem
}{%
  \ifkorrekturansicht\clearpage\fi
}
\makeatother

\IfFileExists{\jobname-pw.ind}{\input{\jobname-pw.ind}}{}

% Quellenangabe nur in der Leseansicht
\ifkorrekturansicht\else
% Fallback-Definitionen, falls die .tex-Datei \titel etc. nicht gesetzt hat
\providecommand{\titel}{}
\providecommand{\editorInnen}{}
\providecommand{\dateiname}{\jobname}

\vspace{3cm}

\vfill

\footnotesize
\textsc{Quelle}: \titel. Herausgegeben von {\editorInnen}. In: \emph{Arthur Schnitzler: Briefwechsel mit Autorinnen und Autoren}.
 Digitale Edition, https://schnitzler-briefe.acdh.oeaw.ac.at/{\dateiname}.html (Stand \today)
\fi

\end{document}


