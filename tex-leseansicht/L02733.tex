%% latex-korrekturansicht-vorspann.tex
%% Vorspann für die Korrekturansicht.
%% Lädt die gemeinsame Datei latex-vorspann.tex mit gesetztem Schalter.

\newif\ifkorrekturansicht
\korrekturansichttrue

\input{../tex-inputs/latex-vorspann}


\section[Paul Goldmann an Arthur Schnitzler, 3. 4. {[}1895{]}]{L02733 Paul Goldmann an Arthur Schnitzler, 3. 4. {[}1895{]}}
\nopagebreak\mylabel{L02733v}
\rehead{ }\normalsize\beginnumbering\briefempfaengerindex{Schnitzler, Arthur@\textsc{Schnitzler, Arthur}!zzzGoldmann, Paul@\emph{von Paul Goldmann}!1895-04-031@{3. 4. {[}1895{]}}|(be}
\toendnotes[C]{\smallbreak\pagebreak[2]}\Standort{DLA, A:Schnitzler, HS.NZ85.1.3165.}
\physDesc{Brief, 1 Blatt, 2 Seiten, 747 Zeichen
\newline{}Handschrift: schwarze Tinte, deutsche Kurrent
\newline{}Schnitzler: 1) mit schwarzer Tinte das Jahr »95« vermerkt  2) mit rotem Buntstift eine Unterstreichung}\toendnotes[C]{\smallbreak}
\pstart
           {\pb}\textcolor{gray}{\textbf{\textbf{Frankfurter Zeitung\orgindex{Frankfurter Zeitung@Frankfurter Zeitung|pw}}}}\pend
           
\pstart
           \textcolor{gray}{\textbf{(\begin{otherlanguage}{french}Gazette de Francfort\end{otherlanguage}\orgindex{Frankfurter Zeitung@Frankfurter Zeitung|pw}). }}\pend
           
\pstart
           \textcolor{gray}{\textbf{\textbf{\begin{otherlanguage}{french}Fondateur M. L.
                              Sonnemann\pwindex{Sonnemann, Leopold 1831-10-29 – 1909-10-30@\textsc{Sonnemann, Leopold} (1831-10-29 – 1909-10-30), \emph{Journalist/Journalistin, Herausgeber/Herausgeberin}|pw}\end{otherlanguage}.}}}\pend
           
\pstart
           \begin{otherlanguage}{french}\textcolor{gray}{\textbf{Journal politique, financier,}}\end{otherlanguage}\pend
           
\pstart
           \begin{otherlanguage}{french}\textcolor{gray}{\textbf{commercial et littéraire.}}\end{otherlanguage}\pend
           
\pstart
           \begin{otherlanguage}{french}\textcolor{gray}{\textbf{\textbf{Paraissant trois fois par jour.}}}\end{otherlanguage}\hfill \textsc{Paris\oindex{Paris@\textbf{Paris}, \emph{P.PPLC}|pw}}, 3. April.\pend
           
\pstart
           \begin{otherlanguage}{french}\textcolor{gray}{\textbf{\textbf{Bureau à Paris\oindex{Paris@\textbf{Paris}, \emph{P.PPLC}|pw}:}}}\end{otherlanguage}\pend
           
\pstart
           \begin{otherlanguage}{french}\textcolor{gray}{\textbf{\textbf{24. Rue Feydeau\oindex{rue Feydeau@\textbf{rue Feydeau}, \emph{Straße (K.STR)}|pw}.}}}\end{otherlanguage}\pend
           
\pstart\center{}Mein lieber Freund,\pend\vspace{0.5em}
\pstart
           In Eile: Dieſen \label{K_L02733-1v}\edtext{Mann\pwindex{Vallette, Gaspard 13.5.1865 – 6.8.1911@\textsc{Vallette, Gaspard} (13.5.1865 – 6.8.1911), \emph{Journalist/Journalistin, Übersetzer/Übersetzerin}|pwuv}}{\lemma{\textnormal{\emph{Mann}}}\Cendnote{\textnormal{Es dürfte sich um Gaspard Vallette\pwindex{Vallette, Gaspard 13.5.1865 – 6.8.1911@\textsc{Vallette, Gaspard} (13.5.1865 – 6.8.1911), \emph{Journalist/Journalistin, Übersetzer/Übersetzerin}|pwk} handeln, der \emph{Sterben}\pwindex{Sterben. Novelle@\emph{Sterben. Novelle}|pwk} ins Französische übersetzte. Nur wenige Tage vor
                  der Entstehung dieses Briefs, am 31. 3. 1895, notierte Schnitzler die Anfrage zur Übersetzung\pwindex{Mourir. Roman@\emph{Mourir. Roman}|pwkv} im \emph{Tagebuch}\pwindex{Tagebuch@\emph{Tagebuch}|pwk}.}}}\label{K_L02733-1} in \textsc{Cannes\oindex{Cannes@\textbf{Cannes}, \emph{P.PPL}|pw}} kenne ich nicht, und Niemand kennt ihn, den ich hier befragt. Die Adreſſe
               deutet auf einen \label{K_L02733-2v}\edtext{\textsc{\begin{otherlanguage}{french}homme cossu\end{otherlanguage}}}{\lemma{\textnormal{\emph{homme cossu}}}\Cendnote{\textnormal{französisch: wohlhabender Mann}}}\label{K_L02733-2}
               hin. Ob er Franzöſiſch kann? Denn es ſcheint \label{K_L02733-3v}\edtext{kein Fran\oindex{Frankreich@\textbf{Frankreich}, \emph{A.PCLI}|pwv}zoſe}{\lemma{\textnormal{\emph{kein Franzoſe}}}\Cendnote{\textnormal{Vallette\pwindex{Vallette, Gaspard 13.5.1865 – 6.8.1911@\textsc{Vallette, Gaspard} (13.5.1865 – 6.8.1911), \emph{Journalist/Journalistin, Übersetzer/Übersetzerin}|pwk} war Schweizer\oindex{Schweiz@\textbf{Schweiz}, \emph{A.PCLI}|pwk}.}}}\label{K_L02733-3} zu ſein. Immerhin gib’ ihm
               die Autoriſation. Eine franzöſiſche Überſetzung\pwindex{Mourir. Roman@\emph{Mourir. Roman}|pwv}, die Du noch dazu nicht zu bezahlen brauchſt, iſt beſſer als gar
               keine. Mache aber aus, daß er die Sache\pwindex{Mourir. Roman@\emph{Mourir. Roman}|pwv} nicht veröffentlicht ohne daß Du die Überſetzung\pwindex{Mourir. Roman@\emph{Mourir. Roman}|pwv}{ }{\pb}geſehen und Deine Zuſtimmung gegeben haſt. Du wirſt
               ſie dann mir zuſenden, und wir werden ſehen.\pend
           
\pstart
           Die Idee, daß \textsc{Langen\pwindex{Langen, Albert 1869-07-08 – 1909-04-30@\textsc{Langen, Albert} (1869-07-08 – 1909-04-30), \emph{Verleger/Verlegerin}|pw}\orgindex{Albert Langen@Albert Langen|pw}} Deine \label{K_L02733-4v}\edtext{Novelle\pwindex{Sterben. Novelle von Arthur Schnitzler@\emph{Sterben. Novelle von Arthur Schnitzler}|pwv}}{\lemma{\textnormal{\emph{Novelle}}}\Cendnote{\textnormal{\emph{Sterben}\pwindex{Sterben. Novelle von Arthur Schnitzler@\emph{Sterben. Novelle von Arthur Schnitzler}|pwk} in französischer Übersetzung}}}\label{K_L02733-4}
               verlegen ſoll, iſt nicht übel. Laß’ mich nur machen. Vielleicht kommt übrigens der
                  Lausbube\pwindex{Langen, Albert 1869-07-08 – 1909-04-30@\textsc{Langen, Albert} (1869-07-08 – 1909-04-30), \emph{Verleger/Verlegerin}|pwv} nach \textsc{Wien\oindex{Wien@\textbf{Wien}, \emph{A.ADM2}|pw}}. \strikeout{D} Dann will ich Dir vorher Inſtruktionen
               geben.\pend
           
\pstart
           Grüß Dich Gott! {\\[\baselineskip]}Dein {\\[\baselineskip]}\spacefill\mbox{Paul Goldmann}\pend
           \leftskip=0em{}\selectlanguage{ngerman}\endnumbering\briefempfaengerindex{Schnitzler, Arthur@\textsc{Schnitzler, Arthur}!zzzGoldmann, Paul@\emph{von Paul Goldmann}!1895-04-031@{3. 4. {[}1895{]}}|)be}\mylabel{L02733h}  \normalsize

\doendnotes{C}
\bigskip
\vfill

\clearpage

\footnotesize

\lohead{\textsc{register}}

% Definiere theindex-Environment komplett neu ohne reledmac
\makeatletter
\renewenvironment{theindex}{%
  \section*{\indexname}%
  \setlength{\parindent}{0pt}%
  \setlength{\parskip}{0pt plus 0.3pt}%
  \let\item\@idxitem
}{%
  \clearpage
}
\makeatother

\IfFileExists{\jobname-pw.ind}{\input{\jobname-pw.ind}}{}

\end{document}

      