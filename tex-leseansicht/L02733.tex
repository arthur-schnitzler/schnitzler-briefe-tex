%% latex-leseansicht-vorspann.tex
%% Vorspann für die Leseansicht.
%% Lädt die gemeinsame Datei latex-vorspann.tex mit nicht gesetztem Schalter.

\newif\ifkorrekturansicht
\korrekturansichtfalse

\input{../tex-inputs/latex-vorspann}


\section[Paul Goldmann an Arthur Schnitzler, 3. 4. [1895]]{L02733 Paul Goldmann an Arthur Schnitzler, 3. 4. [1895]}
\nopagebreak\mylabel{L02733v}
\rehead{ }\normalsize\beginnumbering\briefempfaengerindex{Schnitzler, Arthur@\textsc{Schnitzler, Arthur}!zzzGoldmann, Paul@\emph{von Paul Goldmann}!1895-04-031@{3. 4. [1895]}|(be}
\toendnotes[C]{\smallbreak\pagebreak[2]}
\correspDesc{Versand  durch Paul Goldmann am 3. 4. [1895] in Paris
\newline{}Erhalt  durch Arthur Schnitzler im Zeitraum [4. 4. 1895
                  – 8. 4. 1895?] in Wien}\toendnotes[C]{\smallbreak}
\Standort{DLA, A:Schnitzler, HS.NZ85.1.3165.}
\physDesc{Brief, 1 Blatt, 2 Seiten, 747 Zeichen
\newline{}Handschrift: schwarze Tinte, deutsche Kurrent
\newline{}Schnitzler: 1) mit schwarzer Tinte das Jahr »95« vermerkt  2) mit rotem Buntstift eine Unterstreichung}\toendnotes[C]{\smallbreak}
\pstart
           {\pb}\textcolor{gray}{\textbf{\textbf{Frankfurter Zeitung\orgindex{Frankfurter Zeitung@Frankfurter Zeitung|pw}}}}\pend
           
\pstart
           \textcolor{gray}{\textbf{(\begin{otherlanguage}{french}Gazette de Francfort\end{otherlanguage}\orgindex{Frankfurter Zeitung@Frankfurter Zeitung|pw}).}}\pend
           
\pstart
           \textcolor{gray}{\textbf{\textbf{\begin{otherlanguage}{french}Fondateur M. L.
                              Sonnemann\pwindex{Sonnemann, Leopold 29.\,10.\,1831 Höchberg – 30.\,10.\,1909 Frankfurt am Main@\textsc{Sonnemann, Leopold} (29.\,10.\,1831 Höchberg – 30.\,10.\,1909 Frankfurt am Main), \emph{Journalist, Herausgeber}|pw}\end{otherlanguage}.}}}\pend
           
\pstart
           \begin{otherlanguage}{french}\textcolor{gray}{\textbf{Journal politique, financier,}}\end{otherlanguage}\pend
           
\pstart
           \begin{otherlanguage}{french}\textcolor{gray}{\textbf{commercial et littéraire.}}\end{otherlanguage}\pend
           
\pstart
           \begin{otherlanguage}{french}\textcolor{gray}{\textbf{\textbf{Paraissant trois fois par jour.}}}\end{otherlanguage}\hfill \textsc{Paris\oindex{Paris@\textbf{Paris}, \emph{Hauptstadt}|pw}}, 3. April.\pend
           
\pstart
           \begin{otherlanguage}{french}\textcolor{gray}{\textbf{\textbf{Bureau à Paris\oindex{Paris@\textbf{Paris}, \emph{Hauptstadt}|pw}:}}}\end{otherlanguage}\pend
           
\pstart
           \begin{otherlanguage}{french}\textcolor{gray}{\textbf{\textbf{24. Rue Feydeau\oindex{rue Feydeau@\textbf{rue Feydeau}, \emph{Straße}|pw}.}}}\end{otherlanguage}\pend
           
\pstart\center{}Mein lieber Freund,\pend\vspace{0.5em}
\pstart
           In Eile: Dieſen \label{K_L02733-1v}\edtext{Mann\pwindex{Vallette, Gaspard 13.\,5.\,1865 Jussy – 6.\,8.\,1911 La Tène@\textsc{Vallette, Gaspard} (13.\,5.\,1865 Jussy – 6.\,8.\,1911 La Tène), \emph{Journalist, Übersetzer}|pwuv}}{\lemma{\textnormal{\emph{Mann}}}\Cendnote{\textnormal{Es dürfte sich um Gaspard Vallette\pwindex{Vallette, Gaspard 13.\,5.\,1865 Jussy – 6.\,8.\,1911 La Tène@\textsc{Vallette, Gaspard} (13.\,5.\,1865 Jussy – 6.\,8.\,1911 La Tène), \emph{Journalist, Übersetzer}|pwk} handeln, der \emph{Sterben}\pwindex{Schnitzler, Arthur 15.\,5.\,1862 Wien – 21.\,10.\,1931 ebd.@\textsc{Schnitzler, Arthur} (15.\,5.\,1862 Wien – 21.\,10.\,1931 ebd.), \emph{Schriftsteller, Mediziner}!Sterben. Novelle@\strich\emph{Sterben. Novelle}|pwk} ins Französische übersetzte. Nur wenige Tage vor
                  der Entstehung dieses Briefs, am 31. 3. 1895, notierte Schnitzler die Anfrage zur Übersetzung\pwindex{Schnitzler, Arthur 15.\,5.\,1862 Wien – 21.\,10.\,1931 ebd.@\textsc{Schnitzler, Arthur} (15.\,5.\,1862 Wien – 21.\,10.\,1931 ebd.), \emph{Schriftsteller, Mediziner}!Mourir. Roman@\strich\emph{Mourir. Roman}|pwkv} im \emph{Tagebuch}\pwindex{Schnitzler, Arthur 15.\,5.\,1862 Wien – 21.\,10.\,1931 ebd.@\textsc{Schnitzler, Arthur} (15.\,5.\,1862 Wien – 21.\,10.\,1931 ebd.), \emph{Schriftsteller, Mediziner}!Tagebuch@\strich\emph{Tagebuch}|pwk}.}}}\label{K_L02733-1} in \textsc{Cannes\oindex{Cannes@\textbf{Cannes}|pw}} kenne ich nicht, und Niemand kennt ihn, den ich hier befragt. Die Adreſſe
               deutet auf einen \label{K_L02733-2v}\edtext{\textsc{\begin{otherlanguage}{french}homme cossu\end{otherlanguage}}}{\lemma{\textnormal{\emph{homme cossu}}}\Cendnote{\textnormal{französisch: wohlhabender Mann}}}\label{K_L02733-2}
               hin. Ob er Franzöſiſch kann? Denn es{ }ſcheint \label{K_L02733-3v}\edtext{kein Fran\oindex{Frankreich@\textbf{Frankreich}|pwv}zoſe}{\lemma{\textnormal{\emph{kein Franzose}}}\Cendnote{\textnormal{Vallette\pwindex{Vallette, Gaspard 13.\,5.\,1865 Jussy – 6.\,8.\,1911 La Tène@\textsc{Vallette, Gaspard} (13.\,5.\,1865 Jussy – 6.\,8.\,1911 La Tène), \emph{Journalist, Übersetzer}|pwk} war Schweizer\oindex{Schweiz@\textbf{Schweiz}|pwk}.}}}\label{K_L02733-3} zu{ }ſein. Immerhin gib’ ihm
               die Autoriſation. Eine franzöſiſche Überſetzung\pwindex{Schnitzler, Arthur 15.\,5.\,1862 Wien – 21.\,10.\,1931 ebd.@\textsc{Schnitzler, Arthur} (15.\,5.\,1862 Wien – 21.\,10.\,1931 ebd.), \emph{Schriftsteller, Mediziner}!Mourir. Roman@\strich\emph{Mourir. Roman}|pwv}, die Du noch dazu nicht zu bezahlen brauchſt, iſt beſſer als gar
               keine. Mache aber aus, daß er die Sache\pwindex{Schnitzler, Arthur 15.\,5.\,1862 Wien – 21.\,10.\,1931 ebd.@\textsc{Schnitzler, Arthur} (15.\,5.\,1862 Wien – 21.\,10.\,1931 ebd.), \emph{Schriftsteller, Mediziner}!Mourir. Roman@\strich\emph{Mourir. Roman}|pwv} nicht veröffentlicht ohne daß Du die Überſetzung\pwindex{Schnitzler, Arthur 15.\,5.\,1862 Wien – 21.\,10.\,1931 ebd.@\textsc{Schnitzler, Arthur} (15.\,5.\,1862 Wien – 21.\,10.\,1931 ebd.), \emph{Schriftsteller, Mediziner}!Mourir. Roman@\strich\emph{Mourir. Roman}|pwv}{ }{\pb}geſehen und Deine Zuſtimmung gegeben haſt. Du wirſt{ }ſie dann mir zuſenden, und wir werden{ }ſehen.\pend
           
\pstart
           Die Idee, daß \textsc{Langen\pwindex{Langen, Albert 8.\,7.\,1869 Antwerpen – 30.\,4.\,1909 München@\textsc{Langen, Albert} (8.\,7.\,1869 Antwerpen – 30.\,4.\,1909 München), \emph{Verleger}|pw}\orgindex{Albert Langen@Albert Langen|pw}} Deine \label{K_L02733-4v}\edtext{Novelle\pwindex{Sterben. Novelle von Arthur Schnitzler@\emph{Sterben. Novelle von Arthur Schnitzler}|pwv}}{\lemma{\textnormal{\emph{Novelle}}}\Cendnote{\textnormal{\emph{Sterben}\pwindex{Sterben. Novelle von Arthur Schnitzler@\emph{Sterben. Novelle von Arthur Schnitzler}|pwk} in französischer Übersetzung}}}\label{K_L02733-4}
               verlegen{ }ſoll, iſt nicht übel. Laß’ mich nur machen. Vielleicht kommt übrigens der
                  Lausbube\pwindex{Langen, Albert 8.\,7.\,1869 Antwerpen – 30.\,4.\,1909 München@\textsc{Langen, Albert} (8.\,7.\,1869 Antwerpen – 30.\,4.\,1909 München), \emph{Verleger}|pwv} nach \textsc{Wien\oindex{Wien@\textbf{Wien}, \emph{Verwaltungsgebiet}|pw}}. \strikeout{D} Dann will ich Dir vorher Inſtruktionen
               geben.\pend
           
\pstart
           Grüß Dich Gott! {\\[\baselineskip]}Dein {\\[\baselineskip]}\spacefill\mbox{Paul Goldmann}\pend
           \leftskip=0em{}\selectlanguage{ngerman}\endnumbering\briefempfaengerindex{Schnitzler, Arthur@\textsc{Schnitzler, Arthur}!zzzGoldmann, Paul@\emph{von Paul Goldmann}!1895-04-031@{3. 4. [1895]}|)be}\mylabel{L02733h}  \newcommand{\dateiname}{L02733}\newcommand{\titel}{Paul Goldmann an Arthur Schnitzler, 3. 4. [1895]}\newcommand{\editorInnen}{Martin Anton Müller und Laura Untner}%% latex-leseansicht-abspann.tex
%% Abspann für die Leseansicht.
%% Der Schalter \ifkorrekturansicht ist bereits durch den Vorspann gesetzt.

%% latex-abspann.tex
%% Gemeinsamer Abspann für Korrekturansicht und Leseansicht.
%% Setzt den Schalter \ifkorrekturansicht voraus (gesetzt in den
%% einbindenden Dateien latex-korrekturansicht-abspann.tex bzw.
%% latex-leseansicht-abspann.tex).
%% ---------------------------------------------------------------

\normalsize

% Das esempio-Environment wird nur in der Leseansicht benötigt
\ifkorrekturansicht\else
\newenvironment{esempio}[3]%
{
    \vspace{1.5ex}
    \rlap{\underline{#1}}
    \par
    \setlength{\parindent}{0cm}
    \nopagebreak
    \leftskip=#2cm
    \rightskip=#3cm
}
{
    \par
}
\fi

\doendnotes{C}
\bigskip
\vfill

\clearpage

\footnotesize

\ifkorrekturansicht
  \lohead{\textsc{register}}
\fi

% theindex-Environment neu definieren ohne reledmac
\makeatletter
\renewenvironment{theindex}{%
  \ifkorrekturansicht
    \section*{\indexname}%
  \else
    \subsubsection*{Index der erwähnten Entitäten}%
  \fi
  \setlength{\parindent}{0pt}%
  \setlength{\parskip}{0pt plus 0.3pt}%
  \let\item\@idxitem
}{%
  \ifkorrekturansicht\clearpage\fi
}
\makeatother

\IfFileExists{\jobname-pw.ind}{\input{\jobname-pw.ind}}{}

% Quellenangabe nur in der Leseansicht
\ifkorrekturansicht\else
% Fallback-Definitionen, falls die .tex-Datei \titel etc. nicht gesetzt hat
\providecommand{\titel}{}
\providecommand{\editorInnen}{}
\providecommand{\dateiname}{\jobname}

\vspace{3cm}

\vfill

\footnotesize
\textsc{Quelle}: \titel. Herausgegeben von {\editorInnen}. In: \emph{Arthur Schnitzler: Briefwechsel mit Autorinnen und Autoren}.
 Digitale Edition, https://schnitzler-briefe.acdh.oeaw.ac.at/{\dateiname}.html (Stand \today)
\fi

\end{document}


