%% latex-leseansicht-vorspann.tex
%% Vorspann für die Leseansicht.
%% Lädt die gemeinsame Datei latex-vorspann.tex mit nicht gesetztem Schalter.

\newif\ifkorrekturansicht
\korrekturansichtfalse

\input{../tex-inputs/latex-vorspann}


         
         \renewcommand{\erwaehntePersonen}{Personen: Jakob Wassermann}
         \renewcommand{\erwaehnteOrte}{Orte: Hohe Tauern, Pension Pundschu, Seeboden, Velden am Wörthersee}
         \renewcommand{\erwaehnteWerke}{Werke: Der Tod Georgs}
               \section[Richard Beer-Hofmann an Arthur Schnitzler, 18. 7. 1899]{ Richard Beer-Hofmann an Arthur Schnitzler, 18. 7. 1899}\nopagebreak\mylabel{v}\rehead{ }\begin{ledgroupsized}[t]{13cm}\normalsize\beginnumbering \toendnotes[C]{\smallbreak\pagebreak[2]} \Standort{CUL, Schnitzler, B 8.}
\physDesc{Kartenbrief
\newline{}Handschrift: Bleistift, lateinische Kurrent\newline{}Versand: 1) Stempel: »\nobreak{}\oindex{Seeboden@\textbf{Seeboden}|pwk}Seeboden, 18{[}. 7. 1899{]}\nobreak{}«.   2) Stempel: »\nobreak{}\oindex{Velden am Woerthersee@\textbf{Velden am Wörthersee}|pwk}Velden am Wörthersee, 19 7 99, 18.F\nobreak{}«. \newline{}Ordnung: mit Bleistift von unbekannter Hand nummeriert: »134« }\toendnotes[C]{\smallbreak}\pstart{}{\pb}D\textsuperscript{r}
                  Arthur Schnitzler\pend{}\pstart{}Velden a Wörthersee\oindex{Velden am Woerthersee@\textbf{Velden am Wörthersee}|pw}\pend{}\pstart{}Pension Pundschu\oindex{Pension Pundschu@\textbf{Pension Pundschu}|pw}\pend{}{\bigskip}\pstart
           \raggedleft{}{\pb}Seeboden\oindex{Seeboden@\textbf{Seeboden}|pw}{ }18/VII früh\pend
           \pstart
           Lieber Arthur! Ich hoffe Ende dieser Woche fertig zu werden. Auch
               wenn ich aber nicht fertig\pwindex{Beer-Hofmann, Richard 1866-07-11 – 1945-09-26@\textsc{Beer-Hofmann, Richard} (1866-07-11 – 1945-09-26), \emph{Schriftsteller}!Tod Georgs1900@\strich\emph{Der Tod Georgs} {[}1900{]}|pwv} bin
                  ko{\geminationm} ich Sonntag oder Montag
               zu Ihnen. Jedenfalls telegrafire ich früher.\pend
           \pstart
           Was unsere Tour anlangt, habe ich außer irgendeinem Tauern\oindex{Hohe Tauern@\textbf{Hohe Tauern}|pw}übergang keine besondere Wünsche.\pend
           \pstart
           Grüßen Sie Wassermann\pwindex{Wassermann, Jakob 10.03.1873 – 01.01.1934@\textsc{Wassermann, Jakob} (10.03.1873 – 01.01.1934), \emph{Schriftsteller}|pw}.\pend
           \pstart
           Herzlichst{\\[\baselineskip]}\spacefill\mbox{Richard}\pend
           \leftskip=0em{}
         
         \endnumbering\mylabel{h}\end{ledgroupsized}  \newcommand{\dateiname}{L00946}\newcommand{\titel}{Richard Beer-Hofmann an Arthur Schnitzler, 18. 7. 1899}\newcommand{\editorInnen}{Martin Anton Müller und Gerd-Hermann Susen}%% latex-leseansicht-abspann.tex
%% Abspann für die Leseansicht.
%% Der Schalter \ifkorrekturansicht ist bereits durch den Vorspann gesetzt.

%% latex-abspann.tex
%% Gemeinsamer Abspann für Korrekturansicht und Leseansicht.
%% Setzt den Schalter \ifkorrekturansicht voraus (gesetzt in den
%% einbindenden Dateien latex-korrekturansicht-abspann.tex bzw.
%% latex-leseansicht-abspann.tex).
%% ---------------------------------------------------------------

\normalsize

% Das esempio-Environment wird nur in der Leseansicht benötigt
\ifkorrekturansicht\else
\newenvironment{esempio}[3]%
{
    \vspace{1.5ex}
    \rlap{\underline{#1}}
    \par
    \setlength{\parindent}{0cm}
    \nopagebreak
    \leftskip=#2cm
    \rightskip=#3cm
}
{
    \par
}
\fi

\doendnotes{C}
\bigskip
\vfill

\clearpage

\footnotesize

\ifkorrekturansicht
  \lohead{\textsc{register}}
\fi

% theindex-Environment neu definieren ohne reledmac
\makeatletter
\renewenvironment{theindex}{%
  \ifkorrekturansicht
    \section*{\indexname}%
  \else
    \subsubsection*{Index der erwähnten Entitäten}%
  \fi
  \setlength{\parindent}{0pt}%
  \setlength{\parskip}{0pt plus 0.3pt}%
  \let\item\@idxitem
}{%
  \ifkorrekturansicht\clearpage\fi
}
\makeatother

\IfFileExists{\jobname-pw.ind}{\input{\jobname-pw.ind}}{}

% Quellenangabe nur in der Leseansicht
\ifkorrekturansicht\else
% Fallback-Definitionen, falls die .tex-Datei \titel etc. nicht gesetzt hat
\providecommand{\titel}{}
\providecommand{\editorInnen}{}
\providecommand{\dateiname}{\jobname}

\vspace{3cm}

\vfill

\footnotesize
\textsc{Quelle}: \titel. Herausgegeben von {\editorInnen}. In: \emph{Arthur Schnitzler: Briefwechsel mit Autorinnen und Autoren}.
 Digitale Edition, https://schnitzler-briefe.acdh.oeaw.ac.at/{\dateiname}.html (Stand \today)
\fi

\end{document}


      