%% latex-leseansicht-vorspann.tex
%% Vorspann für die Leseansicht.
%% Lädt die gemeinsame Datei latex-vorspann.tex mit nicht gesetztem Schalter.

\newif\ifkorrekturansicht
\korrekturansichtfalse

\input{../tex-inputs/latex-vorspann}


\section[Hugo von Hofmannsthal an Arthur Schnitzler, 11. 9. [1914]]{L02196 Hugo von Hofmannsthal an Arthur Schnitzler, 11. 9. [1914]}
\nopagebreak\mylabel{L02196v}
\rehead{ }\normalsize\beginnumbering\briefempfaengerindex{Schnitzler, Arthur@\textsc{Schnitzler, Arthur}!zzzHofmannsthal, Hugo von@\emph{von Hugo von Hofmannsthal}!1914-09-111@{11. 9. [1914]}|(be}
\toendnotes[C]{\smallbreak\pagebreak[2]}
\correspDesc{Versand  durch Hugo von Hofmannsthal am 11. 9. [1914] in Bad Aussee
\newline{}Erhalt  durch Arthur Schnitzler im Zeitraum [12. 9. 1914
                  – 16. 9. 1914?] in Wien}\toendnotes[C]{\smallbreak}
\Standort{CUL, Schnitzler, B 43.}
\physDesc{Brief, 1 Blatt, 2 Seiten, 678 Zeichen
\newline{}Handschrift: schwarze Tinte, deutsche Kurrent
\newline{}Schnitzler: 1) mit Bleistift beschriftet: »Hugo«  2) mit rotem Buntstift eine Unterstreichung
\newline{}Ordnung: 1) mit Bleistift von unbekannter Hand nummeriert: »\strikeout{336}«  2) mit Bleistift von unbekannter Hand nummeriert:
                                    »351«}
\buchAbdrucke{\weitereDrucke{Hugo von Hofmannsthal, Arthur Schnitzler: \emph{Briefwechsel}. Herausgegeben von Therese Nickl und Heinrich Schnitzler. Frankfurt am Main: \emph{S. Fischer} 1964, S. 276.} }\toendnotes[C]{\smallbreak}
\pstart
           \raggedleft{}{\pb}Auſſee\oindex{Bad Aussee@\textbf{Bad Aussee}, \emph{Hauptstadt}|pw}{ }11 IX.\pend
           
\pstart{}lieber Arthur\pend\vspace{0.5em}
\pstart
           ich bin für 2–3 Tage hier, dann wieder Eliſabethſtraße\oindex{Wien@\textbf{Wien}!I., Innere Stadt@\textbf{I., Innere Stadt}!Elisabethstraße [Wien]@\textbf{Elisabethstraße [Wien]}, \emph{Straße}|pw}.\hspace*{1.5em}Ich weiß daſs Sie{ }ſchon
               größere Beträge fürs rote Kreuz\orgindex{Internationales Komitee vom Roten Kreuz@Internationales Komitee vom Roten Kreuz|pw} gegeben haben,
               aber \uline{bitte} geben Sie nun noch etwas und das{ }ſogleich
               für die Rettungsgeſellſchaft\orgindex{Wiener freiwillige Rettungsgesellschaft@Wiener freiwillige Rettungsgesellschaft|pw}, die vorzügliches
               leiſtet und dringend Hilfe braucht und bitte geben Sie es \label{K_L02196-1v}\edtext{durch die \textsc{Neue Freie Presse}\orgindex{Neue Freie Presse@Neue Freie Presse|pw}}{\lemma{\textnormal{\emph{durch … Presse}}}\Cendnote{\textnormal{Am 10. 9. 1914 erschien ein
                     »Erster Spendenausweis« der Sammlung, die 819 Kronen nachwies,
                  wobei jeweils 200 von Hofmannsthal\pwindex{Hofmannsthal, Hugo von 1.\,2.\,1874 Wien – 15.\,7.\,1929 Rodaun@\textsc{Hofmannsthal, Hugo von} (1.\,2.\,1874 Wien – 15.\,7.\,1929 Rodaun), \emph{Schriftsteller}|pwk} und
                  seinem Vater\pwindex{Hofmannsthal, Hugo August von 21.\,12.\,1841 Wien – 8.\,12.\,1915 ebd.@\textsc{Hofmannsthal, Hugo August von} (21.\,12.\,1841 Wien – 8.\,12.\,1915 ebd.), \emph{Bankdirektor}|pwkv} stammten (\emph{Neue Freie Presse}\orgindex{Neue Freie Presse@Neue Freie Presse|pwk}, Nr. 17.976, S. 7).
                  In den Folgetagen wurden weitere Spenden ausgewiesen, aber keine von Schnitzler.}}}\label{K_L02196-1}, das zieht wieder andere
                  {\pb}Leute mit, deshalb gab ich
               auch dort, gab nur einen kleinem Beitrag \introOben{}(200)\introOben{}, um mehrmals
               wieder geben zu können, es wird noch allſeits viel zu wenig gegeben, es iſt ein Meer
               von Not und Schwierigkeiten.\pend
           
\pstart
           Ich bitte Sie und Olga\pwindex{Schnitzler, Olga 17.\,1.\,1882 Wien – 13.\,1.\,1970 Lugano@\textsc{Schnitzler, Olga} (17.\,1.\,1882 Wien – 13.\,1.\,1970 Lugano), \emph{Schauspielerin, Sängerin}|pw}, dies unter Euren
               Bekannten \label{K_L02196-2v}\edtext{weiterzuſagen}{\lemma{\textnormal{\emph{weiterzusagen}}}\Cendnote{\textnormal{Am 19. 9. 1914 wurde eine
                  Spende von 300 Kronen durch Paula
                     Beer-Hofmann\pwindex{Beer-Hofmann, Paula 25.\,2.\,1879 Wien – 30.\,10.\,1939 Zürich@\textsc{Beer-Hofmann, Paula} (25.\,2.\,1879 Wien – 30.\,10.\,1939 Zürich)|pwk} ausgewiesen (\emph{Neue Freie Presse}\orgindex{Neue Freie Presse@Neue Freie Presse|pwk}, Nr. 17.985,
                  S. 5).}}}\label{K_L02196-2}, es iſt eine der dringendſten Notwendigkeiten.\pend
           
\pstart
           Von Herzen{\\[\baselineskip]}\spacefill\mbox{Hugo.}\pend
           \leftskip=0em{}\selectlanguage{ngerman}\endnumbering\briefempfaengerindex{Schnitzler, Arthur@\textsc{Schnitzler, Arthur}!zzzHofmannsthal, Hugo von@\emph{von Hugo von Hofmannsthal}!1914-09-111@{11. 9. [1914]}|)be}\mylabel{L02196h}  \newcommand{\dateiname}{L02196}\newcommand{\titel}{Hugo von Hofmannsthal an Arthur Schnitzler, 11. 9. [1914]}\newcommand{\editorInnen}{Martin Anton Müller und Gerd-Hermann Susen}%% latex-leseansicht-abspann.tex
%% Abspann für die Leseansicht.
%% Der Schalter \ifkorrekturansicht ist bereits durch den Vorspann gesetzt.

%% latex-abspann.tex
%% Gemeinsamer Abspann für Korrekturansicht und Leseansicht.
%% Setzt den Schalter \ifkorrekturansicht voraus (gesetzt in den
%% einbindenden Dateien latex-korrekturansicht-abspann.tex bzw.
%% latex-leseansicht-abspann.tex).
%% ---------------------------------------------------------------

\normalsize

% Das esempio-Environment wird nur in der Leseansicht benötigt
\ifkorrekturansicht\else
\newenvironment{esempio}[3]%
{
    \vspace{1.5ex}
    \rlap{\underline{#1}}
    \par
    \setlength{\parindent}{0cm}
    \nopagebreak
    \leftskip=#2cm
    \rightskip=#3cm
}
{
    \par
}
\fi

\doendnotes{C}
\bigskip
\vfill

\clearpage

\footnotesize

\ifkorrekturansicht
  \lohead{\textsc{register}}
\fi

% theindex-Environment neu definieren ohne reledmac
\makeatletter
\renewenvironment{theindex}{%
  \ifkorrekturansicht
    \section*{\indexname}%
  \else
    \subsubsection*{Index der erwähnten Entitäten}%
  \fi
  \setlength{\parindent}{0pt}%
  \setlength{\parskip}{0pt plus 0.3pt}%
  \let\item\@idxitem
}{%
  \ifkorrekturansicht\clearpage\fi
}
\makeatother

\IfFileExists{\jobname-pw.ind}{\input{\jobname-pw.ind}}{}

% Quellenangabe nur in der Leseansicht
\ifkorrekturansicht\else
% Fallback-Definitionen, falls die .tex-Datei \titel etc. nicht gesetzt hat
\providecommand{\titel}{}
\providecommand{\editorInnen}{}
\providecommand{\dateiname}{\jobname}

\vspace{3cm}

\vfill

\footnotesize
\textsc{Quelle}: \titel. Herausgegeben von {\editorInnen}. In: \emph{Arthur Schnitzler: Briefwechsel mit Autorinnen und Autoren}.
 Digitale Edition, https://schnitzler-briefe.acdh.oeaw.ac.at/{\dateiname}.html (Stand \today)
\fi

\end{document}


