\input{../tex-inputs/latex-pdf-vorspann}
\begin{center}
            \textcolor{red}{ENTWURF. ENTZIFFERUNG NOCH NICHT KORREKTURGELESEN}
                      \end{center}
            
               \section[Hugo von Hofmannsthal an Arthur Schnitzler, 11. 9. {[}1914{]}]{ Hugo von Hofmannsthal an Arthur Schnitzler, 11. 9. {[}1914{]}}\nopagebreak\mylabel{v}\rehead{ }\begin{ledgroupsized}[t]{13cm}\normalsize\beginnumbering\briefempfaengerindex{Schnitzler, Arthur@\textsc{Schnitzler, Arthur}!zzzHofmannsthal, Hugo von@\emph{von Hugo von Hofmannsthal}!1914-09-111@{11. 9. {[}1914{]}}|(be} \toendnotes[C]{\smallbreak\pagebreak[2]} \Standort{CUL, Schnitzler, B 43.}
\physDesc{Brief, 1 Blatt, 2 Seiten
\newline{}Handschrift: schwarze Tinte, deutsche Kurrent
\newline{}Schnitzler: 1) mit Bleistift beschriftet: »Hugo« 2) mit rotem Buntstift eine Unterstreichung\newline{}Ordnung: 1) mit Bleistift von unbekannter Hand nummeriert: »\strikeout{336}« 2) mit Bleistift von unbekannter Hand nummeriert:
                                    »351«}\buchAbdrucke{\weitereDrucke{Hugo von Hofmannsthal, Arthur Schnitzler: \emph{Briefwechsel}. Hg. Therese Nickl und Heinrich Schnitzler. Frankfurt am Main: \emph{S. Fischer} 1964, S. 276.} }\toendnotes[C]{\smallbreak}\pstart
           \raggedleft{}{\pb}Auſſee\oindex{Bad Aussee@\textbf{Bad Aussee}|pw}{ }11 IX.\pend
           \pstart{}lieber Arthur\pend\pstart
           ich bin für 2–3 Tage hier, dann wieder Eliſabethſtraße\oindex{Elisabethstrasse@\textbf{Elisabethstraße}|pw}.\hspace*{1.5em}Ich weiß daſs Sie ſchon
               größere Beträge fürs rote Kreuz\orgindex{Internationales Komitee vom Roten Kreuz@Internationales Komitee vom Roten Kreuz|pw} gegeben haben, aber
                  \uline{bitte} geben Sie nun noch etwas und das ſogleich
               für die Rettungsgeſellſchaft\orgindex{Wiener freiwillige Rettungsgesellschaft@Wiener freiwillige Rettungsgesellschaft|pw}, die vorzügliches
               leiſtet und dringend Hilfe braucht und bitte geben Sie es \label{K_L02196_1v}\edtext{durch die \textsc{Neue Freie Presse}\orgindex{Neue Freie Presse@Neue Freie Presse|pw}}{\lemma{\textnormal{\emph{durch … Presse}}}\Cendnote{\textnormal{Am 10. 9. 1914 erschien ein
                     »Erster Spendenausweis« der Sammlung, die 819 Kronen nachwies,
                  wobei jeweils 200 von Hofmannsthal\pwindex{Hofmannsthal, Hugo von 01.02.1874 – 15.07.1929@\textsc{Hofmannsthal, Hugo von} (01.02.1874 – 15.07.1929), \emph{Schriftsteller}|pwk} und seinem
                     Vater\pwindex{Hofmannsthal, Hugo August von 21.12.1841 – 08.12.1915@\textsc{Hofmannsthal, Hugo August von} (21.12.1841 – 08.12.1915), \emph{Bankdirektor}|pwkv} stammten (\emph{Neue Freie Presse}\orgindex{Neue Freie Presse@Neue Freie Presse|pwk}, Nr. 17976, S. 7). In
                  den Folgetagen wurden weitere Spenden ausgwiesen, aber keine von Schnitzler\pwindex{Schnitzler, Arthur 15.05.1862 – 21.10.1931@\textsc{Schnitzler, Arthur} (15.05.1862 – 21.10.1931), \emph{Schriftsteller, Mediziner}|pwk}.}}}\label{K_L02196_1h}, das zieht wieder andere {\pb}Leute mit, deshalb gab ich auch
               dort, gab nur einen kleinem Beitrag \introOben{}(200)\introOben{}, um mehrmals
               wieder geben zu können, es wird noch allſeits viel zu wenig gegeben, es iſt ein Meer
               von Not und Schwierigkeiten.\pend
           \pstart
           Ich bitte Sie und Olga\pwindex{Schnitzler, Olga 17.01.1882 – 13.01.1970@\textsc{Schnitzler, Olga} (17.01.1882 – 13.01.1970), \emph{Schauspielerin, Sängerin}|pw}, dies unter Euren Bekannten
                  \label{K_L02196_2v}\edtext{weiterzuſagen}{\lemma{\textnormal{\emph{weiterzuſagen}}}\Cendnote{\textnormal{Am 19. 9. 1914 wird eine
                  Spende von 300 Kronen durch Paula Beer-Hofmann\pwindex{Beer-Hofmann, Paula 25.02.1879 – 30.10.1939@\textsc{Beer-Hofmann, Paula} (25.02.1879 – 30.10.1939)|pwk}
                  ausgewiesen (\emph{Neue Freie Presse}\orgindex{Neue Freie Presse@Neue Freie Presse|pwk}, Nr. 17985,
                  S. 5).}}}\label{K_L02196_2h}, es iſt eine der dringendſten Notwendigkeiten.\pend
           \pstart
           Von Herzen{\\[\baselineskip]}\spacefill\mbox{Hugo.}\pend
           \leftskip=0em{}\endnumbering\briefempfaengerindex{Schnitzler, Arthur@\textsc{Schnitzler, Arthur}!zzzHofmannsthal, Hugo von@\emph{von Hugo von Hofmannsthal}!1914-09-111@{11. 9. {[}1914{]}}|)be}\mylabel{h}\end{ledgroupsized}  \newcommand{\dateiname}{L02196}\newcommand{\titel}{Hugo von Hofmannsthal an Arthur Schnitzler, 11. 9. [1914]}\newcommand{\editorInnen}{Martin Anton Müller und Gerd-Hermann Susen}\input{../tex-inputs/latex-pdf-abspann}
      