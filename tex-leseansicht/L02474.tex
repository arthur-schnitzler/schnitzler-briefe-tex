%% latex-leseansicht-vorspann.tex
%% Vorspann für die Leseansicht.
%% Lädt die gemeinsame Datei latex-vorspann.tex mit nicht gesetztem Schalter.

\newif\ifkorrekturansicht
\korrekturansichtfalse

\input{../tex-inputs/latex-vorspann}


\section[Thomas und Katia Mann an Arthur Schnitzler, 23. 5. 1926]{L02474 Thomas und Katia Mann an Arthur Schnitzler, 23. 5. 1926}
\nopagebreak\mylabel{L02474v}
\rehead{ }\normalsize\beginnumbering\briefempfaengerindex{Schnitzler, Arthur@\textsc{Schnitzler, Arthur}!zzzMann, Katia@\emph{von Katia Mann}!1926-05-231@{23. 5. 1926}|(be}\briefempfaengerindex{Schnitzler, Arthur@\textsc{Schnitzler, Arthur}!zzzMann, Thomas@\emph{von Thomas Mann}!1926-05-231@{23. 5. 1926}|(be}
\toendnotes[C]{\smallbreak\pagebreak[2]}
\correspDesc{Versand  durch Thomas Mann, Katia Mann am 23. 5. 1926 in Arosa
\newline{}Erhalt  durch Arthur Schnitzler im Zeitraum [24. 5. 1926
                  – 28. 5. 1926?] in Wien}\toendnotes[C]{\smallbreak}
\Standort{CUL, Schnitzler, B 67.}
\physDesc{Bildpostkarte, 236 Zeichen
\newline{}Handschrift: schwarze Tinte, deutsche Kurrent
\newline{}Versand: Stempel: »\nobreak{}\oindex{Arosa@\textbf{Arosa}|pwk}Arosa, 24. V. 26, 12\nobreak{}«.  
\newline{}Schnitzler: mit rotem Buntstift eine Unterstreichung }
\buchAbdrucke{\weitereDrucke{Hertha Krotkoff: \emph{Arthur Schnitzler – Thomas Mann: Briefe.} In: \emph{Modern Austrian Literature}, Jg. 7 (1974) Nr. 1/2, S. 24.} }\pstart{}\textsc{{\pb}Herrn}\pend{}\pstart{}\textsc{Dr. Arthur Schnitzler}\pend{}\pstart{}\textsc{Wien\oindex{Wien@\textbf{Wien}, \emph{Verwaltungsgebiet}|pw}}\pend{}\pstart{}\textsc{Sternwartstr.\oindex{Wien@\textbf{Wien}!XVIII., Währing@\textbf{XVIII., Währing}!Sternwartestraße 71@\textbf{Sternwartestraße 71}, \emph{Wohngebäude}|pw}}\pend{}{\bigskip}
\pstart
           \noindent{}\centering{}\textcolor{gray}{\textbf{{\pb}Arosa\oindex{Arosa@\textbf{Arosa}|pw}. Bergkirchli\oindex{Bergkirchli@\textbf{Bergkirchli}, \emph{Kirche}|pw} m. Weisshorn\oindex{Aroser Weisshorn@\textbf{Aroser Weisshorn}, \emph{Bergspitze}|pw}.}}\pend
           \vspace{1em}
\pstart
           {\pb}Aroſa\oindex{Arosa@\textbf{Arosa}|pw} den 23. V. 26.\pend
           
\pstart{}Lieber und verehrter Dr. Schnitzler,\pend\vspace{0.5em}
\pstart
           hingeriſſen haben wir die Traumnovelle\pwindex{Schnitzler, Arthur 15.\,5.\,1862 Wien – 21.\,10.\,1931 ebd.@\textsc{Schnitzler, Arthur} (15.\,5.\,1862 Wien – 21.\,10.\,1931 ebd.), \emph{Schriftsteller, Mediziner}!Traumnovelle@\strich\emph{Traumnovelle}|pw} hier
               geleſen, beide in einem Zuge, atemlos, und begrüßen Sie voller Bewunderung.\pend
           \pstart \spacefill\mbox{Thomas und Katja Mann.}\pend{}\selectlanguage{ngerman}\endnumbering\briefempfaengerindex{Schnitzler, Arthur@\textsc{Schnitzler, Arthur}!zzzMann, Katia@\emph{von Katia Mann}!1926-05-231@{23. 5. 1926}|)be}\briefempfaengerindex{Schnitzler, Arthur@\textsc{Schnitzler, Arthur}!zzzMann, Thomas@\emph{von Thomas Mann}!1926-05-231@{23. 5. 1926}|)be}\mylabel{L02474h}  \newcommand{\dateiname}{L02474}\newcommand{\titel}{Thomas und Katia Mann an Arthur Schnitzler, 23. 5. 1926}\newcommand{\editorInnen}{Martin Anton Müller und Gerd-Hermann Susen}%% latex-leseansicht-abspann.tex
%% Abspann für die Leseansicht.
%% Der Schalter \ifkorrekturansicht ist bereits durch den Vorspann gesetzt.

%% latex-abspann.tex
%% Gemeinsamer Abspann für Korrekturansicht und Leseansicht.
%% Setzt den Schalter \ifkorrekturansicht voraus (gesetzt in den
%% einbindenden Dateien latex-korrekturansicht-abspann.tex bzw.
%% latex-leseansicht-abspann.tex).
%% ---------------------------------------------------------------

\normalsize

% Das esempio-Environment wird nur in der Leseansicht benötigt
\ifkorrekturansicht\else
\newenvironment{esempio}[3]%
{
    \vspace{1.5ex}
    \rlap{\underline{#1}}
    \par
    \setlength{\parindent}{0cm}
    \nopagebreak
    \leftskip=#2cm
    \rightskip=#3cm
}
{
    \par
}
\fi

\doendnotes{C}
\bigskip
\vfill

\clearpage

\footnotesize

\ifkorrekturansicht
  \lohead{\textsc{register}}
\fi

% theindex-Environment neu definieren ohne reledmac
\makeatletter
\renewenvironment{theindex}{%
  \ifkorrekturansicht
    \section*{\indexname}%
  \else
    \subsubsection*{Index der erwähnten Entitäten}%
  \fi
  \setlength{\parindent}{0pt}%
  \setlength{\parskip}{0pt plus 0.3pt}%
  \let\item\@idxitem
}{%
  \ifkorrekturansicht\clearpage\fi
}
\makeatother

\IfFileExists{\jobname-pw.ind}{\input{\jobname-pw.ind}}{}

% Quellenangabe nur in der Leseansicht
\ifkorrekturansicht\else
% Fallback-Definitionen, falls die .tex-Datei \titel etc. nicht gesetzt hat
\providecommand{\titel}{}
\providecommand{\editorInnen}{}
\providecommand{\dateiname}{\jobname}

\vspace{3cm}

\vfill

\footnotesize
\textsc{Quelle}: \titel. Herausgegeben von {\editorInnen}. In: \emph{Arthur Schnitzler: Briefwechsel mit Autorinnen und Autoren}.
 Digitale Edition, https://schnitzler-briefe.acdh.oeaw.ac.at/{\dateiname}.html (Stand \today)
\fi

\end{document}


