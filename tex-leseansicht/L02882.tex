%% latex-korrekturansicht-vorspann.tex
%% Vorspann für die Korrekturansicht.
%% Lädt die gemeinsame Datei latex-vorspann.tex mit gesetztem Schalter.

\newif\ifkorrekturansicht
\korrekturansichttrue

\input{../tex-inputs/latex-vorspann}


\section[ Paul Goldmann an Arthur Schnitzler, 3. 8. {[}1899{]}]{L02882 Paul Goldmann an Arthur Schnitzler, 3. 8. {[}1899{]}}
\nopagebreak\mylabel{L02882v}
\rehead{ }\normalsize\beginnumbering\briefempfaengerindex{Schnitzler, Arthur@\textsc{Schnitzler, Arthur}!zzzGoldmann, Paul@\emph{von Paul Goldmann}!1899-08-031@{3. 8. {[}1899{]}}|(be}
\toendnotes[C]{\smallbreak\pagebreak[2]}\Standort{DLA, A:Schnitzler, HS.NZ85.1.3169.}
\physDesc{Brief, 1 Blatt, 1 Seite, 256 Zeichen
\newline{}Handschrift: schwarze Tinte, deutsche Kurrent
\newline{}Schnitzler: mit Bleistift das Jahr »99.« vermerkt }\toendnotes[C]{\smallbreak}
\pstart
           \raggedleft{}{\pb}Frankfurt\oindex{Frankfurt am Main@\textbf{Frankfurt am Main}, \emph{P.PPLA3}|pw}{ }3. Auguſt.\pend
           
\pstart{}Mein lieber Freund,\pend\vspace{0.5em}
\pstart
           Haſt Du meinen letzten \label{K_L02882-1v}\edtext{Brief}{\lemma{\textnormal{\emph{Brief}}}\Cendnote{\textnormal{Paul Goldmann an Arthur Schnitzler, 27. 7. [1899].
               }}}\label{K_L02882-1} erhalten?\pend
           
\pstart
           Wie geht es Dir?\pend
           
\pstart
           Heut fahre ich nach \textsc{Rennes\oindex{Rennes@\textbf{Rennes}, \emph{P.PPLA}|pw}}. Briefe erreichen mich dort unter der Adreſſe: \textsc{Hôtel de France\oindex{Hôtel de France@\textbf{Hôtel de France}, \emph{Hotel (K.HTL)}|pw}}.\pend
           
\pstart
           Werden wir uns im \label{K_L02882-2v}\edtext{September oder Oktober}{\lemma{\textnormal{\emph{September oder Oktober}}}\Cendnote{\textnormal{Sie sahen sich zuerst in Frankfurt am Main\oindex{Frankfurt am Main@\textbf{Frankfurt am Main}, \emph{P.PPLA3}|pwk}, wo sich Schnitzler vom 19. 9. 1899 bis zum 24. 9. 1899 aufhielt. Vom 13. 10. 1899 bis zum 21. 10. 1899 war Goldmann\pwindex{Goldmann, Paul 31.01.1865 – 25.09.1935@\textsc{Goldmann, Paul} (31.01.1865 – 25.09.1935), \emph{Schriftsteller/Schriftstellerin, Journalist/Journalistin}|pwk} in Wien\oindex{Wien@\textbf{Wien}, \emph{A.ADM2}|pwk}.}}}\label{K_L02882-2} ſehen?\pend
           
\pstart
           Viele treue Grüße! {\\[\baselineskip]}Dein {\\[\baselineskip]}\spacefill\mbox{Paul Goldmann}\pend
           \leftskip=0em{}\selectlanguage{ngerman}\endnumbering\briefempfaengerindex{Schnitzler, Arthur@\textsc{Schnitzler, Arthur}!zzzGoldmann, Paul@\emph{von Paul Goldmann}!1899-08-031@{3. 8. {[}1899{]}}|)be}\mylabel{L02882h}  \normalsize

\doendnotes{C}
\bigskip
\vfill

\clearpage

\footnotesize

\lohead{\textsc{register}}

% Definiere theindex-Environment komplett neu ohne reledmac
\makeatletter
\renewenvironment{theindex}{%
  \section*{\indexname}%
  \setlength{\parindent}{0pt}%
  \setlength{\parskip}{0pt plus 0.3pt}%
  \let\item\@idxitem
}{%
  \clearpage
}
\makeatother

\IfFileExists{\jobname-pw.ind}{\input{\jobname-pw.ind}}{}

\end{document}

      