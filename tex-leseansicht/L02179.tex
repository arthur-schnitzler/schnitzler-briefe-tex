%% latex-korrekturansicht-vorspann.tex
%% Vorspann für die Korrekturansicht.
%% Lädt die gemeinsame Datei latex-vorspann.tex mit gesetztem Schalter.

\newif\ifkorrekturansicht
\korrekturansichttrue

\input{../tex-inputs/latex-vorspann}


\section[Bertha von Suttner an Arthur und Olga Schnitzler, 14. 5. 1914]{L02179 Bertha von Suttner an Arthur und Olga Schnitzler, 14. 5. 1914}
\nopagebreak\mylabel{L02179v}
\rehead{ }\normalsize\beginnumbering\briefempfaengerindex{Schnitzler, Olga@\textsc{Schnitzler, Olga}!zzzSuttner, Bertha von@\emph{von Bertha von Suttner}!1914-05-141@{14. 5. 1914}|(be}\briefempfaengerindex{Schnitzler, Arthur@\textsc{Schnitzler, Arthur}!zzzSuttner, Bertha von@\emph{von Bertha von Suttner}!1914-05-141@{14. 5. 1914}|(be}
\toendnotes[C]{\smallbreak\pagebreak[2]}\Standort{CUL, Schnitzler, B 104.}
\physDesc{Brief, 1 Blatt, 1 Seite, 519 Zeichen
\newline{}Handschrift: schwarze Tinte, lateinische Kurrent
\newline{}Schnitzler: mit rotem Buntstift eine Unterstreichung }\Standort{DLA, A:Schnitzler, HS.NZ85.1.4773.}
\physDesc{maschinenschriftliche Abschrift1 Blatt, 1 Seite, 519 Zeichen
\newline{}Schreibmaschine}\toendnotes[C]{\smallbreak}
\pstart
           \centering{}{\pb}\textcolor{gray}{\textbf{XXI. WELTFRIEDENS-KONGRESS}}\pend
           
\pstart
           \centering{}\textcolor{gray}{\textbf{WIEN\oindex{Wien@\textbf{Wien}, \emph{A.ADM2}|pw}, September 1914}}\pend
           
\pstart
           \raggedleft{}\textcolor{gray}{\textbf{WIEN I.\oindex{I., Innere Stadt@\textbf{I., Innere Stadt}, \emph{A.ADM3}|pw},..........191{\dots}}}\pend
           
\pstart
           \raggedleft{}\textcolor{gray}{\textbf{SPIEGELGASSE 4\oindex{Spiegelgasse@\textbf{Spiegelgasse}, \emph{Straße (K.STR)}|pw}}}\pend
           
\pstart
           \raggedleft{}14/5 191\substVorne{}\textsuperscript{3}\substDazwischen{}4\substHinten{}\pend
           
\pstart{}Hochgeehrter D\textsuperscript{r.} Schnitzler und liebwerte Frau
                     D\textsuperscript{r.}\pend\vspace{0.5em}
\pstart
           Darf ich Sie nun in aller Form und herzlichst einladen und bitten, sich als
               Teilnehmer unseres \label{K_L02179-1v}\edtext{Congresses}{\lemma{\textnormal{\emph{Congresses}}}\Cendnote{\textnormal{Dieser fand nicht statt. Suttner\pwindex{Suttner, Bertha von 09.06.1843 – 21.06.1914@\textsc{Suttner, Bertha von} (09.06.1843 – 21.06.1914), \emph{Schriftsteller/Schriftstellerin, Pazifist/Pazifistin, Schriftsteller/Schriftstellerin}|pwk} starb am 21. 6. 1914. Auch der
                  Kriegsausbruch machte eine Abhaltung unmöglich.}}}\label{K_L02179-1} anzusagen? Die erlauchten
               Spitzen unserer Geisteswelt sollen doch bei für die Kultur kämpfenden Veranstaltung
               vertreten sein.\pend
           
\pstart
           Würden Sie im letzten Augenblick verhindert (was ich ich nicht hoffe) so wäre schon
               Ihr Name in der Teilnehmerliste uns eine höchst wertvolle Förderung.\pend
           
\pstart
           Mit vielen Grüssen – und g\textcolor{gray}{ewär}tige Erledigung
               hoffend{\\[\baselineskip]}Ihre erg.{\\[\baselineskip]}\spacefill\mbox{Bertha v. Suttner}\pend
           \leftskip=0em{}\selectlanguage{ngerman}\endnumbering\briefempfaengerindex{Schnitzler, Olga@\textsc{Schnitzler, Olga}!zzzSuttner, Bertha von@\emph{von Bertha von Suttner}!1914-05-141@{14. 5. 1914}|)be}\briefempfaengerindex{Schnitzler, Arthur@\textsc{Schnitzler, Arthur}!zzzSuttner, Bertha von@\emph{von Bertha von Suttner}!1914-05-141@{14. 5. 1914}|)be}\mylabel{L02179h}  \normalsize

\doendnotes{C}
\bigskip
\vfill

\clearpage

\footnotesize

\lohead{\textsc{register}}

% Definiere theindex-Environment komplett neu ohne reledmac
\makeatletter
\renewenvironment{theindex}{%
  \section*{\indexname}%
  \setlength{\parindent}{0pt}%
  \setlength{\parskip}{0pt plus 0.3pt}%
  \let\item\@idxitem
}{%
  \clearpage
}
\makeatother

\IfFileExists{\jobname-pw.ind}{\input{\jobname-pw.ind}}{}

\end{document}

      