%% latex-leseansicht-vorspann.tex
%% Vorspann für die Leseansicht.
%% Lädt die gemeinsame Datei latex-vorspann.tex mit nicht gesetztem Schalter.

\newif\ifkorrekturansicht
\korrekturansichtfalse

\input{../tex-inputs/latex-vorspann}


\section[Bertha von Suttner an Arthur und Olga Schnitzler, 14. 5. 1914]{L02179 Bertha von Suttner an Arthur und Olga Schnitzler, 14. 5. 1914}
\nopagebreak\mylabel{L02179v}
\rehead{ }\normalsize\beginnumbering\briefempfaengerindex{Schnitzler, Olga@\textsc{Schnitzler, Olga}!zzzSuttner, Bertha von@\emph{von Bertha von Suttner}!1914-05-141@{14. 5. 1914}|(be}\briefempfaengerindex{Schnitzler, Arthur@\textsc{Schnitzler, Arthur}!zzzSuttner, Bertha von@\emph{von Bertha von Suttner}!1914-05-141@{14. 5. 1914}|(be}
\toendnotes[C]{\smallbreak\pagebreak[2]}
\correspDesc{Versand  durch Bertha von Suttner am 14. 5. 1914 in Wien
\newline{}Erhalt  durch Arthur Schnitzler, Olga Schnitzler im Zeitraum [14. 5. 1914
                  – 18. 5. 1914?] in Wien}\toendnotes[C]{\smallbreak}
\Standort{CUL, Schnitzler, B 104.}
\physDesc{Brief, 1 Blatt, 1 Seite, 519 Zeichen
\newline{}Handschrift: schwarze Tinte, lateinische Kurrent
\newline{}Schnitzler: mit rotem Buntstift eine Unterstreichung }\Standort{DLA, A:Schnitzler, HS.NZ85.1.4773.}
\physDesc{maschinenschriftliche Abschrift, 1 Blatt, 1 Seite, 519 Zeichen
\newline{}Schreibmaschine}\toendnotes[C]{\smallbreak}
\pstart
           \centering{}{\pb}\textcolor{gray}{\textbf{XXI. WELTFRIEDENS-KONGRESS}}\pend
           
\pstart
           \centering{}\textcolor{gray}{\textbf{WIEN\oindex{Wien@\textbf{Wien}, \emph{Verwaltungsgebiet}|pw}, September 1914}}\pend
           
\pstart
           \raggedleft{}\textcolor{gray}{\textbf{WIEN I.\oindex{I., Innere Stadt@\textbf{I., Innere Stadt}, \emph{Verwaltungsgebiet}|pw},..........191{\dots}}}\pend
           
\pstart
           \raggedleft{}\textcolor{gray}{\textbf{SPIEGELGASSE 4\oindex{Wien@\textbf{Wien}!I., Innere Stadt@\textbf{I., Innere Stadt}!Spiegelgasse@\textbf{Spiegelgasse}, \emph{Straße}|pw}}}\pend
           
\pstart
           \raggedleft{}14/5 191\substVorne{}\textsuperscript{3}\substDazwischen{}4\substHinten{}\pend
           
\pstart{}Hochgeehrter D\textsuperscript{r.} Schnitzler und liebwerte Frau
                     D\textsuperscript{r.}\pend\vspace{0.5em}
\pstart
           Darf ich Sie nun in aller Form und herzlichst einladen und bitten, sich als
               Teilnehmer unseres \label{K_L02179-1v}\edtext{Congresses}{\lemma{\textnormal{\emph{Congresses}}}\Cendnote{\textnormal{Dieser fand nicht statt. Suttner\pwindex{Suttner, Bertha von 9.\,6.\,1843 Prag – 21.\,6.\,1914 Wien@\textsc{Suttner, Bertha von} (9.\,6.\,1843 Prag – 21.\,6.\,1914 Wien), \emph{Schriftstellerin, Pazifistin, Schriftstellerin}|pwk} starb am 21. 6. 1914. Auch der
                  Kriegsausbruch machte eine Abhaltung unmöglich.}}}\label{K_L02179-1} anzusagen? Die erlauchten
               Spitzen unserer Geisteswelt sollen doch bei für die Kultur kämpfenden Veranstaltung
               vertreten sein.\pend
           
\pstart
           Würden Sie im letzten Augenblick verhindert (was ich ich nicht hoffe) so wäre schon
               Ihr Name in der Teilnehmerliste uns eine höchst wertvolle Förderung.\pend
           
\pstart
           Mit vielen Grüssen – und g\textcolor{gray}{ewär}tige Erledigung
               hoffend{\\[\baselineskip]}Ihre erg.{\\[\baselineskip]}\spacefill\mbox{Bertha v. Suttner}\pend
           \leftskip=0em{}\selectlanguage{ngerman}\endnumbering\briefempfaengerindex{Schnitzler, Olga@\textsc{Schnitzler, Olga}!zzzSuttner, Bertha von@\emph{von Bertha von Suttner}!1914-05-141@{14. 5. 1914}|)be}\briefempfaengerindex{Schnitzler, Arthur@\textsc{Schnitzler, Arthur}!zzzSuttner, Bertha von@\emph{von Bertha von Suttner}!1914-05-141@{14. 5. 1914}|)be}\mylabel{L02179h}  \newcommand{\dateiname}{L02179}\newcommand{\titel}{Bertha von Suttner an Arthur und Olga Schnitzler, 14. 5. 1914}\newcommand{\editorInnen}{Martin Anton Müller und Gerd-Hermann Susen}%% latex-leseansicht-abspann.tex
%% Abspann für die Leseansicht.
%% Der Schalter \ifkorrekturansicht ist bereits durch den Vorspann gesetzt.

%% latex-abspann.tex
%% Gemeinsamer Abspann für Korrekturansicht und Leseansicht.
%% Setzt den Schalter \ifkorrekturansicht voraus (gesetzt in den
%% einbindenden Dateien latex-korrekturansicht-abspann.tex bzw.
%% latex-leseansicht-abspann.tex).
%% ---------------------------------------------------------------

\normalsize

% Das esempio-Environment wird nur in der Leseansicht benötigt
\ifkorrekturansicht\else
\newenvironment{esempio}[3]%
{
    \vspace{1.5ex}
    \rlap{\underline{#1}}
    \par
    \setlength{\parindent}{0cm}
    \nopagebreak
    \leftskip=#2cm
    \rightskip=#3cm
}
{
    \par
}
\fi

\doendnotes{C}
\bigskip
\vfill

\clearpage

\footnotesize

\ifkorrekturansicht
  \lohead{\textsc{register}}
\fi

% theindex-Environment neu definieren ohne reledmac
\makeatletter
\renewenvironment{theindex}{%
  \ifkorrekturansicht
    \section*{\indexname}%
  \else
    \subsubsection*{Index der erwähnten Entitäten}%
  \fi
  \setlength{\parindent}{0pt}%
  \setlength{\parskip}{0pt plus 0.3pt}%
  \let\item\@idxitem
}{%
  \ifkorrekturansicht\clearpage\fi
}
\makeatother

\IfFileExists{\jobname-pw.ind}{\input{\jobname-pw.ind}}{}

% Quellenangabe nur in der Leseansicht
\ifkorrekturansicht\else
% Fallback-Definitionen, falls die .tex-Datei \titel etc. nicht gesetzt hat
\providecommand{\titel}{}
\providecommand{\editorInnen}{}
\providecommand{\dateiname}{\jobname}

\vspace{3cm}

\vfill

\footnotesize
\textsc{Quelle}: \titel. Herausgegeben von {\editorInnen}. In: \emph{Arthur Schnitzler: Briefwechsel mit Autorinnen und Autoren}.
 Digitale Edition, https://schnitzler-briefe.acdh.oeaw.ac.at/{\dateiname}.html (Stand \today)
\fi

\end{document}


