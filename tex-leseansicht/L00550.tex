%% latex-leseansicht-vorspann.tex
%% Vorspann für die Leseansicht.
%% Lädt die gemeinsame Datei latex-vorspann.tex mit nicht gesetztem Schalter.

\newif\ifkorrekturansicht
\korrekturansichtfalse

\input{../tex-inputs/latex-vorspann}


\section[Alfred Polgar an Arthur Schnitzler, 6. 6. {[}1896?{]}]{L00550 Alfred Polgar an Arthur Schnitzler, 6. 6. [1896?]}
\nopagebreak\mylabel{L00550v}
\rehead{ }\normalsize\beginnumbering\briefempfaengerindex{Schnitzler, Arthur@\textsc{Schnitzler, Arthur}!zzzPolgar, Alfred@\emph{von Alfred Polgar}!1896-06-061@{6. 6. [1896?]}|(be}
\toendnotes[C]{\smallbreak\pagebreak[2]}
\correspDesc{Versand  durch Alfred Polgar am 6. 6. [1896?] \textbf{Ort fehlend} 
\newline{}Erhalt  durch Arthur Schnitzler im Zeitraum [6. 6. 1896
                  – 10. 6. 1896?] in Wien}\toendnotes[C]{\smallbreak}
\Standort{CUL, Schnitzler, B 78.}
\physDesc{Brief, 1 Blatt, 1 Seite, 678 Zeichen
\newline{}Handschrift: schwarze Tinte, lateinische Kurrent
\newline{}Schnitzler: 1) auf der ersten Seite mit rotem Buntstift beschriftet: »\textsc{Polgar}«.  2) mit Bleistift unterhalb der Unterschrift »(Polgar)«, das Datum
                                 wiederum mit der Jahreszahl »9\textcolor{gray}{6}« versehen}\toendnotes[C]{\smallbreak}
\pstart{}{\pb}Hochverehrter Herr Doctor,\pend\vspace{0.5em}
\pstart
           Verzeihen Sie, Herr Doctor, daß ich Sie abermals mit meinen privaten Angelegenheiten
               belästige und Sie dringendst bitte, den Passus Ihres Briefes an D\textsuperscript{r}{ }Ludassy\pwindex{Gans-Ludassy, Julius von 13.\,4.\,1858 Wien – 30.\,9.\,1922 ebd.@\textsc{Gans-Ludassy, Julius von} (13.\,4.\,1858 Wien – 30.\,9.\,1922 ebd.), \emph{Schriftsteller, Journalist, Herausgeber}|pw}, der von meinem Urlaub handelt, zu
               streichen, ev. ein paar neue Zeilen über meinen Gesundheitszustand zu schreiben.\pend
           
\pstart
           Es ist ganz zweifellos, daß mein Chef den Hinweis auf einen Urlaub als von mir
               inspirirt ansehen wird und das könnte die Aversion, die er in letzter Zeit gegen mich
               zu haben scheint, in’s Unheilbare steigern.\pend
           
\pstart
           Ich bitte recht sehr, Herr Doctor, mir die neuerliche Belästigung nicht übel nehmen
               zu wollen und zeichne mit aufrichtigstem Dank\pend
           \pstart hochachtungsvoll erg. \spacefill\mbox{Alfred Pollak.}\pend{}
\pstart
           \label{K_L00550-1v}\edtext{6/VI.}{\lemma{\textnormal{\emph{6/VI.}}}\Cendnote{\textnormal{Die Datierung des Jahres beruht auf
                     dem unsicher gelesenen Zusatz »96« durch Schnitzler. Zusätzliche
                     Argumente für die Datierung in dieser Zeit liefern der Eintrag im \emph{Tagebuch}\pwindex{Schnitzler, Arthur 15.\,5.\,1862 Wien – 21.\,10.\,1931 ebd.@\textsc{Schnitzler, Arthur} (15.\,5.\,1862 Wien – 21.\,10.\,1931 ebd.), \emph{Schriftsteller, Mediziner}!Tagebuch@\strich\emph{Tagebuch}|pwk}{ }Schnitzlers vom 10. 11. 1905, in dem
                     er seine Bekanntschaft mit Polgar\pwindex{Polgar, Alfred 17.\,10.\,1873 Wien – 24.\,4.\,1955 Zürich@\textsc{Polgar, Alfred} (17.\,10.\,1873 Wien – 24.\,4.\,1955 Zürich), \emph{Schriftsteller, Journalist, Kritiker}|pwk} Revue
                     passieren lässt, sowie der Umstand, dass die Unterschrift auf den späteren \emph{nom de plume} verzichtet.}}}\label{K_L00550-1}\pend
           \selectlanguage{ngerman}\endnumbering\briefempfaengerindex{Schnitzler, Arthur@\textsc{Schnitzler, Arthur}!zzzPolgar, Alfred@\emph{von Alfred Polgar}!1896-06-061@{6. 6. [1896?]}|)be}\mylabel{L00550h}  \newcommand{\dateiname}{L00550}\newcommand{\titel}{Alfred Polgar an Arthur Schnitzler, 6. 6. [1896?]}\newcommand{\editorInnen}{Martin Anton Müller und Gerd-Hermann Susen}%% latex-leseansicht-abspann.tex
%% Abspann für die Leseansicht.
%% Der Schalter \ifkorrekturansicht ist bereits durch den Vorspann gesetzt.

%% latex-abspann.tex
%% Gemeinsamer Abspann für Korrekturansicht und Leseansicht.
%% Setzt den Schalter \ifkorrekturansicht voraus (gesetzt in den
%% einbindenden Dateien latex-korrekturansicht-abspann.tex bzw.
%% latex-leseansicht-abspann.tex).
%% ---------------------------------------------------------------

\normalsize

% Das esempio-Environment wird nur in der Leseansicht benötigt
\ifkorrekturansicht\else
\newenvironment{esempio}[3]%
{
    \vspace{1.5ex}
    \rlap{\underline{#1}}
    \par
    \setlength{\parindent}{0cm}
    \nopagebreak
    \leftskip=#2cm
    \rightskip=#3cm
}
{
    \par
}
\fi

\doendnotes{C}
\bigskip
\vfill

\clearpage

\footnotesize

\ifkorrekturansicht
  \lohead{\textsc{register}}
\fi

% theindex-Environment neu definieren ohne reledmac
\makeatletter
\renewenvironment{theindex}{%
  \ifkorrekturansicht
    \section*{\indexname}%
  \else
    \subsubsection*{Index der erwähnten Entitäten}%
  \fi
  \setlength{\parindent}{0pt}%
  \setlength{\parskip}{0pt plus 0.3pt}%
  \let\item\@idxitem
}{%
  \ifkorrekturansicht\clearpage\fi
}
\makeatother

\IfFileExists{\jobname-pw.ind}{\input{\jobname-pw.ind}}{}

% Quellenangabe nur in der Leseansicht
\ifkorrekturansicht\else
% Fallback-Definitionen, falls die .tex-Datei \titel etc. nicht gesetzt hat
\providecommand{\titel}{}
\providecommand{\editorInnen}{}
\providecommand{\dateiname}{\jobname}

\vspace{3cm}

\vfill

\footnotesize
\textsc{Quelle}: \titel. Herausgegeben von {\editorInnen}. In: \emph{Arthur Schnitzler: Briefwechsel mit Autorinnen und Autoren}.
 Digitale Edition, https://schnitzler-briefe.acdh.oeaw.ac.at/{\dateiname}.html (Stand \today)
\fi

\end{document}


