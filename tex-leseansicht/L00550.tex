%% latex-leseansicht-vorspann.tex
%% Vorspann für die Leseansicht.
%% Lädt die gemeinsame Datei latex-vorspann.tex mit nicht gesetztem Schalter.

\newif\ifkorrekturansicht
\korrekturansichtfalse

\input{../tex-inputs/latex-vorspann}


         
         \renewcommand{\erwaehntePersonen}{Personen: Julius von Gans-Ludassy}
         \renewcommand{\erwaehnteOrte}{Orte: Wien}
         \renewcommand{\erwaehnteWerke}{Werke: Tagebuch}
               \section[Alfred Polgar an Arthur Schnitzler, 6. 6. {[}1896?{]}]{ Alfred Polgar an Arthur Schnitzler, 6. 6. {[}1896?{]}}\nopagebreak\mylabel{v}\rehead{ }\begin{ledgroupsized}[t]{13cm}\normalsize\beginnumbering \toendnotes[C]{\smallbreak\pagebreak[2]} \Standort{CUL, Schnitzler, B 78.}
\physDesc{Brief, 1 Blatt, 1 Seite, 678 Zeichen
\newline{}Handschrift: schwarze Tinte, lateinische Kurrent
\newline{}Schnitzler: 1) auf der ersten Seite mit rotem Buntstift beschriftet: »\textsc{Polgar}«.  2) mit Bleistift unterhalb der Unterschrift
                                    »(Polgar)«, das Datum wiederum mit der Jahreszahl »9\textcolor{gray}{6}« versehen}\toendnotes[C]{\smallbreak}\pstart{}{\pb}Hochverehrter Herr Doctor,\pend\pstart
           Verzeihen Sie, Herr Doctor, daß ich Sie abermals mit meinen privaten Angelegenheiten
               belästige und Sie dringendst bitte, den Passus Ihres Briefes an D\textsuperscript{r}{ }Ludassy\pwindex{Gans-Ludassy, Julius von 13.04.1858 – 30.09.1922@\textsc{Gans-Ludassy, Julius von} (13.04.1858 – 30.09.1922), \emph{Schriftsteller, Journalist, Herausgeber}|pw}, der von meinem Urlaub handelt, zu
               streichen, ev. ein paar neue Zeilen über meinen Gesundheitszustand zu schreiben.\pend
           \pstart
           Es ist ganz zweifellos, daß mein Chef den Hinweis auf einen Urlaub als von mir
               inspirirt ansehen wird und das könnte die Aversion, die er in letzter Zeit gegen mich
               zu haben scheint, in’s Unheilbare steigern.\pend
           \pstart
           Ich bitte recht sehr, Herr Doctor, mir die neuerliche Belästigung nicht übel nehmen
               zu wollen und zeichne mit aufrichtigstem Dank\pend
           \pstart hochachtungsvoll erg. \spacefill\mbox{Alfred Pollak.}\pend{}\pstart
           \label{K_L00550_1v}\edtext{6/VI.}{\lemma{\textnormal{\emph{6/VI.}}}\Cendnote{\textnormal{Die Datierung des Jahres beruht auf
                     dem unsicher gelesenen Zusatz »96« durch Schnitzler\pwindex{Schnitzler, Arthur 15.05.1862 – 21.10.1931@\textsc{Schnitzler, Arthur} (15.05.1862 – 21.10.1931), \emph{Schriftsteller, Mediziner}|pwk}. Zusätzliche
                     Argumente für die Datierung in der Zeit liefern der Eintrag im \emph{Tagebuch}\pwindex{Schnitzler, Arthur 15.05.1862 – 21.10.1931@\textsc{Schnitzler, Arthur} (15.05.1862 – 21.10.1931), \emph{Schriftsteller, Mediziner}!Tagebuch1981 – 2000@\strich\emph{Tagebuch} {[}1981 – 2000{]}|pwk}{ }Schnitzlers\pwindex{Schnitzler, Arthur 15.05.1862 – 21.10.1931@\textsc{Schnitzler, Arthur} (15.05.1862 – 21.10.1931), \emph{Schriftsteller, Mediziner}|pwk} vom 10. 11. 1905, in dem
                     er seine Bekanntschaft mit Polgar\pwindex{Polgar, Alfred 17.10.1873 – 24.04.1955@\textsc{Polgar, Alfred} (17.10.1873 – 24.04.1955), \emph{Schriftsteller, Journalist, Kritiker}|pwk} Revue
                     passieren lässt, sowie der Umstand, dass die Unterschrift auf den späteren \emph{nom de plume} verzichtet.}}}\label{K_L00550_1h}\pend
           
         
         \endnumbering\mylabel{h}\end{ledgroupsized}  \newcommand{\dateiname}{L00550}\newcommand{\titel}{Alfred Polgar an Arthur Schnitzler, 6. 6. [1896?]}\newcommand{\editorInnen}{Martin Anton Müller und Gerd-Hermann Susen}%% latex-leseansicht-abspann.tex
%% Abspann für die Leseansicht.
%% Der Schalter \ifkorrekturansicht ist bereits durch den Vorspann gesetzt.

%% latex-abspann.tex
%% Gemeinsamer Abspann für Korrekturansicht und Leseansicht.
%% Setzt den Schalter \ifkorrekturansicht voraus (gesetzt in den
%% einbindenden Dateien latex-korrekturansicht-abspann.tex bzw.
%% latex-leseansicht-abspann.tex).
%% ---------------------------------------------------------------

\normalsize

% Das esempio-Environment wird nur in der Leseansicht benötigt
\ifkorrekturansicht\else
\newenvironment{esempio}[3]%
{
    \vspace{1.5ex}
    \rlap{\underline{#1}}
    \par
    \setlength{\parindent}{0cm}
    \nopagebreak
    \leftskip=#2cm
    \rightskip=#3cm
}
{
    \par
}
\fi

\doendnotes{C}
\bigskip
\vfill

\clearpage

\footnotesize

\ifkorrekturansicht
  \lohead{\textsc{register}}
\fi

% theindex-Environment neu definieren ohne reledmac
\makeatletter
\renewenvironment{theindex}{%
  \ifkorrekturansicht
    \section*{\indexname}%
  \else
    \subsubsection*{Index der erwähnten Entitäten}%
  \fi
  \setlength{\parindent}{0pt}%
  \setlength{\parskip}{0pt plus 0.3pt}%
  \let\item\@idxitem
}{%
  \ifkorrekturansicht\clearpage\fi
}
\makeatother

\IfFileExists{\jobname-pw.ind}{\input{\jobname-pw.ind}}{}

% Quellenangabe nur in der Leseansicht
\ifkorrekturansicht\else
% Fallback-Definitionen, falls die .tex-Datei \titel etc. nicht gesetzt hat
\providecommand{\titel}{}
\providecommand{\editorInnen}{}
\providecommand{\dateiname}{\jobname}

\vspace{3cm}

\vfill

\footnotesize
\textsc{Quelle}: \titel. Herausgegeben von {\editorInnen}. In: \emph{Arthur Schnitzler: Briefwechsel mit Autorinnen und Autoren}.
 Digitale Edition, https://schnitzler-briefe.acdh.oeaw.ac.at/{\dateiname}.html (Stand \today)
\fi

\end{document}


      