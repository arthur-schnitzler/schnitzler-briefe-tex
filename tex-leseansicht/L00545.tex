%% latex-korrekturansicht-vorspann.tex
%% Vorspann für die Korrekturansicht.
%% Lädt die gemeinsame Datei latex-vorspann.tex mit gesetztem Schalter.

\newif\ifkorrekturansicht
\korrekturansichttrue

\input{../tex-inputs/latex-vorspann}


\section[Hugo von Hofmannsthal an Arthur Schnitzler, 17. 5. {[}1896{]}]{L00545 Hugo von Hofmannsthal an Arthur Schnitzler, 17. 5. {[}1896{]}}
\nopagebreak\mylabel{L00545v}
\rehead{ }\normalsize\beginnumbering\briefempfaengerindex{Schnitzler, Arthur@\textsc{Schnitzler, Arthur}!zzzHofmannsthal, Hugo von@\emph{von Hugo von Hofmannsthal}!1896-05-171@{17. 5. {[}1896{]}}|(be}
\toendnotes[C]{\smallbreak\pagebreak[2]}\Standort{CUL, Schnitzler, B 43.}
\physDesc{Brief, 1 Blatt, 4 Seiten, 1872 Zeichen (aufgeprägtes Wappen)
\newline{}Handschrift: schwarze Tinte, deutsche Kurrent
\newline{}Ordnung: von unbekannter Hand nummeriert: »1« }
\buchAbdrucke{\weitereDrucke{1) Hugo von Hofmannsthal: \emph{Briefe. 1890–1901}. Berlin: \emph{S. Fischer} 1935, S. 192–193.} \weitereDrucke{2) Hugo von Hofmannsthal, Arthur Schnitzler: \emph{Briefwechsel}. Frankfurt am Main: \emph{S. Fischer} 1964, S. 65–66.} \weitereDrucke{3) Hermann Bahr, Arthur Schnitzler: \emph{Briefwechsel, Aufzeichnungen, Dokumente (1891–1931)}. Göttingen: \emph{Wallstein} 2018, S. 121.} }\toendnotes[C]{\smallbreak}
\pstart
           \raggedleft{}{\pb}\label{K_L00545-1v}\edtext{\textsc{Tłumacz}}{\lemma{\textnormal{\emph{Tłumacz}}}\Cendnote{\textnormal{Hugo von Hofmannsthal leistete im
                           Mai 1896 seinen Militärdienst in Tłumacz ab.}}}\label{K_L00545-1} bei \textsc{Stanislau}\oindex{Tłumacz@\textbf{Tłumacz}, \emph{P.PPLA2}|pw}{ }\textsc{(Galizien\oindex{Galizien@\textbf{Galizien}, \emph{Region}|pw})}{\\}\textsc{K. u. K. 8\textsuperscript{tes}
                     Uhlanenregiment}{\\}Sonntag 17\textsuperscript{ten} Mai.\pend
           
\pstart{}lieber Arthur!\pend\vspace{0.5em}
\pstart
           vor einer Woche hat mir meine Mutter\pwindex{Hofmannsthal, Anna von 27.01.1849 – 22.03.1904@\textsc{Hofmannsthal, Anna von} (27.01.1849 – 22.03.1904)|pwv} geſchrieben, Sie hätten mit ihr geſprochen und ihr erzählt, daſs im
               Herbſt wieder ein \strikeout{ein}{ }Stück\pwindex{Freiwild. Schauspiel in 3 Akten@\emph{Freiwild. Schauspiel in 3 Akten}|pwv} von Ihnen aufgeführt
               werden wird. Das hat mich, wie es der Zufall manchmal bringt, ſo »hiſtoriſch«
               berührt. Die ganze Zeit, ſeit wir uns kennen, iſt mir als ein ganzes eingefallen, wie
               eine Landſchaft, {\pb}aber viel
               merkwürdiger: als wenn man in einem Thal ſtünde und durch die Wände der Berge
               hindurch die andern Thäler gleichzeitig ſehen würde.\pend
           
\pstart
           Auch der gute Goldmann\pwindex{Goldmann, Paul 31.01.1865 – 25.09.1935@\textsc{Goldmann, Paul} (31.01.1865 – 25.09.1935), \emph{Schriftsteller/Schriftstellerin, Journalist/Journalistin}|pw} ist mir ſehr ſtark
               eingefallen und ſein ſonderbares ſchmerzliches Leben. Es iſt merkwürdig, wie ſtark
               man an Vergangenes denken kann, wenn man ſo allein und abgeſchnitten lebt, wie ich
               hier. Mir iſt eingefallen, wie mir der Goldmann\pwindex{Goldmann, Paul 31.01.1865 – 25.09.1935@\textsc{Goldmann, Paul} (31.01.1865 – 25.09.1935), \emph{Schriftsteller/Schriftstellerin, Journalist/Journalistin}|pw} zum erſten Mal von Nietzſche\pwindex{Nietzsche, Friedrich 15.10.1844 – 25.08.1900@\textsc{Nietzsche, Friedrich} (15.10.1844 – 25.08.1900), \emph{Schriftsteller/Schriftstellerin, Philosoph/Philosophin}|pw}
               und von Bahr\pwindex{Bahr, Hermann 19.07.1863 – 15.01.1934@\textsc{Bahr, Hermann} (19.07.1863 – 15.01.1934), \emph{Schriftsteller/Schriftstellerin, Kritiker/Kritikerin}|pw} erzählt hat, das ganze kleine
               \label{K_L00545-2v}\edtext{Redactionszimmer\orgindex{der schoenen blauen Donau@An der schönen blauen Donau|pwv}}{\lemma{\textnormal{\emph{Redactionszimmer}}}\Cendnote{\textnormal{Goldmann\pwindex{Goldmann, Paul 31.01.1865 – 25.09.1935@\textsc{Goldmann, Paul} (31.01.1865 – 25.09.1935), \emph{Schriftsteller/Schriftstellerin, Journalist/Journalistin}|pwk} war bis 1890
                  verantwortlicher Redakteur der Zeitschrift \emph{An der schönen blauen
                     Donau}\pwindex{der schoenen blauen Donau@\emph{An der schönen blauen Donau}|pwk} gewesen, in der Schnitzler einige
                  frühe Texte publiziert hatte.}}}\label{K_L00545-2} und unſre {\pb}erſten Begegnungen, und alles
               kommt mir ſo unglaublich vergangen vor und ſo nett und altmodiſch wie eine Geſchichte
               aus der Jean Paul\pwindex{Jean Paul 1763-03-21 – 1825-11-14@\textsc{Jean Paul} (1763-03-21 – 1825-11-14), \emph{Schriftsteller/Schriftstellerin}|pw}-Zeit. \uline{Wir haben doch in dieſen paar Jahren ſehr viele ſchöne Stunden gehabt.}
               Wir haben ſehr oft das Leben reich und groß geſehen und waren im Stande, viele Dinge
               auf einander zu beziehen, und immer hat ſichs wieder verändert, das war das ſchönſte.
                  \label{LL435-1v}Auch daſs wir voneinander nicht gar zu viel
                  wiſſen und immer \strikeout{ein} jeder {\pb}wie ein Neuer aus ſeinem Leben
                  hervortritt und wieder hinein geht, ist ſehr ſchön.\label{LL435-1h}\pend
           
\pstart
           Über meinen augenblicklichen Zuſtand will ich lieber nichts erzählen: die Station iſt
               von einer teufliſchen Häſslichkeit, die Menſchen nicht recht erfreulich, das Wetter
               fortwährend elend. Ich habe einige Bändchen Platon\pwindex{Platon 427? v. u. Z. – 347/348 v. u. Z.@\textsc{Platon} (427? v. u. Z. – 347/348 v. u. Z.), \emph{Philosoph/Philosophin}|pw} mit, auch den Pindar\pwindex{Pindaros 522/518 – nach 446@\textsc{Pindaros} (522/518 – nach 446), \emph{Schriftsteller/Schriftstellerin}|pw} und den
               unerſchöpflichen erſten Band von Goethe\pwindex{Goethe, Johann Wolfgang von 1749-08-28 – 1832-03-22@\textsc{Goethe, Johann Wolfgang von} (1749-08-28 – 1832-03-22), \emph{Schriftsteller/Schriftstellerin}|pw}: die
               Lieder, die Elegien, und die Sprüche. Ich freue mich im ſtillen (wenn auch mit
               Zweifeln) Ihr neues Stück\pwindex{Freiwild. Schauspiel in 3 Akten@\emph{Freiwild. Schauspiel in 3 Akten}|pwv} noch
               im Juni bei der Tini\pwindex{Schoenberger, Christine 1875-11-17 – 1971-02-03@\textsc{Schönberger, Christine} (1875-11-17 – 1971-02-03), \emph{Gastwirt/Gastwirtin}|pw} zu
               hören.\pend
           
\pstart
           Herzlich Ihr{\\[\baselineskip]}\spacefill\mbox{Hugo.}\pend
           \leftskip=0em{}\selectlanguage{ngerman}\endnumbering\briefempfaengerindex{Schnitzler, Arthur@\textsc{Schnitzler, Arthur}!zzzHofmannsthal, Hugo von@\emph{von Hugo von Hofmannsthal}!1896-05-171@{17. 5. {[}1896{]}}|)be}\mylabel{L00545h}  \normalsize

\doendnotes{C}
\bigskip
\vfill

\clearpage

\footnotesize

\lohead{\textsc{register}}

% Definiere theindex-Environment komplett neu ohne reledmac
\makeatletter
\renewenvironment{theindex}{%
  \section*{\indexname}%
  \setlength{\parindent}{0pt}%
  \setlength{\parskip}{0pt plus 0.3pt}%
  \let\item\@idxitem
}{%
  \clearpage
}
\makeatother

\IfFileExists{\jobname-pw.ind}{\input{\jobname-pw.ind}}{}

\end{document}

      