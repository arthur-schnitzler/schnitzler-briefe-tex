%% latex-leseansicht-vorspann.tex
%% Vorspann für die Leseansicht.
%% Lädt die gemeinsame Datei latex-vorspann.tex mit nicht gesetztem Schalter.

\newif\ifkorrekturansicht
\korrekturansichtfalse

\input{../tex-inputs/latex-vorspann}


\section[Hugo von Hofmannsthal an Arthur Schnitzler, 17. 5. [1896]]{L00545 Hugo von Hofmannsthal an Arthur Schnitzler, 17. 5. [1896]}
\nopagebreak\mylabel{L00545v}
\rehead{ }\normalsize\beginnumbering\briefempfaengerindex{Schnitzler, Arthur@\textsc{Schnitzler, Arthur}!zzzHofmannsthal, Hugo von@\emph{von Hugo von Hofmannsthal}!1896-05-171@{17. 5. [1896]}|(be}
\toendnotes[C]{\smallbreak\pagebreak[2]}
\correspDesc{Versand  durch Hugo von Hofmannsthal am 17. 5. [1896] in Tlumatsch
\newline{}Erhalt  durch Arthur Schnitzler im Zeitraum [18. 5. 1896
                  – 22. 5. 1896?] in Wien}\toendnotes[C]{\smallbreak}
\Standort{CUL, Schnitzler, B 43.}
\physDesc{Brief, 1 Blatt, 4 Seiten, 1872 Zeichen (aufgeprägtes Wappen)
\newline{}Handschrift: schwarze Tinte, deutsche Kurrent
\newline{}Ordnung: von unbekannter Hand nummeriert: »1« }
\buchAbdrucke{\weitereDrucke{1) Hugo von Hofmannsthal: \emph{Briefe. 1890–1901}. Berlin: \emph{S. Fischer} 1935, S. 192–193.} \weitereDrucke{2) Hugo von Hofmannsthal, Arthur Schnitzler: \emph{Briefwechsel}. Herausgegeben von Therese Nickl und Heinrich Schnitzler. Frankfurt am Main: \emph{S. Fischer} 1964, S. 65–66.} \weitereDrucke{3) Hermann Bahr, Arthur Schnitzler: \emph{Briefwechsel, Aufzeichnungen, Dokumente (1891–1931)}. Herausgegeben von Kurt Ifkovits und Martin Anton Müller. Göttingen: \emph{Wallstein} 2018, S. 121.} }\toendnotes[C]{\smallbreak}
\pstart
           \raggedleft{}{\pb}\label{K_L00545-1v}\edtext{\textsc{Tłumacz}}{\lemma{\textnormal{\emph{Tłumacz}}}\Cendnote{\textnormal{Hugo von Hofmannsthal leistete im
                           Mai 1896 seinen Militärdienst in Tłumacz ab.}}}\label{K_L00545-1} bei \textsc{Stanislau}\oindex{Tłumacz@\textbf{Tłumacz}, \emph{Hauptstadt}|pw}{ }\textsc{(Galizien\oindex{Galizien@\textbf{Galizien}|pw})}{\\}\textsc{K. u. K. 8\textsuperscript{tes}
                     Uhlanenregiment}{\\}Sonntag 17\textsuperscript{ten} Mai.\pend
           
\pstart{}lieber Arthur!\pend\vspace{0.5em}
\pstart
           vor einer Woche hat mir meine Mutter\pwindex{Hofmannsthal, Anna von 27.\,1.\,1849 Wien – 22.\,3.\,1904 Sanatorium Fürth@\textsc{Hofmannsthal, Anna von} (27.\,1.\,1849 Wien – 22.\,3.\,1904 Sanatorium Fürth)|pwv} geſchrieben, Sie hätten mit ihr geſprochen und ihr erzählt, daſs im
               Herbſt wieder ein \strikeout{ein}{ }Stück\pwindex{Schnitzler, Arthur 15.\,5.\,1862 Wien – 21.\,10.\,1931 ebd.@\textsc{Schnitzler, Arthur} (15.\,5.\,1862 Wien – 21.\,10.\,1931 ebd.), \emph{Schriftsteller, Mediziner}!Freiwild. Schauspiel in 3 Akten@\strich\emph{Freiwild. Schauspiel in 3 Akten}|pwv} von Ihnen aufgeführt
               werden wird. Das hat mich, wie es der Zufall manchmal bringt,{ }ſo »hiſtoriſch«
               berührt. Die ganze Zeit,{ }ſeit wir uns kennen, iſt mir als ein ganzes eingefallen, wie
               eine Landſchaft, {\pb}aber viel
               merkwürdiger: als wenn man in einem Thal{ }ſtünde und durch die Wände der Berge
               hindurch die andern Thäler gleichzeitig{ }ſehen würde.\pend
           
\pstart
           Auch der gute Goldmann\pwindex{Goldmann, Paul 31.\,1.\,1865 Breslau – 25.\,9.\,1935 Wien@\textsc{Goldmann, Paul} (31.\,1.\,1865 Breslau – 25.\,9.\,1935 Wien), \emph{Schriftsteller, Journalist}|pw} ist mir{ }ſehr{ }ſtark
               eingefallen und{ }ſein{ }ſonderbares{ }ſchmerzliches Leben. Es iſt merkwürdig, wie{ }ſtark
               man an Vergangenes denken kann, wenn man{ }ſo allein und abgeſchnitten lebt, wie ich
               hier. Mir iſt eingefallen, wie mir der Goldmann\pwindex{Goldmann, Paul 31.\,1.\,1865 Breslau – 25.\,9.\,1935 Wien@\textsc{Goldmann, Paul} (31.\,1.\,1865 Breslau – 25.\,9.\,1935 Wien), \emph{Schriftsteller, Journalist}|pw} zum erſten Mal von Nietzſche\pwindex{Nietzsche, Friedrich 15.\,10.\,1844 Röcken – 25.\,8.\,1900 Weimar@\textsc{Nietzsche, Friedrich} (15.\,10.\,1844 Röcken – 25.\,8.\,1900 Weimar), \emph{Schriftsteller, Philosoph}|pw}
               und von Bahr\pwindex{Bahr, Hermann 19.\,7.\,1863 Linz – 15.\,1.\,1934 München@\textsc{Bahr, Hermann} (19.\,7.\,1863 Linz – 15.\,1.\,1934 München), \emph{Schriftsteller, Kritiker}|pw} erzählt hat, das ganze kleine
               \label{K_L00545-2v}\edtext{Redactionszimmer\orgindex{der schönen blauen Donau@An der schönen blauen Donau|pwv}}{\lemma{\textnormal{\emph{Redactionszimmer}}}\Cendnote{\textnormal{Goldmann\pwindex{Goldmann, Paul 31.\,1.\,1865 Breslau – 25.\,9.\,1935 Wien@\textsc{Goldmann, Paul} (31.\,1.\,1865 Breslau – 25.\,9.\,1935 Wien), \emph{Schriftsteller, Journalist}|pwk} war bis 1890
                  verantwortlicher Redakteur der Zeitschrift \emph{An der schönen blauen
                     Donau}\pwindex{der schönen blauen Donau@\emph{An der schönen blauen Donau}|pwk} gewesen, in der Schnitzler einige
                  frühe Texte publiziert hatte.}}}\label{K_L00545-2} und unſre {\pb}erſten Begegnungen, und alles
               kommt mir{ }ſo unglaublich vergangen vor und{ }ſo nett und altmodiſch wie eine Geſchichte
               aus der Jean Paul\pwindex{Jean Paul 21.\,3.\,1763 Wunsiedel – 14.\,11.\,1825 Bayreuth@\textsc{Jean Paul} (21.\,3.\,1763 Wunsiedel – 14.\,11.\,1825 Bayreuth), \emph{Schriftsteller}|pw}-Zeit. \uline{Wir haben doch in dieſen paar Jahren{ }ſehr viele{ }ſchöne Stunden gehabt.}
               Wir haben{ }ſehr oft das Leben reich und groß geſehen und waren im Stande, viele Dinge
               auf einander zu beziehen, und immer hat{ }ſichs wieder verändert, das war das{ }ſchönſte.
                  \label{LL435-1v}Auch daſs wir voneinander nicht gar zu viel
                  wiſſen und immer \strikeout{ein} jeder {\pb}wie ein Neuer aus{ }ſeinem Leben
                  hervortritt und wieder hinein geht, ist{ }ſehr{ }ſchön.\label{LL435-1h}\pend
           
\pstart
           Über meinen augenblicklichen Zuſtand will ich lieber nichts erzählen: die Station iſt
               von einer teufliſchen Häſslichkeit, die Menſchen nicht recht erfreulich, das Wetter
               fortwährend elend. Ich habe einige Bändchen Platon\pwindex{Platon 427? v.\,u.\,Z. Athen – 347/348 v.\,u.\,Z. ebd.@\textsc{Platon} (427? v.\,u.\,Z. Athen – 347/348 v.\,u.\,Z. ebd.), \emph{Philosoph}|pw} mit, auch den Pindar\pwindex{Pindaros 522/518 – nach 446@\textsc{Pindaros} (522/518 – nach 446), \emph{Schriftsteller}|pw} und den
               unerſchöpflichen erſten Band von Goethe\pwindex{Goethe, Johann Wolfgang von 28.\,8.\,1749 Frankfurt am Main – 22.\,3.\,1832 Weimar@\textsc{Goethe, Johann Wolfgang von} (28.\,8.\,1749 Frankfurt am Main – 22.\,3.\,1832 Weimar), \emph{Schriftsteller}|pw}: die
               Lieder, die Elegien, und die Sprüche. Ich freue mich im{ }ſtillen (wenn auch mit
               Zweifeln) Ihr neues Stück\pwindex{Schnitzler, Arthur 15.\,5.\,1862 Wien – 21.\,10.\,1931 ebd.@\textsc{Schnitzler, Arthur} (15.\,5.\,1862 Wien – 21.\,10.\,1931 ebd.), \emph{Schriftsteller, Mediziner}!Freiwild. Schauspiel in 3 Akten@\strich\emph{Freiwild. Schauspiel in 3 Akten}|pwv} noch
               im Juni bei der Tini\pwindex{Kepert, Christine 17.\,11.\,1875 – 3.\,2.\,1971 Wien@\textsc{Kepert, Christine} (17.\,11.\,1875 – 3.\,2.\,1971 Wien), \emph{Gastwirtin}|pw} zu
               hören.\pend
           
\pstart
           Herzlich Ihr{\\[\baselineskip]}\spacefill\mbox{Hugo.}\pend
           \leftskip=0em{}\selectlanguage{ngerman}\endnumbering\briefempfaengerindex{Schnitzler, Arthur@\textsc{Schnitzler, Arthur}!zzzHofmannsthal, Hugo von@\emph{von Hugo von Hofmannsthal}!1896-05-171@{17. 5. [1896]}|)be}\mylabel{L00545h}  \newcommand{\dateiname}{L00545}\newcommand{\titel}{Hugo von Hofmannsthal an Arthur Schnitzler, 17. 5. [1896]}\newcommand{\editorInnen}{Herausgegeben von Martin Anton Müller}%% latex-leseansicht-abspann.tex
%% Abspann für die Leseansicht.
%% Der Schalter \ifkorrekturansicht ist bereits durch den Vorspann gesetzt.

%% latex-abspann.tex
%% Gemeinsamer Abspann für Korrekturansicht und Leseansicht.
%% Setzt den Schalter \ifkorrekturansicht voraus (gesetzt in den
%% einbindenden Dateien latex-korrekturansicht-abspann.tex bzw.
%% latex-leseansicht-abspann.tex).
%% ---------------------------------------------------------------

\normalsize

% Das esempio-Environment wird nur in der Leseansicht benötigt
\ifkorrekturansicht\else
\newenvironment{esempio}[3]%
{
    \vspace{1.5ex}
    \rlap{\underline{#1}}
    \par
    \setlength{\parindent}{0cm}
    \nopagebreak
    \leftskip=#2cm
    \rightskip=#3cm
}
{
    \par
}
\fi

\doendnotes{C}
\bigskip
\vfill

\clearpage

\footnotesize

\ifkorrekturansicht
  \lohead{\textsc{register}}
\fi

% theindex-Environment neu definieren ohne reledmac
\makeatletter
\renewenvironment{theindex}{%
  \ifkorrekturansicht
    \section*{\indexname}%
  \else
    \subsubsection*{Index der erwähnten Entitäten}%
  \fi
  \setlength{\parindent}{0pt}%
  \setlength{\parskip}{0pt plus 0.3pt}%
  \let\item\@idxitem
}{%
  \ifkorrekturansicht\clearpage\fi
}
\makeatother

\IfFileExists{\jobname-pw.ind}{\input{\jobname-pw.ind}}{}

% Quellenangabe nur in der Leseansicht
\ifkorrekturansicht\else
% Fallback-Definitionen, falls die .tex-Datei \titel etc. nicht gesetzt hat
\providecommand{\titel}{}
\providecommand{\editorInnen}{}
\providecommand{\dateiname}{\jobname}

\vspace{3cm}

\vfill

\footnotesize
\textsc{Quelle}: \titel. Herausgegeben von {\editorInnen}. In: \emph{Arthur Schnitzler: Briefwechsel mit Autorinnen und Autoren}.
 Digitale Edition, https://schnitzler-briefe.acdh.oeaw.ac.at/{\dateiname}.html (Stand \today)
\fi

\end{document}


