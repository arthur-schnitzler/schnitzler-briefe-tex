%% latex-korrekturansicht-vorspann.tex
%% Vorspann für die Korrekturansicht.
%% Lädt die gemeinsame Datei latex-vorspann.tex mit gesetztem Schalter.

\newif\ifkorrekturansicht
\korrekturansichttrue

\input{../tex-inputs/latex-vorspann}


\section[ Paul Goldmann an Arthur Schnitzler, 12. 3. {[}1899{]}]{L02869 Paul Goldmann an Arthur Schnitzler, 12. 3. {[}1899{]}}
\nopagebreak\mylabel{L02869v}
\rehead{ }\normalsize\beginnumbering\briefempfaengerindex{Schnitzler, Arthur@\textsc{Schnitzler, Arthur}!zzzGoldmann, Paul@\emph{von Paul Goldmann}!1899-03-121@{12. 3. {[}1899{]}}|(be}
\toendnotes[C]{\smallbreak\pagebreak[2]}\Standort{DLA, A:Schnitzler, HS.NZ85.1.3169.}
\physDesc{Brief, 1 Blatt, 4 Seiten, 3022 Zeichen
\newline{}Handschrift: schwarze Tinte, deutsche Kurrent
\newline{}Schnitzler: mit rotem Buntstift eine Unterstreichung }\toendnotes[C]{\smallbreak}
\pstart
           \centering{}{\pb}Frankfurt\oindex{Frankfurt am Main@\textbf{Frankfurt am Main}, \emph{P.PPLA3}|pw}, 12. März.\pend
           
\pstart\center{}Mein lieber Freund,\pend\vspace{0.5em}
\pstart
           Wenn Du \label{K_L02869-1v}\edtext{Ende April nach Berlin\oindex{Berlin@\textbf{Berlin}, \emph{P.PPLC}|pw}}{\lemma{\textnormal{\emph{Ende April nach Berlin}}}\Cendnote{\textnormal{Schnitzler hielt sich vom 25. 4. 1899 bis zum 2. 5. 1899 in Berlin\oindex{Berlin@\textbf{Berlin}, \emph{P.PPLC}|pwk} auf. Dort hatte sein neuer \emph{Einakterzyklus}\pwindex{gruene Kakadu – Paracelsus – Die Gefaehrtin. Drei Einakter@\emph{Der grüne Kakadu – Paracelsus – Die Gefährtin. Drei Einakter}|pwk} (\emph{Der grüne Kakadu}\pwindex{gruene Kakadu. Groteske in einem Akt@\emph{Der grüne Kakadu. Groteske in einem Akt}|pwk}, \emph{Paracelsus}\pwindex{Paracelsus. Versspiel in einem Akt@\emph{Paracelsus. Versspiel in einem Akt}|pwk}, \emph{Die Gefährtin}\pwindex{Gefaehrtin. Schauspiel in einem Akt@\emph{Die Gefährtin. Schauspiel in einem Akt}|pwk}) am 29. 4. 1899 am Deutschen Theater\oindex{Deutsches Theater Berlin@\textbf{Deutsches Theater Berlin}, \emph{Theater (K.THE)}|pwk} Premiere. Nach Frankfurt am Main\oindex{Frankfurt am Main@\textbf{Frankfurt am Main}, \emph{P.PPLA3}|pwk} reiste er im Zuge dessen nicht.}}}\label{K_L02869-1}
               gehſt, könnteſt Du da nicht auf der Hin- oder Rückreiſe über Frankfurt\oindex{Frankfurt am Main@\textbf{Frankfurt am Main}, \emph{P.PPLA3}|pw} kommen? Der Umweg iſt freilich groß; aber im Frühling
               iſt Frankfurt\oindex{Frankfurt am Main@\textbf{Frankfurt am Main}, \emph{P.PPLA3}|pw} u. das Rheinland\oindex{Rheinland@\textbf{Rheinland}, \emph{Teil eines Landes (A.LNDX)}|pw} gar ſchön. Von der Freude, die Du mir machen
               würdeſt, rede ich erſt gar nicht.\pend
           
\pstart
           Von den Kritiken über Deine Stücke\pwindex{gruene Kakadu – Paracelsus – Die Gefaehrtin. Drei Einakter@\emph{Der grüne Kakadu – Paracelsus – Die Gefährtin. Drei Einakter}|pwv} hat mir \label{K_L02869-2v}\edtext{die von \textsc{Hirschfeld\pwindex{Hirschfeld, Robert 17.09.1857 – 02.04.1914@\textsc{Hirschfeld, Robert} (17.09.1857 – 02.04.1914), \emph{Journalist/Journalistin, Musikkritiker/Musikkritikerin}|pw}}\pwindex{Burgtheater. (»Paracelsus«, Schauspiel in einem Act. – »Die Gefaehrtin«, Schauspiel in einem Act. – »Der gruene Kakadu«, Groteske in einem Act von Arthur Schnitzler. Erste Auffuehrung am 1. Maerz 1899.)@\emph{Burgtheater. (»Paracelsus«, Schauspiel in einem Act. – »Die Gefährtin«, Schauspiel in einem Act. – »Der grüne Kakadu«, Groteske in einem Act von Arthur Schnitzler. Erste Aufführung am 1. März 1899.)}|pwv}}{\lemma{\textnormal{\emph{die von Hirschfeld}}}\Cendnote{\textnormal{L. A. Terne\pwindex{Hirschfeld, Robert 17.09.1857 – 02.04.1914@\textsc{Hirschfeld, Robert} (17.09.1857 – 02.04.1914), \emph{Journalist/Journalistin, Musikkritiker/Musikkritikerin}|pwkv} [ = Robert Hirschfeld\pwindex{Hirschfeld, Robert 17.09.1857 – 02.04.1914@\textsc{Hirschfeld, Robert} (17.09.1857 – 02.04.1914), \emph{Journalist/Journalistin, Musikkritiker/Musikkritikerin}|pwk}]: \emph{Burgtheater. (»Paracelsus«, Schauspiel in einem Act. – »Die
                        Gefährtin«, Schauspiel in einem Act. – »Der grüne Kakadu«, Groteske in einem
                        Act von Arthur Schnitzler. Erste Aufführung am 1. März 1899)}\pwindex{Burgtheater. (»Paracelsus«, Schauspiel in einem Act. – »Die Gefaehrtin«, Schauspiel in einem Act. – »Der gruene Kakadu«, Groteske in einem Act von Arthur Schnitzler. Erste Auffuehrung am 1. Maerz 1899.)@\emph{Burgtheater. (»Paracelsus«, Schauspiel in einem Act. – »Die Gefährtin«, Schauspiel in einem Act. – »Der grüne Kakadu«, Groteske in einem Act von Arthur Schnitzler. Erste Aufführung am 1. März 1899.)}|pwk}. In: \emph{Wiener Sonn- und Montags-Zeitung}\pwindex{Wiener Sonn- und Montagszeitung@\emph{Wiener Sonn- und Montagszeitung}|pwk}, Jg. 37,
                     Nr. 10, 6. 3. 1899, S. 1–2.}}}\label{K_L02869-2} am
               Beſten gefallen. Auch ſcheint ſie mir die richtigſte zu ſein. Er prägt ein
               treffliches Wort \label{K_L02869-3v}\edtext{»Anatol\pwindex{Anatol@\emph{Anatol}|pw}ismus\pwindex{Burgtheater. (»Paracelsus«, Schauspiel in einem Act. – »Die Gefaehrtin«, Schauspiel in einem Act. – »Der gruene Kakadu«, Groteske in einem Act von Arthur Schnitzler. Erste Auffuehrung am 1. Maerz 1899.)@\emph{Burgtheater. (»Paracelsus«, Schauspiel in einem Act. – »Die Gefährtin«, Schauspiel in einem Act. – »Der grüne Kakadu«, Groteske in einem Act von Arthur Schnitzler. Erste Aufführung am 1. März 1899.)}|pwv}«}{\lemma{\textnormal{\emph{»Anatolismus«}}}\Cendnote{\textnormal{»Heute […] bewundere ich einzig die Virtuosität
                     der wenigstens scheinbar dramatischen Gestaltung, die psychologischen
                     Rechtfertigungen und Folgerungen, Schnitzler’s Tiefe der Beobachtung und Weite der Menschenkenntniß,
                     seinen männlichen Ernst, der unbekümmert um äußere Erfolge, seine Kräfte auch
                     an gewagten dramatischen Stoffen erprobt. Er arbeitet unverdrossen und in sich
                     gekehrt an seiner inneren Klärung – weniger an läppischen ›Erklärungen‹, mit
                     denen eitle Reclamehelden ihr Dasein überflüssigerweise noch betonen möchten.
                     Ein Schritt noch, und Schnitzler hat sich
                     in seinem ebenmäßigen, organischen Entwickelungsgange, wie er nur
                     hervorragenden und grundechten Begabungen zukommt, von seinen Anatol\pwindex{Anatol@\emph{Anatol}|pw}ismen gänzlich losgerungen.«
                  (S. 1–2)}}}\label{K_L02869-3} und ſagt mit Recht, für Dich ſei es wichtig, aus dieſem {\pb}herauszukommen. Ich ſehe, daß Du große Anſtrengungen
               in dieſer Richtung machſt, und ich bin ſicher, daß es Dir gelingen wird. Darum halte
               ich den »Kakadu\pwindex{gruene Kakadu. Groteske in einem Akt@\emph{Der grüne Kakadu. Groteske in einem Akt}|pw}« für ein ſo wichtiges
               Entwickelungs-Stadium; aber immerhin ſteht er noch, wie mir dünkt, mit einem Fuße im
                  Anatol\pwindex{Anatol@\emph{Anatol}|pw}ismus. Daß es Dir auf Anderes dabei
               angekommen, als auf eine Liebesgeſchichte mit einem Theatermädel, iſt klar. Aber das
               Andere iſt, meinem Gefühl nach, nicht ſtark genug herausgekommen. Dies der Eindruck,
               den ich beim Leſen gehabt habe. Der Eindruck iſt vielleicht falſch, und namentlich
               auf der Bühne geſtaltet ſich die ganze Wirkung vielleicht ganz anders. Da ich aber
               dieſen Eindruck beim Leſen gehabt, war ich verpflichtet, ihn \strikeout{\textcolor{gray}{×}} Dir mitzutheilen. »Erſchöpfend {\pb}characteriſiren«, wie Du meinſt, habe ich Dein Werk\pwindex{gruene Kakadu. Groteske in einem Akt@\emph{Der grüne Kakadu. Groteske in einem Akt}|pwv} damit nicht gewollt; und es erſtaunt mich, daß ich Dich
               erſt noch beſonders darauf hinweiſen muß, eine in einem Briefwechſel zwiſchen zwei
               Freunden flüchtig hingeworfene Bemerkung könne doch unmöglich die Prätention haben,
               ein Werk »erſchöpfend zu characteriſiren«.\pend
           
\pstart
           Daß ich Dir ſolange nicht ſchrieb, hatte ſeinen Grund in der Angewißheit der ganzen
               Situation. Du kannſt Dich gewiß nur ſchwer in die Qualen einer ſolchen Wartezeit
               hineindenken. Heut will ich ſchreiben; aber nein, ich warte doch lieber bis auf
               morgen, weil morgen doch endlich die \label{K_L02869-4v}\edtext{entſcheidende Antwort}{\lemma{\textnormal{\emph{entſcheidende Antwort}}}\Cendnote{\textnormal{Siehe Paul Goldmann an Arthur Schnitzler, 5. 3. [1899].
               }}}\label{K_L02869-4} kommen wird. Und das geht ſo, einen Monat lang und darüber! Ich habe Dir
               nicht geſchrieben, weil ich {\pb}thatſächlich von Tag zu
               Tage gezerrt wurde\strikeout{,} und ſchließlich ſo muthlos wurde,
               ſo \label{K_L02869-5v}\edtext{\begin{otherlanguage}{french}\textsc{dégouté de tout}\end{otherlanguage}}{\lemma{\textnormal{\emph{dégouté de tout}}}\Cendnote{\textnormal{französisch: von allem
                  angewidert}}}\label{K_L02869-5}, daß ich mich ſelbſt zu einem Briefe an Dich nicht mehr
               aufzuraffen vermochte.\pend
           
\pstart
           Die N. Fr. Pr.\orgindex{Neue Freie Presse@Neue Freie Presse|pw} iſt übrigens beleidigt und
               entrüſtet und ſucht die Sachlage jetzt ſo zu drehen, als ſei \uline{ich} kontraktbrüchig geworden.\pend
           
\pstart
           Ich lebe ſeit Wochen im Hotel\oindex{Central-Hotel@\textbf{Central-Hotel}, \emph{Hotel (K.HTL)}|pwv}, in einer geradezu verzweifelten Unordnung. So gerieth auch das
               Manuſkript des »Kakadu\pwindex{gruene Kakadu. Groteske in einem Akt@\emph{Der grüne Kakadu. Groteske in einem Akt}|pw}« an einen Platz, wo es
               mir aus den Augen entſchwand; und als ich es \introOben{}zu ſpät\introOben{}
               wiederfand, hatte ich nicht mehr die Energie, Dir meine Schlamperei einzugeſtehen und
               Dich um Entſchuldigung zu bitten. Ich habe meine Nachläſſigkeit ſeitdem oft bereut,
               und die Art, wie Du ſie in Deinem Briefe erwähnſt, iſt die gerechte Strafe dafür, die
               ich nur als verdient hinnehmen kann.\pend
           
\pstart
           Viele treue Grüße! Dein{\\[\baselineskip]}\spacefill\mbox{Paul Goldmann}\pend
           \leftskip=0em{}
\pstart
           \noindent{}Grüße an Deine Freundin\pwindex{Reinhard, Marie 1871-03-13 – 1899-03-18@\textsc{Reinhard, Marie} (1871-03-13 – 1899-03-18), \emph{Gesangspädagoge/Gesangspädagogin}|pwv}!\pend
           
\pstart
           {\pb}\label{T_L02869-1v}\edtext{Ich danke den Deinen, namentlich
                  Deiner Frau Mutter\pwindex{Schnitzler, Louise 1840-07-08 – 1911-09-09@\textsc{Schnitzler, Louise} (1840-07-08 – 1911-09-09)|pwv}, für
                  alle ihre liebenswürdigen Intentionen. Auch mir thut es unendlich leid, daß die
                     Wien\oindex{Wien@\textbf{Wien}, \emph{A.ADM2}|pw}er Projekte ſich nicht realiſirt haben.
                  Meine geſammte Familie grüßt Dich herzlichſt.}{\lemma{\textnormal{\emph{Ich … herzlichſt.}}}\Cendnote{\textnormal{zwischen der Datumszeile und der Anrede kopfüber am oberen
                     Rand der ersten Seite}}}\label{T_L02869-1}\pend
           \selectlanguage{ngerman}\endnumbering\briefempfaengerindex{Schnitzler, Arthur@\textsc{Schnitzler, Arthur}!zzzGoldmann, Paul@\emph{von Paul Goldmann}!1899-03-121@{12. 3. {[}1899{]}}|)be}\mylabel{L02869h}  \normalsize

\doendnotes{C}
\bigskip
\vfill

\clearpage

\footnotesize

\lohead{\textsc{register}}

% Definiere theindex-Environment komplett neu ohne reledmac
\makeatletter
\renewenvironment{theindex}{%
  \section*{\indexname}%
  \setlength{\parindent}{0pt}%
  \setlength{\parskip}{0pt plus 0.3pt}%
  \let\item\@idxitem
}{%
  \clearpage
}
\makeatother

\IfFileExists{\jobname-pw.ind}{\input{\jobname-pw.ind}}{}

\end{document}

      