%% latex-leseansicht-vorspann.tex
%% Vorspann für die Leseansicht.
%% Lädt die gemeinsame Datei latex-vorspann.tex mit nicht gesetztem Schalter.

\newif\ifkorrekturansicht
\korrekturansichtfalse

\input{../tex-inputs/latex-vorspann}


\section[Arthur Schnitzler an Stefan Zweig, 9. 10. 1917]{L03775 Arthur Schnitzler an Stefan Zweig, 9. 10. 1917}
\nopagebreak\mylabel{L03775v}
\rehead{ }\normalsize\beginnumbering\briefempfaengerindex{Zweig, Stefan@\textsc{Zweig, Stefan}!zzzSchnitzler, Arthur@\emph{von Arthur Schnitzler}!1917-10-091@{9. 10. 1917}|(be}
\toendnotes[C]{\smallbreak\pagebreak[2]}
\correspDesc{Versand  durch Arthur Schnitzler am 9. 10. 1917 in Wien
\newline{}Erhalt  durch Stefan Zweig im Zeitraum [9. 10. 1917
                  – 12. 10. 1917?] in Wien}\toendnotes[C]{\smallbreak}
\Standort{Jerusalem, National Library of Israel, ARC. Ms. Var. 305 1 58 Stefan Zweig Collection.}
\physDesc{Briefkarte, 726 Zeichen
\newline{}Handschrift: schwarze Tinte, lateinische Kurrent}\toendnotes[C]{\smallbreak}
\pstart
           \textcolor{gray}{\textbf{Dr. Arthur Schnitzler}}\hfill {\pb}9. X. 917\pend
           
\pstart
           \textcolor{gray}{\textbf{Wien XVIII. Sternwartestrasse 71\oindex{Wien@\textbf{Wien}!XVIII., Währing@\textbf{XVIII., Währing}!Sternwartestraße 71@\textbf{Sternwartestraße 71}, \emph{Wohngebäude}|pw}}}\pend
           
\pstart{}lieber und verehrter Herr Doctor,\pend\vspace{0.5em}
\pstart
           nun hab ich Ihren \label{K_L03775-1v}\edtext{Jeremias\pwindex{Zweig, Stefan 28.\,11.\,1881 Wien – 23.\,2.\,1942 Petrópolis@\textsc{Zweig, Stefan} (28.\,11.\,1881 Wien – 23.\,2.\,1942 Petrópolis), \emph{Schriftsteller}!Jeremias. Eine dramatische Dichtung in neun Bildern@\strich\emph{Jeremias. Eine dramatische Dichtung in neun Bildern}|pw} gelesen}{\lemma{\textnormal{\emph{Jeremias gelesen}}}\Cendnote{\textnormal{Vgl. A. S.: \emph{Tagebuch}, 3. 10. 1917.}}}\label{K_L03775-1}, mit
               stärkster sich von Bild zu Bild erhöhender Theilnahme, und insbesondere den Schluss,
               nicht nur von den dichterischen Schönheiten, sondern auch von der menschlichen Wärme
               ergriffen, die Ihr Werk\pwindex{Zweig, Stefan 28.\,11.\,1881 Wien – 23.\,2.\,1942 Petrópolis@\textsc{Zweig, Stefan} (28.\,11.\,1881 Wien – 23.\,2.\,1942 Petrópolis), \emph{Schriftsteller}!Jeremias. Eine dramatische Dichtung in neun Bildern@\strich\emph{Jeremias. Eine dramatische Dichtung in neun Bildern}|pwv}
               ausstrahlt. Auf der Bühne wird es meiner Überzeugung nach – in der gedrängten Form,
               die Sie für diesen Zweck Ihrem dramatischen {\pb}Gedicht\pwindex{Zweig, Stefan 28.\,11.\,1881 Wien – 23.\,2.\,1942 Petrópolis@\textsc{Zweig, Stefan} (28.\,11.\,1881 Wien – 23.\,2.\,1942 Petrópolis), \emph{Schriftsteller}!Jeremias. Eine dramatische Dichtung in neun Bildern@\strich\emph{Jeremias. Eine dramatische Dichtung in neun Bildern}|pwv} wahrscheinlich geben
               werden, seine Wirkung gleichfalls nicht verfehlen, und ich wünschte sehr, Sie recht
               bald zu seinem ersten Theatererfolg beglückwünschen zu können.\pend
           
\pstart
           Meine Frau\pwindex{Schnitzler, Olga 17.\,1.\,1882 Wien – 13.\,1.\,1970 Lugano@\textsc{Schnitzler, Olga} (17.\,1.\,1882 Wien – 13.\,1.\,1970 Lugano), \emph{Schauspielerin, Sängerin}|pwv}, die den gleichen
               Eindruck von Jeremias\pwindex{Zweig, Stefan 28.\,11.\,1881 Wien – 23.\,2.\,1942 Petrópolis@\textsc{Zweig, Stefan} (28.\,11.\,1881 Wien – 23.\,2.\,1942 Petrópolis), \emph{Schriftsteller}!Jeremias. Eine dramatische Dichtung in neun Bildern@\strich\emph{Jeremias. Eine dramatische Dichtung in neun Bildern}|pw} erhalten, dankt Ihnen,
               lieber Herr Doctor und grüßt Sie so herzlich wie ich.\pend
           
\pstart
           Ihr ergebner{\\[\baselineskip]}\spacefill\mbox{Arthur Schnitzler}\pend
           \leftskip=0em{}\selectlanguage{ngerman}\endnumbering\briefempfaengerindex{Zweig, Stefan@\textsc{Zweig, Stefan}!zzzSchnitzler, Arthur@\emph{von Arthur Schnitzler}!1917-10-091@{9. 10. 1917}|)be}\mylabel{L03775h}  \newcommand{\dateiname}{L03775}\newcommand{\titel}{Arthur Schnitzler an Stefan Zweig, 9. 10. 1917}\newcommand{\editorInnen}{Selma Jahnke und Martin Anton Müller}%% latex-leseansicht-abspann.tex
%% Abspann für die Leseansicht.
%% Der Schalter \ifkorrekturansicht ist bereits durch den Vorspann gesetzt.

%% latex-abspann.tex
%% Gemeinsamer Abspann für Korrekturansicht und Leseansicht.
%% Setzt den Schalter \ifkorrekturansicht voraus (gesetzt in den
%% einbindenden Dateien latex-korrekturansicht-abspann.tex bzw.
%% latex-leseansicht-abspann.tex).
%% ---------------------------------------------------------------

\normalsize

% Das esempio-Environment wird nur in der Leseansicht benötigt
\ifkorrekturansicht\else
\newenvironment{esempio}[3]%
{
    \vspace{1.5ex}
    \rlap{\underline{#1}}
    \par
    \setlength{\parindent}{0cm}
    \nopagebreak
    \leftskip=#2cm
    \rightskip=#3cm
}
{
    \par
}
\fi

\doendnotes{C}
\bigskip
\vfill

\clearpage

\footnotesize

\ifkorrekturansicht
  \lohead{\textsc{register}}
\fi

% theindex-Environment neu definieren ohne reledmac
\makeatletter
\renewenvironment{theindex}{%
  \ifkorrekturansicht
    \section*{\indexname}%
  \else
    \subsubsection*{Index der erwähnten Entitäten}%
  \fi
  \setlength{\parindent}{0pt}%
  \setlength{\parskip}{0pt plus 0.3pt}%
  \let\item\@idxitem
}{%
  \ifkorrekturansicht\clearpage\fi
}
\makeatother

\IfFileExists{\jobname-pw.ind}{\input{\jobname-pw.ind}}{}

% Quellenangabe nur in der Leseansicht
\ifkorrekturansicht\else
% Fallback-Definitionen, falls die .tex-Datei \titel etc. nicht gesetzt hat
\providecommand{\titel}{}
\providecommand{\editorInnen}{}
\providecommand{\dateiname}{\jobname}

\vspace{3cm}

\vfill

\footnotesize
\textsc{Quelle}: \titel. Herausgegeben von {\editorInnen}. In: \emph{Arthur Schnitzler: Briefwechsel mit Autorinnen und Autoren}.
 Digitale Edition, https://schnitzler-briefe.acdh.oeaw.ac.at/{\dateiname}.html (Stand \today)
\fi

\end{document}


