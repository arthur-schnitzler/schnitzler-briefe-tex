%% latex-korrekturansicht-vorspann.tex
%% Vorspann für die Korrekturansicht.
%% Lädt die gemeinsame Datei latex-vorspann.tex mit gesetztem Schalter.

\newif\ifkorrekturansicht
\korrekturansichttrue

\input{../tex-inputs/latex-vorspann}


\section[ Paul Goldmann an Arthur Schnitzler, 22. 3. {[}1900{]}]{L02907 Paul Goldmann an Arthur Schnitzler, 22. 3. {[}1900{]}}
\nopagebreak\mylabel{L02907v}
\rehead{ }\normalsize\beginnumbering\briefempfaengerindex{Schnitzler, Arthur@\textsc{Schnitzler, Arthur}!zzzGoldmann, Paul@\emph{von Paul Goldmann}!1900-03-222@{22. 3. {[}1900{]}}|(be}
\toendnotes[C]{\smallbreak\pagebreak[2]}\Standort{DLA, A:Schnitzler, HS.NZ85.1.3170.}
\physDesc{Brief, 1 Blatt, 4 Seiten, 3049 Zeichen
\newline{}Handschrift: blaue Tinte, deutsche Kurrent
\newline{}Schnitzler: 1) mit rotem Buntstift acht Unterstreichungen  2) mit Bleistift die Jahreszahl »1900« ergänzt}\toendnotes[C]{\smallbreak}
\pstart
           {\pb}\textcolor{gray}{\textbf{DESSAUERSTRASSE 19}}\pend
           
\pstart
           \raggedleft{}Berlin\oindex{Berlin@\textbf{Berlin}, \emph{P.PPLC}|pw}, 22. März.\pend
           
\pstart\center{}Mein lieber Freund,\pend\vspace{0.5em}
\pstart
           Ich danke Dir für Deine lieben Briefe. Zum Antworten komme ich erſt heut, weil ich gar ſo viel zu thun hatte.\pend
           
\pstart
           Es iſt mir ſchmerzlich, daß Dein \label{K_L02907-1v}\edtext{Leid}{\lemma{\textnormal{\emph{Leid}}}\Cendnote{\textnormal{Bezug auf Marie Reinhards\pwindex{Reinhard, Marie 1871-03-13 – 1899-03-18@\textsc{Reinhard, Marie} (1871-03-13 – 1899-03-18), \emph{Gesangspädagoge/Gesangspädagogin}|pwk} Tod am 18. 3. 1899, also rund ein Jahr zuvor. Schnitzler notierte zu dieser Zeit mehrmals
                  damit zusammenhängende Verstimmungen in seinem \emph{Tagebuch}\pwindex{Tagebuch@\emph{Tagebuch}|pwk}. Vgl. A. S.: \emph{Tagebuch}, 13. 3. 1900, 14. 3. 1900, 15. 3. 1900 und 18. 3. 1900.}}}\label{K_L02907-1} ſich gar nicht lindern will. Gewiß, einen Erſatz
               für das Verlorene gibt es nicht. Aber es gibt Anderes, Neues, das auch gut ſein wird
               in ſeiner Art. Du wirſt doch nicht im Ernſt glauben wollen, daß Dein Leben
               abgeſchloſſen iſt? Geh’ nur nach dem Süden, das wird heilſam ſein.\pend
           
\pstart
           \textsc{Salten\pwindex{Salten, Felix 06.09.1869 – 08.10.1945@\textsc{Salten, Felix} (06.09.1869 – 08.10.1945), \emph{Schriftsteller/Schriftstellerin, Journalist/Journalistin, Chefredakteur/Chefredakteurin}|pw}} hat mir \label{K_L02907-2v}\edtext{diesmal}{\lemma{\textnormal{\emph{diesmal}}}\Cendnote{\textnormal{Felix Salten\pwindex{Salten, Felix 06.09.1869 – 08.10.1945@\textsc{Salten, Felix} (06.09.1869 – 08.10.1945), \emph{Schriftsteller/Schriftstellerin, Journalist/Journalistin, Chefredakteur/Chefredakteurin}|pwk} war mit dem Erzherzog Leopold Ferdinand\pwindex{Woelfling, Leopold Ferdinand Salvator 1868-12-02 – 1935-07-04@\textsc{Wölfling, Leopold Ferdinand Salvator} (1868-12-02 – 1935-07-04), \emph{Erzherzog/Erzherzogin}|pwk} seit 1898 gut bekannt. Dadurch gelangte er an brisante Informationen, die
                  als Tratschgeschichten in der Presse Aufsehen erregten und Salten\pwindex{Salten, Felix 06.09.1869 – 08.10.1945@\textsc{Salten, Felix} (06.09.1869 – 08.10.1945), \emph{Schriftsteller/Schriftstellerin, Journalist/Journalistin, Chefredakteur/Chefredakteurin}|pwk} über Wien\oindex{Wien@\textbf{Wien}, \emph{A.ADM2}|pwk} hinaus
                  bekannt machten. Vgl. Siegfried Mattl und Werner Michael Schwarz: \emph{Felix Salten. Annäherung an eine Biografie}. In: Siegfried
                     Mattl und Werner Michael Schwarz, Herausgegeber: \emph{Felix Salten.
                        Schriftsteller – Journalist – Exilant}. Wien:
                        \emph{Holzhausen}{ }2016, S. 17–72, hier: S. 32–35 und 42–44. Vgl. Felix Salten an Arthur Schnitzler, 28. 12. 1902.}}}\label{K_L02907-2}
               nicht ſonderlich gefallen. Lügt er nicht auch ein wenig? Die Geſchichten von dem Erzherzog\pwindex{Woelfling, Leopold Ferdinand Salvator 1868-12-02 – 1935-07-04@\textsc{Wölfling, Leopold Ferdinand Salvator} (1868-12-02 – 1935-07-04), \emph{Erzherzog/Erzherzogin}|pwv} können doch nicht
               alle wahr ſein. Ich glaube, er hält auf eine gewiſſe Anſtändigkeit, weil der Zufall
               es gefügt hat, daß er ſich an Dich angeſchloſſen hat. {\pb}Aber wenn der Zufall ihn zu den Andern geführt
               hätte, ſo wäre er geworden, wie dieſe, und vielleicht wird er es noch einmal.\pend
           
\pstart
           Die Fräuleins \textsc{Glümer}\pwindex{Gluemer, Marie 03.07.1867 – 16.11.1925@\textsc{Glümer, Marie} (03.07.1867 – 16.11.1925), \emph{Schauspieler/Schauspielerin}|pwv}\pwindex{Gluemer, Auguste 1862-03-16 – 1956@\textsc{Glümer, Auguste} (1862-03-16 – 1956), \emph{Lehrer/Lehrerin}|pwv} ſehe ich nicht ſo oft, als ich möchte. \textsc{Gusti\pwindex{Gluemer, Auguste 1862-03-16 – 1956@\textsc{Glümer, Auguste} (1862-03-16 – 1956), \emph{Lehrer/Lehrerin}|pw}}, die ich neulich vertraulich fragte, ob ſie Deinen \label{K_L02907-3v}\edtext{Brief}{\lemma{\textnormal{\emph{Brief}}}\Cendnote{\textnormal{nicht
                  ermittelt}}}\label{K_L02907-3} erhalten, ſagte: Ja.\pend
           
\pstart
           Eine Frau \textsc{Meyer-Cohn}\pwindex{Meyer-Cohn, Helene 1859-12-30 – 1918-11-09@\textsc{Meyer-Cohn, Helene} (1859-12-30 – 1918-11-09), \emph{Übersetzer/Übersetzerin}|pw}, bei der ich hier verkehre, ſagte mir, ſie ſei eine \label{K_L02907-4v}\edtext{Jugendbekannte}{\lemma{\textnormal{\emph{Jugendbekannte}}}\Cendnote{\textnormal{Siehe A. S.: \emph{Tagebuch}, 10. 7. 1893.
               }}}\label{K_L02907-4} von Dir. Mir ſcheint, ſie läßt Dich auch grüßen.\pend
           
\pstart
           Wie iſt \label{K_L02907-5v}\edtext{\textsc{Salten\pwindex{Salten, Felix 06.09.1869 – 08.10.1945@\textsc{Salten, Felix} (06.09.1869 – 08.10.1945), \emph{Schriftsteller/Schriftstellerin, Journalist/Journalistin, Chefredakteur/Chefredakteurin}|pw}’s}{ }Stück\pwindex{Gemeine. Schauspiel in drei Aufzuegen@\emph{Der Gemeine. Schauspiel in drei Aufzügen}|pwv}}{\lemma{\textnormal{\emph{Salten’s Stück}}}\Cendnote{\textnormal{Felix Salten\pwindex{Salten, Felix 06.09.1869 – 08.10.1945@\textsc{Salten, Felix} (06.09.1869 – 08.10.1945), \emph{Schriftsteller/Schriftstellerin, Journalist/Journalistin, Chefredakteur/Chefredakteurin}|pwk} hatte seinen Dreiakter \emph{Der Gemeine}\pwindex{Gemeine. Schauspiel in drei Aufzuegen@\emph{Der Gemeine. Schauspiel in drei Aufzügen}|pwk} am 2. 2. 1900 und 13. 2. 1900{ }Schnitzler vorgelesen.}}}\label{K_L02907-5}? Der
               Glückliche! Ihm iſt jetzt auch eine größere Arbeit gelungen. Ich bleibe allein
               zurück.\pend
           
\pstart
           Bleibe allein zurück in dem Journalismus, der mir unerträglicher iſt, als je. Und wie
               ich behandelt werde! Kein einziges meiner Theaterreferate wird mehr gedruckt, ohne
               daß vorher zwei Drittel herausgeſtrichen wären. {\pb}\strikeout{Ich} Oder: ich referire über ein Stück, und zwei Tage
               ſpäter wird in der Theaterrubrik\pwindex{Neue Freie Presse@\emph{Neue Freie Presse}|pwv} das Referat aus der »Nationalzeitung\pwindex{National-Zeitung@\emph{National-Zeitung}|pw}« abgedruckt, welches das Gegentheil ſagt. Oder: Man trägt
               mir telegraphiſch die Abfaſſung eines Artikels auf. Ich arbeite drei Tage, und der
               Artikel wird weggeworfen. \strikeout{So} So muß ich mich
               behandeln laſſen, ich, ein Menſch von Werth! Manchmal kommt mir das Weinen an über
               die Erniedrigung.\pend
           
\pstart
           \textsc{Herzl\pwindex{Herzl, Theodor 1860-05-02 – 1904-07-03@\textsc{Herzl, Theodor} (1860-05-02 – 1904-07-03), \emph{Schriftsteller/Schriftstellerin, Journalist/Journalistin}|pw}} als Feuilleton\pwindex{Neue Freie Presse@\emph{Neue Freie Presse}|pwv}-Redakteur
               iſt ſehr anſtändig. Das Alles aber muß unter uns bleiben. Du weißt, wie raſch in Wien\oindex{Wien@\textbf{Wien}, \emph{A.ADM2}|pw} ſich ſo etwas herumſpricht; und das könnte mir
               übel bekommen.\pend
           
\pstart
           Kein Weg, der aus dieſem entſetzlichen Berufe herausführt! Und ich werde alt und kann
               auch nicht mehr lange ſo arbeiten, wie bisher.\pend
           
\pstart
           Verkehr habe ich hier ſo gut wie keinen. {\pb}Mit wem
               ſollte ich auch verkehren? Als »Zeitungsſchreiber« bin ich ein Mann zweiten Ranges,
               und jeder Burſche, der einen ſchlechten Einakter aufführen läßt, dünkt ſich mehr als
               ich. \textsc{Kerr\pwindex{Kerr, Alfred 25.12.1867 – 12.10.1948@\textsc{Kerr, Alfred} (25.12.1867 – 12.10.1948), \emph{Schriftsteller/Schriftstellerin, Kritiker/Kritikerin}|pw}} iſt genau ſo eingebildet, als er begabt iſt. Er betrachtet mich nicht als
               gleichberechtigt, folglich bleibe ich ihm fern. \textsc{Brahm\pwindex{Brahm, Otto 05.02.1856 – 28.11.1912@\textsc{Brahm, Otto} (05.02.1856 – 28.11.1912), \emph{Theaterleiter/Theaterleiterin, Regisseur/Regisseurin}|pw}} habe ich einmal geſehen. Ich machte ihm meinen Antrittsbeſuch, und \strikeout{ſ\textcolor{gray}{o}} wir ſprachen über Berlin\oindex{Berlin@\textbf{Berlin}, \emph{P.PPLC}|pw} und Wien\oindex{Wien@\textbf{Wien}, \emph{A.ADM2}|pw}. Ich klagte, daß Berlin\oindex{Berlin@\textbf{Berlin}, \emph{P.PPLC}|pw} ſo unkünſtleriſch ſei. – »Nun, das wird ſich jetzt wohl
               beſſern, wo Sie da ſind«. – Seitdem bin ich natürlich nicht mehr wiedergekommen. Der
               einzig angenehme literariſche Menſch, den ich hier kennen gelernt habe, iſt \label{K_L02907-6v}\edtext{\textsc{Fritz Mauthner\pwindex{Mauthner, Fritz 1849-11-20 – 1923-06-29@\textsc{Mauthner, Fritz} (1849-11-20 – 1923-06-29), \emph{Schriftsteller/Schriftstellerin, Journalist/Journalistin, Philosoph/Philosophin}|pw}}}{\lemma{\textnormal{\emph{Fritz Mauthner}}}\Cendnote{\textnormal{Es ist zwar wahrscheinlich, jedoch nicht
                  eindeutig zu klären, ob sich Schnitzler und
                     Fritz Mauthner\pwindex{Mauthner, Fritz 1849-11-20 – 1923-06-29@\textsc{Mauthner, Fritz} (1849-11-20 – 1923-06-29), \emph{Schriftsteller/Schriftstellerin, Journalist/Journalistin, Philosoph/Philosophin}|pwk} persönlich kannten. Schnitzler las im Laufe seines Lebens
                  jedenfalls einige seiner Werke (vgl. A. S.: \emph{Tagebuch}, 11. 10. 1904 und 17. 12. 1916 sowie A. S.: \emph{Lektüren}, deutschsprachige Literatur).}}}\label{K_L02907-6}.
               Kennſt Du den? Ich ſehe ihn freilich alle ſechs Wochen einmal{\dotsfour}\pend
           
\pstart
           Was macht \textsc{Richard\pwindex{Beer-Hofmann, Richard 1866-07-11 – 1945-09-26@\textsc{Beer-Hofmann, Richard} (1866-07-11 – 1945-09-26), \emph{Schriftsteller/Schriftstellerin}|pw}}? Seht Ihr Euch oft? Wie lebſt Du und was treibſt Du?\pend
           
\pstart
           Schreib’ mir bald wieder!\pend
           
\pstart
           Viele treue Grüße! {\\[\baselineskip]}Dein \spacefill\mbox{Paul Goldmann.}\pend
           \leftskip=0em{}\selectlanguage{ngerman}\endnumbering\briefempfaengerindex{Schnitzler, Arthur@\textsc{Schnitzler, Arthur}!zzzGoldmann, Paul@\emph{von Paul Goldmann}!1900-03-222@{22. 3. {[}1900{]}}|)be}\mylabel{L02907h}  \normalsize

\doendnotes{C}
\bigskip
\vfill

\clearpage

\footnotesize

\lohead{\textsc{register}}

% Definiere theindex-Environment komplett neu ohne reledmac
\makeatletter
\renewenvironment{theindex}{%
  \section*{\indexname}%
  \setlength{\parindent}{0pt}%
  \setlength{\parskip}{0pt plus 0.3pt}%
  \let\item\@idxitem
}{%
  \clearpage
}
\makeatother

\IfFileExists{\jobname-pw.ind}{\input{\jobname-pw.ind}}{}

\end{document}

      