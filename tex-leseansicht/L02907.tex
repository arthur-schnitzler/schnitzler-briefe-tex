%% latex-leseansicht-vorspann.tex
%% Vorspann für die Leseansicht.
%% Lädt die gemeinsame Datei latex-vorspann.tex mit nicht gesetztem Schalter.

\newif\ifkorrekturansicht
\korrekturansichtfalse

\input{../tex-inputs/latex-vorspann}


\section[ Paul Goldmann an Arthur Schnitzler, 22. 3. {[}1900{]}]{L02907 Paul Goldmann an Arthur Schnitzler,  22. 3. [1900]}
\nopagebreak\mylabel{L02907v}
\rehead{ }\normalsize\beginnumbering\briefempfaengerindex{Schnitzler, Arthur@\textsc{Schnitzler, Arthur}!zzzGoldmann, Paul@\emph{von Paul Goldmann}!1900-03-222@{22. 3. [1900]}|(be}
\toendnotes[C]{\smallbreak\pagebreak[2]}
\correspDesc{Versand  durch Paul Goldmann am 22. 3. [1900] in Berlin
\newline{}Erhalt  durch Arthur Schnitzler im Zeitraum [23. 3. 1900
                  – 27. 3. 1900?] in Wien}\toendnotes[C]{\smallbreak}
\Standort{DLA, A:Schnitzler, HS.NZ85.1.3170.}
\physDesc{Brief, 1 Blatt, 4 Seiten, 3049 Zeichen
\newline{}Handschrift: blaue Tinte, deutsche Kurrent
\newline{}Schnitzler: 1) mit rotem Buntstift acht Unterstreichungen  2) mit Bleistift die Jahreszahl »1900« ergänzt}\toendnotes[C]{\smallbreak}
\pstart
           {\pb}\textcolor{gray}{\textbf{DESSAUERSTRASSE 19}}\pend
           
\pstart
           \raggedleft{}Berlin\oindex{Berlin@\textbf{Berlin}, \emph{Hauptstadt}|pw}, 22. März.\pend
           
\pstart\center{}Mein lieber Freund,\pend\vspace{0.5em}
\pstart
           Ich danke Dir für Deine lieben Briefe. Zum Antworten komme ich erſt heut, weil ich gar{ }ſo viel zu thun hatte.\pend
           
\pstart
           Es iſt mir{ }ſchmerzlich, daß Dein \label{K_L02907-1v}\edtext{Leid}{\lemma{\textnormal{\emph{Leid}}}\Cendnote{\textnormal{Bezug auf Marie Reinhards\pwindex{Reinhard, Marie 13.\,3.\,1871 Wien – 18.\,3.\,1899 ebd.@\textsc{Reinhard, Marie} (13.\,3.\,1871 Wien – 18.\,3.\,1899 ebd.), \emph{Gesangspädagogin}|pwk} Tod am 18. 3. 1899, also rund ein Jahr zuvor. Schnitzler notierte zu dieser Zeit mehrmals
                  damit zusammenhängende Verstimmungen in seinem \emph{Tagebuch}\pwindex{Schnitzler, Arthur 15.\,5.\,1862 Wien – 21.\,10.\,1931 ebd.@\textsc{Schnitzler, Arthur} (15.\,5.\,1862 Wien – 21.\,10.\,1931 ebd.), \emph{Schriftsteller, Mediziner}!Tagebuch@\strich\emph{Tagebuch}|pwk}. Vgl. A. S.: \emph{Tagebuch}, 13. 3. 1900, 14. 3. 1900, 15. 3. 1900 und 18. 3. 1900.}}}\label{K_L02907-1}{ }ſich gar nicht lindern will. Gewiß, einen Erſatz
               für das Verlorene gibt es nicht. Aber es gibt Anderes, Neues, das auch gut{ }ſein wird
               in{ }ſeiner Art. Du wirſt doch nicht im Ernſt glauben wollen, daß Dein Leben
               abgeſchloſſen iſt? Geh’ nur nach dem Süden, das wird heilſam{ }ſein.\pend
           
\pstart
           \textsc{Salten\pwindex{Salten, Felix 6.\,9.\,1869 Budapest – 8.\,10.\,1945 Zürich@\textsc{Salten, Felix} (6.\,9.\,1869 Budapest – 8.\,10.\,1945 Zürich), \emph{Schriftsteller, Journalist, Chefredakteur}|pw}} hat mir \label{K_L02907-2v}\edtext{diesmal}{\lemma{\textnormal{\emph{diesmal}}}\Cendnote{\textnormal{Felix Salten\pwindex{Salten, Felix 6.\,9.\,1869 Budapest – 8.\,10.\,1945 Zürich@\textsc{Salten, Felix} (6.\,9.\,1869 Budapest – 8.\,10.\,1945 Zürich), \emph{Schriftsteller, Journalist, Chefredakteur}|pwk} war mit dem Erzherzog Leopold Ferdinand\pwindex{Wölfling, Leopold Ferdinand Salvator 2.\,12.\,1868 Salzburg – 4.\,7.\,1935 Berlin@\textsc{Wölfling, Leopold Ferdinand Salvator} (2.\,12.\,1868 Salzburg – 4.\,7.\,1935 Berlin), \emph{Erzherzog}|pwk} seit 1898 gut bekannt. Dadurch gelangte er an brisante Informationen, die
                  als Tratschgeschichten in der Presse Aufsehen erregten und Salten\pwindex{Salten, Felix 6.\,9.\,1869 Budapest – 8.\,10.\,1945 Zürich@\textsc{Salten, Felix} (6.\,9.\,1869 Budapest – 8.\,10.\,1945 Zürich), \emph{Schriftsteller, Journalist, Chefredakteur}|pwk} über Wien\oindex{Wien@\textbf{Wien}, \emph{Verwaltungsgebiet}|pwk} hinaus
                  bekannt machten. Vgl. Siegfried Mattl und Werner Michael Schwarz: \emph{Felix Salten. Annäherung an eine Biografie}. In: Siegfried
                     Mattl und Werner Michael Schwarz, Herausgegeber: \emph{Felix Salten.
                        Schriftsteller – Journalist – Exilant}. Wien:
                        \emph{Holzhausen}{ }2016, S. 17–72, hier: S. 32–35 und 42–44. Vgl. XXXX Auszeichnungsfehler: Dokument L03338 nicht gefunden.}}}\label{K_L02907-2}
               nicht{ }ſonderlich gefallen. Lügt er nicht auch ein wenig? Die Geſchichten von dem Erzherzog\pwindex{Wölfling, Leopold Ferdinand Salvator 2.\,12.\,1868 Salzburg – 4.\,7.\,1935 Berlin@\textsc{Wölfling, Leopold Ferdinand Salvator} (2.\,12.\,1868 Salzburg – 4.\,7.\,1935 Berlin), \emph{Erzherzog}|pwv} können doch nicht
               alle wahr{ }ſein. Ich glaube, er hält auf eine gewiſſe Anſtändigkeit, weil der Zufall
               es gefügt hat, daß er{ }ſich an Dich angeſchloſſen hat. {\pb}Aber wenn der Zufall ihn zu den Andern geführt
               hätte,{ }ſo wäre er geworden, wie dieſe, und vielleicht wird er es noch einmal.\pend
           
\pstart
           Die Fräuleins \textsc{Glümer}\pwindex{Glümer, Marie 3.\,7.\,1867 Wien – 16.\,11.\,1925 München@\textsc{Glümer, Marie} (3.\,7.\,1867 Wien – 16.\,11.\,1925 München), \emph{Schauspielerin}|pwv}\pwindex{Glümer, Auguste 16.\,3.\,1862 Wien – 1956@\textsc{Glümer, Auguste} (16.\,3.\,1862 Wien – 1956), \emph{Lehrerin}|pwv}{ }ſehe ich nicht{ }ſo oft, als ich möchte. \textsc{Gusti\pwindex{Glümer, Auguste 16.\,3.\,1862 Wien – 1956@\textsc{Glümer, Auguste} (16.\,3.\,1862 Wien – 1956), \emph{Lehrerin}|pw}}, die ich neulich vertraulich fragte, ob{ }ſie Deinen \label{K_L02907-3v}\edtext{Brief}{\lemma{\textnormal{\emph{Brief}}}\Cendnote{\textnormal{nicht
                  ermittelt}}}\label{K_L02907-3} erhalten,{ }ſagte: Ja.\pend
           
\pstart
           Eine Frau \textsc{Meyer-Cohn}\pwindex{Meyer-Cohn, Helene 30.\,12.\,1859 Lviv – 9.\,11.\,1918 Berlin@\textsc{Meyer-Cohn, Helene} (30.\,12.\,1859 Lviv – 9.\,11.\,1918 Berlin), \emph{Übersetzerin}|pw}, bei der ich hier verkehre,{ }ſagte mir,{ }ſie{ }ſei eine \label{K_L02907-4v}\edtext{Jugendbekannte}{\lemma{\textnormal{\emph{Jugendbekannte}}}\Cendnote{\textnormal{Siehe A. S.: \emph{Tagebuch}, 10. 7. 1893.
               }}}\label{K_L02907-4} von Dir. Mir{ }ſcheint,{ }ſie läßt Dich auch grüßen.\pend
           
\pstart
           Wie iſt \label{K_L02907-5v}\edtext{\textsc{Salten\pwindex{Salten, Felix 6.\,9.\,1869 Budapest – 8.\,10.\,1945 Zürich@\textsc{Salten, Felix} (6.\,9.\,1869 Budapest – 8.\,10.\,1945 Zürich), \emph{Schriftsteller, Journalist, Chefredakteur}|pw}’s}{ }Stück\pwindex{Salten, Felix 6.\,9.\,1869 Budapest – 8.\,10.\,1945 Zürich@\textsc{Salten, Felix} (6.\,9.\,1869 Budapest – 8.\,10.\,1945 Zürich), \emph{Schriftsteller, Journalist, Chefredakteur}!Gemeine. Schauspiel in drei Aufzügen@\strich\emph{Der Gemeine. Schauspiel in drei Aufzügen}|pwv}}{\lemma{\textnormal{\emph{Salten’s Stück}}}\Cendnote{\textnormal{Felix Salten\pwindex{Salten, Felix 6.\,9.\,1869 Budapest – 8.\,10.\,1945 Zürich@\textsc{Salten, Felix} (6.\,9.\,1869 Budapest – 8.\,10.\,1945 Zürich), \emph{Schriftsteller, Journalist, Chefredakteur}|pwk} hatte seinen Dreiakter \emph{Der Gemeine}\pwindex{Salten, Felix 6.\,9.\,1869 Budapest – 8.\,10.\,1945 Zürich@\textsc{Salten, Felix} (6.\,9.\,1869 Budapest – 8.\,10.\,1945 Zürich), \emph{Schriftsteller, Journalist, Chefredakteur}!Gemeine. Schauspiel in drei Aufzügen@\strich\emph{Der Gemeine. Schauspiel in drei Aufzügen}|pwk} am 2. 2. 1900 und 13. 2. 1900{ }Schnitzler vorgelesen.}}}\label{K_L02907-5}? Der
               Glückliche! Ihm iſt jetzt auch eine größere Arbeit gelungen. Ich bleibe allein
               zurück.\pend
           
\pstart
           Bleibe allein zurück in dem Journalismus, der mir unerträglicher iſt, als je. Und wie
               ich behandelt werde! Kein einziges meiner Theaterreferate wird mehr gedruckt, ohne
               daß vorher zwei Drittel herausgeſtrichen wären. {\pb}\strikeout{Ich} Oder: ich referire über ein Stück, und zwei Tage{ }ſpäter wird in der Theaterrubrik\pwindex{Neue Freie Presse@\emph{Neue Freie Presse}|pwv} das Referat aus der »Nationalzeitung\pwindex{National-Zeitung@\emph{National-Zeitung}|pw}« abgedruckt, welches das Gegentheil{ }ſagt. Oder: Man trägt
               mir telegraphiſch die Abfaſſung eines Artikels auf. Ich arbeite drei Tage, und der
               Artikel wird weggeworfen. \strikeout{So} So muß ich mich
               behandeln laſſen, ich, ein Menſch von Werth! Manchmal kommt mir das Weinen an über
               die Erniedrigung.\pend
           
\pstart
           \textsc{Herzl\pwindex{Herzl, Theodor 2.\,5.\,1860 Budapest – 3.\,7.\,1904 Edlach@\textsc{Herzl, Theodor} (2.\,5.\,1860 Budapest – 3.\,7.\,1904 Edlach), \emph{Schriftsteller, Journalist}|pw}} als Feuilleton\pwindex{Neue Freie Presse@\emph{Neue Freie Presse}|pwv}-Redakteur
               iſt{ }ſehr anſtändig. Das Alles aber muß unter uns bleiben. Du weißt, wie raſch in Wien\oindex{Wien@\textbf{Wien}, \emph{Verwaltungsgebiet}|pw}{ }ſich{ }ſo etwas herumſpricht; und das könnte mir
               übel bekommen.\pend
           
\pstart
           Kein Weg, der aus dieſem entſetzlichen Berufe herausführt! Und ich werde alt und kann
               auch nicht mehr lange{ }ſo arbeiten, wie bisher.\pend
           
\pstart
           Verkehr habe ich hier{ }ſo gut wie keinen. {\pb}Mit wem{ }ſollte ich auch verkehren? Als »Zeitungsſchreiber« bin ich ein Mann zweiten Ranges,
               und jeder Burſche, der einen{ }ſchlechten Einakter aufführen läßt, dünkt{ }ſich mehr als
               ich. \textsc{Kerr\pwindex{Kerr, Alfred 25.\,12.\,1867 Breslau – 12.\,10.\,1948 Hamburg@\textsc{Kerr, Alfred} (25.\,12.\,1867 Breslau – 12.\,10.\,1948 Hamburg), \emph{Schriftsteller, Kritiker}|pw}} iſt genau{ }ſo eingebildet, als er begabt iſt. Er betrachtet mich nicht als
               gleichberechtigt, folglich bleibe ich ihm fern. \textsc{Brahm\pwindex{Brahm, Otto 5.\,2.\,1856 Hamburg – 28.\,11.\,1912 Berlin@\textsc{Brahm, Otto} (5.\,2.\,1856 Hamburg – 28.\,11.\,1912 Berlin), \emph{Theaterleiter, Regisseur}|pw}} habe ich einmal geſehen. Ich machte ihm meinen Antrittsbeſuch, und \strikeout{ſ\textcolor{gray}{o}} wir{ }ſprachen über Berlin\oindex{Berlin@\textbf{Berlin}, \emph{Hauptstadt}|pw} und Wien\oindex{Wien@\textbf{Wien}, \emph{Verwaltungsgebiet}|pw}. Ich klagte, daß Berlin\oindex{Berlin@\textbf{Berlin}, \emph{Hauptstadt}|pw}{ }ſo unkünſtleriſch{ }ſei. – »Nun, das wird{ }ſich jetzt wohl
               beſſern, wo Sie da{ }ſind«. – Seitdem bin ich natürlich nicht mehr wiedergekommen. Der
               einzig angenehme literariſche Menſch, den ich hier kennen gelernt habe, iſt \label{K_L02907-6v}\edtext{\textsc{Fritz Mauthner\pwindex{Mauthner, Fritz 20.\,11.\,1849 Hořice – 29.\,6.\,1923 Meersburg@\textsc{Mauthner, Fritz} (20.\,11.\,1849 Hořice – 29.\,6.\,1923 Meersburg), \emph{Schriftsteller, Journalist, Philosoph}|pw}}}{\lemma{\textnormal{\emph{Fritz Mauthner}}}\Cendnote{\textnormal{Es ist zwar wahrscheinlich, jedoch nicht
                  eindeutig zu klären, ob sich Schnitzler und
                     Fritz Mauthner\pwindex{Mauthner, Fritz 20.\,11.\,1849 Hořice – 29.\,6.\,1923 Meersburg@\textsc{Mauthner, Fritz} (20.\,11.\,1849 Hořice – 29.\,6.\,1923 Meersburg), \emph{Schriftsteller, Journalist, Philosoph}|pwk} persönlich kannten. Schnitzler las im Laufe seines Lebens
                  jedenfalls einige seiner Werke (vgl. A. S.: \emph{Tagebuch}, 11. 10. 1904 und 17. 12. 1916 sowie A. S.: \emph{Lektüren}, deutschsprachige Literatur).}}}\label{K_L02907-6}.
               Kennſt Du den? Ich{ }ſehe ihn freilich alle{ }ſechs Wochen einmal{\dotsfour}\pend
           
\pstart
           Was macht \textsc{Richard\pwindex{Beer-Hofmann, Richard 11.\,7.\,1866 Wien – 26.\,9.\,1945 New York City@\textsc{Beer-Hofmann, Richard} (11.\,7.\,1866 Wien – 26.\,9.\,1945 New York City), \emph{Schriftsteller}|pw}}? Seht Ihr Euch oft? Wie lebſt Du und was treibſt Du?\pend
           
\pstart
           Schreib’ mir bald wieder!\pend
           
\pstart
           Viele treue Grüße! {\\[\baselineskip]}Dein \spacefill\mbox{Paul Goldmann.}\pend
           \leftskip=0em{}\selectlanguage{ngerman}\endnumbering\briefempfaengerindex{Schnitzler, Arthur@\textsc{Schnitzler, Arthur}!zzzGoldmann, Paul@\emph{von Paul Goldmann}!1900-03-222@{22. 3. [1900]}|)be}\mylabel{L02907h}  \newcommand{\dateiname}{L02907}\newcommand{\titel}{Paul Goldmann an Arthur Schnitzler, 22. 3. [1900]}\newcommand{\editorInnen}{Martin Anton Müller und Laura Untner}%% latex-leseansicht-abspann.tex
%% Abspann für die Leseansicht.
%% Der Schalter \ifkorrekturansicht ist bereits durch den Vorspann gesetzt.

%% latex-abspann.tex
%% Gemeinsamer Abspann für Korrekturansicht und Leseansicht.
%% Setzt den Schalter \ifkorrekturansicht voraus (gesetzt in den
%% einbindenden Dateien latex-korrekturansicht-abspann.tex bzw.
%% latex-leseansicht-abspann.tex).
%% ---------------------------------------------------------------

\normalsize

% Das esempio-Environment wird nur in der Leseansicht benötigt
\ifkorrekturansicht\else
\newenvironment{esempio}[3]%
{
    \vspace{1.5ex}
    \rlap{\underline{#1}}
    \par
    \setlength{\parindent}{0cm}
    \nopagebreak
    \leftskip=#2cm
    \rightskip=#3cm
}
{
    \par
}
\fi

\doendnotes{C}
\bigskip
\vfill

\clearpage

\footnotesize

\ifkorrekturansicht
  \lohead{\textsc{register}}
\fi

% theindex-Environment neu definieren ohne reledmac
\makeatletter
\renewenvironment{theindex}{%
  \ifkorrekturansicht
    \section*{\indexname}%
  \else
    \subsubsection*{Index der erwähnten Entitäten}%
  \fi
  \setlength{\parindent}{0pt}%
  \setlength{\parskip}{0pt plus 0.3pt}%
  \let\item\@idxitem
}{%
  \ifkorrekturansicht\clearpage\fi
}
\makeatother

\IfFileExists{\jobname-pw.ind}{\input{\jobname-pw.ind}}{}

% Quellenangabe nur in der Leseansicht
\ifkorrekturansicht\else
% Fallback-Definitionen, falls die .tex-Datei \titel etc. nicht gesetzt hat
\providecommand{\titel}{}
\providecommand{\editorInnen}{}
\providecommand{\dateiname}{\jobname}

\vspace{3cm}

\vfill

\footnotesize
\textsc{Quelle}: \titel. Herausgegeben von {\editorInnen}. In: \emph{Arthur Schnitzler: Briefwechsel mit Autorinnen und Autoren}.
 Digitale Edition, https://schnitzler-briefe.acdh.oeaw.ac.at/{\dateiname}.html (Stand \today)
\fi

\end{document}


