%% latex-leseansicht-vorspann.tex
%% Vorspann für die Leseansicht.
%% Lädt die gemeinsame Datei latex-vorspann.tex mit nicht gesetztem Schalter.

\newif\ifkorrekturansicht
\korrekturansichtfalse

\input{../tex-inputs/latex-vorspann}

\begin{center}
            \textcolor{red}{ENTWURF, NICHT FERTIG KORRIGIERT}
                      \end{center}
            
         
         \newcommand{\erwaehntePersonen}{Personen: Richard Beer-Hofmann, Otto Brahm, Auguste Chlum, Marie Glümer, Theodor Herzl, Alfred Kerr,  Leopold Ferdinand Salvator, Fritz Mauthner, Helene Meyer-Cohn, Marie Reinhard, Felix Salten}
         \newcommand{\erwaehnteInstitutionen}{}
         \newcommand{\erwaehnteOrte}{Orte: Berlin, Wien}
         \newcommand{\erwaehnteWerke}{Werke: Der Gemeine. Schauspiel in drei Aufzügen, National-Zeitung, Neue Freie Presse, Tagebuch}
               \section[ Paul Goldmann an Arthur Schnitzler, 22. 3. {[}1900{]}]{ Paul Goldmann an Arthur Schnitzler, 22. 3. {[}1900{]}}\nopagebreak\mylabel{v}\rehead{ }\begin{ledgroupsized}[t]{13cm}\normalsize\beginnumbering \toendnotes[C]{\smallbreak\pagebreak[2]} \Standort{DLA, A:Schnitzler, HS.NZ85.1.3170.}
\physDesc{Brief, 1 Blatt, 4 Seiten
\newline{}Handschrift: blaue Tinte, deutsche Kurrent
\newline{}Schnitzler: mit rotem Buntstift acht Unterstreichungen }\toendnotes[C]{\smallbreak}\pstart{}{\pb}\textcolor{gray}{\textbf{DESSAUERSTRASSE 19}}\pend{}{\bigskip}\pstart
           \raggedleft{}Berlin\oindex{Berlin@\textbf{Berlin}|pw}, 22. März.\pend
           \pstart\center{}Mein lieber Freund,\pend\pstart
           Ich danke Dir für Deine lieben Briefe. Zum Antworten komme ich erſt heut, weil ich gar ſo viel zu thun hatte.\pend
           \pstart
           Es iſt mir ſchmerzlich, daß Dein \label{K_L02907-1v}\edtext{Leid}{\lemma{\textnormal{\emph{Leid}}}\Cendnote{\textnormal{Bezug auf Marie Reinhard\pwindex{Reinhard, Marie 1871-03-13 – 1899-03-18@\textsc{Reinhard, Marie} (1871-03-13 – 1899-03-18), \emph{Gesangspädagogin}|pwk}s Tod am 18. 3. 1899, also rund ein Jahr zuvor. Schnitzler\pwindex{Schnitzler, Arthur 15.05.1862 – 21.10.1931@\textsc{Schnitzler, Arthur} (15.05.1862 – 21.10.1931), \emph{Schriftsteller, Mediziner}|pwk} notierte zu dieser Zeit mehrmals
                  damit zusammenhängende Verstimmungen in seinem \emph{Tagebuch}\pwindex{Schnitzler, Arthur 15.05.1862 – 21.10.1931@\textsc{Schnitzler, Arthur} (15.05.1862 – 21.10.1931), \emph{Schriftsteller, Mediziner}!Tagebuch1981 – 2000@\strich\emph{Tagebuch} {[}1981 – 2000{]}|pwk}. Vgl. A. S.: \emph{Tagebuch}, 13. 3. 1900, A. S.: \emph{Tagebuch}, 14. 3. 1900, A. S.: \emph{Tagebuch}, 15. 3. 1900 und A. S.: \emph{Tagebuch}, 18. 3. 1900.}}}\label{K_L02907-1h} ſich gar nicht lindern
               will. Gewiß, einen Erſatz für das Verlorene gibt es nicht. Aber es gibt Anderes,
               Neues, das auch gut ſein wird in ſeiner Art. Du wirſt doch nicht im Ernſt glauben
               wollen, daß Dein Leben abgeſchloſſen iſt? Geh’ nur nach dem Süden, das wird heilſam
               ſein.\pend
           \pstart
           \textsc{Salten\pwindex{Salten, Felix 06.09.1869 – 08.10.1945@\textsc{Salten, Felix} (06.09.1869 – 08.10.1945), \emph{Schriftsteller, Journalist}|pw}} hat mir \label{K_L02907-2v}\edtext{diesmal}{\lemma{\textnormal{\emph{diesmal}}}\Cendnote{\textnormal{Felix Salten\pwindex{Salten, Felix 06.09.1869 – 08.10.1945@\textsc{Salten, Felix} (06.09.1869 – 08.10.1945), \emph{Schriftsteller, Journalist}|pwk} war mit dem Erzherzog Leopold Ferdinand\pwindex{Leopold Ferdinand Salvator 1868-12-02 – 1935-07-04@\textsc{Leopold Ferdinand Salvator} (1868-12-02 – 1935-07-04), \emph{Erzherzog}|pwk} seit 1898 gut bekannt. Dadurch gelangte er an brisante Informationen, die
                  als Tratschgeschichten in der Presse für Aufsehen erregten und Salten\pwindex{Salten, Felix 06.09.1869 – 08.10.1945@\textsc{Salten, Felix} (06.09.1869 – 08.10.1945), \emph{Schriftsteller, Journalist}|pwk} über Wien\oindex{Wien@\textbf{Wien}|pwk} hinaus
                  bekannt machten. Vgl. Siegfried Mattl und Werner Michael Schwarz: \emph{Felix Salten. Annäherung an eine Biografie}. In: Ebd.
                     (Hg.): \emph{Felix Salten. Schriftsteller – Journalist –
                        Exilant}. Wien:
                        \emph{Holzhausen}{ }2016, S. 17–72, hier: S. 32–35 u. 42–44.}}}\label{K_L02907-2h}
               nicht ſonderlich gefallen. Lügt er nicht auch ein wenig? Die Geſchichten von dem Erzherzog\pwindex{Leopold Ferdinand Salvator 1868-12-02 – 1935-07-04@\textsc{Leopold Ferdinand Salvator} (1868-12-02 – 1935-07-04), \emph{Erzherzog}|pwv} können doch nicht
               alle wahr ſein. Ich glaube, er hält auf eine gewiſſe Anſtändigkeit, weil der Zufall
               es gefügt hat, daß er ſich an Dich angeſchloſſen hat. {\pb}Aber wenn der Zufall ihn zu den Andern geführt
               hätte, ſo wäre er geworden, wie dieſe, und vielleicht wird er es noch einmal.\pend
           \pstart
           Die Fräuleins \textsc{Glümer}\pwindex{Gluemer, Marie 03.07.1867 – 16.11.1925@\textsc{Glümer, Marie} (03.07.1867 – 16.11.1925), \emph{Schauspielerin}|pwv}\pwindex{Chlum, Auguste 16.03.1862 – 1956@\textsc{Chlum, Auguste} (16.03.1862 – 1956)|pwv} ſehe ich nicht ſo oft, als ich möchte. \textsc{Gusti\pwindex{Chlum, Auguste 16.03.1862 – 1956@\textsc{Chlum, Auguste} (16.03.1862 – 1956)|pwv}}, die ich neulich vertraulich fragte, ob ſie Deinen \label{K_L02907-3v}\edtext{Brief}{\lemma{\textnormal{\emph{Brief}}}\Cendnote{\textnormal{möglicherweise enthalten in der Mappe 336 oder 337 im Deutschen Literaturarchiv
                  Marbach (HS.1985.1.836, HS.1985.1.837)}}}\label{K_L02907-3h} erhalten, ſagte: Ja.\pend
           \pstart
           Eine Frau \textsc{Meyer-Cohn}\pwindex{Meyer-Cohn, Helene 1859-12-30 – 1918-11-09@\textsc{Meyer-Cohn, Helene} (1859-12-30 – 1918-11-09), \emph{Übersetzerin}|pw}, bei der ich hier verkehre, ſagte mir, ſie ſei eine \label{K_L02907-4v}\edtext{Jugendbekannte}{\lemma{\textnormal{\emph{Jugendbekannte}}}\Cendnote{\textnormal{siehe A. S.: \emph{Tagebuch}, 7. 10. 1893}}}\label{K_L02907-4h} von Dir. Mir ſcheint, ſie läßt Dich auch grüßen.\pend
           \pstart
           Wie iſt \label{K_L02907-5v}\edtext{\textsc{Salten\pwindex{Salten, Felix 06.09.1869 – 08.10.1945@\textsc{Salten, Felix} (06.09.1869 – 08.10.1945), \emph{Schriftsteller, Journalist}|pw}’s}{ }Stück\pwindex{Salten, Felix 06.09.1869 – 08.10.1945@\textsc{Salten, Felix} (06.09.1869 – 08.10.1945), \emph{Schriftsteller, Journalist}!Gemeine. Schauspiel in drei Aufzuegen1901@\strich\emph{Der Gemeine. Schauspiel in drei Aufzügen} {[}1901{]}|pwv}}{\lemma{\textnormal{\emph{Salten’s Stück}}}\Cendnote{\textnormal{Felix Salten\pwindex{Salten, Felix 06.09.1869 – 08.10.1945@\textsc{Salten, Felix} (06.09.1869 – 08.10.1945), \emph{Schriftsteller, Journalist}|pwk} hatte Schnitzler\pwindex{Schnitzler, Arthur 15.05.1862 – 21.10.1931@\textsc{Schnitzler, Arthur} (15.05.1862 – 21.10.1931), \emph{Schriftsteller, Mediziner}|pwk} seinen Dreiakter \emph{Der Gemeine}\pwindex{Salten, Felix 06.09.1869 – 08.10.1945@\textsc{Salten, Felix} (06.09.1869 – 08.10.1945), \emph{Schriftsteller, Journalist}!Gemeine. Schauspiel in drei Aufzuegen1901@\strich\emph{Der Gemeine. Schauspiel in drei Aufzügen} {[}1901{]}|pwk} am 2. 2. 1900 und 13. 2. 1900 vorgelesen.}}}\label{K_L02907-5h}? Der Glückliche! Ihm iſt jetzt auch
               eine größere Arbeit gelungen. Ich bleibe allein zurück.\pend
           \pstart
           Bleibe allein zurück in dem Journalismus, der mir unerträglicher iſt, als je. Und wie
               ich behandelt werde! Kein einziges meiner Theaterreferate wird mehr gedruckt, ohne
               daß vorher zwei Drittel herausgeſtrichen wären. {\pb}\strikeout{Ich} Oder: ich referire über ein Stück, und zwei Tage
               ſpäter wird in der Theaterrubrik\pwindex{Neue Freie Presse1864 – 1939@\emph{Neue Freie Presse} {[}1864 – 1939{]}|pwv} das Referat aus der »Nationalzeitung\pwindex{?? Werk@Nicht ermittelte Verfasserinnen und Verfasser!National-Zeitung1848 – 1938@\emph{National-Zeitung} {[}1848 – 1938{]}|pw}« abgedruckt, welches das Gegentheil ſagt. Oder: Man trägt
               mir telegraphiſch die Abfaſſung eines Artikels auf. Ich arbeite drei Tage, und der
               Artikel wird weggeworfen. \strikeout{So} So muß ich mich
               behandeln laſſen, ich, ein Menſch von Werth! Manchmal kommt mir das Weinen an über
               die Erniedrigung.\pend
           \pstart
           \textsc{Herzl\pwindex{Herzl, Theodor 1860-05-02 – 1904-07-03@\textsc{Herzl, Theodor} (1860-05-02 – 1904-07-03), \emph{Schriftsteller, Journalist}|pw}} als Feuilleton\pwindex{Neue Freie Presse1864 – 1939@\emph{Neue Freie Presse} {[}1864 – 1939{]}|pwv}-Redakteur
               iſt ſehr anſtändig. Das Alles aber muß unter uns bleiben. Du weißt, wie raſch in Wien\oindex{Wien@\textbf{Wien}|pw} ſich ſo etwas herumſpricht; und das könnte mir
               übel bekommen.\pend
           \pstart
           Kein Weg, der aus dieſem entſetzlichen Berufe herausführt! Und ich werde alt und kann
               auch nicht mehr lange ſo arbeiten, wie bisher.\pend
           \pstart
           Verkehr habe ich hier ſo gut wie keinen. {\pb}Mit wem
               ſollte ich auch verkehren? Als »Zeitungsſchreiber« bin ich ein Mann zweiten Ranges,
               und jeder Burſche der einen ſchlechten Einakter aufführen läßt, dünkt ſich mehr als
               ich. \textsc{Kerr\pwindex{Kerr, Alfred 25.12.1867 – 12.10.1948@\textsc{Kerr, Alfred} (25.12.1867 – 12.10.1948), \emph{Schriftsteller, Kritiker}|pw}} iſt genau ſo eingebildet, als er begabt iſt. Er betrachtet mich nicht als
               gleichberechtigt, folglich bleibe ich ihm fern. \textsc{Brahm\pwindex{Brahm, Otto 05.02.1856 – 28.11.1912@\textsc{Brahm, Otto} (05.02.1856 – 28.11.1912), \emph{Theaterleiter, Regisseur}|pw}} habe ich einmal geſehen. Ich machte ihm meinen Antrittsbeſuch, und \strikeout{ſ\textcolor{gray}{o}} wir ſprachen über Berlin\oindex{Berlin@\textbf{Berlin}|pw} und Wien\oindex{Wien@\textbf{Wien}|pw}. Ich klagte, daß Berlin\oindex{Berlin@\textbf{Berlin}|pw} ſo unkünſtleriſch ſei. – »Nun, das wird ſich jetzt wohl
               beſſern, wo Sie da ſind«. – Seitdem bin ich natürlich nicht mehr wiedergekommen. Der
               einzig angenehme literariſche Menſch, den ich hier kennen gelernt habe, iſt \label{K_L02907-7v}\edtext{\textsc{Fritz Mauthner\pwindex{Mauthner, Fritz 1849-11-20 – 1923-06-29@\textsc{Mauthner, Fritz} (1849-11-20 – 1923-06-29), \emph{Schriftsteller, Journalist, Philosoph}|pw}}}{\lemma{\textnormal{\emph{Fritz Mauthner}}}\Cendnote{\textnormal{Es ist zwar wahrscheinlich, jedoch nicht
                  eindeutig zu klären, ob sich Schnitzler\pwindex{Schnitzler, Arthur 15.05.1862 – 21.10.1931@\textsc{Schnitzler, Arthur} (15.05.1862 – 21.10.1931), \emph{Schriftsteller, Mediziner}|pwk} und
                     Fritz Mauthner\pwindex{Mauthner, Fritz 1849-11-20 – 1923-06-29@\textsc{Mauthner, Fritz} (1849-11-20 – 1923-06-29), \emph{Schriftsteller, Journalist, Philosoph}|pwk} persönlich kannten. Schnitzler\pwindex{Schnitzler, Arthur 15.05.1862 – 21.10.1931@\textsc{Schnitzler, Arthur} (15.05.1862 – 21.10.1931), \emph{Schriftsteller, Mediziner}|pwk} las im Laufe seines Lebens
                  jedenfalls einige seiner Werke (vgl. A. S.: \emph{Tagebuch}, 11. 10. 1904, A. S.: \emph{Tagebuch}, 17. 12. 1916, A. S.: \emph{Lektüren}, Deutschsprachige-Literatur).}}}\label{K_L02907-7h}. Kennſt Du den? Ich ſehe ihn
               freilich alle ſechs Wochen einmal{\dotsfour}\pend
           \pstart
           Was macht \textsc{Richard\pwindex{Beer-Hofmann, Richard 1866-07-11 – 1945-09-26@\textsc{Beer-Hofmann, Richard} (1866-07-11 – 1945-09-26), \emph{Schriftsteller}|pw}}? Seht Ihr Euch oft? Wie lebſt Du und was treibſt Du?\pend
           \pstart
           Schreib’ mir bald wieder!\pend
           \pstart
           Viele treue Grüße! {\\[\baselineskip]}Dein \spacefill\mbox{Paul Goldmann.}\pend
           \leftskip=0em{}
         
         \endnumbering\mylabel{h}\end{ledgroupsized}\begin{anhang}\end{anhang}\newcommand{\dateiname}{L02907}\newcommand{\titel}{Paul Goldmann an Arthur Schnitzler, 22. 3. [1900]}\newcommand{\editorInnen}{Martin Anton Müller und Laura Untner}%% latex-leseansicht-abspann.tex
%% Abspann für die Leseansicht.
%% Der Schalter \ifkorrekturansicht ist bereits durch den Vorspann gesetzt.

%% latex-abspann.tex
%% Gemeinsamer Abspann für Korrekturansicht und Leseansicht.
%% Setzt den Schalter \ifkorrekturansicht voraus (gesetzt in den
%% einbindenden Dateien latex-korrekturansicht-abspann.tex bzw.
%% latex-leseansicht-abspann.tex).
%% ---------------------------------------------------------------

\normalsize

% Das esempio-Environment wird nur in der Leseansicht benötigt
\ifkorrekturansicht\else
\newenvironment{esempio}[3]%
{
    \vspace{1.5ex}
    \rlap{\underline{#1}}
    \par
    \setlength{\parindent}{0cm}
    \nopagebreak
    \leftskip=#2cm
    \rightskip=#3cm
}
{
    \par
}
\fi

\doendnotes{C}
\bigskip
\vfill

\clearpage

\footnotesize

\ifkorrekturansicht
  \lohead{\textsc{register}}
\fi

% theindex-Environment neu definieren ohne reledmac
\makeatletter
\renewenvironment{theindex}{%
  \ifkorrekturansicht
    \section*{\indexname}%
  \else
    \subsubsection*{Index der erwähnten Entitäten}%
  \fi
  \setlength{\parindent}{0pt}%
  \setlength{\parskip}{0pt plus 0.3pt}%
  \let\item\@idxitem
}{%
  \ifkorrekturansicht\clearpage\fi
}
\makeatother

\IfFileExists{\jobname-pw.ind}{\input{\jobname-pw.ind}}{}

% Quellenangabe nur in der Leseansicht
\ifkorrekturansicht\else
% Fallback-Definitionen, falls die .tex-Datei \titel etc. nicht gesetzt hat
\providecommand{\titel}{}
\providecommand{\editorInnen}{}
\providecommand{\dateiname}{\jobname}

\vspace{3cm}

\vfill

\footnotesize
\textsc{Quelle}: \titel. Herausgegeben von {\editorInnen}. In: \emph{Arthur Schnitzler: Briefwechsel mit Autorinnen und Autoren}.
 Digitale Edition, https://schnitzler-briefe.acdh.oeaw.ac.at/{\dateiname}.html (Stand \today)
\fi

\end{document}


      