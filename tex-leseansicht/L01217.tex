%% latex-korrekturansicht-vorspann.tex
%% Vorspann für die Korrekturansicht.
%% Lädt die gemeinsame Datei latex-vorspann.tex mit gesetztem Schalter.

\newif\ifkorrekturansicht
\korrekturansichttrue

\input{../tex-inputs/latex-vorspann}


\section[Arthur Schnitzler an Richard Beer-Hofmann, 2. {[}5.?{]} 1902]{L01217 Arthur Schnitzler an Richard Beer-Hofmann, 2. {[}5.?{]} 1902}
\nopagebreak\mylabel{L01217v}
\rehead{ }\normalsize\beginnumbering\briefempfaengerindex{Beer-Hofmann, Richard@\textsc{Beer-Hofmann, Richard}!zzzSchnitzler, Arthur@\emph{von Arthur Schnitzler}!1902-05-021@{2. {[}5.?{]} 1902}|(be}
\toendnotes[C]{\smallbreak\pagebreak[2]}\Standort{YCGL, MSS 31.}
\physDesc{Briefkarte, , Umschlag, 441 Zeichen
\newline{}Handschrift: Bleistift, deutsche Kurrent
\newline{}Versand: 1) Stempel: »\nobreak{}\textcolor{gray}{Wien}, \textcolor{gray}{2 5} 02, 5–6N\nobreak{}«.   2) Stempel: »\nobreak{}\oindex{Rodaun@\textbf{Rodaun}, \emph{A.ADM4}|pwk}{\pb}Rodaun, 3. \textcolor{gray}{5}. 02, \textcolor{gray}{7–9}V\nobreak{}«. 
\newline{}Ordnung: mit Bleistift von unbekannter Hand falsch datiert: »3. 3.« }
\buchAbdrucke{\weitereDrucke{Arthur Schnitzler, Richard Beer-Hofmann: \emph{Briefwechsel 1891–1931}. Wien, Zürich: \emph{Europaverlag} 1992, S. 157.} }\toendnotes[C]{\smallbreak}\pstart{}{\pb}Herrn \textsc{Dr. Richard
                     Beer-Hofmann}\pend{}\pstart{}\textsc{Rodaun}\oindex{Rodaun@\textbf{Rodaun}, \emph{A.ADM4}|pw}\pend{}\pstart{}\textsc{Liesinger Straße 2\oindex{Liesingerstrasse@\textbf{Liesingerstraße}, \emph{Straße (K.STR)}|pw}}\pend{}{\bigskip}\vspace{1em}
\pstart
           \noindent{}{\pb}lieber Richard, ich weiſs nicht, ob Sie Sitze haben, jedenfalls
               laſſe ich Ihnen bis \label{K_L01217-1v}\edtext{Dinſtag}{\lemma{\textnormal{\emph{Dinſtag}}}\Cendnote{\textnormal{Die Poststempel dieses
                  Korrespondenzstücks sind, mit Ausnahme der Jahresangabe, nur unzuverlässig zu
                  entziffern, weswegen es bislang auch mit 2. 3. 1902 datiert wurde. Da
                  es sich aber um einen Zeitraum handeln muss, in dem Brahm\pwindex{Brahm, Otto 05.02.1856 – 28.11.1912@\textsc{Brahm, Otto} (05.02.1856 – 28.11.1912), \emph{Theaterleiter/Theaterleiterin, Regisseur/Regisseurin}|pwk} für das Gastspiel im Carl-Theater\oindex{Carl-Theater@\textbf{Carl-Theater}, \emph{Theater (K.THE)}|pwk} in Wien\oindex{Wien@\textbf{Wien}, \emph{A.ADM2}|pwk} weilte, ist die
                  Monatsangabe mit Mai anzusetzen und mit »Dienstag« der 6. 5. 1902 gemeint, der erste Tag des
                  Gastspiels. Dazu passt auch das Telegramm Brahms\pwindex{Brahm, Otto 05.02.1856 – 28.11.1912@\textsc{Brahm, Otto} (05.02.1856 – 28.11.1912), \emph{Theaterleiter/Theaterleiterin, Regisseur/Regisseurin}|pwk} vom 2. 5. 1902 (\emph{Der Briefwechsel Arthur Schnitzler – Otto Brahm}.
                     Vollständige Ausgabe. Herausgegeben, eingeleitet und erläutert von Oskar
                     Seidlin. Tübingen: \emph{Niemeyer}{ }1975, S. 122), in dem er die hier in Folge an Hofmannsthal\pwindex{Hofmannsthal, Hugo von 1874-02-01 – 1929-07-15@\textsc{Hofmannsthal, Hugo von} (1874-02-01 – 1929-07-15), \emph{Schriftsteller/Schriftstellerin}|pwk} weiterzugebende Antwort
                  kommuniziert.}}}\label{K_L01217-1}{ }Mittag an der Carltheater\oindex{Carl-Theater@\textbf{Carl-Theater}, \emph{Theater (K.THE)}|pw} Caſſe
               2 Parkets reſerviren. Holen Sie ſie nicht, ſo werden ſie anderweitig {\pb}verkauft. – Sie haben ſich alſo nicht weiter zu
               kümmern. –\pend
           
\pstart
           Dem Hugo\pwindex{Hofmannsthal, Hugo von 1874-02-01 – 1929-07-15@\textsc{Hofmannsthal, Hugo von} (1874-02-01 – 1929-07-15), \emph{Schriftsteller/Schriftstellerin}|pw} ſagen Sie bitte, \uline{aber sicher}, dſs Brahm\pwindex{Brahm, Otto 05.02.1856 – 28.11.1912@\textsc{Brahm, Otto} (05.02.1856 – 28.11.1912), \emph{Theaterleiter/Theaterleiterin, Regisseur/Regisseurin}|pw}{ }Dinſtag{ }\uline{nicht} zu mir kommt.\pend
           
\pstart
           Ich hoffe übrigens So{\geminationn}tag{ }Vormittag{ }Rodaun\oindex{Rodaun@\textbf{Rodaun}, \emph{A.ADM4}|pw} zu durchradeln.\pend
           
\pstart
           Herzlichſt Ihr{\\[\baselineskip]}\spacefill\mbox{A.}\pend
           \leftskip=0em{}\selectlanguage{ngerman}\endnumbering\briefempfaengerindex{Beer-Hofmann, Richard@\textsc{Beer-Hofmann, Richard}!zzzSchnitzler, Arthur@\emph{von Arthur Schnitzler}!1902-05-021@{2. {[}5.?{]} 1902}|)be}\mylabel{L01217h}  \normalsize

\doendnotes{C}
\bigskip
\vfill

\clearpage

\footnotesize

\lohead{\textsc{register}}

% Definiere theindex-Environment komplett neu ohne reledmac
\makeatletter
\renewenvironment{theindex}{%
  \section*{\indexname}%
  \setlength{\parindent}{0pt}%
  \setlength{\parskip}{0pt plus 0.3pt}%
  \let\item\@idxitem
}{%
  \clearpage
}
\makeatother

\IfFileExists{\jobname-pw.ind}{\input{\jobname-pw.ind}}{}

\end{document}

      