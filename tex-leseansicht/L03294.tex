%% latex-leseansicht-vorspann.tex
%% Vorspann für die Leseansicht.
%% Lädt die gemeinsame Datei latex-vorspann.tex mit nicht gesetztem Schalter.

\newif\ifkorrekturansicht
\korrekturansichtfalse

\input{../tex-inputs/latex-vorspann}


\section[ Felix Salten an Arthur Schnitzler, {[}27. 6. 1899{]}]{L03294 Felix Salten an Arthur Schnitzler,  [27. 6. 1899]}
\nopagebreak\mylabel{L03294v}
\rehead{ }\normalsize\beginnumbering\briefempfaengerindex{Schnitzler, Arthur@\textsc{Schnitzler, Arthur}!zzzSalten, Felix@\emph{von Felix Salten}!1899-06-271@{{[}27. 6. 1899{]}}|(be}
\toendnotes[C]{\smallbreak\pagebreak[2]}
\correspDesc{Versand  durch Felix Salten am [27. 6. 1899] in Wien
\newline{}Erhalt  durch Arthur Schnitzler im Zeitraum [28. 6. 1899
                  – 2. 7. 1899?] in Wien?}\toendnotes[C]{\smallbreak}
\Standort{CUL, Schnitzler, B 89, A 2.}
\physDesc{Brief, 1 Blatt, 1 Seite, 587 Zeichen
\newline{}Handschrift: schwarze Tinte, lateinische Kurrent
\newline{}Schnitzler: mit Bleistift datiert: »27/6 99« 
\newline{}Ordnung: mit Bleistift von unbekannter Hand nummeriert: »118« }\toendnotes[C]{\smallbreak}
\pstart
           \noindent{}{\pb}Lieber Freund, wegen des »Liebesreigen\pwindex{Schnitzler, Arthur 15.\,5.\,1862 Wien – 21.\,10.\,1931 ebd.@\textsc{Schnitzler, Arthur} (15.\,5.\,1862 Wien – 21.\,10.\,1931 ebd.), \emph{Schriftsteller, Mediziner}!Reigen. Zehn Dialoge@\strich\emph{Reigen. Zehn Dialoge}|pw}« möchte ich so bald wie möglich mit Ihnen \label{K_L03294-1v}\edtext{sprechen}{\lemma{\textnormal{\emph{sprechen}}}\Cendnote{\textnormal{Er dachte an eine Veröffentlichung in der \emph{Wiener Allgemeinen Montags-Zeitung}\pwindex{Wiener Allgemeine Montags-Zeitung@\emph{Wiener Allgemeine Montags-Zeitung}|pwk}, zu der es aber nicht
                  kam (siehe XXXX Auszeichnungsfehler: Dokument L03293 nicht gefunden).}}}\label{K_L03294-1}. D\textsuperscript{r}{ }Szeps\pwindex{Szeps, Julius 27.\,10.\,1867 Wien – 27.\,10.\,1924 ebd.@\textsc{Szeps, Julius} (27.\,10.\,1867 Wien – 27.\,10.\,1924 ebd.), \emph{Journalist}|pw} macht im Ministerium\orgindex{Ministerium für Inneres@Ministerium für Inneres|pwuv} Anstrengungen denselben
               durchzusetzen, und an eventuelle Aufregung im Leserkreis kehre ich mich nicht. Ich
               könnte Ihnen ein nicht unbeträchtliches Honorar dafür bieten, und glaube, wenn es
               durch Vermittlung des Ministers\pwindex{Koerber, Ernest von 6.\,11.\,1850 Trient – 5.\,3.\,1919 Baden bei Wien@\textsc{Koerber, Ernest von} (6.\,11.\,1850 Trient – 5.\,3.\,1919 Baden bei Wien), \emph{Politiker}|pwv} gelingt, die Sache durch die Censur zu drücken, wäre ein wichtiges
               Präjudiz geschaffen, das Ihnen auch für eine \label{K_L03294-2v}\edtext{Buchausgabe}{\lemma{\textnormal{\emph{Buchausgabe}}}\Cendnote{\textnormal{Schnitzler
                     ließ einen Privatdruck von \emph{Reigen}\pwindex{Schnitzler, Arthur 15.\,5.\,1862 Wien – 21.\,10.\,1931 ebd.@\textsc{Schnitzler, Arthur} (15.\,5.\,1862 Wien – 21.\,10.\,1931 ebd.), \emph{Schriftsteller, Mediziner}!Reigen. Zehn Dialoge@\strich\emph{Reigen. Zehn Dialoge}|pwk} mit einer Auflage von 200 Stück erstellen, den er 
                     im 
                     April 1900 an Freunde verteilte (vgl. XXXX Auszeichnungsfehler: Dokument L01032 nicht gefunden). 
                      1903 erschien das Theaterstück\pwindex{Schnitzler, Arthur 15.\,5.\,1862 Wien – 21.\,10.\,1931 ebd.@\textsc{Schnitzler, Arthur} (15.\,5.\,1862 Wien – 21.\,10.\,1931 ebd.), \emph{Schriftsteller, Mediziner}!Reigen. Zehn Dialoge@\strich\emph{Reigen. Zehn Dialoge}|pwkv} im \emph{Wiener Verlag}\orgindex{Wiener Verlag@Wiener Verlag|pwk}.}}}\label{K_L03294-2} sehr
               werthvoll sein könnte. Bitte, theilen Sie mir gleich nach Ihrer \label{K_L03294-3v}\edtext{Rückkunft}{\lemma{\textnormal{\emph{Rückkunft}}}\Cendnote{\textnormal{Schnitzler kehrte am 28. 6. 1899 nach Wien\oindex{Wien@\textbf{Wien}, \emph{Verwaltungsgebiet}|pwk} zurück.}}}\label{K_L03294-3} mit, wann ich Sie sprechen
               kann.\pend
           
\pstart
           Herzlichst Ihr {\\[\baselineskip]}\spacefill\mbox{Salten}\pend
           \leftskip=0em{}\selectlanguage{ngerman}\endnumbering\briefempfaengerindex{Schnitzler, Arthur@\textsc{Schnitzler, Arthur}!zzzSalten, Felix@\emph{von Felix Salten}!1899-06-271@{{[}27. 6. 1899{]}}|)be}\mylabel{L03294h}  \newcommand{\dateiname}{L03294}\newcommand{\titel}{Felix Salten an Arthur Schnitzler, [27. 6. 1899]}\newcommand{\editorInnen}{Martin Anton Müller und Laura Untner}%% latex-leseansicht-abspann.tex
%% Abspann für die Leseansicht.
%% Der Schalter \ifkorrekturansicht ist bereits durch den Vorspann gesetzt.

%% latex-abspann.tex
%% Gemeinsamer Abspann für Korrekturansicht und Leseansicht.
%% Setzt den Schalter \ifkorrekturansicht voraus (gesetzt in den
%% einbindenden Dateien latex-korrekturansicht-abspann.tex bzw.
%% latex-leseansicht-abspann.tex).
%% ---------------------------------------------------------------

\normalsize

% Das esempio-Environment wird nur in der Leseansicht benötigt
\ifkorrekturansicht\else
\newenvironment{esempio}[3]%
{
    \vspace{1.5ex}
    \rlap{\underline{#1}}
    \par
    \setlength{\parindent}{0cm}
    \nopagebreak
    \leftskip=#2cm
    \rightskip=#3cm
}
{
    \par
}
\fi

\doendnotes{C}
\bigskip
\vfill

\clearpage

\footnotesize

\ifkorrekturansicht
  \lohead{\textsc{register}}
\fi

% theindex-Environment neu definieren ohne reledmac
\makeatletter
\renewenvironment{theindex}{%
  \ifkorrekturansicht
    \section*{\indexname}%
  \else
    \subsubsection*{Index der erwähnten Entitäten}%
  \fi
  \setlength{\parindent}{0pt}%
  \setlength{\parskip}{0pt plus 0.3pt}%
  \let\item\@idxitem
}{%
  \ifkorrekturansicht\clearpage\fi
}
\makeatother

\IfFileExists{\jobname-pw.ind}{\input{\jobname-pw.ind}}{}

% Quellenangabe nur in der Leseansicht
\ifkorrekturansicht\else
% Fallback-Definitionen, falls die .tex-Datei \titel etc. nicht gesetzt hat
\providecommand{\titel}{}
\providecommand{\editorInnen}{}
\providecommand{\dateiname}{\jobname}

\vspace{3cm}

\vfill

\footnotesize
\textsc{Quelle}: \titel. Herausgegeben von {\editorInnen}. In: \emph{Arthur Schnitzler: Briefwechsel mit Autorinnen und Autoren}.
 Digitale Edition, https://schnitzler-briefe.acdh.oeaw.ac.at/{\dateiname}.html (Stand \today)
\fi

\end{document}


