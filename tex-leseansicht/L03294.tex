%% latex-leseansicht-vorspann.tex
%% Vorspann für die Leseansicht.
%% Lädt die gemeinsame Datei latex-vorspann.tex mit nicht gesetztem Schalter.

\newif\ifkorrekturansicht
\korrekturansichtfalse

\input{../tex-inputs/latex-vorspann}


         
         \renewcommand{\erwaehntePersonen}{Personen: Ernest von Koerber, Felix Salten, Julius Szeps}
         \renewcommand{\erwaehnteInstitutionen}{Institutionen: Ministerium für Inneres, Wiener Verlag}
         \renewcommand{\erwaehnteOrte}{Orte: Wien}
         \renewcommand{\erwaehnteWerke}{Werke: Reigen. Zehn Dialoge, Wiener Allgemeine Montags-Zeitung}
               \section[ Felix Salten an Arthur Schnitzler, {[}27. 6. 1899{]}]{ Felix Salten an Arthur Schnitzler, {[}27. 6. 1899{]}}\nopagebreak\mylabel{v}\rehead{ }\begin{ledgroupsized}[t]{13cm}\normalsize\beginnumbering \toendnotes[C]{\smallbreak\pagebreak[2]} \Standort{CUL, Schnitzler, B 89, A 2.}
\physDesc{Brief, 1 Blatt, 1 Seite, 587 Zeichen
\newline{}Handschrift: schwarze Tinte, lateinische Kurrent
\newline{}Schnitzler: mit Bleistift datiert: »27/6 99« 
\newline{}Ordnung: mit Bleistift von unbekannter Hand nummeriert: »118« }\toendnotes[C]{\smallbreak}\pstart
           \noindent{}{\pb}Lieber Freund, wegen des »Liebesreigen\pwindex{Schnitzler, Arthur 15.05.1862 – 21.10.1931@\textsc{Schnitzler, Arthur} (15.05.1862 – 21.10.1931), \emph{Schriftsteller, Mediziner}!Reigen. Zehn Dialoge1900@\strich\emph{Reigen. Zehn Dialoge} {[}1900{]}|pw}« möchte ich so bald wie möglich mit Ihnen \label{K_L03294-1v}\edtext{sprechen}{\lemma{\textnormal{\emph{sprechen}}}\Cendnote{\textnormal{wegen einer Veröffentlichung in der \emph{Wiener Allgemeinen Montags-Zeitung}\pwindex{Wiener Allgemeine Montags-Zeitung1899-07-03 – 1899-12-18@\emph{Wiener Allgemeine Montags-Zeitung} {[}1899-07-03 – 1899-12-18{]}|pwk}, zu der es aber nicht
                  kam (siehe Felix Salten an Arthur Schnitzler, 21. 6. 1899)}}}\label{K_L03294-1h}. D\textsuperscript{r}{ }Szeps\pwindex{Szeps, Julius 1867-10-27 – 27.10.1924@\textsc{Szeps, Julius} (1867-10-27 – 27.10.1924), \emph{Journalist}|pw} macht im Ministerium\orgindex{Ministerium fuer Inneres@Ministerium für Inneres|pwuv} Anstrengungen denselben
               durchzusetzen, und an eventuelle Aufregung im Leserkreis kehre ich mich nicht. Ich
               könnte Ihnen ein nicht unbeträchtliches Honorar dafür bieten, und glaube, wenn es
               durch Vermittlung des Minister\pwindex{Koerber, Ernest von 06.11.1850 – 05.03.1919@\textsc{Koerber, Ernest von} (06.11.1850 – 05.03.1919), \emph{Politiker}|pwv}s gelingt, die Sache durch die Censur zu drücken, wäre ein wichtiges
               Präjudiz geschaffen, das Ihnen auch für eine \label{K_L03294-2v}\edtext{Buchausgabe}{\lemma{\textnormal{\emph{Buchausgabe}}}\Cendnote{\textnormal{Schnitzler\pwindex{Schnitzler, Arthur 15.05.1862 – 21.10.1931@\textsc{Schnitzler, Arthur} (15.05.1862 – 21.10.1931), \emph{Schriftsteller, Mediziner}|pwk}
                     ließ einen Privatdruck von \emph{Reigen}\pwindex{Schnitzler, Arthur 15.05.1862 – 21.10.1931@\textsc{Schnitzler, Arthur} (15.05.1862 – 21.10.1931), \emph{Schriftsteller, Mediziner}!Reigen. Zehn Dialoge1900@\strich\emph{Reigen. Zehn Dialoge} {[}1900{]}|pwk} mit einer Auflage von 200 Stück erstellen, den er 
                     im 
                     April 1900 an Freunde verteilte (vgl. Hermann Bahr an Arthur Schnitzler, 20. 4. [1900]). 
                      1903 erschien das Theaterstück\pwindex{Schnitzler, Arthur 15.05.1862 – 21.10.1931@\textsc{Schnitzler, Arthur} (15.05.1862 – 21.10.1931), \emph{Schriftsteller, Mediziner}!Reigen. Zehn Dialoge1900@\strich\emph{Reigen. Zehn Dialoge} {[}1900{]}|pwkv} im \emph{Wiener Verlag}\orgindex{Wiener Verlag@Wiener Verlag|pwk}.}}}\label{K_L03294-2h} sehr
               werthvoll sein könnte. Bitte, theilen Sie mir gleich nach Ihrer \label{K_L03294-3v}\edtext{Rückkunft}{\lemma{\textnormal{\emph{Rückkunft}}}\Cendnote{\textnormal{Schnitzler\pwindex{Schnitzler, Arthur 15.05.1862 – 21.10.1931@\textsc{Schnitzler, Arthur} (15.05.1862 – 21.10.1931), \emph{Schriftsteller, Mediziner}|pwk} kehrte am 28. 6. 1899 nach Wien\oindex{Wien@\textbf{Wien}|pwk} zurück.}}}\label{K_L03294-3h} mit, wann ich Sie sprechen
               kann.\pend
           \pstart
           Herzlichst Ihr {\\[\baselineskip]}\spacefill\mbox{Salten}\pend
           \leftskip=0em{}
         
         \endnumbering\mylabel{h}\end{ledgroupsized}  \newcommand{\dateiname}{L03294}\newcommand{\titel}{Felix Salten an Arthur Schnitzler, [27. 6. 1899]}\newcommand{\editorInnen}{Martin Anton Müller und Laura Untner}%% latex-leseansicht-abspann.tex
%% Abspann für die Leseansicht.
%% Der Schalter \ifkorrekturansicht ist bereits durch den Vorspann gesetzt.

%% latex-abspann.tex
%% Gemeinsamer Abspann für Korrekturansicht und Leseansicht.
%% Setzt den Schalter \ifkorrekturansicht voraus (gesetzt in den
%% einbindenden Dateien latex-korrekturansicht-abspann.tex bzw.
%% latex-leseansicht-abspann.tex).
%% ---------------------------------------------------------------

\normalsize

% Das esempio-Environment wird nur in der Leseansicht benötigt
\ifkorrekturansicht\else
\newenvironment{esempio}[3]%
{
    \vspace{1.5ex}
    \rlap{\underline{#1}}
    \par
    \setlength{\parindent}{0cm}
    \nopagebreak
    \leftskip=#2cm
    \rightskip=#3cm
}
{
    \par
}
\fi

\doendnotes{C}
\bigskip
\vfill

\clearpage

\footnotesize

\ifkorrekturansicht
  \lohead{\textsc{register}}
\fi

% theindex-Environment neu definieren ohne reledmac
\makeatletter
\renewenvironment{theindex}{%
  \ifkorrekturansicht
    \section*{\indexname}%
  \else
    \subsubsection*{Index der erwähnten Entitäten}%
  \fi
  \setlength{\parindent}{0pt}%
  \setlength{\parskip}{0pt plus 0.3pt}%
  \let\item\@idxitem
}{%
  \ifkorrekturansicht\clearpage\fi
}
\makeatother

\IfFileExists{\jobname-pw.ind}{\input{\jobname-pw.ind}}{}

% Quellenangabe nur in der Leseansicht
\ifkorrekturansicht\else
% Fallback-Definitionen, falls die .tex-Datei \titel etc. nicht gesetzt hat
\providecommand{\titel}{}
\providecommand{\editorInnen}{}
\providecommand{\dateiname}{\jobname}

\vspace{3cm}

\vfill

\footnotesize
\textsc{Quelle}: \titel. Herausgegeben von {\editorInnen}. In: \emph{Arthur Schnitzler: Briefwechsel mit Autorinnen und Autoren}.
 Digitale Edition, https://schnitzler-briefe.acdh.oeaw.ac.at/{\dateiname}.html (Stand \today)
\fi

\end{document}


      