%% latex-korrekturansicht-vorspann.tex
%% Vorspann für die Korrekturansicht.
%% Lädt die gemeinsame Datei latex-vorspann.tex mit gesetztem Schalter.

\newif\ifkorrekturansicht
\korrekturansichttrue

\input{../tex-inputs/latex-vorspann}


\section[ Felix Salten an Arthur Schnitzler, {[}27. 6. 1899{]}]{L03294 Felix Salten an Arthur Schnitzler, {[}27. 6. 1899{]}}
\nopagebreak\mylabel{L03294v}
\rehead{ }\normalsize\beginnumbering\briefempfaengerindex{Schnitzler, Arthur@\textsc{Schnitzler, Arthur}!zzzSalten, Felix@\emph{von Felix Salten}!1899-06-271@{{[}27. 6. 1899{]}}|(be}
\toendnotes[C]{\smallbreak\pagebreak[2]}\Standort{CUL, Schnitzler, B 89, A 2.}
\physDesc{Brief, 1 Blatt, 1 Seite, 587 Zeichen
\newline{}Handschrift: schwarze Tinte, lateinische Kurrent
\newline{}Schnitzler: mit Bleistift datiert: »27/6 99« 
\newline{}Ordnung: mit Bleistift von unbekannter Hand nummeriert: »118« }\toendnotes[C]{\smallbreak}
\pstart
           \noindent{}{\pb}Lieber Freund, wegen des »Liebesreigen\pwindex{Reigen. Zehn Dialoge@\emph{Reigen. Zehn Dialoge}|pw}« möchte ich so bald wie möglich mit Ihnen \label{K_L03294-1v}\edtext{sprechen}{\lemma{\textnormal{\emph{sprechen}}}\Cendnote{\textnormal{Er dachte an eine Veröffentlichung in der \emph{Wiener Allgemeinen Montags-Zeitung}\pwindex{Wiener Allgemeine Montags-Zeitung@\emph{Wiener Allgemeine Montags-Zeitung}|pwk}, zu der es aber nicht
                  kam (siehe Felix Salten an Arthur Schnitzler, 21. 6. 1899).}}}\label{K_L03294-1}. D\textsuperscript{r}{ }Szeps\pwindex{Szeps, Julius 1867-10-27 – 27.10.1924@\textsc{Szeps, Julius} (1867-10-27 – 27.10.1924), \emph{Journalist/Journalistin}|pw} macht im Ministerium\orgindex{Ministerium fuer Inneres@Ministerium für Inneres|pwuv} Anstrengungen denselben
               durchzusetzen, und an eventuelle Aufregung im Leserkreis kehre ich mich nicht. Ich
               könnte Ihnen ein nicht unbeträchtliches Honorar dafür bieten, und glaube, wenn es
               durch Vermittlung des Ministers\pwindex{Koerber, Ernest von 06.11.1850 – 05.03.1919@\textsc{Koerber, Ernest von} (06.11.1850 – 05.03.1919), \emph{Politiker/Politikerin}|pwv} gelingt, die Sache durch die Censur zu drücken, wäre ein wichtiges
               Präjudiz geschaffen, das Ihnen auch für eine \label{K_L03294-2v}\edtext{Buchausgabe}{\lemma{\textnormal{\emph{Buchausgabe}}}\Cendnote{\textnormal{Schnitzler
                     ließ einen Privatdruck von \emph{Reigen}\pwindex{Reigen. Zehn Dialoge@\emph{Reigen. Zehn Dialoge}|pwk} mit einer Auflage von 200 Stück erstellen, den er 
                     im 
                     April 1900 an Freunde verteilte (vgl. Hermann Bahr an Arthur Schnitzler, 20. 4. [1900]). 
                      1903 erschien das Theaterstück\pwindex{Reigen. Zehn Dialoge@\emph{Reigen. Zehn Dialoge}|pwkv} im \emph{Wiener Verlag}\orgindex{Wiener Verlag@Wiener Verlag|pwk}.}}}\label{K_L03294-2} sehr
               werthvoll sein könnte. Bitte, theilen Sie mir gleich nach Ihrer \label{K_L03294-3v}\edtext{Rückkunft}{\lemma{\textnormal{\emph{Rückkunft}}}\Cendnote{\textnormal{Schnitzler kehrte am 28. 6. 1899 nach Wien\oindex{Wien@\textbf{Wien}, \emph{A.ADM2}|pwk} zurück.}}}\label{K_L03294-3} mit, wann ich Sie sprechen
               kann.\pend
           
\pstart
           Herzlichst Ihr {\\[\baselineskip]}\spacefill\mbox{Salten}\pend
           \leftskip=0em{}\selectlanguage{ngerman}\endnumbering\briefempfaengerindex{Schnitzler, Arthur@\textsc{Schnitzler, Arthur}!zzzSalten, Felix@\emph{von Felix Salten}!1899-06-271@{{[}27. 6. 1899{]}}|)be}\mylabel{L03294h}  \normalsize

\doendnotes{C}
\bigskip
\vfill

\clearpage

\footnotesize

\lohead{\textsc{register}}

% Definiere theindex-Environment komplett neu ohne reledmac
\makeatletter
\renewenvironment{theindex}{%
  \section*{\indexname}%
  \setlength{\parindent}{0pt}%
  \setlength{\parskip}{0pt plus 0.3pt}%
  \let\item\@idxitem
}{%
  \clearpage
}
\makeatother

\IfFileExists{\jobname-pw.ind}{\input{\jobname-pw.ind}}{}

\end{document}

      