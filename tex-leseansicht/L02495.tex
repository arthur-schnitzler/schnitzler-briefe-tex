%% latex-korrekturansicht-vorspann.tex
%% Vorspann für die Korrekturansicht.
%% Lädt die gemeinsame Datei latex-vorspann.tex mit gesetztem Schalter.

\newif\ifkorrekturansicht
\korrekturansichttrue

\input{../tex-inputs/latex-vorspann}


\section[Thomas Mann an Arthur Schnitzler, 25. 12. 1927]{L02495 Thomas Mann an Arthur Schnitzler, 25. 12. 1927}
\nopagebreak\mylabel{L02495v}
\rehead{ }\normalsize\beginnumbering\briefempfaengerindex{Schnitzler, Arthur@\textsc{Schnitzler, Arthur}!zzzMann, Thomas@\emph{von Thomas Mann}!1927-12-252@{25. 12. 1927}|(be}
\toendnotes[C]{\smallbreak\pagebreak[2]}\Standort{CUL, Schnitzler, B 67.}
\physDesc{Postkarte, 422 Zeichen
\newline{}Handschrift: schwarze Tinte, deutsche Kurrent
\newline{}Versand: Stempel: »\nobreak{}\oindex{Muenchen@\textbf{München}, \emph{P.PPLA}|pwk}München, 25. 12. 1927, 9–10\textcolor{gray}{N}\nobreak{}«.  
\newline{}Schnitzler: 1) mit rotem Buntstift beschrieben mit »\textsc{Aph}{[}orismen{]}«  2) mit rotem Buntstift zwei Unterstreichungen}
\buchAbdrucke{\weitereDrucke{\emph{Modern Austrian Literature}, Jg. 7 (1974) Nr. 1/2, S. 25.} }\toendnotes[C]{\smallbreak}\pstart{}{\pb}Herrn\pend{}\pstart{}Dr. Arthur \textsc{Schnitzler}\pend{}\pstart{}Wien XVIII\oindex{XVIII., Waehring@\textbf{XVIII., Währing}, \emph{A.ADM3}|pw}\pend{}\pstart{}Sternwarteſtr. 78\oindex{Sternwartestrasse 71@\textbf{Sternwartestraße 71}, \emph{Wohngebäude (K.WHS)}|pw}.\pend{}{\bigskip}\vspace{1em}
\pstart
           {\pb}\textcolor{gray}{\textbf{\textsc{Dr. Thomas Mann}}}\hfill \textcolor{gray}{\textbf{MÜNCHEN 27\oindex{Muenchen@\textbf{München}, \emph{P.PPLA}|pw}, den}}{ }25. XII. 27.\pend
           
\pstart
           \raggedleft{}\textcolor{gray}{\textbf{POSCHINGERSTR. 1\oindex{Poschingerstrasse@\textbf{Poschingerstraße}, \emph{Straße (K.STR)}|pw}}}\pend
           
\pstart{}Lieber, verehrter Arthur Schnitzler,\pend\vspace{0.5em}
\pstart
           von Herzen Dank für das Weihnachtsgeſchenk Ihres Spruch-Buches\pwindex{Buch der Sprueche und Bedenken@\emph{Buch der Sprüche und Bedenken}|pwv}, das ſo voll iſt von ſchön und klar geformter
               Weisheit! Sie ſind ganz darin mit Ihrer Unbeſtechlichkeit, Freiheit \damage{\textcolor{gray}{und}} Güte, und nicht nur im Einzelnen, ſondern als Ganzes iſt es
               liebenswert.\pend
           
\pstart
           Ein glückliches neues Jahr wünſcht Ihnen{\\[\baselineskip]}Ihr treu ergebener{\\[\baselineskip]}\spacefill\mbox{Thomas Mann.}\pend
           \leftskip=0em{}\selectlanguage{ngerman}\endnumbering\briefempfaengerindex{Schnitzler, Arthur@\textsc{Schnitzler, Arthur}!zzzMann, Thomas@\emph{von Thomas Mann}!1927-12-252@{25. 12. 1927}|)be}\mylabel{L02495h}  \normalsize

\doendnotes{C}
\bigskip
\vfill

\clearpage

\footnotesize

\lohead{\textsc{register}}

% Definiere theindex-Environment komplett neu ohne reledmac
\makeatletter
\renewenvironment{theindex}{%
  \section*{\indexname}%
  \setlength{\parindent}{0pt}%
  \setlength{\parskip}{0pt plus 0.3pt}%
  \let\item\@idxitem
}{%
  \clearpage
}
\makeatother

\IfFileExists{\jobname-pw.ind}{\input{\jobname-pw.ind}}{}

\end{document}

      