%% latex-leseansicht-vorspann.tex
%% Vorspann für die Leseansicht.
%% Lädt die gemeinsame Datei latex-vorspann.tex mit nicht gesetztem Schalter.

\newif\ifkorrekturansicht
\korrekturansichtfalse

\input{../tex-inputs/latex-vorspann}


\section[Arthur Schnitzler an Hermann Bahr, 23. 1. 1905]{L01493 Arthur Schnitzler an Hermann Bahr, 23. 1. 1905}
\nopagebreak\mylabel{L01493v}
\rehead{ }\normalsize\beginnumbering\briefempfaengerindex{Bahr, Hermann@\textsc{Bahr, Hermann}!zzzSchnitzler, Arthur@\emph{von Arthur Schnitzler}!1905-01-231@{23. 1. 1905}|(be}
\toendnotes[C]{\smallbreak\pagebreak[2]}
\correspDesc{Versand  durch Arthur Schnitzler am 23. 1. 1905 in Wien
\newline{}Erhalt  durch Hermann Bahr im Zeitraum [23. 1. 1905
                  – 27. 1. 1905?] in Wien}\toendnotes[C]{\smallbreak}
\Standort{TMW, HS AM 23371 Ba.}
\physDesc{Brief, 1 Blatt, 1 Seite, 298 Zeichen
\newline{}Handschrift: schwarze Tinte, deutsche Kurrent
\newline{}Ordnung: Lochung }
\buchAbdrucke{\weitereDrucke{1) \emph{23. 1. 1905.} In: Arthur Schnitzler: \emph{The Letters of Arthur Schnitzler to Hermann Bahr}. Edited, annotated, and with an introduction, by Donald G. Daviau. Chapel Hill: \emph{The University of North Carolina Press} 1978, S. 88 (University of North Carolina studies in the Germanic languages
                        and literatures, 89).} \weitereDrucke{2) Hermann Bahr, Arthur Schnitzler: \emph{Briefwechsel, Aufzeichnungen, Dokumente (1891–1931)}. Herausgegeben von Kurt Ifkovits und Martin Anton Müller. Göttingen: \emph{Wallstein} 2018, S. 339.} }\toendnotes[C]{\smallbreak}
\pstart
           \raggedleft{}{\pb}\textsc{Wien XVIII.\oindex{XVIII., Währing@\textbf{XVIII., Währing}, \emph{Verwaltungsgebiet}|pw}}{\\}am \introOben{}23/1 905\introOben{}\pend
           \vspace{0.5em}
\pstart
           leider, mein lieber Hermann, hab ich gar nichts rechtes luſtiges,
               kurzes, ungedrucktes, zur Lectüre geeignetes – aber{ }ſehen möcht ich dich je eher je
               lieber. Hoffentlich nächſte Woche. Und So{\geminationn}tag hörſt du
               dir wohl auch die \label{K_L01493-1v}\edtext{Mahler\pwindex{Mahler, Gustav 7.\,7.\,1860 Kaliště – 18.\,5.\,1911 Wien@\textsc{Mahler, Gustav} (7.\,7.\,1860 Kaliště – 18.\,5.\,1911 Wien), \emph{Theaterleiter, Komponist, Dirigent}|pw} Lieder}{\lemma{\textnormal{\emph{Mahler Lieder}}}\Cendnote{\textnormal{\emph{Des Knaben Wunderhorn}\pwindex{Mahler, Gustav 7.\,7.\,1860 Kaliště – 18.\,5.\,1911 Wien@\textsc{Mahler, Gustav} (7.\,7.\,1860 Kaliště – 18.\,5.\,1911 Wien), \emph{Theaterleiter, Komponist, Dirigent}!Des Knaben Wunderhorn@\strich\emph{Des Knaben Wunderhorn}|pwk} am
                     29. 1. 1905 im Bösendorfer-Saal\oindex{Wien@\textbf{Wien}!I., Innere Stadt@\textbf{I., Innere Stadt}!Bösendorfer-Saal@\textbf{Bösendorfer-Saal}, \emph{Veranstaltungsgebäude}|pwk}.}}}\label{K_L01493-1} an? Wir\pwindex{Schnitzler, Olga 17.\,1.\,1882 Wien – 13.\,1.\,1970 Lugano@\textsc{Schnitzler, Olga} (17.\,1.\,1882 Wien – 13.\,1.\,1970 Lugano), \emph{Schauspielerin, Sängerin}|pwv} grüßen dich beide.\pend
           
\pstart
           Herzlichſt dein{\\[\baselineskip]}\spacefill\mbox{Arthur}\pend
           \leftskip=0em{}\selectlanguage{ngerman}\endnumbering\briefempfaengerindex{Bahr, Hermann@\textsc{Bahr, Hermann}!zzzSchnitzler, Arthur@\emph{von Arthur Schnitzler}!1905-01-231@{23. 1. 1905}|)be}\mylabel{L01493h}  \newcommand{\dateiname}{L01493}\newcommand{\titel}{Arthur Schnitzler an Hermann Bahr, 23. 1. 1905}\newcommand{\editorInnen}{Herausgegeben von Martin Anton Müller}%% latex-leseansicht-abspann.tex
%% Abspann für die Leseansicht.
%% Der Schalter \ifkorrekturansicht ist bereits durch den Vorspann gesetzt.

%% latex-abspann.tex
%% Gemeinsamer Abspann für Korrekturansicht und Leseansicht.
%% Setzt den Schalter \ifkorrekturansicht voraus (gesetzt in den
%% einbindenden Dateien latex-korrekturansicht-abspann.tex bzw.
%% latex-leseansicht-abspann.tex).
%% ---------------------------------------------------------------

\normalsize

% Das esempio-Environment wird nur in der Leseansicht benötigt
\ifkorrekturansicht\else
\newenvironment{esempio}[3]%
{
    \vspace{1.5ex}
    \rlap{\underline{#1}}
    \par
    \setlength{\parindent}{0cm}
    \nopagebreak
    \leftskip=#2cm
    \rightskip=#3cm
}
{
    \par
}
\fi

\doendnotes{C}
\bigskip
\vfill

\clearpage

\footnotesize

\ifkorrekturansicht
  \lohead{\textsc{register}}
\fi

% theindex-Environment neu definieren ohne reledmac
\makeatletter
\renewenvironment{theindex}{%
  \ifkorrekturansicht
    \section*{\indexname}%
  \else
    \subsubsection*{Index der erwähnten Entitäten}%
  \fi
  \setlength{\parindent}{0pt}%
  \setlength{\parskip}{0pt plus 0.3pt}%
  \let\item\@idxitem
}{%
  \ifkorrekturansicht\clearpage\fi
}
\makeatother

\IfFileExists{\jobname-pw.ind}{\input{\jobname-pw.ind}}{}

% Quellenangabe nur in der Leseansicht
\ifkorrekturansicht\else
% Fallback-Definitionen, falls die .tex-Datei \titel etc. nicht gesetzt hat
\providecommand{\titel}{}
\providecommand{\editorInnen}{}
\providecommand{\dateiname}{\jobname}

\vspace{3cm}

\vfill

\footnotesize
\textsc{Quelle}: \titel. Herausgegeben von {\editorInnen}. In: \emph{Arthur Schnitzler: Briefwechsel mit Autorinnen und Autoren}.
 Digitale Edition, https://schnitzler-briefe.acdh.oeaw.ac.at/{\dateiname}.html (Stand \today)
\fi

\end{document}


