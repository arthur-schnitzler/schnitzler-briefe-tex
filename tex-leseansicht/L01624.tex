%% latex-korrekturansicht-vorspann.tex
%% Vorspann für die Korrekturansicht.
%% Lädt die gemeinsame Datei latex-vorspann.tex mit gesetztem Schalter.

\newif\ifkorrekturansicht
\korrekturansichttrue

\input{../tex-inputs/latex-vorspann}


\section[Hugo von Hofmannsthal an Arthur Schnitzler, 4. {[}9. 1906{]}]{L01624 Hugo von Hofmannsthal an Arthur Schnitzler, 4. {[}9. 1906{]}}
\nopagebreak\mylabel{L01624v}
\rehead{ }\normalsize\beginnumbering\briefempfaengerindex{Schnitzler, Arthur@\textsc{Schnitzler, Arthur}!zzzHofmannsthal, Hugo von@\emph{von Hugo von Hofmannsthal}!1906-09-041@{4. {[}9. 1906{]}}|(be}
\toendnotes[C]{\smallbreak\pagebreak[2]}\Standort{CUL, Schnitzler, B 43.}
\physDesc{Brief, 1 Blatt, 4 Seiten, 1245 Zeichen
\newline{}Handschrift: schwarze Tinte, deutsche Kurrent
\newline{}Schnitzler: mit Bleistift zum Datum eine mutmaßliche Monatsangabe ergänzt: »7(?).« 
\newline{}Ordnung: 1) mit Bleistift von unbekannter Hand nummeriert: »\strikeout{214}«  2) mit Bleistift von unbekannter Hand nummeriert:
                                    »197«}
\buchAbdrucke{\weitereDrucke{Hugo von Hofmannsthal, Arthur Schnitzler: \emph{Briefwechsel}. Frankfurt am Main: \emph{S. Fischer} 1964, S. 220.} }\toendnotes[C]{\smallbreak}
\pstart
           \raggedleft{}{\pb}\textsc{Lueg}\oindex{Lueg@\textbf{Lueg}, \emph{Teil eines besiedelten Ortes (A.BSOX)}|pw}{ }4\textsuperscript{ten}\pend
           
\pstart{}mein lieber Arthur \pend\vspace{0.5em}
\pstart
           ich habe rechtes Verlangen, von Ihnen ein bischen ausführlicher zu hören.\hspace*{1.5em}Von mir (und Gerty\pwindex{Hofmannsthal, Gertrude von 16.03.1880 – 09.11.1959@\textsc{Hofmannsthal, Gertrude von} (16.03.1880 – 09.11.1959)|pw}) kann ich, was Stimmung, Laune, Genießen des Sommers betrifft, nur
               Gutes berichten, von einer größeren Arbeit iſt freilich noch nichts zu ſagen,
               manchmal {\pb}ſcheint dergleichen
               recht nahe, dann iſt es wieder, als ob es untertauchte und ſich verbärge, aber nicht
               in Waſſer, ſondern in einer viel härteren undurchſichtigen Subſtanz, doch halte ich
               gar nicht für unmöglich, daſs der Herbſt, der mir oft günſtig war, auch diesmal
               plötzlich und ſpringquellhaft wieder etwas {\pb}hervortreibt – das Gefühl der
               Armut hatte ich jedesfalls nicht, vieles größere und kleinere mehr Gedankenhafte hat
               ſich geordnet, aufgeſchrieben hab ich auch gar nicht weniges und eine gewiſſe
               Möglichkeit, epiſches (kürzeres zunächſt) in mir auszubilden fühle ich auch, mehr als
               ein Vorgefühl {\pb}allerdings. Unſeres
               letzten Zuſa{\geminationm}enſeins, des Spaziergangs bei drohenden
               Wolken und des ſchönen leichten und inhaltsvollen Redens denke ich auch – auf ein
               paar Tage Semmering\oindex{Semmering@\textbf{Semmering}, \emph{A.ADM3}|pw} (vielleicht mit Brahm\pwindex{Brahm, Otto 05.02.1856 – 28.11.1912@\textsc{Brahm, Otto} (05.02.1856 – 28.11.1912), \emph{Theaterleiter/Theaterleiterin, Regisseur/Regisseurin}|pw}) möchte ich jedenfalls rechnen.\pend
           
\pstart
           Ich weiß nicht, (da es ſo wunderſchön iſt) ob ich nicht noch 10–14 Tage hier bleibe,
               die Kinder\pwindex{Zimmer, Christiane 14.05.1902 – 05.01.1987@\textsc{Zimmer, Christiane} (14.05.1902 – 05.01.1987)|pwv}\pwindex{Hofmannsthal, Raimund von 26.5.1906 – 20.03.1974@\textsc{Hofmannsthal, Raimund von} (26.5.1906 – 20.03.1974)|pwv}\pwindex{Hofmannsthal, Franz von 20.10.1903 – 13.07.1929@\textsc{Hofmannsthal, Franz von} (20.10.1903 – 13.07.1929)|pwv}{ }ſind ſchon in Rodaun\oindex{Rodaun@\textbf{Rodaun}, \emph{A.ADM4}|pw}.\pend
           
\pstart
            Schreiben Sie.\hspace*{1.5em}Von Herzen\pend
           \pstart \spacefill\mbox{Hugo.}\pend{}\selectlanguage{ngerman}\endnumbering\briefempfaengerindex{Schnitzler, Arthur@\textsc{Schnitzler, Arthur}!zzzHofmannsthal, Hugo von@\emph{von Hugo von Hofmannsthal}!1906-09-041@{4. {[}9. 1906{]}}|)be}\mylabel{L01624h}  \normalsize

\doendnotes{C}
\bigskip
\vfill

\clearpage

\footnotesize

\lohead{\textsc{register}}

% Definiere theindex-Environment komplett neu ohne reledmac
\makeatletter
\renewenvironment{theindex}{%
  \section*{\indexname}%
  \setlength{\parindent}{0pt}%
  \setlength{\parskip}{0pt plus 0.3pt}%
  \let\item\@idxitem
}{%
  \clearpage
}
\makeatother

\IfFileExists{\jobname-pw.ind}{\input{\jobname-pw.ind}}{}

\end{document}

      