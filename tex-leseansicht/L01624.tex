%% latex-leseansicht-vorspann.tex
%% Vorspann für die Leseansicht.
%% Lädt die gemeinsame Datei latex-vorspann.tex mit nicht gesetztem Schalter.

\newif\ifkorrekturansicht
\korrekturansichtfalse

\input{../tex-inputs/latex-vorspann}


\section[Hugo von Hofmannsthal an Arthur Schnitzler, 4. {[}9. 1906{]}]{L01624 Hugo von Hofmannsthal an Arthur Schnitzler, 4. [9. 1906]}
\nopagebreak\mylabel{L01624v}
\rehead{ }\normalsize\beginnumbering\briefempfaengerindex{Schnitzler, Arthur@\textsc{Schnitzler, Arthur}!zzzHofmannsthal, Hugo von@\emph{von Hugo von Hofmannsthal}!1906-09-041@{4. [9. 1906]}|(be}
\toendnotes[C]{\smallbreak\pagebreak[2]}
\correspDesc{Versand  durch Hugo von Hofmannsthal am 4. [9. 1906] in St. Gilgen
\newline{}Erhalt  durch Arthur Schnitzler im Zeitraum [5. 9. 1906
                  – 9. 9. 1906?] in Wien}\toendnotes[C]{\smallbreak}
\Standort{CUL, Schnitzler, B 43.}
\physDesc{Brief, 1 Blatt, 4 Seiten, 1245 Zeichen
\newline{}Handschrift: schwarze Tinte, deutsche Kurrent
\newline{}Schnitzler: mit Bleistift zum Datum eine mutmaßliche Monatsangabe ergänzt: »7(?).« 
\newline{}Ordnung: 1) mit Bleistift von unbekannter Hand nummeriert: »\strikeout{214}«  2) mit Bleistift von unbekannter Hand nummeriert:
                                    »197«}
\buchAbdrucke{\weitereDrucke{Hugo von Hofmannsthal, Arthur Schnitzler: \emph{Briefwechsel}. Herausgegeben von Therese Nickl und Heinrich Schnitzler. Frankfurt am Main: \emph{S. Fischer} 1964, S. 220.} }\toendnotes[C]{\smallbreak}
\pstart
           \raggedleft{}{\pb}\textsc{Lueg}\oindex{Lueg@\textbf{Lueg}, \emph{Teil eines besiedelten Ortes}|pw}{ }4\textsuperscript{ten}\pend
           
\pstart{}mein lieber Arthur\pend\vspace{0.5em}
\pstart
           ich habe rechtes Verlangen, von Ihnen ein bischen ausführlicher zu hören.\hspace*{1.5em}Von mir (und Gerty\pwindex{Hofmannsthal, Gertrude von 16.\,3.\,1880 Wien – 9.\,11.\,1959 Paddington@\textsc{Hofmannsthal, Gertrude von} (16.\,3.\,1880 Wien – 9.\,11.\,1959 Paddington)|pw}) kann ich, was Stimmung, Laune, Genießen des Sommers betrifft, nur
               Gutes berichten, von einer größeren Arbeit iſt freilich noch nichts zu{ }ſagen,
               manchmal {\pb}ſcheint dergleichen
               recht nahe, dann iſt es wieder, als ob es untertauchte und{ }ſich verbärge, aber nicht
               in Waſſer,{ }ſondern in einer viel härteren undurchſichtigen Subſtanz, doch halte ich
               gar nicht für unmöglich, daſs der Herbſt, der mir oft günſtig war, auch diesmal
               plötzlich und{ }ſpringquellhaft wieder etwas {\pb}hervortreibt – das Gefühl der
               Armut hatte ich jedesfalls nicht, vieles größere und kleinere mehr Gedankenhafte hat{ }ſich geordnet, aufgeſchrieben hab ich auch gar nicht weniges und eine gewiſſe
               Möglichkeit, epiſches (kürzeres zunächſt) in mir auszubilden fühle ich auch, mehr als
               ein Vorgefühl {\pb}allerdings. Unſeres
               letzten Zuſa{\geminationm}enſeins, des Spaziergangs bei drohenden
               Wolken und des{ }ſchönen leichten und inhaltsvollen Redens denke ich auch – auf ein
               paar Tage Semmering\oindex{Semmering@\textbf{Semmering}, \emph{Verwaltungsgebiet}|pw} (vielleicht mit Brahm\pwindex{Brahm, Otto 5.\,2.\,1856 Hamburg – 28.\,11.\,1912 Berlin@\textsc{Brahm, Otto} (5.\,2.\,1856 Hamburg – 28.\,11.\,1912 Berlin), \emph{Theaterleiter, Regisseur}|pw}) möchte ich jedenfalls rechnen.\pend
           
\pstart
           Ich weiß nicht, (da es{ }ſo wunderſchön iſt) ob ich nicht noch 10–14 Tage hier bleibe,
               die Kinder\pwindex{Zimmer, Christiane 14.\,5.\,1902 Rodaun – 5.\,1.\,1987 New York City@\textsc{Zimmer, Christiane} (14.\,5.\,1902 Rodaun – 5.\,1.\,1987 New York City)|pwv}\pwindex{Hofmannsthal, Raimund von 26.\,5.\,1906 Rodaun – 20.\,3.\,1974 London@\textsc{Hofmannsthal, Raimund von} (26.\,5.\,1906 Rodaun – 20.\,3.\,1974 London)|pwv}\pwindex{Hofmannsthal, Franz von 20.\,10.\,1903 Wien – 13.\,7.\,1929 ebd.@\textsc{Hofmannsthal, Franz von} (20.\,10.\,1903 Wien – 13.\,7.\,1929 ebd.)|pwv}{ }ſind{ }ſchon in Rodaun\oindex{Wien@\textbf{Wien}!XXIII., Liesing@\textbf{XXIII., Liesing}!Rodaun@\textbf{Rodaun}, \emph{Region}|pw}.\pend
           
\pstart
           Schreiben Sie.\hspace*{1.5em}Von Herzen\pend
           \pstart \spacefill\mbox{Hugo.}\pend{}\selectlanguage{ngerman}\endnumbering\briefempfaengerindex{Schnitzler, Arthur@\textsc{Schnitzler, Arthur}!zzzHofmannsthal, Hugo von@\emph{von Hugo von Hofmannsthal}!1906-09-041@{4. [9. 1906]}|)be}\mylabel{L01624h}  \newcommand{\dateiname}{L01624}\newcommand{\titel}{Hugo von Hofmannsthal an Arthur Schnitzler, 4. [9. 1906]}\newcommand{\editorInnen}{Martin Anton Müller und Gerd-Hermann Susen}%% latex-leseansicht-abspann.tex
%% Abspann für die Leseansicht.
%% Der Schalter \ifkorrekturansicht ist bereits durch den Vorspann gesetzt.

%% latex-abspann.tex
%% Gemeinsamer Abspann für Korrekturansicht und Leseansicht.
%% Setzt den Schalter \ifkorrekturansicht voraus (gesetzt in den
%% einbindenden Dateien latex-korrekturansicht-abspann.tex bzw.
%% latex-leseansicht-abspann.tex).
%% ---------------------------------------------------------------

\normalsize

% Das esempio-Environment wird nur in der Leseansicht benötigt
\ifkorrekturansicht\else
\newenvironment{esempio}[3]%
{
    \vspace{1.5ex}
    \rlap{\underline{#1}}
    \par
    \setlength{\parindent}{0cm}
    \nopagebreak
    \leftskip=#2cm
    \rightskip=#3cm
}
{
    \par
}
\fi

\doendnotes{C}
\bigskip
\vfill

\clearpage

\footnotesize

\ifkorrekturansicht
  \lohead{\textsc{register}}
\fi

% theindex-Environment neu definieren ohne reledmac
\makeatletter
\renewenvironment{theindex}{%
  \ifkorrekturansicht
    \section*{\indexname}%
  \else
    \subsubsection*{Index der erwähnten Entitäten}%
  \fi
  \setlength{\parindent}{0pt}%
  \setlength{\parskip}{0pt plus 0.3pt}%
  \let\item\@idxitem
}{%
  \ifkorrekturansicht\clearpage\fi
}
\makeatother

\IfFileExists{\jobname-pw.ind}{\input{\jobname-pw.ind}}{}

% Quellenangabe nur in der Leseansicht
\ifkorrekturansicht\else
% Fallback-Definitionen, falls die .tex-Datei \titel etc. nicht gesetzt hat
\providecommand{\titel}{}
\providecommand{\editorInnen}{}
\providecommand{\dateiname}{\jobname}

\vspace{3cm}

\vfill

\footnotesize
\textsc{Quelle}: \titel. Herausgegeben von {\editorInnen}. In: \emph{Arthur Schnitzler: Briefwechsel mit Autorinnen und Autoren}.
 Digitale Edition, https://schnitzler-briefe.acdh.oeaw.ac.at/{\dateiname}.html (Stand \today)
\fi

\end{document}


