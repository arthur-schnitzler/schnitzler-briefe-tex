%% latex-korrekturansicht-vorspann.tex
%% Vorspann für die Korrekturansicht.
%% Lädt die gemeinsame Datei latex-vorspann.tex mit gesetztem Schalter.

\newif\ifkorrekturansicht
\korrekturansichttrue

\input{../tex-inputs/latex-vorspann}


\section[Arthur Schnitzler an Richard Beer-Hofmann, 17. 7. 1905]{L01531 Arthur Schnitzler an Richard Beer-Hofmann, 17. 7. 1905}
\nopagebreak\mylabel{L01531v}
\rehead{ }\normalsize\beginnumbering\briefempfaengerindex{Beer-Hofmann, Richard@\textsc{Beer-Hofmann, Richard}!zzzSchnitzler, Arthur@\emph{von Arthur Schnitzler}!1905-07-171@{17. 7. 1905}|(be}
\toendnotes[C]{\smallbreak\pagebreak[2]}\Standort{YCGL, MSS 31.}
\physDesc{Kartenbrief, 400 Zeichen
\newline{}Handschrift: schwarze Tinte, deutsche Kurrent
\newline{}Versand: Stempel: »\nobreak{}\oindex{Reichenau an der Rax@\textbf{Reichenau an der Rax}, \emph{A.ADM3}|pwk}Reichenau bei Payerbach, 17 \textcolor{gray}{7} 0\textcolor{gray}{5}, 2–6N\nobreak{}«.  }\pstart{}{\pb}Herrn \textsc{Dr. Richard}\pend{}\pstart{}\textsc{Beer-Hofmann}\pend{}\pstart{}\textsc{Rodaun{\\}bei Liesing\oindex{Rodaun@\textbf{Rodaun}, \emph{A.ADM4}|pw}}\pend{}\pstart{}\textsc{Liesingerstraße 2}\oindex{Liesingerstrasse@\textbf{Liesingerstraße}, \emph{Straße (K.STR)}|pw}\pend{}{\bigskip}\vspace{1em}
\pstart
           \raggedleft{}{\pb}\textsc{Reichenau, Kurhaus}\oindex{Kurhaus Rudolfsbad@\textbf{Kurhaus Rudolfsbad}, \emph{Sanatorium (K.SAN)}|pw}{\\}Am 17. 7. 905\pend
           \vspace{0.5em}
\pstart
           lieber Richard, bitte theilen Sie uns mit, wie es Paula\pwindex{Beer-Hofmann, Paula 25.02.1879 – 30.10.1939@\textsc{Beer-Hofmann, Paula} (25.02.1879 – 30.10.1939)|pw} geht. Was iſt de{\geminationn}
               eigentlich geſchehn? Und wie ſtehen Sie jetzt mit Ihren vielen So{\geminationm}erplänen?\pend
           
\pstart
           Vielleicht ko{\geminationm}t man doch irgendwa{\geminationn} u irgendwo zuſammen? Wir bleiben noch 2–3 Wochen da,
               kleinere (2–3 tägige) Fußpartien meinerſeits abgerechnet\pend
           
\pstart
           Herzlichſt Ihr{\\}\spacefill\mbox{Arthur}\pend
           \selectlanguage{ngerman}\endnumbering\briefempfaengerindex{Beer-Hofmann, Richard@\textsc{Beer-Hofmann, Richard}!zzzSchnitzler, Arthur@\emph{von Arthur Schnitzler}!1905-07-171@{17. 7. 1905}|)be}\mylabel{L01531h}  \normalsize

\doendnotes{C}
\bigskip
\vfill

\clearpage

\footnotesize

\lohead{\textsc{register}}

% Definiere theindex-Environment komplett neu ohne reledmac
\makeatletter
\renewenvironment{theindex}{%
  \section*{\indexname}%
  \setlength{\parindent}{0pt}%
  \setlength{\parskip}{0pt plus 0.3pt}%
  \let\item\@idxitem
}{%
  \clearpage
}
\makeatother

\IfFileExists{\jobname-pw.ind}{\input{\jobname-pw.ind}}{}

\end{document}

      