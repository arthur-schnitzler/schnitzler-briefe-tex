%% latex-leseansicht-vorspann.tex
%% Vorspann für die Leseansicht.
%% Lädt die gemeinsame Datei latex-vorspann.tex mit nicht gesetztem Schalter.

\newif\ifkorrekturansicht
\korrekturansichtfalse

\input{../tex-inputs/latex-vorspann}

\begin{center}
            \textcolor{red}{ENTWURF, NICHT FERTIG KORRIGIERT}
                      \end{center}
            
         \renewcommand{\erwaehnteInstitutionen}{Institutionen: Georg Müller Verlag, Wiener Verlag}
         \renewcommand{\erwaehnteOrte}{Orte: Wien}
         \renewcommand{\erwaehnteWerke}{Werke: Der Schrei der Liebe. Novelle, Die Gedenktafel der Prinzessin Anna. Der Schrei der Liebe. Zwei Novellen}
               \section[ Felix Salten an Arthur Schnitzler, 11. 4. 1928]{ Felix Salten an Arthur Schnitzler, 11. 4. 1928}\nopagebreak\mylabel{v}\rehead{ }\begin{ledgroupsized}[t]{13cm}\normalsize\beginnumbering \toendnotes[C]{\smallbreak\pagebreak[2]} \Standort{CUL, Schnitzler, B 89, B 2.}
\physDesc{Karte, 467 Zeichen
\newline{}Handschrift: Bleistift, lateinische Kurrent
\newline{}Schnitzler: mit rotem Buntstift zwei Unterstreichungen 
\newline{}Ordnung: mit Bleistift von unbekannter Hand nummeriert: »299« }\toendnotes[C]{\smallbreak}\pstart
           \raggedleft{}{\pb}11. IV. 28\pend
           \pstart{}Lieber,\pend\pstart
           vielleicht haben Sie noch \label{K_L03586-1v}\edtext{mein Buch
                  »Schrei der Liebe\pwindex{Salten, Felix 06.09.1869 – 08.10.1945@\textsc{Salten, Felix} (06.09.1869 – 08.10.1945), \emph{Schriftsteller, Journalist}!Schrei der Liebe. Novelle1904-10-22@\strich\emph{Der Schrei der Liebe. Novelle} {[}1904-10-22{]}|pw}}{\lemma{\textnormal{\emph{mein … Liebe}}}\Cendnote{\textnormal{erschienen im \emph{Wiener Verlag}\orgindex{Wiener Verlag@Wiener Verlag|pwk} im Oktober 1904,
                     siehe Felix Salten: Widmungsexemplar Der Schrei der Liebe für Arthur
               Schnitzler, 22. 10. 1904}}}\label{K_L03586-1h}«; ich gab es Ihnen damals, als es erschien. Auch die \label{K_L03586-2v}\edtext{zweite Ausgabe bei Georg Müller\orgindex{Georg Mueller Verlag@Georg Müller Verlag|pw}}{\lemma{\textnormal{\emph{zweite … Müller}}}\Cendnote{\textnormal{erschienen 1913}}}\label{K_L03586-2h}, zusammen mit der »Gedenktafel\pwindex{Salten, Felix 06.09.1869 – 08.10.1945@\textsc{Salten, Felix} (06.09.1869 – 08.10.1945), \emph{Schriftsteller, Journalist}!Gedenktafel der Prinzessin Anna. Der Schrei der Liebe. Zwei
                  Novellen1913@\strich\emph{Die Gedenktafel der Prinzessin Anna. Der Schrei der Liebe. Zwei Novellen} {[}1913{]}|pw}«, hab’
               ich Ihnen \label{K_L03586-3v}\edtext{dediziert}{\lemma{\textnormal{\emph{dediziert}}}\Cendnote{\textnormal{Widmungsexemplar nicht erhalten}}}\label{K_L03586-3h}.
               Jetzt ist das kleine Buch\pwindex{Salten, Felix 06.09.1869 – 08.10.1945@\textsc{Salten, Felix} (06.09.1869 – 08.10.1945), \emph{Schriftsteller, Journalist}!Schrei der Liebe. Novelle1904-10-22@\strich\emph{Der Schrei der Liebe. Novelle} {[}1904-10-22{]}|pwv}
               total vergriffen, ich \label{K_L03586-4v}\edtext{brauche es
                  dringend}{\lemma{\textnormal{\emph{brauche es
                  dringend}}}\Cendnote{\textnormal{Salten\pwindex{Salten, Felix 06.09.1869 – 08.10.1945@\textsc{Salten, Felix} (06.09.1869 – 08.10.1945), \emph{Schriftsteller, Journalist}|pwk} benötigte das Buch\pwindex{Salten, Felix 06.09.1869 – 08.10.1945@\textsc{Salten, Felix} (06.09.1869 – 08.10.1945), \emph{Schriftsteller, Journalist}!Schrei der Liebe. Novelle1904-10-22@\strich\emph{Der Schrei der Liebe. Novelle} {[}1904-10-22{]}|pwkv} als Grundlage für eine Neuausgabe,
                     vgl. Felix Salten: Widmungsexemplar Der Schrei der Liebe für Arthur
               Schnitzler, Juli 1928.}}}\label{K_L03586-4h} und kann
               es nirgendwo kriegen. Wenn Sie es noch haben und so gut sein wollen, es mir für zwei
               Wochen zu leihen, wäre sich sehr dankbar. Sie bekommen es unversehrt zurück.\pend
           \pstart
           Herzlichst {\\[\baselineskip]}Ihr {\\[\baselineskip]}\spacefill\mbox{Felix Salten}\pend
           \leftskip=0em{}
         
         \endnumbering\mylabel{h}\end{ledgroupsized}  \newcommand{\dateiname}{L03586}\newcommand{\titel}{Felix Salten an Arthur Schnitzler, 11. 4. 1928}\newcommand{\editorInnen}{Martin Anton Müller und Laura Untner}%% latex-leseansicht-abspann.tex
%% Abspann für die Leseansicht.
%% Der Schalter \ifkorrekturansicht ist bereits durch den Vorspann gesetzt.

%% latex-abspann.tex
%% Gemeinsamer Abspann für Korrekturansicht und Leseansicht.
%% Setzt den Schalter \ifkorrekturansicht voraus (gesetzt in den
%% einbindenden Dateien latex-korrekturansicht-abspann.tex bzw.
%% latex-leseansicht-abspann.tex).
%% ---------------------------------------------------------------

\normalsize

% Das esempio-Environment wird nur in der Leseansicht benötigt
\ifkorrekturansicht\else
\newenvironment{esempio}[3]%
{
    \vspace{1.5ex}
    \rlap{\underline{#1}}
    \par
    \setlength{\parindent}{0cm}
    \nopagebreak
    \leftskip=#2cm
    \rightskip=#3cm
}
{
    \par
}
\fi

\doendnotes{C}
\bigskip
\vfill

\clearpage

\footnotesize

\ifkorrekturansicht
  \lohead{\textsc{register}}
\fi

% theindex-Environment neu definieren ohne reledmac
\makeatletter
\renewenvironment{theindex}{%
  \ifkorrekturansicht
    \section*{\indexname}%
  \else
    \subsubsection*{Index der erwähnten Entitäten}%
  \fi
  \setlength{\parindent}{0pt}%
  \setlength{\parskip}{0pt plus 0.3pt}%
  \let\item\@idxitem
}{%
  \ifkorrekturansicht\clearpage\fi
}
\makeatother

\IfFileExists{\jobname-pw.ind}{\input{\jobname-pw.ind}}{}

% Quellenangabe nur in der Leseansicht
\ifkorrekturansicht\else
% Fallback-Definitionen, falls die .tex-Datei \titel etc. nicht gesetzt hat
\providecommand{\titel}{}
\providecommand{\editorInnen}{}
\providecommand{\dateiname}{\jobname}

\vspace{3cm}

\vfill

\footnotesize
\textsc{Quelle}: \titel. Herausgegeben von {\editorInnen}. In: \emph{Arthur Schnitzler: Briefwechsel mit Autorinnen und Autoren}.
 Digitale Edition, https://schnitzler-briefe.acdh.oeaw.ac.at/{\dateiname}.html (Stand \today)
\fi

\end{document}


      