%% latex-leseansicht-vorspann.tex
%% Vorspann für die Leseansicht.
%% Lädt die gemeinsame Datei latex-vorspann.tex mit nicht gesetztem Schalter.

\newif\ifkorrekturansicht
\korrekturansichtfalse

\input{../tex-inputs/latex-vorspann}


\section[ Felix Salten an Arthur Schnitzler, 11. 4. 1928]{L03586 Felix Salten an Arthur Schnitzler,  11. 4. 1928}
\nopagebreak\mylabel{L03586v}
\rehead{ }\normalsize\beginnumbering\briefempfaengerindex{Schnitzler, Arthur@\textsc{Schnitzler, Arthur}!zzzSalten, Felix@\emph{von Felix Salten}!1928-04-111@{11. 4. 1928}|(be}
\toendnotes[C]{\smallbreak\pagebreak[2]}
\correspDesc{Versand  durch Felix Salten am 11. 4. 1928 in Wien
\newline{}Erhalt  durch Arthur Schnitzler am 11. 4. 1928 in Wien}\toendnotes[C]{\smallbreak}
\Standort{CUL, Schnitzler, B 89, B 2.}
\physDesc{Karte, 471 Zeichen
\newline{}Handschrift: Bleistift, lateinische Kurrent
\newline{}Schnitzler: mit rotem Buntstift zwei Unterstreichungen 
\newline{}Ordnung: mit Bleistift von unbekannter Hand nummeriert: »299« }\toendnotes[C]{\smallbreak}
\pstart
           \raggedleft{}{\pb}11. IV. 28\pend
           
\pstart{}Lieber,\pend\vspace{0.5em}
\pstart
           vielleicht haben Sie noch \label{K_L03586-1v}\edtext{mein Buch
                  »Schrei der Liebe\pwindex{Salten, Felix 6.\,9.\,1869 Budapest – 8.\,10.\,1945 Zürich@\textsc{Salten, Felix} (6.\,9.\,1869 Budapest – 8.\,10.\,1945 Zürich), \emph{Schriftsteller, Journalist, Chefredakteur}!Schrei der Liebe. Novelle@\strich\emph{Der Schrei der Liebe. Novelle}|pw}}{\lemma{\textnormal{\emph{mein … Liebe}}}\Cendnote{\textnormal{erschienen im \emph{Wiener Verlag}\orgindex{Wiener Verlag@Wiener Verlag|pwk} im Oktober 1904,
                     siehe XXXX Auszeichnungsfehler: Dokument L03053 nicht gefunden.}}}\label{K_L03586-1}«; ich gab es Ihnen damals, als es erschien. Auch die \label{K_L03586-2v}\edtext{zweite Ausgabe bei Georg Müller\orgindex{Georg Müller Verlag@Georg Müller Verlag|pw}, zusammen mit der »Gedenktafel\pwindex{Salten, Felix 6.\,9.\,1869 Budapest – 8.\,10.\,1945 Zürich@\textsc{Salten, Felix} (6.\,9.\,1869 Budapest – 8.\,10.\,1945 Zürich), \emph{Schriftsteller, Journalist, Chefredakteur}!Gedenktafel der Prinzessin Anna. Der Schrei der Liebe. Zwei Novellen@\strich\emph{Die Gedenktafel der Prinzessin Anna. Der Schrei der Liebe. Zwei Novellen}|pw}«, hab’ ich Ihnen dediziert}{\lemma{\textnormal{\emph{zweite … dediziert}}}\Cendnote{\textnormal{Das Widmungsexemplar der Ausgabe von 1913 ist 
                     nicht überliefert.}}}\label{K_L03586-2}. Jetzt ist das kleine Buch\pwindex{Salten, Felix 6.\,9.\,1869 Budapest – 8.\,10.\,1945 Zürich@\textsc{Salten, Felix} (6.\,9.\,1869 Budapest – 8.\,10.\,1945 Zürich), \emph{Schriftsteller, Journalist, Chefredakteur}!Schrei der Liebe. Novelle@\strich\emph{Der Schrei der Liebe. Novelle}|pwv} total vergriffen, ich \label{K_L03586-3v}\edtext{brauche es dringend}{\lemma{\textnormal{\emph{brauche es dringend}}}\Cendnote{\textnormal{Salten\pwindex{Salten, Felix 6.\,9.\,1869 Budapest – 8.\,10.\,1945 Zürich@\textsc{Salten, Felix} (6.\,9.\,1869 Budapest – 8.\,10.\,1945 Zürich), \emph{Schriftsteller, Journalist, Chefredakteur}|pwk} benötigte das Buch\pwindex{Salten, Felix 6.\,9.\,1869 Budapest – 8.\,10.\,1945 Zürich@\textsc{Salten, Felix} (6.\,9.\,1869 Budapest – 8.\,10.\,1945 Zürich), \emph{Schriftsteller, Journalist, Chefredakteur}!Schrei der Liebe. Novelle@\strich\emph{Der Schrei der Liebe. Novelle}|pwkv} als Grundlage für eine Neuausgabe,
                     vgl. XXXX Auszeichnungsfehler: Dokument L03044 nicht gefunden.}}}\label{K_L03586-3} und kann
               es nirgendwo kriegen. Wenn Sie es noch haben und so gut sein wollen, es mir für zwei
               Wochen zu leihen, wäre sich sehr dankbar. Sie bekommen es unversehrt zurück.\pend
           
\pstart
           Herzlichst {\\[\baselineskip]}Ihr {\\[\baselineskip]}\spacefill\mbox{Felix Salten}\pend
           \leftskip=0em{}\selectlanguage{ngerman}\endnumbering\briefempfaengerindex{Schnitzler, Arthur@\textsc{Schnitzler, Arthur}!zzzSalten, Felix@\emph{von Felix Salten}!1928-04-111@{11. 4. 1928}|)be}\mylabel{L03586h}  \newcommand{\dateiname}{L03586}\newcommand{\titel}{Felix Salten an Arthur Schnitzler, 11. 4. 1928}\newcommand{\editorInnen}{Martin Anton Müller und Laura Untner}%% latex-leseansicht-abspann.tex
%% Abspann für die Leseansicht.
%% Der Schalter \ifkorrekturansicht ist bereits durch den Vorspann gesetzt.

%% latex-abspann.tex
%% Gemeinsamer Abspann für Korrekturansicht und Leseansicht.
%% Setzt den Schalter \ifkorrekturansicht voraus (gesetzt in den
%% einbindenden Dateien latex-korrekturansicht-abspann.tex bzw.
%% latex-leseansicht-abspann.tex).
%% ---------------------------------------------------------------

\normalsize

% Das esempio-Environment wird nur in der Leseansicht benötigt
\ifkorrekturansicht\else
\newenvironment{esempio}[3]%
{
    \vspace{1.5ex}
    \rlap{\underline{#1}}
    \par
    \setlength{\parindent}{0cm}
    \nopagebreak
    \leftskip=#2cm
    \rightskip=#3cm
}
{
    \par
}
\fi

\doendnotes{C}
\bigskip
\vfill

\clearpage

\footnotesize

\ifkorrekturansicht
  \lohead{\textsc{register}}
\fi

% theindex-Environment neu definieren ohne reledmac
\makeatletter
\renewenvironment{theindex}{%
  \ifkorrekturansicht
    \section*{\indexname}%
  \else
    \subsubsection*{Index der erwähnten Entitäten}%
  \fi
  \setlength{\parindent}{0pt}%
  \setlength{\parskip}{0pt plus 0.3pt}%
  \let\item\@idxitem
}{%
  \ifkorrekturansicht\clearpage\fi
}
\makeatother

\IfFileExists{\jobname-pw.ind}{\input{\jobname-pw.ind}}{}

% Quellenangabe nur in der Leseansicht
\ifkorrekturansicht\else
% Fallback-Definitionen, falls die .tex-Datei \titel etc. nicht gesetzt hat
\providecommand{\titel}{}
\providecommand{\editorInnen}{}
\providecommand{\dateiname}{\jobname}

\vspace{3cm}

\vfill

\footnotesize
\textsc{Quelle}: \titel. Herausgegeben von {\editorInnen}. In: \emph{Arthur Schnitzler: Briefwechsel mit Autorinnen und Autoren}.
 Digitale Edition, https://schnitzler-briefe.acdh.oeaw.ac.at/{\dateiname}.html (Stand \today)
\fi

\end{document}


