%% latex-leseansicht-vorspann.tex
%% Vorspann für die Leseansicht.
%% Lädt die gemeinsame Datei latex-vorspann.tex mit nicht gesetztem Schalter.

\newif\ifkorrekturansicht
\korrekturansichtfalse

\input{../tex-inputs/latex-vorspann}


\section[ Paul Goldmann an Arthur Schnitzler, 25. 8. {[}1904{]}]{L03452 Paul Goldmann an Arthur Schnitzler,  25. 8. [1904]}
\nopagebreak\mylabel{L03452v}
\rehead{ }\normalsize\beginnumbering\briefempfaengerindex{Schnitzler, Arthur@\textsc{Schnitzler, Arthur}!zzzGoldmann, Paul@\emph{von Paul Goldmann}!1904-08-252@{25. 8. [1904]}|(be}
\toendnotes[C]{\smallbreak\pagebreak[2]}
\correspDesc{Versand  durch Paul Goldmann am 25. 8. [1904] in Madonna di Campiglio
\newline{}Erhalt  durch Arthur Schnitzler im Zeitraum [26. 8. 1904
                  – 30. 8. 1904?] in Wien}\toendnotes[C]{\smallbreak}
\Standort{DLA, A:Schnitzler, HS.NZ85.1.3174.}
\physDesc{Bildpostkarte, 246 Zeichen
\newline{}Handschrift: schwarze Tinte, deutsche Kurrent
\newline{}Versand: Stempel: »\nobreak{}\oindex{Madonna di Campiglio@\textbf{Madonna di Campiglio}|pwk}\textcolor{gray}{Madonna di Campiglio}\nobreak{}«.  
\newline{}Schnitzler: mit Bleistift das Jahr »904« vermerkt }\toendnotes[C]{\smallbreak}\pstart{}\textsc{{\pb}Herrn}\pend{}\pstart{}\textsc{Dr. Arthur Schnitzler}\pend{}\pstart{}\textsc{Wien\oindex{Wien@\textbf{Wien}, \emph{Verwaltungsgebiet}|pw}}\pend{}\pstart{}\textsc{XVIII\textcolor{gray}{.}
                  Spöttelgaſse 7\oindex{Wien@\textbf{Wien}!XVIII., Währing@\textbf{XVIII., Währing}!Edmund-Weiß-Gasse 7@\textbf{Edmund-Weiß-Gasse 7}, \emph{Wohngebäude}|pw}.}\pend{}{\bigskip}
\pstart
           \noindent{}\centering{}{\pb}\textcolor{gray}{\textbf{Campo di Carlo Magno\oindex{Campo Carlo Magno@\textbf{Campo Carlo Magno}, \emph{Pass}|pw}}}\pend
           \vspace{1em}
\pstart
           \centering{}{\pb}25. Auguſt.\pend
           \vspace{0.5em}
\pstart
           \label{K_L03452-1v}\edtext{M. l.}{\lemma{\textnormal{\emph{M. l.}}}\Cendnote{\textnormal{Mein lieber}}}\label{K_L03452-1} Freund,{ }\label{K_L03452-2v}\edtext{kennſt Du \textsc{Madonna di Campiglio\oindex{Madonna di Campiglio@\textbf{Madonna di Campiglio}|pw}}}{\lemma{\textnormal{\emph{kennst … Campiglio}}}\Cendnote{\textnormal{Schnitzler hatte im Vorjahr, vom 17. 8. 1903 auf den
                     18. 8. 1903, in
                  dem Ort\oindex{Madonna di Campiglio@\textbf{Madonna di Campiglio}|pwkv} übernachtet und war
                  von dort zu einem Treffen mit Goldmann\pwindex{Goldmann, Paul 31.\,1.\,1865 Breslau – 25.\,9.\,1935 Wien@\textsc{Goldmann, Paul} (31.\,1.\,1865 Breslau – 25.\,9.\,1935 Wien), \emph{Schriftsteller, Journalist}|pwk}
                  geradelt.}}}\label{K_L03452-2}? Ein herrlicher Ort. Köſtlichſte Wald- und Alpen\oindex{Alpen@\textbf{Alpen}|pw}luft. Hätteſt Du nicht Luſt \label{K_L03452-3v}\edtext{hierherzukommen}{\lemma{\textnormal{\emph{hierherzukommen}}}\Cendnote{\textnormal{Dazu kam es nicht.}}}\label{K_L03452-3}? Herzliche Grüße Dir und Deiner Frau\pwindex{Schnitzler, Olga 17.\,1.\,1882 Wien – 13.\,1.\,1970 Lugano@\textsc{Schnitzler, Olga} (17.\,1.\,1882 Wien – 13.\,1.\,1970 Lugano), \emph{Schauspielerin, Sängerin}|pwv} Dein \spacefill\mbox{Paul Goldmann}\pend
           \selectlanguage{ngerman}\endnumbering\briefempfaengerindex{Schnitzler, Arthur@\textsc{Schnitzler, Arthur}!zzzGoldmann, Paul@\emph{von Paul Goldmann}!1904-08-252@{25. 8. [1904]}|)be}\mylabel{L03452h}  \newcommand{\dateiname}{L03452}\newcommand{\titel}{Paul Goldmann an Arthur Schnitzler, 25. 8. [1904]}\newcommand{\editorInnen}{Martin Anton Müller und Laura Untner}%% latex-leseansicht-abspann.tex
%% Abspann für die Leseansicht.
%% Der Schalter \ifkorrekturansicht ist bereits durch den Vorspann gesetzt.

%% latex-abspann.tex
%% Gemeinsamer Abspann für Korrekturansicht und Leseansicht.
%% Setzt den Schalter \ifkorrekturansicht voraus (gesetzt in den
%% einbindenden Dateien latex-korrekturansicht-abspann.tex bzw.
%% latex-leseansicht-abspann.tex).
%% ---------------------------------------------------------------

\normalsize

% Das esempio-Environment wird nur in der Leseansicht benötigt
\ifkorrekturansicht\else
\newenvironment{esempio}[3]%
{
    \vspace{1.5ex}
    \rlap{\underline{#1}}
    \par
    \setlength{\parindent}{0cm}
    \nopagebreak
    \leftskip=#2cm
    \rightskip=#3cm
}
{
    \par
}
\fi

\doendnotes{C}
\bigskip
\vfill

\clearpage

\footnotesize

\ifkorrekturansicht
  \lohead{\textsc{register}}
\fi

% theindex-Environment neu definieren ohne reledmac
\makeatletter
\renewenvironment{theindex}{%
  \ifkorrekturansicht
    \section*{\indexname}%
  \else
    \subsubsection*{Index der erwähnten Entitäten}%
  \fi
  \setlength{\parindent}{0pt}%
  \setlength{\parskip}{0pt plus 0.3pt}%
  \let\item\@idxitem
}{%
  \ifkorrekturansicht\clearpage\fi
}
\makeatother

\IfFileExists{\jobname-pw.ind}{\input{\jobname-pw.ind}}{}

% Quellenangabe nur in der Leseansicht
\ifkorrekturansicht\else
% Fallback-Definitionen, falls die .tex-Datei \titel etc. nicht gesetzt hat
\providecommand{\titel}{}
\providecommand{\editorInnen}{}
\providecommand{\dateiname}{\jobname}

\vspace{3cm}

\vfill

\footnotesize
\textsc{Quelle}: \titel. Herausgegeben von {\editorInnen}. In: \emph{Arthur Schnitzler: Briefwechsel mit Autorinnen und Autoren}.
 Digitale Edition, https://schnitzler-briefe.acdh.oeaw.ac.at/{\dateiname}.html (Stand \today)
\fi

\end{document}


