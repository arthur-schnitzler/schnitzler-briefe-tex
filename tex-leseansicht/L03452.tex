%% latex-leseansicht-vorspann.tex
%% Vorspann für die Leseansicht.
%% Lädt die gemeinsame Datei latex-vorspann.tex mit nicht gesetztem Schalter.

\newif\ifkorrekturansicht
\korrekturansichtfalse

\input{../tex-inputs/latex-vorspann}

\begin{center}
            \textcolor{red}{ENTWURF, NICHT FERTIG KORRIGIERT}
                      \end{center}
            
         
         \renewcommand{\erwaehntePersonen}{Personen: Paul Goldmann, Olga Schnitzler}
         \renewcommand{\erwaehnteOrte}{Orte: Alpen, Campo Carlo Magno, Edmund-Weiß-Gasse 7, Madonna di Campiglio, Wien}
         \renewcommand{\erwaehnteWerke}{}
               \section[ Paul Goldmann an Arthur Schnitzler, 25. 8. {[}1904{]}]{ Paul Goldmann an Arthur Schnitzler, 25. 8. {[}1904{]}}\nopagebreak\mylabel{v}\rehead{ }\begin{ledgroupsized}[t]{13cm}\normalsize\beginnumbering\briefempfaengerindex{Schnitzler, Arthur@\textsc{Schnitzler, Arthur}!zzzGoldmann, Paul@\emph{von Paul Goldmann}!1904-08-251@{25. 8. {[}1904{]}}|(be} \toendnotes[C]{\smallbreak\pagebreak[2]} \Standort{DLA, A:Schnitzler, HS.NZ85.1.3174.}
\physDesc{Bildpostkarte, 247 Zeichen
\newline{}Handschrift: 1) schwarze Tinte, deutsche Kurrent\hspace{1em}2) schwarze Tinte, lateinische Kurrent (\noindent{}Adresse)\hspace{1em}
\newline{}Versand: Stempel: »\nobreak{}\oindex{Madonna di Campiglio@\textbf{Madonna di Campiglio}|pwk}\textcolor{gray}{Madonna di Campiglio}\nobreak{}«.  
\newline{}Schnitzler: mit Bleistift das Jahr »904« vermerkt }\toendnotes[C]{\smallbreak}\pstart{}{\pb}Herrn\pend{}\pstart{}Dr. Arthur Schnitzler\pend{}\pstart{}Wien\oindex{Wien@\textbf{Wien}|pw}\pend{}\pstart{}XVIII. Spöttelgaſse 7\oindex{Edmund-Weiss-Gasse 7@\textbf{Edmund-Weiß-Gasse 7}|pw}.\pend{}{\bigskip}\pstart
           \noindent{}\centering{}{\pb}\textcolor{gray}{\textbf{Campo di Carlo Magno\oindex{Campo Carlo Magno@\textbf{Campo Carlo Magno}|pw}}}\pend
           \pstart
           \raggedleft{}25. Auguſt. \pend
           \pstart
           \label{K_L03452-1v}\edtext{M. l.}{\lemma{\textnormal{\emph{M. l.}}}\Cendnote{\textnormal{Mein lieber}}}\label{K_L03452-1h} Freund, \label{K_L03452-2v}\edtext{kennſt Du \textsc{Madonna di Campiglio\oindex{Madonna di Campiglio@\textbf{Madonna di Campiglio}|pw}}}{\lemma{\textnormal{\emph{kennſt … Campiglio}}}\Cendnote{\textnormal{Schnitzler\pwindex{Schnitzler, Arthur 15.05.1862 – 21.10.1931@\textsc{Schnitzler, Arthur} (15.05.1862 – 21.10.1931), \emph{Schriftsteller, Mediziner}|pwk} hat im Vorjahr vom 17. 8. 1903 auf den
                        18. 8. 1903 in
                     dem Ort übernachtet und war von da aus zu einem Treffen mit Goldmann\pwindex{Goldmann, Paul 31.01.1865 – 25.09.1935@\textsc{Goldmann, Paul} (31.01.1865 – 25.09.1935), \emph{Schriftsteller, Journalist}|pwk} geradelt.}}}\label{K_L03452-2h}? Ein herrlicher Ort. Köſtlichſte Wald- und Alpen\oindex{Alpen@\textbf{Alpen}|pw}luft. Hätteſt Du nicht Luſt \label{K_L03452-3v}\edtext{hierherzukommen}{\lemma{\textnormal{\emph{hierherzukommen}}}\Cendnote{\textnormal{nicht geschehen}}}\label{K_L03452-3h}? Herzliche Grüße Dir und Deiner Frau\pwindex{Schnitzler, Olga 17.01.1882 – 13.01.1970@\textsc{Schnitzler, Olga} (17.01.1882 – 13.01.1970), \emph{Schauspielerin, Sängerin}|pwv}\pend
           \pstart Dein \spacefill\mbox{Paul Goldmann}\pend{}
         
         \endnumbering\mylabel{h}\end{ledgroupsized}  \newcommand{\dateiname}{L03452}\newcommand{\titel}{Paul Goldmann an Arthur Schnitzler, 25. 8. [1904]}\newcommand{\editorInnen}{Martin Anton Müller und Laura Untner}%% latex-leseansicht-abspann.tex
%% Abspann für die Leseansicht.
%% Der Schalter \ifkorrekturansicht ist bereits durch den Vorspann gesetzt.

%% latex-abspann.tex
%% Gemeinsamer Abspann für Korrekturansicht und Leseansicht.
%% Setzt den Schalter \ifkorrekturansicht voraus (gesetzt in den
%% einbindenden Dateien latex-korrekturansicht-abspann.tex bzw.
%% latex-leseansicht-abspann.tex).
%% ---------------------------------------------------------------

\normalsize

% Das esempio-Environment wird nur in der Leseansicht benötigt
\ifkorrekturansicht\else
\newenvironment{esempio}[3]%
{
    \vspace{1.5ex}
    \rlap{\underline{#1}}
    \par
    \setlength{\parindent}{0cm}
    \nopagebreak
    \leftskip=#2cm
    \rightskip=#3cm
}
{
    \par
}
\fi

\doendnotes{C}
\bigskip
\vfill

\clearpage

\footnotesize

\ifkorrekturansicht
  \lohead{\textsc{register}}
\fi

% theindex-Environment neu definieren ohne reledmac
\makeatletter
\renewenvironment{theindex}{%
  \ifkorrekturansicht
    \section*{\indexname}%
  \else
    \subsubsection*{Index der erwähnten Entitäten}%
  \fi
  \setlength{\parindent}{0pt}%
  \setlength{\parskip}{0pt plus 0.3pt}%
  \let\item\@idxitem
}{%
  \ifkorrekturansicht\clearpage\fi
}
\makeatother

\IfFileExists{\jobname-pw.ind}{\input{\jobname-pw.ind}}{}

% Quellenangabe nur in der Leseansicht
\ifkorrekturansicht\else
% Fallback-Definitionen, falls die .tex-Datei \titel etc. nicht gesetzt hat
\providecommand{\titel}{}
\providecommand{\editorInnen}{}
\providecommand{\dateiname}{\jobname}

\vspace{3cm}

\vfill

\footnotesize
\textsc{Quelle}: \titel. Herausgegeben von {\editorInnen}. In: \emph{Arthur Schnitzler: Briefwechsel mit Autorinnen und Autoren}.
 Digitale Edition, https://schnitzler-briefe.acdh.oeaw.ac.at/{\dateiname}.html (Stand \today)
\fi

\end{document}


      