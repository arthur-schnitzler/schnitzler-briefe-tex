%% latex-leseansicht-vorspann.tex
%% Vorspann für die Leseansicht.
%% Lädt die gemeinsame Datei latex-vorspann.tex mit nicht gesetztem Schalter.

\newif\ifkorrekturansicht
\korrekturansichtfalse

\input{../tex-inputs/latex-vorspann}


               \section[Stefan Großmann an Arthur Schnitzler, 5. 12. 1913]{ Stefan Großmann an Arthur Schnitzler, 5. 12. 1913}\nopagebreak\mylabel{v}\rehead{ }\begin{ledgroupsized}[t]{13cm}\normalsize\beginnumbering\briefempfaengerindex{Schnitzler, Arthur@\textsc{Schnitzler, Arthur}!zzzGrossmann, Stefan@\emph{von Stefan Großmann}!1913-12-051@{5. 12. 1913}|(be} \toendnotes[C]{\smallbreak\pagebreak[2]} \Standort{CUL, Schnitzler, B 34.}
\physDesc{Brief, 1 Blatt, 1 Seite
\newline{}Handschrift: schwarze Tinte, deutsche Kurrent
\newline{}Schnitzler: mit rotem Buntstift eine Unterstreichung \newline{}Ordnung: mit Bleistift von unbekannter Hand nummeriert: »14« }\pstart
           \noindent{}{\pb}\textcolor{gray}{\textbf{STEFAN GROSSMANN}}\hfill \textcolor{gray}{\textbf{WIEN,\oindex{Wien@\textbf{Wien}|pw}}}{ }5. \substVorne{}\textsuperscript{\textsc{Janner}}{\allowbreak}\substDazwischen{}\textsc{December}\substHinten{} 1913\pend
           \pstart
           \raggedleft{}I. \textsc{Dominikanerbastei} 5\oindex{Dominikanerbastei@\textbf{Dominikanerbastei}|pw}\pend
           \pstart\center{}Sehr verehrter Herr Schnitzler\pend\pstart
           Der Verlag Ullstein\orgindex{Ullstein Verlag@Ullstein Verlag|pw} theilt mir mit, daſs er bis
                    zum 2. oder 3. Jänner warten will, wenn er mit
                    einiger Beſtimmtheit auf einen Beitrag von ihnen rechnen kann. Nur würden wir
                    bitten, uns ungefähr das Ausmaß Ihres Beitrages vorausſagen zu wollen, wenn Sie
                    das können.\pend
           \pstart
           Selbſtverſtändlich wäre es dem Redakteur eine große Erleichterung, Ihren Beitrag
                    früher zu erhalten.\pend
           \pstart
           Jedenfalls danken wir Ihnen für Ihre Bereitwilligkeit und hoffen ſehr auf Ihre
                    Gabe.\pend
           \pstart
           Sehr ergeben{\\[\baselineskip]} Ihr \spacefill\mbox{Stefan Großmann}\pend
           \leftskip=0em{}          \endnumbering\briefempfaengerindex{Schnitzler, Arthur@\textsc{Schnitzler, Arthur}!zzzGrossmann, Stefan@\emph{von Stefan Großmann}!1913-12-051@{5. 12. 1913}|)be}\mylabel{h}\end{ledgroupsized}  \newcommand{\dateiname}{L02160}\newcommand{\titel}{Stefan Großmann an Arthur Schnitzler, 5. 12. 1913}\newcommand{\editorInnen}{Martin Anton Müller und Gerd-Hermann Susen}%% latex-leseansicht-abspann.tex
%% Abspann für die Leseansicht.
%% Der Schalter \ifkorrekturansicht ist bereits durch den Vorspann gesetzt.

%% latex-abspann.tex
%% Gemeinsamer Abspann für Korrekturansicht und Leseansicht.
%% Setzt den Schalter \ifkorrekturansicht voraus (gesetzt in den
%% einbindenden Dateien latex-korrekturansicht-abspann.tex bzw.
%% latex-leseansicht-abspann.tex).
%% ---------------------------------------------------------------

\normalsize

% Das esempio-Environment wird nur in der Leseansicht benötigt
\ifkorrekturansicht\else
\newenvironment{esempio}[3]%
{
    \vspace{1.5ex}
    \rlap{\underline{#1}}
    \par
    \setlength{\parindent}{0cm}
    \nopagebreak
    \leftskip=#2cm
    \rightskip=#3cm
}
{
    \par
}
\fi

\doendnotes{C}
\bigskip
\vfill

\clearpage

\footnotesize

\ifkorrekturansicht
  \lohead{\textsc{register}}
\fi

% theindex-Environment neu definieren ohne reledmac
\makeatletter
\renewenvironment{theindex}{%
  \ifkorrekturansicht
    \section*{\indexname}%
  \else
    \subsubsection*{Index der erwähnten Entitäten}%
  \fi
  \setlength{\parindent}{0pt}%
  \setlength{\parskip}{0pt plus 0.3pt}%
  \let\item\@idxitem
}{%
  \ifkorrekturansicht\clearpage\fi
}
\makeatother

\IfFileExists{\jobname-pw.ind}{\input{\jobname-pw.ind}}{}

% Quellenangabe nur in der Leseansicht
\ifkorrekturansicht\else
% Fallback-Definitionen, falls die .tex-Datei \titel etc. nicht gesetzt hat
\providecommand{\titel}{}
\providecommand{\editorInnen}{}
\providecommand{\dateiname}{\jobname}

\vspace{3cm}

\vfill

\footnotesize
\textsc{Quelle}: \titel. Herausgegeben von {\editorInnen}. In: \emph{Arthur Schnitzler: Briefwechsel mit Autorinnen und Autoren}.
 Digitale Edition, https://schnitzler-briefe.acdh.oeaw.ac.at/{\dateiname}.html (Stand \today)
\fi

\end{document}


      