%% latex-leseansicht-vorspann.tex
%% Vorspann für die Leseansicht.
%% Lädt die gemeinsame Datei latex-vorspann.tex mit nicht gesetztem Schalter.

\newif\ifkorrekturansicht
\korrekturansichtfalse

\input{../tex-inputs/latex-vorspann}


\section[Hugo von Hofmannsthal an Arthur Schnitzler, {[}27. 10. 1909{]}]{L01882 Hugo von Hofmannsthal an Arthur Schnitzler, {[}27. 10. 1909{]}}
\nopagebreak\mylabel{L01882v}
\rehead{ }\normalsize\beginnumbering\briefempfaengerindex{Schnitzler, Arthur@\textsc{Schnitzler, Arthur}!zzzHofmannsthal, Hugo von@\emph{von Hugo von Hofmannsthal}!1909-10-273@{27. 10. 1909}|(be}
\toendnotes[C]{\smallbreak\pagebreak[2]}
\correspDesc{Versand  durch Hugo von Hofmannsthal am 27. 10. 1909 in Rodaun
\newline{}Erhalt  durch Arthur Schnitzler am 27. 10. 1909 in Wien}\toendnotes[C]{\smallbreak}
\Standort{CUL, Schnitzler, B 43.}
\physDesc{Telegramm, 36 Zeichen
\newline{}HandschriftX2 einer Schreibkraft: blauer Buntstift, deutsche Kurrent
\newline{}Versand: »\noindent{}Roda\damage{\textcolor{gray}{un}}\oindex{Wien@\textbf{Wien}!XXIII., Liesing@\textbf{XXIII., Liesing}!Rodaun@\textbf{Rodaun}, \emph{Region}|pw}{ }\textcolor{gray}{\textbf{Nr.{\dots} Taxw.{\dots} (W.{\dots} Ch.{\dots}) aufgegeben am}}{ }{[}27./{]}10 \textcolor{gray}{\textbf{190{\dotstwo}}}{ }\textcolor{gray}{\textbf{um}}{ }3 \textcolor{gray}{\textbf{Uhr {\dots}
                                             M.}} n \textcolor{gray}{\textbf{Mittag}}\textcolor{gray}{\textbf{.}}« 
\newline{}Schnitzler: mit Bleistift datiert: »2\substVorne{}\textsuperscript{6}\substDazwischen{}7\substHinten{}/X 09« und beschriftet:
                                 »Hofm« 
\newline{}Ordnung: 1) beschnitten  2) mit Bleistift von unbekannter Hand nummeriert:
                                    »310«}
\buchAbdrucke{\weitereDrucke{Hugo von Hofmannsthal, Arthur Schnitzler: \emph{Briefwechsel}. Herausgegeben von Therese Nickl und Heinrich Schnitzler. Frankfurt am Main: \emph{S. Fischer} 1964, S. 246.} }\toendnotes[C]{\smallbreak}
\pstart
           \noindent{}{\pb}= Freue mich rieſig auf \label{K_L01882-1v}\edtext{montag\pwindex{Schnitzler, Arthur 15.\,5.\,1862 Wien – 21.\,10.\,1931 ebd.@\textsc{Schnitzler, Arthur} (15.\,5.\,1862 Wien – 21.\,10.\,1931 ebd.), \emph{Schriftsteller, Mediziner}!junge Medardus. Dramatische Historie in einem Vorspiel und fünf Aufzügen@\strich\emph{Der junge Medardus. Dramatische Historie in einem Vorspiel und fünf Aufzügen}|pwv}}{\lemma{\textnormal{\emph{montag}}}\Cendnote{\textnormal{Vgl. A. S.: \emph{Tagebuch}, 1. 11. 1909.
               }}}\label{K_L01882-1} =\pend
           \pstart \spacefill\mbox{Hugo +}\pend{}\selectlanguage{ngerman}\endnumbering\briefempfaengerindex{Schnitzler, Arthur@\textsc{Schnitzler, Arthur}!zzzHofmannsthal, Hugo von@\emph{von Hugo von Hofmannsthal}!1909-10-273@{27. 10. 1909}|)be}\mylabel{L01882h}  \newcommand{\dateiname}{L01882}\newcommand{\titel}{Hugo von Hofmannsthal an Arthur Schnitzler, [27. 10. 1909]}\newcommand{\editorInnen}{Martin Anton Müller und Gerd-Hermann Susen}%% latex-leseansicht-abspann.tex
%% Abspann für die Leseansicht.
%% Der Schalter \ifkorrekturansicht ist bereits durch den Vorspann gesetzt.

%% latex-abspann.tex
%% Gemeinsamer Abspann für Korrekturansicht und Leseansicht.
%% Setzt den Schalter \ifkorrekturansicht voraus (gesetzt in den
%% einbindenden Dateien latex-korrekturansicht-abspann.tex bzw.
%% latex-leseansicht-abspann.tex).
%% ---------------------------------------------------------------

\normalsize

% Das esempio-Environment wird nur in der Leseansicht benötigt
\ifkorrekturansicht\else
\newenvironment{esempio}[3]%
{
    \vspace{1.5ex}
    \rlap{\underline{#1}}
    \par
    \setlength{\parindent}{0cm}
    \nopagebreak
    \leftskip=#2cm
    \rightskip=#3cm
}
{
    \par
}
\fi

\doendnotes{C}
\bigskip
\vfill

\clearpage

\footnotesize

\ifkorrekturansicht
  \lohead{\textsc{register}}
\fi

% theindex-Environment neu definieren ohne reledmac
\makeatletter
\renewenvironment{theindex}{%
  \ifkorrekturansicht
    \section*{\indexname}%
  \else
    \subsubsection*{Index der erwähnten Entitäten}%
  \fi
  \setlength{\parindent}{0pt}%
  \setlength{\parskip}{0pt plus 0.3pt}%
  \let\item\@idxitem
}{%
  \ifkorrekturansicht\clearpage\fi
}
\makeatother

\IfFileExists{\jobname-pw.ind}{\input{\jobname-pw.ind}}{}

% Quellenangabe nur in der Leseansicht
\ifkorrekturansicht\else
% Fallback-Definitionen, falls die .tex-Datei \titel etc. nicht gesetzt hat
\providecommand{\titel}{}
\providecommand{\editorInnen}{}
\providecommand{\dateiname}{\jobname}

\vspace{3cm}

\vfill

\footnotesize
\textsc{Quelle}: \titel. Herausgegeben von {\editorInnen}. In: \emph{Arthur Schnitzler: Briefwechsel mit Autorinnen und Autoren}.
 Digitale Edition, https://schnitzler-briefe.acdh.oeaw.ac.at/{\dateiname}.html (Stand \today)
\fi

\end{document}


