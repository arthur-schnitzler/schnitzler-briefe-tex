%% latex-korrekturansicht-vorspann.tex
%% Vorspann für die Korrekturansicht.
%% Lädt die gemeinsame Datei latex-vorspann.tex mit gesetztem Schalter.

\newif\ifkorrekturansicht
\korrekturansichttrue

\input{../tex-inputs/latex-vorspann}


\section[ Paul Goldmann an Arthur Schnitzler, 18. 9. {[}1906{]}]{L03251 Paul Goldmann an Arthur Schnitzler, 18. 9. {[}1906{]}}
\nopagebreak\mylabel{L03251v}
\rehead{ }\normalsize\beginnumbering\briefempfaengerindex{Schnitzler, Arthur@\textsc{Schnitzler, Arthur}!zzzGoldmann, Paul@\emph{von Paul Goldmann}!1906-09-181@{18. 9. {[}1906{]}}|(be}
\toendnotes[C]{\smallbreak\pagebreak[2]}\Standort{DLA, A:Schnitzler, HS.NZ85.1.3175.}
\physDesc{Brief, 1 Blatt, 2 Seiten, 820 Zeichen
\newline{}Handschrift: schwarze Tinte, deutsche Kurrent
\newline{}Schnitzler: mit Bleistift das Jahr »{[}19{]}06« vermerkt }\toendnotes[C]{\smallbreak}
\pstart
           \raggedleft{}{\pb}\textcolor{gray}{\textbf{\textbf{GRAND HÔTEL\oindex{Grand Hotel Wien@\textbf{Grand Hotel Wien}, \emph{Hotel (K.HTL)}|pw}, \begin{otherlanguage}{french}VIENNE\oindex{Wien@\textbf{Wien}, \emph{A.ADM2}|pw}\end{otherlanguage}}}}\pend
           
\pstart
           \raggedleft{}\textcolor{gray}{\textbf{I., KÄRNTNERRING. 9\oindex{Kaerntnerring@\textbf{Kärntnerring}, \emph{Straße (K.STR)}|pw}. }}\pend
           
\pstart
           \textsc{18. Sept}.\pend
           
\pstart{}Mein lieber Freund,\pend\vspace{0.5em}
\pstart
           Es thut mir unendlich leid, nicht gewußt zu haben, daß Du auf dem \label{K_L03251-1v}\edtext{\textsc{Semmering\oindex{Semmering@\textbf{Semmering}, \emph{A.ADM3}|pw}}}{\lemma{\textnormal{\emph{Semmering}}}\Cendnote{\textnormal{Schnitzler hielt sich zwischen 10. 9. 1906 und 20. 9. 1906 auf dem
                     Semmering\oindex{Semmering@\textbf{Semmering}, \emph{A.ADM3}|pwk} auf.}}}\label{K_L03251-1} biſt. Denn ich bin
                  \label{K_L03251-2v}\edtext{über den \textsc{Semmering\oindex{Semmering@\textbf{Semmering}, \emph{A.ADM3}|pw}} gefahren}{\lemma{\textnormal{\emph{über … gefahren}}}\Cendnote{\textnormal{Der Semmering\oindex{Semmering@\textbf{Semmering}, \emph{A.ADM3}|pwk} liegt südlich von Wien\oindex{Wien@\textbf{Wien}, \emph{A.ADM2}|pwk}, Goldmann\pwindex{Goldmann, Paul 31.01.1865 – 25.09.1935@\textsc{Goldmann, Paul} (31.01.1865 – 25.09.1935), \emph{Schriftsteller/Schriftstellerin, Journalist/Journalistin}|pwk} dürfte also von
                  einer Reise und nicht aus Berlin\oindex{Berlin@\textbf{Berlin}, \emph{P.PPLC}|pwk} nach Wien\oindex{Wien@\textbf{Wien}, \emph{A.ADM2}|pwk} gekommen sein.}}}\label{K_L03251-2} u. wäre gern
               ausgeſtiegen, um einen Tag mit Dir zu verbringen. Auch in Wien\oindex{Wien@\textbf{Wien}, \emph{A.ADM2}|pw} werde ich Dich leider nicht ſehen, da ich vorausſichtlich
                  übermorgen heimfahre.\pend
           
\pstart
           Deine liebe \label{K_L03251-3v}\edtext{Karte mit den ſchönen
                  Verſen}{\lemma{\textnormal{\emph{Karte … Verſen}}}\Cendnote{\textnormal{Schnitzler hatte Goldmann\pwindex{Goldmann, Paul 31.01.1865 – 25.09.1935@\textsc{Goldmann, Paul} (31.01.1865 – 25.09.1935), \emph{Schriftsteller/Schriftstellerin, Journalist/Journalistin}|pwk} eine gereimte Karte geschrieben, siehe A. S.: \emph{Tagebuch}, 5. 8. 1906. Nachdem Goldmann\pwindex{Goldmann, Paul 31.01.1865 – 25.09.1935@\textsc{Goldmann, Paul} (31.01.1865 – 25.09.1935), \emph{Schriftsteller/Schriftstellerin, Journalist/Journalistin}|pwk} in Folge in einen
                  satirisch-ironischen Ton verfällt, dürfte Schnitzler darin Goldmanns\pwindex{Goldmann, Paul 31.01.1865 – 25.09.1935@\textsc{Goldmann, Paul} (31.01.1865 – 25.09.1935), \emph{Schriftsteller/Schriftstellerin, Journalist/Journalistin}|pwk} Arbeiten gelobt haben.}}}\label{K_L03251-3} (wirklich, welch’ ein Talent!) iſt auch erſt vor
               Kurzem {\pb}in meinen Beſitz gekommen. Ich hätte manches
               darauf zu antworten – aber wozu? Es hat keinen Sinn, \label{K_L03251-4v}\edtext{auch noch \textsc{privatim} zu polemiſiren}{\lemma{\textnormal{\emph{auch … polemiſiren}}}\Cendnote{\textnormal{Sammelbände von Goldmanns\pwindex{Goldmann, Paul 31.01.1865 – 25.09.1935@\textsc{Goldmann, Paul} (31.01.1865 – 25.09.1935), \emph{Schriftsteller/Schriftstellerin, Journalist/Journalistin}|pwk}
                  Theaterkritik erschienen in dieser Zeit durchwegs mit dem Untertitel: »Polemische Aufsätze über Berliner Theater-Aufführungen«.}}}\label{K_L03251-4}. Ich werde mich lieber darauf beſchränken, Dein
               nächſtes Stück öffentlich ſchlecht zu machen. {\\}Im Ernſt: ich hätte Dir ſehr, ſehr
               gern die Hand gedrückt. Vielleicht gibſt Du mir im Laufe des Winters \label{K_L03251-5v}\edtext{Gelegenheit dazu}{\lemma{\textnormal{\emph{Gelegenheit dazu}}}\Cendnote{\textnormal{Schnitzler und Goldmann\pwindex{Goldmann, Paul 31.01.1865 – 25.09.1935@\textsc{Goldmann, Paul} (31.01.1865 – 25.09.1935), \emph{Schriftsteller/Schriftstellerin, Journalist/Journalistin}|pwk} trafen sich erst am 24. 5. 1907 in Wien\oindex{Wien@\textbf{Wien}, \emph{A.ADM2}|pwk} wieder.}}}\label{K_L03251-5} in Berlin\oindex{Berlin@\textbf{Berlin}, \emph{P.PPLC}|pw}. \strikeout{\textcolor{gray}{×}\-\textcolor{gray}{×}}\pend
           
\pstart
           Inzwiſchen ſei ſamt Frau\pwindex{Schnitzler, Olga 17.01.1882 – 13.01.1970@\textsc{Schnitzler, Olga} (17.01.1882 – 13.01.1970), \emph{Schauspieler/Schauspielerin, Sänger/Sängerin}|pwv}
               u. Kind\pwindex{Schnitzler, Heinrich 09.08.1902 – 12.07.1982@\textsc{Schnitzler, Heinrich} (09.08.1902 – 12.07.1982), \emph{Regisseur/Regisseurin, Schauspieler/Schauspielerin}|pwv} herzlichſt gegrüßt
               von {\\[\baselineskip]}Deinem getreuen {\\[\baselineskip]}\spacefill\mbox{Paul Goldmann.}\pend
           \leftskip=0em{}\selectlanguage{ngerman}\endnumbering\briefempfaengerindex{Schnitzler, Arthur@\textsc{Schnitzler, Arthur}!zzzGoldmann, Paul@\emph{von Paul Goldmann}!1906-09-181@{18. 9. {[}1906{]}}|)be}\mylabel{L03251h}  \normalsize

\doendnotes{C}
\bigskip
\vfill

\clearpage

\footnotesize

\lohead{\textsc{register}}

% Definiere theindex-Environment komplett neu ohne reledmac
\makeatletter
\renewenvironment{theindex}{%
  \section*{\indexname}%
  \setlength{\parindent}{0pt}%
  \setlength{\parskip}{0pt plus 0.3pt}%
  \let\item\@idxitem
}{%
  \clearpage
}
\makeatother

\IfFileExists{\jobname-pw.ind}{\input{\jobname-pw.ind}}{}

\end{document}

      