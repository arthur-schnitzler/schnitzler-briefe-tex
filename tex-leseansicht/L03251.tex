%% latex-leseansicht-vorspann.tex
%% Vorspann für die Leseansicht.
%% Lädt die gemeinsame Datei latex-vorspann.tex mit nicht gesetztem Schalter.

\newif\ifkorrekturansicht
\korrekturansichtfalse

\input{../tex-inputs/latex-vorspann}


         
         \renewcommand{\erwaehntePersonen}{Personen: Paul Goldmann, Olga Schnitzler, Heinrich Schnitzler}
         \renewcommand{\erwaehnteOrte}{Orte: Berlin, Grand Hotel Wien, Kärntnerring, Semmering, Wien}
         \renewcommand{\erwaehnteWerke}{}
               \section[ Paul Goldmann an Arthur Schnitzler, 18. 9. {[}1906{]}]{ Paul Goldmann an Arthur Schnitzler, 18. 9. {[}1906{]}}\nopagebreak\mylabel{v}\rehead{ }\begin{ledgroupsized}[t]{13cm}\normalsize\beginnumbering\briefempfaengerindex{Schnitzler, Arthur@\textsc{Schnitzler, Arthur}!zzzGoldmann, Paul@\emph{von Paul Goldmann}!1906-09-181@{18. 9. {[}1906{]}}|(be} \toendnotes[C]{\smallbreak\pagebreak[2]} \Standort{DLA, A:Schnitzler, HS.NZ85.1.3175.}
\physDesc{Brief, 1 Blatt, 2 Seiten, 820 Zeichen
\newline{}Handschrift: schwarze Tinte, deutsche Kurrent
\newline{}Schnitzler: mit Bleistift das Jahr »{[}19{]}06« vermerkt }\toendnotes[C]{\smallbreak}\pstart
           \noindent{}\raggedleft{}{\pb}\textcolor{gray}{\textbf{\textbf{GRAND HÔTEL\oindex{Grand Hotel Wien@\textbf{Grand Hotel Wien}|pw}, \begin{otherlanguage}{french}VIENNE\oindex{Wien@\textbf{Wien}|pw}\end{otherlanguage}}}}\pend
           \pstart
           \noindent{}\raggedleft{}\textcolor{gray}{\textbf{I., KÄRNTNERRING. 9\oindex{Kaerntnerring@\textbf{Kärntnerring}|pw}. }}\pend
           \pstart
           \textsc{18. Sept}.\pend
           \pstart{}Mein lieber Freund,\pend\pstart
           Es thut mir unendlich leid, nicht gewußt zu haben, daß Du auf dem \label{K_L03251-1v}\edtext{\textsc{Semmering\oindex{Semmering@\textbf{Semmering}|pw}}}{\lemma{\textnormal{\emph{Semmering}}}\Cendnote{\textnormal{Schnitzler\pwindex{Schnitzler, Arthur 15.05.1862 – 21.10.1931@\textsc{Schnitzler, Arthur} (15.05.1862 – 21.10.1931), \emph{Schriftsteller, Mediziner}|pwk} hielt sich zwischen 10. 9. 1906 und 20. 9. 1906 auf dem
                     Semmering\oindex{Semmering@\textbf{Semmering}|pwk} auf.}}}\label{K_L03251-1h} biſt. Denn ich bin
                  \label{K_L03251-2v}\edtext{über den \textsc{Semmering\oindex{Semmering@\textbf{Semmering}|pw}} gefahren}{\lemma{\textnormal{\emph{über … gefahren}}}\Cendnote{\textnormal{Der Semmering\oindex{Semmering@\textbf{Semmering}|pwk} liegt südlich von Wien\oindex{Wien@\textbf{Wien}|pwk}, Goldmann\pwindex{Goldmann, Paul 31.01.1865 – 25.09.1935@\textsc{Goldmann, Paul} (31.01.1865 – 25.09.1935), \emph{Schriftsteller, Journalist}|pwk} dürfte also von
                  einer Reise und nicht aus Berlin\oindex{Berlin@\textbf{Berlin}|pwk} nach Wien\oindex{Wien@\textbf{Wien}|pwk} gekommen sein.}}}\label{K_L03251-2h} u. wäre gern
               ausgeſtiegen, um einen Tag mit Dir zu verbringen. Auch in Wien\oindex{Wien@\textbf{Wien}|pw} werde ich Dich leider nicht ſehen, da ich vorausſichtlich
                  übermorgen heimfahre.\pend
           \pstart
           Deine liebe \label{K_L03251-3v}\edtext{Karte mit den ſchönen
                  Verſen}{\lemma{\textnormal{\emph{Karte … Verſen}}}\Cendnote{\textnormal{Schnitzler\pwindex{Schnitzler, Arthur 15.05.1862 – 21.10.1931@\textsc{Schnitzler, Arthur} (15.05.1862 – 21.10.1931), \emph{Schriftsteller, Mediziner}|pwk} hatte Goldmann\pwindex{Goldmann, Paul 31.01.1865 – 25.09.1935@\textsc{Goldmann, Paul} (31.01.1865 – 25.09.1935), \emph{Schriftsteller, Journalist}|pwk} eine gereimte Karte geschrieben, siehe A. S.: \emph{Tagebuch}, 5. 8. 1906. Nachdem Goldmann\pwindex{Goldmann, Paul 31.01.1865 – 25.09.1935@\textsc{Goldmann, Paul} (31.01.1865 – 25.09.1935), \emph{Schriftsteller, Journalist}|pwk} in Folge in einen
                  satirisch-ironischen Ton verfällt, dürfte Schnitzler\pwindex{Schnitzler, Arthur 15.05.1862 – 21.10.1931@\textsc{Schnitzler, Arthur} (15.05.1862 – 21.10.1931), \emph{Schriftsteller, Mediziner}|pwk} darin Goldmanns\pwindex{Goldmann, Paul 31.01.1865 – 25.09.1935@\textsc{Goldmann, Paul} (31.01.1865 – 25.09.1935), \emph{Schriftsteller, Journalist}|pwk} Arbeiten gelobt haben.}}}\label{K_L03251-3h} (wirklich, welch’ ein Talent!) iſt auch erſt vor
               Kurzem {\pb}in meinen Beſitz gekommen. Ich hätte manches
               darauf zu antworten – aber wozu? Es hat keinen Sinn, \label{K_L03251-4v}\edtext{auch noch \textsc{privatim} zu polemiſiren}{\lemma{\textnormal{\emph{auch … polemiſiren}}}\Cendnote{\textnormal{Sammelbände von Goldmanns\pwindex{Goldmann, Paul 31.01.1865 – 25.09.1935@\textsc{Goldmann, Paul} (31.01.1865 – 25.09.1935), \emph{Schriftsteller, Journalist}|pwk}
                  Theaterkritik erschienen in dieser Zeit durchwegs mit dem Untertitel: »Polemische Aufsätze über Berliner Theater-Aufführungen«.}}}\label{K_L03251-4h}. Ich werde mich lieber darauf beſchränken, Dein
               nächſtes Stück öffentlich ſchlecht zu machen. {\\}Im Ernſt: ich hätte Dir ſehr, ſehr
               gern die Hand gedrückt. Vielleicht gibſt Du mir im Laufe des Winters \label{K_L03251-5v}\edtext{Gelegenheit dazu}{\lemma{\textnormal{\emph{Gelegenheit dazu}}}\Cendnote{\textnormal{Schnitzler\pwindex{Schnitzler, Arthur 15.05.1862 – 21.10.1931@\textsc{Schnitzler, Arthur} (15.05.1862 – 21.10.1931), \emph{Schriftsteller, Mediziner}|pwk} und Goldmann\pwindex{Goldmann, Paul 31.01.1865 – 25.09.1935@\textsc{Goldmann, Paul} (31.01.1865 – 25.09.1935), \emph{Schriftsteller, Journalist}|pwk} trafen sich erst am 24. 5. 1907 in Wien\oindex{Wien@\textbf{Wien}|pwk} wieder.}}}\label{K_L03251-5h} in Berlin\oindex{Berlin@\textbf{Berlin}|pw}. \strikeout{\textcolor{gray}{×}\-\textcolor{gray}{×}}\pend
           \pstart
           Inzwiſchen ſei ſamt Frau\pwindex{Schnitzler, Olga 17.01.1882 – 13.01.1970@\textsc{Schnitzler, Olga} (17.01.1882 – 13.01.1970), \emph{Schauspielerin, Sängerin}|pwv}
               u. Kind\pwindex{Schnitzler, Heinrich 09.08.1902 – 12.07.1982@\textsc{Schnitzler, Heinrich} (09.08.1902 – 12.07.1982), \emph{Regisseur, Schauspieler}|pwv} herzlichſt gegrüßt
               von {\\[\baselineskip]}Deinem getreuen {\\[\baselineskip]}\spacefill\mbox{Paul Goldmann.}\pend
           \leftskip=0em{}
         
         \endnumbering\mylabel{h}\end{ledgroupsized}  \newcommand{\dateiname}{L03251}\newcommand{\titel}{Paul Goldmann an Arthur Schnitzler, 18. 9. [1906]}\newcommand{\editorInnen}{Martin Anton Müller und Laura Untner}%% latex-leseansicht-abspann.tex
%% Abspann für die Leseansicht.
%% Der Schalter \ifkorrekturansicht ist bereits durch den Vorspann gesetzt.

%% latex-abspann.tex
%% Gemeinsamer Abspann für Korrekturansicht und Leseansicht.
%% Setzt den Schalter \ifkorrekturansicht voraus (gesetzt in den
%% einbindenden Dateien latex-korrekturansicht-abspann.tex bzw.
%% latex-leseansicht-abspann.tex).
%% ---------------------------------------------------------------

\normalsize

% Das esempio-Environment wird nur in der Leseansicht benötigt
\ifkorrekturansicht\else
\newenvironment{esempio}[3]%
{
    \vspace{1.5ex}
    \rlap{\underline{#1}}
    \par
    \setlength{\parindent}{0cm}
    \nopagebreak
    \leftskip=#2cm
    \rightskip=#3cm
}
{
    \par
}
\fi

\doendnotes{C}
\bigskip
\vfill

\clearpage

\footnotesize

\ifkorrekturansicht
  \lohead{\textsc{register}}
\fi

% theindex-Environment neu definieren ohne reledmac
\makeatletter
\renewenvironment{theindex}{%
  \ifkorrekturansicht
    \section*{\indexname}%
  \else
    \subsubsection*{Index der erwähnten Entitäten}%
  \fi
  \setlength{\parindent}{0pt}%
  \setlength{\parskip}{0pt plus 0.3pt}%
  \let\item\@idxitem
}{%
  \ifkorrekturansicht\clearpage\fi
}
\makeatother

\IfFileExists{\jobname-pw.ind}{\input{\jobname-pw.ind}}{}

% Quellenangabe nur in der Leseansicht
\ifkorrekturansicht\else
% Fallback-Definitionen, falls die .tex-Datei \titel etc. nicht gesetzt hat
\providecommand{\titel}{}
\providecommand{\editorInnen}{}
\providecommand{\dateiname}{\jobname}

\vspace{3cm}

\vfill

\footnotesize
\textsc{Quelle}: \titel. Herausgegeben von {\editorInnen}. In: \emph{Arthur Schnitzler: Briefwechsel mit Autorinnen und Autoren}.
 Digitale Edition, https://schnitzler-briefe.acdh.oeaw.ac.at/{\dateiname}.html (Stand \today)
\fi

\end{document}


      