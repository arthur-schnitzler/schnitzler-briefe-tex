%% latex-leseansicht-vorspann.tex
%% Vorspann für die Leseansicht.
%% Lädt die gemeinsame Datei latex-vorspann.tex mit nicht gesetztem Schalter.

\newif\ifkorrekturansicht
\korrekturansichtfalse

\input{../tex-inputs/latex-vorspann}


\section[Therese Rie-Andro an Arthur Schnitzler, {{[}}Anfang Juli 1923{{]}}]{L02573 Therese Rie-Andro an Arthur Schnitzler, {[}Anfang Juli 1923{]}}
\nopagebreak\mylabel{L02573v}
\rehead{ }\normalsize\beginnumbering\briefempfaengerindex{Schnitzler, Arthur@\textsc{Schnitzler, Arthur}!zzzRie, Therese@\emph{von Therese Rie}!1923-07-201@{{[}Anfang Juli 1923{]}}|(be}
\toendnotes[C]{\smallbreak\pagebreak[2]}
\correspDesc{Versand  durch Therese Rie im Zeitraum [Anfang Juli 1923] in Bernried
\newline{}Erhalt  durch Arthur Schnitzler in Wien}\toendnotes[C]{\smallbreak}
\Standort{DLA, A:Schnitzler, 85.1.4310.}
\physDesc{Brief, 1 Blatt, 3 Seiten, 1185 Zeichen
\newline{}Handschrift: Bleistift, lateinische Kurrent
\newline{}Schnitzler: 1) mit Bleistift beschriftet: »\textsc{Andro.}«, datiert: »Juli 23«  2) mit rotem Buntstift sechs Unterstreichungen}
\pstart
           \raggedleft{}{\pb}Bernried/Starnbergerseee\oindex{Bernried@\textbf{Bernried}, \emph{Hauptstadt}|pw}\pend
           
\pstart
           \raggedleft{}Oberbayern\oindex{Oberbayern@\textbf{Oberbayern}, \emph{Verwaltungsgebiet}|pw}\pend
           
\pstart
           \raggedleft{}Altwirt\oindex{Hotel Seeblick@\textbf{Hotel Seeblick}, \emph{Hotel}|pw}\pend
           
\pstart{}Verehrter Herr Doktor,\pend\vspace{0.5em}
\pstart
           Dieser Ort iſt so lieb,{ }ſtill und schön, daſs ich Ihnen von da einen Gruß schicken
               muß. Vielleicht finden Sie diese Logik nicht zwingend, aber für mich beſteht sie
               doch. Wahrscheinlich entſpringt sie aus dem Wunsch, daß Sie für Ihre Erholung einen
               Platz finden möchten, der Ihren Neigungen ebenso entspricht, wie dieser hier den
               meinen – wo es nichts gibt als See und herrlich bewaldete Ufer und gar keine Städter
               und die netteſten Schafe, Ziegen und Gänse und gar keine Tinte. {\pb}Das einzige Tintenfaß in der Gegend befindet sich auf de\substVorne{}\textsuperscript{m}\substDazwischen{}r\substHinten{} »Amtstube« des Bürgermeiſters, der mir, als ich mich bei meinem erſten
               Aufenthalt – ich war{ }ſchon öfters hier – sagte, als ich mich als Ausländerin melden
               wollte: »Sö san do ka Ausländer, sö reden do wie mir; a Saupreuß\oindex{Deutschland@\textbf{Deutschland}|pw}, \uline{des} is a
               Ausländer!!«\pend
           
\pstart
           Und als ich diesmal sagte, ich käme jetzt selten ins Reich\oindex{Deutschland@\textbf{Deutschland}|pw}, meinte er: »Ja ja, ich ko{\geminationm} auch selten
               hin!« – – Und das alles gibts \uline{wirklich} und es ist
               nicht von Ludwig Thoma\pwindex{Thoma, Ludwig 21.\,1.\,1867 Oberammergau – 26.\,8.\,1921 Rottach-Egern@\textsc{Thoma, Ludwig} (21.\,1.\,1867 Oberammergau – 26.\,8.\,1921 Rottach-Egern), \emph{Schriftsteller}|pw} und es iſt eine Stunde
               von München\oindex{München@\textbf{München}|pw}, wo {\pb}es so
               übel knirscht, daſs man der nächsten Entwicklung der Dinge nur mit Besorgnis
               folgt.\pend
           
\pstart
           Und nun alle guten So{\geminationm}erwünſche für Sie!\pend
           
\pstart
           Ihre{\\[\baselineskip]}\spacefill\mbox{Therese Rie.}\pend
           \leftskip=0em{}\selectlanguage{ngerman}\endnumbering\briefempfaengerindex{Schnitzler, Arthur@\textsc{Schnitzler, Arthur}!zzzRie, Therese@\emph{von Therese Rie}!1923-07-011@{{[}Anfang Juli 1923{]}}|)be}\mylabel{L02573h}  \newcommand{\dateiname}{L02573}\newcommand{\titel}{Therese Rie-Andro an Arthur Schnitzler, [Anfang Juli 1923]}\newcommand{\editorInnen}{Martin Anton Müller und Gerd-Hermann Susen}%% latex-leseansicht-abspann.tex
%% Abspann für die Leseansicht.
%% Der Schalter \ifkorrekturansicht ist bereits durch den Vorspann gesetzt.

%% latex-abspann.tex
%% Gemeinsamer Abspann für Korrekturansicht und Leseansicht.
%% Setzt den Schalter \ifkorrekturansicht voraus (gesetzt in den
%% einbindenden Dateien latex-korrekturansicht-abspann.tex bzw.
%% latex-leseansicht-abspann.tex).
%% ---------------------------------------------------------------

\normalsize

% Das esempio-Environment wird nur in der Leseansicht benötigt
\ifkorrekturansicht\else
\newenvironment{esempio}[3]%
{
    \vspace{1.5ex}
    \rlap{\underline{#1}}
    \par
    \setlength{\parindent}{0cm}
    \nopagebreak
    \leftskip=#2cm
    \rightskip=#3cm
}
{
    \par
}
\fi

\doendnotes{C}
\bigskip
\vfill

\clearpage

\footnotesize

\ifkorrekturansicht
  \lohead{\textsc{register}}
\fi

% theindex-Environment neu definieren ohne reledmac
\makeatletter
\renewenvironment{theindex}{%
  \ifkorrekturansicht
    \section*{\indexname}%
  \else
    \subsubsection*{Index der erwähnten Entitäten}%
  \fi
  \setlength{\parindent}{0pt}%
  \setlength{\parskip}{0pt plus 0.3pt}%
  \let\item\@idxitem
}{%
  \ifkorrekturansicht\clearpage\fi
}
\makeatother

\IfFileExists{\jobname-pw.ind}{\input{\jobname-pw.ind}}{}

% Quellenangabe nur in der Leseansicht
\ifkorrekturansicht\else
% Fallback-Definitionen, falls die .tex-Datei \titel etc. nicht gesetzt hat
\providecommand{\titel}{}
\providecommand{\editorInnen}{}
\providecommand{\dateiname}{\jobname}

\vspace{3cm}

\vfill

\footnotesize
\textsc{Quelle}: \titel. Herausgegeben von {\editorInnen}. In: \emph{Arthur Schnitzler: Briefwechsel mit Autorinnen und Autoren}.
 Digitale Edition, https://schnitzler-briefe.acdh.oeaw.ac.at/{\dateiname}.html (Stand \today)
\fi

\end{document}


