%% latex-korrekturansicht-vorspann.tex
%% Vorspann für die Korrekturansicht.
%% Lädt die gemeinsame Datei latex-vorspann.tex mit gesetztem Schalter.

\newif\ifkorrekturansicht
\korrekturansichttrue

\input{../tex-inputs/latex-vorspann}


\section[Georg Brandes an Arthur Schnitzler, 19. 10. 1911]{L02040 Georg Brandes an Arthur Schnitzler, 19. 10. 1911}
\nopagebreak\mylabel{L02040v}
\rehead{ }\normalsize\beginnumbering\briefempfaengerindex{Schnitzler, Arthur@\textsc{Schnitzler, Arthur}!zzzBrandes, Georg@\emph{von Georg Brandes}!1911-10-191@{19. 10. 1911}|(be}
\toendnotes[C]{\smallbreak\pagebreak[2]}\Standort{CUL, Schnitzler, B 17.}
\physDesc{Brief, 1 Blatt, 4 Seiten, 3301 Zeichen
\newline{}Handschrift: schwarze Tinte, lateinische Kurrent
\newline{}Schnitzler: mit Bleistift beschriftet: »\textsc{Brandes}« 
\newline{}Ordnung: mit Bleistift von unbekannter Hand nummeriert:
                                    »37« }
\buchAbdrucke{\weitereDrucke{Georg Brandes, Arthur Schnitzler: \emph{Ein Briefwechsel}. Bern: \emph{Francke} 1956, S. 102–104.} }\toendnotes[C]{\smallbreak}
\pstart
           \raggedleft{}{\pb}\uline{Kopenhagen}\oindex{Daenemark@\textbf{Dänemark}, \emph{A.PCLI}|pw} (\uline{nicht}{ }Havnegade\oindex{Havnegade@\textbf{Havnegade}, \emph{Straße (K.STR)}|pw}){\\}19 Oct 11\pend
           
\pstart{}Mein verehrter Freund\pend\vspace{0.5em}
\pstart
           Ihr Schauspiel\pwindex{weite Land. Tragikomoedie in fuenf Akten@\emph{Das weite Land. Tragikomödie in fünf Akten}|pwv} und Ihr Brief
               haben mir beide tief bewegt. Der Brief, weil er so herzlich war und weil ich, seit
               lange von allerlei Unglück und Missgeschick verfolgt, für Herzlichkeit sehr
               empfänglich bin, das Schauspiel\pwindex{weite Land. Tragikomoedie in fuenf Akten@\emph{Das weite Land. Tragikomödie in fünf Akten}|pwv}, weil es mir das Werk eines Meisters scheint, vollreif.\pend
           
\pstart
           Diese Menschen, die Sie dort darstellen, stehen uns vor Augen als wirkliche
               Individualitäten, voll und rund und originell, mit Eigenschaften und Eigenheiten, die
               ein Ensemble ausmachen. Die Nebenfiguren wie Natter\pwindex{weite Land. Tragikomoedie in fuenf Akten@\emph{Das weite Land. Tragikomödie in fünf Akten}|pwv}, oder die amüsant Karikierten, wie Rhon\pwindex{weite Land. Tragikomoedie in fuenf Akten@\emph{Das weite Land. Tragikomödie in fünf Akten}|pwv} und Serknitz\pwindex{weite Land. Tragikomoedie in fuenf Akten@\emph{Das weite Land. Tragikomödie in fünf Akten}|pwv}, sind nicht weniger unvergesslich als die
               tiefsinnig studierten und räthselvollen wie Friedrich\pwindex{weite Land. Tragikomoedie in fuenf Akten@\emph{Das weite Land. Tragikomödie in fünf Akten}|pwv}, Genia\pwindex{weite Land. Tragikomoedie in fuenf Akten@\emph{Das weite Land. Tragikomödie in fünf Akten}|pwv} und die eine \uline{ganze}
               Seele, Erna\pwindex{weite Land. Tragikomoedie in fuenf Akten@\emph{Das weite Land. Tragikomödie in fünf Akten}|pwv}. {\pb}Ich würde nichts darüber schreiben
               können, das etwas hinzufügte an die Wirkung, und nichts, das irgend etwas erklärte,
               denn alles erklärt sich von selbst.\pend
           
\pstart
           Sie lieben es, die Nebentriebe und Nebenpassionen zu verfolgen, die Sprünge und
               Seitensprünge des Gefühlslebens, alles Getheilte, das von dem Hauptstamm sich ablöst,
               auszubreiten. Die Welt, so gesehen, ist auf eine specielle Weise traurig. Meiner
               Gefühlart nach wäre, um das Bild zu supplieren, auch das Erhebende, das ab und zu,
               wenn auch sehr selten, uns begegnet, ich meine: das, was das Leben erträglich macht,
                  \strikeout{auch} mit in Rechenschaft zu ziehen.\hspace*{2em}Ich bin, glaub ich, im Ganzen pessimistischer als Sie,
               aber dennoch empfind ich einige Ruhepunkte, und man muss das, soll man sich nicht
               tödten. Man muss z. B. Jemand vertrauen können; {\pb}in der hier vorgeführten, sehr
               reichen und schillernden Welt, ist aber jedes Vertrauen unmöglich; alle arbeiten sich
               von ihren Neigungen und Bänden los.\pend
           
\pstart
           Haben Sie Dank, dass Sie sich um das mir unbekannte Frl. Prozor\pwindex{Prozor, Grete 28.12.1885 – 14.02.1978@\textsc{Prozor, Grete} (28.12.1885 – 14.02.1978), \emph{Schauspieler/Schauspielerin}|pw} bemühten, und dass Sie ihr so nützlich waren.\pend
           
\pstart
           Sie irren sich wenn Sie glauben, ich möchte nicht gern nach Wien\oindex{Wien@\textbf{Wien}, \emph{A.ADM2}|pw} kommen. Im Gegentheil Wien\oindex{Wien@\textbf{Wien}, \emph{A.ADM2}|pw} hat immer für mich eine grosse Anziehungskraft gehabt; ich habe dort
               sehr angenehme Stunden verlebt, besonders – es ist lange her – in 1885,
               als ich den alten Gompertz\pwindex{Gomperz, Theodor 29.03.1832 – 29.08.1912@\textsc{Gomperz, Theodor} (29.03.1832 – 29.08.1912), \emph{Altphilologe/Altphilologin}|pw} kennen lernte.
               Später einmal, ich weiss nicht wann, es ist wohl 20 Jahre her, luden \uline{Sie} sich zu mir ein, und es war bei Ihnen eine \label{K_L02040-1v}\edtext{Herrengesellschaft}{\lemma{\textnormal{\emph{Herrengesellschaft}}}\Cendnote{\textnormal{Brandes dürfte auf den 22. 3. 1900 anspielen,
                  wenngleich Hofmannsthal\pwindex{Hofmannsthal, Hugo von 1874-02-01 – 1929-07-15@\textsc{Hofmannsthal, Hugo von} (1874-02-01 – 1929-07-15), \emph{Schriftsteller/Schriftstellerin}|pwk} im \emph{Tagebuch}\pwindex{Tagebuch@\emph{Tagebuch}|pwk} nicht explizit genannt ist.}}}\label{K_L02040-1}
               spät Abends, wo viele, die später berühmt \strikeout{\textcolor{gray}{×}\-\textcolor{gray}{×}\-\textcolor{gray}{×}} geworden, zusammen waren: Hoffmannsthal\pwindex{Hofmannsthal, Hugo von 1874-02-01 – 1929-07-15@\textsc{Hofmannsthal, Hugo von} (1874-02-01 – 1929-07-15), \emph{Schriftsteller/Schriftstellerin}|pw}, Wassermann\pwindex{Wassermann, Jakob 10.03.1873 – 01.01.1934@\textsc{Wassermann, Jakob} (10.03.1873 – 01.01.1934), \emph{Schriftsteller/Schriftstellerin}|pw}, und andere.
               Sonst habe ich in Wien\oindex{Wien@\textbf{Wien}, \emph{A.ADM2}|pw} nur bei Gompertz\pwindex{Gomperz, Theodor 29.03.1832 – 29.08.1912@\textsc{Gomperz, Theodor} (29.03.1832 – 29.08.1912), \emph{Altphilologe/Altphilologin}|pw} Menschen gesehen. Ich kenne {\pb}ja Niemand dort.\hspace*{2em}Aber ich bin in der Regel wie in einem Schraubenstock;
               ich kann nicht fort, wenn ich wollte, was zu weitläufig zu erklären ist. Leichter zu
               erklären ist, dass ich eigentlich nie Geld zu meinen Reisen habe. Aus Deutschland\oindex{Deutschland@\textbf{Deutschland}, \emph{A.PCLI}|pw} bekomme ich nie einen Pfennig, habe
               dort seit lange nicht einmal mehr einmal einen \uline{Verleger} und stehe mit keiner Zeitung in Verbindung. In Dänemark\oindex{Daenemark@\textbf{Dänemark}, \emph{A.PCLI}|pw} verdiente ich durch ein Buch im Jahr 10375 Kronen, aus
                  England\oindex{England@\textbf{England}, \emph{A.ADM1}|pw} bekomme ich als Royalty für ein
               Dutzend Bände jährlich 400 Kronen. – \uline{Ihre} Einnahmen
               werden sich glücklicherweise anders gestalten.\pend
           
\pstart
           Ich habe das Glück gehabt, meine Mutter\pwindex{Brandes, Emilie 22.03.1818 – 27.12.1898@\textsc{Brandes, Emilie} (22.03.1818 – 27.12.1898)|pwv} etwas länger zu behalten als es Ihnen gestattet wurde. Die Mutter ist
               ja vielleicht das einzige unbedingt sichere, das wir zum Vertrauen haben, um so
               unersetzlicher. Sie müssen jetzt 50 Jahre alt sein, ich bin in wenigen Monaten 70,
               deshalb einigermassen isolirt, obwohl mein Temperament dasselbe geblieben.\pend
           
\pstart
           Ich drücke Ihre Hand in alter Ergebenheit{\\[\baselineskip]}\spacefill\mbox{Georg Brandes}\pend
           \leftskip=0em{}\selectlanguage{ngerman}\endnumbering\briefempfaengerindex{Schnitzler, Arthur@\textsc{Schnitzler, Arthur}!zzzBrandes, Georg@\emph{von Georg Brandes}!1911-10-191@{19. 10. 1911}|)be}\mylabel{L02040h}  \normalsize

\doendnotes{C}
\bigskip
\vfill

\clearpage

\footnotesize

\lohead{\textsc{register}}

% Definiere theindex-Environment komplett neu ohne reledmac
\makeatletter
\renewenvironment{theindex}{%
  \section*{\indexname}%
  \setlength{\parindent}{0pt}%
  \setlength{\parskip}{0pt plus 0.3pt}%
  \let\item\@idxitem
}{%
  \clearpage
}
\makeatother

\IfFileExists{\jobname-pw.ind}{\input{\jobname-pw.ind}}{}

\end{document}

      