%% latex-leseansicht-vorspann.tex
%% Vorspann für die Leseansicht.
%% Lädt die gemeinsame Datei latex-vorspann.tex mit nicht gesetztem Schalter.

\newif\ifkorrekturansicht
\korrekturansichtfalse

\input{../tex-inputs/latex-vorspann}

\begin{center}
            \textcolor{red}{ENTWURF. ENTZIFFERUNG NOCH NICHT KORREKTURGELESEN}
                      \end{center}
            
               \section[Georg Brandes an Arthur Schnitzler, 19. 10. 1911]{ Georg Brandes an Arthur Schnitzler, 19. 10. 1911}\nopagebreak\mylabel{v}\rehead{ }\begin{ledgroupsized}[t]{13cm}\normalsize\beginnumbering\briefempfaengerindex{Schnitzler, Arthur@\textsc{Schnitzler, Arthur}!zzzBrandes, Georg@\emph{von Georg Brandes}!1911-10-191@{19. 10. 1911}|(be} \toendnotes[C]{\smallbreak\pagebreak[2]} \Standort{CUL, Schnitzler, B 17.}
\physDesc{Brief, 1 Blatt, 4 Seiten
\newline{}Handschrift: schwarze Tinte, lateinische Kurrent
\newline{}Schnitzler: mit Bleistift beschriftet: »\textsc{Brandes}« \newline{}Ordnung: mit Bleistift von unbekannter Hand nummeriert:
                                        »37« }\buchAbdrucke{\weitereDrucke{Georg Brandes, Arthur Schnitzler: \emph{Ein Briefwechsel}. Hg. Kurt Bergel. Bern: \emph{Francke} 1956, S. 102–104.} }\toendnotes[C]{\smallbreak}\pstart
           \raggedleft{}{\pb}\uline{Kopenhagen}\oindex{Daenemark@\textbf{Dänemark}|pw} (\uline{nicht}{ }Havnegade\oindex{Havnegade@\textbf{Havnegade}|pw}){\\}19 Oct 11\pend
           \pstart{}Mein verehrter Freund\pend\pstart
           Ihr Schauspiel\pwindex{Schnitzler, Arthur 15.05.1862 – 21.10.1931@\textsc{Schnitzler, Arthur} (15.05.1862 – 21.10.1931), \emph{Schriftsteller, Mediziner}!weite Land. Tragikomoedie in fuenf Akten1910-10-20@\strich\emph{Das weite Land. Tragikomödie in fünf Akten} {[}1910-10-20{]}|pwv} und Ihr Brief
                    haben mir beide tief bewegt. Der Brief, weil er so herzlich war und weil ich,
                    seit lange von allerlei Unglück und Missgeschick verfolgt, für Herzlichkeit sehr
                    empfänglich bin, das Schauspiel\pwindex{Schnitzler, Arthur 15.05.1862 – 21.10.1931@\textsc{Schnitzler, Arthur} (15.05.1862 – 21.10.1931), \emph{Schriftsteller, Mediziner}!weite Land. Tragikomoedie in fuenf Akten1910-10-20@\strich\emph{Das weite Land. Tragikomödie in fünf Akten} {[}1910-10-20{]}|pwv}, weil es mir das Werk eines Meisters scheint, vollreif.\pend
           \pstart
           Diese Menschen, die Sie dort darstellen, stehen uns vor Augen als wirkliche
                    Individualitäten, voll und rund und originell, mit Eigenschaften und
                    Eigenheiten, die ein Ensemble ausmachen. Die Nebenfiguren wie Natter\pwindex{Schnitzler, Arthur 15.05.1862 – 21.10.1931@\textsc{Schnitzler, Arthur} (15.05.1862 – 21.10.1931), \emph{Schriftsteller, Mediziner}!weite Land. Tragikomoedie in fuenf Akten1910-10-20@\strich\emph{Das weite Land. Tragikomödie in fünf Akten} {[}1910-10-20{]}|pwv}, oder die amüsant Karikierten,
                    wie Rhon\pwindex{Schnitzler, Arthur 15.05.1862 – 21.10.1931@\textsc{Schnitzler, Arthur} (15.05.1862 – 21.10.1931), \emph{Schriftsteller, Mediziner}!weite Land. Tragikomoedie in fuenf Akten1910-10-20@\strich\emph{Das weite Land. Tragikomödie in fünf Akten} {[}1910-10-20{]}|pwv} und Serknitz\pwindex{Schnitzler, Arthur 15.05.1862 – 21.10.1931@\textsc{Schnitzler, Arthur} (15.05.1862 – 21.10.1931), \emph{Schriftsteller, Mediziner}!weite Land. Tragikomoedie in fuenf Akten1910-10-20@\strich\emph{Das weite Land. Tragikomödie in fünf Akten} {[}1910-10-20{]}|pwv}, sind nicht weniger
                    unvergesslich als die tiefsinnig studierten und räthselvollen wie Friedrich\pwindex{Schnitzler, Arthur 15.05.1862 – 21.10.1931@\textsc{Schnitzler, Arthur} (15.05.1862 – 21.10.1931), \emph{Schriftsteller, Mediziner}!weite Land. Tragikomoedie in fuenf Akten1910-10-20@\strich\emph{Das weite Land. Tragikomödie in fünf Akten} {[}1910-10-20{]}|pwv}, Genia\pwindex{Schnitzler, Arthur 15.05.1862 – 21.10.1931@\textsc{Schnitzler, Arthur} (15.05.1862 – 21.10.1931), \emph{Schriftsteller, Mediziner}!weite Land. Tragikomoedie in fuenf Akten1910-10-20@\strich\emph{Das weite Land. Tragikomödie in fünf Akten} {[}1910-10-20{]}|pwv} und die eine \uline{ganze} Seele, Erna\pwindex{Schnitzler, Arthur 15.05.1862 – 21.10.1931@\textsc{Schnitzler, Arthur} (15.05.1862 – 21.10.1931), \emph{Schriftsteller, Mediziner}!weite Land. Tragikomoedie in fuenf Akten1910-10-20@\strich\emph{Das weite Land. Tragikomödie in fünf Akten} {[}1910-10-20{]}|pwv}. {\pb}Ich würde nichts darüber
                    schreiben können, das etwas hinzufügte an die Wirkung, und nichts, das irgend
                    etwas erklärte, denn alles erklärt sich von selbst.\pend
           \pstart
           Sie lieben es, die Nebentriebe und Nebenpassionen zu verfolgen, die Sprünge und
                    Seitensprünge des Gefühlslebens, alles Getheilte, das von dem Hauptstamm sich
                    ablöst, auszubreiten. Die Welt, so gesehen, ist auf eine specielle Weise
                    traurig. Meiner Gefühlart nach wäre, um das Bild zu supplieren, auch das
                    Erhebende, das ab und zu, wenn auch sehr selten, uns begegnet, ich meine: das,
                    was das Leben erträglich macht, \strikeout{auch} mit in
                    Rechenschaft zu ziehen.\hspace*{2em}Ich bin, glaub ich, im
                    Ganzen pessimistischer als Sie, aber dennoch empfind ich einige Ruhepunkte, und
                    man muss das, soll man sich nicht tödten. Man muss z. B. Jemand vertrauen
                    können; {\pb}in der hier
                    vorgeführten, sehr reichen und schillernden Welt, ist aber jedes Vertrauen
                    unmöglich; alle arbeiten sich von ihren Neigungen und Bänden los.\pend
           \pstart
           Haben Sie Dank, dass Sie sich um das mir unbekannte Frl. Prozor\pwindex{Prozor, Grete 28.12.1885 – 14.02.1978@\textsc{Prozor, Grete} (28.12.1885 – 14.02.1978), \emph{Schauspieler/Schauspielerin}|pw} bemühten, und dass Sie ihr so nützlich waren.\pend
           \pstart
           Sie irren sich wenn Sie glauben, ich möchte nicht gern nach Wien\oindex{Wien@\textbf{Wien}|pw} kommen. Im Gegentheil Wien\oindex{Wien@\textbf{Wien}|pw} hat immer für mich eine grosse Anziehungskraft gehabt; ich habe
                    dort sehr angenehme Stunden verlebt, besonders – es ist lange her – in
                        1885, als ich den alten Gompertz\pwindex{Gomperz, Theodor 29.03.1832 – 29.08.1912@\textsc{Gomperz, Theodor} (29.03.1832 – 29.08.1912), \emph{Altphilologe}|pw} kennen lernte. Später einmal, ich weiss nicht wann, es ist
                    wohl 20 Jahre her, luden \uline{Sie} sich zu mir ein,
                    und es war bei Ihnen eine \label{K_L02040_1v}\edtext{Herrengesellschaft}{\lemma{\textnormal{\emph{Herrengesellschaft}}}\Cendnote{\textnormal{Brandes dürfte
                        auf den 22. 3. 1900 anspielen,
                        wenngleich Hofmannsthal\pwindex{Hofmannsthal, Hugo von 01.02.1874 – 15.07.1929@\textsc{Hofmannsthal, Hugo von} (01.02.1874 – 15.07.1929), \emph{Schriftsteller}|pwk} im \emph{Tagebuch}\pwindex{Schnitzler, Arthur 15.05.1862 – 21.10.1931@\textsc{Schnitzler, Arthur} (15.05.1862 – 21.10.1931), \emph{Schriftsteller, Mediziner}!Tagebuch1981 – 2000@\strich\emph{Tagebuch} {[}1981 – 2000{]}|pwk} nicht explizit genannt
                        ist.}}}\label{K_L02040_1h} spät Abends, wo viele, die später berühmt \strikeout{\textcolor{gray}{×}\-\textcolor{gray}{×}\-\textcolor{gray}{×}} geworden, zusammen waren: Hoffmannsthal\pwindex{Hofmannsthal, Hugo von 01.02.1874 – 15.07.1929@\textsc{Hofmannsthal, Hugo von} (01.02.1874 – 15.07.1929), \emph{Schriftsteller}|pw}, Wassermann\pwindex{Wassermann, Jakob 10.03.1873 – 01.01.1934@\textsc{Wassermann, Jakob} (10.03.1873 – 01.01.1934), \emph{Schriftsteller}|pw}, und
                    andere. Sonst habe ich in Wien\oindex{Wien@\textbf{Wien}|pw} nur bei Gompertz\pwindex{Gomperz, Theodor 29.03.1832 – 29.08.1912@\textsc{Gomperz, Theodor} (29.03.1832 – 29.08.1912), \emph{Altphilologe}|pw} Menschen gesehen. Ich kenne {\pb}ja Niemand dort.\hspace*{2em}Aber ich bin in der Regel wie in einem
                    Schraubenstock; ich kann nicht fort, wenn ich wollte, was zu weitläufig zu
                    erklären ist. Leichter zu erklären ist, dass ich eigentlich nie Geld zu meinen
                    Reisen habe. Aus Deutschland\oindex{Deutschland@\textbf{Deutschland}|pw} bekomme ich nie
                    einen Pfennig, habe dort seit lange nicht einmal mehr einmal einen \uline{Verleger} und stehe mit keiner Zeitung in Verbindung.
                    In Dänemark\oindex{Daenemark@\textbf{Dänemark}|pw} verdiente ich durch ein Buch im
                    Jahr 10375 Kronen, aus England\oindex{England@\textbf{England}|pw} bekomme ich als
                    Royalty für ein Dutzend Bände jährlich 400 Kronen. – \uline{Ihre} Einnahmen werden sich glücklicherweise anders gestalten.\pend
           \pstart
           Ich habe das Glück gehabt, meine Mutter\pwindex{Brandes, Emilie 22.03.1818 – 27.12.1898@\textsc{Brandes, Emilie} (22.03.1818 – 27.12.1898)|pwv} etwas länger zu behalten als es Ihnen gestattet wurde. Die
                    Mutter ist ja vielleicht das einzige unbedingt sichere, das wir zum Vertrauen
                    haben, um so unersetzlicher. Sie müssen jetzt 50 Jahre alt sein, ich bin in
                    wenigen Monaten 70, deshalb einigermassen isolirt, obwohl mein Temperament
                    dasselbe geblieben.\pend
           \pstart
           Ich drücke Ihre Hand in alter Ergebenheit{\\[\baselineskip]}\spacefill\mbox{Georg Brandes}\pend
           \leftskip=0em{}\endnumbering\briefempfaengerindex{Schnitzler, Arthur@\textsc{Schnitzler, Arthur}!zzzBrandes, Georg@\emph{von Georg Brandes}!1911-10-191@{19. 10. 1911}|)be}\mylabel{h}\end{ledgroupsized}  \newcommand{\dateiname}{L02040}\newcommand{\titel}{Georg Brandes an Arthur Schnitzler, 19. 10. 1911}\newcommand{\editorInnen}{Martin Anton Müller und Gerd-Hermann Susen}%% latex-leseansicht-abspann.tex
%% Abspann für die Leseansicht.
%% Der Schalter \ifkorrekturansicht ist bereits durch den Vorspann gesetzt.

%% latex-abspann.tex
%% Gemeinsamer Abspann für Korrekturansicht und Leseansicht.
%% Setzt den Schalter \ifkorrekturansicht voraus (gesetzt in den
%% einbindenden Dateien latex-korrekturansicht-abspann.tex bzw.
%% latex-leseansicht-abspann.tex).
%% ---------------------------------------------------------------

\normalsize

% Das esempio-Environment wird nur in der Leseansicht benötigt
\ifkorrekturansicht\else
\newenvironment{esempio}[3]%
{
    \vspace{1.5ex}
    \rlap{\underline{#1}}
    \par
    \setlength{\parindent}{0cm}
    \nopagebreak
    \leftskip=#2cm
    \rightskip=#3cm
}
{
    \par
}
\fi

\doendnotes{C}
\bigskip
\vfill

\clearpage

\footnotesize

\ifkorrekturansicht
  \lohead{\textsc{register}}
\fi

% theindex-Environment neu definieren ohne reledmac
\makeatletter
\renewenvironment{theindex}{%
  \ifkorrekturansicht
    \section*{\indexname}%
  \else
    \subsubsection*{Index der erwähnten Entitäten}%
  \fi
  \setlength{\parindent}{0pt}%
  \setlength{\parskip}{0pt plus 0.3pt}%
  \let\item\@idxitem
}{%
  \ifkorrekturansicht\clearpage\fi
}
\makeatother

\IfFileExists{\jobname-pw.ind}{\input{\jobname-pw.ind}}{}

% Quellenangabe nur in der Leseansicht
\ifkorrekturansicht\else
% Fallback-Definitionen, falls die .tex-Datei \titel etc. nicht gesetzt hat
\providecommand{\titel}{}
\providecommand{\editorInnen}{}
\providecommand{\dateiname}{\jobname}

\vspace{3cm}

\vfill

\footnotesize
\textsc{Quelle}: \titel. Herausgegeben von {\editorInnen}. In: \emph{Arthur Schnitzler: Briefwechsel mit Autorinnen und Autoren}.
 Digitale Edition, https://schnitzler-briefe.acdh.oeaw.ac.at/{\dateiname}.html (Stand \today)
\fi

\end{document}


      