%% latex-korrekturansicht-vorspann.tex
%% Vorspann für die Korrekturansicht.
%% Lädt die gemeinsame Datei latex-vorspann.tex mit gesetztem Schalter.

\newif\ifkorrekturansicht
\korrekturansichttrue

\input{../tex-inputs/latex-vorspann}


\section[Marie Herzfeld an Arthur Schnitzler, 12. 3. {[}1931{]}]{L02596 Marie Herzfeld an Arthur Schnitzler, 12. 3. {[}1931{]}}
\nopagebreak\mylabel{L02596v}
\rehead{ }\normalsize\beginnumbering\briefempfaengerindex{Schnitzler, Arthur@\textsc{Schnitzler, Arthur}!zzzHerzfeld, Marie@\emph{von Marie Herzfeld}!1931-03-121@{12. 3. {[}1931{]}}|(be}
\toendnotes[C]{\smallbreak\pagebreak[2]}\Standort{DLA, A:Schnitzler, HS.1985.1.03436,7.}
\physDesc{Briefkarte, 305 Zeichen
\newline{}Handschrift: schwarze Tinte, lateinische Kurrent
\newline{}Schnitzler: mit rotem Buntstift Vermerk »\substVorne{}\textsuperscript{\textcolor{gray}{\textsc{Herzfeld}}}\substDazwischen{}\textsc{Herzfeld}\substHinten{}« und die Jahreszahl »31.« bei der Datumsangabe ergänzt }\toendnotes[C]{\smallbreak}
\pstart
           \centering{}{\pb}12/III\pend
           \vspace{0.5em}
\pstart
           Lieber D\textsuperscript{r} Schnitzler, welche schöne \label{K_L02596-1v}\edtext{Ueberraschung}{\lemma{\textnormal{\emph{Ueberraschung}}}\Cendnote{\textnormal{nicht
               ermittelt}}}\label{K_L02596-1}! Es gibt noch unerwartete Freuden. Am liebsten würde ich Ihnen gar
               nicht danken, \uline{nur} lesen – (anstatt zu arbeiten!),
               aber ich werde erst {\pb}ordentlich danken, wenn ich gelesen
               habe: dann schreibe ich ausführlich. Einstweilen nur: welche Freude! \pend
           \pstart \spacefill\mbox{Marie Herzfeld}\pend{}\selectlanguage{ngerman}\endnumbering\briefempfaengerindex{Schnitzler, Arthur@\textsc{Schnitzler, Arthur}!zzzHerzfeld, Marie@\emph{von Marie Herzfeld}!1931-03-121@{12. 3. {[}1931{]}}|)be}\mylabel{L02596h}  \normalsize

\doendnotes{C}
\bigskip
\vfill

\clearpage

\footnotesize

\lohead{\textsc{register}}

% Definiere theindex-Environment komplett neu ohne reledmac
\makeatletter
\renewenvironment{theindex}{%
  \section*{\indexname}%
  \setlength{\parindent}{0pt}%
  \setlength{\parskip}{0pt plus 0.3pt}%
  \let\item\@idxitem
}{%
  \clearpage
}
\makeatother

\IfFileExists{\jobname-pw.ind}{\input{\jobname-pw.ind}}{}

\end{document}

      