%% latex-leseansicht-vorspann.tex
%% Vorspann für die Leseansicht.
%% Lädt die gemeinsame Datei latex-vorspann.tex mit nicht gesetztem Schalter.

\newif\ifkorrekturansicht
\korrekturansichtfalse

\input{../tex-inputs/latex-vorspann}


\section[Arthur Schnitzler an Auguste Hauschner, 23.\,1.\,1909]{L02583 Arthur Schnitzler an Auguste Hauschner, 23.\,1.\,1909}
\nopagebreak\mylabel{L02583v}
\rehead{ }\normalsize\beginnumbering\briefempfaengerindex{Hauschner, Auguste@\textsc{Hauschner, Auguste}!zzzSchnitzler, Arthur@\emph{von Arthur Schnitzler}!1909-01-231@{23.\,1.\,1909}|(be}
\toendnotes[C]{\smallbreak\pagebreak[2]}
\correspDesc{Versand  durch Arthur Schnitzler am 23. 1. 1909 in Wien
\newline{}Erhalt  durch Auguste Hauschner im Zeitraum [24. 1. 1909
                  – 28. 1. 1909?] in Berlin}\toendnotes[C]{\smallbreak}
\Standort{Staatsbibliothek Berlin – Preußischer Kulturbesitz, Handschriftenabteilung, Nachlass Auguste Hauschner.}
\physDesc{Brief, 1 Blatt, 4 Seiten, 1101 Zeichen
\newline{}Handschrift: schwarze Tinte, lateinische Kurrent
\newline{}Hauschner: mit rotem Buntstift eine Unterstreichung unter
                                    »tautologisch«, eventuell, weil die Entzifferung
                                 Probleme bereitete }
\buchAbdrucke{\weitereDrucke{1) \pwindex{Schnitzler, Arthur 15.\,5.\,1862 Wien – 21.\,10.\,1931 ebd.@\textsc{Schnitzler, Arthur} (15.\,5.\,1862 Wien – 21.\,10.\,1931 ebd.), \emph{Schriftsteller, Mediziner}!Brief an Auguste Hauschner, 23.1.1909]@\strich\emph{[Brief an Auguste Hauschner, 23.1.1909]}|pwk}\pwindex{Briefe an Auguste Hauschner@\emph{Briefe an Auguste Hauschner}|pwk}Arthur Schnitzler: \emph{[Brief an Auguste Hauschner zum Weg ins Freie].} In: \emph{Briefe an Auguste Hauschner}. Herausgegeben von Martin Beradt und Lotte Bloch-Zavřel. Berlin: \emph{Ernst Rowohlt Verlag} [Ende Oktober 1928, vordatiert auf:]
                           1929, S. 106.} \weitereDrucke{2) Arthur Schnitzler: \emph{Briefe 1875–1912}. Herausgegeben von Therese Nickl und Heinrich Schnitzler. Frankfurt am Main: \emph{S. Fischer} 1981, S. 588.} }\toendnotes[C]{\smallbreak}
\pstart
           
\pstart
           {\pb}\textcolor{gray}{\textbf{Dr. Arthur Schnitzler}}\pend
           
\pstart
           \raggedleft{}23.\,1.\,09\pend
           \pend
           
\pstart{}verehrte Frau,\pend\vspace{0.5em}
\pstart
           ich danke Ihnen sehr, dass Sie mir Ihren schönen \label{K_L02583-1v}\edtext{Artikel\pwindex{Hauschner, Auguste 12.\,2.\,1850 Prag – 10.\,4.\,1924 Berlin@\textsc{Hauschner, Auguste} (12.\,2.\,1850 Prag – 10.\,4.\,1924 Berlin), \emph{Schriftstellerin}!Weg ins Freie@\strich\emph{Der Weg ins Freie}|pwv}}{\lemma{\textnormal{\emph{Artikel}}}\Cendnote{\textnormal{Auguste Hauschner\pwindex{Hauschner, Auguste 12.\,2.\,1850 Prag – 10.\,4.\,1924 Berlin@\textsc{Hauschner, Auguste} (12.\,2.\,1850 Prag – 10.\,4.\,1924 Berlin), \emph{Schriftstellerin}|pwk}: \emph{Der Weg ins Freie}\pwindex{Hauschner, Auguste 12.\,2.\,1850 Prag – 10.\,4.\,1924 Berlin@\textsc{Hauschner, Auguste} (12.\,2.\,1850 Prag – 10.\,4.\,1924 Berlin), \emph{Schriftstellerin}!Weg ins Freie@\strich\emph{Der Weg ins Freie}|pwk}. In: \emph{Die Hilfe}\pwindex{Hilfe. Zeitschrift für Politik, Wirtschaft und geistige Bewegung@\emph{Die Hilfe. Zeitschrift für Politik, Wirtschaft und geistige Bewegung}|pwk}, Jg. 15, Nr. 3, 17.\,1.\,1909, S. 39–40.
                     Schnitzler urteilte im \emph{Tagebuch}\pwindex{Schnitzler, Arthur 15.\,5.\,1862 Wien – 21.\,10.\,1931 ebd.@\textsc{Schnitzler, Arthur} (15.\,5.\,1862 Wien – 21.\,10.\,1931 ebd.), \emph{Schriftsteller, Mediziner}!Tagebuch@\strich\emph{Tagebuch}|pwk} am 15. 1. 1909: »Neue Kritikensammlung, von
                        Fischer\pwindex{Fischer, Samuel 24.\,12.\,1859 Liptovský Mikuláš – 15.\,10.\,1934 Berlin@\textsc{Fischer, Samuel} (24.\,12.\,1859 Liptovský Mikuláš – 15.\,10.\,1934 Berlin), \emph{Verleger}|pw} gesandt, über den Weg. Die
                        Hauschner\pwindex{Hauschner, Auguste 12.\,2.\,1850 Prag – 10.\,4.\,1924 Berlin@\textsc{Hauschner, Auguste} (12.\,2.\,1850 Prag – 10.\,4.\,1924 Berlin), \emph{Schriftstellerin}|pw}, fand endlich in der ›Hilfe\pwindex{Hilfe. Zeitschrift für Politik, Wirtschaft und geistige Bewegung@\emph{Die Hilfe. Zeitschrift für Politik, Wirtschaft und geistige Bewegung}|pw}‹ eine Stätte für ihren mir nun erst
                     bekannt werdenden sehr freundlichen Aufsatz.«}}}\label{K_L02583-1} geschickt haben.
               Gar viel wäre darüber zu sagen, wenn es mir nicht so fatal wäre, über meine eignen
               Sachen was niederzuschreiben. Reden kö{\geminationn}t ich schon eher
               drüber, nun vielleicht fügt es mein gutes Glück, dass {\pb}ich Ihnen irgend einmal in der Welt
               begegne. Übrigens, einfacher: we{\geminationn} Sie nach Wien\oindex{Wien@\textbf{Wien}, \emph{Verwaltungsgebiet}|pw} kommen, lassen Sie michs wissen, gnädige Frau,
               und we{\geminationn} ich nach Berlin\oindex{Berlin@\textbf{Berlin}, \emph{Hauptstadt}|pw} komme, darf ich mich wohl auch melden –? Vorher aber noch möcht ich
               Ihnen sagen, daß Sie Unrecht haben Ihren \label{K_L02583-2v}\edtext{Schluss »mislungen«}{\lemma{\textnormal{\emph{Schluss »mislungen«}}}\Cendnote{\textnormal{Siehe XXXX Auszeichnungsfehler: Dokument L02587 nicht gefunden.
               }}}\label{K_L02583-2} zu finden – auch ohne Ihren Brief {\pb}wüßt ich sehr gut, was Sie eigentlich
               sagen wollten. Und so viel tief und liebevoll (oder ist das tautologisch?)
               eindringendes in den vorherigen Absätzen. Wie
               viele Leseri{\geminationn}en Ihrer Art denken Sie gibt es wohl? Und
               gar eine, die zugleich Künstlerin ist { }{\dotsfive} jetzt aber ko{\geminationm}t es immer
               näher, – noch drei Zeilen, und ich fange an etwas über mein {\pb}Buch\pwindex{Schnitzler, Arthur 15.\,5.\,1862 Wien – 21.\,10.\,1931 ebd.@\textsc{Schnitzler, Arthur} (15.\,5.\,1862 Wien – 21.\,10.\,1931 ebd.), \emph{Schriftsteller, Mediziner}!Weg ins Freie. Roman@\strich\emph{Der Weg ins Freie. Roman}|pwv} zu sagen – daher nicht
               mehr als dies: Sie haben mir durch gedrucktes geschriebenes und
               gefühltes herzliche Freude bereitet!\pend
           
\pstart
           Ihr aufrichtig ergebner{\\[\baselineskip]}\spacefill\mbox{Arthur Schnitzler}\pend
           \leftskip=0em{}\selectlanguage{ngerman}\endnumbering\briefempfaengerindex{Hauschner, Auguste@\textsc{Hauschner, Auguste}!zzzSchnitzler, Arthur@\emph{von Arthur Schnitzler}!1909-01-231@{23.\,1.\,1909}|)be}\mylabel{L02583h}  \newcommand{\dateiname}{L02583}\newcommand{\titel}{Arthur Schnitzler an Auguste Hauschner, 23. 1. 1909}\newcommand{\editorInnen}{Martin Anton Müller und Laura Untner}%% latex-leseansicht-abspann.tex
%% Abspann für die Leseansicht.
%% Der Schalter \ifkorrekturansicht ist bereits durch den Vorspann gesetzt.

%% latex-abspann.tex
%% Gemeinsamer Abspann für Korrekturansicht und Leseansicht.
%% Setzt den Schalter \ifkorrekturansicht voraus (gesetzt in den
%% einbindenden Dateien latex-korrekturansicht-abspann.tex bzw.
%% latex-leseansicht-abspann.tex).
%% ---------------------------------------------------------------

\normalsize

% Das esempio-Environment wird nur in der Leseansicht benötigt
\ifkorrekturansicht\else
\newenvironment{esempio}[3]%
{
    \vspace{1.5ex}
    \rlap{\underline{#1}}
    \par
    \setlength{\parindent}{0cm}
    \nopagebreak
    \leftskip=#2cm
    \rightskip=#3cm
}
{
    \par
}
\fi

\doendnotes{C}
\bigskip
\vfill

\clearpage

\footnotesize

\ifkorrekturansicht
  \lohead{\textsc{register}}
\fi

% theindex-Environment neu definieren ohne reledmac
\makeatletter
\renewenvironment{theindex}{%
  \ifkorrekturansicht
    \section*{\indexname}%
  \else
    \subsubsection*{Index der erwähnten Entitäten}%
  \fi
  \setlength{\parindent}{0pt}%
  \setlength{\parskip}{0pt plus 0.3pt}%
  \let\item\@idxitem
}{%
  \ifkorrekturansicht\clearpage\fi
}
\makeatother

\IfFileExists{\jobname-pw.ind}{\input{\jobname-pw.ind}}{}

% Quellenangabe nur in der Leseansicht
\ifkorrekturansicht\else
% Fallback-Definitionen, falls die .tex-Datei \titel etc. nicht gesetzt hat
\providecommand{\titel}{}
\providecommand{\editorInnen}{}
\providecommand{\dateiname}{\jobname}

\vspace{3cm}

\vfill

\footnotesize
\textsc{Quelle}: \titel. Herausgegeben von {\editorInnen}. In: \emph{Arthur Schnitzler: Briefwechsel mit Autorinnen und Autoren}.
 Digitale Edition, https://schnitzler-briefe.acdh.oeaw.ac.at/{\dateiname}.html (Stand \today)
\fi

\end{document}


