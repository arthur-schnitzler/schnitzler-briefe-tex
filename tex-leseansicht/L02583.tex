%% latex-korrekturansicht-vorspann.tex
%% Vorspann für die Korrekturansicht.
%% Lädt die gemeinsame Datei latex-vorspann.tex mit gesetztem Schalter.

\newif\ifkorrekturansicht
\korrekturansichttrue

\input{../tex-inputs/latex-vorspann}


\section[Arthur Schnitzler an Auguste Hauschner, 23. 1. 1909]{L02583 Arthur Schnitzler an Auguste Hauschner, 23. 1. 1909}
\nopagebreak\mylabel{L02583v}
\rehead{ }\normalsize\beginnumbering\briefempfaengerindex{Hauschner, Auguste@\textsc{Hauschner, Auguste}!zzzSchnitzler, Arthur@\emph{von Arthur Schnitzler}!1909-01-231@{23. 1. 1909}|(be}
\toendnotes[C]{\smallbreak\pagebreak[2]}\Standort{Staatsbibliothek Berlin – Preußischer Kulturbesitz, Handschriftenabteilung, Nachlass Auguste Hauschner.}
\physDesc{Brief, 1 Blatt, 4 Seiten, 1101 Zeichen
\newline{}Handschrift: schwarze Tinte, lateinische Kurrent
\newline{}Hauschner: mit rotem Buntstift eine Unterstreichung unter
                                    »tautologisch«, eventuell, weil die Entzifferung
                                 Probleme bereitete }
\buchAbdrucke{\weitereDrucke{1) \pwindex{Brief an Auguste Hauschner, 23.1.1909]@\emph{[Brief an Auguste Hauschner, 23.1.1909]}|pwk}\pwindex{Briefe an Auguste Hauschner@\emph{Briefe an Auguste Hauschner}|pwk}\emph{Briefe an Auguste Hauschner}. Berlin: \emph{Ernst Rowohlt Verlag} [Ende Oktober 1928, vordatiert auf:]
                           1929, S. 106.} \weitereDrucke{2) Arthur Schnitzler: \emph{Briefe 1875–1912}. Frankfurt am Main: \emph{S. Fischer} 1981, S. 588.} }\toendnotes[C]{\smallbreak}
\pstart
           
\pstart
           {\pb}\textcolor{gray}{\textbf{Dr. Arthur Schnitzler}}\pend
           
\pstart
           \raggedleft{}23. 1. 09\pend
           \pend
           
\pstart{}verehrte Frau, \pend\vspace{0.5em}
\pstart
           ich danke Ihnen sehr, dass Sie mir Ihren schönen \label{K_L02583-1v}\edtext{Artikel\pwindex{Weg ins Freie@\emph{Der Weg ins Freie}|pwv}}{\lemma{\textnormal{\emph{Artikel}}}\Cendnote{\textnormal{Auguste Hauschner\pwindex{Hauschner, Auguste 12.02.1850 – 10.04.1924@\textsc{Hauschner, Auguste} (12.02.1850 – 10.04.1924), \emph{Schriftsteller/Schriftstellerin}|pwk}: \emph{Der Weg ins Freie}\pwindex{Weg ins Freie@\emph{Der Weg ins Freie}|pwk}. In: \emph{Die Hilfe}\pwindex{Hilfe. Zeitschrift fuer Politik, Wirtschaft und geistige Bewegung@\emph{Die Hilfe. Zeitschrift für Politik, Wirtschaft und geistige Bewegung}|pwk}, Jg. 15, Nr. 3, 17. 1. 1909, S. 39–40.
                     Schnitzler urteilte im \emph{Tagebuch}\pwindex{Tagebuch@\emph{Tagebuch}|pwk} am 15. 1. 1909: »Neue Kritikensammlung, von
                        Fischer\pwindex{Fischer, Samuel 24.12.1859 – 15.10.1934@\textsc{Fischer, Samuel} (24.12.1859 – 15.10.1934), \emph{Verleger/Verlegerin}|pw} gesandt, über den Weg. Die
                        Hauschner\pwindex{Hauschner, Auguste 12.02.1850 – 10.04.1924@\textsc{Hauschner, Auguste} (12.02.1850 – 10.04.1924), \emph{Schriftsteller/Schriftstellerin}|pw}, fand endlich in der ›Hilfe\pwindex{Hilfe. Zeitschrift fuer Politik, Wirtschaft und geistige Bewegung@\emph{Die Hilfe. Zeitschrift für Politik, Wirtschaft und geistige Bewegung}|pw}‹ eine Stätte für ihren mir nun erst
                     bekannt werdenden sehr freundlichen Aufsatz.«}}}\label{K_L02583-1} geschickt haben.
               Gar viel wäre darüber zu sagen, wenn es mir nicht so fatal wäre, über meine eignen
               Sachen was niederzuschreiben. Reden kö{\geminationn}t ich schon eher
               drüber, nun vielleicht fügt es mein gutes Glück, dass {\pb}ich Ihnen irgend einmal in der Welt
               begegne. Übrigens, einfacher: we{\geminationn} Sie nach Wien\oindex{Wien@\textbf{Wien}, \emph{A.ADM2}|pw} kommen, lassen Sie michs wissen, gnädige Frau,
               und we{\geminationn} ich nach Berlin\oindex{Berlin@\textbf{Berlin}, \emph{P.PPLC}|pw} komme, darf ich mich wohl auch melden –? Vorher aber noch möcht ich
               Ihnen sagen, daß Sie Unrecht haben Ihren \label{K_L02583-2v}\edtext{Schluss »mislungen«}{\lemma{\textnormal{\emph{Schluss »mislungen«}}}\Cendnote{\textnormal{Siehe Auguste Hauschner an Arthur Schnitzler, 16. 1. 1909.
               }}}\label{K_L02583-2} zu finden – auch ohne Ihren Brief {\pb}wüßt ich sehr gut, was Sie eigentlich
               sagen wollten. Und so viel tief und liebevoll (oder ist das tautologisch?)
               eindringendes in den vorherigen Absätzen. Wie
               viele Leseri{\geminationn}en Ihrer Art denken Sie gibt es wohl? Und
               gar eine, die zugleich Künstlerin ist { }{\dotsfive} jetzt aber ko{\geminationm}t es immer
               näher, – noch drei Zeilen, und ich fange an etwas über mein {\pb}Buch\pwindex{Weg ins Freie. Roman@\emph{Der Weg ins Freie. Roman}|pwv} zu sagen – daher nicht
               mehr als dies: Sie haben mir durch gedrucktes geschriebenes und
               gefühltes herzliche Freude bereitet!\pend
           
\pstart
           Ihr aufrichtig ergebner{\\[\baselineskip]}\spacefill\mbox{Arthur Schnitzler}\pend
           \leftskip=0em{}\selectlanguage{ngerman}\endnumbering\briefempfaengerindex{Hauschner, Auguste@\textsc{Hauschner, Auguste}!zzzSchnitzler, Arthur@\emph{von Arthur Schnitzler}!1909-01-231@{23. 1. 1909}|)be}\mylabel{L02583h}  \normalsize

\doendnotes{C}
\bigskip
\vfill

\clearpage

\footnotesize

\lohead{\textsc{register}}

% Definiere theindex-Environment komplett neu ohne reledmac
\makeatletter
\renewenvironment{theindex}{%
  \section*{\indexname}%
  \setlength{\parindent}{0pt}%
  \setlength{\parskip}{0pt plus 0.3pt}%
  \let\item\@idxitem
}{%
  \clearpage
}
\makeatother

\IfFileExists{\jobname-pw.ind}{\input{\jobname-pw.ind}}{}

\end{document}

      