%% latex-korrekturansicht-vorspann.tex
%% Vorspann für die Korrekturansicht.
%% Lädt die gemeinsame Datei latex-vorspann.tex mit gesetztem Schalter.

\newif\ifkorrekturansicht
\korrekturansichttrue

\input{../tex-inputs/latex-vorspann}


\section[Hugo von Hofmannsthal an Arthur Schnitzler, 19. 3. {[}1909{]}]{L01831 Hugo von Hofmannsthal an Arthur Schnitzler, 19. 3. {[}1909{]}}
\nopagebreak\mylabel{L01831v}
\rehead{ }\normalsize\beginnumbering\briefempfaengerindex{Schnitzler, Arthur@\textsc{Schnitzler, Arthur}!zzzHofmannsthal, Hugo von@\emph{von Hugo von Hofmannsthal}!1909-03-191@{19. 3. {[}1909{]}}|(be}
\toendnotes[C]{\smallbreak\pagebreak[2]}\Standort{CUL, Schnitzler, B 43.}
\physDesc{Brief, 1 Blatt, 3 Seiten, 518 Zeichen
\newline{}Handschrift: schwarze Tinte, deutsche Kurrent
\newline{}Schnitzler: mit Bleistift die Jahreszahl ergänzt: »909« und beschriftet: »Hugo
                                 Hofmannsthal« 
\newline{}Ordnung: 1) mit Bleistift von unbekannter Hand nummeriert: »\strikeout{299}«  2) mit Bleistift von unbekannter Hand nummeriert:
                                    »295«}
\buchAbdrucke{\weitereDrucke{Hugo von Hofmannsthal, Arthur Schnitzler: \emph{Briefwechsel}. Frankfurt am Main: \emph{S. Fischer} 1964, S. 243.} }\toendnotes[C]{\smallbreak}
\pstart
           \raggedleft{}{\pb}R.\oindex{Rodaun@\textbf{Rodaun}, \emph{A.ADM4}|pw}{ }19 III.\pend
           \vspace{0.5em}
\pstart
           lieber, bitte erwähnen Sie das Folgende gegen niemanden, am
               wenigſten gegen Waſſermanns\pwindex{Wassermann, Jakob 10.03.1873 – 01.01.1934@\textsc{Wassermann, Jakob} (10.03.1873 – 01.01.1934), \emph{Schriftsteller/Schriftstellerin}|pw}\pwindex{Wassermann, Julie 05.12.1876 – April 1963@\textsc{Wassermann, Julie} (05.12.1876 – April 1963), \emph{Schriftsteller/Schriftstellerin}|pw}, am
               wenigſten gegen \textsc{Trebitsch}\pwindex{Trebitsch, Siegfried 22.12.1868 – 03.06.1956@\textsc{Trebitsch, Siegfried} (22.12.1868 – 03.06.1956), \emph{Schriftsteller/Schriftstellerin, Übersetzer/Übersetzerin}|pw}, am wenigſten gegen \textsc{Sil Vara}\pwindex{Silberer, Geza 01.12.1876 – 05.04.1938@\textsc{Silberer, Geza} (01.12.1876 – 05.04.1938), \emph{Schriftsteller/Schriftstellerin, Journalist/Journalistin}|pw}, alſo gut. Nämlich: bitte ko{\geminationm}en Sie zur
               Generalprobe von unſerer wohltönenden herzigen Elektra\pwindex{Elektra [op. 58]@\emph{Elektra [op. 58]}|pw} d. h. am Montag um ¾ 11{ }{\pb}pünktlich gehen Sie beim Directionseingang\oindex{Oper@\textbf{Oper}, \emph{Oper (K.OPR)}|pwv} hinein (Kärtnerſtraße\oindex{Kaerntner Strasse@\textbf{Kärntner Straße}, \emph{Straße (K.STR)}|pw}
               ) in den erſten Stock hinauf dort im Bureau des Oberrates Ribitſch\pwindex{Ribitsch, Gabriel 1856? – 25.11.1924@\textsc{Ribitsch, Gabriel} (1856? – 25.11.1924), \emph{Rechnungsprüfer/Rechnungsprüferin}|pw}{ }ſteht Ihr werter und angeſehener Name auf einer
               Liste, worauf man Sie in eine Loge führt. Parkett ist nicht.\pend
           
\pstart
           Ihr lieber{\\[\baselineskip]}\spacefill\mbox{Hugo.}\pend
           \leftskip=0em{}\selectlanguage{ngerman}\endnumbering\briefempfaengerindex{Schnitzler, Arthur@\textsc{Schnitzler, Arthur}!zzzHofmannsthal, Hugo von@\emph{von Hugo von Hofmannsthal}!1909-03-191@{19. 3. {[}1909{]}}|)be}\mylabel{L01831h}  \normalsize

\doendnotes{C}
\bigskip
\vfill

\clearpage

\footnotesize

\lohead{\textsc{register}}

% Definiere theindex-Environment komplett neu ohne reledmac
\makeatletter
\renewenvironment{theindex}{%
  \section*{\indexname}%
  \setlength{\parindent}{0pt}%
  \setlength{\parskip}{0pt plus 0.3pt}%
  \let\item\@idxitem
}{%
  \clearpage
}
\makeatother

\IfFileExists{\jobname-pw.ind}{\input{\jobname-pw.ind}}{}

\end{document}

      