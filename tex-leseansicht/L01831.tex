%% latex-leseansicht-vorspann.tex
%% Vorspann für die Leseansicht.
%% Lädt die gemeinsame Datei latex-vorspann.tex mit nicht gesetztem Schalter.

\newif\ifkorrekturansicht
\korrekturansichtfalse

\input{../tex-inputs/latex-vorspann}


         
         \renewcommand{\erwaehntePersonen}{Personen: Hugo von Hofmannsthal, Gabriel Ribitsch, Geza Silberer, Siegfried Trebitsch, Jakob Wassermann, Julie Wassermann}
         \renewcommand{\erwaehnteOrte}{Orte: Kärntner Straße, Oper, Rodaun, Wien}
         \renewcommand{\erwaehnteWerke}{Werke: Elektra (op. 58)}
               \section[Hugo von Hofmannsthal an Arthur Schnitzler, 19. 3. {[}1909{]}]{ Hugo von Hofmannsthal an Arthur Schnitzler, 19. 3. {[}1909{]}}\nopagebreak\mylabel{v}\rehead{ }\begin{ledgroupsized}[t]{13cm}\normalsize\beginnumbering \toendnotes[C]{\smallbreak\pagebreak[2]} \Standort{CUL, Schnitzler, B 43.}
\physDesc{Brief, 1 Blatt, 3 Seiten, 518 Zeichen
\newline{}Handschrift: schwarze Tinte, deutsche Kurrent
\newline{}Schnitzler: mit Bleistift die Jahreszahl ergänzt: »909« und beschriftet: »Hugo
                                 Hofmannsthal« 
\newline{}Ordnung: 1) mit Bleistift von unbekannter Hand nummeriert: »\strikeout{299}«  2) mit Bleistift von unbekannter Hand nummeriert:
                                    »295«}\buchAbdrucke{\weitereDrucke{Hugo von Hofmannsthal, Arthur Schnitzler: \emph{Briefwechsel}. Hg. Therese Nickl und Heinrich Schnitzler. Frankfurt am Main: \emph{S. Fischer} 1964, S. 243.} }\toendnotes[C]{\smallbreak}\pstart
           \raggedleft{}{\pb}R.\oindex{Rodaun@\textbf{Rodaun}|pw}{ }19 III.\pend
           \pstart
           lieber, bitte erwähnen Sie das Folgende gegen niemanden, am
               wenigſten gegen Waſſermanns\pwindex{Wassermann, Jakob 10.03.1873 – 01.01.1934@\textsc{Wassermann, Jakob} (10.03.1873 – 01.01.1934), \emph{Schriftsteller}|pw}\pwindex{Wassermann, Julie 05.12.1876 – April 1963@\textsc{Wassermann, Julie} (05.12.1876 – April 1963), \emph{Schriftstellerin}|pw}, am
               wenigſten gegen \textsc{Trebitsch}\pwindex{Trebitsch, Siegfried 22.12.1868 – 03.06.1956@\textsc{Trebitsch, Siegfried} (22.12.1868 – 03.06.1956), \emph{Schriftsteller, Übersetzer}|pw}, am wenigſten gegen \textsc{Sil Vara}\pwindex{Silberer, Geza 01.12.1876 – 05.04.1938@\textsc{Silberer, Geza} (01.12.1876 – 05.04.1938), \emph{Schriftsteller, Journalist}|pw}, alſo gut. Nämlich: bitte ko{\geminationm}en Sie zur
               Generalprobe von unſerer wohltönenden herzigen Elektra\pwindex{\textcolor{red}{\textsuperscript{XXXX1 indx}}!Elektra (op. 58)25. 1. 1909@\strich\emph{Elektra (op. 58)} {[}Vertonung, 25. 1. 1909{]}|pw} d. h. am Montag um ¾ 11{ }{\pb}pünktlich gehen Sie beim Directionseingang\oindex{Oper@\textbf{Oper}|pwv} hinein (Kärtnerſtraße\oindex{Kaerntner Strasse@\textbf{Kärntner Straße}|pw}) in den erſten Stock hinauf dort im Bureau des Oberrates Ribitſch\pwindex{Ribitsch, Gabriel 1856? – 25.11.1924@\textsc{Ribitsch, Gabriel} (1856? – 25.11.1924), \emph{Rechnungsprüfer}|pw}{ }ſteht Ihr werter und angeſehener Name auf einer
               Liste, worauf man Sie in eine Loge führt. Parkett ist nicht.\pend
           \pstart
           Ihr lieber{\\[\baselineskip]}\spacefill\mbox{Hugo.}\pend
           \leftskip=0em{}
         
         \endnumbering\mylabel{h}\end{ledgroupsized}  \newcommand{\dateiname}{L01831}\newcommand{\titel}{Hugo von Hofmannsthal an Arthur Schnitzler, 19. 3. [1909]}\newcommand{\editorInnen}{Martin Anton Müller und Gerd-Hermann Susen}%% latex-leseansicht-abspann.tex
%% Abspann für die Leseansicht.
%% Der Schalter \ifkorrekturansicht ist bereits durch den Vorspann gesetzt.

%% latex-abspann.tex
%% Gemeinsamer Abspann für Korrekturansicht und Leseansicht.
%% Setzt den Schalter \ifkorrekturansicht voraus (gesetzt in den
%% einbindenden Dateien latex-korrekturansicht-abspann.tex bzw.
%% latex-leseansicht-abspann.tex).
%% ---------------------------------------------------------------

\normalsize

% Das esempio-Environment wird nur in der Leseansicht benötigt
\ifkorrekturansicht\else
\newenvironment{esempio}[3]%
{
    \vspace{1.5ex}
    \rlap{\underline{#1}}
    \par
    \setlength{\parindent}{0cm}
    \nopagebreak
    \leftskip=#2cm
    \rightskip=#3cm
}
{
    \par
}
\fi

\doendnotes{C}
\bigskip
\vfill

\clearpage

\footnotesize

\ifkorrekturansicht
  \lohead{\textsc{register}}
\fi

% theindex-Environment neu definieren ohne reledmac
\makeatletter
\renewenvironment{theindex}{%
  \ifkorrekturansicht
    \section*{\indexname}%
  \else
    \subsubsection*{Index der erwähnten Entitäten}%
  \fi
  \setlength{\parindent}{0pt}%
  \setlength{\parskip}{0pt plus 0.3pt}%
  \let\item\@idxitem
}{%
  \ifkorrekturansicht\clearpage\fi
}
\makeatother

\IfFileExists{\jobname-pw.ind}{\input{\jobname-pw.ind}}{}

% Quellenangabe nur in der Leseansicht
\ifkorrekturansicht\else
% Fallback-Definitionen, falls die .tex-Datei \titel etc. nicht gesetzt hat
\providecommand{\titel}{}
\providecommand{\editorInnen}{}
\providecommand{\dateiname}{\jobname}

\vspace{3cm}

\vfill

\footnotesize
\textsc{Quelle}: \titel. Herausgegeben von {\editorInnen}. In: \emph{Arthur Schnitzler: Briefwechsel mit Autorinnen und Autoren}.
 Digitale Edition, https://schnitzler-briefe.acdh.oeaw.ac.at/{\dateiname}.html (Stand \today)
\fi

\end{document}


      