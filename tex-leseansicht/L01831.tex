%% latex-leseansicht-vorspann.tex
%% Vorspann für die Leseansicht.
%% Lädt die gemeinsame Datei latex-vorspann.tex mit nicht gesetztem Schalter.

\newif\ifkorrekturansicht
\korrekturansichtfalse

\input{../tex-inputs/latex-vorspann}


\section[Hugo von Hofmannsthal an Arthur Schnitzler, 19. 3. [1909]]{L01831 Hugo von Hofmannsthal an Arthur Schnitzler, 19. 3. [1909]}
\nopagebreak\mylabel{L01831v}
\rehead{ }\normalsize\beginnumbering\briefempfaengerindex{Schnitzler, Arthur@\textsc{Schnitzler, Arthur}!zzzHofmannsthal, Hugo von@\emph{von Hugo von Hofmannsthal}!1909-03-191@{19. 3. [1909]}|(be}
\toendnotes[C]{\smallbreak\pagebreak[2]}
\correspDesc{Versand  durch Hugo von Hofmannsthal am 19. 3. [1909] in Rodaun
\newline{}Erhalt  durch Arthur Schnitzler im Zeitraum [20. 3. 1909
                  – 24. 3. 1909?] in Wien}\toendnotes[C]{\smallbreak}
\Standort{CUL, Schnitzler, B 43.}
\physDesc{Brief, 1 Blatt, 3 Seiten, 518 Zeichen
\newline{}Handschrift: schwarze Tinte, deutsche Kurrent
\newline{}Schnitzler: mit Bleistift die Jahreszahl ergänzt: »909« und beschriftet: »Hugo
                                 Hofmannsthal« 
\newline{}Ordnung: 1) mit Bleistift von unbekannter Hand nummeriert: »\strikeout{299}«  2) mit Bleistift von unbekannter Hand nummeriert:
                                    »295«}
\buchAbdrucke{\weitereDrucke{Hugo von Hofmannsthal, Arthur Schnitzler: \emph{Briefwechsel}. Herausgegeben von Therese Nickl und Heinrich Schnitzler. Frankfurt am Main: \emph{S. Fischer} 1964, S. 243.} }\toendnotes[C]{\smallbreak}
\pstart
           \raggedleft{}{\pb}R.\oindex{Wien@\textbf{Wien}!XXIII., Liesing@\textbf{XXIII., Liesing}!Rodaun@\textbf{Rodaun}, \emph{Region}|pw}{ }19 III.\pend
           \vspace{0.5em}
\pstart
           lieber, bitte erwähnen Sie das Folgende gegen niemanden, am
               wenigſten gegen Waſſermanns\pwindex{Wassermann, Jakob 10.\,3.\,1873 Fürth – 1.\,1.\,1934 Altaussee@\textsc{Wassermann, Jakob} (10.\,3.\,1873 Fürth – 1.\,1.\,1934 Altaussee), \emph{Schriftsteller}|pw}\pwindex{Wassermann, Julie 5.\,12.\,1876 Wien – April 1963 Zürich@\textsc{Wassermann, Julie} (5.\,12.\,1876 Wien – April 1963 Zürich), \emph{Schriftstellerin}|pw}, am
               wenigſten gegen \textsc{Trebitsch}\pwindex{Trebitsch, Siegfried 22.\,12.\,1868 Wien – 3.\,6.\,1956 Zürich@\textsc{Trebitsch, Siegfried} (22.\,12.\,1868 Wien – 3.\,6.\,1956 Zürich), \emph{Schriftsteller, Übersetzer}|pw}, am wenigſten gegen \textsc{Sil Vara}\pwindex{Silberer, Geza 1.\,12.\,1876 Vršac – 5.\,4.\,1938 Wien@\textsc{Silberer, Geza} (1.\,12.\,1876 Vršac – 5.\,4.\,1938 Wien), \emph{Schriftsteller, Journalist}|pw}, alſo gut. Nämlich: bitte ko{\geminationm}en Sie zur
               Generalprobe von unſerer wohltönenden herzigen Elektra\pwindex{Hofmannsthal, Hugo von 1.\,2.\,1874 Wien – 15.\,7.\,1929 Rodaun@\textsc{Hofmannsthal, Hugo von} (1.\,2.\,1874 Wien – 15.\,7.\,1929 Rodaun), \emph{Schriftsteller}!Elektra [op. 58]@\strich\emph{Elektra [op. 58]}|pw} d. h. am Montag um ¾ 11{ }{\pb}pünktlich gehen Sie beim Directionseingang\oindex{Wien@\textbf{Wien}!I., Innere Stadt@\textbf{I., Innere Stadt}!Oper@\textbf{Oper}, \emph{Oper}|pwv} hinein (Kärtnerſtraße\oindex{Wien@\textbf{Wien}!I., Innere Stadt@\textbf{I., Innere Stadt}!Kärntner Straße@\textbf{Kärntner Straße}, \emph{Straße}|pw}
               ) in den erſten Stock hinauf dort im Bureau des Oberrates Ribitſch\pwindex{Ribitsch, Gabriel 1856? – 25.\,11.\,1924 Wien@\textsc{Ribitsch, Gabriel} (1856? – 25.\,11.\,1924 Wien), \emph{Rechnungsprüfer}|pw}{ }ſteht Ihr werter und angeſehener Name auf einer
               Liste, worauf man Sie in eine Loge führt. Parkett ist nicht.\pend
           
\pstart
           Ihr lieber{\\[\baselineskip]}\spacefill\mbox{Hugo.}\pend
           \leftskip=0em{}\selectlanguage{ngerman}\endnumbering\briefempfaengerindex{Schnitzler, Arthur@\textsc{Schnitzler, Arthur}!zzzHofmannsthal, Hugo von@\emph{von Hugo von Hofmannsthal}!1909-03-191@{19. 3. [1909]}|)be}\mylabel{L01831h}  \newcommand{\dateiname}{L01831}\newcommand{\titel}{Hugo von Hofmannsthal an Arthur Schnitzler, 19. 3. [1909]}\newcommand{\editorInnen}{Martin Anton Müller und Gerd-Hermann Susen}%% latex-leseansicht-abspann.tex
%% Abspann für die Leseansicht.
%% Der Schalter \ifkorrekturansicht ist bereits durch den Vorspann gesetzt.

%% latex-abspann.tex
%% Gemeinsamer Abspann für Korrekturansicht und Leseansicht.
%% Setzt den Schalter \ifkorrekturansicht voraus (gesetzt in den
%% einbindenden Dateien latex-korrekturansicht-abspann.tex bzw.
%% latex-leseansicht-abspann.tex).
%% ---------------------------------------------------------------

\normalsize

% Das esempio-Environment wird nur in der Leseansicht benötigt
\ifkorrekturansicht\else
\newenvironment{esempio}[3]%
{
    \vspace{1.5ex}
    \rlap{\underline{#1}}
    \par
    \setlength{\parindent}{0cm}
    \nopagebreak
    \leftskip=#2cm
    \rightskip=#3cm
}
{
    \par
}
\fi

\doendnotes{C}
\bigskip
\vfill

\clearpage

\footnotesize

\ifkorrekturansicht
  \lohead{\textsc{register}}
\fi

% theindex-Environment neu definieren ohne reledmac
\makeatletter
\renewenvironment{theindex}{%
  \ifkorrekturansicht
    \section*{\indexname}%
  \else
    \subsubsection*{Index der erwähnten Entitäten}%
  \fi
  \setlength{\parindent}{0pt}%
  \setlength{\parskip}{0pt plus 0.3pt}%
  \let\item\@idxitem
}{%
  \ifkorrekturansicht\clearpage\fi
}
\makeatother

\IfFileExists{\jobname-pw.ind}{\input{\jobname-pw.ind}}{}

% Quellenangabe nur in der Leseansicht
\ifkorrekturansicht\else
% Fallback-Definitionen, falls die .tex-Datei \titel etc. nicht gesetzt hat
\providecommand{\titel}{}
\providecommand{\editorInnen}{}
\providecommand{\dateiname}{\jobname}

\vspace{3cm}

\vfill

\footnotesize
\textsc{Quelle}: \titel. Herausgegeben von {\editorInnen}. In: \emph{Arthur Schnitzler: Briefwechsel mit Autorinnen und Autoren}.
 Digitale Edition, https://schnitzler-briefe.acdh.oeaw.ac.at/{\dateiname}.html (Stand \today)
\fi

\end{document}


