%% latex-leseansicht-vorspann.tex
%% Vorspann für die Leseansicht.
%% Lädt die gemeinsame Datei latex-vorspann.tex mit nicht gesetztem Schalter.

\newif\ifkorrekturansicht
\korrekturansichtfalse

\input{../tex-inputs/latex-vorspann}


\section[Richard Beer-Hofmann an Arthur Schnitzler, 4. 2. 1898]{L00772 Richard Beer-Hofmann an Arthur Schnitzler, 4. 2. 1898}
\nopagebreak\mylabel{L00772v}
\rehead{ }\normalsize\beginnumbering\briefempfaengerindex{Schnitzler, Arthur@\textsc{Schnitzler, Arthur}!zzzBeer-Hofmann, Richard@\emph{von Richard Beer-Hofmann}!1898-02-041@{4. 2. 1898}|(be}
\toendnotes[C]{\smallbreak\pagebreak[2]}
\correspDesc{Versand  durch Richard Beer-Hofmann am 4. 2. 1898 in Wien
\newline{}Erhalt  durch Arthur Schnitzler im Zeitraum [4. 2. 1898
                  – 8. 2. 1898?] \textbf{Ort fehlend} }\toendnotes[C]{\smallbreak}
\Standort{CUL, Schnitzler, B 8.}
\physDesc{Briefkarte, 122 Zeichen
\newline{}Handschrift: blauer Buntstift, lateinische Kurrent
\newline{}Ordnung: mit Bleistift von unbekannter Hand nummeriert:
                                    »109« }\toendnotes[C]{\smallbreak}
\pstart
           \raggedleft{}{\pb}4/II 1898\pend
           \vspace{0.5em}
\pstart
           Lieber Arthur,  also heute Abends im Caffee Royal\oindex{Wien@\textbf{Wien}!I., Innere Stadt@\textbf{I., Innere Stadt}!Café Scheuchenstuel@\textbf{Café Scheuchenstuel}, \emph{Kaffeehaus}|pw} (\label{K_L00772-1v}\edtext{Scheuchen{\pb}stuhl\oindex{Wien@\textbf{Wien}!I., Innere Stadt@\textbf{I., Innere Stadt}!Café Scheuchenstuel@\textbf{Café Scheuchenstuel}, \emph{Kaffeehaus}|pw}}{\lemma{\textnormal{\emph{Scheuchenstuhl}}}\Cendnote{\textnormal{Richtig hieß es Café Scheuchenstuel\oindex{Wien@\textbf{Wien}!I., Innere Stadt@\textbf{I., Innere Stadt}!Café Scheuchenstuel@\textbf{Café Scheuchenstuel}, \emph{Kaffeehaus}|pwk}. Ein Namenswechsel zu Café Royal\oindex{Wien@\textbf{Wien}!I., Innere Stadt@\textbf{I., Innere Stadt}!Café Scheuchenstuel@\textbf{Café Scheuchenstuel}, \emph{Kaffeehaus}|pwk} lässt sich nicht verifizieren.}}}\label{K_L00772-1}) Ecke der
                  Schuler\oindex{Wien@\textbf{Wien}!I., Innere Stadt@\textbf{I., Innere Stadt}!Schulerstraße@\textbf{Schulerstraße}, \emph{Straße}|pw} u. Stroblgasse\oindex{Strobelgasse@\textbf{Strobelgasse}, \emph{Straße}|pw}.\pend
           
\pstart
           \uline{Von Herzen Ihr}{\\[\baselineskip]}\spacefill\mbox{Richard}\pend
           \leftskip=0em{}\selectlanguage{ngerman}\endnumbering\briefempfaengerindex{Schnitzler, Arthur@\textsc{Schnitzler, Arthur}!zzzBeer-Hofmann, Richard@\emph{von Richard Beer-Hofmann}!1898-02-041@{4. 2. 1898}|)be}\mylabel{L00772h}  \newcommand{\dateiname}{L00772}\newcommand{\titel}{Richard Beer-Hofmann an Arthur Schnitzler, 4. 2. 1898}\newcommand{\editorInnen}{Martin Anton Müller und Gerd-Hermann Susen}%% latex-leseansicht-abspann.tex
%% Abspann für die Leseansicht.
%% Der Schalter \ifkorrekturansicht ist bereits durch den Vorspann gesetzt.

%% latex-abspann.tex
%% Gemeinsamer Abspann für Korrekturansicht und Leseansicht.
%% Setzt den Schalter \ifkorrekturansicht voraus (gesetzt in den
%% einbindenden Dateien latex-korrekturansicht-abspann.tex bzw.
%% latex-leseansicht-abspann.tex).
%% ---------------------------------------------------------------

\normalsize

% Das esempio-Environment wird nur in der Leseansicht benötigt
\ifkorrekturansicht\else
\newenvironment{esempio}[3]%
{
    \vspace{1.5ex}
    \rlap{\underline{#1}}
    \par
    \setlength{\parindent}{0cm}
    \nopagebreak
    \leftskip=#2cm
    \rightskip=#3cm
}
{
    \par
}
\fi

\doendnotes{C}
\bigskip
\vfill

\clearpage

\footnotesize

\ifkorrekturansicht
  \lohead{\textsc{register}}
\fi

% theindex-Environment neu definieren ohne reledmac
\makeatletter
\renewenvironment{theindex}{%
  \ifkorrekturansicht
    \section*{\indexname}%
  \else
    \subsubsection*{Index der erwähnten Entitäten}%
  \fi
  \setlength{\parindent}{0pt}%
  \setlength{\parskip}{0pt plus 0.3pt}%
  \let\item\@idxitem
}{%
  \ifkorrekturansicht\clearpage\fi
}
\makeatother

\IfFileExists{\jobname-pw.ind}{\input{\jobname-pw.ind}}{}

% Quellenangabe nur in der Leseansicht
\ifkorrekturansicht\else
% Fallback-Definitionen, falls die .tex-Datei \titel etc. nicht gesetzt hat
\providecommand{\titel}{}
\providecommand{\editorInnen}{}
\providecommand{\dateiname}{\jobname}

\vspace{3cm}

\vfill

\footnotesize
\textsc{Quelle}: \titel. Herausgegeben von {\editorInnen}. In: \emph{Arthur Schnitzler: Briefwechsel mit Autorinnen und Autoren}.
 Digitale Edition, https://schnitzler-briefe.acdh.oeaw.ac.at/{\dateiname}.html (Stand \today)
\fi

\end{document}


