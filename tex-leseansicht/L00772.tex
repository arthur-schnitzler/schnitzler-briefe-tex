%% latex-korrekturansicht-vorspann.tex
%% Vorspann für die Korrekturansicht.
%% Lädt die gemeinsame Datei latex-vorspann.tex mit gesetztem Schalter.

\newif\ifkorrekturansicht
\korrekturansichttrue

\input{../tex-inputs/latex-vorspann}


\section[Richard Beer-Hofmann an Arthur Schnitzler, 4. 2. 1898]{L00772 Richard Beer-Hofmann an Arthur Schnitzler, 4. 2. 1898}
\nopagebreak\mylabel{L00772v}
\rehead{ }\normalsize\beginnumbering\briefempfaengerindex{Schnitzler, Arthur@\textsc{Schnitzler, Arthur}!zzzBeer-Hofmann, Richard@\emph{von Richard Beer-Hofmann}!1898-02-041@{4. 2. 1898}|(be}
\toendnotes[C]{\smallbreak\pagebreak[2]}\Standort{CUL, Schnitzler, B 8.}
\physDesc{Briefkarte, 122 Zeichen
\newline{}Handschrift: blauer Buntstift, lateinische Kurrent
\newline{}Ordnung: mit Bleistift von unbekannter Hand nummeriert:
                                    »109« }\toendnotes[C]{\smallbreak}
\pstart
           \raggedleft{}{\pb}4/II 1898\pend
           \vspace{0.5em}
\pstart
           Lieber Arthur,  also heute Abends im Caffee Royal\oindex{Cafe Scheuchenstuel@\textbf{Café Scheuchenstuel}, \emph{Kaffeehaus (K.KAF)}|pw} (\label{K_L00772-1v}\edtext{Scheuchen{\pb}stuhl\oindex{Cafe Scheuchenstuel@\textbf{Café Scheuchenstuel}, \emph{Kaffeehaus (K.KAF)}|pw}}{\lemma{\textnormal{\emph{Scheuchenstuhl}}}\Cendnote{\textnormal{Richtig hieß es Café Scheuchenstuel\oindex{Cafe Scheuchenstuel@\textbf{Café Scheuchenstuel}, \emph{Kaffeehaus (K.KAF)}|pwk}. Ein Namenswechsel zu Café Royal\oindex{Cafe Scheuchenstuel@\textbf{Café Scheuchenstuel}, \emph{Kaffeehaus (K.KAF)}|pwk} lässt sich nicht verifizieren.}}}\label{K_L00772-1}) Ecke der
                  Schuler\oindex{Schulerstrasse@\textbf{Schulerstraße}, \emph{Straße (K.STR)}|pw} u. Stroblgasse\oindex{Strobelgasse@\textbf{Strobelgasse}, \emph{Straße (K.STR)}|pw}.\pend
           
\pstart
           \uline{Von Herzen Ihr}{\\[\baselineskip]}\spacefill\mbox{Richard}\pend
           \leftskip=0em{}\selectlanguage{ngerman}\endnumbering\briefempfaengerindex{Schnitzler, Arthur@\textsc{Schnitzler, Arthur}!zzzBeer-Hofmann, Richard@\emph{von Richard Beer-Hofmann}!1898-02-041@{4. 2. 1898}|)be}\mylabel{L00772h}  \normalsize

\doendnotes{C}
\bigskip
\vfill

\clearpage

\footnotesize

\lohead{\textsc{register}}

% Definiere theindex-Environment komplett neu ohne reledmac
\makeatletter
\renewenvironment{theindex}{%
  \section*{\indexname}%
  \setlength{\parindent}{0pt}%
  \setlength{\parskip}{0pt plus 0.3pt}%
  \let\item\@idxitem
}{%
  \clearpage
}
\makeatother

\IfFileExists{\jobname-pw.ind}{\input{\jobname-pw.ind}}{}

\end{document}

      