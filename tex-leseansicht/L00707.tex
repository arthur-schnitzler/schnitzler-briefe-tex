%% latex-korrekturansicht-vorspann.tex
%% Vorspann für die Korrekturansicht.
%% Lädt die gemeinsame Datei latex-vorspann.tex mit gesetztem Schalter.

\newif\ifkorrekturansicht
\korrekturansichttrue

\input{../tex-inputs/latex-vorspann}


\section[Arthur Schnitzler an Richard Beer-Hofmann, 20. 7. 1897]{L00707 Arthur Schnitzler an Richard Beer-Hofmann, 20. 7. 1897}
\nopagebreak\mylabel{L00707v}
\rehead{ }\normalsize\beginnumbering\briefempfaengerindex{Beer-Hofmann, Richard@\textsc{Beer-Hofmann, Richard}!zzzSchnitzler, Arthur@\emph{von Arthur Schnitzler}!1897-07-201@{20. 7. 1897}|(be}
\toendnotes[C]{\smallbreak\pagebreak[2]}\Standort{YCGL, MSS 31.}
\physDesc{Visitenkarte, 528 Zeichen
\newline{}Handschrift: Bleistift, deutsche Kurrent}
\buchAbdrucke{\weitereDrucke{Arthur Schnitzler, Richard Beer-Hofmann: \emph{Briefwechsel 1891–1931}. Wien, Zürich: \emph{Europaverlag} 1992, S. 111.} }\toendnotes[C]{\smallbreak}
\pstart{}{\pb}\label{T_L00707-1v}\edtext{Lieber Richard.}{\lemma{\textnormal{\emph{Lieber Richard.}}}\Cendnote{\textnormal{der gesamte Text ignoriert den Vordruck
                     und ist quer zu dessen Ausrichtung verfasst}}}\label{T_L00707-1}\pend\vspace{0.5em}
\pstart
           1.) Ich fahr heut 4 Uhr{ }Hallſtadt\oindex{Hallstatt@\textbf{Hallstatt}, \emph{P.PPL}|pw}{ }\textsc{Loebs}\pwindex{Loeb, Louis 29.06.1842 – 06.06.1921@\textsc{Loeb, Louis} (29.06.1842 – 06.06.1921), \emph{Bankier/Bankierin}|pw}\pwindex{Loeb, Regina 1850 – 5.2.1918@\textsc{Loeb, Regina} (1850 – 5.2.1918)|pw} (die mit der Bahn).\pend
           
\pstart
           2.) Hugo\pwindex{Hofmannsthal, Hugo von 1874-02-01 – 1929-07-15@\textsc{Hofmannsthal, Hugo von} (1874-02-01 – 1929-07-15), \emph{Schriftsteller/Schriftstellerin}|pw} a) aergert ſich, dſs Sie ihm nicht
               ſchreiben\pend
           
\pstart
           b) ka{\geminationn} nicht aus der \textsc{Fusch}\oindex{Fusch an der Grossglocknerstrasse@\textbf{Fusch an der Großglocknerstraße}, \emph{P.PPL}|pw} fort.\pend
           
\pstart
           (Was unſere Partie hoffent. nicht hindert)\pend
           
\pstart
           3.) In Gmunden\oindex{Gmunden@\textbf{Gmunden}, \emph{P.PPL}|pw}{ }ſoll 22. (übermorgen) \uline{Freiwild}\pwindex{Freiwild. Schauspiel in 3 Akten@\emph{Freiwild. Schauspiel in 3 Akten}|pw}{ }ſein (\label{K_L00707-1v}\edtext{Fremdenblatt\pwindex{Fremden-Blatt@\emph{Fremden-Blatt}|pw}\pwindex{Man schreibt uns aus Gmunden@\emph{Man schreibt uns aus Gmunden}|pwv}}{\lemma{\textnormal{\emph{Fremdenblatt}}}\Cendnote{\textnormal{»– Man schreibt uns aus \so{Gmunden}\oindex{Gmunden@\textbf{Gmunden}, \emph{P.PPL}|pw}: Das hiesige Saisontheater sieht einer interessanten Première entgegen.
                        Arthur \so{Schnitzler}’s ›\so{Freiwild}\pwindex{Freiwild. Schauspiel in 3 Akten@\emph{Freiwild. Schauspiel in 3 Akten}|pw}« gelangt hier Donnerstag den 22. d., von Direktor \so{Cavar}\pwindex{Cavar, Alfred 02.12.1859 – 15.09.1920@\textsc{Cavar, Alfred} (02.12.1859 – 15.09.1920), \emph{Theaterleiter/Theaterleiterin, Schauspieler/Schauspielerin, Theaterdirektor/Theaterdirektorin}|pw} inszenirt, zum erstenmale (in Oesterreich\oindex{Oesterreich@\textbf{Österreich}, \emph{A.PCLI}|pw}) zur Aufführung, mit jenen Einschränkungen natürlich,
                     welche die Zensur für nothwendig erachtet hat. In der Novität sind die besten
                     Kräfte beschäftigt, über welche das hiesige Theater verfügt, u. A. die Naive
                     Fräulein \so{Großmüller}\pwindex{Grossmueller, Frieda *~7.1.1880@\textsc{Grossmüller, Frieda} (*~7.1.1880), \emph{Schauspieler/Schauspielerin}|pw}, welche für die nächste Saison an das Deutsche Volkstheater\orgindex{Volkstheater@Volkstheater|pw} engagirt ist, und Herr Alexander \so{Rottmann}\pwindex{Rottmann, Alexander 03.07.1869 – 26.09.1916@\textsc{Rottmann, Alexander} (03.07.1869 – 26.09.1916), \emph{Schauspieler/Schauspielerin}|pw}, der in einer Aufführung von Ohnet\pwindex{Ohnet, Georges 03.04.1848 – 05.05.1918@\textsc{Ohnet, Georges} (03.04.1848 – 05.05.1918), \emph{Schriftsteller/Schriftstellerin}|pw}’s ›Hüttenbesitzer\pwindex{Huettenbesitzer@\emph{Der Hüttenbesitzer}|pw}‹ durch die
                     diskrete Anwendung seiner schönen Mittel und die Natürlichkeit seiner
                     Darstellung des Philippe Derblay einen vollen Erfolg erzielt hat.« (\emph{Fremden-Blatt}\pwindex{Fremden-Blatt@\emph{Fremden-Blatt}|pwk}, Jg. 51, Nr. 198,
                        19. 7. 1897, Abend-Blatt, S. 6.)}}}\label{K_L00707-1}) mit cenſurellen
               Aenderungen. Ich hab an \textsc{Cavar}\pwindex{Cavar, Alfred 02.12.1859 – 15.09.1920@\textsc{Cavar, Alfred} (02.12.1859 – 15.09.1920), \emph{Theaterleiter/Theaterleiterin, Schauspieler/Schauspielerin, Theaterdirektor/Theaterdirektorin}|pw} telegrafirt, mir {\pb}ſofort die Aenderg
               mitzutheilen. Geſindel, mi\textcolor{gray}{ch} nicht vorher zu verſtändg. (Kämen Sie
                     Do{\geminationn}erſtg mit mir hinüber?)\pend
           
\pstart
           4.) Schaun Sie nach dem Nachtmahl zu mir herauf oder laſſen mir ſagen, wo Sie
               ſind.\pend
           
\pstart
           Herzl Gruß{\\[\baselineskip]}Ihr \spacefill\mbox{A.}\pend
           \leftskip=0em{}
\pstart
           \noindent{}\centering{}\textcolor{gray}{\textbf{D\textsuperscript{r} Arthur Schnitzler}}\pend
           
\pstart
           \raggedleft{}Wien\oindex{Wien@\textbf{Wien}, \emph{A.ADM2}|pw}\pend
           \selectlanguage{ngerman}\endnumbering\briefempfaengerindex{Beer-Hofmann, Richard@\textsc{Beer-Hofmann, Richard}!zzzSchnitzler, Arthur@\emph{von Arthur Schnitzler}!1897-07-201@{20. 7. 1897}|)be}\mylabel{L00707h}  \normalsize

\doendnotes{C}
\bigskip
\vfill

\clearpage

\footnotesize

\lohead{\textsc{register}}

% Definiere theindex-Environment komplett neu ohne reledmac
\makeatletter
\renewenvironment{theindex}{%
  \section*{\indexname}%
  \setlength{\parindent}{0pt}%
  \setlength{\parskip}{0pt plus 0.3pt}%
  \let\item\@idxitem
}{%
  \clearpage
}
\makeatother

\IfFileExists{\jobname-pw.ind}{\input{\jobname-pw.ind}}{}

\end{document}

      