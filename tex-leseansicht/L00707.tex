%% latex-leseansicht-vorspann.tex
%% Vorspann für die Leseansicht.
%% Lädt die gemeinsame Datei latex-vorspann.tex mit nicht gesetztem Schalter.

\newif\ifkorrekturansicht
\korrekturansichtfalse

\input{../tex-inputs/latex-vorspann}


         
         \newcommand{\erwaehntePersonen}{Personen: }
         \newcommand{\erwaehnteInstitutionen}{}
         \newcommand{\erwaehnteOrte}{}
         \newcommand{\erwaehnteWerke}{
               \section[Arthur Schnitzler an Richard Beer-Hofmann, 20. 7. 1897]{ Arthur Schnitzler an Richard Beer-Hofmann, 20. 7. 1897}\nopagebreak\mylabel{v}\rehead{ }\begin{ledgroupsized}[t]{13cm}\normalsize\beginnumbering \toendnotes[C]{\smallbreak\pagebreak[2]} \Standort{YCGL, MSS 31.}
\physDesc{Visitenkarte
\newline{}Handschrift: Bleistift, deutsche Kurrent}\buchAbdrucke{\weitereDrucke{Arthur Schnitzler, Richard Beer-Hofmann: \emph{Briefwechsel 1891–1931}. Hg. Konstanze Fliedl. Wien, Zürich: \emph{Europaverlag} 1992, S. 111.} }\toendnotes[C]{\smallbreak}\pstart{}{\pb}\label{T_L00707-1v}\edtext{Lieber Richard.}{\lemma{\textnormal{\emph{Lieber Richard.}}}\Cendnote{\textnormal{der gesamte Text ignoriert den Vordruck und ist quer zu dessen Ausrichtung verfasst}}}\label{T_L00707-1h}\pend\pstart
           1.) Ich fahr heut 4 Uhr{ }Hallſtadt\oindex{XXXX Ortsangabe fehlt|pw}{ }\textsc{Loebs}\pwindex{\textcolor{red}{\textsuperscript{XXXX1 indx}}|pw}\pwindex{\textcolor{red}{\textsuperscript{XXXX1 indx}}|pw} (die mit der Bahn).\pend
           \pstart
           2.) Hugo\pwindex{\textcolor{red}{\textsuperscript{XXXX1 indx}}|pw} a) aergert ſich, dſs Sie ihm nicht
               ſchreiben\pend
           \pstart
           b) ka{\geminationn} nicht aus der \textsc{Fusch}\oindex{XXXX Ortsangabe fehlt|pw} fort.\pend
           \pstart
           (Was unſere Partie hoffent. nicht hindert)\pend
           \pstart
           3.) In Gmunden\oindex{XXXX Ortsangabe fehlt|pw}{ }ſoll 22. (übermorgen) \uline{Freiwild}\textcolor{red}{\textsuperscript{XXXX indx}}{ }ſein (\label{K_L00707_1v}\edtext{Fremdenblatt\textcolor{red}{\textsuperscript{XXXX indx}}\textcolor{red}{\textsuperscript{XXXX indx}}}{\lemma{\textnormal{\emph{Fremdenblatt}}}\Cendnote{\textnormal{»– Man schreibt uns aus \so{Gmunden}\oindex{XXXX Ortsangabe fehlt|pw}: Das hiesige Saisontheater sieht einer interessanten Première entgegen.
                        Arthur \so{Schnitzler}\pwindex{\textcolor{red}{\textsuperscript{XXXX1 indx}}|pw}’s ›\so{Freiwild}\textcolor{red}{\textsuperscript{XXXX indx}}« gelangt hier Donnerstag den 22. d., von Direktor \so{Cavar}\pwindex{\textcolor{red}{\textsuperscript{XXXX1 indx}}|pw} inszenirt, zum erstenmale (in Oesterreich\oindex{XXXX Ortsangabe fehlt|pw}) zur Aufführung, mit jenen Einschränkungen natürlich,
                     welche die Zensur für nothwendig erachtet hat. In der Novität sind die besten
                     Kräfte beschäftigt, über welche das hiesige Theater verfügt, u. A. die Naive
                     Fräulein \so{Großmüller}\pwindex{\textcolor{red}{\textsuperscript{XXXX1 indx}}|pw}, welche für die nächste Saison an das Deutsche VolkstheaterXXXX ORGangabe fehlt engagirt ist, und Herr Alexander \so{Rottmann}\pwindex{\textcolor{red}{\textsuperscript{XXXX1 indx}}|pw}, der in einer Aufführung von Ohnet\pwindex{\textcolor{red}{\textsuperscript{XXXX1 indx}}|pw}’s
                        ›Hüttenbesitzer\textcolor{red}{\textsuperscript{XXXX indx}}‹ durch die diskrete
                     Anwendung seiner schönen Mittel und die Natürlichkeit seiner Darstellung des
                     Philippe Derblay einen vollen Erfolg erzielt hat.« (\emph{Fremden-Blatt}\textcolor{red}{\textsuperscript{XXXX indx}}, Jg. 51, Nr. 198,
                        19. 7. 1897, Abend-Blatt, S. 6)}}}\label{K_L00707_1h}) mit cenſurellen
               Aenderungen. Ich hab an \textsc{Cavar}\pwindex{\textcolor{red}{\textsuperscript{XXXX1 indx}}|pw} telegrafirt, mir {\pb}ſofort die Aenderg
               mitzutheilen. Geſindel, mi\textcolor{gray}{ch} nicht vorher zu verſtändg. (Kämen Sie
                     Do{\geminationn}erſtg mit mir hinüber?)\pend
           \pstart
           4.) Schaun Sie nach dem Nachtmahl zu mir herauf oder laſſen mir ſagen, wo Sie
               ſind.\pend
           \pstart
           Herzl Gruß{\\[\baselineskip]}Ihr \spacefill\mbox{A.}\pend
           \leftskip=0em{}\pstart
           \noindent{}\centering{}\textcolor{gray}{\textbf{D\textsuperscript{r} Arthur Schnitzler}}\pend
           \pstart
           \noindent{}\raggedleft{}Wien\oindex{XXXX Ortsangabe fehlt|pw}\pend
           
         
         \endnumbering\mylabel{h}\end{ledgroupsized}  \newcommand{\dateiname}{L00707}\newcommand{\titel}{Arthur Schnitzler an Richard Beer-Hofmann, 20. 7. 1897}\newcommand{\editorInnen}{Martin Anton Müller und Gerd-Hermann Susen}%% latex-leseansicht-abspann.tex
%% Abspann für die Leseansicht.
%% Der Schalter \ifkorrekturansicht ist bereits durch den Vorspann gesetzt.

%% latex-abspann.tex
%% Gemeinsamer Abspann für Korrekturansicht und Leseansicht.
%% Setzt den Schalter \ifkorrekturansicht voraus (gesetzt in den
%% einbindenden Dateien latex-korrekturansicht-abspann.tex bzw.
%% latex-leseansicht-abspann.tex).
%% ---------------------------------------------------------------

\normalsize

% Das esempio-Environment wird nur in der Leseansicht benötigt
\ifkorrekturansicht\else
\newenvironment{esempio}[3]%
{
    \vspace{1.5ex}
    \rlap{\underline{#1}}
    \par
    \setlength{\parindent}{0cm}
    \nopagebreak
    \leftskip=#2cm
    \rightskip=#3cm
}
{
    \par
}
\fi

\doendnotes{C}
\bigskip
\vfill

\clearpage

\footnotesize

\ifkorrekturansicht
  \lohead{\textsc{register}}
\fi

% theindex-Environment neu definieren ohne reledmac
\makeatletter
\renewenvironment{theindex}{%
  \ifkorrekturansicht
    \section*{\indexname}%
  \else
    \subsubsection*{Index der erwähnten Entitäten}%
  \fi
  \setlength{\parindent}{0pt}%
  \setlength{\parskip}{0pt plus 0.3pt}%
  \let\item\@idxitem
}{%
  \ifkorrekturansicht\clearpage\fi
}
\makeatother

\IfFileExists{\jobname-pw.ind}{\input{\jobname-pw.ind}}{}

% Quellenangabe nur in der Leseansicht
\ifkorrekturansicht\else
% Fallback-Definitionen, falls die .tex-Datei \titel etc. nicht gesetzt hat
\providecommand{\titel}{}
\providecommand{\editorInnen}{}
\providecommand{\dateiname}{\jobname}

\vspace{3cm}

\vfill

\footnotesize
\textsc{Quelle}: \titel. Herausgegeben von {\editorInnen}. In: \emph{Arthur Schnitzler: Briefwechsel mit Autorinnen und Autoren}.
 Digitale Edition, https://schnitzler-briefe.acdh.oeaw.ac.at/{\dateiname}.html (Stand \today)
\fi

\end{document}


      