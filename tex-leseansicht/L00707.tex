%% latex-leseansicht-vorspann.tex
%% Vorspann für die Leseansicht.
%% Lädt die gemeinsame Datei latex-vorspann.tex mit nicht gesetztem Schalter.

\newif\ifkorrekturansicht
\korrekturansichtfalse

\input{../tex-inputs/latex-vorspann}

\begin{center}
            \textcolor{red}{ENTWURF. ENTZIFFERUNG NOCH NICHT KORREKTURGELESEN}
                      \end{center}
            
               \section[Arthur Schnitzler an Richard Beer-Hofmann, 20. 7. 1897]{ Arthur Schnitzler an Richard Beer-Hofmann, 20. 7. 1897}\nopagebreak\mylabel{v}\rehead{ }\begin{ledgroupsized}[t]{13cm}\normalsize\beginnumbering\briefempfaengerindex{Beer-Hofmann, Richard@\textsc{Beer-Hofmann, Richard}!zzzSchnitzler, Arthur@\emph{von Arthur Schnitzler}!1897-07-201@{20. 7. 1897}|(be} \toendnotes[C]{\smallbreak\pagebreak[2]} \Standort{YCGL, MSS 31.}
\physDesc{Visitenkarte
\newline{}Handschrift: Bleistift, deutsche Kurrent}\buchAbdrucke{\weitereDrucke{Arthur Schnitzler, Richard Beer-Hofmann: \emph{Briefwechsel 1891–1931}. Hg. Konstanze Fliedl. Wien, Zürich: \emph{Europaverlag} 1992, S. 111.} }\toendnotes[C]{\smallbreak}\pstart{}{\pb}\label{T_L00707-1v}\edtext{Lieber Richard.}{\lemma{\textnormal{\emph{Lieber Richard.}}}\Cendnote{\textnormal{der gesamte Text ignoriert den Vordruck und ist quer zu dessen Ausrichtung verfasst}}}\label{T_L00707-1h}\pend\pstart
           1.) Ich fahr heut 4 Uhr{ }Hallſtadt\oindex{Hallstatt@\textbf{Hallstatt}|pw}{ }\textsc{Loebs}\pwindex{Loeb, Louis 29.06.1842 – 06.06.1921@\textsc{Loeb, Louis} (29.06.1842 – 06.06.1921), \emph{Bankier}|pw}\pwindex{Loeb, Regina 1850 – 5.2.1918@\textsc{Loeb, Regina} (1850 – 5.2.1918)|pw} (die mit der Bahn).\pend
           \pstart
           2.) Hugo\pwindex{Hofmannsthal, Hugo von 01.02.1874 – 15.07.1929@\textsc{Hofmannsthal, Hugo von} (01.02.1874 – 15.07.1929), \emph{Schriftsteller}|pw} a) aergert ſich, dſs Sie ihm nicht
               ſchreiben\pend
           \pstart
           b) ka{\geminationn} nicht aus der \textsc{Fusch}\oindex{Fusch an der Grossglocknerstrasse@\textbf{Fusch an der Großglocknerstraße}|pw} fort.\pend
           \pstart
           (Was unſere Partie hoffent. nicht hindert)\pend
           \pstart
           3.) In Gmunden\oindex{Gmunden@\textbf{Gmunden}|pw}{ }ſoll 22. (übermorgen) \uline{Freiwild}\pwindex{Schnitzler, Arthur 15.05.1862 – 21.10.1931@\textsc{Schnitzler, Arthur} (15.05.1862 – 21.10.1931), \emph{Schriftsteller, Mediziner}!Freiwild. Schauspiel in 3 Akten1896@\strich\emph{Freiwild. Schauspiel in 3 Akten} {[}1896{]}|pw}{ }ſein (\label{K_L00707_1v}\edtext{Fremdenblatt\pwindex{Fremden-Blatt1.7.1847 – 22.3.1919@\emph{Fremden-Blatt}|pw}\pwindex{?? Werk@Nicht ermittelte Verfasserinnen und Verfasser!Man schreibt uns aus Gmunden19.7.1895 – 19.7.1895@\emph{Man schreibt uns aus Gmunden} {[}19.7.1895 – 19.7.1895{]}|pwv}}{\lemma{\textnormal{\emph{Fremdenblatt}}}\Cendnote{\textnormal{»– Man schreibt uns aus \so{Gmunden}\oindex{Gmunden@\textbf{Gmunden}|pw}: Das hiesige Saisontheater sieht einer interessanten Première entgegen.
                        Arthur \so{Schnitzler}\pwindex{Schnitzler, Arthur 15.05.1862 – 21.10.1931@\textsc{Schnitzler, Arthur} (15.05.1862 – 21.10.1931), \emph{Schriftsteller, Mediziner}|pw}’s ›\so{Freiwild}\pwindex{Schnitzler, Arthur 15.05.1862 – 21.10.1931@\textsc{Schnitzler, Arthur} (15.05.1862 – 21.10.1931), \emph{Schriftsteller, Mediziner}!Freiwild. Schauspiel in 3 Akten1896@\strich\emph{Freiwild. Schauspiel in 3 Akten} {[}1896{]}|pw}« gelangt hier Donnerstag den 22. d., von Direktor \so{Cavar}\pwindex{Cavar, Alfred 02.12.1859 – 15.09.1920@\textsc{Cavar, Alfred} (02.12.1859 – 15.09.1920), \emph{Theaterleiter}|pw} inszenirt, zum erstenmale (in Oesterreich\oindex{Oesterreich@\textbf{Österreich}|pw}) zur Aufführung, mit jenen Einschränkungen natürlich,
                     welche die Zensur für nothwendig erachtet hat. In der Novität sind die besten
                     Kräfte beschäftigt, über welche das hiesige Theater verfügt, u. A. die Naive
                     Fräulein \so{Großmüller}\pwindex{Grossmueller, Frieda *~7.1.1880@\textsc{Grossmüller, Frieda} (*~7.1.1880), \emph{Schauspielerin}|pw}, welche für die nächste Saison an das Deutsche Volkstheater\orgindex{Volkstheater@Volkstheater|pw} engagirt ist, und Herr Alexander \so{Rottmann}\pwindex{Rottmann, Alexander 03.07.1869 – 26.09.1916@\textsc{Rottmann, Alexander} (03.07.1869 – 26.09.1916), \emph{Schauspieler}|pw}, der in einer Aufführung von Ohnet\pwindex{Ohnet, Georges 03.04.1848 – 05.05.1918@\textsc{Ohnet, Georges} (03.04.1848 – 05.05.1918), \emph{Schriftsteller}|pw}’s
                        ›Hüttenbesitzer\pwindex{Ohnet, Georges 03.04.1848 – 05.05.1918@\textsc{Ohnet, Georges} (03.04.1848 – 05.05.1918), \emph{Schriftsteller}!Huettenbesitzer1882@\strich\emph{Der Hüttenbesitzer} {[}1882{]}|pw}‹ durch die diskrete
                     Anwendung seiner schönen Mittel und die Natürlichkeit seiner Darstellung des
                     Philippe Derblay einen vollen Erfolg erzielt hat.« (\emph{Fremden-Blatt}\pwindex{Fremden-Blatt1.7.1847 – 22.3.1919@\emph{Fremden-Blatt}|pwk}, Jg. 51, Nr. 198,
                        19. 7. 1897, Abend-Blatt, S. 6)}}}\label{K_L00707_1h}) mit cenſurellen
               Aenderungen. Ich hab an \textsc{Cavar}\pwindex{Cavar, Alfred 02.12.1859 – 15.09.1920@\textsc{Cavar, Alfred} (02.12.1859 – 15.09.1920), \emph{Theaterleiter}|pw} telegrafirt, mir {\pb}ſofort die Aenderg
               mitzutheilen. Geſindel, mi\textcolor{gray}{ch} nicht vorher zu verſtändg. (Kämen Sie
                     Do{\geminationn}erſtg mit mir hinüber?)\pend
           \pstart
           4.) Schaun Sie nach dem Nachtmahl zu mir herauf oder laſſen mir ſagen, wo Sie
               ſind.\pend
           \pstart
           Herzl Gruß{\\[\baselineskip]}Ihr \spacefill\mbox{A.}\pend
           \leftskip=0em{}\pstart
           \noindent{}\centering{}\textcolor{gray}{\textbf{D\textsuperscript{r} Arthur Schnitzler}}\pend
           \pstart
           \noindent{}\raggedleft{}Wien\oindex{Wien@\textbf{Wien}|pw}\pend
           \endnumbering\briefempfaengerindex{Beer-Hofmann, Richard@\textsc{Beer-Hofmann, Richard}!zzzSchnitzler, Arthur@\emph{von Arthur Schnitzler}!1897-07-201@{20. 7. 1897}|)be}\mylabel{h}\end{ledgroupsized}  \newcommand{\dateiname}{L00707}\newcommand{\titel}{Arthur Schnitzler an Richard Beer-Hofmann, 20. 7. 1897}\newcommand{\editorInnen}{Martin Anton Müller und Gerd-Hermann Susen}%% latex-leseansicht-abspann.tex
%% Abspann für die Leseansicht.
%% Der Schalter \ifkorrekturansicht ist bereits durch den Vorspann gesetzt.

%% latex-abspann.tex
%% Gemeinsamer Abspann für Korrekturansicht und Leseansicht.
%% Setzt den Schalter \ifkorrekturansicht voraus (gesetzt in den
%% einbindenden Dateien latex-korrekturansicht-abspann.tex bzw.
%% latex-leseansicht-abspann.tex).
%% ---------------------------------------------------------------

\normalsize

% Das esempio-Environment wird nur in der Leseansicht benötigt
\ifkorrekturansicht\else
\newenvironment{esempio}[3]%
{
    \vspace{1.5ex}
    \rlap{\underline{#1}}
    \par
    \setlength{\parindent}{0cm}
    \nopagebreak
    \leftskip=#2cm
    \rightskip=#3cm
}
{
    \par
}
\fi

\doendnotes{C}
\bigskip
\vfill

\clearpage

\footnotesize

\ifkorrekturansicht
  \lohead{\textsc{register}}
\fi

% theindex-Environment neu definieren ohne reledmac
\makeatletter
\renewenvironment{theindex}{%
  \ifkorrekturansicht
    \section*{\indexname}%
  \else
    \subsubsection*{Index der erwähnten Entitäten}%
  \fi
  \setlength{\parindent}{0pt}%
  \setlength{\parskip}{0pt plus 0.3pt}%
  \let\item\@idxitem
}{%
  \ifkorrekturansicht\clearpage\fi
}
\makeatother

\IfFileExists{\jobname-pw.ind}{\input{\jobname-pw.ind}}{}

% Quellenangabe nur in der Leseansicht
\ifkorrekturansicht\else
% Fallback-Definitionen, falls die .tex-Datei \titel etc. nicht gesetzt hat
\providecommand{\titel}{}
\providecommand{\editorInnen}{}
\providecommand{\dateiname}{\jobname}

\vspace{3cm}

\vfill

\footnotesize
\textsc{Quelle}: \titel. Herausgegeben von {\editorInnen}. In: \emph{Arthur Schnitzler: Briefwechsel mit Autorinnen und Autoren}.
 Digitale Edition, https://schnitzler-briefe.acdh.oeaw.ac.at/{\dateiname}.html (Stand \today)
\fi

\end{document}


      