%% latex-leseansicht-vorspann.tex
%% Vorspann für die Leseansicht.
%% Lädt die gemeinsame Datei latex-vorspann.tex mit nicht gesetztem Schalter.

\newif\ifkorrekturansicht
\korrekturansichtfalse

\input{../tex-inputs/latex-vorspann}

\begin{center}
            \textcolor{red}{ENTWURF, NICHT FERTIG KORRIGIERT}
                      \end{center}
            
         \renewcommand{\erwaehnteOrte}{Orte: Semmering, Wien}
         \renewcommand{\erwaehnteWerke}{}
               \section[Felix Salten an Arthur Schnitzler, {[}11. 11?. 1903{]}]{ Felix Salten an Arthur Schnitzler, {[}11. 11?. 1903{]}}\nopagebreak\mylabel{v}\rehead{ }\begin{ledgroupsized}[t]{13cm}\normalsize\beginnumbering \toendnotes[C]{\smallbreak\pagebreak[2]} \Standort{CUL, Schnitzler, B 89, A 2.}
\physDesc{Brief, 1 Blatt, 4 Seiten
\newline{}Handschrift: Bleistift, lateinische Kurrent\newline{}Ordnung: mit Bleistift von unbekannter Hand nummeriert:
                                    »181« }\toendnotes[C]{\smallbreak}\pstart
           {\pb}\label{K_L03355-1v}\edtext{Mittwoch}{\lemma{\textnormal{\emph{Mittwoch}}}\Cendnote{\textnormal{Schnitzler\pwindex{Schnitzler, Arthur 15.05.1862 – 21.10.1931@\textsc{Schnitzler, Arthur} (15.05.1862 – 21.10.1931), \emph{Schriftsteller, Mediziner}|pwk} datiert den Brief auf den falschen Monat.}}}\label{K_L03355-1h}.\pend
           \pstart
           Lieber, vor allem: Ihr \label{K_L03355-111v}\edtext{Brief vom Semmering\oindex{Semmering@\textbf{Semmering}|pw}}{\lemma{\textnormal{\emph{Brief vom Semmering}}}\Cendnote{\textnormal{Arthur Schnitzler an Felix Salten, 7. 11. 1903}}}\label{K_L03355-111h} gab mir ein \uline{Recht} zu der Annahme,
               Sie seien verletzt, und seien in manchen Dingen, die bisher zwischen uns fest standen
               erschüttert worden. Das legte mir, in\uline{meiner} Erregung
               (die Sie begreifen müßen) den Gedanken nahe, ob es rathsam sei, sich nach diesem
               brieflichen Unwetter \uline{sogleich}\uline{wieder} auf kritischen Boden zu treffen. \uline{Nur} weil die Vorlesung so \uline{unmittelbar} bevorsteht, kam ich darauf, sie in den Kreis der Discussion zu
               ziehen, und was also ein zufälliges Zusammentreffen war, nehmen Sie als ein
               Misstrauen pro futuro. \pend
           \pstart
           Ich habe lediglich in einem Gefühl – wie soll ich sagen? – des Respektes vor der
               unberührten Sti{\geminationm}ung, die sonst bei unseren Vorlesungen
               obwaltet, lediglich aus dem Wunsch und aus der Besorgnis, die absolute Klarheit
               dieser für uns alle so notwendigen Athmosphäre ungetrübt zu erhalten, darauf
               hingewiesen, dass \uline{diese} Vorlesung Sie vielleicht noch
               nicht ganz in ruhiger Unbefangenheit mir gegenüber finden wird. \pend
           \pstart
           Mein Brief war in seiner sachlichen und gründlichen Ausführlichkeit allein schon ein
               Freundschaftsbeweis. Alles, was ich darin sage, kann garnicht anders gedacht werden,
               als dass ich mir die äußerste und ernsteste Mühe gab, einen Freund über meine {\pb}Intentionen aufzuklären, seine
               Verstimmung zu beheben. Dass ich auch dies Letzte, – ohne böse \uline{Absicht}, \uline{ohne einen schlimmen
                  Nebengedanken} – aussprach, ist wieder nur ein Freundschaftsbeweis. Nichts in
               meinem bisherigen Verhalten gegen Sie, nichts in meinem Brief gibt ihnen ein Recht zu
               der Annahme, ich hätte Ihnen ein häßliches Misstrauen insinuiren wollen. Dagegen muß
               ich mich mit aller Entschiedenheit verwehren. \pend
           \pstart
           Man ist doch nicht »nachträgerisch«, wenn man von einer Sache tiefer berührt wird,
               Sie setzen consequent verletzende Worte, die \uuline{ich} nie
               gemeint habe, für die besseren und einfacheren. \pend
           \pstart
           Ich hielt Sie, und mußte Sie für tiefer berührt halten. Sie überraschen mich jetzt
               durch die Mittheilung, es sei Ihnen nicht möglich, »dergleichen schwer« zu nehmen. \pend
           \pstart
           In Ihrem ersten Brief sagen Sie mir klipp und klar, Sie seien an \uline{meiner kritischen Aufrichtigkeit} irre geworden. Und \uline{das} las ich Sonntag. Weil ich nun besorgt
               wurde, wie das am Donnerstag sein wird, eine Besorgnis, von der Sie sehr
               genau wissen müßten, dass sie mich ebenso wie der Vorwurf kränken muß, werden Sie
               heftig. \pend
           \pstart
           Sie hätten mir ruhig sagen können: »Die Sache wirklich freilich noch in mir nach, –
               aber ko{\geminationm}en {\pb}Sie.« Oder Sie
               hätten mir sagen können: »es ist kein Rest davon mehr in mir!« Ich wollte weder ein
               Vertrauensvotum provoziren, noch Ihnen ein Misstrauen aussprechen, – ich wollte
               Klarheit in einer Sache, die mir so sehr am Herzen liegt, wie unsere Vorlesungen! Ich
               wollte nicht, mit dem leisesten Schatten einer Besorgnis nach dieser Richtung Ihr
               Werk hören. Dass ich solche Dinge aussprach, ist einfach ein Beweis subtiler
               Ehrlichkeit. Dass Sie mir darauf \uline{so} antworten, legt
               auch mir die Frage vor, die Sie am Anfang Ihres großen Briefes aufwarfen, »ob es
               nicht besser sei, ec.« \pend
           \pstart
           Ich will auf die so sehr heftigen und verletzenden Dinge, die Sie mir schreiben,
               nicht eingehen. Jetzt nicht. Vielleicht sprechen wir nach der Vorlesung über den
               Anspruch auf Erregung und Ungerechtigkeit, den Sie für sich selbst geltend machen,
               und den Sie mir nicht zubilligen wollen, über das hohe Niveau »absoluter
               Ehrlichkeit«, auf welchem ich unsere Beziehungen nicht hätte halten können, und auch
               darüber, dass von Ihrer Seite das Wort »Bruch« in dieser Angelegenheit fallen konnte.
               Lieber wäre es mir, und erwünschter freilich gewesen, wenn alles \uline{vorher} zwischen uns ins Reine gekommen wäre. Aber offenbar können
               Briefe, die aus dem Temperament und nicht aus Vorbedacht geschrieben werden, eine
               Sache beid{[}er{]} seits nur verwirren. Ich resümire: \uline{nie} werde ich {\pb}zu der Empfindung zu überreden
               sein, dass \uline{ich} an dem Abbruch unserer Beziehungen
               Schuld trage, und nie werde ich dieses Auseinandergehen verhindern, wenn mir gesagt
               wird, dass ich enervante Wirkungen ausübe, und wenn ich sehe, dass ein noch so zart
               gemeintes Bedenken mir als Misstrauen ausgelegt werden kann. \pend
           \pstart
           Dagegen werde ich \uline{alles} aufbieten, eine Freundschaft
               zu erhalten, die ich als die einzige meines Lebens bezeichnen muß, die mir bisher –
               ich glaube es bewiesen zu haben – menschlich und künstlerisch theuer war, und die man
               in meinem Alter ja auch nicht ohne starke Erschütterung verliert, – wenn mir wie
               sonst die Möglichkeit bleibt, ohne Angst vor Missdeutungen, und ohne Angst vor
               verzehrender Milde \uline{alles}\uline{rückhaltlos} zu sagen was ich denke! Und es erscheint
               mir leider notwendig hier noch etwas hinzuzufügen, dass mein \uline{schärfster} Gedanke \introOben{}gegen Sie\introOben{} bis auf den
               heutigen Tag noch nicht scharf genug gewesen ist, um auch nur eines Kindes Haut zu
               ritzen. Ich meine: darauf kommt es an! \pend
           \pstart Ihr \spacefill\mbox{FS.}\pend{}
         
         \endnumbering\mylabel{h}\end{ledgroupsized}\begin{anhang}\end{anhang}\newcommand{\dateiname}{L03355}\newcommand{\titel}{Felix Salten an Arthur Schnitzler, [11. 11?. 1903]}\newcommand{\editorInnen}{Martin Anton Müller und Laura Untner}%% latex-leseansicht-abspann.tex
%% Abspann für die Leseansicht.
%% Der Schalter \ifkorrekturansicht ist bereits durch den Vorspann gesetzt.

%% latex-abspann.tex
%% Gemeinsamer Abspann für Korrekturansicht und Leseansicht.
%% Setzt den Schalter \ifkorrekturansicht voraus (gesetzt in den
%% einbindenden Dateien latex-korrekturansicht-abspann.tex bzw.
%% latex-leseansicht-abspann.tex).
%% ---------------------------------------------------------------

\normalsize

% Das esempio-Environment wird nur in der Leseansicht benötigt
\ifkorrekturansicht\else
\newenvironment{esempio}[3]%
{
    \vspace{1.5ex}
    \rlap{\underline{#1}}
    \par
    \setlength{\parindent}{0cm}
    \nopagebreak
    \leftskip=#2cm
    \rightskip=#3cm
}
{
    \par
}
\fi

\doendnotes{C}
\bigskip
\vfill

\clearpage

\footnotesize

\ifkorrekturansicht
  \lohead{\textsc{register}}
\fi

% theindex-Environment neu definieren ohne reledmac
\makeatletter
\renewenvironment{theindex}{%
  \ifkorrekturansicht
    \section*{\indexname}%
  \else
    \subsubsection*{Index der erwähnten Entitäten}%
  \fi
  \setlength{\parindent}{0pt}%
  \setlength{\parskip}{0pt plus 0.3pt}%
  \let\item\@idxitem
}{%
  \ifkorrekturansicht\clearpage\fi
}
\makeatother

\IfFileExists{\jobname-pw.ind}{\input{\jobname-pw.ind}}{}

% Quellenangabe nur in der Leseansicht
\ifkorrekturansicht\else
% Fallback-Definitionen, falls die .tex-Datei \titel etc. nicht gesetzt hat
\providecommand{\titel}{}
\providecommand{\editorInnen}{}
\providecommand{\dateiname}{\jobname}

\vspace{3cm}

\vfill

\footnotesize
\textsc{Quelle}: \titel. Herausgegeben von {\editorInnen}. In: \emph{Arthur Schnitzler: Briefwechsel mit Autorinnen und Autoren}.
 Digitale Edition, https://schnitzler-briefe.acdh.oeaw.ac.at/{\dateiname}.html (Stand \today)
\fi

\end{document}


      