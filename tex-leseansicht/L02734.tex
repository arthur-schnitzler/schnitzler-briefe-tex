%% latex-leseansicht-vorspann.tex
%% Vorspann für die Leseansicht.
%% Lädt die gemeinsame Datei latex-vorspann.tex mit nicht gesetztem Schalter.

\newif\ifkorrekturansicht
\korrekturansichtfalse

\input{../tex-inputs/latex-vorspann}


         
         \renewcommand{\erwaehntePersonen}{Personen: Leopold von Andrian-Werburg, Hermann Bahr, Richard Beer-Hofmann, Max Eugen Burckhard, Theodor Herzl, Paul Lindau, Rudolf Lothar, Moriz Necker, Paul Schlenther, Carl von Torresani-Lanzenfeld}
         \renewcommand{\erwaehnteInstitutionen}{Institutionen: Burgtheater, Deutsches Theater Berlin, Verein der Literaturfreunde}
         \renewcommand{\erwaehnteOrte}{Orte: Berlin, Deutschland, Frankfurt am Main, Paris, Wien}
         \renewcommand{\erwaehnteWerke}{Werke: Das junge Österreich, Das junge Österreich [Vortrag], Der Garten der Erkenntnis, Der Garten der Erkenntnis, Die Zeit. Wiener Wochenschrift, Frauenlob. Ein Lustspiel in drei Aufzügen, Liebelei. Schauspiel in drei Akten, Neue Freie Presse}
               \section[Paul Goldmann an Arthur Schnitzler, 24. 4. {[}1895{]}]{ Paul Goldmann an Arthur Schnitzler, 24. 4. {[}1895{]}}\nopagebreak\mylabel{v}\rehead{ }\begin{ledgroupsized}[t]{13cm}\normalsize\beginnumbering \toendnotes[C]{\smallbreak\pagebreak[2]} \Standort{DLA, A:Schnitzler, HS.NZ85.1.3165.}
\physDesc{Brief, 2 Blätter, 8 Seiten
\newline{}Handschrift: blaue Tinte, deutsche Kurrent
\newline{}Schnitzler: 1) mit Bleistift das Jahr »95« vermerkt  2) mit rotem Buntstift sechs Unterstreichungen}\toendnotes[C]{\smallbreak}\pstart
           \raggedleft{}{\pb}\textsc{Frankfurt\oindex{Frankfurt am Main@\textbf{Frankfurt am Main}|pw}}{ }24. April.\pend
           \pstart\center{}Mein lieber Freund,\pend\pstart
           Seit zehn Tagen bin ich in Frankfurt\oindex{Frankfurt am Main@\textbf{Frankfurt am Main}|pw} bei den
               Meinen. Deutſch\oindex{Deutschland@\textbf{Deutschland}|pwv}es Land,
               Frühling und Friede – das thut wohl. Aber drohend ſind die Zukunftsfragen da. Und ich
               war krank und lag einige Tage zu Bette{[}.{]}{ }{\pb}Dieſer Tage gehe ich nach \textsc{Paris\oindex{Paris@\textbf{Paris}|pw}} zurück. Will Dir nur von unterwegs einen Gruß ſenden. Aus \textsc{Paris\oindex{Paris@\textbf{Paris}|pw}} hörſt Du Näheres von mir.\pend
           \pstart
           \textsc{Herzl\pwindex{Herzl, Theodor 1860-05-02 – 1904-07-03@\textsc{Herzl, Theodor} (1860-05-02 – 1904-07-03), \emph{Schriftsteller, Journalist}|pw}} iſt gar ſo ſchweigſam über das \label{K_L02734-1v}\edtext{Beiſammenſein mit Dir}{\lemma{\textnormal{\emph{Beiſammenſein mit Dir}}}\Cendnote{\textnormal{Theodor Herzl\pwindex{Herzl, Theodor 1860-05-02 – 1904-07-03@\textsc{Herzl, Theodor} (1860-05-02 – 1904-07-03), \emph{Schriftsteller, Journalist}|pwk} hielt sich im März 1895 in Wien\oindex{Wien@\textbf{Wien}|pwk} auf.
                  Zwischen 26. 3. 1895
                  und 30. 3. 1895 sah
                  er Schnitzler\pwindex{Schnitzler, Arthur 15.05.1862 – 21.10.1931@\textsc{Schnitzler, Arthur} (15.05.1862 – 21.10.1931), \emph{Schriftsteller, Mediziner}|pwk} jeden Tag. Ein Konflikt
                  zwischen den beiden ist nicht bekannt.}}}\label{K_L02734-1h}. Iſt das nur ſeine eitle {\pb}\label{K_L02734-44v}\edtext{\textsc{Suffisance}}{\lemma{\textnormal{\emph{Suffisance}}}\Cendnote{\textnormal{französisch: Selbstgefälligkeit}}}\label{K_L02734-44h}?
               Oder habt Ihr was gehabt? Wie hat er Dir überhaupt gefallen?\pend
           \pstart
           Ich \substVorne{}\textsuperscript{hö\textcolor{gray}{re,}}\substDazwischen{}höre,\substHinten{}{ }Du\pwindex{Schnitzler, Arthur 15.05.1862 – 21.10.1931@\textsc{Schnitzler, Arthur} (15.05.1862 – 21.10.1931), \emph{Schriftsteller, Mediziner}!Liebelei. Schauspiel in drei Akten1895-10-09@\strich\emph{Liebelei. Schauspiel in drei Akten} {[}1895-10-09{]}|pwv} wirſt erſt im Herbſt
               aufgeführt. Beſſer im Anfang, als am Ende der Saison. Am Beſten wäre es freilich, die
                  \label{K_L02734-2v}\edtext{Berlin\oindex{Berlin@\textbf{Berlin}|pw}er Aufführung\pwindex{Schnitzler, Arthur 15.05.1862 – 21.10.1931@\textsc{Schnitzler, Arthur} (15.05.1862 – 21.10.1931), \emph{Schriftsteller, Mediziner}!Liebelei. Schauspiel in drei Akten1895-10-09@\strich\emph{Liebelei. Schauspiel in drei Akten} {[}1895-10-09{]}|pwv}}{\lemma{\textnormal{\emph{Berliner Aufführung}}}\Cendnote{\textnormal{Am 4. 2. 1896 feierte die \emph{Liebelei}\pwindex{Schnitzler, Arthur 15.05.1862 – 21.10.1931@\textsc{Schnitzler, Arthur} (15.05.1862 – 21.10.1931), \emph{Schriftsteller, Mediziner}!Liebelei. Schauspiel in drei Akten1895-10-09@\strich\emph{Liebelei. Schauspiel in drei Akten} {[}1895-10-09{]}|pwk} am \emph{Deutschen
                     Theater}\orgindex{Deutsches Theater Berlin@Deutsches Theater Berlin|pwk} in Berlin\oindex{Berlin@\textbf{Berlin}|pwk} Premiere.}}}\label{K_L02734-2h}{ }{\pb}ginge der Wien\oindex{Wien@\textbf{Wien}|pw}er
               voran. Publikum und Kritik ſind in Berlin\oindex{Berlin@\textbf{Berlin}|pw} doch im
               Ganzen intelligenter. Ein Berlin\oindex{Berlin@\textbf{Berlin}|pw}er Erfolg wäre
               für Wien\oindex{Wien@\textbf{Wien}|pw} beſtimmend, auch für den ewig zaudernden
                  Burgtheater\orgindex{Burgtheater@Burgtheater|pw}-Direktor\pwindex{Burckhard, Max Eugen 14.07.1854 – 16.03.1912@\textsc{Burckhard, Max Eugen} (14.07.1854 – 16.03.1912), \emph{Schriftsteller, Rechtswissenschaftler, Theaterleiter}|pwv}. (Wie ich hier höre,
               ſtrebt \textsc{Paul Lindau\pwindex{Lindau, Paul 03.06.1839 – 31.01.1919@\textsc{Lindau, Paul} (03.06.1839 – 31.01.1919), \emph{Schriftsteller, Kritiker, Theaterleiter}|pw}} nach \label{K_L02734-3v}\edtext{\textsc{Burckhardt\pwindex{Burckhard, Max Eugen 14.07.1854 – 16.03.1912@\textsc{Burckhard, Max Eugen} (14.07.1854 – 16.03.1912), \emph{Schriftsteller, Rechtswissenschaftler, Theaterleiter}|pw}s} Nachfolgerſchaft}{\lemma{\textnormal{\emph{Burckhardts Nachfolgerſchaft}}}\Cendnote{\textnormal{Max Burckhardt\pwindex{Burckhard, Max Eugen 14.07.1854 – 16.03.1912@\textsc{Burckhard, Max Eugen} (14.07.1854 – 16.03.1912), \emph{Schriftsteller, Rechtswissenschaftler, Theaterleiter}|pwk} war als Jurist eine
                  überraschende Besetzung für die Leitung des \emph{Burgtheater}\orgindex{Burgtheater@Burgtheater|pwk}s gewesen. Ablösegerüchte oder -wünsche bestanden von Anfang
                  an, doch konnte er sich bis 1898 halten. Nachfolger wurde
                     Paul Schlenther\pwindex{Schlenther, Paul 20.08.1854 – 30.04.1916@\textsc{Schlenther, Paul} (20.08.1854 – 30.04.1916), \emph{Schriftsteller, Kritiker, Theaterleiter}|pwk}.}}}\label{K_L02734-3h}). {\pb}Hier ein \label{K_L02734-4v}\edtext{Stück\pwindex{Lothar, Rudolf 23.2.1865 – 2.10.1943@\textsc{Lothar, Rudolf} (23.2.1865 – 2.10.1943), \emph{Schriftsteller, Journalist, Theaterdirektor}!Frauenlob. Ein Lustspiel in drei Aufzuegen1895@\strich\emph{Frauenlob. Ein Lustspiel in drei Aufzügen} {[}1895{]}|pwu}}{\lemma{\textnormal{\emph{Stück}}}\Cendnote{\textnormal{vermutlich \emph{ Frauenlob. Lustspiel in drei Aufzügen }\pwindex{Lothar, Rudolf 23.2.1865 – 2.10.1943@\textsc{Lothar, Rudolf} (23.2.1865 – 2.10.1943), \emph{Schriftsteller, Journalist, Theaterdirektor}!Frauenlob. Ein Lustspiel in drei Aufzuegen1895@\strich\emph{Frauenlob. Ein Lustspiel in drei Aufzügen} {[}1895{]}|pwk}}}}\label{K_L02734-4h} von \textsc{Rudolf Lothar\pwindex{Lothar, Rudolf 23.2.1865 – 2.10.1943@\textsc{Lothar, Rudolf} (23.2.1865 – 2.10.1943), \emph{Schriftsteller, Journalist, Theaterdirektor}|pw}} geſehen. Es iſt unerhört, daß man dieſen Buben\pwindex{Lothar, Rudolf 23.2.1865 – 2.10.1943@\textsc{Lothar, Rudolf} (23.2.1865 – 2.10.1943), \emph{Schriftsteller, Journalist, Theaterdirektor}|pwv} nicht mit Fußtritten vom Theater jagt.\pend
           \pstart
           Haſt Du frohe Oſtern gehabt? Und wie gehts Dir? Du ſchreibſt mir wohl ein kurzes
               Wort, ohne meine {\pb}längere Antwort abzuwarten.\pend
           \pstart
           \textsc{Bahr\pwindex{Bahr, Hermann 19.07.1863 – 15.01.1934@\textsc{Bahr, Hermann} (19.07.1863 – 15.01.1934), \emph{Schriftsteller, Kritiker}|pw}} hat alſo wieder einen \label{K_L02734-5v}\edtext{Vortrag\pwindex{Bahr, Hermann 19.07.1863 – 15.01.1934@\textsc{Bahr, Hermann} (19.07.1863 – 15.01.1934), \emph{Schriftsteller, Kritiker}!junge Oesterreich [Vortrag]1895-03-13@\strich\emph{Das junge Österreich [Vortrag]} {[}1895-03-13{]}|pwv}}{\lemma{\textnormal{\emph{Vortrag}}}\Cendnote{\textnormal{Am 13. 3. 1895 fand eine Veranstaltung des \emph{Vereins der Literaturfreunde}\orgindex{Verein der Literaturfreunde@Verein der Literaturfreunde|pwk} statt, bei der Hermann Bahr\pwindex{Bahr, Hermann 19.07.1863 – 15.01.1934@\textsc{Bahr, Hermann} (19.07.1863 – 15.01.1934), \emph{Schriftsteller, Kritiker}|pwk} einen Vortrag\pwindex{Bahr, Hermann 19.07.1863 – 15.01.1934@\textsc{Bahr, Hermann} (19.07.1863 – 15.01.1934), \emph{Schriftsteller, Kritiker}!junge Oesterreich [Vortrag]1895-03-13@\strich\emph{Das junge Österreich [Vortrag]} {[}1895-03-13{]}|pwkv} mit dem Titel \emph{Das junge Österreich}\pwindex{Bahr, Hermann 19.07.1863 – 15.01.1934@\textsc{Bahr, Hermann} (19.07.1863 – 15.01.1934), \emph{Schriftsteller, Kritiker}!junge Oesterreich [Vortrag]1895-03-13@\strich\emph{Das junge Österreich [Vortrag]} {[}1895-03-13{]}|pwk} hielt. Schnitzler\pwindex{Schnitzler, Arthur 15.05.1862 – 21.10.1931@\textsc{Schnitzler, Arthur} (15.05.1862 – 21.10.1931), \emph{Schriftsteller, Mediziner}|pwk}, dessen Kunstschaffen als
                     »abgethan« geschildert wurde, war empört. Siehe A. S.: \emph{Tagebuch}, 14. 3. 1895. }}}\label{K_L02734-5h}
               gehalten. Der Volksſänger der
                  Moderne\pwindex{Bahr, Hermann 19.07.1863 – 15.01.1934@\textsc{Bahr, Hermann} (19.07.1863 – 15.01.1934), \emph{Schriftsteller, Kritiker}|pwv}! Die \label{K_L02734-6v}\edtext{Brettl-Natur}{\lemma{\textnormal{\emph{Brettl-Natur}}}\Cendnote{\textnormal{abwertend; gemeint ist ein Schauspieler,
                  der nicht auf einer gezimmerten, sondern einer aus einfachen Brettern
                  zusammengefügten Bühne auftritt}}}\label{K_L02734-6h}, das iſt der Grund in dem Weſen des Kerl\pwindex{Bahr, Hermann 19.07.1863 – 15.01.1934@\textsc{Bahr, Hermann} (19.07.1863 – 15.01.1934), \emph{Schriftsteller, Kritiker}|pwv}s. Wie ich den immer mehr
               haſſe! Dieſen Mann\pwindex{Bahr, Hermann 19.07.1863 – 15.01.1934@\textsc{Bahr, Hermann} (19.07.1863 – 15.01.1934), \emph{Schriftsteller, Kritiker}|pwv} von
               Geiſt, aber {\pb}ohne Kunſt, ohne Urtheil, ohne
               Gewiſſen! Merkſt Du, wie er ſich langſam in die \label{K_L02734-66v}\edtext{\textsc{Clique}}{\lemma{\textnormal{\emph{Clique}}}\Cendnote{\textnormal{Hier liegt eine positive Verwendung des
                  Wortes vor, das bei Schnitzler\pwindex{Schnitzler, Arthur 15.05.1862 – 21.10.1931@\textsc{Schnitzler, Arthur} (15.05.1862 – 21.10.1931), \emph{Schriftsteller, Mediziner}|pwk} hingegen
                  meist nur in einer negativen Form vorkommt, insofern er nicht als Teil einer
                  eingeschworenen Gruppe von Literaten wahrgenommen werden wollte. }}}\label{K_L02734-66h}
               hineinſchleicht? In wenig Jahren hat er irgendwo ein officiöſes k. k. Literatur-Amt.
               Daß dieſes Rindvieh\pwindex{Necker, Moriz 1857-10-14 – 1915-02-16@\textsc{Necker, Moriz} (1857-10-14 – 1915-02-16), \emph{Journalist, Kritiker, Literaturwissenschaftler}|pwv}, der
                  \strikeout{\textsc{A}}{ }\label{K_L02734-7v}\edtext{\textsc{Necker\pwindex{junge Oesterreich1895-03-14@\emph{Das junge Österreich} {[}1895-03-14{]}|pwv}\pwindex{Necker, Moriz 1857-10-14 – 1915-02-16@\textsc{Necker, Moriz} (1857-10-14 – 1915-02-16), \emph{Journalist, Kritiker, Literaturwissenschaftler}|pw}}}{\lemma{\textnormal{\emph{Necker}}}\Cendnote{\textnormal{Die Veranstaltung wurde wohlwollend von
                     Moriz Necker\pwindex{Necker, Moriz 1857-10-14 – 1915-02-16@\textsc{Necker, Moriz} (1857-10-14 – 1915-02-16), \emph{Journalist, Kritiker, Literaturwissenschaftler}|pwk} in der \emph{Neuen Freien Presse}\pwindex{Neue Freie Presse1864 – 1939@\emph{Neue Freie Presse} {[}1864 – 1939{]}|pwk} besprochen, einschließlich der
                  überraschenden Volte, dass eine neue Kunstepoche entstehe und dass frühere Wien\oindex{Wien@\textbf{Wien}|pwk}er Vertreter wie »Hermann Bahr\pwindex{Bahr, Hermann 19.07.1863 – 15.01.1934@\textsc{Bahr, Hermann} (19.07.1863 – 15.01.1934), \emph{Schriftsteller, Kritiker}|pw}, Baron Torresani\pwindex{Torresani-Lanzenfeld, Carl von 19.04.1846 – 16.04.1907@\textsc{Torresani-Lanzenfeld, Carl von} (19.04.1846 – 16.04.1907), \emph{Schriftsteller}|pw}, Beer-Hoffmann\pwindex{Beer-Hofmann, Richard 1866-07-11 – 1945-09-26@\textsc{Beer-Hofmann, Richard} (1866-07-11 – 1945-09-26), \emph{Schriftsteller}|pw}« nur eine Übergangszeit repräsentiert hätten. Schnitzler\pwindex{Schnitzler, Arthur 15.05.1862 – 21.10.1931@\textsc{Schnitzler, Arthur} (15.05.1862 – 21.10.1931), \emph{Schriftsteller, Mediziner}|pwk}s Name fällt in der Rezension\pwindex{junge Oesterreich1895-03-14@\emph{Das junge Österreich} {[}1895-03-14{]}|pwkv} nicht. Vgl. [Moriz Necker]\pwindex{Necker, Moriz 1857-10-14 – 1915-02-16@\textsc{Necker, Moriz} (1857-10-14 – 1915-02-16), \emph{Journalist, Kritiker, Literaturwissenschaftler}|pwk}: \emph{Das junge Österreich}\pwindex{junge Oesterreich1895-03-14@\emph{Das junge Österreich} {[}1895-03-14{]}|pwk}. In: \emph{Neue Freie Presse}\pwindex{Neue Freie Presse1864 – 1939@\emph{Neue Freie Presse} {[}1864 – 1939{]}|pwk}, Nr. 10.075, 14. 3. 1895, S. 5.}}}\label{K_L02734-7h}, Dich angreift, iſt ſelbſt{\pb}verſtändlich. \strikeout{Wenn
                  Du} Daran daß Du die \strikeout{Och}{ } Ochſen ſtützig machſt, kannſt Du auch ſehen, daß
               Du Jemand biſt. Aber daß dieſer \label{K_L02734-88v}\edtext{Angriff in der »Zeit\pwindex{Zeit. Wiener Wochenschrift1894 – 1904@\emph{Die Zeit. Wiener Wochenschrift} {[}1894 – 1904{]}|pw}«}{\lemma{\textnormal{\emph{Angriff in der »Zeit«}}}\Cendnote{\textnormal{Gemeint dürfte nicht ein spezifischer Artikel sein – auch
                  wenn Bahr\pwindex{Bahr, Hermann 19.07.1863 – 15.01.1934@\textsc{Bahr, Hermann} (19.07.1863 – 15.01.1934), \emph{Schriftsteller, Kritiker}|pwk} Gedanken davon in seiner Rezension\pwindex{Bahr, Hermann 19.07.1863 – 15.01.1934@\textsc{Bahr, Hermann} (19.07.1863 – 15.01.1934), \emph{Schriftsteller, Kritiker}!Garten der Erkenntnis1895-03-16@\strich\emph{Der Garten der Erkenntnis} {[}1895-03-16{]}|pwkv} von Leopold von Andrian-Werburg\pwindex{Andrian-Werburg, Leopold von 09.05.1875 – 19.11.1951@\textsc{Andrian-Werburg, Leopold von} (09.05.1875 – 19.11.1951), \emph{Schriftsteller, Diplomat}|pwk}s \emph{Der Garten der Erkenntnis}\pwindex{Garten der Erkenntnis1895@\emph{Der Garten der Erkenntnis} {[}1895{]}|pwk} verwendet –, sondern eher die
                  allgemeine Unmut ausdrücken, dass von einem Repräsentanten der Wochenschrift\pwindex{Zeit. Wiener Wochenschrift1894 – 1904@\emph{Die Zeit. Wiener Wochenschrift} {[}1894 – 1904{]}|pwkv}, die man auf der eigenen
                  Seite vermutete, Kritik kam. Vgl. Hermann Bahr\pwindex{Bahr, Hermann 19.07.1863 – 15.01.1934@\textsc{Bahr, Hermann} (19.07.1863 – 15.01.1934), \emph{Schriftsteller, Kritiker}|pwk}: \emph{Der Garten der Erkenntnis}\pwindex{Bahr, Hermann 19.07.1863 – 15.01.1934@\textsc{Bahr, Hermann} (19.07.1863 – 15.01.1934), \emph{Schriftsteller, Kritiker}!Garten der Erkenntnis1895-03-16@\strich\emph{Der Garten der Erkenntnis} {[}1895-03-16{]}|pwk}. In: \emph{Die Zeit. Wiener Wochenschrift}\pwindex{Zeit. Wiener Wochenschrift1894 – 1904@\emph{Die Zeit. Wiener Wochenschrift} {[}1894 – 1904{]}|pwk}, Bd. 2, H. 24, 16. 3. 1895, S. 171–172.}}}\label{K_L02734-88h} ſteht, macht
               mir das Blut wallen. Wenn Ihr könnt, tretet den \textsc{Bahr\pwindex{Bahr, Hermann 19.07.1863 – 15.01.1934@\textsc{Bahr, Hermann} (19.07.1863 – 15.01.1934), \emph{Schriftsteller, Kritiker}|pw}} noch bei Zeiten todt. Sonſt werdet Ihr viel Schlimmeres erleben{\dotsfour}\pend
           \pstart
           Grüß’ Dich Gott, mein lieber Freund!{\\[\baselineskip]}Dein {\\[\baselineskip]}\spacefill\mbox{Paul Goldmnn}\pend
           \leftskip=0em{}
         
         \endnumbering\mylabel{h}\end{ledgroupsized}  \newcommand{\dateiname}{L02734}\newcommand{\titel}{Paul Goldmann an Arthur Schnitzler, 24. 4. [1895]}\newcommand{\editorInnen}{Martin Anton Müller und Laura Untner}%% latex-leseansicht-abspann.tex
%% Abspann für die Leseansicht.
%% Der Schalter \ifkorrekturansicht ist bereits durch den Vorspann gesetzt.

%% latex-abspann.tex
%% Gemeinsamer Abspann für Korrekturansicht und Leseansicht.
%% Setzt den Schalter \ifkorrekturansicht voraus (gesetzt in den
%% einbindenden Dateien latex-korrekturansicht-abspann.tex bzw.
%% latex-leseansicht-abspann.tex).
%% ---------------------------------------------------------------

\normalsize

% Das esempio-Environment wird nur in der Leseansicht benötigt
\ifkorrekturansicht\else
\newenvironment{esempio}[3]%
{
    \vspace{1.5ex}
    \rlap{\underline{#1}}
    \par
    \setlength{\parindent}{0cm}
    \nopagebreak
    \leftskip=#2cm
    \rightskip=#3cm
}
{
    \par
}
\fi

\doendnotes{C}
\bigskip
\vfill

\clearpage

\footnotesize

\ifkorrekturansicht
  \lohead{\textsc{register}}
\fi

% theindex-Environment neu definieren ohne reledmac
\makeatletter
\renewenvironment{theindex}{%
  \ifkorrekturansicht
    \section*{\indexname}%
  \else
    \subsubsection*{Index der erwähnten Entitäten}%
  \fi
  \setlength{\parindent}{0pt}%
  \setlength{\parskip}{0pt plus 0.3pt}%
  \let\item\@idxitem
}{%
  \ifkorrekturansicht\clearpage\fi
}
\makeatother

\IfFileExists{\jobname-pw.ind}{\input{\jobname-pw.ind}}{}

% Quellenangabe nur in der Leseansicht
\ifkorrekturansicht\else
% Fallback-Definitionen, falls die .tex-Datei \titel etc. nicht gesetzt hat
\providecommand{\titel}{}
\providecommand{\editorInnen}{}
\providecommand{\dateiname}{\jobname}

\vspace{3cm}

\vfill

\footnotesize
\textsc{Quelle}: \titel. Herausgegeben von {\editorInnen}. In: \emph{Arthur Schnitzler: Briefwechsel mit Autorinnen und Autoren}.
 Digitale Edition, https://schnitzler-briefe.acdh.oeaw.ac.at/{\dateiname}.html (Stand \today)
\fi

\end{document}


      