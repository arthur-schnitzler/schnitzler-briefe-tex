%% latex-korrekturansicht-vorspann.tex
%% Vorspann für die Korrekturansicht.
%% Lädt die gemeinsame Datei latex-vorspann.tex mit gesetztem Schalter.

\newif\ifkorrekturansicht
\korrekturansichttrue

\input{../tex-inputs/latex-vorspann}


\section[Paul Goldmann an Arthur Schnitzler, 24. 4. {[}1895{]}]{L02734 Paul Goldmann an Arthur Schnitzler, 24. 4. {[}1895{]}}
\nopagebreak\mylabel{L02734v}
\rehead{ }\normalsize\beginnumbering\briefempfaengerindex{Schnitzler, Arthur@\textsc{Schnitzler, Arthur}!zzzGoldmann, Paul@\emph{von Paul Goldmann}!1895-04-242@{24. 4. {[}1895{]}}|(be}
\toendnotes[C]{\smallbreak\pagebreak[2]}\Standort{DLA, A:Schnitzler, HS.NZ85.1.3165.}
\physDesc{Brief, 2 Blätter, 8 Seiten, 1861 Zeichen
\newline{}Handschrift: blaue Tinte, deutsche Kurrent
\newline{}Schnitzler: 1) mit Bleistift das Jahr »95« vermerkt  2) mit rotem Buntstift sechs Unterstreichungen}\toendnotes[C]{\smallbreak}
\pstart
           \raggedleft{}{\pb}\textsc{Frankfurt\oindex{Frankfurt am Main@\textbf{Frankfurt am Main}, \emph{P.PPLA3}|pw}}{ }24. April.\pend
           
\pstart\center{}Mein lieber Freund,\pend\vspace{0.5em}
\pstart
           Seit zehn Tagen bin ich in Frankfurt\oindex{Frankfurt am Main@\textbf{Frankfurt am Main}, \emph{P.PPLA3}|pw} bei den
               Meinen. Deutſch\oindex{Deutschland@\textbf{Deutschland}, \emph{A.PCLI}|pwv}es Land,
               Frühling und Friede – das thut wohl. Aber drohend ſind die Zukunftsfragen da. Und ich
               war krank und lag einige Tage zu Bette{[}.{]}{ }{\pb}Dieſer Tage gehe ich nach \textsc{Paris\oindex{Paris@\textbf{Paris}, \emph{P.PPLC}|pw}} zurück. Will Dir nur von unterwegs einen Gruß ſenden. Aus \textsc{Paris\oindex{Paris@\textbf{Paris}, \emph{P.PPLC}|pw}} hörſt Du Näheres von mir.\pend
           
\pstart
           \textsc{Herzl\pwindex{Herzl, Theodor 1860-05-02 – 1904-07-03@\textsc{Herzl, Theodor} (1860-05-02 – 1904-07-03), \emph{Schriftsteller/Schriftstellerin, Journalist/Journalistin}|pw}} iſt gar ſo ſchweigſam über das \label{K_L02734-1v}\edtext{Beiſammenſein mit Dir}{\lemma{\textnormal{\emph{Beiſammenſein mit Dir}}}\Cendnote{\textnormal{Theodor Herzl\pwindex{Herzl, Theodor 1860-05-02 – 1904-07-03@\textsc{Herzl, Theodor} (1860-05-02 – 1904-07-03), \emph{Schriftsteller/Schriftstellerin, Journalist/Journalistin}|pwk} hielt sich im März 1895 in Wien\oindex{Wien@\textbf{Wien}, \emph{A.ADM2}|pwk} auf.
                  Zwischen 26. 3. 1895
                  und 30. 3. 1895 sah
                  er Schnitzler jeden Tag. Ein Konflikt
                  zwischen den beiden ist nicht bekannt.}}}\label{K_L02734-1}. Iſt das nur ſeine eitle {\pb}\label{K_L02734-2v}\edtext{\textsc{Suffisance}}{\lemma{\textnormal{\emph{Suffisance}}}\Cendnote{\textnormal{französisch: Selbstgefälligkeit}}}\label{K_L02734-2}?
               Oder habt Ihr was gehabt? Wie hat er Dir überhaupt gefallen?\pend
           
\pstart
           Ich \substVorne{}\textsuperscript{hö\textcolor{gray}{re,}}\substDazwischen{}höre,\substHinten{}{ }Du\pwindex{Liebelei. Schauspiel in drei Akten@\emph{Liebelei. Schauspiel in drei Akten}|pwv} wirſt erſt im Herbſt
               aufgeführt. Beſſer im Anfang, als am Ende der Saison. Am Beſten wäre es freilich, die
                  \label{K_L02734-3v}\edtext{Berlin\oindex{Berlin@\textbf{Berlin}, \emph{P.PPLC}|pw}er Aufführung\pwindex{Liebelei. Schauspiel in drei Akten@\emph{Liebelei. Schauspiel in drei Akten}|pwv}}{\lemma{\textnormal{\emph{Berliner Aufführung}}}\Cendnote{\textnormal{Am 4. 2. 1896 feierte \emph{Liebelei}\pwindex{Liebelei. Schauspiel in drei Akten@\emph{Liebelei. Schauspiel in drei Akten}|pwk} am \emph{Deutschen
                     Theater}\orgindex{Deutsches Theater Berlin@Deutsches Theater Berlin|pwk} in Berlin\oindex{Berlin@\textbf{Berlin}, \emph{P.PPLC}|pwk} Premiere.}}}\label{K_L02734-3}{ }{\pb}ginge der Wien\oindex{Wien@\textbf{Wien}, \emph{A.ADM2}|pw}er
               voran. Publikum und Kritik ſind in Berlin\oindex{Berlin@\textbf{Berlin}, \emph{P.PPLC}|pw} doch im
               Ganzen intelligenter. Ein Berlin\oindex{Berlin@\textbf{Berlin}, \emph{P.PPLC}|pw}er Erfolg wäre
               für Wien\oindex{Wien@\textbf{Wien}, \emph{A.ADM2}|pw} beſtimmend, auch für den ewig zaudernden
                  Burgtheater\orgindex{Burgtheater@Burgtheater|pw}-Direktor\pwindex{Burckhard, Max Eugen 14.07.1854 – 16.03.1912@\textsc{Burckhard, Max Eugen} (14.07.1854 – 16.03.1912), \emph{Schriftsteller/Schriftstellerin, Rechtswissenschaftler/Rechtswissenschaftlerin, Theaterleiter/Theaterleiterin}|pwv}. (Wie ich hier höre,
               ſtrebt \textsc{Paul Lindau\pwindex{Lindau, Paul 03.06.1839 – 31.01.1919@\textsc{Lindau, Paul} (03.06.1839 – 31.01.1919), \emph{Schriftsteller/Schriftstellerin, Kritiker/Kritikerin, Theaterleiter/Theaterleiterin}|pw}} nach \label{K_L02734-4v}\edtext{\textsc{Burckhardts\pwindex{Burckhard, Max Eugen 14.07.1854 – 16.03.1912@\textsc{Burckhard, Max Eugen} (14.07.1854 – 16.03.1912), \emph{Schriftsteller/Schriftstellerin, Rechtswissenschaftler/Rechtswissenschaftlerin, Theaterleiter/Theaterleiterin}|pw}} Nachfolgerſchaft}{\lemma{\textnormal{\emph{Burckhardts Nachfolgerſchaft}}}\Cendnote{\textnormal{Max Burckhardt\pwindex{Burckhard, Max Eugen 14.07.1854 – 16.03.1912@\textsc{Burckhard, Max Eugen} (14.07.1854 – 16.03.1912), \emph{Schriftsteller/Schriftstellerin, Rechtswissenschaftler/Rechtswissenschaftlerin, Theaterleiter/Theaterleiterin}|pwk} war als Jurist eine
                     überraschende Besetzung für die Leitung des \emph{Burgtheaters}\orgindex{Burgtheater@Burgtheater|pwk} gewesen. Ablösegerüchte oder -wünsche bestanden von Anfang
                  an, doch konnte er sich bis 1898 halten. Nachfolger wurde
                     Paul Schlenther\pwindex{Schlenther, Paul 20.08.1854 – 30.04.1916@\textsc{Schlenther, Paul} (20.08.1854 – 30.04.1916), \emph{Schriftsteller/Schriftstellerin, Kritiker/Kritikerin, Theaterleiter/Theaterleiterin}|pwk}.}}}\label{K_L02734-4}). {\pb}Hier ein \label{K_L02734-5v}\edtext{Stück\pwindex{Frauenlob. Ein Lustspiel in drei Aufzuegen@\emph{Frauenlob. Ein Lustspiel in drei Aufzügen}|pwu}}{\lemma{\textnormal{\emph{Stück}}}\Cendnote{\textnormal{vermutlich \emph{Frauenlob. Lustspiel in drei Aufzügen}\pwindex{Frauenlob. Ein Lustspiel in drei Aufzuegen@\emph{Frauenlob. Ein Lustspiel in drei Aufzügen}|pwk}}}}\label{K_L02734-5} von \textsc{Rudolf Lothar\pwindex{Lothar, Rudolf 23.2.1865 – 2.10.1943@\textsc{Lothar, Rudolf} (23.2.1865 – 2.10.1943), \emph{Schriftsteller/Schriftstellerin, Journalist/Journalistin, Theaterdirektor/Theaterdirektorin}|pw}} geſehen. Es iſt unerhört, daß man dieſen Buben\pwindex{Lothar, Rudolf 23.2.1865 – 2.10.1943@\textsc{Lothar, Rudolf} (23.2.1865 – 2.10.1943), \emph{Schriftsteller/Schriftstellerin, Journalist/Journalistin, Theaterdirektor/Theaterdirektorin}|pwv} nicht mit Fußtritten vom Theater jagt.\pend
           
\pstart
           Haſt Du frohe Oſtern gehabt? Und wie gehts Dir? Du ſchreibſt mir wohl ein kurzes
               Wort, ohne meine {\pb}längere Antwort abzuwarten.\pend
           
\pstart
           \textsc{Bahr\pwindex{Bahr, Hermann 19.07.1863 – 15.01.1934@\textsc{Bahr, Hermann} (19.07.1863 – 15.01.1934), \emph{Schriftsteller/Schriftstellerin, Kritiker/Kritikerin}|pw}} hat alſo wieder einen \label{K_L02734-6v}\edtext{Vortrag\pwindex{junge Oesterreich [Vortrag]@\emph{Das junge Österreich [Vortrag]}|pwv}}{\lemma{\textnormal{\emph{Vortrag}}}\Cendnote{\textnormal{Am 13. 3. 1895 fand eine Veranstaltung des \emph{Vereins der Literaturfreunde}\orgindex{Verein der Literaturfreunde@Verein der Literaturfreunde|pwk} statt, bei der Hermann Bahr\pwindex{Bahr, Hermann 19.07.1863 – 15.01.1934@\textsc{Bahr, Hermann} (19.07.1863 – 15.01.1934), \emph{Schriftsteller/Schriftstellerin, Kritiker/Kritikerin}|pwk} einen Vortrag\pwindex{junge Oesterreich [Vortrag]@\emph{Das junge Österreich [Vortrag]}|pwkv} mit dem Titel \emph{Das junge Österreich}\pwindex{junge Oesterreich [Vortrag]@\emph{Das junge Österreich [Vortrag]}|pwk} hielt. Schnitzler, dessen Kunstschaffen als
                     »abgethan« geschildert wurde, war empört. Siehe A. S.: \emph{Tagebuch}, 14. 3. 1895. }}}\label{K_L02734-6}
               gehalten. Der Volksſänger der
                  Moderne\pwindex{Bahr, Hermann 19.07.1863 – 15.01.1934@\textsc{Bahr, Hermann} (19.07.1863 – 15.01.1934), \emph{Schriftsteller/Schriftstellerin, Kritiker/Kritikerin}|pwv}! Die \label{K_L02734-7v}\edtext{Brettl-Natur}{\lemma{\textnormal{\emph{Brettl-Natur}}}\Cendnote{\textnormal{Der Verweis auf einen Schauspieler,
                     der auf einer aus einfachen Brettern
                     zusammengefügten Bühne statt auf einem gezimmerten Boden auftritt, soll hier abwertend ausdrücken, dass es nur für das ungebildete Volk von Interesse ist.}}}\label{K_L02734-7}, das iſt der Grund in dem Weſen des Kerls\pwindex{Bahr, Hermann 19.07.1863 – 15.01.1934@\textsc{Bahr, Hermann} (19.07.1863 – 15.01.1934), \emph{Schriftsteller/Schriftstellerin, Kritiker/Kritikerin}|pwv}. Wie ich den immer mehr
               haſſe! Dieſen Mann\pwindex{Bahr, Hermann 19.07.1863 – 15.01.1934@\textsc{Bahr, Hermann} (19.07.1863 – 15.01.1934), \emph{Schriftsteller/Schriftstellerin, Kritiker/Kritikerin}|pwv} von
               Geiſt, aber {\pb}ohne Kunſt, ohne Urtheil, ohne
               Gewiſſen! Merkſt Du, wie er ſich langſam in die \label{K_L02734-8v}\edtext{\textsc{Clique}}{\lemma{\textnormal{\emph{Clique}}}\Cendnote{\textnormal{Hier liegt eine positive Verwendung des
                  Wortes vor, das bei Schnitzler hingegen
                  meist nur in einer negativen Form vorkommt, insofern er nicht als Teil einer
                  eingeschworenen Gruppe von Literaten wahrgenommen werden wollte. }}}\label{K_L02734-8}
               hineinſchleicht? In wenig Jahren hat er irgendwo ein officiöſes k. k. Literatur-Amt.
               Daß dieſes Rindvieh\pwindex{Necker, Moriz 1857-10-14 – 1915-02-16@\textsc{Necker, Moriz} (1857-10-14 – 1915-02-16), \emph{Journalist/Journalistin, Kritiker/Kritikerin, Literaturwissenschaftler/Literaturwissenschaftlerin}|pwv}, der
                  \strikeout{\textsc{A}}{ }\label{K_L02734-9v}\edtext{\textsc{Necker\pwindex{junge Oesterreich@\emph{Das junge Österreich}|pwv}\pwindex{Necker, Moriz 1857-10-14 – 1915-02-16@\textsc{Necker, Moriz} (1857-10-14 – 1915-02-16), \emph{Journalist/Journalistin, Kritiker/Kritikerin, Literaturwissenschaftler/Literaturwissenschaftlerin}|pw}}}{\lemma{\textnormal{\emph{Necker}}}\Cendnote{\textnormal{Die Veranstaltung wurde wohlwollend von
                     Moriz Necker\pwindex{Necker, Moriz 1857-10-14 – 1915-02-16@\textsc{Necker, Moriz} (1857-10-14 – 1915-02-16), \emph{Journalist/Journalistin, Kritiker/Kritikerin, Literaturwissenschaftler/Literaturwissenschaftlerin}|pwk} in der \emph{Neuen Freien Presse}\pwindex{Neue Freie Presse@\emph{Neue Freie Presse}|pwk} besprochen, einschließlich der
                  überraschenden Volte, dass eine neue Kunstepoche entstehe und dass frühere Wien\oindex{Wien@\textbf{Wien}, \emph{A.ADM2}|pwk}er Vertreter wie »Hermann Bahr\pwindex{Bahr, Hermann 19.07.1863 – 15.01.1934@\textsc{Bahr, Hermann} (19.07.1863 – 15.01.1934), \emph{Schriftsteller/Schriftstellerin, Kritiker/Kritikerin}|pw}, Baron Torresani\pwindex{Torresani-Lanzenfeld, Carl von 19.04.1846 – 16.04.1907@\textsc{Torresani-Lanzenfeld, Carl von} (19.04.1846 – 16.04.1907), \emph{Schriftsteller/Schriftstellerin, Offizier/Offizierin}|pw}, Beer-Hoffmann\pwindex{Beer-Hofmann, Richard 1866-07-11 – 1945-09-26@\textsc{Beer-Hofmann, Richard} (1866-07-11 – 1945-09-26), \emph{Schriftsteller/Schriftstellerin}|pw}« nur eine Übergangszeit repräsentiert hätten. Schnitzlers Name fällt in der Rezension\pwindex{junge Oesterreich@\emph{Das junge Österreich}|pwkv} nicht. Vgl. [Moriz Necker]\pwindex{Necker, Moriz 1857-10-14 – 1915-02-16@\textsc{Necker, Moriz} (1857-10-14 – 1915-02-16), \emph{Journalist/Journalistin, Kritiker/Kritikerin, Literaturwissenschaftler/Literaturwissenschaftlerin}|pwk}: \emph{Das junge Österreich}\pwindex{junge Oesterreich@\emph{Das junge Österreich}|pwk}. In: \emph{Neue Freie Presse}\pwindex{Neue Freie Presse@\emph{Neue Freie Presse}|pwk}, Nr. 10.075, 14. 3. 1895, S. 5.}}}\label{K_L02734-9}, Dich angreift, iſt ſelbſt{\pb}verſtändlich. \strikeout{Wenn
                  Du} Daran daß Du die \strikeout{Och}{ } Ochſen ſtützig machſt, kannſt Du auch ſehen, daß
               Du Jemand biſt. Aber daß dieſer \label{K_L02734-10v}\edtext{Angriff in der »Zeit\pwindex{Zeit. Wiener Wochenschrift@\emph{Die Zeit. Wiener Wochenschrift}|pw}«}{\lemma{\textnormal{\emph{Angriff in der »Zeit«}}}\Cendnote{\textnormal{Gemeint dürfte nicht ein spezifischer Artikel sein – auch
                  wenn Bahr\pwindex{Bahr, Hermann 19.07.1863 – 15.01.1934@\textsc{Bahr, Hermann} (19.07.1863 – 15.01.1934), \emph{Schriftsteller/Schriftstellerin, Kritiker/Kritikerin}|pwk} Gedanken davon in seiner Rezension\pwindex{Garten der Erkenntnis@\emph{Der Garten der Erkenntnis}|pwkv} von Leopold von Andrian-Werburgs\pwindex{Andrian-Werburg, Leopold von 09.05.1875 – 19.11.1951@\textsc{Andrian-Werburg, Leopold von} (09.05.1875 – 19.11.1951), \emph{Schriftsteller/Schriftstellerin, Diplomat/Diplomatin}|pwk}{ }\emph{Der Garten der Erkenntnis}\pwindex{Garten der Erkenntnis@\emph{Der Garten der Erkenntnis}|pwk} verwendet –, sondern eher die
                  allgemeine Unmut ausdrücken, dass von einem Repräsentanten der Wochenschrift\pwindex{Zeit. Wiener Wochenschrift@\emph{Die Zeit. Wiener Wochenschrift}|pwkv}, die man auf der eigenen
                  Seite vermutete, Kritik kam. Vgl. Hermann Bahr\pwindex{Bahr, Hermann 19.07.1863 – 15.01.1934@\textsc{Bahr, Hermann} (19.07.1863 – 15.01.1934), \emph{Schriftsteller/Schriftstellerin, Kritiker/Kritikerin}|pwk}: \emph{Der Garten der Erkenntnis}\pwindex{Garten der Erkenntnis@\emph{Der Garten der Erkenntnis}|pwk}. In: \emph{Die Zeit. Wiener Wochenschrift}\pwindex{Zeit. Wiener Wochenschrift@\emph{Die Zeit. Wiener Wochenschrift}|pwk}, Bd. 2, H. 24, 16. 3. 1895, S. 171–172.}}}\label{K_L02734-10} ſteht, macht
               mir das Blut wallen. Wenn Ihr könnt, tretet den \textsc{Bahr\pwindex{Bahr, Hermann 19.07.1863 – 15.01.1934@\textsc{Bahr, Hermann} (19.07.1863 – 15.01.1934), \emph{Schriftsteller/Schriftstellerin, Kritiker/Kritikerin}|pw}} noch bei Zeiten todt. Sonſt werdet Ihr viel Schlimmeres erleben{\dotsfour}\pend
           
\pstart
           Grüß’ Dich Gott, mein lieber Freund!{\\[\baselineskip]}Dein {\\[\baselineskip]}\spacefill\mbox{Paul Goldmnn}\pend
           \leftskip=0em{}\selectlanguage{ngerman}\endnumbering\briefempfaengerindex{Schnitzler, Arthur@\textsc{Schnitzler, Arthur}!zzzGoldmann, Paul@\emph{von Paul Goldmann}!1895-04-242@{24. 4. {[}1895{]}}|)be}\mylabel{L02734h}  \normalsize

\doendnotes{C}
\bigskip
\vfill

\clearpage

\footnotesize

\lohead{\textsc{register}}

% Definiere theindex-Environment komplett neu ohne reledmac
\makeatletter
\renewenvironment{theindex}{%
  \section*{\indexname}%
  \setlength{\parindent}{0pt}%
  \setlength{\parskip}{0pt plus 0.3pt}%
  \let\item\@idxitem
}{%
  \clearpage
}
\makeatother

\IfFileExists{\jobname-pw.ind}{\input{\jobname-pw.ind}}{}

\end{document}

      