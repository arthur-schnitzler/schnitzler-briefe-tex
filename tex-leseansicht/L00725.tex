%% latex-leseansicht-vorspann.tex
%% Vorspann für die Leseansicht.
%% Lädt die gemeinsame Datei latex-vorspann.tex mit nicht gesetztem Schalter.

\newif\ifkorrekturansicht
\korrekturansichtfalse

\input{../tex-inputs/latex-vorspann}

\begin{center}
            \textcolor{red}{ENTWURF. ENTZIFFERUNG NOCH NICHT KORREKTURGELESEN}
                      \end{center}
            
               \section[Hermann Bahr an Arthur Schnitzler, 28. 9. 1897]{ Hermann Bahr an Arthur Schnitzler, 28. 9. 1897}\nopagebreak\mylabel{v}\rehead{ }\begin{ledgroupsized}[t]{13cm}\normalsize\beginnumbering\briefempfaengerindex{Schnitzler, Arthur@\textsc{Schnitzler, Arthur}!zzzBahr, Hermann@\emph{von Hermann Bahr}!1897-09-281@{28. 9. 1897}|(be} \toendnotes[C]{\smallbreak\pagebreak[2]} \Standort{CUL, Schnitzler, B 5b.}
\physDesc{Brief, 1 Blatt, 2 Seiten
\newline{}Handschrift: schwarze Tinte, deutsche Kurrent\newline{}Ordnung: mit Bleistift von unbekannter Hand nummeriert: »54« }\buchAbdrucke{\weitereDrucke{Hermann Bahr, Arthur Schnitzler: \emph{Briefwechsel, Aufzeichnungen, Dokumente (1891–1931)}. Hg. Kurt Ifkovits und Martin Anton Müller. Göttingen: \emph{Wallstein} 2018, S. 152.} }\toendnotes[C]{\smallbreak}\pstart
           \noindent{}{\pb}\textcolor{gray}{\textbf{»Die
                        Zeit\orgindex{Zeit. Wiener Wochenschrift@Die Zeit. Wiener Wochenschrift|pw}«}}\hfill \textcolor{gray}{\textbf{\textbf{Wien\oindex{Wien@\textbf{Wien}|pw}}, den }}28. Sept. \textcolor{gray}{\textbf{189}}7\pend
           \pstart
           \textcolor{gray}{\textbf{Wiener Wochenſchrift}}\hfill \textcolor{gray}{\textbf{IX/3, Günthergaſſe 1\oindex{Guenthergasse@\textbf{Günthergasse}|pw}.}}\pend
           \pstart
           \textcolor{gray}{\textbf{\textbf{Herausgeber}:}}{\\}\textcolor{gray}{\textbf{Profeſſor Dr. I. Singer\pwindex{Singer, Isidor 16.01.1857 – 08.12.1927@\textsc{Singer, Isidor} (16.01.1857 – 08.12.1927), \emph{Journalist, Herausgeber, Soziologe}|pw}, Hermann Bahr\pwindex{Bahr, Hermann 19.07.1863 – 15.01.1934@\textsc{Bahr, Hermann} (19.07.1863 – 15.01.1934), \emph{Schriftsteller, Kritiker}|pw},
                        Dr. Heinrich Kanner\pwindex{Kanner, Heinrich 09.11.1864 – 15.02.1930@\textsc{Kanner, Heinrich} (09.11.1864 – 15.02.1930), \emph{Publizist}|pw}.}}\pend
           \pstart
           \textcolor{gray}{\textbf{Telephon Nr. 6415.}}\pend
           \pstart\center{}Lieber Arthur!\pend\pstart
           Ich danke Dir herzlich für Deine lieben Worte. Es ſcheint in der That, daß das »Tſchaperl\pwindex{Bahr, Hermann 19.07.1863 – 15.01.1934@\textsc{Bahr, Hermann} (19.07.1863 – 15.01.1934), \emph{Schriftsteller, Kritiker}!Tschaperl1896@\strich\emph{Das Tschaperl} {[}1896{]}|pw}« in Berlin\oindex{Berlin@\textbf{Berlin}|pw} gefallen hat, was mir ſehr viel Vergnügen macht: Denn in dieſem Falle
               ſind wohl die Berlin\oindex{Berlin@\textbf{Berlin}|pw}er über den abſoluten Werth
               (als die Entfernteren) eher competent. Oder wenigſtens bilde ich es mir jetzt ein,
               was auf dasſelbe hinausläuft.\pend
           \pstart
           Wann ſchickſt Du mir wieder einmal {\pb}etwas für die
                  »Zeit\orgindex{Zeit. Wiener Wochenschrift@Die Zeit. Wiener Wochenschrift|pw}«? Ich rechne beſtimmt
               darauf.\pend
           \pstart
           Und was macht Dein Stück\pwindex{Schnitzler, Arthur 15.05.1862 – 21.10.1931@\textsc{Schnitzler, Arthur} (15.05.1862 – 21.10.1931), \emph{Schriftsteller, Mediziner}!Vermaechtnis. Schauspiel in drei Akten1898@\strich\emph{Das Vermächtnis. Schauspiel in drei Akten} {[}1898{]}|pwv}? Ich
               möchte, wegen Neumann-Hofer\pwindex{Neumann-Hofer, Gilbert Otto 04.02.1857 – 14.04.1941@\textsc{Neumann-Hofer, Gilbert Otto} (04.02.1857 – 14.04.1941), \emph{Kritiker, Theaterleiter}|pw}, baldigſt darüber
               Näheres wiſſen.\pend
           \pstart
           Nochmals dankend{\\[\baselineskip]}herzlichſt{\\[\baselineskip]}Dein alter{\\[\baselineskip]}\spacefill\mbox{Hermann}\pend
           \leftskip=0em{}\pstart
           \noindent{}Herrn \textsc{D\textsuperscript{r} Arthur Schnitzler}{\\}Wien IX \textsc{Frankgasse}
                     1.\oindex{Frankgasse@\textbf{Frankgasse}|pw}\pend
           \pstart
           \textcolor{gray}{\textbf{\label{T_L00725_1v}\edtext{Alle für »Die Zeit\orgindex{Zeit. Wiener Wochenschrift@Die Zeit. Wiener Wochenschrift|pw}« beſtimmten Zuſchriften und Sendungen ſind an
                  die Redaction der »Zeit\orgindex{Zeit. Wiener Wochenschrift@Die Zeit. Wiener Wochenschrift|pw}« und \textbf{nicht} an die Perſon eines der Herausgeber zu richten.}{\lemma{\textnormal{\emph{Alle … richten.}}}\Cendnote{\textnormal{am unteren Rand der ersten Seite}}}\label{T_L00725_1h}}}\pend
           \endnumbering\briefempfaengerindex{Schnitzler, Arthur@\textsc{Schnitzler, Arthur}!zzzBahr, Hermann@\emph{von Hermann Bahr}!1897-09-281@{28. 9. 1897}|)be}\mylabel{h}\end{ledgroupsized}  \newcommand{\dateiname}{L00725}\newcommand{\titel}{Hermann Bahr an Arthur Schnitzler, 28. 9. 1897}\newcommand{\editorInnen}{ Kurt Ifkovits,  Martin Anton Müller}%% latex-leseansicht-abspann.tex
%% Abspann für die Leseansicht.
%% Der Schalter \ifkorrekturansicht ist bereits durch den Vorspann gesetzt.

%% latex-abspann.tex
%% Gemeinsamer Abspann für Korrekturansicht und Leseansicht.
%% Setzt den Schalter \ifkorrekturansicht voraus (gesetzt in den
%% einbindenden Dateien latex-korrekturansicht-abspann.tex bzw.
%% latex-leseansicht-abspann.tex).
%% ---------------------------------------------------------------

\normalsize

% Das esempio-Environment wird nur in der Leseansicht benötigt
\ifkorrekturansicht\else
\newenvironment{esempio}[3]%
{
    \vspace{1.5ex}
    \rlap{\underline{#1}}
    \par
    \setlength{\parindent}{0cm}
    \nopagebreak
    \leftskip=#2cm
    \rightskip=#3cm
}
{
    \par
}
\fi

\doendnotes{C}
\bigskip
\vfill

\clearpage

\footnotesize

\ifkorrekturansicht
  \lohead{\textsc{register}}
\fi

% theindex-Environment neu definieren ohne reledmac
\makeatletter
\renewenvironment{theindex}{%
  \ifkorrekturansicht
    \section*{\indexname}%
  \else
    \subsubsection*{Index der erwähnten Entitäten}%
  \fi
  \setlength{\parindent}{0pt}%
  \setlength{\parskip}{0pt plus 0.3pt}%
  \let\item\@idxitem
}{%
  \ifkorrekturansicht\clearpage\fi
}
\makeatother

\IfFileExists{\jobname-pw.ind}{\input{\jobname-pw.ind}}{}

% Quellenangabe nur in der Leseansicht
\ifkorrekturansicht\else
% Fallback-Definitionen, falls die .tex-Datei \titel etc. nicht gesetzt hat
\providecommand{\titel}{}
\providecommand{\editorInnen}{}
\providecommand{\dateiname}{\jobname}

\vspace{3cm}

\vfill

\footnotesize
\textsc{Quelle}: \titel. Herausgegeben von {\editorInnen}. In: \emph{Arthur Schnitzler: Briefwechsel mit Autorinnen und Autoren}.
 Digitale Edition, https://schnitzler-briefe.acdh.oeaw.ac.at/{\dateiname}.html (Stand \today)
\fi

\end{document}


      