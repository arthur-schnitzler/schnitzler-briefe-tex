\input{../tex-inputs/latex-pdf-vorspann}
\begin{center}
            \textcolor{red}{ENTWURF. ENTZIFFERUNG NOCH NICHT KORREKTURGELESEN}
                      \end{center}
            
               \section[Hugo von Hofmannsthal an Arthur Schnitzler, {[}Mai 1899{]}]{ Hugo von Hofmannsthal an Arthur Schnitzler, {[}Mai 1899{]}}\nopagebreak\mylabel{v}\rehead{ }\begin{ledgroupsized}[t]{13cm}\normalsize\beginnumbering\briefempfaengerindex{Schnitzler, Arthur@\textsc{Schnitzler, Arthur}!zzzHofmannsthal, Hugo von@\emph{von Hugo von Hofmannsthal}!1899-05-011@{{[}Mai 1899{]}}|(be} \toendnotes[C]{\smallbreak\pagebreak[2]} \Standort{CUL, Schnitzler, B 43.}
\physDesc{Brief, 2 Blätter (Auf einer Rückseite des 1. Blattes gestrichener Text von unbekannter Hand: »zu erhalten. Wir würden Ihnen zu Dank verpflichtet sein wollten Sie uns zwei Gedichtchen oder einen Artikel in Prosa zur Verfügung stellen. Wir bitten um Zusendung sat genauer Unterschrift behufs Facsimilirung bis zum 18. dM. Gestatten Sie«), 2 Seiten, Fragment
\newline{}Handschrift: 1) schwarze Tinte, deutsche Kurrent (\noindent{}zweites Blatt)\hspace{1em}2) schwarze Tinte, lateinische Kurrent (\noindent{}erstes Blatt)\hspace{1em}
\newline{}Schnitzler: mit Bleistift datiert: »\textsc{Mai} 99« und am Ende des Texts »mettre a
                                 table« \newline{}Ordnung: 1) mit Bleistift von unbekannter Hand nummeriert: »\strikeout{148}« 2) mit Bleistift von unbekannter Hand nummeriert: »144«}\buchAbdrucke{\weitereDrucke{Hugo von Hofmannsthal, Arthur Schnitzler: \emph{Briefwechsel}. Hg. Therese Nickl und Heinrich Schnitzler. Frankfurt am Main: \emph{S. Fischer} 1964, S. 122.} }\toendnotes[C]{\smallbreak}\pstart
           \noindent{}{\pb}\label{K_L00914_1v}\edtext{traduction\pwindex{Schnitzler, Arthur 15.05.1862 – 21.10.1931@\textsc{Schnitzler, Arthur} (15.05.1862 – 21.10.1931), \emph{Schriftsteller, Mediziner}!gruene Kakadu. Groteske in einem Akt1.3.1899 – 1.3.1899@\strich\emph{Der grüne Kakadu. Groteske in einem Akt} {[}1.3.1899 – 1.3.1899{]}|pwv} médiocre, sans vigeur
               sans subtilité, exempte de qualités littéraires; elle trahit un ésprit pédant et
               d’une sotte vanité.}{\lemma{\textnormal{\emph{traduction … vanité.}}}\Cendnote{\textnormal{französisch:
                     »mittelmäßige Übersetzung, ohne Kraft, Subtilität, bar jeder
                     literarischen Qualitäten; sie verrät einen kleinkrämerischen Geist und dumme
                     Eitelkeit«. Es handelt sich um die nicht überlieferte französische
                  Übersetzung von \emph{Der grüne Kakadu}\pwindex{Schnitzler, Arthur 15.05.1862 – 21.10.1931@\textsc{Schnitzler, Arthur} (15.05.1862 – 21.10.1931), \emph{Schriftsteller, Mediziner}!gruene Kakadu. Groteske in einem Akt1.3.1899 – 1.3.1899@\strich\emph{Der grüne Kakadu. Groteske in einem Akt} {[}1.3.1899 – 1.3.1899{]}|pwk} durch Émile Soutif\pwindex{Soutif, Emile @\textsc{Soutif, Émile}, \emph{Lehrer}|pwk}.}}}\label{K_L00914_1h}\pend
           \pstart \spacefill\mbox{H. H.}\pend{}\pstart
           \noindent{}\label{K_L00914_2v}\edtext{j’ai ajouté quelques
                     eclaircissements}{\lemma{\textnormal{\emph{j’ai … eclaircissements}}}\Cendnote{\textnormal{frz: »ich
                        habe ein paar Klärungen ergänzt«}}}\label{K_L00914_2h}.\pend
           \pstart
           {\pb}Das \label{K_L00914_3v}\edtext{Wortſpiel mit dem Sitzen}{\lemma{\textnormal{\emph{Wortſpiel mit dem Sitzen}}}\Cendnote{\textnormal{Es ist im Stück in der doppelten
                     Bedeutung von »herumsitzen« und im »Gefängnis sitzen« verwendet.}}}\label{K_L00914_3h} iſt
                  unverſtanden geblieben, iſt auch ſchwer zu überſetzen.\pend
           \pstart
           Die \textsc{replique} des \textsc{Prosper}\pwindex{Schnitzler, Arthur 15.05.1862 – 21.10.1931@\textsc{Schnitzler, Arthur} (15.05.1862 – 21.10.1931), \emph{Schriftsteller, Mediziner}!gruene Kakadu. Groteske in einem Akt1.3.1899 – 1.3.1899@\strich\emph{Der grüne Kakadu. Groteske in einem Akt} {[}1.3.1899 – 1.3.1899{]}|pwv} müſste lauten: \label{K_L00914_4v}\edtext{\textsc{si au moins tu ne faisais que leur tenir compagnie!}}{\lemma{\textnormal{\emph{si … compagnie!}}}\Cendnote{\textnormal{Im Stück heißt es: »– wenn du
                        nur immer mit ihnen gesessen wärst.«.}}}\label{K_L00914_4h} (das iſt aber auch ohne
                  Schärfe)\pend
           \endnumbering\briefempfaengerindex{Schnitzler, Arthur@\textsc{Schnitzler, Arthur}!zzzHofmannsthal, Hugo von@\emph{von Hugo von Hofmannsthal}!1899-05-011@{{[}Mai 1899{]}}|)be}\mylabel{h}\end{ledgroupsized}  \newcommand{\dateiname}{L00914}\newcommand{\titel}{Hugo von Hofmannsthal an Arthur Schnitzler, [Mai 1899]}\newcommand{\editorInnen}{Martin Anton Müller und Gerd-Hermann Susen}\input{../tex-inputs/latex-pdf-abspann}
      