%% latex-korrekturansicht-vorspann.tex
%% Vorspann für die Korrekturansicht.
%% Lädt die gemeinsame Datei latex-vorspann.tex mit gesetztem Schalter.

\newif\ifkorrekturansicht
\korrekturansichttrue

\input{../tex-inputs/latex-vorspann}


\section[Hugo von Hofmannsthal an Arthur Schnitzler, {[}Mai 1899{]}]{L00914 Hugo von Hofmannsthal an Arthur Schnitzler, {[}Mai 1899{]}}
\nopagebreak\mylabel{L00914v}
\rehead{ }\normalsize\beginnumbering\briefempfaengerindex{Schnitzler, Arthur@\textsc{Schnitzler, Arthur}!zzzHofmannsthal, Hugo von@\emph{von Hugo von Hofmannsthal}!1899-05-011@{{[}Mai 1899{]}}|(be}
\toendnotes[C]{\smallbreak\pagebreak[2]}\Standort{CUL, Schnitzler, B 43.}
\physDesc{Brief, 2 Blätter, 2 Seiten, 378 Zeichen, Fragment
\newline{}Handschrift: 1) schwarze Tinte, deutsche Kurrent (\noindent{}zweites Blatt)\hspace{1em}2) schwarze Tinte, lateinische Kurrent (\noindent{}erstes Blatt)\hspace{1em}
\newline{}Schnitzler: mit Bleistift datiert: »\textsc{Mai} 99« und am Ende des Texts »mettre a
                                 table« 
\newline{}Ordnung: 1) mit Bleistift von unbekannter Hand nummeriert: »\strikeout{148}«  2) mit Bleistift von unbekannter Hand nummeriert:
                                    »144«
\newline{}Zusatz: Auf einer Rückseite des 1. Blattes gestrichener Text von
                                 unbekannter Hand: »{\pb}zu erhalten.
                                    Wir würden Ihnen zu Dank verpflichtet sein wollten Sie uns zwei
                                    Gedichtchen oder einen Artikel in Prosa zur Verfügung stellen.
                                    Wir bitten um Zusendung sa{\geminationm}t
                                    genauer Unterschrift behufs Facsimilirung bis \uline{zum 18. dM.} Gestatten
                                 Sie« }
\buchAbdrucke{\weitereDrucke{Hugo von Hofmannsthal, Arthur Schnitzler: \emph{Briefwechsel}. Frankfurt am Main: \emph{S. Fischer} 1964, S. 122.} }\toendnotes[C]{\smallbreak}
\pstart
           \noindent{}{\pb}\label{K_L00914-1v}\edtext{traduction\pwindex{gruene Kakadu. Groteske in einem Akt@\emph{Der grüne Kakadu. Groteske in einem Akt}|pwv} médiocre, sans
               vigeur sans subtilité, exempte de qualités littéraires; elle trahit un ésprit pédant
               et d’une sotte vanité.}{\lemma{\textnormal{\emph{traduction … vanité.}}}\Cendnote{\textnormal{französisch:
                     mittelmäßige Übersetzung, ohne Kraft, Subtilität, bar jeder
                     literarischen Qualitäten; sie verrät einen kleinkrämerischen Geist und dumme
                     Eitelkeit. Es handelt sich um die nicht überlieferte französische
                  Übersetzung von \emph{Der grüne Kakadu}\pwindex{gruene Kakadu. Groteske in einem Akt@\emph{Der grüne Kakadu. Groteske in einem Akt}|pwk} durch Émile Soutif\pwindex{Soutif, Emile @\textsc{Soutif, Émile}, \emph{Lehrer/Lehrerin}|pwk}.}}}\label{K_L00914-1}\pend
           \pstart \spacefill\mbox{H. H.}\pend{}
\pstart
           \noindent{}\label{K_L00914-2v}\edtext{j’ai ajouté quelques
                     eclaircissements}{\lemma{\textnormal{\emph{j’ai … eclaircissements}}}\Cendnote{\textnormal{französisch: ich
                        habe ein paar Klärungen ergänzt}}}\label{K_L00914-2}.\pend
           
\pstart
           {\pb}Das \label{K_L00914-3v}\edtext{Wortſpiel mit dem Sitzen}{\lemma{\textnormal{\emph{Wortſpiel mit dem Sitzen}}}\Cendnote{\textnormal{Es ist im Stück in der doppelten
                     Bedeutung von ›herumsitzen‹ und im ›Gefängnis sitzen‹ verwendet.}}}\label{K_L00914-3} iſt
                  unverſtanden geblieben, iſt auch ſchwer zu überſetzen.\pend
           
\pstart
           Die \textsc{replique} des \textsc{Prosper}\pwindex{gruene Kakadu. Groteske in einem Akt@\emph{Der grüne Kakadu. Groteske in einem Akt}|pwv} müſste lauten: \label{K_L00914-4v}\edtext{\textsc{si au moins tu ne faisais que leur tenir compagnie!}}{\lemma{\textnormal{\emph{si … compagnie!}}}\Cendnote{\textnormal{Im Stück heißt es: »– wenn du
                        nur immer mit ihnen gesessen wärst.«}}}\label{K_L00914-4} (das iſt aber auch ohne
                  Schärfe)\pend
           \selectlanguage{ngerman}\endnumbering\briefempfaengerindex{Schnitzler, Arthur@\textsc{Schnitzler, Arthur}!zzzHofmannsthal, Hugo von@\emph{von Hugo von Hofmannsthal}!1899-05-011@{{[}Mai 1899{]}}|)be}\mylabel{L00914h}  \normalsize

\doendnotes{C}
\bigskip
\vfill

\clearpage

\footnotesize

\lohead{\textsc{register}}

% Definiere theindex-Environment komplett neu ohne reledmac
\makeatletter
\renewenvironment{theindex}{%
  \section*{\indexname}%
  \setlength{\parindent}{0pt}%
  \setlength{\parskip}{0pt plus 0.3pt}%
  \let\item\@idxitem
}{%
  \clearpage
}
\makeatother

\IfFileExists{\jobname-pw.ind}{\input{\jobname-pw.ind}}{}

\end{document}

      