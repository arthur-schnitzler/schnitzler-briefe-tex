%% latex-korrekturansicht-vorspann.tex
%% Vorspann für die Korrekturansicht.
%% Lädt die gemeinsame Datei latex-vorspann.tex mit gesetztem Schalter.

\newif\ifkorrekturansicht
\korrekturansichttrue

\input{../tex-inputs/latex-vorspann}


\section[ Paul Goldmann an Arthur Schnitzler, 13. 6. 1898]{L02844 Paul Goldmann an Arthur Schnitzler, 13. 6. 1898}
\nopagebreak\mylabel{L02844v}
\rehead{ }\normalsize\beginnumbering\briefempfaengerindex{Schnitzler, Arthur@\textsc{Schnitzler, Arthur}!zzzGoldmann, Paul@\emph{von Paul Goldmann}!1898-06-131@{13. 6. 1898}|(be}
\toendnotes[C]{\smallbreak\pagebreak[2]}\Standort{DLA, A:Schnitzler, HS.NZ85.1.3168.}
\physDesc{Brief, 1 Blatt, 2 Seiten, 731 Zeichen
\newline{}Handschrift: blaue Tinte, deutsche Kurrent}\toendnotes[C]{\smallbreak}
\pstart
           \centering{}{\pb}\textcolor{gray}{\textbf{\begin{otherlanguage}{english}The Astor House\oindex{Astor House Hotel [Shanghai]@\textbf{Astor House Hotel [Shanghai]}, \emph{Hotel (K.HTL)}|pw}\end{otherlanguage}}}\pend
           
\pstart
           \centering{}\textcolor{gray}{\textbf{\begin{otherlanguage}{english}MRS.\end{otherlanguage}{ }E. JANSEN\pwindex{Jansen, Ellen 1843-04-05 – 1918-11-12@\textsc{Jansen, Ellen} (1843-04-05 – 1918-11-12), \emph{Hotelbesitzer/Hotelbesitzerin}|pw}, \begin{otherlanguage}{english}PROPRIETRESS\end{otherlanguage}.}}\pend
           
\pstart
           \raggedleft{}\textcolor{gray}{\textbf{Shanghai}}{ }13. Juni \textcolor{gray}{\textbf{189}}8.\pend
           
\pstart\center{}Mein lieber Freund,\pend\vspace{0.5em}
\pstart
           Warum höre ich ſo gar nichts von Dir? Geſtern erhielt
               ich hier Dein neues Buch\pwindex{Frau des Weisen. Novelletten@\emph{Die Frau des Weisen. Novelletten}|pwv}.
               Tauſend Dank dafür. Ich will es leſen, aber einen Brief möchte ich auch haben.\pend
           
\pstart
           Heute ſende ich ein kleines Poſt-Paket an Dich ab. Du
               findeſt darin: 1.) ein paar goldene Manſchetten-Knöpfe für Dich 2.) eine goldene
               Krawatten-Nadel für \textsc{Richard\pwindex{Beer-Hofmann, Richard 1866-07-11 – 1945-09-26@\textsc{Beer-Hofmann, Richard} (1866-07-11 – 1945-09-26), \emph{Schriftsteller/Schriftstellerin}|pw}} 3.) eine Tigerzahn-\strikeout{K}Krawatten-Nadel für \textsc{Leo\pwindex{Van-Jung, Leo 15.10.1866 – 02.07.1939@\textsc{Van-Jung, Leo} (15.10.1866 – 02.07.1939), \emph{Gesangspädagoge/Gesangspädagogin, Mathematiker/Mathematikerin}|pw}} 4.) eine \strikeout{ſ\textcolor{gray}{×}}{ }{\pb}ſilberne \textsc{Broche} für Deine
                  Freundin\pwindex{Reinhard, Marie 1871-03-13 – 1899-03-18@\textsc{Reinhard, Marie} (1871-03-13 – 1899-03-18), \emph{Gesangspädagoge/Gesangspädagogin}|pwv}.\pend
           
\pstart
           Bitte, übergib den drei Anderen\pwindex{Beer-Hofmann, Richard 1866-07-11 – 1945-09-26@\textsc{Beer-Hofmann, Richard} (1866-07-11 – 1945-09-26), \emph{Schriftsteller/Schriftstellerin}|pwv}\pwindex{Van-Jung, Leo 15.10.1866 – 02.07.1939@\textsc{Van-Jung, Leo} (15.10.1866 – 02.07.1939), \emph{Gesangspädagoge/Gesangspädagogin, Mathematiker/Mathematikerin}|pwv}\pwindex{Reinhard, Marie 1871-03-13 – 1899-03-18@\textsc{Reinhard, Marie} (1871-03-13 – 1899-03-18), \emph{Gesangspädagoge/Gesangspädagogin}|pwv} die für ſie beſtimmten Gegenſtände
               mit vielen Grüßen von mir und nimm’ Dir \strikeout{das}{ }\strikeout{\textcolor{gray}{d}} den Deinigen mit derſelben Beigabe.\pend
           
\pstart
           Ich leide furchtbar unter der Hitze, den \textsc{Mosquitos}, dem
               Heimweh, andauernden Kopfſchmerzen und meiner Unfähigkeit, zu ſchreiben.\pend
           
\pstart
           Tauſend Grüße!\pend
           
\pstart
           Dein {\\[\baselineskip]}\spacefill\mbox{Paul Goldmn}\pend
           \leftskip=0em{}\selectlanguage{ngerman}\endnumbering\briefempfaengerindex{Schnitzler, Arthur@\textsc{Schnitzler, Arthur}!zzzGoldmann, Paul@\emph{von Paul Goldmann}!1898-06-131@{13. 6. 1898}|)be}\mylabel{L02844h}  \normalsize

\doendnotes{C}
\bigskip
\vfill

\clearpage

\footnotesize

\lohead{\textsc{register}}

% Definiere theindex-Environment komplett neu ohne reledmac
\makeatletter
\renewenvironment{theindex}{%
  \section*{\indexname}%
  \setlength{\parindent}{0pt}%
  \setlength{\parskip}{0pt plus 0.3pt}%
  \let\item\@idxitem
}{%
  \clearpage
}
\makeatother

\IfFileExists{\jobname-pw.ind}{\input{\jobname-pw.ind}}{}

\end{document}

      