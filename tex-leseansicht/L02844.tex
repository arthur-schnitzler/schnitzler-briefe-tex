%% latex-leseansicht-vorspann.tex
%% Vorspann für die Leseansicht.
%% Lädt die gemeinsame Datei latex-vorspann.tex mit nicht gesetztem Schalter.

\newif\ifkorrekturansicht
\korrekturansichtfalse

\input{../tex-inputs/latex-vorspann}


         
         \renewcommand{\erwaehntePersonen}{Personen: Richard Beer-Hofmann, Paul Goldmann, Ellen Jansen, Marie Reinhard, Leo Van-Jung}
         \renewcommand{\erwaehnteOrte}{Orte: Astor House Hotel [Shanghai], Shanghai, Wien}
         \renewcommand{\erwaehnteWerke}{Werke: Die Frau des Weisen. Novelletten}
               \section[ Paul Goldmann an Arthur Schnitzler, 13. 6. 1898]{ Paul Goldmann an Arthur Schnitzler, 13. 6. 1898}\nopagebreak\mylabel{v}\rehead{ }\begin{ledgroupsized}[t]{13cm}\normalsize\beginnumbering \toendnotes[C]{\smallbreak\pagebreak[2]} \Standort{DLA, A:Schnitzler, HS.NZ85.1.3168.}
\physDesc{Brief, 1 Blatt, 2 Seiten, 731 Zeichen
\newline{}Handschrift: blaue Tinte, deutsche Kurrent}\toendnotes[C]{\smallbreak}\pstart
           \noindent{}\centering{}{\pb}\textcolor{gray}{\textbf{\begin{otherlanguage}{english}The Astor House\oindex{Astor House Hotel [Shanghai]@\textbf{Astor House Hotel [Shanghai]}|pw}\end{otherlanguage}}}\pend
           \pstart
           \noindent{}\centering{}\textcolor{gray}{\textbf{\begin{otherlanguage}{english}MRS.\end{otherlanguage}{ }E. JANSEN\pwindex{Jansen, Ellen 1843-04-05 – 1918-11-12@\textsc{Jansen, Ellen} (1843-04-05 – 1918-11-12), \emph{Hotelbesitzerin}|pw}, \begin{otherlanguage}{english}PROPRIETRESS\end{otherlanguage}.}}\pend
           \pstart
           \raggedleft{}\textcolor{gray}{\textbf{Shanghai}}{ }13. Juni \textcolor{gray}{\textbf{189}}8.\pend
           \pstart\center{}Mein lieber Freund,\pend\pstart
           Warum höre ich ſo gar nichts von Dir? Geſtern erhielt
               ich hier Dein neues Buch\pwindex{Schnitzler, Arthur 15.05.1862 – 21.10.1931@\textsc{Schnitzler, Arthur} (15.05.1862 – 21.10.1931), \emph{Schriftsteller, Mediziner}!Frau des Weisen. Novelletten1898-05-03@\strich\emph{Die Frau des Weisen. Novelletten} {[}1898-05-03{]}|pwv}.
               Tauſend Dank dafür. Ich will es leſen, aber einen Brief möchte ich auch haben.\pend
           \pstart
           Heute ſende ich ein kleines Poſt-Paket an Dich ab. Du
               findeſt darin: 1.) ein paar goldene Manſchetten-Knöpfe für Dich 2.) eine goldene
               Krawatten-Nadel für \textsc{Richard\pwindex{Beer-Hofmann, Richard 1866-07-11 – 1945-09-26@\textsc{Beer-Hofmann, Richard} (1866-07-11 – 1945-09-26), \emph{Schriftsteller}|pw}} 3.) eine Tigerzahn-\strikeout{K}Krawatten-Nadel für \textsc{Leo\pwindex{Van-Jung, Leo 15.10.1866 – 02.07.1939@\textsc{Van-Jung, Leo} (15.10.1866 – 02.07.1939), \emph{Gesangspädagoge, Mathematiker}|pw}} 4.) eine \strikeout{ſ\textcolor{gray}{×}}{ }{\pb}ſilberne \textsc{Broche} für Deine
                  Freundin\pwindex{Reinhard, Marie 1871-03-13 – 1899-03-18@\textsc{Reinhard, Marie} (1871-03-13 – 1899-03-18), \emph{Gesangspädagogin}|pwv}.\pend
           \pstart
           Bitte, übergib den drei Anderen\pwindex{Beer-Hofmann, Richard 1866-07-11 – 1945-09-26@\textsc{Beer-Hofmann, Richard} (1866-07-11 – 1945-09-26), \emph{Schriftsteller}|pwv}\pwindex{Van-Jung, Leo 15.10.1866 – 02.07.1939@\textsc{Van-Jung, Leo} (15.10.1866 – 02.07.1939), \emph{Gesangspädagoge, Mathematiker}|pwv}\pwindex{Reinhard, Marie 1871-03-13 – 1899-03-18@\textsc{Reinhard, Marie} (1871-03-13 – 1899-03-18), \emph{Gesangspädagogin}|pwv} die für ſie beſtimmten Gegenſtände
               mit vielen Grüßen von mir und nimm’ Dir \strikeout{das}{ }\strikeout{\textcolor{gray}{d}} den Deinigen mit derſelben Beigabe.\pend
           \pstart
           Ich leide furchtbar unter der Hitze, den \textsc{Mosquitos}, dem
               Heimweh, andauernden Kopfſchmerzen und meiner Unfähigkeit, zu ſchreiben.\pend
           \pstart
           Tauſend Grüße!\pend
           \pstart
           Dein {\\[\baselineskip]}\spacefill\mbox{Paul Goldmn}\pend
           \leftskip=0em{}
         
         \endnumbering\mylabel{h}\end{ledgroupsized}  \newcommand{\dateiname}{L02844}\newcommand{\titel}{Paul Goldmann an Arthur Schnitzler, 13. 6. 1898}\newcommand{\editorInnen}{Martin Anton Müller und Laura Untner}%% latex-leseansicht-abspann.tex
%% Abspann für die Leseansicht.
%% Der Schalter \ifkorrekturansicht ist bereits durch den Vorspann gesetzt.

%% latex-abspann.tex
%% Gemeinsamer Abspann für Korrekturansicht und Leseansicht.
%% Setzt den Schalter \ifkorrekturansicht voraus (gesetzt in den
%% einbindenden Dateien latex-korrekturansicht-abspann.tex bzw.
%% latex-leseansicht-abspann.tex).
%% ---------------------------------------------------------------

\normalsize

% Das esempio-Environment wird nur in der Leseansicht benötigt
\ifkorrekturansicht\else
\newenvironment{esempio}[3]%
{
    \vspace{1.5ex}
    \rlap{\underline{#1}}
    \par
    \setlength{\parindent}{0cm}
    \nopagebreak
    \leftskip=#2cm
    \rightskip=#3cm
}
{
    \par
}
\fi

\doendnotes{C}
\bigskip
\vfill

\clearpage

\footnotesize

\ifkorrekturansicht
  \lohead{\textsc{register}}
\fi

% theindex-Environment neu definieren ohne reledmac
\makeatletter
\renewenvironment{theindex}{%
  \ifkorrekturansicht
    \section*{\indexname}%
  \else
    \subsubsection*{Index der erwähnten Entitäten}%
  \fi
  \setlength{\parindent}{0pt}%
  \setlength{\parskip}{0pt plus 0.3pt}%
  \let\item\@idxitem
}{%
  \ifkorrekturansicht\clearpage\fi
}
\makeatother

\IfFileExists{\jobname-pw.ind}{\input{\jobname-pw.ind}}{}

% Quellenangabe nur in der Leseansicht
\ifkorrekturansicht\else
% Fallback-Definitionen, falls die .tex-Datei \titel etc. nicht gesetzt hat
\providecommand{\titel}{}
\providecommand{\editorInnen}{}
\providecommand{\dateiname}{\jobname}

\vspace{3cm}

\vfill

\footnotesize
\textsc{Quelle}: \titel. Herausgegeben von {\editorInnen}. In: \emph{Arthur Schnitzler: Briefwechsel mit Autorinnen und Autoren}.
 Digitale Edition, https://schnitzler-briefe.acdh.oeaw.ac.at/{\dateiname}.html (Stand \today)
\fi

\end{document}


      