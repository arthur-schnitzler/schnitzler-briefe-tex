%% latex-leseansicht-vorspann.tex
%% Vorspann für die Leseansicht.
%% Lädt die gemeinsame Datei latex-vorspann.tex mit nicht gesetztem Schalter.

\newif\ifkorrekturansicht
\korrekturansichtfalse

\input{../tex-inputs/latex-vorspann}


         
         \newcommand{\erwaehntePersonen}{Personen: }
         \newcommand{\erwaehnteInstitutionen}{}
         \newcommand{\erwaehnteOrte}{}
         \newcommand{\erwaehnteWerke}{
               \section[Anna von Hofmannsthal und Arthur Schnitzler an Hugo von Hofmannsthal, {[}19. 7. 1898{]}]{ Anna von Hofmannsthal und Arthur Schnitzler an Hugo von Hofmannsthal,
               {[}19. 7. 1898{]}}\nopagebreak\mylabel{v}\rehead{ }\begin{ledgroupsized}[t]{13cm}\normalsize\beginnumbering \toendnotes[C]{\smallbreak\pagebreak[2]} \Standort{FDH, Hofmannsthal, M8.}
\physDesc{Brief, 1 Blatt, 4 Seiten
\newline{}Handschrift  : schwarze Tinte, deutsche Kurrent\newline{}Handschrift  : schwarze Tinte, deutsche Kurrent}\buchAbdrucke{\weitereDrucke{Arthur Schnitzler: \emph{Briefe 1875–1912}. Hg. Therese Nickl und Heinrich Schnitzler. Frankfurt am Main: \emph{S. Fischer} 1981, S. 351.} }\toendnotes[C]{\smallbreak}\pstart
           \raggedleft{}{\pb}\textsc{Fusch}\oindex{XXXX Ortsangabe fehlt|pw} den 19/7.\pend
           \pstart\center{}Mein lieber kleiner \textsc{Hugi}!\pend\pstart
           Heute ein prachtvoller \textsc{So{\geminationm}ertag}! der gute \textsc{Papa}\pwindex{\textcolor{red}{\textsuperscript{XXXX1 indx}}|pwv} iſt mit \textsc{Arthur}, der geſtern nach unſerem \textsc{Souper} angefahren kam, nämlich \textsc{D\textsuperscript{r} Schnitzler} iſt dieſer \textsc{Arthur} in \textsc{Ferleithen}\oindex{XXXX Ortsangabe fehlt|pw} von wo ſie \substVorne{}\textsuperscript{nach}\substDazwischen{}vor\substHinten{} Tiſch zurück kehren wollen. Die liebe kleine \textsc{Dora}\pwindex{\textcolor{red}{\textsuperscript{XXXX1 indx}}|pw}, die einer Erkältung wegen mit ihrer Familie die auch nach \textsc{Ferleithen}\oindex{XXXX Ortsangabe fehlt|pw} iſt nicht mit konnte, ſitzt neben mir auf der \textsc{Veranda}
               und kocht mit den 2 Flatſcherkindern\pwindex{\textcolor{red}{\textsuperscript{XXXX1 indx}}|pw}\pwindex{\textcolor{red}{\textsuperscript{XXXX1 indx}}|pw}. \textsc{Papa}\pwindex{\textcolor{red}{\textsuperscript{XXXX1 indx}}|pwv} hat ein ſehr hübſches Flanellhemd und ſeinen ſchwarzen Gürtel angezogen, eine
                  \textsc{affectirte} ſchottiſche Kappe aufgeſetzt, und iſt mit der
                  »\textsc{Liebelei}\textcolor{red}{\textsuperscript{XXXX indx}}« die ich nicht ſah, weil ich noch im Bette lag, friſchen Muthes um ½ 8
                  Uhr früh ab.\pend
           \pstart
           Seit es ſchön iſt, fühlt ſich \textsc{Papa}\pwindex{\textcolor{red}{\textsuperscript{XXXX1 indx}}|pwv} unberufen ſehr wohl, iſt luſtig und zieht ſich ſehr gepflegt an. Über Alles das
               ſind wir froh, nicht wahr lieber Hugi.\pend
           \pstart
           {\pb}Sehr ſtolz bin ich darauf, daß Du mit meinem Brief ſo
               zufrieden biſt!\pend
           \pstart
           \textsc{Amusantes} kann ich Dir eigentlich nichts ſchreiben, aber
               von alldem was hier vorgeht, und wie uns zu Muthe iſt, davon weißt Du immer! –\pend
           \pstart
           Geſtern war ich faſt den ganzen Nachmittag im Wald oben, und habe ſo recht nach
               Herzensluſt mit den \textsc{Speyermädeln}\pwindex{\textcolor{red}{\textsuperscript{XXXX1 indx}}|pw}\pwindex{\textcolor{red}{\textsuperscript{XXXX1 indx}}|pw}\pwindex{\textcolor{red}{\textsuperscript{XXXX1 indx}}|pw}\pwindex{\textcolor{red}{\textsuperscript{XXXX1 indx}}|pw}\pwindex{\textcolor{red}{\textsuperscript{XXXX1 indx}}|pw}\pwindex{\textcolor{red}{\textsuperscript{XXXX1 indx}}|pw} geplauſcht. Dann bin ich mit \textsc{Papa}\pwindex{\textcolor{red}{\textsuperscript{XXXX1 indx}}|pwv} auf der Anna Bank gemüthlich geſeßen, und bei \textsc{Arthur’s
                  Souper assistirten} wir auch. Wir ſind mit ihm unter den Bäumen vor dem
                  Fliegen\textsc{salon} geſeßen. Alſo 12 Stunden in der beſten
               Luft, die es überhaupt giebt. Ich ſeh ſchon, wie Du jetzt lachſt, daß ich die \textsc{Fusch}\oindex{XXXX Ortsangabe fehlt|pw} ſchon wieder ſo lobe! –\pend
           \pstart
           Während ich mit Dir plaudere, kommt abwechſelnd die kleine \textsc{Nani}\pwindex{\textcolor{red}{\textsuperscript{XXXX1 indx}}|pw} und der \textsc{Martin}\pwindex{\textcolor{red}{\textsuperscript{XXXX1 indx}}|pw}, und zeigen mir die ſchönen Sachen, die ſie am Tiſch neben an, in dem Geſchirrl
               das {\pb}wir ihnen mitbrachten, kochten. Sie ſind wirklich
               liebe Fratzen, und machen mir viel Spaß, und ko{\geminationm}e ich
               mir um Vieles jünger vor wenn ich mit Kindern oder jungen \textsc{Mädeln} bin. Du weißt, daß mich die Frauen in meinem Alter nur mäßig anregen.
               Eigentlich verſti{\geminationm}en ſie mich mehr, und fühle ich dann
               mein Alter! es iſt das eine Schwäche von mir deren ich mich aufrichtig geſagt aber
               nicht ſchäme.\pend
           \pstart
           Abends wollen wir heute wieder zu \textsc{Weilguni}\oindex{XXXX Ortsangabe fehlt|pw} gehen, ſchöne Muſick hören. ich freue mich ſehr darauf, denn das iſt mir ein
               großer Genuß für mich.\pend
           \pstart
           Damit die Schreiberei noch \textsc{animirter} wird, werfen die
               Kinder über unter und neben mich den Ballen. Unglaublich, was ſie heute treiben, aber
               mich ſtört es nicht und ſpiele ich immer wieder ſelbſt mit ihnen. \pend
           {\bigskip}\pstart
           \noindent{}{\pb}{[}hs. :{]} mein lieber Hugo, aus Ferleiten\oindex{XXXX Ortsangabe fehlt|pw} haben Sie ſchon meinen gedruckten Gruſs beko{\geminationm}en, nehmen Sie noch einen geſchriebnen aus der Fuſch\oindex{XXXX Ortsangabe fehlt|pw}. Ich freue mich ſehr hiehergeko{\geminationm}en zu ſein; vor zwanzig Jahren oder mehr bin ich zum
               letzten Mal hier geweſen. Jetzt eben ko{\geminationm} ich mit Ihrem
                  Papa\pwindex{\textcolor{red}{\textsuperscript{XXXX1 indx}}|pwv} aus Ferleiten\oindex{XXXX Ortsangabe fehlt|pw} zurück und Ihre Mama offerirt mir dieſe leere Seite.
               So werd ich mit Liebenswürdigkeiten überſchüttet.\pend
           \pstart
           Auf Wiederſehen!\pend
           \pstart Von Herzen Ihr \spacefill\mbox{Arthur.}\pend{}
         
         \endnumbering\mylabel{h}\end{ledgroupsized}  \newcommand{\dateiname}{L00824}\newcommand{\titel}{Anna von Hofmannsthal und Arthur Schnitzler an Hugo von Hofmannsthal, [19. 7. 1898]}\newcommand{\editorInnen}{Martin Anton Müller und Gerd-Hermann Susen}%% latex-leseansicht-abspann.tex
%% Abspann für die Leseansicht.
%% Der Schalter \ifkorrekturansicht ist bereits durch den Vorspann gesetzt.

%% latex-abspann.tex
%% Gemeinsamer Abspann für Korrekturansicht und Leseansicht.
%% Setzt den Schalter \ifkorrekturansicht voraus (gesetzt in den
%% einbindenden Dateien latex-korrekturansicht-abspann.tex bzw.
%% latex-leseansicht-abspann.tex).
%% ---------------------------------------------------------------

\normalsize

% Das esempio-Environment wird nur in der Leseansicht benötigt
\ifkorrekturansicht\else
\newenvironment{esempio}[3]%
{
    \vspace{1.5ex}
    \rlap{\underline{#1}}
    \par
    \setlength{\parindent}{0cm}
    \nopagebreak
    \leftskip=#2cm
    \rightskip=#3cm
}
{
    \par
}
\fi

\doendnotes{C}
\bigskip
\vfill

\clearpage

\footnotesize

\ifkorrekturansicht
  \lohead{\textsc{register}}
\fi

% theindex-Environment neu definieren ohne reledmac
\makeatletter
\renewenvironment{theindex}{%
  \ifkorrekturansicht
    \section*{\indexname}%
  \else
    \subsubsection*{Index der erwähnten Entitäten}%
  \fi
  \setlength{\parindent}{0pt}%
  \setlength{\parskip}{0pt plus 0.3pt}%
  \let\item\@idxitem
}{%
  \ifkorrekturansicht\clearpage\fi
}
\makeatother

\IfFileExists{\jobname-pw.ind}{\input{\jobname-pw.ind}}{}

% Quellenangabe nur in der Leseansicht
\ifkorrekturansicht\else
% Fallback-Definitionen, falls die .tex-Datei \titel etc. nicht gesetzt hat
\providecommand{\titel}{}
\providecommand{\editorInnen}{}
\providecommand{\dateiname}{\jobname}

\vspace{3cm}

\vfill

\footnotesize
\textsc{Quelle}: \titel. Herausgegeben von {\editorInnen}. In: \emph{Arthur Schnitzler: Briefwechsel mit Autorinnen und Autoren}.
 Digitale Edition, https://schnitzler-briefe.acdh.oeaw.ac.at/{\dateiname}.html (Stand \today)
\fi

\end{document}


      