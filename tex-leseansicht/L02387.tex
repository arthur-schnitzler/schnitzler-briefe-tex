%% latex-leseansicht-vorspann.tex
%% Vorspann für die Leseansicht.
%% Lädt die gemeinsame Datei latex-vorspann.tex mit nicht gesetztem Schalter.

\newif\ifkorrekturansicht
\korrekturansichtfalse

\input{../tex-inputs/latex-vorspann}


\section[Arthur Schnitzler an Georg Brandes, 7. 6. 1922]{L02387 Arthur Schnitzler an Georg Brandes, 7. 6. 1922}
\nopagebreak\mylabel{L02387v}
\rehead{ }\normalsize\beginnumbering\briefempfaengerindex{Brandes, Georg@\textsc{Brandes, Georg}!zzzSchnitzler, Arthur@\emph{von Arthur Schnitzler}!1922-06-071@{7. 6. 1922}|(be}
\toendnotes[C]{\smallbreak\pagebreak[2]}
\correspDesc{Versand  durch Arthur Schnitzler am 7. 6. 1922 in Wien
\newline{}Erhalt  durch Georg Brandes im Zeitraum [8. 6. 1922
                  – 12. 6. 1922?] in Kopenhagen}\toendnotes[C]{\smallbreak}
\Standort{Kopenhagen, Det Kongelige Bibliotek, Georg Brandes Arkiv, box 125.}
\physDesc{Brief, 2 Blätter, 4 Seiten, 2996 Zeichen
\newline{}Handschrift: schwarze Tinte, lateinische Kurrent
\newline{}Ordnung: mit Bleistift von unbekannter Hand beschriftet: »Schnitzler«
                                 und nummeriert: »45.«, das zweite Blatt mit
                                 ergänztem Datum: »7/6 22« }
\buchAbdrucke{\weitereDrucke{Georg Brandes, Arthur Schnitzler: \emph{Ein Briefwechsel}. Herausgegeben von Kurt Bergel. Bern: \emph{Francke} 1956, S. 137–138.} }\toendnotes[C]{\smallbreak}
\pstart
           \raggedleft{}{\pb}Wien\oindex{Wien@\textbf{Wien}, \emph{Verwaltungsgebiet}|pw}, 7. 6. 22\pend
           \vspace{0.5em}
\pstart
           Mein lieber und verehrter Freund, daß ich nicht nach Kopenhagen\oindex{Kopenhagen@\textbf{Kopenhagen}, \emph{Hauptstadt}|pw} gekommen bin, war niemandem
               aergerlicher als mir, aber niemand hatte weniger Schuld daran. Hören Sie wie es war:
               Ein sehr netter junger Mann aus Daenemark\oindex{Dänemark@\textbf{Dänemark}|pw}, Herr
                  Axel Fraenckel\pwindex{Fraenckel, Axel @\textsc{Fraenckel, Axel}, \emph{Geisteswissenschaftler}|pw}, Literat, forderte mich im
               Namen eines »radicalen« Studentenbundes auf, in Kopenhagen\oindex{Kopenhagen@\textbf{Kopenhagen}, \emph{Hauptstadt}|pw} zu lesen. Ich war mit Vergnügen bereit – ja ich spielte mit dem
               Gedanken gerade den 15 Mai in Kopenhagen\oindex{Kopenhagen@\textbf{Kopenhagen}, \emph{Hauptstadt}|pw} und womöglich mit Ihnen zuzubringen. Ich erklärte, daß ich im Haag\oindex{Den Haag@\textbf{Den Haag}, \emph{Hauptstadt}|pw}, (wo ich, wie in Amsterdam\oindex{Amsterdam@\textbf{Amsterdam}, \emph{Hauptstadt}|pw} u Rotterdam\oindex{Rotterdam@\textbf{Rotterdam}|pw} aus
               meinen Werken vorlas) \uline{definitive}{ }\strikeout{Aus} Nachrichten abwarten wolle u. zw. bis spaetestens
                  30. April. Ich war bis zum 8. Mai in Holland\oindex{Niederlande@\textbf{Niederlande}|pw} – es kam keine Zeile, – und ich selbst konnte mich nicht
               an den Studentenbund wenden – schon darum, weil mir weder der officielle Name, noch
               die Adresse noch der Name des Obmanns {\pb}bekannt war
               – so dacht ich man habe in Kopenhagen\oindex{Kopenhagen@\textbf{Kopenhagen}, \emph{Hauptstadt}|pw} auf mein
               Kommen verzichtet, – fuhr nach Berlin\oindex{Berlin@\textbf{Berlin}, \emph{Hauptstadt}|pw}, – wo mir –
               über Haag\oindex{Den Haag@\textbf{Den Haag}, \emph{Hauptstadt}|pw}, – und Wien\oindex{Wien@\textbf{Wien}, \emph{Verwaltungsgebiet}|pw} – (die kürzeste Verbindung) ein Telegramm nachgesandt wurde – von dem
               Studentenbund – ich möge meinen Ankunftstag melden. Nun aber hatt ich meine
               Dispositionen schon total geändert u. es war zu spät, wieder in den Norden zu
               reisen; – auch hatt ich einigermaßen die Lust verloren. So verbracht ich meinen
               Geburtstag – vollko{\geminationm}en allein – in Nürnberg\oindex{Nürnberg@\textbf{Nürnberg}|pw} und fuhr von da nach München\oindex{München@\textbf{München}|pw} und Wien\oindex{Wien@\textbf{Wien}, \emph{Verwaltungsgebiet}|pw}. Entweder ist ein Brief
               in den Haag\oindex{Den Haag@\textbf{Den Haag}, \emph{Hauptstadt}|pw} verloren gegangen oder die Herren
               vom Studentenbund haben die Angelegenheit etwas zu lax behandelt – aber ich hoffe,
               ein nächstes Mal – vielleicht im nächsten Frühling (freilich – schon »am nächsten
               Tag« ist ein kühnes Wort) – wird die Sache zu Stande kommen. {\pb}Morgen fahr ich nach Graz\oindex{Graz@\textbf{Graz}, \emph{Verwaltungsgebiet}|pw}, wo ich zweimal vorlese – ein etwas ärmlicher Ersatz für
                  Kopenhagen\oindex{Kopenhagen@\textbf{Kopenhagen}, \emph{Hauptstadt}|pw} und Sie.\pend
           
\pstart
           Und für Ihre lieben Worte, mein verehrter Georg Brandes, ka{\geminationn} ich Ihnen nur schriftlich danken. (Haben Sie de{\geminationn} auch meinen Brief zu Ihrem soundsovielten Geburtstag
               bekommen?)\pend
           
\pstart
           Anfang Juli bring ich meine Kinder\pwindex{Schnitzler, Heinrich 9.\,8.\,1902 Hinterbrühl – 12.\,7.\,1982 Wien@\textsc{Schnitzler, Heinrich} (9.\,8.\,1902 Hinterbrühl – 12.\,7.\,1982 Wien), \emph{Regisseur, Schauspieler}|pwv}\pwindex{Cappellini, Lili 13.\,9.\,1909 Wien – 26.\,7.\,1928 Venedig@\textsc{Cappellini, Lili} (13.\,9.\,1909 Wien – 26.\,7.\,1928 Venedig)|pwv} an den Starnbergersee\oindex{Starnberger See@\textbf{Starnberger See}, \emph{See}|pw} zu ihrer Mutter\pwindex{Schnitzler, Olga 17.\,1.\,1882 Wien – 13.\,1.\,1970 Lugano@\textsc{Schnitzler, Olga} (17.\,1.\,1882 Wien – 13.\,1.\,1970 Lugano), \emph{Schauspielerin, Sängerin}|pwv}. (Mein Sohn\pwindex{Schnitzler, Heinrich 9.\,8.\,1902 Hinterbrühl – 12.\,7.\,1982 Wien@\textsc{Schnitzler, Heinrich} (9.\,8.\,1902 Hinterbrühl – 12.\,7.\,1982 Wien), \emph{Regisseur, Schauspieler}|pwv},
               bald zwanzig, ist für die nächste Saison schon hier am Raimundtheater\oindex{Wien@\textbf{Wien}!VI., Mariahilf@\textbf{VI., Mariahilf}!Raimund-Theater@\textbf{Raimund-Theater}, \emph{Theater}|pw} engagirt; er studirt auch Philosophie an der Universität\oindex{Wien@\textbf{Wien}!I., Innere Stadt@\textbf{I., Innere Stadt}!Universität Wien@\textbf{Universität Wien}, \emph{Universität}|pwv}, arbeitet auch
               theatergeschichtlich, macht Inszenierungspläne, zeichnet u malt Figurinen, treibt
               viel Musik; meine Tochter\pwindex{Cappellini, Lili 13.\,9.\,1909 Wien – 26.\,7.\,1928 Venedig@\textsc{Cappellini, Lili} (13.\,9.\,1909 Wien – 26.\,7.\,1928 Venedig)|pwv},
               bald dreizehn, geht ins Gymnasium.); meine Sommer{\pb}pläne sind noch etwas unsicher; – ich wünschte sehr, nach ziemlich unruhigen und
               verwirrten Zeiten, ins geordnete Arbeiten zu gelangen – und, insbesondre ein Stück\pwindex{Schnitzler, Arthur 15.\,5.\,1862 Wien – 21.\,10.\,1931 ebd.@\textsc{Schnitzler, Arthur} (15.\,5.\,1862 Wien – 21.\,10.\,1931 ebd.), \emph{Schriftsteller, Mediziner}!Komödie der Verführung. In drei Akten@\strich\emph{Komödie der Verführung. In drei Akten}|pwv} zu vollenden, dessen
               letzter Akt an der daenischen\oindex{Dänemark@\textbf{Dänemark}|pw} Küste spielen
               soll. Ich baue dort ein köstliches Hotel hin wie ich es seinerzeit am Völser Weiher\oindex{Völser Weiher@\textbf{Völser Weiher}, \emph{See}|pw} (im weiten Land\pwindex{Schnitzler, Arthur 15.\,5.\,1862 Wien – 21.\,10.\,1931 ebd.@\textsc{Schnitzler, Arthur} (15.\,5.\,1862 Wien – 21.\,10.\,1931 ebd.), \emph{Schriftsteller, Mediziner}!weite Land. Tragikomödie in fünf Akten@\strich\emph{Das weite Land. Tragikomödie in fünf Akten}|pw}) gethan – mögen mir die Gestalten auch so gelingen, wie das
               Hotel – es ist ersten Ranges.\pend
           
\pstart
           Erhalten Sie mir Ihre Freundschaft und seien Sie von Herzen gegrüßt.\pend
           
\pstart
           Von Ihren athen\oindex{Athen@\textbf{Athen}, \emph{Hauptstadt}|pw}iensischen Abenteuern hatt ich
               hier schon in der Zeitung gelesen. Mein Garten steht voll Rosen; – bin ich auch kein
                  griechischer\oindex{Griechenland@\textbf{Griechenland}|pw} Student – ich streue sie alle
               im Geiste auf Ihr theures Haupt!\pend
           
\pstart
           In Treue{\\[\baselineskip]}Ihr \spacefill\mbox{Arthur Schnitzler}\pend
           \leftskip=0em{}\selectlanguage{ngerman}\endnumbering\briefempfaengerindex{Brandes, Georg@\textsc{Brandes, Georg}!zzzSchnitzler, Arthur@\emph{von Arthur Schnitzler}!1922-06-071@{7. 6. 1922}|)be}\mylabel{L02387h}  \newcommand{\dateiname}{L02387}\newcommand{\titel}{Arthur Schnitzler an Georg Brandes, 7. 6. 1922}\newcommand{\editorInnen}{Martin Anton Müller und Gerd-Hermann Susen}%% latex-leseansicht-abspann.tex
%% Abspann für die Leseansicht.
%% Der Schalter \ifkorrekturansicht ist bereits durch den Vorspann gesetzt.

%% latex-abspann.tex
%% Gemeinsamer Abspann für Korrekturansicht und Leseansicht.
%% Setzt den Schalter \ifkorrekturansicht voraus (gesetzt in den
%% einbindenden Dateien latex-korrekturansicht-abspann.tex bzw.
%% latex-leseansicht-abspann.tex).
%% ---------------------------------------------------------------

\normalsize

% Das esempio-Environment wird nur in der Leseansicht benötigt
\ifkorrekturansicht\else
\newenvironment{esempio}[3]%
{
    \vspace{1.5ex}
    \rlap{\underline{#1}}
    \par
    \setlength{\parindent}{0cm}
    \nopagebreak
    \leftskip=#2cm
    \rightskip=#3cm
}
{
    \par
}
\fi

\doendnotes{C}
\bigskip
\vfill

\clearpage

\footnotesize

\ifkorrekturansicht
  \lohead{\textsc{register}}
\fi

% theindex-Environment neu definieren ohne reledmac
\makeatletter
\renewenvironment{theindex}{%
  \ifkorrekturansicht
    \section*{\indexname}%
  \else
    \subsubsection*{Index der erwähnten Entitäten}%
  \fi
  \setlength{\parindent}{0pt}%
  \setlength{\parskip}{0pt plus 0.3pt}%
  \let\item\@idxitem
}{%
  \ifkorrekturansicht\clearpage\fi
}
\makeatother

\IfFileExists{\jobname-pw.ind}{\input{\jobname-pw.ind}}{}

% Quellenangabe nur in der Leseansicht
\ifkorrekturansicht\else
% Fallback-Definitionen, falls die .tex-Datei \titel etc. nicht gesetzt hat
\providecommand{\titel}{}
\providecommand{\editorInnen}{}
\providecommand{\dateiname}{\jobname}

\vspace{3cm}

\vfill

\footnotesize
\textsc{Quelle}: \titel. Herausgegeben von {\editorInnen}. In: \emph{Arthur Schnitzler: Briefwechsel mit Autorinnen und Autoren}.
 Digitale Edition, https://schnitzler-briefe.acdh.oeaw.ac.at/{\dateiname}.html (Stand \today)
\fi

\end{document}


