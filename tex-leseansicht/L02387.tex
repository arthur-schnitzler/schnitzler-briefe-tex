%% latex-korrekturansicht-vorspann.tex
%% Vorspann für die Korrekturansicht.
%% Lädt die gemeinsame Datei latex-vorspann.tex mit gesetztem Schalter.

\newif\ifkorrekturansicht
\korrekturansichttrue

\input{../tex-inputs/latex-vorspann}


\section[Arthur Schnitzler an Georg Brandes, 7. 6. 1922]{L02387 Arthur Schnitzler an Georg Brandes, 7. 6. 1922}
\nopagebreak\mylabel{L02387v}
\rehead{ }\normalsize\beginnumbering\briefempfaengerindex{Brandes, Georg@\textsc{Brandes, Georg}!zzzSchnitzler, Arthur@\emph{von Arthur Schnitzler}!1922-06-071@{7. 6. 1922}|(be}
\toendnotes[C]{\smallbreak\pagebreak[2]}\Standort{Kopenhagen, Det Kongelige Bibliotek, Georg Brandes Arkiv, box 125.}
\physDesc{Brief, 2 Blätter, 4 Seiten, 2996 Zeichen
\newline{}Handschrift: schwarze Tinte, lateinische Kurrent
\newline{}Ordnung: mit Bleistift von unbekannter Hand beschriftet: »Schnitzler«
                                 und nummeriert: »45.«, das zweite Blatt mit
                                 ergänztem Datum: »7/6 22« }
\buchAbdrucke{\weitereDrucke{Georg Brandes, Arthur Schnitzler: \emph{Ein Briefwechsel}. Bern: \emph{Francke} 1956, S. 137–138.} }\toendnotes[C]{\smallbreak}
\pstart
           \raggedleft{}{\pb}Wien\oindex{Wien@\textbf{Wien}, \emph{A.ADM2}|pw}, 7. 6. 22\pend
           \vspace{0.5em}
\pstart
           Mein lieber und verehrter Freund, daß ich nicht nach Kopenhagen\oindex{Kopenhagen@\textbf{Kopenhagen}, \emph{P.PPLC}|pw} gekommen bin, war niemandem
               aergerlicher als mir, aber niemand hatte weniger Schuld daran. Hören Sie wie es war:
               Ein sehr netter junger Mann aus Daenemark\oindex{Daenemark@\textbf{Dänemark}, \emph{A.PCLI}|pw}, Herr
                  Axel Fraenckel\pwindex{Fraenckel, Axel @\textsc{Fraenckel, Axel}, \emph{Geisteswissenschaftler/Geisteswissenschaftlerin}|pw}, Literat, forderte mich im
               Namen eines »radicalen« Studentenbundes auf, in Kopenhagen\oindex{Kopenhagen@\textbf{Kopenhagen}, \emph{P.PPLC}|pw} zu lesen. Ich war mit Vergnügen bereit – ja ich spielte mit dem
               Gedanken gerade den 15 Mai in Kopenhagen\oindex{Kopenhagen@\textbf{Kopenhagen}, \emph{P.PPLC}|pw} und womöglich mit Ihnen zuzubringen. Ich erklärte, daß ich im Haag\oindex{Den Haag@\textbf{Den Haag}, \emph{P.PPLG}|pw}, (wo ich, wie in Amsterdam\oindex{Amsterdam@\textbf{Amsterdam}, \emph{P.PPLC}|pw} u Rotterdam\oindex{Rotterdam@\textbf{Rotterdam}, \emph{P.PPL}|pw} aus
               meinen Werken vorlas) \uline{definitive}{ }\strikeout{Aus} Nachrichten abwarten wolle u. zw. bis spaetestens
                  30. April. Ich war bis zum 8. Mai in Holland\oindex{Niederlande@\textbf{Niederlande}, \emph{A.PCLI}|pw} – es kam keine Zeile, – und ich selbst konnte mich nicht
               an den Studentenbund wenden – schon darum, weil mir weder der officielle Name, noch
               die Adresse noch der Name des Obmanns {\pb}bekannt war
               – so dacht ich man habe in Kopenhagen\oindex{Kopenhagen@\textbf{Kopenhagen}, \emph{P.PPLC}|pw} auf mein
               Kommen verzichtet, – fuhr nach Berlin\oindex{Berlin@\textbf{Berlin}, \emph{P.PPLC}|pw}, – wo mir –
               über Haag\oindex{Den Haag@\textbf{Den Haag}, \emph{P.PPLG}|pw}, – und Wien\oindex{Wien@\textbf{Wien}, \emph{A.ADM2}|pw} – (die kürzeste Verbindung) ein Telegramm nachgesandt wurde – von dem
               Studentenbund – ich möge meinen Ankunftstag melden. Nun aber hatt ich meine
               Dispositionen schon total geändert u. es war zu spät, wieder in den Norden zu
               reisen; – auch hatt ich einigermaßen die Lust verloren. So verbracht ich meinen
               Geburtstag – vollko{\geminationm}en allein – in Nürnberg\oindex{Nuernberg@\textbf{Nürnberg}, \emph{P.PPL}|pw} und fuhr von da nach München\oindex{Muenchen@\textbf{München}, \emph{P.PPLA}|pw} und Wien\oindex{Wien@\textbf{Wien}, \emph{A.ADM2}|pw}. Entweder ist ein Brief
               in den Haag\oindex{Den Haag@\textbf{Den Haag}, \emph{P.PPLG}|pw} verloren gegangen oder die Herren
               vom Studentenbund haben die Angelegenheit etwas zu lax behandelt – aber ich hoffe,
               ein nächstes Mal – vielleicht im nächsten Frühling (freilich – schon »am nächsten
               Tag« ist ein kühnes Wort) – wird die Sache zu Stande kommen. {\pb}Morgen fahr ich nach Graz\oindex{Graz@\textbf{Graz}, \emph{A.ADM2}|pw}, wo ich zweimal vorlese – ein etwas ärmlicher Ersatz für
                  Kopenhagen\oindex{Kopenhagen@\textbf{Kopenhagen}, \emph{P.PPLC}|pw} und Sie.\pend
           
\pstart
           Und für Ihre lieben Worte, mein verehrter Georg Brandes, ka{\geminationn} ich Ihnen nur schriftlich danken. (Haben Sie de{\geminationn} auch meinen Brief zu Ihrem soundsovielten Geburtstag
               bekommen?)\pend
           
\pstart
           Anfang Juli bring ich meine Kinder\pwindex{Schnitzler, Heinrich 09.08.1902 – 12.07.1982@\textsc{Schnitzler, Heinrich} (09.08.1902 – 12.07.1982), \emph{Regisseur/Regisseurin, Schauspieler/Schauspielerin}|pwv}\pwindex{Cappellini, Lili 13.09.1909 – 26.07.1928@\textsc{Cappellini, Lili} (13.09.1909 – 26.07.1928)|pwv} an den Starnbergersee\oindex{Starnberger See@\textbf{Starnberger See}, \emph{H.LK}|pw} zu ihrer Mutter\pwindex{Schnitzler, Olga 17.01.1882 – 13.01.1970@\textsc{Schnitzler, Olga} (17.01.1882 – 13.01.1970), \emph{Schauspieler/Schauspielerin, Sänger/Sängerin}|pwv}. (Mein Sohn\pwindex{Schnitzler, Heinrich 09.08.1902 – 12.07.1982@\textsc{Schnitzler, Heinrich} (09.08.1902 – 12.07.1982), \emph{Regisseur/Regisseurin, Schauspieler/Schauspielerin}|pwv},
               bald zwanzig, ist für die nächste Saison schon hier am Raimundtheater\oindex{Raimund-Theater@\textbf{Raimund-Theater}, \emph{Theater (K.THE)}|pw} engagirt; er studirt auch Philosophie an der Universität\oindex{Universitaet Wien@\textbf{Universität Wien}, \emph{Universität (K.UNI)}|pwv}, arbeitet auch
               theatergeschichtlich, macht Inszenierungspläne, zeichnet u malt Figurinen, treibt
               viel Musik; meine Tochter\pwindex{Cappellini, Lili 13.09.1909 – 26.07.1928@\textsc{Cappellini, Lili} (13.09.1909 – 26.07.1928)|pwv},
               bald dreizehn, geht ins Gymnasium.); meine Sommer{\pb}pläne sind noch etwas unsicher; – ich wünschte sehr, nach ziemlich unruhigen und
               verwirrten Zeiten, ins geordnete Arbeiten zu gelangen – und, insbesondre ein Stück\pwindex{Komoedie der Verfuehrung. In drei Akten@\emph{Komödie der Verführung. In drei Akten}|pwv} zu vollenden, dessen
               letzter Akt an der daenischen\oindex{Daenemark@\textbf{Dänemark}, \emph{A.PCLI}|pw} Küste spielen
               soll. Ich baue dort ein köstliches Hotel hin wie ich es seinerzeit am Völser Weiher\oindex{Voelser Weiher@\textbf{Völser Weiher}, \emph{See (N.SEE)}|pw} (im weiten Land\pwindex{weite Land. Tragikomoedie in fuenf Akten@\emph{Das weite Land. Tragikomödie in fünf Akten}|pw}) gethan – mögen mir die Gestalten auch so gelingen, wie das
               Hotel – es ist ersten Ranges.\pend
           
\pstart
           Erhalten Sie mir Ihre Freundschaft und seien Sie von Herzen gegrüßt.\pend
           
\pstart
           Von Ihren athen\oindex{Athen@\textbf{Athen}, \emph{P.PPLC}|pw}iensischen Abenteuern hatt ich
               hier schon in der Zeitung gelesen. Mein Garten steht voll Rosen; – bin ich auch kein
                  griechischer\oindex{Griechenland@\textbf{Griechenland}, \emph{A.PCLI}|pw} Student – ich streue sie alle
               im Geiste auf Ihr theures Haupt!\pend
           
\pstart
           In Treue{\\[\baselineskip]}Ihr \spacefill\mbox{Arthur Schnitzler}\pend
           \leftskip=0em{}\selectlanguage{ngerman}\endnumbering\briefempfaengerindex{Brandes, Georg@\textsc{Brandes, Georg}!zzzSchnitzler, Arthur@\emph{von Arthur Schnitzler}!1922-06-071@{7. 6. 1922}|)be}\mylabel{L02387h}  \normalsize

\doendnotes{C}
\bigskip
\vfill

\clearpage

\footnotesize

\lohead{\textsc{register}}

% Definiere theindex-Environment komplett neu ohne reledmac
\makeatletter
\renewenvironment{theindex}{%
  \section*{\indexname}%
  \setlength{\parindent}{0pt}%
  \setlength{\parskip}{0pt plus 0.3pt}%
  \let\item\@idxitem
}{%
  \clearpage
}
\makeatother

\IfFileExists{\jobname-pw.ind}{\input{\jobname-pw.ind}}{}

\end{document}

      