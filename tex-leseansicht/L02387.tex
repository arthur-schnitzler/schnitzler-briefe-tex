%% latex-leseansicht-vorspann.tex
%% Vorspann für die Leseansicht.
%% Lädt die gemeinsame Datei latex-vorspann.tex mit nicht gesetztem Schalter.

\newif\ifkorrekturansicht
\korrekturansichtfalse

\input{../tex-inputs/latex-vorspann}

\begin{center}
            \textcolor{red}{ENTWURF. ENTZIFFERUNG NOCH NICHT KORREKTURGELESEN}
                      \end{center}
            
               \section[Arthur Schnitzler an Georg Brandes, 7. 6. 1922]{ Arthur Schnitzler an Georg Brandes, 7. 6. 1922}\nopagebreak\mylabel{v}\rehead{ }\begin{ledgroupsized}[t]{13cm}\normalsize\beginnumbering\briefempfaengerindex{Brandes, Georg@\textsc{Brandes, Georg}!zzzSchnitzler, Arthur@\emph{von Arthur Schnitzler}!1922-06-071@{7. 6. 1922}|(be} \toendnotes[C]{\smallbreak\pagebreak[2]} \Standort{Kopenhagen, Det Kongelige Bibliotek, Georg Brandes Arkiv, box 125.}
\physDesc{Brief, 2 Blätter, 4 Seiten
\newline{}Handschrift: schwarze Tinte, lateinische Kurrent\newline{}Ordnung: mit Bleistift von unbekannter Hand beschriftet:
                                                »Schnitzler« und nummeriert:
                                                »45.«, das zweite Blatt mit ergänztem
                                            Datum: »7/6 22« }\buchAbdrucke{\weitereDrucke{Georg Brandes, Arthur Schnitzler: \emph{Ein Briefwechsel}. Hg. Kurt Bergel. Bern: \emph{Francke} 1956, S. 137–138.} }\toendnotes[C]{\smallbreak}\pstart
           \raggedleft{}{\pb}Wien\oindex{Wien@\textbf{Wien}|pw}, 7. 6. 22\pend
           \pstart
           Mein lieber und verehrter Freund, daß ich nicht nach Kopenhagen\oindex{Kopenhagen@\textbf{Kopenhagen}|pw} gekommen bin, war niemandem
                    aergerlicher als mir, aber niemand hatte weniger Schuld daran. Hören Sie wie es
                    war: Ein sehr netter junger Mann aus Daenemark\oindex{Daenemark@\textbf{Dänemark}|pw}, Herr Axel Fraenckel\pwindex{Fraenckel, Axel @\textsc{Fraenckel, Axel}, \emph{Geisteswissenschaftler}|pw},
                    Literat, forderte mich im Namen eines »radicalen« Studentenbundes auf, in Kopenhagen\oindex{Kopenhagen@\textbf{Kopenhagen}|pw} zu lesen. Ich war mit Vergnügen
                    bereit – ja ich spielte mit dem Gedanken gerade den 15 Mai in Kopenhagen\oindex{Kopenhagen@\textbf{Kopenhagen}|pw} und womöglich mit Ihnen
               zuzubringen. Ich erklärte, daß ich im Haag\oindex{Den Haag@\textbf{Den Haag}|pw}, (wo
                    ich, wie in Amsterdam\oindex{Amsterdam@\textbf{Amsterdam}|pw} u Rotterdam\oindex{Rotterdam@\textbf{Rotterdam}|pw} aus meinen Werken vorlas) \uline{definitive}{ }\strikeout{Aus} Nachrichten abwarten wolle u. zw. bis
                    spaetestens 30. April. Ich war bis zum 8. Mai in Holland\oindex{Niederlande@\textbf{Niederlande}|pw} – es kam keine Zeile, – und ich
                    selbst konnte mich nicht an den Studentenbund wenden – schon darum, weil mir
                    weder der officielle Name, noch die Adresse noch der Name des Obmanns {\pb}bekannt war – so dacht ich man habe in Kopenhagen\oindex{Kopenhagen@\textbf{Kopenhagen}|pw} auf mein Kommen verzichtet, – fuhr
               nach Berlin\oindex{Berlin@\textbf{Berlin}|pw}, – wo mir – über Haag\oindex{Den Haag@\textbf{Den Haag}|pw}, – und Wien\oindex{Wien@\textbf{Wien}|pw} – (die
                    kürzeste Verbindung) ein Telegramm nachgesandt wurde – von dem Studentenbund –
                    ich möge meinen Ankunftstag melden. Nun aber hatt ich meine Dispositionen schon
                    total geändert u. es war zu spät, wieder in den Norden zu reisen; – auch hatt
                    ich einigermaßen die Lust verloren. So verbracht ich meinen Geburtstag –
                        vollko{\geminationm}en allein – in Nürnberg\oindex{Nuernberg@\textbf{Nürnberg}|pw} und fuhr von da nach München\oindex{Muenchen@\textbf{München}|pw} und Wien\oindex{Wien@\textbf{Wien}|pw}. Entweder
               ist ein Brief in den Haag\oindex{Den Haag@\textbf{Den Haag}|pw} verloren gegangen
                    oder die Herren vom Studentenbund haben die Angelegenheit etwas zu lax behandelt
                    – aber ich hoffe, ein nächstes Mal – vielleicht im nächsten Frühling (freilich –
                    schon »am nächsten Tag« ist ein kühnes Wort) – wird die Sache zu Stande kommen.
                        {\pb}Morgen fahr ich nach Graz\oindex{Graz@\textbf{Graz}|pw}, wo ich zweimal vorlese – ein etwas ärmlicher Ersatz
                    für Kopenhagen\oindex{Kopenhagen@\textbf{Kopenhagen}|pw} und Sie.\pend
           \pstart
           Und für Ihre lieben Worte, mein verehrter Georg Brandes, ka{\geminationn} ich Ihnen nur schriftlich danken. (Haben Sie
                        de{\geminationn} auch meinen Brief zu Ihrem soundsovielten
                    Geburtstag bekommen?)\pend
           \pstart
           Anfang Juli bring ich meine Kinder\pwindex{Schnitzler, Heinrich 09.08.1902 – 12.07.1982@\textsc{Schnitzler, Heinrich} (09.08.1902 – 12.07.1982), \emph{Regisseur, Schauspieler}|pwv}\pwindex{Schnitzler, Lili 13.09.1909 – 26.07.1928@\textsc{Schnitzler, Lili} (13.09.1909 – 26.07.1928)|pwv} an den Starnbergersee zu ihrer Mutter\pwindex{Schnitzler, Olga 17.01.1882 – 13.01.1970@\textsc{Schnitzler, Olga} (17.01.1882 – 13.01.1970), \emph{Schauspielerin, Sängerin}|pwv}. (Mein Sohn\pwindex{Schnitzler, Heinrich 09.08.1902 – 12.07.1982@\textsc{Schnitzler, Heinrich} (09.08.1902 – 12.07.1982), \emph{Regisseur, Schauspieler}|pwv}, bald zwanzig, ist für die nächste Saison schon
                    hier am Raimundtheater\oindex{Raimund-Theater@\textbf{Raimund-Theater}|pw} engagirt; er studirt auch
                    Philosophie an der Universität\oindex{Universitaet Wien@\textbf{Universität Wien}|pwv}, arbeitet auch theatergeschichtlich, macht
                    Inszenierungspläne, zeichnet u malt Figurinen, treibt viel Musik; meine Tochter\pwindex{Schnitzler, Lili 13.09.1909 – 26.07.1928@\textsc{Schnitzler, Lili} (13.09.1909 – 26.07.1928)|pwv}, bald dreizehn,
                    geht ins Gymnasium.); meine Sommer{\pb}pläne sind
                    noch etwas unsicher; – ich wünschte sehr, nach ziemlich unruhigen und verwirrten
                    Zeiten, ins geordnete Arbeiten zu gelangen – und, insbesondre ein Stück\pwindex{Schnitzler, Arthur 15.05.1862 – 21.10.1931@\textsc{Schnitzler, Arthur} (15.05.1862 – 21.10.1931), \emph{Schriftsteller, Mediziner}!Komoedie der Verfuehrung. In drei Akten1924@\strich\emph{Komödie der Verführung. In drei Akten} {[}1924{]}|pwv} zu vollenden, dessen
                    letzter Akt an der daenischen\oindex{Daenemark@\textbf{Dänemark}|pw} Küste spielen
                    soll. Ich baue dort ein köstliches Hotel hin wie ich es seinerzeit am Völser Weiher\oindex{Voelser Weiher@\textbf{Völser Weiher}|pw} (im weiten Land\pwindex{Schnitzler, Arthur 15.05.1862 – 21.10.1931@\textsc{Schnitzler, Arthur} (15.05.1862 – 21.10.1931), \emph{Schriftsteller, Mediziner}!weite Land. Tragikomoedie in fuenf Akten1910-10-20@\strich\emph{Das weite Land. Tragikomödie in fünf Akten} {[}1910-10-20{]}|pw}) gethan – mögen mir die Gestalten auch so gelingen, wie
                    das Hotel – es ist ersten Ranges.\pend
           \pstart
           Erhalten Sie mir Ihre Freundschaft und seien Sie von Herzen gegrüßt.\pend
           \pstart
           Von Ihren athen\oindex{Athen@\textbf{Athen}|pw}iensischen Abenteuern hatt ich
                    hier schon in der Zeitung gelesen. Mein Garten steht voll Rosen; – bin ich auch
                    kein griechischer\oindex{Griechenland@\textbf{Griechenland}|pw} Student – ich streue sie
                    alle im Geiste auf Ihr theures Haupt!\pend
           \pstart
           In Treue{\\[\baselineskip]}Ihr \spacefill\mbox{Arthur Schnitzler}\pend
           \leftskip=0em{}\endnumbering\briefempfaengerindex{Brandes, Georg@\textsc{Brandes, Georg}!zzzSchnitzler, Arthur@\emph{von Arthur Schnitzler}!1922-06-071@{7. 6. 1922}|)be}\mylabel{h}\end{ledgroupsized}  \newcommand{\dateiname}{L02387}\newcommand{\titel}{Arthur Schnitzler an Georg Brandes, 7. 6. 1922}\newcommand{\editorInnen}{Martin Anton Müller und Gerd-Hermann Susen}%% latex-leseansicht-abspann.tex
%% Abspann für die Leseansicht.
%% Der Schalter \ifkorrekturansicht ist bereits durch den Vorspann gesetzt.

%% latex-abspann.tex
%% Gemeinsamer Abspann für Korrekturansicht und Leseansicht.
%% Setzt den Schalter \ifkorrekturansicht voraus (gesetzt in den
%% einbindenden Dateien latex-korrekturansicht-abspann.tex bzw.
%% latex-leseansicht-abspann.tex).
%% ---------------------------------------------------------------

\normalsize

% Das esempio-Environment wird nur in der Leseansicht benötigt
\ifkorrekturansicht\else
\newenvironment{esempio}[3]%
{
    \vspace{1.5ex}
    \rlap{\underline{#1}}
    \par
    \setlength{\parindent}{0cm}
    \nopagebreak
    \leftskip=#2cm
    \rightskip=#3cm
}
{
    \par
}
\fi

\doendnotes{C}
\bigskip
\vfill

\clearpage

\footnotesize

\ifkorrekturansicht
  \lohead{\textsc{register}}
\fi

% theindex-Environment neu definieren ohne reledmac
\makeatletter
\renewenvironment{theindex}{%
  \ifkorrekturansicht
    \section*{\indexname}%
  \else
    \subsubsection*{Index der erwähnten Entitäten}%
  \fi
  \setlength{\parindent}{0pt}%
  \setlength{\parskip}{0pt plus 0.3pt}%
  \let\item\@idxitem
}{%
  \ifkorrekturansicht\clearpage\fi
}
\makeatother

\IfFileExists{\jobname-pw.ind}{\input{\jobname-pw.ind}}{}

% Quellenangabe nur in der Leseansicht
\ifkorrekturansicht\else
% Fallback-Definitionen, falls die .tex-Datei \titel etc. nicht gesetzt hat
\providecommand{\titel}{}
\providecommand{\editorInnen}{}
\providecommand{\dateiname}{\jobname}

\vspace{3cm}

\vfill

\footnotesize
\textsc{Quelle}: \titel. Herausgegeben von {\editorInnen}. In: \emph{Arthur Schnitzler: Briefwechsel mit Autorinnen und Autoren}.
 Digitale Edition, https://schnitzler-briefe.acdh.oeaw.ac.at/{\dateiname}.html (Stand \today)
\fi

\end{document}


      