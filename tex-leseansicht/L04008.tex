%% latex-leseansicht-vorspann.tex
%% Vorspann für die Leseansicht.
%% Lädt die gemeinsame Datei latex-vorspann.tex mit nicht gesetztem Schalter.

\newif\ifkorrekturansicht
\korrekturansichtfalse

\input{../tex-inputs/latex-vorspann}


\section[Berta Zuckerkandl an Arthur Schnitzler, 11. 10. 1924]{L04008 Berta Zuckerkandl an Arthur Schnitzler, 11. 10. 1924}
\nopagebreak\mylabel{L04008v}
\rehead{ }\normalsize\beginnumbering\briefempfaengerindex{Schnitzler, Arthur@\textsc{Schnitzler, Arthur}!zzzZuckerkandl, Berta@\emph{von Berta Zuckerkandl}!1924-10-111@{11. 10. 1924}|(be}
\toendnotes[C]{\smallbreak\pagebreak[2]}
\correspDesc{Versand  durch Berta Zuckerkandl am 11. 10. 1924 in Wien
\newline{}Erhalt  durch Arthur Schnitzler im Zeitraum [11. 10. 1924 – 14. 10. 1924?] in Wien}\toendnotes[C]{\smallbreak}
\Standort{CUL, Schnitzler, B 200.}
\physDesc{Brief, 1 Blatt, 2 Seiten, 788 Zeichen
\newline{}Handschrift: blaue Tinte, lateinische Kurrent
\newline{}Schnitzler: mit rotem Buntstift eine Unterstreichung }\toendnotes[C]{\smallbreak}
\pstart
           \raggedleft{}{\pb}Samstag 11. O. 1924.\pend
           \vspace{0.5em}
\pstart
           Innig verehrter Freund! Seit zwei Tagen, seit Ihrer Generalprobe\eventindex{Burgtheater@\textbf{Burgtheater}!Generalprobe von Komödie der Verführung, 10.10.1924@Generalprobe von Komödie der Verführung, 10.10.1924|pw}, kämpfe ich vergebens mit meinem Telephon. Da
               ich nicht zu Ihnen kann (meine Schwester\pwindex{Clemenceau, Sophie 25.\,5.\,1862 – 24.\,9.\,1937@\textsc{Clemenceau, Sophie} (25.\,5.\,1862 – 24.\,9.\,1937)|pwv} u mein Bruder\pwindex{Szeps, Julius 27.\,10.\,1867 Wien – 27.\,10.\,1924 ebd.@\textsc{Szeps, Julius} (27.\,10.\,1867 Wien – 27.\,10.\,1924 ebd.), \emph{Journalist}|pwv} nehmen mir jede Minute) so will ich Ihnen wenigstens nur das Eine
               sagen.\pend
           
\pstart
           Ich habe mich längst damit abgefunden, und bin sogar froh, nicht mehr Kritiker zu
               sein. Aber diesmal war mir weh um’s Herz. Diesmal hätte ich so sehnsüchtig Vieles,
               Vieles sagen wollen, über dieses Ihr verschleiert wehmütig wissendes Werk\pwindex{Schnitzler, Arthur 15. 5. 1862 Wien – 21. 10. 1931 ebd.@\textsc{Schnitzler, Arthur} (15. 5. 1862 Wien – 21. 10. 1931 ebd.), \emph{Schriftsteller, Mediziner}!Komödie der Verführung. In drei Akten@\strich\emph{Komödie der Verführung. In drei Akten}|pwv}. So kann ich Ihnen nur danken {\pb}für so viel Reichtum den Sie schencken{\dotsfive} wenn man versteht..........\pend
           
\pstart
           Ich hoffe Sie \label{K_L04008-1v}\edtext{noch diese Woche}{\lemma{\textnormal{\emph{noch diese
                  Woche}}}\Cendnote{\textnormal{Am 17. 10. 1924 traf Schnitzler{ }Zuckerkandl\pwindex{Zuckerkandl, Berta 13.\,4.\,1864 Wien – 16.\,10.\,1945 Paris@\textsc{Zuckerkandl, Berta} (13.\,4.\,1864 Wien – 16.\,10.\,1945 Paris), \emph{Schriftstellerin, Journalistin, Übersetzerin}|pwk} und ihre Schwester Sophie Clemenceau\pwindex{Clemenceau, Sophie 25.\,5.\,1862 – 24.\,9.\,1937@\textsc{Clemenceau, Sophie} (25.\,5.\,1862 – 24.\,9.\,1937)|pwk} bei einem Konzert von Jean Untermeyer\eventindex{Musikverein@\textbf{Musikverein}!Konzert von Jean Untermeyer, 17.10.1924@Konzert von Jean Untermeyer, 17.10.1924|pwk}, am 28. 10. 1924 besuchte
                  er Zuckerkandl\pwindex{Zuckerkandl, Berta 13.\,4.\,1864 Wien – 16.\,10.\,1945 Paris@\textsc{Zuckerkandl, Berta} (13.\,4.\,1864 Wien – 16.\,10.\,1945 Paris), \emph{Schriftstellerin, Journalistin, Übersetzerin}|pwk} zuhause, um ihr zu
                  kondilieren, da ihr Bruder Julius Szeps\pwindex{Szeps, Julius 27.\,10.\,1867 Wien – 27.\,10.\,1924 ebd.@\textsc{Szeps, Julius} (27.\,10.\,1867 Wien – 27.\,10.\,1924 ebd.), \emph{Journalist}|pwk} am
                  Tag zuvor verstorben war. }}}\label{K_L04008-1} sprechen zu können.\pend
           
\pstart
           In Treue Ihre {\\[\baselineskip]}\spacefill\mbox{Berta Zuckerkandl}\pend
           \leftskip=0em{}
\pstart
           \noindent{}Bitte grüssen Sie Olga\pwindex{Schnitzler, Olga 17.\,1.\,1882 Wien – 13.\,1.\,1970 Lugano@\textsc{Schnitzler, Olga} (17.\,1.\,1882 Wien – 13.\,1.\,1970 Lugano), \emph{Schauspielerin, Sängerin}|pw} von mir. Wenn ich
                  bisher nicht kommen konnte so sind eben meine Familienangelegenheiten daran
                  Schuld.\pend
           \selectlanguage{ngerman}\endnumbering\briefempfaengerindex{Schnitzler, Arthur@\textsc{Schnitzler, Arthur}!zzzZuckerkandl, Berta@\emph{von Berta Zuckerkandl}!1924-10-111@{11. 10. 1924}|)be}\mylabel{L04008h}
\begin{anhang}
\end{anhang}\newcommand{\dateiname}{L04008}\newcommand{\titel}{Berta Zuckerkandl an Arthur Schnitzler, 11. 10. 1924}\newcommand{\editorInnen}{Herausgegeben von Jahnke, SelmaMüller, Martin Anton}%% latex-leseansicht-abspann.tex
%% Abspann für die Leseansicht.
%% Der Schalter \ifkorrekturansicht ist bereits durch den Vorspann gesetzt.

%% latex-abspann.tex
%% Gemeinsamer Abspann für Korrekturansicht und Leseansicht.
%% Setzt den Schalter \ifkorrekturansicht voraus (gesetzt in den
%% einbindenden Dateien latex-korrekturansicht-abspann.tex bzw.
%% latex-leseansicht-abspann.tex).
%% ---------------------------------------------------------------

\normalsize

% Das esempio-Environment wird nur in der Leseansicht benötigt
\ifkorrekturansicht\else
\newenvironment{esempio}[3]%
{
    \vspace{1.5ex}
    \rlap{\underline{#1}}
    \par
    \setlength{\parindent}{0cm}
    \nopagebreak
    \leftskip=#2cm
    \rightskip=#3cm
}
{
    \par
}
\fi

\doendnotes{C}
\bigskip
\vfill

\clearpage

\footnotesize

\ifkorrekturansicht
  \lohead{\textsc{register}}
\fi

% theindex-Environment neu definieren ohne reledmac
\makeatletter
\renewenvironment{theindex}{%
  \ifkorrekturansicht
    \section*{\indexname}%
  \else
    \subsubsection*{Index der erwähnten Entitäten}%
  \fi
  \setlength{\parindent}{0pt}%
  \setlength{\parskip}{0pt plus 0.3pt}%
  \let\item\@idxitem
}{%
  \ifkorrekturansicht\clearpage\fi
}
\makeatother

\IfFileExists{\jobname-pw.ind}{\input{\jobname-pw.ind}}{}

% Quellenangabe nur in der Leseansicht
\ifkorrekturansicht\else
% Fallback-Definitionen, falls die .tex-Datei \titel etc. nicht gesetzt hat
\providecommand{\titel}{}
\providecommand{\editorInnen}{}
\providecommand{\dateiname}{\jobname}

\vspace{3cm}

\vfill

\footnotesize
\textsc{Quelle}: \titel. Herausgegeben von {\editorInnen}. In: \emph{Arthur Schnitzler: Briefwechsel mit Autorinnen und Autoren}.
 Digitale Edition, https://schnitzler-briefe.acdh.oeaw.ac.at/{\dateiname}.html (Stand \today)
\fi

\end{document}


