%% latex-korrekturansicht-vorspann.tex
%% Vorspann für die Korrekturansicht.
%% Lädt die gemeinsame Datei latex-vorspann.tex mit gesetztem Schalter.

\newif\ifkorrekturansicht
\korrekturansichttrue

\input{../tex-inputs/latex-vorspann}


\section[Thomas Mann an Arthur Schnitzler, 20. 2. 1923]{L02397 Thomas Mann an Arthur Schnitzler, 20. 2. 1923}
\nopagebreak\mylabel{L02397v}
\rehead{ }\normalsize\beginnumbering\briefempfaengerindex{Schnitzler, Arthur@\textsc{Schnitzler, Arthur}!zzzMann, Thomas@\emph{von Thomas Mann}!1923-02-201@{20. 2. 1923}|(be}
\toendnotes[C]{\smallbreak\pagebreak[2]}\Standort{CUL, Schnitzler, B 67.}
\physDesc{Brief, 1 Blatt, 3 Seiten, 1951 Zeichen
\newline{}Handschrift: schwarze Tinte, deutsche Kurrent
\newline{}Schnitzler: mit rotem Buntstift mehrere Unterstreichungen 
\newline{}Ordnung: mit Bleistift von Frieda
                                    Pollak\pwindex{Pollak, Frieda 08.12.1881 – 13.07.1937@\textsc{Pollak, Frieda} (08.12.1881 – 13.07.1937), \emph{Sekretär/Sekretärin}|pw} (?) mit dem Buchstaben »A«
                                 (Abgeschrieben/Abschrift) gekennzeichnet }
\buchAbdrucke{\weitereDrucke{\emph{Modern Austrian Literature}, Jg. 7 (1974) Nr. 1/2, S. 20–21.} }\toendnotes[C]{\smallbreak}
\pstart
           {\pb}\textcolor{gray}{\textbf{\textsc{Dr. Thomas Mann}}}\hfill \textcolor{gray}{\textbf{MÜNCHEN\oindex{Muenchen@\textbf{München}, \emph{P.PPLA}|pw}, DEN}}{ }20. II. 23.\pend
           
\pstart
           \raggedleft{}\textcolor{gray}{\textbf{POSCHINGERSTR. 1\oindex{Poschingerstrasse@\textbf{Poschingerstraße}, \emph{Straße (K.STR)}|pw}}}\pend
           
\pstart{}Verehrter Herr Dr. Schnitzler!\pend\vspace{0.5em}
\pstart
           Für Ihren liebenswürdigen Brief vom Dezember habe ich noch vielmals zu
               danken. Die Abenteuer Caſanovas\pwindex{Casanova, Giacomo Girolamo 02.04.1725 – 04.06.1798@\textsc{Casanova, Giacomo Girolamo} (02.04.1725 – 04.06.1798), \emph{Schriftsteller/Schriftstellerin, Abenteurer/Abenteurerin}|pwv}\pwindex{Casanovas Heimfahrt@\emph{Casanovas Heimfahrt}|pw} in Amerika\oindex{Amerika@\textbf{Amerika}, \emph{kein passender Code gefunden}|pw} haben mich ſehr amüſiert. Auf
               die Frage, die Sie daran knüpfen, weiß ich nicht viel zu antworten, denn meine
               Erfahrungen mit Kirpatrick + Brandt\orgindex{Brandt und Kirkpatrick@Brandt {\kaufmannsund}  Kirkpatrick|pw}{ }ſind beſchränkt. Vor Jahr und Tag wurde »Buddenbrooks\pwindex{Buddenbrooks@\emph{Buddenbrooks}|pw}« nach Amerika\oindex{Amerika@\textbf{Amerika}, \emph{kein passender Code gefunden}|pw} verkauft, das iſt alles. Die Bezahlung war nicht ſchlecht:
               500 Dollars, wenn ich nicht irre. Aber das Riſiko iſt auch wohl groß, – obgleich der
               Roman durch einen Buddenbrook-Film\pwindex{Buddenbrooks@\emph{Buddenbrooks}|pw} geſtützt
               werden wird, den zur Zeit eine Berlin\oindex{Berlin@\textbf{Berlin}, \emph{P.PPLC}|pw}er
               Export-Firma mit meiner ſchamloſen Zuſtimmung {\pb}herzuſtellen im Begriffe iſt. Was wollen
               Sie, – das ist der Krieg!\pend
           
\pstart
           Auf Ihre freundlichen Worte über den republikaniſchen Aufſatz\pwindex{Von deutscher Republik. Gerhart Hauptmann zum sechzigsten Geburtstag@\emph{Von deutscher Republik. Gerhart Hauptmann zum sechzigsten Geburtstag}|pwv} bilde ich mir nicht wenig ein. Seien Sie überzeugt,
               daß ich Ihre Skepſis in Hinſicht auf die Bedeutung poſitiver Staatsformen vollkommen
               teile. Hier handelte es ſich für mich um eine rein praktiſche Aktion, mit der ich in
               gewiſſen Grenzen \uline{genützt} zu haben glaube, denn der
                  Artikel\pwindex{Von deutscher Republik. Gerhart Hauptmann zum sechzigsten Geburtstag@\emph{Von deutscher Republik. Gerhart Hauptmann zum sechzigsten Geburtstag}|pwv} iſt im Auslande
               viel excerpiert worden. Aber freilich gegen die Thorheit der Franzoſen\oindex{Frankreich@\textbf{Frankreich}, \emph{A.PCLI}|pw} iſt nicht aufzuko{\geminationm}en.
               Offenbar haben ſie es ſich in den Kopf geſetzt, jedem das Konzept zu verderben, der
               verſucht, in Deutſchland\oindex{Deutschland@\textbf{Deutschland}, \emph{A.PCLI}|pw} zum Guten zu reden. Man
               verſichert, daß die \textsc{Détails} von der Ruhr\oindex{Ruhrgebiet@\textbf{Ruhrgebiet}, \emph{L.RGN}|pw} nicht nur nicht übertrieben ſind, ſon{\pb}dern ſogar noch hinter der Wahrheit
               zurückbleiben. Der Ingrimm iſt fürchterlich, und man ſieht nicht ab, was einmal
               daraus werden ſoll.\pend
           
\pstart
           Ich wollte Sie um Folgendes bitten. Felix
                  Salten\pwindex{Salten, Felix 06.09.1869 – 08.10.1945@\textsc{Salten, Felix} (06.09.1869 – 08.10.1945), \emph{Schriftsteller/Schriftstellerin, Journalist/Journalistin, Chefredakteur/Chefredakteurin}|pw} hatte die große Freundlichkeit, mir ſein Buch »Bambi\pwindex{Bambi. Eine Lebensgeschichte aus dem Walde@\emph{Bambi. Eine Lebensgeschichte aus dem Walde}|pw}« zu ſchicken, – und ich habe ſeine Adreſſe nicht. Wollen
               Sie es gütigſt übernehmen, ihm in meinem Auftrage zu \uline{danken}? Ich finde dieſe Tier- und Waldgeſchichte reizend, erquickend, voll
               von Humor und Natur. Sagen Sie ihm das!\pend
           
\pstart
           Ich komme Ende März nach Wien\oindex{Wien@\textbf{Wien}, \emph{A.ADM2}|pw} (wahrſcheinlich
               wieder mit meiner Frau\pwindex{Mann, Katia 24.07.1883 – 25.04.1980@\textsc{Mann, Katia} (24.07.1883 – 25.04.1980)|pwv}) und
               will hoffen, daß Sie dann noch nicht im Norden ſind.\pend
           
\pstart
           Ihr ergebener{\\[\baselineskip]}\spacefill\mbox{Thomas Mann.}\pend
           \leftskip=0em{}\selectlanguage{ngerman}\endnumbering\briefempfaengerindex{Schnitzler, Arthur@\textsc{Schnitzler, Arthur}!zzzMann, Thomas@\emph{von Thomas Mann}!1923-02-201@{20. 2. 1923}|)be}\mylabel{L02397h}  \normalsize

\doendnotes{C}
\bigskip
\vfill

\clearpage

\footnotesize

\lohead{\textsc{register}}

% Definiere theindex-Environment komplett neu ohne reledmac
\makeatletter
\renewenvironment{theindex}{%
  \section*{\indexname}%
  \setlength{\parindent}{0pt}%
  \setlength{\parskip}{0pt plus 0.3pt}%
  \let\item\@idxitem
}{%
  \clearpage
}
\makeatother

\IfFileExists{\jobname-pw.ind}{\input{\jobname-pw.ind}}{}

\end{document}

      