%% latex-leseansicht-vorspann.tex
%% Vorspann für die Leseansicht.
%% Lädt die gemeinsame Datei latex-vorspann.tex mit nicht gesetztem Schalter.

\newif\ifkorrekturansicht
\korrekturansichtfalse

\input{../tex-inputs/latex-vorspann}


\section[Thomas Mann an Arthur Schnitzler, 20. 2. 1923]{L02397 Thomas Mann an Arthur Schnitzler, 20. 2. 1923}
\nopagebreak\mylabel{L02397v}
\rehead{ }\normalsize\beginnumbering\briefempfaengerindex{Schnitzler, Arthur@\textsc{Schnitzler, Arthur}!zzzMann, Thomas@\emph{von Thomas Mann}!1923-02-201@{20. 2. 1923}|(be}
\toendnotes[C]{\smallbreak\pagebreak[2]}
\correspDesc{Versand  durch Thomas Mann am 20. 2. 1923 in München
\newline{}Erhalt  durch Arthur Schnitzler im Zeitraum [21. 2. 1923
                  – 25. 2. 1923?] in Wien}\toendnotes[C]{\smallbreak}
\Standort{CUL, Schnitzler, B 67.}
\physDesc{Brief, 1 Blatt, 3 Seiten, 1951 Zeichen
\newline{}Handschrift: schwarze Tinte, deutsche Kurrent
\newline{}Schnitzler: mit rotem Buntstift mehrere Unterstreichungen 
\newline{}Ordnung: mit Bleistift von Frieda
                                    Pollak\pwindex{Pollak, Frieda 8.\,12.\,1881 Wien – 13.\,7.\,1937 ebd.@\textsc{Pollak, Frieda} (8.\,12.\,1881 Wien – 13.\,7.\,1937 ebd.), \emph{Sekretärin}|pw} (?) mit dem Buchstaben »A«
                                 (Abgeschrieben/Abschrift) gekennzeichnet }
\buchAbdrucke{\weitereDrucke{Hertha Krotkoff: \emph{Arthur Schnitzler – Thomas Mann: Briefe.} In: \emph{Modern Austrian Literature}, Jg. 7 (1974) Nr. 1/2, S. 20–21.} }\toendnotes[C]{\smallbreak}
\pstart
           {\pb}\textcolor{gray}{\textbf{\textsc{Dr. Thomas Mann}}}\hfill \textcolor{gray}{\textbf{MÜNCHEN\oindex{München@\textbf{München}|pw}, DEN}}{ }20. II. 23.\pend
           
\pstart
           \raggedleft{}\textcolor{gray}{\textbf{POSCHINGERSTR. 1\oindex{Poschingerstraße@\textbf{Poschingerstraße}, \emph{Straße}|pw}}}\pend
           
\pstart{}Verehrter Herr Dr. Schnitzler!\pend\vspace{0.5em}
\pstart
           Für Ihren liebenswürdigen Brief vom Dezember habe ich noch vielmals zu
               danken. Die Abenteuer Caſanovas\pwindex{Casanova, Giacomo Girolamo 2.\,4.\,1725 Venedig – 4.\,6.\,1798 Duchcov@\textsc{Casanova, Giacomo Girolamo} (2.\,4.\,1725 Venedig – 4.\,6.\,1798 Duchcov), \emph{Schriftsteller, Abenteurer}|pwv}\pwindex{Schnitzler, Arthur 15.\,5.\,1862 Wien – 21.\,10.\,1931 ebd.@\textsc{Schnitzler, Arthur} (15.\,5.\,1862 Wien – 21.\,10.\,1931 ebd.), \emph{Schriftsteller, Mediziner}!Casanovas Heimfahrt@\strich\emph{Casanovas Heimfahrt}|pw} in Amerika\oindex{Amerika@\textbf{Amerika}|pw} haben mich{ }ſehr amüſiert. Auf
               die Frage, die Sie daran knüpfen, weiß ich nicht viel zu antworten, denn meine
               Erfahrungen mit Kirpatrick + Brandt\orgindex{Brandt und Kirkpatrick@Brandt {\kaufmannsund}  Kirkpatrick|pw}{ }ſind beſchränkt. Vor Jahr und Tag wurde »Buddenbrooks\pwindex{Mann, Thomas 6.\,6.\,1875 Lübeck – 12.\,8.\,1955 Zürich@\textsc{Mann, Thomas} (6.\,6.\,1875 Lübeck – 12.\,8.\,1955 Zürich), \emph{Schriftsteller}!Buddenbrooks@\strich\emph{Buddenbrooks}|pw}« nach Amerika\oindex{Amerika@\textbf{Amerika}|pw} verkauft, das iſt alles. Die Bezahlung war nicht{ }ſchlecht:
               500 Dollars, wenn ich nicht irre. Aber das Riſiko iſt auch wohl groß, – obgleich der
               Roman durch einen Buddenbrook-Film\pwindex{\textcolor{red}{\textsuperscript{XXXX indx1}}!Buddenbrooks@\strich\emph{Buddenbrooks}|pw} geſtützt
               werden wird, den zur Zeit eine Berlin\oindex{Berlin@\textbf{Berlin}, \emph{Hauptstadt}|pw}er
               Export-Firma mit meiner{ }ſchamloſen Zuſtimmung {\pb}herzuſtellen im Begriffe iſt. Was wollen
               Sie, – das ist der Krieg!\pend
           
\pstart
           Auf Ihre freundlichen Worte über den republikaniſchen Aufſatz\pwindex{Mann, Thomas 6.\,6.\,1875 Lübeck – 12.\,8.\,1955 Zürich@\textsc{Mann, Thomas} (6.\,6.\,1875 Lübeck – 12.\,8.\,1955 Zürich), \emph{Schriftsteller}!Von deutscher Republik. Gerhart Hauptmann zum sechzigsten Geburtstag@\strich\emph{Von deutscher Republik. Gerhart Hauptmann zum sechzigsten Geburtstag}|pwv} bilde ich mir nicht wenig ein. Seien Sie überzeugt,
               daß ich Ihre Skepſis in Hinſicht auf die Bedeutung poſitiver Staatsformen vollkommen
               teile. Hier handelte es{ }ſich für mich um eine rein praktiſche Aktion, mit der ich in
               gewiſſen Grenzen \uline{genützt} zu haben glaube, denn der
                  Artikel\pwindex{Mann, Thomas 6.\,6.\,1875 Lübeck – 12.\,8.\,1955 Zürich@\textsc{Mann, Thomas} (6.\,6.\,1875 Lübeck – 12.\,8.\,1955 Zürich), \emph{Schriftsteller}!Von deutscher Republik. Gerhart Hauptmann zum sechzigsten Geburtstag@\strich\emph{Von deutscher Republik. Gerhart Hauptmann zum sechzigsten Geburtstag}|pwv} iſt im Auslande
               viel excerpiert worden. Aber freilich gegen die Thorheit der Franzoſen\oindex{Frankreich@\textbf{Frankreich}|pw} iſt nicht aufzuko{\geminationm}en.
               Offenbar haben{ }ſie es{ }ſich in den Kopf geſetzt, jedem das Konzept zu verderben, der
               verſucht, in Deutſchland\oindex{Deutschland@\textbf{Deutschland}|pw} zum Guten zu reden. Man
               verſichert, daß die \textsc{Détails} von der Ruhr\oindex{Ruhrgebiet@\textbf{Ruhrgebiet}, \emph{Region}|pw} nicht nur nicht übertrieben{ }ſind,{ }ſon{\pb}dern{ }ſogar noch hinter der Wahrheit
               zurückbleiben. Der Ingrimm iſt fürchterlich, und man{ }ſieht nicht ab, was einmal
               daraus werden{ }ſoll.\pend
           
\pstart
           Ich wollte Sie um Folgendes bitten. Felix
                  Salten\pwindex{Salten, Felix 6.\,9.\,1869 Budapest – 8.\,10.\,1945 Zürich@\textsc{Salten, Felix} (6.\,9.\,1869 Budapest – 8.\,10.\,1945 Zürich), \emph{Schriftsteller, Journalist, Chefredakteur}|pw} hatte die große Freundlichkeit, mir{ }ſein Buch »Bambi\pwindex{Salten, Felix 6.\,9.\,1869 Budapest – 8.\,10.\,1945 Zürich@\textsc{Salten, Felix} (6.\,9.\,1869 Budapest – 8.\,10.\,1945 Zürich), \emph{Schriftsteller, Journalist, Chefredakteur}!Bambi. Eine Lebensgeschichte aus dem Walde@\strich\emph{Bambi. Eine Lebensgeschichte aus dem Walde}|pw}« zu{ }ſchicken, – und ich habe{ }ſeine Adreſſe nicht. Wollen
               Sie es gütigſt übernehmen, ihm in meinem Auftrage zu \uline{danken}? Ich finde dieſe Tier- und Waldgeſchichte reizend, erquickend, voll
               von Humor und Natur. Sagen Sie ihm das!\pend
           
\pstart
           Ich komme Ende März nach Wien\oindex{Wien@\textbf{Wien}, \emph{Verwaltungsgebiet}|pw} (wahrſcheinlich
               wieder mit meiner Frau\pwindex{Mann, Katia 24.\,7.\,1883 Feldafing – 25.\,4.\,1980 Kilchberg@\textsc{Mann, Katia} (24.\,7.\,1883 Feldafing – 25.\,4.\,1980 Kilchberg)|pwv}) und
               will hoffen, daß Sie dann noch nicht im Norden{ }ſind.\pend
           
\pstart
           Ihr ergebener{\\[\baselineskip]}\spacefill\mbox{Thomas Mann.}\pend
           \leftskip=0em{}\selectlanguage{ngerman}\endnumbering\briefempfaengerindex{Schnitzler, Arthur@\textsc{Schnitzler, Arthur}!zzzMann, Thomas@\emph{von Thomas Mann}!1923-02-201@{20. 2. 1923}|)be}\mylabel{L02397h}  \newcommand{\dateiname}{L02397}\newcommand{\titel}{Thomas Mann an Arthur Schnitzler, 20. 2. 1923}\newcommand{\editorInnen}{Martin Anton Müller und Gerd-Hermann Susen}%% latex-leseansicht-abspann.tex
%% Abspann für die Leseansicht.
%% Der Schalter \ifkorrekturansicht ist bereits durch den Vorspann gesetzt.

%% latex-abspann.tex
%% Gemeinsamer Abspann für Korrekturansicht und Leseansicht.
%% Setzt den Schalter \ifkorrekturansicht voraus (gesetzt in den
%% einbindenden Dateien latex-korrekturansicht-abspann.tex bzw.
%% latex-leseansicht-abspann.tex).
%% ---------------------------------------------------------------

\normalsize

% Das esempio-Environment wird nur in der Leseansicht benötigt
\ifkorrekturansicht\else
\newenvironment{esempio}[3]%
{
    \vspace{1.5ex}
    \rlap{\underline{#1}}
    \par
    \setlength{\parindent}{0cm}
    \nopagebreak
    \leftskip=#2cm
    \rightskip=#3cm
}
{
    \par
}
\fi

\doendnotes{C}
\bigskip
\vfill

\clearpage

\footnotesize

\ifkorrekturansicht
  \lohead{\textsc{register}}
\fi

% theindex-Environment neu definieren ohne reledmac
\makeatletter
\renewenvironment{theindex}{%
  \ifkorrekturansicht
    \section*{\indexname}%
  \else
    \subsubsection*{Index der erwähnten Entitäten}%
  \fi
  \setlength{\parindent}{0pt}%
  \setlength{\parskip}{0pt plus 0.3pt}%
  \let\item\@idxitem
}{%
  \ifkorrekturansicht\clearpage\fi
}
\makeatother

\IfFileExists{\jobname-pw.ind}{\input{\jobname-pw.ind}}{}

% Quellenangabe nur in der Leseansicht
\ifkorrekturansicht\else
% Fallback-Definitionen, falls die .tex-Datei \titel etc. nicht gesetzt hat
\providecommand{\titel}{}
\providecommand{\editorInnen}{}
\providecommand{\dateiname}{\jobname}

\vspace{3cm}

\vfill

\footnotesize
\textsc{Quelle}: \titel. Herausgegeben von {\editorInnen}. In: \emph{Arthur Schnitzler: Briefwechsel mit Autorinnen und Autoren}.
 Digitale Edition, https://schnitzler-briefe.acdh.oeaw.ac.at/{\dateiname}.html (Stand \today)
\fi

\end{document}


