%% latex-korrekturansicht-vorspann.tex
%% Vorspann für die Korrekturansicht.
%% Lädt die gemeinsame Datei latex-vorspann.tex mit gesetztem Schalter.

\newif\ifkorrekturansicht
\korrekturansichttrue

\input{../tex-inputs/latex-vorspann}


\section[Arthur Schnitzler an Richard Beer-Hofmann, 19. 9. 1892]{L00124 Arthur Schnitzler an Richard Beer-Hofmann, 19. 9. 1892}
\nopagebreak\mylabel{L00124v}
\rehead{ }\normalsize\beginnumbering\briefempfaengerindex{Beer-Hofmann, Richard@\textsc{Beer-Hofmann, Richard}!zzzSchnitzler, Arthur@\emph{von Arthur Schnitzler}!1892-09-191@{19. 9. 1892}|(be}
\toendnotes[C]{\smallbreak\pagebreak[2]}\Standort{YCGL, MSS 31.}
\physDesc{Brief, 1 Blatt, 2 Seiten, Umschlag, 313 Zeichen (die Innenseite des Blattes in voller Breite
                                 beschrieben)
\newline{}Handschrift: schwarze Tinte, deutsche Kurrent
\newline{}Versand: 1) Stempel: »\nobreak{}\oindex{Venedig@\textbf{Venedig}, \emph{P.PPLA}|pwk}Venezia, 19 {[}9 92{]}, 8 S\nobreak{}«.   2) Stempel: »\nobreak{}\oindex{Theater in der Josefstadt@\textbf{Theater in der Josefstadt}, \emph{Theater (K.THE)}|pwk}Wien 1/1, 21/9 92, 9–9½ V., Bestellt\nobreak{}«. }
\buchAbdrucke{\weitereDrucke{Arthur Schnitzler, Richard Beer-Hofmann: \emph{Briefwechsel 1891–1931}. Wien, Zürich: \emph{Europaverlag} 1992, S. 39.} }\pstart{}{\pb}\textsc{Austria\oindex{Oesterreich@\textbf{Österreich}, \emph{A.PCLI}|pw}}\pend{}\pstart{}Hrn Dr. \textsc{Richard Beer Hofmann}\pend{}\pstart{}\textsc{Wien\oindex{Wien@\textbf{Wien}, \emph{A.ADM2}|pw}.}\pend{}\pstart{}\textsc{I Wollzeile 15\oindex{Wollzeile@\textbf{Wollzeile}, \emph{Straße (K.STR)}|pw}}.\pend{}{\bigskip}\vspace{1em}
\pstart
           \noindent{}{\pb}Ich will Ihnen nur, mein lieber Richard, aus Venedig\oindex{Venedig@\textbf{Venedig}, \emph{P.PPLA}|pw} die herzlichſten Grüße
               ſagen. Wir ſehn uns wohl bald wieder. Für alle Fälle mögen Sie wiſſen, {\pb}daß ich Montag den 26. September Abends
               nach zehn Uhr im Café \textsc{Kremser}\oindex{Cafe Kremser@\textbf{Café Kremser}, \emph{Kaffeehaus (K.KAF)}|pw} zu ſein beabſichtige.\pend
           
\pstart
           Ganz der Ihre{\\[\baselineskip]}\spacefill\mbox{Arthur}\pend
           \leftskip=0em{}
\pstart
           Am 19. 9. 1892\pend
           \selectlanguage{ngerman}\endnumbering\briefempfaengerindex{Beer-Hofmann, Richard@\textsc{Beer-Hofmann, Richard}!zzzSchnitzler, Arthur@\emph{von Arthur Schnitzler}!1892-09-191@{19. 9. 1892}|)be}\mylabel{L00124h}  \normalsize

\doendnotes{C}
\bigskip
\vfill

\clearpage

\footnotesize

\lohead{\textsc{register}}

% Definiere theindex-Environment komplett neu ohne reledmac
\makeatletter
\renewenvironment{theindex}{%
  \section*{\indexname}%
  \setlength{\parindent}{0pt}%
  \setlength{\parskip}{0pt plus 0.3pt}%
  \let\item\@idxitem
}{%
  \clearpage
}
\makeatother

\IfFileExists{\jobname-pw.ind}{\input{\jobname-pw.ind}}{}

\end{document}

      