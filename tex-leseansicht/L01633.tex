%% latex-korrekturansicht-vorspann.tex
%% Vorspann für die Korrekturansicht.
%% Lädt die gemeinsame Datei latex-vorspann.tex mit gesetztem Schalter.

\newif\ifkorrekturansicht
\korrekturansichttrue

\input{../tex-inputs/latex-vorspann}


\section[Hermann Bahr an Arthur Schnitzler, 16. 10. 1906]{L01633 Hermann Bahr an Arthur Schnitzler, 16. 10. 1906}
\nopagebreak\mylabel{L01633v}
\rehead{ }\normalsize\beginnumbering\briefempfaengerindex{Schnitzler, Arthur@\textsc{Schnitzler, Arthur}!zzzBahr, Hermann@\emph{von Hermann Bahr}!1906-10-161@{16. 10. 1906}|(be}
\toendnotes[C]{\smallbreak\pagebreak[2]}\Standort{CUL, Schnitzler, B 5b.}
\physDesc{Brief, 1 Blatt, 2 Seiten, 846 Zeichen
\newline{}Handschrift Lisa Clarus: blaue Tinte, lateinische Kurrent
\newline{}Handschrift Hermann Bahr: blaue Tinte
\newline{}Schnitzler: mit Bleistift ergänzt »Bahr« 
\newline{}Ordnung: mit Bleistift von unbekannter Hand nummeriert:
                                    »142« }
\buchAbdrucke{\weitereDrucke{Hermann Bahr, Arthur Schnitzler: \emph{Briefwechsel, Aufzeichnungen, Dokumente (1891–1931)}. Göttingen: \emph{Wallstein} 2018, S. 383.} }\toendnotes[C]{\smallbreak}
\pstart
           \raggedleft{}{\pb}16. 10. 06.\pend
           
\pstart\center{}Lieber Arthur!\pend\vspace{0.5em}
\pstart
           Ich sende Dir beiliegend einen kleinen Akt\pwindex{tiefe Natur. Ein Akt@\emph{Die tiefe Natur. Ein Akt}|pwv} mit der Frage, ob Du was dagegen hast, dass ich ihm,
               wenn er gedruckt wird, die folgende Widmung vorsetze:\pend
           
\pstart
           \centering{}»In Erinnerung an meinen lieben Anatol«.\pend
           
\pstart
           Mir ist nämlich folgendes passiert: Ich schrieb den Akt »nach einer wahren
               Begebenheit« (worüber gelegentlich einmal mündlich), eigentlich nur, weil ich Spass
               an der weiblichen Figur fand; nun teilt mir Burckhard\pwindex{Burckhard, Max Eugen 14.07.1854 – 16.03.1912@\textsc{Burckhard, Max Eugen} (14.07.1854 – 16.03.1912), \emph{Schriftsteller/Schriftstellerin, Rechtswissenschaftler/Rechtswissenschaftlerin, Theaterleiter/Theaterleiterin}|pw} mit, dass es eigentlich das Abschiedsouper\pwindex{Abschiedssouper@\emph{Abschiedssouper}|pw} ist, ich erschrecke, denke nach und – {\pb}kann es nicht läugnen. Du wirst mir glauben, dass
               es unbewusst war. Ich möchte aber doch jedenfalls öffentlich irgendwie den wahren
               Autor nennen: daher die Widmung.\pend
           
\pstart
           Ich lasse Deiner lieben Frau\pwindex{Schnitzler, Olga 17.01.1882 – 13.01.1970@\textsc{Schnitzler, Olga} (17.01.1882 – 13.01.1970), \emph{Schauspieler/Schauspielerin, Sänger/Sängerin}|pwv}
               herzlichst für die \label{K_L01633-1v}\edtext{Bilder}{\lemma{\textnormal{\emph{Bilder}}}\Cendnote{\textnormal{nicht ermittelt}}}\label{K_L01633-1} danken und werde
               mich sehr freuen, wenn sie mir erlaubt, ihr gelegentlich eines zu bringen\pwindex{Hermann Bahr@\emph{Hermann Bahr}|pwuv}.\pend
           
\pstart
           Mit vielen herzlichen Grüssen{\\[\baselineskip]}Dein{\\[\baselineskip]}\spacefill\mbox{{[}hs. :{]} Hermann}\pend
           \leftskip=0em{}\selectlanguage{ngerman}\endnumbering\briefempfaengerindex{Schnitzler, Arthur@\textsc{Schnitzler, Arthur}!zzzBahr, Hermann@\emph{von Hermann Bahr}!1906-10-161@{16. 10. 1906}|)be}\mylabel{L01633h}  \normalsize

\doendnotes{C}
\bigskip
\vfill

\clearpage

\footnotesize

\lohead{\textsc{register}}

% Definiere theindex-Environment komplett neu ohne reledmac
\makeatletter
\renewenvironment{theindex}{%
  \section*{\indexname}%
  \setlength{\parindent}{0pt}%
  \setlength{\parskip}{0pt plus 0.3pt}%
  \let\item\@idxitem
}{%
  \clearpage
}
\makeatother

\IfFileExists{\jobname-pw.ind}{\input{\jobname-pw.ind}}{}

\end{document}

      