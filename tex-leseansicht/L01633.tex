%% latex-leseansicht-vorspann.tex
%% Vorspann für die Leseansicht.
%% Lädt die gemeinsame Datei latex-vorspann.tex mit nicht gesetztem Schalter.

\newif\ifkorrekturansicht
\korrekturansichtfalse

\input{../tex-inputs/latex-vorspann}


         
         \newcommand{\erwaehntePersonen}{Personen: }
         \newcommand{\erwaehnteInstitutionen}{}
         \newcommand{\erwaehnteOrte}{}
         \newcommand{\erwaehnteWerke}{
               \section[Hermann Bahr an Arthur Schnitzler, 16. 10. 1906]{ Hermann Bahr an Arthur Schnitzler, 16. 10. 1906}\nopagebreak\mylabel{v}\rehead{ }\begin{ledgroupsized}[t]{13cm}\normalsize\beginnumbering \toendnotes[C]{\smallbreak\pagebreak[2]} \Standort{CUL, Schnitzler, B 5b.}
\physDesc{Brief, 1 Blatt, 2 Seiten
\newline{}Handschrift  : blaue Tinte, lateinische Kurrent\newline{}Handschrift  : blaue Tinte
\newline{}Schnitzler: mit Bleistift ergänzt »Bahr« \newline{}Ordnung: mit Bleistift von unbekannter Hand nummeriert: »142« }\buchAbdrucke{\weitereDrucke{Hermann Bahr, Arthur Schnitzler: \emph{Briefwechsel, Aufzeichnungen, Dokumente (1891–1931)}. Hg. Kurt Ifkovits und Martin Anton Müller. Göttingen: \emph{Wallstein} 2018, S. 383.} }\toendnotes[C]{\smallbreak}\pstart
           \raggedleft{}{\pb}16. 10. 06.\pend
           \pstart\center{}Lieber Arthur!\pend\pstart
           Ich sende Dir beiliegend einen kleinen Akt\textcolor{red}{\textsuperscript{XXXX indx}} mit der Frage, ob Du was dagegen hast, dass ich ihm, wenn er gedruckt
               wird, die folgende Widmung vorsetze:\pend
           \pstart
           \centering{}»In Erinnerung an meinen lieben Anatol«.\pend
           \pstart
           \noindent{}Mir ist nämlich folgendes passiert: Ich schrieb den Akt »nach einer wahren
               Begebenheit« (worüber gelegentlich einmal mündlich), eigentlich nur, weil ich Spass
               an der weiblichen Figur fand; nun teilt mir Burckhard\pwindex{\textcolor{red}{\textsuperscript{XXXX1 indx}}|pw} mit, dass es eigentlich das Abschiedsouper\textcolor{red}{\textsuperscript{XXXX indx}} ist, ich erschrecke, denke nach und – {\pb}kann es nicht läugnen. Du wirst mir glauben, dass
               es unbewusst war. Ich möchte aber doch jedenfalls öffentlich irgendwie den wahren
               Autor nennen: daher die Widmung.\pend
           \pstart
           Ich lasse Deiner lieben Frau\pwindex{\textcolor{red}{\textsuperscript{XXXX1 indx}}|pwv}
               herzlichst für die \label{K_L01633_1v}\edtext{Bilder}{\lemma{\textnormal{\emph{Bilder}}}\Cendnote{\textnormal{nicht ermittelt}}}\label{K_L01633_1h} danken und werde mich
               sehr freuen, wenn sie mir erlaubt, ihr gelegentlich eines zu bringen\textcolor{red}{\textsuperscript{XXXX indx}}.\pend
           \pstart
           Mit vielen herzlichen Grüssen{\\[\baselineskip]}Dein{\\[\baselineskip]}\spacefill\mbox{{[}hs. :{]} Hermann}\pend
           \leftskip=0em{}
         
         \endnumbering\mylabel{h}\end{ledgroupsized}  \newcommand{\dateiname}{L01633}\newcommand{\titel}{Hermann Bahr an Arthur Schnitzler, 16. 10. 1906}\newcommand{\editorInnen}{ Kurt Ifkovits,  Martin Anton Müller}%% latex-leseansicht-abspann.tex
%% Abspann für die Leseansicht.
%% Der Schalter \ifkorrekturansicht ist bereits durch den Vorspann gesetzt.

%% latex-abspann.tex
%% Gemeinsamer Abspann für Korrekturansicht und Leseansicht.
%% Setzt den Schalter \ifkorrekturansicht voraus (gesetzt in den
%% einbindenden Dateien latex-korrekturansicht-abspann.tex bzw.
%% latex-leseansicht-abspann.tex).
%% ---------------------------------------------------------------

\normalsize

% Das esempio-Environment wird nur in der Leseansicht benötigt
\ifkorrekturansicht\else
\newenvironment{esempio}[3]%
{
    \vspace{1.5ex}
    \rlap{\underline{#1}}
    \par
    \setlength{\parindent}{0cm}
    \nopagebreak
    \leftskip=#2cm
    \rightskip=#3cm
}
{
    \par
}
\fi

\doendnotes{C}
\bigskip
\vfill

\clearpage

\footnotesize

\ifkorrekturansicht
  \lohead{\textsc{register}}
\fi

% theindex-Environment neu definieren ohne reledmac
\makeatletter
\renewenvironment{theindex}{%
  \ifkorrekturansicht
    \section*{\indexname}%
  \else
    \subsubsection*{Index der erwähnten Entitäten}%
  \fi
  \setlength{\parindent}{0pt}%
  \setlength{\parskip}{0pt plus 0.3pt}%
  \let\item\@idxitem
}{%
  \ifkorrekturansicht\clearpage\fi
}
\makeatother

\IfFileExists{\jobname-pw.ind}{\input{\jobname-pw.ind}}{}

% Quellenangabe nur in der Leseansicht
\ifkorrekturansicht\else
% Fallback-Definitionen, falls die .tex-Datei \titel etc. nicht gesetzt hat
\providecommand{\titel}{}
\providecommand{\editorInnen}{}
\providecommand{\dateiname}{\jobname}

\vspace{3cm}

\vfill

\footnotesize
\textsc{Quelle}: \titel. Herausgegeben von {\editorInnen}. In: \emph{Arthur Schnitzler: Briefwechsel mit Autorinnen und Autoren}.
 Digitale Edition, https://schnitzler-briefe.acdh.oeaw.ac.at/{\dateiname}.html (Stand \today)
\fi

\end{document}


      