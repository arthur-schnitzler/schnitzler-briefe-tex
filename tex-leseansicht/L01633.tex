%% latex-leseansicht-vorspann.tex
%% Vorspann für die Leseansicht.
%% Lädt die gemeinsame Datei latex-vorspann.tex mit nicht gesetztem Schalter.

\newif\ifkorrekturansicht
\korrekturansichtfalse

\input{../tex-inputs/latex-vorspann}


\section[Hermann Bahr an Arthur Schnitzler, 16. 10. 1906]{L01633 Hermann Bahr an Arthur Schnitzler, 16. 10. 1906}
\nopagebreak\mylabel{L01633v}
\rehead{ }\normalsize\beginnumbering\briefempfaengerindex{Schnitzler, Arthur@\textsc{Schnitzler, Arthur}!zzzBahr, Hermann@\emph{von Hermann Bahr}!1906-10-161@{16. 10. 1906}|(be}
\toendnotes[C]{\smallbreak\pagebreak[2]}
\correspDesc{Versand  durch Hermann Bahr am 16. 10. 1906 in Wien
\newline{}Erhalt  durch Arthur Schnitzler im Zeitraum [16. 10. 1906 – 20. 10. 1906?] in Wien}\toendnotes[C]{\smallbreak}
\Standort{CUL, Schnitzler, B 5b.}
\physDesc{Brief, 1 Blatt, 2 Seiten, 846 Zeichen
\newline{}Handschrift Lisa Clarus: blaue Tinte, lateinische Kurrent
\newline{}Handschrift Hermann Bahr: blaue Tinte
\newline{}Schnitzler: mit Bleistift ergänzt »Bahr« 
\newline{}Ordnung: mit Bleistift von unbekannter Hand nummeriert:
                                    »142« }
\buchAbdrucke{\weitereDrucke{Hermann Bahr, Arthur Schnitzler: \emph{Briefwechsel, Aufzeichnungen, Dokumente (1891–1931)}. Herausgegeben von Kurt Ifkovits und Martin Anton Müller. Göttingen: \emph{Wallstein} 2018, S. 383.} }\toendnotes[C]{\smallbreak}
\pstart
           \raggedleft{}{\pb}16. 10. 06.\pend
           
\pstart\center{}Lieber Arthur!\pend\vspace{0.5em}
\pstart
           Ich sende Dir beiliegend einen kleinen Akt\pwindex{Bahr, Hermann 19.\,7.\,1863 Linz – 15.\,1.\,1934 München@\textsc{Bahr, Hermann} (19.\,7.\,1863 Linz – 15.\,1.\,1934 München), \emph{Schriftsteller, Kritiker}!tiefe Natur. Ein Akt@\strich\emph{Die tiefe Natur. Ein Akt}|pwv} mit der Frage, ob Du was dagegen hast, dass ich ihm,
               wenn er gedruckt wird, die folgende Widmung vorsetze:\pend
           
\pstart
           \centering{}»In Erinnerung an meinen lieben Anatol«.\pend
           
\pstart
           Mir ist nämlich folgendes passiert: Ich schrieb den Akt »nach einer wahren
               Begebenheit« (worüber gelegentlich einmal mündlich), eigentlich nur, weil ich Spass
               an der weiblichen Figur fand; nun teilt mir Burckhard\pwindex{Burckhard, Max Eugen 14.\,7.\,1854 Korneuburg – 16.\,3.\,1912 Wien@\textsc{Burckhard, Max Eugen} (14.\,7.\,1854 Korneuburg – 16.\,3.\,1912 Wien), \emph{Schriftsteller, Rechtswissenschaftler, Theaterleiter}|pw} mit, dass es eigentlich das Abschiedsouper\pwindex{Schnitzler, Arthur 15.\,5.\,1862 Wien – 21.\,10.\,1931 ebd.@\textsc{Schnitzler, Arthur} (15.\,5.\,1862 Wien – 21.\,10.\,1931 ebd.), \emph{Schriftsteller, Mediziner}!Abschiedssouper@\strich\emph{Abschiedssouper}|pw} ist, ich erschrecke, denke nach und – {\pb}kann es nicht läugnen. Du wirst mir glauben, dass
               es unbewusst war. Ich möchte aber doch jedenfalls öffentlich irgendwie den wahren
               Autor nennen: daher die Widmung.\pend
           
\pstart
           Ich lasse Deiner lieben Frau\pwindex{Schnitzler, Olga 17.\,1.\,1882 Wien – 13.\,1.\,1970 Lugano@\textsc{Schnitzler, Olga} (17.\,1.\,1882 Wien – 13.\,1.\,1970 Lugano), \emph{Schauspielerin, Sängerin}|pwv}
               herzlichst für die \label{K_L01633-1v}\edtext{Bilder}{\lemma{\textnormal{\emph{Bilder}}}\Cendnote{\textnormal{nicht ermittelt}}}\label{K_L01633-1} danken und werde
               mich sehr freuen, wenn sie mir erlaubt, ihr gelegentlich eines zu bringen\pwindex{\textcolor{red}{\textsuperscript{XXXX indx1}}!Hermann Bahr@\strich\emph{Hermann Bahr}|pwuv}.\pend
           
\pstart
           Mit vielen herzlichen Grüssen{\\[\baselineskip]}Dein{\\[\baselineskip]}\spacefill\mbox{{[}hs. Bahr:{]} Hermann}\pend
           \leftskip=0em{}\selectlanguage{ngerman}\endnumbering\briefempfaengerindex{Schnitzler, Arthur@\textsc{Schnitzler, Arthur}!zzzBahr, Hermann@\emph{von Hermann Bahr}!1906-10-161@{16. 10. 1906}|)be}\mylabel{L01633h}  \newcommand{\dateiname}{L01633}\newcommand{\titel}{Hermann Bahr an Arthur Schnitzler, 16. 10. 1906}\newcommand{\editorInnen}{Herausgegeben von Martin Anton Müller}%% latex-leseansicht-abspann.tex
%% Abspann für die Leseansicht.
%% Der Schalter \ifkorrekturansicht ist bereits durch den Vorspann gesetzt.

%% latex-abspann.tex
%% Gemeinsamer Abspann für Korrekturansicht und Leseansicht.
%% Setzt den Schalter \ifkorrekturansicht voraus (gesetzt in den
%% einbindenden Dateien latex-korrekturansicht-abspann.tex bzw.
%% latex-leseansicht-abspann.tex).
%% ---------------------------------------------------------------

\normalsize

% Das esempio-Environment wird nur in der Leseansicht benötigt
\ifkorrekturansicht\else
\newenvironment{esempio}[3]%
{
    \vspace{1.5ex}
    \rlap{\underline{#1}}
    \par
    \setlength{\parindent}{0cm}
    \nopagebreak
    \leftskip=#2cm
    \rightskip=#3cm
}
{
    \par
}
\fi

\doendnotes{C}
\bigskip
\vfill

\clearpage

\footnotesize

\ifkorrekturansicht
  \lohead{\textsc{register}}
\fi

% theindex-Environment neu definieren ohne reledmac
\makeatletter
\renewenvironment{theindex}{%
  \ifkorrekturansicht
    \section*{\indexname}%
  \else
    \subsubsection*{Index der erwähnten Entitäten}%
  \fi
  \setlength{\parindent}{0pt}%
  \setlength{\parskip}{0pt plus 0.3pt}%
  \let\item\@idxitem
}{%
  \ifkorrekturansicht\clearpage\fi
}
\makeatother

\IfFileExists{\jobname-pw.ind}{\input{\jobname-pw.ind}}{}

% Quellenangabe nur in der Leseansicht
\ifkorrekturansicht\else
% Fallback-Definitionen, falls die .tex-Datei \titel etc. nicht gesetzt hat
\providecommand{\titel}{}
\providecommand{\editorInnen}{}
\providecommand{\dateiname}{\jobname}

\vspace{3cm}

\vfill

\footnotesize
\textsc{Quelle}: \titel. Herausgegeben von {\editorInnen}. In: \emph{Arthur Schnitzler: Briefwechsel mit Autorinnen und Autoren}.
 Digitale Edition, https://schnitzler-briefe.acdh.oeaw.ac.at/{\dateiname}.html (Stand \today)
\fi

\end{document}


