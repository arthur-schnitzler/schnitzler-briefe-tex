%% latex-leseansicht-vorspann.tex
%% Vorspann für die Leseansicht.
%% Lädt die gemeinsame Datei latex-vorspann.tex mit nicht gesetztem Schalter.

\newif\ifkorrekturansicht
\korrekturansichtfalse

\input{../tex-inputs/latex-vorspann}


\section[Wilhelm von Wymetal u. a. an Arthur Schnitzler, 27. 11. [1907]]{L01735 Wilhelm von Wymetal u. a. an Arthur Schnitzler, 27. 11. [1907]}
\nopagebreak\mylabel{L01735v}
\rehead{ }\normalsize\beginnumbering\briefempfaengerindex{Schnitzler, Arthur@\textsc{Schnitzler, Arthur}!zzzGoldschmied, E.@\emph{von E. Goldschmied}!1907-11-271@{27. 11. [1907]}|(be}\briefempfaengerindex{Schnitzler, Arthur@\textsc{Schnitzler, Arthur}!zzzMarkowitz, Alfred@\emph{von Alfred Markowitz}!1907-11-271@{27. 11. [1907]}|(be}\briefempfaengerindex{Schnitzler, Arthur@\textsc{Schnitzler, Arthur}!zzzKurzweil, Hedwig@\emph{von Hedwig Kurzweil}!1907-11-271@{27. 11. [1907]}|(be}\briefempfaengerindex{Schnitzler, Arthur@\textsc{Schnitzler, Arthur}!zzzStefan, Paul@\emph{von Paul Stefan}!1907-11-271@{27. 11. [1907]}|(be}\briefempfaengerindex{Schnitzler, Arthur@\textsc{Schnitzler, Arthur}!zzzLiliencron, Detlev von@\emph{von Detlev von Liliencron}!1907-11-271@{27. 11. [1907]}|(be}\briefempfaengerindex{Schnitzler, Arthur@\textsc{Schnitzler, Arthur}!zzzWymetal [Musikschriftsteller], Wilhelm@\emph{von Wilhelm Wymetal [Musikschriftsteller]}!1907-11-271@{27. 11. [1907]}|(be}
\toendnotes[C]{\smallbreak\pagebreak[2]}
\correspDesc{Versand  durch Wilhelm von Wymetal, Detlev von Liliencron, Paul Stefan, Hedwig Kurzweil, Alfred Markowitz, E. Goldschmied am 27. 11. [1907] in Wien
\newline{}Erhalt  durch Arthur Schnitzler im Zeitraum [27. 11. 1907 – 1. 12. 1907?] in Wien}\toendnotes[C]{\smallbreak}
\Standort{CUL, Schnitzler, B 59.}
\physDesc{Bildpostkarte, 180 Zeichen
\newline{}Handschrift Wilhelm Wymetal [Musikschriftsteller]: blauer Buntstift, lateinische Kurrent
\newline{}Handschrift Detlev von Liliencron: blauer Buntstift, deutsche Kurrent
\newline{}Handschrift Paul Stefan: blauer Buntstift, deutsche Kurrent
\newline{}Handschrift Hedwig Kurzweil: blauer Buntstift
\newline{}Handschrift Alfred Markowitz: blauer Buntstift
\newline{}Handschrift E. Goldschmied: blauer Buntstift
\newline{}Versand: Stempel: »\nobreak{}\oindex{Wien@\textbf{Wien}, \emph{Verwaltungsgebiet}|pwk}Wien 68, 27. {[}11. 1907{]}\nobreak{}«.  }\toendnotes[C]{\smallbreak}\pstart{}{\pb}Herrn\pend{}\pstart{}Dr. Arthur Schnitzler\pend{}\pstart{}Wien XVIII\oindex{XVIII., Währing@\textbf{XVIII., Währing}, \emph{Verwaltungsgebiet}|pw}/\pend{}\pstart{}Spöttelstr. 7\oindex{Wien@\textbf{Wien}!XVIII., Währing@\textbf{XVIII., Währing}!Edmund-Weiß-Gasse 7@\textbf{Edmund-Weiß-Gasse 7}, \emph{Wohngebäude}|pw}\pend{}{\bigskip}
\pstart
           \noindent{}\centering{}{\pb}\textcolor{gray}{\textbf{Wien, XIII., Schönbrunn.
                     Obelisk-Allee\oindex{Wien@\textbf{Wien}!XIII., Hietzing@\textbf{XIII., Hietzing}!Schloss Schönbrunn@\textbf{Schloss Schönbrunn}, \emph{Schloss}|pw}.}}\pend
           \vspace{1em}
\pstart
           \noindent{}{\pb}{[}hs. Liliencron:{]} Herzlichen Gruß\pend
           \pstart Ihr \spacefill\mbox{Liliencron}\pend{}\selectlanguage{ngerman}\vspace{1em}\pstart {[}hs. Stefan:{]} Ergebenſt Ihr \spacefill\mbox{Paul Stefan.}\pend{}\selectlanguage{ngerman}\vspace{1em}
\pstart
           \noindent{}{[}hs. Wymetal [Musikschriftsteller]:{]} Herzlich ergeben\pend
           \pstart \spacefill\mbox{Will Wymetal}\pend{}\selectlanguage{ngerman}\vspace{1em}\pstart \spacefill\mbox{{[}hs. Kurzweil:{]} Hedwig Kurzweil}\pend{}\selectlanguage{ngerman}\vspace{1em}\pstart \spacefill\mbox{{[}hs. Markowitz:{]} D\textsuperscript{r}Markowitz.}\pend{}\selectlanguage{ngerman}\vspace{1em}\pstart \spacefill\mbox{{[}hs. Goldschmied:{]} EGoldschmied}\pend{}
\pstart
           {[}hs. Wymetal [Musikschriftsteller]:{]} Wien\oindex{Wien@\textbf{Wien}, \emph{Verwaltungsgebiet}|pw}, \label{K_L01735-1v}\edtext{27/11}{\lemma{\textnormal{\emph{27/11}}}\Cendnote{\textnormal{Am 27. 11. 1907
                     veranstaltete der \emph{Ansorge-Verein}\orgindex{Ansorge-Verein. Verein zur Förderung moderner Kunst@Ansorge-Verein. Verein zur Förderung moderner Kunst|pwk} im Gewerbevereinssaal\oindex{Wien@\textbf{Wien}!I., Innere Stadt@\textbf{I., Innere Stadt}!Saal des Niederösterreichischen Gewerbevereins@\textbf{Saal des Niederösterreichischen Gewerbevereins}, \emph{Bürogebäude}|pwk} einen Leseabend mit Detlev von Liliencron\pwindex{Schnitzler, Louise 8.\,7.\,1840 Kőszeg – 9.\,9.\,1911 Wien@\textsc{Schnitzler, Louise} (8.\,7.\,1840 Kőszeg – 9.\,9.\,1911 Wien)|pwk}.}}}\label{K_L01735-1}\pend
           \selectlanguage{ngerman}\endnumbering\briefempfaengerindex{Schnitzler, Arthur@\textsc{Schnitzler, Arthur}!zzzGoldschmied, E.@\emph{von E. Goldschmied}!1907-11-271@{27. 11. [1907]}|)be}\briefempfaengerindex{Schnitzler, Arthur@\textsc{Schnitzler, Arthur}!zzzMarkowitz, Alfred@\emph{von Alfred Markowitz}!1907-11-271@{27. 11. [1907]}|)be}\briefempfaengerindex{Schnitzler, Arthur@\textsc{Schnitzler, Arthur}!zzzKurzweil, Hedwig@\emph{von Hedwig Kurzweil}!1907-11-271@{27. 11. [1907]}|)be}\briefempfaengerindex{Schnitzler, Arthur@\textsc{Schnitzler, Arthur}!zzzStefan, Paul@\emph{von Paul Stefan}!1907-11-271@{27. 11. [1907]}|)be}\briefempfaengerindex{Schnitzler, Arthur@\textsc{Schnitzler, Arthur}!zzzLiliencron, Detlev von@\emph{von Detlev von Liliencron}!1907-11-271@{27. 11. [1907]}|)be}\briefempfaengerindex{Schnitzler, Arthur@\textsc{Schnitzler, Arthur}!zzzWymetal [Musikschriftsteller], Wilhelm@\emph{von Wilhelm Wymetal [Musikschriftsteller]}!1907-11-271@{27. 11. [1907]}|)be}\mylabel{L01735h}  \newcommand{\dateiname}{L01735}\newcommand{\titel}{Wilhelm von Wymetal u. a. an Arthur Schnitzler, 27. 11. [1907]}\newcommand{\editorInnen}{Martin Anton Müller und Gerd-Hermann Susen}%% latex-leseansicht-abspann.tex
%% Abspann für die Leseansicht.
%% Der Schalter \ifkorrekturansicht ist bereits durch den Vorspann gesetzt.

%% latex-abspann.tex
%% Gemeinsamer Abspann für Korrekturansicht und Leseansicht.
%% Setzt den Schalter \ifkorrekturansicht voraus (gesetzt in den
%% einbindenden Dateien latex-korrekturansicht-abspann.tex bzw.
%% latex-leseansicht-abspann.tex).
%% ---------------------------------------------------------------

\normalsize

% Das esempio-Environment wird nur in der Leseansicht benötigt
\ifkorrekturansicht\else
\newenvironment{esempio}[3]%
{
    \vspace{1.5ex}
    \rlap{\underline{#1}}
    \par
    \setlength{\parindent}{0cm}
    \nopagebreak
    \leftskip=#2cm
    \rightskip=#3cm
}
{
    \par
}
\fi

\doendnotes{C}
\bigskip
\vfill

\clearpage

\footnotesize

\ifkorrekturansicht
  \lohead{\textsc{register}}
\fi

% theindex-Environment neu definieren ohne reledmac
\makeatletter
\renewenvironment{theindex}{%
  \ifkorrekturansicht
    \section*{\indexname}%
  \else
    \subsubsection*{Index der erwähnten Entitäten}%
  \fi
  \setlength{\parindent}{0pt}%
  \setlength{\parskip}{0pt plus 0.3pt}%
  \let\item\@idxitem
}{%
  \ifkorrekturansicht\clearpage\fi
}
\makeatother

\IfFileExists{\jobname-pw.ind}{\input{\jobname-pw.ind}}{}

% Quellenangabe nur in der Leseansicht
\ifkorrekturansicht\else
% Fallback-Definitionen, falls die .tex-Datei \titel etc. nicht gesetzt hat
\providecommand{\titel}{}
\providecommand{\editorInnen}{}
\providecommand{\dateiname}{\jobname}

\vspace{3cm}

\vfill

\footnotesize
\textsc{Quelle}: \titel. Herausgegeben von {\editorInnen}. In: \emph{Arthur Schnitzler: Briefwechsel mit Autorinnen und Autoren}.
 Digitale Edition, https://schnitzler-briefe.acdh.oeaw.ac.at/{\dateiname}.html (Stand \today)
\fi

\end{document}


