%% latex-leseansicht-vorspann.tex
%% Vorspann für die Leseansicht.
%% Lädt die gemeinsame Datei latex-vorspann.tex mit nicht gesetztem Schalter.

\newif\ifkorrekturansicht
\korrekturansichtfalse

\input{../tex-inputs/latex-vorspann}


         
         \renewcommand{\erwaehntePersonen}{Personen: Louise Schnitzler}
         \renewcommand{\erwaehnteInstitutionen}{Institutionen: Ansorge-Verein}
         \renewcommand{\erwaehnteOrte}{Orte: Edmund-Weiß-Gasse, Gewerbevereinssaal, Schloß Schönbrunn, Wien, XVIII., Währing}
         \renewcommand{\erwaehnteWerke}{
               \section[Wilhelm von Wymetal u. a. an Arthur Schnitzler, 27. 11. {[}1907{]}]{ Wilhelm von Wymetal u. a. an Arthur Schnitzler,
                    27. 11. {[}1907{]}}\nopagebreak\mylabel{v}\rehead{ }\begin{ledgroupsized}[t]{13cm}\normalsize\beginnumbering \toendnotes[C]{\smallbreak\pagebreak[2]} \Standort{CUL, Schnitzler, B 59.}
\physDesc{Bildpostkarte
\newline{}Handschrift Wilhelm von Wymetal: blauer Buntstift, lateinische Kurrent\newline{}Handschrift Detlev von Liliencron: blauer Buntstift, deutsche Kurrent\newline{}Handschrift Paul Stefan: blauer Buntstift, deutsche Kurrent\newline{}Handschrift Hedwig Kurzweil: blauer Buntstift\newline{}Handschrift Alfred Markowitz: blauer Buntstift\newline{}Handschrift E. Goldschmied: blauer Buntstift\newline{}Versand: Stempel: »\nobreak{}Wien 68, 27. {[}11. 1907{]}\nobreak{}«.  }\toendnotes[C]{\smallbreak}\pstart{}{\pb}Herrn\pend{}\pstart{}Dr. Arthur Schnitzler\pend{}\pstart{}Wien XVIII\oindex{XVIII., Waehring@\textbf{XVIII., Währing}|pw}/\pend{}\pstart{}Spöttelstr. 7\oindex{Edmund-Weiss-Gasse@\textbf{Edmund-Weiß-Gasse}|pw}\pend{}{\bigskip}\pstart
           \noindent{}\centering{}\textcolor{gray}{\textbf{{\pb}Wien, XIII., Schönbrunn.
                                Obelisk-Allee\oindex{Schloss Schoenbrunn@\textbf{Schloß Schönbrunn}|pw}.}}\pend
           \pstart
           {\pb}{[}hs. Liliencron:{]} Herzlichen Gruß\pend
           \pstart Ihr \spacefill\mbox{Liliencron}\pend{}\pstart {[}hs. Stefan:{]} Ergebenſt Ihr \spacefill\mbox{Paul Stefan.}\pend{}\pstart
           \noindent{}{[}hs. Wymetal:{]} Herzlich ergeben\pend
           \pstart \spacefill\mbox{Will Wymetal}\pend{}\pstart \spacefill\mbox{{[}hs. Kurzweil:{]} Hedwig Kurzweil}\pend{}\pstart \spacefill\mbox{{[}hs. Markowitz:{]} D\textsuperscript{r}Markowitz.}\pend{}\pstart \spacefill\mbox{{[}hs. Goldschmied:{]} EGoldschmied}\pend{}\pstart
           {[}hs. Wymetal:{]} Wien\oindex{Wien@\textbf{Wien}|pw}, \label{K_L01735_1v}\edtext{27/11}{\lemma{\textnormal{\emph{27/11}}}\Cendnote{\textnormal{Am 27. 11. 1907
                            veranstaltete der \emph{Ansorge-Verein}\orgindex{Ansorge-Verein@Ansorge-Verein|pwk} im Gewerbevereinssaal\oindex{Gewerbevereinssaal@\textbf{Gewerbevereinssaal}|pwk} einen Leseabend mit
                                Detlev von Liliencron\pwindex{Schnitzler, Louise 1840-07-08 – 1911-09-09@\textsc{Schnitzler, Louise} (1840-07-08 – 1911-09-09)|pwk}.}}}\label{K_L01735_1h}\pend
           
         
         \endnumbering\mylabel{h}\end{ledgroupsized}  \newcommand{\dateiname}{L01735}\newcommand{\titel}{Wilhelm von Wymetal u. a. an Arthur Schnitzler, 27. 11. [1907]}\newcommand{\editorInnen}{Martin Anton Müller und Gerd-Hermann Susen}%% latex-leseansicht-abspann.tex
%% Abspann für die Leseansicht.
%% Der Schalter \ifkorrekturansicht ist bereits durch den Vorspann gesetzt.

%% latex-abspann.tex
%% Gemeinsamer Abspann für Korrekturansicht und Leseansicht.
%% Setzt den Schalter \ifkorrekturansicht voraus (gesetzt in den
%% einbindenden Dateien latex-korrekturansicht-abspann.tex bzw.
%% latex-leseansicht-abspann.tex).
%% ---------------------------------------------------------------

\normalsize

% Das esempio-Environment wird nur in der Leseansicht benötigt
\ifkorrekturansicht\else
\newenvironment{esempio}[3]%
{
    \vspace{1.5ex}
    \rlap{\underline{#1}}
    \par
    \setlength{\parindent}{0cm}
    \nopagebreak
    \leftskip=#2cm
    \rightskip=#3cm
}
{
    \par
}
\fi

\doendnotes{C}
\bigskip
\vfill

\clearpage

\footnotesize

\ifkorrekturansicht
  \lohead{\textsc{register}}
\fi

% theindex-Environment neu definieren ohne reledmac
\makeatletter
\renewenvironment{theindex}{%
  \ifkorrekturansicht
    \section*{\indexname}%
  \else
    \subsubsection*{Index der erwähnten Entitäten}%
  \fi
  \setlength{\parindent}{0pt}%
  \setlength{\parskip}{0pt plus 0.3pt}%
  \let\item\@idxitem
}{%
  \ifkorrekturansicht\clearpage\fi
}
\makeatother

\IfFileExists{\jobname-pw.ind}{\input{\jobname-pw.ind}}{}

% Quellenangabe nur in der Leseansicht
\ifkorrekturansicht\else
% Fallback-Definitionen, falls die .tex-Datei \titel etc. nicht gesetzt hat
\providecommand{\titel}{}
\providecommand{\editorInnen}{}
\providecommand{\dateiname}{\jobname}

\vspace{3cm}

\vfill

\footnotesize
\textsc{Quelle}: \titel. Herausgegeben von {\editorInnen}. In: \emph{Arthur Schnitzler: Briefwechsel mit Autorinnen und Autoren}.
 Digitale Edition, https://schnitzler-briefe.acdh.oeaw.ac.at/{\dateiname}.html (Stand \today)
\fi

\end{document}


      