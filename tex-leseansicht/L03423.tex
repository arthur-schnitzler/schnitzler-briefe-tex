%% latex-leseansicht-vorspann.tex
%% Vorspann für die Leseansicht.
%% Lädt die gemeinsame Datei latex-vorspann.tex mit nicht gesetztem Schalter.

\newif\ifkorrekturansicht
\korrekturansichtfalse

\input{../tex-inputs/latex-vorspann}


\section[Felix Salten, Paul Lindau und Marie Barthel an Arthur Schnitzler, 9. 5. 1906]{L03423 Felix Salten, Paul Lindau und Marie Barthel an Arthur
               Schnitzler,  9. 5. 1906}
\nopagebreak\mylabel{L03423v}
\rehead{ }\normalsize\beginnumbering\briefempfaengerindex{Schnitzler, Arthur@\textsc{Schnitzler, Arthur}!zzzBarthel, Marie@\emph{von Marie Barthel}!1906-05-091@{9. 5. 1906}|(be}\briefempfaengerindex{Schnitzler, Arthur@\textsc{Schnitzler, Arthur}!zzzLindau, Paul@\emph{von Paul Lindau}!1906-05-091@{9. 5. 1906}|(be}\briefempfaengerindex{Schnitzler, Arthur@\textsc{Schnitzler, Arthur}!zzzSalten, Felix@\emph{von Felix Salten}!1906-05-091@{9. 5. 1906}|(be}
\toendnotes[C]{\smallbreak\pagebreak[2]}
\correspDesc{Versand  durch Felix Salten, Paul Lindau, Marie Barthel am 9. 5. 1906 in Wartburg
\newline{}Erhalt  durch Arthur Schnitzler im Zeitraum [10. 5. 1906
                  – 14. 5. 1906?] in Wien}\toendnotes[C]{\smallbreak}
\Standort{CUL, Schnitzler, B 89, B 1.}
\physDesc{Bildpostkarte, 143 Zeichen
\newline{}Handschrift Felix Salten: schwarze Tinte, lateinische Kurrent
\newline{}Handschrift Paul Lindau: schwarze Tinte, deutsche Kurrent
\newline{}Handschrift Marie Barthel: schwarze Tinte
\newline{}Versand: Stempel: »\nobreak{}\oindex{Wartburg@\textbf{Wartburg}, \emph{Burg}|pwk}Wartburg, 9. 5. 06, 6–7 \textcolor{gray}{N}\nobreak{}«.  
\newline{}Schnitzler: mit Bleistift datiert: »9. 5. 90\textcolor{gray}{6}« 
\newline{}Ordnung: mit Bleistift von unbekannter Hand nummeriert: »213« }\pstart{}{\pb}Herrn D\textsuperscript{r} Arthur Schnitzler\pend{}\pstart{}Wien XVIII.\oindex{XVIII., Währing@\textbf{XVIII., Währing}, \emph{Verwaltungsgebiet}|pw}\pend{}\pstart{}Spöttelgaſse 7\oindex{Wien@\textbf{Wien}!XVIII., Währing@\textbf{XVIII., Währing}!Edmund-Weiß-Gasse 7@\textbf{Edmund-Weiß-Gasse 7}, \emph{Wohngebäude}|pw}\pend{}{\bigskip}
\pstart
           \noindent{}\centering{}{\pb}\textcolor{gray}{\textbf{Wartburg\oindex{Wartburg@\textbf{Wartburg}, \emph{Burg}|pw} von Süd-West.}}\pend
           \vspace{1em}
\pstart
           \noindent{}{\pb}Viele herzliche Grüße an Sie u.
               Frau Olga\pwindex{Schnitzler, Olga 17.\,1.\,1882 Wien – 13.\,1.\,1970 Lugano@\textsc{Schnitzler, Olga} (17.\,1.\,1882 Wien – 13.\,1.\,1970 Lugano), \emph{Schauspielerin, Sängerin}|pw}.\pend
           \pstart Ihr \spacefill\mbox{Salten}\pend{}\selectlanguage{ngerman}\vspace{1em}
\pstart
           \noindent{}{[}hs. Lindau:{]} Herzlich grüßend\pend
           
\pstart
           Ihr \spacefill\mbox{Paul Lindau}{\\[\baselineskip]}{[}hs. Barthel:{]} \spacefill\mbox{Marie Barthel}\pend
           \leftskip=0em{}\selectlanguage{ngerman}\endnumbering\briefempfaengerindex{Schnitzler, Arthur@\textsc{Schnitzler, Arthur}!zzzBarthel, Marie@\emph{von Marie Barthel}!1906-05-091@{9. 5. 1906}|)be}\briefempfaengerindex{Schnitzler, Arthur@\textsc{Schnitzler, Arthur}!zzzLindau, Paul@\emph{von Paul Lindau}!1906-05-091@{9. 5. 1906}|)be}\briefempfaengerindex{Schnitzler, Arthur@\textsc{Schnitzler, Arthur}!zzzSalten, Felix@\emph{von Felix Salten}!1906-05-091@{9. 5. 1906}|)be}\mylabel{L03423h}  \newcommand{\dateiname}{L03423}\newcommand{\titel}{Felix Salten, Paul Lindau und Marie Barthel an Arthur Schnitzler, 9. 5. 1906}\newcommand{\editorInnen}{Martin Anton Müller und Laura Untner}%% latex-leseansicht-abspann.tex
%% Abspann für die Leseansicht.
%% Der Schalter \ifkorrekturansicht ist bereits durch den Vorspann gesetzt.

%% latex-abspann.tex
%% Gemeinsamer Abspann für Korrekturansicht und Leseansicht.
%% Setzt den Schalter \ifkorrekturansicht voraus (gesetzt in den
%% einbindenden Dateien latex-korrekturansicht-abspann.tex bzw.
%% latex-leseansicht-abspann.tex).
%% ---------------------------------------------------------------

\normalsize

% Das esempio-Environment wird nur in der Leseansicht benötigt
\ifkorrekturansicht\else
\newenvironment{esempio}[3]%
{
    \vspace{1.5ex}
    \rlap{\underline{#1}}
    \par
    \setlength{\parindent}{0cm}
    \nopagebreak
    \leftskip=#2cm
    \rightskip=#3cm
}
{
    \par
}
\fi

\doendnotes{C}
\bigskip
\vfill

\clearpage

\footnotesize

\ifkorrekturansicht
  \lohead{\textsc{register}}
\fi

% theindex-Environment neu definieren ohne reledmac
\makeatletter
\renewenvironment{theindex}{%
  \ifkorrekturansicht
    \section*{\indexname}%
  \else
    \subsubsection*{Index der erwähnten Entitäten}%
  \fi
  \setlength{\parindent}{0pt}%
  \setlength{\parskip}{0pt plus 0.3pt}%
  \let\item\@idxitem
}{%
  \ifkorrekturansicht\clearpage\fi
}
\makeatother

\IfFileExists{\jobname-pw.ind}{\input{\jobname-pw.ind}}{}

% Quellenangabe nur in der Leseansicht
\ifkorrekturansicht\else
% Fallback-Definitionen, falls die .tex-Datei \titel etc. nicht gesetzt hat
\providecommand{\titel}{}
\providecommand{\editorInnen}{}
\providecommand{\dateiname}{\jobname}

\vspace{3cm}

\vfill

\footnotesize
\textsc{Quelle}: \titel. Herausgegeben von {\editorInnen}. In: \emph{Arthur Schnitzler: Briefwechsel mit Autorinnen und Autoren}.
 Digitale Edition, https://schnitzler-briefe.acdh.oeaw.ac.at/{\dateiname}.html (Stand \today)
\fi

\end{document}


