%% latex-korrekturansicht-vorspann.tex
%% Vorspann für die Korrekturansicht.
%% Lädt die gemeinsame Datei latex-vorspann.tex mit gesetztem Schalter.

\newif\ifkorrekturansicht
\korrekturansichttrue

\input{../tex-inputs/latex-vorspann}


\section[Christiane Hofmannsthal an Arthur Schnitzler, 28. 1. 192{[}2{]}]{L02374 Christiane Hofmannsthal an Arthur Schnitzler, 28. 1. 192{[}2{]}}
\nopagebreak\mylabel{L02374v}
\rehead{ }\normalsize\beginnumbering\briefempfaengerindex{Schnitzler, Arthur@\textsc{Schnitzler, Arthur}!zzzZimmer, Christiane@\emph{von Christiane Zimmer}!1922-01-281@{28. 1. 192{[}2{]}}|(be}
\toendnotes[C]{\smallbreak\pagebreak[2]}\Standort{CUL, Schnitzler, B 43.}
\physDesc{Postkarte, 288 Zeichen
\newline{}Handschrift: schwarze Tinte, lateinische Kurrent
\newline{}Versand: Stempel: »\nobreak{}\oindex{Rodaun@\textbf{Rodaun}, \emph{A.ADM4}|pwk}\textcolor{gray}{R}odau\textcolor{gray}{n}\nobreak{}«.  
\newline{}Ordnung: 1) mit Bleistift von Frieda
                                    Pollak\pwindex{Pollak, Frieda 08.12.1881 – 13.07.1937@\textsc{Pollak, Frieda} (08.12.1881 – 13.07.1937), \emph{Sekretär/Sekretärin}|pw} (?) mit dem Buchstaben »A«
                                 (Abgeschrieben/Abschrift) gekennzeichnet  2) mit Bleistift von unbekannter Hand nummeriert: »\strikeout{375}« 3) mit Bleistift von unbekannter Hand nummeriert:
                                    »363«}
\buchAbdrucke{\weitereDrucke{Hugo von Hofmannsthal, Arthur Schnitzler: \emph{Briefwechsel}. Frankfurt am Main: \emph{S. Fischer} 1964, S. 392.} }\toendnotes[C]{\smallbreak}\pstart{}{\pb}Herrn Arthur Schnitzler\pend{}\pstart{}Wien XVIII.\oindex{XVIII., Waehring@\textbf{XVIII., Währing}, \emph{A.ADM3}|pw}\pend{}\pstart{}Sternwartestr. 71\oindex{Sternwartestrasse 71@\textbf{Sternwartestraße 71}, \emph{Wohngebäude (K.WHS)}|pw}.\pend{}{\bigskip}\vspace{1em}
\pstart
           \raggedleft{}{\pb}\label{K_L02374-1v}\edtext{28. I. 21}{\lemma{\textnormal{\emph{28. I. 21}}}\Cendnote{\textnormal{Bei der Jahresangabe handelt es sich
                     um einen Schreibirrtum, wie sich aus der angekündigten Lesung ergibt.}}}\label{K_L02374-1}\pend
           
\pstart{}Lieber Arthur, \pend\vspace{0.5em}
\pstart
           Im Namen vom Papa\pwindex{Hofmannsthal, Hugo von 1874-02-01 – 1929-07-15@\textsc{Hofmannsthal, Hugo von} (1874-02-01 – 1929-07-15), \emph{Schriftsteller/Schriftstellerin}|pwv} bitte ich
               Dich, sicher am \label{K_L02374-2v}\edtext{Freitag}{\lemma{\textnormal{\emph{Freitag}}}\Cendnote{\textnormal{Vgl. A. S.: \emph{Tagebuch}, 3. 2. 1922.
               }}}\label{K_L02374-2}{ }¾ 7\textsuperscript{h} abends bei der Berta Zuckerkandl\pwindex{Zuckerkandl, Berta 13.04.1864 – 16.10.1945@\textsc{Zuckerkandl, Berta} (13.04.1864 – 16.10.1945), \emph{Journalist/Journalistin, Übersetzer/Übersetzerin}|pw} zu
               sein, wo Papa\pwindex{Hofmannsthal, Hugo von 1874-02-01 – 1929-07-15@\textsc{Hofmannsthal, Hugo von} (1874-02-01 – 1929-07-15), \emph{Schriftsteller/Schriftstellerin}|pwv} das Welttheater\pwindex{Salzburger grosse Welttheater@\emph{Das Salzburger große Welttheater}|pw} vorliest. Er freut sich besonders auf
               Dein Zuhören.\pend
           
\pstart
           Herzliche Grüße von Deiner{\\[\baselineskip]}\spacefill\mbox{Christiane Hofmannsthal}\pend
           \leftskip=0em{}\selectlanguage{ngerman}\endnumbering\briefempfaengerindex{Schnitzler, Arthur@\textsc{Schnitzler, Arthur}!zzzZimmer, Christiane@\emph{von Christiane Zimmer}!1922-01-281@{28. 1. 192{[}2{]}}|)be}\mylabel{L02374h}  \normalsize

\doendnotes{C}
\bigskip
\vfill

\clearpage

\footnotesize

\lohead{\textsc{register}}

% Definiere theindex-Environment komplett neu ohne reledmac
\makeatletter
\renewenvironment{theindex}{%
  \section*{\indexname}%
  \setlength{\parindent}{0pt}%
  \setlength{\parskip}{0pt plus 0.3pt}%
  \let\item\@idxitem
}{%
  \clearpage
}
\makeatother

\IfFileExists{\jobname-pw.ind}{\input{\jobname-pw.ind}}{}

\end{document}

      