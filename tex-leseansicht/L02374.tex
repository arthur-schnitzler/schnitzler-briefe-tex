%% latex-leseansicht-vorspann.tex
%% Vorspann für die Leseansicht.
%% Lädt die gemeinsame Datei latex-vorspann.tex mit nicht gesetztem Schalter.

\newif\ifkorrekturansicht
\korrekturansichtfalse

\input{../tex-inputs/latex-vorspann}


         
         \renewcommand{\erwaehntePersonen}{Personen: Christiane von Hofmannsthal, Hugo von Hofmannsthal, Frieda Pollak, Berta Zuckerkandl}
         \renewcommand{\erwaehnteOrte}{Orte: Rodaun, Sternwartestraße, Wien, XVIII., Währing}
         \renewcommand{\erwaehnteWerke}{Werke: Das Salzburger große Welttheater}
               \section[Christiane von Hofmannsthal an Arthur Schnitzler, 28. 1. 192{[}2{]}]{ Christiane von Hofmannsthal an Arthur Schnitzler, 28. 1. 192{[}2{]}}\nopagebreak\mylabel{v}\rehead{ }\begin{ledgroupsized}[t]{13cm}\normalsize\beginnumbering \toendnotes[C]{\smallbreak\pagebreak[2]} \Standort{CUL, Schnitzler, B 43.}
\physDesc{Postkarte, 288 Zeichen
\newline{}Handschrift: schwarze Tinte, lateinische Kurrent
\newline{}Versand: Stempel: »\nobreak{}\oindex{Rodaun@\textbf{Rodaun}|pwk}\textcolor{gray}{R}odau\textcolor{gray}{n}\nobreak{}«.  
\newline{}Ordnung: 1) mit Bleistift von Frieda
                                    Pollak\pwindex{Pollak, Frieda 08.12.1881 – 13.07.1937@\textsc{Pollak, Frieda} (08.12.1881 – 13.07.1937), \emph{Sekretärin}|pw} (?) mit dem Buchstaben »A«
                                 (Abgeschrieben/Abschrift) gekennzeichnet  2) mit Bleistift von unbekannter Hand nummeriert: »\strikeout{375}« 3) mit Bleistift von unbekannter Hand nummeriert:
                                    »363«}\buchAbdrucke{\weitereDrucke{Hugo von Hofmannsthal, Arthur Schnitzler: \emph{Briefwechsel}. Hg. Therese Nickl und Heinrich Schnitzler. Frankfurt am Main: \emph{S. Fischer} 1964, S. 392.} }\toendnotes[C]{\smallbreak}\pstart{}{\pb}Herrn Arthur Schnitzler\pend{}\pstart{}Wien XVIII.\oindex{XVIII., Waehring@\textbf{XVIII., Währing}|pw}\pend{}\pstart{}Sternwartestr. 71\oindex{XXXX Ortsangabe fehlt|pw}.\pend{}{\bigskip}\pstart
           \raggedleft{}{\pb}\label{K_L02374-1v}\edtext{28. I. 21}{\lemma{\textnormal{\emph{28. I. 21}}}\Cendnote{\textnormal{Bei der Jahresangabe handelt es sich
                     um einen Schreibirrtum, wie sich aus der angekündigten Lesung ergibt.}}}\label{K_L02374-1h}\pend
           \pstart{}Lieber Arthur, \pend\pstart
           Im Namen vom Papa\pwindex{Hofmannsthal, Hugo von 1874-02-01 – 1929-07-15@\textsc{Hofmannsthal, Hugo von} (1874-02-01 – 1929-07-15), \emph{Schriftsteller}|pwv} bitte ich
               Dich, sicher am Freitag{ }¾ 7\textsuperscript{h} abends bei der Berta Zuckerkandl\pwindex{Zuckerkandl, Berta 13.04.1864 – 16.10.1945@\textsc{Zuckerkandl, Berta} (13.04.1864 – 16.10.1945), \emph{Journalistin, Übersetzerin}|pw} zu
               sein, wo Papa\pwindex{Hofmannsthal, Hugo von 1874-02-01 – 1929-07-15@\textsc{Hofmannsthal, Hugo von} (1874-02-01 – 1929-07-15), \emph{Schriftsteller}|pwv} das Welttheater\pwindex{Hofmannsthal, Hugo von 1874-02-01 – 1929-07-15@\textsc{Hofmannsthal, Hugo von} (1874-02-01 – 1929-07-15), \emph{Schriftsteller}!Salzburger grosse Welttheater1922@\strich\emph{Das Salzburger große Welttheater} {[}1922{]}|pw} vorliest. Er freut sich besonders auf
               Dein Zuhören.\pend
           \pstart
           Herzliche Grüße von Deiner{\\[\baselineskip]}\spacefill\mbox{Christiane Hofmannsthal}\pend
           \leftskip=0em{}
         
         \endnumbering\mylabel{h}\end{ledgroupsized}  \newcommand{\dateiname}{L02374}\newcommand{\titel}{Christiane von Hofmannsthal an Arthur Schnitzler, 28. 1. 192[2]}\newcommand{\editorInnen}{Martin Anton Müller und Gerd-Hermann Susen}%% latex-leseansicht-abspann.tex
%% Abspann für die Leseansicht.
%% Der Schalter \ifkorrekturansicht ist bereits durch den Vorspann gesetzt.

%% latex-abspann.tex
%% Gemeinsamer Abspann für Korrekturansicht und Leseansicht.
%% Setzt den Schalter \ifkorrekturansicht voraus (gesetzt in den
%% einbindenden Dateien latex-korrekturansicht-abspann.tex bzw.
%% latex-leseansicht-abspann.tex).
%% ---------------------------------------------------------------

\normalsize

% Das esempio-Environment wird nur in der Leseansicht benötigt
\ifkorrekturansicht\else
\newenvironment{esempio}[3]%
{
    \vspace{1.5ex}
    \rlap{\underline{#1}}
    \par
    \setlength{\parindent}{0cm}
    \nopagebreak
    \leftskip=#2cm
    \rightskip=#3cm
}
{
    \par
}
\fi

\doendnotes{C}
\bigskip
\vfill

\clearpage

\footnotesize

\ifkorrekturansicht
  \lohead{\textsc{register}}
\fi

% theindex-Environment neu definieren ohne reledmac
\makeatletter
\renewenvironment{theindex}{%
  \ifkorrekturansicht
    \section*{\indexname}%
  \else
    \subsubsection*{Index der erwähnten Entitäten}%
  \fi
  \setlength{\parindent}{0pt}%
  \setlength{\parskip}{0pt plus 0.3pt}%
  \let\item\@idxitem
}{%
  \ifkorrekturansicht\clearpage\fi
}
\makeatother

\IfFileExists{\jobname-pw.ind}{\input{\jobname-pw.ind}}{}

% Quellenangabe nur in der Leseansicht
\ifkorrekturansicht\else
% Fallback-Definitionen, falls die .tex-Datei \titel etc. nicht gesetzt hat
\providecommand{\titel}{}
\providecommand{\editorInnen}{}
\providecommand{\dateiname}{\jobname}

\vspace{3cm}

\vfill

\footnotesize
\textsc{Quelle}: \titel. Herausgegeben von {\editorInnen}. In: \emph{Arthur Schnitzler: Briefwechsel mit Autorinnen und Autoren}.
 Digitale Edition, https://schnitzler-briefe.acdh.oeaw.ac.at/{\dateiname}.html (Stand \today)
\fi

\end{document}


      