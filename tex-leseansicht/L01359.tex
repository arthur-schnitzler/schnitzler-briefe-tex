%% latex-korrekturansicht-vorspann.tex
%% Vorspann für die Korrekturansicht.
%% Lädt die gemeinsame Datei latex-vorspann.tex mit gesetztem Schalter.

\newif\ifkorrekturansicht
\korrekturansichttrue

\input{../tex-inputs/latex-vorspann}


\section[Arthur Schnitzler an Hermann Bahr, 8. 1. 1904]{L01359 Arthur Schnitzler an Hermann Bahr, 8. 1. 1904}
\nopagebreak\mylabel{L01359v}
\rehead{ }\normalsize\beginnumbering\briefempfaengerindex{Bahr, Hermann@\textsc{Bahr, Hermann}!zzzSchnitzler, Arthur@\emph{von Arthur Schnitzler}!1904-01-081@{8. 1. 1904}|(be}
\toendnotes[C]{\smallbreak\pagebreak[2]}\Standort{TMW, HS AM 23363 Ba.}
\physDesc{Brief, 1 Blatt, 1 Seite, 700 Zeichen
\newline{}Schreibmaschine
\newline{}Handschrift: schwarze Tinte, deutsche Kurrent (\noindent{}Schlussformel, Unterschrift und Einfügung von
                                 »ev.«)
\newline{}Ordnung: Lochung }
\buchAbdrucke{\weitereDrucke{1) Arthur Schnitzler: \emph{The Letters of Arthur Schnitzler to Hermann Bahr}. Chapel Hill: \emph{The University of North Carolina Press} 1978, S. 83.} \weitereDrucke{2) Hermann Bahr, Arthur Schnitzler: \emph{Briefwechsel, Aufzeichnungen, Dokumente (1891–1931)}. Göttingen: \emph{Wallstein} 2018, S. 288.} }\toendnotes[C]{\smallbreak}
\pstart
           \raggedleft{}{\pb}Wien\oindex{Wien@\textbf{Wien}, \emph{A.ADM2}|pw}, 8. Januar 1904.\pend
           
\pstart
           \raggedleft{}XVIII. Spöttelg. 7.\oindex{Edmund-Weiss-Gasse 7@\textbf{Edmund-Weiß-Gasse 7}, \emph{Wohngebäude (K.WHS)}|pw}\pend
           
\pstart{}Lieber Hermann!\pend\vspace{0.5em}
\pstart
           Die Adresse des Dr. Stephan Epstein\pwindex{Epstein, Stephan 12.11.1866 – 1941@\textsc{Epstein, Stephan} (12.11.1866 – 1941), \emph{Schriftsteller/Schriftstellerin, Dramaturg/Dramaturgin, Übersetzer/Übersetzerin}|pw} ist: Paris, 78, Rue de l’Assomption\oindex{rue de l Assomption@\textbf{rue de l’Assomption}, \emph{Straße (K.STR)}|pw}. Er hat dir wol
               auch über das \introOben{}ev.\introOben{}{ }\label{K_L01359-1v}\edtext{Gastspiel}{\lemma{\textnormal{\emph{Gastspiel}}}\Cendnote{\textnormal{1904 trat Antoine\pwindex{Antoine, Andre 1858-01-31 – 1943-10-23@\textsc{Antoine, André} (1858-01-31 – 1943-10-23), \emph{Theaterleiter/Theaterleiterin, Schauspieler/Schauspielerin}|pwk} nicht in Wien\oindex{Wien@\textbf{Wien}, \emph{A.ADM2}|pwk} auf.}}}\label{K_L01359-1}{ }Antoine\pwindex{Antoine, Andre 1858-01-31 – 1943-10-23@\textsc{Antoine, André} (1858-01-31 – 1943-10-23), \emph{Theaterleiter/Theaterleiterin, Schauspieler/Schauspielerin}|pw} geschrieben. Seine Frau\pwindex{Epstein, Henriette Estelle 19.08.1885 – 25.07.1967@\textsc{Epstein, Henriette Estelle} (19.08.1885 – 25.07.1967)|pwv}, die \label{K_L01359-2v}\edtext{neulich}{\lemma{\textnormal{\emph{neulich}}}\Cendnote{\textnormal{Siehe A. S.: \emph{Tagebuch}, 28. 12. 1903.
               }}}\label{K_L01359-2} in Wien\oindex{Wien@\textbf{Wien}, \emph{A.ADM2}|pw} war, fragte mich, auf welche Weise
               es möglich wäre, die Sezession\oindex{Secession@\textbf{Secession}, \emph{Museum (K.MUS)}|pw} zu veranlassen,
               einen in Paris\oindex{Paris@\textbf{Paris}, \emph{P.PPLC}|pw} lebenden Künstler, Bernhard Hoetger\pwindex{Hoetger, Bernhard 04.05.1874 – 18.07.1949@\textsc{Hoetger, Bernhard} (04.05.1874 – 18.07.1949), \emph{Grafiker/Grafikerin, Bildhauer/Bildhauerin}|pw}, zu einer Ausstellung seiner
               Werke einzuladen. Sie schickt Dir nächstens irgend ein französisches Journal, in
               welchem Hoetgerische\pwindex{Hoetger, Bernhard 04.05.1874 – 18.07.1949@\textsc{Hoetger, Bernhard} (04.05.1874 – 18.07.1949), \emph{Grafiker/Grafikerin, Bildhauer/Bildhauerin}|pw} Arbeiten abgebildet
               sind.\pend
           
\pstart
           Morgen fahre ich auf einige Tage auf den Semmering\oindex{Semmering@\textbf{Semmering}, \emph{A.ADM3}|pw}, komme gleich, wenn ich zurück bin, mit deiner freundlichen
               Erlaubnis zu dir, und hoffe, dich wohl zu finden.\pend
           
\pstart
           {[}hs.:{]} Herzliche Grüße, auch von meiner Frau\pwindex{Schnitzler, Olga 17.01.1882 – 13.01.1970@\textsc{Schnitzler, Olga} (17.01.1882 – 13.01.1970), \emph{Schauspieler/Schauspielerin, Sänger/Sängerin}|pwv}{\\[\baselineskip]}dein \spacefill\mbox{Arthu\damage{r}}\pend
           \leftskip=0em{}\selectlanguage{ngerman}\endnumbering\briefempfaengerindex{Bahr, Hermann@\textsc{Bahr, Hermann}!zzzSchnitzler, Arthur@\emph{von Arthur Schnitzler}!1904-01-081@{8. 1. 1904}|)be}\mylabel{L01359h}  \normalsize

\doendnotes{C}
\bigskip
\vfill

\clearpage

\footnotesize

\lohead{\textsc{register}}

% Definiere theindex-Environment komplett neu ohne reledmac
\makeatletter
\renewenvironment{theindex}{%
  \section*{\indexname}%
  \setlength{\parindent}{0pt}%
  \setlength{\parskip}{0pt plus 0.3pt}%
  \let\item\@idxitem
}{%
  \clearpage
}
\makeatother

\IfFileExists{\jobname-pw.ind}{\input{\jobname-pw.ind}}{}

\end{document}

      