%% latex-leseansicht-vorspann.tex
%% Vorspann für die Leseansicht.
%% Lädt die gemeinsame Datei latex-vorspann.tex mit nicht gesetztem Schalter.

\newif\ifkorrekturansicht
\korrekturansichtfalse

\input{../tex-inputs/latex-vorspann}


\section[Arthur Schnitzler an Hermann Bahr, 8. 1. 1904]{L01359 Arthur Schnitzler an Hermann Bahr, 8. 1. 1904}
\nopagebreak\mylabel{L01359v}
\rehead{ }\normalsize\beginnumbering\briefempfaengerindex{Bahr, Hermann@\textsc{Bahr, Hermann}!zzzSchnitzler, Arthur@\emph{von Arthur Schnitzler}!1904-01-081@{8. 1. 1904}|(be}
\toendnotes[C]{\smallbreak\pagebreak[2]}
\correspDesc{Versand  durch Arthur Schnitzler am 8. 1. 1904 in Wien
\newline{}Erhalt  durch Hermann Bahr im Zeitraum [8. 1. 1904
                  – 12. 1. 1904?] in Wien}\toendnotes[C]{\smallbreak}
\Standort{TMW, HS AM 23363 Ba.}
\physDesc{Brief, 1 Blatt, 1 Seite, 700 Zeichen
\newline{}Schreibmaschine
\newline{}Handschrift: schwarze Tinte, deutsche Kurrent (\noindent{}Schlussformel, Unterschrift und Einfügung von
                                 »ev.«)
\newline{}Ordnung: Lochung }
\buchAbdrucke{\weitereDrucke{1) \emph{8. 1. 1904.} In: Arthur Schnitzler: \emph{The Letters of Arthur Schnitzler to Hermann Bahr}. Edited, annotated, and with an introduction, by Donald G. Daviau. Chapel Hill: \emph{The University of North Carolina Press} 1978, S. 83 (University of North Carolina studies in the Germanic languages
                        and literatures, 89).} \weitereDrucke{2) Hermann Bahr, Arthur Schnitzler: \emph{Briefwechsel, Aufzeichnungen, Dokumente (1891–1931)}. Herausgegeben von Kurt Ifkovits und Martin Anton Müller. Göttingen: \emph{Wallstein} 2018, S. 288.} }\toendnotes[C]{\smallbreak}
\pstart
           \raggedleft{}{\pb}Wien\oindex{Wien@\textbf{Wien}, \emph{Verwaltungsgebiet}|pw}, 8. Januar 1904.\pend
           
\pstart
           \raggedleft{}XVIII. Spöttelg. 7.\oindex{Wien@\textbf{Wien}!XVIII., Währing@\textbf{XVIII., Währing}!Edmund-Weiß-Gasse 7@\textbf{Edmund-Weiß-Gasse 7}, \emph{Wohngebäude}|pw}\pend
           
\pstart{}Lieber Hermann!\pend\vspace{0.5em}
\pstart
           Die Adresse des Dr. Stephan Epstein\pwindex{Epstein, Stephan 12.\,11.\,1866 Warschau – 1941 Paris@\textsc{Epstein, Stephan} (12.\,11.\,1866 Warschau – 1941 Paris), \emph{Schriftsteller, Dramaturg, Übersetzer}|pw} ist: Paris, 78, Rue de l’Assomption\oindex{rue de l’Assomption@\textbf{rue de l’Assomption}, \emph{Straße}|pw}. Er hat dir wol
               auch über das \introOben{}ev.\introOben{}{ }\label{K_L01359-1v}\edtext{Gastspiel}{\lemma{\textnormal{\emph{Gastspiel}}}\Cendnote{\textnormal{1904 trat Antoine\pwindex{Antoine, André 31.\,1.\,1858 Limoges – 23.\,10.\,1943 Le Pouliguen@\textsc{Antoine, André} (31.\,1.\,1858 Limoges – 23.\,10.\,1943 Le Pouliguen), \emph{Theaterleiter, Schauspieler}|pwk} nicht in Wien\oindex{Wien@\textbf{Wien}, \emph{Verwaltungsgebiet}|pwk} auf.}}}\label{K_L01359-1}{ }Antoine\pwindex{Antoine, André 31.\,1.\,1858 Limoges – 23.\,10.\,1943 Le Pouliguen@\textsc{Antoine, André} (31.\,1.\,1858 Limoges – 23.\,10.\,1943 Le Pouliguen), \emph{Theaterleiter, Schauspieler}|pw} geschrieben. Seine Frau\pwindex{Epstein, Henriette Estelle 19.\,8.\,1885 La Chartre-sur-le-Loir – 25.\,7.\,1967 Riom@\textsc{Epstein, Henriette Estelle} (19.\,8.\,1885 La Chartre-sur-le-Loir – 25.\,7.\,1967 Riom)|pwv}, die \label{K_L01359-2v}\edtext{neulich}{\lemma{\textnormal{\emph{neulich}}}\Cendnote{\textnormal{Siehe A. S.: \emph{Tagebuch}, 28. 12. 1903.
               }}}\label{K_L01359-2} in Wien\oindex{Wien@\textbf{Wien}, \emph{Verwaltungsgebiet}|pw} war, fragte mich, auf welche Weise
               es möglich wäre, die Sezession\oindex{Wien@\textbf{Wien}!I., Innere Stadt@\textbf{I., Innere Stadt}!Secession@\textbf{Secession}, \emph{Museum}|pw} zu veranlassen,
               einen in Paris\oindex{Paris@\textbf{Paris}, \emph{Hauptstadt}|pw} lebenden Künstler, Bernhard Hoetger\pwindex{Hoetger, Bernhard 4.\,5.\,1874 Hörde – 18.\,7.\,1949 Interlaken@\textsc{Hoetger, Bernhard} (4.\,5.\,1874 Hörde – 18.\,7.\,1949 Interlaken), \emph{Grafiker, Bildhauer}|pw}, zu einer Ausstellung seiner
               Werke einzuladen. Sie schickt Dir nächstens irgend ein französisches Journal, in
               welchem Hoetgerische\pwindex{Hoetger, Bernhard 4.\,5.\,1874 Hörde – 18.\,7.\,1949 Interlaken@\textsc{Hoetger, Bernhard} (4.\,5.\,1874 Hörde – 18.\,7.\,1949 Interlaken), \emph{Grafiker, Bildhauer}|pw} Arbeiten abgebildet
               sind.\pend
           
\pstart
           Morgen fahre ich auf einige Tage auf den Semmering\oindex{Semmering@\textbf{Semmering}, \emph{Verwaltungsgebiet}|pw}, komme gleich, wenn ich zurück bin, mit deiner freundlichen
               Erlaubnis zu dir, und hoffe, dich wohl zu finden.\pend
           
\pstart
           {[}hs.:{]} Herzliche Grüße, auch von meiner Frau\pwindex{Schnitzler, Olga 17.\,1.\,1882 Wien – 13.\,1.\,1970 Lugano@\textsc{Schnitzler, Olga} (17.\,1.\,1882 Wien – 13.\,1.\,1970 Lugano), \emph{Schauspielerin, Sängerin}|pwv}{\\[\baselineskip]}dein \spacefill\mbox{Arthu\damage{r}}\pend
           \leftskip=0em{}\selectlanguage{ngerman}\endnumbering\briefempfaengerindex{Bahr, Hermann@\textsc{Bahr, Hermann}!zzzSchnitzler, Arthur@\emph{von Arthur Schnitzler}!1904-01-081@{8. 1. 1904}|)be}\mylabel{L01359h}  \newcommand{\dateiname}{L01359}\newcommand{\titel}{Arthur Schnitzler an Hermann Bahr, 8. 1. 1904}\newcommand{\editorInnen}{Herausgegeben von Martin Anton Müller}%% latex-leseansicht-abspann.tex
%% Abspann für die Leseansicht.
%% Der Schalter \ifkorrekturansicht ist bereits durch den Vorspann gesetzt.

%% latex-abspann.tex
%% Gemeinsamer Abspann für Korrekturansicht und Leseansicht.
%% Setzt den Schalter \ifkorrekturansicht voraus (gesetzt in den
%% einbindenden Dateien latex-korrekturansicht-abspann.tex bzw.
%% latex-leseansicht-abspann.tex).
%% ---------------------------------------------------------------

\normalsize

% Das esempio-Environment wird nur in der Leseansicht benötigt
\ifkorrekturansicht\else
\newenvironment{esempio}[3]%
{
    \vspace{1.5ex}
    \rlap{\underline{#1}}
    \par
    \setlength{\parindent}{0cm}
    \nopagebreak
    \leftskip=#2cm
    \rightskip=#3cm
}
{
    \par
}
\fi

\doendnotes{C}
\bigskip
\vfill

\clearpage

\footnotesize

\ifkorrekturansicht
  \lohead{\textsc{register}}
\fi

% theindex-Environment neu definieren ohne reledmac
\makeatletter
\renewenvironment{theindex}{%
  \ifkorrekturansicht
    \section*{\indexname}%
  \else
    \subsubsection*{Index der erwähnten Entitäten}%
  \fi
  \setlength{\parindent}{0pt}%
  \setlength{\parskip}{0pt plus 0.3pt}%
  \let\item\@idxitem
}{%
  \ifkorrekturansicht\clearpage\fi
}
\makeatother

\IfFileExists{\jobname-pw.ind}{\input{\jobname-pw.ind}}{}

% Quellenangabe nur in der Leseansicht
\ifkorrekturansicht\else
% Fallback-Definitionen, falls die .tex-Datei \titel etc. nicht gesetzt hat
\providecommand{\titel}{}
\providecommand{\editorInnen}{}
\providecommand{\dateiname}{\jobname}

\vspace{3cm}

\vfill

\footnotesize
\textsc{Quelle}: \titel. Herausgegeben von {\editorInnen}. In: \emph{Arthur Schnitzler: Briefwechsel mit Autorinnen und Autoren}.
 Digitale Edition, https://schnitzler-briefe.acdh.oeaw.ac.at/{\dateiname}.html (Stand \today)
\fi

\end{document}


