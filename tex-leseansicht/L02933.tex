%% latex-leseansicht-vorspann.tex
%% Vorspann für die Leseansicht.
%% Lädt die gemeinsame Datei latex-vorspann.tex mit nicht gesetztem Schalter.

\newif\ifkorrekturansicht
\korrekturansichtfalse

\input{../tex-inputs/latex-vorspann}

\begin{center}
            \textcolor{red}{ENTWURF, NICHT FERTIG KORRIGIERT}
                      \end{center}
            
         \renewcommand{\erwaehnteOrte}{Orte: Berlin, Breslau, Frankgasse, Wien}
         \renewcommand{\erwaehnteWerke}{Werke: Der Schleier der Beatrice. Schauspiel in fünf Akten, Neue Freie Presse, Theater- und Kunstnachrichten [Schleier der Beatrice in Breslau]}
               \section[ Paul Goldmann an Arthur Schnitzler, 22. 9. 1900]{ Paul Goldmann an Arthur Schnitzler, 22. 9. 1900}\nopagebreak\mylabel{v}\rehead{ }\begin{ledgroupsized}[t]{13cm}\normalsize\beginnumbering \toendnotes[C]{\smallbreak\pagebreak[2]} \Standort{DLA, A:Schnitzler, HS.NZ85.1.3170.}
\physDesc{Postkarte
\newline{}Handschrift: 1) blaue Tinte, deutsche Kurrent\hspace{1em}2) blaue Tinte, lateinische Kurrent (\noindent{}Adresse)\hspace{1em}\newline{}Versand: 1) Stempel: »\nobreak{}\oindex{Berlin@\textbf{Berlin}|pwk}Berlin\textcolor{gray}{,}
                                       S.W., 11. 9. {[}19{]}00, 4–5 N., 46\nobreak{}«.   2) Stempel: »\nobreak{}Wien 9/3 72, 23. 9. 00, 9. V, Bestellt\nobreak{}«. 
\newline{}Schnitzler: mit Bleistift das Jahr »{[}1{]}900« vermerkt }\toendnotes[C]{\smallbreak}\pstart{}{\pb}\textcolor{gray}{\textbf{An}}\pend{}\pstart{}Herrn\pend{}\pstart{}Dr. Arthur Schnitzler\pend{}\pstart{}\textcolor{gray}{\textbf{in}}{ }Wien\oindex{Wien@\textbf{Wien}|pw}\pend{}\pstart{}IX. Frankgaſse 1\oindex{Frankgasse@\textbf{Frankgasse}|pw}.\pend{}{\bigskip}\pstart
           {\pb}Berlin\oindex{Berlin@\textbf{Berlin}|pw}, 22. September.\pend
           \pstart
           \label{K_L02933-1v}\edtext{\textsc{M. l. F.}}{\lemma{\textnormal{\emph{M. l. F.}}}\Cendnote{\textnormal{Mein lieber Freund}}}\label{K_L02933-1h},
               Die aus Berlin\oindex{Berlin@\textbf{Berlin}|pw} datirte \label{K_L02933-2v}\edtext{Mittheilung\pwindex{?? Werk@Nicht ermittelte Verfasserinnen und Verfasser!Theater- und Kunstnachrichten [Schleier der Beatrice in Breslau]1900-09-21@\emph{Theater- und Kunstnachrichten [Schleier der Beatrice in Breslau]} {[}1900-09-21{]}|pwv}}{\lemma{\textnormal{\emph{Mittheilung}}}\Cendnote{\textnormal{o. V.: \emph{Theater- und Kunstnachrichten}\pwindex{?? Werk@Nicht ermittelte Verfasserinnen und Verfasser!Theater- und Kunstnachrichten [Schleier der Beatrice in Breslau]1900-09-21@\emph{Theater- und Kunstnachrichten [Schleier der Beatrice in Breslau]} {[}1900-09-21{]}|pwk}.
                     In: \emph{Neue Freie Presse}\pwindex{Neue Freie Presse1864 – 1939@\emph{Neue Freie Presse} {[}1864 – 1939{]}|pwk}, Nr. 12959, 21. 9. 1900, Morgenblatt, S. 6–7, hier:
                  S. 7.}}}\label{K_L02933-2h} von der Aufführung der »\textsc{Beatrice\pwindex{Schnitzler, Arthur 15.05.1862 – 21.10.1931@\textsc{Schnitzler, Arthur} (15.05.1862 – 21.10.1931), \emph{Schriftsteller, Mediziner}!Schleier der Beatrice. Schauspiel in fuenf Akten1900-12-01@\strich\emph{Der Schleier der Beatrice. Schauspiel in fünf Akten} {[}1900-12-01{]}|pw}}« in Breslau\oindex{Breslau@\textbf{Breslau}|pw}, welche geſtern im Theatertheil der N. Fr.
                  Pr.\textcolor{red}{\textsuperscript{XXXX indx}} zu leſen war, ſtammt ſelbſtverſtändlich nicht von mir.\pend
           \pstart
           Viele Grüße! {\\[\baselineskip]}\spacefill\mbox{P. G.}\pend
           \leftskip=0em{}
         
         \endnumbering\mylabel{h}\end{ledgroupsized}\begin{anhang}\end{anhang}\newcommand{\dateiname}{L02933}\newcommand{\titel}{Paul Goldmann an Arthur Schnitzler, 22. 9. 1900}\newcommand{\editorInnen}{Martin Anton Müller und Laura Untner}%% latex-leseansicht-abspann.tex
%% Abspann für die Leseansicht.
%% Der Schalter \ifkorrekturansicht ist bereits durch den Vorspann gesetzt.

%% latex-abspann.tex
%% Gemeinsamer Abspann für Korrekturansicht und Leseansicht.
%% Setzt den Schalter \ifkorrekturansicht voraus (gesetzt in den
%% einbindenden Dateien latex-korrekturansicht-abspann.tex bzw.
%% latex-leseansicht-abspann.tex).
%% ---------------------------------------------------------------

\normalsize

% Das esempio-Environment wird nur in der Leseansicht benötigt
\ifkorrekturansicht\else
\newenvironment{esempio}[3]%
{
    \vspace{1.5ex}
    \rlap{\underline{#1}}
    \par
    \setlength{\parindent}{0cm}
    \nopagebreak
    \leftskip=#2cm
    \rightskip=#3cm
}
{
    \par
}
\fi

\doendnotes{C}
\bigskip
\vfill

\clearpage

\footnotesize

\ifkorrekturansicht
  \lohead{\textsc{register}}
\fi

% theindex-Environment neu definieren ohne reledmac
\makeatletter
\renewenvironment{theindex}{%
  \ifkorrekturansicht
    \section*{\indexname}%
  \else
    \subsubsection*{Index der erwähnten Entitäten}%
  \fi
  \setlength{\parindent}{0pt}%
  \setlength{\parskip}{0pt plus 0.3pt}%
  \let\item\@idxitem
}{%
  \ifkorrekturansicht\clearpage\fi
}
\makeatother

\IfFileExists{\jobname-pw.ind}{\input{\jobname-pw.ind}}{}

% Quellenangabe nur in der Leseansicht
\ifkorrekturansicht\else
% Fallback-Definitionen, falls die .tex-Datei \titel etc. nicht gesetzt hat
\providecommand{\titel}{}
\providecommand{\editorInnen}{}
\providecommand{\dateiname}{\jobname}

\vspace{3cm}

\vfill

\footnotesize
\textsc{Quelle}: \titel. Herausgegeben von {\editorInnen}. In: \emph{Arthur Schnitzler: Briefwechsel mit Autorinnen und Autoren}.
 Digitale Edition, https://schnitzler-briefe.acdh.oeaw.ac.at/{\dateiname}.html (Stand \today)
\fi

\end{document}


      