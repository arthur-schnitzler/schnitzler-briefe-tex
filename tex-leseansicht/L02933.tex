%% latex-korrekturansicht-vorspann.tex
%% Vorspann für die Korrekturansicht.
%% Lädt die gemeinsame Datei latex-vorspann.tex mit gesetztem Schalter.

\newif\ifkorrekturansicht
\korrekturansichttrue

\input{../tex-inputs/latex-vorspann}


\section[ Paul Goldmann an Arthur Schnitzler, 22. 9. 1900]{L02933 Paul Goldmann an Arthur Schnitzler, 22. 9. 1900}
\nopagebreak\mylabel{L02933v}
\rehead{ }\normalsize\beginnumbering\briefempfaengerindex{Schnitzler, Arthur@\textsc{Schnitzler, Arthur}!zzzGoldmann, Paul@\emph{von Paul Goldmann}!1900-09-221@{22. 9. 1900}|(be}
\toendnotes[C]{\smallbreak\pagebreak[2]}\Standort{DLA, A:Schnitzler, HS.NZ85.1.3170.}
\physDesc{Postkarte, 266 Zeichen
\newline{}Handschrift: 1) blaue Tinte, deutsche Kurrent\hspace{1em}2) blaue Tinte, lateinische Kurrent (\noindent{}Adresse)\hspace{1em}
\newline{}Versand: 1) Stempel: »\nobreak{}\oindex{Berlin@\textbf{Berlin}, \emph{P.PPLC}|pwk}Berlin\textcolor{gray}{,}
                                       S.W., 22. 9. 00, 4–5 N., 46\nobreak{}«.   2) Stempel: »\nobreak{}\oindex{IX., Alsergrund@\textbf{IX., Alsergrund}, \emph{A.ADM3}|pwk}Wien 9/3 72, 23. 9. 00, 9. V, Bestellt\nobreak{}«. 
\newline{}Schnitzler: mit Bleistift das Jahr »900« vermerkt }\toendnotes[C]{\smallbreak}\pstart{}{\pb}Herrn\pend{}\pstart{}Dr. Arthur Schnitzler\pend{}\pstart{}Wien\oindex{Wien@\textbf{Wien}, \emph{A.ADM2}|pw}\pend{}\pstart{}IX. Frankgaſse 1\oindex{Frankgasse 1@\textbf{Frankgasse 1}, \emph{Wohngebäude (K.WHS)}|pw}.\pend{}{\bigskip}\vspace{1em}
\pstart
           {\pb}Berlin\oindex{Berlin@\textbf{Berlin}, \emph{P.PPLC}|pw}, 22. September.\pend
           \vspace{0.5em}
\pstart
           \label{K_L02933-1v}\edtext{M. l. F.}{\lemma{\textnormal{\emph{M. l. F.}}}\Cendnote{\textnormal{Mein lieber Freund}}}\label{K_L02933-1}, die aus Berlin\oindex{Berlin@\textbf{Berlin}, \emph{P.PPLC}|pw} datirte \label{K_L02933-2v}\edtext{Mittheilung\pwindex{Theater- und Kunstnachrichten [Schleier der Beatrice in Breslau]@\emph{Theater- und Kunstnachrichten [Schleier der Beatrice in Breslau]}|pwv}}{\lemma{\textnormal{\emph{Mittheilung}}}\Cendnote{\textnormal{»– Aus \so{Berlin}\oindex{Berlin@\textbf{Berlin}, \emph{P.PPLC}|pw} wird uns gemeldet: ›Der Schleier der
                        Beatrice\pwindex{Schleier der Beatrice. Schauspiel in fuenf Akten@\emph{Der Schleier der Beatrice. Schauspiel in fünf Akten}|pw}‹ von Arthur \so{Schnitzler} wird im Einvernehmen mit dem Dichter demnächst im Breslauer Stadttheater\orgindex{Lobe-Theater@Lobe-Theater|pw} ſeine Erſtaufführung erleben.«
                     [O. V.]: \emph{Theater- und
                        Kunstnachrichten}\pwindex{Theater- und Kunstnachrichten [Schleier der Beatrice in Breslau]@\emph{Theater- und Kunstnachrichten [Schleier der Beatrice in Breslau]}|pwk}. In: \emph{Neue Freie
                        Presse}\pwindex{Neue Freie Presse@\emph{Neue Freie Presse}|pwk}, Nr. 12.959, 21. 9. 1900,
                     Morgenblatt, S. 6–7, hier: S. 7.}}}\label{K_L02933-2} von der Aufführung der »\textsc{Beatrice\pwindex{Schleier der Beatrice. Schauspiel in fuenf Akten@\emph{Der Schleier der Beatrice. Schauspiel in fünf Akten}|pw}}« in Breslau\oindex{Breslau@\textbf{Breslau}, \emph{P.PPLA}|pw}, welche geſtern im Theatertheil der N. Fr.
                  Pr.\pwindex{Neue Freie Presse@\emph{Neue Freie Presse}|pw} zu leſen war, ſtammt ſelbſtverſtändlich nicht von mir.\pend
           
\pstart
           Viele Grüße! {\\[\baselineskip]}\spacefill\mbox{P. G.}\pend
           \leftskip=0em{}\selectlanguage{ngerman}\endnumbering\briefempfaengerindex{Schnitzler, Arthur@\textsc{Schnitzler, Arthur}!zzzGoldmann, Paul@\emph{von Paul Goldmann}!1900-09-221@{22. 9. 1900}|)be}\mylabel{L02933h}  \normalsize

\doendnotes{C}
\bigskip
\vfill

\clearpage

\footnotesize

\lohead{\textsc{register}}

% Definiere theindex-Environment komplett neu ohne reledmac
\makeatletter
\renewenvironment{theindex}{%
  \section*{\indexname}%
  \setlength{\parindent}{0pt}%
  \setlength{\parskip}{0pt plus 0.3pt}%
  \let\item\@idxitem
}{%
  \clearpage
}
\makeatother

\IfFileExists{\jobname-pw.ind}{\input{\jobname-pw.ind}}{}

\end{document}

      