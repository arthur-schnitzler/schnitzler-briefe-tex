%% latex-korrekturansicht-vorspann.tex
%% Vorspann für die Korrekturansicht.
%% Lädt die gemeinsame Datei latex-vorspann.tex mit gesetztem Schalter.

\newif\ifkorrekturansicht
\korrekturansichttrue

\input{../tex-inputs/latex-vorspann}


\section[Hugo Hofmannsthal an Arthur Schnitzler, 15. 5. 1923]{L02399 Hugo Hofmannsthal an Arthur Schnitzler, 15. 5. 1923}
\nopagebreak\mylabel{L02399v}
\rehead{ }\normalsize\beginnumbering\briefempfaengerindex{Schnitzler, Arthur@\textsc{Schnitzler, Arthur}!zzzHofmannsthal, Hugo von@\emph{von Hugo von Hofmannsthal}!1923-05-151@{15. 5. 1923}|(be}
\toendnotes[C]{\smallbreak\pagebreak[2]}\Standort{CUL, Schnitzler, B 43.}
\physDesc{Bildpostkarte, 266 Zeichen
\newline{}Handschrift: Bleistift, lateinische Kurrent
\newline{}Versand: Stempel: »\nobreak{}\oindex{Magglingen@\textbf{Magglingen}, \emph{P.PPL}|pwk}Macolin (Magglingen), 16. V. 23\nobreak{}«.  
\newline{}Ordnung: 1) mit Bleistift von Frieda
                                    Pollak\pwindex{Pollak, Frieda 08.12.1881 – 13.07.1937@\textsc{Pollak, Frieda} (08.12.1881 – 13.07.1937), \emph{Sekretär/Sekretärin}|pw} (?) mit dem Buchstaben »A«
                                 (Abgeschrieben/Abschrift) gekennzeichnet  2) mit Bleistift von unbekannter Hand nummeriert: »\strikeout{373}« 3) mit Bleistift von unbekannter Hand nummeriert:
                                    »377«}
\buchAbdrucke{\weitereDrucke{Hugo von Hofmannsthal, Arthur Schnitzler: \emph{Briefwechsel}. Frankfurt am Main: \emph{S. Fischer} 1964, S. 298.} }\toendnotes[C]{\smallbreak}\pstart{}{\pb}Herrn D\textsuperscript{r} Arthur Schnitzler\pend{}\pstart{}Wien\oindex{Wien@\textbf{Wien}, \emph{A.ADM2}|pw}\pend{}\pstart{}XVIII Sternwartestrasse 71\oindex{Sternwartestrasse 71@\textbf{Sternwartestraße 71}, \emph{Wohngebäude (K.WHS)}|pw}\pend{}{\bigskip}
\pstart
           \noindent{}\centering{}{\pb}\textcolor{gray}{\textbf{Nr. 6508 Biel – Bienne\oindex{Biel@\textbf{Biel}, \emph{P.PPLA2}|pw}}}\pend
           \vspace{1em}
\pstart
           \raggedleft{}{\pb}Biel\oindex{Biel@\textbf{Biel}, \emph{P.PPLA2}|pw} den 15\textsuperscript{ten} Mai\pend
           
\pstart{}mein lieber Arthur \pend\vspace{0.5em}
\pstart
           hier sind wir nämlich vor 25 Jahren (am \label{K_L02399-1v}\edtext{20\textsuperscript{ten} oder 21\textsuperscript{ten} August 1898}{\lemma{\textnormal{\emph{20\textsuperscript{ten} … 1898}}}\Cendnote{\textnormal{Es dürfte sich um den
                     13. 8. 1898 gehandelt haben, vgl. A. S.: \emph{Tagebuch}, 13. 8. 1898.}}}\label{K_L02399-1}) miteinander gesessen!\pend
           
\pstart
           Das ist seltsam und geisterhaft.\pend
           
\pstart
           Ich schicke Ihnen viele freundschaftliche Gedanken! \pend
           
\pstart
           Ihr{\\[\baselineskip]}\spacefill\mbox{Hugo}\pend
           \leftskip=0em{}\selectlanguage{ngerman}\endnumbering\briefempfaengerindex{Schnitzler, Arthur@\textsc{Schnitzler, Arthur}!zzzHofmannsthal, Hugo von@\emph{von Hugo von Hofmannsthal}!1923-05-151@{15. 5. 1923}|)be}\mylabel{L02399h}  \normalsize

\doendnotes{C}
\bigskip
\vfill

\clearpage

\footnotesize

\lohead{\textsc{register}}

% Definiere theindex-Environment komplett neu ohne reledmac
\makeatletter
\renewenvironment{theindex}{%
  \section*{\indexname}%
  \setlength{\parindent}{0pt}%
  \setlength{\parskip}{0pt plus 0.3pt}%
  \let\item\@idxitem
}{%
  \clearpage
}
\makeatother

\IfFileExists{\jobname-pw.ind}{\input{\jobname-pw.ind}}{}

\end{document}

      