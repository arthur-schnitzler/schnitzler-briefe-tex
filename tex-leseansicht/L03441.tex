%% latex-leseansicht-vorspann.tex
%% Vorspann für die Leseansicht.
%% Lädt die gemeinsame Datei latex-vorspann.tex mit nicht gesetztem Schalter.

\newif\ifkorrekturansicht
\korrekturansichtfalse

\input{../tex-inputs/latex-vorspann}


         
         \renewcommand{\erwaehntePersonen}{Personen: Hermann Bahr, Fritz Freund, Wolfgang Heine}
         \renewcommand{\erwaehnteInstitutionen}{Institutionen: Ernst Rowohlt Verlag, Sozialdemokratische Partei Deutschlands (SPD), Staatsanwaltschaft Berlin, Wiener Verlag}
         \renewcommand{\erwaehnteOrte}{Orte: Berlin, Dessauer Straße, Deutschland, Wien}
         \renewcommand{\erwaehnteWerke}{Werke: Der Kampf um den Reigen. Vollständiger Bericht über die sechstägige Verhandlung gegen Direktion und Darsteller des Kleinen Schauspielhauses Berlin, Neue Freie Presse, Reigen. Zehn Dialoge}
               \section[ Paul Goldmann an Arthur Schnitzler, 19. 3. {[}1904{]}]{ Paul Goldmann an Arthur Schnitzler, 19. 3. {[}1904{]}}\nopagebreak\mylabel{v}\rehead{ }\begin{ledgroupsized}[t]{13cm}\normalsize\beginnumbering \toendnotes[C]{\smallbreak\pagebreak[2]} \Standort{DLA, A:Schnitzler, HS.NZ85.1.3174.}
\physDesc{Brief, 1 Blatt, 3 Seiten, 916 Zeichen
\newline{}Handschrift: blaue Tinte, deutsche Kurrent
\newline{}Schnitzler: mit Bleistift das Jahr »904« vermerkt }\toendnotes[C]{\smallbreak}\pstart
           \noindent{}\raggedleft{}{\pb}\textcolor{gray}{\textbf{DESSAUERSTRASSE 19\oindex{Dessauer Strasse@\textbf{Dessauer Straße}|pw}}}\pend
           \pstart
           Berlin\oindex{Berlin@\textbf{Berlin}|pw}, 19. März.\pend
           \pstart\center{}Mein lieber Freund,\pend\pstart
           Das \label{K_L03441-1v}\edtext{Verbot des »Reigen\pwindex{Schnitzler, Arthur 15.05.1862 – 21.10.1931@\textsc{Schnitzler, Arthur} (15.05.1862 – 21.10.1931), \emph{Schriftsteller, Mediziner}!Reigen. Zehn Dialoge1900@\strich\emph{Reigen. Zehn Dialoge} {[}1900{]}|pw}«}{\lemma{\textnormal{\emph{Verbot des »Reigen«}}}\Cendnote{\textnormal{Am 16. 3. 1904 war die
                     1903 im \emph{Wiener
                     Verlag}\orgindex{Wiener Verlag@Wiener Verlag|pwk} erschienene Buchausgabe des \emph{Reigen}\pwindex{Schnitzler, Arthur 15.05.1862 – 21.10.1931@\textsc{Schnitzler, Arthur} (15.05.1862 – 21.10.1931), \emph{Schriftsteller, Mediziner}!Reigen. Zehn Dialoge1900@\strich\emph{Reigen. Zehn Dialoge} {[}1900{]}|pwk} durch die \emph{Berliner
                     Staatsanwaltschaft}\orgindex{Staatsanwaltschaft Berlin@Staatsanwaltschaft Berlin|pwk} im ganzen Deutschen
                     Reich\oindex{Deutschland@\textbf{Deutschland}|pwk} konfisziert worden.}}}\label{K_L03441-1h} durch die Berlin\oindex{Berlin@\textbf{Berlin}|pw}er Staatsanwaltſchaft\orgindex{Staatsanwaltschaft Berlin@Staatsanwaltschaft Berlin|pw} ſcheint ſich
               nun wohl leider zu beſtätigen? Ich bitte Dich, mir mitzutheilen, ob ich Dir in dieſer
               Angelegenheit irgendwie \strikeout{di} dienen kann? Du weißt,
               daß, nach deutsch\oindex{Deutschland@\textbf{Deutschland}|pwv}em Recht, auf
               jede Confiscation ein Prozeß folgen muß. Es iſt alſo dringend nöthig, daß Du oder
               Dein Verleger\pwindex{Freund, Fritz 07.04.1879 – 08.05.1950@\textsc{Freund, Fritz} (07.04.1879 – 08.05.1950), \emph{Verleger}|pw}\orgindex{Wiener Verlag@Wiener Verlag|pw} einen tüchtigen Rechtsanwalt \strikeout{zur} als Berather
               nehmt, – womöglich einen, der auch ein {\pb}Wort
               politiſcher Oppoſition nicht ſcheut. Beiſpielsweiſe würde ich \label{K_L03441-2v}\edtext{\textsc{Heine\pwindex{Heine, Wolfgang 03.05.1861 – 09.05.1944@\textsc{Heine, Wolfgang} (03.05.1861 – 09.05.1944), \emph{Notar, Politiker, Rechtsanwalt}|pw}}}{\lemma{\textnormal{\emph{Heine}}}\Cendnote{\textnormal{Heine\pwindex{Heine, Wolfgang 03.05.1861 – 09.05.1944@\textsc{Heine, Wolfgang} (03.05.1861 – 09.05.1944), \emph{Notar, Politiker, Rechtsanwalt}|pwk} war ein Freund und Studienkollege Hermann Bahr\pwindex{Bahr, Hermann 19.07.1863 – 15.01.1934@\textsc{Bahr, Hermann} (19.07.1863 – 15.01.1934), \emph{Schriftsteller, Kritiker}|pwk}s. Neben seiner politischen
                  Tätigkeit für die \emph{SPD}\orgindex{Sozialdemokratische Partei Deutschlands (SPD)@Sozialdemokratische Partei Deutschlands (SPD)|pwk} war er als Anwalt tätig.
                  Für den \emph{Reigen}\pwindex{Schnitzler, Arthur 15.05.1862 – 21.10.1931@\textsc{Schnitzler, Arthur} (15.05.1862 – 21.10.1931), \emph{Schriftsteller, Mediziner}!Reigen. Zehn Dialoge1900@\strich\emph{Reigen. Zehn Dialoge} {[}1900{]}|pwk} wurde er erst
                     1921 tätig, vgl. \emph{Der Kampf um den Reigen. Vollständiger Bericht
                        über die sechstägige Verhandlung gegen Direktion und Darsteller des Kleinen
                        Schauspielhauses Berlin}\pwindex{Kampf um den Reigen. Vollstaendiger Bericht ueber die sechstaegige Verhandlung gegen Direktion und Darsteller des Kleinen Schauspielhauses Berlin1922@\emph{Der Kampf um den Reigen. Vollständiger Bericht über die sechstägige Verhandlung gegen Direktion und Darsteller des Kleinen Schauspielhauses Berlin} {[}1922{]}|pwk}. Herausgegeben und mit einer Einleitung von Wolfgang Heine\pwindex{Heine, Wolfgang 03.05.1861 – 09.05.1944@\textsc{Heine, Wolfgang} (03.05.1861 – 09.05.1944), \emph{Notar, Politiker, Rechtsanwalt}|pwk}, Rechtsanwalt,
                     Staatsminister a. D. Berlin\oindex{Berlin@\textbf{Berlin}|pwk}: \emph{Rowohlt}\orgindex{Ernst Rowohlt Verlag@Ernst Rowohlt Verlag|pwk}{ }1922.}}}\label{K_L03441-2h} empfehlen.\pend
           \pstart
           Schreibe mir, ob ich igendwelche Schritte in dieſer Angelegenheit für Dich thun kann,
               – ob Du wünſcheſt, daß irgend Etwas in den Berlin\oindex{Berlin@\textbf{Berlin}|pw}er Blättern oder in der N. Fr. Pr.\pwindex{Neue Freie Presse1864 – 1939@\emph{Neue Freie Presse} {[}1864 – 1939{]}|pw}
               veröffentlicht wird?\pend
           \pstart
           Das Verbot richtet hoffentlich keinen großen \label{K_L03441-3v}\edtext{materiellen Schaden}{\lemma{\textnormal{\emph{materiellen Schaden}}}\Cendnote{\textnormal{Das Verbot des \emph{Reigen}\pwindex{Schnitzler, Arthur 15.05.1862 – 21.10.1931@\textsc{Schnitzler, Arthur} (15.05.1862 – 21.10.1931), \emph{Schriftsteller, Mediziner}!Reigen. Zehn Dialoge1900@\strich\emph{Reigen. Zehn Dialoge} {[}1900{]}|pwk}
                  hatte tatsächlich den gegenteiligen Effekt und förderte den Verkauf.}}}\label{K_L03441-3h} mehr an, – im Gegentheil
                  {\pb}wird es wohl, wie immer ſolche Verbote, auf das
                  Buch\pwindex{Schnitzler, Arthur 15.05.1862 – 21.10.1931@\textsc{Schnitzler, Arthur} (15.05.1862 – 21.10.1931), \emph{Schriftsteller, Mediziner}!Reigen. Zehn Dialoge1900@\strich\emph{Reigen. Zehn Dialoge} {[}1900{]}|pwv} erſt recht aufmerkſam
               machen.\pend
           \pstart
           Viele herzliche Grüße! {\\[\baselineskip]}Dein {\\[\baselineskip]}\spacefill\mbox{Paul Goldm}\pend
           \leftskip=0em{}
         
         \endnumbering\mylabel{h}\end{ledgroupsized}  \newcommand{\dateiname}{L03441}\newcommand{\titel}{Paul Goldmann an Arthur Schnitzler, 19. 3. [1904]}\newcommand{\editorInnen}{Martin Anton Müller und Laura Untner}%% latex-leseansicht-abspann.tex
%% Abspann für die Leseansicht.
%% Der Schalter \ifkorrekturansicht ist bereits durch den Vorspann gesetzt.

%% latex-abspann.tex
%% Gemeinsamer Abspann für Korrekturansicht und Leseansicht.
%% Setzt den Schalter \ifkorrekturansicht voraus (gesetzt in den
%% einbindenden Dateien latex-korrekturansicht-abspann.tex bzw.
%% latex-leseansicht-abspann.tex).
%% ---------------------------------------------------------------

\normalsize

% Das esempio-Environment wird nur in der Leseansicht benötigt
\ifkorrekturansicht\else
\newenvironment{esempio}[3]%
{
    \vspace{1.5ex}
    \rlap{\underline{#1}}
    \par
    \setlength{\parindent}{0cm}
    \nopagebreak
    \leftskip=#2cm
    \rightskip=#3cm
}
{
    \par
}
\fi

\doendnotes{C}
\bigskip
\vfill

\clearpage

\footnotesize

\ifkorrekturansicht
  \lohead{\textsc{register}}
\fi

% theindex-Environment neu definieren ohne reledmac
\makeatletter
\renewenvironment{theindex}{%
  \ifkorrekturansicht
    \section*{\indexname}%
  \else
    \subsubsection*{Index der erwähnten Entitäten}%
  \fi
  \setlength{\parindent}{0pt}%
  \setlength{\parskip}{0pt plus 0.3pt}%
  \let\item\@idxitem
}{%
  \ifkorrekturansicht\clearpage\fi
}
\makeatother

\IfFileExists{\jobname-pw.ind}{\input{\jobname-pw.ind}}{}

% Quellenangabe nur in der Leseansicht
\ifkorrekturansicht\else
% Fallback-Definitionen, falls die .tex-Datei \titel etc. nicht gesetzt hat
\providecommand{\titel}{}
\providecommand{\editorInnen}{}
\providecommand{\dateiname}{\jobname}

\vspace{3cm}

\vfill

\footnotesize
\textsc{Quelle}: \titel. Herausgegeben von {\editorInnen}. In: \emph{Arthur Schnitzler: Briefwechsel mit Autorinnen und Autoren}.
 Digitale Edition, https://schnitzler-briefe.acdh.oeaw.ac.at/{\dateiname}.html (Stand \today)
\fi

\end{document}


      