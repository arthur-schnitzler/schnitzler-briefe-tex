%% latex-korrekturansicht-vorspann.tex
%% Vorspann für die Korrekturansicht.
%% Lädt die gemeinsame Datei latex-vorspann.tex mit gesetztem Schalter.

\newif\ifkorrekturansicht
\korrekturansichttrue

\input{../tex-inputs/latex-vorspann}


\section[ Paul Goldmann an Arthur Schnitzler, 19. 3. {[}1904{]}]{L03441 Paul Goldmann an Arthur Schnitzler, 19. 3. {[}1904{]}}
\nopagebreak\mylabel{L03441v}
\rehead{ }\normalsize\beginnumbering\briefempfaengerindex{Schnitzler, Arthur@\textsc{Schnitzler, Arthur}!zzzGoldmann, Paul@\emph{von Paul Goldmann}!1904-03-191@{19. 3. {[}1904{]}}|(be}
\toendnotes[C]{\smallbreak\pagebreak[2]}\Standort{DLA, A:Schnitzler, HS.NZ85.1.3174.}
\physDesc{Brief, 1 Blatt, 3 Seiten, 916 Zeichen
\newline{}Handschrift: blaue Tinte, deutsche Kurrent
\newline{}Schnitzler: mit Bleistift das Jahr »904« vermerkt }\toendnotes[C]{\smallbreak}
\pstart
           \raggedleft{}{\pb}\textcolor{gray}{\textbf{DESSAUERSTRASSE 19\oindex{Dessauer Strasse@\textbf{Dessauer Straße}, \emph{Straße (K.STR)}|pw}}}\pend
           
\pstart
           Berlin\oindex{Berlin@\textbf{Berlin}, \emph{P.PPLC}|pw}, 19. März.\pend
           
\pstart\center{}Mein lieber Freund,\pend\vspace{0.5em}
\pstart
           Das \label{K_L03441-1v}\edtext{Verbot des »Reigen\pwindex{Reigen. Zehn Dialoge@\emph{Reigen. Zehn Dialoge}|pw}«}{\lemma{\textnormal{\emph{Verbot des »Reigen«}}}\Cendnote{\textnormal{Am 16. 3. 1904 war die
                     1903 im \emph{Wiener
                     Verlag}\orgindex{Wiener Verlag@Wiener Verlag|pwk} erschienene Buchausgabe des \emph{Reigen}\pwindex{Reigen. Zehn Dialoge@\emph{Reigen. Zehn Dialoge}|pwk} durch die \emph{Berliner
                     Staatsanwaltschaft}\orgindex{Staatsanwaltschaft Berlin@Staatsanwaltschaft Berlin|pwk} im ganzen Deutschen
                     Reich\oindex{Deutschland@\textbf{Deutschland}, \emph{A.PCLI}|pwk} konfisziert worden.}}}\label{K_L03441-1} durch die Berlin\oindex{Berlin@\textbf{Berlin}, \emph{P.PPLC}|pw}er Staatsanwaltſchaft\orgindex{Staatsanwaltschaft Berlin@Staatsanwaltschaft Berlin|pw} ſcheint ſich
               nun wohl leider zu beſtätigen? Ich bitte Dich, mir mitzutheilen, ob ich Dir in dieſer
               Angelegenheit irgendwie \strikeout{di} dienen kann? Du weißt,
               daß, nach deutsch\oindex{Deutschland@\textbf{Deutschland}, \emph{A.PCLI}|pwv}em Recht, auf
               jede Confiscation ein Prozeß folgen muß. Es iſt alſo dringend nöthig, daß Du oder
               Dein Verleger\pwindex{Freund, Fritz 07.04.1879 – 08.05.1950@\textsc{Freund, Fritz} (07.04.1879 – 08.05.1950), \emph{Verleger/Verlegerin}|pw}\orgindex{Wiener Verlag@Wiener Verlag|pw} einen tüchtigen Rechtsanwalt \strikeout{zur} als Berather
               nehmt, – womöglich einen, der auch ein {\pb}Wort
               politiſcher Oppoſition nicht ſcheut. Beiſpielsweiſe würde ich \label{K_L03441-2v}\edtext{\textsc{Heine\pwindex{Heine, Wolfgang 03.05.1861 – 09.05.1944@\textsc{Heine, Wolfgang} (03.05.1861 – 09.05.1944), \emph{Notar/Notarin, Politiker/Politikerin, Rechtsanwalt/Rechtsanwältin}|pw}}}{\lemma{\textnormal{\emph{Heine}}}\Cendnote{\textnormal{Heine\pwindex{Heine, Wolfgang 03.05.1861 – 09.05.1944@\textsc{Heine, Wolfgang} (03.05.1861 – 09.05.1944), \emph{Notar/Notarin, Politiker/Politikerin, Rechtsanwalt/Rechtsanwältin}|pwk} war ein Freund und Studienkollege Hermann Bahrs\pwindex{Bahr, Hermann 19.07.1863 – 15.01.1934@\textsc{Bahr, Hermann} (19.07.1863 – 15.01.1934), \emph{Schriftsteller/Schriftstellerin, Kritiker/Kritikerin}|pwk}. Neben seiner politischen
                  Tätigkeit für die \emph{SPD}\orgindex{Sozialdemokratische Partei Deutschlands (SPD)@Sozialdemokratische Partei Deutschlands (SPD)|pwk} war er als Anwalt tätig.
                  Für den \emph{Reigen}\pwindex{Reigen. Zehn Dialoge@\emph{Reigen. Zehn Dialoge}|pwk} wurde er erst
                     1921 tätig, vgl. \emph{Der Kampf um den Reigen. Vollständiger Bericht
                        über die sechstägige Verhandlung gegen Direktion und Darsteller des Kleinen
                        Schauspielhauses Berlin}\pwindex{Kampf um den Reigen. Vollstaendiger Bericht ueber die sechstaegige Verhandlung gegen Direktion und Darsteller des Kleinen Schauspielhauses Berlin@\emph{Der Kampf um den Reigen. Vollständiger Bericht über die sechstägige Verhandlung gegen Direktion und Darsteller des Kleinen Schauspielhauses Berlin}|pwk}. Herausgegeben und mit einer Einleitung von Wolfgang Heine\pwindex{Heine, Wolfgang 03.05.1861 – 09.05.1944@\textsc{Heine, Wolfgang} (03.05.1861 – 09.05.1944), \emph{Notar/Notarin, Politiker/Politikerin, Rechtsanwalt/Rechtsanwältin}|pwk}, Rechtsanwalt,
                     Staatsminister a. D. Berlin\oindex{Berlin@\textbf{Berlin}, \emph{P.PPLC}|pwk}: \emph{Rowohlt}\orgindex{Ernst Rowohlt Verlag@Ernst Rowohlt Verlag|pwk}{ }1922.}}}\label{K_L03441-2} empfehlen.\pend
           
\pstart
           Schreibe mir, ob ich igendwelche Schritte in dieſer Angelegenheit für Dich thun kann,
               – ob Du wünſcheſt, daß irgend Etwas in den Berlin\oindex{Berlin@\textbf{Berlin}, \emph{P.PPLC}|pw}er Blättern oder in der N. Fr. Pr.\pwindex{Neue Freie Presse@\emph{Neue Freie Presse}|pw}
               veröffentlicht wird?\pend
           
\pstart
           Das Verbot richtet hoffentlich keinen großen \label{K_L03441-3v}\edtext{materiellen Schaden}{\lemma{\textnormal{\emph{materiellen Schaden}}}\Cendnote{\textnormal{Das Verbot des \emph{Reigen}\pwindex{Reigen. Zehn Dialoge@\emph{Reigen. Zehn Dialoge}|pwk}
                  hatte tatsächlich den gegenteiligen Effekt und förderte den Verkauf.}}}\label{K_L03441-3} mehr an, – im Gegentheil
                  {\pb}wird es wohl, wie immer ſolche Verbote, auf das
                  Buch\pwindex{Reigen. Zehn Dialoge@\emph{Reigen. Zehn Dialoge}|pwv} erſt recht aufmerkſam
               machen.\pend
           
\pstart
           Viele herzliche Grüße! {\\[\baselineskip]}Dein {\\[\baselineskip]}\spacefill\mbox{Paul Goldm}\pend
           \leftskip=0em{}\selectlanguage{ngerman}\endnumbering\briefempfaengerindex{Schnitzler, Arthur@\textsc{Schnitzler, Arthur}!zzzGoldmann, Paul@\emph{von Paul Goldmann}!1904-03-191@{19. 3. {[}1904{]}}|)be}\mylabel{L03441h}  \normalsize

\doendnotes{C}
\bigskip
\vfill

\clearpage

\footnotesize

\lohead{\textsc{register}}

% Definiere theindex-Environment komplett neu ohne reledmac
\makeatletter
\renewenvironment{theindex}{%
  \section*{\indexname}%
  \setlength{\parindent}{0pt}%
  \setlength{\parskip}{0pt plus 0.3pt}%
  \let\item\@idxitem
}{%
  \clearpage
}
\makeatother

\IfFileExists{\jobname-pw.ind}{\input{\jobname-pw.ind}}{}

\end{document}

      