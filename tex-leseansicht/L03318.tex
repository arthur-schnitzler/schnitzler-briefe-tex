%% latex-leseansicht-vorspann.tex
%% Vorspann für die Leseansicht.
%% Lädt die gemeinsame Datei latex-vorspann.tex mit nicht gesetztem Schalter.

\newif\ifkorrekturansicht
\korrekturansichtfalse

\input{../tex-inputs/latex-vorspann}

\begin{center}
            \textcolor{red}{ENTWURF, NICHT FERTIG KORRIGIERT}
                      \end{center}
            
         
         \renewcommand{\erwaehntePersonen}{Personen: Emil Aufricht, Julian Sternberg, Franz Strosse von Hofwehr}
         \renewcommand{\erwaehnteInstitutionen}{Institutionen: Jung-Wiener Theater zum Lieben Augustin}
         \renewcommand{\erwaehnteOrte}{Orte: Bad Ischl, Berlin, Cieszyn, Hamburg, Theater an der Wien, Wien}
         \renewcommand{\erwaehnteWerke}{Werke: Der Schrei der Liebe. Novelle, Die Gedenktafel der Prinzessin Anna, Die Insel. Monatsschrift mit Buchschmuck und Illustrationen, Lieutenant Gustl. Novelle, Neue Freie Presse, Wir erhalten folgende Mittheilung: Das »Jung-Wiener Theater zum lieben Augustin«}
               \section[Felix Salten an Arthur Schnitzler, 18. 8. 1901]{ Felix Salten an Arthur Schnitzler, 18. 8. 1901}\nopagebreak\mylabel{v}\rehead{ }\begin{ledgroupsized}[t]{13cm}\normalsize\beginnumbering \toendnotes[C]{\smallbreak\pagebreak[2]} \Standort{CUL, Schnitzler, B 89, A 2.}
\physDesc{Briefkarte, 728 Zeichen
\newline{}Handschrift: Bleistift, lateinische Kurrent
\newline{}Ordnung: mit Bleistift von unbekannter Hand nummeriert:
                                    »142« }\toendnotes[C]{\smallbreak}\pstart
           \noindent{}{\pb}\textcolor{gray}{\textbf{Jung-Wiener Theater\orgindex{Jung-Wiener Theater zum Lieben Augustin@Jung-Wiener Theater zum Lieben Augustin|pw}}}\hfill \textcolor{gray}{\textbf{Wien\oindex{Wien@\textbf{Wien}|pw},}}{ }18. Aug. \textcolor{gray}{\textbf{190}}1\pend
           \pstart
           \textcolor{gray}{\textbf{Zum lieben Augustin\orgindex{Jung-Wiener Theater zum Lieben Augustin@Jung-Wiener Theater zum Lieben Augustin|pw}.}}\hfill \textcolor{gray}{\textbf{(Theater a. d.
                        Wien\oindex{Theater an der Wien@\textbf{Theater an der Wien}|pw})}}\pend
           \pstart
           \textcolor{gray}{\textbf{Direction.}}\pend
           \pstart
           Lieber Freund, herzl. Dank für Ihre verschiedenen Ansichtskarten.
               Ich war jetzt wieder eine Woche in Ischl\oindex{Bad Ischl@\textbf{Bad Ischl}|pw} und gehe
               dieser Tage nochmals hin. Im September{ }Berlin\oindex{Berlin@\textbf{Berlin}|pw}, Hamburg\oindex{Hamburg@\textbf{Hamburg}|pw}. Ein Exemplar der Insel\pwindex{Salten, Felix 06.09.1869 – 08.10.1945@\textsc{Salten, Felix} (06.09.1869 – 08.10.1945), \emph{Schriftsteller, Journalist}!Gedenktafel der Prinzessin Anna1901-07-01@\strich\emph{Die Gedenktafel der Prinzessin Anna} {[}1901-07-01{]}|pwv}\pwindex{Insel. Monatsschrift mit Buchschmuck und Illustrationen1899 – 1902@\emph{Die Insel. Monatsschrift mit Buchschmuck und Illustrationen} {[}1899 – 1902{]}|pw} kann ich Ihnen doch erst nächste Woche schicken, und da weiß ich nicht, ob's
               noch dafür steht. Geben Sie mir, wenn's noch sein kann, directe Adreße an, damit es
               keinen {\pb}solchen Umweg macht. Was
               sagen Sie, in welch' verschämter Weise st-g\pwindex{Sternberg, Julian 08.11.1868 – 28.06. 1945@\textsc{Sternberg, Julian} (08.11.1868 – 28.06. 1945), \emph{Journalist}|pw} mir
                  \label{K_L03318-1v}\edtext{Reclame\pwindex{Wir erhalten folgende Mittheilung: Das »Jung-Wiener Theater zum lieben
                  Augustin«1901-08-18@\emph{Wir erhalten folgende Mittheilung: Das »Jung-Wiener Theater zum lieben Augustin«} {[}1901-08-18{]}|pwv}}{\lemma{\textnormal{\emph{Reclame}}}\Cendnote{\textnormal{[Julian Sternberg\pwindex{Sternberg, Julian 08.11.1868 – 28.06. 1945@\textsc{Sternberg, Julian} (08.11.1868 – 28.06. 1945), \emph{Journalist}|pwk}]: \emph{Wir erhalten folgende Mittheilung: Das »Jung-Wiener Theater
                        zum lieben Augustin«}\pwindex{Wir erhalten folgende Mittheilung: Das »Jung-Wiener Theater zum lieben
                  Augustin«1901-08-18@\emph{Wir erhalten folgende Mittheilung: Das »Jung-Wiener Theater zum lieben Augustin«} {[}1901-08-18{]}|pwk}. In: \emph{Neue Freie
                        Presse}\pwindex{Neue Freie Presse1864 – 1939@\emph{Neue Freie Presse} {[}1864 – 1939{]}|pwk}, Nr. 13.283, 18. 8. 1901, Morgenblatt,
                  S. 9.}}}\label{K_L03318-1h} gemacht hat? Heuer scheint’s im Sommer nur lauter Lieutenant Gustl\pwindex{Schnitzler, Arthur 15.05.1862 – 21.10.1931@\textsc{Schnitzler, Arthur} (15.05.1862 – 21.10.1931), \emph{Schriftsteller, Mediziner}!Lieutenant Gustl. Novelle1900-12-25@\strich\emph{Lieutenant Gustl. Novelle} {[}1900-12-25{]}|pwv}’s zu geben –
                  (\label{K_L03318-43v}\edtext{Teschen\oindex{Cieszyn@\textbf{Cieszyn}|pw}}{\lemma{\textnormal{\emph{Teschen}}}\Cendnote{\textnormal{Hier war im Juli der Bäckermeister Emil Aufricht\pwindex{Aufricht, Emil *~um 1874@\textsc{Aufricht, Emil} (*~um 1874), \emph{Bäcker}|pwk} vom Lieutenant Franz Strosse, Edler von Hochwehr\pwindex{Strosse von Hofwehr, Franz 08.07.1877 – 15.12.1965@\textsc{Strosse von Hofwehr, Franz} (08.07.1877 – 15.12.1965), \emph{Militär}|pwk} als
                  »Saujud« beschimpft worden. Dieser nannte nun den anderen entweder unmittelbar
                  oder im Gespräch mit Dritten »Lausbub«, woraufhin Strosse\pwindex{Strosse von Hofwehr, Franz 08.07.1877 – 15.12.1965@\textsc{Strosse von Hofwehr, Franz} (08.07.1877 – 15.12.1965), \emph{Militär}|pwk} mit Gefährten dem Bäcker auflauerten und ihn
                  verprügelten, so dass er schwere Kopfverletzungen erlitt und ihm vier Finger
                  amputiert werden mussten.}}}\label{K_L03318-43h} ec.) Neues gibts genug, aber es wär’ zu
               weitläufig. Leben Sie herzlich wol, hoffentlich auf baldiges Wiedersehen.\pend
           \pstart
           Ihr {\\[\baselineskip]}\spacefill\mbox{Salten}\pend
           \leftskip=0em{}\pstart
           \noindent{}Ich schreibe eine Geschichte\pwindex{Salten, Felix 06.09.1869 – 08.10.1945@\textsc{Salten, Felix} (06.09.1869 – 08.10.1945), \emph{Schriftsteller, Journalist}!Schrei der Liebe. Novelle1904-10-22@\strich\emph{Der Schrei der Liebe. Novelle} {[}1904-10-22{]}|pwuv}, die hoffentl. besser ist als die Prinzessin Anna\pwindex{Salten, Felix 06.09.1869 – 08.10.1945@\textsc{Salten, Felix} (06.09.1869 – 08.10.1945), \emph{Schriftsteller, Journalist}!Gedenktafel der Prinzessin Anna1901-07-01@\strich\emph{Die Gedenktafel der Prinzessin Anna} {[}1901-07-01{]}|pw}. \pend
           
         
         \endnumbering\mylabel{h}\end{ledgroupsized}\begin{anhang}\end{anhang}\newcommand{\dateiname}{L03318}\newcommand{\titel}{Felix Salten an Arthur Schnitzler, 18. 8. 1901}\newcommand{\editorInnen}{Martin Anton Müller und Laura Untner}%% latex-leseansicht-abspann.tex
%% Abspann für die Leseansicht.
%% Der Schalter \ifkorrekturansicht ist bereits durch den Vorspann gesetzt.

%% latex-abspann.tex
%% Gemeinsamer Abspann für Korrekturansicht und Leseansicht.
%% Setzt den Schalter \ifkorrekturansicht voraus (gesetzt in den
%% einbindenden Dateien latex-korrekturansicht-abspann.tex bzw.
%% latex-leseansicht-abspann.tex).
%% ---------------------------------------------------------------

\normalsize

% Das esempio-Environment wird nur in der Leseansicht benötigt
\ifkorrekturansicht\else
\newenvironment{esempio}[3]%
{
    \vspace{1.5ex}
    \rlap{\underline{#1}}
    \par
    \setlength{\parindent}{0cm}
    \nopagebreak
    \leftskip=#2cm
    \rightskip=#3cm
}
{
    \par
}
\fi

\doendnotes{C}
\bigskip
\vfill

\clearpage

\footnotesize

\ifkorrekturansicht
  \lohead{\textsc{register}}
\fi

% theindex-Environment neu definieren ohne reledmac
\makeatletter
\renewenvironment{theindex}{%
  \ifkorrekturansicht
    \section*{\indexname}%
  \else
    \subsubsection*{Index der erwähnten Entitäten}%
  \fi
  \setlength{\parindent}{0pt}%
  \setlength{\parskip}{0pt plus 0.3pt}%
  \let\item\@idxitem
}{%
  \ifkorrekturansicht\clearpage\fi
}
\makeatother

\IfFileExists{\jobname-pw.ind}{\input{\jobname-pw.ind}}{}

% Quellenangabe nur in der Leseansicht
\ifkorrekturansicht\else
% Fallback-Definitionen, falls die .tex-Datei \titel etc. nicht gesetzt hat
\providecommand{\titel}{}
\providecommand{\editorInnen}{}
\providecommand{\dateiname}{\jobname}

\vspace{3cm}

\vfill

\footnotesize
\textsc{Quelle}: \titel. Herausgegeben von {\editorInnen}. In: \emph{Arthur Schnitzler: Briefwechsel mit Autorinnen und Autoren}.
 Digitale Edition, https://schnitzler-briefe.acdh.oeaw.ac.at/{\dateiname}.html (Stand \today)
\fi

\end{document}


      