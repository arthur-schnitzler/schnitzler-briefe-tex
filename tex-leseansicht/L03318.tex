%% latex-leseansicht-vorspann.tex
%% Vorspann für die Leseansicht.
%% Lädt die gemeinsame Datei latex-vorspann.tex mit nicht gesetztem Schalter.

\newif\ifkorrekturansicht
\korrekturansichtfalse

\input{../tex-inputs/latex-vorspann}


\section[ Felix Salten an Arthur Schnitzler, 18. 8. 1901]{L03318 Felix Salten an Arthur Schnitzler,  18. 8. 1901}
\nopagebreak\mylabel{L03318v}
\rehead{ }\normalsize\beginnumbering\briefempfaengerindex{Schnitzler, Arthur@\textsc{Schnitzler, Arthur}!zzzSalten, Felix@\emph{von Felix Salten}!1901-08-181@{18. 8. 1901}|(be}
\toendnotes[C]{\smallbreak\pagebreak[2]}
\correspDesc{Versand  durch Felix Salten am 18. 8. 1901 in Wien
\newline{}Erhalt  durch Arthur Schnitzler im Zeitraum [19. 8. 1901
                  – 23. 8. 1901?] in Welsberg-Taisten}\toendnotes[C]{\smallbreak}
\Standort{CUL, Schnitzler, B 89, A 2.}
\physDesc{Briefkarte, 729 Zeichen
\newline{}Handschrift: Bleistift, lateinische Kurrent
\newline{}Ordnung: mit Bleistift von unbekannter Hand nummeriert: »142« }\toendnotes[C]{\smallbreak}
\pstart
           {\pb}\textcolor{gray}{\textbf{Jung-Wiener Theater\orgindex{Jung-Wiener Theater zum Lieben Augustin@Jung-Wiener Theater zum Lieben Augustin|pw}}}\hfill \textcolor{gray}{\textbf{Wien\oindex{Wien@\textbf{Wien}, \emph{Verwaltungsgebiet}|pw},}}{ }18. Aug. \textcolor{gray}{\textbf{190}}1\pend
           
\pstart
           \textcolor{gray}{\textbf{Zum lieben Augustin\orgindex{Jung-Wiener Theater zum Lieben Augustin@Jung-Wiener Theater zum Lieben Augustin|pw}.}}\hfill \textcolor{gray}{\textbf{(Theater a. d.
                        Wien\oindex{Wien@\textbf{Wien}!VI., Mariahilf@\textbf{VI., Mariahilf}!Theater an der Wien@\textbf{Theater an der Wien}, \emph{Theater}|pw})}}\pend
           
\pstart
           \textcolor{gray}{\textbf{Direction.}}\pend
           \vspace{0.5em}
\pstart
           Lieber Freund, herzl. Dank für Ihre verschiedenen Ansichtskarten.
               Ich war jetzt wieder eine Woche in Ischl\oindex{Bad Ischl@\textbf{Bad Ischl}|pw} und gehe
               dieser Tage nochmals hin. Im September{ }Berlin\oindex{Berlin@\textbf{Berlin}, \emph{Hauptstadt}|pw}{ }{\kaufmannsund}{ }Hamburg\oindex{Hamburg@\textbf{Hamburg}|pw}. Ein Exemplar der \label{K_L03318-1v}\edtext{Insel\pwindex{Salten, Felix 6.\,9.\,1869 Budapest – 8.\,10.\,1945 Zürich@\textsc{Salten, Felix} (6.\,9.\,1869 Budapest – 8.\,10.\,1945 Zürich), \emph{Schriftsteller, Journalist, Chefredakteur}!Gedenktafel der Prinzessin Anna@\strich\emph{Die Gedenktafel der Prinzessin Anna}|pwv}\pwindex{Insel. Monatsschrift mit Buchschmuck und Illustrationen@\emph{Die Insel. Monatsschrift mit Buchschmuck und Illustrationen}|pw}}{\lemma{\textnormal{\emph{Insel}}}\Cendnote{\textnormal{Siehe XXXX Auszeichnungsfehler: Dokument L03316 nicht gefunden. Die Übermittlung
                  dürfte erst in Wien\oindex{Wien@\textbf{Wien}, \emph{Verwaltungsgebiet}|pwk} erfolgt sein, am XXXX Auszeichnungsfehler: Dokument L02971 nicht gefunden retournierte Schnitzler das Heft.}}}\label{K_L03318-1} kann ich Ihnen
               doch erst nächste Woche schicken, und da weiß ich nicht, ob’s noch dafürsteht. Geben
               Sie mir, wenn’s noch sein kann, directe Adreße an, damit es keinen {\pb}solchen Umweg macht. Was sagen
               Sie, in welch’ verschämter Weise st-g\pwindex{Sternberg, Julian 8.\,11.\,1868 Wien – 28.\,6.\,1945 Havanna@\textsc{Sternberg, Julian} (8.\,11.\,1868 Wien – 28.\,6.\,1945 Havanna), \emph{Journalist}|pw} mir
                  \label{K_L03318-2v}\edtext{Reclame\pwindex{Wir erhalten folgende Mittheilung: Das »Jung-Wiener Theater zum lieben Augustin«@\emph{Wir erhalten folgende Mittheilung: Das »Jung-Wiener Theater zum lieben Augustin«}|pwv}}{\lemma{\textnormal{\emph{Reclame}}}\Cendnote{\textnormal{[Julian Sternberg\pwindex{Sternberg, Julian 8.\,11.\,1868 Wien – 28.\,6.\,1945 Havanna@\textsc{Sternberg, Julian} (8.\,11.\,1868 Wien – 28.\,6.\,1945 Havanna), \emph{Journalist}|pwk}]: \emph{Wir erhalten folgende Mittheilung: Das »Jung-Wiener Theater
                        zum lieben Augustin«}\pwindex{Wir erhalten folgende Mittheilung: Das »Jung-Wiener Theater zum lieben Augustin«@\emph{Wir erhalten folgende Mittheilung: Das »Jung-Wiener Theater zum lieben Augustin«}|pwk}. In: \emph{Neue Freie
                        Presse}\pwindex{Neue Freie Presse@\emph{Neue Freie Presse}|pwk}, Nr. 13.283, 18. 8. 1901,
                     Morgenblatt, S. 9.}}}\label{K_L03318-2} gemacht hat? Heuer scheint’s im Sommer nur
               lauter Lieutenant Gustl\pwindex{Schnitzler, Arthur 15.\,5.\,1862 Wien – 21.\,10.\,1931 ebd.@\textsc{Schnitzler, Arthur} (15.\,5.\,1862 Wien – 21.\,10.\,1931 ebd.), \emph{Schriftsteller, Mediziner}!Lieutenant Gustl. Novelle@\strich\emph{Lieutenant Gustl. Novelle}|pwv}’s zu
               geben – (\label{K_L03318-3v}\edtext{Teschen\oindex{Cieszyn@\textbf{Cieszyn}, \emph{Hauptstadt}|pw}}{\lemma{\textnormal{\emph{Teschen}}}\Cendnote{\textnormal{In Teschen\oindex{Cieszyn@\textbf{Cieszyn}, \emph{Hauptstadt}|pwk} war im Juli der Bäckermeister Emil Aufricht\pwindex{Aufricht, Emil *~um 1874@\textsc{Aufricht, Emil} (*~um 1874), \emph{Bäcker}|pwk} von Lieutenant Franz Strosse, Edler von Hochwehr\pwindex{Strosse von Hofwehr, Franz 8.\,7.\,1877 Bochnia – 15.\,12.\,1965 Klosterneuburg@\textsc{Strosse von Hofwehr, Franz} (8.\,7.\,1877 Bochnia – 15.\,12.\,1965 Klosterneuburg), \emph{Militär}|pwk}, als
                  »Saujud« beschimpft worden. Dieser nannte folglich den anderen entweder
                  unmittelbar oder im Gespräch mit Dritten »Lausbub«. Daraufhin lauerte Strosse\pwindex{Strosse von Hofwehr, Franz 8.\,7.\,1877 Bochnia – 15.\,12.\,1965 Klosterneuburg@\textsc{Strosse von Hofwehr, Franz} (8.\,7.\,1877 Bochnia – 15.\,12.\,1965 Klosterneuburg), \emph{Militär}|pwk} mit Gefährten dem Bäcker auf. Sie
                  verprügelten ihn, er erlitt schwere Kopfverletzungen und ihm mussten vier Finger
                  amputiert werden.}}}\label{K_L03318-3} ec.) Neues gibts genug, aber es wär’ zu weitläufig. Leben
               Sie herzlich wol, hoffentlich auf baldiges Wiedersehen.\pend
           
\pstart
           Ihr {\\[\baselineskip]}\spacefill\mbox{Salten}\pend
           \leftskip=0em{}
\pstart
           \noindent{}Ich schreibe eine \label{K_L03318-4v}\edtext{Geschichte\pwindex{Salten, Felix 6.\,9.\,1869 Budapest – 8.\,10.\,1945 Zürich@\textsc{Salten, Felix} (6.\,9.\,1869 Budapest – 8.\,10.\,1945 Zürich), \emph{Schriftsteller, Journalist, Chefredakteur}!Schrei der Liebe. Novelle@\strich\emph{Der Schrei der Liebe. Novelle}|pwuv}}{\lemma{\textnormal{\emph{Geschichte}}}\Cendnote{\textnormal{Möglicherweise arbeitete er an 
                     \emph{Der Schrei der Liebe}\pwindex{Salten, Felix 6.\,9.\,1869 Budapest – 8.\,10.\,1945 Zürich@\textsc{Salten, Felix} (6.\,9.\,1869 Budapest – 8.\,10.\,1945 Zürich), \emph{Schriftsteller, Journalist, Chefredakteur}!Schrei der Liebe. Novelle@\strich\emph{Der Schrei der Liebe. Novelle}|pwk} oder an dem nicht näher
                     bestimmbaren Text \emph{Empfängnis}\pwindex{Salten, Felix 6.\,9.\,1869 Budapest – 8.\,10.\,1945 Zürich@\textsc{Salten, Felix} (6.\,9.\,1869 Budapest – 8.\,10.\,1945 Zürich), \emph{Schriftsteller, Journalist, Chefredakteur}!Empfängnis@\strich\emph{Empfängnis}|pwk}, den Salten\pwindex{Salten, Felix 6.\,9.\,1869 Budapest – 8.\,10.\,1945 Zürich@\textsc{Salten, Felix} (6.\,9.\,1869 Budapest – 8.\,10.\,1945 Zürich), \emph{Schriftsteller, Journalist, Chefredakteur}|pwk}{ }Schnitzler am 24. 3. 1902
                     vorlas.}}}\label{K_L03318-4}, die hoffentl. besser ist als die Prinzessin Anna\pwindex{Salten, Felix 6.\,9.\,1869 Budapest – 8.\,10.\,1945 Zürich@\textsc{Salten, Felix} (6.\,9.\,1869 Budapest – 8.\,10.\,1945 Zürich), \emph{Schriftsteller, Journalist, Chefredakteur}!Gedenktafel der Prinzessin Anna@\strich\emph{Die Gedenktafel der Prinzessin Anna}|pw}.\pend
           \selectlanguage{ngerman}\endnumbering\briefempfaengerindex{Schnitzler, Arthur@\textsc{Schnitzler, Arthur}!zzzSalten, Felix@\emph{von Felix Salten}!1901-08-181@{18. 8. 1901}|)be}\mylabel{L03318h}  \newcommand{\dateiname}{L03318}\newcommand{\titel}{Felix Salten an Arthur Schnitzler, 18. 8. 1901}\newcommand{\editorInnen}{Martin Anton Müller und Laura Untner}%% latex-leseansicht-abspann.tex
%% Abspann für die Leseansicht.
%% Der Schalter \ifkorrekturansicht ist bereits durch den Vorspann gesetzt.

%% latex-abspann.tex
%% Gemeinsamer Abspann für Korrekturansicht und Leseansicht.
%% Setzt den Schalter \ifkorrekturansicht voraus (gesetzt in den
%% einbindenden Dateien latex-korrekturansicht-abspann.tex bzw.
%% latex-leseansicht-abspann.tex).
%% ---------------------------------------------------------------

\normalsize

% Das esempio-Environment wird nur in der Leseansicht benötigt
\ifkorrekturansicht\else
\newenvironment{esempio}[3]%
{
    \vspace{1.5ex}
    \rlap{\underline{#1}}
    \par
    \setlength{\parindent}{0cm}
    \nopagebreak
    \leftskip=#2cm
    \rightskip=#3cm
}
{
    \par
}
\fi

\doendnotes{C}
\bigskip
\vfill

\clearpage

\footnotesize

\ifkorrekturansicht
  \lohead{\textsc{register}}
\fi

% theindex-Environment neu definieren ohne reledmac
\makeatletter
\renewenvironment{theindex}{%
  \ifkorrekturansicht
    \section*{\indexname}%
  \else
    \subsubsection*{Index der erwähnten Entitäten}%
  \fi
  \setlength{\parindent}{0pt}%
  \setlength{\parskip}{0pt plus 0.3pt}%
  \let\item\@idxitem
}{%
  \ifkorrekturansicht\clearpage\fi
}
\makeatother

\IfFileExists{\jobname-pw.ind}{\input{\jobname-pw.ind}}{}

% Quellenangabe nur in der Leseansicht
\ifkorrekturansicht\else
% Fallback-Definitionen, falls die .tex-Datei \titel etc. nicht gesetzt hat
\providecommand{\titel}{}
\providecommand{\editorInnen}{}
\providecommand{\dateiname}{\jobname}

\vspace{3cm}

\vfill

\footnotesize
\textsc{Quelle}: \titel. Herausgegeben von {\editorInnen}. In: \emph{Arthur Schnitzler: Briefwechsel mit Autorinnen und Autoren}.
 Digitale Edition, https://schnitzler-briefe.acdh.oeaw.ac.at/{\dateiname}.html (Stand \today)
\fi

\end{document}


