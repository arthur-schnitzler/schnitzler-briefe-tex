%% latex-leseansicht-vorspann.tex
%% Vorspann für die Leseansicht.
%% Lädt die gemeinsame Datei latex-vorspann.tex mit nicht gesetztem Schalter.

\newif\ifkorrekturansicht
\korrekturansichtfalse

\input{../tex-inputs/latex-vorspann}


\section[Robert Adam an Arthur Schnitzler, 22. 6. 1915]{L02209 Robert Adam an Arthur Schnitzler, 22. 6. 1915}
\nopagebreak\mylabel{L02209v}
\rehead{ }\normalsize\beginnumbering\briefempfaengerindex{Schnitzler, Arthur@\textsc{Schnitzler, Arthur}!zzzAdam, Robert@\emph{von Robert Adam}!1915-06-221@{22. 6. 1915}|(be}
\toendnotes[C]{\smallbreak\pagebreak[2]}
\correspDesc{Versand  durch Robert Adam am 22. 6. 1915 in Zistersdorf
\newline{}Erhalt  durch Arthur Schnitzler im Zeitraum [23. 6. 1915
                  – 27. 6. 1915?] in Wien}\toendnotes[C]{\smallbreak}
\Standort{DLA, A:Schnitzler, HS.NZ85.1.4230,9.}
\physDesc{Brief, 1 Blatt, 2 Seiten, 1667 Zeichen
\newline{}Handschrift: schwarze Tinte, deutsche Kurrent
\newline{}Schnitzler: 1) mit Bleistift beschriftet: »\textsc{Adam}«  2) mit rotem Buntstift eine Unterstreichung}\toendnotes[C]{\smallbreak}
\pstart
           \raggedleft{}{\pb}Ziſtersdorf\oindex{Zistersdorf@\textbf{Zistersdorf}, \emph{Verwaltungsgebiet}|pw}, 22. Juni 1915.\pend
           
\pstart{}Hochverehrter Herr Doktor!\pend\vspace{0.5em}
\pstart
           Ich kann Ihnen anzeigen, daß es mir nach längerer Beratung mit unſerem Poſtmeiſter\pwindex{?? [Postmeister in Zistersdorf] *~1915@\textsc{?? [Postmeister in Zistersdorf]} (*~1915)|pwv}, der über den
               Kriegspoſtverkehr mit den Verbündeten nicht viel beſſer informiert zu{ }ſein{ }ſcheint
               als ich, gelungen iſt, das Manuſkript des »\textsc{Fremden}\pwindex{Adam, Robert 20.\,4.\,1877 Wien – 16.\,10.\,1961 Baden bei Wien@\textsc{Adam, Robert} (20.\,4.\,1877 Wien – 16.\,10.\,1961 Baden bei Wien), \emph{Schriftsteller, Richter}!Fremde@\strich\emph{Der Fremde}|pw}« mit einem Briefe an den Fiſcherſchen
                  Verlag\orgindex{S. Fischer Verlag@S. Fischer Verlag|pw} zu{ }ſenden, und ich gebe mich der Hoffnung hin, daß beides den
               Beſtimmungsort erreicht.\pend
           
\pstart
           Zugleich erlaube ich mir, Ihnen das Manuſkript der Komödie: »Geſellſchaft\pwindex{Adam, Robert 20.\,4.\,1877 Wien – 16.\,10.\,1961 Baden bei Wien@\textsc{Adam, Robert} (20.\,4.\,1877 Wien – 16.\,10.\,1961 Baden bei Wien), \emph{Schriftsteller, Richter}!Gesellschaft [Eine Gaunerkomödie]@\strich\emph{Gesellschaft [Eine Gaunerkomödie]}|pw}« zu{ }ſchicken, die, wie ich Ihnen erzählte, vom »Deutſchen Volkstheater\oindex{Wien@\textbf{Wien}!VII., Neubau@\textbf{VII., Neubau}!Volkstheater@\textbf{Volkstheater}, \emph{Theater}|pw}« abgelehnt wurde. Ein
               Meiſterwerk iſt{ }ſie ja gewiß nicht, obwohl ich meinen möchte, daß{ }ſie, vom
               techniſchen Geſichtspunkt aus betrachtet, einem gelernten »Dramaturgen« Freudentränen
               entlocken könnte. Aber \strikeout{iſ}{ }ſie iſt {\pb}wohl
               vergnüglich; allerdings kann ich dieſe ihre Eigenſchaft{ }ſelbſt nicht objektiv
               einſchätzen, aber ich{ }ſchließe es daraus, daß ich{ }ſie mit derſelben Behaglichkeit
               niederſchrieb, die den alten Dumas\pwindex{Dumas, Alexandre père 24.\,7.\,1802 Villers-Cotterêts – 5.\,12.\,1870 Puys@\textsc{Dumas, Alexandre père} (24.\,7.\,1802 Villers-Cotterêts – 5.\,12.\,1870 Puys), \emph{Schriftsteller}|pw} beim
               Verfaſſen{ }ſeiner heitern Romane hell auflachen ließ. Wenn die Erlebniſſe meiner
               Helden, die ich zum größten Teil perſönlich kennen lernen durfte – den Daniel
               Rubinſtein{ }ſchilderten mir nur Perſonen, die er mit{ }ſeiner intereſſanten
               Bekanntſchaft beehrt hatte –, Sie auch nur ein wenig erheitern, wird es mich
               außerordentlich freuen. Eigentlich habe ich doch die Hoffnung noch nicht ganz
               aufgegeben, dieſe Komödie bei einer Bühne anzubringen (allenfalls nach einigen
               Verbeſſerungen); denn ich glaube, daß{ }ſie eine ganze Anzahl »guter Rollen«
               enthält.\pend
           
\pstart
           Indem ich Ihnen, hochverehrter Herr Doktor, für Ihre große Liebenswürdigkeit nochmals
               herzlich danke, verbleibe ich Ihr{ }ſehr ergebener\pend
           \pstart \spacefill\mbox{Robert Adam}\pend{}\selectlanguage{ngerman}\endnumbering\briefempfaengerindex{Schnitzler, Arthur@\textsc{Schnitzler, Arthur}!zzzAdam, Robert@\emph{von Robert Adam}!1915-06-221@{22. 6. 1915}|)be}\mylabel{L02209h}  \newcommand{\dateiname}{L02209}\newcommand{\titel}{Robert Adam an Arthur Schnitzler, 22. 6. 1915}\newcommand{\editorInnen}{Martin Anton Müller und Gerd-Hermann Susen}%% latex-leseansicht-abspann.tex
%% Abspann für die Leseansicht.
%% Der Schalter \ifkorrekturansicht ist bereits durch den Vorspann gesetzt.

%% latex-abspann.tex
%% Gemeinsamer Abspann für Korrekturansicht und Leseansicht.
%% Setzt den Schalter \ifkorrekturansicht voraus (gesetzt in den
%% einbindenden Dateien latex-korrekturansicht-abspann.tex bzw.
%% latex-leseansicht-abspann.tex).
%% ---------------------------------------------------------------

\normalsize

% Das esempio-Environment wird nur in der Leseansicht benötigt
\ifkorrekturansicht\else
\newenvironment{esempio}[3]%
{
    \vspace{1.5ex}
    \rlap{\underline{#1}}
    \par
    \setlength{\parindent}{0cm}
    \nopagebreak
    \leftskip=#2cm
    \rightskip=#3cm
}
{
    \par
}
\fi

\doendnotes{C}
\bigskip
\vfill

\clearpage

\footnotesize

\ifkorrekturansicht
  \lohead{\textsc{register}}
\fi

% theindex-Environment neu definieren ohne reledmac
\makeatletter
\renewenvironment{theindex}{%
  \ifkorrekturansicht
    \section*{\indexname}%
  \else
    \subsubsection*{Index der erwähnten Entitäten}%
  \fi
  \setlength{\parindent}{0pt}%
  \setlength{\parskip}{0pt plus 0.3pt}%
  \let\item\@idxitem
}{%
  \ifkorrekturansicht\clearpage\fi
}
\makeatother

\IfFileExists{\jobname-pw.ind}{\input{\jobname-pw.ind}}{}

% Quellenangabe nur in der Leseansicht
\ifkorrekturansicht\else
% Fallback-Definitionen, falls die .tex-Datei \titel etc. nicht gesetzt hat
\providecommand{\titel}{}
\providecommand{\editorInnen}{}
\providecommand{\dateiname}{\jobname}

\vspace{3cm}

\vfill

\footnotesize
\textsc{Quelle}: \titel. Herausgegeben von {\editorInnen}. In: \emph{Arthur Schnitzler: Briefwechsel mit Autorinnen und Autoren}.
 Digitale Edition, https://schnitzler-briefe.acdh.oeaw.ac.at/{\dateiname}.html (Stand \today)
\fi

\end{document}


