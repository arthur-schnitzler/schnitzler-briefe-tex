%% latex-leseansicht-vorspann.tex
%% Vorspann für die Leseansicht.
%% Lädt die gemeinsame Datei latex-vorspann.tex mit nicht gesetztem Schalter.

\newif\ifkorrekturansicht
\korrekturansichtfalse

\input{../tex-inputs/latex-vorspann}


\section[Adalbert Seligmann an Arthur Schnitzler, 30. 9. [1902?]]{L01236 Adalbert Seligmann an Arthur Schnitzler, 30. 9. [1902?]}
\nopagebreak\mylabel{L01236v}
\rehead{ }\normalsize\beginnumbering\briefempfaengerindex{Schnitzler, Arthur@\textsc{Schnitzler, Arthur}!zzzSeligmann, Adalbert Franz@\emph{von Adalbert Franz Seligmann}!1902-09-303@{30. 9. [1902?]}|(be}
\toendnotes[C]{\smallbreak\pagebreak[2]}
\correspDesc{Versand  durch Adalbert Seligmann am 30. 9. [1902?] in Wien
\newline{}Erhalt  durch Arthur Schnitzler im Zeitraum [30. 9. 1902
                  – 4. 10. 1902?] in Wien}\toendnotes[C]{\smallbreak}
\Standort{CUL, Schnitzler, B 97.}
\physDesc{Briefkarte, 535 Zeichen
\newline{}Handschrift: schwarze Tinte, deutsche Kurrent}\toendnotes[C]{\smallbreak}
\pstart
           \noindent{}{\pb}Verehrter Freund! Ueberbringer dieſes, ein unverſchuldet in Not
               geratener \label{K_L01236-1v}\edtext{Schriftſteller\pwindex{Ferency @\textsc{Ferency}, \emph{Schriftsteller}|pwuv}}{\lemma{\textnormal{\emph{Schriftsteller}}}\Cendnote{\textnormal{Der Karte fehlt die Jahresangabe. Sofern
                  die Person im \emph{Tagebuch}\pwindex{Schnitzler, Arthur 15.\,5.\,1862 Wien – 21.\,10.\,1931 ebd.@\textsc{Schnitzler, Arthur} (15.\,5.\,1862 Wien – 21.\,10.\,1931 ebd.), \emph{Schriftsteller, Mediziner}!Tagebuch@\strich\emph{Tagebuch}|pwk} erwähnt ist, könnte es
                  sich um einen nicht näher bestimmten Ferency\pwindex{Ferency @\textsc{Ferency}, \emph{Schriftsteller}|pwk} handeln, der Schnitzler am
                     30. 9. 1902
                  besucht hat.}}}\label{K_L01236-1}, von \textsc{Jacobsen}\pwindex{Jacobsohn, Siegfried 28.\,1.\,1881 Berlin – 3.\,12.\,1926 ebd.@\textsc{Jacobsohn, Siegfried} (28.\,1.\,1881 Berlin – 3.\,12.\,1926 ebd.), \emph{Journalist, Kritiker, Publizist}|pw} (Berlin\oindex{Berlin@\textbf{Berlin}, \emph{Hauptstadt}|pw}) \textsc{Polgar}\pwindex{Polgar, Alfred 17.\,10.\,1873 Wien – 24.\,4.\,1955 Zürich@\textsc{Polgar, Alfred} (17.\,10.\,1873 Wien – 24.\,4.\,1955 Zürich), \emph{Schriftsteller, Journalist, Kritiker}|pw} u. \textsc{Glücksmann}\pwindex{Glücksmann, Heinrich 8.\,7.\,1864 Rakšice – 1.\,3.\,1943 Buenos Aires@\textsc{Glücksmann, Heinrich} (8.\,7.\,1864 Rakšice – 1.\,3.\,1943 Buenos Aires), \emph{Schriftsteller, Journalist, Dramaturg}|pw} warm empfohlen, erſucht mich um einige Worte an einen München\oindex{München@\textbf{München}|pw}er Verlag. Da ich aber dort keine Beziehungen habe,
               wäre es Ihnen vielleicht möglich, ihm ein {\pb}paar Zeilen mitzugeben. Es handelt{ }ſich ihm nur darum, daß{ }ſeine Sachen in dem
               betreffenden Verlag bald geleſen werden u. er in kurzer Zeit einen zuſagenden oder
               ablehnenden Beſcheid erhält. Verzeihen Sie die Beläſtigung.\pend
           \pstart Ihr ergebenſter\spacefill\mbox{A. F. Seligmann}\pend{}
\pstart
           30/IX.\pend
           \selectlanguage{ngerman}\endnumbering\briefempfaengerindex{Schnitzler, Arthur@\textsc{Schnitzler, Arthur}!zzzSeligmann, Adalbert Franz@\emph{von Adalbert Franz Seligmann}!1902-09-303@{30. 9. [1902?]}|)be}\mylabel{L01236h}  \newcommand{\dateiname}{L01236}\newcommand{\titel}{Adalbert Seligmann an Arthur Schnitzler, 30. 9. [1902?]}\newcommand{\editorInnen}{Martin Anton Müller und Gerd-Hermann Susen}%% latex-leseansicht-abspann.tex
%% Abspann für die Leseansicht.
%% Der Schalter \ifkorrekturansicht ist bereits durch den Vorspann gesetzt.

%% latex-abspann.tex
%% Gemeinsamer Abspann für Korrekturansicht und Leseansicht.
%% Setzt den Schalter \ifkorrekturansicht voraus (gesetzt in den
%% einbindenden Dateien latex-korrekturansicht-abspann.tex bzw.
%% latex-leseansicht-abspann.tex).
%% ---------------------------------------------------------------

\normalsize

% Das esempio-Environment wird nur in der Leseansicht benötigt
\ifkorrekturansicht\else
\newenvironment{esempio}[3]%
{
    \vspace{1.5ex}
    \rlap{\underline{#1}}
    \par
    \setlength{\parindent}{0cm}
    \nopagebreak
    \leftskip=#2cm
    \rightskip=#3cm
}
{
    \par
}
\fi

\doendnotes{C}
\bigskip
\vfill

\clearpage

\footnotesize

\ifkorrekturansicht
  \lohead{\textsc{register}}
\fi

% theindex-Environment neu definieren ohne reledmac
\makeatletter
\renewenvironment{theindex}{%
  \ifkorrekturansicht
    \section*{\indexname}%
  \else
    \subsubsection*{Index der erwähnten Entitäten}%
  \fi
  \setlength{\parindent}{0pt}%
  \setlength{\parskip}{0pt plus 0.3pt}%
  \let\item\@idxitem
}{%
  \ifkorrekturansicht\clearpage\fi
}
\makeatother

\IfFileExists{\jobname-pw.ind}{\input{\jobname-pw.ind}}{}

% Quellenangabe nur in der Leseansicht
\ifkorrekturansicht\else
% Fallback-Definitionen, falls die .tex-Datei \titel etc. nicht gesetzt hat
\providecommand{\titel}{}
\providecommand{\editorInnen}{}
\providecommand{\dateiname}{\jobname}

\vspace{3cm}

\vfill

\footnotesize
\textsc{Quelle}: \titel. Herausgegeben von {\editorInnen}. In: \emph{Arthur Schnitzler: Briefwechsel mit Autorinnen und Autoren}.
 Digitale Edition, https://schnitzler-briefe.acdh.oeaw.ac.at/{\dateiname}.html (Stand \today)
\fi

\end{document}


