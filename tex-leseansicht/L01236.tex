%% latex-korrekturansicht-vorspann.tex
%% Vorspann für die Korrekturansicht.
%% Lädt die gemeinsame Datei latex-vorspann.tex mit gesetztem Schalter.

\newif\ifkorrekturansicht
\korrekturansichttrue

\input{../tex-inputs/latex-vorspann}


\section[Adalbert Seligmann an Arthur Schnitzler, 30. 9. {[}1902?{]}]{L01236 Adalbert Seligmann an Arthur Schnitzler, 30. 9. {[}1902?{]}}
\nopagebreak\mylabel{L01236v}
\rehead{ }\normalsize\beginnumbering\briefempfaengerindex{Schnitzler, Arthur@\textsc{Schnitzler, Arthur}!zzzSeligmann, Adalbert Franz@\emph{von Adalbert Franz Seligmann}!1902-09-303@{30. 9. {[}1902?{]}}|(be}
\toendnotes[C]{\smallbreak\pagebreak[2]}\Standort{CUL, Schnitzler, B 97.}
\physDesc{Briefkarte, 535 Zeichen
\newline{}Handschrift: schwarze Tinte, deutsche Kurrent}\toendnotes[C]{\smallbreak}
\pstart
           \noindent{}{\pb}Verehrter Freund! Ueberbringer dieſes, ein unverſchuldet in Not
               geratener \label{K_L01236-1v}\edtext{Schriftſteller\pwindex{Ferency @\textsc{Ferency}, \emph{Schriftsteller/Schriftstellerin}|pwuv}}{\lemma{\textnormal{\emph{Schriftſteller}}}\Cendnote{\textnormal{Der Karte fehlt die Jahresangabe. Sofern
                  die Person im \emph{Tagebuch}\pwindex{Tagebuch@\emph{Tagebuch}|pwk} erwähnt ist, könnte es
                  sich um einen nicht näher bestimmten Ferency\pwindex{Ferency @\textsc{Ferency}, \emph{Schriftsteller/Schriftstellerin}|pwk} handeln, der Schnitzler am
                     30. 9. 1902
                  besucht hat.}}}\label{K_L01236-1}, von \textsc{Jacobsen}\pwindex{Jacobsohn, Siegfried 28.01.1881 – 03.12.1926@\textsc{Jacobsohn, Siegfried} (28.01.1881 – 03.12.1926), \emph{Journalist/Journalistin, Kritiker/Kritikerin, Publizist/Publizistin}|pw} (Berlin\oindex{Berlin@\textbf{Berlin}, \emph{P.PPLC}|pw}) \textsc{Polgar}\pwindex{Polgar, Alfred 17.10.1873 – 24.04.1955@\textsc{Polgar, Alfred} (17.10.1873 – 24.04.1955), \emph{Schriftsteller/Schriftstellerin, Journalist/Journalistin, Kritiker/Kritikerin}|pw} u. \textsc{Glücksmann}\pwindex{Gluecksmann, Heinrich 08.07.1864 – 01.03.1943@\textsc{Glücksmann, Heinrich} (08.07.1864 – 01.03.1943), \emph{Schriftsteller/Schriftstellerin, Journalist/Journalistin, Dramaturg/Dramaturgin}|pw} warm empfohlen, erſucht mich um einige Worte an einen München\oindex{Muenchen@\textbf{München}, \emph{P.PPLA}|pw}er Verlag. Da ich aber dort keine Beziehungen habe,
               wäre es Ihnen vielleicht möglich, ihm ein {\pb}paar Zeilen mitzugeben. Es handelt ſich ihm nur darum, daß ſeine Sachen in dem
               betreffenden Verlag bald geleſen werden u. er in kurzer Zeit einen zuſagenden oder
               ablehnenden Beſcheid erhält. Verzeihen Sie die Beläſtigung.\pend
           \pstart Ihr ergebenſter\spacefill\mbox{A. F. Seligmann}\pend{}
\pstart
           30/IX.\pend
           \selectlanguage{ngerman}\endnumbering\briefempfaengerindex{Schnitzler, Arthur@\textsc{Schnitzler, Arthur}!zzzSeligmann, Adalbert Franz@\emph{von Adalbert Franz Seligmann}!1902-09-303@{30. 9. {[}1902?{]}}|)be}\mylabel{L01236h}  \normalsize

\doendnotes{C}
\bigskip
\vfill

\clearpage

\footnotesize

\lohead{\textsc{register}}

% Definiere theindex-Environment komplett neu ohne reledmac
\makeatletter
\renewenvironment{theindex}{%
  \section*{\indexname}%
  \setlength{\parindent}{0pt}%
  \setlength{\parskip}{0pt plus 0.3pt}%
  \let\item\@idxitem
}{%
  \clearpage
}
\makeatother

\IfFileExists{\jobname-pw.ind}{\input{\jobname-pw.ind}}{}

\end{document}

      