%% latex-leseansicht-vorspann.tex
%% Vorspann für die Leseansicht.
%% Lädt die gemeinsame Datei latex-vorspann.tex mit nicht gesetztem Schalter.

\newif\ifkorrekturansicht
\korrekturansichtfalse

\input{../tex-inputs/latex-vorspann}


         
         \renewcommand{\erwaehntePersonen}{Personen:  Ferency, Heinrich Glücksmann, Siegfried Jacobsohn, Alfred Polgar, Adalbert Franz Seligmann}
         \renewcommand{\erwaehnteOrte}{Orte: Berlin, München, Wien}
         \renewcommand{\erwaehnteWerke}{Werke: Tagebuch}
               \section[Adalbert Seligmann an Arthur Schnitzler, 30. 9. {[}1902?{]}]{ Adalbert Seligmann an Arthur Schnitzler, 30. 9. {[}1902?{]}}\nopagebreak\mylabel{v}\rehead{ }\begin{ledgroupsized}[t]{13cm}\normalsize\beginnumbering\briefempfaengerindex{Schnitzler, Arthur@\textsc{Schnitzler, Arthur}!zzzSeligmann, Adalbert Franz@\emph{von Adalbert Franz Seligmann}!1902-09-303@{30. 9. {[}1902?{]}}|(be} \toendnotes[C]{\smallbreak\pagebreak[2]} \Standort{CUL, Schnitzler, B 97.}
\physDesc{Briefkarte, 535 Zeichen
\newline{}Handschrift: schwarze Tinte, deutsche Kurrent}\toendnotes[C]{\smallbreak}\pstart
           \noindent{}{\pb}Verehrter Freund! Ueberbringer dieſes, ein unverſchuldet in Not
               geratener \label{K_L01236-1v}\edtext{Schriftſteller\pwindex{Ferency @\textsc{Ferency}, \emph{Schriftsteller}|pwuv}}{\lemma{\textnormal{\emph{Schriftſteller}}}\Cendnote{\textnormal{Der Karte fehlt die Jahresangabe. Sofern
                  die Person im \emph{Tagebuch}\pwindex{\textcolor{red}{\textsuperscript{XXXX1 indx}}!Tagebuch1981 – 2000@\strich\emph{Tagebuch} {[}Hrsg., 1981 – 2000{]}|pwk} erwähnt ist, könnte es
                  sich um einen nicht näher bestimmten Ferency\pwindex{Ferency @\textsc{Ferency}, \emph{Schriftsteller}|pwk} handeln, der Schnitzler\pwindex{Schnitzler, Arthur 15.05.1862 – 21.10.1931@\textsc{Schnitzler, Arthur} (15.05.1862 – 21.10.1931), \emph{Schriftsteller, Mediziner}|pwk} am
                     30. 9. 1902
                  besucht hat.}}}\label{K_L01236-1h}, von \textsc{Jacobsen}\pwindex{Jacobsohn, Siegfried 28.01.1881 – 03.12.1926@\textsc{Jacobsohn, Siegfried} (28.01.1881 – 03.12.1926), \emph{Journalist, Kritiker, Publizist}|pw} (Berlin\oindex{Berlin@\textbf{Berlin}|pw}) \textsc{Polgar}\pwindex{Polgar, Alfred 17.10.1873 – 24.04.1955@\textsc{Polgar, Alfred} (17.10.1873 – 24.04.1955), \emph{Schriftsteller, Journalist, Kritiker}|pw} u. \textsc{Glücksmann}\pwindex{Gluecksmann, Heinrich 08.07.1864 – 01.03.1943@\textsc{Glücksmann, Heinrich} (08.07.1864 – 01.03.1943), \emph{Schriftsteller, Journalist, Dramaturg}|pw} warm empfohlen, erſucht mich um einige Worte an einen München\oindex{Muenchen@\textbf{München}|pw}er Verlag. Da ich aber dort keine Beziehungen habe,
               wäre es Ihnen vielleicht möglich, ihm ein {\pb}paar Zeilen mitzugeben. Es handelt ſich ihm nur darum, daß ſeine Sachen in dem
               betreffenden Verlag bald geleſen werden u. er in kurzer Zeit einen zuſagenden oder
               ablehnenden Beſcheid erhält. Verzeihen Sie die Beläſtigung.\pend
           \pstart Ihr ergebenſter\spacefill\mbox{A. F. Seligmann}\pend{}\pstart
           30/IX.\pend
           
         
         \endnumbering\mylabel{h}\end{ledgroupsized}  \newcommand{\dateiname}{L01236}\newcommand{\titel}{Adalbert Seligmann an Arthur Schnitzler, 30. 9. [1902?]}\newcommand{\editorInnen}{Martin Anton Müller und Gerd-Hermann Susen}%% latex-leseansicht-abspann.tex
%% Abspann für die Leseansicht.
%% Der Schalter \ifkorrekturansicht ist bereits durch den Vorspann gesetzt.

%% latex-abspann.tex
%% Gemeinsamer Abspann für Korrekturansicht und Leseansicht.
%% Setzt den Schalter \ifkorrekturansicht voraus (gesetzt in den
%% einbindenden Dateien latex-korrekturansicht-abspann.tex bzw.
%% latex-leseansicht-abspann.tex).
%% ---------------------------------------------------------------

\normalsize

% Das esempio-Environment wird nur in der Leseansicht benötigt
\ifkorrekturansicht\else
\newenvironment{esempio}[3]%
{
    \vspace{1.5ex}
    \rlap{\underline{#1}}
    \par
    \setlength{\parindent}{0cm}
    \nopagebreak
    \leftskip=#2cm
    \rightskip=#3cm
}
{
    \par
}
\fi

\doendnotes{C}
\bigskip
\vfill

\clearpage

\footnotesize

\ifkorrekturansicht
  \lohead{\textsc{register}}
\fi

% theindex-Environment neu definieren ohne reledmac
\makeatletter
\renewenvironment{theindex}{%
  \ifkorrekturansicht
    \section*{\indexname}%
  \else
    \subsubsection*{Index der erwähnten Entitäten}%
  \fi
  \setlength{\parindent}{0pt}%
  \setlength{\parskip}{0pt plus 0.3pt}%
  \let\item\@idxitem
}{%
  \ifkorrekturansicht\clearpage\fi
}
\makeatother

\IfFileExists{\jobname-pw.ind}{\input{\jobname-pw.ind}}{}

% Quellenangabe nur in der Leseansicht
\ifkorrekturansicht\else
% Fallback-Definitionen, falls die .tex-Datei \titel etc. nicht gesetzt hat
\providecommand{\titel}{}
\providecommand{\editorInnen}{}
\providecommand{\dateiname}{\jobname}

\vspace{3cm}

\vfill

\footnotesize
\textsc{Quelle}: \titel. Herausgegeben von {\editorInnen}. In: \emph{Arthur Schnitzler: Briefwechsel mit Autorinnen und Autoren}.
 Digitale Edition, https://schnitzler-briefe.acdh.oeaw.ac.at/{\dateiname}.html (Stand \today)
\fi

\end{document}


      