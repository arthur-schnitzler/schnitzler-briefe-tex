%% latex-korrekturansicht-vorspann.tex
%% Vorspann für die Korrekturansicht.
%% Lädt die gemeinsame Datei latex-vorspann.tex mit gesetztem Schalter.

\newif\ifkorrekturansicht
\korrekturansichttrue

\input{../tex-inputs/latex-vorspann}


\section[Hugo von Hofmannsthal an Arthur Schnitzler, 8. 4. 1910]{L01923 Hugo von Hofmannsthal an Arthur Schnitzler, 8. 4. 1910}
\nopagebreak\mylabel{L01923v}
\rehead{ }\normalsize\beginnumbering\briefempfaengerindex{Schnitzler, Arthur@\textsc{Schnitzler, Arthur}!zzzHofmannsthal, Hugo von@\emph{von Hugo von Hofmannsthal}!1910-04-082@{8. 4. 1910}|(be}
\toendnotes[C]{\smallbreak\pagebreak[2]}\Standort{CUL, Schnitzler, B 43.}
\physDesc{Postkarte, 645 Zeichen
\newline{}Handschrift: 1) schwarze Tinte, deutsche Kurrent\hspace{1em}2) schwarze Tinte, lateinische Kurrent (\noindent{}Adresse)\hspace{1em}
\newline{}Versand: Stempel: »\nobreak{}\oindex{IX., Alsergrund@\textbf{IX., Alsergrund}, \emph{A.ADM3}|pwk}9/4 Wien 68, 8 IV 10, 12\nobreak{}«.  
\newline{}Schnitzler: mit Bleistift datiert: »\substVorne{}\textsuperscript{März}\substDazwischen{}April\substHinten{} 910« und beschriftet: »Hugo« 
\newline{}Ordnung: 1) mit Bleistift von unbekannter Hand nummeriert:
                                    »314«  2) mit Bleistift von unbekannter Hand nummeriert:
                                    »317«}
\buchAbdrucke{\weitereDrucke{Hugo von Hofmannsthal, Arthur Schnitzler: \emph{Briefwechsel}. Frankfurt am Main: \emph{S. Fischer} 1964, S. 248.} }\pstart{}{\pb}Herrn\pend{}\pstart{}D\textsuperscript{r} Arthur Schnitzler\pend{}\pstart{}Wien\oindex{Wien@\textbf{Wien}, \emph{A.ADM2}|pw}\pend{}\pstart{}XVIII Spöttelgasse 7\oindex{Edmund-Weiss-Gasse 7@\textbf{Edmund-Weiß-Gasse 7}, \emph{Wohngebäude (K.WHS)}|pw}.\pend{}{\bigskip}\vspace{1em}
\pstart
           \raggedleft{}\textsc{Sanatorium Löw\oindex{Sanatorium Loew@\textbf{Sanatorium Loew}, \emph{Sanatorium (K.SAN)}|pw}, Frauenabteilung}\pend
           
\pstart
           \raggedleft{}\textsc{Pelikangasse 15}\oindex{Pelikangasse@\textbf{Pelikangasse}, \emph{Straße (K.STR)}|pw}. \pend
           
\pstart
           Freitag{ }abends.\pend
           \vspace{0.5em}
\pstart
           mein lieber Arthur,\hspace*{1.5em}Gerty\pwindex{Hofmannsthal, Gertrude von 16.03.1880 – 09.11.1959@\textsc{Hofmannsthal, Gertrude von} (16.03.1880 – 09.11.1959)|pw} iſt ſchon ſo ziemlich ſchmerzfrei und
               wäre ſehr erfreut wenn Olga\pwindex{Schnitzler, Olga 17.01.1882 – 13.01.1970@\textsc{Schnitzler, Olga} (17.01.1882 – 13.01.1970), \emph{Schauspieler/Schauspielerin, Sänger/Sängerin}|pw} Sie \substVorne{}\textsuperscript{Sonntag oder Montag}\substDazwischen{}Montag oder Dienstag\substHinten{}{ }{\pb}nachmittags durch ihren Beſuch
               auszeichnen würde, bittet aber um vorherige gütige telephoniſche Anſage.\pend
           
\pstart
           \uline{Mir} würde es große Freude machen wieder einmal – da
               ich jetzt ausnahmsweiſe in Wien\oindex{Wien@\textbf{Wien}, \emph{A.ADM2}|pw} wohne – mit Ihnen
               vormittags ſpazierenzugehen.\pend
           
\pstart
           Dürfte ich Sie Sonntag oder Montag dazu abholen? Um
                  11 Uhr? oder wann? jedenfalls wünſche mir, Sie zu ſehen, doppelt in
               dieſen etwas abnormalen Tagen. Bitte um ein Wort.\pend
           
\pstart
           Ihr{\\[\baselineskip]}\spacefill\mbox{Hugo.}\pend
           \leftskip=0em{}\selectlanguage{ngerman}\endnumbering\briefempfaengerindex{Schnitzler, Arthur@\textsc{Schnitzler, Arthur}!zzzHofmannsthal, Hugo von@\emph{von Hugo von Hofmannsthal}!1910-04-082@{8. 4. 1910}|)be}\mylabel{L01923h}  \normalsize

\doendnotes{C}
\bigskip
\vfill

\clearpage

\footnotesize

\lohead{\textsc{register}}

% Definiere theindex-Environment komplett neu ohne reledmac
\makeatletter
\renewenvironment{theindex}{%
  \section*{\indexname}%
  \setlength{\parindent}{0pt}%
  \setlength{\parskip}{0pt plus 0.3pt}%
  \let\item\@idxitem
}{%
  \clearpage
}
\makeatother

\IfFileExists{\jobname-pw.ind}{\input{\jobname-pw.ind}}{}

\end{document}

      