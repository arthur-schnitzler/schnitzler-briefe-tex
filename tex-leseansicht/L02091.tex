%% latex-korrekturansicht-vorspann.tex
%% Vorspann für die Korrekturansicht.
%% Lädt die gemeinsame Datei latex-vorspann.tex mit gesetztem Schalter.

\newif\ifkorrekturansicht
\korrekturansichttrue

\input{../tex-inputs/latex-vorspann}


\section[Georg Brandes an Arthur Schnitzler, 26. 9. 1912]{L02091 Georg Brandes an Arthur Schnitzler, 26. 9. 1912}
\nopagebreak\mylabel{L02091v}
\rehead{ }\normalsize\beginnumbering\briefempfaengerindex{Schnitzler, Arthur@\textsc{Schnitzler, Arthur}!zzzBrandes, Georg@\emph{von Georg Brandes}!1912-09-261@{26. 9. 1912}|(be}
\toendnotes[C]{\smallbreak\pagebreak[2]}\Standort{CUL, Schnitzler, B 17.}
\physDesc{Postkarte, 632 Zeichen
\newline{}Handschrift: blaue Tinte, lateinische Kurrent
\newline{}Versand: Stempel: »\nobreak{}\oindex{Skagen@\textbf{Skagen}, \emph{P.PPL}|pwk}Skagen, 26.9. 12, 12–3E\nobreak{}«.  
\newline{}Ordnung: mit Bleistift von unbekannter Hand nummeriert:
                                    »39« }
\buchAbdrucke{\weitereDrucke{Georg Brandes, Arthur Schnitzler: \emph{Ein Briefwechsel}. Bern: \emph{Francke} 1956, S. 104–105.} }\toendnotes[C]{\smallbreak}\pstart{}{\pb}Herrn Dr. Arthur
                  Schnitzler\pend{}\pstart{}Sternwartestrasse 71\oindex{Sternwartestrasse 71@\textbf{Sternwartestraße 71}, \emph{Wohngebäude (K.WHS)}|pw}\pend{}\pstart{}Wien XVIII\oindex{XVIII., Waehring@\textbf{XVIII., Währing}, \emph{A.ADM3}|pw}\pend{}{\bigskip}\vspace{1em}
\pstart
           \raggedleft{}{\pb}\label{K_L02091-1v}\edtext{p. T.}{\lemma{\textnormal{\emph{p. T.}}}\Cendnote{\textnormal{pro tempore, lateinisch: zur Zeit}}}\label{K_L02091-1}{ }Skagen\oindex{Skagen@\textbf{Skagen}, \emph{P.PPL}|pw}{\\}Dänemark\oindex{Daenemark@\textbf{Dänemark}, \emph{A.PCLI}|pw}{\\}Adresse: Kopenhagen\oindex{Kopenhagen@\textbf{Kopenhagen}, \emph{P.PPLC}|pw}\pend
           
\pstart{}Verehrter Freund\pend\vspace{0.5em}
\pstart
           Da es zehn oder elf Jahre her ist, dass ich in Wien\oindex{Wien@\textbf{Wien}, \emph{A.ADM2}|pw} war, habe ich um einen Vorwand zu haben, es wiederzusehen, mich
               engagiren lassen, ein Paar erbärmliche Vorträge zwischen 20{ }\strikeout{\textcolor{gray}{×}} und 23 November dort zu halten (\uline{Urania}\oindex{Urania@\textbf{Urania}, \emph{Volksbildungsanstalt (K.VBA)}|pw}). Da nun Sie ein Hauptstück von meinem Wien\oindex{Wien@\textbf{Wien}, \emph{A.ADM2}|pw}
               sind, möchte ich gerne wissen, ob Sie wol zu der Zeit sich in Wien\oindex{Wien@\textbf{Wien}, \emph{A.ADM2}|pw} befinden (nur auf einer Karte antworten).\pend
           
\pstart
           Ich möchte noch gerne Beer-Hoffmann\pwindex{Beer-Hofmann, Richard 1866-07-11 – 1945-09-26@\textsc{Beer-Hofmann, Richard} (1866-07-11 – 1945-09-26), \emph{Schriftsteller/Schriftstellerin}|pw} und Wassermann\pwindex{Wassermann, Jakob 10.03.1873 – 01.01.1934@\textsc{Wassermann, Jakob} (10.03.1873 – 01.01.1934), \emph{Schriftsteller/Schriftstellerin}|pw} und Hofmannsthal\pwindex{Hofmannsthal, Hugo von 1874-02-01 – 1929-07-15@\textsc{Hofmannsthal, Hugo von} (1874-02-01 – 1929-07-15), \emph{Schriftsteller/Schriftstellerin}|pw} sehen, die ja alle Wien\oindex{Wien@\textbf{Wien}, \emph{A.ADM2}|pw}er sind; Sie sind aber für mich die Hauptperson.\pend
           
\pstart
           Ihr sehr ergebener{\\[\baselineskip]}\spacefill\mbox{Georg Brandes}\pend
           \leftskip=0em{}\selectlanguage{ngerman}\endnumbering\briefempfaengerindex{Schnitzler, Arthur@\textsc{Schnitzler, Arthur}!zzzBrandes, Georg@\emph{von Georg Brandes}!1912-09-261@{26. 9. 1912}|)be}\mylabel{L02091h}  \normalsize

\doendnotes{C}
\bigskip
\vfill

\clearpage

\footnotesize

\lohead{\textsc{register}}

% Definiere theindex-Environment komplett neu ohne reledmac
\makeatletter
\renewenvironment{theindex}{%
  \section*{\indexname}%
  \setlength{\parindent}{0pt}%
  \setlength{\parskip}{0pt plus 0.3pt}%
  \let\item\@idxitem
}{%
  \clearpage
}
\makeatother

\IfFileExists{\jobname-pw.ind}{\input{\jobname-pw.ind}}{}

\end{document}

      