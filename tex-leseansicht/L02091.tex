%% latex-leseansicht-vorspann.tex
%% Vorspann für die Leseansicht.
%% Lädt die gemeinsame Datei latex-vorspann.tex mit nicht gesetztem Schalter.

\newif\ifkorrekturansicht
\korrekturansichtfalse

\input{../tex-inputs/latex-vorspann}


         
         \renewcommand{\erwaehntePersonen}{Personen: Richard Beer-Hofmann, Hugo von Hofmannsthal, Jakob Wassermann}
         \renewcommand{\erwaehnteOrte}{Orte: Dänemark, Kopenhagen, Skagen, Sternwartestraße, Urania, Wien, XVIII., Währing}
         \renewcommand{\erwaehnteWerke}{
               \section[Georg Brandes an Arthur Schnitzler, 26. 9. 1912]{ Georg Brandes an Arthur Schnitzler, 26. 9. 1912}\nopagebreak\mylabel{v}\rehead{ }\begin{ledgroupsized}[t]{13cm}\normalsize\beginnumbering \toendnotes[C]{\smallbreak\pagebreak[2]} \Standort{CUL, Schnitzler, B 17.}
\physDesc{Postkarte
\newline{}Handschrift: blaue Tinte, lateinische Kurrent\newline{}Versand: Stempel: »\nobreak{}\oindex{Skagen@\textbf{Skagen}|pwk}Skagen, 26.9. 12, 12–3E\nobreak{}«.  \newline{}Ordnung: mit Bleistift von unbekannter Hand nummeriert:
                                    »39« }\buchAbdrucke{\weitereDrucke{Georg Brandes, Arthur Schnitzler: \emph{Ein Briefwechsel}. Hg. Kurt Bergel. Bern: \emph{Francke} 1956, S. 104–105.} }\toendnotes[C]{\smallbreak}\pstart{}{\pb}Herrn Dr. Arthur
                  Schnitzler\pend{}\pstart{}Sternwartestrasse 71\oindex{Sternwartestrasse@\textbf{Sternwartestraße}|pw}\pend{}\pstart{}Wien XVIII\oindex{XVIII., Waehring@\textbf{XVIII., Währing}|pw}\pend{}{\bigskip}\pstart
           \raggedleft{}{\pb}\label{K_L02091_1v}\edtext{p. T.}{\lemma{\textnormal{\emph{p. T.}}}\Cendnote{\textnormal{pro tempore, lateinisch: zur Zeit}}}\label{K_L02091_1h}{ }Skagen\oindex{Skagen@\textbf{Skagen}|pw}{\\}Dänemark\oindex{Daenemark@\textbf{Dänemark}|pw}{\\}Adresse: Kopenhagen\oindex{Kopenhagen@\textbf{Kopenhagen}|pw}\pend
           \pstart{}Verehrter Freund\pend\pstart
           Da es zehn oder elf Jahre her ist, dass ich in Wien\oindex{Wien@\textbf{Wien}|pw} war, habe ich um einen Vorwand zu haben, es wiederzusehen, mich
               engagiren lassen, ein Paar erbärmliche Vorträge zwischen 20{ }\strikeout{\textcolor{gray}{×}} und 23 November dort zu halten (\uline{Urania}\oindex{Urania@\textbf{Urania}|pw}). Da nun Sie ein Hauptstück von meinem Wien\oindex{Wien@\textbf{Wien}|pw}
               sind, möchte ich gerne wissen, ob Sie wol zu der Zeit sich in Wien\oindex{Wien@\textbf{Wien}|pw} befinden (nur auf einer Karte antworten).\pend
           \pstart
           Ich möchte noch gerne Beer-Hoffmann\pwindex{Beer-Hofmann, Richard 1866-07-11 – 1945-09-26@\textsc{Beer-Hofmann, Richard} (1866-07-11 – 1945-09-26), \emph{Schriftsteller}|pw} und Wassermann\pwindex{Wassermann, Jakob 10.03.1873 – 01.01.1934@\textsc{Wassermann, Jakob} (10.03.1873 – 01.01.1934), \emph{Schriftsteller}|pw} und Hofmannsthal\pwindex{Hofmannsthal, Hugo von 1874-02-01 – 1929-07-15@\textsc{Hofmannsthal, Hugo von} (1874-02-01 – 1929-07-15), \emph{Schriftsteller}|pw} sehen, die ja alle Wien\oindex{Wien@\textbf{Wien}|pw}er sind; Sie sind aber für mich die Hauptperson.\pend
           \pstart
           Ihr sehr ergebener{\\[\baselineskip]}\spacefill\mbox{Georg Brandes}\pend
           \leftskip=0em{}
         
         \endnumbering\mylabel{h}\end{ledgroupsized}  \newcommand{\dateiname}{L02091}\newcommand{\titel}{Georg Brandes an Arthur Schnitzler, 26. 9. 1912}\newcommand{\editorInnen}{Martin Anton Müller und Gerd-Hermann Susen}%% latex-leseansicht-abspann.tex
%% Abspann für die Leseansicht.
%% Der Schalter \ifkorrekturansicht ist bereits durch den Vorspann gesetzt.

%% latex-abspann.tex
%% Gemeinsamer Abspann für Korrekturansicht und Leseansicht.
%% Setzt den Schalter \ifkorrekturansicht voraus (gesetzt in den
%% einbindenden Dateien latex-korrekturansicht-abspann.tex bzw.
%% latex-leseansicht-abspann.tex).
%% ---------------------------------------------------------------

\normalsize

% Das esempio-Environment wird nur in der Leseansicht benötigt
\ifkorrekturansicht\else
\newenvironment{esempio}[3]%
{
    \vspace{1.5ex}
    \rlap{\underline{#1}}
    \par
    \setlength{\parindent}{0cm}
    \nopagebreak
    \leftskip=#2cm
    \rightskip=#3cm
}
{
    \par
}
\fi

\doendnotes{C}
\bigskip
\vfill

\clearpage

\footnotesize

\ifkorrekturansicht
  \lohead{\textsc{register}}
\fi

% theindex-Environment neu definieren ohne reledmac
\makeatletter
\renewenvironment{theindex}{%
  \ifkorrekturansicht
    \section*{\indexname}%
  \else
    \subsubsection*{Index der erwähnten Entitäten}%
  \fi
  \setlength{\parindent}{0pt}%
  \setlength{\parskip}{0pt plus 0.3pt}%
  \let\item\@idxitem
}{%
  \ifkorrekturansicht\clearpage\fi
}
\makeatother

\IfFileExists{\jobname-pw.ind}{\input{\jobname-pw.ind}}{}

% Quellenangabe nur in der Leseansicht
\ifkorrekturansicht\else
% Fallback-Definitionen, falls die .tex-Datei \titel etc. nicht gesetzt hat
\providecommand{\titel}{}
\providecommand{\editorInnen}{}
\providecommand{\dateiname}{\jobname}

\vspace{3cm}

\vfill

\footnotesize
\textsc{Quelle}: \titel. Herausgegeben von {\editorInnen}. In: \emph{Arthur Schnitzler: Briefwechsel mit Autorinnen und Autoren}.
 Digitale Edition, https://schnitzler-briefe.acdh.oeaw.ac.at/{\dateiname}.html (Stand \today)
\fi

\end{document}


      