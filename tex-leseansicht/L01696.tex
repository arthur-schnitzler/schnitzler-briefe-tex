%% latex-korrekturansicht-vorspann.tex
%% Vorspann für die Korrekturansicht.
%% Lädt die gemeinsame Datei latex-vorspann.tex mit gesetztem Schalter.

\newif\ifkorrekturansicht
\korrekturansichttrue

\input{../tex-inputs/latex-vorspann}


\section[Max Mell an Arthur Schnitzler, 29. 7. 1907]{L01696 Max Mell an Arthur Schnitzler, 29. 7. 1907}
\nopagebreak\mylabel{L01696v}
\rehead{ }\normalsize\beginnumbering\briefempfaengerindex{Schnitzler, Arthur@\textsc{Schnitzler, Arthur}!zzzMell, Max@\emph{von Max Mell}!1907-07-292@{29. 7. 1907}|(be}
\toendnotes[C]{\smallbreak\pagebreak[2]}\Standort{CUL, Schnitzler, B 70.}
\physDesc{Briefkarte, 568 Zeichen
\newline{}Handschrift: schwarze Tinte, deutsche Kurrent}\toendnotes[C]{\smallbreak}
\pstart
           \raggedleft{}{\pb}Wien\oindex{Wien@\textbf{Wien}, \emph{A.ADM2}|pw}, 29. Juli 1907\pend
           
\pstart{}Sehr geehrter Herr Doktor,\pend\vspace{0.5em}
\pstart
           es wird mir ſehr ſchmerzlich ſein, in meinem Almanach\pwindex{Almanach der Wiener Werkstaette@\emph{Almanach der Wiener Werkstätte}|pwv} nichts von Ihnen zu haben. Wäre es nicht möglich,
               daß Sie mir ein Fragment aus der größeren Arbeit\pwindex{Weg ins Freie. Roman@\emph{Der Weg ins Freie. Roman}|pwv} die Sie jetzt ſchreiben, gäben? Im ſchlimmſten Fall
               möchte ich wenigſtens etwas ſchon gedrucktes, (etwa Gedichte?) bringen, und bäte Sie
               dafür um Rat.\pend
           
\pstart
           Mit Ihrer Anſichtskarte haben Sie mir eine große Freude gemacht, \textsc{Dr. Servaes\pwindex{Servaes, Franz 17.06.1862 – 14.07.1947@\textsc{Servaes, Franz} (17.06.1862 – 14.07.1947), \emph{Journalist/Journalistin, Kritiker/Kritikerin}|pw}}, der am ſelben Tag zu mir kam, hat mich ordentlich {\pb}beneidet darum. Bitte empfehlen Sie mich
               Ihrer verehrten Frau\pwindex{Schnitzler, Olga 17.01.1882 – 13.01.1970@\textsc{Schnitzler, Olga} (17.01.1882 – 13.01.1970), \emph{Schauspieler/Schauspielerin, Sänger/Sängerin}|pwv}!\pend
           
\pstart
           Mit den herzlichſten Grüßen{\\[\baselineskip]}Ihr{\\[\baselineskip]}\spacefill\mbox{Max Mell.}\pend
           \leftskip=0em{}\selectlanguage{ngerman}\endnumbering\briefempfaengerindex{Schnitzler, Arthur@\textsc{Schnitzler, Arthur}!zzzMell, Max@\emph{von Max Mell}!1907-07-292@{29. 7. 1907}|)be}\mylabel{L01696h}  \normalsize

\doendnotes{C}
\bigskip
\vfill

\clearpage

\footnotesize

\lohead{\textsc{register}}

% Definiere theindex-Environment komplett neu ohne reledmac
\makeatletter
\renewenvironment{theindex}{%
  \section*{\indexname}%
  \setlength{\parindent}{0pt}%
  \setlength{\parskip}{0pt plus 0.3pt}%
  \let\item\@idxitem
}{%
  \clearpage
}
\makeatother

\IfFileExists{\jobname-pw.ind}{\input{\jobname-pw.ind}}{}

\end{document}

      