%% latex-leseansicht-vorspann.tex
%% Vorspann für die Leseansicht.
%% Lädt die gemeinsame Datei latex-vorspann.tex mit nicht gesetztem Schalter.

\newif\ifkorrekturansicht
\korrekturansichtfalse

\input{../tex-inputs/latex-vorspann}


\section[Robert Adam an Arthur Schnitzler, Briefentwurf, 15. 4. 1913]{L02119 Robert Adam an Arthur Schnitzler, Briefentwurf, 15. 4. 1913}
\nopagebreak\mylabel{L02119v}
\rehead{ }\normalsize\beginnumbering\briefempfaengerindex{Schnitzler, Arthur@\textsc{Schnitzler, Arthur}!zzzAdam, Robert@\emph{von Robert Adam}!1913-04-151@{15. 4. 1913}|(be}
\toendnotes[C]{\smallbreak\pagebreak[2]}
\correspDesc{Versand  durch Robert Adam am 15. 4. 1913 in Zistersdorf
\newline{}Erhalt  durch Arthur Schnitzler im Zeitraum [16. 4. 1913
                  – 20. 4. 1913?] in Wien}\toendnotes[C]{\smallbreak}
\Standort{Wien, Österreichische Nationalbibliothek, Cod. ser. 52.266, 161.}
\physDesc{Briefentwurf, 1 Blatt, 2 Seiten, 2966 Zeichen
\newline{}Handschrift: schwarze Tinte, deutsche Kurrent}\toendnotes[C]{\smallbreak}
\pstart
           \raggedleft{}{\pb}Ziſtersdorf\oindex{Zistersdorf@\textbf{Zistersdorf}, \emph{Verwaltungsgebiet}|pw}, am 1\substVorne{}\textsuperscript{4}\substDazwischen{}5\substHinten{}. April 1913\pend
           
\pstart{}Hochverehrter Herr Doktor!\pend\vspace{0.5em}
\pstart
           Ich mache von Ihrer liebenswürdigen \label{K_L02119-1v}\edtext{Erlaubnis Gebrauch}{\lemma{\textnormal{\emph{Erlaubnis Gebrauch}}}\Cendnote{\textnormal{Eine Fassung des
                  Briefes wurde am 15. 4. 1913 abgesandt, wie aus dem unmittelbar auf
                  den Entwurf folgenden Tagebucheintrag hervorgeht.}}}\label{K_L02119-1} und überſende Ihnen das
               Manuſkript \substVorne{}\textsuperscript{der}\substDazwischen{}von\substHinten{} »Fatme\pwindex{Adam, Robert 20.\,4.\,1877 Wien – 16.\,10.\,1961 Baden bei Wien@\textsc{Adam, Robert} (20.\,4.\,1877 Wien – 16.\,10.\,1961 Baden bei Wien), \emph{Schriftsteller, Richter}!Fatme@\strich\emph{Fatme}|pw}«.\pend
           
\pstart
           Hiebei muß ich Sie vor allem deshalb um Nachſicht bitten, weil die
               Schreibmaſchinenabſchrift \substVorne{}\textsuperscript{keineswegs{ }ſo}\substDazwischen{}verſchiedener leidiger Umſtände halber nicht recht\substHinten{} preſentabel ausgefallen iſt \strikeout{wie ich{ }ſie erwarte.
                  Beſonders der blaue Druck der erſten Hälfte iſt mir herzlich unangenehm. Trotzdem{ }ſende ich Ihnen dies und und nicht das Durchſchlagsexemplar, da letzteres doch
                  weniger deutlich iſt.}\pend
           
\pstart
           Und dann bitte ich Sie \introOben{}betreffs\introOben{} der »Fatme\pwindex{Adam, Robert 20.\,4.\,1877 Wien – 16.\,10.\,1961 Baden bei Wien@\textsc{Adam, Robert} (20.\,4.\,1877 Wien – 16.\,10.\,1961 Baden bei Wien), \emph{Schriftsteller, Richter}!Fatme@\strich\emph{Fatme}|pw}«{ }ſelbſt \strikeout{wegen} um Duldung.
               Ich nenne{ }ſie eine »Studie«; ich wage es nicht,{ }ſie eine dramatiſche Studie zu
               nennen. Die beſte Bezeichnung wäre vielleicht: ein Konglomerat. Wenn ich \introOben{}mir\introOben{} die Frage \substVorne{}\textsuperscript{erwäge}\substDazwischen{}ſtelle\substHinten{}, ob dies \substVorne{}\textsuperscript{Konglomerat}\substDazwischen{}\strikeout{Sammelſurium} Gemengſel\substHinten{} von \strikeout{Phantaſie,} Phantaſterei, \introOben{}Theorie, \strikeout{Ökonomie,}\introOben{}{ }Satire, \introOben{}Erlebnis\introOben{}, Roſinen,
                  \introOben{}Geſellſchafts\introOben{}Kritik-\introOben{}Charakteririſierungs-\introOben{} und Dramenanſätzen Sie intereſſieren werde – \strikeout{ſo} zweifle ich \strikeout{über}{ }\strikeout{die Antwort}; ja ich verzweifle geradezu. Ich möchte
               faſt wünſchen, ich hätte mich \introOben{}wegen\introOben{} dieſes \introOben{}höchſt undramatiſchen\introOben{} Miſchlings von Ernſt und Spott \introOben{}der betr. \strikeout{d\textcolor{gray}{och}} jedem Akt, ja jeder Szene \strikeout{nicht} einer
                  Spezialexpoſition \strikeout{eröffnen muß} bedarf\introOben{}{ }\strikeout{wegen} nicht an Sie gewendet, da ich{ }ſehr befürchte,
               eine etwa gute Meinung, die Sie von meinem Geſchmack \introOben{}u.
                  techniſchen Geſchick\introOben{} haben könnten, \strikeout{da}durch
                  \introOben{}ihn\introOben{} zu \substVorne{}\textsuperscript{töten}\substDazwischen{}vernichten\substHinten{}, und ich wünſchte, ich hätte die Vollendung einer \introOben{}weniger
                  exotiſchen u.{ }ſtrafferen\introOben{} Komödie »Geſellſchaft\pwindex{Adam, Robert 20.\,4.\,1877 Wien – 16.\,10.\,1961 Baden bei Wien@\textsc{Adam, Robert} (20.\,4.\,1877 Wien – 16.\,10.\,1961 Baden bei Wien), \emph{Schriftsteller, Richter}!Gesellschaft [Eine Gaunerkomödie]@\strich\emph{Gesellschaft [Eine Gaunerkomödie]}|pw}«, an der ich jetzt arbeite, abgewartet, anſtatt mich »Fatme\pwindex{Adam, Robert 20.\,4.\,1877 Wien – 16.\,10.\,1961 Baden bei Wien@\textsc{Adam, Robert} (20.\,4.\,1877 Wien – 16.\,10.\,1961 Baden bei Wien), \emph{Schriftsteller, Richter}!Fatme@\strich\emph{Fatme}|pw}« \introOben{}gewiſſermaßen\introOben{} zu
               würfeln.\pend
           
\pstart
           Was dieſe betrifft, möchte ich zur Aufklärung nur \substVorne{}\textsuperscript{ſagen}\substDazwischen{}beifügen\substHinten{}, daß ich urſprünglich die \introOben{}einfache\introOben{} Dramatiſierung
               einer Erzählung \textsc{Wells}\pwindex{Wells, H. G. 21.\,9.\,1866 Bromley – 13.\,8.\,1946 London@\textsc{Wells, H. G.} (21.\,9.\,1866 Bromley – 13.\,8.\,1946 London), \emph{Schriftsteller}|pw}{ }\introOben{}(»\textsc{A story of the Days to come}\pwindex{Wells, H. G. 21.\,9.\,1866 Bromley – 13.\,8.\,1946 London@\textsc{Wells, H. G.} (21.\,9.\,1866 Bromley – 13.\,8.\,1946 London), \emph{Schriftsteller}!Story of the Days to Come@\strich\emph{A Story of the Days to Come}|pw}{[}«{]} in \textsc{Tales of Space and Time}\pwindex{Wells, H. G. 21.\,9.\,1866 Bromley – 13.\,8.\,1946 London@\textsc{Wells, H. G.} (21.\,9.\,1866 Bromley – 13.\,8.\,1946 London), \emph{Schriftsteller}!Tales of Space and Time@\strich\emph{Tales of Space and Time}|pw}{ }\strikeout{\textsc{and Space}})\introOben{}{ }\substVorne{}\textsuperscript{beabſichtigte}\substDazwischen{}im Auge hatte\substHinten{}, dann aber, \introOben{}beim Überdenken\introOben{} des Stoffes \strikeout{überdenkend}{ }\strikeout{zur Anſicht}{ }\strikeout{gelangte}{ }\introOben{}mich vor dem \textcolor{gray}{×}\-\textcolor{gray}{×}\-\textcolor{gray}{×}\-\textcolor{gray}{×}weg {\kaufmannsund} die Notwendigkeit geſtellt{ }ſah\introOben{}, \strikeout{ich möchte}{ }\substVorne{}\textsuperscript{den}\substDazwischen{}\strikeout{einen ganzen}\substHinten{}{ }\strikeout{Zukunftsſtaat,}{ }\introOben{}an\introOben{}ſtatt den \textsc{Wells}\pwindex{Wells, H. G. 21.\,9.\,1866 Bromley – 13.\,8.\,1946 London@\textsc{Wells, H. G.} (21.\,9.\,1866 Bromley – 13.\,8.\,1946 London), \emph{Schriftsteller}|pw}’ſchen \introOben{}Zukunftsſta\textcolor{gray}{at}\introOben{} einfach \substVorne{}\textsuperscript{anzunehmen}\substDazwischen{}\strikeout{als gegeben}\substHinten{}, \strikeout{nach}{ }\introOben{}\strikeout{gänzl} zu akzeptieren, in einen Staat zu verlegen,
                  der\introOben{} meinen \strikeout{eigenen} Anſichten \strikeout{raus}{ }\introOben{}zu\introOben{}{ }\introOben{}von einer möglichen Entwicklung der{ }ſozialen Verhältniſſe beſſer
                  entſpräche. So mußte ich für den gegebenen Stoff einen eigenen
                     Zukunftsſta\textcolor{gray}{at}\introOben{} konſtruieren; und kaum {\pb}war \substVorne{}\textsuperscript{damit begonnen}\substDazwischen{}dies geſchehen\substHinten{},{ }ſo \substVorne{}\textsuperscript{ſah ich auch}\substDazwischen{}ergab{ }ſich\substHinten{} die \introOben{}weitere\introOben{} Notwendigkeit \strikeout{vor mir}, \introOben{}auch\introOben{} mit dem \textsc{Wells}\pwindex{Wells, H. G. 21.\,9.\,1866 Bromley – 13.\,8.\,1946 London@\textsc{Wells, H. G.} (21.\,9.\,1866 Bromley – 13.\,8.\,1946 London), \emph{Schriftsteller}|pw}’ſchen Stoff zu brechen \substVorne{}\textsuperscript{und formte meinen eigenen, wie er meinem Staat
                     entſprach.}\substDazwischen{}und die Fabel meinem Staate anzupaſſen. So iſt Fatme\pwindex{Adam, Robert 20.\,4.\,1877 Wien – 16.\,10.\,1961 Baden bei Wien@\textsc{Adam, Robert} (20.\,4.\,1877 Wien – 16.\,10.\,1961 Baden bei Wien), \emph{Schriftsteller, Richter}!Fatme@\strich\emph{Fatme}|pw} die \textsc{Story of the Days to come}\pwindex{Adam, Robert 20.\,4.\,1877 Wien – 16.\,10.\,1961 Baden bei Wien@\textsc{Adam, Robert} (20.\,4.\,1877 Wien – 16.\,10.\,1961 Baden bei Wien), \emph{Schriftsteller, Richter}!Gesellschaft [Eine Gaunerkomödie]@\strich\emph{Gesellschaft [Eine Gaunerkomödie]}|pw};\substHinten{}{ }\substVorne{}\textsuperscript{Alſo wurde zuerſt das Feſt, dann die}\substDazwischen{}dasſelbe Meſſer, doch mit anderem und andrer\substHinten{} Klinge \strikeout{des Meſſers geändert}\pend
           
\pstart
           Sollten Sie, hochverehrter Herr Doktor, der Studie kein Intereſſe ab\substVorne{}\textsuperscript{nötigen}\substDazwischen{}gewinnen\substHinten{} können,{ }ſo bitte ich Sie, mir wegen ihrer Ueberſendung nicht zu grollen und
               mir zu erlauben,{ }ſie \introOben{}ſpäter\introOben{} gegen die »Geſellſchaft\pwindex{Adam, Robert 20.\,4.\,1877 Wien – 16.\,10.\,1961 Baden bei Wien@\textsc{Adam, Robert} (20.\,4.\,1877 Wien – 16.\,10.\,1961 Baden bei Wien), \emph{Schriftsteller, Richter}!Gesellschaft [Eine Gaunerkomödie]@\strich\emph{Gesellschaft [Eine Gaunerkomödie]}|pw}«, \strikeout{die jedenfalls weniger
                  Sammelſurium werden wird,} umzutauſchen.\pend
           \pstart Ich verbleibe mit den ergebenſten Grüßen\hspace*{1.5em}Ihr\hspace*{1.5em}\spacefill\mbox{RA}\pend{}\selectlanguage{ngerman}\endnumbering\briefempfaengerindex{Schnitzler, Arthur@\textsc{Schnitzler, Arthur}!zzzAdam, Robert@\emph{von Robert Adam}!1913-04-151@{15. 4. 1913}|)be}\mylabel{L02119h}  \newcommand{\dateiname}{L02119}\newcommand{\titel}{Robert Adam an Arthur Schnitzler, Briefentwurf, 15. 4. 1913}\newcommand{\editorInnen}{Martin Anton Müller und Gerd-Hermann Susen}%% latex-leseansicht-abspann.tex
%% Abspann für die Leseansicht.
%% Der Schalter \ifkorrekturansicht ist bereits durch den Vorspann gesetzt.

%% latex-abspann.tex
%% Gemeinsamer Abspann für Korrekturansicht und Leseansicht.
%% Setzt den Schalter \ifkorrekturansicht voraus (gesetzt in den
%% einbindenden Dateien latex-korrekturansicht-abspann.tex bzw.
%% latex-leseansicht-abspann.tex).
%% ---------------------------------------------------------------

\normalsize

% Das esempio-Environment wird nur in der Leseansicht benötigt
\ifkorrekturansicht\else
\newenvironment{esempio}[3]%
{
    \vspace{1.5ex}
    \rlap{\underline{#1}}
    \par
    \setlength{\parindent}{0cm}
    \nopagebreak
    \leftskip=#2cm
    \rightskip=#3cm
}
{
    \par
}
\fi

\doendnotes{C}
\bigskip
\vfill

\clearpage

\footnotesize

\ifkorrekturansicht
  \lohead{\textsc{register}}
\fi

% theindex-Environment neu definieren ohne reledmac
\makeatletter
\renewenvironment{theindex}{%
  \ifkorrekturansicht
    \section*{\indexname}%
  \else
    \subsubsection*{Index der erwähnten Entitäten}%
  \fi
  \setlength{\parindent}{0pt}%
  \setlength{\parskip}{0pt plus 0.3pt}%
  \let\item\@idxitem
}{%
  \ifkorrekturansicht\clearpage\fi
}
\makeatother

\IfFileExists{\jobname-pw.ind}{\input{\jobname-pw.ind}}{}

% Quellenangabe nur in der Leseansicht
\ifkorrekturansicht\else
% Fallback-Definitionen, falls die .tex-Datei \titel etc. nicht gesetzt hat
\providecommand{\titel}{}
\providecommand{\editorInnen}{}
\providecommand{\dateiname}{\jobname}

\vspace{3cm}

\vfill

\footnotesize
\textsc{Quelle}: \titel. Herausgegeben von {\editorInnen}. In: \emph{Arthur Schnitzler: Briefwechsel mit Autorinnen und Autoren}.
 Digitale Edition, https://schnitzler-briefe.acdh.oeaw.ac.at/{\dateiname}.html (Stand \today)
\fi

\end{document}


