%% latex-korrekturansicht-vorspann.tex
%% Vorspann für die Korrekturansicht.
%% Lädt die gemeinsame Datei latex-vorspann.tex mit gesetztem Schalter.

\newif\ifkorrekturansicht
\korrekturansichttrue

\input{../tex-inputs/latex-vorspann}


\section[Arthur Schnitzler an Sigmund Freud, 6. 5. 1906, Briefentwurf]{L03815 Arthur Schnitzler an Sigmund Freud, 6. 5. 1906,
               Briefentwurf}
\nopagebreak\mylabel{L03815v}
\rehead{ }\normalsize\beginnumbering\briefempfaengerindex{Freud, Sigmund@\textsc{Freud, Sigmund}!zzzSchnitzler, Arthur@\emph{von Arthur Schnitzler}!1906-05-061@{6. 5. 1906}|(be}
\toendnotes[C]{\smallbreak\pagebreak[2]}\Standort{CUL, Schnitzler, B 31A.}
\physDesc{Kartenbrief, 1 Blatt, 2 Seiten, 453 Zeichen
\newline{}Handschrift: schwarze Tinte, deutsche Kurrent}\toendnotes[C]{\smallbreak}
\pstart
           {\pb}\textcolor{gray}{\textbf{Dr. Arthur Schnitzler}}\hfill 6/5. 906\pend
           
\pstart
           \textcolor{gray}{\textbf{Wien, XVIII.
                  Spoettelgasse 7\oindex{Edmund-Weiss-Gasse@\textbf{Edmund-Weiß-Gasse}, \emph{R.ST}|pw}.}}\pend
           \vspace{0.5em}
\pstart
           verehrteſter Herr Profeſſor, wenn Sie ſich auch \label{K_L03815-1v}\edtext{perſönlich meiner}{\lemma{\textnormal{\emph{perſönlich meiner}}}\Cendnote{\textnormal{XXXX}}}\label{K_L03815-1}
               kaum mehr eri{\geminationn}ern dürften, erlauben Sie mir doch mich den \label{K_L03815-2v}\edtext{Glückwünſchreden beizugeſellen}{\lemma{\textnormal{\emph{Glückwünſchreden beizugeſellen}}}\Cendnote{\textnormal{Der Briefentwurf befindet sich in Schnitzlers Nachlass (heute in der Cambridge University
                  Library), wurde also nicht verschickt. Ein sehr ähnliches Schreiben anläßlich von Freuds\pwindex{Freud, Sigmund 06.05.1856 – 23.09.1939@\textsc{Freud, Sigmund} (06.05.1856 – 23.09.1939), \emph{Psychoanalytiker/Psychoanalytikerin}|pwk} 50. Geburtstag am 6. 5. 1906 muß aber an
                     diesen gegangen sein, denn er
                  antwortete mit Bezugnahme auf die von Schnitzler auch hier verwendete Formulierung der »Anregungen«, vgl. Sigmund Freud an Arthur Schnitzler, 8. 5. 1906.}}}\label{K_L03815-2}, die
                  heute von Ihnen erscheinen. Ich danke Ihren Schriften ſo mannigfache
               ſtarke und tiefe Anregungen, und Ihr fünfzigſter Geburtstag darf {\pb}mir wohl Gelegenheit bieten, es Ihnen
               zu ſagen und Ihnen die Verſicherung meiner aufrichtigſten wärmſten Verehrung
               darzubringen.\pend
           
\pstart
           Ihr ergebner{\\[\baselineskip]}\spacefill\mbox{Arthur Schnitzler}\pend
           \leftskip=0em{}\selectlanguage{ngerman}\endnumbering\briefempfaengerindex{Freud, Sigmund@\textsc{Freud, Sigmund}!zzzSchnitzler, Arthur@\emph{von Arthur Schnitzler}!1906-05-061@{6. 5. 1906}|)be}\mylabel{L03815h}
\begin{anhang}
\end{anhang}\normalsize

\doendnotes{C}
\bigskip
\vfill

\clearpage

\footnotesize

\lohead{\textsc{register}}

% Definiere theindex-Environment komplett neu ohne reledmac
\makeatletter
\renewenvironment{theindex}{%
  \section*{\indexname}%
  \setlength{\parindent}{0pt}%
  \setlength{\parskip}{0pt plus 0.3pt}%
  \let\item\@idxitem
}{%
  \clearpage
}
\makeatother

\IfFileExists{\jobname-pw.ind}{\input{\jobname-pw.ind}}{}

\end{document}

      