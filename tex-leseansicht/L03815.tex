%% latex-leseansicht-vorspann.tex
%% Vorspann für die Leseansicht.
%% Lädt die gemeinsame Datei latex-vorspann.tex mit nicht gesetztem Schalter.

\newif\ifkorrekturansicht
\korrekturansichtfalse

\input{../tex-inputs/latex-vorspann}


\section[Arthur Schnitzler an Sigmund Freud, 6. 5. 1906, Briefentwurf]{L03815 Arthur Schnitzler an Sigmund Freud, 6. 5. 1906, Briefentwurf}
\nopagebreak\mylabel{L03815v}
\rehead{ }\normalsize\beginnumbering\briefempfaengerindex{Freud, Sigmund@\textsc{Freud, Sigmund}!zzzSchnitzler, Arthur@\emph{von Arthur Schnitzler}!1906-05-061@{6. 5. 1906}|(be}
\toendnotes[C]{\smallbreak\pagebreak[2]}
\correspDesc{Versand  durch Arthur Schnitzler am 6. 5. 1906 in Wien
\newline{}Erhalt  durch Sigmund Freud im Zeitraum [6. 5. 1906
                  – 9. 5. 1906?] in Wien}\toendnotes[C]{\smallbreak}
\Standort{CUL, Schnitzler, B 31A.}
\physDesc{Kartenbrief, 453 Zeichen
\newline{}Handschrift: schwarze Tinte, deutsche Kurrent}
\buchAbdrucke{\weitereDrucke{Luigi Reitani: \emph{Besser sublimiert als verdrängt. In Cambridge entdeckt. Ein unbekannter Brief von Arthur Schnitzler an Sigmund Freud.} In: \emph{Die Presse}, Nr. 13.377, Beilage Spectrum, 3. 10. 1992, S. X.} }\toendnotes[C]{\smallbreak}
\pstart
           {\pb}\textcolor{gray}{\textbf{Dr. Arthur Schnitzler}}\hfill 6/5. 906\pend
           
\pstart
           \textcolor{gray}{\textbf{Wien, XVIII. Spoettelgasse 7\oindex{Wien@\textbf{Wien}!XVIII., Währing@\textbf{XVIII., Währing}!Edmund-Weiß-Gasse@\textbf{Edmund-Weiß-Gasse}, \emph{Straße}|pw}.}}\pend
           \vspace{0.5em}
\pstart
           verehrteſter Herr Profeſſor, wenn Sie ſich auch \label{K_L03815-1v}\edtext{perſönlich meiner kaum mehr eri{\geminationn}ern}{\lemma{\textnormal{\emph{persönlich … erinnern}}}\Cendnote{\textnormal{Schnitzler war knapp sechs Jahre jünger als
                     Freud\pwindex{Freud, Sigmund 6.\,5.\,1856 Pribor – 23.\,9.\,1939 London@\textsc{Freud, Sigmund} (6.\,5.\,1856 Pribor – 23.\,9.\,1939 London), \emph{Psychoanalytiker}|pwk}. Die beiden kannten sich aus der
                  Zeit des Medizinstudiums bzw. aus den Jahren, in denen Schnitzler für seinen Vater\pwindex{Schnitzler, Johann 10.\,4.\,1835 Nagykanizsa – 2.\,5.\,1893 Wien@\textsc{Schnitzler, Johann} (10.\,4.\,1835 Nagykanizsa – 2.\,5.\,1893 Wien), \emph{Laryngologe}|pwkv} die \emph{Internationale
                     Klinische Rundschau}\pwindex{Internationale klinische Rundschau@\emph{Internationale klinische Rundschau}|pwk} herausgab (1897–1894), für die auch Freud\pwindex{Freud, Sigmund 6.\,5.\,1856 Pribor – 23.\,9.\,1939 London@\textsc{Freud, Sigmund} (6.\,5.\,1856 Pribor – 23.\,9.\,1939 London), \emph{Psychoanalytiker}|pwk} ein paar Texte beisteuerte. }}}\label{K_L03815-1}
               dürften, erlauben Sie mir doch mich den \label{K_L03815-11v}\edtext{Glückwünſchreden}{\lemma{\textnormal{\emph{Glückwünschreden}}}\Cendnote{\textnormal{Am 6. 5. 1906 beging Freud\pwindex{Freud, Sigmund 6.\,5.\,1856 Pribor – 23.\,9.\,1939 London@\textsc{Freud, Sigmund} (6.\,5.\,1856 Pribor – 23.\,9.\,1939 London), \emph{Psychoanalytiker}|pwk} seinen 50. Geburtstag.}}}\label{K_L03815-11} beizugeſellen, die
                  heute von Ihnen erscheinen. Ich danke Ihren Schriften ſo mannigfache{ }ſtarke und tiefe \label{K_L03815-2v}\edtext{Anregungen}{\lemma{\textnormal{\emph{Anregungen}}}\Cendnote{\textnormal{Die Briefkarte befindet sich in Schnitzlers Nachlass (heute in der
                        \emph{Cambridge University Library}), wurde
                  also aller Wahrscheinlichkeit nicht verschickt. Es könnte sich um einen
                  Briefentwurf oder eine Briefkopie handeln, wobei beide Textsorten für den Nachlass
                     Schnitzlers unüblich sind. Jedenfalls
                  dürfte das tatsächlich versandte inhaltlich nicht stark abgewichen haben, denn Freud\pwindex{Freud, Sigmund 6.\,5.\,1856 Pribor – 23.\,9.\,1939 London@\textsc{Freud, Sigmund} (6.\,5.\,1856 Pribor – 23.\,9.\,1939 London), \emph{Psychoanalytiker}|pwk} nahm in seinem Antwortschreiben die
                  von Schnitzler verwendete Formulierung der
                  »Anregungen« auf, vgl. XXXX Auszeichnungsfehler: Dokument L03819 nicht gefunden.}}}\label{K_L03815-2}, und Ihr fünfzigſter Geburtstag darf {\pb}mir wohl Gelegenheit bieten, es Ihnen
               zu ſagen und Ihnen die Verſicherung meiner aufrichtigſten wärmſten Verehrung
               darzubringen.\pend
           
\pstart
           Ihr ergebner{\\[\baselineskip]}\spacefill\mbox{Arthur Schnitzler}\pend
           \leftskip=0em{}\selectlanguage{ngerman}\endnumbering\briefempfaengerindex{Freud, Sigmund@\textsc{Freud, Sigmund}!zzzSchnitzler, Arthur@\emph{von Arthur Schnitzler}!1906-05-061@{6. 5. 1906}|)be}\mylabel{L03815h}
\begin{anhang}
\end{anhang}\newcommand{\dateiname}{L03815}\newcommand{\titel}{Arthur Schnitzler an Sigmund Freud, 6. 5. 1906, Briefentwurf}\newcommand{\editorInnen}{Selma Jahnke und Martin Anton Müller}%% latex-leseansicht-abspann.tex
%% Abspann für die Leseansicht.
%% Der Schalter \ifkorrekturansicht ist bereits durch den Vorspann gesetzt.

%% latex-abspann.tex
%% Gemeinsamer Abspann für Korrekturansicht und Leseansicht.
%% Setzt den Schalter \ifkorrekturansicht voraus (gesetzt in den
%% einbindenden Dateien latex-korrekturansicht-abspann.tex bzw.
%% latex-leseansicht-abspann.tex).
%% ---------------------------------------------------------------

\normalsize

% Das esempio-Environment wird nur in der Leseansicht benötigt
\ifkorrekturansicht\else
\newenvironment{esempio}[3]%
{
    \vspace{1.5ex}
    \rlap{\underline{#1}}
    \par
    \setlength{\parindent}{0cm}
    \nopagebreak
    \leftskip=#2cm
    \rightskip=#3cm
}
{
    \par
}
\fi

\doendnotes{C}
\bigskip
\vfill

\clearpage

\footnotesize

\ifkorrekturansicht
  \lohead{\textsc{register}}
\fi

% theindex-Environment neu definieren ohne reledmac
\makeatletter
\renewenvironment{theindex}{%
  \ifkorrekturansicht
    \section*{\indexname}%
  \else
    \subsubsection*{Index der erwähnten Entitäten}%
  \fi
  \setlength{\parindent}{0pt}%
  \setlength{\parskip}{0pt plus 0.3pt}%
  \let\item\@idxitem
}{%
  \ifkorrekturansicht\clearpage\fi
}
\makeatother

\IfFileExists{\jobname-pw.ind}{\input{\jobname-pw.ind}}{}

% Quellenangabe nur in der Leseansicht
\ifkorrekturansicht\else
% Fallback-Definitionen, falls die .tex-Datei \titel etc. nicht gesetzt hat
\providecommand{\titel}{}
\providecommand{\editorInnen}{}
\providecommand{\dateiname}{\jobname}

\vspace{3cm}

\vfill

\footnotesize
\textsc{Quelle}: \titel. Herausgegeben von {\editorInnen}. In: \emph{Arthur Schnitzler: Briefwechsel mit Autorinnen und Autoren}.
 Digitale Edition, https://schnitzler-briefe.acdh.oeaw.ac.at/{\dateiname}.html (Stand \today)
\fi

\end{document}


