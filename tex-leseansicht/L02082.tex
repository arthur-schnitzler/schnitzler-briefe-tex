%% latex-leseansicht-vorspann.tex
%% Vorspann für die Leseansicht.
%% Lädt die gemeinsame Datei latex-vorspann.tex mit nicht gesetztem Schalter.

\newif\ifkorrekturansicht
\korrekturansichtfalse

\input{../tex-inputs/latex-vorspann}


\section[Arthur Schnitzler an Richard Beer-Hofmann, 5. 8. 1912]{L02082 Arthur Schnitzler an Richard Beer-Hofmann, 5. 8. 1912}
\nopagebreak\mylabel{L02082v}
\rehead{ }\normalsize\beginnumbering\briefempfaengerindex{Beer-Hofmann, Richard@\textsc{Beer-Hofmann, Richard}!zzzSchnitzler, Arthur@\emph{von Arthur Schnitzler}!1912-08-051@{5. 8. 1912}|(be}
\toendnotes[C]{\smallbreak\pagebreak[2]}
\correspDesc{Versand  durch Arthur Schnitzler am 5. 8. 1912 in Brijuni
\newline{}Erhalt  durch Richard Beer-Hofmann im Zeitraum [6. 8. 1912
                  – 10. 8. 1912?] in Sankt Moritz}\toendnotes[C]{\smallbreak}
\Standort{YCGL, MSS 31.}
\physDesc{Bildpostkarte, 305 Zeichen
\newline{}Handschrift: Bleistift, deutsche Kurrent
\newline{}Versand: Stempel: »\nobreak{}\oindex{Brijuni@\textbf{Brijuni}|pwk}Brioni, 5. 8. 1\textcolor{gray}{2}\nobreak{}«.  }\toendnotes[C]{\smallbreak}\pstart{}\textsc{{\pb}Herrn Dr.}\pend{}\pstart{}\textsc{Richard Beerhofmann}\pend{}\pstart{}\textsc{St Moritz\oindex{St. Moritz@\textbf{St. Moritz}|pw}}\pend{}\pstart{}\textsc{im Engadin\oindex{Engadin@\textbf{Engadin}, \emph{Tal}|pw}}\pend{}\pstart{}\textsc{Waldhaus\oindex{Hotel Waldhaus am See@\textbf{Hotel Waldhaus am See}, \emph{Hotel}|pw}.}\pend{}{\bigskip}
\pstart
           \noindent{}\centering{}{\pb}\textcolor{gray}{\textbf{Insel Brioni i. d. Adria\oindex{Brijuni@\textbf{Brijuni}|pw}.}}\pend
           
\pstart
           \centering{}\textcolor{gray}{\textbf{Val Catena\oindex{Val Catena@\textbf{Val Catena}, \emph{Tal}|pw}.}}\pend
           \vspace{1em}
\pstart
           \noindent{}{\pb}lieber Richard, unſer Plan iſt am 20 od
                  21. über die ital Seen\oindex{Italien@\textbf{Italien}|pw} nach \textsc{Sils Maria}\oindex{Sils im Engadin/Segl@\textbf{Sils im Engadin/Segl}, \emph{Verwaltungsgebiet}|pw}, dort bis Ende Auguſt, da{\geminationn}{ }\textsc{München}\oindex{München@\textbf{München}|pw}, (\textsc{Tutzing}\oindex{Tutzing@\textbf{Tutzing}, \emph{Hauptstadt}|pw}) \textsc{circa} 8 Tage; u \textsc{direct} von
               dort Wien\oindex{Wien@\textbf{Wien}, \emph{Verwaltungsgebiet}|pw}. Vielleicht trifft man{ }ſich in München\oindex{München@\textbf{München}|pw} (\label{K_L02082-1v}\edtext{Sitze beſorgt?}{\lemma{\textnormal{\emph{Sitze besorgt?}}}\Cendnote{\textnormal{Siehe XXXX Auszeichnungsfehler: Dokument L00478 nicht gefunden, XXXX Auszeichnungsfehler: Dokument L02554 nicht gefunden. }}}\label{K_L02082-1}) Wie lange bleibt Kaufma{\geminationn}\pwindex{Kaufmann, Arthur 4.\,4.\,1872 Iași – 25.\,7.\,1938 Wien@\textsc{Kaufmann, Arthur} (4.\,4.\,1872 Iași – 25.\,7.\,1938 Wien), \emph{Rechtswissenschaftler, Privatgelehrte, Privatier}|pw} im Engadin\oindex{Engadin@\textbf{Engadin}, \emph{Tal}|pw}?\pend
           
\pstart
           Herzlichſt{\\[\baselineskip]}Ihr \spacefill\mbox{A.}\pend
           \leftskip=0em{}\selectlanguage{ngerman}\endnumbering\briefempfaengerindex{Beer-Hofmann, Richard@\textsc{Beer-Hofmann, Richard}!zzzSchnitzler, Arthur@\emph{von Arthur Schnitzler}!1912-08-051@{5. 8. 1912}|)be}\mylabel{L02082h}  \newcommand{\dateiname}{L02082}\newcommand{\titel}{Arthur Schnitzler an Richard Beer-Hofmann, 5. 8. 1912}\newcommand{\editorInnen}{Martin Anton Müller und Gerd-Hermann Susen}%% latex-leseansicht-abspann.tex
%% Abspann für die Leseansicht.
%% Der Schalter \ifkorrekturansicht ist bereits durch den Vorspann gesetzt.

%% latex-abspann.tex
%% Gemeinsamer Abspann für Korrekturansicht und Leseansicht.
%% Setzt den Schalter \ifkorrekturansicht voraus (gesetzt in den
%% einbindenden Dateien latex-korrekturansicht-abspann.tex bzw.
%% latex-leseansicht-abspann.tex).
%% ---------------------------------------------------------------

\normalsize

% Das esempio-Environment wird nur in der Leseansicht benötigt
\ifkorrekturansicht\else
\newenvironment{esempio}[3]%
{
    \vspace{1.5ex}
    \rlap{\underline{#1}}
    \par
    \setlength{\parindent}{0cm}
    \nopagebreak
    \leftskip=#2cm
    \rightskip=#3cm
}
{
    \par
}
\fi

\doendnotes{C}
\bigskip
\vfill

\clearpage

\footnotesize

\ifkorrekturansicht
  \lohead{\textsc{register}}
\fi

% theindex-Environment neu definieren ohne reledmac
\makeatletter
\renewenvironment{theindex}{%
  \ifkorrekturansicht
    \section*{\indexname}%
  \else
    \subsubsection*{Index der erwähnten Entitäten}%
  \fi
  \setlength{\parindent}{0pt}%
  \setlength{\parskip}{0pt plus 0.3pt}%
  \let\item\@idxitem
}{%
  \ifkorrekturansicht\clearpage\fi
}
\makeatother

\IfFileExists{\jobname-pw.ind}{\input{\jobname-pw.ind}}{}

% Quellenangabe nur in der Leseansicht
\ifkorrekturansicht\else
% Fallback-Definitionen, falls die .tex-Datei \titel etc. nicht gesetzt hat
\providecommand{\titel}{}
\providecommand{\editorInnen}{}
\providecommand{\dateiname}{\jobname}

\vspace{3cm}

\vfill

\footnotesize
\textsc{Quelle}: \titel. Herausgegeben von {\editorInnen}. In: \emph{Arthur Schnitzler: Briefwechsel mit Autorinnen und Autoren}.
 Digitale Edition, https://schnitzler-briefe.acdh.oeaw.ac.at/{\dateiname}.html (Stand \today)
\fi

\end{document}


