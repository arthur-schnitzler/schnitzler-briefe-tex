%% latex-leseansicht-vorspann.tex
%% Vorspann für die Leseansicht.
%% Lädt die gemeinsame Datei latex-vorspann.tex mit nicht gesetztem Schalter.

\newif\ifkorrekturansicht
\korrekturansichtfalse

\input{../tex-inputs/latex-vorspann}


\section[Paul Goldmann an Arthur Schnitzler, 12. 1. [1895]]{L02727 Paul Goldmann an Arthur Schnitzler, 12. 1. [1895]}
\nopagebreak\mylabel{L02727v}
\rehead{ }\normalsize\beginnumbering\briefempfaengerindex{Schnitzler, Arthur@\textsc{Schnitzler, Arthur}!zzzGoldmann, Paul@\emph{von Paul Goldmann}!1895-01-121@{12. 1. [1895]}|(be}
\toendnotes[C]{\smallbreak\pagebreak[2]}
\correspDesc{Versand  durch Paul Goldmann am 12. 1. [1895] in Paris
\newline{}Erhalt  durch Arthur Schnitzler im Zeitraum [13. 1. 1895
                  – 17. 1. 1895?] in Wien}\toendnotes[C]{\smallbreak}
\Standort{DLA, A:Schnitzler, HS.NZ85.1.3165.}
\physDesc{Brief, 1 Blatt, 3 Seiten, 1272 Zeichen
\newline{}Handschrift: schwarze Tinte, deutsche Kurrent
\newline{}Schnitzler: 1) mit Bleistift das Jahr »95« vermerkt  2) mit rotem Buntstift eine Unterstreichung}\toendnotes[C]{\smallbreak}
\pstart
           {\pb}\textcolor{gray}{\textbf{\textbf{Frankfurter Zeitung\orgindex{Frankfurter Zeitung@Frankfurter Zeitung|pw}}}}\pend
           
\pstart
           \textcolor{gray}{\textbf{(\begin{otherlanguage}{french}Gazette de Francfort\end{otherlanguage}\orgindex{Frankfurter Zeitung@Frankfurter Zeitung|pw}).}}\pend
           
\pstart
           \textcolor{gray}{\textbf{\textbf{\begin{otherlanguage}{french}Fondateur M. L.
                              Sonnemann\pwindex{Sonnemann, Leopold 29.\,10.\,1831 Höchberg – 30.\,10.\,1909 Frankfurt am Main@\textsc{Sonnemann, Leopold} (29.\,10.\,1831 Höchberg – 30.\,10.\,1909 Frankfurt am Main), \emph{Journalist, Herausgeber}|pw}\end{otherlanguage}.}}}\pend
           
\pstart
           \begin{otherlanguage}{french}\textcolor{gray}{\textbf{Journal politique, financier,}}\end{otherlanguage}\hfill \textsc{Paris\oindex{Paris@\textbf{Paris}, \emph{Hauptstadt}|pw}}, 12. Januar.\pend
           
\pstart
           \begin{otherlanguage}{french}\textcolor{gray}{\textbf{commercial et littéraire.}}\end{otherlanguage}\pend
           
\pstart
           \begin{otherlanguage}{french}\textcolor{gray}{\textbf{\textbf{Paraissant trois fois par jour.}}}\end{otherlanguage}\pend
           
\pstart
           \begin{otherlanguage}{french}\textcolor{gray}{\textbf{\textbf{Bureau à Paris\oindex{Paris@\textbf{Paris}, \emph{Hauptstadt}|pw}:}}}\end{otherlanguage}\pend
           
\pstart
           \begin{otherlanguage}{french}\textcolor{gray}{\textbf{\textbf{24. Rue Feydeau\oindex{rue Feydeau@\textbf{rue Feydeau}, \emph{Straße}|pw}.}}}\end{otherlanguage}\pend
           
\pstart{}Mein lieber Freund,\pend\vspace{0.5em}
\pstart
           \textsc{Lalo\pwindex{Lalo, Pierre 6.\,9.\,1866 Puteaux – 9.\,6.\,1943 Paris@\textsc{Lalo, Pierre} (6.\,9.\,1866 Puteaux – 9.\,6.\,1943 Paris), \emph{Kritiker}|pw}}, vom »\textsc{Journal des Débats\orgindex{Journal des débats@Journal des débats|pw}}«, war geſtern bei mir. »Sterben\pwindex{Schnitzler, Arthur 15.\,5.\,1862 Wien – 21.\,10.\,1931 ebd.@\textsc{Schnitzler, Arthur} (15.\,5.\,1862 Wien – 21.\,10.\,1931 ebd.), \emph{Schriftsteller, Mediziner}!Sterben. Novelle@\strich\emph{Sterben. Novelle}|pw}« hat ihm ungemein gefallen, \textsc{Richards\pwindex{Beer-Hofmann, Richard 11.\,7.\,1866 Wien – 26.\,9.\,1945 New York City@\textsc{Beer-Hofmann, Richard} (11.\,7.\,1866 Wien – 26.\,9.\,1945 New York City), \emph{Schriftsteller}|pw}}{ }Buch\pwindex{Beer-Hofmann, Richard 11.\,7.\,1866 Wien – 26.\,9.\,1945 New York City@\textsc{Beer-Hofmann, Richard} (11.\,7.\,1866 Wien – 26.\,9.\,1945 New York City), \emph{Schriftsteller}!Novellen@\strich\emph{Novellen}|pwv} weniger
               (ſags ihm aber nicht). Er hat \substVorne{}\textsuperscript{e}\substDazwischen{}b\substHinten{}eſtimmt verſprochen, über Euch zu \label{K_L02727-1v}\edtext{ſchreiben}{\lemma{\textnormal{\emph{schreiben}}}\Cendnote{\textnormal{Nicht
                  über Richard Beer-Hofmann\pwindex{Beer-Hofmann, Richard 11.\,7.\,1866 Wien – 26.\,9.\,1945 New York City@\textsc{Beer-Hofmann, Richard} (11.\,7.\,1866 Wien – 26.\,9.\,1945 New York City), \emph{Schriftsteller}|pwk}, jedoch über Schnitzler und seine Novelle \emph{Sterben}\pwindex{Schnitzler, Arthur 15.\,5.\,1862 Wien – 21.\,10.\,1931 ebd.@\textsc{Schnitzler, Arthur} (15.\,5.\,1862 Wien – 21.\,10.\,1931 ebd.), \emph{Schriftsteller, Mediziner}!Sterben. Novelle@\strich\emph{Sterben. Novelle}|pwk} schrieb Pierre
                     Lalo\pwindex{Lalo, Pierre 6.\,9.\,1866 Puteaux – 9.\,6.\,1943 Paris@\textsc{Lalo, Pierre} (6.\,9.\,1866 Puteaux – 9.\,6.\,1943 Paris), \emph{Kritiker}|pwk} am 21. 3. 1895: P. L. [ = Pierre Lalo\pwindex{Lalo, Pierre 6.\,9.\,1866 Puteaux – 9.\,6.\,1943 Paris@\textsc{Lalo, Pierre} (6.\,9.\,1866 Puteaux – 9.\,6.\,1943 Paris), \emph{Kritiker}|pwk}]: \emph{Au jour le jour. M. Arthur
                           Schnitzler}\pwindex{Lalo, Pierre 6.\,9.\,1866 Puteaux – 9.\,6.\,1943 Paris@\textsc{Lalo, Pierre} (6.\,9.\,1866 Puteaux – 9.\,6.\,1943 Paris), \emph{Kritiker}!Au jour le jour. M. Arthur Schnitzler@\strich\emph{Au jour le jour. M. Arthur Schnitzler}|pwk}. In: \emph{Journal des débats}\pwindex{Journal des débats. Politiques et littéraires@\emph{Journal des débats. Politiques et littéraires}|pwk}, Jg. 107,
                        21. 3. 1895, S. 1.}}}\label{K_L02727-1}. Ob ers
               halten wird???\pend
           
\pstart
           Bitte,{ }ſchick’ mir \textsc{Torresanis\pwindex{Torresani-Lanzenfeld, Carl von 19.\,4.\,1846 Mailand – 16.\,4.\,1907 Nago-Torbole@\textsc{Torresani-Lanzenfeld, Carl von} (19.\,4.\,1846 Mailand – 16.\,4.\,1907 Nago-Torbole), \emph{Schriftsteller, Offizier}|pw}} Adreſſe.\pend
           
\pstart
           Hat Frl. \textsc{Sandrock\pwindex{Sandrock, Adele 19.\,8.\,1863 Rotterdam – 30.\,8.\,1937 Berlin@\textsc{Sandrock, Adele} (19.\,8.\,1863 Rotterdam – 30.\,8.\,1937 Berlin), \emph{Schauspielerin}|pw}} meine Briefe erhalten?\pend
           
\pstart
           Franzoſ\oindex{Frankreich@\textbf{Frankreich}|pwv}en\pwindex{Donnay, Maurice 12.\,10.\,1859 Paris – 31.\,3.\,1945 ebd.@\textsc{Donnay, Maurice} (12.\,10.\,1859 Paris – 31.\,3.\,1945 ebd.), \emph{Schriftsteller}|pwv}\pwindex{Hervieu, Paul Ernest 2.\,9.\,1857 Neuilly-sur-Seine – 25.\,10.\,1915 Paris@\textsc{Hervieu, Paul Ernest} (2.\,9.\,1857 Neuilly-sur-Seine – 25.\,10.\,1915 Paris), \emph{Schriftsteller}|pwv}\pwindex{Esparbès, Georges d' 24.\,3.\,1863 Valence – 25.\,6.\,1944 Saint-Germain-en-Laye@\textsc{Esparbès, Georges d'} (24.\,3.\,1863 Valence – 25.\,6.\,1944 Saint-Germain-en-Laye), \emph{Schriftsteller}|pwv}\pwindex{Hermant, Abel 3.\,2.\,1862 Paris – 28.\,9.\,1950@\textsc{Hermant, Abel} (3.\,2.\,1862 Paris – 28.\,9.\,1950), \emph{Schriftsteller}|pwv}\pwindex{Lavedan, Henri Léon 9.\,4.\,1859 Orléans – 4.\,9.\,1940 Paris@\textsc{Lavedan, Henri Léon} (9.\,4.\,1859 Orléans – 4.\,9.\,1940 Paris), \emph{Schriftsteller, Journalist}|pwv}\pwindex{Vandérem, Fernand 24.\,6.\,1864 Paris – 11.\,3.\,1939 ebd.@\textsc{Vandérem, Fernand} (24.\,6.\,1864 Paris – 11.\,3.\,1939 ebd.), \emph{Schriftsteller}|pwv}\pwindex{Capus, Alfred 25.\,11.\,1858 Aix-en-Provence – 1.\,11.\,1922 Neuilly-sur-Seine@\textsc{Capus, Alfred} (25.\,11.\,1858 Aix-en-Provence – 1.\,11.\,1922 Neuilly-sur-Seine), \emph{Schriftsteller, Journalist}|pwv}\pwindex{Nion, François de 13.\,8.\,1854 Pierrefonds – 11.\,1.\,1923 Paris@\textsc{Nion, François de} (13.\,8.\,1854 Pierrefonds – 11.\,1.\,1923 Paris), \emph{Schriftsteller}|pwv}\pwindex{Fleurigny, Henry de 10.\,3.\,1849 Paris – 10.\,10.\,1916 Nizza@\textsc{Fleurigny, Henry de} (10.\,3.\,1849 Paris – 10.\,10.\,1916 Nizza), \emph{Schriftsteller, Militär}|pwv}\pwindex{Courteline, Georges 25.\,6.\,1858 Tours – 25.\,6.\,1929 Paris@\textsc{Courteline, Georges} (25.\,6.\,1858 Tours – 25.\,6.\,1929 Paris), \emph{Schriftsteller}|pwv}\pwindex{Ajalbert, Jean 10.\,6.\,1863 Clichy – 14.\,1.\,1947 Cahot@\textsc{Ajalbert, Jean} (10.\,6.\,1863 Clichy – 14.\,1.\,1947 Cahot), \emph{Schriftsteller, Kritiker, Rechtsanwalt}|pwv}\pwindex{Xanrof, Léon 9.\,12.\,1867 Paris – 17.\,5.\,1953 ebd.@\textsc{Xanrof, Léon} (9.\,12.\,1867 Paris – 17.\,5.\,1953 ebd.), \emph{Dramatiker, Humorist}|pwv}\pwindex{Renard, Jules 22.\,2.\,1864 Châlons-du-Maine – 22.\,3.\,1910 Paris@\textsc{Renard, Jules} (22.\,2.\,1864 Châlons-du-Maine – 22.\,3.\,1910 Paris), \emph{Schriftsteller}|pwv}\pwindex{Jules-Bois, Henri Antoine 29.\,9.\,1868 Marseille – 2.\,7.\,1943 New York City@\textsc{Jules-Bois, Henri Antoine} (29.\,9.\,1868 Marseille – 2.\,7.\,1943 New York City), \emph{Schriftsteller}|pwv}\pwindex{Case, Jules 24.\,6.\,1854 Sens – 15.\,12.\,1931 Straßburg@\textsc{Case, Jules} (24.\,6.\,1854 Sens – 15.\,12.\,1931 Straßburg), \emph{Schriftsteller, Journalist, Literaturkritiker}|pwv}\pwindex{Adam, Paul 6.\,12.\,1862 Paris – 2.\,1.\,1920 ebd.@\textsc{Adam, Paul} (6.\,12.\,1862 Paris – 2.\,1.\,1920 ebd.), \emph{Schriftsteller, Kunstkritiker}|pwv}, die kleine
               Geſchichten{ }ſchreiben,{ }ſind: \textsc{Maurice Donnay\pwindex{Donnay, Maurice 12.\,10.\,1859 Paris – 31.\,3.\,1945 ebd.@\textsc{Donnay, Maurice} (12.\,10.\,1859 Paris – 31.\,3.\,1945 ebd.), \emph{Schriftsteller}|pw}}, \textsc{Paul Hervieu\pwindex{Hervieu, Paul Ernest 2.\,9.\,1857 Neuilly-sur-Seine – 25.\,10.\,1915 Paris@\textsc{Hervieu, Paul Ernest} (2.\,9.\,1857 Neuilly-sur-Seine – 25.\,10.\,1915 Paris), \emph{Schriftsteller}|pw}}, {\pb}\textsc{Georges d’Esparbès\pwindex{Esparbès, Georges d' 24.\,3.\,1863 Valence – 25.\,6.\,1944 Saint-Germain-en-Laye@\textsc{Esparbès, Georges d'} (24.\,3.\,1863 Valence – 25.\,6.\,1944 Saint-Germain-en-Laye), \emph{Schriftsteller}|pw}}, \textsc{Abel Hermant\pwindex{Hermant, Abel 3.\,2.\,1862 Paris – 28.\,9.\,1950@\textsc{Hermant, Abel} (3.\,2.\,1862 Paris – 28.\,9.\,1950), \emph{Schriftsteller}|pw}}, \textsc{\strikeout{Hen\pwindex{Lavedan, Henri Léon 9.\,4.\,1859 Orléans – 4.\,9.\,1940 Paris@\textsc{Lavedan, Henri Léon} (9.\,4.\,1859 Orléans – 4.\,9.\,1940 Paris), \emph{Schriftsteller, Journalist}|pwv}}}{ }\textsc{Henri \strikeout{La}
                     Lavedan\pwindex{Lavedan, Henri Léon 9.\,4.\,1859 Orléans – 4.\,9.\,1940 Paris@\textsc{Lavedan, Henri Léon} (9.\,4.\,1859 Orléans – 4.\,9.\,1940 Paris), \emph{Schriftsteller, Journalist}|pw}}, \textsc{Ferdinand Vanderem\pwindex{Vandérem, Fernand 24.\,6.\,1864 Paris – 11.\,3.\,1939 ebd.@\textsc{Vandérem, Fernand} (24.\,6.\,1864 Paris – 11.\,3.\,1939 ebd.), \emph{Schriftsteller}|pw}}, \textsc{Alfred Capus\pwindex{Capus, Alfred 25.\,11.\,1858 Aix-en-Provence – 1.\,11.\,1922 Neuilly-sur-Seine@\textsc{Capus, Alfred} (25.\,11.\,1858 Aix-en-Provence – 1.\,11.\,1922 Neuilly-sur-Seine), \emph{Schriftsteller, Journalist}|pw}}, \textsc{François de Nion\pwindex{Nion, François de 13.\,8.\,1854 Pierrefonds – 11.\,1.\,1923 Paris@\textsc{Nion, François de} (13.\,8.\,1854 Pierrefonds – 11.\,1.\,1923 Paris), \emph{Schriftsteller}|pw}, Henry de Fleurigny\pwindex{Fleurigny, Henry de 10.\,3.\,1849 Paris – 10.\,10.\,1916 Nizza@\textsc{Fleurigny, Henry de} (10.\,3.\,1849 Paris – 10.\,10.\,1916 Nizza), \emph{Schriftsteller, Militär}|pw}}, \textsc{Georges Courteline\pwindex{Courteline, Georges 25.\,6.\,1858 Tours – 25.\,6.\,1929 Paris@\textsc{Courteline, Georges} (25.\,6.\,1858 Tours – 25.\,6.\,1929 Paris), \emph{Schriftsteller}|pw}}, \textsc{Jean Ajalbert\pwindex{Ajalbert, Jean 10.\,6.\,1863 Clichy – 14.\,1.\,1947 Cahot@\textsc{Ajalbert, Jean} (10.\,6.\,1863 Clichy – 14.\,1.\,1947 Cahot), \emph{Schriftsteller, Kritiker, Rechtsanwalt}|pw}, L. Xanrof\pwindex{Xanrof, Léon 9.\,12.\,1867 Paris – 17.\,5.\,1953 ebd.@\textsc{Xanrof, Léon} (9.\,12.\,1867 Paris – 17.\,5.\,1953 ebd.), \emph{Dramatiker, Humorist}|pw}}, \textsc{Jules Renard\pwindex{Renard, Jules 22.\,2.\,1864 Châlons-du-Maine – 22.\,3.\,1910 Paris@\textsc{Renard, Jules} (22.\,2.\,1864 Châlons-du-Maine – 22.\,3.\,1910 Paris), \emph{Schriftsteller}|pw}}, \textsc{Jules Bois\pwindex{Jules-Bois, Henri Antoine 29.\,9.\,1868 Marseille – 2.\,7.\,1943 New York City@\textsc{Jules-Bois, Henri Antoine} (29.\,9.\,1868 Marseille – 2.\,7.\,1943 New York City), \emph{Schriftsteller}|pw}}, \textsc{Jules Case\pwindex{Case, Jules 24.\,6.\,1854 Sens – 15.\,12.\,1931 Straßburg@\textsc{Case, Jules} (24.\,6.\,1854 Sens – 15.\,12.\,1931 Straßburg), \emph{Schriftsteller, Journalist, Literaturkritiker}|pw}}, \textsc{Paul Adam\pwindex{Adam, Paul 6.\,12.\,1862 Paris – 2.\,1.\,1920 ebd.@\textsc{Adam, Paul} (6.\,12.\,1862 Paris – 2.\,1.\,1920 ebd.), \emph{Schriftsteller, Kunstkritiker}|pw}}{ }\textsc{etc}.\pend
           
\pstart
           Wenn Du damit nicht genug haſt, kannſt Du mehr bekommen. Meiſtens{ }ſind{ }ſie recht
               mäßig. Die gegenwärtig aufgehende Saat iſt nicht gut gerathen. Außer den verwöhnten
                  Mode-Pinſeln\pwindex{Prévost, Marcel 1.\,5.\,1862 Paris – 8.\,4.\,1941 Vianne@\textsc{Prévost, Marcel} (1.\,5.\,1862 Paris – 8.\,4.\,1941 Vianne), \emph{Schriftsteller}|pwv}\pwindex{Hermant, Abel 3.\,2.\,1862 Paris – 28.\,9.\,1950@\textsc{Hermant, Abel} (3.\,2.\,1862 Paris – 28.\,9.\,1950), \emph{Schriftsteller}|pwv}\pwindex{Vandérem, Fernand 24.\,6.\,1864 Paris – 11.\,3.\,1939 ebd.@\textsc{Vandérem, Fernand} (24.\,6.\,1864 Paris – 11.\,3.\,1939 ebd.), \emph{Schriftsteller}|pwv} (\textsc{Prevost\pwindex{Prévost, Marcel 1.\,5.\,1862 Paris – 8.\,4.\,1941 Vianne@\textsc{Prévost, Marcel} (1.\,5.\,1862 Paris – 8.\,4.\,1941 Vianne), \emph{Schriftsteller}|pw}}, \textsc{Hermant\pwindex{Hermant, Abel 3.\,2.\,1862 Paris – 28.\,9.\,1950@\textsc{Hermant, Abel} (3.\,2.\,1862 Paris – 28.\,9.\,1950), \emph{Schriftsteller}|pw}, Vanderem\pwindex{Vandérem, Fernand 24.\,6.\,1864 Paris – 11.\,3.\,1939 ebd.@\textsc{Vandérem, Fernand} (24.\,6.\,1864 Paris – 11.\,3.\,1939 ebd.), \emph{Schriftsteller}|pw}}) kann man{ }ſie zum Überſetzen zweifellos {\pb}billig, meiſt umſonſt bekommen. Man{ }ſchreibt ihnen: \label{K_L02727-2v}\edtext{\textsc{\begin{otherlanguage}{french}Nous serions très-heureux d’obtenir l’autorisation de
                     traduire {\dotssix} Cela servirait comme échantillon de vos
                     œuvres pour vous introduire auprès du public autrich\oindex{Österreich@\textbf{Österreich}|pw}ièn.\end{otherlanguage}}}{\lemma{\textnormal{\emph{Nous … autrichièn.}}}\Cendnote{\textnormal{französisch: Wir würden uns sehr freuen,
                  wenn wir die Erlaubnis bekämen, {\dotssix} zu übersetzen. Dies
                  würde als Kostprobe Ihrer Werke dienen, um Sie dem österreich\oindex{Österreich@\textbf{Österreich}|pwk}ischen Publikum bekannt zu machen.}}}\label{K_L02727-2} So
               natürlich nur den Unbekannten. Die Bekannten{ }ſetzen voraus, daß man in Wien\oindex{Wien@\textbf{Wien}, \emph{Verwaltungsgebiet}|pw} nichts mehr lieſt, als{ }ſie. Oder aber man{ }ſchreibt gar nicht. Wer kümmert{ }ſich in \textsc{Paris\oindex{Paris@\textbf{Paris}, \emph{Hauptstadt}|pw}} um die \label{K_L02727-3v}\edtext{Allgemeine Zeitung\orgindex{Wiener Allgemeine Zeitung@Wiener Allgemeine Zeitung|pw}}{\lemma{\textnormal{\emph{Allgemeine Zeitung}}}\Cendnote{\textnormal{Seit Oktober 1894 war Felix Salten\pwindex{Salten, Felix 6.\,9.\,1869 Budapest – 8.\,10.\,1945 Zürich@\textsc{Salten, Felix} (6.\,9.\,1869 Budapest – 8.\,10.\,1945 Zürich), \emph{Schriftsteller, Journalist, Chefredakteur}|pwk} bei der \emph{Wiener Allgemeinen Zeitung}\orgindex{Wiener Allgemeine Zeitung@Wiener Allgemeine Zeitung|pwk} engagiert, was einen möglichen
                  Hintergrund für die Anfrage darstellt. Ob Schnitzler überlegte, sich selbst durch Übersetzungen einen Verdienst zu
                  verschaffen, ist ungewiss.}}}\label{K_L02727-3}?\pend
           
\pstart
           Herzlichſt {\\[\baselineskip]}Dein {\\[\baselineskip]}\spacefill\mbox{Paul Goldmnn}\pend
           \leftskip=0em{}\selectlanguage{ngerman}\endnumbering\briefempfaengerindex{Schnitzler, Arthur@\textsc{Schnitzler, Arthur}!zzzGoldmann, Paul@\emph{von Paul Goldmann}!1895-01-121@{12. 1. [1895]}|)be}\mylabel{L02727h}  \newcommand{\dateiname}{L02727}\newcommand{\titel}{Paul Goldmann an Arthur Schnitzler, 12. 1. [1895]}\newcommand{\editorInnen}{Martin Anton Müller und Laura Untner}%% latex-leseansicht-abspann.tex
%% Abspann für die Leseansicht.
%% Der Schalter \ifkorrekturansicht ist bereits durch den Vorspann gesetzt.

%% latex-abspann.tex
%% Gemeinsamer Abspann für Korrekturansicht und Leseansicht.
%% Setzt den Schalter \ifkorrekturansicht voraus (gesetzt in den
%% einbindenden Dateien latex-korrekturansicht-abspann.tex bzw.
%% latex-leseansicht-abspann.tex).
%% ---------------------------------------------------------------

\normalsize

% Das esempio-Environment wird nur in der Leseansicht benötigt
\ifkorrekturansicht\else
\newenvironment{esempio}[3]%
{
    \vspace{1.5ex}
    \rlap{\underline{#1}}
    \par
    \setlength{\parindent}{0cm}
    \nopagebreak
    \leftskip=#2cm
    \rightskip=#3cm
}
{
    \par
}
\fi

\doendnotes{C}
\bigskip
\vfill

\clearpage

\footnotesize

\ifkorrekturansicht
  \lohead{\textsc{register}}
\fi

% theindex-Environment neu definieren ohne reledmac
\makeatletter
\renewenvironment{theindex}{%
  \ifkorrekturansicht
    \section*{\indexname}%
  \else
    \subsubsection*{Index der erwähnten Entitäten}%
  \fi
  \setlength{\parindent}{0pt}%
  \setlength{\parskip}{0pt plus 0.3pt}%
  \let\item\@idxitem
}{%
  \ifkorrekturansicht\clearpage\fi
}
\makeatother

\IfFileExists{\jobname-pw.ind}{\input{\jobname-pw.ind}}{}

% Quellenangabe nur in der Leseansicht
\ifkorrekturansicht\else
% Fallback-Definitionen, falls die .tex-Datei \titel etc. nicht gesetzt hat
\providecommand{\titel}{}
\providecommand{\editorInnen}{}
\providecommand{\dateiname}{\jobname}

\vspace{3cm}

\vfill

\footnotesize
\textsc{Quelle}: \titel. Herausgegeben von {\editorInnen}. In: \emph{Arthur Schnitzler: Briefwechsel mit Autorinnen und Autoren}.
 Digitale Edition, https://schnitzler-briefe.acdh.oeaw.ac.at/{\dateiname}.html (Stand \today)
\fi

\end{document}


