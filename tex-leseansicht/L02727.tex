%% latex-korrekturansicht-vorspann.tex
%% Vorspann für die Korrekturansicht.
%% Lädt die gemeinsame Datei latex-vorspann.tex mit gesetztem Schalter.

\newif\ifkorrekturansicht
\korrekturansichttrue

\input{../tex-inputs/latex-vorspann}


\section[Paul Goldmann an Arthur Schnitzler, 12. 1. {[}1895{]}]{L02727 Paul Goldmann an Arthur Schnitzler, 12. 1. {[}1895{]}}
\nopagebreak\mylabel{L02727v}
\rehead{ }\normalsize\beginnumbering\briefempfaengerindex{Schnitzler, Arthur@\textsc{Schnitzler, Arthur}!zzzGoldmann, Paul@\emph{von Paul Goldmann}!1895-01-121@{12. 1. {[}1895{]}}|(be}
\toendnotes[C]{\smallbreak\pagebreak[2]}\Standort{DLA, A:Schnitzler, HS.NZ85.1.3165.}
\physDesc{Brief, 1 Blatt, 3 Seiten, 1272 Zeichen
\newline{}Handschrift: schwarze Tinte, deutsche Kurrent
\newline{}Schnitzler: 1) mit Bleistift das Jahr »95« vermerkt  2) mit rotem Buntstift eine Unterstreichung}\toendnotes[C]{\smallbreak}
\pstart
           {\pb}\textcolor{gray}{\textbf{\textbf{Frankfurter Zeitung\orgindex{Frankfurter Zeitung@Frankfurter Zeitung|pw}}}}\pend
           
\pstart
           \textcolor{gray}{\textbf{(\begin{otherlanguage}{french}Gazette de Francfort\end{otherlanguage}\orgindex{Frankfurter Zeitung@Frankfurter Zeitung|pw}). }}\pend
           
\pstart
           \textcolor{gray}{\textbf{\textbf{\begin{otherlanguage}{french}Fondateur M. L.
                              Sonnemann\pwindex{Sonnemann, Leopold 1831-10-29 – 1909-10-30@\textsc{Sonnemann, Leopold} (1831-10-29 – 1909-10-30), \emph{Journalist/Journalistin, Herausgeber/Herausgeberin}|pw}\end{otherlanguage}.}}}\pend
           
\pstart
           \begin{otherlanguage}{french}\textcolor{gray}{\textbf{Journal politique, financier,}}\end{otherlanguage}\hfill \textsc{Paris\oindex{Paris@\textbf{Paris}, \emph{P.PPLC}|pw}}, 12. Januar.\pend
           
\pstart
           \begin{otherlanguage}{french}\textcolor{gray}{\textbf{commercial et littéraire.}}\end{otherlanguage}\pend
           
\pstart
           \begin{otherlanguage}{french}\textcolor{gray}{\textbf{\textbf{Paraissant trois fois par jour.}}}\end{otherlanguage}\pend
           
\pstart
           \begin{otherlanguage}{french}\textcolor{gray}{\textbf{\textbf{Bureau à Paris\oindex{Paris@\textbf{Paris}, \emph{P.PPLC}|pw}:}}}\end{otherlanguage}\pend
           
\pstart
           \begin{otherlanguage}{french}\textcolor{gray}{\textbf{\textbf{24. Rue Feydeau\oindex{rue Feydeau@\textbf{rue Feydeau}, \emph{Straße (K.STR)}|pw}.}}}\end{otherlanguage}\pend
           
\pstart{}Mein lieber Freund,\pend\vspace{0.5em}
\pstart
           \textsc{Lalo\pwindex{Lalo, Pierre 1866-09-06 – 1943-06-09@\textsc{Lalo, Pierre} (1866-09-06 – 1943-06-09), \emph{Kritiker/Kritikerin}|pw}}, vom »\textsc{Journal des Débats\orgindex{Journal des debats@Journal des débats|pw}}«, war geſtern bei mir. »Sterben\pwindex{Sterben. Novelle@\emph{Sterben. Novelle}|pw}« hat ihm ungemein gefallen, \textsc{Richards\pwindex{Beer-Hofmann, Richard 1866-07-11 – 1945-09-26@\textsc{Beer-Hofmann, Richard} (1866-07-11 – 1945-09-26), \emph{Schriftsteller/Schriftstellerin}|pw}}{ }Buch\pwindex{Novellen@\emph{Novellen}|pwv} weniger
               (ſags ihm aber nicht). Er hat \substVorne{}\textsuperscript{e}\substDazwischen{}b\substHinten{}eſtimmt verſprochen, über Euch zu \label{K_L02727-1v}\edtext{ſchreiben}{\lemma{\textnormal{\emph{ſchreiben}}}\Cendnote{\textnormal{Nicht
                  über Richard Beer-Hofmann\pwindex{Beer-Hofmann, Richard 1866-07-11 – 1945-09-26@\textsc{Beer-Hofmann, Richard} (1866-07-11 – 1945-09-26), \emph{Schriftsteller/Schriftstellerin}|pwk}, jedoch über Schnitzler und seine Novelle \emph{Sterben}\pwindex{Sterben. Novelle@\emph{Sterben. Novelle}|pwk} schrieb Pierre
                     Lalo\pwindex{Lalo, Pierre 1866-09-06 – 1943-06-09@\textsc{Lalo, Pierre} (1866-09-06 – 1943-06-09), \emph{Kritiker/Kritikerin}|pwk} am 21. 3. 1895: P. L. [ = Pierre Lalo\pwindex{Lalo, Pierre 1866-09-06 – 1943-06-09@\textsc{Lalo, Pierre} (1866-09-06 – 1943-06-09), \emph{Kritiker/Kritikerin}|pwk}]: \emph{Au jour le jour. M. Arthur
                           Schnitzler}\pwindex{Au jour le jour. M. Arthur Schnitzler@\emph{Au jour le jour. M. Arthur Schnitzler}|pwk}. In: \emph{Journal des débats}\pwindex{Journal des debats. Politiques et litteraires@\emph{Journal des débats. Politiques et littéraires}|pwk}, Jg. 107,
                        21. 3. 1895, S. 1.}}}\label{K_L02727-1}. Ob ers
               halten wird???\pend
           
\pstart
           Bitte, ſchick’ mir \textsc{Torresanis\pwindex{Torresani-Lanzenfeld, Carl von 19.04.1846 – 16.04.1907@\textsc{Torresani-Lanzenfeld, Carl von} (19.04.1846 – 16.04.1907), \emph{Schriftsteller/Schriftstellerin, Offizier/Offizierin}|pw}} Adreſſe.\pend
           
\pstart
           Hat Frl. \textsc{Sandrock\pwindex{Sandrock, Adele 1863-08-19 – 1937-08-30@\textsc{Sandrock, Adele} (1863-08-19 – 1937-08-30), \emph{Schauspieler/Schauspielerin}|pw}} meine Briefe erhalten?\pend
           
\pstart
           Franzoſ\oindex{Frankreich@\textbf{Frankreich}, \emph{A.PCLI}|pwv}en\pwindex{Donnay, Maurice 12.10.1859 – 31.03.1945@\textsc{Donnay, Maurice} (12.10.1859 – 31.03.1945), \emph{Schriftsteller/Schriftstellerin}|pwv}\pwindex{Hervieu, Paul Ernest 2.9.1857 – 25.10.1915@\textsc{Hervieu, Paul Ernest} (2.9.1857 – 25.10.1915), \emph{Schriftsteller/Schriftstellerin}|pwv}\pwindex{Esparbes, Georges d' 1863-03-24 – 1944-06-25@\textsc{Esparbès, Georges d'} (1863-03-24 – 1944-06-25), \emph{Schriftsteller/Schriftstellerin}|pwv}\pwindex{Hermant, Abel 03.02.1862 – 28.09.1950@\textsc{Hermant, Abel} (03.02.1862 – 28.09.1950), \emph{Schriftsteller/Schriftstellerin}|pwv}\pwindex{Lavedan, Henri Leon 09.04.1859 – 4.9.1940@\textsc{Lavedan, Henri Léon} (09.04.1859 – 4.9.1940), \emph{Schriftsteller/Schriftstellerin, Journalist/Journalistin}|pwv}\pwindex{Vanderem, Fernand 1864-06-24 – 1939-03-11@\textsc{Vandérem, Fernand} (1864-06-24 – 1939-03-11), \emph{Schriftsteller/Schriftstellerin}|pwv}\pwindex{Capus, Alfred 25.11.1858 – 01.11.1922@\textsc{Capus, Alfred} (25.11.1858 – 01.11.1922), \emph{Schriftsteller/Schriftstellerin, Journalist/Journalistin}|pwv}\pwindex{Nion, François de 1854-08-13 – 1923-01-11@\textsc{Nion, François de} (1854-08-13 – 1923-01-11), \emph{Schriftsteller/Schriftstellerin}|pwv}\pwindex{Fleurigny, Henry de 1849-03-10 – 1916-10-10@\textsc{Fleurigny, Henry de} (1849-03-10 – 1916-10-10), \emph{Schriftsteller/Schriftstellerin, Militär/Militärin}|pwv}\pwindex{Courteline, Georges 25.06.1858 – 25.06.1929@\textsc{Courteline, Georges} (25.06.1858 – 25.06.1929), \emph{Schriftsteller/Schriftstellerin}|pwv}\pwindex{Ajalbert, Jean 1863-06-10 – 1947-01-14@\textsc{Ajalbert, Jean} (1863-06-10 – 1947-01-14), \emph{Schriftsteller/Schriftstellerin, Kritiker/Kritikerin, Rechtsanwalt/Rechtsanwältin}|pwv}\pwindex{Xanrof, Leon 1867-12-09 – 1953-05-17@\textsc{Xanrof, Léon} (1867-12-09 – 1953-05-17), \emph{Dramatiker/Dramatikerin, Humorist/Humoristin}|pwv}\pwindex{Renard, Jules 22.02.1864 – 22.03.1910@\textsc{Renard, Jules} (22.02.1864 – 22.03.1910), \emph{Schriftsteller/Schriftstellerin}|pwv}\pwindex{Jules-Bois, Henri Antoine 1868-09-29 – 1943-07-02@\textsc{Jules-Bois, Henri Antoine} (1868-09-29 – 1943-07-02), \emph{Schriftsteller/Schriftstellerin}|pwv}\pwindex{Case, Jules 1854-06-24 – 1931-12-15@\textsc{Case, Jules} (1854-06-24 – 1931-12-15), \emph{Schriftsteller/Schriftstellerin, Journalist/Journalistin, Literaturkritiker/Literaturkritikerin}|pwv}\pwindex{Adam, Paul 1862-12-06 – 1920-01-02@\textsc{Adam, Paul} (1862-12-06 – 1920-01-02), \emph{Schriftsteller/Schriftstellerin, Kunstkritiker/Kunstkritikerin}|pwv}, die kleine
               Geſchichten ſchreiben, ſind: \textsc{Maurice Donnay\pwindex{Donnay, Maurice 12.10.1859 – 31.03.1945@\textsc{Donnay, Maurice} (12.10.1859 – 31.03.1945), \emph{Schriftsteller/Schriftstellerin}|pw}}, \textsc{Paul Hervieu\pwindex{Hervieu, Paul Ernest 2.9.1857 – 25.10.1915@\textsc{Hervieu, Paul Ernest} (2.9.1857 – 25.10.1915), \emph{Schriftsteller/Schriftstellerin}|pw}}, {\pb}\textsc{Georges d'Esparbès\pwindex{Esparbes, Georges d' 1863-03-24 – 1944-06-25@\textsc{Esparbès, Georges d'} (1863-03-24 – 1944-06-25), \emph{Schriftsteller/Schriftstellerin}|pw}}, \textsc{Abel Hermant\pwindex{Hermant, Abel 03.02.1862 – 28.09.1950@\textsc{Hermant, Abel} (03.02.1862 – 28.09.1950), \emph{Schriftsteller/Schriftstellerin}|pw}}, \textsc{\strikeout{Hen\pwindex{Lavedan, Henri Leon 09.04.1859 – 4.9.1940@\textsc{Lavedan, Henri Léon} (09.04.1859 – 4.9.1940), \emph{Schriftsteller/Schriftstellerin, Journalist/Journalistin}|pwv}}}{ }\textsc{Henri \strikeout{La}
                     Lavedan\pwindex{Lavedan, Henri Leon 09.04.1859 – 4.9.1940@\textsc{Lavedan, Henri Léon} (09.04.1859 – 4.9.1940), \emph{Schriftsteller/Schriftstellerin, Journalist/Journalistin}|pw}}, \textsc{Ferdinand Vanderem\pwindex{Vanderem, Fernand 1864-06-24 – 1939-03-11@\textsc{Vandérem, Fernand} (1864-06-24 – 1939-03-11), \emph{Schriftsteller/Schriftstellerin}|pw}}, \textsc{Alfred Capus\pwindex{Capus, Alfred 25.11.1858 – 01.11.1922@\textsc{Capus, Alfred} (25.11.1858 – 01.11.1922), \emph{Schriftsteller/Schriftstellerin, Journalist/Journalistin}|pw}}, \textsc{François de Nion\pwindex{Nion, François de 1854-08-13 – 1923-01-11@\textsc{Nion, François de} (1854-08-13 – 1923-01-11), \emph{Schriftsteller/Schriftstellerin}|pw}, Henry de Fleurigny\pwindex{Fleurigny, Henry de 1849-03-10 – 1916-10-10@\textsc{Fleurigny, Henry de} (1849-03-10 – 1916-10-10), \emph{Schriftsteller/Schriftstellerin, Militär/Militärin}|pw}}, \textsc{Georges Courteline\pwindex{Courteline, Georges 25.06.1858 – 25.06.1929@\textsc{Courteline, Georges} (25.06.1858 – 25.06.1929), \emph{Schriftsteller/Schriftstellerin}|pw}}, \textsc{Jean Ajalbert\pwindex{Ajalbert, Jean 1863-06-10 – 1947-01-14@\textsc{Ajalbert, Jean} (1863-06-10 – 1947-01-14), \emph{Schriftsteller/Schriftstellerin, Kritiker/Kritikerin, Rechtsanwalt/Rechtsanwältin}|pw}, L. Xanrof\pwindex{Xanrof, Leon 1867-12-09 – 1953-05-17@\textsc{Xanrof, Léon} (1867-12-09 – 1953-05-17), \emph{Dramatiker/Dramatikerin, Humorist/Humoristin}|pw}}, \textsc{Jules Renard\pwindex{Renard, Jules 22.02.1864 – 22.03.1910@\textsc{Renard, Jules} (22.02.1864 – 22.03.1910), \emph{Schriftsteller/Schriftstellerin}|pw}}, \textsc{Jules Bois\pwindex{Jules-Bois, Henri Antoine 1868-09-29 – 1943-07-02@\textsc{Jules-Bois, Henri Antoine} (1868-09-29 – 1943-07-02), \emph{Schriftsteller/Schriftstellerin}|pw}}, \textsc{Jules Case\pwindex{Case, Jules 1854-06-24 – 1931-12-15@\textsc{Case, Jules} (1854-06-24 – 1931-12-15), \emph{Schriftsteller/Schriftstellerin, Journalist/Journalistin, Literaturkritiker/Literaturkritikerin}|pw}}, \textsc{Paul Adam\pwindex{Adam, Paul 1862-12-06 – 1920-01-02@\textsc{Adam, Paul} (1862-12-06 – 1920-01-02), \emph{Schriftsteller/Schriftstellerin, Kunstkritiker/Kunstkritikerin}|pw}}{ }\textsc{etc}.\pend
           
\pstart
           Wenn Du damit nicht genug haſt, kannſt Du mehr bekommen. Meiſtens ſind ſie recht
               mäßig. Die gegenwärtig aufgehende Saat iſt nicht gut gerathen. Außer den verwöhnten
                  Mode-Pinſeln\pwindex{Prevost, Marcel 01.05.1862 – 08.04.1941@\textsc{Prévost, Marcel} (01.05.1862 – 08.04.1941), \emph{Schriftsteller/Schriftstellerin}|pwv}\pwindex{Hermant, Abel 03.02.1862 – 28.09.1950@\textsc{Hermant, Abel} (03.02.1862 – 28.09.1950), \emph{Schriftsteller/Schriftstellerin}|pwv}\pwindex{Vanderem, Fernand 1864-06-24 – 1939-03-11@\textsc{Vandérem, Fernand} (1864-06-24 – 1939-03-11), \emph{Schriftsteller/Schriftstellerin}|pwv} (\textsc{Prevost\pwindex{Prevost, Marcel 01.05.1862 – 08.04.1941@\textsc{Prévost, Marcel} (01.05.1862 – 08.04.1941), \emph{Schriftsteller/Schriftstellerin}|pw}}, \textsc{Hermant\pwindex{Hermant, Abel 03.02.1862 – 28.09.1950@\textsc{Hermant, Abel} (03.02.1862 – 28.09.1950), \emph{Schriftsteller/Schriftstellerin}|pw}, Vanderem\pwindex{Vanderem, Fernand 1864-06-24 – 1939-03-11@\textsc{Vandérem, Fernand} (1864-06-24 – 1939-03-11), \emph{Schriftsteller/Schriftstellerin}|pw}}) kann man ſie zum Überſetzen zweifellos {\pb}billig, meiſt umſonſt bekommen. Man ſchreibt ihnen: \label{K_L02727-2v}\edtext{\textsc{\begin{otherlanguage}{french}Nous serions très-heureux d’obtenir l’autorisation de
                     traduire {\dotssix} Cela servirait comme échantillon de vos
                     œuvres pour vous introduire auprès du public autrich\oindex{Oesterreich@\textbf{Österreich}, \emph{A.PCLI}|pw}ièn.\end{otherlanguage}}}{\lemma{\textnormal{\emph{Nous … autrichièn.}}}\Cendnote{\textnormal{französisch: Wir würden uns sehr freuen,
                  wenn wir die Erlaubnis bekämen, {\dotssix} zu übersetzen. Dies
                  würde als Kostprobe Ihrer Werke dienen, um Sie dem österreich\oindex{Oesterreich@\textbf{Österreich}, \emph{A.PCLI}|pwk}ischen Publikum bekannt zu machen.}}}\label{K_L02727-2} So
               natürlich nur den Unbekannten. Die Bekannten ſetzen voraus, daß man in Wien\oindex{Wien@\textbf{Wien}, \emph{A.ADM2}|pw} nichts mehr lieſt, als ſie. Oder aber man
               ſchreibt gar nicht. Wer kümmert ſich in \textsc{Paris\oindex{Paris@\textbf{Paris}, \emph{P.PPLC}|pw}} um die \label{K_L02727-3v}\edtext{Allgemeine Zeitung\orgindex{Wiener Allgemeine Zeitung@Wiener Allgemeine Zeitung|pw}}{\lemma{\textnormal{\emph{Allgemeine Zeitung}}}\Cendnote{\textnormal{Seit Oktober 1894 war Felix Salten\pwindex{Salten, Felix 06.09.1869 – 08.10.1945@\textsc{Salten, Felix} (06.09.1869 – 08.10.1945), \emph{Schriftsteller/Schriftstellerin, Journalist/Journalistin, Chefredakteur/Chefredakteurin}|pwk} bei der \emph{Wiener Allgemeinen Zeitung}\orgindex{Wiener Allgemeine Zeitung@Wiener Allgemeine Zeitung|pwk} engagiert, was einen möglichen
                  Hintergrund für die Anfrage darstellt. Ob Schnitzler überlegte, sich selbst durch Übersetzungen einen Verdienst zu
                  verschaffen, ist ungewiss.}}}\label{K_L02727-3}?\pend
           
\pstart
           Herzlichſt {\\[\baselineskip]}Dein {\\[\baselineskip]}\spacefill\mbox{Paul Goldmnn}\pend
           \leftskip=0em{}\selectlanguage{ngerman}\endnumbering\briefempfaengerindex{Schnitzler, Arthur@\textsc{Schnitzler, Arthur}!zzzGoldmann, Paul@\emph{von Paul Goldmann}!1895-01-121@{12. 1. {[}1895{]}}|)be}\mylabel{L02727h}  \normalsize

\doendnotes{C}
\bigskip
\vfill

\clearpage

\footnotesize

\lohead{\textsc{register}}

% Definiere theindex-Environment komplett neu ohne reledmac
\makeatletter
\renewenvironment{theindex}{%
  \section*{\indexname}%
  \setlength{\parindent}{0pt}%
  \setlength{\parskip}{0pt plus 0.3pt}%
  \let\item\@idxitem
}{%
  \clearpage
}
\makeatother

\IfFileExists{\jobname-pw.ind}{\input{\jobname-pw.ind}}{}

\end{document}

      