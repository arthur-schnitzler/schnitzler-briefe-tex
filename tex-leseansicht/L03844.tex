%% latex-leseansicht-vorspann.tex
%% Vorspann für die Leseansicht.
%% Lädt die gemeinsame Datei latex-vorspann.tex mit nicht gesetztem Schalter.

\newif\ifkorrekturansicht
\korrekturansichtfalse

\input{../tex-inputs/latex-vorspann}


\section[Theodor Herzl an Arthur Schnitzler, 4. 1. 1895]{L03844 Theodor Herzl an Arthur Schnitzler, 4. 1. 1895}
\nopagebreak\mylabel{L03844v}
\rehead{ }\normalsize\beginnumbering\briefempfaengerindex{Schnitzler, Arthur@\textsc{Schnitzler, Arthur}!zzzHerzl, Theodor@\emph{von Theodor Herzl}!1895-01-041@{4. 1. 1895}|(be}
\toendnotes[C]{\smallbreak\pagebreak[2]}
\correspDesc{Versand  durch Theodor Herzl am 4. 1. 1895 in Paris
\newline{}Erhalt  durch Arthur Schnitzler im Zeitraum [5. 1. 1895
                  – 9. 1. 1895?] in Wien}\toendnotes[C]{\smallbreak}
\Standort{CUL, Schnitzler, B 39.}
\physDesc{Brief, 1 Blatt, 1 Seite, 800 Zeichen
\newline{}Handschrift: schwarze Tinte, lateinische Kurrent
\newline{}Ordnung: mit Bleistift von unbekannter Hand nummeriert: »23« }
\buchAbdrucke{\weitereDrucke{Theodor Herzl: \emph{Briefe und
                        autobiographische Notizen 1866–1895}. Bearbeitet von Johannes Wachten in Zusammenarbeit mit Chaya Harel, Daisy Tycho und Manfred Winkler. Berlin, Frankfurt am Main, Wien: \emph{Propyläen} 1983, S. 567–568 (Briefe und Tagebücher. Herausgegeben von Alex Bein, Hermann Greive, Moshe Schaerf, Julius H. Schoeps und Johannes Wachten, 1).} }\toendnotes[C]{\smallbreak}
\pstart
           \raggedleft{}{\pb}Cabinet de lecture\oindex{Cabinet de lecture@\textbf{Cabinet de lecture}, \emph{Bibliothek}|pw}\pend
           
\pstart
           \raggedleft{}Passage de l’opéra\oindex{Passage de l'Opéra@\textbf{Passage de l'Opéra}, \emph{Einkaufspassage}|pw}\pend
           
\pstart
           \raggedleft{}4. I. 95\pend
           
\pstart{}Lieber Freund,\pend\vspace{0.5em}
\pstart
           so weit wären wir also. Ich bitte Sie nun, mir nach Absendung des Mscpts\pwindex{Herzl, Theodor 2.\,5.\,1860 Budapest – 3.\,7.\,1904 Edlach@\textsc{Herzl, Theodor} (2.\,5.\,1860 Budapest – 3.\,7.\,1904 Edlach), \emph{Schriftsteller, Journalist}!neue Ghetto. Schauspiel in vier Acten@\strich\emph{Das neue Ghetto. Schauspiel in vier Acten}|pwv} u. \label{K_L03844-1v}\edtext{Brief}{\lemma{\textnormal{\emph{Brief}}}\Cendnote{\textnormal{Beilage zu
                     XXXX Auszeichnungsfehler: Dokument L03843 nicht gefunden}}}\label{K_L03844-1} mitzutheilen was Sie für mich bisher ausgelegt haben.\pend
           
\pstart
           Für die vorläufig noch entfernte Eventualität der Annahme müssen wir einen
               Telegrammschlüssel verabreden. Denn meine Ungeduld braucht den Draht.\pend
           
\pstart
           Annahme des Stückes\pwindex{Herzl, Theodor 2.\,5.\,1860 Budapest – 3.\,7.\,1904 Edlach@\textsc{Herzl, Theodor} (2.\,5.\,1860 Budapest – 3.\,7.\,1904 Edlach), \emph{Schriftsteller, Journalist}!neue Ghetto. Schauspiel in vier Acten@\strich\emph{Das neue Ghetto. Schauspiel in vier Acten}|pwv} im Lessingtheater\orgindex{Lessing-Theater@Lessing-Theater|pw} bitte ich mir in dieser Form zu
               telegraphiren:\pend
           
\pstart
           »Berichtet ausführlich über Progressivsteuer{\\} Moriz«\pend
           
\pstart
           \centering{} – \pend
           
\pstart
           Ablehnung des Lessingtheaters\orgindex{Lessing-Theater@Lessing-Theater|pw}{\\} Gebet Subscriptionsresultat. Moriz«\pend
           
\pstart
           \centering{} – \pend
           
\pstart
           \strikeout{Deutsches}\pend
           
\pstart
           Haben Sie’s zuerst dem Deutschen Theater\orgindex{Deutsches Theater Berlin@Deutsches Theater Berlin|pw}
               geschickt so gilt dasselbe fürs Deutsche Theater\orgindex{Deutsches Theater Berlin@Deutsches Theater Berlin|pw}\pend
           
\pstart
           Ueber spätere Schlüssel werden wir uns verständigen. Verlangt man \uline{nicht} einschneidende Aenderungen so benützen Sie die
               Annahmeformel.\pend
           
\pstart
           Mit herzlichen Grüssen{\\[\baselineskip]}Ihr ergebener{\\[\baselineskip]}\spacefill\mbox{Th H.}\pend
           \leftskip=0em{}\selectlanguage{ngerman}\endnumbering\briefempfaengerindex{Schnitzler, Arthur@\textsc{Schnitzler, Arthur}!zzzHerzl, Theodor@\emph{von Theodor Herzl}!1895-01-041@{4. 1. 1895}|)be}\mylabel{L03844h}
\begin{anhang}
\end{anhang}\newcommand{\dateiname}{L03844}\newcommand{\titel}{Theodor Herzl an Arthur Schnitzler, 4. 1. 1895}\newcommand{\editorInnen}{Selma Jahnke und Martin Anton Müller}%% latex-leseansicht-abspann.tex
%% Abspann für die Leseansicht.
%% Der Schalter \ifkorrekturansicht ist bereits durch den Vorspann gesetzt.

%% latex-abspann.tex
%% Gemeinsamer Abspann für Korrekturansicht und Leseansicht.
%% Setzt den Schalter \ifkorrekturansicht voraus (gesetzt in den
%% einbindenden Dateien latex-korrekturansicht-abspann.tex bzw.
%% latex-leseansicht-abspann.tex).
%% ---------------------------------------------------------------

\normalsize

% Das esempio-Environment wird nur in der Leseansicht benötigt
\ifkorrekturansicht\else
\newenvironment{esempio}[3]%
{
    \vspace{1.5ex}
    \rlap{\underline{#1}}
    \par
    \setlength{\parindent}{0cm}
    \nopagebreak
    \leftskip=#2cm
    \rightskip=#3cm
}
{
    \par
}
\fi

\doendnotes{C}
\bigskip
\vfill

\clearpage

\footnotesize

\ifkorrekturansicht
  \lohead{\textsc{register}}
\fi

% theindex-Environment neu definieren ohne reledmac
\makeatletter
\renewenvironment{theindex}{%
  \ifkorrekturansicht
    \section*{\indexname}%
  \else
    \subsubsection*{Index der erwähnten Entitäten}%
  \fi
  \setlength{\parindent}{0pt}%
  \setlength{\parskip}{0pt plus 0.3pt}%
  \let\item\@idxitem
}{%
  \ifkorrekturansicht\clearpage\fi
}
\makeatother

\IfFileExists{\jobname-pw.ind}{\input{\jobname-pw.ind}}{}

% Quellenangabe nur in der Leseansicht
\ifkorrekturansicht\else
% Fallback-Definitionen, falls die .tex-Datei \titel etc. nicht gesetzt hat
\providecommand{\titel}{}
\providecommand{\editorInnen}{}
\providecommand{\dateiname}{\jobname}

\vspace{3cm}

\vfill

\footnotesize
\textsc{Quelle}: \titel. Herausgegeben von {\editorInnen}. In: \emph{Arthur Schnitzler: Briefwechsel mit Autorinnen und Autoren}.
 Digitale Edition, https://schnitzler-briefe.acdh.oeaw.ac.at/{\dateiname}.html (Stand \today)
\fi

\end{document}


