%% latex-korrekturansicht-vorspann.tex
%% Vorspann für die Korrekturansicht.
%% Lädt die gemeinsame Datei latex-vorspann.tex mit gesetztem Schalter.

\newif\ifkorrekturansicht
\korrekturansichttrue

\input{../tex-inputs/latex-vorspann}


\section[Arthur Schnitzler an Richard Beer-Hofmann, 1. 4. 1902]{L01215 Arthur Schnitzler an Richard Beer-Hofmann, 1. 4. 1902}
\nopagebreak\mylabel{L01215v}
\rehead{ }\normalsize\beginnumbering\briefempfaengerindex{Beer-Hofmann, Richard@\textsc{Beer-Hofmann, Richard}!zzzSchnitzler, Arthur@\emph{von Arthur Schnitzler}!1902-04-011@{1. 4. 1902}|(be}
\toendnotes[C]{\smallbreak\pagebreak[2]}\Standort{YCGL, MSS 31.}
\physDesc{Briefkarte, , Umschlag, 329 Zeichen
\newline{}Handschrift: Bleistift, deutsche Kurrent
\newline{}Versand: 1) Stempel: »\nobreak{}\oindex{I., Innere Stadt@\textbf{I., Innere Stadt}, \emph{A.ADM3}|pwk}Wien 1/1 1, 1. 4. 0\textcolor{gray}{2}, 11–12N\nobreak{}«.   2) Stempel: »\nobreak{}\oindex{Rodaun@\textbf{Rodaun}, \emph{A.ADM4}|pwk}{\pb}Rodaun, 1. 4. {[}02{]}, 7–9V\nobreak{}«. 
\newline{}Ordnung: mit Bleistift von unbekannter Hand datiert: »1. 4.« }\toendnotes[C]{\smallbreak}\pstart{}{\pb}Herrn \textsc{Dr. Rich.
                     Beer-Hofmann}\pend{}\pstart{}\textsc{Rodaun} bei Lieſing\oindex{Rodaun@\textbf{Rodaun}, \emph{A.ADM4}|pw}\pend{}\pstart{}Lieſinger Straße 2\oindex{Liesingerstrasse@\textbf{Liesingerstraße}, \emph{Straße (K.STR)}|pw}.\pend{}{\bigskip}\vspace{1em}
\pstart
           \noindent{}{\pb}lieber Richard, ich habe mich bei Schlichter\pwindex{Schlichter, Felix 11.04.1865 – 03.11.1924@\textsc{Schlichter, Felix} (11.04.1865 – 03.11.1924), \emph{Pädiater/Pädiaterin}|pw} für \uline{\label{K_L01215-1v}\edtext{Samſtag{ }4}{\lemma{\textnormal{\emph{Samſtag 4}}}\Cendnote{\textnormal{Vgl. A. S.: \emph{Tagebuch}, 12. 4. 1902.
                  }}}\label{K_L01215-1}} angeſagt u Ihr wahrſcheinl. Kommen in Ausſicht geſtellt, neuen Impfstoff
               beſtellt.\pend
           
\pstart
           Leider ko{\geminationn}t ich heut nicht zu Ihnen, wir müſſen doch
               endlich wieder {\pb}einige ungehetzte Stunden miteinander
               verbringen find ich.\pend
           
\pstart
           Ihr{\\[\baselineskip]}\spacefill\mbox{A.}\pend
           \leftskip=0em{}\selectlanguage{ngerman}\endnumbering\briefempfaengerindex{Beer-Hofmann, Richard@\textsc{Beer-Hofmann, Richard}!zzzSchnitzler, Arthur@\emph{von Arthur Schnitzler}!1902-04-011@{1. 4. 1902}|)be}\mylabel{L01215h}  \normalsize

\doendnotes{C}
\bigskip
\vfill

\clearpage

\footnotesize

\lohead{\textsc{register}}

% Definiere theindex-Environment komplett neu ohne reledmac
\makeatletter
\renewenvironment{theindex}{%
  \section*{\indexname}%
  \setlength{\parindent}{0pt}%
  \setlength{\parskip}{0pt plus 0.3pt}%
  \let\item\@idxitem
}{%
  \clearpage
}
\makeatother

\IfFileExists{\jobname-pw.ind}{\input{\jobname-pw.ind}}{}

\end{document}

      