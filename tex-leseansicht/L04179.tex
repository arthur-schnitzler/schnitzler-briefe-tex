%% latex-leseansicht-vorspann.tex
%% Vorspann für die Leseansicht.
%% Lädt die gemeinsame Datei latex-vorspann.tex mit nicht gesetztem Schalter.

\newif\ifkorrekturansicht
\korrekturansichtfalse

\input{../tex-inputs/latex-vorspann}


\section[Arthur Schnitzler an Gustav Schwarzkopf, {[}16. 1. 1902?{]}]{L04179 Arthur Schnitzler an Gustav Schwarzkopf, {[}16. 1. 1902?{]}}
\nopagebreak\mylabel{L04179v}
\rehead{ }\normalsize\beginnumbering\briefempfaengerindex{Schwarzkopf, Gustav@\textsc{Schwarzkopf, Gustav}!zzzSchnitzler, Arthur@\emph{von Arthur Schnitzler}!1902-01-161@{{[}16. 1. 1902?{]}}|(be}
\toendnotes[C]{\smallbreak\pagebreak[2]}
\correspDesc{Versand  durch Arthur Schnitzler am [16. 1. 1902?] in Wien
\newline{}Erhalt  durch Gustav Schwarzkopf im Zeitraum [16. 1. 1902
                  – 19. 1. 1902?] in Wien}\toendnotes[C]{\smallbreak}
\Standort{CUL, Schnitzler, B 96.}
\physDesc{Brief, 1 Blatt, 3 Seiten, 388 Zeichen
\newline{}Handschrift: Bleistift, deutsche Kurrent}\toendnotes[C]{\smallbreak}
\pstart{}{\pb}lieber Guſtav,\pend\vspace{0.5em}
\pstart
           bitte nachtmahlen Sie morgen \label{K_L04179-11v}\edtext{Freitag bei uns\pwindex{\textcolor{red}{\textsuperscript{XXXX indx1}}|pwv}\oindex{XXXX Ortsangabe fehlt|pwv}}{\lemma{\textnormal{\emph{Freitag bei uns}}}\Cendnote{\textnormal{Nicht im \emph{Tagebuch}\textcolor{red}{\textsuperscript{XXXX indx2}}-Eintrag zum
                     17. 1. 1902.}}}\label{K_L04179-11}. (8 Uhr.) Vielleicht haben Sie früher Zeit und kommen zwiſchen
                  6 und 7 in die \label{K_L04179-1v}\edtext{Garniſons{\pb}gaſſe\oindex{Wien@\textbf{Wien}!IX., Alsergrund@\textbf{IX., Alsergrund}!?? [Unterkunft von Olga Schnitzler, Dezember 1901 – März 1920, Garnisonsgasse?]@\textbf{?? [Unterkunft von Olga Schnitzler, Dezember 1901 – März 1920, Garnisonsgasse?]}, \emph{Wohngebäude}|pwv}}{\lemma{\textnormal{\emph{Garnisonsgasse}}}\Cendnote{\textnormal{Für den Zeitraum Ende November 1901 bis 3. 2. 1902 ist derzeit nicht ermittelt, wo Olga Gussmann\pwindex{Schnitzler, Olga 17.\,1.\,1882 Wien – 13.\,1.\,1970 Lugano@\textsc{Schnitzler, Olga} (17.\,1.\,1882 Wien – 13.\,1.\,1970 Lugano), \emph{Schauspielerin, Sängerin}|pwk} wohnte. Sie wusste zu diesem
                  Zeitpunkt bereits von ihrer Schwangerschaft mit Heinrich Schnitzler\pwindex{Schnitzler, Heinrich 9.\,8.\,1902 Hinterbrühl – 12.\,7.\,1982 Wien@\textsc{Schnitzler, Heinrich} (9.\,8.\,1902 Hinterbrühl – 12.\,7.\,1982 Wien), \emph{Regisseur, Schauspieler}|pwk}, vgl. A. S.: \emph{Tagebuch}, 10. 11. 1901. Als Vertrauter war Schwarzkopf\pwindex{Schwarzkopf, Gustav 7.\,11.\,1853 Wien – 13.\,11.\,1939 ebd.@\textsc{Schwarzkopf, Gustav} (7.\,11.\,1853 Wien – 13.\,11.\,1939 ebd.), \emph{Schriftsteller}|pwk} mehrfach in die gemeinsamen Wohnungen von Schnitzler und Olga
                     Gussmann\pwindex{Schnitzler, Olga 17.\,1.\,1882 Wien – 13.\,1.\,1970 Lugano@\textsc{Schnitzler, Olga} (17.\,1.\,1882 Wien – 13.\,1.\,1970 Lugano), \emph{Schauspielerin, Sängerin}|pwk} geladen, so dass diese Erwähnung hier zumindest die ungefähre
                  Lage der Wohnung zu bestimmen scheint. Vgl. XXXX Auszeichnungsfehler: Dokument L03536 nicht gefunden. }}}\label{K_L04179-1}? Ich war heute \substVorne{}\textsuperscript{be}\substDazwischen{}tro\substHinten{}tz des \label{K_L04179-2v}\edtext{Orkans in Mödling\oindex{Mödling@\textbf{Mödling}, \emph{Hauptstadt}|pw}}{\lemma{\textnormal{\emph{Orkans in Mödling}}}\Cendnote{\textnormal{Das erlaubt die Datierung, vgl. A. S.: \emph{Tagebuch}, 16. 1. 1902. Er suchte eine
                  Unterkunft für die Zeit der Schwangerschaft und die Geburt von Heinrich\pwindex{Schnitzler, Heinrich 9.\,8.\,1902 Hinterbrühl – 12.\,7.\,1982 Wien@\textsc{Schnitzler, Heinrich} (9.\,8.\,1902 Hinterbrühl – 12.\,7.\,1982 Wien), \emph{Regisseur, Schauspieler}|pwk}, da die Eltern nicht verheiratet waren.}}}\label{K_L04179-2} –
               und hatte ſchon einige Angſt, daſs Sie – aus Höflichkeit – zur Bahn\oindex{Wien@\textbf{Wien}!X., Favoriten@\textbf{X., Favoriten}!Südbahnhof@\textbf{Südbahnhof}, \emph{Bahnhofsgebäude}|pwv} kommen würden. Es war geradezu
                  lebensge{\pb}fährlich. \label{K_L04179-3v}\edtext{Aber was wäre man, we{\geminationn} man den Muth nicht hätte\pwindex{Sudermann, Hermann 30.\,9.\,1857 Macikai – 21.\,11.\,1928 Berlin@\textsc{Sudermann, Hermann} (30.\,9.\,1857 Macikai – 21.\,11.\,1928 Berlin), \emph{Schriftsteller}!Ehre. Schauspiel in 4 Akten@\strich\emph{Die Ehre. Schauspiel in 4 Akten}|pwv}, ſagt Lothar Brandt\pwindex{Sudermann, Hermann 30.\,9.\,1857 Macikai – 21.\,11.\,1928 Berlin@\textsc{Sudermann, Hermann} (30.\,9.\,1857 Macikai – 21.\,11.\,1928 Berlin), \emph{Schriftsteller}!Ehre. Schauspiel in 4 Akten@\strich\emph{Die Ehre. Schauspiel in 4 Akten}|pwv}}{\lemma{\textnormal{\emph{Aber … Brandt}}}\Cendnote{\textnormal{»Ich muß bekennen, mein gnädiges
                     Fräulein, Sie haben mich eingeschüchtert. Und das will etwas sagen! Denn was
                     wäre man, wenn man nicht den Mut besäße?«, Hermann Sudermann\pwindex{Sudermann, Hermann 30.\,9.\,1857 Macikai – 21.\,11.\,1928 Berlin@\textsc{Sudermann, Hermann} (30.\,9.\,1857 Macikai – 21.\,11.\,1928 Berlin), \emph{Schriftsteller}|pwk}: \emph{Die
                     Ehre}\pwindex{Sudermann, Hermann 30.\,9.\,1857 Macikai – 21.\,11.\,1928 Berlin@\textsc{Sudermann, Hermann} (30.\,9.\,1857 Macikai – 21.\,11.\,1928 Berlin), \emph{Schriftsteller}!Ehre. Schauspiel in 4 Akten@\strich\emph{Die Ehre. Schauspiel in 4 Akten}|pwk}, 4. Akt, 5. Szene.}}}\label{K_L04179-3}\pend
           
\pstart
           Herzlich grüßend{\\[\baselineskip]} Ihr{\\[\baselineskip]}\spacefill\mbox{A.}\pend
           \leftskip=0em{}\selectlanguage{ngerman}\endnumbering\briefempfaengerindex{Schwarzkopf, Gustav@\textsc{Schwarzkopf, Gustav}!zzzSchnitzler, Arthur@\emph{von Arthur Schnitzler}!1902-01-161@{{[}16. 1. 1902?{]}}|)be}\mylabel{L04179h}
\begin{anhang}
\end{anhang}\newcommand{\dateiname}{L04179}\newcommand{\titel}{Arthur Schnitzler an Gustav Schwarzkopf, [16. 1. 1902?]}\newcommand{\editorInnen}{Herausgegeben von Jahnke, SelmaMüller, Martin Anton}%% latex-leseansicht-abspann.tex
%% Abspann für die Leseansicht.
%% Der Schalter \ifkorrekturansicht ist bereits durch den Vorspann gesetzt.

%% latex-abspann.tex
%% Gemeinsamer Abspann für Korrekturansicht und Leseansicht.
%% Setzt den Schalter \ifkorrekturansicht voraus (gesetzt in den
%% einbindenden Dateien latex-korrekturansicht-abspann.tex bzw.
%% latex-leseansicht-abspann.tex).
%% ---------------------------------------------------------------

\normalsize

% Das esempio-Environment wird nur in der Leseansicht benötigt
\ifkorrekturansicht\else
\newenvironment{esempio}[3]%
{
    \vspace{1.5ex}
    \rlap{\underline{#1}}
    \par
    \setlength{\parindent}{0cm}
    \nopagebreak
    \leftskip=#2cm
    \rightskip=#3cm
}
{
    \par
}
\fi

\doendnotes{C}
\bigskip
\vfill

\clearpage

\footnotesize

\ifkorrekturansicht
  \lohead{\textsc{register}}
\fi

% theindex-Environment neu definieren ohne reledmac
\makeatletter
\renewenvironment{theindex}{%
  \ifkorrekturansicht
    \section*{\indexname}%
  \else
    \subsubsection*{Index der erwähnten Entitäten}%
  \fi
  \setlength{\parindent}{0pt}%
  \setlength{\parskip}{0pt plus 0.3pt}%
  \let\item\@idxitem
}{%
  \ifkorrekturansicht\clearpage\fi
}
\makeatother

\IfFileExists{\jobname-pw.ind}{\input{\jobname-pw.ind}}{}

% Quellenangabe nur in der Leseansicht
\ifkorrekturansicht\else
% Fallback-Definitionen, falls die .tex-Datei \titel etc. nicht gesetzt hat
\providecommand{\titel}{}
\providecommand{\editorInnen}{}
\providecommand{\dateiname}{\jobname}

\vspace{3cm}

\vfill

\footnotesize
\textsc{Quelle}: \titel. Herausgegeben von {\editorInnen}. In: \emph{Arthur Schnitzler: Briefwechsel mit Autorinnen und Autoren}.
 Digitale Edition, https://schnitzler-briefe.acdh.oeaw.ac.at/{\dateiname}.html (Stand \today)
\fi

\end{document}


