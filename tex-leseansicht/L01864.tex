%% latex-leseansicht-vorspann.tex
%% Vorspann für die Leseansicht.
%% Lädt die gemeinsame Datei latex-vorspann.tex mit nicht gesetztem Schalter.

\newif\ifkorrekturansicht
\korrekturansichtfalse

\input{../tex-inputs/latex-vorspann}


\section[Arthur Schnitzler an Richard Beer-Hofmann, 12. 8. 1909]{L01864 Arthur Schnitzler an Richard Beer-Hofmann, 12. 8. 1909}
\nopagebreak\mylabel{L01864v}
\rehead{ }\normalsize\beginnumbering\briefempfaengerindex{Beer-Hofmann, Richard@\textsc{Beer-Hofmann, Richard}!zzzSchnitzler, Arthur@\emph{von Arthur Schnitzler}!1909-08-121@{12. 8. 1909}|(be}
\toendnotes[C]{\smallbreak\pagebreak[2]}
\correspDesc{Versand  durch Arthur Schnitzler am 12. 8. 1909 in Hochschneeberg
\newline{}Erhalt  durch Richard Beer-Hofmann am [13.?] 8. 1909 in Lido}\toendnotes[C]{\smallbreak}
\Standort{YCGL, MSS 31.}
\physDesc{Bildpostkarte, 102 Zeichen
\newline{}Handschrift: Bleistift, lateinische Kurrent
\newline{}Versand: 1) Stempel: »\nobreak{}\oindex{Hochschneeberg@\textbf{Hochschneeberg}, \emph{Gebirge}|pwk}Hoch{[}schneeberg{]}\nobreak{}«.   2) Stempel: »\nobreak{}\oindex{Santa Maria Elisabetta@\textbf{Santa Maria Elisabetta}, \emph{Bezirk}|pwk}S. Elis{[}abetta{]}{ }\textcolor{gray}{di} Lido – (Venezia), \textcolor{gray}{13} 8 0\textcolor{gray}{9}\nobreak{}«. 
\newline{}Zusatz: die Karte erschien: »Im Selbstverlage des \textbf{St.
                                          Elisabeth-Kirchlein-Baucomité\orgindex{St. Elisabeth-Kirchlein-Baucomité@St. Elisabeth-Kirchlein-Baucomité|pw}}, Wien III.
                                       Gemeindehaus\oindex{Wien@\textbf{Wien}!III., Landstraße@\textbf{III., Landstraße}!Magistratisches Bezirksamt für den 3. Bezirk@\textbf{Magistratisches Bezirksamt für den 3. Bezirk}, \emph{Verwaltungsgebäude}|pw}. Der Erlös dieser Karte fliesst dem
                                    Baufonds zu.« }\pstart{}{\pb}Dr. Richard Beer Hofmann\pend{}\pstart{}Venedig – Lido\oindex{Lido@\textbf{Lido}|pw}\pend{}\pstart{}Grand Hotel des bains\oindex{Grand Hotel des Bains@\textbf{Grand Hotel des Bains}, \emph{Hotel}|pw}\pend{}{\bigskip}
\pstart
           \noindent{}\centering{}{\pb}\textcolor{gray}{\textbf{Hotel am Hochschneeberg\oindex{Berghaus Hochschneeberg@\textbf{Berghaus Hochschneeberg}, \emph{Hotel}|pw}}}\pend
           \vspace{1em}
\pstart
           {\pb}12. 8. 09\pend
           \vspace{0.5em}
\pstart
           Herzliche Grüße Ihnen Allen!\pend
           \pstart \spacefill\mbox{Arthur}\pend{}\selectlanguage{ngerman}\endnumbering\briefempfaengerindex{Beer-Hofmann, Richard@\textsc{Beer-Hofmann, Richard}!zzzSchnitzler, Arthur@\emph{von Arthur Schnitzler}!1909-08-121@{12. 8. 1909}|)be}\mylabel{L01864h}  \newcommand{\dateiname}{L01864}\newcommand{\titel}{Arthur Schnitzler an Richard Beer-Hofmann, 12. 8. 1909}\newcommand{\editorInnen}{Martin Anton Müller und Gerd-Hermann Susen}%% latex-leseansicht-abspann.tex
%% Abspann für die Leseansicht.
%% Der Schalter \ifkorrekturansicht ist bereits durch den Vorspann gesetzt.

%% latex-abspann.tex
%% Gemeinsamer Abspann für Korrekturansicht und Leseansicht.
%% Setzt den Schalter \ifkorrekturansicht voraus (gesetzt in den
%% einbindenden Dateien latex-korrekturansicht-abspann.tex bzw.
%% latex-leseansicht-abspann.tex).
%% ---------------------------------------------------------------

\normalsize

% Das esempio-Environment wird nur in der Leseansicht benötigt
\ifkorrekturansicht\else
\newenvironment{esempio}[3]%
{
    \vspace{1.5ex}
    \rlap{\underline{#1}}
    \par
    \setlength{\parindent}{0cm}
    \nopagebreak
    \leftskip=#2cm
    \rightskip=#3cm
}
{
    \par
}
\fi

\doendnotes{C}
\bigskip
\vfill

\clearpage

\footnotesize

\ifkorrekturansicht
  \lohead{\textsc{register}}
\fi

% theindex-Environment neu definieren ohne reledmac
\makeatletter
\renewenvironment{theindex}{%
  \ifkorrekturansicht
    \section*{\indexname}%
  \else
    \subsubsection*{Index der erwähnten Entitäten}%
  \fi
  \setlength{\parindent}{0pt}%
  \setlength{\parskip}{0pt plus 0.3pt}%
  \let\item\@idxitem
}{%
  \ifkorrekturansicht\clearpage\fi
}
\makeatother

\IfFileExists{\jobname-pw.ind}{\input{\jobname-pw.ind}}{}

% Quellenangabe nur in der Leseansicht
\ifkorrekturansicht\else
% Fallback-Definitionen, falls die .tex-Datei \titel etc. nicht gesetzt hat
\providecommand{\titel}{}
\providecommand{\editorInnen}{}
\providecommand{\dateiname}{\jobname}

\vspace{3cm}

\vfill

\footnotesize
\textsc{Quelle}: \titel. Herausgegeben von {\editorInnen}. In: \emph{Arthur Schnitzler: Briefwechsel mit Autorinnen und Autoren}.
 Digitale Edition, https://schnitzler-briefe.acdh.oeaw.ac.at/{\dateiname}.html (Stand \today)
\fi

\end{document}


