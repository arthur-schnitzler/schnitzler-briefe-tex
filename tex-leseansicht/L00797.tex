%% latex-korrekturansicht-vorspann.tex
%% Vorspann für die Korrekturansicht.
%% Lädt die gemeinsame Datei latex-vorspann.tex mit gesetztem Schalter.

\newif\ifkorrekturansicht
\korrekturansichttrue

\input{../tex-inputs/latex-vorspann}


\section[Arthur Schnitzler an Hugo von Hofmannsthal, {[}29. 5. 1898?{]}]{L00797 Arthur Schnitzler an Hugo von Hofmannsthal, {[}29. 5. 1898?{]}}
\nopagebreak\mylabel{L00797v}
\rehead{ }\normalsize\beginnumbering\briefempfaengerindex{Hofmannsthal, Hugo von@\textsc{Hofmannsthal, Hugo von}!zzzSchnitzler, Arthur@\emph{von Arthur Schnitzler}!1898-05-291@{{[}29. 5. 1898?{]}}|(be}
\toendnotes[C]{\smallbreak\pagebreak[2]}\Standort{FDH, Hs-30885,48.}
\physDesc{Brief, 1 Blatt, 2 Seiten, 307 Zeichen
\newline{}Handschrift: Bleistift, deutsche Kurrent
\newline{}Ordnung: mit Bleistift von Schnitzler mutmaßlich bei der Durchsicht der Korrespondenz
                                    1929 beschriftet: »Datum?
                                       95?« }
\buchAbdrucke{\weitereDrucke{Hugo von Hofmannsthal, Arthur Schnitzler: \emph{Briefwechsel}. Frankfurt am Main: \emph{S. Fischer} 1964, S. 64.} }\toendnotes[C]{\smallbreak}
\pstart
           \noindent{}{\pb}Lieber Hugo, ich höre eben, wir haben eine Loge zu \label{K_L00797-1v}\edtext{\textsc{\uline{Norma}}\pwindex{Norma@\emph{Norma}|pw}}{\lemma{\textnormal{\emph{Norma}}}\Cendnote{\textnormal{Der einzige belegbare Besuch Schnitzlers in \emph{Norma}\pwindex{Norma@\emph{Norma}|pwk} fand am 29. 5. 1898 statt (\emph{Cambridge University Library}, A 179). Für den selben Abend vermerkt das \emph{Tagebuch}\pwindex{Tagebuch@\emph{Tagebuch}|pwk} ein Abendessen
                  mit Hofmannsthal\pwindex{Hofmannsthal, Hugo von 1874-02-01 – 1929-07-15@\textsc{Hofmannsthal, Hugo von} (1874-02-01 – 1929-07-15), \emph{Schriftsteller/Schriftstellerin}|pwk}.}}}\label{K_L00797-1} (\textsc{Lehmann}\pwindex{Lehmann, Lilli 1848-11-24 – 1929-05-16@\textsc{Lehmann, Lilli} (1848-11-24 – 1929-05-16), \emph{Sänger/Sängerin, Gesangspädagoge/Gesangspädagogin}|pw}); bitte kommen Sie vielleicht ſtatt ins Reſidenzhotel\oindex{Residenzhotel@\textbf{Residenzhotel}, \emph{Hotel (K.HTL)}|pw} um ½ 9 oder wann Sie wollen in die Loge
               (2. Stock, \strikeout{links, 9} rechts, 9). – Wenn Sie keine {\pb}Luſt haben (was mir leid thäte), so kommen Sie ins Reſidenzhotel\oindex{Residenzhotel@\textbf{Residenzhotel}, \emph{Hotel (K.HTL)}|pw}, aber etwas ſpäter.\pend
           
\pstart
           Herzlichen Gruſs{\\[\baselineskip]}Ihr \spacefill\mbox{Arthur}\pend
           \leftskip=0em{}\selectlanguage{ngerman}\endnumbering\briefempfaengerindex{Hofmannsthal, Hugo von@\textsc{Hofmannsthal, Hugo von}!zzzSchnitzler, Arthur@\emph{von Arthur Schnitzler}!1898-05-291@{{[}29. 5. 1898?{]}}|)be}\mylabel{L00797h}  \normalsize

\doendnotes{C}
\bigskip
\vfill

\clearpage

\footnotesize

\lohead{\textsc{register}}

% Definiere theindex-Environment komplett neu ohne reledmac
\makeatletter
\renewenvironment{theindex}{%
  \section*{\indexname}%
  \setlength{\parindent}{0pt}%
  \setlength{\parskip}{0pt plus 0.3pt}%
  \let\item\@idxitem
}{%
  \clearpage
}
\makeatother

\IfFileExists{\jobname-pw.ind}{\input{\jobname-pw.ind}}{}

\end{document}

      