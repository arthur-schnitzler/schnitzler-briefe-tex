%% latex-leseansicht-vorspann.tex
%% Vorspann für die Leseansicht.
%% Lädt die gemeinsame Datei latex-vorspann.tex mit nicht gesetztem Schalter.

\newif\ifkorrekturansicht
\korrekturansichtfalse

\input{../tex-inputs/latex-vorspann}


\section[Arthur Schnitzler an Hugo von Hofmannsthal, {{[}}29. 5. 1898?{{]}}]{L00797 Arthur Schnitzler an Hugo von Hofmannsthal, {[}29. 5. 1898?{]}}
\nopagebreak\mylabel{L00797v}
\rehead{ }\normalsize\beginnumbering\briefempfaengerindex{Hofmannsthal, Hugo von@\textsc{Hofmannsthal, Hugo von}!zzzSchnitzler, Arthur@\emph{von Arthur Schnitzler}!1898-05-291@{{[}29. 5. 1898?{]}}|(be}
\toendnotes[C]{\smallbreak\pagebreak[2]}
\correspDesc{Versand  durch Arthur Schnitzler am [29. 5. 1898?] in Wien
\newline{}Erhalt  durch Hugo von Hofmannsthal im Zeitraum [29. 5. 1898
                  – 2. 6. 1898?] in Wien}\toendnotes[C]{\smallbreak}
\Standort{FDH, Hs-30885,48.}
\physDesc{Brief, 1 Blatt, 2 Seiten, 307 Zeichen
\newline{}Handschrift: Bleistift, deutsche Kurrent
\newline{}Ordnung: mit Bleistift von Schnitzler mutmaßlich bei der Durchsicht der Korrespondenz
                                    1929 beschriftet: »Datum?
                                       95?« }
\buchAbdrucke{\weitereDrucke{Hugo von Hofmannsthal, Arthur Schnitzler: \emph{Briefwechsel}. Herausgegeben von Therese Nickl und Heinrich Schnitzler. Frankfurt am Main: \emph{S. Fischer} 1964, S. 64.} }\toendnotes[C]{\smallbreak}
\pstart
           \noindent{}{\pb}Lieber Hugo, ich höre eben, wir haben eine Loge zu \label{K_L00797-1v}\edtext{\textsc{\uline{Norma}}\pwindex{\textcolor{red}{\textsuperscript{XXXX indx1}}!Norma@\strich\emph{Norma}|pw}}{\lemma{\textnormal{\emph{Norma}}}\Cendnote{\textnormal{Der einzige belegbare Besuch Schnitzlers in \emph{Norma}\pwindex{\textcolor{red}{\textsuperscript{XXXX indx1}}!Norma@\strich\emph{Norma}|pwk} fand am 29. 5. 1898 statt (\emph{Cambridge University Library}, A 179). Für den selben Abend vermerkt das \emph{Tagebuch}\pwindex{Schnitzler, Arthur 15.\,5.\,1862 Wien – 21.\,10.\,1931 ebd.@\textsc{Schnitzler, Arthur} (15.\,5.\,1862 Wien – 21.\,10.\,1931 ebd.), \emph{Schriftsteller, Mediziner}!Tagebuch@\strich\emph{Tagebuch}|pwk} ein Abendessen
                  mit Hofmannsthal\pwindex{Hofmannsthal, Hugo von 1.\,2.\,1874 Wien – 15.\,7.\,1929 Rodaun@\textsc{Hofmannsthal, Hugo von} (1.\,2.\,1874 Wien – 15.\,7.\,1929 Rodaun), \emph{Schriftsteller}|pwk}.}}}\label{K_L00797-1} (\textsc{Lehmann}\pwindex{Lehmann, Lilli 24.\,11.\,1848 Würzburg – 16.\,5.\,1929 Berlin@\textsc{Lehmann, Lilli} (24.\,11.\,1848 Würzburg – 16.\,5.\,1929 Berlin), \emph{Sängerin, Gesangspädagogin}|pw}); bitte kommen Sie vielleicht{ }ſtatt ins Reſidenzhotel\oindex{Wien@\textbf{Wien}!I., Innere Stadt@\textbf{I., Innere Stadt}!Residenzhotel@\textbf{Residenzhotel}, \emph{Hotel}|pw} um ½ 9 oder wann Sie wollen in die Loge
               (2. Stock, \strikeout{links, 9} rechts, 9). – Wenn Sie keine {\pb}Luſt haben (was mir leid thäte), so kommen Sie ins Reſidenzhotel\oindex{Wien@\textbf{Wien}!I., Innere Stadt@\textbf{I., Innere Stadt}!Residenzhotel@\textbf{Residenzhotel}, \emph{Hotel}|pw}, aber etwas{ }ſpäter.\pend
           
\pstart
           Herzlichen Gruſs{\\[\baselineskip]}Ihr \spacefill\mbox{Arthur}\pend
           \leftskip=0em{}\selectlanguage{ngerman}\endnumbering\briefempfaengerindex{Hofmannsthal, Hugo von@\textsc{Hofmannsthal, Hugo von}!zzzSchnitzler, Arthur@\emph{von Arthur Schnitzler}!1898-05-291@{{[}29. 5. 1898?{]}}|)be}\mylabel{L00797h}  \newcommand{\dateiname}{L00797}\newcommand{\titel}{Arthur Schnitzler an Hugo von Hofmannsthal, [29. 5. 1898?]}\newcommand{\editorInnen}{Martin Anton Müller und Gerd-Hermann Susen}%% latex-leseansicht-abspann.tex
%% Abspann für die Leseansicht.
%% Der Schalter \ifkorrekturansicht ist bereits durch den Vorspann gesetzt.

%% latex-abspann.tex
%% Gemeinsamer Abspann für Korrekturansicht und Leseansicht.
%% Setzt den Schalter \ifkorrekturansicht voraus (gesetzt in den
%% einbindenden Dateien latex-korrekturansicht-abspann.tex bzw.
%% latex-leseansicht-abspann.tex).
%% ---------------------------------------------------------------

\normalsize

% Das esempio-Environment wird nur in der Leseansicht benötigt
\ifkorrekturansicht\else
\newenvironment{esempio}[3]%
{
    \vspace{1.5ex}
    \rlap{\underline{#1}}
    \par
    \setlength{\parindent}{0cm}
    \nopagebreak
    \leftskip=#2cm
    \rightskip=#3cm
}
{
    \par
}
\fi

\doendnotes{C}
\bigskip
\vfill

\clearpage

\footnotesize

\ifkorrekturansicht
  \lohead{\textsc{register}}
\fi

% theindex-Environment neu definieren ohne reledmac
\makeatletter
\renewenvironment{theindex}{%
  \ifkorrekturansicht
    \section*{\indexname}%
  \else
    \subsubsection*{Index der erwähnten Entitäten}%
  \fi
  \setlength{\parindent}{0pt}%
  \setlength{\parskip}{0pt plus 0.3pt}%
  \let\item\@idxitem
}{%
  \ifkorrekturansicht\clearpage\fi
}
\makeatother

\IfFileExists{\jobname-pw.ind}{\input{\jobname-pw.ind}}{}

% Quellenangabe nur in der Leseansicht
\ifkorrekturansicht\else
% Fallback-Definitionen, falls die .tex-Datei \titel etc. nicht gesetzt hat
\providecommand{\titel}{}
\providecommand{\editorInnen}{}
\providecommand{\dateiname}{\jobname}

\vspace{3cm}

\vfill

\footnotesize
\textsc{Quelle}: \titel. Herausgegeben von {\editorInnen}. In: \emph{Arthur Schnitzler: Briefwechsel mit Autorinnen und Autoren}.
 Digitale Edition, https://schnitzler-briefe.acdh.oeaw.ac.at/{\dateiname}.html (Stand \today)
\fi

\end{document}


