%% latex-leseansicht-vorspann.tex
%% Vorspann für die Leseansicht.
%% Lädt die gemeinsame Datei latex-vorspann.tex mit nicht gesetztem Schalter.

\newif\ifkorrekturansicht
\korrekturansichtfalse

\input{../tex-inputs/latex-vorspann}


\section[ Arthur Schnitzler an Felix Salten, 10. 2. 1927]{L03022 Arthur Schnitzler an Felix Salten,  10. 2. 1927}
\nopagebreak\mylabel{L03022v}
\rehead{ }\normalsize\beginnumbering\briefempfaengerindex{Salten, Felix@\textsc{Salten, Felix}!zzzSchnitzler, Arthur@\emph{von Arthur Schnitzler}!1927-02-101@{10. 2. 1927}|(be}
\toendnotes[C]{\smallbreak\pagebreak[2]}
\correspDesc{Versand  durch Arthur Schnitzler am 10. 2. 1927 in Wien
\newline{}Erhalt  durch Felix Salten im Zeitraum [11. 2. 1927
                  – 15. 2. 1927?] in Dresden}\toendnotes[C]{\smallbreak}
\Standort{Wienbibliothek im Rathaus, ZPH 1681, 2.1.516.}
\physDesc{Brief, 1 Blatt, 2 Seiten, 1237 Zeichen
\newline{}Handschrift: Bleistift, lateinische Kurrent
\newline{}Ordnung: mit Bleistift von unbekannter Hand nummeriert: »3« }
\buchAbdrucke{\weitereDrucke{Arthur Schnitzler: \emph{Briefe 1913–1931}. Herausgegeben von Peter Michael Braunwarth, Richard Miklin, Susanne Pertlik und Heinrich Schnitzler. Frankfurt am Main: \emph{S. Fischer} 1984, S. 470–471.} }\toendnotes[C]{\smallbreak}
\pstart
           \raggedleft{}{\pb}Wien\oindex{Wien@\textbf{Wien}, \emph{Verwaltungsgebiet}|pw}{ }10. 2. 927\pend
           \vspace{0.5em}
\pstart
           lieber, ich dank Ihnen sehr für Ihre \label{K_L03022-1v}\edtext{Karte}{\lemma{\textnormal{\emph{Karte}}}\Cendnote{\textnormal{XXXX Auszeichnungsfehler: Dokument L03596 nicht gefunden.
               }}}\label{K_L03022-1}. Glauben Sie nicht, daſs ich weniger und daſ\textcolor{gray}{s} ich anders
               Ihrer denke als in früherer Zeit. Daſs ich so wenig sicht- u hörbar bin liegt zum
               Theil an der etwas complicirten \introOben{}(\introOben{}und zeitraubenden\introOben{})\introOben{} Form\strikeout{)} die meine Existenz
                  angeno{\geminationm}en hat; und gar nicht daran, dſs \strikeout{ich} es mich nicht kü{\geminationm}ern sollte, wie es Ihnen geht.
               Ich wußte, dſs Sie in Dresden\oindex{Dresden@\textbf{Dresden}|pw} im Sanatorium\oindex{Sanatorium am Königspark@\textbf{Sanatorium am Königspark}, \emph{Sanatorium}|pwv}{ }\strikeout{\textcolor{gray}{×}\-\textcolor{gray}{×}\-\textcolor{gray}{×}\-\textcolor{gray}{×}} sind; \label{K_L03022-2v}\edtext{bei Zsolnays\pwindex{Zsolnay, Adolph von 30.\,9.\,1863 Sillein – 8.\,7.\,1932 Wien@\textsc{Zsolnay, Adolph von} (30.\,9.\,1863 Sillein – 8.\,7.\,1932 Wien), \emph{Kaufmann}|pw}\pwindex{Zsolnay, Amanda von 8.\,2.\,1876 Dresden – 10.\,8.\,1956 London@\textsc{Zsolnay, Amanda von} (8.\,2.\,1876 Dresden – 10.\,8.\,1956 London), \emph{Kunstsammlerin}|pw} (zu Keyserlings\pwindex{Keyserling, Eduard von 15.\,5.\,1855 Tels-Paddern – 28.\,9.\,1918 München@\textsc{Keyserling, Eduard von} (15.\,5.\,1855 Tels-Paddern – 28.\,9.\,1918 München), \emph{Schriftsteller}|pw} Ehren)}{\lemma{\textnormal{\emph{bei … Ehren)}}}\Cendnote{\textnormal{Siehe A. S.: \emph{Tagebuch}, 6. 2. 1927.
               }}}\label{K_L03022-2} hört ichs zuerst, und eben erst sprach auch Benedikt\pwindex{Benedikt, Ernst 20.\,5.\,1882 Wien – 28.\,12.\,1973 ebd.@\textsc{Benedikt, Ernst} (20.\,5.\,1882 Wien – 28.\,12.\,1973 ebd.), \emph{Schriftsteller, Journalist}|pw}, bei dem ich heute zufällg zu Mittag
               aſs, davon, von Ihrer Arbeitskraft und allerlei sehr herzliches. Auch von dem weiten
               Wiederhall Ihres schönen \label{K_L03022-3v}\edtext{Bambibuch\pwindex{Salten, Felix 6.\,9.\,1869 Budapest – 8.\,10.\,1945 Zürich@\textsc{Salten, Felix} (6.\,9.\,1869 Budapest – 8.\,10.\,1945 Zürich), \emph{Schriftsteller, Journalist, Chefredakteur}!Bambi. Eine Lebensgeschichte aus dem Walde@\strich\emph{Bambi. Eine Lebensgeschichte aus dem Walde}|pw}}{\lemma{\textnormal{\emph{Bambibuch}}}\Cendnote{\textnormal{Schnitzler bezog sich hier nicht auf die
                     1922 bei \emph{Ullstein}\orgindex{Ullstein Verlag@Ullstein Verlag|pwk}
                  erschienene \emph{Bambi}\pwindex{Salten, Felix 6.\,9.\,1869 Budapest – 8.\,10.\,1945 Zürich@\textsc{Salten, Felix} (6.\,9.\,1869 Budapest – 8.\,10.\,1945 Zürich), \emph{Schriftsteller, Journalist, Chefredakteur}!Bambi. Eine Lebensgeschichte aus dem Walde@\strich\emph{Bambi. Eine Lebensgeschichte aus dem Walde}|pwk}-Ausgabe, sondern jene, die
                     1926 bei \emph{Paul
                     Zsolnay}\orgindex{Paul Zsolnay Verlag@Paul Zsolnay Verlag|pwk} erschienen war.}}}\label{K_L03022-3}es weiſs ich und dſs Sie einen \label{K_L03022-4v}\edtext{Roman\pwindex{Salten, Felix 6.\,9.\,1869 Budapest – 8.\,10.\,1945 Zürich@\textsc{Salten, Felix} (6.\,9.\,1869 Budapest – 8.\,10.\,1945 Zürich), \emph{Schriftsteller, Journalist, Chefredakteur}!Martin Overbeck. Der Roman eines reichen jungen Mannes@\strich\emph{Martin Overbeck. Der Roman eines reichen jungen Mannes}|pwuv}}{\lemma{\textnormal{\emph{Roman}}}\Cendnote{\textnormal{Eventuell ist \emph{Martin Overbeck. Der Roman eines reichen jungen Mannes}\pwindex{Salten, Felix 6.\,9.\,1869 Budapest – 8.\,10.\,1945 Zürich@\textsc{Salten, Felix} (6.\,9.\,1869 Budapest – 8.\,10.\,1945 Zürich), \emph{Schriftsteller, Journalist, Chefredakteur}!Martin Overbeck. Der Roman eines reichen jungen Mannes@\strich\emph{Martin Overbeck. Der Roman eines reichen jungen Mannes}|pwk} gemeint,
                  der bereits im April 1927 veröffentlicht wurde
                  und folglich schon fertiggeschrieben gewesen sein müsste.}}}\label{K_L03022-4} schreiben\textcolor{gray}{.}{ }{\pb}Un\textcolor{gray}{d} habe neulich mit Ergriffenheit Ihr
                  \label{K_L03022-5v}\edtext{Feu{[}i{]}lleton\pwindex{Salten, Felix 6.\,9.\,1869 Budapest – 8.\,10.\,1945 Zürich@\textsc{Salten, Felix} (6.\,9.\,1869 Budapest – 8.\,10.\,1945 Zürich), \emph{Schriftsteller, Journalist, Chefredakteur}!Theodor@\strich\emph{Theodor}|pwv}}{\lemma{\textnormal{\emph{Feuilleton}}}\Cendnote{\textnormal{Felix Salten\pwindex{Salten, Felix 6.\,9.\,1869 Budapest – 8.\,10.\,1945 Zürich@\textsc{Salten, Felix} (6.\,9.\,1869 Budapest – 8.\,10.\,1945 Zürich), \emph{Schriftsteller, Journalist, Chefredakteur}|pwk}: \emph{Theodor}\pwindex{Salten, Felix 6.\,9.\,1869 Budapest – 8.\,10.\,1945 Zürich@\textsc{Salten, Felix} (6.\,9.\,1869 Budapest – 8.\,10.\,1945 Zürich), \emph{Schriftsteller, Journalist, Chefredakteur}!Theodor@\strich\emph{Theodor}|pwk}. In: \emph{Neue
                        Freie Presse}\pwindex{Neue Freie Presse@\emph{Neue Freie Presse}|pwk}, Nr. 22.381, 6. 1. 1927,
                     Morgenblatt, S. 13.}}}\label{K_L03022-5} (du{\geminationm}es Wort)
               über Ihren Bruder\pwindex{Salzmann, Theodor 1867 Budapest – 12.\,12.\,1926 Wien@\textsc{Salzmann, Theodor} (1867 Budapest – 12.\,12.\,1926 Wien)|pwv} gelesen.
               Und mit Vergnügen gehört, daſs Annerl\pwindex{Rehmann, Anna Katharina 18.\,8.\,1904 Wien – 27.\,3.\,1977 Zürich@\textsc{Rehmann, Anna Katharina} (18.\,8.\,1904 Wien – 27.\,3.\,1977 Zürich), \emph{Schauspielerin, Übersetzerin}|pw} (we{\geminationn} man noch so sagen darf) nun auch ein
               schauspielerisches Talent in sich entdeckt hat und als »\label{K_L03022-6v}\edtext{Mitgefangne\pwindex{Bourdet, Édouard 26.\,10.\,1887 Saint-Germain-en-Laye – 17.\,1.\,1945 Paris@\textsc{Bourdet, Édouard} (26.\,10.\,1887 Saint-Germain-en-Laye – 17.\,1.\,1945 Paris), \emph{Schriftsteller}!Gefangene. Schauspiel in drei Akten@\strich\emph{Die Gefangene. Schauspiel in drei Akten}|pwv}« von 
               Helene Thimig\pwindex{Thimig, Helene 5.\,6.\,1889 Wien – 7.\,11.\,1974 ebd.@\textsc{Thimig, Helene} (5.\,6.\,1889 Wien – 7.\,11.\,1974 ebd.), \emph{Schauspielerin}|pw}}{\lemma{\textnormal{\emph{Mitgefangne« … Thimig}}}\Cendnote{\textnormal{Helene Thimig\pwindex{Thimig, Helene 5.\,6.\,1889 Wien – 7.\,11.\,1974 ebd.@\textsc{Thimig, Helene} (5.\,6.\,1889 Wien – 7.\,11.\,1974 ebd.), \emph{Schauspielerin}|pwk} befand sich
                  mit dem »Schauspiel in drei Akten« \emph{Die Gefangene}\pwindex{Bourdet, Édouard 26.\,10.\,1887 Saint-Germain-en-Laye – 17.\,1.\,1945 Paris@\textsc{Bourdet, Édouard} (26.\,10.\,1887 Saint-Germain-en-Laye – 17.\,1.\,1945 Paris), \emph{Schriftsteller}!Gefangene. Schauspiel in drei Akten@\strich\emph{Die Gefangene. Schauspiel in drei Akten}|pwk} (\emph{La Prisonnière}\pwindex{Bourdet, Édouard 26.\,10.\,1887 Saint-Germain-en-Laye – 17.\,1.\,1945 Paris@\textsc{Bourdet, Édouard} (26.\,10.\,1887 Saint-Germain-en-Laye – 17.\,1.\,1945 Paris), \emph{Schriftsteller}!Prisonnière@\strich\emph{La Prisonnière}|pwk}) von Édouard Bourdet\pwindex{Bourdet, Édouard 26.\,10.\,1887 Saint-Germain-en-Laye – 17.\,1.\,1945 Paris@\textsc{Bourdet, Édouard} (26.\,10.\,1887 Saint-Germain-en-Laye – 17.\,1.\,1945 Paris), \emph{Schriftsteller}|pwk},
                  deutsch von Stefan Hock\pwindex{Hock, Stefan 9.\,1.\,1877 Wien – 19.\,5.\,1947 London@\textsc{Hock, Stefan} (9.\,1.\,1877 Wien – 19.\,5.\,1947 London), \emph{Dramaturg}|pwk} auf Tournee. Das Stück hatte am 21. 5. 1926 in Wien\oindex{Wien@\textbf{Wien}, \emph{Verwaltungsgebiet}|pwk}
                  die deutschsprachige Uraufführung\eventindex{Theater in der Josefstadt@\textbf{Theater in der Josefstadt}!deutschsprachige Uraufführung von Die Gefangene, 21.5.1926@deutschsprachige Uraufführung von Die Gefangene, 21.5.1926|pwkv} gehabt. Schnitzler sah die Aufführung\eventindex{Theater in der Josefstadt@\textbf{Theater in der Josefstadt}!Aufführung von Die Gefangene, 7.6.1926@Aufführung von Die Gefangene, 7.6.1926|pwkv} am 7. 6. 1926.
                  Für die Tournee waren die meisten Rollen neu besetzt worden, unter anderem mit Anna Katharina Salten\pwindex{Rehmann, Anna Katharina 18.\,8.\,1904 Wien – 27.\,3.\,1977 Zürich@\textsc{Rehmann, Anna Katharina} (18.\,8.\,1904 Wien – 27.\,3.\,1977 Zürich), \emph{Schauspielerin, Übersetzerin}|pwk}.}}}\label{K_L03022-6}
               in Deutschland\oindex{Deutschland@\textbf{Deutschland}|pw} herumreist. Bescheidene Stichproben
               von meinem Wissen um Sie. Ich hoffe, Sie ergänzen \substVorne{}\textsuperscript{\textcolor{gray}{m}}\substDazwischen{}es\substHinten{} bald. Wa{\geminationn} ko{\geminationm}en Sie wieder? Ich
               habe vorläufg keine Reise-Absichten. Also »klopfen« oder telefoniren Sie bald. Ich
               freu mich darauf\textcolor{gray}{,} Sie endlich einmal wieder ausführlicher zu sprechen.\pend
           
\pstart
           Von Her\textcolor{gray}{zen} Ihr {\\[\baselineskip]}\spacefill\mbox{Arthur}\pend
           \leftskip=0em{}\selectlanguage{ngerman}\endnumbering\briefempfaengerindex{Salten, Felix@\textsc{Salten, Felix}!zzzSchnitzler, Arthur@\emph{von Arthur Schnitzler}!1927-02-101@{10. 2. 1927}|)be}\mylabel{L03022h}  \newcommand{\dateiname}{L03022}\newcommand{\titel}{Arthur Schnitzler an Felix Salten, 10. 2. 1927}\newcommand{\editorInnen}{Martin Anton Müller und Laura Untner}%% latex-leseansicht-abspann.tex
%% Abspann für die Leseansicht.
%% Der Schalter \ifkorrekturansicht ist bereits durch den Vorspann gesetzt.

%% latex-abspann.tex
%% Gemeinsamer Abspann für Korrekturansicht und Leseansicht.
%% Setzt den Schalter \ifkorrekturansicht voraus (gesetzt in den
%% einbindenden Dateien latex-korrekturansicht-abspann.tex bzw.
%% latex-leseansicht-abspann.tex).
%% ---------------------------------------------------------------

\normalsize

% Das esempio-Environment wird nur in der Leseansicht benötigt
\ifkorrekturansicht\else
\newenvironment{esempio}[3]%
{
    \vspace{1.5ex}
    \rlap{\underline{#1}}
    \par
    \setlength{\parindent}{0cm}
    \nopagebreak
    \leftskip=#2cm
    \rightskip=#3cm
}
{
    \par
}
\fi

\doendnotes{C}
\bigskip
\vfill

\clearpage

\footnotesize

\ifkorrekturansicht
  \lohead{\textsc{register}}
\fi

% theindex-Environment neu definieren ohne reledmac
\makeatletter
\renewenvironment{theindex}{%
  \ifkorrekturansicht
    \section*{\indexname}%
  \else
    \subsubsection*{Index der erwähnten Entitäten}%
  \fi
  \setlength{\parindent}{0pt}%
  \setlength{\parskip}{0pt plus 0.3pt}%
  \let\item\@idxitem
}{%
  \ifkorrekturansicht\clearpage\fi
}
\makeatother

\IfFileExists{\jobname-pw.ind}{\input{\jobname-pw.ind}}{}

% Quellenangabe nur in der Leseansicht
\ifkorrekturansicht\else
% Fallback-Definitionen, falls die .tex-Datei \titel etc. nicht gesetzt hat
\providecommand{\titel}{}
\providecommand{\editorInnen}{}
\providecommand{\dateiname}{\jobname}

\vspace{3cm}

\vfill

\footnotesize
\textsc{Quelle}: \titel. Herausgegeben von {\editorInnen}. In: \emph{Arthur Schnitzler: Briefwechsel mit Autorinnen und Autoren}.
 Digitale Edition, https://schnitzler-briefe.acdh.oeaw.ac.at/{\dateiname}.html (Stand \today)
\fi

\end{document}


