%% latex-korrekturansicht-vorspann.tex
%% Vorspann für die Korrekturansicht.
%% Lädt die gemeinsame Datei latex-vorspann.tex mit gesetztem Schalter.

\newif\ifkorrekturansicht
\korrekturansichttrue

\input{../tex-inputs/latex-vorspann}


\section[ Arthur Schnitzler an Felix Salten, 10. 2. 1927]{L03022 Arthur Schnitzler an Felix Salten, 10. 2. 1927}
\nopagebreak\mylabel{L03022v}
\rehead{ }\normalsize\beginnumbering\briefempfaengerindex{Salten, Felix@\textsc{Salten, Felix}!zzzSchnitzler, Arthur@\emph{von Arthur Schnitzler}!1927-02-101@{10. 2. 1927}|(be}
\toendnotes[C]{\smallbreak\pagebreak[2]}\Standort{Wienbibliothek im Rathaus, ZPH 1681, 2.1.516.}
\physDesc{Brief, 1 Blatt, 2 Seiten, 1237 Zeichen
\newline{}Handschrift: Bleistift, lateinische Kurrent
\newline{}Ordnung: mit Bleistift von unbekannter Hand nummeriert: »3« }
\buchAbdrucke{\weitereDrucke{Arthur Schnitzler: \emph{Briefe 1913–1931}. Frankfurt am Main: \emph{S. Fischer} 1984, S. 470–471.} }\toendnotes[C]{\smallbreak}
\pstart
           \raggedleft{}{\pb}Wien\oindex{Wien@\textbf{Wien}, \emph{A.ADM2}|pw}{ }10. 2. 927\pend
           \vspace{0.5em}
\pstart
           lieber, ich dank Ihnen sehr für Ihre \label{K_L03022-1v}\edtext{Karte}{\lemma{\textnormal{\emph{Karte}}}\Cendnote{\textnormal{Felix Salten an Arthur Schnitzler, 8. 2. 1927.
               }}}\label{K_L03022-1}. Glauben Sie nicht, daſs ich weniger und daſ\textcolor{gray}{s} ich anders
               Ihrer denke als in früherer Zeit. Daſs ich so wenig sicht- u hörbar bin liegt zum
               Theil an der etwas complicirten \introOben{}(\introOben{}und zeitraubenden\introOben{})\introOben{} Form\strikeout{)} die meine Existenz
                  angeno{\geminationm}en hat; und gar nicht daran, dſs \strikeout{ich} es mich nicht kü{\geminationm}ern sollte, wie es Ihnen geht.
               Ich wußte, dſs Sie in Dresden\oindex{Dresden@\textbf{Dresden}, \emph{P.PPLA}|pw} im Sanatorium\oindex{Sanatorium am Koenigspark@\textbf{Sanatorium am Königspark}, \emph{Sanatorium (K.SAN)}|pwv}{ }\strikeout{\textcolor{gray}{×}\-\textcolor{gray}{×}\-\textcolor{gray}{×}\-\textcolor{gray}{×}} sind; \label{K_L03022-2v}\edtext{bei Zsolnays\pwindex{Zsolnay, Adolph von 30.09.1863 – 08.07.1932@\textsc{Zsolnay, Adolph von} (30.09.1863 – 08.07.1932), \emph{Kaufmann/Kauffrau}|pw}\pwindex{Zsolnay, Amanda von 08.02.1876 – 10.08.1956@\textsc{Zsolnay, Amanda von} (08.02.1876 – 10.08.1956), \emph{Kunstsammler/Kunstsammlerin}|pw} (zu Keyserlings\pwindex{Keyserling, Eduard von 15.05.1855 – 28.09.1918@\textsc{Keyserling, Eduard von} (15.05.1855 – 28.09.1918), \emph{Schriftsteller/Schriftstellerin}|pw} Ehren)}{\lemma{\textnormal{\emph{bei … Ehren)}}}\Cendnote{\textnormal{Siehe A. S.: \emph{Tagebuch}, 6. 2. 1927.
               }}}\label{K_L03022-2} hört ichs zuerst, und eben erst sprach auch Benedikt\pwindex{Benedikt, Ernst 20.05.1882 – 28.12.1973@\textsc{Benedikt, Ernst} (20.05.1882 – 28.12.1973), \emph{Schriftsteller/Schriftstellerin, Journalist/Journalistin}|pw}, bei dem ich heute zufällg zu Mittag
               aſs, davon, von Ihrer Arbeitskraft und allerlei sehr herzliches. Auch von dem weiten
               Wiederhall Ihres schönen \label{K_L03022-3v}\edtext{Bambibuch\pwindex{Bambi. Eine Lebensgeschichte aus dem Walde@\emph{Bambi. Eine Lebensgeschichte aus dem Walde}|pw}}{\lemma{\textnormal{\emph{Bambibuch}}}\Cendnote{\textnormal{Schnitzler bezog sich hier nicht auf die
                     1922 bei \emph{Ullstein}\orgindex{Ullstein Verlag@Ullstein Verlag|pwk}
                  erschienene \emph{Bambi}\pwindex{Bambi. Eine Lebensgeschichte aus dem Walde@\emph{Bambi. Eine Lebensgeschichte aus dem Walde}|pwk}-Ausgabe, sondern jene, die
                     1926 bei \emph{Paul
                     Zsolnay}\orgindex{Paul Zsolnay Verlag@Paul Zsolnay Verlag|pwk} erschienen war.}}}\label{K_L03022-3}es weiſs ich und dſs Sie einen \label{K_L03022-4v}\edtext{Roman\pwindex{Martin Overbeck. Der Roman eines reichen jungen Mannes@\emph{Martin Overbeck. Der Roman eines reichen jungen Mannes}|pwuv}}{\lemma{\textnormal{\emph{Roman}}}\Cendnote{\textnormal{Eventuell ist \emph{Martin Overbeck. Der Roman eines reichen jungen Mannes}\pwindex{Martin Overbeck. Der Roman eines reichen jungen Mannes@\emph{Martin Overbeck. Der Roman eines reichen jungen Mannes}|pwk} gemeint,
                  der bereits im April 1927 veröffentlicht wurde
                  und folglich schon fertiggeschrieben gewesen sein müsste.}}}\label{K_L03022-4} schreiben\textcolor{gray}{.}{ }{\pb}Un\textcolor{gray}{d} habe neulich mit Ergriffenheit Ihr
                  \label{K_L03022-5v}\edtext{Feu{[}i{]}lleton\pwindex{Theodor@\emph{Theodor}|pwv}}{\lemma{\textnormal{\emph{Feuilleton}}}\Cendnote{\textnormal{Felix Salten\pwindex{Salten, Felix 06.09.1869 – 08.10.1945@\textsc{Salten, Felix} (06.09.1869 – 08.10.1945), \emph{Schriftsteller/Schriftstellerin, Journalist/Journalistin, Chefredakteur/Chefredakteurin}|pwk}: \emph{Theodor}\pwindex{Theodor@\emph{Theodor}|pwk}. In: \emph{Neue
                        Freie Presse}\pwindex{Neue Freie Presse@\emph{Neue Freie Presse}|pwk}, Nr. 22.381, 6. 1. 1927,
                     Morgenblatt, S. 13.}}}\label{K_L03022-5} (du{\geminationm}es Wort)
               über Ihren Bruder\pwindex{Salzmann, Theodor 1867 – 1926-12-12@\textsc{Salzmann, Theodor} (1867 – 1926-12-12)|pwv} gelesen.
               Und mit Vergnügen gehört, daſs Annerl\pwindex{Rehmann, Anna Katharina 18.08.1904 – 27.03.1977@\textsc{Rehmann, Anna Katharina} (18.08.1904 – 27.03.1977), \emph{Schauspieler/Schauspielerin, Übersetzer/Übersetzerin}|pw} (we{\geminationn} man noch so sagen darf) nun auch ein
               schauspielerisches Talent in sich entdeckt hat und als »\label{K_L03022-6v}\edtext{Mitgefangne\pwindex{Gefangene. Schauspiel in drei Akten@\emph{Die Gefangene. Schauspiel in drei Akten}|pwv}« von 
               Helene Thimig\pwindex{Thimig, Helene 05.06.1889 – 07.11.1974@\textsc{Thimig, Helene} (05.06.1889 – 07.11.1974), \emph{Schauspieler/Schauspielerin}|pw}}{\lemma{\textnormal{\emph{Mitgefangne« … Thimig}}}\Cendnote{\textnormal{Helene Thimig\pwindex{Thimig, Helene 05.06.1889 – 07.11.1974@\textsc{Thimig, Helene} (05.06.1889 – 07.11.1974), \emph{Schauspieler/Schauspielerin}|pwk} befand sich
                  mit dem »Schauspiel in drei Akten« \emph{Die Gefangene}\pwindex{Gefangene. Schauspiel in drei Akten@\emph{Die Gefangene. Schauspiel in drei Akten}|pwk} (\emph{La Prisonnière}\pwindex{Prisonniere@\emph{La Prisonnière}|pwk}) von Édouard Bourdet\pwindex{Bourdet, Edouard 26.10.1887 – 17.01.1945@\textsc{Bourdet, Édouard} (26.10.1887 – 17.01.1945), \emph{Schriftsteller/Schriftstellerin}|pwk},
                  deutsch von Stefan Hock\pwindex{Hock, Stefan 09.01.1877 – 1947-05-19@\textsc{Hock, Stefan} (09.01.1877 – 1947-05-19), \emph{Dramaturg/Dramaturgin}|pwk} auf Tournee. Das Stück hatte am 21. 5. 1921 in Wien\oindex{Wien@\textbf{Wien}, \emph{A.ADM2}|pwk}
                  die deutschsprachige Uraufführung gehabt. Schnitzler sah die Aufführung am 7. 6. 1926.
                  Für die Tournee waren die meisten Rollen neu besetzt worden, unter anderem mit Anna Katharina Salten\pwindex{Rehmann, Anna Katharina 18.08.1904 – 27.03.1977@\textsc{Rehmann, Anna Katharina} (18.08.1904 – 27.03.1977), \emph{Schauspieler/Schauspielerin, Übersetzer/Übersetzerin}|pwk}.}}}\label{K_L03022-6}
               in Deutschland\oindex{Deutschland@\textbf{Deutschland}, \emph{A.PCLI}|pw} herumreist. Bescheidene Stichproben
               von meinem Wissen um Sie. Ich hoffe, Sie ergänzen \substVorne{}\textsuperscript{\textcolor{gray}{m}}\substDazwischen{}es\substHinten{} bald. Wa{\geminationn} ko{\geminationm}en Sie wieder? Ich
               habe vorläufg keine Reise-Absichten. Also »klopfen« oder telefoniren Sie bald. Ich
               freu mich darauf\textcolor{gray}{,} Sie endlich einmal wieder ausführlicher zu sprechen.\pend
           
\pstart
           Von Her\textcolor{gray}{zen} Ihr {\\[\baselineskip]}\spacefill\mbox{Arthur}\pend
           \leftskip=0em{}\selectlanguage{ngerman}\endnumbering\briefempfaengerindex{Salten, Felix@\textsc{Salten, Felix}!zzzSchnitzler, Arthur@\emph{von Arthur Schnitzler}!1927-02-101@{10. 2. 1927}|)be}\mylabel{L03022h}  \normalsize

\doendnotes{C}
\bigskip
\vfill

\clearpage

\footnotesize

\lohead{\textsc{register}}

% Definiere theindex-Environment komplett neu ohne reledmac
\makeatletter
\renewenvironment{theindex}{%
  \section*{\indexname}%
  \setlength{\parindent}{0pt}%
  \setlength{\parskip}{0pt plus 0.3pt}%
  \let\item\@idxitem
}{%
  \clearpage
}
\makeatother

\IfFileExists{\jobname-pw.ind}{\input{\jobname-pw.ind}}{}

\end{document}

      