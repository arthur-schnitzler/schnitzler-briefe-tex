%% latex-leseansicht-vorspann.tex
%% Vorspann für die Leseansicht.
%% Lädt die gemeinsame Datei latex-vorspann.tex mit nicht gesetztem Schalter.

\newif\ifkorrekturansicht
\korrekturansichtfalse

\input{../tex-inputs/latex-vorspann}

\begin{center}
            \textcolor{red}{ENTWURF, NICHT FERTIG KORRIGIERT}
                      \end{center}
            
         
         \renewcommand{\erwaehntePersonen}{Personen: Ernst Benedikt, Eduard von Keyserling, Anna Katharina Rehmann, Felix Salten, Theodor Salzmann, Helene Thimig, Adolph von Zsolnay, Amanda von Zsolnay}
         \renewcommand{\erwaehnteOrte}{Orte: Deutschland, Dresden, Sanatorium am Königspark, Wien}
         \renewcommand{\erwaehnteWerke}{Werke: Bambi. Eine Lebensgeschichte aus dem Walde, Martin Overbeck. Der Roman eines reichen jungen Mannes, Neue Freie Presse, Theodor}
               \section[Arthur Schnitzler an Felix Salten, 10. 2. 1927]{ Arthur Schnitzler an Felix Salten, 10. 2. 1927}\nopagebreak\mylabel{v}\rehead{ }\begin{ledgroupsized}[t]{13cm}\normalsize\beginnumbering \toendnotes[C]{\smallbreak\pagebreak[2]} \Standort{Wienbibliothek im Rathaus, ZPH 1681, 2.1.516.}
\physDesc{Brief, 1 Blatt, 2 Seiten, 1244 Zeichen
\newline{}Handschrift: Bleistift, lateinische Kurrent
\newline{}Ordnung: mit Bleistift von unbekannter Hand Nummerierung der Blätter des
                                 Konvoluts: »3« }\toendnotes[C]{\smallbreak}\pstart
           \raggedleft{}{\pb}Wien\oindex{Wien@\textbf{Wien}|pw}{ }10. 2. 927\pend
           \pstart
           lieber, ich dank Ihnen sehr für Ihre Karte. Glauben Sie nicht, daſs
               ich weniger und daſs ich anders Ihrer denke als in früherer Zeit. Daſs ich so wenig
               sicht- u hörbar bin liegt zum Theil an der etwas complicirten (und zeitraubenden)
                  Form\strikeout{)} den meine Existenz angeno{\geminationm}en hat; und gar nicht daran, dſs \strikeout{ich} es mich nicht kü{\geminationm}ern
               sollte, wie es Ihnen geht. Ich wußte dſs Sie in Dresden\oindex{Dresden@\textbf{Dresden}|pw} im Sanatorium\oindex{Sanatorium am Koenigspark@\textbf{Sanatorium am Königspark}|pwv}{ }\strikeout{\textcolor{gray}{×}\-\textcolor{gray}{×}\-\textcolor{gray}{×}\-\textcolor{gray}{×}} sind; \label{K_L03022-1v}\edtext{bei Zsolnays\pwindex{Zsolnay, Adolph von 30.09.1863 – 08.07.1932@\textsc{Zsolnay, Adolph von} (30.09.1863 – 08.07.1932), \emph{Kaufmann}|pw}\pwindex{Zsolnay, Amanda von 08.02.1876 – 10.08.1956@\textsc{Zsolnay, Amanda von} (08.02.1876 – 10.08.1956), \emph{Sammlerin >> Kunstsammler}|pw}}{\lemma{\textnormal{\emph{bei Zsolnays}}}\Cendnote{\textnormal{siehe A. S.: \emph{Tagebuch}, 6. 2. 1927}}}\label{K_L03022-1h} (zu Keyserling\pwindex{Keyserling, Eduard von 15.05.1855 – 28.09.1918@\textsc{Keyserling, Eduard von} (15.05.1855 – 28.09.1918), \emph{Schriftsteller}|pw}s Ehren) hört ichs
               zuerst, und eben erst sprach auch Benedikt\pwindex{Benedikt, Ernst 20.05.1882 – 28.12.1973@\textsc{Benedikt, Ernst} (20.05.1882 – 28.12.1973), \emph{Schriftsteller, Journalist}|pw},
               bei dem ich heute zufällg zu Mittag aſs, davon, von Ihrer Arbeitskraft und allerlei
               sehr herzliches. Auch von dem weiten Wiederhall Ihres schönen Bambibuches\pwindex{Salten, Felix 06.09.1869 – 08.10.1945@\textsc{Salten, Felix} (06.09.1869 – 08.10.1945), \emph{Schriftsteller, Journalist}!Bambi. Eine Lebensgeschichte aus dem Walde1922-12-08@\strich\emph{Bambi. Eine Lebensgeschichte aus dem Walde} {[}1922-12-08{]}|pw} weiſs ich und dſs Sie einen \label{K_L03022-22v}\edtext{Roman\pwindex{Salten, Felix 06.09.1869 – 08.10.1945@\textsc{Salten, Felix} (06.09.1869 – 08.10.1945), \emph{Schriftsteller, Journalist}!Martin Overbeck. Der Roman eines reichen jungen MannesApril 1927@\strich\emph{Martin Overbeck. Der Roman eines reichen jungen Mannes} {[}April 1927{]}|pwuv}}{\lemma{\textnormal{\emph{Roman}}}\Cendnote{\textnormal{Eventuell gemeint ist \emph{Martin Overbeck. Der Roman eines reichen jungen Mannes}\pwindex{Salten, Felix 06.09.1869 – 08.10.1945@\textsc{Salten, Felix} (06.09.1869 – 08.10.1945), \emph{Schriftsteller, Journalist}!Martin Overbeck. Der Roman eines reichen jungen MannesApril 1927@\strich\emph{Martin Overbeck. Der Roman eines reichen jungen Mannes} {[}April 1927{]}|pwk},
                  der aber bereits im April 1927 zur Ausgabe kam und folglich schon
                  fertiggeschrieben war.}}}\label{K_L03022-22h} schreiben. {\pb}Und habe neulich mit Ergriffenheit Ihr \label{K_L03022-111v}\edtext{Feu{[}i{]}lleton\pwindex{Salten, Felix 06.09.1869 – 08.10.1945@\textsc{Salten, Felix} (06.09.1869 – 08.10.1945), \emph{Schriftsteller, Journalist}!Theodor1927-01-06@\strich\emph{Theodor} {[}1927-01-06{]}|pwv}}{\lemma{\textnormal{\emph{Feuilleton}}}\Cendnote{\textnormal{Felix Salten\pwindex{Salten, Felix 06.09.1869 – 08.10.1945@\textsc{Salten, Felix} (06.09.1869 – 08.10.1945), \emph{Schriftsteller, Journalist}|pwk}: \emph{Theodor}\pwindex{Salten, Felix 06.09.1869 – 08.10.1945@\textsc{Salten, Felix} (06.09.1869 – 08.10.1945), \emph{Schriftsteller, Journalist}!Theodor1927-01-06@\strich\emph{Theodor} {[}1927-01-06{]}|pwk}. In: \emph{Neue
                        Freie Presse}\pwindex{Neue Freie Presse1864 – 1939@\emph{Neue Freie Presse} {[}1864 – 1939{]}|pwk}, Nr. 22.381, 6. 1. 1927, Morgenblatt,
                     S. 13.}}}\label{K_L03022-111h} (du{\geminationm}es Wort) über Ihren Bruder\pwindex{Salzmann, Theodor 1867 – 1926-12-12@\textsc{Salzmann, Theodor} (1867 – 1926-12-12)|pwv} gelesen. Und mit
               Vergnügen gehört, daſs Annerl\pwindex{Rehmann, Anna Katharina 18.08.1904 – 27.03.1977@\textsc{Rehmann, Anna Katharina} (18.08.1904 – 27.03.1977), \emph{Schauspielerin}|pw} (we{\geminationn} man noch so sagen darf) nun auch ein
               schauspielerisches Talent in sich entdeckt hat und als »Mitgefangne« von Helene Thimig\pwindex{Thimig, Helene 05.06.1889 – 07.11.1974@\textsc{Thimig, Helene} (05.06.1889 – 07.11.1974), \emph{Schauspielerin}|pw} in Deutschland\oindex{Deutschland@\textbf{Deutschland}|pw} herumreist. Bescheidene Stichproben von meinem Wissen um Sie.
               Ich hoffe, Sie ergänzen es bald. Wa{\geminationn} ko{\geminationm}en Sie wieder? Ich habe vorläufg keine Reise-Absichten.
               Also »klopfen« oder telefoniren Sie bald. Ich freu mich darauf Sie endlich einmal
               wieder ausführlicher zu sprechen. Von Her\textcolor{gray}{zen} Ihr \pend
           \pstart \spacefill\mbox{Arthur}\pend{}
         
         \endnumbering\mylabel{h}\end{ledgroupsized}\begin{anhang}\end{anhang}\newcommand{\dateiname}{L03022}\newcommand{\titel}{Arthur Schnitzler an Felix Salten, 10. 2. 1927}\newcommand{\editorInnen}{Martin Anton Müller und Laura Untner}%% latex-leseansicht-abspann.tex
%% Abspann für die Leseansicht.
%% Der Schalter \ifkorrekturansicht ist bereits durch den Vorspann gesetzt.

%% latex-abspann.tex
%% Gemeinsamer Abspann für Korrekturansicht und Leseansicht.
%% Setzt den Schalter \ifkorrekturansicht voraus (gesetzt in den
%% einbindenden Dateien latex-korrekturansicht-abspann.tex bzw.
%% latex-leseansicht-abspann.tex).
%% ---------------------------------------------------------------

\normalsize

% Das esempio-Environment wird nur in der Leseansicht benötigt
\ifkorrekturansicht\else
\newenvironment{esempio}[3]%
{
    \vspace{1.5ex}
    \rlap{\underline{#1}}
    \par
    \setlength{\parindent}{0cm}
    \nopagebreak
    \leftskip=#2cm
    \rightskip=#3cm
}
{
    \par
}
\fi

\doendnotes{C}
\bigskip
\vfill

\clearpage

\footnotesize

\ifkorrekturansicht
  \lohead{\textsc{register}}
\fi

% theindex-Environment neu definieren ohne reledmac
\makeatletter
\renewenvironment{theindex}{%
  \ifkorrekturansicht
    \section*{\indexname}%
  \else
    \subsubsection*{Index der erwähnten Entitäten}%
  \fi
  \setlength{\parindent}{0pt}%
  \setlength{\parskip}{0pt plus 0.3pt}%
  \let\item\@idxitem
}{%
  \ifkorrekturansicht\clearpage\fi
}
\makeatother

\IfFileExists{\jobname-pw.ind}{\input{\jobname-pw.ind}}{}

% Quellenangabe nur in der Leseansicht
\ifkorrekturansicht\else
% Fallback-Definitionen, falls die .tex-Datei \titel etc. nicht gesetzt hat
\providecommand{\titel}{}
\providecommand{\editorInnen}{}
\providecommand{\dateiname}{\jobname}

\vspace{3cm}

\vfill

\footnotesize
\textsc{Quelle}: \titel. Herausgegeben von {\editorInnen}. In: \emph{Arthur Schnitzler: Briefwechsel mit Autorinnen und Autoren}.
 Digitale Edition, https://schnitzler-briefe.acdh.oeaw.ac.at/{\dateiname}.html (Stand \today)
\fi

\end{document}


      