%% latex-leseansicht-vorspann.tex
%% Vorspann für die Leseansicht.
%% Lädt die gemeinsame Datei latex-vorspann.tex mit nicht gesetztem Schalter.

\newif\ifkorrekturansicht
\korrekturansichtfalse

\input{../tex-inputs/latex-vorspann}


\section[ Paul Goldmann an Arthur Schnitzler, 31. 7. [1903]]{L03379 Paul Goldmann an Arthur Schnitzler,  31. 7. [1903]}
\nopagebreak\mylabel{L03379v}
\rehead{ }\normalsize\beginnumbering\briefempfaengerindex{Schnitzler, Arthur@\textsc{Schnitzler, Arthur}!zzzGoldmann, Paul@\emph{von Paul Goldmann}!1903-07-311@{31. 7. [1903]}|(be}
\toendnotes[C]{\smallbreak\pagebreak[2]}
\correspDesc{Versand  durch Paul Goldmann am 31. 7. [1903] in Berlin
\newline{}Erhalt  durch Arthur Schnitzler im Zeitraum [1. 8. 1903
                  – 5. 8. 1903?] in Wien}\toendnotes[C]{\smallbreak}
\Standort{DLA, A:Schnitzler, HS.NZ85.1.3173.}
\physDesc{Brief, 1 Blatt, 3 Seiten, 1037 Zeichen
\newline{}Handschrift: blaue Tinte, deutsche Kurrent
\newline{}Schnitzler: 1) mit Bleistift das Jahr »903« vermerkt  2) mit rotem Buntstift zwei Unterstreichungen}\toendnotes[C]{\smallbreak}
\pstart
           \raggedleft{}{\pb}\textcolor{gray}{\textbf{DESSAUERSTRASSE 19\oindex{Dessauer Straße@\textbf{Dessauer Straße}, \emph{Straße}|pw}}}\pend
           
\pstart
           Berlin\oindex{Berlin@\textbf{Berlin}, \emph{Hauptstadt}|pw}, 31. Juli.\pend
           
\pstart\center{}Mein lieber Freund,\pend\vspace{0.5em}
\pstart
           Dank für Deine liebe Karte vom \label{K_L03379-1v}\edtext{Schneeberg\oindex{Schneeberg@\textbf{Schneeberg}, \emph{Berg}|pw}}{\lemma{\textnormal{\emph{Schneeberg}}}\Cendnote{\textnormal{Schnitzler war am 28. 7. 1903 und 29. 7. 1903 auf dem
                     Schneeberg\oindex{Schneeberg@\textbf{Schneeberg}, \emph{Berg}|pwk} gewesen, wohin auch Richard\pwindex{Beer-Hofmann, Richard 11.\,7.\,1866 Wien – 26.\,9.\,1945 New York City@\textsc{Beer-Hofmann, Richard} (11.\,7.\,1866 Wien – 26.\,9.\,1945 New York City), \emph{Schriftsteller}|pwk} und Paula Beer-Hofmann\pwindex{Beer-Hofmann, Paula 25.\,2.\,1879 Wien – 30.\,10.\,1939 Zürich@\textsc{Beer-Hofmann, Paula} (25.\,2.\,1879 Wien – 30.\,10.\,1939 Zürich)|pwk} sowie deren Tochter Mirjam\pwindex{Beer-Hofmann, Mirjam 4.\,9.\,1897 Wien – 24.\,12.\,1984 New York City@\textsc{Beer-Hofmann, Mirjam} (4.\,9.\,1897 Wien – 24.\,12.\,1984 New York City)|pwk} gekommen waren.}}}\label{K_L03379-1} und Deine Briefe.\pend
           
\pstart
           Noch iſts ungewiß, wann ich weggehe. Nächſte Woche wird{ }ſichs entſcheiden, ob »sie\pwindex{Rottenberg, Theodore 7.\,9.\,1875 – 5.\,4.\,1945 Limburg an der Lahn@\textsc{Rottenberg, Theodore} (7.\,9.\,1875 – 5.\,4.\,1945 Limburg an der Lahn)|pwv}« \label{K_L03379-2v}\edtext{mitkommt}{\lemma{\textnormal{\emph{mitkommt}}}\Cendnote{\textnormal{Siehe XXXX Auszeichnungsfehler: Dokument L03375 nicht gefunden.
               }}}\label{K_L03379-2}. Wenn ja,{ }ſo \strikeout{\textcolor{gray}{k}} reiſe ich über Wien\oindex{Wien@\textbf{Wien}, \emph{Verwaltungsgebiet}|pw} nach Tirol\oindex{Tirol@\textbf{Tirol}, \emph{Land}|pw}\oindex{Südtirol@\textbf{Südtirol}, \emph{Verwaltungsgebiet}|pw}; wenn nicht,{ }ſo weiß ich noch gar nicht, was ich
               mache. Da das Alles{ }ſo ungewiß iſt, bitte ich Dich dringend, nicht auf mich zu
               warten, mich aber immer in Kenntniß Deines Aufenthalts {\pb}zu laſſen.\pend
           
\pstart
           \label{K_L03379-3v}\edtext{\textsc{Harden\pwindex{Harden, Maximilian 20.\,10.\,1861 Berlin – 30.\,10.\,1927 Montana@\textsc{Harden, Maximilian} (20.\,10.\,1861 Berlin – 30.\,10.\,1927 Montana), \emph{Schriftsteller, Publizist}|pw}}}{\lemma{\textnormal{\emph{Harden}}}\Cendnote{\textnormal{Dazu kam es nicht.}}}\label{K_L03379-3} hätte nicht übel
               Luſt, mit Dir und mir ein wenig nach Tirol\oindex{Tirol@\textbf{Tirol}, \emph{Land}|pw}\oindex{Südtirol@\textbf{Südtirol}, \emph{Verwaltungsgebiet}|pw} zu kommen, – auch mit Dir allein, wenn ich nicht mitthäte. Ich habe
               ihm geſtern geſagt, daß Du Dich gewiß freuen wirſt,
               ihn zum Begleiter zu haben, und ich bitte Dich, ihm gleich zu{ }ſchreiben\substVorne{}\textsuperscript{,}\substDazwischen{} und\substHinten{} ihn zum Mitkommen zu animiren. Er wäre gewiß ein charmanter und
               unterhaltender Gefährte.\pend
           
\pstart
           Laß’ mich alſo wiſſen, {\pb}welche Reiſe-Entſchlüſſe Du
               gefaßt haſt, ebenſo wie ich Dir sofort Mitteilung machen werde,{ }ſobald ich Genaues
               weiß. (Möglich, daß ich, wenn ich Begleitung\pwindex{Rottenberg, Theodore 7.\,9.\,1875 – 5.\,4.\,1945 Limburg an der Lahn@\textsc{Rottenberg, Theodore} (7.\,9.\,1875 – 5.\,4.\,1945 Limburg an der Lahn)|pwv} habe, doch nach \textsc{Welsberg\oindex{Welsberg-Taisten@\textbf{Welsberg-Taisten}, \emph{Verwaltungsgebiet}|pw}} gehe.)\pend
           
\pstart
           Viele herzliche Grüße an Dich, \textsc{Olga\pwindex{Schnitzler, Olga 17.\,1.\,1882 Wien – 13.\,1.\,1970 Lugano@\textsc{Schnitzler, Olga} (17.\,1.\,1882 Wien – 13.\,1.\,1970 Lugano), \emph{Schauspielerin, Sängerin}|pw}} und \textsc{Heinrich\pwindex{Schnitzler, Heinrich 9.\,8.\,1902 Hinterbrühl – 12.\,7.\,1982 Wien@\textsc{Schnitzler, Heinrich} (9.\,8.\,1902 Hinterbrühl – 12.\,7.\,1982 Wien), \emph{Regisseur, Schauspieler}|pw}}! {\\[\baselineskip]}Dein getreuer {\\[\baselineskip]}\spacefill\mbox{Paul Goldm}\pend
           \leftskip=0em{}\selectlanguage{ngerman}\endnumbering\briefempfaengerindex{Schnitzler, Arthur@\textsc{Schnitzler, Arthur}!zzzGoldmann, Paul@\emph{von Paul Goldmann}!1903-07-311@{31. 7. [1903]}|)be}\mylabel{L03379h}  \newcommand{\dateiname}{L03379}\newcommand{\titel}{Paul Goldmann an Arthur Schnitzler, 31. 7. [1903]}\newcommand{\editorInnen}{Martin Anton Müller und Laura Untner}%% latex-leseansicht-abspann.tex
%% Abspann für die Leseansicht.
%% Der Schalter \ifkorrekturansicht ist bereits durch den Vorspann gesetzt.

%% latex-abspann.tex
%% Gemeinsamer Abspann für Korrekturansicht und Leseansicht.
%% Setzt den Schalter \ifkorrekturansicht voraus (gesetzt in den
%% einbindenden Dateien latex-korrekturansicht-abspann.tex bzw.
%% latex-leseansicht-abspann.tex).
%% ---------------------------------------------------------------

\normalsize

% Das esempio-Environment wird nur in der Leseansicht benötigt
\ifkorrekturansicht\else
\newenvironment{esempio}[3]%
{
    \vspace{1.5ex}
    \rlap{\underline{#1}}
    \par
    \setlength{\parindent}{0cm}
    \nopagebreak
    \leftskip=#2cm
    \rightskip=#3cm
}
{
    \par
}
\fi

\doendnotes{C}
\bigskip
\vfill

\clearpage

\footnotesize

\ifkorrekturansicht
  \lohead{\textsc{register}}
\fi

% theindex-Environment neu definieren ohne reledmac
\makeatletter
\renewenvironment{theindex}{%
  \ifkorrekturansicht
    \section*{\indexname}%
  \else
    \subsubsection*{Index der erwähnten Entitäten}%
  \fi
  \setlength{\parindent}{0pt}%
  \setlength{\parskip}{0pt plus 0.3pt}%
  \let\item\@idxitem
}{%
  \ifkorrekturansicht\clearpage\fi
}
\makeatother

\IfFileExists{\jobname-pw.ind}{\input{\jobname-pw.ind}}{}

% Quellenangabe nur in der Leseansicht
\ifkorrekturansicht\else
% Fallback-Definitionen, falls die .tex-Datei \titel etc. nicht gesetzt hat
\providecommand{\titel}{}
\providecommand{\editorInnen}{}
\providecommand{\dateiname}{\jobname}

\vspace{3cm}

\vfill

\footnotesize
\textsc{Quelle}: \titel. Herausgegeben von {\editorInnen}. In: \emph{Arthur Schnitzler: Briefwechsel mit Autorinnen und Autoren}.
 Digitale Edition, https://schnitzler-briefe.acdh.oeaw.ac.at/{\dateiname}.html (Stand \today)
\fi

\end{document}


