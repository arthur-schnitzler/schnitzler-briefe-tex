%% latex-leseansicht-vorspann.tex
%% Vorspann für die Leseansicht.
%% Lädt die gemeinsame Datei latex-vorspann.tex mit nicht gesetztem Schalter.

\newif\ifkorrekturansicht
\korrekturansichtfalse

\input{../tex-inputs/latex-vorspann}

\begin{center}
            \textcolor{red}{ENTWURF, NICHT FERTIG KORRIGIERT}
                      \end{center}
            
         
         \renewcommand{\erwaehntePersonen}{Personen:  ?? [Partner von Theodore Rottenberg, Ende 1902/Anfang 1903], Otto Brahm, Paul Goldmann, Theodore Rottenberg, Olga Schnitzler, Irene Triesch}
         \renewcommand{\erwaehnteOrte}{Orte: Berlin, Dessauer Straße, Deutsches Theater Berlin, Wien}
         \renewcommand{\erwaehnteWerke}{Werke: Der Schleier der Beatrice. Schauspiel in fünf Akten, Tagebuch}
               \section[ Paul Goldmann an Arthur Schnitzler, 5. 2. {[}1903{]}]{ Paul Goldmann an Arthur Schnitzler, 5. 2. {[}1903{]}}\nopagebreak\mylabel{v}\rehead{ }\begin{ledgroupsized}[t]{13cm}\normalsize\beginnumbering\briefempfaengerindex{Schnitzler, Arthur@\textsc{Schnitzler, Arthur}!zzzGoldmann, Paul@\emph{von Paul Goldmann}!1903-02-051@{5. 2. {[}1903{]}}|(be} \toendnotes[C]{\smallbreak\pagebreak[2]} \Standort{DLA, A:Schnitzler, HS.NZ85.1.3173.}
\physDesc{Brief, 1 Blatt, 4 Seiten, 1636 Zeichen
\newline{}Handschrift: blaue Tinte, deutsche Kurrent
\newline{}Schnitzler: 1) mit Bleistift Antwortskizze (?) normal zum Text: »\textcolor{gray}{Woran ſchuld {\dots} daſs einmal
                                       nachgebend wär} –«  2) mit Bleistift datiert: »903.« 3) mit rotem Buntstift zwei Unterstreichungen}\toendnotes[C]{\smallbreak}\pstart
           \noindent{}\raggedleft{}{\pb}\textcolor{gray}{\textbf{DESSAUERSTRASSE 19\oindex{Dessauer Strasse@\textbf{Dessauer Straße}|pw}}}\pend
           \pstart
           Berlin\oindex{Berlin@\textbf{Berlin}|pw}, 5. Februar.\pend
           \pstart\center{}Mein lieber Freund,\pend\pstart
           Von den \label{K_L03362-1v}\edtext{Aufführungsplänen \textsc{Brahm\pwindex{Brahm, Otto 05.02.1856 – 28.11.1912@\textsc{Brahm, Otto} (05.02.1856 – 28.11.1912), \emph{Theaterleiter, Regisseur}|pw}s }}{\lemma{\textnormal{\emph{Aufführungsplänen Brahms }}}\Cendnote{\textnormal{Bezug auf die bevorstehende Premiere von
                     \emph{Der Schleier der Beatrice}\pwindex{Schnitzler, Arthur 15.05.1862 – 21.10.1931@\textsc{Schnitzler, Arthur} (15.05.1862 – 21.10.1931), \emph{Schriftsteller, Mediziner}!Schleier der Beatrice. Schauspiel in fuenf Akten1900-12-01@\strich\emph{Der Schleier der Beatrice. Schauspiel in fünf Akten} {[}1900-12-01{]}|pwk} am Deutschen Theater Berlin\oindex{Deutsches Theater Berlin@\textbf{Deutsches Theater Berlin}|pwk}. Der Termin war noch
                  nicht fixiert, letztendlich wurde es der 7. 3. 1903. Vgl. \emph{Der
                        Briefwechsel Arthur Schnitzler — Otto Brahm}. Vollständige Ausgabe.
                     Herausgegeben, eingeleitet und erläutert von Oskar Seidlin.
                     Tübingen: \emph{Niemeyer}{ }1975, S. 133–139.}}}\label{K_L03362-1h} weiß ich nichts. Vielleicht kann ich etwas durch die \textsc{Triesch\pwindex{Triesch, Irene 13.04.1877 – 24.11.1964@\textsc{Triesch, Irene} (13.04.1877 – 24.11.1964), \emph{Schauspielerin}|pw}} erfahren, die ich dieſer Tage ſehen werde . Da aber \textsc{Brahm\pwindex{Brahm, Otto 05.02.1856 – 28.11.1912@\textsc{Brahm, Otto} (05.02.1856 – 28.11.1912), \emph{Theaterleiter, Regisseur}|pw}} ein anſtändiger Menſch iſt, \strikeout{h\textcolor{gray}{alte}} nehme ich ſicher an, daß er Dir Wort halten wird.\pend
           \pstart
           Ich hoffe, Du \label{K_L03362-2v}\edtext{kommſt bald}{\lemma{\textnormal{\emph{kommſt bald}}}\Cendnote{\textnormal{siehe Paul Goldmann an Arthur Schnitzler, 27. 1. [1903]}}}\label{K_L03362-2h}. Ich ſehne mich ſchon ſehr danach, mit Dir zu ſprechen. Ich leide {\pb}ganz unbeſchreiblich, weil zu dem Bewußtſein der
               verlorenen Liebe ein marterndes Bewußtſein der Schuld hinzukommt. Ich \uline{mußte} dieſe \label{K_L03362-3v}\edtext{Frau\pwindex{Rottenberg, Theodore 1875-09-07 – 1945-04-05@\textsc{Rottenberg, Theodore} (1875-09-07 – 1945-04-05)|pwv}}{\lemma{\textnormal{\emph{Frau}}}\Cendnote{\textnormal{siehe Paul Goldmann an Arthur Schnitzler, 3. 1. [1903]}}}\label{K_L03362-3h} heirathen, ſchon aus Ehrenpflicht, – trotz aller Bedenken wegen ihrer
               Verläßlichkeit. Und dann paßte ſie zu mir und liebte mich. Und ich ſuchte nach einer
               reichen Parthie! Als ob die Heirath ein Geſchäft wäre! Oh ich verblendeter Thor!
               Jetzt iſt {\pb}Alles aus. Sie\pwindex{Rottenberg, Theodore 1875-09-07 – 1945-04-05@\textsc{Rottenberg, Theodore} (1875-09-07 – 1945-04-05)|pwv} liebt den Andern\pwindex{?? [Partner von Theodore Rottenberg, Ende 1902/Anfang 1903] @\textsc{?? [Partner von Theodore Rottenberg, Ende 1902/Anfang 1903]}|pwv}, geht in ihm auf, findet ſelbſt in
               ſeiner Krankheit, die ihn pflegebedürftig macht, ein \strikeout{\textcolor{gray}{w}} Band, das ſie feſſelt, – von ſeinem Reichthum, der ihr jeden Wunſch erfüllen
               kann, ganz zu ſchweigen! Und er ſpielt \strikeout{die} jetzt die
               leichte und dankbare Rolle des unendlich Guten und Nachſichtigen, – eine Rolle, die
               nach meiner Brutalität von ſelbſt gegeben iſt. Ich habe dieſe Frau\pwindex{Rottenberg, Theodore 1875-09-07 – 1945-04-05@\textsc{Rottenberg, Theodore} (1875-09-07 – 1945-04-05)|pwv}, die mich wahrhaft liebte, wie eine
                  {\pb}Dirne behandelt (freilich nicht ohne Grund, denn
               ſie hatte immer etwas dirnenhaftes in ſich), – \strikeout{d\textcolor{gray}{e}}{ }er\pwindex{?? [Partner von Theodore Rottenberg, Ende 1902/Anfang 1903] @\textsc{?? [Partner von Theodore Rottenberg, Ende 1902/Anfang 1903]}|pwv} behandelt ſie wie eine
               Heilige. Das wirkt; und ſo bin ich längſt erſetzt, und alle meine flehenden,
               ſehnſüchtigen, reumüthigen Briefe bleiben ohne Antwort. Ich ſehe täglich mehr, was
               ich verloren habe. Wie ſoll ich da einen Erſatz finden? In der nüchternen, \label{K_L03362-4v}\edtext{kalten Stadt\oindex{Berlin@\textbf{Berlin}|pwv}}{\lemma{\textnormal{\emph{kalten Stadt}}}\Cendnote{\textnormal{Siehe dazu auch Schnitzler\pwindex{Schnitzler, Arthur 15.05.1862 – 21.10.1931@\textsc{Schnitzler, Arthur} (15.05.1862 – 21.10.1931), \emph{Schriftsteller, Mediziner}|pwk}s Kommentar im \emph{Tagebuch}\pwindex{\textcolor{red}{\textsuperscript{XXXX1 indx}}!Tagebuch1981 – 2000@\strich\emph{Tagebuch} {[}Hrsg., 1981 – 2000{]}|pwk}: »P. Goldmann\pwindex{Goldmann, Paul 31.01.1865 – 25.09.1935@\textsc{Goldmann, Paul} (31.01.1865 – 25.09.1935), \emph{Schriftsteller, Journalist}|pwk} wie
                  gewöhnlich macht Berlin\oindex{Berlin@\textbf{Berlin}|pwk} zum Vertrauten seines
                  Liebesgrams.–« (5. 2. 1903)}}}\label{K_L03362-4h}, in der ich lebe! Und \label{K_L03362-5v}\edtext{dieſer Tage}{\lemma{\textnormal{\emph{dieſer Tage}}}\Cendnote{\textnormal{am
                     31. 1. 1903}}}\label{K_L03362-5h} bin ich 38 Jahre geworden!\pend
           \pstart
           Viele treue Grüße, auch an \textsc{Olga}\pwindex{Schnitzler, Olga 17.01.1882 – 13.01.1970@\textsc{Schnitzler, Olga} (17.01.1882 – 13.01.1970), \emph{Schauspielerin, Sängerin}|pw}! Dein {\\[\baselineskip]}\spacefill\mbox{Paul Goldm}\pend
           \leftskip=0em{}
         
         \endnumbering\mylabel{h}\end{ledgroupsized}\begin{anhang}\end{anhang}\newcommand{\dateiname}{L03362}\newcommand{\titel}{Paul Goldmann an Arthur Schnitzler, 5. 2. [1903]}\newcommand{\editorInnen}{Martin Anton Müller und Laura Untner}%% latex-leseansicht-abspann.tex
%% Abspann für die Leseansicht.
%% Der Schalter \ifkorrekturansicht ist bereits durch den Vorspann gesetzt.

%% latex-abspann.tex
%% Gemeinsamer Abspann für Korrekturansicht und Leseansicht.
%% Setzt den Schalter \ifkorrekturansicht voraus (gesetzt in den
%% einbindenden Dateien latex-korrekturansicht-abspann.tex bzw.
%% latex-leseansicht-abspann.tex).
%% ---------------------------------------------------------------

\normalsize

% Das esempio-Environment wird nur in der Leseansicht benötigt
\ifkorrekturansicht\else
\newenvironment{esempio}[3]%
{
    \vspace{1.5ex}
    \rlap{\underline{#1}}
    \par
    \setlength{\parindent}{0cm}
    \nopagebreak
    \leftskip=#2cm
    \rightskip=#3cm
}
{
    \par
}
\fi

\doendnotes{C}
\bigskip
\vfill

\clearpage

\footnotesize

\ifkorrekturansicht
  \lohead{\textsc{register}}
\fi

% theindex-Environment neu definieren ohne reledmac
\makeatletter
\renewenvironment{theindex}{%
  \ifkorrekturansicht
    \section*{\indexname}%
  \else
    \subsubsection*{Index der erwähnten Entitäten}%
  \fi
  \setlength{\parindent}{0pt}%
  \setlength{\parskip}{0pt plus 0.3pt}%
  \let\item\@idxitem
}{%
  \ifkorrekturansicht\clearpage\fi
}
\makeatother

\IfFileExists{\jobname-pw.ind}{\input{\jobname-pw.ind}}{}

% Quellenangabe nur in der Leseansicht
\ifkorrekturansicht\else
% Fallback-Definitionen, falls die .tex-Datei \titel etc. nicht gesetzt hat
\providecommand{\titel}{}
\providecommand{\editorInnen}{}
\providecommand{\dateiname}{\jobname}

\vspace{3cm}

\vfill

\footnotesize
\textsc{Quelle}: \titel. Herausgegeben von {\editorInnen}. In: \emph{Arthur Schnitzler: Briefwechsel mit Autorinnen und Autoren}.
 Digitale Edition, https://schnitzler-briefe.acdh.oeaw.ac.at/{\dateiname}.html (Stand \today)
\fi

\end{document}


      