%% latex-leseansicht-vorspann.tex
%% Vorspann für die Leseansicht.
%% Lädt die gemeinsame Datei latex-vorspann.tex mit nicht gesetztem Schalter.

\newif\ifkorrekturansicht
\korrekturansichtfalse

\input{../tex-inputs/latex-vorspann}


\section[ Paul Goldmann an Arthur Schnitzler, 5. 2. {[}1903{]}]{L03362 Paul Goldmann an Arthur Schnitzler,  5. 2. [1903]}
\nopagebreak\mylabel{L03362v}
\rehead{ }\normalsize\beginnumbering\briefempfaengerindex{Schnitzler, Arthur@\textsc{Schnitzler, Arthur}!zzzGoldmann, Paul@\emph{von Paul Goldmann}!1903-02-051@{5. 2. [1903]}|(be}
\toendnotes[C]{\smallbreak\pagebreak[2]}
\correspDesc{Versand  durch Paul Goldmann am 5. 2. [1903] in Berlin
\newline{}Erhalt  durch Arthur Schnitzler im Zeitraum [6. 2. 1903
                  – 10. 2. 1903?] in Wien}\toendnotes[C]{\smallbreak}
\Standort{DLA, A:Schnitzler, HS.NZ85.1.3173.}
\physDesc{Brief, 1 Blatt, 4 Seiten, 1635 Zeichen
\newline{}Handschrift: blaue Tinte, deutsche Kurrent
\newline{}Schnitzler: 1) mit Bleistift die Antwortskizze (?) »\textcolor{gray}{Nur ein{ }ſchuld {\dots} daſs er mal
                                       nachgebend wär} –« normal zum Text vermerkt
                                 und das Jahr ergänzt: »903.«  2) mit rotem Buntstift zwei Unterstreichungen}\toendnotes[C]{\smallbreak}
\pstart
           \raggedleft{}{\pb}\textcolor{gray}{\textbf{DESSAUERSTRASSE 19\oindex{Dessauer Straße@\textbf{Dessauer Straße}, \emph{Straße}|pw}}}\pend
           
\pstart
           Berlin\oindex{Berlin@\textbf{Berlin}, \emph{Hauptstadt}|pw}, 5. Februar.\pend
           
\pstart{}Mein lieber Freund,\pend\vspace{0.5em}
\pstart
           Von den \label{K_L03362-1v}\edtext{Aufführungsplänen 
               \textsc{Brahms}\pwindex{Brahm, Otto 5.\,2.\,1856 Hamburg – 28.\,11.\,1912 Berlin@\textsc{Brahm, Otto} (5.\,2.\,1856 Hamburg – 28.\,11.\,1912 Berlin), \emph{Theaterleiter, Regisseur}|pw}}{\lemma{\textnormal{\emph{Aufführungsplänen 
               Brahms}}}\Cendnote{\textnormal{Die Premiere von
                  \emph{Der Schleier der Beatrice}\pwindex{Schnitzler, Arthur 15.\,5.\,1862 Wien – 21.\,10.\,1931 ebd.@\textsc{Schnitzler, Arthur} (15.\,5.\,1862 Wien – 21.\,10.\,1931 ebd.), \emph{Schriftsteller, Mediziner}!Schleier der Beatrice. Schauspiel in fünf Akten@\strich\emph{Der Schleier der Beatrice. Schauspiel in fünf Akten}|pwk} am \emph{Deutschen Theater Berlin}\orgindex{Deutsches Theater Berlin@Deutsches Theater Berlin|pwk} befand sich
                  in Vorbereitung. Der Termin war noch
                  nicht fixiert, letztendlich wurde es der 7. 3. 1903. Vgl. \emph{Der
                     Briefwechsel Arthur Schnitzler – Otto Brahm}. Vollständige Ausgabe.
                     Herausgegeben, eingeleitet und erläutert von Oskar Seidlin.
                     Tübingen: \emph{Niemeyer}{ }1975, S. 133–139.
               }}}\label{K_L03362-1} weiß ich nichts. Vielleicht kann ich etwas durch die \textsc{Triesch\pwindex{Triesch, Irene 13.\,4.\,1877 Wien – 24.\,11.\,1964 Basel@\textsc{Triesch, Irene} (13.\,4.\,1877 Wien – 24.\,11.\,1964 Basel), \emph{Schauspielerin}|pw}} erfahren, die ich dieſer Tage{ }ſehen werde. Da aber \textsc{Brahm\pwindex{Brahm, Otto 5.\,2.\,1856 Hamburg – 28.\,11.\,1912 Berlin@\textsc{Brahm, Otto} (5.\,2.\,1856 Hamburg – 28.\,11.\,1912 Berlin), \emph{Theaterleiter, Regisseur}|pw}} ein anſtändiger Menſch iſt, \strikeout{h\textcolor{gray}{alte}} nehme ich{ }ſicher an, daß er Dir Wort halten wird.\pend
           
\pstart
           Ich hoffe, Du \label{K_L03362-2v}\edtext{kommſt bald}{\lemma{\textnormal{\emph{kommst bald}}}\Cendnote{\textnormal{Siehe XXXX Auszeichnungsfehler: Dokument L03361 nicht gefunden.
               }}}\label{K_L03362-2}. Ich{ }ſehne mich{ }ſchon{ }ſehr danach, mit Dir zu{ }ſprechen. Ich leide {\pb}ganz unbeſchreiblich, weil zu dem Bewußtſein der
               verlorenen Liebe ein marterndes Bewußtſein der Schuld hinzu kommt. Ich \uline{mußte} dieſe \label{K_L03362-3v}\edtext{Frau\pwindex{Rottenberg, Theodore 7.\,9.\,1875 – 5.\,4.\,1945 Limburg an der Lahn@\textsc{Rottenberg, Theodore} (7.\,9.\,1875 – 5.\,4.\,1945 Limburg an der Lahn)|pwv}}{\lemma{\textnormal{\emph{Frau}}}\Cendnote{\textnormal{Siehe XXXX Auszeichnungsfehler: Dokument L03360 nicht gefunden.
               }}}\label{K_L03362-3} heirathen,{ }ſchon aus Ehrenpflicht, – trotz aller Bedenken wegen ihrer
               Verläßlichkeit. Und dann paßte{ }ſie zu mir und liebte mich. Und ich{ }ſuchte nach einer
               reichen Parthie! Als ob die Heirath ein Geſchäft wäre! Oh ich verblendeter Thor!
               Jetzt iſt {\pb}Alles aus. Sie\pwindex{Rottenberg, Theodore 7.\,9.\,1875 – 5.\,4.\,1945 Limburg an der Lahn@\textsc{Rottenberg, Theodore} (7.\,9.\,1875 – 5.\,4.\,1945 Limburg an der Lahn)|pwv} liebt den Andern\pwindex{?? [Partner von Theodore Rottenberg, Ende 1902/Anfang 1903] @\textsc{?? [Partner von Theodore Rottenberg, Ende 1902/Anfang 1903]}|pwv}, geht in ihm auf, findet{ }ſelbſt in{ }ſeiner Krankheit, die ihn pflegebedürftig macht, ein \strikeout{\textcolor{gray}{w}} Band, das{ }ſie feſſelt, – von{ }ſeinem Reichthum, der ihr jeden Wunſch erfüllen
               kann, ganz zu{ }ſchweigen! Und er{ }ſpielt \strikeout{die} jetzt die
               leichte und dankbare Rolle des unendlich Guten und Nachſichtigen, – eine Rolle, die
               nach meiner Brutalität von{ }ſelbſt gegeben iſt. Ich habe dieſe Frau\pwindex{Rottenberg, Theodore 7.\,9.\,1875 – 5.\,4.\,1945 Limburg an der Lahn@\textsc{Rottenberg, Theodore} (7.\,9.\,1875 – 5.\,4.\,1945 Limburg an der Lahn)|pwv}, die mich wahrhaft liebte, wie eine
                  {\pb}Dirne behandelt (freilich nicht ohne Grund, denn{ }ſie hatte immer etwas Dirnenhaftes in{ }ſich), – \strikeout{d\textcolor{gray}{e}}{ }er\pwindex{?? [Partner von Theodore Rottenberg, Ende 1902/Anfang 1903] @\textsc{?? [Partner von Theodore Rottenberg, Ende 1902/Anfang 1903]}|pwv} behandelt{ }ſie wie eine
               Heilige. Das wirkt; und{ }ſo bin ich längſt erſetzt, und alle meine flehenden,{ }ſehnſüchtigen, reumüthigen Briefe bleiben ohne Antwort. Ich{ }ſehe täglich mehr, was
               ich verloren habe. Wie{ }ſoll ich da einen Erſatz finden? In der nüchternen, \label{K_L03362-4v}\edtext{kalten Stadt\oindex{Berlin@\textbf{Berlin}, \emph{Hauptstadt}|pwv}}{\lemma{\textnormal{\emph{kalten Stadt}}}\Cendnote{\textnormal{Siehe dazu auch Schnitzlers Kommentar im \emph{Tagebuch}\pwindex{Schnitzler, Arthur 15.\,5.\,1862 Wien – 21.\,10.\,1931 ebd.@\textsc{Schnitzler, Arthur} (15.\,5.\,1862 Wien – 21.\,10.\,1931 ebd.), \emph{Schriftsteller, Mediziner}!Tagebuch@\strich\emph{Tagebuch}|pwk}: »P. Goldmann\pwindex{Goldmann, Paul 31.\,1.\,1865 Breslau – 25.\,9.\,1935 Wien@\textsc{Goldmann, Paul} (31.\,1.\,1865 Breslau – 25.\,9.\,1935 Wien), \emph{Schriftsteller, Journalist}|pwk} wie
                  gewöhnlich macht Berlin\oindex{Berlin@\textbf{Berlin}, \emph{Hauptstadt}|pwk} zum Vertrauten seines
                  Liebesgrams. –« (5. 2. 1903.)}}}\label{K_L03362-4}, in der ich lebe! Und \label{K_L03362-5v}\edtext{dieſer Tage}{\lemma{\textnormal{\emph{dieser Tage}}}\Cendnote{\textnormal{am
                     31. 1. 1903}}}\label{K_L03362-5} bin ich 38 Jahr\textcolor{gray}{e} geworden!\pend
           
\pstart
           Viele treue Grüße, auch an \textsc{Olga}\pwindex{Schnitzler, Olga 17.\,1.\,1882 Wien – 13.\,1.\,1970 Lugano@\textsc{Schnitzler, Olga} (17.\,1.\,1882 Wien – 13.\,1.\,1970 Lugano), \emph{Schauspielerin, Sängerin}|pw}! Dein {\\[\baselineskip]}\spacefill\mbox{Paul Goldm}\pend
           \leftskip=0em{}\selectlanguage{ngerman}\endnumbering\briefempfaengerindex{Schnitzler, Arthur@\textsc{Schnitzler, Arthur}!zzzGoldmann, Paul@\emph{von Paul Goldmann}!1903-02-051@{5. 2. [1903]}|)be}\mylabel{L03362h}  \newcommand{\dateiname}{L03362}\newcommand{\titel}{Paul Goldmann an Arthur Schnitzler, 5. 2. [1903]}\newcommand{\editorInnen}{Martin Anton Müller und Laura Untner}%% latex-leseansicht-abspann.tex
%% Abspann für die Leseansicht.
%% Der Schalter \ifkorrekturansicht ist bereits durch den Vorspann gesetzt.

%% latex-abspann.tex
%% Gemeinsamer Abspann für Korrekturansicht und Leseansicht.
%% Setzt den Schalter \ifkorrekturansicht voraus (gesetzt in den
%% einbindenden Dateien latex-korrekturansicht-abspann.tex bzw.
%% latex-leseansicht-abspann.tex).
%% ---------------------------------------------------------------

\normalsize

% Das esempio-Environment wird nur in der Leseansicht benötigt
\ifkorrekturansicht\else
\newenvironment{esempio}[3]%
{
    \vspace{1.5ex}
    \rlap{\underline{#1}}
    \par
    \setlength{\parindent}{0cm}
    \nopagebreak
    \leftskip=#2cm
    \rightskip=#3cm
}
{
    \par
}
\fi

\doendnotes{C}
\bigskip
\vfill

\clearpage

\footnotesize

\ifkorrekturansicht
  \lohead{\textsc{register}}
\fi

% theindex-Environment neu definieren ohne reledmac
\makeatletter
\renewenvironment{theindex}{%
  \ifkorrekturansicht
    \section*{\indexname}%
  \else
    \subsubsection*{Index der erwähnten Entitäten}%
  \fi
  \setlength{\parindent}{0pt}%
  \setlength{\parskip}{0pt plus 0.3pt}%
  \let\item\@idxitem
}{%
  \ifkorrekturansicht\clearpage\fi
}
\makeatother

\IfFileExists{\jobname-pw.ind}{\input{\jobname-pw.ind}}{}

% Quellenangabe nur in der Leseansicht
\ifkorrekturansicht\else
% Fallback-Definitionen, falls die .tex-Datei \titel etc. nicht gesetzt hat
\providecommand{\titel}{}
\providecommand{\editorInnen}{}
\providecommand{\dateiname}{\jobname}

\vspace{3cm}

\vfill

\footnotesize
\textsc{Quelle}: \titel. Herausgegeben von {\editorInnen}. In: \emph{Arthur Schnitzler: Briefwechsel mit Autorinnen und Autoren}.
 Digitale Edition, https://schnitzler-briefe.acdh.oeaw.ac.at/{\dateiname}.html (Stand \today)
\fi

\end{document}


