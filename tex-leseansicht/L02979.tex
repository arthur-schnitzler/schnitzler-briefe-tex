%% latex-leseansicht-vorspann.tex
%% Vorspann für die Leseansicht.
%% Lädt die gemeinsame Datei latex-vorspann.tex mit nicht gesetztem Schalter.

\newif\ifkorrekturansicht
\korrekturansichtfalse

\input{../tex-inputs/latex-vorspann}


         
         \renewcommand{\erwaehntePersonen}{Personen: Hermann Bahr, Oskar Bie, Otto Brahm, Samuel Fischer, Gerhart Hauptmann, Eduard von Keyserling, Felix Salten, Émile Zola}
         \renewcommand{\erwaehnteInstitutionen}{Institutionen: Burgtheater, Deutsches Theater Berlin, S. Fischer Verlag, Schiller-Theater}
         \renewcommand{\erwaehnteOrte}{Orte: Berlin, Berliner Theater, Deutsches Theater Berlin, Hotel Bristol Berlin, Jagniątków, Wien}
         \renewcommand{\erwaehnteWerke}{Werke: Beate und Mareile. Eine Schloßgeschichte, Der Schleier der Beatrice. Schauspiel in fünf Akten, Die kleine Veronika, Liebelei. Schauspiel in drei Akten, Monna Vanna. Schauspiel in drei Akten, Neue Deutsche Rundschau, Wienerinnen. Lustspiel in drei Akten}
               \section[ Arthur Schnitzler an Felix Salten, 16. 10. 1902]{ Arthur Schnitzler an Felix Salten, 16. 10. 1902}\nopagebreak\mylabel{v}\rehead{ }\begin{ledgroupsized}[t]{13cm}\normalsize\beginnumbering\briefempfaengerindex{Salten, Felix@\textsc{Salten, Felix}!zzzSchnitzler, Arthur@\emph{von Arthur Schnitzler}!1902-10-161@{16. 10. 1902}|(be} \toendnotes[C]{\smallbreak\pagebreak[2]} \Standort{Wienbibliothek im Rathaus, ZPH 1681, 2.1.516.}
\physDesc{Brief, 1 Blatt, 4 Seiten, 1314 Zeichen
\newline{}Handschrift: Bleistift, deutsche Kurrent
\newline{}Ordnung: mit Bleistift von unbekannter Hand Nummerierung der Doppelseiten des
                                 Konvoluts: »67«–»68« }\toendnotes[C]{\smallbreak}\pstart
           \raggedleft{}{\pb}\textsc{Berlin Bristol\oindex{Hotel Bristol Berlin@\textbf{Hotel Bristol Berlin}|pw}}, \uline{16. X. 902.}\pend
           \pstart
           lieber Freund,{ }\label{K_L02979-1v}\edtext{geſtern}{\lemma{\textnormal{\emph{geſtern}}}\Cendnote{\textnormal{Siehe A. S.: \emph{Tagebuch}, 15. 10. 1902.
               }}}\label{K_L02979-1h} ſprach ich \textsc{S. Fischer\pwindex{Fischer, Samuel 24.12.1859 – 15.10.1934@\textsc{Fischer, Samuel} (24.12.1859 – 15.10.1934), \emph{Verleger}|pw}}; nach einigen
               Einwendungen
               geſtand er der \label{K_L02979-2v}\edtext{Novelle\pwindex{Salten, Felix 06.09.1869 – 08.10.1945@\textsc{Salten, Felix} (06.09.1869 – 08.10.1945), \emph{Schriftsteller, Journalist}!kleine Veronika1902-12-01@\strich\emph{Die kleine Veronika} {[}1902-12-01{]}|pwv}}{\lemma{\textnormal{\emph{Novelle}}}\Cendnote{\textnormal{Wie später im Korrespondenzstück als Möglichkeit
                  thematisiert, erschien die Novelle\pwindex{Salten, Felix 06.09.1869 – 08.10.1945@\textsc{Salten, Felix} (06.09.1869 – 08.10.1945), \emph{Schriftsteller, Journalist}!kleine Veronika1902-12-01@\strich\emph{Die kleine Veronika} {[}1902-12-01{]}|pwkv} noch im 
                     Dezember in der \emph{Neuen
                        Deutschen Rundschau}\pwindex{Neue Deutsche Rundschau1894-01-01 – 1903-12-31@\emph{Neue Deutsche Rundschau} {[}1894-01-01 – 1903-12-31{]}|pwk}: Felix Salten\pwindex{Salten, Felix 06.09.1869 – 08.10.1945@\textsc{Salten, Felix} (06.09.1869 – 08.10.1945), \emph{Schriftsteller, Journalist}|pwk}: \emph{Die kleine Veronika}\pwindex{Salten, Felix 06.09.1869 – 08.10.1945@\textsc{Salten, Felix} (06.09.1869 – 08.10.1945), \emph{Schriftsteller, Journalist}!kleine Veronika1902-12-01@\strich\emph{Die kleine Veronika} {[}1902-12-01{]}|pwk}. In: \emph{Neue Deutsche Rundschau}\pwindex{Neue Deutsche Rundschau1894-01-01 – 1903-12-31@\emph{Neue Deutsche Rundschau} {[}1894-01-01 – 1903-12-31{]}|pwk}, Jg. 13, Nr. 12, Dezember 1902, S. 1285–1333.
                  Die Buchausgabe folgte 1903: \emph{Die kleine Veronika}\pwindex{Salten, Felix 06.09.1869 – 08.10.1945@\textsc{Salten, Felix} (06.09.1869 – 08.10.1945), \emph{Schriftsteller, Journalist}!kleine Veronika1902-12-01@\strich\emph{Die kleine Veronika} {[}1902-12-01{]}|pwk}. Berlin\oindex{Berlin@\textbf{Berlin}|pwk}: \emph{S. Fischer}\orgindex{S. Fischer Verlag@S. Fischer Verlag|pwk}{ }{[}Mitte Mai{]} 1903.}}}\label{K_L02979-2h}, beſonders im letzten Drittel, Zola\pwindex{Zola, Emile 02.04.1840 – 29.09.1902@\textsc{Zola, Émile} (02.04.1840 – 29.09.1902), \emph{Schriftsteller, Journalist}|pw}’ſche Kraft zu, und iſt \uline{jedenfalls ſofort
                  bereit} ſie als Buch zu drucken. Gegen die Veröffentlichung in der \textsc{N. Dtsch Rds\pwindex{Neue Deutsche Rundschau1894-01-01 – 1903-12-31@\emph{Neue Deutsche Rundschau} {[}1894-01-01 – 1903-12-31{]}|pw}} ſprechen \uline{vorläufig} noch einige Bedenken
               ausſchließlich techniſcher Natur. Sie nähme 60 Seiten ein, was für \uline{eine} Nu{\geminationm}er {\pb}zu viel ſei; und neben dem im Jänner beginnenden \label{K_L02979-3v}\edtext{Roman\pwindex{Keyserling, Eduard von 15.05.1855 – 28.09.1918@\textsc{Keyserling, Eduard von} (15.05.1855 – 28.09.1918), \emph{Schriftsteller}!Beate und Mareile. Eine Schlossgeschichte1903-01-01 – 1903-03-01@\strich\emph{Beate und Mareile. Eine Schloßgeschichte} {[}1903-01-01 – 1903-03-01{]}|pwv}}{\lemma{\textnormal{\emph{Roman}}}\Cendnote{\textnormal{\emph{Beate und Mareile. Eine Schloßgeschichte}\pwindex{Keyserling, Eduard von 15.05.1855 – 28.09.1918@\textsc{Keyserling, Eduard von} (15.05.1855 – 28.09.1918), \emph{Schriftsteller}!Beate und Mareile. Eine Schlossgeschichte1903-01-01 – 1903-03-01@\strich\emph{Beate und Mareile. Eine Schloßgeschichte} {[}1903-01-01 – 1903-03-01{]}|pwk} von
                     Eduard von Keyserling\pwindex{Keyserling, Eduard von 15.05.1855 – 28.09.1918@\textsc{Keyserling, Eduard von} (15.05.1855 – 28.09.1918), \emph{Schriftsteller}|pwk} erschien in drei
                  Teilen zwischen Januar und März 1903 in der \emph{Neuen Deutschen
                     Rundschau}\pwindex{Neue Deutsche Rundschau1894-01-01 – 1903-12-31@\emph{Neue Deutsche Rundschau} {[}1894-01-01 – 1903-12-31{]}|pwk}.}}}\label{K_L02979-3h}{ }\label{T_L02979-1v}\edtext{könnten}{\lemma{\textnormal{\emph{könnten}}}\Cendnote{\textnormal{Er schreibt »konnten«.}}}\label{T_L02979-1h} ſie nicht ein
               Ding in 2 Fortſetzungen bringen. Inmit\textcolor{gray}{ten} der Discuſſion kam \textsc{Bie\pwindex{Bie, Oskar 09.02.1864 – 21.04.1938@\textsc{Bie, Oskar} (09.02.1864 – 21.04.1938), \emph{Schriftsteller, Journalist, Redakteur}|pw}}, der die Novelle\pwindex{Salten, Felix 06.09.1869 – 08.10.1945@\textsc{Salten, Felix} (06.09.1869 – 08.10.1945), \emph{Schriftsteller, Journalist}!kleine Veronika1902-12-01@\strich\emph{Die kleine Veronika} {[}1902-12-01{]}|pwv} zur
               Lecture nach Hauſe nahm. Ich habe den Eindruck, wenn ſie ihm gefällt, wird man ſie im
                  Dezemberheft\pwindex{Neue Deutsche Rundschau1894-01-01 – 1903-12-31@\emph{Neue Deutsche Rundschau} {[}1894-01-01 – 1903-12-31{]}|pwv}, trotz der 60 Seiten abdrucken. In
               Hinblick auf die Buchausgabe iſt natürlich {\pb}zuzugreifen. –\pend
           \pstart
           In Hinſicht auf die \label{K_L02979-4v}\edtext{\textsc{Bea\pwindex{Schnitzler, Arthur 15.05.1862 – 21.10.1931@\textsc{Schnitzler, Arthur} (15.05.1862 – 21.10.1931), \emph{Schriftsteller, Mediziner}!Schleier der Beatrice. Schauspiel in fuenf Akten1900-12-01@\strich\emph{Der Schleier der Beatrice. Schauspiel in fünf Akten} {[}1900-12-01{]}|pw}}}{\lemma{\textnormal{\emph{Bea}}}\Cendnote{\textnormal{Schnitzler\pwindex{Schnitzler, Arthur 15.05.1862 – 21.10.1931@\textsc{Schnitzler, Arthur} (15.05.1862 – 21.10.1931), \emph{Schriftsteller, Mediziner}|pwk} hoffte weiterhin,
                  dass \emph{Der Schleier der Beatrice}\pwindex{Schnitzler, Arthur 15.05.1862 – 21.10.1931@\textsc{Schnitzler, Arthur} (15.05.1862 – 21.10.1931), \emph{Schriftsteller, Mediziner}!Schleier der Beatrice. Schauspiel in fuenf Akten1900-12-01@\strich\emph{Der Schleier der Beatrice. Schauspiel in fünf Akten} {[}1900-12-01{]}|pwk} durch eine qualitätvolle Aufführung 
                  in Berlin\oindex{Berlin@\textbf{Berlin}|pwk} Erfolg haben würde. Die Berlin\oindex{Berlin@\textbf{Berlin}|pwk}-Premiere wurde letztlich am 7. 3. 1903 vom \emph{Deutschen Theater}\orgindex{Deutsches Theater Berlin@Deutsches Theater Berlin|pwk} veranstaltet.}}}\label{K_L02979-4h}{ }\substVorne{}\textsuperscript{iſt}\substDazwischen{}bin\substHinten{} ich ſoweit als früher. Vom Schillertheater\orgindex{Schiller-Theater@Schiller-Theater|pw} räth mir \uline{alles} ab; die
                  \label{K_L02979-5v}\edtext{Aufführg der \textsc{\label{K_L02979-6v}\edtext{M. Vanna\pwindex{\textcolor{red}{\textsuperscript{XXXX1 indx}}!Monna Vanna. Schauspiel in drei Akten1903@\strich\emph{Monna Vanna. Schauspiel in drei Akten} {[}1903{]}|pw}}{\lemma{\textnormal{\emph{M. Vanna}}}\Cendnote{\textnormal{Vgl. Paul Goldmann an Arthur Schnitzler, 16. 6. [1902].
                  }}}\label{K_L02979-6h}} im Dtſch Theater\oindex{Deutsches Theater Berlin@\textbf{Deutsches Theater Berlin}|pw}}{\lemma{\textnormal{\emph{Aufführg … Theater}}}\Cendnote{\textnormal{Siehe A. S.: \emph{Tagebuch}, 14. 10. 1902.
               }}}\label{K_L02979-5h} iſt kläglich. Brahm\pwindex{Brahm, Otto 05.02.1856 – 28.11.1912@\textsc{Brahm, Otto} (05.02.1856 – 28.11.1912), \emph{Theaterleiter, Regisseur}|pw} will ſehr; da er
                  vorgeſtern abgereiſt iſt, reiſe ich von hier
               wahrſcheinlich \introOben{}(Samſtag)\introOben{} zu
               ihm \label{K_L02979-7v}\edtext{nach Agnetendorf\oindex{Jagniątków@\textbf{Jagniątków}|pw}}{\lemma{\textnormal{\emph{nach Agnetendorf}}}\Cendnote{\textnormal{Siehe A. S.: \emph{Tagebuch}, 19. 10. 1902.
               }}}\label{K_L02979-7h}, wohin ich auch von Hauptm\pwindex{Hauptmann, Gerhart 15.11.1862 – 06.06.1946@\textsc{Hauptmann, Gerhart} (15.11.1862 – 06.06.1946), \emph{Schriftsteller}|pw} eine
                  \label{K_L02979-8v}\edtext{telegr. Einladg}{\lemma{\textnormal{\emph{telegr. Einladg}}}\Cendnote{\textnormal{nicht überliefert}}}\label{K_L02979-8h} erhalten {\pb}habe, – u bringe dort die Sache ins
               Reine.\pend
           \pstart
           Bahr\pwindex{Bahr, Hermann 19.07.1863 – 15.01.1934@\textsc{Bahr, Hermann} (19.07.1863 – 15.01.1934), \emph{Schriftsteller, Kritiker}|pw} hatte hier einen wirklichen \label{K_L02979-9v}\edtext{Erfolg\pwindex{Bahr, Hermann 19.07.1863 – 15.01.1934@\textsc{Bahr, Hermann} (19.07.1863 – 15.01.1934), \emph{Schriftsteller, Kritiker}!Wienerinnen. Lustspiel in drei Akten1900@\strich\emph{Wienerinnen. Lustspiel in drei Akten} {[}1900{]}|pwv}}{\lemma{\textnormal{\emph{Erfolg}}}\Cendnote{\textnormal{Am 14. 10. 1902 war Bahrs\pwindex{Bahr, Hermann 19.07.1863 – 15.01.1934@\textsc{Bahr, Hermann} (19.07.1863 – 15.01.1934), \emph{Schriftsteller, Kritiker}|pwk}{ }\emph{Wienerinnen}\pwindex{Bahr, Hermann 19.07.1863 – 15.01.1934@\textsc{Bahr, Hermann} (19.07.1863 – 15.01.1934), \emph{Schriftsteller, Kritiker}!Wienerinnen. Lustspiel in drei Akten1900@\strich\emph{Wienerinnen. Lustspiel in drei Akten} {[}1900{]}|pwk} – in Anwesenheit des Autors\pwindex{Bahr, Hermann 19.07.1863 – 15.01.1934@\textsc{Bahr, Hermann} (19.07.1863 – 15.01.1934), \emph{Schriftsteller, Kritiker}|pwkv} – am Berliner Theater\oindex{Berliner Theater@\textbf{Berliner Theater}|pwk} aufgeführt worden.}}}\label{K_L02979-9h}. –\pend
           \pstart
           In Hinſicht auf die \label{K_L02979-10v}\edtext{Kündigungspflicht
               beim Burgtheater\orgindex{Burgtheater@Burgtheater|pw}}{\lemma{\textnormal{\emph{Kündigungspflicht beim Burgtheater}}}\Cendnote{\textnormal{Siehe Felix Salten an Arthur Schnitzler, 15. 10. 1902.
               }}}\label{K_L02979-10h} ſti{\geminationm}t’s. Ich muſs am 9.
                  Nov. dem Theater\orgindex{Burgtheater@Burgtheater|pwv} das
               ausſchließliche Aufführgsrecht der Liebelei\pwindex{Schnitzler, Arthur 15.05.1862 – 21.10.1931@\textsc{Schnitzler, Arthur} (15.05.1862 – 21.10.1931), \emph{Schriftsteller, Mediziner}!Liebelei. Schauspiel in drei Akten1895-10-09@\strich\emph{Liebelei. Schauspiel in drei Akten} {[}1895-10-09{]}|pw} kündigen mit 2 monatlicher Friſt. Näheres
               mündlich. –\pend
           \pstart Herzlichſt Ihr \spacefill\mbox{A. S.}\pend{}
         
         \endnumbering\mylabel{h}\end{ledgroupsized}  \newcommand{\dateiname}{L02979}\newcommand{\titel}{Arthur Schnitzler an Felix Salten, 16. 10. 1902}\newcommand{\editorInnen}{Martin Anton Müller und Laura Untner}%% latex-leseansicht-abspann.tex
%% Abspann für die Leseansicht.
%% Der Schalter \ifkorrekturansicht ist bereits durch den Vorspann gesetzt.

%% latex-abspann.tex
%% Gemeinsamer Abspann für Korrekturansicht und Leseansicht.
%% Setzt den Schalter \ifkorrekturansicht voraus (gesetzt in den
%% einbindenden Dateien latex-korrekturansicht-abspann.tex bzw.
%% latex-leseansicht-abspann.tex).
%% ---------------------------------------------------------------

\normalsize

% Das esempio-Environment wird nur in der Leseansicht benötigt
\ifkorrekturansicht\else
\newenvironment{esempio}[3]%
{
    \vspace{1.5ex}
    \rlap{\underline{#1}}
    \par
    \setlength{\parindent}{0cm}
    \nopagebreak
    \leftskip=#2cm
    \rightskip=#3cm
}
{
    \par
}
\fi

\doendnotes{C}
\bigskip
\vfill

\clearpage

\footnotesize

\ifkorrekturansicht
  \lohead{\textsc{register}}
\fi

% theindex-Environment neu definieren ohne reledmac
\makeatletter
\renewenvironment{theindex}{%
  \ifkorrekturansicht
    \section*{\indexname}%
  \else
    \subsubsection*{Index der erwähnten Entitäten}%
  \fi
  \setlength{\parindent}{0pt}%
  \setlength{\parskip}{0pt plus 0.3pt}%
  \let\item\@idxitem
}{%
  \ifkorrekturansicht\clearpage\fi
}
\makeatother

\IfFileExists{\jobname-pw.ind}{\input{\jobname-pw.ind}}{}

% Quellenangabe nur in der Leseansicht
\ifkorrekturansicht\else
% Fallback-Definitionen, falls die .tex-Datei \titel etc. nicht gesetzt hat
\providecommand{\titel}{}
\providecommand{\editorInnen}{}
\providecommand{\dateiname}{\jobname}

\vspace{3cm}

\vfill

\footnotesize
\textsc{Quelle}: \titel. Herausgegeben von {\editorInnen}. In: \emph{Arthur Schnitzler: Briefwechsel mit Autorinnen und Autoren}.
 Digitale Edition, https://schnitzler-briefe.acdh.oeaw.ac.at/{\dateiname}.html (Stand \today)
\fi

\end{document}


      