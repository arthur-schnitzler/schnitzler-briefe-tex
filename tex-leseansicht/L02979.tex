%% latex-korrekturansicht-vorspann.tex
%% Vorspann für die Korrekturansicht.
%% Lädt die gemeinsame Datei latex-vorspann.tex mit gesetztem Schalter.

\newif\ifkorrekturansicht
\korrekturansichttrue

\input{../tex-inputs/latex-vorspann}


\section[ Arthur Schnitzler an Felix Salten, 16. 10. 1902]{L02979 Arthur Schnitzler an Felix Salten, 16. 10. 1902}
\nopagebreak\mylabel{L02979v}
\rehead{ }\normalsize\beginnumbering\briefempfaengerindex{Salten, Felix@\textsc{Salten, Felix}!zzzSchnitzler, Arthur@\emph{von Arthur Schnitzler}!1902-10-161@{16. 10. 1902}|(be}
\toendnotes[C]{\smallbreak\pagebreak[2]}\Standort{Wienbibliothek im Rathaus, ZPH 1681, 2.1.516.}
\physDesc{Brief, 1 Blatt, 4 Seiten, 1314 Zeichen
\newline{}Handschrift: Bleistift, deutsche Kurrent
\newline{}Ordnung: mit Bleistift von unbekannter Hand Nummerierung der Doppelseiten des
                                 Konvoluts: »67«–»68« }\toendnotes[C]{\smallbreak}
\pstart
           \raggedleft{}{\pb}\textsc{Berlin Bristol\oindex{Hotel Bristol Berlin@\textbf{Hotel Bristol Berlin}, \emph{Hotel (K.HTL)}|pw}}, \uline{16. X. 902.}\pend
           \vspace{0.5em}
\pstart
           lieber Freund,{ }\label{K_L02979-1v}\edtext{geſtern}{\lemma{\textnormal{\emph{geſtern}}}\Cendnote{\textnormal{Siehe A. S.: \emph{Tagebuch}, 15. 10. 1902.
               }}}\label{K_L02979-1} ſprach ich \textsc{S. Fischer\pwindex{Fischer, Samuel 24.12.1859 – 15.10.1934@\textsc{Fischer, Samuel} (24.12.1859 – 15.10.1934), \emph{Verleger/Verlegerin}|pw}}; nach einigen
               Einwendungen
               geſtand er der \label{K_L02979-2v}\edtext{Novelle\pwindex{kleine Veronika@\emph{Die kleine Veronika}|pwv}}{\lemma{\textnormal{\emph{Novelle}}}\Cendnote{\textnormal{Wie später im Korrespondenzstück als Möglichkeit
                  thematisiert, erschien die Novelle\pwindex{kleine Veronika@\emph{Die kleine Veronika}|pwkv} noch im 
                     Dezember in der \emph{Neuen
                        Deutschen Rundschau}\pwindex{Neue Deutsche Rundschau@\emph{Neue Deutsche Rundschau}|pwk}: Felix Salten\pwindex{Salten, Felix 06.09.1869 – 08.10.1945@\textsc{Salten, Felix} (06.09.1869 – 08.10.1945), \emph{Schriftsteller/Schriftstellerin, Journalist/Journalistin, Chefredakteur/Chefredakteurin}|pwk}: \emph{Die kleine Veronika}\pwindex{kleine Veronika@\emph{Die kleine Veronika}|pwk}. In: \emph{Neue Deutsche Rundschau}\pwindex{Neue Deutsche Rundschau@\emph{Neue Deutsche Rundschau}|pwk}, Jg. 13, Nr. 12, Dezember 1902, S. 1285–1333.
                  Die Buchausgabe folgte 1903: \emph{Die kleine Veronika}\pwindex{kleine Veronika@\emph{Die kleine Veronika}|pwk}. Berlin\oindex{Berlin@\textbf{Berlin}, \emph{P.PPLC}|pwk}: \emph{S. Fischer}\orgindex{S. Fischer Verlag@S. Fischer Verlag|pwk}{ }{[}Mitte Mai{]} 1903.}}}\label{K_L02979-2}, beſonders im letzten Drittel, Zola\pwindex{Zola, Emile 02.04.1840 – 29.09.1902@\textsc{Zola, Émile} (02.04.1840 – 29.09.1902), \emph{Schriftsteller/Schriftstellerin, Journalist/Journalistin}|pw}’ſche Kraft zu, und iſt \uline{jedenfalls ſofort
                  bereit} ſie als Buch zu drucken. Gegen die Veröffentlichung in der \textsc{N. Dtsch Rds\pwindex{Neue Deutsche Rundschau@\emph{Neue Deutsche Rundschau}|pw}} ſprechen \uline{vorläufig} noch einige Bedenken
               ausſchließlich techniſcher Natur. Sie nähme 60 Seiten ein, was für \uline{eine} Nu{\geminationm}er {\pb}zu viel ſei; und neben dem im Jänner beginnenden \label{K_L02979-3v}\edtext{Roman\pwindex{Beate und Mareile. Eine Schlossgeschichte@\emph{Beate und Mareile. Eine Schloßgeschichte}|pwv}}{\lemma{\textnormal{\emph{Roman}}}\Cendnote{\textnormal{\emph{Beate und Mareile. Eine Schloßgeschichte}\pwindex{Beate und Mareile. Eine Schlossgeschichte@\emph{Beate und Mareile. Eine Schloßgeschichte}|pwk} von
                     Eduard von Keyserling\pwindex{Keyserling, Eduard von 15.05.1855 – 28.09.1918@\textsc{Keyserling, Eduard von} (15.05.1855 – 28.09.1918), \emph{Schriftsteller/Schriftstellerin}|pwk} erschien in drei
                  Teilen zwischen Januar und März 1903 in der \emph{Neuen Deutschen
                     Rundschau}\pwindex{Neue Deutsche Rundschau@\emph{Neue Deutsche Rundschau}|pwk}.}}}\label{K_L02979-3}{ }\label{T_L02979-1v}\edtext{könnten}{\lemma{\textnormal{\emph{könnten}}}\Cendnote{\textnormal{Er schreibt »konnten«.}}}\label{T_L02979-1} ſie nicht ein
               Ding in 2 Fortſetzungen bringen. Inmit\textcolor{gray}{ten} der Discuſſion kam \textsc{Bie\pwindex{Bie, Oskar 09.02.1864 – 21.04.1938@\textsc{Bie, Oskar} (09.02.1864 – 21.04.1938), \emph{Schriftsteller/Schriftstellerin, Journalist/Journalistin, Redakteur/Redakteurin}|pw}}, der die Novelle\pwindex{kleine Veronika@\emph{Die kleine Veronika}|pwv} zur
               Lecture nach Hauſe nahm. Ich habe den Eindruck, wenn ſie ihm gefällt, wird man ſie im
                  Dezemberheft\pwindex{Neue Deutsche Rundschau@\emph{Neue Deutsche Rundschau}|pwv}, trotz der 60 Seiten abdrucken. In
               Hinblick auf die Buchausgabe iſt natürlich {\pb}zuzugreifen. –\pend
           
\pstart
           In Hinſicht auf die \label{K_L02979-4v}\edtext{\textsc{Bea\pwindex{Schleier der Beatrice. Schauspiel in fuenf Akten@\emph{Der Schleier der Beatrice. Schauspiel in fünf Akten}|pw}}}{\lemma{\textnormal{\emph{Bea}}}\Cendnote{\textnormal{Schnitzler hoffte weiterhin,
                  dass \emph{Der Schleier der Beatrice}\pwindex{Schleier der Beatrice. Schauspiel in fuenf Akten@\emph{Der Schleier der Beatrice. Schauspiel in fünf Akten}|pwk} durch eine qualitätvolle Aufführung 
                  in Berlin\oindex{Berlin@\textbf{Berlin}, \emph{P.PPLC}|pwk} Erfolg haben würde. Die Berlin\oindex{Berlin@\textbf{Berlin}, \emph{P.PPLC}|pwk}-Premiere wurde letztlich am 7. 3. 1903 vom \emph{Deutschen Theater}\orgindex{Deutsches Theater Berlin@Deutsches Theater Berlin|pwk} veranstaltet.}}}\label{K_L02979-4}{ }\substVorne{}\textsuperscript{iſt}\substDazwischen{}bin\substHinten{} ich ſoweit als früher. Vom Schillertheater\orgindex{Schiller-Theater@Schiller-Theater|pw} räth mir \uline{alles} ab; die
                  \label{K_L02979-5v}\edtext{Aufführg der \textsc{\label{K_L02979-6v}\edtext{M. Vanna\pwindex{Monna Vanna. Schauspiel in drei Akten@\emph{Monna Vanna. Schauspiel in drei Akten}|pw}}{\lemma{\textnormal{\emph{M. Vanna}}}\Cendnote{\textnormal{Vgl. Paul Goldmann an Arthur Schnitzler, 16. 6. [1902].
                  }}}\label{K_L02979-6}} im Dtſch Theater\oindex{Deutsches Theater Berlin@\textbf{Deutsches Theater Berlin}, \emph{Theater (K.THE)}|pw}}{\lemma{\textnormal{\emph{Aufführg … Theater}}}\Cendnote{\textnormal{Siehe A. S.: \emph{Tagebuch}, 14. 10. 1902.
               }}}\label{K_L02979-5} iſt kläglich. Brahm\pwindex{Brahm, Otto 05.02.1856 – 28.11.1912@\textsc{Brahm, Otto} (05.02.1856 – 28.11.1912), \emph{Theaterleiter/Theaterleiterin, Regisseur/Regisseurin}|pw} will ſehr; da er
                  vorgeſtern abgereiſt iſt, reiſe ich von hier
               wahrſcheinlich \introOben{}(Samſtag)\introOben{} zu
               ihm \label{K_L02979-7v}\edtext{nach Agnetendorf\oindex{Jagniątków@\textbf{Jagniątków}, \emph{P.PPL}|pw}}{\lemma{\textnormal{\emph{nach Agnetendorf}}}\Cendnote{\textnormal{Siehe A. S.: \emph{Tagebuch}, 19. 10. 1902.
               }}}\label{K_L02979-7}, wohin ich auch von Hauptm\pwindex{Hauptmann, Gerhart 15.11.1862 – 06.06.1946@\textsc{Hauptmann, Gerhart} (15.11.1862 – 06.06.1946), \emph{Schriftsteller/Schriftstellerin}|pw} eine
                  \label{K_L02979-8v}\edtext{telegr. Einladg}{\lemma{\textnormal{\emph{telegr. Einladg}}}\Cendnote{\textnormal{nicht überliefert}}}\label{K_L02979-8} erhalten {\pb}habe, – u bringe dort die Sache ins
               Reine.\pend
           
\pstart
           Bahr\pwindex{Bahr, Hermann 19.07.1863 – 15.01.1934@\textsc{Bahr, Hermann} (19.07.1863 – 15.01.1934), \emph{Schriftsteller/Schriftstellerin, Kritiker/Kritikerin}|pw} hatte hier einen wirklichen \label{K_L02979-9v}\edtext{Erfolg\pwindex{Wienerinnen. Lustspiel in drei Akten@\emph{Wienerinnen. Lustspiel in drei Akten}|pwv}}{\lemma{\textnormal{\emph{Erfolg}}}\Cendnote{\textnormal{Am 14. 10. 1902 war Bahrs\pwindex{Bahr, Hermann 19.07.1863 – 15.01.1934@\textsc{Bahr, Hermann} (19.07.1863 – 15.01.1934), \emph{Schriftsteller/Schriftstellerin, Kritiker/Kritikerin}|pwk}{ }\emph{Wienerinnen}\pwindex{Wienerinnen. Lustspiel in drei Akten@\emph{Wienerinnen. Lustspiel in drei Akten}|pwk} – in Anwesenheit des Autors\pwindex{Bahr, Hermann 19.07.1863 – 15.01.1934@\textsc{Bahr, Hermann} (19.07.1863 – 15.01.1934), \emph{Schriftsteller/Schriftstellerin, Kritiker/Kritikerin}|pwkv} – am Berliner Theater\oindex{Berliner Theater@\textbf{Berliner Theater}, \emph{Theater (K.THE)}|pwk} aufgeführt worden.}}}\label{K_L02979-9}. –\pend
           
\pstart
           In Hinſicht auf die \label{K_L02979-10v}\edtext{Kündigungspflicht
               beim Burgtheater\orgindex{Burgtheater@Burgtheater|pw}}{\lemma{\textnormal{\emph{Kündigungspflicht beim Burgtheater}}}\Cendnote{\textnormal{Siehe Felix Salten an Arthur Schnitzler, 15. 10. 1902.
               }}}\label{K_L02979-10} ſti{\geminationm}t’s. Ich muſs am 9.
                  Nov. dem Theater\orgindex{Burgtheater@Burgtheater|pwv} das
               ausſchließliche Aufführgsrecht der Liebelei\pwindex{Liebelei. Schauspiel in drei Akten@\emph{Liebelei. Schauspiel in drei Akten}|pw} kündigen mit 2 monatlicher Friſt. Näheres
               mündlich. –\pend
           \pstart Herzlichſt Ihr \spacefill\mbox{A. S.}\pend{}\selectlanguage{ngerman}\endnumbering\briefempfaengerindex{Salten, Felix@\textsc{Salten, Felix}!zzzSchnitzler, Arthur@\emph{von Arthur Schnitzler}!1902-10-161@{16. 10. 1902}|)be}\mylabel{L02979h}  \normalsize

\doendnotes{C}
\bigskip
\vfill

\clearpage

\footnotesize

\lohead{\textsc{register}}

% Definiere theindex-Environment komplett neu ohne reledmac
\makeatletter
\renewenvironment{theindex}{%
  \section*{\indexname}%
  \setlength{\parindent}{0pt}%
  \setlength{\parskip}{0pt plus 0.3pt}%
  \let\item\@idxitem
}{%
  \clearpage
}
\makeatother

\IfFileExists{\jobname-pw.ind}{\input{\jobname-pw.ind}}{}

\end{document}

      