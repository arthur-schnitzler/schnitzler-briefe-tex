%% latex-leseansicht-vorspann.tex
%% Vorspann für die Leseansicht.
%% Lädt die gemeinsame Datei latex-vorspann.tex mit nicht gesetztem Schalter.

\newif\ifkorrekturansicht
\korrekturansichtfalse

\input{../tex-inputs/latex-vorspann}


\section[ Paul Goldmann an Arthur Schnitzler, 15. 6. [1897]]{L02814 Paul Goldmann an Arthur Schnitzler,  15. 6. [1897]}
\nopagebreak\mylabel{L02814v}
\rehead{ }\normalsize\beginnumbering\briefempfaengerindex{Schnitzler, Arthur@\textsc{Schnitzler, Arthur}!zzzGoldmann, Paul@\emph{von Paul Goldmann}!1897-06-152@{15. 6. [1897]}|(be}
\toendnotes[C]{\smallbreak\pagebreak[2]}
\correspDesc{Versand  durch Paul Goldmann am 15. 6. [1897] in Paris
\newline{}Erhalt  durch Arthur Schnitzler im Zeitraum [16. 6. 1897
                  – 20. 6. 1897?] in Wien}\toendnotes[C]{\smallbreak}
\Standort{DLA, A:Schnitzler, HS.NZ85.1.3167.}
\physDesc{Brief, 1 Blatt, 3 Seiten, 1776 Zeichen
\newline{}Handschrift: blaue Tinte, deutsche Kurrent
\newline{}Schnitzler: 1) mit Bleistift das Jahr »97« vermerkt  2) mit rotem Buntstift zwei Unterstreichungen}\toendnotes[C]{\smallbreak}
\pstart
           {\pb}\textcolor{gray}{\textbf{\textbf{Frankfurter Zeitung\orgindex{Frankfurter Zeitung@Frankfurter Zeitung|pw}}}}\pend
           
\pstart
           \textcolor{gray}{\textbf{(\begin{otherlanguage}{french}Gazette de Francfort\end{otherlanguage}\orgindex{Frankfurter Zeitung@Frankfurter Zeitung|pw}).}}\pend
           
\pstart
           \textcolor{gray}{\textbf{\textbf{\begin{otherlanguage}{french}Fondateur M.\end{otherlanguage}{ }L. Sonnemann\pwindex{Sonnemann, Leopold 29.\,10.\,1831 Höchberg – 30.\,10.\,1909 Frankfurt am Main@\textsc{Sonnemann, Leopold} (29.\,10.\,1831 Höchberg – 30.\,10.\,1909 Frankfurt am Main), \emph{Journalist, Herausgeber}|pw}.}}}\pend
           
\pstart
           \begin{otherlanguage}{french}\textcolor{gray}{\textbf{Journal politique, financier,}}\end{otherlanguage}\pend
           
\pstart
           \begin{otherlanguage}{french}\textcolor{gray}{\textbf{commercial et littéraire.}}\end{otherlanguage}\pend
           
\pstart
           \begin{otherlanguage}{french}\textcolor{gray}{\textbf{\textbf{Paraissant trois fois par jour.}}}\end{otherlanguage}\hfill \textsc{Paris\oindex{Paris@\textbf{Paris}, \emph{Hauptstadt}|pw}}, 15. Juni.\pend
           
\pstart
           \begin{otherlanguage}{french}\textcolor{gray}{\textbf{\textbf{Bureau à Paris\oindex{Paris@\textbf{Paris}, \emph{Hauptstadt}|pw}}}}\end{otherlanguage}\pend
           
\pstart
           \begin{otherlanguage}{french}\textcolor{gray}{\textbf{\textbf{10 \so{Rue de la Bourse}\oindex{rue de la Bourse@\textbf{rue de la Bourse}, \emph{Straße}|pw}.}}}\end{otherlanguage}\pend
           
\pstart\center{}Mein lieber Freund,\pend\vspace{0.5em}
\pstart
           Ich wollte Dir immerfort{ }ſchon{ }ſchreiben; aber ich habe wieder{ }ſo hunderterlei zu
               thun gehabt, und von Tag zu Tage mußte ich das Project verſchieben, bis endlich Dein
               Brief kam.\pend
           
\pstart
           In der erſten Zeit nach Deiner Abreiſe haſt Du mir an allen Ecken und Enden gefehlt.
               Nur{ }ſchwer habe ich mich wieder an das Alleinſein mit \strikeout{f\textcolor{gray}{re}mde} all’ den fremden Menſchen gewöhnen können.\pend
           
\pstart
           Geſtern habe ich endlich auch eine halbe Stunde Zeit
               gefunden, um zu \textsc{Madame Marni\pwindex{Marni, Jeanne 31.\,1.\,1854 Toulouse – 6.\,1.\,1910 Cannes@\textsc{Marni, Jeanne} (31.\,1.\,1854 Toulouse – 6.\,1.\,1910 Cannes), \emph{Schriftstellerin}|pw}} zu gehen. Sie{ }ſprach{ }ſehr warm von Dir und {\pb}hat Dich offenbar \label{K_L02814-1v}\edtext{ſehr gut
                  verſtanden}{\lemma{\textnormal{\emph{sehr gut
                  verstanden}}}\Cendnote{\textnormal{Schnitzler hatte Jeanne Marni\pwindex{Marni, Jeanne 31.\,1.\,1854 Toulouse – 6.\,1.\,1910 Cannes@\textsc{Marni, Jeanne} (31.\,1.\,1854 Toulouse – 6.\,1.\,1910 Cannes), \emph{Schriftstellerin}|pwk} gemeinsam mit ihrer Tochter Emmy Fournier\pwindex{Fournier, Emmy 1874 – 1944@\textsc{Fournier, Emmy} (1874 – 1944), \emph{Chefredakteurin}|pwk} und Goldmann\pwindex{Goldmann, Paul 31.\,1.\,1865 Breslau – 25.\,9.\,1935 Wien@\textsc{Goldmann, Paul} (31.\,1.\,1865 Breslau – 25.\,9.\,1935 Wien), \emph{Schriftsteller, Journalist}|pwk} am 14. 5. 1897 getroffen. Im \emph{Tagebuch}\pwindex{Schnitzler, Arthur 15.\,5.\,1862 Wien – 21.\,10.\,1931 ebd.@\textsc{Schnitzler, Arthur} (15.\,5.\,1862 Wien – 21.\,10.\,1931 ebd.), \emph{Schriftsteller, Mediziner}!Tagebuch@\strich\emph{Tagebuch}|pwk}
                  notierte Schnitzler einen äußerst positiven
                  Eindruck von ihr. Am 19. 5. 1897 war er ihr mit Goldmann\pwindex{Goldmann, Paul 31.\,1.\,1865 Breslau – 25.\,9.\,1935 Wien@\textsc{Goldmann, Paul} (31.\,1.\,1865 Breslau – 25.\,9.\,1935 Wien), \emph{Schriftsteller, Journalist}|pwk} und Paul Hermann\pwindex{Paul, Hermann 27.\,12.\,1864 Paris – 1940-07@\textsc{Paul, Hermann} (27.\,12.\,1864 Paris – 1940-07), \emph{Maler, Karikaturist}|pwk} noch
                  einmal begegnet.}}}\label{K_L02814-1}. Deine Roſen haben{ }ſie{ }ſehr entzückt. Sie hätte Dir gern gedankt,
               wenn{ }ſie Deine Adreſſe gewußt hätte.\pend
           
\pstart
           Daß ich Ende Juli nicht fortkann, iſt{ }ſo gut wie{ }ſicher.
               Ich muß jetzt auch mit der \label{K_L02814-2v}\edtext{ruſſ\oindex{Russland@\textbf{Russland}|pwv}iſchen Reiſe des Präſident\pwindex{Faure, Félix 30.\,1.\,1841 Paris – 16.\,2.\,1899 ebd.@\textsc{Faure, Félix} (30.\,1.\,1841 Paris – 16.\,2.\,1899 ebd.), \emph{Politiker, Präsident}|pwv}en}{\lemma{\textnormal{\emph{russischen … Präsidenten}}}\Cendnote{\textnormal{Nachdem Nikolaus II.\pwindex{Nikolaus II. von Russland 6.\,5.\,1868 Pushkin – 17.\,7.\,1918 Jekaterinburg@\textsc{Nikolaus II. von Russland} (6.\,5.\,1868 Pushkin – 17.\,7.\,1918 Jekaterinburg), \emph{Zar}|pwk}{ }Frankreich\oindex{Frankreich@\textbf{Frankreich}|pwk} im Vorjahr besucht hatte, reiste
                  der fran\oindex{Frankreich@\textbf{Frankreich}|pwkv}zösische Präsident
                     Félix Faure\pwindex{Faure, Félix 30.\,1.\,1841 Paris – 16.\,2.\,1899 ebd.@\textsc{Faure, Félix} (30.\,1.\,1841 Paris – 16.\,2.\,1899 ebd.), \emph{Politiker, Präsident}|pwk} auf Einladung des Zaren\pwindex{Nikolaus II. von Russland 6.\,5.\,1868 Pushkin – 17.\,7.\,1918 Jekaterinburg@\textsc{Nikolaus II. von Russland} (6.\,5.\,1868 Pushkin – 17.\,7.\,1918 Jekaterinburg), \emph{Zar}|pwkv} im August 1897 nach Russland\oindex{Russland@\textbf{Russland}|pwk}.}}}\label{K_L02814-2} rechnen, während deren ich in \textsc{Paris\oindex{Paris@\textbf{Paris}, \emph{Hauptstadt}|pw}} bleiben muß wegen möglicher Zwiſchenfälle. Könnteſt Du Dir es nicht{ }ſo
               einrichten, daß Du \uline{Mitte}{ }Auguſt auf 8 bis 10 Tage nach \label{K_L02814-3v}\edtext{\textsc{Ischl\oindex{Bad Ischl@\textbf{Bad Ischl}|pw}}}{\lemma{\textnormal{\emph{Ischl}}}\Cendnote{\textnormal{Schnitzler war vom 19. 8. 1897 bis zum 30. 8. 1897 in Bad Ischl\oindex{Bad Ischl@\textbf{Bad Ischl}|pwk}. Goldmann\pwindex{Goldmann, Paul 31.\,1.\,1865 Breslau – 25.\,9.\,1935 Wien@\textsc{Goldmann, Paul} (31.\,1.\,1865 Breslau – 25.\,9.\,1935 Wien), \emph{Schriftsteller, Journalist}|pwk} war zu dieser Zeit auch dort.}}}\label{K_L02814-3} kommſt? Wenn nicht,{ }ſo
               werde ich wohl kaum mich dorthin begeben. Immerhin iſt das Alles noch nicht
               endgiltig. Meine definitiven Dispoſitionen hängen vom Gang der Ereigniſſe ab.\pend
           
\pstart
           An meine Mutter\pwindex{Goldmann, Clementine 15.\,5.\,1842 Breslau – 24.\,2.\,1924 Frankfurt am Main@\textsc{Goldmann, Clementine} (15.\,5.\,1842 Breslau – 24.\,2.\,1924 Frankfurt am Main)|pwv} habe ich {\pb}mindeſtens dreimal geſchrieben, daß{ }ſie Dir den \textsc{Nansen\pwindex{Nansen, Peter 20.\,1.\,1861 Kopenhagen – 31.\,7.\,1918 Mariager@\textsc{Nansen, Peter} (20.\,1.\,1861 Kopenhagen – 31.\,7.\,1918 Mariager), \emph{Schriftsteller, Journalist, Verleger}|pw}schen}{ }\label{K_L02814-4v}\edtext{Artikel\pwindex{Nansen, Peter 20.\,1.\,1861 Kopenhagen – 31.\,7.\,1918 Mariager@\textsc{Nansen, Peter} (20.\,1.\,1861 Kopenhagen – 31.\,7.\,1918 Mariager), \emph{Schriftsteller, Journalist, Verleger}!?? [Artikel von Peter Nansen, Mai/Juni 1897]@\strich\emph{?? [Artikel von Peter Nansen, Mai/Juni 1897]}|pwv}}{\lemma{\textnormal{\emph{Artikel}}}\Cendnote{\textnormal{Siehe XXXX Auszeichnungsfehler: Dokument L02815 nicht gefunden.
               }}}\label{K_L02814-4}{ }ſchicken möge. Hoffentlich haſt Du ihn jetzt endlich erhalten.\pend
           
\pstart
           Daß Frau \textsc{Olga\pwindex{Waissnix, Olga 3.\,11.\,1862 Wien – 4.\,11.\,1897 ebd.@\textsc{Waissnix, Olga} (3.\,11.\,1862 Wien – 4.\,11.\,1897 ebd.), \emph{Hotelière}|pw}} die{ }ſchwere \label{K_L02814-5v}\edtext{Operation}{\lemma{\textnormal{\emph{Operation}}}\Cendnote{\textnormal{Olga Waissnix\pwindex{Waissnix, Olga 3.\,11.\,1862 Wien – 4.\,11.\,1897 ebd.@\textsc{Waissnix, Olga} (3.\,11.\,1862 Wien – 4.\,11.\,1897 ebd.), \emph{Hotelière}|pwk} wurde im Mai 1897 zweimal im Sanatorium Loew\oindex{Wien@\textbf{Wien}!IX., Alsergrund@\textbf{IX., Alsergrund}!Sanatorium Loew@\textbf{Sanatorium Loew}, \emph{Sanatorium}|pwk}
                  in der Mariannengasse\oindex{Wien@\textbf{Wien}!IX., Alsergrund@\textbf{IX., Alsergrund}!Mariannengasse@\textbf{Mariannengasse}, \emph{Straße}|pwk} operiert. Die erste
                  Operation, bei der womöglich eine Krebserkrankung festgestellt wurde, fand am
                     16. 5. 1897 statt, die zweite am 25. 5. 1897. Ihr wurden die Gebärmutter sowie die
                  Eierstöcke entfernt. Vgl. \emph{Arthur Schnitzler, Olga Waissnix. Liebe, die starb vor der
                        Zeit. Ein Briefwechsel}. Mit einem Vorwort von Hans Weigel. Herausgegeben von 
                     Therese Nickl und Heinrich Schnitzler. Wien,
                     München, Zürich: \emph{Fritz
                        Molden}{ }1970, S. 322 u. 324; Elisabeth-Joe
                     Harriet: \emph{Die unvollendete Geliebte. Olga Waissnix {\kaufmannsund} Arthur Schnitzler}.
                     Wien: \emph{Almathea}{ }2015, S. 369–371 [E-Book].}}}\label{K_L02814-5} glücklich überſtanden hat, freut mich von Herzen. Es iſt{ }ſchön, daß{ }ſie{ }ſich
               meiner noch \label{K_L02814-6v}\edtext{erinnert}{\lemma{\textnormal{\emph{erinnert}}}\Cendnote{\textnormal{Im Brief von Olga Waissnix\pwindex{Waissnix, Olga 3.\,11.\,1862 Wien – 4.\,11.\,1897 ebd.@\textsc{Waissnix, Olga} (3.\,11.\,1862 Wien – 4.\,11.\,1897 ebd.), \emph{Hotelière}|pwk} an Schnitzler vom 13. 5. 1897 bat sie ihn,
                     Goldmann\pwindex{Goldmann, Paul 31.\,1.\,1865 Breslau – 25.\,9.\,1935 Wien@\textsc{Goldmann, Paul} (31.\,1.\,1865 Breslau – 25.\,9.\,1935 Wien), \emph{Schriftsteller, Journalist}|pwk} zu grüßen. Vgl. \emph{Arthur Schnitzler, Olga Waissnix. Liebe, die starb vor der Zeit.
                        Ein Briefwechsel}. Mit einem Vorwort von Hans Weigel. Herausgegeben von  Therese
                     Nickl und Heinrich Schnitzler. Wien,
                     München, Zürich: \emph{Fritz
                        Molden}{ }1970, S. 322.}}}\label{K_L02814-6}. Empfiehl’ mich ihr,
               bitte, und{ }ſag’ ihr,{ }ſie{ }ſolle \strikeout{\textcolor{gray}{e}} eine \label{K_L02814-7v}\edtext{Reconvalescenz}{\lemma{\textnormal{\emph{Reconvalescenz}}}\Cendnote{\textnormal{Genesung}}}\label{K_L02814-7}-Reiſe nach \textsc{Paris\oindex{Paris@\textbf{Paris}, \emph{Hauptstadt}|pw}} machen.\pend
           
\pstart
           Die Klatſcherei von \label{K_L02814-8v}\edtext{\textsc{M. B.\pwindex{Schaffgotsch, Hermine von 25.\,11.\,1871 Wien – 25.\,11.\,1928 Purgstall@\textsc{Schaffgotsch, Hermine von} (25.\,11.\,1871 Wien – 25.\,11.\,1928 Purgstall)|pwv}}}{\lemma{\textnormal{\emph{M. B.}}}\Cendnote{\textnormal{Schnitzler hatte am 8. 6. 1897 erfahren, dass Minnie Benedict\pwindex{Schaffgotsch, Hermine von 25.\,11.\,1871 Wien – 25.\,11.\,1928 Purgstall@\textsc{Schaffgotsch, Hermine von} (25.\,11.\,1871 Wien – 25.\,11.\,1928 Purgstall)|pwk} in Anwesenheit verschiedener
                  anderer Leute »erzählte, dass ich mit einem Vorstadtmädel\pwindex{Reinhard, Marie 13.\,3.\,1871 Wien – 18.\,3.\,1899 ebd.@\textsc{Reinhard, Marie} (13.\,3.\,1871 Wien – 18.\,3.\,1899 ebd.), \emph{Gesangspädagogin}|pwv}, wegen Kind\pwindex{?? [Totgeborener Sohn von Arthur Schnitzler und Marie Reinhard] 24.\,9.\,1897 Endresstraße 68 – 24.\,9.\,1897 ebd.@\textsc{?? [Totgeborener Sohn von Arthur Schnitzler und Marie Reinhard]} (24.\,9.\,1897 Endresstraße 68 – 24.\,9.\,1897 ebd.)|pwv} etc. nach Paris\oindex{Paris@\textbf{Paris}, \emph{Hauptstadt}|pw} gereist«.}}}\label{K_L02814-8} iſt widerwärtig. Oh dieſe
               iſraelitiſchen Jungfrauen! {\dotsfour}\pend
           
\pstart
           Ich{ }ſchlafe{ }ſchlecht, bin unzufrieden und mißmuthig{\dotsfive}\pend
           
\pstart
           Könnteſt Du nicht am 9. oder 11. Auguſt zum \label{K_L02814-9v}\edtext{\textsc{Parsifal\pwindex{\textcolor{red}{\textsuperscript{XXXX indx1}}!Parsifal@\strich\emph{Parsifal}|pw}} nach \textsc{Bayreuth\oindex{Bayreuth@\textbf{Bayreuth}, \emph{Hauptstadt}|pw}\orgindex{Bayreuther Festspiele@Bayreuther Festspiele|pwv}}}{\lemma{\textnormal{\emph{Parsifal nach Bayreuth}}}\Cendnote{\textnormal{Dazu kam es nicht. Die \emph{Bayreuther Festspiele}\orgindex{Bayreuther Festspiele@Bayreuther Festspiele|pwk} wurden 1897
                  von Cosima Wagner\pwindex{Wagner, Cosima 25.\,12.\,1837 Como – 1.\,4.\,1930 Bayreuth@\textsc{Wagner, Cosima} (25.\,12.\,1837 Como – 1.\,4.\,1930 Bayreuth)|pwk} geleitet.}}}\label{K_L02814-9}
               kommen?\pend
           
\pstart
           Grüße Deine Freundin\pwindex{Reinhard, Marie 13.\,3.\,1871 Wien – 18.\,3.\,1899 ebd.@\textsc{Reinhard, Marie} (13.\,3.\,1871 Wien – 18.\,3.\,1899 ebd.), \emph{Gesangspädagogin}|pwv}
               recht herzlich und{ }ſei Du{ }ſelbſt vielmals gegrüßt von {\\[\baselineskip]}Deinem treuen \spacefill\mbox{Paul
                  Goldm}\pend
           \leftskip=0em{}\selectlanguage{ngerman}\endnumbering\briefempfaengerindex{Schnitzler, Arthur@\textsc{Schnitzler, Arthur}!zzzGoldmann, Paul@\emph{von Paul Goldmann}!1897-06-152@{15. 6. [1897]}|)be}\mylabel{L02814h}  \newcommand{\dateiname}{L02814}\newcommand{\titel}{Paul Goldmann an Arthur Schnitzler, 15. 6. [1897]}\newcommand{\editorInnen}{Martin Anton Müller und Laura Untner}%% latex-leseansicht-abspann.tex
%% Abspann für die Leseansicht.
%% Der Schalter \ifkorrekturansicht ist bereits durch den Vorspann gesetzt.

%% latex-abspann.tex
%% Gemeinsamer Abspann für Korrekturansicht und Leseansicht.
%% Setzt den Schalter \ifkorrekturansicht voraus (gesetzt in den
%% einbindenden Dateien latex-korrekturansicht-abspann.tex bzw.
%% latex-leseansicht-abspann.tex).
%% ---------------------------------------------------------------

\normalsize

% Das esempio-Environment wird nur in der Leseansicht benötigt
\ifkorrekturansicht\else
\newenvironment{esempio}[3]%
{
    \vspace{1.5ex}
    \rlap{\underline{#1}}
    \par
    \setlength{\parindent}{0cm}
    \nopagebreak
    \leftskip=#2cm
    \rightskip=#3cm
}
{
    \par
}
\fi

\doendnotes{C}
\bigskip
\vfill

\clearpage

\footnotesize

\ifkorrekturansicht
  \lohead{\textsc{register}}
\fi

% theindex-Environment neu definieren ohne reledmac
\makeatletter
\renewenvironment{theindex}{%
  \ifkorrekturansicht
    \section*{\indexname}%
  \else
    \subsubsection*{Index der erwähnten Entitäten}%
  \fi
  \setlength{\parindent}{0pt}%
  \setlength{\parskip}{0pt plus 0.3pt}%
  \let\item\@idxitem
}{%
  \ifkorrekturansicht\clearpage\fi
}
\makeatother

\IfFileExists{\jobname-pw.ind}{\input{\jobname-pw.ind}}{}

% Quellenangabe nur in der Leseansicht
\ifkorrekturansicht\else
% Fallback-Definitionen, falls die .tex-Datei \titel etc. nicht gesetzt hat
\providecommand{\titel}{}
\providecommand{\editorInnen}{}
\providecommand{\dateiname}{\jobname}

\vspace{3cm}

\vfill

\footnotesize
\textsc{Quelle}: \titel. Herausgegeben von {\editorInnen}. In: \emph{Arthur Schnitzler: Briefwechsel mit Autorinnen und Autoren}.
 Digitale Edition, https://schnitzler-briefe.acdh.oeaw.ac.at/{\dateiname}.html (Stand \today)
\fi

\end{document}


