%% latex-korrekturansicht-vorspann.tex
%% Vorspann für die Korrekturansicht.
%% Lädt die gemeinsame Datei latex-vorspann.tex mit gesetztem Schalter.

\newif\ifkorrekturansicht
\korrekturansichttrue

\input{../tex-inputs/latex-vorspann}


\section[ Paul Goldmann an Arthur Schnitzler, 15. 6. {[}1897{]}]{L02814 Paul Goldmann an Arthur Schnitzler, 15. 6. {[}1897{]}}
\nopagebreak\mylabel{L02814v}
\rehead{ }\normalsize\beginnumbering\briefempfaengerindex{Schnitzler, Arthur@\textsc{Schnitzler, Arthur}!zzzGoldmann, Paul@\emph{von Paul Goldmann}!1897-06-152@{15. 6. {[}1897{]}}|(be}
\toendnotes[C]{\smallbreak\pagebreak[2]}\Standort{DLA, A:Schnitzler, HS.NZ85.1.3167.}
\physDesc{Brief, 1 Blatt, 3 Seiten, 1776 Zeichen
\newline{}Handschrift: blaue Tinte, deutsche Kurrent
\newline{}Schnitzler: 1) mit Bleistift das Jahr »97« vermerkt  2) mit rotem Buntstift zwei Unterstreichungen}\toendnotes[C]{\smallbreak}
\pstart
           {\pb}\textcolor{gray}{\textbf{\textbf{Frankfurter Zeitung\orgindex{Frankfurter Zeitung@Frankfurter Zeitung|pw}}}}\pend
           
\pstart
           \textcolor{gray}{\textbf{(\begin{otherlanguage}{french}Gazette de Francfort\end{otherlanguage}\orgindex{Frankfurter Zeitung@Frankfurter Zeitung|pw}).}}\pend
           
\pstart
           \textcolor{gray}{\textbf{\textbf{\begin{otherlanguage}{french}Fondateur M.\end{otherlanguage}{ }L. Sonnemann\pwindex{Sonnemann, Leopold 1831-10-29 – 1909-10-30@\textsc{Sonnemann, Leopold} (1831-10-29 – 1909-10-30), \emph{Journalist/Journalistin, Herausgeber/Herausgeberin}|pw}.}}}\pend
           
\pstart
           \begin{otherlanguage}{french}\textcolor{gray}{\textbf{Journal politique, financier,}}\end{otherlanguage}\pend
           
\pstart
           \begin{otherlanguage}{french}\textcolor{gray}{\textbf{commercial et littéraire.}}\end{otherlanguage}\pend
           
\pstart
           \begin{otherlanguage}{french}\textcolor{gray}{\textbf{\textbf{Paraissant trois fois par jour.}}}\end{otherlanguage}\hfill \textsc{Paris\oindex{Paris@\textbf{Paris}, \emph{P.PPLC}|pw}}, 15. Juni.\pend
           
\pstart
           \begin{otherlanguage}{french}\textcolor{gray}{\textbf{\textbf{Bureau à Paris\oindex{Paris@\textbf{Paris}, \emph{P.PPLC}|pw}}}}\end{otherlanguage}\pend
           
\pstart
           \begin{otherlanguage}{french}\textcolor{gray}{\textbf{\textbf{10 \so{Rue de la Bourse}\oindex{rue de la Bourse@\textbf{rue de la Bourse}, \emph{Straße (K.STR)}|pw}.}}}\end{otherlanguage}\pend
           
\pstart\center{}Mein lieber Freund,\pend\vspace{0.5em}
\pstart
           Ich wollte Dir immerfort ſchon ſchreiben; aber ich habe wieder ſo hunderterlei zu
               thun gehabt, und von Tag zu Tage mußte ich das Project verſchieben, bis endlich Dein
               Brief kam.\pend
           
\pstart
           In der erſten Zeit nach Deiner Abreiſe haſt Du mir an allen Ecken und Enden gefehlt.
               Nur ſchwer habe ich mich wieder an das Alleinſein mit \strikeout{f\textcolor{gray}{re}mde} all’ den fremden Menſchen gewöhnen können.\pend
           
\pstart
           Geſtern habe ich endlich auch eine halbe Stunde Zeit
               gefunden, um zu \textsc{Madame Marni\pwindex{Marni, Jeanne 1854-01-31 – 1910-01-06@\textsc{Marni, Jeanne} (1854-01-31 – 1910-01-06), \emph{Schriftsteller/Schriftstellerin}|pw}} zu gehen. Sie ſprach ſehr warm von Dir und {\pb}hat Dich offenbar \label{K_L02814-1v}\edtext{ſehr gut
                  verſtanden}{\lemma{\textnormal{\emph{ſehr gut
                  verſtanden}}}\Cendnote{\textnormal{Schnitzler hatte Jeanne Marni\pwindex{Marni, Jeanne 1854-01-31 – 1910-01-06@\textsc{Marni, Jeanne} (1854-01-31 – 1910-01-06), \emph{Schriftsteller/Schriftstellerin}|pwk} gemeinsam mit ihrer Tochter Emmy Fournier\pwindex{Fournier, Emmy 1874 – 1944@\textsc{Fournier, Emmy} (1874 – 1944), \emph{Chefredakteur/Chefredakteurin}|pwk} und Goldmann\pwindex{Goldmann, Paul 31.01.1865 – 25.09.1935@\textsc{Goldmann, Paul} (31.01.1865 – 25.09.1935), \emph{Schriftsteller/Schriftstellerin, Journalist/Journalistin}|pwk} am 14. 5. 1897 getroffen. Im \emph{Tagebuch}\pwindex{Tagebuch@\emph{Tagebuch}|pwk}
                  notierte Schnitzler einen äußerst positiven
                  Eindruck von ihr. Am 19. 5. 1897 war er ihr mit Goldmann\pwindex{Goldmann, Paul 31.01.1865 – 25.09.1935@\textsc{Goldmann, Paul} (31.01.1865 – 25.09.1935), \emph{Schriftsteller/Schriftstellerin, Journalist/Journalistin}|pwk} und Paul Hermann\pwindex{Paul, Hermann 27.12.1864 – 1940.07@\textsc{Paul, Hermann} (27.12.1864 – 1940.07), \emph{Maler/Malerin, Karikaturist/Karikaturistin}|pwk} noch
                  einmal begegnet.}}}\label{K_L02814-1}. Deine Roſen haben ſie ſehr entzückt. Sie hätte Dir gern gedankt,
               wenn ſie Deine Adreſſe gewußt hätte.\pend
           
\pstart
           Daß ich Ende Juli nicht fortkann, iſt ſo gut wie ſicher.
               Ich muß jetzt auch mit der \label{K_L02814-2v}\edtext{ruſſ\oindex{Russland@\textbf{Russland}, \emph{A.PCLI}|pwv}iſchen Reiſe des Präſident\pwindex{Faure, Felix 1841-01-30 – 1899-02-16@\textsc{Faure, Félix} (1841-01-30 – 1899-02-16), \emph{Politiker/Politikerin, Präsident/Präsidentin}|pwv}en}{\lemma{\textnormal{\emph{ruſſiſchen … Präſidenten}}}\Cendnote{\textnormal{Nachdem Nikolaus II.\pwindex{Nikolaus II. von Russland 1868-05-06 – 1918-07-17@\textsc{Nikolaus II. von Russland} (1868-05-06 – 1918-07-17), \emph{Zar/Zarin}|pwk}{ }Frankreich\oindex{Frankreich@\textbf{Frankreich}, \emph{A.PCLI}|pwk} im Vorjahr besucht hatte, reiste
                  der fran\oindex{Frankreich@\textbf{Frankreich}, \emph{A.PCLI}|pwkv}zösische Präsident
                     Félix Faure\pwindex{Faure, Felix 1841-01-30 – 1899-02-16@\textsc{Faure, Félix} (1841-01-30 – 1899-02-16), \emph{Politiker/Politikerin, Präsident/Präsidentin}|pwk} auf Einladung des Zaren\pwindex{Nikolaus II. von Russland 1868-05-06 – 1918-07-17@\textsc{Nikolaus II. von Russland} (1868-05-06 – 1918-07-17), \emph{Zar/Zarin}|pwkv} im August 1897 nach Russland\oindex{Russland@\textbf{Russland}, \emph{A.PCLI}|pwk}.}}}\label{K_L02814-2} rechnen, während deren ich in \textsc{Paris\oindex{Paris@\textbf{Paris}, \emph{P.PPLC}|pw}} bleiben muß wegen möglicher Zwiſchenfälle. Könnteſt Du Dir es nicht ſo
               einrichten, daß Du \uline{Mitte}{ }Auguſt auf 8 bis 10 Tage nach \label{K_L02814-3v}\edtext{\textsc{Ischl\oindex{Bad Ischl@\textbf{Bad Ischl}, \emph{P.PPL}|pw}}}{\lemma{\textnormal{\emph{Ischl}}}\Cendnote{\textnormal{Schnitzler war vom 19. 8. 1897 bis zum 30. 8. 1897 in Bad Ischl\oindex{Bad Ischl@\textbf{Bad Ischl}, \emph{P.PPL}|pwk}. Goldmann\pwindex{Goldmann, Paul 31.01.1865 – 25.09.1935@\textsc{Goldmann, Paul} (31.01.1865 – 25.09.1935), \emph{Schriftsteller/Schriftstellerin, Journalist/Journalistin}|pwk} war zu dieser Zeit auch dort.}}}\label{K_L02814-3} kommſt? Wenn nicht, ſo
               werde ich wohl kaum mich dorthin begeben. Immerhin iſt das Alles noch nicht
               endgiltig. Meine definitiven Dispoſitionen hängen vom Gang der Ereigniſſe ab.\pend
           
\pstart
           An meine Mutter\pwindex{Goldmann, Clementine 1842-05-15 – 1924-02-24@\textsc{Goldmann, Clementine} (1842-05-15 – 1924-02-24)|pwv} habe ich {\pb}mindeſtens dreimal geſchrieben, daß ſie Dir den \textsc{Nansen\pwindex{Nansen, Peter 20.01.1861 – 31.07.1918@\textsc{Nansen, Peter} (20.01.1861 – 31.07.1918), \emph{Schriftsteller/Schriftstellerin, Journalist/Journalistin, Verleger/Verlegerin}|pw}schen}{ }\label{K_L02814-4v}\edtext{Artikel\pwindex{?? [Artikel von Peter Nansen, Mai/Juni 1897]@\emph{?? [Artikel von Peter Nansen, Mai/Juni 1897]}|pwv}}{\lemma{\textnormal{\emph{Artikel}}}\Cendnote{\textnormal{Siehe Paul Goldmann an Arthur Schnitzler, 18. 6. [1897].
               }}}\label{K_L02814-4} ſchicken möge. Hoffentlich haſt Du ihn jetzt endlich erhalten.\pend
           
\pstart
           Daß Frau \textsc{Olga\pwindex{Waissnix, Olga 03.11.1862 – 04.11.1897@\textsc{Waissnix, Olga} (03.11.1862 – 04.11.1897), \emph{Hotelier/Hotelière}|pw}} die ſchwere \label{K_L02814-5v}\edtext{Operation}{\lemma{\textnormal{\emph{Operation}}}\Cendnote{\textnormal{Olga Waissnix\pwindex{Waissnix, Olga 03.11.1862 – 04.11.1897@\textsc{Waissnix, Olga} (03.11.1862 – 04.11.1897), \emph{Hotelier/Hotelière}|pwk} wurde im Mai 1897 zweimal im Sanatorium Loew\oindex{Sanatorium Loew@\textbf{Sanatorium Loew}, \emph{Sanatorium (K.SAN)}|pwk}
                  in der Mariannengasse\oindex{Mariannengasse@\textbf{Mariannengasse}, \emph{Straße (K.STR)}|pwk} operiert. Die erste
                  Operation, bei der womöglich eine Krebserkrankung festgestellt wurde, fand am
                     16. 5. 1897 statt, die zweite am 25. 5. 1897. Ihr wurden die Gebärmutter sowie die
                  Eierstöcke entfernt. Vgl. \emph{Arthur Schnitzler, Olga Waissnix. Liebe, die starb vor der
                        Zeit. Ein Briefwechsel}. Mit einem Vorwort von Hans Weigel. Herausgegeben von 
                     Therese Nickl und Heinrich Schnitzler. Wien,
                     München, Zürich: \emph{Fritz
                        Molden}{ }1970, S. 322 u. 324; Elisabeth-Joe
                     Harriet: \emph{Die unvollendete Geliebte. Olga Waissnix {\kaufmannsund} Arthur Schnitzler}.
                     Wien: \emph{Almathea}{ }2015, S. 369–371 [E-Book].}}}\label{K_L02814-5} glücklich überſtanden hat, freut mich von Herzen. Es iſt ſchön, daß ſie ſich
               meiner noch \label{K_L02814-6v}\edtext{erinnert}{\lemma{\textnormal{\emph{erinnert}}}\Cendnote{\textnormal{Im Brief von Olga Waissnix\pwindex{Waissnix, Olga 03.11.1862 – 04.11.1897@\textsc{Waissnix, Olga} (03.11.1862 – 04.11.1897), \emph{Hotelier/Hotelière}|pwk} an Schnitzler vom 13. 5. 1897 bat sie ihn,
                     Goldmann\pwindex{Goldmann, Paul 31.01.1865 – 25.09.1935@\textsc{Goldmann, Paul} (31.01.1865 – 25.09.1935), \emph{Schriftsteller/Schriftstellerin, Journalist/Journalistin}|pwk} zu grüßen. Vgl. \emph{Arthur Schnitzler, Olga Waissnix. Liebe, die starb vor der Zeit.
                        Ein Briefwechsel}. Mit einem Vorwort von Hans Weigel. Herausgegeben von  Therese
                     Nickl und Heinrich Schnitzler. Wien,
                     München, Zürich: \emph{Fritz
                        Molden}{ }1970, S. 322.}}}\label{K_L02814-6}. Empfiehl’ mich ihr,
               bitte, und ſag’ ihr, ſie ſolle \strikeout{\textcolor{gray}{e}} eine \label{K_L02814-7v}\edtext{Reconvalescenz}{\lemma{\textnormal{\emph{Reconvalescenz}}}\Cendnote{\textnormal{Genesung}}}\label{K_L02814-7}-Reiſe nach \textsc{Paris\oindex{Paris@\textbf{Paris}, \emph{P.PPLC}|pw}} machen.\pend
           
\pstart
           Die Klatſcherei von \label{K_L02814-8v}\edtext{\textsc{M. B.\pwindex{Schaffgotsch, Hermine von 25.11.1871 – 25.11.1928@\textsc{Schaffgotsch, Hermine von} (25.11.1871 – 25.11.1928)|pwv}}}{\lemma{\textnormal{\emph{M. B.}}}\Cendnote{\textnormal{Schnitzler hatte am 8. 6. 1897 erfahren, dass Minnie Benedict\pwindex{Schaffgotsch, Hermine von 25.11.1871 – 25.11.1928@\textsc{Schaffgotsch, Hermine von} (25.11.1871 – 25.11.1928)|pwk} in Anwesenheit verschiedener
                  anderer Leute »erzählte, dass ich mit einem Vorstadtmädel\pwindex{Reinhard, Marie 1871-03-13 – 1899-03-18@\textsc{Reinhard, Marie} (1871-03-13 – 1899-03-18), \emph{Gesangspädagoge/Gesangspädagogin}|pwv}, wegen Kind\pwindex{?? [Totgeborener Sohn von Arthur Schnitzler und Marie Reinhard] 1897-09-24 – 1897-09-24@\textsc{?? [Totgeborener Sohn von Arthur Schnitzler und Marie Reinhard]} (1897-09-24 – 1897-09-24)|pwv} etc. nach Paris\oindex{Paris@\textbf{Paris}, \emph{P.PPLC}|pw} gereist«.}}}\label{K_L02814-8} iſt widerwärtig. Oh dieſe
               iſraelitiſchen Jungfrauen! {\dotsfour}\pend
           
\pstart
           Ich ſchlafe ſchlecht, bin unzufrieden und mißmuthig{\dotsfive}\pend
           
\pstart
           Könnteſt Du nicht am 9. oder 11. Auguſt zum \label{K_L02814-9v}\edtext{\textsc{Parsifal\pwindex{Parsifal@\emph{Parsifal}|pw}} nach \textsc{Bayreuth\oindex{Bayreuth@\textbf{Bayreuth}, \emph{P.PPLA2}|pw}\orgindex{Bayreuther Festspiele@Bayreuther Festspiele|pwv}}}{\lemma{\textnormal{\emph{Parsifal nach Bayreuth}}}\Cendnote{\textnormal{Dazu kam es nicht. Die \emph{Bayreuther Festspiele}\orgindex{Bayreuther Festspiele@Bayreuther Festspiele|pwk} wurden 1897
                  von Cosima Wagner\pwindex{Wagner, Cosima 25.12.1837 – 01.04.1930@\textsc{Wagner, Cosima} (25.12.1837 – 01.04.1930)|pwk} geleitet.}}}\label{K_L02814-9}
               kommen?\pend
           
\pstart
           Grüße Deine Freundin\pwindex{Reinhard, Marie 1871-03-13 – 1899-03-18@\textsc{Reinhard, Marie} (1871-03-13 – 1899-03-18), \emph{Gesangspädagoge/Gesangspädagogin}|pwv}
               recht herzlich und ſei Du ſelbſt vielmals gegrüßt von {\\[\baselineskip]}Deinem treuen \spacefill\mbox{Paul
                  Goldm}\pend
           \leftskip=0em{}\selectlanguage{ngerman}\endnumbering\briefempfaengerindex{Schnitzler, Arthur@\textsc{Schnitzler, Arthur}!zzzGoldmann, Paul@\emph{von Paul Goldmann}!1897-06-152@{15. 6. {[}1897{]}}|)be}\mylabel{L02814h}  \normalsize

\doendnotes{C}
\bigskip
\vfill

\clearpage

\footnotesize

\lohead{\textsc{register}}

% Definiere theindex-Environment komplett neu ohne reledmac
\makeatletter
\renewenvironment{theindex}{%
  \section*{\indexname}%
  \setlength{\parindent}{0pt}%
  \setlength{\parskip}{0pt plus 0.3pt}%
  \let\item\@idxitem
}{%
  \clearpage
}
\makeatother

\IfFileExists{\jobname-pw.ind}{\input{\jobname-pw.ind}}{}

\end{document}

      