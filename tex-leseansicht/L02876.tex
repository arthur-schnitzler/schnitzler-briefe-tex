%% latex-leseansicht-vorspann.tex
%% Vorspann für die Leseansicht.
%% Lädt die gemeinsame Datei latex-vorspann.tex mit nicht gesetztem Schalter.

\newif\ifkorrekturansicht
\korrekturansichtfalse

\input{../tex-inputs/latex-vorspann}


         
         \renewcommand{\erwaehntePersonen}{Personen: Hermann Bahr, Julius Bauer, Richard Beer-Hofmann, Stephan Epstein, Theodor Herzl, Hugo von Hofmannsthal, Karl Kraus, Émile Lutz, Marie Reinhard, Theodore Rottenberg, Gustav Schwarzkopf, Émile Soutif, Jean Thorel}
         \renewcommand{\erwaehnteInstitutionen}{Institutionen: Frankfurter Zeitung, Neue Freie Presse, Rütten {\kaufmannsund}  Loening, S. Fischer Verlag}
         \renewcommand{\erwaehnteOrte}{Orte: Bayreuth, Berlin, Dunkirk, Frankfurt am Main, Paris, Théâtre Antoine-Simone Berriau, Wien}
         \renewcommand{\erwaehnteWerke}{Werke: Amourette. Pièce en trois actes. Adaptée de Arthur Schnitzler, Au Perroquet Vert, Der grüne Kakadu – Paracelsus – Die Gefährtin. Drei Einakter, Der grüne Kakadu. Groteske in einem Akt, Die Fackel, Die Gefährtin. Schauspiel in einem Akt, Ein Sommer in China. Reisebilder, Le Perroquet Vert, Liebelei. Schauspiel in drei Akten, Paracelsus. Versspiel in einem Akt}
               \section[ Paul Goldmann an Arthur Schnitzler, 28. 5. {[}1899{]}]{ Paul Goldmann an Arthur Schnitzler, 28. 5. {[}1899{]}}\nopagebreak\mylabel{v}\rehead{ }\begin{ledgroupsized}[t]{13cm}\normalsize\beginnumbering \toendnotes[C]{\smallbreak\pagebreak[2]} \Standort{DLA, A:Schnitzler, HS.NZ85.1.3169.}
\physDesc{Brief, 1 Blatt, 3 Seiten
\newline{}Handschrift: blaue Tinte, deutsche Kurrent
\newline{}Schnitzler: 1) mit Bleistift das Jahr »99« vermerkt  2) mit rotem Buntstift drei Unterstreichungen}\toendnotes[C]{\smallbreak}\pstart
           \raggedleft{}{\pb}Frankfurt\oindex{Frankfurt am Main@\textbf{Frankfurt am Main}|pw}, 28. Mai.\pend
           \pstart\center{}Mein lieber Freund,\pend\pstart
           Wieder habe ich den Sonntag abwarten müſſen, um eine
               freie Stunde für einen Brief an Dich zu finden.\pend
           \pstart
           Ich danke Dir von Herzen für Deine letzten lieben Briefe, ſowie für die Überſendung
               des \label{K_L02876-1v}\edtext{»Grünen Kakadu\pwindex{Schnitzler, Arthur 15.05.1862 – 21.10.1931@\textsc{Schnitzler, Arthur} (15.05.1862 – 21.10.1931), \emph{Schriftsteller, Mediziner}!gruene Kakadu – Paracelsus – Die Gefaehrtin. Drei Einakter1898 – 1899@\strich\emph{Der grüne Kakadu – Paracelsus – Die Gefährtin. Drei Einakter} {[}1898 – 1899{]}|pw}«}{\lemma{\textnormal{\emph{»Grünen Kakadu«}}}\Cendnote{\textnormal{Die Buchausgabe
                  des Einakterzyklus\pwindex{Schnitzler, Arthur 15.05.1862 – 21.10.1931@\textsc{Schnitzler, Arthur} (15.05.1862 – 21.10.1931), \emph{Schriftsteller, Mediziner}!gruene Kakadu – Paracelsus – Die Gefaehrtin. Drei Einakter1898 – 1899@\strich\emph{Der grüne Kakadu – Paracelsus – Die Gefährtin. Drei Einakter} {[}1898 – 1899{]}|pwkv}’ (\emph{Der grüne Kakadu}\pwindex{Schnitzler, Arthur 15.05.1862 – 21.10.1931@\textsc{Schnitzler, Arthur} (15.05.1862 – 21.10.1931), \emph{Schriftsteller, Mediziner}!gruene Kakadu. Groteske in einem Akt1. 3. 1899@\strich\emph{Der grüne Kakadu. Groteske in einem Akt} {[}1. 3. 1899{]}|pwk}, \emph{Paracelsus}\pwindex{Schnitzler, Arthur 15.05.1862 – 21.10.1931@\textsc{Schnitzler, Arthur} (15.05.1862 – 21.10.1931), \emph{Schriftsteller, Mediziner}!Paracelsus. Versspiel in einem Akt01. 11. 1898@\strich\emph{Paracelsus. Versspiel in einem Akt} {[}01. 11. 1898{]}|pwk}, \emph{Die
                     Gefährtin}\pwindex{Schnitzler, Arthur 15.05.1862 – 21.10.1931@\textsc{Schnitzler, Arthur} (15.05.1862 – 21.10.1931), \emph{Schriftsteller, Mediziner}!Gefaehrtin. Schauspiel in einem Akt1899-03-01@\strich\emph{Die Gefährtin. Schauspiel in einem Akt} {[}1899-03-01{]}|pwk}) erschien am 29. 4. 1899 bei \emph{S. Fischer}\orgindex{S. Fischer Verlag@S. Fischer Verlag|pwk} (Berlin\oindex{Berlin@\textbf{Berlin}|pwk}).}}}\label{K_L02876-1h} (das Exemplar iſt vornehm und geſchmackvoll ausgeſtaltet)
               und für die liebe Widmung, die das Titelblatt ziert.\pend
           \pstart
           Deine letzten Briefe ſind, Gott ſei Dank, doch ſchon etwas ruhiger, ſo ſehr es auch
               noch in Dir \label{K_L02876-2v}\edtext{wühlt}{\lemma{\textnormal{\emph{wühlt}}}\Cendnote{\textnormal{der Tod von Marie Reinhard\pwindex{Reinhard, Marie 1871-03-13 – 1899-03-18@\textsc{Reinhard, Marie} (1871-03-13 – 1899-03-18), \emph{Gesangspädagogin}|pwk} am 18. 3. 1899}}}\label{K_L02876-2h}. Ich habe nur den dringenden Wunſch, Dich endlich auch einmal zu ſehen und zu
               ſprechen. Sommerpläne freilich kann ich in dieſem Jahre gar nicht machen. Am 15. Juli ſoll ich für die Zeitung\orgindex{Frankfurter Zeitung@Frankfurter Zeitung|pwv} nach Bayreuth\oindex{Bayreuth@\textbf{Bayreuth}|pw}, dann nach \textsc{Paris\oindex{Paris@\textbf{Paris}|pw}}, um über die Vorarbeiten zur \label{K_L02876-3v}\edtext{Weltausſtellung}{\lemma{\textnormal{\emph{Weltausſtellung}}}\Cendnote{\textnormal{Die Weltausstellung
                  in Paris\oindex{Paris@\textbf{Paris}|pwk} fand von 15. 4. 1900 bis 12. 11. 1900
                  statt.}}}\label{K_L02876-3h} zu berichten. Ich fürchte, mein ganzer Urlaub geht zum Teufel.
               Immerhin mußt Du mich ſtets auf dem Laufenden halten, wo Du biſt; vielleicht kann ich
               doch noch einmal raſch irgendwohin kommen, wo Du \substVorne{}\textsuperscript{\textcolor{gray}{×}\-\textcolor{gray}{×}\-\textcolor{gray}{×}\-\textcolor{gray}{×}}\substDazwischen{}Dich aufhältſt.\substHinten{} Und wenn Du \label{K_L02876-5v}\edtext{im September nach Frankfurt\oindex{Frankfurt am Main@\textbf{Frankfurt am Main}|pw}}{\lemma{\textnormal{\emph{im … Frankfurt}}}\Cendnote{\textnormal{Schnitzler\pwindex{Schnitzler, Arthur 15.05.1862 – 21.10.1931@\textsc{Schnitzler, Arthur} (15.05.1862 – 21.10.1931), \emph{Schriftsteller, Mediziner}|pwk} war von 19. 9. 1899 bis 23. 9. 1899 in Frankfurt am Main\oindex{Frankfurt am Main@\textbf{Frankfurt am Main}|pwk}.}}}\label{K_L02876-5h} kommſt, bin ich
               jedenfalls da.\pend
           \pstart
           \label{K_L02876-6v}\edtext{Affaire \textsc{Thorel\pwindex{Thorel, Jean 1859-09-11 – 1916-08-20@\textsc{Thorel, Jean} (1859-09-11 – 1916-08-20), \emph{Übersetzer, Dramatiker}!Amourette. Piece en trois actes. Adaptee de Arthur Schnitzler1897@\strich\emph{Amourette. Pièce en trois actes. Adaptée de Arthur Schnitzler} {[}Übersetzung, 1897{]}|pw}\pwindex{Thorel, Jean 1859-09-11 – 1916-08-20@\textsc{Thorel, Jean} (1859-09-11 – 1916-08-20), \emph{Übersetzer, Dramatiker}|pw}}}{\lemma{\textnormal{\emph{Affaire Thorel}}}\Cendnote{\textnormal{Gemeint war die von Jean Thorel\pwindex{Thorel, Jean 1859-09-11 – 1916-08-20@\textsc{Thorel, Jean} (1859-09-11 – 1916-08-20), \emph{Übersetzer, Dramatiker}|pwk} in den Jahren 1896 und
                     1897 angefertigte französische Übersetzung\pwindex{Thorel, Jean 1859-09-11 – 1916-08-20@\textsc{Thorel, Jean} (1859-09-11 – 1916-08-20), \emph{Übersetzer, Dramatiker}!Amourette. Piece en trois actes. Adaptee de Arthur Schnitzler1897@\strich\emph{Amourette. Pièce en trois actes. Adaptée de Arthur Schnitzler} {[}Übersetzung, 1897{]}|pwkv} der \emph{Liebelei}\pwindex{Schnitzler, Arthur 15.05.1862 – 21.10.1931@\textsc{Schnitzler, Arthur} (15.05.1862 – 21.10.1931), \emph{Schriftsteller, Mediziner}!Liebelei. Schauspiel in drei Akten1895-10-09@\strich\emph{Liebelei. Schauspiel in drei Akten} {[}1895-10-09{]}|pwk} (\begin{otherlanguage}{french}\emph{Amourette. Pièce en trois actes}\pwindex{Thorel, Jean 1859-09-11 – 1916-08-20@\textsc{Thorel, Jean} (1859-09-11 – 1916-08-20), \emph{Übersetzer, Dramatiker}!Amourette. Piece en trois actes. Adaptee de Arthur Schnitzler1897@\strich\emph{Amourette. Pièce en trois actes. Adaptée de Arthur Schnitzler} {[}Übersetzung, 1897{]}|pwk}\end{otherlanguage}), die jedoch unveröffentlicht blieb.}}}\label{K_L02876-6h}. Ich habe keine Ahnung mehr
               von den getroffenen \label{K_L02876-56v}\edtext{Abmachungen}{\lemma{\textnormal{\emph{Abmachungen}}}\Cendnote{\textnormal{siehe Paul Goldmann an Arthur Schnitzler, 22. 9. [1896]}}}\label{K_L02876-56h}. Jedenfalls haſt Du zum Mindeſten Anſpruch auf die \uline{Hälfte} des Honorars, {\pb}da Du ihm\pwindex{Thorel, Jean 1859-09-11 – 1916-08-20@\textsc{Thorel, Jean} (1859-09-11 – 1916-08-20), \emph{Übersetzer, Dramatiker}|pwv} ja ſein ganzes Honorar,
               das \strikeout{es} aus den \begin{otherlanguage}{french}\textsc{Tantièmen}\end{otherlanguage}{ }der \label{K_L02876-7v}\edtext{Aufführungen\pwindex{Thorel, Jean 1859-09-11 – 1916-08-20@\textsc{Thorel, Jean} (1859-09-11 – 1916-08-20), \emph{Übersetzer, Dramatiker}!Amourette. Piece en trois actes. Adaptee de Arthur Schnitzler1897@\strich\emph{Amourette. Pièce en trois actes. Adaptée de Arthur Schnitzler} {[}Übersetzung, 1897{]}|pwv}}{\lemma{\textnormal{\emph{Aufführungen}}}\Cendnote{\textnormal{Abgesehen von einer Aufführung am 29. 8. 1902 in Dunkerque\oindex{Dunkirk@\textbf{Dunkirk}|pwk} sind keine Vorstellungen der \emph{Liebelei}\pwindex{Schnitzler, Arthur 15.05.1862 – 21.10.1931@\textsc{Schnitzler, Arthur} (15.05.1862 – 21.10.1931), \emph{Schriftsteller, Mediziner}!Liebelei. Schauspiel in drei Akten1895-10-09@\strich\emph{Liebelei. Schauspiel in drei Akten} {[}1895-10-09{]}|pwk} nach Thorel\pwindex{Thorel, Jean 1859-09-11 – 1916-08-20@\textsc{Thorel, Jean} (1859-09-11 – 1916-08-20), \emph{Übersetzer, Dramatiker}|pwk}s Übersetzung\pwindex{Thorel, Jean 1859-09-11 – 1916-08-20@\textsc{Thorel, Jean} (1859-09-11 – 1916-08-20), \emph{Übersetzer, Dramatiker}!Amourette. Piece en trois actes. Adaptee de Arthur Schnitzler1897@\strich\emph{Amourette. Pièce en trois actes. Adaptée de Arthur Schnitzler} {[}Übersetzung, 1897{]}|pwkv} bekannt.}}}\label{K_L02876-7h}
               beſtritten werden ſollte, als \label{K_L02876-8v}\edtext{Vorſchuß}{\lemma{\textnormal{\emph{Vorſchuß}}}\Cendnote{\textnormal{in der Höhe
                  von 500 Francs}}}\label{K_L02876-8h} gezahlt haſt. Auch den »Kakadu\pwindex{Schnitzler, Arthur 15.05.1862 – 21.10.1931@\textsc{Schnitzler, Arthur} (15.05.1862 – 21.10.1931), \emph{Schriftsteller, Mediziner}!gruene Kakadu. Groteske in einem Akt1. 3. 1899@\strich\emph{Der grüne Kakadu. Groteske in einem Akt} {[}1. 3. 1899{]}|pw}« ſollteſt Du ihm \label{K_L02876-9v}\edtext{zu
               überſetzen geben}{\lemma{\textnormal{\emph{zu
               überſetzen geben}}}\Cendnote{\textnormal{\emph{Der grüne Kakadu}\pwindex{Schnitzler, Arthur 15.05.1862 – 21.10.1931@\textsc{Schnitzler, Arthur} (15.05.1862 – 21.10.1931), \emph{Schriftsteller, Mediziner}!gruene Kakadu. Groteske in einem Akt1. 3. 1899@\strich\emph{Der grüne Kakadu. Groteske in einem Akt} {[}1. 3. 1899{]}|pwk} wurde zuerst von Émile Soutif\pwindex{Soutif, Emile @\textsc{Soutif, Émile}, \emph{Lehrer}|pwk} (Übersetzung\pwindex{Schnitzler, Arthur 15.05.1862 – 21.10.1931@\textsc{Schnitzler, Arthur} (15.05.1862 – 21.10.1931), \emph{Schriftsteller, Mediziner}!Le Perroquet VertNone@\strich\emph{Le Perroquet Vert} {[}None{]}|pwkv} nicht überliefert, siehe Arthur Schnitzler an Georg Brandes, 8. 6. 1899) und dann als \emph{Au Perroquet vert}\pwindex{Schnitzler, Arthur 15.05.1862 – 21.10.1931@\textsc{Schnitzler, Arthur} (15.05.1862 – 21.10.1931), \emph{Schriftsteller, Mediziner}!Au Perroquet Vert1903-11-07@\strich\emph{Au Perroquet Vert} {[}1903-11-07{]}|pwk} von Stephan Epstein\pwindex{Epstein, Stephan 12.11.1866 – 1941@\textsc{Epstein, Stephan} (12.11.1866 – 1941), \emph{Schriftsteller, Dramaturg, Übersetzer}|pwk} und Émile
                     Lutz\pwindex{Lutz, Emile 1868-04-08 – 1940-01-18@\textsc{Lutz, Émile} (1868-04-08 – 1940-01-18), \emph{Übersetzer, Dichter}|pwk} ins Französische übersetzt. Die spätere Übersetzung\pwindex{Schnitzler, Arthur 15.05.1862 – 21.10.1931@\textsc{Schnitzler, Arthur} (15.05.1862 – 21.10.1931), \emph{Schriftsteller, Mediziner}!Au Perroquet Vert1903-11-07@\strich\emph{Au Perroquet Vert} {[}1903-11-07{]}|pwkv} war die Grundlage für zwölf
                  Aufführungen zwischen 7. 11. 1903 und 6. 12. 1903 im Théâtre
                     Antoine\oindex{Theâtre Antoine-Simone Berriau@\textbf{Théâtre Antoine-Simone Berriau}|pwk}.}}}\label{K_L02876-9h}. Er iſt als Überſetzer ſo ſchlecht, wie alle Andern, hat
               aber doch wenigſtens Verbindungen{\dotsfive}\pend
           \pstart
           Ich erlebe nichts, was mich \label{K_L02876-88v}\edtext{glücklich
               und unglücklich}{\lemma{\textnormal{\emph{glücklich und unglücklich}}}\Cendnote{\textnormal{Eventuell wird hier
                  neuerlich (vgl. Paul Goldmann an Arthur Schnitzler, 20. 5. [1899]) die
                  Frühphase der intimen Beziehung mit der verheirateten Theodore Rottenberg\pwindex{Rottenberg, Theodore 1875-09-07 – 1945-04-05@\textsc{Rottenberg, Theodore} (1875-09-07 – 1945-04-05)|pwk} etwas kryptisch beschrieben, vgl. Paul Goldmann an Arthur Schnitzler, 8. 10. [1899]}}}\label{K_L02876-88h} zugleich macht, ſondern: Es würde ein großes Glück ſein, aber ich kann es
               nicht erleben. Siehſt Du: Verlieren, durch das Schickſal verlieren, wie es Dein Loos
               war, iſt furchtbar. Aber nicht \strikeout{\textcolor{gray}{×}\-\textcolor{gray}{×}\-\textcolor{gray}{×}\-\textcolor{gray}{×}} beſitzen können, \uline{durch eigene Schuld} nicht
               beſitzen können, iſt entſetzlich, und zudem wird man ſich ſelbſt verächtlich und zum
               Ekel. Das läßt ſich Alles nicht \substVorne{}\textsuperscript{\textcolor{gray}{×}\-\textcolor{gray}{×}\-\textcolor{gray}{×}\-\textcolor{gray}{×}\-\textcolor{gray}{×}}\substDazwischen{}ſchreiben\substHinten{}; ich ſehne mich danach, es Dir zu erzählen{\dotsfour}\pend
           \pstart
           Bitte, ſchreib’ mir bald wieder, wie es Dir geht. Theile mir auch freundlichſt die
               Adreſſe des Herrn \textsc{von Hoffmannsthal\pwindex{Hofmannsthal, Hugo von 1874-02-01 – 1929-07-15@\textsc{Hofmannsthal, Hugo von} (1874-02-01 – 1929-07-15), \emph{Schriftsteller}|pw}} mit, dem ich \label{K_L02876-11v}\edtext{mein Buch\pwindex{Goldmann, Paul 31.01.1865 – 25.09.1935@\textsc{Goldmann, Paul} (31.01.1865 – 25.09.1935), \emph{Schriftsteller, Journalist}!Sommer in China. Reisebilder1899-05-02@\strich\emph{Ein Sommer in China. Reisebilder} {[}1899-05-02{]}|pwv}}{\lemma{\textnormal{\emph{mein Buch}}}\Cendnote{\textnormal{über seine Asienreise 1898; Paul Goldmann\pwindex{Goldmann, Paul 31.01.1865 – 25.09.1935@\textsc{Goldmann, Paul} (31.01.1865 – 25.09.1935), \emph{Schriftsteller, Journalist}|pwk}: \emph{Ein Sommer in China. Reisebilder}\pwindex{Goldmann, Paul 31.01.1865 – 25.09.1935@\textsc{Goldmann, Paul} (31.01.1865 – 25.09.1935), \emph{Schriftsteller, Journalist}!Sommer in China. Reisebilder1899-05-02@\strich\emph{Ein Sommer in China. Reisebilder} {[}1899-05-02{]}|pwk}. 2 Bde. Frankfurt am Main\oindex{Frankfurt am Main@\textbf{Frankfurt am Main}|pwk}: \emph{Literarische Anstalt Rütten {\kaufmannsund}
                     Loening}\orgindex{Ruetten und Loening@Rütten {\kaufmannsund}  Loening|pwk}{ }1899, erschienen Anfang Mai 1899.}}}\label{K_L02876-11h} ſchicken möchte. Was macht \textsc{Richard\pwindex{Beer-Hofmann, Richard 1866-07-11 – 1945-09-26@\textsc{Beer-Hofmann, Richard} (1866-07-11 – 1945-09-26), \emph{Schriftsteller}|pw}}? Ich höre natürlich kein Wort von ihm.\pend
           \pstart
           Was ſagt Ihr zur »Fackel\pwindex{Fackel1899-04 – 1936@\emph{Die Fackel} {[}1899-04 – 1936{]}|pw}«? Der Burſch\pwindex{Kraus, Karl 28.04.1874 – 12.06.1936@\textsc{Kraus, Karl} (28.04.1874 – 12.06.1936), \emph{Schriftsteller, Publizist}|pwv} hat Talent. Schade nur, daß er ein
               ſolcher Lausbub iſt. Denn das \label{K_L02876-12v}\edtext{Ausmiſtungs-Werk}{\lemma{\textnormal{\emph{Ausmiſtungs-Werk}}}\Cendnote{\textnormal{Anspielung auf Karl Kraus\pwindex{Kraus, Karl 28.04.1874 – 12.06.1936@\textsc{Kraus, Karl} (28.04.1874 – 12.06.1936), \emph{Schriftsteller, Publizist}|pwk}’ umfassende polemische Kritik in
                  der damals neu erscheinenden \emph{Fackel}\pwindex{Fackel1899-04 – 1936@\emph{Die Fackel} {[}1899-04 – 1936{]}|pwk}}}}\label{K_L02876-12h}, das er unternimmt, iſt verdienlich. Er ſagt treffliche Worte gegen \textsc{Bauer\pwindex{Bauer, Julius 15.10.1853 – 11.06.1941@\textsc{Bauer, Julius} (15.10.1853 – 11.06.1941), \emph{Schriftsteller, Journalist, Kritiker}|pw}}, \textsc{Herzl\pwindex{Herzl, Theodor 1860-05-02 – 1904-07-03@\textsc{Herzl, Theodor} (1860-05-02 – 1904-07-03), \emph{Schriftsteller, Journalist}|pw}}, \textsc{Bahr\pwindex{Bahr, Hermann 19.07.1863 – 15.01.1934@\textsc{Bahr, Hermann} (19.07.1863 – 15.01.1934), \emph{Schriftsteller, Kritiker}|pw}}, namentlich gegen die »Neue Freie Preſſe\orgindex{Neue Freie Presse@Neue Freie Presse|pw}«,
               und es iſt das Traurige an den jetzigen \strikeout{Wien\oindex{Wien@\textbf{Wien}|pw}er}{ }{\pb}Wien\oindex{Wien@\textbf{Wien}|pw}er Verhältniſſen, daß, wenn endlich einmal
               Jemand kommt, der gegen die Corruption kämpft, er ebenſo corrupt iſt, wie die
               Corruption ſelbſt.\pend
           \pstart
           Grüß’ mir \textsc{Schwarzkopf\pwindex{Schwarzkopf, Gustav 07.11.1853 – 13.11.1939@\textsc{Schwarzkopf, Gustav} (07.11.1853 – 13.11.1939), \emph{Schriftsteller}|pw}}, mit dem Du ja jetzt häufiger zuſammen biſt.\pend
           \pstart
           Ich grüße Dich von Herzen {\\[\baselineskip]}Dein treuer {\\[\baselineskip]}\spacefill\mbox{Paul Goldmann.}\pend
           \leftskip=0em{}
         
         \endnumbering\mylabel{h}\end{ledgroupsized}  \newcommand{\dateiname}{L02876}\newcommand{\titel}{Paul Goldmann an Arthur Schnitzler, 28. 5. [1899]}\newcommand{\editorInnen}{Martin Anton Müller und Laura Untner}%% latex-leseansicht-abspann.tex
%% Abspann für die Leseansicht.
%% Der Schalter \ifkorrekturansicht ist bereits durch den Vorspann gesetzt.

%% latex-abspann.tex
%% Gemeinsamer Abspann für Korrekturansicht und Leseansicht.
%% Setzt den Schalter \ifkorrekturansicht voraus (gesetzt in den
%% einbindenden Dateien latex-korrekturansicht-abspann.tex bzw.
%% latex-leseansicht-abspann.tex).
%% ---------------------------------------------------------------

\normalsize

% Das esempio-Environment wird nur in der Leseansicht benötigt
\ifkorrekturansicht\else
\newenvironment{esempio}[3]%
{
    \vspace{1.5ex}
    \rlap{\underline{#1}}
    \par
    \setlength{\parindent}{0cm}
    \nopagebreak
    \leftskip=#2cm
    \rightskip=#3cm
}
{
    \par
}
\fi

\doendnotes{C}
\bigskip
\vfill

\clearpage

\footnotesize

\ifkorrekturansicht
  \lohead{\textsc{register}}
\fi

% theindex-Environment neu definieren ohne reledmac
\makeatletter
\renewenvironment{theindex}{%
  \ifkorrekturansicht
    \section*{\indexname}%
  \else
    \subsubsection*{Index der erwähnten Entitäten}%
  \fi
  \setlength{\parindent}{0pt}%
  \setlength{\parskip}{0pt plus 0.3pt}%
  \let\item\@idxitem
}{%
  \ifkorrekturansicht\clearpage\fi
}
\makeatother

\IfFileExists{\jobname-pw.ind}{\input{\jobname-pw.ind}}{}

% Quellenangabe nur in der Leseansicht
\ifkorrekturansicht\else
% Fallback-Definitionen, falls die .tex-Datei \titel etc. nicht gesetzt hat
\providecommand{\titel}{}
\providecommand{\editorInnen}{}
\providecommand{\dateiname}{\jobname}

\vspace{3cm}

\vfill

\footnotesize
\textsc{Quelle}: \titel. Herausgegeben von {\editorInnen}. In: \emph{Arthur Schnitzler: Briefwechsel mit Autorinnen und Autoren}.
 Digitale Edition, https://schnitzler-briefe.acdh.oeaw.ac.at/{\dateiname}.html (Stand \today)
\fi

\end{document}


      