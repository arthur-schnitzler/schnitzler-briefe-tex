%% latex-leseansicht-vorspann.tex
%% Vorspann für die Leseansicht.
%% Lädt die gemeinsame Datei latex-vorspann.tex mit nicht gesetztem Schalter.

\newif\ifkorrekturansicht
\korrekturansichtfalse

\input{../tex-inputs/latex-vorspann}


\section[ Paul Goldmann an Arthur Schnitzler, 28. 5. {[}1899{]}]{L02876 Paul Goldmann an Arthur Schnitzler,  28. 5. [1899]}
\nopagebreak\mylabel{L02876v}
\rehead{ }\normalsize\beginnumbering\briefempfaengerindex{Schnitzler, Arthur@\textsc{Schnitzler, Arthur}!zzzGoldmann, Paul@\emph{von Paul Goldmann}!1899-05-281@{28. 5. [1899]}|(be}
\toendnotes[C]{\smallbreak\pagebreak[2]}
\correspDesc{Versand  durch Paul Goldmann am 28. 5. [1899] in Frankfurt am Main
\newline{}Erhalt  durch Arthur Schnitzler im Zeitraum [29. 5. 1899
                  – 2. 6. 1899?] in Wien}\toendnotes[C]{\smallbreak}
\Standort{DLA, A:Schnitzler, HS.NZ85.1.3169.}
\physDesc{Brief, 1 Blatt, 3 Seiten, 2603 Zeichen
\newline{}Handschrift: blaue Tinte, deutsche Kurrent
\newline{}Schnitzler: 1) mit Bleistift das Jahr »99« vermerkt  2) mit rotem Buntstift drei Unterstreichungen}\toendnotes[C]{\smallbreak}
\pstart
           \raggedleft{}{\pb}Frankfurt\oindex{Frankfurt am Main@\textbf{Frankfurt am Main}, \emph{Hauptstadt}|pw}, 28. Mai.\pend
           
\pstart\center{}Mein lieber Freund,\pend\vspace{0.5em}
\pstart
           Wieder habe ich den Sonntag abwarten müſſen, um eine
               freie Stunde für einen Brief an Dich zu finden.\pend
           
\pstart
           Ich danke Dir von Herzen für Deine letzten lieben Briefe,{ }ſowie für die Überſendung
               des \label{K_L02876-1v}\edtext{»Grünen Kakadu\pwindex{Schnitzler, Arthur 15.\,5.\,1862 Wien – 21.\,10.\,1931 ebd.@\textsc{Schnitzler, Arthur} (15.\,5.\,1862 Wien – 21.\,10.\,1931 ebd.), \emph{Schriftsteller, Mediziner}!grüne Kakadu – Paracelsus – Die Gefährtin. Drei Einakter@\strich\emph{Der grüne Kakadu – Paracelsus – Die Gefährtin. Drei Einakter}|pw}«}{\lemma{\textnormal{\emph{»Grünen Kakadu«}}}\Cendnote{\textnormal{Die Buchausgabe
                  des Einakterzyklus\pwindex{Schnitzler, Arthur 15.\,5.\,1862 Wien – 21.\,10.\,1931 ebd.@\textsc{Schnitzler, Arthur} (15.\,5.\,1862 Wien – 21.\,10.\,1931 ebd.), \emph{Schriftsteller, Mediziner}!grüne Kakadu – Paracelsus – Die Gefährtin. Drei Einakter@\strich\emph{Der grüne Kakadu – Paracelsus – Die Gefährtin. Drei Einakter}|pwkv}’ (\emph{Der grüne Kakadu}\pwindex{Schnitzler, Arthur 15.\,5.\,1862 Wien – 21.\,10.\,1931 ebd.@\textsc{Schnitzler, Arthur} (15.\,5.\,1862 Wien – 21.\,10.\,1931 ebd.), \emph{Schriftsteller, Mediziner}!grüne Kakadu. Groteske in einem Akt@\strich\emph{Der grüne Kakadu. Groteske in einem Akt}|pwk}, \emph{Paracelsus}\pwindex{Schnitzler, Arthur 15.\,5.\,1862 Wien – 21.\,10.\,1931 ebd.@\textsc{Schnitzler, Arthur} (15.\,5.\,1862 Wien – 21.\,10.\,1931 ebd.), \emph{Schriftsteller, Mediziner}!Paracelsus. Versspiel in einem Akt@\strich\emph{Paracelsus. Versspiel in einem Akt}|pwk}, \emph{Die
                     Gefährtin}\pwindex{Schnitzler, Arthur 15.\,5.\,1862 Wien – 21.\,10.\,1931 ebd.@\textsc{Schnitzler, Arthur} (15.\,5.\,1862 Wien – 21.\,10.\,1931 ebd.), \emph{Schriftsteller, Mediziner}!Gefährtin. Schauspiel in einem Akt@\strich\emph{Die Gefährtin. Schauspiel in einem Akt}|pwk}) war am 29. 4. 1899 bei \emph{S. Fischer}\orgindex{S. Fischer Verlag@S. Fischer Verlag|pwk} in Berlin\oindex{Berlin@\textbf{Berlin}, \emph{Hauptstadt}|pwk} erschienen.}}}\label{K_L02876-1} (das Exemplar iſt vornehm und geſchmackvoll ausgeſtaltet)
               und für die liebe Widmung, die das Titelblatt ziert.\pend
           
\pstart
           Deine letzten Briefe{ }ſind, Gott{ }ſei Dank, doch{ }ſchon etwas ruhiger,{ }ſo{ }ſehr es auch
               noch in Dir \label{K_L02876-2v}\edtext{wühlt}{\lemma{\textnormal{\emph{wühlt}}}\Cendnote{\textnormal{der Tod von Marie Reinhard\pwindex{Reinhard, Marie 13.\,3.\,1871 Wien – 18.\,3.\,1899 ebd.@\textsc{Reinhard, Marie} (13.\,3.\,1871 Wien – 18.\,3.\,1899 ebd.), \emph{Gesangspädagogin}|pwk} am 18. 3. 1899}}}\label{K_L02876-2}. Ich habe nur den dringenden Wunſch, Dich endlich auch einmal zu{ }ſehen und zu{ }ſprechen. Sommerpläne freilich kann ich in dieſem Jahre gar nicht machen. Am 15. Juli{ }ſoll ich für die Zeitung\orgindex{Frankfurter Zeitung@Frankfurter Zeitung|pwv} nach Bayreuth\oindex{Bayreuth@\textbf{Bayreuth}, \emph{Hauptstadt}|pw}, dann nach \textsc{Paris\oindex{Paris@\textbf{Paris}, \emph{Hauptstadt}|pw}}, um über die Vorarbeiten zur \label{K_L02876-3v}\edtext{Weltausſtellung}{\lemma{\textnormal{\emph{Weltausstellung}}}\Cendnote{\textnormal{Die Weltausstellung
                  in Paris\oindex{Paris@\textbf{Paris}, \emph{Hauptstadt}|pwk} fand von 15. 4. 1900 bis 12. 11. 1900
                  statt.}}}\label{K_L02876-3} zu berichten. Ich fürchte, mein ganzer Urlaub geht zum Teufel.
               Immerhin mußt Du mich{ }ſtets auf dem Laufenden halten, wo Du biſt; vielleicht kann ich
               doch noch einmal raſch irgendwohin kommen, wo Du \substVorne{}\textsuperscript{\textcolor{gray}{×}\-\textcolor{gray}{×}\-\textcolor{gray}{×}\-\textcolor{gray}{×}}\substDazwischen{}Dich aufhältſt.\substHinten{} Und wenn Du \label{K_L02876-4v}\edtext{im September nach Frankfurt\oindex{Frankfurt am Main@\textbf{Frankfurt am Main}, \emph{Hauptstadt}|pw}}{\lemma{\textnormal{\emph{im … Frankfurt}}}\Cendnote{\textnormal{Schnitzler war vom 19. 9. 1899 bis zum 23. 9. 1899 in Frankfurt am Main\oindex{Frankfurt am Main@\textbf{Frankfurt am Main}, \emph{Hauptstadt}|pwk}.}}}\label{K_L02876-4} kommſt, bin ich
               jedenfalls da.\pend
           
\pstart
           \label{K_L02876-5v}\edtext{Affaire \textsc{Thorel\pwindex{Schnitzler, Arthur 15.\,5.\,1862 Wien – 21.\,10.\,1931 ebd.@\textsc{Schnitzler, Arthur} (15.\,5.\,1862 Wien – 21.\,10.\,1931 ebd.), \emph{Schriftsteller, Mediziner}!Amourette. Pièce en trois actes. Adaptée de Arthur Schnitzler@\strich\emph{Amourette. Pièce en trois actes. Adaptée de Arthur Schnitzler}|pw}\pwindex{Thorel, Jean 11.\,9.\,1859 Éragny – 20.\,8.\,1916 Enghien-les-Bains@\textsc{Thorel, Jean} (11.\,9.\,1859 Éragny – 20.\,8.\,1916 Enghien-les-Bains), \emph{Übersetzer, Dramatiker}|pw}}}{\lemma{\textnormal{\emph{Affaire Thorel}}}\Cendnote{\textnormal{Gemeint war die von Jean Thorel\pwindex{Thorel, Jean 11.\,9.\,1859 Éragny – 20.\,8.\,1916 Enghien-les-Bains@\textsc{Thorel, Jean} (11.\,9.\,1859 Éragny – 20.\,8.\,1916 Enghien-les-Bains), \emph{Übersetzer, Dramatiker}|pwk} in den Jahren 1896 und
                     1897 angefertigte französische Übersetzung\pwindex{Schnitzler, Arthur 15.\,5.\,1862 Wien – 21.\,10.\,1931 ebd.@\textsc{Schnitzler, Arthur} (15.\,5.\,1862 Wien – 21.\,10.\,1931 ebd.), \emph{Schriftsteller, Mediziner}!Amourette. Pièce en trois actes. Adaptée de Arthur Schnitzler@\strich\emph{Amourette. Pièce en trois actes. Adaptée de Arthur Schnitzler}|pwkv} von \emph{Liebelei}\pwindex{Schnitzler, Arthur 15.\,5.\,1862 Wien – 21.\,10.\,1931 ebd.@\textsc{Schnitzler, Arthur} (15.\,5.\,1862 Wien – 21.\,10.\,1931 ebd.), \emph{Schriftsteller, Mediziner}!Liebelei. Schauspiel in drei Akten@\strich\emph{Liebelei. Schauspiel in drei Akten}|pwk} (\begin{otherlanguage}{french}\emph{Amourette. Pièce en trois actes}\pwindex{Schnitzler, Arthur 15.\,5.\,1862 Wien – 21.\,10.\,1931 ebd.@\textsc{Schnitzler, Arthur} (15.\,5.\,1862 Wien – 21.\,10.\,1931 ebd.), \emph{Schriftsteller, Mediziner}!Amourette. Pièce en trois actes. Adaptée de Arthur Schnitzler@\strich\emph{Amourette. Pièce en trois actes. Adaptée de Arthur Schnitzler}|pwk}\end{otherlanguage}), die jedoch unveröffentlicht blieb.}}}\label{K_L02876-5}. Ich habe keine Ahnung mehr
               von den getroffenen \label{K_L02876-6v}\edtext{Abmachungen}{\lemma{\textnormal{\emph{Abmachungen}}}\Cendnote{\textnormal{Siehe XXXX Auszeichnungsfehler: Dokument L02785 nicht gefunden.
               }}}\label{K_L02876-6}. Jedenfalls haſt Du zum Mindeſten Anſpruch auf die \uline{Hälfte} des Honorars, {\pb}da Du ihm\pwindex{Thorel, Jean 11.\,9.\,1859 Éragny – 20.\,8.\,1916 Enghien-les-Bains@\textsc{Thorel, Jean} (11.\,9.\,1859 Éragny – 20.\,8.\,1916 Enghien-les-Bains), \emph{Übersetzer, Dramatiker}|pwv} ja{ }ſein ganzes Honorar,
               das \strikeout{es} aus den \begin{otherlanguage}{french}\textsc{Tantièmen}\end{otherlanguage}{ }der \label{K_L02876-7v}\edtext{Aufführungen\pwindex{Schnitzler, Arthur 15.\,5.\,1862 Wien – 21.\,10.\,1931 ebd.@\textsc{Schnitzler, Arthur} (15.\,5.\,1862 Wien – 21.\,10.\,1931 ebd.), \emph{Schriftsteller, Mediziner}!Amourette. Pièce en trois actes. Adaptée de Arthur Schnitzler@\strich\emph{Amourette. Pièce en trois actes. Adaptée de Arthur Schnitzler}|pwv}}{\lemma{\textnormal{\emph{Aufführungen}}}\Cendnote{\textnormal{Abgesehen von einer Aufführung am 29. 8. 1902 in Dunkerque\oindex{Dunkirk@\textbf{Dunkirk}, \emph{Hauptstadt}|pwk} sind keine Vorstellungen von \emph{Liebelei}\pwindex{Schnitzler, Arthur 15.\,5.\,1862 Wien – 21.\,10.\,1931 ebd.@\textsc{Schnitzler, Arthur} (15.\,5.\,1862 Wien – 21.\,10.\,1931 ebd.), \emph{Schriftsteller, Mediziner}!Liebelei. Schauspiel in drei Akten@\strich\emph{Liebelei. Schauspiel in drei Akten}|pwk} nach Thorels\pwindex{Thorel, Jean 11.\,9.\,1859 Éragny – 20.\,8.\,1916 Enghien-les-Bains@\textsc{Thorel, Jean} (11.\,9.\,1859 Éragny – 20.\,8.\,1916 Enghien-les-Bains), \emph{Übersetzer, Dramatiker}|pwk}{ }Übersetzung\pwindex{Schnitzler, Arthur 15.\,5.\,1862 Wien – 21.\,10.\,1931 ebd.@\textsc{Schnitzler, Arthur} (15.\,5.\,1862 Wien – 21.\,10.\,1931 ebd.), \emph{Schriftsteller, Mediziner}!Amourette. Pièce en trois actes. Adaptée de Arthur Schnitzler@\strich\emph{Amourette. Pièce en trois actes. Adaptée de Arthur Schnitzler}|pwkv} bekannt.}}}\label{K_L02876-7}
               beſtritten werden{ }ſollte, als \label{K_L02876-8v}\edtext{Vorſchuß}{\lemma{\textnormal{\emph{Vorschuß}}}\Cendnote{\textnormal{in der Höhe
                  von 500 Francs}}}\label{K_L02876-8} gezahlt haſt. Auch den »Kakadu\pwindex{Schnitzler, Arthur 15.\,5.\,1862 Wien – 21.\,10.\,1931 ebd.@\textsc{Schnitzler, Arthur} (15.\,5.\,1862 Wien – 21.\,10.\,1931 ebd.), \emph{Schriftsteller, Mediziner}!grüne Kakadu. Groteske in einem Akt@\strich\emph{Der grüne Kakadu. Groteske in einem Akt}|pw}«{ }ſollteſt Du ihm \label{K_L02876-9v}\edtext{zu
               überſetzen geben}{\lemma{\textnormal{\emph{zu
               übersetzen geben}}}\Cendnote{\textnormal{\emph{Der grüne Kakadu}\pwindex{Schnitzler, Arthur 15.\,5.\,1862 Wien – 21.\,10.\,1931 ebd.@\textsc{Schnitzler, Arthur} (15.\,5.\,1862 Wien – 21.\,10.\,1931 ebd.), \emph{Schriftsteller, Mediziner}!grüne Kakadu. Groteske in einem Akt@\strich\emph{Der grüne Kakadu. Groteske in einem Akt}|pwk} wurde zuerst von Émile Soutif\pwindex{Soutif, Émile @\textsc{Soutif, Émile}, \emph{Lehrer}|pwk} (Übersetzung\pwindex{Schnitzler, Arthur 15.\,5.\,1862 Wien – 21.\,10.\,1931 ebd.@\textsc{Schnitzler, Arthur} (15.\,5.\,1862 Wien – 21.\,10.\,1931 ebd.), \emph{Schriftsteller, Mediziner}!Le Perroquet Vert@\strich\emph{Le Perroquet Vert}|pwkv} nicht überliefert, siehe XXXX Auszeichnungsfehler: Dokument L00923 nicht gefunden) und dann als \emph{Au Perroquet vert}\pwindex{Schnitzler, Arthur 15.\,5.\,1862 Wien – 21.\,10.\,1931 ebd.@\textsc{Schnitzler, Arthur} (15.\,5.\,1862 Wien – 21.\,10.\,1931 ebd.), \emph{Schriftsteller, Mediziner}!Au Perroquet Vert@\strich\emph{Au Perroquet Vert}|pwk} von Stephan Epstein\pwindex{Epstein, Stephan 12.\,11.\,1866 Warschau – 1941 Paris@\textsc{Epstein, Stephan} (12.\,11.\,1866 Warschau – 1941 Paris), \emph{Schriftsteller, Dramaturg, Übersetzer}|pwk} und Émile
                     Lutz\pwindex{Lutz, Émile 8.\,4.\,1868 Saint-Étienne-du-Rouvray – 18.\,1.\,1940 Paris@\textsc{Lutz, Émile} (8.\,4.\,1868 Saint-Étienne-du-Rouvray – 18.\,1.\,1940 Paris), \emph{Übersetzer, Dichter}|pwk} ins Französische übersetzt. Die spätere Übersetzung\pwindex{Schnitzler, Arthur 15.\,5.\,1862 Wien – 21.\,10.\,1931 ebd.@\textsc{Schnitzler, Arthur} (15.\,5.\,1862 Wien – 21.\,10.\,1931 ebd.), \emph{Schriftsteller, Mediziner}!Au Perroquet Vert@\strich\emph{Au Perroquet Vert}|pwkv} war die Grundlage für zwölf
                  Aufführungen zwischen 7. 11. 1903 und 6. 12. 1903 im Théâtre
                     Antoine\oindex{Théâtre Antoine-Simone Berriau@\textbf{Théâtre Antoine-Simone Berriau}, \emph{Theater}|pwk}.}}}\label{K_L02876-9}. Er iſt als Überſetzer{ }ſo{ }ſchlecht, wie alle Andern, hat
               aber doch wenigſtens Verbindungen{\dotsfive}\pend
           
\pstart
           Ich erlebe nichts, was mich \label{K_L02876-10v}\edtext{glücklich
               und unglücklich}{\lemma{\textnormal{\emph{glücklich und unglücklich}}}\Cendnote{\textnormal{Eventuell wird hier
                  neuerlich (vgl. XXXX Auszeichnungsfehler: Dokument L02875 nicht gefunden) die
                  Frühphase der intimen Beziehung mit der verheirateten Theodore Rottenberg\pwindex{Rottenberg, Theodore 7.\,9.\,1875 – 5.\,4.\,1945 Limburg an der Lahn@\textsc{Rottenberg, Theodore} (7.\,9.\,1875 – 5.\,4.\,1945 Limburg an der Lahn)|pwk} etwas kryptisch beschrieben, vgl. XXXX Auszeichnungsfehler: Dokument L02889 nicht gefunden. }}}\label{K_L02876-10} zugleich
               macht,{ }ſondern: Es würde ein großes Glück{ }ſein, aber ich kann es nicht erleben.
               Siehſt Du: Verlieren, durch das Schickſal verlieren, wie es Dein Loos war, iſt
               furchtbar. Aber nicht \strikeout{\textcolor{gray}{×}\-\textcolor{gray}{×}\-\textcolor{gray}{×}\-\textcolor{gray}{×}} beſitzen können, \uline{durch eigene Schuld} nicht
               beſitzen können, iſt entſetzlich, und zudem wird man{ }ſich{ }ſelbſt verächtlich und zum
               Ekel. Das läßt{ }ſich Alles nicht \substVorne{}\textsuperscript{\textcolor{gray}{×}\-\textcolor{gray}{×}\-\textcolor{gray}{×}\-\textcolor{gray}{×}\-\textcolor{gray}{×}}\substDazwischen{}ſchreiben\substHinten{}; ich{ }ſehne mich danach, es Dir zu erzählen{\dotsfour}\pend
           
\pstart
           Bitte,{ }ſchreib’ mir bald wieder, wie es Dir geht. Theile mir auch freundlichſt die
               Adreſſe des Herrn \textsc{von Hoffmannsthal\pwindex{Hofmannsthal, Hugo von 1.\,2.\,1874 Wien – 15.\,7.\,1929 Rodaun@\textsc{Hofmannsthal, Hugo von} (1.\,2.\,1874 Wien – 15.\,7.\,1929 Rodaun), \emph{Schriftsteller}|pw}} mit, dem ich \label{K_L02876-11v}\edtext{mein Buch\pwindex{Goldmann, Paul 31.\,1.\,1865 Breslau – 25.\,9.\,1935 Wien@\textsc{Goldmann, Paul} (31.\,1.\,1865 Breslau – 25.\,9.\,1935 Wien), \emph{Schriftsteller, Journalist}!Sommer in China. Reisebilder@\strich\emph{Ein Sommer in China. Reisebilder}|pwv}}{\lemma{\textnormal{\emph{mein Buch}}}\Cendnote{\textnormal{über seine Asienreise 1898; Paul Goldmann\pwindex{Goldmann, Paul 31.\,1.\,1865 Breslau – 25.\,9.\,1935 Wien@\textsc{Goldmann, Paul} (31.\,1.\,1865 Breslau – 25.\,9.\,1935 Wien), \emph{Schriftsteller, Journalist}|pwk}: \emph{Ein Sommer in China. Reisebilder}\pwindex{Goldmann, Paul 31.\,1.\,1865 Breslau – 25.\,9.\,1935 Wien@\textsc{Goldmann, Paul} (31.\,1.\,1865 Breslau – 25.\,9.\,1935 Wien), \emph{Schriftsteller, Journalist}!Sommer in China. Reisebilder@\strich\emph{Ein Sommer in China. Reisebilder}|pwk}. 2 Bde. Frankfurt am Main\oindex{Frankfurt am Main@\textbf{Frankfurt am Main}, \emph{Hauptstadt}|pwk}: \emph{Literarische Anstalt Rütten {\kaufmannsund}
                     Loening}\orgindex{Rütten und Loening@Rütten {\kaufmannsund}  Loening|pwk}{ }1899, erschienen Anfang Mai 1899.}}}\label{K_L02876-11}{ }ſchicken möchte. Was macht \textsc{Richard\pwindex{Beer-Hofmann, Richard 11.\,7.\,1866 Wien – 26.\,9.\,1945 New York City@\textsc{Beer-Hofmann, Richard} (11.\,7.\,1866 Wien – 26.\,9.\,1945 New York City), \emph{Schriftsteller}|pw}}? Ich höre natürlich kein Wort von ihm.\pend
           
\pstart
           Was{ }ſagt Ihr zur »Fackel\pwindex{Fackel@\emph{Die Fackel}|pw}«? Der Burſch\pwindex{Kraus, Karl 28.\,4.\,1874 Jičín – 12.\,6.\,1936 Wien@\textsc{Kraus, Karl} (28.\,4.\,1874 Jičín – 12.\,6.\,1936 Wien), \emph{Schriftsteller, Publizist, Schriftsteller}|pwv} hat Talent. Schade nur, daß er ein{ }ſolcher Lausbub iſt. Denn das \label{K_L02876-12v}\edtext{Ausmiſtungs-Werk}{\lemma{\textnormal{\emph{Ausmistungs-Werk}}}\Cendnote{\textnormal{Anspielung auf Karl Kraus\pwindex{Kraus, Karl 28.\,4.\,1874 Jičín – 12.\,6.\,1936 Wien@\textsc{Kraus, Karl} (28.\,4.\,1874 Jičín – 12.\,6.\,1936 Wien), \emph{Schriftsteller, Publizist, Schriftsteller}|pwk}’ umfassende polemische Kritik in
                  der damals neu erscheinenden \emph{Fackel}\pwindex{Fackel@\emph{Die Fackel}|pwk}}}}\label{K_L02876-12}, das er unternimmt, iſt verdienlich. Er{ }ſagt treffliche Worte gegen \textsc{Bauer\pwindex{Bauer, Julius 15.\,10.\,1853 Szigetvár – 11.\,6.\,1941 Wien@\textsc{Bauer, Julius} (15.\,10.\,1853 Szigetvár – 11.\,6.\,1941 Wien), \emph{Schriftsteller, Journalist, Kritiker}|pw}}, \textsc{Herzl\pwindex{Herzl, Theodor 2.\,5.\,1860 Budapest – 3.\,7.\,1904 Edlach@\textsc{Herzl, Theodor} (2.\,5.\,1860 Budapest – 3.\,7.\,1904 Edlach), \emph{Schriftsteller, Journalist}|pw}}, \textsc{Bahr\pwindex{Bahr, Hermann 19.\,7.\,1863 Linz – 15.\,1.\,1934 München@\textsc{Bahr, Hermann} (19.\,7.\,1863 Linz – 15.\,1.\,1934 München), \emph{Schriftsteller, Kritiker}|pw}}, namentlich gegen die »Neue Freie Preſſe\orgindex{Neue Freie Presse@Neue Freie Presse|pw}«,
               und es iſt das Traurige an den jetzigen \strikeout{Wien\oindex{Wien@\textbf{Wien}, \emph{Verwaltungsgebiet}|pw}er}{ }{\pb}Wien\oindex{Wien@\textbf{Wien}, \emph{Verwaltungsgebiet}|pw}er Verhältniſſen, daß, wenn endlich einmal
               Jemand kommt, der gegen die Corruption kämpft, er ebenſo corrupt iſt, wie die
               Corruption{ }ſelbſt.\pend
           
\pstart
           Grüß’ mir \textsc{Schwarzkopf\pwindex{Schwarzkopf, Gustav 7.\,11.\,1853 Wien – 13.\,11.\,1939 ebd.@\textsc{Schwarzkopf, Gustav} (7.\,11.\,1853 Wien – 13.\,11.\,1939 ebd.), \emph{Schriftsteller}|pw}}, mit dem Du ja jetzt häufiger zuſammen biſt.\pend
           
\pstart
           Ich grüße Dich von Herzen {\\[\baselineskip]}Dein treuer {\\[\baselineskip]}\spacefill\mbox{Paul Goldmann.}\pend
           \leftskip=0em{}\selectlanguage{ngerman}\endnumbering\briefempfaengerindex{Schnitzler, Arthur@\textsc{Schnitzler, Arthur}!zzzGoldmann, Paul@\emph{von Paul Goldmann}!1899-05-281@{28. 5. [1899]}|)be}\mylabel{L02876h}  \newcommand{\dateiname}{L02876}\newcommand{\titel}{Paul Goldmann an Arthur Schnitzler, 28. 5. [1899]}\newcommand{\editorInnen}{Martin Anton Müller und Laura Untner}%% latex-leseansicht-abspann.tex
%% Abspann für die Leseansicht.
%% Der Schalter \ifkorrekturansicht ist bereits durch den Vorspann gesetzt.

%% latex-abspann.tex
%% Gemeinsamer Abspann für Korrekturansicht und Leseansicht.
%% Setzt den Schalter \ifkorrekturansicht voraus (gesetzt in den
%% einbindenden Dateien latex-korrekturansicht-abspann.tex bzw.
%% latex-leseansicht-abspann.tex).
%% ---------------------------------------------------------------

\normalsize

% Das esempio-Environment wird nur in der Leseansicht benötigt
\ifkorrekturansicht\else
\newenvironment{esempio}[3]%
{
    \vspace{1.5ex}
    \rlap{\underline{#1}}
    \par
    \setlength{\parindent}{0cm}
    \nopagebreak
    \leftskip=#2cm
    \rightskip=#3cm
}
{
    \par
}
\fi

\doendnotes{C}
\bigskip
\vfill

\clearpage

\footnotesize

\ifkorrekturansicht
  \lohead{\textsc{register}}
\fi

% theindex-Environment neu definieren ohne reledmac
\makeatletter
\renewenvironment{theindex}{%
  \ifkorrekturansicht
    \section*{\indexname}%
  \else
    \subsubsection*{Index der erwähnten Entitäten}%
  \fi
  \setlength{\parindent}{0pt}%
  \setlength{\parskip}{0pt plus 0.3pt}%
  \let\item\@idxitem
}{%
  \ifkorrekturansicht\clearpage\fi
}
\makeatother

\IfFileExists{\jobname-pw.ind}{\input{\jobname-pw.ind}}{}

% Quellenangabe nur in der Leseansicht
\ifkorrekturansicht\else
% Fallback-Definitionen, falls die .tex-Datei \titel etc. nicht gesetzt hat
\providecommand{\titel}{}
\providecommand{\editorInnen}{}
\providecommand{\dateiname}{\jobname}

\vspace{3cm}

\vfill

\footnotesize
\textsc{Quelle}: \titel. Herausgegeben von {\editorInnen}. In: \emph{Arthur Schnitzler: Briefwechsel mit Autorinnen und Autoren}.
 Digitale Edition, https://schnitzler-briefe.acdh.oeaw.ac.at/{\dateiname}.html (Stand \today)
\fi

\end{document}


