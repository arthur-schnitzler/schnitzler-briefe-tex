%% latex-korrekturansicht-vorspann.tex
%% Vorspann für die Korrekturansicht.
%% Lädt die gemeinsame Datei latex-vorspann.tex mit gesetztem Schalter.

\newif\ifkorrekturansicht
\korrekturansichttrue

\input{../tex-inputs/latex-vorspann}


\section[Hugo von Hofmannsthal an Arthur Schnitzler, {[}10. 3. 1898?{]}]{L00782 Hugo von Hofmannsthal an Arthur Schnitzler, {[}10. 3. 1898?{]}}
\nopagebreak\mylabel{L00782v}
\rehead{ }\normalsize\beginnumbering\briefempfaengerindex{Schnitzler, Arthur@\textsc{Schnitzler, Arthur}!zzzHofmannsthal, Hugo von@\emph{von Hugo von Hofmannsthal}!1898-03-101@{{[}10. 3. 1898?{]}}|(be}
\toendnotes[C]{\smallbreak\pagebreak[2]}\Standort{CUL, Schnitzler, B 43b/1.}
\physDesc{Brief, 1 Blatt, 2 Seiten, 482 Zeichen
\newline{}Handschrift: schwarze Tinte, deutsche Kurrent
\newline{}Schnitzler: mit Bleistift datiert: »März 98« 
\newline{}Ordnung: 1) mit Bleistift von unbekannter Hand nummeriert: »\strikeout{108}«  2) mit Bleistift von unbekannter Hand nummeriert:
                                    »109«}
\buchAbdrucke{\weitereDrucke{Hugo von Hofmannsthal, Arthur Schnitzler: \emph{Briefwechsel}. Frankfurt am Main: \emph{S. Fischer} 1964, S. 100.} }\toendnotes[C]{\smallbreak}
\pstart
           {\pb}\textcolor{gray}{\textbf{\label{T_L00782-1v}\edtext{hvH}{\lemma{\textnormal{\emph{hvH}}}\Cendnote{\textnormal{gedrucktes Monogramm mit Krone in blauer Farbe}}}\label{T_L00782-1}}}\pend
           
\pstart
           \raggedleft{}Donnerstag.\pend
           
\pstart{}lieber Arthur\pend\vspace{0.5em}
\pstart
           entſchuldigen Sie daſs ich Sie wegen einer Dummheit beläſtige.\pend
           
\pstart
           Am \label{K_L00782-1v}\edtext{zweiten Jänner}{\lemma{\textnormal{\emph{zweiten Jänner}}}\Cendnote{\textnormal{Das Gastspiel hatte bereits vom
                     25. 11. 1897 bis zum 28. 11. 1897 stattgefunden. Bei der erwähnten
                  Aufführung an einem Sonntag dürfte es sich um die Schlussvorstellung am
                     28. 11. 1897 handeln.}}}\label{K_L00782-1} oder einem dieſem Datum ſehr nahen
               Sonn oder Feiertag hat die \textsc{Réjane}\pwindex{Rejane 1856-06-05 – 1920-06-14@\textsc{Réjane} (1856-06-05 – 1920-06-14), \emph{Schauspieler/Schauspielerin}|pw} im Carltheater\oindex{Carl-Theater@\textbf{Carl-Theater}, \emph{Theater (K.THE)}|pw}{ }\uline{nachmittag} die \textsc{Madame Sans Gêne}\pwindex{Madame Sans-Gêne. Comedie en 3 actes et 1 prologue@\emph{Madame Sans-Gêne. Comédie en 3 actes et 1 prologue}|pw} geſpielt. Ich wär ſehr froh, wenn ich den {\pb}Theaterzettel von dieſer
               Vorſtellung haben könnt, den ſicher noch irgend ein Diener{[},{]}
               Beamter oder ſo jemand im Carltheater\oindex{Carl-Theater@\textbf{Carl-Theater}, \emph{Theater (K.THE)}|pw} beſitzt.
               Vielleicht könnten Sie mir durch die \textsc{Glümer}\pwindex{Gluemer, Marie 03.07.1867 – 16.11.1925@\textsc{Glümer, Marie} (03.07.1867 – 16.11.1925), \emph{Schauspieler/Schauspielerin}|pw} oder ſo mir einen verſchaffen. Das wäre sehr lieb.\pend
           
\pstart
           Ihr{\\[\baselineskip]}\spacefill\mbox{Hugo.}\pend
           \leftskip=0em{}\selectlanguage{ngerman}\endnumbering\briefempfaengerindex{Schnitzler, Arthur@\textsc{Schnitzler, Arthur}!zzzHofmannsthal, Hugo von@\emph{von Hugo von Hofmannsthal}!1898-03-101@{{[}10. 3. 1898?{]}}|)be}\mylabel{L00782h}  \normalsize

\doendnotes{C}
\bigskip
\vfill

\clearpage

\footnotesize

\lohead{\textsc{register}}

% Definiere theindex-Environment komplett neu ohne reledmac
\makeatletter
\renewenvironment{theindex}{%
  \section*{\indexname}%
  \setlength{\parindent}{0pt}%
  \setlength{\parskip}{0pt plus 0.3pt}%
  \let\item\@idxitem
}{%
  \clearpage
}
\makeatother

\IfFileExists{\jobname-pw.ind}{\input{\jobname-pw.ind}}{}

\end{document}

      