%% latex-leseansicht-vorspann.tex
%% Vorspann für die Leseansicht.
%% Lädt die gemeinsame Datei latex-vorspann.tex mit nicht gesetztem Schalter.

\newif\ifkorrekturansicht
\korrekturansichtfalse

\input{../tex-inputs/latex-vorspann}


               \section[Hugo von Hofmannsthal an Arthur Schnitzler, {[}10. 3. 1898?{]}]{ Hugo von Hofmannsthal an Arthur Schnitzler, {[}10. 3. 1898?{]}}\nopagebreak\mylabel{v}\rehead{ }\begin{ledgroupsized}[t]{13cm}\normalsize\beginnumbering\briefempfaengerindex{Schnitzler, Arthur@\textsc{Schnitzler, Arthur}!zzzHofmannsthal, Hugo von@\emph{von Hugo von Hofmannsthal}!1898-03-101@{{[}10. 3. 1898?{]}}|(be} \toendnotes[C]{\smallbreak\pagebreak[2]} \Standort{CUL, Schnitzler, B 43b/1.}
\physDesc{Brief, 1 Blatt, 2 Seiten
\newline{}Handschrift: schwarze Tinte, deutsche Kurrent
\newline{}Schnitzler: mit Bleistift datiert: »März 98« \newline{}Ordnung: 1) mit Bleistift von unbekannter Hand nummeriert: »\strikeout{108}« 2) mit Bleistift von unbekannter Hand nummeriert:
                                    »109«}\buchAbdrucke{\weitereDrucke{Hugo von Hofmannsthal, Arthur Schnitzler: \emph{Briefwechsel}. Hg. Therese Nickl und Heinrich Schnitzler. Frankfurt am Main: \emph{S. Fischer} 1964, S. 100.} }\toendnotes[C]{\smallbreak}\pstart
           \noindent{}{\pb}\textcolor{gray}{\textbf{\label{T_L00782-1v}\edtext{hvH}{\lemma{\textnormal{\emph{hvH}}}\Cendnote{\textnormal{gedrucktes Monogramm mit Krone in blauer Farbe}}}\label{T_L00782-1h}}}\pend
           \pstart
           \raggedleft{}Donnerstag.\pend
           \pstart{}lieber Arthur\pend\pstart
           entſchuldigen Sie daſs ich Sie wegen einer Dummheit beläſtige.\pend
           \pstart
           Am \label{K_L00782_1v}\edtext{zweiten Jänner}{\lemma{\textnormal{\emph{zweiten Jänner}}}\Cendnote{\textnormal{Das Gastspiel hatte bereits von
                     25.–28. 11. 1897 stattgefunden. Bei der erwähnten
                  Aufführung an einem Sonntag dürfte es sich um die Schlussvorstellung am
                  28. 11. 1897 handeln.}}}\label{K_L00782_1h} oder einem dieſem Datum ſehr nahen Sonn oder
               Feiertag hat die \textsc{Réjane}\pwindex{Reju, Gabrielle-Charlotte 06.06.1856 – 14.06.1920@\textsc{Réju, Gabrielle-Charlotte} (06.06.1856 – 14.06.1920), \emph{Schauspielerin}|pw} im Carltheater\oindex{Carl-Theater@\textbf{Carl-Theater}|pw}{ }\uline{nachmittag} die \textsc{Madame Sans Gêne}\pwindex{\textcolor{red}{\textsuperscript{XXXX1 indx}}!Madame Sans-Gêne1893@\strich\emph{Madame Sans-Gêne} {[}1893{]}|pw}\pwindex{\textcolor{red}{\textsuperscript{XXXX1 indx}}!Madame Sans-Gêne1893@\strich\emph{Madame Sans-Gêne} {[}1893{]}|pw} geſpielt. Ich wär ſehr froh, wenn ich den {\pb}Theaterzettel von dieſer
               Vorſtellung haben könnt, den ſicher noch irgend ein Diener{[},{]}
               Beamter oder ſo jemand im Carltheater\oindex{Carl-Theater@\textbf{Carl-Theater}|pw} beſitzt.
               Vielleicht könnten Sie mir durch die \textsc{Glümer}\pwindex{Gluemer, Marie 03.07.1867 – 16.11.1925@\textsc{Glümer, Marie} (03.07.1867 – 16.11.1925), \emph{Schauspielerin}|pw} oder ſo mir einen verſchaffen. Das wäre sehr lieb.\pend
           \pstart
           Ihr{\\[\baselineskip]}\spacefill\mbox{Hugo.}\pend
           \leftskip=0em{}          \endnumbering\briefempfaengerindex{Schnitzler, Arthur@\textsc{Schnitzler, Arthur}!zzzHofmannsthal, Hugo von@\emph{von Hugo von Hofmannsthal}!1898-03-101@{{[}10. 3. 1898?{]}}|)be}\mylabel{h}\end{ledgroupsized}  \newcommand{\dateiname}{L00782}\newcommand{\titel}{Hugo von Hofmannsthal an Arthur Schnitzler, [10. 3. 1898?]}\newcommand{\editorInnen}{Martin Anton Müller und Gerd-Hermann Susen}
            \footnotesize
\begin{ledgroupsized}[t]{11.5cm}
\doendnotes{C}
\end{ledgroupsized}
         %% latex-leseansicht-abspann.tex
%% Abspann für die Leseansicht.
%% Der Schalter \ifkorrekturansicht ist bereits durch den Vorspann gesetzt.

%% latex-abspann.tex
%% Gemeinsamer Abspann für Korrekturansicht und Leseansicht.
%% Setzt den Schalter \ifkorrekturansicht voraus (gesetzt in den
%% einbindenden Dateien latex-korrekturansicht-abspann.tex bzw.
%% latex-leseansicht-abspann.tex).
%% ---------------------------------------------------------------

\normalsize

% Das esempio-Environment wird nur in der Leseansicht benötigt
\ifkorrekturansicht\else
\newenvironment{esempio}[3]%
{
    \vspace{1.5ex}
    \rlap{\underline{#1}}
    \par
    \setlength{\parindent}{0cm}
    \nopagebreak
    \leftskip=#2cm
    \rightskip=#3cm
}
{
    \par
}
\fi

\doendnotes{C}
\bigskip
\vfill

\clearpage

\footnotesize

\ifkorrekturansicht
  \lohead{\textsc{register}}
\fi

% theindex-Environment neu definieren ohne reledmac
\makeatletter
\renewenvironment{theindex}{%
  \ifkorrekturansicht
    \section*{\indexname}%
  \else
    \subsubsection*{Index der erwähnten Entitäten}%
  \fi
  \setlength{\parindent}{0pt}%
  \setlength{\parskip}{0pt plus 0.3pt}%
  \let\item\@idxitem
}{%
  \ifkorrekturansicht\clearpage\fi
}
\makeatother

\IfFileExists{\jobname-pw.ind}{\input{\jobname-pw.ind}}{}

% Quellenangabe nur in der Leseansicht
\ifkorrekturansicht\else
% Fallback-Definitionen, falls die .tex-Datei \titel etc. nicht gesetzt hat
\providecommand{\titel}{}
\providecommand{\editorInnen}{}
\providecommand{\dateiname}{\jobname}

\vspace{3cm}

\vfill

\footnotesize
\textsc{Quelle}: \titel. Herausgegeben von {\editorInnen}. In: \emph{Arthur Schnitzler: Briefwechsel mit Autorinnen und Autoren}.
 Digitale Edition, https://schnitzler-briefe.acdh.oeaw.ac.at/{\dateiname}.html (Stand \today)
\fi

\end{document}


      