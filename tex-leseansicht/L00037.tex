%% latex-leseansicht-vorspann.tex
%% Vorspann für die Leseansicht.
%% Lädt die gemeinsame Datei latex-vorspann.tex mit nicht gesetztem Schalter.

\newif\ifkorrekturansicht
\korrekturansichtfalse

\input{../tex-inputs/latex-vorspann}


\section[Eduard Michael Kafka an Arthur Schnitzler, 30. 8. 1891]{L00037 Eduard Michael Kafka an Arthur Schnitzler, 30. 8. 1891}
\nopagebreak\mylabel{L00037v}
\rehead{ }\normalsize\beginnumbering\briefempfaengerindex{Schnitzler, Arthur@\textsc{Schnitzler, Arthur}!zzzKafka, Eduard Michael@\emph{von Eduard Michael Kafka}!1891-08-301@{30. 8. 1891}|(be}
\toendnotes[C]{\smallbreak\pagebreak[2]}
\correspDesc{Versand  durch Eduard Michael Kafka am 30. 8. 1891 in Brünn
\newline{}Erhalt  durch Arthur Schnitzler im Zeitraum [31. 8. 1891
                  – 4. 9. 1891?] in Wien}\toendnotes[C]{\smallbreak}
\Standort{DLA, A:Schnitzler, HS.NZ85.1.3604.}
\physDesc{Brief, 1 Blatt, 2 Seiten, 1535 Zeichen
\newline{}Handschrift: schwarze Tinte, deutsche Kurrent
\newline{}Schnitzler: mit rotem Buntstift beschriftet: »\textsc{Kafka}« und nummeriert: »(2)«. mit rotem
                                 Buntstift eine Unterstreichung  }\toendnotes[C]{\smallbreak}
\pstart
           \centering{}{\pb}\textcolor{gray}{\textbf{\textbf{Moderne Dichtung.}}}\orgindex{Moderne Dichtung/Moderne Rundschau@Moderne Dichtung/Moderne Rundschau|pw}\pend
           
\pstart
           \centering{}\textcolor{gray}{\textbf{\so{Monatsſchrift für Literatur und Kritik}.}}\pend
           
\pstart
           \centering{}\textcolor{gray}{\textbf{Herausgeber: \textbf{E. M. Kafka, Wien\oindex{Wien@\textbf{Wien}, \emph{Verwaltungsgebiet}|pw}.} – Verlag: \textbf{Holzwarth {\kaufmannsund}
                           Ortony\orgindex{Holzwarth und Ortony@Holzwarth {\kaufmannsund}  Ortony|pw}, Wien\oindex{Wien@\textbf{Wien}, \emph{Verwaltungsgebiet}|pw}.}}}\pend
           
\pstart
           \raggedleft{}Brünn\oindex{Brünn@\textbf{Brünn}|pw}, \textcolor{gray}{\textbf{\textbf{Wien\oindex{Wien@\textbf{Wien}, \emph{Verwaltungsgebiet}|pw},}}}{ }30. August \textcolor{gray}{\textbf{189}}1\pend
           
\pstart{}Sehr verehrter Herr Doctor,\pend\vspace{0.5em}
\pstart
           ich lade Sie freundlichſt ein, an einem »\label{K_L00037-1v}\edtext{\textsc{Oesterreichischen Jahrbuch für moderne Literatur}}{\lemma{\textnormal{\emph{Oesterreichischen … Literatur}}}\Cendnote{\textnormal{Das Jahrbuch sollte Beiträge von 42
                  Schriftstellern enthalten, wurde aber nicht verwirklicht. Vgl. den Brief Kafkas\pwindex{Kafka, Eduard Michael 11.\,3.\,1869 Wien – 6.\,8.\,1893 Brünn@\textsc{Kafka, Eduard Michael} (11.\,3.\,1869 Wien – 6.\,8.\,1893 Brünn), \emph{Redakteur}|pwk} an Ferdinand von Saar\pwindex{Saar, Ferdinand von 30.\,9.\,1833 Wien – 24.\,7.\,1906 ebd.@\textsc{Saar, Ferdinand von} (30.\,9.\,1833 Wien – 24.\,7.\,1906 ebd.), \emph{Schriftsteller}|pwk} vom 25. 8. 1891, in: \emph{Jugend in Wien. Literatur um 1900}. Ausstellung und
                     Katalog von Ludwig Greve und Werner Volke. München:
                        \emph{Kösel}{ }1974, S. 98.}}}\label{K_L00037-1}« mitzuarbeiten, das ich anfangs
                  November herauszugeben beabſichtige. Und zwar erbitte ich mir für
               dasſelbe vor allem »die \textsc{Elixire}\pwindex{Schnitzler, Arthur 15.\,5.\,1862 Wien – 21.\,10.\,1931 ebd.@\textsc{Schnitzler, Arthur} (15.\,5.\,1862 Wien – 21.\,10.\,1931 ebd.), \emph{Schriftsteller, Mediziner}!drei Elixire@\strich\emph{Die drei Elixire}|pw}«, u. wäre Ihnen ganz außerordentlich verbunden, könnte ich hiezu noch eine
               bisher ungedruckte \uline{Bluette} erhalten. Aus dem \textsc{Anatol-Cyclus}\pwindex{Schnitzler, Arthur 15.\,5.\,1862 Wien – 21.\,10.\,1931 ebd.@\textsc{Schnitzler, Arthur} (15.\,5.\,1862 Wien – 21.\,10.\,1931 ebd.), \emph{Schriftsteller, Mediziner}!Anatol@\strich\emph{Anatol}|pw} haben Sie ja noch Etwas, – wenn ich nicht irre. Wenn möglich, bäte ich um recht
               baldige Zuſendung, da das Buch bereits anfangs September in Angriff genommen, alſo
               mit der Drucklegung begonnen werden wird.\pend
           
\pstart
           Ich bäte ferner um Zuſendung Ihres »Märchen\pwindex{Schnitzler, Arthur 15.\,5.\,1862 Wien – 21.\,10.\,1931 ebd.@\textsc{Schnitzler, Arthur} (15.\,5.\,1862 Wien – 21.\,10.\,1931 ebd.), \emph{Schriftsteller, Mediziner}!Märchen. Schauspiel in drei Aufzügen@\strich\emph{Das Märchen. Schauspiel in drei Aufzügen}|pw}«, um
               dasſelbe dem Direktor\pwindex{Baumann, Adolf 10.\,3.\,1855 Karlsruhe – 30.\,1.\,1895 Ärmelkanal@\textsc{Baumann, Adolf} (10.\,3.\,1855 Karlsruhe – 30.\,1.\,1895 Ärmelkanal), \emph{Schauspieler, Theaterdirektor}|pwv} des
               Brünner Stadttheater\oindex{Stadttheater [Brünn]@\textbf{Stadttheater [Brünn]}, \emph{Theater}|pw} zu übermit{\pb}teln. Derſelbe verſprach mir, das Stück binnen 3 Tagen
               gelesen u.{ }ſich bezüglich einer ev. Aufführung entſchieden zu haben. Wenn möglich,{ }ſo
               wär es am beſten, wenn die Einreichung \uline{jetzt}
               geſchähe, da mir Baumann\pwindex{Baumann, Adolf 10.\,3.\,1855 Karlsruhe – 30.\,1.\,1895 Ärmelkanal@\textsc{Baumann, Adolf} (10.\,3.\,1855 Karlsruhe – 30.\,1.\,1895 Ärmelkanal), \emph{Schauspieler, Theaterdirektor}|pw} mittheilt, daſs er
               auf Suche \introOben{}nach Novitäten\introOben{} iſt.\pend
           
\pstart
           Was meine Geſundheit betrifft,{ }ſo vermag ich leider nichts beſonders Günſtiges zu
               vermelden. Doch hoffe ich immerhin, in 4–6 Wochen wieder nach Wien\oindex{Wien@\textbf{Wien}, \emph{Verwaltungsgebiet}|pw} zurückkehren zu können.\pend
           
\pstart
           Sie würden mich durch ein paar Zeilen{ }ſehr erfreuen. Auch bitte ich Sie recht{ }ſehr,
               mich Ihrem Herrn Bruder\pwindex{Schnitzler, Julius 13.\,7.\,1865 Wien – 29.\,6.\,1939 ebd.@\textsc{Schnitzler, Julius} (13.\,7.\,1865 Wien – 29.\,6.\,1939 ebd.), \emph{Chirurg}|pwv}, der
               wohl{ }ſehr böſe auf mich{ }ſein wird, weil ich mich wirklich recht unartig ihm gegenüber
               benommen habe, frdlchst zu empfehlen. Es rächt{ }ſich jetzt an mir, in unangenehmſter
               Weiſe, daſs ich ihm{ }ſo vorzeitig Reißaus genommen!\pend
           
\pstart
           Mit herzlichen Grüßen{\\[\baselineskip]}Ihr{\\[\baselineskip]}Sie aufrichtig hochſchätzender{\\[\baselineskip]}\spacefill\mbox{EMKafka}\pend
           \leftskip=0em{}
\pstart
           \noindent{}Brünn\oindex{Brünn@\textbf{Brünn}|pw}, Straßengaſſe 36\oindex{Hybešova@\textbf{Hybešova}, \emph{Straße}|pw}\pend
           
\pstart
           \label{T_L00037-1v}\edtext{\textcolor{gray}{\textbf{Alle den Inhalt der »Modernen
                        Dichtung\orgindex{Moderne Dichtung/Moderne Rundschau@Moderne Dichtung/Moderne Rundschau|pw}« betreffenden Zuſchriften und Sendungen wolle man an die
                     Redaktion: Wien\oindex{Wien@\textbf{Wien}, \emph{Verwaltungsgebiet}|pw}, VIII., Buchfeldgasse 8\oindex{Wien@\textbf{Wien}!VIII., Josefstadt@\textbf{VIII., Josefstadt}!Buchfeldgasse@\textbf{Buchfeldgasse}, \emph{Straße}|pw} (Sprechſtunden 2–4), alle auf die
                     Adminiſtration und Expedition bezüglichen Zuſchriften, Geldſendungen etc.
                     jedoch an den Verlag\orgindex{Holzwarth und Ortony@Holzwarth {\kaufmannsund}  Ortony|pw}: Wien\oindex{Wien@\textbf{Wien}, \emph{Verwaltungsgebiet}|pw}, IX.,
                        Liechtenſteinſtraße 3\oindex{Liechtensteinstraße [Hinterbrühl]@\textbf{Liechtensteinstraße [Hinterbrühl]}, \emph{Straße}|pw}, richten.}}}{\lemma{\textnormal{\emph{Alle … richten.}}}\Cendnote{\textnormal{quer am Rand der ersten Seite}}}\label{T_L00037-1}\pend
           \selectlanguage{ngerman}\endnumbering\briefempfaengerindex{Schnitzler, Arthur@\textsc{Schnitzler, Arthur}!zzzKafka, Eduard Michael@\emph{von Eduard Michael Kafka}!1891-08-301@{30. 8. 1891}|)be}\mylabel{L00037h}  \newcommand{\dateiname}{L00037}\newcommand{\titel}{Eduard Michael Kafka an Arthur Schnitzler, 30. 8. 1891}\newcommand{\editorInnen}{Martin Anton Müller und Gerd-Hermann Susen}%% latex-leseansicht-abspann.tex
%% Abspann für die Leseansicht.
%% Der Schalter \ifkorrekturansicht ist bereits durch den Vorspann gesetzt.

%% latex-abspann.tex
%% Gemeinsamer Abspann für Korrekturansicht und Leseansicht.
%% Setzt den Schalter \ifkorrekturansicht voraus (gesetzt in den
%% einbindenden Dateien latex-korrekturansicht-abspann.tex bzw.
%% latex-leseansicht-abspann.tex).
%% ---------------------------------------------------------------

\normalsize

% Das esempio-Environment wird nur in der Leseansicht benötigt
\ifkorrekturansicht\else
\newenvironment{esempio}[3]%
{
    \vspace{1.5ex}
    \rlap{\underline{#1}}
    \par
    \setlength{\parindent}{0cm}
    \nopagebreak
    \leftskip=#2cm
    \rightskip=#3cm
}
{
    \par
}
\fi

\doendnotes{C}
\bigskip
\vfill

\clearpage

\footnotesize

\ifkorrekturansicht
  \lohead{\textsc{register}}
\fi

% theindex-Environment neu definieren ohne reledmac
\makeatletter
\renewenvironment{theindex}{%
  \ifkorrekturansicht
    \section*{\indexname}%
  \else
    \subsubsection*{Index der erwähnten Entitäten}%
  \fi
  \setlength{\parindent}{0pt}%
  \setlength{\parskip}{0pt plus 0.3pt}%
  \let\item\@idxitem
}{%
  \ifkorrekturansicht\clearpage\fi
}
\makeatother

\IfFileExists{\jobname-pw.ind}{\input{\jobname-pw.ind}}{}

% Quellenangabe nur in der Leseansicht
\ifkorrekturansicht\else
% Fallback-Definitionen, falls die .tex-Datei \titel etc. nicht gesetzt hat
\providecommand{\titel}{}
\providecommand{\editorInnen}{}
\providecommand{\dateiname}{\jobname}

\vspace{3cm}

\vfill

\footnotesize
\textsc{Quelle}: \titel. Herausgegeben von {\editorInnen}. In: \emph{Arthur Schnitzler: Briefwechsel mit Autorinnen und Autoren}.
 Digitale Edition, https://schnitzler-briefe.acdh.oeaw.ac.at/{\dateiname}.html (Stand \today)
\fi

\end{document}


