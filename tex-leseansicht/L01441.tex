%% latex-leseansicht-vorspann.tex
%% Vorspann für die Leseansicht.
%% Lädt die gemeinsame Datei latex-vorspann.tex mit nicht gesetztem Schalter.

\newif\ifkorrekturansicht
\korrekturansichtfalse

\input{../tex-inputs/latex-vorspann}


\section[Hugo und Gerty von Hofmannsthal an Arthur Schnitzler, [9.?] 9. 1904]{L01441 Hugo und Gerty von Hofmannsthal an Arthur Schnitzler, [9.?] 9. 1904}
\nopagebreak\mylabel{L01441v}
\rehead{ }\normalsize\beginnumbering\briefempfaengerindex{Schnitzler, Arthur@\textsc{Schnitzler, Arthur}!zzzHofmannsthal, Gertrude von@\emph{von Gertrude von Hofmannsthal}!1904-09-091@{[9.?] 9. 1904}|(be}\briefempfaengerindex{Schnitzler, Arthur@\textsc{Schnitzler, Arthur}!zzzHofmannsthal, Hugo von@\emph{von Hugo von Hofmannsthal}!1904-09-091@{[9.?] 9. 1904}|(be}
\toendnotes[C]{\smallbreak\pagebreak[2]}
\correspDesc{Versand  durch Hugo von Hofmannsthal, Gerty von Hofmannsthal am [9.?] 9. 1904 in Bozen
\newline{}Erhalt  durch Arthur Schnitzler am 10. 9. 1904 in St. Gilgen}\toendnotes[C]{\smallbreak}
\Standort{CUL, Schnitzler, B 43.}
\physDesc{Bildpostkarte, 157 Zeichen
\newline{}Handschrift Hugo von Hofmannsthal: Bleistift, lateinische Kurrent
\newline{}Handschrift Gertrude von Hofmannsthal: Bleistift
\newline{}Versand: 1) Stempel: »\nobreak{}\textcolor{gray}{×}. IX. 04, 8\nobreak{}«.   2) Stempel: »\nobreak{}\oindex{St. Gilgen@\textbf{St. Gilgen}, \emph{Verwaltungsgebiet}|pwk}St. Gilgen, 10. 9. 04\nobreak{}«. 
\newline{}Schnitzler: mit Bleistift datiert: »Sept 904« 
\newline{}Ordnung: 1) mit Bleistift von unbekannter Hand nummeriert: »\strikeout{251}«  2) mit Bleistift von unbekannter Hand nummeriert:
                                    »235«}
\buchAbdrucke{\weitereDrucke{Hugo von Hofmannsthal, Arthur Schnitzler: \emph{Briefwechsel}. Herausgegeben von Therese Nickl und Heinrich Schnitzler. Frankfurt am Main: \emph{S. Fischer} 1964, S. 202.} }\pstart{}{\pb}Herrn D\textsuperscript{r} Arthur Schnitzler\pend{}\pstart{}Lueg\oindex{Lueg@\textbf{Lueg}, \emph{Teil eines besiedelten Ortes}|pw}\pend{}\pstart{}bei Sanct Gilgen\oindex{St. Gilgen@\textbf{St. Gilgen}, \emph{Verwaltungsgebiet}|pw}\pend{}\pstart{}Salzka{\geminationm}ergut\oindex{Salzkammergut@\textbf{Salzkammergut}, \emph{Region}|pw}\pend{}{\bigskip}
\pstart
           \noindent{}\centering{}{\pb}\textcolor{gray}{\textbf{Grüsse vom Rosengarten\oindex{Rosengartengruppe@\textbf{Rosengartengruppe}, \emph{Berg}|pw} bei Bozen\oindex{Bozen@\textbf{Bozen}, \emph{Hauptstadt}|pw}}}\pend
           \vspace{1em}
\pstart
           \noindent{}{\pb}Wir sind brav und haben uns die
                  Herzogin von Assy\pwindex{\textcolor{red}{\textsuperscript{XXXX indx1}}!Göttinnen oder Die drei Romane der Herzogin von Assy@\strich\emph{Die Göttinnen oder Die drei Romane der Herzogin von Assy}|pw} und einiges von Tschechow\pwindex{Čechov, Anton Pavlovič 17.\,1.\,1860 Taganrog – 15.\,7.\,1904 Badenweiler@\textsc{Čechov, Anton Pavlovič} (17.\,1.\,1860 Taganrog – 15.\,7.\,1904 Badenweiler), \emph{Schriftsteller}|pw} gekauft.\pend
           
\pstart
           Viele Grüsse \spacefill\mbox{Hugo}{\\[\baselineskip]}\spacefill\mbox{{[}hs. Hofmannsthal:{]} Gerty}\pend
           \leftskip=0em{}\selectlanguage{ngerman}\endnumbering\briefempfaengerindex{Schnitzler, Arthur@\textsc{Schnitzler, Arthur}!zzzHofmannsthal, Gertrude von@\emph{von Gertrude von Hofmannsthal}!1904-09-091@{[9.?] 9. 1904}|)be}\briefempfaengerindex{Schnitzler, Arthur@\textsc{Schnitzler, Arthur}!zzzHofmannsthal, Hugo von@\emph{von Hugo von Hofmannsthal}!1904-09-091@{[9.?] 9. 1904}|)be}\mylabel{L01441h}  \newcommand{\dateiname}{L01441}\newcommand{\titel}{Hugo und Gerty von Hofmannsthal an Arthur Schnitzler, [9.?] 9. 1904}\newcommand{\editorInnen}{Martin Anton Müller und Gerd-Hermann Susen}%% latex-leseansicht-abspann.tex
%% Abspann für die Leseansicht.
%% Der Schalter \ifkorrekturansicht ist bereits durch den Vorspann gesetzt.

%% latex-abspann.tex
%% Gemeinsamer Abspann für Korrekturansicht und Leseansicht.
%% Setzt den Schalter \ifkorrekturansicht voraus (gesetzt in den
%% einbindenden Dateien latex-korrekturansicht-abspann.tex bzw.
%% latex-leseansicht-abspann.tex).
%% ---------------------------------------------------------------

\normalsize

% Das esempio-Environment wird nur in der Leseansicht benötigt
\ifkorrekturansicht\else
\newenvironment{esempio}[3]%
{
    \vspace{1.5ex}
    \rlap{\underline{#1}}
    \par
    \setlength{\parindent}{0cm}
    \nopagebreak
    \leftskip=#2cm
    \rightskip=#3cm
}
{
    \par
}
\fi

\doendnotes{C}
\bigskip
\vfill

\clearpage

\footnotesize

\ifkorrekturansicht
  \lohead{\textsc{register}}
\fi

% theindex-Environment neu definieren ohne reledmac
\makeatletter
\renewenvironment{theindex}{%
  \ifkorrekturansicht
    \section*{\indexname}%
  \else
    \subsubsection*{Index der erwähnten Entitäten}%
  \fi
  \setlength{\parindent}{0pt}%
  \setlength{\parskip}{0pt plus 0.3pt}%
  \let\item\@idxitem
}{%
  \ifkorrekturansicht\clearpage\fi
}
\makeatother

\IfFileExists{\jobname-pw.ind}{\input{\jobname-pw.ind}}{}

% Quellenangabe nur in der Leseansicht
\ifkorrekturansicht\else
% Fallback-Definitionen, falls die .tex-Datei \titel etc. nicht gesetzt hat
\providecommand{\titel}{}
\providecommand{\editorInnen}{}
\providecommand{\dateiname}{\jobname}

\vspace{3cm}

\vfill

\footnotesize
\textsc{Quelle}: \titel. Herausgegeben von {\editorInnen}. In: \emph{Arthur Schnitzler: Briefwechsel mit Autorinnen und Autoren}.
 Digitale Edition, https://schnitzler-briefe.acdh.oeaw.ac.at/{\dateiname}.html (Stand \today)
\fi

\end{document}


