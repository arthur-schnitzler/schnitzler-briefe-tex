\input{../tex-inputs/latex-pdf-vorspann}
\begin{center}
            \textcolor{red}{ENTWURF. ENTZIFFERUNG NOCH NICHT KORREKTURGELESEN}
                      \end{center}
            
               \section[Hugo und Gerty von Hofmannsthal an Arthur Schnitzler, {[}9.?{]} 9. 1904]{ Hugo und Gerty von Hofmannsthal an Arthur Schnitzler,
               {[}9.?{]} 9. 1904}\nopagebreak\mylabel{v}\rehead{ }\begin{ledgroupsized}[t]{13cm}\normalsize\beginnumbering\briefempfaengerindex{Schnitzler, Arthur@\textsc{Schnitzler, Arthur}!zzzHofmannsthal, Gertrude von@\emph{von Gertrude von Hofmannsthal}!1904-09-091@{{[}9.?{]} 9. 1904}|(be}\briefempfaengerindex{Schnitzler, Arthur@\textsc{Schnitzler, Arthur}!zzzHofmannsthal, Hugo von@\emph{von Hugo von Hofmannsthal}!1904-09-091@{{[}9.?{]} 9. 1904}|(be} \toendnotes[C]{\smallbreak\pagebreak[2]} \Standort{CUL, Schnitzler, B 43.}
\physDesc{Bildpostkarte
\newline{}Handschrift Hugo von Hofmannsthal: Bleistift, lateinische Kurrent\newline{}Handschrift Gertrude von Hofmannsthal: Bleistift\newline{}Versand: 1) Stempel: »\nobreak{}\textcolor{gray}{×}. IX. 04, 8\nobreak{}«.  2) Stempel: »\nobreak{}\oindex{St. Gilgen@\textbf{St. Gilgen}|pwk}St. Gilgen, 10. 9. 04\nobreak{}«. 
\newline{}Schnitzler: mit Bleistift datiert: »Sept 904« \newline{}Ordnung: 1) mit Bleistift von unbekannter Hand nummeriert: »\strikeout{251}« 2) mit Bleistift von unbekannter Hand nummeriert: »235«}\buchAbdrucke{\weitereDrucke{Hugo von Hofmannsthal, Arthur Schnitzler: \emph{Briefwechsel}. Hg. Therese Nickl und Heinrich Schnitzler. Frankfurt am Main: \emph{S. Fischer} 1964, S. 202.} }\pstart{}{\pb}Herrn D\textsuperscript{r} Arthur Schnitzler\pend{}\pstart{}Lueg\oindex{Lueg am Wolfgangsee@\textbf{Lueg am Wolfgangsee}|pw}\pend{}\pstart{}bei Sanct Gilgen\oindex{St. Gilgen@\textbf{St. Gilgen}|pw}\pend{}\pstart{}Salzka{\geminationm}ergut\oindex{Salzkammergut@\textbf{Salzkammergut}|pw}\pend{}{\bigskip}\pstart
           \noindent{}\centering{}\textcolor{gray}{\textbf{{\pb}Grüsse vom Rosengarten\oindex{Rosengartengruppe@\textbf{Rosengartengruppe}|pw} bei Bozen\oindex{Bozen@\textbf{Bozen}|pw}}}\pend
           \pstart
           {\pb}Wir sind brav und haben uns die
                  Herzogin von Assy\pwindex{\textcolor{red}{\textsuperscript{XXXX1 indx}}!Goettinnen oder Die drei Romane der Herzogin von Assy1902@\strich\emph{Die Göttinnen oder Die drei Romane der Herzogin von Assy} {[}1902{]}|pw} und einiges von Tschechow\pwindex{Cechov, Anton Pavlovic 1860-01-17 – 1904-07-15@\textsc{Čechov, Anton Pavlovič} (1860-01-17 – 1904-07-15), \emph{Schriftsteller}|pw} gekauft.\pend
           \pstart
           Viele Grüsse \spacefill\mbox{Hugo}{\\[\baselineskip]}\spacefill\mbox{{[}hs. G. Hofmannsthal:{]} Gerty}\pend
           \leftskip=0em{}\endnumbering\briefempfaengerindex{Schnitzler, Arthur@\textsc{Schnitzler, Arthur}!zzzHofmannsthal, Gertrude von@\emph{von Gertrude von Hofmannsthal}!1904-09-091@{{[}9.?{]} 9. 1904}|)be}\briefempfaengerindex{Schnitzler, Arthur@\textsc{Schnitzler, Arthur}!zzzHofmannsthal, Hugo von@\emph{von Hugo von Hofmannsthal}!1904-09-091@{{[}9.?{]} 9. 1904}|)be}\mylabel{h}\end{ledgroupsized}  \newcommand{\dateiname}{L01441}\newcommand{\titel}{Hugo und Gerty von Hofmannsthal an Arthur Schnitzler, [9.?] 9. 1904}\newcommand{\editorInnen}{Martin Anton Müller und Gerd-Hermann Susen}\input{../tex-inputs/latex-pdf-abspann}
      