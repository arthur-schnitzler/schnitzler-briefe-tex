%% latex-leseansicht-vorspann.tex
%% Vorspann für die Leseansicht.
%% Lädt die gemeinsame Datei latex-vorspann.tex mit nicht gesetztem Schalter.

\newif\ifkorrekturansicht
\korrekturansichtfalse

\input{../tex-inputs/latex-vorspann}


\section[Arthur Schnitzler an Hermann Bahr, 1. 2. 1904]{L01368 Arthur Schnitzler an Hermann Bahr, 1. 2. 1904}
\nopagebreak\mylabel{L01368v}
\rehead{ }\normalsize\beginnumbering\briefempfaengerindex{Bahr, Hermann@\textsc{Bahr, Hermann}!zzzSchnitzler, Arthur@\emph{von Arthur Schnitzler}!1904-02-011@{1. 2. 1904}|(be}
\toendnotes[C]{\smallbreak\pagebreak[2]}
\correspDesc{Versand  durch Arthur Schnitzler am 1. 2. 1904 in Wien
\newline{}Erhalt  durch Hermann Bahr im Zeitraum [2. 2. 1904
                  – 6. 2. 1904?] in Radolfzell am Bodensee}\toendnotes[C]{\smallbreak}
\Standort{TMW, HS AM 23365 Ba.}
\physDesc{Brief, 1 Blatt, 4 Seiten, 1307 Zeichen
\newline{}Handschrift: schwarze Tinte, deutsche Kurrent}
\buchAbdrucke{\weitereDrucke{1) \emph{1. 2. 1904.} In: Arthur Schnitzler: \emph{The Letters of Arthur Schnitzler to Hermann Bahr}. Edited, annotated, and with an introduction, by Donald G. Daviau. Chapel Hill: \emph{The University of North Carolina Press} 1978, S. 83–84 (University of North Carolina studies in the Germanic languages
                        and literatures, 89).} \weitereDrucke{2) Hermann Bahr, Arthur Schnitzler: \emph{Briefwechsel, Aufzeichnungen, Dokumente (1891–1931)}. Herausgegeben von Kurt Ifkovits und Martin Anton Müller. Göttingen: \emph{Wallstein} 2018, S. 294.} }\toendnotes[C]{\smallbreak}
\pstart
           \raggedleft{}{\pb}Wien\oindex{Wien@\textbf{Wien}, \emph{Verwaltungsgebiet}|pw}{ }1. 2. 904.\pend
           \vspace{0.5em}
\pstart
           lieber Hermann, aus deinen Worten{ }ſcheint mir eher eine üble Sti{\geminationm}ung als ein übles Befinden hervorzugehen – was für den
               Betroffenen allerdings aufs gleiche herauskommt. Immerhin – ohne Ratſchlägen u
               Entſchlüſſen vorgreifen zu wollen, deine Idee mit Taormina\oindex{Taormina@\textbf{Taormina}, \emph{Hauptstadt}|pw} iſt mir{ }ſehr{ }ſympathiſch – beſonders weil ich große Luſt hätte, im
                  \label{K_L01368-1v}\edtext{April}{\lemma{\textnormal{\emph{April}}}\Cendnote{\textnormal{Die Reise fand erst im Mai statt.}}}\label{K_L01368-1} nach Sicilien\oindex{Sizilien@\textbf{Sizilien}, \emph{Land}|pw}{ }{\pb}zu fahren und es mir
               natürlich höchſt erfreulich wäre, dich dort zu finden. Wir (meine Frau\pwindex{Schnitzler, Olga 17.\,1.\,1882 Wien – 13.\,1.\,1970 Lugano@\textsc{Schnitzler, Olga} (17.\,1.\,1882 Wien – 13.\,1.\,1970 Lugano), \emph{Schauspielerin, Sängerin}|pwv} u ich) möchten gern zu Schiff von \textsc{Fiume\oindex{Rijeka@\textbf{Rijeka}|pw}} nach \textsc{Palermo\oindex{Palermo@\textbf{Palermo}|pw}}.\pend
           
\pstart
           – Donnerſtag reiſe ich nach Berlin\oindex{Berlin@\textbf{Berlin}, \emph{Hauptstadt}|pw},
               wo es{ }ſich zeigen{ }ſoll, wie der Einſame Weg\pwindex{Schnitzler, Arthur 15.\,5.\,1862 Wien – 21.\,10.\,1931 ebd.@\textsc{Schnitzler, Arthur} (15.\,5.\,1862 Wien – 21.\,10.\,1931 ebd.), \emph{Schriftsteller, Mediziner}!einsame Weg. Schauspiel in fünf Akten@\strich\emph{Der einsame Weg. Schauspiel in fünf Akten}|pw} auf
               der Bühne wirkt. Daſs im Gang des Stücks etwas nicht in Ordnung iſt, hat mir während
               der – oft unterbrochenen und ganz neu aufgeno{\geminationm}enen –
               Arbeit oft geſchienen. Die gute Wirkung {\pb}die das Stück im
               Vorleſen machte, hat mich einigermaßen beruhigt; – von den eigentlichen Theaterleuten
               scheint aber keiner ernſtlich an einen äußern Erfolg zu glauben (bei aller möglichen
               Hochachtung \textsc{etc.}). Mir perſönlich{ }ſind an dem Stücke werth:
               die Geſtalten des \textsc{Sala} und der \textsc{Johanna}; ferner der Lauf des 4. u besonders des 5. Aktes. –\pend
           
\pstart
           Deine Grüße werden beſtellt, meine Frau\pwindex{Schnitzler, Olga 17.\,1.\,1882 Wien – 13.\,1.\,1970 Lugano@\textsc{Schnitzler, Olga} (17.\,1.\,1882 Wien – 13.\,1.\,1970 Lugano), \emph{Schauspielerin, Sängerin}|pwv} dankt dir herzlich {\pb}für deine Grüße und
               wünſcht dir gleich mir, alles mögliche gute.\pend
           
\pstart
           Gelegentlich ein Wort von dir zu hören wäre mir höchſt erwünſcht und{ }ſehr
               erbeten.\pend
           
\pstart
           Dein getreuer{\\[\baselineskip]}\spacefill\mbox{Arthur.}\pend
           \leftskip=0em{}\selectlanguage{ngerman}\endnumbering\briefempfaengerindex{Bahr, Hermann@\textsc{Bahr, Hermann}!zzzSchnitzler, Arthur@\emph{von Arthur Schnitzler}!1904-02-011@{1. 2. 1904}|)be}\mylabel{L01368h}  \newcommand{\dateiname}{L01368}\newcommand{\titel}{Arthur Schnitzler an Hermann Bahr, 1. 2. 1904}\newcommand{\editorInnen}{Herausgegeben von Martin Anton Müller}%% latex-leseansicht-abspann.tex
%% Abspann für die Leseansicht.
%% Der Schalter \ifkorrekturansicht ist bereits durch den Vorspann gesetzt.

%% latex-abspann.tex
%% Gemeinsamer Abspann für Korrekturansicht und Leseansicht.
%% Setzt den Schalter \ifkorrekturansicht voraus (gesetzt in den
%% einbindenden Dateien latex-korrekturansicht-abspann.tex bzw.
%% latex-leseansicht-abspann.tex).
%% ---------------------------------------------------------------

\normalsize

% Das esempio-Environment wird nur in der Leseansicht benötigt
\ifkorrekturansicht\else
\newenvironment{esempio}[3]%
{
    \vspace{1.5ex}
    \rlap{\underline{#1}}
    \par
    \setlength{\parindent}{0cm}
    \nopagebreak
    \leftskip=#2cm
    \rightskip=#3cm
}
{
    \par
}
\fi

\doendnotes{C}
\bigskip
\vfill

\clearpage

\footnotesize

\ifkorrekturansicht
  \lohead{\textsc{register}}
\fi

% theindex-Environment neu definieren ohne reledmac
\makeatletter
\renewenvironment{theindex}{%
  \ifkorrekturansicht
    \section*{\indexname}%
  \else
    \subsubsection*{Index der erwähnten Entitäten}%
  \fi
  \setlength{\parindent}{0pt}%
  \setlength{\parskip}{0pt plus 0.3pt}%
  \let\item\@idxitem
}{%
  \ifkorrekturansicht\clearpage\fi
}
\makeatother

\IfFileExists{\jobname-pw.ind}{\input{\jobname-pw.ind}}{}

% Quellenangabe nur in der Leseansicht
\ifkorrekturansicht\else
% Fallback-Definitionen, falls die .tex-Datei \titel etc. nicht gesetzt hat
\providecommand{\titel}{}
\providecommand{\editorInnen}{}
\providecommand{\dateiname}{\jobname}

\vspace{3cm}

\vfill

\footnotesize
\textsc{Quelle}: \titel. Herausgegeben von {\editorInnen}. In: \emph{Arthur Schnitzler: Briefwechsel mit Autorinnen und Autoren}.
 Digitale Edition, https://schnitzler-briefe.acdh.oeaw.ac.at/{\dateiname}.html (Stand \today)
\fi

\end{document}


