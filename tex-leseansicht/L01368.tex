%% latex-korrekturansicht-vorspann.tex
%% Vorspann für die Korrekturansicht.
%% Lädt die gemeinsame Datei latex-vorspann.tex mit gesetztem Schalter.

\newif\ifkorrekturansicht
\korrekturansichttrue

\input{../tex-inputs/latex-vorspann}


\section[Arthur Schnitzler an Hermann Bahr, 1. 2. 1904]{L01368 Arthur Schnitzler an Hermann Bahr, 1. 2. 1904}
\nopagebreak\mylabel{L01368v}
\rehead{ }\normalsize\beginnumbering\briefempfaengerindex{Bahr, Hermann@\textsc{Bahr, Hermann}!zzzSchnitzler, Arthur@\emph{von Arthur Schnitzler}!1904-02-011@{1. 2. 1904}|(be}
\toendnotes[C]{\smallbreak\pagebreak[2]}\Standort{TMW, HS AM 23365 Ba.}
\physDesc{Brief, 1 Blatt, 4 Seiten, 1307 Zeichen
\newline{}Handschrift: schwarze Tinte, deutsche Kurrent}
\buchAbdrucke{\weitereDrucke{1) Arthur Schnitzler: \emph{The Letters of Arthur Schnitzler to Hermann Bahr}. Chapel Hill: \emph{The University of North Carolina Press} 1978, S. 83–84.} \weitereDrucke{2) Hermann Bahr, Arthur Schnitzler: \emph{Briefwechsel, Aufzeichnungen, Dokumente (1891–1931)}. Göttingen: \emph{Wallstein} 2018, S. 294.} }\toendnotes[C]{\smallbreak}
\pstart
           \raggedleft{}{\pb}Wien\oindex{Wien@\textbf{Wien}, \emph{A.ADM2}|pw}{ }1. 2. 904.\pend
           \vspace{0.5em}
\pstart
           lieber Hermann, aus deinen Worten ſcheint mir eher eine üble Sti{\geminationm}ung als ein übles Befinden hervorzugehen – was für den
               Betroffenen allerdings aufs gleiche herauskommt. Immerhin – ohne Ratſchlägen u
               Entſchlüſſen vorgreifen zu wollen, deine Idee mit Taormina\oindex{Taormina@\textbf{Taormina}, \emph{P.PPLA3}|pw} iſt mir ſehr ſympathiſch – beſonders weil ich große Luſt hätte, im
                  \label{K_L01368-1v}\edtext{April}{\lemma{\textnormal{\emph{April}}}\Cendnote{\textnormal{Die Reise fand erst im Mai statt.}}}\label{K_L01368-1} nach Sicilien\oindex{Sizilien@\textbf{Sizilien}, \emph{A.ADM1}|pw}{ }{\pb}zu fahren und es mir
               natürlich höchſt erfreulich wäre, dich dort zu finden. Wir (meine Frau\pwindex{Schnitzler, Olga 17.01.1882 – 13.01.1970@\textsc{Schnitzler, Olga} (17.01.1882 – 13.01.1970), \emph{Schauspieler/Schauspielerin, Sänger/Sängerin}|pwv} u ich) möchten gern zu Schiff von \textsc{Fiume\oindex{Rijeka@\textbf{Rijeka}, \emph{P.PPLA}|pw}} nach \textsc{Palermo\oindex{Palermo@\textbf{Palermo}, \emph{P.PPLA}|pw}}.\pend
           
\pstart
           – Donnerſtag reiſe ich nach Berlin\oindex{Berlin@\textbf{Berlin}, \emph{P.PPLC}|pw},
               wo es ſich zeigen ſoll, wie der Einſame Weg\pwindex{einsame Weg. Schauspiel in fuenf Akten@\emph{Der einsame Weg. Schauspiel in fünf Akten}|pw} auf
               der Bühne wirkt. Daſs im Gang des Stücks etwas nicht in Ordnung iſt, hat mir während
               der – oft unterbrochenen und ganz neu aufgeno{\geminationm}enen –
               Arbeit oft geſchienen. Die gute Wirkung {\pb}die das Stück im
               Vorleſen machte, hat mich einigermaßen beruhigt; – von den eigentlichen Theaterleuten
               scheint aber keiner ernſtlich an einen äußern Erfolg zu glauben (bei aller möglichen
               Hochachtung \textsc{etc.}). Mir perſönlich ſind an dem Stücke werth:
               die Geſtalten des \textsc{Sala} und der \textsc{Johanna}; ferner der Lauf des 4. u besonders des 5. Aktes. –\pend
           
\pstart
           Deine Grüße werden beſtellt, meine Frau\pwindex{Schnitzler, Olga 17.01.1882 – 13.01.1970@\textsc{Schnitzler, Olga} (17.01.1882 – 13.01.1970), \emph{Schauspieler/Schauspielerin, Sänger/Sängerin}|pwv} dankt dir herzlich {\pb}für deine Grüße und
               wünſcht dir gleich mir, alles mögliche gute.\pend
           
\pstart
           Gelegentlich ein Wort von dir zu hören wäre mir höchſt erwünſcht und ſehr
               erbeten.\pend
           
\pstart
           Dein getreuer{\\[\baselineskip]}\spacefill\mbox{Arthur.}\pend
           \leftskip=0em{}\selectlanguage{ngerman}\endnumbering\briefempfaengerindex{Bahr, Hermann@\textsc{Bahr, Hermann}!zzzSchnitzler, Arthur@\emph{von Arthur Schnitzler}!1904-02-011@{1. 2. 1904}|)be}\mylabel{L01368h}  \normalsize

\doendnotes{C}
\bigskip
\vfill

\clearpage

\footnotesize

\lohead{\textsc{register}}

% Definiere theindex-Environment komplett neu ohne reledmac
\makeatletter
\renewenvironment{theindex}{%
  \section*{\indexname}%
  \setlength{\parindent}{0pt}%
  \setlength{\parskip}{0pt plus 0.3pt}%
  \let\item\@idxitem
}{%
  \clearpage
}
\makeatother

\IfFileExists{\jobname-pw.ind}{\input{\jobname-pw.ind}}{}

\end{document}

      