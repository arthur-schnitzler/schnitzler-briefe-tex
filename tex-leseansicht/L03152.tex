%% latex-leseansicht-vorspann.tex
%% Vorspann für die Leseansicht.
%% Lädt die gemeinsame Datei latex-vorspann.tex mit nicht gesetztem Schalter.

\newif\ifkorrekturansicht
\korrekturansichtfalse

\input{../tex-inputs/latex-vorspann}


\section[ Felix Salten an Arthur Schnitzler, 18. 2. 1895]{L03152 Felix Salten an Arthur Schnitzler,  18. 2. 1895}
\nopagebreak\mylabel{L03152v}
\rehead{ }\normalsize\beginnumbering\briefempfaengerindex{Schnitzler, Arthur@\textsc{Schnitzler, Arthur}!zzzSalten, Felix@\emph{von Felix Salten}!1895-02-184@{18. 2. 1895}|(be}
\toendnotes[C]{\smallbreak\pagebreak[2]}
\correspDesc{Versand  durch Felix Salten am 18. 2. 1895 in München
\newline{}Erhalt  durch Arthur Schnitzler im Zeitraum [19. 2. 1895
                  – 23. 2. 1895?] in Wien}\toendnotes[C]{\smallbreak}
\Standort{CUL, Schnitzler, B 89, A 1.}
\physDesc{Brief, 1 Blatt, 4 Seiten, 1282 Zeichen
\newline{}Handschrift: Bleistift, lateinische Kurrent
\newline{}Ordnung: mit Bleistift von unbekannter Hand nummeriert: »53« }
\buchAbdrucke{\weitereDrucke{Hermann Bahr, Arthur Schnitzler: \emph{Briefwechsel, Aufzeichnungen, Dokumente (1891–1931)}. Herausgegeben von Kurt Ifkovits und Martin Anton Müller. Göttingen: \emph{Wallstein} 2018, S. 97–98.} }\toendnotes[C]{\smallbreak}
\pstart
           \raggedleft{}{\pb}München\oindex{München@\textbf{München}|pw}\textcolor{gray}{,}{ }1\substVorne{}\textsuperscript{9}\substDazwischen{}8\substHinten{}./II. 95.\pend
           \vspace{0.5em}
\pstart
           Lieber Freund, ich habe zunächst eine grosse Bitte an
               Sie: da ich vorausssichtlich von hier\oindex{München@\textbf{München}|pwv} nicht wegkomme, telegrafiren Sie mir gleich nach \substVorne{}\textsuperscript{e}\substDazwischen{}E\substHinten{}rhalt dieses Briefes: »Salten \strikeout{Hotel}{ }München\oindex{München@\textbf{München}|pw}{ }Oberpollinger\oindex{Hotel Oberpollinger@\textbf{Hotel Oberpollinger}, \emph{Hotel}|pw}. Ihre Anwesenheit für Donnerstag erwünscht. Die Redaction.«\pend
           
\pstart
           Aus dieser Bitte entnehmen Sie ungefähr auch\textcolor{gray}{,} wie es mir geht. Ich \substVorne{}\textsuperscript{kä}\substDazwischen{}ko\textcolor{gray}{m}\substHinten{}me dann Donnerstag von der Bahn direkt in die
                  Musik {\kaufmannsund}
                  Theatergesellschaft\orgindex{Wiener Musik- und Theatergesellschaft@Wiener Musik- und Theatergesellschaft|pw}, wo wir uns \label{K_L03152-1v}\edtext{treffen können}{\lemma{\textnormal{\emph{treffen können}}}\Cendnote{\textnormal{Sie sahen sich erst am
                     Freitag, dem 22. 2. 1895.}}}\label{K_L03152-1}.\pend
           
\pstart
           {\pb}Ich könnte jetzt sehr
               glücklich sein, wenn ich durch diese freundlichen Straßen mit einem Mädel ginge, das
               ich wirklich liebe. So aber ärgere ich mich ausschließlich, wenn ich mich nicht
               langweile. Morgen will ich ein paar Leute aufsuchen,
               da ich ja heute schon ein \label{K_L03152-2v}\edtext{Zimmer für Lotte\pwindex{Pohl-Glas, Charlotte 1.\,1.\,1873 Wien – 15.\,2.\,1944 Zürich@\textsc{Pohl-Glas, Charlotte} (1.\,1.\,1873 Wien – 15.\,2.\,1944 Zürich), \emph{Schriftstellerin, Politikerin, Sozialistin}|pw}
                  aufgenommen}{\lemma{\textnormal{\emph{Zimmer … aufgenommen}}}\Cendnote{\textnormal{Charlotte Glas\pwindex{Pohl-Glas, Charlotte 1.\,1.\,1873 Wien – 15.\,2.\,1944 Zürich@\textsc{Pohl-Glas, Charlotte} (1.\,1.\,1873 Wien – 15.\,2.\,1944 Zürich), \emph{Schriftstellerin, Politikerin, Sozialistin}|pwk} war mit dem gemeinsamen Kind\pwindex{Lamberg, Maria Charlotte 24.\,3.\,1895 Wien – 27.\,7.\,1895 Gerasdorf bei Wien@\textsc{Lamberg, Maria Charlotte} (24.\,3.\,1895 Wien – 27.\,7.\,1895 Gerasdorf bei Wien)|pwkv} schwanger. Eventuell
                  hätte sie es in München\oindex{München@\textbf{München}|pwk} gebären oder auch
                     nur die letzten Tage der Schwangerschaft dort verbringen sollen.}}}\label{K_L03152-2} habe, {\pb}mich
               also \uline{damit} nicht weiter aufzuhalten brauche.\pend
           
\pstart
           Ein Brief von Ihnen, der nicht schon unterwegs ist, träfe mich nicht mehr hier. Wenn
               etwas Wichtiges geschehen ist, dann telegrafiren Sie mir ja ohnedies noch separat.
               Sobald \label{K_L03152-3v}\edtext{Brahm\pwindex{Brahm, Otto 5.\,2.\,1856 Hamburg – 28.\,11.\,1912 Berlin@\textsc{Brahm, Otto} (5.\,2.\,1856 Hamburg – 28.\,11.\,1912 Berlin), \emph{Theaterleiter, Regisseur}|pw} Ihnen den Contract}{\lemma{\textnormal{\emph{Brahm Ihnen den Contract}}}\Cendnote{\textnormal{Gemeint war der Vertrag für das
                  Aufführungsrecht für \emph{Liebelei}\pwindex{Schnitzler, Arthur 15.\,5.\,1862 Wien – 21.\,10.\,1931 ebd.@\textsc{Schnitzler, Arthur} (15.\,5.\,1862 Wien – 21.\,10.\,1931 ebd.), \emph{Schriftsteller, Mediziner}!Liebelei. Schauspiel in drei Akten@\strich\emph{Liebelei. Schauspiel in drei Akten}|pwk} am \emph{Deutschen Theater}\orgindex{Deutsches Theater Berlin@Deutsches Theater Berlin|pwk}. Der Vertrag dürfte zu dem
                  Zeitpunkt bereits angekommen sein (vgl.
                     \emph{Der Briefwechsel Arthur Schnitzler – Otto Brahm}.
                     Vollständige Ausgabe. Herausgegeben, eingeleitet und erläutert von Oskar
                     Seidlin. Tübingen: \emph{Niemeyer}{ }1975, S. 4).}}}\label{K_L03152-3} gesendet {\kaufmannsund} Sie
               diese Sache in die Zeitungen geben, vergessen Sie nicht, auch {\pb}Ludassy\pwindex{Gans-Ludassy, Julius von 13.\,4.\,1858 Wien – 30.\,9.\,1922 ebd.@\textsc{Gans-Ludassy, Julius von} (13.\,4.\,1858 Wien – 30.\,9.\,1922 ebd.), \emph{Schriftsteller, Journalist, Herausgeber}|pw} zu verständigen.\pend
           
\pstart
           Haben Sie Bahr\pwindex{Bahr, Hermann 19.\,7.\,1863 Linz – 15.\,1.\,1934 München@\textsc{Bahr, Hermann} (19.\,7.\,1863 Linz – 15.\,1.\,1934 München), \emph{Schriftsteller, Kritiker}|pw}’s \label{K_L03152-4v}\edtext{Artikel A. S.\pwindex{Bahr, Hermann 19.\,7.\,1863 Linz – 15.\,1.\,1934 München@\textsc{Bahr, Hermann} (19.\,7.\,1863 Linz – 15.\,1.\,1934 München), \emph{Schriftsteller, Kritiker}!Adele Sandrock@\strich\emph{Adele Sandrock}|pw}}{\lemma{\textnormal{\emph{Artikel A. S.}}}\Cendnote{\textnormal{Hermann Bahr\pwindex{Bahr, Hermann 19.\,7.\,1863 Linz – 15.\,1.\,1934 München@\textsc{Bahr, Hermann} (19.\,7.\,1863 Linz – 15.\,1.\,1934 München), \emph{Schriftsteller, Kritiker}|pwk}: \emph{Adele Sandrock}\pwindex{Bahr, Hermann 19.\,7.\,1863 Linz – 15.\,1.\,1934 München@\textsc{Bahr, Hermann} (19.\,7.\,1863 Linz – 15.\,1.\,1934 München), \emph{Schriftsteller, Kritiker}!Adele Sandrock@\strich\emph{Adele Sandrock}|pwk}. In: \emph{Die
                        Zeit}\pwindex{Zeit. Wiener Wochenschrift@\emph{Die Zeit. Wiener Wochenschrift}|pwk}, Bd. 2, Nr. 20, 16. 2. 1895,
                     S. 108–109.}}}\label{K_L03152-4} gelesen? Ich habe ihn noch Samstag{ }Abend im Theater\oindex{Wien@\textbf{Wien}!VII., Neubau@\textbf{VII., Neubau}!Volkstheater@\textbf{Volkstheater}, \emph{Theater}|pwuv} gesprochen und er war wieder beängstigend freundlich.\pend
           
\pstart
           Leben Sie wol, und grüßen Beer Hofmann\pwindex{Beer-Hofmann, Richard 11.\,7.\,1866 Wien – 26.\,9.\,1945 New York City@\textsc{Beer-Hofmann, Richard} (11.\,7.\,1866 Wien – 26.\,9.\,1945 New York City), \emph{Schriftsteller}|pw}{ }{\kaufmannsund}{ }Loris\pwindex{Hofmannsthal, Hugo von 1.\,2.\,1874 Wien – 15.\,7.\,1929 Rodaun@\textsc{Hofmannsthal, Hugo von} (1.\,2.\,1874 Wien – 15.\,7.\,1929 Rodaun), \emph{Schriftsteller}|pw}. Auf Wiedersehen\pend
           
\pstart
           Herzlichst Ihr {\\[\baselineskip]}\spacefill\mbox{Salten}\pend
           \leftskip=0em{}\selectlanguage{ngerman}\endnumbering\briefempfaengerindex{Schnitzler, Arthur@\textsc{Schnitzler, Arthur}!zzzSalten, Felix@\emph{von Felix Salten}!1895-02-184@{18. 2. 1895}|)be}\mylabel{L03152h}  \newcommand{\dateiname}{L03152}\newcommand{\titel}{Felix Salten an Arthur Schnitzler, 18. 2. 1895}\newcommand{\editorInnen}{Martin Anton Müller und Laura Untner}%% latex-leseansicht-abspann.tex
%% Abspann für die Leseansicht.
%% Der Schalter \ifkorrekturansicht ist bereits durch den Vorspann gesetzt.

%% latex-abspann.tex
%% Gemeinsamer Abspann für Korrekturansicht und Leseansicht.
%% Setzt den Schalter \ifkorrekturansicht voraus (gesetzt in den
%% einbindenden Dateien latex-korrekturansicht-abspann.tex bzw.
%% latex-leseansicht-abspann.tex).
%% ---------------------------------------------------------------

\normalsize

% Das esempio-Environment wird nur in der Leseansicht benötigt
\ifkorrekturansicht\else
\newenvironment{esempio}[3]%
{
    \vspace{1.5ex}
    \rlap{\underline{#1}}
    \par
    \setlength{\parindent}{0cm}
    \nopagebreak
    \leftskip=#2cm
    \rightskip=#3cm
}
{
    \par
}
\fi

\doendnotes{C}
\bigskip
\vfill

\clearpage

\footnotesize

\ifkorrekturansicht
  \lohead{\textsc{register}}
\fi

% theindex-Environment neu definieren ohne reledmac
\makeatletter
\renewenvironment{theindex}{%
  \ifkorrekturansicht
    \section*{\indexname}%
  \else
    \subsubsection*{Index der erwähnten Entitäten}%
  \fi
  \setlength{\parindent}{0pt}%
  \setlength{\parskip}{0pt plus 0.3pt}%
  \let\item\@idxitem
}{%
  \ifkorrekturansicht\clearpage\fi
}
\makeatother

\IfFileExists{\jobname-pw.ind}{\input{\jobname-pw.ind}}{}

% Quellenangabe nur in der Leseansicht
\ifkorrekturansicht\else
% Fallback-Definitionen, falls die .tex-Datei \titel etc. nicht gesetzt hat
\providecommand{\titel}{}
\providecommand{\editorInnen}{}
\providecommand{\dateiname}{\jobname}

\vspace{3cm}

\vfill

\footnotesize
\textsc{Quelle}: \titel. Herausgegeben von {\editorInnen}. In: \emph{Arthur Schnitzler: Briefwechsel mit Autorinnen und Autoren}.
 Digitale Edition, https://schnitzler-briefe.acdh.oeaw.ac.at/{\dateiname}.html (Stand \today)
\fi

\end{document}


