%% latex-korrekturansicht-vorspann.tex
%% Vorspann für die Korrekturansicht.
%% Lädt die gemeinsame Datei latex-vorspann.tex mit gesetztem Schalter.

\newif\ifkorrekturansicht
\korrekturansichttrue

\input{../tex-inputs/latex-vorspann}


\section[ Felix Salten an Arthur Schnitzler, 18. 2. 1895]{L03152 Felix Salten an Arthur Schnitzler, 18. 2. 1895}
\nopagebreak\mylabel{L03152v}
\rehead{ }\normalsize\beginnumbering\briefempfaengerindex{Schnitzler, Arthur@\textsc{Schnitzler, Arthur}!zzzSalten, Felix@\emph{von Felix Salten}!1895-02-181@{18. 2. 1895}|(be}
\toendnotes[C]{\smallbreak\pagebreak[2]}\Standort{CUL, Schnitzler, B 89, A 1.}
\physDesc{Brief, 1 Blatt, 4 Seiten, 1282 Zeichen
\newline{}Handschrift: Bleistift, lateinische Kurrent
\newline{}Ordnung: mit Bleistift von unbekannter Hand nummeriert: »53« }
\buchAbdrucke{\weitereDrucke{Hermann Bahr, Arthur Schnitzler: \emph{Briefwechsel, Aufzeichnungen, Dokumente (1891–1931)}. Göttingen: \emph{Wallstein} 2018, S. 97–98.} }\toendnotes[C]{\smallbreak}
\pstart
           \raggedleft{}{\pb}München\oindex{Muenchen@\textbf{München}, \emph{P.PPLA}|pw}\textcolor{gray}{,}{ }1\substVorne{}\textsuperscript{9}\substDazwischen{}8\substHinten{}./II. 95.\pend
           \vspace{0.5em}
\pstart
           Lieber Freund, ich habe zunächst eine grosse Bitte an
               Sie: da ich vorausssichtlich von hier\oindex{Muenchen@\textbf{München}, \emph{P.PPLA}|pwv} nicht wegkomme, telegrafiren Sie mir gleich nach \substVorne{}\textsuperscript{e}\substDazwischen{}E\substHinten{}rhalt dieses Briefes: »Salten \strikeout{Hotel}{ }München\oindex{Muenchen@\textbf{München}, \emph{P.PPLA}|pw}{ }Oberpollinger\oindex{Hotel Oberpollinger@\textbf{Hotel Oberpollinger}, \emph{Hotel (K.HTL)}|pw}. Ihre Anwesenheit für Donnerstag erwünscht. Die Redaction.«\pend
           
\pstart
           Aus dieser Bitte entnehmen Sie ungefähr auch\textcolor{gray}{,} wie es mir geht. Ich \substVorne{}\textsuperscript{kä}\substDazwischen{}ko\textcolor{gray}{m}\substHinten{}me dann Donnerstag von der Bahn direkt in die
                  Musik {\kaufmannsund}
                  Theatergesellschaft\orgindex{Wiener Musik- und Theatergesellschaft@Wiener Musik- und Theatergesellschaft|pw}, wo wir uns \label{K_L03152-1v}\edtext{treffen können}{\lemma{\textnormal{\emph{treffen können}}}\Cendnote{\textnormal{Sie sahen sich erst am
                     Freitag, dem 22. 2. 1895.}}}\label{K_L03152-1}.\pend
           
\pstart
           {\pb}Ich könnte jetzt sehr
               glücklich sein, wenn ich durch diese freundlichen Straßen mit einem Mädel ginge, das
               ich wirklich liebe. So aber ärgere ich mich ausschließlich, wenn ich mich nicht
               langweile. Morgen will ich ein paar Leute aufsuchen,
               da ich ja heute schon ein \label{K_L03152-2v}\edtext{Zimmer für Lotte\pwindex{Pohl-Glas, Charlotte 1873-01-01 – 1944-02-15@\textsc{Pohl-Glas, Charlotte} (1873-01-01 – 1944-02-15), \emph{Schriftsteller/Schriftstellerin, Politiker/Politikerin, Sozialist/Sozialistin}|pw}
                  aufgenommen}{\lemma{\textnormal{\emph{Zimmer … aufgenommen}}}\Cendnote{\textnormal{Charlotte Glas\pwindex{Pohl-Glas, Charlotte 1873-01-01 – 1944-02-15@\textsc{Pohl-Glas, Charlotte} (1873-01-01 – 1944-02-15), \emph{Schriftsteller/Schriftstellerin, Politiker/Politikerin, Sozialist/Sozialistin}|pwk} war mit dem gemeinsamen Kind\pwindex{Lamberg, Maria Charlotte 1895-03-24 – 1895-07-27@\textsc{Lamberg, Maria Charlotte} (1895-03-24 – 1895-07-27)|pwkv} schwanger. Eventuell
                  hätte sie es in München\oindex{Muenchen@\textbf{München}, \emph{P.PPLA}|pwk} gebären oder auch
                     nur die letzten Tage der Schwangerschaft dort verbringen sollen.}}}\label{K_L03152-2} habe, {\pb}mich
               also \uline{damit} nicht weiter aufzuhalten brauche.\pend
           
\pstart
           Ein Brief von Ihnen, der nicht schon unterwegs ist, träfe mich nicht mehr hier. Wenn
               etwas Wichtiges geschehen ist, dann telegrafiren Sie mir ja ohnedies noch separat.
               Sobald \label{K_L03152-3v}\edtext{Brahm\pwindex{Brahm, Otto 05.02.1856 – 28.11.1912@\textsc{Brahm, Otto} (05.02.1856 – 28.11.1912), \emph{Theaterleiter/Theaterleiterin, Regisseur/Regisseurin}|pw} Ihnen den Contract}{\lemma{\textnormal{\emph{Brahm Ihnen den Contract}}}\Cendnote{\textnormal{Gemeint war der Vertrag für das
                  Aufführungsrecht für \emph{Liebelei}\pwindex{Liebelei. Schauspiel in drei Akten@\emph{Liebelei. Schauspiel in drei Akten}|pwk} am \emph{Deutschen Theater}\orgindex{Deutsches Theater Berlin@Deutsches Theater Berlin|pwk}. Der Vertrag dürfte zu dem
                  Zeitpunkt bereits angekommen sein (vgl.
                     \emph{Der Briefwechsel Arthur Schnitzler – Otto Brahm}.
                     Vollständige Ausgabe. Herausgegeben, eingeleitet und erläutert von Oskar
                     Seidlin. Tübingen: \emph{Niemeyer}{ }1975, S. 4).}}}\label{K_L03152-3} gesendet {\kaufmannsund} Sie
               diese Sache in die Zeitungen geben, vergessen Sie nicht, auch {\pb}Ludassy\pwindex{Gans-Ludassy, Julius von 13.04.1858 – 30.09.1922@\textsc{Gans-Ludassy, Julius von} (13.04.1858 – 30.09.1922), \emph{Schriftsteller/Schriftstellerin, Journalist/Journalistin, Herausgeber/Herausgeberin}|pw} zu verständigen.\pend
           
\pstart
           Haben Sie Bahr\pwindex{Bahr, Hermann 19.07.1863 – 15.01.1934@\textsc{Bahr, Hermann} (19.07.1863 – 15.01.1934), \emph{Schriftsteller/Schriftstellerin, Kritiker/Kritikerin}|pw}’s \label{K_L03152-4v}\edtext{Artikel A. S.\pwindex{Adele Sandrock@\emph{Adele Sandrock}|pw}}{\lemma{\textnormal{\emph{Artikel A. S.}}}\Cendnote{\textnormal{Hermann Bahr\pwindex{Bahr, Hermann 19.07.1863 – 15.01.1934@\textsc{Bahr, Hermann} (19.07.1863 – 15.01.1934), \emph{Schriftsteller/Schriftstellerin, Kritiker/Kritikerin}|pwk}: \emph{Adele Sandrock}\pwindex{Adele Sandrock@\emph{Adele Sandrock}|pwk}. In: \emph{Die
                        Zeit}\pwindex{Zeit. Wiener Wochenschrift@\emph{Die Zeit. Wiener Wochenschrift}|pwk}, Bd. 2, Nr. 20, 16. 2. 1895,
                     S. 108–109.}}}\label{K_L03152-4} gelesen? Ich habe ihn noch Samstag{ }Abend im Theater\oindex{Volkstheater@\textbf{Volkstheater}, \emph{Theater (K.THE)}|pwuv} gesprochen und er war wieder beängstigend freundlich.\pend
           
\pstart
           Leben Sie wol, und grüßen Beer Hofmann\pwindex{Beer-Hofmann, Richard 1866-07-11 – 1945-09-26@\textsc{Beer-Hofmann, Richard} (1866-07-11 – 1945-09-26), \emph{Schriftsteller/Schriftstellerin}|pw}{ }{\kaufmannsund}{ }Loris\pwindex{Hofmannsthal, Hugo von 1874-02-01 – 1929-07-15@\textsc{Hofmannsthal, Hugo von} (1874-02-01 – 1929-07-15), \emph{Schriftsteller/Schriftstellerin}|pw}. Auf Wiedersehen\pend
           
\pstart
           Herzlichst Ihr {\\[\baselineskip]}\spacefill\mbox{Salten}\pend
           \leftskip=0em{}\selectlanguage{ngerman}\endnumbering\briefempfaengerindex{Schnitzler, Arthur@\textsc{Schnitzler, Arthur}!zzzSalten, Felix@\emph{von Felix Salten}!1895-02-181@{18. 2. 1895}|)be}\mylabel{L03152h}  \normalsize

\doendnotes{C}
\bigskip
\vfill

\clearpage

\footnotesize

\lohead{\textsc{register}}

% Definiere theindex-Environment komplett neu ohne reledmac
\makeatletter
\renewenvironment{theindex}{%
  \section*{\indexname}%
  \setlength{\parindent}{0pt}%
  \setlength{\parskip}{0pt plus 0.3pt}%
  \let\item\@idxitem
}{%
  \clearpage
}
\makeatother

\IfFileExists{\jobname-pw.ind}{\input{\jobname-pw.ind}}{}

\end{document}

      