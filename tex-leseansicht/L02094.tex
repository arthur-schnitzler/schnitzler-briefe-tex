%% latex-leseansicht-vorspann.tex
%% Vorspann für die Leseansicht.
%% Lädt die gemeinsame Datei latex-vorspann.tex mit nicht gesetztem Schalter.

\newif\ifkorrekturansicht
\korrekturansichtfalse

\input{../tex-inputs/latex-vorspann}


         
         \renewcommand{\erwaehntePersonen}{Personen: Hermann Bahr, Richard Beer-Hofmann, Hugo von Hofmannsthal}
         \renewcommand{\erwaehnteOrte}{Orte: Hotel Panhans, Semmering, Wien}
         \renewcommand{\erwaehnteWerke}{}
               \section[Peter Altenberg an Arthur Schnitzler, {[}7. 11. 1912{]}]{ Peter Altenberg an Arthur Schnitzler, {[}7. 11. 1912{]}}\nopagebreak\mylabel{v}\rehead{ }\begin{ledgroupsized}[t]{13cm}\normalsize\beginnumbering \toendnotes[C]{\smallbreak\pagebreak[2]} \Standort{CUL, Schnitzler, B 2.}
\physDesc{Brief, 1 Blatt, 2 Seiten
\newline{}Handschrift: schwarze Tinte, deutsche Kurrent
\newline{}Schnitzler: mit Bleistift datiert: »7/11 912« \newline{}Ordnung: mit Bleistift von unbekannter Hand nummeriert:
                                    »10« }\Standort{DLA, A:Schnitzler, 85.1.2342, S. 9–10.}
\physDesc{Maschinenschriftliche Abschrift, 1 Blatt, 1 Seite
\newline{}Schreibmaschine
\newline{}Handschrift einer Schreibkraft: Bleistift (\noindent{}Unterstreichungen, zwei Korrekturen)\newline{}Zusatz: Die Abschrift mit Schnitzlers Schreibmaschine mit weiter
                                 Spationierung erstellt und ist womöglich kurz nach dem Tod Altenbergs
                                 entstanden. }\buchAbdrucke{\weitereDrucke{1) Kurt Bergel: \emph{Arthur Schnitzlers unveröffentlichte Tragikomödie Das Wort.} In: \emph{Studies in Arthur Schnitzler. Centennial Commemorative
                        Volume}. Hg. Herbert W. Reichert und Herman Salinger. Chapel Hill: \emph{University of North Carolina Press} 1963, S. 21 (UNC Studies in the Germanic Languages and Literatures, 42).} \weitereDrucke{2) Arthur Schnitzler: \emph{Das Wort. Tragikomödie in fünf Akten. Fragment}. Aus dem Nachlaß hg. und eingel. von Kurt Bergel. Frankfurt am Main: \emph{S. Fischer} 1966, S. 10.} \weitereDrucke{3) Hermann Bahr, Arthur Schnitzler: \emph{Briefwechsel, Aufzeichnungen, Dokumente (1891–1931)}. Hg. Kurt Ifkovits und Martin Anton Müller. Göttingen: \emph{Wallstein} 2018, S. 478.} }\toendnotes[C]{\smallbreak}\pstart
           \noindent{}{\pb}\textcolor{gray}{\textbf{Peter Altenberg}}\hfill \textcolor{gray}{\textbf{Semmering\oindex{Semmering@\textbf{Semmering}|pw}}}\pend
           \pstart
           \raggedleft{}\textcolor{gray}{\textbf{Hotel Panhans\oindex{Hotel Panhans@\textbf{Hotel Panhans}|pw}.}}\pend
           \pstart{}Lieber \textsc{D\textsuperscript{r}} Arthur Schnitzler,\pend\pstart
           ich ſchreibe es Ihnen ganz klip und klar, denn alles Andere hätte gar keinen
               Sinn:\pend
           \pstart
           Eine Reihe von Menſchen, die mich \uline{bisher} durch \uline{fixe monatliche Beiträge} unterſtützt haben, ſind
               allmälig »\uline{ausgeſprungen}«. Ich frage daher bei Ihnen,
               dem vom Schickſale Begünſtigten, an, ob Sie oder Andere (Beer-Hoffmann\pwindex{Beer-Hofmann, Richard 1866-07-11 – 1945-09-26@\textsc{Beer-Hofmann, Richard} (1866-07-11 – 1945-09-26), \emph{Schriftsteller}|pw}, Hugo
                  Hofmannstal\pwindex{Hofmannsthal, Hugo von 1874-02-01 – 1929-07-15@\textsc{Hofmannsthal, Hugo von} (1874-02-01 – 1929-07-15), \emph{Schriftsteller}|pw}, Hermann Bahr\pwindex{Bahr, Hermann 19.07.1863 – 15.01.1934@\textsc{Bahr, Hermann} (19.07.1863 – 15.01.1934), \emph{Schriftsteller, Kritiker}|pw}{ }\textsc{etc. etc.}) {\\}mir die Sorge meines
               Lebensabends{\\}(»tiefſte Lebensnacht« ſollte es eigentlich lauten) erleichtern
               wollen!?!? \introOben{}Bis zum 53. Jahre habe ich mich ſo »\uline{durchgefrettet}«.\introOben{}\pend
           \pstart
           {\pb}Ich bin ſeit 8 Wochen von einer
               »allgemeinen Nervenentzündung« (\textsc{polyneuritis}) Tag und Nacht
                  \uuline{\edtext{gefoltert}{\Cendnote{dreifach unterstrichen}}}, dazu die ſeeliſche Depreſſion!\pend
           \pstart
           Ich bitte ſehr, dieſes Schreiben als \uuline{\edtext{Geheimnis}{\Cendnote{dreifach unterstrichen}}} zu
               betrachten. \introOben{}Ich appellire an den Menſchen \uline{und} den Dichter.\introOben{}\pend
           \pstart
           Meine Tage ſind \uline{gerichtet} und gezählt, da gibt es
               keine Demütigung mehr, man iſt ſchon halb wo anders, dort wo die Beurteilungen des
               Menſchen und ſeiner Seele \uuline{\edtext{anders}{\Cendnote{dreifach unterstrichen}}} gewertet werden!\pend
           \pstart
           Ihr unglückſeliger{\\[\baselineskip]}\spacefill\mbox{Peter Altenberg}\pend
           \leftskip=0em{}\pstart
           \noindent{}Semmering, Hotel Panhans.\oindex{Hotel Panhans@\textbf{Hotel Panhans}|pw}\pend
           \pstart
           Es iſt ein \uline{Notſchrei} eines ſchwerst
                  Bedrängten.\pend
           \pstart
           \raggedleft{}Geheimnis!!!\pend
           
         
         \endnumbering\mylabel{h}\end{ledgroupsized}  \newcommand{\dateiname}{L02094}\newcommand{\titel}{Peter Altenberg an Arthur Schnitzler, [7. 11. 1912]}\newcommand{\editorInnen}{ Martin Anton Müller und Gerd-Hermann Susen}%% latex-leseansicht-abspann.tex
%% Abspann für die Leseansicht.
%% Der Schalter \ifkorrekturansicht ist bereits durch den Vorspann gesetzt.

%% latex-abspann.tex
%% Gemeinsamer Abspann für Korrekturansicht und Leseansicht.
%% Setzt den Schalter \ifkorrekturansicht voraus (gesetzt in den
%% einbindenden Dateien latex-korrekturansicht-abspann.tex bzw.
%% latex-leseansicht-abspann.tex).
%% ---------------------------------------------------------------

\normalsize

% Das esempio-Environment wird nur in der Leseansicht benötigt
\ifkorrekturansicht\else
\newenvironment{esempio}[3]%
{
    \vspace{1.5ex}
    \rlap{\underline{#1}}
    \par
    \setlength{\parindent}{0cm}
    \nopagebreak
    \leftskip=#2cm
    \rightskip=#3cm
}
{
    \par
}
\fi

\doendnotes{C}
\bigskip
\vfill

\clearpage

\footnotesize

\ifkorrekturansicht
  \lohead{\textsc{register}}
\fi

% theindex-Environment neu definieren ohne reledmac
\makeatletter
\renewenvironment{theindex}{%
  \ifkorrekturansicht
    \section*{\indexname}%
  \else
    \subsubsection*{Index der erwähnten Entitäten}%
  \fi
  \setlength{\parindent}{0pt}%
  \setlength{\parskip}{0pt plus 0.3pt}%
  \let\item\@idxitem
}{%
  \ifkorrekturansicht\clearpage\fi
}
\makeatother

\IfFileExists{\jobname-pw.ind}{\input{\jobname-pw.ind}}{}

% Quellenangabe nur in der Leseansicht
\ifkorrekturansicht\else
% Fallback-Definitionen, falls die .tex-Datei \titel etc. nicht gesetzt hat
\providecommand{\titel}{}
\providecommand{\editorInnen}{}
\providecommand{\dateiname}{\jobname}

\vspace{3cm}

\vfill

\footnotesize
\textsc{Quelle}: \titel. Herausgegeben von {\editorInnen}. In: \emph{Arthur Schnitzler: Briefwechsel mit Autorinnen und Autoren}.
 Digitale Edition, https://schnitzler-briefe.acdh.oeaw.ac.at/{\dateiname}.html (Stand \today)
\fi

\end{document}


      