%% latex-korrekturansicht-vorspann.tex
%% Vorspann für die Korrekturansicht.
%% Lädt die gemeinsame Datei latex-vorspann.tex mit gesetztem Schalter.

\newif\ifkorrekturansicht
\korrekturansichttrue

\input{../tex-inputs/latex-vorspann}


\section[Paul Goldmann an Arthur Schnitzler, 21. 11. 1896]{L02689 Paul Goldmann an Arthur Schnitzler, 21. 11. 1896}
\nopagebreak\mylabel{L02689v}
\rehead{ }\normalsize\beginnumbering\briefempfaengerindex{Schnitzler, Arthur@\textsc{Schnitzler, Arthur}!zzzGoldmann, Paul@\emph{von Paul Goldmann}!1896-11-212@{21. 11. 1896}|(be}
\toendnotes[C]{\smallbreak\pagebreak[2]}\Standort{DLA, A:Schnitzler, HS.NZ85.1.3166.}
\physDesc{Telegramm, 111 Zeichen
\newline{}maschinell
\newline{}Schnitzler: mit Bleistift datiert: »Nov. 96« 
\newline{}Ordnung: beschnitten }
\pstart
           \centering{}{\pb}wien\oindex{Wien@\textbf{Wien}, \emph{A.ADM2}|pw}{ }paris\oindex{Paris@\textbf{Paris}, \emph{P.PPLC}|pw} 8498 16 21{ }4/20’ sr –\pend
           \vspace{0.5em}
\pstart
           nach pistolenduell mit millevoye\pwindex{Millevoye, Lucien 1850-08-01 – 1918-03-25@\textsc{Millevoye, Lucien} (1850-08-01 – 1918-03-25), \emph{Politiker/Politikerin, Journalist/Journalistin}|pw}
               wohlbehalten\pend
           \pstart sende dir herzlichen gruss. – \spacefill\mbox{goldmann}\pend{}\selectlanguage{ngerman}\endnumbering\briefempfaengerindex{Schnitzler, Arthur@\textsc{Schnitzler, Arthur}!zzzGoldmann, Paul@\emph{von Paul Goldmann}!1896-11-212@{21. 11. 1896}|)be}\mylabel{L02689h}  \normalsize

\doendnotes{C}
\bigskip
\vfill

\clearpage

\footnotesize

\lohead{\textsc{register}}

% Definiere theindex-Environment komplett neu ohne reledmac
\makeatletter
\renewenvironment{theindex}{%
  \section*{\indexname}%
  \setlength{\parindent}{0pt}%
  \setlength{\parskip}{0pt plus 0.3pt}%
  \let\item\@idxitem
}{%
  \clearpage
}
\makeatother

\IfFileExists{\jobname-pw.ind}{\input{\jobname-pw.ind}}{}

\end{document}

      