%% latex-korrekturansicht-vorspann.tex
%% Vorspann für die Korrekturansicht.
%% Lädt die gemeinsame Datei latex-vorspann.tex mit gesetztem Schalter.

\newif\ifkorrekturansicht
\korrekturansichttrue

\input{../tex-inputs/latex-vorspann}


\section[Arthur Schnitzler an Hugo von Hofmannsthal, 5. 7. 1893]{L00233 Arthur Schnitzler an Hugo von Hofmannsthal, 5. 7. 1893}
\nopagebreak\mylabel{L00233v}
\rehead{ }\normalsize\beginnumbering\briefempfaengerindex{Hofmannsthal, Hugo von@\textsc{Hofmannsthal, Hugo von}!zzzSchnitzler, Arthur@\emph{von Arthur Schnitzler}!1893-07-051@{5. 7. 1893}|(be}
\toendnotes[C]{\smallbreak\pagebreak[2]}\Standort{FDH, Hs-30885,35.}
\physDesc{Brief, 1 Blatt, 3 Seiten, 471 Zeichen (Briefpapier mit Trauerrand)
\newline{}Handschrift: schwarze Tinte, deutsche Kurrent}
\buchAbdrucke{\weitereDrucke{Hugo von Hofmannsthal, Arthur Schnitzler: \emph{Briefwechsel}. Frankfurt am Main: \emph{S. Fischer} 1964, S. 39–40.} }\toendnotes[C]{\smallbreak}
\pstart
           \raggedleft{}{\pb}\textsc{Ischl, Pens. Leopold}\oindex{Hotel und Pension Rudolfshoehe (Leopold Petter)@\textbf{Hotel und Pension Rudolfshöhe (Leopold Petter)}, \emph{Hotel (K.HTL)}|pw}{\\}5/7. 93. \pend
           
\pstart{}Lieber Loris,\pend\vspace{0.5em}
\pstart
           bin in Iſchl\oindex{Hotel und Pension Rudolfshoehe (Leopold Petter)@\textbf{Hotel und Pension Rudolfshöhe (Leopold Petter)}, \emph{Hotel (K.HTL)}|pw}, war \textsc{per
                  Bic}. u. mit \textsc{Richard}\pwindex{Beer-Hofmann, Richard 1866-07-11 – 1945-09-26@\textsc{Beer-Hofmann, Richard} (1866-07-11 – 1945-09-26), \emph{Schriftsteller/Schriftstellerin}|pw} in \textsc{Strobl}\oindex{Strobl@\textbf{Strobl}, \emph{A.ADM3}|pw}, wo Sie von der Badekabinenvermietherin\pwindex{?? [Badekabinenvermieterin in Strobl] 1893 – 1893@\textsc{?? [Badekabinenvermieterin in Strobl]} (1893 – 1893)|pwv}{ }{\pb}gekannt werden u Ihr Name unorthographiſch auf den
               Brettern ſteht. –\pend
           
\pstart
           Ich bleibe etwa bis zum 14. da, wünſchte was von Ihnen zu hören und
               ſchätze Sie ſowohl als Poeten wie als Menſchen {\pb}ſehr
               hoch. –\pend
           
\pstart
           Geſchrieben hab ich wenig oder nichts oder gar nichts oder doch \label{K_L00233-1v}\edtext{etwas}{\lemma{\textnormal{\emph{etwas}}}\Cendnote{\textnormal{nicht identifiziert}}}\label{K_L00233-1}, und meine Laune iſt kühl, dumpf und
               grau mit grünen Tupfen. –\pend
           
\pstart
           Ihr entarteter{\\[\baselineskip]}\spacefill\mbox{ArthSch}\pend
           \leftskip=0em{}\selectlanguage{ngerman}\endnumbering\briefempfaengerindex{Hofmannsthal, Hugo von@\textsc{Hofmannsthal, Hugo von}!zzzSchnitzler, Arthur@\emph{von Arthur Schnitzler}!1893-07-051@{5. 7. 1893}|)be}\mylabel{L00233h}  \normalsize

\doendnotes{C}
\bigskip
\vfill

\clearpage

\footnotesize

\lohead{\textsc{register}}

% Definiere theindex-Environment komplett neu ohne reledmac
\makeatletter
\renewenvironment{theindex}{%
  \section*{\indexname}%
  \setlength{\parindent}{0pt}%
  \setlength{\parskip}{0pt plus 0.3pt}%
  \let\item\@idxitem
}{%
  \clearpage
}
\makeatother

\IfFileExists{\jobname-pw.ind}{\input{\jobname-pw.ind}}{}

\end{document}

      