%% latex-leseansicht-vorspann.tex
%% Vorspann für die Leseansicht.
%% Lädt die gemeinsame Datei latex-vorspann.tex mit nicht gesetztem Schalter.

\newif\ifkorrekturansicht
\korrekturansichtfalse

\input{../tex-inputs/latex-vorspann}


         
         \renewcommand{\erwaehntePersonen}{Personen:  ?? [Badekabinenvermieterin in Strobl], Richard Beer-Hofmann, Hugo von Hofmannsthal}
         \renewcommand{\erwaehnteOrte}{Orte: Bad Ischl, Hotel und Pension Rudolfshöhe (Leopold Petter), Strobl, Wien}
         \renewcommand{\erwaehnteWerke}{}
               \section[Arthur Schnitzler an Hugo von Hofmannsthal, 5. 7. 1893]{ Arthur Schnitzler an Hugo von Hofmannsthal, 5. 7. 1893}\nopagebreak\mylabel{v}\rehead{ }\begin{ledgroupsized}[t]{13cm}\normalsize\beginnumbering\briefempfaengerindex{Hofmannsthal, Hugo von@\textsc{Hofmannsthal, Hugo von}!zzzSchnitzler, Arthur@\emph{von Arthur Schnitzler}!1893-07-051@{5. 7. 1893}|(be} \toendnotes[C]{\smallbreak\pagebreak[2]} \Standort{FDH, Hs-30885,35.}
\physDesc{Brief, 1 Blatt, 3 Seiten, 471 Zeichen (Briefpapier mit Trauerrand)
\newline{}Handschrift: schwarze Tinte, deutsche Kurrent}\buchAbdrucke{\weitereDrucke{Hugo von Hofmannsthal, Arthur Schnitzler: \emph{Briefwechsel}. Hg. Therese Nickl und Heinrich Schnitzler. Frankfurt am Main: \emph{S. Fischer} 1964, S. 39–40.} }\toendnotes[C]{\smallbreak}\pstart
           \raggedleft{}{\pb}\textsc{Ischl, Pens. Leopold}\oindex{Hotel und Pension Rudolfshoehe (Leopold Petter)@\textbf{Hotel und Pension Rudolfshöhe (Leopold Petter)}|pw}{\\}5/7. 93. \pend
           \pstart{}Lieber Loris,\pend\pstart
           bin in Iſchl\oindex{Hotel und Pension Rudolfshoehe (Leopold Petter)@\textbf{Hotel und Pension Rudolfshöhe (Leopold Petter)}|pw}, war \textsc{per
                  Bic}. u. mit \textsc{Richard}\pwindex{Beer-Hofmann, Richard 1866-07-11 – 1945-09-26@\textsc{Beer-Hofmann, Richard} (1866-07-11 – 1945-09-26), \emph{Schriftsteller}|pw} in \textsc{Strobl}\oindex{Strobl@\textbf{Strobl}|pw}, wo Sie von der Badekabinenvermietherin\pwindex{?? [Badekabinenvermieterin in Strobl] 1893 – 1893@\textsc{?? [Badekabinenvermieterin in Strobl]} (1893 – 1893)|pwv}{ }{\pb}gekannt werden u Ihr Name unorthographiſch auf den
               Brettern ſteht. –\pend
           \pstart
           Ich bleibe etwa bis zum 14. da, wünſchte was von Ihnen zu hören und
               ſchätze Sie ſowohl als Poeten wie als Menſchen {\pb}ſehr
               hoch. –\pend
           \pstart
           Geſchrieben hab ich wenig oder nichts oder gar nichts oder doch \label{K_L00233-1v}\edtext{etwas}{\lemma{\textnormal{\emph{etwas}}}\Cendnote{\textnormal{nicht identifiziert}}}\label{K_L00233-1h}, und meine Laune iſt kühl, dumpf und
               grau mit grünen Tupfen. –\pend
           \pstart
           Ihr entarteter{\\[\baselineskip]}\spacefill\mbox{ArthSch}\pend
           \leftskip=0em{}
         
         \endnumbering\mylabel{h}\end{ledgroupsized}  \newcommand{\dateiname}{L00233}\newcommand{\titel}{Arthur Schnitzler an Hugo von Hofmannsthal, 5. 7. 1893}\newcommand{\editorInnen}{Martin Anton Müller und Gerd-Hermann Susen}%% latex-leseansicht-abspann.tex
%% Abspann für die Leseansicht.
%% Der Schalter \ifkorrekturansicht ist bereits durch den Vorspann gesetzt.

%% latex-abspann.tex
%% Gemeinsamer Abspann für Korrekturansicht und Leseansicht.
%% Setzt den Schalter \ifkorrekturansicht voraus (gesetzt in den
%% einbindenden Dateien latex-korrekturansicht-abspann.tex bzw.
%% latex-leseansicht-abspann.tex).
%% ---------------------------------------------------------------

\normalsize

% Das esempio-Environment wird nur in der Leseansicht benötigt
\ifkorrekturansicht\else
\newenvironment{esempio}[3]%
{
    \vspace{1.5ex}
    \rlap{\underline{#1}}
    \par
    \setlength{\parindent}{0cm}
    \nopagebreak
    \leftskip=#2cm
    \rightskip=#3cm
}
{
    \par
}
\fi

\doendnotes{C}
\bigskip
\vfill

\clearpage

\footnotesize

\ifkorrekturansicht
  \lohead{\textsc{register}}
\fi

% theindex-Environment neu definieren ohne reledmac
\makeatletter
\renewenvironment{theindex}{%
  \ifkorrekturansicht
    \section*{\indexname}%
  \else
    \subsubsection*{Index der erwähnten Entitäten}%
  \fi
  \setlength{\parindent}{0pt}%
  \setlength{\parskip}{0pt plus 0.3pt}%
  \let\item\@idxitem
}{%
  \ifkorrekturansicht\clearpage\fi
}
\makeatother

\IfFileExists{\jobname-pw.ind}{\input{\jobname-pw.ind}}{}

% Quellenangabe nur in der Leseansicht
\ifkorrekturansicht\else
% Fallback-Definitionen, falls die .tex-Datei \titel etc. nicht gesetzt hat
\providecommand{\titel}{}
\providecommand{\editorInnen}{}
\providecommand{\dateiname}{\jobname}

\vspace{3cm}

\vfill

\footnotesize
\textsc{Quelle}: \titel. Herausgegeben von {\editorInnen}. In: \emph{Arthur Schnitzler: Briefwechsel mit Autorinnen und Autoren}.
 Digitale Edition, https://schnitzler-briefe.acdh.oeaw.ac.at/{\dateiname}.html (Stand \today)
\fi

\end{document}


      