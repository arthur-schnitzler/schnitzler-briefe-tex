%% latex-leseansicht-vorspann.tex
%% Vorspann für die Leseansicht.
%% Lädt die gemeinsame Datei latex-vorspann.tex mit nicht gesetztem Schalter.

\newif\ifkorrekturansicht
\korrekturansichtfalse

\input{../tex-inputs/latex-vorspann}


\section[Georg Brandes an Arthur und Olga Schnitzler, 3. 2. 1912]{L02051 Georg Brandes an Arthur und Olga Schnitzler, 3. 2. 1912}
\nopagebreak\mylabel{L02051v}
\rehead{ }\normalsize\beginnumbering\briefempfaengerindex{Schnitzler, Olga@\textsc{Schnitzler, Olga}!zzzBrandes, Georg@\emph{von Georg Brandes}!1912-02-032@{3. 2. 1912}|(be}\briefempfaengerindex{Schnitzler, Arthur@\textsc{Schnitzler, Arthur}!zzzBrandes, Georg@\emph{von Georg Brandes}!1912-02-032@{3. 2. 1912}|(be}
\toendnotes[C]{\smallbreak\pagebreak[2]}
\correspDesc{Versand  durch Georg Brandes am 3. 2. 1912 in Paris
\newline{}Erhalt  durch Arthur Schnitzler, Olga Schnitzler im Zeitraum [4. 2. 1912
                  – 8. 2. 1912?] in Wien}\toendnotes[C]{\smallbreak}
\Standort{CUL, Schnitzler, B 17.}
\physDesc{Postkarte, 935 Zeichen
\newline{}Handschrift: schwarze Tinte, lateinische Kurrent
\newline{}Versand: Stempel: »\nobreak{}\oindex{Paris@\textbf{Paris}, \emph{Hauptstadt}|pwk}Paris, 3-2 12\nobreak{}«.  
\newline{}Ordnung: mit Bleistift von unbekannter Hand nummeriert:
                                    »38« }
\buchAbdrucke{\weitereDrucke{Georg Brandes, Arthur Schnitzler: \emph{Ein Briefwechsel}. Herausgegeben von Kurt Bergel. Bern: \emph{Francke} 1956, S. 104.} }\toendnotes[C]{\smallbreak}\pstart{}{\pb}\textcolor{gray}{\textbf{* Expedié par}}\pend{}\pstart{}\textcolor{gray}{\textbf{M}} Brandes\pend{}\pstart{}\textcolor{gray}{\textbf{Dem\textsuperscript{t} à}}{ }Hotel d’Jéna\oindex{Hotel d’Jéna@\textbf{Hotel d’Jéna}, \emph{Hotel}|pw}\pend{}\pstart{}Paris\oindex{Paris@\textbf{Paris}, \emph{Hauptstadt}|pw}\pend{}{\bigskip}\pstart{}\textcolor{gray}{\textbf{M}}onsieur Arthur Schnitzler\pend{}\pstart{}Sternwartestrasse 71\oindex{Wien@\textbf{Wien}!XVIII., Währing@\textbf{XVIII., Währing}!Sternwartestraße 71@\textbf{Sternwartestraße 71}, \emph{Wohngebäude}|pw}\pend{}\pstart{}Vienne\hspace*{2.5em}Autriche\pend{}{\bigskip}\vspace{1em}
\pstart
           
\pstart
           {\pb}Paris. Hotel d’Jéna\oindex{Hotel d’Jéna@\textbf{Hotel d’Jéna}, \emph{Hotel}|pw}\pend
           
\pstart
           \raggedleft{}3 Febr. 12\pend
           \pend
           
\pstart{}Verehrter Freund, verehrte Freundin\pend\vspace{0.5em}
\pstart
           Ihre lieben und schönen Portraits haben mich hier eingeholt, wohin ich geflohen bin
               um verschiedenen Festlichkeiten in Kopenhagen\oindex{Kopenhagen@\textbf{Kopenhagen}, \emph{Hauptstadt}|pw}
               zu vermeiden. Ich bin Ihnen sehr dankbar, dass auch Sie, die ich so sehr schätze, an
               mich (bei dieser schmählichen tragikomischen Gelegenheit) gedacht haben.\pend
           
\pstart
           Ihnen gegenüber ist mein Herz voll. \label{K_L02051-1v}\edtext{On
               a eu l’idée saugrenue}{\lemma{\textnormal{\emph{On
               a eu l’idée saugrenue}}}\Cendnote{\textnormal{französisch: man
                  hat eine groteske Idee gehabt}}}\label{K_L02051-1} – da ich sowohl das Rathausfest wie einem von
               der Universität und den Schriftstellern veranstalteten ausschlug – einen Saal der Kgl. Bibliotek\orgindex{Det Kongelige Bibliotek@Det Kongelige Bibliotek|pw} zu einem G. B.-Archiv\orgindex{Georg Brandes-arkiv@Georg Brandes-arkiv|pw} zu verwandeln und mit meiner Büste zu versehen.\pend
           
\pstart
           Da sollen idiotische Literaturhistoriker der Zukunft in meinen alten Liebesbriefen
               schnüffeln. Das soll mir Freude machen.\pend
           
\pstart
           Glücklicherweise für Arthur S. halten wir noch immer dieselbe Distanz von
               20 Jahren.\pend
           \pstart Ihr ergebenster \spacefill\mbox{Georg Brandes}\pend{}\selectlanguage{ngerman}\endnumbering\briefempfaengerindex{Schnitzler, Olga@\textsc{Schnitzler, Olga}!zzzBrandes, Georg@\emph{von Georg Brandes}!1912-02-032@{3. 2. 1912}|)be}\briefempfaengerindex{Schnitzler, Arthur@\textsc{Schnitzler, Arthur}!zzzBrandes, Georg@\emph{von Georg Brandes}!1912-02-032@{3. 2. 1912}|)be}\mylabel{L02051h}  \newcommand{\dateiname}{L02051}\newcommand{\titel}{Georg Brandes an Arthur und Olga Schnitzler, 3. 2. 1912}\newcommand{\editorInnen}{Martin Anton Müller und Gerd-Hermann Susen}%% latex-leseansicht-abspann.tex
%% Abspann für die Leseansicht.
%% Der Schalter \ifkorrekturansicht ist bereits durch den Vorspann gesetzt.

%% latex-abspann.tex
%% Gemeinsamer Abspann für Korrekturansicht und Leseansicht.
%% Setzt den Schalter \ifkorrekturansicht voraus (gesetzt in den
%% einbindenden Dateien latex-korrekturansicht-abspann.tex bzw.
%% latex-leseansicht-abspann.tex).
%% ---------------------------------------------------------------

\normalsize

% Das esempio-Environment wird nur in der Leseansicht benötigt
\ifkorrekturansicht\else
\newenvironment{esempio}[3]%
{
    \vspace{1.5ex}
    \rlap{\underline{#1}}
    \par
    \setlength{\parindent}{0cm}
    \nopagebreak
    \leftskip=#2cm
    \rightskip=#3cm
}
{
    \par
}
\fi

\doendnotes{C}
\bigskip
\vfill

\clearpage

\footnotesize

\ifkorrekturansicht
  \lohead{\textsc{register}}
\fi

% theindex-Environment neu definieren ohne reledmac
\makeatletter
\renewenvironment{theindex}{%
  \ifkorrekturansicht
    \section*{\indexname}%
  \else
    \subsubsection*{Index der erwähnten Entitäten}%
  \fi
  \setlength{\parindent}{0pt}%
  \setlength{\parskip}{0pt plus 0.3pt}%
  \let\item\@idxitem
}{%
  \ifkorrekturansicht\clearpage\fi
}
\makeatother

\IfFileExists{\jobname-pw.ind}{\input{\jobname-pw.ind}}{}

% Quellenangabe nur in der Leseansicht
\ifkorrekturansicht\else
% Fallback-Definitionen, falls die .tex-Datei \titel etc. nicht gesetzt hat
\providecommand{\titel}{}
\providecommand{\editorInnen}{}
\providecommand{\dateiname}{\jobname}

\vspace{3cm}

\vfill

\footnotesize
\textsc{Quelle}: \titel. Herausgegeben von {\editorInnen}. In: \emph{Arthur Schnitzler: Briefwechsel mit Autorinnen und Autoren}.
 Digitale Edition, https://schnitzler-briefe.acdh.oeaw.ac.at/{\dateiname}.html (Stand \today)
\fi

\end{document}


