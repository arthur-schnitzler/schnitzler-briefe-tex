\input{../tex-inputs/latex-pdf-vorspann}
\begin{center}
            \textcolor{red}{ENTWURF. ENTZIFFERUNG NOCH NICHT KORREKTURGELESEN}
                      \end{center}
            
               \section[Hugo von Hofmannsthal an Arthur Schnitzler, 12. 10. {[}1898{]}]{ Hugo von Hofmannsthal an Arthur Schnitzler, 12. 10. {[}1898{]}}\nopagebreak\mylabel{v}\rehead{ }\begin{ledgroupsized}[t]{13cm}\normalsize\beginnumbering\briefempfaengerindex{Schnitzler, Arthur@\textsc{Schnitzler, Arthur}!zzzHofmannsthal, Hugo von@\emph{von Hugo von Hofmannsthal}!1898-10-121@{12. 10. {[}1898{]}}|(be} \toendnotes[C]{\smallbreak\pagebreak[2]} \Standort{CUL, Schnitzler, B 43.}
\physDesc{Brief, 1 Blatt, 3 Seiten
\newline{}Handschrift: schwarze Tinte, deutsche Kurrent
\newline{}Schnitzler: mit Bleistift die Jahreszahl ergänzt: »98« \newline{}Ordnung: mit Bleistift von unbekannter Hand nummeriert:
                                        »135« }\buchAbdrucke{\weitereDrucke{Hugo von Hofmannsthal, Arthur Schnitzler: \emph{Briefwechsel}. Hg. Therese Nickl und Heinrich Schnitzler. Frankfurt am Main: \emph{S. Fischer} 1964, S. 112.} }\toendnotes[C]{\smallbreak}\pstart
           {\pb}12. X.\hfill Gießhüblerstraße 2\oindex{Giesshueblerstrasse@\textbf{Gießhüblerstraße}|pw}\pend
           \pstart{}mein lieber Arthur\pend\pstart
           ich bin überaus froh, daſs es in Berlin\oindex{Berlin@\textbf{Berlin}|pw}{ }ſo
                    abſolut gut gegangen iſt, denn ich habe für den zweiten und dritten Act\pwindex{Schnitzler, Arthur 15.05.1862 – 21.10.1931@\textsc{Schnitzler, Arthur} (15.05.1862 – 21.10.1931), \emph{Schriftsteller, Mediziner}!Vermaechtnis. Schauspiel in drei Akten1898@\strich\emph{Das Vermächtnis. Schauspiel in drei Akten} {[}1898{]}|pwv} große Angſt gehabt.\hspace*{1.5em}Mein venezianiſches\oindex{Venedig@\textbf{Venedig}|pw} halb-ernſtes Stück\pwindex{Hofmannsthal, Hugo von 01.02.1874 – 15.07.1929@\textsc{Hofmannsthal, Hugo von} (01.02.1874 – 15.07.1929), \emph{Schriftsteller}!Abenteurer und die Saengerin oder Die Geschenke des Lebens18. 3. 1899@\strich\emph{Der Abenteurer und die Sängerin oder Die Geschenke des Lebens} {[}18. 3. 1899{]}|pwv} iſt nahezu fertig. Ich bin nun noch für 5–6 Tage
                    hier, weil es ſo wunderſchön iſt, zwiſchen {\pb}den purpurrothen und gelben
                    Bäumen radzufahren. Es wäre ſo lieb von Ihnen wenn Sie einen der Wochentage in
                    der Früh herauskämen und bis zum Dunkelwerden hier blieben. Sie wiſſen daſs die
                        Schleſingers\pwindex{Schlesinger, Emil 10.05.1844 – 31.05.1899@\textsc{Schlesinger, Emil} (10.05.1844 – 31.05.1899), \emph{Bankdirektor}|pw}\pwindex{Schlesinger, Franziska 17.08.1851 – 11.08.1932@\textsc{Schlesinger, Franziska} (17.08.1851 – 11.08.1932)|pw} darin keinen auf
                    ſie bezüglichen Beſuch {\pb}ſehen. Ich hätte eine ſehr große Freude darüber. Sie müſsten nur den Abend
                    vorher telegraphieren.\pend
           \pstart
           Von Herzen Ihr{\\[\baselineskip]}\spacefill\mbox{Hugo.}\pend
           \leftskip=0em{}\endnumbering\briefempfaengerindex{Schnitzler, Arthur@\textsc{Schnitzler, Arthur}!zzzHofmannsthal, Hugo von@\emph{von Hugo von Hofmannsthal}!1898-10-121@{12. 10. {[}1898{]}}|)be}\mylabel{h}\end{ledgroupsized}  \newcommand{\dateiname}{L00851}\newcommand{\titel}{Hugo von Hofmannsthal an Arthur Schnitzler, 12. 10. [1898]}\newcommand{\editorInnen}{Martin Anton Müller und Gerd-Hermann Susen}\input{../tex-inputs/latex-pdf-abspann}
      