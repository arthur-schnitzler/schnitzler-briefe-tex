%% latex-korrekturansicht-vorspann.tex
%% Vorspann für die Korrekturansicht.
%% Lädt die gemeinsame Datei latex-vorspann.tex mit gesetztem Schalter.

\newif\ifkorrekturansicht
\korrekturansichttrue

\input{../tex-inputs/latex-vorspann}


\section[Hugo von Hofmannsthal an Arthur Schnitzler, 12. 10. {[}1898{]}]{L00851 Hugo von Hofmannsthal an Arthur Schnitzler, 12. 10. {[}1898{]}}
\nopagebreak\mylabel{L00851v}
\rehead{ }\normalsize\beginnumbering\briefempfaengerindex{Schnitzler, Arthur@\textsc{Schnitzler, Arthur}!zzzHofmannsthal, Hugo von@\emph{von Hugo von Hofmannsthal}!1898-10-121@{12. 10. {[}1898{]}}|(be}
\toendnotes[C]{\smallbreak\pagebreak[2]}\Standort{CUL, Schnitzler, B 43.}
\physDesc{Brief, 1 Blatt, 3 Seiten, 653 Zeichen
\newline{}Handschrift: schwarze Tinte, deutsche Kurrent
\newline{}Schnitzler: mit Bleistift die Jahreszahl ergänzt: »98« 
\newline{}Ordnung: mit Bleistift von unbekannter Hand nummeriert:
                                    »135« }
\buchAbdrucke{\weitereDrucke{Hugo von Hofmannsthal, Arthur Schnitzler: \emph{Briefwechsel}. Frankfurt am Main: \emph{S. Fischer} 1964, S. 112.} }\toendnotes[C]{\smallbreak}
\pstart
           
\pstart
           {\pb}12. X.\pend
           
\pstart
           \raggedleft{}Gießhüblerstraße 2\oindex{Giesshueblerstrasse@\textbf{Gießhüblerstraße}, \emph{Straße (K.STR)}|pw}\pend
           \pend
           
\pstart{}mein lieber Arthur\pend\vspace{0.5em}
\pstart
           ich bin überaus froh, daſs es in Berlin\oindex{Berlin@\textbf{Berlin}, \emph{P.PPLC}|pw}{ }ſo abſolut gut gegangen iſt, denn ich habe für den
               zweiten und dritten Act\pwindex{Vermaechtnis. Schauspiel in drei Akten@\emph{Das Vermächtnis. Schauspiel in drei Akten}|pwv} große
               Angſt gehabt.\hspace*{1.5em}Mein venezianiſches\oindex{Venedig@\textbf{Venedig}, \emph{P.PPLA}|pw} halb-ernſtes Stück\pwindex{Abenteurer und die Saengerin oder Die Geschenke des Lebens@\emph{Der Abenteurer und die Sängerin oder Die Geschenke des Lebens}|pwv} iſt nahezu fertig. Ich bin nun noch für 5–6 Tage
               hier, weil es ſo wunderſchön iſt, zwiſchen {\pb}den purpurrothen und gelben Bäumen
               radzufahren. Es wäre ſo lieb von Ihnen wenn Sie einen der Wochentage in der Früh
               herauskämen und bis zum Dunkelwerden hier blieben. Sie wiſſen daſs die Schleſingers\pwindex{Schlesinger, Emil 10.05.1844 – 31.05.1899@\textsc{Schlesinger, Emil} (10.05.1844 – 31.05.1899), \emph{Bankdirektor/Bankdirektorin}|pw}\pwindex{Schlesinger, Franziska 17.08.1851 – 11.08.1932@\textsc{Schlesinger, Franziska} (17.08.1851 – 11.08.1932)|pw} darin keinen auf ſie
               bezüglichen Beſuch {\pb}ſehen. Ich
               hätte eine ſehr große Freude darüber. Sie müſsten nur den Abend vorher
               telegraphieren.\pend
           
\pstart
           Von Herzen Ihr{\\[\baselineskip]}\spacefill\mbox{Hugo.}\pend
           \leftskip=0em{}\selectlanguage{ngerman}\endnumbering\briefempfaengerindex{Schnitzler, Arthur@\textsc{Schnitzler, Arthur}!zzzHofmannsthal, Hugo von@\emph{von Hugo von Hofmannsthal}!1898-10-121@{12. 10. {[}1898{]}}|)be}\mylabel{L00851h}  \normalsize

\doendnotes{C}
\bigskip
\vfill

\clearpage

\footnotesize

\lohead{\textsc{register}}

% Definiere theindex-Environment komplett neu ohne reledmac
\makeatletter
\renewenvironment{theindex}{%
  \section*{\indexname}%
  \setlength{\parindent}{0pt}%
  \setlength{\parskip}{0pt plus 0.3pt}%
  \let\item\@idxitem
}{%
  \clearpage
}
\makeatother

\IfFileExists{\jobname-pw.ind}{\input{\jobname-pw.ind}}{}

\end{document}

      