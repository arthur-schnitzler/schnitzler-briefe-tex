%% latex-leseansicht-vorspann.tex
%% Vorspann für die Leseansicht.
%% Lädt die gemeinsame Datei latex-vorspann.tex mit nicht gesetztem Schalter.

\newif\ifkorrekturansicht
\korrekturansichtfalse

\input{../tex-inputs/latex-vorspann}


\section[Hugo von Hofmannsthal an Arthur Schnitzler, 12. 10. {[}1898{]}]{L00851 Hugo von Hofmannsthal an Arthur Schnitzler, 12. 10. [1898]}
\nopagebreak\mylabel{L00851v}
\rehead{ }\normalsize\beginnumbering\briefempfaengerindex{Schnitzler, Arthur@\textsc{Schnitzler, Arthur}!zzzHofmannsthal, Hugo von@\emph{von Hugo von Hofmannsthal}!1898-10-121@{12. 10. [1898]}|(be}
\toendnotes[C]{\smallbreak\pagebreak[2]}
\correspDesc{Versand  durch Hugo von Hofmannsthal am 12. 10. [1898] in Hinterbrühl
\newline{}Erhalt  durch Arthur Schnitzler im Zeitraum [13. 10. 1898 – 17. 10. 1898?] in Wien}\toendnotes[C]{\smallbreak}
\Standort{CUL, Schnitzler, B 43.}
\physDesc{Brief, 1 Blatt, 3 Seiten, 653 Zeichen
\newline{}Handschrift: schwarze Tinte, deutsche Kurrent
\newline{}Schnitzler: mit Bleistift die Jahreszahl ergänzt: »98« 
\newline{}Ordnung: mit Bleistift von unbekannter Hand nummeriert:
                                    »135« }
\buchAbdrucke{\weitereDrucke{Hugo von Hofmannsthal, Arthur Schnitzler: \emph{Briefwechsel}. Herausgegeben von Therese Nickl und Heinrich Schnitzler. Frankfurt am Main: \emph{S. Fischer} 1964, S. 112.} }\toendnotes[C]{\smallbreak}
\pstart
           
\pstart
           {\pb}12. X.\pend
           
\pstart
           \raggedleft{}Gießhüblerstraße 2\oindex{Gießhüblerstraße@\textbf{Gießhüblerstraße}, \emph{Straße}|pw}\pend
           \pend
           
\pstart{}mein lieber Arthur\pend\vspace{0.5em}
\pstart
           ich bin überaus froh, daſs es in Berlin\oindex{Berlin@\textbf{Berlin}, \emph{Hauptstadt}|pw}{ }ſo abſolut gut gegangen iſt, denn ich habe für den
               zweiten und dritten Act\pwindex{Schnitzler, Arthur 15.\,5.\,1862 Wien – 21.\,10.\,1931 ebd.@\textsc{Schnitzler, Arthur} (15.\,5.\,1862 Wien – 21.\,10.\,1931 ebd.), \emph{Schriftsteller, Mediziner}!Vermächtnis. Schauspiel in drei Akten@\strich\emph{Das Vermächtnis. Schauspiel in drei Akten}|pwv} große
               Angſt gehabt.\hspace*{1.5em}Mein venezianiſches\oindex{Venedig@\textbf{Venedig}|pw} halb-ernſtes Stück\pwindex{Hofmannsthal, Hugo von 1.\,2.\,1874 Wien – 15.\,7.\,1929 Rodaun@\textsc{Hofmannsthal, Hugo von} (1.\,2.\,1874 Wien – 15.\,7.\,1929 Rodaun), \emph{Schriftsteller}!Abenteurer und die Sängerin oder Die Geschenke des Lebens@\strich\emph{Der Abenteurer und die Sängerin oder Die Geschenke des Lebens}|pwv} iſt nahezu fertig. Ich bin nun noch für 5–6 Tage
               hier, weil es{ }ſo wunderſchön iſt, zwiſchen {\pb}den purpurrothen und gelben Bäumen
               radzufahren. Es wäre{ }ſo lieb von Ihnen wenn Sie einen der Wochentage in der Früh
               herauskämen und bis zum Dunkelwerden hier blieben. Sie wiſſen daſs die Schleſingers\pwindex{Schlesinger, Emil 10.\,5.\,1844 Wien – 31.\,5.\,1899 ebd.@\textsc{Schlesinger, Emil} (10.\,5.\,1844 Wien – 31.\,5.\,1899 ebd.), \emph{Bankdirektor}|pw}\pwindex{Schlesinger, Franziska 17.\,8.\,1851 Wien – 11.\,8.\,1932 ebd.@\textsc{Schlesinger, Franziska} (17.\,8.\,1851 Wien – 11.\,8.\,1932 ebd.)|pw} darin keinen auf{ }ſie
               bezüglichen Beſuch {\pb}ſehen. Ich
               hätte eine{ }ſehr große Freude darüber. Sie müſsten nur den Abend vorher
               telegraphieren.\pend
           
\pstart
           Von Herzen Ihr{\\[\baselineskip]}\spacefill\mbox{Hugo.}\pend
           \leftskip=0em{}\selectlanguage{ngerman}\endnumbering\briefempfaengerindex{Schnitzler, Arthur@\textsc{Schnitzler, Arthur}!zzzHofmannsthal, Hugo von@\emph{von Hugo von Hofmannsthal}!1898-10-121@{12. 10. [1898]}|)be}\mylabel{L00851h}  \newcommand{\dateiname}{L00851}\newcommand{\titel}{Hugo von Hofmannsthal an Arthur Schnitzler, 12. 10. [1898]}\newcommand{\editorInnen}{Martin Anton Müller und Gerd-Hermann Susen}%% latex-leseansicht-abspann.tex
%% Abspann für die Leseansicht.
%% Der Schalter \ifkorrekturansicht ist bereits durch den Vorspann gesetzt.

%% latex-abspann.tex
%% Gemeinsamer Abspann für Korrekturansicht und Leseansicht.
%% Setzt den Schalter \ifkorrekturansicht voraus (gesetzt in den
%% einbindenden Dateien latex-korrekturansicht-abspann.tex bzw.
%% latex-leseansicht-abspann.tex).
%% ---------------------------------------------------------------

\normalsize

% Das esempio-Environment wird nur in der Leseansicht benötigt
\ifkorrekturansicht\else
\newenvironment{esempio}[3]%
{
    \vspace{1.5ex}
    \rlap{\underline{#1}}
    \par
    \setlength{\parindent}{0cm}
    \nopagebreak
    \leftskip=#2cm
    \rightskip=#3cm
}
{
    \par
}
\fi

\doendnotes{C}
\bigskip
\vfill

\clearpage

\footnotesize

\ifkorrekturansicht
  \lohead{\textsc{register}}
\fi

% theindex-Environment neu definieren ohne reledmac
\makeatletter
\renewenvironment{theindex}{%
  \ifkorrekturansicht
    \section*{\indexname}%
  \else
    \subsubsection*{Index der erwähnten Entitäten}%
  \fi
  \setlength{\parindent}{0pt}%
  \setlength{\parskip}{0pt plus 0.3pt}%
  \let\item\@idxitem
}{%
  \ifkorrekturansicht\clearpage\fi
}
\makeatother

\IfFileExists{\jobname-pw.ind}{\input{\jobname-pw.ind}}{}

% Quellenangabe nur in der Leseansicht
\ifkorrekturansicht\else
% Fallback-Definitionen, falls die .tex-Datei \titel etc. nicht gesetzt hat
\providecommand{\titel}{}
\providecommand{\editorInnen}{}
\providecommand{\dateiname}{\jobname}

\vspace{3cm}

\vfill

\footnotesize
\textsc{Quelle}: \titel. Herausgegeben von {\editorInnen}. In: \emph{Arthur Schnitzler: Briefwechsel mit Autorinnen und Autoren}.
 Digitale Edition, https://schnitzler-briefe.acdh.oeaw.ac.at/{\dateiname}.html (Stand \today)
\fi

\end{document}


