%% latex-korrekturansicht-vorspann.tex
%% Vorspann für die Korrekturansicht.
%% Lädt die gemeinsame Datei latex-vorspann.tex mit gesetztem Schalter.

\newif\ifkorrekturansicht
\korrekturansichttrue

\input{../tex-inputs/latex-vorspann}


\section[Arthur Schnitzler an Charlotte Ehrenstein, {[}vor dem 21. 5.? 1906{]}]{L01598 Arthur Schnitzler an Charlotte Ehrenstein, {[}vor dem
               21. 5.? 1906{]}}
\nopagebreak\mylabel{L01598v}
\rehead{ }\normalsize\beginnumbering\briefempfaengerindex{Ehrenstein, Charlotte@\textsc{Ehrenstein, Charlotte}!zzzSchnitzler, Arthur@\emph{von Arthur Schnitzler}!1906-05-201@{{[}vor dem 21. 5.? 1906{]}}|(be}
\toendnotes[C]{\smallbreak\pagebreak[2]}\Standort{Jerusalem, The National Library of Israel, ARC. Ms. Var. 306 1 118.}
\physDesc{Briefkarte, 331 Zeichen
\newline{}Handschrift: schwarze Tinte, deutsche Kurrent
\newline{}Ordnung: mit Bleistift von unbekannter Hand nummeriert:
                                 »2« }\toendnotes[C]{\smallbreak}
\pstart
           {\pb}\textcolor{gray}{\textbf{Dr. Arthur Schnitzler}}{\\}\textcolor{gray}{\textbf{Wien, XVIII. Spoettelgasse 7\oindex{Edmund-Weiss-Gasse 7@\textbf{Edmund-Weiß-Gasse 7}, \emph{Wohngebäude (K.WHS)}|pw}.}}\pend
           \vspace{0.5em}
\pstart
           Sehr geehrte gnädige Frau, ich danke für Ihre freundlichen
               Nachrichten u freue mich, daſs Albert\pwindex{Ehrenstein, Albert 23.12.1886 – 08.04.1950@\textsc{Ehrenstein, Albert} (23.12.1886 – 08.04.1950), \emph{Schriftsteller/Schriftstellerin}|pw}{ }ſich vollſtändig \label{K_L01598-1v}\edtext{erholt}{\lemma{\textnormal{\emph{erholt}}}\Cendnote{\textnormal{Vgl. A. S.: \emph{Tagebuch}, 21. 6. 1906. Entsprechend
                  dürfte dieser Brief, der den Besuch ankündigt, kurz zuvor geschickt worden
                  sein.}}}\label{K_L01598-1} hat. Sobald es meine Zeit erlaubt, werde ich ſo frei ſein, mich
               perſönlich nach ſeinem {\pb}Befinden zu erkundigen.\pend
           
\pstart
           Grüßen Sie ihn beſtens. Meine Empfehlungen Ihnen gnädige Frau und dem Herrn Gemahl\pwindex{Ehrenstein, Alexander 29.03.1857 – 29.05.1925@\textsc{Ehrenstein, Alexander} (29.03.1857 – 29.05.1925), \emph{Kassier/Kassierin}|pwv}.\pend
           
\pstart
           Ihr ſehr ergebener{\\[\baselineskip]}\spacefill\mbox{A. S.}\pend
           \leftskip=0em{}\selectlanguage{ngerman}\endnumbering\briefempfaengerindex{Ehrenstein, Charlotte@\textsc{Ehrenstein, Charlotte}!zzzSchnitzler, Arthur@\emph{von Arthur Schnitzler}!1906-05-201@{{[}vor dem 21. 5.? 1906{]}}|)be}\mylabel{L01598h}  \normalsize

\doendnotes{C}
\bigskip
\vfill

\clearpage

\footnotesize

\lohead{\textsc{register}}

% Definiere theindex-Environment komplett neu ohne reledmac
\makeatletter
\renewenvironment{theindex}{%
  \section*{\indexname}%
  \setlength{\parindent}{0pt}%
  \setlength{\parskip}{0pt plus 0.3pt}%
  \let\item\@idxitem
}{%
  \clearpage
}
\makeatother

\IfFileExists{\jobname-pw.ind}{\input{\jobname-pw.ind}}{}

\end{document}

      