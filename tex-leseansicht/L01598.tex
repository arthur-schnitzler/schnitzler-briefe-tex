%% latex-leseansicht-vorspann.tex
%% Vorspann für die Leseansicht.
%% Lädt die gemeinsame Datei latex-vorspann.tex mit nicht gesetztem Schalter.

\newif\ifkorrekturansicht
\korrekturansichtfalse

\input{../tex-inputs/latex-vorspann}


\section[Arthur Schnitzler an Charlotte Ehrenstein, {{[}}vor dem 21. 5.? 1906{{]}}]{L01598 Arthur Schnitzler an Charlotte Ehrenstein, {[}vor dem 21. 5.? 1906{]}}
\nopagebreak\mylabel{L01598v}
\rehead{ }\normalsize\beginnumbering\briefempfaengerindex{Ehrenstein, Charlotte@\textsc{Ehrenstein, Charlotte}!zzzSchnitzler, Arthur@\emph{von Arthur Schnitzler}!1906-05-201@{{[}vor dem 21. 5.? 1906{]}}|(be}
\toendnotes[C]{\smallbreak\pagebreak[2]}
\correspDesc{Versand  durch Arthur Schnitzler am [vor dem 21. 5.? 1906] in Wien
\newline{}Erhalt  durch Charlotte Ehrenstein im Zeitraum [20. 5. 1906
                  – 24. 5. 1906?] in Wien}\toendnotes[C]{\smallbreak}
\Standort{Jerusalem, The National Library of Israel, ARC. Ms. Var. 306 1 118.}
\physDesc{Briefkarte, 331 Zeichen
\newline{}Handschrift: schwarze Tinte, deutsche Kurrent
\newline{}Ordnung: mit Bleistift von unbekannter Hand nummeriert:
                                 »2« }\toendnotes[C]{\smallbreak}
\pstart
           {\pb}\textcolor{gray}{\textbf{Dr. Arthur Schnitzler}}{\\}\textcolor{gray}{\textbf{Wien, XVIII. Spoettelgasse 7\oindex{Wien@\textbf{Wien}!XVIII., Währing@\textbf{XVIII., Währing}!Edmund-Weiß-Gasse 7@\textbf{Edmund-Weiß-Gasse 7}, \emph{Wohngebäude}|pw}.}}\pend
           \vspace{0.5em}
\pstart
           Sehr geehrte gnädige Frau, ich danke für Ihre freundlichen
               Nachrichten u freue mich, daſs Albert\pwindex{Ehrenstein, Albert 23.\,12.\,1886 Wien – 8.\,4.\,1950 New York City@\textsc{Ehrenstein, Albert} (23.\,12.\,1886 Wien – 8.\,4.\,1950 New York City), \emph{Schriftsteller}|pw}{ }ſich vollſtändig \label{K_L01598-1v}\edtext{erholt}{\lemma{\textnormal{\emph{erholt}}}\Cendnote{\textnormal{Vgl. A. S.: \emph{Tagebuch}, 21. 6. 1906. Entsprechend
                  dürfte dieser Brief, der den Besuch ankündigt, kurz zuvor geschickt worden
                  sein.}}}\label{K_L01598-1} hat. Sobald es meine Zeit erlaubt, werde ich{ }ſo frei{ }ſein, mich
               perſönlich nach{ }ſeinem {\pb}Befinden zu erkundigen.\pend
           
\pstart
           Grüßen Sie ihn beſtens. Meine Empfehlungen Ihnen gnädige Frau und dem Herrn Gemahl\pwindex{Ehrenstein, Alexander 29.\,3.\,1857 Skalice – 29.\,5.\,1925 Wien@\textsc{Ehrenstein, Alexander} (29.\,3.\,1857 Skalice – 29.\,5.\,1925 Wien), \emph{Kassier}|pwv}.\pend
           
\pstart
           Ihr{ }ſehr ergebener{\\[\baselineskip]}\spacefill\mbox{A. S.}\pend
           \leftskip=0em{}\selectlanguage{ngerman}\endnumbering\briefempfaengerindex{Ehrenstein, Charlotte@\textsc{Ehrenstein, Charlotte}!zzzSchnitzler, Arthur@\emph{von Arthur Schnitzler}!1906-05-201@{{[}vor dem 21. 5.? 1906{]}}|)be}\mylabel{L01598h}  \newcommand{\dateiname}{L01598}\newcommand{\titel}{Arthur Schnitzler an Charlotte Ehrenstein, [vor dem 21. 5.? 1906]}\newcommand{\editorInnen}{Martin Anton Müller und Gerd-Hermann Susen}%% latex-leseansicht-abspann.tex
%% Abspann für die Leseansicht.
%% Der Schalter \ifkorrekturansicht ist bereits durch den Vorspann gesetzt.

%% latex-abspann.tex
%% Gemeinsamer Abspann für Korrekturansicht und Leseansicht.
%% Setzt den Schalter \ifkorrekturansicht voraus (gesetzt in den
%% einbindenden Dateien latex-korrekturansicht-abspann.tex bzw.
%% latex-leseansicht-abspann.tex).
%% ---------------------------------------------------------------

\normalsize

% Das esempio-Environment wird nur in der Leseansicht benötigt
\ifkorrekturansicht\else
\newenvironment{esempio}[3]%
{
    \vspace{1.5ex}
    \rlap{\underline{#1}}
    \par
    \setlength{\parindent}{0cm}
    \nopagebreak
    \leftskip=#2cm
    \rightskip=#3cm
}
{
    \par
}
\fi

\doendnotes{C}
\bigskip
\vfill

\clearpage

\footnotesize

\ifkorrekturansicht
  \lohead{\textsc{register}}
\fi

% theindex-Environment neu definieren ohne reledmac
\makeatletter
\renewenvironment{theindex}{%
  \ifkorrekturansicht
    \section*{\indexname}%
  \else
    \subsubsection*{Index der erwähnten Entitäten}%
  \fi
  \setlength{\parindent}{0pt}%
  \setlength{\parskip}{0pt plus 0.3pt}%
  \let\item\@idxitem
}{%
  \ifkorrekturansicht\clearpage\fi
}
\makeatother

\IfFileExists{\jobname-pw.ind}{\input{\jobname-pw.ind}}{}

% Quellenangabe nur in der Leseansicht
\ifkorrekturansicht\else
% Fallback-Definitionen, falls die .tex-Datei \titel etc. nicht gesetzt hat
\providecommand{\titel}{}
\providecommand{\editorInnen}{}
\providecommand{\dateiname}{\jobname}

\vspace{3cm}

\vfill

\footnotesize
\textsc{Quelle}: \titel. Herausgegeben von {\editorInnen}. In: \emph{Arthur Schnitzler: Briefwechsel mit Autorinnen und Autoren}.
 Digitale Edition, https://schnitzler-briefe.acdh.oeaw.ac.at/{\dateiname}.html (Stand \today)
\fi

\end{document}


