%% latex-leseansicht-vorspann.tex
%% Vorspann für die Leseansicht.
%% Lädt die gemeinsame Datei latex-vorspann.tex mit nicht gesetztem Schalter.

\newif\ifkorrekturansicht
\korrekturansichtfalse

\input{../tex-inputs/latex-vorspann}


\section[ Paul Goldmann an Arthur Schnitzler, 8. 9. 1905]{L03236 Paul Goldmann an Arthur Schnitzler,  8. 9. 1905}
\nopagebreak\mylabel{L03236v}
\rehead{ }\normalsize\beginnumbering\briefempfaengerindex{Schnitzler, Arthur@\textsc{Schnitzler, Arthur}!zzzGoldmann, Paul@\emph{von Paul Goldmann}!1905-09-081@{8. 9. 1905}|(be}
\toendnotes[C]{\smallbreak\pagebreak[2]}
\correspDesc{Versand  durch Paul Goldmann am 8. 9. 1905 in Tremezzo
\newline{}Erhalt  durch Arthur Schnitzler im Zeitraum [9. 9. 1905
                  – 13. 9. 1905?] in Wien}\toendnotes[C]{\smallbreak}
\Standort{DLA, A:Schnitzler, HS.NZ85.1.3175.}
\physDesc{Bildpostkarte, 142 Zeichen
\newline{}Handschrift: schwarze Tinte, deutsche Kurrent
\newline{}Versand: Stempel: »\nobreak{}\oindex{Tremezzo@\textbf{Tremezzo}, \emph{Hauptstadt}|pwk}Tremezzo (Como), 8 9 05\nobreak{}«.  
\newline{}Schnitzler: mit Bleistift das Jahr »90\textcolor{gray}{5}« vermerkt }\toendnotes[C]{\smallbreak}\pstart{}\textsc{{\pb}Herrn}\pend{}\pstart{}\textsc{Dr. Arthur Schnitzler}\pend{}\pstart{}\textsc{XVIII. Spöttelgaſse 7\oindex{Wien@\textbf{Wien}!XVIII., Währing@\textbf{XVIII., Währing}!Edmund-Weiß-Gasse 7@\textbf{Edmund-Weiß-Gasse 7}, \emph{Wohngebäude}|pw}}\pend{}\pstart{}\textsc{Wien\oindex{Wien@\textbf{Wien}, \emph{Verwaltungsgebiet}|pw}.}\pend{}\pstart{}\textsc{\strikeout{Ge}{ }\begin{otherlanguage}{italian}Austria\end{otherlanguage}\oindex{Österreich-Ungarn@\textbf{Österreich-Ungarn}|pw}}\pend{}{\bigskip}
\pstart
           \noindent{}\centering{}\label{K_L03236-1v}\edtext{\begin{otherlanguage}{italian}\textcolor{gray}{\textbf{Sul lago\oindex{Lago di Como@\textbf{Lago di Como}, \emph{See}|pw}.}}\end{otherlanguage}}{\lemma{\textnormal{\emph{Sul lago.}}}\Cendnote{\textnormal{Italienisch: auf dem See. Zu sehen
                  ist der Como See\oindex{Lago di Como@\textbf{Lago di Como}, \emph{See}|pwk} mit einem Boot mit
                  Sonnendach, vier Rudern und zwei Männern, die diese stehend bedienen.}}}\label{K_L03236-1}\pend
           \vspace{1em}
\pstart
           {\pb}Tremezzo\oindex{Tremezzo@\textbf{Tremezzo}, \emph{Hauptstadt}|pw}{ }8. Sept.\pend
           \vspace{0.5em}
\pstart
           Herzliche Grüße Dir, lieber Freund, und Deiner Frau\pwindex{Schnitzler, Olga 17.\,1.\,1882 Wien – 13.\,1.\,1970 Lugano@\textsc{Schnitzler, Olga} (17.\,1.\,1882 Wien – 13.\,1.\,1970 Lugano), \emph{Schauspielerin, Sängerin}|pwv}! {\\}\spacefill\mbox{Paul Goldmann.}\pend
           \selectlanguage{ngerman}\endnumbering\briefempfaengerindex{Schnitzler, Arthur@\textsc{Schnitzler, Arthur}!zzzGoldmann, Paul@\emph{von Paul Goldmann}!1905-09-081@{8. 9. 1905}|)be}\mylabel{L03236h}  \newcommand{\dateiname}{L03236}\newcommand{\titel}{Paul Goldmann an Arthur Schnitzler, 8. 9. 1905}\newcommand{\editorInnen}{Martin Anton Müller und Laura Untner}%% latex-leseansicht-abspann.tex
%% Abspann für die Leseansicht.
%% Der Schalter \ifkorrekturansicht ist bereits durch den Vorspann gesetzt.

%% latex-abspann.tex
%% Gemeinsamer Abspann für Korrekturansicht und Leseansicht.
%% Setzt den Schalter \ifkorrekturansicht voraus (gesetzt in den
%% einbindenden Dateien latex-korrekturansicht-abspann.tex bzw.
%% latex-leseansicht-abspann.tex).
%% ---------------------------------------------------------------

\normalsize

% Das esempio-Environment wird nur in der Leseansicht benötigt
\ifkorrekturansicht\else
\newenvironment{esempio}[3]%
{
    \vspace{1.5ex}
    \rlap{\underline{#1}}
    \par
    \setlength{\parindent}{0cm}
    \nopagebreak
    \leftskip=#2cm
    \rightskip=#3cm
}
{
    \par
}
\fi

\doendnotes{C}
\bigskip
\vfill

\clearpage

\footnotesize

\ifkorrekturansicht
  \lohead{\textsc{register}}
\fi

% theindex-Environment neu definieren ohne reledmac
\makeatletter
\renewenvironment{theindex}{%
  \ifkorrekturansicht
    \section*{\indexname}%
  \else
    \subsubsection*{Index der erwähnten Entitäten}%
  \fi
  \setlength{\parindent}{0pt}%
  \setlength{\parskip}{0pt plus 0.3pt}%
  \let\item\@idxitem
}{%
  \ifkorrekturansicht\clearpage\fi
}
\makeatother

\IfFileExists{\jobname-pw.ind}{\input{\jobname-pw.ind}}{}

% Quellenangabe nur in der Leseansicht
\ifkorrekturansicht\else
% Fallback-Definitionen, falls die .tex-Datei \titel etc. nicht gesetzt hat
\providecommand{\titel}{}
\providecommand{\editorInnen}{}
\providecommand{\dateiname}{\jobname}

\vspace{3cm}

\vfill

\footnotesize
\textsc{Quelle}: \titel. Herausgegeben von {\editorInnen}. In: \emph{Arthur Schnitzler: Briefwechsel mit Autorinnen und Autoren}.
 Digitale Edition, https://schnitzler-briefe.acdh.oeaw.ac.at/{\dateiname}.html (Stand \today)
\fi

\end{document}


