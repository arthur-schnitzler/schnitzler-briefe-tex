%% latex-leseansicht-vorspann.tex
%% Vorspann für die Leseansicht.
%% Lädt die gemeinsame Datei latex-vorspann.tex mit nicht gesetztem Schalter.

\newif\ifkorrekturansicht
\korrekturansichtfalse

\input{../tex-inputs/latex-vorspann}


\section[Theodor Herzl an Arthur Schnitzler, 13. 11. 1894]{L03836 Theodor Herzl an Arthur Schnitzler, 13. 11. 1894}
\nopagebreak\mylabel{L03836v}
\rehead{ }\normalsize\beginnumbering\briefempfaengerindex{Schnitzler, Arthur@\textsc{Schnitzler, Arthur}!zzzHerzl, Theodor@\emph{von Theodor Herzl}!1894-11-131@{13. 11. 1894}|(be}
\toendnotes[C]{\smallbreak\pagebreak[2]}
\correspDesc{Versand  durch Theodor Herzl am 13. 11. 1894 in Paris
\newline{}Erhalt  durch Arthur Schnitzler im Zeitraum [14. 11. 1894 – 18. 11. 1894?] in Wien}\toendnotes[C]{\smallbreak}
\Standort{CUL, Schnitzler, B 39.}
\physDesc{Brief, 2 Blätter, 5 Seiten, 5584 Zeichen
\newline{}Handschrift: schwarze Tinte, lateinische Kurrent (\noindent{}Nummerierung des zweiten Bogens: »2 Bl.«)
\newline{}Ordnung: 1) mit Bleistift von unbekannter Hand nummeriert: »15«  2) mit blauem Buntstift mutmaßlich von Leon
                                    Kellner\pwindex{Kellner, Leon 17.\,4.\,1859 Tarnów – 5.\,12.\,1928 Wien@\textsc{Kellner, Leon} (17.\,4.\,1859 Tarnów – 5.\,12.\,1928 Wien), \emph{Zionist, Literaturhistoriker, Anglist}|pw} Markierung interessanter Stellen 3) mit rotem Buntstift eine Anstreichung}
\buchAbdrucke{\weitereDrucke{Theodor Herzl: \emph{Briefe und
                        autobiographische Notizen 1866–1895}. Bearbeitet von Johannes Wachten in Zusammenarbeit mit Chaya Harel, Daisy Tycho und Manfred Winkler. Berlin, Frankfurt am Main, Wien: \emph{Propyläen} 1983, S. 555–557 (Briefe und Tagebücher. Herausgegeben von Alex Bein, Hermann Greive, Moshe Schaerf, Julius H. Schoeps und Johannes Wachten, 1).} }\toendnotes[C]{\smallbreak}
\pstart
           \raggedleft{}{\pb}Paris\oindex{Paris@\textbf{Paris}, \emph{Hauptstadt}|pw}{ }8 November 894\pend
           
\pstart{}Mein lieber Schnitzer!\pend\vspace{0.5em}
\pstart
           Ich habe gar nicht daran gezweifelt, dass Sie mit mir gehen werden. Ich danke Ihnen
               herzlich für Ihre Bereitwilligkeit. Es bleibt also bei dem neulich Entwickelten. Sie
               sind und bleiben bis auf meinen Widerruf der Einzige, der \label{K_L03836-1v}\edtext{Albert Schnabels Geheimniss}{\lemma{\textnormal{\emph{Albert … Geheimniss}}}\Cendnote{\textnormal{Unter diesem Pseudonym sollte Arthur Schnitzler für Herzl\pwindex{Herzl, Theodor 2.\,5.\,1860 Budapest – 3.\,7.\,1904 Edlach@\textsc{Herzl, Theodor} (2.\,5.\,1860 Budapest – 3.\,7.\,1904 Edlach), \emph{Schriftsteller, Journalist}|pwk} das Schauspiel \emph{Das neue Ghetto}\pwindex{Herzl, Theodor 2.\,5.\,1860 Budapest – 3.\,7.\,1904 Edlach@\textsc{Herzl, Theodor} (2.\,5.\,1860 Budapest – 3.\,7.\,1904 Edlach), \emph{Schriftsteller, Journalist}!neue Ghetto. Schauspiel in vier Acten@\strich\emph{Das neue Ghetto. Schauspiel in vier Acten}|pwk}
                  an verschiedenen Theatern\orgindex{Deutsches Theater Berlin@Deutsches Theater Berlin|pwkv}\orgindex{Lessing-Theater@Lessing-Theater|pwkv}\orgindex{Berliner Theater@Berliner Theater|pwkv}\orgindex{Neues Theater@Neues Theater|pwkv}\orgindex{Freie Bühne@Freie Bühne|pwkv}
                  einreichen, siehe XXXX Auszeichnungsfehler: Dokument L03835 nicht gefunden.}}}\label{K_L03836-1}
               kennt. Auch von meiner Familie\pwindex{Herzl, Julie 1.\,2.\,1868 Budapest – 10.\,11.\,1907 Wien@\textsc{Herzl, Julie} (1.\,2.\,1868 Budapest – 10.\,11.\,1907 Wien)|pwv}\pwindex{Herzl, Jakob 14.\,3.\,1837 Zemun – 9.\,6.\,1902 Wien@\textsc{Herzl, Jakob} (14.\,3.\,1837 Zemun – 9.\,6.\,1902 Wien), \emph{Bankdirektor, Großkaufmann}|pwv}\pwindex{Herzl, Jeanette 28.\,7.\,1836 Budapest – 20.\,2.\,1911 Wien@\textsc{Herzl, Jeanette} (28.\,7.\,1836 Budapest – 20.\,2.\,1911 Wien)|pwv} weiss gar Niemand davon. Frauen können nicht
               schweigen, u. ein Geheimniss darf höchstens auf vier Augen stehen.\pend
           
\pstart
           Ich gebe Ihnen keinerlei Detailwünsche zur Wahrung des Geheimnisses bekannt: dass Sie
               fortab alle meine Briefe unter Verschluss halten etc. Niemand erfährt wer Albert
               Schnabel ist, auch der Notar\pwindex{Schik, Friedrich *~6.\,9.\,1857 Wien@\textsc{Schik, Friedrich} (*~6.\,9.\,1857 Wien), \emph{Notar, Journalist, Dramaturg}|pwv}
               nicht, dem ich auf Ihre Worte hin übrigens mein ganzes Vertrauen zuwende. Natürlich
               nehmen Sie ihm das schriftliche Ehrenwort ab, dass er auch Sie Niemandem nennt, denn
               dann käme man allmälig auf die Spur. Dieses \label{K_L03836-2v}\edtext{\begin{otherlanguage}{english}Rallye-Paper\end{otherlanguage}}{\lemma{\textnormal{\emph{Rallye-Paper}}}\Cendnote{\textnormal{englisch: Schnitzeljagd}}}\label{K_L03836-2} auf einen
               anonymen Autor habe ich mit den Wilddieben\pwindex{Herzl, Theodor 2.\,5.\,1860 Budapest – 3.\,7.\,1904 Edlach@\textsc{Herzl, Theodor} (2.\,5.\,1860 Budapest – 3.\,7.\,1904 Edlach), \emph{Schriftsteller, Journalist}!Wilddiebe. Lustspiel in vier Akten@\strich\emph{Wilddiebe. Lustspiel in vier Akten}|pw}\pwindex{\textcolor{red}{\textsuperscript{XXXX indx1}}!Wilddiebe. Lustspiel in vier Akten@\strich\emph{Wilddiebe. Lustspiel in vier Akten}|pw}
               durchgemacht – theils belustigt, theils geärgert. Belustigt, als mich meine Freunde
               angriffen oder meine Feinde lobten – die Wiener allg.
                  Ztg.\orgindex{Wiener Allgemeine Zeitung@Wiener Allgemeine Zeitung|pw} bei der ich damals war brachte \label{K_L03836-3v}\edtext{die schärfsten Angriffe}{\lemma{\textnormal{\emph{die schärfsten Angriffe}}}\Cendnote{\textnormal{vgl. die durchwachsene Kritik der Uraufführung\eventindex{Burgtheater@\textbf{Burgtheater}!Uraufführung von Wilddiebe@Uraufführung von Wilddiebe|pwk} von Oskar Berggruen\pwindex{Berggruen, Oskar 1842 – 1903@\textsc{Berggruen, Oskar} (1842 – 1903), \emph{Journalist, Musikkritiker, Kunstkritiker}|pwk}: \emph{Burgtheater. Erste Aufführung des Lustspiels »Wilddiebe«}. In: \emph{Wiener Allgemeine Zeitung}\pwindex{Wiener Allgemeine Zeitung@\emph{Wiener Allgemeine Zeitung}|pwk}, Nr. 3246,
                     22. 3. 1889, S. 3–4 oder den launigen Artikel über die Suche nach dem
                  anonymen Verfasser von Heinrich Kana\pwindex{Kana, Heinrich 1857 – 1891@\textsc{Kana, Heinrich} (1857 – 1891), \emph{Schriftsteller, Journalist}|pwk}: \emph{Wie ich
                        den Verfasser der »Wilddiebe« auffand}. In: \emph{Wiener Allgemeine Zeitung}\pwindex{Wiener Allgemeine Zeitung@\emph{Wiener Allgemeine Zeitung}|pwk}, Nr. 3243, 19. 3. 1889,
                     S. 3–4.}}}\label{K_L03836-3} etc. – Dem Notar\pwindex{Schik, Friedrich *~6.\,9.\,1857 Wien@\textsc{Schik, Friedrich} (*~6.\,9.\,1857 Wien), \emph{Notar, Journalist, Dramaturg}|pwv} werden Sie bei Vertragsschlüssen Winke aus Ihrer
               Bühnenerfahrung geben – {\pb}Sie haben ja
               auch schon welche, mein lieber Freund – und werden in vorsichtigem Verkehr mit ihm
               bleiben. Er muss auch schlau schweigen.\pend
           
\pstart
           Zunächst will ich also Ihre Meinung über Albert Schnabels 4 actiges Schauspiel »Das Ghetto\pwindex{Herzl, Theodor 2.\,5.\,1860 Budapest – 3.\,7.\,1904 Edlach@\textsc{Herzl, Theodor} (2.\,5.\,1860 Budapest – 3.\,7.\,1904 Edlach), \emph{Schriftsteller, Journalist}!neue Ghetto. Schauspiel in vier Acten@\strich\emph{Das neue Ghetto. Schauspiel in vier Acten}|pw}« hören. In diesem besonderen Falle
               haben Sie mir mit aller Brutalität die Wahrheit zu sagen. Finden Sie es schlecht –
               heraus damit! Ich vertrage einen Puff. Dieses Pseudonymat ist ja für mich der
               Kugelpanzer. Schon lebe ich in der Schnabelschen Fiction, u. weder Lob noch Tadel
               wird mich daraus vertreiben. Mein Gedanke ist 4 – 5 Stücke in 4 – 5 Jahren auf diese
               Weise zu publiciren – u. mich während der Zeit als Journalist verachten oder von
               Reclamehubern im Arsch lecken zu lassen.\pend
           
\pstart
           Zum Stück\pwindex{Herzl, Theodor 2.\,5.\,1860 Budapest – 3.\,7.\,1904 Edlach@\textsc{Herzl, Theodor} (2.\,5.\,1860 Budapest – 3.\,7.\,1904 Edlach), \emph{Schriftsteller, Journalist}!neue Ghetto. Schauspiel in vier Acten@\strich\emph{Das neue Ghetto. Schauspiel in vier Acten}|pwv} habe ich nur zu
               bemerken, dass ich Ihnen den \label{K_L03836-4v}\edtext{\begin{otherlanguage}{french}premier jet\end{otherlanguage}}{\lemma{\textnormal{\emph{premier jet}}}\Cendnote{\textnormal{französisch: erster Entwurf}}}\label{K_L03836-4}{ }geschickt. So ist es ganz fertig. Es fehlt nur eine Scene, die ich aus Fachwerken
               construiren muss. Der Inhalt ist bestimmt und für Sie kurz hineingedeutet. Flüchtig
               gemacht ist nur eine Stelle im 2 Art: der Samuelische Grundsatz, den ich natürlich
               auch in eine definitive u. kürzere als die jetzige Form bringen werde.\pend
           
\pstart
           Sonst hängen nur ganz kleine Fetzen, die abzuschleifen ich bei der Reinschrift\pwindex{Herzl, Theodor 2.\,5.\,1860 Budapest – 3.\,7.\,1904 Edlach@\textsc{Herzl, Theodor} (2.\,5.\,1860 Budapest – 3.\,7.\,1904 Edlach), \emph{Schriftsteller, Journalist}!neue Ghetto. Schauspiel in vier Acten@\strich\emph{Das neue Ghetto. Schauspiel in vier Acten}|pwv} Gelegenheit nehme. Diese Reinschrift\pwindex{Herzl, Theodor 2.\,5.\,1860 Budapest – 3.\,7.\,1904 Edlach@\textsc{Herzl, Theodor} (2.\,5.\,1860 Budapest – 3.\,7.\,1904 Edlach), \emph{Schriftsteller, Journalist}!neue Ghetto. Schauspiel in vier Acten@\strich\emph{Das neue Ghetto. Schauspiel in vier Acten}|pwv} habe ich noch nicht
               gemacht, weil ich auch auf Ihre Bedenken Rücksicht nehmen will. Das mache {\pb}ich dann alles zusammen. Ich bitte Sie
               also mir das Manuscript\pwindex{Herzl, Theodor 2.\,5.\,1860 Budapest – 3.\,7.\,1904 Edlach@\textsc{Herzl, Theodor} (2.\,5.\,1860 Budapest – 3.\,7.\,1904 Edlach), \emph{Schriftsteller, Journalist}!neue Ghetto. Schauspiel in vier Acten@\strich\emph{Das neue Ghetto. Schauspiel in vier Acten}|pwv} gleich
               nachdem Sie damit fertig sind, begleitet von einem ausführlichen u. männlich
               rücksichtslosen Gutachten zurückzuschicken. Adresse:\pend
           
\pstart
           \centering{}Monsieur Théodore Herzl\pend
           
\pstart
           \raggedleft{}\uline{Paris}\oindex{Paris@\textbf{Paris}, \emph{Hauptstadt}|pw}, poste restante \begin{otherlanguage}{french}au Bureau 37\end{otherlanguage}\pend
           
\pstart
           \numberlinefalse{}\centering{}–\numberlinetrue{}\pend
           
\pstart
           Und zwar als recommandirten Brief, wie es meine Sendung war.\pend
           
\pstart
           Wichtigere u. recommandirte Sendungen werde ich immer unter dieser Adresse von Ihnen
               erwarten. Sie sehen, Sie werden mit mir Mühe haben. Eine solche Sendung müssen Sie
               mir immer in einem gewöhnlichen Brief anzeigen in Worten, die nur ich verstehe:
               »Heute habe ich an Albert geschrieben«, wird für mich die Formel sein dass im Bureau
               37 ein Brief erliegt.\pend
           
\pstart
           Diese Uebervorsicht ist nöthig, denn bei den Wilddieben\pwindex{Herzl, Theodor 2.\,5.\,1860 Budapest – 3.\,7.\,1904 Edlach@\textsc{Herzl, Theodor} (2.\,5.\,1860 Budapest – 3.\,7.\,1904 Edlach), \emph{Schriftsteller, Journalist}!Wilddiebe. Lustspiel in vier Akten@\strich\emph{Wilddiebe. Lustspiel in vier Akten}|pw}\pwindex{\textcolor{red}{\textsuperscript{XXXX indx1}}!Wilddiebe. Lustspiel in vier Akten@\strich\emph{Wilddiebe. Lustspiel in vier Akten}|pw} ist das Geheimniss durch meine Familie\pwindex{Herzl, Julie 1.\,2.\,1868 Budapest – 10.\,11.\,1907 Wien@\textsc{Herzl, Julie} (1.\,2.\,1868 Budapest – 10.\,11.\,1907 Wien)|pwv}\pwindex{Herzl, Jakob 14.\,3.\,1837 Zemun – 9.\,6.\,1902 Wien@\textsc{Herzl, Jakob} (14.\,3.\,1837 Zemun – 9.\,6.\,1902 Wien), \emph{Bankdirektor, Großkaufmann}|pwv}\pwindex{Herzl, Jeanette 28.\,7.\,1836 Budapest – 20.\,2.\,1911 Wien@\textsc{Herzl, Jeanette} (28.\,7.\,1836 Budapest – 20.\,2.\,1911 Wien)|pwv} ausgeplaudert
               worden. Schrieben Sie nur recommandirt an meine Domicil-Adresse\oindex{8, rue de Monceau@\textbf{8, rue de Monceau}, \emph{Wohngebäude}|pwv}, so hätte ich bei meiner Frau\pwindex{Herzl, Julie 1.\,2.\,1868 Budapest – 10.\,11.\,1907 Wien@\textsc{Herzl, Julie} (1.\,2.\,1868 Budapest – 10.\,11.\,1907 Wien)|pwv} unzählige Verhöre zu bestehen. Das
               würde mich auf die Dauer sehr nervös machen.\pend
           
\pstart
           Mittheilungen, die leicht unter Anspielungen versteckt werden können, schreiben Sie
                  {\pb}mir nur in gewöhnlichen Briefen. Die
               müssen so gefasst sein, dass Sie auch verloren gehen können. Ich werde mich nicht
               dabei aufhalten, einem Schreibkünstler Anleitungen für unseren Geheimschlüssel zu
               geben.\pend
           
\pstart
           Was Ihre Besorgnisse wegen der Theaterdirectoren betrifft, \introOben{}die\introOben{} theile ich \strikeout{die} nicht. Ja, wenn ich zu
               den Patronen betteln ginge, würden Sie mich nicht anhören. Ich werde Sie aber von
               vornherein an die Wand drücken. Das darf ich als Mitglied einer einflussreichen Zeitung\orgindex{Neue Freie Presse@Neue Freie Presse|pwv} nicht thun – es sähe wie
               Erpressung aus – aber Albert Schnabel, der Unbekannte, kann das in aller Reinheit und
               Schroffheit thun. Auch dazu ist mein Pseudonym gut. Ich werde zwei Tage nach
               Absendung des Manuscripts\pwindex{Herzl, Theodor 2.\,5.\,1860 Budapest – 3.\,7.\,1904 Edlach@\textsc{Herzl, Theodor} (2.\,5.\,1860 Budapest – 3.\,7.\,1904 Edlach), \emph{Schriftsteller, Journalist}!neue Ghetto. Schauspiel in vier Acten@\strich\emph{Das neue Ghetto. Schauspiel in vier Acten}|pwv}
               folgenden Brief an den ersten Director\pwindex{Brahm, Otto 5.\,2.\,1856 Hamburg – 28.\,11.\,1912 Berlin@\textsc{Brahm, Otto} (5.\,2.\,1856 Hamburg – 28.\,11.\,1912 Berlin), \emph{Theaterleiter, Regisseur}|pwv} (zur Weitergabe schicken). »An die Directoren der unten \label{K_L03836-5v}\edtext{benannten Bühnen}{\lemma{\textnormal{\emph{benannten Bühnen}}}\Cendnote{\textnormal{Das Stück\pwindex{Herzl, Theodor 2.\,5.\,1860 Budapest – 3.\,7.\,1904 Edlach@\textsc{Herzl, Theodor} (2.\,5.\,1860 Budapest – 3.\,7.\,1904 Edlach), \emph{Schriftsteller, Journalist}!neue Ghetto. Schauspiel in vier Acten@\strich\emph{Das neue Ghetto. Schauspiel in vier Acten}|pwkv} sollte Herzls\pwindex{Herzl, Theodor 2.\,5.\,1860 Budapest – 3.\,7.\,1904 Edlach@\textsc{Herzl, Theodor} (2.\,5.\,1860 Budapest – 3.\,7.\,1904 Edlach), \emph{Schriftsteller, Journalist}|pwk} ursprünglichen Plan gemäß in Berlin\oindex{Berlin@\textbf{Berlin}, \emph{Hauptstadt}|pwk} zunächst beim \emph{Deutschen
                                 Theater}\orgindex{Deutsches Theater Berlin@Deutsches Theater Berlin|pwk} eingereicht werden, darauf beim \emph{Lessing-Theater}\orgindex{Lessing-Theater@Lessing-Theater|pwk}, beim \emph{Neuen Theater}\orgindex{Neues Theater@Neues Theater|pwk}
                     und schließlich bei der \emph{Freien Bühne}\orgindex{Freie Bühne@Freie Bühne|pwk}.}}}\label{K_L03836-5}.
               Der Verfasser reicht sein Werk\pwindex{Herzl, Theodor 2.\,5.\,1860 Budapest – 3.\,7.\,1904 Edlach@\textsc{Herzl, Theodor} (2.\,5.\,1860 Budapest – 3.\,7.\,1904 Edlach), \emph{Schriftsteller, Journalist}!neue Ghetto. Schauspiel in vier Acten@\strich\emph{Das neue Ghetto. Schauspiel in vier Acten}|pwv}
               dem ersten, dann dem zweiten etc. ein. Verschiedene Gründe können den einen oder
               anderen Director verhindern, es aufzunehmen. Dann soll er es weitergeben. Der später
               drankommt, möge sich erinnern, dass viele Werke, die später Erfolg hatten, zuerst
               zurückgewiesen wurden, und es ohne Vorurtheil lesen. Nimmt es keiner an, so wird es
               veröffentlicht, mit diesem Brief {\pb}und mit
               den Namen der Directoren, die es ablehnten. In einer unbegreiflichen Gutmüthigkeit
               oder Schamhaftigkeit verschwiegen bisher die Bühnendichter ihre Ablehnungen. Der
               Verfasser möchte darin etwas Neues einführen: die Verantwortlichkeit der
               Theaterleiter. Wer ein Stück abdehnt, soll dafür einstehen. Der Gerechte und von
               literarischen Rücksichten Geleitete hat das nicht zu fürchten. Aber auch die
               Irrthümer der Bühnenleiter sollen bekannt werden: Das verspricht schätzbares
               Material. Das Publicum wird erfahren, was abgelehnt worden, nachdem es schon wusste,
               was gespielt wird.\pend
           
\pstart
           \raggedleft{}Der Verfasser«\pend
           
\pstart
           Und so werd’ ich das durchführen, bis ans Ende, mein lieber Freund. Zum Schluss wird
               das Stück\pwindex{Herzl, Theodor 2.\,5.\,1860 Budapest – 3.\,7.\,1904 Edlach@\textsc{Herzl, Theodor} (2.\,5.\,1860 Budapest – 3.\,7.\,1904 Edlach), \emph{Schriftsteller, Journalist}!neue Ghetto. Schauspiel in vier Acten@\strich\emph{Das neue Ghetto. Schauspiel in vier Acten}|pwv} gedruckt – aber ich
               glaube, es wird sich eine Bühne dafür finden. Ich weiss nicht, ob es ein gutes Stück
               ist - aber ich fühle, dass es ein nothwendiges ist. Was sagen Sie?\pend
           
\pstart
           Herzlich Ihr ergebener{\\[\baselineskip]}\spacefill\mbox{Th. H.}\pend
           \leftskip=0em{}\selectlanguage{ngerman}\endnumbering\briefempfaengerindex{Schnitzler, Arthur@\textsc{Schnitzler, Arthur}!zzzHerzl, Theodor@\emph{von Theodor Herzl}!1894-11-131@{13. 11. 1894}|)be}\mylabel{L03836h}
\begin{anhang}
\end{anhang}\newcommand{\dateiname}{L03836}\newcommand{\titel}{Theodor Herzl an Arthur Schnitzler, 13. 11. 1894}\newcommand{\editorInnen}{Selma Jahnke und Martin Anton Müller}%% latex-leseansicht-abspann.tex
%% Abspann für die Leseansicht.
%% Der Schalter \ifkorrekturansicht ist bereits durch den Vorspann gesetzt.

%% latex-abspann.tex
%% Gemeinsamer Abspann für Korrekturansicht und Leseansicht.
%% Setzt den Schalter \ifkorrekturansicht voraus (gesetzt in den
%% einbindenden Dateien latex-korrekturansicht-abspann.tex bzw.
%% latex-leseansicht-abspann.tex).
%% ---------------------------------------------------------------

\normalsize

% Das esempio-Environment wird nur in der Leseansicht benötigt
\ifkorrekturansicht\else
\newenvironment{esempio}[3]%
{
    \vspace{1.5ex}
    \rlap{\underline{#1}}
    \par
    \setlength{\parindent}{0cm}
    \nopagebreak
    \leftskip=#2cm
    \rightskip=#3cm
}
{
    \par
}
\fi

\doendnotes{C}
\bigskip
\vfill

\clearpage

\footnotesize

\ifkorrekturansicht
  \lohead{\textsc{register}}
\fi

% theindex-Environment neu definieren ohne reledmac
\makeatletter
\renewenvironment{theindex}{%
  \ifkorrekturansicht
    \section*{\indexname}%
  \else
    \subsubsection*{Index der erwähnten Entitäten}%
  \fi
  \setlength{\parindent}{0pt}%
  \setlength{\parskip}{0pt plus 0.3pt}%
  \let\item\@idxitem
}{%
  \ifkorrekturansicht\clearpage\fi
}
\makeatother

\IfFileExists{\jobname-pw.ind}{\input{\jobname-pw.ind}}{}

% Quellenangabe nur in der Leseansicht
\ifkorrekturansicht\else
% Fallback-Definitionen, falls die .tex-Datei \titel etc. nicht gesetzt hat
\providecommand{\titel}{}
\providecommand{\editorInnen}{}
\providecommand{\dateiname}{\jobname}

\vspace{3cm}

\vfill

\footnotesize
\textsc{Quelle}: \titel. Herausgegeben von {\editorInnen}. In: \emph{Arthur Schnitzler: Briefwechsel mit Autorinnen und Autoren}.
 Digitale Edition, https://schnitzler-briefe.acdh.oeaw.ac.at/{\dateiname}.html (Stand \today)
\fi

\end{document}


