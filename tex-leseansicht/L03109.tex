%% latex-leseansicht-vorspann.tex
%% Vorspann für die Leseansicht.
%% Lädt die gemeinsame Datei latex-vorspann.tex mit nicht gesetztem Schalter.

\newif\ifkorrekturansicht
\korrekturansichtfalse

\input{../tex-inputs/latex-vorspann}

\begin{center}
            \textcolor{red}{ENTWURF, NICHT FERTIG KORRIGIERT}
                      \end{center}
            
         
         \renewcommand{\erwaehntePersonen}{Personen: Leopold Weiss}
         \renewcommand{\erwaehnteOrte}{Orte: Wien}
         \renewcommand{\erwaehnteWerke}{Werke: Anatol, Muza}
               \section[Felix Salten an Arthur Schnitzler, {[}25. 5. 1892{]}]{ Felix Salten an Arthur Schnitzler, {[}25. 5. 1892{]}}\nopagebreak\mylabel{v}\rehead{ }\begin{ledgroupsized}[t]{13cm}\normalsize\beginnumbering \toendnotes[C]{\smallbreak\pagebreak[2]} \Standort{CUL, Schnitzler, B 89, A 1.}
\physDesc{Brief, 1 Blatt, 3 Seiten
\newline{}Handschrift: Bleistift, lateinische Kurrent
\newline{}Schnitzler: mit Bleistift datiert: »25/5 92« \newline{}Ordnung: mit Bleistift von unbekannter Hand nummeriert: »11« }\toendnotes[C]{\smallbreak}\pstart
           \noindent{}{\pb}Lieber Freund! Seit morgens ½ 11 Uhr arbeite ich unausgesetzt, und
               gedenke auch noch bis Abends zu arbeiten, da ich sehr \label{K_L03109-1v}\edtext{en train}{\lemma{\textnormal{\emph{en train}}}\Cendnote{\textnormal{französisch: im Zug, im Sinne von: es sehr gut läuft}}}\label{K_L03109-1h} bin. Es ist aber
               möglich, dass ich früher ermatte. \pend
           \pstart
           Ins Theater kann ich ja doch nicht mit Ihnen gehen, da ich keinen Sitz habe. Es wäre
               mir lieb, wenn ich Sie abends im Café {\pb}treffen könnte, da \strikeout{\textcolor{gray}{au}} ja morgen Feiertag ist, und Sie
               länger Zeit haben, ich also vielleicht auf eine Stunde zu Ihnen hinauf gehen kann um
               Ihnen meine \label{K_L03109-2v}\edtext{Geschichte\pwindex{Salten, Felix 06.09.1869 – 08.10.1945@\textsc{Salten, Felix} (06.09.1869 – 08.10.1945), \emph{Schriftsteller, Journalist}!MuzaNone@\strich\emph{Muza} {[}None{]}|pwv} vorzulesen}{\lemma{\textnormal{\emph{Geschichte vorzulesen}}}\Cendnote{\textnormal{vgl. A. S.: \emph{Tagebuch}, 27. 5. 1892}}}\label{K_L03109-2h},
               damit ich die Sache endlich los habe. \pend
           \pstart
           Noch eine Bitte! Vielleicht gestatten es Ihre Umstände, mir nach f 5– zu leihen.
               Ich bin sehr {\pb}betrübt darüber, daß ich Sie so übermäßig strapazieren muß,
               aber meine klägliche Situation dürfte sich erst mit nächstem Monate ein wenig
               bessern.\pend
           \pstart
           Waren Sie heute bei \label{K_L03109-3v}\edtext{Weiss\pwindex{Weiss, Leopold *~1853@\textsc{Weiss, Leopold} (*~1853), \emph{Verleger}|pw}}{\lemma{\textnormal{\emph{Weiss}}}\Cendnote{\textnormal{Schnitzler\pwindex{Schnitzler, Arthur 15.05.1862 – 21.10.1931@\textsc{Schnitzler, Arthur} (15.05.1862 – 21.10.1931), \emph{Schriftsteller, Mediziner}|pwk} war
            auf der Suche nach einem Verlag für \emph{Anatol}\pwindex{Schnitzler, Arthur 15.05.1862 – 21.10.1931@\textsc{Schnitzler, Arthur} (15.05.1862 – 21.10.1931), \emph{Schriftsteller, Mediziner}!Anatol1892-10-29@\strich\emph{Anatol} {[}1892-10-29{]}|pwk}, bekam mündlich eine Zusage, aber am 
               18. 6. 1892 eine schriftliche Absage.}}}\label{K_L03109-3h}? »Ist mit dem Manne etwas
               anzufangen?»\pend
           \pstart
           Herzlichst{\\[\baselineskip]}Ihr{\\[\baselineskip]}\spacefill\mbox{FelixSalten}\pend
           \leftskip=0em{}
         
         \endnumbering\mylabel{h}\end{ledgroupsized}\begin{anhang}\end{anhang}\newcommand{\dateiname}{L03109}\newcommand{\titel}{Felix Salten an Arthur Schnitzler, [25. 5. 1892]}\newcommand{\editorInnen}{Martin Anton Müller und Laura Untner}%% latex-leseansicht-abspann.tex
%% Abspann für die Leseansicht.
%% Der Schalter \ifkorrekturansicht ist bereits durch den Vorspann gesetzt.

%% latex-abspann.tex
%% Gemeinsamer Abspann für Korrekturansicht und Leseansicht.
%% Setzt den Schalter \ifkorrekturansicht voraus (gesetzt in den
%% einbindenden Dateien latex-korrekturansicht-abspann.tex bzw.
%% latex-leseansicht-abspann.tex).
%% ---------------------------------------------------------------

\normalsize

% Das esempio-Environment wird nur in der Leseansicht benötigt
\ifkorrekturansicht\else
\newenvironment{esempio}[3]%
{
    \vspace{1.5ex}
    \rlap{\underline{#1}}
    \par
    \setlength{\parindent}{0cm}
    \nopagebreak
    \leftskip=#2cm
    \rightskip=#3cm
}
{
    \par
}
\fi

\doendnotes{C}
\bigskip
\vfill

\clearpage

\footnotesize

\ifkorrekturansicht
  \lohead{\textsc{register}}
\fi

% theindex-Environment neu definieren ohne reledmac
\makeatletter
\renewenvironment{theindex}{%
  \ifkorrekturansicht
    \section*{\indexname}%
  \else
    \subsubsection*{Index der erwähnten Entitäten}%
  \fi
  \setlength{\parindent}{0pt}%
  \setlength{\parskip}{0pt plus 0.3pt}%
  \let\item\@idxitem
}{%
  \ifkorrekturansicht\clearpage\fi
}
\makeatother

\IfFileExists{\jobname-pw.ind}{\input{\jobname-pw.ind}}{}

% Quellenangabe nur in der Leseansicht
\ifkorrekturansicht\else
% Fallback-Definitionen, falls die .tex-Datei \titel etc. nicht gesetzt hat
\providecommand{\titel}{}
\providecommand{\editorInnen}{}
\providecommand{\dateiname}{\jobname}

\vspace{3cm}

\vfill

\footnotesize
\textsc{Quelle}: \titel. Herausgegeben von {\editorInnen}. In: \emph{Arthur Schnitzler: Briefwechsel mit Autorinnen und Autoren}.
 Digitale Edition, https://schnitzler-briefe.acdh.oeaw.ac.at/{\dateiname}.html (Stand \today)
\fi

\end{document}


      