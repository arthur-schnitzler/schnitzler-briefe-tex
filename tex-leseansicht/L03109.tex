%% latex-korrekturansicht-vorspann.tex
%% Vorspann für die Korrekturansicht.
%% Lädt die gemeinsame Datei latex-vorspann.tex mit gesetztem Schalter.

\newif\ifkorrekturansicht
\korrekturansichttrue

\input{../tex-inputs/latex-vorspann}


\section[Felix Salten an Arthur Schnitzler, {[}25. 5. 1892{]}]{L03109 Felix Salten an Arthur Schnitzler, {[}25. 5. 1892{]}}
\nopagebreak\mylabel{L03109v}
\rehead{ }\normalsize\beginnumbering\briefempfaengerindex{Schnitzler, Arthur@\textsc{Schnitzler, Arthur}!zzzSalten, Felix@\emph{von Felix Salten}!1892-05-251@{{[}25. 5. 1892{]}}|(be}
\toendnotes[C]{\smallbreak\pagebreak[2]}\Standort{CUL, Schnitzler, B 89, A 1.}
\physDesc{Brief, 1 Blatt, 3 Seiten, 817 Zeichen
\newline{}Handschrift: Bleistift, lateinische Kurrent
\newline{}Schnitzler: mit Bleistift datiert: »25/5 92« 
\newline{}Ordnung: mit Bleistift von unbekannter Hand nummeriert: »11« }\toendnotes[C]{\smallbreak}
\pstart
           \noindent{}{\pb}lieber Freund! Seit morgens ½ 11 Uhr
               arbeite ich unausgesetzt, und gedenke auch noch bis abends zu arbeiten,
               da ich sehr \label{K_L03109-1v}\edtext{\begin{otherlanguage}{french}en train\end{otherlanguage}}{\lemma{\textnormal{\emph{en train}}}\Cendnote{\textnormal{französisch: im Zug, im Sinne von: ›es
                  läuft sehr gut‹}}}\label{K_L03109-1} bin. Es ist aber möglich, dass ich früher ermatte.\pend
           
\pstart
           Ins \label{K_L03109-2v}\edtext{Theater\oindex{Internationales Ausstellungstheater im k.k. Prater@\textbf{Internationales Ausstellungstheater im k.k. Prater}, \emph{Theater (K.THE)}|pwv}}{\lemma{\textnormal{\emph{Theater}}}\Cendnote{\textnormal{Schnitzler besuchte ein Gastspiel der \emph{Comédie Française}\orgindex{Comedie-Française@Comédie-Française|pwk} im Ausstellungs-Theater\oindex{Internationales Ausstellungstheater im k.k. Prater@\textbf{Internationales Ausstellungstheater im k.k. Prater}, \emph{Theater (K.THE)}|pwk}.}}}\label{K_L03109-2} kann ich ja doch nicht mit
               Ihnen gehen, da ich keinen Sitz habe. Es wäre mir lieb, wenn ich Sie \label{K_L03109-3v}\edtext{abends im Café}{\lemma{\textnormal{\emph{abends im Café}}}\Cendnote{\textnormal{nicht
                  nachweisbar}}}\label{K_L03109-3}{ }{\pb}treffen könnte, da \strikeout{\textcolor{gray}{au}} ja morgen{ }\label{K_L03109-4v}\edtext{Feiertag}{\lemma{\textnormal{\emph{Feiertag}}}\Cendnote{\textnormal{Christi Himmelfahrt}}}\label{K_L03109-4} ist, und Sie länger Zeit haben,
               ich also vielleicht auf eine Stunde zu Ihnen hinauf gehen kann um Ihnen meine \label{K_L03109-5v}\edtext{Geschichte\pwindex{Muza@\emph{Muza}|pwv} vorzulesen}{\lemma{\textnormal{\emph{Geschichte vorzulesen}}}\Cendnote{\textnormal{Siehe A. S.: \emph{Tagebuch}, 27. 5. 1892.
               }}}\label{K_L03109-5}, damit ich die Sache endlich los habe.\pend
           
\pstart
           Noch eine Bitte! Vielleicht gestatten es Ihre Umstände, mir noch f 5– zu leihen. Ich
               bin sehr {\pb}betrübt darüber,
               daß ich Sie so übermäßig strapazieren muß, aber meine klägliche Situation dürfte sich
               erst mit nächstem Monate ein wenig bessern.\pend
           
\pstart
           Waren Sie heute bei \label{K_L03109-6v}\edtext{Weiss\pwindex{Weiss, Leopold *~1853@\textsc{Weiss, Leopold} (*~1853), \emph{Verleger/Verlegerin}|pw}}{\lemma{\textnormal{\emph{Weiss}}}\Cendnote{\textnormal{Schnitzler war auf der Suche nach einem
                  Verlag für \emph{Anatol}\pwindex{Anatol@\emph{Anatol}|pwk}. Zwar hatte ihm Leopold Weiss\pwindex{Weiss, Leopold *~1853@\textsc{Weiss, Leopold} (*~1853), \emph{Verleger/Verlegerin}|pwk} mündlich zugesagt, aber am
                     18. 6. 1892
                  übermittelte dieser eine schriftliche Absage.}}}\label{K_L03109-6}? »Ist mit dem Manne\pwindex{Weiss, Leopold *~1853@\textsc{Weiss, Leopold} (*~1853), \emph{Verleger/Verlegerin}|pwv} etwas anzufangen?«\pend
           
\pstart
           Herzlichst{\\[\baselineskip]}Ihr{\\[\baselineskip]}\spacefill\mbox{FelixSalten}\pend
           \leftskip=0em{}\selectlanguage{ngerman}\endnumbering\briefempfaengerindex{Schnitzler, Arthur@\textsc{Schnitzler, Arthur}!zzzSalten, Felix@\emph{von Felix Salten}!1892-05-251@{{[}25. 5. 1892{]}}|)be}\mylabel{L03109h}  \normalsize

\doendnotes{C}
\bigskip
\vfill

\clearpage

\footnotesize

\lohead{\textsc{register}}

% Definiere theindex-Environment komplett neu ohne reledmac
\makeatletter
\renewenvironment{theindex}{%
  \section*{\indexname}%
  \setlength{\parindent}{0pt}%
  \setlength{\parskip}{0pt plus 0.3pt}%
  \let\item\@idxitem
}{%
  \clearpage
}
\makeatother

\IfFileExists{\jobname-pw.ind}{\input{\jobname-pw.ind}}{}

\end{document}

      