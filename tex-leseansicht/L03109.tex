%% latex-leseansicht-vorspann.tex
%% Vorspann für die Leseansicht.
%% Lädt die gemeinsame Datei latex-vorspann.tex mit nicht gesetztem Schalter.

\newif\ifkorrekturansicht
\korrekturansichtfalse

\input{../tex-inputs/latex-vorspann}


\section[Felix Salten an Arthur Schnitzler, {[}25. 5. 1892{]}]{L03109 Felix Salten an Arthur Schnitzler, {[}25. 5. 1892{]}}
\nopagebreak\mylabel{L03109v}
\rehead{ }\normalsize\beginnumbering\briefempfaengerindex{Schnitzler, Arthur@\textsc{Schnitzler, Arthur}!zzzSalten, Felix@\emph{von Felix Salten}!1892-05-251@{{[}25. 5. 1892{]}}|(be}
\toendnotes[C]{\smallbreak\pagebreak[2]}
\correspDesc{Versand  durch Felix Salten am [25. 5. 1892] in Wien
\newline{}Erhalt  durch Arthur Schnitzler am [25. 5. 1892] in Wien}\toendnotes[C]{\smallbreak}
\Standort{CUL, Schnitzler, B 89, A 1.}
\physDesc{Brief, 1 Blatt, 3 Seiten, 817 Zeichen
\newline{}Handschrift: Bleistift, lateinische Kurrent
\newline{}Schnitzler: mit Bleistift datiert: »25/5 92« 
\newline{}Ordnung: mit Bleistift von unbekannter Hand nummeriert: »11« }\toendnotes[C]{\smallbreak}
\pstart
           \noindent{}{\pb}lieber Freund! Seit morgens ½ 11 Uhr
               arbeite ich unausgesetzt, und gedenke auch noch bis abends zu arbeiten,
               da ich sehr \label{K_L03109-1v}\edtext{\begin{otherlanguage}{french}en train\end{otherlanguage}}{\lemma{\textnormal{\emph{en train}}}\Cendnote{\textnormal{französisch: im Zug, im Sinne von: ›es
                  läuft sehr gut‹}}}\label{K_L03109-1} bin. Es ist aber möglich, dass ich früher ermatte.\pend
           
\pstart
           Ins \label{K_L03109-2v}\edtext{Theater\oindex{Wien@\textbf{Wien}!II., Leopoldstadt@\textbf{II., Leopoldstadt}!Internationales Ausstellungstheater im k.k. Prater@\textbf{Internationales Ausstellungstheater im k.k. Prater}, \emph{Theater}|pwv}}{\lemma{\textnormal{\emph{Theater}}}\Cendnote{\textnormal{Schnitzler besuchte ein Gastspiel der \emph{Comédie Française}\orgindex{Comédie-Française@Comédie-Française|pwk} im Ausstellungs-Theater\oindex{Wien@\textbf{Wien}!II., Leopoldstadt@\textbf{II., Leopoldstadt}!Internationales Ausstellungstheater im k.k. Prater@\textbf{Internationales Ausstellungstheater im k.k. Prater}, \emph{Theater}|pwk}.}}}\label{K_L03109-2} kann ich ja doch nicht mit
               Ihnen gehen, da ich keinen Sitz habe. Es wäre mir lieb, wenn ich Sie \label{K_L03109-3v}\edtext{abends im Café}{\lemma{\textnormal{\emph{abends im Café}}}\Cendnote{\textnormal{nicht
                  nachweisbar}}}\label{K_L03109-3}{ }{\pb}treffen könnte, da \strikeout{\textcolor{gray}{au}} ja morgen{ }\label{K_L03109-4v}\edtext{Feiertag}{\lemma{\textnormal{\emph{Feiertag}}}\Cendnote{\textnormal{Christi Himmelfahrt}}}\label{K_L03109-4} ist, und Sie länger Zeit haben,
               ich also vielleicht auf eine Stunde zu Ihnen hinauf gehen kann um Ihnen meine \label{K_L03109-5v}\edtext{Geschichte\pwindex{Salten, Felix 6.\,9.\,1869 Budapest – 8.\,10.\,1945 Zürich@\textsc{Salten, Felix} (6.\,9.\,1869 Budapest – 8.\,10.\,1945 Zürich), \emph{Schriftsteller, Journalist, Chefredakteur}!Muza@\strich\emph{Muza}|pwv} vorzulesen}{\lemma{\textnormal{\emph{Geschichte vorzulesen}}}\Cendnote{\textnormal{Siehe A. S.: \emph{Tagebuch}, 27. 5. 1892.
               }}}\label{K_L03109-5}, damit ich die Sache endlich los habe.\pend
           
\pstart
           Noch eine Bitte! Vielleicht gestatten es Ihre Umstände, mir noch f 5– zu leihen. Ich
               bin sehr {\pb}betrübt darüber,
               daß ich Sie so übermäßig strapazieren muß, aber meine klägliche Situation dürfte sich
               erst mit nächstem Monate ein wenig bessern.\pend
           
\pstart
           Waren Sie heute bei \label{K_L03109-6v}\edtext{Weiss\pwindex{Weiss, Leopold *~1853@\textsc{Weiss, Leopold} (*~1853), \emph{Verleger}|pw}}{\lemma{\textnormal{\emph{Weiss}}}\Cendnote{\textnormal{Schnitzler war auf der Suche nach einem
                  Verlag für \emph{Anatol}\pwindex{Schnitzler, Arthur 15.\,5.\,1862 Wien – 21.\,10.\,1931 ebd.@\textsc{Schnitzler, Arthur} (15.\,5.\,1862 Wien – 21.\,10.\,1931 ebd.), \emph{Schriftsteller, Mediziner}!Anatol@\strich\emph{Anatol}|pwk}. Zwar hatte ihm Leopold Weiss\pwindex{Weiss, Leopold *~1853@\textsc{Weiss, Leopold} (*~1853), \emph{Verleger}|pwk} mündlich zugesagt, aber am
                     18. 6. 1892
                  übermittelte dieser eine schriftliche Absage.}}}\label{K_L03109-6}? »Ist mit dem Manne\pwindex{Weiss, Leopold *~1853@\textsc{Weiss, Leopold} (*~1853), \emph{Verleger}|pwv} etwas anzufangen?«\pend
           
\pstart
           Herzlichst{\\[\baselineskip]}Ihr{\\[\baselineskip]}\spacefill\mbox{FelixSalten}\pend
           \leftskip=0em{}\selectlanguage{ngerman}\endnumbering\briefempfaengerindex{Schnitzler, Arthur@\textsc{Schnitzler, Arthur}!zzzSalten, Felix@\emph{von Felix Salten}!1892-05-251@{{[}25. 5. 1892{]}}|)be}\mylabel{L03109h}  \newcommand{\dateiname}{L03109}\newcommand{\titel}{Felix Salten an Arthur Schnitzler, [25. 5. 1892]}\newcommand{\editorInnen}{Martin Anton Müller und Laura Untner}%% latex-leseansicht-abspann.tex
%% Abspann für die Leseansicht.
%% Der Schalter \ifkorrekturansicht ist bereits durch den Vorspann gesetzt.

%% latex-abspann.tex
%% Gemeinsamer Abspann für Korrekturansicht und Leseansicht.
%% Setzt den Schalter \ifkorrekturansicht voraus (gesetzt in den
%% einbindenden Dateien latex-korrekturansicht-abspann.tex bzw.
%% latex-leseansicht-abspann.tex).
%% ---------------------------------------------------------------

\normalsize

% Das esempio-Environment wird nur in der Leseansicht benötigt
\ifkorrekturansicht\else
\newenvironment{esempio}[3]%
{
    \vspace{1.5ex}
    \rlap{\underline{#1}}
    \par
    \setlength{\parindent}{0cm}
    \nopagebreak
    \leftskip=#2cm
    \rightskip=#3cm
}
{
    \par
}
\fi

\doendnotes{C}
\bigskip
\vfill

\clearpage

\footnotesize

\ifkorrekturansicht
  \lohead{\textsc{register}}
\fi

% theindex-Environment neu definieren ohne reledmac
\makeatletter
\renewenvironment{theindex}{%
  \ifkorrekturansicht
    \section*{\indexname}%
  \else
    \subsubsection*{Index der erwähnten Entitäten}%
  \fi
  \setlength{\parindent}{0pt}%
  \setlength{\parskip}{0pt plus 0.3pt}%
  \let\item\@idxitem
}{%
  \ifkorrekturansicht\clearpage\fi
}
\makeatother

\IfFileExists{\jobname-pw.ind}{\input{\jobname-pw.ind}}{}

% Quellenangabe nur in der Leseansicht
\ifkorrekturansicht\else
% Fallback-Definitionen, falls die .tex-Datei \titel etc. nicht gesetzt hat
\providecommand{\titel}{}
\providecommand{\editorInnen}{}
\providecommand{\dateiname}{\jobname}

\vspace{3cm}

\vfill

\footnotesize
\textsc{Quelle}: \titel. Herausgegeben von {\editorInnen}. In: \emph{Arthur Schnitzler: Briefwechsel mit Autorinnen und Autoren}.
 Digitale Edition, https://schnitzler-briefe.acdh.oeaw.ac.at/{\dateiname}.html (Stand \today)
\fi

\end{document}


