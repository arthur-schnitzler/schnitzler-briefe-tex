%% latex-korrekturansicht-vorspann.tex
%% Vorspann für die Korrekturansicht.
%% Lädt die gemeinsame Datei latex-vorspann.tex mit gesetztem Schalter.

\newif\ifkorrekturansicht
\korrekturansichttrue

\input{../tex-inputs/latex-vorspann}


\section[Arthur Schnitzler, Karl Kraus und Friedrich Schik an Richard Beer-Hofmann, {[}31. 12. 1893?{]}]{L00288 Arthur Schnitzler, Karl Kraus und Friedrich Schik an Richard
               Beer-Hofmann, {[}31. 12. 1893?{]}}
\nopagebreak\mylabel{L00288v}
\rehead{ }\normalsize\beginnumbering\briefempfaengerindex{Beer-Hofmann, Richard@\textsc{Beer-Hofmann, Richard}!zzzSchik, Friedrich@\emph{von Friedrich Schik}!1893-12-314@{{[}31. 12. 1893?{]}}|(be}\briefempfaengerindex{Beer-Hofmann, Richard@\textsc{Beer-Hofmann, Richard}!zzzKraus, Karl@\emph{von Karl Kraus}!1893-12-314@{{[}31. 12. 1893?{]}}|(be}\briefempfaengerindex{Beer-Hofmann, Richard@\textsc{Beer-Hofmann, Richard}!zzzSchnitzler, Arthur@\emph{von Arthur Schnitzler}!1893-12-314@{{[}31. 12. 1893?{]}}|(be}
\toendnotes[C]{\smallbreak\pagebreak[2]}\Standort{YCGL, MSS 31.}
\physDesc{Visitenkarte, 303 Zeichen (Visitenkarte mit Trauerrand)
\newline{}Handschrift Arthur Schnitzler: Bleistift, deutsche Kurrent
\newline{}Handschrift Karl Kraus: Bleistift, deutsche Kurrent
\newline{}Handschrift Friedrich Schik: Bleistift, deutsche Kurrent}
\buchAbdrucke{\weitereDrucke{Arthur Schnitzler, Richard Beer-Hofmann: \emph{Briefwechsel 1891–1931}. Wien, Zürich: \emph{Europaverlag} 1992, S. 54.} }\toendnotes[C]{\smallbreak}
\pstart
           \noindent{}{\pb}An den Verfaſſer des »Kinds\pwindex{Kind@\emph{Das Kind}|pw}«. –\pend
           
\pstart
           Wir haben ½ Stunde ununterbrochen über Sie \label{K_L00288-1v}\edtext{geſprochen}{\lemma{\textnormal{\emph{geſprochen}}}\Cendnote{\textnormal{Die drei
                  Unterzeichner waren laut \emph{Tagebuch}\pwindex{Tagebuch@\emph{Tagebuch}|pwk} am 31. 12. 1893 gemeinsam
                  im Kaffeehaus.}}}\label{K_L00288-1}. Auch der Autor\pwindex{Salten, Felix 06.09.1869 – 08.10.1945@\textsc{Salten, Felix} (06.09.1869 – 08.10.1945), \emph{Schriftsteller/Schriftstellerin, Journalist/Journalistin, Chefredakteur/Chefredakteurin}|pwv} des »\label{K_L00288-2v}\edtext{Begräbniſſes\pwindex{Begraebnis@\emph{Begräbnis}|pw}}{\lemma{\textnormal{\emph{Begräbniſſes}}}\Cendnote{\textnormal{Felix Salten\pwindex{Salten, Felix 06.09.1869 – 08.10.1945@\textsc{Salten, Felix} (06.09.1869 – 08.10.1945), \emph{Schriftsteller/Schriftstellerin, Journalist/Journalistin, Chefredakteur/Chefredakteurin}|pwk}: \emph{Begräbnis}\pwindex{Begraebnis@\emph{Begräbnis}|pwk}. In: \emph{Mährisches
                        Tagblatt}\pwindex{Maehrisches Tagblatt@\emph{Mährisches Tagblatt}|pwk}, Jg. 14, Nr. 160, 17. 7. 1893, S. 1–2.
               }}}\label{K_L00288-2}« blieb nicht unerwähnt. – Es iſt bedauerlich, daß ſolche Männer ihre Nächte
               in Dominoorgien hinbringen. –\pend
           \pstart {\pb}In Hochachtung\pend{}
\pstart
           \centering{}\textcolor{gray}{\textbf{D\textsuperscript{r}Arthur Schnitzler}}\pend
           \selectlanguage{ngerman}\vspace{1em}
\pstart
           \noindent{}{[}hs. :{]} in aufrichtiger Bewunderung u. Wertschätzung\pend
           \pstart \spacefill\mbox{KarlKraus}\pend{}\selectlanguage{ngerman}\vspace{1em}
\pstart
           \noindent{}{[}hs. :{]} ergebenſt\pend
           \pstart \spacefill\mbox{FSchik}\pend{}\selectlanguage{ngerman}\endnumbering\briefempfaengerindex{Beer-Hofmann, Richard@\textsc{Beer-Hofmann, Richard}!zzzSchik, Friedrich@\emph{von Friedrich Schik}!1893-12-314@{{[}31. 12. 1893?{]}}|)be}\briefempfaengerindex{Beer-Hofmann, Richard@\textsc{Beer-Hofmann, Richard}!zzzKraus, Karl@\emph{von Karl Kraus}!1893-12-314@{{[}31. 12. 1893?{]}}|)be}\briefempfaengerindex{Beer-Hofmann, Richard@\textsc{Beer-Hofmann, Richard}!zzzSchnitzler, Arthur@\emph{von Arthur Schnitzler}!1893-12-314@{{[}31. 12. 1893?{]}}|)be}\mylabel{L00288h}  \normalsize

\doendnotes{C}
\bigskip
\vfill

\clearpage

\footnotesize

\lohead{\textsc{register}}

% Definiere theindex-Environment komplett neu ohne reledmac
\makeatletter
\renewenvironment{theindex}{%
  \section*{\indexname}%
  \setlength{\parindent}{0pt}%
  \setlength{\parskip}{0pt plus 0.3pt}%
  \let\item\@idxitem
}{%
  \clearpage
}
\makeatother

\IfFileExists{\jobname-pw.ind}{\input{\jobname-pw.ind}}{}

\end{document}

      