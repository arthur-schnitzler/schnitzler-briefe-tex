%% latex-leseansicht-vorspann.tex
%% Vorspann für die Leseansicht.
%% Lädt die gemeinsame Datei latex-vorspann.tex mit nicht gesetztem Schalter.

\newif\ifkorrekturansicht
\korrekturansichtfalse

\input{../tex-inputs/latex-vorspann}


\section[Arthur Schnitzler, Karl Kraus und Friedrich Schik an Richard Beer-Hofmann, {{[}}31. 12. 1893?{{]}}]{L00288 Arthur Schnitzler, Karl Kraus und Friedrich Schik an Richard
               Beer-Hofmann, {[}31. 12. 1893?{]}}
\nopagebreak\mylabel{L00288v}
\rehead{ }\normalsize\beginnumbering\briefempfaengerindex{Beer-Hofmann, Richard@\textsc{Beer-Hofmann, Richard}!zzzSchik, Friedrich@\emph{von Friedrich Schik}!1893-12-314@{{[}31. 12. 1893?{]}}|(be}\briefempfaengerindex{Beer-Hofmann, Richard@\textsc{Beer-Hofmann, Richard}!zzzKraus, Karl@\emph{von Karl Kraus}!1893-12-314@{{[}31. 12. 1893?{]}}|(be}\briefempfaengerindex{Beer-Hofmann, Richard@\textsc{Beer-Hofmann, Richard}!zzzSchnitzler, Arthur@\emph{von Arthur Schnitzler}!1893-12-314@{{[}31. 12. 1893?{]}}|(be}
\toendnotes[C]{\smallbreak\pagebreak[2]}
\correspDesc{Versand  durch Arthur Schnitzler, Karl Kraus, Friedrich Schik am [31. 12. 1893?] in Wien
\newline{}Erhalt  durch Richard Beer-Hofmann im Zeitraum [31. 12. 1893 – 4. 1. 1894?] \textbf{Ort fehlend} }\toendnotes[C]{\smallbreak}
\Standort{YCGL, MSS 31.}
\physDesc{Visitenkarte, 303 Zeichen (Visitenkarte mit Trauerrand)
\newline{}Handschrift Arthur Schnitzler: Bleistift, deutsche Kurrent
\newline{}Handschrift Karl Kraus: Bleistift, deutsche Kurrent
\newline{}Handschrift Friedrich Schik: Bleistift, deutsche Kurrent}
\buchAbdrucke{\weitereDrucke{Arthur Schnitzler, Richard Beer-Hofmann: \emph{Briefwechsel 1891–1931}. Herausgegeben von Konstanze Fliedl. Wien, Zürich: \emph{Europaverlag} 1992, S. 54.} }\toendnotes[C]{\smallbreak}
\pstart
           \noindent{}{\pb}An den Verfaſſer des »Kinds\pwindex{Beer-Hofmann, Richard 11.\,7.\,1866 Wien – 26.\,9.\,1945 New York City@\textsc{Beer-Hofmann, Richard} (11.\,7.\,1866 Wien – 26.\,9.\,1945 New York City), \emph{Schriftsteller}!Kind@\strich\emph{Das Kind}|pw}«. –\pend
           
\pstart
           Wir haben ½ Stunde ununterbrochen über Sie \label{K_L00288-1v}\edtext{geſprochen}{\lemma{\textnormal{\emph{gesprochen}}}\Cendnote{\textnormal{Die drei
                  Unterzeichner waren laut \emph{Tagebuch}\pwindex{Schnitzler, Arthur 15.\,5.\,1862 Wien – 21.\,10.\,1931 ebd.@\textsc{Schnitzler, Arthur} (15.\,5.\,1862 Wien – 21.\,10.\,1931 ebd.), \emph{Schriftsteller, Mediziner}!Tagebuch@\strich\emph{Tagebuch}|pwk} am 31. 12. 1893 gemeinsam
                  im Kaffeehaus.}}}\label{K_L00288-1}. Auch der Autor\pwindex{Salten, Felix 6.\,9.\,1869 Budapest – 8.\,10.\,1945 Zürich@\textsc{Salten, Felix} (6.\,9.\,1869 Budapest – 8.\,10.\,1945 Zürich), \emph{Schriftsteller, Journalist, Chefredakteur}|pwv} des »\label{K_L00288-2v}\edtext{Begräbniſſes\pwindex{Salten, Felix 6.\,9.\,1869 Budapest – 8.\,10.\,1945 Zürich@\textsc{Salten, Felix} (6.\,9.\,1869 Budapest – 8.\,10.\,1945 Zürich), \emph{Schriftsteller, Journalist, Chefredakteur}!Begräbnis@\strich\emph{Begräbnis}|pw}}{\lemma{\textnormal{\emph{Begräbnisses}}}\Cendnote{\textnormal{Felix Salten\pwindex{Salten, Felix 6.\,9.\,1869 Budapest – 8.\,10.\,1945 Zürich@\textsc{Salten, Felix} (6.\,9.\,1869 Budapest – 8.\,10.\,1945 Zürich), \emph{Schriftsteller, Journalist, Chefredakteur}|pwk}: \emph{Begräbnis}\pwindex{Salten, Felix 6.\,9.\,1869 Budapest – 8.\,10.\,1945 Zürich@\textsc{Salten, Felix} (6.\,9.\,1869 Budapest – 8.\,10.\,1945 Zürich), \emph{Schriftsteller, Journalist, Chefredakteur}!Begräbnis@\strich\emph{Begräbnis}|pwk}. In: \emph{Mährisches
                        Tagblatt}\pwindex{Mährisches Tagblatt@\emph{Mährisches Tagblatt}|pwk}, Jg. 14, Nr. 160, 17. 7. 1893, S. 1–2.
               }}}\label{K_L00288-2}« blieb nicht unerwähnt. – Es iſt bedauerlich, daß{ }ſolche Männer ihre Nächte
               in Dominoorgien hinbringen. –\pend
           \pstart {\pb}In Hochachtung\pend{}
\pstart
           \centering{}\textcolor{gray}{\textbf{D\textsuperscript{r}Arthur Schnitzler}}\pend
           \selectlanguage{ngerman}\vspace{1em}
\pstart
           \noindent{}{[}hs. Kraus:{]} in aufrichtiger Bewunderung u. Wertschätzung\pend
           \pstart \spacefill\mbox{KarlKraus}\pend{}\selectlanguage{ngerman}\vspace{1em}
\pstart
           \noindent{}{[}hs. Schik:{]} ergebenſt\pend
           \pstart \spacefill\mbox{FSchik}\pend{}\selectlanguage{ngerman}\endnumbering\briefempfaengerindex{Beer-Hofmann, Richard@\textsc{Beer-Hofmann, Richard}!zzzSchik, Friedrich@\emph{von Friedrich Schik}!1893-12-314@{{[}31. 12. 1893?{]}}|)be}\briefempfaengerindex{Beer-Hofmann, Richard@\textsc{Beer-Hofmann, Richard}!zzzKraus, Karl@\emph{von Karl Kraus}!1893-12-314@{{[}31. 12. 1893?{]}}|)be}\briefempfaengerindex{Beer-Hofmann, Richard@\textsc{Beer-Hofmann, Richard}!zzzSchnitzler, Arthur@\emph{von Arthur Schnitzler}!1893-12-314@{{[}31. 12. 1893?{]}}|)be}\mylabel{L00288h}  \newcommand{\dateiname}{L00288}\newcommand{\titel}{Arthur Schnitzler, Karl Kraus und Friedrich Schik an Richard Beer-Hofmann, [31. 12. 1893?]}\newcommand{\editorInnen}{Martin Anton Müller und Gerd-Hermann Susen}%% latex-leseansicht-abspann.tex
%% Abspann für die Leseansicht.
%% Der Schalter \ifkorrekturansicht ist bereits durch den Vorspann gesetzt.

%% latex-abspann.tex
%% Gemeinsamer Abspann für Korrekturansicht und Leseansicht.
%% Setzt den Schalter \ifkorrekturansicht voraus (gesetzt in den
%% einbindenden Dateien latex-korrekturansicht-abspann.tex bzw.
%% latex-leseansicht-abspann.tex).
%% ---------------------------------------------------------------

\normalsize

% Das esempio-Environment wird nur in der Leseansicht benötigt
\ifkorrekturansicht\else
\newenvironment{esempio}[3]%
{
    \vspace{1.5ex}
    \rlap{\underline{#1}}
    \par
    \setlength{\parindent}{0cm}
    \nopagebreak
    \leftskip=#2cm
    \rightskip=#3cm
}
{
    \par
}
\fi

\doendnotes{C}
\bigskip
\vfill

\clearpage

\footnotesize

\ifkorrekturansicht
  \lohead{\textsc{register}}
\fi

% theindex-Environment neu definieren ohne reledmac
\makeatletter
\renewenvironment{theindex}{%
  \ifkorrekturansicht
    \section*{\indexname}%
  \else
    \subsubsection*{Index der erwähnten Entitäten}%
  \fi
  \setlength{\parindent}{0pt}%
  \setlength{\parskip}{0pt plus 0.3pt}%
  \let\item\@idxitem
}{%
  \ifkorrekturansicht\clearpage\fi
}
\makeatother

\IfFileExists{\jobname-pw.ind}{\input{\jobname-pw.ind}}{}

% Quellenangabe nur in der Leseansicht
\ifkorrekturansicht\else
% Fallback-Definitionen, falls die .tex-Datei \titel etc. nicht gesetzt hat
\providecommand{\titel}{}
\providecommand{\editorInnen}{}
\providecommand{\dateiname}{\jobname}

\vspace{3cm}

\vfill

\footnotesize
\textsc{Quelle}: \titel. Herausgegeben von {\editorInnen}. In: \emph{Arthur Schnitzler: Briefwechsel mit Autorinnen und Autoren}.
 Digitale Edition, https://schnitzler-briefe.acdh.oeaw.ac.at/{\dateiname}.html (Stand \today)
\fi

\end{document}


