%% latex-leseansicht-vorspann.tex
%% Vorspann für die Leseansicht.
%% Lädt die gemeinsame Datei latex-vorspann.tex mit nicht gesetztem Schalter.

\newif\ifkorrekturansicht
\korrekturansichtfalse

\input{../tex-inputs/latex-vorspann}


\section[ Paul Goldmann an Arthur Schnitzler, 29. 7. [1901]]{L03075 Paul Goldmann an Arthur Schnitzler,  29. 7. [1901]}
\nopagebreak\mylabel{L03075v}
\rehead{ }\normalsize\beginnumbering\briefempfaengerindex{Schnitzler, Arthur@\textsc{Schnitzler, Arthur}!zzzGoldmann, Paul@\emph{von Paul Goldmann}!1901-07-291@{29. 7. [1901]}|(be}
\toendnotes[C]{\smallbreak\pagebreak[2]}
\correspDesc{Versand  durch Paul Goldmann am 29. 7. [1901] in Pörtschach
\newline{}Erhalt  durch Arthur Schnitzler im Zeitraum [30. 7. 1901
                  – 3. 8. 1901?] in Vahrn}\toendnotes[C]{\smallbreak}
\Standort{DLA, A:Schnitzler, HS.NZ85.1.3171.}
\physDesc{Brief, 1 Blatt, 2 Seiten, 743 Zeichen
\newline{}Handschrift: blaue Tinte, deutsche Kurrent
\newline{}Schnitzler: 1) mit schwarzer Tinte das Jahr »901« vermerkt  2) mit rotem Buntstift eine Unterstreichung}\toendnotes[C]{\smallbreak}
\pstart
           {\pb}\textsc{Pörtschach\oindex{Pörtschach am Wörthersee@\textbf{Pörtschach am Wörthersee}|pw}}, 29. Juli.\pend
           
\pstart{}Mein lieber Freund,\pend\vspace{0.5em}
\pstart
           Ich danke Dir für Deinen lieben Brief und Deine \label{K_L03075-1v}\edtext{Forſchungsreiſen}{\lemma{\textnormal{\emph{Forschungsreisen}}}\Cendnote{\textnormal{Goldmann\pwindex{Goldmann, Paul 31.\,1.\,1865 Breslau – 25.\,9.\,1935 Wien@\textsc{Goldmann, Paul} (31.\,1.\,1865 Breslau – 25.\,9.\,1935 Wien), \emph{Schriftsteller, Journalist}|pwk} dürfte sich auf den Ausflug Schnitzlers vom 22. 7. 1901 bis zum
                     24. 7. 1901
                  bezogen haben, dessen Zweck in der Ermittlung der nächsten Unterkunft gelegen
                  haben dürfte. Entsprechend wäre Schnitzlers nicht überliefertes Schreiben nach der Rückkehr anzusetzen.}}}\label{K_L03075-1}. Finde nur
               etwas Hohes und Kühles. Hier iſt es mir zu lau und die Luft iſt mir zu matt. Trotzdem
               bleibe ich wohl eine Woche hier, weil ich ein wenig das Beiſammenſein mit \textsc{Richard\pwindex{Beer-Hofmann, Richard 11.\,7.\,1866 Wien – 26.\,9.\,1945 New York City@\textsc{Beer-Hofmann, Richard} (11.\,7.\,1866 Wien – 26.\,9.\,1945 New York City), \emph{Schriftsteller}|pw}} genießen will. Könnteſt Du nicht irgend etwas in den Dolomiten\oindex{Dolomiten@\textbf{Dolomiten}, \emph{Gebirge}|pw},{ }ſo um \textsc{Madonna \strikeout{die} di
                     Campiglio\oindex{Madonna di Campiglio@\textbf{Madonna di Campiglio}|pw}} herum, \label{K_L03075-2v}\edtext{finden}{\lemma{\textnormal{\emph{finden}}}\Cendnote{\textnormal{Siehe XXXX Auszeichnungsfehler: Dokument L03064 nicht gefunden.
               }}}\label{K_L03075-2}? Was geht uns die Geſellſchaft an, wenn \strikeout{\textcolor{gray}{×}} wir {\pb}miteinander{ }ſind? Nach einem warmen Ort
               komme ich nicht. Ich{ }ſchlafe keine Nacht und brauche{ }ſtarke Luft, um Schlaf zu
               finden.\pend
           
\pstart
           Wenn Du Dich zu einer Niederlaſſung entſchloſſen haſt,{ }ſo{ }ſende mir Nachricht
               hierher, Etabliſſement \textsc{Werzer}\oindex{Etablissement Werzer@\textbf{Etablissement Werzer}, \emph{Hotel}|pw}, \substVorne{}\textsuperscript{Zimmer}\substDazwischen{}\textsc{Villa}\substHinten{} 8, Zimmer 31.\pend
           
\pstart
           Viele Grüße Dir und den lieblichen Schweſtern\pwindex{Schnitzler, Olga 17.\,1.\,1882 Wien – 13.\,1.\,1970 Lugano@\textsc{Schnitzler, Olga} (17.\,1.\,1882 Wien – 13.\,1.\,1970 Lugano), \emph{Schauspielerin, Sängerin}|pwv}\pwindex{Steinrück, Elisabeth 19.\,11.\,1885 – 7.\,4.\,1920 Partenkirchen@\textsc{Steinrück, Elisabeth} (19.\,11.\,1885 – 7.\,4.\,1920 Partenkirchen)|pwv}! {\\[\baselineskip]}Dein {\\[\baselineskip]}\spacefill\mbox{Paul Goldmnn}\pend
           \leftskip=0em{}\selectlanguage{ngerman}\endnumbering\briefempfaengerindex{Schnitzler, Arthur@\textsc{Schnitzler, Arthur}!zzzGoldmann, Paul@\emph{von Paul Goldmann}!1901-07-291@{29. 7. [1901]}|)be}\mylabel{L03075h}  \newcommand{\dateiname}{L03075}\newcommand{\titel}{Paul Goldmann an Arthur Schnitzler, 29. 7. [1901]}\newcommand{\editorInnen}{Martin Anton Müller und Laura Untner}%% latex-leseansicht-abspann.tex
%% Abspann für die Leseansicht.
%% Der Schalter \ifkorrekturansicht ist bereits durch den Vorspann gesetzt.

%% latex-abspann.tex
%% Gemeinsamer Abspann für Korrekturansicht und Leseansicht.
%% Setzt den Schalter \ifkorrekturansicht voraus (gesetzt in den
%% einbindenden Dateien latex-korrekturansicht-abspann.tex bzw.
%% latex-leseansicht-abspann.tex).
%% ---------------------------------------------------------------

\normalsize

% Das esempio-Environment wird nur in der Leseansicht benötigt
\ifkorrekturansicht\else
\newenvironment{esempio}[3]%
{
    \vspace{1.5ex}
    \rlap{\underline{#1}}
    \par
    \setlength{\parindent}{0cm}
    \nopagebreak
    \leftskip=#2cm
    \rightskip=#3cm
}
{
    \par
}
\fi

\doendnotes{C}
\bigskip
\vfill

\clearpage

\footnotesize

\ifkorrekturansicht
  \lohead{\textsc{register}}
\fi

% theindex-Environment neu definieren ohne reledmac
\makeatletter
\renewenvironment{theindex}{%
  \ifkorrekturansicht
    \section*{\indexname}%
  \else
    \subsubsection*{Index der erwähnten Entitäten}%
  \fi
  \setlength{\parindent}{0pt}%
  \setlength{\parskip}{0pt plus 0.3pt}%
  \let\item\@idxitem
}{%
  \ifkorrekturansicht\clearpage\fi
}
\makeatother

\IfFileExists{\jobname-pw.ind}{\input{\jobname-pw.ind}}{}

% Quellenangabe nur in der Leseansicht
\ifkorrekturansicht\else
% Fallback-Definitionen, falls die .tex-Datei \titel etc. nicht gesetzt hat
\providecommand{\titel}{}
\providecommand{\editorInnen}{}
\providecommand{\dateiname}{\jobname}

\vspace{3cm}

\vfill

\footnotesize
\textsc{Quelle}: \titel. Herausgegeben von {\editorInnen}. In: \emph{Arthur Schnitzler: Briefwechsel mit Autorinnen und Autoren}.
 Digitale Edition, https://schnitzler-briefe.acdh.oeaw.ac.at/{\dateiname}.html (Stand \today)
\fi

\end{document}


