%% latex-korrekturansicht-vorspann.tex
%% Vorspann für die Korrekturansicht.
%% Lädt die gemeinsame Datei latex-vorspann.tex mit gesetztem Schalter.

\newif\ifkorrekturansicht
\korrekturansichttrue

\input{../tex-inputs/latex-vorspann}


\section[ Paul Goldmann an Arthur Schnitzler, 29. 7. {[}1901{]}]{L03075 Paul Goldmann an Arthur Schnitzler, 29. 7. {[}1901{]}}
\nopagebreak\mylabel{L03075v}
\rehead{ }\normalsize\beginnumbering\briefempfaengerindex{Schnitzler, Arthur@\textsc{Schnitzler, Arthur}!zzzGoldmann, Paul@\emph{von Paul Goldmann}!1901-07-291@{29. 7. {[}1901{]}}|(be}
\toendnotes[C]{\smallbreak\pagebreak[2]}\Standort{DLA, A:Schnitzler, HS.NZ85.1.3171.}
\physDesc{Brief, 1 Blatt, 2 Seiten, 743 Zeichen
\newline{}Handschrift: blaue Tinte, deutsche Kurrent
\newline{}Schnitzler: 1) mit schwarzer Tinte das Jahr »901« vermerkt  2) mit rotem Buntstift eine Unterstreichung}\toendnotes[C]{\smallbreak}
\pstart
           {\pb}\textsc{Pörtschach\oindex{Poertschach am Woerthersee@\textbf{Pörtschach am Wörthersee}, \emph{P.PPL}|pw}}, 29. Juli.\pend
           
\pstart{}Mein lieber Freund,\pend\vspace{0.5em}
\pstart
           Ich danke Dir für Deinen lieben Brief und Deine \label{K_L03075-1v}\edtext{Forſchungsreiſen}{\lemma{\textnormal{\emph{Forſchungsreiſen}}}\Cendnote{\textnormal{Goldmann\pwindex{Goldmann, Paul 31.01.1865 – 25.09.1935@\textsc{Goldmann, Paul} (31.01.1865 – 25.09.1935), \emph{Schriftsteller/Schriftstellerin, Journalist/Journalistin}|pwk} dürfte sich auf den Ausflug Schnitzlers vom 22. 7. 1901 bis zum
                     24. 7. 1901
                  bezogen haben, dessen Zweck in der Ermittlung der nächsten Unterkunft gelegen
                  haben dürfte. Entsprechend wäre Schnitzlers nicht überliefertes Schreiben nach der Rückkehr anzusetzen.}}}\label{K_L03075-1}. Finde nur
               etwas Hohes und Kühles. Hier iſt es mir zu lau und die Luft iſt mir zu matt. Trotzdem
               bleibe ich wohl eine Woche hier, weil ich ein wenig das Beiſammenſein mit \textsc{Richard\pwindex{Beer-Hofmann, Richard 1866-07-11 – 1945-09-26@\textsc{Beer-Hofmann, Richard} (1866-07-11 – 1945-09-26), \emph{Schriftsteller/Schriftstellerin}|pw}} genießen will. Könnteſt Du nicht irgend etwas in den Dolomiten\oindex{Dolomiten@\textbf{Dolomiten}, \emph{Gebirge (N.GBR)}|pw}, ſo um \textsc{Madonna \strikeout{die} di
                     Campiglio\oindex{Madonna di Campiglio@\textbf{Madonna di Campiglio}, \emph{P.PPL}|pw}} herum, \label{K_L03075-2v}\edtext{finden}{\lemma{\textnormal{\emph{finden}}}\Cendnote{\textnormal{Siehe Paul Goldmann an Arthur Schnitzler, 26. 4. [1901].
               }}}\label{K_L03075-2}? Was geht uns die Geſellſchaft an, wenn \strikeout{\textcolor{gray}{×}} wir {\pb}miteinander ſind? Nach einem warmen Ort
               komme ich nicht. Ich ſchlafe keine Nacht und brauche ſtarke Luft, um Schlaf zu
               finden.\pend
           
\pstart
           Wenn Du Dich zu einer Niederlaſſung entſchloſſen haſt, ſo ſende mir Nachricht
               hierher, Etabliſſement \textsc{Werzer}\oindex{Etablissement Werzer@\textbf{Etablissement Werzer}, \emph{Hotel (K.HTL)}|pw}, \substVorne{}\textsuperscript{Zimmer}\substDazwischen{}\textsc{Villa}\substHinten{} 8, Zimmer 31.\pend
           
\pstart
           Viele Grüße Dir und den lieblichen Schweſtern\pwindex{Schnitzler, Olga 17.01.1882 – 13.01.1970@\textsc{Schnitzler, Olga} (17.01.1882 – 13.01.1970), \emph{Schauspieler/Schauspielerin, Sänger/Sängerin}|pwv}\pwindex{Steinrueck, Elisabeth 19.11.1885 – 07.04.1920@\textsc{Steinrück, Elisabeth} (19.11.1885 – 07.04.1920)|pwv}! {\\[\baselineskip]}Dein {\\[\baselineskip]}\spacefill\mbox{Paul Goldmnn}\pend
           \leftskip=0em{}\selectlanguage{ngerman}\endnumbering\briefempfaengerindex{Schnitzler, Arthur@\textsc{Schnitzler, Arthur}!zzzGoldmann, Paul@\emph{von Paul Goldmann}!1901-07-291@{29. 7. {[}1901{]}}|)be}\mylabel{L03075h}  \normalsize

\doendnotes{C}
\bigskip
\vfill

\clearpage

\footnotesize

\lohead{\textsc{register}}

% Definiere theindex-Environment komplett neu ohne reledmac
\makeatletter
\renewenvironment{theindex}{%
  \section*{\indexname}%
  \setlength{\parindent}{0pt}%
  \setlength{\parskip}{0pt plus 0.3pt}%
  \let\item\@idxitem
}{%
  \clearpage
}
\makeatother

\IfFileExists{\jobname-pw.ind}{\input{\jobname-pw.ind}}{}

\end{document}

      