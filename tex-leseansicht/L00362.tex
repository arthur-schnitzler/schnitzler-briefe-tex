%% latex-leseansicht-vorspann.tex
%% Vorspann für die Leseansicht.
%% Lädt die gemeinsame Datei latex-vorspann.tex mit nicht gesetztem Schalter.

\newif\ifkorrekturansicht
\korrekturansichtfalse

\input{../tex-inputs/latex-vorspann}


\section[Richard Beer-Hofmann an Arthur Schnitzler, {[}18. 8. 1894{]}]{L00362 Richard Beer-Hofmann an Arthur Schnitzler, {[}18. 8. 1894{]}}
\nopagebreak\mylabel{L00362v}
\rehead{ }\normalsize\beginnumbering\briefempfaengerindex{Schnitzler, Arthur@\textsc{Schnitzler, Arthur}!zzzBeer-Hofmann, Richard@\emph{von Richard Beer-Hofmann}!1894-08-181@{{[}18. 8. 1894{]}}|(be}
\toendnotes[C]{\smallbreak\pagebreak[2]}
\correspDesc{Versand  durch Richard Beer-Hofmann am [18. 8. 1894] in Wien
\newline{}Erhalt  durch Arthur Schnitzler im Zeitraum [18. 8. 1894
                  – 22. 8. 1894?] in Wien}\toendnotes[C]{\smallbreak}
\Standort{CUL, Schnitzler, B 8.}
\physDesc{Briefkarte, 323 Zeichen
\newline{}Handschrift: blauer Buntstift, lateinische Kurrent
\newline{}Schnitzler: mit Bleistift datiert: »18/8 94« und nummeriert: »35« }\toendnotes[C]{\smallbreak}
\pstart
           \noindent{}{\pb}Lieber Arthur! Also
                  Goldmann\pwindex{Goldmann, Paul 31.\,1.\,1865 Breslau – 25.\,9.\,1935 Wien@\textsc{Goldmann, Paul} (31.\,1.\,1865 Breslau – 25.\,9.\,1935 Wien), \emph{Schriftsteller, Journalist}|pw}{ }ko{\geminationm}t. \label{K_L00362-1v}\edtext{Prosceniumsloge}{\lemma{\textnormal{\emph{Prosceniumsloge}}}\Cendnote{\textnormal{seitlich der Vorderbühne befindliche Logen, die sich gut für
                  Repräsentationszwecke eignen}}}\label{K_L00362-1} links sowie die daran anstossenden Logen sind
               Saison über in festen Händen. Zu haben ist {\pb}nur die rechte Prosceniumsloge die
               bei erhöhten Preisen 18 fl. kostet und die \strikeout{daran} mit
               2. (\label{T_L00362-1v}\edtext{rechts}{\lemma{\textnormal{\emph{rechts}}}\Cendnote{\textnormal{in deutscher Kurrentschrift}}}\label{T_L00362-1}) \label{K_L00362-2v}\edtext{bezeichnete Loge}{\lemma{\textnormal{\emph{bezeichnete Loge}}}\Cendnote{\textnormal{Eine Skizze illustriert die Lage der Loge, es ist die dritte seitlich von der
                  Bühne aus gesehen.}}}\label{K_L00362-2} die 12 fl kostet; welche soll ich nehmen? Ko{\geminationm}en Sie bald?\pend
           \pstart Herzlichst Ihr \spacefill\mbox{Rich}\pend{}\selectlanguage{ngerman}\endnumbering\briefempfaengerindex{Schnitzler, Arthur@\textsc{Schnitzler, Arthur}!zzzBeer-Hofmann, Richard@\emph{von Richard Beer-Hofmann}!1894-08-181@{{[}18. 8. 1894{]}}|)be}\mylabel{L00362h}  \newcommand{\dateiname}{L00362}\newcommand{\titel}{Richard Beer-Hofmann an Arthur Schnitzler, [18. 8. 1894]}\newcommand{\editorInnen}{Martin Anton Müller und Gerd-Hermann Susen}%% latex-leseansicht-abspann.tex
%% Abspann für die Leseansicht.
%% Der Schalter \ifkorrekturansicht ist bereits durch den Vorspann gesetzt.

%% latex-abspann.tex
%% Gemeinsamer Abspann für Korrekturansicht und Leseansicht.
%% Setzt den Schalter \ifkorrekturansicht voraus (gesetzt in den
%% einbindenden Dateien latex-korrekturansicht-abspann.tex bzw.
%% latex-leseansicht-abspann.tex).
%% ---------------------------------------------------------------

\normalsize

% Das esempio-Environment wird nur in der Leseansicht benötigt
\ifkorrekturansicht\else
\newenvironment{esempio}[3]%
{
    \vspace{1.5ex}
    \rlap{\underline{#1}}
    \par
    \setlength{\parindent}{0cm}
    \nopagebreak
    \leftskip=#2cm
    \rightskip=#3cm
}
{
    \par
}
\fi

\doendnotes{C}
\bigskip
\vfill

\clearpage

\footnotesize

\ifkorrekturansicht
  \lohead{\textsc{register}}
\fi

% theindex-Environment neu definieren ohne reledmac
\makeatletter
\renewenvironment{theindex}{%
  \ifkorrekturansicht
    \section*{\indexname}%
  \else
    \subsubsection*{Index der erwähnten Entitäten}%
  \fi
  \setlength{\parindent}{0pt}%
  \setlength{\parskip}{0pt plus 0.3pt}%
  \let\item\@idxitem
}{%
  \ifkorrekturansicht\clearpage\fi
}
\makeatother

\IfFileExists{\jobname-pw.ind}{\input{\jobname-pw.ind}}{}

% Quellenangabe nur in der Leseansicht
\ifkorrekturansicht\else
% Fallback-Definitionen, falls die .tex-Datei \titel etc. nicht gesetzt hat
\providecommand{\titel}{}
\providecommand{\editorInnen}{}
\providecommand{\dateiname}{\jobname}

\vspace{3cm}

\vfill

\footnotesize
\textsc{Quelle}: \titel. Herausgegeben von {\editorInnen}. In: \emph{Arthur Schnitzler: Briefwechsel mit Autorinnen und Autoren}.
 Digitale Edition, https://schnitzler-briefe.acdh.oeaw.ac.at/{\dateiname}.html (Stand \today)
\fi

\end{document}


