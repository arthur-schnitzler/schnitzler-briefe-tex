%% latex-leseansicht-vorspann.tex
%% Vorspann für die Leseansicht.
%% Lädt die gemeinsame Datei latex-vorspann.tex mit nicht gesetztem Schalter.

\newif\ifkorrekturansicht
\korrekturansichtfalse

\input{../tex-inputs/latex-vorspann}


         
         \renewcommand{\erwaehntePersonen}{Personen: Andrea Mantegna}
         \renewcommand{\erwaehnteInstitutionen}{Institutionen: Accademia di belle arti di Venezia, Wiener Schachclub}
         \renewcommand{\erwaehnteOrte}{Orte: Florenz, Frankgasse 1, IX., Alsergrund, Stazione di Venezia Santa Lucia, Venedig, Wien, Österreich}
         \renewcommand{\erwaehnteWerke}{Werke: Sankt Georg}
               \section[Richard Beer-Hofmann an Arthur Schnitzler, 7. 3. 1900]{ Richard Beer-Hofmann an Arthur Schnitzler, 7. 3. 1900}\nopagebreak\mylabel{v}\rehead{ }\begin{ledgroupsized}[t]{13cm}\normalsize\beginnumbering \toendnotes[C]{\smallbreak\pagebreak[2]} \Standort{CUL, Schnitzler, B 8.}
\physDesc{Bildpostkarte, 148 Zeichen
\newline{}Handschrift: schwarze Tinte, lateinische Kurrent
\newline{}Versand: 1) Stempel: »\nobreak{}\oindex{Stazione di Venezia Santa Lucia@\textbf{Stazione di Venezia Santa Lucia}|pwk}Fi\textcolor{gray}{renze}
                                          Fer\textcolor{gray}{rovia}, 8 3 00, 11 M\nobreak{}«.   2) Stempel: »\nobreak{}\oindex{IX., Alsergrund@\textbf{IX., Alsergrund}|pwk}Wien 9/3, 10. 3. 00, 8.V, Bestellt\nobreak{}«. 
\newline{}Ordnung: mit Bleistift von unbekannter Hand nummeriert:
                                    »153« }\pstart{}{\pb}D\textsuperscript{r}
                  Arthur Schnitzler\pend{}\pstart{}Wien\oindex{Wien@\textbf{Wien}|pw}\pend{}\pstart{}IX. Frankgasse 1\oindex{Frankgasse 1@\textbf{Frankgasse 1}|pw}\pend{}\pstart{}Austria\oindex{Oesterreich@\textbf{Österreich}|pw}\pend{}{\bigskip}\pstart
           \noindent{}\centering{}\textcolor{gray}{\textbf{{\pb}Venezia\oindex{Venedig@\textbf{Venedig}|pw}{ }Accademia di Belle Arti\orgindex{Accademia di belle arti di Venezia@Accademia di belle arti di Venezia|pw}.}}\pend
           \pstart
           \noindent{}\centering{}\textcolor{gray}{\textbf{Andrea Mantegna\pwindex{Mantegna, Andrea 1431 – 13.09.1506@\textsc{Mantegna, Andrea} (1431 – 13.09.1506), \emph{Maler, Künstler, Kupferstecher}|pw}, San Giorgio\pwindex{Mantegna, Andrea 1431 – 13.09.1506@\textsc{Mantegna, Andrea} (1431 – 13.09.1506), \emph{Maler, Künstler, Kupferstecher}!Sankt Georgum 1460@\strich\emph{Sankt Georg} {[}um 1460{]}|pw}.}}\pend
           \pstart
           \raggedleft{}Florenz\oindex{Florenz@\textbf{Florenz}|pw}{\\}7/III 1900\pend
           \pstart{}Lieber Arthur\pend\pstart
           ich hoffe Sonntag in Wien\oindex{Wien@\textbf{Wien}|pw} zu sein\pend
           \pstart
           Sind Sie Abends im Schach-Klub\orgindex{Wiener Schachclub@Wiener Schachclub|pw}?\pend
           \pstart
           Herzlich{\\[\baselineskip]}Ihr\spacefill\mbox{R.}\pend
           \leftskip=0em{}
         
         \endnumbering\mylabel{h}\end{ledgroupsized}  \newcommand{\dateiname}{L01019}\newcommand{\titel}{Richard Beer-Hofmann an Arthur Schnitzler, 7. 3. 1900}\newcommand{\editorInnen}{Martin Anton Müller und Gerd-Hermann Susen}%% latex-leseansicht-abspann.tex
%% Abspann für die Leseansicht.
%% Der Schalter \ifkorrekturansicht ist bereits durch den Vorspann gesetzt.

%% latex-abspann.tex
%% Gemeinsamer Abspann für Korrekturansicht und Leseansicht.
%% Setzt den Schalter \ifkorrekturansicht voraus (gesetzt in den
%% einbindenden Dateien latex-korrekturansicht-abspann.tex bzw.
%% latex-leseansicht-abspann.tex).
%% ---------------------------------------------------------------

\normalsize

% Das esempio-Environment wird nur in der Leseansicht benötigt
\ifkorrekturansicht\else
\newenvironment{esempio}[3]%
{
    \vspace{1.5ex}
    \rlap{\underline{#1}}
    \par
    \setlength{\parindent}{0cm}
    \nopagebreak
    \leftskip=#2cm
    \rightskip=#3cm
}
{
    \par
}
\fi

\doendnotes{C}
\bigskip
\vfill

\clearpage

\footnotesize

\ifkorrekturansicht
  \lohead{\textsc{register}}
\fi

% theindex-Environment neu definieren ohne reledmac
\makeatletter
\renewenvironment{theindex}{%
  \ifkorrekturansicht
    \section*{\indexname}%
  \else
    \subsubsection*{Index der erwähnten Entitäten}%
  \fi
  \setlength{\parindent}{0pt}%
  \setlength{\parskip}{0pt plus 0.3pt}%
  \let\item\@idxitem
}{%
  \ifkorrekturansicht\clearpage\fi
}
\makeatother

\IfFileExists{\jobname-pw.ind}{\input{\jobname-pw.ind}}{}

% Quellenangabe nur in der Leseansicht
\ifkorrekturansicht\else
% Fallback-Definitionen, falls die .tex-Datei \titel etc. nicht gesetzt hat
\providecommand{\titel}{}
\providecommand{\editorInnen}{}
\providecommand{\dateiname}{\jobname}

\vspace{3cm}

\vfill

\footnotesize
\textsc{Quelle}: \titel. Herausgegeben von {\editorInnen}. In: \emph{Arthur Schnitzler: Briefwechsel mit Autorinnen und Autoren}.
 Digitale Edition, https://schnitzler-briefe.acdh.oeaw.ac.at/{\dateiname}.html (Stand \today)
\fi

\end{document}


      