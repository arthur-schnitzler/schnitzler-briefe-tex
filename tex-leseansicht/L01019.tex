%% latex-korrekturansicht-vorspann.tex
%% Vorspann für die Korrekturansicht.
%% Lädt die gemeinsame Datei latex-vorspann.tex mit gesetztem Schalter.

\newif\ifkorrekturansicht
\korrekturansichttrue

\input{../tex-inputs/latex-vorspann}


\section[Richard Beer-Hofmann an Arthur Schnitzler, 7. 3. 1900]{L01019 Richard Beer-Hofmann an Arthur Schnitzler, 7. 3. 1900}
\nopagebreak\mylabel{L01019v}
\rehead{ }\normalsize\beginnumbering\briefempfaengerindex{Schnitzler, Arthur@\textsc{Schnitzler, Arthur}!zzzBeer-Hofmann, Richard@\emph{von Richard Beer-Hofmann}!1900-03-071@{7. 3. 1900}|(be}
\toendnotes[C]{\smallbreak\pagebreak[2]}\Standort{CUL, Schnitzler, B 8.}
\physDesc{Bildpostkarte, 148 Zeichen
\newline{}Handschrift: schwarze Tinte, lateinische Kurrent
\newline{}Versand: 1) Stempel: »\nobreak{}\oindex{Stazione di Venezia Santa Lucia@\textbf{Stazione di Venezia Santa Lucia}, \emph{Bahnhofsgebäude (K.BHF)}|pwk}Fi\textcolor{gray}{renze}
                                          Fer\textcolor{gray}{rovia}, 8 3 00, 11 M\nobreak{}«.   2) Stempel: »\nobreak{}\oindex{IX., Alsergrund@\textbf{IX., Alsergrund}, \emph{A.ADM3}|pwk}Wien 9/3, 10. 3. 00, 8.V, Bestellt\nobreak{}«. 
\newline{}Ordnung: mit Bleistift von unbekannter Hand nummeriert:
                                    »153« }\pstart{}{\pb}D\textsuperscript{r}
                  Arthur Schnitzler\pend{}\pstart{}Wien\oindex{Wien@\textbf{Wien}, \emph{A.ADM2}|pw}\pend{}\pstart{}IX. Frankgasse 1\oindex{Frankgasse 1@\textbf{Frankgasse 1}, \emph{Wohngebäude (K.WHS)}|pw}\pend{}\pstart{}Austria\oindex{Oesterreich@\textbf{Österreich}, \emph{A.PCLI}|pw}\pend{}{\bigskip}
\pstart
           \noindent{}\centering{}{\pb}\textcolor{gray}{\textbf{\textsc{Venezia\oindex{Venedig@\textbf{Venedig}, \emph{P.PPLA}|pw}}{ }Accademia di Belle Arti\orgindex{Accademia di belle arti di Venezia@Accademia di belle arti di Venezia|pw}.}}\pend
           
\pstart
           \centering{}\textcolor{gray}{\textbf{Andrea Mantegna\pwindex{Mantegna, Andrea 1431 – 13.09.1506@\textsc{Mantegna, Andrea} (1431 – 13.09.1506), \emph{Maler/Malerin, Künstler/Künstlerin, Kupferstecher/Kupferstecherin}|pw}, San Giorgio\pwindex{Sankt Georg@\emph{Sankt Georg}|pw}.}}\pend
           \vspace{1em}
\pstart
           \raggedleft{}{\pb}Florenz\oindex{Florenz@\textbf{Florenz}, \emph{P.PPLA}|pw}{\\}7/III 1900\pend
           
\pstart{}Lieber Arthur\pend\vspace{0.5em}
\pstart
           ich hoffe Sonntag in Wien\oindex{Wien@\textbf{Wien}, \emph{A.ADM2}|pw} zu sein\pend
           
\pstart
           Sind Sie Abends im Schach-Klub\orgindex{Wiener Schachclub@Wiener Schachclub|pw}?\pend
           
\pstart
           Herzlich{\\[\baselineskip]}Ihr\spacefill\mbox{R.}\pend
           \leftskip=0em{}\selectlanguage{ngerman}\endnumbering\briefempfaengerindex{Schnitzler, Arthur@\textsc{Schnitzler, Arthur}!zzzBeer-Hofmann, Richard@\emph{von Richard Beer-Hofmann}!1900-03-071@{7. 3. 1900}|)be}\mylabel{L01019h}  \normalsize

\doendnotes{C}
\bigskip
\vfill

\clearpage

\footnotesize

\lohead{\textsc{register}}

% Definiere theindex-Environment komplett neu ohne reledmac
\makeatletter
\renewenvironment{theindex}{%
  \section*{\indexname}%
  \setlength{\parindent}{0pt}%
  \setlength{\parskip}{0pt plus 0.3pt}%
  \let\item\@idxitem
}{%
  \clearpage
}
\makeatother

\IfFileExists{\jobname-pw.ind}{\input{\jobname-pw.ind}}{}

\end{document}

      