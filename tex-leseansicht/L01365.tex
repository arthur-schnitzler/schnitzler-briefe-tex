\input{../tex-inputs/latex-pdf-vorspann}
\begin{center}
            \textcolor{red}{ENTWURF. ENTZIFFERUNG NOCH NICHT KORREKTURGELESEN}
                      \end{center}
            
               \section[Arthur Schnitzler an Hermann Bahr, 27. 1. 1904]{ Arthur Schnitzler an Hermann Bahr, 27. 1. 1904}\nopagebreak\mylabel{v}\rehead{ }\begin{ledgroupsized}[t]{13cm}\normalsize\beginnumbering\briefempfaengerindex{Bahr, Hermann@\textsc{Bahr, Hermann}!zzzSchnitzler, Arthur@\emph{von Arthur Schnitzler}!1904-01-271@{27. 1. 1904}|(be} \toendnotes[C]{\smallbreak\pagebreak[2]} \Standort{TMW, HS AM 23364 Ba.}
\physDesc{Kartenbrief
\newline{}Handschrift: schwarze Tinte, deutsche Kurrent\newline{}Versand: 1) Stempel: »\nobreak{}\oindex{XVIII., Waehring@\textbf{XVIII., Währing}|pwk}18/1 Wien, 27. 1. 04, 11–12 V\nobreak{}«.  2) Stempel: »\nobreak{}\oindex{Wangen im Allgaeu@\textbf{Wangen im Allgäu}|pwk}Wangen, 29./1. 04, 9–10 V\nobreak{}«. \newline{}Ordnung: Lochung }\buchAbdrucke{\weitereDrucke{1) \emph{27. 1. 1904.} In: Arthur Schnitzler: \emph{The Letters of Arthur Schnitzler to Hermann Bahr}. Edited, annotated, and with an introduction, by Donald G.
                        Daviau. Chapel Hill: \emph{The University of North Carolina Press} 1978, S. 83 (University of North Carolina studies in the Germanic languages
                        and literatures, 89).} \weitereDrucke{2) Hermann Bahr, Arthur Schnitzler: \emph{Briefwechsel, Aufzeichnungen, Dokumente (1891–1931)}. Hg. Kurt Ifkovits und Martin Anton Müller. Göttingen: \emph{Wallstein} 2018, S. 292.} }\pstart{}{\pb}Herrn \textsc{Hermann Bahr}\pend{}\pstart{}\textsc{Marbach (Sanatorium)\oindex{Sanatorium Schloss Marbach am Bodensee@\textbf{Sanatorium Schloss Marbach am Bodensee}|pw}}\pend{}\pstart{}\textsc{Radolfzell\oindex{Radolfzell am Bodensee@\textbf{Radolfzell am Bodensee}|pw} am Bodensee}\pend{}{\bigskip}\pstart{}{\pb}mein lieber
                  Hermann,\pend\pstart
           möchteſt du mir ein Wort ſchreiben, wie’s dir geht? wie lang du in Marbach\oindex{Marbach am Bodensee@\textbf{Marbach am Bodensee}|pw} bleiben wirſt? –\pend
           \pstart
           Anfang Feber fahre ich nach Berlin\oindex{Berlin@\textbf{Berlin}|pw}, den Einſamen Weg\pwindex{Schnitzler, Arthur 15.05.1862 – 21.10.1931@\textsc{Schnitzler, Arthur} (15.05.1862 – 21.10.1931), \emph{Schriftsteller, Mediziner}!einsame Weg. Schauspiel in fuenf Akten1904@\strich\emph{Der einsame Weg. Schauspiel in fünf Akten} {[}1904{]}|pw} hab ich dir durch Fiſcher\pwindex{Fischer, Samuel 24.12.1859 – 15.10.1934@\textsc{Fischer, Samuel} (24.12.1859 – 15.10.1934), \emph{Verleger}|pw}{ }ſchicken laſſen!\pend
           \pstart
           Herzliche Grüße!{\\[\baselineskip]}Dein getreuer{\\[\baselineskip]}\spacefill\mbox{Arthur}\pend
           \leftskip=0em{}\pstart
           27. 1. 904.\pend
           \endnumbering\briefempfaengerindex{Bahr, Hermann@\textsc{Bahr, Hermann}!zzzSchnitzler, Arthur@\emph{von Arthur Schnitzler}!1904-01-271@{27. 1. 1904}|)be}\mylabel{h}\end{ledgroupsized}  \newcommand{\dateiname}{L01365}\newcommand{\titel}{Arthur Schnitzler an Hermann Bahr, 27. 1. 1904}\newcommand{\editorInnen}{ Kurt Ifkovits,  Martin Anton Müller}\input{../tex-inputs/latex-pdf-abspann}
      