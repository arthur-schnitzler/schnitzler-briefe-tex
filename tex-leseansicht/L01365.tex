%% latex-korrekturansicht-vorspann.tex
%% Vorspann für die Korrekturansicht.
%% Lädt die gemeinsame Datei latex-vorspann.tex mit gesetztem Schalter.

\newif\ifkorrekturansicht
\korrekturansichttrue

\input{../tex-inputs/latex-vorspann}


\section[Arthur Schnitzler an Hermann Bahr, 27. 1. 1904]{L01365 Arthur Schnitzler an Hermann Bahr, 27. 1. 1904}
\nopagebreak\mylabel{L01365v}
\rehead{ }\normalsize\beginnumbering\briefempfaengerindex{Bahr, Hermann@\textsc{Bahr, Hermann}!zzzSchnitzler, Arthur@\emph{von Arthur Schnitzler}!1904-01-271@{27. 1. 1904}|(be}
\toendnotes[C]{\smallbreak\pagebreak[2]}\Standort{TMW, HS AM 23364 Ba.}
\physDesc{Kartenbrief, 304 Zeichen
\newline{}Handschrift: schwarze Tinte, deutsche Kurrent
\newline{}Versand: 1) Stempel: »\nobreak{}\oindex{XVIII., Waehring@\textbf{XVIII., Währing}, \emph{A.ADM3}|pwk}18/1 Wien, 27. 1. 04, 11–12 V\nobreak{}«.   2) Stempel: »\nobreak{}\oindex{Wangen im Allgaeu@\textbf{Wangen im Allgäu}, \emph{P.PPL}|pwk}Wangen, 29./1. 04, 9–10 V\nobreak{}«. 
\newline{}Ordnung: Lochung }
\buchAbdrucke{\weitereDrucke{1) Arthur Schnitzler: \emph{The Letters of Arthur Schnitzler to Hermann Bahr}. Chapel Hill: \emph{The University of North Carolina Press} 1978, S. 83.} \weitereDrucke{2) Hermann Bahr, Arthur Schnitzler: \emph{Briefwechsel, Aufzeichnungen, Dokumente (1891–1931)}. Göttingen: \emph{Wallstein} 2018, S. 292.} }\pstart{}{\pb}Herrn \textsc{Hermann Bahr}\pend{}\pstart{}\textsc{Marbach (Sanatorium)\oindex{Sanatorium Schloss Marbach am Bodensee@\textbf{Sanatorium Schloss Marbach am Bodensee}, \emph{Sanatorium (K.SAN)}|pw}}\pend{}\pstart{}\textsc{Radolfzell\oindex{Radolfzell am Bodensee@\textbf{Radolfzell am Bodensee}, \emph{P.PPL}|pw} am Bodensee}\pend{}{\bigskip}\vspace{1em}
\pstart{}{\pb}mein lieber
                  Hermann,\pend\vspace{0.5em}
\pstart
           möchteſt du mir ein Wort ſchreiben, wie’s dir geht? wie lang du in Marbach\oindex{Oehningen@\textbf{Öhningen}, \emph{A.ADM4}|pw} bleiben wirſt? –\pend
           
\pstart
           Anfang Feber fahre ich nach Berlin\oindex{Berlin@\textbf{Berlin}, \emph{P.PPLC}|pw}, den Einſamen Weg\pwindex{einsame Weg. Schauspiel in fuenf Akten@\emph{Der einsame Weg. Schauspiel in fünf Akten}|pw} hab ich dir durch Fiſcher\pwindex{Fischer, Samuel 24.12.1859 – 15.10.1934@\textsc{Fischer, Samuel} (24.12.1859 – 15.10.1934), \emph{Verleger/Verlegerin}|pw}{ }ſchicken laſſen!\pend
           
\pstart
           Herzliche Grüße!{\\[\baselineskip]}Dein getreuer{\\[\baselineskip]}\spacefill\mbox{Arthur}\pend
           \leftskip=0em{}
\pstart
           27. 1. 904.\pend
           \selectlanguage{ngerman}\endnumbering\briefempfaengerindex{Bahr, Hermann@\textsc{Bahr, Hermann}!zzzSchnitzler, Arthur@\emph{von Arthur Schnitzler}!1904-01-271@{27. 1. 1904}|)be}\mylabel{L01365h}  \normalsize

\doendnotes{C}
\bigskip
\vfill

\clearpage

\footnotesize

\lohead{\textsc{register}}

% Definiere theindex-Environment komplett neu ohne reledmac
\makeatletter
\renewenvironment{theindex}{%
  \section*{\indexname}%
  \setlength{\parindent}{0pt}%
  \setlength{\parskip}{0pt plus 0.3pt}%
  \let\item\@idxitem
}{%
  \clearpage
}
\makeatother

\IfFileExists{\jobname-pw.ind}{\input{\jobname-pw.ind}}{}

\end{document}

      