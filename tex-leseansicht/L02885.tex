%% latex-korrekturansicht-vorspann.tex
%% Vorspann für die Korrekturansicht.
%% Lädt die gemeinsame Datei latex-vorspann.tex mit gesetztem Schalter.

\newif\ifkorrekturansicht
\korrekturansichttrue

\input{../tex-inputs/latex-vorspann}


\section[ Paul Goldmann an Arthur Schnitzler, 30. 8. 1899]{L02885 Paul Goldmann an Arthur Schnitzler, 30. 8. 1899}
\nopagebreak\mylabel{L02885v}
\rehead{ }\normalsize\beginnumbering\briefempfaengerindex{Schnitzler, Arthur@\textsc{Schnitzler, Arthur}!zzzGoldmann, Paul@\emph{von Paul Goldmann}!1899-08-301@{30. 8. 1899}|(be}
\toendnotes[C]{\smallbreak\pagebreak[2]}\Standort{DLA, A:Schnitzler, HS.NZ85.1.3169.}
\physDesc{Bildpostkarte, 395 Zeichen
\newline{}Handschrift: 1) blaue Tinte, deutsche Kurrent\hspace{1em}2) blaue Tinte, lateinische Kurrent (\noindent{}Adresse)\hspace{1em}
\newline{}Versand: Stempel: »\nobreak{}\oindex{Rennes@\textbf{Rennes}, \emph{P.PPLA}|pwk}Rennes Ille-et-Vilaine, 30 AOUT 99, 13\textsuperscript{e}\nobreak{}«. Stempel: »\nobreak{}\oindex{Bad Ischl@\textbf{Bad Ischl}, \emph{P.PPL}|pwk}Ischl, 2. 9. {[}99{]}, 12–1 {[}V{]}\nobreak{}«.  
\newline{}Schnitzler: mit rotem Buntstift eine Unterstreichung }\toendnotes[C]{\smallbreak}\pstart{}{\pb}\begin{otherlanguage}{french}Autriche\oindex{Oesterreich@\textbf{Österreich}, \emph{A.PCLI}|pw}.\end{otherlanguage}\pend{}\pstart{}\begin{otherlanguage}{french}\textcolor{gray}{\textbf{M}}onsieur le Dr. \end{otherlanguage}\pend{}\pstart{}Arthur Schnitzler \pend{}\pstart{}Rudolfshoehe, Pension Petter\oindex{Hotel und Pension Rudolfshoehe (Leopold Petter)@\textbf{Hotel und Pension Rudolfshöhe (Leopold Petter)}, \emph{Hotel (K.HTL)}|pw}\pend{}\pstart{}Ischl\oindex{Bad Ischl@\textbf{Bad Ischl}, \emph{P.PPL}|pw}.\pend{}{\bigskip}
\pstart
           \noindent{}\centering{}{\pb}\textcolor{gray}{\textbf{M\textcolor{gray}{.}{ }\label{K_L02885-1v}\edtext{DEMANGE\pwindex{Demange, Edgar 1841-04-22 – 1925-02-11@\textsc{Demange, Edgar} (1841-04-22 – 1925-02-11), \emph{Anwalt/Anwältin}|pw}}{\lemma{\textnormal{\emph{Demange}}}\Cendnote{\textnormal{Edgar Demange\pwindex{Demange, Edgar 1841-04-22 – 1925-02-11@\textsc{Demange, Edgar} (1841-04-22 – 1925-02-11), \emph{Anwalt/Anwältin}|pwk} war der Anwalt von
                        Alfred Dreyfus\pwindex{Dreyfus, Alfred 1859-10-09 – 1935-07-12@\textsc{Dreyfus, Alfred} (1859-10-09 – 1935-07-12), \emph{Militär/Militärin}|pwk}.}}}\label{K_L02885-1} SORTANT
                     DU CONSEIL DU GUERRE\orgindex{Conseil de guerre de la Xe region militaire de Rennes@Conseil de guerre de la Xe région militaire de Rennes|pw}}}\pend
           \vspace{1em}
\pstart
           \raggedleft{}{\pb}\textcolor{gray}{\textbf{Rennes\oindex{Rennes@\textbf{Rennes}, \emph{P.PPLA}|pw} le}}{ }30. \textsc{août}.\pend
           \vspace{0.5em}
\pstart
           Vielen Dank, liebſter Freund, für Deine Karte und Deinen Brief. Ich
               beantworte ihn ausführlich, ſobald ich Zeit finde. Grüße mir das kleine \label{K_L02885-2v}\edtext{Fräulein\pwindex{Ziegler, Alice 1880-01-05 – Dezember 1943@\textsc{Ziegler, Alice} (1880-01-05 – Dezember 1943)|pwv} aus \textsc{Prag\oindex{Prag@\textbf{Prag}, \emph{A.ADM1}|pw}}}{\lemma{\textnormal{\emph{Fräulein aus Prag}}}\Cendnote{\textnormal{Siehe Paul Goldmann an Arthur Schnitzler, 19. 11. [1897]; die Familie
                     Bondy\pwindex{Ziegler, Alice 1880-01-05 – Dezember 1943@\textsc{Ziegler, Alice} (1880-01-05 – Dezember 1943)|pwkv}\pwindex{Bondy, Charlotte 25.03.1854 – 1914-03-07@\textsc{Bondy, Charlotte} (25.03.1854 – 1914-03-07), \emph{Schauspieler/Schauspielerin}|pwkv}\pwindex{Bondy, Vít Šalomoun 09.12.1831 – 31.10.1909@\textsc{Bondy, Vít Šalomoun} (09.12.1831 – 31.10.1909), \emph{Fabrikant/Fabrikantin}|pwkv} dürfte sich zeitgleich mit Schnitzler in Bad Ischl\oindex{Bad Ischl@\textbf{Bad Ischl}, \emph{P.PPL}|pwk} aufgehalten
                  haben.}}}\label{K_L02885-2} und ſage ihr, daß ich ſie nicht vergeſſen habe. So ſchöne Mädchen
               vergißt man nicht! {\dots} Viele treue Grüße\textcolor{gray}{!}\pend
           \pstart \spacefill\mbox{P. G.}{ }\strikeout{P\textcolor{gray}{.} G\textcolor{gray}{.}}\pend{}
\pstart
           \noindent{}Ein \label{K_L02885-3v}\edtext{Freiſpruch\pwindex{Dreyfus, Alfred 1859-10-09 – 1935-07-12@\textsc{Dreyfus, Alfred} (1859-10-09 – 1935-07-12), \emph{Militär/Militärin}|pwv}}{\lemma{\textnormal{\emph{Freiſpruch}}}\Cendnote{\textnormal{Dreyfus\pwindex{Dreyfus, Alfred 1859-10-09 – 1935-07-12@\textsc{Dreyfus, Alfred} (1859-10-09 – 1935-07-12), \emph{Militär/Militärin}|pwk} wurde am 9. 9. 1899 schuldig gesprochen und am 19. 9. 1899 begnadigt.}}}\label{K_L02885-3} wird immer wahrſcheinlicher.\pend
           \selectlanguage{ngerman}\endnumbering\briefempfaengerindex{Schnitzler, Arthur@\textsc{Schnitzler, Arthur}!zzzGoldmann, Paul@\emph{von Paul Goldmann}!1899-08-301@{30. 8. 1899}|)be}\mylabel{L02885h}  \normalsize

\doendnotes{C}
\bigskip
\vfill

\clearpage

\footnotesize

\lohead{\textsc{register}}

% Definiere theindex-Environment komplett neu ohne reledmac
\makeatletter
\renewenvironment{theindex}{%
  \section*{\indexname}%
  \setlength{\parindent}{0pt}%
  \setlength{\parskip}{0pt plus 0.3pt}%
  \let\item\@idxitem
}{%
  \clearpage
}
\makeatother

\IfFileExists{\jobname-pw.ind}{\input{\jobname-pw.ind}}{}

\end{document}

      