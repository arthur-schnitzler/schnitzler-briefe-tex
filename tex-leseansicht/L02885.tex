%% latex-leseansicht-vorspann.tex
%% Vorspann für die Leseansicht.
%% Lädt die gemeinsame Datei latex-vorspann.tex mit nicht gesetztem Schalter.

\newif\ifkorrekturansicht
\korrekturansichtfalse

\input{../tex-inputs/latex-vorspann}


\section[ Paul Goldmann an Arthur Schnitzler, 30. 8. 1899]{L02885 Paul Goldmann an Arthur Schnitzler,  30. 8. 1899}
\nopagebreak\mylabel{L02885v}
\rehead{ }\normalsize\beginnumbering\briefempfaengerindex{Schnitzler, Arthur@\textsc{Schnitzler, Arthur}!zzzGoldmann, Paul@\emph{von Paul Goldmann}!1899-08-301@{30. 8. 1899}|(be}
\toendnotes[C]{\smallbreak\pagebreak[2]}
\correspDesc{Versand  durch Paul Goldmann am 30. 8. 1899 in Rennes
\newline{}Erhalt  durch Arthur Schnitzler im Zeitraum [31. 8. 1899
                  – 4. 9. 1899?] in Bad Ischl}\toendnotes[C]{\smallbreak}
\Standort{DLA, A:Schnitzler, HS.NZ85.1.3169.}
\physDesc{Bildpostkarte, 395 Zeichen
\newline{}Handschrift: blaue Tinte, deutsche Kurrent
\newline{}Versand: Stempel: »\nobreak{}\oindex{Rennes@\textbf{Rennes}|pwk}Rennes Ille-et-Vilaine, 30 AOUT 99, 13\textsuperscript{e}\nobreak{}«. Stempel: »\nobreak{}\oindex{Bad Ischl@\textbf{Bad Ischl}|pwk}Ischl, 2. 9. {[}99{]}, 12–1 {[}V{]}\nobreak{}«.  
\newline{}Schnitzler: mit rotem Buntstift eine Unterstreichung }\toendnotes[C]{\smallbreak}\pstart{}\textsc{{\pb}\begin{otherlanguage}{french}Autriche\oindex{Österreich@\textbf{Österreich}|pw}.\end{otherlanguage}}\pend{}\pstart{}\textsc{\begin{otherlanguage}{french}\textcolor{gray}{\textbf{M}}onsieur le Dr. \end{otherlanguage}}\pend{}\pstart{}\textsc{Arthur Schnitzler}\pend{}\pstart{}\textsc{Rudolfshoehe, Pension Petter\oindex{Hotel und Pension Rudolfshöhe (Leopold Petter)@\textbf{Hotel und Pension Rudolfshöhe (Leopold Petter)}, \emph{Hotel}|pw}}\pend{}\pstart{}\textsc{Ischl\oindex{Bad Ischl@\textbf{Bad Ischl}|pw}.}\pend{}{\bigskip}
\pstart
           \noindent{}\centering{}{\pb}\textcolor{gray}{\textbf{M\textcolor{gray}{.}{ }\label{K_L02885-1v}\edtext{DEMANGE\pwindex{Demange, Edgar 22.\,4.\,1841 Versailles – 11.\,2.\,1925 Paris@\textsc{Demange, Edgar} (22.\,4.\,1841 Versailles – 11.\,2.\,1925 Paris), \emph{Anwalt}|pw}}{\lemma{\textnormal{\emph{Demange}}}\Cendnote{\textnormal{Edgar Demange\pwindex{Demange, Edgar 22.\,4.\,1841 Versailles – 11.\,2.\,1925 Paris@\textsc{Demange, Edgar} (22.\,4.\,1841 Versailles – 11.\,2.\,1925 Paris), \emph{Anwalt}|pwk} war der Anwalt von
                        Alfred Dreyfus\pwindex{Dreyfus, Alfred 9.\,10.\,1859 Mulhouse – 12.\,7.\,1935 Paris@\textsc{Dreyfus, Alfred} (9.\,10.\,1859 Mulhouse – 12.\,7.\,1935 Paris), \emph{Militär}|pwk}.}}}\label{K_L02885-1} SORTANT
                     DU CONSEIL DU GUERRE\orgindex{Conseil de guerre de la Xe région militaire de Rennes@Conseil de guerre de la Xe région militaire de Rennes|pw}}}\pend
           \vspace{1em}
\pstart
           \raggedleft{}{\pb}\textcolor{gray}{\textbf{Rennes\oindex{Rennes@\textbf{Rennes}|pw} le}}{ }30. \textsc{août}.\pend
           \vspace{0.5em}
\pstart
           Vielen Dank, liebſter Freund, für Deine Karte und Deinen Brief. Ich
               beantworte ihn ausführlich,{ }ſobald ich Zeit finde. Grüße mir das kleine \label{K_L02885-2v}\edtext{Fräulein\pwindex{Ziegler, Alice 5.\,1.\,1880 Prag – Dezember 1943 Konzentrationslager Auschwitz-Birkenau@\textsc{Ziegler, Alice} (5.\,1.\,1880 Prag – Dezember 1943 Konzentrationslager Auschwitz-Birkenau)|pwv} aus \textsc{Prag\oindex{Prag@\textbf{Prag}, \emph{Land}|pw}}}{\lemma{\textnormal{\emph{Fräulein aus Prag}}}\Cendnote{\textnormal{Siehe XXXX Auszeichnungsfehler: Dokument L02831 nicht gefunden; die Familie
                     Bondy\pwindex{Ziegler, Alice 5.\,1.\,1880 Prag – Dezember 1943 Konzentrationslager Auschwitz-Birkenau@\textsc{Ziegler, Alice} (5.\,1.\,1880 Prag – Dezember 1943 Konzentrationslager Auschwitz-Birkenau)|pwkv}\pwindex{Bondy, Charlotte 25.\,3.\,1854 Bielsko-Biała – 7.\,3.\,1914 Prag@\textsc{Bondy, Charlotte} (25.\,3.\,1854 Bielsko-Biała – 7.\,3.\,1914 Prag), \emph{Schauspielerin}|pwkv}\pwindex{Bondy, Vít Šalomoun 9.\,12.\,1831 Prag – 31.\,10.\,1909 ebd.@\textsc{Bondy, Vít Šalomoun} (9.\,12.\,1831 Prag – 31.\,10.\,1909 ebd.), \emph{Fabrikant}|pwkv} dürfte sich zeitgleich mit Schnitzler in Bad Ischl\oindex{Bad Ischl@\textbf{Bad Ischl}|pwk} aufgehalten
                  haben.}}}\label{K_L02885-2} und{ }ſage ihr, daß ich{ }ſie nicht vergeſſen habe. So{ }ſchöne Mädchen
               vergißt man nicht! {\dots} Viele treue Grüße\textcolor{gray}{!}\pend
           \pstart \spacefill\mbox{P. G.}{ }\strikeout{P\textcolor{gray}{.} G\textcolor{gray}{.}}\pend{}
\pstart
           \noindent{}Ein \label{K_L02885-3v}\edtext{Freiſpruch\pwindex{Dreyfus, Alfred 9.\,10.\,1859 Mulhouse – 12.\,7.\,1935 Paris@\textsc{Dreyfus, Alfred} (9.\,10.\,1859 Mulhouse – 12.\,7.\,1935 Paris), \emph{Militär}|pwv}}{\lemma{\textnormal{\emph{Freispruch}}}\Cendnote{\textnormal{Dreyfus\pwindex{Dreyfus, Alfred 9.\,10.\,1859 Mulhouse – 12.\,7.\,1935 Paris@\textsc{Dreyfus, Alfred} (9.\,10.\,1859 Mulhouse – 12.\,7.\,1935 Paris), \emph{Militär}|pwk} wurde am 9. 9. 1899 schuldig gesprochen und am 19. 9. 1899 begnadigt.}}}\label{K_L02885-3} wird immer wahrſcheinlicher.\pend
           \selectlanguage{ngerman}\endnumbering\briefempfaengerindex{Schnitzler, Arthur@\textsc{Schnitzler, Arthur}!zzzGoldmann, Paul@\emph{von Paul Goldmann}!1899-08-301@{30. 8. 1899}|)be}\mylabel{L02885h}  \newcommand{\dateiname}{L02885}\newcommand{\titel}{Paul Goldmann an Arthur Schnitzler, 30. 8. 1899}\newcommand{\editorInnen}{Martin Anton Müller und Laura Untner}%% latex-leseansicht-abspann.tex
%% Abspann für die Leseansicht.
%% Der Schalter \ifkorrekturansicht ist bereits durch den Vorspann gesetzt.

%% latex-abspann.tex
%% Gemeinsamer Abspann für Korrekturansicht und Leseansicht.
%% Setzt den Schalter \ifkorrekturansicht voraus (gesetzt in den
%% einbindenden Dateien latex-korrekturansicht-abspann.tex bzw.
%% latex-leseansicht-abspann.tex).
%% ---------------------------------------------------------------

\normalsize

% Das esempio-Environment wird nur in der Leseansicht benötigt
\ifkorrekturansicht\else
\newenvironment{esempio}[3]%
{
    \vspace{1.5ex}
    \rlap{\underline{#1}}
    \par
    \setlength{\parindent}{0cm}
    \nopagebreak
    \leftskip=#2cm
    \rightskip=#3cm
}
{
    \par
}
\fi

\doendnotes{C}
\bigskip
\vfill

\clearpage

\footnotesize

\ifkorrekturansicht
  \lohead{\textsc{register}}
\fi

% theindex-Environment neu definieren ohne reledmac
\makeatletter
\renewenvironment{theindex}{%
  \ifkorrekturansicht
    \section*{\indexname}%
  \else
    \subsubsection*{Index der erwähnten Entitäten}%
  \fi
  \setlength{\parindent}{0pt}%
  \setlength{\parskip}{0pt plus 0.3pt}%
  \let\item\@idxitem
}{%
  \ifkorrekturansicht\clearpage\fi
}
\makeatother

\IfFileExists{\jobname-pw.ind}{\input{\jobname-pw.ind}}{}

% Quellenangabe nur in der Leseansicht
\ifkorrekturansicht\else
% Fallback-Definitionen, falls die .tex-Datei \titel etc. nicht gesetzt hat
\providecommand{\titel}{}
\providecommand{\editorInnen}{}
\providecommand{\dateiname}{\jobname}

\vspace{3cm}

\vfill

\footnotesize
\textsc{Quelle}: \titel. Herausgegeben von {\editorInnen}. In: \emph{Arthur Schnitzler: Briefwechsel mit Autorinnen und Autoren}.
 Digitale Edition, https://schnitzler-briefe.acdh.oeaw.ac.at/{\dateiname}.html (Stand \today)
\fi

\end{document}


