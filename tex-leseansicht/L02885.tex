%% latex-leseansicht-vorspann.tex
%% Vorspann für die Leseansicht.
%% Lädt die gemeinsame Datei latex-vorspann.tex mit nicht gesetztem Schalter.

\newif\ifkorrekturansicht
\korrekturansichtfalse

\input{../tex-inputs/latex-vorspann}


         
         \renewcommand{\erwaehntePersonen}{Personen: Charlotte Bondy, Vít Šalomoun Bondy, Edgar Demange, Alfred Dreyfus, Alice Ziegler}
         \renewcommand{\erwaehnteInstitutionen}{Institutionen: Conseil de guerre de la Xe région militaire de Rennes}
         \renewcommand{\erwaehnteOrte}{Orte: Bad Ischl, Hotel und Pension Rudolfshöhe (Leopold Petter), Prag, Rennes, Österreich}
         \renewcommand{\erwaehnteWerke}{}
               \section[ Paul Goldmann an Arthur Schnitzler, 30. 8. 1899]{ Paul Goldmann an Arthur Schnitzler, 30. 8. 1899}\nopagebreak\mylabel{v}\rehead{ }\begin{ledgroupsized}[t]{13cm}\normalsize\beginnumbering \toendnotes[C]{\smallbreak\pagebreak[2]} \Standort{DLA, A:Schnitzler, HS.NZ85.1.3169.}
\physDesc{Bildpostkarte, 395 Zeichen
\newline{}Handschrift: 1) blaue Tinte, deutsche Kurrent\hspace{1em}2) blaue Tinte, lateinische Kurrent (\noindent{}Adresse)\hspace{1em}
\newline{}Versand: Stempel: »\nobreak{}\oindex{Rennes@\textbf{Rennes}|pwk}Rennes Ille-et-Vilaine, 30 AOUT 99, 13\textsuperscript{e}\nobreak{}«. Stempel: »\nobreak{}\oindex{Bad Ischl@\textbf{Bad Ischl}|pwk}Ischl, 2. 9. {[}99{]}, 12–1 {[}V{]}\nobreak{}«.  
\newline{}Schnitzler: mit rotem Buntstift eine Unterstreichung }\toendnotes[C]{\smallbreak}\pstart{}{\pb}\begin{otherlanguage}{french}Autriche\oindex{Oesterreich@\textbf{Österreich}|pw}.\end{otherlanguage}\pend{}\pstart{}\begin{otherlanguage}{french}\textcolor{gray}{\textbf{M}}onsieur le Dr. \end{otherlanguage}\pend{}\pstart{}Arthur Schnitzler \pend{}\pstart{}Rudolfshoehe, Pension Petter\oindex{Hotel und Pension Rudolfshoehe (Leopold Petter)@\textbf{Hotel und Pension Rudolfshöhe (Leopold Petter)}|pw}\pend{}\pstart{}Ischl\oindex{Bad Ischl@\textbf{Bad Ischl}|pw}.\pend{}{\bigskip}\pstart
           \noindent{}\centering{}{\pb}\textcolor{gray}{\textbf{M\textcolor{gray}{.}{ }\label{K_L02885-1v}\edtext{DEMANGE\pwindex{Demange, Edgar 1841-04-22 – 1925-02-11@\textsc{Demange, Edgar} (1841-04-22 – 1925-02-11), \emph{Anwalt}|pw}}{\lemma{\textnormal{\emph{Demange}}}\Cendnote{\textnormal{Edgar Demange\pwindex{Demange, Edgar 1841-04-22 – 1925-02-11@\textsc{Demange, Edgar} (1841-04-22 – 1925-02-11), \emph{Anwalt}|pwk} war der Anwalt von
                              Alfred Dreyfus\pwindex{Dreyfus, Alfred 1859-10-09 – 1935-07-12@\textsc{Dreyfus, Alfred} (1859-10-09 – 1935-07-12), \emph{Militär}|pwk}.}}}\label{K_L02885-1h} SORTANT
                        DU CONSEIL DU GUERRE\orgindex{Conseil de guerre de la Xe region militaire de Rennes@Conseil de guerre de la Xe région militaire de Rennes|pw}}}\pend
           \pstart
           \raggedleft{}\textcolor{gray}{\textbf{Rennes\oindex{Rennes@\textbf{Rennes}|pw} le}}{ }30. \textsc{août}.\pend
           \pstart
           Vielen Dank, liebſter Freund, für Deine Karte und Deinen Brief. Ich
               beantworte ihn ausführlich, ſobald ich Zeit finde. Grüße mir das kleine \label{K_L02885-2v}\edtext{Fräulein\pwindex{Ziegler, Alice 1880-01-05 – Dezember 1943@\textsc{Ziegler, Alice} (1880-01-05 – Dezember 1943)|pwv} aus \textsc{Prag\oindex{Prag@\textbf{Prag}|pw}}}{\lemma{\textnormal{\emph{Fräulein aus Prag}}}\Cendnote{\textnormal{Siehe Paul Goldmann an Arthur Schnitzler, 19. 11. [1897]; die Familie
                     Bondy\pwindex{Ziegler, Alice 1880-01-05 – Dezember 1943@\textsc{Ziegler, Alice} (1880-01-05 – Dezember 1943)|pwkv}\pwindex{Bondy, Charlotte 25.03.1854 – 1914-03-07@\textsc{Bondy, Charlotte} (25.03.1854 – 1914-03-07), \emph{Schauspielerin}|pwkv}\pwindex{Bondy, Vít Šalomoun 09.12.1831 – 31.10.1909@\textsc{Bondy, Vít Šalomoun} (09.12.1831 – 31.10.1909), \emph{Fabrikant}|pwkv} dürfte sich zeitgleich mit Schnitzler\pwindex{Schnitzler, Arthur 15.05.1862 – 21.10.1931@\textsc{Schnitzler, Arthur} (15.05.1862 – 21.10.1931), \emph{Schriftsteller, Mediziner}|pwk} in Bad Ischl\oindex{Bad Ischl@\textbf{Bad Ischl}|pwk} aufgehalten
                  haben.}}}\label{K_L02885-2h} und ſage ihr, daß ich ſie nicht vergeſſen habe. So ſchöne Mädchen
               vergißt man nicht! {\dots} Viele treue Grüße\textcolor{gray}{!}\pend
           \pstart \spacefill\mbox{P. G.}{ }\strikeout{P\textcolor{gray}{.} G\textcolor{gray}{.}}\pend{}\pstart
           \noindent{}Ein \label{K_L02885-3v}\edtext{Freiſpruch\pwindex{Dreyfus, Alfred 1859-10-09 – 1935-07-12@\textsc{Dreyfus, Alfred} (1859-10-09 – 1935-07-12), \emph{Militär}|pwv}}{\lemma{\textnormal{\emph{Freiſpruch}}}\Cendnote{\textnormal{Dreyfus\pwindex{Dreyfus, Alfred 1859-10-09 – 1935-07-12@\textsc{Dreyfus, Alfred} (1859-10-09 – 1935-07-12), \emph{Militär}|pwk} wurde am 9. 9. 1899 schuldig gesprochen und am 19. 9. 1899 begnadigt.}}}\label{K_L02885-3h} wird immer wahrſcheinlicher.\pend
           
         
         \endnumbering\mylabel{h}\end{ledgroupsized}  \newcommand{\dateiname}{L02885}\newcommand{\titel}{Paul Goldmann an Arthur Schnitzler, 30. 8. 1899}\newcommand{\editorInnen}{Martin Anton Müller und Laura Untner}%% latex-leseansicht-abspann.tex
%% Abspann für die Leseansicht.
%% Der Schalter \ifkorrekturansicht ist bereits durch den Vorspann gesetzt.

%% latex-abspann.tex
%% Gemeinsamer Abspann für Korrekturansicht und Leseansicht.
%% Setzt den Schalter \ifkorrekturansicht voraus (gesetzt in den
%% einbindenden Dateien latex-korrekturansicht-abspann.tex bzw.
%% latex-leseansicht-abspann.tex).
%% ---------------------------------------------------------------

\normalsize

% Das esempio-Environment wird nur in der Leseansicht benötigt
\ifkorrekturansicht\else
\newenvironment{esempio}[3]%
{
    \vspace{1.5ex}
    \rlap{\underline{#1}}
    \par
    \setlength{\parindent}{0cm}
    \nopagebreak
    \leftskip=#2cm
    \rightskip=#3cm
}
{
    \par
}
\fi

\doendnotes{C}
\bigskip
\vfill

\clearpage

\footnotesize

\ifkorrekturansicht
  \lohead{\textsc{register}}
\fi

% theindex-Environment neu definieren ohne reledmac
\makeatletter
\renewenvironment{theindex}{%
  \ifkorrekturansicht
    \section*{\indexname}%
  \else
    \subsubsection*{Index der erwähnten Entitäten}%
  \fi
  \setlength{\parindent}{0pt}%
  \setlength{\parskip}{0pt plus 0.3pt}%
  \let\item\@idxitem
}{%
  \ifkorrekturansicht\clearpage\fi
}
\makeatother

\IfFileExists{\jobname-pw.ind}{\input{\jobname-pw.ind}}{}

% Quellenangabe nur in der Leseansicht
\ifkorrekturansicht\else
% Fallback-Definitionen, falls die .tex-Datei \titel etc. nicht gesetzt hat
\providecommand{\titel}{}
\providecommand{\editorInnen}{}
\providecommand{\dateiname}{\jobname}

\vspace{3cm}

\vfill

\footnotesize
\textsc{Quelle}: \titel. Herausgegeben von {\editorInnen}. In: \emph{Arthur Schnitzler: Briefwechsel mit Autorinnen und Autoren}.
 Digitale Edition, https://schnitzler-briefe.acdh.oeaw.ac.at/{\dateiname}.html (Stand \today)
\fi

\end{document}


      