\input{../tex-inputs/latex-pdf-vorspann}
\begin{center}
            \textcolor{red}{ENTWURF. ENTZIFFERUNG NOCH NICHT KORREKTURGELESEN}
                      \end{center}
            
               \section[Arthur Schnitzler an Robert Adam, 12. 6. 1930]{ Arthur Schnitzler an Robert Adam, 12. 6. 1930}\nopagebreak\mylabel{v}\rehead{ }\begin{ledgroupsized}[t]{13cm}\normalsize\beginnumbering\briefempfaengerindex{Adam, Robert@\textsc{Adam, Robert}!zzzSchnitzler, Arthur@\emph{von Arthur Schnitzler}!1930-06-121@{12. 6. 1930}|(be} \toendnotes[C]{\smallbreak\pagebreak[2]} \Standort{DLA, 96.34.1/6.}
\physDesc{Brief, 1 Blatt, 1 Seite, Umschlag
\newline{}Handschrift: schwarze Tinte, lateinische Kurrent\newline{}Versand: Stempel: »\nobreak{}Wien, 13. 6. 1930, 11\nobreak{}«.  }\toendnotes[C]{\smallbreak}\pstart{}{\pb}\label{T_L02536-1v}\edtext{\textcolor{gray}{\textbf{A. S.}}}{\lemma{\textnormal{\emph{A. S.}}}\Cendnote{\textnormal{ovaler Absenderkleber}}}\label{T_L02536-1h}\pend{}\pstart{}\textcolor{gray}{\textbf{WIEN, XVIII.}}\oindex{XVIII., Waehring@\textbf{XVIII., Währing}|pw}\pend{}\pstart{}\textcolor{gray}{\textbf{STERNWARTESTR. 71}}\oindex{Sternwartestrasse@\textbf{Sternwartestraße}|pw}\pend{}{\bigskip}\pstart{}{\pb}Hrn Ob. Landesgerichts
                        Rath\pend{}\pstart{}Dr. Robert Adam Pollak,\pend{}\pstart{}Vicepraesident des Handelsgerichts\orgindex{Handelsgericht Wien@Handelsgericht Wien|pw}\pend{}\pstart{}Wien XII\oindex{XII., Meidling@\textbf{XII., Meidling}|pw}\pend{}\pstart{}Meidlinger Hptstr 56\oindex{Meidlinger Hauptstrasse@\textbf{Meidlinger Hauptstraße}|pw}.\pend{}{\bigskip}\pstart
           \raggedleft{}{\pb}Wien\oindex{Wien@\textbf{Wien}|pw}, 12. 6. 930\pend
           \pstart{}Verehrter Herr Doctor,\pend\pstart
           lassen Sie mich Ihnen zur Erne{\geminationn}ung zum
                    Vicepraesidenten des Handelsgerichtes\orgindex{Handelsgericht Wien@Handelsgericht Wien|pw}
                    herzlichst gratulieren; – zugleich zur Annahme der Margot\pwindex{Adam, Robert 20.04.1877 – 16.10.1961@\textsc{Adam, Robert} (20.04.1877 – 16.10.1961), \emph{Schriftsteller, Richter}!Margot und das Jugendgericht1931@\strich\emph{Margot und das Jugendgericht} {[}1931{]}|pw} in Frankfurt\oindex{Frankfurt am Main@\textbf{Frankfurt am Main}|pw}; – ich freue
                    mich, daſs der Erfolg von ein paar Seiten zugleich heranko{\geminationm}t; wenigen wünsch ich so überzeugt ein
                    freundschaftliches Gelingen auf jedem Gebiet wie Ihnen – de{\geminationn} wenige verdienen es wie Sie.\pend
           \pstart
           Ich grüße Sie \textcolor{gray}{schönstens} und hoffe wir sehen und sprechen
                    einander wieder.{\\[\baselineskip]}Ihr{\\[\baselineskip]}\spacefill\mbox{ArthSchnitzler}\pend
           \leftskip=0em{}\endnumbering\briefempfaengerindex{Adam, Robert@\textsc{Adam, Robert}!zzzSchnitzler, Arthur@\emph{von Arthur Schnitzler}!1930-06-121@{12. 6. 1930}|)be}\mylabel{h}\end{ledgroupsized}  \newcommand{\dateiname}{L02536}\newcommand{\titel}{Arthur Schnitzler an Robert Adam, 12. 6. 1930}\newcommand{\editorInnen}{Martin Anton Müller und Gerd-Hermann Susen}\input{../tex-inputs/latex-pdf-abspann}
      