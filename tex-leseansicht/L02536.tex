%% latex-korrekturansicht-vorspann.tex
%% Vorspann für die Korrekturansicht.
%% Lädt die gemeinsame Datei latex-vorspann.tex mit gesetztem Schalter.

\newif\ifkorrekturansicht
\korrekturansichttrue

\input{../tex-inputs/latex-vorspann}


\section[Arthur Schnitzler an Robert Adam, 12. 6. 1930]{L02536 Arthur Schnitzler an Robert Adam, 12. 6. 1930}
\nopagebreak\mylabel{L02536v}
\rehead{ }\normalsize\beginnumbering\briefempfaengerindex{Adam, Robert@\textsc{Adam, Robert}!zzzSchnitzler, Arthur@\emph{von Arthur Schnitzler}!1930-06-121@{12. 6. 1930}|(be}
\toendnotes[C]{\smallbreak\pagebreak[2]}\Standort{DLA, 96.34.1/6.}
\physDesc{Brief, 1 Blatt, 1 Seite, Umschlag, 580 Zeichen
\newline{}Handschrift: schwarze Tinte, lateinische Kurrent
\newline{}Versand: Stempel: »\nobreak{}Wien, 13. 6. 1930, 11\nobreak{}«.  }\toendnotes[C]{\smallbreak}\pstart{}{\pb}\label{T_L02536-1v}\edtext{\textcolor{gray}{\textbf{A. S.}}}{\lemma{\textnormal{\emph{A. S.}}}\Cendnote{\textnormal{ovaler Absenderkleber}}}\label{T_L02536-1}\pend{}\pstart{}\textcolor{gray}{\textbf{WIEN, XVIII.}}\oindex{XVIII., Waehring@\textbf{XVIII., Währing}, \emph{A.ADM3}|pw}\pend{}\pstart{}\textcolor{gray}{\textbf{STERNWARTESTR. 71}}\oindex{Sternwartestrasse 71@\textbf{Sternwartestraße 71}, \emph{Wohngebäude (K.WHS)}|pw}\pend{}{\bigskip}\pstart{}{\pb}Hrn Ob. Landesgerichts Rath\pend{}\pstart{}Dr. Robert Adam Pollak,\pend{}\pstart{}Vicepraesident des Handelsgerichts\orgindex{Handelsgericht Wien@Handelsgericht Wien|pw}\pend{}\pstart{}Wien XII\oindex{XII., Meidling@\textbf{XII., Meidling}, \emph{A.ADM3}|pw}\pend{}\pstart{}Meidlinger Hptstr 56\oindex{Meidlinger Hauptstrasse@\textbf{Meidlinger Hauptstraße}, \emph{Straße (K.STR)}|pw}.\pend{}{\bigskip}\vspace{1em}
\pstart
           \raggedleft{}{\pb}Wien\oindex{Wien@\textbf{Wien}, \emph{A.ADM2}|pw}, 12. 6. 930\pend
           
\pstart{}Verehrter Herr Doctor,\pend\vspace{0.5em}
\pstart
           lassen Sie mich Ihnen zur Erne{\geminationn}ung zum Vicepraesidenten
               des Handelsgerichtes\orgindex{Handelsgericht Wien@Handelsgericht Wien|pw} herzlichst gratulieren; –
               zugleich zur Annahme der Margot\pwindex{Margot und das Jugendgericht@\emph{Margot und das Jugendgericht}|pw} in Frankfurt\oindex{Frankfurt am Main@\textbf{Frankfurt am Main}, \emph{P.PPLA3}|pw}; – ich freue mich, daſs der Erfolg von
               ein paar Seiten zugleich heranko{\geminationm}t; wenigen wünsch ich
               so überzeugt ein freundschaftliches Gelingen auf jedem Gebiet wie Ihnen – de{\geminationn} wenige verdienen es wie Sie.\pend
           
\pstart
           Ich grüße Sie \textcolor{gray}{schönstens} und hoffe wir sehen und sprechen
               einander wieder.{\\[\baselineskip]}Ihr{\\[\baselineskip]}\spacefill\mbox{ArthSchnitzler}\pend
           \leftskip=0em{}\selectlanguage{ngerman}\endnumbering\briefempfaengerindex{Adam, Robert@\textsc{Adam, Robert}!zzzSchnitzler, Arthur@\emph{von Arthur Schnitzler}!1930-06-121@{12. 6. 1930}|)be}\mylabel{L02536h}  \normalsize

\doendnotes{C}
\bigskip
\vfill

\clearpage

\footnotesize

\lohead{\textsc{register}}

% Definiere theindex-Environment komplett neu ohne reledmac
\makeatletter
\renewenvironment{theindex}{%
  \section*{\indexname}%
  \setlength{\parindent}{0pt}%
  \setlength{\parskip}{0pt plus 0.3pt}%
  \let\item\@idxitem
}{%
  \clearpage
}
\makeatother

\IfFileExists{\jobname-pw.ind}{\input{\jobname-pw.ind}}{}

\end{document}

      