%% latex-korrekturansicht-vorspann.tex
%% Vorspann für die Korrekturansicht.
%% Lädt die gemeinsame Datei latex-vorspann.tex mit gesetztem Schalter.

\newif\ifkorrekturansicht
\korrekturansichttrue

\input{../tex-inputs/latex-vorspann}


\section[Albert Ehrenstein an Arthur Schnitzler, 25. 11. 1909]{L01889 Albert Ehrenstein an Arthur Schnitzler, 25. 11. 1909}
\nopagebreak\mylabel{L01889v}
\rehead{ }\normalsize\beginnumbering\briefempfaengerindex{Schnitzler, Arthur@\textsc{Schnitzler, Arthur}!zzzEhrenstein, Albert@\emph{von Albert Ehrenstein}!1909-11-251@{25. 11. 1909}|(be}
\toendnotes[C]{\smallbreak\pagebreak[2]}\Standort{CUL, Schnitzler, B 30.}
\physDesc{Brief, 1 Blatt, 2 Seiten, 846 Zeichen
\newline{}Handschrift: schwarze Tinte, deutsche Kurrent
\newline{}Schnitzler: mit Bleistift beschriftet: »\textsc{Ehrenstein}« }\toendnotes[C]{\smallbreak}
\pstart
           
\pstart
           {\pb}XVI. \textsc{Ottakringerstr}.
                        114\oindex{Ottakringer Strasse@\textbf{Ottakringer Straße}, \emph{Straße (K.STR)}|pw}\pend
           
\pstart
           \raggedleft{}25. XI. 09\pend
           \pend
           
\pstart{}Sehr geehrter Herr Doktor,\pend\vspace{0.5em}
\pstart
           von den bei Ihnen liegenden Manuſkripten ſind, wie ich bereits im Begleitſchreiben
               erwähnte, für Sie bloß Saccumum\pwindex{Saccumum@\emph{Saccumum}|pw} und »Mitgefühl\pwindex{Mitgefuehl@\emph{Mitgefühl}|pw}« unbekannt, welche übrigens, wie ich
               fürchte, kaum geeignet ſind, Ihr Urteil über meine dermaligen Leiſtungen zu
               modifizieren. Obwohl ich mir nun nicht verhehlen kann, daß über meine Sachen faſt
               mehr hin- und hergeſchrieben und geſprochen wurde, als {\pb}ſie überhaupt wert ſind, trotzdem wäre ich
               Ihnen, hochverehrter Herr Doktor, ſehr dankbar, wenn Sie die Güte hätten, die zwei
               genannten Skizzen\pwindex{Saccumum@\emph{Saccumum}|pwv}\pwindex{Mitgefuehl@\emph{Mitgefühl}|pwv} zu
               leſen, in den anderen zu blättern und mir dann in der nächſten Woche darüber wie auch
               über die andere Angelegenheit Ihre mir notwendige Meinung zu sagen. Es wird mich
               freuen, wenn all dies Ihre Zeiteinteilung zuläßt.\pend
           
\pstart
           Hochachtungsvoll{\\[\baselineskip]}Ihr ergebenſter{\\[\baselineskip]}\spacefill\mbox{Albert Ehrenstein.}\pend
           \leftskip=0em{}\selectlanguage{ngerman}\endnumbering\briefempfaengerindex{Schnitzler, Arthur@\textsc{Schnitzler, Arthur}!zzzEhrenstein, Albert@\emph{von Albert Ehrenstein}!1909-11-251@{25. 11. 1909}|)be}\mylabel{L01889h}  \normalsize

\doendnotes{C}
\bigskip
\vfill

\clearpage

\footnotesize

\lohead{\textsc{register}}

% Definiere theindex-Environment komplett neu ohne reledmac
\makeatletter
\renewenvironment{theindex}{%
  \section*{\indexname}%
  \setlength{\parindent}{0pt}%
  \setlength{\parskip}{0pt plus 0.3pt}%
  \let\item\@idxitem
}{%
  \clearpage
}
\makeatother

\IfFileExists{\jobname-pw.ind}{\input{\jobname-pw.ind}}{}

\end{document}

      