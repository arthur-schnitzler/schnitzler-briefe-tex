%% latex-leseansicht-vorspann.tex
%% Vorspann für die Leseansicht.
%% Lädt die gemeinsame Datei latex-vorspann.tex mit nicht gesetztem Schalter.

\newif\ifkorrekturansicht
\korrekturansichtfalse

\input{../tex-inputs/latex-vorspann}


\section[Max Burckhard an Arthur Schnitzler, 12. 2. 1894]{L00298 Max Burckhard an Arthur Schnitzler, 12. 2. 1894}
\nopagebreak\mylabel{L00298v}
\rehead{ }\normalsize\beginnumbering\briefempfaengerindex{Schnitzler, Arthur@\textsc{Schnitzler, Arthur}!zzzBurckhard, Max Eugen@\emph{von Max Eugen Burckhard}!1894-02-121@{12. 2. 1894}|(be}
\toendnotes[C]{\smallbreak\pagebreak[2]}
\correspDesc{Versand  durch Max Burckhard am 12. 2. 1894 in Wien
\newline{}Erhalt  durch Arthur Schnitzler im Zeitraum [12. 2. 1894
                  – 16. 2. 1894?] in Wien}\toendnotes[C]{\smallbreak}
\Standort{CUL, Schnitzler, B 20.}
\physDesc{Brief, 1 Blatt, 1 Seite, 345 Zeichen
\newline{}Handschrift: schwarze Tinte, deutsche Kurrent
\newline{}Ordnung: mit rotem Buntstift von unbekannter Hand nummeriert:
                                    »2«, mutmaßlich von anderer Hand mit Bleistift
                                 durchgestrichen und nummeriert: »4« }\toendnotes[C]{\smallbreak}
\pstart
           {\pb}\textcolor{gray}{\textbf{\label{T_L00298-1v}\edtext{k. k. Hofburgtheater
                              Direction}{\lemma{\textnormal{\emph{k. k. … Direction}}}\Cendnote{\textnormal{Prägedruck}}}\label{T_L00298-1}\orgindex{Burgtheater@Burgtheater|pw}}}\hfill Wien\oindex{Wien@\textbf{Wien}, \emph{Verwaltungsgebiet}|pw}, 12. 2. 1894\pend
           
\pstart{}Sehr geehrter Herr Doctor!\pend\vspace{0.5em}
\pstart
           In Beantwortung Ihres freundlichen Schreibens bin ich{ }ſo frei mithzuteilen, daſs ich
               meines beſten Wiſſens und Erinnerns die drei Luſtſpiele\pwindex{Schnitzler, Arthur 15.\,5.\,1862 Wien – 21.\,10.\,1931 ebd.@\textsc{Schnitzler, Arthur} (15.\,5.\,1862 Wien – 21.\,10.\,1931 ebd.), \emph{Schriftsteller, Mediziner}!Anatol@\strich\emph{Anatol}|pwv} nicht erhalten habe und{ }ſie auch weder in der
               Theaterbibliothek noch in meiner Privatbibliothek vorfinden kann.\pend
           
\pstart
           Mit herzlichen Empfehlungen hochachtungsvoll{\\[\baselineskip]}\spacefill\mbox{Burckhard}\pend
           \leftskip=0em{}\selectlanguage{ngerman}\endnumbering\briefempfaengerindex{Schnitzler, Arthur@\textsc{Schnitzler, Arthur}!zzzBurckhard, Max Eugen@\emph{von Max Eugen Burckhard}!1894-02-121@{12. 2. 1894}|)be}\mylabel{L00298h}  \newcommand{\dateiname}{L00298}\newcommand{\titel}{Max Burckhard an Arthur Schnitzler, 12. 2. 1894}\newcommand{\editorInnen}{Martin Anton Müller und Gerd-Hermann Susen}%% latex-leseansicht-abspann.tex
%% Abspann für die Leseansicht.
%% Der Schalter \ifkorrekturansicht ist bereits durch den Vorspann gesetzt.

%% latex-abspann.tex
%% Gemeinsamer Abspann für Korrekturansicht und Leseansicht.
%% Setzt den Schalter \ifkorrekturansicht voraus (gesetzt in den
%% einbindenden Dateien latex-korrekturansicht-abspann.tex bzw.
%% latex-leseansicht-abspann.tex).
%% ---------------------------------------------------------------

\normalsize

% Das esempio-Environment wird nur in der Leseansicht benötigt
\ifkorrekturansicht\else
\newenvironment{esempio}[3]%
{
    \vspace{1.5ex}
    \rlap{\underline{#1}}
    \par
    \setlength{\parindent}{0cm}
    \nopagebreak
    \leftskip=#2cm
    \rightskip=#3cm
}
{
    \par
}
\fi

\doendnotes{C}
\bigskip
\vfill

\clearpage

\footnotesize

\ifkorrekturansicht
  \lohead{\textsc{register}}
\fi

% theindex-Environment neu definieren ohne reledmac
\makeatletter
\renewenvironment{theindex}{%
  \ifkorrekturansicht
    \section*{\indexname}%
  \else
    \subsubsection*{Index der erwähnten Entitäten}%
  \fi
  \setlength{\parindent}{0pt}%
  \setlength{\parskip}{0pt plus 0.3pt}%
  \let\item\@idxitem
}{%
  \ifkorrekturansicht\clearpage\fi
}
\makeatother

\IfFileExists{\jobname-pw.ind}{\input{\jobname-pw.ind}}{}

% Quellenangabe nur in der Leseansicht
\ifkorrekturansicht\else
% Fallback-Definitionen, falls die .tex-Datei \titel etc. nicht gesetzt hat
\providecommand{\titel}{}
\providecommand{\editorInnen}{}
\providecommand{\dateiname}{\jobname}

\vspace{3cm}

\vfill

\footnotesize
\textsc{Quelle}: \titel. Herausgegeben von {\editorInnen}. In: \emph{Arthur Schnitzler: Briefwechsel mit Autorinnen und Autoren}.
 Digitale Edition, https://schnitzler-briefe.acdh.oeaw.ac.at/{\dateiname}.html (Stand \today)
\fi

\end{document}


