%% latex-korrekturansicht-vorspann.tex
%% Vorspann für die Korrekturansicht.
%% Lädt die gemeinsame Datei latex-vorspann.tex mit gesetztem Schalter.

\newif\ifkorrekturansicht
\korrekturansichttrue

\input{../tex-inputs/latex-vorspann}


\section[Max Burckhard an Arthur Schnitzler, 12. 2. 1894]{L00298 Max Burckhard an Arthur Schnitzler, 12. 2. 1894}
\nopagebreak\mylabel{L00298v}
\rehead{ }\normalsize\beginnumbering\briefempfaengerindex{Schnitzler, Arthur@\textsc{Schnitzler, Arthur}!zzzBurckhard, Max Eugen@\emph{von Max Eugen Burckhard}!1894-02-121@{12. 2. 1894}|(be}
\toendnotes[C]{\smallbreak\pagebreak[2]}\Standort{CUL, Schnitzler, B 20.}
\physDesc{Brief, 1 Blatt, 1 Seite, 345 Zeichen
\newline{}Handschrift: schwarze Tinte, deutsche Kurrent
\newline{}Ordnung: mit rotem Buntstift von unbekannter Hand nummeriert:
                                    »2«, mutmaßlich von anderer Hand mit Bleistift
                                 durchgestrichen und nummeriert: »4« }\toendnotes[C]{\smallbreak}
\pstart
           {\pb}\textcolor{gray}{\textbf{\label{T_L00298-1v}\edtext{k. k. Hofburgtheater
                              Direction}{\lemma{\textnormal{\emph{k. k. … Direction}}}\Cendnote{\textnormal{Prägedruck}}}\label{T_L00298-1}\orgindex{Burgtheater@Burgtheater|pw}}}\hfill Wien\oindex{Wien@\textbf{Wien}, \emph{A.ADM2}|pw}, 12. 2. 1894\pend
           
\pstart{}Sehr geehrter Herr Doctor!\pend\vspace{0.5em}
\pstart
           In Beantwortung Ihres freundlichen Schreibens bin ich ſo frei mithzuteilen, daſs ich
               meines beſten Wiſſens und Erinnerns die drei Luſtſpiele\pwindex{Anatol@\emph{Anatol}|pwv} nicht erhalten habe und ſie auch weder in der
               Theaterbibliothek noch in meiner Privatbibliothek vorfinden kann.\pend
           
\pstart
           Mit herzlichen Empfehlungen hochachtungsvoll{\\[\baselineskip]}\spacefill\mbox{Burckhard}\pend
           \leftskip=0em{}\selectlanguage{ngerman}\endnumbering\briefempfaengerindex{Schnitzler, Arthur@\textsc{Schnitzler, Arthur}!zzzBurckhard, Max Eugen@\emph{von Max Eugen Burckhard}!1894-02-121@{12. 2. 1894}|)be}\mylabel{L00298h}  \normalsize

\doendnotes{C}
\bigskip
\vfill

\clearpage

\footnotesize

\lohead{\textsc{register}}

% Definiere theindex-Environment komplett neu ohne reledmac
\makeatletter
\renewenvironment{theindex}{%
  \section*{\indexname}%
  \setlength{\parindent}{0pt}%
  \setlength{\parskip}{0pt plus 0.3pt}%
  \let\item\@idxitem
}{%
  \clearpage
}
\makeatother

\IfFileExists{\jobname-pw.ind}{\input{\jobname-pw.ind}}{}

\end{document}

      