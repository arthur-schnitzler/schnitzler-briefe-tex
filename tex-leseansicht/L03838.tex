%% latex-leseansicht-vorspann.tex
%% Vorspann für die Leseansicht.
%% Lädt die gemeinsame Datei latex-vorspann.tex mit nicht gesetztem Schalter.

\newif\ifkorrekturansicht
\korrekturansichtfalse

\input{../tex-inputs/latex-vorspann}


\section[Theodor Herzl an Arthur Schnitzler, 11. 12. 1894]{L03838 Theodor Herzl an Arthur Schnitzler, 11. 12. 1894}
\nopagebreak\mylabel{L03838v}
\rehead{ }\normalsize\beginnumbering\briefempfaengerindex{Schnitzler, Arthur@\textsc{Schnitzler, Arthur}!zzzHerzl, Theodor@\emph{von Theodor Herzl}!1894-12-111@{11. 12. 1894}|(be}
\toendnotes[C]{\smallbreak\pagebreak[2]}
\correspDesc{Versand  durch Theodor Herzl am 11. 12. 1894 in Paris
\newline{}Erhalt  durch Arthur Schnitzler im Zeitraum [12. 12. 1894 – 16. 12. 1894?] in Wien}\toendnotes[C]{\smallbreak}
\Standort{CUL, Schnitzler, B 39.}
\physDesc{Brief, 1 Blatt, 4 Seiten, 3388 Zeichen
\newline{}Handschrift: schwarze Tinte, lateinische Kurrent
\newline{}Ordnung: mit Bleistift von unbekannter Hand nummeriert: »17« }
\buchAbdrucke{\weitereDrucke{Theodor Herzl: \emph{Briefe und
                        autobiographische Notizen 1866–1895}. Bearbeitet von Johannes Wachten in Zusammenarbeit mit Chaya Harel, Daisy Tycho und Manfred Winkler. Berlin, Frankfurt am Main, Wien: \emph{Propyläen} 1983, S. 560–561 (Briefe und Tagebücher. Herausgegeben von Alex Bein, Hermann Greive, Moshe Schaerf, Julius H. Schoeps und Johannes Wachten, 1).} }\toendnotes[C]{\smallbreak}
\pstart
           \raggedleft{}{\pb}Palais Bourbon\oindex{Palais Bourbon@\textbf{Palais Bourbon}, \emph{Regierungsgebäude}|pw}\pend
           
\pstart
           \raggedleft{}11. XII. 894\pend
           
\pstart{}Mein lieber Freund!\pend\vspace{0.5em}
\pstart
           Die »Glosse\pwindex{Herzl, Theodor 2.\,5.\,1860 Budapest – 3.\,7.\,1904 Edlach@\textsc{Herzl, Theodor} (2.\,5.\,1860 Budapest – 3.\,7.\,1904 Edlach), \emph{Schriftsteller, Journalist}!Glosse. Lustspiel in einem Act@\strich\emph{Die Glosse. Lustspiel in einem Act}|pw}« habe ich Ihnen schon vor
      mehreren Tagen geschickt.\pend
           
\pstart
           Ich bin nur schwer wieder in meine Arbeit\pwindex{Herzl, Theodor 2.\,5.\,1860 Budapest – 3.\,7.\,1904 Edlach@\textsc{Herzl, Theodor} (2.\,5.\,1860 Budapest – 3.\,7.\,1904 Edlach), \emph{Schriftsteller, Journalist}!neue Ghetto. Schauspiel in vier Acten@\strich\emph{Das neue Ghetto. Schauspiel in vier Acten}|pwv}
      hineingekommen. Sie hat mir wochenlang
      einen unsagbaren Ekel eingeflösst – aber
      ich will Sie nicht mit allen neurasthenischen
      Erscheinungen meiner Production langweilen.
      Kurz, ich kann jetzt wieder die Sache
      in die Hand nehmen und feilen.
      Die Abschrift\pwindex{Herzl, Theodor 2.\,5.\,1860 Budapest – 3.\,7.\,1904 Edlach@\textsc{Herzl, Theodor} (2.\,5.\,1860 Budapest – 3.\,7.\,1904 Edlach), \emph{Schriftsteller, Journalist}!neue Ghetto. Schauspiel in vier Acten@\strich\emph{Das neue Ghetto. Schauspiel in vier Acten}|pwv} wird bald fertig werden.
      Wenn ich aber fertig bin, soll keine
      Sekunde versäumt werden. Denn dann
      wird mir wieder das Warten auf die
      Entscheidung fürchterlich auf die Nerven
      gehen. Ich konnte schon Ihr erstes Urtheil
      nicht erwarten, u. da \label{K_L03838-1v}\edtext{Ihr Brief}{\lemma{\textnormal{\emph{Ihr Brief}}}\Cendnote{\textnormal{XXXX17.11.1894}}}\label{K_L03838-1} einige
      Tage auf sich warten liess, war ich
      ganz zapplig. Ich werde mich nun
      offenbar in Geduld fassen müssen, bis
      ich eine Entscheidung \label{K_L03838-2v}\edtext{von den Directoren}{\lemma{\textnormal{\emph{von den Directoren}}}\Cendnote{\textnormal{Herzl\pwindex{Herzl, Theodor 2.\,5.\,1860 Budapest – 3.\,7.\,1904 Edlach@\textsc{Herzl, Theodor} (2.\,5.\,1860 Budapest – 3.\,7.\,1904 Edlach), \emph{Schriftsteller, Journalist}|pwk} wollte sein Stück\pwindex{Herzl, Theodor 2.\,5.\,1860 Budapest – 3.\,7.\,1904 Edlach@\textsc{Herzl, Theodor} (2.\,5.\,1860 Budapest – 3.\,7.\,1904 Edlach), \emph{Schriftsteller, Journalist}!neue Ghetto. Schauspiel in vier Acten@\strich\emph{Das neue Ghetto. Schauspiel in vier Acten}|pwkv} in Berlin\oindex{Berlin@\textbf{Berlin}, \emph{Hauptstadt}|pwk} zunächst beim \emph{Deutschen
            Theater}\orgindex{Deutsches Theater Berlin@Deutsches Theater Berlin|pwk} einreichen lassen, danach beim \emph{Lessing-Theater}\orgindex{Lessing-Theater@Lessing-Theater|pwk}, dieses wiederum sollte es an das \emph{Neue Theater}\orgindex{Neues Theater@Neues Theater|pwk}
         und weitergeben und schließliche an die \emph{Freie Bühne}\orgindex{Freie Bühne@Freie Bühne|pwk}. Die Weiterleitung bei Nichtannahme sollte von den jeweiligen Theaterdirektionen selbst erfolgen.}}}\label{K_L03838-2}
      sehe, aber die Expedition wenigstens soll {\pb}nicht um eine Stunde über das Nöthige
      hinausgezögert werden.\pend
           
\pstart
           Das
               Manuscript\pwindex{Herzl, Theodor 2.\,5.\,1860 Budapest – 3.\,7.\,1904 Edlach@\textsc{Herzl, Theodor} (2.\,5.\,1860 Budapest – 3.\,7.\,1904 Edlach), \emph{Schriftsteller, Journalist}!neue Ghetto. Schauspiel in vier Acten@\strich\emph{Das neue Ghetto. Schauspiel in vier Acten}|pwv} von meiner Hand kann
      ich natürlich nicht einreichen. Ich dachte
      zuerst daran, es hier auf einer Schreibmaschine abzuspielen. Das wäre das Beste,
      auch das leserlichste. Aber ich könnte das
      bei mir zu Hause nicht, ohne Aufsehen
      zu machen. Also will ich mein definitives
         Manuscript\pwindex{Herzl, Theodor 2.\,5.\,1860 Budapest – 3.\,7.\,1904 Edlach@\textsc{Herzl, Theodor} (2.\,5.\,1860 Budapest – 3.\,7.\,1904 Edlach), \emph{Schriftsteller, Journalist}!neue Ghetto. Schauspiel in vier Acten@\strich\emph{Das neue Ghetto. Schauspiel in vier Acten}|pwv} in Wien\oindex{Wien@\textbf{Wien}, \emph{Verwaltungsgebiet}|pw} abschreiben lassen.\pend
           
\pstart
           Vielleicht könnten Sie sich eine Schreibmaschine ausleihen. Hier gibt die Niederlage der Yost-Maschinen auf 14 Tage
      bis 3 Wochen Maschinen zur Probe ins
      Haus. Man kann sie dann zurückgeben
      u. ich hätte in diesem Fall Trinkgelder
      gegeben oder versucht ein Leihgeld zu
      zahlen. Vielleicht können Sie das in Wien\oindex{Wien@\textbf{Wien}, \emph{Verwaltungsgebiet}|pw}
      auch? Dann wurde ich Sie bitten, von
         Schick\pwindex{Schik, Friedrich *~6.\,9.\,1857 Wien@\textsc{Schik, Friedrich} (*~6.\,9.\,1857 Wien), \emph{Notar, Journalist, Dramaturg}|pw} einen intelligenten Abschreiber zu
      verlangen, der das Spielen auf der Maschine
      in einer Stunde heraushätte u. – erst
      langsam dann immer schneller – das Manuscript\pwindex{Herzl, Theodor 2.\,5.\,1860 Budapest – 3.\,7.\,1904 Edlach@\textsc{Herzl, Theodor} (2.\,5.\,1860 Budapest – 3.\,7.\,1904 Edlach), \emph{Schriftsteller, Journalist}!neue Ghetto. Schauspiel in vier Acten@\strich\emph{Das neue Ghetto. Schauspiel in vier Acten}|pwv} abklopfen würde.\pend
           
\pstart
           Geht’s mit der Maschine nicht, so müssten
      wir uns von Schick\pwindex{Schik, Friedrich *~6.\,9.\,1857 Wien@\textsc{Schik, Friedrich} (*~6.\,9.\,1857 Wien), \emph{Notar, Journalist, Dramaturg}|pw} oder vielleicht lieber von {\pb}der Universität\orgindex{Universität Wien@Universität Wien|pwv} einen Abschreiber holen. Die
      Wahl des Abschreibers ist wichtig, weil er ja
      später was verrathen könnte. Darum
      möchte ich auch nicht, dass das Mscpt\pwindex{Herzl, Theodor 2.\,5.\,1860 Budapest – 3.\,7.\,1904 Edlach@\textsc{Herzl, Theodor} (2.\,5.\,1860 Budapest – 3.\,7.\,1904 Edlach), \emph{Schriftsteller, Journalist}!neue Ghetto. Schauspiel in vier Acten@\strich\emph{Das neue Ghetto. Schauspiel in vier Acten}|pwv}
      einem Theater-Abschreibe-Bureau übergeben
      werde. Der Schreiber muss bei Ihnen
      sitzen – ich mache Ihnen viel Mühe! –
      denn ich wollte nicht, dass das Mscpt\pwindex{Herzl, Theodor 2.\,5.\,1860 Budapest – 3.\,7.\,1904 Edlach@\textsc{Herzl, Theodor} (2.\,5.\,1860 Budapest – 3.\,7.\,1904 Edlach), \emph{Schriftsteller, Journalist}!neue Ghetto. Schauspiel in vier Acten@\strich\emph{Das neue Ghetto. Schauspiel in vier Acten}|pwv}
      aus Ihren Händen komme. Natürlich
      muss er eine wunderschöne, sehr klare
      sehr leserliche Schrift haben. Bezahlen
      Sie ihn reichlich, lieber Freund und
      theilen Sie mir gütigst sofort alle Ihre
      Auslagen mit, denn Kosten darf Sie die
      Sache mindestens nichts.\pend
           
\pstart
           Damit die Abschrift\pwindex{Herzl, Theodor 2.\,5.\,1860 Budapest – 3.\,7.\,1904 Edlach@\textsc{Herzl, Theodor} (2.\,5.\,1860 Budapest – 3.\,7.\,1904 Edlach), \emph{Schriftsteller, Journalist}!neue Ghetto. Schauspiel in vier Acten@\strich\emph{Das neue Ghetto. Schauspiel in vier Acten}|pwv} recht schnell fertig
         werde, will ich das Mscpt\pwindex{Herzl, Theodor 2.\,5.\,1860 Budapest – 3.\,7.\,1904 Edlach@\textsc{Herzl, Theodor} (2.\,5.\,1860 Budapest – 3.\,7.\,1904 Edlach), \emph{Schriftsteller, Journalist}!neue Ghetto. Schauspiel in vier Acten@\strich\emph{Das neue Ghetto. Schauspiel in vier Acten}|pwv} in mehreren
      Bruchstücken an Sie absenden. Der Schreiber
      soll sofort nach Eintreffen des ersten
      anfangen. Wenn alles fertig, bitte ich
      sie die Reinschrift von einem Buchbinder
      brochiren zu lassen, dem aber nur ein
      paar Stunden Zeit gelassen werden dürfen,
      damit er das Buch\pwindex{Herzl, Theodor 2.\,5.\,1860 Budapest – 3.\,7.\,1904 Edlach@\textsc{Herzl, Theodor} (2.\,5.\,1860 Budapest – 3.\,7.\,1904 Edlach), \emph{Schriftsteller, Journalist}!neue Ghetto. Schauspiel in vier Acten@\strich\emph{Das neue Ghetto. Schauspiel in vier Acten}|pwv} nicht lese. Natürlich
      ein Buchbinder, der Sie nicht kennt.\pend
           
\pstart
           Das sind vorläufig alle meine Wünsche.
      Mache ich Ihnen zu viel Mühe?\pend
           
\pstart
           Ich bitte Sie auch um den vollen Namen
         Schicks\pwindex{Schik, Friedrich *~6.\,9.\,1857 Wien@\textsc{Schik, Friedrich} (*~6.\,9.\,1857 Wien), \emph{Notar, Journalist, Dramaturg}|pw}, den ich für den Begleitbrief brauche. {\pb}Bleibt die Adresse \label{K_L03838-3v}\edtext{Reisnerstrasse\oindex{Wien@\textbf{Wien}!III., Landstraße@\textbf{III., Landstraße}!Reisnerstraße@\textbf{Reisnerstraße}, \emph{Straße}|pw}}{\lemma{\textnormal{\emph{Reisnerstrasse}}}\Cendnote{\textnormal{Friedrich Schik\pwindex{Schik, Friedrich *~6.\,9.\,1857 Wien@\textsc{Schik, Friedrich} (*~6.\,9.\,1857 Wien), \emph{Notar, Journalist, Dramaturg}|pwk} wohnte in der Reisnerstraße 35\oindex{Wien@\textbf{Wien}!III., Landstraße@\textbf{III., Landstraße}!Reisnerstraße 35@\textbf{Reisnerstraße 35}, \emph{Wohngebäude}|pwk}.}}}\label{K_L03838-3}?\pend
           
\pstart
           – Ich bin neugierig, was Sie zur
         Revision meines Stückes\pwindex{Herzl, Theodor 2.\,5.\,1860 Budapest – 3.\,7.\,1904 Edlach@\textsc{Herzl, Theodor} (2.\,5.\,1860 Budapest – 3.\,7.\,1904 Edlach), \emph{Schriftsteller, Journalist}!neue Ghetto. Schauspiel in vier Acten@\strich\emph{Das neue Ghetto. Schauspiel in vier Acten}|pwv} sagen. Ich
      will Sie bitten, diese Revision erst in
      der Reinschrift des Abschreibers zu lesen,
      damit Sie seine Fehler entdecken u. corrigiren können. Lesen Sie sie vorher,
      so sind Sie dabei nicht mehr aufmerksam.\pend
           
\pstart
           Also für heute Schluss.\pend
           
\pstart
           Mit herzlichen Grüssen Ihr Freund{\\[\baselineskip]}\spacefill\mbox{Herzl}\pend
           \leftskip=0em{}
\pstart
           \noindent{}Antwort\strikeout{e}, auch diesmal, wie immer wenn nichts Besonderes vorliegt, an meine Adresse rue Monceau\oindex{rue Monceau@\textbf{rue Monceau}, \emph{Straße}|pw}\pend
           \selectlanguage{ngerman}\endnumbering\briefempfaengerindex{Schnitzler, Arthur@\textsc{Schnitzler, Arthur}!zzzHerzl, Theodor@\emph{von Theodor Herzl}!1894-12-111@{11. 12. 1894}|)be}\mylabel{L03838h}
\begin{anhang}
\end{anhang}\newcommand{\dateiname}{L03838}\newcommand{\titel}{Theodor Herzl an Arthur Schnitzler, 11. 12. 1894}\newcommand{\editorInnen}{Selma Jahnke und Martin Anton Müller}%% latex-leseansicht-abspann.tex
%% Abspann für die Leseansicht.
%% Der Schalter \ifkorrekturansicht ist bereits durch den Vorspann gesetzt.

%% latex-abspann.tex
%% Gemeinsamer Abspann für Korrekturansicht und Leseansicht.
%% Setzt den Schalter \ifkorrekturansicht voraus (gesetzt in den
%% einbindenden Dateien latex-korrekturansicht-abspann.tex bzw.
%% latex-leseansicht-abspann.tex).
%% ---------------------------------------------------------------

\normalsize

% Das esempio-Environment wird nur in der Leseansicht benötigt
\ifkorrekturansicht\else
\newenvironment{esempio}[3]%
{
    \vspace{1.5ex}
    \rlap{\underline{#1}}
    \par
    \setlength{\parindent}{0cm}
    \nopagebreak
    \leftskip=#2cm
    \rightskip=#3cm
}
{
    \par
}
\fi

\doendnotes{C}
\bigskip
\vfill

\clearpage

\footnotesize

\ifkorrekturansicht
  \lohead{\textsc{register}}
\fi

% theindex-Environment neu definieren ohne reledmac
\makeatletter
\renewenvironment{theindex}{%
  \ifkorrekturansicht
    \section*{\indexname}%
  \else
    \subsubsection*{Index der erwähnten Entitäten}%
  \fi
  \setlength{\parindent}{0pt}%
  \setlength{\parskip}{0pt plus 0.3pt}%
  \let\item\@idxitem
}{%
  \ifkorrekturansicht\clearpage\fi
}
\makeatother

\IfFileExists{\jobname-pw.ind}{\input{\jobname-pw.ind}}{}

% Quellenangabe nur in der Leseansicht
\ifkorrekturansicht\else
% Fallback-Definitionen, falls die .tex-Datei \titel etc. nicht gesetzt hat
\providecommand{\titel}{}
\providecommand{\editorInnen}{}
\providecommand{\dateiname}{\jobname}

\vspace{3cm}

\vfill

\footnotesize
\textsc{Quelle}: \titel. Herausgegeben von {\editorInnen}. In: \emph{Arthur Schnitzler: Briefwechsel mit Autorinnen und Autoren}.
 Digitale Edition, https://schnitzler-briefe.acdh.oeaw.ac.at/{\dateiname}.html (Stand \today)
\fi

\end{document}


