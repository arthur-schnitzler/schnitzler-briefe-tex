%% latex-leseansicht-vorspann.tex
%% Vorspann für die Leseansicht.
%% Lädt die gemeinsame Datei latex-vorspann.tex mit nicht gesetztem Schalter.

\newif\ifkorrekturansicht
\korrekturansichtfalse

\input{../tex-inputs/latex-vorspann}


         
         \renewcommand{\erwaehntePersonen}{Personen: Léon Bourgeois, Baptistin Lucien de Combles de Nayve, Blanche Marie Combles de Nayve, Paul Goldmann, Hippolyte Menaldo, Alexandre Ribot, Leopold Sonnemann}
         \renewcommand{\erwaehnteInstitutionen}{Institutionen: Frankfurter Zeitung, Französische Abgeordnetenkammer, Französische Regierung}
         \renewcommand{\erwaehnteOrte}{Orte: Carmaux, Französische Nationalversammlung, Paris, Wien, rue Feydeau}
         \renewcommand{\erwaehnteWerke}{Werke: Liebelei. Schauspiel in drei Akten}
               \section[Paul Goldmann an Arthur Schnitzler, 6. 11. {[}1895{]}]{ Paul Goldmann an Arthur Schnitzler, 6. 11. {[}1895{]}}\nopagebreak\mylabel{v}\rehead{ }\begin{ledgroupsized}[t]{13cm}\normalsize\beginnumbering\briefempfaengerindex{Schnitzler, Arthur@\textsc{Schnitzler, Arthur}!zzzGoldmann, Paul@\emph{von Paul Goldmann}!1895-11-061@{6. 11. {[}1895{]}}|(be} \toendnotes[C]{\smallbreak\pagebreak[2]} \Standort{DLA, A:Schnitzler, HS.NZ85.1.3165.}
\physDesc{Brief, 1 Blatt, 3 Seiten, 987 Zeichen
\newline{}Handschrift: blaue Tinte, deutsche Kurrent
\newline{}Schnitzler: mit Bleistift das Jahr »95« vermerkt }\toendnotes[C]{\smallbreak}\pstart
           \noindent{}{\pb}\textcolor{gray}{\textbf{\textbf{Frankfurter Zeitung\orgindex{Frankfurter Zeitung@Frankfurter Zeitung|pw}}}}\pend
           \pstart
           \textcolor{gray}{\textbf{(\begin{otherlanguage}{french}Gazette de Francfort\end{otherlanguage}\orgindex{Frankfurter Zeitung@Frankfurter Zeitung|pw}). }}\pend
           \pstart
           \textcolor{gray}{\textbf{\textbf{\begin{otherlanguage}{french}Fondateur M. L.
                              Sonnemann\pwindex{Sonnemann, Leopold 1831-10-29 – 1909-10-30@\textsc{Sonnemann, Leopold} (1831-10-29 – 1909-10-30), \emph{Journalist, Herausgeber}|pw}\end{otherlanguage}.}}}\pend
           \pstart
           \begin{otherlanguage}{french}\textcolor{gray}{\textbf{Journal politique, financier,}}\end{otherlanguage}\hfill \textsc{Paris\oindex{Paris@\textbf{Paris}|pw}}, 6. November.\pend
           \pstart
           \begin{otherlanguage}{french}\textcolor{gray}{\textbf{commercial et littéraire.}}\end{otherlanguage}\pend
           \pstart
           \begin{otherlanguage}{french}\textcolor{gray}{\textbf{\textbf{Paraissant trois fois par jour.}}}\end{otherlanguage}\pend
           \pstart
           \begin{otherlanguage}{french}\textcolor{gray}{\textbf{\textbf{Bureau à Paris\oindex{Paris@\textbf{Paris}|pw}:}}}\end{otherlanguage}\pend
           \pstart
           \begin{otherlanguage}{french}\textcolor{gray}{\textbf{\textbf{24. Rue Feydeau\oindex{rue Feydeau@\textbf{rue Feydeau}|pw}.}}}\end{otherlanguage}\pend
           \pstart\center{}Mein lieber Freund,\pend\pstart
           Seit 14 Tagen warte ich auf jeden neuen Tag, in der Hoffnung, er werde mir eine \strikeout{\textcolor{gray}{ne}} freie Stunde bringen, um Dir antworten zu können, aber die freie Stunde will
               nicht kommen. Endloſe Kammer\orgindex{Franzoesische Abgeordnetenkammer@Französische Abgeordnetenkammer|pw}-Debatten, \label{K_L02754-1v}\edtext{Miniſter\pwindex{Ribot, Alexandre 1842-02-07 – 1923-01-13@\textsc{Ribot, Alexandre} (1842-02-07 – 1923-01-13), \emph{Politiker}|pwv}ſturz, Kriſis, neues
                  Cabinet\orgindex{Franzoesische Regierung@Französische Regierung|pwv}}{\lemma{\textnormal{\emph{Miniſterſturz, … Cabinet}}}\Cendnote{\textnormal{Die Regierung\orgindex{Franzoesische Regierung@Französische Regierung|pwkv}{ }Alexandre Ribots\pwindex{Ribot, Alexandre 1842-02-07 – 1923-01-13@\textsc{Ribot, Alexandre} (1842-02-07 – 1923-01-13), \emph{Politiker}|pwk} war am
                     28. 10. 1895 gestürzt worden. Premierminister\pwindex{Bourgeois, Leon 1851-05-29 – 1925-09-29@\textsc{Bourgeois, Léon} (1851-05-29 – 1925-09-29), \emph{Politiker, Minister, Nobelpreisträger}|pwkv}{ }Léon Bourgeois\pwindex{Bourgeois, Leon 1851-05-29 – 1925-09-29@\textsc{Bourgeois, Léon} (1851-05-29 – 1925-09-29), \emph{Politiker, Minister, Nobelpreisträger}|pwk} bildete ein neues Kabinett\orgindex{Franzoesische Regierung@Französische Regierung|pwkv}.}}}\label{K_L02754-1h}, \label{K_L02754-2v}\edtext{St\textcolor{gray}{re}ike von \textsc{Carmaux\oindex{Carmaux@\textbf{Carmaux}|pw}}}{\lemma{\textnormal{\emph{Streike von Carmaux}}}\Cendnote{\textnormal{In Carmaux\oindex{Carmaux@\textbf{Carmaux}|pwk} streikten Glasarbeiterinnen und Glasarbeiter gegen soziale
                  Missstände.}}}\label{K_L02754-2h}, \label{K_L02754-3v}\edtext{Prozeß \textsc{de Nayve\pwindex{Combles de Nayve, Baptistin Lucien de *~1849-09-06@\textsc{Combles de Nayve, Baptistin Lucien de} (*~1849-09-06), \emph{Beamter}|pw}}}{\lemma{\textnormal{\emph{Prozeß de Nayve}}}\Cendnote{\textnormal{Baptistin de Combles de Nayves\pwindex{Combles de Nayve, Baptistin Lucien de *~1849-09-06@\textsc{Combles de Nayve, Baptistin Lucien de} (*~1849-09-06), \emph{Beamter}|pwk} wurde der
                  Prozess gemacht, weil ihm seine Gattin\pwindex{Combles de Nayve, Blanche Marie *~1854-09-02@\textsc{Combles de Nayve, Blanche Marie} (*~1854-09-02)|pwkv} vorgeworfen hatte, er hätte absichtlich ihr
                  leibliches Kind\pwindex{Menaldo, Hippolyte 1871-07-11 – 1885-11-10@\textsc{Menaldo, Hippolyte} (1871-07-11 – 1885-11-10)|pwkv} aus einer
                  früheren Beziehung einen Felsen hinunter in den Tod gestoßen. Letztlich wurde er
                  im Zweifel freigesprochen.}}}\label{K_L02754-3h}, dazwiſchen Theater und ſonſt allerhand – es
               bleibt gerade Zeit zum Eſſen und zum Schlafen, und auch dieſe nicht immer. Ich \strikeout{hätte}{ }{\pb}hätte Dir ſoviel zu ſagen, möchte Dir für Deine
               letzten ſo lieben Briefe danken, – aber dieſe Arbeits-Woge iſt ſtärker, als mein
               guter Wille, und ich kann nichts machen, als warten, bis ſie vorüber iſt. Dieſer Tage
               hoffe ich endlich Dir ausführlicher ſchreiben zu können. Einſtweilen ſollen dieſe
               wenigen Zeilen mich nur bei Dir entſchuldigen. Wenn ich nach der Kammer\oindex{Franzoesische Nationalversammlung@\textbf{Französische Nationalversammlung}|pwv} gehe, kaufe ich mir hier und da ein
                  Wien\oindex{Wien@\textbf{Wien}|pw}er Blatt auf dem \begin{otherlanguage}{french}\textsc{Boulevard}\end{otherlanguage} und ſehe mit Freude, daß die »Liebelei\pwindex{Schnitzler, Arthur 15.05.1862 – 21.10.1931@\textsc{Schnitzler, Arthur} (15.05.1862 – 21.10.1931), \emph{Schriftsteller, Mediziner}!Liebelei. Schauspiel in drei Akten1895-10-09@\strich\emph{Liebelei. Schauspiel in drei Akten} {[}1895-10-09{]}|pw}« \strikeout{ſe\textcolor{gray}{i}} ihren {\pb}Platz im Repertoire behält. \strikeout{\textcolor{gray}{×}\-\textcolor{gray}{×}\-\textcolor{gray}{×}\-\textcolor{gray}{×}\-\textcolor{gray}{×}} Das iſt ſchön.\pend
           \pstart
           Viele treue Grüße! {\\[\baselineskip]}Dein {\\[\baselineskip]}\spacefill\mbox{Paul Goldmann.}\pend
           \leftskip=0em{}
         
         \endnumbering\mylabel{h}\end{ledgroupsized}  \newcommand{\dateiname}{L02754}\newcommand{\titel}{Paul Goldmann an Arthur Schnitzler, 6. 11. [1895]}\newcommand{\editorInnen}{Martin Anton Müller und Laura Untner}%% latex-leseansicht-abspann.tex
%% Abspann für die Leseansicht.
%% Der Schalter \ifkorrekturansicht ist bereits durch den Vorspann gesetzt.

%% latex-abspann.tex
%% Gemeinsamer Abspann für Korrekturansicht und Leseansicht.
%% Setzt den Schalter \ifkorrekturansicht voraus (gesetzt in den
%% einbindenden Dateien latex-korrekturansicht-abspann.tex bzw.
%% latex-leseansicht-abspann.tex).
%% ---------------------------------------------------------------

\normalsize

% Das esempio-Environment wird nur in der Leseansicht benötigt
\ifkorrekturansicht\else
\newenvironment{esempio}[3]%
{
    \vspace{1.5ex}
    \rlap{\underline{#1}}
    \par
    \setlength{\parindent}{0cm}
    \nopagebreak
    \leftskip=#2cm
    \rightskip=#3cm
}
{
    \par
}
\fi

\doendnotes{C}
\bigskip
\vfill

\clearpage

\footnotesize

\ifkorrekturansicht
  \lohead{\textsc{register}}
\fi

% theindex-Environment neu definieren ohne reledmac
\makeatletter
\renewenvironment{theindex}{%
  \ifkorrekturansicht
    \section*{\indexname}%
  \else
    \subsubsection*{Index der erwähnten Entitäten}%
  \fi
  \setlength{\parindent}{0pt}%
  \setlength{\parskip}{0pt plus 0.3pt}%
  \let\item\@idxitem
}{%
  \ifkorrekturansicht\clearpage\fi
}
\makeatother

\IfFileExists{\jobname-pw.ind}{\input{\jobname-pw.ind}}{}

% Quellenangabe nur in der Leseansicht
\ifkorrekturansicht\else
% Fallback-Definitionen, falls die .tex-Datei \titel etc. nicht gesetzt hat
\providecommand{\titel}{}
\providecommand{\editorInnen}{}
\providecommand{\dateiname}{\jobname}

\vspace{3cm}

\vfill

\footnotesize
\textsc{Quelle}: \titel. Herausgegeben von {\editorInnen}. In: \emph{Arthur Schnitzler: Briefwechsel mit Autorinnen und Autoren}.
 Digitale Edition, https://schnitzler-briefe.acdh.oeaw.ac.at/{\dateiname}.html (Stand \today)
\fi

\end{document}


      