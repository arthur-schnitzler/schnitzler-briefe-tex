%% latex-leseansicht-vorspann.tex
%% Vorspann für die Leseansicht.
%% Lädt die gemeinsame Datei latex-vorspann.tex mit nicht gesetztem Schalter.

\newif\ifkorrekturansicht
\korrekturansichtfalse

\input{../tex-inputs/latex-vorspann}


\section[Paul Goldmann an Arthur Schnitzler, 6. 11. {[}1895{]}]{L02754 Paul Goldmann an Arthur Schnitzler, 6. 11. [1895]}
\nopagebreak\mylabel{L02754v}
\rehead{ }\normalsize\beginnumbering\briefempfaengerindex{Schnitzler, Arthur@\textsc{Schnitzler, Arthur}!zzzGoldmann, Paul@\emph{von Paul Goldmann}!1895-11-061@{6. 11. [1895]}|(be}
\toendnotes[C]{\smallbreak\pagebreak[2]}
\correspDesc{Versand  durch Paul Goldmann am 6. 11. [1895] in Paris
\newline{}Erhalt  durch Arthur Schnitzler im Zeitraum [7. 11. 1895
                  – 11. 11. 1895?] in Wien}\toendnotes[C]{\smallbreak}
\Standort{DLA, A:Schnitzler, HS.NZ85.1.3165.}
\physDesc{Brief, 1 Blatt, 3 Seiten, 987 Zeichen
\newline{}Handschrift: blaue Tinte, deutsche Kurrent
\newline{}Schnitzler: mit Bleistift das Jahr »95« vermerkt }\toendnotes[C]{\smallbreak}
\pstart
           {\pb}\textcolor{gray}{\textbf{\textbf{Frankfurter Zeitung\orgindex{Frankfurter Zeitung@Frankfurter Zeitung|pw}}}}\pend
           
\pstart
           \textcolor{gray}{\textbf{(\begin{otherlanguage}{french}Gazette de Francfort\end{otherlanguage}\orgindex{Frankfurter Zeitung@Frankfurter Zeitung|pw}).}}\pend
           
\pstart
           \textcolor{gray}{\textbf{\textbf{\begin{otherlanguage}{french}Fondateur M. L.
                              Sonnemann\pwindex{Sonnemann, Leopold 29.\,10.\,1831 Höchberg – 30.\,10.\,1909 Frankfurt am Main@\textsc{Sonnemann, Leopold} (29.\,10.\,1831 Höchberg – 30.\,10.\,1909 Frankfurt am Main), \emph{Journalist, Herausgeber}|pw}\end{otherlanguage}.}}}\pend
           
\pstart
           \begin{otherlanguage}{french}\textcolor{gray}{\textbf{Journal politique, financier,}}\end{otherlanguage}\hfill \textsc{Paris\oindex{Paris@\textbf{Paris}, \emph{Hauptstadt}|pw}}, 6. November.\pend
           
\pstart
           \begin{otherlanguage}{french}\textcolor{gray}{\textbf{commercial et littéraire.}}\end{otherlanguage}\pend
           
\pstart
           \begin{otherlanguage}{french}\textcolor{gray}{\textbf{\textbf{Paraissant trois fois par jour.}}}\end{otherlanguage}\pend
           
\pstart
           \begin{otherlanguage}{french}\textcolor{gray}{\textbf{\textbf{Bureau à Paris\oindex{Paris@\textbf{Paris}, \emph{Hauptstadt}|pw}:}}}\end{otherlanguage}\pend
           
\pstart
           \begin{otherlanguage}{french}\textcolor{gray}{\textbf{\textbf{24. Rue Feydeau\oindex{rue Feydeau@\textbf{rue Feydeau}, \emph{Straße}|pw}.}}}\end{otherlanguage}\pend
           
\pstart\center{}Mein lieber Freund,\pend\vspace{0.5em}
\pstart
           Seit 14 Tagen warte ich auf jeden neuen Tag, in der Hoffnung, er werde mir eine \strikeout{\textcolor{gray}{ne}} freie Stunde bringen, um Dir antworten zu können, aber die freie Stunde will
               nicht kommen. Endloſe Kammer\orgindex{Französische Abgeordnetenkammer@Französische Abgeordnetenkammer|pw}-Debatten, \label{K_L02754-1v}\edtext{Miniſter\pwindex{Ribot, Alexandre 7.\,2.\,1842 Saint-Omer – 13.\,1.\,1923 Paris@\textsc{Ribot, Alexandre} (7.\,2.\,1842 Saint-Omer – 13.\,1.\,1923 Paris), \emph{Politiker}|pwv}ſturz, Kriſis, neues
                  Cabinet\orgindex{Französische Regierung@Französische Regierung|pwv}}{\lemma{\textnormal{\emph{Ministersturz, … Cabinet}}}\Cendnote{\textnormal{Die Regierung\orgindex{Französische Regierung@Französische Regierung|pwkv}{ }Alexandre Ribots\pwindex{Ribot, Alexandre 7.\,2.\,1842 Saint-Omer – 13.\,1.\,1923 Paris@\textsc{Ribot, Alexandre} (7.\,2.\,1842 Saint-Omer – 13.\,1.\,1923 Paris), \emph{Politiker}|pwk} war am
                     28. 10. 1895 gestürzt worden. Premierminister\pwindex{Bourgeois, Léon 29.\,5.\,1851 Paris – 29.\,9.\,1925 Épernay@\textsc{Bourgeois, Léon} (29.\,5.\,1851 Paris – 29.\,9.\,1925 Épernay), \emph{Politiker, Minister, Nobelpreisträger}|pwkv}{ }Léon Bourgeois\pwindex{Bourgeois, Léon 29.\,5.\,1851 Paris – 29.\,9.\,1925 Épernay@\textsc{Bourgeois, Léon} (29.\,5.\,1851 Paris – 29.\,9.\,1925 Épernay), \emph{Politiker, Minister, Nobelpreisträger}|pwk} bildete ein neues Kabinett\orgindex{Französische Regierung@Französische Regierung|pwkv}.}}}\label{K_L02754-1}, \label{K_L02754-2v}\edtext{St\textcolor{gray}{re}ike von \textsc{Carmaux\oindex{Carmaux@\textbf{Carmaux}|pw}}}{\lemma{\textnormal{\emph{Streike von Carmaux}}}\Cendnote{\textnormal{In Carmaux\oindex{Carmaux@\textbf{Carmaux}|pwk} streikten Glasarbeiterinnen und Glasarbeiter gegen soziale
                  Missstände.}}}\label{K_L02754-2}, \label{K_L02754-3v}\edtext{Prozeß \textsc{de Nayve\pwindex{Combles de Nayve, Baptistin Lucien de *~6.\,9.\,1849 Grenoble@\textsc{Combles de Nayve, Baptistin Lucien de} (*~6.\,9.\,1849 Grenoble), \emph{Steuerbeamter}|pw}}}{\lemma{\textnormal{\emph{Prozeß de Nayve}}}\Cendnote{\textnormal{Baptistin de Combles de Nayves\pwindex{Combles de Nayve, Baptistin Lucien de *~6.\,9.\,1849 Grenoble@\textsc{Combles de Nayve, Baptistin Lucien de} (*~6.\,9.\,1849 Grenoble), \emph{Steuerbeamter}|pwk} wurde der
                  Prozess gemacht, weil ihm seine Gattin\pwindex{Combles de Nayve, Blanche Marie *~2.\,9.\,1854 Saint-Amand-Montrond@\textsc{Combles de Nayve, Blanche Marie} (*~2.\,9.\,1854 Saint-Amand-Montrond)|pwkv} vorgeworfen hatte, er hätte absichtlich ihr
                  leibliches Kind\pwindex{Menaldo, Hippolyte 11.\,7.\,1871 Le Havre – 10.\,11.\,1885 Vico Equense@\textsc{Menaldo, Hippolyte} (11.\,7.\,1871 Le Havre – 10.\,11.\,1885 Vico Equense)|pwkv} aus einer
                  früheren Beziehung einen Felsen hinunter in den Tod gestoßen. Letztlich wurde er
                  im Zweifel freigesprochen.}}}\label{K_L02754-3}, dazwiſchen Theater und{ }ſonſt allerhand – es
               bleibt gerade Zeit zum Eſſen und zum Schlafen, und auch dieſe nicht immer. Ich \strikeout{hätte}{ }{\pb}hätte Dir{ }ſoviel zu{ }ſagen, möchte Dir für Deine
               letzten{ }ſo lieben Briefe danken, – aber dieſe Arbeits-Woge iſt{ }ſtärker, als mein
               guter Wille, und ich kann nichts machen, als warten, bis{ }ſie vorüber iſt. Dieſer Tage
               hoffe ich endlich Dir ausführlicher{ }ſchreiben zu können. Einſtweilen{ }ſollen dieſe
               wenigen Zeilen mich nur bei Dir entſchuldigen. Wenn ich nach der Kammer\oindex{Französische Nationalversammlung@\textbf{Französische Nationalversammlung}, \emph{Regierungsgebäude}|pwv} gehe, kaufe ich mir hier und da ein
                  Wien\oindex{Wien@\textbf{Wien}, \emph{Verwaltungsgebiet}|pw}er Blatt auf dem \begin{otherlanguage}{french}\textsc{Boulevard}\end{otherlanguage} und{ }ſehe mit Freude, daß die »Liebelei\pwindex{Schnitzler, Arthur 15.\,5.\,1862 Wien – 21.\,10.\,1931 ebd.@\textsc{Schnitzler, Arthur} (15.\,5.\,1862 Wien – 21.\,10.\,1931 ebd.), \emph{Schriftsteller, Mediziner}!Liebelei. Schauspiel in drei Akten@\strich\emph{Liebelei. Schauspiel in drei Akten}|pw}« \strikeout{ſe\textcolor{gray}{i}} ihren {\pb}Platz im Repertoire behält. \strikeout{\textcolor{gray}{×}\-\textcolor{gray}{×}\-\textcolor{gray}{×}\-\textcolor{gray}{×}\-\textcolor{gray}{×}} Das iſt{ }ſchön.\pend
           
\pstart
           Viele treue Grüße! {\\[\baselineskip]}Dein {\\[\baselineskip]}\spacefill\mbox{Paul Goldmann.}\pend
           \leftskip=0em{}\selectlanguage{ngerman}\endnumbering\briefempfaengerindex{Schnitzler, Arthur@\textsc{Schnitzler, Arthur}!zzzGoldmann, Paul@\emph{von Paul Goldmann}!1895-11-061@{6. 11. [1895]}|)be}\mylabel{L02754h}  \newcommand{\dateiname}{L02754}\newcommand{\titel}{Paul Goldmann an Arthur Schnitzler, 6. 11. [1895]}\newcommand{\editorInnen}{Martin Anton Müller und Laura Untner}%% latex-leseansicht-abspann.tex
%% Abspann für die Leseansicht.
%% Der Schalter \ifkorrekturansicht ist bereits durch den Vorspann gesetzt.

%% latex-abspann.tex
%% Gemeinsamer Abspann für Korrekturansicht und Leseansicht.
%% Setzt den Schalter \ifkorrekturansicht voraus (gesetzt in den
%% einbindenden Dateien latex-korrekturansicht-abspann.tex bzw.
%% latex-leseansicht-abspann.tex).
%% ---------------------------------------------------------------

\normalsize

% Das esempio-Environment wird nur in der Leseansicht benötigt
\ifkorrekturansicht\else
\newenvironment{esempio}[3]%
{
    \vspace{1.5ex}
    \rlap{\underline{#1}}
    \par
    \setlength{\parindent}{0cm}
    \nopagebreak
    \leftskip=#2cm
    \rightskip=#3cm
}
{
    \par
}
\fi

\doendnotes{C}
\bigskip
\vfill

\clearpage

\footnotesize

\ifkorrekturansicht
  \lohead{\textsc{register}}
\fi

% theindex-Environment neu definieren ohne reledmac
\makeatletter
\renewenvironment{theindex}{%
  \ifkorrekturansicht
    \section*{\indexname}%
  \else
    \subsubsection*{Index der erwähnten Entitäten}%
  \fi
  \setlength{\parindent}{0pt}%
  \setlength{\parskip}{0pt plus 0.3pt}%
  \let\item\@idxitem
}{%
  \ifkorrekturansicht\clearpage\fi
}
\makeatother

\IfFileExists{\jobname-pw.ind}{\input{\jobname-pw.ind}}{}

% Quellenangabe nur in der Leseansicht
\ifkorrekturansicht\else
% Fallback-Definitionen, falls die .tex-Datei \titel etc. nicht gesetzt hat
\providecommand{\titel}{}
\providecommand{\editorInnen}{}
\providecommand{\dateiname}{\jobname}

\vspace{3cm}

\vfill

\footnotesize
\textsc{Quelle}: \titel. Herausgegeben von {\editorInnen}. In: \emph{Arthur Schnitzler: Briefwechsel mit Autorinnen und Autoren}.
 Digitale Edition, https://schnitzler-briefe.acdh.oeaw.ac.at/{\dateiname}.html (Stand \today)
\fi

\end{document}


