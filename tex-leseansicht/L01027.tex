%% latex-leseansicht-vorspann.tex
%% Vorspann für die Leseansicht.
%% Lädt die gemeinsame Datei latex-vorspann.tex mit nicht gesetztem Schalter.

\newif\ifkorrekturansicht
\korrekturansichtfalse

\input{../tex-inputs/latex-vorspann}


\section[Arthur Schnitzler an Hugo von Hofmannsthal, 3. 4. 1900]{L01027 Arthur Schnitzler an Hugo von Hofmannsthal, 3. 4. 1900}
\nopagebreak\mylabel{L01027v}
\rehead{ }\normalsize\beginnumbering\briefempfaengerindex{Hofmannsthal, Hugo von@\textsc{Hofmannsthal, Hugo von}!zzzSchnitzler, Arthur@\emph{von Arthur Schnitzler}!1900-04-032@{3. 4. 1900}|(be}
\toendnotes[C]{\smallbreak\pagebreak[2]}
\correspDesc{Versand  durch Arthur Schnitzler am 3. 4. 1900 in Split
\newline{}Erhalt  durch Hugo von Hofmannsthal am 5. 4. 1900 in Paris}\toendnotes[C]{\smallbreak}
\Standort{FDH, Hs-30885,91.}
\physDesc{Bildpostkarte, 214 Zeichen
\newline{}Handschrift: Bleistift, deutsche Kurrent
\newline{}Versand: 1) Stempel: »\nobreak{}\oindex{Split@\textbf{Split}|pwk}Spluet – Spalato, 3 4 00\nobreak{}«.   2) Stempel: »\nobreak{}\oindex{Paris@\textbf{Paris}, \emph{Hauptstadt}|pwk}Paris Etranger, 5\textsuperscript{E} Avril 00\nobreak{}«. 
\newline{}Ordnung: von unbekannter Hand datiert: »3/4 00« }
\buchAbdrucke{\weitereDrucke{Hugo von Hofmannsthal, Arthur Schnitzler: \emph{Briefwechsel}. Herausgegeben von Therese Nickl und Heinrich Schnitzler. Frankfurt am Main: \emph{S. Fischer} 1964, S. 137.} }\pstart{}\textsc{{\pb}Hugo von Hofmannsthal}\pend{}\pstart{}\textsc{Paris\oindex{Paris@\textbf{Paris}, \emph{Hauptstadt}|pw}}\pend{}\pstart{}\textsc{192 Boulevard Haussman\oindex{Boulevard Haussmann@\textbf{Boulevard Haussmann}, \emph{Straße}|pw}}\pend{}{\bigskip}
\pstart
           \noindent{}\centering{}{\pb}\textcolor{gray}{\textbf{Spalato – Split. Porta aurea\oindex{Goldenes Tor@\textbf{Goldenes Tor}, \emph{Monument}|pw}.}}\pend
           \vspace{1em}
\pstart
           \noindent{}{\pb}lieber Hugo, ich bin in Trieſt\oindex{Triest@\textbf{Triest}, \emph{Verwaltungsgebiet}|pw} u
                  Raguſa\oindex{Dubrovnik@\textbf{Dubrovnik}|pw} geweſen, habe manches merkwürdige
               geſehen, bin heute in Spalato\oindex{Split@\textbf{Split}|pw}, fahre Nachts nach
                  Abazia\oindex{Opatija@\textbf{Opatija}, \emph{Hauptstadt}|pw} und fühle, dſs ich{ }ſehr allein
               bin.\pend
           \pstart Ihr \spacefill\mbox{Arthur}\pend{}\selectlanguage{ngerman}\endnumbering\briefempfaengerindex{Hofmannsthal, Hugo von@\textsc{Hofmannsthal, Hugo von}!zzzSchnitzler, Arthur@\emph{von Arthur Schnitzler}!1900-04-032@{3. 4. 1900}|)be}\mylabel{L01027h}  \newcommand{\dateiname}{L01027}\newcommand{\titel}{Arthur Schnitzler an Hugo von Hofmannsthal, 3. 4. 1900}\newcommand{\editorInnen}{Martin Anton Müller und Gerd-Hermann Susen}%% latex-leseansicht-abspann.tex
%% Abspann für die Leseansicht.
%% Der Schalter \ifkorrekturansicht ist bereits durch den Vorspann gesetzt.

%% latex-abspann.tex
%% Gemeinsamer Abspann für Korrekturansicht und Leseansicht.
%% Setzt den Schalter \ifkorrekturansicht voraus (gesetzt in den
%% einbindenden Dateien latex-korrekturansicht-abspann.tex bzw.
%% latex-leseansicht-abspann.tex).
%% ---------------------------------------------------------------

\normalsize

% Das esempio-Environment wird nur in der Leseansicht benötigt
\ifkorrekturansicht\else
\newenvironment{esempio}[3]%
{
    \vspace{1.5ex}
    \rlap{\underline{#1}}
    \par
    \setlength{\parindent}{0cm}
    \nopagebreak
    \leftskip=#2cm
    \rightskip=#3cm
}
{
    \par
}
\fi

\doendnotes{C}
\bigskip
\vfill

\clearpage

\footnotesize

\ifkorrekturansicht
  \lohead{\textsc{register}}
\fi

% theindex-Environment neu definieren ohne reledmac
\makeatletter
\renewenvironment{theindex}{%
  \ifkorrekturansicht
    \section*{\indexname}%
  \else
    \subsubsection*{Index der erwähnten Entitäten}%
  \fi
  \setlength{\parindent}{0pt}%
  \setlength{\parskip}{0pt plus 0.3pt}%
  \let\item\@idxitem
}{%
  \ifkorrekturansicht\clearpage\fi
}
\makeatother

\IfFileExists{\jobname-pw.ind}{\input{\jobname-pw.ind}}{}

% Quellenangabe nur in der Leseansicht
\ifkorrekturansicht\else
% Fallback-Definitionen, falls die .tex-Datei \titel etc. nicht gesetzt hat
\providecommand{\titel}{}
\providecommand{\editorInnen}{}
\providecommand{\dateiname}{\jobname}

\vspace{3cm}

\vfill

\footnotesize
\textsc{Quelle}: \titel. Herausgegeben von {\editorInnen}. In: \emph{Arthur Schnitzler: Briefwechsel mit Autorinnen und Autoren}.
 Digitale Edition, https://schnitzler-briefe.acdh.oeaw.ac.at/{\dateiname}.html (Stand \today)
\fi

\end{document}


