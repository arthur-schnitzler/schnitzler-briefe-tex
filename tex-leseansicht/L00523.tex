%% latex-leseansicht-vorspann.tex
%% Vorspann für die Leseansicht.
%% Lädt die gemeinsame Datei latex-vorspann.tex mit nicht gesetztem Schalter.

\newif\ifkorrekturansicht
\korrekturansichtfalse

\input{../tex-inputs/latex-vorspann}


         
         \renewcommand{\erwaehntePersonen}{Personen: Lou Andreas-Salomé, Max Halbe}
         \renewcommand{\erwaehnteOrte}{Orte: Café Griensteidl, Hotel Imperial, Wien}
         \renewcommand{\erwaehnteWerke}{
               \section[Richard Beer-Hofmann an Arthur Schnitzler, {[}17. 12. 1895{]}]{ Richard Beer-Hofmann an Arthur Schnitzler,
               {[}17. 12. 1895{]}}\nopagebreak\mylabel{v}\rehead{ }\begin{ledgroupsized}[t]{13cm}\normalsize\beginnumbering \toendnotes[C]{\smallbreak\pagebreak[2]} \Standort{CUL, Schnitzler, B 8.}
\physDesc{Briefkarte
\newline{}Handschrift: schwarze Tinte, lateinische Kurrent
\newline{}Schnitzler: mit Bleistift datiert: »\substVorne{}\textsuperscript{18}\substDazwischen{}17\substHinten{}. 12. 95« \newline{}Ordnung: mit Bleistift von unbekannter Hand nummeriert: »72« }\buchAbdrucke{\weitereDrucke{Arthur Schnitzler, Richard Beer-Hofmann: \emph{Briefwechsel 1891–1931}. Hg. Konstanze Fliedl. Wien, Zürich: \emph{Europaverlag} 1992, S. 89.} }\pstart
           \noindent{}{\pb}Lieber Arthur Sie sind ja sicher morgen um halb eins im Griensteidel\oindex{Cafe Griensteidl@\textbf{Café Griensteidl}|pw}? Wenn ich nicht \uline{punkt halbeins} dort bin, dann gehen Sie mit Halbe\pwindex{Halbe, Max 04.10.1865 – 30.11.1944@\textsc{Halbe, Max} (04.10.1865 – 30.11.1944), \emph{Schriftsteller}|pw} zu Lou\pwindex{Andreas-Salome, Lou 12.02.1861 – 05.02.1937@\textsc{Andreas-Salomé, Lou} (12.02.1861 – 05.02.1937), \emph{Schriftstellerin}|pw}. Ich ko{\geminationm}e dann gegen halbzwei ins Imperial\oindex{Hotel Imperial@\textbf{Hotel Imperial}|pw}{ }{\pb}direkt.\pend
           \pstart
           Herzlichst{\\}Ihr{\\}\spacefill\mbox{R.}\pend
           
         
         \endnumbering\mylabel{h}\end{ledgroupsized}  \newcommand{\dateiname}{L00523}\newcommand{\titel}{Richard Beer-Hofmann an Arthur Schnitzler, [17. 12. 1895]}\newcommand{\editorInnen}{Martin Anton Müller und Gerd-Hermann Susen}%% latex-leseansicht-abspann.tex
%% Abspann für die Leseansicht.
%% Der Schalter \ifkorrekturansicht ist bereits durch den Vorspann gesetzt.

%% latex-abspann.tex
%% Gemeinsamer Abspann für Korrekturansicht und Leseansicht.
%% Setzt den Schalter \ifkorrekturansicht voraus (gesetzt in den
%% einbindenden Dateien latex-korrekturansicht-abspann.tex bzw.
%% latex-leseansicht-abspann.tex).
%% ---------------------------------------------------------------

\normalsize

% Das esempio-Environment wird nur in der Leseansicht benötigt
\ifkorrekturansicht\else
\newenvironment{esempio}[3]%
{
    \vspace{1.5ex}
    \rlap{\underline{#1}}
    \par
    \setlength{\parindent}{0cm}
    \nopagebreak
    \leftskip=#2cm
    \rightskip=#3cm
}
{
    \par
}
\fi

\doendnotes{C}
\bigskip
\vfill

\clearpage

\footnotesize

\ifkorrekturansicht
  \lohead{\textsc{register}}
\fi

% theindex-Environment neu definieren ohne reledmac
\makeatletter
\renewenvironment{theindex}{%
  \ifkorrekturansicht
    \section*{\indexname}%
  \else
    \subsubsection*{Index der erwähnten Entitäten}%
  \fi
  \setlength{\parindent}{0pt}%
  \setlength{\parskip}{0pt plus 0.3pt}%
  \let\item\@idxitem
}{%
  \ifkorrekturansicht\clearpage\fi
}
\makeatother

\IfFileExists{\jobname-pw.ind}{\input{\jobname-pw.ind}}{}

% Quellenangabe nur in der Leseansicht
\ifkorrekturansicht\else
% Fallback-Definitionen, falls die .tex-Datei \titel etc. nicht gesetzt hat
\providecommand{\titel}{}
\providecommand{\editorInnen}{}
\providecommand{\dateiname}{\jobname}

\vspace{3cm}

\vfill

\footnotesize
\textsc{Quelle}: \titel. Herausgegeben von {\editorInnen}. In: \emph{Arthur Schnitzler: Briefwechsel mit Autorinnen und Autoren}.
 Digitale Edition, https://schnitzler-briefe.acdh.oeaw.ac.at/{\dateiname}.html (Stand \today)
\fi

\end{document}


      