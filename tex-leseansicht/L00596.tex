\input{../tex-inputs/latex-pdf-vorspann}
\begin{center}
            \textcolor{red}{ENTWURF. ENTZIFFERUNG NOCH NICHT KORREKTURGELESEN}
                      \end{center}
            
               \section[Arthur Schnitzler an Richard Beer-Hofmann, 21. 9. 1896]{ Arthur Schnitzler an Richard Beer-Hofmann, 21. 9. 1896}\nopagebreak\mylabel{v}\rehead{ }\begin{ledgroupsized}[t]{13cm}\normalsize\beginnumbering\briefempfaengerindex{Beer-Hofmann, Richard@\textsc{Beer-Hofmann, Richard}!zzzSchnitzler, Arthur@\emph{von Arthur Schnitzler}!1896-09-211@{21. 9. 1896}|(be} \toendnotes[C]{\smallbreak\pagebreak[2]} \Standort{YCGL, MSS 31.}
\physDesc{Brief, 1 Blatt, 4 Seiten, Umschlag
\newline{}Handschrift: Bleistift, deutsche Kurrent\newline{}Versand: 1) Stempel: »\nobreak{}\oindex{IX., Alsergrund@\textbf{IX., Alsergrund}|pwk}Wien 9/3, 21. 9. 96, 3–4N\nobreak{}«.  2) Stempel: »\nobreak{}\oindex{Baden bei Wien@\textbf{Baden bei Wien}|pwk}Baden, 22. 9. 96, 7–10V, Bestellt\nobreak{}«. 3) Stempel: »\nobreak{}\oindex{I., Innere Stadt@\textbf{I., Innere Stadt}|pwk}{[}Wie{]}n 1/1, 22. 9. 96, 3–4½N, {[}Be{]}stellt\nobreak{}«. 4) von
                           unbekannter Hand nachgesandt nach Wien\oindex{Wien@\textbf{Wien}|pw}, I Wollzeile 15\oindex{Wollzeile@\textbf{Wollzeile}|pw}}\buchAbdrucke{\weitereDrucke{Arthur Schnitzler, Richard Beer-Hofmann: \emph{Briefwechsel 1891–1931}. Hg. Konstanze Fliedl. Wien, Zürich: \emph{Europaverlag} 1992, S. 98–99.} }\toendnotes[C]{\smallbreak}\pstart{}{\pb}Herrn Doctor \textsc{Rich.
                     Beer-Hofmann}\pend{}\pstart{}\textsc{Baden bei Wien\oindex{Baden bei Wien@\textbf{Baden bei Wien}|pw}.}\pend{}\pstart{}Franzensgaſſe 54\oindex{Kaiser-Franz-Ring@\textbf{Kaiser-Franz-Ring}|pw}, Th. 8.\pend{}{\bigskip}\pstart
           \noindent{}{\pb}Lieber Richard, gerade wie ich die Sitze nehmen wollte,
                  treff\textcolor{gray}{e} ich Dörma{\geminationn}\pwindex{Doermann, Felix 29.05.1870 – 26.10.1928@\textsc{Dörmann, Felix} (29.05.1870 – 26.10.1928), \emph{Schriftsteller}|pw} der eben einen Brief erhalten (ich las den Brief) daſs Sein Sohn\pwindex{Doermann, Felix 29.05.1870 – 26.10.1928@\textsc{Dörmann, Felix} (29.05.1870 – 26.10.1928), \emph{Schriftsteller}!Sein Sohn16.10.1896 – 16.10.1896@\strich\emph{Sein Sohn} {[}16.10.1896 – 16.10.1896{]}|pw} auf \label{K_L00596_1v}\edtext{unbesti{\geminationm}te Zeit}{\lemma{\textnormal{\emph{unbestite Zeit}}}\Cendnote{\textnormal{Hugo Ranzenberg\pwindex{Ranzenberg, Hugo 1852-09-13 – 1896-09-21@\textsc{Ranzenberg, Hugo} (1852-09-13 – 1896-09-21), \emph{Regisseur, Schauspieler}|pwk} starb am
                     21. 9. 1896, die Uraufführung fand dann am
                     16. 10. 1896 statt.}}}\label{K_L00596_1h} verſchoben wegen {\pb}Erkrankung Ranzenberg\pwindex{Ranzenberg, Hugo 1852-09-13 – 1896-09-21@\textsc{Ranzenberg, Hugo} (1852-09-13 – 1896-09-21), \emph{Regisseur, Schauspieler}|pw}s. –\pend
           \pstart
           Am Mittwoch{ }Abend hole ich Sie gegen acht ab; ich werde unten
               läuten. –\pend
           \pstart
           Im übrigen könnte man auch ein Stück in 9 Akten ſchreiben, Märchen\pwindex{Schnitzler, Arthur 15.05.1862 – 21.10.1931@\textsc{Schnitzler, Arthur} (15.05.1862 – 21.10.1931), \emph{Schriftsteller, Mediziner}!Maerchen. Schauspiel in drei Aufzuegen1891 – 1891@\strich\emph{Das Märchen. Schauspiel in drei Aufzügen} {[}1891 – 1891{]}|pw}, Liebelei\pwindex{Schnitzler, Arthur 15.05.1862 – 21.10.1931@\textsc{Schnitzler, Arthur} (15.05.1862 – 21.10.1931), \emph{Schriftsteller, Mediziner}!Liebelei. Schauspiel in drei Akten9. 10. 1895@\strich\emph{Liebelei. Schauspiel in drei Akten} {[}9. 10. 1895{]}|pw}, u Freiwild\pwindex{Schnitzler, Arthur 15.05.1862 – 21.10.1931@\textsc{Schnitzler, Arthur} (15.05.1862 – 21.10.1931), \emph{Schriftsteller, Mediziner}!Freiwild. Schauspiel in 3 Akten1896@\strich\emph{Freiwild. Schauspiel in 3 Akten} {[}1896{]}|pw} zuſa{\geminationm}en. Nur
               kleine Aenderungen {\pb}wären nothwendig, der alte Geiger\pwindex{Schnitzler, Arthur 15.05.1862 – 21.10.1931@\textsc{Schnitzler, Arthur} (15.05.1862 – 21.10.1931), \emph{Schriftsteller, Mediziner}!Liebelei. Schauspiel in drei Akten9. 10. 1895@\strich\emph{Liebelei. Schauspiel in drei Akten} {[}9. 10. 1895{]}|pwv} wär eine alte Geigerin (bei einer
               Damenkapelle) als Mutter der \label{T_L00596_1v}\edtext{Fanny\pwindex{Schnitzler, Arthur 15.05.1862 – 21.10.1931@\textsc{Schnitzler, Arthur} (15.05.1862 – 21.10.1931), \emph{Schriftsteller, Mediziner}!Maerchen. Schauspiel in drei Aufzuegen1891 – 1891@\strich\emph{Das Märchen. Schauspiel in drei Aufzügen} {[}1891 – 1891{]}|pwv}–Chriſtine\pwindex{Schnitzler, Arthur 15.05.1862 – 21.10.1931@\textsc{Schnitzler, Arthur} (15.05.1862 – 21.10.1931), \emph{Schriftsteller, Mediziner}!Liebelei. Schauspiel in drei Akten9. 10. 1895@\strich\emph{Liebelei. Schauspiel in drei Akten} {[}9. 10. 1895{]}|pwv}–Anna\pwindex{Schnitzler, Arthur 15.05.1862 – 21.10.1931@\textsc{Schnitzler, Arthur} (15.05.1862 – 21.10.1931), \emph{Schriftsteller, Mediziner}!Freiwild. Schauspiel in 3 Akten1896@\strich\emph{Freiwild. Schauspiel in 3 Akten} {[}1896{]}|pwv}}{\lemma{\textnormal{\emph{Fanny–Chriſtine–Anna}}}\Cendnote{\textnormal{Eine geschwungene Klammer oberhalb
                  verbindet die Namen und scheint sie der Damenkapelle zuzuordnen.}}}\label{T_L00596_1h}, der
               Doctor Witte\pwindex{Schnitzler, Arthur 15.05.1862 – 21.10.1931@\textsc{Schnitzler, Arthur} (15.05.1862 – 21.10.1931), \emph{Schriftsteller, Mediziner}!Maerchen. Schauspiel in drei Aufzuegen1891 – 1891@\strich\emph{Das Märchen. Schauspiel in drei Aufzügen} {[}1891 – 1891{]}|pwv} wär \substVorne{}\textsuperscript{d}\substDazwischen{}n\substHinten{}ahe daran, ſeine Praxis niederzulegen weil ſich der Fedor Denner\pwindex{Schnitzler, Arthur 15.05.1862 – 21.10.1931@\textsc{Schnitzler, Arthur} (15.05.1862 – 21.10.1931), \emph{Schriftsteller, Mediziner}!Maerchen. Schauspiel in drei Aufzuegen1891 – 1891@\strich\emph{Das Märchen. Schauspiel in drei Aufzügen} {[}1891 – 1891{]}|pwv} nicht mit ihm ſchlagen will, und
                  {\pb}der Moritzki\pwindex{Schnitzler, Arthur 15.05.1862 – 21.10.1931@\textsc{Schnitzler, Arthur} (15.05.1862 – 21.10.1931), \emph{Schriftsteller, Mediziner}!Freiwild. Schauspiel in 3 Akten1896@\strich\emph{Freiwild. Schauspiel in 3 Akten} {[}1896{]}|pwv} wäre vom Direktor Schneider\pwindex{Schnitzler, Arthur 15.05.1862 – 21.10.1931@\textsc{Schnitzler, Arthur} (15.05.1862 – 21.10.1931), \emph{Schriftsteller, Mediziner}!Freiwild. Schauspiel in 3 Akten1896@\strich\emph{Freiwild. Schauspiel in 3 Akten} {[}1896{]}|pwv} ins Haus der alten Geigerin geſandt. –\pend
           \pstart
           Die Athenerin\pwindex{Ebermann, Leo 16.07.1863 – 09.10.1914@\textsc{Ebermann, Leo} (16.07.1863 – 09.10.1914), \emph{Schriftsteller, Journalist, Rechtswissenschaftler}!Athenerin19.9.1896 – 19.9.1896@\strich\emph{Die Athenerin} {[}19.9.1896 – 19.9.1896{]}|pw} hat großen Erfolg gehabt, und Bauer\pwindex{Bauer, Julius 15.10.1853 – 11.06.1941@\textsc{Bauer, Julius} (15.10.1853 – 11.06.1941), \emph{Schriftsteller, Journalist, Kritiker}|pw} war bei der Première aufgeregter als der Autor\pwindex{Ebermann, Leo 16.07.1863 – 09.10.1914@\textsc{Ebermann, Leo} (16.07.1863 – 09.10.1914), \emph{Schriftsteller, Journalist, Rechtswissenschaftler}|pwv}, (wie er \introOben{}(B.\pwindex{Bauer, Julius 15.10.1853 – 11.06.1941@\textsc{Bauer, Julius} (15.10.1853 – 11.06.1941), \emph{Schriftsteller, Journalist, Kritiker}|pw})\introOben{} ſelbſt im Parquet
               erzählte). –\pend
           \pstart
           Herzlich Ihr{\\[\baselineskip]}\spacefill\mbox{Arthur}\pend
           \leftskip=0em{}\endnumbering\briefempfaengerindex{Beer-Hofmann, Richard@\textsc{Beer-Hofmann, Richard}!zzzSchnitzler, Arthur@\emph{von Arthur Schnitzler}!1896-09-211@{21. 9. 1896}|)be}\mylabel{h}\end{ledgroupsized}  \newcommand{\dateiname}{L00596}\newcommand{\titel}{Arthur Schnitzler an Richard Beer-Hofmann, 21. 9. 1896}\newcommand{\editorInnen}{Martin Anton Müller und Gerd-Hermann Susen}\input{../tex-inputs/latex-pdf-abspann}
      