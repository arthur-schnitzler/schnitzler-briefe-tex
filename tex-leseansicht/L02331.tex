%% latex-leseansicht-vorspann.tex
%% Vorspann für die Leseansicht.
%% Lädt die gemeinsame Datei latex-vorspann.tex mit nicht gesetztem Schalter.

\newif\ifkorrekturansicht
\korrekturansichtfalse

\input{../tex-inputs/latex-vorspann}


\section[Hugo Hofmannsthal an Arthur Schnitzler, 2. 11. 1919]{L02331 Hugo Hofmannsthal an Arthur Schnitzler, 2. 11. 1919}
\nopagebreak\mylabel{L02331v}
\rehead{ }\normalsize\beginnumbering\briefempfaengerindex{Schnitzler, Arthur@\textsc{Schnitzler, Arthur}!zzzHofmannsthal, Hugo von@\emph{von Hugo von Hofmannsthal}!1919-11-021@{2. 11. 1919}|(be}
\toendnotes[C]{\smallbreak\pagebreak[2]}
\correspDesc{Versand  durch Hugo von Hofmannsthal am 2. 11. 1919 in Bad Aussee
\newline{}Erhalt  durch Arthur Schnitzler im Zeitraum [3. 11. 1919
                  – 7. 11. 1919?] in Wien}\toendnotes[C]{\smallbreak}
\Standort{CUL, Schnitzler, B 43.}
\physDesc{Brief, 1 Blatt, 4 Seiten, 2145 Zeichen
\newline{}Handschrift: schwarze Tinte, deutsche Kurrent
\newline{}Schnitzler: 1) mit Bleistift die Jahreszahl »19« ergänzt  2) mit rotem Buntstift einzelne Unterstreichungen
\newline{}Ordnung: 1) mit Bleistift von Frieda
                                    Pollak\pwindex{Pollak, Frieda 8.\,12.\,1881 Wien – 13.\,7.\,1937 ebd.@\textsc{Pollak, Frieda} (8.\,12.\,1881 Wien – 13.\,7.\,1937 ebd.), \emph{Sekretärin}|pw} (?) mit dem Buchstaben »A«
                                 (Abgeschrieben/Abschrift) gekennzeichnet  2) mit Bleistift von unbekannter Hand nummeriert: »\strikeout{354}« 3) mit Bleistift von unbekannter Hand nummeriert: »?
                                    383«, bei der von Schnitzler ergänzten Jahreszahl
                                 ebenfalls ein Fragezeichen hinzugefügt}
\buchAbdrucke{\weitereDrucke{Hugo von Hofmannsthal, Arthur Schnitzler: \emph{Briefwechsel}. Herausgegeben von Therese Nickl und Heinrich Schnitzler. Frankfurt am Main: \emph{S. Fischer} 1964, S. 287.} }\toendnotes[C]{\smallbreak}
\pstart
           \raggedleft{}{\pb}Bad Aussee\oindex{Bad Aussee@\textbf{Bad Aussee}, \emph{Hauptstadt}|pw}{ }2 XI 19\pend
           
\pstart{}mein lieber Arthur\pend\vspace{0.5em}
\pstart
           Sie haben mir vor mehr als einem Monat einen{ }ſo lieben{ }ſchönen Brief hierher
               geſchrieben – ich dank Ihnen vielmals dafür.\hspace*{1.5em}Über
               unſere Vorleſungen denk ich{ }ſo wie Sie:{ }ſie{ }ſind mir auch als Feſte ganz beſonderer
               Art in der Erinnerung, und am{ }ſtärkſten und beſonderſten von allen \label{K_L02331-1v}\edtext{die des »Märchens\pwindex{Schnitzler, Arthur 15.\,5.\,1862 Wien – 21.\,10.\,1931 ebd.@\textsc{Schnitzler, Arthur} (15.\,5.\,1862 Wien – 21.\,10.\,1931 ebd.), \emph{Schriftsteller, Mediziner}!Märchen. Schauspiel in drei Aufzügen@\strich\emph{Das Märchen. Schauspiel in drei Aufzügen}|pw}«}{\lemma{\textnormal{\emph{die des »Märchens«}}}\Cendnote{\textnormal{am
                     25. 6. 1891}}}\label{K_L02331-1} in Richards\pwindex{Beer-Hofmann, Richard 11.\,7.\,1866 Wien – 26.\,9.\,1945 New York City@\textsc{Beer-Hofmann, Richard} (11.\,7.\,1866 Wien – 26.\,9.\,1945 New York City), \emph{Schriftsteller}|pw} verhängter u. nach
               Naphtalin riechender Wohnung in der \label{K_L02331-2v}\edtext{Gärtnergaſſe\oindex{Wien@\textbf{Wien}!III., Landstraße@\textbf{III., Landstraße}!Gärtnergasse@\textbf{Gärtnergasse}, \emph{Straße}|pw}}{\lemma{\textnormal{\emph{Gärtnergasse}}}\Cendnote{\textnormal{Vermutlich eine Verwechslung, er dürfte
                  eine Parallelstraße meinen, die Seidlgasse\oindex{Wien@\textbf{Wien}!III., Landstraße@\textbf{III., Landstraße}!Seidlgasse@\textbf{Seidlgasse}, \emph{Straße}|pwk}.}}}\label{K_L02331-2} – aber auch manche Andere,{ }ſo \label{K_L02331-3v}\edtext{ein Abend}{\lemma{\textnormal{\emph{ein Abend}}}\Cendnote{\textnormal{am 11. 4. 1904, in
                  Anwesenheit von Schwarzkopf\pwindex{Schwarzkopf, Gustav 7.\,11.\,1853 Wien – 13.\,11.\,1939 ebd.@\textsc{Schwarzkopf, Gustav} (7.\,11.\,1853 Wien – 13.\,11.\,1939 ebd.), \emph{Schriftsteller}|pwk}}}}\label{K_L02331-3} wo Sie mir ganz allein – oder mir und Schwarzkopf\pwindex{Schwarzkopf, Gustav 7.\,11.\,1853 Wien – 13.\,11.\,1939 ebd.@\textsc{Schwarzkopf, Gustav} (7.\,11.\,1853 Wien – 13.\,11.\,1939 ebd.), \emph{Schriftsteller}|pw} – in der Wohnung, die Sie vor dieſer jetzigen zuletzt bewohnten
               – die Geſchichte des Freiherrn von \textsc{Leisenbogh}\pwindex{Schnitzler, Arthur 15.\,5.\,1862 Wien – 21.\,10.\,1931 ebd.@\textsc{Schnitzler, Arthur} (15.\,5.\,1862 Wien – 21.\,10.\,1931 ebd.), \emph{Schriftsteller, Mediziner}!Schicksal des Freiherrn von Leisenbohg. Novellette@\strich\emph{Das Schicksal des Freiherrn von Leisenbohg. Novellette}|pw} vorlaſen, die ich{ }ſo beſonders liebe.\pend
           
\pstart
           \label{T_L02331-1v}\edtext{Wenn}{\lemma{\textnormal{\emph{Wenn}}}\Cendnote{\textnormal{Absatztrennmarkierung nachträglich mit Bleistift eingefügt}}}\label{T_L02331-1}{ }{\pb}ich das Geſellſchaftsluſtſpiel\pwindex{Hofmannsthal, Hugo von 1.\,2.\,1874 Wien – 15.\,7.\,1929 Rodaun@\textsc{Hofmannsthal, Hugo von} (1.\,2.\,1874 Wien – 15.\,7.\,1929 Rodaun), \emph{Schriftsteller}!Schwierige. Lustspiel in drei Akten@\strich\emph{Der Schwierige. Lustspiel in drei Akten}|pwv} fertig habe, an dem
               ich immer noch im Einzelnen herumbeſſere,{ }ſo freue ich mich recht{ }ſehr, es Ihnen,{ }ſei
               es Ihnen allein oder mit noch ein paar Menſchen, zu leſen. Vielleicht hätte ich die
               Geſellſchaft, die es darſtellt, die Oeſterreichiſche \strikeout{arſtr} ariſtokratiſche Geſellſchaft, nie mit{ }ſo viel Liebe in ihrem \textsc{charme} und ihrer Qualität darſtellen können als in dem
               hiſtoriſchen Augenblick wo{ }ſie, die bis vor kurzem eine Gegebenheit, ja eine Macht
               war,{ }ſich leiſe u. geiſterhaft ins Nichts auflöst, wie {\pb}ein übriggebliebenes Nebelwölkchen
               am Morgen.\pend
           
\pstart
           Inzwiſchen iſt das Märchen von der Frau ohne
                  Schatten\pwindex{Hofmannsthal, Hugo von 1.\,2.\,1874 Wien – 15.\,7.\,1929 Rodaun@\textsc{Hofmannsthal, Hugo von} (1.\,2.\,1874 Wien – 15.\,7.\,1929 Rodaun), \emph{Schriftsteller}!Frau ohne Schatten. Erzählung@\strich\emph{Die Frau ohne Schatten. Erzählung}|pw} zu Ihnen gewandert, und, hoffentlich,{ }ſeit langem in Ihren
               Händen.\pend
           
\pstart
           Ich habe, in faſt{ }ſieben Jahren, unſäglich viel Mühe an dieſe kleine Arbeit gewandt –
               hoffentlich merkt man ihr dies nicht an. Wenn{ }ſie Ihnen und Olga\pwindex{Schnitzler, Olga 17.\,1.\,1882 Wien – 13.\,1.\,1970 Lugano@\textsc{Schnitzler, Olga} (17.\,1.\,1882 Wien – 13.\,1.\,1970 Lugano), \emph{Schauspielerin, Sängerin}|pw} ein bischen Vergnügen gemacht hat,{ }ſo{ }ſchreiben Sie mir
               ein paar Zeilen darüber – weſſen Beifall{ }ſollte man denn wünſchen u.{ }ſuchen, als der
               paar Menſchen mit {\pb}denen und durch
               die man das Leben gelebt hat.\pend
           
\pstart
           Adieu, Arthur.\hspace*{1.5em}Im Vorübergehen möcht ich Sie auf ein{ }ſehr kluges, zu vielem Denken anregendes Buch aufmerkſam machen, das mir dieſe
               letzten etwas unproductiveren Föhntage{ }ſehr bereichert hat: \textsc{Keyserlings\pwindex{Keyserling, Hermann von 20.\,7.\,1880 Kaisma – 26.\,4.\,1946 Innsbruck@\textsc{Keyserling, Hermann von} (20.\,7.\,1880 Kaisma – 26.\,4.\,1946 Innsbruck), \emph{Philosoph}|pw}}{ }Reiſetagebuch eines
                  Philoſophen\pwindex{Keyserling, Hermann von 20.\,7.\,1880 Kaisma – 26.\,4.\,1946 Innsbruck@\textsc{Keyserling, Hermann von} (20.\,7.\,1880 Kaisma – 26.\,4.\,1946 Innsbruck), \emph{Philosoph}!Reisetagebuch eines Philosophen@\strich\emph{Das Reisetagebuch eines Philosophen}|pw}.{\\}Ihr\spacefill\mbox{Hugo}\pend
           
\pstart
           \noindent{}PS. Iſt es denn richtig daſs ein abſurdes Geſetz einem Händler\pwindex{Henrici, Karl Ernst 1.\,9.\,1879 Leipzig – 9.\,11.\,1944 Hermsdorf@\textsc{Henrici, Karl Ernst} (1.\,9.\,1879 Leipzig – 9.\,11.\,1944 Hermsdorf), \emph{Buchhändler}|pwv} der Brahms\pwindex{Brahm, Otto 5.\,2.\,1856 Hamburg – 28.\,11.\,1912 Berlin@\textsc{Brahm, Otto} (5.\,2.\,1856 Hamburg – 28.\,11.\,1912 Berlin), \emph{Theaterleiter, Regisseur}|pw} ganzen Briefwechſel gekauft hat, jetzt das Recht
                  gibt, unſere{ }ſo ganz vertraulichen Briefe an den Todten, ob wir wollen oder nicht,
                  zu publicieren?\pend
           \selectlanguage{ngerman}\endnumbering\briefempfaengerindex{Schnitzler, Arthur@\textsc{Schnitzler, Arthur}!zzzHofmannsthal, Hugo von@\emph{von Hugo von Hofmannsthal}!1919-11-021@{2. 11. 1919}|)be}\mylabel{L02331h}  \newcommand{\dateiname}{L02331}\newcommand{\titel}{Hugo Hofmannsthal an Arthur Schnitzler, 2. 11. 1919}\newcommand{\editorInnen}{Martin Anton Müller und Gerd-Hermann Susen}%% latex-leseansicht-abspann.tex
%% Abspann für die Leseansicht.
%% Der Schalter \ifkorrekturansicht ist bereits durch den Vorspann gesetzt.

%% latex-abspann.tex
%% Gemeinsamer Abspann für Korrekturansicht und Leseansicht.
%% Setzt den Schalter \ifkorrekturansicht voraus (gesetzt in den
%% einbindenden Dateien latex-korrekturansicht-abspann.tex bzw.
%% latex-leseansicht-abspann.tex).
%% ---------------------------------------------------------------

\normalsize

% Das esempio-Environment wird nur in der Leseansicht benötigt
\ifkorrekturansicht\else
\newenvironment{esempio}[3]%
{
    \vspace{1.5ex}
    \rlap{\underline{#1}}
    \par
    \setlength{\parindent}{0cm}
    \nopagebreak
    \leftskip=#2cm
    \rightskip=#3cm
}
{
    \par
}
\fi

\doendnotes{C}
\bigskip
\vfill

\clearpage

\footnotesize

\ifkorrekturansicht
  \lohead{\textsc{register}}
\fi

% theindex-Environment neu definieren ohne reledmac
\makeatletter
\renewenvironment{theindex}{%
  \ifkorrekturansicht
    \section*{\indexname}%
  \else
    \subsubsection*{Index der erwähnten Entitäten}%
  \fi
  \setlength{\parindent}{0pt}%
  \setlength{\parskip}{0pt plus 0.3pt}%
  \let\item\@idxitem
}{%
  \ifkorrekturansicht\clearpage\fi
}
\makeatother

\IfFileExists{\jobname-pw.ind}{\input{\jobname-pw.ind}}{}

% Quellenangabe nur in der Leseansicht
\ifkorrekturansicht\else
% Fallback-Definitionen, falls die .tex-Datei \titel etc. nicht gesetzt hat
\providecommand{\titel}{}
\providecommand{\editorInnen}{}
\providecommand{\dateiname}{\jobname}

\vspace{3cm}

\vfill

\footnotesize
\textsc{Quelle}: \titel. Herausgegeben von {\editorInnen}. In: \emph{Arthur Schnitzler: Briefwechsel mit Autorinnen und Autoren}.
 Digitale Edition, https://schnitzler-briefe.acdh.oeaw.ac.at/{\dateiname}.html (Stand \today)
\fi

\end{document}


