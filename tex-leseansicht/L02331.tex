%% latex-leseansicht-vorspann.tex
%% Vorspann für die Leseansicht.
%% Lädt die gemeinsame Datei latex-vorspann.tex mit nicht gesetztem Schalter.

\newif\ifkorrekturansicht
\korrekturansichtfalse

\input{../tex-inputs/latex-vorspann}

\begin{center}
            \textcolor{red}{ENTWURF. ENTZIFFERUNG NOCH NICHT KORREKTURGELESEN}
                      \end{center}
            
               \section[Hugo Hofmannsthal an Arthur Schnitzler, 2. 11. 1919]{ Hugo Hofmannsthal an Arthur Schnitzler, 2. 11. 1919}\nopagebreak\mylabel{v}\rehead{ }\begin{ledgroupsized}[t]{13cm}\normalsize\beginnumbering\briefempfaengerindex{Schnitzler, Arthur@\textsc{Schnitzler, Arthur}!zzzHofmannsthal, Hugo von@\emph{von Hugo von Hofmannsthal}!1919-11-021@{2. 11. 1919}|(be} \toendnotes[C]{\smallbreak\pagebreak[2]} \Standort{CUL, Schnitzler, B 43.}
\physDesc{Brief, 1 Blatt, 4 Seiten
\newline{}Handschrift: schwarze Tinte, deutsche Kurrent
\newline{}Schnitzler: 1) mit Bleistift die Jahreszahl »19« ergänzt 2) mit rotem Buntstift einzelne Unterstreichungen\newline{}Ordnung: 1) mit Bleistift von Frieda
                                    Pollak\pwindex{Pollak, Frieda 08.12.1881 – 13.07.1937@\textsc{Pollak, Frieda} (08.12.1881 – 13.07.1937), \emph{Sekretärin}|pw} (?) mit dem Buchstaben »A«
                                 (Abgeschrieben/Abschrift) gekennzeichnet 2) mit Bleistift von unbekannter Hand nummeriert: »\strikeout{354}«3) mit Bleistift von unbekannter Hand nummeriert: »?
                                    383«, bei der von Schnitzler ergänzten Jahreszahl
                                 ebenfalls ein Fragezeichen hinzugefügt}\buchAbdrucke{\weitereDrucke{Hugo von Hofmannsthal, Arthur Schnitzler: \emph{Briefwechsel}. Hg. Therese Nickl und Heinrich Schnitzler. Frankfurt am Main: \emph{S. Fischer} 1964, S. 287.} }\toendnotes[C]{\smallbreak}\pstart
           \raggedleft{}{\pb}Bad Aussee\oindex{Bad Aussee@\textbf{Bad Aussee}|pw}{ }2 XI 19\pend
           \pstart{}mein lieber Arthur\pend\pstart
           Sie haben mir vor mehr als einem Monat einen ſo lieben ſchönen Brief hierher
               geſchrieben – ich dank Ihnen vielmals dafür.\hspace*{1.5em}Über
               unſere Vorleſungen denk ich ſo wie Sie: ſie ſind mir auch als Feſte ganz beſonderer
               Art in der Erinnerung, und am ſtärkſten und beſonderſten von allen \label{K_L02331_1v}\edtext{die des »Märchens\pwindex{Schnitzler, Arthur 15.05.1862 – 21.10.1931@\textsc{Schnitzler, Arthur} (15.05.1862 – 21.10.1931), \emph{Schriftsteller, Mediziner}!Maerchen. Schauspiel in drei Aufzuegen1891 – 1891@\strich\emph{Das Märchen. Schauspiel in drei Aufzügen} {[}1891 – 1891{]}|pw}«}{\lemma{\textnormal{\emph{die des »Märchens«}}}\Cendnote{\textnormal{am
                     25. 6. 1891}}}\label{K_L02331_1h} in Richard\pwindex{Beer-Hofmann, Richard 11.07.1866 – 26.09.1945@\textsc{Beer-Hofmann, Richard} (11.07.1866 – 26.09.1945), \emph{Schriftsteller}|pw}s verhängter u. nach Naphtalin
               riechender Wohnung in der \label{K_L02331_2v}\edtext{Gärtnergaſſe\oindex{Gaertnergasse@\textbf{Gärtnergasse}|pw}}{\lemma{\textnormal{\emph{Gärtnergaſſe}}}\Cendnote{\textnormal{Vermutlich eine Verwechslung, er dürfte
                  eine Parallelstraße meinen, die Seidlgasse\oindex{Seidlgasse@\textbf{Seidlgasse}|pwk}.}}}\label{K_L02331_2h} – aber auch manche Andere, ſo \label{K_L02331_3v}\edtext{ein Abend}{\lemma{\textnormal{\emph{ein Abend}}}\Cendnote{\textnormal{am 11. 4. 1904, in
                  Anwesenheit von Schwarzkopf\pwindex{Schwarzkopf, Gustav 07.11.1853 – 13.11.1939@\textsc{Schwarzkopf, Gustav} (07.11.1853 – 13.11.1939), \emph{Schriftsteller}|pwk}}}}\label{K_L02331_3h} wo Sie mir
               ganz allein – oder mir und Schwarzkopf\pwindex{Schwarzkopf, Gustav 07.11.1853 – 13.11.1939@\textsc{Schwarzkopf, Gustav} (07.11.1853 – 13.11.1939), \emph{Schriftsteller}|pw} – in der
               Wohnung, die Sie vor dieſer jetzigen zuletzt bewohnten – die Geſchichte des Freiherrn von \textsc{Leisenbogh}\pwindex{Schnitzler, Arthur 15.05.1862 – 21.10.1931@\textsc{Schnitzler, Arthur} (15.05.1862 – 21.10.1931), \emph{Schriftsteller, Mediziner}!Schicksal des Freiherrn von Leisenbohg. Novellette01. 07. 1904@\strich\emph{Das Schicksal des Freiherrn von Leisenbohg. Novellette} {[}01. 07. 1904{]}|pw} vorlaſen, die ich ſo beſonders liebe.\pend
           \pstart
           \label{T_L02331_1v}\edtext{Wenn}{\lemma{\textnormal{\emph{Wenn}}}\Cendnote{\textnormal{Absatztrennmarkierung nachträglich mit Bleistift eingefügt}}}\label{T_L02331_1h}{ }{\pb}ich das Geſellſchaftsluſtſpiel\pwindex{Hofmannsthal, Hugo von 01.02.1874 – 15.07.1929@\textsc{Hofmannsthal, Hugo von} (01.02.1874 – 15.07.1929), \emph{Schriftsteller}!Schwierige. Lustspiel in drei Akten1921 – 1921@\strich\emph{Der Schwierige. Lustspiel in drei Akten} {[}1921 – 1921{]}|pwv} fertig habe, an dem ich
               immer noch im Einzelnen herumbeſſere, ſo freue ich mich recht ſehr, es Ihnen, ſei es
               Ihnen allein oder mit noch ein paar Menſchen, zu leſen. Vielleicht hätte ich die
               Geſellſchaft, die es darſtellt, die Oeſterreichiſche \strikeout{arſtr} ariſtokratiſche Geſellſchaft, nie mit ſo viel Liebe in ihrem \textsc{charme} und ihrer Qualität darſtellen können als in dem
               hiſtoriſchen Augenblick wo ſie, die bis vor kurzem eine Gegebenheit, ja eine Macht
               war, ſich leiſe u. geiſterhaft ins Nichts auflöst, wie {\pb}ein übriggebliebenes Nebelwölkchen
               am Morgen.\pend
           \pstart
           Inzwiſchen iſt das Märchen von der Frau ohne
                  Schatten\pwindex{Hofmannsthal, Hugo von 01.02.1874 – 15.07.1929@\textsc{Hofmannsthal, Hugo von} (01.02.1874 – 15.07.1929), \emph{Schriftsteller}!Frau ohne Schatten. Erzaehlung1919 – 1919@\strich\emph{Die Frau ohne Schatten. Erzählung} {[}1919 – 1919{]}|pw} zu Ihnen gewandert, und, hoffentlich, ſeit langem in Ihren
               Händen.\pend
           \pstart
           Ich habe, in faſt ſieben Jahren, unſäglich viel Mühe an dieſe kleine Arbeit gewandt –
               hoffentlich merkt man ihr dies nicht an. Wenn ſie Ihnen und Olga\pwindex{Schnitzler, Olga 17.01.1882 – 13.01.1970@\textsc{Schnitzler, Olga} (17.01.1882 – 13.01.1970), \emph{Schauspielerin, Sängerin}|pw} ein bischen Vergnügen gemacht hat, ſo ſchreiben Sie mir
               ein paar Zeilen darüber – weſſen Beifall ſollte man denn wünſchen u. ſuchen, als der
               paar Menſchen mit {\pb}denen und durch
               die man das Leben gelebt hat.\pend
           \pstart
           Adieu, Arthur.\hspace*{1.5em}Im Vorübergehen möcht ich Sie auf ein
               ſehr kluges, zu vielem Denken anregendes Buch aufmerkſam machen, das mir dieſe
               letzten etwas unproductiveren Föhntage ſehr bereichert hat: \textsc{Keyserling\pwindex{Keyserling, Hermann von 20.07.1880 – 26.04.1946@\textsc{Keyserling, Hermann von} (20.07.1880 – 26.04.1946), \emph{Philosoph}|pw}s}{ }Reiſetagebuch eines
                  Philoſophen\pwindex{Keyserling, Hermann von 20.07.1880 – 26.04.1946@\textsc{Keyserling, Hermann von} (20.07.1880 – 26.04.1946), \emph{Philosoph}!Reisetagebuch eines Philosophen1919 – 1919@\strich\emph{Das Reisetagebuch eines Philosophen} {[}1919 – 1919{]}|pw}.{\\}Ihr\spacefill\mbox{Hugo}\pend
           \pstart
           \noindent{}PS. Iſt es denn richtig daſs ein abſurdes Geſetz einem Händler\pwindex{Henrici, Karl Ernst 1.9.1879 – 9.11.1944@\textsc{Henrici, Karl Ernst} (1.9.1879 – 9.11.1944), \emph{Buchhändler}|pwv} der Brahm\pwindex{Brahm, Otto 05.02.1856 – 28.11.1912@\textsc{Brahm, Otto} (05.02.1856 – 28.11.1912), \emph{Theaterleiter, Regisseur}|pw}s ganzen Briefwechſel gekauft hat, jetzt das Recht gibt, unſere ſo
                  ganz vertraulichen Briefe an den Todten, ob wir wollen oder nicht, zu publicieren?
               \pend
           \endnumbering\briefempfaengerindex{Schnitzler, Arthur@\textsc{Schnitzler, Arthur}!zzzHofmannsthal, Hugo von@\emph{von Hugo von Hofmannsthal}!1919-11-021@{2. 11. 1919}|)be}\mylabel{h}\end{ledgroupsized}  \newcommand{\dateiname}{L02331}\newcommand{\titel}{Hugo Hofmannsthal an Arthur Schnitzler, 2. 11. 1919}\newcommand{\editorInnen}{Martin Anton Müller und Gerd-Hermann Susen}%% latex-leseansicht-abspann.tex
%% Abspann für die Leseansicht.
%% Der Schalter \ifkorrekturansicht ist bereits durch den Vorspann gesetzt.

%% latex-abspann.tex
%% Gemeinsamer Abspann für Korrekturansicht und Leseansicht.
%% Setzt den Schalter \ifkorrekturansicht voraus (gesetzt in den
%% einbindenden Dateien latex-korrekturansicht-abspann.tex bzw.
%% latex-leseansicht-abspann.tex).
%% ---------------------------------------------------------------

\normalsize

% Das esempio-Environment wird nur in der Leseansicht benötigt
\ifkorrekturansicht\else
\newenvironment{esempio}[3]%
{
    \vspace{1.5ex}
    \rlap{\underline{#1}}
    \par
    \setlength{\parindent}{0cm}
    \nopagebreak
    \leftskip=#2cm
    \rightskip=#3cm
}
{
    \par
}
\fi

\doendnotes{C}
\bigskip
\vfill

\clearpage

\footnotesize

\ifkorrekturansicht
  \lohead{\textsc{register}}
\fi

% theindex-Environment neu definieren ohne reledmac
\makeatletter
\renewenvironment{theindex}{%
  \ifkorrekturansicht
    \section*{\indexname}%
  \else
    \subsubsection*{Index der erwähnten Entitäten}%
  \fi
  \setlength{\parindent}{0pt}%
  \setlength{\parskip}{0pt plus 0.3pt}%
  \let\item\@idxitem
}{%
  \ifkorrekturansicht\clearpage\fi
}
\makeatother

\IfFileExists{\jobname-pw.ind}{\input{\jobname-pw.ind}}{}

% Quellenangabe nur in der Leseansicht
\ifkorrekturansicht\else
% Fallback-Definitionen, falls die .tex-Datei \titel etc. nicht gesetzt hat
\providecommand{\titel}{}
\providecommand{\editorInnen}{}
\providecommand{\dateiname}{\jobname}

\vspace{3cm}

\vfill

\footnotesize
\textsc{Quelle}: \titel. Herausgegeben von {\editorInnen}. In: \emph{Arthur Schnitzler: Briefwechsel mit Autorinnen und Autoren}.
 Digitale Edition, https://schnitzler-briefe.acdh.oeaw.ac.at/{\dateiname}.html (Stand \today)
\fi

\end{document}


      