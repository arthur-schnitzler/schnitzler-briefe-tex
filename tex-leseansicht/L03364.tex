%% latex-leseansicht-vorspann.tex
%% Vorspann für die Leseansicht.
%% Lädt die gemeinsame Datei latex-vorspann.tex mit nicht gesetztem Schalter.

\newif\ifkorrekturansicht
\korrekturansichtfalse

\input{../tex-inputs/latex-vorspann}


         
         \renewcommand{\erwaehntePersonen}{Personen: Bjørnstjerne Bjørnson, Paul Goldmann}
         \renewcommand{\erwaehnteOrte}{Orte: Berlin, Dessauer Straße, Palasthotel Berlin, Residenztheater Berlin}
         \renewcommand{\erwaehnteWerke}{Werke: Leonarda}
               \section[ Paul Goldmann an Arthur Schnitzler, 21. 2. {[}1903{]}]{ Paul Goldmann an Arthur Schnitzler, 21. 2. {[}1903{]}}\nopagebreak\mylabel{v}\rehead{ }\begin{ledgroupsized}[t]{13cm}\normalsize\beginnumbering\briefempfaengerindex{Schnitzler, Arthur@\textsc{Schnitzler, Arthur}!zzzGoldmann, Paul@\emph{von Paul Goldmann}!1903-02-211@{21. 2. {[}1903{]}}|(be} \toendnotes[C]{\smallbreak\pagebreak[2]} \Standort{DLA, A:Schnitzler, HS.NZ85.1.3173.}
\physDesc{Brief, 1 Blatt, 2 Seiten, 300 Zeichen
\newline{}Handschrift: blaue Tinte, deutsche Kurrent
\newline{}Schnitzler: mit Bleistift das Jahr »903« vermerkt }\toendnotes[C]{\smallbreak}\pstart
           \noindent{}\raggedleft{}{\pb}\textcolor{gray}{\textbf{DESSAUERSTRASSE 19\oindex{Dessauer Strasse@\textbf{Dessauer Straße}|pw}}}\pend
           \pstart
           Berlin\oindex{Berlin@\textbf{Berlin}|pw}, 21. Februar. \pend
           \pstart\center{}Mein lieber Freund,\pend\pstart
           Herzlichſt \label{K_L03364-1v}\edtext{willkommen}{\lemma{\textnormal{\emph{willkommen}}}\Cendnote{\textnormal{Schnitzler\pwindex{Schnitzler, Arthur 15.05.1862 – 21.10.1931@\textsc{Schnitzler, Arthur} (15.05.1862 – 21.10.1931), \emph{Schriftsteller, Mediziner}|pwk} kam am 22. 2. 1903 morgens in
                     Berlin\oindex{Berlin@\textbf{Berlin}|pwk} an.}}}\label{K_L03364-1h}! Ich muß leider zu
                  eine\textcolor{gray}{r}{ }\textsc{Matinée} ins Reſidenztheater\oindex{Residenztheater Berlin@\textbf{Residenztheater Berlin}|pw} (»\textsc{Leonarda\pwindex{Bjørnson, Bjørnstjerne 1832-12-08 – 1910-04-26@\textsc{Bjørnson, Bjørnstjerne} (1832-12-08 – 1910-04-26), \emph{Schriftsteller}!Leonarda1879@\strich\emph{Leonarda} {[}1879{]}|pw}}« von \textsc{Björnson\pwindex{Bjørnson, Bjørnstjerne 1832-12-08 – 1910-04-26@\textsc{Bjørnson, Bjørnstjerne} (1832-12-08 – 1910-04-26), \emph{Schriftsteller}|pw}}). Anbei ein Billet, für den Fall, daß Du \label{K_L03364-2v}\edtext{mitkommen}{\lemma{\textnormal{\emph{mitkommen}}}\Cendnote{\textnormal{Schnitzler\pwindex{Schnitzler, Arthur 15.05.1862 – 21.10.1931@\textsc{Schnitzler, Arthur} (15.05.1862 – 21.10.1931), \emph{Schriftsteller, Mediziner}|pwk} kam nicht mit, man sah sich am
                  Nachmittag, vgl. A. S.: \emph{Tagebuch}, 22. 2. 1903.}}}\label{K_L03364-2h} willſt. Wenn nicht, ſo {\pb}komme ich
                  zwiſchen 3 und \strikeout{1/} 3 ½
                  Uhr zu Dir ins \textsc{Hôtel\oindex{Palasthotel Berlin@\textbf{Palasthotel Berlin}|pwv}}. Bis 5 Uhr bin ich frei.\pend
           \pstart
           Herzlichſt {\\[\baselineskip]}Dein {\\[\baselineskip]}\spacefill\mbox{Paul Goldm}\pend
           \leftskip=0em{}
         
         \endnumbering\mylabel{h}\end{ledgroupsized}  \newcommand{\dateiname}{L03364}\newcommand{\titel}{Paul Goldmann an Arthur Schnitzler, 21. 2. [1903]}\newcommand{\editorInnen}{Martin Anton Müller und Laura Untner}%% latex-leseansicht-abspann.tex
%% Abspann für die Leseansicht.
%% Der Schalter \ifkorrekturansicht ist bereits durch den Vorspann gesetzt.

%% latex-abspann.tex
%% Gemeinsamer Abspann für Korrekturansicht und Leseansicht.
%% Setzt den Schalter \ifkorrekturansicht voraus (gesetzt in den
%% einbindenden Dateien latex-korrekturansicht-abspann.tex bzw.
%% latex-leseansicht-abspann.tex).
%% ---------------------------------------------------------------

\normalsize

% Das esempio-Environment wird nur in der Leseansicht benötigt
\ifkorrekturansicht\else
\newenvironment{esempio}[3]%
{
    \vspace{1.5ex}
    \rlap{\underline{#1}}
    \par
    \setlength{\parindent}{0cm}
    \nopagebreak
    \leftskip=#2cm
    \rightskip=#3cm
}
{
    \par
}
\fi

\doendnotes{C}
\bigskip
\vfill

\clearpage

\footnotesize

\ifkorrekturansicht
  \lohead{\textsc{register}}
\fi

% theindex-Environment neu definieren ohne reledmac
\makeatletter
\renewenvironment{theindex}{%
  \ifkorrekturansicht
    \section*{\indexname}%
  \else
    \subsubsection*{Index der erwähnten Entitäten}%
  \fi
  \setlength{\parindent}{0pt}%
  \setlength{\parskip}{0pt plus 0.3pt}%
  \let\item\@idxitem
}{%
  \ifkorrekturansicht\clearpage\fi
}
\makeatother

\IfFileExists{\jobname-pw.ind}{\input{\jobname-pw.ind}}{}

% Quellenangabe nur in der Leseansicht
\ifkorrekturansicht\else
% Fallback-Definitionen, falls die .tex-Datei \titel etc. nicht gesetzt hat
\providecommand{\titel}{}
\providecommand{\editorInnen}{}
\providecommand{\dateiname}{\jobname}

\vspace{3cm}

\vfill

\footnotesize
\textsc{Quelle}: \titel. Herausgegeben von {\editorInnen}. In: \emph{Arthur Schnitzler: Briefwechsel mit Autorinnen und Autoren}.
 Digitale Edition, https://schnitzler-briefe.acdh.oeaw.ac.at/{\dateiname}.html (Stand \today)
\fi

\end{document}


      