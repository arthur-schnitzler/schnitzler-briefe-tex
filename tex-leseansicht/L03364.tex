%% latex-korrekturansicht-vorspann.tex
%% Vorspann für die Korrekturansicht.
%% Lädt die gemeinsame Datei latex-vorspann.tex mit gesetztem Schalter.

\newif\ifkorrekturansicht
\korrekturansichttrue

\input{../tex-inputs/latex-vorspann}


\section[ Paul Goldmann an Arthur Schnitzler, 21. 2. {[}1903{]}]{L03364 Paul Goldmann an Arthur Schnitzler, 21. 2. {[}1903{]}}
\nopagebreak\mylabel{L03364v}
\rehead{ }\normalsize\beginnumbering\briefempfaengerindex{Schnitzler, Arthur@\textsc{Schnitzler, Arthur}!zzzGoldmann, Paul@\emph{von Paul Goldmann}!1903-02-211@{21. 2. {[}1903{]}}|(be}
\toendnotes[C]{\smallbreak\pagebreak[2]}\Standort{DLA, A:Schnitzler, HS.NZ85.1.3173.}
\physDesc{Brief, 1 Blatt, 2 Seiten, 300 Zeichen
\newline{}Handschrift: blaue Tinte, deutsche Kurrent
\newline{}Schnitzler: mit Bleistift das Jahr »903« vermerkt }\toendnotes[C]{\smallbreak}
\pstart
           \raggedleft{}{\pb}\textcolor{gray}{\textbf{DESSAUERSTRASSE 19\oindex{Dessauer Strasse@\textbf{Dessauer Straße}, \emph{Straße (K.STR)}|pw}}}\pend
           
\pstart
           Berlin\oindex{Berlin@\textbf{Berlin}, \emph{P.PPLC}|pw}, 21. Februar. \pend
           
\pstart\center{}Mein lieber Freund,\pend\vspace{0.5em}
\pstart
           Herzlichſt \label{K_L03364-1v}\edtext{willkommen}{\lemma{\textnormal{\emph{willkommen}}}\Cendnote{\textnormal{Schnitzler kam am 22. 2. 1903 morgens in
                     Berlin\oindex{Berlin@\textbf{Berlin}, \emph{P.PPLC}|pwk} an.}}}\label{K_L03364-1}! Ich muß leider zu
                  eine\textcolor{gray}{r}{ }\textsc{Matinée} ins Reſidenztheater\oindex{Residenztheater Berlin@\textbf{Residenztheater Berlin}, \emph{Theater (K.THE)}|pw} (»\textsc{Leonarda\pwindex{Leonarda@\emph{Leonarda}|pw}}« von \textsc{Björnson\pwindex{Bjørnson, Bjørnstjerne 1832-12-08 – 1910-04-26@\textsc{Bjørnson, Bjørnstjerne} (1832-12-08 – 1910-04-26), \emph{Schriftsteller/Schriftstellerin}|pw}}). Anbei ein Billet, für den Fall, daß Du \label{K_L03364-2v}\edtext{mitkommen}{\lemma{\textnormal{\emph{mitkommen}}}\Cendnote{\textnormal{Schnitzler kam nicht mit, man sah sich am
                  Nachmittag, vgl. A. S.: \emph{Tagebuch}, 22. 2. 1903.}}}\label{K_L03364-2} willſt. Wenn nicht, ſo {\pb}komme ich
                  zwiſchen 3 und \strikeout{1/}3 ½
                  Uhr zu Dir ins \textsc{Hôtel\oindex{Palasthotel Berlin@\textbf{Palasthotel Berlin}, \emph{Hotel (K.HTL)}|pwv}}. Bis 5 Uhr bin ich frei.\pend
           
\pstart
           Herzlichſt {\\[\baselineskip]}Dein {\\[\baselineskip]}\spacefill\mbox{Paul Goldm}\pend
           \leftskip=0em{}\selectlanguage{ngerman}\endnumbering\briefempfaengerindex{Schnitzler, Arthur@\textsc{Schnitzler, Arthur}!zzzGoldmann, Paul@\emph{von Paul Goldmann}!1903-02-211@{21. 2. {[}1903{]}}|)be}\mylabel{L03364h}  \normalsize

\doendnotes{C}
\bigskip
\vfill

\clearpage

\footnotesize

\lohead{\textsc{register}}

% Definiere theindex-Environment komplett neu ohne reledmac
\makeatletter
\renewenvironment{theindex}{%
  \section*{\indexname}%
  \setlength{\parindent}{0pt}%
  \setlength{\parskip}{0pt plus 0.3pt}%
  \let\item\@idxitem
}{%
  \clearpage
}
\makeatother

\IfFileExists{\jobname-pw.ind}{\input{\jobname-pw.ind}}{}

\end{document}

      