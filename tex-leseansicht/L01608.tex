%% latex-korrekturansicht-vorspann.tex
%% Vorspann für die Korrekturansicht.
%% Lädt die gemeinsame Datei latex-vorspann.tex mit gesetztem Schalter.

\newif\ifkorrekturansicht
\korrekturansichttrue

\input{../tex-inputs/latex-vorspann}


\section[Arthur Schnitzler an Georg Brandes, 11. 7. 1906]{L01608 Arthur Schnitzler an Georg Brandes, 11. 7. 1906}
\nopagebreak\mylabel{L01608v}
\rehead{ }\normalsize\beginnumbering\briefempfaengerindex{Brandes, Georg@\textsc{Brandes, Georg}!zzzSchnitzler, Arthur@\emph{von Arthur Schnitzler}!1906-07-111@{11. 7. 1906}|(be}
\toendnotes[C]{\smallbreak\pagebreak[2]}\Standort{CUL, Schnitzler, B 17-2.}
\physDesc{Postkarte, maschinenschriftliche Abschrift364 Zeichen
\newline{}Schreibmaschine}
\buchAbdrucke{\weitereDrucke{Georg Brandes, Arthur Schnitzler: \emph{Ein Briefwechsel}. Bern: \emph{Francke} 1956, S. 93.} }\toendnotes[C]{\smallbreak}
\pstart
           {\pb}26) (Ansichtskarte)\pend
           
\pstart
           \raggedleft{}11. 7. 906. Marienlyst\oindex{Marienlyst@\textbf{Marienlyst}, \emph{S.EST}|pw}\pend
           \vspace{0.5em}
\pstart
           Verehrtester Herr Brandes, heute erhalte ich eine \label{K_L01608-2v}\edtext{Karte von Brahm\pwindex{Brahm, Otto 05.02.1856 – 28.11.1912@\textsc{Brahm, Otto} (05.02.1856 – 28.11.1912), \emph{Theaterleiter/Theaterleiterin, Regisseur/Regisseurin}|pw}}{\lemma{\textnormal{\emph{Karte von Brahm}}}\Cendnote{\textnormal{Otto Brahm\pwindex{Brahm, Otto 05.02.1856 – 28.11.1912@\textsc{Brahm, Otto} (05.02.1856 – 28.11.1912), \emph{Theaterleiter/Theaterleiterin, Regisseur/Regisseurin}|pwk} an 
                     Arthur Schnitzler, 8. 7. 1906. In:
                     \emph{Der Briefwechsel Arthur Schnitzler – Otto Brahm}.
                     Vollständige Ausgabe. Herausgegeben, eingeleitet und erläutert von Oskar
                     Seidlin. Tübingen: \emph{Niemeyer}{ }1975, S. 231–232.}}}\label{K_L01608-2}, der mich bittet Sie herzlich zu grüssen
               und mir von Ihrem \label{K_L01608-1v}\edtext{Ibsen\pwindex{Ibsen, Henrik 20.03.1828 – 23.05.1906@\textsc{Ibsen, Henrik} (20.03.1828 – 23.05.1906), \emph{Schriftsteller/Schriftstellerin}|pw}-Büchlein\pwindex{Henrik Ibsen@\emph{Henrik Ibsen}|pwv}}{\lemma{\textnormal{\emph{Ibsen-Büchlein}}}\Cendnote{\textnormal{\emph{Henrik Ibsen}\pwindex{Henrik Ibsen@\emph{Henrik Ibsen}|pwk} von Georg Brandes\pwindex{Brandes, Georg 04.02.1842 – 19.02.1927@\textsc{Brandes, Georg} (04.02.1842 – 19.02.1927)|pwk}. Mit zwölf Briefen Henrik Ibsens\pwindex{Ibsen, Henrik 20.03.1828 – 23.05.1906@\textsc{Ibsen, Henrik} (20.03.1828 – 23.05.1906), \emph{Schriftsteller/Schriftstellerin}|pwk}. Siebzehn Vollbilder und vier Faksimiles.
                     Berlin: \emph{Bard Marquardt}{ }{[}1906{]} (Die Literatur, herausgegeben von Georg
                        Brandes\pwindex{Brandes, Georg 04.02.1842 – 19.02.1927@\textsc{Brandes, Georg} (04.02.1842 – 19.02.1927)|pwk}, Bd. 32–33).}}}\label{K_L01608-1} erzählt. Lassen Sie mich doch,
               wenn’s leicht geht, durch eine Zeile wissen, wie’s Ihnen geht. Mir gefällt es hier
               ausnehmend gut. Auf Wiedersehen und herzlichen Gruss. Ihr \spacefill\mbox{Arthur
                  Schnitzler}\pend
           \selectlanguage{ngerman}\endnumbering\briefempfaengerindex{Brandes, Georg@\textsc{Brandes, Georg}!zzzSchnitzler, Arthur@\emph{von Arthur Schnitzler}!1906-07-111@{11. 7. 1906}|)be}\mylabel{L01608h}  \normalsize

\doendnotes{C}
\bigskip
\vfill

\clearpage

\footnotesize

\lohead{\textsc{register}}

% Definiere theindex-Environment komplett neu ohne reledmac
\makeatletter
\renewenvironment{theindex}{%
  \section*{\indexname}%
  \setlength{\parindent}{0pt}%
  \setlength{\parskip}{0pt plus 0.3pt}%
  \let\item\@idxitem
}{%
  \clearpage
}
\makeatother

\IfFileExists{\jobname-pw.ind}{\input{\jobname-pw.ind}}{}

\end{document}

      