%% latex-leseansicht-vorspann.tex
%% Vorspann für die Leseansicht.
%% Lädt die gemeinsame Datei latex-vorspann.tex mit nicht gesetztem Schalter.

\newif\ifkorrekturansicht
\korrekturansichtfalse

\input{../tex-inputs/latex-vorspann}


         
         \renewcommand{\erwaehntePersonen}{Personen: Karl Bleibtreu, Otto Brahm, Paul Goldmann}
         \renewcommand{\erwaehnteInstitutionen}{Institutionen: Carl Krabbe, Frankfurter Zeitung, Gil Blas, L’Écho de Paris}
         \renewcommand{\erwaehnteOrte}{Orte: Brüssel, Deutschland, Frankfurt am Main, Wien}
         \renewcommand{\erwaehnteWerke}{Werke: Die drei Elixire, Dies Irae. Erinnerungen eines französischen Offiziers an die Tage von Sedan, Frankfurter Zeitung, Freie Bühne für modernes Leben}
               \section[Fedor Mamroth an Arthur Schnitzler, 21. 6. 1891]{ Fedor Mamroth an Arthur Schnitzler, 21. 6. 1891}\nopagebreak\mylabel{v}\rehead{ }\begin{ledgroupsized}[t]{13cm}\normalsize\beginnumbering \toendnotes[C]{\smallbreak\pagebreak[2]} \Standort{CUL, Schnitzler, B 68.}
\physDesc{Brief, 1 Blatt, 2 Seiten, 1062 Zeichen
\newline{}Handschrift: blaue Tinte, deutsche Kurrent
\newline{}Schnitzler: 1) mit Bleistift nummeriert: »2.«  2) mit rotem Buntstift eine Unterstreichung}\toendnotes[C]{\smallbreak}\pstart
           \noindent{}{\pb}\textcolor{gray}{\textbf{\textsc{Frankfurter Zeitung}}}{\\}\textcolor{gray}{\textbf{\textsc{und}}}{\\}\textcolor{gray}{\textbf{\textsc{Handelsblatt.}}}\orgindex{Frankfurter Zeitung@Frankfurter Zeitung|pw}\pend
           \pstart
           \textcolor{gray}{\textbf{\textsc{Redaction.}}}\hfill \textcolor{gray}{\textbf{\textsc{Frankfurt a. M.\oindex{Frankfurt am Main@\textbf{Frankfurt am Main}|pw},}}}{ }21. Juni. \textcolor{gray}{\textbf{\textsc{189}}}1\pend
           \pstart
           \textcolor{gray}{\textbf{\textsc{Telegramm-Adresse:}}}\pend
           \pstart
           \textcolor{gray}{\textbf{\textsc{Zeitung Frankfurt Main.}}}\pend
           \pstart{}Hochgeehrter Herr Doctor!\pend\pstart
           Mit aufrichtigem Vergnügen las ich Ihre »Drei
                  Elixire\pwindex{Schnitzler, Arthur 15.05.1862 – 21.10.1931@\textsc{Schnitzler, Arthur} (15.05.1862 – 21.10.1931), \emph{Schriftsteller, Mediziner}!drei Elixire1893@\strich\emph{Die drei Elixire} {[}1893{]}|pw}« und ich verſage es mir ungern, Ihnen eine Menge ſchöner Dinge
               darüber zu ſagen, weil ich in der Hauptſache weder Ihren noch meinen Wünſchen zu
               entſprechen vermag. Vermutlich wird die Frankf.
                  Ztg.\pwindex{?? Werk@Nicht ermittelte Verfasserinnen und Verfasser!Frankfurter Zeitung1856 – 1943@\emph{Frankfurter Zeitung} {[}1856 – 1943{]}|pw} im Jahre 1920 eine Arbeit dieſer Art veröffentlichen
               dürfen, ohne Straßenkämpfe hervorzurufen. Namens unſeres Publikums danke ich Ihnen
               für die Überſchätzung, die Sie ſeinem Niveau zu teil werden laſſen. Außer Brahm\pwindex{Brahm, Otto 05.02.1856 – 28.11.1912@\textsc{Brahm, Otto} (05.02.1856 – 28.11.1912), \emph{Theaterleiter, Regisseur}|pw}’s »Freier
                  Bühne\pwindex{Freie Buehne fuer modernes Leben1890 – 1891@\emph{Freie Bühne für modernes Leben} {[}1890 – 1891{]}|pw}« wüßte ich auch kein deutſches Blatt, das dieſe reizende Dichtung
               veröffentlichen könnte. Es ſei denn, Sie überſetzten ſie ins Franzöſiſche u ſchickten
               ſie dem »\textsc{Echo de Paris}\orgindex{Echo de Paris@L’Écho de Paris|pw}« oder dem »\textsc{Gil Blas}\orgindex{Gil Blas@Gil Blas|pw}«, – dann könnte ſie vielleicht \label{K_L00020-1v}\edtext{von dort aus den Weg}{\lemma{\textnormal{\emph{von dort aus den Weg}}}\Cendnote{\textnormal{Anspielung auf den
                  in Deutschland\oindex{Deutschland@\textbf{Deutschland}|pwk} kaum rezipierten Roman von Karl Bleibtreu\pwindex{Bleibtreu, Karl 13.01.1859 – 30.01.1928@\textsc{Bleibtreu, Karl} (13.01.1859 – 30.01.1928), \emph{Schriftsteller, Journalist, Übersetzer}|pwk}: \emph{Dies Irae. Erinnerungen eines französischen Offiziers an die
                        Tage von Sedan}\pwindex{Bleibtreu, Karl 13.01.1859 – 30.01.1928@\textsc{Bleibtreu, Karl} (13.01.1859 – 30.01.1928), \emph{Schriftsteller, Journalist, Übersetzer}!Dies Irae. Erinnerungen eines franzoesischen Offiziers an die Tage von Sedan1882@\strich\emph{Dies Irae. Erinnerungen eines französischen Offiziers an die Tage von Sedan} {[}1882{]}|pwk}. Stuttgart: \emph{Krabbe}\orgindex{Carl Krabbe@Carl Krabbe|pwk}{ }1882, dessen vielbeachtete französische Übersetzung für das Original gehalten
                  und ins Deutsche rückübersetzt wurde.}}}\label{K_L00020-1h}{ }{\pb}nach Deutſchland\oindex{Deutschland@\textbf{Deutschland}|pw} finden. – – –  Paul\pwindex{Goldmann, Paul 31.01.1865 – 25.09.1935@\textsc{Goldmann, Paul} (31.01.1865 – 25.09.1935), \emph{Schriftsteller, Journalist}|pw}{ }ſcheint es gut zu gehen; ſeine Privatberichte ſind
               zumeiſt ſo mißgeſti{\geminationm}t, daß ich überzeugt bin, es gefalle
               ihm in Brüſſel\oindex{Bruessel@\textbf{Brüssel}|pw} ganz ausgezeichnet. Laſſen Sie
               mich hoffen, daß es Ihnen mindeſtens ebenſo gut gehe u empfangen Sie meine
               herzlichſten Grüße.\pend
           \pstart
           Ihr ergebener{\\[\baselineskip]}\spacefill\mbox{FMamroth}\pend
           \leftskip=0em{}
         
         \endnumbering\mylabel{h}\end{ledgroupsized}  \newcommand{\dateiname}{L00020}\newcommand{\titel}{Fedor Mamroth an Arthur Schnitzler, 21. 6. 1891}\newcommand{\editorInnen}{Martin Anton Müller und Gerd-Hermann Susen}%% latex-leseansicht-abspann.tex
%% Abspann für die Leseansicht.
%% Der Schalter \ifkorrekturansicht ist bereits durch den Vorspann gesetzt.

%% latex-abspann.tex
%% Gemeinsamer Abspann für Korrekturansicht und Leseansicht.
%% Setzt den Schalter \ifkorrekturansicht voraus (gesetzt in den
%% einbindenden Dateien latex-korrekturansicht-abspann.tex bzw.
%% latex-leseansicht-abspann.tex).
%% ---------------------------------------------------------------

\normalsize

% Das esempio-Environment wird nur in der Leseansicht benötigt
\ifkorrekturansicht\else
\newenvironment{esempio}[3]%
{
    \vspace{1.5ex}
    \rlap{\underline{#1}}
    \par
    \setlength{\parindent}{0cm}
    \nopagebreak
    \leftskip=#2cm
    \rightskip=#3cm
}
{
    \par
}
\fi

\doendnotes{C}
\bigskip
\vfill

\clearpage

\footnotesize

\ifkorrekturansicht
  \lohead{\textsc{register}}
\fi

% theindex-Environment neu definieren ohne reledmac
\makeatletter
\renewenvironment{theindex}{%
  \ifkorrekturansicht
    \section*{\indexname}%
  \else
    \subsubsection*{Index der erwähnten Entitäten}%
  \fi
  \setlength{\parindent}{0pt}%
  \setlength{\parskip}{0pt plus 0.3pt}%
  \let\item\@idxitem
}{%
  \ifkorrekturansicht\clearpage\fi
}
\makeatother

\IfFileExists{\jobname-pw.ind}{\input{\jobname-pw.ind}}{}

% Quellenangabe nur in der Leseansicht
\ifkorrekturansicht\else
% Fallback-Definitionen, falls die .tex-Datei \titel etc. nicht gesetzt hat
\providecommand{\titel}{}
\providecommand{\editorInnen}{}
\providecommand{\dateiname}{\jobname}

\vspace{3cm}

\vfill

\footnotesize
\textsc{Quelle}: \titel. Herausgegeben von {\editorInnen}. In: \emph{Arthur Schnitzler: Briefwechsel mit Autorinnen und Autoren}.
 Digitale Edition, https://schnitzler-briefe.acdh.oeaw.ac.at/{\dateiname}.html (Stand \today)
\fi

\end{document}


      