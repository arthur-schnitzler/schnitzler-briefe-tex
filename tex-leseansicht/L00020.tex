%% latex-leseansicht-vorspann.tex
%% Vorspann für die Leseansicht.
%% Lädt die gemeinsame Datei latex-vorspann.tex mit nicht gesetztem Schalter.

\newif\ifkorrekturansicht
\korrekturansichtfalse

\input{../tex-inputs/latex-vorspann}


\section[Fedor Mamroth an Arthur Schnitzler, 21. 6. 1891]{L00020 Fedor Mamroth an Arthur Schnitzler, 21. 6. 1891}
\nopagebreak\mylabel{L00020v}
\rehead{ }\normalsize\beginnumbering\briefempfaengerindex{Schnitzler, Arthur@\textsc{Schnitzler, Arthur}!zzzMamroth, Fedor@\emph{von Fedor Mamroth}!1891-06-211@{21. 6. 1891}|(be}
\toendnotes[C]{\smallbreak\pagebreak[2]}
\correspDesc{Versand  durch Fedor Mamroth am 21. 6. 1891 in Frankfurt am Main
\newline{}Erhalt  durch Arthur Schnitzler im Zeitraum [22. 6. 1891
                  – 26. 6. 1891?] in Wien}\toendnotes[C]{\smallbreak}
\Standort{CUL, Schnitzler, B 68.}
\physDesc{Brief, 1 Blatt, 2 Seiten, 1062 Zeichen
\newline{}Handschrift: blaue Tinte, deutsche Kurrent
\newline{}Schnitzler: 1) mit Bleistift nummeriert: »2.«  2) mit rotem Buntstift eine Unterstreichung}\toendnotes[C]{\smallbreak}
\pstart
           {\pb}\textcolor{gray}{\textbf{\textsc{Frankfurter Zeitung}}}{\\}\textcolor{gray}{\textbf{\textsc{und}}}{\\}\textcolor{gray}{\textbf{\textsc{Handelsblatt.}}}\orgindex{Frankfurter Zeitung@Frankfurter Zeitung|pw}\pend
           
\pstart
           
\pstart
           \textcolor{gray}{\textbf{\textsc{Redaction.}}}\pend
           
\pstart
           \raggedleft{}\textcolor{gray}{\textbf{\textsc{Frankfurt a. M.\oindex{Frankfurt am Main@\textbf{Frankfurt am Main}, \emph{Hauptstadt}|pw},}}}{ }21. Juni. \textcolor{gray}{\textbf{\textsc{189}}}1\pend
           \pend
           
\pstart
           \textcolor{gray}{\textbf{\textsc{Telegramm-Adresse:}}}\pend
           
\pstart
           \textcolor{gray}{\textbf{\textsc{Zeitung Frankfurt Main.}}}\pend
           
\pstart{}Hochgeehrter Herr Doctor!\pend\vspace{0.5em}
\pstart
           Mit aufrichtigem Vergnügen las ich Ihre »Drei
                  Elixire\pwindex{Schnitzler, Arthur 15.\,5.\,1862 Wien – 21.\,10.\,1931 ebd.@\textsc{Schnitzler, Arthur} (15.\,5.\,1862 Wien – 21.\,10.\,1931 ebd.), \emph{Schriftsteller, Mediziner}!drei Elixire@\strich\emph{Die drei Elixire}|pw}« und ich verſage es mir ungern, Ihnen eine Menge{ }ſchöner Dinge
               darüber zu{ }ſagen, weil ich in der Hauptſache weder Ihren noch meinen Wünſchen zu
               entſprechen vermag. Vermutlich wird die Frankf.
                  Ztg.\pwindex{Frankfurter Zeitung@\emph{Frankfurter Zeitung}|pw} im Jahre 1920 eine Arbeit dieſer Art veröffentlichen
               dürfen, ohne Straßenkämpfe hervorzurufen. Namens unſeres Publikums danke ich Ihnen
               für die Überſchätzung, die Sie{ }ſeinem Niveau zu teil werden laſſen. Außer Brahm\pwindex{Brahm, Otto 5.\,2.\,1856 Hamburg – 28.\,11.\,1912 Berlin@\textsc{Brahm, Otto} (5.\,2.\,1856 Hamburg – 28.\,11.\,1912 Berlin), \emph{Theaterleiter, Regisseur}|pw}’s »Freier
                  Bühne\pwindex{Freie Bühne für modernes Leben@\emph{Freie Bühne für modernes Leben}|pw}« wüßte ich auch kein deutſches Blatt, das dieſe reizende Dichtung
               veröffentlichen könnte. Es{ }ſei denn, Sie überſetzten{ }ſie ins Franzöſiſche u{ }ſchickten{ }ſie dem »\textsc{Echo de Paris}\orgindex{Écho de Paris@L’Écho de Paris|pw}« oder dem »\textsc{Gil Blas}\orgindex{Gil Blas@Gil Blas|pw}«, – dann könnte{ }ſie vielleicht \label{K_L00020-1v}\edtext{von dort aus den Weg}{\lemma{\textnormal{\emph{von dort aus den Weg}}}\Cendnote{\textnormal{Anspielung auf den
                  in Deutschland\oindex{Deutschland@\textbf{Deutschland}|pwk} kaum rezipierten Roman von Karl Bleibtreu\pwindex{Bleibtreu, Karl 13.\,1.\,1859 Berlin – 30.\,1.\,1928 Locarno@\textsc{Bleibtreu, Karl} (13.\,1.\,1859 Berlin – 30.\,1.\,1928 Locarno), \emph{Schriftsteller, Journalist, Übersetzer}|pwk}: \emph{Dies Irae. Erinnerungen eines französischen Offiziers an die
                        Tage von Sedan}\pwindex{Bleibtreu, Karl 13.\,1.\,1859 Berlin – 30.\,1.\,1928 Locarno@\textsc{Bleibtreu, Karl} (13.\,1.\,1859 Berlin – 30.\,1.\,1928 Locarno), \emph{Schriftsteller, Journalist, Übersetzer}!Dies Irae. Erinnerungen eines französischen Offiziers an die Tage von Sedan@\strich\emph{Dies Irae. Erinnerungen eines französischen Offiziers an die Tage von Sedan}|pwk}. Stuttgart: \emph{Krabbe}\orgindex{Carl Krabbe@Carl Krabbe|pwk}{ }1882, dessen vielbeachtete französische Übersetzung für das Original gehalten
                  und ins Deutsche rückübersetzt wurde.}}}\label{K_L00020-1}{ }{\pb}nach Deutſchland\oindex{Deutschland@\textbf{Deutschland}|pw} finden. – – –  Paul\pwindex{Goldmann, Paul 31.\,1.\,1865 Breslau – 25.\,9.\,1935 Wien@\textsc{Goldmann, Paul} (31.\,1.\,1865 Breslau – 25.\,9.\,1935 Wien), \emph{Schriftsteller, Journalist}|pw}{ }ſcheint es gut zu gehen;{ }ſeine Privatberichte{ }ſind
               zumeiſt{ }ſo mißgeſti{\geminationm}t, daß ich überzeugt bin, es gefalle
               ihm in Brüſſel\oindex{Brüssel@\textbf{Brüssel}, \emph{Hauptstadt}|pw} ganz ausgezeichnet. Laſſen Sie
               mich hoffen, daß es Ihnen mindeſtens ebenſo gut gehe u empfangen Sie meine
               herzlichſten Grüße.\pend
           
\pstart
           Ihr ergebener{\\[\baselineskip]}\spacefill\mbox{FMamroth}\pend
           \leftskip=0em{}\selectlanguage{ngerman}\endnumbering\briefempfaengerindex{Schnitzler, Arthur@\textsc{Schnitzler, Arthur}!zzzMamroth, Fedor@\emph{von Fedor Mamroth}!1891-06-211@{21. 6. 1891}|)be}\mylabel{L00020h}  \newcommand{\dateiname}{L00020}\newcommand{\titel}{Fedor Mamroth an Arthur Schnitzler, 21. 6. 1891}\newcommand{\editorInnen}{Martin Anton Müller und Gerd-Hermann Susen}%% latex-leseansicht-abspann.tex
%% Abspann für die Leseansicht.
%% Der Schalter \ifkorrekturansicht ist bereits durch den Vorspann gesetzt.

%% latex-abspann.tex
%% Gemeinsamer Abspann für Korrekturansicht und Leseansicht.
%% Setzt den Schalter \ifkorrekturansicht voraus (gesetzt in den
%% einbindenden Dateien latex-korrekturansicht-abspann.tex bzw.
%% latex-leseansicht-abspann.tex).
%% ---------------------------------------------------------------

\normalsize

% Das esempio-Environment wird nur in der Leseansicht benötigt
\ifkorrekturansicht\else
\newenvironment{esempio}[3]%
{
    \vspace{1.5ex}
    \rlap{\underline{#1}}
    \par
    \setlength{\parindent}{0cm}
    \nopagebreak
    \leftskip=#2cm
    \rightskip=#3cm
}
{
    \par
}
\fi

\doendnotes{C}
\bigskip
\vfill

\clearpage

\footnotesize

\ifkorrekturansicht
  \lohead{\textsc{register}}
\fi

% theindex-Environment neu definieren ohne reledmac
\makeatletter
\renewenvironment{theindex}{%
  \ifkorrekturansicht
    \section*{\indexname}%
  \else
    \subsubsection*{Index der erwähnten Entitäten}%
  \fi
  \setlength{\parindent}{0pt}%
  \setlength{\parskip}{0pt plus 0.3pt}%
  \let\item\@idxitem
}{%
  \ifkorrekturansicht\clearpage\fi
}
\makeatother

\IfFileExists{\jobname-pw.ind}{\input{\jobname-pw.ind}}{}

% Quellenangabe nur in der Leseansicht
\ifkorrekturansicht\else
% Fallback-Definitionen, falls die .tex-Datei \titel etc. nicht gesetzt hat
\providecommand{\titel}{}
\providecommand{\editorInnen}{}
\providecommand{\dateiname}{\jobname}

\vspace{3cm}

\vfill

\footnotesize
\textsc{Quelle}: \titel. Herausgegeben von {\editorInnen}. In: \emph{Arthur Schnitzler: Briefwechsel mit Autorinnen und Autoren}.
 Digitale Edition, https://schnitzler-briefe.acdh.oeaw.ac.at/{\dateiname}.html (Stand \today)
\fi

\end{document}


