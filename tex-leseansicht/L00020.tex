%% latex-korrekturansicht-vorspann.tex
%% Vorspann für die Korrekturansicht.
%% Lädt die gemeinsame Datei latex-vorspann.tex mit gesetztem Schalter.

\newif\ifkorrekturansicht
\korrekturansichttrue

\input{../tex-inputs/latex-vorspann}


\section[Fedor Mamroth an Arthur Schnitzler, 21. 6. 1891]{L00020 Fedor Mamroth an Arthur Schnitzler, 21. 6. 1891}
\nopagebreak\mylabel{L00020v}
\rehead{ }\normalsize\beginnumbering\briefempfaengerindex{Schnitzler, Arthur@\textsc{Schnitzler, Arthur}!zzzMamroth, Fedor@\emph{von Fedor Mamroth}!1891-06-211@{21. 6. 1891}|(be}
\toendnotes[C]{\smallbreak\pagebreak[2]}\Standort{CUL, Schnitzler, B 68.}
\physDesc{Brief, 1 Blatt, 2 Seiten, 1062 Zeichen
\newline{}Handschrift: blaue Tinte, deutsche Kurrent
\newline{}Schnitzler: 1) mit Bleistift nummeriert: »2.«  2) mit rotem Buntstift eine Unterstreichung}\toendnotes[C]{\smallbreak}
\pstart
           {\pb}\textcolor{gray}{\textbf{\textsc{Frankfurter Zeitung}}}{\\}\textcolor{gray}{\textbf{\textsc{und}}}{\\}\textcolor{gray}{\textbf{\textsc{Handelsblatt.}}}\orgindex{Frankfurter Zeitung@Frankfurter Zeitung|pw}\pend
           
\pstart
           
\pstart
           \textcolor{gray}{\textbf{\textsc{Redaction.}}}\pend
           
\pstart
           \raggedleft{}\textcolor{gray}{\textbf{\textsc{Frankfurt a. M.\oindex{Frankfurt am Main@\textbf{Frankfurt am Main}, \emph{P.PPLA3}|pw},}}}{ }21. Juni. \textcolor{gray}{\textbf{\textsc{189}}}1\pend
           \pend
           
\pstart
           \textcolor{gray}{\textbf{\textsc{Telegramm-Adresse:}}}\pend
           
\pstart
           \textcolor{gray}{\textbf{\textsc{Zeitung Frankfurt Main.}}}\pend
           
\pstart{}Hochgeehrter Herr Doctor!\pend\vspace{0.5em}
\pstart
           Mit aufrichtigem Vergnügen las ich Ihre »Drei
                  Elixire\pwindex{drei Elixire@\emph{Die drei Elixire}|pw}« und ich verſage es mir ungern, Ihnen eine Menge ſchöner Dinge
               darüber zu ſagen, weil ich in der Hauptſache weder Ihren noch meinen Wünſchen zu
               entſprechen vermag. Vermutlich wird die Frankf.
                  Ztg.\pwindex{Frankfurter Zeitung@\emph{Frankfurter Zeitung}|pw} im Jahre 1920 eine Arbeit dieſer Art veröffentlichen
               dürfen, ohne Straßenkämpfe hervorzurufen. Namens unſeres Publikums danke ich Ihnen
               für die Überſchätzung, die Sie ſeinem Niveau zu teil werden laſſen. Außer Brahm\pwindex{Brahm, Otto 05.02.1856 – 28.11.1912@\textsc{Brahm, Otto} (05.02.1856 – 28.11.1912), \emph{Theaterleiter/Theaterleiterin, Regisseur/Regisseurin}|pw}’s »Freier
                  Bühne\pwindex{Freie Buehne fuer modernes Leben@\emph{Freie Bühne für modernes Leben}|pw}« wüßte ich auch kein deutſches Blatt, das dieſe reizende Dichtung
               veröffentlichen könnte. Es ſei denn, Sie überſetzten ſie ins Franzöſiſche u ſchickten
               ſie dem »\textsc{Echo de Paris}\orgindex{Echo de Paris@L’Écho de Paris|pw}« oder dem »\textsc{Gil Blas}\orgindex{Gil Blas@Gil Blas|pw}«, – dann könnte ſie vielleicht \label{K_L00020-1v}\edtext{von dort aus den Weg}{\lemma{\textnormal{\emph{von dort aus den Weg}}}\Cendnote{\textnormal{Anspielung auf den
                  in Deutschland\oindex{Deutschland@\textbf{Deutschland}, \emph{A.PCLI}|pwk} kaum rezipierten Roman von Karl Bleibtreu\pwindex{Bleibtreu, Karl 13.01.1859 – 30.01.1928@\textsc{Bleibtreu, Karl} (13.01.1859 – 30.01.1928), \emph{Schriftsteller/Schriftstellerin, Journalist/Journalistin, Übersetzer/Übersetzerin}|pwk}: \emph{Dies Irae. Erinnerungen eines französischen Offiziers an die
                        Tage von Sedan}\pwindex{Dies Irae. Erinnerungen eines franzoesischen Offiziers an die Tage von Sedan@\emph{Dies Irae. Erinnerungen eines französischen Offiziers an die Tage von Sedan}|pwk}. Stuttgart: \emph{Krabbe}\orgindex{Carl Krabbe@Carl Krabbe|pwk}{ }1882, dessen vielbeachtete französische Übersetzung für das Original gehalten
                  und ins Deutsche rückübersetzt wurde.}}}\label{K_L00020-1}{ }{\pb}nach Deutſchland\oindex{Deutschland@\textbf{Deutschland}, \emph{A.PCLI}|pw} finden. – – –  Paul\pwindex{Goldmann, Paul 31.01.1865 – 25.09.1935@\textsc{Goldmann, Paul} (31.01.1865 – 25.09.1935), \emph{Schriftsteller/Schriftstellerin, Journalist/Journalistin}|pw}{ }ſcheint es gut zu gehen; ſeine Privatberichte ſind
               zumeiſt ſo mißgeſti{\geminationm}t, daß ich überzeugt bin, es gefalle
               ihm in Brüſſel\oindex{Bruessel@\textbf{Brüssel}, \emph{P.PPLC}|pw} ganz ausgezeichnet. Laſſen Sie
               mich hoffen, daß es Ihnen mindeſtens ebenſo gut gehe u empfangen Sie meine
               herzlichſten Grüße.\pend
           
\pstart
           Ihr ergebener{\\[\baselineskip]}\spacefill\mbox{FMamroth}\pend
           \leftskip=0em{}\selectlanguage{ngerman}\endnumbering\briefempfaengerindex{Schnitzler, Arthur@\textsc{Schnitzler, Arthur}!zzzMamroth, Fedor@\emph{von Fedor Mamroth}!1891-06-211@{21. 6. 1891}|)be}\mylabel{L00020h}  \normalsize

\doendnotes{C}
\bigskip
\vfill

\clearpage

\footnotesize

\lohead{\textsc{register}}

% Definiere theindex-Environment komplett neu ohne reledmac
\makeatletter
\renewenvironment{theindex}{%
  \section*{\indexname}%
  \setlength{\parindent}{0pt}%
  \setlength{\parskip}{0pt plus 0.3pt}%
  \let\item\@idxitem
}{%
  \clearpage
}
\makeatother

\IfFileExists{\jobname-pw.ind}{\input{\jobname-pw.ind}}{}

\end{document}

      