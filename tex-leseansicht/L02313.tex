%% latex-leseansicht-vorspann.tex
%% Vorspann für die Leseansicht.
%% Lädt die gemeinsame Datei latex-vorspann.tex mit nicht gesetztem Schalter.

\newif\ifkorrekturansicht
\korrekturansichtfalse

\input{../tex-inputs/latex-vorspann}


         
         \renewcommand{\erwaehntePersonen}{Personen: Hugo von Hofmannsthal, Gertrude von Hofmannsthal, Frieda Pollak, Olga Schnitzler}
         \renewcommand{\erwaehnteOrte}{Orte: Hofmannsthal-Schlössl, Stallburggasse, Wien}
         \renewcommand{\erwaehnteWerke}{Werke: Professor Bernhardi. Komödie in fünf Akten}
               \section[Hugo von Hofmannsthal an Arthur Schnitzler, {[}Anfang Dezember 1918{]}]{ Hugo von Hofmannsthal an Arthur Schnitzler, {[}Anfang Dezember
               1918{]}}\nopagebreak\mylabel{v}\rehead{ }\begin{ledgroupsized}[t]{13cm}\normalsize\beginnumbering\briefempfaengerindex{Schnitzler, Arthur@\textsc{Schnitzler, Arthur}!zzzHofmannsthal, Hugo von@\emph{von Hugo von Hofmannsthal}!1918-12-071@{{[}Anfang Dezember
                  1918{]}}|(be} \toendnotes[C]{\smallbreak\pagebreak[2]} \Standort{CUL, Schnitzler, B 43.}
\physDesc{Brief, 1 Blatt, 1 Seite, 1112 Zeichen
\newline{}Handschrift: schwarze Tinte, deutsche Kurrent
\newline{}Schnitzler: 1) mit Bleistift datiert: »Anf Dez. 918« und beschriftet: »\textsc{Hugo}«  2) mit rotem Buntstift eine Unterstreichung
\newline{}Ordnung: 1) mit Bleistift von Frieda
                                    Pollak\pwindex{Pollak, Frieda 08.12.1881 – 13.07.1937@\textsc{Pollak, Frieda} (08.12.1881 – 13.07.1937), \emph{Sekretärin}|pw} (?) mit dem Buchstaben »A«
                                 (Abgeschrieben/Abschrift) gekennzeichnet  2) mit Bleistift von unbekannter Hand nummeriert: »\strikeout{351}« 3) mit Bleistift von unbekannter Hand nummeriert:
                                    »360«}\buchAbdrucke{\weitereDrucke{Hugo von Hofmannsthal, Arthur Schnitzler: \emph{Briefwechsel}. Hg. Therese Nickl und Heinrich Schnitzler. Frankfurt am Main: \emph{S. Fischer} 1964, S. 288.} }\toendnotes[C]{\smallbreak}\pstart
           \raggedleft{}{\pb}Wien\oindex{Wien@\textbf{Wien}|pw}{\\}Stallburggaſſe 2\oindex{Stallburggasse@\textbf{Stallburggasse}|pw}\pend
           \pstart{}mein lieber Arthur \pend\pstart
           ſeit mehr als 10 Tagen ſind wir ganz herinnen, Gerty\pwindex{Hofmannsthal, Gertrude von 16.03.1880 – 09.11.1959@\textsc{Hofmannsthal, Gertrude von} (16.03.1880 – 09.11.1959)|pw} ist hier krank geworden, befindet ſich aber ſchon wieder wohl und
                  Sonntag werden wir für einige Zeit wieder hinausziehen, doch läſst
               ſich draußen in einem finſteren und kaum über {\pb}11° heizbaren Haus\oindex{Hofmannsthal-Schloessl@\textbf{Hofmannsthal-Schlössl}|pwv} mehr vegetieren als leben.\hspace*{1.5em}– Aber nicht davon wollte ich ſprechen ſondern ſagen
               daſs ich Sie und Olga\pwindex{Schnitzler, Olga 17.01.1882 – 13.01.1970@\textsc{Schnitzler, Olga} (17.01.1882 – 13.01.1970), \emph{Schauspielerin, Sängerin}|pw} unendlich gern ſehen
               möchte und in dieſen Tagen durch wiederholtes Anrufen vergeblich dies zu betätigen
               verſucht habe. Ich wollte anfragen ob ich eines Vormittags zu Ihnen hinausko{\geminationm}en könnte! Indeſſen hab ich aber gehört daſs Sie {\pb}Proben zum Profeſſor Bernhardi\pwindex{Schnitzler, Arthur 15.05.1862 – 21.10.1931@\textsc{Schnitzler, Arthur} (15.05.1862 – 21.10.1931), \emph{Schriftsteller, Mediziner}!Professor Bernhardi. Komoedie in fuenf Akten1912@\strich\emph{Professor Bernhardi. Komödie in fünf Akten} {[}1912{]}|pw} haben – ſo nehme ich an daſs Ihre Vormittage
               beſetzt ſind und zwar wie ich hoffe in einer Weiſe die Sie über das halb Gräſsliche
               halb Läppiſche das uns umgibt einigermaßen hinaushebt wofür ich Sie gewiſſermaßen
               beneide.\pend
           \pstart
           Bitte wenn das vorbei iſt, {\pb}ſo
               ſchreiben Sie mir eine Zeile und vielleicht ko{\geminationm}t Ihr
               dann endlich einmal in die Stallburggaſſe\oindex{Stallburggasse@\textbf{Stallburggasse}|pw}, etwa
               mit einem Concert es verbindend – oder wenn Ihnen das lieber ist, ſo ko{\geminationm}e ich hinaus.\hspace*{1.5em}Ihnen
               und Olga\pwindex{Schnitzler, Olga 17.01.1882 – 13.01.1970@\textsc{Schnitzler, Olga} (17.01.1882 – 13.01.1970), \emph{Schauspielerin, Sängerin}|pw} alles Liebe \pend
           \pstart
           von Ihrem{\\[\baselineskip]}\spacefill\mbox{Hugo.}\pend
           \leftskip=0em{}
         
         \endnumbering\mylabel{h}\end{ledgroupsized}  \newcommand{\dateiname}{L02313}\newcommand{\titel}{Hugo von Hofmannsthal an Arthur Schnitzler, [Anfang Dezember 1918]}\newcommand{\editorInnen}{Martin Anton Müller und Gerd-Hermann Susen}%% latex-leseansicht-abspann.tex
%% Abspann für die Leseansicht.
%% Der Schalter \ifkorrekturansicht ist bereits durch den Vorspann gesetzt.

%% latex-abspann.tex
%% Gemeinsamer Abspann für Korrekturansicht und Leseansicht.
%% Setzt den Schalter \ifkorrekturansicht voraus (gesetzt in den
%% einbindenden Dateien latex-korrekturansicht-abspann.tex bzw.
%% latex-leseansicht-abspann.tex).
%% ---------------------------------------------------------------

\normalsize

% Das esempio-Environment wird nur in der Leseansicht benötigt
\ifkorrekturansicht\else
\newenvironment{esempio}[3]%
{
    \vspace{1.5ex}
    \rlap{\underline{#1}}
    \par
    \setlength{\parindent}{0cm}
    \nopagebreak
    \leftskip=#2cm
    \rightskip=#3cm
}
{
    \par
}
\fi

\doendnotes{C}
\bigskip
\vfill

\clearpage

\footnotesize

\ifkorrekturansicht
  \lohead{\textsc{register}}
\fi

% theindex-Environment neu definieren ohne reledmac
\makeatletter
\renewenvironment{theindex}{%
  \ifkorrekturansicht
    \section*{\indexname}%
  \else
    \subsubsection*{Index der erwähnten Entitäten}%
  \fi
  \setlength{\parindent}{0pt}%
  \setlength{\parskip}{0pt plus 0.3pt}%
  \let\item\@idxitem
}{%
  \ifkorrekturansicht\clearpage\fi
}
\makeatother

\IfFileExists{\jobname-pw.ind}{\input{\jobname-pw.ind}}{}

% Quellenangabe nur in der Leseansicht
\ifkorrekturansicht\else
% Fallback-Definitionen, falls die .tex-Datei \titel etc. nicht gesetzt hat
\providecommand{\titel}{}
\providecommand{\editorInnen}{}
\providecommand{\dateiname}{\jobname}

\vspace{3cm}

\vfill

\footnotesize
\textsc{Quelle}: \titel. Herausgegeben von {\editorInnen}. In: \emph{Arthur Schnitzler: Briefwechsel mit Autorinnen und Autoren}.
 Digitale Edition, https://schnitzler-briefe.acdh.oeaw.ac.at/{\dateiname}.html (Stand \today)
\fi

\end{document}


      