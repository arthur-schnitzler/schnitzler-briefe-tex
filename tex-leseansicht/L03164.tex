%% latex-leseansicht-vorspann.tex
%% Vorspann für die Leseansicht.
%% Lädt die gemeinsame Datei latex-vorspann.tex mit nicht gesetztem Schalter.

\newif\ifkorrekturansicht
\korrekturansichtfalse

\input{../tex-inputs/latex-vorspann}

\begin{center}
            \textcolor{red}{ENTWURF, NICHT FERTIG KORRIGIERT}
                      \end{center}
            
         
         \renewcommand{\erwaehntePersonen}{Personen: Lou Andreas-Salomé, Richard Beer-Hofmann, Siegfried Bing, Emma Fr., Vincent van Gogh, Paul Goldmann, Franz Irresberger, Karl Kraus, Charlotte Pohl-Glas, Adele Sandrock}
         \renewcommand{\erwaehnteOrte}{Orte: Asien, Bad Ischl, Gmunden, Japan, München, Paris, Salzburg, Wien, Österreichischer Hof}
         \renewcommand{\erwaehnteWerke}{Werke: Ischler Brief. (Wiener Dichter auf der Esplanade.), Wiener Familien-Journal}
               \section[Felix Salten an Arthur Schnitzler, 30. 8. 1895]{ Felix Salten an Arthur Schnitzler, 30. 8. 1895}\nopagebreak\mylabel{v}\rehead{ }\begin{ledgroupsized}[t]{13cm}\normalsize\beginnumbering \toendnotes[C]{\smallbreak\pagebreak[2]} \Standort{CUL, Schnitzler, B 89, A 1.}
\physDesc{Brief, 1 Blatt, 3 Seiten, 1784 Zeichen
\newline{}Handschrift: schwarze Tinte, lateinische Kurrent
\newline{}Ordnung: mit Bleistift von unbekannter Hand nummeriert:
                                    »64« }\toendnotes[C]{\smallbreak}\pstart
           \noindent{}\centering{}{\pb}\textcolor{gray}{\textbf{\textsc{Hôtel oesterreichischer Hof\oindex{Oesterreichischer Hof@\textbf{Österreichischer Hof}|pw}}}}\pend
           \pstart
           \noindent{}\centering{}\textcolor{gray}{\textbf{Franz Irresberger\pwindex{Irresberger, Franz 1862 – 1929-07-14@\textsc{Irresberger, Franz} (1862 – 1929-07-14), \emph{Hotelier}|pw}}}\pend
           \pstart
           \raggedleft{}30. VIII. 95\pend
           \pstart
           Lieber Freund, ich habe bei meiner Ankunft nur \uline{die Hälfte} des so bestimmt erwarteten Betrages erhalten und auf meine
               telegrafische Urgenz ist bis jetzt noch nichts eingelangt, so dass ich wegen der
               Rückreise selbst in arger Verlegenheit bin. Seien Sie mir deshalb nicht böse, wenn in
               der Sache eine Verzögerung von einigen Tagen eintritt, ich empfinde das ohnehin
               peinlich genug, und leide darunter, dass auch unsere 2\textsuperscript{te}
               Bicycle tour mit einem solchen Nachspiel endet. Sollte ich aber heute oder
                  morgen noch das erhoffte bekommen, dann sende ich es \uline{Ihnen} sofort, wo nicht, \uline{gleich} nach meiner Rückkehr nach Wien\oindex{Wien@\textbf{Wien}|pw}.
               Das ist ganz sicher. \pend
           \pstart
           L.\pwindex{Pohl-Glas, Charlotte 1873-01-01 – 1944-02-15@\textsc{Pohl-Glas, Charlotte} (1873-01-01 – 1944-02-15), \emph{Schriftstellerin, Politikerin, Sozialistin}|pw} kam hier an voll Erbitterung und ich lebe
               schwere Tage. Irgend {\pb}ein
               Mensch – wer, das bringe ich noch nicht heraus, – hat ihr in Gmunden\oindex{Gmunden@\textbf{Gmunden}|pw} oder Ischl\oindex{Bad Ischl@\textbf{Bad Ischl}|pw} erzählt,
               dass ich das erste mal in Ischl\oindex{Bad Ischl@\textbf{Bad Ischl}|pw} war. Ferner, dass
               ich voriges Jahr, als sie hieherkam, auch in Ischl\oindex{Bad Ischl@\textbf{Bad Ischl}|pw} gewesen, hat ihr sonst allerhand Geschichten von Frau Fr.\pwindex{Fr., Emma @\textsc{Fr., Emma}|pw} ferner von Frl. S.\pwindex{Sandrock, Adele 1863-08-19 – 1937-08-30@\textsc{Sandrock, Adele} (1863-08-19 – 1937-08-30), \emph{Schauspielerin}|pw} erzählt, – kurz Sie können sich denken, wie das arme Mädel
               zugerichtet war. So hatte ich hier zu thun und habe es noch, um alles wieder ins
               Gleichgewicht zu bringen. \pend
           \pstart
           Außerdem hat man ihr erzählt, wir seien in Salzburg\oindex{Salzburg@\textbf{Salzburg}|pw} mit einer »jungen chicken Blondine\pwindex{Andreas-Salome, Lou 12.02.1861 – 05.02.1937@\textsc{Andreas-Salomé, Lou} (12.02.1861 – 05.02.1937), \emph{Schriftstellerin}|pwuv}« »umhergelaufen«. Dass sie mir viel
               Tratsch über Sie, Beer-Hofmann\pwindex{Beer-Hofmann, Richard 1866-07-11 – 1945-09-26@\textsc{Beer-Hofmann, Richard} (1866-07-11 – 1945-09-26), \emph{Schriftsteller}|pw} und mich
               mitgebracht, gehört {\pb}wol mit
               dazu. Von Kraus\pwindex{Kraus, Karl 28.04.1874 – 12.06.1936@\textsc{Kraus, Karl} (28.04.1874 – 12.06.1936), \emph{Schriftsteller, Publizist}|pw} ist im Familien-Journal\pwindex{?? Werk@Nicht ermittelte Verfasserinnen und Verfasser!Wiener Familien-JournalNone@\emph{Wiener Familien-Journal} {[}None{]}|pw} eine Geschichte erschienen, »\label{K_L03164-1v}\edtext{Esplanade Dichter\pwindex{Ischler Brief. (Wiener Dichter auf der Esplanade.)1895-08-23@\emph{Ischler Brief. (Wiener Dichter auf der Esplanade.)} {[}1895-08-23{]}|pwv}}{\lemma{\textnormal{\emph{Esplanade Dichter}}}\Cendnote{\textnormal{Crêpedechine\pwindex{Kraus, Karl 28.04.1874 – 12.06.1936@\textsc{Kraus, Karl} (28.04.1874 – 12.06.1936), \emph{Schriftsteller, Publizist}|pwk} [ = Karl Kraus\pwindex{Kraus, Karl 28.04.1874 – 12.06.1936@\textsc{Kraus, Karl} (28.04.1874 – 12.06.1936), \emph{Schriftsteller, Publizist}|pwk}: \emph{Ischler
                        Brief. (Wiener Dichter auf der Esplanade.)}\pwindex{Ischler Brief. (Wiener Dichter auf der Esplanade.)1895-08-23@\emph{Ischler Brief. (Wiener Dichter auf der Esplanade.)} {[}1895-08-23{]}|pwk}. In: \emph{Wiener Familien-Journal}\pwindex{?? Werk@Nicht ermittelte Verfasserinnen und Verfasser!Wiener Familien-JournalNone@\emph{Wiener Familien-Journal} {[}None{]}|pwk}, Nr. 230, 23. 8. 1895, S. 914–915. Während die
                  satirischen Bemerkungen über Beer Hofmann\pwindex{Beer-Hofmann, Richard 1866-07-11 – 1945-09-26@\textsc{Beer-Hofmann, Richard} (1866-07-11 – 1945-09-26), \emph{Schriftsteller}|pwk}
                     (»ein junger Dichter, der die beſten Erfolge auf dem Gebiete der Mode
                     aufzuweiſen hat«) und Hofmannsthal\pwindex{\textcolor{red}{\textsuperscript{XXXX1 indx}}|pwk} (»{[}e{]}in Wien\oindex{Wien@\textbf{Wien}|pw}er Dichter, der den Schulſchluß abwarten muß, um nach
                        Iſchl\oindex{Bad Ischl@\textbf{Bad Ischl}|pw} gehen zu können«) gut
                  zuordenbar scheinen, lässt sich im Text keine unzweifelhafte Spitze gegen Schnitzler\pwindex{Schnitzler, Arthur 15.05.1862 – 21.10.1931@\textsc{Schnitzler, Arthur} (15.05.1862 – 21.10.1931), \emph{Schriftsteller, Mediziner}|pwk} ausmachen.}}}\label{K_L03164-1h}«, das sind Beer Hofmann\pwindex{Beer-Hofmann, Richard 1866-07-11 – 1945-09-26@\textsc{Beer-Hofmann, Richard} (1866-07-11 – 1945-09-26), \emph{Schriftsteller}|pw} und Sie, und sollen »Eure
               Affectationen und Posen« darin mit vielem Witz »gegeißelt« worden sein. Ich habs
               nicht gesehen. \pend
           \pstart
           Bitte, sagen Sie an Hr. D\textsuperscript{r.}{ }Goldmann\pwindex{Goldmann, Paul 31.01.1865 – 25.09.1935@\textsc{Goldmann, Paul} (31.01.1865 – 25.09.1935), \emph{Schriftsteller, Journalist}|pw}, er möge Ihnen die Adresse von
                  \label{K_L03164-11v}\edtext{Bing\pwindex{Bing, Siegfried 1838-02-26 – 1905-09-06@\textsc{Bing, Siegfried} (1838-02-26 – 1905-09-06), \emph{Japanologe, Kunstsammler, Kunsthändler}|pw}}{\lemma{\textnormal{\emph{Bing}}}\Cendnote{\textnormal{Gemeint dürfte der in Paris\oindex{Paris@\textbf{Paris}|pwk} lebende Kunsthändler Siegfried Bing\pwindex{Bing, Siegfried 1838-02-26 – 1905-09-06@\textsc{Bing, Siegfried} (1838-02-26 – 1905-09-06), \emph{Japanologe, Kunstsammler, Kunsthändler}|pwk} sein, der sich auf japanische\oindex{Japan@\textbf{Japan}|pwk} und asiatische\oindex{Asien@\textbf{Asien}|pwk} Kunst spezialisiert hatte. Vincent van Gogh\pwindex{Gogh, Vincent van 30.03.1853 – 29.07.1890@\textsc{Gogh, Vincent van} (30.03.1853 – 29.07.1890), \emph{Bildender Künstler}|pwk} frequentierte seine Sammlung.}}}\label{K_L03164-11h} oder Bingen\pwindex{Bing, Siegfried 1838-02-26 – 1905-09-06@\textsc{Bing, Siegfried} (1838-02-26 – 1905-09-06), \emph{Japanologe, Kunstsammler, Kunsthändler}|pw}, das ist der Japaner\oindex{Japan@\textbf{Japan}|pwv}, mittheilen, und schreiben Sie mir
               nach Wien\oindex{Wien@\textbf{Wien}|pw}, wo\strikeout{hin}
               ich ohnedies bald einen Brief von Ihnen erwarte. \pend
           \pstart
           Mit vielen Empfehlungen an D\textsuperscript{r}{ }G.\pwindex{Goldmann, Paul 31.01.1865 – 25.09.1935@\textsc{Goldmann, Paul} (31.01.1865 – 25.09.1935), \emph{Schriftsteller, Journalist}|pw} herzlichst \pend
           \pstart
           Ihr {\\[\baselineskip]}\spacefill\mbox{Salten}\pend
           \leftskip=0em{}
         
         \endnumbering\mylabel{h}\end{ledgroupsized}\begin{anhang}\end{anhang}\newcommand{\dateiname}{L03164}\newcommand{\titel}{Felix Salten an Arthur Schnitzler, 30. 8. 1895}\newcommand{\editorInnen}{Martin Anton Müller und Laura Untner}%% latex-leseansicht-abspann.tex
%% Abspann für die Leseansicht.
%% Der Schalter \ifkorrekturansicht ist bereits durch den Vorspann gesetzt.

%% latex-abspann.tex
%% Gemeinsamer Abspann für Korrekturansicht und Leseansicht.
%% Setzt den Schalter \ifkorrekturansicht voraus (gesetzt in den
%% einbindenden Dateien latex-korrekturansicht-abspann.tex bzw.
%% latex-leseansicht-abspann.tex).
%% ---------------------------------------------------------------

\normalsize

% Das esempio-Environment wird nur in der Leseansicht benötigt
\ifkorrekturansicht\else
\newenvironment{esempio}[3]%
{
    \vspace{1.5ex}
    \rlap{\underline{#1}}
    \par
    \setlength{\parindent}{0cm}
    \nopagebreak
    \leftskip=#2cm
    \rightskip=#3cm
}
{
    \par
}
\fi

\doendnotes{C}
\bigskip
\vfill

\clearpage

\footnotesize

\ifkorrekturansicht
  \lohead{\textsc{register}}
\fi

% theindex-Environment neu definieren ohne reledmac
\makeatletter
\renewenvironment{theindex}{%
  \ifkorrekturansicht
    \section*{\indexname}%
  \else
    \subsubsection*{Index der erwähnten Entitäten}%
  \fi
  \setlength{\parindent}{0pt}%
  \setlength{\parskip}{0pt plus 0.3pt}%
  \let\item\@idxitem
}{%
  \ifkorrekturansicht\clearpage\fi
}
\makeatother

\IfFileExists{\jobname-pw.ind}{\input{\jobname-pw.ind}}{}

% Quellenangabe nur in der Leseansicht
\ifkorrekturansicht\else
% Fallback-Definitionen, falls die .tex-Datei \titel etc. nicht gesetzt hat
\providecommand{\titel}{}
\providecommand{\editorInnen}{}
\providecommand{\dateiname}{\jobname}

\vspace{3cm}

\vfill

\footnotesize
\textsc{Quelle}: \titel. Herausgegeben von {\editorInnen}. In: \emph{Arthur Schnitzler: Briefwechsel mit Autorinnen und Autoren}.
 Digitale Edition, https://schnitzler-briefe.acdh.oeaw.ac.at/{\dateiname}.html (Stand \today)
\fi

\end{document}


      