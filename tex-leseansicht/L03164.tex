%% latex-korrekturansicht-vorspann.tex
%% Vorspann für die Korrekturansicht.
%% Lädt die gemeinsame Datei latex-vorspann.tex mit gesetztem Schalter.

\newif\ifkorrekturansicht
\korrekturansichttrue

\input{../tex-inputs/latex-vorspann}


\section[ Felix Salten an Arthur Schnitzler, 30. 8. 1895]{L03164 Felix Salten an Arthur Schnitzler, 30. 8. 1895}
\nopagebreak\mylabel{L03164v}
\rehead{ }\normalsize\beginnumbering\briefempfaengerindex{Schnitzler, Arthur@\textsc{Schnitzler, Arthur}!zzzSalten, Felix@\emph{von Felix Salten}!1895-08-301@{30. 8. 1895}|(be}
\toendnotes[C]{\smallbreak\pagebreak[2]}\Standort{CUL, Schnitzler, B 89, A 1.}
\physDesc{Brief, 1 Blatt, 3 Seiten, 1772 Zeichen
\newline{}Handschrift: schwarze Tinte, lateinische Kurrent
\newline{}Ordnung: mit Bleistift von unbekannter Hand nummeriert: »64« }\toendnotes[C]{\smallbreak}
\pstart
           \centering{}{\pb}\textcolor{gray}{\textbf{\textsc{Hôtel oesterreichischer Hof\oindex{Oesterreichischer Hof@\textbf{Österreichischer Hof}, \emph{Hotel (K.HTL)}|pw}}.}}\pend
           
\pstart
           \centering{}\textcolor{gray}{\textbf{Franz Irresberger\pwindex{Irresberger, Franz 1862 – 1929-07-14@\textsc{Irresberger, Franz} (1862 – 1929-07-14), \emph{Hotelier/Hotelière}|pw}}}\pend
           
\pstart
           \centering{}\textcolor{gray}{\textbf{SALZBURG\oindex{Salzburg@\textbf{Salzburg}, \emph{A.ADM2}|pw}.}}\pend
           
\pstart
           \raggedleft{}30. VIII. 95\pend
           \vspace{0.5em}
\pstart
           Lieber Freund, ich habe bei meiner Ankunft nur \uline{die Hälfte} des so bestimmt erwarteten \label{K_L03164-1v}\edtext{Betrages}{\lemma{\textnormal{\emph{Betrages}}}\Cendnote{\textnormal{Offenbar
                  hatte Salten\pwindex{Salten, Felix 06.09.1869 – 08.10.1945@\textsc{Salten, Felix} (06.09.1869 – 08.10.1945), \emph{Schriftsteller/Schriftstellerin, Journalist/Journalistin, Chefredakteur/Chefredakteurin}|pwk} Geld von Schnitzler geliehen und konnte es nicht rechtzeitig
                  zurückzahlen.}}}\label{K_L03164-1} erhalten und auf meine telegrafische Urgenz ist bis jetzt
               noch nichts eingelangt, so dass ich wegen der Rückreise selbst in arger Verlegenheit
               bin. Seien Sie mir deshalb nicht böse, wenn in der Sache eine Verzögerung von einigen
               Tagen eintritt, ich empfinde das ohnedies peinlich genug und leide darunter, dass
               auch unsere \label{K_L03164-2v}\edtext{2\textsuperscript{te} Bicycle Tour}{\lemma{\textnormal{\emph{2\textsuperscript{te} … Tour}}}\Cendnote{\textnormal{Schon zwischen
                     24. 8. 1895 und
                     27. 8. 1895
                  hatten Salten\pwindex{Salten, Felix 06.09.1869 – 08.10.1945@\textsc{Salten, Felix} (06.09.1869 – 08.10.1945), \emph{Schriftsteller/Schriftstellerin, Journalist/Journalistin, Chefredakteur/Chefredakteurin}|pwk} und Schnitzler eine Radtour von Salzburg\oindex{Salzburg@\textbf{Salzburg}, \emph{A.ADM2}|pwk} nach München\oindex{Muenchen@\textbf{München}, \emph{P.PPLA}|pwk}
                  unternommen.}}}\label{K_L03164-2} mit einem solchen Nachspiel endet. Sollte ich aber heute oder morgen noch das
               Erhoffte bekommen, dann sende ich es Ihnen \uline{sofort}, wo
               nicht, \uline{gleich} nach meiner Rückkehr nach Wien\oindex{Wien@\textbf{Wien}, \emph{A.ADM2}|pw}. Das ist ganz sicher.\pend
           
\pstart
           L.\pwindex{Pohl-Glas, Charlotte 1873-01-01 – 1944-02-15@\textsc{Pohl-Glas, Charlotte} (1873-01-01 – 1944-02-15), \emph{Schriftsteller/Schriftstellerin, Politiker/Politikerin, Sozialist/Sozialistin}|pw} kam hier\oindex{Salzburg@\textbf{Salzburg}, \emph{A.ADM2}|pw}
               an voll Erbitterung und ich lebe schwere Tage. Irgend {\pb}ein Mensch, – wer, das bringe
               ich noch nicht heraus, – hat ihr in Gmunden\oindex{Gmunden@\textbf{Gmunden}, \emph{P.PPL}|pw} oder
                  Ischl\oindex{Bad Ischl@\textbf{Bad Ischl}, \emph{P.PPL}|pw} erzählt, dass ich das erste mal in Ischl\oindex{Bad Ischl@\textbf{Bad Ischl}, \emph{P.PPL}|pw} war, ferner, dass ich voriges Jahr, als sie
               hieherkam, \label{K_L03164-3v}\edtext{auch in Ischl\oindex{Bad Ischl@\textbf{Bad Ischl}, \emph{P.PPL}|pw} gewesen}{\lemma{\textnormal{\emph{auch in Ischl gewesen}}}\Cendnote{\textnormal{In
                  seinen \emph{Erinnerungen}\pwindex{Erinnerungen@\emph{Erinnerungen}|pwk} schreibt Salten\pwindex{Salten, Felix 06.09.1869 – 08.10.1945@\textsc{Salten, Felix} (06.09.1869 – 08.10.1945), \emph{Schriftsteller/Schriftstellerin, Journalist/Journalistin, Chefredakteur/Chefredakteurin}|pwk}: »Ich beredete meinen damaligen
                     Freund Karl Kraus\pwindex{Kraus, Karl 28.04.1874 – 12.06.1936@\textsc{Kraus, Karl} (28.04.1874 – 12.06.1936), \emph{Schriftsteller/Schriftstellerin, Publizist/Publizistin, Schriftsteller/Schriftstellerin}|pw}, dass er Lotte\pwindex{Pohl-Glas, Charlotte 1873-01-01 – 1944-02-15@\textsc{Pohl-Glas, Charlotte} (1873-01-01 – 1944-02-15), \emph{Schriftsteller/Schriftstellerin, Politiker/Politikerin, Sozialist/Sozialistin}|pw} an der Bahn erwarte und ihr sagen
                     solle, ich sei über eine Schafbergpartie\oindex{Schafberg [St. Gilgen]@\textbf{Schafberg [St. Gilgen]}, \emph{T.MT}|pw}
                     nach Unterach\oindex{Unterach am Attersee@\textbf{Unterach am Attersee}, \emph{P.PPL}|pw} und von dort auf dem Wege
                     nach Wien\oindex{Wien@\textbf{Wien}, \emph{A.ADM2}|pw}. Womöglich solle er sie zur
                     sofortigen Abreise nach Wien\oindex{Wien@\textbf{Wien}, \emph{A.ADM2}|pw} überreden. Kraus\pwindex{Kraus, Karl 28.04.1874 – 12.06.1936@\textsc{Kraus, Karl} (28.04.1874 – 12.06.1936), \emph{Schriftsteller/Schriftstellerin, Publizist/Publizistin, Schriftsteller/Schriftstellerin}|pw} war gerne dazu bereit und ich
                     wollte mich von Lottens Abreise persönlich überzeugen. Damals gab es in Ischl\oindex{Bad Ischl@\textbf{Bad Ischl}, \emph{P.PPL}|pw} noch Sänften. Also liess ich mich in
                     einer Sänfte auf den Perron tragen, liess dort die Sänfte an die Wand stellen
                     und beobachtete durch einen Vorhangsspalt, wie Lotte\pwindex{Pohl-Glas, Charlotte 1873-01-01 – 1944-02-15@\textsc{Pohl-Glas, Charlotte} (1873-01-01 – 1944-02-15), \emph{Schriftsteller/Schriftstellerin, Politiker/Politikerin, Sozialist/Sozialistin}|pw} von Kraus\pwindex{Kraus, Karl 28.04.1874 – 12.06.1936@\textsc{Kraus, Karl} (28.04.1874 – 12.06.1936), \emph{Schriftsteller/Schriftstellerin, Publizist/Publizistin, Schriftsteller/Schriftstellerin}|pw} überredet den
                     Zug bestieg um wieder nach Wien\oindex{Wien@\textbf{Wien}, \emph{A.ADM2}|pw} zu fahren.
                     Als der Zug abgegangen war verliess ich die Sänfte und erschien vor dem
                     entgeistert dastehenden Kraus\pwindex{Kraus, Karl 28.04.1874 – 12.06.1936@\textsc{Kraus, Karl} (28.04.1874 – 12.06.1936), \emph{Schriftsteller/Schriftstellerin, Publizist/Publizistin, Schriftsteller/Schriftstellerin}|pw}.«
                     (\emph{Wienbibliothek im Rathaus}, Nachlass Salten, ZPH 1681/1 1.1.1.9.1, [S. 47].)}}}\label{K_L03164-3}, hat ihr sonst
               allerhand Geschichten von Frau \label{K_L03164-4v}\edtext{Fr.\pwindex{Fr., Emma @\textsc{Fr., Emma}|pw} ferner von Frl. S.\pwindex{Sandrock, Adele 1863-08-19 – 1937-08-30@\textsc{Sandrock, Adele} (1863-08-19 – 1937-08-30), \emph{Schauspieler/Schauspielerin}|pw}}{\lemma{\textnormal{\emph{Fr. ferner von Frl. S.}}}\Cendnote{\textnormal{Emma Fr.\pwindex{Fr., Emma @\textsc{Fr., Emma}|pwk} und Adele Sandrock\pwindex{Sandrock, Adele 1863-08-19 – 1937-08-30@\textsc{Sandrock, Adele} (1863-08-19 – 1937-08-30), \emph{Schauspieler/Schauspielerin}|pwk}, die beide mit Salten\pwindex{Salten, Felix 06.09.1869 – 08.10.1945@\textsc{Salten, Felix} (06.09.1869 – 08.10.1945), \emph{Schriftsteller/Schriftstellerin, Journalist/Journalistin, Chefredakteur/Chefredakteurin}|pwk} in einer intimen Beziehung standen}}}\label{K_L03164-4} erzählt, –
               kurz, Sie können sich denken wie das arme Mädel\pwindex{Pohl-Glas, Charlotte 1873-01-01 – 1944-02-15@\textsc{Pohl-Glas, Charlotte} (1873-01-01 – 1944-02-15), \emph{Schriftsteller/Schriftstellerin, Politiker/Politikerin, Sozialist/Sozialistin}|pwv} zugerichtet war. So hatte ich hier zu thun, und habe es
               noch, um alles wieder ins Gleichgewicht zu bringen.\pend
           
\pstart
           Außerdem hat man ihr erzählt, wir seien in Salzburg\oindex{Salzburg@\textbf{Salzburg}, \emph{A.ADM2}|pw} mit einer \label{K_L03164-5v}\edtext{»jungen chic\substVorne{}\textsuperscript{k}\substDazwischen{}e\substHinten{}n Blondine\pwindex{Andreas-Salome, Lou 12.02.1861 – 05.02.1937@\textsc{Andreas-Salomé, Lou} (12.02.1861 – 05.02.1937), \emph{Schriftsteller/Schriftstellerin}|pwuv}«}{\lemma{\textnormal{\emph{»jungen chicen Blondine«}}}\Cendnote{\textnormal{Lou Andreas-Salomé\pwindex{Andreas-Salome, Lou 12.02.1861 – 05.02.1937@\textsc{Andreas-Salomé, Lou} (12.02.1861 – 05.02.1937), \emph{Schriftsteller/Schriftstellerin}|pwk}, siehe A. S.: \emph{Tagebuch}, 20. 8. 1895.}}}\label{K_L03164-5}
               »umhergelaufen«. Dass sie mir viel Tratsch über Sie, Beer-Hofmann\pwindex{Beer-Hofmann, Richard 1866-07-11 – 1945-09-26@\textsc{Beer-Hofmann, Richard} (1866-07-11 – 1945-09-26), \emph{Schriftsteller/Schriftstellerin}|pw} und mich mitgebracht, gehört {\pb}wol mit dazu. Von Kraus\pwindex{Kraus, Karl 28.04.1874 – 12.06.1936@\textsc{Kraus, Karl} (28.04.1874 – 12.06.1936), \emph{Schriftsteller/Schriftstellerin, Publizist/Publizistin, Schriftsteller/Schriftstellerin}|pw} ist im Familien-Journal\pwindex{Wiener Familien-Journal@\emph{Wiener Familien-Journal}|pw} eine Geschichte erschienen, »\label{K_L03164-6v}\edtext{Esplanade-Dichter\pwindex{Ischler Brief. (Wiener Dichter auf der Esplanade.)@\emph{Ischler Brief. (Wiener Dichter auf der Esplanade.)}|pwv}}{\lemma{\textnormal{\emph{Esplanade-Dichter}}}\Cendnote{\textnormal{Crêpedechine\pwindex{Kraus, Karl 28.04.1874 – 12.06.1936@\textsc{Kraus, Karl} (28.04.1874 – 12.06.1936), \emph{Schriftsteller/Schriftstellerin, Publizist/Publizistin, Schriftsteller/Schriftstellerin}|pwk} [ = Karl Kraus\pwindex{Kraus, Karl 28.04.1874 – 12.06.1936@\textsc{Kraus, Karl} (28.04.1874 – 12.06.1936), \emph{Schriftsteller/Schriftstellerin, Publizist/Publizistin, Schriftsteller/Schriftstellerin}|pwk}]: \emph{Ischler
                        Brief. (Wiener Dichter auf der Esplanade)}\pwindex{Ischler Brief. (Wiener Dichter auf der Esplanade.)@\emph{Ischler Brief. (Wiener Dichter auf der Esplanade.)}|pwk}. In: \emph{Wiener Familien-Journal}\pwindex{Wiener Familien-Journal@\emph{Wiener Familien-Journal}|pwk}, Nr. 230, 23. 8. 1895, S. 914–915. Während die
                  satirischen Bemerkungen über Beer-Hofmann\pwindex{Beer-Hofmann, Richard 1866-07-11 – 1945-09-26@\textsc{Beer-Hofmann, Richard} (1866-07-11 – 1945-09-26), \emph{Schriftsteller/Schriftstellerin}|pwk}
                     (»ein junger Dichter, der die beſten Erfolge auf dem Gebiete der Mode
                     aufzuweiſen hat«) und Hofmannsthal\pwindex{Hofmannsthal, Hugo von 1874-02-01 – 1929-07-15@\textsc{Hofmannsthal, Hugo von} (1874-02-01 – 1929-07-15), \emph{Schriftsteller/Schriftstellerin}|pwk} (»{[}e{]}in Wien\oindex{Wien@\textbf{Wien}, \emph{A.ADM2}|pw}er Dichter, der
                     den Schulſchluß abwarten muß, um nach Iſchl\oindex{Bad Ischl@\textbf{Bad Ischl}, \emph{P.PPL}|pw}
                     gehen zu können«) gut zuordenbar scheinen, lässt sich im Text\pwindex{Ischler Brief. (Wiener Dichter auf der Esplanade.)@\emph{Ischler Brief. (Wiener Dichter auf der Esplanade.)}|pwkv} keine unzweifelhafte
                  Spitze gegen Schnitzler ausmachen.}}}\label{K_L03164-6}«,
               das sind Beer Hofmann\pwindex{Beer-Hofmann, Richard 1866-07-11 – 1945-09-26@\textsc{Beer-Hofmann, Richard} (1866-07-11 – 1945-09-26), \emph{Schriftsteller/Schriftstellerin}|pw} und Sie, und sollen
               »Eure Affectationen und Posen« darin mit vielem Witz »gegeißelt« worden sein. Ich
               habs nicht gesehen.\pend
           
\pstart
           Bitte, sagen Sie an Hr. D\textsuperscript{r}{ }Goldmann\pwindex{Goldmann, Paul 31.01.1865 – 25.09.1935@\textsc{Goldmann, Paul} (31.01.1865 – 25.09.1935), \emph{Schriftsteller/Schriftstellerin, Journalist/Journalistin}|pw}, er möge Ihnen die Adresse von
                  \label{K_L03164-7v}\edtext{Bing\pwindex{Bing, Siegfried 1838-02-26 – 1905-09-06@\textsc{Bing, Siegfried} (1838-02-26 – 1905-09-06), \emph{Kunsthändler/Kunsthändlerin, Kunstsammler/Kunstsammlerin, Japanologe/Japanologin}|pw}}{\lemma{\textnormal{\emph{Bing}}}\Cendnote{\textnormal{Gemeint dürfte der in Paris\oindex{Paris@\textbf{Paris}, \emph{P.PPLC}|pwk} lebende Kunsthändler Siegfried Bing\pwindex{Bing, Siegfried 1838-02-26 – 1905-09-06@\textsc{Bing, Siegfried} (1838-02-26 – 1905-09-06), \emph{Kunsthändler/Kunsthändlerin, Kunstsammler/Kunstsammlerin, Japanologe/Japanologin}|pwk} sein, der sich auf japan\oindex{Japan@\textbf{Japan}, \emph{A.PCLI}|pwk}ische und asia\oindex{Asien@\textbf{Asien}, \emph{kein passender Code gefunden}|pwk}tische Kunst spezialisiert hatte. Vincent van Gogh\pwindex{Gogh, Vincent van 30.03.1853 – 29.07.1890@\textsc{Gogh, Vincent van} (30.03.1853 – 29.07.1890), \emph{Maler/Malerin}|pwk} frequentierte seine Sammlung. Goldmann\pwindex{Goldmann, Paul 31.01.1865 – 25.09.1935@\textsc{Goldmann, Paul} (31.01.1865 – 25.09.1935), \emph{Schriftsteller/Schriftstellerin, Journalist/Journalistin}|pwk} hielt sich zu diesem Zeitpunkt mit Schnitzler in München\oindex{Muenchen@\textbf{München}, \emph{P.PPLA}|pwk} auf.}}}\label{K_L03164-7} oder Bingen\pwindex{Bing, Siegfried 1838-02-26 – 1905-09-06@\textsc{Bing, Siegfried} (1838-02-26 – 1905-09-06), \emph{Kunsthändler/Kunsthändlerin, Kunstsammler/Kunstsammlerin, Japanologe/Japanologin}|pw}, das ist der Japaner\oindex{Japan@\textbf{Japan}, \emph{A.PCLI}|pwv}{[},{]} mittheilen, und schreiben Sie mir nach Wien\oindex{Wien@\textbf{Wien}, \emph{A.ADM2}|pw}, wo\strikeout{hin} ich ohnedies bald
               einen Brief von Ihnen erwarte.\pend
           
\pstart
           Mit vielen Empfehlungen an D\textsuperscript{r}{ }G.\pwindex{Goldmann, Paul 31.01.1865 – 25.09.1935@\textsc{Goldmann, Paul} (31.01.1865 – 25.09.1935), \emph{Schriftsteller/Schriftstellerin, Journalist/Journalistin}|pw} herzlichst\pend
           
\pstart
           Ihr {\\[\baselineskip]}\spacefill\mbox{Salten}\pend
           \leftskip=0em{}\selectlanguage{ngerman}\endnumbering\briefempfaengerindex{Schnitzler, Arthur@\textsc{Schnitzler, Arthur}!zzzSalten, Felix@\emph{von Felix Salten}!1895-08-301@{30. 8. 1895}|)be}\mylabel{L03164h}  \normalsize

\doendnotes{C}
\bigskip
\vfill

\clearpage

\footnotesize

\lohead{\textsc{register}}

% Definiere theindex-Environment komplett neu ohne reledmac
\makeatletter
\renewenvironment{theindex}{%
  \section*{\indexname}%
  \setlength{\parindent}{0pt}%
  \setlength{\parskip}{0pt plus 0.3pt}%
  \let\item\@idxitem
}{%
  \clearpage
}
\makeatother

\IfFileExists{\jobname-pw.ind}{\input{\jobname-pw.ind}}{}

\end{document}

      