%% latex-leseansicht-vorspann.tex
%% Vorspann für die Leseansicht.
%% Lädt die gemeinsame Datei latex-vorspann.tex mit nicht gesetztem Schalter.

\newif\ifkorrekturansicht
\korrekturansichtfalse

\input{../tex-inputs/latex-vorspann}


\section[ Felix Salten an Arthur Schnitzler, 30. 8. 1895]{L03164 Felix Salten an Arthur Schnitzler,  30. 8. 1895}
\nopagebreak\mylabel{L03164v}
\rehead{ }\normalsize\beginnumbering\briefempfaengerindex{Schnitzler, Arthur@\textsc{Schnitzler, Arthur}!zzzSalten, Felix@\emph{von Felix Salten}!1895-08-301@{30. 8. 1895}|(be}
\toendnotes[C]{\smallbreak\pagebreak[2]}
\correspDesc{Versand  durch Felix Salten am 30. 8. 1895 in Salzburg
\newline{}Erhalt  durch Arthur Schnitzler im Zeitraum [31. 8. 1895
                  – 4. 9. 1895?] in München}\toendnotes[C]{\smallbreak}
\Standort{CUL, Schnitzler, B 89, A 1.}
\physDesc{Brief, 1 Blatt, 3 Seiten, 1772 Zeichen
\newline{}Handschrift: schwarze Tinte, lateinische Kurrent
\newline{}Ordnung: mit Bleistift von unbekannter Hand nummeriert: »64« }\toendnotes[C]{\smallbreak}
\pstart
           \centering{}{\pb}\textcolor{gray}{\textbf{\textsc{Hôtel oesterreichischer Hof\oindex{Österreichischer Hof@\textbf{Österreichischer Hof}, \emph{Hotel}|pw}}.}}\pend
           
\pstart
           \centering{}\textcolor{gray}{\textbf{Franz Irresberger\pwindex{Irresberger, Franz 1862 – 14.\,7.\,1929 Salzburg@\textsc{Irresberger, Franz} (1862 – 14.\,7.\,1929 Salzburg), \emph{Hotelier}|pw}}}\pend
           
\pstart
           \centering{}\textcolor{gray}{\textbf{SALZBURG\oindex{Salzburg@\textbf{Salzburg}, \emph{Verwaltungsgebiet}|pw}.}}\pend
           
\pstart
           \raggedleft{}30. VIII. 95\pend
           \vspace{0.5em}
\pstart
           Lieber Freund, ich habe bei meiner Ankunft nur \uline{die Hälfte} des so bestimmt erwarteten \label{K_L03164-1v}\edtext{Betrages}{\lemma{\textnormal{\emph{Betrages}}}\Cendnote{\textnormal{Offenbar
                  hatte Salten\pwindex{Salten, Felix 6.\,9.\,1869 Budapest – 8.\,10.\,1945 Zürich@\textsc{Salten, Felix} (6.\,9.\,1869 Budapest – 8.\,10.\,1945 Zürich), \emph{Schriftsteller, Journalist, Chefredakteur}|pwk} Geld von Schnitzler geliehen und konnte es nicht rechtzeitig
                  zurückzahlen.}}}\label{K_L03164-1} erhalten und auf meine telegrafische Urgenz ist bis jetzt
               noch nichts eingelangt, so dass ich wegen der Rückreise selbst in arger Verlegenheit
               bin. Seien Sie mir deshalb nicht böse, wenn in der Sache eine Verzögerung von einigen
               Tagen eintritt, ich empfinde das ohnedies peinlich genug und leide darunter, dass
               auch unsere \label{K_L03164-2v}\edtext{2\textsuperscript{te} Bicycle Tour}{\lemma{\textnormal{\emph{2\textsuperscript{te} … Tour}}}\Cendnote{\textnormal{Schon zwischen
                     24. 8. 1895 und
                     27. 8. 1895
                  hatten Salten\pwindex{Salten, Felix 6.\,9.\,1869 Budapest – 8.\,10.\,1945 Zürich@\textsc{Salten, Felix} (6.\,9.\,1869 Budapest – 8.\,10.\,1945 Zürich), \emph{Schriftsteller, Journalist, Chefredakteur}|pwk} und Schnitzler eine Radtour von Salzburg\oindex{Salzburg@\textbf{Salzburg}, \emph{Verwaltungsgebiet}|pwk} nach München\oindex{München@\textbf{München}|pwk}
                  unternommen.}}}\label{K_L03164-2} mit einem solchen Nachspiel endet. Sollte ich aber heute oder morgen noch das
               Erhoffte bekommen, dann sende ich es Ihnen \uline{sofort}, wo
               nicht, \uline{gleich} nach meiner Rückkehr nach Wien\oindex{Wien@\textbf{Wien}, \emph{Verwaltungsgebiet}|pw}. Das ist ganz sicher.\pend
           
\pstart
           L.\pwindex{Pohl-Glas, Charlotte 1.\,1.\,1873 Wien – 15.\,2.\,1944 Zürich@\textsc{Pohl-Glas, Charlotte} (1.\,1.\,1873 Wien – 15.\,2.\,1944 Zürich), \emph{Schriftstellerin, Politikerin, Sozialistin}|pw} kam hier\oindex{Salzburg@\textbf{Salzburg}, \emph{Verwaltungsgebiet}|pw}
               an voll Erbitterung und ich lebe schwere Tage. Irgend {\pb}ein Mensch, – wer, das bringe
               ich noch nicht heraus, – hat ihr in Gmunden\oindex{Gmunden@\textbf{Gmunden}|pw} oder
                  Ischl\oindex{Bad Ischl@\textbf{Bad Ischl}|pw} erzählt, dass ich das erste mal in Ischl\oindex{Bad Ischl@\textbf{Bad Ischl}|pw} war, ferner, dass ich voriges Jahr, als sie
               hieherkam, \label{K_L03164-3v}\edtext{auch in Ischl\oindex{Bad Ischl@\textbf{Bad Ischl}|pw} gewesen}{\lemma{\textnormal{\emph{auch in Ischl gewesen}}}\Cendnote{\textnormal{In
                  seinen \emph{Erinnerungen}\pwindex{Salten, Felix 6.\,9.\,1869 Budapest – 8.\,10.\,1945 Zürich@\textsc{Salten, Felix} (6.\,9.\,1869 Budapest – 8.\,10.\,1945 Zürich), \emph{Schriftsteller, Journalist, Chefredakteur}!Erinnerungen@\strich\emph{Erinnerungen}|pwk} schreibt Salten\pwindex{Salten, Felix 6.\,9.\,1869 Budapest – 8.\,10.\,1945 Zürich@\textsc{Salten, Felix} (6.\,9.\,1869 Budapest – 8.\,10.\,1945 Zürich), \emph{Schriftsteller, Journalist, Chefredakteur}|pwk}: »Ich beredete meinen damaligen
                     Freund Karl Kraus\pwindex{Kraus, Karl 28.\,4.\,1874 Jičín – 12.\,6.\,1936 Wien@\textsc{Kraus, Karl} (28.\,4.\,1874 Jičín – 12.\,6.\,1936 Wien), \emph{Schriftsteller, Publizist, Schriftsteller}|pw}, dass er Lotte\pwindex{Pohl-Glas, Charlotte 1.\,1.\,1873 Wien – 15.\,2.\,1944 Zürich@\textsc{Pohl-Glas, Charlotte} (1.\,1.\,1873 Wien – 15.\,2.\,1944 Zürich), \emph{Schriftstellerin, Politikerin, Sozialistin}|pw} an der Bahn erwarte und ihr sagen
                     solle, ich sei über eine Schafbergpartie\oindex{Schafberg [St. Gilgen]@\textbf{Schafberg [St. Gilgen]}, \emph{Berg}|pw}
                     nach Unterach\oindex{Unterach am Attersee@\textbf{Unterach am Attersee}|pw} und von dort auf dem Wege
                     nach Wien\oindex{Wien@\textbf{Wien}, \emph{Verwaltungsgebiet}|pw}. Womöglich solle er sie zur
                     sofortigen Abreise nach Wien\oindex{Wien@\textbf{Wien}, \emph{Verwaltungsgebiet}|pw} überreden. Kraus\pwindex{Kraus, Karl 28.\,4.\,1874 Jičín – 12.\,6.\,1936 Wien@\textsc{Kraus, Karl} (28.\,4.\,1874 Jičín – 12.\,6.\,1936 Wien), \emph{Schriftsteller, Publizist, Schriftsteller}|pw} war gerne dazu bereit und ich
                     wollte mich von Lottens Abreise persönlich überzeugen. Damals gab es in Ischl\oindex{Bad Ischl@\textbf{Bad Ischl}|pw} noch Sänften. Also liess ich mich in
                     einer Sänfte auf den Perron tragen, liess dort die Sänfte an die Wand stellen
                     und beobachtete durch einen Vorhangsspalt, wie Lotte\pwindex{Pohl-Glas, Charlotte 1.\,1.\,1873 Wien – 15.\,2.\,1944 Zürich@\textsc{Pohl-Glas, Charlotte} (1.\,1.\,1873 Wien – 15.\,2.\,1944 Zürich), \emph{Schriftstellerin, Politikerin, Sozialistin}|pw} von Kraus\pwindex{Kraus, Karl 28.\,4.\,1874 Jičín – 12.\,6.\,1936 Wien@\textsc{Kraus, Karl} (28.\,4.\,1874 Jičín – 12.\,6.\,1936 Wien), \emph{Schriftsteller, Publizist, Schriftsteller}|pw} überredet den
                     Zug bestieg um wieder nach Wien\oindex{Wien@\textbf{Wien}, \emph{Verwaltungsgebiet}|pw} zu fahren.
                     Als der Zug abgegangen war verliess ich die Sänfte und erschien vor dem
                     entgeistert dastehenden Kraus\pwindex{Kraus, Karl 28.\,4.\,1874 Jičín – 12.\,6.\,1936 Wien@\textsc{Kraus, Karl} (28.\,4.\,1874 Jičín – 12.\,6.\,1936 Wien), \emph{Schriftsteller, Publizist, Schriftsteller}|pw}.«
                     (\emph{Wienbibliothek im Rathaus}, Nachlass Salten, ZPH 1681/1 1.1.1.9.1, [S. 47].)}}}\label{K_L03164-3}, hat ihr sonst
               allerhand Geschichten von Frau \label{K_L03164-4v}\edtext{Fr.\pwindex{Fr., Emma @\textsc{Fr., Emma}|pw} ferner von Frl. S.\pwindex{Sandrock, Adele 19.\,8.\,1863 Rotterdam – 30.\,8.\,1937 Berlin@\textsc{Sandrock, Adele} (19.\,8.\,1863 Rotterdam – 30.\,8.\,1937 Berlin), \emph{Schauspielerin}|pw}}{\lemma{\textnormal{\emph{Fr. ferner von Frl. S.}}}\Cendnote{\textnormal{Emma Fr.\pwindex{Fr., Emma @\textsc{Fr., Emma}|pwk} und Adele Sandrock\pwindex{Sandrock, Adele 19.\,8.\,1863 Rotterdam – 30.\,8.\,1937 Berlin@\textsc{Sandrock, Adele} (19.\,8.\,1863 Rotterdam – 30.\,8.\,1937 Berlin), \emph{Schauspielerin}|pwk}, die beide mit Salten\pwindex{Salten, Felix 6.\,9.\,1869 Budapest – 8.\,10.\,1945 Zürich@\textsc{Salten, Felix} (6.\,9.\,1869 Budapest – 8.\,10.\,1945 Zürich), \emph{Schriftsteller, Journalist, Chefredakteur}|pwk} in einer intimen Beziehung standen}}}\label{K_L03164-4} erzählt, –
               kurz, Sie können sich denken wie das arme Mädel\pwindex{Pohl-Glas, Charlotte 1.\,1.\,1873 Wien – 15.\,2.\,1944 Zürich@\textsc{Pohl-Glas, Charlotte} (1.\,1.\,1873 Wien – 15.\,2.\,1944 Zürich), \emph{Schriftstellerin, Politikerin, Sozialistin}|pwv} zugerichtet war. So hatte ich hier zu thun, und habe es
               noch, um alles wieder ins Gleichgewicht zu bringen.\pend
           
\pstart
           Außerdem hat man ihr erzählt, wir seien in Salzburg\oindex{Salzburg@\textbf{Salzburg}, \emph{Verwaltungsgebiet}|pw} mit einer \label{K_L03164-5v}\edtext{»jungen chic\substVorne{}\textsuperscript{k}\substDazwischen{}e\substHinten{}n Blondine\pwindex{Andreas-Salomé, Lou 12.\,2.\,1861 Sankt Petersburg – 5.\,2.\,1937 Göttingen@\textsc{Andreas-Salomé, Lou} (12.\,2.\,1861 Sankt Petersburg – 5.\,2.\,1937 Göttingen), \emph{Schriftstellerin}|pwuv}«}{\lemma{\textnormal{\emph{»jungen chicen Blondine«}}}\Cendnote{\textnormal{Lou Andreas-Salomé\pwindex{Andreas-Salomé, Lou 12.\,2.\,1861 Sankt Petersburg – 5.\,2.\,1937 Göttingen@\textsc{Andreas-Salomé, Lou} (12.\,2.\,1861 Sankt Petersburg – 5.\,2.\,1937 Göttingen), \emph{Schriftstellerin}|pwk}, siehe A. S.: \emph{Tagebuch}, 20. 8. 1895.}}}\label{K_L03164-5}
               »umhergelaufen«. Dass sie mir viel Tratsch über Sie, Beer-Hofmann\pwindex{Beer-Hofmann, Richard 11.\,7.\,1866 Wien – 26.\,9.\,1945 New York City@\textsc{Beer-Hofmann, Richard} (11.\,7.\,1866 Wien – 26.\,9.\,1945 New York City), \emph{Schriftsteller}|pw} und mich mitgebracht, gehört {\pb}wol mit dazu. Von Kraus\pwindex{Kraus, Karl 28.\,4.\,1874 Jičín – 12.\,6.\,1936 Wien@\textsc{Kraus, Karl} (28.\,4.\,1874 Jičín – 12.\,6.\,1936 Wien), \emph{Schriftsteller, Publizist, Schriftsteller}|pw} ist im Familien-Journal\pwindex{Wiener Familien-Journal@\emph{Wiener Familien-Journal}|pw} eine Geschichte erschienen, »\label{K_L03164-6v}\edtext{Esplanade-Dichter\pwindex{Kraus, Karl 28.\,4.\,1874 Jičín – 12.\,6.\,1936 Wien@\textsc{Kraus, Karl} (28.\,4.\,1874 Jičín – 12.\,6.\,1936 Wien), \emph{Schriftsteller, Publizist, Schriftsteller}!Ischler Brief. (Wiener Dichter auf der Esplanade.)@\strich\emph{Ischler Brief. (Wiener Dichter auf der Esplanade.)}|pwv}}{\lemma{\textnormal{\emph{Esplanade-Dichter}}}\Cendnote{\textnormal{Crêpedechine\pwindex{Kraus, Karl 28.\,4.\,1874 Jičín – 12.\,6.\,1936 Wien@\textsc{Kraus, Karl} (28.\,4.\,1874 Jičín – 12.\,6.\,1936 Wien), \emph{Schriftsteller, Publizist, Schriftsteller}|pwk} [ = Karl Kraus\pwindex{Kraus, Karl 28.\,4.\,1874 Jičín – 12.\,6.\,1936 Wien@\textsc{Kraus, Karl} (28.\,4.\,1874 Jičín – 12.\,6.\,1936 Wien), \emph{Schriftsteller, Publizist, Schriftsteller}|pwk}]: \emph{Ischler
                        Brief. (Wiener Dichter auf der Esplanade)}\pwindex{Kraus, Karl 28.\,4.\,1874 Jičín – 12.\,6.\,1936 Wien@\textsc{Kraus, Karl} (28.\,4.\,1874 Jičín – 12.\,6.\,1936 Wien), \emph{Schriftsteller, Publizist, Schriftsteller}!Ischler Brief. (Wiener Dichter auf der Esplanade.)@\strich\emph{Ischler Brief. (Wiener Dichter auf der Esplanade.)}|pwk}. In: \emph{Wiener Familien-Journal}\pwindex{Wiener Familien-Journal@\emph{Wiener Familien-Journal}|pwk}, Nr. 230, 23. 8. 1895, S. 914–915. Während die
                  satirischen Bemerkungen über Beer-Hofmann\pwindex{Beer-Hofmann, Richard 11.\,7.\,1866 Wien – 26.\,9.\,1945 New York City@\textsc{Beer-Hofmann, Richard} (11.\,7.\,1866 Wien – 26.\,9.\,1945 New York City), \emph{Schriftsteller}|pwk}
                     (»ein junger Dichter, der die beſten Erfolge auf dem Gebiete der Mode
                     aufzuweiſen hat«) und Hofmannsthal\pwindex{Hofmannsthal, Hugo von 1.\,2.\,1874 Wien – 15.\,7.\,1929 Rodaun@\textsc{Hofmannsthal, Hugo von} (1.\,2.\,1874 Wien – 15.\,7.\,1929 Rodaun), \emph{Schriftsteller}|pwk} (»{[}e{]}in Wien\oindex{Wien@\textbf{Wien}, \emph{Verwaltungsgebiet}|pw}er Dichter, der
                     den Schulſchluß abwarten muß, um nach Iſchl\oindex{Bad Ischl@\textbf{Bad Ischl}|pw}
                     gehen zu können«) gut zuordenbar scheinen, lässt sich im Text\pwindex{Kraus, Karl 28.\,4.\,1874 Jičín – 12.\,6.\,1936 Wien@\textsc{Kraus, Karl} (28.\,4.\,1874 Jičín – 12.\,6.\,1936 Wien), \emph{Schriftsteller, Publizist, Schriftsteller}!Ischler Brief. (Wiener Dichter auf der Esplanade.)@\strich\emph{Ischler Brief. (Wiener Dichter auf der Esplanade.)}|pwkv} keine unzweifelhafte
                  Spitze gegen Schnitzler ausmachen.}}}\label{K_L03164-6}«,
               das sind Beer Hofmann\pwindex{Beer-Hofmann, Richard 11.\,7.\,1866 Wien – 26.\,9.\,1945 New York City@\textsc{Beer-Hofmann, Richard} (11.\,7.\,1866 Wien – 26.\,9.\,1945 New York City), \emph{Schriftsteller}|pw} und Sie, und sollen
               »Eure Affectationen und Posen« darin mit vielem Witz »gegeißelt« worden sein. Ich
               habs nicht gesehen.\pend
           
\pstart
           Bitte, sagen Sie an Hr. D\textsuperscript{r}{ }Goldmann\pwindex{Goldmann, Paul 31.\,1.\,1865 Breslau – 25.\,9.\,1935 Wien@\textsc{Goldmann, Paul} (31.\,1.\,1865 Breslau – 25.\,9.\,1935 Wien), \emph{Schriftsteller, Journalist}|pw}, er möge Ihnen die Adresse von
                  \label{K_L03164-7v}\edtext{Bing\pwindex{Bing, Siegfried 26.\,2.\,1838 Hamburg – 6.\,9.\,1905 Vaucresson@\textsc{Bing, Siegfried} (26.\,2.\,1838 Hamburg – 6.\,9.\,1905 Vaucresson), \emph{Kunsthändler, Kunstsammler, Japanologe}|pw}}{\lemma{\textnormal{\emph{Bing}}}\Cendnote{\textnormal{Gemeint dürfte der in Paris\oindex{Paris@\textbf{Paris}, \emph{Hauptstadt}|pwk} lebende Kunsthändler Siegfried Bing\pwindex{Bing, Siegfried 26.\,2.\,1838 Hamburg – 6.\,9.\,1905 Vaucresson@\textsc{Bing, Siegfried} (26.\,2.\,1838 Hamburg – 6.\,9.\,1905 Vaucresson), \emph{Kunsthändler, Kunstsammler, Japanologe}|pwk} sein, der sich auf japan\oindex{Japan@\textbf{Japan}|pwk}ische und asia\oindex{Asien@\textbf{Asien}|pwk}tische Kunst spezialisiert hatte. Vincent van Gogh\pwindex{Gogh, Vincent van 30.\,3.\,1853 Zundert – 29.\,7.\,1890 Auvers-sur-Oise@\textsc{Gogh, Vincent van} (30.\,3.\,1853 Zundert – 29.\,7.\,1890 Auvers-sur-Oise), \emph{Maler}|pwk} frequentierte seine Sammlung. Goldmann\pwindex{Goldmann, Paul 31.\,1.\,1865 Breslau – 25.\,9.\,1935 Wien@\textsc{Goldmann, Paul} (31.\,1.\,1865 Breslau – 25.\,9.\,1935 Wien), \emph{Schriftsteller, Journalist}|pwk} hielt sich zu diesem Zeitpunkt mit Schnitzler in München\oindex{München@\textbf{München}|pwk} auf.}}}\label{K_L03164-7} oder Bingen\pwindex{Bing, Siegfried 26.\,2.\,1838 Hamburg – 6.\,9.\,1905 Vaucresson@\textsc{Bing, Siegfried} (26.\,2.\,1838 Hamburg – 6.\,9.\,1905 Vaucresson), \emph{Kunsthändler, Kunstsammler, Japanologe}|pw}, das ist der Japaner\oindex{Japan@\textbf{Japan}|pwv}{[},{]} mittheilen, und schreiben Sie mir nach Wien\oindex{Wien@\textbf{Wien}, \emph{Verwaltungsgebiet}|pw}, wo\strikeout{hin} ich ohnedies bald
               einen Brief von Ihnen erwarte.\pend
           
\pstart
           Mit vielen Empfehlungen an D\textsuperscript{r}{ }G.\pwindex{Goldmann, Paul 31.\,1.\,1865 Breslau – 25.\,9.\,1935 Wien@\textsc{Goldmann, Paul} (31.\,1.\,1865 Breslau – 25.\,9.\,1935 Wien), \emph{Schriftsteller, Journalist}|pw} herzlichst\pend
           
\pstart
           Ihr {\\[\baselineskip]}\spacefill\mbox{Salten}\pend
           \leftskip=0em{}\selectlanguage{ngerman}\endnumbering\briefempfaengerindex{Schnitzler, Arthur@\textsc{Schnitzler, Arthur}!zzzSalten, Felix@\emph{von Felix Salten}!1895-08-301@{30. 8. 1895}|)be}\mylabel{L03164h}  \newcommand{\dateiname}{L03164}\newcommand{\titel}{Felix Salten an Arthur Schnitzler, 30. 8. 1895}\newcommand{\editorInnen}{Martin Anton Müller und Laura Untner}%% latex-leseansicht-abspann.tex
%% Abspann für die Leseansicht.
%% Der Schalter \ifkorrekturansicht ist bereits durch den Vorspann gesetzt.

%% latex-abspann.tex
%% Gemeinsamer Abspann für Korrekturansicht und Leseansicht.
%% Setzt den Schalter \ifkorrekturansicht voraus (gesetzt in den
%% einbindenden Dateien latex-korrekturansicht-abspann.tex bzw.
%% latex-leseansicht-abspann.tex).
%% ---------------------------------------------------------------

\normalsize

% Das esempio-Environment wird nur in der Leseansicht benötigt
\ifkorrekturansicht\else
\newenvironment{esempio}[3]%
{
    \vspace{1.5ex}
    \rlap{\underline{#1}}
    \par
    \setlength{\parindent}{0cm}
    \nopagebreak
    \leftskip=#2cm
    \rightskip=#3cm
}
{
    \par
}
\fi

\doendnotes{C}
\bigskip
\vfill

\clearpage

\footnotesize

\ifkorrekturansicht
  \lohead{\textsc{register}}
\fi

% theindex-Environment neu definieren ohne reledmac
\makeatletter
\renewenvironment{theindex}{%
  \ifkorrekturansicht
    \section*{\indexname}%
  \else
    \subsubsection*{Index der erwähnten Entitäten}%
  \fi
  \setlength{\parindent}{0pt}%
  \setlength{\parskip}{0pt plus 0.3pt}%
  \let\item\@idxitem
}{%
  \ifkorrekturansicht\clearpage\fi
}
\makeatother

\IfFileExists{\jobname-pw.ind}{\input{\jobname-pw.ind}}{}

% Quellenangabe nur in der Leseansicht
\ifkorrekturansicht\else
% Fallback-Definitionen, falls die .tex-Datei \titel etc. nicht gesetzt hat
\providecommand{\titel}{}
\providecommand{\editorInnen}{}
\providecommand{\dateiname}{\jobname}

\vspace{3cm}

\vfill

\footnotesize
\textsc{Quelle}: \titel. Herausgegeben von {\editorInnen}. In: \emph{Arthur Schnitzler: Briefwechsel mit Autorinnen und Autoren}.
 Digitale Edition, https://schnitzler-briefe.acdh.oeaw.ac.at/{\dateiname}.html (Stand \today)
\fi

\end{document}


