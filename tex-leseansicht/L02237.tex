%% latex-leseansicht-vorspann.tex
%% Vorspann für die Leseansicht.
%% Lädt die gemeinsame Datei latex-vorspann.tex mit nicht gesetztem Schalter.

\newif\ifkorrekturansicht
\korrekturansichtfalse

\input{../tex-inputs/latex-vorspann}

\begin{center}
            \textcolor{red}{ENTWURF. ENTZIFFERUNG NOCH NICHT KORREKTURGELESEN}
                      \end{center}
            
               \section[Arthur Schnitzler an Richard Beer-Hofmann, 20. 8. 1916]{ Arthur Schnitzler an Richard Beer-Hofmann, 20. 8. 1916}\nopagebreak\mylabel{v}\rehead{ }\begin{ledgroupsized}[t]{13cm}\normalsize\beginnumbering\briefempfaengerindex{Beer-Hofmann, Richard@\textsc{Beer-Hofmann, Richard}!zzzSchnitzler, Arthur@\emph{von Arthur Schnitzler}!1916-08-201@{20. 8. 1916}|(be} \toendnotes[C]{\smallbreak\pagebreak[2]} \Standort{YCGL, MSS 31.}
\physDesc{Kartenbrief
\newline{}Handschrift: Bleistift, deutsche Kurrent\newline{}Versand: Stempel: »\nobreak{}\oindex{Altaussee@\textbf{Altaussee}|pwk}Alt Aussee, 21. VIII. 16\nobreak{}«.  
\newline{}Beer-Hofmann: mit blauem Buntstift den Empfang vermerkt:
                                 »E« }\buchAbdrucke{\weitereDrucke{Arthur Schnitzler, Richard Beer-Hofmann: \emph{Briefwechsel 1891–1931}. Hg. Konstanze Fliedl. Wien, Zürich: \emph{Europaverlag} 1992, S. 222.} }\pstart{}{\pb}Abſ. \textsc{Schnitzler}\pend{}\pstart{}\textsc{Altaussee\oindex{Altaussee@\textbf{Altaussee}|pw}, Fischerndorf 79\oindex{Fischerndorf@\textbf{Fischerndorf}|pw}}\pend{}{\bigskip}\pstart{}\textsc{Herrn Doctor Richard Beer-Hofmann}\pend{}\pstart{}\textsc{Bad Ischl}\oindex{Bad Ischl@\textbf{Bad Ischl}|pw}\pend{}\pstart{}\textsc{Grazerstr 52}\oindex{Grazer Strasse@\textbf{Grazer Straße}|pw}\pend{}{\bigskip}\pstart
           \raggedleft{}{\pb}\textsc{Altaussee\oindex{Altaussee@\textbf{Altaussee}|pw}, Fischerndorf 79\oindex{Fischerndorf@\textbf{Fischerndorf}|pw}}{\\}20. 8. 1916\pend
           \pstart
           lieber Richard, darf ich Sie um die große Gefälligkeit bitten, uns
               für Mittwoch oder Donnerſtag{ }Mittag zwei einbettige Zimmer (womöglich nebeneinander) in der Kaiſerkrone\oindex{Hotel Kaiserkrone@\textbf{Hotel Kaiserkrone}|pw} – event. Poſt\oindex{Hotel Post@\textbf{Hotel Post}|pw} reſerviren zu laſſen? Lieber Mittwoch als Do{\geminationn}erſtag und lieber Kaiſerkrone\oindex{Hotel Kaiserkrone@\textbf{Hotel Kaiserkrone}|pw} als Poſt\oindex{Hotel Post@\textbf{Hotel Post}|pw}\footnote{\noindent{}aber im Grunde gleichgiltig, insbeſondre ob Mittwoch oder
                        Donnerſtag.} Ich ko{\geminationm}e bei ſchönem Wetter zu Fuſs hinüber,
               ev. über Koppen\oindex{Hoher Koppen@\textbf{Hoher Koppen}|pw} oder Pötſchen\oindex{Poetschenpass@\textbf{Pötschenpass}|pw}, wohl erſt Nachmittag\pend
           \pstart
           Am Abend des Ankunfttages
               hoffen wir mit Ihnen beim S.–ſchein\oindex{Restaurant Sonnenschein@\textbf{Restaurant Sonnenschein}|pw} zu
               nachtmahlen, vorher kommen wir natürlich (we{\geminationn}’s Ihnen
               paſſt) zu Ihnen. Am nächſten Tag Aſchau\oindex{Aschau@\textbf{Aschau}|pw} und
               Retourfahrt nach Auſſee\oindex{Bad Aussee@\textbf{Bad Aussee}|pw}. Für telegraf. Verſtändigung
               – welcher Tag welches Hotel wären wir ſehr dankbar!\pend
           \pstart
           Von Herzen Ihr{\\[\baselineskip]}\spacefill\mbox{Arthur}\pend
           \leftskip=0em{}\endnumbering\briefempfaengerindex{Beer-Hofmann, Richard@\textsc{Beer-Hofmann, Richard}!zzzSchnitzler, Arthur@\emph{von Arthur Schnitzler}!1916-08-201@{20. 8. 1916}|)be}\mylabel{h}\end{ledgroupsized}  \newcommand{\dateiname}{L02237}\newcommand{\titel}{Arthur Schnitzler an Richard Beer-Hofmann, 20. 8. 1916}\newcommand{\editorInnen}{Martin Anton Müller und Gerd-Hermann Susen}%% latex-leseansicht-abspann.tex
%% Abspann für die Leseansicht.
%% Der Schalter \ifkorrekturansicht ist bereits durch den Vorspann gesetzt.

%% latex-abspann.tex
%% Gemeinsamer Abspann für Korrekturansicht und Leseansicht.
%% Setzt den Schalter \ifkorrekturansicht voraus (gesetzt in den
%% einbindenden Dateien latex-korrekturansicht-abspann.tex bzw.
%% latex-leseansicht-abspann.tex).
%% ---------------------------------------------------------------

\normalsize

% Das esempio-Environment wird nur in der Leseansicht benötigt
\ifkorrekturansicht\else
\newenvironment{esempio}[3]%
{
    \vspace{1.5ex}
    \rlap{\underline{#1}}
    \par
    \setlength{\parindent}{0cm}
    \nopagebreak
    \leftskip=#2cm
    \rightskip=#3cm
}
{
    \par
}
\fi

\doendnotes{C}
\bigskip
\vfill

\clearpage

\footnotesize

\ifkorrekturansicht
  \lohead{\textsc{register}}
\fi

% theindex-Environment neu definieren ohne reledmac
\makeatletter
\renewenvironment{theindex}{%
  \ifkorrekturansicht
    \section*{\indexname}%
  \else
    \subsubsection*{Index der erwähnten Entitäten}%
  \fi
  \setlength{\parindent}{0pt}%
  \setlength{\parskip}{0pt plus 0.3pt}%
  \let\item\@idxitem
}{%
  \ifkorrekturansicht\clearpage\fi
}
\makeatother

\IfFileExists{\jobname-pw.ind}{\input{\jobname-pw.ind}}{}

% Quellenangabe nur in der Leseansicht
\ifkorrekturansicht\else
% Fallback-Definitionen, falls die .tex-Datei \titel etc. nicht gesetzt hat
\providecommand{\titel}{}
\providecommand{\editorInnen}{}
\providecommand{\dateiname}{\jobname}

\vspace{3cm}

\vfill

\footnotesize
\textsc{Quelle}: \titel. Herausgegeben von {\editorInnen}. In: \emph{Arthur Schnitzler: Briefwechsel mit Autorinnen und Autoren}.
 Digitale Edition, https://schnitzler-briefe.acdh.oeaw.ac.at/{\dateiname}.html (Stand \today)
\fi

\end{document}


      