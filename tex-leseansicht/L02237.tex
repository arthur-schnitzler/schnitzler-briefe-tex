%% latex-leseansicht-vorspann.tex
%% Vorspann für die Leseansicht.
%% Lädt die gemeinsame Datei latex-vorspann.tex mit nicht gesetztem Schalter.

\newif\ifkorrekturansicht
\korrekturansichtfalse

\input{../tex-inputs/latex-vorspann}


\section[Arthur Schnitzler an Richard Beer-Hofmann, 20. 8. 1916]{L02237 Arthur Schnitzler an Richard Beer-Hofmann, 20. 8. 1916}
\nopagebreak\mylabel{L02237v}
\rehead{ }\normalsize\beginnumbering\briefempfaengerindex{Beer-Hofmann, Richard@\textsc{Beer-Hofmann, Richard}!zzzSchnitzler, Arthur@\emph{von Arthur Schnitzler}!1916-08-201@{20. 8. 1916}|(be}
\toendnotes[C]{\smallbreak\pagebreak[2]}
\correspDesc{Versand  durch Arthur Schnitzler am 20. 8. 1916 in Altaussee
\newline{}Übermittlung  am 21. 8. 1916 in Altaussee
\newline{}Erhalt  durch Richard Beer-Hofmann im Zeitraum [21. 8. 1916
                  – 25. 8. 1916?] in Bad Ischl}\toendnotes[C]{\smallbreak}
\Standort{YCGL, MSS 31.}
\physDesc{Kartenbrief, 755 Zeichen
\newline{}Handschrift: Bleistift, deutsche Kurrent
\newline{}Versand: Stempel: »\nobreak{}\oindex{Altaussee@\textbf{Altaussee}, \emph{Verwaltungsgebiet}|pwk}Alt Aussee, 21. VIII. 16\nobreak{}«.  
\newline{}Beer-Hofmann: mit blauem Buntstift den Empfang vermerkt:
                                 »E« }
\buchAbdrucke{\weitereDrucke{Arthur Schnitzler, Richard Beer-Hofmann: \emph{Briefwechsel 1891–1931}. Herausgegeben von Konstanze Fliedl. Wien, Zürich: \emph{Europaverlag} 1992, S. 222.} }\pstart{}{\pb}Abſ. \textsc{Schnitzler}\pend{}\pstart{}\textsc{Altaussee\oindex{Altaussee@\textbf{Altaussee}, \emph{Verwaltungsgebiet}|pw}, Fischerndorf 79\oindex{Fischerndorf@\textbf{Fischerndorf}|pw}}\pend{}{\bigskip}\pstart{}\textsc{Herrn Doctor Richard Beer-Hofmann}\pend{}\pstart{}\textsc{Bad Ischl}\oindex{Bad Ischl@\textbf{Bad Ischl}|pw}\pend{}\pstart{}\textsc{Grazerstr 52}\oindex{Grazer Straße [Bad Ischl]@\textbf{Grazer Straße [Bad Ischl]}, \emph{Straße}|pw}\pend{}{\bigskip}\vspace{1em}
\pstart
           \raggedleft{}{\pb}\textsc{Altaussee\oindex{Altaussee@\textbf{Altaussee}, \emph{Verwaltungsgebiet}|pw}, Fischerndorf 79\oindex{Fischerndorf@\textbf{Fischerndorf}|pw}}{\\}20. 8. 1916\pend
           \vspace{0.5em}
\pstart
           lieber Richard, darf ich Sie um die große Gefälligkeit bitten, uns
               für Mittwoch oder Donnerſtag{ }Mittag zwei einbettige Zimmer (womöglich nebeneinander) in der Kaiſerkrone\oindex{Hotel Kaiserkrone@\textbf{Hotel Kaiserkrone}, \emph{Hotel}|pw} – event. Poſt\oindex{Hotel Post [Bad Ischl]@\textbf{Hotel Post [Bad Ischl]}, \emph{Hotel}|pw} reſerviren zu laſſen? Lieber Mittwoch als
                     Do{\geminationn}erſtag und lieber Kaiſerkrone\oindex{Hotel Kaiserkrone@\textbf{Hotel Kaiserkrone}, \emph{Hotel}|pw} als Poſt\oindex{Hotel Post [Bad Ischl]@\textbf{Hotel Post [Bad Ischl]}, \emph{Hotel}|pw}\footnote{\noindent{}aber im Grunde gleichgiltig, insbeſondre ob Mittwoch oder
                        Donnerſtag.} Ich ko{\geminationm}e bei{ }ſchönem Wetter zu Fuſs hinüber, ev.
               über Koppen\oindex{Hoher Koppen@\textbf{Hoher Koppen}, \emph{Berg}|pw} oder Pötſchen\oindex{Pötschenpass@\textbf{Pötschenpass}, \emph{Pass}|pw}, wohl erſt Nachmittag\pend
           
\pstart
           Am Abend des Ankunfttages hoffen wir mit Ihnen beim S.–ſchein\oindex{Restaurant Sonnenschein@\textbf{Restaurant Sonnenschein}, \emph{Restaurant}|pw} zu nachtmahlen, vorher kommen wir natürlich (we{\geminationn}’s Ihnen paſſt) zu Ihnen. Am nächſten Tag Aſchau\oindex{Aschau@\textbf{Aschau}|pw} und Retourfahrt nach Auſſee\oindex{Bad Aussee@\textbf{Bad Aussee}, \emph{Hauptstadt}|pw}. Für telegraf. Verſtändigung – welcher Tag welches Hotel
               wären wir{ }ſehr dankbar!\pend
           
\pstart
           Von Herzen Ihr{\\[\baselineskip]}\spacefill\mbox{Arthur}\pend
           \leftskip=0em{}\selectlanguage{ngerman}\endnumbering\briefempfaengerindex{Beer-Hofmann, Richard@\textsc{Beer-Hofmann, Richard}!zzzSchnitzler, Arthur@\emph{von Arthur Schnitzler}!1916-08-201@{20. 8. 1916}|)be}\mylabel{L02237h}  \newcommand{\dateiname}{L02237}\newcommand{\titel}{Arthur Schnitzler an Richard Beer-Hofmann, 20. 8. 1916}\newcommand{\editorInnen}{Martin Anton Müller und Gerd-Hermann Susen}%% latex-leseansicht-abspann.tex
%% Abspann für die Leseansicht.
%% Der Schalter \ifkorrekturansicht ist bereits durch den Vorspann gesetzt.

%% latex-abspann.tex
%% Gemeinsamer Abspann für Korrekturansicht und Leseansicht.
%% Setzt den Schalter \ifkorrekturansicht voraus (gesetzt in den
%% einbindenden Dateien latex-korrekturansicht-abspann.tex bzw.
%% latex-leseansicht-abspann.tex).
%% ---------------------------------------------------------------

\normalsize

% Das esempio-Environment wird nur in der Leseansicht benötigt
\ifkorrekturansicht\else
\newenvironment{esempio}[3]%
{
    \vspace{1.5ex}
    \rlap{\underline{#1}}
    \par
    \setlength{\parindent}{0cm}
    \nopagebreak
    \leftskip=#2cm
    \rightskip=#3cm
}
{
    \par
}
\fi

\doendnotes{C}
\bigskip
\vfill

\clearpage

\footnotesize

\ifkorrekturansicht
  \lohead{\textsc{register}}
\fi

% theindex-Environment neu definieren ohne reledmac
\makeatletter
\renewenvironment{theindex}{%
  \ifkorrekturansicht
    \section*{\indexname}%
  \else
    \subsubsection*{Index der erwähnten Entitäten}%
  \fi
  \setlength{\parindent}{0pt}%
  \setlength{\parskip}{0pt plus 0.3pt}%
  \let\item\@idxitem
}{%
  \ifkorrekturansicht\clearpage\fi
}
\makeatother

\IfFileExists{\jobname-pw.ind}{\input{\jobname-pw.ind}}{}

% Quellenangabe nur in der Leseansicht
\ifkorrekturansicht\else
% Fallback-Definitionen, falls die .tex-Datei \titel etc. nicht gesetzt hat
\providecommand{\titel}{}
\providecommand{\editorInnen}{}
\providecommand{\dateiname}{\jobname}

\vspace{3cm}

\vfill

\footnotesize
\textsc{Quelle}: \titel. Herausgegeben von {\editorInnen}. In: \emph{Arthur Schnitzler: Briefwechsel mit Autorinnen und Autoren}.
 Digitale Edition, https://schnitzler-briefe.acdh.oeaw.ac.at/{\dateiname}.html (Stand \today)
\fi

\end{document}


