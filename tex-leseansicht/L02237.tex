%% latex-korrekturansicht-vorspann.tex
%% Vorspann für die Korrekturansicht.
%% Lädt die gemeinsame Datei latex-vorspann.tex mit gesetztem Schalter.

\newif\ifkorrekturansicht
\korrekturansichttrue

\input{../tex-inputs/latex-vorspann}


\section[Arthur Schnitzler an Richard Beer-Hofmann, 20. 8. 1916]{L02237 Arthur Schnitzler an Richard Beer-Hofmann, 20. 8. 1916}
\nopagebreak\mylabel{L02237v}
\rehead{ }\normalsize\beginnumbering\briefempfaengerindex{Beer-Hofmann, Richard@\textsc{Beer-Hofmann, Richard}!zzzSchnitzler, Arthur@\emph{von Arthur Schnitzler}!1916-08-201@{20. 8. 1916}|(be}
\toendnotes[C]{\smallbreak\pagebreak[2]}\Standort{YCGL, MSS 31.}
\physDesc{Kartenbrief, 755 Zeichen
\newline{}Handschrift: Bleistift, deutsche Kurrent
\newline{}Versand: Stempel: »\nobreak{}\oindex{Altaussee@\textbf{Altaussee}, \emph{A.ADM3}|pwk}Alt Aussee, 21. VIII. 16\nobreak{}«.  
\newline{}Beer-Hofmann: mit blauem Buntstift den Empfang vermerkt:
                                 »E« }
\buchAbdrucke{\weitereDrucke{Arthur Schnitzler, Richard Beer-Hofmann: \emph{Briefwechsel 1891–1931}. Wien, Zürich: \emph{Europaverlag} 1992, S. 222.} }\pstart{}{\pb}Abſ. \textsc{Schnitzler}\pend{}\pstart{}\textsc{Altaussee\oindex{Altaussee@\textbf{Altaussee}, \emph{A.ADM3}|pw}, Fischerndorf 79\oindex{Fischerndorf@\textbf{Fischerndorf}, \emph{P.PPL}|pw}}\pend{}{\bigskip}\pstart{}\textsc{Herrn Doctor Richard Beer-Hofmann}\pend{}\pstart{}\textsc{Bad Ischl}\oindex{Bad Ischl@\textbf{Bad Ischl}, \emph{P.PPL}|pw}\pend{}\pstart{}\textsc{Grazerstr 52}\oindex{Grazer Strasse [Bad Ischl]@\textbf{Grazer Straße [Bad Ischl]}, \emph{Straße (K.STR)}|pw}\pend{}{\bigskip}\vspace{1em}
\pstart
           \raggedleft{}{\pb}\textsc{Altaussee\oindex{Altaussee@\textbf{Altaussee}, \emph{A.ADM3}|pw}, Fischerndorf 79\oindex{Fischerndorf@\textbf{Fischerndorf}, \emph{P.PPL}|pw}}{\\}20. 8. 1916\pend
           \vspace{0.5em}
\pstart
           lieber Richard, darf ich Sie um die große Gefälligkeit bitten, uns
               für Mittwoch oder Donnerſtag{ }Mittag zwei einbettige Zimmer (womöglich nebeneinander) in der Kaiſerkrone\oindex{Hotel Kaiserkrone@\textbf{Hotel Kaiserkrone}, \emph{Hotel (K.HTL)}|pw} – event. Poſt\oindex{Hotel Post [Bad Ischl]@\textbf{Hotel Post [Bad Ischl]}, \emph{Hotel (K.HTL)}|pw} reſerviren zu laſſen? Lieber Mittwoch als
                     Do{\geminationn}erſtag und lieber Kaiſerkrone\oindex{Hotel Kaiserkrone@\textbf{Hotel Kaiserkrone}, \emph{Hotel (K.HTL)}|pw} als Poſt\oindex{Hotel Post [Bad Ischl]@\textbf{Hotel Post [Bad Ischl]}, \emph{Hotel (K.HTL)}|pw}\noindent{}aber im Grunde gleichgiltig, insbeſondre ob Mittwoch oder
                        Donnerſtag. Ich ko{\geminationm}e bei ſchönem Wetter zu Fuſs hinüber, ev.
               über Koppen\oindex{Hoher Koppen@\textbf{Hoher Koppen}, \emph{Berg (N.BRG)}|pw} oder Pötſchen\oindex{Poetschenpass@\textbf{Pötschenpass}, \emph{Pass (N.PAS)}|pw}, wohl erſt Nachmittag\pend
           
\pstart
           Am Abend des Ankunfttages hoffen wir mit Ihnen beim S.–ſchein\oindex{Restaurant Sonnenschein@\textbf{Restaurant Sonnenschein}, \emph{S.REST}|pw} zu nachtmahlen, vorher kommen wir natürlich (we{\geminationn}’s Ihnen paſſt) zu Ihnen. Am nächſten Tag Aſchau\oindex{Aschau@\textbf{Aschau}, \emph{eingemeindeter Ort (A.VOO)}|pw} und Retourfahrt nach Auſſee\oindex{Bad Aussee@\textbf{Bad Aussee}, \emph{P.PPLA3}|pw}. Für telegraf. Verſtändigung – welcher Tag welches Hotel
               wären wir ſehr dankbar!\pend
           
\pstart
           Von Herzen Ihr{\\[\baselineskip]}\spacefill\mbox{Arthur}\pend
           \leftskip=0em{}\selectlanguage{ngerman}\endnumbering\briefempfaengerindex{Beer-Hofmann, Richard@\textsc{Beer-Hofmann, Richard}!zzzSchnitzler, Arthur@\emph{von Arthur Schnitzler}!1916-08-201@{20. 8. 1916}|)be}\mylabel{L02237h}  \normalsize

\doendnotes{C}
\bigskip
\vfill

\clearpage

\footnotesize

\lohead{\textsc{register}}

% Definiere theindex-Environment komplett neu ohne reledmac
\makeatletter
\renewenvironment{theindex}{%
  \section*{\indexname}%
  \setlength{\parindent}{0pt}%
  \setlength{\parskip}{0pt plus 0.3pt}%
  \let\item\@idxitem
}{%
  \clearpage
}
\makeatother

\IfFileExists{\jobname-pw.ind}{\input{\jobname-pw.ind}}{}

\end{document}

      