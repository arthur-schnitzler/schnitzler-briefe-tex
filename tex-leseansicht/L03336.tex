%% latex-korrekturansicht-vorspann.tex
%% Vorspann für die Korrekturansicht.
%% Lädt die gemeinsame Datei latex-vorspann.tex mit gesetztem Schalter.

\newif\ifkorrekturansicht
\korrekturansichttrue

\input{../tex-inputs/latex-vorspann}


\section[ Felix Salten an Arthur Schnitzler, {[}24. 10. 1902{]}]{L03336 Felix Salten an Arthur Schnitzler, {[}24. 10. 1902{]}}
\nopagebreak\mylabel{L03336v}
\rehead{ }\normalsize\beginnumbering\briefempfaengerindex{Schnitzler, Arthur@\textsc{Schnitzler, Arthur}!zzzSalten, Felix@\emph{von Felix Salten}!1902-10-241@{{[}24. 10. 1902{]}}|(be}
\toendnotes[C]{\smallbreak\pagebreak[2]}\Standort{CUL, Schnitzler, B 89, A 2.}
\physDesc{Brief, 1 Blatt, 1 Seite, 157 Zeichen
\newline{}Handschrift: blaue Tinte, lateinische Kurrent
\newline{}Schnitzler: mit Bleistift datiert: »24/X 902« 
\newline{}Ordnung: mit Bleistift von unbekannter Hand nummeriert: »161« }\toendnotes[C]{\smallbreak}
\pstart
           \noindent{}{\pb}Lieber, die \label{K_L03336-1v}\edtext{kl. Veronika\pwindex{kleine Veronika@\emph{Die kleine Veronika}|pw}}{\lemma{\textnormal{\emph{kl. Veronika}}}\Cendnote{\textnormal{Felix Salten\pwindex{Salten, Felix 06.09.1869 – 08.10.1945@\textsc{Salten, Felix} (06.09.1869 – 08.10.1945), \emph{Schriftsteller/Schriftstellerin, Journalist/Journalistin, Chefredakteur/Chefredakteurin}|pwk}: \emph{Die kleine Veronika}\pwindex{kleine Veronika@\emph{Die kleine Veronika}|pwk}. In: \emph{Neue Deutsche Rundschau}\pwindex{Neue Deutsche Rundschau@\emph{Neue Deutsche Rundschau}|pwk}, Jg. 13, Nr. 12, Dezember 1902, S. 1285–1333.}}}\label{K_L03336-1} erscheint
               also am 1. Dezember in der »N. D. R\pwindex{Neue Deutsche Rundschau@\emph{Neue Deutsche Rundschau}|pw}« Eben theilt es mir Fischer\pwindex{Fischer, Samuel 24.12.1859 – 15.10.1934@\textsc{Fischer, Samuel} (24.12.1859 – 15.10.1934), \emph{Verleger/Verlegerin}|pw} mit. Ich freue mich aufrichtig und \label{K_L03336-2v}\edtext{danke Ihnen}{\lemma{\textnormal{\emph{danke Ihnen}}}\Cendnote{\textnormal{Siehe Arthur Schnitzler an Felix Salten, 16. 10. 1902.
               }}}\label{K_L03336-2} herzlichst.\pend
           
\pstart
           Ihr {\\[\baselineskip]}\spacefill\mbox{Salten}\pend
           \leftskip=0em{}\selectlanguage{ngerman}\endnumbering\briefempfaengerindex{Schnitzler, Arthur@\textsc{Schnitzler, Arthur}!zzzSalten, Felix@\emph{von Felix Salten}!1902-10-241@{{[}24. 10. 1902{]}}|)be}\mylabel{L03336h}  \normalsize

\doendnotes{C}
\bigskip
\vfill

\clearpage

\footnotesize

\lohead{\textsc{register}}

% Definiere theindex-Environment komplett neu ohne reledmac
\makeatletter
\renewenvironment{theindex}{%
  \section*{\indexname}%
  \setlength{\parindent}{0pt}%
  \setlength{\parskip}{0pt plus 0.3pt}%
  \let\item\@idxitem
}{%
  \clearpage
}
\makeatother

\IfFileExists{\jobname-pw.ind}{\input{\jobname-pw.ind}}{}

\end{document}

      