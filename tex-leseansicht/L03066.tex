%% latex-leseansicht-vorspann.tex
%% Vorspann für die Leseansicht.
%% Lädt die gemeinsame Datei latex-vorspann.tex mit nicht gesetztem Schalter.

\newif\ifkorrekturansicht
\korrekturansichtfalse

\input{../tex-inputs/latex-vorspann}


\section[ Paul Goldmann an Arthur Schnitzler, 13. 5. {[}1901{]}]{L03066 Paul Goldmann an Arthur Schnitzler,  13. 5. [1901]}
\nopagebreak\mylabel{L03066v}
\rehead{ }\normalsize\beginnumbering\briefempfaengerindex{Schnitzler, Arthur@\textsc{Schnitzler, Arthur}!zzzGoldmann, Paul@\emph{von Paul Goldmann}!1901-05-133@{13. 5. [1901]}|(be}
\toendnotes[C]{\smallbreak\pagebreak[2]}
\correspDesc{Versand  durch Paul Goldmann am 13. 5. [1901] in Berlin
\newline{}Erhalt  durch Arthur Schnitzler im Zeitraum [14. 5. 1901
                  – 18. 5. 1901?] in Wien}\toendnotes[C]{\smallbreak}
\Standort{DLA, A:Schnitzler, HS.NZ85.1.3171.}
\physDesc{Brief, 1 Blatt, 4 Seiten, 1453 Zeichen
\newline{}Handschrift: blaue Tinte, deutsche Kurrent
\newline{}Schnitzler: 1) mit Bleistift das Jahr »901« vermerkt  2) mit rotem Buntstift drei Unterstreichungen}\toendnotes[C]{\smallbreak}
\pstart
           \raggedleft{}{\pb}\textcolor{gray}{\textbf{DESSAUERSTRASSE 19}}\oindex{Dessauer Straße@\textbf{Dessauer Straße}, \emph{Straße}|pw}\pend
           
\pstart
           Berlin\oindex{Berlin@\textbf{Berlin}, \emph{Hauptstadt}|pw}, 13. Mai.\pend
           
\pstart\center{}Mein lieber Freund,\pend\vspace{0.5em}
\pstart
           Es thut mir \label{K_L03066-1v}\edtext{unendlich leid}{\lemma{\textnormal{\emph{unendlich leid}}}\Cendnote{\textnormal{ Am 10. 5. 1901 musste die Schwangerschaft von Olga Gussmann\pwindex{Schnitzler, Olga 17.\,1.\,1882 Wien – 13.\,1.\,1970 Lugano@\textsc{Schnitzler, Olga} (17.\,1.\,1882 Wien – 13.\,1.\,1970 Lugano), \emph{Schauspielerin, Sängerin}|pwk} mit dem gemeinsamen Kind
                  operativ beendet werden.}}}\label{K_L03066-1}, daß es{ }ſo gekommen iſt. Da kann man{ }ſich zum
               Troſt immer nur{ }ſagen: Wer weiß, wozu es gut war? Jedenfalls{ }ſind auch manche Sorgen
               dadurch beſeitigt. Und wenn wirklich Anämie daran Schuld war,{ }ſo iſt es vielleicht
               beſſer, wenn die Mutter\pwindex{Schnitzler, Olga 17.\,1.\,1882 Wien – 13.\,1.\,1970 Lugano@\textsc{Schnitzler, Olga} (17.\,1.\,1882 Wien – 13.\,1.\,1970 Lugano), \emph{Schauspielerin, Sängerin}|pwv} erſt
               einmal ordentlich gekräftigt wird, um {\pb}auch ein
               kräftiges Kind zur Welt zu bringen. Oder iſt das ein naturwiſſenſchaftlicher Unſinn?
               Schſade,{ }ſchade! Ihr{ }ſcheint Euch Beide\pwindex{Schnitzler, Olga 17.\,1.\,1882 Wien – 13.\,1.\,1970 Lugano@\textsc{Schnitzler, Olga} (17.\,1.\,1882 Wien – 13.\,1.\,1970 Lugano), \emph{Schauspielerin, Sängerin}|pwv}{ }ſehr darauf gefreut zu haben. Hoffen wir alſo auf \label{K_L03066-2v}\edtext{das nächſte Mal}{\lemma{\textnormal{\emph{das nächste Mal}}}\Cendnote{\textnormal{Das nächste Mal wurde Olga
                     Gussmann\pwindex{Schnitzler, Olga 17.\,1.\,1882 Wien – 13.\,1.\,1970 Lugano@\textsc{Schnitzler, Olga} (17.\,1.\,1882 Wien – 13.\,1.\,1970 Lugano), \emph{Schauspielerin, Sängerin}|pwk} Ende des Jahres schwanger. Am 9. 8. 1902 gebar sie Heinrich Schnitzler\pwindex{Schnitzler, Heinrich 9.\,8.\,1902 Hinterbrühl – 12.\,7.\,1982 Wien@\textsc{Schnitzler, Heinrich} (9.\,8.\,1902 Hinterbrühl – 12.\,7.\,1982 Wien), \emph{Regisseur, Schauspieler}|pwk}.}}}\label{K_L03066-2}!\pend
           
\pstart
           Wenn die \label{K_L03066-3v}\edtext{Sommerpläne}{\lemma{\textnormal{\emph{Sommerpläne}}}\Cendnote{\textnormal{Siehe XXXX Auszeichnungsfehler: Dokument L03064 nicht gefunden.
               }}}\label{K_L03066-3} gar{ }ſo{ }ſchwankend{ }ſind,{ }ſo iſt es vielleicht am Beſten, daß ich \textsc{Hirschfelds\pwindex{Hirschfeld, Robert 17.\,9.\,1857 Žďár nad Sázavou – 2.\,4.\,1914 Salzburg@\textsc{Hirschfeld, Robert} (17.\,9.\,1857 Žďár nad Sázavou – 2.\,4.\,1914 Salzburg), \emph{Journalist, Musikkritiker}|pw}} Einladung annehme, zu ihm
               an den Wörther See\oindex{Wörthersee@\textbf{Wörthersee}, \emph{See}|pw}{ }{\pb}zu kommen. Oder ich gehe nach Velden\oindex{Velden am Wörthersee@\textbf{Velden am Wörthersee}|pw}{ }\strikeout{\textcolor{gray}{×}\-\textcolor{gray}{×}\-\textcolor{gray}{×}\-\textcolor{gray}{×}} oder Pörtſchach\oindex{Pörtschach am Wörthersee@\textbf{Pörtschach am Wörthersee}|pw}. Ihr kommt dann hin,
                  \strikeout{ſ\textcolor{gray}{oe}\textcolor{gray}{×}\-\textcolor{gray}{×} Ihr k}{ }ſobald Ihr könnt.
               Ich wiederhole nochmals: ich will diesmal ruhig{ }ſitzen und nicht herumreiſen. Möchte
               auch in dieſen paar Wochen in einer Wien\oindex{Wien@\textbf{Wien}, \emph{Verwaltungsgebiet}|pw}er
               Sommerfriſche ein Bischen Wien\oindex{Wien@\textbf{Wien}, \emph{Verwaltungsgebiet}|pw}er Leben mitmachen.
               Iſt Deine Frau \label{K_L03066-4v}\edtext{Mutter\pwindex{Schnitzler, Louise 8.\,7.\,1840 Kőszeg – 9.\,9.\,1911 Wien@\textsc{Schnitzler, Louise} (8.\,7.\,1840 Kőszeg – 9.\,9.\,1911 Wien)|pwv} im Auguſt am Wörtherſee\oindex{Wörthersee@\textbf{Wörthersee}, \emph{See}|pw}}{\lemma{\textnormal{\emph{Mutter … Wörthersee}}}\Cendnote{\textnormal{Louise Schnitzler\pwindex{Schnitzler, Louise 8.\,7.\,1840 Kőszeg – 9.\,9.\,1911 Wien@\textsc{Schnitzler, Louise} (8.\,7.\,1840 Kőszeg – 9.\,9.\,1911 Wien)|pwk} war im Sommer 1901 höchstwahrscheinlich nicht am Wörthersee\oindex{Wörthersee@\textbf{Wörthersee}, \emph{See}|pwk}. Den Briefen Schnitzlers an sie ist zu entnehmen, dass sie in Klosters\oindex{Klosters Dorf@\textbf{Klosters Dorf}|pwk} (Schweiz\oindex{Schweiz@\textbf{Schweiz}|pwk})
                  war.}}}\label{K_L03066-4}?\pend
           
\pstart
           Ich muß mich jetzt wieder namenlos {\pb}mit der N. Fr. Pr.\orgindex{Neue Freie Presse@Neue Freie Presse|pw} herumkränken. Dem Herrn Nachtredakteur\pwindex{Kohler, Karl Felix 22.\,5.\,1838 Prag – 4.\,10.\,1911 Wien@\textsc{Kohler, Karl Felix} (22.\,5.\,1838 Prag – 4.\,10.\,1911 Wien), \emph{Journalist, Zeitungsredakteur}|pwv} (\textsc{Kohler\pwindex{Kohler, Karl Felix 22.\,5.\,1838 Prag – 4.\,10.\,1911 Wien@\textsc{Kohler, Karl Felix} (22.\,5.\,1838 Prag – 4.\,10.\,1911 Wien), \emph{Journalist, Zeitungsredakteur}|pw}}) bin ich antipathiſch. Infolgedeſſen verſchwinden alle meine Berlin\oindex{Berlin@\textbf{Berlin}, \emph{Hauptstadt}|pw}er Theatertelegramme{ }ſpurlos. Wenn ich mich beſchwere,
               heißt es: Raummangel, und dann wird ruhig weiter weggeworfen, was ich{ }ſchicke. Hätte
               ich eine andere Stellung, ich würde meine \label{K_L03066-5v}\edtext{Demiſſion}{\lemma{\textnormal{\emph{Demission}}}\Cendnote{\textnormal{Rücktritt}}}\label{K_L03066-5} geben{\dots}\pend
           
\pstart
           Bitte, Fräulein \textsc{Olga\pwindex{Schnitzler, Olga 17.\,1.\,1882 Wien – 13.\,1.\,1970 Lugano@\textsc{Schnitzler, Olga} (17.\,1.\,1882 Wien – 13.\,1.\,1970 Lugano), \emph{Schauspielerin, Sängerin}|pw}} recht herzlich zu grüßen, und{ }ſei auch Du vielmals gegrüßt von {\\[\baselineskip]}Deinem
               treuen {\\[\baselineskip]}\spacefill\mbox{Paul Goldmnn.}\pend
           \leftskip=0em{}\selectlanguage{ngerman}\endnumbering\briefempfaengerindex{Schnitzler, Arthur@\textsc{Schnitzler, Arthur}!zzzGoldmann, Paul@\emph{von Paul Goldmann}!1901-05-133@{13. 5. [1901]}|)be}\mylabel{L03066h}  \newcommand{\dateiname}{L03066}\newcommand{\titel}{Paul Goldmann an Arthur Schnitzler, 13. 5. [1901]}\newcommand{\editorInnen}{Martin Anton Müller und Laura Untner}%% latex-leseansicht-abspann.tex
%% Abspann für die Leseansicht.
%% Der Schalter \ifkorrekturansicht ist bereits durch den Vorspann gesetzt.

%% latex-abspann.tex
%% Gemeinsamer Abspann für Korrekturansicht und Leseansicht.
%% Setzt den Schalter \ifkorrekturansicht voraus (gesetzt in den
%% einbindenden Dateien latex-korrekturansicht-abspann.tex bzw.
%% latex-leseansicht-abspann.tex).
%% ---------------------------------------------------------------

\normalsize

% Das esempio-Environment wird nur in der Leseansicht benötigt
\ifkorrekturansicht\else
\newenvironment{esempio}[3]%
{
    \vspace{1.5ex}
    \rlap{\underline{#1}}
    \par
    \setlength{\parindent}{0cm}
    \nopagebreak
    \leftskip=#2cm
    \rightskip=#3cm
}
{
    \par
}
\fi

\doendnotes{C}
\bigskip
\vfill

\clearpage

\footnotesize

\ifkorrekturansicht
  \lohead{\textsc{register}}
\fi

% theindex-Environment neu definieren ohne reledmac
\makeatletter
\renewenvironment{theindex}{%
  \ifkorrekturansicht
    \section*{\indexname}%
  \else
    \subsubsection*{Index der erwähnten Entitäten}%
  \fi
  \setlength{\parindent}{0pt}%
  \setlength{\parskip}{0pt plus 0.3pt}%
  \let\item\@idxitem
}{%
  \ifkorrekturansicht\clearpage\fi
}
\makeatother

\IfFileExists{\jobname-pw.ind}{\input{\jobname-pw.ind}}{}

% Quellenangabe nur in der Leseansicht
\ifkorrekturansicht\else
% Fallback-Definitionen, falls die .tex-Datei \titel etc. nicht gesetzt hat
\providecommand{\titel}{}
\providecommand{\editorInnen}{}
\providecommand{\dateiname}{\jobname}

\vspace{3cm}

\vfill

\footnotesize
\textsc{Quelle}: \titel. Herausgegeben von {\editorInnen}. In: \emph{Arthur Schnitzler: Briefwechsel mit Autorinnen und Autoren}.
 Digitale Edition, https://schnitzler-briefe.acdh.oeaw.ac.at/{\dateiname}.html (Stand \today)
\fi

\end{document}


