%% latex-korrekturansicht-vorspann.tex
%% Vorspann für die Korrekturansicht.
%% Lädt die gemeinsame Datei latex-vorspann.tex mit gesetztem Schalter.

\newif\ifkorrekturansicht
\korrekturansichttrue

\input{../tex-inputs/latex-vorspann}


\section[ Paul Goldmann an Arthur Schnitzler, 13. 5. {[}1901{]}]{L03066 Paul Goldmann an Arthur Schnitzler, 13. 5. {[}1901{]}}
\nopagebreak\mylabel{L03066v}
\rehead{ }\normalsize\beginnumbering\briefempfaengerindex{Schnitzler, Arthur@\textsc{Schnitzler, Arthur}!zzzGoldmann, Paul@\emph{von Paul Goldmann}!1901-05-133@{13. 5. {[}1901{]}}|(be}
\toendnotes[C]{\smallbreak\pagebreak[2]}\Standort{DLA, A:Schnitzler, HS.NZ85.1.3171.}
\physDesc{Brief, 1 Blatt, 4 Seiten, 1453 Zeichen
\newline{}Handschrift: blaue Tinte, deutsche Kurrent
\newline{}Schnitzler: 1) mit Bleistift das Jahr »901« vermerkt  2) mit rotem Buntstift drei Unterstreichungen}\toendnotes[C]{\smallbreak}
\pstart
           \raggedleft{}{\pb}\textcolor{gray}{\textbf{DESSAUERSTRASSE 19}}\oindex{Dessauer Strasse@\textbf{Dessauer Straße}, \emph{Straße (K.STR)}|pw}\pend
           
\pstart
           Berlin\oindex{Berlin@\textbf{Berlin}, \emph{P.PPLC}|pw}, 13. Mai.\pend
           
\pstart\center{}Mein lieber Freund,\pend\vspace{0.5em}
\pstart
           Es thut mir \label{K_L03066-1v}\edtext{unendlich leid}{\lemma{\textnormal{\emph{unendlich leid}}}\Cendnote{\textnormal{ Am 10. 5. 1901 musste die Schwangerschaft von Olga Gussmann\pwindex{Schnitzler, Olga 17.01.1882 – 13.01.1970@\textsc{Schnitzler, Olga} (17.01.1882 – 13.01.1970), \emph{Schauspieler/Schauspielerin, Sänger/Sängerin}|pwk} mit dem gemeinsamen Kind
                  operativ beendet werden.}}}\label{K_L03066-1}, daß es ſo gekommen iſt. Da kann man ſich zum
               Troſt immer nur ſagen: Wer weiß, wozu es gut war? Jedenfalls ſind auch manche Sorgen
               dadurch beſeitigt. Und wenn wirklich Anämie daran Schuld war, ſo iſt es vielleicht
               beſſer, wenn die Mutter\pwindex{Schnitzler, Olga 17.01.1882 – 13.01.1970@\textsc{Schnitzler, Olga} (17.01.1882 – 13.01.1970), \emph{Schauspieler/Schauspielerin, Sänger/Sängerin}|pwv} erſt
               einmal ordentlich gekräftigt wird, um {\pb}auch ein
               kräftiges Kind zur Welt zu bringen. Oder iſt das ein naturwiſſenſchaftlicher Unſinn?
               Schſade, ſchade! Ihr ſcheint Euch Beide\pwindex{Schnitzler, Olga 17.01.1882 – 13.01.1970@\textsc{Schnitzler, Olga} (17.01.1882 – 13.01.1970), \emph{Schauspieler/Schauspielerin, Sänger/Sängerin}|pwv} ſehr darauf gefreut zu haben. Hoffen wir alſo auf \label{K_L03066-2v}\edtext{das nächſte Mal}{\lemma{\textnormal{\emph{das nächſte Mal}}}\Cendnote{\textnormal{Das nächste Mal wurde Olga
                     Gussmann\pwindex{Schnitzler, Olga 17.01.1882 – 13.01.1970@\textsc{Schnitzler, Olga} (17.01.1882 – 13.01.1970), \emph{Schauspieler/Schauspielerin, Sänger/Sängerin}|pwk} Ende des Jahres schwanger. Am 9. 8. 1902 gebar sie Heinrich Schnitzler\pwindex{Schnitzler, Heinrich 09.08.1902 – 12.07.1982@\textsc{Schnitzler, Heinrich} (09.08.1902 – 12.07.1982), \emph{Regisseur/Regisseurin, Schauspieler/Schauspielerin}|pwk}.}}}\label{K_L03066-2}!\pend
           
\pstart
           Wenn die \label{K_L03066-3v}\edtext{Sommerpläne}{\lemma{\textnormal{\emph{Sommerpläne}}}\Cendnote{\textnormal{Siehe Paul Goldmann an Arthur Schnitzler, 26. 4. [1901].
               }}}\label{K_L03066-3} gar ſo ſchwankend ſind, ſo iſt es vielleicht am Beſten, daß ich \textsc{Hirschfelds\pwindex{Hirschfeld, Robert 17.09.1857 – 02.04.1914@\textsc{Hirschfeld, Robert} (17.09.1857 – 02.04.1914), \emph{Journalist/Journalistin, Musikkritiker/Musikkritikerin}|pw}} Einladung annehme, zu ihm
               an den Wörther See\oindex{Woerthersee@\textbf{Wörthersee}, \emph{H.LK}|pw}{ }{\pb}zu kommen. Oder ich gehe nach Velden\oindex{Velden am Woerthersee@\textbf{Velden am Wörthersee}, \emph{P.PPL}|pw}{ }\strikeout{\textcolor{gray}{×}\-\textcolor{gray}{×}\-\textcolor{gray}{×}\-\textcolor{gray}{×}} oder Pörtſchach\oindex{Poertschach am Woerthersee@\textbf{Pörtschach am Wörthersee}, \emph{P.PPL}|pw}. Ihr kommt dann hin,
                  \strikeout{ſ\textcolor{gray}{oe}\textcolor{gray}{×}\-\textcolor{gray}{×} Ihr k} ſobald Ihr könnt.
               Ich wiederhole nochmals: ich will diesmal ruhig ſitzen und nicht herumreiſen. Möchte
               auch in dieſen paar Wochen in einer Wien\oindex{Wien@\textbf{Wien}, \emph{A.ADM2}|pw}er
               Sommerfriſche ein Bischen Wien\oindex{Wien@\textbf{Wien}, \emph{A.ADM2}|pw}er Leben mitmachen.
               Iſt Deine Frau \label{K_L03066-4v}\edtext{Mutter\pwindex{Schnitzler, Louise 1840-07-08 – 1911-09-09@\textsc{Schnitzler, Louise} (1840-07-08 – 1911-09-09)|pwv} im Auguſt am Wörtherſee\oindex{Woerthersee@\textbf{Wörthersee}, \emph{H.LK}|pw}}{\lemma{\textnormal{\emph{Mutter … Wörtherſee}}}\Cendnote{\textnormal{Louise Schnitzler\pwindex{Schnitzler, Louise 1840-07-08 – 1911-09-09@\textsc{Schnitzler, Louise} (1840-07-08 – 1911-09-09)|pwk} war im Sommer 1901 höchstwahrscheinlich nicht am Wörthersee\oindex{Woerthersee@\textbf{Wörthersee}, \emph{H.LK}|pwk}. Den Briefen Schnitzlers an sie ist zu entnehmen, dass sie in Klosters\oindex{Klosters Dorf@\textbf{Klosters Dorf}, \emph{P.PPL}|pwk} (Schweiz\oindex{Schweiz@\textbf{Schweiz}, \emph{A.PCLI}|pwk})
                  war.}}}\label{K_L03066-4}?\pend
           
\pstart
           Ich muß mich jetzt wieder namenlos {\pb}mit der N. Fr. Pr.\orgindex{Neue Freie Presse@Neue Freie Presse|pw} herumkränken. Dem Herrn Nachtredakteur\pwindex{Kohler, Karl Felix 1838-05-22 – 1911-10-04@\textsc{Kohler, Karl Felix} (1838-05-22 – 1911-10-04), \emph{Journalist/Journalistin, Zeitungsredakteur/Zeitungsredakteurin}|pwv} (\textsc{Kohler\pwindex{Kohler, Karl Felix 1838-05-22 – 1911-10-04@\textsc{Kohler, Karl Felix} (1838-05-22 – 1911-10-04), \emph{Journalist/Journalistin, Zeitungsredakteur/Zeitungsredakteurin}|pw}}) bin ich antipathiſch. Infolgedeſſen verſchwinden alle meine Berlin\oindex{Berlin@\textbf{Berlin}, \emph{P.PPLC}|pw}er Theatertelegramme ſpurlos. Wenn ich mich beſchwere,
               heißt es: Raummangel, und dann wird ruhig weiter weggeworfen, was ich ſchicke. Hätte
               ich eine andere Stellung, ich würde meine \label{K_L03066-5v}\edtext{Demiſſion}{\lemma{\textnormal{\emph{Demiſſion}}}\Cendnote{\textnormal{Rücktritt}}}\label{K_L03066-5} geben{\dots}\pend
           
\pstart
           Bitte, Fräulein \textsc{Olga\pwindex{Schnitzler, Olga 17.01.1882 – 13.01.1970@\textsc{Schnitzler, Olga} (17.01.1882 – 13.01.1970), \emph{Schauspieler/Schauspielerin, Sänger/Sängerin}|pw}} recht herzlich zu grüßen, und ſei auch Du vielmals gegrüßt von {\\[\baselineskip]}Deinem
               treuen {\\[\baselineskip]}\spacefill\mbox{Paul Goldmnn.}\pend
           \leftskip=0em{}\selectlanguage{ngerman}\endnumbering\briefempfaengerindex{Schnitzler, Arthur@\textsc{Schnitzler, Arthur}!zzzGoldmann, Paul@\emph{von Paul Goldmann}!1901-05-133@{13. 5. {[}1901{]}}|)be}\mylabel{L03066h}  \normalsize

\doendnotes{C}
\bigskip
\vfill

\clearpage

\footnotesize

\lohead{\textsc{register}}

% Definiere theindex-Environment komplett neu ohne reledmac
\makeatletter
\renewenvironment{theindex}{%
  \section*{\indexname}%
  \setlength{\parindent}{0pt}%
  \setlength{\parskip}{0pt plus 0.3pt}%
  \let\item\@idxitem
}{%
  \clearpage
}
\makeatother

\IfFileExists{\jobname-pw.ind}{\input{\jobname-pw.ind}}{}

\end{document}

      