%% latex-leseansicht-vorspann.tex
%% Vorspann für die Leseansicht.
%% Lädt die gemeinsame Datei latex-vorspann.tex mit nicht gesetztem Schalter.

\newif\ifkorrekturansicht
\korrekturansichtfalse

\input{../tex-inputs/latex-vorspann}


\section[Thomas Mann an Arthur Schnitzler, 27. 8. 1917]{L02270 Thomas Mann an Arthur Schnitzler, 27. 8. 1917}
\nopagebreak\mylabel{L02270v}
\rehead{ }\normalsize\beginnumbering\briefempfaengerindex{Schnitzler, Arthur@\textsc{Schnitzler, Arthur}!zzzMann, Thomas@\emph{von Thomas Mann}!1917-08-272@{27. 8. 1917}|(be}
\toendnotes[C]{\smallbreak\pagebreak[2]}
\correspDesc{Versand  durch Thomas Mann am 27. 8. 1917 in Bad Tölz
\newline{}Erhalt  durch Arthur Schnitzler im Zeitraum [28. 8. 1917
                  – 1. 9. 1917?] in Wien}\toendnotes[C]{\smallbreak}
\Standort{CUL, Schnitzler, B 67.}
\physDesc{Brief, 1 Blatt, 2 Seiten, 937 Zeichen
\newline{}Handschrift: schwarze Tinte, deutsche Kurrent
\newline{}Schnitzler: 1) mit Bleistift beschriftet: »\textsc{Th. Mann}«  2) mit rotem Buntstift eine Unterstreichung}
\buchAbdrucke{\weitereDrucke{Hertha Krotkoff: \emph{Arthur Schnitzler – Thomas Mann: Briefe.} In: \emph{Modern Austrian Literature}, Jg. 7 (1974) Nr. 1/2, S. 17.} }\toendnotes[C]{\smallbreak}
\pstart
           \raggedleft{}{\pb}Bad Tölz\oindex{Bad Tölz@\textbf{Bad Tölz}, \emph{Hauptstadt}|pw} den
                  27. VIII. 17.\pend
           
\pstart{}Verehrter Herr Doctor:\pend\vspace{0.5em}
\pstart
           Der Verlag\orgindex{S. Fischer Verlag@S. Fischer Verlag|pwv} überſandte mir in
               Ihrem gütigen Auftrage den »Doktor Gräsler\pwindex{Schnitzler, Arthur 15.\,5.\,1862 Wien – 21.\,10.\,1931 ebd.@\textsc{Schnitzler, Arthur} (15.\,5.\,1862 Wien – 21.\,10.\,1931 ebd.), \emph{Schriftsteller, Mediziner}!Doktor Gräsler, Badearzt@\strich\emph{Doktor Gräsler, Badearzt}|pw}«. Von
               Herzen danke ich Ihnen für die koſtbare Gabe, die echteſter Arthur Schnitzler iſt,
               anmutsvoll wie je eine frühere und bei aller Weichheit und Süßigkeit doch wieder \strikeout{dies} irgendwie dies{ }ſtrenge Lebensgefühl vermittelnd –
               ich werde nie aufhören, das zu bewundern.\pend
           
\pstart
           Mein oeffentliches Verſtummen iſt Ihnen {\pb}möglicherweiſe aufgefallen. Ich war nicht imſtand, meine Schuhe weiter zu machen.
               Seit Jahr und Tag{ }ſchreibe ich an einer Art von Buch\pwindex{Mann, Thomas 6.\,6.\,1875 Lübeck – 12.\,8.\,1955 Zürich@\textsc{Mann, Thomas} (6.\,6.\,1875 Lübeck – 12.\,8.\,1955 Zürich), \emph{Schriftsteller}!Betrachtungen eines Unpolitischen@\strich\emph{Betrachtungen eines Unpolitischen}|pwv}, \substVorne{}\textsuperscript{\textcolor{gray}{×}}\substDazwischen{}es\substHinten{}{ }ſind Betrachtungen, politiſch-antipolitiſch, zeit-
               und{ }ſelbſtkritiſch, kurz, eigentlich uferlos, aber nun doch leidlich eingedämmt, und
               bis zum Spätherbſt darf ich hoffen, es abſorbiert zu haben. Als Motto verdiente es
               den Satz: »\label{K_L02270-1v}\edtext{\textsc{Mais que diable allait-il faire dans cette galère?}\pwindex{Molière 14.\,1.\,1622 Paris – 17.\,2.\,1673 ebd.@\textsc{Molière} (14.\,1.\,1622 Paris – 17.\,2.\,1673 ebd.), \emph{Schriftsteller, Theaterleiter, Schauspieler}!Scapins Streiche@\strich\emph{Scapins Streiche}|pwv}}{\lemma{\textnormal{\emph{Mais … galère?}}}\Cendnote{\textnormal{französisch: Was zum Teufel hatte er auf
                  diesem Schiff zu suchen? (Molière\pwindex{Molière 14.\,1.\,1622 Paris – 17.\,2.\,1673 ebd.@\textsc{Molière} (14.\,1.\,1622 Paris – 17.\,2.\,1673 ebd.), \emph{Schriftsteller, Theaterleiter, Schauspieler}|pwk}: \emph{Les Fourberies de Scapin}\pwindex{Molière 14.\,1.\,1622 Paris – 17.\,2.\,1673 ebd.@\textsc{Molière} (14.\,1.\,1622 Paris – 17.\,2.\,1673 ebd.), \emph{Schriftsteller, Theaterleiter, Schauspieler}!Scapins Streiche@\strich\emph{Scapins Streiche}|pwk}, II,6.)}}}\label{K_L02270-1}« und doch mußte es{ }ſein.\pend
           
\pstart
           Mit den verbindlichſten Grüßen bin ich, verehrter Herr Doctor{\\[\baselineskip]}Ihr{\\[\baselineskip]}\spacefill\mbox{Thomas Mann.}\pend
           \leftskip=0em{}\selectlanguage{ngerman}\endnumbering\briefempfaengerindex{Schnitzler, Arthur@\textsc{Schnitzler, Arthur}!zzzMann, Thomas@\emph{von Thomas Mann}!1917-08-272@{27. 8. 1917}|)be}\mylabel{L02270h}  \newcommand{\dateiname}{L02270}\newcommand{\titel}{Thomas Mann an Arthur Schnitzler, 27. 8. 1917}\newcommand{\editorInnen}{Martin Anton Müller und Gerd-Hermann Susen}%% latex-leseansicht-abspann.tex
%% Abspann für die Leseansicht.
%% Der Schalter \ifkorrekturansicht ist bereits durch den Vorspann gesetzt.

%% latex-abspann.tex
%% Gemeinsamer Abspann für Korrekturansicht und Leseansicht.
%% Setzt den Schalter \ifkorrekturansicht voraus (gesetzt in den
%% einbindenden Dateien latex-korrekturansicht-abspann.tex bzw.
%% latex-leseansicht-abspann.tex).
%% ---------------------------------------------------------------

\normalsize

% Das esempio-Environment wird nur in der Leseansicht benötigt
\ifkorrekturansicht\else
\newenvironment{esempio}[3]%
{
    \vspace{1.5ex}
    \rlap{\underline{#1}}
    \par
    \setlength{\parindent}{0cm}
    \nopagebreak
    \leftskip=#2cm
    \rightskip=#3cm
}
{
    \par
}
\fi

\doendnotes{C}
\bigskip
\vfill

\clearpage

\footnotesize

\ifkorrekturansicht
  \lohead{\textsc{register}}
\fi

% theindex-Environment neu definieren ohne reledmac
\makeatletter
\renewenvironment{theindex}{%
  \ifkorrekturansicht
    \section*{\indexname}%
  \else
    \subsubsection*{Index der erwähnten Entitäten}%
  \fi
  \setlength{\parindent}{0pt}%
  \setlength{\parskip}{0pt plus 0.3pt}%
  \let\item\@idxitem
}{%
  \ifkorrekturansicht\clearpage\fi
}
\makeatother

\IfFileExists{\jobname-pw.ind}{\input{\jobname-pw.ind}}{}

% Quellenangabe nur in der Leseansicht
\ifkorrekturansicht\else
% Fallback-Definitionen, falls die .tex-Datei \titel etc. nicht gesetzt hat
\providecommand{\titel}{}
\providecommand{\editorInnen}{}
\providecommand{\dateiname}{\jobname}

\vspace{3cm}

\vfill

\footnotesize
\textsc{Quelle}: \titel. Herausgegeben von {\editorInnen}. In: \emph{Arthur Schnitzler: Briefwechsel mit Autorinnen und Autoren}.
 Digitale Edition, https://schnitzler-briefe.acdh.oeaw.ac.at/{\dateiname}.html (Stand \today)
\fi

\end{document}


