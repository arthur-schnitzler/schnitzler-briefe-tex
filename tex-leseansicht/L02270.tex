\input{../tex-inputs/latex-pdf-vorspann}
\begin{center}
            \textcolor{red}{ENTWURF. ENTZIFFERUNG NOCH NICHT KORREKTURGELESEN}
                      \end{center}
            
               \section[Thomas Mann an Arthur Schnitzler, 27. 8. 1917]{ Thomas Mann an Arthur Schnitzler, 27. 8. 1917}\nopagebreak\mylabel{v}\rehead{ }\begin{ledgroupsized}[t]{13cm}\normalsize\beginnumbering\briefempfaengerindex{Schnitzler, Arthur@\textsc{Schnitzler, Arthur}!zzzMann, Thomas@\emph{von Thomas Mann}!1917-08-272@{27. 8. 1917}|(be} \toendnotes[C]{\smallbreak\pagebreak[2]} \Standort{CUL, Schnitzler, B 67.}
\physDesc{Brief, 1 Blatt, 2 Seiten
\newline{}Handschrift: schwarze Tinte, deutsche Kurrent
\newline{}Schnitzler: 1) mit Bleistift beschriftet: »\textsc{Th. Mann}« 2) mit rotem Buntstift eine Unterstreichung}\buchAbdrucke{\weitereDrucke{Hertha Krotkoff: \emph{Arthur Schnitzler – Thomas Mann: Briefe.} In: \emph{Modern Austrian Literature}, Jg. 7 (1974) Nr. 1/2, S. 17.} }\toendnotes[C]{\smallbreak}\pstart
           \raggedleft{}{\pb}Bad Tölz\oindex{Bad Toelz@\textbf{Bad Tölz}|pw} den
                  27. VIII. 17.\pend
           \pstart{}Verehrter Herr Doctor:\pend\pstart
           Der Verlag\orgindex{S. Fischer Verlag@S. Fischer Verlag|pwv} überſandte mir in Ihrem
               gütigen Auftrage den »Doktor Gräsler\pwindex{Schnitzler, Arthur 15.05.1862 – 21.10.1931@\textsc{Schnitzler, Arthur} (15.05.1862 – 21.10.1931), \emph{Schriftsteller, Mediziner}!Doktor Graesler, Badearzt1917-02-10 – 1917-03-18@\strich\emph{Doktor Gräsler, Badearzt} {[}1917-02-10 – 1917-03-18{]}|pw}«. Von Herzen
               danke ich Ihnen für die koſtbare Gabe, die echteſter Arthur Schnitzler iſt,
               anmutsvoll wie je eine frühere und bei aller Weichheit und Süßigkeit doch wieder \strikeout{dies} irgendwie dies ſtrenge Lebensgefühl vermittelnd –
               ich werde nie aufhören, das zu bewundern.\pend
           \pstart
           Mein oeffentliches Verſtummen iſt Ihnen {\pb}möglicherweiſe aufgefallen. Ich war nicht imſtand, meine Schuhe weiter zu machen.
               Seit Jahr und Tag ſchreibe ich an einer Art von Buch\pwindex{Mann, Thomas 06.06.1875 – 12.08.1955@\textsc{Mann, Thomas} (06.06.1875 – 12.08.1955), \emph{Schriftsteller}!Betrachtungen eines Unpolitischen1918@\strich\emph{Betrachtungen eines Unpolitischen} {[}1918{]}|pwv}, \substVorne{}\textsuperscript{\textcolor{gray}{×}}\substDazwischen{}es\substHinten{}{ }ſind Betrachtungen, politiſch-antipolitiſch, zeit-
               und ſelbſtkritiſch, kurz, eigentlich uferlos, aber nun doch leidlich eingedämmt, und
               bis zum Spätherbſt darf ich hoffen, es abſorbiert zu haben. Als Motto verdiente es
               den Satz: »\label{K_L02270_1v}\edtext{\textsc{Mais que diable allait-il faire dans cette galère?}\pwindex{Moliere 14.01.1622 – 17.02.1673@\textsc{Molière} (14.01.1622 – 17.02.1673), \emph{Schriftsteller, Theaterleiter, Schauspieler}!Scapins Streiche1671@\strich\emph{Scapins Streiche} {[}1671{]}|pwv}}{\lemma{\textnormal{\emph{Mais … galère?}}}\Cendnote{\textnormal{französisch: Was zum Teufel hatte er auf
                  diesem Schiff zu suchen? (Molière\pwindex{Moliere 14.01.1622 – 17.02.1673@\textsc{Molière} (14.01.1622 – 17.02.1673), \emph{Schriftsteller, Theaterleiter, Schauspieler}|pwk}: \emph{Les
                        Fourberies de Scapin}\pwindex{Moliere 14.01.1622 – 17.02.1673@\textsc{Molière} (14.01.1622 – 17.02.1673), \emph{Schriftsteller, Theaterleiter, Schauspieler}!Scapins Streiche1671@\strich\emph{Scapins Streiche} {[}1671{]}|pwk}, II,6).}}}\label{K_L02270_1h}« und doch mußte es ſein.\pend
           \pstart
           Mit den verbindlichſten Grüßen bin ich, verehrter Herr Doctor{\\[\baselineskip]}Ihr{\\[\baselineskip]}\spacefill\mbox{Thomas Mann.}\pend
           \leftskip=0em{}\endnumbering\briefempfaengerindex{Schnitzler, Arthur@\textsc{Schnitzler, Arthur}!zzzMann, Thomas@\emph{von Thomas Mann}!1917-08-272@{27. 8. 1917}|)be}\mylabel{h}\end{ledgroupsized}  \newcommand{\dateiname}{L02270}\newcommand{\titel}{Thomas Mann an Arthur Schnitzler, 27. 8. 1917}\newcommand{\editorInnen}{Martin Anton Müller und Gerd-Hermann Susen}\input{../tex-inputs/latex-pdf-abspann}
      