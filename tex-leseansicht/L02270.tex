%% latex-korrekturansicht-vorspann.tex
%% Vorspann für die Korrekturansicht.
%% Lädt die gemeinsame Datei latex-vorspann.tex mit gesetztem Schalter.

\newif\ifkorrekturansicht
\korrekturansichttrue

\input{../tex-inputs/latex-vorspann}


\section[Thomas Mann an Arthur Schnitzler, 27. 8. 1917]{L02270 Thomas Mann an Arthur Schnitzler, 27. 8. 1917}
\nopagebreak\mylabel{L02270v}
\rehead{ }\normalsize\beginnumbering\briefempfaengerindex{Schnitzler, Arthur@\textsc{Schnitzler, Arthur}!zzzMann, Thomas@\emph{von Thomas Mann}!1917-08-272@{27. 8. 1917}|(be}
\toendnotes[C]{\smallbreak\pagebreak[2]}\Standort{CUL, Schnitzler, B 67.}
\physDesc{Brief, 1 Blatt, 2 Seiten, 937 Zeichen
\newline{}Handschrift: schwarze Tinte, deutsche Kurrent
\newline{}Schnitzler: 1) mit Bleistift beschriftet: »\textsc{Th. Mann}«  2) mit rotem Buntstift eine Unterstreichung}
\buchAbdrucke{\weitereDrucke{\emph{Modern Austrian Literature}, Jg. 7 (1974) Nr. 1/2, S. 17.} }\toendnotes[C]{\smallbreak}
\pstart
           \raggedleft{}{\pb}Bad Tölz\oindex{Bad Toelz@\textbf{Bad Tölz}, \emph{P.PPLA3}|pw} den
                  27. VIII. 17.\pend
           
\pstart{}Verehrter Herr Doctor:\pend\vspace{0.5em}
\pstart
           Der Verlag\orgindex{S. Fischer Verlag@S. Fischer Verlag|pwv} überſandte mir in
               Ihrem gütigen Auftrage den »Doktor Gräsler\pwindex{Doktor Graesler, Badearzt@\emph{Doktor Gräsler, Badearzt}|pw}«. Von
               Herzen danke ich Ihnen für die koſtbare Gabe, die echteſter Arthur Schnitzler iſt,
               anmutsvoll wie je eine frühere und bei aller Weichheit und Süßigkeit doch wieder \strikeout{dies} irgendwie dies ſtrenge Lebensgefühl vermittelnd –
               ich werde nie aufhören, das zu bewundern.\pend
           
\pstart
           Mein oeffentliches Verſtummen iſt Ihnen {\pb}möglicherweiſe aufgefallen. Ich war nicht imſtand, meine Schuhe weiter zu machen.
               Seit Jahr und Tag ſchreibe ich an einer Art von Buch\pwindex{Betrachtungen eines Unpolitischen@\emph{Betrachtungen eines Unpolitischen}|pwv}, \substVorne{}\textsuperscript{\textcolor{gray}{×}}\substDazwischen{}es\substHinten{}{ }ſind Betrachtungen, politiſch-antipolitiſch, zeit-
               und ſelbſtkritiſch, kurz, eigentlich uferlos, aber nun doch leidlich eingedämmt, und
               bis zum Spätherbſt darf ich hoffen, es abſorbiert zu haben. Als Motto verdiente es
               den Satz: »\label{K_L02270-1v}\edtext{\textsc{Mais que diable allait-il faire dans cette galère?}\pwindex{Scapins Streiche@\emph{Scapins Streiche}|pwv}}{\lemma{\textnormal{\emph{Mais … galère?}}}\Cendnote{\textnormal{französisch: Was zum Teufel hatte er auf
                  diesem Schiff zu suchen? (Molière\pwindex{Moliere 14.01.1622 – 17.02.1673@\textsc{Molière} (14.01.1622 – 17.02.1673), \emph{Schriftsteller/Schriftstellerin, Theaterleiter/Theaterleiterin, Schauspieler/Schauspielerin}|pwk}: \emph{Les Fourberies de Scapin}\pwindex{Scapins Streiche@\emph{Scapins Streiche}|pwk}, II,6.)}}}\label{K_L02270-1}« und doch mußte es
               ſein.\pend
           
\pstart
           Mit den verbindlichſten Grüßen bin ich, verehrter Herr Doctor{\\[\baselineskip]}Ihr{\\[\baselineskip]}\spacefill\mbox{Thomas Mann.}\pend
           \leftskip=0em{}\selectlanguage{ngerman}\endnumbering\briefempfaengerindex{Schnitzler, Arthur@\textsc{Schnitzler, Arthur}!zzzMann, Thomas@\emph{von Thomas Mann}!1917-08-272@{27. 8. 1917}|)be}\mylabel{L02270h}  \normalsize

\doendnotes{C}
\bigskip
\vfill

\clearpage

\footnotesize

\lohead{\textsc{register}}

% Definiere theindex-Environment komplett neu ohne reledmac
\makeatletter
\renewenvironment{theindex}{%
  \section*{\indexname}%
  \setlength{\parindent}{0pt}%
  \setlength{\parskip}{0pt plus 0.3pt}%
  \let\item\@idxitem
}{%
  \clearpage
}
\makeatother

\IfFileExists{\jobname-pw.ind}{\input{\jobname-pw.ind}}{}

\end{document}

      