%% latex-korrekturansicht-vorspann.tex
%% Vorspann für die Korrekturansicht.
%% Lädt die gemeinsame Datei latex-vorspann.tex mit gesetztem Schalter.

\newif\ifkorrekturansicht
\korrekturansichttrue

\input{../tex-inputs/latex-vorspann}


\section[Marie von Ebner-Eschenbach an Arthur Schnitzler, 26. 9. 1901]{L02581 Marie von Ebner-Eschenbach an Arthur Schnitzler, 26. 9. 1901}
\nopagebreak\mylabel{L02581v}
\rehead{ }\normalsize\beginnumbering\briefempfaengerindex{Schnitzler, Arthur@\textsc{Schnitzler, Arthur}!zzzEbner-Eschenbach, Marie von@\emph{von Marie von Ebner-Eschenbach}!1901-09-261@{26. 9. 1901}|(be}
\toendnotes[C]{\smallbreak\pagebreak[2]}\Standort{DLA, A:Schnitzler, HS.1985.1.5718.}
\physDesc{Brief, fotografische Vervielfältigung1 Blatt, 2 Seiten, 764 Zeichen
\newline{}Handschrift: schwarze Tinte, lateinische Kurrent
\newline{}Schnitzler: vermutlich mit rotem Buntstift »\textsc{\textcolor{gray}{Leutnt\pwindex{Lieutenant Gustl. Novelle@\emph{Lieutenant Gustl. Novelle}|pw}}}«, »\textsc{Ebner Eschenbach}« und eine Unterstreichung }\toendnotes[C]{\smallbreak}
\pstart
           \raggedleft{}{\pb}\textcolor{gray}{\textbf{SCHLOSS ZDISSLAWITZ}}\oindex{Schloss Zdislavice@\textbf{Schloss Zdislavice}, \emph{Schloss (K.SLS)}|pw}{\\}\textcolor{gray}{\textbf{POST ZDOUNEK. MÄHREN\oindex{Zdounky@\textbf{Zdounky}, \emph{P.PPL}|pw}}}\pend
           
\pstart
           \raggedleft{}26. Sept. 1901.\pend
           
\pstart\center{}Verehrter Herr Doctor!\pend\vspace{0.5em}
\pstart
           Viel zu spät danke ich Ihnen, verzeihen Sie es mir. So manche Entschuldigung hätte
               ich vorzubringen, will Sie aber nicht damit langweilen, sondern gleich anfangen das
               allzu lang Versäumte nachzuholen. Sie haben mir mit Ihrer großmütigen \label{K_L02581-1v}\edtext{Spende}{\lemma{\textnormal{\emph{Spende}}}\Cendnote{\textnormal{Vermutlich hat ihr Schnitzler seine beiden im April erschienenen Bücher \emph{Lieutenant Gustl}\pwindex{Lieutenant Gustl. Novelle@\emph{Lieutenant Gustl. Novelle}|pwk} und \emph{Frau Bertha Garlan}\pwindex{Frau Bertha Garlan. Roman@\emph{Frau Bertha Garlan. Roman}|pwk} geschenkt.}}}\label{K_L02581-1} Ehre erwiesen und
               Freude gemacht, Ihre beiden letzten Werke\pwindex{Lieutenant Gustl. Novelle@\emph{Lieutenant Gustl. Novelle}|pwv}\pwindex{Frau Bertha Garlan. Roman@\emph{Frau Bertha Garlan. Roman}|pwv} sind mir – wie deren Vorgänger – lieb und
               wert geworden und ich bewundere sie. Mit wärmster Zustimmung {\pb}las ich eben im Westermannschen Octoberheft\pwindex{Westermanns Monatshefte@\emph{Westermanns Monatshefte}|pw}
               die \label{K_L02581-2v}\edtext{Besprechung\pwindex{Romane und Novellen@\emph{Romane und Novellen}|pwv}}{\lemma{\textnormal{\emph{Besprechung}}}\Cendnote{\textnormal{F. D. [ = Friedrich Düsel]\pwindex{Duesel, Friedrich 11.02.1869 – 08.12.1945@\textsc{Düsel, Friedrich} (11.02.1869 – 08.12.1945), \emph{Kritiker/Kritikerin}|pwk}: \emph{Romane und Novellen}\pwindex{Romane und Novellen@\emph{Romane und Novellen}|pwk}. In: \emph{Westermanns Monatshefte}\pwindex{Westermanns Monatshefte@\emph{Westermanns Monatshefte}|pwk}, Jg. 46, Nr. 541, Oktober
                        1901, S. 157–160.}}}\label{K_L02581-2} des »Leutnant Gustl\pwindex{Lieutenant Gustl. Novelle@\emph{Lieutenant Gustl. Novelle}|pw}«\textcolor{gray}{.}\pend
           
\pstart
           Mir uralten Erzählerin ist das Zeichen des Wohlwollens das eines der glänzendsten
               Vertreter der neuen Richtung unserer Litteratur mir gegeben hat, eine Quelle
                  \textcolor{gray}{ewigster} Befriedigung.\pend
           
\pstart
           Dankbarst, verehrter Herr Doctor, {\\[\baselineskip]}Ihre ergebene {\\[\baselineskip]}\spacefill\mbox{Marie Ebner-Eschenbach.}\pend
           \leftskip=0em{}\selectlanguage{ngerman}\endnumbering\briefempfaengerindex{Schnitzler, Arthur@\textsc{Schnitzler, Arthur}!zzzEbner-Eschenbach, Marie von@\emph{von Marie von Ebner-Eschenbach}!1901-09-261@{26. 9. 1901}|)be}\mylabel{L02581h}  \normalsize

\doendnotes{C}
\bigskip
\vfill

\clearpage

\footnotesize

\lohead{\textsc{register}}

% Definiere theindex-Environment komplett neu ohne reledmac
\makeatletter
\renewenvironment{theindex}{%
  \section*{\indexname}%
  \setlength{\parindent}{0pt}%
  \setlength{\parskip}{0pt plus 0.3pt}%
  \let\item\@idxitem
}{%
  \clearpage
}
\makeatother

\IfFileExists{\jobname-pw.ind}{\input{\jobname-pw.ind}}{}

\end{document}

      