%% latex-leseansicht-vorspann.tex
%% Vorspann für die Leseansicht.
%% Lädt die gemeinsame Datei latex-vorspann.tex mit nicht gesetztem Schalter.

\newif\ifkorrekturansicht
\korrekturansichtfalse

\input{../tex-inputs/latex-vorspann}


               \section[Marie von Ebner-Eschenbach an Arthur Schnitzler, 26. 9. 1901]{ Marie von Ebner-Eschenbach an Arthur Schnitzler, 26. 9. 1901}\nopagebreak\mylabel{v}\rehead{ }\begin{ledgroupsized}[t]{13cm}\normalsize\beginnumbering\briefempfaengerindex{Schnitzler, Arthur@\textsc{Schnitzler, Arthur}!zzzEbner-Eschenbach, Marie von@\emph{von Marie von Ebner-Eschenbach}!1901-09-261@{26. 9. 1901}|(be} \toendnotes[C]{\smallbreak\pagebreak[2]} \Standort{DLA, A:Schnitzler, HS.1985.1.5718.}
\physDesc{Brief, 1 Blatt, 2 Seiten, fotografische Vervielfältigung
\newline{}Handschrift: schwarze Tinte, lateinische Kurrent
\newline{}Schnitzler: vermutlich mit rotem Buntstift »\textsc{\textcolor{gray}{Leutnt\pwindex{Schnitzler, Arthur 15.05.1862 – 21.10.1931@\textsc{Schnitzler, Arthur} (15.05.1862 – 21.10.1931), \emph{Schriftsteller, Mediziner}!Lieutenant Gustl. Novelle25. 12. 1900@\strich\emph{Lieutenant Gustl. Novelle} {[}25. 12. 1900{]}|pw}}}«, »\textsc{Ebner Eschenbach}« und eine Unterstreichung }\toendnotes[C]{\smallbreak}\pstart
           \noindent{}\raggedleft{}{\pb}\textcolor{gray}{\textbf{SCHLOSS ZDISSLAWITZ}}\oindex{Schloss Zdislavice@\textbf{Schloss Zdislavice}|pw}{\\}\textcolor{gray}{\textbf{POST ZDOUNEK. MÄHREN\oindex{Zdounky@\textbf{Zdounky}|pw}}}\pend
           \pstart
           \raggedleft{}26. Sept. 1901.\pend
           \pstart\center{}Verehrter Herr Doctor!\pend\pstart
           Viel zu spät danke ich Ihnen, verzeihen Sie es mir. So manche Entschuldigung hätte
               ich vorzubringen, will Sie aber nicht damit langweilen, sondern gleich anfangen das
               allzu lang Versäumte nachzuholen. Sie haben mir mit Ihrer großmütigen \label{K_L02581-1v}\edtext{Spende}{\lemma{\textnormal{\emph{Spende}}}\Cendnote{\textnormal{Vermutlich hat ihr Schnitzler\pwindex{Schnitzler, Arthur 15.05.1862 – 21.10.1931@\textsc{Schnitzler, Arthur} (15.05.1862 – 21.10.1931), \emph{Schriftsteller, Mediziner}|pwk} seine beiden im April erschienenen \emph{Lieutenant Gustl}\pwindex{Schnitzler, Arthur 15.05.1862 – 21.10.1931@\textsc{Schnitzler, Arthur} (15.05.1862 – 21.10.1931), \emph{Schriftsteller, Mediziner}!Lieutenant Gustl. Novelle25. 12. 1900@\strich\emph{Lieutenant Gustl. Novelle} {[}25. 12. 1900{]}|pwk} und \emph{Frau Bertha Garlan}\pwindex{Schnitzler, Arthur 15.05.1862 – 21.10.1931@\textsc{Schnitzler, Arthur} (15.05.1862 – 21.10.1931), \emph{Schriftsteller, Mediziner}!Frau Bertha Garlan. Roman15.1.1901 – 15.3.1901@\strich\emph{Frau Bertha Garlan. Roman} {[}15.1.1901 – 15.3.1901{]}|pwk} geschenkt.}}}\label{K_L02581-1h} Ehre erwiesen und
               Freude gemacht, Ihre beiden letzten Werke\pwindex{Schnitzler, Arthur 15.05.1862 – 21.10.1931@\textsc{Schnitzler, Arthur} (15.05.1862 – 21.10.1931), \emph{Schriftsteller, Mediziner}!Lieutenant Gustl. Novelle25. 12. 1900@\strich\emph{Lieutenant Gustl. Novelle} {[}25. 12. 1900{]}|pwv}\pwindex{Schnitzler, Arthur 15.05.1862 – 21.10.1931@\textsc{Schnitzler, Arthur} (15.05.1862 – 21.10.1931), \emph{Schriftsteller, Mediziner}!Frau Bertha Garlan. Roman15.1.1901 – 15.3.1901@\strich\emph{Frau Bertha Garlan. Roman} {[}15.1.1901 – 15.3.1901{]}|pwv} sind mir – wie deren Vorgänger – lieb und
               wert geworden und ich bewundere sie. Mit wärmster Zustimmung {\pb}las ich eben im Westermannschen Octoberheft\pwindex{Westermanns MonatshefteOktober 1856@\emph{Westermanns Monatshefte}|pw}
               die \label{K_L02581-2v}\edtext{Besprechung\pwindex{Romane und NovellenOktober 1901@\emph{Romane und Novellen} {[}Oktober 1901{]}|pwv}}{\lemma{\textnormal{\emph{Besprechung}}}\Cendnote{\textnormal{F. D. [=Friedrich Düsel]\pwindex{Duesel, Friedrich 11.02.1869 – 08.12.1945@\textsc{Düsel, Friedrich} (11.02.1869 – 08.12.1945), \emph{Kritiker}|pwk}: \emph{Romane und Novellen}\pwindex{Romane und NovellenOktober 1901@\emph{Romane und Novellen} {[}Oktober 1901{]}|pwk}. In: \emph{Westermanns Monatshefte}\pwindex{Westermanns MonatshefteOktober 1856@\emph{Westermanns Monatshefte}|pwk}, Jg. 46, Nr. 541, Oktober
                        1901, S. 157–160.}}}\label{K_L02581-2h} des »Leutnant Gustl\pwindex{Schnitzler, Arthur 15.05.1862 – 21.10.1931@\textsc{Schnitzler, Arthur} (15.05.1862 – 21.10.1931), \emph{Schriftsteller, Mediziner}!Lieutenant Gustl. Novelle25. 12. 1900@\strich\emph{Lieutenant Gustl. Novelle} {[}25. 12. 1900{]}|pw}«\textcolor{gray}{.}\pend
           \pstart
           Mir uralten Erzählerin ist das Zeichen des Wohlwollens das eines der glänzendsten
               Vertreter der neuen Richtung unserer Litteratur mir gegeben hat, eine Quelle
                  \textcolor{gray}{ewigster} Befriedigung.\pend
           \pstart
           Dankbarst, verehrter Herr Doctor, {\\[\baselineskip]}Ihre ergebene {\\[\baselineskip]}\spacefill\mbox{Marie Ebner-Eschenbach.}\pend
           \leftskip=0em{}          \endnumbering\briefempfaengerindex{Schnitzler, Arthur@\textsc{Schnitzler, Arthur}!zzzEbner-Eschenbach, Marie von@\emph{von Marie von Ebner-Eschenbach}!1901-09-261@{26. 9. 1901}|)be}\mylabel{h}\end{ledgroupsized}  \newcommand{\dateiname}{L02581}\newcommand{\titel}{Marie von Ebner-Eschenbach an Arthur Schnitzler, 26. 9. 1901}\newcommand{\editorInnen}{Martin Anton Müller und Laura Untner}
            \footnotesize
\begin{ledgroupsized}[t]{11.5cm}
\doendnotes{C}
\end{ledgroupsized}
         %% latex-leseansicht-abspann.tex
%% Abspann für die Leseansicht.
%% Der Schalter \ifkorrekturansicht ist bereits durch den Vorspann gesetzt.

%% latex-abspann.tex
%% Gemeinsamer Abspann für Korrekturansicht und Leseansicht.
%% Setzt den Schalter \ifkorrekturansicht voraus (gesetzt in den
%% einbindenden Dateien latex-korrekturansicht-abspann.tex bzw.
%% latex-leseansicht-abspann.tex).
%% ---------------------------------------------------------------

\normalsize

% Das esempio-Environment wird nur in der Leseansicht benötigt
\ifkorrekturansicht\else
\newenvironment{esempio}[3]%
{
    \vspace{1.5ex}
    \rlap{\underline{#1}}
    \par
    \setlength{\parindent}{0cm}
    \nopagebreak
    \leftskip=#2cm
    \rightskip=#3cm
}
{
    \par
}
\fi

\doendnotes{C}
\bigskip
\vfill

\clearpage

\footnotesize

\ifkorrekturansicht
  \lohead{\textsc{register}}
\fi

% theindex-Environment neu definieren ohne reledmac
\makeatletter
\renewenvironment{theindex}{%
  \ifkorrekturansicht
    \section*{\indexname}%
  \else
    \subsubsection*{Index der erwähnten Entitäten}%
  \fi
  \setlength{\parindent}{0pt}%
  \setlength{\parskip}{0pt plus 0.3pt}%
  \let\item\@idxitem
}{%
  \ifkorrekturansicht\clearpage\fi
}
\makeatother

\IfFileExists{\jobname-pw.ind}{\input{\jobname-pw.ind}}{}

% Quellenangabe nur in der Leseansicht
\ifkorrekturansicht\else
% Fallback-Definitionen, falls die .tex-Datei \titel etc. nicht gesetzt hat
\providecommand{\titel}{}
\providecommand{\editorInnen}{}
\providecommand{\dateiname}{\jobname}

\vspace{3cm}

\vfill

\footnotesize
\textsc{Quelle}: \titel. Herausgegeben von {\editorInnen}. In: \emph{Arthur Schnitzler: Briefwechsel mit Autorinnen und Autoren}.
 Digitale Edition, https://schnitzler-briefe.acdh.oeaw.ac.at/{\dateiname}.html (Stand \today)
\fi

\end{document}


      