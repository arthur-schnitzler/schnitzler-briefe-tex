%% latex-korrekturansicht-vorspann.tex
%% Vorspann für die Korrekturansicht.
%% Lädt die gemeinsame Datei latex-vorspann.tex mit gesetztem Schalter.

\newif\ifkorrekturansicht
\korrekturansichttrue

\input{../tex-inputs/latex-vorspann}


\section[Arno Holz an Arthur Schnitzler, 11. 2. 1917]{L02255 Arno Holz an Arthur Schnitzler, 11. 2. 1917}
\nopagebreak\mylabel{L02255v}
\rehead{ }\normalsize\beginnumbering\briefempfaengerindex{Schnitzler, Arthur@\textsc{Schnitzler, Arthur}!zzzHolz, Arno@\emph{von Arno Holz}!1917-02-171@{11. 2. 1917}|(be}
\toendnotes[C]{\smallbreak\pagebreak[2]}\Standort{DLA, A:Schnitzler, HS.NZ85.1.5728.}
\physDesc{Brief, Fotokopie1 Blatt, 1 Seite, 1553 Zeichen
\newline{}Handschrift: schwarze Tinte, deutsche Kurrent
\newline{}Schnitzler: mit (mutmaßlich) rotem Buntstift beschriftet: »\textcolor{gray}{\textbf{Holz}}« }
\pstart
           \raggedleft{}{\pb}Berlin W, 30. Stübbenſtr. 5\oindex{Stuebbenstrasse@\textbf{Stübbenstraße}, \emph{Straße (K.STR)}|pw}.\hspace*{1.5em}11. II. 17. \pend
           
\pstart\center{}Sehr verehrter Herr Doktor!\pend\vspace{0.5em}
\pstart
           Durch die Ungunſt der Zeitumſtände bin ich gezwungen von meinem ſatiriſchen
               Gedichtwerk »Die Blechſchmiede\pwindex{Blechschmiede@\emph{Die Blechschmiede}|pw}« (Leipzig\oindex{Leipzig@\textbf{Leipzig}, \emph{P.PPLA3}|pw}, Inſel-Verlag\orgindex{Insel Verlag@Insel Verlag|pw} vergriffen) die neue, ſtark über das doppelte vermehrte Ausgabe
                  \uline{lediglich auf private Subſkription} herauszugeben.
               Das Werk ſoll mit einer ſchönen Type auf gutem Bütten in Großquart (34 zu 25\textsuperscript{cm}) erſcheinen, und ich ſchätze ſeinen Umfang auf etwa
               320 Seiten. Der Preis – 100 Mark – ſcheint ein hoher, läßt ſich aber bei der
               geplanten Ausſtattung und der Kleinheit der Auflage – vermutlich nur hundert
               Exemplare – niedriger nicht ſtellen. Durch eine liebens{[}würdige{]}
               Zeichnung eines Exemplars würden Sie mir eine dankenswerte Hülfe gewähren! Dürfte ich
               Sie um eine ſolche bitten? Falls ja, ſo bäte ich um freundliche Zuſtellung der Hälfte
               des Betrages, mit der ferneren Bitte, mir den Reſt nach Verſendung des Werkes
               anweiſen zu wollen, die pünktlich am erſten Oktober erfolgen würde.\pend
           
\pstart
           In beſonderer Hochſchätzung{\\[\baselineskip]}Ihr{\\[\baselineskip]}ganz ergebenſter{\\[\baselineskip]}\spacefill\mbox{ArnoHolz}\pend
           \leftskip=0em{}
\pstart
           PS. Als Schlußvermerk – das Eingeklammerte ausgedruckt – käme auf die letzte
               Seite:\pend
           
\pstart
           »Dieſes Werk wurde im Sommer 1917 durch die Druckerei von Fletzſchke und Gretſchel\orgindex{Petzschke und Gretschel@Petzschke {\kaufmannsund}  Gretschel|pw} in Dresden\oindex{Dresden@\textbf{Dresden}, \emph{P.PPLA}|pw} im Auftrage des Verfaſſers für (Zahl) Subſkribenten
               hergeſtellt und nach deren alphabetiſcher Folge numeriert; das vorliegende Exemplar
               iſt das (Zahl)te und Eigentum von (Name, Ort).« –\pend
           
\pstart
           Sollte es Ihnen zugleich möglich ſein, mir freundlichſt auch noch den einen oder
               andern weiteren Subſkribenten zu beſchaffen, ſo wäre ich Ihnen dafür ganz beſonders
               dankbar!\pend
           \selectlanguage{ngerman}\endnumbering\briefempfaengerindex{Schnitzler, Arthur@\textsc{Schnitzler, Arthur}!zzzHolz, Arno@\emph{von Arno Holz}!1917-02-171@{11. 2. 1917}|)be}\mylabel{L02255h}  \normalsize

\doendnotes{C}
\bigskip
\vfill

\clearpage

\footnotesize

\lohead{\textsc{register}}

% Definiere theindex-Environment komplett neu ohne reledmac
\makeatletter
\renewenvironment{theindex}{%
  \section*{\indexname}%
  \setlength{\parindent}{0pt}%
  \setlength{\parskip}{0pt plus 0.3pt}%
  \let\item\@idxitem
}{%
  \clearpage
}
\makeatother

\IfFileExists{\jobname-pw.ind}{\input{\jobname-pw.ind}}{}

\end{document}

      