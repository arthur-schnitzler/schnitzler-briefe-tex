%% latex-leseansicht-vorspann.tex
%% Vorspann für die Leseansicht.
%% Lädt die gemeinsame Datei latex-vorspann.tex mit nicht gesetztem Schalter.

\newif\ifkorrekturansicht
\korrekturansichtfalse

\input{../tex-inputs/latex-vorspann}


               \section[Arno Holz an Arthur Schnitzler, 11. 2. 1917]{ Arno Holz an Arthur Schnitzler, 11. 2. 1917}\nopagebreak\mylabel{v}\rehead{ }\begin{ledgroupsized}[t]{13cm}\normalsize\beginnumbering\briefempfaengerindex{Schnitzler, Arthur@\textsc{Schnitzler, Arthur}!zzzHolz, Arno@\emph{von Arno Holz}!1917-02-171@{11. 2. 1917}|(be} \toendnotes[C]{\smallbreak\pagebreak[2]} \Standort{DLA, A:Schnitzler, HS.NZ85.1.5728.}
\physDesc{Brief, 1 Blatt, 1 Seite, Fotokopie
\newline{}Schnitzler: mit (mutmaßlich) rotem Buntstift beschriftet: »\textcolor{gray}{\textbf{Holz}}« }\pstart
           \raggedleft{}{\pb}Berlin W, 30. Stübbenſtr. 5\oindex{Stuebbenstrasse@\textbf{Stübbenstraße}|pw}.\hspace*{1.5em}11. II. 17. \pend
           \pstart\center{}Sehr verehrter Herr Doktor!\pend\pstart
           Durch die Ungunſt der Zeitumſtände bin ich gezwungen von meinem ſatiriſchen
                    Gedichtwerk »Die Blechſchmiede\pwindex{Holz, Arno 26.04.1863 – 26.10.1929@\textsc{Holz, Arno} (26.04.1863 – 26.10.1929), \emph{Schriftsteller}!Blechschmiede1902@\strich\emph{Die Blechschmiede} {[}1902{]}|pw}« (Leipzig\oindex{Leipzig@\textbf{Leipzig}|pw}, Inſel-Verlag\orgindex{Insel-Verlag@Insel-Verlag|pw} vergriffen) die neue, ſtark über das doppelte vermehrte
                    Ausgabe \uline{lediglich auf private Subſkription}
                    herauszugeben. Das Werk ſoll mit einer ſchönen Type auf gutem Bütten in
                    Großquart (34 zu 25\textsuperscript{cm}) erſcheinen, und ich ſchätze
                    ſeinen Umfang auf etwa 320 Seiten. Der Preis – 100 Mark – ſcheint ein hoher,
                    läßt ſich aber bei der geplanten Ausſtattung und der Kleinheit der Auflage –
                    vermutlich nur hundert Exemplare – niedriger nicht ſtellen. Durch eine
                        liebens{[}würdige{]} Zeichnung eines Exemplars würden Sie mir
                    eine dankenswerte Hülfe gewähren! Dürfte ich Sie um eine ſolche bitten? Falls
                    ja, ſo bäte ich um freundliche Zuſtellung der Hälfte des Betrages, mit der
                    ferneren Bitte, mir den Reſt nach Verſendung des Werkes anweiſen zu wollen, die
                    pünktlich am erſten Oktober erfolgen würde.\pend
           \pstart
           In beſonderer Hochſchätzung{\\[\baselineskip]}Ihr{\\[\baselineskip]}ganz ergebenſter{\\[\baselineskip]}\spacefill\mbox{ArnoHolz}\pend
           \leftskip=0em{}\pstart
           PS. Als Schlußvermerk – das Eingeklammerte ausgedruckt – käme auf die letzte
                    Seite:\pend
           \pstart
           »Dieſes Werk wurde im Sommer 1917 durch die Druckerei von Fletzſchke und Gretſchel\orgindex{Petzschke und Gretschel@Petzschke {\kaufmannsund}  Gretschel|pw} in Dresden\oindex{Dresden@\textbf{Dresden}|pw} im Auftrage des Verfaſſers für (Zahl)
                    Subſkribenten hergeſtellt und nach deren alphabetiſcher Folge numeriert; das
                    vorliegende Exemplar iſt das (Zahl)te und Eigentum von (Name, Ort).« –\pend
           \pstart
           Sollte es Ihnen zugleich möglich ſein, mir freundlichſt auch noch den einen oder
                    andern weiteren Subſkribenten zu beſchaffen, ſo wäre ich Ihnen dafür ganz
                    beſonders dankbar!\pend
                     \endnumbering\briefempfaengerindex{Schnitzler, Arthur@\textsc{Schnitzler, Arthur}!zzzHolz, Arno@\emph{von Arno Holz}!1917-02-171@{11. 2. 1917}|)be}\mylabel{h}\end{ledgroupsized}  \newcommand{\dateiname}{L02255}\newcommand{\titel}{Arno Holz an Arthur Schnitzler, 11. 2. 1917}\newcommand{\editorInnen}{Martin Anton Müller und Gerd-Hermann Susen}%% latex-leseansicht-abspann.tex
%% Abspann für die Leseansicht.
%% Der Schalter \ifkorrekturansicht ist bereits durch den Vorspann gesetzt.

%% latex-abspann.tex
%% Gemeinsamer Abspann für Korrekturansicht und Leseansicht.
%% Setzt den Schalter \ifkorrekturansicht voraus (gesetzt in den
%% einbindenden Dateien latex-korrekturansicht-abspann.tex bzw.
%% latex-leseansicht-abspann.tex).
%% ---------------------------------------------------------------

\normalsize

% Das esempio-Environment wird nur in der Leseansicht benötigt
\ifkorrekturansicht\else
\newenvironment{esempio}[3]%
{
    \vspace{1.5ex}
    \rlap{\underline{#1}}
    \par
    \setlength{\parindent}{0cm}
    \nopagebreak
    \leftskip=#2cm
    \rightskip=#3cm
}
{
    \par
}
\fi

\doendnotes{C}
\bigskip
\vfill

\clearpage

\footnotesize

\ifkorrekturansicht
  \lohead{\textsc{register}}
\fi

% theindex-Environment neu definieren ohne reledmac
\makeatletter
\renewenvironment{theindex}{%
  \ifkorrekturansicht
    \section*{\indexname}%
  \else
    \subsubsection*{Index der erwähnten Entitäten}%
  \fi
  \setlength{\parindent}{0pt}%
  \setlength{\parskip}{0pt plus 0.3pt}%
  \let\item\@idxitem
}{%
  \ifkorrekturansicht\clearpage\fi
}
\makeatother

\IfFileExists{\jobname-pw.ind}{\input{\jobname-pw.ind}}{}

% Quellenangabe nur in der Leseansicht
\ifkorrekturansicht\else
% Fallback-Definitionen, falls die .tex-Datei \titel etc. nicht gesetzt hat
\providecommand{\titel}{}
\providecommand{\editorInnen}{}
\providecommand{\dateiname}{\jobname}

\vspace{3cm}

\vfill

\footnotesize
\textsc{Quelle}: \titel. Herausgegeben von {\editorInnen}. In: \emph{Arthur Schnitzler: Briefwechsel mit Autorinnen und Autoren}.
 Digitale Edition, https://schnitzler-briefe.acdh.oeaw.ac.at/{\dateiname}.html (Stand \today)
\fi

\end{document}


      