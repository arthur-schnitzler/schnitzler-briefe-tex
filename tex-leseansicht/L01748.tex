%% latex-korrekturansicht-vorspann.tex
%% Vorspann für die Korrekturansicht.
%% Lädt die gemeinsame Datei latex-vorspann.tex mit gesetztem Schalter.

\newif\ifkorrekturansicht
\korrekturansichttrue

\input{../tex-inputs/latex-vorspann}


\section[Olga Schnitzler an Richard Beer-Hofmann, {[}1908?{]}]{L01748 Olga Schnitzler an Richard Beer-Hofmann, {[}1908?{]}}
\nopagebreak\mylabel{L01748v}
\rehead{ }\normalsize\beginnumbering\briefempfaengerindex{Beer-Hofmann, Richard@\textsc{Beer-Hofmann, Richard}!zzzSchnitzler, Olga@\emph{von Olga Schnitzler}!1908-12-311@{{[}1908?{]}}|(be}
\toendnotes[C]{\smallbreak\pagebreak[2]}\Standort{YCGL, MSS 31.}
\physDesc{Visitenkarte, 33 Zeichen
\newline{}Handschrift: schwarze Tinte, lateinische Kurrent}\toendnotes[C]{\smallbreak}
\pstart
           \centering{}{\pb}\textcolor{gray}{\textbf{\strikeout{\label{T_L01748-1v}\edtext{Frau}{\lemma{\textnormal{\emph{Frau}}}\Cendnote{\textnormal{diese und die folgende Streichung mit schwarzer
                           Tinte}}}\label{T_L01748-1}} Olga \strikeout{Schnitzler}}}\pend
           \vspace{0.5em}
\pstart
           Mit bestem \label{K_L01748-1v}\edtext{Dank}{\lemma{\textnormal{\emph{Dank}}}\Cendnote{\textnormal{Die Datierung bezieht sich allein auf die
                  Aufbewahrung im Archiv mit den Korrespondenzstücken des Jahres
                  1908.}}}\label{K_L01748-1} und vielen Grüssen!\pend
           \selectlanguage{ngerman}\endnumbering\briefempfaengerindex{Beer-Hofmann, Richard@\textsc{Beer-Hofmann, Richard}!zzzSchnitzler, Olga@\emph{von Olga Schnitzler}!1908-01-011@{{[}1908?{]}}|)be}\mylabel{L01748h}  \normalsize

\doendnotes{C}
\bigskip
\vfill

\clearpage

\footnotesize

\lohead{\textsc{register}}

% Definiere theindex-Environment komplett neu ohne reledmac
\makeatletter
\renewenvironment{theindex}{%
  \section*{\indexname}%
  \setlength{\parindent}{0pt}%
  \setlength{\parskip}{0pt plus 0.3pt}%
  \let\item\@idxitem
}{%
  \clearpage
}
\makeatother

\IfFileExists{\jobname-pw.ind}{\input{\jobname-pw.ind}}{}

\end{document}

      