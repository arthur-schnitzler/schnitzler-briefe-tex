%% latex-korrekturansicht-vorspann.tex
%% Vorspann für die Korrekturansicht.
%% Lädt die gemeinsame Datei latex-vorspann.tex mit gesetztem Schalter.

\newif\ifkorrekturansicht
\korrekturansichttrue

\input{../tex-inputs/latex-vorspann}


\section[Olga und Arthur Schnitzler an Richard und Paula Beer-Hofmann, 26. 6. 1908]{L01778 Olga und Arthur Schnitzler an Richard und Paula Beer-Hofmann,
               26. 6. 1908}
\nopagebreak\mylabel{L01778v}
\rehead{ }\normalsize\beginnumbering\briefempfaengerindex{Beer-Hofmann, Paula@\textsc{Beer-Hofmann, Paula}!zzzSchnitzler, Arthur@\emph{von Arthur Schnitzler}!1908-06-261@{26. 6. 1908}|(be}\briefempfaengerindex{Beer-Hofmann, Paula@\textsc{Beer-Hofmann, Paula}!zzzSchnitzler, Olga@\emph{von Olga Schnitzler}!1908-06-261@{26. 6. 1908}|(be}\briefempfaengerindex{Beer-Hofmann, Richard@\textsc{Beer-Hofmann, Richard}!zzzSchnitzler, Arthur@\emph{von Arthur Schnitzler}!1908-06-261@{26. 6. 1908}|(be}\briefempfaengerindex{Beer-Hofmann, Richard@\textsc{Beer-Hofmann, Richard}!zzzSchnitzler, Olga@\emph{von Olga Schnitzler}!1908-06-261@{26. 6. 1908}|(be}
\toendnotes[C]{\smallbreak\pagebreak[2]}\Standort{YCGL, MSS 31.}
\physDesc{Bildpostkarte, 555 Zeichen
\newline{}Handschrift Arthur Schnitzler: schwarze Tinte, deutsche Kurrent
\newline{}Handschrift Olga Schnitzler: schwarze Tinte, lateinische Kurrent
\newline{}Versand: Stempel: »\nobreak{}\oindex{Seis am Schlern@\textbf{Seis am Schlern}, \emph{P.PPL}|pwk}Seis, 26. 6. {[}1906{]}\nobreak{}«.  }\toendnotes[C]{\smallbreak}\pstart{}{\pb}Herrn u. Frau\pend{}\pstart{}D\textsuperscript{r} Richard Beer-Hofmann\pend{}\pstart{}Wien XVIII\oindex{XVIII., Waehring@\textbf{XVIII., Währing}, \emph{A.ADM3}|pw}\pend{}\pstart{}Hasenauserstrasse 59\oindex{Hasenauerstrasse 59@\textbf{Hasenauerstraße 59}, \emph{Wohngebäude (K.WHS)}|pw}.\pend{}{\bigskip}
\pstart
           \noindent{}\centering{}{\pb}\textcolor{gray}{\textbf{Tirol\oindex{Suedtirol@\textbf{Südtirol}, \emph{A.ADM2}|pw}: \label{T_L01778-1v}\edtext{\uline{Villa Heufler, Seis am Schlern}}{\lemma{\textnormal{\emph{Villa … Schlern}}}\Cendnote{\textnormal{Unterstreichung mit schwarzer
                        Tinte.}}}\label{T_L01778-1}\oindex{Villa Heufler@\textbf{Villa Heufler}, \emph{Beherbergungsgebäude (K.BHB)}|pw}, 1000m. Nach dem Aquarell\pwindex{Partie in Seis am Schlern@\emph{Partie in Seis am Schlern}|pwv} von F. A. C. M.
                     Reisch\pwindex{Reisch, Franz August Carl Maria 1862-05-01 – 1942?@\textsc{Reisch, Franz August Carl Maria} (1862-05-01 – 1942?), \emph{Maler/Malerin}|pw}, Meran\oindex{Meran@\textbf{Meran}, \emph{P.PPLA3}|pw}.}}\pend
           \vspace{1em}
\pstart
           \raggedleft{}{\pb}26. Juni 08.\pend
           \vspace{0.5em}
\pstart
           \label{T_L01778-2v}\edtext{Mein Fenster}{\lemma{\textnormal{\emph{Mein Fenster}}}\Cendnote{\textnormal{Ein Pfeil weist auf das Fenster links des Balkons im zweiten
                  Stock.}}}\label{T_L01778-2}\pend
           
\pstart
           \label{T_L01778-3v}\edtext{Heini\pwindex{Schnitzler, Heinrich 09.08.1902 – 12.07.1982@\textsc{Schnitzler, Heinrich} (09.08.1902 – 12.07.1982), \emph{Regisseur/Regisseurin, Schauspieler/Schauspielerin}|pw}}{\lemma{\textnormal{\emph{Heini}}}\Cendnote{\textnormal{Ein Pfeil weist auf das zweite Fenster
                  von links beim Balkon im zweiten Stock.}}}\label{T_L01778-3}\pend
           
\pstart
           Um’s Eck hab ich auch noch ein Fenster, daneben ist auch Arthurs Balcon-Zimmer.\pend
           
\pstart
           Salegg\oindex{Hotel Salegg@\textbf{Hotel Salegg}, \emph{Hotel (K.HTL)}|pw} war rasch erledigt, da der schlaue {\pb}Wirt\pwindex{Honeck, Michael 1864-09-04 – 1944-01-25@\textsc{Honeck, Michael} (1864-09-04 – 1944-01-25), \emph{Hotelbesitzer/Hotelbesitzerin}|pwv} die Mängel seines
               Hauses durch wohlwollendes Schulterklopfen zu ersetzen suchte.\pend
           
\pstart
           Wir speisen im gegenüberliegenden Seiserhof\oindex{Seiserhof@\textbf{Seiserhof}, \emph{Gastgewerbegebäude (K.GGW)}|pw},
               wollen lange bleiben. Heut ist Heini\pwindex{Schnitzler, Heinrich 09.08.1902 – 12.07.1982@\textsc{Schnitzler, Heinrich} (09.08.1902 – 12.07.1982), \emph{Regisseur/Regisseurin, Schauspieler/Schauspielerin}|pw}
                  einge\textcolor{gray}{t}roffen.\pend
           
\pstart
           Die Wälder sind unglaublich schön. Hoffentlich sind Sie ebenso zufrieden, aber wo???
               Die allerherzlichsten Wünsche und Grüsse, auch den Kindern\pwindex{Beer-Hofmann, Naemah 20.12.1898 – 10.11.1971@\textsc{Beer-Hofmann, Naëmah} (20.12.1898 – 10.11.1971)|pwv}\pwindex{Beer-Hofmann, Mirjam 04.09.1897 – 24.12.1984@\textsc{Beer-Hofmann, Mirjam} (04.09.1897 – 24.12.1984)|pwv}\pwindex{Beer-Hofmann, Gabriel 09.01.1901 – 24.03.1971@\textsc{Beer-Hofmann, Gabriel} (09.01.1901 – 24.03.1971), \emph{Schriftsteller/Schriftstellerin, Filmagent/Filmagentin}|pwv}.\pend
           \pstart \spacefill\mbox{O. S.}\pend{}\selectlanguage{ngerman}\vspace{1em}{\vspace{1\baselineskip}}
\pstart
           {\pb}{[}hs. :{]} Herzlichſt\pend
           \pstart \spacefill\mbox{Arthur}\pend{}\selectlanguage{ngerman}\endnumbering\briefempfaengerindex{Beer-Hofmann, Paula@\textsc{Beer-Hofmann, Paula}!zzzSchnitzler, Arthur@\emph{von Arthur Schnitzler}!1908-06-261@{26. 6. 1908}|)be}\briefempfaengerindex{Beer-Hofmann, Paula@\textsc{Beer-Hofmann, Paula}!zzzSchnitzler, Olga@\emph{von Olga Schnitzler}!1908-06-261@{26. 6. 1908}|)be}\briefempfaengerindex{Beer-Hofmann, Richard@\textsc{Beer-Hofmann, Richard}!zzzSchnitzler, Arthur@\emph{von Arthur Schnitzler}!1908-06-261@{26. 6. 1908}|)be}\briefempfaengerindex{Beer-Hofmann, Richard@\textsc{Beer-Hofmann, Richard}!zzzSchnitzler, Olga@\emph{von Olga Schnitzler}!1908-06-261@{26. 6. 1908}|)be}\mylabel{L01778h}  \normalsize

\doendnotes{C}
\bigskip
\vfill

\clearpage

\footnotesize

\lohead{\textsc{register}}

% Definiere theindex-Environment komplett neu ohne reledmac
\makeatletter
\renewenvironment{theindex}{%
  \section*{\indexname}%
  \setlength{\parindent}{0pt}%
  \setlength{\parskip}{0pt plus 0.3pt}%
  \let\item\@idxitem
}{%
  \clearpage
}
\makeatother

\IfFileExists{\jobname-pw.ind}{\input{\jobname-pw.ind}}{}

\end{document}

      