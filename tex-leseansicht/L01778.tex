%% latex-leseansicht-vorspann.tex
%% Vorspann für die Leseansicht.
%% Lädt die gemeinsame Datei latex-vorspann.tex mit nicht gesetztem Schalter.

\newif\ifkorrekturansicht
\korrekturansichtfalse

\input{../tex-inputs/latex-vorspann}


\section[Olga und Arthur Schnitzler an Richard und Paula Beer-Hofmann, 26. 6. 1908]{L01778 Olga und Arthur Schnitzler an Richard und Paula Beer-Hofmann, 26. 6. 1908}
\nopagebreak\mylabel{L01778v}
\rehead{ }\normalsize\beginnumbering\briefempfaengerindex{Beer-Hofmann, Paula@\textsc{Beer-Hofmann, Paula}!zzzSchnitzler, Arthur@\emph{von Arthur Schnitzler}!1908-06-261@{26. 6. 1908}|(be}\briefempfaengerindex{Beer-Hofmann, Paula@\textsc{Beer-Hofmann, Paula}!zzzSchnitzler, Olga@\emph{von Olga Schnitzler}!1908-06-261@{26. 6. 1908}|(be}\briefempfaengerindex{Beer-Hofmann, Richard@\textsc{Beer-Hofmann, Richard}!zzzSchnitzler, Arthur@\emph{von Arthur Schnitzler}!1908-06-261@{26. 6. 1908}|(be}\briefempfaengerindex{Beer-Hofmann, Richard@\textsc{Beer-Hofmann, Richard}!zzzSchnitzler, Olga@\emph{von Olga Schnitzler}!1908-06-261@{26. 6. 1908}|(be}
\toendnotes[C]{\smallbreak\pagebreak[2]}
\correspDesc{Versand  durch Olga Schnitzler, Arthur Schnitzler am 26. 6. 1908 in Seis am Schlern
\newline{}Erhalt  durch Richard Beer-Hofmann, Paula Beer-Hofmann im Zeitraum [27. 6. 1908
                  – 1. 7. 1908?] in Wien}\toendnotes[C]{\smallbreak}
\Standort{YCGL, MSS 31.}
\physDesc{Bildpostkarte, 555 Zeichen
\newline{}Handschrift Arthur Schnitzler: schwarze Tinte, deutsche Kurrent
\newline{}Handschrift Olga Schnitzler: schwarze Tinte, lateinische Kurrent
\newline{}Versand: Stempel: »\nobreak{}\oindex{Seis am Schlern@\textbf{Seis am Schlern}|pwk}Seis, 26. 6. {[}1906{]}\nobreak{}«.  }\toendnotes[C]{\smallbreak}\pstart{}{\pb}Herrn u. Frau\pend{}\pstart{}D\textsuperscript{r} Richard Beer-Hofmann\pend{}\pstart{}Wien XVIII\oindex{XVIII., Währing@\textbf{XVIII., Währing}, \emph{Verwaltungsgebiet}|pw}\pend{}\pstart{}Hasenauserstrasse 59\oindex{Wien@\textbf{Wien}!XVIII., Währing@\textbf{XVIII., Währing}!Hasenauerstraße 59@\textbf{Hasenauerstraße 59}, \emph{Wohngebäude}|pw}.\pend{}{\bigskip}
\pstart
           \noindent{}\centering{}{\pb}\textcolor{gray}{\textbf{Tirol\oindex{Südtirol@\textbf{Südtirol}, \emph{Verwaltungsgebiet}|pw}: \label{T_L01778-1v}\edtext{\uline{Villa Heufler, Seis am Schlern}}{\lemma{\textnormal{\emph{Villa … Schlern}}}\Cendnote{\textnormal{Unterstreichung mit schwarzer
                        Tinte.}}}\label{T_L01778-1}\oindex{Villa Heufler@\textbf{Villa Heufler}, \emph{Beherbergungsgebäude}|pw}, 1000m. Nach dem Aquarell\pwindex{Reisch, Franz August Carl Maria 1.\,5.\,1862 Wien – 1942? Meran@\textsc{Reisch, Franz August Carl Maria} (1.\,5.\,1862 Wien – 1942? Meran), \emph{Maler}!Partie in Seis am Schlern@\strich\emph{Partie in Seis am Schlern}|pwv} von F. A. C. M.
                     Reisch\pwindex{Reisch, Franz August Carl Maria 1.\,5.\,1862 Wien – 1942? Meran@\textsc{Reisch, Franz August Carl Maria} (1.\,5.\,1862 Wien – 1942? Meran), \emph{Maler}|pw}, Meran\oindex{Meran@\textbf{Meran}, \emph{Hauptstadt}|pw}.}}\pend
           \vspace{1em}
\pstart
           \raggedleft{}{\pb}26. Juni 08.\pend
           \vspace{0.5em}
\pstart
           \label{T_L01778-2v}\edtext{Mein Fenster}{\lemma{\textnormal{\emph{Mein Fenster}}}\Cendnote{\textnormal{Ein Pfeil weist auf das Fenster links des Balkons im zweiten
                  Stock.}}}\label{T_L01778-2}\pend
           
\pstart
           \label{T_L01778-3v}\edtext{Heini\pwindex{Schnitzler, Heinrich 9.\,8.\,1902 Hinterbrühl – 12.\,7.\,1982 Wien@\textsc{Schnitzler, Heinrich} (9.\,8.\,1902 Hinterbrühl – 12.\,7.\,1982 Wien), \emph{Regisseur, Schauspieler}|pw}}{\lemma{\textnormal{\emph{Heini}}}\Cendnote{\textnormal{Ein Pfeil weist auf das zweite Fenster
                  von links beim Balkon im zweiten Stock.}}}\label{T_L01778-3}\pend
           
\pstart
           Um’s Eck hab ich auch noch ein Fenster, daneben ist auch Arthurs Balcon-Zimmer.\pend
           
\pstart
           Salegg\oindex{Hotel Salegg@\textbf{Hotel Salegg}, \emph{Hotel}|pw} war rasch erledigt, da der schlaue {\pb}Wirt\pwindex{Honeck, Michael 4.\,9.\,1864 Bisamberg – 25.\,1.\,1944 Obermais@\textsc{Honeck, Michael} (4.\,9.\,1864 Bisamberg – 25.\,1.\,1944 Obermais), \emph{Hotelbesitzer}|pwv} die Mängel seines
               Hauses durch wohlwollendes Schulterklopfen zu ersetzen suchte.\pend
           
\pstart
           Wir speisen im gegenüberliegenden Seiserhof\oindex{Seiserhof@\textbf{Seiserhof}, \emph{Gastgewerbegebäude}|pw},
               wollen lange bleiben. Heut ist Heini\pwindex{Schnitzler, Heinrich 9.\,8.\,1902 Hinterbrühl – 12.\,7.\,1982 Wien@\textsc{Schnitzler, Heinrich} (9.\,8.\,1902 Hinterbrühl – 12.\,7.\,1982 Wien), \emph{Regisseur, Schauspieler}|pw}
                  einge\textcolor{gray}{t}roffen.\pend
           
\pstart
           Die Wälder sind unglaublich schön. Hoffentlich sind Sie ebenso zufrieden, aber wo???
               Die allerherzlichsten Wünsche und Grüsse, auch den Kindern\pwindex{Beer-Hofmann, Naëmah 20.\,12.\,1898 Wien – 10.\,11.\,1971 New York City@\textsc{Beer-Hofmann, Naëmah} (20.\,12.\,1898 Wien – 10.\,11.\,1971 New York City)|pwv}\pwindex{Beer-Hofmann, Mirjam 4.\,9.\,1897 Wien – 24.\,12.\,1984 New York City@\textsc{Beer-Hofmann, Mirjam} (4.\,9.\,1897 Wien – 24.\,12.\,1984 New York City)|pwv}\pwindex{Beer-Hofmann, Gabriel 9.\,1.\,1901 Wien – 24.\,3.\,1971 St Albans@\textsc{Beer-Hofmann, Gabriel} (9.\,1.\,1901 Wien – 24.\,3.\,1971 St Albans), \emph{Schriftsteller, Filmagent}|pwv}.\pend
           \pstart \spacefill\mbox{O. S.}\pend{}\selectlanguage{ngerman}\vspace{1em}{\vspace{1\baselineskip}}
\pstart
           {\pb}{[}hs. Schnitzler:{]} Herzlichſt\pend
           \pstart \spacefill\mbox{Arthur}\pend{}\selectlanguage{ngerman}\endnumbering\briefempfaengerindex{Beer-Hofmann, Paula@\textsc{Beer-Hofmann, Paula}!zzzSchnitzler, Arthur@\emph{von Arthur Schnitzler}!1908-06-261@{26. 6. 1908}|)be}\briefempfaengerindex{Beer-Hofmann, Paula@\textsc{Beer-Hofmann, Paula}!zzzSchnitzler, Olga@\emph{von Olga Schnitzler}!1908-06-261@{26. 6. 1908}|)be}\briefempfaengerindex{Beer-Hofmann, Richard@\textsc{Beer-Hofmann, Richard}!zzzSchnitzler, Arthur@\emph{von Arthur Schnitzler}!1908-06-261@{26. 6. 1908}|)be}\briefempfaengerindex{Beer-Hofmann, Richard@\textsc{Beer-Hofmann, Richard}!zzzSchnitzler, Olga@\emph{von Olga Schnitzler}!1908-06-261@{26. 6. 1908}|)be}\mylabel{L01778h}  \newcommand{\dateiname}{L01778}\newcommand{\titel}{Olga und Arthur Schnitzler an Richard und Paula Beer-Hofmann, 26. 6. 1908}\newcommand{\editorInnen}{Martin Anton Müller und Gerd-Hermann Susen}%% latex-leseansicht-abspann.tex
%% Abspann für die Leseansicht.
%% Der Schalter \ifkorrekturansicht ist bereits durch den Vorspann gesetzt.

%% latex-abspann.tex
%% Gemeinsamer Abspann für Korrekturansicht und Leseansicht.
%% Setzt den Schalter \ifkorrekturansicht voraus (gesetzt in den
%% einbindenden Dateien latex-korrekturansicht-abspann.tex bzw.
%% latex-leseansicht-abspann.tex).
%% ---------------------------------------------------------------

\normalsize

% Das esempio-Environment wird nur in der Leseansicht benötigt
\ifkorrekturansicht\else
\newenvironment{esempio}[3]%
{
    \vspace{1.5ex}
    \rlap{\underline{#1}}
    \par
    \setlength{\parindent}{0cm}
    \nopagebreak
    \leftskip=#2cm
    \rightskip=#3cm
}
{
    \par
}
\fi

\doendnotes{C}
\bigskip
\vfill

\clearpage

\footnotesize

\ifkorrekturansicht
  \lohead{\textsc{register}}
\fi

% theindex-Environment neu definieren ohne reledmac
\makeatletter
\renewenvironment{theindex}{%
  \ifkorrekturansicht
    \section*{\indexname}%
  \else
    \subsubsection*{Index der erwähnten Entitäten}%
  \fi
  \setlength{\parindent}{0pt}%
  \setlength{\parskip}{0pt plus 0.3pt}%
  \let\item\@idxitem
}{%
  \ifkorrekturansicht\clearpage\fi
}
\makeatother

\IfFileExists{\jobname-pw.ind}{\input{\jobname-pw.ind}}{}

% Quellenangabe nur in der Leseansicht
\ifkorrekturansicht\else
% Fallback-Definitionen, falls die .tex-Datei \titel etc. nicht gesetzt hat
\providecommand{\titel}{}
\providecommand{\editorInnen}{}
\providecommand{\dateiname}{\jobname}

\vspace{3cm}

\vfill

\footnotesize
\textsc{Quelle}: \titel. Herausgegeben von {\editorInnen}. In: \emph{Arthur Schnitzler: Briefwechsel mit Autorinnen und Autoren}.
 Digitale Edition, https://schnitzler-briefe.acdh.oeaw.ac.at/{\dateiname}.html (Stand \today)
\fi

\end{document}


