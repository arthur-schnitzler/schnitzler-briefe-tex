%% latex-leseansicht-vorspann.tex
%% Vorspann für die Leseansicht.
%% Lädt die gemeinsame Datei latex-vorspann.tex mit nicht gesetztem Schalter.

\newif\ifkorrekturansicht
\korrekturansichtfalse

\input{../tex-inputs/latex-vorspann}


         
         \renewcommand{\erwaehntePersonen}{Personen: Richard Beer-Hofmann, Paula Beer-Hofmann, Naëmah Beer-Hofmann, Mirjam Beer-Hofmann, Gabriel Beer-Hofmann, Michael Honeck, Franz August Carl Maria Reisch, Olga Schnitzler, Heinrich Schnitzler}
         \renewcommand{\erwaehnteOrte}{Orte: Hasenauerstraße, Hotel Salegg, Meran, Seis am Schlern, Seiserhof, Südtirol, Villa Heufler, Wien, XVIII., Währing}
         \renewcommand{\erwaehnteWerke}{Werke: Partie in Seis am Schlern}
               \section[Olga und Arthur Schnitzler an Richard und Paula Beer-Hofmann, 26. 6. 1908]{ Olga und Arthur Schnitzler an Richard und Paula Beer-Hofmann,
               26. 6. 1908}\nopagebreak\mylabel{v}\rehead{ }\begin{ledgroupsized}[t]{13cm}\normalsize\beginnumbering\briefempfaengerindex{Beer-Hofmann, Paula@\textsc{Beer-Hofmann, Paula}!zzzSchnitzler, Arthur@\emph{von Arthur Schnitzler}!1908-06-261@{26. 6. 1908}|(be}\briefempfaengerindex{Beer-Hofmann, Paula@\textsc{Beer-Hofmann, Paula}!zzzSchnitzler, Olga@\emph{von Olga Schnitzler}!1908-06-261@{26. 6. 1908}|(be}\briefempfaengerindex{Beer-Hofmann, Richard@\textsc{Beer-Hofmann, Richard}!zzzSchnitzler, Arthur@\emph{von Arthur Schnitzler}!1908-06-261@{26. 6. 1908}|(be}\briefempfaengerindex{Beer-Hofmann, Richard@\textsc{Beer-Hofmann, Richard}!zzzSchnitzler, Olga@\emph{von Olga Schnitzler}!1908-06-261@{26. 6. 1908}|(be} \toendnotes[C]{\smallbreak\pagebreak[2]} \Standort{YCGL, MSS 31.}
\physDesc{Bildpostkarte, 555 Zeichen
\newline{}Handschrift Arthur Schnitzler: schwarze Tinte, deutsche Kurrent\newline{}Handschrift Olga Schnitzler: schwarze Tinte, lateinische Kurrent
\newline{}Versand: Stempel: »\nobreak{}\oindex{Seis am Schlern@\textbf{Seis am Schlern}|pwk}Seis, 26. 6. {[}1906{]}\nobreak{}«.  }\toendnotes[C]{\smallbreak}\pstart{}{\pb}Herrn u. Frau\pend{}\pstart{}D\textsuperscript{r} Richard Beer-Hofmann\pend{}\pstart{}Wien XVIII\oindex{XVIII., Waehring@\textbf{XVIII., Währing}|pw}\pend{}\pstart{}Hasenauserstrasse 59\oindex{Hasenauerstrasse@\textbf{Hasenauerstraße}|pw}.\pend{}{\bigskip}\pstart
           \noindent{}\centering{}{\pb}\textcolor{gray}{\textbf{Tirol\oindex{Suedtirol@\textbf{Südtirol}|pw}: \label{T_L01778-1v}\edtext{\uline{Villa Heufler, Seis am Schlern}}{\lemma{\textnormal{\emph{Villa … Schlern}}}\Cendnote{\textnormal{Unterstreichung mit schwarzer
                           Tinte.}}}\label{T_L01778-1h}\oindex{Villa Heufler@\textbf{Villa Heufler}|pw}, 1000m. Nach dem Aquarell\pwindex{Reisch, Franz August Carl Maria 1862-05-01 – 1942?@\textsc{Reisch, Franz August Carl Maria} (1862-05-01 – 1942?), \emph{Maler}!Partie in Seis am Schlern@\strich\emph{Partie in Seis am Schlern}|pwv} von F. A. C. M.
                     Reisch\pwindex{Reisch, Franz August Carl Maria 1862-05-01 – 1942?@\textsc{Reisch, Franz August Carl Maria} (1862-05-01 – 1942?), \emph{Maler}|pw}, Meran\oindex{Meran@\textbf{Meran}|pw}.}}\pend
           \pstart
           \raggedleft{}26. Juni 08.\pend
           \pstart
           \label{T_L01778-2v}\edtext{Mein Fenster}{\lemma{\textnormal{\emph{Mein Fenster}}}\Cendnote{\textnormal{Ein Pfeil weist auf das Fenster links des Balkons im zweiten
                  Stock.}}}\label{T_L01778-2h}\pend
           \pstart
           \label{T_L01778-3v}\edtext{Heini\pwindex{Schnitzler, Heinrich 09.08.1902 – 12.07.1982@\textsc{Schnitzler, Heinrich} (09.08.1902 – 12.07.1982), \emph{Regisseur, Schauspieler}|pw}}{\lemma{\textnormal{\emph{Heini}}}\Cendnote{\textnormal{Ein Pfeil weist auf das zweite Fenster
                  von links beim Balkon im zweiten Stock.}}}\label{T_L01778-3h}\pend
           \pstart
           Um’s Eck hab ich auch noch ein Fenster, daneben ist auch Arthurs Balcon-Zimmer.\pend
           \pstart
           Salegg\oindex{Hotel Salegg@\textbf{Hotel Salegg}|pw} war rasch erledigt, da der schlaue {\pb}Wirt\pwindex{Honeck, Michael 1864-09-04 – 1944-01-25@\textsc{Honeck, Michael} (1864-09-04 – 1944-01-25), \emph{Hotelbesitzer}|pwv} die Mängel seines
               Hauses durch wohlwollendes Schulterklopfen zu ersetzen suchte.\pend
           \pstart
           Wir speisen im gegenüberliegenden Seiserhof\oindex{Seiserhof@\textbf{Seiserhof}|pw},
               wollen lange bleiben. Heut ist Heini\pwindex{Schnitzler, Heinrich 09.08.1902 – 12.07.1982@\textsc{Schnitzler, Heinrich} (09.08.1902 – 12.07.1982), \emph{Regisseur, Schauspieler}|pw}
                  einge\textcolor{gray}{t}roffen.\pend
           \pstart
           Die Wälder sind unglaublich schön. Hoffentlich sind Sie ebenso zufrieden, aber wo???
               Die allerherzlichsten Wünsche und Grüsse, auch den Kindern\pwindex{Beer-Hofmann, Naemah 20.12.1898 – 10.11.1971@\textsc{Beer-Hofmann, Naëmah} (20.12.1898 – 10.11.1971)|pwv}\pwindex{Beer-Hofmann, Mirjam 04.09.1897 – 24.12.1984@\textsc{Beer-Hofmann, Mirjam} (04.09.1897 – 24.12.1984)|pwv}\pwindex{Beer-Hofmann, Gabriel 09.01.1901 – 24.03.1971@\textsc{Beer-Hofmann, Gabriel} (09.01.1901 – 24.03.1971), \emph{Schriftsteller, Filmagent}|pwv}.\pend
           \pstart \spacefill\mbox{O. S.}\pend{}{\bigskip}\pstart
           \noindent{}{\pb}{[}hs. Arthur Schnitzler:{]} Herzlichſt\pend
           \pstart \spacefill\mbox{Arthur}\pend{}
         
         \endnumbering\mylabel{h}\end{ledgroupsized}  \newcommand{\dateiname}{L01778}\newcommand{\titel}{Olga und Arthur Schnitzler an Richard und Paula Beer-Hofmann, 26. 6. 1908}\newcommand{\editorInnen}{Martin Anton Müller und Gerd-Hermann Susen}%% latex-leseansicht-abspann.tex
%% Abspann für die Leseansicht.
%% Der Schalter \ifkorrekturansicht ist bereits durch den Vorspann gesetzt.

%% latex-abspann.tex
%% Gemeinsamer Abspann für Korrekturansicht und Leseansicht.
%% Setzt den Schalter \ifkorrekturansicht voraus (gesetzt in den
%% einbindenden Dateien latex-korrekturansicht-abspann.tex bzw.
%% latex-leseansicht-abspann.tex).
%% ---------------------------------------------------------------

\normalsize

% Das esempio-Environment wird nur in der Leseansicht benötigt
\ifkorrekturansicht\else
\newenvironment{esempio}[3]%
{
    \vspace{1.5ex}
    \rlap{\underline{#1}}
    \par
    \setlength{\parindent}{0cm}
    \nopagebreak
    \leftskip=#2cm
    \rightskip=#3cm
}
{
    \par
}
\fi

\doendnotes{C}
\bigskip
\vfill

\clearpage

\footnotesize

\ifkorrekturansicht
  \lohead{\textsc{register}}
\fi

% theindex-Environment neu definieren ohne reledmac
\makeatletter
\renewenvironment{theindex}{%
  \ifkorrekturansicht
    \section*{\indexname}%
  \else
    \subsubsection*{Index der erwähnten Entitäten}%
  \fi
  \setlength{\parindent}{0pt}%
  \setlength{\parskip}{0pt plus 0.3pt}%
  \let\item\@idxitem
}{%
  \ifkorrekturansicht\clearpage\fi
}
\makeatother

\IfFileExists{\jobname-pw.ind}{\input{\jobname-pw.ind}}{}

% Quellenangabe nur in der Leseansicht
\ifkorrekturansicht\else
% Fallback-Definitionen, falls die .tex-Datei \titel etc. nicht gesetzt hat
\providecommand{\titel}{}
\providecommand{\editorInnen}{}
\providecommand{\dateiname}{\jobname}

\vspace{3cm}

\vfill

\footnotesize
\textsc{Quelle}: \titel. Herausgegeben von {\editorInnen}. In: \emph{Arthur Schnitzler: Briefwechsel mit Autorinnen und Autoren}.
 Digitale Edition, https://schnitzler-briefe.acdh.oeaw.ac.at/{\dateiname}.html (Stand \today)
\fi

\end{document}


      