%% latex-leseansicht-vorspann.tex
%% Vorspann für die Leseansicht.
%% Lädt die gemeinsame Datei latex-vorspann.tex mit nicht gesetztem Schalter.

\newif\ifkorrekturansicht
\korrekturansichtfalse

\input{../tex-inputs/latex-vorspann}


\section[ Arthur Schnitzler an Paul Goldmann, 1. 2. 1911]{L03521 Arthur Schnitzler an Paul Goldmann,  1. 2. 1911}
\nopagebreak\mylabel{L03521v}
\rehead{ }\normalsize\beginnumbering\briefempfaengerindex{Goldmann, Paul@\textsc{Goldmann, Paul}!zzzSchnitzler, Arthur@\emph{von Arthur Schnitzler}!1911-02-011@{1. 2. 1911}|(be}
\toendnotes[C]{\smallbreak\pagebreak[2]}
\correspDesc{Versand  durch Arthur Schnitzler am 1. 2. 1911 in Wien
\newline{}Erhalt  durch Paul Goldmann im Zeitraum [2. 2. 1911
                  – 6. 2. 1911?] in Berlin}\toendnotes[C]{\smallbreak}
\Standort{CUL, Schnitzler, A 20.}
\physDesc{Brief, Durchschlag, 7 Blätter, 13 Seiten, 14381 Zeichen
\newline{}Schreibmaschine
\newline{}Handschrift Arthur Schnitzler: 1) roter Buntstift, deutsche Kurrent (\noindent{}»\uline{Goldmann\pwindex{Goldmann, Paul 31.\,1.\,1865 Breslau – 25.\,9.\,1935 Wien@\textsc{Goldmann, Paul} (31.\,1.\,1865 Breslau – 25.\,9.\,1935 Wien), \emph{Schriftsteller, Journalist}|pw}}«)\hspace{1em}2) Bleistift, lateinische Kurrent (\noindent{}kleinere Korrekturen, Schlussformel und Unterschrift)\hspace{1em}3) schwarze Tinte, lateinische Kurrent (\noindent{}eine Ersetzung)\hspace{1em}
\newline{}Handschrift Schreibkraft: Bleistift, lateinische Kurrent (\noindent{}Korrekturen, Ergänzung des letzten Satzes der sechsten Seite)}\toendnotes[C]{\smallbreak}
\pstart
           \raggedleft{}{\pb}1. 2. 1911.\pend
           \vspace{0.5em}
\pstart
           Gewiss, lieber Freund, schon in Deinen \label{K_L03521-1v}\edtext{Briefen}{\lemma{\textnormal{\emph{Briefen}}}\Cendnote{\textnormal{Am 14. 1. 1911 hatte Schnitzler{ }Goldmanns\pwindex{Goldmann, Paul 31.\,1.\,1865 Breslau – 25.\,9.\,1935 Wien@\textsc{Goldmann, Paul} (31.\,1.\,1865 Breslau – 25.\,9.\,1935 Wien), \emph{Schriftsteller, Journalist}|pwk} Brief vom XXXX Auszeichnungsfehler: Dokument L03475 nicht gefunden erhalten. Am
                  Folgetag, dem 15. 1. 1911, hatte er begonnen, Notizen für seine Antwort anzulegen.
                  Der Sonderstatus, den für Schnitzler sein
                  Zerwürfnis mit Goldmann\pwindex{Goldmann, Paul 31.\,1.\,1865 Breslau – 25.\,9.\,1935 Wien@\textsc{Goldmann, Paul} (31.\,1.\,1865 Breslau – 25.\,9.\,1935 Wien), \emph{Schriftsteller, Journalist}|pwk} eingenommen hat, drückt
                  sich in Abweichungen beim Verwahren der Korrespondenzstücke auf. So stellen Goldmanns\pwindex{Goldmann, Paul 31.\,1.\,1865 Breslau – 25.\,9.\,1935 Wien@\textsc{Goldmann, Paul} (31.\,1.\,1865 Breslau – 25.\,9.\,1935 Wien), \emph{Schriftsteller, Journalist}|pwk} Briefe mit Schnitzler die umfangreichste berufliche Korrespondenz dar,
                  die sich nicht in der \emph{Cambridge University Library} (\emph{CUL}) aufbewahrt findet. Schnitzler ließ
                  auch keine Abschrift der an ihn gesandten Schreiben herstellen. Und der
                  Verwahrort des vorliegenden Antwortschreibens ist ungewöhnlich. Es wird nicht bei
                  den Briefdurchschlägen im \emph{Deutschen Literaturarchiv Marbach} aufbewahrt, sondern findet sich im literarischen Nachlass in der \emph{CUL}.}}}\label{K_L03521-1} hattest Du allerlei Bedenken gegen die »Beatrice\pwindex{Schnitzler, Arthur 15.\,5.\,1862 Wien – 21.\,10.\,1931 ebd.@\textsc{Schnitzler, Arthur} (15.\,5.\,1862 Wien – 21.\,10.\,1931 ebd.), \emph{Schriftsteller, Mediziner}!Schleier der Beatrice. Schauspiel in fünf Akten@\strich\emph{Der Schleier der Beatrice. Schauspiel in fünf Akten}|pw}« ausgesprochen; und in Deinem Feuilleton\pwindex{Goldmann, Paul 31.\,1.\,1865 Breslau – 25.\,9.\,1935 Wien@\textsc{Goldmann, Paul} (31.\,1.\,1865 Breslau – 25.\,9.\,1935 Wien), \emph{Schriftsteller, Journalist}!Berliner Theater. (»Der Schleier der Beatrice« von Arthur Schnitzler.)@\strich\emph{Berliner Theater. (»Der Schleier der Beatrice« von Arthur Schnitzler.)}|pwv} über dasselbe Stück\pwindex{Schnitzler, Arthur 15.\,5.\,1862 Wien – 21.\,10.\,1931 ebd.@\textsc{Schnitzler, Arthur} (15.\,5.\,1862 Wien – 21.\,10.\,1931 ebd.), \emph{Schriftsteller, Mediziner}!Schleier der Beatrice. Schauspiel in fünf Akten@\strich\emph{Der Schleier der Beatrice. Schauspiel in fünf Akten}|pwv} war manches Lob enthalten.
               Nichtsdestoweniger wird jeder objektiv Urteilende von Deinen Briefen über die »Beatrice\pwindex{Schnitzler, Arthur 15.\,5.\,1862 Wien – 21.\,10.\,1931 ebd.@\textsc{Schnitzler, Arthur} (15.\,5.\,1862 Wien – 21.\,10.\,1931 ebd.), \emph{Schriftsteller, Mediziner}!Schleier der Beatrice. Schauspiel in fünf Akten@\strich\emph{Der Schleier der Beatrice. Schauspiel in fünf Akten}|pw}« den Eindruck empfangen: Freudige
               Begrüssung des Werks\pwindex{Schnitzler, Arthur 15.\,5.\,1862 Wien – 21.\,10.\,1931 ebd.@\textsc{Schnitzler, Arthur} (15.\,5.\,1862 Wien – 21.\,10.\,1931 ebd.), \emph{Schriftsteller, Mediziner}!Schleier der Beatrice. Schauspiel in fünf Akten@\strich\emph{Der Schleier der Beatrice. Schauspiel in fünf Akten}|pwv} nicht
               ohne Einwendungen; – von Deinem Feuilleton\pwindex{Goldmann, Paul 31.\,1.\,1865 Breslau – 25.\,9.\,1935 Wien@\textsc{Goldmann, Paul} (31.\,1.\,1865 Breslau – 25.\,9.\,1935 Wien), \emph{Schriftsteller, Journalist}!Berliner Theater. (»Der Schleier der Beatrice« von Arthur Schnitzler.)@\strich\emph{Berliner Theater. (»Der Schleier der Beatrice« von Arthur Schnitzler.)}|pwv}: \label{K_L03521-2v}\edtext{Ablehnung mit
                  Zibeben}{\lemma{\textnormal{\emph{Ablehnung mit
                  Zibeben}}}\Cendnote{\textnormal{Zibebe, österr.: Rosine. Die
                  Phrase ist abseits dieses Briefes nicht belegt.}}}\label{K_L03521-2}; – so verschieden ist der
               Grundton Deiner Privatäusserungen gegenüber dem Deines Zeitungsartikels\pwindex{Goldmann, Paul 31.\,1.\,1865 Breslau – 25.\,9.\,1935 Wien@\textsc{Goldmann, Paul} (31.\,1.\,1865 Breslau – 25.\,9.\,1935 Wien), \emph{Schriftsteller, Journalist}!Berliner Theater. (»Der Schleier der Beatrice« von Arthur Schnitzler.)@\strich\emph{Berliner Theater. (»Der Schleier der Beatrice« von Arthur Schnitzler.)}|pwv}. Wenn ich also schon
               keinen Grund sehe, dass Dich die Lektüre der Briefkopien vor Erstaunen starr gemacht
               hat, so begreife ich noch weniger Deine Behauptung, dass die Briefkopien von mir als
               Dokumente gegen Deine Ehre gedacht waren. Sie waren und sind nichts anderes als
               Beweise, dass Deine Ansichten über ein Stück\pwindex{Schnitzler, Arthur 15.\,5.\,1862 Wien – 21.\,10.\,1931 ebd.@\textsc{Schnitzler, Arthur} (15.\,5.\,1862 Wien – 21.\,10.\,1931 ebd.), \emph{Schriftsteller, Mediziner}!Schleier der Beatrice. Schauspiel in fünf Akten@\strich\emph{Der Schleier der Beatrice. Schauspiel in fünf Akten}|pwv} im Laufe von zwei Jahren erheblich gewechselt haben;
               und als solche bleiben sie bestehen.\pend
           
\pstart
           Zu dem Fall der »Lebendigen Stunden\pwindex{Schnitzler, Arthur 15.\,5.\,1862 Wien – 21.\,10.\,1931 ebd.@\textsc{Schnitzler, Arthur} (15.\,5.\,1862 Wien – 21.\,10.\,1931 ebd.), \emph{Schriftsteller, Mediziner}!Lebendige Stunden. Vier Einakter@\strich\emph{Lebendige Stunden. Vier Einakter}|pw}« übergehend
               möchte ich vor allem erklären, dass ich die Dir ge{\pb}sprächsweise zugeschriebene
               Aeusserung: »Du möchtest Dich erschiessen, weil Du so etwas nicht leisten
                  könntest{[}«{]}, in diesem Wortlaut nicht aufrecht zu erhalten
               vermag; dass hier möglicherweise eine Erinnerungstäuschung meinerseits vorliegt und
               Du Dich wirklich nicht – um Dein Wort zu gebrauchen – mit so »weibischem
                  Schwulst{[}«{]} ausgedrüc\damage{kt} hast – eine Bemerkung übrigens, durch die sich im weitesten Umkreis niemand
               getroffen fühlt. Es ist ferner festzustellen, dass Du tatsächlich schon nach jener
                  Vorlesung\pwindex{Schnitzler, Arthur 15.\,5.\,1862 Wien – 21.\,10.\,1931 ebd.@\textsc{Schnitzler, Arthur} (15.\,5.\,1862 Wien – 21.\,10.\,1931 ebd.), \emph{Schriftsteller, Mediziner}!Lebendige Stunden. Vier Einakter@\strich\emph{Lebendige Stunden. Vier Einakter}|pwv} im Walde\oindex{Welsberg-Taisten@\textbf{Welsberg-Taisten}, \emph{Verwaltungsgebiet}|pwv} (wie auch in unserem
               letzten Gespräch ausdrücklich vermerkt wurde) gewisse Einwendungen erhoben hast; –
               sie richteten sich ausschliesslich gegen die »Literatur\pwindex{Schnitzler, Arthur 15.\,5.\,1862 Wien – 21.\,10.\,1931 ebd.@\textsc{Schnitzler, Arthur} (15.\,5.\,1862 Wien – 21.\,10.\,1931 ebd.), \emph{Schriftsteller, Mediziner}!Literatur@\strich\emph{Literatur}|pw}«, also gegen dasjenige Stück\pwindex{Schnitzler, Arthur 15.\,5.\,1862 Wien – 21.\,10.\,1931 ebd.@\textsc{Schnitzler, Arthur} (15.\,5.\,1862 Wien – 21.\,10.\,1931 ebd.), \emph{Schriftsteller, Mediziner}!Literatur@\strich\emph{Literatur}|pwv}, das Du als einziges von den vieren\pwindex{Schnitzler, Arthur 15.\,5.\,1862 Wien – 21.\,10.\,1931 ebd.@\textsc{Schnitzler, Arthur} (15.\,5.\,1862 Wien – 21.\,10.\,1931 ebd.), \emph{Schriftsteller, Mediziner}!Lebendige Stunden. Vier Einakter@\strich\emph{Lebendige Stunden. Vier Einakter}|pwv} nach der Aufführung hast gelten
               lassen. (»Die letzten Masken\pwindex{Schnitzler, Arthur 15.\,5.\,1862 Wien – 21.\,10.\,1931 ebd.@\textsc{Schnitzler, Arthur} (15.\,5.\,1862 Wien – 21.\,10.\,1931 ebd.), \emph{Schriftsteller, Mediziner}!letzten Masken@\strich\emph{Die letzten Masken}|pw}«, die Du erst von
               der Bühne\oindex{Deutsches Theater Berlin@\textbf{Deutsches Theater Berlin}, \emph{Theater}|pwv} herab kennen lerntest,
               fallen aus dem Bereich dieser Erörterungen). »Lebendige Stunden\pwindex{Schnitzler, Arthur 15.\,5.\,1862 Wien – 21.\,10.\,1931 ebd.@\textsc{Schnitzler, Arthur} (15.\,5.\,1862 Wien – 21.\,10.\,1931 ebd.), \emph{Schriftsteller, Mediziner}!Lebendige Stunden@\strich\emph{Lebendige Stunden}|pw}« und »Die Frau mit dem
                  Dolch\pwindex{Schnitzler, Arthur 15.\,5.\,1862 Wien – 21.\,10.\,1931 ebd.@\textsc{Schnitzler, Arthur} (15.\,5.\,1862 Wien – 21.\,10.\,1931 ebd.), \emph{Schriftsteller, Mediziner}!Frau mit dem Dolche@\strich\emph{Die Frau mit dem Dolche}|pw}«, besonders letztere erkanntest Du nach jener Vorlesung\pwindex{Schnitzler, Arthur 15.\,5.\,1862 Wien – 21.\,10.\,1931 ebd.@\textsc{Schnitzler, Arthur} (15.\,5.\,1862 Wien – 21.\,10.\,1931 ebd.), \emph{Schriftsteller, Mediziner}!Lebendige Stunden. Vier Einakter@\strich\emph{Lebendige Stunden. Vier Einakter}|pwv} im Walde\oindex{Welsberg-Taisten@\textbf{Welsberg-Taisten}, \emph{Verwaltungsgebiet}|pwv} rückhaltlos ja enthusiastisch an und
               liessest sie fallen, sobald sie auf der Bühne\oindex{Deutsches Theater Berlin@\textbf{Deutsches Theater Berlin}, \emph{Theater}|pwv} erschienen war\introOben{}en\introOben{}. Deine Bemerkung,
               dass der geringe Erfolg der vier Stücke\pwindex{Schnitzler, Arthur 15.\,5.\,1862 Wien – 21.\,10.\,1931 ebd.@\textsc{Schnitzler, Arthur} (15.\,5.\,1862 Wien – 21.\,10.\,1931 ebd.), \emph{Schriftsteller, Mediziner}!Lebendige Stunden. Vier Einakter@\strich\emph{Lebendige Stunden. Vier Einakter}|pwv} Dein in der Zeitung\pwindex{Neue Freie Presse@\emph{Neue Freie Presse}|pwv} ausgespro{\pb}chenes Urteil\pwindex{Goldmann, Paul 31.\,1.\,1865 Breslau – 25.\,9.\,1935 Wien@\textsc{Goldmann, Paul} (31.\,1.\,1865 Breslau – 25.\,9.\,1935 Wien), \emph{Schriftsteller, Journalist}!Berliner Theater. (»Lebendige Stunden« von Arthur Schnitzler.)@\strich\emph{Berliner Theater. (»Lebendige Stunden« von Arthur Schnitzler.)}|pwv} bestätige, ist aus
               mannigfachen Gründen nicht ernst zu nehmen. In dem Bühnenschicksal eines Stückes kann
               der Kritiker niemals die Bestätigung und niemals die Widerlegung seiner Ansichten
               (höchstens einer Vorhersage) ausgedrückt sehen; es sei denn, dass er sich
               bedingungslos mit dem Publikum solidarisch erklärte. Das aber ist bei Dir gewiss
               nicht der Fall; denn Du hast Dich (mit vollem Recht) noch nie darum für geschlagen
               erachtet, weil ein von Dir verworfenes Stück \introOben{}dem Publikum\introOben{}
               behagt und eine lange Reihe von Aufführungen erlebt hat. Also selbst wenn die »Lebendigen Stunden\pwindex{Schnitzler, Arthur 15.\,5.\,1862 Wien – 21.\,10.\,1931 ebd.@\textsc{Schnitzler, Arthur} (15.\,5.\,1862 Wien – 21.\,10.\,1931 ebd.), \emph{Schriftsteller, Mediziner}!Lebendige Stunden. Vier Einakter@\strich\emph{Lebendige Stunden. Vier Einakter}|pw}« missfallen und sich nicht auf
               der Bühne erhalten hätten, wäre damit keineswege die Treffsicherheit Deiner Zeitungskritik\pwindex{Goldmann, Paul 31.\,1.\,1865 Breslau – 25.\,9.\,1935 Wien@\textsc{Goldmann, Paul} (31.\,1.\,1865 Breslau – 25.\,9.\,1935 Wien), \emph{Schriftsteller, Journalist}!Berliner Theater. (»Lebendige Stunden« von Arthur Schnitzler.)@\strich\emph{Berliner Theater. (»Lebendige Stunden« von Arthur Schnitzler.)}|pwv} erwiesen. Nun
               kommt aber noch dazu, dass Deine Behauptung von dem geringen Erfolg der vier Einakter\pwindex{Schnitzler, Arthur 15.\,5.\,1862 Wien – 21.\,10.\,1931 ebd.@\textsc{Schnitzler, Arthur} (15.\,5.\,1862 Wien – 21.\,10.\,1931 ebd.), \emph{Schriftsteller, Mediziner}!Lebendige Stunden. Vier Einakter@\strich\emph{Lebendige Stunden. Vier Einakter}|pwv} den Tatsachen
               durchaus widerspricht. Nicht als Beweis für die Vortrefflichheit der Stücke\pwindex{Schnitzler, Arthur 15.\,5.\,1862 Wien – 21.\,10.\,1931 ebd.@\textsc{Schnitzler, Arthur} (15.\,5.\,1862 Wien – 21.\,10.\,1931 ebd.), \emph{Schriftsteller, Mediziner}!Lebendige Stunden. Vier Einakter@\strich\emph{Lebendige Stunden. Vier Einakter}|pwv}, sondern eben nur als Tatsache führe
               ich an, dass die »Lebendigen Stunden\pwindex{Schnitzler, Arthur 15.\,5.\,1862 Wien – 21.\,10.\,1931 ebd.@\textsc{Schnitzler, Arthur} (15.\,5.\,1862 Wien – 21.\,10.\,1931 ebd.), \emph{Schriftsteller, Mediziner}!Lebendige Stunden. Vier Einakter@\strich\emph{Lebendige Stunden. Vier Einakter}|pw}« nach der
                  »Liebelei\pwindex{Schnitzler, Arthur 15.\,5.\,1862 Wien – 21.\,10.\,1931 ebd.@\textsc{Schnitzler, Arthur} (15.\,5.\,1862 Wien – 21.\,10.\,1931 ebd.), \emph{Schriftsteller, Mediziner}!Liebelei. Schauspiel in drei Akten@\strich\emph{Liebelei. Schauspiel in drei Akten}|pw}« bisher meinen stärksten
               Theatererfolg bedeutet haben. So ist bei Brahm\oindex{Deutsches Theater Berlin@\textbf{Deutsches Theater Berlin}, \emph{Theater}|pwv}\pwindex{Brahm, Otto 5.\,2.\,1856 Hamburg – 28.\,11.\,1912 Berlin@\textsc{Brahm, Otto} (5.\,2.\,1856 Hamburg – 28.\,11.\,1912 Berlin), \emph{Theaterleiter, Regisseur}|pw} der ganze Zyklus\pwindex{Schnitzler, Arthur 15.\,5.\,1862 Wien – 21.\,10.\,1931 ebd.@\textsc{Schnitzler, Arthur} (15.\,5.\,1862 Wien – 21.\,10.\,1931 ebd.), \emph{Schriftsteller, Mediziner}!Lebendige Stunden. Vier Einakter@\strich\emph{Lebendige Stunden. Vier Einakter}|pwv} über
               vierzig Mal aufgeführt worden. »Letzte Masken\pwindex{Schnitzler, Arthur 15.\,5.\,1862 Wien – 21.\,10.\,1931 ebd.@\textsc{Schnitzler, Arthur} (15.\,5.\,1862 Wien – 21.\,10.\,1931 ebd.), \emph{Schriftsteller, Mediziner}!letzten Masken@\strich\emph{Die letzten Masken}|pw}«
               und »Literatur\pwindex{Schnitzler, Arthur 15.\,5.\,1862 Wien – 21.\,10.\,1931 ebd.@\textsc{Schnitzler, Arthur} (15.\,5.\,1862 Wien – 21.\,10.\,1931 ebd.), \emph{Schriftsteller, Mediziner}!Literatur@\strich\emph{Literatur}|pw}« {\pb}im Zyklus\pwindex{Schnitzler, Arthur 15.\,5.\,1862 Wien – 21.\,10.\,1931 ebd.@\textsc{Schnitzler, Arthur} (15.\,5.\,1862 Wien – 21.\,10.\,1931 ebd.), \emph{Schriftsteller, Mediziner}!Lebendige Stunden. Vier Einakter@\strich\emph{Lebendige Stunden. Vier Einakter}|pwv} am \label{K_L03521-3v}\edtext{Münchner Residenztheater\oindex{Residenztheater München@\textbf{Residenztheater München}, \emph{Theater}|pw}}{\lemma{\textnormal{\emph{Münchner Residenztheater}}}\Cendnote{\textnormal{Die Premiere von \emph{Lebendige Stunden}\pwindex{Schnitzler, Arthur 15.\,5.\,1862 Wien – 21.\,10.\,1931 ebd.@\textsc{Schnitzler, Arthur} (15.\,5.\,1862 Wien – 21.\,10.\,1931 ebd.), \emph{Schriftsteller, Mediziner}!Lebendige Stunden. Vier Einakter@\strich\emph{Lebendige Stunden. Vier Einakter}|pwk} am \emph{Residenztheater München}\orgindex{Residenztheater München@Residenztheater München|pwk} hatte am 6. 3. 1902 stattgefunden.}}}\label{K_L03521-3} oft gespielt, habe ich neulich
               anlässlich ihrer 16. Aufführung im \label{K_L03521-4v}\edtext{Schauspielhaus\oindex{Münchner Schauspielhaus@\textbf{Münchner Schauspielhaus}, \emph{Theater}|pw}}{\lemma{\textnormal{\emph{Schauspielhaus}}}\Cendnote{\textnormal{Siehe A. S.: \emph{Tagebuch}, 10. 12. 1910.
               }}}\label{K_L03521-4} derselben Stadt\oindex{München@\textbf{München}|pwv} bei
               total ausverkauftem Hause zu sehen Gelegenheit gehabt. »Die Frau mit dem Dolch\pwindex{Schnitzler, Arthur 15.\,5.\,1862 Wien – 21.\,10.\,1931 ebd.@\textsc{Schnitzler, Arthur} (15.\,5.\,1862 Wien – 21.\,10.\,1931 ebd.), \emph{Schriftsteller, Mediziner}!Frau mit dem Dolche@\strich\emph{Die Frau mit dem Dolche}|pw}« brachte mir erst kürzlich aus \label{K_L03521-5v}\edtext{Schweden\oindex{Schweden@\textbf{Schweden}|pw}}{\lemma{\textnormal{\emph{Schweden}}}\Cendnote{\textnormal{Gustaf Linden\pwindex{Linden, Gustaf 10.\,11.\,1875 Malmö – 19.\,3.\,1936 Stockholm@\textsc{Linden, Gustaf} (10.\,11.\,1875 Malmö – 19.\,3.\,1936 Stockholm), \emph{Theaterleiter, Regisseur}|pwk} hatte bereits einige Stücke
                     Schnitzlers übersetzt. Die schwed\oindex{Schweden@\textbf{Schweden}|pwkv}ische Premiere von
                     \emph{Comtesse Mizzi}\pwindex{Schnitzler, Arthur 15.\,5.\,1862 Wien – 21.\,10.\,1931 ebd.@\textsc{Schnitzler, Arthur} (15.\,5.\,1862 Wien – 21.\,10.\,1931 ebd.), \emph{Schriftsteller, Mediziner}!Comtesse Mizzi [schwedisch]@\strich\emph{Comtesse Mizzi [schwedisch]}|pwk} (\emph{Komtesse Mizzi}\pwindex{Schnitzler, Arthur 15.\,5.\,1862 Wien – 21.\,10.\,1931 ebd.@\textsc{Schnitzler, Arthur} (15.\,5.\,1862 Wien – 21.\,10.\,1931 ebd.), \emph{Schriftsteller, Mediziner}!Komtesse Mizzi oder: Der Familientag@\strich\emph{Komtesse Mizzi oder: Der Familientag}|pwk}), \emph{Damen
                     med dolken}\pwindex{Schnitzler, Arthur 15.\,5.\,1862 Wien – 21.\,10.\,1931 ebd.@\textsc{Schnitzler, Arthur} (15.\,5.\,1862 Wien – 21.\,10.\,1931 ebd.), \emph{Schriftsteller, Mediziner}!Damen med dolken@\strich\emph{Damen med dolken}|pwk} (\emph{Die Frau mit dem Dolche}\pwindex{Schnitzler, Arthur 15.\,5.\,1862 Wien – 21.\,10.\,1931 ebd.@\textsc{Schnitzler, Arthur} (15.\,5.\,1862 Wien – 21.\,10.\,1931 ebd.), \emph{Schriftsteller, Mediziner}!Frau mit dem Dolche@\strich\emph{Die Frau mit dem Dolche}|pwk})
                  und \emph{Den gröna papegojan}\pwindex{Schnitzler, Arthur 15.\,5.\,1862 Wien – 21.\,10.\,1931 ebd.@\textsc{Schnitzler, Arthur} (15.\,5.\,1862 Wien – 21.\,10.\,1931 ebd.), \emph{Schriftsteller, Mediziner}!Den gröna papegojan@\strich\emph{Den gröna papegojan}|pwk} (\emph{Der grüne Kakadu}\pwindex{Schnitzler, Arthur 15.\,5.\,1862 Wien – 21.\,10.\,1931 ebd.@\textsc{Schnitzler, Arthur} (15.\,5.\,1862 Wien – 21.\,10.\,1931 ebd.), \emph{Schriftsteller, Mediziner}!grüne Kakadu. Groteske in einem Akt@\strich\emph{Der grüne Kakadu. Groteske in einem Akt}|pwk}) hatte am 30. 3. 1910 im Kungliga Dramatiska
                     Teatern\oindex{Kungliga Dramatiska Teatern@\textbf{Kungliga Dramatiska Teatern}, \emph{Theater}|pwk} (Königlich Dramatisches
                     Theater\oindex{Kungliga Dramatiska Teatern@\textbf{Kungliga Dramatiska Teatern}, \emph{Theater}|pwk}) stattgefunden.}}}\label{K_L03521-5} Tantiemen. »Die letzten Masken\pwindex{Schnitzler, Arthur 15.\,5.\,1862 Wien – 21.\,10.\,1931 ebd.@\textsc{Schnitzler, Arthur} (15.\,5.\,1862 Wien – 21.\,10.\,1931 ebd.), \emph{Schriftsteller, Mediziner}!letzten Masken@\strich\emph{Die letzten Masken}|pw}« wurden \label{K_L03521-6v}\edtext{in
                  England\oindex{England@\textbf{England}, \emph{Land}|pw}}{\lemma{\textnormal{\emph{in
                  England}}}\Cendnote{\textnormal{In London\oindex{London@\textbf{London}, \emph{Hauptstadt}|pwk} hatte die Premiere von \emph{In the
                     Hospital}\pwindex{Schnitzler, Arthur 15.\,5.\,1862 Wien – 21.\,10.\,1931 ebd.@\textsc{Schnitzler, Arthur} (15.\,5.\,1862 Wien – 21.\,10.\,1931 ebd.), \emph{Schriftsteller, Mediziner}!In the Hospital@\strich\emph{In the Hospital}|pwk} am 28. 2. 1905 am Court Theatre\oindex{Royal Court Theatre@\textbf{Royal Court Theatre}, \emph{Theater}|pwk} stattgefunden. Die Übersetzung stammte von Christopher Horne\pwindex{Horne, Christopher @\textsc{Horne, Christopher}, \emph{Übersetzer}|pwk}. Es handelte sich um die
                  erste englische Schnitzler-Aufführung
                  überhaupt.}}}\label{K_L03521-6} und \label{K_L03521-7v}\edtext{in Italien\oindex{Italien@\textbf{Italien}|pw}}{\lemma{\textnormal{\emph{in Italien}}}\Cendnote{\textnormal{Am Teatro Alfieri\oindex{Teatro Alfieri@\textbf{Teatro Alfieri}, \emph{Theater}|pwk} in Turin\oindex{Turin@\textbf{Turin}|pwk} war \emph{Le ultime maschere}\pwindex{Schnitzler, Arthur 15.\,5.\,1862 Wien – 21.\,10.\,1931 ebd.@\textsc{Schnitzler, Arthur} (15.\,5.\,1862 Wien – 21.\,10.\,1931 ebd.), \emph{Schriftsteller, Mediziner}!Le ultime maschere. Drama in un atto@\strich\emph{Le ultime maschere. Drama in un atto}|pwk} gemeinsam mit \emph{Cena d’addio}\pwindex{Schnitzler, Arthur 15.\,5.\,1862 Wien – 21.\,10.\,1931 ebd.@\textsc{Schnitzler, Arthur} (15.\,5.\,1862 Wien – 21.\,10.\,1931 ebd.), \emph{Schriftsteller, Mediziner}!Cena d’addio@\strich\emph{Cena d’addio}|pwk} (\emph{Abschiedssouper}\pwindex{Schnitzler, Arthur 15.\,5.\,1862 Wien – 21.\,10.\,1931 ebd.@\textsc{Schnitzler, Arthur} (15.\,5.\,1862 Wien – 21.\,10.\,1931 ebd.), \emph{Schriftsteller, Mediziner}!Abschiedssouper@\strich\emph{Abschiedssouper}|pwk}) in der Übersetzung von Cesare Levi\pwindex{Levi, Cesare 1874 Triest – 18.\,7.\,1926 Sand in Taufers@\textsc{Levi, Cesare} (1874 Triest – 18.\,7.\,1926 Sand in Taufers), \emph{Journalist, Übersetzer, Theaterkritiker}|pwk} aufgeführt worden. Dazu war der Einakter \emph{Monsignore in vacanza}\pwindex{Claretie, Jules 3.\,12.\,1840 Limoges – 23.\,12.\,1913 Paris@\textsc{Claretie, Jules} (3.\,12.\,1840 Limoges – 23.\,12.\,1913 Paris), \emph{Schriftsteller, Journalist, Theaterleiter}!Monsignore in vacanza. Commedia in un atto@\strich\emph{Monsignore in vacanza. Commedia in un atto}|pwk} von Jules Claretie\pwindex{Claretie, Jules 3.\,12.\,1840 Limoges – 23.\,12.\,1913 Paris@\textsc{Claretie, Jules} (3.\,12.\,1840 Limoges – 23.\,12.\,1913 Paris), \emph{Schriftsteller, Journalist, Theaterleiter}|pwk}, übersetzt von Ugo Piperno\pwindex{Piperno, Ugo 27.\,3.\,1871 Livorno – 4.\,5.\,1922 Casalecchio di Reno@\textsc{Piperno, Ugo} (27.\,3.\,1871 Livorno – 4.\,5.\,1922 Casalecchio di Reno), \emph{Schauspieler, Übersetzer}|pwk}, gegeben worden.}}}\label{K_L03521-7} gegeben und »Literatur\pwindex{Schnitzler, Arthur 15.\,5.\,1862 Wien – 21.\,10.\,1931 ebd.@\textsc{Schnitzler, Arthur} (15.\,5.\,1862 Wien – 21.\,10.\,1931 ebd.), \emph{Schriftsteller, Mediziner}!Literatur@\strich\emph{Literatur}|pw}« hat schon eine kleine Reise um die
               Welt gemacht.\pend
           
\pstart
           Wenn Du es weiters als eine Lächerlichheit erklärst »gegen das öffentlich abgegebene
                  Urteil\pwindex{Goldmann, Paul 31.\,1.\,1865 Breslau – 25.\,9.\,1935 Wien@\textsc{Goldmann, Paul} (31.\,1.\,1865 Breslau – 25.\,9.\,1935 Wien), \emph{Schriftsteller, Journalist}!Berliner Theater. (»Lebendige Stunden« von Arthur Schnitzler.)@\strich\emph{Berliner Theater. (»Lebendige Stunden« von Arthur Schnitzler.)}|pwv} eines Kritikers,
               das er genau und sachlich begründet habe, Aeusserungen ausspielen zu wollen, die er
               nach einer Vorlesung\pwindex{Schnitzler, Arthur 15.\,5.\,1862 Wien – 21.\,10.\,1931 ebd.@\textsc{Schnitzler, Arthur} (15.\,5.\,1862 Wien – 21.\,10.\,1931 ebd.), \emph{Schriftsteller, Mediziner}!Lebendige Stunden. Vier Einakter@\strich\emph{Lebendige Stunden. Vier Einakter}|pwv} im Walde\oindex{Welsberg-Taisten@\textbf{Welsberg-Taisten}, \emph{Verwaltungsgebiet}|pwv} getan«, so dürfte ich Dir
               mit demselben Recht entgegnen, es sei lächerlich{[},{]} ein
               gedrucktes, für die Oeffentlichkeit bestimmtes Feuilleton\pwindex{Goldmann, Paul 31.\,1.\,1865 Breslau – 25.\,9.\,1935 Wien@\textsc{Goldmann, Paul} (31.\,1.\,1865 Breslau – 25.\,9.\,1935 Wien), \emph{Schriftsteller, Journalist}!Berliner Theater. (»Lebendige Stunden« von Arthur Schnitzler.)@\strich\emph{Berliner Theater. (»Lebendige Stunden« von Arthur Schnitzler.)}|pwv} gegen die rückhaltlos anerkennenden Worte
               auszuspielen, die man sechs Monate vorher als Freund zum Freunde gesprochen. Ob aber
               Aeusserungen in einem Walde oder in einem geschlossenen Raum gefallen sind, das kann
               wohl für deren Wertung unter ernsthaften Leuten nicht in Betracht kommen.\pend
           
\pstart
           Nun könnte Einer, der nur Deinen Brief und nicht {\pb}auch meine Erwiderung zu lesen
                  bekäme{[},{]} leicht zu der irrigen Meinung verleitet werden als
               hätte ich jemals gewünscht oder gar von Dir verlangt, dass Du über meine Werke keine
               abfälligen Kritiken \introOben{}veröffentlichen\introOben{} oder dass Du solche
               wenigstens nicht in Deine Bücher\pwindex{Goldmann, Paul 31.\,1.\,1865 Breslau – 25.\,9.\,1935 Wien@\textsc{Goldmann, Paul} (31.\,1.\,1865 Breslau – 25.\,9.\,1935 Wien), \emph{Schriftsteller, Journalist}!Aus dem dramatischen Irrgarten. Polemische Aufsätze über Berliner Theateraufführungen@\strich\emph{Aus dem dramatischen Irrgarten. Polemische Aufsätze über Berliner Theateraufführungen}|pwv}\pwindex{Goldmann, Paul 31.\,1.\,1865 Breslau – 25.\,9.\,1935 Wien@\textsc{Goldmann, Paul} (31.\,1.\,1865 Breslau – 25.\,9.\,1935 Wien), \emph{Schriftsteller, Journalist}!Vom Rückgang der deutschen Bühne. Polemische Aufsätze über Berliner Theater-Aufführungen@\strich\emph{Vom Rückgang der deutschen Bühne. Polemische Aufsätze über Berliner Theater-Aufführungen}|pwv}\pwindex{Goldmann, Paul 31.\,1.\,1865 Breslau – 25.\,9.\,1935 Wien@\textsc{Goldmann, Paul} (31.\,1.\,1865 Breslau – 25.\,9.\,1935 Wien), \emph{Schriftsteller, Journalist}!Literatenstücke und Ausstattungsregie. Polemische Aufsätze über Berliner Theater-Aufführungen@\strich\emph{Literatenstücke und Ausstattungsregie. Polemische Aufsätze über Berliner Theater-Aufführungen}|pwv} aufnehmen solltest. Dass mir
               dies jederzeit so ferne lag wie nur möglich sei hier nur der Vollständigkeit wegen
               ausgesprochen. Du selbst hast allerdings nun schon wiederholt den Wunsch \substVorne{}\textsuperscript{ausgesprochen}\substDazwischen{}geäußert\substHinten{} über mich nicht mehr schreiben zu müssen. Da dieser Wunsch entweder Deiner
               Meinung entspringt, ich würde niemals etwas Deinem Geschmack nach Gutes zu
               produzieren imstande sein oder Deinem Gefühl, Du würdest niemals zu einer meiner
               Arbeiten ein Verhältnis finden können, so schiene es mir ja allerdings angemessen,
               dass Du Dich Deiner Verpflichtung über mich zu schreiben \introOben{}auf
                  irgend eine Weise\introOben{} zu entledigen suchtest. Doch das ist eine Sache, die Du
               mit Dir selber auszumachen hast. Was ich konstatieren wollte ist einfach, dass Deine
               kritischen Ueberzeugungen nicht sonderlich stark fundiert sind, dass in den zur
               Diskussion stehenden Fällen\pwindex{Schnitzler, Arthur 15.\,5.\,1862 Wien – 21.\,10.\,1931 ebd.@\textsc{Schnitzler, Arthur} (15.\,5.\,1862 Wien – 21.\,10.\,1931 ebd.), \emph{Schriftsteller, Mediziner}!Lebendige Stunden. Vier Einakter@\strich\emph{Lebendige Stunden. Vier Einakter}|pwv}\pwindex{Schnitzler, Arthur 15.\,5.\,1862 Wien – 21.\,10.\,1931 ebd.@\textsc{Schnitzler, Arthur} (15.\,5.\,1862 Wien – 21.\,10.\,1931 ebd.), \emph{Schriftsteller, Mediziner}!Schleier der Beatrice. Schauspiel in fünf Akten@\strich\emph{Der Schleier der Beatrice. Schauspiel in fünf Akten}|pwv} jedesmal das Publikum es war und nicht ich, das von Deinen beiden Urteilen\pwindex{Goldmann, Paul 31.\,1.\,1865 Breslau – 25.\,9.\,1935 Wien@\textsc{Goldmann, Paul} (31.\,1.\,1865 Breslau – 25.\,9.\,1935 Wien), \emph{Schriftsteller, Journalist}!Berliner Theater. (»Lebendige Stunden« von Arthur Schnitzler.)@\strich\emph{Berliner Theater. (»Lebendige Stunden« von Arthur Schnitzler.)}|pwv}\pwindex{Goldmann, Paul 31.\,1.\,1865 Breslau – 25.\,9.\,1935 Wien@\textsc{Goldmann, Paul} (31.\,1.\,1865 Breslau – 25.\,9.\,1935 Wien), \emph{Schriftsteller, Journalist}!Berliner Theater. (»Der Schleier der Beatrice« von Arthur Schnitzler.)@\strich\emph{Berliner Theater. (»Der Schleier der Beatrice« von Arthur Schnitzler.)}|pwv} das
               ungünstigere zu hören resp. zu lesen bekam, und ich füge {\pb}heute noch hinzu, dass es sich beide Male, ganz
               besonders im Fall der »Lebendigen
                  Stunden\pwindex{Schnitzler, Arthur 15.\,5.\,1862 Wien – 21.\,10.\,1931 ebd.@\textsc{Schnitzler, Arthur} (15.\,5.\,1862 Wien – 21.\,10.\,1931 ebd.), \emph{Schriftsteller, Mediziner}!Lebendige Stunden. Vier Einakter@\strich\emph{Lebendige Stunden. Vier Einakter}|pw}«{[},{]} nicht um Differenzen der Ausdrucksnuance, wie
               Du es nun darstellen möchtest, sondern um solche des Grundtons gehandelt hat.\pend
           
\pstart
           Warum Du Dich gegen diese Feststellung so heftig zur Wehre setzst, ist umso
               unverständlicher als Du ja selbst noch vor \label{T_L03521-1v}\edtext{Nachprüfung}{\lemma{\textnormal{\emph{Nachprüfung}}}\Cendnote{\textnormal{korrigiert aus »Nachrpüfung«}}}\label{T_L03521-1} Deiner Briefkopien und Deines
                  Feuilletons\pwindex{Goldmann, Paul 31.\,1.\,1865 Breslau – 25.\,9.\,1935 Wien@\textsc{Goldmann, Paul} (31.\,1.\,1865 Breslau – 25.\,9.\,1935 Wien), \emph{Schriftsteller, Journalist}!Berliner Theater. (»Lebendige Stunden« von Arthur Schnitzler.)@\strich\emph{Berliner Theater. (»Lebendige Stunden« von Arthur Schnitzler.)}|pwv} das Bestehen
               solcher Widersprüche zwischen Deinen privaten und öffentlichen Aeusserungen
               ohneweiters zugabst und um Erklärungen dafür keineswegs verlegen warst. Du sprachst
               die Meinung aus, dass man im privaten Verkehr einem Freunde nicht gern wehe tun wolle
               und daher zuweilen Rücksichten nehme, die man bei Besprechung seiner Leistungen vor
               der Oeffentlichkeit ausser Acht lassen könne, ja sogar müsse. Du gabst ferner zu,
               dass die Aufführung eines Werkes Dich manchmal Schwächen erkennen liesse (warum
               niemals Vorzüge?), die Dir bei Lektüre oder Vorlesung desselben Werkes nicht
               aufgefallen wären. Ob diese Erklärungsversuche nun stimmen oder nicht, mir sind und
               bleiben sie Beweise, dass wir sowohl über das Wesen freundschaftlicher Beziehungen
               als über die Vorbedingungen eines \label{T_L03521-2v}\edtext{kritischen Richteramts recht verschieden denken.}{\lemma{\textnormal{\emph{kritischen … denken.}}}\Cendnote{\textnormal{von der Schreibkraft am rechten Rand normal zum Text}}}\label{T_L03521-2}{ }{\pb}Meine Ansicht geht dahin, dass man
               einem Freund im Privatverkehr seine Meinung mindestens so aufrichtig zu sagen \substVorne{}\textsuperscript{hätte wie}\substDazwischen{}habe als\substHinten{} in einem Feuilleton und dass ein Rezensent – besonders einer, der sich
               nebstbei auch zum Theaterdirektor berufen fühlt – sich von dem Wesen eines
               Theaterstücks, von dessen innerem Wert, nicht von dessen Erfolgschancen meine ich,
               auch schon aus dem Buch eine bestimmte Vorstellung müsse bilden können. Habe ich in
               unserem letzten Gespräch diese Ansichten dahin formuliert, dass Du gerade durch Deine
               Erklärungsversuche sowohl als Freund wie als Kritiker Selbstmord begangen hättest, so
               war dies möglicherweise \strikeout{in} etwas zu temperamentvoll
               vorgebracht, immerhin aber in harmloseren Ton gehalten als Deine briefliche Replik,
               in der Du mir – wörtlich – vorwirfst, ich sei über Dich hergefallen wie über einen
               charakterlosen Lumpen und mir mitteilst, dass Du an diese Unterredung mit einer
               Mischung von Scham, Widerwillen und Empörung zurückdenkst. Ohne die subjektive
               Echtheit Deiner Empfindung anzweifeln zu wollen, stelle ich es Dir anheim, ob Du
               Deine Ausdrucksweise als männlichen, weiblichen oder sächlichen Schwulst bezeichnen
               willst.\pend
           
\pstart
           {\pb}Dass dieses Gespräch im Hause\oindex{Wien@\textbf{Wien}!IX., Alsergrund@\textbf{IX., Alsergrund}!Frankgasse 1@\textbf{Frankgasse 1}, \emph{Wohngebäude}|pwv} meiner Mutter\pwindex{Schnitzler, Louise 8.\,7.\,1840 Kőszeg – 9.\,9.\,1911 Wien@\textsc{Schnitzler, Louise} (8.\,7.\,1840 Kőszeg – 9.\,9.\,1911 Wien)|pwv} stattfand, worauf Du
               besonderes Gewicht zu legen scheinst{[},{]} ist für meine Auffassung
               so belanglos als es in jenem früheren Fall die Welsberg\oindex{Welsberg-Taisten@\textbf{Welsberg-Taisten}, \emph{Verwaltungsgebiet}|pw}er Waldlandschaft gewesen ist. Und wenn ich mir die ruhige, fast
               herzliche Art in Erinnerung zurückrufe, in der wir uns im Vorzimmer meiner Mutter\pwindex{Schnitzler, Louise 8.\,7.\,1840 Kőszeg – 9.\,9.\,1911 Wien@\textsc{Schnitzler, Louise} (8.\,7.\,1840 Kőszeg – 9.\,9.\,1911 Wien)|pwv}{ }\introOben{}von einander\introOben{} verabschiedet haben, so scheint mir Deine
               Betonung des verletzten Gastrechtes viel eher feuilletoni\introOben{}sti\introOben{}sch-polemischen Erwägungen ihre Entstehung zu verdanken als spontaner
               Ueberzeugung. Jedenfalls aber möchte ich nochmals bemerken, dass jene oben zitierten
                  Versuche{[},{]} die Widersprüche zwischen Deinen privaten und
               öffentlichen Aeusserungen\pwindex{Goldmann, Paul 31.\,1.\,1865 Breslau – 25.\,9.\,1935 Wien@\textsc{Goldmann, Paul} (31.\,1.\,1865 Breslau – 25.\,9.\,1935 Wien), \emph{Schriftsteller, Journalist}!Berliner Theater. (»Lebendige Stunden« von Arthur Schnitzler.)@\strich\emph{Berliner Theater. (»Lebendige Stunden« von Arthur Schnitzler.)}|pwv}\pwindex{Goldmann, Paul 31.\,1.\,1865 Breslau – 25.\,9.\,1935 Wien@\textsc{Goldmann, Paul} (31.\,1.\,1865 Breslau – 25.\,9.\,1935 Wien), \emph{Schriftsteller, Journalist}!Berliner Theater. (»Der Schleier der Beatrice« von Arthur Schnitzler.)@\strich\emph{Berliner Theater. (»Der Schleier der Beatrice« von Arthur Schnitzler.)}|pwv} aufzuklären{[},{]} von Dir herrühren und nicht
               von mir und überdies betonen, dass ich selbst den Grund dieser Widersprüche stets
               viel weniger in etwaigen Mängeln Deines menschlichen Wesens als in solchen Deiner
               kritischen Begabung erblickt habe. Es ist mir nicht unangenehm, dass ich bescheidenen
               Zweifeln in dieser Richtung schon vor vielen Jahren, lang ehe Du zu öffentlichen
               Aeusserungen über mich Gelegenheit hattest{[},{]} aus Anlass \introOben{}eines\introOben{} Deiner ersten \label{K_L03521-8v}\edtext{Hauptmanns\pwindex{Hauptmann, Gerhart 15.\,11.\,1862 Szczawno-Zdrój – 6.\,6.\,1946 Jagniątków@\textsc{Hauptmann, Gerhart} (15.\,11.\,1862 Szczawno-Zdrój – 6.\,6.\,1946 Jagniątków), \emph{Schriftsteller}|pw}\pwindex{Goldmann, Paul 31.\,1.\,1865 Breslau – 25.\,9.\,1935 Wien@\textsc{Goldmann, Paul} (31.\,1.\,1865 Breslau – 25.\,9.\,1935 Wien), \emph{Schriftsteller, Journalist}!Michael Kramer.«@\strich\emph{»Michael Kramer.«}|pwv}}{\lemma{\textnormal{\emph{Hauptmanns}}}\Cendnote{\textnormal{Bezugnahme auf Paul Goldmann\pwindex{Goldmann, Paul 31.\,1.\,1865 Breslau – 25.\,9.\,1935 Wien@\textsc{Goldmann, Paul} (31.\,1.\,1865 Breslau – 25.\,9.\,1935 Wien), \emph{Schriftsteller, Journalist}|pwk}: \emph{»Michael Kramer«}\pwindex{Goldmann, Paul 31.\,1.\,1865 Breslau – 25.\,9.\,1935 Wien@\textsc{Goldmann, Paul} (31.\,1.\,1865 Breslau – 25.\,9.\,1935 Wien), \emph{Schriftsteller, Journalist}!Michael Kramer.«@\strich\emph{»Michael Kramer.«}|pwk}. In: \emph{Neue Freie Presse}\pwindex{Neue Freie Presse@\emph{Neue Freie Presse}|pwk}, Nr. 13.055, 28. 12. 1900, Morgenblatt, S. 1–3, bzw. auf darauffolgende
                  Feuilletons und damit einhergehende Auseinandersetzungen, vgl. XXXX Auszeichnungsfehler: Dokument L02947 nicht gefunden, XXXX Auszeichnungsfehler: Dokument L03090 nicht gefunden und XXXX Auszeichnungsfehler: Dokument L03091 nicht gefunden; siehe auch XXXX Auszeichnungsfehler: Dokument L03520 nicht gefunden.
               }}}\label{K_L03521-8} brieflichen Ausdruck gab. Und so darf mir wohl ge{\pb}stattet sein, freilich nicht aus
               diesem Grunde allein, Deinen Versuch, mich als einen \label{K_L03521-9v}\edtext{»durch Grössengefühl und Selbstgefällikeit jeden Urteils
               beraubten Autor«}{\lemma{\textnormal{\emph{»durch … Autor«}}}\Cendnote{\textnormal{XXXX Auszeichnungsfehler: Dokument L03475 nicht gefunden.
               }}}\label{K_L03521-9} hinzustellen, mit jener Gleichgültigkeit aufzunehmen, die mir so bedenklicher
               Polemik gegenüber am Platze scheint. Doch möchte ich in diesem Zusammenhang, wie
               gleichfalls schon mündlich geschehen, betonen, dass ich Deiner öffentlichen
               kritischen Tätigkeit wie der Durchschnittskritik überhaupt, keineswegs jene
               Wichtigkeit beimesse, die die Ausführlichheit dieses Schreibens Uneingeweihte könnte
               vermuten lassen. Was Du auf journalistischem Gebiete insbesondere als \label{K_L03521-10v}\edtext{politischer Korrespondent und
                  Reiseschilderer}{\lemma{\textnormal{\emph{politischer … Reiseschilderer}}}\Cendnote{\textnormal{Neben seiner
                  Tätigkeit als Theater- und Kulturjournalist war Goldmann\pwindex{Goldmann, Paul 31.\,1.\,1865 Breslau – 25.\,9.\,1935 Wien@\textsc{Goldmann, Paul} (31.\,1.\,1865 Breslau – 25.\,9.\,1935 Wien), \emph{Schriftsteller, Journalist}|pwk} bei der \emph{Frankfurter Zeitung}\orgindex{Frankfurter Zeitung@Frankfurter Zeitung|pwk}
                  politischer Korrespondent in Paris\oindex{Paris@\textbf{Paris}, \emph{Hauptstadt}|pwk} gewesen.
                  Sein ausführlichster Reisebericht, \emph{Ein Sommer in
                     China}\pwindex{Goldmann, Paul 31.\,1.\,1865 Breslau – 25.\,9.\,1935 Wien@\textsc{Goldmann, Paul} (31.\,1.\,1865 Breslau – 25.\,9.\,1935 Wien), \emph{Schriftsteller, Journalist}!Sommer in China. Reisebilder@\strich\emph{Ein Sommer in China. Reisebilder}|pwk} (1899), erschien auch in Buchform.}}}\label{K_L03521-10}
               geleistet hast, soll nach wie vor anerkannt werden, was Du als Kritiker zu wirken
               vermochtest sei hier in kurzen Worten zusammengefasst: Es ist Dir manchmal gelungen
               einem Autor auf ein paar Stunden die Stimmung zu verderben\substVorne{}\textsuperscript{. F}\substDazwischen{}; f\substHinten{}erner mag es manchmal vorgekommen sein, dass Deine Feuilletons, dadurch, dass
               sie in einem weit verbreiteten Blatt\pwindex{Neue Freie Presse@\emph{Neue Freie Presse}|pwv} erschienen sind, manchen Stücken höheren Ranges in Wien\oindex{Wien@\textbf{Wien}, \emph{Verwaltungsgebiet}|pw} ein ungünstiges Vorurteil bereitet und sie dadurch
               geschäftlich geschädigt haben. Aber {\pb}damit sind die Grenzen Deines Einflusses aufs Weiteste umrissen. Unbeirrt geht die
               deutsche Literatur ihren Weg, die Dichter schreiben nach wie vor was sie wollen und
               nicht was Dir manchmal beliebt ihnen \label{K_L03521-11v}\edtext{vorzuschlagen}{\lemma{\textnormal{\emph{vorzuschlagen}}}\Cendnote{\textnormal{Anspielung auf Goldmanns\pwindex{Goldmann, Paul 31.\,1.\,1865 Breslau – 25.\,9.\,1935 Wien@\textsc{Goldmann, Paul} (31.\,1.\,1865 Breslau – 25.\,9.\,1935 Wien), \emph{Schriftsteller, Journalist}|pwk} wiederholte Forderungen, was Schnitzler schreiben solle – etwa ein
                  Lustspiel (vgl. XXXX Auszeichnungsfehler: Dokument L02723 nicht gefunden, XXXX Auszeichnungsfehler: Dokument L02724 nicht gefunden, XXXX Auszeichnungsfehler: Dokument L02792 nicht gefunden, XXXX Auszeichnungsfehler: Dokument L02914 nicht gefunden, XXXX Auszeichnungsfehler: Dokument L03092 nicht gefunden und XXXX Auszeichnungsfehler: Dokument L03204 nicht gefunden) oder ein
                  historisches Wien\oindex{Wien@\textbf{Wien}, \emph{Verwaltungsgebiet}|pwk}er Stück (vgl. XXXX Auszeichnungsfehler: Dokument L02790 nicht gefunden und XXXX Auszeichnungsfehler: Dokument L02794 nicht gefunden).}}}\label{K_L03521-11}. An den
               Urteilen selbständig denkender Leute hast Du niemals das Geringste zu ändern
               vermocht; – wenn Du also auch ein oder das andere Mal im Einzelnen das Richtige zu
               treffen, öfter noch irrtümliche und voreingenommene Ansichten mit Witz und
               Geschicklichkeit zur Geltung zu bringen imstande warst – Dein Gesamtwirken hat bisher
               niemanden dauernd geschadet als Dir selbst, dessen Bild schon heute eines
               Ehrenplatzes in der Galerie jener berühmten Missversteher gewiss ist, die zu jeder
               Zeit die Schaffensfreude gerade der Besten mit ihrem \substVorne{}\textsuperscript{unerfreulichen Spott und Warnungsrufen}\substDazwischen{}respekt- u ahnungslosen Geschwätz\substHinten{} begleitet haben. Schade. Denn einmal sah es aus, wie wenn Du im Geistesleben
               unserer Zeit zu anderem berufen wärest, als
                  dazu{[},{]} der Kunst mit jener
               Fremdheit, ja mit jenem halb unbewussten Groll gegenüberzustehen, zu dem der
               unproduktive Mensch (nicht der Kritiker sage ich, denn es gibt auch produktive
               Kritik) dem produktiven Menschen gegenüber nun einmal verdammt zu {\pb}sein scheint.\pend
           
\pstart
           Du magst es Dir weiter in dem Wahne wohl sein lassen, dass aus all dem, was ich hier
               gesagt habe, am Ende doch nichts anderes spräche als die verletzte Empfindlichkeit
               des getadelten oder des nicht genügend gelobten dramatischen Autors. So frei ich mich
               von solcher Empfindlichkeit weiss{[},{]} ganz besonders Dir gegenüber,
               so lässt sich hier eine \introOben{}allgemeinere\introOben{}, gewissermassen
               abschliessende Bemerkung, nicht wohl vermeiden. Es ist nicht zu bestreiten, dass wir
               mit der Mehrzahl der Menschen ganz ungestört weiter verkehren können und dürfen, auch
               dann, wenn wir uns ge\substVorne{}\textsuperscript{drungen}\substDazwischen{}nötigt\substHinten{} sehen, ihre beruflichen Leistungen gering zu schätzen. Ein
                  Schuhfabrikant{[},{]} auch wenn er das miserabelste Zeug liefert
               (besonders, wenn Du Deine Stiefel anderswo beziehst), ein \label{T_L03521-3v}\edtext{schlechter}{\lemma{\textnormal{\emph{schlechter}}}\Cendnote{\textnormal{korrigiert aus »scglechter«}}}\label{T_L03521-3} Jurist, ein untüchtiger Arzt,
               ein mässiger Klavierspieler und selbst ein Schriftsteller, der ohne innere
               Beteiligung, vielleicht fürs tägliche Brot und nur dafür seine sogenannten Novellen
               und Stücke verfasst – sie alle können Deinem Herzen nahe bleiben, wenn sie nur sonst
               redliche, nette und verträgliche Leute vorstellen. Der Einzige, mit dem es Dir nicht
               gelingen wird{[},{]}{ }\strikeout{und darf}{ }\substVorne{}\textsuperscript{freundschaftliche}\substDazwischen{}innere\substHinten{}{ }{\pb}Beziehungen aufrecht zu erhalten,
               wenn Du sein Wirken missbilligst, ist der Dichter. Es wird Dir umso weniger gelingen
               je öfter Du \introOben{}dich gedrungen fühlst\introOben{} nicht nur die eine oder
               andere seiner Leistungen, sondern das Wesentliche seiner Produktion und überdies die
               ganze Richtung, der er als einer der \substVorne{}\textsuperscript{vornehmsten}\substDazwischen{}bekanntesten\substHinten{} Vertreter angehört, als eine unfruchtbare verderbliche und im Niedergang
               befindliche ab\introOben{}zu\introOben{}lehn\substVorne{}\textsuperscript{st}\substDazwischen{}en\substHinten{}. Denn der Beruf des Dichters stellt ja nicht wie der so vieler anderer Leute
               eine zufällige Lebensäusserung dar, die am Ende auch gegen eine andere vertauscht
               werden könnte, nein, sein Beruf ist – je ehrlicher er es mit seiner Kunst meint
               umsomehr – der tiefste Ausdruck seines Wesens, ja seine Seele selbst. Und wer sich
               von dem \introOben{}Gesa{\geminationm}t-\introOben{}Werk eines
               Dichters ohne Anteil abkehrt oder es gar verdammt, der hat \substVorne{}\textsuperscript{ihm}\substDazwischen{}damit\substHinten{} auch \substVorne{}\textsuperscript{schon persönlich}\substDazwischen{}seiner Person\substHinten{} den Rücken gewendet. Und da ich nun einmal zu der Art von Dichtern gehöre,
               die \strikeout{in ihren Werken sich selbst zu geben suchen,
                  jedenfalls} durchaus aus ihrer Persönlichkeit heraus schaffen und Du dem, was
               ich schaffe, wenigstens seit geraumer Zeit so gegenüberstehst, wie wir ja wissen, so
               ist es nur natürlich und konnte gar nicht anders kommen, als dass zwichen Dir und mir
               allmählich {\pb}jene Entfremdung eintreten
               musste, deren wir uns ja längst bewusst sind und \substVorne{}\textsuperscript{unbedingt stimme ich Deiner}\substDazwischen{}kein vernünftiger Mensch wird Deiner\substHinten{} Behauptung \substVorne{}\textsuperscript{bei}\substDazwischen{}widersprechen\substHinten{}, dass Deine und meine Entwicklung seit lange eine gänzlich verschiedene
               Richtung eingeschlagen haben. Es frägt sich eben nur, welche von diesen Richtungen am
               Ende zu einem besseren Ziele führt und das werden Andere zu entscheiden haben als Du
               und ich.\pend
           
\pstart
           {[}hs.:{]} Mit bestem Gruß {\\[\baselineskip]}Dein {\\[\baselineskip]}\spacefill\mbox{A. S.}\pend
           \leftskip=0em{}\selectlanguage{ngerman}\endnumbering\briefempfaengerindex{Goldmann, Paul@\textsc{Goldmann, Paul}!zzzSchnitzler, Arthur@\emph{von Arthur Schnitzler}!1911-02-011@{1. 2. 1911}|)be}\mylabel{L03521h}  \newcommand{\dateiname}{L03521}\newcommand{\titel}{Arthur Schnitzler an Paul Goldmann, 1. 2. 1911}\newcommand{\editorInnen}{Martin Anton Müller und Laura Untner}%% latex-leseansicht-abspann.tex
%% Abspann für die Leseansicht.
%% Der Schalter \ifkorrekturansicht ist bereits durch den Vorspann gesetzt.

%% latex-abspann.tex
%% Gemeinsamer Abspann für Korrekturansicht und Leseansicht.
%% Setzt den Schalter \ifkorrekturansicht voraus (gesetzt in den
%% einbindenden Dateien latex-korrekturansicht-abspann.tex bzw.
%% latex-leseansicht-abspann.tex).
%% ---------------------------------------------------------------

\normalsize

% Das esempio-Environment wird nur in der Leseansicht benötigt
\ifkorrekturansicht\else
\newenvironment{esempio}[3]%
{
    \vspace{1.5ex}
    \rlap{\underline{#1}}
    \par
    \setlength{\parindent}{0cm}
    \nopagebreak
    \leftskip=#2cm
    \rightskip=#3cm
}
{
    \par
}
\fi

\doendnotes{C}
\bigskip
\vfill

\clearpage

\footnotesize

\ifkorrekturansicht
  \lohead{\textsc{register}}
\fi

% theindex-Environment neu definieren ohne reledmac
\makeatletter
\renewenvironment{theindex}{%
  \ifkorrekturansicht
    \section*{\indexname}%
  \else
    \subsubsection*{Index der erwähnten Entitäten}%
  \fi
  \setlength{\parindent}{0pt}%
  \setlength{\parskip}{0pt plus 0.3pt}%
  \let\item\@idxitem
}{%
  \ifkorrekturansicht\clearpage\fi
}
\makeatother

\IfFileExists{\jobname-pw.ind}{\input{\jobname-pw.ind}}{}

% Quellenangabe nur in der Leseansicht
\ifkorrekturansicht\else
% Fallback-Definitionen, falls die .tex-Datei \titel etc. nicht gesetzt hat
\providecommand{\titel}{}
\providecommand{\editorInnen}{}
\providecommand{\dateiname}{\jobname}

\vspace{3cm}

\vfill

\footnotesize
\textsc{Quelle}: \titel. Herausgegeben von {\editorInnen}. In: \emph{Arthur Schnitzler: Briefwechsel mit Autorinnen und Autoren}.
 Digitale Edition, https://schnitzler-briefe.acdh.oeaw.ac.at/{\dateiname}.html (Stand \today)
\fi

\end{document}


