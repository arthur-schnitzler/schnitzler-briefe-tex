%% latex-korrekturansicht-vorspann.tex
%% Vorspann für die Korrekturansicht.
%% Lädt die gemeinsame Datei latex-vorspann.tex mit gesetztem Schalter.

\newif\ifkorrekturansicht
\korrekturansichttrue

\input{../tex-inputs/latex-vorspann}


\section[ Paul Goldmann an Arthur Schnitzler, 27. 11. {[}1905{]}]{L03238 Paul Goldmann an Arthur Schnitzler, 27. 11. {[}1905{]}}
\nopagebreak\mylabel{L03238v}
\rehead{ }\normalsize\beginnumbering\briefempfaengerindex{Schnitzler, Arthur@\textsc{Schnitzler, Arthur}!zzzGoldmann, Paul@\emph{von Paul Goldmann}!1905-11-271@{27. 11. {[}1905{]}}|(be}
\toendnotes[C]{\smallbreak\pagebreak[2]}\Standort{DLA, A:Schnitzler, HS.NZ85.1.3175.}
\physDesc{Brief, 1 Blatt, 2 Seiten, 437 Zeichen
\newline{}Handschrift: blaue Tinte, deutsche Kurrent
\newline{}Schnitzler: mit Bleistift das Jahr »905« vermerkt }\toendnotes[C]{\smallbreak}
\pstart
           \raggedleft{}{\pb}\textcolor{gray}{\textbf{DESSAUERSTRASSE 19}}\oindex{Dessauer Strasse@\textbf{Dessauer Straße}, \emph{Straße (K.STR)}|pw}\pend
           
\pstart
           Berlin\oindex{Berlin@\textbf{Berlin}, \emph{P.PPLC}|pw}, 27. Nov.\pend
           
\pstart\center{}Lieber Freund,\pend\vspace{0.5em}
\pstart
           Ich danke Dir herzlichſt für die \label{K_L03238-1v}\edtext{Überſendung des Buch\pwindex{Zwischenspiel. Komoedie in drei Akten@\emph{Zwischenspiel. Komödie in drei Akten}|pwv}es}{\lemma{\textnormal{\emph{Überſendung des Buches}}}\Cendnote{\textnormal{\emph{Zwischenspiel}\pwindex{Zwischenspiel. Komoedie in drei Akten@\emph{Zwischenspiel. Komödie in drei Akten}|pwk}. Die Widmungsexemplare wurden
                  am 24. 11. 1905
                  versandt (vgl. Arthur Schnitzler: Widmungsexemplar Zwischenspiel für Hugo von
               Hofmannsthal, 24. 11. 1905 und Max Burckhard an Arthur Schnitzler, 30. 11. 1905).}}}\label{K_L03238-1} und freue mich ſchon ſehr
               darauf, es in der erſten freien Stunde zu leſen. \pend
           
\pstart
           Soweit ich nach den Zeitungen urteilen kann, darf man Dich zum Erfolge der \label{K_L03238-2v}\edtext{\textsc{Première}\pwindex{Zwischenspiel. Komoedie in drei Akten@\emph{Zwischenspiel. Komödie in drei Akten}|pwv}}{\lemma{\textnormal{\emph{Première}}}\Cendnote{\textnormal{Am 25. 11. 1905 hatte die Premiere von Schnitzlers{ }\emph{Zwischenspiel}\pwindex{Zwischenspiel. Komoedie in drei Akten@\emph{Zwischenspiel. Komödie in drei Akten}|pwk} am Deutschen Theater
                     Berlin\oindex{Deutsches Theater Berlin@\textbf{Deutsches Theater Berlin}, \emph{Theater (K.THE)}|pwk} in Anwesenheit des Autors stattgefunden.}}}\label{K_L03238-2} beglückwünſchen,
               was ich denn auch mit aller Herzlichkeit thue.\pend
           
\pstart
           {\pb}Hoffentlich biſt Du wohlbehalten \label{K_L03238-3v}\edtext{heimgekehrt}{\lemma{\textnormal{\emph{heimgekehrt}}}\Cendnote{\textnormal{Schnitzler kam am 27. 11. 1905 wieder in
                     Wien\oindex{Wien@\textbf{Wien}, \emph{A.ADM2}|pwk} an.}}}\label{K_L03238-3}. Grüße mir, bitte, Deine Frau\pwindex{Schnitzler, Olga 17.01.1882 – 13.01.1970@\textsc{Schnitzler, Olga} (17.01.1882 – 13.01.1970), \emph{Schauspieler/Schauspielerin, Sänger/Sängerin}|pwv} und ſei ſelbſt \strikeout{von} vielmals gegrüßt von {\\}Deinem getreuen {\\}\spacefill\mbox{Paul Goldmnn}\pend
           \selectlanguage{ngerman}\endnumbering\briefempfaengerindex{Schnitzler, Arthur@\textsc{Schnitzler, Arthur}!zzzGoldmann, Paul@\emph{von Paul Goldmann}!1905-11-271@{27. 11. {[}1905{]}}|)be}\mylabel{L03238h}  \normalsize

\doendnotes{C}
\bigskip
\vfill

\clearpage

\footnotesize

\lohead{\textsc{register}}

% Definiere theindex-Environment komplett neu ohne reledmac
\makeatletter
\renewenvironment{theindex}{%
  \section*{\indexname}%
  \setlength{\parindent}{0pt}%
  \setlength{\parskip}{0pt plus 0.3pt}%
  \let\item\@idxitem
}{%
  \clearpage
}
\makeatother

\IfFileExists{\jobname-pw.ind}{\input{\jobname-pw.ind}}{}

\end{document}

      