%% latex-leseansicht-vorspann.tex
%% Vorspann für die Leseansicht.
%% Lädt die gemeinsame Datei latex-vorspann.tex mit nicht gesetztem Schalter.

\newif\ifkorrekturansicht
\korrekturansichtfalse

\input{../tex-inputs/latex-vorspann}


               \section[Arthur Schnitzler an Richard Beer-Hofmann, 23. 8. 1903]{ Arthur Schnitzler an Richard Beer-Hofmann, 23. 8. 1903}\nopagebreak\mylabel{v}\rehead{ }\begin{ledgroupsized}[t]{13cm}\normalsize\beginnumbering\briefempfaengerindex{Beer-Hofmann, Richard@\textsc{Beer-Hofmann, Richard}!zzzSchnitzler, Arthur@\emph{von Arthur Schnitzler}!1903-08-232@{23. 8. 1903}|(be} \toendnotes[C]{\smallbreak\pagebreak[2]} \Standort{YCGL, MSS 31.}
\physDesc{Brief, 1 Blatt, 3 Seiten, Umschlag
\newline{}Handschrift: Bleistift, deutsche Kurrent\newline{}Versand: Stempel: »\nobreak{}\oindex{IX., Alsergrund@\textbf{IX., Alsergrund}|pwk}Wien \textcolor{gray}{9/3}, 24. \textcolor{gray}{8}. 0\textcolor{gray}{3}, 7–\textcolor{gray}{9}V\nobreak{}«.  }\buchAbdrucke{\weitereDrucke{Arthur Schnitzler, Richard Beer-Hofmann: \emph{Briefwechsel 1891–1931}. Hg. Konstanze Fliedl. Wien, Zürich: \emph{Europaverlag} 1992, S. 163–164.} }\toendnotes[C]{\smallbreak}\pstart{}{\pb}Herrn \textsc{Dr Richard }\pend{}\pstart{}\textsc{Beer-Hofma{\geminationn}}\pend{}\pstart{}\textsc{Rodaun}\oindex{Rodaun@\textbf{Rodaun}|pw}{ }\textsuperscript{b}/\textsc{Wien}\oindex{Wien@\textbf{Wien}|pw}\pend{}\pstart{}\textsc{Liesinger Straße 2\oindex{Liesingerstrasse@\textbf{Liesingerstraße}|pw}.}\pend{}{\bigskip}\pstart
           \raggedleft{}{\pb}23. 8. 903. \pend
           \pstart
           lieber Richard, mein Telegr. iſt eben an Sie abgegangen; ich füge
               brieflich den Vorſchlag bei, dſs Sie dann gleich bei uns in der Gentzgaſſe\oindex{Gentzgasse@\textbf{Gentzgasse}|pw} eſſen mögen. Vielleicht hat Ihre Frau\pwindex{Beer-Hofmann, Paula 25.02.1879 – 30.10.1939@\textsc{Beer-Hofmann, Paula} (25.02.1879 – 30.10.1939)|pwv} am gleichen Tag etwas in Wien\oindex{Wien@\textbf{Wien}|pw} zu thun, u da{\geminationn} gilt das
               gleiche, ebenſo herzlich, für ſie. –\pend
           \pstart
           Möchten Sie mir auch in Kürze mittheilen, wie {\pb}Sie das
               ſ. Z. in Ihrem Fall mit Honoraren und Trinkgeldern (von den Taxen abgeſehen) gehalten
               haben?\pend
           \pstart
           Ich verſtändige niemanden von dem Vorgang, ehe meine Mama\pwindex{Schnitzler, Louise 08.07.1840 – 09.09.1911@\textsc{Schnitzler, Louise} (08.07.1840 – 09.09.1911)|pwv} wieder zurück iſt der ich auch erſt dann Mittheilg
               machen werde. Alſo ſagen Sie bitte auch niemandem was davon. –\pend
           \pstart
           Meine Reiſe war ſehr ſchön; das neue Hotel\oindex{Palast Hotel Lido@\textbf{Palast Hotel Lido}|pwv} in Riva\oindex{Riva del Garda@\textbf{Riva del Garda}|pw} ſcheint angenehm zu ſein; ich
               denke {\pb}mit Olga\pwindex{Schnitzler, Olga 17.01.1882 – 13.01.1970@\textsc{Schnitzler, Olga} (17.01.1882 – 13.01.1970), \emph{Schauspielerin, Sängerin}|pw}{ }Mitte September dorthin zu reiſen. Vielleicht ſpäter Meran\oindex{Meran@\textbf{Meran}|pw}.\pend
           \pstart
           Herzlichſt Ihr{\\[\baselineskip]}\spacefill\mbox{Arthur}\pend
           \leftskip=0em{}          \endnumbering\briefempfaengerindex{Beer-Hofmann, Richard@\textsc{Beer-Hofmann, Richard}!zzzSchnitzler, Arthur@\emph{von Arthur Schnitzler}!1903-08-232@{23. 8. 1903}|)be}\mylabel{h}\end{ledgroupsized}  \newcommand{\dateiname}{L01313}\newcommand{\titel}{Arthur Schnitzler an Richard Beer-Hofmann, 23. 8. 1903}\newcommand{\editorInnen}{Martin Anton Müller und Gerd-Hermann Susen}%% latex-leseansicht-abspann.tex
%% Abspann für die Leseansicht.
%% Der Schalter \ifkorrekturansicht ist bereits durch den Vorspann gesetzt.

%% latex-abspann.tex
%% Gemeinsamer Abspann für Korrekturansicht und Leseansicht.
%% Setzt den Schalter \ifkorrekturansicht voraus (gesetzt in den
%% einbindenden Dateien latex-korrekturansicht-abspann.tex bzw.
%% latex-leseansicht-abspann.tex).
%% ---------------------------------------------------------------

\normalsize

% Das esempio-Environment wird nur in der Leseansicht benötigt
\ifkorrekturansicht\else
\newenvironment{esempio}[3]%
{
    \vspace{1.5ex}
    \rlap{\underline{#1}}
    \par
    \setlength{\parindent}{0cm}
    \nopagebreak
    \leftskip=#2cm
    \rightskip=#3cm
}
{
    \par
}
\fi

\doendnotes{C}
\bigskip
\vfill

\clearpage

\footnotesize

\ifkorrekturansicht
  \lohead{\textsc{register}}
\fi

% theindex-Environment neu definieren ohne reledmac
\makeatletter
\renewenvironment{theindex}{%
  \ifkorrekturansicht
    \section*{\indexname}%
  \else
    \subsubsection*{Index der erwähnten Entitäten}%
  \fi
  \setlength{\parindent}{0pt}%
  \setlength{\parskip}{0pt plus 0.3pt}%
  \let\item\@idxitem
}{%
  \ifkorrekturansicht\clearpage\fi
}
\makeatother

\IfFileExists{\jobname-pw.ind}{\input{\jobname-pw.ind}}{}

% Quellenangabe nur in der Leseansicht
\ifkorrekturansicht\else
% Fallback-Definitionen, falls die .tex-Datei \titel etc. nicht gesetzt hat
\providecommand{\titel}{}
\providecommand{\editorInnen}{}
\providecommand{\dateiname}{\jobname}

\vspace{3cm}

\vfill

\footnotesize
\textsc{Quelle}: \titel. Herausgegeben von {\editorInnen}. In: \emph{Arthur Schnitzler: Briefwechsel mit Autorinnen und Autoren}.
 Digitale Edition, https://schnitzler-briefe.acdh.oeaw.ac.at/{\dateiname}.html (Stand \today)
\fi

\end{document}


      