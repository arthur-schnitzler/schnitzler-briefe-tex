%% latex-korrekturansicht-vorspann.tex
%% Vorspann für die Korrekturansicht.
%% Lädt die gemeinsame Datei latex-vorspann.tex mit gesetztem Schalter.

\newif\ifkorrekturansicht
\korrekturansichttrue

\input{../tex-inputs/latex-vorspann}


\section[Arthur Schnitzler an Richard Beer-Hofmann, 23. 8. 1903]{L01313 Arthur Schnitzler an Richard Beer-Hofmann, 23. 8. 1903}
\nopagebreak\mylabel{L01313v}
\rehead{ }\normalsize\beginnumbering\briefempfaengerindex{Beer-Hofmann, Richard@\textsc{Beer-Hofmann, Richard}!zzzSchnitzler, Arthur@\emph{von Arthur Schnitzler}!1903-08-232@{23. 8. 1903}|(be}
\toendnotes[C]{\smallbreak\pagebreak[2]}\Standort{YCGL, MSS 31.}
\physDesc{Brief, 1 Blatt, 3 Seiten, Umschlag, 811 Zeichen
\newline{}Handschrift: Bleistift, deutsche Kurrent
\newline{}Versand: Stempel: »\nobreak{}\oindex{IX., Alsergrund@\textbf{IX., Alsergrund}, \emph{A.ADM3}|pwk}Wien \textcolor{gray}{9/3}, 24. \textcolor{gray}{8}. 0\textcolor{gray}{3}, 7–\textcolor{gray}{9}V\nobreak{}«.  }
\buchAbdrucke{\weitereDrucke{Arthur Schnitzler, Richard Beer-Hofmann: \emph{Briefwechsel 1891–1931}. Wien, Zürich: \emph{Europaverlag} 1992, S. 163–164.} }\toendnotes[C]{\smallbreak}\pstart{}{\pb}Herrn \textsc{Dr Richard}\pend{}\pstart{}\textsc{Beer-Hofma{\geminationn}}\pend{}\pstart{}\textsc{Rodaun}\oindex{Rodaun@\textbf{Rodaun}, \emph{A.ADM4}|pw}{ }\textsuperscript{b}/\textsc{Wien}\oindex{Wien@\textbf{Wien}, \emph{A.ADM2}|pw}\pend{}\pstart{}\textsc{Liesinger Straße 2\oindex{Liesingerstrasse@\textbf{Liesingerstraße}, \emph{Straße (K.STR)}|pw}.}\pend{}{\bigskip}\vspace{1em}
\pstart
           \raggedleft{}{\pb}23. 8. 903. \pend
           \vspace{0.5em}
\pstart
           lieber Richard, mein Telegr. iſt eben an Sie abgegangen; ich füge
               brieflich den Vorſchlag bei, dſs Sie dann gleich bei uns in der Gentzgaſſe\oindex{Gentzgasse@\textbf{Gentzgasse}, \emph{Straße (K.STR)}|pw} eſſen mögen. Vielleicht hat Ihre Frau\pwindex{Beer-Hofmann, Paula 25.02.1879 – 30.10.1939@\textsc{Beer-Hofmann, Paula} (25.02.1879 – 30.10.1939)|pwv} am gleichen Tag etwas in Wien\oindex{Wien@\textbf{Wien}, \emph{A.ADM2}|pw} zu thun, u da{\geminationn}
               gilt das gleiche, ebenſo herzlich, für ſie. –\pend
           
\pstart
           Möchten Sie mir auch in Kürze mittheilen, wie {\pb}Sie das
               ſ. Z. in Ihrem Fall mit Honoraren und Trinkgeldern (von den Taxen abgeſehen) gehalten
               haben?\pend
           
\pstart
           Ich verſtändige niemanden von dem Vorgang, ehe meine Mama\pwindex{Schnitzler, Louise 1840-07-08 – 1911-09-09@\textsc{Schnitzler, Louise} (1840-07-08 – 1911-09-09)|pwv} wieder zurück iſt der ich auch erſt
               dann Mittheilg machen werde. Alſo ſagen Sie bitte auch niemandem was davon. –\pend
           
\pstart
           Meine Reiſe war ſehr ſchön; das neue Hotel\oindex{Palast Hotel Lido@\textbf{Palast Hotel Lido}, \emph{Hotel (K.HTL)}|pwv} in Riva\oindex{Riva del Garda@\textbf{Riva del Garda}, \emph{P.PPLA3}|pw} ſcheint
               angenehm zu ſein; ich denke {\pb}mit Olga\pwindex{Schnitzler, Olga 17.01.1882 – 13.01.1970@\textsc{Schnitzler, Olga} (17.01.1882 – 13.01.1970), \emph{Schauspieler/Schauspielerin, Sänger/Sängerin}|pw}{ }Mitte September dorthin zu reiſen. Vielleicht ſpäter Meran\oindex{Meran@\textbf{Meran}, \emph{P.PPLA3}|pw}.\pend
           
\pstart
           Herzlichſt Ihr{\\[\baselineskip]}\spacefill\mbox{Arthur}\pend
           \leftskip=0em{}\selectlanguage{ngerman}\endnumbering\briefempfaengerindex{Beer-Hofmann, Richard@\textsc{Beer-Hofmann, Richard}!zzzSchnitzler, Arthur@\emph{von Arthur Schnitzler}!1903-08-232@{23. 8. 1903}|)be}\mylabel{L01313h}  \normalsize

\doendnotes{C}
\bigskip
\vfill

\clearpage

\footnotesize

\lohead{\textsc{register}}

% Definiere theindex-Environment komplett neu ohne reledmac
\makeatletter
\renewenvironment{theindex}{%
  \section*{\indexname}%
  \setlength{\parindent}{0pt}%
  \setlength{\parskip}{0pt plus 0.3pt}%
  \let\item\@idxitem
}{%
  \clearpage
}
\makeatother

\IfFileExists{\jobname-pw.ind}{\input{\jobname-pw.ind}}{}

\end{document}

      