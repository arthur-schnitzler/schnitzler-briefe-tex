%% latex-leseansicht-vorspann.tex
%% Vorspann für die Leseansicht.
%% Lädt die gemeinsame Datei latex-vorspann.tex mit nicht gesetztem Schalter.

\newif\ifkorrekturansicht
\korrekturansichtfalse

\input{../tex-inputs/latex-vorspann}


\section[Hugo von Hofmannsthal an Arthur Schnitzler, 31. 7. [1899]]{L00955 Hugo von Hofmannsthal an Arthur Schnitzler, 31. 7. [1899]}
\nopagebreak\mylabel{L00955v}
\rehead{ }\normalsize\beginnumbering\briefempfaengerindex{Schnitzler, Arthur@\textsc{Schnitzler, Arthur}!zzzHofmannsthal, Hugo von@\emph{von Hugo von Hofmannsthal}!1899-07-314@{31. 7. [1899]}|(be}
\toendnotes[C]{\smallbreak\pagebreak[2]}
\correspDesc{Versand  durch Hugo von Hofmannsthal am 31. 7. [1899] in Altaussee
\newline{}Erhalt  durch Arthur Schnitzler im Zeitraum [1. 8. 1899
                  – 5. 8. 1899?] in Toblach}\toendnotes[C]{\smallbreak}
\Standort{CUL, Schnitzler, B 43.}
\physDesc{Brief, 1 Blatt, 3 Seiten, 573 Zeichen
\newline{}Handschrift: schwarze Tinte, deutsche Kurrent
\newline{}Schnitzler: mit Bleistift die Jahreszahl ergänzt: »99« 
\newline{}Ordnung: mit Bleistift von unbekannter Hand eine frühere Zählung
                                 überarbeitet: »15\substVorne{}\textsuperscript{6}\substDazwischen{}3\substHinten{}« }
\buchAbdrucke{\weitereDrucke{Hugo von Hofmannsthal, Arthur Schnitzler: \emph{Briefwechsel}. Herausgegeben von Therese Nickl und Heinrich Schnitzler. Frankfurt am Main: \emph{S. Fischer} 1964, S. 128.} }\toendnotes[C]{\smallbreak}
\pstart
           \raggedleft{}{\pb}Alt-Aussee\oindex{Altaussee@\textbf{Altaussee}, \emph{Verwaltungsgebiet}|pw}{ }31. VII.\pend
           
\pstart{}mein lieber Arthur\pend\vspace{0.5em}
\pstart
           denken Sie doch was uns ein neues Stück\pwindex{Schnitzler, Arthur 15.\,5.\,1862 Wien – 21.\,10.\,1931 ebd.@\textsc{Schnitzler, Arthur} (15.\,5.\,1862 Wien – 21.\,10.\,1931 ebd.), \emph{Schriftsteller, Mediziner}!Schleier der Beatrice. Schauspiel in fünf Akten@\strich\emph{Der Schleier der Beatrice. Schauspiel in fünf Akten}|pwv} von Ihnen für eine Freude iſt, dem Richard\pwindex{Beer-Hofmann, Richard 11.\,7.\,1866 Wien – 26.\,9.\,1945 New York City@\textsc{Beer-Hofmann, Richard} (11.\,7.\,1866 Wien – 26.\,9.\,1945 New York City), \emph{Schriftsteller}|pw} und mir. Ich war{ }ſo froh, daſs Sie mir über Ihre Arbeit\pwindex{Schnitzler, Arthur 15.\,5.\,1862 Wien – 21.\,10.\,1931 ebd.@\textsc{Schnitzler, Arthur} (15.\,5.\,1862 Wien – 21.\,10.\,1931 ebd.), \emph{Schriftsteller, Mediziner}!Schleier der Beatrice. Schauspiel in fünf Akten@\strich\emph{Der Schleier der Beatrice. Schauspiel in fünf Akten}|pwv} und über eine Beſſerung in Richards\pwindex{Beer-Hofmann, Richard 11.\,7.\,1866 Wien – 26.\,9.\,1945 New York City@\textsc{Beer-Hofmann, Richard} (11.\,7.\,1866 Wien – 26.\,9.\,1945 New York City), \emph{Schriftsteller}|pw} Sti{\geminationm}ung{ }ſchreiben. Ich lebe jetzt hier ein gedankenloſes Leben mit \textsc{tennys} und {\pb}\textsc{bycicle-polo}, nach einer Zeit werde ich an den 3\textsuperscript{ten}{ }Act\pwindex{Hofmannsthal, Hugo von 1.\,2.\,1874 Wien – 15.\,7.\,1929 Rodaun@\textsc{Hofmannsthal, Hugo von} (1.\,2.\,1874 Wien – 15.\,7.\,1929 Rodaun), \emph{Schriftsteller}!Bergwerk zu Falun@\strich\emph{Das Bergwerk zu Falun}|pwv} gehen. Vielleicht, wenn
               Sie nach Iſchl\oindex{Bad Ischl@\textbf{Bad Ischl}|pw} gehen, in Iſchl\oindex{Bad Ischl@\textbf{Bad Ischl}|pw}! oder beide in Salzburg\oindex{Salzburg@\textbf{Salzburg}, \emph{Verwaltungsgebiet}|pw}?\pend
           
\pstart
           Ich wünſche Ihnen und den andern möglichſt viel Freude von der Fußpartie.\pend
           
\pstart
           Clemens Franckenstein\pwindex{Franckenstein, Clemens von 14.\,7.\,1875 Wiesentheid – 19.\,8.\,1942 Hechendorf am Pilsensee@\textsc{Franckenstein, Clemens von} (14.\,7.\,1875 Wiesentheid – 19.\,8.\,1942 Hechendorf am Pilsensee), \emph{Theaterleiter, Komponist, Dirigent}|pw}{ }{\pb}läſst den Waſſermann\pwindex{Wassermann, Jakob 10.\,3.\,1873 Fürth – 1.\,1.\,1934 Altaussee@\textsc{Wassermann, Jakob} (10.\,3.\,1873 Fürth – 1.\,1.\,1934 Altaussee), \emph{Schriftsteller}|pw} fragen, was mit dem \label{K_L00955-1v}\edtext{Operntext}{\lemma{\textnormal{\emph{Operntext}}}\Cendnote{\textnormal{Wassermann\pwindex{Wassermann, Jakob 10.\,3.\,1873 Fürth – 1.\,1.\,1934 Altaussee@\textsc{Wassermann, Jakob} (10.\,3.\,1873 Fürth – 1.\,1.\,1934 Altaussee), \emph{Schriftsteller}|pwk} hatte im Vorjahr \emph{Lorenza Burgkmair. Karnevals-Stück in drei Akten}\pwindex{Wassermann, Jakob 10.\,3.\,1873 Fürth – 1.\,1.\,1934 Altaussee@\textsc{Wassermann, Jakob} (10.\,3.\,1873 Fürth – 1.\,1.\,1934 Altaussee), \emph{Schriftsteller}!Lorenza Burgkmair. Karnevals-Stück in drei Akten@\strich\emph{Lorenza Burgkmair. Karnevals-Stück in drei Akten}|pwk}
                  veröffentlicht. Er arbeitete es als Libretto um, das die Textgrundlage von Clemens Franckensteins\pwindex{Franckenstein, Clemens von 14.\,7.\,1875 Wiesentheid – 19.\,8.\,1942 Hechendorf am Pilsensee@\textsc{Franckenstein, Clemens von} (14.\,7.\,1875 Wiesentheid – 19.\,8.\,1942 Hechendorf am Pilsensee), \emph{Theaterleiter, Komponist, Dirigent}|pwk} dreiaktiger Oper \emph{Fortunatus}\pwindex{Wassermann, Jakob 10.\,3.\,1873 Fürth – 1.\,1.\,1934 Altaussee@\textsc{Wassermann, Jakob} (10.\,3.\,1873 Fürth – 1.\,1.\,1934 Altaussee), \emph{Schriftsteller}!Fortunatus. Oper in 3 Akten, op. 16@\strich\emph{Fortunatus. Oper in 3 Akten, op. 16}|pwk} (1903) bildete.}}}\label{K_L00955-1} iſt.\pend
           
\pstart
           Herzlich Ihr{\\[\baselineskip]}\spacefill\mbox{Hugo.}\pend
           \leftskip=0em{}\selectlanguage{ngerman}\endnumbering\briefempfaengerindex{Schnitzler, Arthur@\textsc{Schnitzler, Arthur}!zzzHofmannsthal, Hugo von@\emph{von Hugo von Hofmannsthal}!1899-07-314@{31. 7. [1899]}|)be}\mylabel{L00955h}  \newcommand{\dateiname}{L00955}\newcommand{\titel}{Hugo von Hofmannsthal an Arthur Schnitzler, 31. 7. [1899]}\newcommand{\editorInnen}{Martin Anton Müller und Gerd-Hermann Susen}%% latex-leseansicht-abspann.tex
%% Abspann für die Leseansicht.
%% Der Schalter \ifkorrekturansicht ist bereits durch den Vorspann gesetzt.

%% latex-abspann.tex
%% Gemeinsamer Abspann für Korrekturansicht und Leseansicht.
%% Setzt den Schalter \ifkorrekturansicht voraus (gesetzt in den
%% einbindenden Dateien latex-korrekturansicht-abspann.tex bzw.
%% latex-leseansicht-abspann.tex).
%% ---------------------------------------------------------------

\normalsize

% Das esempio-Environment wird nur in der Leseansicht benötigt
\ifkorrekturansicht\else
\newenvironment{esempio}[3]%
{
    \vspace{1.5ex}
    \rlap{\underline{#1}}
    \par
    \setlength{\parindent}{0cm}
    \nopagebreak
    \leftskip=#2cm
    \rightskip=#3cm
}
{
    \par
}
\fi

\doendnotes{C}
\bigskip
\vfill

\clearpage

\footnotesize

\ifkorrekturansicht
  \lohead{\textsc{register}}
\fi

% theindex-Environment neu definieren ohne reledmac
\makeatletter
\renewenvironment{theindex}{%
  \ifkorrekturansicht
    \section*{\indexname}%
  \else
    \subsubsection*{Index der erwähnten Entitäten}%
  \fi
  \setlength{\parindent}{0pt}%
  \setlength{\parskip}{0pt plus 0.3pt}%
  \let\item\@idxitem
}{%
  \ifkorrekturansicht\clearpage\fi
}
\makeatother

\IfFileExists{\jobname-pw.ind}{\input{\jobname-pw.ind}}{}

% Quellenangabe nur in der Leseansicht
\ifkorrekturansicht\else
% Fallback-Definitionen, falls die .tex-Datei \titel etc. nicht gesetzt hat
\providecommand{\titel}{}
\providecommand{\editorInnen}{}
\providecommand{\dateiname}{\jobname}

\vspace{3cm}

\vfill

\footnotesize
\textsc{Quelle}: \titel. Herausgegeben von {\editorInnen}. In: \emph{Arthur Schnitzler: Briefwechsel mit Autorinnen und Autoren}.
 Digitale Edition, https://schnitzler-briefe.acdh.oeaw.ac.at/{\dateiname}.html (Stand \today)
\fi

\end{document}


