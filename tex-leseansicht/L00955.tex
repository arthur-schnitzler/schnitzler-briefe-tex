%% latex-korrekturansicht-vorspann.tex
%% Vorspann für die Korrekturansicht.
%% Lädt die gemeinsame Datei latex-vorspann.tex mit gesetztem Schalter.

\newif\ifkorrekturansicht
\korrekturansichttrue

\input{../tex-inputs/latex-vorspann}


\section[Hugo von Hofmannsthal an Arthur Schnitzler, 31. 7. {[}1899{]}]{L00955 Hugo von Hofmannsthal an Arthur Schnitzler, 31. 7. {[}1899{]}}
\nopagebreak\mylabel{L00955v}
\rehead{ }\normalsize\beginnumbering\briefempfaengerindex{Schnitzler, Arthur@\textsc{Schnitzler, Arthur}!zzzHofmannsthal, Hugo von@\emph{von Hugo von Hofmannsthal}!1899-07-314@{31. 7. {[}1899{]}}|(be}
\toendnotes[C]{\smallbreak\pagebreak[2]}\Standort{CUL, Schnitzler, B 43.}
\physDesc{Brief, 1 Blatt, 3 Seiten, 573 Zeichen
\newline{}Handschrift: schwarze Tinte, deutsche Kurrent
\newline{}Schnitzler: mit Bleistift die Jahreszahl ergänzt: »99« 
\newline{}Ordnung: mit Bleistift von unbekannter Hand eine frühere Zählung
                                 überarbeitet: »15\substVorne{}\textsuperscript{6}\substDazwischen{}3\substHinten{}« }
\buchAbdrucke{\weitereDrucke{Hugo von Hofmannsthal, Arthur Schnitzler: \emph{Briefwechsel}. Frankfurt am Main: \emph{S. Fischer} 1964, S. 128.} }\toendnotes[C]{\smallbreak}
\pstart
           \raggedleft{}{\pb}Alt-Aussee\oindex{Altaussee@\textbf{Altaussee}, \emph{A.ADM3}|pw}{ }31. VII.\pend
           
\pstart{}mein lieber Arthur\pend\vspace{0.5em}
\pstart
           denken Sie doch was uns ein neues Stück\pwindex{Schleier der Beatrice. Schauspiel in fuenf Akten@\emph{Der Schleier der Beatrice. Schauspiel in fünf Akten}|pwv} von Ihnen für eine Freude iſt, dem Richard\pwindex{Beer-Hofmann, Richard 1866-07-11 – 1945-09-26@\textsc{Beer-Hofmann, Richard} (1866-07-11 – 1945-09-26), \emph{Schriftsteller/Schriftstellerin}|pw} und mir. Ich war ſo froh, daſs Sie mir über Ihre Arbeit\pwindex{Schleier der Beatrice. Schauspiel in fuenf Akten@\emph{Der Schleier der Beatrice. Schauspiel in fünf Akten}|pwv} und über eine Beſſerung in Richards\pwindex{Beer-Hofmann, Richard 1866-07-11 – 1945-09-26@\textsc{Beer-Hofmann, Richard} (1866-07-11 – 1945-09-26), \emph{Schriftsteller/Schriftstellerin}|pw} Sti{\geminationm}ung
               ſchreiben. Ich lebe jetzt hier ein gedankenloſes Leben mit \textsc{tennys} und {\pb}\textsc{bycicle-polo}, nach einer Zeit werde ich an den 3\textsuperscript{ten}{ }Act\pwindex{Bergwerk zu Falun@\emph{Das Bergwerk zu Falun}|pwv} gehen. Vielleicht, wenn
               Sie nach Iſchl\oindex{Bad Ischl@\textbf{Bad Ischl}, \emph{P.PPL}|pw} gehen, in Iſchl\oindex{Bad Ischl@\textbf{Bad Ischl}, \emph{P.PPL}|pw}! oder beide in Salzburg\oindex{Salzburg@\textbf{Salzburg}, \emph{A.ADM2}|pw}?\pend
           
\pstart
           Ich wünſche Ihnen und den andern möglichſt viel Freude von der Fußpartie.\pend
           
\pstart
           Clemens Franckenstein\pwindex{Franckenstein, Clemens von 14.07.1875 – 19.08.1942@\textsc{Franckenstein, Clemens von} (14.07.1875 – 19.08.1942), \emph{Theaterleiter/Theaterleiterin, Komponist/Komponistin, Dirigent/Dirigentin}|pw}{ }{\pb}läſst den Waſſermann\pwindex{Wassermann, Jakob 10.03.1873 – 01.01.1934@\textsc{Wassermann, Jakob} (10.03.1873 – 01.01.1934), \emph{Schriftsteller/Schriftstellerin}|pw} fragen, was mit dem \label{K_L00955-1v}\edtext{Operntext}{\lemma{\textnormal{\emph{Operntext}}}\Cendnote{\textnormal{Wassermann\pwindex{Wassermann, Jakob 10.03.1873 – 01.01.1934@\textsc{Wassermann, Jakob} (10.03.1873 – 01.01.1934), \emph{Schriftsteller/Schriftstellerin}|pwk} hatte im Vorjahr \emph{Lorenza Burgkmair. Karnevals-Stück in drei Akten}\pwindex{Lorenza Burgkmair. Karnevals-Stueck in drei Akten@\emph{Lorenza Burgkmair. Karnevals-Stück in drei Akten}|pwk}
                  veröffentlicht. Er arbeitete es als Libretto um, das die Textgrundlage von Clemens Franckensteins\pwindex{Franckenstein, Clemens von 14.07.1875 – 19.08.1942@\textsc{Franckenstein, Clemens von} (14.07.1875 – 19.08.1942), \emph{Theaterleiter/Theaterleiterin, Komponist/Komponistin, Dirigent/Dirigentin}|pwk} dreiaktiger Oper \emph{Fortunatus}\pwindex{Fortunatus. Oper in 3 Akten, op. 16@\emph{Fortunatus. Oper in 3 Akten, op. 16}|pwk} (1903) bildete.}}}\label{K_L00955-1} iſt.\pend
           
\pstart
           Herzlich Ihr{\\[\baselineskip]}\spacefill\mbox{Hugo.}\pend
           \leftskip=0em{}\selectlanguage{ngerman}\endnumbering\briefempfaengerindex{Schnitzler, Arthur@\textsc{Schnitzler, Arthur}!zzzHofmannsthal, Hugo von@\emph{von Hugo von Hofmannsthal}!1899-07-314@{31. 7. {[}1899{]}}|)be}\mylabel{L00955h}  \normalsize

\doendnotes{C}
\bigskip
\vfill

\clearpage

\footnotesize

\lohead{\textsc{register}}

% Definiere theindex-Environment komplett neu ohne reledmac
\makeatletter
\renewenvironment{theindex}{%
  \section*{\indexname}%
  \setlength{\parindent}{0pt}%
  \setlength{\parskip}{0pt plus 0.3pt}%
  \let\item\@idxitem
}{%
  \clearpage
}
\makeatother

\IfFileExists{\jobname-pw.ind}{\input{\jobname-pw.ind}}{}

\end{document}

      