%% latex-korrekturansicht-vorspann.tex
%% Vorspann für die Korrekturansicht.
%% Lädt die gemeinsame Datei latex-vorspann.tex mit gesetztem Schalter.

\newif\ifkorrekturansicht
\korrekturansichttrue

\input{../tex-inputs/latex-vorspann}


\section[Hugo von Hofmannsthal an Arthur Schnitzler, 9. {[}7. 1897{]}]{L00697 Hugo von Hofmannsthal an Arthur Schnitzler, 9. {[}7. 1897{]}}
\nopagebreak\mylabel{L00697v}
\rehead{ }\normalsize\beginnumbering\briefempfaengerindex{Schnitzler, Arthur@\textsc{Schnitzler, Arthur}!zzzHofmannsthal, Hugo von@\emph{von Hugo von Hofmannsthal}!1897-07-092@{9. {[}7. 1897{]}}|(be}
\toendnotes[C]{\smallbreak\pagebreak[2]}\Standort{CUL, Schnitzler, B 43.}
\physDesc{Brief, 1 Blatt, 2 Seiten, 703 Zeichen
\newline{}Handschrift: schwarze Tinte, deutsche Kurrent
\newline{}Schnitzler: mit Bleistift Monat und Jahreszahl ergänzt: »7. 97« 
\newline{}Ordnung: 1) mit Bleistift von unbekannter Hand nummeriert: »\strikeout{95}«  2) mit Bleistift von unbekannter Hand nummeriert:
                                    »93«}
\buchAbdrucke{\weitereDrucke{Hugo von Hofmannsthal, Arthur Schnitzler: \emph{Briefwechsel}. Frankfurt am Main: \emph{S. Fischer} 1964, S. 89–90.} }
\pstart
           \raggedleft{}{\pb}Bad Fuſch\oindex{Bad Fusch@\textbf{Bad Fusch}, \emph{A.ADM3}|pw}{ }9\textsuperscript{ten}\pend
           \vspace{0.5em}
\pstart
           lieber Arthur, danke für Ihren lieben Brief. Ich bin durch
               aufeinanderfolgende ſehr angſtvolle und undeutliche Telegramme von Poldy\pwindex{Andrian-Werburg, Leopold von 09.05.1875 – 19.11.1951@\textsc{Andrian-Werburg, Leopold von} (09.05.1875 – 19.11.1951), \emph{Schriftsteller/Schriftstellerin, Diplomat/Diplomatin}|pw}{ }ſehr beunruhigt. Er will mich bei ſich haben, was
               mir begreiflicherweiſe aus vielen Gründen ſehr ſchwer fällt. Bitte antworten Sie mir
                  \uline{umgehend} mit 2 Zeilen, ob Sie Ihre Fahrt nach Wien\oindex{Wien@\textbf{Wien}, \emph{A.ADM2}|pw}, die doch unvermeidlich ſcheint, nicht ſchon
               in den nächſten {\pb}Tagen machen und
               ihn dabei (Vorderbrühl Liechtenſteinſtraße 10\oindex{Liechtensteinstrasse [Hinterbruehl]@\textbf{Liechtensteinstraße [Hinterbrühl]}, \emph{Straße (K.STR)}|pw})
               beſuchen könnten, ebenſo als Arzt wie als Freund. Ich kenne mich nicht aus, werde
               alſo eventuell doch hinfahren.\pend
           
\pstart
           Unſer \textsc{rendez vous} in \textsc{Salzburg}\oindex{Salzburg@\textbf{Salzburg}, \emph{A.ADM2}|pw} bleibt, wenn was Gott verhüte nichts ganz beſondres dazwiſchenkommt, für den
                     23\textsuperscript{ten} oder 24\textsuperscript{ten} July.\pend
           
\pstart
           Von Herzen{\\[\baselineskip]}Ihr{\\[\baselineskip]}\spacefill\mbox{Hugo.}\pend
           \leftskip=0em{}\selectlanguage{ngerman}\endnumbering\briefempfaengerindex{Schnitzler, Arthur@\textsc{Schnitzler, Arthur}!zzzHofmannsthal, Hugo von@\emph{von Hugo von Hofmannsthal}!1897-07-092@{9. {[}7. 1897{]}}|)be}\mylabel{L00697h}  \normalsize

\doendnotes{C}
\bigskip
\vfill

\clearpage

\footnotesize

\lohead{\textsc{register}}

% Definiere theindex-Environment komplett neu ohne reledmac
\makeatletter
\renewenvironment{theindex}{%
  \section*{\indexname}%
  \setlength{\parindent}{0pt}%
  \setlength{\parskip}{0pt plus 0.3pt}%
  \let\item\@idxitem
}{%
  \clearpage
}
\makeatother

\IfFileExists{\jobname-pw.ind}{\input{\jobname-pw.ind}}{}

\end{document}

      