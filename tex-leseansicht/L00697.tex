%% latex-leseansicht-vorspann.tex
%% Vorspann für die Leseansicht.
%% Lädt die gemeinsame Datei latex-vorspann.tex mit nicht gesetztem Schalter.

\newif\ifkorrekturansicht
\korrekturansichtfalse

\input{../tex-inputs/latex-vorspann}


\section[Hugo von Hofmannsthal an Arthur Schnitzler, 9. [7. 1897]]{L00697 Hugo von Hofmannsthal an Arthur Schnitzler, 9. [7. 1897]}
\nopagebreak\mylabel{L00697v}
\rehead{ }\normalsize\beginnumbering\briefempfaengerindex{Schnitzler, Arthur@\textsc{Schnitzler, Arthur}!zzzHofmannsthal, Hugo von@\emph{von Hugo von Hofmannsthal}!1897-07-092@{9. [7. 1897]}|(be}
\toendnotes[C]{\smallbreak\pagebreak[2]}
\correspDesc{Versand  durch Hugo von Hofmannsthal am 9. [7. 1897] in Bad Fusch
\newline{}Erhalt  durch Arthur Schnitzler im Zeitraum [10. 7. 1897
                  – 14. 7. 1897?] in Bad Ischl}\toendnotes[C]{\smallbreak}
\Standort{CUL, Schnitzler, B 43.}
\physDesc{Brief, 1 Blatt, 2 Seiten, 703 Zeichen
\newline{}Handschrift: schwarze Tinte, deutsche Kurrent
\newline{}Schnitzler: mit Bleistift Monat und Jahreszahl ergänzt: »7. 97« 
\newline{}Ordnung: 1) mit Bleistift von unbekannter Hand nummeriert: »\strikeout{95}«  2) mit Bleistift von unbekannter Hand nummeriert:
                                    »93«}
\buchAbdrucke{\weitereDrucke{Hugo von Hofmannsthal, Arthur Schnitzler: \emph{Briefwechsel}. Herausgegeben von Therese Nickl und Heinrich Schnitzler. Frankfurt am Main: \emph{S. Fischer} 1964, S. 89–90.} }
\pstart
           \raggedleft{}{\pb}Bad Fuſch\oindex{Bad Fusch@\textbf{Bad Fusch}|pw}{ }9\textsuperscript{ten}\pend
           \vspace{0.5em}
\pstart
           lieber Arthur, danke für Ihren lieben Brief. Ich bin durch
               aufeinanderfolgende{ }ſehr angſtvolle und undeutliche Telegramme von Poldy\pwindex{Andrian-Werburg, Leopold von 9.\,5.\,1875 Berlin – 19.\,11.\,1951 Fribourg@\textsc{Andrian-Werburg, Leopold von} (9.\,5.\,1875 Berlin – 19.\,11.\,1951 Fribourg), \emph{Schriftsteller, Diplomat}|pw}{ }ſehr beunruhigt. Er will mich bei{ }ſich haben, was
               mir begreiflicherweiſe aus vielen Gründen{ }ſehr{ }ſchwer fällt. Bitte antworten Sie mir
                  \uline{umgehend} mit 2 Zeilen, ob Sie Ihre Fahrt nach Wien\oindex{Wien@\textbf{Wien}, \emph{Verwaltungsgebiet}|pw}, die doch unvermeidlich{ }ſcheint, nicht{ }ſchon
               in den nächſten {\pb}Tagen machen und
               ihn dabei (Vorderbrühl Liechtenſteinſtraße 10\oindex{Liechtensteinstraße [Hinterbrühl]@\textbf{Liechtensteinstraße [Hinterbrühl]}, \emph{Straße}|pw})
               beſuchen könnten, ebenſo als Arzt wie als Freund. Ich kenne mich nicht aus, werde
               alſo eventuell doch hinfahren.\pend
           
\pstart
           Unſer \textsc{rendez vous} in \textsc{Salzburg}\oindex{Salzburg@\textbf{Salzburg}, \emph{Verwaltungsgebiet}|pw} bleibt, wenn was Gott verhüte nichts ganz beſondres dazwiſchenkommt, für den
                     23\textsuperscript{ten} oder 24\textsuperscript{ten} July.\pend
           
\pstart
           Von Herzen{\\[\baselineskip]}Ihr{\\[\baselineskip]}\spacefill\mbox{Hugo.}\pend
           \leftskip=0em{}\selectlanguage{ngerman}\endnumbering\briefempfaengerindex{Schnitzler, Arthur@\textsc{Schnitzler, Arthur}!zzzHofmannsthal, Hugo von@\emph{von Hugo von Hofmannsthal}!1897-07-092@{9. [7. 1897]}|)be}\mylabel{L00697h}  \newcommand{\dateiname}{L00697}\newcommand{\titel}{Hugo von Hofmannsthal an Arthur Schnitzler, 9. [7. 1897]}\newcommand{\editorInnen}{Martin Anton Müller und Gerd-Hermann Susen}%% latex-leseansicht-abspann.tex
%% Abspann für die Leseansicht.
%% Der Schalter \ifkorrekturansicht ist bereits durch den Vorspann gesetzt.

%% latex-abspann.tex
%% Gemeinsamer Abspann für Korrekturansicht und Leseansicht.
%% Setzt den Schalter \ifkorrekturansicht voraus (gesetzt in den
%% einbindenden Dateien latex-korrekturansicht-abspann.tex bzw.
%% latex-leseansicht-abspann.tex).
%% ---------------------------------------------------------------

\normalsize

% Das esempio-Environment wird nur in der Leseansicht benötigt
\ifkorrekturansicht\else
\newenvironment{esempio}[3]%
{
    \vspace{1.5ex}
    \rlap{\underline{#1}}
    \par
    \setlength{\parindent}{0cm}
    \nopagebreak
    \leftskip=#2cm
    \rightskip=#3cm
}
{
    \par
}
\fi

\doendnotes{C}
\bigskip
\vfill

\clearpage

\footnotesize

\ifkorrekturansicht
  \lohead{\textsc{register}}
\fi

% theindex-Environment neu definieren ohne reledmac
\makeatletter
\renewenvironment{theindex}{%
  \ifkorrekturansicht
    \section*{\indexname}%
  \else
    \subsubsection*{Index der erwähnten Entitäten}%
  \fi
  \setlength{\parindent}{0pt}%
  \setlength{\parskip}{0pt plus 0.3pt}%
  \let\item\@idxitem
}{%
  \ifkorrekturansicht\clearpage\fi
}
\makeatother

\IfFileExists{\jobname-pw.ind}{\input{\jobname-pw.ind}}{}

% Quellenangabe nur in der Leseansicht
\ifkorrekturansicht\else
% Fallback-Definitionen, falls die .tex-Datei \titel etc. nicht gesetzt hat
\providecommand{\titel}{}
\providecommand{\editorInnen}{}
\providecommand{\dateiname}{\jobname}

\vspace{3cm}

\vfill

\footnotesize
\textsc{Quelle}: \titel. Herausgegeben von {\editorInnen}. In: \emph{Arthur Schnitzler: Briefwechsel mit Autorinnen und Autoren}.
 Digitale Edition, https://schnitzler-briefe.acdh.oeaw.ac.at/{\dateiname}.html (Stand \today)
\fi

\end{document}


