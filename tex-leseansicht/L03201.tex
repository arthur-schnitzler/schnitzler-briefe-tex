%% latex-korrekturansicht-vorspann.tex
%% Vorspann für die Korrekturansicht.
%% Lädt die gemeinsame Datei latex-vorspann.tex mit gesetztem Schalter.

\newif\ifkorrekturansicht
\korrekturansichttrue

\input{../tex-inputs/latex-vorspann}


\section[ Paul Goldmann an Arthur Schnitzler, 21. 3. 1902]{L03201 Paul Goldmann an Arthur Schnitzler, 21. 3. 1902}
\nopagebreak\mylabel{L03201v}
\rehead{ }\normalsize\beginnumbering\briefempfaengerindex{Schnitzler, Arthur@\textsc{Schnitzler, Arthur}!zzzGoldmann, Paul@\emph{von Paul Goldmann}!1902-03-211@{21. 3. 1902}|(be}
\toendnotes[C]{\smallbreak\pagebreak[2]}\Standort{DLA, A:Schnitzler, HS.NZ85.1.3172.}
\physDesc{Postkarte, 309 Zeichen
\newline{}Handschrift: 1) blaue Tinte, deutsche Kurrent\hspace{1em}2) blaue Tinte, lateinische Kurrent (\noindent{}Adresse)\hspace{1em}
\newline{}Versand: 1) Stempel: »\nobreak{}\oindex{Berlin@\textbf{Berlin}, \emph{P.PPLC}|pwk}Berlin S. W. 46 a, 21. 3. 02, 12–1 N.\nobreak{}«.   2) Stempel: »\nobreak{}\oindex{IX., Alsergrund@\textbf{IX., Alsergrund}, \emph{A.ADM3}|pwk}9/3 Wien 7{[}2{]}, 22. 3. {[}1902{]}, 11., Beste{[}llt{]}\nobreak{}«. }\toendnotes[C]{\smallbreak}\pstart{}{\pb}Herrn\pend{}\pstart{}Dr. Arthur Schnitzler\pend{}\pstart{}Wien\oindex{Wien@\textbf{Wien}, \emph{A.ADM2}|pw}\pend{}\pstart{}IX. Frankgaſse 1\oindex{Frankgasse 1@\textbf{Frankgasse 1}, \emph{Wohngebäude (K.WHS)}|pw}.\pend{}{\bigskip}\vspace{1em}
\pstart
           {\pb}21. 3. 1902.\pend
           
\pstart{}Mein lieber Freund,\pend\vspace{0.5em}
\pstart
           Im ſoeben erſchienenen Heft der »Zukunft\pwindex{Zukunft@\emph{Die Zukunft}|pw}« (ich
               habe es nicht zur Hand u. kann es Dir daher nicht ſchicken) ſagt \textsc{Harden\pwindex{Harden, Maximilian 20.10.1861 – 30.10.1927@\textsc{Harden, Maximilian} (20.10.1861 – 30.10.1927), \emph{Schriftsteller/Schriftstellerin, Publizist/Publizistin}|pw}} gegen Schluß ſeines \label{K_L03201-1v}\edtext{Theaterartikels\pwindex{Theater@\emph{Theater}|pwv}}{\lemma{\textnormal{\emph{Theaterartikels}}}\Cendnote{\textnormal{M. H.\pwindex{Harden, Maximilian 20.10.1861 – 30.10.1927@\textsc{Harden, Maximilian} (20.10.1861 – 30.10.1927), \emph{Schriftsteller/Schriftstellerin, Publizist/Publizistin}|pwkv} [ = Maximilian Harden\pwindex{Harden, Maximilian 20.10.1861 – 30.10.1927@\textsc{Harden, Maximilian} (20.10.1861 – 30.10.1927), \emph{Schriftsteller/Schriftstellerin, Publizist/Publizistin}|pwk}]: \emph{Theater}\pwindex{Theater@\emph{Theater}|pwk}. In: \emph{Die
                        Zukunft}\pwindex{Zukunft@\emph{Die Zukunft}|pwk}, Jg. 38, 22. 3. 1902,
                     S. 490–498, hier: S. 497: »Herr Arthur Schnitzler, den der Erfolg doch schon bekannt gemacht und
                     gesegnet hat, harrt vergebens noch immer der Stunde, die sein reifstes Werk,
                     den ›Schleier der Beatrice\pwindex{Schleier der Beatrice. Schauspiel in fuenf Akten@\emph{Der Schleier der Beatrice. Schauspiel in fünf Akten}|pw}‹, auf einer
                     großen Bühne zum Leben erweckt. Und seine ›Lebendigen Stunden\pwindex{Lebendige Stunden. Vier Einakter@\emph{Lebendige Stunden. Vier Einakter}|pw}‹, drei sehr feine und ein effektvoller Einakter,
                     von denen noch zu reden sein wird, mußten nach kurzer Frist dem
                     Coulissenschmöker des Kollegen Sudermann\pwindex{Sudermann, Hermann 30.09.1857 – 21.11.1928@\textsc{Sudermann, Hermann} (30.09.1857 – 21.11.1928), \emph{Schriftsteller/Schriftstellerin}|pw}
                     weichen.«}}}\label{K_L03201-1} einige freundliche Worte über den »Schleier der \textsc{Beatrice}\pwindex{Schleier der Beatrice. Schauspiel in fuenf Akten@\emph{Der Schleier der Beatrice. Schauspiel in fünf Akten}|pw}«.\pend
           
\pstart
           Viele Grüße! Dein {\\[\baselineskip]}\spacefill\mbox{P. G.}\pend
           \leftskip=0em{}\selectlanguage{ngerman}\endnumbering\briefempfaengerindex{Schnitzler, Arthur@\textsc{Schnitzler, Arthur}!zzzGoldmann, Paul@\emph{von Paul Goldmann}!1902-03-211@{21. 3. 1902}|)be}\mylabel{L03201h}  \normalsize

\doendnotes{C}
\bigskip
\vfill

\clearpage

\footnotesize

\lohead{\textsc{register}}

% Definiere theindex-Environment komplett neu ohne reledmac
\makeatletter
\renewenvironment{theindex}{%
  \section*{\indexname}%
  \setlength{\parindent}{0pt}%
  \setlength{\parskip}{0pt plus 0.3pt}%
  \let\item\@idxitem
}{%
  \clearpage
}
\makeatother

\IfFileExists{\jobname-pw.ind}{\input{\jobname-pw.ind}}{}

\end{document}

      