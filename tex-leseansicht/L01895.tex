%% latex-korrekturansicht-vorspann.tex
%% Vorspann für die Korrekturansicht.
%% Lädt die gemeinsame Datei latex-vorspann.tex mit gesetztem Schalter.

\newif\ifkorrekturansicht
\korrekturansichttrue

\input{../tex-inputs/latex-vorspann}


\section[Richard Beer-Hofmann an Arthur Schnitzler, 9. 12. 1909]{L01895 Richard Beer-Hofmann an Arthur Schnitzler, 9. 12. 1909}
\nopagebreak\mylabel{L01895v}
\rehead{ }\normalsize\beginnumbering\briefempfaengerindex{Schnitzler, Arthur@\textsc{Schnitzler, Arthur}!zzzBeer-Hofmann, Richard@\emph{von Richard Beer-Hofmann}!1909-12-091@{9. 12. 1909}|(be}
\toendnotes[C]{\smallbreak\pagebreak[2]}\Standort{CUL, Schnitzler, B 8.}
\physDesc{Kartenbrief, 543 Zeichen
\newline{}Handschrift: Bleistift, lateinische Kurrent
\newline{}Versand: ohne postalischen Übermittlungsvermerk 
\newline{}Schnitzler: mit Bleistift beschriftet: »\textsc{Beerh.}« 
\newline{}Ordnung: mit Bleistift von unbekannter Hand nummeriert:
                                    »226« }
\buchAbdrucke{\weitereDrucke{Arthur Schnitzler, Richard Beer-Hofmann: \emph{Briefwechsel 1891–1931}. Wien, Zürich: \emph{Europaverlag} 1992, S. 196.} }\toendnotes[C]{\smallbreak}\pstart{}{\pb}nicht dringend\pend{}\pstart{}Herrn\pend{}\pstart{}Arthur Schnitzler\pend{}\pstart{}Spöttelgasse 7\oindex{Edmund-Weiss-Gasse 7@\textbf{Edmund-Weiß-Gasse 7}, \emph{Wohngebäude (K.WHS)}|pw}\pend{}{\bigskip}\vspace{1em}
\pstart
           \raggedleft{}{\pb}9./XII. 09\pend
           \vspace{0.5em}
\pstart
           Lieber Arthur! Soeben überfällt mich folgendes Telegra{\geminationm}: »Bin morgen, Freitag 2{ }Wien\oindex{Wien@\textbf{Wien}, \emph{A.ADM2}|pw} wäre sehr dankbar wenn mich 3
                  Uhr{ }Hasenauerstr\oindex{Hasenauerstrasse 59@\textbf{Hasenauerstraße 59}, \emph{Wohngebäude (K.WHS)}|pw} erwarten und mir baldmöglichst
               consultation Arthur Schnitzler ermoeglichen wollten herzlichst poldi andrian\pwindex{Andrian-Werburg, Leopold von 09.05.1875 – 19.11.1951@\textsc{Andrian-Werburg, Leopold von} (09.05.1875 – 19.11.1951), \emph{Schriftsteller/Schriftstellerin, Diplomat/Diplomatin}|pw}«. Ich sehe Sie ja morgen Vorm (\strikeout{voraussichtlich} – hoffentlich) schreibe {\pb}Ihnen aber jetzt, – damit Sie es
               sich einteilen können. Entweder – dass ich ihn zu Ihnen hinüberschicke, oder dass Sie
               zu mir herüberko{\geminationm}en. \label{K_L01895-1v}\edtext{Grossvater Giacomo\pwindex{Meyerbeer, Giacomo 05.09.1791 – 02.05.1864@\textsc{Meyerbeer, Giacomo} (05.09.1791 – 02.05.1864), \emph{Komponist/Komponistin}|pw}}{\lemma{\textnormal{\emph{Grossvater Giacomo}}}\Cendnote{\textnormal{Leopold Andrian\pwindex{Andrian-Werburg, Leopold von 09.05.1875 – 19.11.1951@\textsc{Andrian-Werburg, Leopold von} (09.05.1875 – 19.11.1951), \emph{Schriftsteller/Schriftstellerin, Diplomat/Diplomatin}|pwk} war mütterlicherseits ein
                  Enkel des Komponisten Giacomo
                  Meyerbeer\pwindex{Meyerbeer, Giacomo 05.09.1791 – 02.05.1864@\textsc{Meyerbeer, Giacomo} (05.09.1791 – 02.05.1864), \emph{Komponist/Komponistin}|pwk}.}}}\label{K_L01895-1}’s Nerven?\pend
           
\pstart
           Herzlichst{\\[\baselineskip]}\spacefill\mbox{Richard}\pend
           \leftskip=0em{}\selectlanguage{ngerman}\endnumbering\briefempfaengerindex{Schnitzler, Arthur@\textsc{Schnitzler, Arthur}!zzzBeer-Hofmann, Richard@\emph{von Richard Beer-Hofmann}!1909-12-091@{9. 12. 1909}|)be}\mylabel{L01895h}  \normalsize

\doendnotes{C}
\bigskip
\vfill

\clearpage

\footnotesize

\lohead{\textsc{register}}

% Definiere theindex-Environment komplett neu ohne reledmac
\makeatletter
\renewenvironment{theindex}{%
  \section*{\indexname}%
  \setlength{\parindent}{0pt}%
  \setlength{\parskip}{0pt plus 0.3pt}%
  \let\item\@idxitem
}{%
  \clearpage
}
\makeatother

\IfFileExists{\jobname-pw.ind}{\input{\jobname-pw.ind}}{}

\end{document}

      