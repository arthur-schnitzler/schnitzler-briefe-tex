%% latex-leseansicht-vorspann.tex
%% Vorspann für die Leseansicht.
%% Lädt die gemeinsame Datei latex-vorspann.tex mit nicht gesetztem Schalter.

\newif\ifkorrekturansicht
\korrekturansichtfalse

\input{../tex-inputs/latex-vorspann}


         
         \renewcommand{\erwaehntePersonen}{Personen: Leopold von Andrian-Werburg, Giacomo Meyerbeer}
         \renewcommand{\erwaehnteOrte}{Orte: Edmund-Weiß-Gasse, Hasenauerstraße, Wien}
         \renewcommand{\erwaehnteWerke}{}
               \section[Richard Beer-Hofmann an Arthur Schnitzler, 9. 12. 1909]{ Richard Beer-Hofmann an Arthur Schnitzler, 9. 12. 1909}\nopagebreak\mylabel{v}\rehead{ }\begin{ledgroupsized}[t]{13cm}\normalsize\beginnumbering \toendnotes[C]{\smallbreak\pagebreak[2]} \Standort{CUL, Schnitzler, B 8.}
\physDesc{Kartenbrief
\newline{}Handschrift: Bleistift, lateinische Kurrent\newline{}Versand: ohne postalischen Übermittlungsvermerk 
\newline{}Schnitzler: mit Bleistift beschriftet: »\textsc{Beerh.}« \newline{}Ordnung: mit Bleistift von unbekannter Hand nummeriert:
                              »226« }\buchAbdrucke{\weitereDrucke{Arthur Schnitzler, Richard Beer-Hofmann: \emph{Briefwechsel 1891–1931}. Hg. Konstanze Fliedl. Wien, Zürich: \emph{Europaverlag} 1992, S. 196.} }\toendnotes[C]{\smallbreak}\pstart{}{\pb}nicht dringend\pend{}\pstart{}Herrn\pend{}\pstart{}Arthur Schnitzler\pend{}\pstart{}Spöttelgasse 7\oindex{Edmund-Weiss-Gasse@\textbf{Edmund-Weiß-Gasse}|pw}\pend{}{\bigskip}\pstart
           \raggedleft{}{\pb}9./XII. 09\pend
           \pstart
           Lieber Arthur! Soeben überfällt mich folgendes Telegra{\geminationm}: »Bin morgen, Freitag 2{ }Wien\oindex{Wien@\textbf{Wien}|pw} wäre sehr dankbar wenn mich 3 Uhr{ }Hasenauerstr\oindex{Hasenauerstrasse@\textbf{Hasenauerstraße}|pw} erwarten und mir baldmöglichst
               consultation Arthur Schnitzler ermoeglichen wollten herzlichst poldi andrian\pwindex{Andrian-Werburg, Leopold von 09.05.1875 – 19.11.1951@\textsc{Andrian-Werburg, Leopold von} (09.05.1875 – 19.11.1951), \emph{Schriftsteller, Diplomat}|pw}«. Ich sehe Sie ja morgen Vorm (\strikeout{voraussichtlich} – hoffentlich) schreibe {\pb}Ihnen aber jetzt, – damit Sie es sich
               einteilen können. Entweder – dass ich ihn zu Ihnen hinüberschicke, oder dass Sie zu mir herüberko{\geminationm}en. \label{K_L01895-1v}\edtext{Grossvater Giacomo\pwindex{Meyerbeer, Giacomo 05.09.1791 – 02.05.1864@\textsc{Meyerbeer, Giacomo} (05.09.1791 – 02.05.1864), \emph{Komponist}|pw}}{\lemma{\textnormal{\emph{Grossvater Giacomo}}}\Cendnote{\textnormal{Leopold Andrian\pwindex{Andrian-Werburg, Leopold von 09.05.1875 – 19.11.1951@\textsc{Andrian-Werburg, Leopold von} (09.05.1875 – 19.11.1951), \emph{Schriftsteller, Diplomat}|pwk} war mütterlicherseits ein Enkel des Komponisten Giacomo Meyerbeer\pwindex{Meyerbeer, Giacomo 05.09.1791 – 02.05.1864@\textsc{Meyerbeer, Giacomo} (05.09.1791 – 02.05.1864), \emph{Komponist}|pwk}.}}}\label{K_L01895-1h}’s Nerven?\pend
           \pstart
           Herzlichst{\\[\baselineskip]}\spacefill\mbox{Richard}\pend
           \leftskip=0em{}
         
         \endnumbering\mylabel{h}\end{ledgroupsized}  \newcommand{\dateiname}{L01895}\newcommand{\titel}{Richard Beer-Hofmann an Arthur Schnitzler, 9. 12. 1909}\newcommand{\editorInnen}{Martin Anton Müller und Gerd-Hermann Susen}%% latex-leseansicht-abspann.tex
%% Abspann für die Leseansicht.
%% Der Schalter \ifkorrekturansicht ist bereits durch den Vorspann gesetzt.

%% latex-abspann.tex
%% Gemeinsamer Abspann für Korrekturansicht und Leseansicht.
%% Setzt den Schalter \ifkorrekturansicht voraus (gesetzt in den
%% einbindenden Dateien latex-korrekturansicht-abspann.tex bzw.
%% latex-leseansicht-abspann.tex).
%% ---------------------------------------------------------------

\normalsize

% Das esempio-Environment wird nur in der Leseansicht benötigt
\ifkorrekturansicht\else
\newenvironment{esempio}[3]%
{
    \vspace{1.5ex}
    \rlap{\underline{#1}}
    \par
    \setlength{\parindent}{0cm}
    \nopagebreak
    \leftskip=#2cm
    \rightskip=#3cm
}
{
    \par
}
\fi

\doendnotes{C}
\bigskip
\vfill

\clearpage

\footnotesize

\ifkorrekturansicht
  \lohead{\textsc{register}}
\fi

% theindex-Environment neu definieren ohne reledmac
\makeatletter
\renewenvironment{theindex}{%
  \ifkorrekturansicht
    \section*{\indexname}%
  \else
    \subsubsection*{Index der erwähnten Entitäten}%
  \fi
  \setlength{\parindent}{0pt}%
  \setlength{\parskip}{0pt plus 0.3pt}%
  \let\item\@idxitem
}{%
  \ifkorrekturansicht\clearpage\fi
}
\makeatother

\IfFileExists{\jobname-pw.ind}{\input{\jobname-pw.ind}}{}

% Quellenangabe nur in der Leseansicht
\ifkorrekturansicht\else
% Fallback-Definitionen, falls die .tex-Datei \titel etc. nicht gesetzt hat
\providecommand{\titel}{}
\providecommand{\editorInnen}{}
\providecommand{\dateiname}{\jobname}

\vspace{3cm}

\vfill

\footnotesize
\textsc{Quelle}: \titel. Herausgegeben von {\editorInnen}. In: \emph{Arthur Schnitzler: Briefwechsel mit Autorinnen und Autoren}.
 Digitale Edition, https://schnitzler-briefe.acdh.oeaw.ac.at/{\dateiname}.html (Stand \today)
\fi

\end{document}


      