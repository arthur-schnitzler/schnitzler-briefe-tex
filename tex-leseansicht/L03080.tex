%% latex-leseansicht-vorspann.tex
%% Vorspann für die Leseansicht.
%% Lädt die gemeinsame Datei latex-vorspann.tex mit nicht gesetztem Schalter.

\newif\ifkorrekturansicht
\korrekturansichtfalse

\input{../tex-inputs/latex-vorspann}


         
         \renewcommand{\erwaehntePersonen}{Personen: Louis Cima, Paul Goldmann, Olga Schnitzler, Anton Pavlovič Čechov}
         \renewcommand{\erwaehnteOrte}{Orte: Bad Ischl, Berlin, Engadin, Hôtel Metropole, Innsbruck, Russland, Sankt Moritz, Sankt Moritz-Bad, Schweiz, Wien}
         \renewcommand{\erwaehnteWerke}{Werke: Ein Zweikampf}
               \section[ Paul Goldmann an Arthur Schnitzler, 21. 8. {[}1905?{]}]{ Paul Goldmann an Arthur Schnitzler, 21. 8. {[}1905?{]}}\nopagebreak\mylabel{v}\rehead{ }\begin{ledgroupsized}[t]{13cm}\normalsize\beginnumbering\briefempfaengerindex{Schnitzler, Arthur@\textsc{Schnitzler, Arthur}!zzzGoldmann, Paul@\emph{von Paul Goldmann}!1905-08-212@{21. 8. {[}1905?{]}}|(be} \toendnotes[C]{\smallbreak\pagebreak[2]} \Standort{DLA, A:Schnitzler, HS.NZ85.1.3171.}
\physDesc{Brief, 1 Blatt, 4 Seiten, 1679 Zeichen
\newline{}Handschrift: schwarze Tinte, deutsche Kurrent
\newline{}Schnitzler: 1) mit Bleistift das Jahr »901« vermerkt  2) mit rotem Buntstift eine Unterstreichung}\toendnotes[C]{\smallbreak}\pstart
           \noindent{}{\pb}\textcolor{gray}{\textbf{\textbf{\begin{otherlanguage}{french}HÔTEL MÉTROPOLE\oindex{Hôtel Metropole@\textbf{Hôtel Metropole}|pw}{ }ST. MORITZ\oindex{Sankt Moritz-Bad@\textbf{Sankt Moritz-Bad}|pw}\end{otherlanguage}}}}\pend
           \pstart
           \textcolor{gray}{\textbf{\emph{\begin{otherlanguage}{french}Hôtel de I\textsuperscript{er}
                           Ordre\end{otherlanguage}}}}\pend
           \pstart
           \textcolor{gray}{\textbf{\begin{otherlanguage}{french}ENGADINE\oindex{Engadin@\textbf{Engadin}|pw} · SUISSE\oindex{Schweiz@\textbf{Schweiz}|pw}\end{otherlanguage}}}\pend
           \pstart
           \textcolor{gray}{\textbf{\begin{otherlanguage}{french}NOUVELLEMENT CONSTRUIT AVEC TOUS LES CONFORTS
                           MODERNES\end{otherlanguage}}}\pend
           \pstart
           \textcolor{gray}{\textbf{\begin{otherlanguage}{french}120 CHAMBRES\end{otherlanguage}}}\pend
           \pstart
           \textcolor{gray}{\textbf{\begin{otherlanguage}{french}SITUATION SPLENDIDE\end{otherlanguage}}}\pend
           \pstart
           \textcolor{gray}{\textbf{\begin{otherlanguage}{french}ASCENSEUR ET LUMIÈRE ELECTRIQUE\end{otherlanguage}}}\pend
           \pstart
           \textcolor{gray}{\textbf{\begin{otherlanguage}{french}RESTAURANT A LA CARTE ET ARRANGEMENTS POUR
                           FAMILLES\end{otherlanguage}}}\pend
           \pstart
           \textcolor{gray}{\textbf{\emph{LOUIS CIMA\pwindex{Cima, Louis @\textsc{Cima, Louis}, \emph{Hotelbesitzer}|pw},{ }\begin{otherlanguage}{french}PROPR. \end{otherlanguage}}}}\pend
           \pstart
           \raggedleft{}\textcolor{gray}{\textbf{St. Moritz-Bad\oindex{Sankt Moritz-Bad@\textbf{Sankt Moritz-Bad}|pw}, \begin{otherlanguage}{french}le\end{otherlanguage}}}{ }\label{K_L03080-1v}\edtext{21. Auguſt}{\lemma{\textnormal{\emph{21. Auguſt}}}\Cendnote{\textnormal{Schnitzlers\pwindex{Schnitzler, Arthur 15.05.1862 – 21.10.1931@\textsc{Schnitzler, Arthur} (15.05.1862 – 21.10.1931), \emph{Schriftsteller, Mediziner}|pwk} Datierung des Briefs auf den
                        21. 8. 1901 ist falsch. Er und Goldmann\pwindex{Goldmann, Paul 31.01.1865 – 25.09.1935@\textsc{Goldmann, Paul} (31.01.1865 – 25.09.1935), \emph{Schriftsteller, Journalist}|pwk} waren zu dieser Zeit im Jahr
                        1901 gemeinsam auf Reisen (vgl. Paul Goldmann und Arthur Schnitzler an Georg Brandes,
               21. 8. 1901). 1905 war Goldmann\pwindex{Goldmann, Paul 31.01.1865 – 25.09.1935@\textsc{Goldmann, Paul} (31.01.1865 – 25.09.1935), \emph{Schriftsteller, Journalist}|pwk}
                     nachweislich in Sankt Moritz\oindex{Sankt Moritz@\textbf{Sankt Moritz}|pwk} (vgl. Paul Goldmann an Arthur Schnitzler, 26. 8. 1905). Davor, am 31. 7. 1905, hatte
                     er Schnitzler\pwindex{Schnitzler, Arthur 15.05.1862 – 21.10.1931@\textsc{Schnitzler, Arthur} (15.05.1862 – 21.10.1931), \emph{Schriftsteller, Mediziner}|pwk} und dessen Frau\pwindex{Schnitzler, Olga 17.01.1882 – 13.01.1970@\textsc{Schnitzler, Olga} (17.01.1882 – 13.01.1970), \emph{Schauspielerin, Sängerin}|pwkv} in Wien\oindex{Wien@\textbf{Wien}|pwk} einen Besuch abgestattet.}}}\label{K_L03080-1h}.\pend
           \pstart\center{}Mein lieber Freund,\pend\pstart
           Ich komme erſt heut dazu, Dir und Deiner Frau\pwindex{Schnitzler, Olga 17.01.1882 – 13.01.1970@\textsc{Schnitzler, Olga} (17.01.1882 – 13.01.1970), \emph{Schauspielerin, Sängerin}|pwv} für die Freundſchaft zu
               danken, mit der Ihr in Wien\oindex{Wien@\textbf{Wien}|pw} mich aufgenommen
               habt.\pend
           \pstart
           Die erſte Hälfte meines Urlaubs habe ich leider ſehr unzweckmäßig verbracht. Der
               Aufenthalt in \textsc{Ischl\oindex{Bad Ischl@\textbf{Bad Ischl}|pw}} hat mir gar keine Erholung gewährt, und ich bedaure {\pb}es ſehr, daß ich nicht die Energie gefunden habe, mich
               früher von dort loszureißen, obwohl doch eigentlich nichts mich hielt. Seit vorigem
                  Donnerſtag bin ich hier, und jetzt erſt beginne
               ich, mich zu kräftigen und zu erfriſchen. Du kennſt ja den Ort\oindex{Sankt Moritz-Bad@\textbf{Sankt Moritz-Bad}|pwv} von unſerem \label{K_L03080-2v}\edtext{gemeinſamen Aufenthalt}{\lemma{\textnormal{\emph{gemeinſamen Aufenthalt}}}\Cendnote{\textnormal{A. S.: \emph{Tagebuch}, 21. 8. 1900 und 6. 8. 1930}}}\label{K_L03080-2h} her, an den \strikeout{\textcolor{gray}{ich}} mich \strikeout{h\textcolor{gray}{×}\-\textcolor{gray}{×}} hier Manches erinnert, aber in ſeiner ganzen Herrlichkeit entfaltet ſich das
                  Engadin\oindex{Engadin@\textbf{Engadin}|pw} doch erſt bei längerem Aufenthalt.
               Mein Entſchluß iſt gefaßt: Ich werde fortan \uline{jeden}
               Urlaub im Engadin\oindex{Engadin@\textbf{Engadin}|pw} verbringen. Nirgends wieder
               gibt es eine {\pb}ſolche Luft, das Athmen allein iſt ein
               Vergnügen, und für abgearbeitete Menſchen iſt hier und hier allein die rechte
               Erholung. Obwohl Du ja nicht abgearbeitet biſt, rate ich Dir auch dringend, nächſten
               Sommer hier einen längeren \label{K_L03080-3v}\edtext{Aufenthalt}{\lemma{\textnormal{\emph{Aufenthalt}}}\Cendnote{\textnormal{Schnitzler\pwindex{Schnitzler, Arthur 15.05.1862 – 21.10.1931@\textsc{Schnitzler, Arthur} (15.05.1862 – 21.10.1931), \emph{Schriftsteller, Mediziner}|pwk} kam erst am 26. 8. 1913 wieder
                  nach Sankt Moritz\oindex{Sankt Moritz@\textbf{Sankt Moritz}|pwk}.}}}\label{K_L03080-3h} zu nehmen. Da die
               Bahn jetzt bis \textsc{St. Moritz\oindex{Sankt Moritz@\textbf{Sankt Moritz}|pw}} fährt, kommt man bequem hin (von Innsbruck\oindex{Innsbruck@\textbf{Innsbruck}|pw}
               in 10 Stunden).\pend
           \pstart
           Das \label{K_L03080-4v}\edtext{Buch\pwindex{Cechov, Anton Pavlovic 1860-01-17 – 1904-07-15@\textsc{Čechov, Anton Pavlovič} (1860-01-17 – 1904-07-15), \emph{Schriftsteller}!Zweikampf1897@\strich\emph{Ein Zweikampf} {[}1897{]}|pwuv}}{\lemma{\textnormal{\emph{Buch}}}\Cendnote{\textnormal{Es dürfte sich um die Novelle \emph{Ein Zweikampf}\pwindex{Cechov, Anton Pavlovic 1860-01-17 – 1904-07-15@\textsc{Čechov, Anton Pavlovič} (1860-01-17 – 1904-07-15), \emph{Schriftsteller}!Zweikampf1897@\strich\emph{Ein Zweikampf} {[}1897{]}|pwk} (zumeist übersetzt als \emph{Das Duell}\pwindex{Cechov, Anton Pavlovic 1860-01-17 – 1904-07-15@\textsc{Čechov, Anton Pavlovič} (1860-01-17 – 1904-07-15), \emph{Schriftsteller}!Zweikampf1897@\strich\emph{Ein Zweikampf} {[}1897{]}|pwk}) handeln, deren Lektüre durch Schnitzler\pwindex{Schnitzler, Arthur 15.05.1862 – 21.10.1931@\textsc{Schnitzler, Arthur} (15.05.1862 – 21.10.1931), \emph{Schriftsteller, Mediziner}|pwk} für den 7. 10. 1904 belegt
                  ist. Vgl. A. S.: \emph{»Das Zeitlose ist von kürzester Dauer«}, Tschechow, 18. 1. 1910. }}}\label{K_L03080-4h} von
                  \textsc{Tschechow\pwindex{Cechov, Anton Pavlovic 1860-01-17 – 1904-07-15@\textsc{Čechov, Anton Pavlovič} (1860-01-17 – 1904-07-15), \emph{Schriftsteller}|pw}} hat mich nicht begeiſtert. Es enthält manches Feine, im Übrigen habe ich es vor
               allen Dingen quälend gefunden, und Quälen iſt nicht Dichten. {\pb}Meine Anſicht, daß \textsc{Tschechow\pwindex{Cechov, Anton Pavlovic 1860-01-17 – 1904-07-15@\textsc{Čechov, Anton Pavlovič} (1860-01-17 – 1904-07-15), \emph{Schriftsteller}|pw}} ein feines Talent iſt, aber zu den bedeutenden und eigenartigen
               Perſönlichkeiten der ruſſiſchen\oindex{Russland@\textbf{Russland}|pwv} Literatur \uline{nicht} gehört, hat durch
               dieſes Buch\pwindex{Cechov, Anton Pavlovic 1860-01-17 – 1904-07-15@\textsc{Čechov, Anton Pavlovič} (1860-01-17 – 1904-07-15), \emph{Schriftsteller}!Zweikampf1897@\strich\emph{Ein Zweikampf} {[}1897{]}|pwuv} eine
               Beſtärkung erfahren.\pend
           \pstart
           Auf der Rückreiſe komme ich nicht über Wien\oindex{Wien@\textbf{Wien}|pw}, ich
               hoffe aber, Dich im \label{K_L03080-5v}\edtext{Winter in Berlin\oindex{Berlin@\textbf{Berlin}|pw}}{\lemma{\textnormal{\emph{Winter in Berlin}}}\Cendnote{\textnormal{Schnitzler\pwindex{Schnitzler, Arthur 15.05.1862 – 21.10.1931@\textsc{Schnitzler, Arthur} (15.05.1862 – 21.10.1931), \emph{Schriftsteller, Mediziner}|pwk} und Goldmann\pwindex{Goldmann, Paul 31.01.1865 – 25.09.1935@\textsc{Goldmann, Paul} (31.01.1865 – 25.09.1935), \emph{Schriftsteller, Journalist}|pwk} trafen sich jedenfalls am 21. 11. 1905 und am
                     23. 11. 1905 in
                     Berlin\oindex{Berlin@\textbf{Berlin}|pwk}.}}}\label{K_L03080-5h} wiederzuſehen.\pend
           \pstart
           Mit vielen herzlichen Grüßen an Deine Frau\pwindex{Schnitzler, Olga 17.01.1882 – 13.01.1970@\textsc{Schnitzler, Olga} (17.01.1882 – 13.01.1970), \emph{Schauspielerin, Sängerin}|pwv} und Dich bin ich {\\[\baselineskip]}Dein getreuer {\\[\baselineskip]}\spacefill\mbox{Paul Goldmann.}\pend
           \leftskip=0em{}
         
         \endnumbering\mylabel{h}\end{ledgroupsized}  \newcommand{\dateiname}{L03080}\newcommand{\titel}{Paul Goldmann an Arthur Schnitzler, 21. 8. [1905?]}\newcommand{\editorInnen}{Martin Anton Müller und Laura Untner}%% latex-leseansicht-abspann.tex
%% Abspann für die Leseansicht.
%% Der Schalter \ifkorrekturansicht ist bereits durch den Vorspann gesetzt.

%% latex-abspann.tex
%% Gemeinsamer Abspann für Korrekturansicht und Leseansicht.
%% Setzt den Schalter \ifkorrekturansicht voraus (gesetzt in den
%% einbindenden Dateien latex-korrekturansicht-abspann.tex bzw.
%% latex-leseansicht-abspann.tex).
%% ---------------------------------------------------------------

\normalsize

% Das esempio-Environment wird nur in der Leseansicht benötigt
\ifkorrekturansicht\else
\newenvironment{esempio}[3]%
{
    \vspace{1.5ex}
    \rlap{\underline{#1}}
    \par
    \setlength{\parindent}{0cm}
    \nopagebreak
    \leftskip=#2cm
    \rightskip=#3cm
}
{
    \par
}
\fi

\doendnotes{C}
\bigskip
\vfill

\clearpage

\footnotesize

\ifkorrekturansicht
  \lohead{\textsc{register}}
\fi

% theindex-Environment neu definieren ohne reledmac
\makeatletter
\renewenvironment{theindex}{%
  \ifkorrekturansicht
    \section*{\indexname}%
  \else
    \subsubsection*{Index der erwähnten Entitäten}%
  \fi
  \setlength{\parindent}{0pt}%
  \setlength{\parskip}{0pt plus 0.3pt}%
  \let\item\@idxitem
}{%
  \ifkorrekturansicht\clearpage\fi
}
\makeatother

\IfFileExists{\jobname-pw.ind}{\input{\jobname-pw.ind}}{}

% Quellenangabe nur in der Leseansicht
\ifkorrekturansicht\else
% Fallback-Definitionen, falls die .tex-Datei \titel etc. nicht gesetzt hat
\providecommand{\titel}{}
\providecommand{\editorInnen}{}
\providecommand{\dateiname}{\jobname}

\vspace{3cm}

\vfill

\footnotesize
\textsc{Quelle}: \titel. Herausgegeben von {\editorInnen}. In: \emph{Arthur Schnitzler: Briefwechsel mit Autorinnen und Autoren}.
 Digitale Edition, https://schnitzler-briefe.acdh.oeaw.ac.at/{\dateiname}.html (Stand \today)
\fi

\end{document}


      