%% latex-leseansicht-vorspann.tex
%% Vorspann für die Leseansicht.
%% Lädt die gemeinsame Datei latex-vorspann.tex mit nicht gesetztem Schalter.

\newif\ifkorrekturansicht
\korrekturansichtfalse

\input{../tex-inputs/latex-vorspann}


\section[Arthur Schnitzler an Richard Beer-Hofmann, 2. 6. 1903]{L01295 Arthur Schnitzler an Richard Beer-Hofmann, 2. 6. 1903}
\nopagebreak\mylabel{L01295v}
\rehead{ }\normalsize\beginnumbering\briefempfaengerindex{Beer-Hofmann, Richard@\textsc{Beer-Hofmann, Richard}!zzzSchnitzler, Arthur@\emph{von Arthur Schnitzler}!1903-06-021@{2. 6. 1903}|(be}
\toendnotes[C]{\smallbreak\pagebreak[2]}
\correspDesc{Versand  durch Arthur Schnitzler am 2. 6. 1903 in Padua
\newline{}Erhalt  durch Richard Beer-Hofmann am 4. 6. 1903 in Rodaun}\toendnotes[C]{\smallbreak}
\Standort{YCGL, MSS 31.}
\physDesc{Bildpostkarte, 88 Zeichen
\newline{}Handschrift: Bleistift, deutsche Kurrent
\newline{}Versand: Stempel: »\nobreak{}\oindex{Wien@\textbf{Wien}!XXIII., Liesing@\textbf{XXIII., Liesing}!Rodaun@\textbf{Rodaun}, \emph{Region}|pwk}R{[}odau{]}n, 4. {[}6. 03{]}\nobreak{}«.  
\newline{}Ordnung: mit Bleistift von unbekannter Hand datiert: »2. 6.« }\pstart{}\textsc{{\pb}Dr Richard Beer-Hofmann}\pend{}\pstart{}\textsc{Rodaun bei Wien\oindex{Wien@\textbf{Wien}!XXIII., Liesing@\textbf{XXIII., Liesing}!Rodaun@\textbf{Rodaun}, \emph{Region}|pw}}\pend{}\pstart{}\textsc{Liesingerstraße 2\oindex{Liesingerstraße@\textbf{Liesingerstraße}, \emph{Straße}|pw}}\pend{}\pstart{}\textsc{Austria\oindex{Österreich@\textbf{Österreich}|pw}}\pend{}{\bigskip}
\pstart
           \noindent{}{\pb}\textcolor{gray}{\textbf{PADOVA – Chiesa Eremitani – Capella Mantegna\oindex{Cappella Ovetari@\textbf{Cappella Ovetari}, \emph{Kirche}|pw}}}\pend
           
\pstart
           \centering{}\textcolor{gray}{\textbf{Condanna di S. Giacomo\pwindex{Mantegna, Andrea 1431 – 13.\,9.\,1506 Mantova@\textsc{Mantegna, Andrea} (1431 – 13.\,9.\,1506 Mantova), \emph{Maler, Künstler, Kupferstecher}!Verurteilung des Hl. Jakobus@\strich\emph{Verurteilung des Hl. Jakobus}|pw} (Mantegna\pwindex{Mantegna, Andrea 1431 – 13.\,9.\,1506 Mantova@\textsc{Mantegna, Andrea} (1431 – 13.\,9.\,1506 Mantova), \emph{Maler, Künstler, Kupferstecher}|pw})}}\pend
           \vspace{1em}
\pstart
           \raggedleft{}{\pb}2/6 903.\pend
           \vspace{0.5em}
\pstart
           Herzliche Grüße!\pend
           \pstart \spacefill\mbox{A.}\pend{}\selectlanguage{ngerman}\endnumbering\briefempfaengerindex{Beer-Hofmann, Richard@\textsc{Beer-Hofmann, Richard}!zzzSchnitzler, Arthur@\emph{von Arthur Schnitzler}!1903-06-021@{2. 6. 1903}|)be}\mylabel{L01295h}  \newcommand{\dateiname}{L01295}\newcommand{\titel}{Arthur Schnitzler an Richard Beer-Hofmann, 2. 6. 1903}\newcommand{\editorInnen}{Martin Anton Müller und Gerd-Hermann Susen}%% latex-leseansicht-abspann.tex
%% Abspann für die Leseansicht.
%% Der Schalter \ifkorrekturansicht ist bereits durch den Vorspann gesetzt.

%% latex-abspann.tex
%% Gemeinsamer Abspann für Korrekturansicht und Leseansicht.
%% Setzt den Schalter \ifkorrekturansicht voraus (gesetzt in den
%% einbindenden Dateien latex-korrekturansicht-abspann.tex bzw.
%% latex-leseansicht-abspann.tex).
%% ---------------------------------------------------------------

\normalsize

% Das esempio-Environment wird nur in der Leseansicht benötigt
\ifkorrekturansicht\else
\newenvironment{esempio}[3]%
{
    \vspace{1.5ex}
    \rlap{\underline{#1}}
    \par
    \setlength{\parindent}{0cm}
    \nopagebreak
    \leftskip=#2cm
    \rightskip=#3cm
}
{
    \par
}
\fi

\doendnotes{C}
\bigskip
\vfill

\clearpage

\footnotesize

\ifkorrekturansicht
  \lohead{\textsc{register}}
\fi

% theindex-Environment neu definieren ohne reledmac
\makeatletter
\renewenvironment{theindex}{%
  \ifkorrekturansicht
    \section*{\indexname}%
  \else
    \subsubsection*{Index der erwähnten Entitäten}%
  \fi
  \setlength{\parindent}{0pt}%
  \setlength{\parskip}{0pt plus 0.3pt}%
  \let\item\@idxitem
}{%
  \ifkorrekturansicht\clearpage\fi
}
\makeatother

\IfFileExists{\jobname-pw.ind}{\input{\jobname-pw.ind}}{}

% Quellenangabe nur in der Leseansicht
\ifkorrekturansicht\else
% Fallback-Definitionen, falls die .tex-Datei \titel etc. nicht gesetzt hat
\providecommand{\titel}{}
\providecommand{\editorInnen}{}
\providecommand{\dateiname}{\jobname}

\vspace{3cm}

\vfill

\footnotesize
\textsc{Quelle}: \titel. Herausgegeben von {\editorInnen}. In: \emph{Arthur Schnitzler: Briefwechsel mit Autorinnen und Autoren}.
 Digitale Edition, https://schnitzler-briefe.acdh.oeaw.ac.at/{\dateiname}.html (Stand \today)
\fi

\end{document}


