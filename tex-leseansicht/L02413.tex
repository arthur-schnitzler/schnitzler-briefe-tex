%% latex-korrekturansicht-vorspann.tex
%% Vorspann für die Korrekturansicht.
%% Lädt die gemeinsame Datei latex-vorspann.tex mit gesetztem Schalter.

\newif\ifkorrekturansicht
\korrekturansichttrue

\input{../tex-inputs/latex-vorspann}


\section[Arthur Schnitzler an Georg Brandes, 23. 6. 1924]{L02413 Arthur Schnitzler an Georg Brandes, 23. 6. 1924}
\nopagebreak\mylabel{L02413v}
\rehead{ }\normalsize\beginnumbering\briefempfaengerindex{Brandes, Georg@\textsc{Brandes, Georg}!zzzSchnitzler, Arthur@\emph{von Arthur Schnitzler}!1924-06-231@{23. 6. 1924}|(be}
\toendnotes[C]{\smallbreak\pagebreak[2]}\Standort{Kopenhagen, Det Kongelige Bibliotek, Georg Brandes Arkiv, box 125.}
\physDesc{Postkarte, 812 Zeichen
\newline{}Handschrift: Bleistift, lateinische Kurrent
\newline{}Versand: 1) Stempel: »\nobreak{}\oindex{XVIII., Waehring@\textbf{XVIII., Währing}, \emph{A.ADM3}|pwk}18 W\textcolor{gray}{i}en
                                          \textcolor{gray}{110}, 23. VI. 24, 17\nobreak{}«.   2) mit blauer Tinte von unbekannter Hand die Ortsangabe in der
                                 Adresse geändert zu: »\noindent{}Villa Iris\oindex{Villa Iris@\textbf{Villa Iris}, \emph{Wohngebäude (K.WHS)}|pw}{ / }\uline{Hornbæk}\oindex{Hornbæk@\textbf{Hornbæk}, \emph{P.PPL}|pw}«
\newline{}Ordnung: mit Bleistift von unbekannter Hand nummeriert:
                                    »Schnitzler 48.« }
\buchAbdrucke{\weitereDrucke{Georg Brandes, Arthur Schnitzler: \emph{Ein Briefwechsel}. Bern: \emph{Francke} 1956, S. 139–140.} }\toendnotes[C]{\smallbreak}\pstart{}{\pb}\label{T_L02413-1v}\edtext{\textcolor{gray}{\textbf{A. S.}}}{\lemma{\textnormal{\emph{A. S.}}}\Cendnote{\textnormal{ovaler Absenderkleber}}}\label{T_L02413-1}\pend{}\pstart{}\textcolor{gray}{\textbf{WIEN, XVIII.}}\oindex{XVIII., Waehring@\textbf{XVIII., Währing}, \emph{A.ADM3}|pw}\pend{}\pstart{}\textcolor{gray}{\textbf{STERNWARTESTR. 71}}\oindex{Sternwartestrasse 71@\textbf{Sternwartestraße 71}, \emph{Wohngebäude (K.WHS)}|pw}\pend{}{\bigskip}\pstart{}Hr\pend{}\pstart{}Georg Brandes\pend{}\pstart{}Kopenhagen\oindex{Kopenhagen@\textbf{Kopenhagen}, \emph{P.PPLC}|pw}\pend{}{\bigskip}\vspace{1em}
\pstart
           \raggedleft{}{\pb}Wien\oindex{Wien@\textbf{Wien}, \emph{A.ADM2}|pw}, 23. 6. 24\pend
           \vspace{0.5em}
\pstart
           Mein lieber und verehrter Freund, vor kurzem erst hab ich Ihren
               wunderbaren Voltaire\pwindex{Voltaire und sein Jahrhundert@\emph{Voltaire und sein Jahrhundert}|pw} mit wahrem Entzücken
               gelesen und wieder erfreuen Sie mich durch die gütige Übersendg der zwei Bände Ihrer
                  Hauptströmungen\pwindex{Hauptstroemungen der Literatur des neunzehnten Jahrhunderts@\emph{Hauptströmungen der Literatur des neunzehnten Jahrhunderts}|pw}, – die, eine theure
               Jugenderinnerung, mich nun in ihrer \label{K_L02413-1v}\edtext{neuen Form}{\lemma{\textnormal{\emph{neuen Form}}}\Cendnote{\textnormal{Georg Brandes\pwindex{Altenberg, Peter 09.03.1859 – 08.01.1919@\textsc{Altenberg, Peter} (09.03.1859 – 08.01.1919), \emph{Schriftsteller/Schriftstellerin}|pwk}: \emph{Hauptströmungen der Literatur des 19. Jahrhunderts}\pwindex{Hauptstroemungen der Literatur des neunzehnten Jahrhunderts@\emph{Hauptströmungen der Literatur des neunzehnten Jahrhunderts}|pwk}. Vom
                     Verfasser neu bearbeitete endgültige Ausgabe. Berlin: \emph{Erich Reiss}\orgindex{Erich-Reiss-Verlag@Erich-Reiss-Verlag|pwk}{ }1924.}}}\label{K_L02413-1} in den Sommer begleiten sollen, wie der Michel Angelo\pwindex{Michelangelo Buonarotti@\emph{Michelangelo Buonarotti}|pw}. Wie werd ich Ihnen immer von neuem, – und wie gern
               immer wieder Dank schuldig. – Ich bin in den letzten Monaten nicht ganz unthätig
               gewesen, und hoffe mich für Ihre kostbaren {\pb}Gaben,
               in recht bescheidener Weise\pwindex{Fraeulein Else@\emph{Fräulein Else}|pwv}\pwindex{Komoedie der Verfuehrung. In drei Akten@\emph{Komödie der Verführung. In drei Akten}|pwv}, bald revanchiren zu dürfen. Ich hoffe liebster un\textcolor{gray}{d}
               verehrtester Georg Brandes, Sie befin\textcolor{gray}{d}en sich wohl. Lassen Sie
               mich auch darüber ein Wort vernehmen; ich schreibe demnächst mehr. In Freundschaft u.
               Bewunderung stets der Ihre \spacefill\mbox{Arthur Schnitzler}\pend
           \selectlanguage{ngerman}\endnumbering\briefempfaengerindex{Brandes, Georg@\textsc{Brandes, Georg}!zzzSchnitzler, Arthur@\emph{von Arthur Schnitzler}!1924-06-231@{23. 6. 1924}|)be}\mylabel{L02413h}  \normalsize

\doendnotes{C}
\bigskip
\vfill

\clearpage

\footnotesize

\lohead{\textsc{register}}

% Definiere theindex-Environment komplett neu ohne reledmac
\makeatletter
\renewenvironment{theindex}{%
  \section*{\indexname}%
  \setlength{\parindent}{0pt}%
  \setlength{\parskip}{0pt plus 0.3pt}%
  \let\item\@idxitem
}{%
  \clearpage
}
\makeatother

\IfFileExists{\jobname-pw.ind}{\input{\jobname-pw.ind}}{}

\end{document}

      