%% latex-leseansicht-vorspann.tex
%% Vorspann für die Leseansicht.
%% Lädt die gemeinsame Datei latex-vorspann.tex mit nicht gesetztem Schalter.

\newif\ifkorrekturansicht
\korrekturansichtfalse

\input{../tex-inputs/latex-vorspann}


\section[Arthur Schnitzler an Georg Brandes, 23. 6. 1924]{L02413 Arthur Schnitzler an Georg Brandes, 23. 6. 1924}
\nopagebreak\mylabel{L02413v}
\rehead{ }\normalsize\beginnumbering\briefempfaengerindex{Brandes, Georg@\textsc{Brandes, Georg}!zzzSchnitzler, Arthur@\emph{von Arthur Schnitzler}!1924-06-231@{23. 6. 1924}|(be}
\toendnotes[C]{\smallbreak\pagebreak[2]}
\correspDesc{Versand  durch Arthur Schnitzler am 23. 6. 1924 in Wien
\newline{}Erhalt  durch Georg Brandes im Zeitraum [24. 6. 1924
                  – 28. 6. 1924?] in Kopenhagen}\toendnotes[C]{\smallbreak}
\Standort{Kopenhagen, Det Kongelige Bibliotek, Georg Brandes Arkiv, box 125.}
\physDesc{Postkarte, 812 Zeichen
\newline{}Handschrift: Bleistift, lateinische Kurrent
\newline{}Versand: 1) Stempel: »\nobreak{}\oindex{XVIII., Währing@\textbf{XVIII., Währing}, \emph{Verwaltungsgebiet}|pwk}18 W\textcolor{gray}{i}en
                                          \textcolor{gray}{110}, 23. VI. 24, 17\nobreak{}«.   2) mit blauer Tinte von unbekannter Hand die Ortsangabe in der
                                 Adresse geändert zu: »\noindent{}Villa Iris\oindex{Villa Iris@\textbf{Villa Iris}, \emph{Wohngebäude}|pw}{ / }\uline{Hornbæk}\oindex{Hornbæk@\textbf{Hornbæk}|pw}«
\newline{}Ordnung: mit Bleistift von unbekannter Hand nummeriert:
                                    »Schnitzler 48.« }
\buchAbdrucke{\weitereDrucke{Georg Brandes, Arthur Schnitzler: \emph{Ein Briefwechsel}. Herausgegeben von Kurt Bergel. Bern: \emph{Francke} 1956, S. 139–140.} }\toendnotes[C]{\smallbreak}\pstart{}{\pb}\label{T_L02413-1v}\edtext{\textcolor{gray}{\textbf{A. S.}}}{\lemma{\textnormal{\emph{A. S.}}}\Cendnote{\textnormal{ovaler Absenderkleber}}}\label{T_L02413-1}\pend{}\pstart{}\textcolor{gray}{\textbf{WIEN, XVIII.}}\oindex{XVIII., Währing@\textbf{XVIII., Währing}, \emph{Verwaltungsgebiet}|pw}\pend{}\pstart{}\textcolor{gray}{\textbf{STERNWARTESTR. 71}}\oindex{Wien@\textbf{Wien}!XVIII., Währing@\textbf{XVIII., Währing}!Sternwartestraße 71@\textbf{Sternwartestraße 71}, \emph{Wohngebäude}|pw}\pend{}{\bigskip}\pstart{}Hr\pend{}\pstart{}Georg Brandes\pend{}\pstart{}Kopenhagen\oindex{Kopenhagen@\textbf{Kopenhagen}, \emph{Hauptstadt}|pw}\pend{}{\bigskip}\vspace{1em}
\pstart
           \raggedleft{}{\pb}Wien\oindex{Wien@\textbf{Wien}, \emph{Verwaltungsgebiet}|pw}, 23. 6. 24\pend
           \vspace{0.5em}
\pstart
           Mein lieber und verehrter Freund, vor kurzem erst hab ich Ihren
               wunderbaren Voltaire\pwindex{Brandes, Georg 4.\,2.\,1842 Kopenhagen – 19.\,2.\,1927 ebd.@\textsc{Brandes, Georg} (4.\,2.\,1842 Kopenhagen – 19.\,2.\,1927 ebd.)!Voltaire und sein Jahrhundert@\strich\emph{Voltaire und sein Jahrhundert}|pw} mit wahrem Entzücken
               gelesen und wieder erfreuen Sie mich durch die gütige Übersendg der zwei Bände Ihrer
                  Hauptströmungen\pwindex{Brandes, Georg 4.\,2.\,1842 Kopenhagen – 19.\,2.\,1927 ebd.@\textsc{Brandes, Georg} (4.\,2.\,1842 Kopenhagen – 19.\,2.\,1927 ebd.)!Hauptströmungen der Literatur des neunzehnten Jahrhunderts@\strich\emph{Hauptströmungen der Literatur des neunzehnten Jahrhunderts}|pw}, – die, eine theure
               Jugenderinnerung, mich nun in ihrer \label{K_L02413-1v}\edtext{neuen Form}{\lemma{\textnormal{\emph{neuen Form}}}\Cendnote{\textnormal{Georg Brandes\pwindex{Altenberg, Peter 9.\,3.\,1859 Wien – 8.\,1.\,1919 ebd.@\textsc{Altenberg, Peter} (9.\,3.\,1859 Wien – 8.\,1.\,1919 ebd.), \emph{Schriftsteller}|pwk}: \emph{Hauptströmungen der Literatur des 19. Jahrhunderts}\pwindex{Brandes, Georg 4.\,2.\,1842 Kopenhagen – 19.\,2.\,1927 ebd.@\textsc{Brandes, Georg} (4.\,2.\,1842 Kopenhagen – 19.\,2.\,1927 ebd.)!Hauptströmungen der Literatur des neunzehnten Jahrhunderts@\strich\emph{Hauptströmungen der Literatur des neunzehnten Jahrhunderts}|pwk}. Vom
                     Verfasser neu bearbeitete endgültige Ausgabe. Berlin: \emph{Erich Reiss}\orgindex{Erich-Reiss-Verlag@Erich-Reiss-Verlag|pwk}{ }1924.}}}\label{K_L02413-1} in den Sommer begleiten sollen, wie der Michel Angelo\pwindex{Brandes, Georg 4.\,2.\,1842 Kopenhagen – 19.\,2.\,1927 ebd.@\textsc{Brandes, Georg} (4.\,2.\,1842 Kopenhagen – 19.\,2.\,1927 ebd.)!Michelangelo Buonarotti@\strich\emph{Michelangelo Buonarotti}|pw}. Wie werd ich Ihnen immer von neuem, – und wie gern
               immer wieder Dank schuldig. – Ich bin in den letzten Monaten nicht ganz unthätig
               gewesen, und hoffe mich für Ihre kostbaren {\pb}Gaben,
               in recht bescheidener Weise\pwindex{Schnitzler, Arthur 15.\,5.\,1862 Wien – 21.\,10.\,1931 ebd.@\textsc{Schnitzler, Arthur} (15.\,5.\,1862 Wien – 21.\,10.\,1931 ebd.), \emph{Schriftsteller, Mediziner}!Fräulein Else@\strich\emph{Fräulein Else}|pwv}\pwindex{Schnitzler, Arthur 15.\,5.\,1862 Wien – 21.\,10.\,1931 ebd.@\textsc{Schnitzler, Arthur} (15.\,5.\,1862 Wien – 21.\,10.\,1931 ebd.), \emph{Schriftsteller, Mediziner}!Komödie der Verführung. In drei Akten@\strich\emph{Komödie der Verführung. In drei Akten}|pwv}, bald revanchiren zu dürfen. Ich hoffe liebster un\textcolor{gray}{d}
               verehrtester Georg Brandes, Sie befin\textcolor{gray}{d}en sich wohl. Lassen Sie
               mich auch darüber ein Wort vernehmen; ich schreibe demnächst mehr. In Freundschaft u.
               Bewunderung stets der Ihre \spacefill\mbox{Arthur Schnitzler}\pend
           \selectlanguage{ngerman}\endnumbering\briefempfaengerindex{Brandes, Georg@\textsc{Brandes, Georg}!zzzSchnitzler, Arthur@\emph{von Arthur Schnitzler}!1924-06-231@{23. 6. 1924}|)be}\mylabel{L02413h}  \newcommand{\dateiname}{L02413}\newcommand{\titel}{Arthur Schnitzler an Georg Brandes, 23. 6. 1924}\newcommand{\editorInnen}{Martin Anton Müller und Gerd-Hermann Susen}%% latex-leseansicht-abspann.tex
%% Abspann für die Leseansicht.
%% Der Schalter \ifkorrekturansicht ist bereits durch den Vorspann gesetzt.

%% latex-abspann.tex
%% Gemeinsamer Abspann für Korrekturansicht und Leseansicht.
%% Setzt den Schalter \ifkorrekturansicht voraus (gesetzt in den
%% einbindenden Dateien latex-korrekturansicht-abspann.tex bzw.
%% latex-leseansicht-abspann.tex).
%% ---------------------------------------------------------------

\normalsize

% Das esempio-Environment wird nur in der Leseansicht benötigt
\ifkorrekturansicht\else
\newenvironment{esempio}[3]%
{
    \vspace{1.5ex}
    \rlap{\underline{#1}}
    \par
    \setlength{\parindent}{0cm}
    \nopagebreak
    \leftskip=#2cm
    \rightskip=#3cm
}
{
    \par
}
\fi

\doendnotes{C}
\bigskip
\vfill

\clearpage

\footnotesize

\ifkorrekturansicht
  \lohead{\textsc{register}}
\fi

% theindex-Environment neu definieren ohne reledmac
\makeatletter
\renewenvironment{theindex}{%
  \ifkorrekturansicht
    \section*{\indexname}%
  \else
    \subsubsection*{Index der erwähnten Entitäten}%
  \fi
  \setlength{\parindent}{0pt}%
  \setlength{\parskip}{0pt plus 0.3pt}%
  \let\item\@idxitem
}{%
  \ifkorrekturansicht\clearpage\fi
}
\makeatother

\IfFileExists{\jobname-pw.ind}{\input{\jobname-pw.ind}}{}

% Quellenangabe nur in der Leseansicht
\ifkorrekturansicht\else
% Fallback-Definitionen, falls die .tex-Datei \titel etc. nicht gesetzt hat
\providecommand{\titel}{}
\providecommand{\editorInnen}{}
\providecommand{\dateiname}{\jobname}

\vspace{3cm}

\vfill

\footnotesize
\textsc{Quelle}: \titel. Herausgegeben von {\editorInnen}. In: \emph{Arthur Schnitzler: Briefwechsel mit Autorinnen und Autoren}.
 Digitale Edition, https://schnitzler-briefe.acdh.oeaw.ac.at/{\dateiname}.html (Stand \today)
\fi

\end{document}


