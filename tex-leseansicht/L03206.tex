%% latex-leseansicht-vorspann.tex
%% Vorspann für die Leseansicht.
%% Lädt die gemeinsame Datei latex-vorspann.tex mit nicht gesetztem Schalter.

\newif\ifkorrekturansicht
\korrekturansichtfalse

\input{../tex-inputs/latex-vorspann}

\begin{center}
            \textcolor{red}{ENTWURF, NICHT FERTIG KORRIGIERT}
                      \end{center}
            
         
         \renewcommand{\erwaehntePersonen}{Personen: Gerhart Hauptmann, Olga Schnitzler, Karl Schönherr}
         \renewcommand{\erwaehnteInstitutionen}{Institutionen: Tägliche Rundschau}
         \renewcommand{\erwaehnteOrte}{Orte: Berlin, Brühl, Dessauer Straße, Deutschland, Wien}
         \renewcommand{\erwaehnteWerke}{Werke: Der Sonnwendtag. Drama in fünf Akten, Die Weber, Tägliche Rundschau}
               \section[ Paul Goldmann an Arthur Schnitzler, 2. 5. {[}1902{]}]{ Paul Goldmann an Arthur Schnitzler, 2. 5. {[}1902{]}}\nopagebreak\mylabel{v}\rehead{ }\begin{ledgroupsized}[t]{13cm}\normalsize\beginnumbering \toendnotes[C]{\smallbreak\pagebreak[2]} \Standort{DLA, A:Schnitzler, HS.NZ85.1.3172.}
\physDesc{Brief, 1 Blatt, 2 Seiten
\newline{}Handschrift: blaue Tinte, deutsche Kurrent
\newline{}Schnitzler: mit Bleistift das Jahr »{[}1{]}902«
                                            vermerkt }\toendnotes[C]{\smallbreak}\pstart
           \noindent{}\raggedleft{}{\pb}\textcolor{gray}{\textbf{DESSAUERSTRASSE 19}}\oindex{Dessauer Strasse@\textbf{Dessauer Straße}|pw}\pend
           \pstart
           Berlin\oindex{Berlin@\textbf{Berlin}|pw}, 2. Mai.\pend
           \pstart\center{}Mein lieber Freund,\pend\pstart
           Daß Du den \label{K_L03206-1v}\edtext{Schwindler\textcolor{red}{\textsuperscript{\textbf{KEY}}}}{\lemma{\textnormal{\emph{Schwindler}}}\Cendnote{\textnormal{siehe Paul Goldmann an Arthur Schnitzler, 29. 4. [1902]}}}\label{K_L03206-1h}, den \textsc{Jurco\textcolor{red}{\textsuperscript{\textbf{KEY}}}} ſelbſt, laufen läßt, verſtehe ich. Der Kerl hat ſein Theil. Aber ganz and
                    gar nicht einverſtanden bin ich damit, daß Du Herrn \textsc{Stecker\textcolor{red}{\textsuperscript{\textbf{KEY}}}}, dem deutſchen Mann\textcolor{red}{\textsuperscript{\textbf{KEY}}} und
                    literariſchen Kritiker\textcolor{red}{\textsuperscript{\textbf{KEY}}}, ſo
                    vollſtändig nachgibſt. Das Benehmen dieſes Menſchen\textcolor{red}{\textsuperscript{\textbf{KEY}}} iſt von einer ſo unerhörten Unanſtändigkeit, daß Du
                    gerade darum energiſch auf Deinem Recht beſtehen müßteſt. Die Leſer der »täglichen Rundſchau\pwindex{?? Werk@Nicht ermittelte Verfasserinnen und Verfasser!Taegliche Rundschau1881 – 1933@\emph{Tägliche Rundschau} {[}1881 – 1933{]}|pw}« (und das Blatt\pwindex{?? Werk@Nicht ermittelte Verfasserinnen und Verfasser!Taegliche Rundschau1881 – 1933@\emph{Tägliche Rundschau} {[}1881 – 1933{]}|pwv} iſt in Deutſchland\oindex{Deutschland@\textbf{Deutschland}|pw} mehr geleſen, als irgendeine Wien\oindex{Wien@\textbf{Wien}|pw}er Zeitung) müſſen glauben, daß Du, da Du
                    auf die »offene Frage\textcolor{red}{\textsuperscript{\textbf{KEY}}}« nicht
                    geantwortet haſt, an dem Schwindel des Herrn \textsc{Jurco\textcolor{red}{\textsuperscript{\textbf{KEY}}}} mitbetheiligt biſt. Ich würde es nicht begreifen, wenn {\pb}Du \strikeout{es} darauf
                    verzichteteſt, in dieſer Angelegenheit entſchieden Dein Recht zu verlangen. Du
                    mußt es um Deintwegen thun, und dann beſteht auch ein gewiſſes allgemeines
                    Intereſſe, daß die Unanſtändigkeit eines ehrenfeſten deutſch\oindex{Deutschland@\textbf{Deutschland}|pwv}en Mann\textcolor{red}{\textsuperscript{\textbf{KEY}}}es, des Kritikers eines alldeutſch\oindex{Deutschland@\textbf{Deutschland}|pwv}en und
                    antiſemitiſchen Blatt\orgindex{Taegliche Rundschau@Tägliche Rundschau|pwv}es,
                    an die Öffentlichkeit gebracht wird. Du \strikeout{\textcolor{gray}{m}} mußt ihm ſofort ſchreiben und auf der Veröffentlichung Deiner Antwort\textcolor{red}{\textsuperscript{\textbf{KEY}}} beſtehen. Das wird dem Herr\textcolor{red}{\textsuperscript{\textbf{KEY}}}n lehren, im nächſten »Fall \textsc{Schnitzler}\textcolor{red}{\textsuperscript{\textbf{KEY}}}« vorſichtiger zu ſein.\pend
           \pstart
           Ich habe eben den »Sonnwendtag\pwindex{Schoenherr, Karl 24.02.1867 – 15.03.1943@\textsc{Schönherr, Karl} (24.02.1867 – 15.03.1943), \emph{Schriftsteller, Mediziner}!Sonnwendtag. Drama in fuenf Akten1902-01-06@\strich\emph{Der Sonnwendtag. Drama in fünf Akten} {[}1902-01-06{]}|pw}« geleſen. Das
                        Stück\pwindex{Schoenherr, Karl 24.02.1867 – 15.03.1943@\textsc{Schönherr, Karl} (24.02.1867 – 15.03.1943), \emph{Schriftsteller, Mediziner}!Sonnwendtag. Drama in fuenf Akten1902-01-06@\strich\emph{Der Sonnwendtag. Drama in fünf Akten} {[}1902-01-06{]}|pwv} hat mich ſehr
                    ergriffen. Wieviel höher ſteht dieſes Werk\pwindex{Schoenherr, Karl 24.02.1867 – 15.03.1943@\textsc{Schönherr, Karl} (24.02.1867 – 15.03.1943), \emph{Schriftsteller, Mediziner}!Sonnwendtag. Drama in fuenf Akten1902-01-06@\strich\emph{Der Sonnwendtag. Drama in fünf Akten} {[}1902-01-06{]}|pwv} eines Dichter\pwindex{Schoenherr, Karl 24.02.1867 – 15.03.1943@\textsc{Schönherr, Karl} (24.02.1867 – 15.03.1943), \emph{Schriftsteller, Mediziner}|pwv}s als ſämmtliche \textsc{Hauptmann\pwindex{Hauptmann, Gerhart 15.11.1862 – 06.06.1946@\textsc{Hauptmann, Gerhart} (15.11.1862 – 06.06.1946), \emph{Schriftsteller}|pw}sche} Dramen (mit
                    Ausnahme der »\label{K_L03206-4v}\edtext{Weber\pwindex{Hauptmann, Gerhart 15.11.1862 – 06.06.1946@\textsc{Hauptmann, Gerhart} (15.11.1862 – 06.06.1946), \emph{Schriftsteller}!Weber1892@\strich\emph{Die Weber} {[}1892{]}|pw}}{\lemma{\textnormal{\emph{Weber}}}\Cendnote{\textnormal{siehe Paul Goldmann an Arthur Schnitzler, 31. 12. [1900]}}}\label{K_L03206-4h}«)!\pend
           \pstart
           Grüße \textsc{Olga\pwindex{Schnitzler, Olga 17.01.1882 – 13.01.1970@\textsc{Schnitzler, Olga} (17.01.1882 – 13.01.1970), \emph{Schauspielerin, Sängerin}|pw}} und ſei vielmals und von Herzen gegrüßt von Deinem {\\[\baselineskip]}\spacefill\mbox{Paul
                        Goldmnn}\pend
           \leftskip=0em{}\pstart
           \noindent{}Biſt Du Pfingſten in Wien\oindex{Wien@\textbf{Wien}|pw}? Vielleicht \label{K_L03206-3v}\edtext{komme ich}{\lemma{\textnormal{\emph{komme ich}}}\Cendnote{\textnormal{Goldmann\pwindex{Goldmann, Paul 31.01.1865 – 25.09.1935@\textsc{Goldmann, Paul} (31.01.1865 – 25.09.1935), \emph{Schriftsteller, Journalist}|pwk} war
                            von 18. 5. 1902 bis jedenfalls 25. 5. 1902 in Wien\oindex{Wien@\textbf{Wien}|pwk} bzw. der Brühl\oindex{Bruehl@\textbf{Brühl}|pwk}.}}}\label{K_L03206-3h} hin.\pend
           
         
         \endnumbering\mylabel{h}\end{ledgroupsized}\begin{anhang}\end{anhang}\newcommand{\dateiname}{L03206}\newcommand{\titel}{Paul Goldmann an Arthur Schnitzler, 2. 5. [1902]}\newcommand{\editorInnen}{Martin Anton Müller und Laura Untner}%% latex-leseansicht-abspann.tex
%% Abspann für die Leseansicht.
%% Der Schalter \ifkorrekturansicht ist bereits durch den Vorspann gesetzt.

%% latex-abspann.tex
%% Gemeinsamer Abspann für Korrekturansicht und Leseansicht.
%% Setzt den Schalter \ifkorrekturansicht voraus (gesetzt in den
%% einbindenden Dateien latex-korrekturansicht-abspann.tex bzw.
%% latex-leseansicht-abspann.tex).
%% ---------------------------------------------------------------

\normalsize

% Das esempio-Environment wird nur in der Leseansicht benötigt
\ifkorrekturansicht\else
\newenvironment{esempio}[3]%
{
    \vspace{1.5ex}
    \rlap{\underline{#1}}
    \par
    \setlength{\parindent}{0cm}
    \nopagebreak
    \leftskip=#2cm
    \rightskip=#3cm
}
{
    \par
}
\fi

\doendnotes{C}
\bigskip
\vfill

\clearpage

\footnotesize

\ifkorrekturansicht
  \lohead{\textsc{register}}
\fi

% theindex-Environment neu definieren ohne reledmac
\makeatletter
\renewenvironment{theindex}{%
  \ifkorrekturansicht
    \section*{\indexname}%
  \else
    \subsubsection*{Index der erwähnten Entitäten}%
  \fi
  \setlength{\parindent}{0pt}%
  \setlength{\parskip}{0pt plus 0.3pt}%
  \let\item\@idxitem
}{%
  \ifkorrekturansicht\clearpage\fi
}
\makeatother

\IfFileExists{\jobname-pw.ind}{\input{\jobname-pw.ind}}{}

% Quellenangabe nur in der Leseansicht
\ifkorrekturansicht\else
% Fallback-Definitionen, falls die .tex-Datei \titel etc. nicht gesetzt hat
\providecommand{\titel}{}
\providecommand{\editorInnen}{}
\providecommand{\dateiname}{\jobname}

\vspace{3cm}

\vfill

\footnotesize
\textsc{Quelle}: \titel. Herausgegeben von {\editorInnen}. In: \emph{Arthur Schnitzler: Briefwechsel mit Autorinnen und Autoren}.
 Digitale Edition, https://schnitzler-briefe.acdh.oeaw.ac.at/{\dateiname}.html (Stand \today)
\fi

\end{document}


      