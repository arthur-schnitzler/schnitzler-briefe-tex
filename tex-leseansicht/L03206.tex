%% latex-korrekturansicht-vorspann.tex
%% Vorspann für die Korrekturansicht.
%% Lädt die gemeinsame Datei latex-vorspann.tex mit gesetztem Schalter.

\newif\ifkorrekturansicht
\korrekturansichttrue

\input{../tex-inputs/latex-vorspann}


\section[ Paul Goldmann an Arthur Schnitzler, 2. 5. {[}1902{]}]{L03206 Paul Goldmann an Arthur Schnitzler, 2. 5. {[}1902{]}}
\nopagebreak\mylabel{L03206v}
\rehead{ }\normalsize\beginnumbering\briefempfaengerindex{Schnitzler, Arthur@\textsc{Schnitzler, Arthur}!zzzGoldmann, Paul@\emph{von Paul Goldmann}!1902-05-022@{2. 5. {[}1902{]}}|(be}
\toendnotes[C]{\smallbreak\pagebreak[2]}\Standort{DLA, A:Schnitzler, HS.NZ85.1.3172.}
\physDesc{Brief, 1 Blatt, 2 Seiten, 1467 Zeichen
\newline{}Handschrift: blaue Tinte, deutsche Kurrent
\newline{}Schnitzler: 1) mit Bleistift das Jahr »902« vermerkt  2) mit rotem Buntstift eine Unterstreichung}\toendnotes[C]{\smallbreak}
\pstart
           \raggedleft{}{\pb}\textcolor{gray}{\textbf{DESSAUERSTRASSE 19}}\oindex{Dessauer Strasse@\textbf{Dessauer Straße}, \emph{Straße (K.STR)}|pw}\pend
           
\pstart
           Berlin\oindex{Berlin@\textbf{Berlin}, \emph{P.PPLC}|pw}, 2. Mai.\pend
           
\pstart\center{}Mein lieber Freund,\pend\vspace{0.5em}
\pstart
           Daß Du den \label{K_L03206-1v}\edtext{Schwindler\pwindex{Greger-Jurco, Ernest von *~11.08.1860@\textsc{Gréger-Jurco, Ernest von} (*~11.08.1860), \emph{Schriftsteller/Schriftstellerin}|pwv}}{\lemma{\textnormal{\emph{Schwindler}}}\Cendnote{\textnormal{Siehe Paul Goldmann an Arthur Schnitzler, 29. 4. [1902].
               }}}\label{K_L03206-1}, den \textsc{Jurco\pwindex{Greger-Jurco, Ernest von *~11.08.1860@\textsc{Gréger-Jurco, Ernest von} (*~11.08.1860), \emph{Schriftsteller/Schriftstellerin}|pw}} ſelbſt, laufen läßt, verſtehe ich. Der Kerl hat ſein Theil. Aber ganz and gar
               nicht einverſtanden bin ich damit, daß Du Herrn \textsc{Karl Strecker\pwindex{Strecker, Karl 1862-04-08 – 1933-02-19@\textsc{Strecker, Karl} (1862-04-08 – 1933-02-19), \emph{Theaterkritiker/Theaterkritikerin}|pw}}, dem deutſchen Mann\pwindex{Strecker, Karl 1862-04-08 – 1933-02-19@\textsc{Strecker, Karl} (1862-04-08 – 1933-02-19), \emph{Theaterkritiker/Theaterkritikerin}|pwv} und
               literariſchen Kritiker\pwindex{Strecker, Karl 1862-04-08 – 1933-02-19@\textsc{Strecker, Karl} (1862-04-08 – 1933-02-19), \emph{Theaterkritiker/Theaterkritikerin}|pwv}, ſo
               vollſtändig nachgibſt. Das Benehmen dieſes Menſchen\pwindex{Strecker, Karl 1862-04-08 – 1933-02-19@\textsc{Strecker, Karl} (1862-04-08 – 1933-02-19), \emph{Theaterkritiker/Theaterkritikerin}|pwv} iſt von einer ſo unerhörten Unanſtändigkeit, daß Du
               gerade darum energiſch auf Deinem Recht beſtehen müßteſt. Die Leſer der »Täglichen Rundſchau\pwindex{Taegliche Rundschau@\emph{Tägliche Rundschau}|pw}« (und das Blatt\pwindex{Taegliche Rundschau@\emph{Tägliche Rundschau}|pwv} iſt in Deutſchland\oindex{Deutschland@\textbf{Deutschland}, \emph{A.PCLI}|pw} mehr geleſen, als irgendeine Wien\oindex{Wien@\textbf{Wien}, \emph{A.ADM2}|pw}er Zeitung) müſſen glauben, daß Du, da Du auf die \label{K_L03206-2v}\edtext{»offene Frage\pwindex{litterarisch-dramatisches Hochstapler-Stuecklein@\emph{Ein litterarisch-dramatisches Hochstapler-Stücklein}|pwv}«}{\lemma{\textnormal{\emph{»offene Frage«}}}\Cendnote{\textnormal{Siehe Paul Goldmann an Arthur Schnitzler, 26. 4. 1902.
               }}}\label{K_L03206-2} nicht geantwortet haſt, an dem Schwindel des Herrn \textsc{Jurco\pwindex{Greger-Jurco, Ernest von *~11.08.1860@\textsc{Gréger-Jurco, Ernest von} (*~11.08.1860), \emph{Schriftsteller/Schriftstellerin}|pw}} mitbetheiligt biſt. Ich würde es nicht begreifen, wenn {\pb}Du \strikeout{es} darauf
               verzichteteſt, in dieſer Angelegenheit entſchieden Dein Recht zu verlangen. Du mußt
               es um Deinetwegen thun, und dann beſteht auch ein gewiſſes allgemeines Intereſſe, daß
               die Unanſtändigkeit eines ehrenfeſten deutſch\oindex{Deutschland@\textbf{Deutschland}, \emph{A.PCLI}|pwv}en Mann\pwindex{Strecker, Karl 1862-04-08 – 1933-02-19@\textsc{Strecker, Karl} (1862-04-08 – 1933-02-19), \emph{Theaterkritiker/Theaterkritikerin}|pwv}es, des Kritikers eines alldeutſch\oindex{Deutschland@\textbf{Deutschland}, \emph{A.PCLI}|pwv}en und antiſemitiſchen Blatt\orgindex{Taegliche Rundschau@Tägliche Rundschau|pwv}es, an die Öffentlichkeit gebracht wird. Du \strikeout{\textcolor{gray}{m}} mußt ihm ſofort ſchreiben und auf der Veröffentlichung Deiner Antwort\pwindex{angebliche Telegramm Arthur Schnitzlers@\emph{Das angebliche Telegramm Arthur Schnitzlers}|pwv} beſtehen. Das wird dem Herr\pwindex{Strecker, Karl 1862-04-08 – 1933-02-19@\textsc{Strecker, Karl} (1862-04-08 – 1933-02-19), \emph{Theaterkritiker/Theaterkritikerin}|pwv}n lehren, im nächſten »Fall \textsc{Schnitzler}\pwindex{litterarisch-dramatisches Hochstapler-Stuecklein@\emph{Ein litterarisch-dramatisches Hochstapler-Stücklein}|pwv}« vorſichtiger zu ſein.\pend
           
\pstart
           Ich habe eben den »Sonnwendtag\pwindex{Sonnwendtag. Drama in fuenf Akten@\emph{Der Sonnwendtag. Drama in fünf Akten}|pw}« geleſen. Das Stück\pwindex{Sonnwendtag. Drama in fuenf Akten@\emph{Der Sonnwendtag. Drama in fünf Akten}|pwv} hat mich ſehr ergriffen.
               Wieviel höher ſteht dieſes Werk\pwindex{Sonnwendtag. Drama in fuenf Akten@\emph{Der Sonnwendtag. Drama in fünf Akten}|pwv} eines Dichters\pwindex{Schoenherr, Karl 24.02.1867 – 15.03.1943@\textsc{Schönherr, Karl} (24.02.1867 – 15.03.1943), \emph{Schriftsteller/Schriftstellerin, Mediziner/Medizinerin}|pwv}
               als ſämmtliche \textsc{Hauptmann\pwindex{Hauptmann, Gerhart 15.11.1862 – 06.06.1946@\textsc{Hauptmann, Gerhart} (15.11.1862 – 06.06.1946), \emph{Schriftsteller/Schriftstellerin}|pw}sche} Dramen (mit Ausnahme der
                  »\label{K_L03206-3v}\edtext{Weber\pwindex{Weber@\emph{Die Weber}|pw}}{\lemma{\textnormal{\emph{Weber}}}\Cendnote{\textnormal{Siehe Paul Goldmann an Arthur Schnitzler, 31. 12. [1900].
               }}}\label{K_L03206-3}«)!\pend
           
\pstart
           Grüße \textsc{Olga\pwindex{Schnitzler, Olga 17.01.1882 – 13.01.1970@\textsc{Schnitzler, Olga} (17.01.1882 – 13.01.1970), \emph{Schauspieler/Schauspielerin, Sänger/Sängerin}|pw}} und ſei vielmals und von Herzen gegrüßt von Deinem {\\[\baselineskip]}\spacefill\mbox{Paul Goldmnn}\pend
           \leftskip=0em{}
\pstart
           \noindent{}Biſt Du Pfingſten in Wien\oindex{Wien@\textbf{Wien}, \emph{A.ADM2}|pw}? Vielleicht \label{K_L03206-4v}\edtext{komme
                     ich}{\lemma{\textnormal{\emph{komme
                     ich}}}\Cendnote{\textnormal{Schnitzler und Goldmann\pwindex{Goldmann, Paul 31.01.1865 – 25.09.1935@\textsc{Goldmann, Paul} (31.01.1865 – 25.09.1935), \emph{Schriftsteller/Schriftstellerin, Journalist/Journalistin}|pwk} sahen sich zwischen 18. 5. 1902 und
                        25. 5. 1902 in
                        Wien\oindex{Wien@\textbf{Wien}, \emph{A.ADM2}|pwk} und teilweise auf Tagesausflügen
                     nach Hinterbrühl\oindex{Hinterbruehl@\textbf{Hinterbrühl}, \emph{P.PPLA3}|pwk}.}}}\label{K_L03206-4} hin.\pend
           \selectlanguage{ngerman}\endnumbering\briefempfaengerindex{Schnitzler, Arthur@\textsc{Schnitzler, Arthur}!zzzGoldmann, Paul@\emph{von Paul Goldmann}!1902-05-022@{2. 5. {[}1902{]}}|)be}\mylabel{L03206h}  \normalsize

\doendnotes{C}
\bigskip
\vfill

\clearpage

\footnotesize

\lohead{\textsc{register}}

% Definiere theindex-Environment komplett neu ohne reledmac
\makeatletter
\renewenvironment{theindex}{%
  \section*{\indexname}%
  \setlength{\parindent}{0pt}%
  \setlength{\parskip}{0pt plus 0.3pt}%
  \let\item\@idxitem
}{%
  \clearpage
}
\makeatother

\IfFileExists{\jobname-pw.ind}{\input{\jobname-pw.ind}}{}

\end{document}

      