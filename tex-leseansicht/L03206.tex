%% latex-leseansicht-vorspann.tex
%% Vorspann für die Leseansicht.
%% Lädt die gemeinsame Datei latex-vorspann.tex mit nicht gesetztem Schalter.

\newif\ifkorrekturansicht
\korrekturansichtfalse

\input{../tex-inputs/latex-vorspann}


\section[ Paul Goldmann an Arthur Schnitzler, 2. 5. [1902]]{L03206 Paul Goldmann an Arthur Schnitzler,  2. 5. [1902]}
\nopagebreak\mylabel{L03206v}
\rehead{ }\normalsize\beginnumbering\briefempfaengerindex{Schnitzler, Arthur@\textsc{Schnitzler, Arthur}!zzzGoldmann, Paul@\emph{von Paul Goldmann}!1902-05-022@{2. 5. [1902]}|(be}
\toendnotes[C]{\smallbreak\pagebreak[2]}
\correspDesc{Versand  durch Paul Goldmann am 2. 5. [1902] in Berlin
\newline{}Erhalt  durch Arthur Schnitzler im Zeitraum [3. 5. 1902
                  – 7. 5. 1902?] in Wien}\toendnotes[C]{\smallbreak}
\Standort{DLA, A:Schnitzler, HS.NZ85.1.3172.}
\physDesc{Brief, 1 Blatt, 2 Seiten, 1467 Zeichen
\newline{}Handschrift: blaue Tinte, deutsche Kurrent
\newline{}Schnitzler: 1) mit Bleistift das Jahr »902« vermerkt  2) mit rotem Buntstift eine Unterstreichung}\toendnotes[C]{\smallbreak}
\pstart
           \raggedleft{}{\pb}\textcolor{gray}{\textbf{DESSAUERSTRASSE 19}}\oindex{Dessauer Straße@\textbf{Dessauer Straße}, \emph{Straße}|pw}\pend
           
\pstart
           Berlin\oindex{Berlin@\textbf{Berlin}, \emph{Hauptstadt}|pw}, 2. Mai.\pend
           
\pstart\center{}Mein lieber Freund,\pend\vspace{0.5em}
\pstart
           Daß Du den \label{K_L03206-1v}\edtext{Schwindler\pwindex{Gréger-Jurco, Ernest von *~11.\,8.\,1860 Orăştie@\textsc{Gréger-Jurco, Ernest von} (*~11.\,8.\,1860 Orăştie), \emph{Schriftsteller}|pwv}}{\lemma{\textnormal{\emph{Schwindler}}}\Cendnote{\textnormal{Siehe XXXX Auszeichnungsfehler: Dokument L03205 nicht gefunden.
               }}}\label{K_L03206-1}, den \textsc{Jurco\pwindex{Gréger-Jurco, Ernest von *~11.\,8.\,1860 Orăştie@\textsc{Gréger-Jurco, Ernest von} (*~11.\,8.\,1860 Orăştie), \emph{Schriftsteller}|pw}}{ }ſelbſt, laufen läßt, verſtehe ich. Der Kerl hat{ }ſein Theil. Aber ganz and gar
               nicht einverſtanden bin ich damit, daß Du Herrn \textsc{Karl Strecker\pwindex{Strecker, Karl 8.\,4.\,1862 Tąpadły – 19.\,2.\,1933 Garmisch-Partenkirchen@\textsc{Strecker, Karl} (8.\,4.\,1862 Tąpadły – 19.\,2.\,1933 Garmisch-Partenkirchen), \emph{Theaterkritiker}|pw}}, dem deutſchen Mann\pwindex{Strecker, Karl 8.\,4.\,1862 Tąpadły – 19.\,2.\,1933 Garmisch-Partenkirchen@\textsc{Strecker, Karl} (8.\,4.\,1862 Tąpadły – 19.\,2.\,1933 Garmisch-Partenkirchen), \emph{Theaterkritiker}|pwv} und
               literariſchen Kritiker\pwindex{Strecker, Karl 8.\,4.\,1862 Tąpadły – 19.\,2.\,1933 Garmisch-Partenkirchen@\textsc{Strecker, Karl} (8.\,4.\,1862 Tąpadły – 19.\,2.\,1933 Garmisch-Partenkirchen), \emph{Theaterkritiker}|pwv},{ }ſo
               vollſtändig nachgibſt. Das Benehmen dieſes Menſchen\pwindex{Strecker, Karl 8.\,4.\,1862 Tąpadły – 19.\,2.\,1933 Garmisch-Partenkirchen@\textsc{Strecker, Karl} (8.\,4.\,1862 Tąpadły – 19.\,2.\,1933 Garmisch-Partenkirchen), \emph{Theaterkritiker}|pwv} iſt von einer{ }ſo unerhörten Unanſtändigkeit, daß Du
               gerade darum energiſch auf Deinem Recht beſtehen müßteſt. Die Leſer der »Täglichen Rundſchau\pwindex{Tägliche Rundschau@\emph{Tägliche Rundschau}|pw}« (und das Blatt\pwindex{Tägliche Rundschau@\emph{Tägliche Rundschau}|pwv} iſt in Deutſchland\oindex{Deutschland@\textbf{Deutschland}|pw} mehr geleſen, als irgendeine Wien\oindex{Wien@\textbf{Wien}, \emph{Verwaltungsgebiet}|pw}er Zeitung) müſſen glauben, daß Du, da Du auf die \label{K_L03206-2v}\edtext{»offene Frage\pwindex{Strecker, Karl 8.\,4.\,1862 Tąpadły – 19.\,2.\,1933 Garmisch-Partenkirchen@\textsc{Strecker, Karl} (8.\,4.\,1862 Tąpadły – 19.\,2.\,1933 Garmisch-Partenkirchen), \emph{Theaterkritiker}!litterarisch-dramatisches Hochstapler-Stücklein@\strich\emph{Ein litterarisch-dramatisches Hochstapler-Stücklein}|pwv}«}{\lemma{\textnormal{\emph{»offene Frage«}}}\Cendnote{\textnormal{Siehe XXXX Auszeichnungsfehler: Dokument L02634 nicht gefunden.
               }}}\label{K_L03206-2} nicht geantwortet haſt, an dem Schwindel des Herrn \textsc{Jurco\pwindex{Gréger-Jurco, Ernest von *~11.\,8.\,1860 Orăştie@\textsc{Gréger-Jurco, Ernest von} (*~11.\,8.\,1860 Orăştie), \emph{Schriftsteller}|pw}} mitbetheiligt biſt. Ich würde es nicht begreifen, wenn {\pb}Du \strikeout{es} darauf
               verzichteteſt, in dieſer Angelegenheit entſchieden Dein Recht zu verlangen. Du mußt
               es um Deinetwegen thun, und dann beſteht auch ein gewiſſes allgemeines Intereſſe, daß
               die Unanſtändigkeit eines ehrenfeſten deutſch\oindex{Deutschland@\textbf{Deutschland}|pwv}en Mann\pwindex{Strecker, Karl 8.\,4.\,1862 Tąpadły – 19.\,2.\,1933 Garmisch-Partenkirchen@\textsc{Strecker, Karl} (8.\,4.\,1862 Tąpadły – 19.\,2.\,1933 Garmisch-Partenkirchen), \emph{Theaterkritiker}|pwv}es, des Kritikers eines alldeutſch\oindex{Deutschland@\textbf{Deutschland}|pwv}en und antiſemitiſchen Blatt\orgindex{Tägliche Rundschau@Tägliche Rundschau|pwv}es, an die Öffentlichkeit gebracht wird. Du \strikeout{\textcolor{gray}{m}} mußt ihm{ }ſofort{ }ſchreiben und auf der Veröffentlichung Deiner Antwort\pwindex{Strecker, Karl 8.\,4.\,1862 Tąpadły – 19.\,2.\,1933 Garmisch-Partenkirchen@\textsc{Strecker, Karl} (8.\,4.\,1862 Tąpadły – 19.\,2.\,1933 Garmisch-Partenkirchen), \emph{Theaterkritiker}!angebliche Telegramm Arthur Schnitzlers@\strich\emph{Das angebliche Telegramm Arthur Schnitzlers}|pwv} beſtehen. Das wird dem Herr\pwindex{Strecker, Karl 8.\,4.\,1862 Tąpadły – 19.\,2.\,1933 Garmisch-Partenkirchen@\textsc{Strecker, Karl} (8.\,4.\,1862 Tąpadły – 19.\,2.\,1933 Garmisch-Partenkirchen), \emph{Theaterkritiker}|pwv}n lehren, im nächſten »Fall \textsc{Schnitzler}\pwindex{Strecker, Karl 8.\,4.\,1862 Tąpadły – 19.\,2.\,1933 Garmisch-Partenkirchen@\textsc{Strecker, Karl} (8.\,4.\,1862 Tąpadły – 19.\,2.\,1933 Garmisch-Partenkirchen), \emph{Theaterkritiker}!litterarisch-dramatisches Hochstapler-Stücklein@\strich\emph{Ein litterarisch-dramatisches Hochstapler-Stücklein}|pwv}« vorſichtiger zu{ }ſein.\pend
           
\pstart
           Ich habe eben den »Sonnwendtag\pwindex{Schönherr, Karl 24.\,2.\,1867 Axams – 15.\,3.\,1943 Wien@\textsc{Schönherr, Karl} (24.\,2.\,1867 Axams – 15.\,3.\,1943 Wien), \emph{Schriftsteller, Mediziner}!Sonnwendtag. Drama in fünf Akten@\strich\emph{Der Sonnwendtag. Drama in fünf Akten}|pw}« geleſen. Das Stück\pwindex{Schönherr, Karl 24.\,2.\,1867 Axams – 15.\,3.\,1943 Wien@\textsc{Schönherr, Karl} (24.\,2.\,1867 Axams – 15.\,3.\,1943 Wien), \emph{Schriftsteller, Mediziner}!Sonnwendtag. Drama in fünf Akten@\strich\emph{Der Sonnwendtag. Drama in fünf Akten}|pwv} hat mich{ }ſehr ergriffen.
               Wieviel höher{ }ſteht dieſes Werk\pwindex{Schönherr, Karl 24.\,2.\,1867 Axams – 15.\,3.\,1943 Wien@\textsc{Schönherr, Karl} (24.\,2.\,1867 Axams – 15.\,3.\,1943 Wien), \emph{Schriftsteller, Mediziner}!Sonnwendtag. Drama in fünf Akten@\strich\emph{Der Sonnwendtag. Drama in fünf Akten}|pwv} eines Dichters\pwindex{Schönherr, Karl 24.\,2.\,1867 Axams – 15.\,3.\,1943 Wien@\textsc{Schönherr, Karl} (24.\,2.\,1867 Axams – 15.\,3.\,1943 Wien), \emph{Schriftsteller, Mediziner}|pwv}
               als{ }ſämmtliche \textsc{Hauptmann\pwindex{Hauptmann, Gerhart 15.\,11.\,1862 Szczawno-Zdrój – 6.\,6.\,1946 Jagniątków@\textsc{Hauptmann, Gerhart} (15.\,11.\,1862 Szczawno-Zdrój – 6.\,6.\,1946 Jagniątków), \emph{Schriftsteller}|pw}sche} Dramen (mit Ausnahme der
                  »\label{K_L03206-3v}\edtext{Weber\pwindex{Hauptmann, Gerhart 15.\,11.\,1862 Szczawno-Zdrój – 6.\,6.\,1946 Jagniątków@\textsc{Hauptmann, Gerhart} (15.\,11.\,1862 Szczawno-Zdrój – 6.\,6.\,1946 Jagniątków), \emph{Schriftsteller}!Weber@\strich\emph{Die Weber}|pw}}{\lemma{\textnormal{\emph{Weber}}}\Cendnote{\textnormal{Siehe XXXX Auszeichnungsfehler: Dokument L02947 nicht gefunden.
               }}}\label{K_L03206-3}«)!\pend
           
\pstart
           Grüße \textsc{Olga\pwindex{Schnitzler, Olga 17.\,1.\,1882 Wien – 13.\,1.\,1970 Lugano@\textsc{Schnitzler, Olga} (17.\,1.\,1882 Wien – 13.\,1.\,1970 Lugano), \emph{Schauspielerin, Sängerin}|pw}} und{ }ſei vielmals und von Herzen gegrüßt von Deinem {\\[\baselineskip]}\spacefill\mbox{Paul Goldmnn}\pend
           \leftskip=0em{}
\pstart
           \noindent{}Biſt Du Pfingſten in Wien\oindex{Wien@\textbf{Wien}, \emph{Verwaltungsgebiet}|pw}? Vielleicht \label{K_L03206-4v}\edtext{komme
                     ich}{\lemma{\textnormal{\emph{komme
                     ich}}}\Cendnote{\textnormal{Schnitzler und Goldmann\pwindex{Goldmann, Paul 31.\,1.\,1865 Breslau – 25.\,9.\,1935 Wien@\textsc{Goldmann, Paul} (31.\,1.\,1865 Breslau – 25.\,9.\,1935 Wien), \emph{Schriftsteller, Journalist}|pwk} sahen sich zwischen 18. 5. 1902 und
                        25. 5. 1902 in
                        Wien\oindex{Wien@\textbf{Wien}, \emph{Verwaltungsgebiet}|pwk} und teilweise auf Tagesausflügen
                     nach Hinterbrühl\oindex{Hinterbrühl@\textbf{Hinterbrühl}, \emph{Hauptstadt}|pwk}.}}}\label{K_L03206-4} hin.\pend
           \selectlanguage{ngerman}\endnumbering\briefempfaengerindex{Schnitzler, Arthur@\textsc{Schnitzler, Arthur}!zzzGoldmann, Paul@\emph{von Paul Goldmann}!1902-05-022@{2. 5. [1902]}|)be}\mylabel{L03206h}  \newcommand{\dateiname}{L03206}\newcommand{\titel}{Paul Goldmann an Arthur Schnitzler, 2. 5. [1902]}\newcommand{\editorInnen}{Martin Anton Müller und Laura Untner}%% latex-leseansicht-abspann.tex
%% Abspann für die Leseansicht.
%% Der Schalter \ifkorrekturansicht ist bereits durch den Vorspann gesetzt.

%% latex-abspann.tex
%% Gemeinsamer Abspann für Korrekturansicht und Leseansicht.
%% Setzt den Schalter \ifkorrekturansicht voraus (gesetzt in den
%% einbindenden Dateien latex-korrekturansicht-abspann.tex bzw.
%% latex-leseansicht-abspann.tex).
%% ---------------------------------------------------------------

\normalsize

% Das esempio-Environment wird nur in der Leseansicht benötigt
\ifkorrekturansicht\else
\newenvironment{esempio}[3]%
{
    \vspace{1.5ex}
    \rlap{\underline{#1}}
    \par
    \setlength{\parindent}{0cm}
    \nopagebreak
    \leftskip=#2cm
    \rightskip=#3cm
}
{
    \par
}
\fi

\doendnotes{C}
\bigskip
\vfill

\clearpage

\footnotesize

\ifkorrekturansicht
  \lohead{\textsc{register}}
\fi

% theindex-Environment neu definieren ohne reledmac
\makeatletter
\renewenvironment{theindex}{%
  \ifkorrekturansicht
    \section*{\indexname}%
  \else
    \subsubsection*{Index der erwähnten Entitäten}%
  \fi
  \setlength{\parindent}{0pt}%
  \setlength{\parskip}{0pt plus 0.3pt}%
  \let\item\@idxitem
}{%
  \ifkorrekturansicht\clearpage\fi
}
\makeatother

\IfFileExists{\jobname-pw.ind}{\input{\jobname-pw.ind}}{}

% Quellenangabe nur in der Leseansicht
\ifkorrekturansicht\else
% Fallback-Definitionen, falls die .tex-Datei \titel etc. nicht gesetzt hat
\providecommand{\titel}{}
\providecommand{\editorInnen}{}
\providecommand{\dateiname}{\jobname}

\vspace{3cm}

\vfill

\footnotesize
\textsc{Quelle}: \titel. Herausgegeben von {\editorInnen}. In: \emph{Arthur Schnitzler: Briefwechsel mit Autorinnen und Autoren}.
 Digitale Edition, https://schnitzler-briefe.acdh.oeaw.ac.at/{\dateiname}.html (Stand \today)
\fi

\end{document}


