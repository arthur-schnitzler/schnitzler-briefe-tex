%% latex-korrekturansicht-vorspann.tex
%% Vorspann für die Korrekturansicht.
%% Lädt die gemeinsame Datei latex-vorspann.tex mit gesetztem Schalter.

\newif\ifkorrekturansicht
\korrekturansichttrue

\input{../tex-inputs/latex-vorspann}


\section[Wilhelm Bölsche an Arthur Schnitzler, 6. 10. 1891]{L00042 Wilhelm Bölsche an Arthur Schnitzler, 6. 10. 1891}
\nopagebreak\mylabel{L00042v}
\rehead{ }\normalsize\beginnumbering\briefempfaengerindex{Schnitzler, Arthur@\textsc{Schnitzler, Arthur}!zzzBoelsche, Wilhelm@\emph{von Wilhelm Bölsche}!1891-10-061@{6. 10. 1891}|(be}
\toendnotes[C]{\smallbreak\pagebreak[2]}\Standort{DLA, A:Schnitzler, HS.NZ85.1.2577,2.}
\physDesc{Brief, 1 Blatt, 1 Seite, 393 Zeichen
\newline{}Handschrift: schwarze Tinte, deutsche Kurrent
\newline{}Ordnung: mit rotem Buntstift von unbekannter Hand nummeriert:
                                    »2« }
\buchAbdrucke{\weitereDrucke{Wilhelm Bölsche: \emph{Briefwechsel. Mit Autoren der Freien Bühne}. Berlin: \emph{Weidler} 2010, S. 672.} }\toendnotes[C]{\smallbreak}
\pstart
           \raggedleft{}{\pb}Friedrichshagen\oindex{Friedrichshagen@\textbf{Friedrichshagen}, \emph{P.PPLX}|pw}{\\}b. Berlin\oindex{Berlin@\textbf{Berlin}, \emph{P.PPLC}|pw}.{\\}Wilhelmſtr 72\oindex{Peter-Hille-Strasse@\textbf{Peter-Hille-Straße}, \emph{Straße (K.STR)}|pw}.{\\}6. X. 91.\pend
           
\pstart\center{}Hochgeehrter Herr Doktor!\pend\vspace{0.5em}
\pstart
           Ich ſehe eben mit Bedauern, daß \label{K_L00042-1v}\edtext{mein
                  Stellvertreter\pwindex{Hart, Julius 1859-04-09 – 1930@\textsc{Hart, Julius} (1859-04-09 – 1930), \emph{Schriftsteller/Schriftstellerin, Journalist/Journalistin}|pwv}}{\lemma{\textnormal{\emph{mein
                  Stellvertreter}}}\Cendnote{\textnormal{Julius Hart\pwindex{Hart, Julius 1859-04-09 – 1930@\textsc{Hart, Julius} (1859-04-09 – 1930), \emph{Schriftsteller/Schriftstellerin, Journalist/Journalistin}|pwk} betreute die Redaktion der \emph{Freien Bühne}\pwindex{Freie Buehne fuer modernes Leben@\emph{Freie Bühne für modernes Leben}|pwk} vom 26. 8. 1891 bis
                  zum 23. 9. 1891.}}}\label{K_L00042-1} während meiner mehrmonatlichen Abweſenheit
               Sie nicht benachrichtigt hat, daß Ihre Novelle »Der
                  Sohn\pwindex{Sohn. Aus den Papieren eines Arztes@\emph{Der Sohn. Aus den Papieren eines Arztes}|pw}« von mir angenommen worden war. Nur etwas warten muß ſie leider, das
                  \label{K_L00042-2v}\edtext{Drama\pwindex{Lumpengesindel@\emph{Das Lumpengesindel}|pwv}}{\lemma{\textnormal{\emph{Drama}}}\Cendnote{\textnormal{Ernst von Wolzogen\pwindex{Wolzogen, Ernst von 23.04.1855 – 30.07.1934@\textsc{Wolzogen, Ernst von} (23.04.1855 – 30.07.1934), \emph{Schriftsteller/Schriftstellerin}|pwk}: \emph{Das Lumpengesindel. Komödie in 5 Aufzügen}\pwindex{Lumpengesindel@\emph{Das Lumpengesindel}|pwk}. In: \emph{Freie Bühne für modernes Leben}\pwindex{Freie Buehne fuer modernes Leben@\emph{Freie Bühne für modernes Leben}|pwk}, Jg. 2,
                     H. 40–52, 7. 10. 1891 – 30. 10. 1891 (13
                     Teile).}}}\label{K_L00042-2}, das wir jetzt abdrucken, ſchiebt alle Novellen
               zurück.\pend
           
\pstart
           Mit vorzüglicher Hochachtung{\\[\baselineskip]}\spacefill\mbox{Wilhelm Bölsche}\pend
           \leftskip=0em{}\selectlanguage{ngerman}\endnumbering\briefempfaengerindex{Schnitzler, Arthur@\textsc{Schnitzler, Arthur}!zzzBoelsche, Wilhelm@\emph{von Wilhelm Bölsche}!1891-10-061@{6. 10. 1891}|)be}\mylabel{L00042h}  \normalsize

\doendnotes{C}
\bigskip
\vfill

\clearpage

\footnotesize

\lohead{\textsc{register}}

% Definiere theindex-Environment komplett neu ohne reledmac
\makeatletter
\renewenvironment{theindex}{%
  \section*{\indexname}%
  \setlength{\parindent}{0pt}%
  \setlength{\parskip}{0pt plus 0.3pt}%
  \let\item\@idxitem
}{%
  \clearpage
}
\makeatother

\IfFileExists{\jobname-pw.ind}{\input{\jobname-pw.ind}}{}

\end{document}

      