%% latex-leseansicht-vorspann.tex
%% Vorspann für die Leseansicht.
%% Lädt die gemeinsame Datei latex-vorspann.tex mit nicht gesetztem Schalter.

\newif\ifkorrekturansicht
\korrekturansichtfalse

\input{../tex-inputs/latex-vorspann}


         
         \renewcommand{\erwaehntePersonen}{Personen: Julius Hart, Ernst von Wolzogen}
         \renewcommand{\erwaehnteOrte}{Orte: Berlin, Friedrichshagen, Peter-Hille-Straße, Wien}
         \renewcommand{\erwaehnteWerke}{Werke: Das Lumpengesindel, Der Sohn. Aus den Papieren eines Arztes, Freie Bühne für modernes Leben}
               \section[Wilhelm Bölsche an Arthur Schnitzler, 6. 10. 1891]{ Wilhelm Bölsche an Arthur Schnitzler, 6. 10. 1891}\nopagebreak\mylabel{v}\rehead{ }\begin{ledgroupsized}[t]{13cm}\normalsize\beginnumbering \toendnotes[C]{\smallbreak\pagebreak[2]} \Standort{DLA, A:Schnitzler, HS.NZ85.1.2577,2.}
\physDesc{Brief, 1 Blatt, 1 Seite
\newline{}Handschrift: schwarze Tinte, deutsche Kurrent\newline{}Ordnung: mit rotem Buntstift von unbekannter Hand nummeriert: »2« }\buchAbdrucke{\weitereDrucke{Wilhelm Bölsche: \emph{Briefwechsel. Mit Autoren der Freien Bühne}. Hg. Gerd-Hermann Susen. Berlin: \emph{Weidler} 2010, S. 672 (Werke und Briefe. Wissenschaftliche Ausgabe, Briefe I).} }\toendnotes[C]{\smallbreak}\pstart
           \raggedleft{}{\pb}Friedrichshagen\oindex{Friedrichshagen@\textbf{Friedrichshagen}|pw}{\\}b. Berlin\oindex{Berlin@\textbf{Berlin}|pw}.{\\}Wilhelmſtr 72\oindex{Peter-Hille-Strasse@\textbf{Peter-Hille-Straße}|pw}.{\\}6. X. 91.\pend
           \pstart\center{}Hochgeehrter Herr Doktor!\pend\pstart
           Ich ſehe eben mit Bedauern, daß \label{K_L00042_1v}\edtext{mein Stellvertreter\pwindex{Hart, Julius 1859-04-09 – 1930@\textsc{Hart, Julius} (1859-04-09 – 1930), \emph{Schriftsteller, Journalist}|pwv}}{\lemma{\textnormal{\emph{mein Stellvertreter}}}\Cendnote{\textnormal{Julius Hart\pwindex{Hart, Julius 1859-04-09 – 1930@\textsc{Hart, Julius} (1859-04-09 – 1930), \emph{Schriftsteller, Journalist}|pwk} betreute die Redaktion der
                            \emph{Freien Bühne}\pwindex{Freie Buehne fuer modernes Leben1890 – 1891@\emph{Freie Bühne für modernes Leben} {[}1890 – 1891{]}|pwk} vom
                            26. 8. 1891 bis zum 23. 9. 1891.}}}\label{K_L00042_1h}
                    während meiner mehrmonatlichen Abweſenheit Sie nicht benachrichtigt hat, daß
                    Ihre Novelle »Der Sohn\pwindex{Schnitzler, Arthur 15.05.1862 – 21.10.1931@\textsc{Schnitzler, Arthur} (15.05.1862 – 21.10.1931), \emph{Schriftsteller, Mediziner}!Sohn. Aus den Papieren eines Arztes1. 1. 1892@\strich\emph{Der Sohn. Aus den Papieren eines Arztes} {[}1. 1. 1892{]}|pw}« von mir angenommen
                    worden war. Nur etwas warten muß ſie leider, das \label{K_L00042_2v}\edtext{Drama\pwindex{Wolzogen, Ernst von 23.04.1855 – 30.07.1934@\textsc{Wolzogen, Ernst von} (23.04.1855 – 30.07.1934), \emph{Schriftsteller}!Lumpengesindel1891-10-07 – 1891-10-30@\strich\emph{Das Lumpengesindel} {[}1891-10-07 – 1891-10-30{]}|pwv}}{\lemma{\textnormal{\emph{Drama}}}\Cendnote{\textnormal{Ernst von Wolzogen\pwindex{Wolzogen, Ernst von 23.04.1855 – 30.07.1934@\textsc{Wolzogen, Ernst von} (23.04.1855 – 30.07.1934), \emph{Schriftsteller}|pwk}: \emph{Das Lumpengesindel. Komödie in 5 Aufzügen}\pwindex{Wolzogen, Ernst von 23.04.1855 – 30.07.1934@\textsc{Wolzogen, Ernst von} (23.04.1855 – 30.07.1934), \emph{Schriftsteller}!Lumpengesindel1891-10-07 – 1891-10-30@\strich\emph{Das Lumpengesindel} {[}1891-10-07 – 1891-10-30{]}|pwk}. In: \emph{Freie Bühne für modernes Leben}\pwindex{Freie Buehne fuer modernes Leben1890 – 1891@\emph{Freie Bühne für modernes Leben} {[}1890 – 1891{]}|pwk}, Jg. 2,
                            H. 40–52, 7. 10. 1891 – 30. 10. 1891 (13
                            Teile).}}}\label{K_L00042_2h}, das wir jetzt abdrucken, ſchiebt alle Novellen
                    zurück.\pend
           \pstart
           Mit vorzüglicher Hochachtung{\\[\baselineskip]}\spacefill\mbox{Wilhelm Bölsche}\pend
           \leftskip=0em{}
         
         \endnumbering\mylabel{h}\end{ledgroupsized}  \newcommand{\dateiname}{L00042}\newcommand{\titel}{Wilhelm Bölsche an Arthur Schnitzler, 6. 10. 1891}\newcommand{\editorInnen}{Martin Anton Müller und Gerd-Hermann Susen}%% latex-leseansicht-abspann.tex
%% Abspann für die Leseansicht.
%% Der Schalter \ifkorrekturansicht ist bereits durch den Vorspann gesetzt.

%% latex-abspann.tex
%% Gemeinsamer Abspann für Korrekturansicht und Leseansicht.
%% Setzt den Schalter \ifkorrekturansicht voraus (gesetzt in den
%% einbindenden Dateien latex-korrekturansicht-abspann.tex bzw.
%% latex-leseansicht-abspann.tex).
%% ---------------------------------------------------------------

\normalsize

% Das esempio-Environment wird nur in der Leseansicht benötigt
\ifkorrekturansicht\else
\newenvironment{esempio}[3]%
{
    \vspace{1.5ex}
    \rlap{\underline{#1}}
    \par
    \setlength{\parindent}{0cm}
    \nopagebreak
    \leftskip=#2cm
    \rightskip=#3cm
}
{
    \par
}
\fi

\doendnotes{C}
\bigskip
\vfill

\clearpage

\footnotesize

\ifkorrekturansicht
  \lohead{\textsc{register}}
\fi

% theindex-Environment neu definieren ohne reledmac
\makeatletter
\renewenvironment{theindex}{%
  \ifkorrekturansicht
    \section*{\indexname}%
  \else
    \subsubsection*{Index der erwähnten Entitäten}%
  \fi
  \setlength{\parindent}{0pt}%
  \setlength{\parskip}{0pt plus 0.3pt}%
  \let\item\@idxitem
}{%
  \ifkorrekturansicht\clearpage\fi
}
\makeatother

\IfFileExists{\jobname-pw.ind}{\input{\jobname-pw.ind}}{}

% Quellenangabe nur in der Leseansicht
\ifkorrekturansicht\else
% Fallback-Definitionen, falls die .tex-Datei \titel etc. nicht gesetzt hat
\providecommand{\titel}{}
\providecommand{\editorInnen}{}
\providecommand{\dateiname}{\jobname}

\vspace{3cm}

\vfill

\footnotesize
\textsc{Quelle}: \titel. Herausgegeben von {\editorInnen}. In: \emph{Arthur Schnitzler: Briefwechsel mit Autorinnen und Autoren}.
 Digitale Edition, https://schnitzler-briefe.acdh.oeaw.ac.at/{\dateiname}.html (Stand \today)
\fi

\end{document}


      