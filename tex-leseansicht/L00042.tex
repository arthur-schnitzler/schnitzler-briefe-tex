%% latex-leseansicht-vorspann.tex
%% Vorspann für die Leseansicht.
%% Lädt die gemeinsame Datei latex-vorspann.tex mit nicht gesetztem Schalter.

\newif\ifkorrekturansicht
\korrekturansichtfalse

\input{../tex-inputs/latex-vorspann}


\section[Wilhelm Bölsche an Arthur Schnitzler, 6. 10. 1891]{L00042 Wilhelm Bölsche an Arthur Schnitzler, 6. 10. 1891}
\nopagebreak\mylabel{L00042v}
\rehead{ }\normalsize\beginnumbering\briefempfaengerindex{Schnitzler, Arthur@\textsc{Schnitzler, Arthur}!zzzBölsche, Wilhelm@\emph{von Wilhelm Bölsche}!1891-10-061@{6. 10. 1891}|(be}
\toendnotes[C]{\smallbreak\pagebreak[2]}
\correspDesc{Versand  durch Wilhelm Bölsche am 6. 10. 1891 in Berlin
\newline{}Erhalt  durch Arthur Schnitzler im Zeitraum [7. 10. 1891
                  – 11. 10. 1891?] in Wien}\toendnotes[C]{\smallbreak}
\Standort{DLA, A:Schnitzler, HS.NZ85.1.2577,2.}
\physDesc{Brief, 1 Blatt, 1 Seite, 393 Zeichen
\newline{}Handschrift: schwarze Tinte, deutsche Kurrent
\newline{}Ordnung: mit rotem Buntstift von unbekannter Hand nummeriert:
                                    »2« }
\buchAbdrucke{\weitereDrucke{Wilhelm Bölsche: \emph{Briefwechsel. Mit Autoren der Freien Bühne}. Herausgegeben von Gerd-Hermann Susen. Berlin: \emph{Weidler} 2010, S. 672 (Werke und Briefe. Wissenschaftliche Ausgabe, Briefe I).} }\toendnotes[C]{\smallbreak}
\pstart
           \raggedleft{}{\pb}Friedrichshagen\oindex{Friedrichshagen@\textbf{Friedrichshagen}, \emph{Ehemaliger Ort}|pw}{\\}b. Berlin\oindex{Berlin@\textbf{Berlin}, \emph{Hauptstadt}|pw}.{\\}Wilhelmſtr 72\oindex{Peter-Hille-Straße@\textbf{Peter-Hille-Straße}, \emph{Straße}|pw}.{\\}6. X. 91.\pend
           
\pstart\center{}Hochgeehrter Herr Doktor!\pend\vspace{0.5em}
\pstart
           Ich{ }ſehe eben mit Bedauern, daß \label{K_L00042-1v}\edtext{mein
                  Stellvertreter\pwindex{Hart, Julius 9.\,4.\,1859 Münster – 1930 Berlin@\textsc{Hart, Julius} (9.\,4.\,1859 Münster – 1930 Berlin), \emph{Schriftsteller, Journalist}|pwv}}{\lemma{\textnormal{\emph{mein
                  Stellvertreter}}}\Cendnote{\textnormal{Julius Hart\pwindex{Hart, Julius 9.\,4.\,1859 Münster – 1930 Berlin@\textsc{Hart, Julius} (9.\,4.\,1859 Münster – 1930 Berlin), \emph{Schriftsteller, Journalist}|pwk} betreute die Redaktion der \emph{Freien Bühne}\pwindex{Freie Bühne für modernes Leben@\emph{Freie Bühne für modernes Leben}|pwk} vom 26. 8. 1891 bis
                  zum 23. 9. 1891.}}}\label{K_L00042-1} während meiner mehrmonatlichen Abweſenheit
               Sie nicht benachrichtigt hat, daß Ihre Novelle »Der
                  Sohn\pwindex{Schnitzler, Arthur 15.\,5.\,1862 Wien – 21.\,10.\,1931 ebd.@\textsc{Schnitzler, Arthur} (15.\,5.\,1862 Wien – 21.\,10.\,1931 ebd.), \emph{Schriftsteller, Mediziner}!Sohn. Aus den Papieren eines Arztes@\strich\emph{Der Sohn. Aus den Papieren eines Arztes}|pw}« von mir angenommen worden war. Nur etwas warten muß{ }ſie leider, das
                  \label{K_L00042-2v}\edtext{Drama\pwindex{Wolzogen, Ernst von 23.\,4.\,1855 Breslau – 30.\,7.\,1934 Puppling@\textsc{Wolzogen, Ernst von} (23.\,4.\,1855 Breslau – 30.\,7.\,1934 Puppling), \emph{Schriftsteller}!Lumpengesindel@\strich\emph{Das Lumpengesindel}|pwv}}{\lemma{\textnormal{\emph{Drama}}}\Cendnote{\textnormal{Ernst von Wolzogen\pwindex{Wolzogen, Ernst von 23.\,4.\,1855 Breslau – 30.\,7.\,1934 Puppling@\textsc{Wolzogen, Ernst von} (23.\,4.\,1855 Breslau – 30.\,7.\,1934 Puppling), \emph{Schriftsteller}|pwk}: \emph{Das Lumpengesindel. Komödie in 5 Aufzügen}\pwindex{Wolzogen, Ernst von 23.\,4.\,1855 Breslau – 30.\,7.\,1934 Puppling@\textsc{Wolzogen, Ernst von} (23.\,4.\,1855 Breslau – 30.\,7.\,1934 Puppling), \emph{Schriftsteller}!Lumpengesindel@\strich\emph{Das Lumpengesindel}|pwk}. In: \emph{Freie Bühne für modernes Leben}\pwindex{Freie Bühne für modernes Leben@\emph{Freie Bühne für modernes Leben}|pwk}, Jg. 2,
                     H. 40–52, 7. 10. 1891 – 30. 10. 1891 (13
                     Teile).}}}\label{K_L00042-2}, das wir jetzt abdrucken,{ }ſchiebt alle Novellen
               zurück.\pend
           
\pstart
           Mit vorzüglicher Hochachtung{\\[\baselineskip]}\spacefill\mbox{Wilhelm Bölsche}\pend
           \leftskip=0em{}\selectlanguage{ngerman}\endnumbering\briefempfaengerindex{Schnitzler, Arthur@\textsc{Schnitzler, Arthur}!zzzBölsche, Wilhelm@\emph{von Wilhelm Bölsche}!1891-10-061@{6. 10. 1891}|)be}\mylabel{L00042h}  \newcommand{\dateiname}{L00042}\newcommand{\titel}{Wilhelm Bölsche an Arthur Schnitzler, 6. 10. 1891}\newcommand{\editorInnen}{Martin Anton Müller und Gerd-Hermann Susen}%% latex-leseansicht-abspann.tex
%% Abspann für die Leseansicht.
%% Der Schalter \ifkorrekturansicht ist bereits durch den Vorspann gesetzt.

%% latex-abspann.tex
%% Gemeinsamer Abspann für Korrekturansicht und Leseansicht.
%% Setzt den Schalter \ifkorrekturansicht voraus (gesetzt in den
%% einbindenden Dateien latex-korrekturansicht-abspann.tex bzw.
%% latex-leseansicht-abspann.tex).
%% ---------------------------------------------------------------

\normalsize

% Das esempio-Environment wird nur in der Leseansicht benötigt
\ifkorrekturansicht\else
\newenvironment{esempio}[3]%
{
    \vspace{1.5ex}
    \rlap{\underline{#1}}
    \par
    \setlength{\parindent}{0cm}
    \nopagebreak
    \leftskip=#2cm
    \rightskip=#3cm
}
{
    \par
}
\fi

\doendnotes{C}
\bigskip
\vfill

\clearpage

\footnotesize

\ifkorrekturansicht
  \lohead{\textsc{register}}
\fi

% theindex-Environment neu definieren ohne reledmac
\makeatletter
\renewenvironment{theindex}{%
  \ifkorrekturansicht
    \section*{\indexname}%
  \else
    \subsubsection*{Index der erwähnten Entitäten}%
  \fi
  \setlength{\parindent}{0pt}%
  \setlength{\parskip}{0pt plus 0.3pt}%
  \let\item\@idxitem
}{%
  \ifkorrekturansicht\clearpage\fi
}
\makeatother

\IfFileExists{\jobname-pw.ind}{\input{\jobname-pw.ind}}{}

% Quellenangabe nur in der Leseansicht
\ifkorrekturansicht\else
% Fallback-Definitionen, falls die .tex-Datei \titel etc. nicht gesetzt hat
\providecommand{\titel}{}
\providecommand{\editorInnen}{}
\providecommand{\dateiname}{\jobname}

\vspace{3cm}

\vfill

\footnotesize
\textsc{Quelle}: \titel. Herausgegeben von {\editorInnen}. In: \emph{Arthur Schnitzler: Briefwechsel mit Autorinnen und Autoren}.
 Digitale Edition, https://schnitzler-briefe.acdh.oeaw.ac.at/{\dateiname}.html (Stand \today)
\fi

\end{document}


