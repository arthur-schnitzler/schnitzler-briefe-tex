%% latex-leseansicht-vorspann.tex
%% Vorspann für die Leseansicht.
%% Lädt die gemeinsame Datei latex-vorspann.tex mit nicht gesetztem Schalter.

\newif\ifkorrekturansicht
\korrekturansichtfalse

\input{../tex-inputs/latex-vorspann}


\section[ Paul Goldmann an Arthur Schnitzler, 19. 1. [1898]]{L02836 Paul Goldmann an Arthur Schnitzler,  19. 1. [1898]}
\nopagebreak\mylabel{L02836v}
\rehead{ }\normalsize\beginnumbering\briefempfaengerindex{Schnitzler, Arthur@\textsc{Schnitzler, Arthur}!zzzGoldmann, Paul@\emph{von Paul Goldmann}!1898-01-191@{19. 1. [1898]}|(be}
\toendnotes[C]{\smallbreak\pagebreak[2]}
\correspDesc{Versand  durch Paul Goldmann am 19. 1. [1898] in Paris
\newline{}Erhalt  durch Arthur Schnitzler im Zeitraum [20. 1. 1898
                  – 24. 1. 1898?] in Wien}\toendnotes[C]{\smallbreak}
\Standort{DLA, A:Schnitzler, HS.NZ85.1.3168.}
\physDesc{Brief, 1 Blatt, 4 Seiten, 2177 Zeichen
\newline{}Handschrift: blaue Tinte, deutsche Kurrent
\newline{}Schnitzler: 1) mit Bleistift das Jahr »98« vermerkt  2) mit rotem Buntstift fünf Unterstreichungen}\toendnotes[C]{\smallbreak}
\pstart
           {\pb}\textcolor{gray}{\textbf{\textbf{Frankfurter Zeitung\orgindex{Frankfurter Zeitung@Frankfurter Zeitung|pw}}}}\pend
           
\pstart
           \textcolor{gray}{\textbf{(\begin{otherlanguage}{french}Gazette de Francfort\end{otherlanguage}\orgindex{Frankfurter Zeitung@Frankfurter Zeitung|pw}).}}\pend
           
\pstart
           \textcolor{gray}{\textbf{\textbf{\begin{otherlanguage}{french}Fondateur M.\end{otherlanguage}{ }L. Sonnemann\pwindex{Sonnemann, Leopold 29.\,10.\,1831 Höchberg – 30.\,10.\,1909 Frankfurt am Main@\textsc{Sonnemann, Leopold} (29.\,10.\,1831 Höchberg – 30.\,10.\,1909 Frankfurt am Main), \emph{Journalist, Herausgeber}|pw}.}}}\pend
           
\pstart
           \begin{otherlanguage}{french}\textcolor{gray}{\textbf{Journal politique, financier,}}\end{otherlanguage}\pend
           
\pstart
           \begin{otherlanguage}{french}\textcolor{gray}{\textbf{commercial et littéraire.}}\end{otherlanguage}\pend
           
\pstart
           \begin{otherlanguage}{french}\textcolor{gray}{\textbf{\textbf{Paraissant trois fois par jour.}}}\end{otherlanguage}\hfill \textsc{Paris\oindex{Paris@\textbf{Paris}, \emph{Hauptstadt}|pw}}, 19. Januar.\pend
           
\pstart
           \begin{otherlanguage}{french}\textcolor{gray}{\textbf{\textbf{Bureau à Paris\oindex{Paris@\textbf{Paris}, \emph{Hauptstadt}|pw}}}}\end{otherlanguage}\pend
           
\pstart
           \begin{otherlanguage}{french}\textcolor{gray}{\textbf{\textbf{10 \so{Rue de la Bourse}\oindex{rue de la Bourse@\textbf{rue de la Bourse}, \emph{Straße}|pw}.}}}\end{otherlanguage}\pend
           
\pstart\center{}Mein lieber Freund,\pend\vspace{0.5em}
\pstart
           Ich kann Dir nur in aller Kürze für Deinen lieben Brief danken; denn ich habe
               unmenſchlich viel zu thun.\pend
           
\pstart
           Mein Schwager\pwindex{Rosengart, Josef 8.\,2.\,1860 Laupheim – 4.\,8.\,1927 Frankfurt am Main@\textsc{Rosengart, Josef} (8.\,2.\,1860 Laupheim – 4.\,8.\,1927 Frankfurt am Main), \emph{Arzt}|pwv} hat die
               verrückte Idee gehabt, ich könnte \label{K_L02836-1v}\edtext{\textsc{Schlenthers\pwindex{Schlenther, Paul 20.\,8.\,1854 Chernyakhovsk – 30.\,4.\,1916 Berlin@\textsc{Schlenther, Paul} (20.\,8.\,1854 Chernyakhovsk – 30.\,4.\,1916 Berlin), \emph{Schriftsteller, Kritiker, Theaterleiter}|pw}} Nachfolger bei der Voſſiſchen Ztg.\orgindex{Vossische Zeitung@Vossische Zeitung|pw}}{\lemma{\textnormal{\emph{Schlenthers … Ztg.}}}\Cendnote{\textnormal{Paul Schlenther\pwindex{Schlenther, Paul 20.\,8.\,1854 Chernyakhovsk – 30.\,4.\,1916 Berlin@\textsc{Schlenther, Paul} (20.\,8.\,1854 Chernyakhovsk – 30.\,4.\,1916 Berlin), \emph{Schriftsteller, Kritiker, Theaterleiter}|pwk} war von 1886 bis 1898, als Theodor Fontanes\pwindex{Fontane, Theodor 30.\,12.\,1819 Neuruppin – 20.\,9.\,1898 Berlin@\textsc{Fontane, Theodor} (30.\,12.\,1819 Neuruppin – 20.\,9.\,1898 Berlin), \emph{Schriftsteller, Kritiker, Apotheker}|pwk} Nachfolger, Theaterkritiker der \emph{Vossischen Zeitung}\orgindex{Vossische Zeitung@Vossische Zeitung|pwk}. Danach, bis 1910, war er Direktor des \emph{Burgtheaters}\orgindex{Burgtheater@Burgtheater|pwk}. Siehe auch XXXX Auszeichnungsfehler: Dokument L02837 nicht gefunden.}}}\label{K_L02836-1} werden, und ich glaube, man hat{ }ſogar Dich in der Angelegenheit \label{K_L02836-2v}\edtext{beläſtigt}{\lemma{\textnormal{\emph{belästigt}}}\Cendnote{\textnormal{Siehe XXXX Auszeichnungsfehler: Dokument L02797 nicht gefunden.
               }}}\label{K_L02836-2}. Sei nicht böſe deßwegen!\pend
           
\pstart
           Von meinen Projecten für die nächſte Zukunft{ }ſteht die Reiſe nach \textsc{China\oindex{China@\textbf{China}|pw}} im Vordergrund. Es wäre gar herrlich, in \textsc{Wien\oindex{Wien@\textbf{Wien}, \emph{Verwaltungsgebiet}|pw}} wieder mit Euch zu leben. Aber denke an den Sumpf des Wien\oindex{Wien@\textbf{Wien}, \emph{Verwaltungsgebiet}|pw}er Journalismus. {\pb}Was{ }ſoll
               ich da machen? Was kann ich dort werden? Das iſt ein Boden, auf welchem Sumpfplanzen
               wie \textsc{Bahr\pwindex{Bahr, Hermann 19.\,7.\,1863 Linz – 15.\,1.\,1934 München@\textsc{Bahr, Hermann} (19.\,7.\,1863 Linz – 15.\,1.\,1934 München), \emph{Schriftsteller, Kritiker}|pw}} gedeihen, nicht ich. Da heißt es,{ }ſeine Sehnſucht bezwingen und{ }ſtark{ }ſein.\pend
           
\pstart
           Ich lernte hier den \textsc{Prof. Singer\pwindex{Singer, Isidor 16.\,1.\,1857 Budapest – 8.\,12.\,1927 Wien@\textsc{Singer, Isidor} (16.\,1.\,1857 Budapest – 8.\,12.\,1927 Wien), \emph{Journalist, Herausgeber, Soziologe}|pw}} kennen. Braver Mann. Aber durchaus unkünſtleriſch und auch unperſönlich; iſt
               ganz von \textsc{Kanner\pwindex{Kanner, Heinrich 9.\,11.\,1864 Galați – 15.\,2.\,1930 Wien@\textsc{Kanner, Heinrich} (9.\,11.\,1864 Galați – 15.\,2.\,1930 Wien), \emph{Herausgeber, Publizist}|pw}} hypnotiſirt; und iſt{ }ſchon{ }ſehr »Zeitung\pwindex{Zeit. Wiener Wochenschrift@\emph{Die Zeit. Wiener Wochenschrift}|pwv}s-Herausgeber«, welcher durchdrungen davon iſt, daß
               die »Zeit\orgindex{Zeit. Wiener Wochenschrift@Die Zeit. Wiener Wochenschrift|pw}« Öſterreich\oindex{Österreich@\textbf{Österreich}|pw} und auch ein wenig die Welt regiert.\pend
           
\pstart
           Wie{ }ſtehts mit »\label{K_L02836-3v}\edtext{Freiwild\pwindex{Schnitzler, Arthur 15.\,5.\,1862 Wien – 21.\,10.\,1931 ebd.@\textsc{Schnitzler, Arthur} (15.\,5.\,1862 Wien – 21.\,10.\,1931 ebd.), \emph{Schriftsteller, Mediziner}!Freiwild. Schauspiel in 3 Akten@\strich\emph{Freiwild. Schauspiel in 3 Akten}|pw}}{\lemma{\textnormal{\emph{Freiwild}}}\Cendnote{\textnormal{Zu diesem Zeitpunkt liefen
                  Vorbereitungen für die bevorstehende Premiere von \emph{Freiwild}\pwindex{Schnitzler, Arthur 15.\,5.\,1862 Wien – 21.\,10.\,1931 ebd.@\textsc{Schnitzler, Arthur} (15.\,5.\,1862 Wien – 21.\,10.\,1931 ebd.), \emph{Schriftsteller, Mediziner}!Freiwild. Schauspiel in 3 Akten@\strich\emph{Freiwild. Schauspiel in 3 Akten}|pwk}\eventindex{Carl-Theater@\textbf{Carl-Theater}!Premiere von Freiwild, 4.2.1898@Premiere von Freiwild, 4.2.1898|pwk} im Wien\oindex{Wien@\textbf{Wien}, \emph{Verwaltungsgebiet}|pwk}er \emph{Carl-Theater}\orgindex{Carl-Theater@Carl-Theater|pwk} am 4. 2. 1898.}}}\label{K_L02836-3}« und Deinem \label{K_L02836-4v}\edtext{neuen Stück\pwindex{Schnitzler, Arthur 15.\,5.\,1862 Wien – 21.\,10.\,1931 ebd.@\textsc{Schnitzler, Arthur} (15.\,5.\,1862 Wien – 21.\,10.\,1931 ebd.), \emph{Schriftsteller, Mediziner}!Vermächtnis. Schauspiel in drei Akten@\strich\emph{Das Vermächtnis. Schauspiel in drei Akten}|pwv}}{\lemma{\textnormal{\emph{neuen Stück}}}\Cendnote{\textnormal{Schnitzler las Max Burckhard\pwindex{Burckhard, Max Eugen 14.\,7.\,1854 Korneuburg – 16.\,3.\,1912 Wien@\textsc{Burckhard, Max Eugen} (14.\,7.\,1854 Korneuburg – 16.\,3.\,1912 Wien), \emph{Schriftsteller, Rechtswissenschaftler, Theaterleiter}|pwk} sein Schauspiel \emph{Das Vermächtnis}\pwindex{Schnitzler, Arthur 15.\,5.\,1862 Wien – 21.\,10.\,1931 ebd.@\textsc{Schnitzler, Arthur} (15.\,5.\,1862 Wien – 21.\,10.\,1931 ebd.), \emph{Schriftsteller, Mediziner}!Vermächtnis. Schauspiel in drei Akten@\strich\emph{Das Vermächtnis. Schauspiel in drei Akten}|pwk} am 27. 12. 1897 vor und schickte es ihm in Folge. Burckhard\pwindex{Burckhard, Max Eugen 14.\,7.\,1854 Korneuburg – 16.\,3.\,1912 Wien@\textsc{Burckhard, Max Eugen} (14.\,7.\,1854 Korneuburg – 16.\,3.\,1912 Wien), \emph{Schriftsteller, Rechtswissenschaftler, Theaterleiter}|pwk} gefiel das Stück\pwindex{Schnitzler, Arthur 15.\,5.\,1862 Wien – 21.\,10.\,1931 ebd.@\textsc{Schnitzler, Arthur} (15.\,5.\,1862 Wien – 21.\,10.\,1931 ebd.), \emph{Schriftsteller, Mediziner}!Vermächtnis. Schauspiel in drei Akten@\strich\emph{Das Vermächtnis. Schauspiel in drei Akten}|pwkv} und er wollte es gleich in der
                  nächsten Saison auf die Bühne\orgindex{Burgtheater@Burgtheater|pwkv} bringen. Mit dem neuen Direktor\orgindex{Burgtheater@Burgtheater|pwkv}{ }Paul Schlenther\pwindex{Schlenther, Paul 20.\,8.\,1854 Chernyakhovsk – 30.\,4.\,1916 Berlin@\textsc{Schlenther, Paul} (20.\,8.\,1854 Chernyakhovsk – 30.\,4.\,1916 Berlin), \emph{Schriftsteller, Kritiker, Theaterleiter}|pwk} kam es jedoch zu einer
                  Verschiebung (vgl. A. S.: \emph{Tagebuch}, 13. 2. 1898),
                  wodurch \emph{Das Vermächtnis}\pwindex{Schnitzler, Arthur 15.\,5.\,1862 Wien – 21.\,10.\,1931 ebd.@\textsc{Schnitzler, Arthur} (15.\,5.\,1862 Wien – 21.\,10.\,1931 ebd.), \emph{Schriftsteller, Mediziner}!Vermächtnis. Schauspiel in drei Akten@\strich\emph{Das Vermächtnis. Schauspiel in drei Akten}|pwk} die Uraufführung\eventindex{Deutsches Theater Berlin@\textbf{Deutsches Theater Berlin}!Uraufführung von Das Vermächtnis, 8.10.1898@Uraufführung von Das Vermächtnis, 8.10.1898|pwkv} in
                  Berlin\oindex{Berlin@\textbf{Berlin}, \emph{Hauptstadt}|pwk} hatte und erst am 31. 5. 1899 am Wien\oindex{Wien@\textbf{Wien}, \emph{Verwaltungsgebiet}|pwk}er Burgtheater\oindex{Wien@\textbf{Wien}!I., Innere Stadt@\textbf{I., Innere Stadt}!Burgtheater@\textbf{Burgtheater}, \emph{Theater}|pwk}{ }aufgeführt\eventindex{Burgtheater@\textbf{Burgtheater}!Aufführung von Das Vermächtnis, 31.5.1899@Aufführung von Das Vermächtnis, 31.5.1899|pwkv} wurde.}}}\label{K_L02836-4}? \textsc{Schlenthers\pwindex{Schlenther, Paul 20.\,8.\,1854 Chernyakhovsk – 30.\,4.\,1916 Berlin@\textsc{Schlenther, Paul} (20.\,8.\,1854 Chernyakhovsk – 30.\,4.\,1916 Berlin), \emph{Schriftsteller, Kritiker, Theaterleiter}|pw}}{ }Amtsantritt\orgindex{Burgtheater@Burgtheater|pwv} ändert natürlich
               nichts an der Thatſache, daß Dein Stück\pwindex{Schnitzler, Arthur 15.\,5.\,1862 Wien – 21.\,10.\,1931 ebd.@\textsc{Schnitzler, Arthur} (15.\,5.\,1862 Wien – 21.\,10.\,1931 ebd.), \emph{Schriftsteller, Mediziner}!Vermächtnis. Schauspiel in drei Akten@\strich\emph{Das Vermächtnis. Schauspiel in drei Akten}|pwv} bald geſpielt wird? {\dotsfive}\pend
           
\pstart
           {\pb}Mit dem kleinen \label{K_L02836-5v}\edtext{Fräulein\pwindex{Ziegler, Alice 5.\,1.\,1880 Prag – Dezember 1943 Konzentrationslager Auschwitz-Birkenau@\textsc{Ziegler, Alice} (5.\,1.\,1880 Prag – Dezember 1943 Konzentrationslager Auschwitz-Birkenau)|pwv} in \textsc{Prag\oindex{Prag@\textbf{Prag}, \emph{Land}|pw}}}{\lemma{\textnormal{\emph{Fräulein in Prag}}}\Cendnote{\textnormal{Siehe XXXX Auszeichnungsfehler: Dokument L02831 nicht gefunden.
               }}}\label{K_L02836-5} hat die Sache ein jähes Ende genommen. Ich bekam ihre Photographie. Ich war
               gerade{ }ſehr einſam und das Bild war{ }ſehr lieb. Das ging mir tief zu Herzen, und ich
               machte einige Verſe. Seit ich dieſelben abgeſandt, iſt die Correſpondenz abgebrochen.
               Das thut mir{ }ſehr weh, \strikeout{\textcolor{gray}{v}} vor Allem wegen des Affronts, der darin liegt. Ich{ }ſende Dir \label{K_L02836-6v}\edtext{anbei die Verſe}{\lemma{\textnormal{\emph{anbei die Verse}}}\Cendnote{\textnormal{Beilage nicht erhalten}}}\label{K_L02836-6}. Es iſt jetzt hier{ }ſo viel von
               Sachverſtändigen die Rede; ich rufe Dich als \textsc{Experten} an,
               und Du{ }ſollſt mir{ }ſagen, ob das, was ich da geſchrieben habe, verletzend oder taktlos
               iſt. Bitte,{ }ſende mir die Verſe zurück. Ich komme mir recht ekelhaft vor, daß ich{ }ſo
               mein volles Herz zu Markte trage und es einer Jeden anbiete. Aber ich habe ein{ }ſolches {\pb}Bedürfniß nach Zärtlichkeit, welches das
               Leben mir noch nicht ein einziges Mal befriedigt hat. Überall werde ich
               zurückgeſtoßen und bleibe einſam und voll unerfüllter Sehnſucht. \label{K_L02836-7v}\edtext{\begin{otherlanguage}{french}\textsc{Raté}\end{otherlanguage}}{\lemma{\textnormal{\emph{Raté}}}\Cendnote{\textnormal{französisch: Versager}}}\label{K_L02836-7} auch hier,
               erſt recht hier. Kurzum, ich will nach \textsc{China\oindex{China@\textbf{China}|pw}}.\pend
           
\pstart
           Grüß’ Dich Gott, liebſter Freund! Schreib’ mir bald\textcolor{gray}{!}\pend
           
\pstart
           Dein treuer {\\[\baselineskip]}\spacefill\mbox{Paul Goldmann}\pend
           \leftskip=0em{}
\pstart
           \noindent{}Viele Grüße an Deine Freundin\pwindex{Reinhard, Marie 13.\,3.\,1871 Wien – 18.\,3.\,1899 ebd.@\textsc{Reinhard, Marie} (13.\,3.\,1871 Wien – 18.\,3.\,1899 ebd.), \emph{Gesangspädagogin}|pwv}!\pend
           \selectlanguage{ngerman}\endnumbering\briefempfaengerindex{Schnitzler, Arthur@\textsc{Schnitzler, Arthur}!zzzGoldmann, Paul@\emph{von Paul Goldmann}!1898-01-191@{19. 1. [1898]}|)be}\mylabel{L02836h}  \newcommand{\dateiname}{L02836}\newcommand{\titel}{Paul Goldmann an Arthur Schnitzler, 19. 1. [1898]}\newcommand{\editorInnen}{Martin Anton Müller und Laura Untner}%% latex-leseansicht-abspann.tex
%% Abspann für die Leseansicht.
%% Der Schalter \ifkorrekturansicht ist bereits durch den Vorspann gesetzt.

%% latex-abspann.tex
%% Gemeinsamer Abspann für Korrekturansicht und Leseansicht.
%% Setzt den Schalter \ifkorrekturansicht voraus (gesetzt in den
%% einbindenden Dateien latex-korrekturansicht-abspann.tex bzw.
%% latex-leseansicht-abspann.tex).
%% ---------------------------------------------------------------

\normalsize

% Das esempio-Environment wird nur in der Leseansicht benötigt
\ifkorrekturansicht\else
\newenvironment{esempio}[3]%
{
    \vspace{1.5ex}
    \rlap{\underline{#1}}
    \par
    \setlength{\parindent}{0cm}
    \nopagebreak
    \leftskip=#2cm
    \rightskip=#3cm
}
{
    \par
}
\fi

\doendnotes{C}
\bigskip
\vfill

\clearpage

\footnotesize

\ifkorrekturansicht
  \lohead{\textsc{register}}
\fi

% theindex-Environment neu definieren ohne reledmac
\makeatletter
\renewenvironment{theindex}{%
  \ifkorrekturansicht
    \section*{\indexname}%
  \else
    \subsubsection*{Index der erwähnten Entitäten}%
  \fi
  \setlength{\parindent}{0pt}%
  \setlength{\parskip}{0pt plus 0.3pt}%
  \let\item\@idxitem
}{%
  \ifkorrekturansicht\clearpage\fi
}
\makeatother

\IfFileExists{\jobname-pw.ind}{\input{\jobname-pw.ind}}{}

% Quellenangabe nur in der Leseansicht
\ifkorrekturansicht\else
% Fallback-Definitionen, falls die .tex-Datei \titel etc. nicht gesetzt hat
\providecommand{\titel}{}
\providecommand{\editorInnen}{}
\providecommand{\dateiname}{\jobname}

\vspace{3cm}

\vfill

\footnotesize
\textsc{Quelle}: \titel. Herausgegeben von {\editorInnen}. In: \emph{Arthur Schnitzler: Briefwechsel mit Autorinnen und Autoren}.
 Digitale Edition, https://schnitzler-briefe.acdh.oeaw.ac.at/{\dateiname}.html (Stand \today)
\fi

\end{document}


