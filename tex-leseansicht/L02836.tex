%% latex-leseansicht-vorspann.tex
%% Vorspann für die Leseansicht.
%% Lädt die gemeinsame Datei latex-vorspann.tex mit nicht gesetztem Schalter.

\newif\ifkorrekturansicht
\korrekturansichtfalse

\input{../tex-inputs/latex-vorspann}


         
         \renewcommand{\erwaehntePersonen}{Personen: Hermann Bahr, Max Eugen Burckhard, Theodor Fontane, Paul Goldmann, Heinrich Kanner, Marie Reinhard, Josef Rosengart, Paul Schlenther, Isidor Singer, Leopold Sonnemann, Alice Ziegler}
         \renewcommand{\erwaehnteInstitutionen}{Institutionen: Burgtheater, Carl-Theater, Die Zeit. Wiener Wochenschrift, Frankfurter Zeitung, Vossische Zeitung}
         \renewcommand{\erwaehnteOrte}{Orte: Berlin, Burgtheater, China, Paris, Prag, Wien, rue de la Bourse, Österreich}
         \renewcommand{\erwaehnteWerke}{Werke: Das Vermächtnis. Schauspiel in drei Akten, Die Zeit. Wiener Wochenschrift, Freiwild. Schauspiel in 3 Akten}
               \section[ Paul Goldmann an Arthur Schnitzler, 19. 1. {[}1898{]}]{ Paul Goldmann an Arthur Schnitzler, 19. 1. {[}1898{]}}\nopagebreak\mylabel{v}\rehead{ }\begin{ledgroupsized}[t]{13cm}\normalsize\beginnumbering\briefempfaengerindex{Schnitzler, Arthur@\textsc{Schnitzler, Arthur}!zzzGoldmann, Paul@\emph{von Paul Goldmann}!1898-01-191@{19. 1. {[}1898{]}}|(be} \toendnotes[C]{\smallbreak\pagebreak[2]} \Standort{DLA, A:Schnitzler, HS.NZ85.1.3168.}
\physDesc{Brief, 1 Blatt, 4 Seiten, 2177 Zeichen
\newline{}Handschrift: blaue Tinte, deutsche Kurrent
\newline{}Schnitzler: 1) mit Bleistift das Jahr »98« vermerkt  2) mit rotem Buntstift fünf Unterstreichungen}\toendnotes[C]{\smallbreak}\pstart
           \noindent{}{\pb}\textcolor{gray}{\textbf{\textbf{Frankfurter Zeitung\orgindex{Frankfurter Zeitung@Frankfurter Zeitung|pw}}}}\pend
           \pstart
           \textcolor{gray}{\textbf{(\begin{otherlanguage}{french}Gazette de Francfort\end{otherlanguage}\orgindex{Frankfurter Zeitung@Frankfurter Zeitung|pw}).}}\pend
           \pstart
           \textcolor{gray}{\textbf{\textbf{\begin{otherlanguage}{french}Fondateur M.\end{otherlanguage}{ }L. Sonnemann\pwindex{Sonnemann, Leopold 1831-10-29 – 1909-10-30@\textsc{Sonnemann, Leopold} (1831-10-29 – 1909-10-30), \emph{Journalist, Herausgeber}|pw}.}}}\pend
           \pstart
           \begin{otherlanguage}{french}\textcolor{gray}{\textbf{Journal politique, financier,}}\end{otherlanguage}\pend
           \pstart
           \begin{otherlanguage}{french}\textcolor{gray}{\textbf{commercial et littéraire.}}\end{otherlanguage}\pend
           \pstart
           \begin{otherlanguage}{french}\textcolor{gray}{\textbf{\textbf{Paraissant trois fois par jour.}}}\end{otherlanguage}\hfill \textsc{Paris\oindex{Paris@\textbf{Paris}|pw}}, 19. Januar.\pend
           \pstart
           \begin{otherlanguage}{french}\textcolor{gray}{\textbf{\textbf{Bureau à Paris\oindex{Paris@\textbf{Paris}|pw}}}}\end{otherlanguage}\pend
           \pstart
           \begin{otherlanguage}{french}\textcolor{gray}{\textbf{\textbf{10 \so{Rue de la Bourse}\oindex{rue de la Bourse@\textbf{rue de la Bourse}|pw}.}}}\end{otherlanguage}\pend
           \pstart\center{}Mein lieber Freund,\pend\pstart
           Ich kann Dir nur in aller Kürze für Deinen lieben Brief danken; denn ich habe
               unmenſchlich viel zu thun.\pend
           \pstart
           Mein Schwager\pwindex{Rosengart, Josef 1860-02-08 – 1927-08-04@\textsc{Rosengart, Josef} (1860-02-08 – 1927-08-04), \emph{Arzt}|pwv} hat die
               verrückte Idee gehabt, ich könnte \label{K_L02836-1v}\edtext{\textsc{Schlenthers\pwindex{Schlenther, Paul 20.08.1854 – 30.04.1916@\textsc{Schlenther, Paul} (20.08.1854 – 30.04.1916), \emph{Schriftsteller, Kritiker, Theaterleiter}|pw}} Nachfolger bei der Voſſiſchen Ztg.\orgindex{Vossische Zeitung@Vossische Zeitung|pw}}{\lemma{\textnormal{\emph{Schlenthers … Ztg.}}}\Cendnote{\textnormal{Paul Schlenther\pwindex{Schlenther, Paul 20.08.1854 – 30.04.1916@\textsc{Schlenther, Paul} (20.08.1854 – 30.04.1916), \emph{Schriftsteller, Kritiker, Theaterleiter}|pwk} war von 1886 bis 1898, als Theodor Fontanes\pwindex{Fontane, Theodor 30.12.1819 – 20.09.1898@\textsc{Fontane, Theodor} (30.12.1819 – 20.09.1898), \emph{Schriftsteller, Kritiker, Apotheker}|pwk} Nachfolger, Theaterkritiker der \emph{Vossischen Zeitung}\orgindex{Vossische Zeitung@Vossische Zeitung|pwk}. Danach, bis 1910, war er Direktor des \emph{Burgtheaters}\orgindex{Burgtheater@Burgtheater|pwk}. Siehe auch Paul Goldmann an Arthur Schnitzler, 26. 1. [1898].}}}\label{K_L02836-1h} werden, und ich glaube, man hat ſogar Dich in der Angelegenheit \label{K_L02836-2v}\edtext{beläſtigt}{\lemma{\textnormal{\emph{beläſtigt}}}\Cendnote{\textnormal{Siehe Vally Rosengart an Arthur Schnitzler, [16. 1. 1898].
               }}}\label{K_L02836-2h}. Sei nicht böſe deßwegen!\pend
           \pstart
           Von meinen Projecten für die nächſte Zukunft ſteht die Reiſe nach \textsc{China\oindex{China@\textbf{China}|pw}} im Vordergrund. Es wäre gar herrlich, in \textsc{Wien\oindex{Wien@\textbf{Wien}|pw}} wieder mit Euch zu leben. Aber denke an den Sumpf des Wien\oindex{Wien@\textbf{Wien}|pw}er Journalismus. {\pb}Was ſoll
               ich da machen? Was kann ich dort werden? Das iſt ein Boden, auf welchem Sumpfplanzen
               wie \textsc{Bahr\pwindex{Bahr, Hermann 19.07.1863 – 15.01.1934@\textsc{Bahr, Hermann} (19.07.1863 – 15.01.1934), \emph{Schriftsteller, Kritiker}|pw}} gedeihen, nicht ich. Da heißt es, ſeine Sehnſucht bezwingen und ſtark ſein.\pend
           \pstart
           Ich lernte hier den \textsc{Prof. Singer\pwindex{Singer, Isidor 16.01.1857 – 08.12.1927@\textsc{Singer, Isidor} (16.01.1857 – 08.12.1927), \emph{Journalist, Herausgeber, Soziologe}|pw}} kennen. Braver Mann. Aber durchaus unkünſtleriſch und auch unperſönlich; iſt
               ganz von \textsc{Kanner\pwindex{Kanner, Heinrich 09.11.1864 – 15.02.1930@\textsc{Kanner, Heinrich} (09.11.1864 – 15.02.1930), \emph{Herausgeber, Publizist}|pw}} hypnotiſirt; und iſt ſchon ſehr »Zeitung\pwindex{Zeit. Wiener Wochenschrift1894 – 1904@\emph{Die Zeit. Wiener Wochenschrift} {[}1894 – 1904{]}|pwv}s-Herausgeber«, welcher durchdrungen davon iſt, daß
               die »Zeit\orgindex{Zeit. Wiener Wochenschrift@Die Zeit. Wiener Wochenschrift|pw}« Öſterreich\oindex{Oesterreich@\textbf{Österreich}|pw} und auch ein wenig die Welt regiert.\pend
           \pstart
           Wie ſtehts mit »\label{K_L02836-3v}\edtext{Freiwild\pwindex{Schnitzler, Arthur 15.05.1862 – 21.10.1931@\textsc{Schnitzler, Arthur} (15.05.1862 – 21.10.1931), \emph{Schriftsteller, Mediziner}!Freiwild. Schauspiel in 3 Akten1896@\strich\emph{Freiwild. Schauspiel in 3 Akten} {[}1896{]}|pw}}{\lemma{\textnormal{\emph{Freiwild}}}\Cendnote{\textnormal{Zu diesem Zeitpunkt liefen
                  Vorbereitungen für die bevorstehende Premiere von \emph{Freiwild}\pwindex{Schnitzler, Arthur 15.05.1862 – 21.10.1931@\textsc{Schnitzler, Arthur} (15.05.1862 – 21.10.1931), \emph{Schriftsteller, Mediziner}!Freiwild. Schauspiel in 3 Akten1896@\strich\emph{Freiwild. Schauspiel in 3 Akten} {[}1896{]}|pwk} im Wien\oindex{Wien@\textbf{Wien}|pwk}er \emph{Carl-Theater}\orgindex{Carl-Theater@Carl-Theater|pwk} am 4. 2. 1898.}}}\label{K_L02836-3h}« und Deinem \label{K_L02836-4v}\edtext{neuen Stück\pwindex{Schnitzler, Arthur 15.05.1862 – 21.10.1931@\textsc{Schnitzler, Arthur} (15.05.1862 – 21.10.1931), \emph{Schriftsteller, Mediziner}!Vermaechtnis. Schauspiel in drei Akten1898@\strich\emph{Das Vermächtnis. Schauspiel in drei Akten} {[}1898{]}|pwv}}{\lemma{\textnormal{\emph{neuen Stück}}}\Cendnote{\textnormal{Schnitzler\pwindex{Schnitzler, Arthur 15.05.1862 – 21.10.1931@\textsc{Schnitzler, Arthur} (15.05.1862 – 21.10.1931), \emph{Schriftsteller, Mediziner}|pwk} las Max Burckhard\pwindex{Burckhard, Max Eugen 14.07.1854 – 16.03.1912@\textsc{Burckhard, Max Eugen} (14.07.1854 – 16.03.1912), \emph{Schriftsteller, Wissenschaftler, Theaterleiter}|pwk} sein Schauspiel \emph{Das Vermächtnis}\pwindex{Schnitzler, Arthur 15.05.1862 – 21.10.1931@\textsc{Schnitzler, Arthur} (15.05.1862 – 21.10.1931), \emph{Schriftsteller, Mediziner}!Vermaechtnis. Schauspiel in drei Akten1898@\strich\emph{Das Vermächtnis. Schauspiel in drei Akten} {[}1898{]}|pwk} am 27. 12. 1897 vor und schickte es ihm in Folge. Burckhard\pwindex{Burckhard, Max Eugen 14.07.1854 – 16.03.1912@\textsc{Burckhard, Max Eugen} (14.07.1854 – 16.03.1912), \emph{Schriftsteller, Wissenschaftler, Theaterleiter}|pwk} gefiel das Stück\pwindex{Schnitzler, Arthur 15.05.1862 – 21.10.1931@\textsc{Schnitzler, Arthur} (15.05.1862 – 21.10.1931), \emph{Schriftsteller, Mediziner}!Vermaechtnis. Schauspiel in drei Akten1898@\strich\emph{Das Vermächtnis. Schauspiel in drei Akten} {[}1898{]}|pwkv} und er wollte es gleich in der
                  nächsten Saison auf die Bühne\orgindex{Burgtheater@Burgtheater|pwkv} bringen. Mit dem neuen Direktor\orgindex{Burgtheater@Burgtheater|pwkv}{ }Paul Schlenther\pwindex{Schlenther, Paul 20.08.1854 – 30.04.1916@\textsc{Schlenther, Paul} (20.08.1854 – 30.04.1916), \emph{Schriftsteller, Kritiker, Theaterleiter}|pwk} kam es jedoch zu einer
                  Verschiebung (vgl. A. S.: \emph{Tagebuch}, 13. 2. 1898),
                  wodurch \emph{Das Vermächtnis}\pwindex{Schnitzler, Arthur 15.05.1862 – 21.10.1931@\textsc{Schnitzler, Arthur} (15.05.1862 – 21.10.1931), \emph{Schriftsteller, Mediziner}!Vermaechtnis. Schauspiel in drei Akten1898@\strich\emph{Das Vermächtnis. Schauspiel in drei Akten} {[}1898{]}|pwk} die Uraufführung in
                     Berlin\oindex{Berlin@\textbf{Berlin}|pwk} hatte und erst am 31. 5. 1899 am Wien\oindex{Wien@\textbf{Wien}|pwk}er Burgtheater\oindex{Burgtheater@\textbf{Burgtheater}|pwk} aufgeführt wurde.}}}\label{K_L02836-4h}? \textsc{Schlenthers\pwindex{Schlenther, Paul 20.08.1854 – 30.04.1916@\textsc{Schlenther, Paul} (20.08.1854 – 30.04.1916), \emph{Schriftsteller, Kritiker, Theaterleiter}|pw}}{ }Amtsantritt\orgindex{Burgtheater@Burgtheater|pwv} ändert natürlich
               nichts an der Thatſache, daß Dein Stück\pwindex{Schnitzler, Arthur 15.05.1862 – 21.10.1931@\textsc{Schnitzler, Arthur} (15.05.1862 – 21.10.1931), \emph{Schriftsteller, Mediziner}!Vermaechtnis. Schauspiel in drei Akten1898@\strich\emph{Das Vermächtnis. Schauspiel in drei Akten} {[}1898{]}|pwv} bald geſpielt wird? {\dotsfive}\pend
           \pstart
           {\pb}Mit dem kleinen \label{K_L02836-5v}\edtext{Fräulein\pwindex{Ziegler, Alice 1880-01-05 – Dezember 1943@\textsc{Ziegler, Alice} (1880-01-05 – Dezember 1943)|pwv} in \textsc{Prag\oindex{Prag@\textbf{Prag}|pw}}}{\lemma{\textnormal{\emph{Fräulein in Prag}}}\Cendnote{\textnormal{Siehe Paul Goldmann an Arthur Schnitzler, 19. 11. [1897].
               }}}\label{K_L02836-5h} hat die Sache ein jähes Ende genommen. Ich bekam ihre Photographie. Ich war
               gerade ſehr einſam und das Bild war ſehr lieb. Das ging mir tief zu Herzen, und ich
               machte einige Verſe. Seit ich dieſelben abgeſandt, iſt die Correſpondenz abgebrochen.
               Das thut mir ſehr weh, \strikeout{\textcolor{gray}{v}} vor Allem wegen des Affronts, der darin liegt. Ich ſende Dir \label{K_L02836-6v}\edtext{anbei die Verſe}{\lemma{\textnormal{\emph{anbei die Verſe}}}\Cendnote{\textnormal{Beilage nicht erhalten}}}\label{K_L02836-6h}. Es iſt jetzt hier ſo viel von
               Sachverſtändigen die Rede; ich rufe Dich als \textsc{Experten} an,
               und Du ſollſt mir ſagen, ob das, was ich da geſchrieben habe, verletzend oder taktlos
               iſt. Bitte, ſende mir die Verſe zurück. Ich komme mir recht ekelhaft vor, daß ich ſo
               mein volles Herz zu Markte trage und es einer Jeden anbiete. Aber ich habe ein
               ſolches {\pb}Bedürfniß nach Zärtlichkeit, welches das
               Leben mir noch nicht ein einziges Mal befriedigt hat. Überall werde ich
               zurückgeſtoßen und bleibe einſam und voll unerfüllter Sehnſucht. \label{K_L02836-7v}\edtext{\begin{otherlanguage}{french}\textsc{Raté}\end{otherlanguage}}{\lemma{\textnormal{\emph{Raté}}}\Cendnote{\textnormal{französisch: Versager}}}\label{K_L02836-7h} auch hier,
               erſt recht hier. Kurzum, ich will nach \textsc{China\oindex{China@\textbf{China}|pw}}.\pend
           \pstart
           Grüß’ Dich Gott, liebſter Freund! Schreib’ mir bald\textcolor{gray}{!}\pend
           \pstart
           Dein treuer {\\[\baselineskip]}\spacefill\mbox{Paul Goldmann}\pend
           \leftskip=0em{}\pstart
           \noindent{}Viele Grüße an Deine Freundin\pwindex{Reinhard, Marie 1871-03-13 – 1899-03-18@\textsc{Reinhard, Marie} (1871-03-13 – 1899-03-18), \emph{Gesangspädagogin}|pwv}!\pend
           
         
         \endnumbering\mylabel{h}\end{ledgroupsized}  \newcommand{\dateiname}{L02836}\newcommand{\titel}{Paul Goldmann an Arthur Schnitzler, 19. 1. [1898]}\newcommand{\editorInnen}{Martin Anton Müller und Laura Untner}%% latex-leseansicht-abspann.tex
%% Abspann für die Leseansicht.
%% Der Schalter \ifkorrekturansicht ist bereits durch den Vorspann gesetzt.

%% latex-abspann.tex
%% Gemeinsamer Abspann für Korrekturansicht und Leseansicht.
%% Setzt den Schalter \ifkorrekturansicht voraus (gesetzt in den
%% einbindenden Dateien latex-korrekturansicht-abspann.tex bzw.
%% latex-leseansicht-abspann.tex).
%% ---------------------------------------------------------------

\normalsize

% Das esempio-Environment wird nur in der Leseansicht benötigt
\ifkorrekturansicht\else
\newenvironment{esempio}[3]%
{
    \vspace{1.5ex}
    \rlap{\underline{#1}}
    \par
    \setlength{\parindent}{0cm}
    \nopagebreak
    \leftskip=#2cm
    \rightskip=#3cm
}
{
    \par
}
\fi

\doendnotes{C}
\bigskip
\vfill

\clearpage

\footnotesize

\ifkorrekturansicht
  \lohead{\textsc{register}}
\fi

% theindex-Environment neu definieren ohne reledmac
\makeatletter
\renewenvironment{theindex}{%
  \ifkorrekturansicht
    \section*{\indexname}%
  \else
    \subsubsection*{Index der erwähnten Entitäten}%
  \fi
  \setlength{\parindent}{0pt}%
  \setlength{\parskip}{0pt plus 0.3pt}%
  \let\item\@idxitem
}{%
  \ifkorrekturansicht\clearpage\fi
}
\makeatother

\IfFileExists{\jobname-pw.ind}{\input{\jobname-pw.ind}}{}

% Quellenangabe nur in der Leseansicht
\ifkorrekturansicht\else
% Fallback-Definitionen, falls die .tex-Datei \titel etc. nicht gesetzt hat
\providecommand{\titel}{}
\providecommand{\editorInnen}{}
\providecommand{\dateiname}{\jobname}

\vspace{3cm}

\vfill

\footnotesize
\textsc{Quelle}: \titel. Herausgegeben von {\editorInnen}. In: \emph{Arthur Schnitzler: Briefwechsel mit Autorinnen und Autoren}.
 Digitale Edition, https://schnitzler-briefe.acdh.oeaw.ac.at/{\dateiname}.html (Stand \today)
\fi

\end{document}


      