%% latex-korrekturansicht-vorspann.tex
%% Vorspann für die Korrekturansicht.
%% Lädt die gemeinsame Datei latex-vorspann.tex mit gesetztem Schalter.

\newif\ifkorrekturansicht
\korrekturansichttrue

\input{../tex-inputs/latex-vorspann}


\section[Max Burckhard an Arthur Schnitzler, {[}22. 11. 1894{]}]{L00404 Max Burckhard an Arthur Schnitzler, {[}22. 11. 1894{]}}
\nopagebreak\mylabel{L00404v}
\rehead{ }\normalsize\beginnumbering\briefempfaengerindex{Schnitzler, Arthur@\textsc{Schnitzler, Arthur}!zzzBurckhard, Max Eugen@\emph{von Max Eugen Burckhard}!1894-11-221@{{[}22. 11. 1894{]}}|(be}
\toendnotes[C]{\smallbreak\pagebreak[2]}\Standort{CUL, Schnitzler, B 20.}
\physDesc{Brief, 1 Blatt, 1 Seite, 159 Zeichen
\newline{}Handschrift: schwarze Tinte, deutsche Kurrent
\newline{}Schnitzler: mit Bleistift datiert: »22/11 94« 
\newline{}Ordnung: 1) mit rotem Buntstift von unbekannter Hand nummeriert: »\strikeout{5}«  2) mit Bleistift von unbekannter Hand nummeriert:
                                 »6«}
\pstart{}{\pb}Sehr geehrter Herr Doctor!\pend\vspace{0.5em}
\pstart
           Ihrem Wunſche entſprechend bin ich ſo frei Ihnen vorläufig »Liebelei\pwindex{Liebelei. Schauspiel in drei Akten@\emph{Liebelei. Schauspiel in drei Akten}|pw}« zurückzuſenden.\pend
           
\pstart
           Mit beſten Empfehlungen\hspace*{3.5em}Ihr ergebener \pend
           \pstart \spacefill\mbox{D\textsuperscript{r}Burckhard}\pend{}\selectlanguage{ngerman}\endnumbering\briefempfaengerindex{Schnitzler, Arthur@\textsc{Schnitzler, Arthur}!zzzBurckhard, Max Eugen@\emph{von Max Eugen Burckhard}!1894-11-221@{{[}22. 11. 1894{]}}|)be}\mylabel{L00404h}  \normalsize

\doendnotes{C}
\bigskip
\vfill

\clearpage

\footnotesize

\lohead{\textsc{register}}

% Definiere theindex-Environment komplett neu ohne reledmac
\makeatletter
\renewenvironment{theindex}{%
  \section*{\indexname}%
  \setlength{\parindent}{0pt}%
  \setlength{\parskip}{0pt plus 0.3pt}%
  \let\item\@idxitem
}{%
  \clearpage
}
\makeatother

\IfFileExists{\jobname-pw.ind}{\input{\jobname-pw.ind}}{}

\end{document}

      