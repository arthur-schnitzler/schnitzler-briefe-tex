%% latex-leseansicht-vorspann.tex
%% Vorspann für die Leseansicht.
%% Lädt die gemeinsame Datei latex-vorspann.tex mit nicht gesetztem Schalter.

\newif\ifkorrekturansicht
\korrekturansichtfalse

\input{../tex-inputs/latex-vorspann}


         
         \renewcommand{\erwaehntePersonen}{Personen: Paul Goldmann, Ferdinand von Holzinger, Maria Charlotte Lamberg, Hilda von Mitis, Maximilian von Mitis, Maria Pia Mitis, Charlotte Pohl-Glas}
         \renewcommand{\erwaehnteOrte}{Orte: Alser Straße, Bad Ischl, Salzburg, Steyr, Veronikagasse, Wien, XVII., Hernals}
         \renewcommand{\erwaehnteWerke}{Werke: Die kleine Veronika, Heldentod, Neue Deutsche Rundschau, Wiener Allgemeine Zeitung}
               \section[ Felix Salten an Arthur Schnitzler, 7. 8. 1894]{ Felix Salten an Arthur Schnitzler, 7. 8. 1894}\nopagebreak\mylabel{v}\rehead{ }\begin{ledgroupsized}[t]{13cm}\normalsize\beginnumbering \toendnotes[C]{\smallbreak\pagebreak[2]} \Standort{CUL, Schnitzler, B 89, A 1.}
\physDesc{Brief, 1 Blatt, 2 Seiten, 491 Zeichen
\newline{}Handschrift: schwarze Tinte, lateinische Kurrent
\newline{}Ordnung: mit Bleistift von unbekannter Hand nummeriert: »43« }\toendnotes[C]{\smallbreak}\pstart
           \raggedleft{}{\pb}Wien\oindex{Wien@\textbf{Wien}|pw}, 7. VIII. 94\pend
           \pstart{}Lieber Arthur,\pend\pstart
           I. \label{K_L03142-1v}\edtext{Process}{\lemma{\textnormal{\emph{Process}}}\Cendnote{\textnormal{Es dürfte sich um den zweiten Prozess gegen Salten\pwindex{Salten, Felix 06.09.1869 – 08.10.1945@\textsc{Salten, Felix} (06.09.1869 – 08.10.1945), \emph{Schriftsteller, Journalist}|pwk}s Partnerin Charlotte Glas\pwindex{Pohl-Glas, Charlotte 1873-01-01 – 1944-02-15@\textsc{Pohl-Glas, Charlotte} (1873-01-01 – 1944-02-15), \emph{Schriftstellerin, Politikerin, Sozialistin}|pwk} handeln. Der erste hatte wenige Tage zuvor, am 25. 7. 1894, stattgefunden. Bei einer Versammlung am
                     1. 5. 1894 hatte sie einen Hochruf auf die
                     »internationale revolutionäre Sozialdemokratie« ausgerufen. Die
                  Verwendung des Wortes ›revolutionär‹ wurde ihr als umstürzlerisch zur Last
                  gelegt worden. Der Richter\pwindex{Holzinger, Ferdinand von 21.09.1836 – 30.12.1901@\textsc{Holzinger, Ferdinand von} (21.09.1836 – 30.12.1901), \emph{Richter}|pwkv} verurteilte sie zu 14 Tagen Arrest, die sie Mitte September 1894 absolvierte, vgl. Felix Salten an Arthur Schnitzler, [11. 9. 1894]. Am 30. 11. 1894 wurde sie dann neuerlich in Steyr\oindex{Steyr@\textbf{Steyr}|pwk} für ein ähnliches Vergehen zu einem
                  weiteren Monat verurteilt. Diesen Arrest trat sie am 15. 1. 1895 in Wien\oindex{Wien@\textbf{Wien}|pwk} an, vgl. Felix Salten an Arthur Schnitzler, [14?. 1. 1895]. Zu diesem Zeitpunkt
                  war sie bereits mit dem gemeinsamen Kind\pwindex{Lamberg, Maria Charlotte 1895-03-24 – 1895-07-27@\textsc{Lamberg, Maria Charlotte} (1895-03-24 – 1895-07-27)|pwkv} mit Salten\pwindex{Salten, Felix 06.09.1869 – 08.10.1945@\textsc{Salten, Felix} (06.09.1869 – 08.10.1945), \emph{Schriftsteller, Journalist}|pwk} schwanger.}}}\label{K_L03142-1h}
               ist neuerdings vertagt.\pend
           \pstart
           II. Wie ich Ihnen auf meiner \label{K_L03142-2v}\edtext{Karte
               nach Salzburg\oindex{Salzburg@\textbf{Salzburg}|pw}}{\lemma{\textnormal{\emph{Karte
               nach Salzburg}}}\Cendnote{\textnormal{Nicht erhalten. Schnitzler\pwindex{Schnitzler, Arthur 15.05.1862 – 21.10.1931@\textsc{Schnitzler, Arthur} (15.05.1862 – 21.10.1931), \emph{Schriftsteller, Mediziner}|pwk} war zwischen 1. 8. 1894 und 5. 8. 1894 in Salzburg\oindex{Salzburg@\textbf{Salzburg}|pwkv}. Schnitzler\pwindex{Schnitzler, Arthur 15.05.1862 – 21.10.1931@\textsc{Schnitzler, Arthur} (15.05.1862 – 21.10.1931), \emph{Schriftsteller, Mediziner}|pwk} hatte sich für Goldmann\pwindex{Goldmann, Paul 31.01.1865 – 25.09.1935@\textsc{Goldmann, Paul} (31.01.1865 – 25.09.1935), \emph{Schriftsteller, Journalist}|pwk}
                  bei
                  Salten\pwindex{Salten, Felix 06.09.1869 – 08.10.1945@\textsc{Salten, Felix} (06.09.1869 – 08.10.1945), \emph{Schriftsteller, Journalist}|pwk} um Informationen zu Hilda von Mitis\pwindex{Mitis, Hilda von 1876-08-30 – 1894-12-14@\textsc{Mitis, Hilda von} (1876-08-30 – 1894-12-14), \emph{Schriftstellerin, Telefonistin}|pwk}  erkundigt, vgl. Paul Goldmann an Arthur Schnitzler, 29. 7. [1894].}}}\label{K_L03142-2h} berichtet, lebt
                  H. M.\pwindex{Mitis, Hilda von 1876-08-30 – 1894-12-14@\textsc{Mitis, Hilda von} (1876-08-30 – 1894-12-14), \emph{Schriftstellerin, Telefonistin}|pw} bei ihren Eltern\pwindex{Mitis, Maximilian von 1840-07-12 – 1894-12-10@\textsc{Mitis, Maximilian von} (1840-07-12 – 1894-12-10), \emph{Sekretär}|pwv}\pwindex{Mitis, Maria Pia @\textsc{Mitis, Maria Pia}|pwv}, welche bis zum 1. d. M.{ }Alserstrasse 42\oindex{Alser Strasse@\textbf{Alser Straße}|pw} wohnten, aber übersiedelt sind.
               Ich konnte damals die neue Adresse nicht ermitteln, habe sie jedoch heute erfragt. \uline{H. M.}\pwindex{Mitis, Hilda von 1876-08-30 – 1894-12-14@\textsc{Mitis, Hilda von} (1876-08-30 – 1894-12-14), \emph{Schriftstellerin, Telefonistin}|pw} wohnt: \uline{Hernals}\oindex{XVII., Hernals@\textbf{XVII., Hernals}|pw}, Veronikagasse 25\oindex{Veronikagasse@\textbf{Veronikagasse}|pw}, II. Stock
               Thür 19.\pend
           \pstart
           III. \label{K_L03142-3v}\edtext{Heldentod\pwindex{Salten, Felix 06.09.1869 – 08.10.1945@\textsc{Salten, Felix} (06.09.1869 – 08.10.1945), \emph{Schriftsteller, Journalist}!Heldentod1895-01-01@\strich\emph{Heldentod} {[}1895-01-01{]}|pw} ruht}{\lemma{\textnormal{\emph{Heldentod ruht}}}\Cendnote{\textnormal{Salten\pwindex{Salten, Felix 06.09.1869 – 08.10.1945@\textsc{Salten, Felix} (06.09.1869 – 08.10.1945), \emph{Schriftsteller, Journalist}|pwk}s Novelle \emph{Heldentod}\pwindex{Salten, Felix 06.09.1869 – 08.10.1945@\textsc{Salten, Felix} (06.09.1869 – 08.10.1945), \emph{Schriftsteller, Journalist}!Heldentod1895-01-01@\strich\emph{Heldentod} {[}1895-01-01{]}|pwk} wurde im Jahr darauf
                  publiziert: Felix Salten\pwindex{Salten, Felix 06.09.1869 – 08.10.1945@\textsc{Salten, Felix} (06.09.1869 – 08.10.1945), \emph{Schriftsteller, Journalist}|pwk}: \emph{Heldentod}\pwindex{Salten, Felix 06.09.1869 – 08.10.1945@\textsc{Salten, Felix} (06.09.1869 – 08.10.1945), \emph{Schriftsteller, Journalist}!Heldentod1895-01-01@\strich\emph{Heldentod} {[}1895-01-01{]}|pwk}. In: \emph{Wiener
                        Allgemeine Zeitung}\pwindex{?? Werk@Nicht ermittelte Verfasserinnen und Verfasser!Wiener Allgemeine Zeitung1.3.1880 – 11.2.1934@\emph{Wiener Allgemeine Zeitung} {[}1.3.1880 – 11.2.1934{]}|pwk}, Nr. 504, 1. 1. 1895,
                     S. 2–3.}}}\label{K_L03142-3h}.\pend
           \pstart
           IV. \label{K_L03142-4v}\edtext{Confirmandin\pwindex{Salten, Felix 06.09.1869 – 08.10.1945@\textsc{Salten, Felix} (06.09.1869 – 08.10.1945), \emph{Schriftsteller, Journalist}!kleine Veronika1902-12-01@\strich\emph{Die kleine Veronika} {[}1902-12-01{]}|pw}}{\lemma{\textnormal{\emph{Confirmandin}}}\Cendnote{\textnormal{Die Novelle mit dem Arbeitstitel \emph{Die Confirmandin}\pwindex{Salten, Felix 06.09.1869 – 08.10.1945@\textsc{Salten, Felix} (06.09.1869 – 08.10.1945), \emph{Schriftsteller, Journalist}!kleine Veronika1902-12-01@\strich\emph{Die kleine Veronika} {[}1902-12-01{]}|pwk} erschien Jahre später unter
                  dem Titel \emph{Die kleine Veronika}\pwindex{Salten, Felix 06.09.1869 – 08.10.1945@\textsc{Salten, Felix} (06.09.1869 – 08.10.1945), \emph{Schriftsteller, Journalist}!kleine Veronika1902-12-01@\strich\emph{Die kleine Veronika} {[}1902-12-01{]}|pwk}, \emph{Neue Deutsche Rundschau}\pwindex{Neue Deutsche Rundschau1894-01-01 – 1903-12-31@\emph{Neue Deutsche Rundschau} {[}1894-01-01 – 1903-12-31{]}|pwk}, Jg. 13, Nr. 12,
                        Dezember 1902, S. 1285–1333.}}}\label{K_L03142-4h} geht
               langsam vorwärts, doch war ich in diesen {\pb}Tagen durch Besuch
               aufgehalten.\pend
           \pstart
           V. ..........!!\pend
           \pstart
           Herzlichst {\\[\baselineskip]}\spacefill\mbox{Salten.}\pend
           \leftskip=0em{}
         
         \endnumbering\mylabel{h}\end{ledgroupsized}  \newcommand{\dateiname}{L03142}\newcommand{\titel}{Felix Salten an Arthur Schnitzler, 7. 8. 1894}\newcommand{\editorInnen}{Martin Anton Müller und Laura Untner}%% latex-leseansicht-abspann.tex
%% Abspann für die Leseansicht.
%% Der Schalter \ifkorrekturansicht ist bereits durch den Vorspann gesetzt.

%% latex-abspann.tex
%% Gemeinsamer Abspann für Korrekturansicht und Leseansicht.
%% Setzt den Schalter \ifkorrekturansicht voraus (gesetzt in den
%% einbindenden Dateien latex-korrekturansicht-abspann.tex bzw.
%% latex-leseansicht-abspann.tex).
%% ---------------------------------------------------------------

\normalsize

% Das esempio-Environment wird nur in der Leseansicht benötigt
\ifkorrekturansicht\else
\newenvironment{esempio}[3]%
{
    \vspace{1.5ex}
    \rlap{\underline{#1}}
    \par
    \setlength{\parindent}{0cm}
    \nopagebreak
    \leftskip=#2cm
    \rightskip=#3cm
}
{
    \par
}
\fi

\doendnotes{C}
\bigskip
\vfill

\clearpage

\footnotesize

\ifkorrekturansicht
  \lohead{\textsc{register}}
\fi

% theindex-Environment neu definieren ohne reledmac
\makeatletter
\renewenvironment{theindex}{%
  \ifkorrekturansicht
    \section*{\indexname}%
  \else
    \subsubsection*{Index der erwähnten Entitäten}%
  \fi
  \setlength{\parindent}{0pt}%
  \setlength{\parskip}{0pt plus 0.3pt}%
  \let\item\@idxitem
}{%
  \ifkorrekturansicht\clearpage\fi
}
\makeatother

\IfFileExists{\jobname-pw.ind}{\input{\jobname-pw.ind}}{}

% Quellenangabe nur in der Leseansicht
\ifkorrekturansicht\else
% Fallback-Definitionen, falls die .tex-Datei \titel etc. nicht gesetzt hat
\providecommand{\titel}{}
\providecommand{\editorInnen}{}
\providecommand{\dateiname}{\jobname}

\vspace{3cm}

\vfill

\footnotesize
\textsc{Quelle}: \titel. Herausgegeben von {\editorInnen}. In: \emph{Arthur Schnitzler: Briefwechsel mit Autorinnen und Autoren}.
 Digitale Edition, https://schnitzler-briefe.acdh.oeaw.ac.at/{\dateiname}.html (Stand \today)
\fi

\end{document}


      