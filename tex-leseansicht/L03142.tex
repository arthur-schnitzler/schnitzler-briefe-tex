%% latex-leseansicht-vorspann.tex
%% Vorspann für die Leseansicht.
%% Lädt die gemeinsame Datei latex-vorspann.tex mit nicht gesetztem Schalter.

\newif\ifkorrekturansicht
\korrekturansichtfalse

\input{../tex-inputs/latex-vorspann}


\section[ Felix Salten an Arthur Schnitzler, 7. 8. 1894]{L03142 Felix Salten an Arthur Schnitzler,  7. 8. 1894}
\nopagebreak\mylabel{L03142v}
\rehead{ }\normalsize\beginnumbering\briefempfaengerindex{Schnitzler, Arthur@\textsc{Schnitzler, Arthur}!zzzSalten, Felix@\emph{von Felix Salten}!1894-08-071@{7. 8. 1894}|(be}
\toendnotes[C]{\smallbreak\pagebreak[2]}
\correspDesc{Versand  durch Felix Salten am 7. 8. 1894 in Wien
\newline{}Erhalt  durch Arthur Schnitzler im Zeitraum [8. 8. 1894
                  – 12. 8. 1894?] in Bad Ischl}\toendnotes[C]{\smallbreak}
\Standort{CUL, Schnitzler, B 89, A 1.}
\physDesc{Brief, 1 Blatt, 2 Seiten, 491 Zeichen
\newline{}Handschrift: schwarze Tinte, lateinische Kurrent
\newline{}Ordnung: mit Bleistift von unbekannter Hand nummeriert: »43« }\toendnotes[C]{\smallbreak}
\pstart
           \raggedleft{}{\pb}Wien\oindex{Wien@\textbf{Wien}, \emph{Verwaltungsgebiet}|pw}, 7. VIII. 94\pend
           
\pstart{}Lieber Arthur,\pend\vspace{0.5em}
\pstart
           I. \label{K_L03142-1v}\edtext{Process}{\lemma{\textnormal{\emph{Process}}}\Cendnote{\textnormal{Es dürfte sich um den zweiten Prozess gegen Saltens\pwindex{Salten, Felix 6.\,9.\,1869 Budapest – 8.\,10.\,1945 Zürich@\textsc{Salten, Felix} (6.\,9.\,1869 Budapest – 8.\,10.\,1945 Zürich), \emph{Schriftsteller, Journalist, Chefredakteur}|pwk} Partnerin Charlotte Glas\pwindex{Pohl-Glas, Charlotte 1.\,1.\,1873 Wien – 15.\,2.\,1944 Zürich@\textsc{Pohl-Glas, Charlotte} (1.\,1.\,1873 Wien – 15.\,2.\,1944 Zürich), \emph{Schriftstellerin, Politikerin, Sozialistin}|pwk} handeln. Der erste hatte wenige Tage zuvor, am 25. 7. 1894, stattgefunden. Bei einer Versammlung am
                     1. 5. 1894 hatte sie einen Hochruf auf die
                     »internationale revolutionäre Sozialdemokratie« ausgerufen. Die
                  Verwendung des Wortes ›revolutionär‹ wurde ihr als umstürzlerisch zur Last gelegt. Der Richter\pwindex{Holzinger, Ferdinand von 21.\,9.\,1836 Wien – 30.\,12.\,1901 ebd.@\textsc{Holzinger, Ferdinand von} (21.\,9.\,1836 Wien – 30.\,12.\,1901 ebd.), \emph{Richter}|pwkv}
                  verurteilte sie zu 14 Tagen Arrest, die sie Mitte September 1894 absolvierte, vgl. XXXX Auszeichnungsfehler: Dokument L03145 nicht gefunden. Am 30. 11. 1894
                  wurde sie dann neuerlich in Steyr\oindex{Steyr@\textbf{Steyr}, \emph{Hauptstadt}|pwk} für ein
                  ähnliches Vergehen zu einem weiteren Monat verurteilt. Diesen Arrest trat sie am
                     15. 1. 1895 in Wien\oindex{Wien@\textbf{Wien}, \emph{Verwaltungsgebiet}|pwk} an, vgl. XXXX Auszeichnungsfehler: Dokument L03148 nicht gefunden. Zu
                  diesem Zeitpunkt war sie bereits mit dem gemeinsamen Kind\pwindex{Lamberg, Maria Charlotte 24.\,3.\,1895 Wien – 27.\,7.\,1895 Gerasdorf bei Wien@\textsc{Lamberg, Maria Charlotte} (24.\,3.\,1895 Wien – 27.\,7.\,1895 Gerasdorf bei Wien)|pwkv} mit Salten\pwindex{Salten, Felix 6.\,9.\,1869 Budapest – 8.\,10.\,1945 Zürich@\textsc{Salten, Felix} (6.\,9.\,1869 Budapest – 8.\,10.\,1945 Zürich), \emph{Schriftsteller, Journalist, Chefredakteur}|pwk} schwanger.}}}\label{K_L03142-1} ist neuerdings vertagt.\pend
           
\pstart
           II. Wie ich Ihnen auf meiner \label{K_L03142-2v}\edtext{Karte
               nach Salzburg\oindex{Salzburg@\textbf{Salzburg}, \emph{Verwaltungsgebiet}|pw}}{\lemma{\textnormal{\emph{Karte
               nach Salzburg}}}\Cendnote{\textnormal{Nicht erhalten. Schnitzler war zwischen 1. 8. 1894 und 5. 8. 1894 in Salzburg\oindex{Salzburg@\textbf{Salzburg}, \emph{Verwaltungsgebiet}|pwkv}. Er
                  hatte sich für Goldmann\pwindex{Goldmann, Paul 31.\,1.\,1865 Breslau – 25.\,9.\,1935 Wien@\textsc{Goldmann, Paul} (31.\,1.\,1865 Breslau – 25.\,9.\,1935 Wien), \emph{Schriftsteller, Journalist}|pwk} bei Salten\pwindex{Salten, Felix 6.\,9.\,1869 Budapest – 8.\,10.\,1945 Zürich@\textsc{Salten, Felix} (6.\,9.\,1869 Budapest – 8.\,10.\,1945 Zürich), \emph{Schriftsteller, Journalist, Chefredakteur}|pwk} um Informationen zu Hilda von Mitis\pwindex{Mitis, Hilda von 30.\,8.\,1876 Wien – 14.\,12.\,1894 Bratislava@\textsc{Mitis, Hilda von} (30.\,8.\,1876 Wien – 14.\,12.\,1894 Bratislava), \emph{Schriftstellerin, Telefonistin}|pwk} erkundigt, vgl. XXXX Auszeichnungsfehler: Dokument L02608 nicht gefunden.}}}\label{K_L03142-2} berichtet, lebt H. M.\pwindex{Mitis, Hilda von 30.\,8.\,1876 Wien – 14.\,12.\,1894 Bratislava@\textsc{Mitis, Hilda von} (30.\,8.\,1876 Wien – 14.\,12.\,1894 Bratislava), \emph{Schriftstellerin, Telefonistin}|pw} bei ihren Eltern\pwindex{Mitis, Maximilian von 12.\,7.\,1840 Wien – 10.\,12.\,1894 ebd.@\textsc{Mitis, Maximilian von} (12.\,7.\,1840 Wien – 10.\,12.\,1894 ebd.), \emph{Sekretär}|pwv}\pwindex{Mitis, Maria Pia @\textsc{Mitis, Maria Pia}|pwv}, welche bis zum 1. d. M.{ }Alserstrasse 42\oindex{Wien@\textbf{Wien}!VIII., Josefstadt@\textbf{VIII., Josefstadt}!Alser Straße@\textbf{Alser Straße}, \emph{Straße}|pw}\oindex{Wien@\textbf{Wien}!IX., Alsergrund@\textbf{IX., Alsergrund}!Alser Straße@\textbf{Alser Straße}, \emph{Straße}|pw} wohnten, aber übersiedelt sind.
               Ich konnte damals die neue Adresse nicht ermitteln, habe sie jedoch heute erfragt. \uline{H. M.}\pwindex{Mitis, Hilda von 30.\,8.\,1876 Wien – 14.\,12.\,1894 Bratislava@\textsc{Mitis, Hilda von} (30.\,8.\,1876 Wien – 14.\,12.\,1894 Bratislava), \emph{Schriftstellerin, Telefonistin}|pw} wohnt: \uline{Hernals}\oindex{XVII., Hernals@\textbf{XVII., Hernals}, \emph{Verwaltungsgebiet}|pw}, Veronikagasse 25\oindex{Wien@\textbf{Wien}!XVII., Hernals@\textbf{XVII., Hernals}!Veronikagasse@\textbf{Veronikagasse}, \emph{Straße}|pw}, II. Stock
               Thür 19.\pend
           
\pstart
           III. \label{K_L03142-3v}\edtext{Heldentod\pwindex{Salten, Felix 6.\,9.\,1869 Budapest – 8.\,10.\,1945 Zürich@\textsc{Salten, Felix} (6.\,9.\,1869 Budapest – 8.\,10.\,1945 Zürich), \emph{Schriftsteller, Journalist, Chefredakteur}!Heldentod@\strich\emph{Heldentod}|pw} ruht}{\lemma{\textnormal{\emph{Heldentod ruht}}}\Cendnote{\textnormal{Saltens\pwindex{Salten, Felix 6.\,9.\,1869 Budapest – 8.\,10.\,1945 Zürich@\textsc{Salten, Felix} (6.\,9.\,1869 Budapest – 8.\,10.\,1945 Zürich), \emph{Schriftsteller, Journalist, Chefredakteur}|pwk} Novelle wurde im Jahr darauf
                  publiziert: Felix Salten\pwindex{Salten, Felix 6.\,9.\,1869 Budapest – 8.\,10.\,1945 Zürich@\textsc{Salten, Felix} (6.\,9.\,1869 Budapest – 8.\,10.\,1945 Zürich), \emph{Schriftsteller, Journalist, Chefredakteur}|pwk}: \emph{Heldentod}\pwindex{Salten, Felix 6.\,9.\,1869 Budapest – 8.\,10.\,1945 Zürich@\textsc{Salten, Felix} (6.\,9.\,1869 Budapest – 8.\,10.\,1945 Zürich), \emph{Schriftsteller, Journalist, Chefredakteur}!Heldentod@\strich\emph{Heldentod}|pwk}. In: \emph{Wiener
                        Allgemeine Zeitung}\pwindex{Wiener Allgemeine Zeitung@\emph{Wiener Allgemeine Zeitung}|pwk}, Nr. 504, 1. 1. 1895,
                     S. 2–3.}}}\label{K_L03142-3}.\pend
           
\pstart
           IV. \label{K_L03142-4v}\edtext{Confirmandin\pwindex{Salten, Felix 6.\,9.\,1869 Budapest – 8.\,10.\,1945 Zürich@\textsc{Salten, Felix} (6.\,9.\,1869 Budapest – 8.\,10.\,1945 Zürich), \emph{Schriftsteller, Journalist, Chefredakteur}!kleine Veronika@\strich\emph{Die kleine Veronika}|pw}}{\lemma{\textnormal{\emph{Confirmandin}}}\Cendnote{\textnormal{Die Novelle mit dem Arbeitstitel \emph{Die Confirmandin}\pwindex{Salten, Felix 6.\,9.\,1869 Budapest – 8.\,10.\,1945 Zürich@\textsc{Salten, Felix} (6.\,9.\,1869 Budapest – 8.\,10.\,1945 Zürich), \emph{Schriftsteller, Journalist, Chefredakteur}!kleine Veronika@\strich\emph{Die kleine Veronika}|pwk} erschien Jahre später unter
                  dem Titel \emph{Die kleine Veronika}\pwindex{Salten, Felix 6.\,9.\,1869 Budapest – 8.\,10.\,1945 Zürich@\textsc{Salten, Felix} (6.\,9.\,1869 Budapest – 8.\,10.\,1945 Zürich), \emph{Schriftsteller, Journalist, Chefredakteur}!kleine Veronika@\strich\emph{Die kleine Veronika}|pwk} in: \emph{Neue Deutsche Rundschau}\pwindex{Neue Deutsche Rundschau@\emph{Neue Deutsche Rundschau}|pwk}, Jg. 13, Nr. 12,
                        Dezember 1902, S. 1285–1333.}}}\label{K_L03142-4} geht
               langsam vorwärts, doch war ich in diesen {\pb}Tagen durch Besuch
               aufgehalten.\pend
           
\pstart
           V. ..........!!\pend
           
\pstart
           Herzlichst {\\[\baselineskip]}\spacefill\mbox{Salten.}\pend
           \leftskip=0em{}\selectlanguage{ngerman}\endnumbering\briefempfaengerindex{Schnitzler, Arthur@\textsc{Schnitzler, Arthur}!zzzSalten, Felix@\emph{von Felix Salten}!1894-08-071@{7. 8. 1894}|)be}\mylabel{L03142h}  \newcommand{\dateiname}{L03142}\newcommand{\titel}{Felix Salten an Arthur Schnitzler, 7. 8. 1894}\newcommand{\editorInnen}{Martin Anton Müller und Laura Untner}%% latex-leseansicht-abspann.tex
%% Abspann für die Leseansicht.
%% Der Schalter \ifkorrekturansicht ist bereits durch den Vorspann gesetzt.

%% latex-abspann.tex
%% Gemeinsamer Abspann für Korrekturansicht und Leseansicht.
%% Setzt den Schalter \ifkorrekturansicht voraus (gesetzt in den
%% einbindenden Dateien latex-korrekturansicht-abspann.tex bzw.
%% latex-leseansicht-abspann.tex).
%% ---------------------------------------------------------------

\normalsize

% Das esempio-Environment wird nur in der Leseansicht benötigt
\ifkorrekturansicht\else
\newenvironment{esempio}[3]%
{
    \vspace{1.5ex}
    \rlap{\underline{#1}}
    \par
    \setlength{\parindent}{0cm}
    \nopagebreak
    \leftskip=#2cm
    \rightskip=#3cm
}
{
    \par
}
\fi

\doendnotes{C}
\bigskip
\vfill

\clearpage

\footnotesize

\ifkorrekturansicht
  \lohead{\textsc{register}}
\fi

% theindex-Environment neu definieren ohne reledmac
\makeatletter
\renewenvironment{theindex}{%
  \ifkorrekturansicht
    \section*{\indexname}%
  \else
    \subsubsection*{Index der erwähnten Entitäten}%
  \fi
  \setlength{\parindent}{0pt}%
  \setlength{\parskip}{0pt plus 0.3pt}%
  \let\item\@idxitem
}{%
  \ifkorrekturansicht\clearpage\fi
}
\makeatother

\IfFileExists{\jobname-pw.ind}{\input{\jobname-pw.ind}}{}

% Quellenangabe nur in der Leseansicht
\ifkorrekturansicht\else
% Fallback-Definitionen, falls die .tex-Datei \titel etc. nicht gesetzt hat
\providecommand{\titel}{}
\providecommand{\editorInnen}{}
\providecommand{\dateiname}{\jobname}

\vspace{3cm}

\vfill

\footnotesize
\textsc{Quelle}: \titel. Herausgegeben von {\editorInnen}. In: \emph{Arthur Schnitzler: Briefwechsel mit Autorinnen und Autoren}.
 Digitale Edition, https://schnitzler-briefe.acdh.oeaw.ac.at/{\dateiname}.html (Stand \today)
\fi

\end{document}


