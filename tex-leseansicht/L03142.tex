%% latex-korrekturansicht-vorspann.tex
%% Vorspann für die Korrekturansicht.
%% Lädt die gemeinsame Datei latex-vorspann.tex mit gesetztem Schalter.

\newif\ifkorrekturansicht
\korrekturansichttrue

\input{../tex-inputs/latex-vorspann}


\section[ Felix Salten an Arthur Schnitzler, 7. 8. 1894]{L03142 Felix Salten an Arthur Schnitzler, 7. 8. 1894}
\nopagebreak\mylabel{L03142v}
\rehead{ }\normalsize\beginnumbering\briefempfaengerindex{Schnitzler, Arthur@\textsc{Schnitzler, Arthur}!zzzSalten, Felix@\emph{von Felix Salten}!1894-08-071@{7. 8. 1894}|(be}
\toendnotes[C]{\smallbreak\pagebreak[2]}\Standort{CUL, Schnitzler, B 89, A 1.}
\physDesc{Brief, 1 Blatt, 2 Seiten, 491 Zeichen
\newline{}Handschrift: schwarze Tinte, lateinische Kurrent
\newline{}Ordnung: mit Bleistift von unbekannter Hand nummeriert: »43« }\toendnotes[C]{\smallbreak}
\pstart
           \raggedleft{}{\pb}Wien\oindex{Wien@\textbf{Wien}, \emph{A.ADM2}|pw}, 7. VIII. 94\pend
           
\pstart{}Lieber Arthur,\pend\vspace{0.5em}
\pstart
           I. \label{K_L03142-1v}\edtext{Process}{\lemma{\textnormal{\emph{Process}}}\Cendnote{\textnormal{Es dürfte sich um den zweiten Prozess gegen Saltens\pwindex{Salten, Felix 06.09.1869 – 08.10.1945@\textsc{Salten, Felix} (06.09.1869 – 08.10.1945), \emph{Schriftsteller/Schriftstellerin, Journalist/Journalistin, Chefredakteur/Chefredakteurin}|pwk} Partnerin Charlotte Glas\pwindex{Pohl-Glas, Charlotte 1873-01-01 – 1944-02-15@\textsc{Pohl-Glas, Charlotte} (1873-01-01 – 1944-02-15), \emph{Schriftsteller/Schriftstellerin, Politiker/Politikerin, Sozialist/Sozialistin}|pwk} handeln. Der erste hatte wenige Tage zuvor, am 25. 7. 1894, stattgefunden. Bei einer Versammlung am
                     1. 5. 1894 hatte sie einen Hochruf auf die
                     »internationale revolutionäre Sozialdemokratie« ausgerufen. Die
                  Verwendung des Wortes ›revolutionär‹ wurde ihr als umstürzlerisch zur Last gelegt. Der Richter\pwindex{Holzinger, Ferdinand von 21.09.1836 – 30.12.1901@\textsc{Holzinger, Ferdinand von} (21.09.1836 – 30.12.1901), \emph{Richter/Richterin}|pwkv}
                  verurteilte sie zu 14 Tagen Arrest, die sie Mitte
                     September 1894 absolvierte, vgl. Felix Salten an Arthur Schnitzler, [11. 9. 1894]. Am 30. 11. 1894
                  wurde sie dann neuerlich in Steyr\oindex{Steyr@\textbf{Steyr}, \emph{P.PPLA3}|pwk} für ein
                  ähnliches Vergehen zu einem weiteren Monat verurteilt. Diesen Arrest trat sie am
                     15. 1. 1895 in Wien\oindex{Wien@\textbf{Wien}, \emph{A.ADM2}|pwk} an, vgl. Felix Salten an Arthur Schnitzler, [14?. 1. 1895]. Zu
                  diesem Zeitpunkt war sie bereits mit dem gemeinsamen Kind\pwindex{Lamberg, Maria Charlotte 1895-03-24 – 1895-07-27@\textsc{Lamberg, Maria Charlotte} (1895-03-24 – 1895-07-27)|pwkv} mit Salten\pwindex{Salten, Felix 06.09.1869 – 08.10.1945@\textsc{Salten, Felix} (06.09.1869 – 08.10.1945), \emph{Schriftsteller/Schriftstellerin, Journalist/Journalistin, Chefredakteur/Chefredakteurin}|pwk} schwanger.}}}\label{K_L03142-1} ist neuerdings vertagt.\pend
           
\pstart
           II. Wie ich Ihnen auf meiner \label{K_L03142-2v}\edtext{Karte
               nach Salzburg\oindex{Salzburg@\textbf{Salzburg}, \emph{A.ADM2}|pw}}{\lemma{\textnormal{\emph{Karte
               nach Salzburg}}}\Cendnote{\textnormal{Nicht erhalten. Schnitzler war zwischen 1. 8. 1894 und 5. 8. 1894 in Salzburg\oindex{Salzburg@\textbf{Salzburg}, \emph{A.ADM2}|pwkv}. Er
                  hatte sich für Goldmann\pwindex{Goldmann, Paul 31.01.1865 – 25.09.1935@\textsc{Goldmann, Paul} (31.01.1865 – 25.09.1935), \emph{Schriftsteller/Schriftstellerin, Journalist/Journalistin}|pwk} bei Salten\pwindex{Salten, Felix 06.09.1869 – 08.10.1945@\textsc{Salten, Felix} (06.09.1869 – 08.10.1945), \emph{Schriftsteller/Schriftstellerin, Journalist/Journalistin, Chefredakteur/Chefredakteurin}|pwk} um Informationen zu Hilda von Mitis\pwindex{Mitis, Hilda von 1876-08-30 – 1894-12-14@\textsc{Mitis, Hilda von} (1876-08-30 – 1894-12-14), \emph{Schriftsteller/Schriftstellerin, Telefonist/Telefonistin}|pwk} erkundigt, vgl. Paul Goldmann an Arthur Schnitzler, 29. 7. [1894].}}}\label{K_L03142-2} berichtet, lebt H. M.\pwindex{Mitis, Hilda von 1876-08-30 – 1894-12-14@\textsc{Mitis, Hilda von} (1876-08-30 – 1894-12-14), \emph{Schriftsteller/Schriftstellerin, Telefonist/Telefonistin}|pw} bei ihren Eltern\pwindex{Mitis, Maximilian von 1840-07-12 – 1894-12-10@\textsc{Mitis, Maximilian von} (1840-07-12 – 1894-12-10), \emph{Sekretär/Sekretärin}|pwv}\pwindex{Mitis, Maria Pia @\textsc{Mitis, Maria Pia}|pwv}, welche bis zum 1. d.
                  M.{ }Alserstrasse 42\oindex{Alser Strasse@\textbf{Alser Straße}, \emph{Straße (K.STR)}|pw} wohnten, aber übersiedelt sind.
               Ich konnte damals die neue Adresse nicht ermitteln, habe sie jedoch heute erfragt. \uline{H. M.}\pwindex{Mitis, Hilda von 1876-08-30 – 1894-12-14@\textsc{Mitis, Hilda von} (1876-08-30 – 1894-12-14), \emph{Schriftsteller/Schriftstellerin, Telefonist/Telefonistin}|pw} wohnt: \uline{Hernals}\oindex{XVII., Hernals@\textbf{XVII., Hernals}, \emph{A.ADM3}|pw}, Veronikagasse 25\oindex{Veronikagasse@\textbf{Veronikagasse}, \emph{Straße (K.STR)}|pw}, II. Stock
               Thür 19.\pend
           
\pstart
           III. \label{K_L03142-3v}\edtext{Heldentod\pwindex{Heldentod@\emph{Heldentod}|pw} ruht}{\lemma{\textnormal{\emph{Heldentod ruht}}}\Cendnote{\textnormal{Saltens\pwindex{Salten, Felix 06.09.1869 – 08.10.1945@\textsc{Salten, Felix} (06.09.1869 – 08.10.1945), \emph{Schriftsteller/Schriftstellerin, Journalist/Journalistin, Chefredakteur/Chefredakteurin}|pwk} Novelle wurde im Jahr darauf
                  publiziert: Felix Salten\pwindex{Salten, Felix 06.09.1869 – 08.10.1945@\textsc{Salten, Felix} (06.09.1869 – 08.10.1945), \emph{Schriftsteller/Schriftstellerin, Journalist/Journalistin, Chefredakteur/Chefredakteurin}|pwk}: \emph{Heldentod}\pwindex{Heldentod@\emph{Heldentod}|pwk}. In: \emph{Wiener
                        Allgemeine Zeitung}\pwindex{Wiener Allgemeine Zeitung@\emph{Wiener Allgemeine Zeitung}|pwk}, Nr. 504, 1. 1. 1895,
                     S. 2–3.}}}\label{K_L03142-3}.\pend
           
\pstart
           IV. \label{K_L03142-4v}\edtext{Confirmandin\pwindex{kleine Veronika@\emph{Die kleine Veronika}|pw}}{\lemma{\textnormal{\emph{Confirmandin}}}\Cendnote{\textnormal{Die Novelle mit dem Arbeitstitel \emph{Die Confirmandin}\pwindex{kleine Veronika@\emph{Die kleine Veronika}|pwk} erschien Jahre später unter
                  dem Titel \emph{Die kleine Veronika}\pwindex{kleine Veronika@\emph{Die kleine Veronika}|pwk} in: \emph{Neue Deutsche Rundschau}\pwindex{Neue Deutsche Rundschau@\emph{Neue Deutsche Rundschau}|pwk}, Jg. 13, Nr. 12,
                        Dezember 1902, S. 1285–1333.}}}\label{K_L03142-4} geht
               langsam vorwärts, doch war ich in diesen {\pb}Tagen durch Besuch
               aufgehalten.\pend
           
\pstart
           V. ..........!!\pend
           
\pstart
           Herzlichst {\\[\baselineskip]}\spacefill\mbox{Salten.}\pend
           \leftskip=0em{}\selectlanguage{ngerman}\endnumbering\briefempfaengerindex{Schnitzler, Arthur@\textsc{Schnitzler, Arthur}!zzzSalten, Felix@\emph{von Felix Salten}!1894-08-071@{7. 8. 1894}|)be}\mylabel{L03142h}  \normalsize

\doendnotes{C}
\bigskip
\vfill

\clearpage

\footnotesize

\lohead{\textsc{register}}

% Definiere theindex-Environment komplett neu ohne reledmac
\makeatletter
\renewenvironment{theindex}{%
  \section*{\indexname}%
  \setlength{\parindent}{0pt}%
  \setlength{\parskip}{0pt plus 0.3pt}%
  \let\item\@idxitem
}{%
  \clearpage
}
\makeatother

\IfFileExists{\jobname-pw.ind}{\input{\jobname-pw.ind}}{}

\end{document}

      