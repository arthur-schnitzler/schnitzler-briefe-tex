%% latex-leseansicht-vorspann.tex
%% Vorspann für die Leseansicht.
%% Lädt die gemeinsame Datei latex-vorspann.tex mit nicht gesetztem Schalter.

\newif\ifkorrekturansicht
\korrekturansichtfalse

\input{../tex-inputs/latex-vorspann}


\section[Berta Zuckerkandl an Olga Schnitzler, {{[}}21. 1. 1914?{{]}}]{L04010 Berta Zuckerkandl an Olga Schnitzler, {[}21. 1. 1914?{]}}
\nopagebreak\mylabel{L04010v}
\rehead{ }\normalsize\beginnumbering\briefempfaengerindex{Schnitzler, Olga@\textsc{Schnitzler, Olga}!zzzZuckerkandl, Berta@\emph{von Berta Zuckerkandl}!1914-01-211@{{[}21. 1. 1914?{]}}|(be}
\toendnotes[C]{\smallbreak\pagebreak[2]}
\correspDesc{Versand  durch Berta Zuckerkandl am [21. 1. 1914?] in Wien
\newline{}Erhalt  durch Olga Schnitzler im Zeitraum [21. 1. 1914 – 24. 1. 1914?] in Wien}\toendnotes[C]{\smallbreak}
\Standort{DLA, A:Schnitzler, HS.NZ85.1.5418/41.}
\physDesc{Brief, 1 Blatt, 3 Seiten, 536 Zeichen
\newline{}Handschrift: schwarze Tinte, lateinische Kurrent}\toendnotes[C]{\smallbreak}
\pstart
           \noindent{}{\pb}Liebste Olga! Nur ein Wort bevor Du abreist. Ich lasse
               Dich mit einem \uline{so guten Gefühl} weg. Es war \label{K_L04010-1v}\edtext{gestern}{\lemma{\textnormal{\emph{gestern}}}\Cendnote{\textnormal{Das Korrespondenzstück
                  ist nicht datiert. Aus dem Inhalt geht hervor, dass Olga Schnitzler\pwindex{Schnitzler, Olga 17.\,1.\,1882 Wien – 13.\,1.\,1970 Lugano@\textsc{Schnitzler, Olga} (17.\,1.\,1882 Wien – 13.\,1.\,1970 Lugano), \emph{Schauspielerin, Sängerin}|pwk} am Vorabend bei Felix\pwindex{Salten, Felix 6.\,9.\,1869 Budapest – 8.\,10.\,1945 Zürich@\textsc{Salten, Felix} (6.\,9.\,1869 Budapest – 8.\,10.\,1945 Zürich), \emph{Schriftsteller, Journalist, Chefredakteur}|pwk}\pwindex{Salten, Ottilie 7.\,3.\,1868 Prag – 22.\,6.\,1942 Zürich@\textsc{Salten, Ottilie} (7.\,3.\,1868 Prag – 22.\,6.\,1942 Zürich), \emph{Schauspielerin}|pwk} und Ottilie Salten\pwindex{Salten, Ottilie 7.\,3.\,1868 Prag – 22.\,6.\,1942 Zürich@\textsc{Salten, Ottilie} (7.\,3.\,1868 Prag – 22.\,6.\,1942 Zürich), \emph{Schauspielerin}|pwk} aufgetreten war und ein öffentlicher Auftritt in einer anderen
                  Stadt bevorstand. Das dürfte sich auf den Auftritt am 24. 1. 1914 in
                  Regensburg\oindex{Regensburg@\textbf{Regensburg}, \emph{Hauptstadt}|pwk} beziehen, den sie am 20. 1. 1914
                  bei Saltens\pwindex{Salten, Felix 6.\,9.\,1869 Budapest – 8.\,10.\,1945 Zürich@\textsc{Salten, Felix} (6.\,9.\,1869 Budapest – 8.\,10.\,1945 Zürich), \emph{Schriftsteller, Journalist, Chefredakteur}|pwk}\pwindex{Salten, Ottilie 7.\,3.\,1868 Prag – 22.\,6.\,1942 Zürich@\textsc{Salten, Ottilie} (7.\,3.\,1868 Prag – 22.\,6.\,1942 Zürich), \emph{Schauspielerin}|pwk} probte.}}}\label{K_L04010-1}
               wirklich Alles so ausgezeichnet. Sowol Material, wie seelische Disposition. {\pb}Glaube mir Du kannst ruhig sein, und gieb Dich dort nur so rückhaltlos
               aus wie gestern Abend – so wirst Du u musst Du Dir Dein Publikum erwerben.\pend
           
\pstart
           Fufi\pwindex{Zuckerkandl, Victor 2.\,7.\,1896 – 24.\,4.\,1965@\textsc{Zuckerkandl, Victor} (2.\,7.\,1896 – 24.\,4.\,1965), \emph{Dirigent}|pwu} dem
               ich’s er{\pb}zählte freut sich mit mir, u trägt mir sehr auf – Dir’s zu
               sagen. Er war sehr böse auf mich weil ich Salten’s\pwindex{Salten, Felix 6.\,9.\,1869 Budapest – 8.\,10.\,1945 Zürich@\textsc{Salten, Felix} (6.\,9.\,1869 Budapest – 8.\,10.\,1945 Zürich), \emph{Schriftsteller, Journalist, Chefredakteur}|pw}\pwindex{Salten, Ottilie 7.\,3.\,1868 Prag – 22.\,6.\,1942 Zürich@\textsc{Salten, Ottilie} (7.\,3.\,1868 Prag – 22.\,6.\,1942 Zürich), \emph{Schauspielerin}|pw} nicht bat ob er Dich hören ko{\geminationm}en
               dürfe.\pend
           
\pstart
           Also – i{\geminationm}er Vorwärts!! Du hast es in Dir.\pend
           \pstart Deine \spacefill\mbox{Berte.}\pend{}\selectlanguage{ngerman}\endnumbering\briefempfaengerindex{Schnitzler, Olga@\textsc{Schnitzler, Olga}!zzzZuckerkandl, Berta@\emph{von Berta Zuckerkandl}!1914-01-211@{{[}21. 1. 1914?{]}}|)be}\mylabel{L04010h}
\begin{anhang}
\end{anhang}\newcommand{\dateiname}{L04010}\newcommand{\titel}{Berta Zuckerkandl an Olga Schnitzler, [21. 1. 1914?]}\newcommand{\editorInnen}{Herausgegeben von Jahnke, SelmaMüller, Martin Anton}%% latex-leseansicht-abspann.tex
%% Abspann für die Leseansicht.
%% Der Schalter \ifkorrekturansicht ist bereits durch den Vorspann gesetzt.

%% latex-abspann.tex
%% Gemeinsamer Abspann für Korrekturansicht und Leseansicht.
%% Setzt den Schalter \ifkorrekturansicht voraus (gesetzt in den
%% einbindenden Dateien latex-korrekturansicht-abspann.tex bzw.
%% latex-leseansicht-abspann.tex).
%% ---------------------------------------------------------------

\normalsize

% Das esempio-Environment wird nur in der Leseansicht benötigt
\ifkorrekturansicht\else
\newenvironment{esempio}[3]%
{
    \vspace{1.5ex}
    \rlap{\underline{#1}}
    \par
    \setlength{\parindent}{0cm}
    \nopagebreak
    \leftskip=#2cm
    \rightskip=#3cm
}
{
    \par
}
\fi

\doendnotes{C}
\bigskip
\vfill

\clearpage

\footnotesize

\ifkorrekturansicht
  \lohead{\textsc{register}}
\fi

% theindex-Environment neu definieren ohne reledmac
\makeatletter
\renewenvironment{theindex}{%
  \ifkorrekturansicht
    \section*{\indexname}%
  \else
    \subsubsection*{Index der erwähnten Entitäten}%
  \fi
  \setlength{\parindent}{0pt}%
  \setlength{\parskip}{0pt plus 0.3pt}%
  \let\item\@idxitem
}{%
  \ifkorrekturansicht\clearpage\fi
}
\makeatother

\IfFileExists{\jobname-pw.ind}{\input{\jobname-pw.ind}}{}

% Quellenangabe nur in der Leseansicht
\ifkorrekturansicht\else
% Fallback-Definitionen, falls die .tex-Datei \titel etc. nicht gesetzt hat
\providecommand{\titel}{}
\providecommand{\editorInnen}{}
\providecommand{\dateiname}{\jobname}

\vspace{3cm}

\vfill

\footnotesize
\textsc{Quelle}: \titel. Herausgegeben von {\editorInnen}. In: \emph{Arthur Schnitzler: Briefwechsel mit Autorinnen und Autoren}.
 Digitale Edition, https://schnitzler-briefe.acdh.oeaw.ac.at/{\dateiname}.html (Stand \today)
\fi

\end{document}


