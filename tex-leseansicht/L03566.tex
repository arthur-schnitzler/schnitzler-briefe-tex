%% latex-korrekturansicht-vorspann.tex
%% Vorspann für die Korrekturansicht.
%% Lädt die gemeinsame Datei latex-vorspann.tex mit gesetztem Schalter.

\newif\ifkorrekturansicht
\korrekturansichttrue

\input{../tex-inputs/latex-vorspann}


\section[ Felix Salten an Arthur Schnitzler, 15. 5. 1917]{L03566 Felix Salten an Arthur Schnitzler, 15. 5. 1917}
\nopagebreak\mylabel{L03566v}
\rehead{ }\normalsize\beginnumbering\briefempfaengerindex{Schnitzler, Arthur@\textsc{Schnitzler, Arthur}!zzzSalten, Felix@\emph{von Felix Salten}!1917-05-151@{15. 5. 1917}|(be}
\toendnotes[C]{\smallbreak\pagebreak[2]}\Standort{CUL, Schnitzler, B 89, B 2.}
\physDesc{Brief, 1 Blatt, 1 Seite, 502 Zeichen
\newline{}Handschrift: schwarze Tinte, lateinische Kurrent
\newline{}Schnitzler: 1) mit Bleistift Vermerk: »\textsc{Salten}«  2) mit rotem Buntstift eine Unterstreichung
\newline{}Ordnung: 1) mit Bleistift von Frieda Pollak\pwindex{Pollak, Frieda 08.12.1881 – 13.07.1937@\textsc{Pollak, Frieda} (08.12.1881 – 13.07.1937), \emph{Sekretär/Sekretärin}|pw} (?) mit
                                 dem Buchstaben »A« (Abgeschrieben/Abschrift)
                                 gekennzeichnet  2) mit Bleistift von unbekannter Hand nummeriert: »279«}\toendnotes[C]{\smallbreak}
\pstart
           \raggedleft{}{\pb}Wien\oindex{Wien@\textbf{Wien}, \emph{A.ADM2}|pw}, 15. 5. 17\pend
           
\pstart{}Lieber,\pend\vspace{0.5em}
\pstart
           in Ergänzung der Einladung zu dem Vortrag des Schweizer\oindex{Schweiz@\textbf{Schweiz}, \emph{A.PCLI}|pw} Regierungsrates Wettstein\pwindex{Wettstein, Oscar 1866-03-26 – 1952-02-16@\textsc{Wettstein, Oscar} (1866-03-26 – 1952-02-16), \emph{Politiker/Politikerin, Journalist/Journalistin, Jurist/Juristin}|pw}
               am Samstag habe ich es übernommen, Sie auch zu dem
               kleinen Souper zu bitten, das Samstag{ }Abd. ½ 9 im Hotel Imperial\oindex{Hotel Imperial@\textbf{Hotel Imperial}, \emph{Hotel (K.HTL)}|pw} für Herrn Wettstein\pwindex{Wettstein, Oscar 1866-03-26 – 1952-02-16@\textsc{Wettstein, Oscar} (1866-03-26 – 1952-02-16), \emph{Politiker/Politikerin, Journalist/Journalistin, Jurist/Juristin}|pw}
               gegeben wird. Es ist wirklich nur ein kleines Souper (ohne Toaste). Ihre frdl. Zusage
               bitte ich Sie, an den Grafen Adolf Dubsky\pwindex{Dubsky-Třembomyslic, Adolf Oswald von 30.06.1878 – 16.11.1953@\textsc{Dubsky-Třembomyslic, Adolf Oswald von} (30.06.1878 – 16.11.1953), \emph{Diplomat/Diplomatin}|pw} im
                  Ministerium des Äußeren\orgindex{Ministerium fuer Aeusseres@Ministerium für Äußeres|pw} richten zu wollen.
               Hoffentlich \label{K_L03566-1v}\edtext{kommen Sie}{\lemma{\textnormal{\emph{kommen Sie}}}\Cendnote{\textnormal{Schnitzler kam nicht, vgl. Arthur Schnitzler an Felix Salten, 17. 5. 1917.}}}\label{K_L03566-1} sowol zu dem Vortrag, wie zum Souper.\pend
           
\pstart
           Herzliche Grüße von Haus zu Haus {\\[\baselineskip]}Ihr {\\[\baselineskip]}\spacefill\mbox{Felix Salten}\pend
           \leftskip=0em{}\selectlanguage{ngerman}\endnumbering\briefempfaengerindex{Schnitzler, Arthur@\textsc{Schnitzler, Arthur}!zzzSalten, Felix@\emph{von Felix Salten}!1917-05-151@{15. 5. 1917}|)be}\mylabel{L03566h}  \normalsize

\doendnotes{C}
\bigskip
\vfill

\clearpage

\footnotesize

\lohead{\textsc{register}}

% Definiere theindex-Environment komplett neu ohne reledmac
\makeatletter
\renewenvironment{theindex}{%
  \section*{\indexname}%
  \setlength{\parindent}{0pt}%
  \setlength{\parskip}{0pt plus 0.3pt}%
  \let\item\@idxitem
}{%
  \clearpage
}
\makeatother

\IfFileExists{\jobname-pw.ind}{\input{\jobname-pw.ind}}{}

\end{document}

      