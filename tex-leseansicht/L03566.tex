%% latex-leseansicht-vorspann.tex
%% Vorspann für die Leseansicht.
%% Lädt die gemeinsame Datei latex-vorspann.tex mit nicht gesetztem Schalter.

\newif\ifkorrekturansicht
\korrekturansichtfalse

\input{../tex-inputs/latex-vorspann}


\section[ Felix Salten an Arthur Schnitzler, 15. 5. 1917]{L03566 Felix Salten an Arthur Schnitzler,  15. 5. 1917}
\nopagebreak\mylabel{L03566v}
\rehead{ }\normalsize\beginnumbering\briefempfaengerindex{Schnitzler, Arthur@\textsc{Schnitzler, Arthur}!zzzSalten, Felix@\emph{von Felix Salten}!1917-05-151@{15. 5. 1917}|(be}
\toendnotes[C]{\smallbreak\pagebreak[2]}
\correspDesc{Versand  durch Felix Salten am 15. 5. 1917 in Wien
\newline{}Erhalt  durch Arthur Schnitzler im Zeitraum [15. 5. 1917
                  – 17. 5. 1917?] in Wien}\toendnotes[C]{\smallbreak}
\Standort{CUL, Schnitzler, B 89, B 2.}
\physDesc{Brief, 1 Blatt, 1 Seite, 502 Zeichen
\newline{}Handschrift: schwarze Tinte, lateinische Kurrent
\newline{}Schnitzler: 1) mit Bleistift Vermerk: »\textsc{Salten}«  2) mit rotem Buntstift eine Unterstreichung
\newline{}Ordnung: 1) mit Bleistift von Frieda Pollak\pwindex{Pollak, Frieda 8.\,12.\,1881 Wien – 13.\,7.\,1937 ebd.@\textsc{Pollak, Frieda} (8.\,12.\,1881 Wien – 13.\,7.\,1937 ebd.), \emph{Sekretärin}|pw} (?) mit
                                 dem Buchstaben »A« (Abgeschrieben/Abschrift)
                                 gekennzeichnet  2) mit Bleistift von unbekannter Hand nummeriert: »279«}\toendnotes[C]{\smallbreak}
\pstart
           \raggedleft{}{\pb}Wien\oindex{Wien@\textbf{Wien}, \emph{Verwaltungsgebiet}|pw}, 15. 5. 17\pend
           
\pstart{}Lieber,\pend\vspace{0.5em}
\pstart
           in Ergänzung der Einladung zu dem Vortrag des Schweizer\oindex{Schweiz@\textbf{Schweiz}|pw} Regierungsrates Wettstein\pwindex{Wettstein, Oscar 26.\,3.\,1866 Zürich – 16.\,2.\,1952 ebd.@\textsc{Wettstein, Oscar} (26.\,3.\,1866 Zürich – 16.\,2.\,1952 ebd.), \emph{Politiker, Journalist, Jurist}|pw}
               am Samstag habe ich es übernommen, Sie auch zu dem
               kleinen Souper zu bitten, das Samstag{ }Abd. ½ 9 im Hotel Imperial\oindex{Wien@\textbf{Wien}!I., Innere Stadt@\textbf{I., Innere Stadt}!Hotel Imperial@\textbf{Hotel Imperial}, \emph{Hotel}|pw} für Herrn Wettstein\pwindex{Wettstein, Oscar 26.\,3.\,1866 Zürich – 16.\,2.\,1952 ebd.@\textsc{Wettstein, Oscar} (26.\,3.\,1866 Zürich – 16.\,2.\,1952 ebd.), \emph{Politiker, Journalist, Jurist}|pw}
               gegeben wird. Es ist wirklich nur ein kleines Souper (ohne Toaste). Ihre frdl. Zusage
               bitte ich Sie, an den Grafen Adolf Dubsky\pwindex{Dubsky-Třembomyslic, Adolf Oswald von 30.\,6.\,1878 Athen – 16.\,11.\,1953 Wasserburg@\textsc{Dubsky-Třembomyslic, Adolf Oswald von} (30.\,6.\,1878 Athen – 16.\,11.\,1953 Wasserburg), \emph{Diplomat}|pw} im
                  Ministerium des Äußeren\orgindex{Ministerium für Äußeres@Ministerium für Äußeres|pw} richten zu wollen.
               Hoffentlich \label{K_L03566-1v}\edtext{kommen Sie}{\lemma{\textnormal{\emph{kommen Sie}}}\Cendnote{\textnormal{Schnitzler kam nicht, vgl. XXXX Auszeichnungsfehler: Dokument L03019 nicht gefunden.}}}\label{K_L03566-1} sowol zu dem Vortrag, wie zum Souper.\pend
           
\pstart
           Herzliche Grüße von Haus zu Haus {\\[\baselineskip]}Ihr {\\[\baselineskip]}\spacefill\mbox{Felix Salten}\pend
           \leftskip=0em{}\selectlanguage{ngerman}\endnumbering\briefempfaengerindex{Schnitzler, Arthur@\textsc{Schnitzler, Arthur}!zzzSalten, Felix@\emph{von Felix Salten}!1917-05-151@{15. 5. 1917}|)be}\mylabel{L03566h}  \newcommand{\dateiname}{L03566}\newcommand{\titel}{Felix Salten an Arthur Schnitzler, 15. 5. 1917}\newcommand{\editorInnen}{Martin Anton Müller und Laura Untner}%% latex-leseansicht-abspann.tex
%% Abspann für die Leseansicht.
%% Der Schalter \ifkorrekturansicht ist bereits durch den Vorspann gesetzt.

%% latex-abspann.tex
%% Gemeinsamer Abspann für Korrekturansicht und Leseansicht.
%% Setzt den Schalter \ifkorrekturansicht voraus (gesetzt in den
%% einbindenden Dateien latex-korrekturansicht-abspann.tex bzw.
%% latex-leseansicht-abspann.tex).
%% ---------------------------------------------------------------

\normalsize

% Das esempio-Environment wird nur in der Leseansicht benötigt
\ifkorrekturansicht\else
\newenvironment{esempio}[3]%
{
    \vspace{1.5ex}
    \rlap{\underline{#1}}
    \par
    \setlength{\parindent}{0cm}
    \nopagebreak
    \leftskip=#2cm
    \rightskip=#3cm
}
{
    \par
}
\fi

\doendnotes{C}
\bigskip
\vfill

\clearpage

\footnotesize

\ifkorrekturansicht
  \lohead{\textsc{register}}
\fi

% theindex-Environment neu definieren ohne reledmac
\makeatletter
\renewenvironment{theindex}{%
  \ifkorrekturansicht
    \section*{\indexname}%
  \else
    \subsubsection*{Index der erwähnten Entitäten}%
  \fi
  \setlength{\parindent}{0pt}%
  \setlength{\parskip}{0pt plus 0.3pt}%
  \let\item\@idxitem
}{%
  \ifkorrekturansicht\clearpage\fi
}
\makeatother

\IfFileExists{\jobname-pw.ind}{\input{\jobname-pw.ind}}{}

% Quellenangabe nur in der Leseansicht
\ifkorrekturansicht\else
% Fallback-Definitionen, falls die .tex-Datei \titel etc. nicht gesetzt hat
\providecommand{\titel}{}
\providecommand{\editorInnen}{}
\providecommand{\dateiname}{\jobname}

\vspace{3cm}

\vfill

\footnotesize
\textsc{Quelle}: \titel. Herausgegeben von {\editorInnen}. In: \emph{Arthur Schnitzler: Briefwechsel mit Autorinnen und Autoren}.
 Digitale Edition, https://schnitzler-briefe.acdh.oeaw.ac.at/{\dateiname}.html (Stand \today)
\fi

\end{document}


