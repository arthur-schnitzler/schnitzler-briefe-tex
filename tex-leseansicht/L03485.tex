%% latex-leseansicht-vorspann.tex
%% Vorspann für die Leseansicht.
%% Lädt die gemeinsame Datei latex-vorspann.tex mit nicht gesetztem Schalter.

\newif\ifkorrekturansicht
\korrekturansichtfalse

\input{../tex-inputs/latex-vorspann}

\begin{center}
            \textcolor{red}{ENTWURF, NICHT FERTIG KORRIGIERT}
                      \end{center}
            
         
         \renewcommand{\erwaehntePersonen}{Personen: Marco Brociner, Ludwig Klinenberger}
         \renewcommand{\erwaehnteInstitutionen}{Institutionen: Berliner Tageblatt}
         \renewcommand{\erwaehnteOrte}{Orte: Berlin, Wien}
         \renewcommand{\erwaehnteWerke}{Werke: Arthur Schnitzlers »Haus Delorme«, Berliner Tageblatt, Das Haus Delorme. Eine Familienszene, Haus Delorme. (Eine Richtigstellung von Arthur Schnitzler.), Neue Freie Presse, Neues Wiener Tagblatt, Schnitzlers »Haus Delorme«}
               \section[Paul Goldmann an Arthur Schnitzler, {[}23. 11. 1904?{]}]{ Paul Goldmann an Arthur Schnitzler, {[}23. 11. 1904?{]}}\nopagebreak\mylabel{v}\rehead{ }\begin{ledgroupsized}[t]{13cm}\normalsize\beginnumbering \toendnotes[C]{\smallbreak\pagebreak[2]} \Standort{DLA, A:Schnitzler, HS.NZ85.1.3176.}
\physDesc{Brief, 1 Blatt, 1 Seite, 617 Zeichen
\newline{}Handschrift: Bleistift, deutsche Kurrent
\newline{}Schnitzler: mit Bleistift »XXXX« vermerkt und »daß
                                    eine Freigabe \strikeout{nichts}{ }Dienſtag nicht erfolgen
                                       k\textcolor{gray}{ö}nne« vor »Nur aus dieſem
                                    Grunde« hinzugefügt }\toendnotes[C]{\smallbreak}\pstart
           \noindent{}{\pb}\strikeout{\textcolor{gray}{E}}{ }Sehr geehrte \label{K_L03485-1v}\edtext{Redaktion\orgindex{Berliner Tageblatt@Berliner Tageblatt|pwv}}{\lemma{\textnormal{\emph{Redaktion}}}\Cendnote{\textnormal{Bezug auf folgenden Artikel\pwindex{?? Werk@Nicht ermittelte Verfasserinnen und Verfasser!Schnitzlers »Haus Delorme«1904-11-22@\emph{Schnitzlers »Haus Delorme«} {[}1904-11-22{]}|pwkv} im \emph{Berliner Tageblatt}\pwindex{?? Werk@Nicht ermittelte Verfasserinnen und Verfasser!Berliner Tageblatt1872 – 1939@\emph{Berliner Tageblatt} {[}1872 – 1939{]}|pwk} über die Zensurierung des Einakters
                        \emph{Das Haus Delorme}\pwindex{Schnitzler, Arthur 15.05.1862 – 21.10.1931@\textsc{Schnitzler, Arthur} (15.05.1862 – 21.10.1931), \emph{Schriftsteller, Mediziner}!Haus Delorme. Eine Familienszene1977@\strich\emph{Das Haus Delorme. Eine Familienszene} {[}1977{]}|pwk} am 19. 11. 1904, den Schnitzler\pwindex{Schnitzler, Arthur 15.05.1862 – 21.10.1931@\textsc{Schnitzler, Arthur} (15.05.1862 – 21.10.1931), \emph{Schriftsteller, Mediziner}|pwk} zur Kenntnis genommen hatte (vgl. A. S.: \emph{Tagebuch}, 22. 11. 1904): [O. V.]: \emph{Schnitzlers »Haus Delorme«}\pwindex{?? Werk@Nicht ermittelte Verfasserinnen und Verfasser!Schnitzlers »Haus Delorme«1904-11-22@\emph{Schnitzlers »Haus Delorme«} {[}1904-11-22{]}|pwk}. In:
                           \emph{Berliner Tageblatt und
                           -Handelszeitung}\pwindex{?? Werk@Nicht ermittelte Verfasserinnen und Verfasser!Berliner Tageblatt1872 – 1939@\emph{Berliner Tageblatt} {[}1872 – 1939{]}|pwk}, Jg. 33, Nr. 595, 22. 11. 1904, Abend-Ausgabe, S. 2. Womöglich entstand das
                     vorliegende Schreiben am 23. 11. 1904 als Schnitzler\pwindex{Schnitzler, Arthur 15.05.1862 – 21.10.1931@\textsc{Schnitzler, Arthur} (15.05.1862 – 21.10.1931), \emph{Schriftsteller, Mediziner}|pwk}
                     bei Goldmann\pwindex{Goldmann, Paul 31.01.1865 – 25.09.1935@\textsc{Goldmann, Paul} (31.01.1865 – 25.09.1935), \emph{Schriftsteller, Journalist}|pwk} zu Besuch war und wurde Schnitzler\pwindex{Schnitzler, Arthur 15.05.1862 – 21.10.1931@\textsc{Schnitzler, Arthur} (15.05.1862 – 21.10.1931), \emph{Schriftsteller, Mediziner}|pwk} postalisch nach Wien\oindex{Wien@\textbf{Wien}|pwk} nachgeschickt. Wahrscheinlich ist auch,
                     dass es eine Art Vorbereitung auf Schnitzler\pwindex{Schnitzler, Arthur 15.05.1862 – 21.10.1931@\textsc{Schnitzler, Arthur} (15.05.1862 – 21.10.1931), \emph{Schriftsteller, Mediziner}|pwk}s offizielle Stellungnahme zu der im \emph{Berliner Tageblatt}\pwindex{?? Werk@Nicht ermittelte Verfasserinnen und Verfasser!Berliner Tageblatt1872 – 1939@\emph{Berliner Tageblatt} {[}1872 – 1939{]}|pwk} abgedruckten Meldung\pwindex{?? Werk@Nicht ermittelte Verfasserinnen und Verfasser!Schnitzlers »Haus Delorme«1904-11-22@\emph{Schnitzlers »Haus Delorme«} {[}1904-11-22{]}|pwkv} war, die in der \emph{Neuen Freien Presse}\pwindex{Neue Freie Presse1864 – 1939@\emph{Neue Freie Presse} {[}1864 – 1939{]}|pwk} und im \emph{Neuen Wiener Tagblatt}\pwindex{?? Werk@Nicht ermittelte Verfasserinnen und Verfasser!Neues Wiener Tagblatt1867 – 1945@\emph{Neues Wiener Tagblatt} {[}1867 – 1945{]}|pwk} abgedruckt wurde: [Ludwig Klinenberger\pwindex{Klinenberger, Ludwig 16.09.1873 – 27.04.1942@\textsc{Klinenberger, Ludwig} (16.09.1873 – 27.04.1942), \emph{Journalist}|pwk}]: \emph{Arthur Schnitzlers »Haus Delorme«}\pwindex{Arthur Schnitzlers »Haus Delorme«1904-11-25@\emph{Arthur Schnitzlers »Haus Delorme«} {[}1904-11-25{]}|pwk}. In: \emph{Neue Freie Presse}\pwindex{Neue Freie Presse1864 – 1939@\emph{Neue Freie Presse} {[}1864 – 1939{]}|pwk}, Nr. 14.460, 25. 11. 1904, S. 12; [Marco Brociner\pwindex{Brociner, Marco 20.10.1852 – 12.04.1942@\textsc{Brociner, Marco} (20.10.1852 – 12.04.1942), \emph{Schriftsteller, Journalist, Kritiker}|pwk}]: \emph{Haus Delorme. (Eine Richtigstellung von Arthur
                           Schnitzler.)}\pwindex{Haus Delorme. (Eine Richtigstellung von Arthur Schnitzler.)1904-11-25@\emph{Haus Delorme. (Eine Richtigstellung von Arthur Schnitzler.)} {[}1904-11-25{]}|pwk}. In: \emph{Neues Wiener
                           Tagblatt}\pwindex{?? Werk@Nicht ermittelte Verfasserinnen und Verfasser!Neues Wiener Tagblatt1867 – 1945@\emph{Neues Wiener Tagblatt} {[}1867 – 1945{]}|pwk}, Jg. 38, Nr. 327, 25. 11. 1904, S. 13. Siehe dazu auch A. S.: \emph{Tagebuch}, 25. 11. 1904.}}}\label{K_L03485-1h}, Geſtatten
               Sie mir, zur Richtigſtellung der Meldungen\pwindex{?? Werk@Nicht ermittelte Verfasserinnen und Verfasser!Schnitzlers »Haus Delorme«1904-11-22@\emph{Schnitzlers »Haus Delorme«} {[}1904-11-22{]}|pwv}, die Sie geſtern bezüglich \strikeout{d} meines noch unveröffentlichten Einakters »Das Haus Delorme\pwindex{Schnitzler, Arthur 15.05.1862 – 21.10.1931@\textsc{Schnitzler, Arthur} (15.05.1862 – 21.10.1931), \emph{Schriftsteller, Mediziner}!Haus Delorme. Eine Familienszene1977@\strich\emph{Das Haus Delorme. Eine Familienszene} {[}1977{]}|pw}« publizirt haben, Ihnen
               Folgendes mitzutheilen: \strikeout{Es iſt
                  \textcolor{gray}{manc}he} Es entſpricht nicht den Thatſachen, daß die
                  \label{K_L03485-2v}\edtext{Schauſpieler}{\lemma{\textnormal{\emph{Schauſpieler}}}\Cendnote{\textnormal{nicht namentlich genannt, es ist nur von »der Regie und den mitwirkenden
                     Künstlern\pwindex{?? Werk@Nicht ermittelte Verfasserinnen und Verfasser!Schnitzlers »Haus Delorme«1904-11-22@\emph{Schnitzlers »Haus Delorme«} {[}1904-11-22{]}|pwkv}« die Rede}}}\label{K_L03485-2h} ſich geweigert haben, \strikeout{daß} das Stück\pwindex{Schnitzler, Arthur 15.05.1862 – 21.10.1931@\textsc{Schnitzler, Arthur} (15.05.1862 – 21.10.1931), \emph{Schriftsteller, Mediziner}!Haus Delorme. Eine Familienszene1977@\strich\emph{Das Haus Delorme. Eine Familienszene} {[}1977{]}|pwv} zu
               ſpielen. Freitag war noch Probe. \strikeout{Abends infolge die das Cenſur} Am Freitag{ }Abend, vor der auf Sonnabend angeſetzten
               Generalprobe, \strikeout{\textcolor{gray}{er}} erfolgte \substVorne{}\textsuperscript{das Cenſurverbot}{\allowbreak}\substDazwischen{}die Meldung von Seiten der Cenſur\substHinten{}. Nur aus dieſem Grunde wurde das Stück\pwindex{Schnitzler, Arthur 15.05.1862 – 21.10.1931@\textsc{Schnitzler, Arthur} (15.05.1862 – 21.10.1931), \emph{Schriftsteller, Mediziner}!Haus Delorme. Eine Familienszene1977@\strich\emph{Das Haus Delorme. Eine Familienszene} {[}1977{]}|pwv} abgeſetzt. Der Inhalt des Stück\pwindex{Schnitzler, Arthur 15.05.1862 – 21.10.1931@\textsc{Schnitzler, Arthur} (15.05.1862 – 21.10.1931), \emph{Schriftsteller, Mediziner}!Haus Delorme. Eine Familienszene1977@\strich\emph{Das Haus Delorme. Eine Familienszene} {[}1977{]}|pwv}es iſt in \strikeout{der Ihrem Blatte\pwindex{?? Werk@Nicht ermittelte Verfasserinnen und Verfasser!Berliner Tageblatt1872 – 1939@\emph{Berliner Tageblatt} {[}1872 – 1939{]}|pw}} Ihrem Berichte\pwindex{?? Werk@Nicht ermittelte Verfasserinnen und Verfasser!Schnitzlers »Haus Delorme«1904-11-22@\emph{Schnitzlers »Haus Delorme«} {[}1904-11-22{]}|pwv}
               unrichtig wiedergegeben.\pend
           
         
         \endnumbering\mylabel{h}\end{ledgroupsized}\begin{anhang}\end{anhang}\newcommand{\dateiname}{L03485}\newcommand{\titel}{Paul Goldmann an Arthur Schnitzler, [23. 11. 1904?]}\newcommand{\editorInnen}{Martin Anton Müller und Laura Untner}%% latex-leseansicht-abspann.tex
%% Abspann für die Leseansicht.
%% Der Schalter \ifkorrekturansicht ist bereits durch den Vorspann gesetzt.

%% latex-abspann.tex
%% Gemeinsamer Abspann für Korrekturansicht und Leseansicht.
%% Setzt den Schalter \ifkorrekturansicht voraus (gesetzt in den
%% einbindenden Dateien latex-korrekturansicht-abspann.tex bzw.
%% latex-leseansicht-abspann.tex).
%% ---------------------------------------------------------------

\normalsize

% Das esempio-Environment wird nur in der Leseansicht benötigt
\ifkorrekturansicht\else
\newenvironment{esempio}[3]%
{
    \vspace{1.5ex}
    \rlap{\underline{#1}}
    \par
    \setlength{\parindent}{0cm}
    \nopagebreak
    \leftskip=#2cm
    \rightskip=#3cm
}
{
    \par
}
\fi

\doendnotes{C}
\bigskip
\vfill

\clearpage

\footnotesize

\ifkorrekturansicht
  \lohead{\textsc{register}}
\fi

% theindex-Environment neu definieren ohne reledmac
\makeatletter
\renewenvironment{theindex}{%
  \ifkorrekturansicht
    \section*{\indexname}%
  \else
    \subsubsection*{Index der erwähnten Entitäten}%
  \fi
  \setlength{\parindent}{0pt}%
  \setlength{\parskip}{0pt plus 0.3pt}%
  \let\item\@idxitem
}{%
  \ifkorrekturansicht\clearpage\fi
}
\makeatother

\IfFileExists{\jobname-pw.ind}{\input{\jobname-pw.ind}}{}

% Quellenangabe nur in der Leseansicht
\ifkorrekturansicht\else
% Fallback-Definitionen, falls die .tex-Datei \titel etc. nicht gesetzt hat
\providecommand{\titel}{}
\providecommand{\editorInnen}{}
\providecommand{\dateiname}{\jobname}

\vspace{3cm}

\vfill

\footnotesize
\textsc{Quelle}: \titel. Herausgegeben von {\editorInnen}. In: \emph{Arthur Schnitzler: Briefwechsel mit Autorinnen und Autoren}.
 Digitale Edition, https://schnitzler-briefe.acdh.oeaw.ac.at/{\dateiname}.html (Stand \today)
\fi

\end{document}


      