%% latex-korrekturansicht-vorspann.tex
%% Vorspann für die Korrekturansicht.
%% Lädt die gemeinsame Datei latex-vorspann.tex mit gesetztem Schalter.

\newif\ifkorrekturansicht
\korrekturansichttrue

\input{../tex-inputs/latex-vorspann}


\section[Arthur Schnitzler an Hugo Hofmannsthal, 3. 9. 1923]{L02403 Arthur Schnitzler an Hugo Hofmannsthal, 3. 9. 1923}
\nopagebreak\mylabel{L02403v}
\rehead{ }\normalsize\beginnumbering\briefempfaengerindex{Hofmannsthal, Hugo von@\textsc{Hofmannsthal, Hugo von}!zzzSchnitzler, Arthur@\emph{von Arthur Schnitzler}!1923-09-031@{3. 9. 1923}|(be}
\toendnotes[C]{\smallbreak\pagebreak[2]}\Standort{FDH, Hs-30885,150.}
\physDesc{Postkarte, 491 Zeichen
\newline{}Handschrift: Bleistift, lateinische Kurrent
\newline{}Versand: 1) nachgesandt nach Bad-Aussee\oindex{Bad Aussee@\textbf{Bad Aussee}, \emph{P.PPLA3}|pw}  2) Stempel: »\nobreak{}\oindex{Celerina@\textbf{Celerina}, \emph{P.PPL}|pwk}Celerina (Graubünden), 3. IX. 23, 12\nobreak{}«.  3) Stempel: »\nobreak{}\oindex{Rodaun@\textbf{Rodaun}, \emph{A.ADM4}|pwk}Rodaun, \textcolor{gray}{6} 9 23\nobreak{}«. }
\buchAbdrucke{\weitereDrucke{Hugo von Hofmannsthal, Arthur Schnitzler: \emph{Briefwechsel}. Frankfurt am Main: \emph{S. Fischer} 1964, S. 298.} }\toendnotes[C]{\smallbreak}\pstart{}{\pb}Herrn\pend{}\pstart{}Dr. Hugo von Hofmannsthal\pend{}\pstart{}Rodaun\oindex{Rodaun@\textbf{Rodaun}, \emph{A.ADM4}|pw}\pend{}\pstart{}bei Wien\oindex{Wien@\textbf{Wien}, \emph{A.ADM2}|pw}\pend{}\pstart{}(Südbahn\oindex{Suedbahnhof@\textbf{Südbahnhof}, \emph{Bahnhofsgebäude (K.BHF)}|pw}{[}){]}\pend{}\pstart{}NiederOesterreich\oindex{Niederoesterreich@\textbf{Niederösterreich}, \emph{A.ADM1}|pw}\pend{}{\bigskip}\vspace{1em}
\pstart
           \centering{}\textcolor{gray}{\textbf{{\pb}Celerina\oindex{Celerina@\textbf{Celerina}, \emph{P.PPL}|pw}}}\pend
           
\pstart
           \raggedleft{}{\pb}3/9. 23.\pend
           \vspace{0.5em}
\pstart
           mein lieber Hugo – Ihr letztes Lebenszeichen hab ich vor \label{K_L02403-1v}\edtext{Monaten}{\lemma{\textnormal{\emph{Monaten}}}\Cendnote{\textnormal{Siehe Hugo Hofmannsthal an Arthur Schnitzler, 15. 5. 1923.
               }}}\label{K_L02403-1} aus der Schweiz\oindex{Schweiz@\textbf{Schweiz}, \emph{A.PCLI}|pw} erhalten – und heut
               erst, auch aus der Schweiz\oindex{Schweiz@\textbf{Schweiz}, \emph{A.PCLI}|pw}, aus Celerina\oindex{Celerina@\textbf{Celerina}, \emph{P.PPL}|pw}, wo mich vor 9 Jahren der Krieg
               überrascht hat un\textcolor{gray}{d} ich \introOben{}heuer\introOben{} ein paar
               gute Wochen allein verlebt habe, {\pb}erwider ich Ihren
               lieben Gruſs. Heute reis ich ab, seh mir noch im Engadin\oindex{Engadin@\textbf{Engadin}, \emph{T.VAL}|pw} einiges an, und geh da{\geminationn} an den Bodensee\oindex{Bodensee@\textbf{Bodensee}, \emph{H.LK}|pw} (Bregenz\oindex{Bregenz@\textbf{Bregenz}, \emph{P.PPLA}|pw}), von wo ich Lili\pwindex{Cappellini, Lili 13.09.1909 – 26.07.1928@\textsc{Cappellini, Lili} (13.09.1909 – 26.07.1928)|pw} abhole. Auf
               Wiedersehen hoffentlich! \pend
           \pstart Ihr \spacefill\mbox{Arthur}\pend{}\selectlanguage{ngerman}\endnumbering\briefempfaengerindex{Hofmannsthal, Hugo von@\textsc{Hofmannsthal, Hugo von}!zzzSchnitzler, Arthur@\emph{von Arthur Schnitzler}!1923-09-031@{3. 9. 1923}|)be}\mylabel{L02403h}  \normalsize

\doendnotes{C}
\bigskip
\vfill

\clearpage

\footnotesize

\lohead{\textsc{register}}

% Definiere theindex-Environment komplett neu ohne reledmac
\makeatletter
\renewenvironment{theindex}{%
  \section*{\indexname}%
  \setlength{\parindent}{0pt}%
  \setlength{\parskip}{0pt plus 0.3pt}%
  \let\item\@idxitem
}{%
  \clearpage
}
\makeatother

\IfFileExists{\jobname-pw.ind}{\input{\jobname-pw.ind}}{}

\end{document}

      