%% latex-leseansicht-vorspann.tex
%% Vorspann für die Leseansicht.
%% Lädt die gemeinsame Datei latex-vorspann.tex mit nicht gesetztem Schalter.

\newif\ifkorrekturansicht
\korrekturansichtfalse

\input{../tex-inputs/latex-vorspann}


\section[Arthur Schnitzler an Hugo Hofmannsthal, 3. 9. 1923]{L02403 Arthur Schnitzler an Hugo Hofmannsthal, 3. 9. 1923}
\nopagebreak\mylabel{L02403v}
\rehead{ }\normalsize\beginnumbering\briefempfaengerindex{Hofmannsthal, Hugo von@\textsc{Hofmannsthal, Hugo von}!zzzSchnitzler, Arthur@\emph{von Arthur Schnitzler}!1923-09-031@{3. 9. 1923}|(be}
\toendnotes[C]{\smallbreak\pagebreak[2]}
\correspDesc{Versand  durch Arthur Schnitzler am 3. 9. 1923 in Celerina
\newline{}Weiterleitung  am 5. 9. 1923 in Rodaun
\newline{}Erhalt  durch Hugo von Hofmannsthal im Zeitraum [6. 9. 1923
                  – 8. 9. 1923?] in Bad Aussee}\toendnotes[C]{\smallbreak}
\Standort{FDH, Hs-30885,150.}
\physDesc{Postkarte, 491 Zeichen
\newline{}Handschrift: Bleistift, lateinische Kurrent
\newline{}Versand: 1) nachgesandt nach Bad-Aussee\oindex{Bad Aussee@\textbf{Bad Aussee}, \emph{Hauptstadt}|pw}  2) Stempel: »\nobreak{}\oindex{Celerina@\textbf{Celerina}|pwk}Celerina (Graubünden), 3. IX. 23, 12\nobreak{}«.  3) Stempel: »\nobreak{}\oindex{Wien@\textbf{Wien}!XXIII., Liesing@\textbf{XXIII., Liesing}!Rodaun@\textbf{Rodaun}, \emph{Region}|pwk}Rodaun, \textcolor{gray}{6} 9 23\nobreak{}«. }
\buchAbdrucke{\weitereDrucke{Hugo von Hofmannsthal, Arthur Schnitzler: \emph{Briefwechsel}. Herausgegeben von Therese Nickl und Heinrich Schnitzler. Frankfurt am Main: \emph{S. Fischer} 1964, S. 298.} }\toendnotes[C]{\smallbreak}\pstart{}{\pb}Herrn\pend{}\pstart{}Dr. Hugo von Hofmannsthal\pend{}\pstart{}Rodaun\oindex{Wien@\textbf{Wien}!XXIII., Liesing@\textbf{XXIII., Liesing}!Rodaun@\textbf{Rodaun}, \emph{Region}|pw}\pend{}\pstart{}bei Wien\oindex{Wien@\textbf{Wien}, \emph{Verwaltungsgebiet}|pw}\pend{}\pstart{}(Südbahn\oindex{Wien@\textbf{Wien}!X., Favoriten@\textbf{X., Favoriten}!Südbahnhof@\textbf{Südbahnhof}, \emph{Bahnhofsgebäude}|pw}{[}){]}\pend{}\pstart{}NiederOesterreich\oindex{Niederösterreich@\textbf{Niederösterreich}, \emph{Land}|pw}\pend{}{\bigskip}\vspace{1em}
\pstart
           \centering{}\textcolor{gray}{\textbf{{\pb}Celerina\oindex{Celerina@\textbf{Celerina}|pw}}}\pend
           
\pstart
           \raggedleft{}{\pb}3/9. 23.\pend
           \vspace{0.5em}
\pstart
           mein lieber Hugo – Ihr letztes Lebenszeichen hab ich vor \label{K_L02403-1v}\edtext{Monaten}{\lemma{\textnormal{\emph{Monaten}}}\Cendnote{\textnormal{Siehe XXXX Auszeichnungsfehler: Dokument L02399 nicht gefunden.
               }}}\label{K_L02403-1} aus der Schweiz\oindex{Schweiz@\textbf{Schweiz}|pw} erhalten – und heut
               erst, auch aus der Schweiz\oindex{Schweiz@\textbf{Schweiz}|pw}, aus Celerina\oindex{Celerina@\textbf{Celerina}|pw}, wo mich vor 9 Jahren der Krieg
               überrascht hat un\textcolor{gray}{d} ich \introOben{}heuer\introOben{} ein paar
               gute Wochen allein verlebt habe, {\pb}erwider ich Ihren
               lieben Gruſs. Heute reis ich ab, seh mir noch im Engadin\oindex{Engadin@\textbf{Engadin}, \emph{Tal}|pw} einiges an, und geh da{\geminationn} an den Bodensee\oindex{Bodensee@\textbf{Bodensee}, \emph{See}|pw} (Bregenz\oindex{Bregenz@\textbf{Bregenz}|pw}), von wo ich Lili\pwindex{Cappellini, Lili 13.\,9.\,1909 Wien – 26.\,7.\,1928 Venedig@\textsc{Cappellini, Lili} (13.\,9.\,1909 Wien – 26.\,7.\,1928 Venedig)|pw} abhole. Auf
               Wiedersehen hoffentlich!\pend
           \pstart Ihr \spacefill\mbox{Arthur}\pend{}\selectlanguage{ngerman}\endnumbering\briefempfaengerindex{Hofmannsthal, Hugo von@\textsc{Hofmannsthal, Hugo von}!zzzSchnitzler, Arthur@\emph{von Arthur Schnitzler}!1923-09-031@{3. 9. 1923}|)be}\mylabel{L02403h}  \newcommand{\dateiname}{L02403}\newcommand{\titel}{Arthur Schnitzler an Hugo Hofmannsthal, 3. 9. 1923}\newcommand{\editorInnen}{Martin Anton Müller und Gerd-Hermann Susen}%% latex-leseansicht-abspann.tex
%% Abspann für die Leseansicht.
%% Der Schalter \ifkorrekturansicht ist bereits durch den Vorspann gesetzt.

%% latex-abspann.tex
%% Gemeinsamer Abspann für Korrekturansicht und Leseansicht.
%% Setzt den Schalter \ifkorrekturansicht voraus (gesetzt in den
%% einbindenden Dateien latex-korrekturansicht-abspann.tex bzw.
%% latex-leseansicht-abspann.tex).
%% ---------------------------------------------------------------

\normalsize

% Das esempio-Environment wird nur in der Leseansicht benötigt
\ifkorrekturansicht\else
\newenvironment{esempio}[3]%
{
    \vspace{1.5ex}
    \rlap{\underline{#1}}
    \par
    \setlength{\parindent}{0cm}
    \nopagebreak
    \leftskip=#2cm
    \rightskip=#3cm
}
{
    \par
}
\fi

\doendnotes{C}
\bigskip
\vfill

\clearpage

\footnotesize

\ifkorrekturansicht
  \lohead{\textsc{register}}
\fi

% theindex-Environment neu definieren ohne reledmac
\makeatletter
\renewenvironment{theindex}{%
  \ifkorrekturansicht
    \section*{\indexname}%
  \else
    \subsubsection*{Index der erwähnten Entitäten}%
  \fi
  \setlength{\parindent}{0pt}%
  \setlength{\parskip}{0pt plus 0.3pt}%
  \let\item\@idxitem
}{%
  \ifkorrekturansicht\clearpage\fi
}
\makeatother

\IfFileExists{\jobname-pw.ind}{\input{\jobname-pw.ind}}{}

% Quellenangabe nur in der Leseansicht
\ifkorrekturansicht\else
% Fallback-Definitionen, falls die .tex-Datei \titel etc. nicht gesetzt hat
\providecommand{\titel}{}
\providecommand{\editorInnen}{}
\providecommand{\dateiname}{\jobname}

\vspace{3cm}

\vfill

\footnotesize
\textsc{Quelle}: \titel. Herausgegeben von {\editorInnen}. In: \emph{Arthur Schnitzler: Briefwechsel mit Autorinnen und Autoren}.
 Digitale Edition, https://schnitzler-briefe.acdh.oeaw.ac.at/{\dateiname}.html (Stand \today)
\fi

\end{document}


