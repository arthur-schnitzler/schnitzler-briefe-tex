%% latex-leseansicht-vorspann.tex
%% Vorspann für die Leseansicht.
%% Lädt die gemeinsame Datei latex-vorspann.tex mit nicht gesetztem Schalter.

\newif\ifkorrekturansicht
\korrekturansichtfalse

\input{../tex-inputs/latex-vorspann}

\begin{center}
            \textcolor{red}{ENTWURF. ENTZIFFERUNG NOCH NICHT KORREKTURGELESEN}
                      \end{center}
            
               \section[Arthur Schnitzler an Hugo Hofmannsthal, 3. 9. 1923]{ Arthur Schnitzler an Hugo Hofmannsthal, 3. 9. 1923}\nopagebreak\mylabel{v}\rehead{ }\begin{ledgroupsized}[t]{13cm}\normalsize\beginnumbering\briefempfaengerindex{Hofmannsthal, Hugo von@\textsc{Hofmannsthal, Hugo von}!zzzSchnitzler, Arthur@\emph{von Arthur Schnitzler}!1923-09-031@{3. 9. 1923}|(be} \toendnotes[C]{\smallbreak\pagebreak[2]} \Standort{FDH, Hs-30885,150.}
\physDesc{Postkarte
\newline{}Handschrift: Bleistift, lateinische Kurrent\newline{}Versand: 1) nachgesandt nach Bad-Aussee\oindex{Bad Aussee@\textbf{Bad Aussee}|pw} 2) Stempel: »\nobreak{}\oindex{Celerina@\textbf{Celerina}|pwk}Celerina (Graubünden), 3. IX. 23, 12\nobreak{}«. 3) Stempel: »\nobreak{}\oindex{Rodaun@\textbf{Rodaun}|pwk}Rodaun, \textcolor{gray}{6} 9 23\nobreak{}«. }\buchAbdrucke{\weitereDrucke{Hugo von Hofmannsthal, Arthur Schnitzler: \emph{Briefwechsel}. Hg. Therese Nickl und Heinrich Schnitzler. Frankfurt am Main: \emph{S. Fischer} 1964, S. 298.} }\toendnotes[C]{\smallbreak}\pstart{}{\pb}Herrn\pend{}\pstart{}Dr. Hugo von Hofmannsthal\pend{}\pstart{}Rodaun\oindex{Rodaun@\textbf{Rodaun}|pw}\pend{}\pstart{}bei Wien\oindex{Wien@\textbf{Wien}|pw}\pend{}\pstart{}(Südbahn\oindex{Suedbahnhof@\textbf{Südbahnhof}|pw}{[}){]}\pend{}\pstart{}NiederOesterreich\oindex{Niederoesterreich@\textbf{Niederösterreich}|pw}\pend{}{\bigskip}\pstart
           \noindent{}\centering{}\textcolor{gray}{\textbf{{\pb}Celerina\oindex{Celerina@\textbf{Celerina}|pw}}}\pend
           \pstart
           \raggedleft{}{\pb}3/9. 23.\pend
           \pstart
           mein lieber Hugo – Ihr letztes Lebenszeichen hab ich vor \label{K_L02403_1v}\edtext{Monaten}{\lemma{\textnormal{\emph{Monaten}}}\Cendnote{\textnormal{siehe Hugo Hofmannsthal an Arthur Schnitzler, 15. 5. 1923}}}\label{K_L02403_1h} aus der Schweiz\oindex{Schweiz@\textbf{Schweiz}|pw} erhalten – und heut
               erst, auch aus der Schweiz\oindex{Schweiz@\textbf{Schweiz}|pw}, aus Celerina\oindex{Celerina@\textbf{Celerina}|pw}, wo mich vor 9 Jahren der Krieg überrascht hat
                  un\textcolor{gray}{d} ich \introOben{}heuer\introOben{} ein paar gute Wochen
               allein verlebt habe, {\pb}erwider ich Ihren lieben Gruſs.
               Heute reis ich ab, seh mir noch im Engadin\oindex{Engadin@\textbf{Engadin}|pw} einiges
               an, und geh da{\geminationn} an den Bodensee\oindex{Bodensee@\textbf{Bodensee}|pw} (Bregenz\oindex{Bregenz@\textbf{Bregenz}|pw}), von wo ich Lili\pwindex{Schnitzler, Lili 13.09.1909 – 26.07.1928@\textsc{Schnitzler, Lili} (13.09.1909 – 26.07.1928)|pw} abhole. Auf Wiedersehen hoffentlich! \pend
           \pstart Ihr \spacefill\mbox{Arthur}\pend{}\endnumbering\briefempfaengerindex{Hofmannsthal, Hugo von@\textsc{Hofmannsthal, Hugo von}!zzzSchnitzler, Arthur@\emph{von Arthur Schnitzler}!1923-09-031@{3. 9. 1923}|)be}\mylabel{h}\end{ledgroupsized}  \newcommand{\dateiname}{L02403}\newcommand{\titel}{Arthur Schnitzler an Hugo Hofmannsthal, 3. 9. 1923}\newcommand{\editorInnen}{Martin Anton Müller und Gerd-Hermann Susen}%% latex-leseansicht-abspann.tex
%% Abspann für die Leseansicht.
%% Der Schalter \ifkorrekturansicht ist bereits durch den Vorspann gesetzt.

%% latex-abspann.tex
%% Gemeinsamer Abspann für Korrekturansicht und Leseansicht.
%% Setzt den Schalter \ifkorrekturansicht voraus (gesetzt in den
%% einbindenden Dateien latex-korrekturansicht-abspann.tex bzw.
%% latex-leseansicht-abspann.tex).
%% ---------------------------------------------------------------

\normalsize

% Das esempio-Environment wird nur in der Leseansicht benötigt
\ifkorrekturansicht\else
\newenvironment{esempio}[3]%
{
    \vspace{1.5ex}
    \rlap{\underline{#1}}
    \par
    \setlength{\parindent}{0cm}
    \nopagebreak
    \leftskip=#2cm
    \rightskip=#3cm
}
{
    \par
}
\fi

\doendnotes{C}
\bigskip
\vfill

\clearpage

\footnotesize

\ifkorrekturansicht
  \lohead{\textsc{register}}
\fi

% theindex-Environment neu definieren ohne reledmac
\makeatletter
\renewenvironment{theindex}{%
  \ifkorrekturansicht
    \section*{\indexname}%
  \else
    \subsubsection*{Index der erwähnten Entitäten}%
  \fi
  \setlength{\parindent}{0pt}%
  \setlength{\parskip}{0pt plus 0.3pt}%
  \let\item\@idxitem
}{%
  \ifkorrekturansicht\clearpage\fi
}
\makeatother

\IfFileExists{\jobname-pw.ind}{\input{\jobname-pw.ind}}{}

% Quellenangabe nur in der Leseansicht
\ifkorrekturansicht\else
% Fallback-Definitionen, falls die .tex-Datei \titel etc. nicht gesetzt hat
\providecommand{\titel}{}
\providecommand{\editorInnen}{}
\providecommand{\dateiname}{\jobname}

\vspace{3cm}

\vfill

\footnotesize
\textsc{Quelle}: \titel. Herausgegeben von {\editorInnen}. In: \emph{Arthur Schnitzler: Briefwechsel mit Autorinnen und Autoren}.
 Digitale Edition, https://schnitzler-briefe.acdh.oeaw.ac.at/{\dateiname}.html (Stand \today)
\fi

\end{document}


      