%% latex-leseansicht-vorspann.tex
%% Vorspann für die Leseansicht.
%% Lädt die gemeinsame Datei latex-vorspann.tex mit nicht gesetztem Schalter.

\newif\ifkorrekturansicht
\korrekturansichtfalse

\input{../tex-inputs/latex-vorspann}


\section[ Paul Goldmann an Arthur Schnitzler, 30. 12. 1910]{L03472 Paul Goldmann an Arthur Schnitzler,  30. 12. 1910}
\nopagebreak\mylabel{L03472v}
\rehead{ }\normalsize\beginnumbering\briefempfaengerindex{Schnitzler, Arthur@\textsc{Schnitzler, Arthur}!zzzGoldmann, Paul@\emph{von Paul Goldmann}!1910-12-301@{30. 12. 1910}|(be}
\toendnotes[C]{\smallbreak\pagebreak[2]}
\correspDesc{Versand  durch Paul Goldmann am 30. 12. 1910 in Wien
\newline{}Erhalt  durch Arthur Schnitzler im Zeitraum [30. 12. 1910 – 3. 1. 1911?] in Wien}\toendnotes[C]{\smallbreak}
\Standort{DLA, A:Schnitzler, HS.NZ85.1.3175.}
\physDesc{Brief, 1 Blatt, 1 Seite, 437 Zeichen
\newline{}Handschrift: schwarze Tinte, deutsche Kurrent}\toendnotes[C]{\smallbreak}
\pstart
           \centering{}{\pb}\textcolor{gray}{\textbf{Hotel Sacher\oindex{Wien@\textbf{Wien}!I., Innere Stadt@\textbf{I., Innere Stadt}!Hotel Sacher@\textbf{Hotel Sacher}, \emph{Hotel}|pw}}}\pend
           
\pstart
           \textcolor{gray}{\textbf{Telefon Nr 8008.}}\hfill \textcolor{gray}{\textbf{Wien I.\oindex{Wien@\textbf{Wien}, \emph{Verwaltungsgebiet}|pw}}}\pend
           
\pstart
           30. 12. 10.\hfill Lieber Freund,\pend
           \vspace{0.5em}
\pstart
           Ich danke Dir herzlich für die Überſendung der \label{K_L03472-1v}\edtext{Kopien meiner Briefe}{\lemma{\textnormal{\emph{Kopien meiner Briefe}}}\Cendnote{\textnormal{Eine vollständige Abschrift der Korrespondenz ist nicht
                  überliefert. Goldmanns\pwindex{Goldmann, Paul 31.\,1.\,1865 Breslau – 25.\,9.\,1935 Wien@\textsc{Goldmann, Paul} (31.\,1.\,1865 Breslau – 25.\,9.\,1935 Wien), \emph{Schriftsteller, Journalist}|pwk} Briefen aus dem Jahr
                     1900 ist eine mit Schreibmaschine erstellte Abschrift einzelner
                  Briefstellen desselben Jahres beigelegt (\emph{DLA Marbach}, HS.1985.1.3170, zwei Durchschläge).
                  Dass diese neun Seiten hier gemeint sind, ist naheliegend, da die
                  Ausschnitte sich auf Werkaussagen konzentrieren. Aus Briefen Goldmanns\pwindex{Goldmann, Paul 31.\,1.\,1865 Breslau – 25.\,9.\,1935 Wien@\textsc{Goldmann, Paul} (31.\,1.\,1865 Breslau – 25.\,9.\,1935 Wien), \emph{Schriftsteller, Journalist}|pwk} von folgenden Tagen sind 
               Stellen entnommen:
                     XXXX Auszeichnungsfehler: Dokument L02904 nicht gefunden, XXXX Auszeichnungsfehler: Dokument L02921 nicht gefunden, XXXX Auszeichnungsfehler: Dokument L02931 nicht gefunden und XXXX Auszeichnungsfehler: Dokument L02936 nicht gefunden. Ein Zitat
                  stammt aus der Beilage des Schreibens vom XXXX Auszeichnungsfehler: Dokument L02944 nicht gefunden.}}}\label{K_L03472-1}. Nun bitte ich nur noch um die Erlaubis,{ }ſie nach
                  Berlin\oindex{Berlin@\textbf{Berlin}, \emph{Hauptstadt}|pw} mitzunehmen u. dort meiner Frau\pwindex{Goldmann, Eva Marie 27.\,10.\,1877 Wien – 2.\,11.\,1937 ebd.@\textsc{Goldmann, Eva Marie} (27.\,10.\,1877 Wien – 2.\,11.\,1937 ebd.)|pwv} zu zeigen. Von Berlin\oindex{Berlin@\textbf{Berlin}, \emph{Hauptstadt}|pw} werde ich{ }ſie Dir zurückſchicken u. Dir
               zugleich ein abſchließendes Wort über die \label{K_L03472-2v}\edtext{letzte Unterredung}{\lemma{\textnormal{\emph{letzte Unterredung}}}\Cendnote{\textnormal{Siehe XXXX Auszeichnungsfehler: Dokument L03471 nicht gefunden.
               }}}\label{K_L03472-2}{ }ſchreiben, die doch mehr in mir nachwirkt, als ich es gewünſcht hätte. – Mit
               herzlichen Grüßen an Deine Frau\pwindex{Schnitzler, Olga 17.\,1.\,1882 Wien – 13.\,1.\,1970 Lugano@\textsc{Schnitzler, Olga} (17.\,1.\,1882 Wien – 13.\,1.\,1970 Lugano), \emph{Schauspielerin, Sängerin}|pwv} u. Dich bin ich Dein \spacefill\mbox{Paul Goldmann.}\pend
           \selectlanguage{ngerman}\endnumbering\briefempfaengerindex{Schnitzler, Arthur@\textsc{Schnitzler, Arthur}!zzzGoldmann, Paul@\emph{von Paul Goldmann}!1910-12-301@{30. 12. 1910}|)be}\mylabel{L03472h}  \newcommand{\dateiname}{L03472}\newcommand{\titel}{Paul Goldmann an Arthur Schnitzler, 30. 12. 1910}\newcommand{\editorInnen}{Martin Anton Müller und Laura Untner}%% latex-leseansicht-abspann.tex
%% Abspann für die Leseansicht.
%% Der Schalter \ifkorrekturansicht ist bereits durch den Vorspann gesetzt.

%% latex-abspann.tex
%% Gemeinsamer Abspann für Korrekturansicht und Leseansicht.
%% Setzt den Schalter \ifkorrekturansicht voraus (gesetzt in den
%% einbindenden Dateien latex-korrekturansicht-abspann.tex bzw.
%% latex-leseansicht-abspann.tex).
%% ---------------------------------------------------------------

\normalsize

% Das esempio-Environment wird nur in der Leseansicht benötigt
\ifkorrekturansicht\else
\newenvironment{esempio}[3]%
{
    \vspace{1.5ex}
    \rlap{\underline{#1}}
    \par
    \setlength{\parindent}{0cm}
    \nopagebreak
    \leftskip=#2cm
    \rightskip=#3cm
}
{
    \par
}
\fi

\doendnotes{C}
\bigskip
\vfill

\clearpage

\footnotesize

\ifkorrekturansicht
  \lohead{\textsc{register}}
\fi

% theindex-Environment neu definieren ohne reledmac
\makeatletter
\renewenvironment{theindex}{%
  \ifkorrekturansicht
    \section*{\indexname}%
  \else
    \subsubsection*{Index der erwähnten Entitäten}%
  \fi
  \setlength{\parindent}{0pt}%
  \setlength{\parskip}{0pt plus 0.3pt}%
  \let\item\@idxitem
}{%
  \ifkorrekturansicht\clearpage\fi
}
\makeatother

\IfFileExists{\jobname-pw.ind}{\input{\jobname-pw.ind}}{}

% Quellenangabe nur in der Leseansicht
\ifkorrekturansicht\else
% Fallback-Definitionen, falls die .tex-Datei \titel etc. nicht gesetzt hat
\providecommand{\titel}{}
\providecommand{\editorInnen}{}
\providecommand{\dateiname}{\jobname}

\vspace{3cm}

\vfill

\footnotesize
\textsc{Quelle}: \titel. Herausgegeben von {\editorInnen}. In: \emph{Arthur Schnitzler: Briefwechsel mit Autorinnen und Autoren}.
 Digitale Edition, https://schnitzler-briefe.acdh.oeaw.ac.at/{\dateiname}.html (Stand \today)
\fi

\end{document}


