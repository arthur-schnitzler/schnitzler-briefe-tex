%% latex-leseansicht-vorspann.tex
%% Vorspann für die Leseansicht.
%% Lädt die gemeinsame Datei latex-vorspann.tex mit nicht gesetztem Schalter.

\newif\ifkorrekturansicht
\korrekturansichtfalse

\input{../tex-inputs/latex-vorspann}

\begin{center}
            \textcolor{red}{ENTWURF, NICHT FERTIG KORRIGIERT}
                      \end{center}
            
         
         \renewcommand{\erwaehntePersonen}{Personen: Paul Goldmann, Eva Marie Goldmann, Olga Schnitzler}
         \renewcommand{\erwaehnteOrte}{Orte: Berlin, Hotel Sacher, Wien}
         \renewcommand{\erwaehnteWerke}{}
               \section[ Paul Goldmann an Arthur Schnitzler, 30. 12. 1910]{ Paul Goldmann an Arthur Schnitzler, 30. 12. 1910}\nopagebreak\mylabel{v}\rehead{ }\begin{ledgroupsized}[t]{13cm}\normalsize\beginnumbering\briefempfaengerindex{Schnitzler, Arthur@\textsc{Schnitzler, Arthur}!zzzGoldmann, Paul@\emph{von Paul Goldmann}!1910-12-301@{30. 12. 1910}|(be} \toendnotes[C]{\smallbreak\pagebreak[2]} \Standort{DLA, A:Schnitzler, HS.NZ85.1.3175.}
\physDesc{Brief, 1 Blatt, 1 Seite, 437 Zeichen
\newline{}Handschrift: schwarze Tinte, deutsche Kurrent}\toendnotes[C]{\smallbreak}\pstart
           \noindent{}\centering{}{\pb}\textcolor{gray}{\textbf{Hotel Sacher\oindex{Hotel Sacher@\textbf{Hotel Sacher}|pw}}}\pend
           \pstart
           \noindent{}\textcolor{gray}{\textbf{Telefon Nr 8008.}}\hfill \textcolor{gray}{\textbf{Wien I.\oindex{Wien@\textbf{Wien}|pw}}}\pend
           \pstart
           30. 12. 10. \hfill Lieber Freund,\pend
           \pstart
           Ich danke Dir herzlich für die Überſendung der \label{K_L03472-1v}\edtext{Kopien meiner Briefe}{\lemma{\textnormal{\emph{Kopien meiner Briefe}}}\Cendnote{\textnormal{Eine vollständige Abschrift der Korrespondenz ist nicht
                  überliefert. Goldmann\pwindex{Goldmann, Paul 31.01.1865 – 25.09.1935@\textsc{Goldmann, Paul} (31.01.1865 – 25.09.1935), \emph{Schriftsteller, Journalist}|pwk}s Briefen aus dem
                  Jahr 1900 sind eine mit Schreibmaschine erstellte
                     Abschrift einzelner Briefstellen desselben Jahres beigelegt (\emph{DLA Marbach}, HS.1985.1.3170, 2 Durchschläge). 
                     Dass diese 9 Seiten hier gemeint sind, wäre zumindest naheliegend, da die Ausschnitte
                     sich auf Werkaussagen konzentrieren. Die Ausschnitte sind folgenden Briefen Goldmann\pwindex{Goldmann, Paul 31.01.1865 – 25.09.1935@\textsc{Goldmann, Paul} (31.01.1865 – 25.09.1935), \emph{Schriftsteller, Journalist}|pwk}s entnommen: 11. 2. 1900, 21. 6. [1900],
                     19. 9. [1900] und 14. 10. [1900],
                     sowie ein Zitat aus der Beilage des Schreibens vom 9. 12. [1900].}}}\label{K_L03472-1h}. Nun bitte ich nur noch um die Erlaubis, ſie nach Berlin\oindex{Berlin@\textbf{Berlin}|pw} mitzunehmen u. dort meiner Frau\pwindex{Goldmann, Eva Marie 27.10.1877 – 02.11.1937@\textsc{Goldmann, Eva Marie} (27.10.1877 – 02.11.1937)|pwv} zu zeigen. Von Berlin\oindex{Berlin@\textbf{Berlin}|pw} werde ich ſie Dir zurückſchicken u. Dir zugleich ein
               abſchließendes Wort über die \label{K_L03472-2v}\edtext{letzte
                  Unterredung}{\lemma{\textnormal{\emph{letzte
                  Unterredung}}}\Cendnote{\textnormal{siehe Paul Goldmann an Arthur Schnitzler, 26. 12. 1910}}}\label{K_L03472-2h} ſchreiben, die doch mehr in mir nachwirkt, als ich es gewünſcht hätte. – Mit
               herzlichen Grüßen an Deine Frau\pwindex{Schnitzler, Olga 17.01.1882 – 13.01.1970@\textsc{Schnitzler, Olga} (17.01.1882 – 13.01.1970), \emph{Schauspielerin, Sängerin}|pwv} u. Dich bin ich Dein \spacefill\mbox{Paul Goldmann.}\pend
           
         
         \endnumbering\mylabel{h}\end{ledgroupsized}  \newcommand{\dateiname}{L03472}\newcommand{\titel}{Paul Goldmann an Arthur Schnitzler, 30. 12. 1910}\newcommand{\editorInnen}{Martin Anton Müller und Laura Untner}%% latex-leseansicht-abspann.tex
%% Abspann für die Leseansicht.
%% Der Schalter \ifkorrekturansicht ist bereits durch den Vorspann gesetzt.

%% latex-abspann.tex
%% Gemeinsamer Abspann für Korrekturansicht und Leseansicht.
%% Setzt den Schalter \ifkorrekturansicht voraus (gesetzt in den
%% einbindenden Dateien latex-korrekturansicht-abspann.tex bzw.
%% latex-leseansicht-abspann.tex).
%% ---------------------------------------------------------------

\normalsize

% Das esempio-Environment wird nur in der Leseansicht benötigt
\ifkorrekturansicht\else
\newenvironment{esempio}[3]%
{
    \vspace{1.5ex}
    \rlap{\underline{#1}}
    \par
    \setlength{\parindent}{0cm}
    \nopagebreak
    \leftskip=#2cm
    \rightskip=#3cm
}
{
    \par
}
\fi

\doendnotes{C}
\bigskip
\vfill

\clearpage

\footnotesize

\ifkorrekturansicht
  \lohead{\textsc{register}}
\fi

% theindex-Environment neu definieren ohne reledmac
\makeatletter
\renewenvironment{theindex}{%
  \ifkorrekturansicht
    \section*{\indexname}%
  \else
    \subsubsection*{Index der erwähnten Entitäten}%
  \fi
  \setlength{\parindent}{0pt}%
  \setlength{\parskip}{0pt plus 0.3pt}%
  \let\item\@idxitem
}{%
  \ifkorrekturansicht\clearpage\fi
}
\makeatother

\IfFileExists{\jobname-pw.ind}{\input{\jobname-pw.ind}}{}

% Quellenangabe nur in der Leseansicht
\ifkorrekturansicht\else
% Fallback-Definitionen, falls die .tex-Datei \titel etc. nicht gesetzt hat
\providecommand{\titel}{}
\providecommand{\editorInnen}{}
\providecommand{\dateiname}{\jobname}

\vspace{3cm}

\vfill

\footnotesize
\textsc{Quelle}: \titel. Herausgegeben von {\editorInnen}. In: \emph{Arthur Schnitzler: Briefwechsel mit Autorinnen und Autoren}.
 Digitale Edition, https://schnitzler-briefe.acdh.oeaw.ac.at/{\dateiname}.html (Stand \today)
\fi

\end{document}


      