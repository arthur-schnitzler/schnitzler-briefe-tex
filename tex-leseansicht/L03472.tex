%% latex-korrekturansicht-vorspann.tex
%% Vorspann für die Korrekturansicht.
%% Lädt die gemeinsame Datei latex-vorspann.tex mit gesetztem Schalter.

\newif\ifkorrekturansicht
\korrekturansichttrue

\input{../tex-inputs/latex-vorspann}


\section[ Paul Goldmann an Arthur Schnitzler, 30. 12. 1910]{L03472 Paul Goldmann an Arthur Schnitzler, 30. 12. 1910}
\nopagebreak\mylabel{L03472v}
\rehead{ }\normalsize\beginnumbering\briefempfaengerindex{Schnitzler, Arthur@\textsc{Schnitzler, Arthur}!zzzGoldmann, Paul@\emph{von Paul Goldmann}!1910-12-301@{30. 12. 1910}|(be}
\toendnotes[C]{\smallbreak\pagebreak[2]}\Standort{DLA, A:Schnitzler, HS.NZ85.1.3175.}
\physDesc{Brief, 1 Blatt, 1 Seite, 437 Zeichen
\newline{}Handschrift: schwarze Tinte, deutsche Kurrent}\toendnotes[C]{\smallbreak}
\pstart
           \centering{}{\pb}\textcolor{gray}{\textbf{Hotel Sacher\oindex{Hotel Sacher@\textbf{Hotel Sacher}, \emph{Hotel (K.HTL)}|pw}}}\pend
           
\pstart
           \textcolor{gray}{\textbf{Telefon Nr 8008.}}\hfill \textcolor{gray}{\textbf{Wien I.\oindex{Wien@\textbf{Wien}, \emph{A.ADM2}|pw}}}\pend
           
\pstart
           30. 12. 10. \hfill Lieber Freund,\pend
           \vspace{0.5em}
\pstart
           Ich danke Dir herzlich für die Überſendung der \label{K_L03472-1v}\edtext{Kopien meiner Briefe}{\lemma{\textnormal{\emph{Kopien meiner Briefe}}}\Cendnote{\textnormal{Eine vollständige Abschrift der Korrespondenz ist nicht
                  überliefert. Goldmanns\pwindex{Goldmann, Paul 31.01.1865 – 25.09.1935@\textsc{Goldmann, Paul} (31.01.1865 – 25.09.1935), \emph{Schriftsteller/Schriftstellerin, Journalist/Journalistin}|pwk} Briefen aus dem Jahr
                     1900 ist eine mit Schreibmaschine erstellte Abschrift einzelner
                  Briefstellen desselben Jahres beigelegt (\emph{DLA Marbach}, HS.1985.1.3170, zwei Durchschläge).
                  Dass diese neun Seiten hier gemeint sind, ist naheliegend, da die
                  Ausschnitte sich auf Werkaussagen konzentrieren. Aus Briefen Goldmanns\pwindex{Goldmann, Paul 31.01.1865 – 25.09.1935@\textsc{Goldmann, Paul} (31.01.1865 – 25.09.1935), \emph{Schriftsteller/Schriftstellerin, Journalist/Journalistin}|pwk} von folgenden Tagen sind 
               Stellen entnommen:
                     11. 2. 1900, 21. 6. [1900], 19. 9. [1900] und 14. 10. [1900]. Ein Zitat
                  stammt aus der Beilage des Schreibens vom 9. 12. [1900].}}}\label{K_L03472-1}. Nun bitte ich nur noch um die Erlaubis, ſie nach
                  Berlin\oindex{Berlin@\textbf{Berlin}, \emph{P.PPLC}|pw} mitzunehmen u. dort meiner Frau\pwindex{Goldmann, Eva Marie 27.10.1877 – 02.11.1937@\textsc{Goldmann, Eva Marie} (27.10.1877 – 02.11.1937)|pwv} zu zeigen. Von Berlin\oindex{Berlin@\textbf{Berlin}, \emph{P.PPLC}|pw} werde ich ſie Dir zurückſchicken u. Dir
               zugleich ein abſchließendes Wort über die \label{K_L03472-2v}\edtext{letzte Unterredung}{\lemma{\textnormal{\emph{letzte Unterredung}}}\Cendnote{\textnormal{Siehe Paul Goldmann an Arthur Schnitzler, 26. 12. 1910.
               }}}\label{K_L03472-2} ſchreiben, die doch mehr in mir nachwirkt, als ich es gewünſcht hätte. – Mit
               herzlichen Grüßen an Deine Frau\pwindex{Schnitzler, Olga 17.01.1882 – 13.01.1970@\textsc{Schnitzler, Olga} (17.01.1882 – 13.01.1970), \emph{Schauspieler/Schauspielerin, Sänger/Sängerin}|pwv} u. Dich bin ich Dein \spacefill\mbox{Paul Goldmann.}\pend
           \selectlanguage{ngerman}\endnumbering\briefempfaengerindex{Schnitzler, Arthur@\textsc{Schnitzler, Arthur}!zzzGoldmann, Paul@\emph{von Paul Goldmann}!1910-12-301@{30. 12. 1910}|)be}\mylabel{L03472h}  \normalsize

\doendnotes{C}
\bigskip
\vfill

\clearpage

\footnotesize

\lohead{\textsc{register}}

% Definiere theindex-Environment komplett neu ohne reledmac
\makeatletter
\renewenvironment{theindex}{%
  \section*{\indexname}%
  \setlength{\parindent}{0pt}%
  \setlength{\parskip}{0pt plus 0.3pt}%
  \let\item\@idxitem
}{%
  \clearpage
}
\makeatother

\IfFileExists{\jobname-pw.ind}{\input{\jobname-pw.ind}}{}

\end{document}

      