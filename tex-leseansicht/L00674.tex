%% latex-leseansicht-vorspann.tex
%% Vorspann für die Leseansicht.
%% Lädt die gemeinsame Datei latex-vorspann.tex mit nicht gesetztem Schalter.

\newif\ifkorrekturansicht
\korrekturansichtfalse

\input{../tex-inputs/latex-vorspann}


\section[Arthur Schnitzler an Hugo von Hofmannsthal, 6. 5. 1897]{L00674 Arthur Schnitzler an Hugo von Hofmannsthal, 6. 5. 1897}
\nopagebreak\mylabel{L00674v}
\rehead{ }\normalsize\beginnumbering\briefempfaengerindex{Hofmannsthal, Hugo von@\textsc{Hofmannsthal, Hugo von}!zzzSchnitzler, Arthur@\emph{von Arthur Schnitzler}!1897-05-061@{6. 5. 1897}|(be}
\toendnotes[C]{\smallbreak\pagebreak[2]}
\correspDesc{Versand  durch Arthur Schnitzler am 6. 5. 1897 in Paris
\newline{}Erhalt  durch Hugo von Hofmannsthal im Zeitraum [7. 5. 1897
                  – 11. 5. 1897?] in Wien}\toendnotes[C]{\smallbreak}
\Standort{FDH, Hs-30885,13.}
\physDesc{Brief, 2 Blätter, 8 Seiten, 3213 Zeichen
\newline{}Handschrift: schwarze Tinte, deutsche Kurrent
\newline{}Ordnung: mit Bleistift von Schnitzler mutmaßlich bei der Durchsicht der Korrespondenz
                                    1929 auf dem ersten und zweiten Blatt
                                 datiert: »6/5 97« }
\buchAbdrucke{\weitereDrucke{Hugo von Hofmannsthal, Arthur Schnitzler: \emph{Briefwechsel}. Herausgegeben von Therese Nickl und Heinrich Schnitzler. Frankfurt am Main: \emph{S. Fischer} 1964, S. 84–85.} }\toendnotes[C]{\smallbreak}
\pstart
           \noindent{}{\pb}Mein lieber Hugo,{ }Fiſcher\pwindex{Fischer, Samuel 24.\,12.\,1859 Liptovský Mikuláš – 15.\,10.\,1934 Berlin@\textsc{Fischer, Samuel} (24.\,12.\,1859 Liptovský Mikuláš – 15.\,10.\,1934 Berlin), \emph{Verleger}|pw} hat den Satz von Mimi\pwindex{Pollaczek, Clara Katharina 15.\,1.\,1875 Wien – 22.\,7.\,1951 ebd.@\textsc{Pollaczek, Clara Katharina} (15.\,1.\,1875 Wien – 22.\,7.\,1951 ebd.), \emph{Schriftstellerin}!Mimi. Schattenbilder aus einem Mädchenleben@\strich\emph{Mimi. Schattenbilder aus einem Mädchenleben}|pw} auf meinen Wunſch bereits ablegen laſſen, und{ }ſo iſt die
               letzte Gefahr geſchwunden. Ich hoffe, Sie haben meinen vorigen, zweiten Brief, in dem
               ich Ihnen auf Ihr diesbezügliches Erſuchen geantwortet, erhalten? – Iſt es ruhig
               geworden im Hauſe Loeb\pwindex{Loeb, Regina 1850 – 5.\,2.\,1918 Wien@\textsc{Loeb, Regina} (1850 – 5.\,2.\,1918 Wien)|pw}\pwindex{Loeb, Louis 29.\,6.\,1842 Mattersdorf – 6.\,6.\,1921 Wien@\textsc{Loeb, Louis} (29.\,6.\,1842 Mattersdorf – 6.\,6.\,1921 Wien), \emph{Bankier}|pw}? – Wie geht
               es der geſchädigten Verfaſſerin\pwindex{Pollaczek, Clara Katharina 15.\,1.\,1875 Wien – 22.\,7.\,1951 ebd.@\textsc{Pollaczek, Clara Katharina} (15.\,1.\,1875 Wien – 22.\,7.\,1951 ebd.), \emph{Schriftstellerin}|pwv} der Scenen aus
                  einem Mädchenleben\pwindex{Pollaczek, Clara Katharina 15.\,1.\,1875 Wien – 22.\,7.\,1951 ebd.@\textsc{Pollaczek, Clara Katharina} (15.\,1.\,1875 Wien – 22.\,7.\,1951 ebd.), \emph{Schriftstellerin}!Mimi. Schattenbilder aus einem Mädchenleben@\strich\emph{Mimi. Schattenbilder aus einem Mädchenleben}|pwv}?\pend
           
\pstart
           – Die Delna\pwindex{Delna, Marie 3.\,4.\,1875 Paris – 23.\,7.\,1932@\textsc{Delna, Marie} (3.\,4.\,1875 Paris – 23.\,7.\,1932), \emph{Sängerin}|pw} hab ich{ }ſchon gehört; gerade am
               Abend bevor Ihr Brief kam, als Orpheus\pwindex{Gluck, Christoph Willibald 2.\,7.\,1714 Erasbach – 15.\,11.\,1787 Wien@\textsc{Gluck, Christoph Willibald} (2.\,7.\,1714 Erasbach – 15.\,11.\,1787 Wien), \emph{Komponist}!Orpheus und Eurydike@\strich\emph{Orpheus und Eurydike}|pwv}. Sie hat eine {\pb}mächtige, nicht immer
               edle Stimme; eine beſondre Höhe der Darſtellung und des Geſangs erreicht{ }ſie am
               Schluſs; da bin ich tief ergriffen geweſen – bis dahin hatt’ ich die Papier\pwindex{Papier, Rosa 18.\,9.\,1858 Baden bei Wien – 9.\,2.\,1932 Wien@\textsc{Papier, Rosa} (18.\,9.\,1858 Baden bei Wien – 9.\,2.\,1932 Wien), \emph{Sängerin, Gesangspädagogin}|pw} nicht vergeſſen können. –\pend
           
\pstart
           Jetzt eben ko{\geminationm}e ich von einer \textsc{Matinée} im \textsc{Français}\oindex{Comédie française@\textbf{Comédie française}, \emph{Theater}|pw}, wo man den \textsc{Misanthropen}\pwindex{Molière 14.\,1.\,1622 Paris – 17.\,2.\,1673 ebd.@\textsc{Molière} (14.\,1.\,1622 Paris – 17.\,2.\,1673 ebd.), \emph{Schriftsteller, Theaterleiter, Schauspieler}!Le Misanthrope ou l’Atrabilaire amoureux@\strich\emph{Le Misanthrope ou l’Atrabilaire amoureux}|pw} gegeben hat. Um hier der abſoluten Größe inne zu werden, muſs man{ }ſich doch
               erſt hiſtoriſch montieren, was weder bei \textsc{Sophokles}\pwindex{Sophokles 497/496? v.\,u.\,Z. Kolonos – 406/405 v.\,u.\,Z. Athen@\textsc{Sophokles} (497/496? v.\,u.\,Z. Kolonos – 406/405 v.\,u.\,Z. Athen), \emph{Schriftsteller}|pw} noch bei \textsc{Shakespeare}\pwindex{Shakespeare, William 23.\,4.\,1564? Stratford-upon-Avon – 3.\,5.\,1616 ebd.@\textsc{Shakespeare, William} (23.\,4.\,1564? Stratford-upon-Avon – 3.\,5.\,1616 ebd.), \emph{Schauspieler, Dramatiker}|pw} notwendig iſt. Erſt im letzten Akt, {\pb}wo nicht mehr
                  \textsc{\uline{le} misanthrope},{ }ſondern \textsc{\uline{un} misanthrope} vor einem{ }ſteht,{ }ſpürt man was
               ewig menſchliches. Es liegt wohl daran, daſs alles, was in dieſem Stück\pwindex{Molière 14.\,1.\,1622 Paris – 17.\,2.\,1673 ebd.@\textsc{Molière} (14.\,1.\,1622 Paris – 17.\,2.\,1673 ebd.), \emph{Schriftsteller, Theaterleiter, Schauspieler}!Le Misanthrope ou l’Atrabilaire amoureux@\strich\emph{Le Misanthrope ou l’Atrabilaire amoureux}|pwv} vorgeht, einfach die Anſicht des
               Helden beſtätigt; er erfährt nichts neues, denn{ }ſchon im erſten Auftritt weiſs er,
               was die Menſchen für ein Geſindel{ }ſind. Erſt{ }ſein Entſchluſs, in die Einſamkeit{ }ſich
               zurückzuziehen, bewegt uns; wahrſcheinlich weil wir wiſſen, daſs{ }ſeine ganze
               Menſchenfeindſchaft nichts {\pb}iſt als Sehnſucht nach guten
               Menſchen, die er jetzt ein für alle Mal{ }ſelbſt zu etwas unerfüllbarem macht; denn er
               wird niemanden mehr kennen lernen. –\pend
           
\pstart
           Tröſten Sie{ }ſich wegen des gemiſchten Hausbrotes: Wochenlang hab ich ein weißes
               trocknes gegeſſen (wer nie{ }ſein Brod mit Thränen
                  aſs\pwindex{Goethe, Johann Wolfgang von 28.\,8.\,1749 Frankfurt am Main – 22.\,3.\,1832 Weimar@\textsc{Goethe, Johann Wolfgang von} (28.\,8.\,1749 Frankfurt am Main – 22.\,3.\,1832 Weimar), \emph{Schriftsteller}!Wer nie sein Brod mit Thränen as@\strich\emph{Wer nie sein Brod mit Thränen as}|pw}– !); und auch jetzt nehm ich meine Mahlzeiten in einer ſtockfranzöſiſchen\oindex{Frankreich@\textbf{Frankreich}|pw} Familie ein, wo keine heimatlichen {\pb}Gulyasdüfte aufſteigen. Sie ahnen nicht, wie viel »ganz
               andres« ich eſſe. Die hieſige Einteilung 12 Uhr Dejeuner, 7 Diner, 9 Theater, behagt
               mir außerordentlich.\pend
           
\pstart
           Schöne Radpartien? Z. B. fahren Sie von der Tini\pwindex{Kepert, Christine 17.\,11.\,1875 – 3.\,2.\,1971 Wien@\textsc{Kepert, Christine} (17.\,11.\,1875 – 3.\,2.\,1971 Wien), \emph{Gastwirtin}|pw} aus über Heiligenkreuz\oindex{Heiligenkreuz@\textbf{Heiligenkreuz}, \emph{Verwaltungsgebiet}|pw} – Alland\oindex{Alland@\textbf{Alland}, \emph{Verwaltungsgebiet}|pw} – \uline{Neuhaus}\oindex{Neuhaus@\textbf{Neuhaus}|pw} (bei Nöſtach\oindex{Nöstach@\textbf{Nöstach}|pw}) – Pottenſtein\oindex{Pottenstein@\textbf{Pottenstein}, \emph{Hauptstadt}|pw} – Vöslau\oindex{Bad Vöslau@\textbf{Bad Vöslau}, \emph{Hauptstadt}|pw}. Oder:
               Rohrerhütte\oindex{Wien@\textbf{Wien}!XVII., Hernals@\textbf{XVII., Hernals}!Rohrerhütte@\textbf{Rohrerhütte}, \emph{Gastgewerbegebäude}|pw} – Königſtetten\oindex{Königstetten@\textbf{Königstetten}, \emph{Hauptstadt}|pw} (ſehr bergig,{ }ſchieben!) – Tulln\oindex{Tulln an der Donau@\textbf{Tulln an der Donau}, \emph{Verwaltungsgebiet}|pw}, dann an der Donau\oindex{Donau@\textbf{Donau}, \emph{Fluss}|pw} zurück nach Kloſterneuburg\oindex{Weidling@\textbf{Weidling}|pw}. – Sehr hübſch auch die kleine
               Tour Tulln\oindex{Tulln an der Donau@\textbf{Tulln an der Donau}, \emph{Verwaltungsgebiet}|pw} – Stockerau\oindex{Stockerau@\textbf{Stockerau}, \emph{Hauptstadt}|pw}. {\pb}Oder: Rekawinkel\oindex{Rekawinkel@\textbf{Rekawinkel}|pw} – Hütteldorf\oindex{Wien@\textbf{Wien}!XIV., Penzing@\textbf{XIV., Penzing}!Hütteldorf@\textbf{Hütteldorf}|pw}
               (Weſtbahnſtrecke.) Od: Wiener Neuſtadt\oindex{Wiener Neustadt@\textbf{Wiener Neustadt}, \emph{Verwaltungsgebiet}|pw} – Reichenau\oindex{Reichenau an der Rax@\textbf{Reichenau an der Rax}, \emph{Verwaltungsgebiet}|pw}. – Ich freue mich{ }ſehr, we{\geminationn} wir zuſa{\geminationm}en fahren
               werden.\pend
           
\pstart
           Wie lang bleiben Sie de{\geminationn} in Wien\oindex{Wien@\textbf{Wien}, \emph{Verwaltungsgebiet}|pw}? Und wie wird heuer der Sommer werden? Ich möchte{ }ſo gern
               zum Arbeiten ko{\geminationm}en; hier{ }ſpiele ich höchſtens mit
               Plänen; aber möglicherweiſe iſt \substVorne{}\textsuperscript{mehr}\substDazwischen{}mir\substHinten{} durch ein merkwürdiges \label{K_L00674-1v}\edtext{Zuſammenfließen zweier Pläne}{\lemma{\textnormal{\emph{Zusammenfließen … Pläne}}}\Cendnote{\textnormal{Am 30. 4. 1897 überlegte
                     Schnitzler, die Stoffe »Die Entrüsteten«
                  und »Rettung« zusammenzufügen. Ersteres handelte vom Zusammenleben ohne zu
                  heiraten (in Anlehnung an sein Leben mit Marie
                     Reinhard\pwindex{Reinhard, Marie 13.\,3.\,1871 Wien – 18.\,3.\,1899 ebd.@\textsc{Reinhard, Marie} (13.\,3.\,1871 Wien – 18.\,3.\,1899 ebd.), \emph{Gesangspädagogin}|pwk}). Aus der vorliegenden Stelle geht hervor, dass der zweite Plan in Beziehung mit Hermine Benedict\pwindex{Schaffgotsch, Hermine von 25.\,11.\,1871 Wien – 25.\,11.\,1928 Purgstall@\textsc{Schaffgotsch, Hermine von} (25.\,11.\,1871 Wien – 25.\,11.\,1928 Purgstall)|pwk} steht. Aus dem Projekt, das in diesem
                  Stadium noch als Stück gedacht war, entwickelte sich im nächsten Jahrzehnt der
                  Roman \emph{Der Weg ins Freie}\pwindex{Schnitzler, Arthur 15. 5. 1862 Wien – 21. 10. 1931 ebd.@\textsc{Schnitzler, Arthur} (15. 5. 1862 Wien – 21. 10. 1931 ebd.), \emph{Schriftsteller, Mediziner}!Weg ins Freie. Roman@\strich\emph{Der Weg ins Freie. Roman}|pwk}.}}}\label{K_L00674-1}, worunter
               einer der mit der Minni\pwindex{Schaffgotsch, Hermine von 25.\,11.\,1871 Wien – 25.\,11.\,1928 Purgstall@\textsc{Schaffgotsch, Hermine von} (25.\,11.\,1871 Wien – 25.\,11.\,1928 Purgstall)|pw}, etwas gutes {\pb}eingefallen. –\pend
           
\pstart
           Den Götterliebling\pwindex{Beer-Hofmann, Richard 11.\,7.\,1866 Wien – 26.\,9.\,1945 New York City@\textsc{Beer-Hofmann, Richard} (11.\,7.\,1866 Wien – 26.\,9.\,1945 New York City), \emph{Schriftsteller}!Tod Georgs@\strich\emph{Der Tod Georgs}|pw} hoff ich ganz fertig
               anzutreffen. Bei dem Stück\pwindex{Hirschfeld, Georg 11.\,2.\,1873 Berlin – 17.\,1.\,1942 München@\textsc{Hirschfeld, Georg} (11.\,2.\,1873 Berlin – 17.\,1.\,1942 München), \emph{Schriftsteller}!Agnes Jordan. Schauspiel in fünf Akten@\strich\emph{Agnes Jordan. Schauspiel in fünf Akten}|pwv} von
                  Hirſchf.\pwindex{Hirschfeld, Georg 11.\,2.\,1873 Berlin – 17.\,1.\,1942 München@\textsc{Hirschfeld, Georg} (11.\,2.\,1873 Berlin – 17.\,1.\,1942 München), \emph{Schriftsteller}|pw} zweifle ich gar nicht daran. – Iſt
               bei Ben.\pwindex{Benedict, Marianne 1.\,1.\,1848 Bratislava – 12.\,5.\,1930 Wien@\textsc{Benedict, Marianne} (1.\,1.\,1848 Bratislava – 12.\,5.\,1930 Wien)|pw}\pwindex{Benedict, Markus 17.\,9.\,1834 Mikulov – 26.\,2.\,1909 Kärntnerring 13@\textsc{Benedict, Markus} (17.\,9.\,1834 Mikulov – 26.\,2.\,1909 Kärntnerring 13), \emph{Industrieller}|pw} nach mir gefragt
               worden? –\pend
           
\pstart
           Paul Goldma{\geminationn}\pwindex{Goldmann, Paul 31.\,1.\,1865 Breslau – 25.\,9.\,1935 Wien@\textsc{Goldmann, Paul} (31.\,1.\,1865 Breslau – 25.\,9.\,1935 Wien), \emph{Schriftsteller, Journalist}|pw} hat unglaublich viel zu thun, u. we{\geminationn} ich ihn nicht
               gerade auf{ }ſeinen Excurſionen zwiſchen Bureau u. Telegraphenamt begleite, wie z. B.
               geſtern, wo das Brandunglück im \textsc{Bazar de la Charité}\orgindex{Bazar de la Charité@Bazar de la Charité|pw} den Zeitungen{ }ſo {\pb}viel zu thun gab, hab ich
               eigentlich wenig von ihm. Aber{ }ſein Weſen macht mir{ }ſehr viel Freude; und er gehört
               zu den wenigen, an denen ich mich erhole, von denen aus mir der Weg zu mir{ }ſelbſt am
               freieſten und klarſten daliegt.\pend
           
\pstart
           Herzlich der Ihre{\\[\baselineskip]}\spacefill\mbox{Arth}\pend
           \leftskip=0em{}
\pstart
           Paris\oindex{Paris@\textbf{Paris}, \emph{Hauptstadt}|pw}{ }6. 5. 97.\pend
           \selectlanguage{ngerman}\endnumbering\briefempfaengerindex{Hofmannsthal, Hugo von@\textsc{Hofmannsthal, Hugo von}!zzzSchnitzler, Arthur@\emph{von Arthur Schnitzler}!1897-05-061@{6. 5. 1897}|)be}\mylabel{L00674h}  \newcommand{\dateiname}{L00674}\newcommand{\titel}{Arthur Schnitzler an Hugo von Hofmannsthal, 6. 5. 1897}\newcommand{\editorInnen}{Martin Anton Müller und Gerd-Hermann Susen}%% latex-leseansicht-abspann.tex
%% Abspann für die Leseansicht.
%% Der Schalter \ifkorrekturansicht ist bereits durch den Vorspann gesetzt.

%% latex-abspann.tex
%% Gemeinsamer Abspann für Korrekturansicht und Leseansicht.
%% Setzt den Schalter \ifkorrekturansicht voraus (gesetzt in den
%% einbindenden Dateien latex-korrekturansicht-abspann.tex bzw.
%% latex-leseansicht-abspann.tex).
%% ---------------------------------------------------------------

\normalsize

% Das esempio-Environment wird nur in der Leseansicht benötigt
\ifkorrekturansicht\else
\newenvironment{esempio}[3]%
{
    \vspace{1.5ex}
    \rlap{\underline{#1}}
    \par
    \setlength{\parindent}{0cm}
    \nopagebreak
    \leftskip=#2cm
    \rightskip=#3cm
}
{
    \par
}
\fi

\doendnotes{C}
\bigskip
\vfill

\clearpage

\footnotesize

\ifkorrekturansicht
  \lohead{\textsc{register}}
\fi

% theindex-Environment neu definieren ohne reledmac
\makeatletter
\renewenvironment{theindex}{%
  \ifkorrekturansicht
    \section*{\indexname}%
  \else
    \subsubsection*{Index der erwähnten Entitäten}%
  \fi
  \setlength{\parindent}{0pt}%
  \setlength{\parskip}{0pt plus 0.3pt}%
  \let\item\@idxitem
}{%
  \ifkorrekturansicht\clearpage\fi
}
\makeatother

\IfFileExists{\jobname-pw.ind}{\input{\jobname-pw.ind}}{}

% Quellenangabe nur in der Leseansicht
\ifkorrekturansicht\else
% Fallback-Definitionen, falls die .tex-Datei \titel etc. nicht gesetzt hat
\providecommand{\titel}{}
\providecommand{\editorInnen}{}
\providecommand{\dateiname}{\jobname}

\vspace{3cm}

\vfill

\footnotesize
\textsc{Quelle}: \titel. Herausgegeben von {\editorInnen}. In: \emph{Arthur Schnitzler: Briefwechsel mit Autorinnen und Autoren}.
 Digitale Edition, https://schnitzler-briefe.acdh.oeaw.ac.at/{\dateiname}.html (Stand \today)
\fi

\end{document}


