%% latex-korrekturansicht-vorspann.tex
%% Vorspann für die Korrekturansicht.
%% Lädt die gemeinsame Datei latex-vorspann.tex mit gesetztem Schalter.

\newif\ifkorrekturansicht
\korrekturansichttrue

\input{../tex-inputs/latex-vorspann}


\section[Arthur Schnitzler an Hugo von Hofmannsthal, 6. 5. 1897]{L00674 Arthur Schnitzler an Hugo von Hofmannsthal, 6. 5. 1897}
\nopagebreak\mylabel{L00674v}
\rehead{ }\normalsize\beginnumbering\briefempfaengerindex{Hofmannsthal, Hugo von@\textsc{Hofmannsthal, Hugo von}!zzzSchnitzler, Arthur@\emph{von Arthur Schnitzler}!1897-05-061@{6. 5. 1897}|(be}
\toendnotes[C]{\smallbreak\pagebreak[2]}\Standort{FDH, Hs-30885,13.}
\physDesc{Brief, 2 Blätter, 8 Seiten, 3213 Zeichen
\newline{}Handschrift: schwarze Tinte, deutsche Kurrent
\newline{}Ordnung: mit Bleistift von Schnitzler mutmaßlich bei der Durchsicht der Korrespondenz
                                    1929 auf dem ersten und zweiten Blatt
                                 datiert: »6/5 97« }
\buchAbdrucke{\weitereDrucke{Hugo von Hofmannsthal, Arthur Schnitzler: \emph{Briefwechsel}. Frankfurt am Main: \emph{S. Fischer} 1964, S. 84–85.} }\toendnotes[C]{\smallbreak}
\pstart
           \noindent{}{\pb}Mein lieber Hugo,{ }Fiſcher\pwindex{Fischer, Samuel 24.12.1859 – 15.10.1934@\textsc{Fischer, Samuel} (24.12.1859 – 15.10.1934), \emph{Verleger/Verlegerin}|pw} hat den Satz von Mimi\pwindex{Mimi. Schattenbilder aus einem Maedchenleben@\emph{Mimi. Schattenbilder aus einem Mädchenleben}|pw} auf meinen Wunſch bereits ablegen laſſen, und ſo iſt die
               letzte Gefahr geſchwunden. Ich hoffe, Sie haben meinen vorigen, zweiten Brief, in dem
               ich Ihnen auf Ihr diesbezügliches Erſuchen geantwortet, erhalten? – Iſt es ruhig
               geworden im Hauſe Loeb\pwindex{Loeb, Regina 1850 – 5.2.1918@\textsc{Loeb, Regina} (1850 – 5.2.1918)|pw}\pwindex{Loeb, Louis 29.06.1842 – 06.06.1921@\textsc{Loeb, Louis} (29.06.1842 – 06.06.1921), \emph{Bankier/Bankierin}|pw}? – Wie geht
               es der geſchädigten Verfaſſerin\pwindex{Pollaczek, Clara Katharina 15.01.1875 – 22.07.1951@\textsc{Pollaczek, Clara Katharina} (15.01.1875 – 22.07.1951), \emph{Schriftsteller/Schriftstellerin}|pwv} der Scenen aus
                  einem Mädchenleben\pwindex{Mimi. Schattenbilder aus einem Maedchenleben@\emph{Mimi. Schattenbilder aus einem Mädchenleben}|pwv}?\pend
           
\pstart
           – Die Delna\pwindex{Delna, Marie 03.04.1875 – 23.07.1932@\textsc{Delna, Marie} (03.04.1875 – 23.07.1932), \emph{Sänger/Sängerin}|pw} hab ich ſchon gehört; gerade am
               Abend bevor Ihr Brief kam, als Orpheus\pwindex{Orpheus und Eurydike@\emph{Orpheus und Eurydike}|pwv}. Sie hat eine {\pb}mächtige, nicht immer
               edle Stimme; eine beſondre Höhe der Darſtellung und des Geſangs erreicht ſie am
               Schluſs; da bin ich tief ergriffen geweſen – bis dahin hatt’ ich die Papier\pwindex{Papier, Rosa 1858-09-18 – 1932-02-09@\textsc{Papier, Rosa} (1858-09-18 – 1932-02-09), \emph{Sänger/Sängerin, Gesangspädagoge/Gesangspädagogin}|pw} nicht vergeſſen können. –\pend
           
\pstart
           Jetzt eben ko{\geminationm}e ich von einer \textsc{Matinée} im \textsc{Français}\oindex{Comedie française@\textbf{Comédie française}, \emph{S.THTR}|pw}, wo man den \textsc{Misanthropen}\pwindex{Le Misanthrope ou l Atrabilaire amoureux@\emph{Le Misanthrope ou l’Atrabilaire amoureux}|pw} gegeben hat. Um hier der abſoluten Größe inne zu werden, muſs man ſich doch
               erſt hiſtoriſch montieren, was weder bei \textsc{Sophokles}\pwindex{Sophokles 497/496? v. u. Z. – 406/405 v. u. Z.@\textsc{Sophokles} (497/496? v. u. Z. – 406/405 v. u. Z.), \emph{Schriftsteller/Schriftstellerin}|pw} noch bei \textsc{Shakespeare}\pwindex{Shakespeare, William 23.4.1564? – 03.05.1616@\textsc{Shakespeare, William} (23.4.1564? – 03.05.1616), \emph{Schauspieler/Schauspielerin, Dramatiker/Dramatikerin}|pw} notwendig iſt. Erſt im letzten Akt, {\pb}wo nicht mehr
                  \textsc{\uline{le} misanthrope}, ſondern \textsc{\uline{un} misanthrope} vor einem ſteht, ſpürt man was
               ewig menſchliches. Es liegt wohl daran, daſs alles, was in dieſem Stück\pwindex{Le Misanthrope ou l Atrabilaire amoureux@\emph{Le Misanthrope ou l’Atrabilaire amoureux}|pwv} vorgeht, einfach die Anſicht des
               Helden beſtätigt; er erfährt nichts neues, denn ſchon im erſten Auftritt weiſs er,
               was die Menſchen für ein Geſindel ſind. Erſt ſein Entſchluſs, in die Einſamkeit ſich
               zurückzuziehen, bewegt uns; wahrſcheinlich weil wir wiſſen, daſs ſeine ganze
               Menſchenfeindſchaft nichts {\pb}iſt als Sehnſucht nach guten
               Menſchen, die er jetzt ein für alle Mal ſelbſt zu etwas unerfüllbarem macht; denn er
               wird niemanden mehr kennen lernen. –\pend
           
\pstart
           Tröſten Sie ſich wegen des gemiſchten Hausbrotes: Wochenlang hab ich ein weißes
               trocknes gegeſſen (wer nie ſein Brod mit Thränen
                  aſs\pwindex{Wer nie sein Brod mit Thraenen as@\emph{Wer nie sein Brod mit Thränen as}|pw}– !); und auch jetzt nehm ich meine Mahlzeiten in einer ſtockfranzöſiſchen\oindex{Frankreich@\textbf{Frankreich}, \emph{A.PCLI}|pw} Familie ein, wo keine heimatlichen {\pb}Gulyasdüfte aufſteigen. Sie ahnen nicht, wie viel »ganz
               andres« ich eſſe. Die hieſige Einteilung 12 Uhr Dejeuner, 7 Diner, 9 Theater, behagt
               mir außerordentlich.\pend
           
\pstart
           Schöne Radpartien? Z. B. fahren Sie von der Tini\pwindex{Schoenberger, Christine 1875-11-17 – 1971-02-03@\textsc{Schönberger, Christine} (1875-11-17 – 1971-02-03), \emph{Gastwirt/Gastwirtin}|pw} aus über Heiligenkreuz\oindex{Heiligenkreuz@\textbf{Heiligenkreuz}, \emph{A.ADM3}|pw} – Alland\oindex{Alland@\textbf{Alland}, \emph{A.ADM3}|pw} – \uline{Neuhaus}\oindex{Neuhaus@\textbf{Neuhaus}, \emph{P.PPL}|pw} (bei Nöſtach\oindex{Noestach@\textbf{Nöstach}, \emph{P.PPL}|pw}) – Pottenſtein\oindex{Pottenstein@\textbf{Pottenstein}, \emph{P.PPLA3}|pw} – Vöslau\oindex{Bad Voeslau@\textbf{Bad Vöslau}, \emph{P.PPLA3}|pw}. Oder:
               Rohrerhütte\oindex{Rohrerhuette@\textbf{Rohrerhütte}, \emph{Gastgewerbegebäude (K.GGW)}|pw} – Königſtetten\oindex{Koenigstetten@\textbf{Königstetten}, \emph{P.PPLA3}|pw} (ſehr bergig, ſchieben!) – Tulln\oindex{Tulln an der Donau@\textbf{Tulln an der Donau}, \emph{A.ADM3}|pw}, dann an der Donau\oindex{Donau@\textbf{Donau}, \emph{Fluss (N.FLS)}|pw} zurück nach Kloſterneuburg\oindex{Weidling@\textbf{Weidling}, \emph{P.PPL}|pw}. – Sehr hübſch auch die kleine
               Tour Tulln\oindex{Tulln an der Donau@\textbf{Tulln an der Donau}, \emph{A.ADM3}|pw} – Stockerau\oindex{Stockerau@\textbf{Stockerau}, \emph{P.PPLA3}|pw}. {\pb}Oder: Rekawinkel\oindex{Rekawinkel@\textbf{Rekawinkel}, \emph{P.PPL}|pw} – Hütteldorf\oindex{Huetteldorf@\textbf{Hütteldorf}, \emph{eingemeindeter Ort (A.VOO)}|pw}
               (Weſtbahnſtrecke.) Od: Wiener Neuſtadt\oindex{Wiener Neustadt@\textbf{Wiener Neustadt}, \emph{A.ADM2}|pw} – Reichenau\oindex{Reichenau an der Rax@\textbf{Reichenau an der Rax}, \emph{A.ADM3}|pw}. – Ich freue mich ſehr, we{\geminationn} wir zuſa{\geminationm}en fahren
               werden.\pend
           
\pstart
           Wie lang bleiben Sie de{\geminationn} in Wien\oindex{Wien@\textbf{Wien}, \emph{A.ADM2}|pw}? Und wie wird heuer der Sommer werden? Ich möchte ſo gern
               zum Arbeiten ko{\geminationm}en; hier ſpiele ich höchſtens mit
               Plänen; aber möglicherweiſe iſt \substVorne{}\textsuperscript{mehr}\substDazwischen{}mir\substHinten{} durch ein merkwürdiges \label{K_L00674-1v}\edtext{Zuſammenfließen zweier Pläne}{\lemma{\textnormal{\emph{Zuſammenfließen … Pläne}}}\Cendnote{\textnormal{Am 30. 4. 1897 überlegte
                     Schnitzler, die Stoffe »Die Entrüsteten«
                  und »Rettung« zusammenzufügen. Ersteres handelte vom Zusammenleben ohne zu
                  heiraten (in Anlehnung an sein Leben mit Marie
                     Reinhard\pwindex{Reinhard, Marie 1871-03-13 – 1899-03-18@\textsc{Reinhard, Marie} (1871-03-13 – 1899-03-18), \emph{Gesangspädagoge/Gesangspädagogin}|pwk}). Aus der vorliegenden Stelle geht hervor, dass der zweite Plan in Beziehung mit Hermine Benedict\pwindex{Schaffgotsch, Hermine von 25.11.1871 – 25.11.1928@\textsc{Schaffgotsch, Hermine von} (25.11.1871 – 25.11.1928)|pwk} steht. Aus dem Projekt, das in diesem
                  Stadium noch als Stück gedacht war, entwickelte sich im nächsten Jahrzehnt der
                  Roman \emph{Der Weg ins Freie}\pwindex{Weg ins Freie. Roman@\emph{Der Weg ins Freie. Roman}|pwk}.}}}\label{K_L00674-1}, worunter
               einer der mit der Minni\pwindex{Schaffgotsch, Hermine von 25.11.1871 – 25.11.1928@\textsc{Schaffgotsch, Hermine von} (25.11.1871 – 25.11.1928)|pw}, etwas gutes {\pb}eingefallen. –\pend
           
\pstart
           Den Götterliebling\pwindex{Tod Georgs@\emph{Der Tod Georgs}|pw} hoff ich ganz fertig
               anzutreffen. Bei dem Stück\pwindex{Agnes Jordan. Schauspiel in fuenf Akten@\emph{Agnes Jordan. Schauspiel in fünf Akten}|pwv} von
                  Hirſchf.\pwindex{Hirschfeld, Georg 11.02.1873 – 17.01.1942@\textsc{Hirschfeld, Georg} (11.02.1873 – 17.01.1942), \emph{Schriftsteller/Schriftstellerin}|pw} zweifle ich gar nicht daran. – Iſt
               bei Ben.\pwindex{Benedict, Marianne 01.01.1848 – 12.05.1930@\textsc{Benedict, Marianne} (01.01.1848 – 12.05.1930)|pw}\pwindex{Benedict, Markus 17.09.1834 – 26.2.1909@\textsc{Benedict, Markus} (17.09.1834 – 26.2.1909), \emph{Industrieller/Industrielle}|pw} nach mir gefragt
               worden? –\pend
           
\pstart
           Paul Goldma{\geminationn}\pwindex{Goldmann, Paul 31.01.1865 – 25.09.1935@\textsc{Goldmann, Paul} (31.01.1865 – 25.09.1935), \emph{Schriftsteller/Schriftstellerin, Journalist/Journalistin}|pw} hat unglaublich viel zu thun, u. we{\geminationn} ich ihn nicht
               gerade auf ſeinen Excurſionen zwiſchen Bureau u. Telegraphenamt begleite, wie z. B.
               geſtern, wo das Brandunglück im \textsc{Bazar de la Charité}\orgindex{Bazar de la Charite@Bazar de la Charité|pw} den Zeitungen ſo {\pb}viel zu thun gab, hab ich
               eigentlich wenig von ihm. Aber ſein Weſen macht mir ſehr viel Freude; und er gehört
               zu den wenigen, an denen ich mich erhole, von denen aus mir der Weg zu mir ſelbſt am
               freieſten und klarſten daliegt.\pend
           
\pstart
           Herzlich der Ihre{\\[\baselineskip]}\spacefill\mbox{Arth}\pend
           \leftskip=0em{}
\pstart
           Paris\oindex{Paris@\textbf{Paris}, \emph{P.PPLC}|pw}{ }6. 5. 97.\pend
           \selectlanguage{ngerman}\endnumbering\briefempfaengerindex{Hofmannsthal, Hugo von@\textsc{Hofmannsthal, Hugo von}!zzzSchnitzler, Arthur@\emph{von Arthur Schnitzler}!1897-05-061@{6. 5. 1897}|)be}\mylabel{L00674h}  \normalsize

\doendnotes{C}
\bigskip
\vfill

\clearpage

\footnotesize

\lohead{\textsc{register}}

% Definiere theindex-Environment komplett neu ohne reledmac
\makeatletter
\renewenvironment{theindex}{%
  \section*{\indexname}%
  \setlength{\parindent}{0pt}%
  \setlength{\parskip}{0pt plus 0.3pt}%
  \let\item\@idxitem
}{%
  \clearpage
}
\makeatother

\IfFileExists{\jobname-pw.ind}{\input{\jobname-pw.ind}}{}

\end{document}

      