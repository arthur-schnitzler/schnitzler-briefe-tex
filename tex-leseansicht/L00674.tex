%% latex-leseansicht-vorspann.tex
%% Vorspann für die Leseansicht.
%% Lädt die gemeinsame Datei latex-vorspann.tex mit nicht gesetztem Schalter.

\newif\ifkorrekturansicht
\korrekturansichtfalse

\input{../tex-inputs/latex-vorspann}


         
         \newcommand{\erwaehntePersonen}{Personen: Marianne Benedict, Markus Benedict, Marie Delna, Samuel Fischer, Paul Goldmann, Georg Hirschfeld, Hugo von Hofmannsthal, Regina Loeb, Louis Loeb, Rosa Papier, Clara Katharina Pollaczek, Marie Reinhard, Hermine von Schaffgotsch, Christine Schönberger, William Shakespeare,  Sophokles}
         \newcommand{\erwaehnteInstitutionen}{Institutionen: Bazar de la Charité}
         \newcommand{\erwaehnteOrte}{Orte: Alland, Bad Vöslau, Comédie française, Frankreich, Heiligenkreuz, Hütteldorf, Königstetten, Neuhaus, Nöstach, Paris, Pottenstein, Reichenau an der Rax, Rekawinkel, Rohrerhütte, Stockerau, Tulln an der Donau, Weidling, Wien, Wiener Neustadt}
         \newcommand{\erwaehnteWerke}{Werke: Agnes Jordan. Schauspiel in fünf Akten, Der Menschenfeind, Der Tod Georgs, Der Weg ins Freie. Roman, Mimi. Schattenbilder aus einem Mädchenleben, Orpheus und Eurydike, Wer nie sein Brod mit Thränen as}
               \section[Arthur Schnitzler an Hugo von Hofmannsthal, 6. 5. 1897]{ Arthur Schnitzler an Hugo von Hofmannsthal, 6. 5. 1897}\nopagebreak\mylabel{v}\rehead{ }\begin{ledgroupsized}[t]{13cm}\normalsize\beginnumbering \toendnotes[C]{\smallbreak\pagebreak[2]} \Standort{FDH, Hs-30885,13.}
\physDesc{Brief, 2 Blätter, 8 Seiten
\newline{}Handschrift: schwarze Tinte, deutsche Kurrent\newline{}Ordnung: von Schnitzler mutmaßlich bei der Durchsicht der Korrespondenz
                                    1929 auf dem ersten und zweiten Blatt mit Bleistift
                                 datiert: »6/5 97« }\buchAbdrucke{\weitereDrucke{Hugo von Hofmannsthal, Arthur Schnitzler: \emph{Briefwechsel}. Hg. Therese Nickl und Heinrich Schnitzler. Frankfurt am Main: \emph{S. Fischer} 1964, S. 84–85.} }\toendnotes[C]{\smallbreak}\pstart
           \noindent{}{\pb}Mein lieber Hugo, Fiſcher\pwindex{Fischer, Samuel 24.12.1859 – 15.10.1934@\textsc{Fischer, Samuel} (24.12.1859 – 15.10.1934), \emph{Verleger}|pw} hat den Satz von Mimi\pwindex{Mimi. Schattenbilder aus einem Maedchenleben1. 4. 1897@\emph{Mimi. Schattenbilder aus einem Mädchenleben} {[}1. 4. 1897{]}|pw} auf meinen Wunſch bereits ablegen laſſen, und ſo iſt die
               letzte Gefahr geſchwunden. Ich hoffe, Sie haben meinen vorigen, zweiten Brief, in dem
               ich Ihnen auf Ihr diesbezügliches Erſuchen geantwortet, erhalten? – Iſt es ruhig
               geworden im Hauſe Loeb\pwindex{Loeb, Regina 1850 – 5.2.1918@\textsc{Loeb, Regina} (1850 – 5.2.1918)|pw}\pwindex{Loeb, Louis 29.06.1842 – 06.06.1921@\textsc{Loeb, Louis} (29.06.1842 – 06.06.1921), \emph{Bankier}|pw}? – Wie geht es der
               geſchädigten Verfaſſerin\pwindex{Pollaczek, Clara Katharina 15.01.1875 – 22.07.1951@\textsc{Pollaczek, Clara Katharina} (15.01.1875 – 22.07.1951), \emph{Schriftstellerin}|pwv} der
                  Scenen aus einem
               Mädchenleben\pwindex{Mimi. Schattenbilder aus einem Maedchenleben1. 4. 1897@\emph{Mimi. Schattenbilder aus einem Mädchenleben} {[}1. 4. 1897{]}|pwv}?\pend
           \pstart
           – Die Delna\pwindex{Delna, Marie 03.04.1875 – 23.07.1932@\textsc{Delna, Marie} (03.04.1875 – 23.07.1932), \emph{Sängerin}|pw} hab ich ſchon gehört; gerade am Abend
               bevor Ihr Brief kam, als Orpheus\pwindex{\textcolor{red}{\textsuperscript{XXXX1 indx}}!Orpheus und Eurydike5. 10. 1762@\strich\emph{Orpheus und Eurydike} {[}5. 10. 1762{]}|pwv}.
               Sie hat eine {\pb}mächtige, nicht immer edle Stimme; eine
               beſondre Höhe der Darſtellung und des Geſangs erreicht ſie am Schluſs; da bin ich
               tief ergriffen geweſen – bis dahin hatt’ ich die Papier\pwindex{Papier, Rosa 1858-09-18 – 1932-02-09@\textsc{Papier, Rosa} (1858-09-18 – 1932-02-09), \emph{Sängerin, Gesangspädagogin}|pw} nicht vergeſſen können. –\pend
           \pstart
           Jetzt eben ko{\geminationm}e ich von einer \textsc{Matinée} im \textsc{Français}\oindex{Comedie française@\textbf{Comédie française}|pw}, wo man den \textsc{Misanthropen}\pwindex{\textcolor{red}{\textsuperscript{XXXX1 indx}}!Menschenfeind4. 6. 1666@\strich\emph{Der Menschenfeind} {[}4. 6. 1666{]}|pw} gegeben hat. Um hier der abſoluten Größe inne zu werden, muſs man ſich doch
               erſt hiſtoriſch montieren, was weder bei \textsc{Sophokles}\pwindex{Sophokles 497/496? v. u. Z. – 406/405 v. u. Z.@\textsc{Sophokles} (497/496? v. u. Z. – 406/405 v. u. Z.), \emph{Schriftsteller}|pw} noch bei \textsc{Shakespeare}\pwindex{Shakespeare, William 23.4.1564? – 03.05.1616@\textsc{Shakespeare, William} (23.4.1564? – 03.05.1616), \emph{Schauspieler, Dramatiker}|pw} notwendig iſt. Erſt im letzten Akt, {\pb}wo nicht mehr
                  \textsc{\uline{le} misanthrope}, ſondern \textsc{\uline{un} misanthrope} vor einem ſteht, ſpürt man was
               ewig menſchliches. Es liegt wohl daran, daſs alles, was in dieſem Stück\pwindex{\textcolor{red}{\textsuperscript{XXXX1 indx}}!Menschenfeind4. 6. 1666@\strich\emph{Der Menschenfeind} {[}4. 6. 1666{]}|pwv} vorgeht, einfach die Anſicht des Helden
               beſtätigt; er erfährt nichts neues, denn ſchon im erſten Auftritt weiſs er, was die
               Menſchen für ein Geſindel ſind. Erſt ſein Entſchluſs, in die Einſamkeit ſich
               zurückzuziehen, bewegt uns; wahrſcheinlich weil wir wiſſen, daſs ſeine ganze
               Menſchenfeindſchaft nichts {\pb}iſt als Sehnſucht nach guten
               Menſchen, die er jetzt ein für alle Mal ſelbſt zu etwas unerfüllbarem macht; denn er
               wird niemanden mehr kennen lernen. –\pend
           \pstart
           Tröſten Sie ſich wegen des gemiſchten Hausbrotes: Wochenlang hab ich ein weißes
               trocknes gegeſſen (wer nie ſein Brod mit Thränen
                  aſs\pwindex{\textcolor{red}{\textsuperscript{XXXX1 indx}}!Wer nie sein Brod mit Thraenen as1795 – 1796@\strich\emph{Wer nie sein Brod mit Thränen as} {[}1795 – 1796{]}|pw}– !); und auch jetzt nehm ich meine Mahlzeiten in einer ſtockfranzöſiſchen\oindex{Frankreich@\textbf{Frankreich}|pw} Familie ein, wo keine heimatlichen {\pb}Gulyasdüfte aufſteigen. Sie ahnen nicht, wie viel »ganz
               andres« ich eſſe. Die hieſige Einteilung 12 Uhr Dejeuner, 7 Diner, 9 Theater, behagt
               mir außerordentlich.\pend
           \pstart
           Schöne Radpartien? Z. B. fahren Sie von der Tini\pwindex{Schoenberger, Christine 1875-11-17 – 1971-02-03@\textsc{Schönberger, Christine} (1875-11-17 – 1971-02-03), \emph{Gastwirtin}|pw}
               aus über Heiligenkreuz\oindex{Heiligenkreuz@\textbf{Heiligenkreuz}|pw} – Alland\oindex{Alland@\textbf{Alland}|pw} – \uline{Neuhaus}\oindex{Neuhaus@\textbf{Neuhaus}|pw} (bei Nöſtach\oindex{Noestach@\textbf{Nöstach}|pw}) – Pottenſtein\oindex{Pottenstein@\textbf{Pottenstein}|pw} – Vöslau\oindex{Bad Voeslau@\textbf{Bad Vöslau}|pw}. Oder: Rohrerhütte\oindex{Rohrerhuette@\textbf{Rohrerhütte}|pw} – Königſtetten\oindex{Koenigstetten@\textbf{Königstetten}|pw} (ſehr bergig, ſchieben!) – Tulln\oindex{Tulln an der Donau@\textbf{Tulln an der Donau}|pw}, dann an der Donau zurück nach Kloſterneuburg\oindex{Weidling@\textbf{Weidling}|pw}. – Sehr hübſch auch die kleine Tour
                  Tulln\oindex{Tulln an der Donau@\textbf{Tulln an der Donau}|pw} – Stockerau\oindex{Stockerau@\textbf{Stockerau}|pw}. {\pb}Oder: Rekawinkel\oindex{Rekawinkel@\textbf{Rekawinkel}|pw} – Hütteldorf\oindex{Huetteldorf@\textbf{Hütteldorf}|pw} (Weſtbahnſtrecke.)
               Od: Wiener Neuſtadt\oindex{Wiener Neustadt@\textbf{Wiener Neustadt}|pw} – Reichenau\oindex{Reichenau an der Rax@\textbf{Reichenau an der Rax}|pw}. – Ich freue mich ſehr, we{\geminationn} wir
                  zuſa{\geminationm}en fahren werden.\pend
           \pstart
           Wie lang bleiben Sie de{\geminationn} in Wien\oindex{Wien@\textbf{Wien}|pw}? Und wie wird heuer der Sommer werden? Ich möchte ſo gern zum Arbeiten
                  ko{\geminationm}en; hier ſpiele ich höchſtens mit Plänen; aber
               möglicherweiſe iſt \substVorne{}\textsuperscript{mehr}\substDazwischen{}mir\substHinten{} durch ein merkwürdiges \label{K_L00674_1v}\edtext{Zuſammenfließen zweier Pläne}{\lemma{\textnormal{\emph{Zuſammenfließen … Pläne}}}\Cendnote{\textnormal{Am 30. 4. 1897 überlegte Schnitzler\pwindex{Schnitzler, Arthur 15.05.1862 – 21.10.1931@\textsc{Schnitzler, Arthur} (15.05.1862 – 21.10.1931), \emph{Schriftsteller, Mediziner}|pwk}, die Stoffe »Die Entrüsteten« und
                  »Rettung« zusammenzufügen. Ersteres handelte vom Zusammenleben ohne zu heiraten
                  (in Anlehnung an sein Leben mit Marie
                  Reinhard\pwindex{Reinhard, Marie 1871-03-13 – 1899-03-18@\textsc{Reinhard, Marie} (1871-03-13 – 1899-03-18), \emph{Gesangspädagogin}|pwk}), sodass der zweite in Beziehung mit Hermine Benedict\pwindex{Schaffgotsch, Hermine von 25.11.1871 – 25.11.1928@\textsc{Schaffgotsch, Hermine von} (25.11.1871 – 25.11.1928)|pwk} steht. Aus dem Projekt, das in diesem Stadium noch als
                  Stück gedacht war, entwickelte sich im nächsten Jahrzehnt der Roman \emph{Der Weg ins Freie}\pwindex{Schnitzler, Arthur 15.05.1862 – 21.10.1931@\textsc{Schnitzler, Arthur} (15.05.1862 – 21.10.1931), \emph{Schriftsteller, Mediziner}!Weg ins Freie. Roman1.1.1908 – 1.6.1908@\strich\emph{Der Weg ins Freie. Roman} {[}1.1.1908 – 1.6.1908{]}|pwk}.}}}\label{K_L00674_1h}, worunter einer der mit
               der Minni\pwindex{Schaffgotsch, Hermine von 25.11.1871 – 25.11.1928@\textsc{Schaffgotsch, Hermine von} (25.11.1871 – 25.11.1928)|pw}, etwas gutes {\pb}eingefallen. –\pend
           \pstart
           Den Götterliebling\pwindex{\textcolor{red}{\textsuperscript{XXXX1 indx}}!Tod Georgs1900@\strich\emph{Der Tod Georgs} {[}1900{]}|pw} hoff ich ganz fertig
               anzutreffen. Bei dem Stück\pwindex{Hirschfeld, Georg 11.02.1873 – 17.01.1942@\textsc{Hirschfeld, Georg} (11.02.1873 – 17.01.1942), \emph{Schriftsteller}!Agnes Jordan. Schauspiel in fuenf Akten1897@\strich\emph{Agnes Jordan. Schauspiel in fünf Akten} {[}1897{]}|pwv} von
                  Hirſchf.\pwindex{Hirschfeld, Georg 11.02.1873 – 17.01.1942@\textsc{Hirschfeld, Georg} (11.02.1873 – 17.01.1942), \emph{Schriftsteller}|pw} zweifle ich gar nicht daran. – Iſt
               bei Ben.\pwindex{Benedict, Marianne 01.01.1848 – 12.05.1930@\textsc{Benedict, Marianne} (01.01.1848 – 12.05.1930)|pw}\pwindex{Benedict, Markus 17.09.1834 – 26.2.1909@\textsc{Benedict, Markus} (17.09.1834 – 26.2.1909), \emph{Industrieller}|pw} nach mir gefragt worden? –\pend
           \pstart
           Paul Goldma{\geminationn}\pwindex{Goldmann, Paul 31.01.1865 – 25.09.1935@\textsc{Goldmann, Paul} (31.01.1865 – 25.09.1935), \emph{Schriftsteller, Journalist}|pw} hat unglaublich viel zu thun, u. we{\geminationn} ich ihn nicht
               gerade auf ſeinen Excurſionen zwiſchen Bureau u. Telegraphenamt begleite, wie z. B.
               geſtern, wo das Brandunglück im \textsc{Bazar de la Charité}\orgindex{Bazar de la Charite@Bazar de la Charité|pw} den Zeitungen ſo {\pb}viel zu thun gab, hab ich
               eigentlich wenig von ihm. Aber ſein Weſen macht mir ſehr viel Freude; und er gehört
               zu den wenigen, an denen ich mich erhole, von denen aus mir der Weg zu mir ſelbſt am
               freieſten und klarſten daliegt.\pend
           \pstart
           Herzlich der Ihre{\\[\baselineskip]}\spacefill\mbox{Arth}\pend
           \leftskip=0em{}\pstart
           Paris\oindex{Paris@\textbf{Paris}|pw}{ }6. 5. 97.\pend
           
         
         \endnumbering\mylabel{h}\end{ledgroupsized}  \newcommand{\dateiname}{L00674}\newcommand{\titel}{Arthur Schnitzler an Hugo von Hofmannsthal, 6. 5. 1897}\newcommand{\editorInnen}{Martin Anton Müller und Gerd-Hermann Susen}%% latex-leseansicht-abspann.tex
%% Abspann für die Leseansicht.
%% Der Schalter \ifkorrekturansicht ist bereits durch den Vorspann gesetzt.

%% latex-abspann.tex
%% Gemeinsamer Abspann für Korrekturansicht und Leseansicht.
%% Setzt den Schalter \ifkorrekturansicht voraus (gesetzt in den
%% einbindenden Dateien latex-korrekturansicht-abspann.tex bzw.
%% latex-leseansicht-abspann.tex).
%% ---------------------------------------------------------------

\normalsize

% Das esempio-Environment wird nur in der Leseansicht benötigt
\ifkorrekturansicht\else
\newenvironment{esempio}[3]%
{
    \vspace{1.5ex}
    \rlap{\underline{#1}}
    \par
    \setlength{\parindent}{0cm}
    \nopagebreak
    \leftskip=#2cm
    \rightskip=#3cm
}
{
    \par
}
\fi

\doendnotes{C}
\bigskip
\vfill

\clearpage

\footnotesize

\ifkorrekturansicht
  \lohead{\textsc{register}}
\fi

% theindex-Environment neu definieren ohne reledmac
\makeatletter
\renewenvironment{theindex}{%
  \ifkorrekturansicht
    \section*{\indexname}%
  \else
    \subsubsection*{Index der erwähnten Entitäten}%
  \fi
  \setlength{\parindent}{0pt}%
  \setlength{\parskip}{0pt plus 0.3pt}%
  \let\item\@idxitem
}{%
  \ifkorrekturansicht\clearpage\fi
}
\makeatother

\IfFileExists{\jobname-pw.ind}{\input{\jobname-pw.ind}}{}

% Quellenangabe nur in der Leseansicht
\ifkorrekturansicht\else
% Fallback-Definitionen, falls die .tex-Datei \titel etc. nicht gesetzt hat
\providecommand{\titel}{}
\providecommand{\editorInnen}{}
\providecommand{\dateiname}{\jobname}

\vspace{3cm}

\vfill

\footnotesize
\textsc{Quelle}: \titel. Herausgegeben von {\editorInnen}. In: \emph{Arthur Schnitzler: Briefwechsel mit Autorinnen und Autoren}.
 Digitale Edition, https://schnitzler-briefe.acdh.oeaw.ac.at/{\dateiname}.html (Stand \today)
\fi

\end{document}


      