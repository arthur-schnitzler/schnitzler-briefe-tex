%% latex-leseansicht-vorspann.tex
%% Vorspann für die Leseansicht.
%% Lädt die gemeinsame Datei latex-vorspann.tex mit nicht gesetztem Schalter.

\newif\ifkorrekturansicht
\korrekturansichtfalse

\input{../tex-inputs/latex-vorspann}


\section[ Arthur Schnitzler: Widmungsexemplar Anatol für Felix Salten, 29. 9. 1900]{L03601 Arthur Schnitzler: Widmungsexemplar Anatol für Felix
               Salten,  29. 9. 1900}
\nopagebreak\mylabel{L03601v}
\rehead{ }\normalsize\beginnumbering\briefempfaengerindex{Salten, Felix@\textsc{Salten, Felix}!zzzSchnitzler, Arthur@\emph{von Arthur Schnitzler}!1900-09-291@{29. 9. 1900}|(be}
\toendnotes[C]{\smallbreak\pagebreak[2]}
\correspDesc{Versand  durch Arthur Schnitzler am 29. 9. 1900 in Wien
\newline{}Erhalt  durch Felix Salten im Zeitraum [29. 9. 1900
                  – 3. 10. 1900?] in Wien}\toendnotes[C]{\smallbreak}
\Standort{Wienbibliothek im Rathaus, A-72116/2.Ex., DS-2018-9601.}
\physDesc{Widmung am Schmutztitel, 53 Zeichen
\newline{}Handschrift: schwarze Tinte, deutsche Kurrent
\newline{}Salten: 1) mit schwarzer Tinte am Titelblatt gestrichene Regalerfassung: »\noindent{}IN\textsuperscript{o} 2458 WN\textsuperscript{o} 1527{ / }XI b«  2) mit schwarzer Tinte ausgefüllter Stempel: »\noindent{}\textcolor{gray}{\textbf{\textit{Felix Salten}}}{ / }\textcolor{gray}{\textbf{\textit{Inv. Nr.}}}{ }4470{ / }\textcolor{gray}{\textbf{\textit{Werk Nr.}}}{ }2198{ / }\textcolor{gray}{\textbf{\textit{Schrank}}}{ }XIV A. Z. \textcolor{gray}{\textbf{\textit{Fach}}} b« }
\pstart
           \noindent{}{\pb}Meinem lieben Felix Salten\pend
           \pstart \spacefill\mbox{Arth Sch}\pend{}
\pstart
           Wien\oindex{Wien@\textbf{Wien}, \emph{Verwaltungsgebiet}|pw}{ }29. 9. 900.\pend
           {\vspace{1\baselineskip}}
\pstart
           \centering{}\textcolor{gray}{\textbf{ANATOL\pwindex{Schnitzler, Arthur 15.\,5.\,1862 Wien – 21.\,10.\,1931 ebd.@\textsc{Schnitzler, Arthur} (15.\,5.\,1862 Wien – 21.\,10.\,1931 ebd.), \emph{Schriftsteller, Mediziner}!Anatol@\strich\emph{Anatol}|pw}}}\pend
           \selectlanguage{ngerman}\vspace{1em}{\vspace{1\baselineskip}}
\pstart
           \centering{}{\pb}\textcolor{gray}{\textbf{ARTHUR SCHNITZLER}}\pend
           
\pstart
           \centering{}\textcolor{gray}{\textbf{ANATOL\pwindex{Schnitzler, Arthur 15.\,5.\,1862 Wien – 21.\,10.\,1931 ebd.@\textsc{Schnitzler, Arthur} (15.\,5.\,1862 Wien – 21.\,10.\,1931 ebd.), \emph{Schriftsteller, Mediziner}!Anatol. Illustriert@\strich\emph{Anatol. Illustriert}|pw}}}\pend
           
\pstart
           \centering{}\textcolor{gray}{\textbf{illustriert von}}\pend
           
\pstart
           \centering{}\textcolor{gray}{\textbf{M. COSCHELL.\pwindex{Coschell, Moritz 18.\,9.\,1872 Wien – 11.\,7.\,1943 ebd.@\textsc{Coschell, Moritz} (18.\,9.\,1872 Wien – 11.\,7.\,1943 ebd.), \emph{Maler}|pw}}}\pend
           {\vspace{1\baselineskip}}
\pstart
           \centering{}\textcolor{gray}{\textbf{Erste bis dritte Auflage}}\pend
           
\pstart
           \centering{}\textcolor{gray}{\textbf{Berlin\oindex{Berlin@\textbf{Berlin}, \emph{Hauptstadt}|pw}}}\pend
           
\pstart
           \centering{}\textcolor{gray}{\textbf{\so{S. Fischer, Verlag}\orgindex{S. Fischer Verlag@S. Fischer Verlag|pw}}}\pend
           
\pstart
           \centering{}\textcolor{gray}{\textbf{1901.}}\pend
           \selectlanguage{ngerman}\endnumbering\briefempfaengerindex{Salten, Felix@\textsc{Salten, Felix}!zzzSchnitzler, Arthur@\emph{von Arthur Schnitzler}!1900-09-291@{29. 9. 1900}|)be}\mylabel{L03601h}  \newcommand{\dateiname}{L03601}\newcommand{\titel}{Arthur Schnitzler: Widmungsexemplar Anatol für Felix Salten, 29. 9. 1900}\newcommand{\editorInnen}{Martin Anton Müller und Laura Untner}%% latex-leseansicht-abspann.tex
%% Abspann für die Leseansicht.
%% Der Schalter \ifkorrekturansicht ist bereits durch den Vorspann gesetzt.

%% latex-abspann.tex
%% Gemeinsamer Abspann für Korrekturansicht und Leseansicht.
%% Setzt den Schalter \ifkorrekturansicht voraus (gesetzt in den
%% einbindenden Dateien latex-korrekturansicht-abspann.tex bzw.
%% latex-leseansicht-abspann.tex).
%% ---------------------------------------------------------------

\normalsize

% Das esempio-Environment wird nur in der Leseansicht benötigt
\ifkorrekturansicht\else
\newenvironment{esempio}[3]%
{
    \vspace{1.5ex}
    \rlap{\underline{#1}}
    \par
    \setlength{\parindent}{0cm}
    \nopagebreak
    \leftskip=#2cm
    \rightskip=#3cm
}
{
    \par
}
\fi

\doendnotes{C}
\bigskip
\vfill

\clearpage

\footnotesize

\ifkorrekturansicht
  \lohead{\textsc{register}}
\fi

% theindex-Environment neu definieren ohne reledmac
\makeatletter
\renewenvironment{theindex}{%
  \ifkorrekturansicht
    \section*{\indexname}%
  \else
    \subsubsection*{Index der erwähnten Entitäten}%
  \fi
  \setlength{\parindent}{0pt}%
  \setlength{\parskip}{0pt plus 0.3pt}%
  \let\item\@idxitem
}{%
  \ifkorrekturansicht\clearpage\fi
}
\makeatother

\IfFileExists{\jobname-pw.ind}{\input{\jobname-pw.ind}}{}

% Quellenangabe nur in der Leseansicht
\ifkorrekturansicht\else
% Fallback-Definitionen, falls die .tex-Datei \titel etc. nicht gesetzt hat
\providecommand{\titel}{}
\providecommand{\editorInnen}{}
\providecommand{\dateiname}{\jobname}

\vspace{3cm}

\vfill

\footnotesize
\textsc{Quelle}: \titel. Herausgegeben von {\editorInnen}. In: \emph{Arthur Schnitzler: Briefwechsel mit Autorinnen und Autoren}.
 Digitale Edition, https://schnitzler-briefe.acdh.oeaw.ac.at/{\dateiname}.html (Stand \today)
\fi

\end{document}


