%% latex-leseansicht-vorspann.tex
%% Vorspann für die Leseansicht.
%% Lädt die gemeinsame Datei latex-vorspann.tex mit nicht gesetztem Schalter.

\newif\ifkorrekturansicht
\korrekturansichtfalse

\input{../tex-inputs/latex-vorspann}


         
         \renewcommand{\erwaehntePersonen}{Personen: Robert Ehrhart-Ehrhartstein, Hugo von Hofmannsthal}
         \renewcommand{\erwaehnteOrte}{Orte: Café Pfob, Volkstheater, Wien}
         \renewcommand{\erwaehnteWerke}{Werke: Die kleine Lydia, Musotte}
               \section[Hugo von Hofmannsthal an Arthur Schnitzler, {[}8. 11. 1892{]}]{ Hugo von Hofmannsthal an Arthur Schnitzler, {[}8. 11. 1892{]}}\nopagebreak\mylabel{v}\rehead{ }\begin{ledgroupsized}[t]{13cm}\normalsize\beginnumbering\briefempfaengerindex{Schnitzler, Arthur@\textsc{Schnitzler, Arthur}!zzzHofmannsthal, Hugo von@\emph{von Hugo von Hofmannsthal}!1892-11-081@{{[}8. 11. 1892{]}}|(be} \toendnotes[C]{\smallbreak\pagebreak[2]} \Standort{CUL, Schnitzler, B 43.}
\physDesc{Brief, 1 Blatt, 2 Seiten, 523 Zeichen (aufgeprägtes Wappen)
\newline{}Handschrift: schwarze Tinte, deutsche Kurrent
\newline{}Schnitzler: mit Bleistift nummeriert: »33« und datiert: »Nov. 92« }\buchAbdrucke{\weitereDrucke{Hugo von Hofmannsthal, Arthur Schnitzler: \emph{Briefwechsel}. Hg. Therese Nickl und Heinrich Schnitzler. Frankfurt am Main: \emph{S. Fischer} 1964, S. 30.} }\toendnotes[C]{\smallbreak}\pstart
           \raggedleft{}{\pb}Dienstag.\pend
           \pstart\center{}lieber Doctor.\pend\pstart
           Ich kann leider einer Familienverpflichtung wegen abſolut nicht zu \textsc{Pfob}\oindex{Cafe Pfob@\textbf{Café Pfob}|pw} kommen. \label{K_L00132-1v}\edtext{Samſtag}{\lemma{\textnormal{\emph{Samſtag}}}\Cendnote{\textnormal{Erstaufführung im Deutschen Volkstheater\oindex{Volkstheater@\textbf{Volkstheater}|pwk} am 12. 11. 1892}}}\label{K_L00132-1h} gehe ich in »\textsc{Musotte}\pwindex{\textcolor{red}{\textsuperscript{XXXX1 indx}}!Musotte1891@\strich\emph{Musotte} {[}1891{]}|pw}\pwindex{\textcolor{red}{\textsuperscript{XXXX1 indx}}!Musotte1891@\strich\emph{Musotte} {[}1891{]}|pw}«; könnten wir nicht miteinander ſoupieren? bitte gelegentlich Antwort. Falls
                  \textsc{Robert Ehrhart}\pwindex{Ehrhart-Ehrhartstein, Robert 12.09.1870 – 11.11.1956@\textsc{Ehrhart-Ehrhartstein, Robert} (12.09.1870 – 11.11.1956), \emph{Schriftsteller, Ministerialbeamter}|pw} da iſt, ſo ſagen Sie ihm, bitte, daß ich ſeinen leider wieder verfehlten Beſuch
                  {\pb}wenn er mir nicht abſchreibt,
                  Donnerstag zwiſchen 10 u 11 erwidern werde,
               um über die Novelle\pwindex{Ehrhart-Ehrhartstein, Robert 12.09.1870 – 11.11.1956@\textsc{Ehrhart-Ehrhartstein, Robert} (12.09.1870 – 11.11.1956), \emph{Schriftsteller, Ministerialbeamter}!kleine Lydia1892@\strich\emph{Die kleine Lydia} {[}1892{]}|pwuv}
               zu reden. Ich finde ſie ſehr gut gemacht und wenn auch ein bißchen \label{K_L00132-2v}\edtext{\textsc{vieux jeu}}{\lemma{\textnormal{\emph{vieux jeu}}}\Cendnote{\textnormal{französisch: altes Spiel}}}\label{K_L00132-2h}, doch im
               ganzen fertig u. verwendbar.\pend
           \pstart
           Grüße alle herzlichſt\pend
           \pstart \spacefill\mbox{Loris.}\pend{}
         
         \endnumbering\mylabel{h}\end{ledgroupsized}  \newcommand{\dateiname}{L00132}\newcommand{\titel}{Hugo von Hofmannsthal an Arthur Schnitzler, [8. 11. 1892]}\newcommand{\editorInnen}{Martin Anton Müller und Gerd-Hermann Susen}%% latex-leseansicht-abspann.tex
%% Abspann für die Leseansicht.
%% Der Schalter \ifkorrekturansicht ist bereits durch den Vorspann gesetzt.

%% latex-abspann.tex
%% Gemeinsamer Abspann für Korrekturansicht und Leseansicht.
%% Setzt den Schalter \ifkorrekturansicht voraus (gesetzt in den
%% einbindenden Dateien latex-korrekturansicht-abspann.tex bzw.
%% latex-leseansicht-abspann.tex).
%% ---------------------------------------------------------------

\normalsize

% Das esempio-Environment wird nur in der Leseansicht benötigt
\ifkorrekturansicht\else
\newenvironment{esempio}[3]%
{
    \vspace{1.5ex}
    \rlap{\underline{#1}}
    \par
    \setlength{\parindent}{0cm}
    \nopagebreak
    \leftskip=#2cm
    \rightskip=#3cm
}
{
    \par
}
\fi

\doendnotes{C}
\bigskip
\vfill

\clearpage

\footnotesize

\ifkorrekturansicht
  \lohead{\textsc{register}}
\fi

% theindex-Environment neu definieren ohne reledmac
\makeatletter
\renewenvironment{theindex}{%
  \ifkorrekturansicht
    \section*{\indexname}%
  \else
    \subsubsection*{Index der erwähnten Entitäten}%
  \fi
  \setlength{\parindent}{0pt}%
  \setlength{\parskip}{0pt plus 0.3pt}%
  \let\item\@idxitem
}{%
  \ifkorrekturansicht\clearpage\fi
}
\makeatother

\IfFileExists{\jobname-pw.ind}{\input{\jobname-pw.ind}}{}

% Quellenangabe nur in der Leseansicht
\ifkorrekturansicht\else
% Fallback-Definitionen, falls die .tex-Datei \titel etc. nicht gesetzt hat
\providecommand{\titel}{}
\providecommand{\editorInnen}{}
\providecommand{\dateiname}{\jobname}

\vspace{3cm}

\vfill

\footnotesize
\textsc{Quelle}: \titel. Herausgegeben von {\editorInnen}. In: \emph{Arthur Schnitzler: Briefwechsel mit Autorinnen und Autoren}.
 Digitale Edition, https://schnitzler-briefe.acdh.oeaw.ac.at/{\dateiname}.html (Stand \today)
\fi

\end{document}


      