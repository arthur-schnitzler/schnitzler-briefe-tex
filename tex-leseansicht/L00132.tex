%% latex-korrekturansicht-vorspann.tex
%% Vorspann für die Korrekturansicht.
%% Lädt die gemeinsame Datei latex-vorspann.tex mit gesetztem Schalter.

\newif\ifkorrekturansicht
\korrekturansichttrue

\input{../tex-inputs/latex-vorspann}


\section[Hugo von Hofmannsthal an Arthur Schnitzler, {[}8. 11. 1892{]}]{L00132 Hugo von Hofmannsthal an Arthur Schnitzler, {[}8. 11. 1892{]}}
\nopagebreak\mylabel{L00132v}
\rehead{ }\normalsize\beginnumbering\briefempfaengerindex{Schnitzler, Arthur@\textsc{Schnitzler, Arthur}!zzzHofmannsthal, Hugo von@\emph{von Hugo von Hofmannsthal}!1892-11-081@{{[}8. 11. 1892{]}}|(be}
\toendnotes[C]{\smallbreak\pagebreak[2]}\Standort{CUL, Schnitzler, B 43.}
\physDesc{Brief, 1 Blatt, 2 Seiten, 523 Zeichen (aufgeprägtes Wappen)
\newline{}Handschrift: schwarze Tinte, deutsche Kurrent
\newline{}Schnitzler: mit Bleistift nummeriert: »33« und datiert: »Nov. 92« }
\buchAbdrucke{\weitereDrucke{Hugo von Hofmannsthal, Arthur Schnitzler: \emph{Briefwechsel}. Frankfurt am Main: \emph{S. Fischer} 1964, S. 30.} }\toendnotes[C]{\smallbreak}
\pstart
           \raggedleft{}{\pb}Dienstag.\pend
           
\pstart\center{}lieber Doctor.\pend\vspace{0.5em}
\pstart
           Ich kann leider einer Familienverpflichtung wegen abſolut nicht zu \textsc{Pfob}\oindex{Cafe Pfob@\textbf{Café Pfob}, \emph{Kaffeehaus (K.KAF)}|pw} kommen. \label{K_L00132-1v}\edtext{Samſtag}{\lemma{\textnormal{\emph{Samſtag}}}\Cendnote{\textnormal{Die Erstaufführung von \emph{Musotte}\pwindex{Musotte@\emph{Musotte}|pwk} fand am 12. 11. 1892 im Deutschen Volkstheater\oindex{Volkstheater@\textbf{Volkstheater}, \emph{Theater (K.THE)}|pwk} statt.
               }}}\label{K_L00132-1} gehe ich in »\textsc{Musotte}\pwindex{Musotte@\emph{Musotte}|pw}«; könnten wir nicht miteinander ſoupieren? bitte gelegentlich Antwort. Falls
                  \textsc{Robert Ehrhart}\pwindex{Ehrhart-Ehrhartstein, Robert 12.09.1870 – 11.11.1956@\textsc{Ehrhart-Ehrhartstein, Robert} (12.09.1870 – 11.11.1956), \emph{Schriftsteller/Schriftstellerin, Ministerialbeamter/Ministerialbeamte}|pw} da iſt, ſo ſagen Sie ihm, bitte, daß ich ſeinen leider wieder verfehlten Beſuch
                  {\pb}wenn er mir nicht abſchreibt,
                  Donnerstag zwiſchen 10 u 11 erwidern werde,
               um über die Novelle\pwindex{kleine Lydia@\emph{Die kleine Lydia}|pwuv}
               zu reden. Ich finde ſie ſehr gut gemacht und wenn auch ein bißchen \label{K_L00132-2v}\edtext{\textsc{vieux jeu}}{\lemma{\textnormal{\emph{vieux jeu}}}\Cendnote{\textnormal{französisch: altes Spiel}}}\label{K_L00132-2}, doch im
               ganzen fertig u. verwendbar.\pend
           
\pstart
           Grüße alle herzlichſt\pend
           \pstart \spacefill\mbox{Loris.}\pend{}\selectlanguage{ngerman}\endnumbering\briefempfaengerindex{Schnitzler, Arthur@\textsc{Schnitzler, Arthur}!zzzHofmannsthal, Hugo von@\emph{von Hugo von Hofmannsthal}!1892-11-081@{{[}8. 11. 1892{]}}|)be}\mylabel{L00132h}  \normalsize

\doendnotes{C}
\bigskip
\vfill

\clearpage

\footnotesize

\lohead{\textsc{register}}

% Definiere theindex-Environment komplett neu ohne reledmac
\makeatletter
\renewenvironment{theindex}{%
  \section*{\indexname}%
  \setlength{\parindent}{0pt}%
  \setlength{\parskip}{0pt plus 0.3pt}%
  \let\item\@idxitem
}{%
  \clearpage
}
\makeatother

\IfFileExists{\jobname-pw.ind}{\input{\jobname-pw.ind}}{}

\end{document}

      