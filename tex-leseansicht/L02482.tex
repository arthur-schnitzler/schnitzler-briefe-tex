\input{../tex-inputs/latex-pdf-vorspann}
\begin{center}
            \textcolor{red}{ENTWURF. ENTZIFFERUNG NOCH NICHT KORREKTURGELESEN}
                      \end{center}
            
               \section[Arthur Schnitzler an Hugo Hofmannsthal, 26. 2. 1927]{ Arthur Schnitzler an Hugo Hofmannsthal, 26. 2. 1927}\nopagebreak\mylabel{v}\rehead{ }\begin{ledgroupsized}[t]{13cm}\normalsize\beginnumbering\briefempfaengerindex{Hofmannsthal, Hugo von@\textsc{Hofmannsthal, Hugo von}!zzzSchnitzler, Arthur@\emph{von Arthur Schnitzler}!1927-02-261@{26. 2. 1927}|(be} \toendnotes[C]{\smallbreak\pagebreak[2]} \Standort{FDH, Hs-30885,157.}
\physDesc{Postkarte
\newline{}Handschrift: Bleistift, lateinische Kurrent}\buchAbdrucke{\weitereDrucke{Hugo von Hofmannsthal, Arthur Schnitzler: \emph{Briefwechsel}. Hg. Therese Nickl und Heinrich Schnitzler. Frankfurt am Main: \emph{S. Fischer} 1964, S. 307.} }\toendnotes[C]{\smallbreak}\pstart{}{\pb}\label{T_L02482-1v}\edtext{\textcolor{gray}{\textbf{A. S.}}}{\lemma{\textnormal{\emph{A. S.}}}\Cendnote{\textnormal{ovaler Absenderkleber}}}\label{T_L02482-1h}\pend{}\pstart{}\textcolor{gray}{\textbf{WIEN, XVIII.}}\oindex{XVIII., Waehring@\textbf{XVIII., Währing}|pw}\pend{}\pstart{}\textcolor{gray}{\textbf{STERNWARTESTR. 71}}\oindex{Sternwartestrasse@\textbf{Sternwartestraße}|pw}\pend{}{\bigskip}\pstart{}Herrn Hugo v Hofmannsthal,\pend{}\pstart{}Rodaun\oindex{Badgasse@\textbf{Badgasse}|pw}\pend{}\pstart{}bei Wien-Liesing\oindex{Rodaun@\textbf{Rodaun}|pw}\pend{}{\bigskip}\pstart
           \raggedleft{}{\pb}Wien\oindex{Wien@\textbf{Wien}|pw}, 26. 2. 927\pend
           \pstart
           mein lieber Hugo, ich danke Ihnen für Ihren Gruſs aus Girgenti\oindex{Agrigento@\textbf{Agrigento}|pw}.\pend
           \pstart
           Der treffliche Regisseur \uline{Schulbaur}\pwindex{Schulbaur, Heinz 30.12.1884 – 03.07.1964@\textsc{Schulbaur, Heinz} (30.12.1884 – 03.07.1964), \emph{Regisseur}|pw}, früher Volkstheater\oindex{Volkstheater@\textbf{Volkstheater}|pw} wendet sich an mich:
                    ich möchte seine Bitte bei Ihnen unterstützen. Er will in der Akademie\oindex{Hochschule und Akademie fuer Musik und Darstellende Kunst@\textbf{Hochschule und Akademie für Musik und Darstellende Kunst}|pw} mit seinen Schülern den weißen Fächer\pwindex{Hofmannsthal, Hugo von 01.02.1874 – 15.07.1929@\textsc{Hofmannsthal, Hugo von} (01.02.1874 – 15.07.1929), \emph{Schriftsteller}!weisse Faecher. Ein Zwischenspiel05. 02. 1898@\strich\emph{Der weiße Fächer. Ein Zwischenspiel} {[}05. 02. 1898{]}|pw} aufführen. Sie werden wohl nichts dagegen
                    haben, so wenig ich mich gegen dergleichen zu wehren pflege.\pend
           \pstart
           Auf Wiedersehen nach Ihrer Rückkehr\hspace*{1em}Ich wünsche
                    Ihnen weiterhin schöne Sicilianer\oindex{Sizilien@\textbf{Sizilien}|pw} Tage. Ich war
                        1904 in \label{K_L02482_1v}\edtext{Taormina\oindex{Taormina@\textbf{Taormina}|pw}}{\lemma{\textnormal{\emph{Taormina}}}\Cendnote{\textnormal{vgl. A. S.: \emph{Tagebuch}, 19. 5. 1904}}}\label{K_L02482_1h} u \label{K_L02482_2v}\edtext{Syrakus\oindex{Syrakus@\textbf{Syrakus}|pw}}{\lemma{\textnormal{\emph{Syrakus}}}\Cendnote{\textnormal{vgl. A. S.: \emph{Tagebuch}, 17. 5. 1904}}}\label{K_L02482_2h}.\pend
           \pstart Herzlichst Ihr \spacefill\mbox{Arthur}\pend{}\endnumbering\briefempfaengerindex{Hofmannsthal, Hugo von@\textsc{Hofmannsthal, Hugo von}!zzzSchnitzler, Arthur@\emph{von Arthur Schnitzler}!1927-02-261@{26. 2. 1927}|)be}\mylabel{h}\end{ledgroupsized}  \newcommand{\dateiname}{L02482}\newcommand{\titel}{Arthur Schnitzler an Hugo Hofmannsthal, 26. 2. 1927}\newcommand{\editorInnen}{Martin Anton Müller und Gerd-Hermann Susen}\input{../tex-inputs/latex-pdf-abspann}
      