%% latex-korrekturansicht-vorspann.tex
%% Vorspann für die Korrekturansicht.
%% Lädt die gemeinsame Datei latex-vorspann.tex mit gesetztem Schalter.

\newif\ifkorrekturansicht
\korrekturansichttrue

\input{../tex-inputs/latex-vorspann}


\section[Arthur Schnitzler an Hugo Hofmannsthal, 26. 2. 1927]{L02482 Arthur Schnitzler an Hugo Hofmannsthal, 26. 2. 1927}
\nopagebreak\mylabel{L02482v}
\rehead{ }\normalsize\beginnumbering\briefempfaengerindex{Hofmannsthal, Hugo von@\textsc{Hofmannsthal, Hugo von}!zzzSchnitzler, Arthur@\emph{von Arthur Schnitzler}!1927-02-261@{26. 2. 1927}|(be}
\toendnotes[C]{\smallbreak\pagebreak[2]}\Standort{FDH, Hs-30885,157.}
\physDesc{Postkarte, 540 Zeichen
\newline{}Handschrift: Bleistift, lateinische Kurrent}
\buchAbdrucke{\weitereDrucke{Hugo von Hofmannsthal, Arthur Schnitzler: \emph{Briefwechsel}. Frankfurt am Main: \emph{S. Fischer} 1964, S. 307.} }\toendnotes[C]{\smallbreak}\pstart{}{\pb}\label{T_L02482-1v}\edtext{\textcolor{gray}{\textbf{A. S.}}}{\lemma{\textnormal{\emph{A. S.}}}\Cendnote{\textnormal{ovaler Absenderkleber}}}\label{T_L02482-1}\pend{}\pstart{}\textcolor{gray}{\textbf{WIEN, XVIII.}}\oindex{XVIII., Waehring@\textbf{XVIII., Währing}, \emph{A.ADM3}|pw}\pend{}\pstart{}\textcolor{gray}{\textbf{STERNWARTESTR. 71}}\oindex{Sternwartestrasse 71@\textbf{Sternwartestraße 71}, \emph{Wohngebäude (K.WHS)}|pw}\pend{}{\bigskip}\pstart{}Herrn Hugo v Hofmannsthal,\pend{}\pstart{}Rodaun\oindex{Badgasse@\textbf{Badgasse}, \emph{Straße (K.STR)}|pw}\pend{}\pstart{}bei Wien-Liesing\oindex{Rodaun@\textbf{Rodaun}, \emph{A.ADM4}|pw}\pend{}{\bigskip}\vspace{1em}
\pstart
           \raggedleft{}{\pb}Wien\oindex{Wien@\textbf{Wien}, \emph{A.ADM2}|pw}, 26. 2. 927\pend
           \vspace{0.5em}
\pstart
           mein lieber Hugo, ich danke Ihnen für Ihren Gruſs aus Girgenti\oindex{Agrigento@\textbf{Agrigento}, \emph{L.LCTY}|pw}.\pend
           
\pstart
           Der treffliche Regisseur \uline{Schulbaur}\pwindex{Schulbaur, Heinz 30.12.1884 – 03.07.1964@\textsc{Schulbaur, Heinz} (30.12.1884 – 03.07.1964), \emph{Regisseur/Regisseurin}|pw}, früher Volkstheater\oindex{Volkstheater@\textbf{Volkstheater}, \emph{Theater (K.THE)}|pw} wendet sich an mich:
               ich möchte seine Bitte bei Ihnen unterstützen. Er will in der Akademie\oindex{Hochschule und Akademie fuer Musik und Darstellende Kunst@\textbf{Hochschule und Akademie für Musik und Darstellende Kunst}, \emph{Universität (K.UNI)}|pw} mit seinen Schülern den weißen Fächer\pwindex{weisse Faecher. Ein Zwischenspiel@\emph{Der weiße Fächer. Ein Zwischenspiel}|pw} aufführen. Sie werden wohl nichts dagegen haben, so wenig ich
               mich gegen dergleichen zu wehren pflege.\pend
           
\pstart
           Auf Wiedersehen nach Ihrer Rückkehr\hspace*{1em}Ich wünsche
               Ihnen weiterhin schöne Sicilianer\oindex{Sizilien@\textbf{Sizilien}, \emph{A.ADM1}|pw} Tage. Ich war
                  1904 in \label{K_L02482-1v}\edtext{Taormina\oindex{Taormina@\textbf{Taormina}, \emph{P.PPLA3}|pw}}{\lemma{\textnormal{\emph{Taormina}}}\Cendnote{\textnormal{Vgl. A. S.: \emph{Tagebuch}, 19. 5. 1904.
               }}}\label{K_L02482-1} u \label{K_L02482-2v}\edtext{Syrakus\oindex{Syrakus@\textbf{Syrakus}, \emph{P.PPLA2}|pw}}{\lemma{\textnormal{\emph{Syrakus}}}\Cendnote{\textnormal{Vgl. A. S.: \emph{Tagebuch}, 17. 5. 1904.
               }}}\label{K_L02482-2}.\pend
           \pstart Herzlichst Ihr \spacefill\mbox{Arthur}\pend{}\selectlanguage{ngerman}\endnumbering\briefempfaengerindex{Hofmannsthal, Hugo von@\textsc{Hofmannsthal, Hugo von}!zzzSchnitzler, Arthur@\emph{von Arthur Schnitzler}!1927-02-261@{26. 2. 1927}|)be}\mylabel{L02482h}  \normalsize

\doendnotes{C}
\bigskip
\vfill

\clearpage

\footnotesize

\lohead{\textsc{register}}

% Definiere theindex-Environment komplett neu ohne reledmac
\makeatletter
\renewenvironment{theindex}{%
  \section*{\indexname}%
  \setlength{\parindent}{0pt}%
  \setlength{\parskip}{0pt plus 0.3pt}%
  \let\item\@idxitem
}{%
  \clearpage
}
\makeatother

\IfFileExists{\jobname-pw.ind}{\input{\jobname-pw.ind}}{}

\end{document}

      