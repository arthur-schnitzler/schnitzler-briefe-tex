%% latex-leseansicht-vorspann.tex
%% Vorspann für die Leseansicht.
%% Lädt die gemeinsame Datei latex-vorspann.tex mit nicht gesetztem Schalter.

\newif\ifkorrekturansicht
\korrekturansichtfalse

\input{../tex-inputs/latex-vorspann}


\section[Arthur Schnitzler an Hugo Hofmannsthal, 26. 2. 1927]{L02482 Arthur Schnitzler an Hugo Hofmannsthal, 26. 2. 1927}
\nopagebreak\mylabel{L02482v}
\rehead{ }\normalsize\beginnumbering\briefempfaengerindex{Hofmannsthal, Hugo von@\textsc{Hofmannsthal, Hugo von}!zzzSchnitzler, Arthur@\emph{von Arthur Schnitzler}!1927-02-261@{26. 2. 1927}|(be}
\toendnotes[C]{\smallbreak\pagebreak[2]}
\correspDesc{Versand  durch Arthur Schnitzler am 26. 2. 1927 in Wien
\newline{}Erhalt  durch Hugo von Hofmannsthal im Zeitraum [26. 2. 1927
                  – 2. 3. 1927?] in Wien}\toendnotes[C]{\smallbreak}
\Standort{FDH, Hs-30885,157.}
\physDesc{Postkarte, 540 Zeichen
\newline{}Handschrift: Bleistift, lateinische Kurrent}
\buchAbdrucke{\weitereDrucke{Hugo von Hofmannsthal, Arthur Schnitzler: \emph{Briefwechsel}. Herausgegeben von Therese Nickl und Heinrich Schnitzler. Frankfurt am Main: \emph{S. Fischer} 1964, S. 307.} }\toendnotes[C]{\smallbreak}\pstart{}{\pb}\label{T_L02482-1v}\edtext{\textcolor{gray}{\textbf{A. S.}}}{\lemma{\textnormal{\emph{A. S.}}}\Cendnote{\textnormal{ovaler Absenderkleber}}}\label{T_L02482-1}\pend{}\pstart{}\textcolor{gray}{\textbf{WIEN, XVIII.}}\oindex{XVIII., Währing@\textbf{XVIII., Währing}, \emph{Verwaltungsgebiet}|pw}\pend{}\pstart{}\textcolor{gray}{\textbf{STERNWARTESTR. 71}}\oindex{Wien@\textbf{Wien}!XVIII., Währing@\textbf{XVIII., Währing}!Sternwartestraße 71@\textbf{Sternwartestraße 71}, \emph{Wohngebäude}|pw}\pend{}{\bigskip}\pstart{}Herrn Hugo v Hofmannsthal,\pend{}\pstart{}Rodaun\oindex{Wien@\textbf{Wien}!XXIII., Liesing@\textbf{XXIII., Liesing}!Badgasse@\textbf{Badgasse}, \emph{Straße}|pw}\pend{}\pstart{}bei Wien-Liesing\oindex{Wien@\textbf{Wien}!XXIII., Liesing@\textbf{XXIII., Liesing}!Rodaun@\textbf{Rodaun}, \emph{Region}|pw}\pend{}{\bigskip}\vspace{1em}
\pstart
           \raggedleft{}{\pb}Wien\oindex{Wien@\textbf{Wien}, \emph{Verwaltungsgebiet}|pw}, 26. 2. 927\pend
           \vspace{0.5em}
\pstart
           mein lieber Hugo, ich danke Ihnen für Ihren Gruſs aus Girgenti\oindex{Agrigento@\textbf{Agrigento}, \emph{Örtlichkeit}|pw}.\pend
           
\pstart
           Der treffliche Regisseur \uline{Schulbaur}\pwindex{Schulbaur, Heinz 30.\,12.\,1884 Wien – 3.\,7.\,1964 ebd.@\textsc{Schulbaur, Heinz} (30.\,12.\,1884 Wien – 3.\,7.\,1964 ebd.), \emph{Regisseur}|pw}, früher Volkstheater\oindex{Wien@\textbf{Wien}!VII., Neubau@\textbf{VII., Neubau}!Volkstheater@\textbf{Volkstheater}, \emph{Theater}|pw} wendet sich an mich:
               ich möchte seine Bitte bei Ihnen unterstützen. Er will in der Akademie\oindex{Wien@\textbf{Wien}!I., Innere Stadt@\textbf{I., Innere Stadt}!Hochschule und Akademie für Musik und Darstellende Kunst@\textbf{Hochschule und Akademie für Musik und Darstellende Kunst}, \emph{Universität}|pw} mit seinen Schülern den weißen Fächer\pwindex{Hofmannsthal, Hugo von 1.\,2.\,1874 Wien – 15.\,7.\,1929 Rodaun@\textsc{Hofmannsthal, Hugo von} (1.\,2.\,1874 Wien – 15.\,7.\,1929 Rodaun), \emph{Schriftsteller}!weiße Fächer. Ein Zwischenspiel@\strich\emph{Der weiße Fächer. Ein Zwischenspiel}|pw} aufführen. Sie werden wohl nichts dagegen haben, so wenig ich
               mich gegen dergleichen zu wehren pflege.\pend
           
\pstart
           Auf Wiedersehen nach Ihrer Rückkehr\hspace*{1em}Ich wünsche
               Ihnen weiterhin schöne Sicilianer\oindex{Sizilien@\textbf{Sizilien}, \emph{Land}|pw} Tage. Ich war
                  1904 in \label{K_L02482-1v}\edtext{Taormina\oindex{Taormina@\textbf{Taormina}, \emph{Hauptstadt}|pw}}{\lemma{\textnormal{\emph{Taormina}}}\Cendnote{\textnormal{Vgl. A. S.: \emph{Tagebuch}, 19. 5. 1904.
               }}}\label{K_L02482-1} u \label{K_L02482-2v}\edtext{Syrakus\oindex{Syrakus@\textbf{Syrakus}, \emph{Hauptstadt}|pw}}{\lemma{\textnormal{\emph{Syrakus}}}\Cendnote{\textnormal{Vgl. A. S.: \emph{Tagebuch}, 17. 5. 1904.
               }}}\label{K_L02482-2}.\pend
           \pstart Herzlichst Ihr \spacefill\mbox{Arthur}\pend{}\selectlanguage{ngerman}\endnumbering\briefempfaengerindex{Hofmannsthal, Hugo von@\textsc{Hofmannsthal, Hugo von}!zzzSchnitzler, Arthur@\emph{von Arthur Schnitzler}!1927-02-261@{26. 2. 1927}|)be}\mylabel{L02482h}  \newcommand{\dateiname}{L02482}\newcommand{\titel}{Arthur Schnitzler an Hugo Hofmannsthal, 26. 2. 1927}\newcommand{\editorInnen}{Martin Anton Müller und Gerd-Hermann Susen}%% latex-leseansicht-abspann.tex
%% Abspann für die Leseansicht.
%% Der Schalter \ifkorrekturansicht ist bereits durch den Vorspann gesetzt.

%% latex-abspann.tex
%% Gemeinsamer Abspann für Korrekturansicht und Leseansicht.
%% Setzt den Schalter \ifkorrekturansicht voraus (gesetzt in den
%% einbindenden Dateien latex-korrekturansicht-abspann.tex bzw.
%% latex-leseansicht-abspann.tex).
%% ---------------------------------------------------------------

\normalsize

% Das esempio-Environment wird nur in der Leseansicht benötigt
\ifkorrekturansicht\else
\newenvironment{esempio}[3]%
{
    \vspace{1.5ex}
    \rlap{\underline{#1}}
    \par
    \setlength{\parindent}{0cm}
    \nopagebreak
    \leftskip=#2cm
    \rightskip=#3cm
}
{
    \par
}
\fi

\doendnotes{C}
\bigskip
\vfill

\clearpage

\footnotesize

\ifkorrekturansicht
  \lohead{\textsc{register}}
\fi

% theindex-Environment neu definieren ohne reledmac
\makeatletter
\renewenvironment{theindex}{%
  \ifkorrekturansicht
    \section*{\indexname}%
  \else
    \subsubsection*{Index der erwähnten Entitäten}%
  \fi
  \setlength{\parindent}{0pt}%
  \setlength{\parskip}{0pt plus 0.3pt}%
  \let\item\@idxitem
}{%
  \ifkorrekturansicht\clearpage\fi
}
\makeatother

\IfFileExists{\jobname-pw.ind}{\input{\jobname-pw.ind}}{}

% Quellenangabe nur in der Leseansicht
\ifkorrekturansicht\else
% Fallback-Definitionen, falls die .tex-Datei \titel etc. nicht gesetzt hat
\providecommand{\titel}{}
\providecommand{\editorInnen}{}
\providecommand{\dateiname}{\jobname}

\vspace{3cm}

\vfill

\footnotesize
\textsc{Quelle}: \titel. Herausgegeben von {\editorInnen}. In: \emph{Arthur Schnitzler: Briefwechsel mit Autorinnen und Autoren}.
 Digitale Edition, https://schnitzler-briefe.acdh.oeaw.ac.at/{\dateiname}.html (Stand \today)
\fi

\end{document}


