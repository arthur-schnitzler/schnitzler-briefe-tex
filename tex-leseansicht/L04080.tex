%% latex-leseansicht-vorspann.tex
%% Vorspann für die Leseansicht.
%% Lädt die gemeinsame Datei latex-vorspann.tex mit nicht gesetztem Schalter.

\newif\ifkorrekturansicht
\korrekturansichtfalse

\input{../tex-inputs/latex-vorspann}


\section[Arthur Schnitzler an Gustav Schwarzkopf, 9. 7. 1900]{L04080 Arthur Schnitzler an Gustav Schwarzkopf, 9. 7. 1900}
\nopagebreak\mylabel{L04080v}
\rehead{ }\normalsize\beginnumbering\briefempfaengerindex{Schwarzkopf, Gustav@\textsc{Schwarzkopf, Gustav}!zzzSchnitzler, Arthur@\emph{von Arthur Schnitzler}!1900-07-091@{9. 7. 1900}|(be}
\toendnotes[C]{\smallbreak\pagebreak[2]}
\correspDesc{Versand  durch Arthur Schnitzler am 9. 7. 1900 in Reichenau an der Rax
\newline{}Erhalt  durch Gustav Schwarzkopf im Zeitraum [10. 7. 1900 – 14. 7. 1900?] in Wien}\toendnotes[C]{\smallbreak}
\Standort{CUL, Schnitzler, B 96.}
\physDesc{Brief, 1 Blatt, 4 Seiten, 1398 Zeichen
\newline{}Handschrift: schwarze Tinte, deutsche Kurrent}
\buchAbdrucke{\weitereDrucke{Arthur Schnitzler: \emph{Briefe 1875–1912}. Herausgegeben von Therese Nickl und Heinrich Schnitzler. Frankfurt am Main: \emph{S. Fischer} 1981, S. 386–387.} }\toendnotes[C]{\smallbreak}
\pstart
           \raggedleft{}{\pb}9. 7. 900\pend
           
\pstart
           \raggedleft{}\textsc{Reichenau} bei \textsc{Payerbach}\oindex{Reichenau an der Rax@\textbf{Reichenau an der Rax}, \emph{Verwaltungsgebiet}|pw}\pend
           
\pstart
           \raggedleft{}Curhaus\oindex{Kurhaus Rudolfsbad@\textbf{Kurhaus Rudolfsbad}, \emph{Sanatorium}|pwv}\pend
           \vspace{0.5em}
\pstart
           lieber Guſtav, ſeit \label{K_L04080-1v}\edtext{Donnerſ\textcolor{gray}{t}g}{\lemma{\textnormal{\emph{Donnerstg}}}\Cendnote{\textnormal{Vgl. A. S.: \emph{Wiener Schnitzler}, 5. 7. 1900.}}}\label{K_L04080-1} Abend bin ich hier.
            Sie haben wohl meine \label{K_L04080-2v}\edtext{Karten beko{\geminationm}en}{\lemma{\textnormal{\emph{Karten bekommen}}}\Cendnote{\textnormal{{XXXX ref} XXXX}}}\label{K_L04080-2}. Ein paar Tage hab ich
            in Altauſſee\oindex{Altaussee@\textbf{Altaussee}, \emph{Verwaltungsgebiet}|pw} zugebracht, da{\geminationn} kam eine ſchöne Radtour ins Geſäuſe\oindex{Gesäuse@\textbf{Gesäuse}, \emph{Schlucht}|pw}, wo es ein paar recht angenehme Stunden gab. Hier hab
            ich mich ans Arbeiten gemacht; d\strikeout{i}as \strikeout{ſehr} Scenarium der 5aktigen Komödie\pwindex{Schnitzler, Arthur 15. 5. 1862 Wien – 21. 10. 1931 ebd.@\textsc{Schnitzler, Arthur} (15. 5. 1862 Wien – 21. 10. 1931 ebd.), \emph{Schriftsteller, Mediziner}!Weg ins Freie. Roman@\strich\emph{Der Weg ins Freie. Roman}|pwv} entworfen und die Nilquellen\pwindex{Schnitzler, Arthur 15. 5. 1862 Wien – 21. 10. 1931 ebd.@\textsc{Schnitzler, Arthur} (15. 5. 1862 Wien – 21. 10. 1931 ebd.), \emph{Schriftsteller, Mediziner}!Quellen des Nil@\strich\emph{Die Quellen des Nil}|pw} neu begonnen,
            deren erſte mis{\pb}glückte Faſſung Sie
            kennen. Ich folge ganz Ihren Rathschlägen in der »Umarbeitung«. Meine Novelle\pwindex{Schnitzler, Arthur 15. 5. 1862 Wien – 21. 10. 1931 ebd.@\textsc{Schnitzler, Arthur} (15. 5. 1862 Wien – 21. 10. 1931 ebd.), \emph{Schriftsteller, Mediziner}!Frau Bertha Garlan. Roman@\strich\emph{Frau Bertha Garlan. Roman}|pwv} haben Sie wohl durch die Grünwald\pwindex{Grünwald, Ida 28.\,6.\,1873 Wien – Mai 1908 Alland@\textsc{Grünwald, Ida} (28.\,6.\,1873 Wien – Mai 1908 Alland), \emph{Stenotypistin}|pw} erhalten. Bitte behalten Sie ſie \introOben{}(die Novelle\pwindex{Schnitzler, Arthur 15. 5. 1862 Wien – 21. 10. 1931 ebd.@\textsc{Schnitzler, Arthur} (15. 5. 1862 Wien – 21. 10. 1931 ebd.), \emph{Schriftsteller, Mediziner}!Frau Bertha Garlan. Roman@\strich\emph{Frau Bertha Garlan. Roman}|pwv}!!)\introOben{} vorläufig ruhig in Ihrem Hauſe.
            Falls Sie mir was darüber ſagen wollen, \uline{mündlich}. Am
            liebſten natürlich hier, in Reichenau\oindex{Reichenau an der Rax@\textbf{Reichenau an der Rax}, \emph{Verwaltungsgebiet}|pw}. Entſchließen
            Sie ſich doch ein bischen herzuko{\geminationm}en. Im Curhaus\oindex{Kurhaus Rudolfsbad@\textbf{Kurhaus Rudolfsbad}, \emph{Sanatorium}|pwv} lebt{ }ſichs angenehm und nicht theuer.
            Die Reiſe iſt nicht weit wie {\pb}Sie wiſſen.
            Sie finden außer mir meine Mama\pwindex{Schnitzler, Louise 8.\,7.\,1840 Kőszeg – 9.\,9.\,1911 Wien@\textsc{Schnitzler, Louise} (8.\,7.\,1840 Kőszeg – 9.\,9.\,1911 Wien)|pwv}, meine Schweſter\pwindex{Hajek, Gisela 20.\,12.\,1867 Wien – 3.\,2.\,1953 Cambridge@\textsc{Hajek, Gisela} (20.\,12.\,1867 Wien – 3.\,2.\,1953 Cambridge)|pwv}. In
               Edlach\oindex{Edlach@\textbf{Edlach}|pw}{ }Schwägerin\pwindex{Schnitzler, Helene 16.\,7.\,1871 Budapest – September 1941 Atlantischer Ozean@\textsc{Schnitzler, Helene} (16.\,7.\,1871 Budapest – September 1941 Atlantischer Ozean)|pwv} und Neffen\pwindex{Schnitzler, Hans 11.\,7.\,1895 Wien – 26.\,3.\,1967 Chicago@\textsc{Schnitzler, Hans} (11.\,7.\,1895 Wien – 26.\,3.\,1967 Chicago), \emph{Chirurg}|pwv}. – Vormittags
            plauder  ich gewöhnlich mit einer nicht hübſchen (das zu Ihrer Beruhigung) aber
            ausnehmend geſcheidten jungen Dame\pwindex{Schnitzler, Olga 17.\,1.\,1882 Wien – 13.\,1.\,1970 Lugano@\textsc{Schnitzler, Olga} (17.\,1.\,1882 Wien – 13.\,1.\,1970 Lugano), \emph{Schauspielerin, Sängerin}|pwv},
               derſelben, die ich dem \label{K_L04080-3v}\edtext{Brahm\pwindex{Brahm, Otto 5.\,2.\,1856 Hamburg – 28.\,11.\,1912 Berlin@\textsc{Brahm, Otto} (5.\,2.\,1856 Hamburg – 28.\,11.\,1912 Berlin), \emph{Theaterleiter, Regisseur}|pw} empfohlen}{\lemma{\textnormal{\emph{Brahm empfohlen}}}\Cendnote{\textnormal{Brahm\pwindex{Brahm, Otto 5.\,2.\,1856 Hamburg – 28.\,11.\,1912 Berlin@\textsc{Brahm, Otto} (5.\,2.\,1856 Hamburg – 28.\,11.\,1912 Berlin), \emph{Theaterleiter, Regisseur}|pwk} hatte sein Urteil über die zukünftige Ehefrau von Schnitzler am 12. 6. 1900 schriftlich
                  abgegeben: 
                  »wenn ich ihr Vormund wäre, würde ich doch sagen: das Kind\pwindex{Schnitzler, Olga 17.\,1.\,1882 Wien – 13.\,1.\,1970 Lugano@\textsc{Schnitzler, Olga} (17.\,1.\,1882 Wien – 13.\,1.\,1970 Lugano), \emph{Schauspielerin, Sängerin}|pwv} soll mir nicht zur Bühne.« \emph{Der Briefwechsel Arthur Schnitzler – Otto Brahm}.
                     Vollständige Ausgabe. Herausgegeben, eingeleitet und erläutert von Oskar
                     Seidlin. Tübingen: \emph{Niemeyer}{ }1975, S. 87.
               }}}\label{K_L04080-3}. (Halten Sie das
            »junge Dame« nicht für verdächtig!) – Ich bleibe hier wahrſcheinlich bis gegen den 20.,
               da{\geminationn} ko{\geminationm} ich jedenfalls auf
            ein Reihe von Tagen {\pb}nach Wien\oindex{Wien@\textbf{Wien}, \emph{Verwaltungsgebiet}|pw}. Aber das darf Sie nicht umſtimmen, wenn
            Sie ſchon nahe daran waren, meiner Auffordg von Seite 2 zu folgen. Jedenfalls hoff
            ich ſehr bald von Ihnen zu hören.\pend
           
\pstart
           Von Herzen Ihr{\\[\baselineskip]}\spacefill\mbox{Arthur Sch}\pend
           \leftskip=0em{}\selectlanguage{ngerman}\endnumbering\briefempfaengerindex{Schwarzkopf, Gustav@\textsc{Schwarzkopf, Gustav}!zzzSchnitzler, Arthur@\emph{von Arthur Schnitzler}!1900-07-091@{9. 7. 1900}|)be}\mylabel{L04080h}
\begin{anhang}
\end{anhang}\newcommand{\dateiname}{L04080}\newcommand{\titel}{Arthur Schnitzler an Gustav Schwarzkopf, 9. 7. 1900}\newcommand{\editorInnen}{Herausgegeben von Jahnke, SelmaMüller, Martin Anton}%% latex-leseansicht-abspann.tex
%% Abspann für die Leseansicht.
%% Der Schalter \ifkorrekturansicht ist bereits durch den Vorspann gesetzt.

%% latex-abspann.tex
%% Gemeinsamer Abspann für Korrekturansicht und Leseansicht.
%% Setzt den Schalter \ifkorrekturansicht voraus (gesetzt in den
%% einbindenden Dateien latex-korrekturansicht-abspann.tex bzw.
%% latex-leseansicht-abspann.tex).
%% ---------------------------------------------------------------

\normalsize

% Das esempio-Environment wird nur in der Leseansicht benötigt
\ifkorrekturansicht\else
\newenvironment{esempio}[3]%
{
    \vspace{1.5ex}
    \rlap{\underline{#1}}
    \par
    \setlength{\parindent}{0cm}
    \nopagebreak
    \leftskip=#2cm
    \rightskip=#3cm
}
{
    \par
}
\fi

\doendnotes{C}
\bigskip
\vfill

\clearpage

\footnotesize

\ifkorrekturansicht
  \lohead{\textsc{register}}
\fi

% theindex-Environment neu definieren ohne reledmac
\makeatletter
\renewenvironment{theindex}{%
  \ifkorrekturansicht
    \section*{\indexname}%
  \else
    \subsubsection*{Index der erwähnten Entitäten}%
  \fi
  \setlength{\parindent}{0pt}%
  \setlength{\parskip}{0pt plus 0.3pt}%
  \let\item\@idxitem
}{%
  \ifkorrekturansicht\clearpage\fi
}
\makeatother

\IfFileExists{\jobname-pw.ind}{\input{\jobname-pw.ind}}{}

% Quellenangabe nur in der Leseansicht
\ifkorrekturansicht\else
% Fallback-Definitionen, falls die .tex-Datei \titel etc. nicht gesetzt hat
\providecommand{\titel}{}
\providecommand{\editorInnen}{}
\providecommand{\dateiname}{\jobname}

\vspace{3cm}

\vfill

\footnotesize
\textsc{Quelle}: \titel. Herausgegeben von {\editorInnen}. In: \emph{Arthur Schnitzler: Briefwechsel mit Autorinnen und Autoren}.
 Digitale Edition, https://schnitzler-briefe.acdh.oeaw.ac.at/{\dateiname}.html (Stand \today)
\fi

\end{document}


