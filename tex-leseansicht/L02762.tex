%% latex-leseansicht-vorspann.tex
%% Vorspann für die Leseansicht.
%% Lädt die gemeinsame Datei latex-vorspann.tex mit nicht gesetztem Schalter.

\newif\ifkorrekturansicht
\korrekturansichtfalse

\input{../tex-inputs/latex-vorspann}


\section[ Paul Goldmann an Arthur Schnitzler, 11. 1. {[}1896{]}]{L02762 Paul Goldmann an Arthur Schnitzler,  11. 1. [1896]}
\nopagebreak\mylabel{L02762v}
\rehead{ }\normalsize\beginnumbering\briefempfaengerindex{Schnitzler, Arthur@\textsc{Schnitzler, Arthur}!zzzGoldmann, Paul@\emph{von Paul Goldmann}!1896-01-111@{11. 1. [1896]}|(be}
\toendnotes[C]{\smallbreak\pagebreak[2]}
\correspDesc{Versand  durch Paul Goldmann am 11. 1. [1896] in Paris
\newline{}Erhalt  durch Arthur Schnitzler im Zeitraum [12. 1. 1896
                  – 16. 1. 1896?] in Frankfurt am Main}\toendnotes[C]{\smallbreak}
\Standort{DLA, A:Schnitzler, HS.NZ85.1.3166.}
\physDesc{Brief, 4 Blätter, 15 Seiten, 5924 Zeichen
\newline{}Handschrift: blaue Tinte, deutsche Kurrent
\newline{}Beilagen: 1) Vordruck mit handschriftlicher
                                 Nachricht: 1 Blatt, 1 Seite  2) handschriftlicher Brief: 1 Blatt, 1 Seite
\newline{}Schnitzler: 1) mit Bleistift das Jahr »96« vermerkt  2) mit rotem Buntstift zwei Unterstreichungen}\toendnotes[C]{\smallbreak}
\pstart
           {\pb}\textcolor{gray}{\textbf{\textbf{Frankfurter Zeitung\orgindex{Frankfurter Zeitung@Frankfurter Zeitung|pw}}}}\pend
           
\pstart
           \textcolor{gray}{\textbf{(\begin{otherlanguage}{french}Gazette de Francfort\end{otherlanguage}\orgindex{Frankfurter Zeitung@Frankfurter Zeitung|pw}).}}\pend
           
\pstart
           \textcolor{gray}{\textbf{\textbf{\begin{otherlanguage}{french}Fondateur M.\end{otherlanguage}{ }L. Sonnemann\pwindex{Sonnemann, Leopold 29.\,10.\,1831 Höchberg – 30.\,10.\,1909 Frankfurt am Main@\textsc{Sonnemann, Leopold} (29.\,10.\,1831 Höchberg – 30.\,10.\,1909 Frankfurt am Main), \emph{Journalist, Herausgeber}|pw}.}}}\pend
           
\pstart
           \begin{otherlanguage}{french}\textcolor{gray}{\textbf{Journal\pwindex{Frankfurter Zeitung@\emph{Frankfurter Zeitung}|pwv} politique,
                        financier,}}\end{otherlanguage}\pend
           
\pstart
           \begin{otherlanguage}{french}\textcolor{gray}{\textbf{commercial et littéraire.}}\end{otherlanguage}\pend
           
\pstart
           \begin{otherlanguage}{french}\textcolor{gray}{\textbf{\textbf{Paraissant trois fois par jour.}}}\end{otherlanguage}\pend
           
\pstart
           \begin{otherlanguage}{french}\textcolor{gray}{\textbf{\textbf{Bureau à Paris\oindex{Paris@\textbf{Paris}, \emph{Hauptstadt}|pw}:}}}\end{otherlanguage}\hfill \textsc{Paris\oindex{Paris@\textbf{Paris}, \emph{Hauptstadt}|pw}}, 11. Januar.\pend
           
\pstart
           \begin{otherlanguage}{french}\textcolor{gray}{\textbf{\textbf{24. Rue Feydeau\oindex{rue Feydeau@\textbf{rue Feydeau}, \emph{Straße}|pw}.}}}\end{otherlanguage}\pend
           
\pstart\center{}Mein lieber Freund,\pend\vspace{0.5em}
\pstart
           Heut geht das Opernglas an Dich ab. Ich habe Dich
               lange warten laſſen müſſen. Erſtens hatte ich viel zu thun, zweitens war es keine
               leichte Geſchichte. Ich bin bei allen möglichen Optikern herumgelaufen. Die große
               Schwierigkeit war der Ausſchluß von Perlmutter. Alles, was hier hübſch und Pariſ\oindex{Paris@\textbf{Paris}, \emph{Hauptstadt}|pw}eriſch ausſieht, wird in Perlmutter aller
               Arten und Farben gemacht. Dann hat man noch ganz{ }ſchwarze {\pb}Operngläſer, endlich \label{K_L02762-1v}\edtext{Schildpatt}{\lemma{\textnormal{\emph{Schildpatt}}}\Cendnote{\textnormal{Material aus Schuppen von Meeresschildkröten, das auch als Musterbezeichnung
                  gebräuchlich ist.}}}\label{K_L02762-1}. Ich habe mich zu letzterem entſchloſſen, damit wenigſtens
               etwas Farbe daran iſt. Weitere Schwierigkeit: Die wirklich guten Gläſer finden{ }ſich
               nur bei den großen Inſtrumenten. Je kleiner die Gläſer, umſo weniger gut{ }ſieht man.
               Je kleiner die Gläſer, umſo zierlicher freilich und umſo reicher ornamentirt iſt die
               Form des Ganzen. Mich{ }ſtrict an Deine Weiſungen haltend, habe ich das den Gläſern {\pb}nach beſte Opernglas genommen, das ich in \strikeout{de} der betreffenden Preislage finden konnte. Es enthält
               zwölf Gläſer und{ }ſtammt von einem in \textsc{Paris\oindex{Paris@\textbf{Paris}, \emph{Hauptstadt}|pw}} beſtbekannten Optiker. \strikeout{Um eine gewiſſe} Man{ }ſieht gut dadurch, freilich mußte ich deshalb ein etwas größeres Format wählen. Es
               iſt zur Herſtellung \textsc{Aluminium} verwendet, was jetzt hier{ }ſehr in der Mode iſt. Ich kann das zwar abſolut nicht leiden, aber das Opernglas hat
               dadurch den Vortheil {\pb}größter Leichtigkeit. Auch{ }ſonſt gefällt mir meine Wahl äußerlich gar nicht; \strikeout{abe}
               aber Du haſt mir zu enge Grenzen geſteckt, und mein Geſchmack konnte{ }ſich darin nicht
               frei bewegen. Jedenfalls habe ich mit dem Optiker den \uline{Umtauſch} ausgemacht. Gefällts Dir alſo nicht,{ }ſo{ }ſchickſt Du mirs zurück und
               gibſt mir nähere Weiſungen. Koſten{ }ſollte es 60 \textsc{Frcs}, ich
               habe aber einige Tage {\pb}manövrirt und{ }ſchließlich 50 \textsc{Frcs} herausgehandelt. Freilich dürfte{ }ſich der Ehrenmann wohl noch 5 \textsc{Frcs} für Verpackung, Porto
                  \textsc{etc.} herausſchwindeln. Soll ich Dir den Reſt{ }ſchicken
               oder{ }ſoll ich noch etwas dafür hier kaufen?\pend
           
\pstart
           Über die verſchwundenen Goldſtücke hat die hieſige Poſt\orgindex{Französische Post@Französische Post|pw} auf meine Beſchwerde eine Unterſuchung eingeleitet, wie beifolgendes
                  {\pb}Papier beſtätigt, das Dir vielleicht als Ausweis
               gegenüber der öſterreichiſchen Poſt\orgindex{Österreichische Post@Österreichische Post|pw} dienen
               kann.\pend
           
\pstart
           Auch{ }ſende ich Dir einen Brief von \textsc{Thorel\pwindex{Thorel, Jean 11.\,9.\,1859 Éragny – 20.\,8.\,1916 Enghien-les-Bains@\textsc{Thorel, Jean} (11.\,9.\,1859 Éragny – 20.\,8.\,1916 Enghien-les-Bains), \emph{Übersetzer, Dramatiker}|pw}}, der ein \label{K_L02762-2v}\edtext{Stück\pwindex{Thorel, Jean 11.\,9.\,1859 Éragny – 20.\,8.\,1916 Enghien-les-Bains@\textsc{Thorel, Jean} (11.\,9.\,1859 Éragny – 20.\,8.\,1916 Enghien-les-Bains), \emph{Übersetzer, Dramatiker}!Deux sœurs, pièce en 3 actes@\strich\emph{Deux sœurs, pièce en 3 actes}|pwv}}{\lemma{\textnormal{\emph{Stück}}}\Cendnote{\textnormal{\emph{Deux sœurs, pièce en 3 actes}\pwindex{Thorel, Jean 11.\,9.\,1859 Éragny – 20.\,8.\,1916 Enghien-les-Bains@\textsc{Thorel, Jean} (11.\,9.\,1859 Éragny – 20.\,8.\,1916 Enghien-les-Bains), \emph{Übersetzer, Dramatiker}!Deux sœurs, pièce en 3 actes@\strich\emph{Deux sœurs, pièce en 3 actes}|pwk} wurde am 23. 4. 1896 im Paris\oindex{Paris@\textbf{Paris}, \emph{Hauptstadt}|pwk}er \emph{Odéon}\orgindex{Odéon@Odéon|pwk} uraufgeführt.}}}\label{K_L02762-2} im
                  »\textsc{Odéon\orgindex{Odéon@Odéon|pw}}« aufgeführt bekommen{ }ſoll. Man zieht ihn furchtbar damit herum, und das macht
               ihm den Kopf verrückt. Laſſen wir ihm noch etwas \label{K_L02762-3v}\edtext{Zeit}{\lemma{\textnormal{\emph{Zeit}}}\Cendnote{\textnormal{mit der Übersetzung\pwindex{Schnitzler, Arthur 15.\,5.\,1862 Wien – 21.\,10.\,1931 ebd.@\textsc{Schnitzler, Arthur} (15.\,5.\,1862 Wien – 21.\,10.\,1931 ebd.), \emph{Schriftsteller, Mediziner}!Amourette. Pièce en trois actes. Adaptée de Arthur Schnitzler@\strich\emph{Amourette. Pièce en trois actes. Adaptée de Arthur Schnitzler}|pwkv} von \emph{Liebelei}\pwindex{Schnitzler, Arthur 15.\,5.\,1862 Wien – 21.\,10.\,1931 ebd.@\textsc{Schnitzler, Arthur} (15.\,5.\,1862 Wien – 21.\,10.\,1931 ebd.), \emph{Schriftsteller, Mediziner}!Liebelei. Schauspiel in drei Akten@\strich\emph{Liebelei. Schauspiel in drei Akten}|pwk}}}}\label{K_L02762-3}.\pend
           
\pstart
           Den guten \label{K_L02762-4v}\edtext{Mann\pwindex{Riaz, Henri de 1871 Lyon – 1951 Lausanne@\textsc{Riaz, Henri de} (1871 Lyon – 1951 Lausanne), \emph{Dichter}|pwv}{ }{\pb}aus \textsc{Lyon\oindex{Lyon@\textbf{Lyon}|pw}}}{\lemma{\textnormal{\emph{Mann aus Lyon}}}\Cendnote{\textnormal{Siehe XXXX Auszeichnungsfehler: Dokument L02758 nicht gefunden.
               }}}\label{K_L02762-4} beſcheide aufſchiebend. Viel Vertrauen flößt er mir nicht ein. Die
               Zeitſchriften, die er \strikeout{vor} nennt,{ }ſind unbedeutend,
               die \strikeout{Refer} Beziehungen, die er angibt, noch mehr. Für
               das »\textsc{Œuvre\orgindex{Théâtre de l’Œuvre@Théâtre de l’Œuvre|pw}}« oder das »\textsc{Théâtre Libre\orgindex{Théâtre Libre@Théâtre Libre|pw}}« brauchen wir ihn nicht. Mit denen{ }ſtehe ich allein in Verbindung. Auch{ }ſpielt
               man dort{ }ſo erbärmlich, daß ich Dich nicht gern dort aufgeführt{ }ſehen {\pb}möchte. Endlich{ }ſoll Dein Stück\pwindex{Schnitzler, Arthur 15.\,5.\,1862 Wien – 21.\,10.\,1931 ebd.@\textsc{Schnitzler, Arthur} (15.\,5.\,1862 Wien – 21.\,10.\,1931 ebd.), \emph{Schriftsteller, Mediziner}!Liebelei. Schauspiel in drei Akten@\strich\emph{Liebelei. Schauspiel in drei Akten}|pwv} in \textsc{\uline{Paris\oindex{Paris@\textbf{Paris}, \emph{Hauptstadt}|pw}}} überſetzt werden. Was aus der Provinz, aus \textsc{Lyon\oindex{Lyon@\textbf{Lyon}|pw}} kommt, darüber rümpfen{ }ſie in \textsc{Paris\oindex{Paris@\textbf{Paris}, \emph{Hauptstadt}|pw}} bereits die Naſe. Nach einem großen Erfolge in Berlin\oindex{Berlin@\textbf{Berlin}, \emph{Hauptstadt}|pw} – den ich \strikeout{\textcolor{gray}{×}} vorausſehe – werden{ }ſich Dir ganz andere Leute anbieten; vorher darfſt Du wohl
               kein Engagement eingehen.\pend
           
\pstart
           {\pb}Vielen Dank noch für Deine Einladung zum
               Zuſammentreffen in \label{K_L02762-5v}\edtext{\textsc{Frankfurt\oindex{Frankfurt am Main@\textbf{Frankfurt am Main}, \emph{Hauptstadt}|pw}}}{\lemma{\textnormal{\emph{Frankfurt}}}\Cendnote{\textnormal{Schnitzler hielt sich zwischen 10. 1. 1896 und 13. 1. 1896 in Frankfurt am Main\oindex{Frankfurt am Main@\textbf{Frankfurt am Main}, \emph{Hauptstadt}|pwk} auf.}}}\label{K_L02762-5}! Das wäre{ }ſchön
               geweſen. Aber die Idee war phantaſtiſch. Im Januar von
               hier fort! Ich glaube, ich wäre entlaſſen worden. Und kein Geld zur Reiſe! Nur
               Schulden! Nie im Leben bin ich dem Bankerott{ }ſo nahe geweſen. Aber es war lieb, daß
               Du an mich gedacht haſt. Wann {\pb}werden wir uns
                  \label{K_L02762-6v}\edtext{wiederſehen}{\lemma{\textnormal{\emph{wiedersehen}}}\Cendnote{\textnormal{Sie sahen sich am 5. 8. 1896
                  in Kopenhagen\oindex{Kopenhagen@\textbf{Kopenhagen}, \emph{Hauptstadt}|pwk} wieder.}}}\label{K_L02762-6}? Gott weiß! Ich
               glaube, ich gehe nicht mehr aus \textsc{Paris\oindex{Paris@\textbf{Paris}, \emph{Hauptstadt}|pw}} heraus. Hier bin ich vergraben, die Welt draußen aber thut mir \strikeout{weh\textcolor{gray}{e}} weh. Neugierig bin ich auf das Ergebniß der \label{K_L02762-7v}\edtext{erſten Aufführung\pwindex{Schnitzler, Arthur 15.\,5.\,1862 Wien – 21.\,10.\,1931 ebd.@\textsc{Schnitzler, Arthur} (15.\,5.\,1862 Wien – 21.\,10.\,1931 ebd.), \emph{Schriftsteller, Mediziner}!Liebelei. Schauspiel in drei Akten@\strich\emph{Liebelei. Schauspiel in drei Akten}|pwv} in Deutſchland\oindex{Deutschland@\textbf{Deutschland}|pw}}{\lemma{\textnormal{\emph{ersten … Deutschland}}}\Cendnote{\textnormal{\emph{Liebelei}\pwindex{Schnitzler, Arthur 15.\,5.\,1862 Wien – 21.\,10.\,1931 ebd.@\textsc{Schnitzler, Arthur} (15.\,5.\,1862 Wien – 21.\,10.\,1931 ebd.), \emph{Schriftsteller, Mediziner}!Liebelei. Schauspiel in drei Akten@\strich\emph{Liebelei. Schauspiel in drei Akten}|pwk}-Premiere am 4. 2. 1896 im Deutschen Theater Berlin\oindex{Deutsches Theater Berlin@\textbf{Deutsches Theater Berlin}, \emph{Theater}|pwk}}}}\label{K_L02762-7} und – auf meinen Onkel\pwindex{Mamroth, Fedor 21.\,2.\,1851 Breslau – 25.\,6.\,1907 Frankfurt am Main@\textsc{Mamroth, Fedor} (21.\,2.\,1851 Breslau – 25.\,6.\,1907 Frankfurt am Main), \emph{Journalist, Kritiker}|pwv}. Ich habe ihm dieſer Tage geſchrieben, weil ich \strikeout{furch} fürchte, daß er Dir wehthut aus Haß gegen \textsc{\strikeout{Speid}}{ }\label{K_L02762-8v}\edtext{\textsc{Speidel\pwindex{Speidel, Ludwig 11.\,4.\,1830 Ulm – 3.\,2.\,1906 Wien@\textsc{Speidel, Ludwig} (11.\,4.\,1830 Ulm – 3.\,2.\,1906 Wien), \emph{Journalist, Kritiker}|pw}}}{\lemma{\textnormal{\emph{Speidel}}}\Cendnote{\textnormal{Ludwig Speidel\pwindex{Speidel, Ludwig 11.\,4.\,1830 Ulm – 3.\,2.\,1906 Wien@\textsc{Speidel, Ludwig} (11.\,4.\,1830 Ulm – 3.\,2.\,1906 Wien), \emph{Journalist, Kritiker}|pwk} hatte sich zuvor sehr
                  positiv zu \emph{Liebelei}\pwindex{Schnitzler, Arthur 15.\,5.\,1862 Wien – 21.\,10.\,1931 ebd.@\textsc{Schnitzler, Arthur} (15.\,5.\,1862 Wien – 21.\,10.\,1931 ebd.), \emph{Schriftsteller, Mediziner}!Liebelei. Schauspiel in drei Akten@\strich\emph{Liebelei. Schauspiel in drei Akten}|pwk} geäußert. Siehe
                        [Ludwig Speidel\pwindex{Speidel, Ludwig 11.\,4.\,1830 Ulm – 3.\,2.\,1906 Wien@\textsc{Speidel, Ludwig} (11.\,4.\,1830 Ulm – 3.\,2.\,1906 Wien), \emph{Journalist, Kritiker}|pwk}]: \emph{Theater- und Kunstnachrichten.
                        [Burgtheater]}\pwindex{Theater- und Kunstnachrichten. [Burgtheater] [Liebelei, Rechte der Seele]@\emph{Theater- und Kunstnachrichten. [Burgtheater] [Liebelei, Rechte der Seele]}|pwk}. In: \emph{Neue Freie
                        Presse}\pwindex{Neue Freie Presse@\emph{Neue Freie Presse}|pwk}, Nr. 11.181, 10. 10. 1895,
                     S. 7 und L. Sp.\pwindex{Speidel, Ludwig 11.\,4.\,1830 Ulm – 3.\,2.\,1906 Wien@\textsc{Speidel, Ludwig} (11.\,4.\,1830 Ulm – 3.\,2.\,1906 Wien), \emph{Journalist, Kritiker}|pwkv} [ = Ludwig Speidel\pwindex{Speidel, Ludwig 11.\,4.\,1830 Ulm – 3.\,2.\,1906 Wien@\textsc{Speidel, Ludwig} (11.\,4.\,1830 Ulm – 3.\,2.\,1906 Wien), \emph{Journalist, Kritiker}|pwk}]: \emph{Burgtheater. (»Liebelei«, Schauspiel in drei Aufzügen von
                        Arthur Schnitzler. – »Rechte der Seele«, Schauspiel in einem Act von
                        Giuseppe Giacosa, deutsch von Otto Eisenschitz)}\pwindex{Speidel, Ludwig 11.\,4.\,1830 Ulm – 3.\,2.\,1906 Wien@\textsc{Speidel, Ludwig} (11.\,4.\,1830 Ulm – 3.\,2.\,1906 Wien), \emph{Journalist, Kritiker}!Burgtheater. (»Liebelei«, Schauspiel in drei Aufzügen von Arthur Schnitzler. – »Rechte der Seele«, Schauspiel in einem Act von Giuseppe Giacosa, deutsch von Otto Eisenschitz.)@\strich\emph{Burgtheater. (»Liebelei«, Schauspiel in drei Aufzügen von Arthur Schnitzler. – »Rechte der Seele«, Schauspiel in einem Act von Giuseppe Giacosa, deutsch von Otto Eisenschitz.)}|pwk}. In: \emph{Neue Freie Presse}\pwindex{Neue Freie Presse@\emph{Neue Freie Presse}|pwk}, Nr. 11.184, 13. 10. 1895, Morgenblatt, S. 1–3.}}}\label{K_L02762-8}. Im Grunde {\pb}aber iſt er doch ein hochanſtändiger und
               kunſtliebender Mann – und darauf hoffe ich.\pend
           
\pstart
           Ich habe Dir für{ }ſo viele liebe Briefe zu danken. Dein letzter war melancholiſch.
               Dein Talent{ }ſoll nur Deine Jugend geweſen{ }ſein. Oh Du Kind! Wenn irgend ein Talent zu
               reifen beſtimmt iſt,{ }ſo iſt es Deines. Es iſt kein Schwindel und kein Dunſt darin. Es
               beruht auf klarer {\pb}und \strikeout{ern} ernſter Anſchauung des Lebens. \strikeout{Das} Das
               kann nicht altern. Im Gegentheil. Da{ }ſich Einem das Leben immer größer und
               vielgeſtaltiger aufthut, je älter man wird – was wird Dein Talent erſt daraus ziehen,
                  \strikeout{wenn} nachdem es aus dem Bischen Jugend und Liebe{ }ſchon{ }ſo viel gezogen hat! Oder wirſt Du vielleicht morgen plötzlich {\pb}aufhören, ein Poet zu{ }ſein? Glaubſt Du, das verliert{ }ſich mit den Jahren? Oh Du Kind! {\dotsfive}\pend
           
\pstart
           Von meinem Leben will ich Dir nicht{ }ſprechen. Ich{ }ſchäme mich. Es iſt zu{ }ſehr
               dieſelbe Geſchichte. Das Leben, unermündlich mir \strikeout{ne}
               neue Glücks-Möglichkeiten in die Hand zu{ }ſpielen, und {\pb}ich unermüdlich,{ }ſie mir{ }ſtets auf dieſelbe Weiſe zu
               verderben: durch Schwäche, durch mangelnde Mannhaftigkeit \textsc{etc}. Wenn man 31 Jahre geworden iſt,{ }ſo ändert man{ }ſein Leben nicht mehr.
               Und wenn es einmal in eine falſche Richtung eingelenkt iſt,{ }ſo geht es unaufhaltſam
               in dieſer Richtung weiter. Verfahren! Unglücklich{ }ſein, das kann man {\pb}ertragen. Aber wenn man{ }ſtets durch eigene Schuld
               unglücklich iſt, – das erträgt man kaum.\pend
           
\pstart
           Grüß’ Dich Gott, mein lieber Freund! Schreib’ mir bald! Wie{ }ſtehts mit dem neuen Stücke\pwindex{Schnitzler, Arthur 15.\,5.\,1862 Wien – 21.\,10.\,1931 ebd.@\textsc{Schnitzler, Arthur} (15.\,5.\,1862 Wien – 21.\,10.\,1931 ebd.), \emph{Schriftsteller, Mediziner}!Freiwild. Schauspiel in 3 Akten@\strich\emph{Freiwild. Schauspiel in 3 Akten}|pwv}? Rückt die \label{K_L02762-9v}\edtext{zweite Niederſchrift}{\lemma{\textnormal{\emph{zweite Niederschrift}}}\Cendnote{\textnormal{Schnitzler, der mit \emph{Freiwild}\pwindex{Schnitzler, Arthur 15.\,5.\,1862 Wien – 21.\,10.\,1931 ebd.@\textsc{Schnitzler, Arthur} (15.\,5.\,1862 Wien – 21.\,10.\,1931 ebd.), \emph{Schriftsteller, Mediziner}!Freiwild. Schauspiel in 3 Akten@\strich\emph{Freiwild. Schauspiel in 3 Akten}|pwk} äußerst unzufrieden war, hatte das Stück\pwindex{Schnitzler, Arthur 15.\,5.\,1862 Wien – 21.\,10.\,1931 ebd.@\textsc{Schnitzler, Arthur} (15.\,5.\,1862 Wien – 21.\,10.\,1931 ebd.), \emph{Schriftsteller, Mediziner}!Freiwild. Schauspiel in 3 Akten@\strich\emph{Freiwild. Schauspiel in 3 Akten}|pwkv} am 31. 12. 1895
                  neu begonnen.}}}\label{K_L02762-9} vorwärts?\pend
           
\pstart
           Viele Grüße an \textsc{Richard\pwindex{Beer-Hofmann, Richard 11.\,7.\,1866 Wien – 26.\,9.\,1945 New York City@\textsc{Beer-Hofmann, Richard} (11.\,7.\,1866 Wien – 26.\,9.\,1945 New York City), \emph{Schriftsteller}|pw}}!\pend
           
\pstart
           In Treue {\\[\baselineskip]}Dein {\\[\baselineskip]}\spacefill\mbox{Paul Goldmann.}\pend
           \leftskip=0em{}\selectlanguage{ngerman}\vspace{1em}{\vspace{1\baselineskip}}
\pstart
           {\pb}\textcolor{gray}{\textbf{\textsc{\textbf{N\textsuperscript{o} 1293 G⋅A⋅C.}}}}\pend
           
\pstart
           \textcolor{gray}{\textbf{\textbf{\begin{otherlanguage}{french}Décembre 91\end{otherlanguage} – Coq. 55.}}}\pend
           
\pstart
           \begin{otherlanguage}{french}\textcolor{gray}{\textbf{\emph{Ministère} du Commerce, de l’Industrie et des
                           Colonies\orgindex{Ministère du Commerce, de l’Industrie et des Colonies@Ministère du Commerce, de l’Industrie et des Colonies|pwv}.}}\end{otherlanguage}\pend
           
\pstart
           \begin{otherlanguage}{french}\textcolor{gray}{\textbf{\emph{Direction Générale} des Postes et des
                           Télégraphes.\orgindex{Ministère du Commerce, de l’Industrie et des Colonies@Ministère du Commerce, de l’Industrie et des Colonies|pwv}}}\end{otherlanguage}\pend
           
\pstart
           \begin{otherlanguage}{french}\textcolor{gray}{\textbf{\emph{Exploitation Postale.}}}\end{otherlanguage}\pend
           
\pstart
           \begin{otherlanguage}{french}\textcolor{gray}{\textbf{\textbf{Bureau des Réclamations.}}}\end{otherlanguage}\pend
           
\pstart
           \begin{otherlanguage}{french}\textcolor{gray}{\textbf{\emph{2\textsuperscript{e} Sectione.}}}\end{otherlanguage}\pend
           
\pstart
           \textcolor{gray}{\textbf{–6–}}\pend
           
\pstart
           \textcolor{gray}{\textbf{\emph{N\textsuperscript{o}}}} sp. 344\textcolor{gray}{\textbf{\emph{.}}}\pend
           
\pstart
           \begin{otherlanguage}{french}\textcolor{gray}{\textbf{\textbf{Avis d’enquête.}}}\end{otherlanguage}\pend
           
\pstart
           \centering{}\textcolor{gray}{\textbf{\textbf{\begin{otherlanguage}{french}République Française\oindex{Frankreich@\textbf{Frankreich}|pw}.\end{otherlanguage}}}}\pend
           
\pstart
           \raggedleft{}\begin{otherlanguage}{french}\textcolor{gray}{\textbf{\emph{Paris\oindex{Paris@\textbf{Paris}, \emph{Hauptstadt}|pw}, le}}}{ }23 décembre \textcolor{gray}{\textbf{\emph{189}}}5\textcolor{gray}{\textbf{\emph{.}}}\end{otherlanguage}\pend
           
\pstart{}\begin{otherlanguage}{french}\textcolor{gray}{\textbf{\emph{M}}}onsieur,\end{otherlanguage}\pend\vspace{0.5em}
\pstart
           \label{K_L02762-10v}\edtext{\begin{otherlanguage}{french}\textcolor{gray}{\textbf{\emph{J’ai reçu la réclamation que vous m’avez adressée le}}}{ }21 décembre courant, à l’occasion d’une lettre
                  recommandée qui vous a été expédiée de Vienne\oindex{Wien@\textbf{Wien}, \emph{Verwaltungsgebiet}|pw}
                     (Autriche\oindex{Österreich@\textbf{Österreich}|pw}), le 19 décembre, sous le N\textsuperscript{o} 745, par M.
                  Schnitzler, {\kaufmannsund} dans laquelle vous déclarez n’avoir
                  plus trouvé trois pièces de 20 \textsuperscript{f}. qui y auraient été
                  insérées.\end{otherlanguage}}{\lemma{\textnormal{\emph{J’ai … insérées.}}}\Cendnote{\textnormal{französisch: Ich habe Ihre Beschwerde
                  erhalten, die Sie am 21. Dezember an mich gerichtet
                  haben, betreffs eines eingeschriebenen Briefes, der Ihnen am 19. Dezember von Herrn Schnitzler aus Wien\oindex{Wien@\textbf{Wien}, \emph{Verwaltungsgebiet}|pwk} (Österreich\oindex{Österreich@\textbf{Österreich}|pwk}) zugesandt wurde und in dem Sie
                  angeben, drei Münzen zu 20 f. nicht mehr gefunden zu haben, die darin enthalten
                  gewesen sein sollen.}}}\label{K_L02762-10}\pend
           
\pstart
           \label{K_L02762-11v}\edtext{\begin{otherlanguage}{french}\textcolor{gray}{\textbf{\emph{Des ordres ont été immédiatement donnés pour que les faits
                        que vous m’avez signalés soient l’objet d’une d’une enquête dont je vous
                        ferai connaître le résultat dès qu’elle sera terminée.}}}\end{otherlanguage}}{\lemma{\textnormal{\emph{Des … terminée.}}}\Cendnote{\textnormal{französisch: Anweisungen wurden
                  unmittelbar getroffen, dass die Tatsachen, auf die Sie mich hingewiesen haben, die
                  Grundlage einer Untersuchung bilden, deren Ergebnis ich Ihnen nach Abschluss
                  mitteilen werde.}}}\label{K_L02762-11}\pend
           
\pstart
           \label{K_L02762-12v}\edtext{\begin{otherlanguage}{french}\textcolor{gray}{\textbf{\emph{Agréez, M}}}onsieur, \textcolor{gray}{\textbf{\emph{l’assurance de ma considération distinguée}}}\end{otherlanguage}}{\lemma{\textnormal{\emph{Agréez, … distinguée}}}\Cendnote{\textnormal{französisch: Gestatten Sie mir, mein
                  Herr, die Versicherung meiner vorzüglichen Hochachtung}}}\label{K_L02762-12}\pend
           
\pstart
           \begin{otherlanguage}{french}\textcolor{gray}{\textbf{\textbf{Pour le Directeur\pwindex{Selves, Justin de 19.\,7.\,1848 Toulouse – 13.\,1.\,1934 Paris@\textsc{Selves, Justin de} (19.\,7.\,1848 Toulouse – 13.\,1.\,1934 Paris), \emph{Politiker, Beamter}|pwv} Général des Postes et des Télégraphes:}}}{ }{\\}Par \textcolor{gray}{\textbf{\emph{L’Administrateure\pwindex{[Bl]an[qui?] @\textsc{[Bl]an[qui?]}, \emph{Ministerialbeamter}|pwv},}}}\end{otherlanguage}{ }{\\[\baselineskip]}\spacefill\mbox{\textcolor{gray}{Bl}an\textcolor{gray}{qui}\pwindex{[Bl]an[qui?] @\textsc{[Bl]an[qui?]}, \emph{Ministerialbeamter}|pw}\textcolor{gray}{×}\-\textcolor{gray}{×}}\pend
           \leftskip=0em{}
\pstart
           \noindent{}\begin{otherlanguage}{french}\textcolor{gray}{\textbf{\emph{M}}}onsieur\end{otherlanguage} Paul Goldmann\pend
           \selectlanguage{ngerman}\vspace{1em}{\vspace{1\baselineskip}}
\pstart
           \raggedleft{}{\pb}12 rue de Milan\oindex{Rue de Milan@\textbf{Rue de Milan}, \emph{Straße}|pw}\pend
           
\pstart\center{}\begin{otherlanguage}{french}Cher Monsieur Goldmann,\end{otherlanguage}\pend\vspace{0.5em}
\pstart
           \label{K_L02762-13v}\edtext{\begin{otherlanguage}{french}Très touché de votre aimable attention du jour de l’an. Je
                  vous envoie auſsi tous mes meilheurs souhaits.\end{otherlanguage}}{\lemma{\textnormal{\emph{Très … souhaits.}}}\Cendnote{\textnormal{französisch: Sehr berührt von Ihrer
                  freundlichen Aufmerksamkeit zum Neujahrstag.}}}\label{K_L02762-13}\pend
           
\pstart
           \label{K_L02762-14v}\edtext{\begin{otherlanguage}{french}Pourriez-vous me dire l’adreſse de Schnitzler? Elle était
                  bien sur sa lettre, mais illisible. J’ai été très pris ce mois-ci par une affaire
                  que je voudrais entreprendre\strikeout{,} et je n’ai pas
                  encore eu le temps de lire »Liebelei\pwindex{Schnitzler, Arthur 15.\,5.\,1862 Wien – 21.\,10.\,1931 ebd.@\textsc{Schnitzler, Arthur} (15.\,5.\,1862 Wien – 21.\,10.\,1931 ebd.), \emph{Schriftsteller, Mediziner}!Liebelei. Schauspiel in drei Akten@\strich\emph{Liebelei. Schauspiel in drei Akten}|pw}«, mais
                  je pense bien pouvoir le lire ces jours-ci.\end{otherlanguage}}{\lemma{\textnormal{\emph{Pourriez-vous … jours-ci.}}}\Cendnote{\textnormal{französisch: Können Sie mir die Adresse
                  von Schnitzler mitteilen? Sie stand wohl auf
                  seinem Brief, aber unleserlich. Ich war diesen Monat von einer Sache mit Beschlag
                  belegt, die ich unternehmen möchte, und hatte noch keine Zeit, »\emph{Liebelei}\pwindex{Schnitzler, Arthur 15.\,5.\,1862 Wien – 21.\,10.\,1931 ebd.@\textsc{Schnitzler, Arthur} (15.\,5.\,1862 Wien – 21.\,10.\,1931 ebd.), \emph{Schriftsteller, Mediziner}!Liebelei. Schauspiel in drei Akten@\strich\emph{Liebelei. Schauspiel in drei Akten}|pwk}« zu lesen, aber ich bin zuversichtlich, in den
                  kommenden Tagen dazu zu kommen.}}}\label{K_L02762-14}\pend
           
\pstart
           \label{K_L02762-15v}\edtext{\begin{otherlanguage}{french}Votre très dévoué\end{otherlanguage}}{\lemma{\textnormal{\emph{Votre très dévoué}}}\Cendnote{\textnormal{französisch: Ihr sehr ergebener}}}\label{K_L02762-15}{\\[\baselineskip]}\spacefill\mbox{Jean Thorel\pwindex{Thorel, Jean 11.\,9.\,1859 Éragny – 20.\,8.\,1916 Enghien-les-Bains@\textsc{Thorel, Jean} (11.\,9.\,1859 Éragny – 20.\,8.\,1916 Enghien-les-Bains), \emph{Übersetzer, Dramatiker}|pw}}\pend
           \leftskip=0em{}\selectlanguage{ngerman}\endnumbering\briefempfaengerindex{Schnitzler, Arthur@\textsc{Schnitzler, Arthur}!zzzGoldmann, Paul@\emph{von Paul Goldmann}!1896-01-111@{11. 1. [1896]}|)be}\mylabel{L02762h}  \newcommand{\dateiname}{L02762}\newcommand{\titel}{Paul Goldmann an Arthur Schnitzler, 11. 1. [1896]}\newcommand{\editorInnen}{Martin Anton Müller und Laura Untner}%% latex-leseansicht-abspann.tex
%% Abspann für die Leseansicht.
%% Der Schalter \ifkorrekturansicht ist bereits durch den Vorspann gesetzt.

%% latex-abspann.tex
%% Gemeinsamer Abspann für Korrekturansicht und Leseansicht.
%% Setzt den Schalter \ifkorrekturansicht voraus (gesetzt in den
%% einbindenden Dateien latex-korrekturansicht-abspann.tex bzw.
%% latex-leseansicht-abspann.tex).
%% ---------------------------------------------------------------

\normalsize

% Das esempio-Environment wird nur in der Leseansicht benötigt
\ifkorrekturansicht\else
\newenvironment{esempio}[3]%
{
    \vspace{1.5ex}
    \rlap{\underline{#1}}
    \par
    \setlength{\parindent}{0cm}
    \nopagebreak
    \leftskip=#2cm
    \rightskip=#3cm
}
{
    \par
}
\fi

\doendnotes{C}
\bigskip
\vfill

\clearpage

\footnotesize

\ifkorrekturansicht
  \lohead{\textsc{register}}
\fi

% theindex-Environment neu definieren ohne reledmac
\makeatletter
\renewenvironment{theindex}{%
  \ifkorrekturansicht
    \section*{\indexname}%
  \else
    \subsubsection*{Index der erwähnten Entitäten}%
  \fi
  \setlength{\parindent}{0pt}%
  \setlength{\parskip}{0pt plus 0.3pt}%
  \let\item\@idxitem
}{%
  \ifkorrekturansicht\clearpage\fi
}
\makeatother

\IfFileExists{\jobname-pw.ind}{\input{\jobname-pw.ind}}{}

% Quellenangabe nur in der Leseansicht
\ifkorrekturansicht\else
% Fallback-Definitionen, falls die .tex-Datei \titel etc. nicht gesetzt hat
\providecommand{\titel}{}
\providecommand{\editorInnen}{}
\providecommand{\dateiname}{\jobname}

\vspace{3cm}

\vfill

\footnotesize
\textsc{Quelle}: \titel. Herausgegeben von {\editorInnen}. In: \emph{Arthur Schnitzler: Briefwechsel mit Autorinnen und Autoren}.
 Digitale Edition, https://schnitzler-briefe.acdh.oeaw.ac.at/{\dateiname}.html (Stand \today)
\fi

\end{document}


