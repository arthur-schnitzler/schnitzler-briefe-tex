%% latex-korrekturansicht-vorspann.tex
%% Vorspann für die Korrekturansicht.
%% Lädt die gemeinsame Datei latex-vorspann.tex mit gesetztem Schalter.

\newif\ifkorrekturansicht
\korrekturansichttrue

\input{../tex-inputs/latex-vorspann}


\section[ Felix Salten an Arthur Schnitzler, 11. 7. 1901]{L03315 Felix Salten an Arthur Schnitzler, 11. 7. 1901}
\nopagebreak\mylabel{L03315v}
\rehead{ }\normalsize\beginnumbering\briefempfaengerindex{Schnitzler, Arthur@\textsc{Schnitzler, Arthur}!zzzSalten, Felix@\emph{von Felix Salten}!1901-07-112@{11. 7. 1901}|(be}
\toendnotes[C]{\smallbreak\pagebreak[2]}\Standort{CUL, Schnitzler, B 89, A 2.}
\physDesc{Brief, 1 Blatt, 1 Seite, 1319 Zeichen
\newline{}Handschrift: schwarze Tinte, lateinische Kurrent
\newline{}Ordnung: mit Bleistift von unbekannter Hand nummeriert: »139« }\toendnotes[C]{\smallbreak}
\pstart
           {\pb}\textcolor{gray}{\textbf{\textsc{Telephon interurban Nr. 124}}}\hfill \textcolor{gray}{\textbf{\textsc{Hotel}}}\pend
           
\pstart
           \textcolor{gray}{\textbf{\textsc{Telegramm-Adresse:}}}\hfill \textcolor{gray}{\textbf{\textsc{Bristol\oindex{Hotel Bristol Salzburg@\textbf{Hotel Bristol Salzburg}, \emph{Hotel (K.HTL)}|pw}}}}\pend
           
\pstart
           \textcolor{gray}{\textbf{\textsc{\textbf{H}otel \textbf{B}ristol \textbf{S}alzburg\oindex{Hotel Bristol Salzburg@\textbf{Hotel Bristol Salzburg}, \emph{Hotel (K.HTL)}|pw}}.}}\hfill \textcolor{gray}{\textbf{\textsc{Salzburg\oindex{Salzburg@\textbf{Salzburg}, \emph{A.ADM2}|pw}}}}\pend
           
\pstart
           \raggedleft{}\textcolor{gray}{\textbf{(AUSTRIA\oindex{Oesterreich@\textbf{Österreich}, \emph{A.PCLI}|pw})}}\pend
           
\pstart
           \raggedleft{}Salzburg\oindex{Salzburg@\textbf{Salzburg}, \emph{A.ADM2}|pw}, 11. Juli 01\pend
           \vspace{0.5em}
\pstart
           Lieber Freund,{ }heute fand ich hier Ihre \label{K_L03315-1v}\edtext{Karte aus 
               S\textsuperscript{t} Anton\oindex{St. Anton am Arlberg@\textbf{St. Anton am Arlberg}, \emph{A.ADM3}|pw}}{\lemma{\textnormal{\emph{Karte … Anton}}}\Cendnote{\textnormal{Schnitzler hielt sich zwischen 30. 6. 1901 und 12. 7. 1901 in St. Anton am Arlberg\oindex{St. Anton am Arlberg@\textbf{St. Anton am Arlberg}, \emph{A.ADM3}|pwk} auf. Salten\pwindex{Salten, Felix 06.09.1869 – 08.10.1945@\textsc{Salten, Felix} (06.09.1869 – 08.10.1945), \emph{Schriftsteller/Schriftstellerin, Journalist/Journalistin, Chefredakteur/Chefredakteurin}|pwk} war am Vortag also
                  mit
               dem Zug direkt durch den Ort gefahren, an dem Schnitzler sich aufhielt.}}}\label{K_L03315-1}. Ich kam erst
                  gestern{ }Abend aus Darmstadt\oindex{Darmstadt@\textbf{Darmstadt}, \emph{P.PPLA2}|pw} hierher. Gehe
               jetzt nach Ischl\oindex{Bad Ischl@\textbf{Bad Ischl}, \emph{P.PPL}|pw}, und von da erst in 14 Tagen
               nach Wien\oindex{Wien@\textbf{Wien}, \emph{A.ADM2}|pw}. Durch den Arlberg\oindex{Arlberg@\textbf{Arlberg}, \emph{Berg (N.BRG)}|pw} fuhr ich gestern{ }Vormittag. Meine Reise war gut, und wol auch ergiebig. Die Allg. Ztg.\orgindex{Wiener Allgemeine Zeitung@Wiener Allgemeine Zeitung|pw} hatte die Nachricht\pwindex{Lieutenant Gustl.« (Ein ehrenraetliches Urtheil.)@\emph{»Lieutenant Gustl.« (Ein ehrenrätliches Urtheil.)}|pwv} von D\textsuperscript{r}{ }Szeps\pwindex{Szeps, Julius 1867-10-27 – 27.10.1924@\textsc{Szeps, Julius} (1867-10-27 – 27.10.1924), \emph{Journalist/Journalistin}|pw}, der seine Quelle nicht nennen wollte.
               Es war am Tag meiner Abreise. D\textsuperscript{r}{ }Szeps\pwindex{Szeps, Julius 1867-10-27 – 27.10.1924@\textsc{Szeps, Julius} (1867-10-27 – 27.10.1924), \emph{Journalist/Journalistin}|pw} ließ mich rufen, {\kaufmannsund} fragte mich, ob ich etwas gegen die Veröffentlichung
               hätte. Mit Rücksicht auf unser Gespräch über diesen Punkt, sagte ich, es wäre mir
               recht. Sie erinnern sich wol, dass ich Ihnen einmal sagte, wenn die Sache
               durchsickert, wäre ein Verschweigen seitens der Ihnen freundlichen Presse unklug. Das
               sähe so aus, als fühlten Sie sich wirklich getroffen {\kaufmannsund}
               bestraft, und die antis. Presse würde das zweifellos auch so darstellen. Den \label{K_L03315-2v}\edtext{Artikel\pwindex{Lieutenant Gustl.« (Ein ehrenraetliches Urtheil.)@\emph{»Lieutenant Gustl.« (Ein ehrenrätliches Urtheil.)}|pwv}}{\lemma{\textnormal{\emph{Artikel}}}\Cendnote{\textnormal{Es dürfte von dem ohne Autornennung
                  erschienenen Text \emph{»Lieutenant Gustl.« (Ein
                     ehrenrätliches Urtheil.)}\pwindex{Lieutenant Gustl.« (Ein ehrenraetliches Urtheil.)@\emph{»Lieutenant Gustl.« (Ein ehrenrätliches Urtheil.)}|pwk} (\emph{Wiener Allgemeine Zeitung}\pwindex{Wiener Allgemeine Zeitung@\emph{Wiener Allgemeine Zeitung}|pwk}, Nr. 6982,
                        21. 6. 1901, 6 Uhr-Blatt, S. 4) die
                  Rede gewesen sein. Darin wurde von der Aberkennung der Offiziers-Charge berichtet.
                  Da mehrere Zeitungen die gleiche Nachricht am selben Tag brachten, ist nicht
                  unmittelbar zu bestimmen, ob Schnitzler
                  hatte wissen wollen, wie die Information in die Zeitungen gelangt war, oder ob
                  hier eine besondere Information verbreitet worden war, über die kein anderes Blatt
                  verfügte.}}}\label{K_L03315-2} selbst hab’ ich dann erst Abends auf der Bahn lesen
               können. Was meine weiteren Pläne betrifft, ließe viel sich darüber sagen, – brieflich
               ist’s wol aber zu umständlich. Hoffentlich \label{K_L03315-3v}\edtext{sehen wir uns bald}{\lemma{\textnormal{\emph{sehen wir uns bald}}}\Cendnote{\textnormal{Nachweislich sahen sich Salten\pwindex{Salten, Felix 06.09.1869 – 08.10.1945@\textsc{Salten, Felix} (06.09.1869 – 08.10.1945), \emph{Schriftsteller/Schriftstellerin, Journalist/Journalistin, Chefredakteur/Chefredakteurin}|pwk} und Schnitzler erst am 1. 9. 1901
                  wieder.}}}\label{K_L03315-3}. Wenn nicht, – im September? Ich habe
               die Fragerolles\pwindex{Fragerolle, Georges 1855-03-11 – 1920-02-19@\textsc{Fragerolle, Georges} (1855-03-11 – 1920-02-19), \emph{Komponist/Komponistin, Musiker/Musikerin}|pw}-Rivière\pwindex{Riviere, Henri 1864-03-11 – 1951-08-24@\textsc{Rivière, Henri} (1864-03-11 – 1951-08-24), \emph{Maler/Malerin, Radierer/Radiererin, Künstler/Künstlerin}|pw}’schen \label{K_L03315-4v}\edtext{Schattenspiele}{\lemma{\textnormal{\emph{Schattenspiele}}}\Cendnote{\textnormal{Im Kabarett \emph{Le chat noir}\orgindex{Le Chat Noir@Le Chat Noir|pwk} wurden zwischen 1888 und 1897 fast 50 Stücke aufgeführt, für
                  die Henri Rivière\pwindex{Riviere, Henri 1864-03-11 – 1951-08-24@\textsc{Rivière, Henri} (1864-03-11 – 1951-08-24), \emph{Maler/Malerin, Radierer/Radiererin, Künstler/Künstlerin}|pwk} die Ausstattung und Georges Fragerolles\pwindex{Fragerolle, Georges 1855-03-11 – 1920-02-19@\textsc{Fragerolle, Georges} (1855-03-11 – 1920-02-19), \emph{Komponist/Komponistin, Musiker/Musikerin}|pwk} die Musik
                  verantwortete.}}}\label{K_L03315-4} erworben (Geheimnis) und in Zürich\oindex{Zuerich@\textbf{Zürich}, \emph{P.PPLA}|pw} mit Felix\pwindex{Felix, Hugo 19.11.1866 – 25.08.1934@\textsc{Felix, Hugo} (19.11.1866 – 25.08.1934), \emph{Komponist/Komponistin, Chemiker/Chemikerin}|pw}{ }\label{K_L03315-5v}\edtext{Contract}{\lemma{\textnormal{\emph{Contract}}}\Cendnote{\textnormal{für das \emph{Jung-Wiener Theater
                     zum Lieben Augustin}\orgindex{Jung-Wiener Theater zum Lieben Augustin@Jung-Wiener Theater zum Lieben Augustin|pwk}}}}\label{K_L03315-5} gemacht. Vielleicht komme ich in Ischl\oindex{Bad Ischl@\textbf{Bad Ischl}, \emph{P.PPL}|pw}
               dazu \label{K_L03315-6v}\edtext{über Bertha Garlan\pwindex{Frau Bertha Garlan. Roman@\emph{Frau Bertha Garlan. Roman}|pw} zu schreiben}{\lemma{\textnormal{\emph{über … schreiben}}}\Cendnote{\textnormal{Dazu kam es nicht, vgl. Felix Salten an Arthur Schnitzler, 22. 5. 1902.}}}\label{K_L03315-6}, wenn nicht, dann im August in Wien\oindex{Wien@\textbf{Wien}, \emph{A.ADM2}|pw}. Schreiben
               Sie mir bald wieder.\pend
           \pstart Herzlichst Ihr \spacefill\mbox{Salten}\pend{}\selectlanguage{ngerman}\endnumbering\briefempfaengerindex{Schnitzler, Arthur@\textsc{Schnitzler, Arthur}!zzzSalten, Felix@\emph{von Felix Salten}!1901-07-112@{11. 7. 1901}|)be}\mylabel{L03315h}  \normalsize

\doendnotes{C}
\bigskip
\vfill

\clearpage

\footnotesize

\lohead{\textsc{register}}

% Definiere theindex-Environment komplett neu ohne reledmac
\makeatletter
\renewenvironment{theindex}{%
  \section*{\indexname}%
  \setlength{\parindent}{0pt}%
  \setlength{\parskip}{0pt plus 0.3pt}%
  \let\item\@idxitem
}{%
  \clearpage
}
\makeatother

\IfFileExists{\jobname-pw.ind}{\input{\jobname-pw.ind}}{}

\end{document}

      