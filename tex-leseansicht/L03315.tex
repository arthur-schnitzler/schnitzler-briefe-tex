%% latex-leseansicht-vorspann.tex
%% Vorspann für die Leseansicht.
%% Lädt die gemeinsame Datei latex-vorspann.tex mit nicht gesetztem Schalter.

\newif\ifkorrekturansicht
\korrekturansichtfalse

\input{../tex-inputs/latex-vorspann}


\section[ Felix Salten an Arthur Schnitzler, 11. 7. 1901]{L03315 Felix Salten an Arthur Schnitzler,  11. 7. 1901}
\nopagebreak\mylabel{L03315v}
\rehead{ }\normalsize\beginnumbering\briefempfaengerindex{Schnitzler, Arthur@\textsc{Schnitzler, Arthur}!zzzSalten, Felix@\emph{von Felix Salten}!1901-07-112@{11. 7. 1901}|(be}
\toendnotes[C]{\smallbreak\pagebreak[2]}
\correspDesc{Versand  durch Felix Salten am 11. 7. 1901 in Salzburg
\newline{}Erhalt  durch Arthur Schnitzler im Zeitraum [12. 7. 1901
                  – 16. 7. 1901?] in Innsbruck?}\toendnotes[C]{\smallbreak}
\Standort{CUL, Schnitzler, B 89, A 2.}
\physDesc{Brief, 1 Blatt, 1 Seite, 1319 Zeichen
\newline{}Handschrift: schwarze Tinte, lateinische Kurrent
\newline{}Ordnung: mit Bleistift von unbekannter Hand nummeriert: »139« }\toendnotes[C]{\smallbreak}
\pstart
           {\pb}\textcolor{gray}{\textbf{\textsc{Telephon interurban Nr. 124}}}\hfill \textcolor{gray}{\textbf{\textsc{Hotel}}}\pend
           
\pstart
           \textcolor{gray}{\textbf{\textsc{Telegramm-Adresse:}}}\hfill \textcolor{gray}{\textbf{\textsc{Bristol\oindex{Hotel Bristol Salzburg@\textbf{Hotel Bristol Salzburg}, \emph{Hotel}|pw}}}}\pend
           
\pstart
           \textcolor{gray}{\textbf{\textsc{\textbf{H}otel \textbf{B}ristol \textbf{S}alzburg\oindex{Hotel Bristol Salzburg@\textbf{Hotel Bristol Salzburg}, \emph{Hotel}|pw}}.}}\hfill \textcolor{gray}{\textbf{\textsc{Salzburg\oindex{Salzburg@\textbf{Salzburg}, \emph{Verwaltungsgebiet}|pw}}}}\pend
           
\pstart
           \raggedleft{}\textcolor{gray}{\textbf{(AUSTRIA\oindex{Österreich@\textbf{Österreich}|pw})}}\pend
           
\pstart
           \raggedleft{}Salzburg\oindex{Salzburg@\textbf{Salzburg}, \emph{Verwaltungsgebiet}|pw}, 11. Juli 01\pend
           \vspace{0.5em}
\pstart
           Lieber Freund,{ }heute fand ich hier Ihre \label{K_L03315-1v}\edtext{Karte aus 
               S\textsuperscript{t} Anton\oindex{St. Anton am Arlberg@\textbf{St. Anton am Arlberg}, \emph{Verwaltungsgebiet}|pw}}{\lemma{\textnormal{\emph{Karte … Anton}}}\Cendnote{\textnormal{Schnitzler hielt sich zwischen 30. 6. 1901 und 12. 7. 1901 in St. Anton am Arlberg\oindex{St. Anton am Arlberg@\textbf{St. Anton am Arlberg}, \emph{Verwaltungsgebiet}|pwk} auf. Salten\pwindex{Salten, Felix 6.\,9.\,1869 Budapest – 8.\,10.\,1945 Zürich@\textsc{Salten, Felix} (6.\,9.\,1869 Budapest – 8.\,10.\,1945 Zürich), \emph{Schriftsteller, Journalist, Chefredakteur}|pwk} war am Vortag also
                  mit
               dem Zug direkt durch den Ort gefahren, an dem Schnitzler sich aufhielt.}}}\label{K_L03315-1}. Ich kam erst
                  gestern{ }Abend aus Darmstadt\oindex{Darmstadt@\textbf{Darmstadt}, \emph{Hauptstadt}|pw} hierher. Gehe
               jetzt nach Ischl\oindex{Bad Ischl@\textbf{Bad Ischl}|pw}, und von da erst in 14 Tagen
               nach Wien\oindex{Wien@\textbf{Wien}, \emph{Verwaltungsgebiet}|pw}. Durch den Arlberg\oindex{Arlberg@\textbf{Arlberg}, \emph{Berg}|pw} fuhr ich gestern{ }Vormittag. Meine Reise war gut, und wol auch ergiebig. Die Allg. Ztg.\orgindex{Wiener Allgemeine Zeitung@Wiener Allgemeine Zeitung|pw} hatte die Nachricht\pwindex{Lieutenant Gustl.« (Ein ehrenrätliches Urtheil.)@\emph{»Lieutenant Gustl.« (Ein ehrenrätliches Urtheil.)}|pwv} von D\textsuperscript{r}{ }Szeps\pwindex{Szeps, Julius 27.\,10.\,1867 Wien – 27.\,10.\,1924 ebd.@\textsc{Szeps, Julius} (27.\,10.\,1867 Wien – 27.\,10.\,1924 ebd.), \emph{Journalist}|pw}, der seine Quelle nicht nennen wollte.
               Es war am Tag meiner Abreise. D\textsuperscript{r}{ }Szeps\pwindex{Szeps, Julius 27.\,10.\,1867 Wien – 27.\,10.\,1924 ebd.@\textsc{Szeps, Julius} (27.\,10.\,1867 Wien – 27.\,10.\,1924 ebd.), \emph{Journalist}|pw} ließ mich rufen, {\kaufmannsund} fragte mich, ob ich etwas gegen die Veröffentlichung
               hätte. Mit Rücksicht auf unser Gespräch über diesen Punkt, sagte ich, es wäre mir
               recht. Sie erinnern sich wol, dass ich Ihnen einmal sagte, wenn die Sache
               durchsickert, wäre ein Verschweigen seitens der Ihnen freundlichen Presse unklug. Das
               sähe so aus, als fühlten Sie sich wirklich getroffen {\kaufmannsund}
               bestraft, und die antis. Presse würde das zweifellos auch so darstellen. Den \label{K_L03315-2v}\edtext{Artikel\pwindex{Lieutenant Gustl.« (Ein ehrenrätliches Urtheil.)@\emph{»Lieutenant Gustl.« (Ein ehrenrätliches Urtheil.)}|pwv}}{\lemma{\textnormal{\emph{Artikel}}}\Cendnote{\textnormal{Es dürfte von dem ohne Autornennung
                  erschienenen Text \emph{»Lieutenant Gustl.« (Ein
                     ehrenrätliches Urtheil.)}\pwindex{Lieutenant Gustl.« (Ein ehrenrätliches Urtheil.)@\emph{»Lieutenant Gustl.« (Ein ehrenrätliches Urtheil.)}|pwk} (\emph{Wiener Allgemeine Zeitung}\pwindex{Wiener Allgemeine Zeitung@\emph{Wiener Allgemeine Zeitung}|pwk}, Nr. 6982,
                        21. 6. 1901, 6 Uhr-Blatt, S. 4) die
                  Rede gewesen sein. Darin wurde von der Aberkennung der Offiziers-Charge berichtet.
                  Da mehrere Zeitungen die gleiche Nachricht am selben Tag brachten, ist nicht
                  unmittelbar zu bestimmen, ob Schnitzler
                  hatte wissen wollen, wie die Information in die Zeitungen gelangt war, oder ob
                  hier eine besondere Information verbreitet worden war, über die kein anderes Blatt
                  verfügte.}}}\label{K_L03315-2} selbst hab’ ich dann erst Abends auf der Bahn lesen
               können. Was meine weiteren Pläne betrifft, ließe viel sich darüber sagen, – brieflich
               ist’s wol aber zu umständlich. Hoffentlich \label{K_L03315-3v}\edtext{sehen wir uns bald}{\lemma{\textnormal{\emph{sehen wir uns bald}}}\Cendnote{\textnormal{Nachweislich sahen sich Salten\pwindex{Salten, Felix 6.\,9.\,1869 Budapest – 8.\,10.\,1945 Zürich@\textsc{Salten, Felix} (6.\,9.\,1869 Budapest – 8.\,10.\,1945 Zürich), \emph{Schriftsteller, Journalist, Chefredakteur}|pwk} und Schnitzler erst am 1. 9. 1901
                  wieder.}}}\label{K_L03315-3}. Wenn nicht, – im September? Ich habe
               die Fragerolles\pwindex{Fragerolle, Georges 11.\,3.\,1855 Paris – 19.\,2.\,1920 Asnières-sur-Seine@\textsc{Fragerolle, Georges} (11.\,3.\,1855 Paris – 19.\,2.\,1920 Asnières-sur-Seine), \emph{Komponist, Musiker}|pw}-Rivière\pwindex{Rivière, Henri 11.\,3.\,1864 Paris – 24.\,8.\,1951 Sucy-en-Brie@\textsc{Rivière, Henri} (11.\,3.\,1864 Paris – 24.\,8.\,1951 Sucy-en-Brie), \emph{Maler, Radierer, Künstler}|pw}’schen \label{K_L03315-4v}\edtext{Schattenspiele}{\lemma{\textnormal{\emph{Schattenspiele}}}\Cendnote{\textnormal{Im Kabarett \emph{Le chat noir}\orgindex{Le Chat Noir@Le Chat Noir|pwk} wurden zwischen 1888 und 1897 fast 50 Stücke aufgeführt, für
                  die Henri Rivière\pwindex{Rivière, Henri 11.\,3.\,1864 Paris – 24.\,8.\,1951 Sucy-en-Brie@\textsc{Rivière, Henri} (11.\,3.\,1864 Paris – 24.\,8.\,1951 Sucy-en-Brie), \emph{Maler, Radierer, Künstler}|pwk} die Ausstattung und Georges Fragerolles\pwindex{Fragerolle, Georges 11.\,3.\,1855 Paris – 19.\,2.\,1920 Asnières-sur-Seine@\textsc{Fragerolle, Georges} (11.\,3.\,1855 Paris – 19.\,2.\,1920 Asnières-sur-Seine), \emph{Komponist, Musiker}|pwk} die Musik
                  verantwortete.}}}\label{K_L03315-4} erworben (Geheimnis) und in Zürich\oindex{Zürich@\textbf{Zürich}|pw} mit Felix\pwindex{Felix, Hugo 19.\,11.\,1866 Budapest – 25.\,8.\,1934 Hollywood@\textsc{Felix, Hugo} (19.\,11.\,1866 Budapest – 25.\,8.\,1934 Hollywood), \emph{Komponist, Chemiker}|pw}{ }\label{K_L03315-5v}\edtext{Contract}{\lemma{\textnormal{\emph{Contract}}}\Cendnote{\textnormal{für das \emph{Jung-Wiener Theater
                     zum Lieben Augustin}\orgindex{Jung-Wiener Theater zum Lieben Augustin@Jung-Wiener Theater zum Lieben Augustin|pwk}}}}\label{K_L03315-5} gemacht. Vielleicht komme ich in Ischl\oindex{Bad Ischl@\textbf{Bad Ischl}|pw}
               dazu \label{K_L03315-6v}\edtext{über Bertha Garlan\pwindex{Schnitzler, Arthur 15.\,5.\,1862 Wien – 21.\,10.\,1931 ebd.@\textsc{Schnitzler, Arthur} (15.\,5.\,1862 Wien – 21.\,10.\,1931 ebd.), \emph{Schriftsteller, Mediziner}!Frau Bertha Garlan. Roman@\strich\emph{Frau Bertha Garlan. Roman}|pw} zu schreiben}{\lemma{\textnormal{\emph{über … schreiben}}}\Cendnote{\textnormal{Dazu kam es nicht, vgl. XXXX Auszeichnungsfehler: Dokument L03330 nicht gefunden.}}}\label{K_L03315-6}, wenn nicht, dann im August in Wien\oindex{Wien@\textbf{Wien}, \emph{Verwaltungsgebiet}|pw}. Schreiben
               Sie mir bald wieder.\pend
           \pstart Herzlichst Ihr \spacefill\mbox{Salten}\pend{}\selectlanguage{ngerman}\endnumbering\briefempfaengerindex{Schnitzler, Arthur@\textsc{Schnitzler, Arthur}!zzzSalten, Felix@\emph{von Felix Salten}!1901-07-112@{11. 7. 1901}|)be}\mylabel{L03315h}  \newcommand{\dateiname}{L03315}\newcommand{\titel}{Felix Salten an Arthur Schnitzler, 11. 7. 1901}\newcommand{\editorInnen}{Martin Anton Müller und Laura Untner}%% latex-leseansicht-abspann.tex
%% Abspann für die Leseansicht.
%% Der Schalter \ifkorrekturansicht ist bereits durch den Vorspann gesetzt.

%% latex-abspann.tex
%% Gemeinsamer Abspann für Korrekturansicht und Leseansicht.
%% Setzt den Schalter \ifkorrekturansicht voraus (gesetzt in den
%% einbindenden Dateien latex-korrekturansicht-abspann.tex bzw.
%% latex-leseansicht-abspann.tex).
%% ---------------------------------------------------------------

\normalsize

% Das esempio-Environment wird nur in der Leseansicht benötigt
\ifkorrekturansicht\else
\newenvironment{esempio}[3]%
{
    \vspace{1.5ex}
    \rlap{\underline{#1}}
    \par
    \setlength{\parindent}{0cm}
    \nopagebreak
    \leftskip=#2cm
    \rightskip=#3cm
}
{
    \par
}
\fi

\doendnotes{C}
\bigskip
\vfill

\clearpage

\footnotesize

\ifkorrekturansicht
  \lohead{\textsc{register}}
\fi

% theindex-Environment neu definieren ohne reledmac
\makeatletter
\renewenvironment{theindex}{%
  \ifkorrekturansicht
    \section*{\indexname}%
  \else
    \subsubsection*{Index der erwähnten Entitäten}%
  \fi
  \setlength{\parindent}{0pt}%
  \setlength{\parskip}{0pt plus 0.3pt}%
  \let\item\@idxitem
}{%
  \ifkorrekturansicht\clearpage\fi
}
\makeatother

\IfFileExists{\jobname-pw.ind}{\input{\jobname-pw.ind}}{}

% Quellenangabe nur in der Leseansicht
\ifkorrekturansicht\else
% Fallback-Definitionen, falls die .tex-Datei \titel etc. nicht gesetzt hat
\providecommand{\titel}{}
\providecommand{\editorInnen}{}
\providecommand{\dateiname}{\jobname}

\vspace{3cm}

\vfill

\footnotesize
\textsc{Quelle}: \titel. Herausgegeben von {\editorInnen}. In: \emph{Arthur Schnitzler: Briefwechsel mit Autorinnen und Autoren}.
 Digitale Edition, https://schnitzler-briefe.acdh.oeaw.ac.at/{\dateiname}.html (Stand \today)
\fi

\end{document}


