%% latex-korrekturansicht-vorspann.tex
%% Vorspann für die Korrekturansicht.
%% Lädt die gemeinsame Datei latex-vorspann.tex mit gesetztem Schalter.

\newif\ifkorrekturansicht
\korrekturansichttrue

\input{../tex-inputs/latex-vorspann}


\section[Hugo von Hofmannsthal an Arthur Schnitzler, 16. 11. 1907]{L01732 Hugo von Hofmannsthal an Arthur Schnitzler, 16. 11. 1907}
\nopagebreak\mylabel{L01732v}
\rehead{ }\normalsize\beginnumbering\briefempfaengerindex{Schnitzler, Arthur@\textsc{Schnitzler, Arthur}!zzzHofmannsthal, Hugo von@\emph{von Hugo von Hofmannsthal}!1907-11-161@{16. 11. 1907}|(be}
\toendnotes[C]{\smallbreak\pagebreak[2]}\Standort{CUL, Schnitzler, B 43.}
\physDesc{Brief, 1 Blatt, 1 Seite, 688 Zeichen
\newline{}Schreibmaschine
\newline{}Handschrift: schwarze Tinte (\noindent{}Unterschrift)
\newline{}Ordnung: 1) mit Bleistift von unbekannter Hand nummeriert: »\strikeout{286}«  2) mit Bleistift von unbekannter Hand nummeriert:
                                    »289«}
\buchAbdrucke{\weitereDrucke{Hugo von Hofmannsthal, Arthur Schnitzler: \emph{Briefwechsel}. Frankfurt am Main: \emph{S. Fischer} 1964, S. 234.} }\toendnotes[C]{\smallbreak}
\pstart
           \raggedleft{}{\pb}Rodaun\oindex{Rodaun@\textbf{Rodaun}, \emph{A.ADM4}|pw}, den 16. November
                  1907.\pend
           
\pstart{}Mein lieber Arthur!\pend\vspace{0.5em}
\pstart
           Ich danke Ihnen herzlich für den lieben Gedanken, Papa\pwindex{Hofmannsthal, Hugo August von 21.12.1841 – 08.12.1915@\textsc{Hofmannsthal, Hugo August von} (21.12.1841 – 08.12.1915), \emph{Bankdirektor/Bankdirektorin}|pwv} einzuladen. Bitte, tun Sie es. Er wohnt I. Himmelpfortgasse 17\oindex{Himmelpfortgasse@\textbf{Himmelpfortgasse}, \emph{Straße (K.STR)}|pw}. Er wird erst zum
               Nachtmahl kommen und wir sind dann also vorher ja doch allein, umsomehr als ich Sie
               durch diese Zeilen vielmals bitte, mir zu erlauben, dass ich für meine Person schon
               um ½ 6 kommen darf, um Ihnen das Vorhandene von meinem Stück\pwindex{Silvia im »Stern«@\emph{Silvia im »Stern«}|pwv} vorzulesen. Ich stehe dieser Sache so
               unbeschreiblich ratlos und verworren gegenüber und weiss, dass Sie mir helfen können.
               Also erlauben Sie mir das. Es bedarf weiter keiner Antwort, und ich komme.\pend
           \pstart Herzlich Ihr\spacefill\mbox{{[}hs.:{]} \strikeout{Hofm} Hugo.}\pend{}
\pstart
           \noindent{}{[}ms.:{]} P. S. S.\pwindex{Schwarzkopf, Gustav 07.11.1853 – 13.11.1939@\textsc{Schwarzkopf, Gustav} (07.11.1853 – 13.11.1939), \emph{Schriftsteller/Schriftstellerin}|pw}
                  wäre mir bei so schlechter eigener Verfassung eine Qual.\pend
           \selectlanguage{ngerman}\endnumbering\briefempfaengerindex{Schnitzler, Arthur@\textsc{Schnitzler, Arthur}!zzzHofmannsthal, Hugo von@\emph{von Hugo von Hofmannsthal}!1907-11-161@{16. 11. 1907}|)be}\mylabel{L01732h}  \normalsize

\doendnotes{C}
\bigskip
\vfill

\clearpage

\footnotesize

\lohead{\textsc{register}}

% Definiere theindex-Environment komplett neu ohne reledmac
\makeatletter
\renewenvironment{theindex}{%
  \section*{\indexname}%
  \setlength{\parindent}{0pt}%
  \setlength{\parskip}{0pt plus 0.3pt}%
  \let\item\@idxitem
}{%
  \clearpage
}
\makeatother

\IfFileExists{\jobname-pw.ind}{\input{\jobname-pw.ind}}{}

\end{document}

      