%% latex-leseansicht-vorspann.tex
%% Vorspann für die Leseansicht.
%% Lädt die gemeinsame Datei latex-vorspann.tex mit nicht gesetztem Schalter.

\newif\ifkorrekturansicht
\korrekturansichtfalse

\input{../tex-inputs/latex-vorspann}


         
         \renewcommand{\erwaehntePersonen}{Personen: Hugo August von Hofmannsthal, Gustav Schwarzkopf}
         \renewcommand{\erwaehnteOrte}{Orte: Himmelpfortgasse, Rodaun, Wien}
         \renewcommand{\erwaehnteWerke}{Werke: Silvia im »Stern«}
               \section[Hugo von Hofmannsthal an Arthur Schnitzler, 16. 11. 1907]{ Hugo von Hofmannsthal an Arthur Schnitzler, 16. 11. 1907}\nopagebreak\mylabel{v}\rehead{ }\begin{ledgroupsized}[t]{13cm}\normalsize\beginnumbering \toendnotes[C]{\smallbreak\pagebreak[2]} \Standort{CUL, Schnitzler, B 43.}
\physDesc{Brief, 1 Blatt, 1 Seite
\newline{}Schreibmaschine
\newline{}Handschrift: schwarze Tinte (\noindent{}Unterschrift)\newline{}Ordnung: 1) mit Bleistift von unbekannter Hand nummeriert: »\strikeout{286}«  2) mit Bleistift von unbekannter Hand nummeriert: »289«}\buchAbdrucke{\weitereDrucke{Hugo von Hofmannsthal, Arthur Schnitzler: \emph{Briefwechsel}. Hg. Therese Nickl und Heinrich Schnitzler. Frankfurt am Main: \emph{S. Fischer} 1964, S. 234.} }\toendnotes[C]{\smallbreak}\pstart
           \raggedleft{}{\pb}Rodaun\oindex{Rodaun@\textbf{Rodaun}|pw}, den 16. November
                  1907.\pend
           \pstart{}Mein lieber Arthur!\pend\pstart
           Ich danke Ihnen herzlich für den lieben Gedanken, Papa\pwindex{Hofmannsthal, Hugo August von 21.12.1841 – 08.12.1915@\textsc{Hofmannsthal, Hugo August von} (21.12.1841 – 08.12.1915), \emph{Bankdirektor}|pwv} einzuladen. Bitte, tun Sie es. Er wohnt I. Himmelpfortgasse 17\oindex{Himmelpfortgasse@\textbf{Himmelpfortgasse}|pw}. Er wird erst zum Nachtmahl
               kommen und wir sind dann also vorher ja doch allein, umsomehr als ich Sie durch
               diese Zeilen vielmals bitte, mir zu erlauben, dass ich für meine Person schon um
                  ½ 6 kommen darf, um Ihnen das Vorhandene von meinem Stück\pwindex{Hofmannsthal, Hugo von 1874-02-01 – 1929-07-15@\textsc{Hofmannsthal, Hugo von} (1874-02-01 – 1929-07-15), \emph{Schriftsteller}!Silvia im »Stern«1909@\strich\emph{Silvia im »Stern«} {[}1909{]}|pwv} vorzulesen. Ich stehe dieser Sache so
               unbeschreiblich ratlos und verworren gegenüber und weiss, dass Sie mir helfen
               können. Also erlauben Sie mir das. Es bedarf weiter keiner Antwort, und ich
               komme.\pend
           \pstart Herzlich Ihr\spacefill\mbox{{[}hs.:{]} \strikeout{Hofm} Hugo.}\pend{}\pstart
           \noindent{}{[}ms.:{]} P. S. S.\pwindex{Schwarzkopf, Gustav 07.11.1853 – 13.11.1939@\textsc{Schwarzkopf, Gustav} (07.11.1853 – 13.11.1939), \emph{Schriftsteller}|pw} wäre mir bei so schlechter eigener
                  Verfassung eine Qual.\pend
           
         
         \endnumbering\mylabel{h}\end{ledgroupsized}  \newcommand{\dateiname}{L01732}\newcommand{\titel}{Hugo von Hofmannsthal an Arthur Schnitzler, 16. 11. 1907}\newcommand{\editorInnen}{Martin Anton Müller und Gerd-Hermann Susen}%% latex-leseansicht-abspann.tex
%% Abspann für die Leseansicht.
%% Der Schalter \ifkorrekturansicht ist bereits durch den Vorspann gesetzt.

%% latex-abspann.tex
%% Gemeinsamer Abspann für Korrekturansicht und Leseansicht.
%% Setzt den Schalter \ifkorrekturansicht voraus (gesetzt in den
%% einbindenden Dateien latex-korrekturansicht-abspann.tex bzw.
%% latex-leseansicht-abspann.tex).
%% ---------------------------------------------------------------

\normalsize

% Das esempio-Environment wird nur in der Leseansicht benötigt
\ifkorrekturansicht\else
\newenvironment{esempio}[3]%
{
    \vspace{1.5ex}
    \rlap{\underline{#1}}
    \par
    \setlength{\parindent}{0cm}
    \nopagebreak
    \leftskip=#2cm
    \rightskip=#3cm
}
{
    \par
}
\fi

\doendnotes{C}
\bigskip
\vfill

\clearpage

\footnotesize

\ifkorrekturansicht
  \lohead{\textsc{register}}
\fi

% theindex-Environment neu definieren ohne reledmac
\makeatletter
\renewenvironment{theindex}{%
  \ifkorrekturansicht
    \section*{\indexname}%
  \else
    \subsubsection*{Index der erwähnten Entitäten}%
  \fi
  \setlength{\parindent}{0pt}%
  \setlength{\parskip}{0pt plus 0.3pt}%
  \let\item\@idxitem
}{%
  \ifkorrekturansicht\clearpage\fi
}
\makeatother

\IfFileExists{\jobname-pw.ind}{\input{\jobname-pw.ind}}{}

% Quellenangabe nur in der Leseansicht
\ifkorrekturansicht\else
% Fallback-Definitionen, falls die .tex-Datei \titel etc. nicht gesetzt hat
\providecommand{\titel}{}
\providecommand{\editorInnen}{}
\providecommand{\dateiname}{\jobname}

\vspace{3cm}

\vfill

\footnotesize
\textsc{Quelle}: \titel. Herausgegeben von {\editorInnen}. In: \emph{Arthur Schnitzler: Briefwechsel mit Autorinnen und Autoren}.
 Digitale Edition, https://schnitzler-briefe.acdh.oeaw.ac.at/{\dateiname}.html (Stand \today)
\fi

\end{document}


      