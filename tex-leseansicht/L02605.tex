%% latex-leseansicht-vorspann.tex
%% Vorspann für die Leseansicht.
%% Lädt die gemeinsame Datei latex-vorspann.tex mit nicht gesetztem Schalter.

\newif\ifkorrekturansicht
\korrekturansichtfalse

\input{../tex-inputs/latex-vorspann}


               \section[Paul Goldmann an Arthur Schnitzler, 8. 1. {[}1894{]}]{ Paul Goldmann an Arthur Schnitzler, 8. 1. {[}1894{]}}\nopagebreak\mylabel{v}\rehead{ }\begin{ledgroupsized}[t]{13cm}\normalsize\beginnumbering\briefempfaengerindex{Schnitzler, Arthur@\textsc{Schnitzler, Arthur}!zzzGoldmann, Paul@\emph{von Paul Goldmann}!1894-01-081@{8. 1. {[}1894{]}}|(be} \toendnotes[C]{\smallbreak\pagebreak[2]} \Standort{DLA, A:Schnitzler, HS.NZ85.1.3164.}
\physDesc{Brief, 2 Blätter, 6 Seiten
\newline{}Handschrift: schwarze Tinte, deutsche Kurrent
\newline{}Schnitzler: 1) mit Bleistift auf dem ersten Blatt die Jahreszahl »94« vermerkt und auf dem zweiten Blatt »Jän 8/94« 2) mit rotem Buntstift eine Unterstreichung}\toendnotes[C]{\smallbreak}\pstart
           \raggedleft{}{\pb}\textsc{Paris\oindex{Paris@\textbf{Paris}|pw},}{ }8. Januar.\pend
           \pstart\center{}Mein lieber Freund,\pend\pstart
           Ich bin heute ſo ganz verzweifelt ins Bureau\oindex{rue Richelieu@\textbf{rue Richelieu}|pwv} gekommen und habe Deinen lieben Brief gefunden! Du
               biſt wirklich mein einziger Troſt in dieſer ſo bitterlich ſchweren Zeit, und ich
               danke Dir von ganzem Herzen für dieſe Güte, dieſe Treue, dieſe Freundſchaft, die das
               Beſte iſt, was mir das Leben noch geboten. Ich habe wirklich keine Ahnung, ob ich
               irgend etwas leiſte, und in der Entmuthigung, in die ich ſo verſunken bin, iſt mir
                  ein\strikeout{\textcolor{gray}{e}} Beifallszeichen, wie das Deinige ein Halt und {\pb}ein Anſporn, deſſen Werth ich Dir nicht mit Worten zu ſchildern vermag. Ich weiß
               ja, wie ſehr der Wunſch, mir Gutes zu erweiſen, Dein Urtheil zu meinen Gunſten
               beeinflußt. Aber wenn auch die Selbſterkenntniß die nöthigen Subtractionen macht, ſo
               bleibt doch noch genug übrig, um Einem das Herz mit freudigem Stolz zu erfüllen. Ich
               danke Dir viel tauſendmal.\pend
           \pstart
           Gerade in dieſen Tagen bin ich wieder einmal vor die Exiſtenzfrage geſtellt. Mein Blatt\orgindex{Frankfurter Zeitung@Frankfurter Zeitung|pwv} beutet mich {\pb}in ſchamloſer Weise aus. Ganz abgeſehen davon, daß
               es fraglich iſt, ob meine Kräfte noch zur weiteren Leiſtung der Rieſenarbeit
               ausreichen, kann ich mit dem Bettellohn, den man mir zahlt, nicht mehr auskommen. Ich
               habe nach zwei Jahren zum erſten Mal um eine kleine Erhöhung gebeten. Man hat ſie mi\substVorne{}\textsuperscript{\textcolor{gray}{×}}\substDazwischen{}r\substHinten{} rundweg abgeſchlagen; noch mehr: man hat mir mein Speſenconto, das ſchon
               jetzt in keiner Weiſe mehr ausreicht, um die Hälfte reducirt; und man hat mir barſch
                  {\pb}zu verſtehen gegeben: wenn mir das nicht paßte,
               ſo ſollte ich es umgehend mittheilen, damit die Zeitung\orgindex{Frankfurter Zeitung@Frankfurter Zeitung|pwv} Schritte zur Neubeſetzung meines Poſtens thun könne.
               Ich bin ſchon ſo gedehmüthigt, daß \strikeout{d\textcolor{gray}{i}} ich die moraliſche Erniedrigung in dem Allen kaum mehr verſpüre. Aber die
               praktiſche Frage tritt drohend vor mich heran. Ich ſtehe vor meinem Ruin. Nirgends
               ein Ausweg zu finden. Wäre es nicht möglich, daß Du oder einer der Freunde mir
               irgendwo einen \substVorne{}\textsuperscript{\textcolor{gray}{ſ}}\substDazwischen{}k\substHinten{}leinen ſtillen Poſten verſchaffen könntet? Gleichgiltig in welchem Beruf.\pend
           \pstart
           {\pb}Bitte, liebſter Freund, ſchick’ mir noch zwei \textsc{Anatol\pwindex{Schnitzler, Arthur 15.05.1862 – 21.10.1931@\textsc{Schnitzler, Arthur} (15.05.1862 – 21.10.1931), \emph{Schriftsteller, Mediziner}!Anatol1892-10-29 – 1892-10-29@\strich\emph{Anatol} {[}1892-10-29 – 1892-10-29{]}|pw}}-Exemplare. Ich brauche ſie hier in Deinem Intereſſe. Vielleicht kann ich Dir
               doch hier eine Beſprechung verſchaffen. In der Frankfurter Zeitg.\orgindex{Frankfurter Zeitung@Frankfurter Zeitung|pw} kommſt Du \label{K_L02605-2v}\edtext{demnächſt}{\lemma{\textnormal{\emph{demnächſt}}}\Cendnote{\textnormal{XXXX Rezension\pwindex{?? Werk@Nicht ermittelte Verfasserinnen und Verfasser!?? Rezension von Anatol in der Frankfurter Zeitung?]nach dem 8.1.1894@\emph{[?? Rezension von Anatol in der Frankfurter Zeitung?]} {[}nach dem 8.1.1894{]}|pwuvk} erschienen?}}}\label{K_L02605-2h}
               an die Reihe.\pend
           \pstart
           Bitte, danke auch Herrn \textsc{Salten\pwindex{Salten, Felix 06.09.1869 – 08.10.1945@\textsc{Salten, Felix} (06.09.1869 – 08.10.1945), \emph{Schriftsteller, Journalist}|pw}} für ſeine freundlichen Worte, die mich ſehr bewegt haben, und verſichere ihn
               meiner aufrichtigen Ergebenheit. Er möchte mir auch einmal etwas {\pb}von ſich ſchicken, und er ſoll nach \textsc{Paris\oindex{Paris@\textbf{Paris}|pw}} kommen. Danke auch all’ den lieben \label{K_L02605-1v}\edtext{Leuten}{\lemma{\textnormal{\emph{Leuten}}}\Cendnote{\textnormal{nicht
                  identifiziert}}}\label{K_L02605-1h} für ihren Neujahrswunſch\pend
           \pstart
           Ich grüße Dich von Herzen, mein theurer Freund, und bitte Dich, mir ſo treu zu
               bleiben, wie ich Dir bin. {\\[\baselineskip]}Dein {\\[\baselineskip]}\spacefill\mbox{Paul Goldmann.}\pend
           \leftskip=0em{}          \endnumbering\briefempfaengerindex{Schnitzler, Arthur@\textsc{Schnitzler, Arthur}!zzzGoldmann, Paul@\emph{von Paul Goldmann}!1894-01-081@{8. 1. {[}1894{]}}|)be}\mylabel{h}\end{ledgroupsized}  \newcommand{\dateiname}{L02605}\newcommand{\titel}{Paul Goldmann an Arthur Schnitzler, 8. 1. [1894]}\newcommand{\editorInnen}{Martin Anton Müller und Laura Untner}
            \footnotesize
\begin{ledgroupsized}[t]{11.5cm}
\doendnotes{C}
\end{ledgroupsized}
         %% latex-leseansicht-abspann.tex
%% Abspann für die Leseansicht.
%% Der Schalter \ifkorrekturansicht ist bereits durch den Vorspann gesetzt.

%% latex-abspann.tex
%% Gemeinsamer Abspann für Korrekturansicht und Leseansicht.
%% Setzt den Schalter \ifkorrekturansicht voraus (gesetzt in den
%% einbindenden Dateien latex-korrekturansicht-abspann.tex bzw.
%% latex-leseansicht-abspann.tex).
%% ---------------------------------------------------------------

\normalsize

% Das esempio-Environment wird nur in der Leseansicht benötigt
\ifkorrekturansicht\else
\newenvironment{esempio}[3]%
{
    \vspace{1.5ex}
    \rlap{\underline{#1}}
    \par
    \setlength{\parindent}{0cm}
    \nopagebreak
    \leftskip=#2cm
    \rightskip=#3cm
}
{
    \par
}
\fi

\doendnotes{C}
\bigskip
\vfill

\clearpage

\footnotesize

\ifkorrekturansicht
  \lohead{\textsc{register}}
\fi

% theindex-Environment neu definieren ohne reledmac
\makeatletter
\renewenvironment{theindex}{%
  \ifkorrekturansicht
    \section*{\indexname}%
  \else
    \subsubsection*{Index der erwähnten Entitäten}%
  \fi
  \setlength{\parindent}{0pt}%
  \setlength{\parskip}{0pt plus 0.3pt}%
  \let\item\@idxitem
}{%
  \ifkorrekturansicht\clearpage\fi
}
\makeatother

\IfFileExists{\jobname-pw.ind}{\input{\jobname-pw.ind}}{}

% Quellenangabe nur in der Leseansicht
\ifkorrekturansicht\else
% Fallback-Definitionen, falls die .tex-Datei \titel etc. nicht gesetzt hat
\providecommand{\titel}{}
\providecommand{\editorInnen}{}
\providecommand{\dateiname}{\jobname}

\vspace{3cm}

\vfill

\footnotesize
\textsc{Quelle}: \titel. Herausgegeben von {\editorInnen}. In: \emph{Arthur Schnitzler: Briefwechsel mit Autorinnen und Autoren}.
 Digitale Edition, https://schnitzler-briefe.acdh.oeaw.ac.at/{\dateiname}.html (Stand \today)
\fi

\end{document}


      