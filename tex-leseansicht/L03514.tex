%% latex-leseansicht-vorspann.tex
%% Vorspann für die Leseansicht.
%% Lädt die gemeinsame Datei latex-vorspann.tex mit nicht gesetztem Schalter.

\newif\ifkorrekturansicht
\korrekturansichtfalse

\input{../tex-inputs/latex-vorspann}

\begin{center}
            \textcolor{red}{ENTWURF, NICHT FERTIG KORRIGIERT}
                      \end{center}
            
         \renewcommand{\erwaehnteInstitutionen}{Institutionen: Lessing-Theater}
         \renewcommand{\erwaehnteOrte}{Orte: Berlin, Charlottenburg, Edmund-Weiß-Gasse, Friedrichstraße, Hotel Victoria und Victoria-Café, Unter den Linden, Wien, XVIII., Währing}
         \renewcommand{\erwaehnteWerke}{Werke: Vom andern Ufer. Einakter}
               \section[Felix Salten an Arthur Schnitzler, 22. 10. 1907]{ Felix Salten an Arthur Schnitzler, 22. 10. 1907}\nopagebreak\mylabel{v}\rehead{ }\begin{ledgroupsized}[t]{13cm}\normalsize\beginnumbering \toendnotes[C]{\smallbreak\pagebreak[2]} \Standort{CUL, Schnitzler, B 89, B 1.}
\physDesc{Bildpostkarte, 247 Zeichen
\newline{}Handschrift: schwarze Tinte, lateinische Kurrent
\newline{}Versand: Stempel: »\nobreak{}\oindex{Charlottenburg@\textbf{Charlottenburg}|pwk}Charlottenburg, 22. 10. 07, 4\nobreak{}«.  
\newline{}Schnitzler: mit Bleistift datiert: »22/\textcolor{gray}{×} 07« 
\newline{}Ordnung: mit Bleistift von unbekannter Hand nummeriert:
                                    »237« }\toendnotes[C]{\smallbreak}\pstart{}{\pb}Herrn D\textsuperscript{r} Arthur Schnitzler\pend{}\pstart{}Wien XVIII.\oindex{XVIII., Waehring@\textbf{XVIII., Währing}|pw}\pend{}\pstart{}Spöttelgaſse 7\oindex{Edmund-Weiss-Gasse@\textbf{Edmund-Weiß-Gasse}|pw}\pend{}{\bigskip}\pstart
           \noindent{}\centering{}{\pb}\textcolor{gray}{\textbf{Berlin}}\oindex{Berlin@\textbf{Berlin}|pw}\pend
           \pstart
           \noindent{}\centering{}\textcolor{gray}{\textbf{Unter den Linden\oindex{Unter den Linden@\textbf{Unter den Linden}|pw}}}\pend
           \pstart
           \noindent{}\centering{}\textcolor{gray}{\textbf{Ecke Friedrichstr.\oindex{Friedrichstrasse@\textbf{Friedrichstraße}|pw} (Victoria-Cafe\oindex{Hotel Victoria und Victoria-Cafe@\textbf{Hotel Victoria und Victoria-Café}|pw})}}\pend
           \pstart{}{\pb}Lieber,\pend\pstart
           vielen dank für die \label{K_L03514-1v}\edtext{Depesche}{\lemma{\textnormal{\emph{Depesche}}}\Cendnote{\textnormal{Vermutlich anlässlich der Uraufführung von
                     \emph{Vom andern Ufer}\pwindex{Salten, Felix 06.09.1869 – 08.10.1945@\textsc{Salten, Felix} (06.09.1869 – 08.10.1945), \emph{Schriftsteller, Journalist}!Vom andern Ufer. Einakter1907-10-15@\strich\emph{Vom andern Ufer. Einakter} {[}1907-10-15{]}|pwk} am
                     15. 10. 1907 am \emph{Lessing-Theater}\orgindex{Lessing-Theater@Lessing-Theater|pwk}.}}}\label{K_L03514-1h}. Wir sind diese Woche in Wien\oindex{Wien@\textbf{Wien}|pw}. Wenn’s noch schön ist, komm’ ich auf den Tennisplatz,
               hoffe aber jedenfalls, Sie bald zu sehen.\pend
           \pstart
           Herzlichst von uns zu Ihnen Ihr {\\[\baselineskip]}\spacefill\mbox{Salten}\pend
           \leftskip=0em{}
         
         \endnumbering\mylabel{h}\end{ledgroupsized}\begin{anhang}\end{anhang}\newcommand{\dateiname}{L03514}\newcommand{\titel}{Felix Salten an Arthur Schnitzler, 22. 10. 1907}\newcommand{\editorInnen}{Martin Anton Müller und Laura Untner}%% latex-leseansicht-abspann.tex
%% Abspann für die Leseansicht.
%% Der Schalter \ifkorrekturansicht ist bereits durch den Vorspann gesetzt.

%% latex-abspann.tex
%% Gemeinsamer Abspann für Korrekturansicht und Leseansicht.
%% Setzt den Schalter \ifkorrekturansicht voraus (gesetzt in den
%% einbindenden Dateien latex-korrekturansicht-abspann.tex bzw.
%% latex-leseansicht-abspann.tex).
%% ---------------------------------------------------------------

\normalsize

% Das esempio-Environment wird nur in der Leseansicht benötigt
\ifkorrekturansicht\else
\newenvironment{esempio}[3]%
{
    \vspace{1.5ex}
    \rlap{\underline{#1}}
    \par
    \setlength{\parindent}{0cm}
    \nopagebreak
    \leftskip=#2cm
    \rightskip=#3cm
}
{
    \par
}
\fi

\doendnotes{C}
\bigskip
\vfill

\clearpage

\footnotesize

\ifkorrekturansicht
  \lohead{\textsc{register}}
\fi

% theindex-Environment neu definieren ohne reledmac
\makeatletter
\renewenvironment{theindex}{%
  \ifkorrekturansicht
    \section*{\indexname}%
  \else
    \subsubsection*{Index der erwähnten Entitäten}%
  \fi
  \setlength{\parindent}{0pt}%
  \setlength{\parskip}{0pt plus 0.3pt}%
  \let\item\@idxitem
}{%
  \ifkorrekturansicht\clearpage\fi
}
\makeatother

\IfFileExists{\jobname-pw.ind}{\input{\jobname-pw.ind}}{}

% Quellenangabe nur in der Leseansicht
\ifkorrekturansicht\else
% Fallback-Definitionen, falls die .tex-Datei \titel etc. nicht gesetzt hat
\providecommand{\titel}{}
\providecommand{\editorInnen}{}
\providecommand{\dateiname}{\jobname}

\vspace{3cm}

\vfill

\footnotesize
\textsc{Quelle}: \titel. Herausgegeben von {\editorInnen}. In: \emph{Arthur Schnitzler: Briefwechsel mit Autorinnen und Autoren}.
 Digitale Edition, https://schnitzler-briefe.acdh.oeaw.ac.at/{\dateiname}.html (Stand \today)
\fi

\end{document}


      