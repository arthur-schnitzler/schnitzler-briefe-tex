%% latex-korrekturansicht-vorspann.tex
%% Vorspann für die Korrekturansicht.
%% Lädt die gemeinsame Datei latex-vorspann.tex mit gesetztem Schalter.

\newif\ifkorrekturansicht
\korrekturansichttrue

\input{../tex-inputs/latex-vorspann}


\section[ Felix Salten an Arthur Schnitzler, 22. 10. 1907]{L03514 Felix Salten an Arthur Schnitzler, 22. 10. 1907}
\nopagebreak\mylabel{L03514v}
\rehead{ }\normalsize\beginnumbering\briefempfaengerindex{Schnitzler, Arthur@\textsc{Schnitzler, Arthur}!zzzSalten, Felix@\emph{von Felix Salten}!1907-10-221@{22. 10. 1907}|(be}
\toendnotes[C]{\smallbreak\pagebreak[2]}\Standort{CUL, Schnitzler, B 89, B 1.}
\physDesc{Bildpostkarte, 241 Zeichen
\newline{}Handschrift: schwarze Tinte, lateinische Kurrent
\newline{}Versand: Stempel: »\nobreak{}\oindex{Charlottenburg@\textbf{Charlottenburg}, \emph{P.PPLX}|pwk}Charlottenburg 2, 22. 10. 07, 4\textcolor{gray}{–5N.}\nobreak{}«.  
\newline{}Schnitzler: mit Bleistift datiert: »22/\textcolor{gray}{5} 07« 
\newline{}Ordnung: mit Bleistift von unbekannter Hand nummeriert: »237« }\toendnotes[C]{\smallbreak}\pstart{}{\pb}Herrn D\textsuperscript{r} Arthur Schnitzler\pend{}\pstart{}Wien XVIII.\oindex{XVIII., Waehring@\textbf{XVIII., Währing}, \emph{A.ADM3}|pw}\pend{}\pstart{}Spöttelgaße 7\oindex{Edmund-Weiss-Gasse 7@\textbf{Edmund-Weiß-Gasse 7}, \emph{Wohngebäude (K.WHS)}|pw}\pend{}{\bigskip}
\pstart
           \noindent{}\centering{}{\pb}\textcolor{gray}{\textbf{Berlin}}\oindex{Berlin@\textbf{Berlin}, \emph{P.PPLC}|pw}\pend
           
\pstart
           \centering{}\textcolor{gray}{\textbf{Unter den Linden\oindex{Unter den Linden@\textbf{Unter den Linden}, \emph{P.PPLX}|pw}}}\pend
           
\pstart
           \centering{}\textcolor{gray}{\textbf{Ecke Friedrichstr.\oindex{Friedrichstrasse [Berlin]@\textbf{Friedrichstraße [Berlin]}, \emph{Straße (K.STR)}|pw} (Victoria-Cafe\oindex{Hotel Victoria und Victoria-Cafe@\textbf{Hotel Victoria und Victoria-Café}, \emph{Hotel (K.HTL)}|pw})}}\pend
           \vspace{1em}
\pstart{}{\pb}Lieber,\pend\vspace{0.5em}
\pstart
           vielen dank für die \label{K_L03514-1v}\edtext{Depesche}{\lemma{\textnormal{\emph{Depesche}}}\Cendnote{\textnormal{Schnitzler dürfte anlässlich der Uraufführung von
                     \emph{Vom andern Ufer}\pwindex{Vom andern Ufer. Einakter@\emph{Vom andern Ufer. Einakter}|pwk} am 15. 10. 1907 am \emph{Lessing-Theater}\orgindex{Lessing-Theater@Lessing-Theater|pwk} ein Gratulationstelegramm geschrieben haben.}}}\label{K_L03514-1}. Wir\pwindex{Salten, Ottilie 07.03.1868 – 22.06.1942@\textsc{Salten, Ottilie} (07.03.1868 – 22.06.1942), \emph{Schauspieler/Schauspielerin}|pwv} sind diese Woche in Wien\oindex{Wien@\textbf{Wien}, \emph{A.ADM2}|pw}. Wenn’s noch
               schön ist, \label{K_L03514-2v}\edtext{komm’ ich auf den
                  Tennisplatz}{\lemma{\textnormal{\emph{komm’ … Tennisplatz}}}\Cendnote{\textnormal{Vgl. A. S.: \emph{Tagebuch}, 23. 10. 1907.
               }}}\label{K_L03514-2}, hoffe aber jedenfalls, Sie bald zu sehen.\pend
           
\pstart
           Herzlichst von uns\pwindex{Salten, Ottilie 07.03.1868 – 22.06.1942@\textsc{Salten, Ottilie} (07.03.1868 – 22.06.1942), \emph{Schauspieler/Schauspielerin}|pwv} zu
               Ihnen Ihr {\\[\baselineskip]}\spacefill\mbox{Salten}\pend
           \leftskip=0em{}\selectlanguage{ngerman}\endnumbering\briefempfaengerindex{Schnitzler, Arthur@\textsc{Schnitzler, Arthur}!zzzSalten, Felix@\emph{von Felix Salten}!1907-10-221@{22. 10. 1907}|)be}\mylabel{L03514h}  \normalsize

\doendnotes{C}
\bigskip
\vfill

\clearpage

\footnotesize

\lohead{\textsc{register}}

% Definiere theindex-Environment komplett neu ohne reledmac
\makeatletter
\renewenvironment{theindex}{%
  \section*{\indexname}%
  \setlength{\parindent}{0pt}%
  \setlength{\parskip}{0pt plus 0.3pt}%
  \let\item\@idxitem
}{%
  \clearpage
}
\makeatother

\IfFileExists{\jobname-pw.ind}{\input{\jobname-pw.ind}}{}

\end{document}

      