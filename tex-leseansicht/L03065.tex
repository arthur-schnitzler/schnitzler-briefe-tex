%% latex-leseansicht-vorspann.tex
%% Vorspann für die Leseansicht.
%% Lädt die gemeinsame Datei latex-vorspann.tex mit nicht gesetztem Schalter.

\newif\ifkorrekturansicht
\korrekturansichtfalse

\input{../tex-inputs/latex-vorspann}


         
         \renewcommand{\erwaehntePersonen}{Personen: Richard Beer-Hofmann, Ludwig Brefeld, Rosa Freudenthal, Paul Goldmann, Clementine Goldmann, Ernst von Hammerstein-Loxten, Johannes von Miquel, Emil Rechert, Olga Schnitzler, Elisabeth Steinrück}
         \renewcommand{\erwaehnteInstitutionen}{Institutionen: Neue Freie Presse}
         \renewcommand{\erwaehnteOrte}{Orte: Berlin, Dessauer Straße, Elbe, Engadin, Hannover, Makedonien, Mittellandkanal, Nordmazedonien, Preußen, Welsberg-Taisten, Wien, Wörthersee}
         \renewcommand{\erwaehnteWerke}{Werke: Die verhaßte Korrektheit. Wiener Novellette, Frankfurter Zeitung, Frau Bertha Garlan. Roman}
               \section[ Paul Goldmann an Arthur Schnitzler, 7. 5. {[}1901{]}]{ Paul Goldmann an Arthur Schnitzler, 7. 5. {[}1901{]}}\nopagebreak\mylabel{v}\rehead{ }\begin{ledgroupsized}[t]{13cm}\normalsize\beginnumbering\briefempfaengerindex{Schnitzler, Arthur@\textsc{Schnitzler, Arthur}!zzzGoldmann, Paul@\emph{von Paul Goldmann}!1901-05-071@{7. 5. {[}1901{]}}|(be} \toendnotes[C]{\smallbreak\pagebreak[2]} \Standort{DLA, A:Schnitzler, HS.NZ85.1.3171.}
\physDesc{Brief, 1 Blatt, 4 Seiten, 2164 Zeichen
\newline{}Handschrift: blaue Tinte, deutsche Kurrent
\newline{}Schnitzler: 1) mit Bleistift das Jahr »901.« vermerkt  2) mit rotem Buntstift vier Unterstreichungen}\toendnotes[C]{\smallbreak}\pstart
           \noindent{}\raggedleft{}{\pb}\textcolor{gray}{\textbf{DESSAUERSTRASSE 19}}\oindex{Dessauer Strasse@\textbf{Dessauer Straße}|pw}\pend
           \pstart
           Berlin\oindex{Berlin@\textbf{Berlin}|pw}, 7. Mai\pend
           \pstart\center{}Mein lieber Freund,\pend\pstart
           Ich habe bei der N. Fr. Pr.\orgindex{Neue Freie Presse@Neue Freie Presse|pw} angeregt, mich
                  \label{K_L03065-1v}\edtext{nach Macedonien\oindex{Nordmazedonien@\textbf{Nordmazedonien}|pw}\oindex{Makedonien@\textbf{Makedonien}|pw} zu ſchicken}{\lemma{\textnormal{\emph{nach … ſchicken}}}\Cendnote{\textnormal{Dazu kam es nicht.}}}\label{K_L03065-1h}. Denn ich fühle immer unabweisbarer
               das Bedürfniß, die Kraft, die ich in mir ſpüre, wieder einmal an eine ſchwere Aufgabe
               zu ſetzen, und meinem Schickſal, das mir hart und höhniſch alle Wünſche verſagt,
               wieder einmal davonzugehen. Da ich verflucht bin, nicht geliebt zu werden, will ich
               mich \strikeout{\textcolor{gray}{×}\-\textcolor{gray}{×}\-\textcolor{gray}{×}\-\textcolor{gray}{×}\-\textcolor{gray}{×}\-\textcolor{gray}{×}\-\textcolor{gray}{×}\-\textcolor{gray}{×}\-\textcolor{gray}{×}\-\textcolor{gray}{×}\-\textcolor{gray}{×}\-\textcolor{gray}{×}} durch neue Eindrücke, harte Arbeit und hoffentlich auch ein wenig Gefahr
               betäuben. \strikeout{\textcolor{gray}{Ob}} Ob man meiner Anregung Folge geben wird, iſt fraglich. Die Herren\orgindex{Neue Freie Presse@Neue Freie Presse|pwv}, die mein Talent verwalten, benutzen
               dasſelbe lieber zu {\pb}\strikeout{\textcolor{gray}{Ber}} Depeſchen über die \label{K_L03065-2v}\edtext{preuß\oindex{Preussen@\textbf{Preußen}|pw}iſche Miniſterkriſis}{\lemma{\textnormal{\emph{preußiſche Miniſterkriſis}}}\Cendnote{\textnormal{Bezug auf den von konservativer Seite kritisierten Bau des
                     Mittellandkanals\oindex{Mittellandkanal@\textbf{Mittellandkanal}|pwk} (zwischen Hannover\oindex{Hannover@\textbf{Hannover}|pwk} und der Elbe\oindex{Elbe@\textbf{Elbe}|pwk}); Anfang Mai 1901 hatte
                  dieser Konflikt zum Rücktritt des Finanzministers Johannes von Miquel\pwindex{Miquel, Johannes von 1828-02-19 – 1901-09-08@\textsc{Miquel, Johannes von} (1828-02-19 – 1901-09-08), \emph{Politiker, Finanzminister, Staatsminister}|pwk}, des Landwirtschaftsministers Ernst von Hammerstein-Loxten\pwindex{Hammerstein-Loxten, Ernst von 1827-10-02 – 1914-06-05@\textsc{Hammerstein-Loxten, Ernst von} (1827-10-02 – 1914-06-05), \emph{Politiker, Minister, Jurist}|pwk} und des
                  Handelsministers Ludwig Brefeld\pwindex{Brefeld, Ludwig 1837-03-31 – 1907-02-15@\textsc{Brefeld, Ludwig} (1837-03-31 – 1907-02-15), \emph{Politiker, Minister, Beamter}|pwk}
                  geführt.}}}\label{K_L03065-2h} und Berichten über die Lage des Berlin\oindex{Berlin@\textbf{Berlin}|pw}er \label{K_L03065-3v}\edtext{Effektenmarkt}{\lemma{\textnormal{\emph{Effektenmarkt}}}\Cendnote{\textnormal{Wertpapiermarkt}}}\label{K_L03065-3h}es.\pend
           \pstart
           Mache ich alſo nicht die Reiſe, die ich der Redaktion\orgindex{Neue Freie Presse@Neue Freie Presse|pwv} vorgeſchlagen habe, ſo werde ich Anfangs Auguſt meinen Urlaub antreten. Diesmal kann es ſich für mich
               nur um den Aufenthalt an einem Ort handeln. Es iſt wieder die leidige Geldfrage.
               Sparen habe ich während des ganzen Jahres nicht gekonnt, dann muß ich meine Mutter\pwindex{Goldmann, Clementine 1842-05-15 – 1924-02-24@\textsc{Goldmann, Clementine} (1842-05-15 – 1924-02-24)|pwv} ins \label{K_L03065-4v}\edtext{Bad}{\lemma{\textnormal{\emph{Bad}}}\Cendnote{\textnormal{gemeint war eine Kur}}}\label{K_L03065-4h} ſchicken; und iſt dies gethan, ſo
               bleiben mir im \strikeout{S} beſten Falle etwa 400 \textsc{MK}. Damit kann ich nicht ins \label{K_L03065-5v}\edtext{Engadin\oindex{Engadin@\textbf{Engadin}|pw}}{\lemma{\textnormal{\emph{Engadin}}}\Cendnote{\textnormal{Das Engadin\oindex{Engadin@\textbf{Engadin}|pwk} war eines seiner bevorzugten Reiseziele.}}}\label{K_L03065-5h} reiſen; ich hätte
               auch keine Luſt {\pb}dazu. Suche es alſo, bitte, ſo
               einzurichten, daß wir im \uline{Auguſt} uns \label{K_L03065-6v}\edtext{am Wörther See\oindex{Woerthersee@\textbf{Wörthersee}|pw}}{\lemma{\textnormal{\emph{am Wörther See}}}\Cendnote{\textnormal{Dazu kam es nicht, vgl. Paul Goldmann an Arthur Schnitzler, 26. 4. [1901].
               }}}\label{K_L03065-6h} treffen. \textsc{Olga\pwindex{Schnitzler, Olga 17.01.1882 – 13.01.1970@\textsc{Schnitzler, Olga} (17.01.1882 – 13.01.1970), \emph{Schauspielerin, Sängerin}|pw}} und \textsc{Liesl\pwindex{Steinrueck, Elisabeth 19.11.1885 – 07.04.1920@\textsc{Steinrück, Elisabeth} (19.11.1885 – 07.04.1920)|pw}} ſollen auch \label{K_L03065-7v}\edtext{hinkommen}{\lemma{\textnormal{\emph{hinkommen}}}\Cendnote{\textnormal{Olga\pwindex{Schnitzler, Olga 17.01.1882 – 13.01.1970@\textsc{Schnitzler, Olga} (17.01.1882 – 13.01.1970), \emph{Schauspielerin, Sängerin}|pwk} und Elisabeth Gussmann\pwindex{Steinrueck, Elisabeth 19.11.1885 – 07.04.1920@\textsc{Steinrück, Elisabeth} (19.11.1885 – 07.04.1920)|pwk} waren jedenfalls am 7. 8. 1901 gemeinsam mit Schnitzler\pwindex{Schnitzler, Arthur 15.05.1862 – 21.10.1931@\textsc{Schnitzler, Arthur} (15.05.1862 – 21.10.1931), \emph{Schriftsteller, Mediziner}|pwk} in Welsberg\oindex{Welsberg-Taisten@\textbf{Welsberg-Taisten}|pwk},
                  wo sich auch Goldmann\pwindex{Goldmann, Paul 31.01.1865 – 25.09.1935@\textsc{Goldmann, Paul} (31.01.1865 – 25.09.1935), \emph{Schriftsteller, Journalist}|pwk} aufhielt.}}}\label{K_L03065-7h}. Mit
                  \label{K_L03065-8v}\edtext{\textsc{Richard\pwindex{Beer-Hofmann, Richard 1866-07-11 – 1945-09-26@\textsc{Beer-Hofmann, Richard} (1866-07-11 – 1945-09-26), \emph{Schriftsteller}|pw}}}{\lemma{\textnormal{\emph{Richard}}}\Cendnote{\textnormal{Goldmann\pwindex{Goldmann, Paul 31.01.1865 – 25.09.1935@\textsc{Goldmann, Paul} (31.01.1865 – 25.09.1935), \emph{Schriftsteller, Journalist}|pwk} und Beer-Hofmann\pwindex{Beer-Hofmann, Richard 1866-07-11 – 1945-09-26@\textsc{Beer-Hofmann, Richard} (1866-07-11 – 1945-09-26), \emph{Schriftsteller}|pwk} trafen in den Tagen nach dem 22. 8. 1901 in Welsberg\oindex{Welsberg-Taisten@\textbf{Welsberg-Taisten}|pwk} zusammen.}}}\label{K_L03065-8h} treffe ich nicht gern
               zuſammen, weil ich wirklich erbittert darüber bin, daß er mir nicht eine Zeile
               geſchrieben hat, ſeit wir uns im letzten Sommer getrennt haben.\pend
           \pstart
           Was Du mir über Deinen Seelenzuſtand ſchreibſt, iſt wunderſchön. Du haſt zur
               richtigen Zeit offenbar die richtige Frau\pwindex{Schnitzler, Olga 17.01.1882 – 13.01.1970@\textsc{Schnitzler, Olga} (17.01.1882 – 13.01.1970), \emph{Schauspielerin, Sängerin}|pwv} getroffen, und ich hoffe, dieſe Liebe ſoll reiche
               Frucht tragen an dichteriſchen Werken und an Lebensglück.\pend
           \pstart
           In der Frankf. Zeit.\pwindex{?? Werk@Nicht ermittelte Verfasserinnen und Verfasser!Frankfurter Zeitung1856 – 1943@\emph{Frankfurter Zeitung} {[}1856 – 1943{]}|pw} fand ich \label{K_L03065-9v}\edtext{beifolgende {\pb}Novellette\pwindex{Rechert, Emil 24.03.1868 – 02.10.1921@\textsc{Rechert, Emil} (24.03.1868 – 02.10.1921), \emph{Schriftsteller, Rechtsanwalt}!verhasste Korrektheit. Wiener Novellette1901-05-05@\strich\emph{Die verhaßte Korrektheit. Wiener Novellette} {[}1901-05-05{]}|pwv}}{\lemma{\textnormal{\emph{beifolgende Novellette}}}\Cendnote{\textnormal{Beilage nicht erhalten; Emil Rechert\pwindex{Rechert, Emil 24.03.1868 – 02.10.1921@\textsc{Rechert, Emil} (24.03.1868 – 02.10.1921), \emph{Schriftsteller, Rechtsanwalt}|pwk}: \emph{Die verhaßte Korrektheit. Wiener Novellette}\pwindex{Rechert, Emil 24.03.1868 – 02.10.1921@\textsc{Rechert, Emil} (24.03.1868 – 02.10.1921), \emph{Schriftsteller, Rechtsanwalt}!verhasste Korrektheit. Wiener Novellette1901-05-05@\strich\emph{Die verhaßte Korrektheit. Wiener Novellette} {[}1901-05-05{]}|pwk}. In: \emph{Frankfurter Zeitung}\pwindex{?? Werk@Nicht ermittelte Verfasserinnen und Verfasser!Frankfurter Zeitung1856 – 1943@\emph{Frankfurter Zeitung} {[}1856 – 1943{]}|pwk}, Jg. 45, Nr. 124, 5. 5. 1901, Drittes Morgenblatt, S. 1–2.
               }}}\label{K_L03065-9h}. Ich finde, daß ſie feine Beobachtungen und echte Wien\oindex{Wien@\textbf{Wien}|pw}er Stimmung enthält. Wer iſt dieſer \textsc{Dr.
                     Rechert\pwindex{Rechert, Emil 24.03.1868 – 02.10.1921@\textsc{Rechert, Emil} (24.03.1868 – 02.10.1921), \emph{Schriftsteller, Rechtsanwalt}|pw}}?\pend
           \pstart
           Grüße mir die Damen \textsc{Olga\pwindex{Schnitzler, Olga 17.01.1882 – 13.01.1970@\textsc{Schnitzler, Olga} (17.01.1882 – 13.01.1970), \emph{Schauspielerin, Sängerin}|pw}} und \textsc{Liesl\pwindex{Steinrueck, Elisabeth 19.11.1885 – 07.04.1920@\textsc{Steinrück, Elisabeth} (19.11.1885 – 07.04.1920)|pw}} und ſei Du ſelbſt herzlichſt gegrüßt! {\\[\baselineskip]}Dein treuer {\\[\baselineskip]}\spacefill\mbox{Paul Goldmann.}\pend
           \leftskip=0em{}\pstart
           \noindent{}Bei der blödſinnigen Arbeitsmenge, die ich zu verrichten habe, konnte ich »Bertha Garlan\pwindex{Schnitzler, Arthur 15.05.1862 – 21.10.1931@\textsc{Schnitzler, Arthur} (15.05.1862 – 21.10.1931), \emph{Schriftsteller, Mediziner}!Frau Bertha Garlan. Roman15.1.1901 – 15.3.1901@\strich\emph{Frau Bertha Garlan. Roman} {[}15.1.1901 – 15.3.1901{]}|pw}« noch nicht leſen. \strikeout{Inzwiſchen} Meine Mutter\pwindex{Goldmann, Clementine 1842-05-15 – 1924-02-24@\textsc{Goldmann, Clementine} (1842-05-15 – 1924-02-24)|pwv} iſt ſehr entzückt davon. Inzwiſchen habe ich das
                     Buch\pwindex{Schnitzler, Arthur 15.05.1862 – 21.10.1931@\textsc{Schnitzler, Arthur} (15.05.1862 – 21.10.1931), \emph{Schriftsteller, Mediziner}!Frau Bertha Garlan. Roman15.1.1901 – 15.3.1901@\strich\emph{Frau Bertha Garlan. Roman} {[}15.1.1901 – 15.3.1901{]}|pwv} der \label{K_L03065-10v}\edtext{Frau Rechtsanwalt\pwindex{Freudenthal, Rosa 1862 – 18.06.1905@\textsc{Freudenthal, Rosa} (1862 – 18.06.1905)|pwv}}{\lemma{\textnormal{\emph{Frau Rechtsanwalt}}}\Cendnote{\textnormal{Siehe Paul Goldmann an Arthur Schnitzler, 20. 2. 1900.
                  }}}\label{K_L03065-10h} borgen müſſen, die an Gelenkrheumatismus erkrankt iſt.\pend
           
         
         \endnumbering\mylabel{h}\end{ledgroupsized}  \newcommand{\dateiname}{L03065}\newcommand{\titel}{Paul Goldmann an Arthur Schnitzler, 7. 5. [1901]}\newcommand{\editorInnen}{Martin Anton Müller und Laura Untner}%% latex-leseansicht-abspann.tex
%% Abspann für die Leseansicht.
%% Der Schalter \ifkorrekturansicht ist bereits durch den Vorspann gesetzt.

%% latex-abspann.tex
%% Gemeinsamer Abspann für Korrekturansicht und Leseansicht.
%% Setzt den Schalter \ifkorrekturansicht voraus (gesetzt in den
%% einbindenden Dateien latex-korrekturansicht-abspann.tex bzw.
%% latex-leseansicht-abspann.tex).
%% ---------------------------------------------------------------

\normalsize

% Das esempio-Environment wird nur in der Leseansicht benötigt
\ifkorrekturansicht\else
\newenvironment{esempio}[3]%
{
    \vspace{1.5ex}
    \rlap{\underline{#1}}
    \par
    \setlength{\parindent}{0cm}
    \nopagebreak
    \leftskip=#2cm
    \rightskip=#3cm
}
{
    \par
}
\fi

\doendnotes{C}
\bigskip
\vfill

\clearpage

\footnotesize

\ifkorrekturansicht
  \lohead{\textsc{register}}
\fi

% theindex-Environment neu definieren ohne reledmac
\makeatletter
\renewenvironment{theindex}{%
  \ifkorrekturansicht
    \section*{\indexname}%
  \else
    \subsubsection*{Index der erwähnten Entitäten}%
  \fi
  \setlength{\parindent}{0pt}%
  \setlength{\parskip}{0pt plus 0.3pt}%
  \let\item\@idxitem
}{%
  \ifkorrekturansicht\clearpage\fi
}
\makeatother

\IfFileExists{\jobname-pw.ind}{\input{\jobname-pw.ind}}{}

% Quellenangabe nur in der Leseansicht
\ifkorrekturansicht\else
% Fallback-Definitionen, falls die .tex-Datei \titel etc. nicht gesetzt hat
\providecommand{\titel}{}
\providecommand{\editorInnen}{}
\providecommand{\dateiname}{\jobname}

\vspace{3cm}

\vfill

\footnotesize
\textsc{Quelle}: \titel. Herausgegeben von {\editorInnen}. In: \emph{Arthur Schnitzler: Briefwechsel mit Autorinnen und Autoren}.
 Digitale Edition, https://schnitzler-briefe.acdh.oeaw.ac.at/{\dateiname}.html (Stand \today)
\fi

\end{document}


      