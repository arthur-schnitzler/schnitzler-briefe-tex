%% latex-leseansicht-vorspann.tex
%% Vorspann für die Leseansicht.
%% Lädt die gemeinsame Datei latex-vorspann.tex mit nicht gesetztem Schalter.

\newif\ifkorrekturansicht
\korrekturansichtfalse

\input{../tex-inputs/latex-vorspann}


\section[ Paul Goldmann an Arthur Schnitzler, 7. 5. {[}1901{]}]{L03065 Paul Goldmann an Arthur Schnitzler,  7. 5. [1901]}
\nopagebreak\mylabel{L03065v}
\rehead{ }\normalsize\beginnumbering\briefempfaengerindex{Schnitzler, Arthur@\textsc{Schnitzler, Arthur}!zzzGoldmann, Paul@\emph{von Paul Goldmann}!1901-05-071@{7. 5. [1901]}|(be}
\toendnotes[C]{\smallbreak\pagebreak[2]}
\correspDesc{Versand  durch Paul Goldmann am 7. 5. [1901] in Berlin
\newline{}Erhalt  durch Arthur Schnitzler im Zeitraum [8. 5. 1901
                  – 12. 5. 1901?] in Wien}\toendnotes[C]{\smallbreak}
\Standort{DLA, A:Schnitzler, HS.NZ85.1.3171.}
\physDesc{Brief, 1 Blatt, 4 Seiten, 2164 Zeichen
\newline{}Handschrift: blaue Tinte, deutsche Kurrent
\newline{}Schnitzler: 1) mit Bleistift das Jahr »901.« vermerkt  2) mit rotem Buntstift vier Unterstreichungen}\toendnotes[C]{\smallbreak}
\pstart
           \raggedleft{}{\pb}\textcolor{gray}{\textbf{DESSAUERSTRASSE 19}}\oindex{Dessauer Straße@\textbf{Dessauer Straße}, \emph{Straße}|pw}\pend
           
\pstart
           Berlin\oindex{Berlin@\textbf{Berlin}, \emph{Hauptstadt}|pw}, 7. Mai\pend
           
\pstart\center{}Mein lieber Freund,\pend\vspace{0.5em}
\pstart
           Ich habe bei der N. Fr. Pr.\orgindex{Neue Freie Presse@Neue Freie Presse|pw} angeregt, mich
                  \label{K_L03065-1v}\edtext{nach Macedonien\oindex{Nordmazedonien@\textbf{Nordmazedonien}|pw}\oindex{Makedonien@\textbf{Makedonien}, \emph{Region}|pw} zu{ }ſchicken}{\lemma{\textnormal{\emph{nach … schicken}}}\Cendnote{\textnormal{Dazu kam es nicht.}}}\label{K_L03065-1}. Denn ich fühle immer unabweisbarer
               das Bedürfniß, die Kraft, die ich in mir{ }ſpüre, wieder einmal an eine{ }ſchwere Aufgabe
               zu{ }ſetzen, und meinem Schickſal, das mir hart und höhniſch alle Wünſche verſagt,
               wieder einmal davonzugehen. Da ich verflucht bin, nicht geliebt zu werden, will ich
               mich \strikeout{\textcolor{gray}{×}\-\textcolor{gray}{×}\-\textcolor{gray}{×}\-\textcolor{gray}{×}\-\textcolor{gray}{×}\-\textcolor{gray}{×}\-\textcolor{gray}{×}\-\textcolor{gray}{×}\-\textcolor{gray}{×}\-\textcolor{gray}{×}\-\textcolor{gray}{×}\-\textcolor{gray}{×}} durch neue Eindrücke, harte Arbeit und hoffentlich auch ein wenig Gefahr
               betäuben. \strikeout{\textcolor{gray}{Ob}} Ob man meiner Anregung Folge geben wird, iſt fraglich. Die Herren\orgindex{Neue Freie Presse@Neue Freie Presse|pwv}, die mein Talent verwalten, benutzen
               dasſelbe lieber zu {\pb}\strikeout{\textcolor{gray}{Ber}} Depeſchen über die \label{K_L03065-2v}\edtext{preuß\oindex{Preußen@\textbf{Preußen}|pw}iſche Miniſterkriſis}{\lemma{\textnormal{\emph{preußische Ministerkrisis}}}\Cendnote{\textnormal{Bezug auf den von konservativer Seite kritisierten Bau des
                     Mittellandkanals\oindex{Mittellandkanal@\textbf{Mittellandkanal}, \emph{Kanal}|pwk} (zwischen Hannover\oindex{Hannover@\textbf{Hannover}|pwk} und der Elbe\oindex{Elbe@\textbf{Elbe}, \emph{Fluss}|pwk}); Anfang Mai 1901 hatte
                  dieser Konflikt zum Rücktritt des Finanzministers Johannes von Miquel\pwindex{Miquel, Johannes von 19.\,2.\,1828 Neuenhaus – 8.\,9.\,1901 Frankfurt am Main@\textsc{Miquel, Johannes von} (19.\,2.\,1828 Neuenhaus – 8.\,9.\,1901 Frankfurt am Main), \emph{Politiker, Finanzminister, Staatsminister}|pwk}, des Landwirtschaftsministers Ernst von Hammerstein-Loxten\pwindex{Hammerstein-Loxten, Ernst von 2.\,10.\,1827 Loxten – 5.\,6.\,1914 ebd.@\textsc{Hammerstein-Loxten, Ernst von} (2.\,10.\,1827 Loxten – 5.\,6.\,1914 ebd.), \emph{Politiker, Minister, Jurist}|pwk} und des
                  Handelsministers Ludwig Brefeld\pwindex{Brefeld, Ludwig 31.\,3.\,1837 Telgte – 15.\,2.\,1907 Freiburg im Breisgau@\textsc{Brefeld, Ludwig} (31.\,3.\,1837 Telgte – 15.\,2.\,1907 Freiburg im Breisgau), \emph{Politiker, Minister, Beamter}|pwk}
                  geführt.}}}\label{K_L03065-2} und Berichten über die Lage des Berlin\oindex{Berlin@\textbf{Berlin}, \emph{Hauptstadt}|pw}er \label{K_L03065-3v}\edtext{Effektenmarkt}{\lemma{\textnormal{\emph{Effektenmarkt}}}\Cendnote{\textnormal{Wertpapiermarkt}}}\label{K_L03065-3}es.\pend
           
\pstart
           Mache ich alſo nicht die Reiſe, die ich der Redaktion\orgindex{Neue Freie Presse@Neue Freie Presse|pwv} vorgeſchlagen habe,{ }ſo werde ich Anfangs Auguſt meinen Urlaub antreten. Diesmal kann es{ }ſich für mich
               nur um den Aufenthalt an einem Ort handeln. Es iſt wieder die leidige Geldfrage.
               Sparen habe ich während des ganzen Jahres nicht gekonnt, dann muß ich meine Mutter\pwindex{Goldmann, Clementine 15.\,5.\,1842 Breslau – 24.\,2.\,1924 Frankfurt am Main@\textsc{Goldmann, Clementine} (15.\,5.\,1842 Breslau – 24.\,2.\,1924 Frankfurt am Main)|pwv} ins \label{K_L03065-4v}\edtext{Bad}{\lemma{\textnormal{\emph{Bad}}}\Cendnote{\textnormal{gemeint war eine Kur}}}\label{K_L03065-4}{ }ſchicken; und iſt dies gethan,{ }ſo
               bleiben mir im \strikeout{S} beſten Falle etwa 400 \textsc{MK}. Damit kann ich nicht ins \label{K_L03065-5v}\edtext{Engadin\oindex{Engadin@\textbf{Engadin}, \emph{Tal}|pw}}{\lemma{\textnormal{\emph{Engadin}}}\Cendnote{\textnormal{Das Engadin\oindex{Engadin@\textbf{Engadin}, \emph{Tal}|pwk} war eines seiner bevorzugten Reiseziele.}}}\label{K_L03065-5} reiſen; ich hätte
               auch keine Luſt {\pb}dazu. Suche es alſo, bitte,{ }ſo
               einzurichten, daß wir im \uline{Auguſt} uns \label{K_L03065-6v}\edtext{am Wörther See\oindex{Wörthersee@\textbf{Wörthersee}, \emph{See}|pw}}{\lemma{\textnormal{\emph{am Wörther See}}}\Cendnote{\textnormal{Dazu kam es nicht, vgl. XXXX Auszeichnungsfehler: Dokument L03064 nicht gefunden.
               }}}\label{K_L03065-6} treffen. \textsc{Olga\pwindex{Schnitzler, Olga 17.\,1.\,1882 Wien – 13.\,1.\,1970 Lugano@\textsc{Schnitzler, Olga} (17.\,1.\,1882 Wien – 13.\,1.\,1970 Lugano), \emph{Schauspielerin, Sängerin}|pw}} und \textsc{Liesl\pwindex{Steinrück, Elisabeth 19.\,11.\,1885 – 7.\,4.\,1920 Partenkirchen@\textsc{Steinrück, Elisabeth} (19.\,11.\,1885 – 7.\,4.\,1920 Partenkirchen)|pw}}{ }ſollen auch \label{K_L03065-7v}\edtext{hinkommen}{\lemma{\textnormal{\emph{hinkommen}}}\Cendnote{\textnormal{Olga\pwindex{Schnitzler, Olga 17.\,1.\,1882 Wien – 13.\,1.\,1970 Lugano@\textsc{Schnitzler, Olga} (17.\,1.\,1882 Wien – 13.\,1.\,1970 Lugano), \emph{Schauspielerin, Sängerin}|pwk} und Elisabeth Gussmann\pwindex{Steinrück, Elisabeth 19.\,11.\,1885 – 7.\,4.\,1920 Partenkirchen@\textsc{Steinrück, Elisabeth} (19.\,11.\,1885 – 7.\,4.\,1920 Partenkirchen)|pwk} waren jedenfalls am 7. 8. 1901 gemeinsam mit Schnitzler in Welsberg\oindex{Welsberg-Taisten@\textbf{Welsberg-Taisten}, \emph{Verwaltungsgebiet}|pwk},
                  wo sich auch Goldmann\pwindex{Goldmann, Paul 31.\,1.\,1865 Breslau – 25.\,9.\,1935 Wien@\textsc{Goldmann, Paul} (31.\,1.\,1865 Breslau – 25.\,9.\,1935 Wien), \emph{Schriftsteller, Journalist}|pwk} aufhielt.}}}\label{K_L03065-7}. Mit
                  \label{K_L03065-8v}\edtext{\textsc{Richard\pwindex{Beer-Hofmann, Richard 11.\,7.\,1866 Wien – 26.\,9.\,1945 New York City@\textsc{Beer-Hofmann, Richard} (11.\,7.\,1866 Wien – 26.\,9.\,1945 New York City), \emph{Schriftsteller}|pw}}}{\lemma{\textnormal{\emph{Richard}}}\Cendnote{\textnormal{Goldmann\pwindex{Goldmann, Paul 31.\,1.\,1865 Breslau – 25.\,9.\,1935 Wien@\textsc{Goldmann, Paul} (31.\,1.\,1865 Breslau – 25.\,9.\,1935 Wien), \emph{Schriftsteller, Journalist}|pwk} und Beer-Hofmann\pwindex{Beer-Hofmann, Richard 11.\,7.\,1866 Wien – 26.\,9.\,1945 New York City@\textsc{Beer-Hofmann, Richard} (11.\,7.\,1866 Wien – 26.\,9.\,1945 New York City), \emph{Schriftsteller}|pwk} trafen in den Tagen nach dem 22. 8. 1901 in Welsberg\oindex{Welsberg-Taisten@\textbf{Welsberg-Taisten}, \emph{Verwaltungsgebiet}|pwk} zusammen.}}}\label{K_L03065-8} treffe ich nicht gern
               zuſammen, weil ich wirklich erbittert darüber bin, daß er mir nicht eine Zeile
               geſchrieben hat,{ }ſeit wir uns im letzten Sommer getrennt haben.\pend
           
\pstart
           Was Du mir über Deinen Seelenzuſtand{ }ſchreibſt, iſt wunderſchön. Du haſt zur
               richtigen Zeit offenbar die richtige Frau\pwindex{Schnitzler, Olga 17.\,1.\,1882 Wien – 13.\,1.\,1970 Lugano@\textsc{Schnitzler, Olga} (17.\,1.\,1882 Wien – 13.\,1.\,1970 Lugano), \emph{Schauspielerin, Sängerin}|pwv} getroffen, und ich hoffe, dieſe Liebe{ }ſoll reiche
               Frucht tragen an dichteriſchen Werken und an Lebensglück.\pend
           
\pstart
           In der Frankf. Zeit.\pwindex{Frankfurter Zeitung@\emph{Frankfurter Zeitung}|pw} fand ich \label{K_L03065-9v}\edtext{beifolgende {\pb}Novellette\pwindex{Rechert, Emil 24.\,3.\,1868 Wien – 2.\,10.\,1921 ebd.@\textsc{Rechert, Emil} (24.\,3.\,1868 Wien – 2.\,10.\,1921 ebd.), \emph{Schriftsteller, Rechtsanwalt}!verhaßte Korrektheit. Wiener Novellette@\strich\emph{Die verhaßte Korrektheit. Wiener Novellette}|pwv}}{\lemma{\textnormal{\emph{beifolgende Novellette}}}\Cendnote{\textnormal{Beilage nicht erhalten; Emil Rechert\pwindex{Rechert, Emil 24.\,3.\,1868 Wien – 2.\,10.\,1921 ebd.@\textsc{Rechert, Emil} (24.\,3.\,1868 Wien – 2.\,10.\,1921 ebd.), \emph{Schriftsteller, Rechtsanwalt}|pwk}: \emph{Die verhaßte Korrektheit. Wiener Novellette}\pwindex{Rechert, Emil 24.\,3.\,1868 Wien – 2.\,10.\,1921 ebd.@\textsc{Rechert, Emil} (24.\,3.\,1868 Wien – 2.\,10.\,1921 ebd.), \emph{Schriftsteller, Rechtsanwalt}!verhaßte Korrektheit. Wiener Novellette@\strich\emph{Die verhaßte Korrektheit. Wiener Novellette}|pwk}. In: \emph{Frankfurter Zeitung}\pwindex{Frankfurter Zeitung@\emph{Frankfurter Zeitung}|pwk}, Jg. 45, Nr. 124, 5. 5. 1901, Drittes Morgenblatt, S. 1–2.
               }}}\label{K_L03065-9}. Ich finde, daß{ }ſie feine Beobachtungen und echte Wien\oindex{Wien@\textbf{Wien}, \emph{Verwaltungsgebiet}|pw}er Stimmung enthält. Wer iſt dieſer \textsc{Dr.
                     Rechert\pwindex{Rechert, Emil 24.\,3.\,1868 Wien – 2.\,10.\,1921 ebd.@\textsc{Rechert, Emil} (24.\,3.\,1868 Wien – 2.\,10.\,1921 ebd.), \emph{Schriftsteller, Rechtsanwalt}|pw}}?\pend
           
\pstart
           Grüße mir die Damen \textsc{Olga\pwindex{Schnitzler, Olga 17.\,1.\,1882 Wien – 13.\,1.\,1970 Lugano@\textsc{Schnitzler, Olga} (17.\,1.\,1882 Wien – 13.\,1.\,1970 Lugano), \emph{Schauspielerin, Sängerin}|pw}} und \textsc{Liesl\pwindex{Steinrück, Elisabeth 19.\,11.\,1885 – 7.\,4.\,1920 Partenkirchen@\textsc{Steinrück, Elisabeth} (19.\,11.\,1885 – 7.\,4.\,1920 Partenkirchen)|pw}} und{ }ſei Du{ }ſelbſt herzlichſt gegrüßt! {\\[\baselineskip]}Dein treuer {\\[\baselineskip]}\spacefill\mbox{Paul Goldmann.}\pend
           \leftskip=0em{}
\pstart
           \noindent{}Bei der blödſinnigen Arbeitsmenge, die ich zu verrichten habe, konnte ich »Bertha Garlan\pwindex{Schnitzler, Arthur 15.\,5.\,1862 Wien – 21.\,10.\,1931 ebd.@\textsc{Schnitzler, Arthur} (15.\,5.\,1862 Wien – 21.\,10.\,1931 ebd.), \emph{Schriftsteller, Mediziner}!Frau Bertha Garlan. Roman@\strich\emph{Frau Bertha Garlan. Roman}|pw}« noch nicht leſen. \strikeout{Inzwiſchen} Meine Mutter\pwindex{Goldmann, Clementine 15.\,5.\,1842 Breslau – 24.\,2.\,1924 Frankfurt am Main@\textsc{Goldmann, Clementine} (15.\,5.\,1842 Breslau – 24.\,2.\,1924 Frankfurt am Main)|pwv} iſt{ }ſehr entzückt davon. Inzwiſchen habe ich das
                     Buch\pwindex{Schnitzler, Arthur 15.\,5.\,1862 Wien – 21.\,10.\,1931 ebd.@\textsc{Schnitzler, Arthur} (15.\,5.\,1862 Wien – 21.\,10.\,1931 ebd.), \emph{Schriftsteller, Mediziner}!Frau Bertha Garlan. Roman@\strich\emph{Frau Bertha Garlan. Roman}|pwv} der \label{K_L03065-10v}\edtext{Frau Rechtsanwalt\pwindex{Freudenthal, Rosa 1862 – 18.\,6.\,1905 Berlin@\textsc{Freudenthal, Rosa} (1862 – 18.\,6.\,1905 Berlin)|pwv}}{\lemma{\textnormal{\emph{Frau Rechtsanwalt}}}\Cendnote{\textnormal{Siehe XXXX Auszeichnungsfehler: Dokument L02905 nicht gefunden.
                  }}}\label{K_L03065-10} borgen müſſen, die an Gelenkrheumatismus erkrankt iſt.\pend
           \selectlanguage{ngerman}\endnumbering\briefempfaengerindex{Schnitzler, Arthur@\textsc{Schnitzler, Arthur}!zzzGoldmann, Paul@\emph{von Paul Goldmann}!1901-05-071@{7. 5. [1901]}|)be}\mylabel{L03065h}  \newcommand{\dateiname}{L03065}\newcommand{\titel}{Paul Goldmann an Arthur Schnitzler, 7. 5. [1901]}\newcommand{\editorInnen}{Martin Anton Müller und Laura Untner}%% latex-leseansicht-abspann.tex
%% Abspann für die Leseansicht.
%% Der Schalter \ifkorrekturansicht ist bereits durch den Vorspann gesetzt.

%% latex-abspann.tex
%% Gemeinsamer Abspann für Korrekturansicht und Leseansicht.
%% Setzt den Schalter \ifkorrekturansicht voraus (gesetzt in den
%% einbindenden Dateien latex-korrekturansicht-abspann.tex bzw.
%% latex-leseansicht-abspann.tex).
%% ---------------------------------------------------------------

\normalsize

% Das esempio-Environment wird nur in der Leseansicht benötigt
\ifkorrekturansicht\else
\newenvironment{esempio}[3]%
{
    \vspace{1.5ex}
    \rlap{\underline{#1}}
    \par
    \setlength{\parindent}{0cm}
    \nopagebreak
    \leftskip=#2cm
    \rightskip=#3cm
}
{
    \par
}
\fi

\doendnotes{C}
\bigskip
\vfill

\clearpage

\footnotesize

\ifkorrekturansicht
  \lohead{\textsc{register}}
\fi

% theindex-Environment neu definieren ohne reledmac
\makeatletter
\renewenvironment{theindex}{%
  \ifkorrekturansicht
    \section*{\indexname}%
  \else
    \subsubsection*{Index der erwähnten Entitäten}%
  \fi
  \setlength{\parindent}{0pt}%
  \setlength{\parskip}{0pt plus 0.3pt}%
  \let\item\@idxitem
}{%
  \ifkorrekturansicht\clearpage\fi
}
\makeatother

\IfFileExists{\jobname-pw.ind}{\input{\jobname-pw.ind}}{}

% Quellenangabe nur in der Leseansicht
\ifkorrekturansicht\else
% Fallback-Definitionen, falls die .tex-Datei \titel etc. nicht gesetzt hat
\providecommand{\titel}{}
\providecommand{\editorInnen}{}
\providecommand{\dateiname}{\jobname}

\vspace{3cm}

\vfill

\footnotesize
\textsc{Quelle}: \titel. Herausgegeben von {\editorInnen}. In: \emph{Arthur Schnitzler: Briefwechsel mit Autorinnen und Autoren}.
 Digitale Edition, https://schnitzler-briefe.acdh.oeaw.ac.at/{\dateiname}.html (Stand \today)
\fi

\end{document}


