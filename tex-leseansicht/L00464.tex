%% latex-leseansicht-vorspann.tex
%% Vorspann für die Leseansicht.
%% Lädt die gemeinsame Datei latex-vorspann.tex mit nicht gesetztem Schalter.

\newif\ifkorrekturansicht
\korrekturansichtfalse

\input{../tex-inputs/latex-vorspann}


         
         \renewcommand{\erwaehntePersonen}{Personen: Richard Beer-Hofmann, Paul Goldmann, Hugo von Hofmannsthal, Edgar von Karg-Bebenburg, Felix Salten}
         \renewcommand{\erwaehnteOrte}{Orte: Bad Ischl, Hodonín, Hotel und Pension Rudolfshöhe (Leopold Petter), Salesianergasse, Salzburg}
         \renewcommand{\erwaehnteWerke}{Werke: Pan}
               \section[Hugo von Hofmannsthal an Arthur Schnitzler, 17. {[}7. 1895{]}]{ Hugo von Hofmannsthal an Arthur Schnitzler, 17. {[}7. 1895{]}}\nopagebreak\mylabel{v}\rehead{ }\begin{ledgroupsized}[t]{13cm}\normalsize\beginnumbering \toendnotes[C]{\smallbreak\pagebreak[2]} \Standort{CUL, Schnitzler, B 43.}
\physDesc{Brief, 1 Blatt, 3 Seiten, 1457 Zeichen
\newline{}Handschrift: schwarze Tinte, deutsche Kurrent
\newline{}Schnitzler: mit Bleistift Datum der Beantwortung vermerkt: »7 95« und nummeriert: »73« }\buchAbdrucke{\weitereDrucke{1) Hugo von Hofmannsthal: \emph{Briefe. 1890–1901}. Berlin: \emph{S. Fischer} 1935, S. 152–153.} \weitereDrucke{2) Hugo von Hofmannsthal, Arthur Schnitzler: \emph{Briefwechsel}. Hg. Therese Nickl und Heinrich Schnitzler. Frankfurt am Main: \emph{S. Fischer} 1964, S. 56.} }\toendnotes[C]{\smallbreak}\pstart
           \raggedleft{}{\pb}Göding\oindex{Hodonín@\textbf{Hodonín}|pw}, 17\textsuperscript{ten}{ }11 Uhr. \pend
           \pstart
           \raggedleft{}\textcolor{gray}{\textbf{\strikeout{Salesianergasse 12\oindex{Salesianergasse@\textbf{Salesianergasse}|pw}}}}\pend
           \pstart
           es macht mir eine merkwürdige Freude, dieſem Brief in Gedanken nachzugehen. Ich habe
               voriges Jahr ſehr glücklich vor mich hingelebt, von den \label{K_L00464-1v}\edtext{Tagen in Salzburg\oindex{Salzburg@\textbf{Salzburg}|pw}}{\lemma{\textnormal{\emph{Tagen in Salzburg}}}\Cendnote{\textnormal{siehe A. S.: \emph{Tagebuch}, 2. 8. 1894}}}\label{K_L00464-1h} bis in den September
               fühle ich im Zurückdenken das complexe Glück von Bewegung, Blick und Gedanken,
               ſich-Hergeben und ſich-Behalten, Mitleid, Verliebtheit und Einſamkeit, dunklen
               Gewittern am Abend und blaßgelben lautloſen Blitzen in der Nacht; am Anfang mehr die
               Melancholie der kleinen Eiſenbahn mit dem Roth vom Sonnenuntergang auf den
               Kupfernägeln der Bänke, mit den geſchminkten und lautredenden {\pb}Frauen in allen Stationen, mit dem
               plötzlichen Dunkel- und Kaltwerden in dem kleinen Tunnel und gleich darauf den
               harmloſen von nichts wiſſenden Bauernhäuſern und kleinen Gärten; am Ende mehr die
               ſtundenlangen Geſpräche in der Nacht im Regen, im Wald und auf der weißen naſſen
               Landſtraße mit Edgar\pwindex{Karg-Bebenburg, Edgar von 22.12.1872 – 23.06.1905@\textsc{Karg-Bebenburg, Edgar von} (22.12.1872 – 23.06.1905), \emph{Militär}|pw} und das ſo ſtarke
               aufgeregte Fühlen von ſein und meinem Leben wie in einem.\pend
           \pstart
           Als ein beſonders merkwürdiger \label{K_L00464-2v}\edtext{Tag}{\lemma{\textnormal{\emph{Tag}}}\Cendnote{\textnormal{siehe A. S.: \emph{Tagebuch}, 3. 9. 1894}}}\label{K_L00464-2h} erſcheint mir der, wo wir mit Goldmann\pwindex{Goldmann, Paul 31.01.1865 – 25.09.1935@\textsc{Goldmann, Paul} (31.01.1865 – 25.09.1935), \emph{Schriftsteller, Journalist}|pw}
               vor ſeiner Abreiſe zuerſt beim Leopold\oindex{Hotel und Pension Rudolfshoehe (Leopold Petter)@\textbf{Hotel und Pension Rudolfshöhe (Leopold Petter)}|pw} waren und
               dann ein großes Gewitter gekommen iſt. Ich kann aber nicht finden, warum.\pend
           \pstart
           {\pb}Heute nachmittag gehe ich auf
               Patrouille und bleib über Nacht aus. Morgen wenn ich zurückkomm und gebadet hab, wird
               der \label{K_L00464-3v}\edtext{Pan\pwindex{Pan1895 – 1915@\emph{Pan} {[}1895 – 1915{]}|pw} daliegen, den mir der Salten\pwindex{Salten, Felix 06.09.1869 – 08.10.1945@\textsc{Salten, Felix} (06.09.1869 – 08.10.1945), \emph{Schriftsteller, Journalist}|pw}
                  geſchickt}{\lemma{\textnormal{\emph{Pan … geſchickt}}}\Cendnote{\textnormal{siehe Felix Salten an Arthur Schnitzler, 16. 7. [1895]}}}\label{K_L00464-3h} hat. An ſolchen
               kleinen Freuden bringe ich mich wie an Springſtöcken von Stein zu Stein über dieſe
               Öde hinüber.\pend
           \pstart
           Adieu, ſchreiben Sie und Richard\pwindex{Beer-Hofmann, Richard 1866-07-11 – 1945-09-26@\textsc{Beer-Hofmann, Richard} (1866-07-11 – 1945-09-26), \emph{Schriftsteller}|pw} mir doch
               bald.{\\[\baselineskip]} Ihr{\\[\baselineskip]}\spacefill\mbox{Hugo.}\pend
           \leftskip=0em{}
         
         \endnumbering\mylabel{h}\end{ledgroupsized}  \newcommand{\dateiname}{L00464}\newcommand{\titel}{Hugo von Hofmannsthal an Arthur Schnitzler, 17. [7. 1895]}\newcommand{\editorInnen}{Martin Anton Müller und Gerd-Hermann Susen}%% latex-leseansicht-abspann.tex
%% Abspann für die Leseansicht.
%% Der Schalter \ifkorrekturansicht ist bereits durch den Vorspann gesetzt.

%% latex-abspann.tex
%% Gemeinsamer Abspann für Korrekturansicht und Leseansicht.
%% Setzt den Schalter \ifkorrekturansicht voraus (gesetzt in den
%% einbindenden Dateien latex-korrekturansicht-abspann.tex bzw.
%% latex-leseansicht-abspann.tex).
%% ---------------------------------------------------------------

\normalsize

% Das esempio-Environment wird nur in der Leseansicht benötigt
\ifkorrekturansicht\else
\newenvironment{esempio}[3]%
{
    \vspace{1.5ex}
    \rlap{\underline{#1}}
    \par
    \setlength{\parindent}{0cm}
    \nopagebreak
    \leftskip=#2cm
    \rightskip=#3cm
}
{
    \par
}
\fi

\doendnotes{C}
\bigskip
\vfill

\clearpage

\footnotesize

\ifkorrekturansicht
  \lohead{\textsc{register}}
\fi

% theindex-Environment neu definieren ohne reledmac
\makeatletter
\renewenvironment{theindex}{%
  \ifkorrekturansicht
    \section*{\indexname}%
  \else
    \subsubsection*{Index der erwähnten Entitäten}%
  \fi
  \setlength{\parindent}{0pt}%
  \setlength{\parskip}{0pt plus 0.3pt}%
  \let\item\@idxitem
}{%
  \ifkorrekturansicht\clearpage\fi
}
\makeatother

\IfFileExists{\jobname-pw.ind}{\input{\jobname-pw.ind}}{}

% Quellenangabe nur in der Leseansicht
\ifkorrekturansicht\else
% Fallback-Definitionen, falls die .tex-Datei \titel etc. nicht gesetzt hat
\providecommand{\titel}{}
\providecommand{\editorInnen}{}
\providecommand{\dateiname}{\jobname}

\vspace{3cm}

\vfill

\footnotesize
\textsc{Quelle}: \titel. Herausgegeben von {\editorInnen}. In: \emph{Arthur Schnitzler: Briefwechsel mit Autorinnen und Autoren}.
 Digitale Edition, https://schnitzler-briefe.acdh.oeaw.ac.at/{\dateiname}.html (Stand \today)
\fi

\end{document}


      