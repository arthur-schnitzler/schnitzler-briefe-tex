%% latex-korrekturansicht-vorspann.tex
%% Vorspann für die Korrekturansicht.
%% Lädt die gemeinsame Datei latex-vorspann.tex mit gesetztem Schalter.

\newif\ifkorrekturansicht
\korrekturansichttrue

\input{../tex-inputs/latex-vorspann}


\section[Hugo von Hofmannsthal an Arthur Schnitzler, 17. {[}7. 1895{]}]{L00464 Hugo von Hofmannsthal an Arthur Schnitzler, 17. {[}7. 1895{]}}
\nopagebreak\mylabel{L00464v}
\rehead{ }\normalsize\beginnumbering\briefempfaengerindex{Schnitzler, Arthur@\textsc{Schnitzler, Arthur}!zzzHofmannsthal, Hugo von@\emph{von Hugo von Hofmannsthal}!1895-07-172@{17. {[}7. 1895{]}}|(be}
\toendnotes[C]{\smallbreak\pagebreak[2]}\Standort{CUL, Schnitzler, B 43.}
\physDesc{Brief, 1 Blatt, 3 Seiten, 1457 Zeichen
\newline{}Handschrift: schwarze Tinte, deutsche Kurrent
\newline{}Schnitzler: mit Bleistift Datum der Beantwortung vermerkt: »7 95« und nummeriert: »73« }
\buchAbdrucke{\weitereDrucke{1) Hugo von Hofmannsthal: \emph{Briefe. 1890–1901}. Berlin: \emph{S. Fischer} 1935, S. 152–153.} \weitereDrucke{2) Hugo von Hofmannsthal, Arthur Schnitzler: \emph{Briefwechsel}. Frankfurt am Main: \emph{S. Fischer} 1964, S. 56.} }\toendnotes[C]{\smallbreak}
\pstart
           \raggedleft{}{\pb}Göding\oindex{Hodonín@\textbf{Hodonín}, \emph{P.PPL}|pw}, 17\textsuperscript{ten}{ }11 Uhr. \pend
           
\pstart
           \raggedleft{}\textcolor{gray}{\textbf{\strikeout{Salesianergasse 12\oindex{Salesianergasse 12@\textbf{Salesianergasse 12}, \emph{Wohngebäude (K.WHS)}|pw}}}}\pend
           \vspace{0.5em}
\pstart
           es macht mir eine merkwürdige Freude, dieſem Brief in Gedanken nachzugehen. Ich habe
               voriges Jahr ſehr glücklich vor mich hingelebt, von den \label{K_L00464-1v}\edtext{Tagen in Salzburg\oindex{Salzburg@\textbf{Salzburg}, \emph{A.ADM2}|pw}}{\lemma{\textnormal{\emph{Tagen in Salzburg}}}\Cendnote{\textnormal{Siehe A. S.: \emph{Tagebuch}, 2. 8. 1894.
               }}}\label{K_L00464-1} bis in den September
               fühle ich im Zurückdenken das complexe Glück von Bewegung, Blick und Gedanken,
               ſich-Hergeben und ſich-Behalten, Mitleid, Verliebtheit und Einſamkeit, dunklen
               Gewittern am Abend und blaßgelben lautloſen Blitzen in der Nacht; am Anfang mehr die
               Melancholie der kleinen Eiſenbahn mit dem Roth vom Sonnenuntergang auf den
               Kupfernägeln der Bänke, mit den geſchminkten und lautredenden {\pb}Frauen in allen Stationen, mit dem
               plötzlichen Dunkel- und Kaltwerden in dem kleinen Tunnel und gleich darauf den
               harmloſen von nichts wiſſenden Bauernhäuſern und kleinen Gärten; am Ende mehr die
               ſtundenlangen Geſpräche in der Nacht im Regen, im Wald und auf der weißen naſſen
               Landſtraße mit Edgar\pwindex{Karg-Bebenburg, Edgar von 22.12.1872 – 23.06.1905@\textsc{Karg-Bebenburg, Edgar von} (22.12.1872 – 23.06.1905), \emph{Militär/Militärin}|pw} und das ſo ſtarke
               aufgeregte Fühlen von ſein und meinem Leben wie in einem.\pend
           
\pstart
           Als ein beſonders merkwürdiger \label{K_L00464-2v}\edtext{Tag}{\lemma{\textnormal{\emph{Tag}}}\Cendnote{\textnormal{Siehe A. S.: \emph{Tagebuch}, 3. 9. 1894.
               }}}\label{K_L00464-2} erſcheint mir der, wo wir mit Goldmann\pwindex{Goldmann, Paul 31.01.1865 – 25.09.1935@\textsc{Goldmann, Paul} (31.01.1865 – 25.09.1935), \emph{Schriftsteller/Schriftstellerin, Journalist/Journalistin}|pw}
               vor ſeiner Abreiſe zuerſt beim Leopold\oindex{Hotel und Pension Rudolfshoehe (Leopold Petter)@\textbf{Hotel und Pension Rudolfshöhe (Leopold Petter)}, \emph{Hotel (K.HTL)}|pw} waren und
               dann ein großes Gewitter gekommen iſt. Ich kann aber nicht finden, warum.\pend
           
\pstart
           {\pb}Heute nachmittag gehe ich auf
               Patrouille und bleib über Nacht aus. Morgen wenn ich zurückkomm und gebadet hab, wird
               der \label{K_L00464-3v}\edtext{Pan\pwindex{Pan@\emph{Pan}|pw} daliegen, den mir der Salten\pwindex{Salten, Felix 06.09.1869 – 08.10.1945@\textsc{Salten, Felix} (06.09.1869 – 08.10.1945), \emph{Schriftsteller/Schriftstellerin, Journalist/Journalistin, Chefredakteur/Chefredakteurin}|pw}
                  geſchickt}{\lemma{\textnormal{\emph{Pan … geſchickt}}}\Cendnote{\textnormal{Siehe Felix Salten an Arthur Schnitzler, 16. 7. [1895].
               }}}\label{K_L00464-3} hat. An ſolchen
               kleinen Freuden bringe ich mich wie an Springſtöcken von Stein zu Stein über dieſe
               Öde hinüber.\pend
           
\pstart
           Adieu, ſchreiben Sie und Richard\pwindex{Beer-Hofmann, Richard 1866-07-11 – 1945-09-26@\textsc{Beer-Hofmann, Richard} (1866-07-11 – 1945-09-26), \emph{Schriftsteller/Schriftstellerin}|pw} mir doch
               bald.{\\[\baselineskip]} Ihr{\\[\baselineskip]}\spacefill\mbox{Hugo.}\pend
           \leftskip=0em{}\selectlanguage{ngerman}\endnumbering\briefempfaengerindex{Schnitzler, Arthur@\textsc{Schnitzler, Arthur}!zzzHofmannsthal, Hugo von@\emph{von Hugo von Hofmannsthal}!1895-07-172@{17. {[}7. 1895{]}}|)be}\mylabel{L00464h}  \normalsize

\doendnotes{C}
\bigskip
\vfill

\clearpage

\footnotesize

\lohead{\textsc{register}}

% Definiere theindex-Environment komplett neu ohne reledmac
\makeatletter
\renewenvironment{theindex}{%
  \section*{\indexname}%
  \setlength{\parindent}{0pt}%
  \setlength{\parskip}{0pt plus 0.3pt}%
  \let\item\@idxitem
}{%
  \clearpage
}
\makeatother

\IfFileExists{\jobname-pw.ind}{\input{\jobname-pw.ind}}{}

\end{document}

      