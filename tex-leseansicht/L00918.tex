%% latex-korrekturansicht-vorspann.tex
%% Vorspann für die Korrekturansicht.
%% Lädt die gemeinsame Datei latex-vorspann.tex mit gesetztem Schalter.

\newif\ifkorrekturansicht
\korrekturansichttrue

\input{../tex-inputs/latex-vorspann}


\section[Hermann Bahr: Widmungsexemplar Wenn es Euch gefällt für Arthur Schnitzler, 21. 5. 1899]{L00918 Hermann Bahr: Widmungsexemplar Wenn es Euch gefällt für Arthur
               Schnitzler, 21. 5. 1899}
\nopagebreak\mylabel{L00918v}
\rehead{ }\normalsize\beginnumbering\briefempfaengerindex{Schnitzler, Arthur@\textsc{Schnitzler, Arthur}!zzzBahr, Hermann@\emph{von Hermann Bahr}!1899-05-211@{21. 5. 1899}|(be}
\toendnotes[C]{\smallbreak\pagebreak[2]}\Standort{DLA, G:Schnitzler, Arthur (Sammlung Heinrich Schnitzler).}
\physDesc{Widmung am Vorsatzblatt, 75 Zeichen
\newline{}Handschrift: schwarze Tinte, deutsche Kurrent
\newline{}Ordnung: bei der Enteignung des Exemplars 1938 von
                                 unbekannter Hand mit Bleistift ergänzte Information:
                                    »Dublette zu 101.290-B« }
\pstart
           \noindent{}{\pb}Seinem lieben Freunde{\\}Arthur
               Schnitzler\pend
           
\pstart
           herzlichſt{\\[\baselineskip]}\spacefill\mbox{HermannBahr}\pend
           \leftskip=0em{}
\pstart
           \noindent{}Pfingsten 1899\pend
           
\pstart
           \centering{}\textcolor{gray}{\textbf{\textbf{Wenn es Euch gefällt}\pwindex{Wenn es euch gefaellt. Wiener Revue in drei Bildern und einem Vorspiel@\emph{Wenn es euch gefällt. Wiener Revue in drei Bildern und einem Vorspiel}|pw}.}}\pend
           \selectlanguage{ngerman}\vspace{1em}{\vspace{1\baselineskip}}
\pstart
           \centering{}{\pb}\textcolor{gray}{\textbf{\textbf{Wenn es Euch gefällt}\pwindex{Wenn es euch gefaellt. Wiener Revue in drei Bildern und einem Vorspiel@\emph{Wenn es euch gefällt. Wiener Revue in drei Bildern und einem Vorspiel}|pw}.}}\pend
           
\pstart
           \centering{}\textcolor{gray}{\textbf{\textbf{Wien\oindex{Wien@\textbf{Wien}, \emph{A.ADM2}|pw}er Revue}}}\pend
           
\pstart
           \centering{}\textcolor{gray}{\textbf{in}}\pend
           
\pstart
           \centering{}\textcolor{gray}{\textbf{\textbf{drei Bildern und einem Vorſpiel}}}\pend
           
\pstart
           \centering{}\textcolor{gray}{\textbf{von}}\pend
           
\pstart
           \centering{}\textcolor{gray}{\textbf{\textbf{Hermann Bahr} und \textbf{C. Karlweis}\pwindex{Karlweis, Carl 23.11.1850 – 27.10.1901@\textsc{Karlweis, Carl} (23.11.1850 – 27.10.1901), \emph{Schriftsteller/Schriftstellerin}|pw}.}}\pend
           {\vspace{1\baselineskip}}
\pstart
           \centering{}\textcolor{gray}{\textbf{\textbf{Wien}\oindex{Wien@\textbf{Wien}, \emph{A.ADM2}|pw}.}}\pend
           
\pstart
           \centering{}\textcolor{gray}{\textbf{\so{Verlag von Carl Konegen}\orgindex{Carl Konegen@Carl Konegen|pw}}}\pend
           
\pstart
           \centering{}\textcolor{gray}{\textbf{1899.}}\pend
           \selectlanguage{ngerman}\endnumbering\briefempfaengerindex{Schnitzler, Arthur@\textsc{Schnitzler, Arthur}!zzzBahr, Hermann@\emph{von Hermann Bahr}!1899-05-211@{21. 5. 1899}|)be}\mylabel{L00918h}  \normalsize

\doendnotes{C}
\bigskip
\vfill

\clearpage

\footnotesize

\lohead{\textsc{register}}

% Definiere theindex-Environment komplett neu ohne reledmac
\makeatletter
\renewenvironment{theindex}{%
  \section*{\indexname}%
  \setlength{\parindent}{0pt}%
  \setlength{\parskip}{0pt plus 0.3pt}%
  \let\item\@idxitem
}{%
  \clearpage
}
\makeatother

\IfFileExists{\jobname-pw.ind}{\input{\jobname-pw.ind}}{}

\end{document}

      