%% latex-leseansicht-vorspann.tex
%% Vorspann für die Leseansicht.
%% Lädt die gemeinsame Datei latex-vorspann.tex mit nicht gesetztem Schalter.

\newif\ifkorrekturansicht
\korrekturansichtfalse

\input{../tex-inputs/latex-vorspann}


         
         \renewcommand{\erwaehntePersonen}{Personen: Carl Karlweis}
         \renewcommand{\erwaehnteInstitutionen}{Institutionen: Carl Konegen}
         \renewcommand{\erwaehnteOrte}{Orte: Wien}
         \renewcommand{\erwaehnteWerke}{Werke: Wenn es euch gefällt. Wiener Revue in drei Bildern und einem Vorspiel}
               \section[Hermann Bahr: Widmungsexemplar Wenn es Euch gefällt für Arthur Schnitzler, 21. 5. 1899]{ Hermann Bahr: Widmungsexemplar Wenn es Euch gefällt für Arthur
               Schnitzler, 21. 5. 1899}\nopagebreak\mylabel{v}\rehead{ }\begin{ledgroupsized}[t]{13cm}\normalsize\beginnumbering \toendnotes[C]{\smallbreak\pagebreak[2]} \Standort{DLA, G:Schnitzler, Arthur (Sammlung Heinrich Schnitzler).}
\physDesc{Widmung am Vorsatzblatt, 75 Zeichen
\newline{}Handschrift: schwarze Tinte, deutsche Kurrent
\newline{}Ordnung: bei der Enteignung des Exemplars 1938 von
                                 unbekannter Hand mit Bleistift ergänzte Information:
                                    »Dublette zu 101.290-B« }\pstart
           \noindent{}{\pb}Seinem lieben Freunde{\\}Arthur
               Schnitzler\pend
           \pstart
           herzlichſt{\\[\baselineskip]}\spacefill\mbox{HermannBahr}\pend
           \leftskip=0em{}\pstart
           \noindent{}Pfingsten 1899\pend
           \pstart
           \centering{}\textcolor{gray}{\textbf{\textbf{Wenn es Euch gefällt}\pwindex{Bahr, Hermann 19.07.1863 – 15.01.1934@\textsc{Bahr, Hermann} (19.07.1863 – 15.01.1934), \emph{Schriftsteller, Kritiker}!Wenn es euch gefaellt. Wiener Revue in drei Bildern und einem Vorspiel1899@\strich\emph{Wenn es euch gefällt. Wiener Revue in drei Bildern und einem Vorspiel} {[}1899{]}|pw}\pwindex{Karlweis, Carl 23.11.1850 – 27.10.1901@\textsc{Karlweis, Carl} (23.11.1850 – 27.10.1901), \emph{Schriftsteller}!Wenn es euch gefaellt. Wiener Revue in drei Bildern und einem Vorspiel1899@\strich\emph{Wenn es euch gefällt. Wiener Revue in drei Bildern und einem Vorspiel} {[}1899{]}|pw}.}}\pend
           {\bigskip}\pstart
           \noindent{}\centering{}{\pb}\textcolor{gray}{\textbf{\textbf{Wenn es Euch gefällt}\pwindex{Bahr, Hermann 19.07.1863 – 15.01.1934@\textsc{Bahr, Hermann} (19.07.1863 – 15.01.1934), \emph{Schriftsteller, Kritiker}!Wenn es euch gefaellt. Wiener Revue in drei Bildern und einem Vorspiel1899@\strich\emph{Wenn es euch gefällt. Wiener Revue in drei Bildern und einem Vorspiel} {[}1899{]}|pw}\pwindex{Karlweis, Carl 23.11.1850 – 27.10.1901@\textsc{Karlweis, Carl} (23.11.1850 – 27.10.1901), \emph{Schriftsteller}!Wenn es euch gefaellt. Wiener Revue in drei Bildern und einem Vorspiel1899@\strich\emph{Wenn es euch gefällt. Wiener Revue in drei Bildern und einem Vorspiel} {[}1899{]}|pw}.}}\pend
           \pstart
           \noindent{}\centering{}\textcolor{gray}{\textbf{\textbf{Wien\oindex{Wien@\textbf{Wien}|pw}er Revue}}}\pend
           \pstart
           \noindent{}\centering{}\textcolor{gray}{\textbf{in}}\pend
           \pstart
           \noindent{}\centering{}\textcolor{gray}{\textbf{\textbf{drei Bildern und einem Vorſpiel}}}\pend
           \pstart
           \noindent{}\centering{}\textcolor{gray}{\textbf{von}}\pend
           \pstart
           \noindent{}\centering{}\textcolor{gray}{\textbf{\textbf{Hermann Bahr} und \textbf{C. Karlweis}\pwindex{Karlweis, Carl 23.11.1850 – 27.10.1901@\textsc{Karlweis, Carl} (23.11.1850 – 27.10.1901), \emph{Schriftsteller}|pw}.}}\pend
           {\bigskip}\pstart
           \noindent{}\centering{}\textcolor{gray}{\textbf{\textbf{Wien}\oindex{Wien@\textbf{Wien}|pw}.}}\pend
           \pstart
           \noindent{}\centering{}\textcolor{gray}{\textbf{\so{Verlag von Carl Konegen}\orgindex{Carl Konegen@Carl Konegen|pw}}}\pend
           \pstart
           \noindent{}\centering{}\textcolor{gray}{\textbf{1899.}}\pend
           
         
         \endnumbering\mylabel{h}\end{ledgroupsized}  \newcommand{\dateiname}{L00918}\newcommand{\titel}{Hermann Bahr: Widmungsexemplar Wenn es Euch gefällt für Arthur Schnitzler, 21. 5. 1899}\newcommand{\editorInnen}{ Martin Anton Müller und  Gerd Hermann Susen}%% latex-leseansicht-abspann.tex
%% Abspann für die Leseansicht.
%% Der Schalter \ifkorrekturansicht ist bereits durch den Vorspann gesetzt.

%% latex-abspann.tex
%% Gemeinsamer Abspann für Korrekturansicht und Leseansicht.
%% Setzt den Schalter \ifkorrekturansicht voraus (gesetzt in den
%% einbindenden Dateien latex-korrekturansicht-abspann.tex bzw.
%% latex-leseansicht-abspann.tex).
%% ---------------------------------------------------------------

\normalsize

% Das esempio-Environment wird nur in der Leseansicht benötigt
\ifkorrekturansicht\else
\newenvironment{esempio}[3]%
{
    \vspace{1.5ex}
    \rlap{\underline{#1}}
    \par
    \setlength{\parindent}{0cm}
    \nopagebreak
    \leftskip=#2cm
    \rightskip=#3cm
}
{
    \par
}
\fi

\doendnotes{C}
\bigskip
\vfill

\clearpage

\footnotesize

\ifkorrekturansicht
  \lohead{\textsc{register}}
\fi

% theindex-Environment neu definieren ohne reledmac
\makeatletter
\renewenvironment{theindex}{%
  \ifkorrekturansicht
    \section*{\indexname}%
  \else
    \subsubsection*{Index der erwähnten Entitäten}%
  \fi
  \setlength{\parindent}{0pt}%
  \setlength{\parskip}{0pt plus 0.3pt}%
  \let\item\@idxitem
}{%
  \ifkorrekturansicht\clearpage\fi
}
\makeatother

\IfFileExists{\jobname-pw.ind}{\input{\jobname-pw.ind}}{}

% Quellenangabe nur in der Leseansicht
\ifkorrekturansicht\else
% Fallback-Definitionen, falls die .tex-Datei \titel etc. nicht gesetzt hat
\providecommand{\titel}{}
\providecommand{\editorInnen}{}
\providecommand{\dateiname}{\jobname}

\vspace{3cm}

\vfill

\footnotesize
\textsc{Quelle}: \titel. Herausgegeben von {\editorInnen}. In: \emph{Arthur Schnitzler: Briefwechsel mit Autorinnen und Autoren}.
 Digitale Edition, https://schnitzler-briefe.acdh.oeaw.ac.at/{\dateiname}.html (Stand \today)
\fi

\end{document}


      