%% latex-leseansicht-vorspann.tex
%% Vorspann für die Leseansicht.
%% Lädt die gemeinsame Datei latex-vorspann.tex mit nicht gesetztem Schalter.

\newif\ifkorrekturansicht
\korrekturansichtfalse

\input{../tex-inputs/latex-vorspann}


         
         \renewcommand{\erwaehntePersonen}{Personen: Richard Beer-Hofmann, Paula Beer-Hofmann, Max Eugen Burckhard, Hugo von Hofmannsthal, Olga Schnitzler}
         \renewcommand{\erwaehnteOrte}{Orte: Lueg am Wolfgangsee, Salzburg, St. Gilgen, Wien}
         \renewcommand{\erwaehnteWerke}{}
               \section[Arthur Schnitzler an Hugo von Hofmannsthal, 16. 9. 1904]{ Arthur Schnitzler an Hugo von Hofmannsthal, 16. 9. 1904}\nopagebreak\mylabel{v}\rehead{ }\begin{ledgroupsized}[t]{13cm}\normalsize\beginnumbering \toendnotes[C]{\smallbreak\pagebreak[2]} \Standort{FDH, Hs-30885,114.}
\physDesc{Brief, 1 Blatt, 3 Seiten, 776 Zeichen
\newline{}Handschrift: Bleistift, deutsche Kurrent}\buchAbdrucke{\weitereDrucke{Hugo von Hofmannsthal, Arthur Schnitzler: \emph{Briefwechsel}. Hg. Therese Nickl und Heinrich Schnitzler. Frankfurt am Main: \emph{S. Fischer} 1964, S. 201.} }\toendnotes[C]{\smallbreak}\pstart
           \raggedleft{}{\pb}16. 9. 904{\\}\textsc{Lueg a Wolfg}ſee\oindex{Lueg am Wolfgangsee@\textbf{Lueg am Wolfgangsee}|pw}\pend
           \pstart
           lieber Hugo, bis heute ſind wir dageblieben, ſeit vorgeſtern arges
               Regenwetter, heute Nm fährt Richard\pwindex{Beer-Hofmann, Richard 1866-07-11 – 1945-09-26@\textsc{Beer-Hofmann, Richard} (1866-07-11 – 1945-09-26), \emph{Schriftsteller}|pw} vorbei; wir ſteigen zu ihm ein u bleiben noch ein paar Tage in Salzburg\oindex{Salzburg@\textbf{Salzburg}|pw}. Da{\geminationn}
               wahrſcheinlich direct Wien\oindex{Wien@\textbf{Wien}|pw}. Gearbeitet ſo gut wie
               nichts, aber große {\pb}Sehnſucht danach. Mit Burckhard\pwindex{Burckhard, Max Eugen 14.07.1854 – 16.03.1912@\textsc{Burckhard, Max Eugen} (14.07.1854 – 16.03.1912), \emph{Schriftsteller, Wissenschaftler, Theaterleiter}|pw} ein paar ſehr angenehme Stunden. Das
               Rad ununterbrochen ſchwer krank – es zeigte sich daſs die Tretkurbel u noch einiges
               andre total hin war. Bin \uline{ein} Mal von \textsc{St. Gilgen}\oindex{St. Gilgen@\textbf{St. Gilgen}|pw} nach \textsc{Lueg}\oindex{Lueg am Wolfgangsee@\textbf{Lueg am Wolfgangsee}|pw} gefahren. Jetzt iſt es ganz in Ordnung und wird wahrſcheinlich auf der
               Eiſenbahn zer{\pb}trümmert werden. Ihre (eine) Karte
               erhalten. Ob Sie ſchönes Wetter auf der Tour gehabt haben? Eine neulich gekommene
               Karte leg ich bei.\pend
           \pstart
           Laſſen Sie ſehr bald nach Wien\oindex{Wien@\textbf{Wien}|pw} einiges
               vernehmen.\pend
           \pstart
           Wir grüßen Sie Beide\pwindex{Schnitzler, Olga 17.01.1882 – 13.01.1970@\textsc{Schnitzler, Olga} (17.01.1882 – 13.01.1970), \emph{Schauspielerin, Sängerin}|pwv}{ }Beide\pwindex{Beer-Hofmann, Paula 25.02.1879 – 30.10.1939@\textsc{Beer-Hofmann, Paula} (25.02.1879 – 30.10.1939)|pwv}.\pend
           \pstart
           Herzlichſt Ihr{\\[\baselineskip]}\spacefill\mbox{A.}\pend
           \leftskip=0em{}
         
         \endnumbering\mylabel{h}\end{ledgroupsized}  \newcommand{\dateiname}{L01446}\newcommand{\titel}{Arthur Schnitzler an Hugo von Hofmannsthal, 16. 9. 1904}\newcommand{\editorInnen}{Martin Anton Müller und Gerd-Hermann Susen}%% latex-leseansicht-abspann.tex
%% Abspann für die Leseansicht.
%% Der Schalter \ifkorrekturansicht ist bereits durch den Vorspann gesetzt.

%% latex-abspann.tex
%% Gemeinsamer Abspann für Korrekturansicht und Leseansicht.
%% Setzt den Schalter \ifkorrekturansicht voraus (gesetzt in den
%% einbindenden Dateien latex-korrekturansicht-abspann.tex bzw.
%% latex-leseansicht-abspann.tex).
%% ---------------------------------------------------------------

\normalsize

% Das esempio-Environment wird nur in der Leseansicht benötigt
\ifkorrekturansicht\else
\newenvironment{esempio}[3]%
{
    \vspace{1.5ex}
    \rlap{\underline{#1}}
    \par
    \setlength{\parindent}{0cm}
    \nopagebreak
    \leftskip=#2cm
    \rightskip=#3cm
}
{
    \par
}
\fi

\doendnotes{C}
\bigskip
\vfill

\clearpage

\footnotesize

\ifkorrekturansicht
  \lohead{\textsc{register}}
\fi

% theindex-Environment neu definieren ohne reledmac
\makeatletter
\renewenvironment{theindex}{%
  \ifkorrekturansicht
    \section*{\indexname}%
  \else
    \subsubsection*{Index der erwähnten Entitäten}%
  \fi
  \setlength{\parindent}{0pt}%
  \setlength{\parskip}{0pt plus 0.3pt}%
  \let\item\@idxitem
}{%
  \ifkorrekturansicht\clearpage\fi
}
\makeatother

\IfFileExists{\jobname-pw.ind}{\input{\jobname-pw.ind}}{}

% Quellenangabe nur in der Leseansicht
\ifkorrekturansicht\else
% Fallback-Definitionen, falls die .tex-Datei \titel etc. nicht gesetzt hat
\providecommand{\titel}{}
\providecommand{\editorInnen}{}
\providecommand{\dateiname}{\jobname}

\vspace{3cm}

\vfill

\footnotesize
\textsc{Quelle}: \titel. Herausgegeben von {\editorInnen}. In: \emph{Arthur Schnitzler: Briefwechsel mit Autorinnen und Autoren}.
 Digitale Edition, https://schnitzler-briefe.acdh.oeaw.ac.at/{\dateiname}.html (Stand \today)
\fi

\end{document}


      