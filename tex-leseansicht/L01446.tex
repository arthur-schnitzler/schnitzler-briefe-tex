%% latex-korrekturansicht-vorspann.tex
%% Vorspann für die Korrekturansicht.
%% Lädt die gemeinsame Datei latex-vorspann.tex mit gesetztem Schalter.

\newif\ifkorrekturansicht
\korrekturansichttrue

\input{../tex-inputs/latex-vorspann}


\section[Arthur Schnitzler an Hugo von Hofmannsthal, 16. 9. 1904]{L01446 Arthur Schnitzler an Hugo von Hofmannsthal, 16. 9. 1904}
\nopagebreak\mylabel{L01446v}
\rehead{ }\normalsize\beginnumbering\briefempfaengerindex{Hofmannsthal, Hugo von@\textsc{Hofmannsthal, Hugo von}!zzzSchnitzler, Arthur@\emph{von Arthur Schnitzler}!1904-09-161@{16. 9. 1904}|(be}
\toendnotes[C]{\smallbreak\pagebreak[2]}\Standort{FDH, Hs-30885,114.}
\physDesc{Brief, 1 Blatt, 3 Seiten, 776 Zeichen
\newline{}Handschrift: Bleistift, deutsche Kurrent}
\buchAbdrucke{\weitereDrucke{Hugo von Hofmannsthal, Arthur Schnitzler: \emph{Briefwechsel}. Frankfurt am Main: \emph{S. Fischer} 1964, S. 201.} }\toendnotes[C]{\smallbreak}
\pstart
           \raggedleft{}{\pb}16. 9. 904{\\}\textsc{Lueg a Wolfg}ſee\oindex{Lueg@\textbf{Lueg}, \emph{Teil eines besiedelten Ortes (A.BSOX)}|pw}\pend
           \vspace{0.5em}
\pstart
           lieber Hugo, bis heute ſind wir dageblieben, ſeit vorgeſtern arges
               Regenwetter, heute Nm fährt Richard\pwindex{Beer-Hofmann, Richard 1866-07-11 – 1945-09-26@\textsc{Beer-Hofmann, Richard} (1866-07-11 – 1945-09-26), \emph{Schriftsteller/Schriftstellerin}|pw} vorbei; wir ſteigen zu ihm ein u bleiben noch ein paar Tage in Salzburg\oindex{Salzburg@\textbf{Salzburg}, \emph{A.ADM2}|pw}. Da{\geminationn}
               wahrſcheinlich direct Wien\oindex{Wien@\textbf{Wien}, \emph{A.ADM2}|pw}. Gearbeitet ſo gut wie
               nichts, aber große {\pb}Sehnſucht danach. Mit Burckhard\pwindex{Burckhard, Max Eugen 14.07.1854 – 16.03.1912@\textsc{Burckhard, Max Eugen} (14.07.1854 – 16.03.1912), \emph{Schriftsteller/Schriftstellerin, Rechtswissenschaftler/Rechtswissenschaftlerin, Theaterleiter/Theaterleiterin}|pw} ein paar ſehr angenehme Stunden. Das
               Rad ununterbrochen ſchwer krank – es zeigte sich daſs die Tretkurbel u noch einiges
               andre total hin war. Bin \uline{ein} Mal von \textsc{St. Gilgen}\oindex{St. Gilgen@\textbf{St. Gilgen}, \emph{A.ADM3}|pw} nach \textsc{Lueg}\oindex{Lueg@\textbf{Lueg}, \emph{Teil eines besiedelten Ortes (A.BSOX)}|pw} gefahren. Jetzt iſt es ganz in Ordnung und wird wahrſcheinlich auf der
               Eiſenbahn zer{\pb}trümmert werden. Ihre (eine) Karte
               erhalten. Ob Sie ſchönes Wetter auf der Tour gehabt haben? Eine neulich gekommene
               Karte leg ich bei.\pend
           
\pstart
           Laſſen Sie ſehr bald nach Wien\oindex{Wien@\textbf{Wien}, \emph{A.ADM2}|pw} einiges
               vernehmen.\pend
           
\pstart
           Wir grüßen Sie Beide\pwindex{Schnitzler, Olga 17.01.1882 – 13.01.1970@\textsc{Schnitzler, Olga} (17.01.1882 – 13.01.1970), \emph{Schauspieler/Schauspielerin, Sänger/Sängerin}|pwv}{ }Beide\pwindex{Beer-Hofmann, Paula 25.02.1879 – 30.10.1939@\textsc{Beer-Hofmann, Paula} (25.02.1879 – 30.10.1939)|pwv}.\pend
           
\pstart
           Herzlichſt Ihr{\\[\baselineskip]}\spacefill\mbox{A.}\pend
           \leftskip=0em{}\selectlanguage{ngerman}\endnumbering\briefempfaengerindex{Hofmannsthal, Hugo von@\textsc{Hofmannsthal, Hugo von}!zzzSchnitzler, Arthur@\emph{von Arthur Schnitzler}!1904-09-161@{16. 9. 1904}|)be}\mylabel{L01446h}  \normalsize

\doendnotes{C}
\bigskip
\vfill

\clearpage

\footnotesize

\lohead{\textsc{register}}

% Definiere theindex-Environment komplett neu ohne reledmac
\makeatletter
\renewenvironment{theindex}{%
  \section*{\indexname}%
  \setlength{\parindent}{0pt}%
  \setlength{\parskip}{0pt plus 0.3pt}%
  \let\item\@idxitem
}{%
  \clearpage
}
\makeatother

\IfFileExists{\jobname-pw.ind}{\input{\jobname-pw.ind}}{}

\end{document}

      