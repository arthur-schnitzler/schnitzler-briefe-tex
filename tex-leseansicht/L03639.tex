%% latex-leseansicht-vorspann.tex
%% Vorspann für die Leseansicht.
%% Lädt die gemeinsame Datei latex-vorspann.tex mit nicht gesetztem Schalter.

\newif\ifkorrekturansicht
\korrekturansichtfalse

\input{../tex-inputs/latex-vorspann}


\section[Stefan Zweig an Arthur Schnitzler, 12. 11. 1912]{L03639 Stefan Zweig an Arthur Schnitzler, 12. 11. 1912}
\nopagebreak\mylabel{L03639v}
\rehead{ }\normalsize\beginnumbering\briefempfaengerindex{Schnitzler, Arthur@\textsc{Schnitzler, Arthur}!zzzZweig, Stefan@\emph{von Stefan Zweig}!1912-11-121@{12. 11. 1912}|(be}
\toendnotes[C]{\smallbreak\pagebreak[2]}
\correspDesc{Versand  durch Stefan Zweig am 12. 11. 1912 in Wien
\newline{}Zustellung  im Zeitraum [12. 11. 1912 – 14. 11. 1912?] in Wien
\newline{}Erhalt  durch Arthur Schnitzler am [14. 11. 1912] in Wien}\toendnotes[C]{\smallbreak}
\Standort{CUL, Schnitzler, B 118.}
\physDesc{Brief, 1 Blatt, 4 Seiten, 2458 Zeichen
\newline{}Handschrift: lila Tinte, lateinische Kurrent
\newline{}Schnitzler: 1) mit Bleistift »\textsc{Zweig}«  2) mit rotem Buntstift zwei Unterstreichungen}
\buchAbdrucke{\weitereDrucke{1) Stefan Zweig: \emph{Briefwechsel mit Hermann Bahr, Sigmund Freud, Rainer Maria
                        Rilke und Arthur Schnitzler}. Herausgegeben von Jeffrey B. Berlin, Hans-Ulrich Lindken und Donald A. Prater. Frankfurt am Main: \emph{S. Fischer} 1987, S. 370–372.} \weitereDrucke{2) Stefan Zweig: \emph{Briefe. Bd. I: 1897–1914}. Herausgegeben von Knut Beck, Jeffrey B. Berlin und Natascha Weschenbach-Feggeler. Frankfurt am Main: \emph{S. Fischer} 1995, S. 266–267.} }\toendnotes[C]{\smallbreak}
\pstart
           {\pb}\textcolor{gray}{\textbf{SZ}}\hfill \textcolor{gray}{\textbf{VIII. KOCHGASSE 8\oindex{Wien@\textbf{Wien}!VIII., Josefstadt@\textbf{VIII., Josefstadt}!Kochgasse 8@\textbf{Kochgasse 8}, \emph{Wohngebäude}|pw}}}\pend
           
\pstart
           \raggedleft{}\textcolor{gray}{\textbf{WIEN\oindex{Wien@\textbf{Wien}, \emph{Verwaltungsgebiet}|pw},}}{ }12. Nov 12\pend
           {\vspace{1\baselineskip}}
\pstart{}Verehrter lieber Herr Doktor,\pend\vspace{0.5em}
\pstart
           mit ungemeiner Freude habe ich Ihren »Professor
                  Bernhardi\pwindex{Schnitzler, Arthur 15.\,5.\,1862 Wien – 21.\,10.\,1931 ebd.@\textsc{Schnitzler, Arthur} (15.\,5.\,1862 Wien – 21.\,10.\,1931 ebd.), \emph{Schriftsteller, Mediziner}!Professor Bernhardi. Komödie in fünf Akten@\strich\emph{Professor Bernhardi. Komödie in fünf Akten}|pw}« empfangen, mit Leidenschaft ihn sofort gelesen und eigentlich noch
               immer nicht aus der Hand gelegt, wiewohl ich schon längst bei der letzten Seite war
               und wieder mitten darin und wieder am Ende. Aber es ist ja unsere engste Welt, die
               sich hier auftut, weit freilich, unendlich weit, bis man den Himmel der grossen
               seelischen Gerechtigkeit über ihr mit allen guten Sternen sieht. Ich weiss nicht, ob
               ich Ihnen etwas Liebes damit sage, aber meine Empfindung will doch aufrichtig sein:
                  {\pb}ich spürte im ersten Lesen gar nicht
               mehr, dass dies ein Drama ist, ein Theaterstück, ein Kunstwerk, ich spürte nur
               lebendigstes Leben, das mich ergriff wie ein \label{K_L03639-1v}\edtext{\begin{otherlanguage}{french}fait divers\end{otherlanguage}}{\lemma{\textnormal{\emph{fait divers}}}\Cendnote{\textnormal{französisch: Nachricht der Rubrik
                  vermischte Meldungen}}}\label{K_L03639-1} der Zeitung, ein politischer Fall, spürte erst nur
               menschliche Empörung, Freude, Hass und Liebe. Dann später erst kam das Besinnen, dass
               dies Gestaltetes, Verwandeltes, Kunstwerk und nicht unmittelbares Leben ist. Und \strikeout{noch} immer habe ich noch keine Ruhe, um den Bernhardi\pwindex{Schnitzler, Arthur 15.\,5.\,1862 Wien – 21.\,10.\,1931 ebd.@\textsc{Schnitzler, Arthur} (15.\,5.\,1862 Wien – 21.\,10.\,1931 ebd.), \emph{Schriftsteller, Mediziner}!Professor Bernhardi. Komödie in fünf Akten@\strich\emph{Professor Bernhardi. Komödie in fünf Akten}|pw} als Kunstwerk oder gar auf den
               Theatererfolg hin betrachten zu können, ich bin zu passioniert davon, zu sehr mit
               Sympathie und Zorn gegen und für seine so herrlich lebendigen, so atemnahen Menschen.
                  \label{K_L03639-2v}\edtext{Nostra ipsissima res agitur}{\lemma{\textnormal{\emph{Nostra … agitur}}}\Cendnote{\textnormal{lateinisch: Es geht um unsere ureigene
                  Angelegenheit.}}}\label{K_L03639-2} – ich spür es zu sehr und {\pb}kann gar nicht recht heraus, mir’s zu
               betrachten, so sehr bin ich darin. Jedesfalls: Sie haben nie eine grössere Scene
               geschrieben als die im vierten Akt zwischen dem Geistlichen und Bernhardi, es ist die
               \label{K_L03639-3v}\edtext{Grossinquisitorscene\pwindex{Dostojevskij, Fjodor Mihajlovič 11.\,11.\,1821 Moskau – 9.\,2.\,1881 Sankt Petersburg@\textsc{Dostojevskij, Fjodor Mihajlovič} (11.\,11.\,1821 Moskau – 9.\,2.\,1881 Sankt Petersburg), \emph{Schriftsteller}!Brüder Karamasow@\strich\emph{Die Brüder Karamasow}|pwv}}{\lemma{\textnormal{\emph{Grossinquisitorscene}}}\Cendnote{\textnormal{Das fünfte Kapitel im fünften Buch von Fjodor Dostojevskijs\pwindex{Dostojevskij, Fjodor Mihajlovič 11.\,11.\,1821 Moskau – 9.\,2.\,1881 Sankt Petersburg@\textsc{Dostojevskij, Fjodor Mihajlovič} (11.\,11.\,1821 Moskau – 9.\,2.\,1881 Sankt Petersburg), \emph{Schriftsteller}|pwk} Roman \emph{Die Brüder Karamasow}\pwindex{Dostojevskij, Fjodor Mihajlovič 11.\,11.\,1821 Moskau – 9.\,2.\,1881 Sankt Petersburg@\textsc{Dostojevskij, Fjodor Mihajlovič} (11.\,11.\,1821 Moskau – 9.\,2.\,1881 Sankt Petersburg), \emph{Schriftsteller}!Brüder Karamasow@\strich\emph{Die Brüder Karamasow}|pwk} verhandelt in der Legende vom Großinquisitor, der Jesus\pwindex{Jesus 7–4 v.\,u.\,Z. Nazareth – 30/31 Jerusalem@\textsc{Jesus} (7–4 v.\,u.\,Z. Nazareth – 30/31 Jerusalem), \emph{Wanderprediger}|pwk} anklagt, Fragen zu Theodizee, menschlicher Freiheit und Glück sowie Christentum und Kirchenmoral.}}}\label{K_L03639-3} Ihres dramatischen Werks, ganz weit blickend, hart und doch voll
               Güte, gross in jedem, im menschlichen, im künstlerischen Sinn. Nie waren Ihre
               Menschen lebendiger, nie Sie selbst dichterisch so weit, das spüre ich mit Beglückung
               und – verzeihen Sie! – mit Stolz, denn man darf doch niemandem versagen, auf die
               stolz zu sein, die man liebt.\pend
           
\pstart
           Dramaturgisch den Bernhardi\pwindex{Schnitzler, Arthur 15.\,5.\,1862 Wien – 21.\,10.\,1931 ebd.@\textsc{Schnitzler, Arthur} (15.\,5.\,1862 Wien – 21.\,10.\,1931 ebd.), \emph{Schriftsteller, Mediziner}!Professor Bernhardi. Komödie in fünf Akten@\strich\emph{Professor Bernhardi. Komödie in fünf Akten}|pw} zu betrachten,
               vermag ich noch nicht, ich sagte es ja, er ist noch zu heiß in mir. Aber ich weiß,
               solchen letzten menschlichen Entäußerungen kann nie {\pb}die Bewunderung fehlen. Ich weiß Ihr Werk\pwindex{Schnitzler, Arthur 15.\,5.\,1862 Wien – 21.\,10.\,1931 ebd.@\textsc{Schnitzler, Arthur} (15.\,5.\,1862 Wien – 21.\,10.\,1931 ebd.), \emph{Schriftsteller, Mediziner}!Professor Bernhardi. Komödie in fünf Akten@\strich\emph{Professor Bernhardi. Komödie in fünf Akten}|pwv} wird wirken (im banalen
               bühnentechnischen Sinn und um wie viel mehr im höheren!), ich werde jedesfalls in Berlin\oindex{Berlin@\textbf{Berlin}, \emph{Hauptstadt}|pw} bei der \label{K_L03639-4v}\edtext{Première\eventindex{Kleines Theater@\textbf{Kleines Theater}!Uraufführung von Professor Bernhardi, 28.11.1912@Uraufführung von Professor Bernhardi, 28.11.1912|pwv}}{\lemma{\textnormal{\emph{Première}}}\Cendnote{\textnormal{Die Uraufführung\eventindex{Kleines Theater@\textbf{Kleines Theater}!Uraufführung von Professor Bernhardi, 28.11.1912@Uraufführung von Professor Bernhardi, 28.11.1912|pwkv} von \emph{Professor Bernhardi}\pwindex{Schnitzler, Arthur 15.\,5.\,1862 Wien – 21.\,10.\,1931 ebd.@\textsc{Schnitzler, Arthur} (15.\,5.\,1862 Wien – 21.\,10.\,1931 ebd.), \emph{Schriftsteller, Mediziner}!Professor Bernhardi. Komödie in fünf Akten@\strich\emph{Professor Bernhardi. Komödie in fünf Akten}|pwk} fand am 28. 11. 1912 am Kleinen Theater\oindex{Kleines Theater@\textbf{Kleines Theater}, \emph{Theater}|pwk} in Berlin\oindex{Berlin@\textbf{Berlin}, \emph{Hauptstadt}|pwk}
                  statt.}}}\label{K_L03639-4} sein, sobald ich das Datum erfahre und eine Einteilung zu treffen
               vermag. Denn ich \label{K_L03639-5v}\edtext{möchte nicht
                  fehlen}{\lemma{\textnormal{\emph{möchte nicht
                  fehlen}}}\Cendnote{\textnormal{Er dürfte den Plan, zur Uraufführung\eventindex{Kleines Theater@\textbf{Kleines Theater}!Uraufführung von Professor Bernhardi, 28.11.1912@Uraufführung von Professor Bernhardi, 28.11.1912|pwkv} nach Berlin\oindex{Berlin@\textbf{Berlin}, \emph{Hauptstadt}|pwk} zu reisen, nicht umgesetzt haben.
                  Zumindest erwähnt Schnitzler im \emph{Tagebuch}\pwindex{Schnitzler, Arthur 15.\,5.\,1862 Wien – 21.\,10.\,1931 ebd.@\textsc{Schnitzler, Arthur} (15.\,5.\,1862 Wien – 21.\,10.\,1931 ebd.), \emph{Schriftsteller, Mediziner}!Tagebuch@\strich\emph{Tagebuch}|pwk}-Eintrag zum 28. 11. 1912 seine
                  Anwesenheit nicht.}}}\label{K_L03639-5}, wenn ein solches Werk aus Buch zum Wort und vom Wort
               zur lebendigen Wirkung wird.\pend
           
\pstart
           Viele Grüsse Ihrer verehrten Frau Gemahlin\pwindex{Schnitzler, Olga 17.\,1.\,1882 Wien – 13.\,1.\,1970 Lugano@\textsc{Schnitzler, Olga} (17.\,1.\,1882 Wien – 13.\,1.\,1970 Lugano), \emph{Schauspielerin, Sängerin}|pwv}! Innigst getreu und mit frohem Glückwunsch{\\[\baselineskip]}Ihr {\\[\baselineskip]}\spacefill\mbox{Stefan Zweig}\pend
           \leftskip=0em{}\selectlanguage{ngerman}\endnumbering\briefempfaengerindex{Schnitzler, Arthur@\textsc{Schnitzler, Arthur}!zzzZweig, Stefan@\emph{von Stefan Zweig}!1912-11-121@{12. 11. 1912}|)be}\mylabel{L03639h}  \newcommand{\dateiname}{L03639}\newcommand{\titel}{Stefan Zweig an Arthur Schnitzler, 12. 11. 1912}\newcommand{\editorInnen}{Selma Jahnke und Martin Anton Müller}%% latex-leseansicht-abspann.tex
%% Abspann für die Leseansicht.
%% Der Schalter \ifkorrekturansicht ist bereits durch den Vorspann gesetzt.

%% latex-abspann.tex
%% Gemeinsamer Abspann für Korrekturansicht und Leseansicht.
%% Setzt den Schalter \ifkorrekturansicht voraus (gesetzt in den
%% einbindenden Dateien latex-korrekturansicht-abspann.tex bzw.
%% latex-leseansicht-abspann.tex).
%% ---------------------------------------------------------------

\normalsize

% Das esempio-Environment wird nur in der Leseansicht benötigt
\ifkorrekturansicht\else
\newenvironment{esempio}[3]%
{
    \vspace{1.5ex}
    \rlap{\underline{#1}}
    \par
    \setlength{\parindent}{0cm}
    \nopagebreak
    \leftskip=#2cm
    \rightskip=#3cm
}
{
    \par
}
\fi

\doendnotes{C}
\bigskip
\vfill

\clearpage

\footnotesize

\ifkorrekturansicht
  \lohead{\textsc{register}}
\fi

% theindex-Environment neu definieren ohne reledmac
\makeatletter
\renewenvironment{theindex}{%
  \ifkorrekturansicht
    \section*{\indexname}%
  \else
    \subsubsection*{Index der erwähnten Entitäten}%
  \fi
  \setlength{\parindent}{0pt}%
  \setlength{\parskip}{0pt plus 0.3pt}%
  \let\item\@idxitem
}{%
  \ifkorrekturansicht\clearpage\fi
}
\makeatother

\IfFileExists{\jobname-pw.ind}{\input{\jobname-pw.ind}}{}

% Quellenangabe nur in der Leseansicht
\ifkorrekturansicht\else
% Fallback-Definitionen, falls die .tex-Datei \titel etc. nicht gesetzt hat
\providecommand{\titel}{}
\providecommand{\editorInnen}{}
\providecommand{\dateiname}{\jobname}

\vspace{3cm}

\vfill

\footnotesize
\textsc{Quelle}: \titel. Herausgegeben von {\editorInnen}. In: \emph{Arthur Schnitzler: Briefwechsel mit Autorinnen und Autoren}.
 Digitale Edition, https://schnitzler-briefe.acdh.oeaw.ac.at/{\dateiname}.html (Stand \today)
\fi

\end{document}


