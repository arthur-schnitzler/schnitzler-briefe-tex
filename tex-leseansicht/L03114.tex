%% latex-leseansicht-vorspann.tex
%% Vorspann für die Leseansicht.
%% Lädt die gemeinsame Datei latex-vorspann.tex mit nicht gesetztem Schalter.

\newif\ifkorrekturansicht
\korrekturansichtfalse

\input{../tex-inputs/latex-vorspann}

\begin{center}
            \textcolor{red}{ENTWURF, NICHT FERTIG KORRIGIERT}
                      \end{center}
            
         
         \renewcommand{\erwaehntePersonen}{Personen: Richard Beer-Hofmann}
         \renewcommand{\erwaehnteOrte}{Orte: Bad Ischl, Berghof, Unterach am Attersee, Weißenbach am Attersee, Wien}
         \renewcommand{\erwaehnteWerke}{}
               \section[Felix Salten an Arthur Schnitzler, {[}24. oder 25.? 8. 1892{]}]{ Felix Salten an Arthur Schnitzler, {[}24. oder 25.? 8. 1892{]}}\nopagebreak\mylabel{v}\rehead{ }\begin{ledgroupsized}[t]{13cm}\normalsize\beginnumbering \toendnotes[C]{\smallbreak\pagebreak[2]} \Standort{CUL, Schnitzler, B 89, A 1.}
\physDesc{Brief, 1 Blatt, 2 Seiten, 609 Zeichen
\newline{}Handschrift: schwarze Tinte, lateinische Kurrent
\newline{}Schnitzler: mit Bleistift datiert: »\strikeout{Anf} En{[}de{]}
                                       Aug 92« 
\newline{}Ordnung: mit Bleistift von unbekannter Hand nummeriert:
                                    »18« }\toendnotes[C]{\smallbreak}\pstart
           \noindent{}{\pb}Verehrtester Besten Dank für \label{K_L03114-1v}\edtext{Ihren Brief}{\lemma{\textnormal{\emph{Ihren Brief}}}\Cendnote{\textnormal{Die
                  grobe Einordnung des undatierten Korrespondenzstücks gelingt durch die Datierung
                     Schnitzler\pwindex{Schnitzler, Arthur 15.05.1862 – 21.10.1931@\textsc{Schnitzler, Arthur} (15.05.1862 – 21.10.1931), \emph{Schriftsteller, Mediziner}|pwk}s auf »En{[}de{]} Aug 92«. Innerhalb der Korrespondenzstücke dürfte es sich um Schnitzler\pwindex{Schnitzler, Arthur 15.05.1862 – 21.10.1931@\textsc{Schnitzler, Arthur} (15.05.1862 – 21.10.1931), \emph{Schriftsteller, Mediziner}|pwk}s Reaktion auf das Schreiben vom 23. 8. 1892 handeln, da in
                  diesem noch nicht von einem persönlichen Treffen die Rede ist. Schnitzler\pwindex{Schnitzler, Arthur 15.05.1862 – 21.10.1931@\textsc{Schnitzler, Arthur} (15.05.1862 – 21.10.1931), \emph{Schriftsteller, Mediziner}|pwk} war ab 27. 8. 1892 in Ischl\oindex{Bad Ischl@\textbf{Bad Ischl}|pwk} und damit wurde ein Treffen erst möglich. Für 31. 8. 1892 ist eine
                  Zusammenkunft belegt, bildet also den letzten möglichen Zeitpunkt. Weniger gewiss,
                  aber doch wahrscheinlich ist die Annahme, dass Schnitzler\pwindex{Schnitzler, Arthur 15.05.1862 – 21.10.1931@\textsc{Schnitzler, Arthur} (15.05.1862 – 21.10.1931), \emph{Schriftsteller, Mediziner}|pwk} vor seiner Ankunft in Ischl\oindex{Bad Ischl@\textbf{Bad Ischl}|pwk} das Treffen einforderte und diese Kommunikation noch nach Wien\oindex{Wien@\textbf{Wien}|pwk} lief. Damit wären der 24. oder
                     25. 8. 1892 wahrscheinliche Daten für dieses
                  Korrespondenzstück.}}}\label{K_L03114-1h}. Ob gerade eine persönliche breite Aussprache für mich
               beruhigend wäre, weiss ich nicht, – doch darauf ko{\geminationm}t es
               gewiss nicht an. Ich freue mich jedenfalls aufrichtig Sie zu sehen, u bitte Sie mir
               den Tag zu bestimmen, wann ich nach Ischl\oindex{Bad Ischl@\textbf{Bad Ischl}|pw} kommen
               kann, oder wann Sie nach Weissenbach\oindex{Weissenbach am Attersee@\textbf{Weißenbach am Attersee}|pw} kommen
               wollen. Auch {\pb}am Berghof\oindex{Berghof@\textbf{Berghof}|pw} würde man Sie gerne sehen, und bin ich beauftragt, Sie
               für einen Tag herüberzubitten. Auch Beer-Hofmann\pwindex{Beer-Hofmann, Richard 1866-07-11 – 1945-09-26@\textsc{Beer-Hofmann, Richard} (1866-07-11 – 1945-09-26), \emph{Schriftsteller}|pw} soll, wenn er will mitkommen. Dass es mir hauptsächlich jetzt
               um die Aussprache mit Ihnen zu thun ist brauche ich nicht erst zu sagen.\pend
           \pstart
           Also auf Wiedersehen {\\[\baselineskip]}Ihr \spacefill\mbox{Salten}\pend
           \leftskip=0em{}
         
         \endnumbering\mylabel{h}\end{ledgroupsized}\begin{anhang}\end{anhang}\newcommand{\dateiname}{L03114}\newcommand{\titel}{Felix Salten an Arthur Schnitzler, [24. oder 25.? 8. 1892]}\newcommand{\editorInnen}{Martin Anton Müller und Laura Untner}%% latex-leseansicht-abspann.tex
%% Abspann für die Leseansicht.
%% Der Schalter \ifkorrekturansicht ist bereits durch den Vorspann gesetzt.

%% latex-abspann.tex
%% Gemeinsamer Abspann für Korrekturansicht und Leseansicht.
%% Setzt den Schalter \ifkorrekturansicht voraus (gesetzt in den
%% einbindenden Dateien latex-korrekturansicht-abspann.tex bzw.
%% latex-leseansicht-abspann.tex).
%% ---------------------------------------------------------------

\normalsize

% Das esempio-Environment wird nur in der Leseansicht benötigt
\ifkorrekturansicht\else
\newenvironment{esempio}[3]%
{
    \vspace{1.5ex}
    \rlap{\underline{#1}}
    \par
    \setlength{\parindent}{0cm}
    \nopagebreak
    \leftskip=#2cm
    \rightskip=#3cm
}
{
    \par
}
\fi

\doendnotes{C}
\bigskip
\vfill

\clearpage

\footnotesize

\ifkorrekturansicht
  \lohead{\textsc{register}}
\fi

% theindex-Environment neu definieren ohne reledmac
\makeatletter
\renewenvironment{theindex}{%
  \ifkorrekturansicht
    \section*{\indexname}%
  \else
    \subsubsection*{Index der erwähnten Entitäten}%
  \fi
  \setlength{\parindent}{0pt}%
  \setlength{\parskip}{0pt plus 0.3pt}%
  \let\item\@idxitem
}{%
  \ifkorrekturansicht\clearpage\fi
}
\makeatother

\IfFileExists{\jobname-pw.ind}{\input{\jobname-pw.ind}}{}

% Quellenangabe nur in der Leseansicht
\ifkorrekturansicht\else
% Fallback-Definitionen, falls die .tex-Datei \titel etc. nicht gesetzt hat
\providecommand{\titel}{}
\providecommand{\editorInnen}{}
\providecommand{\dateiname}{\jobname}

\vspace{3cm}

\vfill

\footnotesize
\textsc{Quelle}: \titel. Herausgegeben von {\editorInnen}. In: \emph{Arthur Schnitzler: Briefwechsel mit Autorinnen und Autoren}.
 Digitale Edition, https://schnitzler-briefe.acdh.oeaw.ac.at/{\dateiname}.html (Stand \today)
\fi

\end{document}


      