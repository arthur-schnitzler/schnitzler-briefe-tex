%% latex-leseansicht-vorspann.tex
%% Vorspann für die Leseansicht.
%% Lädt die gemeinsame Datei latex-vorspann.tex mit nicht gesetztem Schalter.

\newif\ifkorrekturansicht
\korrekturansichtfalse

\input{../tex-inputs/latex-vorspann}

\begin{center}
            \textcolor{red}{ENTWURF, NICHT FERTIG KORRIGIERT}
                      \end{center}
            
         
         \renewcommand{\erwaehntePersonen}{Personen: Richard Beer-Hofmann, Ignaz Brüll, Marie Brüll, Felix Salten}
         \renewcommand{\erwaehnteOrte}{Orte: Bad Ischl, Berghof, Unterach am Attersee, Weißenbach am Attersee, Wien}
         \renewcommand{\erwaehnteWerke}{}
               \section[Felix Salten an Arthur Schnitzler, {[}25.? 8. 1892{]}]{ Felix Salten an Arthur Schnitzler, {[}25.? 8. 1892{]}}\nopagebreak\mylabel{v}\rehead{ }\begin{ledgroupsized}[t]{13cm}\normalsize\beginnumbering\briefempfaengerindex{Schnitzler, Arthur@\textsc{Schnitzler, Arthur}!zzzSalten, Felix@\emph{von Felix Salten}!1892-08-251@{{[}25.? 8. 1892{]}}|(be} \toendnotes[C]{\smallbreak\pagebreak[2]} \Standort{CUL, Schnitzler, B 89, A 1.}
\physDesc{Brief, 1 Blatt, 2 Seiten, 611 Zeichen
\newline{}Handschrift: schwarze Tinte, lateinische Kurrent
\newline{}Schnitzler: mit Bleistift datiert: »\substVorne{}\textsuperscript{Anf}\substDazwischen{}En{[}de{]}\substHinten{} Au{[}g{]} 92« 
\newline{}Ordnung: mit Bleistift von unbekannter Hand nummeriert: »18« }\toendnotes[C]{\smallbreak}\pstart
           \noindent{}{\pb}Verehrtester\textcolor{gray}{!}{ }Besten Dank für \label{K_L03114-1v}\edtext{Ihren
                  Brief}{\lemma{\textnormal{\emph{Ihren
                  Brief}}}\Cendnote{\textnormal{Die grobe Einordnung des
                  undatierten Korrespondenzstücks gelingt durch die Datierung Schnitzler\pwindex{Schnitzler, Arthur 15.05.1862 – 21.10.1931@\textsc{Schnitzler, Arthur} (15.05.1862 – 21.10.1931), \emph{Schriftsteller, Mediziner}|pwk}s auf »En{[}de{]} Au{[}g{]} 92«. Innerhalb der Korrespondenzstücke dürfte es sich um Schnitzler\pwindex{Schnitzler, Arthur 15.05.1862 – 21.10.1931@\textsc{Schnitzler, Arthur} (15.05.1862 – 21.10.1931), \emph{Schriftsteller, Mediziner}|pwk}s Reaktion auf das Schreiben vom 23. 8. 1892 handeln, da in
                  diesem noch nicht von einem persönlichen Treffen die Rede war. Schnitzler\pwindex{Schnitzler, Arthur 15.05.1862 – 21.10.1931@\textsc{Schnitzler, Arthur} (15.05.1862 – 21.10.1931), \emph{Schriftsteller, Mediziner}|pwk} war ab 27. 8. 1892 in Ischl\oindex{Bad Ischl@\textbf{Bad Ischl}|pwk} – erst damit wurde ein Treffen möglich. Für den 31. 8. 1892 ist eine
                  Zusammenkunft belegt. Dieser Tag bildet also den letzten möglichen Zeitpunkt.
                  Weniger gewiss, aber doch wahrscheinlich ist die Annahme, dass Schnitzler\pwindex{Schnitzler, Arthur 15.05.1862 – 21.10.1931@\textsc{Schnitzler, Arthur} (15.05.1862 – 21.10.1931), \emph{Schriftsteller, Mediziner}|pwk} vor seiner Ankunft in Ischl\oindex{Bad Ischl@\textbf{Bad Ischl}|pwk} das Treffen eingefordert hatte und diese Kommunikation
                  noch nach Wien\oindex{Wien@\textbf{Wien}|pwk} lief. Damit wäre der 25. 8. 1892 das wahrscheinliche Datum für dieses
                  Korrespondenzstück.}}}\label{K_L03114-1h}. Ob gerade eine persönliche breite Aussprache für mich
               beruhigend wäre, weiss ich nicht, – doch darauf ko{\geminationm}t es
               gewiss nicht an. Ich freue mich jedenfalls aufrichtig Sie zu sehen, u bitte Sie mir
               den Tag zu bestimmen, wann ich nach Ischl\oindex{Bad Ischl@\textbf{Bad Ischl}|pw} kommen
               kann, oder wann Sie nach Weissenbach\oindex{Weissenbach am Attersee@\textbf{Weißenbach am Attersee}|pw} kommen
               wollen. Auch {\pb}am \label{K_L03114-2v}\edtext{Berghof\oindex{Berghof@\textbf{Berghof}|pw}}{\lemma{\textnormal{\emph{Berghof}}}\Cendnote{\textnormal{Schnitzler\pwindex{Schnitzler, Arthur 15.05.1862 – 21.10.1931@\textsc{Schnitzler, Arthur} (15.05.1862 – 21.10.1931), \emph{Schriftsteller, Mediziner}|pwk} war zwar in seinem Leben mehrfach
                  in Unterach am Attersee\oindex{Unterach am Attersee@\textbf{Unterach am Attersee}|pwk}, ein Aufenthalt im Berghof\oindex{Berghof@\textbf{Berghof}|pwk} ist aber nur für den 1. 7. 1897
                  nachgewiesen. Das ist umso auffälliger, weil die Korrespondenz Salten\pwindex{Salten, Felix 06.09.1869 – 08.10.1945@\textsc{Salten, Felix} (06.09.1869 – 08.10.1945), \emph{Schriftsteller, Journalist, Chefredakteur}|pwk}s, der hier über Jahrzehnte große Teile des Sommers
                  verbrachte, durchzogen ist mit nachdrücklichen Bitten, auf ein paar Tage
                  vorbeizukommen (vgl. die Korrespondenzstücke vom 17. 8. 1910, 31. 7. 1916, 17. [8.?] 1921 und 17. 8. 1922). Da Schnitzler\pwindex{Schnitzler, Arthur 15.05.1862 – 21.10.1931@\textsc{Schnitzler, Arthur} (15.05.1862 – 21.10.1931), \emph{Schriftsteller, Mediziner}|pwk} dem zu keinem Zeitpunkt nachkommt, dürfte es nicht an
                  seinem inneren Verhältnis zu Salten\pwindex{Salten, Felix 06.09.1869 – 08.10.1945@\textsc{Salten, Felix} (06.09.1869 – 08.10.1945), \emph{Schriftsteller, Journalist, Chefredakteur}|pwk} liegen, sondern eher an
                  der Gastgeberfamilie Ignaz\pwindex{Bruell, Ignaz 07.11.1846 – 17.09.1907@\textsc{Brüll, Ignaz} (07.11.1846 – 17.09.1907), \emph{Komponist}|pwk} und Marie Brüll\pwindex{Bruell, Marie 24.05.1861 – 27.11.1932@\textsc{Brüll, Marie} (24.05.1861 – 27.11.1932)|pwk}, die den Berghof\oindex{Berghof@\textbf{Berghof}|pwk} als Refugium für Künstlerinnen und Künstler
                  betrieben, und zu denen Schnitzler\pwindex{Schnitzler, Arthur 15.05.1862 – 21.10.1931@\textsc{Schnitzler, Arthur} (15.05.1862 – 21.10.1931), \emph{Schriftsteller, Mediziner}|pwk} Abstand
                  gehalten haben scheint.}}}\label{K_L03114-2h} würde man Sie gerne sehen, und bin ich beauftragt,
               Sie für einen Tag herüberzubitten. Auch Beer-Hofmann\pwindex{Beer-Hofmann, Richard 1866-07-11 – 1945-09-26@\textsc{Beer-Hofmann, Richard} (1866-07-11 – 1945-09-26), \emph{Schriftsteller}|pw} soll, wenn er will{[},{]} mitkommen. Dass es mir
               hauptsächlich jetzt um die Aussprache mit Ihnen zu thun ist\textcolor{gray}{,}
               brauche ich nicht erst zu sagen.\pend
           \pstart
           Also auf Wiedersehen {\\[\baselineskip]}Ihr \spacefill\mbox{Salten}\pend
           \leftskip=0em{}
         
         \endnumbering\mylabel{h}\end{ledgroupsized}  \newcommand{\dateiname}{L03114}\newcommand{\titel}{Felix Salten an Arthur Schnitzler, [25.? 8. 1892]}\newcommand{\editorInnen}{Martin Anton Müller und Laura Untner}%% latex-leseansicht-abspann.tex
%% Abspann für die Leseansicht.
%% Der Schalter \ifkorrekturansicht ist bereits durch den Vorspann gesetzt.

%% latex-abspann.tex
%% Gemeinsamer Abspann für Korrekturansicht und Leseansicht.
%% Setzt den Schalter \ifkorrekturansicht voraus (gesetzt in den
%% einbindenden Dateien latex-korrekturansicht-abspann.tex bzw.
%% latex-leseansicht-abspann.tex).
%% ---------------------------------------------------------------

\normalsize

% Das esempio-Environment wird nur in der Leseansicht benötigt
\ifkorrekturansicht\else
\newenvironment{esempio}[3]%
{
    \vspace{1.5ex}
    \rlap{\underline{#1}}
    \par
    \setlength{\parindent}{0cm}
    \nopagebreak
    \leftskip=#2cm
    \rightskip=#3cm
}
{
    \par
}
\fi

\doendnotes{C}
\bigskip
\vfill

\clearpage

\footnotesize

\ifkorrekturansicht
  \lohead{\textsc{register}}
\fi

% theindex-Environment neu definieren ohne reledmac
\makeatletter
\renewenvironment{theindex}{%
  \ifkorrekturansicht
    \section*{\indexname}%
  \else
    \subsubsection*{Index der erwähnten Entitäten}%
  \fi
  \setlength{\parindent}{0pt}%
  \setlength{\parskip}{0pt plus 0.3pt}%
  \let\item\@idxitem
}{%
  \ifkorrekturansicht\clearpage\fi
}
\makeatother

\IfFileExists{\jobname-pw.ind}{\input{\jobname-pw.ind}}{}

% Quellenangabe nur in der Leseansicht
\ifkorrekturansicht\else
% Fallback-Definitionen, falls die .tex-Datei \titel etc. nicht gesetzt hat
\providecommand{\titel}{}
\providecommand{\editorInnen}{}
\providecommand{\dateiname}{\jobname}

\vspace{3cm}

\vfill

\footnotesize
\textsc{Quelle}: \titel. Herausgegeben von {\editorInnen}. In: \emph{Arthur Schnitzler: Briefwechsel mit Autorinnen und Autoren}.
 Digitale Edition, https://schnitzler-briefe.acdh.oeaw.ac.at/{\dateiname}.html (Stand \today)
\fi

\end{document}


      