%% latex-korrekturansicht-vorspann.tex
%% Vorspann für die Korrekturansicht.
%% Lädt die gemeinsame Datei latex-vorspann.tex mit gesetztem Schalter.

\newif\ifkorrekturansicht
\korrekturansichttrue

\input{../tex-inputs/latex-vorspann}


\section[Hugo von Hofmannsthal an Arthur Schnitzler, {[}23. 3. 1899{]}]{L00909 Hugo von Hofmannsthal an Arthur Schnitzler, {[}23. 3. 1899{]}}
\nopagebreak\mylabel{L00909v}
\rehead{ }\normalsize\beginnumbering\briefempfaengerindex{Schnitzler, Arthur@\textsc{Schnitzler, Arthur}!zzzHofmannsthal, Hugo von@\emph{von Hugo von Hofmannsthal}!1899-03-231@{{[}23. 3. 1899{]}}|(be}
\toendnotes[C]{\smallbreak\pagebreak[2]}\Standort{CUL, Schnitzler, B 43.}
\physDesc{Brief, 1 Blatt, 3 Seiten, 632 Zeichen
\newline{}Handschrift: schwarze Tinte, deutsche Kurrent
\newline{}Schnitzler: mit Bleistift datiert: »23/3? 99« 
\newline{}Ordnung: 1) mit Bleistift von unbekannter Hand nummeriert: »\strikeout{144}«  2) mit Bleistift von unbekannter Hand nummeriert:
                                    »141«}
\buchAbdrucke{\weitereDrucke{Hugo von Hofmannsthal, Arthur Schnitzler: \emph{Briefwechsel}. Frankfurt am Main: \emph{S. Fischer} 1964, S. 120.} }\toendnotes[C]{\smallbreak}
\pstart
           \raggedleft{}{\pb}Berlin, Windsor Behrenſtraße\oindex{Hotel Windsor@\textbf{Hotel Windsor}, \emph{Hotel (K.HTL)}|pw}\pend
           
\pstart{}Mein guter lieber Arthur\pend\vspace{0.5em}
\pstart
           Könnten Sie nicht hierher\oindex{Berlin@\textbf{Berlin}, \emph{P.PPLC}|pwv} ko{\geminationm}en? wir könnten ſehr viel beiſammen ſein und auch ſonſt
               ſieht man viele ernſte und liebenswürdige Menſchen und es wäre Ihnen doch leichter,
               ſich ein biſſl in die Höh zu bringen, als in Wien\oindex{Wien@\textbf{Wien}, \emph{A.ADM2}|pw},
               wo die Erinnerung\pwindex{Reinhard, Marie 1871-03-13 – 1899-03-18@\textsc{Reinhard, Marie} (1871-03-13 – 1899-03-18), \emph{Gesangspädagoge/Gesangspädagogin}|pwv} Ihnen bei
               jedem Schritt {\pb}friſch weh thut.
               Ich ſehne mich ſehr, mit Ihnen zu ſprechen, zu ſchreiben bin ich nicht im Stand.\pend
           
\pstart
           Daſs dieſe Erinnerung immer mit meinen erſten Stücken\pwindex{Hochzeit der Sobeide@\emph{Die Hochzeit der Sobeide}|pwv}\pwindex{Abenteurer und die Saengerin oder Die Geschenke des Lebens@\emph{Der Abenteurer und die Sängerin oder Die Geschenke des Lebens}|pwv} verknüpft bleiben
               muſs!\pend
           
\pstart
           Von Herzen Ihr{\\[\baselineskip]}\spacefill\mbox{Hugo.}\pend
           \leftskip=0em{}
\pstart
           \noindent{}P. S. \label{K_L00909-1v}\edtext{Hier}{\lemma{\textnormal{\emph{Hier}}}\Cendnote{\textnormal{Die Uraufführung\eventindex{Deutsches Theater Berlin@\textbf{Deutsches Theater Berlin}!Berliner Urauffuehrung von Der Abenteurer und die Saengerin und Die Hochzeit der Sobeide, 18.3.1899@Berliner Uraufführung von Der Abenteurer und die Sängerin und Die Hochzeit der Sobeide, 18.3.1899|pwkv} im \emph{Deutschen Theater}\orgindex{Deutsches Theater Berlin@Deutsches Theater Berlin|pwk} war am 18. 3. 1899 und damit zugleich mit der
                        Wien\oindex{Wien@\textbf{Wien}, \emph{A.ADM2}|pwk}er Uraufführung\eventindex{Burgtheater@\textbf{Burgtheater}!Wiener Urauffuehrung von Der Abenteurer und die Saengerin und Die Hochzeit der Sobeide, 18.3.1899@Wiener Uraufführung von Der Abenteurer und die Sängerin und Die Hochzeit der Sobeide, 18.3.1899|pwkv} angesetzt.}}}\label{K_L00909-1}{ }ſind meine armen Stücke\pwindex{Hochzeit der Sobeide@\emph{Die Hochzeit der Sobeide}|pwv}\pwindex{Abenteurer und die Saengerin oder Die Geschenke des Lebens@\emph{Der Abenteurer und die Sängerin oder Die Geschenke des Lebens}|pwv} von einer beiſpiellos
                  böſen {\pb}Preſſe erſchlagen worden
                  und muſsten nach dem dritten Mal abgeſetzt werden.\pend
           \selectlanguage{ngerman}\endnumbering\briefempfaengerindex{Schnitzler, Arthur@\textsc{Schnitzler, Arthur}!zzzHofmannsthal, Hugo von@\emph{von Hugo von Hofmannsthal}!1899-03-231@{{[}23. 3. 1899{]}}|)be}\mylabel{L00909h}  \normalsize

\doendnotes{C}
\bigskip
\vfill

\clearpage

\footnotesize

\lohead{\textsc{register}}

% Definiere theindex-Environment komplett neu ohne reledmac
\makeatletter
\renewenvironment{theindex}{%
  \section*{\indexname}%
  \setlength{\parindent}{0pt}%
  \setlength{\parskip}{0pt plus 0.3pt}%
  \let\item\@idxitem
}{%
  \clearpage
}
\makeatother

\IfFileExists{\jobname-pw.ind}{\input{\jobname-pw.ind}}{}

\end{document}

      