%% latex-leseansicht-vorspann.tex
%% Vorspann für die Leseansicht.
%% Lädt die gemeinsame Datei latex-vorspann.tex mit nicht gesetztem Schalter.

\newif\ifkorrekturansicht
\korrekturansichtfalse

\input{../tex-inputs/latex-vorspann}


         
         \renewcommand{\erwaehntePersonen}{Personen: Marie Reinhard}
         \renewcommand{\erwaehnteInstitutionen}{Institutionen: Deutsches Theater Berlin}
         \renewcommand{\erwaehnteOrte}{Orte: Berlin, Hotel Windsor, Wien}
         \renewcommand{\erwaehnteWerke}{Werke: Der Abenteurer und die Sängerin oder Die Geschenke des Lebens, Die Hochzeit der Sobeide}
               \section[Hugo von Hofmannsthal an Arthur Schnitzler, {[}23. 3. 1899{]}]{ Hugo von Hofmannsthal an Arthur Schnitzler, {[}23. 3. 1899{]}}\nopagebreak\mylabel{v}\rehead{ }\begin{ledgroupsized}[t]{13cm}\normalsize\beginnumbering \toendnotes[C]{\smallbreak\pagebreak[2]} \Standort{CUL, Schnitzler, B 43.}
\physDesc{Brief, 1 Blatt, 3 Seiten, 632 Zeichen
\newline{}Handschrift: schwarze Tinte, deutsche Kurrent
\newline{}Schnitzler: mit Bleistift datiert: »23/3? 99« 
\newline{}Ordnung: 1) mit Bleistift von unbekannter Hand nummeriert: »\strikeout{144}«  2) mit Bleistift von unbekannter Hand nummeriert:
                                    »141«}\buchAbdrucke{\weitereDrucke{Hugo von Hofmannsthal, Arthur Schnitzler: \emph{Briefwechsel}. Hg. Therese Nickl und Heinrich Schnitzler. Frankfurt am Main: \emph{S. Fischer} 1964, S. 120.} }\toendnotes[C]{\smallbreak}\pstart
           \raggedleft{}{\pb}Berlin, Windsor Behrenſtraße\oindex{Hotel Windsor@\textbf{Hotel Windsor}|pw}\pend
           \pstart{}Mein guter lieber Arthur\pend\pstart
           Könnten Sie nicht hierher\oindex{Berlin@\textbf{Berlin}|pwv} ko{\geminationm}en? wir könnten ſehr viel beiſammen ſein und auch ſonſt
               ſieht man viele ernſte und liebenswürdige Menſchen und es wäre Ihnen doch leichter,
               ſich ein biſſl in die Höh zu bringen, als in Wien\oindex{Wien@\textbf{Wien}|pw},
               wo die Erinnerung\pwindex{Reinhard, Marie 1871-03-13 – 1899-03-18@\textsc{Reinhard, Marie} (1871-03-13 – 1899-03-18), \emph{Gesangspädagogin}|pwv} Ihnen bei
               jedem Schritt {\pb}friſch weh thut.
               Ich ſehne mich ſehr, mit Ihnen zu ſprechen, zu ſchreiben bin ich nicht im Stand.\pend
           \pstart
           Daſs dieſe Erinnerung immer mit meinen erſten Stücken\pwindex{Hofmannsthal, Hugo von 1874-02-01 – 1929-07-15@\textsc{Hofmannsthal, Hugo von} (1874-02-01 – 1929-07-15), \emph{Schriftsteller}!Hochzeit der Sobeide1899-03-18@\strich\emph{Die Hochzeit der Sobeide} {[}1899-03-18{]}|pwv}\pwindex{Hofmannsthal, Hugo von 1874-02-01 – 1929-07-15@\textsc{Hofmannsthal, Hugo von} (1874-02-01 – 1929-07-15), \emph{Schriftsteller}!Abenteurer und die Saengerin oder Die Geschenke des Lebens18. 3. 1899@\strich\emph{Der Abenteurer und die Sängerin oder Die Geschenke des Lebens} {[}18. 3. 1899{]}|pwv} verknüpft bleiben
               muſs!\pend
           \pstart
           Von Herzen Ihr{\\[\baselineskip]}\spacefill\mbox{Hugo.}\pend
           \leftskip=0em{}\pstart
           \noindent{}P. S. \label{K_L00909_1v}\edtext{Hier}{\lemma{\textnormal{\emph{Hier}}}\Cendnote{\textnormal{Die Uraufführung im \emph{Deutschen Theater}\orgindex{Deutsches Theater Berlin@Deutsches Theater Berlin|pwk} war am 18. 3. 1899 und damit zugleich mit der
                        Wien\oindex{Wien@\textbf{Wien}|pwk}er Uraufführung angesetzt.}}}\label{K_L00909_1h}{ }ſind meine armen Stücke\pwindex{Hofmannsthal, Hugo von 1874-02-01 – 1929-07-15@\textsc{Hofmannsthal, Hugo von} (1874-02-01 – 1929-07-15), \emph{Schriftsteller}!Hochzeit der Sobeide1899-03-18@\strich\emph{Die Hochzeit der Sobeide} {[}1899-03-18{]}|pwv}\pwindex{Hofmannsthal, Hugo von 1874-02-01 – 1929-07-15@\textsc{Hofmannsthal, Hugo von} (1874-02-01 – 1929-07-15), \emph{Schriftsteller}!Abenteurer und die Saengerin oder Die Geschenke des Lebens18. 3. 1899@\strich\emph{Der Abenteurer und die Sängerin oder Die Geschenke des Lebens} {[}18. 3. 1899{]}|pwv} von einer beiſpiellos
                  böſen {\pb}Preſſe erſchlagen worden
                  und muſsten nach dem dritten Mal abgeſetzt werden.\pend
           
         
         \endnumbering\mylabel{h}\end{ledgroupsized}  \newcommand{\dateiname}{L00909}\newcommand{\titel}{Hugo von Hofmannsthal an Arthur Schnitzler, [23. 3. 1899]}\newcommand{\editorInnen}{Martin Anton Müller und Gerd-Hermann Susen}%% latex-leseansicht-abspann.tex
%% Abspann für die Leseansicht.
%% Der Schalter \ifkorrekturansicht ist bereits durch den Vorspann gesetzt.

%% latex-abspann.tex
%% Gemeinsamer Abspann für Korrekturansicht und Leseansicht.
%% Setzt den Schalter \ifkorrekturansicht voraus (gesetzt in den
%% einbindenden Dateien latex-korrekturansicht-abspann.tex bzw.
%% latex-leseansicht-abspann.tex).
%% ---------------------------------------------------------------

\normalsize

% Das esempio-Environment wird nur in der Leseansicht benötigt
\ifkorrekturansicht\else
\newenvironment{esempio}[3]%
{
    \vspace{1.5ex}
    \rlap{\underline{#1}}
    \par
    \setlength{\parindent}{0cm}
    \nopagebreak
    \leftskip=#2cm
    \rightskip=#3cm
}
{
    \par
}
\fi

\doendnotes{C}
\bigskip
\vfill

\clearpage

\footnotesize

\ifkorrekturansicht
  \lohead{\textsc{register}}
\fi

% theindex-Environment neu definieren ohne reledmac
\makeatletter
\renewenvironment{theindex}{%
  \ifkorrekturansicht
    \section*{\indexname}%
  \else
    \subsubsection*{Index der erwähnten Entitäten}%
  \fi
  \setlength{\parindent}{0pt}%
  \setlength{\parskip}{0pt plus 0.3pt}%
  \let\item\@idxitem
}{%
  \ifkorrekturansicht\clearpage\fi
}
\makeatother

\IfFileExists{\jobname-pw.ind}{\input{\jobname-pw.ind}}{}

% Quellenangabe nur in der Leseansicht
\ifkorrekturansicht\else
% Fallback-Definitionen, falls die .tex-Datei \titel etc. nicht gesetzt hat
\providecommand{\titel}{}
\providecommand{\editorInnen}{}
\providecommand{\dateiname}{\jobname}

\vspace{3cm}

\vfill

\footnotesize
\textsc{Quelle}: \titel. Herausgegeben von {\editorInnen}. In: \emph{Arthur Schnitzler: Briefwechsel mit Autorinnen und Autoren}.
 Digitale Edition, https://schnitzler-briefe.acdh.oeaw.ac.at/{\dateiname}.html (Stand \today)
\fi

\end{document}


      