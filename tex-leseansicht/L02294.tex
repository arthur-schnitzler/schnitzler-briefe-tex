%% latex-leseansicht-vorspann.tex
%% Vorspann für die Leseansicht.
%% Lädt die gemeinsame Datei latex-vorspann.tex mit nicht gesetztem Schalter.

\newif\ifkorrekturansicht
\korrekturansichtfalse

\input{../tex-inputs/latex-vorspann}


\section[Robert Adam an Arthur Schnitzler, 10. 8. 1918]{L02294 Robert Adam an Arthur Schnitzler, 10. 8. 1918}
\nopagebreak\mylabel{L02294v}
\rehead{ }\normalsize\beginnumbering\briefempfaengerindex{Schnitzler, Arthur@\textsc{Schnitzler, Arthur}!zzzAdam, Robert@\emph{von Robert Adam}!1918-08-101@{10. 8. 1918}|(be}
\toendnotes[C]{\smallbreak\pagebreak[2]}
\correspDesc{Versand  durch Robert Adam am 10. 8. 1918 in Wien
\newline{}Erhalt  durch Arthur Schnitzler im Zeitraum [10. 8. 1918
                  – 14. 8. 1918?] in Wien}\toendnotes[C]{\smallbreak}
\Standort{CUL, Schnitzler, B 1.}
\physDesc{Brief, 1 Blatt, 2 Seiten, 1083 Zeichen
\newline{}Handschrift: schwarze Tinte, deutsche Kurrent
\newline{}Schnitzler: 1) mit Bleistift beschriftet: »\textsc{Adam}«  2) mit rotem Buntstift eine Unterstreichung
\newline{}Ordnung: von unbekannter Hand nummeriert: »6« }\Standort{Wien, Österreichische Nationalbibliothek, Cod.ser. 52.263, 209 verso.}
\physDesc{Brief, maschinenschriftliche Abschrift, 1 Blatt, 1 Seite, 1083 Zeichen
\newline{}Schreibmaschine}\toendnotes[C]{\smallbreak}
\pstart
           \centering{}{\pb}Wien\oindex{Wien@\textbf{Wien}, \emph{Verwaltungsgebiet}|pw}{ }10/8 1918\pend
           
\pstart\center{}Hochverehrter Herr Doktor!\pend\vspace{0.5em}
\pstart
           Ich{ }ſende Ihnen ein kleines Verzeichnis von Büchern über jugendliche Verbrecher, die
               ich dem Katalog der »Privatbibliothek der
                  Juſtizbeamten\orgindex{Privatbibliothek der Wiener Justizbeamten@Privatbibliothek der Wiener Justizbeamten|pw}« entnehme. Dieſe Bücher – wenn auch nur nach und nach – könnte
               ich Ihnen beſchaffen. Die Bibliothek\orgindex{Privatbibliothek der Wiener Justizbeamten@Privatbibliothek der Wiener Justizbeamten|pwv} enthält aber gewiß – da{ }ſie an kriminaliſtiſchen Werken{ }ſehr
               reichhaltig iſt – noch viele andere Bücher, die das Sie intereſſierende Thema
               behandeln; der Katalog iſt aber äußerſt{ }ſchlecht angelegt, die Titel{ }ſind oft
               unrichtig oder {\pb}unvollſtändig angegeben.
               Wenn ich wieder einmal vormittags einige freie Zeit erübrige, durchſtöbere ich die
                  Bibliothek\orgindex{Privatbibliothek der Wiener Justizbeamten@Privatbibliothek der Wiener Justizbeamten|pwv}{ }ſelbſt und{ }ſchlage insbeſondere in den
               Inhaltsverzeichniſſen der kriminaliſtiſchen Zeitſchriften nach; es{ }ſollte mich dann{ }ſehr wundern, wenn{ }ſich nicht Arbeiten fänden – insbeſondere auch Wiedergabe
               konkreter Rechtsfälle –, die Ihnen von Nutzen{ }ſein könnten.\pend
           
\pstart
           Die weniger in Betracht kommenden Bücher habe ich eingeklammert.\pend
           
\pstart
           Auch die Abteilung: »Pſychiatrie und Kriminalpſychologie« unſerer Bibliothek\orgindex{Privatbibliothek der Wiener Justizbeamten@Privatbibliothek der Wiener Justizbeamten|pwv} iſt ziemlich reichhaltig.\pend
           
\pstart
           Mit ergebenſten Grüßen\pend
           
\pstart
           Ihr{\\[\baselineskip]}\spacefill\mbox{D\textsuperscript{r}RAdam}\pend
           \leftskip=0em{}\selectlanguage{ngerman}\endnumbering\briefempfaengerindex{Schnitzler, Arthur@\textsc{Schnitzler, Arthur}!zzzAdam, Robert@\emph{von Robert Adam}!1918-08-101@{10. 8. 1918}|)be}\mylabel{L02294h}  \newcommand{\dateiname}{L02294}\newcommand{\titel}{Robert Adam an Arthur Schnitzler, 10. 8. 1918}\newcommand{\editorInnen}{Martin Anton Müller und Gerd-Hermann Susen}%% latex-leseansicht-abspann.tex
%% Abspann für die Leseansicht.
%% Der Schalter \ifkorrekturansicht ist bereits durch den Vorspann gesetzt.

%% latex-abspann.tex
%% Gemeinsamer Abspann für Korrekturansicht und Leseansicht.
%% Setzt den Schalter \ifkorrekturansicht voraus (gesetzt in den
%% einbindenden Dateien latex-korrekturansicht-abspann.tex bzw.
%% latex-leseansicht-abspann.tex).
%% ---------------------------------------------------------------

\normalsize

% Das esempio-Environment wird nur in der Leseansicht benötigt
\ifkorrekturansicht\else
\newenvironment{esempio}[3]%
{
    \vspace{1.5ex}
    \rlap{\underline{#1}}
    \par
    \setlength{\parindent}{0cm}
    \nopagebreak
    \leftskip=#2cm
    \rightskip=#3cm
}
{
    \par
}
\fi

\doendnotes{C}
\bigskip
\vfill

\clearpage

\footnotesize

\ifkorrekturansicht
  \lohead{\textsc{register}}
\fi

% theindex-Environment neu definieren ohne reledmac
\makeatletter
\renewenvironment{theindex}{%
  \ifkorrekturansicht
    \section*{\indexname}%
  \else
    \subsubsection*{Index der erwähnten Entitäten}%
  \fi
  \setlength{\parindent}{0pt}%
  \setlength{\parskip}{0pt plus 0.3pt}%
  \let\item\@idxitem
}{%
  \ifkorrekturansicht\clearpage\fi
}
\makeatother

\IfFileExists{\jobname-pw.ind}{\input{\jobname-pw.ind}}{}

% Quellenangabe nur in der Leseansicht
\ifkorrekturansicht\else
% Fallback-Definitionen, falls die .tex-Datei \titel etc. nicht gesetzt hat
\providecommand{\titel}{}
\providecommand{\editorInnen}{}
\providecommand{\dateiname}{\jobname}

\vspace{3cm}

\vfill

\footnotesize
\textsc{Quelle}: \titel. Herausgegeben von {\editorInnen}. In: \emph{Arthur Schnitzler: Briefwechsel mit Autorinnen und Autoren}.
 Digitale Edition, https://schnitzler-briefe.acdh.oeaw.ac.at/{\dateiname}.html (Stand \today)
\fi

\end{document}


