%% latex-korrekturansicht-vorspann.tex
%% Vorspann für die Korrekturansicht.
%% Lädt die gemeinsame Datei latex-vorspann.tex mit gesetztem Schalter.

\newif\ifkorrekturansicht
\korrekturansichttrue

\input{../tex-inputs/latex-vorspann}


\section[Theodor Herzl an Arthur Schnitzler, 10. 10. 1892]{L03825 Theodor Herzl an Arthur Schnitzler, 10. 10. 1892}
\nopagebreak\mylabel{L03825v}
\rehead{ }\normalsize\beginnumbering\briefempfaengerindex{Schnitzler, Arthur@\textsc{Schnitzler, Arthur}!zzzHerzl, Theodor@\emph{von Theodor Herzl}!1892-10-101@{10. 10. 1892}|(be}
\toendnotes[C]{\smallbreak\pagebreak[2]}\Standort{CUL, Schnitzler, B 39.}
\physDesc{Brief, 1 Blatt, 2 Seiten, 1405 Zeichen
\newline{}Handschrift: schwarze Tinte, lateinische Kurrent
\newline{}Schnitzler: mit Bleistift beschriftet »\noindent{}92, Ende{ / }od Anf 9\textcolor{gray}{3}«  
\newline{}Ordnung: mit Bleistift von unbekannter Hand nummeriert: »6« }\toendnotes[C]{\smallbreak}
\pstart
           {\pb}\textcolor{gray}{\textbf{NOUVELLE PRESSE LIBRE}}\orgindex{Neue Freie Presse@Neue Freie Presse|pw}\hfill \textcolor{gray}{\textbf{8, Rue de Monceau }}\oindex{8, Rue de Monceau@\textbf{8, Rue de Monceau}, \emph{Wohngebäude (K.WHS)}|pw}\pend
           
\pstart
           \textcolor{gray}{\textbf{D\textsuperscript{R} TH. HERZL}}\pend
           
\pstart{}Verehrtester Freund,\pend\vspace{0.5em}
\pstart
           Sie müssen mich schon für sehr ungezogen gehalten haben, und ich war nur beschäftigt. \pend
           
\pstart
           Besten Dank für die Uebersendung Ihres Buches\pwindex{Anatol@\emph{Anatol}|pwv}. Ich wollte Ihnen erst schreiben, nachdem ich es
               ausgelesen hätte. Gestern hab ich es angefangen und bisher drei Stücke\pwindex{Frage an das Schicksal@\emph{Die Frage an das Schicksal}|pwv}\pwindex{Weihnachts-Einkaeufe@\emph{Weihnachts-Einkäufe}|pwv}\pwindex{Episode@\emph{Episode}|pwv}
               gelesen. Inzwischen ist mir der \label{K_L03825-1v}\edtext{sybaritische}{\lemma{\textnormal{\emph{sybaritische}}}\Cendnote{\textnormal{schwelgerisch
                  (zurückgehend auf die antike Stadt Sybaris, die mit legendärem Reichtum gesegnet
                  war)}}}\label{K_L03825-1} Einfall gekommen, in den seltenen halben Stunden, wo ich zum Träumeln
               Zeit habe, immer nur eins Ihrer Stückchen\pwindex{Denksteine@\emph{Denksteine}|pwv}\pwindex{Agonie@\emph{Agonie}|pwv}\pwindex{Anatols Hochzeitsmorgen@\emph{Anatols Hochzeitsmorgen}|pwv}\pwindex{Abschiedssouper@\emph{Abschiedssouper}|pwv} zu
               lesen. So wart ich also nicht, bis ich zu Ende bin, um Ihnen zu danken. \pend
           
\pstart
           Die erste Geschichte\pwindex{Frage an das Schicksal@\emph{Die Frage an das Schicksal}|pwv} (Frage an das {\pb}Schicksal\pwindex{Frage an das Schicksal@\emph{Die Frage an das Schicksal}|pw}) finde ich sehr
               gelungen. Ich kannte sie schon – woher nur? Aus der \label{K_L03825-2v}\edtext{Brünner\oindex{Bruenn@\textbf{Brünn}, \emph{P.PPLA}|pw}{ }Monatsschrift\pwindex{Moderne Dichtung. Monatsschrift fuer Literatur und Kritik@\emph{Moderne Dichtung. Monatsschrift für Literatur und Kritik}|pwv}}{\lemma{\textnormal{\emph{Brünner Monatsschrift}}}\Cendnote{\textnormal{Der Einakter \emph{Die Frage an das Schicksal}\pwindex{Frage an das Schicksal@\emph{Die Frage an das Schicksal}|pwk} erschien zuerst in: \emph{Moderne Dichtung}\pwindex{Moderne Dichtung. Monatsschrift fuer Literatur und Kritik@\emph{Moderne Dichtung. Monatsschrift für Literatur und Kritik}|pwk}, Bd. 1, H. 5, 1. 5. 1890, S. 299–306.}}}\label{K_L03825-2} (die man Leichtes \uline{Tuch} nennen könnte) oder anderswoher?\pend
           
\pstart
           In der zweiten\pwindex{Weihnachts-Einkaeufe@\emph{Weihnachts-Einkäufe}|pwv}, die mir zu
               lang ausgesponnen scheint, erwartete ich eine andere weiberkundigere Pointe. Die Frau
               erfährt, dass das »Mädl« nichts Anderes auf der Welt hat, als den Anatol – \uline{darum} nimmt sie ihr ihn weg. Heh?\pend
           
\pstart
           Ich habe nicht Zeit genug, Ihnen alles Gute zu sagen, was ich über die dritte Episode\pwindex{Episode@\emph{Episode}|pwv} denke.\pend
           
\pstart
           Wer ist Loris\pwindex{Hofmannsthal, Hugo von 1874-02-01 – 1929-07-15@\textsc{Hofmannsthal, Hugo von} (1874-02-01 – 1929-07-15), \emph{Schriftsteller/Schriftstellerin}|pwv}? Auch Sie?
               Jedenfalls sind diese paar Verse\pwindex{Prolog [zum Anatol]@\emph{Prolog [zum Anatol]}|pwv} zum Küssen. Schreiben Sie mir, wer Loris\pwindex{Hofmannsthal, Hugo von 1874-02-01 – 1929-07-15@\textsc{Hofmannsthal, Hugo von} (1874-02-01 – 1929-07-15), \emph{Schriftsteller/Schriftstellerin}|pwv} ist. Ergreifen Sie überhaupt Ihre gute \label{K_L03825-3v}\edtext{Feder von Toledo\oindex{Toledo@\textbf{Toledo}, \emph{A.ADM3}|pw}}{\lemma{\textnormal{\emph{Feder von Toledo}}}\Cendnote{\textnormal{Toledostahl war seit der Antike berühmt für seine Härte und Eignung zum Schmieden von Waffen}}}\label{K_L03825-3} und erzählen Sie mir, was Wien\oindex{Wien@\textbf{Wien}, \emph{A.ADM2}|pw}{ }\label{K_L03825-4v}\edtext{\begin{otherlanguage}{french}en l'an de grâce\end{otherlanguage}}{\lemma{\textnormal{\emph{en l'an de grâce}}}\Cendnote{\textnormal{französisch: im Jahr der Gnade}}}\label{K_L03825-4}{ }1892 ist. Recht ausführlich, denn Sie haben Zeit, Sie vielleicht
               Glücklicher.\pend
           
\pstart
           Ich grüsse Sie recht herzlich und ergeben Ihr{\\[\baselineskip]}\spacefill\mbox{Herzl}\pend
           \leftskip=0em{}
\pstart
           10/X 92\pend
           
\pstart
           \label{T_L03825-1v}\edtext{Erzählen Sie mir was es in Kunst u.
                  Zeitung in Wien\oindex{Wien@\textbf{Wien}, \emph{A.ADM2}|pw} gibt. Ich kenne Alles nur aus
                  den Journalen}{\lemma{\textnormal{\emph{Erzählen … Journalen}}}\Cendnote{\textnormal{seitlich entlang des
                     Falzes geschrieben}}}\label{T_L03825-1}\pend
           \selectlanguage{ngerman}\endnumbering\briefempfaengerindex{Schnitzler, Arthur@\textsc{Schnitzler, Arthur}!zzzHerzl, Theodor@\emph{von Theodor Herzl}!1892-10-101@{10. 10. 1892}|)be}\mylabel{L03825h}
\begin{anhang}
\end{anhang}\normalsize

\doendnotes{C}
\bigskip
\vfill

\clearpage

\footnotesize

\lohead{\textsc{register}}

% Definiere theindex-Environment komplett neu ohne reledmac
\makeatletter
\renewenvironment{theindex}{%
  \section*{\indexname}%
  \setlength{\parindent}{0pt}%
  \setlength{\parskip}{0pt plus 0.3pt}%
  \let\item\@idxitem
}{%
  \clearpage
}
\makeatother

\IfFileExists{\jobname-pw.ind}{\input{\jobname-pw.ind}}{}

\end{document}

      