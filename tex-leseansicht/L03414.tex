%% latex-korrekturansicht-vorspann.tex
%% Vorspann für die Korrekturansicht.
%% Lädt die gemeinsame Datei latex-vorspann.tex mit gesetztem Schalter.

\newif\ifkorrekturansicht
\korrekturansichttrue

\input{../tex-inputs/latex-vorspann}


\section[ Felix Salten an Arthur Schnitzler, 24. 3. 1906]{L03414 Felix Salten an Arthur Schnitzler, 24. 3. 1906}
\nopagebreak\mylabel{L03414v}
\rehead{ }\normalsize\beginnumbering\briefempfaengerindex{Schnitzler, Arthur@\textsc{Schnitzler, Arthur}!zzzSalten, Felix@\emph{von Felix Salten}!1906-03-241@{24. 3. 1906}|(be}
\toendnotes[C]{\smallbreak\pagebreak[2]}\Standort{CUL, Schnitzler, B 89, B 1.}
\physDesc{Brief, 1 Blatt, 1 Seite, 873 Zeichen (Briefpapier mit Trauerrand)
\newline{}Handschrift: schwarze Tinte, lateinische Kurrent
\newline{}Ordnung: mit Bleistift von unbekannter Hand nummeriert: »205« }\toendnotes[C]{\smallbreak}
\pstart
           \raggedleft{}{\pb}Berlin\oindex{Berlin@\textbf{Berlin}, \emph{P.PPLC}|pw}, 24. III. 06.\pend
           \vspace{0.5em}
\pstart
           Lieber, in Eile und Arbeit nur ganz kurz: gegen das \label{K_L03414-1v}\edtext{»Kleine
                  Theater\orgindex{Kleines Theater@Kleines Theater|pw}«}{\lemma{\textnormal{\emph{»Kleine
                  Theater«}}}\Cendnote{\textnormal{Er beantwortet die
                  Frage, ob es für eine Inszenierung von \emph{Zum großen
                     Wurstel}\pwindex{Zum grossen Wurstel. Burleske in einem Akt@\emph{Zum großen Wurstel. Burleske in einem Akt}|pwk} infrage käme, vgl. A. S.: \emph{Tagebuch}, 25. 3. 1906.
               }}}\label{K_L03414-1} bin ich unbedingt. Es ist mit seinem jetzigen Bestand an Schauspielern, und
               der retorischen Unfähigkeit des Herrn D\textsuperscript{r}{ }Oberländer\pwindex{Oberlaender, Hans 1870-03-14 – 1942-01-30@\textsc{Oberländer, Hans} (1870-03-14 – 1942-01-30), \emph{Regisseur/Regisseurin, Theaterwissenschaftler/Theaterwissenschaftlerin, Filmregisseur/Filmregisseurin}|pw} garnicht imstande ein so
               stilisirtes und in seinen Reizen vom Dutzend-Regisseur so schwer auffindbares Stück\pwindex{Zum grossen Wurstel. Burleske in einem Akt@\emph{Zum großen Wurstel. Burleske in einem Akt}|pwuv} zu
               reproduziren. Ich hielte es für aussichtslos. Auch wäre, bei der jetzigen Conjunctur
               von so einem Experiment nur abzurathen. Besser, Sie warten auf Reinhardts\pwindex{Reinhardt, Max 09.09.1873 – 30.10.1943@\textsc{Reinhardt, Max} (09.09.1873 – 30.10.1943), \emph{Theaterleiter/Theaterleiterin, Regisseur/Regisseurin, Schauspieler/Schauspielerin}|pw}{ }\label{K_L03414-2v}\edtext{»intimes Theater\orgindex{Kammerspiele Berlin@Kammerspiele Berlin|pwv}«, das im nächsten Jahr bestehen und von 
               Bahr\pwindex{Bahr, Hermann 19.07.1863 – 15.01.1934@\textsc{Bahr, Hermann} (19.07.1863 – 15.01.1934), \emph{Schriftsteller/Schriftstellerin, Kritiker/Kritikerin}|pw} geleitet}{\lemma{\textnormal{\emph{»intimes … geleitet}}}\Cendnote{\textnormal{Bei der Eröffnung hieß das intime Theater\orgindex{Kammerspiele Berlin@Kammerspiele Berlin|pwkv}{ }Max Reinhardts\pwindex{Reinhardt, Max 09.09.1873 – 30.10.1943@\textsc{Reinhardt, Max} (09.09.1873 – 30.10.1943), \emph{Theaterleiter/Theaterleiterin, Regisseur/Regisseurin, Schauspieler/Schauspielerin}|pwk}{ }\emph{Kammerspiele}\orgindex{Kammerspiele Berlin@Kammerspiele Berlin|pwk}. Als kleinere Bühne sollte sie für experimentellere und anspruchsvollere Stücke Verwendung finden. Bahr\pwindex{Bahr, Hermann 19.07.1863 – 15.01.1934@\textsc{Bahr, Hermann} (19.07.1863 – 15.01.1934), \emph{Schriftsteller/Schriftstellerin, Kritiker/Kritikerin}|pwk} arbeitete zwar in Folge für mehrere Inszenierungen als Regisseur
                  bei Reinhardt\pwindex{Reinhardt, Max 09.09.1873 – 30.10.1943@\textsc{Reinhardt, Max} (09.09.1873 – 30.10.1943), \emph{Theaterleiter/Theaterleiterin, Regisseur/Regisseurin, Schauspieler/Schauspielerin}|pwk}, doch tatsächliche
                  Verantwortung als Theaterleiter bekam er nicht übertragen. Nach vier Aufenthalten zwischen November 1906
               und März 1908 endete die Zusammenarbeit.}}}\label{K_L03414-2} wird. Folgen Sie
               mir!\pend
           
\pstart
           Ich schreibe bald und mehr. Dass wir einander wieder herzlich nah sind, empfinde ich
               auch, und es hat mir meinen Abgang von Wien\oindex{Wien@\textbf{Wien}, \emph{A.ADM2}|pw}
               erschwert. Dass etwas Unverlierbares, an das jederzeit ohneweiters angeknüpft werden
               kann, uns verbindet, hab ich immer geglaubt. Viele Grüße von Otti\pwindex{Salten, Ottilie 07.03.1868 – 22.06.1942@\textsc{Salten, Ottilie} (07.03.1868 – 22.06.1942), \emph{Schauspieler/Schauspielerin}|pw} u. mir an Sie Beide\pwindex{Schnitzler, Olga 17.01.1882 – 13.01.1970@\textsc{Schnitzler, Olga} (17.01.1882 – 13.01.1970), \emph{Schauspieler/Schauspielerin, Sänger/Sängerin}|pwv}.\pend
           \pstart Ihr \spacefill\mbox{Salten}\pend{}\selectlanguage{ngerman}\endnumbering\briefempfaengerindex{Schnitzler, Arthur@\textsc{Schnitzler, Arthur}!zzzSalten, Felix@\emph{von Felix Salten}!1906-03-241@{24. 3. 1906}|)be}\mylabel{L03414h}  \normalsize

\doendnotes{C}
\bigskip
\vfill

\clearpage

\footnotesize

\lohead{\textsc{register}}

% Definiere theindex-Environment komplett neu ohne reledmac
\makeatletter
\renewenvironment{theindex}{%
  \section*{\indexname}%
  \setlength{\parindent}{0pt}%
  \setlength{\parskip}{0pt plus 0.3pt}%
  \let\item\@idxitem
}{%
  \clearpage
}
\makeatother

\IfFileExists{\jobname-pw.ind}{\input{\jobname-pw.ind}}{}

\end{document}

      