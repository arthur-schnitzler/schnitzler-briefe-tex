%% latex-leseansicht-vorspann.tex
%% Vorspann für die Leseansicht.
%% Lädt die gemeinsame Datei latex-vorspann.tex mit nicht gesetztem Schalter.

\newif\ifkorrekturansicht
\korrekturansichtfalse

\input{../tex-inputs/latex-vorspann}

\begin{center}
            \textcolor{red}{ENTWURF, NICHT FERTIG KORRIGIERT}
                      \end{center}
            
         
         \renewcommand{\erwaehntePersonen}{Personen: Hermann Bahr, Hans Oberländer, Max Reinhardt, Ottilie Salten, Olga Schnitzler}
         \renewcommand{\erwaehnteInstitutionen}{Institutionen: Deutsches Theater Berlin, Kleines Theater}
         \renewcommand{\erwaehnteOrte}{Orte: Berlin, Wien}
         \renewcommand{\erwaehnteWerke}{Werke: Zum großen Wurstel. Burleske in einem Akt}
               \section[Felix Salten an Arthur Schnitzler, 24. 3. 1906]{ Felix Salten an Arthur Schnitzler, 24. 3. 1906}\nopagebreak\mylabel{v}\rehead{ }\begin{ledgroupsized}[t]{13cm}\normalsize\beginnumbering \toendnotes[C]{\smallbreak\pagebreak[2]} \Standort{CUL, Schnitzler, B 89, B 1.}
\physDesc{Brief, 1 Blatt, 1 Seite
\newline{}Handschrift: schwarze Tinte, lateinische Kurrent\newline{}Ordnung: mit Bleistift von unbekannter Hand nummeriert:
                                    »205« }\toendnotes[C]{\smallbreak}\pstart
           \raggedleft{}{\pb}Berlin\oindex{Berlin@\textbf{Berlin}|pw}, 24. III. 06.
               \pend
           \pstart
           Lieber, in Eile und Arbeit nur ganz kurz: gegen das »Kleine Theater\orgindex{Kleines Theater@Kleines Theater|pw}« bin ich unbedingt. Es ist mit seinem jetzigen
               Bestand an Schauspielern, und der retorischen Unfähigkeit des Herrn D\textsuperscript{r}{ }Oberländer\pwindex{Oberlaender, Hans 1870-03-14 – 1942-01-30@\textsc{Oberländer, Hans} (1870-03-14 – 1942-01-30), \emph{Regisseur, Theaterwissenschaftler, Filmregisseur}|pw} garnicht imstande ein so
               stilisirtes und in seinen Reizen vom Dutzend-Regisseur so schwer auffindbares Stück\pwindex{Schnitzler, Arthur 15.05.1862 – 21.10.1931@\textsc{Schnitzler, Arthur} (15.05.1862 – 21.10.1931), \emph{Schriftsteller, Mediziner}!Zum grossen Wurstel. Burleske in einem Akt08. 03. 1901@\strich\emph{Zum großen Wurstel. Burleske in einem Akt} {[}08. 03. 1901{]}|pwuv} zu reproduziren. Ich hielte es für
               aussichtslos. Auch wäre, bei der jetzigen Conjunctur von so einem Experiment nur
               abzurathen. Besser, Sie warten auf Reinhardt\pwindex{Reinhardt, Max 09.09.1873 – 30.10.1943@\textsc{Reinhardt, Max} (09.09.1873 – 30.10.1943), \emph{Theaterleiter, Regisseur, Schauspieler}|pw}s
                  »intimes Theater\orgindex{Deutsches Theater Berlin@Deutsches Theater Berlin|pw}«, das im nächsten Jahr
               bestehen und von \label{K_L03414-1v}\edtext{Bahr\pwindex{Bahr, Hermann 19.07.1863 – 15.01.1934@\textsc{Bahr, Hermann} (19.07.1863 – 15.01.1934), \emph{Schriftsteller, Kritiker}|pw} geleitet}{\lemma{\textnormal{\emph{Bahr geleitet}}}\Cendnote{\textnormal{vgl. A. S.: \emph{Tagebuch}, 19. 3. 1906. Bahr\pwindex{Bahr, Hermann 19.07.1863 – 15.01.1934@\textsc{Bahr, Hermann} (19.07.1863 – 15.01.1934), \emph{Schriftsteller, Kritiker}|pwk} arbeitete zwar in Folge als Regisseur
                  bei Reinhardt\pwindex{Reinhardt, Max 09.09.1873 – 30.10.1943@\textsc{Reinhardt, Max} (09.09.1873 – 30.10.1943), \emph{Theaterleiter, Regisseur, Schauspieler}|pwk}, doch tatsächliche
                  Verantwortung als Theaterleiter bekam er nicht übertragen.}}}\label{K_L03414-1h} wird. Folgen Sie
               mir! \pend
           \pstart
           Ich schreibe bald und mehr. Dass wir einander wieder herzlich nah sind, empfinde ich
               auch, und es hat mir meinen Abgang von Wien erschwert. Dass etwas Unverlierbares, an
               das jederzeit ohneweiters angeknüpft werden kann, uns verbindet, hab ich immer
               geglaubt. Viele Grüße von Otti\pwindex{Salten, Ottilie 07.03.1868 – 22.06.1942@\textsc{Salten, Ottilie} (07.03.1868 – 22.06.1942), \emph{Schauspielerin}|pw} u. mir an Sie
                  Beide\pwindex{Schnitzler, Olga 17.01.1882 – 13.01.1970@\textsc{Schnitzler, Olga} (17.01.1882 – 13.01.1970), \emph{Schauspielerin, Sängerin}|pwv}. \pend
           \pstart
           {\\[\baselineskip]}Ihr \spacefill\mbox{Salten}\pend
           \leftskip=0em{}
         
         \endnumbering\mylabel{h}\end{ledgroupsized}\begin{anhang}\end{anhang}\newcommand{\dateiname}{L03414}\newcommand{\titel}{Felix Salten an Arthur Schnitzler, 24. 3. 1906}\newcommand{\editorInnen}{Martin Anton Müller und Laura Untner}%% latex-leseansicht-abspann.tex
%% Abspann für die Leseansicht.
%% Der Schalter \ifkorrekturansicht ist bereits durch den Vorspann gesetzt.

%% latex-abspann.tex
%% Gemeinsamer Abspann für Korrekturansicht und Leseansicht.
%% Setzt den Schalter \ifkorrekturansicht voraus (gesetzt in den
%% einbindenden Dateien latex-korrekturansicht-abspann.tex bzw.
%% latex-leseansicht-abspann.tex).
%% ---------------------------------------------------------------

\normalsize

% Das esempio-Environment wird nur in der Leseansicht benötigt
\ifkorrekturansicht\else
\newenvironment{esempio}[3]%
{
    \vspace{1.5ex}
    \rlap{\underline{#1}}
    \par
    \setlength{\parindent}{0cm}
    \nopagebreak
    \leftskip=#2cm
    \rightskip=#3cm
}
{
    \par
}
\fi

\doendnotes{C}
\bigskip
\vfill

\clearpage

\footnotesize

\ifkorrekturansicht
  \lohead{\textsc{register}}
\fi

% theindex-Environment neu definieren ohne reledmac
\makeatletter
\renewenvironment{theindex}{%
  \ifkorrekturansicht
    \section*{\indexname}%
  \else
    \subsubsection*{Index der erwähnten Entitäten}%
  \fi
  \setlength{\parindent}{0pt}%
  \setlength{\parskip}{0pt plus 0.3pt}%
  \let\item\@idxitem
}{%
  \ifkorrekturansicht\clearpage\fi
}
\makeatother

\IfFileExists{\jobname-pw.ind}{\input{\jobname-pw.ind}}{}

% Quellenangabe nur in der Leseansicht
\ifkorrekturansicht\else
% Fallback-Definitionen, falls die .tex-Datei \titel etc. nicht gesetzt hat
\providecommand{\titel}{}
\providecommand{\editorInnen}{}
\providecommand{\dateiname}{\jobname}

\vspace{3cm}

\vfill

\footnotesize
\textsc{Quelle}: \titel. Herausgegeben von {\editorInnen}. In: \emph{Arthur Schnitzler: Briefwechsel mit Autorinnen und Autoren}.
 Digitale Edition, https://schnitzler-briefe.acdh.oeaw.ac.at/{\dateiname}.html (Stand \today)
\fi

\end{document}


      