%% latex-leseansicht-vorspann.tex
%% Vorspann für die Leseansicht.
%% Lädt die gemeinsame Datei latex-vorspann.tex mit nicht gesetztem Schalter.

\newif\ifkorrekturansicht
\korrekturansichtfalse

\input{../tex-inputs/latex-vorspann}


\section[Paul Goldmann an Arthur Schnitzler, 3. 6. 1893]{L02709 Paul Goldmann an Arthur Schnitzler, 3. 6. 1893}
\nopagebreak\mylabel{L02709v}
\rehead{ }\normalsize\beginnumbering\briefempfaengerindex{Schnitzler, Arthur@\textsc{Schnitzler, Arthur}!zzzGoldmann, Paul@\emph{von Paul Goldmann}!1893-06-031@{3. 6. 1893}|(be}
\toendnotes[C]{\smallbreak\pagebreak[2]}
\correspDesc{Versand  durch Paul Goldmann am 3. 6. 1893 in Frankfurt am Main
\newline{}Erhalt  durch Arthur Schnitzler im Zeitraum [4. 6. 1893
                  – 8. 6. 1893?] in Wien}\toendnotes[C]{\smallbreak}
\Standort{DLA, A:Schnitzler, HS.NZ85.1.3163.}
\physDesc{Brief, 3 Blätter, 10 Seiten, 4086 Zeichen
\newline{}Handschrift: blaue Tinte, deutsche Kurrent
\newline{}Schnitzler: 1) mit Bleistift das erste Blatt mit »1.« nummeriert  2) mit rotem Buntstift eine Unterstreichung}\toendnotes[C]{\smallbreak}
\pstart
           {\pb}\textcolor{gray}{\textbf{\textbf{Frankfurter Zeitung}}}\orgindex{Frankfurter Zeitung@Frankfurter Zeitung|pw}\pend
           
\pstart
           \textcolor{gray}{\textbf{und}}\pend
           
\pstart
           \textcolor{gray}{\textbf{\textbf{Handelsblatt}.}}\hfill \textcolor{gray}{\textbf{Frankfurt a. M.\oindex{Frankfurt am Main@\textbf{Frankfurt am Main}, \emph{Hauptstadt}|pw},}}{ }3. Juni \textcolor{gray}{\textbf{189}}3.\pend
           
\pstart
           \textcolor{gray}{\textbf{\textbf{Redaktion.}\footnote{\noindent{}\textcolor{gray}{\textbf{Für die Redaktion\orgindex{Frankfurter Zeitung@Frankfurter Zeitung|pwv} bestimmte Briefe und Sendungen wolle
                              man \so{nicht} an die Person eines Redakteurs,
                              sondern stets \textbf{an die Redaktion\orgindex{Frankfurter Zeitung@Frankfurter Zeitung|pwv} der Frankfurter Zeitung\pwindex{Frankfurter Zeitung@\emph{Frankfurter Zeitung}|pw}} adressiren.}}}}}\pend
           
\pstart
           \textcolor{gray}{\textbf{\textbf{Telegramm-Adresse:}}}\pend
           
\pstart
           \textcolor{gray}{\textbf{\textbf{Zeitung Frankfurt
                        Main\oindex{Frankfurt am Main@\textbf{Frankfurt am Main}, \emph{Hauptstadt}|pw}.}}}\pend
           
\pstart\center{}Mein lieber Arthur!\pend\vspace{0.5em}
\pstart
           Ich bin für wenige Tage zum Beſuch in Frankfurt\oindex{Frankfurt am Main@\textbf{Frankfurt am Main}, \emph{Hauptstadt}|pw},
               um der \label{K_L02709-1v}\edtext{Hochzeit meiner Schweſter\pwindex{Rosengart, Vally 29.\,12.\,1866 Breslau – nach 1926@\textsc{Rosengart, Vally} (29.\,12.\,1866 Breslau – nach 1926)|pwv}}{\lemma{\textnormal{\emph{Hochzeit … Schwester}}}\Cendnote{\textnormal{Vally Goldmann\pwindex{Rosengart, Vally 29.\,12.\,1866 Breslau – nach 1926@\textsc{Rosengart, Vally} (29.\,12.\,1866 Breslau – nach 1926)|pwk} heiratete den in Laupheim\oindex{Laupheim@\textbf{Laupheim}|pwk} geborenen Arzt Josef Rosengart\pwindex{Rosengart, Josef 8.\,2.\,1860 Laupheim – 4.\,8.\,1927 Frankfurt am Main@\textsc{Rosengart, Josef} (8.\,2.\,1860 Laupheim – 4.\,8.\,1927 Frankfurt am Main), \emph{Arzt}|pwk}.}}}\label{K_L02709-1} beizuwohnen. Mein Onkel\pwindex{Mamroth, Fedor 21.\,2.\,1851 Breslau – 25.\,6.\,1907 Frankfurt am Main@\textsc{Mamroth, Fedor} (21.\,2.\,1851 Breslau – 25.\,6.\,1907 Frankfurt am Main), \emph{Journalist, Kritiker}|pwv}{ }ſpricht mir natürlich
               von Dir, erzählt mir mit wahrem Enthuſiasmus von Deinem Roman\pwindex{Schnitzler, Arthur 15.\,5.\,1862 Wien – 21.\,10.\,1931 ebd.@\textsc{Schnitzler, Arthur} (15.\,5.\,1862 Wien – 21.\,10.\,1931 ebd.), \emph{Schriftsteller, Mediziner}!Sterben. Novelle@\strich\emph{Sterben. Novelle}|pwv}, den er als ein bedeutendes Werk\pwindex{Schnitzler, Arthur 15.\,5.\,1862 Wien – 21.\,10.\,1931 ebd.@\textsc{Schnitzler, Arthur} (15.\,5.\,1862 Wien – 21.\,10.\,1931 ebd.), \emph{Schriftsteller, Mediziner}!Sterben. Novelle@\strich\emph{Sterben. Novelle}|pwv} bezeichnet, und zeigt mir{ }ſchließlich Deinen \label{K_L02709-2v}\edtext{Brief}{\lemma{\textnormal{\emph{Brief}}}\Cendnote{\textnormal{Nicht erhalten. In seinen Antwortbriefen
                  vom XXXX Auszeichnungsfehler: Dokument L00186 nicht gefunden und XXXX Auszeichnungsfehler: Dokument L00216 nicht gefunden lobte Fedor Mamroth\pwindex{Mamroth, Fedor 21.\,2.\,1851 Breslau – 25.\,6.\,1907 Frankfurt am Main@\textsc{Mamroth, Fedor} (21.\,2.\,1851 Breslau – 25.\,6.\,1907 Frankfurt am Main), \emph{Journalist, Kritiker}|pwk} jedoch ausdrücklich Schnitzlers Novelle \emph{Sterben}\pwindex{Schnitzler, Arthur 15.\,5.\,1862 Wien – 21.\,10.\,1931 ebd.@\textsc{Schnitzler, Arthur} (15.\,5.\,1862 Wien – 21.\,10.\,1931 ebd.), \emph{Schriftsteller, Mediziner}!Sterben. Novelle@\strich\emph{Sterben. Novelle}|pwk} (vgl. XXXX Auszeichnungsfehler: Dokument L00186 nicht gefunden). Gedruckt wurde \emph{Sterben}\pwindex{Schnitzler, Arthur 15.\,5.\,1862 Wien – 21.\,10.\,1931 ebd.@\textsc{Schnitzler, Arthur} (15.\,5.\,1862 Wien – 21.\,10.\,1931 ebd.), \emph{Schriftsteller, Mediziner}!Sterben. Novelle@\strich\emph{Sterben. Novelle}|pwk}
                  zuerst von Oktober bis Dezember 1894 in den Heften 10 bis 12 der \emph{Neuen Deutschen Rundschau}\pwindex{Neue Deutsche Rundschau@\emph{Neue Deutsche Rundschau}|pwk}.}}}\label{K_L02709-2}, es tief beklagend, daß
                  \label{K_L02709-3v}\edtext{zwiſchen Dich und ih\substVorne{}\textsuperscript{\textcolor{gray}{m}}\substDazwischen{}n\substHinten{} etwas getreten}{\lemma{\textnormal{\emph{zwischen … getreten}}}\Cendnote{\textnormal{Im Kern geht es,
                  wie aus dem Folgenden deutlich wird, um das Ausbleiben einer Rezension des \emph{Anatol}\pwindex{Schnitzler, Arthur 15.\,5.\,1862 Wien – 21.\,10.\,1931 ebd.@\textsc{Schnitzler, Arthur} (15.\,5.\,1862 Wien – 21.\,10.\,1931 ebd.), \emph{Schriftsteller, Mediziner}!Anatol@\strich\emph{Anatol}|pwk} in der \emph{Frankfurter Zeitung}\orgindex{Frankfurter Zeitung@Frankfurter Zeitung|pwk}. In einem größeren Zusammenhang könnte es auch eine
                  Kränkung Schnitzlers aufgrund der
                  wiederholten Ablehnungen Fedor Mamroths\pwindex{Mamroth, Fedor 21.\,2.\,1851 Breslau – 25.\,6.\,1907 Frankfurt am Main@\textsc{Mamroth, Fedor} (21.\,2.\,1851 Breslau – 25.\,6.\,1907 Frankfurt am Main), \emph{Journalist, Kritiker}|pwk} –
                  zuletzt \emph{Das Märchen}\pwindex{Schnitzler, Arthur 15.\,5.\,1862 Wien – 21.\,10.\,1931 ebd.@\textsc{Schnitzler, Arthur} (15.\,5.\,1862 Wien – 21.\,10.\,1931 ebd.), \emph{Schriftsteller, Mediziner}!Märchen. Schauspiel in drei Aufzügen@\strich\emph{Das Märchen. Schauspiel in drei Aufzügen}|pwk} und \emph{Sterben}\pwindex{Schnitzler, Arthur 15.\,5.\,1862 Wien – 21.\,10.\,1931 ebd.@\textsc{Schnitzler, Arthur} (15.\,5.\,1862 Wien – 21.\,10.\,1931 ebd.), \emph{Schriftsteller, Mediziner}!Sterben. Novelle@\strich\emph{Sterben. Novelle}|pwk} – gegeben haben. Der Brief Mamroths\pwindex{Mamroth, Fedor 21.\,2.\,1851 Breslau – 25.\,6.\,1907 Frankfurt am Main@\textsc{Mamroth, Fedor} (21.\,2.\,1851 Breslau – 25.\,6.\,1907 Frankfurt am Main), \emph{Journalist, Kritiker}|pwk} an Schnitzler vom XXXX Auszeichnungsfehler: Dokument L00135 nicht gefunden legt nahe, dass Schnitzler den ausbleibenden Kontakt nach der Ablehnung des \emph{Märchens}\pwindex{Schnitzler, Arthur 15.\,5.\,1862 Wien – 21.\,10.\,1931 ebd.@\textsc{Schnitzler, Arthur} (15.\,5.\,1862 Wien – 21.\,10.\,1931 ebd.), \emph{Schriftsteller, Mediziner}!Märchen. Schauspiel in drei Aufzügen@\strich\emph{Das Märchen. Schauspiel in drei Aufzügen}|pwk} als unhöflich empfunden hat.}}}\label{K_L02709-3} iſt, das beſſer nicht da wäre. Dein Brief, mein lieber Freund, iſt
                  {\pb}ebenſo an mich gerichtet, wie an meinen Onkel\pwindex{Mamroth, Fedor 21.\,2.\,1851 Breslau – 25.\,6.\,1907 Frankfurt am Main@\textsc{Mamroth, Fedor} (21.\,2.\,1851 Breslau – 25.\,6.\,1907 Frankfurt am Main), \emph{Journalist, Kritiker}|pwv}. Vieles von dem, was Du
               zu ihm{ }ſagſt, bezieht{ }ſich auch auf mich. Und ich kann mich von der Schuld nicht
               freiſprechen, ein wenig die Bitterkeit mitveranlaßt zu haben, von der ich Dich
               erfüllt{ }ſehe. Objectiv haſt Du vollſtändig Recht. Nun aber{ }ſubjektiv: Gewiß, wenn ein
               Menſch auf der Welt verpflichtet war, über »\textsc{Anatol\pwindex{Schnitzler, Arthur 15.\,5.\,1862 Wien – 21.\,10.\,1931 ebd.@\textsc{Schnitzler, Arthur} (15.\,5.\,1862 Wien – 21.\,10.\,1931 ebd.), \emph{Schriftsteller, Mediziner}!Anatol@\strich\emph{Anatol}|pw}}« zu{ }ſchreiben,{ }ſo war ich es. Das Buch\pwindex{Schnitzler, Arthur 15.\,5.\,1862 Wien – 21.\,10.\,1931 ebd.@\textsc{Schnitzler, Arthur} (15.\,5.\,1862 Wien – 21.\,10.\,1931 ebd.), \emph{Schriftsteller, Mediziner}!Anatol@\strich\emph{Anatol}|pwv} kam bei mir an in einer meiner{ }ſchwerſten Arbeitszeiten
               – Arbeit, von deren Wucht und Depreſ{\pb}ſionsmacht Du
               keinerlei Ahnung haben kannſt. Ich mußte es zurücklegen für{ }ſpäter. Und als dann das
               »ſpäter« kam, kam über mich das \label{K_L02709-4v}\edtext{Unheil}{\lemma{\textnormal{\emph{Unheil}}}\Cendnote{\textnormal{die Erkrankung an einer
                  Geschlechtskrankheit}}}\label{K_L02709-4}, das Du kennſt, mit der Unmöglichkeit, auch nur ein
               wenig Spannkraft zu finden, um aus dem mechaniſchen Trott der täglichen Arbeit
               herauszugehen und \strikeout{\textcolor{gray}{×}\-\textcolor{gray}{×}} ein Werk\pwindex{Schnitzler, Arthur 15.\,5.\,1862 Wien – 21.\,10.\,1931 ebd.@\textsc{Schnitzler, Arthur} (15.\,5.\,1862 Wien – 21.\,10.\,1931 ebd.), \emph{Schriftsteller, Mediziner}!Anatol@\strich\emph{Anatol}|pwv} von Dir in
               einer Deiner würdigen Weiſe zu bearbeiten. Eine kleine Reklamenotiz hätte ich als
               einen Affront für Dich empfunden. Es mußte etwas Hübſches und Feines {\pb}ſein. Das aber war ich außerſtande zu{ }ſchaffen. Noch
               heut bin ich es nicht imſtande. Denn ich bin nicht geheilt, werde es wohl auch nie
               werden, und bin durch dieſen Schlag und durch gewiſſen{ }ſchweren Familien- und
               Berufs-Kummer, durch die entſetzliche Zukunftsloſigkeit meiner \textsc{Carrière} zerbrochener als je. Um Dich nicht warten zu laſſen,{ }ſandte mein
                  Onkel\pwindex{Mamroth, Fedor 21.\,2.\,1851 Breslau – 25.\,6.\,1907 Frankfurt am Main@\textsc{Mamroth, Fedor} (21.\,2.\,1851 Breslau – 25.\,6.\,1907 Frankfurt am Main), \emph{Journalist, Kritiker}|pwv}{ }ſofort Dein Buch\pwindex{Schnitzler, Arthur 15.\,5.\,1862 Wien – 21.\,10.\,1931 ebd.@\textsc{Schnitzler, Arthur} (15.\,5.\,1862 Wien – 21.\,10.\,1931 ebd.), \emph{Schriftsteller, Mediziner}!Anatol@\strich\emph{Anatol}|pwv} unſerem \label{K_L02709-5v}\edtext{Berlin\oindex{Berlin@\textbf{Berlin}, \emph{Hauptstadt}|pw}er Berichterſtatter\pwindex{Stein, August 2.\,6.\,1851 Breslau – 12.\,10.\,1920 Berlin@\textsc{Stein, August} (2.\,6.\,1851 Breslau – 12.\,10.\,1920 Berlin), \emph{Schriftsteller, Journalist}|pwuv}\pwindex{Eisner, Kurt 14.\,5.\,1867 Berlin – 21.\,2.\,1919 München@\textsc{Eisner, Kurt} (14.\,5.\,1867 Berlin – 21.\,2.\,1919 München), \emph{Schriftsteller, Politiker}|pwuv}}{\lemma{\textnormal{\emph{Berliner Berichterstatter}}}\Cendnote{\textnormal{Es könnte sich hierbei um August Stein\pwindex{Stein, August 2.\,6.\,1851 Breslau – 12.\,10.\,1920 Berlin@\textsc{Stein, August} (2.\,6.\,1851 Breslau – 12.\,10.\,1920 Berlin), \emph{Schriftsteller, Journalist}|pwk} handeln, der seit 1883 das Berlin\oindex{Berlin@\textbf{Berlin}, \emph{Hauptstadt}|pwk}er Büro
                  der \emph{Frankfurter Zeitung}\orgindex{Frankfurter Zeitung@Frankfurter Zeitung|pwk} leitete, oder um Kurt Eisner\pwindex{Eisner, Kurt 14.\,5.\,1867 Berlin – 21.\,2.\,1919 München@\textsc{Eisner, Kurt} (14.\,5.\,1867 Berlin – 21.\,2.\,1919 München), \emph{Schriftsteller, Politiker}|pwk}.}}}\label{K_L02709-5}. Der Herr\pwindex{Stein, August 2.\,6.\,1851 Breslau – 12.\,10.\,1920 Berlin@\textsc{Stein, August} (2.\,6.\,1851 Breslau – 12.\,10.\,1920 Berlin), \emph{Schriftsteller, Journalist}|pwuv}\pwindex{Eisner, Kurt 14.\,5.\,1867 Berlin – 21.\,2.\,1919 München@\textsc{Eisner, Kurt} (14.\,5.\,1867 Berlin – 21.\,2.\,1919 München), \emph{Schriftsteller, Politiker}|pwuv} hat einfach nicht
               darüber geſchrieben. Und wie {\pb}bei unſerem Blatte\orgindex{Frankfurter Zeitung@Frankfurter Zeitung|pwv} die Verhältniſſe liegen,
               iſt mein Onkel\pwindex{Mamroth, Fedor 21.\,2.\,1851 Breslau – 25.\,6.\,1907 Frankfurt am Main@\textsc{Mamroth, Fedor} (21.\,2.\,1851 Breslau – 25.\,6.\,1907 Frankfurt am Main), \emph{Journalist, Kritiker}|pwv} machtlos, ihn
               dazu zu zwingen. Mein Onkel\pwindex{Mamroth, Fedor 21.\,2.\,1851 Breslau – 25.\,6.\,1907 Frankfurt am Main@\textsc{Mamroth, Fedor} (21.\,2.\,1851 Breslau – 25.\,6.\,1907 Frankfurt am Main), \emph{Journalist, Kritiker}|pwv}{ }ſelbſt hat{ }ſich dann längere Zeit mit dem Gedanken getragen,{ }ſelber darüber zu{ }ſchreiben. Aber es iſt eine Unproductivität über ihn gekommen, die auch ihm die Feder
               lähmt,{ }ſoweit es{ }ſich nicht um Arbeiten handelt, die der Dienſt von ihm erzwingt. Das
               Alles iſt {\pb}\strikeout{mündlich}{ }ſchriftlich{ }ſchwer auseinanderzuſetzen.
               Mündlich würde ich es Dir leicht begreiflich machen. Das praktiſche Reſultat: Ich
               gehe nach \textsc{Paris}\oindex{Paris@\textbf{Paris}, \emph{Hauptstadt}|pw} zurück, mit dem feſten Vorſatz, doch über Dein Werk\pwindex{Schnitzler, Arthur 15.\,5.\,1862 Wien – 21.\,10.\,1931 ebd.@\textsc{Schnitzler, Arthur} (15.\,5.\,1862 Wien – 21.\,10.\,1931 ebd.), \emph{Schriftsteller, Mediziner}!Anatol@\strich\emph{Anatol}|pwv} zu \label{K_L02709-6v}\edtext{ſchreiben}{\lemma{\textnormal{\emph{schreiben}}}\Cendnote{\textnormal{Dazu
                  kam es nicht.}}}\label{K_L02709-6}, kann aber bei meinem{ }ſchwachen Character für nichts
               einſtehen. Das Geſcheiteſte, im Intereſſe einer raſchen Erledigung, wäre, wenn einer
               von den Wien\oindex{Wien@\textbf{Wien}, \emph{Verwaltungsgebiet}|pw}er Freunden, \textsc{Richard\pwindex{Beer-Hofmann, Richard 11.\,7.\,1866 Wien – 26.\,9.\,1945 New York City@\textsc{Beer-Hofmann, Richard} (11.\,7.\,1866 Wien – 26.\,9.\,1945 New York City), \emph{Schriftsteller}|pw}} oder \textsc{Loris\pwindex{Hofmannsthal, Hugo von 1.\,2.\,1874 Wien – 15.\,7.\,1929 Rodaun@\textsc{Hofmannsthal, Hugo von} (1.\,2.\,1874 Wien – 15.\,7.\,1929 Rodaun), \emph{Schriftsteller}|pw}}, uns ein kleines \introOben{}\label{K_L02709-7v}\edtext{Artikelchen}{\lemma{\textnormal{\emph{Artikelchen}}}\Cendnote{\textnormal{Dazu kam es nicht.}}}\label{K_L02709-7}\introOben{}{ }\strikeout{\textcolor{gray}{×}\-\textcolor{gray}{×}\-\textcolor{gray}{×}\-\textcolor{gray}{×}\-\textcolor{gray}{×}\-\textcolor{gray}{×}} darüber machen wollte. Mein Onkel\pwindex{Mamroth, Fedor 21.\,2.\,1851 Breslau – 25.\,6.\,1907 Frankfurt am Main@\textsc{Mamroth, Fedor} (21.\,2.\,1851 Breslau – 25.\,6.\,1907 Frankfurt am Main), \emph{Journalist, Kritiker}|pwv} verſpricht {\pb}ſofortigen Abdruck. Wenn
               nicht,{ }ſo gewähre mir, liebſter Freund, noch eine Friſt, und ich will alle Kraft
               aufbieten, um zu thun, was ich Dir{ }ſchulde und was ich auch gar{ }ſo gern thun
               möchte.\pend
           
\pstart
           Über den Roman\pwindex{Schnitzler, Arthur 15.\,5.\,1862 Wien – 21.\,10.\,1931 ebd.@\textsc{Schnitzler, Arthur} (15.\,5.\,1862 Wien – 21.\,10.\,1931 ebd.), \emph{Schriftsteller, Mediziner}!Sterben. Novelle@\strich\emph{Sterben. Novelle}|pwv} haben wir lange
               geſprochen, mein Onkel\pwindex{Mamroth, Fedor 21.\,2.\,1851 Breslau – 25.\,6.\,1907 Frankfurt am Main@\textsc{Mamroth, Fedor} (21.\,2.\,1851 Breslau – 25.\,6.\,1907 Frankfurt am Main), \emph{Journalist, Kritiker}|pwv} und
               ich. Ein Abdruck in der Frkf. Ztg.\orgindex{Frankfurter Zeitung@Frankfurter Zeitung|pw} iſt unmöglich
               wegen der \label{K_L02709-8v}\edtext{Philiſtroſität}{\lemma{\textnormal{\emph{Philistrosität}}}\Cendnote{\textnormal{Spießbürgerlichkeit, Engstirnigkeit}}}\label{K_L02709-8}
               des Publicums. Weder mein Onkel\pwindex{Mamroth, Fedor 21.\,2.\,1851 Breslau – 25.\,6.\,1907 Frankfurt am Main@\textsc{Mamroth, Fedor} (21.\,2.\,1851 Breslau – 25.\,6.\,1907 Frankfurt am Main), \emph{Journalist, Kritiker}|pwv} noch ich{ }ſind in keinen Beziehungen mit einem Verleger. {\pb}Das Einzige, was man für’s Erſte thun könnte, wäre
               ein Brief, den Du dann beifügſt, wenn Du das Manuſkript\pwindex{Schnitzler, Arthur 15.\,5.\,1862 Wien – 21.\,10.\,1931 ebd.@\textsc{Schnitzler, Arthur} (15.\,5.\,1862 Wien – 21.\,10.\,1931 ebd.), \emph{Schriftsteller, Mediziner}!Sterben. Novelle@\strich\emph{Sterben. Novelle}|pwv} einem \label{K_L02709-9v}\edtext{Verleger Deiner Wahl}{\lemma{\textnormal{\emph{Verleger Deiner Wahl}}}\Cendnote{\textnormal{In Buchform erschien \emph{Sterben}\pwindex{Schnitzler, Arthur 15.\,5.\,1862 Wien – 21.\,10.\,1931 ebd.@\textsc{Schnitzler, Arthur} (15.\,5.\,1862 Wien – 21.\,10.\,1931 ebd.), \emph{Schriftsteller, Mediziner}!Sterben. Novelle@\strich\emph{Sterben. Novelle}|pwk} erstmals im November 1894 (vordatiert
                     auf 1895) bei \emph{S.
                     Fischer}\orgindex{S. Fischer Verlag@S. Fischer Verlag|pwk}.}}}\label{K_L02709-9} einſchickſt und der wenigſtens den Vortheil hat, Dir durch
               den Namen der Frankf. Ztg.\orgindex{Frankfurter Zeitung@Frankfurter Zeitung|pw} jene Accredition zu
               geben, deren Du bei jenen urtheilsloſen Buch-Handwerkern noch bedarfſt. Dein Stolz
               wird{ }ſich gegen dieſes Mittel wehren, Dein Verſtand wird Dir zeigen, daß es doch {\pb}nicht zu verſchmähen iſt. Biſt Du aber erſt einmal
               mit einem Verleger in Beziehung und brauchſt Du meinen Onkel\pwindex{Mamroth, Fedor 21.\,2.\,1851 Breslau – 25.\,6.\,1907 Frankfurt am Main@\textsc{Mamroth, Fedor} (21.\,2.\,1851 Breslau – 25.\,6.\,1907 Frankfurt am Main), \emph{Journalist, Kritiker}|pwv} oder mich zur weiteren Förderung der
               Angelegenheit,{ }ſo wirſt Du uns auf dem Laufenden erhalten, und vielleicht ergibt{ }ſich
               am Ende doch die Möglichkeit, etwas Poſitiveres und Specielleres zu erwirken.\pend
           
\pstart
           Der Brief folgt anbei.\pend
           
\pstart
           {\pb}\substVorne{}\textsuperscript{M}\substDazwischen{}N\substHinten{}imm’ dieſen Brief auch als Antwort meines Onkels\pwindex{Mamroth, Fedor 21.\,2.\,1851 Breslau – 25.\,6.\,1907 Frankfurt am Main@\textsc{Mamroth, Fedor} (21.\,2.\,1851 Breslau – 25.\,6.\,1907 Frankfurt am Main), \emph{Journalist, Kritiker}|pwv}, der Dich lieb hat und Dir gern das Blaue vom Himmel
               herunterholen würde, wenn er könnte. Aber Du haſt keine Ahnung, w\substVorne{}\textsuperscript{ie}\substDazwischen{}a\substHinten{}s für arme, macht- und bedeutungsloſe Menſchen wir{ }ſind, er und ich, wir \substVorne{}\textsuperscript{z}\substDazwischen{}Z\substHinten{}wei mit dem verfehlten Leben.\pend
           
\pstart
           Grüß’ Dich Gott, mein theurer Freund! {\\[\baselineskip]}Dein {\\[\baselineskip]}\spacefill\mbox{Paul Goldmann.}\pend
           \leftskip=0em{}\selectlanguage{ngerman}\endnumbering\briefempfaengerindex{Schnitzler, Arthur@\textsc{Schnitzler, Arthur}!zzzGoldmann, Paul@\emph{von Paul Goldmann}!1893-06-031@{3. 6. 1893}|)be}\mylabel{L02709h}  \newcommand{\dateiname}{L02709}\newcommand{\titel}{Paul Goldmann an Arthur Schnitzler, 3. 6. 1893}\newcommand{\editorInnen}{Martin Anton Müller und Laura Untner}%% latex-leseansicht-abspann.tex
%% Abspann für die Leseansicht.
%% Der Schalter \ifkorrekturansicht ist bereits durch den Vorspann gesetzt.

%% latex-abspann.tex
%% Gemeinsamer Abspann für Korrekturansicht und Leseansicht.
%% Setzt den Schalter \ifkorrekturansicht voraus (gesetzt in den
%% einbindenden Dateien latex-korrekturansicht-abspann.tex bzw.
%% latex-leseansicht-abspann.tex).
%% ---------------------------------------------------------------

\normalsize

% Das esempio-Environment wird nur in der Leseansicht benötigt
\ifkorrekturansicht\else
\newenvironment{esempio}[3]%
{
    \vspace{1.5ex}
    \rlap{\underline{#1}}
    \par
    \setlength{\parindent}{0cm}
    \nopagebreak
    \leftskip=#2cm
    \rightskip=#3cm
}
{
    \par
}
\fi

\doendnotes{C}
\bigskip
\vfill

\clearpage

\footnotesize

\ifkorrekturansicht
  \lohead{\textsc{register}}
\fi

% theindex-Environment neu definieren ohne reledmac
\makeatletter
\renewenvironment{theindex}{%
  \ifkorrekturansicht
    \section*{\indexname}%
  \else
    \subsubsection*{Index der erwähnten Entitäten}%
  \fi
  \setlength{\parindent}{0pt}%
  \setlength{\parskip}{0pt plus 0.3pt}%
  \let\item\@idxitem
}{%
  \ifkorrekturansicht\clearpage\fi
}
\makeatother

\IfFileExists{\jobname-pw.ind}{\input{\jobname-pw.ind}}{}

% Quellenangabe nur in der Leseansicht
\ifkorrekturansicht\else
% Fallback-Definitionen, falls die .tex-Datei \titel etc. nicht gesetzt hat
\providecommand{\titel}{}
\providecommand{\editorInnen}{}
\providecommand{\dateiname}{\jobname}

\vspace{3cm}

\vfill

\footnotesize
\textsc{Quelle}: \titel. Herausgegeben von {\editorInnen}. In: \emph{Arthur Schnitzler: Briefwechsel mit Autorinnen und Autoren}.
 Digitale Edition, https://schnitzler-briefe.acdh.oeaw.ac.at/{\dateiname}.html (Stand \today)
\fi

\end{document}


