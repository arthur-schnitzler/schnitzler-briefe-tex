%% latex-korrekturansicht-vorspann.tex
%% Vorspann für die Korrekturansicht.
%% Lädt die gemeinsame Datei latex-vorspann.tex mit gesetztem Schalter.

\newif\ifkorrekturansicht
\korrekturansichttrue

\input{../tex-inputs/latex-vorspann}


\section[Paul Goldmann an Arthur Schnitzler, 3. 6. 1893]{L02709 Paul Goldmann an Arthur Schnitzler, 3. 6. 1893}
\nopagebreak\mylabel{L02709v}
\rehead{ }\normalsize\beginnumbering\briefempfaengerindex{Schnitzler, Arthur@\textsc{Schnitzler, Arthur}!zzzGoldmann, Paul@\emph{von Paul Goldmann}!1893-06-031@{3. 6. 1893}|(be}
\toendnotes[C]{\smallbreak\pagebreak[2]}\Standort{DLA, A:Schnitzler, HS.NZ85.1.3163.}
\physDesc{Brief, 3 Blätter, 10 Seiten, 4086 Zeichen
\newline{}Handschrift: blaue Tinte, deutsche Kurrent
\newline{}Schnitzler: 1) mit Bleistift das erste Blatt mit »1.« nummeriert  2) mit rotem Buntstift eine Unterstreichung}\toendnotes[C]{\smallbreak}
\pstart
           {\pb}\textcolor{gray}{\textbf{\textbf{Frankfurter Zeitung}}}\orgindex{Frankfurter Zeitung@Frankfurter Zeitung|pw}\pend
           
\pstart
           \textcolor{gray}{\textbf{und}}\pend
           
\pstart
           \textcolor{gray}{\textbf{\textbf{Handelsblatt}.}}\hfill \textcolor{gray}{\textbf{Frankfurt a. M.\oindex{Frankfurt am Main@\textbf{Frankfurt am Main}, \emph{P.PPLA3}|pw},}}{ }3. Juni \textcolor{gray}{\textbf{189}}3.\pend
           
\pstart
           \textcolor{gray}{\textbf{\textbf{Redaktion.}\noindent{}\textcolor{gray}{\textbf{Für die Redaktion\orgindex{Frankfurter Zeitung@Frankfurter Zeitung|pwv} bestimmte Briefe und Sendungen wolle
                              man \so{nicht} an die Person eines Redakteurs,
                              sondern stets \textbf{an die Redaktion\orgindex{Frankfurter Zeitung@Frankfurter Zeitung|pwv} der Frankfurter Zeitung\pwindex{Frankfurter Zeitung@\emph{Frankfurter Zeitung}|pw}} adressiren.}}}}\pend
           
\pstart
           \textcolor{gray}{\textbf{\textbf{Telegramm-Adresse:}}}\pend
           
\pstart
           \textcolor{gray}{\textbf{\textbf{Zeitung Frankfurt
                        Main\oindex{Frankfurt am Main@\textbf{Frankfurt am Main}, \emph{P.PPLA3}|pw}.}}}\pend
           
\pstart\center{}Mein lieber Arthur!\pend\vspace{0.5em}
\pstart
           Ich bin für wenige Tage zum Beſuch in Frankfurt\oindex{Frankfurt am Main@\textbf{Frankfurt am Main}, \emph{P.PPLA3}|pw},
               um der \label{K_L02709-1v}\edtext{Hochzeit meiner Schweſter\pwindex{Rosengart, Vally 1866-12-29 – nach 1926@\textsc{Rosengart, Vally} (1866-12-29 – nach 1926)|pwv}}{\lemma{\textnormal{\emph{Hochzeit … Schweſter}}}\Cendnote{\textnormal{Vally Goldmann\pwindex{Rosengart, Vally 1866-12-29 – nach 1926@\textsc{Rosengart, Vally} (1866-12-29 – nach 1926)|pwk} heiratete den in Laupheim\oindex{Laupheim@\textbf{Laupheim}, \emph{P.PPL}|pwk} geborenen Arzt Josef Rosengart\pwindex{Rosengart, Josef 1860-02-08 – 1927-08-04@\textsc{Rosengart, Josef} (1860-02-08 – 1927-08-04), \emph{Arzt/Ärztin}|pwk}.}}}\label{K_L02709-1} beizuwohnen. Mein Onkel\pwindex{Mamroth, Fedor 21.02.1851 – 25.06.1907@\textsc{Mamroth, Fedor} (21.02.1851 – 25.06.1907), \emph{Journalist/Journalistin, Kritiker/Kritikerin}|pwv} ſpricht mir natürlich
               von Dir, erzählt mir mit wahrem Enthuſiasmus von Deinem Roman\pwindex{Sterben. Novelle@\emph{Sterben. Novelle}|pwv}, den er als ein bedeutendes Werk\pwindex{Sterben. Novelle@\emph{Sterben. Novelle}|pwv} bezeichnet, und zeigt mir
               ſchließlich Deinen \label{K_L02709-2v}\edtext{Brief}{\lemma{\textnormal{\emph{Brief}}}\Cendnote{\textnormal{Nicht erhalten. In seinen Antwortbriefen
                  vom 5. 3. 1893 und 4. 6. 1893 lobte Fedor Mamroth\pwindex{Mamroth, Fedor 21.02.1851 – 25.06.1907@\textsc{Mamroth, Fedor} (21.02.1851 – 25.06.1907), \emph{Journalist/Journalistin, Kritiker/Kritikerin}|pwk} jedoch ausdrücklich Schnitzlers Novelle \emph{Sterben}\pwindex{Sterben. Novelle@\emph{Sterben. Novelle}|pwk} (vgl. Fedor Mamroth an Arthur Schnitzler, 5. 3. 1893). Gedruckt wurde \emph{Sterben}\pwindex{Sterben. Novelle@\emph{Sterben. Novelle}|pwk}
                  zuerst von Oktober bis Dezember 1894 in den Heften 10 bis 12 der \emph{Neuen Deutschen Rundschau}\pwindex{Neue Deutsche Rundschau@\emph{Neue Deutsche Rundschau}|pwk}.}}}\label{K_L02709-2}, es tief beklagend, daß
                  \label{K_L02709-3v}\edtext{zwiſchen Dich und ih\substVorne{}\textsuperscript{\textcolor{gray}{m}}\substDazwischen{}n\substHinten{} etwas getreten}{\lemma{\textnormal{\emph{zwiſchen … getreten}}}\Cendnote{\textnormal{Im Kern geht es,
                  wie aus dem Folgenden deutlich wird, um das Ausbleiben einer Rezension des \emph{Anatol}\pwindex{Anatol@\emph{Anatol}|pwk} in der \emph{Frankfurter Zeitung}\orgindex{Frankfurter Zeitung@Frankfurter Zeitung|pwk}. In einem größeren Zusammenhang könnte es auch eine
                  Kränkung Schnitzlers aufgrund der
                  wiederholten Ablehnungen Fedor Mamroths\pwindex{Mamroth, Fedor 21.02.1851 – 25.06.1907@\textsc{Mamroth, Fedor} (21.02.1851 – 25.06.1907), \emph{Journalist/Journalistin, Kritiker/Kritikerin}|pwk} –
                  zuletzt \emph{Das Märchen}\pwindex{Maerchen. Schauspiel in drei Aufzuegen@\emph{Das Märchen. Schauspiel in drei Aufzügen}|pwk} und \emph{Sterben}\pwindex{Sterben. Novelle@\emph{Sterben. Novelle}|pwk} – gegeben haben. Der Brief Mamroths\pwindex{Mamroth, Fedor 21.02.1851 – 25.06.1907@\textsc{Mamroth, Fedor} (21.02.1851 – 25.06.1907), \emph{Journalist/Journalistin, Kritiker/Kritikerin}|pwk} an Schnitzler vom 17. 11. 1892 legt nahe, dass Schnitzler den ausbleibenden Kontakt nach der Ablehnung des \emph{Märchens}\pwindex{Maerchen. Schauspiel in drei Aufzuegen@\emph{Das Märchen. Schauspiel in drei Aufzügen}|pwk} als unhöflich empfunden hat.}}}\label{K_L02709-3} iſt, das beſſer nicht da wäre. Dein Brief, mein lieber Freund, iſt
                  {\pb}ebenſo an mich gerichtet, wie an meinen Onkel\pwindex{Mamroth, Fedor 21.02.1851 – 25.06.1907@\textsc{Mamroth, Fedor} (21.02.1851 – 25.06.1907), \emph{Journalist/Journalistin, Kritiker/Kritikerin}|pwv}. Vieles von dem, was Du
               zu ihm ſagſt, bezieht ſich auch auf mich. Und ich kann mich von der Schuld nicht
               freiſprechen, ein wenig die Bitterkeit mitveranlaßt zu haben, von der ich Dich
               erfüllt ſehe. Objectiv haſt Du vollſtändig Recht. Nun aber ſubjektiv: Gewiß, wenn ein
               Menſch auf der Welt verpflichtet war, über »\textsc{Anatol\pwindex{Anatol@\emph{Anatol}|pw}}« zu ſchreiben, ſo war ich es. Das Buch\pwindex{Anatol@\emph{Anatol}|pwv} kam bei mir an in einer meiner ſchwerſten Arbeitszeiten
               – Arbeit, von deren Wucht und Depreſ{\pb}ſionsmacht Du
               keinerlei Ahnung haben kannſt. Ich mußte es zurücklegen für ſpäter. Und als dann das
               »ſpäter« kam, kam über mich das \label{K_L02709-4v}\edtext{Unheil}{\lemma{\textnormal{\emph{Unheil}}}\Cendnote{\textnormal{die Erkrankung an einer
                  Geschlechtskrankheit}}}\label{K_L02709-4}, das Du kennſt, mit der Unmöglichkeit, auch nur ein
               wenig Spannkraft zu finden, um aus dem mechaniſchen Trott der täglichen Arbeit
               herauszugehen und \strikeout{\textcolor{gray}{×}\-\textcolor{gray}{×}} ein Werk\pwindex{Anatol@\emph{Anatol}|pwv} von Dir in
               einer Deiner würdigen Weiſe zu bearbeiten. Eine kleine Reklamenotiz hätte ich als
               einen Affront für Dich empfunden. Es mußte etwas Hübſches und Feines {\pb}ſein. Das aber war ich außerſtande zu ſchaffen. Noch
               heut bin ich es nicht imſtande. Denn ich bin nicht geheilt, werde es wohl auch nie
               werden, und bin durch dieſen Schlag und durch gewiſſen ſchweren Familien- und
               Berufs-Kummer, durch die entſetzliche Zukunftsloſigkeit meiner \textsc{Carrière} zerbrochener als je. Um Dich nicht warten zu laſſen, ſandte mein
                  Onkel\pwindex{Mamroth, Fedor 21.02.1851 – 25.06.1907@\textsc{Mamroth, Fedor} (21.02.1851 – 25.06.1907), \emph{Journalist/Journalistin, Kritiker/Kritikerin}|pwv} ſofort Dein Buch\pwindex{Anatol@\emph{Anatol}|pwv} unſerem \label{K_L02709-5v}\edtext{Berlin\oindex{Berlin@\textbf{Berlin}, \emph{P.PPLC}|pw}er Berichterſtatter\pwindex{Stein, August 1851-06-02 – 1920-10-12@\textsc{Stein, August} (1851-06-02 – 1920-10-12), \emph{Schriftsteller/Schriftstellerin, Journalist/Journalistin}|pwuv}\pwindex{Eisner, Kurt 14.05.1867 – 21.02.1919@\textsc{Eisner, Kurt} (14.05.1867 – 21.02.1919), \emph{Schriftsteller/Schriftstellerin, Politiker/Politikerin}|pwuv}}{\lemma{\textnormal{\emph{Berliner Berichterſtatter}}}\Cendnote{\textnormal{Es könnte sich hierbei um August Stein\pwindex{Stein, August 1851-06-02 – 1920-10-12@\textsc{Stein, August} (1851-06-02 – 1920-10-12), \emph{Schriftsteller/Schriftstellerin, Journalist/Journalistin}|pwk} handeln, der seit 1883 das Berlin\oindex{Berlin@\textbf{Berlin}, \emph{P.PPLC}|pwk}er Büro
                  der \emph{Frankfurter Zeitung}\orgindex{Frankfurter Zeitung@Frankfurter Zeitung|pwk} leitete, oder um Kurt Eisner\pwindex{Eisner, Kurt 14.05.1867 – 21.02.1919@\textsc{Eisner, Kurt} (14.05.1867 – 21.02.1919), \emph{Schriftsteller/Schriftstellerin, Politiker/Politikerin}|pwk}.}}}\label{K_L02709-5}. Der Herr\pwindex{Stein, August 1851-06-02 – 1920-10-12@\textsc{Stein, August} (1851-06-02 – 1920-10-12), \emph{Schriftsteller/Schriftstellerin, Journalist/Journalistin}|pwuv}\pwindex{Eisner, Kurt 14.05.1867 – 21.02.1919@\textsc{Eisner, Kurt} (14.05.1867 – 21.02.1919), \emph{Schriftsteller/Schriftstellerin, Politiker/Politikerin}|pwuv} hat einfach nicht
               darüber geſchrieben. Und wie {\pb}bei unſerem Blatte\orgindex{Frankfurter Zeitung@Frankfurter Zeitung|pwv} die Verhältniſſe liegen,
               iſt mein Onkel\pwindex{Mamroth, Fedor 21.02.1851 – 25.06.1907@\textsc{Mamroth, Fedor} (21.02.1851 – 25.06.1907), \emph{Journalist/Journalistin, Kritiker/Kritikerin}|pwv} machtlos, ihn
               dazu zu zwingen. Mein Onkel\pwindex{Mamroth, Fedor 21.02.1851 – 25.06.1907@\textsc{Mamroth, Fedor} (21.02.1851 – 25.06.1907), \emph{Journalist/Journalistin, Kritiker/Kritikerin}|pwv}
               ſelbſt hat ſich dann längere Zeit mit dem Gedanken getragen, ſelber darüber zu
               ſchreiben. Aber es iſt eine Unproductivität über ihn gekommen, die auch ihm die Feder
               lähmt, ſoweit es ſich nicht um Arbeiten handelt, die der Dienſt von ihm erzwingt. Das
               Alles iſt {\pb}\strikeout{mündlich} ſchriftlich ſchwer auseinanderzuſetzen.
               Mündlich würde ich es Dir leicht begreiflich machen. Das praktiſche Reſultat: Ich
               gehe nach \textsc{Paris}\oindex{Paris@\textbf{Paris}, \emph{P.PPLC}|pw} zurück, mit dem feſten Vorſatz, doch über Dein Werk\pwindex{Anatol@\emph{Anatol}|pwv} zu \label{K_L02709-6v}\edtext{ſchreiben}{\lemma{\textnormal{\emph{ſchreiben}}}\Cendnote{\textnormal{Dazu
                  kam es nicht.}}}\label{K_L02709-6}, kann aber bei meinem ſchwachen Character für nichts
               einſtehen. Das Geſcheiteſte, im Intereſſe einer raſchen Erledigung, wäre, wenn einer
               von den Wien\oindex{Wien@\textbf{Wien}, \emph{A.ADM2}|pw}er Freunden, \textsc{Richard\pwindex{Beer-Hofmann, Richard 1866-07-11 – 1945-09-26@\textsc{Beer-Hofmann, Richard} (1866-07-11 – 1945-09-26), \emph{Schriftsteller/Schriftstellerin}|pw}} oder \textsc{Loris\pwindex{Hofmannsthal, Hugo von 1874-02-01 – 1929-07-15@\textsc{Hofmannsthal, Hugo von} (1874-02-01 – 1929-07-15), \emph{Schriftsteller/Schriftstellerin}|pw}}, uns ein kleines \introOben{}\label{K_L02709-7v}\edtext{Artikelchen}{\lemma{\textnormal{\emph{Artikelchen}}}\Cendnote{\textnormal{Dazu kam es nicht.}}}\label{K_L02709-7}\introOben{}{ }\strikeout{\textcolor{gray}{×}\-\textcolor{gray}{×}\-\textcolor{gray}{×}\-\textcolor{gray}{×}\-\textcolor{gray}{×}\-\textcolor{gray}{×}} darüber machen wollte. Mein Onkel\pwindex{Mamroth, Fedor 21.02.1851 – 25.06.1907@\textsc{Mamroth, Fedor} (21.02.1851 – 25.06.1907), \emph{Journalist/Journalistin, Kritiker/Kritikerin}|pwv} verſpricht {\pb}ſofortigen Abdruck. Wenn
               nicht, ſo gewähre mir, liebſter Freund, noch eine Friſt, und ich will alle Kraft
               aufbieten, um zu thun, was ich Dir ſchulde und was ich auch gar ſo gern thun
               möchte.\pend
           
\pstart
           Über den Roman\pwindex{Sterben. Novelle@\emph{Sterben. Novelle}|pwv} haben wir lange
               geſprochen, mein Onkel\pwindex{Mamroth, Fedor 21.02.1851 – 25.06.1907@\textsc{Mamroth, Fedor} (21.02.1851 – 25.06.1907), \emph{Journalist/Journalistin, Kritiker/Kritikerin}|pwv} und
               ich. Ein Abdruck in der Frkf. Ztg.\orgindex{Frankfurter Zeitung@Frankfurter Zeitung|pw} iſt unmöglich
               wegen der \label{K_L02709-8v}\edtext{Philiſtroſität}{\lemma{\textnormal{\emph{Philiſtroſität}}}\Cendnote{\textnormal{Spießbürgerlichkeit, Engstirnigkeit}}}\label{K_L02709-8}
               des Publicums. Weder mein Onkel\pwindex{Mamroth, Fedor 21.02.1851 – 25.06.1907@\textsc{Mamroth, Fedor} (21.02.1851 – 25.06.1907), \emph{Journalist/Journalistin, Kritiker/Kritikerin}|pwv} noch ich ſind in keinen Beziehungen mit einem Verleger. {\pb}Das Einzige, was man für’s Erſte thun könnte, wäre
               ein Brief, den Du dann beifügſt, wenn Du das Manuſkript\pwindex{Sterben. Novelle@\emph{Sterben. Novelle}|pwv} einem \label{K_L02709-9v}\edtext{Verleger Deiner Wahl}{\lemma{\textnormal{\emph{Verleger Deiner Wahl}}}\Cendnote{\textnormal{In Buchform erschien \emph{Sterben}\pwindex{Sterben. Novelle@\emph{Sterben. Novelle}|pwk} erstmals im November 1894 (vordatiert
                     auf 1895) bei \emph{S.
                     Fischer}\orgindex{S. Fischer Verlag@S. Fischer Verlag|pwk}.}}}\label{K_L02709-9} einſchickſt und der wenigſtens den Vortheil hat, Dir durch
               den Namen der Frankf. Ztg.\orgindex{Frankfurter Zeitung@Frankfurter Zeitung|pw} jene Accredition zu
               geben, deren Du bei jenen urtheilsloſen Buch-Handwerkern noch bedarfſt. Dein Stolz
               wird ſich gegen dieſes Mittel wehren, Dein Verſtand wird Dir zeigen, daß es doch {\pb}nicht zu verſchmähen iſt. Biſt Du aber erſt einmal
               mit einem Verleger in Beziehung und brauchſt Du meinen Onkel\pwindex{Mamroth, Fedor 21.02.1851 – 25.06.1907@\textsc{Mamroth, Fedor} (21.02.1851 – 25.06.1907), \emph{Journalist/Journalistin, Kritiker/Kritikerin}|pwv} oder mich zur weiteren Förderung der
               Angelegenheit, ſo wirſt Du uns auf dem Laufenden erhalten, und vielleicht ergibt ſich
               am Ende doch die Möglichkeit, etwas Poſitiveres und Specielleres zu erwirken.\pend
           
\pstart
           Der Brief folgt anbei.\pend
           
\pstart
           {\pb}\substVorne{}\textsuperscript{M}\substDazwischen{}N\substHinten{}imm' dieſen Brief auch als Antwort meines Onkels\pwindex{Mamroth, Fedor 21.02.1851 – 25.06.1907@\textsc{Mamroth, Fedor} (21.02.1851 – 25.06.1907), \emph{Journalist/Journalistin, Kritiker/Kritikerin}|pwv}, der Dich lieb hat und Dir gern das Blaue vom Himmel
               herunterholen würde, wenn er könnte. Aber Du haſt keine Ahnung, w\substVorne{}\textsuperscript{ie}\substDazwischen{}a\substHinten{}s für arme, macht- und bedeutungsloſe Menſchen wir ſind, er und ich, wir \substVorne{}\textsuperscript{z}\substDazwischen{}Z\substHinten{}wei mit dem verfehlten Leben.\pend
           
\pstart
           Grüß’ Dich Gott, mein theurer Freund! {\\[\baselineskip]}Dein {\\[\baselineskip]}\spacefill\mbox{Paul Goldmann.}\pend
           \leftskip=0em{}\selectlanguage{ngerman}\endnumbering\briefempfaengerindex{Schnitzler, Arthur@\textsc{Schnitzler, Arthur}!zzzGoldmann, Paul@\emph{von Paul Goldmann}!1893-06-031@{3. 6. 1893}|)be}\mylabel{L02709h}  \normalsize

\doendnotes{C}
\bigskip
\vfill

\clearpage

\footnotesize

\lohead{\textsc{register}}

% Definiere theindex-Environment komplett neu ohne reledmac
\makeatletter
\renewenvironment{theindex}{%
  \section*{\indexname}%
  \setlength{\parindent}{0pt}%
  \setlength{\parskip}{0pt plus 0.3pt}%
  \let\item\@idxitem
}{%
  \clearpage
}
\makeatother

\IfFileExists{\jobname-pw.ind}{\input{\jobname-pw.ind}}{}

\end{document}

      