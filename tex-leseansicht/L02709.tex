\input{../tex-inputs/latex-pdf-vorspann}
\begin{center}
            \textcolor{red}{ENTWURF. ENTZIFFERUNG NOCH NICHT KORREKTURGELESEN}
                      \end{center}
            
               \section[Paul Goldmann an Arthur Schnitzler, 3. 6. 1893]{ Paul Goldmann an Arthur Schnitzler, 3. 6. 1893}\nopagebreak\mylabel{v}\rehead{ }\begin{ledgroupsized}[t]{13cm}\normalsize\beginnumbering\briefempfaengerindex{Schnitzler, Arthur@\textsc{Schnitzler, Arthur}!zzzGoldmann, Paul@\emph{von Paul Goldmann}!1893-06-031@{3. 6. 1893}|(be} \toendnotes[C]{\smallbreak\pagebreak[2]} \Standort{DLA, A:Schnitzler, HS.NZ85.1.3163.}
\physDesc{Brief, 3 Blätter, 10 Seiten
\newline{}Handschrift: blaue Tinte, deutsche Kurrent
\newline{}Schnitzler: 1) mit Bleistift das erste Blatt mit »1.«
                                 nummeriert 2) mit rotem Buntstift eine Unterstreichung}\toendnotes[C]{\smallbreak}\pstart
           \noindent{}{\pb}\footnote{}\textcolor{gray}{\textbf{\textbf{Frankfurter Zeitung}}}\orgindex{Frankfurter Zeitung@Frankfurter Zeitung|pw}\pend
           \pstart
           \textcolor{gray}{\textbf{und}}\pend
           \pstart
           \textcolor{gray}{\textbf{\textbf{Handelsblatt}\orgindex{Frankfurter Zeitung@Frankfurter Zeitung|pwv}.}}\hfill \textcolor{gray}{\textbf{Frankfurt a. M.}}\oindex{Frankfurt am Main@\textbf{Frankfurt am Main}|pw}, 3. Juni \textcolor{gray}{\textbf{189}}3.
               \pend
           \pstart
           \textcolor{gray}{\textbf{\textbf{Redaktion.}\orgindex{Frankfurter Zeitung@Frankfurter Zeitung|pwv}*)}}\pend
           \pstart
           \textcolor{gray}{\textbf{\textbf{Telegramm-Adresse:}}}\pend
           \pstart
           \textcolor{gray}{\textbf{\textbf{Zeitung\orgindex{Frankfurter Zeitung@Frankfurter Zeitung|pwv}{ }Frankfurt Main\oindex{Frankfurt am Main@\textbf{Frankfurt am Main}|pw}.}}}\pend
           \pstart
           \centering{}Mein lieber Arthur!\pend
           \pstart
           \noindent{}Ich bin für wenige Tage zum Beſuch in Frankfurt\oindex{Frankfurt am Main@\textbf{Frankfurt am Main}|pw},
               um der \label{K_L02709-1v}\edtext{Hochzeit meiner Schweſter\pwindex{Rosengart, Vally *~1866-12-29@\textsc{Rosengart, Vally} (*~1866-12-29)|pwv}}{\lemma{\textnormal{\emph{Hochzeit … Schweſter}}}\Cendnote{\textnormal{Vally Rosengart\pwindex{Rosengart, Vally *~1866-12-29@\textsc{Rosengart, Vally} (*~1866-12-29)|pwk}, vormals Goldmann\pwindex{Rosengart, Vally *~1866-12-29@\textsc{Rosengart, Vally} (*~1866-12-29)|pwkv}, heiratete den in Laupheim\oindex{Laupheim@\textbf{Laupheim}|pwk} geborenen Arzt\pwindex{Rosengart, Josef 1860-02-08 – 1927-08-04@\textsc{Rosengart, Josef} (1860-02-08 – 1927-08-04), \emph{Arzt}|pwkv}{ }Josef Rosengart\pwindex{Rosengart, Josef 1860-02-08 – 1927-08-04@\textsc{Rosengart, Josef} (1860-02-08 – 1927-08-04), \emph{Arzt}|pwk}.}}}\label{K_L02709-1h} beizuwohnen. Mein
                  Onkel\pwindex{Mamroth, Fedor 21.02.1851 – 25.06.1907@\textsc{Mamroth, Fedor} (21.02.1851 – 25.06.1907), \emph{Journalist, Kritiker}|pwv} ſpricht mir
               natürlich von Dir, erzählt mir mit wahrem Enthuſiasmus von Deinem Roman\pwindex{Schnitzler, Arthur 15.05.1862 – 21.10.1931@\textsc{Schnitzler, Arthur} (15.05.1862 – 21.10.1931), \emph{Schriftsteller, Mediziner}!Sterben. Novelle1.10.1894 – 1.12.1894@\strich\emph{Sterben. Novelle} {[}1.10.1894 – 1.12.1894{]}|pwv}, den er als ein bedeutendes Werk\pwindex{Schnitzler, Arthur 15.05.1862 – 21.10.1931@\textsc{Schnitzler, Arthur} (15.05.1862 – 21.10.1931), \emph{Schriftsteller, Mediziner}!Sterben. Novelle1.10.1894 – 1.12.1894@\strich\emph{Sterben. Novelle} {[}1.10.1894 – 1.12.1894{]}|pwv} bezeichnet, und zeigt mir
               ſchließlich Deinen \label{K_L02709-2v}\edtext{Brief}{\lemma{\textnormal{\emph{Brief}}}\Cendnote{\textnormal{In seinen Antwortbriefen vom 4. 6. 1893 und 17. 11. 1892 lobte Fedor Mamroth\pwindex{Mamroth, Fedor 21.02.1851 – 25.06.1907@\textsc{Mamroth, Fedor} (21.02.1851 – 25.06.1907), \emph{Journalist, Kritiker}|pwk} ausdrücklich Schnitzler\pwindex{Schnitzler, Arthur 15.05.1862 – 21.10.1931@\textsc{Schnitzler, Arthur} (15.05.1862 – 21.10.1931), \emph{Schriftsteller, Mediziner}|pwk}s Novelle\pwindex{Schnitzler, Arthur 15.05.1862 – 21.10.1931@\textsc{Schnitzler, Arthur} (15.05.1862 – 21.10.1931), \emph{Schriftsteller, Mediziner}!Sterben. Novelle1.10.1894 – 1.12.1894@\strich\emph{Sterben. Novelle} {[}1.10.1894 – 1.12.1894{]}|pwkv}{ }\emph{Sterben}\pwindex{Schnitzler, Arthur 15.05.1862 – 21.10.1931@\textsc{Schnitzler, Arthur} (15.05.1862 – 21.10.1931), \emph{Schriftsteller, Mediziner}!Sterben. Novelle1.10.1894 – 1.12.1894@\strich\emph{Sterben. Novelle} {[}1.10.1894 – 1.12.1894{]}|pwk} – ihm unter dem Titel »Der sterbende Herr\pwindex{Schnitzler, Arthur 15.05.1862 – 21.10.1931@\textsc{Schnitzler, Arthur} (15.05.1862 – 21.10.1931), \emph{Schriftsteller, Mediziner}!Sterben. Novelle1.10.1894 – 1.12.1894@\strich\emph{Sterben. Novelle} {[}1.10.1894 – 1.12.1894{]}|pwkv}« bekannt:
                     »Ich habe Ihren Roman\pwindex{Schnitzler, Arthur 15.05.1862 – 21.10.1931@\textsc{Schnitzler, Arthur} (15.05.1862 – 21.10.1931), \emph{Schriftsteller, Mediziner}!Sterben. Novelle1.10.1894 – 1.12.1894@\strich\emph{Sterben. Novelle} {[}1.10.1894 – 1.12.1894{]}|pwv} »Der sterbende
                        Herr\pwindex{Schnitzler, Arthur 15.05.1862 – 21.10.1931@\textsc{Schnitzler, Arthur} (15.05.1862 – 21.10.1931), \emph{Schriftsteller, Mediziner}!Sterben. Novelle1.10.1894 – 1.12.1894@\strich\emph{Sterben. Novelle} {[}1.10.1894 – 1.12.1894{]}|pwv}« mit einer Theilnahme gelesen, die mir noch selten eine
                     eingereichte Arbeit eingeflößt hat. Ich beglückwünsche Sie zu dieser Dichtung,
                     in der sie den feinen Geist eines Poeten und die scharfe Beobachtungsgabe des
                     Arztes mit merkwürdiger Ergänzungskunst verschmolzen haben.« Außerdem
                  empfahl er ihm den Druck als Buch, nicht als Feuilleton, und plädierte für eine
                  Änderung des Titels. Vgl. Fedor Mamroth an Arthur Schnitzler, 5. 3. 1893 Gedruckt wurde
                     \emph{Sterben}\pwindex{Schnitzler, Arthur 15.05.1862 – 21.10.1931@\textsc{Schnitzler, Arthur} (15.05.1862 – 21.10.1931), \emph{Schriftsteller, Mediziner}!Sterben. Novelle1.10.1894 – 1.12.1894@\strich\emph{Sterben. Novelle} {[}1.10.1894 – 1.12.1894{]}|pwk}, das erste Mal am 25. 9. 1893 unter diesem
                  Titel im \emph{Tagebuch}\pwindex{Schnitzler, Arthur 15.05.1862 – 21.10.1931@\textsc{Schnitzler, Arthur} (15.05.1862 – 21.10.1931), \emph{Schriftsteller, Mediziner}!Tagebuch1981 – 2000@\strich\emph{Tagebuch} {[}1981 – 2000{]}|pwk} notiert, in Heft 10–12
                     (1894) der \emph{Neuen
                     Deutschen Rundschau}\pwindex{Neue Deutsche Rundschau1894-01-01 – 1903-12-31@\emph{Neue Deutsche Rundschau}|pwk}.}}}\label{K_L02709-2h}, es tief beklagend, daß \label{K_L02709-3v}\edtext{zwiſchen Dich und ihn etwas getreten}{\lemma{\textnormal{\emph{zwiſchen … getreten}}}\Cendnote{\textnormal{Möglich ist, dass Schnitzler\pwindex{Schnitzler, Arthur 15.05.1862 – 21.10.1931@\textsc{Schnitzler, Arthur} (15.05.1862 – 21.10.1931), \emph{Schriftsteller, Mediziner}|pwk} nicht nur wegen der ausbleibenden Besprechung
                  des \emph{Anatol}\pwindex{Schnitzler, Arthur 15.05.1862 – 21.10.1931@\textsc{Schnitzler, Arthur} (15.05.1862 – 21.10.1931), \emph{Schriftsteller, Mediziner}!Anatol1892-10-29 – 1892-10-29@\strich\emph{Anatol} {[}1892-10-29 – 1892-10-29{]}|pwk}, sondern auch aufgrund der
                  wiederholten Ablehnungen seiner Werke durch Fedor
                     Mamroth\pwindex{Mamroth, Fedor 21.02.1851 – 25.06.1907@\textsc{Mamroth, Fedor} (21.02.1851 – 25.06.1907), \emph{Journalist, Kritiker}|pwk} – zuletzt \emph{Das Märchen}\pwindex{Schnitzler, Arthur 15.05.1862 – 21.10.1931@\textsc{Schnitzler, Arthur} (15.05.1862 – 21.10.1931), \emph{Schriftsteller, Mediziner}!Maerchen. Schauspiel in drei Aufzuegen1891 – 1891@\strich\emph{Das Märchen. Schauspiel in drei Aufzügen} {[}1891 – 1891{]}|pwk} und \emph{Sterben}\pwindex{Schnitzler, Arthur 15.05.1862 – 21.10.1931@\textsc{Schnitzler, Arthur} (15.05.1862 – 21.10.1931), \emph{Schriftsteller, Mediziner}!Sterben. Novelle1.10.1894 – 1.12.1894@\strich\emph{Sterben. Novelle} {[}1.10.1894 – 1.12.1894{]}|pwk} – gekränkt war. Insbesondere der Brief
                     Mamroth\pwindex{Mamroth, Fedor 21.02.1851 – 25.06.1907@\textsc{Mamroth, Fedor} (21.02.1851 – 25.06.1907), \emph{Journalist, Kritiker}|pwk}s an Schnitzler\pwindex{Schnitzler, Arthur 15.05.1862 – 21.10.1931@\textsc{Schnitzler, Arthur} (15.05.1862 – 21.10.1931), \emph{Schriftsteller, Mediziner}|pwk} vom 17. 11. 1892 lässt vermuten, dass Schnitzler\pwindex{Schnitzler, Arthur 15.05.1862 – 21.10.1931@\textsc{Schnitzler, Arthur} (15.05.1862 – 21.10.1931), \emph{Schriftsteller, Mediziner}|pwk} zudem den ausbleibenden Kontakt nach der Ablehnung des \emph{Märchen}\pwindex{Schnitzler, Arthur 15.05.1862 – 21.10.1931@\textsc{Schnitzler, Arthur} (15.05.1862 – 21.10.1931), \emph{Schriftsteller, Mediziner}!Maerchen. Schauspiel in drei Aufzuegen1891 – 1891@\strich\emph{Das Märchen. Schauspiel in drei Aufzügen} {[}1891 – 1891{]}|pwk}s als unhöflich empfunden haben
                  dürfte.}}}\label{K_L02709-3h} iſt, das beſſer nicht da wäre. Dein Brief, mein lieber Freund, iſt
                  {\pb}ebenſo an mich gerichtet, wie an meinen Onkel\pwindex{Mamroth, Fedor 21.02.1851 – 25.06.1907@\textsc{Mamroth, Fedor} (21.02.1851 – 25.06.1907), \emph{Journalist, Kritiker}|pwv}. Vieles von dem, was Du
               zu ihm ſagt, bezieht ſich auch auf mich. Und ich kann mich von der Schuld nicht
               freiſprechen, ein wenig die Bitterkeit mitveranlaßt zu haben, von der ich Dich
               erfüllt ſehe. Objectiv haſt Du vollſtändig Recht. Nun aber ſubjektiv: Gewiß, wenn ein
               Menſch auf der Welt verpflichtet war, über »Anatol\pwindex{Schnitzler, Arthur 15.05.1862 – 21.10.1931@\textsc{Schnitzler, Arthur} (15.05.1862 – 21.10.1931), \emph{Schriftsteller, Mediziner}!Anatol1892-10-29 – 1892-10-29@\strich\emph{Anatol} {[}1892-10-29 – 1892-10-29{]}|pw}« zu ſchreiben, ſo war ich es. Das Buch\pwindex{Schnitzler, Arthur 15.05.1862 – 21.10.1931@\textsc{Schnitzler, Arthur} (15.05.1862 – 21.10.1931), \emph{Schriftsteller, Mediziner}!Anatol1892-10-29 – 1892-10-29@\strich\emph{Anatol} {[}1892-10-29 – 1892-10-29{]}|pwv} kam bei mir an in einer meiner ſchwerſten Arbeitszeiten
               – Arbeit, von deren Wucht und Depreſ{\pb}ſionsmacht Du
               keinerlei Ahnung haben kannſt. Ich mußte es zurücklegen für ſpäter. Und als dann das
               »ſpäter« kam, kam über mich das Unheil, das Du kennſt, mit der Unmöglichkeit, auch
               nur ein wenig Spannkraft zu finden, um aus dem mechaniſchen Trott der täglichen
               Arbeit heraus zugehen und \strikeout{\textcolor{gray}{×}} ein Werk\pwindex{Schnitzler, Arthur 15.05.1862 – 21.10.1931@\textsc{Schnitzler, Arthur} (15.05.1862 – 21.10.1931), \emph{Schriftsteller, Mediziner}!Anatol1892-10-29 – 1892-10-29@\strich\emph{Anatol} {[}1892-10-29 – 1892-10-29{]}|pwv} von Dir in
               einer Deiner würdigen Weiſe zu bearbeiten. Eine kleine Reklamenotiz hätte ich als
               einen \textsc{Affront} für Dich empfunden. Es mußte etwas Hübſches
               und Feines {\pb}ſein. Das aber war ich außerſtande zu
               ſchaffen. Noch heut bin ich es nicht imſtande. Denn
               ich bin nicht geheilt, werde es wohl auch nie werden, und bin durch dieſen Schlag und
               durch gewiſſen ſchweren Familien und Berufs-Kummer, durch die entſetzliche
               Zukunftsloſigkeit meiner \textsc{Carrière} zerbrochener als je. Um
               Dich nicht warten zu laſſen, ſandte mein Onkel\pwindex{Mamroth, Fedor 21.02.1851 – 25.06.1907@\textsc{Mamroth, Fedor} (21.02.1851 – 25.06.1907), \emph{Journalist, Kritiker}|pwv} ſofort Dein Buch\pwindex{Schnitzler, Arthur 15.05.1862 – 21.10.1931@\textsc{Schnitzler, Arthur} (15.05.1862 – 21.10.1931), \emph{Schriftsteller, Mediziner}!Anatol1892-10-29 – 1892-10-29@\strich\emph{Anatol} {[}1892-10-29 – 1892-10-29{]}|pwv} unſerem \label{K_L02709-5v}\edtext{Berlin\oindex{Berlin@\textbf{Berlin}|pw}er Berichterſtatter\pwindex{Stein, August 1851-06-02 – 1920-10-12@\textsc{Stein, August} (1851-06-02 – 1920-10-12), \emph{Schriftsteller, Journalist}|pwuv}}{\lemma{\textnormal{\emph{Berliner Berichterſtatter}}}\Cendnote{\textnormal{Es könnte sich hierbei um August Stein\pwindex{Stein, August 1851-06-02 – 1920-10-12@\textsc{Stein, August} (1851-06-02 – 1920-10-12), \emph{Schriftsteller, Journalist}|pwk} handeln, der das Berlin\oindex{Berlin@\textbf{Berlin}|pwk}er Büro der \emph{Frankfurter Zeitung}\orgindex{Frankfurter Zeitung@Frankfurter Zeitung|pwk} seit 1883
                  leitete.}}}\label{K_L02709-5h}. Der Herr\pwindex{Stein, August 1851-06-02 – 1920-10-12@\textsc{Stein, August} (1851-06-02 – 1920-10-12), \emph{Schriftsteller, Journalist}|pwuv} hat einfach nicht darüber geſchrieben. Und wie {\pb}bei unſerem Blatte\orgindex{Frankfurter Zeitung@Frankfurter Zeitung|pwv} die Verhältniſſe liegen, iſt mein Onkel\pwindex{Mamroth, Fedor 21.02.1851 – 25.06.1907@\textsc{Mamroth, Fedor} (21.02.1851 – 25.06.1907), \emph{Journalist, Kritiker}|pwv} machtlos, ihn dazu zu zwingen. Mein
                  Onkel\pwindex{Mamroth, Fedor 21.02.1851 – 25.06.1907@\textsc{Mamroth, Fedor} (21.02.1851 – 25.06.1907), \emph{Journalist, Kritiker}|pwv} ſelbſt hat ſich dann
               längere Zeit mit dem Gedanken getragen, ſelber darüber zu ſchreiben. Aber es iſt eine
               Unproductivität über ihn gekommen, die auch ihm die Feder lähmt, ſoweit es ſich nicht
               um Arbeiten handelt, die der Dienſt von ihm er zwingt. Das Alles iſt {\pb}\strikeout{mündlich} ſchriftlich ſchwer auseinanderzuſetzen.
               Mündlich würde ich es Dir leicht begreiflich machen. Das praktiſche Reſultat: Ich
               gehe nach \textsc{Paris}\oindex{Paris@\textbf{Paris}|pw} zurück, mit dem feſten Vorſatz, doch über Dein Werk\pwindex{Schnitzler, Arthur 15.05.1862 – 21.10.1931@\textsc{Schnitzler, Arthur} (15.05.1862 – 21.10.1931), \emph{Schriftsteller, Mediziner}!Anatol1892-10-29 – 1892-10-29@\strich\emph{Anatol} {[}1892-10-29 – 1892-10-29{]}|pwv} zu \label{K_L02709-7v}\edtext{ſchreiben}{\lemma{\textnormal{\emph{ſchreiben}}}\Cendnote{\textnormal{nicht
                  geschehen}}}\label{K_L02709-7h}, kann aber bei meinem ſchwachen Character für nichts einſtehen.
               Das Geſcheiteſte, im Intereſſe einer raſchen Erledigung, wäre, wenn einer von den Wien\oindex{Wien@\textbf{Wien}|pw}er Freunden, \textsc{Richard\pwindex{Beer-Hofmann, Richard 11.07.1866 – 26.09.1945@\textsc{Beer-Hofmann, Richard} (11.07.1866 – 26.09.1945), \emph{Schriftsteller}|pw}} oder \textsc{Loris\pwindex{Hofmannsthal, Hugo von 01.02.1874 – 15.07.1929@\textsc{Hofmannsthal, Hugo von} (01.02.1874 – 15.07.1929), \emph{Schriftsteller}|pw}}, uns ein kleines \introOben{}\label{K_L02709-8v}\edtext{Artikelchen}{\lemma{\textnormal{\emph{Artikelchen}}}\Cendnote{\textnormal{nicht geschehen}}}\label{K_L02709-8h}\introOben{}{ }\strikeout{\textcolor{gray}{×}\-\textcolor{gray}{×}\-\textcolor{gray}{×}\-\textcolor{gray}{×}\-\textcolor{gray}{×}\-\textcolor{gray}{×}} darüber machen wollte. Mein Onkel\pwindex{Mamroth, Fedor 21.02.1851 – 25.06.1907@\textsc{Mamroth, Fedor} (21.02.1851 – 25.06.1907), \emph{Journalist, Kritiker}|pwv} verſpricht {\pb}ſofortigen Abdruck. Wenn
               nicht, ſo gewähre mir, liebſter Freund, noch eine Friſt, und ich will alle Kraft
               aufbieten, um zu thun, was ich Dir ſchulde und was ich auch gar ſo gern thun
               möchte.\pend
           \pstart
           Über den Roman\pwindex{Schnitzler, Arthur 15.05.1862 – 21.10.1931@\textsc{Schnitzler, Arthur} (15.05.1862 – 21.10.1931), \emph{Schriftsteller, Mediziner}!Sterben. Novelle1.10.1894 – 1.12.1894@\strich\emph{Sterben. Novelle} {[}1.10.1894 – 1.12.1894{]}|pwv} haben wir lange
               geſprochen, mein Onkel\pwindex{Mamroth, Fedor 21.02.1851 – 25.06.1907@\textsc{Mamroth, Fedor} (21.02.1851 – 25.06.1907), \emph{Journalist, Kritiker}|pwv} und
               ich. Ein Abdruck in der Frkf. Ztg.\pwindex{Frankfurter Zeitung1856 – 1943@\emph{Frankfurter Zeitung}|pw} iſt unmöglich
               wegen der \label{K_L02709-4v}\edtext{Philiſterſität}{\lemma{\textnormal{\emph{Philiſterſität}}}\Cendnote{\textnormal{Kleinbürgerlichkeit, Engstirnigkeit}}}\label{K_L02709-4h}
               des Publicums. Weder mein Onkel\pwindex{Mamroth, Fedor 21.02.1851 – 25.06.1907@\textsc{Mamroth, Fedor} (21.02.1851 – 25.06.1907), \emph{Journalist, Kritiker}|pwv} noch ich ſind in keinen Beziehungen mit einem Verleger. {\pb}Das Einzige, was man für’s Erſte thun könnte, wäre
               ein Brief, den Du dann beifügſt, wenn Du das Manuſkript\pwindex{Schnitzler, Arthur 15.05.1862 – 21.10.1931@\textsc{Schnitzler, Arthur} (15.05.1862 – 21.10.1931), \emph{Schriftsteller, Mediziner}!Sterben. Novelle1.10.1894 – 1.12.1894@\strich\emph{Sterben. Novelle} {[}1.10.1894 – 1.12.1894{]}|pwv} einem \label{K_L02709-6v}\edtext{Verleger\orgindex{S. Fischer Verlag@S. Fischer Verlag|pwv} Deiner Wahl}{\lemma{\textnormal{\emph{Verleger Deiner Wahl}}}\Cendnote{\textnormal{In Buchform erschien \emph{Sterben}\pwindex{Schnitzler, Arthur 15.05.1862 – 21.10.1931@\textsc{Schnitzler, Arthur} (15.05.1862 – 21.10.1931), \emph{Schriftsteller, Mediziner}!Sterben. Novelle1.10.1894 – 1.12.1894@\strich\emph{Sterben. Novelle} {[}1.10.1894 – 1.12.1894{]}|pwk} erstmals im November 1894 (vordatiert auf 1895) bei \emph{S. Fischer}\orgindex{S. Fischer Verlag@S. Fischer Verlag|pwk}.}}}\label{K_L02709-6h} einſchickſt und der
               wenigſtens den Vortheil hat, Dir durch den Namen der Frankf. Ztg.\orgindex{Frankfurter Zeitung@Frankfurter Zeitung|pw} jene Accredition zu geben, deren Du bei jenen urtheilsloſen
               Buch-Handwerkern noch bedarfſt. Dein Stolz wird ſich gegen dieſes Mittel wehren, Dein
               Verſtand wird Dir zeigen, daß es doch {\pb}nicht zu
               verſchmähen iſt. Biſt Du aber erſt ein mal mit einem Verleger\orgindex{S. Fischer Verlag@S. Fischer Verlag|pwv} in Beziehung und brauchſt Du meinen Onkel\pwindex{Mamroth, Fedor 21.02.1851 – 25.06.1907@\textsc{Mamroth, Fedor} (21.02.1851 – 25.06.1907), \emph{Journalist, Kritiker}|pwv} oder mich zur weiteren
               Förderung der Angelegenheit, ſo wirſt Du uns auf dem Laufenden erhalten, und
               vielleicht ergibt ſich am Ende doch die Möglichkeit, etwas Poſitiveres und
               Specielleres zu erwirken.\pend
           \pstart
           Der Brief folgt anbei. {\pb}\strikeout{M}Nimm' dieſen Brief auch als Antwort meines Onkel\pwindex{Mamroth, Fedor 21.02.1851 – 25.06.1907@\textsc{Mamroth, Fedor} (21.02.1851 – 25.06.1907), \emph{Journalist, Kritiker}|pwv}s, der Dich lieb hat und
               Dir gern das Blaue vom Himmel herunterholen würde, wenn er könnte. Aber Du haſt keine
               Ahnung, w\strikeout{ie}as für arme, macht- und bedeutungsloſe
               Menſchen wir ſind, er und ich, wir zwei mit dem verfehlten Leben.\pend
           \pstart
           Grüß’ Dich Gott, mein theurer Freund! {\\[\baselineskip]}Dein {\\[\baselineskip]}\spacefill\mbox{Paul Goldmann.}\pend
           \leftskip=0em{}\endnumbering\briefempfaengerindex{Schnitzler, Arthur@\textsc{Schnitzler, Arthur}!zzzGoldmann, Paul@\emph{von Paul Goldmann}!1893-06-031@{3. 6. 1893}|)be}\mylabel{h}\end{ledgroupsized}\begin{anhang}\end{anhang}\newcommand{\dateiname}{L02709}\newcommand{\titel}{Paul Goldmann an Arthur Schnitzler, 3. 6. 1893}\newcommand{\editorInnen}{Martin Anton Müller und Laura Untner}\input{../tex-inputs/latex-pdf-abspann}
      