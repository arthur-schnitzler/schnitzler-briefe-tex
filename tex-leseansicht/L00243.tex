%% latex-leseansicht-vorspann.tex
%% Vorspann für die Leseansicht.
%% Lädt die gemeinsame Datei latex-vorspann.tex mit nicht gesetztem Schalter.

\newif\ifkorrekturansicht
\korrekturansichtfalse

\input{../tex-inputs/latex-vorspann}


         
         \renewcommand{\erwaehntePersonen}{Personen: Julius Bauer, Richard Beer-Hofmann, Bertha Flegmann, Carl Freund, Paul Horn, Josef Jarno, Felix Salten, Hermine von Schaffgotsch,  Waldner,  Waldner, Grethe Wreden}
         \renewcommand{\erwaehnteInstitutionen}{Institutionen: Freund {\kaufmannsund}  Jeckel}
         \renewcommand{\erwaehnteOrte}{Orte: Bad Ischl, Berlin, Café Tomaselli, Dauphiné, Salzburg, Wien}
         \renewcommand{\erwaehnteWerke}{Werke: Aus Ischl, Die Presse}
               \section[Richard Beer-Hofmann an Arthur Schnitzler, {[}25. 7. 1893{]}]{ Richard Beer-Hofmann an Arthur Schnitzler, {[}25. 7. 1893{]}}\nopagebreak\mylabel{v}\rehead{ }\begin{ledgroupsized}[t]{13cm}\normalsize\beginnumbering\briefempfaengerindex{Schnitzler, Arthur@\textsc{Schnitzler, Arthur}!zzzBeer-Hofmann, Richard@\emph{von Richard Beer-Hofmann}!1893-07-252@{{[}25. 7. 1893{]}}|(be} \toendnotes[C]{\smallbreak\pagebreak[2]} \Standort{CUL, Schnitzler, B 8.}
\physDesc{Brief, 2 Blätter, 5 Seiten, 1329 Zeichen
\newline{}Handschrift: Bleistift, deutsche Kurrent
\newline{}Schnitzler: 1) mit Bleistift datiert: »29/7 93«  2) mit Bleistift nummeriert: »21.«}\buchAbdrucke{\weitereDrucke{Arthur Schnitzler, Richard Beer-Hofmann: \emph{Briefwechsel 1891–1931}. Hg. Konstanze Fliedl. Wien, Zürich: \emph{Europaverlag} 1992, S. 47–48.} }\toendnotes[C]{\smallbreak}\pstart
           \raggedleft{}{\pb}Salzburg\oindex{Salzburg@\textbf{Salzburg}|pw}{ }Dienst.{ }Nachmittag{\\}\uline{bei Tomaselli\oindex{Cafe Tomaselli@\textbf{Café Tomaselli}|pw}}\pend
           \pstart
           Lieber Arthur! Soeben erhalte ich Ihren Brief nachgeschickt – ich
               bin in Salzburg\oindex{Salzburg@\textbf{Salzburg}|pw}; vielen Dank für Ihre Mühe – Ich
               bin seit Samst.{ }Nachm. hier – von Samstag{ }Abends bis gestern{ }Mittag in Gesellschaft. Lesen Sie die alte Presse\pwindex{?? Werk@Nicht ermittelte Verfasserinnen und Verfasser!Presse1848-07-03@\emph{Die Presse} {[}1848-07-03{]}|pw}, von Freitag »Ischler Brief\pwindex{Aus Ischl21. 7. 1893@\emph{Aus Ischl} {[}21. 7. 1893{]}|pw}«\substVorne{}\textsuperscript{,}\substDazwischen{}:\substHinten{} ganz vernünftig {\pb}anerkennungsvoll, hält es nur für die Bühne zu stark. Aber \uuline{lesen Sie selbst}. Mich beschimpft man noch manchmal, vom moralischen
               Standp. aus.\pend
           \pstart
           Jemand – ich glaube \uline{Frau}{ }Wald\textcolor{gray}{ner}\pwindex{Waldner @\textsc{Waldner}|pw}, er\pwindex{Waldner @\textsc{Waldner}|pwv} ist doch nicht so
                  du{\geminationm} – behauptete es wäre irgendetwas zwischen Ihnen
               und M. B{\dotssix}t\pwindex{Schaffgotsch, Hermine von 25.11.1871 – 25.11.1928@\textsc{Schaffgotsch, Hermine von} (25.11.1871 – 25.11.1928)|pw} im Zuge
               gewesen; aber {\pb}nachdem Sie
               derartige Sachen, \uuline{\edtext{aus Ihrem Leben!}{\Cendnote{achtfach unterstrichen}}} auf die Bühne
                  bringe{[}n{]}, scheine man eingesehen zu haben daß es denn doch
               nicht gienge; Jarno\pwindex{Jarno, Josef 24.08.1865 – 11.01.1932@\textsc{Jarno, Josef} (24.08.1865 – 11.01.1932), \emph{Theaterleiter, Schauspieler}|pw} habe ich ein einziges mal
               gesprochen. {\pb}Er kam zur Wreden\pwindex{Wreden, Grethe @\textsc{Wreden, Grethe}, \emph{Schauspielerin}|pw}, während ich u. Paul Horn\pwindex{Horn, Paul 13.02.1867 – 18.01.1936@\textsc{Horn, Paul} (13.02.1867 – 18.01.1936), \emph{Fabrikant}|pw} dort waren. Sind Sie mit Julius Bauer\pwindex{Bauer, Julius 15.10.1853 – 11.06.1941@\textsc{Bauer, Julius} (15.10.1853 – 11.06.1941), \emph{Schriftsteller, Journalist, Kritiker}|pw} zufrieden? Hier ist’s herrlich! ich schreibe ein
               wenig und feiere Orgien im Entbinden von Plänen; ich ergreife Pauschalbesitz von Salzburg\oindex{Salzburg@\textbf{Salzburg}|pw} – sagen Sie es Salten\pwindex{Salten, Felix 06.09.1869 – 08.10.1945@\textsc{Salten, Felix} (06.09.1869 – 08.10.1945), \emph{Schriftsteller, Journalist}|pw}, den ich herzlich grüße. Sie auch
                  \spacefill\mbox{Richard}\pend
           \pstart
           \noindent{}{\pb}Soeben fällt mir ein daß ich
                  bez. Verlag v. Freund\orgindex{Freund und Jeckel@Freund {\kaufmannsund}  Jeckel|pw} nicht geantwortet habe.
                     Flegmann\pwindex{Flegmann, Bertha 27.05.1852 – 24.6.1933@\textsc{Flegmann, Bertha} (27.05.1852 – 24.6.1933), \emph{Salonnière}|pw} bat mich Ihnen mitzuteilen daß
                     Freund\pwindex{Freund, Carl @\textsc{Freund, Carl}, \emph{Verleger}|pw} nicht in Berlin\oindex{Berlin@\textbf{Berlin}|pw}, \uline{nicht} in d. Bädern
                  sei, sondern in der – Dauphinée\oindex{Dauphine@\textbf{Dauphiné}|pw} – bitte
                  nachzusehen ob die Orthographie richtig – Bis zu seiner Rückkehr kann man nichts
                  tun\pend
           \pstart
           \raggedleft{}\spacefill\mbox{R.}\pend
           \pstart
           \noindent{}\label{T_L00243-1v}\edtext{Ich reise morgen nach Ischl\oindex{Bad Ischl@\textbf{Bad Ischl}|pw} zurück.}{\lemma{\textnormal{\emph{Ich … zurück.}}}\Cendnote{\textnormal{quer am rechten Rand der vierten Seite}}}\label{T_L00243-1h}\pend
           
         
         \endnumbering\mylabel{h}\end{ledgroupsized}  \newcommand{\dateiname}{L00243}\newcommand{\titel}{Richard Beer-Hofmann an Arthur Schnitzler, [25. 7. 1893]}\newcommand{\editorInnen}{Martin Anton Müller und Gerd-Hermann Susen}%% latex-leseansicht-abspann.tex
%% Abspann für die Leseansicht.
%% Der Schalter \ifkorrekturansicht ist bereits durch den Vorspann gesetzt.

%% latex-abspann.tex
%% Gemeinsamer Abspann für Korrekturansicht und Leseansicht.
%% Setzt den Schalter \ifkorrekturansicht voraus (gesetzt in den
%% einbindenden Dateien latex-korrekturansicht-abspann.tex bzw.
%% latex-leseansicht-abspann.tex).
%% ---------------------------------------------------------------

\normalsize

% Das esempio-Environment wird nur in der Leseansicht benötigt
\ifkorrekturansicht\else
\newenvironment{esempio}[3]%
{
    \vspace{1.5ex}
    \rlap{\underline{#1}}
    \par
    \setlength{\parindent}{0cm}
    \nopagebreak
    \leftskip=#2cm
    \rightskip=#3cm
}
{
    \par
}
\fi

\doendnotes{C}
\bigskip
\vfill

\clearpage

\footnotesize

\ifkorrekturansicht
  \lohead{\textsc{register}}
\fi

% theindex-Environment neu definieren ohne reledmac
\makeatletter
\renewenvironment{theindex}{%
  \ifkorrekturansicht
    \section*{\indexname}%
  \else
    \subsubsection*{Index der erwähnten Entitäten}%
  \fi
  \setlength{\parindent}{0pt}%
  \setlength{\parskip}{0pt plus 0.3pt}%
  \let\item\@idxitem
}{%
  \ifkorrekturansicht\clearpage\fi
}
\makeatother

\IfFileExists{\jobname-pw.ind}{\input{\jobname-pw.ind}}{}

% Quellenangabe nur in der Leseansicht
\ifkorrekturansicht\else
% Fallback-Definitionen, falls die .tex-Datei \titel etc. nicht gesetzt hat
\providecommand{\titel}{}
\providecommand{\editorInnen}{}
\providecommand{\dateiname}{\jobname}

\vspace{3cm}

\vfill

\footnotesize
\textsc{Quelle}: \titel. Herausgegeben von {\editorInnen}. In: \emph{Arthur Schnitzler: Briefwechsel mit Autorinnen und Autoren}.
 Digitale Edition, https://schnitzler-briefe.acdh.oeaw.ac.at/{\dateiname}.html (Stand \today)
\fi

\end{document}


      