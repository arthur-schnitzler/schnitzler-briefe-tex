%% latex-leseansicht-vorspann.tex
%% Vorspann für die Leseansicht.
%% Lädt die gemeinsame Datei latex-vorspann.tex mit nicht gesetztem Schalter.

\newif\ifkorrekturansicht
\korrekturansichtfalse

\input{../tex-inputs/latex-vorspann}


\section[Arthur Schnitzler an Stefan Zweig, 27. 10. 1912]{L03783 Arthur Schnitzler an Stefan Zweig, 27. 10. 1912}
\nopagebreak\mylabel{L03783v}
\rehead{ }\normalsize\beginnumbering\briefempfaengerindex{Zweig, Stefan@\textsc{Zweig, Stefan}!zzzSchnitzler, Arthur@\emph{von Arthur Schnitzler}!1912-10-271@{27. 10. 1912}|(be}
\toendnotes[C]{\smallbreak\pagebreak[2]}
\correspDesc{Versand  durch Arthur Schnitzler am 27. 10. 1912 in Wien
\newline{}Erhalt  durch Stefan Zweig im Zeitraum [27. 10. 1912 – 30. 10. 1912?] in Wien}\toendnotes[C]{\smallbreak}
\Standort{Jerusalem, National Library of Israel, ARC. Ms. Var. 305 1 58 Stefan Zweig Collection.}
\physDesc{Briefkarte, 312 Zeichen
\newline{}Handschrift: schwarze Tinte, deutsche Kurrent}\toendnotes[C]{\smallbreak}
\pstart
           {\pb}\textcolor{gray}{\textbf{Dr. Arthur Schnitzler}}\hfill 27. X. 912\pend
           
\pstart
           \textcolor{gray}{\textbf{Wien XVIII. Sternwartestrasse 71\oindex{Wien@\textbf{Wien}!XVIII., Währing@\textbf{XVIII., Währing}!Sternwartestraße 71@\textbf{Sternwartestraße 71}, \emph{Wohngebäude}|pw}}}\pend
           \vspace{0.5em}
\pstart
           lieber Doctor Zweig, ich freue mich über Ihren \label{K_L03783-1v}\edtext{Erfolg\eventindex{Burgtheater@\textbf{Burgtheater}!Uraufführung von Das Haus am Meer, 26.10.1912@Uraufführung von Das Haus am Meer, 26.10.1912|pwv}\pwindex{Zweig, Stefan 28.\,11.\,1881 Wien – 23.\,2.\,1942 Petrópolis@\textsc{Zweig, Stefan} (28.\,11.\,1881 Wien – 23.\,2.\,1942 Petrópolis), \emph{Schriftsteller}!Haus am Meer. Ein Schauspiel in zwei Teilen (drei Aufzügen)@\strich\emph{Das Haus am Meer. Ein Schauspiel in zwei Teilen (drei Aufzügen)}|pwv}}{\lemma{\textnormal{\emph{Erfolg}}}\Cendnote{\textnormal{Am 25. 10. 1912 besuchte Schnitzler die Generalprobe\eventindex{Burgtheater@\textbf{Burgtheater}!Generalprobe von Das Haus am Meer, 25.10.1912@Generalprobe von Das Haus am Meer, 25.10.1912|pwkv} von \emph{Das
                     Haus am Meer}\pwindex{Zweig, Stefan 28.\,11.\,1881 Wien – 23.\,2.\,1942 Petrópolis@\textsc{Zweig, Stefan} (28.\,11.\,1881 Wien – 23.\,2.\,1942 Petrópolis), \emph{Schriftsteller}!Haus am Meer. Ein Schauspiel in zwei Teilen (drei Aufzügen)@\strich\emph{Das Haus am Meer. Ein Schauspiel in zwei Teilen (drei Aufzügen)}|pwk} im Burgtheater\oindex{Wien@\textbf{Wien}!I., Innere Stadt@\textbf{I., Innere Stadt}!Burgtheater@\textbf{Burgtheater}, \emph{Theater}|pwk}. Am
                  Folgetag fand die Uraufführung\eventindex{Burgtheater@\textbf{Burgtheater}!Uraufführung von Das Haus am Meer, 26.10.1912@Uraufführung von Das Haus am Meer, 26.10.1912|pwkv} statt, auf deren Rezeption Schnitzler hier reagiert.}}}\label{K_L03783-1} und beglückwünſche Sie herzlich. Der erſte\pwindex{Zweig, Stefan 28.\,11.\,1881 Wien – 23.\,2.\,1942 Petrópolis@\textsc{Zweig, Stefan} (28.\,11.\,1881 Wien – 23.\,2.\,1942 Petrópolis), \emph{Schriftsteller}!Haus am Meer. Ein Schauspiel in zwei Teilen (drei Aufzügen)@\strich\emph{Das Haus am Meer. Ein Schauspiel in zwei Teilen (drei Aufzügen)}|pwv} und zweite Akt\pwindex{Zweig, Stefan 28.\,11.\,1881 Wien – 23.\,2.\,1942 Petrópolis@\textsc{Zweig, Stefan} (28.\,11.\,1881 Wien – 23.\,2.\,1942 Petrópolis), \emph{Schriftsteller}!Haus am Meer. Ein Schauspiel in zwei Teilen (drei Aufzügen)@\strich\emph{Das Haus am Meer. Ein Schauspiel in zwei Teilen (drei Aufzügen)}|pwv} haben auch von der Bühne\eventindex{Burgtheater@\textbf{Burgtheater}!Generalprobe von Das Haus am Meer, 25.10.1912@Generalprobe von Das Haus am Meer, 25.10.1912|pwv}
               her{ }ſtark auf mich {\pb}gewirkt; was ich etwa hinſichtlich des
                  dritten\pwindex{Zweig, Stefan 28.\,11.\,1881 Wien – 23.\,2.\,1942 Petrópolis@\textsc{Zweig, Stefan} (28.\,11.\,1881 Wien – 23.\,2.\,1942 Petrópolis), \emph{Schriftsteller}!Haus am Meer. Ein Schauspiel in zwei Teilen (drei Aufzügen)@\strich\emph{Das Haus am Meer. Ein Schauspiel in zwei Teilen (drei Aufzügen)}|pwv} auf dem Herzen
               hätte, darf ich Ihnen vielleicht bei \label{K_L03783-2v}\edtext{Gelegenheit}{\lemma{\textnormal{\emph{Gelegenheit}}}\Cendnote{\textnormal{Das nächste
                  nachgewiesene Zusammentreffen fand beim Hauptmann-Bankett\eventindex{Österreichischer Ingenieur- und Architektenverein@\textbf{Österreichischer Ingenieur- und Architektenverein}!Hauptmann-Bankett der Concordia, 17.11.1912@Hauptmann-Bankett der Concordia, 17.11.1912|pwk} am 17. 11. 1912 statt.}}}\label{K_L03783-2}{ }ſagen.\pend
           
\pstart
           Auf baldgs Wiederſehen.{\\[\baselineskip]}Ihr{\\[\baselineskip]}\spacefill\mbox{Arthur Schnitzler}\pend
           \leftskip=0em{}\selectlanguage{ngerman}\endnumbering\briefempfaengerindex{Zweig, Stefan@\textsc{Zweig, Stefan}!zzzSchnitzler, Arthur@\emph{von Arthur Schnitzler}!1912-10-271@{27. 10. 1912}|)be}\mylabel{L03783h}  \newcommand{\dateiname}{L03783}\newcommand{\titel}{Arthur Schnitzler an Stefan Zweig, 27. 10. 1912}\newcommand{\editorInnen}{Selma Jahnke und Martin Anton Müller}%% latex-leseansicht-abspann.tex
%% Abspann für die Leseansicht.
%% Der Schalter \ifkorrekturansicht ist bereits durch den Vorspann gesetzt.

%% latex-abspann.tex
%% Gemeinsamer Abspann für Korrekturansicht und Leseansicht.
%% Setzt den Schalter \ifkorrekturansicht voraus (gesetzt in den
%% einbindenden Dateien latex-korrekturansicht-abspann.tex bzw.
%% latex-leseansicht-abspann.tex).
%% ---------------------------------------------------------------

\normalsize

% Das esempio-Environment wird nur in der Leseansicht benötigt
\ifkorrekturansicht\else
\newenvironment{esempio}[3]%
{
    \vspace{1.5ex}
    \rlap{\underline{#1}}
    \par
    \setlength{\parindent}{0cm}
    \nopagebreak
    \leftskip=#2cm
    \rightskip=#3cm
}
{
    \par
}
\fi

\doendnotes{C}
\bigskip
\vfill

\clearpage

\footnotesize

\ifkorrekturansicht
  \lohead{\textsc{register}}
\fi

% theindex-Environment neu definieren ohne reledmac
\makeatletter
\renewenvironment{theindex}{%
  \ifkorrekturansicht
    \section*{\indexname}%
  \else
    \subsubsection*{Index der erwähnten Entitäten}%
  \fi
  \setlength{\parindent}{0pt}%
  \setlength{\parskip}{0pt plus 0.3pt}%
  \let\item\@idxitem
}{%
  \ifkorrekturansicht\clearpage\fi
}
\makeatother

\IfFileExists{\jobname-pw.ind}{\input{\jobname-pw.ind}}{}

% Quellenangabe nur in der Leseansicht
\ifkorrekturansicht\else
% Fallback-Definitionen, falls die .tex-Datei \titel etc. nicht gesetzt hat
\providecommand{\titel}{}
\providecommand{\editorInnen}{}
\providecommand{\dateiname}{\jobname}

\vspace{3cm}

\vfill

\footnotesize
\textsc{Quelle}: \titel. Herausgegeben von {\editorInnen}. In: \emph{Arthur Schnitzler: Briefwechsel mit Autorinnen und Autoren}.
 Digitale Edition, https://schnitzler-briefe.acdh.oeaw.ac.at/{\dateiname}.html (Stand \today)
\fi

\end{document}


