%% latex-korrekturansicht-vorspann.tex
%% Vorspann für die Korrekturansicht.
%% Lädt die gemeinsame Datei latex-vorspann.tex mit gesetztem Schalter.

\newif\ifkorrekturansicht
\korrekturansichttrue

\input{../tex-inputs/latex-vorspann}


\section[Hugo von Hofmannsthal an Arthur Schnitzler, 10. 4. 1892]{L00093 Hugo von Hofmannsthal an Arthur Schnitzler, 10. 4. 1892}
\nopagebreak\mylabel{L00093v}
\rehead{ }\normalsize\beginnumbering\briefempfaengerindex{Schnitzler, Arthur@\textsc{Schnitzler, Arthur}!zzzHofmannsthal, Hugo von@\emph{von Hugo von Hofmannsthal}!1892-04-101@{10. 4. 1892}|(be}
\toendnotes[C]{\smallbreak\pagebreak[2]}\Standort{CUL, Schnitzler, B 43.}
\physDesc{Kartenbrief, 453 Zeichen
\newline{}Handschrift: Bleistift, deutsche Kurrent
\newline{}Versand: Stempel: »\nobreak{}Wien 3/3, 10. 4. 92, 3–4N\nobreak{}«.  
\newline{}Schnitzler: mit Bleistift datiert: »12/4 92« 
\newline{}Ordnung: mit Bleistift von unbekannter Hand nummeriert:
                                    »22« }
\buchAbdrucke{\weitereDrucke{Hugo von Hofmannsthal, Arthur Schnitzler: \emph{Briefwechsel}. Frankfurt am Main: \emph{S. Fischer} 1964, S. 20.} }\pstart{}{\pb}Herrn \textsc{D\textsuperscript{r} Arthur Schnitzler}\pend{}\pstart{}\textsc{Wien\oindex{Wien@\textbf{Wien}, \emph{A.ADM2}|pw}}\pend{}\pstart{}\textsc{I Kärntnerring 12\oindex{Kaerntnerring 12/Boesendorferstrasse 11@\textbf{Kärntnerring 12/Bösendorferstraße 11}, \emph{Wohngebäude (K.WHS)}|pw}.}\pend{}{\bigskip}\vspace{1em}
\pstart{}{\pb}Lieber Arthur.\pend\vspace{0.5em}
\pstart
           Schwarzkopf\pwindex{Schwarzkopf, Gustav 07.11.1853 – 13.11.1939@\textsc{Schwarzkopf, Gustav} (07.11.1853 – 13.11.1939), \emph{Schriftsteller/Schriftstellerin}|pw} und Karlweiß\pwindex{Karlweis, Carl 23.11.1850 – 27.10.1901@\textsc{Karlweis, Carl} (23.11.1850 – 27.10.1901), \emph{Schriftsteller/Schriftstellerin}|pw} möchten ſich an unſerer Landpartie betheiligen. Und
               zwar wurde (auch Richard\pwindex{Beer-Hofmann, Richard 1866-07-11 – 1945-09-26@\textsc{Beer-Hofmann, Richard} (1866-07-11 – 1945-09-26), \emph{Schriftsteller/Schriftstellerin}|pw} iſt einverſtanden)
               ein \textsc{rendez vous} für Charfreitag pünktlich ½ 3 Uhr bei Grienſteidl\oindex{Cafe Griensteidl@\textbf{Café Griensteidl}, \emph{Kaffeehaus (K.KAF)}|pw} verabredet. Sie brauchen nur \textsc{Salten}\pwindex{Salten, Felix 06.09.1869 – 08.10.1945@\textsc{Salten, Felix} (06.09.1869 – 08.10.1945), \emph{Schriftsteller/Schriftstellerin, Journalist/Journalistin, Chefredakteur/Chefredakteurin}|pw} zu verſtändigen und mir nur dann zu ſchreiben, wenn es Ihnen \uline{nicht} paſst, was mir natürlich mehr als unangenehm wäre.
               Dann müſste man eben eine neue Verabredung treffen.\pend
           
\pstart
           Herzlichſt{\\[\baselineskip]}Ihr{\\[\baselineskip]}\spacefill\mbox{Loris}\pend
           \leftskip=0em{}\selectlanguage{ngerman}\endnumbering\briefempfaengerindex{Schnitzler, Arthur@\textsc{Schnitzler, Arthur}!zzzHofmannsthal, Hugo von@\emph{von Hugo von Hofmannsthal}!1892-04-101@{10. 4. 1892}|)be}\mylabel{L00093h}  \normalsize

\doendnotes{C}
\bigskip
\vfill

\clearpage

\footnotesize

\lohead{\textsc{register}}

% Definiere theindex-Environment komplett neu ohne reledmac
\makeatletter
\renewenvironment{theindex}{%
  \section*{\indexname}%
  \setlength{\parindent}{0pt}%
  \setlength{\parskip}{0pt plus 0.3pt}%
  \let\item\@idxitem
}{%
  \clearpage
}
\makeatother

\IfFileExists{\jobname-pw.ind}{\input{\jobname-pw.ind}}{}

\end{document}

      