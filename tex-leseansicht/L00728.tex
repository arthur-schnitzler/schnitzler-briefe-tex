%% latex-leseansicht-vorspann.tex
%% Vorspann für die Leseansicht.
%% Lädt die gemeinsame Datei latex-vorspann.tex mit nicht gesetztem Schalter.

\newif\ifkorrekturansicht
\korrekturansichtfalse

\input{../tex-inputs/latex-vorspann}


         
         \newcommand{\erwaehntePersonen}{Personen: Richard Beer-Hofmann, Hugo August von Hofmannsthal, Anna von Hofmannsthal, Marie Reinhard, Ermete Zacconi}
         \newcommand{\erwaehnteOrte}{Orte: Carl-Theater, Hinterbrühl, Wien}
         \newcommand{\erwaehnteWerke}{Werke: König Lear}
               \section[Hugo von Hofmannsthal an Arthur Schnitzler, 3. 10. {[}1897{]}]{ Hugo von Hofmannsthal an Arthur Schnitzler, 3. 10. {[}1897{]}}\nopagebreak\mylabel{v}\rehead{ }\begin{ledgroupsized}[t]{13cm}\normalsize\beginnumbering \toendnotes[C]{\smallbreak\pagebreak[2]} \Standort{CUL, Schnitzler, B 43.}
\physDesc{Brief, 1 Blatt, 3 Seiten
\newline{}Handschrift: schwarze Tinte, deutsche Kurrent
\newline{}Schnitzler: mit Bleistift die Jahreszahl ergänzt: »97« \newline{}Ordnung: 1) mit Bleistift von unbekannter Hand nummeriert: »\strikeout{103}«  2) mit Bleistift von unbekannter Hand nummeriert:
                                    »96«}\buchAbdrucke{\weitereDrucke{Hugo von Hofmannsthal, Arthur Schnitzler: \emph{Briefwechsel}. Hg. Therese Nickl und Heinrich Schnitzler. Frankfurt am Main: \emph{S. Fischer} 1964, S. 95–96.} }\toendnotes[C]{\smallbreak}\pstart
           \raggedleft{}{\pb}Hinterbrühl\oindex{Hinterbruehl@\textbf{Hinterbrühl}|pw}{ }3\textsuperscript{ten} X\textsuperscript{ten}.\pend
           \pstart{}mein lieber Arthur\pend\pstart
           Ihr Geſicht iſt mir \label{K_L00728_1v}\edtext{neulich}{\lemma{\textnormal{\emph{neulich}}}\Cendnote{\textnormal{Am 26., 28. und
                     30. 9. 1897 besuchte Schnitzler\pwindex{Schnitzler, Arthur 15.05.1862 – 21.10.1931@\textsc{Schnitzler, Arthur} (15.05.1862 – 21.10.1931), \emph{Schriftsteller, Mediziner}|pwk}
                  das Gastspiel von Ermete Zacconi\pwindex{Zacconi, Ermete 14.09.1857 – 14.10.1948@\textsc{Zacconi, Ermete} (14.09.1857 – 14.10.1948), \emph{Regisseur, Schauspieler}|pwk} im Carl-Theater\oindex{Carl-Theater@\textbf{Carl-Theater}|pwk}. Hofmannsthal\pwindex{Hofmannsthal, Hugo von 1874-02-01 – 1929-07-15@\textsc{Hofmannsthal, Hugo von} (1874-02-01 – 1929-07-15), \emph{Schriftsteller}|pwk} hielt sich am Land auf, konnte aber in die Stadt reisen und
                  war aber nachweislich in der Vorstellung des \emph{König
                     Lear}\pwindex{\textcolor{red}{\textsuperscript{XXXX1 indx}}!Koenig Lear1606@\strich\emph{König Lear} {[}1606{]}|pwk} am letzten der genannten Tage (Brief an die Eltern\pwindex{Hofmannsthal, Hugo August von 21.12.1841 – 08.12.1915@\textsc{Hofmannsthal, Hugo August von} (21.12.1841 – 08.12.1915), \emph{Bankdirektor}|pwkv}\pwindex{Hofmannsthal, Anna von 27.01.1849 – 22.03.1904@\textsc{Hofmannsthal, Anna von} (27.01.1849 – 22.03.1904)|pwkv}).}}}\label{K_L00728_1h} ſchon von der Loge aus ſehr ernſt und
               traurig erſchienen, ich bin dann zu Richard\pwindex{Beer-Hofmann, Richard 1866-07-11 – 1945-09-26@\textsc{Beer-Hofmann, Richard} (1866-07-11 – 1945-09-26), \emph{Schriftsteller}|pw}
               gegangen, er hat mir alles \label{K_L00728_2v}\edtext{erzählt}{\lemma{\textnormal{\emph{erzählt}}}\Cendnote{\textnormal{Marie Reinhard\pwindex{Reinhard, Marie 1871-03-13 – 1899-03-18@\textsc{Reinhard, Marie} (1871-03-13 – 1899-03-18), \emph{Gesangspädagogin}|pwk} und er betrauerten gerade ein am
                     24. 9. 1897 totgeborenes Kind.}}}\label{K_L00728_2h} und deshalb habe ich Ihnen
               unter den vielen fremden Leuten nur die Hand gegeben und nichts geſagt. Ich weiß
               Ihnen {\pb}nichts tröſtliches zu ſagen
               und ob Ihnen meine Zuneigung und Anhänglichkeit irgend eine wirkliche Freude macht,
               weiß ich nicht, deshalb will ich auch nicht davon ſprechen. Ich hoffe von Herzen,
               daſs Sie bald wieder oder ſchon wieder arbeiten können. Ich werde {\pb}wohl die nächſte Woche nach Wien\oindex{Wien@\textbf{Wien}|pw} kommen und hätte Ihnen und dem Richard\pwindex{Beer-Hofmann, Richard 1866-07-11 – 1945-09-26@\textsc{Beer-Hofmann, Richard} (1866-07-11 – 1945-09-26), \emph{Schriftsteller}|pw}, wenn Sie beide aufgelegt ſind, recht viel
               vorzuleſen.\pend
           \pstart
           Herzlich{\\[\baselineskip]}Ihr{\\[\baselineskip]}\spacefill\mbox{Hugo.}\pend
           \leftskip=0em{}
         
         \endnumbering\mylabel{h}\end{ledgroupsized}  \newcommand{\dateiname}{L00728}\newcommand{\titel}{Hugo von Hofmannsthal an Arthur Schnitzler, 3. 10. [1897]}\newcommand{\editorInnen}{Martin Anton Müller und Gerd-Hermann Susen}%% latex-leseansicht-abspann.tex
%% Abspann für die Leseansicht.
%% Der Schalter \ifkorrekturansicht ist bereits durch den Vorspann gesetzt.

%% latex-abspann.tex
%% Gemeinsamer Abspann für Korrekturansicht und Leseansicht.
%% Setzt den Schalter \ifkorrekturansicht voraus (gesetzt in den
%% einbindenden Dateien latex-korrekturansicht-abspann.tex bzw.
%% latex-leseansicht-abspann.tex).
%% ---------------------------------------------------------------

\normalsize

% Das esempio-Environment wird nur in der Leseansicht benötigt
\ifkorrekturansicht\else
\newenvironment{esempio}[3]%
{
    \vspace{1.5ex}
    \rlap{\underline{#1}}
    \par
    \setlength{\parindent}{0cm}
    \nopagebreak
    \leftskip=#2cm
    \rightskip=#3cm
}
{
    \par
}
\fi

\doendnotes{C}
\bigskip
\vfill

\clearpage

\footnotesize

\ifkorrekturansicht
  \lohead{\textsc{register}}
\fi

% theindex-Environment neu definieren ohne reledmac
\makeatletter
\renewenvironment{theindex}{%
  \ifkorrekturansicht
    \section*{\indexname}%
  \else
    \subsubsection*{Index der erwähnten Entitäten}%
  \fi
  \setlength{\parindent}{0pt}%
  \setlength{\parskip}{0pt plus 0.3pt}%
  \let\item\@idxitem
}{%
  \ifkorrekturansicht\clearpage\fi
}
\makeatother

\IfFileExists{\jobname-pw.ind}{\input{\jobname-pw.ind}}{}

% Quellenangabe nur in der Leseansicht
\ifkorrekturansicht\else
% Fallback-Definitionen, falls die .tex-Datei \titel etc. nicht gesetzt hat
\providecommand{\titel}{}
\providecommand{\editorInnen}{}
\providecommand{\dateiname}{\jobname}

\vspace{3cm}

\vfill

\footnotesize
\textsc{Quelle}: \titel. Herausgegeben von {\editorInnen}. In: \emph{Arthur Schnitzler: Briefwechsel mit Autorinnen und Autoren}.
 Digitale Edition, https://schnitzler-briefe.acdh.oeaw.ac.at/{\dateiname}.html (Stand \today)
\fi

\end{document}


      