%% latex-korrekturansicht-vorspann.tex
%% Vorspann für die Korrekturansicht.
%% Lädt die gemeinsame Datei latex-vorspann.tex mit gesetztem Schalter.

\newif\ifkorrekturansicht
\korrekturansichttrue

\input{../tex-inputs/latex-vorspann}


\section[Hugo von Hofmannsthal an Arthur Schnitzler, 3. 10. {[}1897{]}]{L00728 Hugo von Hofmannsthal an Arthur Schnitzler, 3. 10. {[}1897{]}}
\nopagebreak\mylabel{L00728v}
\rehead{ }\normalsize\beginnumbering\briefempfaengerindex{Schnitzler, Arthur@\textsc{Schnitzler, Arthur}!zzzHofmannsthal, Hugo von@\emph{von Hugo von Hofmannsthal}!1897-10-031@{3. 10. {[}1897{]}}|(be}
\toendnotes[C]{\smallbreak\pagebreak[2]}\Standort{CUL, Schnitzler, B 43.}
\physDesc{Brief, 1 Blatt, 3 Seiten, 689 Zeichen
\newline{}Handschrift: schwarze Tinte, deutsche Kurrent
\newline{}Schnitzler: mit Bleistift die Jahreszahl ergänzt: »97« 
\newline{}Ordnung: 1) mit Bleistift von unbekannter Hand nummeriert: »\strikeout{103}«  2) mit Bleistift von unbekannter Hand nummeriert: »96«}
\buchAbdrucke{\weitereDrucke{Hugo von Hofmannsthal, Arthur Schnitzler: \emph{Briefwechsel}. Frankfurt am Main: \emph{S. Fischer} 1964, S. 95–96.} }\toendnotes[C]{\smallbreak}
\pstart
           \raggedleft{}{\pb}Hinterbrühl\oindex{Hinterbruehl@\textbf{Hinterbrühl}, \emph{P.PPLA3}|pw}{ }3\textsuperscript{ten} X\textsuperscript{ten}.\pend
           
\pstart{}mein lieber Arthur\pend\vspace{0.5em}
\pstart
           Ihr Geſicht iſt mir \label{K_L00728-1v}\edtext{neulich}{\lemma{\textnormal{\emph{neulich}}}\Cendnote{\textnormal{Am 26. 9. 1897, 28. 9. 1897 und
               30. 9. 1897 hatte Schnitzler das Gastspiel von Ermete
                     Zacconi\pwindex{Zacconi, Ermete 14.09.1857 – 14.10.1948@\textsc{Zacconi, Ermete} (14.09.1857 – 14.10.1948), \emph{Regisseur/Regisseurin, Schauspieler/Schauspielerin}|pwk} im Carl-Theater\oindex{Carl-Theater@\textbf{Carl-Theater}, \emph{Theater (K.THE)}|pwk} besucht. Hofmannsthal\pwindex{Hofmannsthal, Hugo von 1874-02-01 – 1929-07-15@\textsc{Hofmannsthal, Hugo von} (1874-02-01 – 1929-07-15), \emph{Schriftsteller/Schriftstellerin}|pwk} hielt sich am Land auf, konnte
                  aber in die Stadt reisen und war nachweislich in der Vorstellung des \emph{König Lear}\pwindex{Koenig Lear. Trauerspiel in fuenf Aufzuegen@\emph{König Lear. Trauerspiel in fünf Aufzügen}|pwk} am letzten der genannten Tage
                  (Brief an die Eltern\pwindex{Hofmannsthal, Hugo August von 21.12.1841 – 08.12.1915@\textsc{Hofmannsthal, Hugo August von} (21.12.1841 – 08.12.1915), \emph{Bankdirektor/Bankdirektorin}|pwkv}\pwindex{Hofmannsthal, Anna von 27.01.1849 – 22.03.1904@\textsc{Hofmannsthal, Anna von} (27.01.1849 – 22.03.1904)|pwkv}).}}}\label{K_L00728-1} ſchon von der Loge aus ſehr ernſt und traurig erſchienen,
               ich bin dann zu Richard\pwindex{Beer-Hofmann, Richard 1866-07-11 – 1945-09-26@\textsc{Beer-Hofmann, Richard} (1866-07-11 – 1945-09-26), \emph{Schriftsteller/Schriftstellerin}|pw} gegangen, er hat mir
               alles \label{K_L00728-2v}\edtext{erzählt}{\lemma{\textnormal{\emph{erzählt}}}\Cendnote{\textnormal{Marie Reinhard\pwindex{Reinhard, Marie 1871-03-13 – 1899-03-18@\textsc{Reinhard, Marie} (1871-03-13 – 1899-03-18), \emph{Gesangspädagoge/Gesangspädagogin}|pwk} und er betrauerten das
                  am 24. 9. 1897 totgeborene gemeinsame Kind\pwindex{?? [Totgeborener Sohn von Arthur Schnitzler und Marie Reinhard] 1897-09-24 – 1897-09-24@\textsc{?? [Totgeborener Sohn von Arthur Schnitzler und Marie Reinhard]} (1897-09-24 – 1897-09-24)|pwkv}.}}}\label{K_L00728-2} und deshalb habe ich Ihnen
               unter den vielen fremden Leuten nur die Hand gegeben und nichts geſagt. Ich weiß
               Ihnen {\pb}nichts tröſtliches zu ſagen
               und ob Ihnen meine Zuneigung und Anhänglichkeit irgend eine wirkliche Freude macht,
               weiß ich nicht, deshalb will ich auch nicht davon ſprechen. Ich hoffe von Herzen,
               daſs Sie bald wieder oder ſchon wieder arbeiten können. Ich werde {\pb}wohl die nächſte Woche nach Wien\oindex{Wien@\textbf{Wien}, \emph{A.ADM2}|pw} kommen und hätte Ihnen und dem Richard\pwindex{Beer-Hofmann, Richard 1866-07-11 – 1945-09-26@\textsc{Beer-Hofmann, Richard} (1866-07-11 – 1945-09-26), \emph{Schriftsteller/Schriftstellerin}|pw}, wenn Sie beide aufgelegt ſind, recht
               viel vorzuleſen.\pend
           
\pstart
           Herzlich{\\[\baselineskip]}Ihr{\\[\baselineskip]}\spacefill\mbox{Hugo.}\pend
           \leftskip=0em{}\selectlanguage{ngerman}\endnumbering\briefempfaengerindex{Schnitzler, Arthur@\textsc{Schnitzler, Arthur}!zzzHofmannsthal, Hugo von@\emph{von Hugo von Hofmannsthal}!1897-10-031@{3. 10. {[}1897{]}}|)be}\mylabel{L00728h}  \normalsize

\doendnotes{C}
\bigskip
\vfill

\clearpage

\footnotesize

\lohead{\textsc{register}}

% Definiere theindex-Environment komplett neu ohne reledmac
\makeatletter
\renewenvironment{theindex}{%
  \section*{\indexname}%
  \setlength{\parindent}{0pt}%
  \setlength{\parskip}{0pt plus 0.3pt}%
  \let\item\@idxitem
}{%
  \clearpage
}
\makeatother

\IfFileExists{\jobname-pw.ind}{\input{\jobname-pw.ind}}{}

\end{document}

      