%% latex-leseansicht-vorspann.tex
%% Vorspann für die Leseansicht.
%% Lädt die gemeinsame Datei latex-vorspann.tex mit nicht gesetztem Schalter.

\newif\ifkorrekturansicht
\korrekturansichtfalse

\input{../tex-inputs/latex-vorspann}


\section[Hugo von Hofmannsthal an Arthur Schnitzler, 3. 10. {[}1897{]}]{L00728 Hugo von Hofmannsthal an Arthur Schnitzler, 3. 10. [1897]}
\nopagebreak\mylabel{L00728v}
\rehead{ }\normalsize\beginnumbering\briefempfaengerindex{Schnitzler, Arthur@\textsc{Schnitzler, Arthur}!zzzHofmannsthal, Hugo von@\emph{von Hugo von Hofmannsthal}!1897-10-031@{3. 10. [1897]}|(be}
\toendnotes[C]{\smallbreak\pagebreak[2]}
\correspDesc{Versand  durch Hugo von Hofmannsthal am 3. 10. [1897] in Hinterbrühl
\newline{}Erhalt  durch Arthur Schnitzler im Zeitraum [4. 10. 1897
                  – 8. 10. 1897?] in Wien}\toendnotes[C]{\smallbreak}
\Standort{CUL, Schnitzler, B 43.}
\physDesc{Brief, 1 Blatt, 3 Seiten, 689 Zeichen
\newline{}Handschrift: schwarze Tinte, deutsche Kurrent
\newline{}Schnitzler: mit Bleistift die Jahreszahl ergänzt: »97« 
\newline{}Ordnung: 1) mit Bleistift von unbekannter Hand nummeriert: »\strikeout{103}«  2) mit Bleistift von unbekannter Hand nummeriert: »96«}
\buchAbdrucke{\weitereDrucke{Hugo von Hofmannsthal, Arthur Schnitzler: \emph{Briefwechsel}. Herausgegeben von Therese Nickl und Heinrich Schnitzler. Frankfurt am Main: \emph{S. Fischer} 1964, S. 95–96.} }\toendnotes[C]{\smallbreak}
\pstart
           \raggedleft{}{\pb}Hinterbrühl\oindex{Hinterbrühl@\textbf{Hinterbrühl}, \emph{Hauptstadt}|pw}{ }3\textsuperscript{ten} X\textsuperscript{ten}.\pend
           
\pstart{}mein lieber Arthur\pend\vspace{0.5em}
\pstart
           Ihr Geſicht iſt mir \label{K_L00728-1v}\edtext{neulich}{\lemma{\textnormal{\emph{neulich}}}\Cendnote{\textnormal{Am 26. 9. 1897, 28. 9. 1897 und
               30. 9. 1897 hatte Schnitzler das Gastspiel von Ermete
                     Zacconi\pwindex{Zacconi, Ermete 14.\,9.\,1857 Montecchio Emilia – 14.\,10.\,1948 Viareggio@\textsc{Zacconi, Ermete} (14.\,9.\,1857 Montecchio Emilia – 14.\,10.\,1948 Viareggio), \emph{Regisseur, Schauspieler}|pwk} im Carl-Theater\oindex{Wien@\textbf{Wien}!II., Leopoldstadt@\textbf{II., Leopoldstadt}!Carl-Theater@\textbf{Carl-Theater}, \emph{Theater}|pwk} besucht. Hofmannsthal\pwindex{Hofmannsthal, Hugo von 1.\,2.\,1874 Wien – 15.\,7.\,1929 Rodaun@\textsc{Hofmannsthal, Hugo von} (1.\,2.\,1874 Wien – 15.\,7.\,1929 Rodaun), \emph{Schriftsteller}|pwk} hielt sich am Land auf, konnte
                  aber in die Stadt reisen und war nachweislich in der Vorstellung des \emph{König Lear}\pwindex{\textcolor{red}{\textsuperscript{XXXX indx1}}!König Lear. Trauerspiel in fünf Aufzügen@\strich\emph{König Lear. Trauerspiel in fünf Aufzügen}|pwk} am letzten der genannten Tage
                  (Brief an die Eltern\pwindex{Hofmannsthal, Hugo August von 21.\,12.\,1841 Wien – 8.\,12.\,1915 ebd.@\textsc{Hofmannsthal, Hugo August von} (21.\,12.\,1841 Wien – 8.\,12.\,1915 ebd.), \emph{Bankdirektor}|pwkv}\pwindex{Hofmannsthal, Anna von 27.\,1.\,1849 Wien – 22.\,3.\,1904 Sanatorium Fürth@\textsc{Hofmannsthal, Anna von} (27.\,1.\,1849 Wien – 22.\,3.\,1904 Sanatorium Fürth)|pwkv}).}}}\label{K_L00728-1}{ }ſchon von der Loge aus{ }ſehr ernſt und traurig erſchienen,
               ich bin dann zu Richard\pwindex{Beer-Hofmann, Richard 11.\,7.\,1866 Wien – 26.\,9.\,1945 New York City@\textsc{Beer-Hofmann, Richard} (11.\,7.\,1866 Wien – 26.\,9.\,1945 New York City), \emph{Schriftsteller}|pw} gegangen, er hat mir
               alles \label{K_L00728-2v}\edtext{erzählt}{\lemma{\textnormal{\emph{erzählt}}}\Cendnote{\textnormal{Marie Reinhard\pwindex{Reinhard, Marie 13.\,3.\,1871 Wien – 18.\,3.\,1899 ebd.@\textsc{Reinhard, Marie} (13.\,3.\,1871 Wien – 18.\,3.\,1899 ebd.), \emph{Gesangspädagogin}|pwk} und er betrauerten das
                  am 24. 9. 1897 totgeborene gemeinsame Kind\pwindex{?? [Totgeborener Sohn von Arthur Schnitzler und Marie Reinhard] 24.\,9.\,1897 Endresstraße 68 – 24.\,9.\,1897 ebd.@\textsc{?? [Totgeborener Sohn von Arthur Schnitzler und Marie Reinhard]} (24.\,9.\,1897 Endresstraße 68 – 24.\,9.\,1897 ebd.)|pwkv}.}}}\label{K_L00728-2} und deshalb habe ich Ihnen
               unter den vielen fremden Leuten nur die Hand gegeben und nichts geſagt. Ich weiß
               Ihnen {\pb}nichts tröſtliches zu{ }ſagen
               und ob Ihnen meine Zuneigung und Anhänglichkeit irgend eine wirkliche Freude macht,
               weiß ich nicht, deshalb will ich auch nicht davon{ }ſprechen. Ich hoffe von Herzen,
               daſs Sie bald wieder oder{ }ſchon wieder arbeiten können. Ich werde {\pb}wohl die nächſte Woche nach Wien\oindex{Wien@\textbf{Wien}, \emph{Verwaltungsgebiet}|pw} kommen und hätte Ihnen und dem Richard\pwindex{Beer-Hofmann, Richard 11.\,7.\,1866 Wien – 26.\,9.\,1945 New York City@\textsc{Beer-Hofmann, Richard} (11.\,7.\,1866 Wien – 26.\,9.\,1945 New York City), \emph{Schriftsteller}|pw}, wenn Sie beide aufgelegt{ }ſind, recht
               viel vorzuleſen.\pend
           
\pstart
           Herzlich{\\[\baselineskip]}Ihr{\\[\baselineskip]}\spacefill\mbox{Hugo.}\pend
           \leftskip=0em{}\selectlanguage{ngerman}\endnumbering\briefempfaengerindex{Schnitzler, Arthur@\textsc{Schnitzler, Arthur}!zzzHofmannsthal, Hugo von@\emph{von Hugo von Hofmannsthal}!1897-10-031@{3. 10. [1897]}|)be}\mylabel{L00728h}  \newcommand{\dateiname}{L00728}\newcommand{\titel}{Hugo von Hofmannsthal an Arthur Schnitzler, 3. 10. [1897]}\newcommand{\editorInnen}{Martin Anton Müller und Gerd-Hermann Susen}%% latex-leseansicht-abspann.tex
%% Abspann für die Leseansicht.
%% Der Schalter \ifkorrekturansicht ist bereits durch den Vorspann gesetzt.

%% latex-abspann.tex
%% Gemeinsamer Abspann für Korrekturansicht und Leseansicht.
%% Setzt den Schalter \ifkorrekturansicht voraus (gesetzt in den
%% einbindenden Dateien latex-korrekturansicht-abspann.tex bzw.
%% latex-leseansicht-abspann.tex).
%% ---------------------------------------------------------------

\normalsize

% Das esempio-Environment wird nur in der Leseansicht benötigt
\ifkorrekturansicht\else
\newenvironment{esempio}[3]%
{
    \vspace{1.5ex}
    \rlap{\underline{#1}}
    \par
    \setlength{\parindent}{0cm}
    \nopagebreak
    \leftskip=#2cm
    \rightskip=#3cm
}
{
    \par
}
\fi

\doendnotes{C}
\bigskip
\vfill

\clearpage

\footnotesize

\ifkorrekturansicht
  \lohead{\textsc{register}}
\fi

% theindex-Environment neu definieren ohne reledmac
\makeatletter
\renewenvironment{theindex}{%
  \ifkorrekturansicht
    \section*{\indexname}%
  \else
    \subsubsection*{Index der erwähnten Entitäten}%
  \fi
  \setlength{\parindent}{0pt}%
  \setlength{\parskip}{0pt plus 0.3pt}%
  \let\item\@idxitem
}{%
  \ifkorrekturansicht\clearpage\fi
}
\makeatother

\IfFileExists{\jobname-pw.ind}{\input{\jobname-pw.ind}}{}

% Quellenangabe nur in der Leseansicht
\ifkorrekturansicht\else
% Fallback-Definitionen, falls die .tex-Datei \titel etc. nicht gesetzt hat
\providecommand{\titel}{}
\providecommand{\editorInnen}{}
\providecommand{\dateiname}{\jobname}

\vspace{3cm}

\vfill

\footnotesize
\textsc{Quelle}: \titel. Herausgegeben von {\editorInnen}. In: \emph{Arthur Schnitzler: Briefwechsel mit Autorinnen und Autoren}.
 Digitale Edition, https://schnitzler-briefe.acdh.oeaw.ac.at/{\dateiname}.html (Stand \today)
\fi

\end{document}


