%% latex-leseansicht-vorspann.tex
%% Vorspann für die Leseansicht.
%% Lädt die gemeinsame Datei latex-vorspann.tex mit nicht gesetztem Schalter.

\newif\ifkorrekturansicht
\korrekturansichtfalse

\input{../tex-inputs/latex-vorspann}


         
         \renewcommand{\erwaehntePersonen}{Personen: Hermann Bahr, Paul Marx, Felix Salten, Ottilie Salten, Paul Salten, Olga Schnitzler, Gustav Schwarzkopf}
         \renewcommand{\erwaehnteOrte}{Orte: Eisenerz, Hochschwab, Hotel Hochschwab, Kurhaus Rudolfsbad, Mariazell, Reichenau an der Rax, Weichselboden, Wien, Wildalpen}
         \renewcommand{\erwaehnteWerke}{Werke: Der Ruf des Lebens. Schauspiel in drei Akten, Die Andere, Tagebuch}
               \section[ Arthur Schnitzler an Felix Salten, 20. 7. 1905]{ Arthur Schnitzler an Felix Salten, 20. 7. 1905}\nopagebreak\mylabel{v}\rehead{ }\begin{ledgroupsized}[t]{13cm}\normalsize\beginnumbering\briefempfaengerindex{Salten, Felix@\textsc{Salten, Felix}!zzzSchnitzler, Arthur@\emph{von Arthur Schnitzler}!1905-07-201@{20. 7. 1905}|(be} \toendnotes[C]{\smallbreak\pagebreak[2]} \Standort{Wienbibliothek im Rathaus, ZPH 1681, 2.1.516.}
\physDesc{Brief, 1 Blatt, 4 Seiten, 1240 Zeichen
\newline{}Handschrift: Bleistift, deutsche Kurrent
\newline{}Ordnung: mit Bleistift von unbekannter Hand Nummerierung der Doppelseiten des
                                 Konvoluts: »16«–»17« }\toendnotes[C]{\smallbreak}\pstart
           \raggedleft{}{\pb}\textsc{Reichenau\oindex{Reichenau an der Rax@\textbf{Reichenau an der Rax}|pw}},{\\}20/7 905\pend
           \pstart
           lieber, unſre \label{K_L03000-1v}\edtext{Briefe
               haben ſich gekreuzt}{\lemma{\textnormal{\emph{Briefe … gekreuzt}}}\Cendnote{\textnormal{Der Brief Salten\pwindex{Salten, Felix 06.09.1869 – 08.10.1945@\textsc{Salten, Felix} (06.09.1869 – 08.10.1945), \emph{Schriftsteller, Journalist}|pwk}s ist jener vom 18. 7. 1905, der von Schnitzler\pwindex{Schnitzler, Arthur 15.05.1862 – 21.10.1931@\textsc{Schnitzler, Arthur} (15.05.1862 – 21.10.1931), \emph{Schriftsteller, Mediziner}|pwk} ist nicht erhalten.}}}\label{K_L03000-1h}. Sie
               wiſſen alſo ſchon, daſs ich Sie bitten werde, unſre Tour, \textsc{resp.} Ihr Hieherkommen um etliche Tage zu verſchieben. Heute fahren wir ins Hochſchwab\oindex{Hochschwab@\textbf{Hochschwab}|pw}gebiet, denken Samſtag wieder da zu
               ſein (ich und Paul Marx\pwindex{Marx, Paul 21.07.1879 – 1956-10-30@\textsc{Marx, Paul} (21.07.1879 – 1956-10-30), \emph{Regisseur, Schauspieler}|pw}). Ob \label{K_L03000-2v}\edtext{Gustav Schwarzkopf\pwindex{Schwarzkopf, Gustav 07.11.1853 – 13.11.1939@\textsc{Schwarzkopf, Gustav} (07.11.1853 – 13.11.1939), \emph{Schriftsteller}|pw}}{\lemma{\textnormal{\emph{Gustav Schwarzkopf}}}\Cendnote{\textnormal{Gustav Schwarzkopf\pwindex{Schwarzkopf, Gustav 07.11.1853 – 13.11.1939@\textsc{Schwarzkopf, Gustav} (07.11.1853 – 13.11.1939), \emph{Schriftsteller}|pwk} kam am Montag, dem 24. 7. 1905, in Reichenau an der Rax\oindex{Reichenau an der Rax@\textbf{Reichenau an der Rax}|pwk} an. Im \emph{Tagebuch}\pwindex{\textcolor{red}{\textsuperscript{XXXX1 indx}}!Tagebuch1981 – 2000@\strich\emph{Tagebuch} {[}Hrsg., 1981 – 2000{]}|pwk} wird er in den darauf folgenden Tagen nicht
                  erwähnt. An der hier verhandelten Reise nach Mariazell\oindex{Mariazell@\textbf{Mariazell}|pwk} nahm er nicht teil.}}}\label{K_L03000-2h} iſt noch nicht ausgemacht; das wäre
               etwa Montag auf 2 Tage denk ich. \label{K_L03000-3v}\edtext{Mitte {\pb}oder Ende nächſter Woche}{\lemma{\textnormal{\emph{Mitte … Woche}}}\Cendnote{\textnormal{Arthur\pwindex{Schnitzler, Arthur 15.05.1862 – 21.10.1931@\textsc{Schnitzler, Arthur} (15.05.1862 – 21.10.1931), \emph{Schriftsteller, Mediziner}|pwk} und Olga Schnitzler\pwindex{Schnitzler, Olga 17.01.1882 – 13.01.1970@\textsc{Schnitzler, Olga} (17.01.1882 – 13.01.1970), \emph{Schauspielerin, Sängerin}|pwk} blieben bis zum 29. 7. 1905 in Reichenau an der Rax\oindex{Reichenau an der Rax@\textbf{Reichenau an der Rax}|pwk} und kehrten dann nach Wien\oindex{Wien@\textbf{Wien}|pwk} zurück. Salten\pwindex{Salten, Felix 06.09.1869 – 08.10.1945@\textsc{Salten, Felix} (06.09.1869 – 08.10.1945), \emph{Schriftsteller, Journalist}|pwk} kam am 26. 7. 1905 in Reichenau an der Rax\oindex{Reichenau an der Rax@\textbf{Reichenau an der Rax}|pwk} an und blieb bis zumindest 29. 7. 1905.  trafen
                  sich die vier\pwindex{Schnitzler, Arthur 15.05.1862 – 21.10.1931@\textsc{Schnitzler, Arthur} (15.05.1862 – 21.10.1931), \emph{Schriftsteller, Mediziner}|pwkv}\pwindex{Schnitzler, Olga 17.01.1882 – 13.01.1970@\textsc{Schnitzler, Olga} (17.01.1882 – 13.01.1970), \emph{Schauspielerin, Sängerin}|pwkv}\pwindex{Salten, Ottilie 07.03.1868 – 22.06.1942@\textsc{Salten, Ottilie} (07.03.1868 – 22.06.1942), \emph{Schauspielerin}|pwkv}\pwindex{Salten, Felix 06.09.1869 – 08.10.1945@\textsc{Salten, Felix} (06.09.1869 – 08.10.1945), \emph{Schriftsteller, Journalist}|pwkv} noch.}}}\label{K_L03000-3h} ſtänden wir
               dann gern un\textcolor{gray}{d} auf möglichſt lang zur Verfügung. Vielleicht auch,
               daſs unſre Wegfahrt mit Ihnen ſchon ein Verlaſſen Reichenau\oindex{Reichenau an der Rax@\textbf{Reichenau an der Rax}|pw}s zu bedeuten hätte (der Ort\oindex{Reichenau an der Rax@\textbf{Reichenau an der Rax}|pwv} bleibt wundervoll, aber das \textsc{Curhaus\oindex{Kurhaus Rudolfsbad@\textbf{Kurhaus Rudolfsbad}|pw}}{ }\label{K_L03000-4v}\edtext{verbeiſelt}{\lemma{\textnormal{\emph{verbeiſelt}}}\Cendnote{\textnormal{Beisl, österreichisch: Kneipe, Wirtshaus. Vgl. A. S.: \emph{Tagebuch}, 28. 7. 1905.}}}\label{K_L03000-4h} ſich i{\geminationm}er mehr) und daſs wir uns da{\geminationn} noch auf einige Tage wo anders anſiedeln. Das berühmte
                  {\pb}\label{K_L03000-5v}\edtext{\textsc{Fölzhotel\oindex{Hotel Hochschwab@\textbf{Hotel Hochschwab}|pw}} hoff ich noch heute}{\lemma{\textnormal{\emph{Fölzhotel … heute}}}\Cendnote{\textnormal{siehe A. S.: \emph{Tagebuch}, 20. 7. 1905}}}\label{K_L03000-5h} zu betreten. Eventuell gingen \textsc{resp} führen wir von \textsc{Mariazell\oindex{Mariazell@\textbf{Mariazell}|pw}}, Ihren Intentionen entſprechend, über \textsc{Wilda\textcolor{gray}{l}pe\oindex{Wildalpen@\textbf{Wildalpen}|pw}}, \textsc{Weichsel\oindex{Weichselboden@\textbf{Weichselboden}|pw}boden } nach Eisenerz\oindex{Eisenerz@\textbf{Eisenerz}|pw}. Das weſentliche bleibt, daſs man ein paar Sommertage
               wieder einmal zuſa{\geminationm}en verbringt. Ich hoffe bei meiner
               Rückkehr einige Zeilen von Ihnen zu finden. Was hat denn {\pb}Ihrem Paul\pwindex{Salten, Paul 11.08.1903 – 08.05.1937@\textsc{Salten, Paul} (11.08.1903 – 08.05.1937), \emph{Filmcutter}|pw} gefehlt? Wieder ſo eine Kehlkopfſache?\pend
           \pstart
           Wir grüßen Sie alle herzlich {\\[\baselineskip]}Ihr {\\[\baselineskip]}\spacefill\mbox{A.}\pend
           \leftskip=0em{}\pstart
           \noindent{}Wohin iſt das Bahr\pwindex{Bahr, Hermann 19.07.1863 – 15.01.1934@\textsc{Bahr, Hermann} (19.07.1863 – 15.01.1934), \emph{Schriftsteller, Kritiker}|pw}-Stück\pwindex{Bahr, Hermann 19.07.1863 – 15.01.1934@\textsc{Bahr, Hermann} (19.07.1863 – 15.01.1934), \emph{Schriftsteller, Kritiker}!Andere1905-11-04@\strich\emph{Die Andere} {[}1905-11-04{]}|pwv} zu ſenden? – Ich \label{K_L03000-6v}\edtext{leſe es erſt nach meiner Rückkehr}{\lemma{\textnormal{\emph{leſe … Rückkehr}}}\Cendnote{\textnormal{Schnitzler\pwindex{Schnitzler, Arthur 15.05.1862 – 21.10.1931@\textsc{Schnitzler, Arthur} (15.05.1862 – 21.10.1931), \emph{Schriftsteller, Mediziner}|pwk} las \emph{Die Andere}\pwindex{Bahr, Hermann 19.07.1863 – 15.01.1934@\textsc{Bahr, Hermann} (19.07.1863 – 15.01.1934), \emph{Schriftsteller, Kritiker}!Andere1905-11-04@\strich\emph{Die Andere} {[}1905-11-04{]}|pwk} am 26. 7. 1905. Siehe auch Arthur Schnitzler an Hermann Bahr, 30. 7. 1905.}}}\label{K_L03000-6h}{ }\introOben{}(Samstag)\introOben{}, da ich, ſelbſt
                     \label{K_L03000-7v}\edtext{dramatiſch verſunken}{\lemma{\textnormal{\emph{dramatiſch verſunken}}}\Cendnote{\textnormal{Schnitzler\pwindex{Schnitzler, Arthur 15.05.1862 – 21.10.1931@\textsc{Schnitzler, Arthur} (15.05.1862 – 21.10.1931), \emph{Schriftsteller, Mediziner}|pwk} arbeitete an \emph{Der Ruf des Lebens}\pwindex{Schnitzler, Arthur 15.05.1862 – 21.10.1931@\textsc{Schnitzler, Arthur} (15.05.1862 – 21.10.1931), \emph{Schriftsteller, Mediziner}!Ruf des Lebens. Schauspiel in drei Akten1906-02-20@\strich\emph{Der Ruf des Lebens. Schauspiel in drei Akten} {[}1906-02-20{]}|pwk}.}}}\label{K_L03000-7h}, in nichts andres der Art zu
                  ſteigen mich getraue.\pend
           
         
         \endnumbering\mylabel{h}\end{ledgroupsized}  \newcommand{\dateiname}{L03000}\newcommand{\titel}{Arthur Schnitzler an Felix Salten, 20. 7. 1905}\newcommand{\editorInnen}{Martin Anton Müller und Laura Untner}%% latex-leseansicht-abspann.tex
%% Abspann für die Leseansicht.
%% Der Schalter \ifkorrekturansicht ist bereits durch den Vorspann gesetzt.

%% latex-abspann.tex
%% Gemeinsamer Abspann für Korrekturansicht und Leseansicht.
%% Setzt den Schalter \ifkorrekturansicht voraus (gesetzt in den
%% einbindenden Dateien latex-korrekturansicht-abspann.tex bzw.
%% latex-leseansicht-abspann.tex).
%% ---------------------------------------------------------------

\normalsize

% Das esempio-Environment wird nur in der Leseansicht benötigt
\ifkorrekturansicht\else
\newenvironment{esempio}[3]%
{
    \vspace{1.5ex}
    \rlap{\underline{#1}}
    \par
    \setlength{\parindent}{0cm}
    \nopagebreak
    \leftskip=#2cm
    \rightskip=#3cm
}
{
    \par
}
\fi

\doendnotes{C}
\bigskip
\vfill

\clearpage

\footnotesize

\ifkorrekturansicht
  \lohead{\textsc{register}}
\fi

% theindex-Environment neu definieren ohne reledmac
\makeatletter
\renewenvironment{theindex}{%
  \ifkorrekturansicht
    \section*{\indexname}%
  \else
    \subsubsection*{Index der erwähnten Entitäten}%
  \fi
  \setlength{\parindent}{0pt}%
  \setlength{\parskip}{0pt plus 0.3pt}%
  \let\item\@idxitem
}{%
  \ifkorrekturansicht\clearpage\fi
}
\makeatother

\IfFileExists{\jobname-pw.ind}{\input{\jobname-pw.ind}}{}

% Quellenangabe nur in der Leseansicht
\ifkorrekturansicht\else
% Fallback-Definitionen, falls die .tex-Datei \titel etc. nicht gesetzt hat
\providecommand{\titel}{}
\providecommand{\editorInnen}{}
\providecommand{\dateiname}{\jobname}

\vspace{3cm}

\vfill

\footnotesize
\textsc{Quelle}: \titel. Herausgegeben von {\editorInnen}. In: \emph{Arthur Schnitzler: Briefwechsel mit Autorinnen und Autoren}.
 Digitale Edition, https://schnitzler-briefe.acdh.oeaw.ac.at/{\dateiname}.html (Stand \today)
\fi

\end{document}


      