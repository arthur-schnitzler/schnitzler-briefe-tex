%% latex-leseansicht-vorspann.tex
%% Vorspann für die Leseansicht.
%% Lädt die gemeinsame Datei latex-vorspann.tex mit nicht gesetztem Schalter.

\newif\ifkorrekturansicht
\korrekturansichtfalse

\input{../tex-inputs/latex-vorspann}

\begin{center}
            \textcolor{red}{ENTWURF, NICHT FERTIG KORRIGIERT}
                      \end{center}
            
         
         \renewcommand{\erwaehntePersonen}{Personen: Hermann Bahr, Paul Marx, Felix Salten, Paul Salten, Gustav Schwarzkopf}
         \renewcommand{\erwaehnteOrte}{Orte: Eisenerz, Hochschwab, Hotel Hochschwab, Kurhaus Rudolfsbad, Mariazell, Reichenau an der Rax, Weichselboden, Wien, Wildalpen}
         \renewcommand{\erwaehnteWerke}{Werke: Die Andere}
               \section[Arthur Schnitzler an Felix Salten, 20. 7. 1905]{ Arthur Schnitzler an Felix Salten, 20. 7. 1905}\nopagebreak\mylabel{v}\rehead{ }\begin{ledgroupsized}[t]{13cm}\normalsize\beginnumbering \toendnotes[C]{\smallbreak\pagebreak[2]} \Standort{Wienbibliothek im Rathaus, ZPH 1681, 2.1.516.}
\physDesc{Brief, 1 Blatt, 4 Seiten
\newline{}Handschrift: Bleistift, deutsche Kurrent\newline{}Ordnung: mit Bleistift von unbekannter Hand Nummerierung der ungeraden Seiten: »16« }\toendnotes[C]{\smallbreak}\pstart
           \raggedleft{}{\pb}\textsc{Reichenau\oindex{Reichenau an der Rax@\textbf{Reichenau an der Rax}|pw}}, 20/7 905\pend
           \pstart
           lieber, unſre Briefe haben ſich gekreuzt. Sie wiſſen alſo ſchon,
               daſs ich Sie bitten werde, unſre Tour, \textsc{resp.} Ihr
               Hieherkommen um etliche Tage zu verſchieben. Heute fahren wir ins Hochſchwabgebiet\oindex{Hochschwab@\textbf{Hochschwab}|pw}, denken Samſtag wieder da zu
               ſein (ich und \textsc{Paul Marx\pwindex{Marx, Paul 21.07.1879 – 1956-10-30@\textsc{Marx, Paul} (21.07.1879 – 1956-10-30), \emph{Regisseur, Schauspieler}|pw}}). Ob \textsc{Gustav Schwarzkopf\pwindex{Schwarzkopf, Gustav 07.11.1853 – 13.11.1939@\textsc{Schwarzkopf, Gustav} (07.11.1853 – 13.11.1939), \emph{Schriftsteller}|pw}} iſt noch nicht ausgemacht; das wäre etwa Montag auf 2 Tage denk ich. Mitte {\pb}oder Ende nächſter Woche ſtänden wir dann
               gern und auf möglichſt lang zur Verfügung. Vielleicht auch, daſs unſre Wegfahrt mit
               Ihnen ſchon ein Verlaſſen Reichenaus\oindex{Reichenau an der Rax@\textbf{Reichenau an der Rax}|pw} zu bedeuten
               hätte (der Ort bleibt wundervoll, aber das \textsc{Curhaus\oindex{Kurhaus Rudolfsbad@\textbf{Kurhaus Rudolfsbad}|pw}} verbeiſelt ſich i{\geminationm}er mehr) und daſs wir uns da{\geminationn} noch auf einige Tage wo anders anſiedeln. Das berühmte
                  {\pb}\textsc{Fölzhotel\oindex{Hotel Hochschwab@\textbf{Hotel Hochschwab}|pw}} hoff ich noch heute zu betreten. Eventuell gingen \textsc{resp.} führen wir von \textsc{Mariazell\oindex{Mariazell@\textbf{Mariazell}|pw}}, Ihren Intentionen entſprechend, über\textsc{Wilda\textcolor{gray}{l}pe\oindex{Wildalpen@\textbf{Wildalpen}|pw}}, \textsc{Weichselboden\oindex{Weichselboden@\textbf{Weichselboden}|pw}} nach Eisenerz\oindex{Eisenerz@\textbf{Eisenerz}|pw}. Das weſentliche bleibt, daſs
               man ein paar Sommertage wieder einmal zuſa{\geminationm}en verbringt.
               Ich hoffe bei meiner Rückkehr einige Zeilen von Ihnen zu finden. Was hat denn {\pb}Ihrem Paul\pwindex{Salten, Paul 11.08.1903 – 08.05.1937@\textsc{Salten, Paul} (11.08.1903 – 08.05.1937), \emph{Filmcutter}|pw} gefehlt? Wieder ſo eine Kehlkopfſache? \pend
           \pstart
           Wir grüßen Sie alle herzlich {\\[\baselineskip]}Ihr {\\[\baselineskip]}\spacefill\mbox{A.}\pend
           \leftskip=0em{}\pstart
           \noindent{} Wohin iſt das Bahr\pwindex{Bahr, Hermann 19.07.1863 – 15.01.1934@\textsc{Bahr, Hermann} (19.07.1863 – 15.01.1934), \emph{Schriftsteller, Kritiker}|pw}-Stück\pwindex{Bahr, Hermann 19.07.1863 – 15.01.1934@\textsc{Bahr, Hermann} (19.07.1863 – 15.01.1934), \emph{Schriftsteller, Kritiker}!Andere1905-11-04@\strich\emph{Die Andere} {[}1905-11-04{]}|pwv} zu ſenden? – Ich leſe es erſt nach
                  meiner Rückkehr \introOben{}(Samstag)\introOben{}, da ich, ſelbſt dramatiſch verſunken, in
                  nichts andres der Art zu ſteigen mich getraue. \pend
           
         
         \endnumbering\mylabel{h}\end{ledgroupsized}\begin{anhang}\end{anhang}\newcommand{\dateiname}{L03000}\newcommand{\titel}{Arthur Schnitzler an Felix Salten, 20. 7. 1905}\newcommand{\editorInnen}{Martin Anton Müller und Laura Untner}%% latex-leseansicht-abspann.tex
%% Abspann für die Leseansicht.
%% Der Schalter \ifkorrekturansicht ist bereits durch den Vorspann gesetzt.

%% latex-abspann.tex
%% Gemeinsamer Abspann für Korrekturansicht und Leseansicht.
%% Setzt den Schalter \ifkorrekturansicht voraus (gesetzt in den
%% einbindenden Dateien latex-korrekturansicht-abspann.tex bzw.
%% latex-leseansicht-abspann.tex).
%% ---------------------------------------------------------------

\normalsize

% Das esempio-Environment wird nur in der Leseansicht benötigt
\ifkorrekturansicht\else
\newenvironment{esempio}[3]%
{
    \vspace{1.5ex}
    \rlap{\underline{#1}}
    \par
    \setlength{\parindent}{0cm}
    \nopagebreak
    \leftskip=#2cm
    \rightskip=#3cm
}
{
    \par
}
\fi

\doendnotes{C}
\bigskip
\vfill

\clearpage

\footnotesize

\ifkorrekturansicht
  \lohead{\textsc{register}}
\fi

% theindex-Environment neu definieren ohne reledmac
\makeatletter
\renewenvironment{theindex}{%
  \ifkorrekturansicht
    \section*{\indexname}%
  \else
    \subsubsection*{Index der erwähnten Entitäten}%
  \fi
  \setlength{\parindent}{0pt}%
  \setlength{\parskip}{0pt plus 0.3pt}%
  \let\item\@idxitem
}{%
  \ifkorrekturansicht\clearpage\fi
}
\makeatother

\IfFileExists{\jobname-pw.ind}{\input{\jobname-pw.ind}}{}

% Quellenangabe nur in der Leseansicht
\ifkorrekturansicht\else
% Fallback-Definitionen, falls die .tex-Datei \titel etc. nicht gesetzt hat
\providecommand{\titel}{}
\providecommand{\editorInnen}{}
\providecommand{\dateiname}{\jobname}

\vspace{3cm}

\vfill

\footnotesize
\textsc{Quelle}: \titel. Herausgegeben von {\editorInnen}. In: \emph{Arthur Schnitzler: Briefwechsel mit Autorinnen und Autoren}.
 Digitale Edition, https://schnitzler-briefe.acdh.oeaw.ac.at/{\dateiname}.html (Stand \today)
\fi

\end{document}


      