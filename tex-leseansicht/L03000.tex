%% latex-leseansicht-vorspann.tex
%% Vorspann für die Leseansicht.
%% Lädt die gemeinsame Datei latex-vorspann.tex mit nicht gesetztem Schalter.

\newif\ifkorrekturansicht
\korrekturansichtfalse

\input{../tex-inputs/latex-vorspann}


\section[ Arthur Schnitzler an Felix Salten, 20. 7. 1905]{L03000 Arthur Schnitzler an Felix Salten,  20. 7. 1905}
\nopagebreak\mylabel{L03000v}
\rehead{ }\normalsize\beginnumbering\briefempfaengerindex{Salten, Felix@\textsc{Salten, Felix}!zzzSchnitzler, Arthur@\emph{von Arthur Schnitzler}!1905-07-201@{20. 7. 1905}|(be}
\toendnotes[C]{\smallbreak\pagebreak[2]}
\correspDesc{Versand  durch Arthur Schnitzler am 20. 7. 1905 in Reichenau an der Rax
\newline{}Erhalt  durch Felix Salten im Zeitraum [21. 7. 1905
                  – 25. 7. 1905?] in Wien}\toendnotes[C]{\smallbreak}
\Standort{Wienbibliothek im Rathaus, ZPH 1681, 2.1.516.}
\physDesc{Brief, 1 Blatt, 4 Seiten, 1240 Zeichen
\newline{}Handschrift: Bleistift, deutsche Kurrent
\newline{}Ordnung: mit Bleistift von unbekannter Hand Nummerierung der Doppelseiten des
                                 Konvoluts: »16«–»17« }\toendnotes[C]{\smallbreak}
\pstart
           \raggedleft{}{\pb}\textsc{Reichenau\oindex{Reichenau an der Rax@\textbf{Reichenau an der Rax}, \emph{Verwaltungsgebiet}|pw}},{\\}20/7 905\pend
           \vspace{0.5em}
\pstart
           lieber, unſre \label{K_L03000-1v}\edtext{Briefe
               haben{ }ſich gekreuzt}{\lemma{\textnormal{\emph{Briefe … gekreuzt}}}\Cendnote{\textnormal{Der Brief Saltens\pwindex{Salten, Felix 6.\,9.\,1869 Budapest – 8.\,10.\,1945 Zürich@\textsc{Salten, Felix} (6.\,9.\,1869 Budapest – 8.\,10.\,1945 Zürich), \emph{Schriftsteller, Journalist, Chefredakteur}|pwk} ist jener vom XXXX Auszeichnungsfehler: Dokument L03412 nicht gefunden, der von Schnitzler ist nicht erhalten.}}}\label{K_L03000-1}. Sie
               wiſſen alſo{ }ſchon, daſs ich Sie bitten werde, unſre Tour, \textsc{resp.} Ihr Hieherkommen um etliche Tage zu verſchieben. Heute fahren wir ins Hochſchwab\oindex{Hochschwab@\textbf{Hochschwab}, \emph{Gebirge}|pw}gebiet, denken Samſtag wieder da zu{ }ſein (ich und Paul Marx\pwindex{Marx, Paul 21.\,7.\,1879 Wien – 30.\,10.\,1956 ebd.@\textsc{Marx, Paul} (21.\,7.\,1879 Wien – 30.\,10.\,1956 ebd.), \emph{Regisseur, Schauspieler}|pw}). Ob \label{K_L03000-2v}\edtext{Gustav Schwarzkopf\pwindex{Schwarzkopf, Gustav 7.\,11.\,1853 Wien – 13.\,11.\,1939 ebd.@\textsc{Schwarzkopf, Gustav} (7.\,11.\,1853 Wien – 13.\,11.\,1939 ebd.), \emph{Schriftsteller}|pw}}{\lemma{\textnormal{\emph{Gustav Schwarzkopf}}}\Cendnote{\textnormal{Gustav Schwarzkopf\pwindex{Schwarzkopf, Gustav 7.\,11.\,1853 Wien – 13.\,11.\,1939 ebd.@\textsc{Schwarzkopf, Gustav} (7.\,11.\,1853 Wien – 13.\,11.\,1939 ebd.), \emph{Schriftsteller}|pwk} kam am Montag, dem 24. 7. 1905 in Reichenau an der Rax\oindex{Reichenau an der Rax@\textbf{Reichenau an der Rax}, \emph{Verwaltungsgebiet}|pwk} an. Im \emph{Tagebuch}\pwindex{Schnitzler, Arthur 15.\,5.\,1862 Wien – 21.\,10.\,1931 ebd.@\textsc{Schnitzler, Arthur} (15.\,5.\,1862 Wien – 21.\,10.\,1931 ebd.), \emph{Schriftsteller, Mediziner}!Tagebuch@\strich\emph{Tagebuch}|pwk} wird er in den darauf folgenden Tagen nicht
                  erwähnt. An der hier verhandelten Reise nach Mariazell\oindex{Mariazell@\textbf{Mariazell}, \emph{Hauptstadt}|pwk} nahm er nicht teil.}}}\label{K_L03000-2} iſt noch nicht ausgemacht; das wäre
               etwa Montag auf 2 Tage denk ich. \label{K_L03000-3v}\edtext{Mitte {\pb}oder Ende nächſter Woche}{\lemma{\textnormal{\emph{Mitte … Woche}}}\Cendnote{\textnormal{Arthur und Olga Schnitzler\pwindex{Schnitzler, Olga 17.\,1.\,1882 Wien – 13.\,1.\,1970 Lugano@\textsc{Schnitzler, Olga} (17.\,1.\,1882 Wien – 13.\,1.\,1970 Lugano), \emph{Schauspielerin, Sängerin}|pwk} blieben bis zum 29. 7. 1905 in Reichenau an der Rax\oindex{Reichenau an der Rax@\textbf{Reichenau an der Rax}, \emph{Verwaltungsgebiet}|pwk} und kehrten dann nach Wien\oindex{Wien@\textbf{Wien}, \emph{Verwaltungsgebiet}|pwk} zurück. Salten\pwindex{Salten, Felix 6.\,9.\,1869 Budapest – 8.\,10.\,1945 Zürich@\textsc{Salten, Felix} (6.\,9.\,1869 Budapest – 8.\,10.\,1945 Zürich), \emph{Schriftsteller, Journalist, Chefredakteur}|pwk} kam am 26. 7. 1905 in Reichenau an der Rax\oindex{Reichenau an der Rax@\textbf{Reichenau an der Rax}, \emph{Verwaltungsgebiet}|pwk} an und blieb bis zumindest
                     29. 7. 1905.}}}\label{K_L03000-3}{ }ſtänden wir
               dann gern un\textcolor{gray}{d} auf möglichſt lang zur Verfügung. Vielleicht auch,
               daſs unſre Wegfahrt mit Ihnen{ }ſchon ein Verlaſſen Reichenaus\oindex{Reichenau an der Rax@\textbf{Reichenau an der Rax}, \emph{Verwaltungsgebiet}|pw} zu bedeuten hätte (der Ort\oindex{Reichenau an der Rax@\textbf{Reichenau an der Rax}, \emph{Verwaltungsgebiet}|pwv} bleibt wundervoll, aber das \textsc{Curhaus\oindex{Kurhaus Rudolfsbad@\textbf{Kurhaus Rudolfsbad}, \emph{Sanatorium}|pw}}{ }\label{K_L03000-4v}\edtext{verbeiſelt}{\lemma{\textnormal{\emph{verbeiselt}}}\Cendnote{\textnormal{Beisl, österreichisch: Kneipe, Wirtshaus. Vgl. A. S.: \emph{Tagebuch}, 28. 7. 1905.}}}\label{K_L03000-4}{ }ſich i{\geminationm}er mehr) und daſs wir uns da{\geminationn} noch auf einige Tage wo anders anſiedeln. Das berühmte
                  {\pb}\label{K_L03000-5v}\edtext{\textsc{Fölzhotel\oindex{Hotel Hochschwab@\textbf{Hotel Hochschwab}, \emph{Hotel}|pw}} hoff ich noch heute}{\lemma{\textnormal{\emph{Fölzhotel … heute}}}\Cendnote{\textnormal{Siehe A. S.: \emph{Tagebuch}, 20. 7. 1905. }}}\label{K_L03000-5} zu
               betreten. Eventuell gingen \textsc{resp} führen wir von \textsc{Mariazell\oindex{Mariazell@\textbf{Mariazell}, \emph{Hauptstadt}|pw}}, Ihren Intentionen entſprechend, über \textsc{Wilda\textcolor{gray}{l}pe\oindex{Wildalpen@\textbf{Wildalpen}, \emph{Hauptstadt}|pw}}, \textsc{Weichsel\oindex{Weichselboden@\textbf{Weichselboden}|pw}boden} nach Eisenerz\oindex{Eisenerz@\textbf{Eisenerz}, \emph{Hauptstadt}|pw}. Das weſentliche bleibt, daſs man ein paar Sommertage
               wieder einmal zuſa{\geminationm}en verbringt. Ich hoffe bei meiner
               Rückkehr einige Zeilen von Ihnen zu finden. Was hat denn {\pb}Ihrem Paul\pwindex{Salten, Paul 11.\,8.\,1903 Wien – 8.\,5.\,1937 ebd.@\textsc{Salten, Paul} (11.\,8.\,1903 Wien – 8.\,5.\,1937 ebd.), \emph{Filmcutter}|pw} gefehlt? Wieder{ }ſo eine Kehlkopfſache?\pend
           
\pstart
           Wir grüßen Sie alle herzlich {\\[\baselineskip]}Ihr {\\[\baselineskip]}\spacefill\mbox{A.}\pend
           \leftskip=0em{}
\pstart
           \noindent{}Wohin iſt das Bahr\pwindex{Bahr, Hermann 19.\,7.\,1863 Linz – 15.\,1.\,1934 München@\textsc{Bahr, Hermann} (19.\,7.\,1863 Linz – 15.\,1.\,1934 München), \emph{Schriftsteller, Kritiker}|pw}-Stück\pwindex{Bahr, Hermann 19.\,7.\,1863 Linz – 15.\,1.\,1934 München@\textsc{Bahr, Hermann} (19.\,7.\,1863 Linz – 15.\,1.\,1934 München), \emph{Schriftsteller, Kritiker}!Andere@\strich\emph{Die Andere}|pwv} zu{ }ſenden? – Ich \label{K_L03000-6v}\edtext{leſe es erſt nach meiner Rückkehr}{\lemma{\textnormal{\emph{lese … Rückkehr}}}\Cendnote{\textnormal{Schnitzler las \emph{Die Andere}\pwindex{Bahr, Hermann 19.\,7.\,1863 Linz – 15.\,1.\,1934 München@\textsc{Bahr, Hermann} (19.\,7.\,1863 Linz – 15.\,1.\,1934 München), \emph{Schriftsteller, Kritiker}!Andere@\strich\emph{Die Andere}|pwk} am 26. 7. 1905. Siehe auch XXXX Auszeichnungsfehler: Dokument L01534 nicht gefunden.}}}\label{K_L03000-6}{ }\introOben{}(Samstag)\introOben{}, da ich,{ }ſelbſt
                     \label{K_L03000-7v}\edtext{dramatiſch verſunken}{\lemma{\textnormal{\emph{dramatisch versunken}}}\Cendnote{\textnormal{Schnitzler arbeitete an \emph{Der Ruf des Lebens}\pwindex{Schnitzler, Arthur 15.\,5.\,1862 Wien – 21.\,10.\,1931 ebd.@\textsc{Schnitzler, Arthur} (15.\,5.\,1862 Wien – 21.\,10.\,1931 ebd.), \emph{Schriftsteller, Mediziner}!Ruf des Lebens. Schauspiel in drei Akten@\strich\emph{Der Ruf des Lebens. Schauspiel in drei Akten}|pwk}.}}}\label{K_L03000-7}, in nichts andres der Art zu{ }ſteigen mich getraue.\pend
           \selectlanguage{ngerman}\endnumbering\briefempfaengerindex{Salten, Felix@\textsc{Salten, Felix}!zzzSchnitzler, Arthur@\emph{von Arthur Schnitzler}!1905-07-201@{20. 7. 1905}|)be}\mylabel{L03000h}  \newcommand{\dateiname}{L03000}\newcommand{\titel}{Arthur Schnitzler an Felix Salten, 20. 7. 1905}\newcommand{\editorInnen}{Martin Anton Müller und Laura Untner}%% latex-leseansicht-abspann.tex
%% Abspann für die Leseansicht.
%% Der Schalter \ifkorrekturansicht ist bereits durch den Vorspann gesetzt.

%% latex-abspann.tex
%% Gemeinsamer Abspann für Korrekturansicht und Leseansicht.
%% Setzt den Schalter \ifkorrekturansicht voraus (gesetzt in den
%% einbindenden Dateien latex-korrekturansicht-abspann.tex bzw.
%% latex-leseansicht-abspann.tex).
%% ---------------------------------------------------------------

\normalsize

% Das esempio-Environment wird nur in der Leseansicht benötigt
\ifkorrekturansicht\else
\newenvironment{esempio}[3]%
{
    \vspace{1.5ex}
    \rlap{\underline{#1}}
    \par
    \setlength{\parindent}{0cm}
    \nopagebreak
    \leftskip=#2cm
    \rightskip=#3cm
}
{
    \par
}
\fi

\doendnotes{C}
\bigskip
\vfill

\clearpage

\footnotesize

\ifkorrekturansicht
  \lohead{\textsc{register}}
\fi

% theindex-Environment neu definieren ohne reledmac
\makeatletter
\renewenvironment{theindex}{%
  \ifkorrekturansicht
    \section*{\indexname}%
  \else
    \subsubsection*{Index der erwähnten Entitäten}%
  \fi
  \setlength{\parindent}{0pt}%
  \setlength{\parskip}{0pt plus 0.3pt}%
  \let\item\@idxitem
}{%
  \ifkorrekturansicht\clearpage\fi
}
\makeatother

\IfFileExists{\jobname-pw.ind}{\input{\jobname-pw.ind}}{}

% Quellenangabe nur in der Leseansicht
\ifkorrekturansicht\else
% Fallback-Definitionen, falls die .tex-Datei \titel etc. nicht gesetzt hat
\providecommand{\titel}{}
\providecommand{\editorInnen}{}
\providecommand{\dateiname}{\jobname}

\vspace{3cm}

\vfill

\footnotesize
\textsc{Quelle}: \titel. Herausgegeben von {\editorInnen}. In: \emph{Arthur Schnitzler: Briefwechsel mit Autorinnen und Autoren}.
 Digitale Edition, https://schnitzler-briefe.acdh.oeaw.ac.at/{\dateiname}.html (Stand \today)
\fi

\end{document}


