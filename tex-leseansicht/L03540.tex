%% latex-korrekturansicht-vorspann.tex
%% Vorspann für die Korrekturansicht.
%% Lädt die gemeinsame Datei latex-vorspann.tex mit gesetztem Schalter.

\newif\ifkorrekturansicht
\korrekturansichttrue

\input{../tex-inputs/latex-vorspann}


\section[ Eva Marie Goldmann an Arthur Schnitzler, 21. 9. 1911]{L03540 Eva Marie Goldmann an Arthur Schnitzler, 21. 9. 1911}
\nopagebreak\mylabel{L03540v}
\rehead{ }\normalsize\beginnumbering\briefempfaengerindex{Schnitzler, Arthur@\textsc{Schnitzler, Arthur}!zzzGoldmann, Eva Marie@\emph{von Eva Marie Goldmann}!1911-09-211@{21. 9. 1911}|(be}
\toendnotes[C]{\smallbreak\pagebreak[2]}\Standort{DLA, A:Schnitzler, HS.NZ85.1.3160.}
\physDesc{Brief, 1 Blatt, 3 Seiten, 725 Zeichen
\newline{}Handschrift: lila Tinte, lateinische Kurrent
\newline{}Schnitzler: mit Bleistift Vermerk »Goldmann\pwindex{Goldmann, Eva Marie 27.10.1877 – 02.11.1937@\textsc{Goldmann, Eva Marie} (27.10.1877 – 02.11.1937)|pw}.«
                               }\toendnotes[C]{\smallbreak}
\pstart
           \raggedleft{}{\pb}21. IX. 1911.\pend
           
\pstart
           \textcolor{gray}{\textbf{EG}}\hfill \textcolor{gray}{\textbf{W. SCHÖNEBERGER-UFER 34\oindex{Schoeneberger Ufer@\textbf{Schöneberger Ufer}, \emph{Straße (K.STR)}|pw}.}}\pend
           
\pstart{}Verehrter Herr Doctor, \pend\vspace{0.5em}
\pstart
           ich danke Ihnen vielmals für Ihre \label{K_L03540-1v}\edtext{freundlichen Zeilen}{\lemma{\textnormal{\emph{freundlichen Zeilen}}}\Cendnote{\textnormal{Es ist anzunehmen,
                  dass auch sie Schnitzler anlässlich des
                  Todes seiner Mutter\pwindex{Schnitzler, Louise 1840-07-08 – 1911-09-09@\textsc{Schnitzler, Louise} (1840-07-08 – 1911-09-09)|pwkv}
                  kondoliert hatte.}}}\label{K_L03540-1}.\pend
           
\pstart
           Und ich möchte Ihnen sagen, dass Paul\pwindex{Goldmann, Paul 31.01.1865 – 25.09.1935@\textsc{Goldmann, Paul} (31.01.1865 – 25.09.1935), \emph{Schriftsteller/Schriftstellerin, Journalist/Journalistin}|pw} unter
               dem \label{K_L03540-2v}\edtext{Zerwürfnis}{\lemma{\textnormal{\emph{Zerwürfnis}}}\Cendnote{\textnormal{Ende 1910/Anfang 1911, siehe insbesondere Paul Goldmann an Arthur Schnitzler, 13. 1. 1911 und Arthur Schnitzler an Paul Goldmann, 1. 2. 1911.
               }}}\label{K_L03540-2} mit Ihnen sehr
                  \label{K_L03540-3v}\edtext{gelitten}{\lemma{\textnormal{\emph{gelitten}}}\Cendnote{\textnormal{Siehe A. S.: \emph{Tagebuch}, 23. 9. 1911.
               }}}\label{K_L03540-3} hat.\pend
           
\pstart
           In seinem, und auch {\pb}schon in meinem Alter, kommt kein
               Ersatz mehr für das, was einem genommen wird, was einem theuer war und ein Stück
               Jugend bedeutet hat.\pend
           
\pstart
           Ich würde Sie, verehrter Herr Doctor, gerne einmal wieder sprechen, und ich bilde mir
               ein, dass alles anders gekommen wäre, wenn ich im vergangenen Winter mit in Wien\oindex{Wien@\textbf{Wien}, \emph{A.ADM2}|pw} gewesen wäre.\pend
           
\pstart
           \label{K_L03540-4v}\edtext{Paul\pwindex{Goldmann, Paul 31.01.1865 – 25.09.1935@\textsc{Goldmann, Paul} (31.01.1865 – 25.09.1935), \emph{Schriftsteller/Schriftstellerin, Journalist/Journalistin}|pw}{ }\uline{ahnt nicht}}{\lemma{\textnormal{\emph{Paul ahnt nicht}}}\Cendnote{\textnormal{Schnitzlers Antwort auf diesen Brief ist
                  nicht überliefert. Seine Einordnung dieses und des folgenden Schreibens von Eva Marie Goldmann\pwindex{Goldmann, Eva Marie 27.10.1877 – 02.11.1937@\textsc{Goldmann, Eva Marie} (27.10.1877 – 02.11.1937)|pwk} ist im \emph{Tagebuch}\pwindex{Tagebuch@\emph{Tagebuch}|pwk}eintrag zum 5. 10. 1911 nachzulesen.}}}\label{K_L03540-4}, dass ich Ihnen
                  heute schreibe, u. wird es auch nicht erfahren.\pend
           
\pstart
           Er ist augenblicklich nicht hier\oindex{Berlin@\textbf{Berlin}, \emph{P.PPLC}|pwv}, sondern wegen eines widerwärtigen \label{K_L03540-5v}\edtext{Processes in Wien\oindex{Wien@\textbf{Wien}, \emph{A.ADM2}|pw}}{\lemma{\textnormal{\emph{Processes in Wien}}}\Cendnote{\textnormal{Wieso Goldmann\pwindex{Goldmann, Paul 31.01.1865 – 25.09.1935@\textsc{Goldmann, Paul} (31.01.1865 – 25.09.1935), \emph{Schriftsteller/Schriftstellerin, Journalist/Journalistin}|pwk} wegen dieses Prozesses in Wien\oindex{Wien@\textbf{Wien}, \emph{A.ADM2}|pwk} weilte, ist unklar. Es dürfte sich aber um eine Seitenaffäre der
                  vielfach beachteten Anklage gegen Graf Gisbert
                     von Wolff-Metternich\pwindex{Wolff-Metternich, Gisbert von *~1885@\textsc{Wolff-Metternich, Gisbert von} (*~1885), \emph{Schauspieler/Schauspielerin}|pwk} gehandelt haben. Dieser war bereits mehrfach des
                  Betrugs beschuldigt worden und hatte hohe Schulden. Einen Prozess wollte er
                  abwehren, indem er angab, die wohlhabende Dolly
                     Landsberger\pwindex{Landsberger, Dolly 1892-11-11 – 1923-04-15@\textsc{Landsberger, Dolly} (1892-11-11 – 1923-04-15)|pwk} zu heiraten. Deren Mutter Gertrud Wertheim\pwindex{Wertheim, Gertrud 1867-12-05 – 1927-11-27@\textsc{Wertheim, Gertrud} (1867-12-05 – 1927-11-27), \emph{Schriftsteller/Schriftstellerin}|pwk} (bekannt als Schriftstellerin Truth\pwindex{Wertheim, Gertrud 1867-12-05 – 1927-11-27@\textsc{Wertheim, Gertrud} (1867-12-05 – 1927-11-27), \emph{Schriftsteller/Schriftstellerin}|pwk}) trat als Belastungszeugin gegen ihn auf, wodurch
                  auch ihr ›Vorleben‹ als Dichterin und das Erbe aus erster Ehe für die Verteidigung
                  und in Folge die Klatschpresse interessant wurden. Goldmann\pwindex{Goldmann, Paul 31.01.1865 – 25.09.1935@\textsc{Goldmann, Paul} (31.01.1865 – 25.09.1935), \emph{Schriftsteller/Schriftstellerin, Journalist/Journalistin}|pwk} dürfte sich in Bozen\oindex{Bozen@\textbf{Bozen}, \emph{P.PPLA2}|pwk} dafür stark gemacht haben, dass weder Landsberger\pwindex{Landsberger, Dolly 1892-11-11 – 1923-04-15@\textsc{Landsberger, Dolly} (1892-11-11 – 1923-04-15)|pwk} noch ihre Mutter\pwindex{Wertheim, Gertrud 1867-12-05 – 1927-11-27@\textsc{Wertheim, Gertrud} (1867-12-05 – 1927-11-27), \emph{Schriftsteller/Schriftstellerin}|pwkv} an dem Prozess teilnahmen: »Frau \so{Wertheim}\pwindex{Wertheim, Gertrud 1867-12-05 – 1927-11-27@\textsc{Wertheim, Gertrud} (1867-12-05 – 1927-11-27), \emph{Schriftsteller/Schriftstellerin}|pw} hat der ›Voss. Ztg.\orgindex{Vossische Zeitung@Vossische Zeitung|pw}‹ einen Brief\pwindex{Leserbrief von Gertrud Wertheim mit Klage ueber Paul Goldmann]@\emph{[Leserbrief von Gertrud Wertheim mit Klage über Paul Goldmann]}|pwv} zugeſchickt, in
                        dem ſie gegen den Berlin\oindex{Berlin@\textbf{Berlin}, \emph{P.PPLC}|pw}er
                        Schriftſteller \so{Goldmann}\pwindex{Goldmann, Paul 31.01.1865 – 25.09.1935@\textsc{Goldmann, Paul} (31.01.1865 – 25.09.1935), \emph{Schriftsteller/Schriftstellerin, Journalist/Journalistin}|pw} den Vorwurf erhebt, ſich in eine ihm ganz fremde Angelegenheit
                        unbefugt eingemengt zu haben. Er habe in Bozen\oindex{Bozen@\textbf{Bozen}, \emph{P.PPLA2}|pw} ihren Mann\pwindex{Wertheim, Wolf Waldemar 1867-03-23 – 1940@\textsc{Wertheim, Wolf Waldemar} (1867-03-23 – 1940), \emph{Warenhausbesitzer/Warenhausbesitzerin}|pwv}, ihre Tochter\pwindex{Landsberger, Dolly 1892-11-11 – 1923-04-15@\textsc{Landsberger, Dolly} (1892-11-11 – 1923-04-15)|pwv} und ſie ſelbſt in der intenſivſten Art und Weiſe
                        beſchworen, daß Frau Wertheim\pwindex{Wertheim, Gertrud 1867-12-05 – 1927-11-27@\textsc{Wertheim, Gertrud} (1867-12-05 – 1927-11-27), \emph{Schriftsteller/Schriftstellerin}|pw} nicht
                        zum Metternich\pwindex{Wolff-Metternich, Gisbert von *~1885@\textsc{Wolff-Metternich, Gisbert von} (*~1885), \emph{Schauspieler/Schauspielerin}|pw}-Prozeß fahre. ›Er
                        malte jedem einzelnen‹, heißt es in dem Brief\pwindex{Leserbrief von Gertrud Wertheim mit Klage ueber Paul Goldmann]@\emph{[Leserbrief von Gertrud Wertheim mit Klage über Paul Goldmann]}|pwv}, ›in den düſterſten Farben mein
                        bevorſtehendes Geſchick aus und ſagte weiter, aus der Zeugin\pwindex{Wertheim, Gertrud 1867-12-05 – 1927-11-27@\textsc{Wertheim, Gertrud} (1867-12-05 – 1927-11-27), \emph{Schriftsteller/Schriftstellerin}|pwv} würde eine Angeklagte
                        werden. Eine Kataſtrophe würde eintreten. Da wir alle Herrn Paul Goldmann\pwindex{Goldmann, Paul 31.01.1865 – 25.09.1935@\textsc{Goldmann, Paul} (31.01.1865 – 25.09.1935), \emph{Schriftsteller/Schriftstellerin, Journalist/Journalistin}|pw} nur ganz flüchtig
                        kennen, erregte ſeine Art und Weiſe begreifliche Verwunderung. Er gab mir
                        ſogar den Rat, mich, die ich damals noch heiter und vergnügt war, durch
                        ärztliche Atteſte zu ſchützen. Dieſes gewiß befremdende Benehmen konnte ich
                        mir nur dadurch erklären, daß Herr Goldmann\pwindex{Goldmann, Paul 31.01.1865 – 25.09.1935@\textsc{Goldmann, Paul} (31.01.1865 – 25.09.1935), \emph{Schriftsteller/Schriftstellerin, Journalist/Journalistin}|pw} von irgendeiner Seite beauftragt war. Denn ein derartiges
                        Eingreifen würde höchſtens bei Freunden oder ſonſt Nächſtſtehenden zu
                        erklären oder zu entſchuldigen sein.{[}‹{]}\pwindex{Prozess gegen den Grafen Wolff-Metternich. Eine Erklaerung der Frau Wertheim@\emph{Der Prozeß gegen den Grafen Wolff-Metternich. Eine Erklärung der Frau Wertheim}|pwv}« ([O. V.]: \emph{Der Prozeß gegen den
                        Grafen Wolff-Metternich. Eine Erklärung der Frau Wertheim}\pwindex{Prozess gegen den Grafen Wolff-Metternich. Eine Erklaerung der Frau Wertheim@\emph{Der Prozeß gegen den Grafen Wolff-Metternich. Eine Erklärung der Frau Wertheim}|pwk}. In: \emph{Neues Wiener Journal}\pwindex{Neues Wiener Journal@\emph{Neues Wiener Journal}|pwk}, Jg. 19, Nr. 6454,
                        10. 10. 1911, S. 9.)}}}\label{K_L03540-5}.\pend
           
\pstart
           Mit den besten Wünschen für {\pb}Sie u. die Ihren
               {\\[\baselineskip]}Ihre {\\[\baselineskip]}\spacefill\mbox{EvaMarieGoldmann.}\pend
           \leftskip=0em{}\selectlanguage{ngerman}\endnumbering\briefempfaengerindex{Schnitzler, Arthur@\textsc{Schnitzler, Arthur}!zzzGoldmann, Eva Marie@\emph{von Eva Marie Goldmann}!1911-09-211@{21. 9. 1911}|)be}\mylabel{L03540h}  \normalsize

\doendnotes{C}
\bigskip
\vfill

\clearpage

\footnotesize

\lohead{\textsc{register}}

% Definiere theindex-Environment komplett neu ohne reledmac
\makeatletter
\renewenvironment{theindex}{%
  \section*{\indexname}%
  \setlength{\parindent}{0pt}%
  \setlength{\parskip}{0pt plus 0.3pt}%
  \let\item\@idxitem
}{%
  \clearpage
}
\makeatother

\IfFileExists{\jobname-pw.ind}{\input{\jobname-pw.ind}}{}

\end{document}

      