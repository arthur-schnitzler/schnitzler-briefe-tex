%% latex-leseansicht-vorspann.tex
%% Vorspann für die Leseansicht.
%% Lädt die gemeinsame Datei latex-vorspann.tex mit nicht gesetztem Schalter.

\newif\ifkorrekturansicht
\korrekturansichtfalse

\input{../tex-inputs/latex-vorspann}


\section[ Eva Marie Goldmann an Arthur Schnitzler, 21. 9. 1911]{L03540 Eva Marie Goldmann an Arthur Schnitzler,  21. 9. 1911}
\nopagebreak\mylabel{L03540v}
\rehead{ }\normalsize\beginnumbering\briefempfaengerindex{Schnitzler, Arthur@\textsc{Schnitzler, Arthur}!zzzGoldmann, Eva Marie@\emph{von Eva Marie Goldmann}!1911-09-211@{21. 9. 1911}|(be}
\toendnotes[C]{\smallbreak\pagebreak[2]}
\correspDesc{Versand  durch Eva Marie Goldmann am 21. 9. 1911 in Berlin
\newline{}Erhalt  durch Arthur Schnitzler am [23. 9. 1911] in Wien}\toendnotes[C]{\smallbreak}
\Standort{DLA, A:Schnitzler, HS.NZ85.1.3160.}
\physDesc{Brief, 1 Blatt, 3 Seiten, 725 Zeichen
\newline{}Handschrift: lila Tinte, lateinische Kurrent
\newline{}Schnitzler: mit Bleistift Vermerk »Goldmann\pwindex{Goldmann, Eva Marie 27.\,10.\,1877 Wien – 2.\,11.\,1937 ebd.@\textsc{Goldmann, Eva Marie} (27.\,10.\,1877 Wien – 2.\,11.\,1937 ebd.)|pw}.«
                               }\toendnotes[C]{\smallbreak}
\pstart
           \raggedleft{}{\pb}21. IX. 1911.\pend
           
\pstart
           \textcolor{gray}{\textbf{EG}}\hfill \textcolor{gray}{\textbf{W. SCHÖNEBERGER-UFER 34\oindex{Schöneberger Ufer@\textbf{Schöneberger Ufer}, \emph{Straße}|pw}.}}\pend
           
\pstart{}Verehrter Herr Doctor,\pend\vspace{0.5em}
\pstart
           ich danke Ihnen vielmals für Ihre \label{K_L03540-1v}\edtext{freundlichen Zeilen}{\lemma{\textnormal{\emph{freundlichen Zeilen}}}\Cendnote{\textnormal{Es ist anzunehmen,
                  dass auch sie Schnitzler anlässlich des
                  Todes seiner Mutter\pwindex{Schnitzler, Louise 8.\,7.\,1840 Kőszeg – 9.\,9.\,1911 Wien@\textsc{Schnitzler, Louise} (8.\,7.\,1840 Kőszeg – 9.\,9.\,1911 Wien)|pwkv}
                  kondoliert hatte.}}}\label{K_L03540-1}.\pend
           
\pstart
           Und ich möchte Ihnen sagen, dass Paul\pwindex{Goldmann, Paul 31.\,1.\,1865 Breslau – 25.\,9.\,1935 Wien@\textsc{Goldmann, Paul} (31.\,1.\,1865 Breslau – 25.\,9.\,1935 Wien), \emph{Schriftsteller, Journalist}|pw} unter
               dem \label{K_L03540-2v}\edtext{Zerwürfnis}{\lemma{\textnormal{\emph{Zerwürfnis}}}\Cendnote{\textnormal{Ende 1910/Anfang 1911, siehe insbesondere XXXX Auszeichnungsfehler: Dokument L03475 nicht gefunden und XXXX Auszeichnungsfehler: Dokument L03521 nicht gefunden.
               }}}\label{K_L03540-2} mit Ihnen sehr
                  \label{K_L03540-3v}\edtext{gelitten}{\lemma{\textnormal{\emph{gelitten}}}\Cendnote{\textnormal{Siehe A. S.: \emph{Tagebuch}, 23. 9. 1911.
               }}}\label{K_L03540-3} hat.\pend
           
\pstart
           In seinem, und auch {\pb}schon in meinem Alter, kommt kein
               Ersatz mehr für das, was einem genommen wird, was einem theuer war und ein Stück
               Jugend bedeutet hat.\pend
           
\pstart
           Ich würde Sie, verehrter Herr Doctor, gerne einmal wieder sprechen, und ich bilde mir
               ein, dass alles anders gekommen wäre, wenn ich im vergangenen Winter mit in Wien\oindex{Wien@\textbf{Wien}, \emph{Verwaltungsgebiet}|pw} gewesen wäre.\pend
           
\pstart
           \label{K_L03540-4v}\edtext{Paul\pwindex{Goldmann, Paul 31.\,1.\,1865 Breslau – 25.\,9.\,1935 Wien@\textsc{Goldmann, Paul} (31.\,1.\,1865 Breslau – 25.\,9.\,1935 Wien), \emph{Schriftsteller, Journalist}|pw}{ }\uline{ahnt nicht}}{\lemma{\textnormal{\emph{Paul ahnt nicht}}}\Cendnote{\textnormal{Schnitzlers Antwort auf diesen Brief ist
                  nicht überliefert. Seine Einordnung dieses und des folgenden Schreibens von Eva Marie Goldmann\pwindex{Goldmann, Eva Marie 27.\,10.\,1877 Wien – 2.\,11.\,1937 ebd.@\textsc{Goldmann, Eva Marie} (27.\,10.\,1877 Wien – 2.\,11.\,1937 ebd.)|pwk} ist im \emph{Tagebuch}\pwindex{Schnitzler, Arthur 15.\,5.\,1862 Wien – 21.\,10.\,1931 ebd.@\textsc{Schnitzler, Arthur} (15.\,5.\,1862 Wien – 21.\,10.\,1931 ebd.), \emph{Schriftsteller, Mediziner}!Tagebuch@\strich\emph{Tagebuch}|pwk}eintrag zum 5. 10. 1911 nachzulesen.}}}\label{K_L03540-4}, dass ich Ihnen
                  heute schreibe, u. wird es auch nicht erfahren.\pend
           
\pstart
           Er ist augenblicklich nicht hier\oindex{Berlin@\textbf{Berlin}, \emph{Hauptstadt}|pwv}, sondern wegen eines widerwärtigen \label{K_L03540-5v}\edtext{Processes in Wien\oindex{Wien@\textbf{Wien}, \emph{Verwaltungsgebiet}|pw}}{\lemma{\textnormal{\emph{Processes in Wien}}}\Cendnote{\textnormal{Wieso Goldmann\pwindex{Goldmann, Paul 31.\,1.\,1865 Breslau – 25.\,9.\,1935 Wien@\textsc{Goldmann, Paul} (31.\,1.\,1865 Breslau – 25.\,9.\,1935 Wien), \emph{Schriftsteller, Journalist}|pwk} wegen dieses Prozesses in Wien\oindex{Wien@\textbf{Wien}, \emph{Verwaltungsgebiet}|pwk} weilte, ist unklar. Es dürfte sich aber um eine Seitenaffäre der
                  vielfach beachteten Anklage gegen Graf Gisbert
                     von Wolff-Metternich\pwindex{Wolff-Metternich, Gisbert von *~1885@\textsc{Wolff-Metternich, Gisbert von} (*~1885), \emph{Schauspieler}|pwk} gehandelt haben. Dieser war bereits mehrfach des
                  Betrugs beschuldigt worden und hatte hohe Schulden. Einen Prozess wollte er
                  abwehren, indem er angab, die wohlhabende Dolly
                     Landsberger\pwindex{Landsberger, Dolly 11.\,11.\,1892 Berlin – 15.\,4.\,1923@\textsc{Landsberger, Dolly} (11.\,11.\,1892 Berlin – 15.\,4.\,1923)|pwk} zu heiraten. Deren Mutter Gertrud Wertheim\pwindex{Wertheim, Gertrud 5.\,12.\,1867 Berlin – 27.\,11.\,1927 Montreux@\textsc{Wertheim, Gertrud} (5.\,12.\,1867 Berlin – 27.\,11.\,1927 Montreux), \emph{Schriftstellerin}|pwk} (bekannt als Schriftstellerin Truth\pwindex{Wertheim, Gertrud 5.\,12.\,1867 Berlin – 27.\,11.\,1927 Montreux@\textsc{Wertheim, Gertrud} (5.\,12.\,1867 Berlin – 27.\,11.\,1927 Montreux), \emph{Schriftstellerin}|pwk}) trat als Belastungszeugin gegen ihn auf, wodurch
                  auch ihr ›Vorleben‹ als Dichterin und das Erbe aus erster Ehe für die Verteidigung
                  und in Folge die Klatschpresse interessant wurden. Goldmann\pwindex{Goldmann, Paul 31.\,1.\,1865 Breslau – 25.\,9.\,1935 Wien@\textsc{Goldmann, Paul} (31.\,1.\,1865 Breslau – 25.\,9.\,1935 Wien), \emph{Schriftsteller, Journalist}|pwk} dürfte sich in Bozen\oindex{Bozen@\textbf{Bozen}, \emph{Hauptstadt}|pwk} dafür stark gemacht haben, dass weder Landsberger\pwindex{Landsberger, Dolly 11.\,11.\,1892 Berlin – 15.\,4.\,1923@\textsc{Landsberger, Dolly} (11.\,11.\,1892 Berlin – 15.\,4.\,1923)|pwk} noch ihre Mutter\pwindex{Wertheim, Gertrud 5.\,12.\,1867 Berlin – 27.\,11.\,1927 Montreux@\textsc{Wertheim, Gertrud} (5.\,12.\,1867 Berlin – 27.\,11.\,1927 Montreux), \emph{Schriftstellerin}|pwkv} an dem Prozess teilnahmen: »Frau \so{Wertheim}\pwindex{Wertheim, Gertrud 5.\,12.\,1867 Berlin – 27.\,11.\,1927 Montreux@\textsc{Wertheim, Gertrud} (5.\,12.\,1867 Berlin – 27.\,11.\,1927 Montreux), \emph{Schriftstellerin}|pw} hat der ›Voss. Ztg.\orgindex{Vossische Zeitung@Vossische Zeitung|pw}‹ einen Brief\pwindex{Wertheim, Gertrud 5.\,12.\,1867 Berlin – 27.\,11.\,1927 Montreux@\textsc{Wertheim, Gertrud} (5.\,12.\,1867 Berlin – 27.\,11.\,1927 Montreux), \emph{Schriftstellerin}!Leserbrief von Gertrud Wertheim mit Klage über Paul Goldmann]@\strich\emph{[Leserbrief von Gertrud Wertheim mit Klage über Paul Goldmann]}|pwv} zugeſchickt, in
                        dem{ }ſie gegen den Berlin\oindex{Berlin@\textbf{Berlin}, \emph{Hauptstadt}|pw}er
                        Schriftſteller \so{Goldmann}\pwindex{Goldmann, Paul 31.\,1.\,1865 Breslau – 25.\,9.\,1935 Wien@\textsc{Goldmann, Paul} (31.\,1.\,1865 Breslau – 25.\,9.\,1935 Wien), \emph{Schriftsteller, Journalist}|pw} den Vorwurf erhebt,{ }ſich in eine ihm ganz fremde Angelegenheit
                        unbefugt eingemengt zu haben. Er habe in Bozen\oindex{Bozen@\textbf{Bozen}, \emph{Hauptstadt}|pw} ihren Mann\pwindex{Wertheim, Wolf Waldemar 23.\,3.\,1867 – 1940@\textsc{Wertheim, Wolf Waldemar} (23.\,3.\,1867 – 1940), \emph{Warenhausbesitzer}|pwv}, ihre Tochter\pwindex{Landsberger, Dolly 11.\,11.\,1892 Berlin – 15.\,4.\,1923@\textsc{Landsberger, Dolly} (11.\,11.\,1892 Berlin – 15.\,4.\,1923)|pwv} und{ }ſie{ }ſelbſt in der intenſivſten Art und Weiſe
                        beſchworen, daß Frau Wertheim\pwindex{Wertheim, Gertrud 5.\,12.\,1867 Berlin – 27.\,11.\,1927 Montreux@\textsc{Wertheim, Gertrud} (5.\,12.\,1867 Berlin – 27.\,11.\,1927 Montreux), \emph{Schriftstellerin}|pw} nicht
                        zum Metternich\pwindex{Wolff-Metternich, Gisbert von *~1885@\textsc{Wolff-Metternich, Gisbert von} (*~1885), \emph{Schauspieler}|pw}-Prozeß fahre. ›Er
                        malte jedem einzelnen‹, heißt es in dem Brief\pwindex{Wertheim, Gertrud 5.\,12.\,1867 Berlin – 27.\,11.\,1927 Montreux@\textsc{Wertheim, Gertrud} (5.\,12.\,1867 Berlin – 27.\,11.\,1927 Montreux), \emph{Schriftstellerin}!Leserbrief von Gertrud Wertheim mit Klage über Paul Goldmann]@\strich\emph{[Leserbrief von Gertrud Wertheim mit Klage über Paul Goldmann]}|pwv}, ›in den düſterſten Farben mein
                        bevorſtehendes Geſchick aus und{ }ſagte weiter, aus der Zeugin\pwindex{Wertheim, Gertrud 5.\,12.\,1867 Berlin – 27.\,11.\,1927 Montreux@\textsc{Wertheim, Gertrud} (5.\,12.\,1867 Berlin – 27.\,11.\,1927 Montreux), \emph{Schriftstellerin}|pwv} würde eine Angeklagte
                        werden. Eine Kataſtrophe würde eintreten. Da wir alle Herrn Paul Goldmann\pwindex{Goldmann, Paul 31.\,1.\,1865 Breslau – 25.\,9.\,1935 Wien@\textsc{Goldmann, Paul} (31.\,1.\,1865 Breslau – 25.\,9.\,1935 Wien), \emph{Schriftsteller, Journalist}|pw} nur ganz flüchtig
                        kennen, erregte{ }ſeine Art und Weiſe begreifliche Verwunderung. Er gab mir{ }ſogar den Rat, mich, die ich damals noch heiter und vergnügt war, durch
                        ärztliche Atteſte zu{ }ſchützen. Dieſes gewiß befremdende Benehmen konnte ich
                        mir nur dadurch erklären, daß Herr Goldmann\pwindex{Goldmann, Paul 31.\,1.\,1865 Breslau – 25.\,9.\,1935 Wien@\textsc{Goldmann, Paul} (31.\,1.\,1865 Breslau – 25.\,9.\,1935 Wien), \emph{Schriftsteller, Journalist}|pw} von irgendeiner Seite beauftragt war. Denn ein derartiges
                        Eingreifen würde höchſtens bei Freunden oder{ }ſonſt Nächſtſtehenden zu
                        erklären oder zu entſchuldigen sein.{[}‹{]}\pwindex{Prozeß gegen den Grafen Wolff-Metternich. Eine Erklärung der Frau Wertheim@\emph{Der Prozeß gegen den Grafen Wolff-Metternich. Eine Erklärung der Frau Wertheim}|pwv}« ([O. V.]: \emph{Der Prozeß gegen den
                        Grafen Wolff-Metternich. Eine Erklärung der Frau Wertheim}\pwindex{Prozeß gegen den Grafen Wolff-Metternich. Eine Erklärung der Frau Wertheim@\emph{Der Prozeß gegen den Grafen Wolff-Metternich. Eine Erklärung der Frau Wertheim}|pwk}. In: \emph{Neues Wiener Journal}\pwindex{Neues Wiener Journal@\emph{Neues Wiener Journal}|pwk}, Jg. 19, Nr. 6454,
                        10. 10. 1911, S. 9.)}}}\label{K_L03540-5}.\pend
           
\pstart
           Mit den besten Wünschen für {\pb}Sie u. die Ihren
               {\\[\baselineskip]}Ihre {\\[\baselineskip]}\spacefill\mbox{EvaMarieGoldmann.}\pend
           \leftskip=0em{}\selectlanguage{ngerman}\endnumbering\briefempfaengerindex{Schnitzler, Arthur@\textsc{Schnitzler, Arthur}!zzzGoldmann, Eva Marie@\emph{von Eva Marie Goldmann}!1911-09-211@{21. 9. 1911}|)be}\mylabel{L03540h}  \newcommand{\dateiname}{L03540}\newcommand{\titel}{Eva Marie Goldmann an Arthur Schnitzler, 21. 9. 1911}\newcommand{\editorInnen}{Martin Anton Müller und Laura Untner}%% latex-leseansicht-abspann.tex
%% Abspann für die Leseansicht.
%% Der Schalter \ifkorrekturansicht ist bereits durch den Vorspann gesetzt.

%% latex-abspann.tex
%% Gemeinsamer Abspann für Korrekturansicht und Leseansicht.
%% Setzt den Schalter \ifkorrekturansicht voraus (gesetzt in den
%% einbindenden Dateien latex-korrekturansicht-abspann.tex bzw.
%% latex-leseansicht-abspann.tex).
%% ---------------------------------------------------------------

\normalsize

% Das esempio-Environment wird nur in der Leseansicht benötigt
\ifkorrekturansicht\else
\newenvironment{esempio}[3]%
{
    \vspace{1.5ex}
    \rlap{\underline{#1}}
    \par
    \setlength{\parindent}{0cm}
    \nopagebreak
    \leftskip=#2cm
    \rightskip=#3cm
}
{
    \par
}
\fi

\doendnotes{C}
\bigskip
\vfill

\clearpage

\footnotesize

\ifkorrekturansicht
  \lohead{\textsc{register}}
\fi

% theindex-Environment neu definieren ohne reledmac
\makeatletter
\renewenvironment{theindex}{%
  \ifkorrekturansicht
    \section*{\indexname}%
  \else
    \subsubsection*{Index der erwähnten Entitäten}%
  \fi
  \setlength{\parindent}{0pt}%
  \setlength{\parskip}{0pt plus 0.3pt}%
  \let\item\@idxitem
}{%
  \ifkorrekturansicht\clearpage\fi
}
\makeatother

\IfFileExists{\jobname-pw.ind}{\input{\jobname-pw.ind}}{}

% Quellenangabe nur in der Leseansicht
\ifkorrekturansicht\else
% Fallback-Definitionen, falls die .tex-Datei \titel etc. nicht gesetzt hat
\providecommand{\titel}{}
\providecommand{\editorInnen}{}
\providecommand{\dateiname}{\jobname}

\vspace{3cm}

\vfill

\footnotesize
\textsc{Quelle}: \titel. Herausgegeben von {\editorInnen}. In: \emph{Arthur Schnitzler: Briefwechsel mit Autorinnen und Autoren}.
 Digitale Edition, https://schnitzler-briefe.acdh.oeaw.ac.at/{\dateiname}.html (Stand \today)
\fi

\end{document}


