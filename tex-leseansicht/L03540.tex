%% latex-leseansicht-vorspann.tex
%% Vorspann für die Leseansicht.
%% Lädt die gemeinsame Datei latex-vorspann.tex mit nicht gesetztem Schalter.

\newif\ifkorrekturansicht
\korrekturansichtfalse

\input{../tex-inputs/latex-vorspann}

\begin{center}
            \textcolor{red}{ENTWURF, NICHT FERTIG KORRIGIERT}
                      \end{center}
            
         
         \renewcommand{\erwaehntePersonen}{Personen: Eva Marie Goldmann, Paul Goldmann, Dolly Landsberger, Louise Schnitzler, Gertrud Wertheim, Wolf Waldemar Wertheim, Gisbert von Wolff-Metternich}
         \renewcommand{\erwaehnteInstitutionen}{Institutionen: Vossische Zeitung}
         \renewcommand{\erwaehnteOrte}{Orte: Berlin, Bozen, Schöneberger Ufer, Wien}
         \renewcommand{\erwaehnteWerke}{Werke: Der Prozeß gegen den Grafen Wolff-Metternich. Eine Erklärung der Frau Wertheim, Neues Wiener Journal, Tagebuch, [Leserbrief von Gertrud Wertheim mit Klage über Paul Goldmann]}
               \section[ Eva Marie Goldmann an Arthur Schnitzler, 21. 9. 1911]{ Eva Marie Goldmann an Arthur Schnitzler, 21. 9. 1911}\nopagebreak\mylabel{v}\rehead{ }\begin{ledgroupsized}[t]{13cm}\normalsize\beginnumbering\briefempfaengerindex{Schnitzler, Arthur@\textsc{Schnitzler, Arthur}!zzzGoldmann, Eva Marie@\emph{von Eva Marie Goldmann}!1911-09-211@{21. 9. 1911}|(be} \toendnotes[C]{\smallbreak\pagebreak[2]} \Standort{DLA, A:Schnitzler, HS.NZ85.1.3160.}
\physDesc{Brief, 1 Blatt, 3 Seiten, 725 Zeichen
\newline{}Handschrift: lila Tinte, lateinische Kurrent
\newline{}Schnitzler: mit Bleistift Vermerk »Goldmann\pwindex{Goldmann, Eva Marie 27.10.1877 – 02.11.1937@\textsc{Goldmann, Eva Marie} (27.10.1877 – 02.11.1937)|pw}.«
                               }\toendnotes[C]{\smallbreak}\pstart
           \raggedleft{}{\pb}21. IX. 1911.\pend
           \pstart
           \textcolor{gray}{\textbf{EG}}\hfill \textcolor{gray}{\textbf{W. SCHÖNEBERGER-UFER 34\oindex{Schoeneberger Ufer@\textbf{Schöneberger Ufer}|pw}.}}\pend
           \pstart{}Verehrter Herr Doctor, \pend\pstart
           ich danke Ihnen vielmals für Ihre \label{K_L03540-1v}\edtext{freundlichen Zeilen}{\lemma{\textnormal{\emph{freundlichen Zeilen}}}\Cendnote{\textnormal{Es ist anzunehmen,
                  dass auch sie Schnitzler\pwindex{Schnitzler, Arthur 15.05.1862 – 21.10.1931@\textsc{Schnitzler, Arthur} (15.05.1862 – 21.10.1931), \emph{Schriftsteller, Mediziner}|pwk} anlässlich des
                  Todes seiner Mutter\pwindex{Schnitzler, Louise 1840-07-08 – 1911-09-09@\textsc{Schnitzler, Louise} (1840-07-08 – 1911-09-09)|pwkv}
                  kondoliert hatte.}}}\label{K_L03540-1h}.\pend
           \pstart
           Und ich möchte Ihnen sagen, dass Paul\pwindex{Goldmann, Paul 31.01.1865 – 25.09.1935@\textsc{Goldmann, Paul} (31.01.1865 – 25.09.1935), \emph{Schriftsteller, Journalist}|pw} unter
               dem \label{K_L03540-2v}\edtext{Zerwürfnis}{\lemma{\textnormal{\emph{Zerwürfnis}}}\Cendnote{\textnormal{Ende 1910/Anfang 1911, siehe insbesondere Paul Goldmann an Arthur Schnitzler, 13. 1. 1911 und Arthur Schnitzler an Paul Goldmann, 1. 2. 1911 u }}}\label{K_L03540-2h} mit Ihnen sehr
                  \label{K_L03540-3v}\edtext{gelitten}{\lemma{\textnormal{\emph{gelitten}}}\Cendnote{\textnormal{vgl. A. S.: \emph{Tagebuch}, 23. 9. 1911}}}\label{K_L03540-3h} hat.\pend
           \pstart
           In seinem, und auch {\pb}schon in meinem Alter, kommt kein
               Ersatz mehr für das, was einem genommen wird, was einem theuer war und ein Stück
               Jugend bedeutet hat.\pend
           \pstart
           Ich würde Sie, verehrter Herr Doctor, gerne einmal wieder sprechen, und ich bilde mir
               ein, dass alles anders gekommen wäre, wenn ich im vergangenen Winter mit in Wien\oindex{Wien@\textbf{Wien}|pw} gewesen wäre.\pend
           \pstart
           \label{K_L03540-4v}\edtext{Paul\pwindex{Goldmann, Paul 31.01.1865 – 25.09.1935@\textsc{Goldmann, Paul} (31.01.1865 – 25.09.1935), \emph{Schriftsteller, Journalist}|pw}{ }\uline{ahnt nicht}}{\lemma{\textnormal{\emph{Paul ahnt nicht}}}\Cendnote{\textnormal{Schnitzler\pwindex{Schnitzler, Arthur 15.05.1862 – 21.10.1931@\textsc{Schnitzler, Arthur} (15.05.1862 – 21.10.1931), \emph{Schriftsteller, Mediziner}|pwk}s Antwort auf diesen Brief ist nicht überliefert. 
                  Seine Einordnung dieser und des folgenden Schreibens von Eva Marie Goldmann\pwindex{Goldmann, Eva Marie 27.10.1877 – 02.11.1937@\textsc{Goldmann, Eva Marie} (27.10.1877 – 02.11.1937)|pwk}
                  gibt Schnitzler\pwindex{Schnitzler, Arthur 15.05.1862 – 21.10.1931@\textsc{Schnitzler, Arthur} (15.05.1862 – 21.10.1931), \emph{Schriftsteller, Mediziner}|pwk} im \emph{Tagebuch}\pwindex{\textcolor{red}{\textsuperscript{XXXX1 indx}}!Tagebuch1981 – 2000@\strich\emph{Tagebuch} {[}Hrsg., 1981 – 2000{]}|pwk} zum
                  vgl. A. S.: \emph{Tagebuch}, 5. 10. 1911.}}}\label{K_L03540-4h}, dass ich
               Ihnen heute schreibe, u. wird es auch nicht
               erfahren.\pend
           \pstart
           Er ist augenblicklich nicht hier\oindex{Berlin@\textbf{Berlin}|pwv}, sondern wegen eines widerwärtigen \label{K_L03540-5v}\edtext{Processes in Wien\oindex{Wien@\textbf{Wien}|pw}}{\lemma{\textnormal{\emph{Processes in Wien}}}\Cendnote{\textnormal{Wieso Goldmann\pwindex{Goldmann, Paul 31.01.1865 – 25.09.1935@\textsc{Goldmann, Paul} (31.01.1865 – 25.09.1935), \emph{Schriftsteller, Journalist}|pwk} wegen dieses Prozesses in Wien\oindex{Wien@\textbf{Wien}|pwk} weilte, ist unklar. Es dürfte sich aber um eine Seitenaffäre einer
                  sehr beachteten Ankage gegen Graf Gisbert von
                     Wolff-Metternich\pwindex{Wolff-Metternich, Gisbert von *~1885@\textsc{Wolff-Metternich, Gisbert von} (*~1885), \emph{Schauspieler}|pwk} handeln. Dieser war bereits mehrfach des Betrugs
                  beschuldigt worden und hatte hohe Schulden. Einen Prozess wollte er abwehren,
                  indem er angab, die wohlhabende Dolly
                     Landsberger\pwindex{Landsberger, Dolly 1892-11-11 – 1923-04-15@\textsc{Landsberger, Dolly} (1892-11-11 – 1923-04-15)|pwk} zu heiraten. Deren Mutter Gertrud Wertheim\pwindex{Wertheim, Gertrud 1867-12-05 – 1927-11-27@\textsc{Wertheim, Gertrud} (1867-12-05 – 1927-11-27), \emph{Schriftstellerin}|pwk} (bekannt als Schriftstellerin Truth\pwindex{Wertheim, Gertrud 1867-12-05 – 1927-11-27@\textsc{Wertheim, Gertrud} (1867-12-05 – 1927-11-27), \emph{Schriftstellerin}|pwk}) trat als Belastungszeugin gegen ihn auf, wodurch
                  auch ihr »Vorleben« als Dichterin und das Erbe aus erster Ehe für die Verteidigung
                  und in Folge die Klatschpresse interessant wurde. Goldmann\pwindex{Goldmann, Paul 31.01.1865 – 25.09.1935@\textsc{Goldmann, Paul} (31.01.1865 – 25.09.1935), \emph{Schriftsteller, Journalist}|pwk} dürfte sich in Bozen\oindex{Bozen@\textbf{Bozen}|pwk} dafür stark gemacht haben, dass weder Landsberger\pwindex{Landsberger, Dolly 1892-11-11 – 1923-04-15@\textsc{Landsberger, Dolly} (1892-11-11 – 1923-04-15)|pwk} noch ihre Mutter\pwindex{Wertheim, Gertrud 1867-12-05 – 1927-11-27@\textsc{Wertheim, Gertrud} (1867-12-05 – 1927-11-27), \emph{Schriftstellerin}|pwkv} an dem Prozess teilnahmen: »Frau \so{Wertheim}\pwindex{Wertheim, Gertrud 1867-12-05 – 1927-11-27@\textsc{Wertheim, Gertrud} (1867-12-05 – 1927-11-27), \emph{Schriftstellerin}|pw} hat der ›Voss. Ztg.\orgindex{Vossische Zeitung@Vossische Zeitung|pw}‹ einen Brief\pwindex{Wertheim, Gertrud 1867-12-05 – 1927-11-27@\textsc{Wertheim, Gertrud} (1867-12-05 – 1927-11-27), \emph{Schriftstellerin}!Leserbrief von Gertrud Wertheim mit Klage ueber Paul Goldmann]Oktober 1911@\strich\emph{[Leserbrief von Gertrud Wertheim mit Klage über Paul Goldmann]} {[}Oktober 1911{]}|pwv} zugeſchickt, in
                     dem ſie gegen den Berlin\oindex{Berlin@\textbf{Berlin}|pw}er Schriftſteller
                        \so{Goldmann}\pwindex{Goldmann, Paul 31.01.1865 – 25.09.1935@\textsc{Goldmann, Paul} (31.01.1865 – 25.09.1935), \emph{Schriftsteller, Journalist}|pw} den Vorwurf erhebt, ſich in eine ihm ganz fremde Angelegenheit unbefugt
                     eingemengt zu haben. Er habe in Bozen\oindex{Bozen@\textbf{Bozen}|pw}
                     ihren Mann\pwindex{Wertheim, Wolf Waldemar 1867-03-23 – 1940@\textsc{Wertheim, Wolf Waldemar} (1867-03-23 – 1940), \emph{Warenhausbesitzer}|pwv}, ihre Tochter\pwindex{Landsberger, Dolly 1892-11-11 – 1923-04-15@\textsc{Landsberger, Dolly} (1892-11-11 – 1923-04-15)|pwv} und ſie ſelbſt
                     in der intenſivſten Art und Weiſe beſchworen, daß Frau Wertheim\pwindex{Wertheim, Gertrud 1867-12-05 – 1927-11-27@\textsc{Wertheim, Gertrud} (1867-12-05 – 1927-11-27), \emph{Schriftstellerin}|pw} nicht zum Metternich\pwindex{Wolff-Metternich, Gisbert von *~1885@\textsc{Wolff-Metternich, Gisbert von} (*~1885), \emph{Schauspieler}|pw}-Prozeß fahre. ›Er malte jedem einzelnen‹, heißt es in dem
                        Brief\pwindex{Wertheim, Gertrud 1867-12-05 – 1927-11-27@\textsc{Wertheim, Gertrud} (1867-12-05 – 1927-11-27), \emph{Schriftstellerin}!Leserbrief von Gertrud Wertheim mit Klage ueber Paul Goldmann]Oktober 1911@\strich\emph{[Leserbrief von Gertrud Wertheim mit Klage über Paul Goldmann]} {[}Oktober 1911{]}|pwv}, ›in den
                     düſterſten Farben mein bevorſtehendes Geſchick aus und ſagte weiter, aus der
                        Zeugin\pwindex{Wertheim, Gertrud 1867-12-05 – 1927-11-27@\textsc{Wertheim, Gertrud} (1867-12-05 – 1927-11-27), \emph{Schriftstellerin}|pwv} würde eine
                     Angeklagte werden. Eine Kataſtrophe würde eintreten. Da wir alle Herrn Paul Goldmann\pwindex{Goldmann, Paul 31.01.1865 – 25.09.1935@\textsc{Goldmann, Paul} (31.01.1865 – 25.09.1935), \emph{Schriftsteller, Journalist}|pw} nur ganz flüchtig kennen,
                     erregte ſeine Art und Weiſe begreifliche Verwunderung. Er gab mir ſogar den
                     Rat, mich, die ich damals noch heiter und vergnügt war, durch ärztliche Atteſte
                     zu ſchützen. Dieſes gewiß befremdende Benehmen konnte ich mir nur dadurch
                     erklären, daß Herr Goldmann\pwindex{Goldmann, Paul 31.01.1865 – 25.09.1935@\textsc{Goldmann, Paul} (31.01.1865 – 25.09.1935), \emph{Schriftsteller, Journalist}|pw} von
                     irgendeiner Seite beauftragt war. Denn ein derartiges Eingreifen würde
                     höchſtens bei Freunden oder ſonſt Nächſtſtehenden zu erklären oder zu
                     entſchuldigen sein.{[}‹{]}\pwindex{?? Werk@Nicht ermittelte Verfasserinnen und Verfasser!Prozess gegen den Grafen Wolff-Metternich. Eine Erklaerung der Frau Wertheim1911-10-10@\emph{Der Prozeß gegen den Grafen Wolff-Metternich. Eine Erklärung der Frau Wertheim} {[}1911-10-10{]}|pwv}« ([O. V.]: \emph{Der Prozeß gegen den
                        Grafen Wolff-Metternich. Eine Erklärung der Frau Wertheim}\pwindex{?? Werk@Nicht ermittelte Verfasserinnen und Verfasser!Prozess gegen den Grafen Wolff-Metternich. Eine Erklaerung der Frau Wertheim1911-10-10@\emph{Der Prozeß gegen den Grafen Wolff-Metternich. Eine Erklärung der Frau Wertheim} {[}1911-10-10{]}|pwk}. In: \emph{Neues Wiener Journal}\pwindex{Neues Wiener Journal1893 – 1939@\emph{Neues Wiener Journal} {[}1893 – 1939{]}|pwk}, Jg. 19, Nr. 6.454,
                        10. 10. 1911, S. 9)}}}\label{K_L03540-5h}.\pend
           \pstart
           Mit den besten Wünschen für {\pb}Sie u. die Ihren
               {\\[\baselineskip]}Ihre {\\[\baselineskip]}\spacefill\mbox{EvaMarieGoldmann.}\pend
           \leftskip=0em{}
         
         \endnumbering\mylabel{h}\end{ledgroupsized}  \newcommand{\dateiname}{L03540}\newcommand{\titel}{Eva Marie Goldmann an Arthur Schnitzler, 21. 9. 1911}\newcommand{\editorInnen}{Martin Anton Müller und Laura Untner}%% latex-leseansicht-abspann.tex
%% Abspann für die Leseansicht.
%% Der Schalter \ifkorrekturansicht ist bereits durch den Vorspann gesetzt.

%% latex-abspann.tex
%% Gemeinsamer Abspann für Korrekturansicht und Leseansicht.
%% Setzt den Schalter \ifkorrekturansicht voraus (gesetzt in den
%% einbindenden Dateien latex-korrekturansicht-abspann.tex bzw.
%% latex-leseansicht-abspann.tex).
%% ---------------------------------------------------------------

\normalsize

% Das esempio-Environment wird nur in der Leseansicht benötigt
\ifkorrekturansicht\else
\newenvironment{esempio}[3]%
{
    \vspace{1.5ex}
    \rlap{\underline{#1}}
    \par
    \setlength{\parindent}{0cm}
    \nopagebreak
    \leftskip=#2cm
    \rightskip=#3cm
}
{
    \par
}
\fi

\doendnotes{C}
\bigskip
\vfill

\clearpage

\footnotesize

\ifkorrekturansicht
  \lohead{\textsc{register}}
\fi

% theindex-Environment neu definieren ohne reledmac
\makeatletter
\renewenvironment{theindex}{%
  \ifkorrekturansicht
    \section*{\indexname}%
  \else
    \subsubsection*{Index der erwähnten Entitäten}%
  \fi
  \setlength{\parindent}{0pt}%
  \setlength{\parskip}{0pt plus 0.3pt}%
  \let\item\@idxitem
}{%
  \ifkorrekturansicht\clearpage\fi
}
\makeatother

\IfFileExists{\jobname-pw.ind}{\input{\jobname-pw.ind}}{}

% Quellenangabe nur in der Leseansicht
\ifkorrekturansicht\else
% Fallback-Definitionen, falls die .tex-Datei \titel etc. nicht gesetzt hat
\providecommand{\titel}{}
\providecommand{\editorInnen}{}
\providecommand{\dateiname}{\jobname}

\vspace{3cm}

\vfill

\footnotesize
\textsc{Quelle}: \titel. Herausgegeben von {\editorInnen}. In: \emph{Arthur Schnitzler: Briefwechsel mit Autorinnen und Autoren}.
 Digitale Edition, https://schnitzler-briefe.acdh.oeaw.ac.at/{\dateiname}.html (Stand \today)
\fi

\end{document}


      