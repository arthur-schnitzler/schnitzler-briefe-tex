%% latex-korrekturansicht-vorspann.tex
%% Vorspann für die Korrekturansicht.
%% Lädt die gemeinsame Datei latex-vorspann.tex mit gesetztem Schalter.

\newif\ifkorrekturansicht
\korrekturansichttrue

\input{../tex-inputs/latex-vorspann}


\section[Georg Engländer an Arthur Schnitzler, 18. 4. 1913]{L02123 Georg Engländer an Arthur Schnitzler, 18. 4. 1913}
\nopagebreak\mylabel{L02123v}
\rehead{ }\normalsize\beginnumbering\briefempfaengerindex{Schnitzler, Arthur@\textsc{Schnitzler, Arthur}!zzzEnglaender, Georg@\emph{von Georg Engländer}!1913-04-182@{18. 4. 1913}|(be}
\toendnotes[C]{\smallbreak\pagebreak[2]}\Standort{DLA, A:Schnitzler, HS.NZ85.1.2889.}
\physDesc{Brief, 2 Blätter, 8 Seiten, 2673 Zeichen (mit lateinischen Zahlen nummeriert)
\newline{}Handschrift: schwarze Tinte, deutsche Kurrent
\newline{}Schnitzler: 1) mit Bleistift beschriftet: »\textsc{G. Engländer}«  2) mit rotem Buntstift eine Unterstreichung und Markierungen}\toendnotes[C]{\smallbreak}
\pstart
           \raggedleft{}{\pb}\textsc{Freitag} d. 18/IV 1913.\pend
           
\pstart{}Hochgeehrter Herr!\pend\vspace{0.5em}
\pstart
           Vielen innigſten Dank für Ihre ſo liebe Theilnahme an meines Bruder\pwindex{Altenberg, Peter 09.03.1859 – 08.01.1919@\textsc{Altenberg, Peter} (09.03.1859 – 08.01.1919), \emph{Schriftsteller/Schriftstellerin}|pwv}\textsuperscript{s} Geſchick. Ich antworte erſt heute, da ich nach geſtern
               eine erſchöpfende Ausſprache mit dem \textsc{Primarius}\pwindex{Richter, Karl 09.03.1862 – 25.06.1937@\textsc{Richter, Karl} (09.03.1862 – 25.06.1937), \emph{Mediziner/Medizinerin, Sanatoriumsleiter/Sanatoriumsleiterin}|pwv} des \textsc{Sanat. Steinhof}\oindex{Otto-Wagner-Spital@\textbf{Otto-Wagner-Spital}, \emph{Krankenhaus (K.KKH)}|pw} vor hatte {\kaufmannsund} Ihnen darüber Bericht geben
               wollte.\pend
           
\pstart
           \textsc{Primarius} D\textsuperscript{r}{ }\textsc{Richter}\pwindex{Richter, Karl 09.03.1862 – 25.06.1937@\textsc{Richter, Karl} (09.03.1862 – 25.06.1937), \emph{Mediziner/Medizinerin, Sanatoriumsleiter/Sanatoriumsleiterin}|pw} hält \textsc{Peter}\pwindex{Altenberg, Peter 09.03.1859 – 08.01.1919@\textsc{Altenberg, Peter} (09.03.1859 – 08.01.1919), \emph{Schriftsteller/Schriftstellerin}|pw} für entlaſ{\pb}ſungsmöglich; ſo bald \uline{ich} damit einverſtanden, \uline{der}{ }\textsc{Peter}\pwindex{Altenberg, Peter 09.03.1859 – 08.01.1919@\textsc{Altenberg, Peter} (09.03.1859 – 08.01.1919), \emph{Schriftsteller/Schriftstellerin}|pw} hingebracht, iſt er ſofort freigegeben.\pend
           
\pstart
           Die Schwierigkeit liegt aber wo anders.\pend
           
\pstart
           Bier dürfte er höchſtens Abends ein kleines Glas trinken, eigentlich gar keines, denn
               seit 10 \textsc{December} 1912 erhielt er keinen \textsc{Tropfen Alkohol} mehr, er iſt in der \textsc{Anstalt} zu \textsc{Paraldehyd} als \textsc{Schlafmittel} gewöhnt {\pb}worden, auch daher darf
               ihm nichts ausgefolgt werden er müßte es unter ſtrenger \textsc{Aufsicht} regelmäßig \textsc{dosirt} erhalten. Sie kennen ja
                  \textsc{Peter}\pwindex{Altenberg, Peter 09.03.1859 – 08.01.1919@\textsc{Altenberg, Peter} (09.03.1859 – 08.01.1919), \emph{Schriftsteller/Schriftstellerin}|pw}, ſein \textsc{Freiheitsdrang} geht ja nur dahin, ſich \textsc{Auszutollen}, dann, mit was i{\geminationm}er, \textsc{wahllos} ſich \textsc{Schlaf} oder
                  \textsc{Betäubung} verſchaffen.\pend
           
\pstart
           Anders kann er ja in \textsc{Freiheit} wieder nicht leben.\pend
           
\pstart
           Seine \textsc{Skizze}{ }\textsc{Nerven-Sanatorium}\pwindex{Sanatorium fuer Nervenkranke (aber nicht die, in denen ich mich befand)@\emph{Sanatorium für Nervenkranke (aber nicht die, in denen ich mich befand{\rufezeichen})}|pw} gibt uns \label{K_L02123-1v}\edtext{ein Bild}{\lemma{\textnormal{\emph{ein Bild}}}\Cendnote{\textnormal{In der Prosaskizze \emph{Sanatorium für Nervenkranke (aber nicht die, in denen ich mich
                     befand!)}\pwindex{Sanatorium fuer Nervenkranke (aber nicht die, in denen ich mich befand)@\emph{Sanatorium für Nervenkranke (aber nicht die, in denen ich mich befand{\rufezeichen})}|pwk} (\emph{Simplicissimus}\pwindex{Simplicissimus@\emph{Simplicissimus}|pwk}, Jg. 16, H. 41,
                        8. 2. 1912, S. 724) besticht das Alter Ego des Autors
                  einen Wärter, um an Alkohol zu kommen.}}}\label{K_L02123-1}, wie er es in freieren \textsc{An{\pb}stalten} treibt, damals war
               er in der \textsc{Sulz}\oindex{Sulz im Wienerwald@\textbf{Sulz im Wienerwald}, \emph{P.PPL}|pw}; da am \textsc{Steinhof}\oindex{Otto-Wagner-Spital@\textbf{Otto-Wagner-Spital}, \emph{Krankenhaus (K.KKH)}|pw} bin ich \textsc{sicher} daſs kein \textsc{Unfug}, weder mit \textsc{Alkohol} noch mit \textsc{Schlafmitteln} getrieben werden kann {\kaufmannsund} ſein \textcolor{gray}{Cerebral}zuſtand iſt noch ſo
               unruhig ſo \textsc{aufgeregt}{ }{\kaufmannsund}{ }\textsc{unstet}, daſs ich erſt da eine \textsc{Beſſerung}{ }{\kaufmannsund}{ }\textsc{Beruhigung} abwarten möchte.\pend
           
\pstart
           Es wäre denn, daſs thatſächlich eine \textsc{Garantie}, darunter
               meine ich aber nicht \textsc{Versprechungen} od. \textsc{Versicherungen}{ }\textsc{Peter}\pwindex{Altenberg, Peter 09.03.1859 – 08.01.1919@\textsc{Altenberg, Peter} (09.03.1859 – 08.01.1919), \emph{Schriftsteller/Schriftstellerin}|pw}\textsuperscript{s}, geſchaffen werden {\pb}könnte, ſondern wirklich eine \textsc{Sicherheit}, daſs \textsc{Peter}\pwindex{Altenberg, Peter 09.03.1859 – 08.01.1919@\textsc{Altenberg, Peter} (09.03.1859 – 08.01.1919), \emph{Schriftsteller/Schriftstellerin}|pw} zumindeſt nach 4–6 Wochen wol frei ſei, aber punkto \textsc{Alkohol}{ }{\kaufmannsund}{ }\textsc{Schlafmittel} unter \uuline{\edtext{ſtrengſter}{\Cendnote{dreifach unterstrichen}}}{ }\textsc{Aufsicht}.\pend
           
\pstart
           Gewiß wäre dies das \textsc{Ideal}, da ich mich ja nicht darüber
               täuſche, daſs ſeine Erregung über die ihm vorenthaltene \textsc{Freiheit}, jetzt gewiß auch ungünſtig auf ſeine \textsc{Nerven} einwirkt.\pend
           
\pstart
           Aber lieber noch dieſer {\pb}Nachtheil, als das andere {\kaufmannsund} gewiß größere \textsc{risico} eines
               neuerlichen \textsc{Verfalles}!\pend
           
\pstart
           Der \textsc{financielle Punkt} den er Ihnen gegenüber erwähnte, iſt
               völlig aufgeklärt; von ſeiner \textsc{Seite} ein \textsc{Versehen}, für das er nichts kann. In meiner \textsc{Rechnungführung} fand ich Ihren w. Namen nicht vor {\kaufmannsund} als er mich darum fragte, ſagte ich nein, von D\textsuperscript{r}{ }\textsc{Schnitzler}{ }{\pb}iſt nichts eingelaufen, da ich ja monatlich \uline{von \textsc{S. Fischer}\orgindex{S. Fischer Verlag@S. Fischer Verlag|pw}} c\textsuperscript{a} 100 K zugeſandt erhielt aber nicht wußte daſs
               dieſe mit dieſer \textsc{Sa{\geminationm}lung
                  identisch} ſeien, was ich ihm alſo \textsc{Sonntag} aufklären werde {\kaufmannsund} Sie hiemit frdl. entſchuldigen
               wollen.\pend
           
\pstart
           \textsc{Peter}\pwindex{Altenberg, Peter 09.03.1859 – 08.01.1919@\textsc{Altenberg, Peter} (09.03.1859 – 08.01.1919), \emph{Schriftsteller/Schriftstellerin}|pw} kann täglich ab 2 \textsc{Uhr} beſucht werden, übrigens auch
               in den \textsc{Vormittags-Stunden}, die \textsc{Ärzte} dort aber {\pb}treffen Sie nur zwiſchen 2
                  {\kaufmannsund} 4 \textsc{Uhr} an; dem Herrn \textsc{Primarius}{ }Richter\pwindex{Richter, Karl 09.03.1862 – 25.06.1937@\textsc{Richter, Karl} (09.03.1862 – 25.06.1937), \emph{Mediziner/Medizinerin, Sanatoriumsleiter/Sanatoriumsleiterin}|pw} habe ich von Ihrem vorausſichtlichen
               Beſuch u. Rückſprache mit ihm Meldung erſtattet.\pend
           
\pstart
           Für Ihre wirklich herzlich ſchöne \textsc{Absicht} mitzuhelfen
               wiederholten innigſten Dank.\pend
           
\pstart
           von Ihren Sie hochſchätzenden{\\[\baselineskip]}Ergebenſten{\\[\baselineskip]}\spacefill\mbox{G. Engländer.}\pend
           \leftskip=0em{}
\pstart
           \noindent{}\textsc{III Seidlgasse 23\oindex{Seidlgasse@\textbf{Seidlgasse}, \emph{Straße (K.STR)}|pw}.}\pend
           \selectlanguage{ngerman}\endnumbering\briefempfaengerindex{Schnitzler, Arthur@\textsc{Schnitzler, Arthur}!zzzEnglaender, Georg@\emph{von Georg Engländer}!1913-04-182@{18. 4. 1913}|)be}\mylabel{L02123h}  \normalsize

\doendnotes{C}
\bigskip
\vfill

\clearpage

\footnotesize

\lohead{\textsc{register}}

% Definiere theindex-Environment komplett neu ohne reledmac
\makeatletter
\renewenvironment{theindex}{%
  \section*{\indexname}%
  \setlength{\parindent}{0pt}%
  \setlength{\parskip}{0pt plus 0.3pt}%
  \let\item\@idxitem
}{%
  \clearpage
}
\makeatother

\IfFileExists{\jobname-pw.ind}{\input{\jobname-pw.ind}}{}

\end{document}

      