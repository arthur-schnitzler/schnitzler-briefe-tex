%% latex-leseansicht-vorspann.tex
%% Vorspann für die Leseansicht.
%% Lädt die gemeinsame Datei latex-vorspann.tex mit nicht gesetztem Schalter.

\newif\ifkorrekturansicht
\korrekturansichtfalse

\input{../tex-inputs/latex-vorspann}


\section[ Felix Salten an Arthur Schnitzler, [3. 12.? 1902]]{L03337 Felix Salten an Arthur Schnitzler,  [3. 12.? 1902]}
\nopagebreak\mylabel{L03337v}
\rehead{ }\normalsize\beginnumbering\briefempfaengerindex{Schnitzler, Arthur@\textsc{Schnitzler, Arthur}!zzzSalten, Felix@\emph{von Felix Salten}!1902-12-031@{{[}3. 12.? 1902{]}}|(be}
\toendnotes[C]{\smallbreak\pagebreak[2]}
\correspDesc{Versand  durch Felix Salten am [3. 12.? 1902] in Wien
\newline{}Erhalt  durch Arthur Schnitzler im Zeitraum [3. 12. 1902
                  – 4. 12. 1902?] in Wien}\toendnotes[C]{\smallbreak}
\Standort{CUL, Schnitzler, B 89, A 2.}
\physDesc{Brief, 1 Blatt, 1 Seite, 527 Zeichen
\newline{}Handschrift: Bleistift, lateinische Kurrent
\newline{}Schnitzler: mit Bleistift auf »Nov 902« datiert und dem Vermerk »\textsc{Salten}« versehen 
\newline{}Ordnung: mit Bleistift von unbekannter Hand nummeriert: »162« }\toendnotes[C]{\smallbreak}
\pstart
           \centering{}{\pb}\uline{\label{K_L03337-1v}\edtext{Mittwoch}{\lemma{\textnormal{\emph{Mittwoch}}}\Cendnote{\textnormal{Folgt man der Datierung Schnitzlers, müsste das
                        Korrespondenzstück an einem der vier Mittwoche im November 1902 verfasst worden sein. Im \emph{Tagebuch}\pwindex{Schnitzler, Arthur 15.\,5.\,1862 Wien – 21.\,10.\,1931 ebd.@\textsc{Schnitzler, Arthur} (15.\,5.\,1862 Wien – 21.\,10.\,1931 ebd.), \emph{Schriftsteller, Mediziner}!Tagebuch@\strich\emph{Tagebuch}|pwk} wird Salten\pwindex{Salten, Felix 6.\,9.\,1869 Budapest – 8.\,10.\,1945 Zürich@\textsc{Salten, Felix} (6.\,9.\,1869 Budapest – 8.\,10.\,1945 Zürich), \emph{Schriftsteller, Journalist, Chefredakteur}|pwk} im November 1902 nicht erwähnt.
                        Für den 4. 12. 1902 – einen Donnerstag – ist hingegen ein Treffen mit
                           Salten\pwindex{Salten, Felix 6.\,9.\,1869 Budapest – 8.\,10.\,1945 Zürich@\textsc{Salten, Felix} (6.\,9.\,1869 Budapest – 8.\,10.\,1945 Zürich), \emph{Schriftsteller, Journalist, Chefredakteur}|pwk} in einem Kaffeehaus
                        vermerkt, bei dem es um die Reise Saltens\pwindex{Salten, Felix 6.\,9.\,1869 Budapest – 8.\,10.\,1945 Zürich@\textsc{Salten, Felix} (6.\,9.\,1869 Budapest – 8.\,10.\,1945 Zürich), \emph{Schriftsteller, Journalist, Chefredakteur}|pwk} zur Uraufführung von \emph{Der
                           Gemeine}\pwindex{Salten, Felix 6.\,9.\,1869 Budapest – 8.\,10.\,1945 Zürich@\textsc{Salten, Felix} (6.\,9.\,1869 Budapest – 8.\,10.\,1945 Zürich), \emph{Schriftsteller, Journalist, Chefredakteur}!Gemeine. Schauspiel in drei Aufzügen@\strich\emph{Der Gemeine. Schauspiel in drei Aufzügen}|pwk} nach Berlin\oindex{Berlin@\textbf{Berlin}, \emph{Hauptstadt}|pwk} ging. Diese
                        hatte am 25. 11. 1902 am \emph{Kleinen Theater}\orgindex{Kleines Theater@Kleines Theater|pwk} stattgefunden, einem Dienstag. Da
                        eine Rückkehr nach Wien\oindex{Wien@\textbf{Wien}, \emph{Verwaltungsgebiet}|pwk} am Folgetag
                        unwahrscheinlich ist, kann das Korrespondenzstück auf den nächsten Mittwoch nach der Uraufführung datiert
                        werden.}}}\label{K_L03337-1}.}\pend
           \vspace{0.5em}
\pstart
           Lieber Freund, seit gestern bin ich
               wieder da, und möchte Sie sehr gerne bald sehen. Hätten Sie morgen, Donnerstag{ }Abds, um 10\textcolor{gray}{,} Lust in den Kaiserhof\oindex{Wien@\textbf{Wien}!I., Innere Stadt@\textbf{I., Innere Stadt}!Café Kaiserhof (Inh. Johann Wortner) [Wien]@\textbf{Café Kaiserhof (Inh. Johann Wortner) [Wien]}, \emph{Kaffeehaus}|pw} zu
               kommen? Mir ist es über Erwarten, wie über Verdienst gut gegangen, nur war ich durch
               die verschiedensten Dinge so gehetzt und absorbirt, dass ich außer Depeschen nichts
               schrieb.\pend
           
\pstart
           Entschuldigen Sie mein\strikeout{e} Schweigen, – Sie werden es
               gewiß {[}verstehen{]}, wenn ich Ihnen einiges erzähle. Wenn Sie mir
               nicht abschreiben, bin ich Donnerstag{ }Abds d. i. also morgen im Café\oindex{Wien@\textbf{Wien}!I., Innere Stadt@\textbf{I., Innere Stadt}!Café Kaiserhof (Inh. Johann Wortner) [Wien]@\textbf{Café Kaiserhof (Inh. Johann Wortner) [Wien]}, \emph{Kaffeehaus}|pwv}\textcolor{gray}{.}\pend
           
\pstart
           herzlichst Ihr {\\[\baselineskip]}\spacefill\mbox{Salten}\pend
           \leftskip=0em{}\selectlanguage{ngerman}\endnumbering\briefempfaengerindex{Schnitzler, Arthur@\textsc{Schnitzler, Arthur}!zzzSalten, Felix@\emph{von Felix Salten}!1902-12-031@{{[}3. 12.? 1902{]}}|)be}\mylabel{L03337h}  \newcommand{\dateiname}{L03337}\newcommand{\titel}{Felix Salten an Arthur Schnitzler, [3. 12.? 1902]}\newcommand{\editorInnen}{Martin Anton Müller und Laura Untner}%% latex-leseansicht-abspann.tex
%% Abspann für die Leseansicht.
%% Der Schalter \ifkorrekturansicht ist bereits durch den Vorspann gesetzt.

%% latex-abspann.tex
%% Gemeinsamer Abspann für Korrekturansicht und Leseansicht.
%% Setzt den Schalter \ifkorrekturansicht voraus (gesetzt in den
%% einbindenden Dateien latex-korrekturansicht-abspann.tex bzw.
%% latex-leseansicht-abspann.tex).
%% ---------------------------------------------------------------

\normalsize

% Das esempio-Environment wird nur in der Leseansicht benötigt
\ifkorrekturansicht\else
\newenvironment{esempio}[3]%
{
    \vspace{1.5ex}
    \rlap{\underline{#1}}
    \par
    \setlength{\parindent}{0cm}
    \nopagebreak
    \leftskip=#2cm
    \rightskip=#3cm
}
{
    \par
}
\fi

\doendnotes{C}
\bigskip
\vfill

\clearpage

\footnotesize

\ifkorrekturansicht
  \lohead{\textsc{register}}
\fi

% theindex-Environment neu definieren ohne reledmac
\makeatletter
\renewenvironment{theindex}{%
  \ifkorrekturansicht
    \section*{\indexname}%
  \else
    \subsubsection*{Index der erwähnten Entitäten}%
  \fi
  \setlength{\parindent}{0pt}%
  \setlength{\parskip}{0pt plus 0.3pt}%
  \let\item\@idxitem
}{%
  \ifkorrekturansicht\clearpage\fi
}
\makeatother

\IfFileExists{\jobname-pw.ind}{\input{\jobname-pw.ind}}{}

% Quellenangabe nur in der Leseansicht
\ifkorrekturansicht\else
% Fallback-Definitionen, falls die .tex-Datei \titel etc. nicht gesetzt hat
\providecommand{\titel}{}
\providecommand{\editorInnen}{}
\providecommand{\dateiname}{\jobname}

\vspace{3cm}

\vfill

\footnotesize
\textsc{Quelle}: \titel. Herausgegeben von {\editorInnen}. In: \emph{Arthur Schnitzler: Briefwechsel mit Autorinnen und Autoren}.
 Digitale Edition, https://schnitzler-briefe.acdh.oeaw.ac.at/{\dateiname}.html (Stand \today)
\fi

\end{document}


