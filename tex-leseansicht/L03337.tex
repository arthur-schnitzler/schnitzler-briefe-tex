%% latex-leseansicht-vorspann.tex
%% Vorspann für die Leseansicht.
%% Lädt die gemeinsame Datei latex-vorspann.tex mit nicht gesetztem Schalter.

\newif\ifkorrekturansicht
\korrekturansichtfalse

\input{../tex-inputs/latex-vorspann}

\begin{center}
            \textcolor{red}{ENTWURF, NICHT FERTIG KORRIGIERT}
                      \end{center}
            
         \renewcommand{\erwaehnteInstitutionen}{Institutionen: Kleines Theater}
         \renewcommand{\erwaehnteOrte}{Orte: Berlin, Café Kaiserhof (Inh. Johann Wortner), Wien}
         \renewcommand{\erwaehnteWerke}{Werke: Der Gemeine. Schauspiel in drei Aufzügen, Tagebuch}
               \section[Felix Salten an Arthur Schnitzler, {[}3. 12.? 1902{]}]{ Felix Salten an Arthur Schnitzler, {[}3. 12.? 1902{]}}\nopagebreak\mylabel{v}\rehead{ }\begin{ledgroupsized}[t]{13cm}\normalsize\beginnumbering \toendnotes[C]{\smallbreak\pagebreak[2]} \Standort{CUL, Schnitzler, B 89, A 2.}
\physDesc{Brief, 1 Blatt, 1 Seite
\newline{}Handschrift: Bleistift, lateinische Kurrent
\newline{}Schnitzler: mit Bleistift datiert: »Nov 902« und Vermerk: »\textsc{Salten}« \newline{}Ordnung: mit Bleistift von unbekannter Hand nummeriert:
                                    »162« }\toendnotes[C]{\smallbreak}\pstart
           \centering{}{\pb}\uline{\label{K_L03337-1v}\edtext{Mittwoch}{\lemma{\textnormal{\emph{Mittwoch}}}\Cendnote{\textnormal{Folgt man der Datierung von Schnitzler\pwindex{Schnitzler, Arthur 15.05.1862 – 21.10.1931@\textsc{Schnitzler, Arthur} (15.05.1862 – 21.10.1931), \emph{Schriftsteller, Mediziner}|pwk}s, würde das
                        Korrespondenzstück an einem der vier Mittwoche im November 1902
                        verfasst sein. Im \emph{Tagebuch}\pwindex{Schnitzler, Arthur 15.05.1862 – 21.10.1931@\textsc{Schnitzler, Arthur} (15.05.1862 – 21.10.1931), \emph{Schriftsteller, Mediziner}!Tagebuch1981 – 2000@\strich\emph{Tagebuch} {[}1981 – 2000{]}|pwk} wird Salten\pwindex{Salten, Felix 06.09.1869 – 08.10.1945@\textsc{Salten, Felix} (06.09.1869 – 08.10.1945), \emph{Schriftsteller, Journalist}|pwk} im November 1902
                        nicht erwähnt. Am 4. 12. 1902 – einem Donnerstag – ist hingegen ein Treffen
                        erwähnt, bei der es um die Reise Salten\pwindex{Salten, Felix 06.09.1869 – 08.10.1945@\textsc{Salten, Felix} (06.09.1869 – 08.10.1945), \emph{Schriftsteller, Journalist}|pwk}s zur Uraufführung von \emph{Der
                           Gemeine}\pwindex{Salten, Felix 06.09.1869 – 08.10.1945@\textsc{Salten, Felix} (06.09.1869 – 08.10.1945), \emph{Schriftsteller, Journalist}!Gemeine. Schauspiel in drei Aufzuegen1901@\strich\emph{Der Gemeine. Schauspiel in drei Aufzügen} {[}1901{]}|pwk} nach Berlin\oindex{Berlin@\textbf{Berlin}|pwk} geht. Diese
                        hatte am 25. 11. 1902 am \emph{Kleinen
                           Theater}\orgindex{Kleines Theater@Kleines Theater|pwk} stattgefunden, einem Dienstag. Dadurch ist eine
                        Rückkehr nach Wien\oindex{Wien@\textbf{Wien}|pwk} am Folgetag
                        unwahrscheinlich, so dass das Korrespondenzstück auf den nächsten Mittwoch
                        nach der Uraufführung datiert werden kann.}}}\label{K_L03337-1h}} . \pend
           \pstart
           Lieber Freund, seit gestern bin ich wieder da, und
               möchte Sie sehr gern bald sehen. Hätten Sie morgen, Donnerstag Abds, um
               10, Lust in den Kaiserhof\oindex{Cafe Kaiserhof (Inh. Johann Wortner)@\textbf{Café Kaiserhof (Inh. Johann Wortner)}|pw} zu kommen? Mir ist es
               über Erwarten, weit über Verdienst gut gegangen, nur war ich durch die
               verschiedensten Dinge so gehetzt und absorbirt, dass ich außer Depeschen nichts
               schrieb. \pend
           \pstart
           Entschuldigen Sie mein\strikeout{e} Schweigen, – Sie werden es
               gewiß, wenn ich Ihnen einiges erzähle. Wenn Sie mir nicht abschreiben, bin ich
               Donnerstag Abds d. i. also morgen im Café, \pend
           \pstart herzlichst Ihr \spacefill\mbox{Salten}\pend{}
         
         \endnumbering\mylabel{h}\end{ledgroupsized}\begin{anhang}\end{anhang}\newcommand{\dateiname}{L03337}\newcommand{\titel}{Felix Salten an Arthur Schnitzler, [3. 12.? 1902]}\newcommand{\editorInnen}{Martin Anton Müller und Laura Untner}%% latex-leseansicht-abspann.tex
%% Abspann für die Leseansicht.
%% Der Schalter \ifkorrekturansicht ist bereits durch den Vorspann gesetzt.

%% latex-abspann.tex
%% Gemeinsamer Abspann für Korrekturansicht und Leseansicht.
%% Setzt den Schalter \ifkorrekturansicht voraus (gesetzt in den
%% einbindenden Dateien latex-korrekturansicht-abspann.tex bzw.
%% latex-leseansicht-abspann.tex).
%% ---------------------------------------------------------------

\normalsize

% Das esempio-Environment wird nur in der Leseansicht benötigt
\ifkorrekturansicht\else
\newenvironment{esempio}[3]%
{
    \vspace{1.5ex}
    \rlap{\underline{#1}}
    \par
    \setlength{\parindent}{0cm}
    \nopagebreak
    \leftskip=#2cm
    \rightskip=#3cm
}
{
    \par
}
\fi

\doendnotes{C}
\bigskip
\vfill

\clearpage

\footnotesize

\ifkorrekturansicht
  \lohead{\textsc{register}}
\fi

% theindex-Environment neu definieren ohne reledmac
\makeatletter
\renewenvironment{theindex}{%
  \ifkorrekturansicht
    \section*{\indexname}%
  \else
    \subsubsection*{Index der erwähnten Entitäten}%
  \fi
  \setlength{\parindent}{0pt}%
  \setlength{\parskip}{0pt plus 0.3pt}%
  \let\item\@idxitem
}{%
  \ifkorrekturansicht\clearpage\fi
}
\makeatother

\IfFileExists{\jobname-pw.ind}{\input{\jobname-pw.ind}}{}

% Quellenangabe nur in der Leseansicht
\ifkorrekturansicht\else
% Fallback-Definitionen, falls die .tex-Datei \titel etc. nicht gesetzt hat
\providecommand{\titel}{}
\providecommand{\editorInnen}{}
\providecommand{\dateiname}{\jobname}

\vspace{3cm}

\vfill

\footnotesize
\textsc{Quelle}: \titel. Herausgegeben von {\editorInnen}. In: \emph{Arthur Schnitzler: Briefwechsel mit Autorinnen und Autoren}.
 Digitale Edition, https://schnitzler-briefe.acdh.oeaw.ac.at/{\dateiname}.html (Stand \today)
\fi

\end{document}


      