%% latex-leseansicht-vorspann.tex
%% Vorspann für die Leseansicht.
%% Lädt die gemeinsame Datei latex-vorspann.tex mit nicht gesetztem Schalter.

\newif\ifkorrekturansicht
\korrekturansichtfalse

\input{../tex-inputs/latex-vorspann}


\section[ Paul Goldmann an Arthur Schnitzler, 12. 2. {[}1901{]}]{L03057 Paul Goldmann an Arthur Schnitzler,  12. 2. [1901]}
\nopagebreak\mylabel{L03057v}
\rehead{ }\normalsize\beginnumbering\briefempfaengerindex{Schnitzler, Arthur@\textsc{Schnitzler, Arthur}!zzzGoldmann, Paul@\emph{von Paul Goldmann}!1901-02-121@{12. 2. [1901]}|(be}
\toendnotes[C]{\smallbreak\pagebreak[2]}
\correspDesc{Versand  durch Paul Goldmann am 12. 2. [1901] in Berlin
\newline{}Erhalt  durch Arthur Schnitzler im Zeitraum [13. 2. 1901
                  – 17. 2. 1901?] in Wien}\toendnotes[C]{\smallbreak}
\Standort{DLA, A:Schnitzler, HS.NZ85.1.3171.}
\physDesc{Brief, 1 Blatt, 3 Seiten, 830 Zeichen
\newline{}Handschrift: blaue Tinte, deutsche Kurrent
\newline{}Beilage: ein Zeitungsausschnitt, beschnitten und aufgeklebt 
\newline{}Schnitzler: 1) mit Bleistift das Jahr »1901« vermerkt  2) mit rotem Buntstift drei Unterstreichungen}\toendnotes[C]{\smallbreak}
\pstart
           \raggedleft{}{\pb}\textcolor{gray}{\textbf{DESSAUERSTRASSE 19}}\oindex{Dessauer Straße@\textbf{Dessauer Straße}, \emph{Straße}|pw}\pend
           
\pstart
           Berlin\oindex{Berlin@\textbf{Berlin}, \emph{Hauptstadt}|pw}, 12. Februar.\pend
           
\pstart\center{}Mein lieber Freund,\pend\vspace{0.5em}
\pstart
           Wie gehts?\pend
           
\pstart
           Nach \label{K_L03057-1v}\edtext{Italien\oindex{Italien@\textbf{Italien}|pw}}{\lemma{\textnormal{\emph{Italien}}}\Cendnote{\textnormal{Schnitzler reiste zwischen 26. 3. 1901 und 18. 4. 1901 nach Genua\oindex{Genua@\textbf{Genua}|pwk}, Pisa\oindex{Pisa@\textbf{Pisa}, \emph{Hauptstadt}|pwk}, Rom\oindex{Rom@\textbf{Rom}, \emph{Hauptstadt}|pwk}, Florenz\oindex{Florenz@\textbf{Florenz}|pwk} und Bologna\oindex{Bologna@\textbf{Bologna}|pwk}.}}}\label{K_L03057-1} kann ich{ }ſelbſtverſtändlich nicht mitkommen. Aber es iſt{ }ſchön, daß Du hingehſt.\pend
           
\pstart
           Frau \textsc{Fulda}\pwindex{d’Albert, Ida 5.\,12.\,1869 Wien – 6.\,10.\,1926 Berlin@\textsc{d’Albert, Ida} (5.\,12.\,1869 Wien – 6.\,10.\,1926 Berlin), \emph{Schauspielerin}|pw} (welche ein geiſt- und herzloſes Weib iſt und mir immer weniger{ }ſympathiſch
               wird){ }ſuchte dieſer Tage aus mir herauszubekommen, ob Du in \label{K_L03057-2v}\edtext{weiblicher Geſellſchaft}{\lemma{\textnormal{\emph{weiblicher Gesellschaft}}}\Cendnote{\textnormal{Schnitzler reiste, abgesehen von seiner
                  Mutter Louise\pwindex{Schnitzler, Louise 8.\,7.\,1840 Kőszeg – 9.\,9.\,1911 Wien@\textsc{Schnitzler, Louise} (8.\,7.\,1840 Kőszeg – 9.\,9.\,1911 Wien)|pwk}, die am 11. 4. 1901 in Florenz\oindex{Florenz@\textbf{Florenz}|pwk} ankam, allein.}}}\label{K_L03057-2} nach Italien\oindex{Italien@\textbf{Italien}|pw} gehſt? Ich{ }ſagte: nein.\pend
           
\pstart
           {\pb}Was macht die \label{K_L03057-3v}\edtext{Rotheſterngaſſe\oindex{Wien@\textbf{Wien}!II., Leopoldstadt@\textbf{II., Leopoldstadt}!Rotensterngasse@\textbf{Rotensterngasse}, \emph{Straße}|pw}\pwindex{Schnitzler, Olga 17.\,1.\,1882 Wien – 13.\,1.\,1970 Lugano@\textsc{Schnitzler, Olga} (17.\,1.\,1882 Wien – 13.\,1.\,1970 Lugano), \emph{Schauspielerin, Sängerin}|pwv}\pwindex{Steinrück, Elisabeth 19.\,11.\,1885 – 7.\,4.\,1920 Partenkirchen@\textsc{Steinrück, Elisabeth} (19.\,11.\,1885 – 7.\,4.\,1920 Partenkirchen)|pwv}}{\lemma{\textnormal{\emph{Rothesterngasse}}}\Cendnote{\textnormal{Bezugnahme auf Schnitzlers nachmalige Ehefrau Olga\pwindex{Schnitzler, Olga 17.\,1.\,1882 Wien – 13.\,1.\,1970 Lugano@\textsc{Schnitzler, Olga} (17.\,1.\,1882 Wien – 13.\,1.\,1970 Lugano), \emph{Schauspielerin, Sängerin}|pwk} und ihre Schwester Elisabeth\pwindex{Steinrück, Elisabeth 19.\,11.\,1885 – 7.\,4.\,1920 Partenkirchen@\textsc{Steinrück, Elisabeth} (19.\,11.\,1885 – 7.\,4.\,1920 Partenkirchen)|pwk}, die in der Rotensterngasse\oindex{Wien@\textbf{Wien}!II., Leopoldstadt@\textbf{II., Leopoldstadt}!Rotensterngasse@\textbf{Rotensterngasse}, \emph{Straße}|pwk} wohnten}}}\label{K_L03057-3}?\pend
           
\pstart
           Bitte, lies \label{K_L03057-4v}\edtext{\textsc{Multatuli\pwindex{Dekker, Eduard Douwes 2.\,3.\,1820 Amsterdam – 19.\,2.\,1887 Ingelheim am Rhein@\textsc{Dekker, Eduard Douwes} (2.\,3.\,1820 Amsterdam – 19.\,2.\,1887 Ingelheim am Rhein), \emph{Schriftsteller}|pw}}}{\lemma{\textnormal{\emph{Multatuli}}}\Cendnote{\textnormal{Pseudonym des niederländ\oindex{Niederlande@\textbf{Niederlande}|pwkv}ischen Autors Eduard Douwes Dekker\pwindex{Dekker, Eduard Douwes 2.\,3.\,1820 Amsterdam – 19.\,2.\,1887 Ingelheim am Rhein@\textsc{Dekker, Eduard Douwes} (2.\,3.\,1820 Amsterdam – 19.\,2.\,1887 Ingelheim am Rhein), \emph{Schriftsteller}|pwk}; Lektüre mittels \emph{Tagebuch}\pwindex{Schnitzler, Arthur 15.\,5.\,1862 Wien – 21.\,10.\,1931 ebd.@\textsc{Schnitzler, Arthur} (15.\,5.\,1862 Wien – 21.\,10.\,1931 ebd.), \emph{Schriftsteller, Mediziner}!Tagebuch@\strich\emph{Tagebuch}|pwk} und Leseliste belegbar, vgl. A. S.: \emph{Lektüren}, Norden sowie A. S.: \emph{Tagebuch}, 28. 11. 1907, 30. 11. 1907, 12. 1. 1908, 26. 1. 1908.
               }}}\label{K_L03057-4}!\pend
           
\pstart
           \textsc{Richard\pwindex{Beer-Hofmann, Richard 11.\,7.\,1866 Wien – 26.\,9.\,1945 New York City@\textsc{Beer-Hofmann, Richard} (11.\,7.\,1866 Wien – 26.\,9.\,1945 New York City), \emph{Schriftsteller}|pw}} hat{ }ſich in der That nicht dazu aufſchwingen können, mir die \label{K_L03057-5v}\edtext{Geburt{ }ſeines Sohn\pwindex{Beer-Hofmann, Gabriel 9.\,1.\,1901 Wien – 24.\,3.\,1971 St Albans@\textsc{Beer-Hofmann, Gabriel} (9.\,1.\,1901 Wien – 24.\,3.\,1971 St Albans), \emph{Schriftsteller, Filmagent}|pwv}es}{\lemma{\textnormal{\emph{Geburt seines Sohnes}}}\Cendnote{\textnormal{Gabriel Beer-Hofmann\pwindex{Beer-Hofmann, Gabriel 9.\,1.\,1901 Wien – 24.\,3.\,1971 St Albans@\textsc{Beer-Hofmann, Gabriel} (9.\,1.\,1901 Wien – 24.\,3.\,1971 St Albans), \emph{Schriftsteller, Filmagent}|pwk} wurde am 9. 1. 1901 in Wien\oindex{Wien@\textbf{Wien}, \emph{Verwaltungsgebiet}|pwk}
                  geboren.}}}\label{K_L03057-5} anzuzeigen. Ich habe keine Worte mehr für dieſes Benehmen.
               Nichtsdeſtoweniger{ }ſchicke ich ihm die nachfolgende \label{K_L03057-6v}\edtext{Zeitungsnotiz\pwindex{Tagesneuigkeiten. Richtig@\emph{Tagesneuigkeiten. Richtig}|pwv}}{\lemma{\textnormal{\emph{Zeitungsnotiz}}}\Cendnote{\textnormal{Die Meldung war Ende Januar 1901 in diversen Zeitungen gebracht worden, etwa:
                     [O. V.]: \emph{Tagesneuigkeiten. Richtig}\pwindex{Tagesneuigkeiten. Richtig@\emph{Tagesneuigkeiten. Richtig}|pwk}.
                     In: \emph{Arbeiter-Zeitung}\pwindex{Arbeiter-Zeitung@\emph{Arbeiter-Zeitung}|pwk}, Jg. 13, Nr. 28,
                        28. 1. 1901, Mittagsblatt,
                  S. 3.}}}\label{K_L03057-6}:\pend
           { }
\pstart
           \textcolor{gray}{\textbf{\textbf{Die verkannte Muſe.} Dem Briefkasten eines \label{K_L03057-7v}\edtext{ſüdungar\oindex{Ungarn@\textbf{Ungarn}|pwv}iſchen Blattes}{\lemma{\textnormal{\emph{südungarischen Blattes}}}\Cendnote{\textnormal{nicht ermittelt}}}\label{K_L03057-7} entnimmt die »Bresl. Ztg.\orgindex{Breslauer Zeitung@Breslauer Zeitung|pw}« folgende merkwürdige Antwort: »Alter
                  Abonnent. Sie haben Ihre Wette gewonnen. \label{K_L03057-8v}\edtext{Terpſichore}{\lemma{\textnormal{\emph{Terpsichore}}}\Cendnote{\textnormal{eine der neun Musen aus der griech\oindex{Griechenland@\textbf{Griechenland}|pwkv}ischen Mythologie, die stellvertretend für die Chorlyrik, den
                     Tanz und die Wissenschaften steht; unklarer Bezug zu Beer-Hofmann\pwindex{Beer-Hofmann, Richard 11.\,7.\,1866 Wien – 26.\,9.\,1945 New York City@\textsc{Beer-Hofmann, Richard} (11.\,7.\,1866 Wien – 26.\,9.\,1945 New York City), \emph{Schriftsteller}|pwk}}}}\label{K_L03057-8} iſt kein jüdiſcher Feiertag«}}\pend
           { }
\pstart
           {\pb}Frl. \textsc{Mizzi Glümer\pwindex{Glümer, Marie 3.\,7.\,1867 Wien – 16.\,11.\,1925 München@\textsc{Glümer, Marie} (3.\,7.\,1867 Wien – 16.\,11.\,1925 München), \emph{Schauspielerin}|pw}} hatte wieder einen \label{K_L03057-9v}\edtext{Rückfall}{\lemma{\textnormal{\emph{Rückfall}}}\Cendnote{\textnormal{Siehe XXXX Auszeichnungsfehler: Dokument L03055 nicht gefunden.
               }}}\label{K_L03057-9}, nachdem{ }ſie{ }ſich bereits ganz geneſen geglaubt. Es iſt ein Jammer mit dem
                  Mädel\pwindex{Glümer, Marie 3.\,7.\,1867 Wien – 16.\,11.\,1925 München@\textsc{Glümer, Marie} (3.\,7.\,1867 Wien – 16.\,11.\,1925 München), \emph{Schauspielerin}|pwv}. Kann das wirklich
               nur \label{K_L03057-10v}\edtext{\textsc{Neuralgie}}{\lemma{\textnormal{\emph{Neuralgie}}}\Cendnote{\textnormal{Nervenschmerzen; siehe A. S.: \emph{Tagebuch}, 22. 2. 1901, 3. 3. 1901, 5. 3. 1901.
               }}}\label{K_L03057-10}{ }ſein? Oder was{ }ſonſt?\pend
           
\pstart
           Schreib’ mir bald!\pend
           
\pstart
           Viele treue Grüße! {\\[\baselineskip]}Dein {\\[\baselineskip]}\spacefill\mbox{Paul Goldmann}\pend
           \leftskip=0em{}\selectlanguage{ngerman}\endnumbering\briefempfaengerindex{Schnitzler, Arthur@\textsc{Schnitzler, Arthur}!zzzGoldmann, Paul@\emph{von Paul Goldmann}!1901-02-121@{12. 2. [1901]}|)be}\mylabel{L03057h}  \newcommand{\dateiname}{L03057}\newcommand{\titel}{Paul Goldmann an Arthur Schnitzler, 12. 2. [1901]}\newcommand{\editorInnen}{Martin Anton Müller und Laura Untner}%% latex-leseansicht-abspann.tex
%% Abspann für die Leseansicht.
%% Der Schalter \ifkorrekturansicht ist bereits durch den Vorspann gesetzt.

%% latex-abspann.tex
%% Gemeinsamer Abspann für Korrekturansicht und Leseansicht.
%% Setzt den Schalter \ifkorrekturansicht voraus (gesetzt in den
%% einbindenden Dateien latex-korrekturansicht-abspann.tex bzw.
%% latex-leseansicht-abspann.tex).
%% ---------------------------------------------------------------

\normalsize

% Das esempio-Environment wird nur in der Leseansicht benötigt
\ifkorrekturansicht\else
\newenvironment{esempio}[3]%
{
    \vspace{1.5ex}
    \rlap{\underline{#1}}
    \par
    \setlength{\parindent}{0cm}
    \nopagebreak
    \leftskip=#2cm
    \rightskip=#3cm
}
{
    \par
}
\fi

\doendnotes{C}
\bigskip
\vfill

\clearpage

\footnotesize

\ifkorrekturansicht
  \lohead{\textsc{register}}
\fi

% theindex-Environment neu definieren ohne reledmac
\makeatletter
\renewenvironment{theindex}{%
  \ifkorrekturansicht
    \section*{\indexname}%
  \else
    \subsubsection*{Index der erwähnten Entitäten}%
  \fi
  \setlength{\parindent}{0pt}%
  \setlength{\parskip}{0pt plus 0.3pt}%
  \let\item\@idxitem
}{%
  \ifkorrekturansicht\clearpage\fi
}
\makeatother

\IfFileExists{\jobname-pw.ind}{\input{\jobname-pw.ind}}{}

% Quellenangabe nur in der Leseansicht
\ifkorrekturansicht\else
% Fallback-Definitionen, falls die .tex-Datei \titel etc. nicht gesetzt hat
\providecommand{\titel}{}
\providecommand{\editorInnen}{}
\providecommand{\dateiname}{\jobname}

\vspace{3cm}

\vfill

\footnotesize
\textsc{Quelle}: \titel. Herausgegeben von {\editorInnen}. In: \emph{Arthur Schnitzler: Briefwechsel mit Autorinnen und Autoren}.
 Digitale Edition, https://schnitzler-briefe.acdh.oeaw.ac.at/{\dateiname}.html (Stand \today)
\fi

\end{document}


