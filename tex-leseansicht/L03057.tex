%% latex-leseansicht-vorspann.tex
%% Vorspann für die Leseansicht.
%% Lädt die gemeinsame Datei latex-vorspann.tex mit nicht gesetztem Schalter.

\newif\ifkorrekturansicht
\korrekturansichtfalse

\input{../tex-inputs/latex-vorspann}


         
         \renewcommand{\erwaehntePersonen}{Personen: Richard Beer-Hofmann, Gabriel Beer-Hofmann, Eduard Douwes Dekker, Marie Glümer, Paul Goldmann, Louise Schnitzler, Olga Schnitzler, Elisabeth Steinrück, Ida d’Albert}
         \renewcommand{\erwaehnteInstitutionen}{Institutionen: Breslauer Zeitung}
         \renewcommand{\erwaehnteOrte}{Orte: Berlin, Bologna, Dessauer Straße, Florenz, Genua, Griechenland, Italien, Niederlande, Pisa, Rom, Rotensterngasse, Ungarn, Wien}
         \renewcommand{\erwaehnteWerke}{Werke: Arbeiter-Zeitung, Tagebuch, Tagesneuigkeiten. Richtig}
               \section[ Paul Goldmann an Arthur Schnitzler, 12. 2. {[}1901{]}]{ Paul Goldmann an Arthur Schnitzler, 12. 2. {[}1901{]}}\nopagebreak\mylabel{v}\rehead{ }\begin{ledgroupsized}[t]{13cm}\normalsize\beginnumbering \toendnotes[C]{\smallbreak\pagebreak[2]} \Standort{DLA, A:Schnitzler, HS.NZ85.1.3171.}
\physDesc{Brief, 1 Blatt, 3 Seiten, 830 Zeichen
\newline{}Handschrift: blaue Tinte, deutsche Kurrent
\newline{}Beilage: ein Zeitungsausschnitt, beschnitten und aufgeklebt 
\newline{}Schnitzler: 1) mit Bleistift das Jahr »1901« vermerkt  2) mit rotem Buntstift drei Unterstreichungen}\toendnotes[C]{\smallbreak}\pstart
           \noindent{}\raggedleft{}{\pb}\textcolor{gray}{\textbf{DESSAUERSTRASSE 19}}\oindex{Dessauer Strasse@\textbf{Dessauer Straße}|pw}\pend
           \pstart
           Berlin\oindex{Berlin@\textbf{Berlin}|pw}, 12. Februar.\pend
           \pstart\center{}Mein lieber Freund,\pend\pstart
           Wie gehts?\pend
           \pstart
           Nach \label{K_L03057-1v}\edtext{Italien\oindex{Italien@\textbf{Italien}|pw}}{\lemma{\textnormal{\emph{Italien}}}\Cendnote{\textnormal{Schnitzler\pwindex{Schnitzler, Arthur 15.05.1862 – 21.10.1931@\textsc{Schnitzler, Arthur} (15.05.1862 – 21.10.1931), \emph{Schriftsteller, Mediziner}|pwk} reiste zwischen 26. 3. 1901 und 18. 4. 1901 nach Genua\oindex{Genua@\textbf{Genua}|pwk}, Pisa\oindex{Pisa@\textbf{Pisa}|pwk}, Rom\oindex{Rom@\textbf{Rom}|pwk}, Florenz\oindex{Florenz@\textbf{Florenz}|pwk} und Bologna\oindex{Bologna@\textbf{Bologna}|pwk}.}}}\label{K_L03057-1h} kann ich ſelbſtverſtändlich nicht mitkommen. Aber es iſt
               ſchön, daß Du hingehſt.\pend
           \pstart
           Frau \textsc{Fulda}\pwindex{DAlbert, Ida 05.12.1869 – 1926-10-06@\textsc{d’Albert, Ida} (05.12.1869 – 1926-10-06)|pw} (welche ein geiſt- und herzloſes Weib iſt und mir immer weniger ſympathiſch
               wird) ſuchte dieſer Tage aus mir herauszubekommen, ob Du in \label{K_L03057-2v}\edtext{weiblicher Geſellſchaft}{\lemma{\textnormal{\emph{weiblicher Geſellſchaft}}}\Cendnote{\textnormal{Schnitzler\pwindex{Schnitzler, Arthur 15.05.1862 – 21.10.1931@\textsc{Schnitzler, Arthur} (15.05.1862 – 21.10.1931), \emph{Schriftsteller, Mediziner}|pwk} reiste, abgesehen von seiner
                  Mutter Louise\pwindex{Schnitzler, Louise 1840-07-08 – 1911-09-09@\textsc{Schnitzler, Louise} (1840-07-08 – 1911-09-09)|pwk}, die am 11. 4. 1901 in Florenz\oindex{Florenz@\textbf{Florenz}|pwk} ankam, allein.}}}\label{K_L03057-2h} nach Italien\oindex{Italien@\textbf{Italien}|pw} gehſt? Ich ſagte: nein.\pend
           \pstart
           {\pb}Was macht die \label{K_L03057-3v}\edtext{Rotheſterngaſſe\oindex{Rotensterngasse@\textbf{Rotensterngasse}|pw}\pwindex{Schnitzler, Olga 17.01.1882 – 13.01.1970@\textsc{Schnitzler, Olga} (17.01.1882 – 13.01.1970), \emph{Schauspielerin, Sängerin}|pwv}\pwindex{Steinrueck, Elisabeth 19.11.1885 – 07.04.1920@\textsc{Steinrück, Elisabeth} (19.11.1885 – 07.04.1920)|pwv}}{\lemma{\textnormal{\emph{Rotheſterngaſſe}}}\Cendnote{\textnormal{Bezugnahme auf Schnitzler\pwindex{Schnitzler, Arthur 15.05.1862 – 21.10.1931@\textsc{Schnitzler, Arthur} (15.05.1862 – 21.10.1931), \emph{Schriftsteller, Mediziner}|pwk}s nachmalige Ehefrau Olga\pwindex{Schnitzler, Olga 17.01.1882 – 13.01.1970@\textsc{Schnitzler, Olga} (17.01.1882 – 13.01.1970), \emph{Schauspielerin, Sängerin}|pwk} und ihre Schwester Elisabeth\pwindex{Steinrueck, Elisabeth 19.11.1885 – 07.04.1920@\textsc{Steinrück, Elisabeth} (19.11.1885 – 07.04.1920)|pwk}, die in der Rotensterngasse\oindex{Rotensterngasse@\textbf{Rotensterngasse}|pwk} wohnten}}}\label{K_L03057-3h}?\pend
           \pstart
           Bitte, lies \label{K_L03057-4v}\edtext{\textsc{Multatuli\pwindex{Dekker, Eduard Douwes 02.03.1820 – 19.02.1887@\textsc{Dekker, Eduard Douwes} (02.03.1820 – 19.02.1887), \emph{Schriftsteller}|pw}}}{\lemma{\textnormal{\emph{Multatuli}}}\Cendnote{\textnormal{Pseudonym des niederländ\oindex{Niederlande@\textbf{Niederlande}|pwkv}ischen Autors Eduard Douwes Dekker\pwindex{Dekker, Eduard Douwes 02.03.1820 – 19.02.1887@\textsc{Dekker, Eduard Douwes} (02.03.1820 – 19.02.1887), \emph{Schriftsteller}|pwk}; Lektüre mittels \emph{Tagebuch}\pwindex{\textcolor{red}{\textsuperscript{XXXX1 indx}}!Tagebuch1981 – 2000@\strich\emph{Tagebuch} {[}Hrsg., 1981 – 2000{]}|pwk} und Leseliste belegbar, vgl. A. S.: \emph{Lektüren}, Norden sowie A. S.: \emph{Tagebuch}, 28. 11. 1907, 30. 11. 1907, 12. 1. 1908, 26. 1. 1908}}}\label{K_L03057-4h}!\pend
           \pstart
           \textsc{Richard\pwindex{Beer-Hofmann, Richard 1866-07-11 – 1945-09-26@\textsc{Beer-Hofmann, Richard} (1866-07-11 – 1945-09-26), \emph{Schriftsteller}|pw}} hat ſich in der That nicht dazu aufſchwingen können, mir die \label{K_L03057-5v}\edtext{Geburt ſeines Sohn\pwindex{Beer-Hofmann, Gabriel 09.01.1901 – 24.03.1971@\textsc{Beer-Hofmann, Gabriel} (09.01.1901 – 24.03.1971), \emph{Schriftsteller, Filmagent}|pwv}es}{\lemma{\textnormal{\emph{Geburt ſeines Sohnes}}}\Cendnote{\textnormal{Gabriel Beer-Hofmann\pwindex{Beer-Hofmann, Gabriel 09.01.1901 – 24.03.1971@\textsc{Beer-Hofmann, Gabriel} (09.01.1901 – 24.03.1971), \emph{Schriftsteller, Filmagent}|pwk} wurde am 9. 1. 1901 in Wien\oindex{Wien@\textbf{Wien}|pwk}
                  geboren.}}}\label{K_L03057-5h} anzuzeigen. Ich habe keine Worte mehr für dieſes Benehmen.
               Nichtsdeſtoweniger ſchicke ich ihm die nachfolgende \label{K_L03057-6v}\edtext{Zeitungsnotiz\pwindex{?? Werk@Nicht ermittelte Verfasserinnen und Verfasser!Tagesneuigkeiten. Richtig1901-01-28@\emph{Tagesneuigkeiten. Richtig} {[}1901-01-28{]}|pwv}}{\lemma{\textnormal{\emph{Zeitungsnotiz}}}\Cendnote{\textnormal{Die Meldung wurde Ende Januar 1901 in diversen Zeitungen gebracht, etwa:
                     [O. V.]: \emph{Tagesneuigkeiten. Richtig}\pwindex{?? Werk@Nicht ermittelte Verfasserinnen und Verfasser!Tagesneuigkeiten. Richtig1901-01-28@\emph{Tagesneuigkeiten. Richtig} {[}1901-01-28{]}|pwk}.
                     In: \emph{Arbeiter-Zeitung}\pwindex{Arbeiter-Zeitung12.7.1881 – 31.10.1991@\emph{Arbeiter-Zeitung} {[}12.7.1881 – 31.10.1991{]}|pwk}, Jg. 13, Nr. 28,
                        28. 1. 1901, Mittagsblatt,
                  S. 3.}}}\label{K_L03057-6h}:\pend
           { }\pstart
           \textcolor{gray}{\textbf{\textbf{Die verkannte Muſe.} Dem Briefkasten eines \label{K_L03057-7v}\edtext{ſüdungar\oindex{Ungarn@\textbf{Ungarn}|pwv}iſchen Blattes}{\lemma{\textnormal{\emph{ſüdungariſchen Blattes}}}\Cendnote{\textnormal{nicht ermittelt}}}\label{K_L03057-7h} entnimmt die »Bresl. Ztg.\orgindex{Breslauer Zeitung@Breslauer Zeitung|pw}« folgende merkwürdige Antwort: »Alter
                  Abonnent. Sie haben Ihre Wette gewonnen. \label{K_L03057-8v}\edtext{Terpſichore}{\lemma{\textnormal{\emph{Terpſichore}}}\Cendnote{\textnormal{eine der neun Musen aus der griech\oindex{Griechenland@\textbf{Griechenland}|pwkv}ischen Mythologie, die stellvertretend für die Chorlyrik, den
                     Tanz und die Wissenschaften steht; unklarer Bezug zu Beer-Hofmann\pwindex{Beer-Hofmann, Richard 1866-07-11 – 1945-09-26@\textsc{Beer-Hofmann, Richard} (1866-07-11 – 1945-09-26), \emph{Schriftsteller}|pwk}}}}\label{K_L03057-8h} iſt kein jüdiſcher Feiertag«}}\pend
           { }\pstart
           {\pb}Frl. \textsc{Mizzi Glümer\pwindex{Gluemer, Marie 03.07.1867 – 16.11.1925@\textsc{Glümer, Marie} (03.07.1867 – 16.11.1925), \emph{Schauspielerin}|pw}} hatte wieder einen \label{K_L03057-9v}\edtext{Rückfall}{\lemma{\textnormal{\emph{Rückfall}}}\Cendnote{\textnormal{siehe Paul Goldmann an Arthur Schnitzler, 22. 1. [1901]}}}\label{K_L03057-9h}, nachdem ſie ſich bereits ganz geneſen geglaubt. Es iſt ein Jammer mit dem
                  Mädel\pwindex{Gluemer, Marie 03.07.1867 – 16.11.1925@\textsc{Glümer, Marie} (03.07.1867 – 16.11.1925), \emph{Schauspielerin}|pwv}. Kann das wirklich
               nur \label{K_L03057-10v}\edtext{\textsc{Neuralgie}}{\lemma{\textnormal{\emph{Neuralgie}}}\Cendnote{\textnormal{Nervenschmerzen; siehe A. S.: \emph{Tagebuch}, 22. 2. 1901, 3. 3. 1901, 5. 3. 1901}}}\label{K_L03057-10h} ſein? Oder was ſonſt?\pend
           \pstart
           Schreib’ mir bald!\pend
           \pstart
           Viele treue Grüße! {\\[\baselineskip]}Dein {\\[\baselineskip]}\spacefill\mbox{Paul Goldmann}\pend
           \leftskip=0em{}
         
         \endnumbering\mylabel{h}\end{ledgroupsized}  \newcommand{\dateiname}{L03057}\newcommand{\titel}{Paul Goldmann an Arthur Schnitzler, 12. 2. [1901]}\newcommand{\editorInnen}{Martin Anton Müller und Laura Untner}%% latex-leseansicht-abspann.tex
%% Abspann für die Leseansicht.
%% Der Schalter \ifkorrekturansicht ist bereits durch den Vorspann gesetzt.

%% latex-abspann.tex
%% Gemeinsamer Abspann für Korrekturansicht und Leseansicht.
%% Setzt den Schalter \ifkorrekturansicht voraus (gesetzt in den
%% einbindenden Dateien latex-korrekturansicht-abspann.tex bzw.
%% latex-leseansicht-abspann.tex).
%% ---------------------------------------------------------------

\normalsize

% Das esempio-Environment wird nur in der Leseansicht benötigt
\ifkorrekturansicht\else
\newenvironment{esempio}[3]%
{
    \vspace{1.5ex}
    \rlap{\underline{#1}}
    \par
    \setlength{\parindent}{0cm}
    \nopagebreak
    \leftskip=#2cm
    \rightskip=#3cm
}
{
    \par
}
\fi

\doendnotes{C}
\bigskip
\vfill

\clearpage

\footnotesize

\ifkorrekturansicht
  \lohead{\textsc{register}}
\fi

% theindex-Environment neu definieren ohne reledmac
\makeatletter
\renewenvironment{theindex}{%
  \ifkorrekturansicht
    \section*{\indexname}%
  \else
    \subsubsection*{Index der erwähnten Entitäten}%
  \fi
  \setlength{\parindent}{0pt}%
  \setlength{\parskip}{0pt plus 0.3pt}%
  \let\item\@idxitem
}{%
  \ifkorrekturansicht\clearpage\fi
}
\makeatother

\IfFileExists{\jobname-pw.ind}{\input{\jobname-pw.ind}}{}

% Quellenangabe nur in der Leseansicht
\ifkorrekturansicht\else
% Fallback-Definitionen, falls die .tex-Datei \titel etc. nicht gesetzt hat
\providecommand{\titel}{}
\providecommand{\editorInnen}{}
\providecommand{\dateiname}{\jobname}

\vspace{3cm}

\vfill

\footnotesize
\textsc{Quelle}: \titel. Herausgegeben von {\editorInnen}. In: \emph{Arthur Schnitzler: Briefwechsel mit Autorinnen und Autoren}.
 Digitale Edition, https://schnitzler-briefe.acdh.oeaw.ac.at/{\dateiname}.html (Stand \today)
\fi

\end{document}


      