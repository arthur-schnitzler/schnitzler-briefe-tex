%% latex-korrekturansicht-vorspann.tex
%% Vorspann für die Korrekturansicht.
%% Lädt die gemeinsame Datei latex-vorspann.tex mit gesetztem Schalter.

\newif\ifkorrekturansicht
\korrekturansichttrue

\input{../tex-inputs/latex-vorspann}


\section[ Paul Goldmann an Arthur Schnitzler, 12. 2. {[}1901{]}]{L03057 Paul Goldmann an Arthur Schnitzler, 12. 2. {[}1901{]}}
\nopagebreak\mylabel{L03057v}
\rehead{ }\normalsize\beginnumbering\briefempfaengerindex{Schnitzler, Arthur@\textsc{Schnitzler, Arthur}!zzzGoldmann, Paul@\emph{von Paul Goldmann}!1901-02-121@{12. 2. {[}1901{]}}|(be}
\toendnotes[C]{\smallbreak\pagebreak[2]}\Standort{DLA, A:Schnitzler, HS.NZ85.1.3171.}
\physDesc{Brief, 1 Blatt, 3 Seiten, 830 Zeichen
\newline{}Handschrift: blaue Tinte, deutsche Kurrent
\newline{}Beilage: ein Zeitungsausschnitt, beschnitten und aufgeklebt 
\newline{}Schnitzler: 1) mit Bleistift das Jahr »1901« vermerkt  2) mit rotem Buntstift drei Unterstreichungen}\toendnotes[C]{\smallbreak}
\pstart
           \raggedleft{}{\pb}\textcolor{gray}{\textbf{DESSAUERSTRASSE 19}}\oindex{Dessauer Strasse@\textbf{Dessauer Straße}, \emph{Straße (K.STR)}|pw}\pend
           
\pstart
           Berlin\oindex{Berlin@\textbf{Berlin}, \emph{P.PPLC}|pw}, 12. Februar.\pend
           
\pstart\center{}Mein lieber Freund,\pend\vspace{0.5em}
\pstart
           Wie gehts?\pend
           
\pstart
           Nach \label{K_L03057-1v}\edtext{Italien\oindex{Italien@\textbf{Italien}, \emph{A.PCLI}|pw}}{\lemma{\textnormal{\emph{Italien}}}\Cendnote{\textnormal{Schnitzler reiste zwischen 26. 3. 1901 und 18. 4. 1901 nach Genua\oindex{Genua@\textbf{Genua}, \emph{P.PPLA}|pwk}, Pisa\oindex{Pisa@\textbf{Pisa}, \emph{P.PPLA2}|pwk}, Rom\oindex{Rom@\textbf{Rom}, \emph{P.PPLC}|pwk}, Florenz\oindex{Florenz@\textbf{Florenz}, \emph{P.PPLA}|pwk} und Bologna\oindex{Bologna@\textbf{Bologna}, \emph{P.PPLA}|pwk}.}}}\label{K_L03057-1} kann ich ſelbſtverſtändlich nicht mitkommen. Aber es iſt
               ſchön, daß Du hingehſt.\pend
           
\pstart
           Frau \textsc{Fulda}\pwindex{DAlbert, Ida 05.12.1869 – 1926-10-06@\textsc{d’Albert, Ida} (05.12.1869 – 1926-10-06), \emph{Schauspieler/Schauspielerin}|pw} (welche ein geiſt- und herzloſes Weib iſt und mir immer weniger ſympathiſch
               wird) ſuchte dieſer Tage aus mir herauszubekommen, ob Du in \label{K_L03057-2v}\edtext{weiblicher Geſellſchaft}{\lemma{\textnormal{\emph{weiblicher Geſellſchaft}}}\Cendnote{\textnormal{Schnitzler reiste, abgesehen von seiner
                  Mutter Louise\pwindex{Schnitzler, Louise 1840-07-08 – 1911-09-09@\textsc{Schnitzler, Louise} (1840-07-08 – 1911-09-09)|pwk}, die am 11. 4. 1901 in Florenz\oindex{Florenz@\textbf{Florenz}, \emph{P.PPLA}|pwk} ankam, allein.}}}\label{K_L03057-2} nach Italien\oindex{Italien@\textbf{Italien}, \emph{A.PCLI}|pw} gehſt? Ich ſagte: nein.\pend
           
\pstart
           {\pb}Was macht die \label{K_L03057-3v}\edtext{Rotheſterngaſſe\oindex{Rotensterngasse@\textbf{Rotensterngasse}, \emph{Straße (K.STR)}|pw}\pwindex{Schnitzler, Olga 17.01.1882 – 13.01.1970@\textsc{Schnitzler, Olga} (17.01.1882 – 13.01.1970), \emph{Schauspieler/Schauspielerin, Sänger/Sängerin}|pwv}\pwindex{Steinrueck, Elisabeth 19.11.1885 – 07.04.1920@\textsc{Steinrück, Elisabeth} (19.11.1885 – 07.04.1920)|pwv}}{\lemma{\textnormal{\emph{Rotheſterngaſſe}}}\Cendnote{\textnormal{Bezugnahme auf Schnitzlers nachmalige Ehefrau Olga\pwindex{Schnitzler, Olga 17.01.1882 – 13.01.1970@\textsc{Schnitzler, Olga} (17.01.1882 – 13.01.1970), \emph{Schauspieler/Schauspielerin, Sänger/Sängerin}|pwk} und ihre Schwester Elisabeth\pwindex{Steinrueck, Elisabeth 19.11.1885 – 07.04.1920@\textsc{Steinrück, Elisabeth} (19.11.1885 – 07.04.1920)|pwk}, die in der Rotensterngasse\oindex{Rotensterngasse@\textbf{Rotensterngasse}, \emph{Straße (K.STR)}|pwk} wohnten}}}\label{K_L03057-3}?\pend
           
\pstart
           Bitte, lies \label{K_L03057-4v}\edtext{\textsc{Multatuli\pwindex{Dekker, Eduard Douwes 02.03.1820 – 19.02.1887@\textsc{Dekker, Eduard Douwes} (02.03.1820 – 19.02.1887), \emph{Schriftsteller/Schriftstellerin}|pw}}}{\lemma{\textnormal{\emph{Multatuli}}}\Cendnote{\textnormal{Pseudonym des niederländ\oindex{Niederlande@\textbf{Niederlande}, \emph{A.PCLI}|pwkv}ischen Autors Eduard Douwes Dekker\pwindex{Dekker, Eduard Douwes 02.03.1820 – 19.02.1887@\textsc{Dekker, Eduard Douwes} (02.03.1820 – 19.02.1887), \emph{Schriftsteller/Schriftstellerin}|pwk}; Lektüre mittels \emph{Tagebuch}\pwindex{Tagebuch@\emph{Tagebuch}|pwk} und Leseliste belegbar, vgl. A. S.: \emph{Lektüren}, Norden sowie A. S.: \emph{Tagebuch}, 28. 11. 1907, 30. 11. 1907, 12. 1. 1908, 26. 1. 1908.
               }}}\label{K_L03057-4}!\pend
           
\pstart
           \textsc{Richard\pwindex{Beer-Hofmann, Richard 1866-07-11 – 1945-09-26@\textsc{Beer-Hofmann, Richard} (1866-07-11 – 1945-09-26), \emph{Schriftsteller/Schriftstellerin}|pw}} hat ſich in der That nicht dazu aufſchwingen können, mir die \label{K_L03057-5v}\edtext{Geburt ſeines Sohn\pwindex{Beer-Hofmann, Gabriel 09.01.1901 – 24.03.1971@\textsc{Beer-Hofmann, Gabriel} (09.01.1901 – 24.03.1971), \emph{Schriftsteller/Schriftstellerin, Filmagent/Filmagentin}|pwv}es}{\lemma{\textnormal{\emph{Geburt ſeines Sohnes}}}\Cendnote{\textnormal{Gabriel Beer-Hofmann\pwindex{Beer-Hofmann, Gabriel 09.01.1901 – 24.03.1971@\textsc{Beer-Hofmann, Gabriel} (09.01.1901 – 24.03.1971), \emph{Schriftsteller/Schriftstellerin, Filmagent/Filmagentin}|pwk} wurde am 9. 1. 1901 in Wien\oindex{Wien@\textbf{Wien}, \emph{A.ADM2}|pwk}
                  geboren.}}}\label{K_L03057-5} anzuzeigen. Ich habe keine Worte mehr für dieſes Benehmen.
               Nichtsdeſtoweniger ſchicke ich ihm die nachfolgende \label{K_L03057-6v}\edtext{Zeitungsnotiz\pwindex{Tagesneuigkeiten. Richtig@\emph{Tagesneuigkeiten. Richtig}|pwv}}{\lemma{\textnormal{\emph{Zeitungsnotiz}}}\Cendnote{\textnormal{Die Meldung war Ende Januar 1901 in diversen Zeitungen gebracht worden, etwa:
                     [O. V.]: \emph{Tagesneuigkeiten. Richtig}\pwindex{Tagesneuigkeiten. Richtig@\emph{Tagesneuigkeiten. Richtig}|pwk}.
                     In: \emph{Arbeiter-Zeitung}\pwindex{Arbeiter-Zeitung@\emph{Arbeiter-Zeitung}|pwk}, Jg. 13, Nr. 28,
                        28. 1. 1901, Mittagsblatt,
                  S. 3.}}}\label{K_L03057-6}:\pend
           { }
\pstart
           \textcolor{gray}{\textbf{\textbf{Die verkannte Muſe.} Dem Briefkasten eines \label{K_L03057-7v}\edtext{ſüdungar\oindex{Ungarn@\textbf{Ungarn}, \emph{A.PCLI}|pwv}iſchen Blattes}{\lemma{\textnormal{\emph{ſüdungariſchen Blattes}}}\Cendnote{\textnormal{nicht ermittelt}}}\label{K_L03057-7} entnimmt die »Bresl. Ztg.\orgindex{Breslauer Zeitung@Breslauer Zeitung|pw}« folgende merkwürdige Antwort: »Alter
                  Abonnent. Sie haben Ihre Wette gewonnen. \label{K_L03057-8v}\edtext{Terpſichore}{\lemma{\textnormal{\emph{Terpſichore}}}\Cendnote{\textnormal{eine der neun Musen aus der griech\oindex{Griechenland@\textbf{Griechenland}, \emph{A.PCLI}|pwkv}ischen Mythologie, die stellvertretend für die Chorlyrik, den
                     Tanz und die Wissenschaften steht; unklarer Bezug zu Beer-Hofmann\pwindex{Beer-Hofmann, Richard 1866-07-11 – 1945-09-26@\textsc{Beer-Hofmann, Richard} (1866-07-11 – 1945-09-26), \emph{Schriftsteller/Schriftstellerin}|pwk}}}}\label{K_L03057-8} iſt kein jüdiſcher Feiertag«}}\pend
           { }
\pstart
           {\pb}Frl. \textsc{Mizzi Glümer\pwindex{Gluemer, Marie 03.07.1867 – 16.11.1925@\textsc{Glümer, Marie} (03.07.1867 – 16.11.1925), \emph{Schauspieler/Schauspielerin}|pw}} hatte wieder einen \label{K_L03057-9v}\edtext{Rückfall}{\lemma{\textnormal{\emph{Rückfall}}}\Cendnote{\textnormal{Siehe Paul Goldmann an Arthur Schnitzler, 22. 1. [1901].
               }}}\label{K_L03057-9}, nachdem ſie ſich bereits ganz geneſen geglaubt. Es iſt ein Jammer mit dem
                  Mädel\pwindex{Gluemer, Marie 03.07.1867 – 16.11.1925@\textsc{Glümer, Marie} (03.07.1867 – 16.11.1925), \emph{Schauspieler/Schauspielerin}|pwv}. Kann das wirklich
               nur \label{K_L03057-10v}\edtext{\textsc{Neuralgie}}{\lemma{\textnormal{\emph{Neuralgie}}}\Cendnote{\textnormal{Nervenschmerzen; siehe A. S.: \emph{Tagebuch}, 22. 2. 1901, 3. 3. 1901, 5. 3. 1901.
               }}}\label{K_L03057-10} ſein? Oder was ſonſt?\pend
           
\pstart
           Schreib’ mir bald!\pend
           
\pstart
           Viele treue Grüße! {\\[\baselineskip]}Dein {\\[\baselineskip]}\spacefill\mbox{Paul Goldmann}\pend
           \leftskip=0em{}\selectlanguage{ngerman}\endnumbering\briefempfaengerindex{Schnitzler, Arthur@\textsc{Schnitzler, Arthur}!zzzGoldmann, Paul@\emph{von Paul Goldmann}!1901-02-121@{12. 2. {[}1901{]}}|)be}\mylabel{L03057h}  \normalsize

\doendnotes{C}
\bigskip
\vfill

\clearpage

\footnotesize

\lohead{\textsc{register}}

% Definiere theindex-Environment komplett neu ohne reledmac
\makeatletter
\renewenvironment{theindex}{%
  \section*{\indexname}%
  \setlength{\parindent}{0pt}%
  \setlength{\parskip}{0pt plus 0.3pt}%
  \let\item\@idxitem
}{%
  \clearpage
}
\makeatother

\IfFileExists{\jobname-pw.ind}{\input{\jobname-pw.ind}}{}

\end{document}

      