%% latex-leseansicht-vorspann.tex
%% Vorspann für die Leseansicht.
%% Lädt die gemeinsame Datei latex-vorspann.tex mit nicht gesetztem Schalter.

\newif\ifkorrekturansicht
\korrekturansichtfalse

\input{../tex-inputs/latex-vorspann}


\section[Arthur Schnitzler an Georg Brandes, 14. 5. 1901]{L01120 Arthur Schnitzler an Georg Brandes, 14. 5. 1901}
\nopagebreak\mylabel{L01120v}
\rehead{ }\normalsize\beginnumbering\briefempfaengerindex{Brandes, Georg@\textsc{Brandes, Georg}!zzzSchnitzler, Arthur@\emph{von Arthur Schnitzler}!1901-05-141@{14. 5. 1901}|(be}
\toendnotes[C]{\smallbreak\pagebreak[2]}
\correspDesc{Versand  durch Arthur Schnitzler am 14. 5. 1901 in Wien
\newline{}Erhalt  durch Georg Brandes im Zeitraum [14. 5. 1901
                  – 18. 5. 1901?] \textbf{Ort fehlend} }\toendnotes[C]{\smallbreak}
\Standort{Kopenhagen, Det Kongelige Bibliotek, Georg Brandes Arkiv, box 125.}
\physDesc{Brief, 1 Blatt, 3 Seiten, 831 Zeichen
\newline{}Handschrift: schwarze Tinte, deutsche Kurrent
\newline{}Ordnung: mit Bleistift von unbekannter Hand nummeriert:
                                    »23.«, datiert: »14. 5.01« und beschriftet mit: »Arth.
                                 Schnitzler« }
\buchAbdrucke{\weitereDrucke{Georg Brandes, Arthur Schnitzler: \emph{Ein Briefwechsel}. Herausgegeben von Kurt Bergel. Bern: \emph{Francke} 1956, S. 86.} }
\pstart{}{\pb}Liebſter Herr Brandes\pend\vspace{0.5em}
\pstart
           da meine Wohnung etwa zwiſchen Ihren beiden Bahnhöfen liegt, iſt es am beſten, Sie
               fahren mit Ihrem Gepäck zu mir (der Portier in unſerm Haus kann es aufbewahren; er
               wird aviſirt{ }ſein) wenn Sie es nicht vorziehen, das Gepäck vom Nordbahnhof\oindex{Wien@\textbf{Wien}!II., Leopoldstadt@\textbf{II., Leopoldstadt}!Nordbahnhof@\textbf{Nordbahnhof}, \emph{Bahnhofsgebäude}|pw} direct zum Südbahnhof\oindex{Wien@\textbf{Wien}!X., Favoriten@\textbf{X., Favoriten}!Südbahnhof@\textbf{Südbahnhof}, \emph{Bahnhofsgebäude}|pw}{ }ſchaffen zu laſſen. {\pb}Aber ich würde den Vortheil dieſer letztern
               Anordnung nicht einſehen es wäre nicht einmal eine Erſparnis.\pend
           
\pstart
           Unſer Eſſen werden wir{ }ſo einrichten, daſs Sie bequem zu Ihrem Zug auf der Südbahn\oindex{Wien@\textbf{Wien}!X., Favoriten@\textbf{X., Favoriten}!Südbahnhof@\textbf{Südbahnhof}, \emph{Bahnhofsgebäude}|pw}{ }ſind.\pend
           
\pstart
           Somit hoff ich Sie am Donnerſtg kurz nach 4 bei mir zu
               begrüßen. (Ich wohne jetzt 2 Treppen höher.) Natürlich würde ich Sie auch gerne von
               der Bahn {\pb}abholen aber es gibt Menſchen, denen das
               unangenehm iſt u ich weiſs nicht ob Sie am Ende zu dieſen gehören.\pend
           
\pstart
           Alſo auf Wiederſehen.\pend
           
\pstart
           Mit den herzlichſten Grüßen.\pend
           
\pstart
           Ihr treuer{\\[\baselineskip]}\spacefill\mbox{ArthSchnitzler}\pend
           \leftskip=0em{}
\pstart
           Wien\oindex{Wien@\textbf{Wien}, \emph{Verwaltungsgebiet}|pw}, 14. 5. 901.\pend
           \selectlanguage{ngerman}\endnumbering\briefempfaengerindex{Brandes, Georg@\textsc{Brandes, Georg}!zzzSchnitzler, Arthur@\emph{von Arthur Schnitzler}!1901-05-141@{14. 5. 1901}|)be}\mylabel{L01120h}  \newcommand{\dateiname}{L01120}\newcommand{\titel}{Arthur Schnitzler an Georg Brandes, 14. 5. 1901}\newcommand{\editorInnen}{Martin Anton Müller und Gerd-Hermann Susen}%% latex-leseansicht-abspann.tex
%% Abspann für die Leseansicht.
%% Der Schalter \ifkorrekturansicht ist bereits durch den Vorspann gesetzt.

%% latex-abspann.tex
%% Gemeinsamer Abspann für Korrekturansicht und Leseansicht.
%% Setzt den Schalter \ifkorrekturansicht voraus (gesetzt in den
%% einbindenden Dateien latex-korrekturansicht-abspann.tex bzw.
%% latex-leseansicht-abspann.tex).
%% ---------------------------------------------------------------

\normalsize

% Das esempio-Environment wird nur in der Leseansicht benötigt
\ifkorrekturansicht\else
\newenvironment{esempio}[3]%
{
    \vspace{1.5ex}
    \rlap{\underline{#1}}
    \par
    \setlength{\parindent}{0cm}
    \nopagebreak
    \leftskip=#2cm
    \rightskip=#3cm
}
{
    \par
}
\fi

\doendnotes{C}
\bigskip
\vfill

\clearpage

\footnotesize

\ifkorrekturansicht
  \lohead{\textsc{register}}
\fi

% theindex-Environment neu definieren ohne reledmac
\makeatletter
\renewenvironment{theindex}{%
  \ifkorrekturansicht
    \section*{\indexname}%
  \else
    \subsubsection*{Index der erwähnten Entitäten}%
  \fi
  \setlength{\parindent}{0pt}%
  \setlength{\parskip}{0pt plus 0.3pt}%
  \let\item\@idxitem
}{%
  \ifkorrekturansicht\clearpage\fi
}
\makeatother

\IfFileExists{\jobname-pw.ind}{\input{\jobname-pw.ind}}{}

% Quellenangabe nur in der Leseansicht
\ifkorrekturansicht\else
% Fallback-Definitionen, falls die .tex-Datei \titel etc. nicht gesetzt hat
\providecommand{\titel}{}
\providecommand{\editorInnen}{}
\providecommand{\dateiname}{\jobname}

\vspace{3cm}

\vfill

\footnotesize
\textsc{Quelle}: \titel. Herausgegeben von {\editorInnen}. In: \emph{Arthur Schnitzler: Briefwechsel mit Autorinnen und Autoren}.
 Digitale Edition, https://schnitzler-briefe.acdh.oeaw.ac.at/{\dateiname}.html (Stand \today)
\fi

\end{document}


