%% latex-leseansicht-vorspann.tex
%% Vorspann für die Leseansicht.
%% Lädt die gemeinsame Datei latex-vorspann.tex mit nicht gesetztem Schalter.

\newif\ifkorrekturansicht
\korrekturansichtfalse

\input{../tex-inputs/latex-vorspann}


\section[Hugo von Hofmannsthal an Arthur Schnitzler, 26. 3. 1899]{L00911 Hugo von Hofmannsthal an Arthur Schnitzler, 26. 3. 1899}
\nopagebreak\mylabel{L00911v}
\rehead{ }\normalsize\beginnumbering\briefempfaengerindex{Schnitzler, Arthur@\textsc{Schnitzler, Arthur}!zzzHofmannsthal, Hugo von@\emph{von Hugo von Hofmannsthal}!1899-03-261@{26. 3. 1899}|(be}
\toendnotes[C]{\smallbreak\pagebreak[2]}
\correspDesc{Versand  durch Hugo von Hofmannsthal am 26. 3. 1899 in Berlin
\newline{}Erhalt  durch Arthur Schnitzler im Zeitraum [27. 3. 1899
                  – 31. 3. 1899?] in Wien}\toendnotes[C]{\smallbreak}
\Standort{CUL, Schnitzler, B 43.}
\physDesc{Brief, 1 Blatt, 2 Seiten, 668 Zeichen
\newline{}Handschrift: schwarze Tinte, deutsche Kurrent
\newline{}Beilage: maschinelles Telegramm nach Berlin\oindex{Berlin@\textbf{Berlin}, \emph{Hauptstadt}|pw} 
\newline{}Schnitzler: mit Bleistift datiert: »2\strikeout{9}6/3 99« 
\newline{}Ordnung: 1) mit Bleistift von unbekannter Hand nummeriert: »\strikeout{143}«  2) mit Bleistift von unbekannter Hand nummeriert:
                                    »140«}
\buchAbdrucke{\weitereDrucke{Hugo von Hofmannsthal, Arthur Schnitzler: \emph{Briefwechsel}. Herausgegeben von Therese Nickl und Heinrich Schnitzler. Frankfurt am Main: \emph{S. Fischer} 1964, S. 121.} }
\pstart
           \raggedleft{}{\pb}Berlin\oindex{Berlin@\textbf{Berlin}, \emph{Hauptstadt}|pw}{ }Sonntg\pend
           \vspace{0.5em}
\pstart
           lieber, eben bekomm ich dieſes Telegra{\geminationm}
               von dem armen Poldy\pwindex{Andrian-Werburg, Leopold von 9.\,5.\,1875 Berlin – 19.\,11.\,1951 Fribourg@\textsc{Andrian-Werburg, Leopold von} (9.\,5.\,1875 Berlin – 19.\,11.\,1951 Fribourg), \emph{Schriftsteller, Diplomat}|pw}. Er bildet{ }ſich dieſmal
               ein, daſs er wahnſinnig wird. Vielleicht können Sie irgendwas machen.\pend
           
\pstart
           Ich ko{\geminationm}e, da Sie nicht herko{\geminationm}en,{ }ſchon{ }ſpäteſtens Samstag nach Wien\oindex{Wien@\textbf{Wien}, \emph{Verwaltungsgebiet}|pw}.\pend
           
\pstart
           Ich{ }ſehe viele Menſchen: Hauptmann\pwindex{Hauptmann, Gerhart 15.\,11.\,1862 Szczawno-Zdrój – 6.\,6.\,1946 Jagniątków@\textsc{Hauptmann, Gerhart} (15.\,11.\,1862 Szczawno-Zdrój – 6.\,6.\,1946 Jagniątków), \emph{Schriftsteller}|pw}, Ludwig von Hofmann\pwindex{Hofmann, Ludwig von 17.\,8.\,1861 Darmstadt – 23.\,8.\,1945 Pillnitz@\textsc{Hofmann, Ludwig von} (17.\,8.\,1861 Darmstadt – 23.\,8.\,1945 Pillnitz), \emph{Maler, Grafiker}|pw}, \textsc{Kessler}\pwindex{Kessler, Harry von 23.\,5.\,1868 Paris – 4.\,12.\,1937 Lyon@\textsc{Kessler, Harry von} (23.\,5.\,1868 Paris – 4.\,12.\,1937 Lyon), \emph{Schriftsteller, Verleger, Diplomat}|pw}, Bodenhauſen\pwindex{Bodenhausen, Eberhard von 12.\,6.\,1868 Wiesbaden – 6.\,5.\,1918 Meineweh@\textsc{Bodenhausen, Eberhard von} (12.\,6.\,1868 Wiesbaden – 6.\,5.\,1918 Meineweh)|pw}, Kainz\pwindex{Kainz, Josef 2.\,1.\,1858 Mosonmagyaróvár – 20.\,9.\,1910 Wien@\textsc{Kainz, Josef} (2.\,1.\,1858 Mosonmagyaróvár – 20.\,9.\,1910 Wien), \emph{Schauspieler}|pw}, die Dumont\pwindex{Dumont, Louise 22.\,2.\,1862 Köln – 16.\,5.\,1932 Düsseldorf@\textsc{Dumont, Louise} (22.\,2.\,1862 Köln – 16.\,5.\,1932 Düsseldorf), \emph{Theaterleiterin, Schauspielerin}|pw}{ }\textsc{etc. etc.} auch viele gute Vorſtellungen, wie Fuhrmann Henſchel\pwindex{Hauptmann, Gerhart 15.\,11.\,1862 Szczawno-Zdrój – 6.\,6.\,1946 Jagniątków@\textsc{Hauptmann, Gerhart} (15.\,11.\,1862 Szczawno-Zdrój – 6.\,6.\,1946 Jagniątków), \emph{Schriftsteller}!Fuhrmann Henschel. Schauspiel in 5 Akten@\strich\emph{Fuhrmann Henschel. Schauspiel in 5 Akten}|pw}. {\pb}Bin aber nicht im Stand einen
               Brief zu{ }ſchreiben.\pend
           
\pstart
           Von Herzen Ihr{\\[\baselineskip]}\spacefill\mbox{Hugo.}\pend
           \leftskip=0em{}\selectlanguage{ngerman}\vspace{1em}{\vspace{1\baselineskip}}
\pstart
           {\pb}v insbruck\oindex{Innsbruck@\textbf{Innsbruck}, \emph{Verwaltungsgebiet}|pw} 3747 31 26/3{ }9 40m\pend
           
\pstart
           {[}bef{]}uerchtungen geisteszustand fast eingetroffen bin sofort insbruck\oindex{Innsbruck@\textbf{Innsbruck}, \emph{Verwaltungsgebiet}|pw} gefahren {[}prof{]}essor
                  meyer\pwindex{Mayer, Karl *~9.\,12.\,1862 Wien@\textsc{Mayer, Karl} (*~9.\,12.\,1862 Wien), \emph{Psychiater, Neurologe}|pw} consultiren dieser verreist. bitte
               wenn kannst sofort herkommen wo ist schnitzler?
               = poldi\pwindex{Andrian-Werburg, Leopold von 9.\,5.\,1875 Berlin – 19.\,11.\,1951 Fribourg@\textsc{Andrian-Werburg, Leopold von} (9.\,5.\,1875 Berlin – 19.\,11.\,1951 Fribourg), \emph{Schriftsteller, Diplomat}|pw}{ }goldner adler\oindex{Hotel Goldener Adler [Innsbruck]@\textbf{Hotel Goldener Adler [Innsbruck]}, \emph{Hotel}|pw}.+=\pend
           \selectlanguage{ngerman}\endnumbering\briefempfaengerindex{Schnitzler, Arthur@\textsc{Schnitzler, Arthur}!zzzHofmannsthal, Hugo von@\emph{von Hugo von Hofmannsthal}!1899-03-261@{26. 3. 1899}|)be}\mylabel{L00911h}  \newcommand{\dateiname}{L00911}\newcommand{\titel}{Hugo von Hofmannsthal an Arthur Schnitzler, 26. 3. 1899}\newcommand{\editorInnen}{Martin Anton Müller und Gerd-Hermann Susen}%% latex-leseansicht-abspann.tex
%% Abspann für die Leseansicht.
%% Der Schalter \ifkorrekturansicht ist bereits durch den Vorspann gesetzt.

%% latex-abspann.tex
%% Gemeinsamer Abspann für Korrekturansicht und Leseansicht.
%% Setzt den Schalter \ifkorrekturansicht voraus (gesetzt in den
%% einbindenden Dateien latex-korrekturansicht-abspann.tex bzw.
%% latex-leseansicht-abspann.tex).
%% ---------------------------------------------------------------

\normalsize

% Das esempio-Environment wird nur in der Leseansicht benötigt
\ifkorrekturansicht\else
\newenvironment{esempio}[3]%
{
    \vspace{1.5ex}
    \rlap{\underline{#1}}
    \par
    \setlength{\parindent}{0cm}
    \nopagebreak
    \leftskip=#2cm
    \rightskip=#3cm
}
{
    \par
}
\fi

\doendnotes{C}
\bigskip
\vfill

\clearpage

\footnotesize

\ifkorrekturansicht
  \lohead{\textsc{register}}
\fi

% theindex-Environment neu definieren ohne reledmac
\makeatletter
\renewenvironment{theindex}{%
  \ifkorrekturansicht
    \section*{\indexname}%
  \else
    \subsubsection*{Index der erwähnten Entitäten}%
  \fi
  \setlength{\parindent}{0pt}%
  \setlength{\parskip}{0pt plus 0.3pt}%
  \let\item\@idxitem
}{%
  \ifkorrekturansicht\clearpage\fi
}
\makeatother

\IfFileExists{\jobname-pw.ind}{\input{\jobname-pw.ind}}{}

% Quellenangabe nur in der Leseansicht
\ifkorrekturansicht\else
% Fallback-Definitionen, falls die .tex-Datei \titel etc. nicht gesetzt hat
\providecommand{\titel}{}
\providecommand{\editorInnen}{}
\providecommand{\dateiname}{\jobname}

\vspace{3cm}

\vfill

\footnotesize
\textsc{Quelle}: \titel. Herausgegeben von {\editorInnen}. In: \emph{Arthur Schnitzler: Briefwechsel mit Autorinnen und Autoren}.
 Digitale Edition, https://schnitzler-briefe.acdh.oeaw.ac.at/{\dateiname}.html (Stand \today)
\fi

\end{document}


