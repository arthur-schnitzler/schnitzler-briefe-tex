%% latex-korrekturansicht-vorspann.tex
%% Vorspann für die Korrekturansicht.
%% Lädt die gemeinsame Datei latex-vorspann.tex mit gesetztem Schalter.

\newif\ifkorrekturansicht
\korrekturansichttrue

\input{../tex-inputs/latex-vorspann}


\section[Hugo von Hofmannsthal an Arthur Schnitzler, 26. 3. 1899]{L00911 Hugo von Hofmannsthal an Arthur Schnitzler, 26. 3. 1899}
\nopagebreak\mylabel{L00911v}
\rehead{ }\normalsize\beginnumbering\briefempfaengerindex{Schnitzler, Arthur@\textsc{Schnitzler, Arthur}!zzzHofmannsthal, Hugo von@\emph{von Hugo von Hofmannsthal}!1899-03-261@{26. 3. 1899}|(be}
\toendnotes[C]{\smallbreak\pagebreak[2]}\Standort{CUL, Schnitzler, B 43.}
\physDesc{Brief, 1 Blatt, 2 Seiten, 668 Zeichen
\newline{}Handschrift: schwarze Tinte, deutsche Kurrent
\newline{}Beilage: maschinelles Telegramm nach Berlin\oindex{Berlin@\textbf{Berlin}, \emph{P.PPLC}|pw} 
\newline{}Schnitzler: mit Bleistift datiert: »2\strikeout{9}6/3 99« 
\newline{}Ordnung: 1) mit Bleistift von unbekannter Hand nummeriert: »\strikeout{143}«  2) mit Bleistift von unbekannter Hand nummeriert:
                                    »140«}
\buchAbdrucke{\weitereDrucke{Hugo von Hofmannsthal, Arthur Schnitzler: \emph{Briefwechsel}. Frankfurt am Main: \emph{S. Fischer} 1964, S. 121.} }
\pstart
           \raggedleft{}{\pb}Berlin\oindex{Berlin@\textbf{Berlin}, \emph{P.PPLC}|pw}{ }Sonntg\pend
           \vspace{0.5em}
\pstart
           lieber, eben bekomm ich dieſes Telegra{\geminationm}
               von dem armen Poldy\pwindex{Andrian-Werburg, Leopold von 09.05.1875 – 19.11.1951@\textsc{Andrian-Werburg, Leopold von} (09.05.1875 – 19.11.1951), \emph{Schriftsteller/Schriftstellerin, Diplomat/Diplomatin}|pw}. Er bildet ſich dieſmal
               ein, daſs er wahnſinnig wird. Vielleicht können Sie irgendwas machen.\pend
           
\pstart
           Ich ko{\geminationm}e, da Sie nicht herko{\geminationm}en, ſchon ſpäteſtens Samstag nach Wien\oindex{Wien@\textbf{Wien}, \emph{A.ADM2}|pw}.\pend
           
\pstart
           Ich ſehe viele Menſchen: Hauptmann\pwindex{Hauptmann, Gerhart 15.11.1862 – 06.06.1946@\textsc{Hauptmann, Gerhart} (15.11.1862 – 06.06.1946), \emph{Schriftsteller/Schriftstellerin}|pw}, Ludwig von Hofmann\pwindex{Hofmann, Ludwig von 1861-08-17 – 1945-08-23@\textsc{Hofmann, Ludwig von} (1861-08-17 – 1945-08-23), \emph{Maler/Malerin, Grafiker/Grafikerin}|pw}, \textsc{Kessler}\pwindex{Kessler, Harry von 23.05.1868 – 04.12.1937@\textsc{Kessler, Harry von} (23.05.1868 – 04.12.1937), \emph{Schriftsteller/Schriftstellerin, Verleger/Verlegerin, Diplomat/Diplomatin}|pw}, Bodenhauſen\pwindex{Bodenhausen, Eberhard von 12.06.1868 – 06.05.1918@\textsc{Bodenhausen, Eberhard von} (12.06.1868 – 06.05.1918)|pw}, Kainz\pwindex{Kainz, Josef 02.01.1858 – 20.09.1910@\textsc{Kainz, Josef} (02.01.1858 – 20.09.1910), \emph{Schauspieler/Schauspielerin}|pw}, die Dumont\pwindex{Dumont, Louise 22.02.1862 – 16.05.1932@\textsc{Dumont, Louise} (22.02.1862 – 16.05.1932), \emph{Theaterleiter/Theaterleiterin, Schauspieler/Schauspielerin}|pw}{ }\textsc{etc. etc.} auch viele gute Vorſtellungen, wie Fuhrmann Henſchel\pwindex{Fuhrmann Henschel@\emph{Fuhrmann Henschel}|pw}. {\pb}Bin aber nicht im Stand einen
               Brief zu ſchreiben.\pend
           
\pstart
           Von Herzen Ihr{\\[\baselineskip]}\spacefill\mbox{Hugo.}\pend
           \leftskip=0em{}\selectlanguage{ngerman}\vspace{1em}{\vspace{1\baselineskip}}
\pstart
           {\pb}v insbruck\oindex{Innsbruck@\textbf{Innsbruck}, \emph{A.ADM2}|pw} 3747 31 26/3{ }9 40m\pend
           
\pstart
           {[}bef{]}uerchtungen geisteszustand fast eingetroffen bin sofort insbruck\oindex{Innsbruck@\textbf{Innsbruck}, \emph{A.ADM2}|pw} gefahren {[}prof{]}essor
                  meyer\pwindex{Mayer, Karl *~9.12.1862@\textsc{Mayer, Karl} (*~9.12.1862), \emph{Psychiater/Psychiaterin, Neurologe/Neurologin}|pw} consultiren dieser verreist. bitte
               wenn kannst sofort herkommen wo ist schnitzler?
               = poldi\pwindex{Andrian-Werburg, Leopold von 09.05.1875 – 19.11.1951@\textsc{Andrian-Werburg, Leopold von} (09.05.1875 – 19.11.1951), \emph{Schriftsteller/Schriftstellerin, Diplomat/Diplomatin}|pw}{ }goldner adler\oindex{Hotel Goldener Adler [Innsbruck]@\textbf{Hotel Goldener Adler [Innsbruck]}, \emph{Hotel (K.HTL)}|pw}.+=\pend
           \selectlanguage{ngerman}\endnumbering\briefempfaengerindex{Schnitzler, Arthur@\textsc{Schnitzler, Arthur}!zzzHofmannsthal, Hugo von@\emph{von Hugo von Hofmannsthal}!1899-03-261@{26. 3. 1899}|)be}\mylabel{L00911h}  \normalsize

\doendnotes{C}
\bigskip
\vfill

\clearpage

\footnotesize

\lohead{\textsc{register}}

% Definiere theindex-Environment komplett neu ohne reledmac
\makeatletter
\renewenvironment{theindex}{%
  \section*{\indexname}%
  \setlength{\parindent}{0pt}%
  \setlength{\parskip}{0pt plus 0.3pt}%
  \let\item\@idxitem
}{%
  \clearpage
}
\makeatother

\IfFileExists{\jobname-pw.ind}{\input{\jobname-pw.ind}}{}

\end{document}

      