%% latex-leseansicht-vorspann.tex
%% Vorspann für die Leseansicht.
%% Lädt die gemeinsame Datei latex-vorspann.tex mit nicht gesetztem Schalter.

\newif\ifkorrekturansicht
\korrekturansichtfalse

\input{../tex-inputs/latex-vorspann}


         
         \newcommand{\erwaehntePersonen}{Personen: Leopold von Andrian-Werburg, Eberhard von Bodenhausen, Louise Dumont, Gerhart Hauptmann, Ludwig von Hofmann, Josef Kainz, Harry von Kessler, Karl Mayer}
         \newcommand{\erwaehnteOrte}{Orte: Berlin, Hotel Goldener Adler, Innsbruck, Wien}
         \newcommand{\erwaehnteWerke}{Werke: Fuhrmann Henschel}
               \section[Hugo von Hofmannsthal an Arthur Schnitzler, 26. 3. 1899]{ Hugo von Hofmannsthal an Arthur Schnitzler, 26. 3. 1899}\nopagebreak\mylabel{v}\rehead{ }\begin{ledgroupsized}[t]{13cm}\normalsize\beginnumbering \toendnotes[C]{\smallbreak\pagebreak[2]} \Standort{CUL, Schnitzler, B 43.}
\physDesc{Brief, 1 Blatt, 2 Seiten
\newline{}Handschrift: schwarze Tinte, deutsche Kurrent\newline{}Beilage: maschinelles Telegramm nach Berlin\oindex{Berlin@\textbf{Berlin}|pw} 
\newline{}Schnitzler: mit Bleistift datiert: »2\strikeout{9}6/3 99« \newline{}Ordnung: 1) mit Bleistift von unbekannter Hand nummeriert: »\strikeout{143}«  2) mit Bleistift von unbekannter Hand nummeriert: »140«}\buchAbdrucke{\weitereDrucke{Hugo von Hofmannsthal, Arthur Schnitzler: \emph{Briefwechsel}. Hg. Therese Nickl und Heinrich Schnitzler. Frankfurt am Main: \emph{S. Fischer} 1964, S. 121.} }\pstart
           \raggedleft{}{\pb}Berlin\oindex{Berlin@\textbf{Berlin}|pw}{ }Sonntg\pend
           \pstart
           lieber, eben bekomm ich dieſes Telegra{\geminationm} von dem armen Poldy\pwindex{Andrian-Werburg, Leopold von 09.05.1875 – 19.11.1951@\textsc{Andrian-Werburg, Leopold von} (09.05.1875 – 19.11.1951), \emph{Schriftsteller, Diplomat}|pw}. Er bildet ſich dieſmal ein, daſs er wahnſinnig wird. Vielleicht
                    können Sie irgendwas machen.\pend
           \pstart
           Ich ko{\geminationm}e, da Sie nicht herko{\geminationm}en, ſchon ſpäteſtens Samstag nach
                        Wien\oindex{Wien@\textbf{Wien}|pw}.\pend
           \pstart
           Ich ſehe viele Menſchen: Hauptmann\pwindex{Hauptmann, Gerhart 15.11.1862 – 06.06.1946@\textsc{Hauptmann, Gerhart} (15.11.1862 – 06.06.1946), \emph{Schriftsteller}|pw}, Ludwig von Hofmann\pwindex{Hofmann, Ludwig von 1861-08-17 – 1945-08-23@\textsc{Hofmann, Ludwig von} (1861-08-17 – 1945-08-23), \emph{Maler, Grafiker}|pw}, \textsc{Kessler}\pwindex{Kessler, Harry von 23.05.1868 – 04.12.1937@\textsc{Kessler, Harry von} (23.05.1868 – 04.12.1937), \emph{Schriftsteller, Verleger, Diplomat}|pw}, Bodenhauſen\pwindex{Bodenhausen, Eberhard von 12.06.1868 – 06.05.1918@\textsc{Bodenhausen, Eberhard von} (12.06.1868 – 06.05.1918)|pw}, Kainz\pwindex{Kainz, Josef 02.01.1858 – 20.09.1910@\textsc{Kainz, Josef} (02.01.1858 – 20.09.1910), \emph{Schauspieler}|pw}, die Dumont\pwindex{Dumont, Louise 22.02.1862 – 16.05.1932@\textsc{Dumont, Louise} (22.02.1862 – 16.05.1932), \emph{Theaterleiterin, Schauspielerin}|pw}{ }\textsc{etc. etc.} auch viele gute Vorſtellungen, wie Fuhrmann Henſchel\pwindex{Hauptmann, Gerhart 15.11.1862 – 06.06.1946@\textsc{Hauptmann, Gerhart} (15.11.1862 – 06.06.1946), \emph{Schriftsteller}!Fuhrmann Henschel1898@\strich\emph{Fuhrmann Henschel} {[}1898{]}|pw}. {\pb}Bin aber nicht im Stand
                    einen Brief zu ſchreiben.\pend
           \pstart
           Von Herzen Ihr{\\[\baselineskip]}\spacefill\mbox{Hugo.}\pend
           \leftskip=0em{}{\bigskip}\pstart
           \noindent{}{\pb}v insbruck\oindex{Innsbruck@\textbf{Innsbruck}|pw} 3747 31 26/3{ }9 40m\pend
           \pstart
           {[}bef{]}uerchtungen geisteszustand fast eingetroffen bin sofort
                        insbruck\oindex{Innsbruck@\textbf{Innsbruck}|pw} gefahren
                    {[}prof{]}essor meyer\pwindex{Mayer, Karl *~9.12.1862@\textsc{Mayer, Karl} (*~9.12.1862), \emph{Psychiater, Neurologe}|pw}
                    consultiren dieser verreist. bitte wenn kannst sofort herkommen wo ist schnitzler\pwindex{Schnitzler, Arthur 15.05.1862 – 21.10.1931@\textsc{Schnitzler, Arthur} (15.05.1862 – 21.10.1931), \emph{Schriftsteller, Mediziner}|pw}? = poldi\pwindex{Andrian-Werburg, Leopold von 09.05.1875 – 19.11.1951@\textsc{Andrian-Werburg, Leopold von} (09.05.1875 – 19.11.1951), \emph{Schriftsteller, Diplomat}|pw}{ }goldner adler\oindex{Hotel Goldener Adler@\textbf{Hotel Goldener Adler}|pw}.+=\pend
           
         
         \endnumbering\mylabel{h}\end{ledgroupsized}  \newcommand{\dateiname}{L00911}\newcommand{\titel}{Hugo von Hofmannsthal an Arthur Schnitzler, 26. 3. 1899}\newcommand{\editorInnen}{Martin Anton Müller und Gerd-Hermann Susen}%% latex-leseansicht-abspann.tex
%% Abspann für die Leseansicht.
%% Der Schalter \ifkorrekturansicht ist bereits durch den Vorspann gesetzt.

%% latex-abspann.tex
%% Gemeinsamer Abspann für Korrekturansicht und Leseansicht.
%% Setzt den Schalter \ifkorrekturansicht voraus (gesetzt in den
%% einbindenden Dateien latex-korrekturansicht-abspann.tex bzw.
%% latex-leseansicht-abspann.tex).
%% ---------------------------------------------------------------

\normalsize

% Das esempio-Environment wird nur in der Leseansicht benötigt
\ifkorrekturansicht\else
\newenvironment{esempio}[3]%
{
    \vspace{1.5ex}
    \rlap{\underline{#1}}
    \par
    \setlength{\parindent}{0cm}
    \nopagebreak
    \leftskip=#2cm
    \rightskip=#3cm
}
{
    \par
}
\fi

\doendnotes{C}
\bigskip
\vfill

\clearpage

\footnotesize

\ifkorrekturansicht
  \lohead{\textsc{register}}
\fi

% theindex-Environment neu definieren ohne reledmac
\makeatletter
\renewenvironment{theindex}{%
  \ifkorrekturansicht
    \section*{\indexname}%
  \else
    \subsubsection*{Index der erwähnten Entitäten}%
  \fi
  \setlength{\parindent}{0pt}%
  \setlength{\parskip}{0pt plus 0.3pt}%
  \let\item\@idxitem
}{%
  \ifkorrekturansicht\clearpage\fi
}
\makeatother

\IfFileExists{\jobname-pw.ind}{\input{\jobname-pw.ind}}{}

% Quellenangabe nur in der Leseansicht
\ifkorrekturansicht\else
% Fallback-Definitionen, falls die .tex-Datei \titel etc. nicht gesetzt hat
\providecommand{\titel}{}
\providecommand{\editorInnen}{}
\providecommand{\dateiname}{\jobname}

\vspace{3cm}

\vfill

\footnotesize
\textsc{Quelle}: \titel. Herausgegeben von {\editorInnen}. In: \emph{Arthur Schnitzler: Briefwechsel mit Autorinnen und Autoren}.
 Digitale Edition, https://schnitzler-briefe.acdh.oeaw.ac.at/{\dateiname}.html (Stand \today)
\fi

\end{document}


      