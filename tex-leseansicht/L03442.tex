%% latex-leseansicht-vorspann.tex
%% Vorspann für die Leseansicht.
%% Lädt die gemeinsame Datei latex-vorspann.tex mit nicht gesetztem Schalter.

\newif\ifkorrekturansicht
\korrekturansichtfalse

\input{../tex-inputs/latex-vorspann}

\begin{center}
            \textcolor{red}{ENTWURF, NICHT FERTIG KORRIGIERT}
                      \end{center}
            
         
         \renewcommand{\erwaehntePersonen}{Personen: Charlotte Bondy, Fritz Freund, Theodore Rottenberg, Olga Schnitzler, Heinrich Schnitzler, Moritz von Schwind, Giovanni Battista Tiepolo}
         \renewcommand{\erwaehnteInstitutionen}{Institutionen: Wiener Verlag}
         \renewcommand{\erwaehnteOrte}{Orte: Bamberg, Domberg (Bamberg), Domplatz (Bamberg), Eisenach, Hotel Marienbad, München, Opletalova, Prag, Regensburg, Regensburger Dom, Sizilien, Wartburg, Weimar, Wien, Würzburg}
         \renewcommand{\erwaehnteWerke}{Werke: Deckenfresko im Treppenhaus der Würzburger Residenz, Reigen. Zehn Dialoge, Schwindsche Wartburgfresken}
               \section[ Paul Goldmann an Arthur Schnitzler, 8. 4. {[}1904{]}]{ Paul Goldmann an Arthur Schnitzler, 8. 4. {[}1904{]}}\nopagebreak\mylabel{v}\rehead{ }\begin{ledgroupsized}[t]{13cm}\normalsize\beginnumbering \toendnotes[C]{\smallbreak\pagebreak[2]} \Standort{DLA, A:Schnitzler, HS.NZ85.1.3174.}
\physDesc{Brief, 1 Blatt, 4 Seiten, 2232 Zeichen
\newline{}Handschrift: schwarze Tinte, deutsche Kurrent
\newline{}Schnitzler: 1) mit Bleistift das Jahr »904« vermerkt  2) mit rotem Buntstift eine Unterstreichung}\toendnotes[C]{\smallbreak}\pstart
           \raggedleft{}{\pb}München\oindex{Muenchen@\textbf{München}|pw}{ }8. April.\pend
           \pstart{}Mein lieber Freund,\pend\pstart
           Dein lieber Brief (mit dem ich mich ſehr gefreut habe) und Deine Karte wurden mir
               hierher nachgeſandt (\label{K_L03442-1v}\edtext{Frau \textsc{Bondy\pwindex{Bondy, Charlotte 25.03.1854 – 1914-03-07@\textsc{Bondy, Charlotte} (25.03.1854 – 1914-03-07), \emph{Schauspielerin}|pw}}: \textsc{Prag\oindex{Prag@\textbf{Prag}|pw}}, \textsc{Mariengaſse} 45\oindex{Opletalova@\textbf{Opletalova}|pw}}{\lemma{\textnormal{\emph{Frau … Mariengaſse 45}}}\Cendnote{\textnormal{Schnitzler\pwindex{Schnitzler, Arthur 15.05.1862 – 21.10.1931@\textsc{Schnitzler, Arthur} (15.05.1862 – 21.10.1931), \emph{Schriftsteller, Mediziner}|pwk} 
                     dürfte in der Karte nach ihrer Adresse gefragt haben.}}}\label{K_L03442-1h}). Ich habe eine kleine
               Erholungsreiſe gemacht, bei der ich mich freilich wenig erholt habe. Ins Gebirge
               konnte ich nicht wegen des ſchlechten Wetters. So bin ich in Etappen nach München\oindex{Muenchen@\textbf{München}|pw} gefahren: Weimar\oindex{Weimar@\textbf{Weimar}|pw}, Eiſenach\oindex{Eisenach@\textbf{Eisenach}|pw} (mit der reizend
               gelegenen und wegen der Fresken\pwindex{Schwind, Moritz von 21.01.1804 – 08.02.1871@\textsc{Schwind, Moritz von} (21.01.1804 – 08.02.1871), \emph{Maler}!Schwindsche Wartburgfresken1854 – 1855@\strich\emph{Schwindsche Wartburgfresken} {[}1854 – 1855{]}|pwv}{ }\textsc{Schwindt\pwindex{Schwind, Moritz von 21.01.1804 – 08.02.1871@\textsc{Schwind, Moritz von} (21.01.1804 – 08.02.1871), \emph{Maler}|pw}s} überaus ſehenswerthen Wartburg\oindex{Wartburg@\textbf{Wartburg}|pw}), Würzburg\oindex{Wuerzburg@\textbf{Würzburg}|pw} (herrliche Fresken\pwindex{Tiepolo, Giovanni Battista 05.03.1696 – 27.03.1770@\textsc{Tiepolo, Giovanni Battista} (05.03.1696 – 27.03.1770)!Deckenfresko im Treppenhaus der Wuerzburger Residenz1752 – 1753@\strich\emph{Deckenfresko im Treppenhaus der Würzburger Residenz} {[}1752 – 1753{]}|pwv} von \textsc{Tiepolo\pwindex{Tiepolo, Giovanni Battista 05.03.1696 – 27.03.1770@\textsc{Tiepolo, Giovanni Battista} (05.03.1696 – 27.03.1770)|pw}}), Bamberg\oindex{Bamberg@\textbf{Bamberg}|pw}\strikeout{,} (ein großartiger Domplatz\oindex{Domplatz (Bamberg)@\textbf{Domplatz (Bamberg)}|pw} auf
               einem Berge\oindex{Domberg (Bamberg)@\textbf{Domberg (Bamberg)}|pwv}), Regensburg\oindex{Regensburg@\textbf{Regensburg}|pw}{ }{\pb}(ſchöner gothiſcher Dom\oindex{Regensburger Dom@\textbf{Regensburger Dom}|pw}) und München\oindex{Muenchen@\textbf{München}|pw}. Ich
               wohne wieder im \textsc{Hotel Marienbad\oindex{Hotel Marienbad@\textbf{Hotel Marienbad}|pw}} und gedenke \strikeout{D\textcolor{gray}{ein}} der ſchönen Tage, die wir \label{K_L03442-2v}\edtext{vor
                  Jahren}{\lemma{\textnormal{\emph{vor
                  Jahren}}}\Cendnote{\textnormal{zwischen 28. 8. 1895 und 6. 9. 1895}}}\label{K_L03442-2h}
               hier verbracht haben.\pend
           \pstart
           Daß das \label{K_L03442-3v}\edtext{Verbot des »Reigen\pwindex{Schnitzler, Arthur 15.05.1862 – 21.10.1931@\textsc{Schnitzler, Arthur} (15.05.1862 – 21.10.1931), \emph{Schriftsteller, Mediziner}!Reigen. Zehn Dialoge1900@\strich\emph{Reigen. Zehn Dialoge} {[}1900{]}|pw}«}{\lemma{\textnormal{\emph{Verbot des »Reigen«}}}\Cendnote{\textnormal{siehe Paul Goldmann an Arthur Schnitzler, 19. 3. [1904]}}}\label{K_L03442-3h} Dir keinen Schaden gethan hat, freut mich ſehr. Auch haſt Du ganz Recht, daß
               Du vorläufig in der Öffentlichkeit nichts darüber verlauten laſſen willſt. Wenn es
               zum Prozeß kommen ſollte, wird dazu immer noch Zeit ſein, – falls es überhaupt
               nothwendig werden ſollte. Immerhin iſt es wichtig, daß in dem Prozeß Dein Verleger\pwindex{Freund, Fritz 07.04.1879 – 08.05.1950@\textsc{Freund, Fritz} (07.04.1879 – 08.05.1950), \emph{Verleger}|pw}\orgindex{Wiener Verlag@Wiener Verlag|pwv} durch einen tüchtigen Anwalt vertreten wird, der \strikeout{i\textcolor{gray}{m Sta}} fähig {\pb}iſt, die Angelegenheit von einem
               höheren Standpunkte aus zu erörtern.\pend
           \pstart
           Eure\pwindex{Schnitzler, Olga 17.01.1882 – 13.01.1970@\textsc{Schnitzler, Olga} (17.01.1882 – 13.01.1970), \emph{Schauspielerin, Sängerin}|pwv}{ }\label{K_L03442-4v}\edtext{Frühjahrsreiſe}{\lemma{\textnormal{\emph{Frühjahrsreiſe}}}\Cendnote{\textnormal{siehe Paul Goldmann an Arthur Schnitzler, 14. 3. [1904]}}}\label{K_L03442-4h} nach Sizilien\oindex{Sizilien@\textbf{Sizilien}|pw} wird ſehr ſchön werden.
               Durch den Aufſchub iſt Euch das ſchlechte Wetter erſpart geblieben. Ich wünſche Euch
               den ſchönſten Sonnenſchein{[}.{]} Nur follteſt Du länger als einen
               Monat bleiben. In vier Wochen iſt die Reiſe vielleicht etwas anſtrengend.\pend
           \pstart
           Meiner Freundin\pwindex{Rottenberg, Theodore 1875-09-07 – 1945-04-05@\textsc{Rottenberg, Theodore} (1875-09-07 – 1945-04-05)|pwv} geht es,
               nachdem die drohende \label{K_L03442-5v}\edtext{Gefahr}{\lemma{\textnormal{\emph{Gefahr}}}\Cendnote{\textnormal{siehe Paul Goldmann an Arthur Schnitzler, 14. 3. [1904]}}}\label{K_L03442-5h}{ }\strikeout{abg} glücklich abgewendet iſt, recht gut. Sie hat mir
                  \strikeout{mehrsm} mehrmals Grüße für Dich aufgetragen. Wie
               ſich unſere Zukunft geſtalten wird, weiß Gott allein. Wenn {\pb}\strikeout{ſ\textcolor{gray}{ie}} ich ſie\pwindex{Rottenberg, Theodore 1875-09-07 – 1945-04-05@\textsc{Rottenberg, Theodore} (1875-09-07 – 1945-04-05)|pwv} nicht habe,
               wie jetzt, ſo ſehne ich mich nach ihr; war ich aber vier Wochen mit ihr zuſammen, ſo
               habe ich, wenn ſie wegfährt, ein Gefühl, \strikeout{a\textcolor{gray}{bſ}} der Freiheit. Es ſcheint, daß man von einer Frau niemals gerade ſo viel hat,
               als man \substVorne{}\textsuperscript{\textcolor{gray}{will},}\substDazwischen{}braucht,\substHinten{} ſondern immer nur entweder zu wenig oder zu viel.\pend
           \pstart
           Ich leide ſeit einer Woche an Kopfſchmerzen, die ich mir durch Zuviel-Sehen und
               Zuviel-Herumreiſen zugezogen habe. Nimm’ Dir ein warnendes Beiſpiel für Sizilien\oindex{Sizilien@\textbf{Sizilien}|pw}!\pend
           \pstart
           Schreib’ mir bald wieder und ſei, ſammt Frau\pwindex{Schnitzler, Olga 17.01.1882 – 13.01.1970@\textsc{Schnitzler, Olga} (17.01.1882 – 13.01.1970), \emph{Schauspielerin, Sängerin}|pwv} und Kind\pwindex{Schnitzler, Heinrich 09.08.1902 – 12.07.1982@\textsc{Schnitzler, Heinrich} (09.08.1902 – 12.07.1982), \emph{Regisseur, Schauspieler}|pwv} (was macht Heinrich\pwindex{Schnitzler, Heinrich 09.08.1902 – 12.07.1982@\textsc{Schnitzler, Heinrich} (09.08.1902 – 12.07.1982), \emph{Regisseur, Schauspieler}|pw}?) herzlichſt
               gegrüßt von Deinem getreuen {\\[\baselineskip]}\spacefill\mbox{Paul Goldmann}\pend
           \leftskip=0em{}
         
         \endnumbering\mylabel{h}\end{ledgroupsized}  \newcommand{\dateiname}{L03442}\newcommand{\titel}{Paul Goldmann an Arthur Schnitzler, 8. 4. [1904]}\newcommand{\editorInnen}{Martin Anton Müller und Laura Untner}%% latex-leseansicht-abspann.tex
%% Abspann für die Leseansicht.
%% Der Schalter \ifkorrekturansicht ist bereits durch den Vorspann gesetzt.

%% latex-abspann.tex
%% Gemeinsamer Abspann für Korrekturansicht und Leseansicht.
%% Setzt den Schalter \ifkorrekturansicht voraus (gesetzt in den
%% einbindenden Dateien latex-korrekturansicht-abspann.tex bzw.
%% latex-leseansicht-abspann.tex).
%% ---------------------------------------------------------------

\normalsize

% Das esempio-Environment wird nur in der Leseansicht benötigt
\ifkorrekturansicht\else
\newenvironment{esempio}[3]%
{
    \vspace{1.5ex}
    \rlap{\underline{#1}}
    \par
    \setlength{\parindent}{0cm}
    \nopagebreak
    \leftskip=#2cm
    \rightskip=#3cm
}
{
    \par
}
\fi

\doendnotes{C}
\bigskip
\vfill

\clearpage

\footnotesize

\ifkorrekturansicht
  \lohead{\textsc{register}}
\fi

% theindex-Environment neu definieren ohne reledmac
\makeatletter
\renewenvironment{theindex}{%
  \ifkorrekturansicht
    \section*{\indexname}%
  \else
    \subsubsection*{Index der erwähnten Entitäten}%
  \fi
  \setlength{\parindent}{0pt}%
  \setlength{\parskip}{0pt plus 0.3pt}%
  \let\item\@idxitem
}{%
  \ifkorrekturansicht\clearpage\fi
}
\makeatother

\IfFileExists{\jobname-pw.ind}{\input{\jobname-pw.ind}}{}

% Quellenangabe nur in der Leseansicht
\ifkorrekturansicht\else
% Fallback-Definitionen, falls die .tex-Datei \titel etc. nicht gesetzt hat
\providecommand{\titel}{}
\providecommand{\editorInnen}{}
\providecommand{\dateiname}{\jobname}

\vspace{3cm}

\vfill

\footnotesize
\textsc{Quelle}: \titel. Herausgegeben von {\editorInnen}. In: \emph{Arthur Schnitzler: Briefwechsel mit Autorinnen und Autoren}.
 Digitale Edition, https://schnitzler-briefe.acdh.oeaw.ac.at/{\dateiname}.html (Stand \today)
\fi

\end{document}


      