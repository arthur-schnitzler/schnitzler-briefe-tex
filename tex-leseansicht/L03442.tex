%% latex-leseansicht-vorspann.tex
%% Vorspann für die Leseansicht.
%% Lädt die gemeinsame Datei latex-vorspann.tex mit nicht gesetztem Schalter.

\newif\ifkorrekturansicht
\korrekturansichtfalse

\input{../tex-inputs/latex-vorspann}


\section[ Paul Goldmann an Arthur Schnitzler, 8. 4. [1904]]{L03442 Paul Goldmann an Arthur Schnitzler,  8. 4. [1904]}
\nopagebreak\mylabel{L03442v}
\rehead{ }\normalsize\beginnumbering\briefempfaengerindex{Schnitzler, Arthur@\textsc{Schnitzler, Arthur}!zzzGoldmann, Paul@\emph{von Paul Goldmann}!1904-04-081@{8. 4. [1904]}|(be}
\toendnotes[C]{\smallbreak\pagebreak[2]}
\correspDesc{Versand  durch Paul Goldmann am 8. 4. [1904] in München
\newline{}Erhalt  durch Arthur Schnitzler im Zeitraum [9. 4. 1904
                  – 13. 4. 1904?] in Wien}\toendnotes[C]{\smallbreak}
\Standort{DLA, A:Schnitzler, HS.NZ85.1.3174.}
\physDesc{Brief, 1 Blatt, 4 Seiten, 2231 Zeichen
\newline{}Handschrift: schwarze Tinte, deutsche Kurrent
\newline{}Schnitzler: 1) mit Bleistift das Jahr »904« vermerkt  2) mit rotem Buntstift eine Unterstreichung}\toendnotes[C]{\smallbreak}
\pstart
           \raggedleft{}{\pb}München\oindex{München@\textbf{München}|pw}{ }8. April.\pend
           
\pstart{}Mein lieber Freund,\pend\vspace{0.5em}
\pstart
           Dein lieber Brief (mit dem ich mich{ }ſehr gefreut habe) und Deine Karte wurden mir
               hierher nachgeſandt (\label{K_L03442-1v}\edtext{Frau \textsc{Bondy\pwindex{Bondy, Charlotte 25.\,3.\,1854 Bielsko-Biała – 7.\,3.\,1914 Prag@\textsc{Bondy, Charlotte} (25.\,3.\,1854 Bielsko-Biała – 7.\,3.\,1914 Prag), \emph{Schauspielerin}|pw}}: \textsc{Prag\oindex{Prag@\textbf{Prag}, \emph{Land}|pw}}, \textsc{Mariengaſse} 45\oindex{Opletalova@\textbf{Opletalova}, \emph{Straße}|pw}}{\lemma{\textnormal{\emph{Frau … Mariengasse 45}}}\Cendnote{\textnormal{Schnitzler dürfte in der Karte nach ihrer
                  Adresse gefragt haben.}}}\label{K_L03442-1}). Ich habe eine kleine Erholungsreiſe gemacht, bei
               der ich mich freilich wenig erholt habe. Ins Gebirge konnte ich nicht wegen des{ }ſchlechten Wetters. So bin ich in Etappen nach München\oindex{München@\textbf{München}|pw} gefahren: Weimar\oindex{Weimar@\textbf{Weimar}, \emph{Verwaltungsgebiet}|pw}, Eiſenach\oindex{Eisenach@\textbf{Eisenach}|pw} (mit der reizend gelegenen und wegen
               der Fresken\pwindex{Schwind, Moritz von 21.\,1.\,1804 Wien – 8.\,2.\,1871 München@\textsc{Schwind, Moritz von} (21.\,1.\,1804 Wien – 8.\,2.\,1871 München), \emph{Maler}!Schwindsche Wartburgfresken@\strich\emph{Schwindsche Wartburgfresken}|pwv}{ }\textsc{Schwindts\pwindex{Schwind, Moritz von 21.\,1.\,1804 Wien – 8.\,2.\,1871 München@\textsc{Schwind, Moritz von} (21.\,1.\,1804 Wien – 8.\,2.\,1871 München), \emph{Maler}|pw}} überaus{ }ſehenswerthen Wartburg\oindex{Wartburg@\textbf{Wartburg}, \emph{Burg}|pw}), Würzburg\oindex{Würzburg@\textbf{Würzburg}, \emph{Hauptstadt}|pw} (herrliche Fresken\pwindex{Tiepolo, Giovanni Battista 5.\,3.\,1696 Venedig – 27.\,3.\,1770 Madrid@\textsc{Tiepolo, Giovanni Battista} (5.\,3.\,1696 Venedig – 27.\,3.\,1770 Madrid)!Deckenfresko im Treppenhaus der Würzburger Residenz@\strich\emph{Deckenfresko im Treppenhaus der Würzburger Residenz}|pwv} von \textsc{Tiepolo\pwindex{Tiepolo, Giovanni Battista 5.\,3.\,1696 Venedig – 27.\,3.\,1770 Madrid@\textsc{Tiepolo, Giovanni Battista} (5.\,3.\,1696 Venedig – 27.\,3.\,1770 Madrid)|pw}}), Bamberg\oindex{Bamberg@\textbf{Bamberg}, \emph{Hauptstadt}|pw}\strikeout{,} (ein großartiger Domplatz\oindex{Domplatz [Bamberg]@\textbf{Domplatz [Bamberg]}, \emph{Platz}|pw} auf einem Berge\oindex{Domberg [Bamberg]@\textbf{Domberg [Bamberg]}, \emph{Berg}|pwv}), Regensburg\oindex{Regensburg@\textbf{Regensburg}, \emph{Hauptstadt}|pw}{ }{\pb}(ſchöner gothiſcher Dom\oindex{Regensburger Dom@\textbf{Regensburger Dom}, \emph{Kirche}|pw}) und München\oindex{München@\textbf{München}|pw}. Ich
               wohne wieder im \textsc{Hotel Marienbad\oindex{Hotel Marienbad [München]@\textbf{Hotel Marienbad [München]}, \emph{Hotel}|pw}} und gedenke \strikeout{Dein} der{ }ſchönen Tage, die wir
                  \label{K_L03442-2v}\edtext{vor Jahren}{\lemma{\textnormal{\emph{vor Jahren}}}\Cendnote{\textnormal{zwischen 28. 8. 1895 und 6. 9. 1895}}}\label{K_L03442-2} hier verbracht haben.\pend
           
\pstart
           Daß das \label{K_L03442-3v}\edtext{Verbot des »Reigen\pwindex{Schnitzler, Arthur 15.\,5.\,1862 Wien – 21.\,10.\,1931 ebd.@\textsc{Schnitzler, Arthur} (15.\,5.\,1862 Wien – 21.\,10.\,1931 ebd.), \emph{Schriftsteller, Mediziner}!Reigen. Zehn Dialoge@\strich\emph{Reigen. Zehn Dialoge}|pw}«}{\lemma{\textnormal{\emph{Verbot des »Reigen«}}}\Cendnote{\textnormal{Siehe XXXX Auszeichnungsfehler: Dokument L03441 nicht gefunden.
               }}}\label{K_L03442-3} Dir keinen Schaden gethan hat, freut mich{ }ſehr. Auch haſt Du ganz Recht, daß
               Du vorläufig in der Öffentlichkeit nichts darüber verlauten laſſen willſt. Wenn es
               zum Prozeß kommen{ }ſollte, wird dazu immer noch Zeit{ }ſein, – falls es überhaupt
               nothwendig werden{ }ſollte. Immerhin iſt es wichtig, daß in dem Prozeß Dein Verleger\pwindex{Freund, Fritz 7.\,4.\,1879 Wien – 8.\,5.\,1950 ebd.@\textsc{Freund, Fritz} (7.\,4.\,1879 Wien – 8.\,5.\,1950 ebd.), \emph{Verleger}|pw}\orgindex{Wiener Verlag@Wiener Verlag|pwv} durch einen tüchtigen Anwalt vertreten wird, der \strikeout{i\textcolor{gray}{m} S\textcolor{gray}{ta}} fähig {\pb}iſt, die Angelegenheit von einem höheren Standpunkte
               aus zu erörtern.\pend
           
\pstart
           Eure\pwindex{Schnitzler, Olga 17.\,1.\,1882 Wien – 13.\,1.\,1970 Lugano@\textsc{Schnitzler, Olga} (17.\,1.\,1882 Wien – 13.\,1.\,1970 Lugano), \emph{Schauspielerin, Sängerin}|pwv}{ }\label{K_L03442-4v}\edtext{Frühjahrsreiſe}{\lemma{\textnormal{\emph{Frühjahrsreise}}}\Cendnote{\textnormal{Siehe XXXX Auszeichnungsfehler: Dokument L03440 nicht gefunden.
               }}}\label{K_L03442-4} nach Sizilien\oindex{Sizilien@\textbf{Sizilien}, \emph{Land}|pw} wird{ }ſehr{ }ſchön werden.
               Durch den Aufſchub iſt Euch das{ }ſchlechte Wetter erſpart geblieben. Ich wünſche Euch
               den{ }ſchönſten Sonnenſchein{[}.{]} Nur follteſt Du länger als einen
               Monat bleiben. In vier Wochen iſt die Reiſe vielleicht etwas anſtrengend.\pend
           
\pstart
           Meiner Freundin\pwindex{Rottenberg, Theodore 7.\,9.\,1875 – 5.\,4.\,1945 Limburg an der Lahn@\textsc{Rottenberg, Theodore} (7.\,9.\,1875 – 5.\,4.\,1945 Limburg an der Lahn)|pwv} geht es,
               nachdem \strikeout{\textcolor{gray}{d}} die drohende \label{K_L03442-5v}\edtext{Gefahr}{\lemma{\textnormal{\emph{Gefahr}}}\Cendnote{\textnormal{Siehe XXXX Auszeichnungsfehler: Dokument L03440 nicht gefunden.
               }}}\label{K_L03442-5}{ }\strikeout{abg} glücklich abgewendet iſt, recht gut. Sie hat mir
                  \strikeout{mehrsm} mehrmals Grüße für Dich aufgetragen. Wie{ }ſich unſere Zukunft geſtalten wird, weiß Gott allein. Wenn {\pb}\strikeout{ſ\textcolor{gray}{ie}} ich ſie\pwindex{Rottenberg, Theodore 7.\,9.\,1875 – 5.\,4.\,1945 Limburg an der Lahn@\textsc{Rottenberg, Theodore} (7.\,9.\,1875 – 5.\,4.\,1945 Limburg an der Lahn)|pwv} nicht habe,
               wie jetzt,{ }ſo{ }ſehne ich mich nach ihr; war ich aber vier Wochen mit ihr zuſammen,{ }ſo
               habe ich, wenn{ }ſie wegfährt, ein Gefühl, \strikeout{a\textcolor{gray}{lt}} der Freiheit. Es{ }ſcheint, daß man von einer Frau niemals gerade{ }ſo viel hat,
               als man \substVorne{}\textsuperscript{\textcolor{gray}{will},}\substDazwischen{}braucht,\substHinten{}{ }ſondern immer nur entweder zu wenig oder zu viel.\pend
           
\pstart
           Ich leide{ }ſeit einer Woche an Kopfſchmerzen, die ich mir durch Zuviel-Sehen und
               Zuviel-Herumreiſen zugezogen habe. Nimm’ Dir ein warnendes Beiſpiel für Sizilien\oindex{Sizilien@\textbf{Sizilien}, \emph{Land}|pw}!\pend
           
\pstart
           Schreib’ mir bald wieder und{ }ſei,{ }ſammt Frau\pwindex{Schnitzler, Olga 17.\,1.\,1882 Wien – 13.\,1.\,1970 Lugano@\textsc{Schnitzler, Olga} (17.\,1.\,1882 Wien – 13.\,1.\,1970 Lugano), \emph{Schauspielerin, Sängerin}|pwv} und Kind\pwindex{Schnitzler, Heinrich 9.\,8.\,1902 Hinterbrühl – 12.\,7.\,1982 Wien@\textsc{Schnitzler, Heinrich} (9.\,8.\,1902 Hinterbrühl – 12.\,7.\,1982 Wien), \emph{Regisseur, Schauspieler}|pwv} (was macht \textsc{Heinrich\pwindex{Schnitzler, Heinrich 9.\,8.\,1902 Hinterbrühl – 12.\,7.\,1982 Wien@\textsc{Schnitzler, Heinrich} (9.\,8.\,1902 Hinterbrühl – 12.\,7.\,1982 Wien), \emph{Regisseur, Schauspieler}|pw}}?) herzlichſt gegrüßt von Deinem getreuen {\\[\baselineskip]}\spacefill\mbox{Paul Goldmann}\pend
           \leftskip=0em{}\selectlanguage{ngerman}\endnumbering\briefempfaengerindex{Schnitzler, Arthur@\textsc{Schnitzler, Arthur}!zzzGoldmann, Paul@\emph{von Paul Goldmann}!1904-04-081@{8. 4. [1904]}|)be}\mylabel{L03442h}  \newcommand{\dateiname}{L03442}\newcommand{\titel}{Paul Goldmann an Arthur Schnitzler, 8. 4. [1904]}\newcommand{\editorInnen}{Martin Anton Müller und Laura Untner}%% latex-leseansicht-abspann.tex
%% Abspann für die Leseansicht.
%% Der Schalter \ifkorrekturansicht ist bereits durch den Vorspann gesetzt.

%% latex-abspann.tex
%% Gemeinsamer Abspann für Korrekturansicht und Leseansicht.
%% Setzt den Schalter \ifkorrekturansicht voraus (gesetzt in den
%% einbindenden Dateien latex-korrekturansicht-abspann.tex bzw.
%% latex-leseansicht-abspann.tex).
%% ---------------------------------------------------------------

\normalsize

% Das esempio-Environment wird nur in der Leseansicht benötigt
\ifkorrekturansicht\else
\newenvironment{esempio}[3]%
{
    \vspace{1.5ex}
    \rlap{\underline{#1}}
    \par
    \setlength{\parindent}{0cm}
    \nopagebreak
    \leftskip=#2cm
    \rightskip=#3cm
}
{
    \par
}
\fi

\doendnotes{C}
\bigskip
\vfill

\clearpage

\footnotesize

\ifkorrekturansicht
  \lohead{\textsc{register}}
\fi

% theindex-Environment neu definieren ohne reledmac
\makeatletter
\renewenvironment{theindex}{%
  \ifkorrekturansicht
    \section*{\indexname}%
  \else
    \subsubsection*{Index der erwähnten Entitäten}%
  \fi
  \setlength{\parindent}{0pt}%
  \setlength{\parskip}{0pt plus 0.3pt}%
  \let\item\@idxitem
}{%
  \ifkorrekturansicht\clearpage\fi
}
\makeatother

\IfFileExists{\jobname-pw.ind}{\input{\jobname-pw.ind}}{}

% Quellenangabe nur in der Leseansicht
\ifkorrekturansicht\else
% Fallback-Definitionen, falls die .tex-Datei \titel etc. nicht gesetzt hat
\providecommand{\titel}{}
\providecommand{\editorInnen}{}
\providecommand{\dateiname}{\jobname}

\vspace{3cm}

\vfill

\footnotesize
\textsc{Quelle}: \titel. Herausgegeben von {\editorInnen}. In: \emph{Arthur Schnitzler: Briefwechsel mit Autorinnen und Autoren}.
 Digitale Edition, https://schnitzler-briefe.acdh.oeaw.ac.at/{\dateiname}.html (Stand \today)
\fi

\end{document}


