%% latex-leseansicht-vorspann.tex
%% Vorspann für die Leseansicht.
%% Lädt die gemeinsame Datei latex-vorspann.tex mit nicht gesetztem Schalter.

\newif\ifkorrekturansicht
\korrekturansichtfalse

\input{../tex-inputs/latex-vorspann}


\section[Berta Zuckerkandl an Arthur Schnitzler, 27. 7. {[}1931?{]}]{L04007 Berta Zuckerkandl an Arthur Schnitzler, 27. 7. [1931?]}
\nopagebreak\mylabel{L04007v}
\rehead{ }\normalsize\beginnumbering\briefempfaengerindex{Schnitzler, Arthur@\textsc{Schnitzler, Arthur}!zzzZuckerkandl, Berta@\emph{von Berta Zuckerkandl}!1931-07-271@{27. 7. [1931?]}|(be}
\toendnotes[C]{\smallbreak\pagebreak[2]}
\correspDesc{Versand  durch Berta Zuckerkandl am 27. 7. [1931?] in Purkersdorf
\newline{}Erhalt  durch Arthur Schnitzler im Zeitraum [28. 7. 1931
                  – 1. 8. 1931?] in Wien}\toendnotes[C]{\smallbreak}
\Standort{CUL, Schnitzler, B 200.}
\physDesc{Brief, 1 Blatt, 3 Seiten, 1138 Zeichen
\newline{}Handschrift: schwarze Tinte, lateinische Kurrent}\toendnotes[C]{\smallbreak}
\pstart
           {\pb}\textcolor{gray}{\textbf{SANATORIUM WESTEND}}\oindex{Sanatorium Purkersdorf@\textbf{Sanatorium Purkersdorf}, \emph{Sanatorium}|pw}\pend
           
\pstart
           \textcolor{gray}{\textbf{Tel. (Wiener\oindex{Wien@\textbf{Wien}, \emph{Verwaltungsgebiet}|pw} Netz)
                     R-33-5-65}}\pend
           
\pstart
           \textcolor{gray}{\textbf{PURKERSDORF BEI WIEN}}\oindex{Purkersdorf@\textbf{Purkersdorf}, \emph{Verwaltungsgebiet}|pw}\hfill \textcolor{gray}{\textbf{Purkersdorf\oindex{Purkersdorf@\textbf{Purkersdorf}, \emph{Verwaltungsgebiet}|pw}, am}}{ }27. 7.\pend
           
\pstart
           \textcolor{gray}{\textbf{Modernste Heilanstalt für innere, nervöse und
                        Stoffwechsel-Krankheiten}}\oindex{Sanatorium Purkersdorf@\textbf{Sanatorium Purkersdorf}, \emph{Sanatorium}|pw}\pend
           \vspace{0.5em}
\pstart
           Verehrter Freund! Mit grosser Freude höre ich, dass Sie sich am Semmering\oindex{Semmering@\textbf{Semmering}, \emph{Verwaltungsgebiet}|pw} wol befinden. Ihre Schwägerin Schnitzler\pwindex{Schnitzler, Helene 16.\,7.\,1871 Budapest – September 1941 Atlantischer Ozean@\textsc{Schnitzler, Helene} (16.\,7.\,1871 Budapest – September 1941 Atlantischer Ozean)|pw} hat es meiner Schwägerin Redlich\pwindex{Redlich, Amalie 18.\,4.\,1868 Budapest – 1941 Łódź@\textsc{Redlich, Amalie} (18.\,4.\,1868 Budapest – 1941 Łódź), \emph{Kunstsammlerin}|pw} gesagt. Es ist aber auch ein \label{K_L04007-1v}\edtext{herrlicher Sommer}{\lemma{\textnormal{\emph{herrlicher Sommer}}}\Cendnote{\textnormal{Das Korrespondenzstück trägt keine Jahreszahl. Seit Beginn
                  der näheren Bekanntschaft von Schnitzler und
                     Zuckerkandl\pwindex{Zuckerkandl, Berta 13.\,4.\,1864 Wien – 16.\,10.\,1945 Paris@\textsc{Zuckerkandl, Berta} (13.\,4.\,1864 Wien – 16.\,10.\,1945 Paris), \emph{Schriftstellerin, Journalistin, Übersetzerin}|pwk} im Jahr 1911
                  verbrachte Schnitzler nur im Jahr
                     1931 einen Zeitraum von mehreren Tagen im Monat Juli am Semmering\oindex{Semmering@\textbf{Semmering}, \emph{Verwaltungsgebiet}|pwk}. In diesem Jahr, auf das der Brief zu
                  datieren ist, reiste er am 16. 7. 1931 an und blieb bist zum 28. 7. 1931.}}}\label{K_L04007-1}. Als ob der liebe Gott uns
               für den Weltgerisl ein wenig entschädigen möchte.\pend
           
\pstart
           Ich bin hier \label{K_L04007-2v}\edtext{bei den Kindern\pwindex{Zuckerkandl, Fritz 30.\,7.\,1895 Wien – 14.\,12.\,1983 Krattigen@\textsc{Zuckerkandl, Fritz} (30.\,7.\,1895 Wien – 14.\,12.\,1983 Krattigen), \emph{Chemiker}|pwv}\pwindex{Zuckerkandl, Gertrude 18.\,9.\,1895 Wien – 13.\,7.\,1981 Paris@\textsc{Zuckerkandl, Gertrude} (18.\,9.\,1895 Wien – 13.\,7.\,1981 Paris), \emph{Malerin}|pwv}\pwindex{Zuckerkandl, Emile 4.\,7.\,1922 Wien – 9.\,11.\,2013 Palo Alto@\textsc{Zuckerkandl, Emile} (4.\,7.\,1922 Wien – 9.\,11.\,2013 Palo Alto), \emph{Biologe}|pwv}}{\lemma{\textnormal{\emph{bei den Kindern}}}\Cendnote{\textnormal{Zuckerkandls\pwindex{Zuckerkandl, Berta 13.\,4.\,1864 Wien – 16.\,10.\,1945 Paris@\textsc{Zuckerkandl, Berta} (13.\,4.\,1864 Wien – 16.\,10.\,1945 Paris), \emph{Schriftstellerin, Journalistin, Übersetzerin}|pwk} Sohn Fritz\pwindex{Zuckerkandl, Fritz 30.\,7.\,1895 Wien – 14.\,12.\,1983 Krattigen@\textsc{Zuckerkandl, Fritz} (30.\,7.\,1895 Wien – 14.\,12.\,1983 Krattigen), \emph{Chemiker}|pwk} und seine Frau seine Frau Gertrude\pwindex{Zuckerkandl, Gertrude 18.\,9.\,1895 Wien – 13.\,7.\,1981 Paris@\textsc{Zuckerkandl, Gertrude} (18.\,9.\,1895 Wien – 13.\,7.\,1981 Paris), \emph{Malerin}|pwk} hatten mit ihrem Sohn Emile\pwindex{Zuckerkandl, Emile 4.\,7.\,1922 Wien – 9.\,11.\,2013 Palo Alto@\textsc{Zuckerkandl, Emile} (4.\,7.\,1922 Wien – 9.\,11.\,2013 Palo Alto), \emph{Biologe}|pwk}{ }1929 ein Haus bezogen auf dem Gelände des Sanatoriums Purkersdorf\oindex{Sanatorium Purkersdorf@\textbf{Sanatorium Purkersdorf}, \emph{Sanatorium}|pwk}, das Berta Zuckerkandls\pwindex{Zuckerkandl, Berta 13.\,4.\,1864 Wien – 16.\,10.\,1945 Paris@\textsc{Zuckerkandl, Berta} (13.\,4.\,1864 Wien – 16.\,10.\,1945 Paris), \emph{Schriftstellerin, Journalistin, Übersetzerin}|pwk} Bruder Viktor Zuckerkandl\pwindex{Zuckerkandl, Victor 11.\,4.\,1851 Győr – 9.\,2.\,1927 Berlin@\textsc{Zuckerkandl, Victor} (11.\,4.\,1851 Győr – 9.\,2.\,1927 Berlin), \emph{Industrieller}|pwk} erworben und gestaltet hatte.}}}\label{K_L04007-2} gut aufgehoben.
               Und tobe mich grossmütterich aus. Arbeite sehr viel, da ich mir sogesagen das
               Tägliche {\pb}verdienen muss. Sollte mir eine
               dieser Arbeiten pünktlich gezahlt werden, so will ich von den 10 A. nach
                  Salzburg\oindex{Salzburg@\textbf{Salzburg}, \emph{Verwaltungsgebiet}|pw}. Gern den 24\textsuperscript{ten} auf 16 Tage nach Gastein\oindex{Bad Gastein@\textbf{Bad Gastein}, \emph{Hauptstadt}|pw}.\pend
           
\pstart
           Vorher aber hätte ich die grosse Sehnsucht Sie zu sehen, und frage an ob Sie noch
                  nächste Woche (so \label{K_L04007-3v}\edtext{zwischen den 3\textsuperscript{ten} und den 5\textsuperscript{ten} August}{\lemma{\textnormal{\emph{zwischen … August}}}\Cendnote{\textnormal{Zu dem Treffen am Semmering\oindex{Semmering@\textbf{Semmering}, \emph{Verwaltungsgebiet}|pwk} kam es nicht, aber man traf sich am 3. 8. 1931
                  stattdessen in Wien\oindex{Wien@\textbf{Wien}, \emph{Verwaltungsgebiet}|pwk}.}}}\label{K_L04007-3}) noch oben sind?
               Die Kinder\pwindex{Zuckerkandl, Fritz 30.\,7.\,1895 Wien – 14.\,12.\,1983 Krattigen@\textsc{Zuckerkandl, Fritz} (30.\,7.\,1895 Wien – 14.\,12.\,1983 Krattigen), \emph{Chemiker}|pwv}\pwindex{Zuckerkandl, Gertrude 18.\,9.\,1895 Wien – 13.\,7.\,1981 Paris@\textsc{Zuckerkandl, Gertrude} (18.\,9.\,1895 Wien – 13.\,7.\,1981 Paris), \emph{Malerin}|pwv}\pwindex{Zuckerkandl, Emile 4.\,7.\,1922 Wien – 9.\,11.\,2013 Palo Alto@\textsc{Zuckerkandl, Emile} (4.\,7.\,1922 Wien – 9.\,11.\,2013 Palo Alto), \emph{Biologe}|pwv} werden nämlich einen Tagesausflug auf die Rax\oindex{Rax@\textbf{Rax}, \emph{Berg}|pw} machen. In den Stunden wo sie oben sind, könnte ich über
               Tisch zu Ih{\pb}nen fahren da ich ein Auto
               haben werde. Hoffentlich ist auch Frau
                  Pollatscheck\pwindex{Pollaczek, Clara Katharina 15.\,1.\,1875 Wien – 22.\,7.\,1951 ebd.@\textsc{Pollaczek, Clara Katharina} (15.\,1.\,1875 Wien – 22.\,7.\,1951 ebd.), \emph{Schriftstellerin}|pw} die ich herzlichst grüsse bei Ihnen. –\pend
           
\pstart
           Mein projektirter Besuch \label{K_L04007-4v}\edtext{bei Alma\pwindex{Mahler-Werfel, Alma Maria 31.\,8.\,1879 Wien – 11.\,12.\,1964 New York City@\textsc{Mahler-Werfel, Alma Maria} (31.\,8.\,1879 Wien – 11.\,12.\,1964 New York City)|pw}}{\lemma{\textnormal{\emph{bei Alma}}}\Cendnote{\textnormal{Alma Mahler-Werfel\pwindex{Mahler-Werfel, Alma Maria 31.\,8.\,1879 Wien – 11.\,12.\,1964 New York City@\textsc{Mahler-Werfel, Alma Maria} (31.\,8.\,1879 Wien – 11.\,12.\,1964 New York City)|pwk} besaß ein Haus\oindex{Haus Mahler@\textbf{Haus Mahler}, \emph{Gebäude}|pwkv} bei Breitenstein am Semmering\oindex{Breitenstein am Semmering@\textbf{Breitenstein am Semmering}|pwk}. Schnitzler und Clara
                     Katharina Pollaczek\pwindex{Pollaczek, Clara Katharina 15.\,1.\,1875 Wien – 22.\,7.\,1951 ebd.@\textsc{Pollaczek, Clara Katharina} (15.\,1.\,1875 Wien – 22.\,7.\,1951 ebd.), \emph{Schriftstellerin}|pwk} hatten ihr am 25. 7. 1931 einen Besuch dort abgestattet.}}}\label{K_L04007-4}
               musste leider unterbleiben. – Also – bitte nur um eine beiläufige Bestimmung Ihrer
               Aufenthaltsdauer.\pend
           
\pstart
           In Verehrung und inniger Freundschaft {\\[\baselineskip]}Ihre getreueste{\\[\baselineskip]}\spacefill\mbox{Berta Zuckerkandl-Szeps.}\pend
           \leftskip=0em{}\selectlanguage{ngerman}\endnumbering\briefempfaengerindex{Schnitzler, Arthur@\textsc{Schnitzler, Arthur}!zzzZuckerkandl, Berta@\emph{von Berta Zuckerkandl}!1931-07-271@{27. 7. [1931?]}|)be}\mylabel{L04007h}
\begin{anhang}
\end{anhang}\newcommand{\dateiname}{L04007}\newcommand{\titel}{Berta Zuckerkandl an Arthur Schnitzler, 27. 7. [1931?]}\newcommand{\editorInnen}{Herausgegeben von Jahnke, SelmaMüller, Martin Anton}%% latex-leseansicht-abspann.tex
%% Abspann für die Leseansicht.
%% Der Schalter \ifkorrekturansicht ist bereits durch den Vorspann gesetzt.

%% latex-abspann.tex
%% Gemeinsamer Abspann für Korrekturansicht und Leseansicht.
%% Setzt den Schalter \ifkorrekturansicht voraus (gesetzt in den
%% einbindenden Dateien latex-korrekturansicht-abspann.tex bzw.
%% latex-leseansicht-abspann.tex).
%% ---------------------------------------------------------------

\normalsize

% Das esempio-Environment wird nur in der Leseansicht benötigt
\ifkorrekturansicht\else
\newenvironment{esempio}[3]%
{
    \vspace{1.5ex}
    \rlap{\underline{#1}}
    \par
    \setlength{\parindent}{0cm}
    \nopagebreak
    \leftskip=#2cm
    \rightskip=#3cm
}
{
    \par
}
\fi

\doendnotes{C}
\bigskip
\vfill

\clearpage

\footnotesize

\ifkorrekturansicht
  \lohead{\textsc{register}}
\fi

% theindex-Environment neu definieren ohne reledmac
\makeatletter
\renewenvironment{theindex}{%
  \ifkorrekturansicht
    \section*{\indexname}%
  \else
    \subsubsection*{Index der erwähnten Entitäten}%
  \fi
  \setlength{\parindent}{0pt}%
  \setlength{\parskip}{0pt plus 0.3pt}%
  \let\item\@idxitem
}{%
  \ifkorrekturansicht\clearpage\fi
}
\makeatother

\IfFileExists{\jobname-pw.ind}{\input{\jobname-pw.ind}}{}

% Quellenangabe nur in der Leseansicht
\ifkorrekturansicht\else
% Fallback-Definitionen, falls die .tex-Datei \titel etc. nicht gesetzt hat
\providecommand{\titel}{}
\providecommand{\editorInnen}{}
\providecommand{\dateiname}{\jobname}

\vspace{3cm}

\vfill

\footnotesize
\textsc{Quelle}: \titel. Herausgegeben von {\editorInnen}. In: \emph{Arthur Schnitzler: Briefwechsel mit Autorinnen und Autoren}.
 Digitale Edition, https://schnitzler-briefe.acdh.oeaw.ac.at/{\dateiname}.html (Stand \today)
\fi

\end{document}


