%% latex-korrekturansicht-vorspann.tex
%% Vorspann für die Korrekturansicht.
%% Lädt die gemeinsame Datei latex-vorspann.tex mit gesetztem Schalter.

\newif\ifkorrekturansicht
\korrekturansichttrue

\input{../tex-inputs/latex-vorspann}


\section[Friedrich M. Fels an Arthur Schnitzler, {[}Ende 1892?{]}]{L00151 Friedrich M. Fels an Arthur Schnitzler, {[}Ende 1892?{]}}
\nopagebreak\mylabel{L00151v}
\rehead{ }\normalsize\beginnumbering\briefempfaengerindex{Schnitzler, Arthur@\textsc{Schnitzler, Arthur}!zzzFels, Friedrich Michael@\emph{von Friedrich Michael Fels}!1892-12-312@{{[}Ende 1892?{]}}|(be}
\toendnotes[C]{\smallbreak\pagebreak[2]}\Standort{DLA, A:Schnitzler, HS.NZ85.1.2956.}
\physDesc{Briefkarte, 502 Zeichen
\newline{}Handschrift: schwarze Tinte, lateinische Kurrent}\toendnotes[C]{\smallbreak}
\pstart
           \noindent{}{\pb}Lieber Dr Schnitzler! Warum sind Sie heute nicht geko{\geminationm}en? Ich bin \label{K_L00151-1v}\edtext{schwach}{\lemma{\textnormal{\emph{schwach}}}\Cendnote{\textnormal{Am 20. 12. 1892 notiert
                     Schnitzler erstmals nach einem Besuch von
                     Fels\pwindex{Fels, Friedrich Michael *~1864@\textsc{Fels, Friedrich Michael} (*~1864), \emph{Journalist/Journalistin}|pwk} dessen desolaten Zustand:
                     »der beinahe hungert. – Schrecklich ist das. –«. In den
                  folgenden Wochen involvierte sich Schnitzler
                  stärker, mehrere undatierte Korrespondenzstücke dürften in der Zeit, bis der
                  Kranke Mitte Februar 1893 nach Meran\oindex{Meran@\textbf{Meran}, \emph{P.PPLA3}|pwk} abreiste, zu verorten sein. Nur teilweise lassen sich implizite
                  Reihungen vornehmen.}}}\label{K_L00151-1}, weil ich gestern den ganzen Nachmittag vom Durchfall
               geplagt war. Deshalb ka{\geminationn} ich nicht zu Ihnen ko{\geminationm}en. Bitte dem Boten\pwindex{?? [Bote von Friedrich M. Fels] *~1893@\textsc{?? [Bote von Friedrich M. Fels]} (*~1893)|pwv} etwas Geld mitzugeben; ich brauche zum Leben, für
               Schneider, Schuster, Hutmacher; der Bote\pwindex{?? [Bote von Friedrich M. Fels] *~1893@\textsc{?? [Bote von Friedrich M. Fels]} (*~1893)|pwv} ist \uline{ganz sicher}, der
               Sohn meines Hauswirts\pwindex{?? [Vermieter von F. M. Fels] 1893 – 1894@\textsc{?? [Vermieter von F. M. Fels]} (1893 – 1894)|pwv} –
               können ihm also die \uline{gröſte}{ }Su{\geminationm}e mitgeben. Ich
               sitze \label{K_L00151-2v}\edtext{NB}{\lemma{\textnormal{\emph{NB}}}\Cendnote{\textnormal{Fels\pwindex{Fels, Friedrich Michael *~1864@\textsc{Fels, Friedrich Michael} (*~1864), \emph{Journalist/Journalistin}|pwk} nutzt die Abkürzung »NB«, ›notabene‹ in
                  der Bedeutung von ›übrigens‹.}}}\label{K_L00151-2} ohne alles hier; nicht einmal die Cigarette
                  {\pb}die ich rauche ist bezahlt. NB. Bitte um Adreſse
               (genaue) von Beer-Hofma{\geminationn}\pwindex{Beer-Hofmann, Richard 1866-07-11 – 1945-09-26@\textsc{Beer-Hofmann, Richard} (1866-07-11 – 1945-09-26), \emph{Schriftsteller/Schriftstellerin}|pw} u. Loris\pwindex{Hofmannsthal, Hugo von 1874-02-01 – 1929-07-15@\textsc{Hofmannsthal, Hugo von} (1874-02-01 – 1929-07-15), \emph{Schriftsteller/Schriftstellerin}|pw}.\pend
           
\pstart
           H. {\\[\baselineskip]}\spacefill\mbox{Fels}\pend
           \leftskip=0em{}\selectlanguage{ngerman}\endnumbering\briefempfaengerindex{Schnitzler, Arthur@\textsc{Schnitzler, Arthur}!zzzFels, Friedrich Michael@\emph{von Friedrich Michael Fels}!1892-12-312@{{[}Ende 1892?{]}}|)be}\mylabel{L00151h}  \normalsize

\doendnotes{C}
\bigskip
\vfill

\clearpage

\footnotesize

\lohead{\textsc{register}}

% Definiere theindex-Environment komplett neu ohne reledmac
\makeatletter
\renewenvironment{theindex}{%
  \section*{\indexname}%
  \setlength{\parindent}{0pt}%
  \setlength{\parskip}{0pt plus 0.3pt}%
  \let\item\@idxitem
}{%
  \clearpage
}
\makeatother

\IfFileExists{\jobname-pw.ind}{\input{\jobname-pw.ind}}{}

\end{document}

      