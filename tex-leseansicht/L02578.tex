%% latex-leseansicht-vorspann.tex
%% Vorspann für die Leseansicht.
%% Lädt die gemeinsame Datei latex-vorspann.tex mit nicht gesetztem Schalter.

\newif\ifkorrekturansicht
\korrekturansichtfalse

\input{../tex-inputs/latex-vorspann}


\section[Felix Salten, Jakob Wassermann, Otto Brahm, Ludwig Brahm an Arthur Schnitzler, 21. 12. [1907?]]{L02578 Felix Salten, Jakob Wassermann, Otto Brahm, Ludwig Brahm an Arthur
               Schnitzler, 21. 12. [1907?]}
\nopagebreak\mylabel{L02578v}
\rehead{ }\normalsize\beginnumbering\briefempfaengerindex{Schnitzler, Arthur@\textsc{Schnitzler, Arthur}!zzzWassermann, Jakob@\emph{von Jakob Wassermann}!1907-12-211@{21. 12. [1907?]}|(be}\briefempfaengerindex{Schnitzler, Arthur@\textsc{Schnitzler, Arthur}!zzzBrahm, Otto@\emph{von Otto Brahm}!1907-12-211@{21. 12. [1907?]}|(be}\briefempfaengerindex{Schnitzler, Arthur@\textsc{Schnitzler, Arthur}!zzzBrahm, Ludwig@\emph{von Ludwig Brahm}!1907-12-211@{21. 12. [1907?]}|(be}\briefempfaengerindex{Schnitzler, Arthur@\textsc{Schnitzler, Arthur}!zzzSalten, Felix@\emph{von Felix Salten}!1907-12-211@{21. 12. [1907?]}|(be}
\toendnotes[C]{\smallbreak\pagebreak[2]}
\correspDesc{Versand  durch Felix Salten, Otto Brahm, Ludwig Brahm, Jakob Wassermann am 21. 12. [1907?] in Semmering
\newline{}Erhalt  durch Arthur Schnitzler im Zeitraum [22. 12. 1907 – 26. 12. 1907?] in Wien}\toendnotes[C]{\smallbreak}
\Standort{CUL, Schnitzler, B 113.}
\physDesc{Bildpostkarte, 650 Zeichen
\newline{}Handschrift Felix Salten: Bleistift, lateinische Kurrent
\newline{}Handschrift Ludwig Brahm: Bleistift, deutsche Kurrent
\newline{}Handschrift Jakob Wassermann: Bleistift, deutsche Kurrent
\newline{}Handschrift Otto Brahm: Bleistift, lateinische Kurrent
\newline{}Versand: 1) mit rotem Buntstift Adresse gestrichen und ursprüngliche Adresszeile durch
                                    »Bahnhofstraße« ersetzt  2) Stempel: »\nobreak{}\oindex{Semmering@\textbf{Semmering}, \emph{Verwaltungsgebiet}|pwk}Semmer\textcolor{gray}{ing}, 21. XII. \textcolor{gray}{07}, 9\nobreak{}«. 
\newline{}Schnitzler: mit Bleistift eine Unterstreichung  }\pstart{}{\pb}Herrn D\textsuperscript{r}
                  Arthur Schnitzler\pend{}\pstart{}Wien XVIII.\oindex{XVIII., Währing@\textbf{XVIII., Währing}, \emph{Verwaltungsgebiet}|pw}\pend{}\pstart{}Spoettelgasse 7\oindex{Wien@\textbf{Wien}!XVIII., Währing@\textbf{XVIII., Währing}!Edmund-Weiß-Gasse 7@\textbf{Edmund-Weiß-Gasse 7}, \emph{Wohngebäude}|pw}\pend{}{\bigskip}
\pstart
           \noindent{}\centering{}{\pb}\textcolor{gray}{\textbf{Winter-Idylle.}}\pend
           \vspace{1em}
\pstart
           \noindent{}{\pb}{[}hs. Wassermann:{]} Lieber Arthur! Wie{ }ſehr leid tut uns allen Ihr Nichtdaſein! Wir
               denken und sprechen viel von Ihnen.\pend
           \pstart Der Ihre \spacefill\mbox{Wassermann}\pend{}
\pstart
           \introOben{}Für Olga\pwindex{Schnitzler, Olga 17.\,1.\,1882 Wien – 13.\,1.\,1970 Lugano@\textsc{Schnitzler, Olga} (17.\,1.\,1882 Wien – 13.\,1.\,1970 Lugano), \emph{Schauspielerin, Sängerin}|pw} das Herzlichste an
                  Wünschen\introOben{}\pend
           \selectlanguage{ngerman}\vspace{1em}
\pstart
           \noindent{}{[}hs. Salten:{]} Hoffentlich geht es Frau Olga\pwindex{Schnitzler, Olga 17.\,1.\,1882 Wien – 13.\,1.\,1970 Lugano@\textsc{Schnitzler, Olga} (17.\,1.\,1882 Wien – 13.\,1.\,1970 Lugano), \emph{Schauspielerin, Sängerin}|pw} täglich besser und besser. Viele herzliche Grüße an Sie
               Beide!\pend
           \pstart Ihr \spacefill\mbox{Salten.}\pend{}
\pstart
           \noindent{}Die Bücher sende ich Montag.\pend
           \selectlanguage{ngerman}\vspace{1em}
\pstart
           \noindent{}{[}hs. Brahm:{]} Lieber Freund, da wir Fr. O.\pwindex{Schnitzler, Olga 17.\,1.\,1882 Wien – 13.\,1.\,1970 Lugano@\textsc{Schnitzler, Olga} (17.\,1.\,1882 Wien – 13.\,1.\,1970 Lugano), \emph{Schauspielerin, Sängerin}|pw}
               und Sie leider, leider nicht hier haben, huldigten wir Ihnen und verspürten Ihres
               Geistes ein Hauch auf dem Wasserleitungswege. Alles Gute wünschet von Herzen\pend
           \pstart Ihr \spacefill\mbox{Otto Brahm}\pend{}\selectlanguage{ngerman}\vspace{1em}
\pstart
           \noindent{}{[}hs. Brahm:{]} Den herzlichsten Wünſchen für die schnelle Geneſung
               Ihrer Gattin\pwindex{Schnitzler, Olga 17.\,1.\,1882 Wien – 13.\,1.\,1970 Lugano@\textsc{Schnitzler, Olga} (17.\,1.\,1882 Wien – 13.\,1.\,1970 Lugano), \emph{Schauspielerin, Sängerin}|pw} schließt{ }ſich mit den besten
               Grüßen für Sie an\pend
           
\pstart
           Ihr{\\[\baselineskip]}\spacefill\mbox{Ludwig Brahm.}\pend
           \leftskip=0em{}\selectlanguage{ngerman}\endnumbering\briefempfaengerindex{Schnitzler, Arthur@\textsc{Schnitzler, Arthur}!zzzWassermann, Jakob@\emph{von Jakob Wassermann}!1907-12-211@{21. 12. [1907?]}|)be}\briefempfaengerindex{Schnitzler, Arthur@\textsc{Schnitzler, Arthur}!zzzBrahm, Otto@\emph{von Otto Brahm}!1907-12-211@{21. 12. [1907?]}|)be}\briefempfaengerindex{Schnitzler, Arthur@\textsc{Schnitzler, Arthur}!zzzBrahm, Ludwig@\emph{von Ludwig Brahm}!1907-12-211@{21. 12. [1907?]}|)be}\briefempfaengerindex{Schnitzler, Arthur@\textsc{Schnitzler, Arthur}!zzzSalten, Felix@\emph{von Felix Salten}!1907-12-211@{21. 12. [1907?]}|)be}\mylabel{L02578h}  \newcommand{\dateiname}{L02578}\newcommand{\titel}{Felix Salten, Jakob Wassermann, Otto Brahm, Ludwig Brahm an Arthur Schnitzler, 21. 12. [1907?]}\newcommand{\editorInnen}{Herausgegeben von Martin Anton Müller}%% latex-leseansicht-abspann.tex
%% Abspann für die Leseansicht.
%% Der Schalter \ifkorrekturansicht ist bereits durch den Vorspann gesetzt.

%% latex-abspann.tex
%% Gemeinsamer Abspann für Korrekturansicht und Leseansicht.
%% Setzt den Schalter \ifkorrekturansicht voraus (gesetzt in den
%% einbindenden Dateien latex-korrekturansicht-abspann.tex bzw.
%% latex-leseansicht-abspann.tex).
%% ---------------------------------------------------------------

\normalsize

% Das esempio-Environment wird nur in der Leseansicht benötigt
\ifkorrekturansicht\else
\newenvironment{esempio}[3]%
{
    \vspace{1.5ex}
    \rlap{\underline{#1}}
    \par
    \setlength{\parindent}{0cm}
    \nopagebreak
    \leftskip=#2cm
    \rightskip=#3cm
}
{
    \par
}
\fi

\doendnotes{C}
\bigskip
\vfill

\clearpage

\footnotesize

\ifkorrekturansicht
  \lohead{\textsc{register}}
\fi

% theindex-Environment neu definieren ohne reledmac
\makeatletter
\renewenvironment{theindex}{%
  \ifkorrekturansicht
    \section*{\indexname}%
  \else
    \subsubsection*{Index der erwähnten Entitäten}%
  \fi
  \setlength{\parindent}{0pt}%
  \setlength{\parskip}{0pt plus 0.3pt}%
  \let\item\@idxitem
}{%
  \ifkorrekturansicht\clearpage\fi
}
\makeatother

\IfFileExists{\jobname-pw.ind}{\input{\jobname-pw.ind}}{}

% Quellenangabe nur in der Leseansicht
\ifkorrekturansicht\else
% Fallback-Definitionen, falls die .tex-Datei \titel etc. nicht gesetzt hat
\providecommand{\titel}{}
\providecommand{\editorInnen}{}
\providecommand{\dateiname}{\jobname}

\vspace{3cm}

\vfill

\footnotesize
\textsc{Quelle}: \titel. Herausgegeben von {\editorInnen}. In: \emph{Arthur Schnitzler: Briefwechsel mit Autorinnen und Autoren}.
 Digitale Edition, https://schnitzler-briefe.acdh.oeaw.ac.at/{\dateiname}.html (Stand \today)
\fi

\end{document}


