%% latex-korrekturansicht-vorspann.tex
%% Vorspann für die Korrekturansicht.
%% Lädt die gemeinsame Datei latex-vorspann.tex mit gesetztem Schalter.

\newif\ifkorrekturansicht
\korrekturansichttrue

\input{../tex-inputs/latex-vorspann}


\section[Felix Salten, Jakob Wassermann, Otto Brahm, Ludwig Brahm an Arthur Schnitzler, 21. 12. {[}1907?{]}]{L02578 Felix Salten, Jakob Wassermann, Otto Brahm, Ludwig Brahm an Arthur
               Schnitzler, 21. 12. {[}1907?{]}}
\nopagebreak\mylabel{L02578v}
\rehead{ }\normalsize\beginnumbering\briefempfaengerindex{Schnitzler, Arthur@\textsc{Schnitzler, Arthur}!zzzWassermann, Jakob@\emph{von Jakob Wassermann}!1907-12-211@{21. 12. {[}1907?{]}}|(be}\briefempfaengerindex{Schnitzler, Arthur@\textsc{Schnitzler, Arthur}!zzzBrahm, Otto@\emph{von Otto Brahm}!1907-12-211@{21. 12. {[}1907?{]}}|(be}\briefempfaengerindex{Schnitzler, Arthur@\textsc{Schnitzler, Arthur}!zzzBrahm, Ludwig@\emph{von Ludwig Brahm}!1907-12-211@{21. 12. {[}1907?{]}}|(be}\briefempfaengerindex{Schnitzler, Arthur@\textsc{Schnitzler, Arthur}!zzzSalten, Felix@\emph{von Felix Salten}!1907-12-211@{21. 12. {[}1907?{]}}|(be}
\toendnotes[C]{\smallbreak\pagebreak[2]}\Standort{CUL, Schnitzler, B 113.}
\physDesc{Bildpostkarte, 650 Zeichen
\newline{}Handschrift Felix Salten: Bleistift, lateinische Kurrent
\newline{}Handschrift Ludwig Brahm: Bleistift, deutsche Kurrent
\newline{}Handschrift Jakob Wassermann: Bleistift, deutsche Kurrent
\newline{}Handschrift Otto Brahm: Bleistift, lateinische Kurrent
\newline{}Versand: 1) mit rotem Buntstift Adresse gestrichen und ursprüngliche Adresszeile durch
                                    »Bahnhofstraße« ersetzt  2) Stempel: »\nobreak{}\oindex{Semmering@\textbf{Semmering}, \emph{A.ADM3}|pwk}Semmer\textcolor{gray}{ing}, 21. XII. \textcolor{gray}{07}, 9\nobreak{}«. 
\newline{}Schnitzler: mit Bleistift eine Unterstreichung  }\pstart{}{\pb}Herrn D\textsuperscript{r}
                  Arthur Schnitzler\pend{}\pstart{}Wien XVIII.\oindex{XVIII., Waehring@\textbf{XVIII., Währing}, \emph{A.ADM3}|pw}\pend{}\pstart{}Spoettelgasse 7\oindex{Edmund-Weiss-Gasse 7@\textbf{Edmund-Weiß-Gasse 7}, \emph{Wohngebäude (K.WHS)}|pw}\pend{}{\bigskip}
\pstart
           \noindent{}\centering{}{\pb}\textcolor{gray}{\textbf{Winter-Idylle.}}\pend
           \vspace{1em}
\pstart
           \noindent{}{\pb}{[}hs. :{]} Lieber Arthur! Wie ſehr leid tut uns allen Ihr Nichtdaſein! Wir
               denken und sprechen viel von Ihnen.\pend
           \pstart Der Ihre \spacefill\mbox{Wassermann}\pend{}
\pstart
           \introOben{}Für Olga\pwindex{Schnitzler, Olga 17.01.1882 – 13.01.1970@\textsc{Schnitzler, Olga} (17.01.1882 – 13.01.1970), \emph{Schauspieler/Schauspielerin, Sänger/Sängerin}|pw} das Herzlichste an
                  Wünschen\introOben{}\pend
           \selectlanguage{ngerman}\vspace{1em}
\pstart
           \noindent{}{[}hs. :{]} Hoffentlich geht es Frau Olga\pwindex{Schnitzler, Olga 17.01.1882 – 13.01.1970@\textsc{Schnitzler, Olga} (17.01.1882 – 13.01.1970), \emph{Schauspieler/Schauspielerin, Sänger/Sängerin}|pw} täglich besser und besser. Viele herzliche Grüße an Sie
               Beide!\pend
           \pstart Ihr \spacefill\mbox{Salten.}\pend{}
\pstart
           \noindent{}Die Bücher sende ich Montag.\pend
           \selectlanguage{ngerman}\vspace{1em}
\pstart
           \noindent{}{[}hs. :{]} Lieber Freund, da wir Fr. O.\pwindex{Schnitzler, Olga 17.01.1882 – 13.01.1970@\textsc{Schnitzler, Olga} (17.01.1882 – 13.01.1970), \emph{Schauspieler/Schauspielerin, Sänger/Sängerin}|pw}
               und Sie leider, leider nicht hier haben, huldigten wir Ihnen und verspürten Ihres
               Geistes ein Hauch auf dem Wasserleitungswege. Alles Gute wünschet von Herzen\pend
           \pstart Ihr \spacefill\mbox{Otto Brahm}\pend{}\selectlanguage{ngerman}\vspace{1em}
\pstart
           \noindent{}{[}hs. :{]} Den herzlichsten Wünſchen für die schnelle Geneſung
               Ihrer Gattin\pwindex{Schnitzler, Olga 17.01.1882 – 13.01.1970@\textsc{Schnitzler, Olga} (17.01.1882 – 13.01.1970), \emph{Schauspieler/Schauspielerin, Sänger/Sängerin}|pw} schließt ſich mit den besten
               Grüßen für Sie an\pend
           
\pstart
           Ihr{\\[\baselineskip]}\spacefill\mbox{Ludwig Brahm.}\pend
           \leftskip=0em{}\selectlanguage{ngerman}\endnumbering\briefempfaengerindex{Schnitzler, Arthur@\textsc{Schnitzler, Arthur}!zzzWassermann, Jakob@\emph{von Jakob Wassermann}!1907-12-211@{21. 12. {[}1907?{]}}|)be}\briefempfaengerindex{Schnitzler, Arthur@\textsc{Schnitzler, Arthur}!zzzBrahm, Otto@\emph{von Otto Brahm}!1907-12-211@{21. 12. {[}1907?{]}}|)be}\briefempfaengerindex{Schnitzler, Arthur@\textsc{Schnitzler, Arthur}!zzzBrahm, Ludwig@\emph{von Ludwig Brahm}!1907-12-211@{21. 12. {[}1907?{]}}|)be}\briefempfaengerindex{Schnitzler, Arthur@\textsc{Schnitzler, Arthur}!zzzSalten, Felix@\emph{von Felix Salten}!1907-12-211@{21. 12. {[}1907?{]}}|)be}\mylabel{L02578h}  \normalsize

\doendnotes{C}
\bigskip
\vfill

\clearpage

\footnotesize

\lohead{\textsc{register}}

% Definiere theindex-Environment komplett neu ohne reledmac
\makeatletter
\renewenvironment{theindex}{%
  \section*{\indexname}%
  \setlength{\parindent}{0pt}%
  \setlength{\parskip}{0pt plus 0.3pt}%
  \let\item\@idxitem
}{%
  \clearpage
}
\makeatother

\IfFileExists{\jobname-pw.ind}{\input{\jobname-pw.ind}}{}

\end{document}

      