%% latex-korrekturansicht-vorspann.tex
%% Vorspann für die Korrekturansicht.
%% Lädt die gemeinsame Datei latex-vorspann.tex mit gesetztem Schalter.

\newif\ifkorrekturansicht
\korrekturansichttrue

\input{../tex-inputs/latex-vorspann}


\section[Stefan Zweig an Arthur Schnitzler, {[}26. 4. 1915{]}]{L03654 Stefan Zweig an Arthur Schnitzler, {[}26. 4. 1915{]}}
\nopagebreak\mylabel{L03654v}
\rehead{ }\normalsize\beginnumbering\briefempfaengerindex{Schnitzler, Arthur@\textsc{Schnitzler, Arthur}!zzzZweig, Stefan@\emph{von Stefan Zweig}!1915-04-261@{{[}26. 4. 1915{]}}|(be}
\toendnotes[C]{\smallbreak\pagebreak[2]}\Standort{CUL, Schnitzler, B 118.}
\physDesc{Brief, 1 Blatt, 3 Seiten, 1693 Zeichen
\newline{}Handschrift: lila Tinte, lateinische Kurrent
\newline{}Schnitzler: 1) mit Bleistift »\textsc{Zweig}«  2) mit rotem Buntstift drei Unterstreichungen}
\buchAbdrucke{\weitereDrucke{Stefan Zweig: \emph{Briefwechsel mit Hermann Bahr, Sigmund Freud, Rainer Maria
                        Rilke und Arthur Schnitzler}. Frankfurt am Main: \emph{S. Fischer} 1987, S. 393–394.} }\toendnotes[C]{\smallbreak}
\pstart
           {\pb}\textcolor{gray}{\textbf{SZ}}\hfill \textcolor{gray}{\textbf{VIII. KOCHGASSE\oindex{Kochgasse 8@\textbf{Kochgasse 8}, \emph{Wohngebäude (K.WHS)}|pw}}}\pend
           
\pstart
           \raggedleft{}\textcolor{gray}{\textbf{WIEN\oindex{Wien@\textbf{Wien}, \emph{A.ADM2}|pw},}}\pend
           
\pstart{}Sehr verehrter lieber Herr Doktor,\pend\vspace{0.5em}
\pstart
           ich war innerlich noch sehr bedrückt, Ihnen für den \label{K_L03654-1v}\edtext{schönen Abend}{\lemma{\textnormal{\emph{schönen Abend}}}\Cendnote{\textnormal{Schnitzler hatte Zweig\pwindex{Zweig, Stefan 28.11.1881 – 23.02.1942@\textsc{Zweig, Stefan} (28.11.1881 – 23.02.1942), \emph{Schriftsteller/Schriftstellerin}|pwk} und Berta
                     Zuckerkandl\pwindex{Zuckerkandl, Berta 13.04.1864 – 16.10.1945@\textsc{Zuckerkandl, Berta} (13.04.1864 – 16.10.1945), \emph{Journalist/Journalistin, Übersetzer/Übersetzerin}|pwk} am 11. 4. 1915 die \emph{Komödie der
                     Worte}\pwindex{Komoedie der Worte. Drei Einakter@\emph{Komödie der Worte. Drei Einakter}|pwk} vorgelesen.}}}\label{K_L03654-1} von damals nicht noch besonders gedankt zu haben:
               der Grund für dieses Unterlassen war, dass ich mich innerlich um den Titel für das
                  Werk\pwindex{Komoedie der Worte. Drei Einakter@\emph{Komödie der Worte. Drei Einakter}|pwv} mühte und ohne diese
               bescheidene Gegengabe Ihnen nicht schreiben wollte. Und nun muss ich Ihnen für
               neuerliche Güte danken: glauben Sie mir, bitte, dass ich gerade in dieser Zeit, wo
               sonst alle Menschen das Harte in sich herauskehren, Ihnen dafür besonders erkenntlich
                  {\pb}bin.\pend
           
\pstart
           In der \label{K_L03654-2v}\edtext{Sache D\textsuperscript{r}{ }Rosenbaums\pwindex{Rosenbaum, Richard 04.11.1867 – 25.06.1942@\textsc{Rosenbaum, Richard} (04.11.1867 – 25.06.1942), \emph{Dramaturg/Dramaturgin, Verleger/Verlegerin}|pw}}{\lemma{\textnormal{\emph{Sache … Rosenbaums}}}\Cendnote{\textnormal{siehe Stefan Zweig an Arthur Schnitzler, [zwischen
               5. 4. 1915–9. 4. 1915?]. }}}\label{K_L03654-2} habe ich
                  \label{K_L03654-3v}\edtext{von Gerhardt Hauptmann\pwindex{Hauptmann, Gerhart 15.11.1862 – 06.06.1946@\textsc{Hauptmann, Gerhart} (15.11.1862 – 06.06.1946), \emph{Schriftsteller/Schriftstellerin}|pw} noch keine Antwort}{\lemma{\textnormal{\emph{von … Antwort}}}\Cendnote{\textnormal{Sowohl Schnitzler wie
                  auch Gerhart Hauptmann\pwindex{Hauptmann, Gerhart 15.11.1862 – 06.06.1946@\textsc{Hauptmann, Gerhart} (15.11.1862 – 06.06.1946), \emph{Schriftsteller/Schriftstellerin}|pwk} traten mit einer
                  Erklärung für Rosenbaum\pwindex{Rosenbaum, Richard 04.11.1867 – 25.06.1942@\textsc{Rosenbaum, Richard} (04.11.1867 – 25.06.1942), \emph{Dramaturg/Dramaturgin, Verleger/Verlegerin}|pwk} öffentlich für
                  diesen ein, siehe A. S.: \emph{»Das Zeitlose ist von kürzester Dauer«}, Der Rücktritt des Burgtheatersekretärs Dr. Rosenbaum, 16. 5. 1915. »Zweig\pwindex{Zweig, Stefan 28.11.1881 – 23.02.1942@\textsc{Zweig, Stefan} (28.11.1881 – 23.02.1942), \emph{Schriftsteller/Schriftstellerin}|pw} hatte Gerhart Hauptmann\pwindex{Hauptmann, Gerhart 15.11.1862 – 06.06.1946@\textsc{Hauptmann, Gerhart} (15.11.1862 – 06.06.1946), \emph{Schriftsteller/Schriftstellerin}|pw} in einem (unveröffentlichten) Brief
                     vom 13. 4. 1915 um ›ein Wort zum Abschied‹ Rosenbaums\pwindex{Rosenbaum, Richard 04.11.1867 – 25.06.1942@\textsc{Rosenbaum, Richard} (04.11.1867 – 25.06.1942), \emph{Dramaturg/Dramaturgin, Verleger/Verlegerin}|pw} vom Burgtheaters\orgindex{Burgtheater@Burgtheater|pw} gebeten. Hauptmann\pwindex{Hauptmann, Gerhart 15.11.1862 – 06.06.1946@\textsc{Hauptmann, Gerhart} (15.11.1862 – 06.06.1946), \emph{Schriftsteller/Schriftstellerin}|pw}
                     hatte darauf am 20. 4. kurz geantwortet: ›Der Weggang Dr. Rosenbaums\pwindex{Rosenbaum, Richard 04.11.1867 – 25.06.1942@\textsc{Rosenbaum, Richard} (04.11.1867 – 25.06.1942), \emph{Dramaturg/Dramaturgin, Verleger/Verlegerin}|pw} vom Burgtheaters\orgindex{Burgtheater@Burgtheater|pw} hat mich sehr schmerzlich berührt, weil
                     ich weiß, mit welcher Hingebung er dem Institute verbunden ist. Ich begrüsse
                     Sie herzlich, danke Ihnen wärmstens für Ihre lieben Zeilen und füge ein paar
                     Abschiedsworte {[}\ldots{]} für Dr.
                        Rosenbaum\pwindex{Rosenbaum, Richard 04.11.1867 – 25.06.1942@\textsc{Rosenbaum, Richard} (04.11.1867 – 25.06.1942), \emph{Dramaturg/Dramaturgin, Verleger/Verlegerin}|pw} hier bei.‹ Am
                        4. 5. bedankte Zweig\pwindex{Zweig, Stefan 28.11.1881 – 23.02.1942@\textsc{Zweig, Stefan} (28.11.1881 – 23.02.1942), \emph{Schriftsteller/Schriftstellerin}|pw}
                     sich für Hauptmanns\pwindex{Hauptmann, Gerhart 15.11.1862 – 06.06.1946@\textsc{Hauptmann, Gerhart} (15.11.1862 – 06.06.1946), \emph{Schriftsteller/Schriftstellerin}|pw} Brief und schrieb
                     (in einem ebenfalls unveröffentlichten Brief): ›Erst heute bekam ich Ihren
                     Brief vom 20. April, aber diesmals darf die Post nicht gescholten
                     sein: die ungeheuren Truppentransporte haben die Strecken für sich genommen und
                     für die verzögerte Freude einzelner Briefe haben wir heute die gemeinsame des
                     großen Sieges.‹« (\emph{Briefwechsel mit Hermann Bahr, Sigmund Freud, Rainer Maria
                        Rilke und Arthur Schnitzler}, S. 472.)}}}\label{K_L03654-3}: ist es die
               Post, die den Brief so lange hält oder irgend Etwas in ihm? Jedesfalls bin ich sehr
               erbittert, wie gut Thimig\pwindex{Thimig, Hugo 16.06.1854 – 24.09.1944@\textsc{Thimig, Hugo} (16.06.1854 – 24.09.1944), \emph{Theaterleiter/Theaterleiterin, Schauspieler/Schauspielerin}|pw} alles gelungen ist.
               In aller Stille hat man diesen guten Mann\pwindex{Rosenbaum, Richard 04.11.1867 – 25.06.1942@\textsc{Rosenbaum, Richard} (04.11.1867 – 25.06.1942), \emph{Dramaturg/Dramaturgin, Verleger/Verlegerin}|pwv} begraben und in einem Jahr wird niemand mehr von ihm
               wissen. Ich hoffe noch immer, etwas tun zu können: es wäre ja sehr nötig und nicht
               nur im moralischen Sinne, denn D\textsuperscript{r}{ }R\pwindex{Rosenbaum, Richard 04.11.1867 – 25.06.1942@\textsc{Rosenbaum, Richard} (04.11.1867 – 25.06.1942), \emph{Dramaturg/Dramaturgin, Verleger/Verlegerin}|pw}, der jetzt ein Vierteljahrhundert in
               unablässiger Arbeit gelebt hat, braucht Wirksamkeit, um nicht bitter zu werden.
               Hoffentlich findet sich da ein Weg.\pend
           
\pstart
           Ich freue mich sehr, Sie {\pb}und Ihre
               verehrte Frau Gemahlin\pwindex{Schnitzler, Olga 17.01.1882 – 13.01.1970@\textsc{Schnitzler, Olga} (17.01.1882 – 13.01.1970), \emph{Schauspieler/Schauspielerin, Sänger/Sängerin}|pwv} bald
               wieder sehen zu dürfen: heute abends habe ich mir den \label{K_L03654-4v}\edtext{Sonatenabend Walter\pwindex{Walter, Bruno 15.09.1876 – 17.02.1962@\textsc{Walter, Bruno} (15.09.1876 – 17.02.1962), \emph{Theaterleiter/Theaterleiterin, Komponist/Komponistin, Dirigent/Dirigentin}|pw}{ }Rosé\pwindex{Rose, Arnold 24.10.1863 – 25.08.1946@\textsc{Rosé, Arnold} (24.10.1863 – 25.08.1946), \emph{Violinist/Violinistin}|pw}}{\lemma{\textnormal{\emph{Sonatenabend Walter Rosé}}}\Cendnote{\textnormal{Im Mittleren Konzerthaussaal\oindex{Wiener Konzerthaus@\textbf{Wiener Konzerthaus}, \emph{Konzertsaal (K.KNZ)}|pwk}.}}}\label{K_L03654-4}, \label{K_L03654-5v}\edtext{morgen das Lied von der Erde\pwindex{Lied von der Erde@\emph{Das Lied von der Erde}|pw}}{\lemma{\textnormal{\emph{morgen … Erde}}}\Cendnote{\textnormal{Von den hier aufgeführten
                  Veranstaltungen besuchten Olga\pwindex{Schnitzler, Olga 17.01.1882 – 13.01.1970@\textsc{Schnitzler, Olga} (17.01.1882 – 13.01.1970), \emph{Schauspieler/Schauspielerin, Sänger/Sängerin}|pwk} und Arthur Schnitzler nur die Aufführung von \emph{Das Lied von der Erde}\pwindex{Lied von der Erde@\emph{Das Lied von der Erde}|pwk} am 27. 4. 1915 im Großen Musikvereinssaal\oindex{Musikverein@\textbf{Musikverein}, \emph{Konzertsaal (K.KNZ)}|pwk}.}}}\label{K_L03654-5}, \label{K_L03654-6v}\edtext{Mittwoch{ }Elektra\pwindex{Elektra [op. 58]@\emph{Elektra [op. 58]}|pw}}{\lemma{\textnormal{\emph{Mittwoch Elektra}}}\Cendnote{\textnormal{Am 28. 4. 1915 wurde \emph{Elektra}\pwindex{Elektra [op. 58]@\emph{Elektra [op. 58]}|pwk} in der \emph{Wiener Oper}\orgindex{K.K. Hof-Oper@K.K. Hof-Oper|pwk} gespielt.}}}\label{K_L03654-6} zugedacht, ich lebe jetzt wirklich von
               Musik, denn sonst wäre es nicht zu ertragen.\pend
           
\pstart
           In dankbarer Verehrung getreu Ihr{\\[\baselineskip]}\spacefill\mbox{Stefan Zweig}\pend
           \leftskip=0em{}
\pstart
           \noindent{}Viele Grüsse Ihrer Frau Gemahlin\pwindex{Schnitzler, Olga 17.01.1882 – 13.01.1970@\textsc{Schnitzler, Olga} (17.01.1882 – 13.01.1970), \emph{Schauspieler/Schauspielerin, Sänger/Sängerin}|pwv}!\pend
           
\pstart
           \noindent{}Und noch die Erinnerung: wenn Sie einmal Zeit und Lust haben gedenken Sie jenes
                  Bildhauers Gustinus Ambrosi\pwindex{Ambrosi, Gustinus 24.02.1893 – 30.06.1975@\textsc{Ambrosi, Gustinus} (24.02.1893 – 30.06.1975), \emph{Schriftsteller/Schriftstellerin, Bildhauer/Bildhauerin, Philosoph/Philosophin}|pw}, der so gerne
                  Ihre Büste machte. Ich halte diesen taubstummen Menschen für einen wahrhaft
                  genialen Künstler, er ist auch menschlich, ein unvergleichliches Erlebnis.\pend
           \selectlanguage{ngerman}\endnumbering\briefempfaengerindex{Schnitzler, Arthur@\textsc{Schnitzler, Arthur}!zzzZweig, Stefan@\emph{von Stefan Zweig}!1915-04-261@{{[}26. 4. 1915{]}}|)be}\mylabel{L03654h}
\begin{anhang}
\end{anhang}\normalsize

\doendnotes{C}
\bigskip
\vfill

\clearpage

\footnotesize

\lohead{\textsc{register}}

% Definiere theindex-Environment komplett neu ohne reledmac
\makeatletter
\renewenvironment{theindex}{%
  \section*{\indexname}%
  \setlength{\parindent}{0pt}%
  \setlength{\parskip}{0pt plus 0.3pt}%
  \let\item\@idxitem
}{%
  \clearpage
}
\makeatother

\IfFileExists{\jobname-pw.ind}{\input{\jobname-pw.ind}}{}

\end{document}

      