%% latex-leseansicht-vorspann.tex
%% Vorspann für die Leseansicht.
%% Lädt die gemeinsame Datei latex-vorspann.tex mit nicht gesetztem Schalter.

\newif\ifkorrekturansicht
\korrekturansichtfalse

\input{../tex-inputs/latex-vorspann}


\section[Stefan Zweig an Arthur Schnitzler, {[}26. 4. 1915{]}]{L03654 Stefan Zweig an Arthur Schnitzler, {[}26. 4. 1915{]}}
\nopagebreak\mylabel{L03654v}
\rehead{ }\normalsize\beginnumbering\briefempfaengerindex{Schnitzler, Arthur@\textsc{Schnitzler, Arthur}!zzzZweig, Stefan@\emph{von Stefan Zweig}!1915-04-261@{{[}26. 4. 1915{]}}|(be}
\toendnotes[C]{\smallbreak\pagebreak[2]}
\correspDesc{Versand  durch Stefan Zweig am [26. 4. 1915] in Wien
\newline{}Erhalt  durch Arthur Schnitzler im Zeitraum [26. 4. 1915
                  – 1. 5. 1915?] in Wien}\toendnotes[C]{\smallbreak}
\Standort{CUL, Schnitzler, B 118.}
\physDesc{Brief, 1 Blatt, 3 Seiten, 1692 Zeichen
\newline{}Handschrift: lila Tinte, lateinische Kurrent
\newline{}Schnitzler: 1) mit Bleistift »\textsc{Zweig}« und datiert »Mai 915«  2) mit rotem Buntstift drei Unterstreichungen}
\buchAbdrucke{\weitereDrucke{Stefan Zweig: \emph{Briefwechsel mit Hermann Bahr, Sigmund Freud, Rainer Maria
                        Rilke und Arthur Schnitzler}. Herausgegeben von Jeffrey B. Berlin, Hans-Ulrich Lindken und Donald A. Prater. Frankfurt am Main: \emph{S. Fischer} 1987, S. 393–394.} }\toendnotes[C]{\smallbreak}
\pstart
           {\pb}\textcolor{gray}{\textbf{SZ}}\hfill \textcolor{gray}{\textbf{VIII. KOCHGASSE\oindex{Wien@\textbf{Wien}!VIII., Josefstadt@\textbf{VIII., Josefstadt}!Kochgasse 8@\textbf{Kochgasse 8}, \emph{Wohngebäude}|pw}}}\pend
           
\pstart
           \raggedleft{}\textcolor{gray}{\textbf{WIEN\oindex{Wien@\textbf{Wien}, \emph{Verwaltungsgebiet}|pw},}}\pend
           
\pstart{}Sehr verehrter lieber Herr Doktor,\pend\vspace{0.5em}
\pstart
           ich war innerlich noch sehr bedrückt, Ihnen für den \label{K_L03654-1v}\edtext{schönen Abend}{\lemma{\textnormal{\emph{schönen Abend}}}\Cendnote{\textnormal{Schnitzler hatte Zweig\pwindex{Zweig, Stefan 28.\,11.\,1881 Wien – 23.\,2.\,1942 Petrópolis@\textsc{Zweig, Stefan} (28.\,11.\,1881 Wien – 23.\,2.\,1942 Petrópolis), \emph{Schriftsteller}|pwk} und Berta
                     Zuckerkandl\pwindex{Zuckerkandl, Berta 13.\,4.\,1864 Wien – 16.\,10.\,1945 Paris@\textsc{Zuckerkandl, Berta} (13.\,4.\,1864 Wien – 16.\,10.\,1945 Paris), \emph{Schriftstellerin, Journalistin, Übersetzerin}|pwk} am 11. 4. 1915 die \emph{Komödie der
                     Worte}\pwindex{Schnitzler, Arthur 15.\,5.\,1862 Wien – 21.\,10.\,1931 ebd.@\textsc{Schnitzler, Arthur} (15.\,5.\,1862 Wien – 21.\,10.\,1931 ebd.), \emph{Schriftsteller, Mediziner}!Komödie der Worte. Drei Einakter@\strich\emph{Komödie der Worte. Drei Einakter}|pwk} vorgelesen und bei der Gelegenheit wurde auch die Frage nach einem
                  passenden Titel für das Werk diskutiert.}}}\label{K_L03654-1} von damals nicht noch besonders
               gedankt zu haben: der Grund für dieses Unterlassen war, dass ich mich innerlich um
               den Titel für das Werk\pwindex{Schnitzler, Arthur 15.\,5.\,1862 Wien – 21.\,10.\,1931 ebd.@\textsc{Schnitzler, Arthur} (15.\,5.\,1862 Wien – 21.\,10.\,1931 ebd.), \emph{Schriftsteller, Mediziner}!Komödie der Worte. Drei Einakter@\strich\emph{Komödie der Worte. Drei Einakter}|pwv} mühte
               und ohne diese bescheidene Gegengabe Ihnen nicht schreiben wollte. Und nun muss ich
               Ihnen für neuerliche Güte danken: glauben Sie mir, bitte, dass ich gerade in dieser
               Zeit, wo sonst alle Menschen das Harte in sich herauskehren, Ihnen dafür besonders
               erkenntlich {\pb}bin.\pend
           
\pstart
           In der \label{K_L03654-2v}\edtext{Sache D\textsuperscript{r}{ }Rosenbaums\pwindex{Rosenbaum, Richard 4.\,11.\,1867 Žikov – 25.\,6.\,1942 Konzentrationslager Theresienstadt@\textsc{Rosenbaum, Richard} (4.\,11.\,1867 Žikov – 25.\,6.\,1942 Konzentrationslager Theresienstadt), \emph{Dramaturg, Verleger}|pw}}{\lemma{\textnormal{\emph{Sache … Rosenbaums}}}\Cendnote{\textnormal{Siehe XXXX Auszeichnungsfehler: Dokument L03653 nicht gefunden. }}}\label{K_L03654-2} habe ich
                  \label{K_L03654-3v}\edtext{von Gerhardt Hauptmann\pwindex{Hauptmann, Gerhart 15.\,11.\,1862 Szczawno-Zdrój – 6.\,6.\,1946 Jagniątków@\textsc{Hauptmann, Gerhart} (15.\,11.\,1862 Szczawno-Zdrój – 6.\,6.\,1946 Jagniątków), \emph{Schriftsteller}|pw} noch keine Antwort}{\lemma{\textnormal{\emph{von … Antwort}}}\Cendnote{\textnormal{Sowohl Schnitzler wie
                  auch Gerhart Hauptmann\pwindex{Hauptmann, Gerhart 15.\,11.\,1862 Szczawno-Zdrój – 6.\,6.\,1946 Jagniątków@\textsc{Hauptmann, Gerhart} (15.\,11.\,1862 Szczawno-Zdrój – 6.\,6.\,1946 Jagniątków), \emph{Schriftsteller}|pwk} traten mit einer
                  Erklärung für Rosenbaum\pwindex{Rosenbaum, Richard 4.\,11.\,1867 Žikov – 25.\,6.\,1942 Konzentrationslager Theresienstadt@\textsc{Rosenbaum, Richard} (4.\,11.\,1867 Žikov – 25.\,6.\,1942 Konzentrationslager Theresienstadt), \emph{Dramaturg, Verleger}|pwk} öffentlich für
                  diesen ein, siehe A. S.: \emph{»Das Zeitlose ist von kürzester Dauer«}, Der Rücktritt des Burgtheatersekretärs Dr. Rosenbaum, 16. 5. 1915. »Zweig\pwindex{Zweig, Stefan 28.\,11.\,1881 Wien – 23.\,2.\,1942 Petrópolis@\textsc{Zweig, Stefan} (28.\,11.\,1881 Wien – 23.\,2.\,1942 Petrópolis), \emph{Schriftsteller}|pw} hatte Gerhart Hauptmann\pwindex{Hauptmann, Gerhart 15.\,11.\,1862 Szczawno-Zdrój – 6.\,6.\,1946 Jagniątków@\textsc{Hauptmann, Gerhart} (15.\,11.\,1862 Szczawno-Zdrój – 6.\,6.\,1946 Jagniątków), \emph{Schriftsteller}|pw} in einem (unveröffentlichten) Brief
                     vom 13. 4. 1915 um ›ein Wort zum Abschied‹ Rosenbaums\pwindex{Rosenbaum, Richard 4.\,11.\,1867 Žikov – 25.\,6.\,1942 Konzentrationslager Theresienstadt@\textsc{Rosenbaum, Richard} (4.\,11.\,1867 Žikov – 25.\,6.\,1942 Konzentrationslager Theresienstadt), \emph{Dramaturg, Verleger}|pw} vom Burgtheaters\orgindex{Burgtheater@Burgtheater|pw} gebeten. Hauptmann\pwindex{Hauptmann, Gerhart 15.\,11.\,1862 Szczawno-Zdrój – 6.\,6.\,1946 Jagniątków@\textsc{Hauptmann, Gerhart} (15.\,11.\,1862 Szczawno-Zdrój – 6.\,6.\,1946 Jagniątków), \emph{Schriftsteller}|pw}
                     hatte darauf am 20. 4. kurz geantwortet: ›Der Weggang Dr. Rosenbaums\pwindex{Rosenbaum, Richard 4.\,11.\,1867 Žikov – 25.\,6.\,1942 Konzentrationslager Theresienstadt@\textsc{Rosenbaum, Richard} (4.\,11.\,1867 Žikov – 25.\,6.\,1942 Konzentrationslager Theresienstadt), \emph{Dramaturg, Verleger}|pw} vom Burgtheaters\orgindex{Burgtheater@Burgtheater|pw} hat mich sehr schmerzlich berührt, weil
                     ich weiß, mit welcher Hingebung er dem Institute verbunden ist. Ich begrüsse
                     Sie herzlich, danke Ihnen wärmstens für Ihre lieben Zeilen und füge ein paar
                     Abschiedsworte {[}\ldots{]} für Dr.
                        Rosenbaum\pwindex{Rosenbaum, Richard 4.\,11.\,1867 Žikov – 25.\,6.\,1942 Konzentrationslager Theresienstadt@\textsc{Rosenbaum, Richard} (4.\,11.\,1867 Žikov – 25.\,6.\,1942 Konzentrationslager Theresienstadt), \emph{Dramaturg, Verleger}|pw} hier bei.‹ Am
                        4. 5. bedankte Zweig\pwindex{Zweig, Stefan 28.\,11.\,1881 Wien – 23.\,2.\,1942 Petrópolis@\textsc{Zweig, Stefan} (28.\,11.\,1881 Wien – 23.\,2.\,1942 Petrópolis), \emph{Schriftsteller}|pw}
                     sich für Hauptmanns\pwindex{Hauptmann, Gerhart 15.\,11.\,1862 Szczawno-Zdrój – 6.\,6.\,1946 Jagniątków@\textsc{Hauptmann, Gerhart} (15.\,11.\,1862 Szczawno-Zdrój – 6.\,6.\,1946 Jagniątków), \emph{Schriftsteller}|pw} Brief und schrieb
                     (in einem ebenfalls unveröffentlichten Brief): ›Erst heute bekam ich Ihren
                     Brief vom 20. April, aber diesmals darf die Post nicht gescholten
                     sein: die ungeheuren Truppentransporte haben die Strecken für sich genommen und
                     für die verzögerte Freude einzelner Briefe haben wir heute die gemeinsame des
                     großen Sieges.‹« (\emph{Briefwechsel mit Hermann Bahr, Sigmund Freud, Rainer Maria
                        Rilke und Arthur Schnitzler}, S. 472.)}}}\label{K_L03654-3}: ist es die
               Post, die den Brief so lange hält oder irgend Etwas in ihm? Jedesfalls bin ich sehr
               erbittert, wie gut Thimig\pwindex{Thimig, Hugo 16.\,6.\,1854 Dresden – 24.\,9.\,1944 Wien@\textsc{Thimig, Hugo} (16.\,6.\,1854 Dresden – 24.\,9.\,1944 Wien), \emph{Theaterleiter, Schauspieler}|pw} alles gelungen ist.
               In aller Stille hat man diesen guten Mann\pwindex{Rosenbaum, Richard 4.\,11.\,1867 Žikov – 25.\,6.\,1942 Konzentrationslager Theresienstadt@\textsc{Rosenbaum, Richard} (4.\,11.\,1867 Žikov – 25.\,6.\,1942 Konzentrationslager Theresienstadt), \emph{Dramaturg, Verleger}|pwv} begraben und in einem Jahr wird niemand mehr von ihm
               wissen. Ich hoffe noch immer, etwas tun zu können: es wäre ja sehr nötig und nicht
               nur im moralischen Sinne, denn D\textsuperscript{r}{ }R\pwindex{Rosenbaum, Richard 4.\,11.\,1867 Žikov – 25.\,6.\,1942 Konzentrationslager Theresienstadt@\textsc{Rosenbaum, Richard} (4.\,11.\,1867 Žikov – 25.\,6.\,1942 Konzentrationslager Theresienstadt), \emph{Dramaturg, Verleger}|pw}, der jetzt ein Vierteljahrhundert in
               unablässiger Arbeit gelebt hat, braucht Wirksamkeit, um nicht bitter zu werden.
               Hoffentlich findet sich da ein Weg.\pend
           
\pstart
           Ich freue mich sehr, Sie {\pb}und Ihre
               verehrte Frau Gemahlin\pwindex{Schnitzler, Olga 17.\,1.\,1882 Wien – 13.\,1.\,1970 Lugano@\textsc{Schnitzler, Olga} (17.\,1.\,1882 Wien – 13.\,1.\,1970 Lugano), \emph{Schauspielerin, Sängerin}|pwv} bald
               wieder sehen zu dürfen: heute abends habe ich mir den \label{K_L03654-4v}\edtext{Sonatenabend Walter\pwindex{Walter, Bruno 15.\,9.\,1876 Berlin – 17.\,2.\,1962 Beverly Hills@\textsc{Walter, Bruno} (15.\,9.\,1876 Berlin – 17.\,2.\,1962 Beverly Hills), \emph{Theaterleiter, Komponist, Dirigent}|pw}{ }Rosé\pwindex{Rosé, Arnold 24.\,10.\,1863 Iași – 25.\,8.\,1946 London@\textsc{Rosé, Arnold} (24.\,10.\,1863 Iași – 25.\,8.\,1946 London), \emph{Violinist}|pw}\eventindex{Wiener Konzerthaus@\textbf{Wiener Konzerthaus}!Sonatenabend von Bruno Walter und Arnold Rosé, 26.4.1915@Sonatenabend von Bruno Walter und Arnold Rosé, 26.4.1915|pw}}{\lemma{\textnormal{\emph{Sonatenabend Walter Rosé}}}\Cendnote{\textnormal{Das gemeinsame Konzert von Bruno
                        Walter\pwindex{Walter, Bruno 15.\,9.\,1876 Berlin – 17.\,2.\,1962 Beverly Hills@\textsc{Walter, Bruno} (15.\,9.\,1876 Berlin – 17.\,2.\,1962 Beverly Hills), \emph{Theaterleiter, Komponist, Dirigent}|pwk} und Arnold Rosé\pwindex{Rosé, Arnold 24.\,10.\,1863 Iași – 25.\,8.\,1946 London@\textsc{Rosé, Arnold} (24.\,10.\,1863 Iași – 25.\,8.\,1946 London), \emph{Violinist}|pwk}\eventindex{Wiener Konzerthaus@\textbf{Wiener Konzerthaus}!Sonatenabend von Bruno Walter und Arnold Rosé, 26.4.1915@Sonatenabend von Bruno Walter und Arnold Rosé, 26.4.1915|pwk} fand im Mittleren Konzerthaussaal\oindex{Wien@\textbf{Wien}!III., Landstraße@\textbf{III., Landstraße}!Wiener Konzerthaus@\textbf{Wiener Konzerthaus}, \emph{Konzertsaal}|pwk}
                  statt.}}}\label{K_L03654-4}, \label{K_L03654-5v}\edtext{morgen das Lied von der Erde\pwindex{\textcolor{red}{\textsuperscript{XXXX indx1}}!Lied von der Erde@\strich\emph{Das Lied von der Erde}|pw}\eventindex{Musikverein@\textbf{Musikverein}!Aufführung von Das Lied von der Erde, 27.4.1915@Aufführung von Das Lied von der Erde, 27.4.1915|pw}}{\lemma{\textnormal{\emph{morgen … Erde}}}\Cendnote{\textnormal{Von den hier aufgeführten
                  Veranstaltungen besuchten Olga\pwindex{Schnitzler, Olga 17.\,1.\,1882 Wien – 13.\,1.\,1970 Lugano@\textsc{Schnitzler, Olga} (17.\,1.\,1882 Wien – 13.\,1.\,1970 Lugano), \emph{Schauspielerin, Sängerin}|pwk} und Arthur Schnitzler nur die Aufführung von \emph{Das Lied von
                           der Erde}\pwindex{\textcolor{red}{\textsuperscript{XXXX indx1}}!Lied von der Erde@\strich\emph{Das Lied von der Erde}|pwk}\eventindex{Musikverein@\textbf{Musikverein}!Aufführung von Das Lied von der Erde, 27.4.1915@Aufführung von Das Lied von der Erde, 27.4.1915|pwkv} am 27. 4. 1915 im
                  Großen Musikvereinssaal\oindex{Wien@\textbf{Wien}!I., Innere Stadt@\textbf{I., Innere Stadt}!Musikverein@\textbf{Musikverein}, \emph{Konzertsaal}|pwk}.}}}\label{K_L03654-5}, \label{K_L03654-6v}\edtext{Mittwoch{ }Elektra\pwindex{\textcolor{red}{\textsuperscript{XXXX indx1}}!Elektra [op. 58]@\strich\emph{Elektra [op. 58]}|pw}\eventindex{Oper@\textbf{Oper}!Aufführung von Elektra, 28.4.1915@Aufführung von Elektra, 28.4.1915|pwv}}{\lemma{\textnormal{\emph{Mittwoch Elektra}}}\Cendnote{\textnormal{Am 28.\,4.\,1915 wurde \emph{Elektra}\pwindex{\textcolor{red}{\textsuperscript{XXXX indx1}}!Elektra [op. 58]@\strich\emph{Elektra [op. 58]}|pwk} in der \emph{Wiener Oper}\orgindex{K.K. Hof-Oper@K.K. Hof-Oper|pwk}\eventindex{Oper@\textbf{Oper}!Aufführung von Elektra, 28.4.1915@Aufführung von Elektra, 28.4.1915|pwkv} gespielt.}}}\label{K_L03654-6} zugedacht, ich lebe jetzt wirklich von Musik, denn sonst
               wäre es nicht zu ertragen.\pend
           
\pstart
           In dankbarer Verehrung getreu Ihr{\\[\baselineskip]}\spacefill\mbox{Stefan Zweig}\pend
           \leftskip=0em{}
\pstart
           \noindent{}Viele Grüsse Ihrer Frau Gemahlin\pwindex{Schnitzler, Olga 17.\,1.\,1882 Wien – 13.\,1.\,1970 Lugano@\textsc{Schnitzler, Olga} (17.\,1.\,1882 Wien – 13.\,1.\,1970 Lugano), \emph{Schauspielerin, Sängerin}|pwv}! Und noch die \label{K_L03654-7v}\edtext{Erinnerung}{\lemma{\textnormal{\emph{Erinnerung}}}\Cendnote{\textnormal{Siehe XXXX Auszeichnungsfehler: Dokument L03645 nicht gefunden.}}}\label{K_L03654-7}: wenn Sie
                  einmal Zeit und Lust haben gedenken Sie jenes Bildhauers Gustinus Ambrosi\pwindex{Ambrosi, Gustinus 24.\,2.\,1893 Eisenstadt – 30.\,6.\,1975 Wien@\textsc{Ambrosi, Gustinus} (24.\,2.\,1893 Eisenstadt – 30.\,6.\,1975 Wien), \emph{Schriftsteller, Bildhauer, Philosoph}|pw}, der so gerne Ihre Büste machte. Ich
                  halte diesen taubstummen Menschen für einen wahrhaft genialen Künstler, \label{T_L03654-1v}\edtext{er ist auch menschlich, ein
                  unvergleichliches Erlebnis.}{\lemma{\textnormal{\emph{er … Erlebnis.}}}\Cendnote{\textnormal{seitlich
                     entlang des linken Blattrandes}}}\label{T_L03654-1}\pend
           \selectlanguage{ngerman}\endnumbering\briefempfaengerindex{Schnitzler, Arthur@\textsc{Schnitzler, Arthur}!zzzZweig, Stefan@\emph{von Stefan Zweig}!1915-04-261@{{[}26. 4. 1915{]}}|)be}\mylabel{L03654h}  \newcommand{\dateiname}{L03654}\newcommand{\titel}{Stefan Zweig an Arthur Schnitzler, [26. 4. 1915]}\newcommand{\editorInnen}{Selma Jahnke und Martin Anton Müller}%% latex-leseansicht-abspann.tex
%% Abspann für die Leseansicht.
%% Der Schalter \ifkorrekturansicht ist bereits durch den Vorspann gesetzt.

%% latex-abspann.tex
%% Gemeinsamer Abspann für Korrekturansicht und Leseansicht.
%% Setzt den Schalter \ifkorrekturansicht voraus (gesetzt in den
%% einbindenden Dateien latex-korrekturansicht-abspann.tex bzw.
%% latex-leseansicht-abspann.tex).
%% ---------------------------------------------------------------

\normalsize

% Das esempio-Environment wird nur in der Leseansicht benötigt
\ifkorrekturansicht\else
\newenvironment{esempio}[3]%
{
    \vspace{1.5ex}
    \rlap{\underline{#1}}
    \par
    \setlength{\parindent}{0cm}
    \nopagebreak
    \leftskip=#2cm
    \rightskip=#3cm
}
{
    \par
}
\fi

\doendnotes{C}
\bigskip
\vfill

\clearpage

\footnotesize

\ifkorrekturansicht
  \lohead{\textsc{register}}
\fi

% theindex-Environment neu definieren ohne reledmac
\makeatletter
\renewenvironment{theindex}{%
  \ifkorrekturansicht
    \section*{\indexname}%
  \else
    \subsubsection*{Index der erwähnten Entitäten}%
  \fi
  \setlength{\parindent}{0pt}%
  \setlength{\parskip}{0pt plus 0.3pt}%
  \let\item\@idxitem
}{%
  \ifkorrekturansicht\clearpage\fi
}
\makeatother

\IfFileExists{\jobname-pw.ind}{\input{\jobname-pw.ind}}{}

% Quellenangabe nur in der Leseansicht
\ifkorrekturansicht\else
% Fallback-Definitionen, falls die .tex-Datei \titel etc. nicht gesetzt hat
\providecommand{\titel}{}
\providecommand{\editorInnen}{}
\providecommand{\dateiname}{\jobname}

\vspace{3cm}

\vfill

\footnotesize
\textsc{Quelle}: \titel. Herausgegeben von {\editorInnen}. In: \emph{Arthur Schnitzler: Briefwechsel mit Autorinnen und Autoren}.
 Digitale Edition, https://schnitzler-briefe.acdh.oeaw.ac.at/{\dateiname}.html (Stand \today)
\fi

\end{document}


