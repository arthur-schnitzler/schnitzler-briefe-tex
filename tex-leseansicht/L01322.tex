%% latex-korrekturansicht-vorspann.tex
%% Vorspann für die Korrekturansicht.
%% Lädt die gemeinsame Datei latex-vorspann.tex mit gesetztem Schalter.

\newif\ifkorrekturansicht
\korrekturansichttrue

\input{../tex-inputs/latex-vorspann}


\section[Hugo von Hofmannsthal an Arthur Schnitzler, {[}Anfang Oktober 1903{]}]{L01322 Hugo von Hofmannsthal an Arthur Schnitzler,
               {[}Anfang Oktober 1903{]}}
\nopagebreak\mylabel{L01322v}
\rehead{ }\normalsize\beginnumbering\briefempfaengerindex{Schnitzler, Arthur@\textsc{Schnitzler, Arthur}!zzzHofmannsthal, Hugo von@\emph{von Hugo von Hofmannsthal}!1903-10-011@{{[}Anfang Oktober 1903{]}}|(be}
\toendnotes[C]{\smallbreak\pagebreak[2]}\Standort{CUL, Schnitzler, B 43.}
\physDesc{Brief, 1 Blatt, 1 Seite, 115 Zeichen
\newline{}Handschrift: Bleistift, deutsche Kurrent
\newline{}Schnitzler: mit Bleistift datiert: »Anf Oct. 903« 
\newline{}Ordnung: 1) mit Bleistift von unbekannter Hand nummeriert: »\strikeout{214}«  2) mit Bleistift von unbekannter Hand nummeriert:
                                    »203«}
\buchAbdrucke{\weitereDrucke{Hugo von Hofmannsthal, Arthur Schnitzler: \emph{Briefwechsel}. Frankfurt am Main: \emph{S. Fischer} 1964, S. 175.} }\toendnotes[C]{\smallbreak}
\pstart
           \noindent{}{\pb}Bitte \label{K_L01322-1v}\edtext{durchzuleſen}{\lemma{\textnormal{\emph{durchzuleſen}}}\Cendnote{\textnormal{vermutlich \emph{Elektra}\pwindex{Elektra. Tragoedie in einem Aufzug@\emph{Elektra. Tragödie in einem Aufzug}|pwk}}}}\label{K_L01322-1} und dann zu ſchicken an meinen Papa\pwindex{Hofmannsthal, Hugo August von 21.12.1841 – 08.12.1915@\textsc{Hofmannsthal, Hugo August von} (21.12.1841 – 08.12.1915), \emph{Bankdirektor/Bankdirektorin}|pwv}\pend
           \leftskip=3em{}
\pstart
           \noindent{}III Salesianergasse 12\oindex{Salesianergasse 12@\textbf{Salesianergasse 12}, \emph{Wohngebäude (K.WHS)}|pw}.\pend
           \leftskip=0em{}\pstart \spacefill\mbox{Hugo.}\pend{}
\pstart
           \noindent{}(Aufführung\pwindex{Elektra. Tragoedie in einem Aufzug@\emph{Elektra. Tragödie in einem Aufzug}|pwv} anſcheinend
                     \label{K_L01322-2v}\edtext{Ende October}{\lemma{\textnormal{\emph{Ende October}}}\Cendnote{\textnormal{Am 30. 10. 1903 wurde
                     \emph{Elektra}\pwindex{Elektra. Tragoedie in einem Aufzug@\emph{Elektra. Tragödie in einem Aufzug}|pwk} am \emph{Kleinen Theater}\orgindex{Kleines Theater@Kleines Theater|pwk} aufgeführt.}}}\label{K_L01322-2}.)\pend
           \selectlanguage{ngerman}\endnumbering\briefempfaengerindex{Schnitzler, Arthur@\textsc{Schnitzler, Arthur}!zzzHofmannsthal, Hugo von@\emph{von Hugo von Hofmannsthal}!1903-10-011@{{[}Anfang Oktober 1903{]}}|)be}\mylabel{L01322h}  \normalsize

\doendnotes{C}
\bigskip
\vfill

\clearpage

\footnotesize

\lohead{\textsc{register}}

% Definiere theindex-Environment komplett neu ohne reledmac
\makeatletter
\renewenvironment{theindex}{%
  \section*{\indexname}%
  \setlength{\parindent}{0pt}%
  \setlength{\parskip}{0pt plus 0.3pt}%
  \let\item\@idxitem
}{%
  \clearpage
}
\makeatother

\IfFileExists{\jobname-pw.ind}{\input{\jobname-pw.ind}}{}

\end{document}

      