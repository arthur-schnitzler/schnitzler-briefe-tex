%% latex-leseansicht-vorspann.tex
%% Vorspann für die Leseansicht.
%% Lädt die gemeinsame Datei latex-vorspann.tex mit nicht gesetztem Schalter.

\newif\ifkorrekturansicht
\korrekturansichtfalse

\input{../tex-inputs/latex-vorspann}


\section[Arthur Schnitzler an Detlev von Liliencron, 10. 12. 1896]{L00627 Arthur Schnitzler an Detlev von Liliencron, 10. 12. 1896}
\nopagebreak\mylabel{L00627v}
\rehead{ }\normalsize\beginnumbering\briefempfaengerindex{Liliencron, Detlev von@\textsc{Liliencron, Detlev von}!zzzSchnitzler, Arthur@\emph{von Arthur Schnitzler}!1896-12-101@{10. 12. 1896}|(be}
\toendnotes[C]{\smallbreak\pagebreak[2]}
\correspDesc{Versand  durch Arthur Schnitzler am 10. 12. 1896 in Wien
\newline{}Erhalt  durch Detlev von Liliencron am 11. 12. 1896 in Hamburg}\toendnotes[C]{\smallbreak}
\Standort{Kiel, Institut für Neuere deutsche Literatur und Medien an der Christian-Albrechts-Universität, DE-611-HS-149475.}
\physDesc{Postkarte, 162 Zeichen
\newline{}Handschrift: schwarze Tinte, deutsche Kurrent
\newline{}Versand: 1) Stempel: »\nobreak{}\oindex{IX., Alsergrund@\textbf{IX., Alsergrund}, \emph{Verwaltungsgebiet}|pwk}Wien 9/3, 10. 12. 96, 3–\textcolor{gray}{4} N\nobreak{}«.   2) Stempel: »\nobreak{}\oindex{Altona@\textbf{Altona}, \emph{Ehemaliger Ort}|pwk}Altona, 11. 12. 96, 6–7 N\nobreak{}«. }\pstart{}{\pb}Herrn \textsc{Detlev Frhrn v
                     Liliencron}\pend{}\pstart{}\textsc{Hamburg\oindex{Altona@\textbf{Altona}, \emph{Ehemaliger Ort}|pw}}\pend{}\pstart{}\textsc{Pallmaille 5\oindex{Palmaille@\textbf{Palmaille}, \emph{Straße}|pw}}\pend{}{\bigskip}\vspace{1em}
\pstart
           \noindent{}{\pb}Herzlichen Dank für die freundliche Erinnerung und
               viele Grüße von Ihrem Sie verehrenden\pend
           \pstart \spacefill\mbox{ArthSchnitzler}\pend{}
\pstart
           10. 12. 9\textcolor{gray}{6}.\pend
           \selectlanguage{ngerman}\endnumbering\briefempfaengerindex{Liliencron, Detlev von@\textsc{Liliencron, Detlev von}!zzzSchnitzler, Arthur@\emph{von Arthur Schnitzler}!1896-12-101@{10. 12. 1896}|)be}\mylabel{L00627h}  \newcommand{\dateiname}{L00627}\newcommand{\titel}{Arthur Schnitzler an Detlev von Liliencron, 10. 12. 1896}\newcommand{\editorInnen}{Martin Anton Müller und Gerd-Hermann Susen}%% latex-leseansicht-abspann.tex
%% Abspann für die Leseansicht.
%% Der Schalter \ifkorrekturansicht ist bereits durch den Vorspann gesetzt.

%% latex-abspann.tex
%% Gemeinsamer Abspann für Korrekturansicht und Leseansicht.
%% Setzt den Schalter \ifkorrekturansicht voraus (gesetzt in den
%% einbindenden Dateien latex-korrekturansicht-abspann.tex bzw.
%% latex-leseansicht-abspann.tex).
%% ---------------------------------------------------------------

\normalsize

% Das esempio-Environment wird nur in der Leseansicht benötigt
\ifkorrekturansicht\else
\newenvironment{esempio}[3]%
{
    \vspace{1.5ex}
    \rlap{\underline{#1}}
    \par
    \setlength{\parindent}{0cm}
    \nopagebreak
    \leftskip=#2cm
    \rightskip=#3cm
}
{
    \par
}
\fi

\doendnotes{C}
\bigskip
\vfill

\clearpage

\footnotesize

\ifkorrekturansicht
  \lohead{\textsc{register}}
\fi

% theindex-Environment neu definieren ohne reledmac
\makeatletter
\renewenvironment{theindex}{%
  \ifkorrekturansicht
    \section*{\indexname}%
  \else
    \subsubsection*{Index der erwähnten Entitäten}%
  \fi
  \setlength{\parindent}{0pt}%
  \setlength{\parskip}{0pt plus 0.3pt}%
  \let\item\@idxitem
}{%
  \ifkorrekturansicht\clearpage\fi
}
\makeatother

\IfFileExists{\jobname-pw.ind}{\input{\jobname-pw.ind}}{}

% Quellenangabe nur in der Leseansicht
\ifkorrekturansicht\else
% Fallback-Definitionen, falls die .tex-Datei \titel etc. nicht gesetzt hat
\providecommand{\titel}{}
\providecommand{\editorInnen}{}
\providecommand{\dateiname}{\jobname}

\vspace{3cm}

\vfill

\footnotesize
\textsc{Quelle}: \titel. Herausgegeben von {\editorInnen}. In: \emph{Arthur Schnitzler: Briefwechsel mit Autorinnen und Autoren}.
 Digitale Edition, https://schnitzler-briefe.acdh.oeaw.ac.at/{\dateiname}.html (Stand \today)
\fi

\end{document}


