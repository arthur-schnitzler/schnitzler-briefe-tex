%% latex-korrekturansicht-vorspann.tex
%% Vorspann für die Korrekturansicht.
%% Lädt die gemeinsame Datei latex-vorspann.tex mit gesetztem Schalter.

\newif\ifkorrekturansicht
\korrekturansichttrue

\input{../tex-inputs/latex-vorspann}


\section[Arthur Schnitzler an Detlev von Liliencron, 10. 12. 1896]{L00627 Arthur Schnitzler an Detlev von Liliencron, 10. 12. 1896}
\nopagebreak\mylabel{L00627v}
\rehead{ }\normalsize\beginnumbering\briefempfaengerindex{Liliencron, Detlev von@\textsc{Liliencron, Detlev von}!zzzSchnitzler, Arthur@\emph{von Arthur Schnitzler}!1896-12-101@{10. 12. 1896}|(be}
\toendnotes[C]{\smallbreak\pagebreak[2]}\Standort{Kiel, Institut für Neuere deutsche Literatur und Medien an der Christian-Albrechts-Universität, DE-611-HS-149475.}
\physDesc{Postkarte, 162 Zeichen
\newline{}Handschrift: schwarze Tinte, deutsche Kurrent
\newline{}Versand: 1) Stempel: »\nobreak{}\oindex{IX., Alsergrund@\textbf{IX., Alsergrund}, \emph{A.ADM3}|pwk}Wien 9/3, 10. 12. 96, 3–\textcolor{gray}{4} N\nobreak{}«.   2) Stempel: »\nobreak{}\oindex{Altona@\textbf{Altona}, \emph{P.PPLX}|pwk}Altona, 11. 12. 96, 6–7 N\nobreak{}«. }\pstart{}{\pb}Herrn \textsc{Detlev Frhrn v
                     Liliencron}\pend{}\pstart{}\textsc{Hamburg\oindex{Altona@\textbf{Altona}, \emph{P.PPLX}|pw}}\pend{}\pstart{}\textsc{Pallmaille 5\oindex{Palmaille@\textbf{Palmaille}, \emph{Straße (K.STR)}|pw}}\pend{}{\bigskip}\vspace{1em}
\pstart
           \noindent{}{\pb}Herzlichen Dank für die freundliche Erinnerung und
               viele Grüße von Ihrem Sie verehrenden\pend
           \pstart \spacefill\mbox{ArthSchnitzler}\pend{}
\pstart
           10. 12. 9\textcolor{gray}{6}. \pend
           \selectlanguage{ngerman}\endnumbering\briefempfaengerindex{Liliencron, Detlev von@\textsc{Liliencron, Detlev von}!zzzSchnitzler, Arthur@\emph{von Arthur Schnitzler}!1896-12-101@{10. 12. 1896}|)be}\mylabel{L00627h}  \normalsize

\doendnotes{C}
\bigskip
\vfill

\clearpage

\footnotesize

\lohead{\textsc{register}}

% Definiere theindex-Environment komplett neu ohne reledmac
\makeatletter
\renewenvironment{theindex}{%
  \section*{\indexname}%
  \setlength{\parindent}{0pt}%
  \setlength{\parskip}{0pt plus 0.3pt}%
  \let\item\@idxitem
}{%
  \clearpage
}
\makeatother

\IfFileExists{\jobname-pw.ind}{\input{\jobname-pw.ind}}{}

\end{document}

      