%% latex-korrekturansicht-vorspann.tex
%% Vorspann für die Korrekturansicht.
%% Lädt die gemeinsame Datei latex-vorspann.tex mit gesetztem Schalter.

\newif\ifkorrekturansicht
\korrekturansichttrue

\input{../tex-inputs/latex-vorspann}


\section[ Felix Salten an Arthur Schnitzler, 2. 1. 1902]{L03321 Felix Salten an Arthur Schnitzler, 2. 1. 1902}
\nopagebreak\mylabel{L03321v}
\rehead{ }\normalsize\beginnumbering\briefempfaengerindex{Schnitzler, Arthur@\textsc{Schnitzler, Arthur}!zzzSalten, Felix@\emph{von Felix Salten}!1902-01-021@{2. 1. 1902}|(be}
\toendnotes[C]{\smallbreak\pagebreak[2]}\Standort{CUL, Schnitzler, B 89, A 2.}
\physDesc{Postkarte, 463 Zeichen
\newline{}Handschrift: Bleistift, lateinische Kurrent
\newline{}Versand: 1) Stempel: »\nobreak{}\oindex{I., Innere Stadt@\textbf{I., Innere Stadt}, \emph{A.ADM3}|pwk}Wien 1/1 1, 2. 1. 02, 8–9 N\nobreak{}«.   2) Stempel: »\nobreak{}\textcolor{gray}{×}.
                                          \textcolor{gray}{1}. 02, Bestellt vom Postamte 64\nobreak{}«. 
\newline{}Schnitzler: mit Bleistift datiert: »2/1 902« 
\newline{}Ordnung: mit Bleistift von unbekannter Hand nummeriert: »145« }\toendnotes[C]{\smallbreak}\pstart{}{\pb}Herrn D\textsuperscript{r} Arthur Schnitzler\pend{}\pstart{}Berlin \strikeout{W.}\oindex{Berlin@\textbf{Berlin}, \emph{P.PPLC}|pw}\pend{}\pstart{}Hotel Bristol\oindex{Hotel Bristol Berlin@\textbf{Hotel Bristol Berlin}, \emph{Hotel (K.HTL)}|pw}\pend{}{\bigskip}\vspace{1em}
\pstart
           \noindent{}{\pb}Lieber, danke für Ihre \label{K_L03321-1v}\edtext{C. C.}{\lemma{\textnormal{\emph{C. C.}}}\Cendnote{\textnormal{Correspondenz-Carte}}}\label{K_L03321-1} und für Ihr frdl. Anerbieten. Wenn Sie \label{K_L03321-2v}\edtext{Entsch\pwindex{Entsch, Theodor 17.01.1853 – 19.12.1913@\textsc{Entsch, Theodor} (17.01.1853 – 19.12.1913), \emph{Verleger/Verlegerin, Theateragent/Theateragentin}|pw} sehen}{\lemma{\textnormal{\emph{Entsch sehen}}}\Cendnote{\textnormal{Schnitzler traf den Theateragenten und Verleger Theodor Entsch\pwindex{Entsch, Theodor 17.01.1853 – 19.12.1913@\textsc{Entsch, Theodor} (17.01.1853 – 19.12.1913), \emph{Verleger/Verlegerin, Theateragent/Theateragentin}|pwk} am 6. 1. 1902.}}}\label{K_L03321-2},
               dann bitte sagen Sie ihm, dass P. M.\pwindex{M., P. @\textsc{M., P.}, \emph{Theaterverleger/Theaterverlegerin}|pw} mein Stück\pwindex{Gemeine. Schauspiel in drei Aufzuegen@\emph{Der Gemeine. Schauspiel in drei Aufzügen}|pwv} gerne los wäre, dass ich
               es aber jedesfalls darauf ankommen laße, dass \uline{er} den
               Contract bricht. Wenn Sie mir Kerr\pwindex{Kerr, Alfred 25.12.1867 – 12.10.1948@\textsc{Kerr, Alfred} (25.12.1867 – 12.10.1948), \emph{Schriftsteller/Schriftstellerin, Kritiker/Kritikerin}|pw}’s Adreße
               angeben könnten, wäre ich Ihnen sehr dankbar. Wenn Sie Zeit haben, schreiben Sie mir
               ein paar Zeilen über den Ausgang von \label{K_L03321-3v}\edtext{Samstag{ }Abend}{\lemma{\textnormal{\emph{Samstag Abend}}}\Cendnote{\textnormal{Am Samstag, dem 4. 1. 1902 fand am
                     Deutschen Theater Berlin\oindex{Deutsches Theater Berlin@\textbf{Deutsches Theater Berlin}, \emph{Theater (K.THE)}|pwk} die Uraufführung der
                  vier Einakter \emph{Lebendige Stunden}\pwindex{Lebendige Stunden. Vier Einakter@\emph{Lebendige Stunden. Vier Einakter}|pwk}
               statt.}}}\label{K_L03321-3}. Grüßen Sie Goldmann\pwindex{Goldmann, Paul 31.01.1865 – 25.09.1935@\textsc{Goldmann, Paul} (31.01.1865 – 25.09.1935), \emph{Schriftsteller/Schriftstellerin, Journalist/Journalistin}|pw} ec.\pend
           
\pstart
           Herzlichst Ihr {\\[\baselineskip]}\spacefill\mbox{Salten}\pend
           \leftskip=0em{}\selectlanguage{ngerman}\endnumbering\briefempfaengerindex{Schnitzler, Arthur@\textsc{Schnitzler, Arthur}!zzzSalten, Felix@\emph{von Felix Salten}!1902-01-021@{2. 1. 1902}|)be}\mylabel{L03321h}  \normalsize

\doendnotes{C}
\bigskip
\vfill

\clearpage

\footnotesize

\lohead{\textsc{register}}

% Definiere theindex-Environment komplett neu ohne reledmac
\makeatletter
\renewenvironment{theindex}{%
  \section*{\indexname}%
  \setlength{\parindent}{0pt}%
  \setlength{\parskip}{0pt plus 0.3pt}%
  \let\item\@idxitem
}{%
  \clearpage
}
\makeatother

\IfFileExists{\jobname-pw.ind}{\input{\jobname-pw.ind}}{}

\end{document}

      