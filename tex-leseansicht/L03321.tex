%% latex-leseansicht-vorspann.tex
%% Vorspann für die Leseansicht.
%% Lädt die gemeinsame Datei latex-vorspann.tex mit nicht gesetztem Schalter.

\newif\ifkorrekturansicht
\korrekturansichtfalse

\input{../tex-inputs/latex-vorspann}


         
         \renewcommand{\erwaehntePersonen}{Personen: Theodor Entsch, Paul Goldmann, Alfred Kerr, P. M., Felix Salten}
         \renewcommand{\erwaehnteOrte}{Orte: Berlin, Deutsches Theater Berlin, Hotel Bristol Berlin, I., Innere Stadt, Wien}
         \renewcommand{\erwaehnteWerke}{Werke: Der Gemeine. Schauspiel in drei Aufzügen, Lebendige Stunden. Vier Einakter}
               \section[ Felix Salten an Arthur Schnitzler, 2. 1. 1902]{ Felix Salten an Arthur Schnitzler, 2. 1. 1902}\nopagebreak\mylabel{v}\rehead{ }\begin{ledgroupsized}[t]{13cm}\normalsize\beginnumbering\briefempfaengerindex{Schnitzler, Arthur@\textsc{Schnitzler, Arthur}!zzzSalten, Felix@\emph{von Felix Salten}!1902-01-021@{2. 1. 1902}|(be} \toendnotes[C]{\smallbreak\pagebreak[2]} \Standort{CUL, Schnitzler, B 89, A 2.}
\physDesc{Postkarte, 463 Zeichen
\newline{}Handschrift: Bleistift, lateinische Kurrent
\newline{}Versand: 1) Stempel: »\nobreak{}\oindex{I., Innere Stadt@\textbf{I., Innere Stadt}|pwk}Wien 1/1 1, 2. 1. 02, 8–9 N\nobreak{}«.   2) Stempel: »\nobreak{}\textcolor{gray}{×}.
                                          \textcolor{gray}{1}. 02, Bestellt vom Postamte 64\nobreak{}«. 
\newline{}Schnitzler: mit Bleistift datiert: »2/1 902« 
\newline{}Ordnung: mit Bleistift von unbekannter Hand nummeriert: »145« }\toendnotes[C]{\smallbreak}\pstart{}{\pb}Herrn D\textsuperscript{r} Arthur Schnitzler\pend{}\pstart{}Berlin \strikeout{W.}\oindex{Berlin@\textbf{Berlin}|pw}\pend{}\pstart{}Hotel Bristol\oindex{Hotel Bristol Berlin@\textbf{Hotel Bristol Berlin}|pw}\pend{}{\bigskip}\pstart
           \noindent{}{\pb}Lieber, danke für Ihre \label{K_L03321-1v}\edtext{C. C.}{\lemma{\textnormal{\emph{C. C.}}}\Cendnote{\textnormal{Correspondenz-Carte}}}\label{K_L03321-1h} und für Ihr frdl. Anerbieten. Wenn Sie \label{K_L03321-2v}\edtext{Entsch\pwindex{Entsch, Theodor 17.01.1853 – 19.12.1913@\textsc{Entsch, Theodor} (17.01.1853 – 19.12.1913), \emph{Verleger, Theateragent}|pw} sehen}{\lemma{\textnormal{\emph{Entsch sehen}}}\Cendnote{\textnormal{Schnitzler\pwindex{Schnitzler, Arthur 15.05.1862 – 21.10.1931@\textsc{Schnitzler, Arthur} (15.05.1862 – 21.10.1931), \emph{Schriftsteller, Mediziner}|pwk} traf den Theateragenten und Verleger Theodor Entsch\pwindex{Entsch, Theodor 17.01.1853 – 19.12.1913@\textsc{Entsch, Theodor} (17.01.1853 – 19.12.1913), \emph{Verleger, Theateragent}|pwk} am 6. 1. 1902.}}}\label{K_L03321-2h},
               dann bitte sagen Sie ihm, dass P. M.\pwindex{M., P. @\textsc{M., P.}, \emph{Theaterverleger}|pw} mein Stück\pwindex{Salten, Felix 06.09.1869 – 08.10.1945@\textsc{Salten, Felix} (06.09.1869 – 08.10.1945), \emph{Schriftsteller, Journalist, Chefredakteur}!Gemeine. Schauspiel in drei Aufzuegen1901@\strich\emph{Der Gemeine. Schauspiel in drei Aufzügen} {[}1901{]}|pwv} gerne los wäre, dass ich
               es aber jedesfalls darauf ankommen laße, dass \uline{er} den
               Contract bricht. Wenn Sie mir Kerr\pwindex{Kerr, Alfred 25.12.1867 – 12.10.1948@\textsc{Kerr, Alfred} (25.12.1867 – 12.10.1948), \emph{Schriftsteller, Kritiker}|pw}’s Adreße
               angeben könnten, wäre ich Ihnen sehr dankbar. Wenn Sie Zeit haben, schreiben Sie mir
               ein paar Zeilen über den Ausgang von \label{K_L03321-3v}\edtext{Samstag{ }Abend}{\lemma{\textnormal{\emph{Samstag Abend}}}\Cendnote{\textnormal{Am Samstag, dem 4. 1. 1902 fand am
                     Deutschen Theater Berlin\oindex{Deutsches Theater Berlin@\textbf{Deutsches Theater Berlin}|pwk} die Uraufführung der
                  vier Einakter \emph{Lebendige Stunden}\pwindex{Schnitzler, Arthur 15.05.1862 – 21.10.1931@\textsc{Schnitzler, Arthur} (15.05.1862 – 21.10.1931), \emph{Schriftsteller, Mediziner}!Lebendige Stunden. Vier Einakter1901-12-23@\strich\emph{Lebendige Stunden. Vier Einakter} {[}1901-12-23{]}|pwk}
               statt.}}}\label{K_L03321-3h}. Grüßen Sie Goldmann\pwindex{Goldmann, Paul 31.01.1865 – 25.09.1935@\textsc{Goldmann, Paul} (31.01.1865 – 25.09.1935), \emph{Schriftsteller, Journalist}|pw} ec.\pend
           \pstart
           Herzlichst Ihr {\\[\baselineskip]}\spacefill\mbox{Salten}\pend
           \leftskip=0em{}
         
         \endnumbering\mylabel{h}\end{ledgroupsized}  \newcommand{\dateiname}{L03321}\newcommand{\titel}{Felix Salten an Arthur Schnitzler, 2. 1. 1902}\newcommand{\editorInnen}{Martin Anton Müller und Laura Untner}%% latex-leseansicht-abspann.tex
%% Abspann für die Leseansicht.
%% Der Schalter \ifkorrekturansicht ist bereits durch den Vorspann gesetzt.

%% latex-abspann.tex
%% Gemeinsamer Abspann für Korrekturansicht und Leseansicht.
%% Setzt den Schalter \ifkorrekturansicht voraus (gesetzt in den
%% einbindenden Dateien latex-korrekturansicht-abspann.tex bzw.
%% latex-leseansicht-abspann.tex).
%% ---------------------------------------------------------------

\normalsize

% Das esempio-Environment wird nur in der Leseansicht benötigt
\ifkorrekturansicht\else
\newenvironment{esempio}[3]%
{
    \vspace{1.5ex}
    \rlap{\underline{#1}}
    \par
    \setlength{\parindent}{0cm}
    \nopagebreak
    \leftskip=#2cm
    \rightskip=#3cm
}
{
    \par
}
\fi

\doendnotes{C}
\bigskip
\vfill

\clearpage

\footnotesize

\ifkorrekturansicht
  \lohead{\textsc{register}}
\fi

% theindex-Environment neu definieren ohne reledmac
\makeatletter
\renewenvironment{theindex}{%
  \ifkorrekturansicht
    \section*{\indexname}%
  \else
    \subsubsection*{Index der erwähnten Entitäten}%
  \fi
  \setlength{\parindent}{0pt}%
  \setlength{\parskip}{0pt plus 0.3pt}%
  \let\item\@idxitem
}{%
  \ifkorrekturansicht\clearpage\fi
}
\makeatother

\IfFileExists{\jobname-pw.ind}{\input{\jobname-pw.ind}}{}

% Quellenangabe nur in der Leseansicht
\ifkorrekturansicht\else
% Fallback-Definitionen, falls die .tex-Datei \titel etc. nicht gesetzt hat
\providecommand{\titel}{}
\providecommand{\editorInnen}{}
\providecommand{\dateiname}{\jobname}

\vspace{3cm}

\vfill

\footnotesize
\textsc{Quelle}: \titel. Herausgegeben von {\editorInnen}. In: \emph{Arthur Schnitzler: Briefwechsel mit Autorinnen und Autoren}.
 Digitale Edition, https://schnitzler-briefe.acdh.oeaw.ac.at/{\dateiname}.html (Stand \today)
\fi

\end{document}


      