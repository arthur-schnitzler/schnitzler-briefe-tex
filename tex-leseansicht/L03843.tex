%% latex-leseansicht-vorspann.tex
%% Vorspann für die Leseansicht.
%% Lädt die gemeinsame Datei latex-vorspann.tex mit nicht gesetztem Schalter.

\newif\ifkorrekturansicht
\korrekturansichtfalse

\input{../tex-inputs/latex-vorspann}


\section[Theodor Herzl an Arthur Schnitzler, 1. 1. 1895]{L03843 Theodor Herzl an Arthur Schnitzler, 1. 1. 1895}
\nopagebreak\mylabel{L03843v}
\rehead{ }\normalsize\beginnumbering\briefempfaengerindex{Schnitzler, Arthur@\textsc{Schnitzler, Arthur}!zzzHerzl, Theodor@\emph{von Theodor Herzl}!1895-01-011@{1. 1. 1895}|(be}
\toendnotes[C]{\smallbreak\pagebreak[2]}
\correspDesc{Versand  durch Theodor Herzl am 1. 1. 1895 in Paris
\newline{}Erhalt  durch Arthur Schnitzler im Zeitraum [2. 1. 1895
                  – 6. 1. 1895?] in Wien}\toendnotes[C]{\smallbreak}
\Standort{CUL, Schnitzler, B 39.}
\physDesc{Brief, 1 Blatt, 3 Seiten, 5159 Zeichen
\newline{}Handschrift: schwarze Tinte, lateinische Kurrent
\newline{}Beilage: Briefkonzept von Theodor
                                    Herzl\pwindex{Herzl, Theodor 2.\,5.\,1860 Budapest – 3.\,7.\,1904 Edlach@\textsc{Herzl, Theodor} (2.\,5.\,1860 Budapest – 3.\,7.\,1904 Edlach), \emph{Schriftsteller, Journalist}|pw} an Theaterdirektionen in Berlin, 1 Blatt, 2 Seiten,
                                 schwarze Tinte, lateinische Kurrent, Schnitzler: mit Bleistift zwei
                                 Korrekturen 
\newline{}Ordnung: mit Bleistift von unbekannter Hand nummeriert: »22« }
\buchAbdrucke{\weitereDrucke{Theodor Herzl: \emph{Briefe und autobiographische Notizen 1866–1895}. Bearbeitet von Johannes Wachten in Zusammenarbeit mit Chaya Harel, Daisy Tycho und Manfred Winkler. Berlin, Frankfurt am Main, Wien: \emph{Propyläen} 1983, S. 566–567 (Briefe und Tagebücher. Herausgegeben von Alex Bein, Hermann Greive, Moshe Schaerf, Julius H. Schoeps und Johannes Wachten, 1).} }\toendnotes[C]{\smallbreak}
\pstart
           {\pb}\textcolor{gray}{\textbf{NOUVELLE PRESSE LIBRE}}\orgindex{Neue Freie Presse@Neue Freie Presse|pw}\hfill \textcolor{gray}{\textbf{8, RUE DE MONCEAU }}\oindex{8, rue de Monceau@\textbf{8, rue de Monceau}, \emph{Wohngebäude}|pw}\pend
           
\pstart
           \textcolor{gray}{\textbf{D\textsuperscript{r}{ }TH. HERZL}}\hfill 1. I. 95\pend
           
\pstart{}Mein lieber Schnitzler!\pend\vspace{0.5em}
\pstart
           Nun haben Sie mein ganzes Werk\pwindex{Herzl, Theodor 2.\,5.\,1860 Budapest – 3.\,7.\,1904 Edlach@\textsc{Herzl, Theodor} (2.\,5.\,1860 Budapest – 3.\,7.\,1904 Edlach), \emph{Schriftsteller, Journalist}!neue Ghetto. Schauspiel in vier Acten@\strich\emph{Das neue Ghetto. Schauspiel in vier Acten}|pwv}. Bald werden Sie die Hauptmühe überstanden haben.\pend
           
\pstart
           Sie kennen meine Ungeduld in dem Augenblick wo ich etwas definitiv aus der Hand
               gegeben habe. Dann möchte ich, dass es eilig erledigt werde, im Guten oder Schlechten
               – es \substVorne{}\textsuperscript{soll}\substDazwischen{}soll\substHinten{} nur fort aus meinen Sorgen. So möchte ich mir nun auch das G.\pwindex{Herzl, Theodor 2.\,5.\,1860 Budapest – 3.\,7.\,1904 Edlach@\textsc{Herzl, Theodor} (2.\,5.\,1860 Budapest – 3.\,7.\,1904 Edlach), \emph{Schriftsteller, Journalist}!neue Ghetto. Schauspiel in vier Acten@\strich\emph{Das neue Ghetto. Schauspiel in vier Acten}|pwv} bald aus dem Kopf geschlagen haben.\pend
           
\pstart
           Ich bitte Sie mir folgende Fragen zu beantworten:\pend
           
\pstart
           Hat der Abschreiber\pwindex{?? [Schreibkraft für Arthur Schnitzler] @\textsc{?? [Schreibkraft für Arthur Schnitzler]}|pwu} bei Ihnen
               geschrieben?\pend
           
\pstart
           Ist das Geheimniss gewahrt geblieben?\pend
           
\pstart
           Wann ist er fertig geworden?\pend
           
\pstart
           Hatte er eine schöne Schrift?\pend
           
\pstart
           \strikeout{Wann ist das Mscpt ab}\pend
           
\pstart
           \label{K_L03843-1v}\edtext{Meinen Nachtrag}{\lemma{\textnormal{\emph{Meinen Nachtrag}}}\Cendnote{\textnormal{Siehe XXXX Auszeichnungsfehler: Dokument L03842 nicht gefunden.}}}\label{K_L03843-1}, die
               Berichtigung auf Seite 9 (III A.) haben Sie erhalten?\pend
           
\pstart
           Einzuflicken \label{K_L03843-2v}\edtext{»Monat \hebraeisch{Ab} 5112«}{\lemma{\textnormal{\emph{»Monat … 5112«}}}\Cendnote{\textnormal{Der Monat \hebraeisch{אָב} fällt auf Juli
                  oder August und ist der elfte Monat im jüdischen Kalender, das jüdische Jahr 5112
                  entspricht dem Jahr 1352 in christlicher Zeitrechnung. Die Figur des Rabbiners
                  Dr. Friedheimer referiert in der sechsten Szene des dritten Aktes von Herzls\pwindex{Herzl, Theodor 2.\,5.\,1860 Budapest – 3.\,7.\,1904 Edlach@\textsc{Herzl, Theodor} (2.\,5.\,1860 Budapest – 3.\,7.\,1904 Edlach), \emph{Schriftsteller, Journalist}|pwk} Schauspiel \emph{Das neue Ghetto}\pwindex{Herzl, Theodor 2.\,5.\,1860 Budapest – 3.\,7.\,1904 Edlach@\textsc{Herzl, Theodor} (2.\,5.\,1860 Budapest – 3.\,7.\,1904 Edlach), \emph{Schriftsteller, Journalist}!neue Ghetto. Schauspiel in vier Acten@\strich\emph{Das neue Ghetto. Schauspiel in vier Acten}|pwk} eine Anekdote aus einer alten Chronik mit
                  dieser Datierung, die allerings in der Druckfassung auf den Monat »Ab des Jahres 5143« des jüdischen Kalenders geändert wird, s. Theodor Herzl\pwindex{Herzl, Theodor 2.\,5.\,1860 Budapest – 3.\,7.\,1904 Edlach@\textsc{Herzl, Theodor} (2.\,5.\,1860 Budapest – 3.\,7.\,1904 Edlach), \emph{Schriftsteller, Journalist}|pwk}: \emph{Das neue Ghetto. Schauspiel in 4 Acten}\pwindex{Herzl, Theodor 2.\,5.\,1860 Budapest – 3.\,7.\,1904 Edlach@\textsc{Herzl, Theodor} (2.\,5.\,1860 Budapest – 3.\,7.\,1904 Edlach), \emph{Schriftsteller, Journalist}!neue Ghetto. Schauspiel in vier Acten@\strich\emph{Das neue Ghetto. Schauspiel in vier Acten}|pwk},
                     Wien: \emph{Buchdruckerei »Industrie« –
                        Selbstverlag}{ }1903, S. 74.}}}\label{K_L03843-2}\pend
           
\pstart
           Weiters bitte ich Sie das Mscpt\pwindex{Herzl, Theodor 2.\,5.\,1860 Budapest – 3.\,7.\,1904 Edlach@\textsc{Herzl, Theodor} (2.\,5.\,1860 Budapest – 3.\,7.\,1904 Edlach), \emph{Schriftsteller, Journalist}!neue Ghetto. Schauspiel in vier Acten@\strich\emph{Das neue Ghetto. Schauspiel in vier Acten}|pwv} von Anfang an durchzulesen und alle Paren{\pb}thesen-Spielbemerkungen mit Bleistift zu
               unterzeichnen. Das erleichtert das Lesen.\pend
           
\pstart
           In der Beilage finden Sie die Einbegleitung. Sie besteht \label{K_L03843-3v}\edtext{aus zwei Theilen}{\lemma{\textnormal{\emph{aus zwei Theilen}}}\Cendnote{\textnormal{Das Vorwort siehe anbei in Form der »Vorbemerkung für den
                  Director«, das kurze Briefkonzept an den ersten Theaterdirektor (Otto Brahm\pwindex{Brahm, Otto 5.\,2.\,1856 Hamburg – 28.\,11.\,1912 Berlin@\textsc{Brahm, Otto} (5.\,2.\,1856 Hamburg – 28.\,11.\,1912 Berlin), \emph{Theaterleiter, Regisseur}|pwk}) ist nicht überliefert.}}}\label{K_L03843-3}.
               1.) ein Vorwort welches auf das Respectblatt vor der Titelseite zu schreiben ist
                  (\label{K_L03843-4v}\edtext{u. zw. von der Hand des
                  Abschreibers}{\lemma{\textnormal{\emph{u. … Abschreibers}}}\Cendnote{\textnormal{Neben Herzls\pwindex{Herzl, Theodor 2.\,5.\,1860 Budapest – 3.\,7.\,1904 Edlach@\textsc{Herzl, Theodor} (2.\,5.\,1860 Budapest – 3.\,7.\,1904 Edlach), \emph{Schriftsteller, Journalist}|pwk} Entwurf in der Beilage befindet sich in Schnitzlers Nachlass in der \emph{Cambridge University Library}\orgindex{Cambridge University Library@Cambridge University Library|pwk} eine Abschrift des Textes von
                  Schreiberhand.}}}\label{K_L03843-4}) oder, wenn kein Respectblatt frei gelassen wurde, in
               auffälliger Weise hinzukleben ist.\pend
           
\pstart
           2.) ein kurzer \label{K_L03843-5v}\edtext{Brief}{\lemma{\textnormal{\emph{Brief}}}\Cendnote{\textnormal{nicht überliefert}}}\label{K_L03843-5} an den ersten Director\pwindex{Blumenthal, Oskar 13.\,3.\,1852 Berlin – 24.\,4.\,1917 ebd.@\textsc{Blumenthal, Oskar} (13.\,3.\,1852 Berlin – 24.\,4.\,1917 ebd.), \emph{Schriftsteller, Journalist, Theaterleiter}|pw}\pwindex{Brahm, Otto 5.\,2.\,1856 Hamburg – 28.\,11.\,1912 Berlin@\textsc{Brahm, Otto} (5.\,2.\,1856 Hamburg – 28.\,11.\,1912 Berlin), \emph{Theaterleiter, Regisseur}|pw}. Dieser Brief hat gesondert
               recommandirt am \strikeout{selben} Tage nach Absendung des Mscpts\pwindex{Herzl, Theodor 2.\,5.\,1860 Budapest – 3.\,7.\,1904 Edlach@\textsc{Herzl, Theodor} (2.\,5.\,1860 Budapest – 3.\,7.\,1904 Edlach), \emph{Schriftsteller, Journalist}!neue Ghetto. Schauspiel in vier Acten@\strich\emph{Das neue Ghetto. Schauspiel in vier Acten}|pwv} abzugehen. Zu adressiren
               ist dieser Brief an Blumenthal\pwindex{Blumenthal, Oskar 13.\,3.\,1852 Berlin – 24.\,4.\,1917 ebd.@\textsc{Blumenthal, Oskar} (13.\,3.\,1852 Berlin – 24.\,4.\,1917 ebd.), \emph{Schriftsteller, Journalist, Theaterleiter}|pw} oder Brahm\pwindex{Brahm, Otto 5.\,2.\,1856 Hamburg – 28.\,11.\,1912 Berlin@\textsc{Brahm, Otto} (5.\,2.\,1856 Hamburg – 28.\,11.\,1912 Berlin), \emph{Theaterleiter, Regisseur}|pw}. Ich stelle es Ihnen ganz frei, möchte
               aber nicht, dass Sie sich darin irgendwie von Gefälligkeit leiten lassen. \uline{Mir} ist es absolut gleichgiltig. ich würde es Ihnen
               freimüthig sagen, wenn ich ein Interesse daran hätte, zuerst {\pb}zu Brahm\pwindex{Brahm, Otto 5.\,2.\,1856 Hamburg – 28.\,11.\,1912 Berlin@\textsc{Brahm, Otto} (5.\,2.\,1856 Hamburg – 28.\,11.\,1912 Berlin), \emph{Theaterleiter, Regisseur}|pw} zu gehen. Umsomehr als es vermuthlich ohnehin eine Rundreise werden
               wird. Also thun Sie, was Sie wollen, nur müssen Sie danach den Text des Vorwortes
               einrichten, nämlich die Reihenfolge der Bühnen. Den Brief 2.) möchte ich nicht von
               der Hand des Abschreibers\pwindex{?? [Schreibkraft für Arthur Schnitzler] @\textsc{?? [Schreibkraft für Arthur Schnitzler]}|pwu}, sondern
               von einer anderen haben, u. zw. Schicks\pwindex{Schik, Friedrich *~6.\,9.\,1857 Wien@\textsc{Schik, Friedrich} (*~6.\,9.\,1857 Wien), \emph{Notar, Journalist, Dramaturg}|pw}. Bitte
               also mein Concept zuerst vom Abschreiber\pwindex{?? [Schreibkraft für Arthur Schnitzler] @\textsc{?? [Schreibkraft für Arthur Schnitzler]}|pwu} machen zu lassen u. dann Schick\pwindex{Schik, Friedrich *~6.\,9.\,1857 Wien@\textsc{Schik, Friedrich} (*~6.\,9.\,1857 Wien), \emph{Notar, Journalist, Dramaturg}|pw} zu bitten, dass er die paar Zeilen schreibe.\pend
           
\pstart
           Verständigen Sie mich, lieber Freund, vom Tag des glücklichen Abgangs. Denn von da ab
               werde ich die Stunden zählen. Ich rechne, dass das Mscpt\pwindex{Herzl, Theodor 2.\,5.\,1860 Budapest – 3.\,7.\,1904 Edlach@\textsc{Herzl, Theodor} (2.\,5.\,1860 Budapest – 3.\,7.\,1904 Edlach), \emph{Schriftsteller, Journalist}!neue Ghetto. Schauspiel in vier Acten@\strich\emph{Das neue Ghetto. Schauspiel in vier Acten}|pwv}{ }Freitag den 4 oder Samstag den 5 ds Mts. von Wien\oindex{Wien@\textbf{Wien}, \emph{Verwaltungsgebiet}|pw} abgehen kann.\pend
           
\pstart
           Tausend Dank und herzliche Grüsse {\\[\baselineskip]}Ihr{\\[\baselineskip]}\spacefill\mbox{Th H}\pend
           \leftskip=0em{}
\pstart
           \noindent{}Wenn Sie mich nicht ärgern wollen, bitte ich um sofortige Einsendung des
                  Kostenverzeichnisses.\pend
           \selectlanguage{ngerman}\vspace{1em}
\pstart
           \noindent{}\centering{}{\pb}I\pend
           
\pstart
           \centering{}\uline{Vorbemerkung für den Director.}\pend
           
\pstart
           Der Verfasser täuscht sich nicht über die Schwierigkeiten, mit welchen dieses Stück\pwindex{Herzl, Theodor 2.\,5.\,1860 Budapest – 3.\,7.\,1904 Edlach@\textsc{Herzl, Theodor} (2.\,5.\,1860 Budapest – 3.\,7.\,1904 Edlach), \emph{Schriftsteller, Journalist}!neue Ghetto. Schauspiel in vier Acten@\strich\emph{Das neue Ghetto. Schauspiel in vier Acten}|pwv} zu kämpfen haben wird. Es
               nimmt zu einer heissen Zeitfrage das Wort. Das kann die Aufführung freilich
               ebensowohl erleichtern als erschweren.\pend
           
\pstart
           Der Verfasser weiss auch, dass sein in der Literatur bisher unbekannter Name keine
               Empfehlung ist. Es gehört Vorurtheilslosigkeit dazu, ein solches Werk\pwindex{Herzl, Theodor 2.\,5.\,1860 Budapest – 3.\,7.\,1904 Edlach@\textsc{Herzl, Theodor} (2.\,5.\,1860 Budapest – 3.\,7.\,1904 Edlach), \emph{Schriftsteller, Journalist}!neue Ghetto. Schauspiel in vier Acten@\strich\emph{Das neue Ghetto. Schauspiel in vier Acten}|pwv} aufmerksam zu lesen. Der Verfasser
               verlangt aber noch mehr, und muss es verlangen. Er fordert eine Entscheidung über die
               Annahme binnen kurzer Zeit.\pend
           
\pstart
           Dieses Stück\pwindex{Herzl, Theodor 2.\,5.\,1860 Budapest – 3.\,7.\,1904 Edlach@\textsc{Herzl, Theodor} (2.\,5.\,1860 Budapest – 3.\,7.\,1904 Edlach), \emph{Schriftsteller, Journalist}!neue Ghetto. Schauspiel in vier Acten@\strich\emph{Das neue Ghetto. Schauspiel in vier Acten}|pwv} darf nicht durch
               theatermässige Verschleppungen um eine nutzbare Zeit gebracht werden. Ist die
               Aufführung nicht bald zu erreichen, so wird das Werk\pwindex{Herzl, Theodor 2.\,5.\,1860 Budapest – 3.\,7.\,1904 Edlach@\textsc{Herzl, Theodor} (2.\,5.\,1860 Budapest – 3.\,7.\,1904 Edlach), \emph{Schriftsteller, Journalist}!neue Ghetto. Schauspiel in vier Acten@\strich\emph{Das neue Ghetto. Schauspiel in vier Acten}|pwv} einfach als Buch erscheinen, genau in seiner
               vorliegenden Form, vom ersten Worte dieser Vorbemerkung angefangen bis zum letzten
               »der Vorhang fällt.« Dann wird das Publicum ohnehin erfahren, was der Verfasser sagen
               will. Und darauf kommt es wesentlich an.\pend
           
\pstart
           Die Directoren\pwindex{Blumenthal, Oskar 13.\,3.\,1852 Berlin – 24.\,4.\,1917 ebd.@\textsc{Blumenthal, Oskar} (13.\,3.\,1852 Berlin – 24.\,4.\,1917 ebd.), \emph{Schriftsteller, Journalist, Theaterleiter}|pwv}\pwindex{Brahm, Otto 5.\,2.\,1856 Hamburg – 28.\,11.\,1912 Berlin@\textsc{Brahm, Otto} (5.\,2.\,1856 Hamburg – 28.\,11.\,1912 Berlin), \emph{Theaterleiter, Regisseur}|pwv}
               dreier Bühnen\orgindex{Deutsches Theater Berlin@Deutsches Theater Berlin|pwv}\orgindex{Lessing-Theater@Lessing-Theater|pwv}\orgindex{Berliner Theater@Berliner Theater|pwv} erhalten das Stück\pwindex{Herzl, Theodor 2.\,5.\,1860 Budapest – 3.\,7.\,1904 Edlach@\textsc{Herzl, Theodor} (2.\,5.\,1860 Budapest – 3.\,7.\,1904 Edlach), \emph{Schriftsteller, Journalist}!neue Ghetto. Schauspiel in vier Acten@\strich\emph{Das neue Ghetto. Schauspiel in vier Acten}|pwv} nacheinander. Jeder kann es drei Wochen behalten. Nach Ablauf dieser
               unwiderruflichen Fallfrist wolle es der Ablehnende gütigst dem Nächsten zuschicken.
               Die Reihenfolge ist 1. Lessing-Theater\orgindex{Lessing-Theater@Lessing-Theater|pw} 2. Deutsches Theater\orgindex{Deutsches Theater Berlin@Deutsches Theater Berlin|pw}, 3. Berliner Theater\orgindex{Berliner Theater@Berliner Theater|pw}. Die Gefälligkeit dieser Weitergabe glaubt
               der Verfasser dafür erbitten zu dürfen, dass er seine Arbeit\pwindex{Herzl, Theodor 2.\,5.\,1860 Budapest – 3.\,7.\,1904 Edlach@\textsc{Herzl, Theodor} (2.\,5.\,1860 Budapest – 3.\,7.\,1904 Edlach), \emph{Schriftsteller, Journalist}!neue Ghetto. Schauspiel in vier Acten@\strich\emph{Das neue Ghetto. Schauspiel in vier Acten}|pwv} den Herren\pwindex{Blumenthal, Oskar 13.\,3.\,1852 Berlin – 24.\,4.\,1917 ebd.@\textsc{Blumenthal, Oskar} (13.\,3.\,1852 Berlin – 24.\,4.\,1917 ebd.), \emph{Schriftsteller, Journalist, Theaterleiter}|pwv}\pwindex{Brahm, Otto 5.\,2.\,1856 Hamburg – 28.\,11.\,1912 Berlin@\textsc{Brahm, Otto} (5.\,2.\,1856 Hamburg – 28.\,11.\,1912 Berlin), \emph{Theaterleiter, Regisseur}|pwv} unterbreitet. Es soll dadurch unnützer Zeitverlust
               erspart werden.\pend
           
\pstart
           Nimmt auch der Dritte\pwindex{Blumenthal, Oskar 13.\,3.\,1852 Berlin – 24.\,4.\,1917 ebd.@\textsc{Blumenthal, Oskar} (13.\,3.\,1852 Berlin – 24.\,4.\,1917 ebd.), \emph{Schriftsteller, Journalist, Theaterleiter}|pwv} nicht
               an, so wolle dieser das Manuscript\pwindex{Herzl, Theodor 2.\,5.\,1860 Budapest – 3.\,7.\,1904 Edlach@\textsc{Herzl, Theodor} (2.\,5.\,1860 Budapest – 3.\,7.\,1904 Edlach), \emph{Schriftsteller, Journalist}!neue Ghetto. Schauspiel in vier Acten@\strich\emph{Das neue Ghetto. Schauspiel in vier Acten}|pwv} dem Obmann\pwindex{Schlenther, Paul 20.\,8.\,1854 Chernyakhovsk – 30.\,4.\,1916 Berlin@\textsc{Schlenther, Paul} (20.\,8.\,1854 Chernyakhovsk – 30.\,4.\,1916 Berlin), \emph{Schriftsteller, Kritiker, Theaterleiter}|pwv} der »Freien Bühne\orgindex{Freie Bühne@Freie Bühne|pw}« zuschicken,
               welch Letzterer\pwindex{Schlenther, Paul 20.\,8.\,1854 Chernyakhovsk – 30.\,4.\,1916 Berlin@\textsc{Schlenther, Paul} (20.\,8.\,1854 Chernyakhovsk – 30.\,4.\,1916 Berlin), \emph{Schriftsteller, Kritiker, Theaterleiter}|pwv} es im Falle
               der Ablehnung an den weiterhin bezeichneten Freund\pwindex{Schik, Friedrich *~6.\,9.\,1857 Wien@\textsc{Schik, Friedrich} (*~6.\,9.\,1857 Wien), \emph{Notar, Journalist, Dramaturg}|pwv}{ }{\pb}des Verfassers freundlichst zurücksenden
               möge.\pend
           
\pstart
           Selbstverständlich unterbleibt, die Weiterreise, wenn einer der Directoren\pwindex{Blumenthal, Oskar 13.\,3.\,1852 Berlin – 24.\,4.\,1917 ebd.@\textsc{Blumenthal, Oskar} (13.\,3.\,1852 Berlin – 24.\,4.\,1917 ebd.), \emph{Schriftsteller, Journalist, Theaterleiter}|pwv}\pwindex{Brahm, Otto 5.\,2.\,1856 Hamburg – 28.\,11.\,1912 Berlin@\textsc{Brahm, Otto} (5.\,2.\,1856 Hamburg – 28.\,11.\,1912 Berlin), \emph{Theaterleiter, Regisseur}|pwv}\pwindex{Schlenther, Paul 20.\,8.\,1854 Chernyakhovsk – 30.\,4.\,1916 Berlin@\textsc{Schlenther, Paul} (20.\,8.\,1854 Chernyakhovsk – 30.\,4.\,1916 Berlin), \emph{Schriftsteller, Kritiker, Theaterleiter}|pwv} das Stück\pwindex{Herzl, Theodor 2.\,5.\,1860 Budapest – 3.\,7.\,1904 Edlach@\textsc{Herzl, Theodor} (2.\,5.\,1860 Budapest – 3.\,7.\,1904 Edlach), \emph{Schriftsteller, Journalist}!neue Ghetto. Schauspiel in vier Acten@\strich\emph{Das neue Ghetto. Schauspiel in vier Acten}|pwv}{ }\strikeout{z} annimmt. Die Aufführung hat innerhalb eines Monats
               nach der förmlichen Annahme zu erfolgen. Auch diese Fallfrist ist genügend. Der
               Director, der es mit dem Stücke\pwindex{Herzl, Theodor 2.\,5.\,1860 Budapest – 3.\,7.\,1904 Edlach@\textsc{Herzl, Theodor} (2.\,5.\,1860 Budapest – 3.\,7.\,1904 Edlach), \emph{Schriftsteller, Journalist}!neue Ghetto. Schauspiel in vier Acten@\strich\emph{Das neue Ghetto. Schauspiel in vier Acten}|pwv} ernst meint, hat ausreichende Zeit, es einstudiren zu lassen und
               dafür den Spielplan freizumachen.\pend
           
\pstart
           Die im Vorstehenden erbetene Weiterbeförderung ist vielleicht etwas Ungewöhnliches.
               Der Verfasser glaubt jedoch nicht, dass man solche Ablehnungen verschweigen müsse. Im
               Gegentheil, man soll sie bekanntgeben.\pend
           
\pstart
           Die Bühnenleiter, welche dieses Werk\pwindex{Herzl, Theodor 2.\,5.\,1860 Budapest – 3.\,7.\,1904 Edlach@\textsc{Herzl, Theodor} (2.\,5.\,1860 Budapest – 3.\,7.\,1904 Edlach), \emph{Schriftsteller, Journalist}!neue Ghetto. Schauspiel in vier Acten@\strich\emph{Das neue Ghetto. Schauspiel in vier Acten}|pwv} später in die Hand bekommen, mögen ohne Vorurtheil daran gehen. Es ist
               nicht einmal vorgekommen, dass Directoren sich über den Bühnenwerth eines Stückes
               irrten.\pend
           
\pstart
           Am Text des Manuscripts\pwindex{Herzl, Theodor 2.\,5.\,1860 Budapest – 3.\,7.\,1904 Edlach@\textsc{Herzl, Theodor} (2.\,5.\,1860 Budapest – 3.\,7.\,1904 Edlach), \emph{Schriftsteller, Journalist}!neue Ghetto. Schauspiel in vier Acten@\strich\emph{Das neue Ghetto. Schauspiel in vier Acten}|pwv} darf
               nichts geändert werden, auch die Bezeichnung als Schauspiel ist unveränderlich.\pend
           
\pstart
           Für den Vertragsschluss und alle Abmachungen, die nöthig werden \strikeout{könnten, stellt der Verfasser einen Bevollmächtigten auf in
                  der Person des Herrn Friedrich Schick\pwindex{Schik, Friedrich *~6.\,9.\,1857 Wien@\textsc{Schik, Friedrich} (*~6.\,9.\,1857 Wien), \emph{Notar, Journalist, Dramaturg}|pw}, Wien III\oindex{III., Landstraße@\textbf{III., Landstraße}, \emph{Verwaltungsgebiet}|pw}{ }\textcolor{gray}{×}\-\textcolor{gray}{×}\-\textcolor{gray}{×}\-\textcolor{gray}{×}\-\textcolor{gray}{×}{ }Reisnerstrasse 25\oindex{Wien@\textbf{Wien}!III., Landstraße@\textbf{III., Landstraße}!Reisnerstraße 25@\textbf{Reisnerstraße 25}, \emph{Wohngebäude}|pw}} könnten, stellt der Verfasser – weil er häufig auf \introOben{}\strikeout{Italien\oindex{Italien@\textbf{Italien}|pw}-}\introOben{}Reisen geht – einen Bevollmächtigten auf in der Person des Herrn \strikeout{(Dr)}{ }Friedrich Schick\pwindex{Schik, Friedrich *~6.\,9.\,1857 Wien@\textsc{Schik, Friedrich} (*~6.\,9.\,1857 Wien), \emph{Notar, Journalist, Dramaturg}|pw}, Wien III \label{K_L03843-6v}\edtext{Reisnerstrasse 25}{\lemma{\textnormal{\emph{Reisnerstrasse 25}}}\Cendnote{\textnormal{Friedrich Schik
                     wohnte in der Reisnerstraße 35\oindex{Wien@\textbf{Wien}!III., Landstraße@\textbf{III., Landstraße}!Reisnerstraße 35@\textbf{Reisnerstraße 35}, \emph{Wohngebäude}|pwk},
                     allerdings hatte Schnitzler{ }Herzl\pwindex{Herzl, Theodor 2.\,5.\,1860 Budapest – 3.\,7.\,1904 Edlach@\textsc{Herzl, Theodor} (2.\,5.\,1860 Budapest – 3.\,7.\,1904 Edlach), \emph{Schriftsteller, Journalist}|pwk} zunächst eine falsche Hausnummer
                     genannt, s. XXXX17.11.1894.}}}\label{K_L03843-6}\oindex{Wien@\textbf{Wien}!III., Landstraße@\textbf{III., Landstraße}!Reisnerstraße 25@\textbf{Reisnerstraße 25}, \emph{Wohngebäude}|pw}\pend
           
\pstart
           \centering{}– . –\pend
           
\pstart
           \raggedleft{}Wien\oindex{Wien@\textbf{Wien}, \emph{Verwaltungsgebiet}|pw} am 4 Januar 1894 \pend
           \selectlanguage{ngerman}\endnumbering\briefempfaengerindex{Schnitzler, Arthur@\textsc{Schnitzler, Arthur}!zzzHerzl, Theodor@\emph{von Theodor Herzl}!1895-01-011@{1. 1. 1895}|)be}\mylabel{L03843h}
\begin{anhang}
\end{anhang}\newcommand{\dateiname}{L03843}\newcommand{\titel}{Theodor Herzl an Arthur Schnitzler, 1. 1. 1895}\newcommand{\editorInnen}{Selma Jahnke und Martin Anton Müller}%% latex-leseansicht-abspann.tex
%% Abspann für die Leseansicht.
%% Der Schalter \ifkorrekturansicht ist bereits durch den Vorspann gesetzt.

%% latex-abspann.tex
%% Gemeinsamer Abspann für Korrekturansicht und Leseansicht.
%% Setzt den Schalter \ifkorrekturansicht voraus (gesetzt in den
%% einbindenden Dateien latex-korrekturansicht-abspann.tex bzw.
%% latex-leseansicht-abspann.tex).
%% ---------------------------------------------------------------

\normalsize

% Das esempio-Environment wird nur in der Leseansicht benötigt
\ifkorrekturansicht\else
\newenvironment{esempio}[3]%
{
    \vspace{1.5ex}
    \rlap{\underline{#1}}
    \par
    \setlength{\parindent}{0cm}
    \nopagebreak
    \leftskip=#2cm
    \rightskip=#3cm
}
{
    \par
}
\fi

\doendnotes{C}
\bigskip
\vfill

\clearpage

\footnotesize

\ifkorrekturansicht
  \lohead{\textsc{register}}
\fi

% theindex-Environment neu definieren ohne reledmac
\makeatletter
\renewenvironment{theindex}{%
  \ifkorrekturansicht
    \section*{\indexname}%
  \else
    \subsubsection*{Index der erwähnten Entitäten}%
  \fi
  \setlength{\parindent}{0pt}%
  \setlength{\parskip}{0pt plus 0.3pt}%
  \let\item\@idxitem
}{%
  \ifkorrekturansicht\clearpage\fi
}
\makeatother

\IfFileExists{\jobname-pw.ind}{\input{\jobname-pw.ind}}{}

% Quellenangabe nur in der Leseansicht
\ifkorrekturansicht\else
% Fallback-Definitionen, falls die .tex-Datei \titel etc. nicht gesetzt hat
\providecommand{\titel}{}
\providecommand{\editorInnen}{}
\providecommand{\dateiname}{\jobname}

\vspace{3cm}

\vfill

\footnotesize
\textsc{Quelle}: \titel. Herausgegeben von {\editorInnen}. In: \emph{Arthur Schnitzler: Briefwechsel mit Autorinnen und Autoren}.
 Digitale Edition, https://schnitzler-briefe.acdh.oeaw.ac.at/{\dateiname}.html (Stand \today)
\fi

\end{document}


