%% latex-leseansicht-vorspann.tex
%% Vorspann für die Leseansicht.
%% Lädt die gemeinsame Datei latex-vorspann.tex mit nicht gesetztem Schalter.

\newif\ifkorrekturansicht
\korrekturansichtfalse

\input{../tex-inputs/latex-vorspann}


\section[Arthur Schnitzler an Richard Beer-Hofmann, 29. 12. 1892]{L00149 Arthur Schnitzler an Richard Beer-Hofmann, 29. 12. 1892}
\nopagebreak\mylabel{L00149v}
\rehead{ }\normalsize\beginnumbering\briefempfaengerindex{Beer-Hofmann, Richard@\textsc{Beer-Hofmann, Richard}!zzzSchnitzler, Arthur@\emph{von Arthur Schnitzler}!1892-12-291@{29. 12. 1892}|(be}
\toendnotes[C]{\smallbreak\pagebreak[2]}
\correspDesc{Versand  durch Arthur Schnitzler am 29. 12. 1892 in Wien
\newline{}Erhalt  durch Richard Beer-Hofmann im Zeitraum [29. 12. 1892 – 2. 1. 1893?] in Wien}\toendnotes[C]{\smallbreak}
\Standort{YCGL, MSS 31.}
\physDesc{Kartenbrief, 242 Zeichen
\newline{}Handschrift: Bleistift, deutsche Kurrent
\newline{}Versand: Stempel: »\nobreak{}\oindex{Wien@\textbf{Wien}, \emph{Verwaltungsgebiet}|pwk}Wien 1/1, 29. 12. 92, 7–8 N\nobreak{}«.  }\pstart{}{\pb}Hrn \textsc{Doctor Richard}\pend{}\pstart{}\textsc{Beer-Hofmann}\pend{}\pstart{}\textsc{Wien\oindex{Wien@\textbf{Wien}, \emph{Verwaltungsgebiet}|pw}}\pend{}\pstart{}I Wollzeile 15\oindex{Wien@\textbf{Wien}!I., Innere Stadt@\textbf{I., Innere Stadt}!Wollzeile 15 (»Berthahof«)@\textbf{Wollzeile 15 (»Berthahof«)}, \emph{Wohngebäude}|pw}\pend{}{\bigskip}\vspace{1em}
\pstart{}{\pb}Lieber Richard!\pend\vspace{0.5em}
\pstart
           Paul Horn\pwindex{Horn, Paul 13.\,2.\,1867 Wien – 18.\,1.\,1936 Menton@\textsc{Horn, Paul} (13.\,2.\,1867 Wien – 18.\,1.\,1936 Menton), \emph{Fabrikant}|pw} hat abgeſchrieben, ich theils bei
                  \textsc{Frau Fl.\pwindex{Flegmann, Bertha 27.\,5.\,1852 Dubrovsky, Polen – 24.\,6.\,1933 Bad Ischl@\textsc{Flegmann, Bertha} (27.\,5.\,1852 Dubrovsky, Polen – 24.\,6.\,1933 Bad Ischl), \emph{Salonnière}|pw}} mit.\pend
           
\pstart
           Alſo morgen iſt nichts. –\pend
           
\pstart
           Wann{ }ſehn wir uns wieder? Ich kann in der Sylveſternacht{ }ſehr{ }ſpät ins Pfob\oindex{Wien@\textbf{Wien}!I., Innere Stadt@\textbf{I., Innere Stadt}!Café Pfob@\textbf{Café Pfob}, \emph{Kaffeehaus}|pw},{ }ſo um 2. Herzlichst\pend
           \pstart Ihr \spacefill\mbox{Arthur}\pend{}\selectlanguage{ngerman}\endnumbering\briefempfaengerindex{Beer-Hofmann, Richard@\textsc{Beer-Hofmann, Richard}!zzzSchnitzler, Arthur@\emph{von Arthur Schnitzler}!1892-12-291@{29. 12. 1892}|)be}\mylabel{L00149h}  \newcommand{\dateiname}{L00149}\newcommand{\titel}{Arthur Schnitzler an Richard Beer-Hofmann, 29. 12. 1892}\newcommand{\editorInnen}{Martin Anton Müller und Gerd-Hermann Susen}%% latex-leseansicht-abspann.tex
%% Abspann für die Leseansicht.
%% Der Schalter \ifkorrekturansicht ist bereits durch den Vorspann gesetzt.

%% latex-abspann.tex
%% Gemeinsamer Abspann für Korrekturansicht und Leseansicht.
%% Setzt den Schalter \ifkorrekturansicht voraus (gesetzt in den
%% einbindenden Dateien latex-korrekturansicht-abspann.tex bzw.
%% latex-leseansicht-abspann.tex).
%% ---------------------------------------------------------------

\normalsize

% Das esempio-Environment wird nur in der Leseansicht benötigt
\ifkorrekturansicht\else
\newenvironment{esempio}[3]%
{
    \vspace{1.5ex}
    \rlap{\underline{#1}}
    \par
    \setlength{\parindent}{0cm}
    \nopagebreak
    \leftskip=#2cm
    \rightskip=#3cm
}
{
    \par
}
\fi

\doendnotes{C}
\bigskip
\vfill

\clearpage

\footnotesize

\ifkorrekturansicht
  \lohead{\textsc{register}}
\fi

% theindex-Environment neu definieren ohne reledmac
\makeatletter
\renewenvironment{theindex}{%
  \ifkorrekturansicht
    \section*{\indexname}%
  \else
    \subsubsection*{Index der erwähnten Entitäten}%
  \fi
  \setlength{\parindent}{0pt}%
  \setlength{\parskip}{0pt plus 0.3pt}%
  \let\item\@idxitem
}{%
  \ifkorrekturansicht\clearpage\fi
}
\makeatother

\IfFileExists{\jobname-pw.ind}{\input{\jobname-pw.ind}}{}

% Quellenangabe nur in der Leseansicht
\ifkorrekturansicht\else
% Fallback-Definitionen, falls die .tex-Datei \titel etc. nicht gesetzt hat
\providecommand{\titel}{}
\providecommand{\editorInnen}{}
\providecommand{\dateiname}{\jobname}

\vspace{3cm}

\vfill

\footnotesize
\textsc{Quelle}: \titel. Herausgegeben von {\editorInnen}. In: \emph{Arthur Schnitzler: Briefwechsel mit Autorinnen und Autoren}.
 Digitale Edition, https://schnitzler-briefe.acdh.oeaw.ac.at/{\dateiname}.html (Stand \today)
\fi

\end{document}


