%% latex-leseansicht-vorspann.tex
%% Vorspann für die Leseansicht.
%% Lädt die gemeinsame Datei latex-vorspann.tex mit nicht gesetztem Schalter.

\newif\ifkorrekturansicht
\korrekturansichtfalse

\input{../tex-inputs/latex-vorspann}


\section[Adalbert Seligmann an Arthur Schnitzler, 13. 10. 1905]{L01563 Adalbert Seligmann an Arthur Schnitzler, 13. 10. 1905}
\nopagebreak\mylabel{L01563v}
\rehead{ }\normalsize\beginnumbering\briefempfaengerindex{Schnitzler, Arthur@\textsc{Schnitzler, Arthur}!zzzSeligmann, Adalbert Franz@\emph{von Adalbert Franz Seligmann}!1905-10-134@{13. 10. 1905}|(be}
\toendnotes[C]{\smallbreak\pagebreak[2]}
\correspDesc{Versand  durch Adalbert Seligmann am 13. 10. 1905 in Wien
\newline{}Erhalt  durch Arthur Schnitzler im Zeitraum [13. 10. 1905 – 17. 10. 1905?] in Wien}\toendnotes[C]{\smallbreak}
\Standort{CUL, Schnitzler, B 97.}
\physDesc{Brief, 1 Blatt, 3 Seiten, 1334 Zeichen
\newline{}Handschrift: schwarze Tinte, deutsche Kurrent
\newline{}Schnitzler: 1) mit Bleistift beschriftet: »\textsc{Seligma{\geminationn}}«  2) mit rotem Buntstift eine Unterstreichung}\toendnotes[C]{\smallbreak}
\pstart
           \noindent{}{\pb}Verehrteſter! Wenn Sie wüßten, wieviel es
               braucht, um mich zu einem Brief zu zwingen,{ }ſo könnten Sie daraus allein ermeſſen,
               wie groß die Bewunderung iſt, die ich für Ihr »Zwischenspiel\pwindex{Schnitzler, Arthur 15.\,5.\,1862 Wien – 21.\,10.\,1931 ebd.@\textsc{Schnitzler, Arthur} (15.\,5.\,1862 Wien – 21.\,10.\,1931 ebd.), \emph{Schriftsteller, Mediziner}!Zwischenspiel. Komödie in drei Akten@\strich\emph{Zwischenspiel. Komödie in drei Akten}|pw}« habe«. Was mir aber außerdem noch die Feder in die Hand
               drückt, iſt das Gefühl – verzeihen Sie dieſe Arroganz – daſs ich zu den nicht gar
               Vielen gehöre\strikeout{n}, die das Stück verſtanden haben, und
               daſs ich Ihnen gleich{ }ſagen möchte wie wundervoll ich die Pſychologie darin finde, es
               Ihnen{ }ſagen möchte, bevor die Dickhäuter kommen, die Ihnen verſichern werden, daſs es
               zwar {\pb}geiſtvoll, aber zu compliciert iſt! Wenn{ }ſchon ein ganz feiner
               Kopf, wie Wittmann\pwindex{Wittmann, Hugo 16.\,10.\,1839 Ulm – 6.\,2.\,1923 Wien@\textsc{Wittmann, Hugo} (16.\,10.\,1839 Ulm – 6.\,2.\,1923 Wien), \emph{Schriftsteller, Journalist}|pw}{ }ſich \label{K_L01563-1v}\edtext{nicht darin zurecht findet\pwindex{Burgtheater. [Zwischenspiel]@\emph{Burgtheater. [Zwischenspiel]}|pwv}}{\lemma{\textnormal{\emph{nicht … findet}}}\Cendnote{\textnormal{In seiner Nachtkritik\pwindex{Burgtheater. [Zwischenspiel]@\emph{Burgtheater. [Zwischenspiel]}|pwkv} schreibt – nicht namentlich
                  genannt – Wittmann\pwindex{Wittmann, Hugo 16.\,10.\,1839 Ulm – 6.\,2.\,1923 Wien@\textsc{Wittmann, Hugo} (16.\,10.\,1839 Ulm – 6.\,2.\,1923 Wien), \emph{Schriftsteller, Journalist}|pwk}: »Ein Ibsen\pwindex{Ibsen, Henrik 20.\,3.\,1828 Skien – 23.\,5.\,1906 Oslo@\textsc{Ibsen, Henrik} (20.\,3.\,1828 Skien – 23.\,5.\,1906 Oslo), \emph{Schriftsteller}|pw}-Problem im Grunde, aber schrecklich
                     verkünstelt und hineingepflanzt in einen psychologischen Irrgarten, wo die
                     Menschen auf dem Kopf zu tanzen scheinen.« (\emph{Neue Freie Presse}\pwindex{Neue Freie Presse@\emph{Neue Freie Presse}|pwk}, Nr. 14.778,
                        13. 10. 1905, S. 9).}}}\label{K_L01563-1}, wie{ }ſollen dann die vielen
               Andern folgen können? Ich finde es unglaublich wahr, und mit prachtvoller Conſequenz
               angelegt und durchgeführt. Es{ }ſind eben wirklich, wie der »Raiſonneur«{ }ſagt, zwei
                  \strikeout{unglaublich}{ }\uline{feine} Menſchen, zwiſchen denen{ }ſich das alles
               abſpielt, abſpielen muß! Soll ich noch hinzufügen daſs ich die Oekonomie und den
               Aufbau ganz vollendet finde? Ich will Sie nicht langweilen – was ich Ihnen{ }ſagen
               könnte, wiſſen Sie {\pb}ja alles – und noch viel mehr;{ }ſonſt hätten Sie das
                  Stück\pwindex{Schnitzler, Arthur 15.\,5.\,1862 Wien – 21.\,10.\,1931 ebd.@\textsc{Schnitzler, Arthur} (15.\,5.\,1862 Wien – 21.\,10.\,1931 ebd.), \emph{Schriftsteller, Mediziner}!Zwischenspiel. Komödie in drei Akten@\strich\emph{Zwischenspiel. Komödie in drei Akten}|pwv} ja nicht geſchrieben,
               ein Stück\pwindex{Schnitzler, Arthur 15.\,5.\,1862 Wien – 21.\,10.\,1931 ebd.@\textsc{Schnitzler, Arthur} (15.\,5.\,1862 Wien – 21.\,10.\,1931 ebd.), \emph{Schriftsteller, Mediziner}!Zwischenspiel. Komödie in drei Akten@\strich\emph{Zwischenspiel. Komödie in drei Akten}|pwv}, von dem ich
               überzeugt bin, daſs man es erſt in zwanzig Jahren richtig erfaſſen und würdigen
               wird.\pend
           
\pstart
           Mit den beſten Empfehlungen{\\[\baselineskip]}Ihr\spacefill\mbox{A. F. Seligmann}\pend
           \leftskip=0em{}
\pstart
           13/10 1905\pend
           \selectlanguage{ngerman}\endnumbering\briefempfaengerindex{Schnitzler, Arthur@\textsc{Schnitzler, Arthur}!zzzSeligmann, Adalbert Franz@\emph{von Adalbert Franz Seligmann}!1905-10-134@{13. 10. 1905}|)be}\mylabel{L01563h}  \newcommand{\dateiname}{L01563}\newcommand{\titel}{Adalbert Seligmann an Arthur Schnitzler, 13. 10. 1905}\newcommand{\editorInnen}{Martin Anton Müller und Gerd-Hermann Susen}%% latex-leseansicht-abspann.tex
%% Abspann für die Leseansicht.
%% Der Schalter \ifkorrekturansicht ist bereits durch den Vorspann gesetzt.

%% latex-abspann.tex
%% Gemeinsamer Abspann für Korrekturansicht und Leseansicht.
%% Setzt den Schalter \ifkorrekturansicht voraus (gesetzt in den
%% einbindenden Dateien latex-korrekturansicht-abspann.tex bzw.
%% latex-leseansicht-abspann.tex).
%% ---------------------------------------------------------------

\normalsize

% Das esempio-Environment wird nur in der Leseansicht benötigt
\ifkorrekturansicht\else
\newenvironment{esempio}[3]%
{
    \vspace{1.5ex}
    \rlap{\underline{#1}}
    \par
    \setlength{\parindent}{0cm}
    \nopagebreak
    \leftskip=#2cm
    \rightskip=#3cm
}
{
    \par
}
\fi

\doendnotes{C}
\bigskip
\vfill

\clearpage

\footnotesize

\ifkorrekturansicht
  \lohead{\textsc{register}}
\fi

% theindex-Environment neu definieren ohne reledmac
\makeatletter
\renewenvironment{theindex}{%
  \ifkorrekturansicht
    \section*{\indexname}%
  \else
    \subsubsection*{Index der erwähnten Entitäten}%
  \fi
  \setlength{\parindent}{0pt}%
  \setlength{\parskip}{0pt plus 0.3pt}%
  \let\item\@idxitem
}{%
  \ifkorrekturansicht\clearpage\fi
}
\makeatother

\IfFileExists{\jobname-pw.ind}{\input{\jobname-pw.ind}}{}

% Quellenangabe nur in der Leseansicht
\ifkorrekturansicht\else
% Fallback-Definitionen, falls die .tex-Datei \titel etc. nicht gesetzt hat
\providecommand{\titel}{}
\providecommand{\editorInnen}{}
\providecommand{\dateiname}{\jobname}

\vspace{3cm}

\vfill

\footnotesize
\textsc{Quelle}: \titel. Herausgegeben von {\editorInnen}. In: \emph{Arthur Schnitzler: Briefwechsel mit Autorinnen und Autoren}.
 Digitale Edition, https://schnitzler-briefe.acdh.oeaw.ac.at/{\dateiname}.html (Stand \today)
\fi

\end{document}


