%% latex-korrekturansicht-vorspann.tex
%% Vorspann für die Korrekturansicht.
%% Lädt die gemeinsame Datei latex-vorspann.tex mit gesetztem Schalter.

\newif\ifkorrekturansicht
\korrekturansichttrue

\input{../tex-inputs/latex-vorspann}


\section[Adalbert Seligmann an Arthur Schnitzler, 13. 10. 1905]{L01563 Adalbert Seligmann an Arthur Schnitzler, 13. 10. 1905}
\nopagebreak\mylabel{L01563v}
\rehead{ }\normalsize\beginnumbering\briefempfaengerindex{Schnitzler, Arthur@\textsc{Schnitzler, Arthur}!zzzSeligmann, Adalbert Franz@\emph{von Adalbert Franz Seligmann}!1905-10-133@{13. 10. 1905}|(be}
\toendnotes[C]{\smallbreak\pagebreak[2]}\Standort{CUL, Schnitzler, B 97.}
\physDesc{Brief, 1 Blatt, 3 Seiten, 1334 Zeichen
\newline{}Handschrift: schwarze Tinte, deutsche Kurrent
\newline{}Schnitzler: 1) mit Bleistift beschriftet: »\textsc{Seligma{\geminationn}}«  2) mit rotem Buntstift eine Unterstreichung}\toendnotes[C]{\smallbreak}
\pstart
           \noindent{}{\pb}Verehrteſter! Wenn Sie wüßten, wieviel es
               braucht, um mich zu einem Brief zu zwingen, ſo könnten Sie daraus allein ermeſſen,
               wie groß die Bewunderung iſt, die ich für Ihr »Zwischenspiel\pwindex{Zwischenspiel. Komoedie in drei Akten@\emph{Zwischenspiel. Komödie in drei Akten}|pw}« habe«. Was mir aber außerdem noch die Feder in die Hand
               drückt, iſt das Gefühl – verzeihen Sie dieſe Arroganz – daſs ich zu den nicht gar
               Vielen gehöre\strikeout{n}, die das Stück verſtanden haben, und
               daſs ich Ihnen gleich ſagen möchte wie wundervoll ich die Pſychologie darin finde, es
               Ihnen ſagen möchte, bevor die Dickhäuter kommen, die Ihnen verſichern werden, daſs es
               zwar {\pb}geiſtvoll, aber zu compliciert iſt! Wenn ſchon ein ganz feiner
               Kopf, wie Wittmann\pwindex{Wittmann, Hugo 16.10.1839 – 06.02.1923@\textsc{Wittmann, Hugo} (16.10.1839 – 06.02.1923), \emph{Schriftsteller/Schriftstellerin, Journalist/Journalistin}|pw} ſich \label{K_L01563-1v}\edtext{nicht darin zurecht findet\pwindex{Burgtheater. [Zwischenspiel]@\emph{Burgtheater. [Zwischenspiel]}|pwv}}{\lemma{\textnormal{\emph{nicht … findet}}}\Cendnote{\textnormal{In seiner Nachtkritik\pwindex{Burgtheater. [Zwischenspiel]@\emph{Burgtheater. [Zwischenspiel]}|pwkv} schreibt – nicht namentlich
                  genannt – Wittmann\pwindex{Wittmann, Hugo 16.10.1839 – 06.02.1923@\textsc{Wittmann, Hugo} (16.10.1839 – 06.02.1923), \emph{Schriftsteller/Schriftstellerin, Journalist/Journalistin}|pwk}: »Ein Ibsen\pwindex{Ibsen, Henrik 20.03.1828 – 23.05.1906@\textsc{Ibsen, Henrik} (20.03.1828 – 23.05.1906), \emph{Schriftsteller/Schriftstellerin}|pw}-Problem im Grunde, aber schrecklich
                     verkünstelt und hineingepflanzt in einen psychologischen Irrgarten, wo die
                     Menschen auf dem Kopf zu tanzen scheinen.« (\emph{Neue Freie Presse}\pwindex{Neue Freie Presse@\emph{Neue Freie Presse}|pwk}, Nr. 14.778,
                        13. 10. 1905, S. 9).}}}\label{K_L01563-1}, wie ſollen dann die vielen
               Andern folgen können? Ich finde es unglaublich wahr, und mit prachtvoller Conſequenz
               angelegt und durchgeführt. Es ſind eben wirklich, wie der »Raiſonneur« ſagt, zwei
                  \strikeout{unglaublich}{ }\uline{feine} Menſchen, zwiſchen denen ſich das alles
               abſpielt, abſpielen muß! Soll ich noch hinzufügen daſs ich die Oekonomie und den
               Aufbau ganz vollendet finde? Ich will Sie nicht langweilen – was ich Ihnen ſagen
               könnte, wiſſen Sie {\pb}ja alles – und noch viel mehr; ſonſt hätten Sie das
                  Stück\pwindex{Zwischenspiel. Komoedie in drei Akten@\emph{Zwischenspiel. Komödie in drei Akten}|pwv} ja nicht geſchrieben,
               ein Stück\pwindex{Zwischenspiel. Komoedie in drei Akten@\emph{Zwischenspiel. Komödie in drei Akten}|pwv}, von dem ich
               überzeugt bin, daſs man es erſt in zwanzig Jahren richtig erfaſſen und würdigen
               wird.\pend
           
\pstart
           Mit den beſten Empfehlungen{\\[\baselineskip]}Ihr\spacefill\mbox{A. F. Seligmann}\pend
           \leftskip=0em{}
\pstart
           13/10 1905\pend
           \selectlanguage{ngerman}\endnumbering\briefempfaengerindex{Schnitzler, Arthur@\textsc{Schnitzler, Arthur}!zzzSeligmann, Adalbert Franz@\emph{von Adalbert Franz Seligmann}!1905-10-133@{13. 10. 1905}|)be}\mylabel{L01563h}  \normalsize

\doendnotes{C}
\bigskip
\vfill

\clearpage

\footnotesize

\lohead{\textsc{register}}

% Definiere theindex-Environment komplett neu ohne reledmac
\makeatletter
\renewenvironment{theindex}{%
  \section*{\indexname}%
  \setlength{\parindent}{0pt}%
  \setlength{\parskip}{0pt plus 0.3pt}%
  \let\item\@idxitem
}{%
  \clearpage
}
\makeatother

\IfFileExists{\jobname-pw.ind}{\input{\jobname-pw.ind}}{}

\end{document}

      