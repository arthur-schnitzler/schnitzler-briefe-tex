%% latex-leseansicht-vorspann.tex
%% Vorspann für die Leseansicht.
%% Lädt die gemeinsame Datei latex-vorspann.tex mit nicht gesetztem Schalter.

\newif\ifkorrekturansicht
\korrekturansichtfalse

\input{../tex-inputs/latex-vorspann}


\section[Arthur Schnitzler an Richard Beer-Hofmann, 28. 6. 1898]{L00809 Arthur Schnitzler an Richard Beer-Hofmann, 28. 6. 1898}
\nopagebreak\mylabel{L00809v}
\rehead{ }\normalsize\beginnumbering\briefempfaengerindex{Beer-Hofmann, Richard@\textsc{Beer-Hofmann, Richard}!zzzSchnitzler, Arthur@\emph{von Arthur Schnitzler}!1898-06-281@{28. 6. 1898}|(be}
\toendnotes[C]{\smallbreak\pagebreak[2]}
\correspDesc{Versand  durch Arthur Schnitzler am 28. 6. 1898 in Wien
\newline{}Erhalt  durch Richard Beer-Hofmann am 29. 6. 1898 in Steindorf am Ossiacher See}\toendnotes[C]{\smallbreak}
\Standort{YCGL, MSS 31.}
\physDesc{Brief, 1 Blatt, 4 Seiten, Kuvert, 1495 Zeichen
\newline{}Handschrift: Bleistift, deutsche Kurrent
\newline{}Versand: 1) Stempel: »\nobreak{}\oindex{IX., Alsergrund@\textbf{IX., Alsergrund}, \emph{Verwaltungsgebiet}|pwk}Wien 9/3 72, 28. 6. 98, 2–\textcolor{gray}{3N}\nobreak{}«.   2) Stempel: »\nobreak{}\oindex{Steindorf am Ossiacher See@\textbf{Steindorf am Ossiacher See}, \emph{Verwaltungsgebiet}|pwk}{\pb}{[}Stein{]}dorf am Ossiacher See, 29 6 98\nobreak{}«. }
\buchAbdrucke{\weitereDrucke{Arthur Schnitzler, Richard Beer-Hofmann: \emph{Briefwechsel 1891–1931}. Herausgegeben von Konstanze Fliedl. Wien, Zürich: \emph{Europaverlag} 1992, S. 120–121.} }\toendnotes[C]{\smallbreak}\pstart{}{\pb}\textsc{Kärnthen}\oindex{Kärnten@\textbf{Kärnten}, \emph{Land}|pw}.\pend{}\pstart{} Herrn \textsc{Dr. Richard Beer-Hofmann}\pend{}\pstart{}\textsc{Steindorf\oindex{Steindorf am Ossiacher See@\textbf{Steindorf am Ossiacher See}, \emph{Verwaltungsgebiet}|pw}}\pend{}\pstart{}\textsc{am Ossiacher\oindex{Ossiacher See@\textbf{Ossiacher See}, \emph{See}|pw}}ſee \pend{}{\bigskip}\vspace{1em}
\pstart
           \raggedleft{}{\pb}28. 6. 98.\pend
           \vspace{0.5em}
\pstart
           Mein lieber Richard, ich bin die letzten Tage wirklich{ }ſehr fleißig
               geweſen. Habe Vermächtnis\pwindex{Schnitzler, Arthur 15.\,5.\,1862 Wien – 21.\,10.\,1931 ebd.@\textsc{Schnitzler, Arthur} (15.\,5.\,1862 Wien – 21.\,10.\,1931 ebd.), \emph{Schriftsteller, Mediziner}!Vermächtnis. Schauspiel in drei Akten@\strich\emph{Das Vermächtnis. Schauspiel in drei Akten}|pw} insbeſondre 2. u
               3. Akt ziemlich gründlich hergeno{\geminationm}en und glaube, dſs ich
               mit dieſem Stück heute kaum viel weiter ko{\geminationm}en könnte als
               es iſt. Morgen gebe ich Schlenther\pwindex{Schlenther, Paul 20.\,8.\,1854 Chernyakhovsk – 30.\,4.\,1916 Berlin@\textsc{Schlenther, Paul} (20.\,8.\,1854 Chernyakhovsk – 30.\,4.\,1916 Berlin), \emph{Schriftsteller, Kritiker, Theaterleiter}|pw} die
               Aenderungen. Auch die Einakter\pwindex{Schnitzler, Arthur 15.\,5.\,1862 Wien – 21.\,10.\,1931 ebd.@\textsc{Schnitzler, Arthur} (15.\,5.\,1862 Wien – 21.\,10.\,1931 ebd.), \emph{Schriftsteller, Mediziner}!grüne Kakadu – Paracelsus – Die Gefährtin. Drei Einakter@\strich\emph{Der grüne Kakadu – Paracelsus – Die Gefährtin. Drei Einakter}|pwv}{ }ſind{ }ſo gut wie fertig – »und wie geht es
               Ihnen?«\pend
           
\pstart
           Ich ke{\geminationn} mich heuer mit dem So{\geminationm}er gar nicht ordentlich aus. Hoffentlich können wir uns
               im Auguſt, erſte Hälfte treffen – doch{ }ſowohl \introOben{}ich\introOben{} als Hugo\pwindex{Hofmannsthal, Hugo von 1.\,2.\,1874 Wien – 15.\,7.\,1929 Rodaun@\textsc{Hofmannsthal, Hugo von} (1.\,2.\,1874 Wien – 15.\,7.\,1929 Rodaun), \emph{Schriftsteller}|pw} wären{ }ſehr für was {\pb}andres als Salzburg\oindex{Salzburg@\textbf{Salzburg}, \emph{Verwaltungsgebiet}|pw}
                  eingeno{\geminationm}en \introOben{}(\introOben{}(wo ich im Lauf
               des Juli (20–27 herum) jedenfalls{ }ſein
               werde.)) – Schweiz\oindex{Schweiz@\textbf{Schweiz}|pw} – Luzern\oindex{Luzern@\textbf{Luzern}|pw} – mit Rad gemiſcht –\pend
           
\pstart
           Es ist nemlich auch{ }ſehr möglich, daſs meine Mama\pwindex{Schnitzler, Louise 8.\,7.\,1840 Kőszeg – 9.\,9.\,1911 Wien@\textsc{Schnitzler, Louise} (8.\,7.\,1840 Kőszeg – 9.\,9.\,1911 Wien)|pwv} nach Luzern\oindex{Luzern@\textbf{Luzern}|pw} geht,
               in welchem Fall ich mich beinah verpflichtet habe hinzugehn. \uline{Hier} bleib ich noch bis 12, 13, 14, 15 Juli. –\pend
           
\pstart
           – Heut hab ich von Mirjam\pwindex{Beer-Hofmann, Mirjam 4.\,9.\,1897 Wien – 24.\,12.\,1984 New York City@\textsc{Beer-Hofmann, Mirjam} (4.\,9.\,1897 Wien – 24.\,12.\,1984 New York City)|pw} geträumt, aber es war
               eigentlich ein kleines Kind, das ich behandelt habe, und ich {\pb}war rieſig{ }ſtolz, daſs eine Patientin von mir{ }ſo gut
               ausſieht – und ich hab{ }ſie Ihnen gezeigt, wir{ }ſind vor dem Haus, das an der Donau\oindex{Donau@\textbf{Donau}, \emph{Fluss}|pw} war, zuſa{\geminationm}en geſtanden,
               und Mirjam\pwindex{Beer-Hofmann, Mirjam 4.\,9.\,1897 Wien – 24.\,12.\,1984 New York City@\textsc{Beer-Hofmann, Mirjam} (4.\,9.\,1897 Wien – 24.\,12.\,1984 New York City)|pw} war am Fenſter, 2. Stock, in den
               Armen einer \label{K_L00809-1v}\edtext{\textsc{sage femme}}{\lemma{\textnormal{\emph{sage femme}}}\Cendnote{\textnormal{französisch: Hebamme}}}\label{K_L00809-1}\pwindex{Kirchrath, Leopoldine 28.\,11.\,1845 Wien – 14.\,2.\,1908 ebd.@\textsc{Kirchrath, Leopoldine} (28.\,11.\,1845 Wien – 14.\,2.\,1908 ebd.), \emph{Hebamme}|pwuv} (\introOben{}der\introOben{}{ }\label{K_L00809-2v}\edtext{mir bekannten}{\lemma{\textnormal{\emph{mir bekannten}}}\Cendnote{\textnormal{Gemeint dürfte Leopoldine Kirchrath\pwindex{Kirchrath, Leopoldine 28.\,11.\,1845 Wien – 14.\,2.\,1908 ebd.@\textsc{Kirchrath, Leopoldine} (28.\,11.\,1845 Wien – 14.\,2.\,1908 ebd.), \emph{Hebamme}|pwk} sein.}}}\label{K_L00809-2}) – und war{ }ſo dick und
               glücklich, daſs{ }ſie halb beim Fenſter draußen war. (Dieſer Traum iſt ein Geſchenk für
                  Paula\pwindex{Beer-Hofmann, Paula 25.\,2.\,1879 Wien – 30.\,10.\,1939 Zürich@\textsc{Beer-Hofmann, Paula} (25.\,2.\,1879 Wien – 30.\,10.\,1939 Zürich)|pw}. –)\pend
           
\pstart
           – Wir machen gelegentlich kleine Aus{\pb}flüge per Rad,
                  Rohrerhütte\oindex{Wien@\textbf{Wien}!XVII., Hernals@\textbf{XVII., Hernals}!Rohrerhütte@\textbf{Rohrerhütte}, \emph{Gastgewerbegebäude}|pw}, Weidlingau\oindex{Wien@\textbf{Wien}!XIV., Penzing@\textbf{XIV., Penzing}!Weidlingau@\textbf{Weidlingau}, \emph{Ehemaliger Ort}|pw}.\pend
           
\pstart
           Wie iſt Ihre Sti{\geminationm}ung? Verſuchen Sie zu radeln? Arbeiten
               Sie?\pend
           
\pstart
           Leben Sie wohl. Herzlicher Gruſs. Ihr \spacefill\mbox{Arth}\pend
           \selectlanguage{ngerman}\endnumbering\briefempfaengerindex{Beer-Hofmann, Richard@\textsc{Beer-Hofmann, Richard}!zzzSchnitzler, Arthur@\emph{von Arthur Schnitzler}!1898-06-281@{28. 6. 1898}|)be}\mylabel{L00809h}  \newcommand{\dateiname}{L00809}\newcommand{\titel}{Arthur Schnitzler an Richard Beer-Hofmann, 28. 6. 1898}\newcommand{\editorInnen}{Martin Anton Müller und Gerd-Hermann Susen}%% latex-leseansicht-abspann.tex
%% Abspann für die Leseansicht.
%% Der Schalter \ifkorrekturansicht ist bereits durch den Vorspann gesetzt.

%% latex-abspann.tex
%% Gemeinsamer Abspann für Korrekturansicht und Leseansicht.
%% Setzt den Schalter \ifkorrekturansicht voraus (gesetzt in den
%% einbindenden Dateien latex-korrekturansicht-abspann.tex bzw.
%% latex-leseansicht-abspann.tex).
%% ---------------------------------------------------------------

\normalsize

% Das esempio-Environment wird nur in der Leseansicht benötigt
\ifkorrekturansicht\else
\newenvironment{esempio}[3]%
{
    \vspace{1.5ex}
    \rlap{\underline{#1}}
    \par
    \setlength{\parindent}{0cm}
    \nopagebreak
    \leftskip=#2cm
    \rightskip=#3cm
}
{
    \par
}
\fi

\doendnotes{C}
\bigskip
\vfill

\clearpage

\footnotesize

\ifkorrekturansicht
  \lohead{\textsc{register}}
\fi

% theindex-Environment neu definieren ohne reledmac
\makeatletter
\renewenvironment{theindex}{%
  \ifkorrekturansicht
    \section*{\indexname}%
  \else
    \subsubsection*{Index der erwähnten Entitäten}%
  \fi
  \setlength{\parindent}{0pt}%
  \setlength{\parskip}{0pt plus 0.3pt}%
  \let\item\@idxitem
}{%
  \ifkorrekturansicht\clearpage\fi
}
\makeatother

\IfFileExists{\jobname-pw.ind}{\input{\jobname-pw.ind}}{}

% Quellenangabe nur in der Leseansicht
\ifkorrekturansicht\else
% Fallback-Definitionen, falls die .tex-Datei \titel etc. nicht gesetzt hat
\providecommand{\titel}{}
\providecommand{\editorInnen}{}
\providecommand{\dateiname}{\jobname}

\vspace{3cm}

\vfill

\footnotesize
\textsc{Quelle}: \titel. Herausgegeben von {\editorInnen}. In: \emph{Arthur Schnitzler: Briefwechsel mit Autorinnen und Autoren}.
 Digitale Edition, https://schnitzler-briefe.acdh.oeaw.ac.at/{\dateiname}.html (Stand \today)
\fi

\end{document}


