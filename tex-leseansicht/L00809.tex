%% latex-korrekturansicht-vorspann.tex
%% Vorspann für die Korrekturansicht.
%% Lädt die gemeinsame Datei latex-vorspann.tex mit gesetztem Schalter.

\newif\ifkorrekturansicht
\korrekturansichttrue

\input{../tex-inputs/latex-vorspann}


\section[Arthur Schnitzler an Richard Beer-Hofmann, 28. 6. 1898]{L00809 Arthur Schnitzler an Richard Beer-Hofmann, 28. 6. 1898}
\nopagebreak\mylabel{L00809v}
\rehead{ }\normalsize\beginnumbering\briefempfaengerindex{Beer-Hofmann, Richard@\textsc{Beer-Hofmann, Richard}!zzzSchnitzler, Arthur@\emph{von Arthur Schnitzler}!1898-06-281@{28. 6. 1898}|(be}
\toendnotes[C]{\smallbreak\pagebreak[2]}\Standort{YCGL, MSS 31.}
\physDesc{Brief, 1 Blatt, 4 Seiten, Umschlag, 1495 Zeichen
\newline{}Handschrift: Bleistift, deutsche Kurrent
\newline{}Versand: 1) Stempel: »\nobreak{}\oindex{IX., Alsergrund@\textbf{IX., Alsergrund}, \emph{A.ADM3}|pwk}Wien 9/3 72, 28. 6. 98, 2–\textcolor{gray}{3N}\nobreak{}«.   2) Stempel: »\nobreak{}\oindex{Steindorf am Ossiacher See@\textbf{Steindorf am Ossiacher See}, \emph{A.ADM3}|pwk}{\pb}{[}Stein{]}dorf am Ossiacher See, 29 6 98\nobreak{}«. }
\buchAbdrucke{\weitereDrucke{Arthur Schnitzler, Richard Beer-Hofmann: \emph{Briefwechsel 1891–1931}. Wien, Zürich: \emph{Europaverlag} 1992, S. 120–121.} }\toendnotes[C]{\smallbreak}\pstart{}{\pb}\textsc{Kärnthen}\oindex{Kaernten@\textbf{Kärnten}, \emph{A.ADM1}|pw}.\pend{}\pstart{} Herrn \textsc{Dr. Richard Beer-Hofmann}\pend{}\pstart{}\textsc{Steindorf\oindex{Steindorf am Ossiacher See@\textbf{Steindorf am Ossiacher See}, \emph{A.ADM3}|pw}}\pend{}\pstart{}\textsc{am Ossiacher\oindex{Ossiacher See@\textbf{Ossiacher See}, \emph{See (N.SEE)}|pw}}ſee \pend{}{\bigskip}\vspace{1em}
\pstart
           \raggedleft{}{\pb}28. 6. 98.\pend
           \vspace{0.5em}
\pstart
           Mein lieber Richard, ich bin die letzten Tage wirklich ſehr fleißig
               geweſen. Habe Vermächtnis\pwindex{Vermaechtnis. Schauspiel in drei Akten@\emph{Das Vermächtnis. Schauspiel in drei Akten}|pw} insbeſondre 2. u
               3. Akt ziemlich gründlich hergeno{\geminationm}en und glaube, dſs ich
               mit dieſem Stück heute kaum viel weiter ko{\geminationm}en könnte als
               es iſt. Morgen gebe ich Schlenther\pwindex{Schlenther, Paul 20.08.1854 – 30.04.1916@\textsc{Schlenther, Paul} (20.08.1854 – 30.04.1916), \emph{Schriftsteller/Schriftstellerin, Kritiker/Kritikerin, Theaterleiter/Theaterleiterin}|pw} die
               Aenderungen. Auch die Einakter\pwindex{gruene Kakadu – Paracelsus – Die Gefaehrtin. Drei Einakter@\emph{Der grüne Kakadu – Paracelsus – Die Gefährtin. Drei Einakter}|pwv}{ }ſind ſo gut wie fertig – »und wie geht es
               Ihnen?«\pend
           
\pstart
           Ich ke{\geminationn} mich heuer mit dem So{\geminationm}er gar nicht ordentlich aus. Hoffentlich können wir uns
               im Auguſt, erſte Hälfte treffen – doch ſowohl \introOben{}ich\introOben{} als Hugo\pwindex{Hofmannsthal, Hugo von 1874-02-01 – 1929-07-15@\textsc{Hofmannsthal, Hugo von} (1874-02-01 – 1929-07-15), \emph{Schriftsteller/Schriftstellerin}|pw} wären ſehr für was {\pb}andres als Salzburg\oindex{Salzburg@\textbf{Salzburg}, \emph{A.ADM2}|pw}
                  eingeno{\geminationm}en \introOben{}(\introOben{}(wo ich im Lauf
               des Juli (20–27 herum) jedenfalls ſein
               werde.)) – Schweiz\oindex{Schweiz@\textbf{Schweiz}, \emph{A.PCLI}|pw} – Luzern\oindex{Luzern@\textbf{Luzern}, \emph{P.PPLA}|pw} – mit Rad gemiſcht –\pend
           
\pstart
           Es ist nemlich auch ſehr möglich, daſs meine Mama\pwindex{Schnitzler, Louise 1840-07-08 – 1911-09-09@\textsc{Schnitzler, Louise} (1840-07-08 – 1911-09-09)|pwv} nach Luzern\oindex{Luzern@\textbf{Luzern}, \emph{P.PPLA}|pw} geht,
               in welchem Fall ich mich beinah verpflichtet habe hinzugehn. \uline{Hier} bleib ich noch bis 12, 13, 14,
               15 Juli. –\pend
           
\pstart
           – Heut hab ich von Mirjam\pwindex{Beer-Hofmann, Mirjam 04.09.1897 – 24.12.1984@\textsc{Beer-Hofmann, Mirjam} (04.09.1897 – 24.12.1984)|pw} geträumt, aber es war
               eigentlich ein kleines Kind, das ich behandelt habe, und ich {\pb}war rieſig ſtolz, daſs eine Patientin von mir ſo gut
               ausſieht – und ich hab ſie Ihnen gezeigt, wir ſind vor dem Haus, das an der Donau\oindex{Donau@\textbf{Donau}, \emph{Fluss (N.FLS)}|pw} war, zuſa{\geminationm}en geſtanden,
               und Mirjam\pwindex{Beer-Hofmann, Mirjam 04.09.1897 – 24.12.1984@\textsc{Beer-Hofmann, Mirjam} (04.09.1897 – 24.12.1984)|pw} war am Fenſter, 2. Stock, in den
               Armen einer \label{K_L00809-1v}\edtext{\textsc{sage femme}}{\lemma{\textnormal{\emph{sage femme}}}\Cendnote{\textnormal{französisch: Hebamme}}}\label{K_L00809-1}\pwindex{Kirchrath, Leopoldine 28.11.1845 – 14.02.1908@\textsc{Kirchrath, Leopoldine} (28.11.1845 – 14.02.1908), \emph{männliche Hebamme/Hebamme}|pwuv} (\introOben{}der\introOben{}{ }\label{K_L00809-2v}\edtext{mir bekannten}{\lemma{\textnormal{\emph{mir bekannten}}}\Cendnote{\textnormal{Gemeint dürfte Leopoldine Kirchrath\pwindex{Kirchrath, Leopoldine 28.11.1845 – 14.02.1908@\textsc{Kirchrath, Leopoldine} (28.11.1845 – 14.02.1908), \emph{männliche Hebamme/Hebamme}|pwk} sein.}}}\label{K_L00809-2}) – und war ſo dick und
               glücklich, daſs ſie halb beim Fenſter draußen war. (Dieſer Traum iſt ein Geſchenk für
                  Paula\pwindex{Beer-Hofmann, Paula 25.02.1879 – 30.10.1939@\textsc{Beer-Hofmann, Paula} (25.02.1879 – 30.10.1939)|pw}. –)\pend
           
\pstart
           – Wir machen gelegentlich kleine Aus{\pb}flüge per Rad,
                  Rohrerhütte\oindex{Rohrerhuette@\textbf{Rohrerhütte}, \emph{Gastgewerbegebäude (K.GGW)}|pw}, Weidlingau\oindex{Weidlingau@\textbf{Weidlingau}, \emph{P.PPLX}|pw}.\pend
           
\pstart
           Wie iſt Ihre Sti{\geminationm}ung? Verſuchen Sie zu radeln? Arbeiten
               Sie?\pend
           
\pstart
           Leben Sie wohl. Herzlicher Gruſs. Ihr \spacefill\mbox{Arth}\pend
           \selectlanguage{ngerman}\endnumbering\briefempfaengerindex{Beer-Hofmann, Richard@\textsc{Beer-Hofmann, Richard}!zzzSchnitzler, Arthur@\emph{von Arthur Schnitzler}!1898-06-281@{28. 6. 1898}|)be}\mylabel{L00809h}  \normalsize

\doendnotes{C}
\bigskip
\vfill

\clearpage

\footnotesize

\lohead{\textsc{register}}

% Definiere theindex-Environment komplett neu ohne reledmac
\makeatletter
\renewenvironment{theindex}{%
  \section*{\indexname}%
  \setlength{\parindent}{0pt}%
  \setlength{\parskip}{0pt plus 0.3pt}%
  \let\item\@idxitem
}{%
  \clearpage
}
\makeatother

\IfFileExists{\jobname-pw.ind}{\input{\jobname-pw.ind}}{}

\end{document}

      