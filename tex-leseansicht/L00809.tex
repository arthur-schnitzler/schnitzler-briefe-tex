\input{../tex-inputs/latex-pdf-vorspann}
\begin{center}
            \textcolor{red}{ENTWURF. ENTZIFFERUNG NOCH NICHT KORREKTURGELESEN}
                      \end{center}
            
               \section[Arthur Schnitzler an Richard Beer-Hofmann, 28. 6. 1898]{ Arthur Schnitzler an Richard Beer-Hofmann, 28. 6. 1898}\nopagebreak\mylabel{v}\rehead{ }\begin{ledgroupsized}[t]{13cm}\normalsize\beginnumbering\briefempfaengerindex{Beer-Hofmann, Richard@\textsc{Beer-Hofmann, Richard}!zzzSchnitzler, Arthur@\emph{von Arthur Schnitzler}!1898-06-281@{28. 6. 1898}|(be} \toendnotes[C]{\smallbreak\pagebreak[2]} \Standort{YCGL, MSS 31.}
\physDesc{Brief, 1 Blatt, 4 Seiten, Umschlag
\newline{}Handschrift: Bleistift, deutsche Kurrent\newline{}Versand: 1) Stempel: »\nobreak{}\oindex{IX., Alsergrund@\textbf{IX., Alsergrund}|pwk}Wien 9/3 72, 28. 6. 98, 2–\textcolor{gray}{3N}\nobreak{}«.  2) Stempel: »\nobreak{}\oindex{Steindorf am Ossiacher See@\textbf{Steindorf am Ossiacher See}|pwk}{\pb}{[}Stein{]}dorf am Ossiacher See, 29 6 98\nobreak{}«. }\buchAbdrucke{\weitereDrucke{Arthur Schnitzler, Richard Beer-Hofmann: \emph{Briefwechsel 1891–1931}. Hg. Konstanze Fliedl. Wien, Zürich: \emph{Europaverlag} 1992, S. 120–121.} }\toendnotes[C]{\smallbreak}\pstart{}{\pb}\textsc{Kärnthen}\oindex{Kaernten@\textbf{Kärnten}|pw}.\pend{}\pstart{}
                  Herrn \textsc{Dr. Richard
                     Beer-Hofmann}\pend{}\pstart{}\textsc{Steindorf\oindex{Steindorf am Ossiacher See@\textbf{Steindorf am Ossiacher See}|pw}}\pend{}\pstart{}\textsc{am Ossiacher\oindex{Ossiacher See@\textbf{Ossiacher See}|pw}}ſee \pend{}{\bigskip}\pstart
           \raggedleft{}{\pb}28. 6. 98.\pend
           \pstart
           Mein lieber Richard, ich bin die letzten Tage wirklich ſehr fleißig
               geweſen. Habe Vermächtnis\pwindex{Schnitzler, Arthur 15.05.1862 – 21.10.1931@\textsc{Schnitzler, Arthur} (15.05.1862 – 21.10.1931), \emph{Schriftsteller, Mediziner}!Vermaechtnis. Schauspiel in drei Akten1898@\strich\emph{Das Vermächtnis. Schauspiel in drei Akten} {[}1898{]}|pw} insbeſondre 2. u 3. Akt
               ziemlich gründlich hergeno{\geminationm}en und glaube, dſs ich mit
               dieſem Stück heute kaum viel weiter ko{\geminationm}en könnte als es
               iſt. Morgen gebe ich Schlenther\pwindex{Schlenther, Paul 20.08.1854 – 30.04.1916@\textsc{Schlenther, Paul} (20.08.1854 – 30.04.1916), \emph{Schriftsteller, Kritiker, Theaterleiter}|pw} die Aenderungen.
               Auch die Einakter\pwindex{Schnitzler, Arthur 15.05.1862 – 21.10.1931@\textsc{Schnitzler, Arthur} (15.05.1862 – 21.10.1931), \emph{Schriftsteller, Mediziner}!gruene Kakadu – Paracelsus – Die Gefaehrtin. Drei Einakter1.3.1899 – 1.3.1899@\strich\emph{Der grüne Kakadu – Paracelsus – Die Gefährtin. Drei Einakter} {[}1.3.1899 – 1.3.1899{]}|pwv}{ }ſind ſo gut wie fertig – »und wie geht es
               Ihnen?«\pend
           \pstart
           Ich ke{\geminationn} mich heuer mit dem So{\geminationm}er gar nicht ordentlich aus. Hoffentlich können wir uns
               im Auguſt, erſte Hälfte treffen – doch ſowohl \introOben{}ich\introOben{} als Hugo\pwindex{Hofmannsthal, Hugo von 01.02.1874 – 15.07.1929@\textsc{Hofmannsthal, Hugo von} (01.02.1874 – 15.07.1929), \emph{Schriftsteller}|pw} wären ſehr für was {\pb}andres als Salzburg\oindex{Salzburg@\textbf{Salzburg}|pw}
                  eingeno{\geminationm}en \introOben{}(\introOben{}(wo ich im Lauf
               des Juli (20–27 herum) jedenfalls ſein
               werde.)) – Schweiz\oindex{Schweiz@\textbf{Schweiz}|pw} – Luzern\oindex{Luzern@\textbf{Luzern}|pw} – mit Rad gemiſcht –\pend
           \pstart
           Es ist nemlich auch ſehr möglich, daſs meine Mama\pwindex{Schnitzler, Louise 08.07.1840 – 09.09.1911@\textsc{Schnitzler, Louise} (08.07.1840 – 09.09.1911)|pwv} nach Luzern\oindex{Luzern@\textbf{Luzern}|pw} geht, in
               welchem Fall ich mich beinah verpflichtet habe hinzugehn. \uline{Hier} bleib ich noch bis 12, 13, 14, 15 Juli. –\pend
           \pstart
           – Heut hab ich von Mirjam\pwindex{Beer-Hofmann, Mirjam 04.09.1897 – 24.12.1984@\textsc{Beer-Hofmann, Mirjam} (04.09.1897 – 24.12.1984)|pw} geträumt, aber es war
               eigentlich ein kleines Kind, das ich behandelt habe, und ich {\pb}war rieſig ſtolz, daſs eine Patientin von mir ſo gut
               ausſieht – und ich hab ſie Ihnen gezeigt, wir ſind vor dem Haus, das an der Donau war, zuſa{\geminationm}en geſtanden,
               und Mirjam\pwindex{Beer-Hofmann, Mirjam 04.09.1897 – 24.12.1984@\textsc{Beer-Hofmann, Mirjam} (04.09.1897 – 24.12.1984)|pw} war am Fenſter, 2. Stock, in den Armen
               einer \label{K_L00809_1v}\edtext{\textsc{sage femme}}{\lemma{\textnormal{\emph{sage femme}}}\Cendnote{\textnormal{französisch: Hebamme}}}\label{K_L00809_1h}\pwindex{Kirchrath, Leopoldine 28.11.1845 – 14.02.1908@\textsc{Kirchrath, Leopoldine} (28.11.1845 – 14.02.1908), \emph{Hebamme}|pwuv} (\introOben{}der\introOben{}{ }\label{K_L00809_2v}\edtext{mir
                  bekannten}{\lemma{\textnormal{\emph{mir
                  bekannten}}}\Cendnote{\textnormal{Gemeint dürfte Leopoldine Kirchrath\pwindex{Kirchrath, Leopoldine 28.11.1845 – 14.02.1908@\textsc{Kirchrath, Leopoldine} (28.11.1845 – 14.02.1908), \emph{Hebamme}|pwk}
                  sein.}}}\label{K_L00809_2h}) – und war ſo dick und glücklich, daſs ſie halb beim Fenſter draußen
               war. (Dieſer Traum iſt ein Geſchenk für Paula\pwindex{Beer-Hofmann, Paula 25.02.1879 – 30.10.1939@\textsc{Beer-Hofmann, Paula} (25.02.1879 – 30.10.1939)|pw}. –)\pend
           \pstart
           – Wir machen gelegentlich kleine Aus{\pb}flüge per Rad,
                  Rohrerhütte\oindex{Rohrerhuette@\textbf{Rohrerhütte}|pw}, Weidlingau\oindex{Weidlingau@\textbf{Weidlingau}|pw}.\pend
           \pstart
           Wie iſt Ihre Sti{\geminationm}ung? Verſuchen Sie zu radeln? Arbeiten
               Sie?\pend
           \pstart
           Leben Sie wohl. Herzlicher Gruſs. Ihr \spacefill\mbox{Arth}\pend
           \endnumbering\briefempfaengerindex{Beer-Hofmann, Richard@\textsc{Beer-Hofmann, Richard}!zzzSchnitzler, Arthur@\emph{von Arthur Schnitzler}!1898-06-281@{28. 6. 1898}|)be}\mylabel{h}\end{ledgroupsized}  \newcommand{\dateiname}{L00809}\newcommand{\titel}{Arthur Schnitzler an Richard Beer-Hofmann, 28. 6. 1898}\newcommand{\editorInnen}{Martin Anton Müller und Gerd-Hermann Susen}\input{../tex-inputs/latex-pdf-abspann}
      