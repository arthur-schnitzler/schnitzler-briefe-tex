%% latex-leseansicht-vorspann.tex
%% Vorspann für die Leseansicht.
%% Lädt die gemeinsame Datei latex-vorspann.tex mit nicht gesetztem Schalter.

\newif\ifkorrekturansicht
\korrekturansichtfalse

\input{../tex-inputs/latex-vorspann}


         
         \renewcommand{\erwaehntePersonen}{Personen: Friedrich Mitterwurzer}
         \renewcommand{\erwaehnteOrte}{Orte: Agrigento, Concordia-Tempel, Sternwartestraße, Wien, Österreich}
         \renewcommand{\erwaehnteWerke}{}
               \section[Hugo Hofmannsthal an Arthur Schnitzler, 18. 2. 1927]{ Hugo Hofmannsthal an Arthur Schnitzler, 18. 2. 1927}\nopagebreak\mylabel{v}\rehead{ }\begin{ledgroupsized}[t]{13cm}\normalsize\beginnumbering \toendnotes[C]{\smallbreak\pagebreak[2]} \Standort{CUL, Schnitzler, B 43.}
\physDesc{Bildpostkarte, 269 Zeichen
\newline{}Handschrift: schwarze Tinte, lateinische Kurrent
\newline{}Versand: Stempel: »\nobreak{}\oindex{Agrigento@\textbf{Agrigento}|pwk}Girgen{[}ti{]}, 1\textcolor{gray}{8}. 2. 2{[}7{]}\nobreak{}«.  
\newline{}Ordnung: 1) mit Bleistift von unbekannter Hand nummeriert: »\strikeout{383}\strikeout{386}«  2) mit Bleistift von unbekannter Hand nummeriert:
                                    »391«}\buchAbdrucke{\weitereDrucke{Hugo von Hofmannsthal, Arthur Schnitzler: \emph{Briefwechsel}. Hg. Therese Nickl und Heinrich Schnitzler. Frankfurt am Main: \emph{S. Fischer} 1964, S. 306.} }\toendnotes[C]{\smallbreak}\pstart{}{\pb}Herrn D\textsuperscript{r} Arthur Schnitzler\pend{}\pstart{}Wien\oindex{Wien@\textbf{Wien}|pw}\pend{}\pstart{}XVIII. Sternwartestrasse 71\oindex{Sternwartestrasse@\textbf{Sternwartestraße}|pw}. \pend{}\pstart{}Austria\oindex{Oesterreich@\textbf{Österreich}|pw}\pend{}{\bigskip}\pstart
           \noindent{}\centering{}{\pb}\textcolor{gray}{\textbf{{[}Agrigent\oindex{Agrigento@\textbf{Agrigento}|pw}, Concordia-Tempel\oindex{Concordia-Tempel@\textbf{Concordia-Tempel}|pw}{]}}}\pend
           \pstart
           \raggedleft{}{\pb}Girgenti\oindex{Agrigento@\textbf{Agrigento}|pw}{ }18 II 27.\pend
           \pstart
           Viele gute Gedanken, lieber Arthur. Der \label{K_L02481_1v}\edtext{Todestag Mitterwurzers\pwindex{Mitterwurzer, Friedrich 16.10.1844 – 13.02.1897@\textsc{Mitterwurzer, Friedrich} (16.10.1844 – 13.02.1897), \emph{Schauspieler}|pw}}{\lemma{\textnormal{\emph{Todestag Mitterwurzers}}}\Cendnote{\textnormal{Er starb am
                  13. 2. 1897.}}}\label{K_L02481_1h} eri{\geminationn}erte mich so
               lebhaft an unsere Freundschaft.\hspace*{1.5em}Ich war gar nicht
               wohl, hoffe erholt zurückzuko{\geminationm}en u. Sie da{\geminationn} zu finden!\pend
           \pstart
           Ihr{\\[\baselineskip]}\spacefill\mbox{Hugo}\pend
           \leftskip=0em{}
         
         \endnumbering\mylabel{h}\end{ledgroupsized}  \newcommand{\dateiname}{L02481}\newcommand{\titel}{Hugo Hofmannsthal an Arthur Schnitzler, 18. 2. 1927}\newcommand{\editorInnen}{Martin Anton Müller und Gerd-Hermann Susen}%% latex-leseansicht-abspann.tex
%% Abspann für die Leseansicht.
%% Der Schalter \ifkorrekturansicht ist bereits durch den Vorspann gesetzt.

%% latex-abspann.tex
%% Gemeinsamer Abspann für Korrekturansicht und Leseansicht.
%% Setzt den Schalter \ifkorrekturansicht voraus (gesetzt in den
%% einbindenden Dateien latex-korrekturansicht-abspann.tex bzw.
%% latex-leseansicht-abspann.tex).
%% ---------------------------------------------------------------

\normalsize

% Das esempio-Environment wird nur in der Leseansicht benötigt
\ifkorrekturansicht\else
\newenvironment{esempio}[3]%
{
    \vspace{1.5ex}
    \rlap{\underline{#1}}
    \par
    \setlength{\parindent}{0cm}
    \nopagebreak
    \leftskip=#2cm
    \rightskip=#3cm
}
{
    \par
}
\fi

\doendnotes{C}
\bigskip
\vfill

\clearpage

\footnotesize

\ifkorrekturansicht
  \lohead{\textsc{register}}
\fi

% theindex-Environment neu definieren ohne reledmac
\makeatletter
\renewenvironment{theindex}{%
  \ifkorrekturansicht
    \section*{\indexname}%
  \else
    \subsubsection*{Index der erwähnten Entitäten}%
  \fi
  \setlength{\parindent}{0pt}%
  \setlength{\parskip}{0pt plus 0.3pt}%
  \let\item\@idxitem
}{%
  \ifkorrekturansicht\clearpage\fi
}
\makeatother

\IfFileExists{\jobname-pw.ind}{\input{\jobname-pw.ind}}{}

% Quellenangabe nur in der Leseansicht
\ifkorrekturansicht\else
% Fallback-Definitionen, falls die .tex-Datei \titel etc. nicht gesetzt hat
\providecommand{\titel}{}
\providecommand{\editorInnen}{}
\providecommand{\dateiname}{\jobname}

\vspace{3cm}

\vfill

\footnotesize
\textsc{Quelle}: \titel. Herausgegeben von {\editorInnen}. In: \emph{Arthur Schnitzler: Briefwechsel mit Autorinnen und Autoren}.
 Digitale Edition, https://schnitzler-briefe.acdh.oeaw.ac.at/{\dateiname}.html (Stand \today)
\fi

\end{document}


      