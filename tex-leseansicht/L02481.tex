%% latex-korrekturansicht-vorspann.tex
%% Vorspann für die Korrekturansicht.
%% Lädt die gemeinsame Datei latex-vorspann.tex mit gesetztem Schalter.

\newif\ifkorrekturansicht
\korrekturansichttrue

\input{../tex-inputs/latex-vorspann}


\section[Hugo Hofmannsthal an Arthur Schnitzler, 18. 2. 1927]{L02481 Hugo Hofmannsthal an Arthur Schnitzler, 18. 2. 1927}
\nopagebreak\mylabel{L02481v}
\rehead{ }\normalsize\beginnumbering\briefempfaengerindex{Schnitzler, Arthur@\textsc{Schnitzler, Arthur}!zzzHofmannsthal, Hugo von@\emph{von Hugo von Hofmannsthal}!1927-02-181@{18. 2. 1927}|(be}
\toendnotes[C]{\smallbreak\pagebreak[2]}\Standort{CUL, Schnitzler, B 43.}
\physDesc{Bildpostkarte, 269 Zeichen
\newline{}Handschrift: schwarze Tinte, lateinische Kurrent
\newline{}Versand: Stempel: »\nobreak{}\oindex{Agrigento@\textbf{Agrigento}, \emph{L.LCTY}|pwk}Girgen{[}ti{]}, 1\textcolor{gray}{8}. 2. 2{[}7{]}\nobreak{}«.  
\newline{}Ordnung: 1) mit Bleistift von unbekannter Hand nummeriert: »\strikeout{383}\strikeout{386}«  2) mit Bleistift von unbekannter Hand nummeriert:
                                    »391«}
\buchAbdrucke{\weitereDrucke{Hugo von Hofmannsthal, Arthur Schnitzler: \emph{Briefwechsel}. Frankfurt am Main: \emph{S. Fischer} 1964, S. 306.} }\toendnotes[C]{\smallbreak}\pstart{}{\pb}Herrn D\textsuperscript{r} Arthur Schnitzler\pend{}\pstart{}Wien\oindex{Wien@\textbf{Wien}, \emph{A.ADM2}|pw}\pend{}\pstart{}XVIII. Sternwartestrasse 71\oindex{Sternwartestrasse 71@\textbf{Sternwartestraße 71}, \emph{Wohngebäude (K.WHS)}|pw}. \pend{}\pstart{}Austria\oindex{Oesterreich@\textbf{Österreich}, \emph{A.PCLI}|pw}\pend{}{\bigskip}
\pstart
           \noindent{}\centering{}{\pb}\textcolor{gray}{\textbf{{[}Agrigent\oindex{Agrigento@\textbf{Agrigento}, \emph{L.LCTY}|pw}, Concordia-Tempel\oindex{Concordia-Tempel@\textbf{Concordia-Tempel}, \emph{Monument (K.MON)}|pw}{]}}}\pend
           \vspace{1em}
\pstart
           \raggedleft{}{\pb}Girgenti\oindex{Agrigento@\textbf{Agrigento}, \emph{L.LCTY}|pw}{ }18 II 27.\pend
           \vspace{0.5em}
\pstart
           Viele gute Gedanken, lieber Arthur. Der \label{K_L02481-1v}\edtext{Todestag Mitterwurzers\pwindex{Mitterwurzer, Friedrich 16.10.1844 – 13.02.1897@\textsc{Mitterwurzer, Friedrich} (16.10.1844 – 13.02.1897), \emph{Schauspieler/Schauspielerin}|pw}}{\lemma{\textnormal{\emph{Todestag Mitterwurzers}}}\Cendnote{\textnormal{Er starb am
                  13. 2. 1897.}}}\label{K_L02481-1} eri{\geminationn}erte mich so
               lebhaft an unsere Freundschaft.\hspace*{1.5em}Ich war gar nicht
               wohl, hoffe erholt zurückzuko{\geminationm}en u. Sie da{\geminationn} zu finden!\pend
           
\pstart
           Ihr{\\[\baselineskip]}\spacefill\mbox{Hugo}\pend
           \leftskip=0em{}\selectlanguage{ngerman}\endnumbering\briefempfaengerindex{Schnitzler, Arthur@\textsc{Schnitzler, Arthur}!zzzHofmannsthal, Hugo von@\emph{von Hugo von Hofmannsthal}!1927-02-181@{18. 2. 1927}|)be}\mylabel{L02481h}  \normalsize

\doendnotes{C}
\bigskip
\vfill

\clearpage

\footnotesize

\lohead{\textsc{register}}

% Definiere theindex-Environment komplett neu ohne reledmac
\makeatletter
\renewenvironment{theindex}{%
  \section*{\indexname}%
  \setlength{\parindent}{0pt}%
  \setlength{\parskip}{0pt plus 0.3pt}%
  \let\item\@idxitem
}{%
  \clearpage
}
\makeatother

\IfFileExists{\jobname-pw.ind}{\input{\jobname-pw.ind}}{}

\end{document}

      