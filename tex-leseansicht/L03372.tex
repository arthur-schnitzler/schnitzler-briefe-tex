%% latex-korrekturansicht-vorspann.tex
%% Vorspann für die Korrekturansicht.
%% Lädt die gemeinsame Datei latex-vorspann.tex mit gesetztem Schalter.

\newif\ifkorrekturansicht
\korrekturansichttrue

\input{../tex-inputs/latex-vorspann}


\section[ Paul Goldmann an Arthur Schnitzler, 13. 5. {[}1903{]}]{L03372 Paul Goldmann an Arthur Schnitzler, 13. 5. {[}1903{]}}
\nopagebreak\mylabel{L03372v}
\rehead{ }\normalsize\beginnumbering\briefempfaengerindex{Schnitzler, Arthur@\textsc{Schnitzler, Arthur}!zzzGoldmann, Paul@\emph{von Paul Goldmann}!1903-05-131@{13. 5. {[}1903{]}}|(be}
\toendnotes[C]{\smallbreak\pagebreak[2]}\Standort{DLA, A:Schnitzler, HS.NZ85.1.3173.}
\physDesc{Brief, 1 Blatt, 4 Seiten, 1047 Zeichen
\newline{}Handschrift: blaue Tinte, deutsche Kurrent
\newline{}Schnitzler: 1) mit Bleistift das Jahr »903« vermerkt  2) mit rotem Buntstift eine Unterstreichung}\toendnotes[C]{\smallbreak}
\pstart
           \raggedleft{}{\pb}\textcolor{gray}{\textbf{DESSAUERSTRASSE 19\oindex{Dessauer Strasse@\textbf{Dessauer Straße}, \emph{Straße (K.STR)}|pw}}}\pend
           
\pstart
           Berlin\oindex{Berlin@\textbf{Berlin}, \emph{P.PPLC}|pw}, 13. Mai.\pend
           
\pstart\center{}Mein lieber Freund,\pend\vspace{0.5em}
\pstart
           Ich ſende heut an \textsc{Olga\pwindex{Schnitzler, Olga 17.01.1882 – 13.01.1970@\textsc{Schnitzler, Olga} (17.01.1882 – 13.01.1970), \emph{Schauspieler/Schauspielerin, Sänger/Sängerin}|pw}} die verſprochene \label{K_L03372-1v}\edtext{Tiſchglocke}{\lemma{\textnormal{\emph{Tiſchglocke}}}\Cendnote{\textnormal{eine Glocke, die
                  geklingelt wird, wenn das Essen angerichtet ist}}}\label{K_L03372-1} ab. Ich konnte ſie nicht
               früher ſenden, weil ich ſeit meiner \label{K_L03372-2v}\edtext{Rückkehr aus Wien\oindex{Wien@\textbf{Wien}, \emph{A.ADM2}|pw}}{\lemma{\textnormal{\emph{Rückkehr aus Wien}}}\Cendnote{\textnormal{Goldmann\pwindex{Goldmann, Paul 31.01.1865 – 25.09.1935@\textsc{Goldmann, Paul} (31.01.1865 – 25.09.1935), \emph{Schriftsteller/Schriftstellerin, Journalist/Journalistin}|pwk} war nachweislich zwischen 14. 4. 1903 und 21. 4. 1903 in Wien\oindex{Wien@\textbf{Wien}, \emph{A.ADM2}|pwk}. In dieser Zeit traf er mehrmals mit Schnitzler zusammen, von dem er ursprünglich gedacht hatte, dass er
                  auf Reisen sei (vgl. A. S.: \emph{Tagebuch}, 14. 4. 1903).}}}\label{K_L03372-2} ohne Diener war, der \strikeout{\textcolor{gray}{×}\textcolor{gray}{it}} mir das Paket hätte machen und expediren können. Entſchuldige mich bei \textsc{Olga\pwindex{Schnitzler, Olga 17.01.1882 – 13.01.1970@\textsc{Schnitzler, Olga} (17.01.1882 – 13.01.1970), \emph{Schauspieler/Schauspielerin, Sänger/Sängerin}|pw}} wegen der Verſpätung und grüße ſie herzlichſt.\pend
           
\pstart
           {\pb}Ich habe \strikeout{die} in
               letzter Zeit \strikeout{\textcolor{gray}{vie}}{ }\textsc{Oscar Wilde\pwindex{Wilde, Oscar 16.10.1854 – 30.11.1900@\textsc{Wilde, Oscar} (16.10.1854 – 30.11.1900), \emph{Schriftsteller/Schriftstellerin}|pw}} geleſen und in ihm einen der glänzendſten modernen Geiſter gefunden. Lies’
                  \label{K_L03372-3v}\edtext{»Fingerzeige\pwindex{Fingerzeige@\emph{Fingerzeige}|pw}«, in der Überſetzung von \textsc{Greve\pwindex{Greve, Felix Paul 1879-02-14 – 1948-08-19@\textsc{Greve, Felix Paul} (1879-02-14 – 1948-08-19), \emph{Schriftsteller/Schriftstellerin, Übersetzer/Übersetzerin}|pw}}}{\lemma{\textnormal{\emph{»Fingerzeige«, … Greve}}}\Cendnote{\textnormal{Oscar Wilde\pwindex{Wilde, Oscar 16.10.1854 – 30.11.1900@\textsc{Wilde, Oscar} (16.10.1854 – 30.11.1900), \emph{Schriftsteller/Schriftstellerin}|pwk}: \emph{Fingerzeige}\pwindex{Fingerzeige@\emph{Fingerzeige}|pwk}. Übersetzt von Felix Paul Greve\pwindex{Greve, Felix Paul 1879-02-14 – 1948-08-19@\textsc{Greve, Felix Paul} (1879-02-14 – 1948-08-19), \emph{Schriftsteller/Schriftstellerin, Übersetzer/Übersetzerin}|pwk}. Minden\oindex{Minden@\textbf{Minden}, \emph{P.PPLA3}|pwk}: \emph{J. C. C. Bruns’ Verlag}\orgindex{J. C. C. Bruns@J. C. C. Bruns|pwk}
                        [1903?]. Eine Lektüre durch Schnitzler ist nicht bekannt.}}}\label{K_L03372-3} (Verlag von \textsc{Bruns}\orgindex{J. C. C. Bruns@J. C. C. Bruns|pw} in \textsc{Minden\oindex{Minden@\textbf{Minden}, \emph{P.PPLA3}|pw}}). Die beiden Dialoge\pwindex{Fingerzeige@\emph{Fingerzeige}|pwv}
               über die Kritik als \uline{ſchaffende} Kunst geben wieder,
               was ich im Innerſten über die Kritik denke, – \substVorne{}\textsuperscript{i}\substDazwischen{}m\substHinten{}it den { }{\pb}Worten eines großen Dichters und ſprühenden Geiſtes
               allerdings, die ich nie im ſtande geweſen wäre zu finden.\pend
           
\pstart
           Meine \label{K_L03372-4v}\edtext{\textsc{Musset}-Überſetzung\pwindex{Man soll nichts verschwoeren. Komoedie in 3 Akten@\emph{Man soll nichts verschwören. Komödie in 3 Akten}|pwv}}{\lemma{\textnormal{\emph{Musset-Überſetzung}}}\Cendnote{\textnormal{\emph{Man soll nichts verschwören}\pwindex{Man soll nichts verschwoeren. Komoedie in 3 Akten@\emph{Man soll nichts verschwören. Komödie in 3 Akten}|pwk} (\emph{Il ne faut jurer de rien}\pwindex{ne faut jurer de rien@\emph{Il ne faut jurer de rien}|pwk}, 1836/1848) war in der Übersetzung von Goldmann\pwindex{Goldmann, Paul 31.01.1865 – 25.09.1935@\textsc{Goldmann, Paul} (31.01.1865 – 25.09.1935), \emph{Schriftsteller/Schriftstellerin, Journalist/Journalistin}|pwk} erstmals am 5. 3. 1903 am Prager Deutschen
                     Landestheater\oindex{Staendetheater@\textbf{Ständetheater}, \emph{Theater (K.THE)}|pwk} aufgeführt worden. Am 9. 5. 1903 hatte die Premiere am \emph{Frankfurter Schauspielhaus}XXXX ORGangabe fehlt stattgefunden.}}}\label{K_L03372-4} iſt in Frankfurt\oindex{Frankfurt am Main@\textbf{Frankfurt am Main}, \emph{P.PPLA3}|pw} durchgefallen. \textsc{Musset\pwindex{Musset, Alfred de 11.12.1810 – 02.05.1857@\textsc{Musset, Alfred de} (11.12.1810 – 02.05.1857), \emph{Schriftsteller/Schriftstellerin}|pw}} ſcheint nicht mehr bühnenmöglich zu ſein; ich habe mich durch den glänzenden
               Dialog irreführen laſſen. Wahrſcheinlich ziehe ich das Stück\pwindex{Man soll nichts verschwoeren. Komoedie in 3 Akten@\emph{Man soll nichts verschwören. Komödie in 3 Akten}|pwv} nun auch in {\pb}Berlin\oindex{Berlin@\textbf{Berlin}, \emph{P.PPLC}|pw} zurück.\pend
           
\pstart
           Ich vermiſſe ſehr Deine lieben Nachrichten. Wie geht es Dir? Warum \label{K_L03372-5v}\edtext{schweigſt}{\lemma{\textnormal{\emph{schweigſt}}}\Cendnote{\textnormal{Schnitzler dürfte aufgrund von Goldmanns\pwindex{Goldmann, Paul 31.01.1865 – 25.09.1935@\textsc{Goldmann, Paul} (31.01.1865 – 25.09.1935), \emph{Schriftsteller/Schriftstellerin, Journalist/Journalistin}|pwk}{ }Feuilleton\pwindex{Berliner Theater. (»Der Schleier der Beatrice« von Arthur Schnitzler.)@\emph{Berliner Theater. (»Der Schleier der Beatrice« von Arthur Schnitzler.)}|pwkv} über \emph{Der
                     Schleier der Beatrice}\pwindex{Schleier der Beatrice. Schauspiel in fuenf Akten@\emph{Der Schleier der Beatrice. Schauspiel in fünf Akten}|pwk} anhaltend gekränkt gewesen sein.}}}\label{K_L03372-5} Du ſo
               sehr?\pend
           
\pstart
           Viele treue Grüße! {\\[\baselineskip]}Dein {\\[\baselineskip]}\spacefill\mbox{Paul Goldm}\pend
           \leftskip=0em{}\selectlanguage{ngerman}\endnumbering\briefempfaengerindex{Schnitzler, Arthur@\textsc{Schnitzler, Arthur}!zzzGoldmann, Paul@\emph{von Paul Goldmann}!1903-05-131@{13. 5. {[}1903{]}}|)be}\mylabel{L03372h}  \normalsize

\doendnotes{C}
\bigskip
\vfill

\clearpage

\footnotesize

\lohead{\textsc{register}}

% Definiere theindex-Environment komplett neu ohne reledmac
\makeatletter
\renewenvironment{theindex}{%
  \section*{\indexname}%
  \setlength{\parindent}{0pt}%
  \setlength{\parskip}{0pt plus 0.3pt}%
  \let\item\@idxitem
}{%
  \clearpage
}
\makeatother

\IfFileExists{\jobname-pw.ind}{\input{\jobname-pw.ind}}{}

\end{document}

      