%% latex-leseansicht-vorspann.tex
%% Vorspann für die Leseansicht.
%% Lädt die gemeinsame Datei latex-vorspann.tex mit nicht gesetztem Schalter.

\newif\ifkorrekturansicht
\korrekturansichtfalse

\input{../tex-inputs/latex-vorspann}


         
         \renewcommand{\erwaehntePersonen}{Personen: Paul Goldmann, Felix Paul Greve, Alfred de Musset, Olga Schnitzler, Oscar Wilde}
         \renewcommand{\erwaehnteInstitutionen}{Institutionen: J. C. C. Bruns}
         \renewcommand{\erwaehnteOrte}{Orte: Berlin, Dessauer Straße, Frankfurt am Main, Minden, Ständetheater, Wien}
         \renewcommand{\erwaehnteWerke}{Werke: Berliner Theater. (»Der Schleier der Beatrice« von Arthur Schnitzler.), Der Schleier der Beatrice. Schauspiel in fünf Akten, Fingerzeige, Il ne faut jurer de rien, Man soll nichts verschwören. Komödie in 3 Akten}
               \section[ Paul Goldmann an Arthur Schnitzler, 13. 5. {[}1903{]}]{ Paul Goldmann an Arthur Schnitzler, 13. 5. {[}1903{]}}\nopagebreak\mylabel{v}\rehead{ }\begin{ledgroupsized}[t]{13cm}\normalsize\beginnumbering \toendnotes[C]{\smallbreak\pagebreak[2]} \Standort{DLA, A:Schnitzler, HS.NZ85.1.3173.}
\physDesc{Brief, 1 Blatt, 4 Seiten, 1047 Zeichen
\newline{}Handschrift: blaue Tinte, deutsche Kurrent
\newline{}Schnitzler: 1) mit Bleistift das Jahr »903« vermerkt  2) mit rotem Buntstift eine Unterstreichung}\toendnotes[C]{\smallbreak}\pstart
           \noindent{}\raggedleft{}{\pb}\textcolor{gray}{\textbf{DESSAUERSTRASSE 19\oindex{Dessauer Strasse@\textbf{Dessauer Straße}|pw}}}\pend
           \pstart
           Berlin\oindex{Berlin@\textbf{Berlin}|pw}, 13. Mai.\pend
           \pstart\center{}Mein lieber Freund,\pend\pstart
           Ich ſende heut an \textsc{Olga\pwindex{Schnitzler, Olga 17.01.1882 – 13.01.1970@\textsc{Schnitzler, Olga} (17.01.1882 – 13.01.1970), \emph{Schauspielerin, Sängerin}|pw}} die verſprochene \label{K_L03372-1v}\edtext{Tiſchglocke}{\lemma{\textnormal{\emph{Tiſchglocke}}}\Cendnote{\textnormal{eine Glocke, die
                  geklingelt wird, wenn das Essen angerichtet ist}}}\label{K_L03372-1h} ab. Ich konnte ſie nicht
               früher ſenden, weil ich ſeit meiner \label{K_L03372-2v}\edtext{Rückkehr aus Wien\oindex{Wien@\textbf{Wien}|pw}}{\lemma{\textnormal{\emph{Rückkehr aus Wien}}}\Cendnote{\textnormal{Goldmann\pwindex{Goldmann, Paul 31.01.1865 – 25.09.1935@\textsc{Goldmann, Paul} (31.01.1865 – 25.09.1935), \emph{Schriftsteller, Journalist}|pwk} war nachweislich zwischen 14. 4. 1903 und 21. 4. 1903 in Wien\oindex{Wien@\textbf{Wien}|pwk}. In dieser Zeit traf er Schnitzler\pwindex{Schnitzler, Arthur 15.05.1862 – 21.10.1931@\textsc{Schnitzler, Arthur} (15.05.1862 – 21.10.1931), \emph{Schriftsteller, Mediziner}|pwk}, von dem er ursprünglich gedacht hatte, dass er
                  auf Reisen sei (vgl. A. S.: \emph{Tagebuch}, 14. 4. 1903), mehrmals.}}}\label{K_L03372-2h} ohne Diener war, der \strikeout{\textcolor{gray}{×}\textcolor{gray}{it}} mir das Paket hätte machen und expediren können. Entſchuldige mich bei \textsc{Olga\pwindex{Schnitzler, Olga 17.01.1882 – 13.01.1970@\textsc{Schnitzler, Olga} (17.01.1882 – 13.01.1970), \emph{Schauspielerin, Sängerin}|pw}} wegen der Verſpätung und grüße ſie herzlichſt.\pend
           \pstart
           {\pb}Ich habe \strikeout{die} in
               letzter Zeit \strikeout{\textcolor{gray}{vie}}{ }\textsc{Oscar Wilde\pwindex{Wilde, Oscar 16.10.1854 – 30.11.1900@\textsc{Wilde, Oscar} (16.10.1854 – 30.11.1900), \emph{Schriftsteller}|pw}} geleſen und in ihm einen der glänzendſten modernen Geiſter gefunden. Lies’
                  \label{K_L03372-3v}\edtext{»Fingerzeige\pwindex{Wilde, Oscar 16.10.1854 – 30.11.1900@\textsc{Wilde, Oscar} (16.10.1854 – 30.11.1900), \emph{Schriftsteller}!Fingerzeige1903?@\strich\emph{Fingerzeige} {[}1903?{]}|pw}«, in der Überſetzung von \textsc{Greve\pwindex{Greve, Felix Paul 1879-02-14 – 1948-08-19@\textsc{Greve, Felix Paul} (1879-02-14 – 1948-08-19), \emph{Schriftsteller, Übersetzer}|pw}}}{\lemma{\textnormal{\emph{»Fingerzeige«, … Greve}}}\Cendnote{\textnormal{Oscar Wilde\pwindex{Wilde, Oscar 16.10.1854 – 30.11.1900@\textsc{Wilde, Oscar} (16.10.1854 – 30.11.1900), \emph{Schriftsteller}|pwk}: \emph{Fingerzeige}\pwindex{Wilde, Oscar 16.10.1854 – 30.11.1900@\textsc{Wilde, Oscar} (16.10.1854 – 30.11.1900), \emph{Schriftsteller}!Fingerzeige1903?@\strich\emph{Fingerzeige} {[}1903?{]}|pwk}. Übersetzt von Felix Paul Greve\pwindex{Greve, Felix Paul 1879-02-14 – 1948-08-19@\textsc{Greve, Felix Paul} (1879-02-14 – 1948-08-19), \emph{Schriftsteller, Übersetzer}|pwk}. Minden\oindex{Minden@\textbf{Minden}|pwk}: \emph{J. C. C. Bruns’ Verlag}\orgindex{J. C. C. Bruns@J. C. C. Bruns|pwk}
                        [1903?]. Eine Lektüre durch Schnitzler\pwindex{Schnitzler, Arthur 15.05.1862 – 21.10.1931@\textsc{Schnitzler, Arthur} (15.05.1862 – 21.10.1931), \emph{Schriftsteller, Mediziner}|pwk} ist nicht bekannt.}}}\label{K_L03372-3h} (Verlag von \textsc{Bruns}\orgindex{J. C. C. Bruns@J. C. C. Bruns|pw} in \textsc{Minden\oindex{Minden@\textbf{Minden}|pw}}). Die beiden Dialoge\pwindex{Wilde, Oscar 16.10.1854 – 30.11.1900@\textsc{Wilde, Oscar} (16.10.1854 – 30.11.1900), \emph{Schriftsteller}!Fingerzeige1903?@\strich\emph{Fingerzeige} {[}1903?{]}|pwv}
               über die Kritik als \uline{ſchaffende} Kunst geben wieder,
               was ich im Innerſten über die Kritik denke, – \substVorne{}\textsuperscript{i}\substDazwischen{}m\substHinten{}it den { }{\pb}Worten eines großen Dichters und ſprühenden Geiſtes
               allerdings, die ich nie im ſtande geweſen wäre zu finden.\pend
           \pstart
           Meine \label{K_L03372-4v}\edtext{\textsc{Musset}-Überſetzung\pwindex{Musset, Alfred de 11.12.1810 – 02.05.1857@\textsc{Musset, Alfred de} (11.12.1810 – 02.05.1857), \emph{Schriftsteller}!Man soll nichts verschwoeren. Komoedie in 3 Akten1902-10-17@\strich\emph{Man soll nichts verschwören. Komödie in 3 Akten} {[}1902-10-17{]}|pwv}}{\lemma{\textnormal{\emph{Musset-Überſetzung}}}\Cendnote{\textnormal{\emph{Man soll nichts verschwören}\pwindex{Musset, Alfred de 11.12.1810 – 02.05.1857@\textsc{Musset, Alfred de} (11.12.1810 – 02.05.1857), \emph{Schriftsteller}!Man soll nichts verschwoeren. Komoedie in 3 Akten1902-10-17@\strich\emph{Man soll nichts verschwören. Komödie in 3 Akten} {[}1902-10-17{]}|pwk} (\emph{Il ne faut jurer de rien}\pwindex{Musset, Alfred de 11.12.1810 – 02.05.1857@\textsc{Musset, Alfred de} (11.12.1810 – 02.05.1857), \emph{Schriftsteller}!ne faut jurer de rien1848@\strich\emph{Il ne faut jurer de rien} {[}1848{]}|pwk}, 1836/1848) war in der Übersetzung von Goldmann\pwindex{Goldmann, Paul 31.01.1865 – 25.09.1935@\textsc{Goldmann, Paul} (31.01.1865 – 25.09.1935), \emph{Schriftsteller, Journalist}|pwk} erstmals am 5. 3. 1903 am Prager Deutschen
                     Landestheater\oindex{Staendetheater@\textbf{Ständetheater}|pwk} aufgeführt worden. Am 9. 5. 1903 hatte die Premiere am \emph{Frankfurter Schauspielhaus}XXXX ORGangabe fehlt stattgefunden.}}}\label{K_L03372-4h} iſt in Frankfurt\oindex{Frankfurt am Main@\textbf{Frankfurt am Main}|pw} durchgefallen. \textsc{Musset\pwindex{Musset, Alfred de 11.12.1810 – 02.05.1857@\textsc{Musset, Alfred de} (11.12.1810 – 02.05.1857), \emph{Schriftsteller}|pw}} ſcheint nicht mehr bühnenmöglich zu ſein; ich habe mich durch den glänzenden
               Dialog irreführen laſſen. Wahrſcheinlich ziehe ich das Stück\pwindex{Musset, Alfred de 11.12.1810 – 02.05.1857@\textsc{Musset, Alfred de} (11.12.1810 – 02.05.1857), \emph{Schriftsteller}!Man soll nichts verschwoeren. Komoedie in 3 Akten1902-10-17@\strich\emph{Man soll nichts verschwören. Komödie in 3 Akten} {[}1902-10-17{]}|pwv} nun auch in {\pb}Berlin\oindex{Berlin@\textbf{Berlin}|pw} zurück.\pend
           \pstart
           Ich vermiſſe ſehr Deine lieben Nachrichten. Wie geht es Dir? Warum \label{K_L03372-5v}\edtext{schweigſt}{\lemma{\textnormal{\emph{schweigſt}}}\Cendnote{\textnormal{Schnitzler\pwindex{Schnitzler, Arthur 15.05.1862 – 21.10.1931@\textsc{Schnitzler, Arthur} (15.05.1862 – 21.10.1931), \emph{Schriftsteller, Mediziner}|pwk} dürfte aufgrund von Goldmann\pwindex{Goldmann, Paul 31.01.1865 – 25.09.1935@\textsc{Goldmann, Paul} (31.01.1865 – 25.09.1935), \emph{Schriftsteller, Journalist}|pwk}s Feuilleton\pwindex{Goldmann, Paul 31.01.1865 – 25.09.1935@\textsc{Goldmann, Paul} (31.01.1865 – 25.09.1935), \emph{Schriftsteller, Journalist}!Berliner Theater. (»Der Schleier der Beatrice« von Arthur Schnitzler.)1903-03-19@\strich\emph{Berliner Theater. (»Der Schleier der Beatrice« von Arthur Schnitzler.)} {[}1903-03-19{]}|pwkv} über \emph{Der
                     Schleier der Beatrice}\pwindex{Schnitzler, Arthur 15.05.1862 – 21.10.1931@\textsc{Schnitzler, Arthur} (15.05.1862 – 21.10.1931), \emph{Schriftsteller, Mediziner}!Schleier der Beatrice. Schauspiel in fuenf Akten1900-12-01@\strich\emph{Der Schleier der Beatrice. Schauspiel in fünf Akten} {[}1900-12-01{]}|pwk} anhaltend gekränkt gewesen sein.}}}\label{K_L03372-5h} Du ſo
               sehr?\pend
           \pstart
           Viele treue Grüße! {\\[\baselineskip]}Dein {\\[\baselineskip]}\spacefill\mbox{Paul Goldm}\pend
           \leftskip=0em{}
         
         \endnumbering\mylabel{h}\end{ledgroupsized}  \newcommand{\dateiname}{L03372}\newcommand{\titel}{Paul Goldmann an Arthur Schnitzler, 13. 5. [1903]}\newcommand{\editorInnen}{Martin Anton Müller und Laura Untner}%% latex-leseansicht-abspann.tex
%% Abspann für die Leseansicht.
%% Der Schalter \ifkorrekturansicht ist bereits durch den Vorspann gesetzt.

%% latex-abspann.tex
%% Gemeinsamer Abspann für Korrekturansicht und Leseansicht.
%% Setzt den Schalter \ifkorrekturansicht voraus (gesetzt in den
%% einbindenden Dateien latex-korrekturansicht-abspann.tex bzw.
%% latex-leseansicht-abspann.tex).
%% ---------------------------------------------------------------

\normalsize

% Das esempio-Environment wird nur in der Leseansicht benötigt
\ifkorrekturansicht\else
\newenvironment{esempio}[3]%
{
    \vspace{1.5ex}
    \rlap{\underline{#1}}
    \par
    \setlength{\parindent}{0cm}
    \nopagebreak
    \leftskip=#2cm
    \rightskip=#3cm
}
{
    \par
}
\fi

\doendnotes{C}
\bigskip
\vfill

\clearpage

\footnotesize

\ifkorrekturansicht
  \lohead{\textsc{register}}
\fi

% theindex-Environment neu definieren ohne reledmac
\makeatletter
\renewenvironment{theindex}{%
  \ifkorrekturansicht
    \section*{\indexname}%
  \else
    \subsubsection*{Index der erwähnten Entitäten}%
  \fi
  \setlength{\parindent}{0pt}%
  \setlength{\parskip}{0pt plus 0.3pt}%
  \let\item\@idxitem
}{%
  \ifkorrekturansicht\clearpage\fi
}
\makeatother

\IfFileExists{\jobname-pw.ind}{\input{\jobname-pw.ind}}{}

% Quellenangabe nur in der Leseansicht
\ifkorrekturansicht\else
% Fallback-Definitionen, falls die .tex-Datei \titel etc. nicht gesetzt hat
\providecommand{\titel}{}
\providecommand{\editorInnen}{}
\providecommand{\dateiname}{\jobname}

\vspace{3cm}

\vfill

\footnotesize
\textsc{Quelle}: \titel. Herausgegeben von {\editorInnen}. In: \emph{Arthur Schnitzler: Briefwechsel mit Autorinnen und Autoren}.
 Digitale Edition, https://schnitzler-briefe.acdh.oeaw.ac.at/{\dateiname}.html (Stand \today)
\fi

\end{document}


      