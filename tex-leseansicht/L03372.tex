%% latex-leseansicht-vorspann.tex
%% Vorspann für die Leseansicht.
%% Lädt die gemeinsame Datei latex-vorspann.tex mit nicht gesetztem Schalter.

\newif\ifkorrekturansicht
\korrekturansichtfalse

\input{../tex-inputs/latex-vorspann}


\section[ Paul Goldmann an Arthur Schnitzler, 13. 5. [1903]]{L03372 Paul Goldmann an Arthur Schnitzler,  13. 5. [1903]}
\nopagebreak\mylabel{L03372v}
\rehead{ }\normalsize\beginnumbering\briefempfaengerindex{Schnitzler, Arthur@\textsc{Schnitzler, Arthur}!zzzGoldmann, Paul@\emph{von Paul Goldmann}!1903-05-131@{13. 5. [1903]}|(be}
\toendnotes[C]{\smallbreak\pagebreak[2]}
\correspDesc{Versand  durch Paul Goldmann am 13. 5. [1903] in Berlin
\newline{}Erhalt  durch Arthur Schnitzler im Zeitraum [14. 5. 1903
                  – 18. 5. 1903?] in Wien}\toendnotes[C]{\smallbreak}
\Standort{DLA, A:Schnitzler, HS.NZ85.1.3173.}
\physDesc{Brief, 1 Blatt, 4 Seiten, 1047 Zeichen
\newline{}Handschrift: blaue Tinte, deutsche Kurrent
\newline{}Schnitzler: 1) mit Bleistift das Jahr »903« vermerkt  2) mit rotem Buntstift eine Unterstreichung}\toendnotes[C]{\smallbreak}
\pstart
           \raggedleft{}{\pb}\textcolor{gray}{\textbf{DESSAUERSTRASSE 19\oindex{Dessauer Straße@\textbf{Dessauer Straße}, \emph{Straße}|pw}}}\pend
           
\pstart
           Berlin\oindex{Berlin@\textbf{Berlin}, \emph{Hauptstadt}|pw}, 13. Mai.\pend
           
\pstart\center{}Mein lieber Freund,\pend\vspace{0.5em}
\pstart
           Ich{ }ſende heut an \textsc{Olga\pwindex{Schnitzler, Olga 17.\,1.\,1882 Wien – 13.\,1.\,1970 Lugano@\textsc{Schnitzler, Olga} (17.\,1.\,1882 Wien – 13.\,1.\,1970 Lugano), \emph{Schauspielerin, Sängerin}|pw}} die verſprochene \label{K_L03372-1v}\edtext{Tiſchglocke}{\lemma{\textnormal{\emph{Tischglocke}}}\Cendnote{\textnormal{eine Glocke, die
                  geklingelt wird, wenn das Essen angerichtet ist}}}\label{K_L03372-1} ab. Ich konnte{ }ſie nicht
               früher{ }ſenden, weil ich{ }ſeit meiner \label{K_L03372-2v}\edtext{Rückkehr aus Wien\oindex{Wien@\textbf{Wien}, \emph{Verwaltungsgebiet}|pw}}{\lemma{\textnormal{\emph{Rückkehr aus Wien}}}\Cendnote{\textnormal{Goldmann\pwindex{Goldmann, Paul 31.\,1.\,1865 Breslau – 25.\,9.\,1935 Wien@\textsc{Goldmann, Paul} (31.\,1.\,1865 Breslau – 25.\,9.\,1935 Wien), \emph{Schriftsteller, Journalist}|pwk} war nachweislich zwischen 14. 4. 1903 und 21. 4. 1903 in Wien\oindex{Wien@\textbf{Wien}, \emph{Verwaltungsgebiet}|pwk}. In dieser Zeit traf er mehrmals mit Schnitzler zusammen, von dem er ursprünglich gedacht hatte, dass er
                  auf Reisen sei (vgl. A. S.: \emph{Tagebuch}, 14. 4. 1903).}}}\label{K_L03372-2} ohne Diener war, der \strikeout{\textcolor{gray}{×}\textcolor{gray}{it}} mir das Paket hätte machen und expediren können. Entſchuldige mich bei \textsc{Olga\pwindex{Schnitzler, Olga 17.\,1.\,1882 Wien – 13.\,1.\,1970 Lugano@\textsc{Schnitzler, Olga} (17.\,1.\,1882 Wien – 13.\,1.\,1970 Lugano), \emph{Schauspielerin, Sängerin}|pw}} wegen der Verſpätung und grüße{ }ſie herzlichſt.\pend
           
\pstart
           {\pb}Ich habe \strikeout{die} in
               letzter Zeit \strikeout{\textcolor{gray}{vie}}{ }\textsc{Oscar Wilde\pwindex{Wilde, Oscar 16.\,10.\,1854 Dublin – 30.\,11.\,1900 Paris@\textsc{Wilde, Oscar} (16.\,10.\,1854 Dublin – 30.\,11.\,1900 Paris), \emph{Schriftsteller}|pw}} geleſen und in ihm einen der glänzendſten modernen Geiſter gefunden. Lies’
                  \label{K_L03372-3v}\edtext{»Fingerzeige\pwindex{Wilde, Oscar 16.\,10.\,1854 Dublin – 30.\,11.\,1900 Paris@\textsc{Wilde, Oscar} (16.\,10.\,1854 Dublin – 30.\,11.\,1900 Paris), \emph{Schriftsteller}!Fingerzeige@\strich\emph{Fingerzeige}|pw}«, in der Überſetzung von \textsc{Greve\pwindex{Greve, Felix Paul 14.\,2.\,1879 Radomno – 19.\,8.\,1948 Simcoe@\textsc{Greve, Felix Paul} (14.\,2.\,1879 Radomno – 19.\,8.\,1948 Simcoe), \emph{Schriftsteller, Übersetzer}|pw}}}{\lemma{\textnormal{\emph{»Fingerzeige«, … Greve}}}\Cendnote{\textnormal{Oscar Wilde\pwindex{Wilde, Oscar 16.\,10.\,1854 Dublin – 30.\,11.\,1900 Paris@\textsc{Wilde, Oscar} (16.\,10.\,1854 Dublin – 30.\,11.\,1900 Paris), \emph{Schriftsteller}|pwk}: \emph{Fingerzeige}\pwindex{Wilde, Oscar 16.\,10.\,1854 Dublin – 30.\,11.\,1900 Paris@\textsc{Wilde, Oscar} (16.\,10.\,1854 Dublin – 30.\,11.\,1900 Paris), \emph{Schriftsteller}!Fingerzeige@\strich\emph{Fingerzeige}|pwk}. Übersetzt von Felix Paul Greve\pwindex{Greve, Felix Paul 14.\,2.\,1879 Radomno – 19.\,8.\,1948 Simcoe@\textsc{Greve, Felix Paul} (14.\,2.\,1879 Radomno – 19.\,8.\,1948 Simcoe), \emph{Schriftsteller, Übersetzer}|pwk}. Minden\oindex{Minden@\textbf{Minden}, \emph{Hauptstadt}|pwk}: \emph{J. C. C. Bruns’ Verlag}\orgindex{J. C. C. Bruns@J. C. C. Bruns|pwk}
                        [1903?]. Eine Lektüre durch Schnitzler ist nicht bekannt.}}}\label{K_L03372-3} (Verlag von \textsc{Bruns}\orgindex{J. C. C. Bruns@J. C. C. Bruns|pw} in \textsc{Minden\oindex{Minden@\textbf{Minden}, \emph{Hauptstadt}|pw}}). Die beiden Dialoge\pwindex{Wilde, Oscar 16.\,10.\,1854 Dublin – 30.\,11.\,1900 Paris@\textsc{Wilde, Oscar} (16.\,10.\,1854 Dublin – 30.\,11.\,1900 Paris), \emph{Schriftsteller}!Fingerzeige@\strich\emph{Fingerzeige}|pwv}
               über die Kritik als \uline{ſchaffende} Kunst geben wieder,
               was ich im Innerſten über die Kritik denke, – \substVorne{}\textsuperscript{i}\substDazwischen{}m\substHinten{}it den { }{\pb}Worten eines großen Dichters und{ }ſprühenden Geiſtes
               allerdings, die ich nie im{ }ſtande geweſen wäre zu finden.\pend
           
\pstart
           Meine \label{K_L03372-4v}\edtext{\textsc{Musset}-Überſetzung\pwindex{Musset, Alfred de 11.\,12.\,1810 Paris – 2.\,5.\,1857 ebd.@\textsc{Musset, Alfred de} (11.\,12.\,1810 Paris – 2.\,5.\,1857 ebd.), \emph{Schriftsteller}!Man soll nichts verschwören. Komödie in 3 Akten@\strich\emph{Man soll nichts verschwören. Komödie in 3 Akten}|pwv}}{\lemma{\textnormal{\emph{Musset-Übersetzung}}}\Cendnote{\textnormal{\emph{Man soll nichts verschwören}\pwindex{Musset, Alfred de 11.\,12.\,1810 Paris – 2.\,5.\,1857 ebd.@\textsc{Musset, Alfred de} (11.\,12.\,1810 Paris – 2.\,5.\,1857 ebd.), \emph{Schriftsteller}!Man soll nichts verschwören. Komödie in 3 Akten@\strich\emph{Man soll nichts verschwören. Komödie in 3 Akten}|pwk} (\emph{Il ne faut jurer de rien}\pwindex{Musset, Alfred de 11.\,12.\,1810 Paris – 2.\,5.\,1857 ebd.@\textsc{Musset, Alfred de} (11.\,12.\,1810 Paris – 2.\,5.\,1857 ebd.), \emph{Schriftsteller}!ne faut jurer de rien@\strich\emph{Il ne faut jurer de rien}|pwk}, 1836/1848) war in der Übersetzung von Goldmann\pwindex{Goldmann, Paul 31.\,1.\,1865 Breslau – 25.\,9.\,1935 Wien@\textsc{Goldmann, Paul} (31.\,1.\,1865 Breslau – 25.\,9.\,1935 Wien), \emph{Schriftsteller, Journalist}|pwk} erstmals am 5. 3. 1903 am Prager Deutschen
                     Landestheater\oindex{Ständetheater@\textbf{Ständetheater}, \emph{Theater}|pwk} aufgeführt worden. Am 9. 5. 1903 hatte die Premiere am \emph{Frankfurter Schauspielhaus}\orgindex{Frankfurter Stadttheater@Frankfurter Stadttheater|pwk} stattgefunden.}}}\label{K_L03372-4} iſt in Frankfurt\oindex{Frankfurt am Main@\textbf{Frankfurt am Main}, \emph{Hauptstadt}|pw} durchgefallen. \textsc{Musset\pwindex{Musset, Alfred de 11.\,12.\,1810 Paris – 2.\,5.\,1857 ebd.@\textsc{Musset, Alfred de} (11.\,12.\,1810 Paris – 2.\,5.\,1857 ebd.), \emph{Schriftsteller}|pw}}{ }ſcheint nicht mehr bühnenmöglich zu{ }ſein; ich habe mich durch den glänzenden
               Dialog irreführen laſſen. Wahrſcheinlich ziehe ich das Stück\pwindex{Musset, Alfred de 11.\,12.\,1810 Paris – 2.\,5.\,1857 ebd.@\textsc{Musset, Alfred de} (11.\,12.\,1810 Paris – 2.\,5.\,1857 ebd.), \emph{Schriftsteller}!Man soll nichts verschwören. Komödie in 3 Akten@\strich\emph{Man soll nichts verschwören. Komödie in 3 Akten}|pwv} nun auch in {\pb}Berlin\oindex{Berlin@\textbf{Berlin}, \emph{Hauptstadt}|pw} zurück.\pend
           
\pstart
           Ich vermiſſe{ }ſehr Deine lieben Nachrichten. Wie geht es Dir? Warum \label{K_L03372-5v}\edtext{schweigſt}{\lemma{\textnormal{\emph{schweigst}}}\Cendnote{\textnormal{Schnitzler dürfte aufgrund von Goldmanns\pwindex{Goldmann, Paul 31.\,1.\,1865 Breslau – 25.\,9.\,1935 Wien@\textsc{Goldmann, Paul} (31.\,1.\,1865 Breslau – 25.\,9.\,1935 Wien), \emph{Schriftsteller, Journalist}|pwk}{ }Feuilleton\pwindex{Goldmann, Paul 31.\,1.\,1865 Breslau – 25.\,9.\,1935 Wien@\textsc{Goldmann, Paul} (31.\,1.\,1865 Breslau – 25.\,9.\,1935 Wien), \emph{Schriftsteller, Journalist}!Berliner Theater. (»Der Schleier der Beatrice« von Arthur Schnitzler.)@\strich\emph{Berliner Theater. (»Der Schleier der Beatrice« von Arthur Schnitzler.)}|pwkv} über \emph{Der
                     Schleier der Beatrice}\pwindex{Schnitzler, Arthur 15. 5. 1862 Wien – 21. 10. 1931 ebd.@\textsc{Schnitzler, Arthur} (15. 5. 1862 Wien – 21. 10. 1931 ebd.), \emph{Schriftsteller, Mediziner}!Schleier der Beatrice. Schauspiel in fünf Akten@\strich\emph{Der Schleier der Beatrice. Schauspiel in fünf Akten}|pwk} anhaltend gekränkt gewesen sein.}}}\label{K_L03372-5} Du{ }ſo
               sehr?\pend
           
\pstart
           Viele treue Grüße! {\\[\baselineskip]}Dein {\\[\baselineskip]}\spacefill\mbox{Paul Goldm}\pend
           \leftskip=0em{}\selectlanguage{ngerman}\endnumbering\briefempfaengerindex{Schnitzler, Arthur@\textsc{Schnitzler, Arthur}!zzzGoldmann, Paul@\emph{von Paul Goldmann}!1903-05-131@{13. 5. [1903]}|)be}\mylabel{L03372h}  \newcommand{\dateiname}{L03372}\newcommand{\titel}{Paul Goldmann an Arthur Schnitzler, 13. 5. [1903]}\newcommand{\editorInnen}{Martin Anton Müller und Laura Untner}%% latex-leseansicht-abspann.tex
%% Abspann für die Leseansicht.
%% Der Schalter \ifkorrekturansicht ist bereits durch den Vorspann gesetzt.

%% latex-abspann.tex
%% Gemeinsamer Abspann für Korrekturansicht und Leseansicht.
%% Setzt den Schalter \ifkorrekturansicht voraus (gesetzt in den
%% einbindenden Dateien latex-korrekturansicht-abspann.tex bzw.
%% latex-leseansicht-abspann.tex).
%% ---------------------------------------------------------------

\normalsize

% Das esempio-Environment wird nur in der Leseansicht benötigt
\ifkorrekturansicht\else
\newenvironment{esempio}[3]%
{
    \vspace{1.5ex}
    \rlap{\underline{#1}}
    \par
    \setlength{\parindent}{0cm}
    \nopagebreak
    \leftskip=#2cm
    \rightskip=#3cm
}
{
    \par
}
\fi

\doendnotes{C}
\bigskip
\vfill

\clearpage

\footnotesize

\ifkorrekturansicht
  \lohead{\textsc{register}}
\fi

% theindex-Environment neu definieren ohne reledmac
\makeatletter
\renewenvironment{theindex}{%
  \ifkorrekturansicht
    \section*{\indexname}%
  \else
    \subsubsection*{Index der erwähnten Entitäten}%
  \fi
  \setlength{\parindent}{0pt}%
  \setlength{\parskip}{0pt plus 0.3pt}%
  \let\item\@idxitem
}{%
  \ifkorrekturansicht\clearpage\fi
}
\makeatother

\IfFileExists{\jobname-pw.ind}{\input{\jobname-pw.ind}}{}

% Quellenangabe nur in der Leseansicht
\ifkorrekturansicht\else
% Fallback-Definitionen, falls die .tex-Datei \titel etc. nicht gesetzt hat
\providecommand{\titel}{}
\providecommand{\editorInnen}{}
\providecommand{\dateiname}{\jobname}

\vspace{3cm}

\vfill

\footnotesize
\textsc{Quelle}: \titel. Herausgegeben von {\editorInnen}. In: \emph{Arthur Schnitzler: Briefwechsel mit Autorinnen und Autoren}.
 Digitale Edition, https://schnitzler-briefe.acdh.oeaw.ac.at/{\dateiname}.html (Stand \today)
\fi

\end{document}


