%% latex-leseansicht-vorspann.tex
%% Vorspann für die Leseansicht.
%% Lädt die gemeinsame Datei latex-vorspann.tex mit nicht gesetztem Schalter.

\newif\ifkorrekturansicht
\korrekturansichtfalse

\input{../tex-inputs/latex-vorspann}


         
         \renewcommand{\erwaehntePersonen}{Personen: Hermann Bahr, Richard Beer-Hofmann, Clément Massier}
         \renewcommand{\erwaehnteInstitutionen}{Institutionen: S. Fischer Verlag}
         \renewcommand{\erwaehnteOrte}{Orte: Golfe-Juan, Nizza, Wien}
         \renewcommand{\erwaehnteWerke}{Werke: Das Tschaperl. Ein Wiener Stück in vier Aufzügen}
               \section[Richard Beer-Hofmann an Arthur Schnitzler, 24. 12. 1898]{ Richard Beer-Hofmann an Arthur Schnitzler, 24. 12. 1898}\nopagebreak\mylabel{v}\rehead{ }\begin{ledgroupsized}[t]{13cm}\normalsize\beginnumbering\briefempfaengerindex{Schnitzler, Arthur@\textsc{Schnitzler, Arthur}!zzzBeer-Hofmann, Richard@\emph{von Richard Beer-Hofmann}!1898-12-241@{24. 12. 1898}|(be} \toendnotes[C]{\smallbreak\pagebreak[2]} \Standort{CUL, Schnitzler, B 8.}
\physDesc{Brief, 1 Blatt, 2 Seiten, 365 Zeichen
\newline{}Handschrift: schwarze Tinte, lateinische Kurrent
\newline{}Ordnung: mit Bleistift von unbekannter Hand nummeriert: »125« }\buchAbdrucke{\weitereDrucke{Arthur Schnitzler, Richard Beer-Hofmann: \emph{Briefwechsel 1891–1931}. Hg. Konstanze Fliedl. Wien, Zürich: \emph{Europaverlag} 1992, S. 126.} }\toendnotes[C]{\smallbreak}\pstart
           \raggedleft{}{\pb}24/XII 98\pend
           \pstart
           Da Sie mir die Wahl lassen, lieber Arthur – so betrachte ich es als Hochzeitsgeschenk
               damit ich erst bei Ihrer Hochzeit Ihnen ein Geschenk machen muß, als
               Geschmacklosigkeit, »\label{K_L00871-1v}\edtext{no ja weil’s
               wahr ist}{\lemma{\textnormal{\emph{no ja weil’s
               wahr ist}}}\Cendnote{\textnormal{Es dürfte sich um ein Zitat aus
                     Bahr\pwindex{Bahr, Hermann 19.07.1863 – 15.01.1934@\textsc{Bahr, Hermann} (19.07.1863 – 15.01.1934), \emph{Schriftsteller, Kritiker}|pwk}s Stück \emph{Das Tschaperl}\pwindex{Bahr, Hermann 19.07.1863 – 15.01.1934@\textsc{Bahr, Hermann} (19.07.1863 – 15.01.1934), \emph{Schriftsteller, Kritiker}!Tschaperl. Ein Wiener Stueck in vier Aufzuegen1896@\strich\emph{Das Tschaperl. Ein Wiener Stück in vier Aufzügen} {[}1896{]}|pwk} handeln, in dem es heißt: »I bin ja bloß an
                     alter Wien\oindex{Wien@\textbf{Wien}|pw}er. – Natürlich, ich verſteh’ ja
                     nix. – Ös ſeid’s ja heutigen Tags viel g’ſcheiter. Mein Gott, Ös müßt’s halt
                     noch a bißl warten, lang werd’n mer Enk eh net mehr genieren. […] Na ja, weil’s
                     wahr iſt! Was thut er mi’ denn frozzeln? Wann i a an alter Tepp bin – i kann
                     mi’ ja nit ſelber derſchlagen!« (\emph{Das Tschaperl. Ein Wiener Stück in vier Aufzügen}\pwindex{Bahr, Hermann 19.07.1863 – 15.01.1934@\textsc{Bahr, Hermann} (19.07.1863 – 15.01.1934), \emph{Schriftsteller, Kritiker}!Tschaperl. Ein Wiener Stueck in vier Aufzuegen1896@\strich\emph{Das Tschaperl. Ein Wiener Stück in vier Aufzügen} {[}1896{]}|pwk}. Berlin: 
                        \emph{S. Fischer}\orgindex{S. Fischer Verlag@S. Fischer Verlag|pwk}{ }1898,
                        S. 39). Vgl. Arthur Schnitzler an Felix Salten, 30. 5. 1908.}}}\label{K_L00871-1h}«. Diese Vase ist
                  {\pb}»Clement Massier\pwindex{Massier, Clement 1844-11-08 – 1917-03-24@\textsc{Massier, Clément} (1844-11-08 – 1917-03-24), \emph{Keramiker}|pw}. Golf St.
                  Juan\oindex{Golfe-Juan@\textbf{Golfe-Juan}|pw} bei Nizza\oindex{Nizza@\textbf{Nizza}|pw}, Reflêt metallic
                  (que?){[}«{]}. Sie müßen aber nicht glauben daß das was Besonderes
               ist.\pend
           \pstart
           Von Herzen Ihr{\\[\baselineskip]}\spacefill\mbox{Richard}\pend
           \leftskip=0em{}
         
         \endnumbering\mylabel{h}\end{ledgroupsized}  \newcommand{\dateiname}{L00871}\newcommand{\titel}{Richard Beer-Hofmann an Arthur Schnitzler, 24. 12. 1898}\newcommand{\editorInnen}{Martin Anton Müller und Gerd-Hermann Susen}%% latex-leseansicht-abspann.tex
%% Abspann für die Leseansicht.
%% Der Schalter \ifkorrekturansicht ist bereits durch den Vorspann gesetzt.

%% latex-abspann.tex
%% Gemeinsamer Abspann für Korrekturansicht und Leseansicht.
%% Setzt den Schalter \ifkorrekturansicht voraus (gesetzt in den
%% einbindenden Dateien latex-korrekturansicht-abspann.tex bzw.
%% latex-leseansicht-abspann.tex).
%% ---------------------------------------------------------------

\normalsize

% Das esempio-Environment wird nur in der Leseansicht benötigt
\ifkorrekturansicht\else
\newenvironment{esempio}[3]%
{
    \vspace{1.5ex}
    \rlap{\underline{#1}}
    \par
    \setlength{\parindent}{0cm}
    \nopagebreak
    \leftskip=#2cm
    \rightskip=#3cm
}
{
    \par
}
\fi

\doendnotes{C}
\bigskip
\vfill

\clearpage

\footnotesize

\ifkorrekturansicht
  \lohead{\textsc{register}}
\fi

% theindex-Environment neu definieren ohne reledmac
\makeatletter
\renewenvironment{theindex}{%
  \ifkorrekturansicht
    \section*{\indexname}%
  \else
    \subsubsection*{Index der erwähnten Entitäten}%
  \fi
  \setlength{\parindent}{0pt}%
  \setlength{\parskip}{0pt plus 0.3pt}%
  \let\item\@idxitem
}{%
  \ifkorrekturansicht\clearpage\fi
}
\makeatother

\IfFileExists{\jobname-pw.ind}{\input{\jobname-pw.ind}}{}

% Quellenangabe nur in der Leseansicht
\ifkorrekturansicht\else
% Fallback-Definitionen, falls die .tex-Datei \titel etc. nicht gesetzt hat
\providecommand{\titel}{}
\providecommand{\editorInnen}{}
\providecommand{\dateiname}{\jobname}

\vspace{3cm}

\vfill

\footnotesize
\textsc{Quelle}: \titel. Herausgegeben von {\editorInnen}. In: \emph{Arthur Schnitzler: Briefwechsel mit Autorinnen und Autoren}.
 Digitale Edition, https://schnitzler-briefe.acdh.oeaw.ac.at/{\dateiname}.html (Stand \today)
\fi

\end{document}


      