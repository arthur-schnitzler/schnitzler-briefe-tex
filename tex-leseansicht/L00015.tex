%% latex-korrekturansicht-vorspann.tex
%% Vorspann für die Korrekturansicht.
%% Lädt die gemeinsame Datei latex-vorspann.tex mit gesetztem Schalter.

\newif\ifkorrekturansicht
\korrekturansichttrue

\input{../tex-inputs/latex-vorspann}


\section[Arthur Schnitzler an Richard Beer-Hofmann, 23. 5. 1891]{L00015 Arthur Schnitzler an Richard Beer-Hofmann,23. 5. 1891}
\nopagebreak\mylabel{L00015v}
\rehead{ }\normalsize\beginnumbering\briefempfaengerindex{Beer-Hofmann, Richard@\textsc{Beer-Hofmann, Richard}!zzzSchnitzler, Arthur@\emph{von Arthur Schnitzler}!1891-05-232@{23. 5. 1891}|(be}
\toendnotes[C]{\smallbreak\pagebreak[2]}\Standort{YCGL, MSS 31.}
\physDesc{Briefkarte, , 500 Zeichen
\newline{}Handschrift: 1) schwarze Tinte, deutsche Kurrent\hspace{1em}2) schwarze Tinte, lateinische Kurrent (\noindent{}Adresse)\hspace{1em}
\newline{}Versand: 1) Stempel: »\nobreak{}Wien, 23 {[}5{]} 91, \textcolor{gray}{8} A\nobreak{}«.   2) Stempel: »\nobreak{}\oindex{Bruenn@\textbf{Brünn}|pwk}Brünn Stadt Brno Mesto, 24 5 91, 5\nobreak{}«.  3) Stempel: »\nobreak{}\oindex{Bruenn@\textbf{Brünn}|pwk}Brünn, 24 Mai\nobreak{}«. }
\buchAbdrucke{\weitereDrucke{Arthur Schnitzler, Richard Beer-Hofmann: \emph{Briefwechsel 1891–1931}. Wien, Zürich: \emph{Europaverlag} 1992, S. 29.} }\pstart{}{\pb}Hrn Dr. Rich. Beer-Hofmann\pend{}\pstart{}Brünn\oindex{Bruenn@\textbf{Brünn}|pw}\pend{}\pstart{}Hotel Neuhauser\oindex{Hotel Neuhauser@\textbf{Hotel Neuhauser}|pw}\pend{}{\bigskip}\vspace{1em}
\pstart
           \noindent{}{\pb}Mein lieber Richard!\pend
           
\pstart
           beſten Dank für Ihre Karte. Ich wohne \textsc{Giselastraße 11}\oindex{Ordination Arthur Schnitzler [Boesendorferstrasse 11]@\textbf{Ordination Arthur Schnitzler [Bösendorferstraße 11]}|pw}. Ihre Grüße habe ich theilweiſe ausgerichtet – habe nemlich nur \textsc{Salten}\pwindex{Salten, Felix 6.\,9.\,1869 Budapest – 8.\,10.\,1945 Zuerich@\textsc{Salten, Felix} (6.\,9.\,1869 Budapest – 8.\,10.\,1945 Zürich), \emph{Schriftsteller, Journalist, Chefredakteur}|pw} bisher geſehn, der eben bei mir {\pb}iſt und{ }ſie herzlich grüßt.\pend
           
\pstart
           Das gleiche thue ich; ob ich Sie beſuchen werde, weiß ich noch nicht; laſſen Sie
               jedenfalls in Kürze was von{ }ſich hören, Sie können auch viel und geiſtreich{ }ſchreiben. Sobald Sie zurück{ }ſind, melden Sie{ }ſich gef. bei Ihrem aufrichtig eigens
                  ergeb\textcolor{gray}{nen}\pend
           \pstart \spacefill\mbox{Arthur}\pend{}\selectlanguage{ngerman}\endnumbering\briefempfaengerindex{Beer-Hofmann, Richard@\textsc{Beer-Hofmann, Richard}!zzzSchnitzler, Arthur@\emph{von Arthur Schnitzler}!1891-05-232@{23. 5. 1891}|)be}\mylabel{L00015h}  \normalsize

\doendnotes{C}
\bigskip
\vfill

\clearpage

\footnotesize

\lohead{\textsc{register}}

% Definiere theindex-Environment komplett neu ohne reledmac
\makeatletter
\renewenvironment{theindex}{%
  \section*{\indexname}%
  \setlength{\parindent}{0pt}%
  \setlength{\parskip}{0pt plus 0.3pt}%
  \let\item\@idxitem
}{%
  \clearpage
}
\makeatother

\IfFileExists{\jobname-pw.ind}{\input{\jobname-pw.ind}}{}

\end{document}

      