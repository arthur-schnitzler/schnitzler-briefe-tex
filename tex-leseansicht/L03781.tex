%% latex-leseansicht-vorspann.tex
%% Vorspann für die Leseansicht.
%% Lädt die gemeinsame Datei latex-vorspann.tex mit nicht gesetztem Schalter.

\newif\ifkorrekturansicht
\korrekturansichtfalse

\input{../tex-inputs/latex-vorspann}


\section[Arthur Schnitzler an Stefan Zweig, {[}zwischen 25. und 31.?{]} 5. 1912]{L03781 Arthur Schnitzler an Stefan Zweig, [zwischen 25. und 31.?] 5. 1912}
\nopagebreak\mylabel{L03781v}
\rehead{ }\normalsize\beginnumbering\briefempfaengerindex{Zweig, Stefan@\textsc{Zweig, Stefan}!zzzSchnitzler, Arthur@\emph{von Arthur Schnitzler}!1912-05-312@{[zwischen 25. und 31.?] 5. 1912}|(be}
\toendnotes[C]{\smallbreak\pagebreak[2]}
\correspDesc{Versand  durch Arthur Schnitzler im Zeitraum [zwischen 25. und 31.?] 5. 1912 in Wien
\newline{}Erhalt  durch Stefan Zweig im Zeitraum [zwischen 25. 5. 1912 und 1. 6. 1912?] in Wien}\toendnotes[C]{\smallbreak}
\Standort{Jerusalem, National Library of Israel, ARC. Ms. Var. 305 1 58 Stefan Zweig Collection.}
\physDesc{Karte, 160 Zeichen
\newline{}Handschrift: schwarze Tinte, deutsche Kurrent}\toendnotes[C]{\smallbreak}
\pstart
           \noindent{}{\pb}Herzlichsten Dank, und ich möchte Ihnen doch \label{K_L03781-1v}\edtext{noch einmal}{\lemma{\textnormal{\emph{noch einmal}}}\Cendnote{\textnormal{Es gibt, abseits dieser Karte, keine erhaltene Korrespondenz
                  zwischen Schnitzler und Zweig\pwindex{Zweig, Stefan 28.\,11.\,1881 Wien – 23.\,2.\,1942 Petrópolis@\textsc{Zweig, Stefan} (28.\,11.\,1881 Wien – 23.\,2.\,1942 Petrópolis), \emph{Schriftsteller}|pwk} aus diesem Zeitraum. Am 24. 5. 1912 begegnete
                  man sich (zufällig?) bei Eugenie Bachrach\pwindex{Bachrach, Eugenie 4.\,3.\,1857 Wien – 4.\,12.\,1937 Purkersdorf@\textsc{Bachrach, Eugenie} (4.\,3.\,1857 Wien – 4.\,12.\,1937 Purkersdorf)|pwk}.
                     Schnitzler notierte sich im \emph{Tagebuch}\pwindex{Schnitzler, Arthur 15.\,5.\,1862 Wien – 21.\,10.\,1931 ebd.@\textsc{Schnitzler, Arthur} (15.\,5.\,1862 Wien – 21.\,10.\,1931 ebd.), \emph{Schriftsteller, Mediziner}!Tagebuch@\strich\emph{Tagebuch}|pwk}: »Es kamen später ›Gicki\pwindex{Grünfeld, Max 1881 – 2.\,3.\,1915 Jabłonki@\textsc{Grünfeld, Max} (1881 – 2.\,3.\,1915 Jabłonki), \emph{Rechtsanwalt}|pw}‹, Stefan Zweig\pwindex{Zweig, Stefan 28.\,11.\,1881 Wien – 23.\,2.\,1942 Petrópolis@\textsc{Zweig, Stefan} (28.\,11.\,1881 Wien – 23.\,2.\,1942 Petrópolis), \emph{Schriftsteller}|pw}, der eigentlich wie ich ihm sagte, durch seine Anregung
                     an meinem 50. Geburtstag schuld. (Er hatte mir liebe Verse geschickt und im Merker\pwindex{Merker. Österreichische Zeitschrift für Musik und Theater@\emph{Der Merker. Österreichische Zeitschrift für Musik und Theater}|pw} einen warmen Artikel\pwindex{Zweig, Stefan 28.\,11.\,1881 Wien – 23.\,2.\,1942 Petrópolis@\textsc{Zweig, Stefan} (28.\,11.\,1881 Wien – 23.\,2.\,1942 Petrópolis), \emph{Schriftsteller}!Schnitzler und die Jugend@\strich\emph{Schnitzler und die Jugend}|pwv} über mich geschrieben.) –« Das an der vorliegenden
                  Stelle gebrauchte »noch einmal« deutet darauf hin, dass die Karte nach dieser 
                  Begegnung abgefasst wurde.}}}\label{K_L03781-1}
               sagen, wie{ }ſehr mich Ihre lieben \label{K_L03781-2v}\edtext{Worte\pwindex{Zweig, Stefan 28.\,11.\,1881 Wien – 23.\,2.\,1942 Petrópolis@\textsc{Zweig, Stefan} (28.\,11.\,1881 Wien – 23.\,2.\,1942 Petrópolis), \emph{Schriftsteller}!Schnitzler und die Jugend@\strich\emph{Schnitzler und die Jugend}|pwv}}{\lemma{\textnormal{\emph{Worte}}}\Cendnote{\textnormal{Stefan Zweig\pwindex{Zweig, Stefan 28.\,11.\,1881 Wien – 23.\,2.\,1942 Petrópolis@\textsc{Zweig, Stefan} (28.\,11.\,1881 Wien – 23.\,2.\,1942 Petrópolis), \emph{Schriftsteller}|pwk}: \emph{Schnitzler und die Jugend}\pwindex{Zweig, Stefan 28.\,11.\,1881 Wien – 23.\,2.\,1942 Petrópolis@\textsc{Zweig, Stefan} (28.\,11.\,1881 Wien – 23.\,2.\,1942 Petrópolis), \emph{Schriftsteller}!Schnitzler und die Jugend@\strich\emph{Schnitzler und die Jugend}|pwk}. In: \emph{Der Merker}\pwindex{Merker. Österreichische Zeitschrift für Musik und Theater@\emph{Der Merker. Österreichische Zeitschrift für Musik und Theater}|pwk}, Jg. 3, Nr. 9, 1. 5. 1912,
                     S. 349–350. }}}\label{K_L03781-2} u Ihre{ }ſchöne \label{K_L03781-3v}\edtext{Verſe\pwindex{Zweig, Stefan 28.\,11.\,1881 Wien – 23.\,2.\,1942 Petrópolis@\textsc{Zweig, Stefan} (28.\,11.\,1881 Wien – 23.\,2.\,1942 Petrópolis), \emph{Schriftsteller}!?? [Verse zu Arthur Schnitzlers 50. Geburtstag]@\strich\emph{?? [Verse zu Arthur Schnitzlers 50. Geburtstag]}|pw}}{\lemma{\textnormal{\emph{Verse}}}\Cendnote{\textnormal{nicht erhalten}}}\label{K_L03781-3} erfreut
               haben!\pend
           
\pstart
           Ihr{\\[\baselineskip]}\spacefill\mbox{Arthur Schnitzler}\pend
           \leftskip=0em{}
\pstart
           Wien\oindex{Wien@\textbf{Wien}, \emph{Verwaltungsgebiet}|pw}, im Mai 1912\pend
           \selectlanguage{ngerman}\endnumbering\briefempfaengerindex{Zweig, Stefan@\textsc{Zweig, Stefan}!zzzSchnitzler, Arthur@\emph{von Arthur Schnitzler}!1912-05-252@{[zwischen 25. und 31.?] 5. 1912}|)be}\mylabel{L03781h}  \newcommand{\dateiname}{L03781}\newcommand{\titel}{Arthur Schnitzler an Stefan Zweig, [zwischen 25. und 31.?] 5. 1912}\newcommand{\editorInnen}{Selma Jahnke und Martin Anton Müller}%% latex-leseansicht-abspann.tex
%% Abspann für die Leseansicht.
%% Der Schalter \ifkorrekturansicht ist bereits durch den Vorspann gesetzt.

%% latex-abspann.tex
%% Gemeinsamer Abspann für Korrekturansicht und Leseansicht.
%% Setzt den Schalter \ifkorrekturansicht voraus (gesetzt in den
%% einbindenden Dateien latex-korrekturansicht-abspann.tex bzw.
%% latex-leseansicht-abspann.tex).
%% ---------------------------------------------------------------

\normalsize

% Das esempio-Environment wird nur in der Leseansicht benötigt
\ifkorrekturansicht\else
\newenvironment{esempio}[3]%
{
    \vspace{1.5ex}
    \rlap{\underline{#1}}
    \par
    \setlength{\parindent}{0cm}
    \nopagebreak
    \leftskip=#2cm
    \rightskip=#3cm
}
{
    \par
}
\fi

\doendnotes{C}
\bigskip
\vfill

\clearpage

\footnotesize

\ifkorrekturansicht
  \lohead{\textsc{register}}
\fi

% theindex-Environment neu definieren ohne reledmac
\makeatletter
\renewenvironment{theindex}{%
  \ifkorrekturansicht
    \section*{\indexname}%
  \else
    \subsubsection*{Index der erwähnten Entitäten}%
  \fi
  \setlength{\parindent}{0pt}%
  \setlength{\parskip}{0pt plus 0.3pt}%
  \let\item\@idxitem
}{%
  \ifkorrekturansicht\clearpage\fi
}
\makeatother

\IfFileExists{\jobname-pw.ind}{\input{\jobname-pw.ind}}{}

% Quellenangabe nur in der Leseansicht
\ifkorrekturansicht\else
% Fallback-Definitionen, falls die .tex-Datei \titel etc. nicht gesetzt hat
\providecommand{\titel}{}
\providecommand{\editorInnen}{}
\providecommand{\dateiname}{\jobname}

\vspace{3cm}

\vfill

\footnotesize
\textsc{Quelle}: \titel. Herausgegeben von {\editorInnen}. In: \emph{Arthur Schnitzler: Briefwechsel mit Autorinnen und Autoren}.
 Digitale Edition, https://schnitzler-briefe.acdh.oeaw.ac.at/{\dateiname}.html (Stand \today)
\fi

\end{document}


