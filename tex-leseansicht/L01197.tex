%% latex-korrekturansicht-vorspann.tex
%% Vorspann für die Korrekturansicht.
%% Lädt die gemeinsame Datei latex-vorspann.tex mit gesetztem Schalter.

\newif\ifkorrekturansicht
\korrekturansichttrue

\input{../tex-inputs/latex-vorspann}


\section[Hermann Bahr an Arthur Schnitzler, 9. 1. 1902]{L01197 Hermann Bahr an Arthur Schnitzler, 9. 1. 1902}
\nopagebreak\mylabel{L01197v}
\rehead{ }\normalsize\beginnumbering\briefempfaengerindex{Schnitzler, Arthur@\textsc{Schnitzler, Arthur}!zzzBahr, Hermann@\emph{von Hermann Bahr}!1902-01-091@{9. 1. 1902}|(be}
\toendnotes[C]{\smallbreak\pagebreak[2]}\Standort{CUL, Schnitzler, B 5b.}
\physDesc{Brief, 1 Blatt, 2 Seiten, 500 Zeichen
\newline{}Handschrift: schwarze Tinte, deutsche Kurrent
\newline{}Schnitzler: mit Bleistift die Jahreszahl »902« ergänzt 
\newline{}Ordnung: mit Bleistift von unbekannter Hand nummeriert:
                                    »85« }
\buchAbdrucke{\weitereDrucke{Hermann Bahr, Arthur Schnitzler: \emph{Briefwechsel, Aufzeichnungen, Dokumente (1891–1931)}. Göttingen: \emph{Wallstein} 2018, S. 223.} }\toendnotes[C]{\smallbreak}
\pstart
           \centering{}{\pb}\textcolor{gray}{\textbf{Redaktion des Neuen Wiener Tagblatt\orgindex{Neues Wiener Tagblatt@Neues Wiener Tagblatt|pw}}}\pend
           
\pstart
           \centering{}\textcolor{gray}{\textbf{\textsc{Wien, I., Rothenturmstrasse,
                        Steyrerhof\oindex{Steyrerhof@\textbf{Steyrerhof}, \emph{Gebäude (K.GBD)}|pw}.}}}\pend
           
\pstart
           \centering{}\textcolor{gray}{\textbf{Telegramm-Adresse: Tagblatt\orgindex{Neues Wiener Tagblatt@Neues Wiener Tagblatt|pw}, Steyrerhof, Wien\oindex{Steyrerhof@\textbf{Steyrerhof}, \emph{Gebäude (K.GBD)}|pw}. –
                     Telephon Nr. 384. Staats-Telephon Nr. 36.}}\pend
           
\pstart
           \raggedleft{}9/I\pend
           
\pstart\center{}Lieber Arthur!\pend\vspace{0.5em}
\pstart
           Eben erfahre ich von meinem Sendboten, der bei Schlenther\pwindex{Schlenther, Paul 20.08.1854 – 30.04.1916@\textsc{Schlenther, Paul} (20.08.1854 – 30.04.1916), \emph{Schriftsteller/Schriftstellerin, Kritiker/Kritikerin, Theaterleiter/Theaterleiterin}|pw} war\pend
           
\pstart
           1) Schnitzler bekommt den Grillparzerpreis\orgindex{Franz-Grillparzer-Preis@Franz-Grillparzer-Preis|pw}{ }\uline{nicht};\pend
           
\pstart
           2) Schlenther\pwindex{Schlenther, Paul 20.08.1854 – 30.04.1916@\textsc{Schlenther, Paul} (20.08.1854 – 30.04.1916), \emph{Schriftsteller/Schriftstellerin, Kritiker/Kritikerin, Theaterleiter/Theaterleiterin}|pw} bezeichnet es als abſolut
               falſch, wenn man meine, Schnitzler ſei durch die Guſtl\pwindex{Lieutenant Gustl. Novelle@\emph{Lieutenant Gustl. Novelle}|pw}-Affaire burgtheaterunfähig\oindex{Burgtheater@\textbf{Burgtheater}, \emph{S.THTR}|pw}
               geworden; diese Auffaſſung bestehe weder in der Intendanz noch bei ihm ſelbſt; die
                  »Lebendigen Stunden\pwindex{Lebendige Stunden. Vier Einakter@\emph{Lebendige Stunden. Vier Einakter}|pw}« kenne er leider
               nicht.\pend
           
\pstart
           Ich \label{K_L01197-1v}\edtext{fahre in einer Stunde ab}{\lemma{\textnormal{\emph{fahre in einer Stunde ab}}}\Cendnote{\textnormal{zur Premiere von \emph{Der Krampus}\pwindex{Krampus. Lustspiel in drei Aufzuegen@\emph{Der Krampus. Lustspiel in drei Aufzügen}|pwk} in Hamburg\oindex{Hamburg@\textbf{Hamburg}, \emph{P.PPLA}|pwk}}}}\label{K_L01197-1}. Überleg Dir, bis {\pb}ich wiederkomm’, ob ich
               nicht doch mit den Stücken\pwindex{Frau mit dem Dolche@\emph{Die Frau mit dem Dolche}|pwv}\pwindex{Literatur@\emph{Literatur}|pwv}\pwindex{Lebendige Stunden@\emph{Lebendige Stunden}|pwv} reſolut hingehen darf.\pend
           
\pstart
           Herzlichſt{\\[\baselineskip]}\spacefill\mbox{Hermann}\pend
           \leftskip=0em{}\selectlanguage{ngerman}\endnumbering\briefempfaengerindex{Schnitzler, Arthur@\textsc{Schnitzler, Arthur}!zzzBahr, Hermann@\emph{von Hermann Bahr}!1902-01-091@{9. 1. 1902}|)be}\mylabel{L01197h}  \normalsize

\doendnotes{C}
\bigskip
\vfill

\clearpage

\footnotesize

\lohead{\textsc{register}}

% Definiere theindex-Environment komplett neu ohne reledmac
\makeatletter
\renewenvironment{theindex}{%
  \section*{\indexname}%
  \setlength{\parindent}{0pt}%
  \setlength{\parskip}{0pt plus 0.3pt}%
  \let\item\@idxitem
}{%
  \clearpage
}
\makeatother

\IfFileExists{\jobname-pw.ind}{\input{\jobname-pw.ind}}{}

\end{document}

      