%% latex-leseansicht-vorspann.tex
%% Vorspann für die Leseansicht.
%% Lädt die gemeinsame Datei latex-vorspann.tex mit nicht gesetztem Schalter.

\newif\ifkorrekturansicht
\korrekturansichtfalse

\input{../tex-inputs/latex-vorspann}


\section[Hermann Bahr an Arthur Schnitzler, 9. 1. 1902]{L01197 Hermann Bahr an Arthur Schnitzler, 9. 1. 1902}
\nopagebreak\mylabel{L01197v}
\rehead{ }\normalsize\beginnumbering\briefempfaengerindex{Schnitzler, Arthur@\textsc{Schnitzler, Arthur}!zzzBahr, Hermann@\emph{von Hermann Bahr}!1902-01-091@{9. 1. 1902}|(be}
\toendnotes[C]{\smallbreak\pagebreak[2]}
\correspDesc{Versand  durch Hermann Bahr am 9. 1. 1902 in Wien
\newline{}Erhalt  durch Arthur Schnitzler im Zeitraum [9. 1. 1902
                  – 13. 1. 1902?] in Wien}\toendnotes[C]{\smallbreak}
\Standort{CUL, Schnitzler, B 5b.}
\physDesc{Brief, 1 Blatt, 2 Seiten, 500 Zeichen
\newline{}Handschrift: schwarze Tinte, deutsche Kurrent
\newline{}Schnitzler: mit Bleistift die Jahreszahl »902« ergänzt 
\newline{}Ordnung: mit Bleistift von unbekannter Hand nummeriert:
                                    »85« }
\buchAbdrucke{\weitereDrucke{Hermann Bahr, Arthur Schnitzler: \emph{Briefwechsel, Aufzeichnungen, Dokumente (1891–1931)}. Herausgegeben von Kurt Ifkovits und Martin Anton Müller. Göttingen: \emph{Wallstein} 2018, S. 223.} }\toendnotes[C]{\smallbreak}
\pstart
           \centering{}{\pb}\textcolor{gray}{\textbf{Redaktion des Neuen Wiener Tagblatt\orgindex{Neues Wiener Tagblatt@Neues Wiener Tagblatt|pw}}}\pend
           
\pstart
           \centering{}\textcolor{gray}{\textbf{\textsc{Wien, I., Rothenturmstrasse,
                        Steyrerhof\oindex{Wien@\textbf{Wien}!I., Innere Stadt@\textbf{I., Innere Stadt}!Steyrerhof@\textbf{Steyrerhof}, \emph{Gebäude}|pw}.}}}\pend
           
\pstart
           \centering{}\textcolor{gray}{\textbf{Telegramm-Adresse: Tagblatt\orgindex{Neues Wiener Tagblatt@Neues Wiener Tagblatt|pw}, Steyrerhof, Wien\oindex{Wien@\textbf{Wien}!I., Innere Stadt@\textbf{I., Innere Stadt}!Steyrerhof@\textbf{Steyrerhof}, \emph{Gebäude}|pw}. –
                     Telephon Nr. 384. Staats-Telephon Nr. 36.}}\pend
           
\pstart
           \raggedleft{}9/I\pend
           
\pstart\center{}Lieber Arthur!\pend\vspace{0.5em}
\pstart
           Eben erfahre ich von meinem Sendboten, der bei Schlenther\pwindex{Schlenther, Paul 20.\,8.\,1854 Chernyakhovsk – 30.\,4.\,1916 Berlin@\textsc{Schlenther, Paul} (20.\,8.\,1854 Chernyakhovsk – 30.\,4.\,1916 Berlin), \emph{Schriftsteller, Kritiker, Theaterleiter}|pw} war\pend
           
\pstart
           1) Schnitzler bekommt den Grillparzerpreis\orgindex{Franz-Grillparzer-Preis@Franz-Grillparzer-Preis|pw}{ }\uline{nicht};\pend
           
\pstart
           2) Schlenther\pwindex{Schlenther, Paul 20.\,8.\,1854 Chernyakhovsk – 30.\,4.\,1916 Berlin@\textsc{Schlenther, Paul} (20.\,8.\,1854 Chernyakhovsk – 30.\,4.\,1916 Berlin), \emph{Schriftsteller, Kritiker, Theaterleiter}|pw} bezeichnet es als abſolut
               falſch, wenn man meine, Schnitzler{ }ſei durch die Guſtl\pwindex{Schnitzler, Arthur 15.\,5.\,1862 Wien – 21.\,10.\,1931 ebd.@\textsc{Schnitzler, Arthur} (15.\,5.\,1862 Wien – 21.\,10.\,1931 ebd.), \emph{Schriftsteller, Mediziner}!Lieutenant Gustl. Novelle@\strich\emph{Lieutenant Gustl. Novelle}|pw}-Affaire burgtheaterunfähig\oindex{Wien@\textbf{Wien}!I., Innere Stadt@\textbf{I., Innere Stadt}!Burgtheater@\textbf{Burgtheater}, \emph{Theater}|pw}
               geworden; diese Auffaſſung bestehe weder in der Intendanz noch bei ihm{ }ſelbſt; die
                  »Lebendigen Stunden\pwindex{Schnitzler, Arthur 15.\,5.\,1862 Wien – 21.\,10.\,1931 ebd.@\textsc{Schnitzler, Arthur} (15.\,5.\,1862 Wien – 21.\,10.\,1931 ebd.), \emph{Schriftsteller, Mediziner}!Lebendige Stunden. Vier Einakter@\strich\emph{Lebendige Stunden. Vier Einakter}|pw}« kenne er leider
               nicht.\pend
           
\pstart
           Ich \label{K_L01197-1v}\edtext{fahre in einer Stunde ab}{\lemma{\textnormal{\emph{fahre in einer Stunde ab}}}\Cendnote{\textnormal{zur Premiere von \emph{Der Krampus}\pwindex{Bahr, Hermann 19.\,7.\,1863 Linz – 15.\,1.\,1934 München@\textsc{Bahr, Hermann} (19.\,7.\,1863 Linz – 15.\,1.\,1934 München), \emph{Schriftsteller, Kritiker}!Krampus. Lustspiel in drei Aufzügen@\strich\emph{Der Krampus. Lustspiel in drei Aufzügen}|pwk} in Hamburg\oindex{Hamburg@\textbf{Hamburg}|pwk}}}}\label{K_L01197-1}. Überleg Dir, bis {\pb}ich wiederkomm’, ob ich
               nicht doch mit den Stücken\pwindex{Schnitzler, Arthur 15.\,5.\,1862 Wien – 21.\,10.\,1931 ebd.@\textsc{Schnitzler, Arthur} (15.\,5.\,1862 Wien – 21.\,10.\,1931 ebd.), \emph{Schriftsteller, Mediziner}!Frau mit dem Dolche@\strich\emph{Die Frau mit dem Dolche}|pwv}\pwindex{Schnitzler, Arthur 15.\,5.\,1862 Wien – 21.\,10.\,1931 ebd.@\textsc{Schnitzler, Arthur} (15.\,5.\,1862 Wien – 21.\,10.\,1931 ebd.), \emph{Schriftsteller, Mediziner}!Literatur@\strich\emph{Literatur}|pwv}\pwindex{Schnitzler, Arthur 15.\,5.\,1862 Wien – 21.\,10.\,1931 ebd.@\textsc{Schnitzler, Arthur} (15.\,5.\,1862 Wien – 21.\,10.\,1931 ebd.), \emph{Schriftsteller, Mediziner}!Lebendige Stunden@\strich\emph{Lebendige Stunden}|pwv} reſolut hingehen darf.\pend
           
\pstart
           Herzlichſt{\\[\baselineskip]}\spacefill\mbox{Hermann}\pend
           \leftskip=0em{}\selectlanguage{ngerman}\endnumbering\briefempfaengerindex{Schnitzler, Arthur@\textsc{Schnitzler, Arthur}!zzzBahr, Hermann@\emph{von Hermann Bahr}!1902-01-091@{9. 1. 1902}|)be}\mylabel{L01197h}  \newcommand{\dateiname}{L01197}\newcommand{\titel}{Hermann Bahr an Arthur Schnitzler, 9. 1. 1902}\newcommand{\editorInnen}{Herausgegeben von Martin Anton Müller}%% latex-leseansicht-abspann.tex
%% Abspann für die Leseansicht.
%% Der Schalter \ifkorrekturansicht ist bereits durch den Vorspann gesetzt.

%% latex-abspann.tex
%% Gemeinsamer Abspann für Korrekturansicht und Leseansicht.
%% Setzt den Schalter \ifkorrekturansicht voraus (gesetzt in den
%% einbindenden Dateien latex-korrekturansicht-abspann.tex bzw.
%% latex-leseansicht-abspann.tex).
%% ---------------------------------------------------------------

\normalsize

% Das esempio-Environment wird nur in der Leseansicht benötigt
\ifkorrekturansicht\else
\newenvironment{esempio}[3]%
{
    \vspace{1.5ex}
    \rlap{\underline{#1}}
    \par
    \setlength{\parindent}{0cm}
    \nopagebreak
    \leftskip=#2cm
    \rightskip=#3cm
}
{
    \par
}
\fi

\doendnotes{C}
\bigskip
\vfill

\clearpage

\footnotesize

\ifkorrekturansicht
  \lohead{\textsc{register}}
\fi

% theindex-Environment neu definieren ohne reledmac
\makeatletter
\renewenvironment{theindex}{%
  \ifkorrekturansicht
    \section*{\indexname}%
  \else
    \subsubsection*{Index der erwähnten Entitäten}%
  \fi
  \setlength{\parindent}{0pt}%
  \setlength{\parskip}{0pt plus 0.3pt}%
  \let\item\@idxitem
}{%
  \ifkorrekturansicht\clearpage\fi
}
\makeatother

\IfFileExists{\jobname-pw.ind}{\input{\jobname-pw.ind}}{}

% Quellenangabe nur in der Leseansicht
\ifkorrekturansicht\else
% Fallback-Definitionen, falls die .tex-Datei \titel etc. nicht gesetzt hat
\providecommand{\titel}{}
\providecommand{\editorInnen}{}
\providecommand{\dateiname}{\jobname}

\vspace{3cm}

\vfill

\footnotesize
\textsc{Quelle}: \titel. Herausgegeben von {\editorInnen}. In: \emph{Arthur Schnitzler: Briefwechsel mit Autorinnen und Autoren}.
 Digitale Edition, https://schnitzler-briefe.acdh.oeaw.ac.at/{\dateiname}.html (Stand \today)
\fi

\end{document}


