%% latex-korrekturansicht-vorspann.tex
%% Vorspann für die Korrekturansicht.
%% Lädt die gemeinsame Datei latex-vorspann.tex mit gesetztem Schalter.

\newif\ifkorrekturansicht
\korrekturansichttrue

\input{../tex-inputs/latex-vorspann}


\section[Arthur Schnitzler an Robert Adam, 4. 6. 1912]{L02073 Arthur Schnitzler an Robert Adam, 4. 6. 1912}
\nopagebreak\mylabel{L02073v}
\rehead{ }\normalsize\beginnumbering\briefempfaengerindex{Adam, Robert@\textsc{Adam, Robert}!zzzSchnitzler, Arthur@\emph{von Arthur Schnitzler}!1912-06-041@{4. 6. 1912}|(be}
\toendnotes[C]{\smallbreak\pagebreak[2]}\Standort{DLA, 96.34.1/8.}
\physDesc{Postkarte, , Umschlag, 103 Zeichen
\newline{}\noindent{}Umschlag mit Schreibmaschine\noindent{}Umschlag mit Schreibmaschine
\newline{}Handschrift: schwarze Tinte, deutsche Kurrent
\newline{}Versand: Stempel: »\nobreak{}\oindex{XVIII., Waehring@\textbf{XVIII., Währing}, \emph{A.ADM3}|pwk}18/1 Wien, 4. VI. {[}1912{]}\nobreak{}«.  
\newline{}Ordnung: mit Bleistift von unbekannter Hand Kuvert datiert: »Mai 1912« }\pstart{}{\pb}{[}ms.:{]} Herrn Dr. R. A. Pollak\pend{}\pstart{}Bezirksrichter\pend{}\pstart{}Zistersdorf\oindex{Zistersdorf@\textbf{Zistersdorf}, \emph{A.ADM3}|pw}.\pend{}\pstart{}N.Oe.\oindex{Niederoesterreich@\textbf{Niederösterreich}, \emph{A.ADM1}|pw}\pend{}{\bigskip}\vspace{1em}
\pstart
           \noindent{}{\pb}Herzlichſten Dank\pend
           \pstart \spacefill\mbox{Arthur Schnitzler}\pend{}
\pstart
           Wien\oindex{Wien@\textbf{Wien}, \emph{A.ADM2}|pw}, im Mai 1912\pend
           \selectlanguage{ngerman}\endnumbering\briefempfaengerindex{Adam, Robert@\textsc{Adam, Robert}!zzzSchnitzler, Arthur@\emph{von Arthur Schnitzler}!1912-06-041@{4. 6. 1912}|)be}\mylabel{L02073h}  \normalsize

\doendnotes{C}
\bigskip
\vfill

\clearpage

\footnotesize

\lohead{\textsc{register}}

% Definiere theindex-Environment komplett neu ohne reledmac
\makeatletter
\renewenvironment{theindex}{%
  \section*{\indexname}%
  \setlength{\parindent}{0pt}%
  \setlength{\parskip}{0pt plus 0.3pt}%
  \let\item\@idxitem
}{%
  \clearpage
}
\makeatother

\IfFileExists{\jobname-pw.ind}{\input{\jobname-pw.ind}}{}

\end{document}

      