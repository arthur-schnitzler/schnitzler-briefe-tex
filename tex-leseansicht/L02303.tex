%% latex-leseansicht-vorspann.tex
%% Vorspann für die Leseansicht.
%% Lädt die gemeinsame Datei latex-vorspann.tex mit nicht gesetztem Schalter.

\newif\ifkorrekturansicht
\korrekturansichtfalse

\input{../tex-inputs/latex-vorspann}


               \section[Georg Brandes an Arthur Schnitzler, 18. 9. 1918]{ Georg Brandes an Arthur Schnitzler, 18. 9. 1918}\nopagebreak\mylabel{v}\rehead{ }\begin{ledgroupsized}[t]{13cm}\normalsize\beginnumbering\briefempfaengerindex{Schnitzler, Arthur@\textsc{Schnitzler, Arthur}!zzzBrandes, Georg@\emph{von Georg Brandes}!1918-09-181@{18. 9. 1918}|(be} \toendnotes[C]{\smallbreak\pagebreak[2]} \Standort{CUL, Schnitzler, B 17.}
\physDesc{Brief, 1 Blatt, 4 Seiten
\newline{}Handschrift: schwarze Tinte, lateinische Kurrent
\newline{}Schnitzler: 1) mit Bleistift beschriftet: »\textsc{Brandes}« 2) mit rotem Buntstift vereinzelte Unterstreichungen\newline{}Ordnung: von unbekannter Hand nummeriert: »49« }\buchAbdrucke{\weitereDrucke{Georg Brandes, Arthur Schnitzler: \emph{Ein Briefwechsel}. Hg. Kurt Bergel. Bern: \emph{Francke} 1956, S. 124–125.} }\toendnotes[C]{\smallbreak}\pstart
           \raggedleft{}{\pb}Kopenhagen\oindex{Kopenhagen@\textbf{Kopenhagen}|pw}{ }18 Sept 18\pend
           \pstart{}Lieber verehrter Freund\pend\pstart
           Mein Trieb war, augenblicklich einen so herzlichen Brief zu beantworten. Es war
                    mir nicht möglich Zeit zu finden. Endlich nach anderthalb Jahren Arbeit sind die
                        zwei Bände über Cäsar\pwindex{Caesar, Gaius Iulius 13.7.100? v. Chr. – 15.3.44 v. Chr.@\textsc{Caesar, Gaius Iulius} (13.7.100? v. Chr. – 15.3.44 v. Chr.), \emph{Politiker, Kaiser, Heerführer}|pw}\pwindex{Brandes, Georg 04.02.1842 – 19.02.1927@\textsc{Brandes, Georg} (04.02.1842 – 19.02.1927)!Gaius Julius Cæsar1918@\strich\emph{Gaius Julius Cæsar} {[}1918{]}|pwv}, der erste von 500, der andere von 600 Seiten grossen Formats, vollendet,
                    und ich kann aufatmen.\pend
           \pstart
           Erinnern Sie sich einmal vor Jahren, es war eben an Ihrem Geburtstag und Sie
                    waren so freundlich gewesen, mich zu Tisch einzuladen; ich sagte: Sie sind
                    gerade 20 Jahre jünger als ich; Sie antworteten: Und wir beabsichtigen auch
                    ferner diese Distanz von einander zu halten. – So ist es gegangen, die Distanz
                    ist geblieben, eine seelische Entfernung nicht eingetreten.\pend
           \pstart
           Ich habe Sie nie vergessen, mich immer mit Ihnen beschäftigt, und auch Sie
                    gedenken freundlich meiner, obwohl wir uns nur selten sahen.\pend
           \pstart
           {\pb}Hier hat man in der \label{K_L02303_1v}\edtext{vorigen Saison}{\lemma{\textnormal{\emph{vorigen Saison}}}\Cendnote{\textnormal{Premiere von \emph{Erkendelsens
                            Time}\pwindex{Schnitzler, Arthur 15.05.1862 – 21.10.1931@\textsc{Schnitzler, Arthur} (15.05.1862 – 21.10.1931), \emph{Schriftsteller, Mediziner}!Stunde des Erkennens1915@\strich\emph{Stunde des Erkennens} {[}1915{]}|pwk} (\emph{Stunde der Erkennens}\pwindex{Schnitzler, Arthur 15.05.1862 – 21.10.1931@\textsc{Schnitzler, Arthur} (15.05.1862 – 21.10.1931), \emph{Schriftsteller, Mediziner}!Stunde des Erkennens1915@\strich\emph{Stunde des Erkennens} {[}1915{]}|pwk}) und
                            \emph{Den store Scene}\pwindex{Schnitzler, Arthur 15.05.1862 – 21.10.1931@\textsc{Schnitzler, Arthur} (15.05.1862 – 21.10.1931), \emph{Schriftsteller, Mediziner}!Grosse Szene1915@\strich\emph{Große Szene} {[}1915{]}|pwk} (\emph{Große Szene}\pwindex{Schnitzler, Arthur 15.05.1862 – 21.10.1931@\textsc{Schnitzler, Arthur} (15.05.1862 – 21.10.1931), \emph{Schriftsteller, Mediziner}!Grosse Szene1915@\strich\emph{Große Szene} {[}1915{]}|pwk}) am 22. 3. 1918 am Königlichen Theater\oindex{Det Kongelige Teater@\textbf{Det Kongelige Teater}|pwk}.}}}\label{K_L02303_1h} versucht, zwei
                    Ihrer Stücke\pwindex{Schnitzler, Arthur 15.05.1862 – 21.10.1931@\textsc{Schnitzler, Arthur} (15.05.1862 – 21.10.1931), \emph{Schriftsteller, Mediziner}!Stunde des Erkennens1915@\strich\emph{Stunde des Erkennens} {[}1915{]}|pwv}\pwindex{Schnitzler, Arthur 15.05.1862 – 21.10.1931@\textsc{Schnitzler, Arthur} (15.05.1862 – 21.10.1931), \emph{Schriftsteller, Mediziner}!Grosse Szene1915@\strich\emph{Große Szene} {[}1915{]}|pwv} zu
                    spielen, ich sah das eine, das Stück\pwindex{Schnitzler, Arthur 15.05.1862 – 21.10.1931@\textsc{Schnitzler, Arthur} (15.05.1862 – 21.10.1931), \emph{Schriftsteller, Mediziner}!Grosse Szene1915@\strich\emph{Große Szene} {[}1915{]}|pwv} über den Schauspieler, das sehr gefiel und nicht übel gegeben
                    wurde. Jetzt wird wieder \label{K_L02303_2v}\edtext{etwas\pwindex{Schnitzler, Arthur 15.05.1862 – 21.10.1931@\textsc{Schnitzler, Arthur} (15.05.1862 – 21.10.1931), \emph{Schriftsteller, Mediziner}!Literatur1901@\strich\emph{Literatur} {[}1901{]}|pwv}}{\lemma{\textnormal{\emph{etwas}}}\Cendnote{\textnormal{\emph{Literatur}\pwindex{Schnitzler, Arthur 15.05.1862 – 21.10.1931@\textsc{Schnitzler, Arthur} (15.05.1862 – 21.10.1931), \emph{Schriftsteller, Mediziner}!Literatur1901@\strich\emph{Literatur} {[}1901{]}|pwk} wurde als Gastspiel, gemeinsam
                        mit \emph{Große Szene}\pwindex{Schnitzler, Arthur 15.05.1862 – 21.10.1931@\textsc{Schnitzler, Arthur} (15.05.1862 – 21.10.1931), \emph{Schriftsteller, Mediziner}!Grosse Szene1915@\strich\emph{Große Szene} {[}1915{]}|pwk} am
                            30. 9. 1918 gegeben.}}}\label{K_L02303_2h} von Ihnen, an einem anderen
                    Theater, gespielt werden. Man hat hier leider immer weniger Kunstverstand; doch
                    werden Sie geschätzt; nur sagt unsere unglaublich idiotische Kritik, Sie seien
                    von Peter Nansen\pwindex{Nansen, Peter 20.01.1861 – 31.07.1918@\textsc{Nansen, Peter} (20.01.1861 – 31.07.1918), \emph{Schriftsteller, Journalist, Verleger}|pw}{ }\uline{beeinflusst}. Ich glaube, Sie schrieben, bevor
                    Sie seinen Namen gehört hatten. Und wo wäre die Ähnlichkeit!\pend
           \pstart
           Nansen\pwindex{Nansen, Peter 20.01.1861 – 31.07.1918@\textsc{Nansen, Peter} (20.01.1861 – 31.07.1918), \emph{Schriftsteller, Journalist, Verleger}|pw}s Tod war die Veranlassung Ihres guten
                    Briefes. Dieser Tod hat mich tief ergriffen, so tief, dass es mir ist, als lebte
                    er noch. Mir gegenüber ein sonderbarer Mensch. Dreissig Jahre hat er mich
                    gekannt, und \introOben{}in 25\introOben{} mir nie näher getreten. In seinen
                    beiden Ehen war ich nie in sein Haus geladen, ich habe nicht einmal in einem
                    flüchtigen Besuch je seine Wohnung {\pb}gesehen. Dann plötzlich in
                    den fünf–sechs letzten Lebensjahren schloss er sich mit einer Innigkeit an mich,
                    dass ich eine Art Hauptperson in seiner Gedankenwelt wurde, er widmete mir
                    öffentlich seine Bücher\pwindex{Nansen, Peter 20.01.1861 – 31.07.1918@\textsc{Nansen, Peter} (20.01.1861 – 31.07.1918), \emph{Schriftsteller, Journalist, Verleger}!Brueder Menthe1915@\strich\emph{Die Brüder Menthe} {[}1915{]}|pwv},
                    schrieb öfters über mich – natürlich meistens irrthümlich – aber mit dem besten
                    Willen.\pend
           \pstart
           Es war sehr, sehr traurig, die Abnahme seiner Kräfte zu verfolgen. Man litt fast
                    mit ihm.\pend
           \pstart
           Und doch ertrinkt dies Einzelne in dem allgemeinen Jammer der Menschheit. Glauben
                    Sie nicht \strikeout{ab} auch, dass diese Kugel, Erde
                    genannt, in dem Weltall den Record bestialischer Stupidität geschlagen hat? Es
                    scheint mir unmöglich, dass ein anderer Globus von dümmeren und ekelhafteren
                    Wesen bewohnt sein kann.\pend
           \pstart
           Ab und zu werde ich von Oesterreichern\oindex{Oesterreich@\textbf{Österreich}|pw}
                    aufgesucht, aber es ist zuletzt unerträglich, von seinen Landsleuten als
                    Gebrauchsgegenstand, {\pb}von
                    Fremden als Sehenswürdigkeit aufgesucht zu werden. Wenn vierzig Briefe und
                    12 Bände pr. Tag \strikeout{kommen} mit der Post \introOben{}kommen,\introOben{} und wenn es alle drei Minuten an der Türe
                    schellt, so ist es unmöglich, nicht zu wüthen.\pend
           \pstart
           Sie irren sich völlig, wenn Sie glauben, dass ich hier für einen Vertreter dänischen\oindex{Daenemark@\textbf{Dänemark}|pw} Geisteslebens gelte. Die Zeit ist
                    längst vorüber. Ich habe mich von allem äusseren Leben zurückgezogen um zu
                    arbeiten, und betrachte es als meine einzige Aufgabe, der nordischen\oindex{Skandinavien@\textbf{Skandinavien}|pw} Jugend gegenüber, sie mir vom Halse zu halten.
                    Ich überlasse anderen die Freuden des öffentlichen Vortrags und des
                    Beifallklatschens.\pend
           \pstart
           Ihre Frau Gemahlin\pwindex{Schnitzler, Olga 17.01.1882 – 13.01.1970@\textsc{Schnitzler, Olga} (17.01.1882 – 13.01.1970), \emph{Schauspielerin, Sängerin}|pwv} war mir
                    in Wien\oindex{Wien@\textbf{Wien}|pw}{ }1913 eine liebe Wirthin. Ich sage ihr meinen Dank; hoffe, dass Sie
                    Freude an den Kindern\pwindex{Schnitzler, Lili 13.09.1909 – 26.07.1928@\textsc{Schnitzler, Lili} (13.09.1909 – 26.07.1928)|pwv}\pwindex{Schnitzler, Heinrich 09.08.1902 – 12.07.1982@\textsc{Schnitzler, Heinrich} (09.08.1902 – 12.07.1982), \emph{Regisseur, Schauspieler}|pwv} haben. Ich habe ein paar kleine Enkel\pwindex{Philipp, Gerda 27.11.1907 – 1968@\textsc{Philipp, Gerda} (27.11.1907 – 1968)|pwv}\pwindex{Philipp, Georg 1912-06-21 – 1995-11-08@\textsc{Philipp, Georg} (1912-06-21 – 1995-11-08), \emph{Schauspieler}|pwv}, 10\pwindex{Philipp, Gerda 27.11.1907 – 1968@\textsc{Philipp, Gerda} (27.11.1907 – 1968)|pwv} und 5 Jahre\pwindex{Philipp, Georg 1912-06-21 – 1995-11-08@\textsc{Philipp, Georg} (1912-06-21 – 1995-11-08), \emph{Schauspieler}|pwv}, die selten hier sind, aber sehr lieb.\pend
           \pstart Ihr Freund \spacefill\mbox{Georg Brandes}\pend{}\endnumbering\briefempfaengerindex{Schnitzler, Arthur@\textsc{Schnitzler, Arthur}!zzzBrandes, Georg@\emph{von Georg Brandes}!1918-09-181@{18. 9. 1918}|)be}\mylabel{h}\end{ledgroupsized}  \newcommand{\dateiname}{L02303}\newcommand{\titel}{Georg Brandes an Arthur Schnitzler, 18. 9. 1918}\newcommand{\editorInnen}{Martin Anton Müller und Gerd-Hermann Susen}
            \footnotesize
\begin{ledgroupsized}[t]{11.5cm}
\doendnotes{C}
\end{ledgroupsized}
         %% latex-leseansicht-abspann.tex
%% Abspann für die Leseansicht.
%% Der Schalter \ifkorrekturansicht ist bereits durch den Vorspann gesetzt.

%% latex-abspann.tex
%% Gemeinsamer Abspann für Korrekturansicht und Leseansicht.
%% Setzt den Schalter \ifkorrekturansicht voraus (gesetzt in den
%% einbindenden Dateien latex-korrekturansicht-abspann.tex bzw.
%% latex-leseansicht-abspann.tex).
%% ---------------------------------------------------------------

\normalsize

% Das esempio-Environment wird nur in der Leseansicht benötigt
\ifkorrekturansicht\else
\newenvironment{esempio}[3]%
{
    \vspace{1.5ex}
    \rlap{\underline{#1}}
    \par
    \setlength{\parindent}{0cm}
    \nopagebreak
    \leftskip=#2cm
    \rightskip=#3cm
}
{
    \par
}
\fi

\doendnotes{C}
\bigskip
\vfill

\clearpage

\footnotesize

\ifkorrekturansicht
  \lohead{\textsc{register}}
\fi

% theindex-Environment neu definieren ohne reledmac
\makeatletter
\renewenvironment{theindex}{%
  \ifkorrekturansicht
    \section*{\indexname}%
  \else
    \subsubsection*{Index der erwähnten Entitäten}%
  \fi
  \setlength{\parindent}{0pt}%
  \setlength{\parskip}{0pt plus 0.3pt}%
  \let\item\@idxitem
}{%
  \ifkorrekturansicht\clearpage\fi
}
\makeatother

\IfFileExists{\jobname-pw.ind}{\input{\jobname-pw.ind}}{}

% Quellenangabe nur in der Leseansicht
\ifkorrekturansicht\else
% Fallback-Definitionen, falls die .tex-Datei \titel etc. nicht gesetzt hat
\providecommand{\titel}{}
\providecommand{\editorInnen}{}
\providecommand{\dateiname}{\jobname}

\vspace{3cm}

\vfill

\footnotesize
\textsc{Quelle}: \titel. Herausgegeben von {\editorInnen}. In: \emph{Arthur Schnitzler: Briefwechsel mit Autorinnen und Autoren}.
 Digitale Edition, https://schnitzler-briefe.acdh.oeaw.ac.at/{\dateiname}.html (Stand \today)
\fi

\end{document}


      