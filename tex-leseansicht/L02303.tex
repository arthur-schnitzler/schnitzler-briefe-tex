%% latex-korrekturansicht-vorspann.tex
%% Vorspann für die Korrekturansicht.
%% Lädt die gemeinsame Datei latex-vorspann.tex mit gesetztem Schalter.

\newif\ifkorrekturansicht
\korrekturansichttrue

\input{../tex-inputs/latex-vorspann}


\section[Georg Brandes an Arthur Schnitzler, 18. 9. 1918]{L02303 Georg Brandes an Arthur Schnitzler, 18. 9. 1918}
\nopagebreak\mylabel{L02303v}
\rehead{ }\normalsize\beginnumbering\briefempfaengerindex{Schnitzler, Arthur@\textsc{Schnitzler, Arthur}!zzzBrandes, Georg@\emph{von Georg Brandes}!1918-09-181@{18. 9. 1918}|(be}
\toendnotes[C]{\smallbreak\pagebreak[2]}\Standort{CUL, Schnitzler, B 17.}
\physDesc{Brief, 1 Blatt, 4 Seiten, 3324 Zeichen
\newline{}Handschrift: schwarze Tinte, lateinische Kurrent
\newline{}Schnitzler: 1) mit Bleistift beschriftet: »\textsc{Brandes}«  2) mit rotem Buntstift vereinzelte Unterstreichungen
\newline{}Ordnung: von unbekannter Hand nummeriert: »49« }
\buchAbdrucke{\weitereDrucke{Georg Brandes, Arthur Schnitzler: \emph{Ein Briefwechsel}. Bern: \emph{Francke} 1956, S. 124–125.} }\toendnotes[C]{\smallbreak}
\pstart
           \raggedleft{}{\pb}Kopenhagen\oindex{Kopenhagen@\textbf{Kopenhagen}, \emph{P.PPLC}|pw}{ }18 Sept 18\pend
           
\pstart{}Lieber verehrter Freund\pend\vspace{0.5em}
\pstart
           Mein Trieb war, augenblicklich einen so herzlichen Brief zu beantworten. Es war mir
               nicht möglich Zeit zu finden. Endlich nach anderthalb Jahren Arbeit sind die zwei Bände über Cäsar\pwindex{Caesar, Gaius Iulius 13.7.100? v. Chr. – 15.3.44 v. Chr.@\textsc{Caesar, Gaius Iulius} (13.7.100? v. Chr. – 15.3.44 v. Chr.), \emph{Politiker/Politikerin, Kaiser/Kaiserin, Heerführer/Heerführerin}|pw}\pwindex{Gaius Julius Cæsar@\emph{Gaius Julius Cæsar}|pwv}, der erste von 500, der andere von 600 Seiten grossen Formats, vollendet, und
               ich kann aufatmen.\pend
           
\pstart
           Erinnern Sie sich einmal vor Jahren, es war eben an Ihrem Geburtstag und Sie waren so
               freundlich gewesen, mich zu Tisch einzuladen; ich sagte: Sie sind gerade 20 Jahre
               jünger als ich; Sie antworteten: Und wir beabsichtigen auch ferner diese Distanz von
               einander zu halten. – So ist es gegangen, die Distanz ist geblieben, eine seelische
               Entfernung nicht eingetreten.\pend
           
\pstart
           Ich habe Sie nie vergessen, mich immer mit Ihnen beschäftigt, und auch Sie gedenken
               freundlich meiner, obwohl wir uns nur selten sahen.\pend
           
\pstart
           {\pb}Hier hat man in der \label{K_L02303-1v}\edtext{vorigen Saison}{\lemma{\textnormal{\emph{vorigen Saison}}}\Cendnote{\textnormal{Die Premiere von \emph{Erkendelsens
                     Time}\pwindex{Erkendelsens Time@\emph{Erkendelsens Time}|pwk} (\emph{Stunde der Erkennens}\pwindex{Stunde des Erkennens@\emph{Stunde des Erkennens}|pwk}) und \emph{Den store Scene}\pwindex{Den store Scene@\emph{Den store Scene}|pwk} (\emph{Große Szene}\pwindex{Grosse Szene@\emph{Große Szene}|pwk}) fand am 22. 3. 1918 am \emph{Det Kongelige Teater}\orgindex{Det Kongelige Teater@Det Kongelige Teater|pwk} statt.}}}\label{K_L02303-1} versucht, zwei Ihrer Stücke\pwindex{Stunde des Erkennens@\emph{Stunde des Erkennens}|pwv}\pwindex{Grosse Szene@\emph{Große Szene}|pwv} zu spielen,
               ich sah das eine, das Stück\pwindex{Grosse Szene@\emph{Große Szene}|pwv}
               über den Schauspieler, das sehr gefiel und nicht übel gegeben wurde. Jetzt wird
               wieder \label{K_L02303-2v}\edtext{etwas\pwindex{Literatur@\emph{Literatur}|pwv}}{\lemma{\textnormal{\emph{etwas}}}\Cendnote{\textnormal{\emph{Literatur}\pwindex{Literatur@\emph{Literatur}|pwk} wurde gemeinsam mit
                  \emph{Große Szene}\pwindex{Grosse Szene@\emph{Große Szene}|pwk} am 30. 9. 1918 als Gastspiel  
                  gegeben.}}}\label{K_L02303-2} von Ihnen, an einem anderen Theater, gespielt werden. Man hat hier
               leider immer weniger Kunstverstand; doch werden Sie geschätzt; nur sagt unsere
               unglaublich idiotische Kritik, Sie seien von Peter
                  Nansen\pwindex{Nansen, Peter 20.01.1861 – 31.07.1918@\textsc{Nansen, Peter} (20.01.1861 – 31.07.1918), \emph{Schriftsteller/Schriftstellerin, Journalist/Journalistin, Verleger/Verlegerin}|pw}{ }\uline{beeinflusst}. Ich glaube, Sie schrieben, bevor Sie
               seinen Namen gehört hatten. Und wo wäre die Ähnlichkeit!\pend
           
\pstart
           Nansens\pwindex{Nansen, Peter 20.01.1861 – 31.07.1918@\textsc{Nansen, Peter} (20.01.1861 – 31.07.1918), \emph{Schriftsteller/Schriftstellerin, Journalist/Journalistin, Verleger/Verlegerin}|pw} Tod war die Veranlassung Ihres guten
               Briefes. Dieser Tod hat mich tief ergriffen, so tief, dass es mir ist, als lebte er
               noch. Mir gegenüber ein sonderbarer Mensch. Dreissig Jahre hat er mich gekannt, und
                  \introOben{}in 25\introOben{} mir nie näher getreten. In seinen beiden Ehen war
               ich nie in sein Haus geladen, ich habe nicht einmal in einem flüchtigen Besuch je
               seine Wohnung {\pb}gesehen. Dann
               plötzlich in den fünf–sechs letzten Lebensjahren schloss er sich mit einer Innigkeit
               an mich, dass ich eine Art Hauptperson in seiner Gedankenwelt wurde, er widmete mir
               öffentlich seine Bücher\pwindex{Brueder Menthe@\emph{Die Brüder Menthe}|pwv},
               schrieb öfters über mich – natürlich meistens irrthümlich – aber mit dem besten
               Willen.\pend
           
\pstart
           Es war sehr, sehr traurig, die Abnahme seiner Kräfte zu verfolgen. Man litt fast mit
               ihm.\pend
           
\pstart
           Und doch ertrinkt dies Einzelne in dem allgemeinen Jammer der Menschheit. Glauben Sie
               nicht \strikeout{ab} auch, dass diese Kugel, Erde genannt, in dem
               Weltall den Record bestialischer Stupidität geschlagen hat? Es scheint mir unmöglich,
               dass ein anderer Globus von dümmeren und ekelhafteren Wesen bewohnt sein kann.\pend
           
\pstart
           Ab und zu werde ich von Oesterreichern\oindex{Oesterreich@\textbf{Österreich}, \emph{A.PCLI}|pw}
               aufgesucht, aber es ist zuletzt unerträglich, von seinen Landsleuten als
               Gebrauchsgegenstand, {\pb}von Fremden
               als Sehenswürdigkeit aufgesucht zu werden. Wenn vierzig Briefe und 12 Bände pr. Tag
                  \strikeout{kommen} mit der Post \introOben{}kommen,\introOben{} und wenn es alle drei Minuten an der Türe schellt, so ist es
               unmöglich, nicht zu wüthen.\pend
           
\pstart
           Sie irren sich völlig, wenn Sie glauben, dass ich hier für einen Vertreter dänischen\oindex{Daenemark@\textbf{Dänemark}, \emph{A.PCLI}|pw} Geisteslebens gelte. Die Zeit ist
               längst vorüber. Ich habe mich von allem äusseren Leben zurückgezogen um zu arbeiten,
               und betrachte es als meine einzige Aufgabe, der nordischen\oindex{Skandinavien@\textbf{Skandinavien}, \emph{Region}|pw} Jugend gegenüber, sie mir vom Halse zu halten. Ich überlasse
               anderen die Freuden des öffentlichen Vortrags und des Beifallklatschens.\pend
           
\pstart
           Ihre Frau Gemahlin\pwindex{Schnitzler, Olga 17.01.1882 – 13.01.1970@\textsc{Schnitzler, Olga} (17.01.1882 – 13.01.1970), \emph{Schauspieler/Schauspielerin, Sänger/Sängerin}|pwv} war mir in
                  Wien\oindex{Wien@\textbf{Wien}, \emph{A.ADM2}|pw}{ }1913 eine liebe Wirthin. Ich sage ihr meinen Dank; hoffe, dass Sie
               Freude an den Kindern\pwindex{Cappellini, Lili 13.09.1909 – 26.07.1928@\textsc{Cappellini, Lili} (13.09.1909 – 26.07.1928)|pwv}\pwindex{Schnitzler, Heinrich 09.08.1902 – 12.07.1982@\textsc{Schnitzler, Heinrich} (09.08.1902 – 12.07.1982), \emph{Regisseur/Regisseurin, Schauspieler/Schauspielerin}|pwv} haben. Ich habe ein paar kleine Enkel\pwindex{Philipp, Gerda 27.11.1907 – 1968@\textsc{Philipp, Gerda} (27.11.1907 – 1968)|pwv}\pwindex{Philipp, Georg 1912-06-21 – 1995-11-08@\textsc{Philipp, Georg} (1912-06-21 – 1995-11-08), \emph{Schauspieler/Schauspielerin}|pwv}, 10\pwindex{Philipp, Gerda 27.11.1907 – 1968@\textsc{Philipp, Gerda} (27.11.1907 – 1968)|pwv} und 5 Jahre\pwindex{Philipp, Georg 1912-06-21 – 1995-11-08@\textsc{Philipp, Georg} (1912-06-21 – 1995-11-08), \emph{Schauspieler/Schauspielerin}|pwv}, die selten hier sind, aber sehr lieb.\pend
           \pstart Ihr Freund \spacefill\mbox{Georg Brandes}\pend{}\selectlanguage{ngerman}\endnumbering\briefempfaengerindex{Schnitzler, Arthur@\textsc{Schnitzler, Arthur}!zzzBrandes, Georg@\emph{von Georg Brandes}!1918-09-181@{18. 9. 1918}|)be}\mylabel{L02303h}  \normalsize

\doendnotes{C}
\bigskip
\vfill

\clearpage

\footnotesize

\lohead{\textsc{register}}

% Definiere theindex-Environment komplett neu ohne reledmac
\makeatletter
\renewenvironment{theindex}{%
  \section*{\indexname}%
  \setlength{\parindent}{0pt}%
  \setlength{\parskip}{0pt plus 0.3pt}%
  \let\item\@idxitem
}{%
  \clearpage
}
\makeatother

\IfFileExists{\jobname-pw.ind}{\input{\jobname-pw.ind}}{}

\end{document}

      