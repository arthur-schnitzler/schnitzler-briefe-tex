%% latex-korrekturansicht-vorspann.tex
%% Vorspann für die Korrekturansicht.
%% Lädt die gemeinsame Datei latex-vorspann.tex mit gesetztem Schalter.

\newif\ifkorrekturansicht
\korrekturansichttrue

\input{../tex-inputs/latex-vorspann}


\section[Elsa Plessner an Arthur Schnitzler, 7. 8. {[}1897{]}]{L03697 Elsa Plessner an Arthur Schnitzler, 7. 8. {[}1897{]}}
\nopagebreak\mylabel{L03697v}
\rehead{ }\normalsize\beginnumbering\briefempfaengerindex{Schnitzler, Arthur@\textsc{Schnitzler, Arthur}!zzzPlessner, Elsa@\emph{von Elsa Plessner}!1897-08-072@{7. 8. {[}1897{]}}|(be}
\toendnotes[C]{\smallbreak\pagebreak[2]}\Standort{DLA, A:Schnitzler, 85.1.4198.}
\physDesc{Brief, 2 Blätter, 8 Seiten, 3496 Zeichen
\newline{}Handschrift: , lateinische Kurrent}
\buchAbdrucke{\weitereDrucke{Hermann Bahr, Arthur Schnitzler: \emph{Briefwechsel, Aufzeichnungen, Dokumente (1891–1931)}. Göttingen: \emph{Wallstein} 2018, S. 151–152.} }\toendnotes[C]{\smallbreak}
\pstart
           {\pb}Wien-Sievring, Fröschlgasse 6\oindex{Froeschelgasse 6@\textbf{Fröschelgasse 6}, \emph{Wohngebäude (K.WHS)}|pw} den 7. August\pend
           
\pstart\center{}Hochverehrter Herr Doctor!\pend\vspace{0.5em}
\pstart
           In ganz unsagbarer Aufregung richte ich diese Zeilen an Sie, über die Sie ja
               vielleicht lachen werden, aber ich gestehe Ihnen, dass ich gottesjämmerlich geweint
               habe und noch weine. Nur um ein Verfahren vor Ihnen zu rechtfertigen, das ich
               genöthigt bin, gezwungen sogar, anzuwenden. Das {\pb}Erscheinen der heutigen
                  »Zeit\pwindex{Zeit. Wiener Wochenschrift@\emph{Die Zeit. Wiener Wochenschrift}|pw}« brachte mir statt der so lange
               erhofften Freude einen so intensiven Verdruss, dass ich noch jetzt vor Wuth am ganzen
               Leibe zittere. Es hat nämlich jemand, da \uline{leider}, \uline{leider} Herr Bahr\pwindex{Bahr, Hermann 19.07.1863 – 15.01.1934@\textsc{Bahr, Hermann} (19.07.1863 – 15.01.1934), \emph{Schriftsteller/Schriftstellerin, Kritiker/Kritikerin}|pw} abwesend ist, meiner Arbeit\pwindex{glaeserne Kaefig. Eine Parabel@\emph{Der gläserne Käfig. Eine Parabel}|pwv} den bösen Dienst mehr als geschmackloser Correcturen geleistet und
               zwar nachträglich, das heißt, \uline{nachdem} der
               Correcturbogen von mir als \uline{endgiltige} Form der Arbeit\pwindex{glaeserne Kaefig. Eine Parabel@\emph{Der gläserne Käfig. Eine Parabel}|pwv} abgesandt war. Nicht
               nur, dass ich die Berech{\pb}tigung zu dieser Handlungsweise jedem noch so
               vorsichtigen Schriftleiter bestreite, bin ich außer mir darüber mit meinem
               künstlerisch noch unbescholtenen Namen mehr als andere, als unvermeidliche eigene
               Geschmacklosigkeit decken zu müssen, wie unter anderm die Zerstörung einer meiner
               besten st\substVorne{}\textsuperscript{i}\substDazwischen{}y\substHinten{}l\substVorne{}\textsuperscript{y}\substDazwischen{}i\substHinten{}stischen Wendungen bezüglich der »\label{K_L03697-1v}\edtext{grausamen Fülle}{\lemma{\textnormal{\emph{grausamen Fülle}}}\Cendnote{\textnormal{Im Erstdruck\pwindex{glaeserne Kaefig. Eine Parabel@\emph{Der gläserne Käfig. Eine Parabel}|pwkv} (\emph{Die Zeit}\orgindex{Zeit. Wiener Wochenschrift@Die Zeit. Wiener Wochenschrift|pwk}, Bd. 12, Nr. 149, 7. 8. 1897, S. 95–96) lautet der Satz: »Ermattet von dem
                     ziellosen Wünschen und der grausamen Fülle ruhte sie nun unbeweglich mit
                     geschlossenen Augen – wie schlafend.« Für die Erstausgabe (\emph{Der gläserne Käfig. Skizzen und Novellen}\pwindex{glaeserne Kaefig. Skizzen und Novellen@\emph{Der gläserne Käfig. Skizzen und Novellen}|pwk}.
                        Wien\oindex{Wien@\textbf{Wien}, \emph{A.ADM2}|pwk}, Leipzig\oindex{Leipzig@\textbf{Leipzig}, \emph{P.PPLA3}|pwk}: \emph{Leopold Weiss}\orgindex{Leopold Weiss@Leopold Weiss|pwk}{ }1901) wurde er nicht geändert.}}}\label{K_L03697-1}«. Aber insbesondere das künstlerisch
               geradezu unsühnbare Verbrechen, meine Arbeit auf eigene Hand und \uline{ohne mein Wissen} »\label{K_L03697-2v}\edtext{eine
                  Parabel}{\lemma{\textnormal{\emph{eine
                  Parabel}}}\Cendnote{\textnormal{Nicht in der Erstausgabe\pwindex{glaeserne Kaefig. Skizzen und Novellen@\emph{Der gläserne Käfig. Skizzen und Novellen}|pwkv}.}}}\label{K_L03697-2}« zu nennen, wo ich
               mit {\pb}\uline{wohlerwogener Absicht} überhaupt keine Bezeichnung
               hingesetzt habe, meine Arbeit aus einer naiv-bedeutungsvollen Sphäre in die einer
               lehrhaften zu schubsen – – \textcolor{gray}{×} was
               kann ich dazu sagen? Ich habe keine andere Waffe als die, morgen Sonntag mit dem
               Frühesten einen Rundgang durch alle wichtigen Kaffeehäuser, Pucher\oindex{Cafe Pucher@\textbf{Café Pucher}, \emph{Kaffeehaus (K.KAF)}|pw}, Scheidl\oindex{Cafe Scheidl@\textbf{Café Scheidl}, \emph{Kaffeehaus (K.KAF)}|pw} u. s. w. zu
               thun, und dortselbst die mir aufoctroirten Correcturen in den aufliegenden Nummern
               der »Zeit\orgindex{Zeit. Wiener Wochenschrift@Die Zeit. Wiener Wochenschrift|pw}« auf eigene Hand mit Blaustift
               auszumerzen und dazu {\pb}ganz ehrlich und offen meinen Namen zu
               unterschreiben. Ich will eben mit der Redaction der »Zeit\orgindex{Zeit. Wiener Wochenschrift@Die Zeit. Wiener Wochenschrift|pw}« \uline{keine} Differenz haben wo ich
               eigentlich dem Himmel d. h. Herrn Bahr\pwindex{Bahr, Hermann 19.07.1863 – 15.01.1934@\textsc{Bahr, Hermann} (19.07.1863 – 15.01.1934), \emph{Schriftsteller/Schriftstellerin, Kritiker/Kritikerin}|pw} und
               Ihnen so herzlich und bestens für die Aufnahme meiner Arbeit danke. Wenn es irgendwie
               Unannehmlichkeiten geben sollte, was ich kaum glaube, da ja für andere eine Lappalie,
               was mir bei meinem Debut eine Staatsaction ist – Sie werden mich verstehen und
               entschuldigen, wenn nicht rechtfertigen.\pend
           
\pstart
           {\pb}Ich wünsche gar \uline{nicht} zu wissen, wer sich – in
               der besten Absicht gegen mich vielleicht – so unliebsam meiner Arbeit angenommen hat,
               dass ich an dem Erscheinen derselben so gar keine Freude mehr habe. Durch die
               Redaction selbst ist ja keine Redressur möglich, deshalb – so weit es geht – versuche
               ich auf eigene Rechnung, was leider ziemlich wenig helfen wird, da ich nicht die
               ganze Abonnentenliste der »Zeit\orgindex{Zeit. Wiener Wochenschrift@Die Zeit. Wiener Wochenschrift|pw}« damit behelligen
               kann. Allein ich bitte Sie, lieber, guter Herr Doctor so herzlich ich kann, dafür zu
               sorgen, dass die literarischen Kreise, an deren Urtheil {\pb}mir ja
               hauptsächlich liegt – ein wenig von der Vergewaltigung erfahren, die meiner
               literarischen Ehre angethan wurde! Ich bitte Sie, lieber guter Herr Doctor vielmals
               um diese Gefälligkeit, soweit sie natürlich Ihnen nicht unbequem ist – und wenn Sie
               mich ein bisschen lieb haben und mir beistehen und helfen wollen wie schon so oft, so
               werden Sie mir diesen \uline{innigen Wunsch} erfüllen.
               {\pb}Ich bitte nochmals! Herrn Bahr\pwindex{Bahr, Hermann 19.07.1863 – 15.01.1934@\textsc{Bahr, Hermann} (19.07.1863 – 15.01.1934), \emph{Schriftsteller/Schriftstellerin, Kritiker/Kritikerin}|pw} will ich
               für Jetzt damit noch nicht kommen aber es folgt schon noch, und ich bin \uline{überzeugt}, dass es ihm nicht gleichgiltig sein wird,
               wie man mich behandelt hat. Nicht wahr, Sie missverstehen mich nicht und sind nicht
               sehr böse auf mich? Sie sind doch so gut!\pend
           
\pstart
           Viele viele dankbare Grüße{\\[\baselineskip]}\spacefill\mbox{ElsaPlessner}\pend
           \leftskip=0em{}\selectlanguage{ngerman}\endnumbering\briefempfaengerindex{Schnitzler, Arthur@\textsc{Schnitzler, Arthur}!zzzPlessner, Elsa@\emph{von Elsa Plessner}!1897-08-072@{7. 8. {[}1897{]}}|)be}\mylabel{L03697h}
\begin{anhang}
\end{anhang}\normalsize

\doendnotes{C}
\bigskip
\vfill

\clearpage

\footnotesize

\lohead{\textsc{register}}

% Definiere theindex-Environment komplett neu ohne reledmac
\makeatletter
\renewenvironment{theindex}{%
  \section*{\indexname}%
  \setlength{\parindent}{0pt}%
  \setlength{\parskip}{0pt plus 0.3pt}%
  \let\item\@idxitem
}{%
  \clearpage
}
\makeatother

\IfFileExists{\jobname-pw.ind}{\input{\jobname-pw.ind}}{}

\end{document}

      