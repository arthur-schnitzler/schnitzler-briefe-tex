%% latex-leseansicht-vorspann.tex
%% Vorspann für die Leseansicht.
%% Lädt die gemeinsame Datei latex-vorspann.tex mit nicht gesetztem Schalter.

\newif\ifkorrekturansicht
\korrekturansichtfalse

\input{../tex-inputs/latex-vorspann}


\section[Elsa Plessner an Arthur Schnitzler, 7. 8. [1897]]{L03697 Elsa Plessner an Arthur Schnitzler, 7. 8. [1897]}
\nopagebreak\mylabel{L03697v}
\rehead{ }\normalsize\beginnumbering\briefempfaengerindex{Schnitzler, Arthur@\textsc{Schnitzler, Arthur}!zzzPlessner, Elsa@\emph{von Elsa Plessner}!1897-08-072@{7. 8. [1897]}|(be}
\toendnotes[C]{\smallbreak\pagebreak[2]}
\correspDesc{Versand  durch Elsa Plessner am 7. 8. [1897] in Wien
\newline{}Erhalt  durch Arthur Schnitzler im Zeitraum [7. 8. 1897
                  – 12. 8. 1897?] in Wien}\toendnotes[C]{\smallbreak}
\Standort{DLA, A:Schnitzler, 85.1.4198.}
\physDesc{Brief, 2 Blätter, 8 Seiten, 3495 Zeichen
\newline{}Handschrift: schwarze Tinte, lateinische Kurrent}
\buchAbdrucke{\weitereDrucke{Hermann Bahr, Arthur Schnitzler: \emph{Briefwechsel, Aufzeichnungen, Dokumente (1891–1931)}. Herausgegeben von Kurt Ifkovits und Martin Anton Müller. Göttingen: \emph{Wallstein} 2018, S. 151–152.} }\toendnotes[C]{\smallbreak}
\pstart
           \raggedleft{}{\pb}Wien-Sievring, Fröschlgasse 6\oindex{Wien@\textbf{Wien}!XIX., Döbling@\textbf{XIX., Döbling}!Fröschelgasse 6@\textbf{Fröschelgasse 6}, \emph{Wohngebäude}|pw}{\\}den 7. August\pend
           
\pstart\center{}Hochverehrter Herr Doctor!\pend\vspace{0.5em}
\pstart
           In ganz unsagbarer Aufregung richte ich diese Zeilen an Sie, über die Sie ja
               vielleicht lachen werden, aber ich gestehe Ihnen, dass ich gottesjämmerlich geweint
               habe und noch weine. Nur um ein Verfahren vor Ihnen zu rechtfertigen, das ich
               genöthigt bin, gezwungen sogar, anzuwenden. Das {\pb}Erscheinen der heutigen »Zeit\pwindex{Zeit. Wiener Wochenschrift@\emph{Die Zeit. Wiener Wochenschrift}|pw}« brachte mir
               statt der so lange erhofften Freude einen so intensiven Verdruss, dass ich noch jetzt
               vor Wuth am ganzen Leibe zittere. Es hat nämlich jemand, da \uline{leider}, \uline{leider} Herr Bahr\pwindex{Bahr, Hermann 19.\,7.\,1863 Linz – 15.\,1.\,1934 München@\textsc{Bahr, Hermann} (19.\,7.\,1863 Linz – 15.\,1.\,1934 München), \emph{Schriftsteller, Kritiker}|pw} abwesend ist, meiner Arbeit\pwindex{Plessner, Elsa 22.\,8.\,1875 Wien – 7.\,5.\,1932 Alicante@\textsc{Plessner, Elsa} (22.\,8.\,1875 Wien – 7.\,5.\,1932 Alicante), \emph{Schriftstellerin}!gläserne Käfig. Eine Parabel@\strich\emph{Der gläserne Käfig. Eine Parabel}|pwv} den bösen Dienst mehr als geschmackloser Correcturen
               geleistet und zwar nachträglich, das heißt, \uline{nachdem}
               der Correcturbogen von mir als \uline{endgiltige} Form der
                  Arbeit\pwindex{Plessner, Elsa 22.\,8.\,1875 Wien – 7.\,5.\,1932 Alicante@\textsc{Plessner, Elsa} (22.\,8.\,1875 Wien – 7.\,5.\,1932 Alicante), \emph{Schriftstellerin}!gläserne Käfig. Eine Parabel@\strich\emph{Der gläserne Käfig. Eine Parabel}|pwv} abgesandt war. Nicht
               nur, dass ich die Berech{\pb}tigung zu dieser
               Handlungsweise jedem noch so vorsichtigen Schriftleiter bestreite, bin ich außer mir
               darüber mit meinem künstlerisch noch unbescholtenen Namen mehr als andere, als
               unvermeidliche eigene Geschmacklosigkeit decken zu müssen, wie unter anderm die
               Zerstörung einer meiner besten st\substVorne{}\textsuperscript{i}\substDazwischen{}y\substHinten{}l\substVorne{}\textsuperscript{y}\substDazwischen{}i\substHinten{}stischen Wendungen bezüglich der »\label{K_L03697-1v}\edtext{grausamen Fülle}{\lemma{\textnormal{\emph{grausamen Fülle}}}\Cendnote{\textnormal{Im Erstdruck\pwindex{Plessner, Elsa 22.\,8.\,1875 Wien – 7.\,5.\,1932 Alicante@\textsc{Plessner, Elsa} (22.\,8.\,1875 Wien – 7.\,5.\,1932 Alicante), \emph{Schriftstellerin}!gläserne Käfig. Eine Parabel@\strich\emph{Der gläserne Käfig. Eine Parabel}|pwkv} (\emph{Die Zeit}\pwindex{Zeit. Wiener Wochenschrift@\emph{Die Zeit. Wiener Wochenschrift}|pwk}, Bd. 12, Nr. 149,
                        7. 8. 1897, S. 95–96) lautet der Satz:
                     »Ermattet von dem ziellosen Wünschen und der grausamen Fülle ruhte sie
                     nun unbeweglich mit geschlossenen Augen – wie schlafend.« Für die spätere
                  Erstausgabe des Novellenbandes wurde er nicht geändert (\emph{Der gläserne Käfig. Skizzen und Novellen}\pwindex{Plessner, Elsa 22.\,8.\,1875 Wien – 7.\,5.\,1932 Alicante@\textsc{Plessner, Elsa} (22.\,8.\,1875 Wien – 7.\,5.\,1932 Alicante), \emph{Schriftstellerin}!gläserne Käfig. Skizzen und Novellen@\strich\emph{Der gläserne Käfig. Skizzen und Novellen}|pwk}.
                        Wien\oindex{Wien@\textbf{Wien}, \emph{Verwaltungsgebiet}|pwk}, Leipzig\oindex{Leipzig@\textbf{Leipzig}, \emph{Hauptstadt}|pwk}: \emph{Leopold Weiss}\orgindex{Leopold Weiss@Leopold Weiss|pwk}{ }1901, S. 7).}}}\label{K_L03697-1}«. Aber
               insbesondere das künstlerisch geradezu unsühnbare Verbrechen, meine Arbeit auf eigene
               Hand und \uline{ohne mein Wissen} »\label{K_L03697-2v}\edtext{eine Parabel}{\lemma{\textnormal{\emph{eine Parabel}}}\Cendnote{\textnormal{Nicht in der Erstausgabe\pwindex{Plessner, Elsa 22.\,8.\,1875 Wien – 7.\,5.\,1932 Alicante@\textsc{Plessner, Elsa} (22.\,8.\,1875 Wien – 7.\,5.\,1932 Alicante), \emph{Schriftstellerin}!gläserne Käfig. Skizzen und Novellen@\strich\emph{Der gläserne Käfig. Skizzen und Novellen}|pwkv}.}}}\label{K_L03697-2}« zu nennen, wo ich mit {\pb}\uline{wohlerwogener Absicht} überhaupt keine Bezeichnung
               hingesetzt habe, meine Arbeit aus einer naiv-bedeutungsvollen Sphäre in die einer
               lehrhaften zu schubsen – – \textcolor{gray}{×} was
               kann ich dazu sagen? Ich habe keine andere Waffe als die, morgen, Sonntag mit dem
               Frühesten einen Rundgang durch alle wichtigen Kaffeehäuser, Pucher\oindex{Wien@\textbf{Wien}!I., Innere Stadt@\textbf{I., Innere Stadt}!Café Pucher@\textbf{Café Pucher}, \emph{Kaffeehaus}|pw}, Scheidl\oindex{Wien@\textbf{Wien}!I., Innere Stadt@\textbf{I., Innere Stadt}!Café Scheidl@\textbf{Café Scheidl}, \emph{Kaffeehaus}|pw} u. s. w. zu
               thun, und dortselbst die mir aufoctroirten Correcturen in den aufliegenden Nummern
               der »Zeit\pwindex{Zeit. Wiener Wochenschrift@\emph{Die Zeit. Wiener Wochenschrift}|pw}« auf eigene Hand mit Blaustift
               auszumerzen und dazu {\pb}ganz ehrlich und offen meinen
               Namen zu unterschreiben. Ich will eben mit der Redaction der »Zeit\orgindex{Zeit. Wiener Wochenschrift@Die Zeit. Wiener Wochenschrift|pw}« \uline{keine} Differenz haben
               wo ich eigentlich dem Himmel – d. h. Herrn Bahr\pwindex{Bahr, Hermann 19.\,7.\,1863 Linz – 15.\,1.\,1934 München@\textsc{Bahr, Hermann} (19.\,7.\,1863 Linz – 15.\,1.\,1934 München), \emph{Schriftsteller, Kritiker}|pw} und Ihnen so herzlich und bestens für die Aufnahme meiner Arbeit\pwindex{Plessner, Elsa 22.\,8.\,1875 Wien – 7.\,5.\,1932 Alicante@\textsc{Plessner, Elsa} (22.\,8.\,1875 Wien – 7.\,5.\,1932 Alicante), \emph{Schriftstellerin}!gläserne Käfig. Eine Parabel@\strich\emph{Der gläserne Käfig. Eine Parabel}|pwv} danke. Wenn es irgendwie
               Unannehmlichkeiten geben sollte, was ich kaum glaube, da ja für andere eine Lappalie,
               was mir bei meinem Debut eine Staatsaction ist – Sie werden mich verstehen und
               entschuldigen, wenn nicht rechtfertigen.\pend
           
\pstart
           {\pb}Ich wünsche gar \uline{nicht}
               zu wissen, wer sich – in der besten Absicht gegen mich vielleicht – so unliebsam
               meiner Arbeit angenommen hat, dass ich an dem Erscheinen derselben so gar keine
               Freude mehr habe. Durch die Redaction\orgindex{Zeit. Wiener Wochenschrift@Die Zeit. Wiener Wochenschrift|pwv} selbst ist ja keine Redressur möglich, deshalb
               – so weit es geht – versuche ich sie auf eigene Rechnung, was leider ziemlich wenig
               helfen wird, da ich nicht die ganze Abonnentenliste der »Zeit\orgindex{Zeit. Wiener Wochenschrift@Die Zeit. Wiener Wochenschrift|pw}« damit behelligen kann. Allein ich bitte Sie, lieber,
               guter Herr Doctor so herzlich ich kann, dafür zu sorgen, dass die literarischen
               Kreise, an deren Urtheil {\pb}mir ja hauptsächlich liegt –
               ein wenig von der Vergewaltigung erfahren, die meiner literarischen Ehre angethan
               wurde! Ich bitte Sie, lieber guter Herr Doctor vielmals um diese Gefälligkeit, soweit
               sie natürlich Ihnen nicht unbequem ist – und wenn Sie mich ein bisschen lieb haben
               und mir beistehen und helfen wollen wie schon so oft, so werden Sie mir diesen \uline{innigen Wunsch} erfüllen. {\pb}Ich bitte nochmals! Herrn Bahr\pwindex{Bahr, Hermann 19.\,7.\,1863 Linz – 15.\,1.\,1934 München@\textsc{Bahr, Hermann} (19.\,7.\,1863 Linz – 15.\,1.\,1934 München), \emph{Schriftsteller, Kritiker}|pw} will ich für
               Jetzt damit noch nicht kommen aber es folgt schon noch, und ich bin \uline{überzeugt}, dass es ihm nicht gleichgiltig sein wird,
               wie man mich behandelt hat. Nicht wahr, Sie missverstehen mich nicht und sind nicht
               sehr böse auf mich? Sie sind doch so gut!\pend
           
\pstart
           Viele viele dankbare Grüße{\\[\baselineskip]}\spacefill\mbox{ElsaPlessner}\pend
           \leftskip=0em{}\selectlanguage{ngerman}\endnumbering\briefempfaengerindex{Schnitzler, Arthur@\textsc{Schnitzler, Arthur}!zzzPlessner, Elsa@\emph{von Elsa Plessner}!1897-08-072@{7. 8. [1897]}|)be}\mylabel{L03697h}  \newcommand{\dateiname}{L03697}\newcommand{\titel}{Elsa Plessner an Arthur Schnitzler, 7. 8. [1897]}\newcommand{\editorInnen}{Kurt Ifkovits, Selma Jahnke und Martin Anton Müller}%% latex-leseansicht-abspann.tex
%% Abspann für die Leseansicht.
%% Der Schalter \ifkorrekturansicht ist bereits durch den Vorspann gesetzt.

%% latex-abspann.tex
%% Gemeinsamer Abspann für Korrekturansicht und Leseansicht.
%% Setzt den Schalter \ifkorrekturansicht voraus (gesetzt in den
%% einbindenden Dateien latex-korrekturansicht-abspann.tex bzw.
%% latex-leseansicht-abspann.tex).
%% ---------------------------------------------------------------

\normalsize

% Das esempio-Environment wird nur in der Leseansicht benötigt
\ifkorrekturansicht\else
\newenvironment{esempio}[3]%
{
    \vspace{1.5ex}
    \rlap{\underline{#1}}
    \par
    \setlength{\parindent}{0cm}
    \nopagebreak
    \leftskip=#2cm
    \rightskip=#3cm
}
{
    \par
}
\fi

\doendnotes{C}
\bigskip
\vfill

\clearpage

\footnotesize

\ifkorrekturansicht
  \lohead{\textsc{register}}
\fi

% theindex-Environment neu definieren ohne reledmac
\makeatletter
\renewenvironment{theindex}{%
  \ifkorrekturansicht
    \section*{\indexname}%
  \else
    \subsubsection*{Index der erwähnten Entitäten}%
  \fi
  \setlength{\parindent}{0pt}%
  \setlength{\parskip}{0pt plus 0.3pt}%
  \let\item\@idxitem
}{%
  \ifkorrekturansicht\clearpage\fi
}
\makeatother

\IfFileExists{\jobname-pw.ind}{\input{\jobname-pw.ind}}{}

% Quellenangabe nur in der Leseansicht
\ifkorrekturansicht\else
% Fallback-Definitionen, falls die .tex-Datei \titel etc. nicht gesetzt hat
\providecommand{\titel}{}
\providecommand{\editorInnen}{}
\providecommand{\dateiname}{\jobname}

\vspace{3cm}

\vfill

\footnotesize
\textsc{Quelle}: \titel. Herausgegeben von {\editorInnen}. In: \emph{Arthur Schnitzler: Briefwechsel mit Autorinnen und Autoren}.
 Digitale Edition, https://schnitzler-briefe.acdh.oeaw.ac.at/{\dateiname}.html (Stand \today)
\fi

\end{document}


