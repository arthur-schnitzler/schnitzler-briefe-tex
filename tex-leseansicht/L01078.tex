%% latex-leseansicht-vorspann.tex
%% Vorspann für die Leseansicht.
%% Lädt die gemeinsame Datei latex-vorspann.tex mit nicht gesetztem Schalter.

\newif\ifkorrekturansicht
\korrekturansichtfalse

\input{../tex-inputs/latex-vorspann}


         
         \newcommand{\erwaehntePersonen}{Personen: Hermann Bahr}
         \newcommand{\erwaehnteInstitutionen}{}
         \newcommand{\erwaehnteOrte}{Orte: Baden bei Wien, Wien, Österreich}
         \newcommand{\erwaehnteWerke}{Werke: Der Franzl. Fünf Bilder aus dem Leben eines guten Mannes, Der Schleier der Beatrice. Schauspiel in fünf Akten, Lieutenant Gustl. Novelle}
               \section[Arthur Schnitzler an Hermann Bahr, 18. 10. 1900]{ Arthur Schnitzler an Hermann Bahr, 18. 10. 1900}\nopagebreak\mylabel{v}\rehead{ }\begin{ledgroupsized}[t]{13cm}\normalsize\beginnumbering \toendnotes[C]{\smallbreak\pagebreak[2]} \Standort{TMW, HS AM 23338 Ba.}
\physDesc{Brief, 1 Blatt, 3 Seiten
\newline{}Handschrift: schwarze Tinte, deutsche Kurrent}\buchAbdrucke{\weitereDrucke{1) \emph{18. 10. 1900.} In: Arthur Schnitzler: \emph{The Letters of Arthur Schnitzler to Hermann Bahr}. Edited, annotated, and with an introduction, by Donald G.
                        Daviau. Chapel Hill: \emph{The University of North Carolina Press} 1978, S. 67 (University of North Carolina studies in the Germanic languages
                        and literatures, 89).} \weitereDrucke{2) Hermann Bahr, Arthur Schnitzler: \emph{Briefwechsel, Aufzeichnungen, Dokumente (1891–1931)}. Hg. Kurt Ifkovits und Martin Anton Müller. Göttingen: \emph{Wallstein} 2018, S. 192.} }\toendnotes[C]{\smallbreak}\pstart
           {\pb}\textsc{Baden b/W.\oindex{Baden bei Wien@\textbf{Baden bei Wien}|pw}}{ }18. 10. 900\pend
           \pstart
           lieber Hermann, deine Sympathie für die \textsc{Beatrice}\pwindex{Schnitzler, Arthur 15.05.1862 – 21.10.1931@\textsc{Schnitzler, Arthur} (15.05.1862 – 21.10.1931), \emph{Schriftsteller, Mediziner}!Schleier der Beatrice. Schauspiel in fuenf Akten1900-12-01@\strich\emph{Der Schleier der Beatrice. Schauspiel in fünf Akten} {[}1900-12-01{]}|pw} freut mich herzlich. Vielen Dank für die lieben Worte, in denen du mirs geſagt
               haſt. We{\geminationn} du erlaubſt, bring ich dir das \textsc{Mscrpt} der Novelle\pwindex{Schnitzler, Arthur 15.05.1862 – 21.10.1931@\textsc{Schnitzler, Arthur} (15.05.1862 – 21.10.1931), \emph{Schriftsteller, Mediziner}!Lieutenant Gustl. Novelle1900-12-25@\strich\emph{Lieutenant Gustl. Novelle} {[}1900-12-25{]}|pwv} nächſtens, vielleicht Mitte oder
               Ende nächſter Woche, bis ich wieder {\pb}in Wien\oindex{Wien@\textbf{Wien}|pw} bin. Mit beſonderem Vergnügen habe ich den Franzl\pwindex{Bahr, Hermann 19.07.1863 – 15.01.1934@\textsc{Bahr, Hermann} (19.07.1863 – 15.01.1934), \emph{Schriftsteller, Kritiker}!Franzl. Fuenf Bilder aus dem Leben eines guten Mannes1900@\strich\emph{Der Franzl. Fünf Bilder aus dem Leben eines guten Mannes} {[}1900{]}|pw} geleſen, beſonders den erſten, dritten und vierten Akt.
               Aber manchem werden gewiſs die beiden andern Akte mit dem \textcolor{gray}{vielen}
               Gemüth noch beſſer gefallen. Es iſt eine köſtliche Lebendigkeit in den Bauernburſchen
               wie in den Hofräthen, {\pb}der Himmel über dem ganzen echt oeſterreichiſch\oindex{Oesterreich@\textbf{Österreich}|pw}
               – nur die Geſtirne ko{\geminationm}en mir \substVorne{}\textsuperscript{ſozuſagen }{\allowbreak}\substDazwischen{}zu weilen\substHinten{} ein biſſel »Theater\textcolor{gray}{«} vor.\pend
           \pstart
           Auf Wiederſehen.{\\[\baselineskip]}Herzlichſt dein{\\[\baselineskip]}\spacefill\mbox{Arth Sch.}\pend
           \leftskip=0em{}\pstart
           18. 10. 900.\pend
           
         
         \endnumbering\mylabel{h}\end{ledgroupsized}  \newcommand{\dateiname}{L01078}\newcommand{\titel}{Arthur Schnitzler an Hermann Bahr, 18. 10. 1900}\newcommand{\editorInnen}{ Kurt Ifkovits,  Martin Anton Müller}%% latex-leseansicht-abspann.tex
%% Abspann für die Leseansicht.
%% Der Schalter \ifkorrekturansicht ist bereits durch den Vorspann gesetzt.

%% latex-abspann.tex
%% Gemeinsamer Abspann für Korrekturansicht und Leseansicht.
%% Setzt den Schalter \ifkorrekturansicht voraus (gesetzt in den
%% einbindenden Dateien latex-korrekturansicht-abspann.tex bzw.
%% latex-leseansicht-abspann.tex).
%% ---------------------------------------------------------------

\normalsize

% Das esempio-Environment wird nur in der Leseansicht benötigt
\ifkorrekturansicht\else
\newenvironment{esempio}[3]%
{
    \vspace{1.5ex}
    \rlap{\underline{#1}}
    \par
    \setlength{\parindent}{0cm}
    \nopagebreak
    \leftskip=#2cm
    \rightskip=#3cm
}
{
    \par
}
\fi

\doendnotes{C}
\bigskip
\vfill

\clearpage

\footnotesize

\ifkorrekturansicht
  \lohead{\textsc{register}}
\fi

% theindex-Environment neu definieren ohne reledmac
\makeatletter
\renewenvironment{theindex}{%
  \ifkorrekturansicht
    \section*{\indexname}%
  \else
    \subsubsection*{Index der erwähnten Entitäten}%
  \fi
  \setlength{\parindent}{0pt}%
  \setlength{\parskip}{0pt plus 0.3pt}%
  \let\item\@idxitem
}{%
  \ifkorrekturansicht\clearpage\fi
}
\makeatother

\IfFileExists{\jobname-pw.ind}{\input{\jobname-pw.ind}}{}

% Quellenangabe nur in der Leseansicht
\ifkorrekturansicht\else
% Fallback-Definitionen, falls die .tex-Datei \titel etc. nicht gesetzt hat
\providecommand{\titel}{}
\providecommand{\editorInnen}{}
\providecommand{\dateiname}{\jobname}

\vspace{3cm}

\vfill

\footnotesize
\textsc{Quelle}: \titel. Herausgegeben von {\editorInnen}. In: \emph{Arthur Schnitzler: Briefwechsel mit Autorinnen und Autoren}.
 Digitale Edition, https://schnitzler-briefe.acdh.oeaw.ac.at/{\dateiname}.html (Stand \today)
\fi

\end{document}


      