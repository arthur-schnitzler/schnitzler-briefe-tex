%% latex-korrekturansicht-vorspann.tex
%% Vorspann für die Korrekturansicht.
%% Lädt die gemeinsame Datei latex-vorspann.tex mit gesetztem Schalter.

\newif\ifkorrekturansicht
\korrekturansichttrue

\input{../tex-inputs/latex-vorspann}


\section[Arthur Schnitzler an Hermann Bahr, 18. 10. 1900]{L01078 Arthur Schnitzler an Hermann Bahr, 18. 10. 1900}
\nopagebreak\mylabel{L01078v}
\rehead{ }\normalsize\beginnumbering\briefempfaengerindex{Bahr, Hermann@\textsc{Bahr, Hermann}!zzzSchnitzler, Arthur@\emph{von Arthur Schnitzler}!1900-10-181@{18. 10. 1900}|(be}
\toendnotes[C]{\smallbreak\pagebreak[2]}\Standort{TMW, HS AM 23338 Ba.}
\physDesc{Brief, 1 Blatt, 3 Seiten, 719 Zeichen
\newline{}Handschrift: schwarze Tinte, deutsche Kurrent}
\buchAbdrucke{\weitereDrucke{1) Arthur Schnitzler: \emph{The Letters of Arthur Schnitzler to Hermann Bahr}. Chapel Hill: \emph{The University of North Carolina Press} 1978, S. 67.} \weitereDrucke{2) Hermann Bahr, Arthur Schnitzler: \emph{Briefwechsel, Aufzeichnungen, Dokumente (1891–1931)}. Göttingen: \emph{Wallstein} 2018, S. 192.} }\toendnotes[C]{\smallbreak}
\pstart
           {\pb}\textsc{Baden b/W.\oindex{Baden bei Wien@\textbf{Baden bei Wien}, \emph{P.PPLA3}|pw}}{ }18. 10. 900\pend
           \vspace{0.5em}
\pstart
           lieber Hermann, deine Sympathie für die \textsc{Beatrice}\pwindex{Schleier der Beatrice. Schauspiel in fuenf Akten@\emph{Der Schleier der Beatrice. Schauspiel in fünf Akten}|pw} freut mich herzlich. Vielen Dank für die lieben Worte, in denen du mirs geſagt
               haſt. We{\geminationn} du erlaubſt, bring ich dir das \textsc{Mscrpt} der Novelle\pwindex{Lieutenant Gustl. Novelle@\emph{Lieutenant Gustl. Novelle}|pwv} nächſtens, vielleicht Mitte oder
               Ende nächſter Woche, bis ich wieder {\pb}in Wien\oindex{Wien@\textbf{Wien}, \emph{A.ADM2}|pw} bin. Mit beſonderem Vergnügen habe ich den Franzl\pwindex{Franzl. Fuenf Bilder aus dem Leben eines guten Mannes@\emph{Der Franzl. Fünf Bilder aus dem Leben eines guten Mannes}|pw} geleſen, beſonders den erſten, dritten und
               vierten Akt. Aber manchem werden gewiſs die beiden andern Akte mit dem
                  \textcolor{gray}{vielen} Gemüth noch beſſer gefallen. Es iſt eine köſtliche
               Lebendigkeit in den Bauernburſchen wie in den Hofräthen, {\pb}der Himmel über dem
               ganzen echt oeſterreichiſch\oindex{Oesterreich@\textbf{Österreich}, \emph{A.PCLI}|pw} – nur die Geſtirne
                  ko{\geminationm}en mir \substVorne{}\textsuperscript{ſozuſagen }\substDazwischen{}zu weilen\substHinten{} ein biſſel »Theater\textcolor{gray}{«} vor.\pend
           
\pstart
           Auf Wiederſehen.{\\[\baselineskip]}Herzlichſt dein{\\[\baselineskip]}\spacefill\mbox{Arth Sch.}\pend
           \leftskip=0em{}
\pstart
           18. 10. 900.\pend
           \selectlanguage{ngerman}\endnumbering\briefempfaengerindex{Bahr, Hermann@\textsc{Bahr, Hermann}!zzzSchnitzler, Arthur@\emph{von Arthur Schnitzler}!1900-10-181@{18. 10. 1900}|)be}\mylabel{L01078h}  \normalsize

\doendnotes{C}
\bigskip
\vfill

\clearpage

\footnotesize

\lohead{\textsc{register}}

% Definiere theindex-Environment komplett neu ohne reledmac
\makeatletter
\renewenvironment{theindex}{%
  \section*{\indexname}%
  \setlength{\parindent}{0pt}%
  \setlength{\parskip}{0pt plus 0.3pt}%
  \let\item\@idxitem
}{%
  \clearpage
}
\makeatother

\IfFileExists{\jobname-pw.ind}{\input{\jobname-pw.ind}}{}

\end{document}

      