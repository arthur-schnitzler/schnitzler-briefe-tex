%% latex-leseansicht-vorspann.tex
%% Vorspann für die Leseansicht.
%% Lädt die gemeinsame Datei latex-vorspann.tex mit nicht gesetztem Schalter.

\newif\ifkorrekturansicht
\korrekturansichtfalse

\input{../tex-inputs/latex-vorspann}


         
         \renewcommand{\erwaehntePersonen}{Personen: Alfred Dreyfus, Paul Goldmann}
         \renewcommand{\erwaehnteInstitutionen}{Institutionen: Frankfurter Zeitung}
         \renewcommand{\erwaehnteOrte}{Orte: Belišće, Budapest, Engadin, Frankfurt am Main, Orahovica, Paris, Rennes, Schweiz, Slawonien, Wien, Österreich}
         \renewcommand{\erwaehnteWerke}{}
               \section[ Paul Goldmann an Arthur Schnitzler, 2. 7. 1899]{ Paul Goldmann an Arthur Schnitzler, 2. 7. 1899}\nopagebreak\mylabel{v}\rehead{ }\begin{ledgroupsized}[t]{13cm}\normalsize\beginnumbering \toendnotes[C]{\smallbreak\pagebreak[2]} \Standort{DLA, A:Schnitzler, HS.NZ85.1.3169.}
\physDesc{Brief, 1 Blatt, 3 Seiten, 1823 Zeichen
\newline{}Handschrift: blaue Tinte, deutsche Kurrent}\toendnotes[C]{\smallbreak}\pstart
           \noindent{}{\pb}\textcolor{gray}{\textbf{\textbf{Frankfurter Zeitung}}}\orgindex{Frankfurter Zeitung@Frankfurter Zeitung|pw}\hfill \textcolor{gray}{\textbf{\textbf{Frankfurt a. M.\oindex{Frankfurt am Main@\textbf{Frankfurt am Main}|pw},}}}{ }2. Juli \textcolor{gray}{\textbf{189}}9.\pend
           \pstart
           \textcolor{gray}{\textbf{und}}\pend
           \pstart
           \textcolor{gray}{\textbf{Handelsblatt.}}\pend
           \pstart
           \textcolor{gray}{\textbf{\textbf{Redaktion\orgindex{Frankfurter Zeitung@Frankfurter Zeitung|pwv}.}\footnote{\noindent{}\textcolor{gray}{\textbf{Für die Redaktion\orgindex{Frankfurter Zeitung@Frankfurter Zeitung|pwv} beſtimmte Briefe und Sendungen wolle man
                                 \so{nicht} an die Perſon eines Redakteurs,
                              ſondern ſtets \textbf{an die Redaktion der Frankfurter Zeitung\orgindex{Frankfurter Zeitung@Frankfurter Zeitung|pw}} adreſſiren. }}}}}\pend
           \pstart
           \textcolor{gray}{\textbf{Telegramm-Adreſſe:}}\pend
           \pstart
           \textcolor{gray}{\textbf{\textbf{Zeitung\orgindex{Frankfurter Zeitung@Frankfurter Zeitung|pwv}{ }Frankfurt Main\oindex{Frankfurt am Main@\textbf{Frankfurt am Main}|pw}.}}}\pend
           \pstart\center{}Mein lieber Freund,\pend\pstart
           Nun biſt Du wohl wieder \label{K_L02878-1v}\edtext{aus \textsc{Slavonien\oindex{Slawonien@\textbf{Slawonien}|pw}} zurück}{\lemma{\textnormal{\emph{aus Slavonien zurück}}}\Cendnote{\textnormal{Am 21. 6. 1899 reiste Schnitzler\pwindex{Schnitzler, Arthur 15.05.1862 – 21.10.1931@\textsc{Schnitzler, Arthur} (15.05.1862 – 21.10.1931), \emph{Schriftsteller, Mediziner}|pwk} nach Belišće\oindex{Belišće@\textbf{Belišće}|pwk}, blieb 2 Tage, dann weiter nach Orahovica\oindex{Orahovica@\textbf{Orahovica}|pwk}, wo er ebenfalls für zwei Tage blieb. Über Budapest\oindex{Budapest@\textbf{Budapest}|pwk} reiste er am 21. 6. 1899
                  retour.}}}\label{K_L02878-1h} und haſt Dich hoffentlich recht erholt. Ich habe die Antwort auf
               Deinen letzten lieben Brief von Tag zu Tage hinausgeſchoben, in der Hoffnung, Dir
               Genaueres über meine Reiſepläne mittheilen zu können; aber es will ſich nicht klären.
               Jetzt iſt wieder einmal beſtimmt, daß ich nach \textsc{Rennes\oindex{Rennes@\textbf{Rennes}|pw}} gehen ſoll. Der \label{K_L02878-2v}\edtext{\textsc{Dreyfus\pwindex{Dreyfus, Alfred 1859-10-09 – 1935-07-12@\textsc{Dreyfus, Alfred} (1859-10-09 – 1935-07-12), \emph{Militär}|pw}}-Prozeß}{\lemma{\textnormal{\emph{Dreyfus-Prozeß}}}\Cendnote{\textnormal{Der neue
                  Kriegsgerichtsprozess in der Affäre Dreyfus\pwindex{Dreyfus, Alfred 1859-10-09 – 1935-07-12@\textsc{Dreyfus, Alfred} (1859-10-09 – 1935-07-12), \emph{Militär}|pwk}
                  begann am 8. 8. 1899 in Rennes\oindex{Rennes@\textbf{Rennes}|pwk}.}}}\label{K_L02878-2h} dürfte Mitte oder Ende Juli ſtattfinden. Dann habe ich noch einige Zeit in \textsc{Paris\oindex{Paris@\textbf{Paris}|pw}} zu thun. Vorausſichtlich werde ich ſo gegen Mitte Auguſt fertig ſein, aber ſicher iſt das auch nicht. {\pb}Kann ich im Auguſt
               meinen Urlaub antreten, ſo will ich nach der Schweiz\oindex{Schweiz@\textbf{Schweiz}|pw} gehen. Das iſt von hier aus das Nächſte und Billigſte. Auch brauche
               ich ſtarke Höhenluft und denke darum an ſo etwas wie das Engadin\oindex{Engadin@\textbf{Engadin}|pw}. Nach Öſterreich\oindex{Oesterreich@\textbf{Österreich}|pw}
               kann ich diesmal nicht kommen, aus mancherlei Gründen nicht. Wenn ich alſo nach der
                  Schweiz\oindex{Schweiz@\textbf{Schweiz}|pw} gehe, ſo wirſt Du Dich, wie ich
               zuverſichtlich hoffe, mit mir \label{K_L02878-3v}\edtext{vereinigen}{\lemma{\textnormal{\emph{vereinigen}}}\Cendnote{\textnormal{nicht geschehen}}}\label{K_L02878-3h}.
               Auch Dir wird es gut thun, einmal aus Öſterreich\oindex{Oesterreich@\textbf{Österreich}|pw} herauszukommen. Da ich aber noch gar nichts Beſtimmtes ſagen
               kann, ſo bleibt mir nichts übrig, als Dich zu bitten, mich ſtets auf dem Laufenden
               über Deine Adreſſe zu erhalten. Du kannſt mir immer nach Frankfurt\oindex{Frankfurt am Main@\textbf{Frankfurt am Main}|pw} an die Zeitung\orgindex{Frankfurter Zeitung@Frankfurter Zeitung|pwv} ſchreiben; alle Briefe werden mir nachgeſchickt. {\pb}Hoffentlich höre ich alſo bald wieder von Dir.\pend
           \pstart
           Sonſt war Dein letzter Brief \strikeout{wi} wieder einmal recht
               traurig. Ich wünſche mit Ungeduld den Augenblick herbei, wo ich Dich endlich wieder
               einmal \strikeout{ſ\textcolor{gray}{r}} ſehen und ſprechen kann. Das Reiſen verfehlt hoffentlich auf Dich nicht ſeine
               bewährte Wirkung. Aber nur nicht allein reiſen! Jemanden mußt Du mitnehmen, und wenn
               es der größte Schafskopf wäre. In ein paar Wochen hoffentlich \label{K_L02878-4v}\edtext{komme ich dann zu Dir}{\lemma{\textnormal{\emph{komme ich dann zu Dir}}}\Cendnote{\textnormal{Goldmann\pwindex{Goldmann, Paul 31.01.1865 – 25.09.1935@\textsc{Goldmann, Paul} (31.01.1865 – 25.09.1935), \emph{Schriftsteller, Journalist}|pwk} kam am 13. 10. 1899 nach Wien\oindex{Wien@\textbf{Wien}|pwk} und blieb bis zum 21. 10. 1899.}}}\label{K_L02878-4h},
                  \strikeout{\textcolor{gray}{×}\-\textcolor{gray}{×}} obwohl ich diesmal gerade keine heitere Geſellſchaft für Dich ſein werde.\pend
           \pstart
           Bitte, ſchreib’ mir bald!\pend
           \pstart
           Viele treue Grüße! {\\[\baselineskip]}Dein {\\[\baselineskip]}\spacefill\mbox{Paul Goldmann.}\pend
           \leftskip=0em{}
         
         \endnumbering\mylabel{h}\end{ledgroupsized}  \newcommand{\dateiname}{L02878}\newcommand{\titel}{Paul Goldmann an Arthur Schnitzler, 2. 7. 1899}\newcommand{\editorInnen}{Martin Anton Müller und Laura Untner}%% latex-leseansicht-abspann.tex
%% Abspann für die Leseansicht.
%% Der Schalter \ifkorrekturansicht ist bereits durch den Vorspann gesetzt.

%% latex-abspann.tex
%% Gemeinsamer Abspann für Korrekturansicht und Leseansicht.
%% Setzt den Schalter \ifkorrekturansicht voraus (gesetzt in den
%% einbindenden Dateien latex-korrekturansicht-abspann.tex bzw.
%% latex-leseansicht-abspann.tex).
%% ---------------------------------------------------------------

\normalsize

% Das esempio-Environment wird nur in der Leseansicht benötigt
\ifkorrekturansicht\else
\newenvironment{esempio}[3]%
{
    \vspace{1.5ex}
    \rlap{\underline{#1}}
    \par
    \setlength{\parindent}{0cm}
    \nopagebreak
    \leftskip=#2cm
    \rightskip=#3cm
}
{
    \par
}
\fi

\doendnotes{C}
\bigskip
\vfill

\clearpage

\footnotesize

\ifkorrekturansicht
  \lohead{\textsc{register}}
\fi

% theindex-Environment neu definieren ohne reledmac
\makeatletter
\renewenvironment{theindex}{%
  \ifkorrekturansicht
    \section*{\indexname}%
  \else
    \subsubsection*{Index der erwähnten Entitäten}%
  \fi
  \setlength{\parindent}{0pt}%
  \setlength{\parskip}{0pt plus 0.3pt}%
  \let\item\@idxitem
}{%
  \ifkorrekturansicht\clearpage\fi
}
\makeatother

\IfFileExists{\jobname-pw.ind}{\input{\jobname-pw.ind}}{}

% Quellenangabe nur in der Leseansicht
\ifkorrekturansicht\else
% Fallback-Definitionen, falls die .tex-Datei \titel etc. nicht gesetzt hat
\providecommand{\titel}{}
\providecommand{\editorInnen}{}
\providecommand{\dateiname}{\jobname}

\vspace{3cm}

\vfill

\footnotesize
\textsc{Quelle}: \titel. Herausgegeben von {\editorInnen}. In: \emph{Arthur Schnitzler: Briefwechsel mit Autorinnen und Autoren}.
 Digitale Edition, https://schnitzler-briefe.acdh.oeaw.ac.at/{\dateiname}.html (Stand \today)
\fi

\end{document}


      