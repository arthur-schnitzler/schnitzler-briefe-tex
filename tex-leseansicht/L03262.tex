%% latex-korrekturansicht-vorspann.tex
%% Vorspann für die Korrekturansicht.
%% Lädt die gemeinsame Datei latex-vorspann.tex mit gesetztem Schalter.

\newif\ifkorrekturansicht
\korrekturansichttrue

\input{../tex-inputs/latex-vorspann}


\section[ Felix Salten an Arthur Schnitzler, {[}10. 1. 1897{]}]{L03262 Felix Salten an Arthur Schnitzler, {[}10. 1. 1897{]}}
\nopagebreak\mylabel{L03262v}
\rehead{ }\normalsize\beginnumbering\briefempfaengerindex{Schnitzler, Arthur@\textsc{Schnitzler, Arthur}!zzzSalten, Felix@\emph{von Felix Salten}!1897-01-101@{{[}10. 1. 1897{]}}|(be}
\toendnotes[C]{\smallbreak\pagebreak[2]}\Standort{CUL, Schnitzler, B 89, A 2.}
\physDesc{Karte, 142 Zeichen
\newline{}Handschrift: Bleistift, lateinische Kurrent
\newline{}Schnitzler: mit Bleistift datiert: »10. 1. 97« und die Tagesangabe ungenau eingekringelt 
\newline{}Ordnung: mit Bleistift von unbekannter Hand nummeriert: »84« }\toendnotes[C]{\smallbreak}
\pstart
           \noindent{}{\pb}Lieber Arthur, ich muß Sie nothwendig noch \label{K_L03262-1v}\edtext{heute{ }Abend sprechen}{\lemma{\textnormal{\emph{heute Abend sprechen}}}\Cendnote{\textnormal{Paul Blasel\pwindex{Blasel, Paul 1855-06-29 – 1940-06-21@\textsc{Blasel, Paul} (1855-06-29 – 1940-06-21), \emph{Schauspieler/Schauspielerin, Theaterdirektor/Theaterdirektorin, Tenor/}|pwk} hatte zum Jahreswechsel
                  bekanntgegeben, dass er nach zwei Spielzeiten die Leitung des \emph{Stadttheaters}\orgindex{Stadttheater Teplitz@Stadttheater Teplitz|pwk} in Teplitz\oindex{Teplice@\textbf{Teplice}, \emph{P.PPL}|pwk} mit Ablauf der Saison zurückgeben werde. Salten\pwindex{Salten, Felix 06.09.1869 – 08.10.1945@\textsc{Salten, Felix} (06.09.1869 – 08.10.1945), \emph{Schriftsteller/Schriftstellerin, Journalist/Journalistin, Chefredakteur/Chefredakteurin}|pwk} bemühte sich um die Nachfolge. Am Abend versuchte er
                  (vergeblich), von Schnitzler das Geld für
                  die Kaution zu leihen. Zwei Jahre später wurde der Plan von Salten\pwindex{Salten, Felix 06.09.1869 – 08.10.1945@\textsc{Salten, Felix} (06.09.1869 – 08.10.1945), \emph{Schriftsteller/Schriftstellerin, Journalist/Journalistin, Chefredakteur/Chefredakteurin}|pwk} nochmals aufgenommen,
                     Felix Salten an Arthur Schnitzler, 6. 5. 1899.}}}\label{K_L03262-1}. Um 12
               etwa werde ich Sie im Arcaden-Café\oindex{Cafe Arkaden@\textbf{Café Arkaden}, \emph{Kaffeehaus (K.KAF)}|pw} erwarten.
               Kommen Sie hin, ja?\pend
           
\pstart
           Herzlich {\\[\baselineskip]}\spacefill\mbox{Salten}\pend
           \leftskip=0em{}\selectlanguage{ngerman}\endnumbering\briefempfaengerindex{Schnitzler, Arthur@\textsc{Schnitzler, Arthur}!zzzSalten, Felix@\emph{von Felix Salten}!1897-01-101@{{[}10. 1. 1897{]}}|)be}\mylabel{L03262h}  \normalsize

\doendnotes{C}
\bigskip
\vfill

\clearpage

\footnotesize

\lohead{\textsc{register}}

% Definiere theindex-Environment komplett neu ohne reledmac
\makeatletter
\renewenvironment{theindex}{%
  \section*{\indexname}%
  \setlength{\parindent}{0pt}%
  \setlength{\parskip}{0pt plus 0.3pt}%
  \let\item\@idxitem
}{%
  \clearpage
}
\makeatother

\IfFileExists{\jobname-pw.ind}{\input{\jobname-pw.ind}}{}

\end{document}

      