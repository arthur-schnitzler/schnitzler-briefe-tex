%% latex-korrekturansicht-vorspann.tex
%% Vorspann für die Korrekturansicht.
%% Lädt die gemeinsame Datei latex-vorspann.tex mit gesetztem Schalter.

\newif\ifkorrekturansicht
\korrekturansichttrue

\input{../tex-inputs/latex-vorspann}


\section[ Felix Salten an Arthur Schnitzler, 2. 7. 1904]{L03398 Felix Salten an Arthur Schnitzler, 2. 7. 1904}
\nopagebreak\mylabel{L03398v}
\rehead{ }\normalsize\beginnumbering\briefempfaengerindex{Schnitzler, Arthur@\textsc{Schnitzler, Arthur}!zzzSalten, Felix@\emph{von Felix Salten}!1904-07-022@{2. 7. 1904}|(be}
\toendnotes[C]{\smallbreak\pagebreak[2]}\Standort{CUL, Schnitzler, B 89, B 1.}
\physDesc{Brief, 1 Blatt, 1 Seite, 675 Zeichen
\newline{}maschinell
\newline{}Handschrift: schwarze Tinte, lateinische Kurrent (\noindent{}einen Wortabstand eingefügt, Unterschrift und Nachschrift)
\newline{}Ordnung: mit Bleistift von unbekannter Hand nummeriert: »190« }\toendnotes[C]{\smallbreak}
\pstart
           {\pb}\textcolor{gray}{\textbf{DIE}}\hfill \textcolor{gray}{\textbf{\emph{WIEN, I.}\oindex{I., Innere Stadt@\textbf{I., Innere Stadt}, \emph{A.ADM3}|pw}}}\pend
           
\pstart
           \textcolor{gray}{\textbf{ZEIT\orgindex{Zeit@Die Zeit|pw}}}\hfill \textcolor{gray}{\textbf{\emph{Wipplingerstrasse 38\oindex{Wipplingerstrasse@\textbf{Wipplingerstraße}, \emph{Straße (K.STR)}|pw}}}}{ }2. Juli 1904\pend
           
\pstart
           \textcolor{gray}{\textbf{WIEN\oindex{Wien@\textbf{Wien}, \emph{A.ADM2}|pw}ER TAGESZEITUNG}}\pend
           
\pstart
           \textcolor{gray}{\textbf{Herausgeber:}}\pend
           
\pstart
           \textcolor{gray}{\textbf{\textbf{Prof. Dr. I. Singer\pwindex{Singer, Isidor 16.01.1857 – 08.12.1927@\textsc{Singer, Isidor} (16.01.1857 – 08.12.1927), \emph{Journalist/Journalistin, Herausgeber/Herausgeberin, Soziologe/Soziologin}|pw}}}}\pend
           
\pstart
           \textcolor{gray}{\textbf{\textbf{Dr. Heinrich Kanner\pwindex{Kanner, Heinrich 09.11.1864 – 15.02.1930@\textsc{Kanner, Heinrich} (09.11.1864 – 15.02.1930), \emph{Herausgeber/Herausgeberin, Publizist/Publizistin}|pw}}}}\pend
           
\pstart
           \textcolor{gray}{\textbf{\textbf{Redaction}}}\pend
           
\pstart
           \textcolor{gray}{\textbf{Telegramm-Adresse: \so{Zeit}\orgindex{Zeit@Die Zeit|pw}\so{,}{ }\so{Wien}\oindex{Wien@\textbf{Wien}, \emph{A.ADM2}|pw}}}\pend
           
\pstart
           \textcolor{gray}{\textbf{Interurbanes Telephon Nr. 15.988}}\pend
           
\pstart
           \textcolor{gray}{\textbf{= Telephone Nr. 17.040, 17.041 =}}\pend
           
\pstart\center{}Lieber Freund!\pend\vspace{0.5em}
\pstart
           Den Einakter »Giulia\pwindex{Giulia. Drama in einem Akt@\emph{Giulia. Drama in einem Akt}|pw}« von Artur Vollmöller\pwindex{Vollmoeller, Karl Gustav 07.05.1878 – 18.10.1948@\textsc{Vollmoeller, Karl Gustav} (07.05.1878 – 18.10.1948), \emph{Schriftsteller/Schriftstellerin}|pw} kann ich leider in der »Zeit\pwindex{Zeit@\emph{Die Zeit}|pw}« nicht bringen. Die Situation lässt sich unmöglich vom
               Bett aus auf ein anderes Möbelstück verlegen. Das wäre aber noch das
                  wenigste{[}.{]} Ich kann der ganzen Arbeit\pwindex{Giulia. Drama in einem Akt@\emph{Giulia. Drama in einem Akt}|pwv} keinen Geschmack abgewinnen; sie
               erscheint mir forciert, vollständig dem D’Annunzio\pwindex{DAnnunzio, Gabriele 12.03.1863 – 01.03.1938@\textsc{D’Annunzio, Gabriele} (12.03.1863 – 01.03.1938), \emph{Schriftsteller/Schriftstellerin}|pw} nachgebildet und unnötig. Ich glaube, dass Vollmöller\pwindex{Vollmoeller, Karl Gustav 07.05.1878 – 18.10.1948@\textsc{Vollmoeller, Karl Gustav} (07.05.1878 – 18.10.1948), \emph{Schriftsteller/Schriftstellerin}|pw} zuletzt doch eine Enttäuschung sein wird, ausser,
               man hat sich von ihm überhaupt nichts versprochen.\pend
           
\pstart
           Hoffentlich sind Sie bald wieder ganz \label{K_L03398-1v}\edtext{gesund}{\lemma{\textnormal{\emph{gesund}}}\Cendnote{\textnormal{Vgl. Hugo von Hofmannsthal an Arthur Schnitzler, 28. 6. 1904. Saltens\pwindex{Salten, Felix 06.09.1869 – 08.10.1945@\textsc{Salten, Felix} (06.09.1869 – 08.10.1945), \emph{Schriftsteller/Schriftstellerin, Journalist/Journalistin, Chefredakteur/Chefredakteurin}|pwk} letzter Besuch fand 
               am 29. 6. 1904 statt. Ein neuerlicher Besuch ist nicht belegt,
               stattdessen war Schnitzler am 6. 7. 1904
               bei Salten\pwindex{Salten, Felix 06.09.1869 – 08.10.1945@\textsc{Salten, Felix} (06.09.1869 – 08.10.1945), \emph{Schriftsteller/Schriftstellerin, Journalist/Journalistin, Chefredakteur/Chefredakteurin}|pwk}.}}}\label{K_L03398-1}, ich
               schaue jedenfalls dieser Tage noch einmal zu Ihnen.\pend
           \pstart Herzlichst Ihr \spacefill\mbox{{[}hs.:{]} Salten}\pend{}
\pstart
           \noindent{}{[}ms.:{]} Herrn Dr. Arthur Schnitzler\pend
           
\pstart
           Wien, XVIII. Spöttelgasse 7\oindex{Edmund-Weiss-Gasse 7@\textbf{Edmund-Weiß-Gasse 7}, \emph{Wohngebäude (K.WHS)}|pw}\pend
           
\pstart
           {[}hs.:{]} \label{K_L03398-2v}\edtext{1 Manuscript}{\lemma{\textnormal{\emph{1 Manuscript}}}\Cendnote{\textnormal{Beilage nicht erhalten}}}\label{K_L03398-2}\pend
           \selectlanguage{ngerman}\endnumbering\briefempfaengerindex{Schnitzler, Arthur@\textsc{Schnitzler, Arthur}!zzzSalten, Felix@\emph{von Felix Salten}!1904-07-022@{2. 7. 1904}|)be}\mylabel{L03398h}  \normalsize

\doendnotes{C}
\bigskip
\vfill

\clearpage

\footnotesize

\lohead{\textsc{register}}

% Definiere theindex-Environment komplett neu ohne reledmac
\makeatletter
\renewenvironment{theindex}{%
  \section*{\indexname}%
  \setlength{\parindent}{0pt}%
  \setlength{\parskip}{0pt plus 0.3pt}%
  \let\item\@idxitem
}{%
  \clearpage
}
\makeatother

\IfFileExists{\jobname-pw.ind}{\input{\jobname-pw.ind}}{}

\end{document}

      