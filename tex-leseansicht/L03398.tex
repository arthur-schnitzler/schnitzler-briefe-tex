%% latex-leseansicht-vorspann.tex
%% Vorspann für die Leseansicht.
%% Lädt die gemeinsame Datei latex-vorspann.tex mit nicht gesetztem Schalter.

\newif\ifkorrekturansicht
\korrekturansichtfalse

\input{../tex-inputs/latex-vorspann}


         
         \renewcommand{\erwaehntePersonen}{Personen: Gabriele D’Annunzio, Heinrich Kanner, Isidor Singer, Karl Gustav Vollmoeller}
         \renewcommand{\erwaehnteInstitutionen}{Institutionen: Die Zeit}
         \renewcommand{\erwaehnteOrte}{Orte: Edmund-Weiß-Gasse 7, I., Innere Stadt, Wien, Wipplingerstraße}
         \renewcommand{\erwaehnteWerke}{Werke: Die Zeit, Giulia. Drama in einem Akt}
               \section[ Felix Salten an Arthur Schnitzler, 2. 7. 1904]{ Felix Salten an Arthur Schnitzler, 2. 7. 1904}\nopagebreak\mylabel{v}\rehead{ }\begin{ledgroupsized}[t]{13cm}\normalsize\beginnumbering \toendnotes[C]{\smallbreak\pagebreak[2]} \Standort{CUL, Schnitzler, B 89, B 1.}
\physDesc{Brief, 1 Blatt, 1 Seite, 675 Zeichen
\newline{}maschinell
\newline{}Handschrift: schwarze Tinte, lateinische Kurrent (\noindent{}einen Wortabstand eingefügt, Unterschrift und Nachschrift)
\newline{}Ordnung: mit Bleistift von unbekannter Hand nummeriert: »190« }\toendnotes[C]{\smallbreak}\pstart
           \noindent{}{\pb}\textcolor{gray}{\textbf{DIE}}\hfill \textcolor{gray}{\textbf{\emph{WIEN, I.}\oindex{I., Innere Stadt@\textbf{I., Innere Stadt}|pw}}}\pend
           \pstart
           \textcolor{gray}{\textbf{ZEIT\orgindex{Zeit@Die Zeit|pw}}}\hfill \textcolor{gray}{\textbf{\emph{Wipplingerstrasse 38\oindex{Wipplingerstrasse@\textbf{Wipplingerstraße}|pw}}}}{ }2. Juli 1904\pend
           \pstart
           \textcolor{gray}{\textbf{WIEN\oindex{Wien@\textbf{Wien}|pw}ER TAGESZEITUNG}}\pend
           \pstart
           \textcolor{gray}{\textbf{Herausgeber:}}\pend
           \pstart
           \textcolor{gray}{\textbf{\textbf{Prof. Dr. I. Singer\pwindex{Singer, Isidor 16.01.1857 – 08.12.1927@\textsc{Singer, Isidor} (16.01.1857 – 08.12.1927), \emph{Journalist, Herausgeber, Soziologe}|pw}}}}\pend
           \pstart
           \textcolor{gray}{\textbf{\textbf{Dr. Heinrich Kanner\pwindex{Kanner, Heinrich 09.11.1864 – 15.02.1930@\textsc{Kanner, Heinrich} (09.11.1864 – 15.02.1930), \emph{Herausgeber, Publizist}|pw}}}}\pend
           \pstart
           \textcolor{gray}{\textbf{\textbf{Redaction}}}\pend
           \pstart
           \textcolor{gray}{\textbf{Telegramm-Adresse: \so{Zeit}\orgindex{Zeit@Die Zeit|pw}\so{,}{ }\so{Wien}\oindex{Wien@\textbf{Wien}|pw}}}\pend
           \pstart
           \textcolor{gray}{\textbf{Interurbanes Telephon Nr. 15.988}}\pend
           \pstart
           \textcolor{gray}{\textbf{= Telephone Nr. 17.040, 17.041 =}}\pend
           \pstart\center{}Lieber Freund!\pend\pstart
           Den Einakter »Giulia\pwindex{Vollmoeller, Karl Gustav 07.05.1878 – 18.10.1948@\textsc{Vollmoeller, Karl Gustav} (07.05.1878 – 18.10.1948), \emph{Schriftsteller}!Giulia. Drama in einem Akt1904-12-04@\strich\emph{Giulia. Drama in einem Akt} {[}1904-12-04{]}|pw}« von Artur Vollmöller\pwindex{Vollmoeller, Karl Gustav 07.05.1878 – 18.10.1948@\textsc{Vollmoeller, Karl Gustav} (07.05.1878 – 18.10.1948), \emph{Schriftsteller}|pw} kann ich leider in der »Zeit\pwindex{Zeit1902-09-27 – 1919@\emph{Die Zeit} {[}1902-09-27 – 1919{]}|pw}« nicht bringen. Die Situation lässt sich unmöglich vom
               Bett aus auf ein anderes Möbelstück verlegen. Das wäre aber noch das
                  wenigste{[}.{]} Ich kann der ganzen Arbeit\pwindex{Vollmoeller, Karl Gustav 07.05.1878 – 18.10.1948@\textsc{Vollmoeller, Karl Gustav} (07.05.1878 – 18.10.1948), \emph{Schriftsteller}!Giulia. Drama in einem Akt1904-12-04@\strich\emph{Giulia. Drama in einem Akt} {[}1904-12-04{]}|pwv} keinen Geschmack abgewinnen; sie
               erscheint mir forciert, vollständig dem D’Annunzio\pwindex{DAnnunzio, Gabriele 12.03.1863 – 01.03.1938@\textsc{D’Annunzio, Gabriele} (12.03.1863 – 01.03.1938), \emph{Schriftsteller}|pw} nachgebildet und unnötig. Ich glaube, dass Vollmöller\pwindex{Vollmoeller, Karl Gustav 07.05.1878 – 18.10.1948@\textsc{Vollmoeller, Karl Gustav} (07.05.1878 – 18.10.1948), \emph{Schriftsteller}|pw} zuletzt doch eine Enttäuschung sein wird, ausser,
               man hat sich von ihm überhaupt nichts versprochen.\pend
           \pstart
           Hoffentlich sind Sie bald wieder ganz \label{K_L03398-1v}\edtext{gesund}{\lemma{\textnormal{\emph{gesund}}}\Cendnote{\textnormal{vgl. Hugo von Hofmannsthal an Arthur Schnitzler, 28. 6. 1904. Salten\pwindex{Salten, Felix 06.09.1869 – 08.10.1945@\textsc{Salten, Felix} (06.09.1869 – 08.10.1945), \emph{Schriftsteller, Journalist}|pwk}s letzter Besuch fand 
               am 29. 6. 1904 statt. Ein neuerlicher Besuch ist nicht belegt,
               stattdessen war Schnitzler\pwindex{Schnitzler, Arthur 15.05.1862 – 21.10.1931@\textsc{Schnitzler, Arthur} (15.05.1862 – 21.10.1931), \emph{Schriftsteller, Mediziner}|pwk} am 6. 7. 1904
               bei Salten\pwindex{Salten, Felix 06.09.1869 – 08.10.1945@\textsc{Salten, Felix} (06.09.1869 – 08.10.1945), \emph{Schriftsteller, Journalist}|pwk}.}}}\label{K_L03398-1h}, ich
               schaue jedenfalls dieser Tage noch einmal zu Ihnen.\pend
           \pstart Herzlichst Ihr \spacefill\mbox{{[}hs.:{]} Salten}\pend{}\pstart
           \noindent{}{[}ms.:{]} Herrn Dr. Arthur Schnitzler\pend
           \pstart
           Wien, XVIII. Spöttelgasse 7\oindex{Edmund-Weiss-Gasse 7@\textbf{Edmund-Weiß-Gasse 7}|pw}\pend
           \pstart
           {[}hs.:{]} \label{K_L03398-2v}\edtext{1 Manuscript}{\lemma{\textnormal{\emph{1 Manuscript}}}\Cendnote{\textnormal{Beilage nicht erhalten}}}\label{K_L03398-2h}\pend
           
         
         \endnumbering\mylabel{h}\end{ledgroupsized}  \newcommand{\dateiname}{L03398}\newcommand{\titel}{Felix Salten an Arthur Schnitzler, 2. 7. 1904}\newcommand{\editorInnen}{Martin Anton Müller und Laura Untner}%% latex-leseansicht-abspann.tex
%% Abspann für die Leseansicht.
%% Der Schalter \ifkorrekturansicht ist bereits durch den Vorspann gesetzt.

%% latex-abspann.tex
%% Gemeinsamer Abspann für Korrekturansicht und Leseansicht.
%% Setzt den Schalter \ifkorrekturansicht voraus (gesetzt in den
%% einbindenden Dateien latex-korrekturansicht-abspann.tex bzw.
%% latex-leseansicht-abspann.tex).
%% ---------------------------------------------------------------

\normalsize

% Das esempio-Environment wird nur in der Leseansicht benötigt
\ifkorrekturansicht\else
\newenvironment{esempio}[3]%
{
    \vspace{1.5ex}
    \rlap{\underline{#1}}
    \par
    \setlength{\parindent}{0cm}
    \nopagebreak
    \leftskip=#2cm
    \rightskip=#3cm
}
{
    \par
}
\fi

\doendnotes{C}
\bigskip
\vfill

\clearpage

\footnotesize

\ifkorrekturansicht
  \lohead{\textsc{register}}
\fi

% theindex-Environment neu definieren ohne reledmac
\makeatletter
\renewenvironment{theindex}{%
  \ifkorrekturansicht
    \section*{\indexname}%
  \else
    \subsubsection*{Index der erwähnten Entitäten}%
  \fi
  \setlength{\parindent}{0pt}%
  \setlength{\parskip}{0pt plus 0.3pt}%
  \let\item\@idxitem
}{%
  \ifkorrekturansicht\clearpage\fi
}
\makeatother

\IfFileExists{\jobname-pw.ind}{\input{\jobname-pw.ind}}{}

% Quellenangabe nur in der Leseansicht
\ifkorrekturansicht\else
% Fallback-Definitionen, falls die .tex-Datei \titel etc. nicht gesetzt hat
\providecommand{\titel}{}
\providecommand{\editorInnen}{}
\providecommand{\dateiname}{\jobname}

\vspace{3cm}

\vfill

\footnotesize
\textsc{Quelle}: \titel. Herausgegeben von {\editorInnen}. In: \emph{Arthur Schnitzler: Briefwechsel mit Autorinnen und Autoren}.
 Digitale Edition, https://schnitzler-briefe.acdh.oeaw.ac.at/{\dateiname}.html (Stand \today)
\fi

\end{document}


      