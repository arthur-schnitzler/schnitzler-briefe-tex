%% latex-leseansicht-vorspann.tex
%% Vorspann für die Leseansicht.
%% Lädt die gemeinsame Datei latex-vorspann.tex mit nicht gesetztem Schalter.

\newif\ifkorrekturansicht
\korrekturansichtfalse

\input{../tex-inputs/latex-vorspann}


\section[Olga Schnitzler an Stefan Zweig, 20. 11. 1916]{L03766 Olga Schnitzler an Stefan Zweig, 20. 11. 1916}
\nopagebreak\mylabel{L03766v}
\rehead{ }\normalsize\beginnumbering\briefempfaengerindex{Zweig, Stefan@\textsc{Zweig, Stefan}!zzzSchnitzler, Olga@\emph{von Olga Schnitzler}!1916-11-202@{20. 11. 1916}|(be}
\toendnotes[C]{\smallbreak\pagebreak[2]}
\correspDesc{Versand  durch Olga Schnitzler am 20. 11. 1916 in Wien
\newline{}Erhalt  durch Stefan Zweig im Zeitraum [20. 11. 1916 – 23. 11. 1916?] in Wien}\toendnotes[C]{\smallbreak}
\Standort{Jerusalem, National Library of Israel, ARC. Ms. Var. 305 1 58 Stefan Zweig Collection.}
\physDesc{Briefkarte, 427 Zeichen
\newline{}Handschrift: schwarze Tinte, lateinische Kurrent}\toendnotes[C]{\smallbreak}
\pstart
           \noindent{}{\pb}Herzlichsten Dank, lieber Herr Doctor für die \label{K_L03766-1v}\edtext{prachtvollen Blumen}{\lemma{\textnormal{\emph{prachtvollen Blumen}}}\Cendnote{\textnormal{Am 18. 11. 1916 war Olga Schnitzler\pwindex{Schnitzler, Olga 17.\,1.\,1882 Wien – 13.\,1.\,1970 Lugano@\textsc{Schnitzler, Olga} (17.\,1.\,1882 Wien – 13.\,1.\,1970 Lugano), \emph{Schauspielerin, Sängerin}|pwk}
                  an einem Liederkonzert\eventindex{Wiener Konzerthaus@\textbf{Wiener Konzerthaus}!Gesangskonzert von Olga Schnitzler, 18.11.1916@Gesangskonzert von Olga Schnitzler, 18.11.1916|pwk} im Wiener Konzerthaus\oindex{Wien@\textbf{Wien}!III., Landstraße@\textbf{III., Landstraße}!Wiener Konzerthaus@\textbf{Wiener Konzerthaus}, \emph{Konzertsaal}|pwk} beteiligt gewesen.}}}\label{K_L03766-1}! ich habe mich
               sehr gefreut! Und Dank vor Allem für Ihr immer reges und warmes Interesse, das mir
               sehr wol tut. Gestern war’s mir leider nicht möglich, sie telefonisch zu erreichen, –
               ich wollte Sie für den \label{K_L03766-2v}\edtext{Abend mit der
                  Hofrätin\pwindex{Zuckerkandl, Berta 13.\,4.\,1864 Wien – 16.\,10.\,1945 Paris@\textsc{Zuckerkandl, Berta} (13.\,4.\,1864 Wien – 16.\,10.\,1945 Paris), \emph{Schriftstellerin, Journalistin, Übersetzerin}|pwv}}{\lemma{\textnormal{\emph{Abend mit der
                  Hofrätin}}}\Cendnote{\textnormal{Vgl. A. S.: \emph{Tagebuch}, 19. 11. 1916.}}}\label{K_L03766-2}
               herbitten. So hoff ich auf ein anderes Mal, – und {\pb}auf
               bald!\pend
           
\pstart
           Mit den herzlichsten Grüssen, auch von Arthur, Ihre{\\[\baselineskip]}\spacefill\mbox{Olga Schnitzler.}\pend
           \leftskip=0em{}
\pstart
           20. Nov. 1916.\pend
           \selectlanguage{ngerman}\endnumbering\briefempfaengerindex{Zweig, Stefan@\textsc{Zweig, Stefan}!zzzSchnitzler, Olga@\emph{von Olga Schnitzler}!1916-11-202@{20. 11. 1916}|)be}\mylabel{L03766h}  \newcommand{\dateiname}{L03766}\newcommand{\titel}{Olga Schnitzler an Stefan Zweig, 20. 11. 1916}\newcommand{\editorInnen}{Selma Jahnke und Martin Anton Müller}%% latex-leseansicht-abspann.tex
%% Abspann für die Leseansicht.
%% Der Schalter \ifkorrekturansicht ist bereits durch den Vorspann gesetzt.

%% latex-abspann.tex
%% Gemeinsamer Abspann für Korrekturansicht und Leseansicht.
%% Setzt den Schalter \ifkorrekturansicht voraus (gesetzt in den
%% einbindenden Dateien latex-korrekturansicht-abspann.tex bzw.
%% latex-leseansicht-abspann.tex).
%% ---------------------------------------------------------------

\normalsize

% Das esempio-Environment wird nur in der Leseansicht benötigt
\ifkorrekturansicht\else
\newenvironment{esempio}[3]%
{
    \vspace{1.5ex}
    \rlap{\underline{#1}}
    \par
    \setlength{\parindent}{0cm}
    \nopagebreak
    \leftskip=#2cm
    \rightskip=#3cm
}
{
    \par
}
\fi

\doendnotes{C}
\bigskip
\vfill

\clearpage

\footnotesize

\ifkorrekturansicht
  \lohead{\textsc{register}}
\fi

% theindex-Environment neu definieren ohne reledmac
\makeatletter
\renewenvironment{theindex}{%
  \ifkorrekturansicht
    \section*{\indexname}%
  \else
    \subsubsection*{Index der erwähnten Entitäten}%
  \fi
  \setlength{\parindent}{0pt}%
  \setlength{\parskip}{0pt plus 0.3pt}%
  \let\item\@idxitem
}{%
  \ifkorrekturansicht\clearpage\fi
}
\makeatother

\IfFileExists{\jobname-pw.ind}{\input{\jobname-pw.ind}}{}

% Quellenangabe nur in der Leseansicht
\ifkorrekturansicht\else
% Fallback-Definitionen, falls die .tex-Datei \titel etc. nicht gesetzt hat
\providecommand{\titel}{}
\providecommand{\editorInnen}{}
\providecommand{\dateiname}{\jobname}

\vspace{3cm}

\vfill

\footnotesize
\textsc{Quelle}: \titel. Herausgegeben von {\editorInnen}. In: \emph{Arthur Schnitzler: Briefwechsel mit Autorinnen und Autoren}.
 Digitale Edition, https://schnitzler-briefe.acdh.oeaw.ac.at/{\dateiname}.html (Stand \today)
\fi

\end{document}


