%% latex-leseansicht-vorspann.tex
%% Vorspann für die Leseansicht.
%% Lädt die gemeinsame Datei latex-vorspann.tex mit nicht gesetztem Schalter.

\newif\ifkorrekturansicht
\korrekturansichtfalse

\input{../tex-inputs/latex-vorspann}


\section[ Paul Goldmann an Arthur Schnitzler, 22. 9. {[}1896{]}]{L02785 Paul Goldmann an Arthur Schnitzler,  22. 9. [1896]}
\nopagebreak\mylabel{L02785v}
\rehead{ }\normalsize\beginnumbering\briefempfaengerindex{Schnitzler, Arthur@\textsc{Schnitzler, Arthur}!zzzGoldmann, Paul@\emph{von Paul Goldmann}!1896-09-221@{22. 9. [1896]}|(be}
\toendnotes[C]{\smallbreak\pagebreak[2]}
\correspDesc{Versand  durch Paul Goldmann am 22. 9. [1896] in Paris
\newline{}Erhalt  durch Arthur Schnitzler im Zeitraum [23. 9. 1896
                  – 27. 9. 1896?] in Wien}\toendnotes[C]{\smallbreak}
\Standort{DLA, A:Schnitzler, HS.NZ85.1.3166.}
\physDesc{Brief, 1 Blatt, 4 Seiten, 2680 Zeichen
\newline{}Handschrift: blaue Tinte, deutsche Kurrent
\newline{}Beilage: handschriftlicher Brief: 1 Blatt, 2 Seiten, lila (evtl.
                                 ursprünglich schwarze?) Tinte, lateinische Kurrent; mit Bleistift
                                 Vermerk des Datums von Schnitzler mit »Sept{[}ember{]} 96« 
\newline{}Schnitzler: 1) mit Bleistift das Jahr »96« vermerkt  2) mit rotem Buntstift drei Unterstreichungen}\toendnotes[C]{\smallbreak}
\pstart
           {\pb}\textcolor{gray}{\textbf{\textbf{Frankfurter Zeitung\orgindex{Frankfurter Zeitung@Frankfurter Zeitung|pw}}}}\pend
           
\pstart
           \textcolor{gray}{\textbf{(\begin{otherlanguage}{french}Gazette de Francfort\end{otherlanguage}\orgindex{Frankfurter Zeitung@Frankfurter Zeitung|pw}).}}\pend
           
\pstart
           \textcolor{gray}{\textbf{\textbf{\begin{otherlanguage}{french}Fondateur M.\end{otherlanguage}{ }L. Sonnemann\pwindex{Sonnemann, Leopold 29.\,10.\,1831 Höchberg – 30.\,10.\,1909 Frankfurt am Main@\textsc{Sonnemann, Leopold} (29.\,10.\,1831 Höchberg – 30.\,10.\,1909 Frankfurt am Main), \emph{Journalist, Herausgeber}|pw}.}}}\pend
           
\pstart
           \begin{otherlanguage}{french}\textcolor{gray}{\textbf{Journal\pwindex{Frankfurter Zeitung@\emph{Frankfurter Zeitung}|pwv} politique,
                        financier,}}\end{otherlanguage}\pend
           
\pstart
           \begin{otherlanguage}{french}\textcolor{gray}{\textbf{commercial et littéraire.}}\end{otherlanguage}\pend
           
\pstart
           \begin{otherlanguage}{french}\textcolor{gray}{\textbf{\textbf{Paraissant trois fois par jour.}}}\end{otherlanguage}\hfill \textsc{Paris\oindex{Paris@\textbf{Paris}, \emph{Hauptstadt}|pw}}, 22. September.\pend
           
\pstart
           \begin{otherlanguage}{french}\textcolor{gray}{\textbf{\textbf{Bureau à Paris\oindex{Paris@\textbf{Paris}, \emph{Hauptstadt}|pw}}}}\end{otherlanguage}\pend
           
\pstart
           \begin{otherlanguage}{french}\textcolor{gray}{\textbf{\textbf{24. Rue Feydeau\oindex{rue Feydeau@\textbf{rue Feydeau}, \emph{Straße}|pw}.}}}\end{otherlanguage}\pend
           
\pstart\center{}Mein lieber Freund,\pend\vspace{0.5em}
\pstart
           Ich habe in dieſen Tagen ungewöhnlich viel zu thun gehabt. Auch gab es allerlei
               Aufregungen. Man \label{K_L02785-1v}\edtext{beſchimpft mich in
               der hieſigen Preſſe und verlangt meine Ausweiſung}{\lemma{\textnormal{\emph{beschimpft … Ausweisung}}}\Cendnote{\textnormal{Goldmanns\pwindex{Goldmann, Paul 31.\,1.\,1865 Breslau – 25.\,9.\,1935 Wien@\textsc{Goldmann, Paul} (31.\,1.\,1865 Breslau – 25.\,9.\,1935 Wien), \emph{Schriftsteller, Journalist}|pwk}{ }Berichterstattung\pwindex{Goldmann, Paul 31.\,1.\,1865 Breslau – 25.\,9.\,1935 Wien@\textsc{Goldmann, Paul} (31.\,1.\,1865 Breslau – 25.\,9.\,1935 Wien), \emph{Schriftsteller, Journalist}!Enthüllungen über die Affaire Dreyfus@\strich\emph{Die Enthüllungen über die Affaire Dreyfus}|pwkv} über die Dreyfus\pwindex{Dreyfus, Alfred 9.\,10.\,1859 Mulhouse – 12.\,7.\,1935 Paris@\textsc{Dreyfus, Alfred} (9.\,10.\,1859 Mulhouse – 12.\,7.\,1935 Paris), \emph{Militär}|pwk}-Affäre (G.\pwindex{Goldmann, Paul 31.\,1.\,1865 Breslau – 25.\,9.\,1935 Wien@\textsc{Goldmann, Paul} (31.\,1.\,1865 Breslau – 25.\,9.\,1935 Wien), \emph{Schriftsteller, Journalist}|pwk} [ = Paul Goldmann\pwindex{Goldmann, Paul 31.\,1.\,1865 Breslau – 25.\,9.\,1935 Wien@\textsc{Goldmann, Paul} (31.\,1.\,1865 Breslau – 25.\,9.\,1935 Wien), \emph{Schriftsteller, Journalist}|pwk}]: \emph{Die
                        Enthüllungen über die Affaire Dreyfus}\pwindex{Goldmann, Paul 31.\,1.\,1865 Breslau – 25.\,9.\,1935 Wien@\textsc{Goldmann, Paul} (31.\,1.\,1865 Breslau – 25.\,9.\,1935 Wien), \emph{Schriftsteller, Journalist}!Enthüllungen über die Affaire Dreyfus@\strich\emph{Die Enthüllungen über die Affaire Dreyfus}|pwk}. In: \emph{Frankfurter Zeitung}\pwindex{Frankfurter Zeitung@\emph{Frankfurter Zeitung}|pwk}, Jg. 41, Nr. 258,
                        16. 9. 1896, Erstes Morgenblatt, S. 1) wurde als
                  Einmischung empfunden. In weiterer Folge führte das zu einem Pistolenduell
                  zwischen Goldmann\pwindex{Goldmann, Paul 31.\,1.\,1865 Breslau – 25.\,9.\,1935 Wien@\textsc{Goldmann, Paul} (31.\,1.\,1865 Breslau – 25.\,9.\,1935 Wien), \emph{Schriftsteller, Journalist}|pwk} und dem antisemitischen
                  Chefredakteur Lucien Millevoye\pwindex{Millevoye, Lucien 1.\,8.\,1850 Grenoble – 25.\,3.\,1918 Paris@\textsc{Millevoye, Lucien} (1.\,8.\,1850 Grenoble – 25.\,3.\,1918 Paris), \emph{Politiker, Journalist}|pwk}, das am
                     21. 11. 1896 stattfand (siehe XXXX Auszeichnungsfehler: Dokument L02684 nicht gefunden). Die publizierten Invektiven gegen Goldmann\pwindex{Goldmann, Paul 31.\,1.\,1865 Breslau – 25.\,9.\,1935 Wien@\textsc{Goldmann, Paul} (31.\,1.\,1865 Breslau – 25.\,9.\,1935 Wien), \emph{Schriftsteller, Journalist}|pwk} aus dem September 1896, auf die er hier Bezug nimmt, konnten bislang nicht
                  belegt werden.}}}\label{K_L02785-1}, weil ich \strikeout{\textcolor{gray}{von}} für die Unſchuld des \textsc{Dreyfus\pwindex{Dreyfus, Alfred 9.\,10.\,1859 Mulhouse – 12.\,7.\,1935 Paris@\textsc{Dreyfus, Alfred} (9.\,10.\,1859 Mulhouse – 12.\,7.\,1935 Paris), \emph{Militär}|pw}} eingetreten bin, von der ich, nach den neueſten Enthüllungen, feſter als je
               überzeugt bin. Zudem geht in meiner Familie Alles drunter und drüber. Kurzum ich weiß
               nicht recht, wo mir {\pb}der Kopf{ }ſteht.\pend
           
\pstart
           Dies um mich zu entſchuldigen, daß ich \strikeout{D} beifolgenden
               Brief von \textsc{Thorel\pwindex{Thorel, Jean 11.\,9.\,1859 Éragny – 20.\,8.\,1916 Enghien-les-Bains@\textsc{Thorel, Jean} (11.\,9.\,1859 Éragny – 20.\,8.\,1916 Enghien-les-Bains), \emph{Übersetzer, Dramatiker}|pw}}{ }ſolange liegen ließ. Heb’ ihn Dir gut auf, denn, wie Du aus{ }ſeinem Inhalt
               erſiehſt, vertritt er die Stelle eines Contracts. Ich habe ihn unter irgend einem
               Vorwand von 6 auf 500 heruntergeſchraubt und habe mir ausdrücklich ausbedungen, daß
               dieſe Zahlung nur als Vorſchuß auf etwaige {\pb}\strikeout{\textcolor{gray}{×}}{ }\textsc{Tantièmen} oder Honorare zu betrachten iſt. Ich fürchte
               allerdings, daß letztere Clauſel platoniſch bleiben dürfte. Nun kannſt Du das Geld
               dieſer Tage an mich{ }ſchicken, wenn Du willſt (aber nicht wieder \label{K_L02785-2v}\edtext{in Goldſtücken in einem recommandirten
                  Brief}{\lemma{\textnormal{\emph{in … Brief}}}\Cendnote{\textnormal{Siehe XXXX Auszeichnungsfehler: Dokument L02760 nicht gefunden. }}}\label{K_L02785-2}). Ich werde
               bei dieſem Geſchäft leider nichts verdienen können, aber Du brauchſt hoffentlich bald
               wieder ein \label{K_L02785-3v}\edtext{Opernglas}{\lemma{\textnormal{\emph{Opernglas}}}\Cendnote{\textnormal{Siehe XXXX Auszeichnungsfehler: Dokument L02762 nicht gefunden. }}}\label{K_L02785-3}.\pend
           
\pstart
           Bei \textsc{Forain\pwindex{Forain, Jean-Louis 23.\,10.\,1852 Reims – 11.\,7.\,1931 Paris@\textsc{Forain, Jean-Louis} (23.\,10.\,1852 Reims – 11.\,7.\,1931 Paris), \emph{Maler, Grafiker, Karikaturist}|pw}} war ich auch, aber er iſt noch auf {\pb}dem
               Lande.\pend
           
\pstart
           Was gibts \strikeout{es} Neues bei Dir? Leben und Dichten?
                  \label{K_L02785-4v}\edtext{Was hörſt Du von Berlin\oindex{Berlin@\textbf{Berlin}, \emph{Hauptstadt}|pw} und wann gehſt Du hin?}{\lemma{\textnormal{\emph{Was … hin?}}}\Cendnote{\textnormal{Goldmann\pwindex{Goldmann, Paul 31.\,1.\,1865 Breslau – 25.\,9.\,1935 Wien@\textsc{Goldmann, Paul} (31.\,1.\,1865 Breslau – 25.\,9.\,1935 Wien), \emph{Schriftsteller, Journalist}|pwk} bezieht sich auf die bevorstehende
                     Uraufführung\eventindex{Deutsches Theater Berlin@\textbf{Deutsches Theater Berlin}!Uraufführung von Freiwild, 3.11.1896@Uraufführung von Freiwild, 3.11.1896|pwkv} des Dreiakters \emph{Freiwild}\pwindex{Schnitzler, Arthur 15.\,5.\,1862 Wien – 21.\,10.\,1931 ebd.@\textsc{Schnitzler, Arthur} (15.\,5.\,1862 Wien – 21.\,10.\,1931 ebd.), \emph{Schriftsteller, Mediziner}!Freiwild. Schauspiel in 3 Akten@\strich\emph{Freiwild. Schauspiel in 3 Akten}|pwk} am 3. 11. 1896 am Deutschen Theater\oindex{Deutsches Theater Berlin@\textbf{Deutsches Theater Berlin}, \emph{Theater}|pwk} in Berlin\oindex{Berlin@\textbf{Berlin}, \emph{Hauptstadt}|pwk}. Siehe dazu vor allem \emph{Der Briefwechsel Arthur Schnitzler – Otto Brahm}.
                     Vollständige Ausgabe. Herausgegeben, eingeleitet und erläutert von Oskar
                     Seidlin. Tübingen: \emph{Niemeyer}{ }1975, S. 14–28. { }Schnitzler war dafür zwischen 26. 10. 1896 und 9. 11. 1896 in Berlin\oindex{Berlin@\textbf{Berlin}, \emph{Hauptstadt}|pwk}.}}}\label{K_L02785-4}{ }\label{K_L02785-5v}\edtext{\textsc{Ebermann\pwindex{Ebermann, Leo 16.\,7.\,1863 Draganovka – 9.\,10.\,1914 Wien@\textsc{Ebermann, Leo} (16.\,7.\,1863 Draganovka – 9.\,10.\,1914 Wien), \emph{Schriftsteller, Journalist, Rechtswissenschaftler}!Athenerin. Drama in drei Aufzügen@\strich\emph{Die Athenerin. Drama in drei Aufzügen}|pwv}\pwindex{Ebermann, Leo 16.\,7.\,1863 Draganovka – 9.\,10.\,1914 Wien@\textsc{Ebermann, Leo} (16.\,7.\,1863 Draganovka – 9.\,10.\,1914 Wien), \emph{Schriftsteller, Journalist, Rechtswissenschaftler}|pw}}}{\lemma{\textnormal{\emph{Ebermann}}}\Cendnote{\textnormal{Schnitzler war nicht nur bei Proben von Leo Ebermanns\pwindex{Ebermann, Leo 16.\,7.\,1863 Draganovka – 9.\,10.\,1914 Wien@\textsc{Ebermann, Leo} (16.\,7.\,1863 Draganovka – 9.\,10.\,1914 Wien), \emph{Schriftsteller, Journalist, Rechtswissenschaftler}|pwk} Stück \emph{Die Athenerin}\pwindex{Ebermann, Leo 16.\,7.\,1863 Draganovka – 9.\,10.\,1914 Wien@\textsc{Ebermann, Leo} (16.\,7.\,1863 Draganovka – 9.\,10.\,1914 Wien), \emph{Schriftsteller, Journalist, Rechtswissenschaftler}!Athenerin. Drama in drei Aufzügen@\strich\emph{Die Athenerin. Drama in drei Aufzügen}|pwk} anwesend gewesen, sondern hatte am 19. 9. 1896 auch die
                  Uraufführung\eventindex{Burgtheater@\textbf{Burgtheater}!Uraufführung von Die Athenerin, 19.9.1896@Uraufführung von Die Athenerin, 19.9.1896|pwkv} im Burgtheater\oindex{Wien@\textbf{Wien}!I., Innere Stadt@\textbf{I., Innere Stadt}!Burgtheater@\textbf{Burgtheater}, \emph{Theater}|pwk} besucht. Siehe zum
                  Erfolg des Stücks\pwindex{Ebermann, Leo 16.\,7.\,1863 Draganovka – 9.\,10.\,1914 Wien@\textsc{Ebermann, Leo} (16.\,7.\,1863 Draganovka – 9.\,10.\,1914 Wien), \emph{Schriftsteller, Journalist, Rechtswissenschaftler}!Athenerin. Drama in drei Aufzügen@\strich\emph{Die Athenerin. Drama in drei Aufzügen}|pwkv} etwa auch
                     XXXX Auszeichnungsfehler: Dokument L00596 nicht gefunden und A. S.: \emph{Tagebuch}, 22. 9. 1896.}}}\label{K_L02785-5}{ }ſcheint ja wohl einen
               großen Erfolg gehabt zu haben?\pend
           
\pstart
           Lies \label{K_L02785-6v}\edtext{\textsc{Karl Hillebrand\pwindex{Hillebrand, Karl 26.\,12.\,1861 Wien – 10.\,1.\,1939@\textsc{Hillebrand, Karl} (26.\,12.\,1861 Wien – 10.\,1.\,1939), \emph{Astronom}|pw}}: Frankreich und die Franzoſen\pwindex{Hillebrand, Karl 26.\,12.\,1861 Wien – 10.\,1.\,1939@\textsc{Hillebrand, Karl} (26.\,12.\,1861 Wien – 10.\,1.\,1939), \emph{Astronom}!Frankreich und die Franzosen in der zweiten Hälfte des XIX. Jahrhunderts: Eindrücke und Erfahrungen@\strich\emph{Frankreich und die Franzosen in der zweiten Hälfte des XIX. Jahrhunderts: Eindrücke und Erfahrungen}|pw}}{\lemma{\textnormal{\emph{Karl … Franzosen}}}\Cendnote{\textnormal{Lektüre durch Schnitzler nicht bekannt}}}\label{K_L02785-6}. Der einzige Deutſche, der
                  Frankreich\oindex{Frankreich@\textbf{Frankreich}|pw} kennt, – und eine Perſönlichkeit\pwindex{Hillebrand, Karl 26.\,12.\,1861 Wien – 10.\,1.\,1939@\textsc{Hillebrand, Karl} (26.\,12.\,1861 Wien – 10.\,1.\,1939), \emph{Astronom}|pwv}. Ich leſe \textsc{Schillers\pwindex{Schiller, Friedrich von 10.\,11.\,1759 Marbach am Neckar – 9.\,5.\,1805 Weimar@\textsc{Schiller, Friedrich von} (10.\,11.\,1759 Marbach am Neckar – 9.\,5.\,1805 Weimar), \emph{Schriftsteller, Historiker, Philosoph}|pw}} und \textsc{Goethes\pwindex{Goethe, Johann Wolfgang von 28.\,8.\,1749 Frankfurt am Main – 22.\,3.\,1832 Weimar@\textsc{Goethe, Johann Wolfgang von} (28.\,8.\,1749 Frankfurt am Main – 22.\,3.\,1832 Weimar), \emph{Schriftsteller}|pw}}{ }Briefwechſel\pwindex{Schiller, Friedrich von 10.\,11.\,1759 Marbach am Neckar – 9.\,5.\,1805 Weimar@\textsc{Schiller, Friedrich von} (10.\,11.\,1759 Marbach am Neckar – 9.\,5.\,1805 Weimar), \emph{Schriftsteller, Historiker, Philosoph}!Briefwechsel zwischen Schiller und Goethe@\strich\emph{Briefwechsel zwischen Schiller und Goethe}|pwv}\pwindex{Goethe, Johann Wolfgang von 28.\,8.\,1749 Frankfurt am Main – 22.\,3.\,1832 Weimar@\textsc{Goethe, Johann Wolfgang von} (28.\,8.\,1749 Frankfurt am Main – 22.\,3.\,1832 Weimar), \emph{Schriftsteller}!Briefwechsel zwischen Schiller und Goethe@\strich\emph{Briefwechsel zwischen Schiller und Goethe}|pwv}. Bisher iſt er
               mir unſympathiſch, und \strikeout{\textcolor{gray}{ba}} beſonders der \textsc{Schiller\pwindex{Schiller, Friedrich von 10.\,11.\,1759 Marbach am Neckar – 9.\,5.\,1805 Weimar@\textsc{Schiller, Friedrich von} (10.\,11.\,1759 Marbach am Neckar – 9.\,5.\,1805 Weimar), \emph{Schriftsteller, Historiker, Philosoph}|pw}} langweilt mich mit{ }ſeinem verfluchten Theoretiſiren.\pend
           \pstart \label{T_L02785-1v}\edtext{Grüß’ Dich Gott, liebſter Freund!
               Schreib’ bald! Dein \spacefill\mbox{P. G.}}{\lemma{\textnormal{\emph{Grüß’ … G.}}}\Cendnote{\textnormal{seitlich am linken Rand}}}\label{T_L02785-1}\pend{}\selectlanguage{ngerman}\vspace{1em}{\vspace{1\baselineskip}}
\pstart
           \raggedleft{}{\pb}{[}hs. Thorel:{]} \begin{otherlanguage}{french}\label{K_L02785-7v}\edtext{Chez}{\lemma{\textnormal{\emph{Chez}}}\Cendnote{\textnormal{französisch: bei}}}\label{K_L02785-7}{ }Francis Vielé-Griffin\pwindex{Vielé-Griffin, Francis 26.\,5.\,1864 Norfolk – 11.\,12.\,1937 Frankreich@\textsc{Vielé-Griffin, Francis} (26.\,5.\,1864 Norfolk – 11.\,12.\,1937 Frankreich), \emph{Schriftsteller}|pw}\end{otherlanguage}\pend
           
\pstart
           \raggedleft{}\begin{otherlanguage}{french} au Château de
                        Nazelles\oindex{Château de Nazelles@\textbf{Château de Nazelles}, \emph{Hotel}|pw}\end{otherlanguage}\pend
           
\pstart
           \raggedleft{}\begin{otherlanguage}{french}(Indre-et-Loire\oindex{Indre-et-Loire@\textbf{Indre-et-Loire}, \emph{Verwaltungsgebiet}|pw})\end{otherlanguage}\pend
           
\pstart{}\begin{otherlanguage}{french}\label{K_L02785-8v}\edtext{Cher
                        Ami}{\lemma{\textnormal{\emph{Cher
                        Ami}}}\Cendnote{\textnormal{Lieber
                     Freund}}}\label{K_L02785-8},\end{otherlanguage}\pend\vspace{0.5em}
\pstart
           \label{K_L02785-9v}\edtext{\begin{otherlanguage}{french}La chose est donc convenue, aux conditions que vous dites:
                  cinq cent francs que vous me verserez aux premiers jours d’octobre. Et moi, je vais me mettre tout de suite à l’œuvre\pwindex{Schnitzler, Arthur 15.\,5.\,1862 Wien – 21.\,10.\,1931 ebd.@\textsc{Schnitzler, Arthur} (15.\,5.\,1862 Wien – 21.\,10.\,1931 ebd.), \emph{Schriftsteller, Mediziner}!Amourette. Pièce en trois actes. Adaptée de Arthur Schnitzler@\strich\emph{Amourette. Pièce en trois actes. Adaptée de Arthur Schnitzler}|pwv}, afin d’arriver en temps utile
                  pour profiter des chances de cette saison.\end{otherlanguage}}{\lemma{\textnormal{\emph{La … saison.}}}\Cendnote{\textnormal{französisch: Die Sache ist also
                  ausgemacht, zu den von Ihnen genannten Bedingungen: fünfhundert Francs, die Sie
                  mir in den ersten Oktobertagen auszahlen werden.
                  Und ich werde mich sofort an die Arbeit\pwindex{Schnitzler, Arthur 15.\,5.\,1862 Wien – 21.\,10.\,1931 ebd.@\textsc{Schnitzler, Arthur} (15.\,5.\,1862 Wien – 21.\,10.\,1931 ebd.), \emph{Schriftsteller, Mediziner}!Amourette. Pièce en trois actes. Adaptée de Arthur Schnitzler@\strich\emph{Amourette. Pièce en trois actes. Adaptée de Arthur Schnitzler}|pwkv} machen, damit die Gelegenheiten genützt werden können, die die
                  Saison bietet.}}}\label{K_L02785-9}\pend
           
\pstart
           \begin{otherlanguage}{french}\label{K_L02785-10v}\edtext{Pour achever
                  de préciser le côté affaire, et pour que vous pouissiez envoyer un engagement
                  signé de moi à M. Schnitzler, si vous le désirez, – je rappelle ici qu’il est bien
                  entendu que cette somme de cinq cents francs n’est qu’une avance sur les droits de
                  toute nature que pourra produire la traduction\pwindex{Schnitzler, Arthur 15.\,5.\,1862 Wien – 21.\,10.\,1931 ebd.@\textsc{Schnitzler, Arthur} (15.\,5.\,1862 Wien – 21.\,10.\,1931 ebd.), \emph{Schriftsteller, Mediziner}!Amourette. Pièce en trois actes. Adaptée de Arthur Schnitzler@\strich\emph{Amourette. Pièce en trois actes. Adaptée de Arthur Schnitzler}|pwv} de \uline{Liebelei\pwindex{Schnitzler, Arthur 15.\,5.\,1862 Wien – 21.\,10.\,1931 ebd.@\textsc{Schnitzler, Arthur} (15.\,5.\,1862 Wien – 21.\,10.\,1931 ebd.), \emph{Schriftsteller, Mediziner}!Liebelei. Schauspiel in drei Akten@\strich\emph{Liebelei. Schauspiel in drei Akten}|pw}}, droits de représentation, ou de publication en revue ou en librairie; – Et
                  pour les droits, il va de soi qu’ils seront partagés par moitiés égales entre M.
                  Schnitzler et moi}{\lemma{\textnormal{\emph{Pour … moi}}}\Cendnote{\textnormal{Um die
                        geschäftliche Seite zu präzisieren, und damit Sie, wenn Sie dies wünschen,
                        Herrn Schnitzler eine von mir
                        unterschriebene Verpflichtungserklärung schicken können, – halte ich sie
                        hier fest, um zu verdeutlichen, dass diese Summe von fünfhundert Francs nur
                        ein Vorschuss auf die Rechte jeglicher Art ist, die die Übersetzung\pwindex{Schnitzler, Arthur 15.\,5.\,1862 Wien – 21.\,10.\,1931 ebd.@\textsc{Schnitzler, Arthur} (15.\,5.\,1862 Wien – 21.\,10.\,1931 ebd.), \emph{Schriftsteller, Mediziner}!Amourette. Pièce en trois actes. Adaptée de Arthur Schnitzler@\strich\emph{Amourette. Pièce en trois actes. Adaptée de Arthur Schnitzler}|pwv} von \uline{Liebelei\pwindex{Schnitzler, Arthur 15.\,5.\,1862 Wien – 21.\,10.\,1931 ebd.@\textsc{Schnitzler, Arthur} (15.\,5.\,1862 Wien – 21.\,10.\,1931 ebd.), \emph{Schriftsteller, Mediziner}!Liebelei. Schauspiel in drei Akten@\strich\emph{Liebelei. Schauspiel in drei Akten}|pw}} mit sich bringt, wie Aufführungsrechte oder Veröffentlichungen in
                        Zeitschriften oder Buchhandlungen; – und die Rechte werden
                        selbstverständlich zu gleichen Teilen zwischen Herrn Schnitzler und mir geteilt.}}}\label{K_L02785-10}
                  –\end{otherlanguage}\pend
           
\pstart
           \label{K_L02785-11v}\edtext{\begin{otherlanguage}{french}Je rentrerai à Paris\oindex{Paris@\textbf{Paris}, \emph{Hauptstadt}|pw},
                  vers la fin de {\pb}septembre. Mon adreſse est à Nazelles\oindex{Château de Nazelles@\textbf{Château de Nazelles}, \emph{Hotel}|pwv} jusqu’au 14;{\\}et à partir du 15 elle sera (et moi aussi) chez madame Paul Bert\pwindex{Bert, Paul 19.\,10.\,1833 Auxerre – 11.\,11.\,1886 Hanoi@\textsc{Bert, Paul} (19.\,10.\,1833 Auxerre – 11.\,11.\,1886 Hanoi), \emph{Politiker, Physiologe}|pw}\pwindex{Clayton, Josephine 3.\,10.\,1846 Banff – 1916-06 Auxerre@\textsc{Clayton, Josephine} (3.\,10.\,1846 Banff – 1916-06 Auxerre)|pwv} à \uline{Auxerre}\oindex{Auxerre@\textbf{Auxerre}, \emph{Hauptstadt}|pw} (\uline{Yonne}\oindex{Yonne@\textbf{Yonne}, \emph{Verwaltungsgebiet}|pw})\end{otherlanguage}}{\lemma{\textnormal{\emph{Je … (Yonne)}}}\Cendnote{\textnormal{französisch: Ich werde gegen Ende September nach Paris\oindex{Paris@\textbf{Paris}, \emph{Hauptstadt}|pwk}
                  zurückkehren. Meine Adresse ist in Nazelles\oindex{Château de Nazelles@\textbf{Château de Nazelles}, \emph{Hotel}|pwkv} bis zum 14.; und
                  ab dem 15. ist sie (und ich auch) bei Madame Paul Bert\pwindex{Bert, Paul 19.\,10.\,1833 Auxerre – 11.\,11.\,1886 Hanoi@\textsc{Bert, Paul} (19.\,10.\,1833 Auxerre – 11.\,11.\,1886 Hanoi), \emph{Politiker, Physiologe}|pwk}\pwindex{Clayton, Josephine 3.\,10.\,1846 Banff – 1916-06 Auxerre@\textsc{Clayton, Josephine} (3.\,10.\,1846 Banff – 1916-06 Auxerre)|pwkv} in \uline{Auxerre}\oindex{Auxerre@\textbf{Auxerre}, \emph{Hauptstadt}|pwk} (\uline{Yonne}\oindex{Yonne@\textbf{Yonne}, \emph{Verwaltungsgebiet}|pwk}). }}}\label{K_L02785-11}\pend
           
\pstart
           \label{K_L02785-12v}\edtext{\begin{otherlanguage}{french}Votre bien dévoué\end{otherlanguage}}{\lemma{\textnormal{\emph{Votre bien dévoué}}}\Cendnote{\textnormal{französisch: Ihr sehr ergebener}}}\label{K_L02785-12}{ }{\\[\baselineskip]}\spacefill\mbox{Jean Thorel\pwindex{Thorel, Jean 11.\,9.\,1859 Éragny – 20.\,8.\,1916 Enghien-les-Bains@\textsc{Thorel, Jean} (11.\,9.\,1859 Éragny – 20.\,8.\,1916 Enghien-les-Bains), \emph{Übersetzer, Dramatiker}|pw}}\pend
           \leftskip=0em{}\selectlanguage{ngerman}\endnumbering\briefempfaengerindex{Schnitzler, Arthur@\textsc{Schnitzler, Arthur}!zzzGoldmann, Paul@\emph{von Paul Goldmann}!1896-09-221@{22. 9. [1896]}|)be}\mylabel{L02785h}  \newcommand{\dateiname}{L02785}\newcommand{\titel}{Paul Goldmann an Arthur Schnitzler, 22. 9. [1896]}\newcommand{\editorInnen}{Martin Anton Müller und Laura Untner}%% latex-leseansicht-abspann.tex
%% Abspann für die Leseansicht.
%% Der Schalter \ifkorrekturansicht ist bereits durch den Vorspann gesetzt.

%% latex-abspann.tex
%% Gemeinsamer Abspann für Korrekturansicht und Leseansicht.
%% Setzt den Schalter \ifkorrekturansicht voraus (gesetzt in den
%% einbindenden Dateien latex-korrekturansicht-abspann.tex bzw.
%% latex-leseansicht-abspann.tex).
%% ---------------------------------------------------------------

\normalsize

% Das esempio-Environment wird nur in der Leseansicht benötigt
\ifkorrekturansicht\else
\newenvironment{esempio}[3]%
{
    \vspace{1.5ex}
    \rlap{\underline{#1}}
    \par
    \setlength{\parindent}{0cm}
    \nopagebreak
    \leftskip=#2cm
    \rightskip=#3cm
}
{
    \par
}
\fi

\doendnotes{C}
\bigskip
\vfill

\clearpage

\footnotesize

\ifkorrekturansicht
  \lohead{\textsc{register}}
\fi

% theindex-Environment neu definieren ohne reledmac
\makeatletter
\renewenvironment{theindex}{%
  \ifkorrekturansicht
    \section*{\indexname}%
  \else
    \subsubsection*{Index der erwähnten Entitäten}%
  \fi
  \setlength{\parindent}{0pt}%
  \setlength{\parskip}{0pt plus 0.3pt}%
  \let\item\@idxitem
}{%
  \ifkorrekturansicht\clearpage\fi
}
\makeatother

\IfFileExists{\jobname-pw.ind}{\input{\jobname-pw.ind}}{}

% Quellenangabe nur in der Leseansicht
\ifkorrekturansicht\else
% Fallback-Definitionen, falls die .tex-Datei \titel etc. nicht gesetzt hat
\providecommand{\titel}{}
\providecommand{\editorInnen}{}
\providecommand{\dateiname}{\jobname}

\vspace{3cm}

\vfill

\footnotesize
\textsc{Quelle}: \titel. Herausgegeben von {\editorInnen}. In: \emph{Arthur Schnitzler: Briefwechsel mit Autorinnen und Autoren}.
 Digitale Edition, https://schnitzler-briefe.acdh.oeaw.ac.at/{\dateiname}.html (Stand \today)
\fi

\end{document}


