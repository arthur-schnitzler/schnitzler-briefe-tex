%% latex-leseansicht-vorspann.tex
%% Vorspann für die Leseansicht.
%% Lädt die gemeinsame Datei latex-vorspann.tex mit nicht gesetztem Schalter.

\newif\ifkorrekturansicht
\korrekturansichtfalse

\input{../tex-inputs/latex-vorspann}


         
         \newcommand{\erwaehntePersonen}{Personen: Paul Bert, Josephine Clayton, Alfred Dreyfus, Leo Ebermann, Jean-Louis Forain, Johann Wolfgang von Goethe, Karl Hillebrand, Lucien Millevoye, Friedrich von Schiller, Leopold Sonnemann, Jean Thorel, Francis Vielé-Griffin}
         \newcommand{\erwaehnteInstitutionen}{Institutionen: Frankfurter Zeitung}
         \newcommand{\erwaehnteOrte}{Orte: Auxerre, Berlin, Burgtheater, Chateau De Noyelles, Deutsches Theater Berlin, Frankreich, Indre-et-Loire, Paris, Wien, Yonne, rue Feydeau}
         \newcommand{\erwaehnteWerke}{Werke: Amourette. Pièce en trois actes. Adaptée de Arthur Schnitzler, Briefwechsel zwischen Schiller und Goethe, Die Athenerin, Die Enthüllungen über die Affaire Dreyfus, Frankfurter Zeitung, Frankreich und die Franzosen in der zweiten Hälfte des XIX. Jahrhunderts: Eindrücke und Erfahrungen, Freiwild. Schauspiel in 3 Akten, Liebelei. Schauspiel in drei Akten}
               \section[ Paul Goldmann an Arthur Schnitzler, 22. 9. {[}1896{]}]{ Paul Goldmann an Arthur Schnitzler, 22. 9. {[}1896{]}}\nopagebreak\mylabel{v}\rehead{ }\begin{ledgroupsized}[t]{13cm}\normalsize\beginnumbering \toendnotes[C]{\smallbreak\pagebreak[2]} \Standort{DLA, A:Schnitzler, HS.NZ85.1.3166.}
\physDesc{Brief, 1 Blatt, 4 Seiten
\newline{}Handschrift: blaue Tinte, deutsche Kurrent\newline{}Beilage: handschriftlicher Brief: 1 Blatt, 2 Seiten, lila (evtl.
                                 ursprünglich schwarze?) Tinte, lateinische Kurrent; mit Bleistift
                                 Vermerk des Datums von Schnitzler\pwindex{Schnitzler, Arthur 15.05.1862 – 21.10.1931@\textsc{Schnitzler, Arthur} (15.05.1862 – 21.10.1931), \emph{Schriftsteller, Mediziner}|pw} mit »Sept{[}ember{]} 96« 
\newline{}Schnitzler: 1) mit Bleistift das Jahr »96« vermerkt  2) mit rotem Buntstift drei Unterstreichungen}\toendnotes[C]{\smallbreak}\pstart
           \noindent{}{\pb}\textcolor{gray}{\textbf{\textbf{Frankfurter Zeitung\orgindex{Frankfurter Zeitung@Frankfurter Zeitung|pw}}}}\pend
           \pstart
           \textcolor{gray}{\textbf{(\begin{otherlanguage}{french}Gazette de Francfort\end{otherlanguage}\orgindex{Frankfurter Zeitung@Frankfurter Zeitung|pw}).}}\pend
           \pstart
           \textcolor{gray}{\textbf{\textbf{\begin{otherlanguage}{french}Fondateur M.\end{otherlanguage}{ }L. Sonnemann\pwindex{Sonnemann, Leopold 1831-10-29 – 1909-10-30@\textsc{Sonnemann, Leopold} (1831-10-29 – 1909-10-30), \emph{Journalist, Herausgeber}|pw}.}}}\pend
           \pstart
           \begin{otherlanguage}{french}\textcolor{gray}{\textbf{Journal\pwindex{?? Werk@Nicht ermittelte Verfasserinnen und Verfasser!Frankfurter Zeitung1856 – 1943@\emph{Frankfurter Zeitung} {[}1856 – 1943{]}|pwv} politique,
                        financier,}}\end{otherlanguage}\pend
           \pstart
           \begin{otherlanguage}{french}\textcolor{gray}{\textbf{commercial et littéraire.}}\end{otherlanguage}\pend
           \pstart
           \begin{otherlanguage}{french}\textcolor{gray}{\textbf{\textbf{Paraissant trois fois par jour.}}}\end{otherlanguage}\hfill \textsc{Paris\oindex{Paris@\textbf{Paris}|pw}}, 22. September.\pend
           \pstart
           \begin{otherlanguage}{french}\textcolor{gray}{\textbf{\textbf{Bureau à Paris\oindex{Paris@\textbf{Paris}|pw}}}}\end{otherlanguage}\pend
           \pstart
           \begin{otherlanguage}{french}\textcolor{gray}{\textbf{\textbf{24. Rue Feydeau\oindex{rue Feydeau@\textbf{rue Feydeau}|pw}.}}}\end{otherlanguage}\pend
           \pstart\center{}Mein lieber Freund,\pend\pstart
           Ich habe in dieſen Tagen ungewöhnlich viel zu thun gehabt. Auch gab es allerlei
               Aufregungen. Man \label{K_L02785-1v}\edtext{beſchimpft mich in
               der hieſigen Preſſe und verlangt meine Ausweiſung}{\lemma{\textnormal{\emph{beſchimpft … Ausweiſung}}}\Cendnote{\textnormal{Goldmann\pwindex{Goldmann, Paul 31.01.1865 – 25.09.1935@\textsc{Goldmann, Paul} (31.01.1865 – 25.09.1935), \emph{Schriftsteller, Journalist}|pwk}s Berichterstattung\pwindex{Enthuellungen ueber die Affaire Dreyfus1896-09-16@\emph{Die Enthüllungen über die Affaire Dreyfus} {[}1896-09-16{]}|pwkv} über die Dreyfus\pwindex{Dreyfus, Alfred 1859-10-09 – 1935-07-12@\textsc{Dreyfus, Alfred} (1859-10-09 – 1935-07-12), \emph{Militär}|pwk}-Affäre (G.\pwindex{Goldmann, Paul 31.01.1865 – 25.09.1935@\textsc{Goldmann, Paul} (31.01.1865 – 25.09.1935), \emph{Schriftsteller, Journalist}|pwk} [=Paul Goldmann\pwindex{Goldmann, Paul 31.01.1865 – 25.09.1935@\textsc{Goldmann, Paul} (31.01.1865 – 25.09.1935), \emph{Schriftsteller, Journalist}|pwk}]: \emph{Die Enthüllungen über
                        die Affaire Dreyfus}\pwindex{Enthuellungen ueber die Affaire Dreyfus1896-09-16@\emph{Die Enthüllungen über die Affaire Dreyfus} {[}1896-09-16{]}|pwk}. In: \emph{Frankfurter
                        Zeitung}\pwindex{?? Werk@Nicht ermittelte Verfasserinnen und Verfasser!Frankfurter Zeitung1856 – 1943@\emph{Frankfurter Zeitung} {[}1856 – 1943{]}|pwk}, Jg. 41, Nr. 258, 16. 9. 1896, Erstes Morgenblatt,
                     S. 1) wurde als Einmischung empfunden. In weiterer Folge führte das zu
                  einem Pistolenduell zwischen Goldmann\pwindex{Goldmann, Paul 31.01.1865 – 25.09.1935@\textsc{Goldmann, Paul} (31.01.1865 – 25.09.1935), \emph{Schriftsteller, Journalist}|pwk} und
                  dem antisemitischen Chefredakteur Lucien
                     Millevoye\pwindex{Millevoye, Lucien 1850-08-01 – 1918-03-25@\textsc{Millevoye, Lucien} (1850-08-01 – 1918-03-25), \emph{Politiker, Journalist}|pwk}, das am 21. 11. 1896 stattfand (siehe Arthur Schnitzler an Paul Goldmann, 21. 11. 1896). Die publizierten
                  Invektiven gegen Goldmann\pwindex{Goldmann, Paul 31.01.1865 – 25.09.1935@\textsc{Goldmann, Paul} (31.01.1865 – 25.09.1935), \emph{Schriftsteller, Journalist}|pwk} aus dem September 1896, auf die er hier Bezug nimmt, konnten
                  bislang nicht belegt werden.}}}\label{K_L02785-1h}, weil ich \strikeout{\textcolor{gray}{von}} für die Unſchuld des \textsc{Dreyfus\pwindex{Dreyfus, Alfred 1859-10-09 – 1935-07-12@\textsc{Dreyfus, Alfred} (1859-10-09 – 1935-07-12), \emph{Militär}|pw}} eingetreten bin, von der ich, nach den neueſten Enthüllungen, feſter als je
               überzeugt bin. Zudem geht in meiner Familie Alles drunter und drüber. Kurzum ich weiß
               nicht recht, wo mir {\pb}der Kopf ſteht.\pend
           \pstart
           Dies um mich zu entſchuldigen, daß ich \strikeout{D} beifolgenden
               Brief von \textsc{Thorel\pwindex{Thorel, Jean 1859-09-11 – 1916-08-20@\textsc{Thorel, Jean} (1859-09-11 – 1916-08-20), \emph{Übersetzer, Dramatiker}|pw}} ſolange liegen ließ. Heb’ ihn Dir gut auf, denn, wie Du aus ſeinem Inhalt
               erſiehſt, vertritt er die Stelle eines Contracts. Ich habe ihn unter irgend einem
               Vorwand von 6 auf 500 heruntergeſchraubt und habe mir ausdrücklich ausbedungen, daß
               dieſe Zahlung nur als Vorſchuß auf etwaige {\pb}\strikeout{\textcolor{gray}{×}}{ }\textsc{Tantièmen} oder Honorare zu betrachten iſt. Ich fürchte
               allerdings, daß letztere Clauſel platoniſch bleiben dürfte. Nun kannſt Du das Geld
               dieſer Tage an mich ſchicken, wenn Du willſt (aber nicht wieder \label{K_L02785-2v}\edtext{in Goldſtücken in einem recommandirten
                  Brief}{\lemma{\textnormal{\emph{in … Brief}}}\Cendnote{\textnormal{siehe Paul Goldmann an Arthur Schnitzler, 21. 12. [1895]}}}\label{K_L02785-2h}). Ich werde bei dieſem Geſchäft leider nichts verdienen können, aber Du
               brauchſt hoffentlich bald wieder ein \label{K_L02785-3v}\edtext{Opernglas}{\lemma{\textnormal{\emph{Opernglas}}}\Cendnote{\textnormal{siehe Paul Goldmann an Arthur Schnitzler, 11. 1. [1896]}}}\label{K_L02785-3h}.\pend
           \pstart
           Bei \textsc{Forain\pwindex{Forain, Jean-Louis 1852-10-23 – 1931-07-11@\textsc{Forain, Jean-Louis} (1852-10-23 – 1931-07-11), \emph{Maler, Grafiker, Karikaturist}|pw}} war ich auch, aber er iſt noch auf {\pb}dem
               Lande.\pend
           \pstart
           Was gibts \strikeout{es} Neues bei Dir? Leben und Dichten?
                  \label{K_L02785-4v}\edtext{Was hörſt Du von Berlin\oindex{Berlin@\textbf{Berlin}|pw} und wann gehſt Du hin?}{\lemma{\textnormal{\emph{Was … hin?}}}\Cendnote{\textnormal{Goldmann\pwindex{Goldmann, Paul 31.01.1865 – 25.09.1935@\textsc{Goldmann, Paul} (31.01.1865 – 25.09.1935), \emph{Schriftsteller, Journalist}|pwk} bezieht sich auf die bevorstehende
                  Uraufführung des Dreiakters \emph{Freiwild}\pwindex{Schnitzler, Arthur 15.05.1862 – 21.10.1931@\textsc{Schnitzler, Arthur} (15.05.1862 – 21.10.1931), \emph{Schriftsteller, Mediziner}!Freiwild. Schauspiel in 3 Akten1896@\strich\emph{Freiwild. Schauspiel in 3 Akten} {[}1896{]}|pwk} am 3. 11. 1896 am Deutschen Theater\oindex{Deutsches Theater Berlin@\textbf{Deutsches Theater Berlin}|pwk} in Berlin\oindex{Berlin@\textbf{Berlin}|pwk}. Siehe dazu vor allem \emph{Der Briefwechsel Arthur Schnitzler — Otto Brahm}.
                     Vollständige Ausgabe. Herausgegeben, eingeleitet und erläutert von Oskar
                     Seidlin. Tübingen: \emph{Niemeyer}{ }1975, S. 14–28.{ }Schnitzler\pwindex{Schnitzler, Arthur 15.05.1862 – 21.10.1931@\textsc{Schnitzler, Arthur} (15.05.1862 – 21.10.1931), \emph{Schriftsteller, Mediziner}|pwk} war dafür zwischen 26. 10. 1896 und 9. 11. 1896 in Berlin\oindex{Berlin@\textbf{Berlin}|pwk}.}}}\label{K_L02785-4h}{ }\label{K_L02785-6v}\edtext{\textsc{Ebermann\pwindex{Ebermann, Leo 16.07.1863 – 09.10.1914@\textsc{Ebermann, Leo} (16.07.1863 – 09.10.1914), \emph{Schriftsteller, Journalist, Rechtswissenschaftler}!Athenerin19. 9. 1896@\strich\emph{Die Athenerin} {[}19. 9. 1896{]}|pwv}\pwindex{Ebermann, Leo 16.07.1863 – 09.10.1914@\textsc{Ebermann, Leo} (16.07.1863 – 09.10.1914), \emph{Schriftsteller, Journalist, Rechtswissenschaftler}|pw}}}{\lemma{\textnormal{\emph{Ebermann}}}\Cendnote{\textnormal{Schnitzler\pwindex{Schnitzler, Arthur 15.05.1862 – 21.10.1931@\textsc{Schnitzler, Arthur} (15.05.1862 – 21.10.1931), \emph{Schriftsteller, Mediziner}|pwk} war nicht nur bei Proben von Leo Ebermann\pwindex{Ebermann, Leo 16.07.1863 – 09.10.1914@\textsc{Ebermann, Leo} (16.07.1863 – 09.10.1914), \emph{Schriftsteller, Journalist, Rechtswissenschaftler}|pwk}s Stück \emph{Die Athenerin}\pwindex{Ebermann, Leo 16.07.1863 – 09.10.1914@\textsc{Ebermann, Leo} (16.07.1863 – 09.10.1914), \emph{Schriftsteller, Journalist, Rechtswissenschaftler}!Athenerin19. 9. 1896@\strich\emph{Die Athenerin} {[}19. 9. 1896{]}|pwk} anwesend, sondern besuchte am 19. 9. 1896 auch die
                  Uraufführung im Burgtheater\oindex{Burgtheater@\textbf{Burgtheater}|pwk}. Siehe zum Erfolg
                  des Stück\pwindex{Ebermann, Leo 16.07.1863 – 09.10.1914@\textsc{Ebermann, Leo} (16.07.1863 – 09.10.1914), \emph{Schriftsteller, Journalist, Rechtswissenschaftler}!Athenerin19. 9. 1896@\strich\emph{Die Athenerin} {[}19. 9. 1896{]}|pwkv}s etwa auch Arthur Schnitzler an Richard Beer-Hofmann, 21. 9. 1896 und A. S.: \emph{Tagebuch}, 22. 9. 1896.}}}\label{K_L02785-6h} ſcheint ja wohl einen großen Erfolg gehabt zu
               haben?\pend
           \pstart
           Lies \label{K_L02785-7v}\edtext{\textsc{Karl Hillebrand\pwindex{Hillebrand, Karl 1861-12-26 – 1939-01-10@\textsc{Hillebrand, Karl} (1861-12-26 – 1939-01-10), \emph{Astronom}|pw}}: Frankreich und die Franzoſen\pwindex{Hillebrand, Karl 1861-12-26 – 1939-01-10@\textsc{Hillebrand, Karl} (1861-12-26 – 1939-01-10), \emph{Astronom}!Frankreich und die Franzosen in der zweiten Haelfte des XIX. Jahrhunderts: Eindruecke und Erfahrungen1873@\strich\emph{Frankreich und die Franzosen in der zweiten Hälfte des XIX. Jahrhunderts: Eindrücke und Erfahrungen} {[}1873{]}|pw}}{\lemma{\textnormal{\emph{Karl … Franzoſen}}}\Cendnote{\textnormal{Lektüre durch Schnitzler\pwindex{Schnitzler, Arthur 15.05.1862 – 21.10.1931@\textsc{Schnitzler, Arthur} (15.05.1862 – 21.10.1931), \emph{Schriftsteller, Mediziner}|pwk} nicht bekannt}}}\label{K_L02785-7h}. Der einzige Deutſche, der
                  Frankreich\oindex{Frankreich@\textbf{Frankreich}|pw} kennt, – und eine Perſönlichkeit\pwindex{Hillebrand, Karl 1861-12-26 – 1939-01-10@\textsc{Hillebrand, Karl} (1861-12-26 – 1939-01-10), \emph{Astronom}|pwv}. Ich leſe \textsc{Schiller\pwindex{Schiller, Friedrich von 10.11.1759 – 09.05.1805@\textsc{Schiller, Friedrich von} (10.11.1759 – 09.05.1805), \emph{Schriftsteller, Historiker, Philosoph}|pw}s} und \textsc{Goethe\pwindex{Goethe, Johann Wolfgang von 1749-08-28 – 1832-03-22@\textsc{Goethe, Johann Wolfgang von} (1749-08-28 – 1832-03-22), \emph{Schriftsteller}|pw}s}{ }Briefwechſel\pwindex{Schiller, Friedrich von 10.11.1759 – 09.05.1805@\textsc{Schiller, Friedrich von} (10.11.1759 – 09.05.1805), \emph{Schriftsteller, Historiker, Philosoph}!Briefwechsel zwischen Schiller und Goethe1791 – 1805@\strich\emph{Briefwechsel zwischen Schiller und Goethe} {[}1791 – 1805{]}|pwv}\pwindex{Goethe, Johann Wolfgang von 1749-08-28 – 1832-03-22@\textsc{Goethe, Johann Wolfgang von} (1749-08-28 – 1832-03-22), \emph{Schriftsteller}!Briefwechsel zwischen Schiller und Goethe1791 – 1805@\strich\emph{Briefwechsel zwischen Schiller und Goethe} {[}1791 – 1805{]}|pwv}. Bisher iſt er
               mir unſympathiſch, und \strikeout{\textcolor{gray}{ba}} beſonders der \textsc{Schiller\pwindex{Schiller, Friedrich von 10.11.1759 – 09.05.1805@\textsc{Schiller, Friedrich von} (10.11.1759 – 09.05.1805), \emph{Schriftsteller, Historiker, Philosoph}|pw}} langweilt mich mit ſeinem verfluchten Theoretiſiren.\pend
           \pstart \label{T_L02785-1v}\edtext{Grüß’ Dich Gott, liebſter Freund!
               Schreib’ bald! Dein \spacefill\mbox{P. G.}}{\lemma{\textnormal{\emph{Grüß’ … G.}}}\Cendnote{\textnormal{seitlich am linken Rand}}}\label{T_L02785-1h}\pend{}{\bigskip}\pstart
           \raggedleft{}{\pb}{[}hs. Thorel:{]} \begin{otherlanguage}{french}\label{K_L02785-44v}\edtext{Chez Francis Vielé-Griffin\pwindex{Viele-Griffin, Francis 1864-05-26 – 1937-12-11@\textsc{Vielé-Griffin, Francis} (1864-05-26 – 1937-12-11), \emph{Schriftsteller}|pw}}{\lemma{\textnormal{\emph{Chez … Vielé-Griffin}}}\Cendnote{\textnormal{französisch: Bei Francis Vielé-Griffin\pwindex{Viele-Griffin, Francis 1864-05-26 – 1937-12-11@\textsc{Vielé-Griffin, Francis} (1864-05-26 – 1937-12-11), \emph{Schriftsteller}|pwk}}}}\label{K_L02785-44h}\end{otherlanguage}\pend
           \pstart
           \raggedleft{}\begin{otherlanguage}{french} au château de
                        Noyelles\oindex{Chateau De Noyelles@\textbf{Chateau De Noyelles}|pw}\end{otherlanguage}\pend
           \pstart
           \noindent{}\raggedleft{}\begin{otherlanguage}{french}(Indre-et-Loire\oindex{Indre-et-Loire@\textbf{Indre-et-Loire}|pw})\end{otherlanguage}\pend
           \pstart{}\label{K_L02785-55v}\edtext{\begin{otherlanguage}{french}Cher Ami,\end{otherlanguage}}{\lemma{\textnormal{\emph{Cher Ami,}}}\Cendnote{\textnormal{Lieber Freund!}}}\label{K_L02785-55h}\pend\pstart
           \label{K_L02785-23v}\edtext{\begin{otherlanguage}{french}La chose est donc convenue, aux conditions que vous dites:
                  cinq cent francs que vous me verserez aux premiers jours d’octobre. Et moi, je vais me mettre tout de suite à l’œuvre\pwindex{Thorel, Jean 1859-09-11 – 1916-08-20@\textsc{Thorel, Jean} (1859-09-11 – 1916-08-20), \emph{Übersetzer, Dramatiker}!Amourette. Piece en trois actes. Adaptee de Arthur Schnitzler1897@\strich\emph{Amourette. Pièce en trois actes. Adaptée de Arthur Schnitzler} {[}Übersetzung, 1897{]}|pwv}, afin d’arriver en temps utile
                  pour profiter des chances de cette saison.\end{otherlanguage}}{\lemma{\textnormal{\emph{La … saison.}}}\Cendnote{\textnormal{französisch: Die Sache ist also
                  ausgemacht, zu den von Ihnen genannten Bedingungen: fünfhundert Francs, die Sie
                  mir in den ersten Oktobertagen auszahlen werden.
                  Und ich werde mich sofort an die Arbeit\pwindex{Thorel, Jean 1859-09-11 – 1916-08-20@\textsc{Thorel, Jean} (1859-09-11 – 1916-08-20), \emph{Übersetzer, Dramatiker}!Amourette. Piece en trois actes. Adaptee de Arthur Schnitzler1897@\strich\emph{Amourette. Pièce en trois actes. Adaptée de Arthur Schnitzler} {[}Übersetzung, 1897{]}|pwkv} machen, damit die Gelegenheiten genützt werden können, die die
                  Saison bietet.}}}\label{K_L02785-23h}\pend
           \pstart
           \label{K_L02785-66v}\edtext{\begin{otherlanguage}{french}Pour acter et de préciser le côté affaire, et pour que vous
                  pourriez envoyer un engagement signé de moi à M. Schnitzler, si vous le désirez, –
                  je rappelle ici qu’il est bien entendu que cette somme de cinq cents francs n’est
                  qu’une avance sur les droits de toute nature que pourra produire la traduction\pwindex{Thorel, Jean 1859-09-11 – 1916-08-20@\textsc{Thorel, Jean} (1859-09-11 – 1916-08-20), \emph{Übersetzer, Dramatiker}!Amourette. Piece en trois actes. Adaptee de Arthur Schnitzler1897@\strich\emph{Amourette. Pièce en trois actes. Adaptée de Arthur Schnitzler} {[}Übersetzung, 1897{]}|pwv} de \uline{Liebelei\pwindex{Schnitzler, Arthur 15.05.1862 – 21.10.1931@\textsc{Schnitzler, Arthur} (15.05.1862 – 21.10.1931), \emph{Schriftsteller, Mediziner}!Liebelei. Schauspiel in drei Akten1895-10-09@\strich\emph{Liebelei. Schauspiel in drei Akten} {[}1895-10-09{]}|pw}}, droits de représentation, ou de publication en revue ou en librairie; – Et
                  pour les droits, il va de soi qu’ils seront partagés par moitiés égales entre M.
                  Schnitzler et moi –\end{otherlanguage}}{\lemma{\textnormal{\emph{Pour … –}}}\Cendnote{\textnormal{Um vorwärtszukommen und die
                  geschäftliche Seite zu präzisieren, und damit Sie, wenn Sie dies wünschen, Herrn
                     Schnitzler\pwindex{Schnitzler, Arthur 15.05.1862 – 21.10.1931@\textsc{Schnitzler, Arthur} (15.05.1862 – 21.10.1931), \emph{Schriftsteller, Mediziner}|pwk} eine von mir unterschriebene
                  Verpflichtungserklärung schicken können, – halte ich sie hier fest, um zu
                  verdeutlichen, dass diese Summe von fünfhundert Francs nur ein Vorschuss auf die
                  Rechte jeglicher Art ist, die die Übersetzung\pwindex{Thorel, Jean 1859-09-11 – 1916-08-20@\textsc{Thorel, Jean} (1859-09-11 – 1916-08-20), \emph{Übersetzer, Dramatiker}!Amourette. Piece en trois actes. Adaptee de Arthur Schnitzler1897@\strich\emph{Amourette. Pièce en trois actes. Adaptée de Arthur Schnitzler} {[}Übersetzung, 1897{]}|pwkv} der \uline{\emph{Liebelei}\pwindex{Schnitzler, Arthur 15.05.1862 – 21.10.1931@\textsc{Schnitzler, Arthur} (15.05.1862 – 21.10.1931), \emph{Schriftsteller, Mediziner}!Liebelei. Schauspiel in drei Akten1895-10-09@\strich\emph{Liebelei. Schauspiel in drei Akten} {[}1895-10-09{]}|pwk}} mit sich bringt, wie Aufführungsrechte oder Veröffentlichungen in
                  Zeitschriften oder Buchhandlungen; – Und die Rechte werden selbstverständlich zu
                  gleichen Teilen zwischen Herrn Schnitzler\pwindex{Schnitzler, Arthur 15.05.1862 – 21.10.1931@\textsc{Schnitzler, Arthur} (15.05.1862 – 21.10.1931), \emph{Schriftsteller, Mediziner}|pwk}
                  und mir geteilt. }}}\label{K_L02785-66h}\pend
           \pstart
           \label{K_L02785-99v}\edtext{\begin{otherlanguage}{french}Je rentrerai à Paris\oindex{Paris@\textbf{Paris}|pw},
                  vers la fin de {\pb}septembre. Mon adreſse est: Noyelles\oindex{Chateau De Noyelles@\textbf{Chateau De Noyelles}|pwv} jusqu’au 14;{\\}et à partir du 15 elle sera (et moi aussi) chez madame Paul Bert\pwindex{Bert, Paul 1833-10-19 – 1886-11-11@\textsc{Bert, Paul} (1833-10-19 – 1886-11-11), \emph{Politiker, Physiologe}|pw}\pwindex{Clayton, Josephine 1846-10-03 – 1916-06@\textsc{Clayton, Josephine} (1846-10-03 – 1916-06)|pwv} à \uline{Auxerre}\oindex{Auxerre@\textbf{Auxerre}|pw} (\uline{Yonne}\oindex{Yonne@\textbf{Yonne}|pw})\end{otherlanguage}}{\lemma{\textnormal{\emph{Je … (Yonne)}}}\Cendnote{\textnormal{französisch: Ich werde gegen Ende September nach Paris\oindex{Paris@\textbf{Paris}|pwk}
                  zurückkehren. Meine Adresse ist: Noyelles\oindex{Chateau De Noyelles@\textbf{Chateau De Noyelles}|pwkv} bis zum 14.; und
                  ab dem 15. ist sie (und ich auch) bei Madame Paul Bert\pwindex{Bert, Paul 1833-10-19 – 1886-11-11@\textsc{Bert, Paul} (1833-10-19 – 1886-11-11), \emph{Politiker, Physiologe}|pwk}\pwindex{Clayton, Josephine 1846-10-03 – 1916-06@\textsc{Clayton, Josephine} (1846-10-03 – 1916-06)|pwkv} in \uline{Auxerre}\oindex{Auxerre@\textbf{Auxerre}|pwk} (\uline{Yonne}\oindex{Yonne@\textbf{Yonne}|pwk}). }}}\label{K_L02785-99h}\pend
           \pstart
           \label{K_L02785-22v}\edtext{\begin{otherlanguage}{french}Votre bien
                  dévoué\end{otherlanguage}}{\lemma{\textnormal{\emph{Votre bien
                  dévoué}}}\Cendnote{\textnormal{französisch: Ihr
                  sehr ergebener}}}\label{K_L02785-22h}{ }{\\[\baselineskip]}\spacefill\mbox{Jean Thorel\pwindex{Thorel, Jean 1859-09-11 – 1916-08-20@\textsc{Thorel, Jean} (1859-09-11 – 1916-08-20), \emph{Übersetzer, Dramatiker}|pw}}\pend
           \leftskip=0em{}
         
         \endnumbering\mylabel{h}\end{ledgroupsized}  \newcommand{\dateiname}{L02785}\newcommand{\titel}{Paul Goldmann an Arthur Schnitzler, 22. 9. [1896]}\newcommand{\editorInnen}{Martin Anton Müller und Laura Untner}%% latex-leseansicht-abspann.tex
%% Abspann für die Leseansicht.
%% Der Schalter \ifkorrekturansicht ist bereits durch den Vorspann gesetzt.

%% latex-abspann.tex
%% Gemeinsamer Abspann für Korrekturansicht und Leseansicht.
%% Setzt den Schalter \ifkorrekturansicht voraus (gesetzt in den
%% einbindenden Dateien latex-korrekturansicht-abspann.tex bzw.
%% latex-leseansicht-abspann.tex).
%% ---------------------------------------------------------------

\normalsize

% Das esempio-Environment wird nur in der Leseansicht benötigt
\ifkorrekturansicht\else
\newenvironment{esempio}[3]%
{
    \vspace{1.5ex}
    \rlap{\underline{#1}}
    \par
    \setlength{\parindent}{0cm}
    \nopagebreak
    \leftskip=#2cm
    \rightskip=#3cm
}
{
    \par
}
\fi

\doendnotes{C}
\bigskip
\vfill

\clearpage

\footnotesize

\ifkorrekturansicht
  \lohead{\textsc{register}}
\fi

% theindex-Environment neu definieren ohne reledmac
\makeatletter
\renewenvironment{theindex}{%
  \ifkorrekturansicht
    \section*{\indexname}%
  \else
    \subsubsection*{Index der erwähnten Entitäten}%
  \fi
  \setlength{\parindent}{0pt}%
  \setlength{\parskip}{0pt plus 0.3pt}%
  \let\item\@idxitem
}{%
  \ifkorrekturansicht\clearpage\fi
}
\makeatother

\IfFileExists{\jobname-pw.ind}{\input{\jobname-pw.ind}}{}

% Quellenangabe nur in der Leseansicht
\ifkorrekturansicht\else
% Fallback-Definitionen, falls die .tex-Datei \titel etc. nicht gesetzt hat
\providecommand{\titel}{}
\providecommand{\editorInnen}{}
\providecommand{\dateiname}{\jobname}

\vspace{3cm}

\vfill

\footnotesize
\textsc{Quelle}: \titel. Herausgegeben von {\editorInnen}. In: \emph{Arthur Schnitzler: Briefwechsel mit Autorinnen und Autoren}.
 Digitale Edition, https://schnitzler-briefe.acdh.oeaw.ac.at/{\dateiname}.html (Stand \today)
\fi

\end{document}


      