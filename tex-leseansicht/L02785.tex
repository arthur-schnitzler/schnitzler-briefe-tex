%% latex-korrekturansicht-vorspann.tex
%% Vorspann für die Korrekturansicht.
%% Lädt die gemeinsame Datei latex-vorspann.tex mit gesetztem Schalter.

\newif\ifkorrekturansicht
\korrekturansichttrue

\input{../tex-inputs/latex-vorspann}


\section[ Paul Goldmann an Arthur Schnitzler, 22. 9. {[}1896{]}]{L02785 Paul Goldmann an Arthur Schnitzler, 22. 9. {[}1896{]}}
\nopagebreak\mylabel{L02785v}
\rehead{ }\normalsize\beginnumbering\briefempfaengerindex{Schnitzler, Arthur@\textsc{Schnitzler, Arthur}!zzzGoldmann, Paul@\emph{von Paul Goldmann}!1896-09-221@{22. 9. {[}1896{]}}|(be}
\toendnotes[C]{\smallbreak\pagebreak[2]}\Standort{DLA, A:Schnitzler, HS.NZ85.1.3166.}
\physDesc{Brief, 1 Blatt, 4 Seiten, 2680 Zeichen
\newline{}Handschrift: blaue Tinte, deutsche Kurrent
\newline{}Beilage: handschriftlicher Brief: 1 Blatt, 2 Seiten, lila (evtl.
                                 ursprünglich schwarze?) Tinte, lateinische Kurrent; mit Bleistift
                                 Vermerk des Datums von Schnitzler mit »Sept{[}ember{]} 96« 
\newline{}Schnitzler: 1) mit Bleistift das Jahr »96« vermerkt  2) mit rotem Buntstift drei Unterstreichungen}\toendnotes[C]{\smallbreak}
\pstart
           {\pb}\textcolor{gray}{\textbf{\textbf{Frankfurter Zeitung\orgindex{Frankfurter Zeitung@Frankfurter Zeitung|pw}}}}\pend
           
\pstart
           \textcolor{gray}{\textbf{(\begin{otherlanguage}{french}Gazette de Francfort\end{otherlanguage}\orgindex{Frankfurter Zeitung@Frankfurter Zeitung|pw}).}}\pend
           
\pstart
           \textcolor{gray}{\textbf{\textbf{\begin{otherlanguage}{french}Fondateur M.\end{otherlanguage}{ }L. Sonnemann\pwindex{Sonnemann, Leopold 1831-10-29 – 1909-10-30@\textsc{Sonnemann, Leopold} (1831-10-29 – 1909-10-30), \emph{Journalist/Journalistin, Herausgeber/Herausgeberin}|pw}.}}}\pend
           
\pstart
           \begin{otherlanguage}{french}\textcolor{gray}{\textbf{Journal\pwindex{Frankfurter Zeitung@\emph{Frankfurter Zeitung}|pwv} politique,
                        financier,}}\end{otherlanguage}\pend
           
\pstart
           \begin{otherlanguage}{french}\textcolor{gray}{\textbf{commercial et littéraire.}}\end{otherlanguage}\pend
           
\pstart
           \begin{otherlanguage}{french}\textcolor{gray}{\textbf{\textbf{Paraissant trois fois par jour.}}}\end{otherlanguage}\hfill \textsc{Paris\oindex{Paris@\textbf{Paris}, \emph{P.PPLC}|pw}}, 22. September.\pend
           
\pstart
           \begin{otherlanguage}{french}\textcolor{gray}{\textbf{\textbf{Bureau à Paris\oindex{Paris@\textbf{Paris}, \emph{P.PPLC}|pw}}}}\end{otherlanguage}\pend
           
\pstart
           \begin{otherlanguage}{french}\textcolor{gray}{\textbf{\textbf{24. Rue Feydeau\oindex{rue Feydeau@\textbf{rue Feydeau}, \emph{Straße (K.STR)}|pw}.}}}\end{otherlanguage}\pend
           
\pstart\center{}Mein lieber Freund,\pend\vspace{0.5em}
\pstart
           Ich habe in dieſen Tagen ungewöhnlich viel zu thun gehabt. Auch gab es allerlei
               Aufregungen. Man \label{K_L02785-1v}\edtext{beſchimpft mich in
               der hieſigen Preſſe und verlangt meine Ausweiſung}{\lemma{\textnormal{\emph{beſchimpft … Ausweiſung}}}\Cendnote{\textnormal{Goldmanns\pwindex{Goldmann, Paul 31.01.1865 – 25.09.1935@\textsc{Goldmann, Paul} (31.01.1865 – 25.09.1935), \emph{Schriftsteller/Schriftstellerin, Journalist/Journalistin}|pwk}{ }Berichterstattung\pwindex{Enthuellungen ueber die Affaire Dreyfus@\emph{Die Enthüllungen über die Affaire Dreyfus}|pwkv} über die Dreyfus\pwindex{Dreyfus, Alfred 1859-10-09 – 1935-07-12@\textsc{Dreyfus, Alfred} (1859-10-09 – 1935-07-12), \emph{Militär/Militärin}|pwk}-Affäre (G.\pwindex{Goldmann, Paul 31.01.1865 – 25.09.1935@\textsc{Goldmann, Paul} (31.01.1865 – 25.09.1935), \emph{Schriftsteller/Schriftstellerin, Journalist/Journalistin}|pwk} [ = Paul Goldmann\pwindex{Goldmann, Paul 31.01.1865 – 25.09.1935@\textsc{Goldmann, Paul} (31.01.1865 – 25.09.1935), \emph{Schriftsteller/Schriftstellerin, Journalist/Journalistin}|pwk}]: \emph{Die
                        Enthüllungen über die Affaire Dreyfus}\pwindex{Enthuellungen ueber die Affaire Dreyfus@\emph{Die Enthüllungen über die Affaire Dreyfus}|pwk}. In: \emph{Frankfurter Zeitung}\pwindex{Frankfurter Zeitung@\emph{Frankfurter Zeitung}|pwk}, Jg. 41, Nr. 258,
                        16. 9. 1896, Erstes Morgenblatt, S. 1) wurde als
                  Einmischung empfunden. In weiterer Folge führte das zu einem Pistolenduell
                  zwischen Goldmann\pwindex{Goldmann, Paul 31.01.1865 – 25.09.1935@\textsc{Goldmann, Paul} (31.01.1865 – 25.09.1935), \emph{Schriftsteller/Schriftstellerin, Journalist/Journalistin}|pwk} und dem antisemitischen
                  Chefredakteur Lucien Millevoye\pwindex{Millevoye, Lucien 1850-08-01 – 1918-03-25@\textsc{Millevoye, Lucien} (1850-08-01 – 1918-03-25), \emph{Politiker/Politikerin, Journalist/Journalistin}|pwk}, das am
                     21. 11. 1896 stattfand (siehe Arthur Schnitzler an Paul Goldmann, 21. 11. 1896). Die publizierten Invektiven gegen Goldmann\pwindex{Goldmann, Paul 31.01.1865 – 25.09.1935@\textsc{Goldmann, Paul} (31.01.1865 – 25.09.1935), \emph{Schriftsteller/Schriftstellerin, Journalist/Journalistin}|pwk} aus dem September 1896, auf die er hier Bezug nimmt, konnten bislang nicht
                  belegt werden.}}}\label{K_L02785-1}, weil ich \strikeout{\textcolor{gray}{von}} für die Unſchuld des \textsc{Dreyfus\pwindex{Dreyfus, Alfred 1859-10-09 – 1935-07-12@\textsc{Dreyfus, Alfred} (1859-10-09 – 1935-07-12), \emph{Militär/Militärin}|pw}} eingetreten bin, von der ich, nach den neueſten Enthüllungen, feſter als je
               überzeugt bin. Zudem geht in meiner Familie Alles drunter und drüber. Kurzum ich weiß
               nicht recht, wo mir {\pb}der Kopf ſteht.\pend
           
\pstart
           Dies um mich zu entſchuldigen, daß ich \strikeout{D} beifolgenden
               Brief von \textsc{Thorel\pwindex{Thorel, Jean 1859-09-11 – 1916-08-20@\textsc{Thorel, Jean} (1859-09-11 – 1916-08-20), \emph{Übersetzer/Übersetzerin, Dramatiker/Dramatikerin}|pw}} ſolange liegen ließ. Heb’ ihn Dir gut auf, denn, wie Du aus ſeinem Inhalt
               erſiehſt, vertritt er die Stelle eines Contracts. Ich habe ihn unter irgend einem
               Vorwand von 6 auf 500 heruntergeſchraubt und habe mir ausdrücklich ausbedungen, daß
               dieſe Zahlung nur als Vorſchuß auf etwaige {\pb}\strikeout{\textcolor{gray}{×}}{ }\textsc{Tantièmen} oder Honorare zu betrachten iſt. Ich fürchte
               allerdings, daß letztere Clauſel platoniſch bleiben dürfte. Nun kannſt Du das Geld
               dieſer Tage an mich ſchicken, wenn Du willſt (aber nicht wieder \label{K_L02785-2v}\edtext{in Goldſtücken in einem recommandirten
                  Brief}{\lemma{\textnormal{\emph{in … Brief}}}\Cendnote{\textnormal{Siehe Paul Goldmann an Arthur Schnitzler, 21. 12. [1895]. }}}\label{K_L02785-2}). Ich werde
               bei dieſem Geſchäft leider nichts verdienen können, aber Du brauchſt hoffentlich bald
               wieder ein \label{K_L02785-3v}\edtext{Opernglas}{\lemma{\textnormal{\emph{Opernglas}}}\Cendnote{\textnormal{Siehe Paul Goldmann an Arthur Schnitzler, 11. 1. [1896]. }}}\label{K_L02785-3}.\pend
           
\pstart
           Bei \textsc{Forain\pwindex{Forain, Jean-Louis 1852-10-23 – 1931-07-11@\textsc{Forain, Jean-Louis} (1852-10-23 – 1931-07-11), \emph{Maler/Malerin, Grafiker/Grafikerin, Karikaturist/Karikaturistin}|pw}} war ich auch, aber er iſt noch auf {\pb}dem
               Lande.\pend
           
\pstart
           Was gibts \strikeout{es} Neues bei Dir? Leben und Dichten?
                  \label{K_L02785-4v}\edtext{Was hörſt Du von Berlin\oindex{Berlin@\textbf{Berlin}, \emph{P.PPLC}|pw} und wann gehſt Du hin?}{\lemma{\textnormal{\emph{Was … hin?}}}\Cendnote{\textnormal{Goldmann\pwindex{Goldmann, Paul 31.01.1865 – 25.09.1935@\textsc{Goldmann, Paul} (31.01.1865 – 25.09.1935), \emph{Schriftsteller/Schriftstellerin, Journalist/Journalistin}|pwk} bezieht sich auf die bevorstehende
                  Uraufführung des Dreiakters \emph{Freiwild}\pwindex{Freiwild. Schauspiel in 3 Akten@\emph{Freiwild. Schauspiel in 3 Akten}|pwk} am 3. 11. 1896 am Deutschen Theater\oindex{Deutsches Theater Berlin@\textbf{Deutsches Theater Berlin}, \emph{Theater (K.THE)}|pwk} in Berlin\oindex{Berlin@\textbf{Berlin}, \emph{P.PPLC}|pwk}. Siehe dazu vor allem \emph{Der Briefwechsel Arthur Schnitzler – Otto Brahm}.
                     Vollständige Ausgabe. Herausgegeben, eingeleitet und erläutert von Oskar
                     Seidlin. Tübingen: \emph{Niemeyer}{ }1975, S. 14–28. { }Schnitzler war dafür zwischen 26. 10. 1896 und 9. 11. 1896 in Berlin\oindex{Berlin@\textbf{Berlin}, \emph{P.PPLC}|pwk}.}}}\label{K_L02785-4}{ }\label{K_L02785-5v}\edtext{\textsc{Ebermann\pwindex{Athenerin. Drama in drei Aufzuegen@\emph{Die Athenerin. Drama in drei Aufzügen}|pwv}\pwindex{Ebermann, Leo 16.07.1863 – 09.10.1914@\textsc{Ebermann, Leo} (16.07.1863 – 09.10.1914), \emph{Schriftsteller/Schriftstellerin, Journalist/Journalistin, Rechtswissenschaftler/Rechtswissenschaftlerin}|pw}}}{\lemma{\textnormal{\emph{Ebermann}}}\Cendnote{\textnormal{Schnitzler war nicht nur bei Proben von Leo Ebermanns\pwindex{Ebermann, Leo 16.07.1863 – 09.10.1914@\textsc{Ebermann, Leo} (16.07.1863 – 09.10.1914), \emph{Schriftsteller/Schriftstellerin, Journalist/Journalistin, Rechtswissenschaftler/Rechtswissenschaftlerin}|pwk} Stück \emph{Die Athenerin}\pwindex{Athenerin. Drama in drei Aufzuegen@\emph{Die Athenerin. Drama in drei Aufzügen}|pwk} anwesend gewesen, sondern hatte am 19. 9. 1896 auch die
                  Uraufführung im Burgtheater\oindex{Burgtheater@\textbf{Burgtheater}, \emph{S.THTR}|pwk} besucht. Siehe zum
                  Erfolg des Stücks\pwindex{Athenerin. Drama in drei Aufzuegen@\emph{Die Athenerin. Drama in drei Aufzügen}|pwkv} etwa auch
                     Arthur Schnitzler an Richard Beer-Hofmann, 21. 9. 1896 und A. S.: \emph{Tagebuch}, 22. 9. 1896.}}}\label{K_L02785-5} ſcheint ja wohl einen
               großen Erfolg gehabt zu haben?\pend
           
\pstart
           Lies \label{K_L02785-6v}\edtext{\textsc{Karl Hillebrand\pwindex{Hillebrand, Karl 1861-12-26 – 1939-01-10@\textsc{Hillebrand, Karl} (1861-12-26 – 1939-01-10), \emph{Astronom/Astronomin}|pw}}: Frankreich und die Franzoſen\pwindex{Frankreich und die Franzosen in der zweiten Haelfte des XIX. Jahrhunderts: Eindruecke und Erfahrungen@\emph{Frankreich und die Franzosen in der zweiten Hälfte des XIX. Jahrhunderts: Eindrücke und Erfahrungen}|pw}}{\lemma{\textnormal{\emph{Karl … Franzoſen}}}\Cendnote{\textnormal{Lektüre durch Schnitzler nicht bekannt}}}\label{K_L02785-6}. Der einzige Deutſche, der
                  Frankreich\oindex{Frankreich@\textbf{Frankreich}, \emph{A.PCLI}|pw} kennt, – und eine Perſönlichkeit\pwindex{Hillebrand, Karl 1861-12-26 – 1939-01-10@\textsc{Hillebrand, Karl} (1861-12-26 – 1939-01-10), \emph{Astronom/Astronomin}|pwv}. Ich leſe \textsc{Schillers\pwindex{Schiller, Friedrich von 10.11.1759 – 09.05.1805@\textsc{Schiller, Friedrich von} (10.11.1759 – 09.05.1805), \emph{Schriftsteller/Schriftstellerin, Historiker/Historikerin, Philosoph/Philosophin}|pw}} und \textsc{Goethes\pwindex{Goethe, Johann Wolfgang von 1749-08-28 – 1832-03-22@\textsc{Goethe, Johann Wolfgang von} (1749-08-28 – 1832-03-22), \emph{Schriftsteller/Schriftstellerin}|pw}}{ }Briefwechſel\pwindex{Briefwechsel zwischen Schiller und Goethe@\emph{Briefwechsel zwischen Schiller und Goethe}|pwv}. Bisher iſt er
               mir unſympathiſch, und \strikeout{\textcolor{gray}{ba}} beſonders der \textsc{Schiller\pwindex{Schiller, Friedrich von 10.11.1759 – 09.05.1805@\textsc{Schiller, Friedrich von} (10.11.1759 – 09.05.1805), \emph{Schriftsteller/Schriftstellerin, Historiker/Historikerin, Philosoph/Philosophin}|pw}} langweilt mich mit ſeinem verfluchten Theoretiſiren.\pend
           \pstart \label{T_L02785-1v}\edtext{Grüß’ Dich Gott, liebſter Freund!
               Schreib’ bald! Dein \spacefill\mbox{P. G.}}{\lemma{\textnormal{\emph{Grüß’ … G.}}}\Cendnote{\textnormal{seitlich am linken Rand}}}\label{T_L02785-1}\pend{}\selectlanguage{ngerman}\vspace{1em}{\vspace{1\baselineskip}}
\pstart
           \raggedleft{}{\pb}{[}hs. :{]} \begin{otherlanguage}{french}\label{K_L02785-7v}\edtext{Chez}{\lemma{\textnormal{\emph{Chez}}}\Cendnote{\textnormal{französisch: bei}}}\label{K_L02785-7}{ }Francis Vielé-Griffin\pwindex{Viele-Griffin, Francis 1864-05-26 – 1937-12-11@\textsc{Vielé-Griffin, Francis} (1864-05-26 – 1937-12-11), \emph{Schriftsteller/Schriftstellerin}|pw}\end{otherlanguage}\pend
           
\pstart
           \raggedleft{}\begin{otherlanguage}{french} au Château de
                        Nazelles\oindex{Château de Nazelles@\textbf{Château de Nazelles}, \emph{Hotel (K.HTL)}|pw}\end{otherlanguage}\pend
           
\pstart
           \raggedleft{}\begin{otherlanguage}{french}(Indre-et-Loire\oindex{Indre-et-Loire@\textbf{Indre-et-Loire}, \emph{A.ADM2}|pw})\end{otherlanguage}\pend
           
\pstart{}\begin{otherlanguage}{french}\label{K_L02785-8v}\edtext{Cher
                        Ami}{\lemma{\textnormal{\emph{Cher
                        Ami}}}\Cendnote{\textnormal{Lieber
                     Freund}}}\label{K_L02785-8},\end{otherlanguage}\pend\vspace{0.5em}
\pstart
           \label{K_L02785-9v}\edtext{\begin{otherlanguage}{french}La chose est donc convenue, aux conditions que vous dites:
                  cinq cent francs que vous me verserez aux premiers jours d’octobre. Et moi, je vais me mettre tout de suite à l’œuvre\pwindex{Amourette. Piece en trois actes. Adaptee de Arthur Schnitzler@\emph{Amourette. Pièce en trois actes. Adaptée de Arthur Schnitzler}|pwv}, afin d’arriver en temps utile
                  pour profiter des chances de cette saison.\end{otherlanguage}}{\lemma{\textnormal{\emph{La … saison.}}}\Cendnote{\textnormal{französisch: Die Sache ist also
                  ausgemacht, zu den von Ihnen genannten Bedingungen: fünfhundert Francs, die Sie
                  mir in den ersten Oktobertagen auszahlen werden.
                  Und ich werde mich sofort an die Arbeit\pwindex{Amourette. Piece en trois actes. Adaptee de Arthur Schnitzler@\emph{Amourette. Pièce en trois actes. Adaptée de Arthur Schnitzler}|pwkv} machen, damit die Gelegenheiten genützt werden können, die die
                  Saison bietet.}}}\label{K_L02785-9}\pend
           
\pstart
           \begin{otherlanguage}{french}\label{K_L02785-10v}\edtext{Pour achever
                  de préciser le côté affaire, et pour que vous pouissiez envoyer un engagement
                  signé de moi à M. Schnitzler, si vous le désirez, – je rappelle ici qu’il est bien
                  entendu que cette somme de cinq cents francs n’est qu’une avance sur les droits de
                  toute nature que pourra produire la traduction\pwindex{Amourette. Piece en trois actes. Adaptee de Arthur Schnitzler@\emph{Amourette. Pièce en trois actes. Adaptée de Arthur Schnitzler}|pwv} de \uline{Liebelei\pwindex{Liebelei. Schauspiel in drei Akten@\emph{Liebelei. Schauspiel in drei Akten}|pw}}, droits de représentation, ou de publication en revue ou en librairie; – Et
                  pour les droits, il va de soi qu’ils seront partagés par moitiés égales entre M.
                  Schnitzler et moi}{\lemma{\textnormal{\emph{Pour … moi}}}\Cendnote{\textnormal{Um die
                        geschäftliche Seite zu präzisieren, und damit Sie, wenn Sie dies wünschen,
                        Herrn Schnitzler eine von mir
                        unterschriebene Verpflichtungserklärung schicken können, – halte ich sie
                        hier fest, um zu verdeutlichen, dass diese Summe von fünfhundert Francs nur
                        ein Vorschuss auf die Rechte jeglicher Art ist, die die Übersetzung\pwindex{Amourette. Piece en trois actes. Adaptee de Arthur Schnitzler@\emph{Amourette. Pièce en trois actes. Adaptée de Arthur Schnitzler}|pwv} von \uline{Liebelei\pwindex{Liebelei. Schauspiel in drei Akten@\emph{Liebelei. Schauspiel in drei Akten}|pw}} mit sich bringt, wie Aufführungsrechte oder Veröffentlichungen in
                        Zeitschriften oder Buchhandlungen; – und die Rechte werden
                        selbstverständlich zu gleichen Teilen zwischen Herrn Schnitzler und mir geteilt.}}}\label{K_L02785-10}
                  –\end{otherlanguage}\pend
           
\pstart
           \label{K_L02785-11v}\edtext{\begin{otherlanguage}{french}Je rentrerai à Paris\oindex{Paris@\textbf{Paris}, \emph{P.PPLC}|pw},
                  vers la fin de {\pb}septembre. Mon adreſse est à Nazelles\oindex{Château de Nazelles@\textbf{Château de Nazelles}, \emph{Hotel (K.HTL)}|pwv} jusqu’au 14;{\\}et à partir du 15 elle sera (et moi aussi) chez madame Paul Bert\pwindex{Bert, Paul 1833-10-19 – 1886-11-11@\textsc{Bert, Paul} (1833-10-19 – 1886-11-11), \emph{Politiker/Politikerin, Physiologe/Physiologin}|pw}\pwindex{Clayton, Josephine 1846-10-03 – 1916-06@\textsc{Clayton, Josephine} (1846-10-03 – 1916-06)|pwv} à \uline{Auxerre}\oindex{Auxerre@\textbf{Auxerre}, \emph{P.PPLA2}|pw} (\uline{Yonne}\oindex{Yonne@\textbf{Yonne}, \emph{A.ADM2}|pw})\end{otherlanguage}}{\lemma{\textnormal{\emph{Je … (Yonne)}}}\Cendnote{\textnormal{französisch: Ich werde gegen Ende September nach Paris\oindex{Paris@\textbf{Paris}, \emph{P.PPLC}|pwk}
                  zurückkehren. Meine Adresse ist in Nazelles\oindex{Château de Nazelles@\textbf{Château de Nazelles}, \emph{Hotel (K.HTL)}|pwkv} bis zum 14.; und
                  ab dem 15. ist sie (und ich auch) bei Madame Paul Bert\pwindex{Bert, Paul 1833-10-19 – 1886-11-11@\textsc{Bert, Paul} (1833-10-19 – 1886-11-11), \emph{Politiker/Politikerin, Physiologe/Physiologin}|pwk}\pwindex{Clayton, Josephine 1846-10-03 – 1916-06@\textsc{Clayton, Josephine} (1846-10-03 – 1916-06)|pwkv} in \uline{Auxerre}\oindex{Auxerre@\textbf{Auxerre}, \emph{P.PPLA2}|pwk} (\uline{Yonne}\oindex{Yonne@\textbf{Yonne}, \emph{A.ADM2}|pwk}). }}}\label{K_L02785-11}\pend
           
\pstart
           \label{K_L02785-12v}\edtext{\begin{otherlanguage}{french}Votre bien dévoué\end{otherlanguage}}{\lemma{\textnormal{\emph{Votre bien dévoué}}}\Cendnote{\textnormal{französisch: Ihr sehr ergebener}}}\label{K_L02785-12}{ }{\\[\baselineskip]}\spacefill\mbox{Jean Thorel\pwindex{Thorel, Jean 1859-09-11 – 1916-08-20@\textsc{Thorel, Jean} (1859-09-11 – 1916-08-20), \emph{Übersetzer/Übersetzerin, Dramatiker/Dramatikerin}|pw}}\pend
           \leftskip=0em{}\selectlanguage{ngerman}\endnumbering\briefempfaengerindex{Schnitzler, Arthur@\textsc{Schnitzler, Arthur}!zzzGoldmann, Paul@\emph{von Paul Goldmann}!1896-09-221@{22. 9. {[}1896{]}}|)be}\mylabel{L02785h}  \normalsize

\doendnotes{C}
\bigskip
\vfill

\clearpage

\footnotesize

\lohead{\textsc{register}}

% Definiere theindex-Environment komplett neu ohne reledmac
\makeatletter
\renewenvironment{theindex}{%
  \section*{\indexname}%
  \setlength{\parindent}{0pt}%
  \setlength{\parskip}{0pt plus 0.3pt}%
  \let\item\@idxitem
}{%
  \clearpage
}
\makeatother

\IfFileExists{\jobname-pw.ind}{\input{\jobname-pw.ind}}{}

\end{document}

      