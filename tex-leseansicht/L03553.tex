%% latex-korrekturansicht-vorspann.tex
%% Vorspann für die Korrekturansicht.
%% Lädt die gemeinsame Datei latex-vorspann.tex mit gesetztem Schalter.

\newif\ifkorrekturansicht
\korrekturansichttrue

\input{../tex-inputs/latex-vorspann}


\section[ Felix Salten an Arthur Schnitzler, 16. 8. 1911]{L03553 Felix Salten an Arthur Schnitzler, 16. 8. 1911}
\nopagebreak\mylabel{L03553v}
\rehead{ }\normalsize\beginnumbering\briefempfaengerindex{Schnitzler, Arthur@\textsc{Schnitzler, Arthur}!zzzSalten, Felix@\emph{von Felix Salten}!1911-08-161@{16. 8. 1911}|(be}
\toendnotes[C]{\smallbreak\pagebreak[2]}\Standort{CUL, Schnitzler, B 89, B 2.}
\physDesc{Briefkarte, 2 Karten, 4045 Zeichen (die zweite Karte markiert: »II« )
\newline{}Handschrift: schwarze Tinte, lateinische Kurrent
\newline{}Ordnung: mit Bleistift von unbekannter Hand nummeriert: »268« }\toendnotes[C]{\smallbreak}
\pstart
           \raggedleft{}{\pb}Unterach a. Attersee, Berghof\oindex{Berghof@\textbf{Berghof}, \emph{Wohngebäude (K.WHS)}|pw}\pend
           
\pstart
           \raggedleft{}16. VIII. 11\pend
           
\pstart
           \textcolor{gray}{\textbf{\textsc{Felix Salten}}}\pend
           
\pstart{}Lieber,\pend\vspace{0.5em}
\pstart
           ich danke Ihnen herzlich für Ihren ausführlichen \label{K_L03553-1v}\edtext{Brief}{\lemma{\textnormal{\emph{Brief}}}\Cendnote{\textnormal{Der Brief ist nicht
                  erhalten. Schnitzler dürfte darin von seinem Gespräch
                     mit Moriz Benedikt\pwindex{Benedikt, Moriz 27.05.1849 – 18.03.1920@\textsc{Benedikt, Moriz} (27.05.1849 – 18.03.1920), \emph{Journalist/Journalistin, Herausgeber/Herausgeberin}|pwk} berichtet haben, das am 9. 8. 1911
                     am Semmering\oindex{Semmering@\textbf{Semmering}, \emph{A.ADM3}|pwk} stattgefunden hat. Er dürfte auch
                     begründet haben, warum er Salten\pwindex{Salten, Felix 06.09.1869 – 08.10.1945@\textsc{Salten, Felix} (06.09.1869 – 08.10.1945), \emph{Schriftsteller/Schriftstellerin, Journalist/Journalistin, Chefredakteur/Chefredakteurin}|pwk} nicht thematisierte. In seinen \emph{Erinnerungen}\pwindex{Erinnerungen@\emph{Erinnerungen}|pwk} ging Salten\pwindex{Salten, Felix 06.09.1869 – 08.10.1945@\textsc{Salten, Felix} (06.09.1869 – 08.10.1945), \emph{Schriftsteller/Schriftstellerin, Journalist/Journalistin, Chefredakteur/Chefredakteurin}|pwk} zweimal
                     darauf ein, dass ihn Schnitzler an dieser
                     Stelle nicht unterstützt habe und lässt es dadurch zu einem zentralen Moment ihrer
                     Beziehung werden: Schnitzler »lehnte viele Jahre später auch ab, als ich ihn in
                        einer Daseinskrisis bat, so beiläufig zu erkunden, was für eine Meinung der Herausgeber\pwindex{Benedikt, Moriz 27.05.1849 – 18.03.1920@\textsc{Benedikt, Moriz} (27.05.1849 – 18.03.1920), \emph{Journalist/Journalistin, Herausgeber/Herausgeberin}|pwv} der Neuen Freien Presse\orgindex{Neue Freie Presse@Neue Freie Presse|pw} von mir hege, und sagte,
                        das könne er aus Freundschaft für Auernheimer\pwindex{Auernheimer, Raoul 15.04.1876 – 06.01.1948@\textsc{Auernheimer, Raoul} (15.04.1876 – 06.01.1948), \emph{Schriftsteller/Schriftstellerin, Journalist/Journalistin, Kritiker/Kritikerin}|pw} nicht tun. Diese Freundschaft für Auernheimer\pwindex{Auernheimer, Raoul 15.04.1876 – 06.01.1948@\textsc{Auernheimer, Raoul} (15.04.1876 – 06.01.1948), \emph{Schriftsteller/Schriftstellerin, Journalist/Journalistin, Kritiker/Kritikerin}|pw} war ganz neu und ganz einseitig«. \emph{Wienbibliothek im Rathaus}, Nachlass Salten, ZPH 1681/1
                                 1.1.1.9.1, S. [6], vgl. S. [52].}}}\label{K_L03553-1}. Sie erinnern sich ja gewiß, dass Sie selbst mir \label{K_L03553-2v}\edtext{in St.
                  Gilgen\oindex{St. Gilgen@\textbf{St. Gilgen}, \emph{A.ADM3}|pw}}{\lemma{\textnormal{\emph{in St.
                  Gilgen}}}\Cendnote{\textnormal{Schnitzler war zwischen 24. 7. 1911 und 29. 7. 1911 in St. Gilgen\oindex{St. Gilgen@\textbf{St. Gilgen}, \emph{A.ADM3}|pwk}; das Gespräch mit Salten\pwindex{Salten, Felix 06.09.1869 – 08.10.1945@\textsc{Salten, Felix} (06.09.1869 – 08.10.1945), \emph{Schriftsteller/Schriftstellerin, Journalist/Journalistin, Chefredakteur/Chefredakteurin}|pwk} hatte am 27. 7. 1911
                  stattgefunden.}}}\label{K_L03553-2} sagten, Sie kämen jetzt auf dem Semmering\oindex{Semmering@\textbf{Semmering}, \emph{A.ADM3}|pw} mit Herrn Benedikt\pwindex{Benedikt, Moriz 27.05.1849 – 18.03.1920@\textsc{Benedikt, Moriz} (27.05.1849 – 18.03.1920), \emph{Journalist/Journalistin, Herausgeber/Herausgeberin}|pw} zusammmen, und ob es mir da recht sei, wenn Sie bei einer sich
               ergebenden Gelegenheit meiner Erwähnung tun würden. Ich wäre ja nicht auf diesen
               Einfall gerathen, denn einmal dachte ist nicht daran, dass Sie jetzt mit Herrn Benedikt\pwindex{Benedikt, Moriz 27.05.1849 – 18.03.1920@\textsc{Benedikt, Moriz} (27.05.1849 – 18.03.1920), \emph{Journalist/Journalistin, Herausgeber/Herausgeberin}|pw} zusammentreffen, dann auch wußte ich
               ja, dass Sie sich durch freundschaftliche Rücksichtnahme auf Herrn D\textsuperscript{r}{ }Auernheimer\pwindex{Auernheimer, Raoul 15.04.1876 – 06.01.1948@\textsc{Auernheimer, Raoul} (15.04.1876 – 06.01.1948), \emph{Schriftsteller/Schriftstellerin, Journalist/Journalistin, Kritiker/Kritikerin}|pw} in dieser Sache behindert fühlen. Eine Erwähnung meiner
               Person und \label{K_L03553-3v}\edtext{meines Austritts aus der »Zeit\orgindex{Zeit@Die Zeit|pw}}{\lemma{\textnormal{\emph{meines … »Zeit}}}\Cendnote{\textnormal{Salten\pwindex{Salten, Felix 06.09.1869 – 08.10.1945@\textsc{Salten, Felix} (06.09.1869 – 08.10.1945), \emph{Schriftsteller/Schriftstellerin, Journalist/Journalistin, Chefredakteur/Chefredakteurin}|pwk}
                  war gekündigt worden, vgl. Arthur Schnitzler an Felix Salten, [14. 4. 1910?]. Danach platzierte 
                  Salten\pwindex{Salten, Felix 06.09.1869 – 08.10.1945@\textsc{Salten, Felix} (06.09.1869 – 08.10.1945), \emph{Schriftsteller/Schriftstellerin, Journalist/Journalistin, Chefredakteur/Chefredakteurin}|pwk} als freier Mitarbeiter Texte bei verschiedenen Zeitungen, auch der \emph{Zeit}\orgindex{Zeit@Die Zeit|pwk},
                  und wurde mit Mai 1912 fester Mitarbeiter beim \emph{Fremden-Blatt}\orgindex{Fremden-Blatt@Fremden-Blatt|pwk}.
                  }}}\label{K_L03553-3}« Herrn
                  Benedikt\pwindex{Benedikt, Moriz 27.05.1849 – 18.03.1920@\textsc{Benedikt, Moriz} (27.05.1849 – 18.03.1920), \emph{Journalist/Journalistin, Herausgeber/Herausgeberin}|pw} gegenüber, hätte für mich wol auch
               nur informativen Erfolg haben sollen. Denn wie Sie wißen, waren wir übereingekommen,
               dass Sie nichts Intervenirendes sagen. Wenn Sie nun den Eindruck erhielten, dass
               selbst ein noch so beiläufiges Erwähnen meines Namens bei Herrn Benedikt\pwindex{Benedikt, Moriz 27.05.1849 – 18.03.1920@\textsc{Benedikt, Moriz} (27.05.1849 – 18.03.1920), \emph{Journalist/Journalistin, Herausgeber/Herausgeberin}|pw} die Vermutung des Absichtlichen und Intervenirenden
               wecken würde, dann war es natürlich sehr gut, derartiges ganz zu vermeiden, und ich
               danke Ihnen vielmals dafür. Was Ihren Rat betrifft, glaube ich nicht, dass ich ihn
               befolgen werde. Erstens weiß ich ja noch selber nicht, ob ich jemals wieder eine fixe
                  {\pb}Stellung annehmen werde.
               Dann aber würde diese Stellung wol für mich nicht acceptabel sein, wenn ich noch so
               offen und geradezu mich darum bewerbe, {\dotstwo} eben \uline{weil} ich mich bewerbe! Zuletzt aber gibt es für mich
               noch einen höheren Grund, mich \strikeout{\textcolor{gray}{×}\-\textcolor{gray}{×}\-\textcolor{gray}{×}\-\textcolor{gray}{×}\-\textcolor{gray}{×}\-\textcolor{gray}{×}\-\textcolor{gray}{×}\-\textcolor{gray}{×}\-\textcolor{gray}{×}\-\textcolor{gray}{×}\-\textcolor{gray}{×}\-\textcolor{gray}{×}\-\textcolor{gray}{×}\-\textcolor{gray}{×}\-\textcolor{gray}{×}\-\textcolor{gray}{×}\-\textcolor{gray}{×}\-\textcolor{gray}{×}\-\textcolor{gray}{×}\-\textcolor{gray}{×}} niemals Herrn Benedikt\pwindex{Benedikt, Moriz 27.05.1849 – 18.03.1920@\textsc{Benedikt, Moriz} (27.05.1849 – 18.03.1920), \emph{Journalist/Journalistin, Herausgeber/Herausgeberin}|pw} oder sonst
               Jemandem anzubieten. Ich habe das in meinen kleinsten und schwersten Anfängen nicht
               getan. Jetzt schreibe ich seit achtzehn Jahren; meine Leistung ist zu offenkundig und
               – wenn das Wort erlaubt ist, – mein Anspruch auf eine Stelle in einem Blatt Österreichs\oindex{Oesterreich@\textbf{Österreich}, \emph{A.PCLI}|pw} zu gerecht, als dass ich selbst auf
               diese Leistung hinweisen oder diesen Anspruch geltend machen möchte.\pend
           
\pstart
           In einem einzigen Betracht bedaure ich es lebhaft, dass Sie nicht dazu gelangen, mit
               Herrn Benedikt\pwindex{Benedikt, Moriz 27.05.1849 – 18.03.1920@\textsc{Benedikt, Moriz} (27.05.1849 – 18.03.1920), \emph{Journalist/Journalistin, Herausgeber/Herausgeberin}|pw} zu sprechen. Und aus diesem
               Grund allein tut es mir leid, dass es nicht möglich ist, eine im Metier so viel
               beredte Angelegenheit, wie mein Austritt aus der »Zeit\orgindex{Zeit@Die Zeit|pw}« es ist, vor Herrn Benedikt\pwindex{Benedikt, Moriz 27.05.1849 – 18.03.1920@\textsc{Benedikt, Moriz} (27.05.1849 – 18.03.1920), \emph{Journalist/Journalistin, Herausgeber/Herausgeberin}|pw} zu
               erwähnen. Es ist mir nämlich dieser Tage zugetragen worden, Herr Benedikt\pwindex{Benedikt, Moriz 27.05.1849 – 18.03.1920@\textsc{Benedikt, Moriz} (27.05.1849 – 18.03.1920), \emph{Journalist/Journalistin, Herausgeber/Herausgeberin}|pw} sei – wahrscheinlich von einer mir schlecht
               gesinnten Seite – zu der Ansicht gebracht, ich lebe in völlig desolaten
               Geldverhältnissen, stecke bis über die Ohren in Schulden, und führe ein prassendes
               Verschwenderleben. Wenn er nun aufgeklärt hätte werden können, dass ich wol Schulden
               hatte (Familie \textcolor{gray}{usw}.) jetzt aber keine mehr habe, dass ich wol
               anständig, aber nicht verschwenderisch lebe, hoch versichert bin, und auch sonst
               keine materiellen Krisen habe, wäre mir das schon in einem ganz allgemeinen und
               prinzipiellen Sinn \uline{sehr} erwünscht gewesen, und es
               wäre nur eine einfache Richtigstellung, welche keine anderen, konkurrirenden
               Interessen verletzt. Nun wird es doch wol am besten sein, wenn ich in dieser ganzen
               Sache ruhig zuwarte. Ich weiß ja heute selbst {\pb}noch nicht, wofür ich mich
               entscheiden werde, und es liegen noch mehrere Monate vor mir, in denen ich alle
               Umstände prüfen, verschiedene größere Arbeiten fördern und alles zusa{\geminationm}en überlegen muß. Es kann ja auch sein, dass Herr Benedikt\pwindex{Benedikt, Moriz 27.05.1849 – 18.03.1920@\textsc{Benedikt, Moriz} (27.05.1849 – 18.03.1920), \emph{Journalist/Journalistin, Herausgeber/Herausgeberin}|pw} und ich nicht zusammenko{\geminationm}en, weil er auf eine Deklaration von mir und ich auf
               eine von ihm warte. Es kann ja auch (so leicht) sein, dass wir, \uline{wenn} wir schon einmal zusammenkommen, nicht mit einander einig werden.
               Und es kann auch sein, dass er mich überhaupt nicht mag und eine Verbindung mit mir
               garnicht in Erwägung zieht. Auch damit rechne ich.\pend
           
\pstart
           Bei uns\pwindex{Salten, Ottilie 07.03.1868 – 22.06.1942@\textsc{Salten, Ottilie} (07.03.1868 – 22.06.1942), \emph{Schauspieler/Schauspielerin}|pwv} geht alles ziemlich
               wol. Arbeit, Gäste, Geburtstage, Ausflüge. Das wechselt so ab und ist bisher vom
               schönsten Wetter besonnt. Ich habe eine Kur begonnen und bin seither die Schmerzen
               los; habe die »Zeit\orgindex{Zeit@Die Zeit|pw}« ersucht, mich noch hier\oindex{Berghof@\textbf{Berghof}, \emph{Wohngebäude (K.WHS)}|pwv} zu laßen, damit ich diese
               Kur beendigen kann, und ihr dafür angeboten, von hier aus zu schreiben. Kann sein,
               dass sie mich trotzdem zwingt\textcolor{gray}{,} nach Wien\oindex{Wien@\textbf{Wien}, \emph{A.ADM2}|pw} zu gehen. Fischer\pwindex{Fischer, Samuel 24.12.1859 – 15.10.1934@\textsc{Fischer, Samuel} (24.12.1859 – 15.10.1934), \emph{Verleger/Verlegerin}|pw} ist schon in
                  Gastein\oindex{Bad Gastein@\textbf{Bad Gastein}, \emph{P.PPLA3}|pw}. Wir grüßen Sie alle in
               Herzlichkeit.\pend
           \pstart Ihr \spacefill\mbox{Salten}\pend{}\selectlanguage{ngerman}\endnumbering\briefempfaengerindex{Schnitzler, Arthur@\textsc{Schnitzler, Arthur}!zzzSalten, Felix@\emph{von Felix Salten}!1911-08-161@{16. 8. 1911}|)be}\mylabel{L03553h}  \normalsize

\doendnotes{C}
\bigskip
\vfill

\clearpage

\footnotesize

\lohead{\textsc{register}}

% Definiere theindex-Environment komplett neu ohne reledmac
\makeatletter
\renewenvironment{theindex}{%
  \section*{\indexname}%
  \setlength{\parindent}{0pt}%
  \setlength{\parskip}{0pt plus 0.3pt}%
  \let\item\@idxitem
}{%
  \clearpage
}
\makeatother

\IfFileExists{\jobname-pw.ind}{\input{\jobname-pw.ind}}{}

\end{document}

      