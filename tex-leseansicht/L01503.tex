%% latex-korrekturansicht-vorspann.tex
%% Vorspann für die Korrekturansicht.
%% Lädt die gemeinsame Datei latex-vorspann.tex mit gesetztem Schalter.

\newif\ifkorrekturansicht
\korrekturansichttrue

\input{../tex-inputs/latex-vorspann}


\section[Hermann Bahr: Widmungsexemplar Sanna für Arthur Schnitzler, {[}1.?{]} 3. 1905]{L01503 Hermann Bahr: Widmungsexemplar Sanna für Arthur Schnitzler,
               {[}1.?{]} 3. 1905}
\nopagebreak\mylabel{L01503v}
\rehead{ }\normalsize\beginnumbering\briefempfaengerindex{Schnitzler, Arthur@\textsc{Schnitzler, Arthur}!zzzBahr, Hermann@\emph{von Hermann Bahr}!1905-03-011@{{[}1.?{]} 3. 1905}|(be}
\toendnotes[C]{\smallbreak\pagebreak[2]}\Standort{DLA, G:Schnitzler, Arthur (Sammlung Heinrich Schnitzler).}
\physDesc{Widmung am Vorsatzblatt, 39 Zeichen
\newline{}Handschrift: schwarze Tinte, deutsche Kurrent
\newline{}Ordnung: bei der Enteignung des Exemplars 1938 von
                                 unbekannter Hand mit Bleistift ergänzte Informationen:
                                    »Dubl. zu 439.421-B« }
\buchAbdrucke{\weitereDrucke{Hermann Bahr, Arthur Schnitzler: \emph{Briefwechsel, Aufzeichnungen, Dokumente (1891–1931)}. Göttingen: \emph{Wallstein} 2018, S. 344.} }\toendnotes[C]{\smallbreak}
\pstart
           \noindent{}{\pb}Herzlichſt\pend
           
\pstart
           herzlichſt{\\[\baselineskip]}\spacefill\mbox{Hermann}\pend
           \leftskip=0em{}
\pstart
           \noindent{}\label{K_L01503-1v}\edtext{März 1905}{\lemma{\textnormal{\emph{März 1905}}}\Cendnote{\textnormal{am 28. 2. 1905 vom \emph{Börsenblatt für den deutschen Buchhandel}\pwindex{Boersenblatt fuer den Deutschen Buchhandel@\emph{Börsenblatt für den Deutschen Buchhandel}|pwk}
                        als Neuerscheinung gemeldet}}}\label{K_L01503-1}\pend
           \selectlanguage{ngerman}\vspace{1em}{\vspace{1\baselineskip}}
\pstart
           \centering{}{\pb}\textcolor{gray}{\textbf{\textbf{Sanna}\pwindex{Sanna. Schauspiel in fuenf Aufzuegen@\emph{Sanna. Schauspiel in fünf Aufzügen}|pw}}}\pend
           
\pstart
           \centering{}\textcolor{gray}{\textbf{\so{Schauſpiel in fünf Aufzügen}}}\pend
           
\pstart
           \centering{}\textcolor{gray}{\textbf{von}}\pend
           
\pstart
           \centering{}\textcolor{gray}{\textbf{Hermann Bahr}}\pend
           {\vspace{1\baselineskip}}
\pstart
           \raggedleft{}\textcolor{gray}{\textbf{»\label{K_L01503-2v}\edtext{Endlich gewinnt
                  doch nur unſer}{\lemma{\textnormal{\emph{Endlich … unſer}}}\Cendnote{\textnormal{in einem Brief an Mathilde Wesendonck\pwindex{Wesendonck, Mathilde 23.12.1828 – 31.08.1902@\textsc{Wesendonck, Mathilde} (23.12.1828 – 31.08.1902), \emph{Schriftsteller/Schriftstellerin}|pwk},
                        15. 4. 1859}}}\label{K_L01503-2}}}{\\}\textcolor{gray}{\textbf{Herz, wer am meiſten leidet, und}}{\\}\textcolor{gray}{\textbf{eine Stimme ſagt uns auch, daß}}{\\}\textcolor{gray}{\textbf{er am tiefſten blickt: eben weil er}}{\\}\textcolor{gray}{\textbf{in jedem Falle alle Fälle ſieht, dünkt}}{\\}\textcolor{gray}{\textbf{ihm der kleinſte so ungeheuer.«}}\pend
           
\pstart
           \raggedleft{}\textcolor{gray}{\textbf{Richard Wagner\pwindex{Wagner, Richard 22.05.1813 – 13.02.1883@\textsc{Wagner, Richard} (22.05.1813 – 13.02.1883), \emph{Komponist/Komponistin}|pw}}}\pend
           {\vspace{1\baselineskip}}
\pstart
           \centering{}\textcolor{gray}{\textbf{Berlin\oindex{Berlin@\textbf{Berlin}, \emph{P.PPLC}|pw}{ }1905}}\pend
           
\pstart
           \centering{}\textcolor{gray}{\textbf{\so{S. Fiſcher, Verlag}\orgindex{S. Fischer Verlag@S. Fischer Verlag|pw}}}\pend
           \selectlanguage{ngerman}\endnumbering\briefempfaengerindex{Schnitzler, Arthur@\textsc{Schnitzler, Arthur}!zzzBahr, Hermann@\emph{von Hermann Bahr}!1905-03-011@{{[}1.?{]} 3. 1905}|)be}\mylabel{L01503h}  \normalsize

\doendnotes{C}
\bigskip
\vfill

\clearpage

\footnotesize

\lohead{\textsc{register}}

% Definiere theindex-Environment komplett neu ohne reledmac
\makeatletter
\renewenvironment{theindex}{%
  \section*{\indexname}%
  \setlength{\parindent}{0pt}%
  \setlength{\parskip}{0pt plus 0.3pt}%
  \let\item\@idxitem
}{%
  \clearpage
}
\makeatother

\IfFileExists{\jobname-pw.ind}{\input{\jobname-pw.ind}}{}

\end{document}

      