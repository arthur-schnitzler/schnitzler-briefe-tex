%% latex-korrekturansicht-vorspann.tex
%% Vorspann für die Korrekturansicht.
%% Lädt die gemeinsame Datei latex-vorspann.tex mit gesetztem Schalter.

\newif\ifkorrekturansicht
\korrekturansichttrue

\input{../tex-inputs/latex-vorspann}


\section[Arthur Schnitzler und Felix Salten an Richard Beer-Hofmann, 27. 8. 1893]{L00260 Arthur Schnitzler und Felix Salten an Richard Beer-Hofmann,
               27. 8. 1893}
\nopagebreak\mylabel{L00260v}
\rehead{ }\normalsize\beginnumbering\briefempfaengerindex{Beer-Hofmann, Richard@\textsc{Beer-Hofmann, Richard}!zzzSalten, Felix@\emph{von Felix Salten}!1893-08-271@{27. 8. 1893}|(be}\briefempfaengerindex{Beer-Hofmann, Richard@\textsc{Beer-Hofmann, Richard}!zzzSchnitzler, Arthur@\emph{von Arthur Schnitzler}!1893-08-271@{27. 8. 1893}|(be}
\toendnotes[C]{\smallbreak\pagebreak[2]}\Standort{YCGL, MSS 31.}
\physDesc{Kartenbrief, 604 Zeichen
\newline{}Handschrift Arthur Schnitzler: Bleistift, deutsche Kurrent
\newline{}Handschrift Felix Salten: Bleistift, lateinische Kurrent
\newline{}Versand: 1) Stempel: »\nobreak{}\oindex{Poertschach am Woerthersee@\textbf{Pörtschach am Wörthersee}, \emph{P.PPL}|pwk}{[}Pört{]}schach am See, 27 8 93\nobreak{}«.   2) Stempel: »\nobreak{}\oindex{Znaim@\textbf{Znaim}, \emph{P.PPLA2}|pwk}Znaim
                                       Znoj\textcolor{gray}{m}o, 28 8 93, 1\textcolor{gray}{2}–4 N\nobreak{}«. }
\buchAbdrucke{\weitereDrucke{Arthur Schnitzler, Richard Beer-Hofmann: \emph{Briefwechsel 1891–1931}. Wien, Zürich: \emph{Europaverlag} 1992, S. 51.} }\toendnotes[C]{\smallbreak}\pstart{}{\pb}Herrn \textsc{Dr. Richard
                     Beer-Hofmann}\pend{}\pstart{}k. k. Lieutenant im Infanterie-Regimente Nr. 99\pend{}\pstart{}\textsc{Znaim\oindex{Znaim@\textbf{Znaim}, \emph{P.PPLA2}|pw}}\pend{}\pstart{}Mähren\oindex{Maehren@\textbf{Mähren}, \emph{L.RGN}|pw} (?)\pend{}{\bigskip}\vspace{1em}
\pstart
           \noindent{}{\pb}Lieber Richard, aus \textsc{Pieve di Cadore}\oindex{Pieve di Cadore@\textbf{Pieve di Cadore}, \emph{A.ADM3}|pw}{ }ſchrieben wir dem Verfaſſer\pwindex{Hofmannsthal, Hugo von 1874-02-01 – 1929-07-15@\textsc{Hofmannsthal, Hugo von} (1874-02-01 – 1929-07-15), \emph{Schriftsteller/Schriftstellerin}|pwv} von Tizians
                  Tod\pwindex{Tod des Tizian. Ein Bruchstueck@\emph{Der Tod des Tizian. Ein Bruchstück}|pw}; – aus \textsc{Pörtschach}\oindex{Poertschach am Woerthersee@\textbf{Pörtschach am Wörthersee}, \emph{P.PPL}|pw} dem Verfaſſer des Kindes\pwindex{Kind@\emph{Das Kind}|pw} – denn ebenſowahr
               es iſt dß \textsc{Tizian}\pwindex{Tizian zwischen 1488 und 1490 – 27.08.1576@\textsc{Tizian} (zwischen 1488 und 1490 – 27.08.1576), \emph{Maler/Malerin}|pw} in \textsc{Pieve di Cadore}\oindex{Pieve di Cadore@\textbf{Pieve di Cadore}, \emph{A.ADM3}|pw} geboren word\textcolor{gray}{en}, ebenſo wahr iſt es, dß hier ſchon manches
               Kind geboren ward.\pend
           
\pstart
           – Wir haben eine ſchöne Tour gemacht; näheres mündlich. Ihnen gehts hoffentlich gut,
               und wir grüßen Sie herzlich!\pend
           \pstart \spacefill\mbox{Arthur}\pend{}\selectlanguage{ngerman}\vspace{1em}
\pstart
           \noindent{}{[}hs. :{]} Ich habe Sie hier ohne Backenbart gesehen, sorgen
               dafür, dass er rasch wieder wächst. Frl. Anna
                  Hiller\pwindex{Kupelwieser, Anna 20.09.1874 – 08.04.1937@\textsc{Kupelwieser, Anna} (20.09.1874 – 08.04.1937)|pw}, die mir das Bild zeigte grüßt Sie. Ich auch\pend
           
\pstart
           Ihr{\\[\baselineskip]}\spacefill\mbox{Salten}\pend
           \leftskip=0em{}\selectlanguage{ngerman}\endnumbering\briefempfaengerindex{Beer-Hofmann, Richard@\textsc{Beer-Hofmann, Richard}!zzzSalten, Felix@\emph{von Felix Salten}!1893-08-271@{27. 8. 1893}|)be}\briefempfaengerindex{Beer-Hofmann, Richard@\textsc{Beer-Hofmann, Richard}!zzzSchnitzler, Arthur@\emph{von Arthur Schnitzler}!1893-08-271@{27. 8. 1893}|)be}\mylabel{L00260h}  \normalsize

\doendnotes{C}
\bigskip
\vfill

\clearpage

\footnotesize

\lohead{\textsc{register}}

% Definiere theindex-Environment komplett neu ohne reledmac
\makeatletter
\renewenvironment{theindex}{%
  \section*{\indexname}%
  \setlength{\parindent}{0pt}%
  \setlength{\parskip}{0pt plus 0.3pt}%
  \let\item\@idxitem
}{%
  \clearpage
}
\makeatother

\IfFileExists{\jobname-pw.ind}{\input{\jobname-pw.ind}}{}

\end{document}

      