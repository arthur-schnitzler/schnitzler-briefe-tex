%% latex-leseansicht-vorspann.tex
%% Vorspann für die Leseansicht.
%% Lädt die gemeinsame Datei latex-vorspann.tex mit nicht gesetztem Schalter.

\newif\ifkorrekturansicht
\korrekturansichtfalse

\input{../tex-inputs/latex-vorspann}


         
         \renewcommand{\erwaehntePersonen}{Personen: Richard Beer-Hofmann, Hugo von Hofmannsthal, Anna Kupelwieser,  Tizian}
         \renewcommand{\erwaehnteOrte}{Orte: Mähren, Pieve di Cadore, Pörtschach, Znaim}
         \renewcommand{\erwaehnteWerke}{Werke: Das Kind, Der Tod des Tizian}
               \section[Arthur Schnitzler und Felix Salten an Richard Beer-Hofmann, 27. 8. 1893]{ Arthur Schnitzler und Felix Salten an Richard Beer-Hofmann,
                    27. 8. 1893}\nopagebreak\mylabel{v}\rehead{ }\begin{ledgroupsized}[t]{13cm}\normalsize\beginnumbering \toendnotes[C]{\smallbreak\pagebreak[2]} \Standort{YCGL, MSS 31.}
\physDesc{Kartenbrief
\newline{}Handschrift Arthur Schnitzler: Bleistift, deutsche Kurrent\newline{}Handschrift Felix Salten: Bleistift, lateinische Kurrent\newline{}Versand: 1) Stempel: »\nobreak{}\oindex{Poertschach@\textbf{Pörtschach}|pwk}{[}Pört{]}schach am See, 27 8 93\nobreak{}«.   2) Stempel: »\nobreak{}\oindex{Znaim@\textbf{Znaim}|pwk}Znaim
                                        Znoj\textcolor{gray}{m}o, 28 8 93, 1\textcolor{gray}{2}–4 N\nobreak{}«. }\buchAbdrucke{\weitereDrucke{Arthur Schnitzler, Richard Beer-Hofmann: \emph{Briefwechsel 1891–1931}. Hg. Konstanze Fliedl. Wien, Zürich: \emph{Europaverlag} 1992, S. 51.} }\toendnotes[C]{\smallbreak}\pstart{}{\pb}Herrn \textsc{Dr. Richard
                            Beer-Hofmann}\pend{}\pstart{}k. k. Lieutenant im Infanterie-Regimente Nr. 99\pend{}\pstart{}\textsc{Znaim\oindex{Znaim@\textbf{Znaim}|pw}}\pend{}\pstart{}Mähren\oindex{Maehren@\textbf{Mähren}|pw}
                            (?)\pend{}{\bigskip}\pstart
           \noindent{}{\pb}Lieber Richard, aus \textsc{Pieve di Cadore}\oindex{Pieve di Cadore@\textbf{Pieve di Cadore}|pw}{ }ſchrieben wir dem Verfaſſer\pwindex{Hofmannsthal, Hugo von 1874-02-01 – 1929-07-15@\textsc{Hofmannsthal, Hugo von} (1874-02-01 – 1929-07-15), \emph{Schriftsteller}|pwv}
                    von Tizians Tod\pwindex{Hofmannsthal, Hugo von 1874-02-01 – 1929-07-15@\textsc{Hofmannsthal, Hugo von} (1874-02-01 – 1929-07-15), \emph{Schriftsteller}!Tod des TizianOktober 1892@\strich\emph{Der Tod des Tizian} {[}Oktober 1892{]}|pw}; – aus \textsc{Pörtschach}\oindex{Poertschach@\textbf{Pörtschach}|pw} dem Verfaſſer des Kindes\pwindex{Beer-Hofmann, Richard 1866-07-11 – 1945-09-26@\textsc{Beer-Hofmann, Richard} (1866-07-11 – 1945-09-26), \emph{Schriftsteller}!Kind1893@\strich\emph{Das Kind} {[}1893{]}|pw} – denn
                    ebenſowahr es iſt dß \textsc{Tizian}\pwindex{Tizian zwischen 1488 und 1490 – 27.08.1576@\textsc{Tizian} (zwischen 1488 und 1490 – 27.08.1576), \emph{Maler}|pw} in \textsc{Pieve di Cadore}\oindex{Pieve di Cadore@\textbf{Pieve di Cadore}|pw} geboren word\textcolor{gray}{en}, ebenſo wahr iſt es, dß hier ſchon manches Kind
                    geboren ward.\pend
           \pstart
           – Wir haben eine ſchöne Tour gemacht; näheres mündlich. Ihnen gehts hoffentlich
                    gut, und wir grüßen Sie herzlich!\pend
           \pstart \spacefill\mbox{Arthur}\pend{}\pstart
           \noindent{}{[}hs. Salten:{]} Ich habe Sie hier ohne Backenbart gesehen, sorgen
                    dafür, dass er rasch wieder wächst. Frl. Anna
                        Hiller\pwindex{Kupelwieser, Anna 20.09.1874 – 08.04.1937@\textsc{Kupelwieser, Anna} (20.09.1874 – 08.04.1937)|pw}, die mir das Bild zeigte grüßt Sie. Ich auch\pend
           \pstart
           Ihr{\\[\baselineskip]}\spacefill\mbox{Salten}\pend
           \leftskip=0em{}
         
         \endnumbering\mylabel{h}\end{ledgroupsized}  \newcommand{\dateiname}{L00260}\newcommand{\titel}{Arthur Schnitzler und Felix Salten an Richard Beer-Hofmann, 27. 8. 1893}\newcommand{\editorInnen}{Martin Anton Müller und Gerd-Hermann Susen}%% latex-leseansicht-abspann.tex
%% Abspann für die Leseansicht.
%% Der Schalter \ifkorrekturansicht ist bereits durch den Vorspann gesetzt.

%% latex-abspann.tex
%% Gemeinsamer Abspann für Korrekturansicht und Leseansicht.
%% Setzt den Schalter \ifkorrekturansicht voraus (gesetzt in den
%% einbindenden Dateien latex-korrekturansicht-abspann.tex bzw.
%% latex-leseansicht-abspann.tex).
%% ---------------------------------------------------------------

\normalsize

% Das esempio-Environment wird nur in der Leseansicht benötigt
\ifkorrekturansicht\else
\newenvironment{esempio}[3]%
{
    \vspace{1.5ex}
    \rlap{\underline{#1}}
    \par
    \setlength{\parindent}{0cm}
    \nopagebreak
    \leftskip=#2cm
    \rightskip=#3cm
}
{
    \par
}
\fi

\doendnotes{C}
\bigskip
\vfill

\clearpage

\footnotesize

\ifkorrekturansicht
  \lohead{\textsc{register}}
\fi

% theindex-Environment neu definieren ohne reledmac
\makeatletter
\renewenvironment{theindex}{%
  \ifkorrekturansicht
    \section*{\indexname}%
  \else
    \subsubsection*{Index der erwähnten Entitäten}%
  \fi
  \setlength{\parindent}{0pt}%
  \setlength{\parskip}{0pt plus 0.3pt}%
  \let\item\@idxitem
}{%
  \ifkorrekturansicht\clearpage\fi
}
\makeatother

\IfFileExists{\jobname-pw.ind}{\input{\jobname-pw.ind}}{}

% Quellenangabe nur in der Leseansicht
\ifkorrekturansicht\else
% Fallback-Definitionen, falls die .tex-Datei \titel etc. nicht gesetzt hat
\providecommand{\titel}{}
\providecommand{\editorInnen}{}
\providecommand{\dateiname}{\jobname}

\vspace{3cm}

\vfill

\footnotesize
\textsc{Quelle}: \titel. Herausgegeben von {\editorInnen}. In: \emph{Arthur Schnitzler: Briefwechsel mit Autorinnen und Autoren}.
 Digitale Edition, https://schnitzler-briefe.acdh.oeaw.ac.at/{\dateiname}.html (Stand \today)
\fi

\end{document}


      