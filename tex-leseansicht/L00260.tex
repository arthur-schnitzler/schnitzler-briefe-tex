%% latex-leseansicht-vorspann.tex
%% Vorspann für die Leseansicht.
%% Lädt die gemeinsame Datei latex-vorspann.tex mit nicht gesetztem Schalter.

\newif\ifkorrekturansicht
\korrekturansichtfalse

\input{../tex-inputs/latex-vorspann}


\section[Arthur Schnitzler und Felix Salten an Richard Beer-Hofmann, 27. 8. 1893]{L00260 Arthur Schnitzler und Felix Salten an Richard Beer-Hofmann, 27. 8. 1893}
\nopagebreak\mylabel{L00260v}
\rehead{ }\normalsize\beginnumbering\briefempfaengerindex{Beer-Hofmann, Richard@\textsc{Beer-Hofmann, Richard}!zzzSalten, Felix@\emph{von Felix Salten}!1893-08-271@{27. 8. 1893}|(be}\briefempfaengerindex{Beer-Hofmann, Richard@\textsc{Beer-Hofmann, Richard}!zzzSchnitzler, Arthur@\emph{von Arthur Schnitzler}!1893-08-271@{27. 8. 1893}|(be}
\toendnotes[C]{\smallbreak\pagebreak[2]}
\correspDesc{Versand  durch Arthur Schnitzler, Felix Salten am 27. 8. 1893 in Pörtschach
\newline{}Erhalt  durch Richard Beer-Hofmann im Zeitraum [28. 8. 1893
                  – 1. 9. 1893?] in Znaim}\toendnotes[C]{\smallbreak}
\Standort{YCGL, MSS 31.}
\physDesc{Kartenbrief, 604 Zeichen
\newline{}Handschrift Arthur Schnitzler: Bleistift, deutsche Kurrent
\newline{}Handschrift Felix Salten: Bleistift, lateinische Kurrent
\newline{}Versand: 1) Stempel: »\nobreak{}\oindex{Pörtschach am Wörthersee@\textbf{Pörtschach am Wörthersee}|pwk}{[}Pört{]}schach am See, 27 8 93\nobreak{}«.   2) Stempel: »\nobreak{}\oindex{Znaim@\textbf{Znaim}, \emph{Hauptstadt}|pwk}Znaim
                                       Znoj\textcolor{gray}{m}o, 28 8 93, 1\textcolor{gray}{2}–4 N\nobreak{}«. }
\buchAbdrucke{\weitereDrucke{Arthur Schnitzler, Richard Beer-Hofmann: \emph{Briefwechsel 1891–1931}. Herausgegeben von Konstanze Fliedl. Wien, Zürich: \emph{Europaverlag} 1992, S. 51.} }\toendnotes[C]{\smallbreak}\pstart{}{\pb}Herrn \textsc{Dr. Richard
                     Beer-Hofmann}\pend{}\pstart{}k. k. Lieutenant im Infanterie-Regimente Nr. 99\pend{}\pstart{}\textsc{Znaim\oindex{Znaim@\textbf{Znaim}, \emph{Hauptstadt}|pw}}\pend{}\pstart{}Mähren\oindex{Mähren@\textbf{Mähren}, \emph{Region}|pw} (?)\pend{}{\bigskip}\vspace{1em}
\pstart
           \noindent{}{\pb}Lieber Richard, aus \textsc{Pieve di Cadore}\oindex{Pieve di Cadore@\textbf{Pieve di Cadore}, \emph{Verwaltungsgebiet}|pw}{ }ſchrieben wir dem Verfaſſer\pwindex{Hofmannsthal, Hugo von 1.\,2.\,1874 Wien – 15.\,7.\,1929 Rodaun@\textsc{Hofmannsthal, Hugo von} (1.\,2.\,1874 Wien – 15.\,7.\,1929 Rodaun), \emph{Schriftsteller}|pwv} von Tizians
                  Tod\pwindex{Hofmannsthal, Hugo von 1.\,2.\,1874 Wien – 15.\,7.\,1929 Rodaun@\textsc{Hofmannsthal, Hugo von} (1.\,2.\,1874 Wien – 15.\,7.\,1929 Rodaun), \emph{Schriftsteller}!Tod des Tizian. Ein Bruchstück@\strich\emph{Der Tod des Tizian. Ein Bruchstück}|pw}; – aus \textsc{Pörtschach}\oindex{Pörtschach am Wörthersee@\textbf{Pörtschach am Wörthersee}|pw} dem Verfaſſer des Kindes\pwindex{Beer-Hofmann, Richard 11.\,7.\,1866 Wien – 26.\,9.\,1945 New York City@\textsc{Beer-Hofmann, Richard} (11.\,7.\,1866 Wien – 26.\,9.\,1945 New York City), \emph{Schriftsteller}!Kind@\strich\emph{Das Kind}|pw} – denn ebenſowahr
               es iſt dß \textsc{Tizian}\pwindex{Tizian zwischen 1488 und 1490 Pieve di Cadore – 27.\,8.\,1576 Venedig@\textsc{Tizian} (zwischen 1488 und 1490 Pieve di Cadore – 27.\,8.\,1576 Venedig), \emph{Maler}|pw} in \textsc{Pieve di Cadore}\oindex{Pieve di Cadore@\textbf{Pieve di Cadore}, \emph{Verwaltungsgebiet}|pw} geboren word\textcolor{gray}{en}, ebenſo wahr iſt es, dß hier{ }ſchon manches
               Kind geboren ward.\pend
           
\pstart
           – Wir haben eine{ }ſchöne Tour gemacht; näheres mündlich. Ihnen gehts hoffentlich gut,
               und wir grüßen Sie herzlich!\pend
           \pstart \spacefill\mbox{Arthur}\pend{}\selectlanguage{ngerman}\vspace{1em}
\pstart
           \noindent{}{[}hs. Salten:{]} Ich habe Sie hier ohne Backenbart gesehen, sorgen
               dafür, dass er rasch wieder wächst. Frl. Anna
                  Hiller\pwindex{Kupelwieser, Anna 20.\,9.\,1874 Wien – 8.\,4.\,1937 ebd.@\textsc{Kupelwieser, Anna} (20.\,9.\,1874 Wien – 8.\,4.\,1937 ebd.)|pw}, die mir das Bild zeigte grüßt Sie. Ich auch\pend
           
\pstart
           Ihr{\\[\baselineskip]}\spacefill\mbox{Salten}\pend
           \leftskip=0em{}\selectlanguage{ngerman}\endnumbering\briefempfaengerindex{Beer-Hofmann, Richard@\textsc{Beer-Hofmann, Richard}!zzzSalten, Felix@\emph{von Felix Salten}!1893-08-271@{27. 8. 1893}|)be}\briefempfaengerindex{Beer-Hofmann, Richard@\textsc{Beer-Hofmann, Richard}!zzzSchnitzler, Arthur@\emph{von Arthur Schnitzler}!1893-08-271@{27. 8. 1893}|)be}\mylabel{L00260h}  \newcommand{\dateiname}{L00260}\newcommand{\titel}{Arthur Schnitzler und Felix Salten an Richard Beer-Hofmann, 27. 8. 1893}\newcommand{\editorInnen}{Martin Anton Müller und Gerd-Hermann Susen}%% latex-leseansicht-abspann.tex
%% Abspann für die Leseansicht.
%% Der Schalter \ifkorrekturansicht ist bereits durch den Vorspann gesetzt.

%% latex-abspann.tex
%% Gemeinsamer Abspann für Korrekturansicht und Leseansicht.
%% Setzt den Schalter \ifkorrekturansicht voraus (gesetzt in den
%% einbindenden Dateien latex-korrekturansicht-abspann.tex bzw.
%% latex-leseansicht-abspann.tex).
%% ---------------------------------------------------------------

\normalsize

% Das esempio-Environment wird nur in der Leseansicht benötigt
\ifkorrekturansicht\else
\newenvironment{esempio}[3]%
{
    \vspace{1.5ex}
    \rlap{\underline{#1}}
    \par
    \setlength{\parindent}{0cm}
    \nopagebreak
    \leftskip=#2cm
    \rightskip=#3cm
}
{
    \par
}
\fi

\doendnotes{C}
\bigskip
\vfill

\clearpage

\footnotesize

\ifkorrekturansicht
  \lohead{\textsc{register}}
\fi

% theindex-Environment neu definieren ohne reledmac
\makeatletter
\renewenvironment{theindex}{%
  \ifkorrekturansicht
    \section*{\indexname}%
  \else
    \subsubsection*{Index der erwähnten Entitäten}%
  \fi
  \setlength{\parindent}{0pt}%
  \setlength{\parskip}{0pt plus 0.3pt}%
  \let\item\@idxitem
}{%
  \ifkorrekturansicht\clearpage\fi
}
\makeatother

\IfFileExists{\jobname-pw.ind}{\input{\jobname-pw.ind}}{}

% Quellenangabe nur in der Leseansicht
\ifkorrekturansicht\else
% Fallback-Definitionen, falls die .tex-Datei \titel etc. nicht gesetzt hat
\providecommand{\titel}{}
\providecommand{\editorInnen}{}
\providecommand{\dateiname}{\jobname}

\vspace{3cm}

\vfill

\footnotesize
\textsc{Quelle}: \titel. Herausgegeben von {\editorInnen}. In: \emph{Arthur Schnitzler: Briefwechsel mit Autorinnen und Autoren}.
 Digitale Edition, https://schnitzler-briefe.acdh.oeaw.ac.at/{\dateiname}.html (Stand \today)
\fi

\end{document}


