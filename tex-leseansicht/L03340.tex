%% latex-leseansicht-vorspann.tex
%% Vorspann für die Leseansicht.
%% Lädt die gemeinsame Datei latex-vorspann.tex mit nicht gesetztem Schalter.

\newif\ifkorrekturansicht
\korrekturansichtfalse

\input{../tex-inputs/latex-vorspann}


\section[ Felix Salten an Arthur Schnitzler, [19.? 3. 1903]]{L03340 Felix Salten an Arthur Schnitzler,  [19.? 3. 1903]}
\nopagebreak\mylabel{L03340v}
\rehead{ }\normalsize\beginnumbering\briefempfaengerindex{Schnitzler, Arthur@\textsc{Schnitzler, Arthur}!zzzSalten, Felix@\emph{von Felix Salten}!1903-03-191@{{[}19.? 3. 1903{]}}|(be}
\toendnotes[C]{\smallbreak\pagebreak[2]}
\correspDesc{Versand  durch Felix Salten am [19.? 3. 1903] in Wien
\newline{}Erhalt  durch Arthur Schnitzler am [19.? 3. 1903] in Wien}\toendnotes[C]{\smallbreak}
\Standort{CUL, Schnitzler, B 89, A 2.}
\physDesc{Brief, 1 Blatt, 4 Seiten, 1373 Zeichen
\newline{}Handschrift: Bleistift, lateinische Kurrent
\newline{}Schnitzler: mit Bleistift datiert: »März 903.« 
\newline{}Ordnung: mit Bleistift von unbekannter Hand nummeriert: »165« }\toendnotes[C]{\smallbreak}
\pstart
           \noindent{}{\pb}Lieber – Goldmanns\pwindex{Goldmann, Paul 31.\,1.\,1865 Breslau – 25.\,9.\,1935 Wien@\textsc{Goldmann, Paul} (31.\,1.\,1865 Breslau – 25.\,9.\,1935 Wien), \emph{Schriftsteller, Journalist}|pw}{ }\label{K_L03340-1v}\edtext{Feuilleton\pwindex{Goldmann, Paul 31.\,1.\,1865 Breslau – 25.\,9.\,1935 Wien@\textsc{Goldmann, Paul} (31.\,1.\,1865 Breslau – 25.\,9.\,1935 Wien), \emph{Schriftsteller, Journalist}!Berliner Theater. (»Der Schleier der Beatrice« von Arthur Schnitzler.)@\strich\emph{Berliner Theater. (»Der Schleier der Beatrice« von Arthur Schnitzler.)}|pwv}}{\lemma{\textnormal{\emph{Feuilleton}}}\Cendnote{\textnormal{Paul Goldmann\pwindex{Goldmann, Paul 31.\,1.\,1865 Breslau – 25.\,9.\,1935 Wien@\textsc{Goldmann, Paul} (31.\,1.\,1865 Breslau – 25.\,9.\,1935 Wien), \emph{Schriftsteller, Journalist}|pwk}: \emph{Berliner Theater. (»Der Schleier der Beatrice« von Arthur
                        Schnitzler)}\pwindex{Goldmann, Paul 31.\,1.\,1865 Breslau – 25.\,9.\,1935 Wien@\textsc{Goldmann, Paul} (31.\,1.\,1865 Breslau – 25.\,9.\,1935 Wien), \emph{Schriftsteller, Journalist}!Berliner Theater. (»Der Schleier der Beatrice« von Arthur Schnitzler.)@\strich\emph{Berliner Theater. (»Der Schleier der Beatrice« von Arthur Schnitzler.)}|pwk}. In: \emph{Neue Freie
                        Presse}\pwindex{Neue Freie Presse@\emph{Neue Freie Presse}|pwk}, Nr. 13.851, 19. 3. 1903,
                     Morgenblatt, S. 1–5.}}}\label{K_L03340-1} ist mir – bei allen Erklärungen, die wir uns
               darüber geben und finden können – doch räthselhaft. Ich bin über die kleinliche und
               kleingeistige Form erstaunt, und wundere mich, dass einem Werk wie dem »Schleier\pwindex{Schnitzler, Arthur 15.\,5.\,1862 Wien – 21.\,10.\,1931 ebd.@\textsc{Schnitzler, Arthur} (15.\,5.\,1862 Wien – 21.\,10.\,1931 ebd.), \emph{Schriftsteller, Mediziner}!Schleier der Beatrice. Schauspiel in fünf Akten@\strich\emph{Der Schleier der Beatrice. Schauspiel in fünf Akten}|pw}« gegenüber, der schärfste kritische
               Angriff in der Plattitüde gipfelt: »denn es ist besser lebendig sein ec.« So gesehen
               allerdings müßen sich alle Zusammenhänge verlieren. Dass Filippo\pwindex{Schnitzler, Arthur 15.\,5.\,1862 Wien – 21.\,10.\,1931 ebd.@\textsc{Schnitzler, Arthur} (15.\,5.\,1862 Wien – 21.\,10.\,1931 ebd.), \emph{Schriftsteller, Mediziner}!Schleier der Beatrice. Schauspiel in fünf Akten@\strich\emph{Der Schleier der Beatrice. Schauspiel in fünf Akten}|pwv} durch den Treuebruch gegen die Teresina\pwindex{Schnitzler, Arthur 15.\,5.\,1862 Wien – 21.\,10.\,1931 ebd.@\textsc{Schnitzler, Arthur} (15.\,5.\,1862 Wien – 21.\,10.\,1931 ebd.), \emph{Schriftsteller, Mediziner}!Schleier der Beatrice. Schauspiel in fünf Akten@\strich\emph{Der Schleier der Beatrice. Schauspiel in fünf Akten}|pwv}{ }{\pb}aus den Angeln
                  ge\textcolor{gray}{stoß}en wird, und dass er im Verlust dieser edelsten
               Doppelbeziehung (Teresina\pwindex{Schnitzler, Arthur 15.\,5.\,1862 Wien – 21.\,10.\,1931 ebd.@\textsc{Schnitzler, Arthur} (15.\,5.\,1862 Wien – 21.\,10.\,1931 ebd.), \emph{Schriftsteller, Mediziner}!Schleier der Beatrice. Schauspiel in fünf Akten@\strich\emph{Der Schleier der Beatrice. Schauspiel in fünf Akten}|pw}{ }{\kaufmannsund} ihr Bruder) schon sich selbst verloren hat, das
               übersieht G.\pwindex{Goldmann, Paul 31.\,1.\,1865 Breslau – 25.\,9.\,1935 Wien@\textsc{Goldmann, Paul} (31.\,1.\,1865 Breslau – 25.\,9.\,1935 Wien), \emph{Schriftsteller, Journalist}|pw} oder er unterschlägt es. Ich
               bedauere dieses Feuilleton\pwindex{Schnitzler, Arthur 15.\,5.\,1862 Wien – 21.\,10.\,1931 ebd.@\textsc{Schnitzler, Arthur} (15.\,5.\,1862 Wien – 21.\,10.\,1931 ebd.), \emph{Schriftsteller, Mediziner}!Schleier der Beatrice. Schauspiel in fünf Akten@\strich\emph{Der Schleier der Beatrice. Schauspiel in fünf Akten}|pwv} aus
               vielen künstlerischen und menschlichen Gründen, und vor allem deshalb, weil es der in
                  Wien\oindex{Wien@\textbf{Wien}, \emph{Verwaltungsgebiet}|pw} spielenden \label{K_L03340-2v}\edtext{Schleier\pwindex{Schnitzler, Arthur 15.\,5.\,1862 Wien – 21.\,10.\,1931 ebd.@\textsc{Schnitzler, Arthur} (15.\,5.\,1862 Wien – 21.\,10.\,1931 ebd.), \emph{Schriftsteller, Mediziner}!Schleier der Beatrice. Schauspiel in fünf Akten@\strich\emph{Der Schleier der Beatrice. Schauspiel in fünf Akten}|pw}-Affaire}{\lemma{\textnormal{\emph{Schleier-Affaire}}}\Cendnote{\textnormal{Bezug auf die teilweise in der Presse berichteten Vorgänge
                  aus dem Jahr 1901 um die halbherzige Zu- und nachmalige
                  Absage Paul Schlenthers\pwindex{Schlenther, Paul 20.\,8.\,1854 Chernyakhovsk – 30.\,4.\,1916 Berlin@\textsc{Schlenther, Paul} (20.\,8.\,1854 Chernyakhovsk – 30.\,4.\,1916 Berlin), \emph{Schriftsteller, Kritiker, Theaterleiter}|pwk}, das Stück\pwindex{Schnitzler, Arthur 15.\,5.\,1862 Wien – 21.\,10.\,1931 ebd.@\textsc{Schnitzler, Arthur} (15.\,5.\,1862 Wien – 21.\,10.\,1931 ebd.), \emph{Schriftsteller, Mediziner}!Schleier der Beatrice. Schauspiel in fünf Akten@\strich\emph{Der Schleier der Beatrice. Schauspiel in fünf Akten}|pwkv} am \emph{Burgtheater}\orgindex{Burgtheater@Burgtheater|pwk} aufzuführen}}}\label{K_L03340-2} vorläufig {\pb}einen unrühmlichen Abschluß
               gibt. Gerade mit Bezug \uline{darauf} bin ich von diesem
               Vorgehen doppelt impressionirt, denn \uline{G.} war in Wien\oindex{Wien@\textbf{Wien}, \emph{Verwaltungsgebiet}|pw} als die Affaire spielte, er hat mitgeholfen
               und mitgerathen, ist mitempört gewesen, war mit mir bei Burckhard\pwindex{Burckhard, Max Eugen 14.\,7.\,1854 Korneuburg – 16.\,3.\,1912 Wien@\textsc{Burckhard, Max Eugen} (14.\,7.\,1854 Korneuburg – 16.\,3.\,1912 Wien), \emph{Schriftsteller, Rechtswissenschaftler, Theaterleiter}|pw}{ }{\kaufmannsund} hat sich für dieses Werk\pwindex{Schnitzler, Arthur 15.\,5.\,1862 Wien – 21.\,10.\,1931 ebd.@\textsc{Schnitzler, Arthur} (15.\,5.\,1862 Wien – 21.\,10.\,1931 ebd.), \emph{Schriftsteller, Mediziner}!Schleier der Beatrice. Schauspiel in fünf Akten@\strich\emph{Der Schleier der Beatrice. Schauspiel in fünf Akten}|pwv}, über das er damals freilich anders sprach als
                  heute{[},{]} sehr engagirt.\pend
           
\pstart
           Entschuldigen Sie diese {\pb}»Kundgebung.« Sehe ich Sie \label{K_L03340-3v}\edtext{heute{ }Abend im Café}{\lemma{\textnormal{\emph{heute Abend im Café}}}\Cendnote{\textnormal{Nachweisbar
                  war Schnitzler abends bei Olga Gussmann\pwindex{Schnitzler, Olga 17.\,1.\,1882 Wien – 13.\,1.\,1970 Lugano@\textsc{Schnitzler, Olga} (17.\,1.\,1882 Wien – 13.\,1.\,1970 Lugano), \emph{Schauspielerin, Sängerin}|pwk}, vgl. A. S.: \emph{Tagebuch}, 19. 3. 1903.}}}\label{K_L03340-3}? Ich bin etwa um 11
               dort.\pend
           
\pstart
           Der Titel \label{K_L03340-4v}\edtext{\uline{Interview\pwindex{Bauernfeld-Preis. Eine Interpellation@\emph{Der Bauernfeld-Preis. Eine Interpellation}|pwv}} ist durch ein Missverständnis}{\lemma{\textnormal{\emph{Interview … Missverständnis}}}\Cendnote{\textnormal{[Felix Salten\pwindex{Salten, Felix 6.\,9.\,1869 Budapest – 8.\,10.\,1945 Zürich@\textsc{Salten, Felix} (6.\,9.\,1869 Budapest – 8.\,10.\,1945 Zürich), \emph{Schriftsteller, Journalist, Chefredakteur}|pwk}]: \emph{Der Bauernfeld-Preis. Eine Interpellation}\pwindex{Bauernfeld-Preis. Eine Interpellation@\emph{Der Bauernfeld-Preis. Eine Interpellation}|pwk}. In: \emph{Die Zeit}\pwindex{Zeit@\emph{Die Zeit}|pwk}, Jg. 2, Nr. 169, 19. 3. 1903, S. 5. Darin ist die den
                  Aussagen Schnitzlers gewidmete Stelle mit
                  der Überschrift »Ein Interview mit Arthur
                        Schnitzler\pwindex{Bauernfeld-Preis. Eine Interpellation@\emph{Der Bauernfeld-Preis. Eine Interpellation}|pwv}« versehen.}}}\label{K_L03340-4}{ }\label{K_L03340-5v}\edtext{heute{ }Nachts}{\lemma{\textnormal{\emph{heute Nachts}}}\Cendnote{\textnormal{Das erlaubt die Datierung des
                  Korrespondenzstücks auf den Tag, an dem \emph{Der
                     Bauernfeld-Preis. Eine Interpellation}\pwindex{Bauernfeld-Preis. Eine Interpellation@\emph{Der Bauernfeld-Preis. Eine Interpellation}|pwk} erschienen ist.}}}\label{K_L03340-5}{ }3\textsuperscript{h} als ich schon fort war ins Blatt\pwindex{Zeit@\emph{Die Zeit}|pwv} gekommen. D\textsuperscript{r}{ }Kanner\pwindex{Kanner, Heinrich 9.\,11.\,1864 Galați – 15.\,2.\,1930 Wien@\textsc{Kanner, Heinrich} (9.\,11.\,1864 Galați – 15.\,2.\,1930 Wien), \emph{Herausgeber, Publizist}|pw} läßt Sie um Entschuldigung bitten.\pend
           
\pstart
           Herzlichst {\\[\baselineskip]}Ihr \spacefill\mbox{FS}\pend
           \leftskip=0em{}\selectlanguage{ngerman}\endnumbering\briefempfaengerindex{Schnitzler, Arthur@\textsc{Schnitzler, Arthur}!zzzSalten, Felix@\emph{von Felix Salten}!1903-03-191@{{[}19.? 3. 1903{]}}|)be}\mylabel{L03340h}  \newcommand{\dateiname}{L03340}\newcommand{\titel}{Felix Salten an Arthur Schnitzler, [19.? 3. 1903]}\newcommand{\editorInnen}{Martin Anton Müller und Laura Untner}%% latex-leseansicht-abspann.tex
%% Abspann für die Leseansicht.
%% Der Schalter \ifkorrekturansicht ist bereits durch den Vorspann gesetzt.

%% latex-abspann.tex
%% Gemeinsamer Abspann für Korrekturansicht und Leseansicht.
%% Setzt den Schalter \ifkorrekturansicht voraus (gesetzt in den
%% einbindenden Dateien latex-korrekturansicht-abspann.tex bzw.
%% latex-leseansicht-abspann.tex).
%% ---------------------------------------------------------------

\normalsize

% Das esempio-Environment wird nur in der Leseansicht benötigt
\ifkorrekturansicht\else
\newenvironment{esempio}[3]%
{
    \vspace{1.5ex}
    \rlap{\underline{#1}}
    \par
    \setlength{\parindent}{0cm}
    \nopagebreak
    \leftskip=#2cm
    \rightskip=#3cm
}
{
    \par
}
\fi

\doendnotes{C}
\bigskip
\vfill

\clearpage

\footnotesize

\ifkorrekturansicht
  \lohead{\textsc{register}}
\fi

% theindex-Environment neu definieren ohne reledmac
\makeatletter
\renewenvironment{theindex}{%
  \ifkorrekturansicht
    \section*{\indexname}%
  \else
    \subsubsection*{Index der erwähnten Entitäten}%
  \fi
  \setlength{\parindent}{0pt}%
  \setlength{\parskip}{0pt plus 0.3pt}%
  \let\item\@idxitem
}{%
  \ifkorrekturansicht\clearpage\fi
}
\makeatother

\IfFileExists{\jobname-pw.ind}{\input{\jobname-pw.ind}}{}

% Quellenangabe nur in der Leseansicht
\ifkorrekturansicht\else
% Fallback-Definitionen, falls die .tex-Datei \titel etc. nicht gesetzt hat
\providecommand{\titel}{}
\providecommand{\editorInnen}{}
\providecommand{\dateiname}{\jobname}

\vspace{3cm}

\vfill

\footnotesize
\textsc{Quelle}: \titel. Herausgegeben von {\editorInnen}. In: \emph{Arthur Schnitzler: Briefwechsel mit Autorinnen und Autoren}.
 Digitale Edition, https://schnitzler-briefe.acdh.oeaw.ac.at/{\dateiname}.html (Stand \today)
\fi

\end{document}


