%% latex-korrekturansicht-vorspann.tex
%% Vorspann für die Korrekturansicht.
%% Lädt die gemeinsame Datei latex-vorspann.tex mit gesetztem Schalter.

\newif\ifkorrekturansicht
\korrekturansichttrue

\input{../tex-inputs/latex-vorspann}


\section[ Felix Salten an Arthur Schnitzler, {[}19.? 3. 1903{]}]{L03340 Felix Salten an Arthur Schnitzler, {[}19.? 3. 1903{]}}
\nopagebreak\mylabel{L03340v}
\rehead{ }\normalsize\beginnumbering\briefempfaengerindex{Schnitzler, Arthur@\textsc{Schnitzler, Arthur}!zzzSalten, Felix@\emph{von Felix Salten}!1903-03-191@{{[}19.? 3. 1903{]}}|(be}
\toendnotes[C]{\smallbreak\pagebreak[2]}\Standort{CUL, Schnitzler, B 89, A 2.}
\physDesc{Brief, 1 Blatt, 4 Seiten, 1373 Zeichen
\newline{}Handschrift: Bleistift, lateinische Kurrent
\newline{}Schnitzler: mit Bleistift datiert: »März 903.« 
\newline{}Ordnung: mit Bleistift von unbekannter Hand nummeriert: »165« }\toendnotes[C]{\smallbreak}
\pstart
           \noindent{}{\pb}Lieber – Goldmanns\pwindex{Goldmann, Paul 31.01.1865 – 25.09.1935@\textsc{Goldmann, Paul} (31.01.1865 – 25.09.1935), \emph{Schriftsteller/Schriftstellerin, Journalist/Journalistin}|pw}{ }\label{K_L03340-1v}\edtext{Feuilleton\pwindex{Berliner Theater. (»Der Schleier der Beatrice« von Arthur Schnitzler.)@\emph{Berliner Theater. (»Der Schleier der Beatrice« von Arthur Schnitzler.)}|pwv}}{\lemma{\textnormal{\emph{Feuilleton}}}\Cendnote{\textnormal{Paul Goldmann\pwindex{Goldmann, Paul 31.01.1865 – 25.09.1935@\textsc{Goldmann, Paul} (31.01.1865 – 25.09.1935), \emph{Schriftsteller/Schriftstellerin, Journalist/Journalistin}|pwk}: \emph{Berliner Theater. (»Der Schleier der Beatrice« von Arthur
                        Schnitzler)}\pwindex{Berliner Theater. (»Der Schleier der Beatrice« von Arthur Schnitzler.)@\emph{Berliner Theater. (»Der Schleier der Beatrice« von Arthur Schnitzler.)}|pwk}. In: \emph{Neue Freie
                        Presse}\pwindex{Neue Freie Presse@\emph{Neue Freie Presse}|pwk}, Nr. 13.851, 19. 3. 1903,
                     Morgenblatt, S. 1–5.}}}\label{K_L03340-1} ist mir – bei allen Erklärungen, die wir uns
               darüber geben und finden können – doch räthselhaft. Ich bin über die kleinliche und
               kleingeistige Form erstaunt, und wundere mich, dass einem Werk wie dem »Schleier\pwindex{Schleier der Beatrice. Schauspiel in fuenf Akten@\emph{Der Schleier der Beatrice. Schauspiel in fünf Akten}|pw}« gegenüber, der schärfste kritische
               Angriff in der Plattitüde gipfelt: »denn es ist besser lebendig sein ec.« So gesehen
               allerdings müßen sich alle Zusammenhänge verlieren. Dass Filippo\pwindex{Schleier der Beatrice. Schauspiel in fuenf Akten@\emph{Der Schleier der Beatrice. Schauspiel in fünf Akten}|pwv} durch den Treuebruch gegen die Teresina\pwindex{Schleier der Beatrice. Schauspiel in fuenf Akten@\emph{Der Schleier der Beatrice. Schauspiel in fünf Akten}|pwv}{ }{\pb}aus den Angeln
                  ge\textcolor{gray}{stoß}en wird, und dass er im Verlust dieser edelsten
               Doppelbeziehung (Teresina\pwindex{Schleier der Beatrice. Schauspiel in fuenf Akten@\emph{Der Schleier der Beatrice. Schauspiel in fünf Akten}|pw}{ }{\kaufmannsund} ihr Bruder) schon sich selbst verloren hat, das
               übersieht G.\pwindex{Goldmann, Paul 31.01.1865 – 25.09.1935@\textsc{Goldmann, Paul} (31.01.1865 – 25.09.1935), \emph{Schriftsteller/Schriftstellerin, Journalist/Journalistin}|pw} oder er unterschlägt es. Ich
               bedauere dieses Feuilleton\pwindex{Schleier der Beatrice. Schauspiel in fuenf Akten@\emph{Der Schleier der Beatrice. Schauspiel in fünf Akten}|pwv} aus
               vielen künstlerischen und menschlichen Gründen, und vor allem deshalb, weil es der in
                  Wien\oindex{Wien@\textbf{Wien}, \emph{A.ADM2}|pw} spielenden \label{K_L03340-2v}\edtext{Schleier\pwindex{Schleier der Beatrice. Schauspiel in fuenf Akten@\emph{Der Schleier der Beatrice. Schauspiel in fünf Akten}|pw}-Affaire}{\lemma{\textnormal{\emph{Schleier-Affaire}}}\Cendnote{\textnormal{Bezug auf die teilweise in der Presse berichteten Vorgänge
                  aus dem Jahr 1901 um die halbherzige Zu- und nachmalige
                  Absage Paul Schlenthers\pwindex{Schlenther, Paul 20.08.1854 – 30.04.1916@\textsc{Schlenther, Paul} (20.08.1854 – 30.04.1916), \emph{Schriftsteller/Schriftstellerin, Kritiker/Kritikerin, Theaterleiter/Theaterleiterin}|pwk}, das Stück\pwindex{Schleier der Beatrice. Schauspiel in fuenf Akten@\emph{Der Schleier der Beatrice. Schauspiel in fünf Akten}|pwkv} am \emph{Burgtheater}\orgindex{Burgtheater@Burgtheater|pwk} aufzuführen}}}\label{K_L03340-2} vorläufig {\pb}einen unrühmlichen Abschluß
               gibt. Gerade mit Bezug \uline{darauf} bin ich von diesem
               Vorgehen doppelt impressionirt, denn \uline{G.} war in Wien\oindex{Wien@\textbf{Wien}, \emph{A.ADM2}|pw} als die Affaire spielte, er hat mitgeholfen
               und mitgerathen, ist mitempört gewesen, war mit mir bei Burckhard\pwindex{Burckhard, Max Eugen 14.07.1854 – 16.03.1912@\textsc{Burckhard, Max Eugen} (14.07.1854 – 16.03.1912), \emph{Schriftsteller/Schriftstellerin, Rechtswissenschaftler/Rechtswissenschaftlerin, Theaterleiter/Theaterleiterin}|pw}{ }{\kaufmannsund} hat sich für dieses Werk\pwindex{Schleier der Beatrice. Schauspiel in fuenf Akten@\emph{Der Schleier der Beatrice. Schauspiel in fünf Akten}|pwv}, über das er damals freilich anders sprach als
                  heute{[},{]} sehr engagirt.\pend
           
\pstart
           Entschuldigen Sie diese {\pb}»Kundgebung.« Sehe ich Sie \label{K_L03340-3v}\edtext{heute{ }Abend im Café}{\lemma{\textnormal{\emph{heute Abend im Café}}}\Cendnote{\textnormal{Nachweisbar
                  war Schnitzler abends bei Olga Gussmann\pwindex{Schnitzler, Olga 17.01.1882 – 13.01.1970@\textsc{Schnitzler, Olga} (17.01.1882 – 13.01.1970), \emph{Schauspieler/Schauspielerin, Sänger/Sängerin}|pwk}, vgl. A. S.: \emph{Tagebuch}, 19. 3. 1903.}}}\label{K_L03340-3}? Ich bin etwa um 11
               dort.\pend
           
\pstart
           Der Titel \label{K_L03340-4v}\edtext{\uline{Interview\pwindex{Bauernfeld-Preis. Eine Interpellation@\emph{Der Bauernfeld-Preis. Eine Interpellation}|pwv}} ist durch ein Missverständnis}{\lemma{\textnormal{\emph{Interview … Missverständnis}}}\Cendnote{\textnormal{[Felix Salten\pwindex{Salten, Felix 06.09.1869 – 08.10.1945@\textsc{Salten, Felix} (06.09.1869 – 08.10.1945), \emph{Schriftsteller/Schriftstellerin, Journalist/Journalistin, Chefredakteur/Chefredakteurin}|pwk}]: \emph{Der Bauernfeld-Preis. Eine Interpellation}\pwindex{Bauernfeld-Preis. Eine Interpellation@\emph{Der Bauernfeld-Preis. Eine Interpellation}|pwk}. In: \emph{Die Zeit}\pwindex{Zeit@\emph{Die Zeit}|pwk}, Jg. 2, Nr. 169, 19. 3. 1903, S. 5. Darin ist die den
                  Aussagen Schnitzlers gewidmete Stelle mit
                  der Überschrift »Ein Interview mit Arthur
                        Schnitzler\pwindex{Bauernfeld-Preis. Eine Interpellation@\emph{Der Bauernfeld-Preis. Eine Interpellation}|pwv}« versehen.}}}\label{K_L03340-4}{ }\label{K_L03340-5v}\edtext{heute{ }Nachts}{\lemma{\textnormal{\emph{heute Nachts}}}\Cendnote{\textnormal{Das erlaubt die Datierung des
                  Korrespondenzstücks auf den Tag, an dem \emph{Der
                     Bauernfeld-Preis. Eine Interpellation}\pwindex{Bauernfeld-Preis. Eine Interpellation@\emph{Der Bauernfeld-Preis. Eine Interpellation}|pwk} erschienen ist.}}}\label{K_L03340-5}{ }3\textsuperscript{h} als ich schon fort war ins Blatt\pwindex{Zeit@\emph{Die Zeit}|pwv} gekommen. D\textsuperscript{r}{ }Kanner\pwindex{Kanner, Heinrich 09.11.1864 – 15.02.1930@\textsc{Kanner, Heinrich} (09.11.1864 – 15.02.1930), \emph{Herausgeber/Herausgeberin, Publizist/Publizistin}|pw} läßt Sie um Entschuldigung bitten.\pend
           
\pstart
           Herzlichst {\\[\baselineskip]}Ihr \spacefill\mbox{FS}\pend
           \leftskip=0em{}\selectlanguage{ngerman}\endnumbering\briefempfaengerindex{Schnitzler, Arthur@\textsc{Schnitzler, Arthur}!zzzSalten, Felix@\emph{von Felix Salten}!1903-03-191@{{[}19.? 3. 1903{]}}|)be}\mylabel{L03340h}  \normalsize

\doendnotes{C}
\bigskip
\vfill

\clearpage

\footnotesize

\lohead{\textsc{register}}

% Definiere theindex-Environment komplett neu ohne reledmac
\makeatletter
\renewenvironment{theindex}{%
  \section*{\indexname}%
  \setlength{\parindent}{0pt}%
  \setlength{\parskip}{0pt plus 0.3pt}%
  \let\item\@idxitem
}{%
  \clearpage
}
\makeatother

\IfFileExists{\jobname-pw.ind}{\input{\jobname-pw.ind}}{}

\end{document}

      