%% latex-korrekturansicht-vorspann.tex
%% Vorspann für die Korrekturansicht.
%% Lädt die gemeinsame Datei latex-vorspann.tex mit gesetztem Schalter.

\newif\ifkorrekturansicht
\korrekturansichttrue

\input{../tex-inputs/latex-vorspann}


\section[Arthur Schnitzler an Peter Altenberg, 29. 10. 1896]{L00611 Arthur Schnitzler an Peter Altenberg, 29. 10. 1896}
\nopagebreak\mylabel{L00611v}
\rehead{ }\normalsize\beginnumbering\briefempfaengerindex{Altenberg, Peter@\textsc{Altenberg, Peter}!zzzSchnitzler, Arthur@\emph{von Arthur Schnitzler}!1896-10-291@{29. 10. 1896}|(be}
\toendnotes[C]{\smallbreak\pagebreak[2]}\Standort{Wienbibliothek im Rathaus, H.I.N.-137077.}
\physDesc{Brief, Fotokopie1 Blatt, 2 Seiten, 465 Zeichen
\newline{}Handschrift: schwarze Tinte, lateinische Kurrent
\newline{}Altenberg: Ergänzung, nur zwei der vier Zeilen der Notiz sind ansatzweise
                                 zu entziffern: »\noindent{}\textsc{Lendway}{ / }\textsc{II. \textcolor{gray}{A}\textcolor{gray}{×}\-\textcolor{gray}{×}\-\textcolor{gray}{×}\-\textcolor{gray}{×}\-\textcolor{gray}{×}gaſſe 5}«. Karl Kraus\pwindex{Kraus, Karl 28.04.1874 – 12.06.1936@\textsc{Kraus, Karl} (28.04.1874 – 12.06.1936), \emph{Schriftsteller/Schriftstellerin, Publizist/Publizistin, Schriftsteller/Schriftstellerin}|pw}
                                 beschrieb diesen Text: »Der Wert des Autogramms ist
                                    allerdings beträchtlich erhöht durch eine Randnotiz Peter
                                    Altenbergs, der die ihm widerfahrene literarische Weihe mit den
                                    Adressen eines Nachtcafés und offenbar einer von dessen
                                    Besucherinnen quittiert hat«. \emph{Die Fackel}\pwindex{Fackel@\emph{Die Fackel}|pwk}, Jg. 24,
                                    Nr. 608–612, Ende Dezember 1922,
                                 S. 52. 
\newline{}Ordnung: Im Nachlass von Karl
                                    Kraus\pwindex{Kraus, Karl 28.04.1874 – 12.06.1936@\textsc{Kraus, Karl} (28.04.1874 – 12.06.1936), \emph{Schriftsteller/Schriftstellerin, Publizist/Publizistin, Schriftsteller/Schriftstellerin}|pw} überliefert. Kraus\pwindex{Kraus, Karl 28.04.1874 – 12.06.1936@\textsc{Kraus, Karl} (28.04.1874 – 12.06.1936), \emph{Schriftsteller/Schriftstellerin, Publizist/Publizistin, Schriftsteller/Schriftstellerin}|pw} ergänzte (vor der Kopie) am Objekt:
                                    »handſchriftliche Notiz von Peter Altenberg. Das Dokument
                                       1896 von ihm empfangen. Wien\oindex{Wien@\textbf{Wien}, \emph{A.ADM2}|pw}, im November 1922\hspace*{1.5em}Karl Kraus\pwindex{Kraus, Karl 28.04.1874 – 12.06.1936@\textsc{Kraus, Karl} (28.04.1874 – 12.06.1936), \emph{Schriftsteller/Schriftstellerin, Publizist/Publizistin, Schriftsteller/Schriftstellerin}|pw}« 
\newline{}Zusatz: Kraus\pwindex{Kraus, Karl 28.04.1874 – 12.06.1936@\textsc{Kraus, Karl} (28.04.1874 – 12.06.1936), \emph{Schriftsteller/Schriftstellerin, Publizist/Publizistin, Schriftsteller/Schriftstellerin}|pw} ließ das Original
                                 versteigern. Schnitzler bot selber mit, wurde aber überboten.
                                 Vgl. \emph{Briefe 1913–1931}, S. 293–296 und
                                    Die Fackel\pwindex{Fackel@\emph{Die Fackel}|pw} von Ende
                                    1922 bis Anfang 1923.
                               }
\buchAbdrucke{\weitereDrucke{1) \emph{Vorlesung Karl Kraus [Programm]}. (26. 11. 1922).} \weitereDrucke{2) \pwindex{Fackel@\emph{Die Fackel}|pwk}\emph{Die Fackel}, Jg. 24, Nr. 608–612, Ende Dezember 1922, S. 51.} \weitereDrucke{3) \emph{Literatur und Kritik}, Jg. 3 (1968), S. 292–304, hier S. 293.} }\toendnotes[C]{\smallbreak}
\pstart{}{\pb}Lieber Herr Peter Altenberg,\pend\vspace{0.5em}
\pstart
           geſtern ſprach ich mit \textsc{Gerhard Hauptmann}\pwindex{Hauptmann, Gerhart 15.11.1862 – 06.06.1946@\textsc{Hauptmann, Gerhart} (15.11.1862 – 06.06.1946), \emph{Schriftsteller/Schriftstellerin}|pw}, der ſich über Ihr Buch\pwindex{Wie ich es sehe@\emph{Wie ich es sehe}|pwv}
               in unendlich ſympathiſcher Weiſe äußerte u. unter anderm sagte, ſeit \uline{Jahren} habe kein Buch\pwindex{Wie ich es sehe@\emph{Wie ich es sehe}|pwv} einen ſo ſtarken Eindruck auf ihn gemacht als das
               Ihre.\pend
           
\pstart
           Da dieſe Bemerkung für Sie \label{K_L00611-1v}\edtext{intereſſant
                  ſein}{\lemma{\textnormal{\emph{intereſſant
                  ſein}}}\Cendnote{\textnormal{Für Altenberg\pwindex{Altenberg, Peter 09.03.1859 – 08.01.1919@\textsc{Altenberg, Peter} (09.03.1859 – 08.01.1919), \emph{Schriftsteller/Schriftstellerin}|pwk} bot sie den Anlass, Hauptmann\pwindex{Hauptmann, Gerhart 15.11.1862 – 06.06.1946@\textsc{Hauptmann, Gerhart} (15.11.1862 – 06.06.1946), \emph{Schriftsteller/Schriftstellerin}|pwk} direkt einen Brief zu schreiben. (\emph{Selbsterfindung eines Dichters}, S. 80.)}}}\label{K_L00611-1}
               dürfte und ſie ſonſt kaum an Sie {\pb}gelangen
               könnte, fühle ich mich in gewiſſem Sinne angenehm verpflichtet, ſie Ihnen
               mitzutheilen.\pend
           
\pstart
           Mit beſtem Gruſs Ihr ergebener{\\[\baselineskip]}\spacefill\mbox{ArthurSchnitzler}\pend
           \leftskip=0em{}
\pstart
           Berlin\oindex{Berlin@\textbf{Berlin}, \emph{P.PPLC}|pw}, 29. X. 96.\pend
           \selectlanguage{ngerman}\endnumbering\briefempfaengerindex{Altenberg, Peter@\textsc{Altenberg, Peter}!zzzSchnitzler, Arthur@\emph{von Arthur Schnitzler}!1896-10-291@{29. 10. 1896}|)be}\mylabel{L00611h}  \normalsize

\doendnotes{C}
\bigskip
\vfill

\clearpage

\footnotesize

\lohead{\textsc{register}}

% Definiere theindex-Environment komplett neu ohne reledmac
\makeatletter
\renewenvironment{theindex}{%
  \section*{\indexname}%
  \setlength{\parindent}{0pt}%
  \setlength{\parskip}{0pt plus 0.3pt}%
  \let\item\@idxitem
}{%
  \clearpage
}
\makeatother

\IfFileExists{\jobname-pw.ind}{\input{\jobname-pw.ind}}{}

\end{document}

      