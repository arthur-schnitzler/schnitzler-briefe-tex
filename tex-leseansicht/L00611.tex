%% latex-leseansicht-vorspann.tex
%% Vorspann für die Leseansicht.
%% Lädt die gemeinsame Datei latex-vorspann.tex mit nicht gesetztem Schalter.

\newif\ifkorrekturansicht
\korrekturansichtfalse

\input{../tex-inputs/latex-vorspann}


               \section[Arthur Schnitzler an Peter Altenberg, 29. 10. 1896]{ Arthur Schnitzler an Peter Altenberg, 29. 10. 1896}\nopagebreak\mylabel{v}\rehead{ }\begin{ledgroupsized}[t]{13cm}\normalsize\beginnumbering\briefempfaengerindex{Altenberg, Peter@\textsc{Altenberg, Peter}!zzzSchnitzler, Arthur@\emph{von Arthur Schnitzler}!1896-10-291@{29. 10. 1896}|(be} \toendnotes[C]{\smallbreak\pagebreak[2]} \Standort{Wienbibliothek im Rathaus, H.I.N.-137077.}
\physDesc{Brief, 1 Blatt, 2 Seiten, Fotokopie
\newline{}Altenberg: Ergänzung, nur zwei der vier Zeilen der Notiz sind ansatzweise zu
                                 entziffern: »\noindent{}\textsc{Lendway}{ / }\textsc{II. \textcolor{gray}{A}\textcolor{gray}{×}\-\textcolor{gray}{×}\-\textcolor{gray}{×}\-\textcolor{gray}{×}\-\textcolor{gray}{×}gaſſe 5}«. Karl Kraus\pwindex{Kraus, Karl 28.04.1874 – 12.06.1936@\textsc{Kraus, Karl} (28.04.1874 – 12.06.1936), \emph{Schriftsteller, Publizist}|pw} beschrieb
                                 diesen Text: »Der Wert des Autogramms ist allerdings
                                    beträchtlich erhöht durch eine Randnotiz Peter Altenbergs, der
                                    die ihm widerfahrene literarische Weihe mit den Adressen eines
                                    Nachtcafés und offenbar einer von dessen Besucherinnen quittiert
                                    hat«. Die Fackel\pwindex{Fackel1899 – 1936@\emph{Die Fackel}|pw},
                                 Jg. 24, Nr. 608–612, Ende Dezember
                                 1922, S. 52. \newline{}Ordnung: Im Nachlass von Karl Kraus\pwindex{Kraus, Karl 28.04.1874 – 12.06.1936@\textsc{Kraus, Karl} (28.04.1874 – 12.06.1936), \emph{Schriftsteller, Publizist}|pw}
                                 überliefert. Kraus\pwindex{Kraus, Karl 28.04.1874 – 12.06.1936@\textsc{Kraus, Karl} (28.04.1874 – 12.06.1936), \emph{Schriftsteller, Publizist}|pw} ergänzte
                                 (vor der Kopie) am Objekt: »handſchriftliche Notiz von Peter
                                    Altenberg. Das Dokument 1896 von ihm empfangen. Wien\oindex{Wien@\textbf{Wien}|pw}, im November
                                          1922\hspace*{1.5em}Karl Kraus\pwindex{Kraus, Karl 28.04.1874 – 12.06.1936@\textsc{Kraus, Karl} (28.04.1874 – 12.06.1936), \emph{Schriftsteller, Publizist}|pw}« \newline{}Zusatz: Kraus\pwindex{Kraus, Karl 28.04.1874 – 12.06.1936@\textsc{Kraus, Karl} (28.04.1874 – 12.06.1936), \emph{Schriftsteller, Publizist}|pw} ließ das Original
                                 versteigern. Schnitzler bot selber mit, wurde aber überboten.
                                    Vgl. \emph{Briefe} II,293–296 und
                                    Die Fackel\pwindex{Fackel1899 – 1936@\emph{Die Fackel}|pw} von Ende
                                    1922 bis Anfang 1923 }\buchAbdrucke{\weitereDrucke{1) \emph{Vorlesung Karl Kraus [Programm]}. (26. 11. 1922).} \weitereDrucke{2) \pwindex{Fackel1899 – 1936@\emph{Die Fackel}|pwk}\emph{Die Fackel}, Jg. 24, Nr. 608–612, Ende Dezember 1922, S. 51.} \weitereDrucke{3) Reinhard Urbach: \emph{»Schwätzer sind Verbrecher«. Bemerkungen zu Schnitzlers Dramenfragment »Das Wort«.} In: \emph{Literatur und Kritik}, Jg. 3 (1968), S. 292–304, hier S. 293.} }\toendnotes[C]{\smallbreak}\pstart{}{\pb}Lieber Herr Peter Altenberg,\pend\pstart
           geſtern ſprach ich mit \textsc{Gerhard Hauptmann}\pwindex{Hauptmann, Gerhart 15.11.1862 – 06.06.1946@\textsc{Hauptmann, Gerhart} (15.11.1862 – 06.06.1946), \emph{Schriftsteller}|pw}, der ſich über Ihr Buch\pwindex{Altenberg, Peter 09.03.1859 – 08.01.1919@\textsc{Altenberg, Peter} (09.03.1859 – 08.01.1919), \emph{Schriftsteller}!Wie ich es sehe1896@\strich\emph{Wie ich es sehe} {[}1896{]}|pwv} in
               unendlich ſympathiſcher Weiſe äußerte u. unter anderm sagte, ſeit \uline{Jahren} habe kein Buch\pwindex{Altenberg, Peter 09.03.1859 – 08.01.1919@\textsc{Altenberg, Peter} (09.03.1859 – 08.01.1919), \emph{Schriftsteller}!Wie ich es sehe1896@\strich\emph{Wie ich es sehe} {[}1896{]}|pwv} einen ſo ſtarken Eindruck auf ihn gemacht als das
               Ihre.\pend
           \pstart
           Da dieſe Bemerkung für Sie \label{K_L00611_1v}\edtext{intereſſant
                  ſein}{\lemma{\textnormal{\emph{intereſſant
                  ſein}}}\Cendnote{\textnormal{Für Altenberg\pwindex{Altenberg, Peter 09.03.1859 – 08.01.1919@\textsc{Altenberg, Peter} (09.03.1859 – 08.01.1919), \emph{Schriftsteller}|pwk} bot sie den Anlass, Hauptmann\pwindex{Hauptmann, Gerhart 15.11.1862 – 06.06.1946@\textsc{Hauptmann, Gerhart} (15.11.1862 – 06.06.1946), \emph{Schriftsteller}|pwk} direkt einen Brief zu schreiben. (\emph{Selbsterfindung eines Dichters}, S. 80.)}}}\label{K_L00611_1h}
               dürfte und ſie ſonſt kaum an Sie {\pb}gelangen
               könnte, fühle ich mich in gewiſſem Sinne angenehm verpflichtet, ſie Ihnen
               mitzutheilen.\pend
           \pstart
           Mit beſtem Gruſs Ihr ergebener{\\[\baselineskip]}\spacefill\mbox{ArthurSchnitzler}\pend
           \leftskip=0em{}\pstart
           Berlin\oindex{Berlin@\textbf{Berlin}|pw}, 29. X. 96.\pend
                     \endnumbering\briefempfaengerindex{Altenberg, Peter@\textsc{Altenberg, Peter}!zzzSchnitzler, Arthur@\emph{von Arthur Schnitzler}!1896-10-291@{29. 10. 1896}|)be}\mylabel{h}\end{ledgroupsized}  \newcommand{\dateiname}{L00611}\newcommand{\titel}{Arthur Schnitzler an Peter Altenberg, 29. 10. 1896}\newcommand{\editorInnen}{Martin Anton Müller und Gerd-Hermann Susen}
            \footnotesize
\begin{ledgroupsized}[t]{11.5cm}
\doendnotes{C}
\end{ledgroupsized}
         %% latex-leseansicht-abspann.tex
%% Abspann für die Leseansicht.
%% Der Schalter \ifkorrekturansicht ist bereits durch den Vorspann gesetzt.

%% latex-abspann.tex
%% Gemeinsamer Abspann für Korrekturansicht und Leseansicht.
%% Setzt den Schalter \ifkorrekturansicht voraus (gesetzt in den
%% einbindenden Dateien latex-korrekturansicht-abspann.tex bzw.
%% latex-leseansicht-abspann.tex).
%% ---------------------------------------------------------------

\normalsize

% Das esempio-Environment wird nur in der Leseansicht benötigt
\ifkorrekturansicht\else
\newenvironment{esempio}[3]%
{
    \vspace{1.5ex}
    \rlap{\underline{#1}}
    \par
    \setlength{\parindent}{0cm}
    \nopagebreak
    \leftskip=#2cm
    \rightskip=#3cm
}
{
    \par
}
\fi

\doendnotes{C}
\bigskip
\vfill

\clearpage

\footnotesize

\ifkorrekturansicht
  \lohead{\textsc{register}}
\fi

% theindex-Environment neu definieren ohne reledmac
\makeatletter
\renewenvironment{theindex}{%
  \ifkorrekturansicht
    \section*{\indexname}%
  \else
    \subsubsection*{Index der erwähnten Entitäten}%
  \fi
  \setlength{\parindent}{0pt}%
  \setlength{\parskip}{0pt plus 0.3pt}%
  \let\item\@idxitem
}{%
  \ifkorrekturansicht\clearpage\fi
}
\makeatother

\IfFileExists{\jobname-pw.ind}{\input{\jobname-pw.ind}}{}

% Quellenangabe nur in der Leseansicht
\ifkorrekturansicht\else
% Fallback-Definitionen, falls die .tex-Datei \titel etc. nicht gesetzt hat
\providecommand{\titel}{}
\providecommand{\editorInnen}{}
\providecommand{\dateiname}{\jobname}

\vspace{3cm}

\vfill

\footnotesize
\textsc{Quelle}: \titel. Herausgegeben von {\editorInnen}. In: \emph{Arthur Schnitzler: Briefwechsel mit Autorinnen und Autoren}.
 Digitale Edition, https://schnitzler-briefe.acdh.oeaw.ac.at/{\dateiname}.html (Stand \today)
\fi

\end{document}


      