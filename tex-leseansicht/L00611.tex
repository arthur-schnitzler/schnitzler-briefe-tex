%% latex-leseansicht-vorspann.tex
%% Vorspann für die Leseansicht.
%% Lädt die gemeinsame Datei latex-vorspann.tex mit nicht gesetztem Schalter.

\newif\ifkorrekturansicht
\korrekturansichtfalse

\input{../tex-inputs/latex-vorspann}


\section[Arthur Schnitzler an Peter Altenberg, 29. 10. 1896]{L00611 Arthur Schnitzler an Peter Altenberg, 29. 10. 1896}
\nopagebreak\mylabel{L00611v}
\rehead{ }\normalsize\beginnumbering\briefempfaengerindex{Altenberg, Peter@\textsc{Altenberg, Peter}!zzzSchnitzler, Arthur@\emph{von Arthur Schnitzler}!1896-10-291@{29. 10. 1896}|(be}
\toendnotes[C]{\smallbreak\pagebreak[2]}
\correspDesc{Versand  durch Arthur Schnitzler am 29. 10. 1896 in Berlin
\newline{}Erhalt  durch Peter Altenberg am 30. 10. 1896 in Wien}\toendnotes[C]{\smallbreak}
\Standort{Wienbibliothek im Rathaus, H.I.N.-137077.}
\physDesc{Brief, Fotokopie, 1 Blatt, 2 Seiten, 465 Zeichen
\newline{}Handschrift: schwarze Tinte, lateinische Kurrent
\newline{}Altenberg: Ergänzung, nur zwei der vier Zeilen der Notiz sind ansatzweise
                                 zu entziffern: »\noindent{}\textsc{Lendway}{ / }\textsc{II. \textcolor{gray}{A}\textcolor{gray}{×}\-\textcolor{gray}{×}\-\textcolor{gray}{×}\-\textcolor{gray}{×}\-\textcolor{gray}{×}gaſſe 5}«. Karl Kraus\pwindex{Kraus, Karl 28.\,4.\,1874 Jičín – 12.\,6.\,1936 Wien@\textsc{Kraus, Karl} (28.\,4.\,1874 Jičín – 12.\,6.\,1936 Wien), \emph{Schriftsteller, Publizist, Schriftsteller}|pw}
                                 beschrieb diesen Text: »Der Wert des Autogramms ist
                                    allerdings beträchtlich erhöht durch eine Randnotiz Peter
                                    Altenbergs, der die ihm widerfahrene literarische Weihe mit den
                                    Adressen eines Nachtcafés und offenbar einer von dessen
                                    Besucherinnen quittiert hat«. \emph{Die Fackel}\pwindex{Fackel@\emph{Die Fackel}|pwk}, Jg. 24,
                                    Nr. 608–612, Ende Dezember 1922,
                                 S. 52. 
\newline{}Ordnung: Im Nachlass von Karl
                                    Kraus\pwindex{Kraus, Karl 28.\,4.\,1874 Jičín – 12.\,6.\,1936 Wien@\textsc{Kraus, Karl} (28.\,4.\,1874 Jičín – 12.\,6.\,1936 Wien), \emph{Schriftsteller, Publizist, Schriftsteller}|pw} überliefert. Kraus\pwindex{Kraus, Karl 28.\,4.\,1874 Jičín – 12.\,6.\,1936 Wien@\textsc{Kraus, Karl} (28.\,4.\,1874 Jičín – 12.\,6.\,1936 Wien), \emph{Schriftsteller, Publizist, Schriftsteller}|pw} ergänzte (vor der Kopie) am Objekt:
                                    »handſchriftliche Notiz von Peter Altenberg. Das Dokument
                                       1896 von ihm empfangen. Wien\oindex{Wien@\textbf{Wien}, \emph{Verwaltungsgebiet}|pw}, im November 1922\hspace*{1.5em}Karl Kraus\pwindex{Kraus, Karl 28.\,4.\,1874 Jičín – 12.\,6.\,1936 Wien@\textsc{Kraus, Karl} (28.\,4.\,1874 Jičín – 12.\,6.\,1936 Wien), \emph{Schriftsteller, Publizist, Schriftsteller}|pw}« 
\newline{}Zusatz: Kraus\pwindex{Kraus, Karl 28.\,4.\,1874 Jičín – 12.\,6.\,1936 Wien@\textsc{Kraus, Karl} (28.\,4.\,1874 Jičín – 12.\,6.\,1936 Wien), \emph{Schriftsteller, Publizist, Schriftsteller}|pw} ließ das Original
                                 versteigern. Schnitzler bot selber mit, wurde aber überboten.
                                 Vgl. \emph{Briefe 1913–1931}, S. 293–296 und
                                    Die Fackel\pwindex{Fackel@\emph{Die Fackel}|pw} von Ende 1922 bis Anfang 1923.
                               }
\buchAbdrucke{\weitereDrucke{1) \emph{Vorlesung Karl Kraus [Programm]}. (26. 11. 1922).} \weitereDrucke{2) \pwindex{Fackel@\emph{Die Fackel}|pwk}\emph{Die Fackel}, Jg. 24, Nr. 608–612, Ende Dezember 1922, S. 51.} \weitereDrucke{3) Reinhard Urbach: \emph{»Schwätzer sind Verbrecher«. Bemerkungen zu Schnitzlers Dramenfragment »Das Wort«.} In: \emph{Literatur und Kritik}, Jg. 3 (1968), S. 292–304, hier S. 293.} }\toendnotes[C]{\smallbreak}
\pstart{}{\pb}Lieber Herr Peter Altenberg,\pend\vspace{0.5em}
\pstart
           geſtern{ }ſprach ich mit \textsc{Gerhard Hauptmann}\pwindex{Hauptmann, Gerhart 15.\,11.\,1862 Szczawno-Zdrój – 6.\,6.\,1946 Jagniątków@\textsc{Hauptmann, Gerhart} (15.\,11.\,1862 Szczawno-Zdrój – 6.\,6.\,1946 Jagniątków), \emph{Schriftsteller}|pw}, der{ }ſich über Ihr Buch\pwindex{Altenberg, Peter 9.\,3.\,1859 Wien – 8.\,1.\,1919 ebd.@\textsc{Altenberg, Peter} (9.\,3.\,1859 Wien – 8.\,1.\,1919 ebd.), \emph{Schriftsteller}!Wie ich es sehe@\strich\emph{Wie ich es sehe}|pwv}
               in unendlich{ }ſympathiſcher Weiſe äußerte u. unter anderm sagte,{ }ſeit \uline{Jahren} habe kein Buch\pwindex{Altenberg, Peter 9.\,3.\,1859 Wien – 8.\,1.\,1919 ebd.@\textsc{Altenberg, Peter} (9.\,3.\,1859 Wien – 8.\,1.\,1919 ebd.), \emph{Schriftsteller}!Wie ich es sehe@\strich\emph{Wie ich es sehe}|pwv} einen{ }ſo{ }ſtarken Eindruck auf ihn gemacht als das
               Ihre.\pend
           
\pstart
           Da dieſe Bemerkung für Sie \label{K_L00611-1v}\edtext{intereſſant{ }ſein}{\lemma{\textnormal{\emph{interessant sein}}}\Cendnote{\textnormal{Für Altenberg\pwindex{Altenberg, Peter 9.\,3.\,1859 Wien – 8.\,1.\,1919 ebd.@\textsc{Altenberg, Peter} (9.\,3.\,1859 Wien – 8.\,1.\,1919 ebd.), \emph{Schriftsteller}|pwk} bot sie den Anlass, Hauptmann\pwindex{Hauptmann, Gerhart 15.\,11.\,1862 Szczawno-Zdrój – 6.\,6.\,1946 Jagniątków@\textsc{Hauptmann, Gerhart} (15.\,11.\,1862 Szczawno-Zdrój – 6.\,6.\,1946 Jagniątków), \emph{Schriftsteller}|pwk} direkt einen Brief zu schreiben. (\emph{Selbsterfindung eines Dichters}, S. 80.)}}}\label{K_L00611-1}
               dürfte und{ }ſie{ }ſonſt kaum an Sie {\pb}gelangen
               könnte, fühle ich mich in gewiſſem Sinne angenehm verpflichtet,{ }ſie Ihnen
               mitzutheilen.\pend
           
\pstart
           Mit beſtem Gruſs Ihr ergebener{\\[\baselineskip]}\spacefill\mbox{ArthurSchnitzler}\pend
           \leftskip=0em{}
\pstart
           Berlin\oindex{Berlin@\textbf{Berlin}, \emph{Hauptstadt}|pw}, 29. X. 96.\pend
           \selectlanguage{ngerman}\endnumbering\briefempfaengerindex{Altenberg, Peter@\textsc{Altenberg, Peter}!zzzSchnitzler, Arthur@\emph{von Arthur Schnitzler}!1896-10-291@{29. 10. 1896}|)be}\mylabel{L00611h}  \newcommand{\dateiname}{L00611}\newcommand{\titel}{Arthur Schnitzler an Peter Altenberg, 29. 10. 1896}\newcommand{\editorInnen}{Martin Anton Müller und Gerd-Hermann Susen}%% latex-leseansicht-abspann.tex
%% Abspann für die Leseansicht.
%% Der Schalter \ifkorrekturansicht ist bereits durch den Vorspann gesetzt.

%% latex-abspann.tex
%% Gemeinsamer Abspann für Korrekturansicht und Leseansicht.
%% Setzt den Schalter \ifkorrekturansicht voraus (gesetzt in den
%% einbindenden Dateien latex-korrekturansicht-abspann.tex bzw.
%% latex-leseansicht-abspann.tex).
%% ---------------------------------------------------------------

\normalsize

% Das esempio-Environment wird nur in der Leseansicht benötigt
\ifkorrekturansicht\else
\newenvironment{esempio}[3]%
{
    \vspace{1.5ex}
    \rlap{\underline{#1}}
    \par
    \setlength{\parindent}{0cm}
    \nopagebreak
    \leftskip=#2cm
    \rightskip=#3cm
}
{
    \par
}
\fi

\doendnotes{C}
\bigskip
\vfill

\clearpage

\footnotesize

\ifkorrekturansicht
  \lohead{\textsc{register}}
\fi

% theindex-Environment neu definieren ohne reledmac
\makeatletter
\renewenvironment{theindex}{%
  \ifkorrekturansicht
    \section*{\indexname}%
  \else
    \subsubsection*{Index der erwähnten Entitäten}%
  \fi
  \setlength{\parindent}{0pt}%
  \setlength{\parskip}{0pt plus 0.3pt}%
  \let\item\@idxitem
}{%
  \ifkorrekturansicht\clearpage\fi
}
\makeatother

\IfFileExists{\jobname-pw.ind}{\input{\jobname-pw.ind}}{}

% Quellenangabe nur in der Leseansicht
\ifkorrekturansicht\else
% Fallback-Definitionen, falls die .tex-Datei \titel etc. nicht gesetzt hat
\providecommand{\titel}{}
\providecommand{\editorInnen}{}
\providecommand{\dateiname}{\jobname}

\vspace{3cm}

\vfill

\footnotesize
\textsc{Quelle}: \titel. Herausgegeben von {\editorInnen}. In: \emph{Arthur Schnitzler: Briefwechsel mit Autorinnen und Autoren}.
 Digitale Edition, https://schnitzler-briefe.acdh.oeaw.ac.at/{\dateiname}.html (Stand \today)
\fi

\end{document}


