%% latex-leseansicht-vorspann.tex
%% Vorspann für die Leseansicht.
%% Lädt die gemeinsame Datei latex-vorspann.tex mit nicht gesetztem Schalter.

\newif\ifkorrekturansicht
\korrekturansichtfalse

\input{../tex-inputs/latex-vorspann}


         
         \newcommand{\erwaehntePersonen}{Personen: Leopold von Andrian-Werburg, Wolfgang Amadeus Mozart}
         \newcommand{\erwaehnteInstitutionen}{Institutionen: S. Fischer Verlag}
         \newcommand{\erwaehnteOrte}{Orte: Bad Fusch, Bad Ischl, Berchtesgaden, Hirschbichl, Lofer, Ramsau bei Berchtesgaden, Saalfelden am Steinernen Meer, Salzburg, Zell am See}
         \newcommand{\erwaehnteWerke}{Werke: Geschichte eines österreichischen Officiers, W. A. Mozart}
               \section[Hugo von Hofmannsthal an Arthur Schnitzler, 16. {[}7. 1897{]}]{ Hugo von Hofmannsthal an Arthur Schnitzler,
                    16. {[}7. 1897{]}}\nopagebreak\mylabel{v}\rehead{ }\begin{ledgroupsized}[t]{13cm}\normalsize\beginnumbering \toendnotes[C]{\smallbreak\pagebreak[2]} \Standort{CUL, Schnitzler, B 43.}
\physDesc{Brief, 1 Blatt, 4 Seiten
\newline{}Handschrift: schwarze Tinte, deutsche Kurrent
\newline{}Schnitzler: mit Bleistift Monat und Jahreszahl ergänzt: »7 97« \newline{}Ordnung: 1) mit Bleistift von unbekannter Hand nummeriert:
                                    »97«  2) mit Bleistift von unbekannter Hand nummeriert:
                                    »94«}\buchAbdrucke{\weitereDrucke{Hugo von Hofmannsthal, Arthur Schnitzler: \emph{Briefwechsel}. Hg. Therese Nickl und Heinrich Schnitzler. Frankfurt am Main: \emph{S. Fischer} 1964, S. 92.} }\toendnotes[C]{\smallbreak}\pstart
           \raggedleft{}{\pb}Fuſch\oindex{Bad Fusch@\textbf{Bad Fusch}|pw}{ }16\textsuperscript{ten}.\pend
           \pstart{}mein lieber Arthur\pend\pstart
           ich danke herzlich für Brief und Vorſchlag. Auch den Mozart\pwindex{Mozart, Wolfgang Amadeus 27.01.1756 – 05.12.1791@\textsc{Mozart, Wolfgang Amadeus} (27.01.1756 – 05.12.1791), \emph{Komponist}|pw}band\pwindex{\textcolor{red}{\textsuperscript{XXXX1 indx}}!W. A. Mozart1856 – 1859@\strich\emph{W. A. Mozart} {[}1856 – 1859{]}|pwv} hab ich bekommen. Es
                    thut mir ſehr ſehr leid, daſs es mit Salzburg\oindex{Salzburg@\textbf{Salzburg}|pw}
                    nicht zuſammengeht und wenn es ein geringerer Grund wäre als der völlig
                    zuſammengebrochene Zuſtand Poldys\pwindex{Andrian-Werburg, Leopold von 09.05.1875 – 19.11.1951@\textsc{Andrian-Werburg, Leopold von} (09.05.1875 – 19.11.1951), \emph{Schriftsteller, Diplomat}|pw} der mich
                    ſehr nötig braucht und den ich in dieſen nächſten 14 Tagen nicht mehr Stunden
                    allein laſſen will, {\pb}als
                    täglich meine Arbeit nöthig macht, ſo würde ich noch jetzt trachten, es möglich
                    zu machen. Auch hab ich eine kleine Arbeit\pwindex{Hofmannsthal, Hugo von 1874-02-01 – 1929-07-15@\textsc{Hofmannsthal, Hugo von} (1874-02-01 – 1929-07-15), \emph{Schriftsteller}!Geschichte eines oesterreichischen Officiers1978@\strich\emph{Geschichte eines österreichischen Officiers} {[}1978{]}|pwv} in Verſen angefangen, deren \label{K_L00703_1v}\edtext{Hintergrund}{\lemma{\textnormal{\emph{Hintergrund}}}\Cendnote{\textnormal{In seinen Aufzeichnungen (Hugo von Hofmannsthal\pwindex{Hofmannsthal, Hugo von 1874-02-01 – 1929-07-15@\textsc{Hofmannsthal, Hugo von} (1874-02-01 – 1929-07-15), \emph{Schriftsteller}|pwk}: \emph{Aufzeichnungen}. Hg. Rudolf Hirsch † und Ellen
                            Ritter † in Zusammenarbeit mit Konrad Heumann und Peter Michael
                            Braunwarth. Frankfurt am Main: \emph{S. Fischer}\orgindex{S. Fischer Verlag@S. Fischer Verlag|pwk}{ }2013, S. 381 (\emph{Sämtliche Werke},
                            XXXIX)) erwähnt Hofmannsthal\pwindex{Hofmannsthal, Hugo von 1874-02-01 – 1929-07-15@\textsc{Hofmannsthal, Hugo von} (1874-02-01 – 1929-07-15), \emph{Schriftsteller}|pwk}
                        eine Stiftsdame aus Salzburg\oindex{Salzburg@\textbf{Salzburg}|pwk} für die Arbeit
                        an der zu Lebzeiten unveröffentlicht gebliebenen \emph{Geschichte eines österreichischen Officiers}\pwindex{Hofmannsthal, Hugo von 1874-02-01 – 1929-07-15@\textsc{Hofmannsthal, Hugo von} (1874-02-01 – 1929-07-15), \emph{Schriftsteller}!Geschichte eines oesterreichischen Officiers1978@\strich\emph{Geschichte eines österreichischen Officiers} {[}1978{]}|pwk}.}}}\label{K_L00703_1h}
                    etwas mit Salzburg\oindex{Salzburg@\textbf{Salzburg}|pw} zu thun hat und habe mich in
                    übertriebener Weiſe darauf gefreut, es Euch dort, wo wir immer ſo glücklich
                    zuſammen waren, vorzuleſen. Dieſe kleine Arbeit\pwindex{Hofmannsthal, Hugo von 1874-02-01 – 1929-07-15@\textsc{Hofmannsthal, Hugo von} (1874-02-01 – 1929-07-15), \emph{Schriftsteller}!Geschichte eines oesterreichischen Officiers1978@\strich\emph{Geschichte eines österreichischen Officiers} {[}1978{]}|pwv} wird freilich jetzt {\pb}durch das finſtere
                    regneriſche Wetter etwas verzögert und wäre wohl erſt Ende Juli
                    fertig geworden.\pend
           \pstart
           Auf Euren Vorſchlag möchte ich am liebſten folgendes antworten: wenn das Wetter
                    gut wird und Ihr nur etwas Luſt habt die ſchöne Radtour zu machen (\uline{Salzburg}\oindex{Salzburg@\textbf{Salzburg}|pw} – Berchtesgaden\oindex{Berchtesgaden@\textbf{Berchtesgaden}|pw} – Ramſau\oindex{Ramsau bei Berchtesgaden@\textbf{Ramsau bei Berchtesgaden}|pw} – Hirſchbichel\oindex{Hirschbichl@\textbf{Hirschbichl}|pw} –
                        Saalfelden\oindex{Saalfelden am Steinernen Meer@\textbf{Saalfelden am Steinernen Meer}|pw} – \uline{Zell a See}\oindex{Zell am See@\textbf{Zell am See}|pw}; wozu Lofer\oindex{Lofer@\textbf{Lofer}|pw}?) ſo macht ſie und
                    verſtändigt {\pb}mich unmittelbar
                    vorher \introOben{}recht genau\introOben{}, damit ich rechtzeitig
                    hinunterkommen eventuell ein Stück (Saalfelden\oindex{Saalfelden am Steinernen Meer@\textbf{Saalfelden am Steinernen Meer}|pw}!) entgegenfahren kann. Geht es dann wegen Poldy\pwindex{Andrian-Werburg, Leopold von 09.05.1875 – 19.11.1951@\textsc{Andrian-Werburg, Leopold von} (09.05.1875 – 19.11.1951), \emph{Schriftsteller, Diplomat}|pw} oder anderm nicht, ſo habt Ihr doch nichts
                    ſchlechtes gemacht.\pend
           \pstart
           Herzlich Ihr{\\[\baselineskip]}\spacefill\mbox{Hugo.}\pend
           \leftskip=0em{}
         
         \endnumbering\mylabel{h}\end{ledgroupsized}  \newcommand{\dateiname}{L00703}\newcommand{\titel}{Hugo von Hofmannsthal an Arthur Schnitzler, 16. [7. 1897]}\newcommand{\editorInnen}{Martin Anton Müller und Gerd-Hermann Susen}%% latex-leseansicht-abspann.tex
%% Abspann für die Leseansicht.
%% Der Schalter \ifkorrekturansicht ist bereits durch den Vorspann gesetzt.

%% latex-abspann.tex
%% Gemeinsamer Abspann für Korrekturansicht und Leseansicht.
%% Setzt den Schalter \ifkorrekturansicht voraus (gesetzt in den
%% einbindenden Dateien latex-korrekturansicht-abspann.tex bzw.
%% latex-leseansicht-abspann.tex).
%% ---------------------------------------------------------------

\normalsize

% Das esempio-Environment wird nur in der Leseansicht benötigt
\ifkorrekturansicht\else
\newenvironment{esempio}[3]%
{
    \vspace{1.5ex}
    \rlap{\underline{#1}}
    \par
    \setlength{\parindent}{0cm}
    \nopagebreak
    \leftskip=#2cm
    \rightskip=#3cm
}
{
    \par
}
\fi

\doendnotes{C}
\bigskip
\vfill

\clearpage

\footnotesize

\ifkorrekturansicht
  \lohead{\textsc{register}}
\fi

% theindex-Environment neu definieren ohne reledmac
\makeatletter
\renewenvironment{theindex}{%
  \ifkorrekturansicht
    \section*{\indexname}%
  \else
    \subsubsection*{Index der erwähnten Entitäten}%
  \fi
  \setlength{\parindent}{0pt}%
  \setlength{\parskip}{0pt plus 0.3pt}%
  \let\item\@idxitem
}{%
  \ifkorrekturansicht\clearpage\fi
}
\makeatother

\IfFileExists{\jobname-pw.ind}{\input{\jobname-pw.ind}}{}

% Quellenangabe nur in der Leseansicht
\ifkorrekturansicht\else
% Fallback-Definitionen, falls die .tex-Datei \titel etc. nicht gesetzt hat
\providecommand{\titel}{}
\providecommand{\editorInnen}{}
\providecommand{\dateiname}{\jobname}

\vspace{3cm}

\vfill

\footnotesize
\textsc{Quelle}: \titel. Herausgegeben von {\editorInnen}. In: \emph{Arthur Schnitzler: Briefwechsel mit Autorinnen und Autoren}.
 Digitale Edition, https://schnitzler-briefe.acdh.oeaw.ac.at/{\dateiname}.html (Stand \today)
\fi

\end{document}


      