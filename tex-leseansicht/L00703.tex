%% latex-korrekturansicht-vorspann.tex
%% Vorspann für die Korrekturansicht.
%% Lädt die gemeinsame Datei latex-vorspann.tex mit gesetztem Schalter.

\newif\ifkorrekturansicht
\korrekturansichttrue

\input{../tex-inputs/latex-vorspann}


\section[Hugo von Hofmannsthal an Arthur Schnitzler, 16. {[}7. 1897{]}]{L00703 Hugo von Hofmannsthal an Arthur Schnitzler, 16. {[}7. 1897{]}}
\nopagebreak\mylabel{L00703v}
\rehead{ }\normalsize\beginnumbering\briefempfaengerindex{Schnitzler, Arthur@\textsc{Schnitzler, Arthur}!zzzHofmannsthal, Hugo von@\emph{von Hugo von Hofmannsthal}!1897-07-161@{16. {[}7. 1897{]}}|(be}
\toendnotes[C]{\smallbreak\pagebreak[2]}\Standort{CUL, Schnitzler, B 43.}
\physDesc{Brief, 1 Blatt, 4 Seiten, 1297 Zeichen
\newline{}Handschrift: schwarze Tinte, deutsche Kurrent
\newline{}Schnitzler: mit Bleistift Monat und Jahreszahl ergänzt: »7 97« 
\newline{}Ordnung: 1) mit Bleistift von unbekannter Hand nummeriert:
                                    »97«  2) mit Bleistift von unbekannter Hand nummeriert:
                                    »94«}
\buchAbdrucke{\weitereDrucke{Hugo von Hofmannsthal, Arthur Schnitzler: \emph{Briefwechsel}. Frankfurt am Main: \emph{S. Fischer} 1964, S. 92.} }\toendnotes[C]{\smallbreak}
\pstart
           \raggedleft{}{\pb}Fuſch\oindex{Bad Fusch@\textbf{Bad Fusch}, \emph{A.ADM3}|pw}{ }16\textsuperscript{ten}.\pend
           
\pstart{}mein lieber Arthur\pend\vspace{0.5em}
\pstart
           ich danke herzlich für Brief und Vorſchlag. Auch den Mozart\pwindex{Mozart, Wolfgang Amadeus 27.01.1756 – 05.12.1791@\textsc{Mozart, Wolfgang Amadeus} (27.01.1756 – 05.12.1791), \emph{Komponist/Komponistin}|pw}band\pwindex{W. A. Mozart@\emph{W. A. Mozart}|pwv} hab ich bekommen. Es thut
               mir ſehr ſehr leid, daſs es mit Salzburg\oindex{Salzburg@\textbf{Salzburg}, \emph{A.ADM2}|pw} nicht
               zuſammengeht und wenn es ein geringerer Grund wäre als der völlig zuſammengebrochene
               Zuſtand Poldys\pwindex{Andrian-Werburg, Leopold von 09.05.1875 – 19.11.1951@\textsc{Andrian-Werburg, Leopold von} (09.05.1875 – 19.11.1951), \emph{Schriftsteller/Schriftstellerin, Diplomat/Diplomatin}|pw} der mich ſehr nötig braucht und
               den ich in dieſen nächſten 14 Tagen nicht mehr Stunden allein laſſen will, {\pb}als täglich meine Arbeit nöthig
               macht, ſo würde ich noch jetzt trachten, es möglich zu machen. Auch hab ich eine
               kleine Arbeit\pwindex{Geschichte eines oesterreichischen Officiers@\emph{Geschichte eines österreichischen Officiers}|pwv} in Verſen
               angefangen, deren \label{K_L00703-1v}\edtext{Hintergrund}{\lemma{\textnormal{\emph{Hintergrund}}}\Cendnote{\textnormal{In seinen Aufzeichnungen (Hugo von Hofmannsthal\pwindex{Hofmannsthal, Hugo von 1874-02-01 – 1929-07-15@\textsc{Hofmannsthal, Hugo von} (1874-02-01 – 1929-07-15), \emph{Schriftsteller/Schriftstellerin}|pwk}: \emph{Aufzeichnungen}. Herausgegeben von  Rudolf Hirsch † und Ellen Ritter † in
                     Zusammenarbeit mit Konrad Heumann und Peter Michael Braunwarth. Frankfurt am
                     Main: \emph{S. Fischer}\orgindex{S. Fischer Verlag@S. Fischer Verlag|pwk}{ }2013, S. 381 (\emph{Sämtliche Werke},
                     XXXIX)) erwähnt Hofmannsthal\pwindex{Hofmannsthal, Hugo von 1874-02-01 – 1929-07-15@\textsc{Hofmannsthal, Hugo von} (1874-02-01 – 1929-07-15), \emph{Schriftsteller/Schriftstellerin}|pwk} eine
                  Stiftsdame aus Salzburg\oindex{Salzburg@\textbf{Salzburg}, \emph{A.ADM2}|pwk} für die Arbeit an der
                  zu Lebzeiten unveröffentlicht gebliebenen \emph{Geschichte eines österreichischen Officiers}\pwindex{Geschichte eines oesterreichischen Officiers@\emph{Geschichte eines österreichischen Officiers}|pwk}.}}}\label{K_L00703-1} etwas mit Salzburg\oindex{Salzburg@\textbf{Salzburg}, \emph{A.ADM2}|pw} zu thun hat und habe mich in übertriebener
               Weiſe darauf gefreut, es Euch dort, wo wir immer ſo glücklich zuſammen waren,
               vorzuleſen. Dieſe kleine Arbeit\pwindex{Geschichte eines oesterreichischen Officiers@\emph{Geschichte eines österreichischen Officiers}|pwv} wird freilich jetzt {\pb}durch das finſtere regneriſche
               Wetter etwas verzögert und wäre wohl erſt Ende Juli fertig geworden.\pend
           
\pstart
           Auf Euren Vorſchlag möchte ich am liebſten folgendes antworten: wenn das Wetter gut
               wird und Ihr nur etwas Luſt habt die ſchöne Radtour zu machen (\uline{Salzburg}\oindex{Salzburg@\textbf{Salzburg}, \emph{A.ADM2}|pw} – Berchtesgaden\oindex{Berchtesgaden@\textbf{Berchtesgaden}, \emph{P.PPL}|pw} – Ramſau\oindex{Ramsau bei Berchtesgaden@\textbf{Ramsau bei Berchtesgaden}, \emph{P.PPL}|pw} – Hirſchbichel\oindex{Hirschbichl@\textbf{Hirschbichl}, \emph{P.PPL}|pw} –
                  Saalfelden\oindex{Saalfelden am Steinernen Meer@\textbf{Saalfelden am Steinernen Meer}, \emph{A.ADM3}|pw} – \uline{Zell a See}\oindex{Zell am See@\textbf{Zell am See}, \emph{P.PPLA3}|pw}; wozu Lofer\oindex{Lofer@\textbf{Lofer}, \emph{P.PPLA3}|pw}?) ſo macht ſie und
               verſtändigt {\pb}mich unmittelbar
               vorher \introOben{}recht genau\introOben{}, damit ich rechtzeitig hinunterkommen
               eventuell ein Stück (Saalfelden\oindex{Saalfelden am Steinernen Meer@\textbf{Saalfelden am Steinernen Meer}, \emph{A.ADM3}|pw}!)
               entgegenfahren kann. Geht es dann wegen Poldy\pwindex{Andrian-Werburg, Leopold von 09.05.1875 – 19.11.1951@\textsc{Andrian-Werburg, Leopold von} (09.05.1875 – 19.11.1951), \emph{Schriftsteller/Schriftstellerin, Diplomat/Diplomatin}|pw}
               oder anderm nicht, ſo habt Ihr doch nichts ſchlechtes gemacht.\pend
           
\pstart
           Herzlich Ihr{\\[\baselineskip]}\spacefill\mbox{Hugo.}\pend
           \leftskip=0em{}\selectlanguage{ngerman}\endnumbering\briefempfaengerindex{Schnitzler, Arthur@\textsc{Schnitzler, Arthur}!zzzHofmannsthal, Hugo von@\emph{von Hugo von Hofmannsthal}!1897-07-161@{16. {[}7. 1897{]}}|)be}\mylabel{L00703h}  \normalsize

\doendnotes{C}
\bigskip
\vfill

\clearpage

\footnotesize

\lohead{\textsc{register}}

% Definiere theindex-Environment komplett neu ohne reledmac
\makeatletter
\renewenvironment{theindex}{%
  \section*{\indexname}%
  \setlength{\parindent}{0pt}%
  \setlength{\parskip}{0pt plus 0.3pt}%
  \let\item\@idxitem
}{%
  \clearpage
}
\makeatother

\IfFileExists{\jobname-pw.ind}{\input{\jobname-pw.ind}}{}

\end{document}

      