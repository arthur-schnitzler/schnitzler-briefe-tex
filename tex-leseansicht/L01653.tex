%% latex-leseansicht-vorspann.tex
%% Vorspann für die Leseansicht.
%% Lädt die gemeinsame Datei latex-vorspann.tex mit nicht gesetztem Schalter.

\newif\ifkorrekturansicht
\korrekturansichtfalse

\input{../tex-inputs/latex-vorspann}


\section[Hugo von Hofmannsthal an Arthur Schnitzler, 19. 1. 1907]{L01653 Hugo von Hofmannsthal an Arthur Schnitzler, 19. 1. 1907}
\nopagebreak\mylabel{L01653v}
\rehead{ }\normalsize\beginnumbering\briefempfaengerindex{Schnitzler, Arthur@\textsc{Schnitzler, Arthur}!zzzHofmannsthal, Hugo von@\emph{von Hugo von Hofmannsthal}!1907-01-191@{19. 1. 1907}|(be}
\toendnotes[C]{\smallbreak\pagebreak[2]}
\correspDesc{Versand  durch Hugo von Hofmannsthal am 19. 1. 1907 in Wien
\newline{}Erhalt  durch Arthur Schnitzler am 19. 1. 1907 in Wien}\toendnotes[C]{\smallbreak}
\Standort{CUL, Schnitzler, B 43.}
\physDesc{Postkarte, 357 Zeichen
\newline{}Handschrift: schwarze Tinte, lateinische Kurrent
\newline{}Versand: 1) Rohrpost  2) Stempel: »\nobreak{}\oindex{I., Innere Stadt@\textbf{I., Innere Stadt}, \emph{Verwaltungsgebiet}|pwk}Wien 14/, 19 I 07, 5\textsuperscript{30}N\nobreak{}«.  3) Stempel: »\nobreak{}\oindex{XVIII., Währing@\textbf{XVIII., Währing}, \emph{Verwaltungsgebiet}|pwk}18/1 Wien 111, 19 I 07, 6\textsuperscript{50}\nobreak{}«. 
\newline{}Schnitzler: mit Bleistift datiert: »10/1« 
\newline{}Ordnung: 1) mit Bleistift von unbekannter Hand nummeriert: »\strikeout{273}«  2) mit Bleistift von unbekannter Hand nummeriert: »271«}
\buchAbdrucke{\weitereDrucke{Hugo von Hofmannsthal, Arthur Schnitzler: \emph{Briefwechsel}. Herausgegeben von Therese Nickl und Heinrich Schnitzler. Frankfurt am Main: \emph{S. Fischer} 1964, S. 226.} }\toendnotes[C]{\smallbreak}\pstart{}{\pb}Herrn D\textsuperscript{r} Arthur Schnitzler\pend{}\pstart{}Wien\oindex{Wien@\textbf{Wien}, \emph{Verwaltungsgebiet}|pw}\pend{}\pstart{}XVII Spöttelgasse 7\oindex{Wien@\textbf{Wien}!XVIII., Währing@\textbf{XVIII., Währing}!Edmund-Weiß-Gasse 7@\textbf{Edmund-Weiß-Gasse 7}, \emph{Wohngebäude}|pw}\pend{}\pstart{}nächſt Türkenschanzstrasse\oindex{Wien@\textbf{Wien}!XVIII., Währing@\textbf{XVIII., Währing}!Türkenschanzstraße@\textbf{Türkenschanzstraße}, \emph{Straße}|pw}\pend{}{\bigskip}\vspace{1em}
\pstart
           \noindent{}{\pb}Wir\pwindex{Hofmannsthal, Gertrude von 16.\,3.\,1880 Wien – 9.\,11.\,1959 Paddington@\textsc{Hofmannsthal, Gertrude von} (16.\,3.\,1880 Wien – 9.\,11.\,1959 Paddington)|pwv}\pwindex{Schlesinger, Hans Bernhard 20.\,7.\,1875 Wien – 13.\,3.\,1932 Salzburg@\textsc{Schlesinger, Hans Bernhard} (20.\,7.\,1875 Wien – 13.\,3.\,1932 Salzburg), \emph{Maler}|pwv} möchten \label{K_L01653-1v}\edtext{morgen}{\lemma{\textnormal{\emph{morgen}}}\Cendnote{\textnormal{Vgl. A. S.: \emph{Tagebuch}, 20. 1. 1907.
               }}}\label{K_L01653-1}{ }\introOben{}Sonntag\introOben{} für den späteren Nachmittag noch lieber für den Abend zu Euch.
               Hoffentlich passt es diesmal. Ohne Telephon ist es für uns so furchtbar schwer und
               man sieht sich ja nie!!\hspace*{1.5em}Die Antwort bitte
               pneumatisch oder telephonisch (229) an die \uline{Elisabethstrasse}\oindex{Wien@\textbf{Wien}!I., Innere Stadt@\textbf{I., Innere Stadt}!Elisabethstraße [Wien]@\textbf{Elisabethstraße [Wien]}, \emph{Straße}|pw}\pend
           \pstart \spacefill\mbox{Hugo.}\pend{}
\pstart
           \noindent{}\label{T_L01653-1v}\edtext{Samstag. 4\textsuperscript{h}.}{\lemma{\textnormal{\emph{Samstag. 4\textsuperscript{h}.}}}\Cendnote{\textnormal{nachträglich unten rechts
                     eingefügt}}}\label{T_L01653-1}\pend
           \selectlanguage{ngerman}\endnumbering\briefempfaengerindex{Schnitzler, Arthur@\textsc{Schnitzler, Arthur}!zzzHofmannsthal, Hugo von@\emph{von Hugo von Hofmannsthal}!1907-01-191@{19. 1. 1907}|)be}\mylabel{L01653h}  \newcommand{\dateiname}{L01653}\newcommand{\titel}{Hugo von Hofmannsthal an Arthur Schnitzler, 19. 1. 1907}\newcommand{\editorInnen}{Martin Anton Müller und Gerd-Hermann Susen}%% latex-leseansicht-abspann.tex
%% Abspann für die Leseansicht.
%% Der Schalter \ifkorrekturansicht ist bereits durch den Vorspann gesetzt.

%% latex-abspann.tex
%% Gemeinsamer Abspann für Korrekturansicht und Leseansicht.
%% Setzt den Schalter \ifkorrekturansicht voraus (gesetzt in den
%% einbindenden Dateien latex-korrekturansicht-abspann.tex bzw.
%% latex-leseansicht-abspann.tex).
%% ---------------------------------------------------------------

\normalsize

% Das esempio-Environment wird nur in der Leseansicht benötigt
\ifkorrekturansicht\else
\newenvironment{esempio}[3]%
{
    \vspace{1.5ex}
    \rlap{\underline{#1}}
    \par
    \setlength{\parindent}{0cm}
    \nopagebreak
    \leftskip=#2cm
    \rightskip=#3cm
}
{
    \par
}
\fi

\doendnotes{C}
\bigskip
\vfill

\clearpage

\footnotesize

\ifkorrekturansicht
  \lohead{\textsc{register}}
\fi

% theindex-Environment neu definieren ohne reledmac
\makeatletter
\renewenvironment{theindex}{%
  \ifkorrekturansicht
    \section*{\indexname}%
  \else
    \subsubsection*{Index der erwähnten Entitäten}%
  \fi
  \setlength{\parindent}{0pt}%
  \setlength{\parskip}{0pt plus 0.3pt}%
  \let\item\@idxitem
}{%
  \ifkorrekturansicht\clearpage\fi
}
\makeatother

\IfFileExists{\jobname-pw.ind}{\input{\jobname-pw.ind}}{}

% Quellenangabe nur in der Leseansicht
\ifkorrekturansicht\else
% Fallback-Definitionen, falls die .tex-Datei \titel etc. nicht gesetzt hat
\providecommand{\titel}{}
\providecommand{\editorInnen}{}
\providecommand{\dateiname}{\jobname}

\vspace{3cm}

\vfill

\footnotesize
\textsc{Quelle}: \titel. Herausgegeben von {\editorInnen}. In: \emph{Arthur Schnitzler: Briefwechsel mit Autorinnen und Autoren}.
 Digitale Edition, https://schnitzler-briefe.acdh.oeaw.ac.at/{\dateiname}.html (Stand \today)
\fi

\end{document}


