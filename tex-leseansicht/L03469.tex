%% latex-korrekturansicht-vorspann.tex
%% Vorspann für die Korrekturansicht.
%% Lädt die gemeinsame Datei latex-vorspann.tex mit gesetztem Schalter.

\newif\ifkorrekturansicht
\korrekturansichttrue

\input{../tex-inputs/latex-vorspann}


\section[ Paul Goldmann und Eva Marie Goldmann an Arthur Schnitzler, 1. 10. 1909]{L03469 Paul Goldmann und Eva Marie Goldmann an Arthur
               Schnitzler, 1. 10. 1909}
\nopagebreak\mylabel{L03469v}
\rehead{ }\normalsize\beginnumbering\briefempfaengerindex{Schnitzler, Arthur@\textsc{Schnitzler, Arthur}!zzzGoldmann, Eva Marie@\emph{von Eva Marie Goldmann}!1909-10-011@{1. 10. 1909}|(be}\briefempfaengerindex{Schnitzler, Arthur@\textsc{Schnitzler, Arthur}!zzzGoldmann, Paul@\emph{von Paul Goldmann}!1909-10-011@{1. 10. 1909}|(be}
\toendnotes[C]{\smallbreak\pagebreak[2]}\Standort{DLA, A:Schnitzler, HS.NZ85.1.3175.}
\physDesc{Kartenbrief, 587 Zeichen
\newline{}Handschrift Paul Goldmann: 1) schwarze Tinte, deutsche Kurrent\hspace{1em}2) schwarze Tinte, lateinische Kurrent (\noindent{}Adresse)\hspace{1em}
\newline{}Handschrift Eva Marie Goldmann: schwarze Tinte, lateinische Kurrent (\noindent{}Fußnote)
\newline{}Versand: Stempel: »\nobreak{}\textcolor{gray}{1/1}{ }\textcolor{gray}{Wi}{[}en{]}, 1. X. {[}09{]}, 2\nobreak{}«.  }\toendnotes[C]{\smallbreak}\pstart{}{\pb}Herrn\pend{}\pstart{}Dr. Arthur Schnitzler\pend{}\pstart{}Wien\oindex{Wien@\textbf{Wien}, \emph{A.ADM2}|pw}\pend{}\pstart{}XVIII. Spöttelgaſse 7\oindex{Edmund-Weiss-Gasse 7@\textbf{Edmund-Weiß-Gasse 7}, \emph{Wohngebäude (K.WHS)}|pw}.\pend{}{\bigskip}\vspace{1em}
\pstart
           1. 10. 09.\pend
           \vspace{0.5em}
\pstart
           Lieber Freund, Ich fahre heut{ }Mittag{ }\label{K_L03469-1v}\edtext{ab}{\lemma{\textnormal{\emph{ab}}}\Cendnote{\textnormal{aus Wien\oindex{Wien@\textbf{Wien}, \emph{A.ADM2}|pwk}, am 28. 9. 1909 hatte er
                     Schnitzler noch besucht}}}\label{K_L03469-1} u. will
               Dir nur raſch vorher mitteilen, daß meine Schwägerin\pwindex{Fraenkel, Margarethe 1880-10-25 – 1939-07-21@\textsc{Fränkel, Margarethe} (1880-10-25 – 1939-07-21)|pwv}, Frl. \textsc{Fränkel\pwindex{Fraenkel, Margarethe 1880-10-25 – 1939-07-21@\textsc{Fränkel, Margarethe} (1880-10-25 – 1939-07-21)|pw}}, die im \textsc{Hotel Sacher\oindex{Hotel Sacher@\textbf{Hotel Sacher}, \emph{Hotel (K.HTL)}|pw}} wohnt, gern bereit iſt, Dich in das Haus\oindex{Armbrustergasse@\textbf{Armbrustergasse}, \emph{R.ST}|pwv} des \label{K_L03469-2v}\edtext{\textsc{Dr. Tietze\pwindex{Tietze, Hans 01.03.1880 – 12.04.1954@\textsc{Tietze, Hans} (01.03.1880 – 12.04.1954), \emph{Ministerialbeamter/Ministerialbeamte, Kunsthistoriker/Kunsthistorikerin}|pw}}}{\lemma{\textnormal{\emph{Dr. Tietze}}}\Cendnote{\textnormal{Durch die Geburt des zweiten Kindes
                  Lili\pwindex{Cappellini, Lili 13.09.1909 – 26.07.1928@\textsc{Cappellini, Lili} (13.09.1909 – 26.07.1928)|pwk} am 13. 9. 1909
                  waren die Wohnverhältnisse der Familie Schnitzler\pwindex{Schnitzler, Olga 17.01.1882 – 13.01.1970@\textsc{Schnitzler, Olga} (17.01.1882 – 13.01.1970), \emph{Schauspieler/Schauspielerin, Sänger/Sängerin}|pwk} zu beengt.
                  Deswegen war die Familie auf Wohnungs- bzw. Haussuche, die am 16. 7. 1910 in die
                  Übersiedelung in die Sternwartestraße 71\oindex{Sternwartestrasse 71@\textbf{Sternwartestraße 71}, \emph{Wohngebäude (K.WHS)}|pwk}
                  mündete. Ob sie das Haus\oindex{Armbrustergasse@\textbf{Armbrustergasse}, \emph{R.ST}|pwkv}
                  besichtigten, in dem Hans Tietze\pwindex{Tietze, Hans 01.03.1880 – 12.04.1954@\textsc{Tietze, Hans} (01.03.1880 – 12.04.1954), \emph{Ministerialbeamter/Ministerialbeamte, Kunsthistoriker/Kunsthistorikerin}|pwk} mit seiner
                  Frau Erica Tietze-Conrat\pwindex{Tietze-Conrat, Erica 1883-06-20 – 1958-12-12@\textsc{Tietze-Conrat, Erica} (1883-06-20 – 1958-12-12), \emph{Kunsthistoriker/Kunsthistorikerin}|pwk} wohnte, ist nicht
                  geklärt.}}}\label{K_L03469-2}, der eine Couſine\pwindex{Tietze-Conrat, Erica 1883-06-20 – 1958-12-12@\textsc{Tietze-Conrat, Erica} (1883-06-20 – 1958-12-12), \emph{Kunsthistoriker/Kunsthistorikerin}|pwv} von ihr {\pb}geheiratet hat, zu führen.
               Du brauchſt ihr\pwindex{Fraenkel, Margarethe 1880-10-25 – 1939-07-21@\textsc{Fränkel, Margarethe} (1880-10-25 – 1939-07-21)|pwv} nur ins \textsc{Hotel Sacher\orgindex{Hotel Sacher@Hotel Sacher|pw}} zu telephoniren\noindent{}{[}hs. :{]} Lieber zu Sacher\orgindex{Hotel Sacher@Hotel Sacher|pw} ein paar Zeilen schreiben. Telephoniren ist fast nicht zu
                     machen. {\\}Viele Grüsse \spacefill\mbox{EvaG.}. Du ſollteſt Dir das Haus\oindex{Armbrustergasse@\textbf{Armbrustergasse}, \emph{R.ST}|pwv}, das tatſächlich mit den billigſten Mitteln erbaut iſt u. auf der Hohen Warte\oindex{Hohe Warte@\textbf{Hohe Warte}, \emph{Erhebung}|pw}, Armbruſterſtraße 20\oindex{Armbrustergasse@\textbf{Armbrustergasse}, \emph{R.ST}|pw}, ſteht, einmal anſehen, ehe Du daran gehſt, die
               Wohnungsfrage zu löſen.\pend
           
\pstart
           Herzliche Grüße Deiner Frau\pwindex{Schnitzler, Olga 17.01.1882 – 13.01.1970@\textsc{Schnitzler, Olga} (17.01.1882 – 13.01.1970), \emph{Schauspieler/Schauspielerin, Sänger/Sängerin}|pwv} u. Dir! Dein {\\[\baselineskip]}\spacefill\mbox{Paul Goldmann.}\pend
           \leftskip=0em{}\selectlanguage{ngerman}\endnumbering\briefempfaengerindex{Schnitzler, Arthur@\textsc{Schnitzler, Arthur}!zzzGoldmann, Eva Marie@\emph{von Eva Marie Goldmann}!1909-10-011@{1. 10. 1909}|)be}\briefempfaengerindex{Schnitzler, Arthur@\textsc{Schnitzler, Arthur}!zzzGoldmann, Paul@\emph{von Paul Goldmann}!1909-10-011@{1. 10. 1909}|)be}\mylabel{L03469h}  \normalsize

\doendnotes{C}
\bigskip
\vfill

\clearpage

\footnotesize

\lohead{\textsc{register}}

% Definiere theindex-Environment komplett neu ohne reledmac
\makeatletter
\renewenvironment{theindex}{%
  \section*{\indexname}%
  \setlength{\parindent}{0pt}%
  \setlength{\parskip}{0pt plus 0.3pt}%
  \let\item\@idxitem
}{%
  \clearpage
}
\makeatother

\IfFileExists{\jobname-pw.ind}{\input{\jobname-pw.ind}}{}

\end{document}

      