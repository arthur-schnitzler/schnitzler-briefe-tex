%% latex-leseansicht-vorspann.tex
%% Vorspann für die Leseansicht.
%% Lädt die gemeinsame Datei latex-vorspann.tex mit nicht gesetztem Schalter.

\newif\ifkorrekturansicht
\korrekturansichtfalse

\input{../tex-inputs/latex-vorspann}


\section[ Paul Goldmann und Eva Marie Goldmann an Arthur Schnitzler, 1. 10. 1909]{L03469 Paul Goldmann und Eva Marie Goldmann an Arthur
               Schnitzler,  1. 10. 1909}
\nopagebreak\mylabel{L03469v}
\rehead{ }\normalsize\beginnumbering\briefempfaengerindex{Schnitzler, Arthur@\textsc{Schnitzler, Arthur}!zzzGoldmann, Eva Marie@\emph{von Eva Marie Goldmann}!1909-10-012@{1. 10. 1909}|(be}\briefempfaengerindex{Schnitzler, Arthur@\textsc{Schnitzler, Arthur}!zzzGoldmann, Paul@\emph{von Paul Goldmann}!1909-10-012@{1. 10. 1909}|(be}
\toendnotes[C]{\smallbreak\pagebreak[2]}
\correspDesc{Versand  durch Paul Goldmann, Eva Marie Goldmann am 1. 10. 1909 in Wien
\newline{}Erhalt  durch Arthur Schnitzler im Zeitraum [1. 10. 1909
                  – 5. 10. 1909?] in Wien}\toendnotes[C]{\smallbreak}
\Standort{DLA, A:Schnitzler, HS.NZ85.1.3175.}
\physDesc{Kartenbrief, 587 Zeichen
\newline{}Handschrift Paul Goldmann: schwarze Tinte, deutsche Kurrent
\newline{}Handschrift Eva Marie Goldmann: schwarze Tinte, lateinische Kurrent (\noindent{}Fußnote)
\newline{}Versand: Stempel: »\nobreak{}\oindex{Wien@\textbf{Wien}, \emph{Verwaltungsgebiet}|pwk}\textcolor{gray}{1/1}{ }\textcolor{gray}{Wi}{[}en{]}, 1. X. {[}09{]}, 2\nobreak{}«.  }\toendnotes[C]{\smallbreak}\pstart{}\textsc{{\pb}Herrn}\pend{}\pstart{}\textsc{Dr. Arthur Schnitzler}\pend{}\pstart{}\textsc{Wien\oindex{Wien@\textbf{Wien}, \emph{Verwaltungsgebiet}|pw}}\pend{}\pstart{}\textsc{XVIII. Spöttelgaſse 7\oindex{Wien@\textbf{Wien}!XVIII., Währing@\textbf{XVIII., Währing}!Edmund-Weiß-Gasse 7@\textbf{Edmund-Weiß-Gasse 7}, \emph{Wohngebäude}|pw}.}\pend{}{\bigskip}\vspace{1em}
\pstart
           1. 10. 09.\pend
           \vspace{0.5em}
\pstart
           Lieber Freund, Ich fahre heut{ }Mittag{ }\label{K_L03469-1v}\edtext{ab}{\lemma{\textnormal{\emph{ab}}}\Cendnote{\textnormal{aus Wien\oindex{Wien@\textbf{Wien}, \emph{Verwaltungsgebiet}|pwk}, am 28. 9. 1909 hatte er
                     Schnitzler noch besucht}}}\label{K_L03469-1} u. will
               Dir nur raſch vorher mitteilen, daß meine Schwägerin\pwindex{Fränkel, Margarethe 25.\,10.\,1880 Wien – 21.\,7.\,1939@\textsc{Fränkel, Margarethe} (25.\,10.\,1880 Wien – 21.\,7.\,1939)|pwv}, Frl. \textsc{Fränkel\pwindex{Fränkel, Margarethe 25.\,10.\,1880 Wien – 21.\,7.\,1939@\textsc{Fränkel, Margarethe} (25.\,10.\,1880 Wien – 21.\,7.\,1939)|pw}}, die im \textsc{Hotel Sacher\oindex{Wien@\textbf{Wien}!I., Innere Stadt@\textbf{I., Innere Stadt}!Hotel Sacher@\textbf{Hotel Sacher}, \emph{Hotel}|pw}} wohnt, gern bereit iſt, Dich in das Haus\oindex{Wien@\textbf{Wien}!XIX., Döbling@\textbf{XIX., Döbling}!Armbrustergasse@\textbf{Armbrustergasse}, \emph{Straße}|pwv} des \label{K_L03469-2v}\edtext{\textsc{Dr. Tietze\pwindex{Tietze, Hans 1.\,3.\,1880 Prag – 12.\,4.\,1954 New York City@\textsc{Tietze, Hans} (1.\,3.\,1880 Prag – 12.\,4.\,1954 New York City), \emph{Ministerialbeamter, Kunsthistoriker}|pw}}}{\lemma{\textnormal{\emph{Dr. Tietze}}}\Cendnote{\textnormal{Durch die Geburt des zweiten Kindes
                  Lili\pwindex{Cappellini, Lili 13.\,9.\,1909 Wien – 26.\,7.\,1928 Venedig@\textsc{Cappellini, Lili} (13.\,9.\,1909 Wien – 26.\,7.\,1928 Venedig)|pwk} am 13. 9. 1909
                  waren die Wohnverhältnisse der Familie Schnitzler\pwindex{Schnitzler, Olga 17.\,1.\,1882 Wien – 13.\,1.\,1970 Lugano@\textsc{Schnitzler, Olga} (17.\,1.\,1882 Wien – 13.\,1.\,1970 Lugano), \emph{Schauspielerin, Sängerin}|pwk} zu beengt.
                  Deswegen war die Familie auf Wohnungs- bzw. Haussuche, die am 16. 7. 1910 in die
                  Übersiedelung in die Sternwartestraße 71\oindex{Wien@\textbf{Wien}!XVIII., Währing@\textbf{XVIII., Währing}!Sternwartestraße 71@\textbf{Sternwartestraße 71}, \emph{Wohngebäude}|pwk}
                  mündete. Ob sie das Haus\oindex{Wien@\textbf{Wien}!XIX., Döbling@\textbf{XIX., Döbling}!Armbrustergasse@\textbf{Armbrustergasse}, \emph{Straße}|pwkv}
                  besichtigten, in dem Hans Tietze\pwindex{Tietze, Hans 1.\,3.\,1880 Prag – 12.\,4.\,1954 New York City@\textsc{Tietze, Hans} (1.\,3.\,1880 Prag – 12.\,4.\,1954 New York City), \emph{Ministerialbeamter, Kunsthistoriker}|pwk} mit seiner
                  Frau Erica Tietze-Conrat\pwindex{Tietze-Conrat, Erica 20.\,6.\,1883 Wien – 12.\,12.\,1958 New York City@\textsc{Tietze-Conrat, Erica} (20.\,6.\,1883 Wien – 12.\,12.\,1958 New York City), \emph{Kunsthistorikerin}|pwk} wohnte, ist nicht
                  geklärt.}}}\label{K_L03469-2}, der eine Couſine\pwindex{Tietze-Conrat, Erica 20.\,6.\,1883 Wien – 12.\,12.\,1958 New York City@\textsc{Tietze-Conrat, Erica} (20.\,6.\,1883 Wien – 12.\,12.\,1958 New York City), \emph{Kunsthistorikerin}|pwv} von ihr {\pb}geheiratet hat, zu führen.
               Du brauchſt ihr\pwindex{Fränkel, Margarethe 25.\,10.\,1880 Wien – 21.\,7.\,1939@\textsc{Fränkel, Margarethe} (25.\,10.\,1880 Wien – 21.\,7.\,1939)|pwv} nur ins \textsc{Hotel Sacher\orgindex{Hotel Sacher@Hotel Sacher|pw}} zu telephoniren\footnote{\noindent{}{[}hs. Goldmann:{]} Lieber zu Sacher\orgindex{Hotel Sacher@Hotel Sacher|pw} ein paar Zeilen schreiben. Telephoniren ist fast nicht zu
                     machen. {\\}Viele Grüsse \spacefill\mbox{EvaG.}}. Du{ }ſollteſt Dir das Haus\oindex{Wien@\textbf{Wien}!XIX., Döbling@\textbf{XIX., Döbling}!Armbrustergasse@\textbf{Armbrustergasse}, \emph{Straße}|pwv}, das tatſächlich mit den billigſten Mitteln erbaut iſt u. auf der Hohen Warte\oindex{Wien@\textbf{Wien}!XIX., Döbling@\textbf{XIX., Döbling}!Hohe Warte@\textbf{Hohe Warte}, \emph{Erhebung}|pw}, Armbruſterſtraße 20\oindex{Wien@\textbf{Wien}!XIX., Döbling@\textbf{XIX., Döbling}!Armbrustergasse@\textbf{Armbrustergasse}, \emph{Straße}|pw},{ }ſteht, einmal anſehen, ehe Du daran gehſt, die
               Wohnungsfrage zu löſen.\pend
           
\pstart
           Herzliche Grüße Deiner Frau\pwindex{Schnitzler, Olga 17.\,1.\,1882 Wien – 13.\,1.\,1970 Lugano@\textsc{Schnitzler, Olga} (17.\,1.\,1882 Wien – 13.\,1.\,1970 Lugano), \emph{Schauspielerin, Sängerin}|pwv} u. Dir! Dein {\\[\baselineskip]}\spacefill\mbox{Paul Goldmann.}\pend
           \leftskip=0em{}\selectlanguage{ngerman}\endnumbering\briefempfaengerindex{Schnitzler, Arthur@\textsc{Schnitzler, Arthur}!zzzGoldmann, Eva Marie@\emph{von Eva Marie Goldmann}!1909-10-012@{1. 10. 1909}|)be}\briefempfaengerindex{Schnitzler, Arthur@\textsc{Schnitzler, Arthur}!zzzGoldmann, Paul@\emph{von Paul Goldmann}!1909-10-012@{1. 10. 1909}|)be}\mylabel{L03469h}  \newcommand{\dateiname}{L03469}\newcommand{\titel}{Paul Goldmann und Eva Marie Goldmann an Arthur Schnitzler, 1. 10. 1909}\newcommand{\editorInnen}{Martin Anton Müller und Laura Untner}%% latex-leseansicht-abspann.tex
%% Abspann für die Leseansicht.
%% Der Schalter \ifkorrekturansicht ist bereits durch den Vorspann gesetzt.

%% latex-abspann.tex
%% Gemeinsamer Abspann für Korrekturansicht und Leseansicht.
%% Setzt den Schalter \ifkorrekturansicht voraus (gesetzt in den
%% einbindenden Dateien latex-korrekturansicht-abspann.tex bzw.
%% latex-leseansicht-abspann.tex).
%% ---------------------------------------------------------------

\normalsize

% Das esempio-Environment wird nur in der Leseansicht benötigt
\ifkorrekturansicht\else
\newenvironment{esempio}[3]%
{
    \vspace{1.5ex}
    \rlap{\underline{#1}}
    \par
    \setlength{\parindent}{0cm}
    \nopagebreak
    \leftskip=#2cm
    \rightskip=#3cm
}
{
    \par
}
\fi

\doendnotes{C}
\bigskip
\vfill

\clearpage

\footnotesize

\ifkorrekturansicht
  \lohead{\textsc{register}}
\fi

% theindex-Environment neu definieren ohne reledmac
\makeatletter
\renewenvironment{theindex}{%
  \ifkorrekturansicht
    \section*{\indexname}%
  \else
    \subsubsection*{Index der erwähnten Entitäten}%
  \fi
  \setlength{\parindent}{0pt}%
  \setlength{\parskip}{0pt plus 0.3pt}%
  \let\item\@idxitem
}{%
  \ifkorrekturansicht\clearpage\fi
}
\makeatother

\IfFileExists{\jobname-pw.ind}{\input{\jobname-pw.ind}}{}

% Quellenangabe nur in der Leseansicht
\ifkorrekturansicht\else
% Fallback-Definitionen, falls die .tex-Datei \titel etc. nicht gesetzt hat
\providecommand{\titel}{}
\providecommand{\editorInnen}{}
\providecommand{\dateiname}{\jobname}

\vspace{3cm}

\vfill

\footnotesize
\textsc{Quelle}: \titel. Herausgegeben von {\editorInnen}. In: \emph{Arthur Schnitzler: Briefwechsel mit Autorinnen und Autoren}.
 Digitale Edition, https://schnitzler-briefe.acdh.oeaw.ac.at/{\dateiname}.html (Stand \today)
\fi

\end{document}


