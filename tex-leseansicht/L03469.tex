%% latex-leseansicht-vorspann.tex
%% Vorspann für die Leseansicht.
%% Lädt die gemeinsame Datei latex-vorspann.tex mit nicht gesetztem Schalter.

\newif\ifkorrekturansicht
\korrekturansichtfalse

\input{../tex-inputs/latex-vorspann}

\begin{center}
            \textcolor{red}{ENTWURF, NICHT FERTIG KORRIGIERT}
                      \end{center}
            
         
         \renewcommand{\erwaehntePersonen}{Personen: Margarethe Fränkel, Olga Schnitzler, Hans Tietze, Erica Tietze-Conrat}
         \renewcommand{\erwaehnteInstitutionen}{Institutionen: Hotel Sacher}
         \renewcommand{\erwaehnteOrte}{Orte: Armbrustergasse, Edmund-Weiß-Gasse, Hohe Warte, Hotel Sacher, Sternwartestraße, Wien}
         \renewcommand{\erwaehnteWerke}{}
               \section[ Paul Goldmann und Eva Marie Goldmann an Arthur Schnitzler, 1. 10. 1909]{ Paul Goldmann und Eva Marie Goldmann an Arthur
               Schnitzler, 1. 10. 1909}\nopagebreak\mylabel{v}\rehead{ }\begin{ledgroupsized}[t]{13cm}\normalsize\beginnumbering \toendnotes[C]{\smallbreak\pagebreak[2]} \Standort{DLA, A:Schnitzler, HS.NZ85.1.3175.}
\physDesc{Kartenbrief
\newline{}Handschrift Paul Goldmann: 1) schwarze Tinte, deutsche Kurrent\hspace{1em}2) schwarze Tinte, lateinische Kurrent (\noindent{}Adresse)\hspace{1em}\newline{}Handschrift Eva Marie Goldmann: schwarze Tinte, lateinische Kurrent\newline{}Versand: Stempel: »\nobreak{}\textcolor{gray}{1/1} \textcolor{gray}{Wi}{[}en{]} 3a, 1. X. {[}01{]}, \textcolor{gray}{1}–2\nobreak{}«.  }\toendnotes[C]{\smallbreak}\pstart{}{\pb}Herrn\pend{}\pstart{}Dr. Arthur Schnitzler\pend{}\pstart{}Wien\oindex{Wien@\textbf{Wien}|pw}\pend{}\pstart{}XVIII. Spöttelgaſse 7\oindex{Edmund-Weiss-Gasse@\textbf{Edmund-Weiß-Gasse}|pw}.\pend{}{\bigskip}\pstart
           1. 10. 09.\pend
           \pstart
           Lieber Freund, Ich fahre heut{ }Mittag{ }\label{K_L03469-1v}\edtext{ab}{\lemma{\textnormal{\emph{ab}}}\Cendnote{\textnormal{aus Wien\oindex{Wien@\textbf{Wien}|pwk}, am 28. 9. 1909 hatte er
                     Schnitzler\pwindex{Schnitzler, Arthur 15.05.1862 – 21.10.1931@\textsc{Schnitzler, Arthur} (15.05.1862 – 21.10.1931), \emph{Schriftsteller, Mediziner}|pwk} noch besucht}}}\label{K_L03469-1h} u. will
               Dir nur raſch vorher mitteilen, daß meine Schwägerin\pwindex{Fraenkel, Margarethe *~25.10.1880@\textsc{Fränkel, Margarethe} (*~25.10.1880)|pwv}, Frl. \textsc{Fränkel\pwindex{Fraenkel, Margarethe *~25.10.1880@\textsc{Fränkel, Margarethe} (*~25.10.1880)|pw}}, die im \textsc{Hotel Sacher\oindex{Hotel Sacher@\textbf{Hotel Sacher}|pw}} wohnt, gern bereit iſt, Dich in das Haus\oindex{Armbrustergasse@\textbf{Armbrustergasse}|pwv} des \label{K_L03469-2v}\edtext{\textsc{Dr. Tietze\pwindex{Tietze, Hans 01.03.1880 – 12.04.1954@\textsc{Tietze, Hans} (01.03.1880 – 12.04.1954), \emph{Ministerialbeamter}|pw}}}{\lemma{\textnormal{\emph{Dr. Tietze}}}\Cendnote{\textnormal{Arthur und Olga Schnitzler\pwindex{Schnitzler, Arthur 15.05.1862 – 21.10.1931@\textsc{Schnitzler, Arthur} (15.05.1862 – 21.10.1931), \emph{Schriftsteller, Mediziner}|pwk}\pwindex{Schnitzler, Olga 17.01.1882 – 13.01.1970@\textsc{Schnitzler, Olga} (17.01.1882 – 13.01.1970), \emph{Schauspielerin, Sängerin}|pwk} waren
                  auf Wohnungs- bzw. Haussuche. Am 16. 7. 1910 übersiedelten sie schließlich in die
                     Sternwartestraße 71\oindex{Sternwartestrasse@\textbf{Sternwartestraße}|pwk}. Ob sie das Haus\oindex{Armbrustergasse@\textbf{Armbrustergasse}|pwkv} besichtigten, in dem Hans Tietze\pwindex{Tietze, Hans 01.03.1880 – 12.04.1954@\textsc{Tietze, Hans} (01.03.1880 – 12.04.1954), \emph{Ministerialbeamter}|pwk} mit seiner Frau Erica Tietze-Conrat\pwindex{Tietze-Conrat, Erica 1883-06-20 – 1958-12-12@\textsc{Tietze-Conrat, Erica} (1883-06-20 – 1958-12-12), \emph{Kunsthistorikerin}|pwk} wohnte, ist nicht zu
                  klären.}}}\label{K_L03469-2h}, der eine Couſine\pwindex{Tietze-Conrat, Erica 1883-06-20 – 1958-12-12@\textsc{Tietze-Conrat, Erica} (1883-06-20 – 1958-12-12), \emph{Kunsthistorikerin}|pwv} von ihr {\pb}geheiratet hat, zu führen.
               Du brauchſt ihr\pwindex{Fraenkel, Margarethe *~25.10.1880@\textsc{Fränkel, Margarethe} (*~25.10.1880)|pwv} nur ins \textsc{Hotel Sacher\orgindex{Hotel Sacher@Hotel Sacher|pw}} zu telephoniren\footnote{\noindent{}{[}hs. Eva Marie Goldmann:{]} Lieber zu Sacher\orgindex{Hotel Sacher@Hotel Sacher|pw} ein paar Zeilen schreiben. Telephoniren ist fast nicht zu
                     machen. {\\}Viele Grüsse \spacefill\mbox{EvaG.}}. {[}hs. Paul Goldmann:{]} Du ſollteſt Dir das Haus\oindex{Armbrustergasse@\textbf{Armbrustergasse}|pwv}, das tatſächlich mit den billigſten
               Mitteln erbaut iſt u. auf der Hohen Warte\oindex{Hohe Warte@\textbf{Hohe Warte}|pw}, Armbruſterſtraße 20\oindex{Armbrustergasse@\textbf{Armbrustergasse}|pw}, ſteht, einmal anſehen, ehe Du
               daran gehſt, die Wohnungsfrage zu löſen.\pend
           \pstart
           Herzliche Grüße Deiner Frau\pwindex{Schnitzler, Olga 17.01.1882 – 13.01.1970@\textsc{Schnitzler, Olga} (17.01.1882 – 13.01.1970), \emph{Schauspielerin, Sängerin}|pwv} u. Dir! Dein {\\[\baselineskip]}\spacefill\mbox{Paul Goldmann.}\pend
           \leftskip=0em{}
         
         \endnumbering\mylabel{h}\end{ledgroupsized}\begin{anhang}\end{anhang}\newcommand{\dateiname}{L03469}\newcommand{\titel}{Paul Goldmann und Eva Marie Goldmann an Arthur Schnitzler, 1. 10. 1909}\newcommand{\editorInnen}{Martin Anton Müller und Laura Untner}%% latex-leseansicht-abspann.tex
%% Abspann für die Leseansicht.
%% Der Schalter \ifkorrekturansicht ist bereits durch den Vorspann gesetzt.

%% latex-abspann.tex
%% Gemeinsamer Abspann für Korrekturansicht und Leseansicht.
%% Setzt den Schalter \ifkorrekturansicht voraus (gesetzt in den
%% einbindenden Dateien latex-korrekturansicht-abspann.tex bzw.
%% latex-leseansicht-abspann.tex).
%% ---------------------------------------------------------------

\normalsize

% Das esempio-Environment wird nur in der Leseansicht benötigt
\ifkorrekturansicht\else
\newenvironment{esempio}[3]%
{
    \vspace{1.5ex}
    \rlap{\underline{#1}}
    \par
    \setlength{\parindent}{0cm}
    \nopagebreak
    \leftskip=#2cm
    \rightskip=#3cm
}
{
    \par
}
\fi

\doendnotes{C}
\bigskip
\vfill

\clearpage

\footnotesize

\ifkorrekturansicht
  \lohead{\textsc{register}}
\fi

% theindex-Environment neu definieren ohne reledmac
\makeatletter
\renewenvironment{theindex}{%
  \ifkorrekturansicht
    \section*{\indexname}%
  \else
    \subsubsection*{Index der erwähnten Entitäten}%
  \fi
  \setlength{\parindent}{0pt}%
  \setlength{\parskip}{0pt plus 0.3pt}%
  \let\item\@idxitem
}{%
  \ifkorrekturansicht\clearpage\fi
}
\makeatother

\IfFileExists{\jobname-pw.ind}{\input{\jobname-pw.ind}}{}

% Quellenangabe nur in der Leseansicht
\ifkorrekturansicht\else
% Fallback-Definitionen, falls die .tex-Datei \titel etc. nicht gesetzt hat
\providecommand{\titel}{}
\providecommand{\editorInnen}{}
\providecommand{\dateiname}{\jobname}

\vspace{3cm}

\vfill

\footnotesize
\textsc{Quelle}: \titel. Herausgegeben von {\editorInnen}. In: \emph{Arthur Schnitzler: Briefwechsel mit Autorinnen und Autoren}.
 Digitale Edition, https://schnitzler-briefe.acdh.oeaw.ac.at/{\dateiname}.html (Stand \today)
\fi

\end{document}


      