%% latex-leseansicht-vorspann.tex
%% Vorspann für die Leseansicht.
%% Lädt die gemeinsame Datei latex-vorspann.tex mit nicht gesetztem Schalter.

\newif\ifkorrekturansicht
\korrekturansichtfalse

\input{../tex-inputs/latex-vorspann}


\section[Detlev von Liliencron und Marcell Salzer an Arthur Schnitzler, 7. 12. 1896]{L00626 Detlev von Liliencron und Marcell Salzer an Arthur Schnitzler, 7. 12. 1896}
\nopagebreak\mylabel{L00626v}
\rehead{ }\normalsize\beginnumbering\briefempfaengerindex{Schnitzler, Arthur@\textsc{Schnitzler, Arthur}!zzzSalzer, Marcell@\emph{von Marcell Salzer}!1896-12-071@{7. 12. 1896}|(be}\briefempfaengerindex{Schnitzler, Arthur@\textsc{Schnitzler, Arthur}!zzzLiliencron, Detlev von@\emph{von Detlev von Liliencron}!1896-12-071@{7. 12. 1896}|(be}
\toendnotes[C]{\smallbreak\pagebreak[2]}
\correspDesc{Versand  durch Detlev von Liliencron, Marcell Salzer am 7. 12. 1896 in Hamburg
\newline{}Erhalt  durch Arthur Schnitzler am 9. 12. 1896 in Wien}\toendnotes[C]{\smallbreak}
\Standort{CUL, Schnitzler, B 59.}
\physDesc{Postkarte, 288 Zeichen
\newline{}Handschrift Detlev von Liliencron: schwarze Tinte, deutsche Kurrent
\newline{}Handschrift Marcell Salzer: schwarze Tinte, deutsche Kurrent
\newline{}Versand: 1) Stempel: »\nobreak{}\oindex{Altona@\textbf{Altona}, \emph{Ehemaliger Ort}|pwk}Altona (Elbe), 7. 12. 96, 8–9 N\nobreak{}«.   2) Stempel: »\nobreak{}\oindex{IX., Alsergrund@\textbf{IX., Alsergrund}, \emph{Verwaltungsgebiet}|pwk}Wien 9/3, 9. 12. 96, 8.V, Bestellt\nobreak{}«. }\pstart{}{\pb}Herrn Dr. Arthur
                  Schnitzler,\pend{}\pstart{}Wien, IX\oindex{IX., Alsergrund@\textbf{IX., Alsergrund}, \emph{Verwaltungsgebiet}|pw}\pend{}\pstart{}Frankgaſſe 1\oindex{Wien@\textbf{Wien}!IX., Alsergrund@\textbf{IX., Alsergrund}!Frankgasse 1@\textbf{Frankgasse 1}, \emph{Wohngebäude}|pw}.\pend{}{\bigskip}\vspace{1em}
\pstart
           \noindent{}{\pb}{[}hs. Liliencron:{]} Marcell Salzer, der herrliche Vorleſer{ }ſitzt neben mir
               und erzählt mir von Ihnen! Hurrah! Ihr\pend
           \pstart \spacefill\mbox{Liliencron.}\pend{}\selectlanguage{ngerman}\vspace{1em}
\pstart
           \noindent{}{[}hs. Salzer:{]} Danke herzlichſt \substVorne{}\textsuperscript{H}\substDazwischen{}v\substHinten{}erehrteſter Herr Doctor für Ihren Hamburg\oindex{Hamburg@\textbf{Hamburg}|pw}-Brief. Nehmen Sie meine i{\geminationn}igſten
               verehrungsvollſten Grüße entgegen.\pend
           
\pstart
           Ihr{\\[\baselineskip]}\spacefill\mbox{Salzer}\pend
           \leftskip=0em{}\selectlanguage{ngerman}\endnumbering\briefempfaengerindex{Schnitzler, Arthur@\textsc{Schnitzler, Arthur}!zzzSalzer, Marcell@\emph{von Marcell Salzer}!1896-12-071@{7. 12. 1896}|)be}\briefempfaengerindex{Schnitzler, Arthur@\textsc{Schnitzler, Arthur}!zzzLiliencron, Detlev von@\emph{von Detlev von Liliencron}!1896-12-071@{7. 12. 1896}|)be}\mylabel{L00626h}  \newcommand{\dateiname}{L00626}\newcommand{\titel}{Detlev von Liliencron und Marcell Salzer an Arthur Schnitzler, 7. 12. 1896}\newcommand{\editorInnen}{Martin Anton Müller und Gerd-Hermann Susen}%% latex-leseansicht-abspann.tex
%% Abspann für die Leseansicht.
%% Der Schalter \ifkorrekturansicht ist bereits durch den Vorspann gesetzt.

%% latex-abspann.tex
%% Gemeinsamer Abspann für Korrekturansicht und Leseansicht.
%% Setzt den Schalter \ifkorrekturansicht voraus (gesetzt in den
%% einbindenden Dateien latex-korrekturansicht-abspann.tex bzw.
%% latex-leseansicht-abspann.tex).
%% ---------------------------------------------------------------

\normalsize

% Das esempio-Environment wird nur in der Leseansicht benötigt
\ifkorrekturansicht\else
\newenvironment{esempio}[3]%
{
    \vspace{1.5ex}
    \rlap{\underline{#1}}
    \par
    \setlength{\parindent}{0cm}
    \nopagebreak
    \leftskip=#2cm
    \rightskip=#3cm
}
{
    \par
}
\fi

\doendnotes{C}
\bigskip
\vfill

\clearpage

\footnotesize

\ifkorrekturansicht
  \lohead{\textsc{register}}
\fi

% theindex-Environment neu definieren ohne reledmac
\makeatletter
\renewenvironment{theindex}{%
  \ifkorrekturansicht
    \section*{\indexname}%
  \else
    \subsubsection*{Index der erwähnten Entitäten}%
  \fi
  \setlength{\parindent}{0pt}%
  \setlength{\parskip}{0pt plus 0.3pt}%
  \let\item\@idxitem
}{%
  \ifkorrekturansicht\clearpage\fi
}
\makeatother

\IfFileExists{\jobname-pw.ind}{\input{\jobname-pw.ind}}{}

% Quellenangabe nur in der Leseansicht
\ifkorrekturansicht\else
% Fallback-Definitionen, falls die .tex-Datei \titel etc. nicht gesetzt hat
\providecommand{\titel}{}
\providecommand{\editorInnen}{}
\providecommand{\dateiname}{\jobname}

\vspace{3cm}

\vfill

\footnotesize
\textsc{Quelle}: \titel. Herausgegeben von {\editorInnen}. In: \emph{Arthur Schnitzler: Briefwechsel mit Autorinnen und Autoren}.
 Digitale Edition, https://schnitzler-briefe.acdh.oeaw.ac.at/{\dateiname}.html (Stand \today)
\fi

\end{document}


