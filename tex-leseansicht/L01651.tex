%% latex-leseansicht-vorspann.tex
%% Vorspann für die Leseansicht.
%% Lädt die gemeinsame Datei latex-vorspann.tex mit nicht gesetztem Schalter.

\newif\ifkorrekturansicht
\korrekturansichtfalse

\input{../tex-inputs/latex-vorspann}


               \section[Hugo von Hofmannsthal an Arthur Schnitzler, {[}14. 1. 1907{]}]{ Hugo von Hofmannsthal an Arthur Schnitzler, {[}14. 1. 1907{]}}\nopagebreak\mylabel{v}\rehead{ }\begin{ledgroupsized}[t]{13cm}\normalsize\beginnumbering\briefempfaengerindex{Schnitzler, Arthur@\textsc{Schnitzler, Arthur}!zzzHofmannsthal, Hugo von@\emph{von Hugo von Hofmannsthal}!1907-01-141@{{[}14. 1. 1907{]}}|(be} \toendnotes[C]{\smallbreak\pagebreak[2]} \Standort{CUL, Schnitzler, B 43.}
\physDesc{Brief, 1 Blatt, 2 Seiten
\newline{}Handschrift: schwarze Tinte, deutsche Kurrent
\newline{}Schnitzler: mit Bleistift datiert: »14/1 907« \newline{}Ordnung: 1) mit Bleistift von unbekannter Hand nummeriert: »\strikeout{264}« 2) mit Bleistift von unbekannter Hand nummeriert:
                                    »270«}\buchAbdrucke{\weitereDrucke{Hugo von Hofmannsthal, Arthur Schnitzler: \emph{Briefwechsel}. Hg. Therese Nickl und Heinrich Schnitzler. Frankfurt am Main: \emph{S. Fischer} 1964, S. 226.} }\toendnotes[C]{\smallbreak}\pstart{}{\pb}mein lieber Arthur\pend\pstart
           es iſt mir natürlich äußerſt zuwider, gerade Ihnen auf einen directen Wunſch \strikeout{ſie} »nein« zu ſagen, aber das geht abſolut nicht\pend
           \pstart
           1.) (und das dürfte ſchon hinreichen) bin ich 2\textsuperscript{te}
                  Hälfte Februar fort\pend
           \pstart
           2.) habe ich mir präcis vorgeno{\geminationm}en, wohl noch Vorträge
               zu halten nie mehr aber verſa{\geminationm}elten Schweinen meine
               schönen Werke vorzuleſen\pend
           \pstart
           {\pb}3 würde ein öffentliches Leſen
               (wenn auch \label{K_L01651_1v}\edtext{zu wohlthätigem Zweck}{\lemma{\textnormal{\emph{zu wohlthätigem Zweck}}}\Cendnote{\textnormal{Am 10. 2. 1907 lasen Jakob Wassermann\pwindex{Wassermann, Jakob 10.03.1873 – 01.01.1934@\textsc{Wassermann, Jakob} (10.03.1873 – 01.01.1934), \emph{Schriftsteller}|pwk} seinen Aufsatz \emph{Das Los der Juden}\pwindex{Wassermann, Jakob 10.03.1873 – 01.01.1934@\textsc{Wassermann, Jakob} (10.03.1873 – 01.01.1934), \emph{Schriftsteller}!Los der Juden1904@\strich\emph{Das Los der Juden} {[}1904{]}|pwk}, Richard
                     Beer-Hofmann\pwindex{Beer-Hofmann, Richard 11.07.1866 – 26.09.1945@\textsc{Beer-Hofmann, Richard} (11.07.1866 – 26.09.1945), \emph{Schriftsteller}|pwk} Gedichte (darunter \emph{Schlaflied
                     für Mirjam}\pwindex{Beer-Hofmann, Richard 11.07.1866 – 26.09.1945@\textsc{Beer-Hofmann, Richard} (11.07.1866 – 26.09.1945), \emph{Schriftsteller}!Schlaflied fuer Mirjam15.11.1898 – 15.11.1898@\strich\emph{Schlaflied für Mirjam} {[}15.11.1898 – 15.11.1898{]}|pwk}), Felix Salten\pwindex{Salten, Felix 06.09.1869 – 08.10.1945@\textsc{Salten, Felix} (06.09.1869 – 08.10.1945), \emph{Schriftsteller, Journalist}|pwk} seine Novelle
                     \emph{Der Ernst des Lebens}\pwindex{Salten, Felix 06.09.1869 – 08.10.1945@\textsc{Salten, Felix} (06.09.1869 – 08.10.1945), \emph{Schriftsteller, Journalist}!Ernst des Lebens1902@\strich\emph{Der Ernst des Lebens} {[}1902{]}|pwk} sowie Schnitzler\pwindex{Schnitzler, Arthur 15.05.1862 – 21.10.1931@\textsc{Schnitzler, Arthur} (15.05.1862 – 21.10.1931), \emph{Schriftsteller, Mediziner}|pwk}{ }\emph{Lieutenant Gustl}\pwindex{Schnitzler, Arthur 15.05.1862 – 21.10.1931@\textsc{Schnitzler, Arthur} (15.05.1862 – 21.10.1931), \emph{Schriftsteller, Mediziner}!Lieutenant Gustl. Novelle25. 12. 1900@\strich\emph{Lieutenant Gustl. Novelle} {[}25. 12. 1900{]}|pwk} vor.}}}\label{K_L01651_1h}) die Demonſtration
               die in meiner jetzigen \label{K_L01651_2v}\edtext{kl. Veranſtaltung\pwindex{Hofmannsthal, Hugo von 01.02.1874 – 15.07.1929@\textsc{Hofmannsthal, Hugo von} (01.02.1874 – 15.07.1929), \emph{Schriftsteller}!Dichter und diese Zeit30.11.1906 – 30.11.1906@\strich\emph{Der Dichter und diese Zeit} {[}30.11.1906 – 30.11.1906{]}|pwv}}{\lemma{\textnormal{\emph{kl. Veranſtaltung}}}\Cendnote{\textnormal{Am 17. 1. 1907 hielt Hofmannsthal\pwindex{Hofmannsthal, Hugo von 01.02.1874 – 15.07.1929@\textsc{Hofmannsthal, Hugo von} (01.02.1874 – 15.07.1929), \emph{Schriftsteller}|pwk} den Vortrag \emph{Der Dichter und diese Zeit}\pwindex{Hofmannsthal, Hugo von 01.02.1874 – 15.07.1929@\textsc{Hofmannsthal, Hugo von} (01.02.1874 – 15.07.1929), \emph{Schriftsteller}!Dichter und diese Zeit30.11.1906 – 30.11.1906@\strich\emph{Der Dichter und diese Zeit} {[}30.11.1906 – 30.11.1906{]}|pwk} im \emph{Kunstsalon Miethke}\orgindex{Galerie Miethke@Galerie Miethke|pwk} vor geladenen, zehn Kronen zahlenden Gästen.}}}\label{K_L01651_2h}
               liegt (Hinauswurf von Preſſe und Premièrenpack) geradezu auf den Kopf ſtellen.\pend
           \pstart
           Ihr{\\[\baselineskip]}\spacefill\mbox{Hugo.}\pend
           \leftskip=0em{}          \endnumbering\briefempfaengerindex{Schnitzler, Arthur@\textsc{Schnitzler, Arthur}!zzzHofmannsthal, Hugo von@\emph{von Hugo von Hofmannsthal}!1907-01-141@{{[}14. 1. 1907{]}}|)be}\mylabel{h}\end{ledgroupsized}  \newcommand{\dateiname}{L01651}\newcommand{\titel}{Hugo von Hofmannsthal an Arthur Schnitzler, [14. 1. 1907]}\newcommand{\editorInnen}{Martin Anton Müller und Gerd-Hermann Susen}
            \footnotesize
\begin{ledgroupsized}[t]{11.5cm}
\doendnotes{C}
\end{ledgroupsized}
         %% latex-leseansicht-abspann.tex
%% Abspann für die Leseansicht.
%% Der Schalter \ifkorrekturansicht ist bereits durch den Vorspann gesetzt.

%% latex-abspann.tex
%% Gemeinsamer Abspann für Korrekturansicht und Leseansicht.
%% Setzt den Schalter \ifkorrekturansicht voraus (gesetzt in den
%% einbindenden Dateien latex-korrekturansicht-abspann.tex bzw.
%% latex-leseansicht-abspann.tex).
%% ---------------------------------------------------------------

\normalsize

% Das esempio-Environment wird nur in der Leseansicht benötigt
\ifkorrekturansicht\else
\newenvironment{esempio}[3]%
{
    \vspace{1.5ex}
    \rlap{\underline{#1}}
    \par
    \setlength{\parindent}{0cm}
    \nopagebreak
    \leftskip=#2cm
    \rightskip=#3cm
}
{
    \par
}
\fi

\doendnotes{C}
\bigskip
\vfill

\clearpage

\footnotesize

\ifkorrekturansicht
  \lohead{\textsc{register}}
\fi

% theindex-Environment neu definieren ohne reledmac
\makeatletter
\renewenvironment{theindex}{%
  \ifkorrekturansicht
    \section*{\indexname}%
  \else
    \subsubsection*{Index der erwähnten Entitäten}%
  \fi
  \setlength{\parindent}{0pt}%
  \setlength{\parskip}{0pt plus 0.3pt}%
  \let\item\@idxitem
}{%
  \ifkorrekturansicht\clearpage\fi
}
\makeatother

\IfFileExists{\jobname-pw.ind}{\input{\jobname-pw.ind}}{}

% Quellenangabe nur in der Leseansicht
\ifkorrekturansicht\else
% Fallback-Definitionen, falls die .tex-Datei \titel etc. nicht gesetzt hat
\providecommand{\titel}{}
\providecommand{\editorInnen}{}
\providecommand{\dateiname}{\jobname}

\vspace{3cm}

\vfill

\footnotesize
\textsc{Quelle}: \titel. Herausgegeben von {\editorInnen}. In: \emph{Arthur Schnitzler: Briefwechsel mit Autorinnen und Autoren}.
 Digitale Edition, https://schnitzler-briefe.acdh.oeaw.ac.at/{\dateiname}.html (Stand \today)
\fi

\end{document}


      