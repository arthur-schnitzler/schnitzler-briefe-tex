%% latex-leseansicht-vorspann.tex
%% Vorspann für die Leseansicht.
%% Lädt die gemeinsame Datei latex-vorspann.tex mit nicht gesetztem Schalter.

\newif\ifkorrekturansicht
\korrekturansichtfalse

\input{../tex-inputs/latex-vorspann}


\section[Arthur Schnitzler an Richard Beer-Hofmann, 6. 7. 1899]{L00933 Arthur Schnitzler an Richard Beer-Hofmann, 6. 7. 1899}
\nopagebreak\mylabel{L00933v}
\rehead{ }\normalsize\beginnumbering\briefempfaengerindex{Beer-Hofmann, Richard@\textsc{Beer-Hofmann, Richard}!zzzSchnitzler, Arthur@\emph{von Arthur Schnitzler}!1899-07-061@{6. 7. 1899}|(be}
\toendnotes[C]{\smallbreak\pagebreak[2]}
\correspDesc{Versand  durch Arthur Schnitzler am 6. 7. 1899 in Wien
\newline{}Erhalt  durch Richard Beer-Hofmann am 7. 7. 1899 in Seeboden}\toendnotes[C]{\smallbreak}
\Standort{YCGL, MSS 31.}
\physDesc{Brief, 1 Blatt, 2 Seiten, Kuvert, 500 Zeichen
\newline{}Handschrift: Bleistift, deutsche Kurrent
\newline{}Versand: 1) Stempel: »\nobreak{}\oindex{I., Innere Stadt@\textbf{I., Innere Stadt}, \emph{Verwaltungsgebiet}|pwk}Wien 1/1, 6. 7. 99, 2–3N\nobreak{}«.   2) Stempel: »\nobreak{}\oindex{Seeboden am Millstättersee@\textbf{Seeboden am Millstättersee}|pwk}{\pb}Seeboden, \textcolor{gray}{7}. 7. 99\nobreak{}«. }
\buchAbdrucke{\weitereDrucke{Arthur Schnitzler, Richard Beer-Hofmann: \emph{Briefwechsel 1891–1931}. Herausgegeben von Konstanze Fliedl. Wien, Zürich: \emph{Europaverlag} 1992, S. 131.} }\toendnotes[C]{\smallbreak}\pstart{}{\pb}\textsc{Kärnthen}\oindex{Kärnten@\textbf{Kärnten}, \emph{Land}|pw}\pend{}\pstart{}\textsc{Herrn Dr. Rich. Beer-Hofmann}\pend{}\pstart{}\textsc{Villa Platzer}\oindex{Villa Platzer@\textbf{Villa Platzer}, \emph{Gebäude}|pw}\pend{}\pstart{}\textsc{Seeboden am Millstätter}ſee\oindex{Seeboden am Millstättersee@\textbf{Seeboden am Millstättersee}|pw}\pend{}{\bigskip}\vspace{1em}
\pstart
           \raggedleft{}{\pb}6/7 99\pend
           \vspace{0.5em}
\pstart
           lieber,{ }Mayer\pwindex{Mayer, Oskar 1876 – 15.\,5.\,1915 München@\textsc{Mayer, Oskar} (1876 – 15.\,5.\,1915 München), \emph{Schriftsteller, Beamter}|pw} ko{\geminationm}t ja
               keineswegs mit; hat ers Ihnen noch nicht geſchrieben?\pend
           
\pstart
           – Ich ko{\geminationm}e Mitte Juli nach \textsc{Velden}\oindex{Velden am Wörthersee@\textbf{Velden am Wörthersee}|pw} zu meiner Mama\pwindex{Schnitzler, Louise 8.\,7.\,1840 Kőszeg – 9.\,9.\,1911 Wien@\textsc{Schnitzler, Louise} (8.\,7.\,1840 Kőszeg – 9.\,9.\,1911 Wien)|pwv}, beſuch
               Sie dann gleich (oder Sie mich?) wir beſprechen dann näheres.\pend
           
\pstart
           Eigentlich möchte ich am {\pb}31. Juli in \textsc{Bayreuth}\oindex{Bayreuth@\textbf{Bayreuth}, \emph{Hauptstadt}|pw} zu \textsc{Parsifal}\pwindex{\textcolor{red}{\textsuperscript{XXXX indx1}}!Parsifal@\strich\emph{Parsifal}|pw}{ }ſein.\pend
           
\pstart
           Es ärgert mich dſs Sie mir mit keinem Wort{ }ſchreiben was Sie thun oder nicht
               thun.\pend
           
\pstart
           – Den Todten muſs es{ }ſehr komiſch vorkommen, was wir »Erleben« nennen. –\pend
           \pstart Herzlichſt Ihr \spacefill\mbox{Arthur}\pend{}\selectlanguage{ngerman}\endnumbering\briefempfaengerindex{Beer-Hofmann, Richard@\textsc{Beer-Hofmann, Richard}!zzzSchnitzler, Arthur@\emph{von Arthur Schnitzler}!1899-07-061@{6. 7. 1899}|)be}\mylabel{L00933h}  \newcommand{\dateiname}{L00933}\newcommand{\titel}{Arthur Schnitzler an Richard Beer-Hofmann, 6. 7. 1899}\newcommand{\editorInnen}{Martin Anton Müller und Gerd-Hermann Susen}%% latex-leseansicht-abspann.tex
%% Abspann für die Leseansicht.
%% Der Schalter \ifkorrekturansicht ist bereits durch den Vorspann gesetzt.

%% latex-abspann.tex
%% Gemeinsamer Abspann für Korrekturansicht und Leseansicht.
%% Setzt den Schalter \ifkorrekturansicht voraus (gesetzt in den
%% einbindenden Dateien latex-korrekturansicht-abspann.tex bzw.
%% latex-leseansicht-abspann.tex).
%% ---------------------------------------------------------------

\normalsize

% Das esempio-Environment wird nur in der Leseansicht benötigt
\ifkorrekturansicht\else
\newenvironment{esempio}[3]%
{
    \vspace{1.5ex}
    \rlap{\underline{#1}}
    \par
    \setlength{\parindent}{0cm}
    \nopagebreak
    \leftskip=#2cm
    \rightskip=#3cm
}
{
    \par
}
\fi

\doendnotes{C}
\bigskip
\vfill

\clearpage

\footnotesize

\ifkorrekturansicht
  \lohead{\textsc{register}}
\fi

% theindex-Environment neu definieren ohne reledmac
\makeatletter
\renewenvironment{theindex}{%
  \ifkorrekturansicht
    \section*{\indexname}%
  \else
    \subsubsection*{Index der erwähnten Entitäten}%
  \fi
  \setlength{\parindent}{0pt}%
  \setlength{\parskip}{0pt plus 0.3pt}%
  \let\item\@idxitem
}{%
  \ifkorrekturansicht\clearpage\fi
}
\makeatother

\IfFileExists{\jobname-pw.ind}{\input{\jobname-pw.ind}}{}

% Quellenangabe nur in der Leseansicht
\ifkorrekturansicht\else
% Fallback-Definitionen, falls die .tex-Datei \titel etc. nicht gesetzt hat
\providecommand{\titel}{}
\providecommand{\editorInnen}{}
\providecommand{\dateiname}{\jobname}

\vspace{3cm}

\vfill

\footnotesize
\textsc{Quelle}: \titel. Herausgegeben von {\editorInnen}. In: \emph{Arthur Schnitzler: Briefwechsel mit Autorinnen und Autoren}.
 Digitale Edition, https://schnitzler-briefe.acdh.oeaw.ac.at/{\dateiname}.html (Stand \today)
\fi

\end{document}


