%% latex-korrekturansicht-vorspann.tex
%% Vorspann für die Korrekturansicht.
%% Lädt die gemeinsame Datei latex-vorspann.tex mit gesetztem Schalter.

\newif\ifkorrekturansicht
\korrekturansichttrue

\input{../tex-inputs/latex-vorspann}


\section[Arthur Schnitzler an Richard Beer-Hofmann, 6. 7. 1899]{L00933 Arthur Schnitzler an Richard Beer-Hofmann, 6. 7. 1899}
\nopagebreak\mylabel{L00933v}
\rehead{ }\normalsize\beginnumbering\briefempfaengerindex{Beer-Hofmann, Richard@\textsc{Beer-Hofmann, Richard}!zzzSchnitzler, Arthur@\emph{von Arthur Schnitzler}!1899-07-061@{6. 7. 1899}|(be}
\toendnotes[C]{\smallbreak\pagebreak[2]}\Standort{YCGL, MSS 31.}
\physDesc{Brief, 1 Blatt, 2 Seiten, Umschlag, 500 Zeichen
\newline{}Handschrift: Bleistift, deutsche Kurrent
\newline{}Versand: 1) Stempel: »\nobreak{}\oindex{I., Innere Stadt@\textbf{I., Innere Stadt}, \emph{A.ADM3}|pwk}Wien 1/1, 6. 7. 99, 2–3N\nobreak{}«.   2) Stempel: »\nobreak{}\oindex{Seeboden@\textbf{Seeboden}, \emph{A.ADM3}|pwk}{\pb}Seeboden, \textcolor{gray}{7}. 7. 99\nobreak{}«. }
\buchAbdrucke{\weitereDrucke{Arthur Schnitzler, Richard Beer-Hofmann: \emph{Briefwechsel 1891–1931}. Wien, Zürich: \emph{Europaverlag} 1992, S. 131.} }\toendnotes[C]{\smallbreak}\pstart{}{\pb}\textsc{Kärnthen}\oindex{Kaernten@\textbf{Kärnten}, \emph{A.ADM1}|pw}\pend{}\pstart{}\textsc{Herrn Dr. Rich. Beer-Hofmann}\pend{}\pstart{}\textsc{Villa Platzer}\oindex{Villa Platzer@\textbf{Villa Platzer}, \emph{Gebäude (K.GBD)}|pw}\pend{}\pstart{}\textsc{Seeboden am Millstätter}ſee\oindex{Seeboden@\textbf{Seeboden}, \emph{A.ADM3}|pw}\pend{}{\bigskip}\vspace{1em}
\pstart
           \raggedleft{}{\pb}6/7 99\pend
           \vspace{0.5em}
\pstart
           lieber,{ }Mayer\pwindex{Mayer, Oskar 1876 – 15.05.1915@\textsc{Mayer, Oskar} (1876 – 15.05.1915), \emph{Schriftsteller/Schriftstellerin, Beamter/Beamte}|pw} ko{\geminationm}t ja
               keineswegs mit; hat ers Ihnen noch nicht geſchrieben?\pend
           
\pstart
           – Ich ko{\geminationm}e Mitte Juli nach \textsc{Velden}\oindex{Velden am Woerthersee@\textbf{Velden am Wörthersee}, \emph{P.PPL}|pw} zu meiner Mama\pwindex{Schnitzler, Louise 1840-07-08 – 1911-09-09@\textsc{Schnitzler, Louise} (1840-07-08 – 1911-09-09)|pwv}, beſuch
               Sie dann gleich (oder Sie mich?) wir beſprechen dann näheres.\pend
           
\pstart
           Eigentlich möchte ich am {\pb}31. Juli in \textsc{Bayreuth}\oindex{Bayreuth@\textbf{Bayreuth}, \emph{P.PPLA2}|pw} zu \textsc{Parsifal}\pwindex{Parsifal@\emph{Parsifal}|pw}{ }ſein.\pend
           
\pstart
           Es ärgert mich dſs Sie mir mit keinem Wort ſchreiben was Sie thun oder nicht
               thun.\pend
           
\pstart
           – Den Todten muſs es ſehr komiſch vorkommen, was wir »Erleben« nennen. – \pend
           \pstart Herzlichſt Ihr \spacefill\mbox{Arthur}\pend{}\selectlanguage{ngerman}\endnumbering\briefempfaengerindex{Beer-Hofmann, Richard@\textsc{Beer-Hofmann, Richard}!zzzSchnitzler, Arthur@\emph{von Arthur Schnitzler}!1899-07-061@{6. 7. 1899}|)be}\mylabel{L00933h}  \normalsize

\doendnotes{C}
\bigskip
\vfill

\clearpage

\footnotesize

\lohead{\textsc{register}}

% Definiere theindex-Environment komplett neu ohne reledmac
\makeatletter
\renewenvironment{theindex}{%
  \section*{\indexname}%
  \setlength{\parindent}{0pt}%
  \setlength{\parskip}{0pt plus 0.3pt}%
  \let\item\@idxitem
}{%
  \clearpage
}
\makeatother

\IfFileExists{\jobname-pw.ind}{\input{\jobname-pw.ind}}{}

\end{document}

      