%% latex-korrekturansicht-vorspann.tex
%% Vorspann für die Korrekturansicht.
%% Lädt die gemeinsame Datei latex-vorspann.tex mit gesetztem Schalter.

\newif\ifkorrekturansicht
\korrekturansichttrue

\input{../tex-inputs/latex-vorspann}


\section[Paul Goldmann an Arthur Schnitzler, 3. 12. 1891]{L02673 Paul Goldmann an Arthur Schnitzler, 3. 12. 1891}
\nopagebreak\mylabel{L02673v}
\rehead{ }\normalsize\beginnumbering\briefempfaengerindex{Schnitzler, Arthur@\textsc{Schnitzler, Arthur}!zzzGoldmann, Paul@\emph{von Paul Goldmann}!1891-12-031@{3. 12. 1891}|(be}
\toendnotes[C]{\smallbreak\pagebreak[2]}\Standort{DLA, A:Schnitzler, HS.NZ85.1.3162.}
\physDesc{Kartenbrief, 962 Zeichen
\newline{}Handschrift: 1) schwarze Tinte, deutsche Kurrent\hspace{1em}2) schwarze Tinte, lateinische Kurrent (\noindent{}Adresse)\hspace{1em}
\newline{}Versand: 1) Stempel: »\nobreak{}\oindex{place de la Bourse@\textbf{place de la Bourse}, \emph{Platz (K.PLT)}|pwk}Paris 1 Pl. de la Bourse, 3 Dec 91, 7\textsuperscript{E}\nobreak{}«.   2) Stempel: »\nobreak{}\oindex{I., Innere Stadt@\textbf{I., Innere Stadt}, \emph{A.ADM3}|pwk}Wien \textcolor{gray}{1/1}, 5{[}.{]} 12. 91, 8–9½V.\nobreak{}«. }\toendnotes[C]{\smallbreak}\pstart{}{\pb}\textsc{\begin{otherlanguage}{french}Autriche\end{otherlanguage}\oindex{Oesterreich@\textbf{Österreich}, \emph{A.PCLI}|pw}}!\pend{}\pstart{}\begin{otherlanguage}{french}\textcolor{gray}{\textbf{M}}\textsc{onsieur le docteur}\end{otherlanguage}\pend{}\pstart{}\textsc{Arthur Schnitzler}\pend{}\pstart{}\textsc{\begin{otherlanguage}{french}Vienne\end{otherlanguage}\oindex{Wien@\textbf{Wien}, \emph{A.ADM2}|pw}}\pend{}\pstart{}\textsc{I. Giselastraſse 11\oindex{Ordination Arthur Schnitzler [Boesendorferstrasse 11]@\textbf{Ordination Arthur Schnitzler [Bösendorferstraße 11]}, \emph{Ordination}|pw}. }\pend{}{\bigskip}\vspace{1em}
\pstart
           \raggedleft{}{\pb}\textsc{Paris\oindex{Paris@\textbf{Paris}, \emph{P.PPLC}|pw}}, 3. Dezember.\pend
           
\pstart\center{}Mein lieber Arthur!\pend\vspace{0.5em}
\pstart
           Ich bin in Paris\oindex{Paris@\textbf{Paris}, \emph{P.PPLC}|pw}, das iſt nicht mehr zu leugnen,
               und in den erſten äußeren Eindrücken habe ich beſtätigt gefunden, was Du mir
               geſchrieben: Das iſt eher \label{K_L02673-1v}\edtext{heimlich}{\lemma{\textnormal{\emph{heimlich}}}\Cendnote{\textnormal{im Sinne von: heimatlich
                  (das Gegenteil von ›unheimlich‹)}}}\label{K_L02673-1} als fremd, viel weniger fremd als Brüſſel\oindex{Bruessel@\textbf{Brüssel}, \emph{P.PPLC}|pw}; das iſt im Weſentlichen Wien\oindex{Wien@\textbf{Wien}, \emph{A.ADM2}|pw}, nur farbiger und lebensvoller. Freilich, was mich hier im
                  Büreau\orgindex{Pariser Buero der Frankfurter Zeitung@Pariser Büro der Frankfurter Zeitung|pwv} erwartetete, war
               geeignet, alle freundlichen Eindrücke des Anfangs zu verwiſchen. Ich ſehe es jetzt
               klar, \label{K_L02673-2v}\edtext{was ich Dir ſchrieb}{\lemma{\textnormal{\emph{was ich Dir ſchrieb}}}\Cendnote{\textnormal{Siehe Paul Goldmann an Arthur Schnitzler, 15. 11. 1891.
               }}}\label{K_L02673-2}: zu meinem Beſten hat man mich nicht hergeſandt; es wird ein wilder Kampf
               werden, ſolange ich die Kräfte habe; und auf die Dauer iſt die Stellung unhaltbar.
               Dies unter uns. Wunder Dich nicht, wenn ich Dir in der erſten Zeit wenig ſchreibe.
               Meine Arbeitslaſt hat ſich verfünffacht. Mein Arbeitstag iſt von 7 Uhr Morgens
               bis 1 Uhr Nachts. Viele Grüße an Dich, \textsc{Kapper\pwindex{Kapper, Friedrich 21.04.1861 – 22.07.1939@\textsc{Kapper, Friedrich} (21.04.1861 – 22.07.1939), \emph{Mediziner/Medizinerin}|pw}}, \textsc{Richard\pwindex{Beer-Hofmann, Richard 1866-07-11 – 1945-09-26@\textsc{Beer-Hofmann, Richard} (1866-07-11 – 1945-09-26), \emph{Schriftsteller/Schriftstellerin}|pw}} u. \textsc{Loris\pwindex{Hofmannsthal, Hugo von 1874-02-01 – 1929-07-15@\textsc{Hofmannsthal, Hugo von} (1874-02-01 – 1929-07-15), \emph{Schriftsteller/Schriftstellerin}|pw}}. Dein \spacefill\mbox{P. G.}\pend
           
\pstart
           \noindent{}\label{T_L02673-1v}\edtext{Adreſſe: \textsc{51. Rue Vivienne\oindex{rue Vivienne@\textbf{rue Vivienne}, \emph{Straße (K.STR)}|pw}}, »\textsc{Gazette de Francfort\orgindex{Frankfurter Zeitung@Frankfurter Zeitung|pw}}«\orgindex{Pariser Buero der Frankfurter Zeitung@Pariser Büro der Frankfurter Zeitung|pwv}.}{\lemma{\textnormal{\emph{Adreſſe: … Francfort«.}}}\Cendnote{\textnormal{kopfüber am oberen
                     Rand}}}\label{T_L02673-1}\pend
           \selectlanguage{ngerman}\endnumbering\briefempfaengerindex{Schnitzler, Arthur@\textsc{Schnitzler, Arthur}!zzzGoldmann, Paul@\emph{von Paul Goldmann}!1891-12-031@{3. 12. 1891}|)be}\mylabel{L02673h}  \normalsize

\doendnotes{C}
\bigskip
\vfill

\clearpage

\footnotesize

\lohead{\textsc{register}}

% Definiere theindex-Environment komplett neu ohne reledmac
\makeatletter
\renewenvironment{theindex}{%
  \section*{\indexname}%
  \setlength{\parindent}{0pt}%
  \setlength{\parskip}{0pt plus 0.3pt}%
  \let\item\@idxitem
}{%
  \clearpage
}
\makeatother

\IfFileExists{\jobname-pw.ind}{\input{\jobname-pw.ind}}{}

\end{document}

      