%% latex-leseansicht-vorspann.tex
%% Vorspann für die Leseansicht.
%% Lädt die gemeinsame Datei latex-vorspann.tex mit nicht gesetztem Schalter.

\newif\ifkorrekturansicht
\korrekturansichtfalse

\input{../tex-inputs/latex-vorspann}


         
         \renewcommand{\erwaehntePersonen}{Personen: Richard Beer-Hofmann, Hugo von Hofmannsthal, Friedrich Kapper}
         \renewcommand{\erwaehnteInstitutionen}{Institutionen: Frankfurter Zeitung, Pariser Büro der Frankfurter Zeitung}
         \renewcommand{\erwaehnteOrte}{Orte: Brüssel, Bösendorferstraße, I., Innere Stadt, Paris, Place de la Bourse, Wien, rue Vivienne, Österreich}
         \renewcommand{\erwaehnteWerke}{}
               \section[Paul Goldmann an Arthur Schnitzler, 3. 12. 1891]{ Paul Goldmann an Arthur Schnitzler, 3. 12. 1891}\nopagebreak\mylabel{v}\rehead{ }\begin{ledgroupsized}[t]{13cm}\normalsize\beginnumbering \toendnotes[C]{\smallbreak\pagebreak[2]} \Standort{DLA, A:Schnitzler, HS.NZ85.1.3162.}
\physDesc{Kartenbrief
\newline{}Handschrift: 1) schwarze Tinte, deutsche Kurrent\hspace{1em}2) schwarze Tinte, lateinische Kurrent (\noindent{}Adresse)\hspace{1em}\newline{}Versand: 1) Stempel: »\nobreak{}\oindex{Place de la Bourse@\textbf{Place de la Bourse}|pwk}Paris 1 Pl. de la Bourse, 3 Dec 91, 7\textsuperscript{E}\nobreak{}«.   2) Stempel: »\nobreak{}\oindex{I., Innere Stadt@\textbf{I., Innere Stadt}|pwk}Wien \textcolor{gray}{1/1}, 5{[}.{]} 12. 91, 8–9½V.\nobreak{}«. }\toendnotes[C]{\smallbreak}\pstart{}{\pb}\textsc{\begin{otherlanguage}{french}Autriche\end{otherlanguage}\oindex{Oesterreich@\textbf{Österreich}|pw}}!\pend{}\pstart{}\begin{otherlanguage}{french}\textcolor{gray}{\textbf{M
                     }}\textsc{onsieur le docteur}\end{otherlanguage}\pend{}\pstart{}\textsc{Arthur Schnitzler}\pend{}\pstart{}\textsc{\begin{otherlanguage}{french}Vienne\end{otherlanguage}\oindex{Wien@\textbf{Wien}|pw}}\pend{}\pstart{}\textsc{I. Giselastraße 11.\oindex{Boesendorferstrasse@\textbf{Bösendorferstraße}|pw}}\pend{}{\bigskip}\pstart
           \raggedleft{}{\pb}\textsc{Paris\oindex{Paris@\textbf{Paris}|pw}}, 3. Dezember.\pend
           \pstart\center{}Mein lieber Arthur!\pend\pstart
           Ich bin in Paris\oindex{Paris@\textbf{Paris}|pw}, das iſt nicht mehr zu leugnen,
               und in den erſten äußeren Eindrücken habe ich beſtätigt gefunden, was Du mir
               geſchrieben: Das iſt eher \label{K_L02673-2v}\edtext{heimlich}{\lemma{\textnormal{\emph{heimlich}}}\Cendnote{\textnormal{im Sinne von: heimatlich
                  (das Gegenteil von ›unheimlich‹)}}}\label{K_L02673-2h} als fremd, viel weniger fremd als Brüſſel\oindex{Bruessel@\textbf{Brüssel}|pw}; das iſt im Weſentlichen Wien\oindex{Wien@\textbf{Wien}|pw}, nur farbiger und lebensvoller. Freilich, was mich hier im
                  Büreau\orgindex{Pariser Buero der Frankfurter Zeitung@Pariser Büro der Frankfurter Zeitung|pwv} erwartetete, war
               geeignet, alle freundlichen Eindrücke des Anfangs zu verwiſchen. Ich ſehe es jetzt
               klar, \label{K_L02673-5v}\edtext{was ich Dir ſchrieb}{\lemma{\textnormal{\emph{was ich Dir ſchrieb}}}\Cendnote{\textnormal{siehe Paul Goldmann an Arthur Schnitzler, 15. 11. 1891}}}\label{K_L02673-5h}: zu meinem Beſten hat man mich nicht hergeſandt; es wird ein wilder Kampf
               werden, ſolange ich die Kräfte habe; und auf die Dauer iſt die Stellung unhaltbar.
               Dies unter uns. Wunder Dich nicht, wenn ich Dir in der erſten Zeit wenig ſchreibe.
               Meine Arbeitslaſt hat ſich verfünffacht. Mein Arbeitstag iſt von 7 Uhr Morgens
               bis 1 Uhr Nachts. Viele Grüße an Dich, \textsc{Kapper\pwindex{Kapper, Friedrich 21.04.1861 – 22.07.1939@\textsc{Kapper, Friedrich} (21.04.1861 – 22.07.1939), \emph{Mediziner}|pw}}, \textsc{Richard\pwindex{Beer-Hofmann, Richard 1866-07-11 – 1945-09-26@\textsc{Beer-Hofmann, Richard} (1866-07-11 – 1945-09-26), \emph{Schriftsteller}|pw}} u. \textsc{Loris\pwindex{Hofmannsthal, Hugo von 1874-02-01 – 1929-07-15@\textsc{Hofmannsthal, Hugo von} (1874-02-01 – 1929-07-15), \emph{Schriftsteller}|pw}}. Dein \spacefill\mbox{P. G.}\pend
           \pstart
           \noindent{}\label{T_L02673-1v}\edtext{Adreſſe: \textsc{51. Rue Vivienne\oindex{rue Vivienne@\textbf{rue Vivienne}|pw}}, »\textsc{Gazette de Francfort\orgindex{Frankfurter Zeitung@Frankfurter Zeitung|pw}}«\orgindex{Pariser Buero der Frankfurter Zeitung@Pariser Büro der Frankfurter Zeitung|pwv}.}{\lemma{\textnormal{\emph{Adreſſe: … Francfort«.}}}\Cendnote{\textnormal{kopfüber am oberen
                     Rand}}}\label{T_L02673-1h}\pend
           
         
         \endnumbering\mylabel{h}\end{ledgroupsized}  \newcommand{\dateiname}{L02673}\newcommand{\titel}{Paul Goldmann an Arthur Schnitzler, 3. 12. 1891}\newcommand{\editorInnen}{Martin Anton Müller und Laura Untner}%% latex-leseansicht-abspann.tex
%% Abspann für die Leseansicht.
%% Der Schalter \ifkorrekturansicht ist bereits durch den Vorspann gesetzt.

%% latex-abspann.tex
%% Gemeinsamer Abspann für Korrekturansicht und Leseansicht.
%% Setzt den Schalter \ifkorrekturansicht voraus (gesetzt in den
%% einbindenden Dateien latex-korrekturansicht-abspann.tex bzw.
%% latex-leseansicht-abspann.tex).
%% ---------------------------------------------------------------

\normalsize

% Das esempio-Environment wird nur in der Leseansicht benötigt
\ifkorrekturansicht\else
\newenvironment{esempio}[3]%
{
    \vspace{1.5ex}
    \rlap{\underline{#1}}
    \par
    \setlength{\parindent}{0cm}
    \nopagebreak
    \leftskip=#2cm
    \rightskip=#3cm
}
{
    \par
}
\fi

\doendnotes{C}
\bigskip
\vfill

\clearpage

\footnotesize

\ifkorrekturansicht
  \lohead{\textsc{register}}
\fi

% theindex-Environment neu definieren ohne reledmac
\makeatletter
\renewenvironment{theindex}{%
  \ifkorrekturansicht
    \section*{\indexname}%
  \else
    \subsubsection*{Index der erwähnten Entitäten}%
  \fi
  \setlength{\parindent}{0pt}%
  \setlength{\parskip}{0pt plus 0.3pt}%
  \let\item\@idxitem
}{%
  \ifkorrekturansicht\clearpage\fi
}
\makeatother

\IfFileExists{\jobname-pw.ind}{\input{\jobname-pw.ind}}{}

% Quellenangabe nur in der Leseansicht
\ifkorrekturansicht\else
% Fallback-Definitionen, falls die .tex-Datei \titel etc. nicht gesetzt hat
\providecommand{\titel}{}
\providecommand{\editorInnen}{}
\providecommand{\dateiname}{\jobname}

\vspace{3cm}

\vfill

\footnotesize
\textsc{Quelle}: \titel. Herausgegeben von {\editorInnen}. In: \emph{Arthur Schnitzler: Briefwechsel mit Autorinnen und Autoren}.
 Digitale Edition, https://schnitzler-briefe.acdh.oeaw.ac.at/{\dateiname}.html (Stand \today)
\fi

\end{document}


      