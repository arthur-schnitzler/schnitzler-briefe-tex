%% latex-leseansicht-vorspann.tex
%% Vorspann für die Leseansicht.
%% Lädt die gemeinsame Datei latex-vorspann.tex mit nicht gesetztem Schalter.

\newif\ifkorrekturansicht
\korrekturansichtfalse

\input{../tex-inputs/latex-vorspann}


\section[Paul Goldmann an Arthur Schnitzler, 3. 12. 1891]{L02673 Paul Goldmann an Arthur Schnitzler, 3. 12. 1891}
\nopagebreak\mylabel{L02673v}
\rehead{ }\normalsize\beginnumbering\briefempfaengerindex{Schnitzler, Arthur@\textsc{Schnitzler, Arthur}!zzzGoldmann, Paul@\emph{von Paul Goldmann}!1891-12-031@{3. 12. 1891}|(be}
\toendnotes[C]{\smallbreak\pagebreak[2]}
\correspDesc{Versand  durch Paul Goldmann am 3. 12. 1891 in Paris
\newline{}Erhalt  durch Arthur Schnitzler am 5. 12. 1891 in Wien}\toendnotes[C]{\smallbreak}
\Standort{DLA, A:Schnitzler, HS.NZ85.1.3162.}
\physDesc{Kartenbrief, 962 Zeichen
\newline{}Handschrift: schwarze Tinte, deutsche Kurrent
\newline{}Versand: 1) Stempel: »\nobreak{}\oindex{place de la Bourse@\textbf{place de la Bourse}, \emph{Platz}|pwk}Paris 1 Pl. de la Bourse, 3 Dec 91, 7\textsuperscript{E}\nobreak{}«.   2) Stempel: »\nobreak{}\oindex{I., Innere Stadt@\textbf{I., Innere Stadt}, \emph{Verwaltungsgebiet}|pwk}Wien \textcolor{gray}{1/1}, 5{[}.{]} 12. 91, 8–9½V.\nobreak{}«. }\toendnotes[C]{\smallbreak}\pstart{}\textsc{{\pb}\textsc{\begin{otherlanguage}{french}Autriche\end{otherlanguage}\oindex{Österreich@\textbf{Österreich}|pw}}!}\pend{}\pstart{}\textsc{\begin{otherlanguage}{french}\textcolor{gray}{\textbf{M}}\textsc{onsieur le docteur}\end{otherlanguage}}\pend{}\pstart{}\textsc{\textsc{Arthur Schnitzler}}\pend{}\pstart{}\textsc{\textsc{\begin{otherlanguage}{french}Vienne\end{otherlanguage}\oindex{Wien@\textbf{Wien}, \emph{Verwaltungsgebiet}|pw}}}\pend{}\pstart{}\textsc{\textsc{I. Giselastraſse 11\oindex{Wien@\textbf{Wien}!I., Innere Stadt@\textbf{I., Innere Stadt}!Ordination Arthur Schnitzler [Bösendorferstraße 11]@\textbf{Ordination Arthur Schnitzler [Bösendorferstraße 11]}, \emph{Ordination}|pw}.}}\pend{}{\bigskip}\vspace{1em}
\pstart
           \raggedleft{}{\pb}\textsc{Paris\oindex{Paris@\textbf{Paris}, \emph{Hauptstadt}|pw}}, 3. Dezember.\pend
           
\pstart\center{}Mein lieber Arthur!\pend\vspace{0.5em}
\pstart
           Ich bin in Paris\oindex{Paris@\textbf{Paris}, \emph{Hauptstadt}|pw}, das iſt nicht mehr zu leugnen,
               und in den erſten äußeren Eindrücken habe ich beſtätigt gefunden, was Du mir
               geſchrieben: Das iſt eher \label{K_L02673-1v}\edtext{heimlich}{\lemma{\textnormal{\emph{heimlich}}}\Cendnote{\textnormal{im Sinne von: heimatlich
                  (das Gegenteil von ›unheimlich‹)}}}\label{K_L02673-1} als fremd, viel weniger fremd als Brüſſel\oindex{Brüssel@\textbf{Brüssel}, \emph{Hauptstadt}|pw}; das iſt im Weſentlichen Wien\oindex{Wien@\textbf{Wien}, \emph{Verwaltungsgebiet}|pw}, nur farbiger und lebensvoller. Freilich, was mich hier im
                  Büreau\orgindex{Pariser Büro der Frankfurter Zeitung@Pariser Büro der Frankfurter Zeitung|pwv} erwartetete, war
               geeignet, alle freundlichen Eindrücke des Anfangs zu verwiſchen. Ich{ }ſehe es jetzt
               klar, \label{K_L02673-2v}\edtext{was ich Dir{ }ſchrieb}{\lemma{\textnormal{\emph{was ich Dir schrieb}}}\Cendnote{\textnormal{Siehe XXXX Auszeichnungsfehler: Dokument L02670 nicht gefunden.
               }}}\label{K_L02673-2}: zu meinem Beſten hat man mich nicht hergeſandt; es wird ein wilder Kampf
               werden,{ }ſolange ich die Kräfte habe; und auf die Dauer iſt die Stellung unhaltbar.
               Dies unter uns. Wunder Dich nicht, wenn ich Dir in der erſten Zeit wenig{ }ſchreibe.
               Meine Arbeitslaſt hat{ }ſich verfünffacht. Mein Arbeitstag iſt von 7 Uhr Morgens
               bis 1 Uhr Nachts. Viele Grüße an Dich, \textsc{Kapper\pwindex{Kapper, Friedrich 21.\,4.\,1861 Wien – 22.\,7.\,1939 ebd.@\textsc{Kapper, Friedrich} (21.\,4.\,1861 Wien – 22.\,7.\,1939 ebd.), \emph{Mediziner}|pw}}, \textsc{Richard\pwindex{Beer-Hofmann, Richard 11.\,7.\,1866 Wien – 26.\,9.\,1945 New York City@\textsc{Beer-Hofmann, Richard} (11.\,7.\,1866 Wien – 26.\,9.\,1945 New York City), \emph{Schriftsteller}|pw}} u. \textsc{Loris\pwindex{Hofmannsthal, Hugo von 1.\,2.\,1874 Wien – 15.\,7.\,1929 Rodaun@\textsc{Hofmannsthal, Hugo von} (1.\,2.\,1874 Wien – 15.\,7.\,1929 Rodaun), \emph{Schriftsteller}|pw}}. Dein \spacefill\mbox{P. G.}\pend
           
\pstart
           \noindent{}\label{T_L02673-1v}\edtext{Adreſſe: \textsc{51. Rue Vivienne\oindex{rue Vivienne@\textbf{rue Vivienne}, \emph{Straße}|pw}}, »\textsc{Gazette de Francfort\orgindex{Frankfurter Zeitung@Frankfurter Zeitung|pw}}«\orgindex{Pariser Büro der Frankfurter Zeitung@Pariser Büro der Frankfurter Zeitung|pwv}.}{\lemma{\textnormal{\emph{Adresse: … Francfort«.}}}\Cendnote{\textnormal{kopfüber am oberen
                     Rand}}}\label{T_L02673-1}\pend
           \selectlanguage{ngerman}\endnumbering\briefempfaengerindex{Schnitzler, Arthur@\textsc{Schnitzler, Arthur}!zzzGoldmann, Paul@\emph{von Paul Goldmann}!1891-12-031@{3. 12. 1891}|)be}\mylabel{L02673h}  \newcommand{\dateiname}{L02673}\newcommand{\titel}{Paul Goldmann an Arthur Schnitzler, 3. 12. 1891}\newcommand{\editorInnen}{Martin Anton Müller und Laura Untner}%% latex-leseansicht-abspann.tex
%% Abspann für die Leseansicht.
%% Der Schalter \ifkorrekturansicht ist bereits durch den Vorspann gesetzt.

%% latex-abspann.tex
%% Gemeinsamer Abspann für Korrekturansicht und Leseansicht.
%% Setzt den Schalter \ifkorrekturansicht voraus (gesetzt in den
%% einbindenden Dateien latex-korrekturansicht-abspann.tex bzw.
%% latex-leseansicht-abspann.tex).
%% ---------------------------------------------------------------

\normalsize

% Das esempio-Environment wird nur in der Leseansicht benötigt
\ifkorrekturansicht\else
\newenvironment{esempio}[3]%
{
    \vspace{1.5ex}
    \rlap{\underline{#1}}
    \par
    \setlength{\parindent}{0cm}
    \nopagebreak
    \leftskip=#2cm
    \rightskip=#3cm
}
{
    \par
}
\fi

\doendnotes{C}
\bigskip
\vfill

\clearpage

\footnotesize

\ifkorrekturansicht
  \lohead{\textsc{register}}
\fi

% theindex-Environment neu definieren ohne reledmac
\makeatletter
\renewenvironment{theindex}{%
  \ifkorrekturansicht
    \section*{\indexname}%
  \else
    \subsubsection*{Index der erwähnten Entitäten}%
  \fi
  \setlength{\parindent}{0pt}%
  \setlength{\parskip}{0pt plus 0.3pt}%
  \let\item\@idxitem
}{%
  \ifkorrekturansicht\clearpage\fi
}
\makeatother

\IfFileExists{\jobname-pw.ind}{\input{\jobname-pw.ind}}{}

% Quellenangabe nur in der Leseansicht
\ifkorrekturansicht\else
% Fallback-Definitionen, falls die .tex-Datei \titel etc. nicht gesetzt hat
\providecommand{\titel}{}
\providecommand{\editorInnen}{}
\providecommand{\dateiname}{\jobname}

\vspace{3cm}

\vfill

\footnotesize
\textsc{Quelle}: \titel. Herausgegeben von {\editorInnen}. In: \emph{Arthur Schnitzler: Briefwechsel mit Autorinnen und Autoren}.
 Digitale Edition, https://schnitzler-briefe.acdh.oeaw.ac.at/{\dateiname}.html (Stand \today)
\fi

\end{document}


