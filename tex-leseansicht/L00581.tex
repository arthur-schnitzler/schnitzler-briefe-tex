%% latex-leseansicht-vorspann.tex
%% Vorspann für die Leseansicht.
%% Lädt die gemeinsame Datei latex-vorspann.tex mit nicht gesetztem Schalter.

\newif\ifkorrekturansicht
\korrekturansichtfalse

\input{../tex-inputs/latex-vorspann}


         
         \newcommand{\erwaehntePersonen}{Personen: Ernst Pinkert}
         \newcommand{\erwaehnteInstitutionen}{}
         \newcommand{\erwaehnteOrte}{Orte: Baden bei Wien, Frankgasse, IX., Alsergrund, Leipzig, Wien, Zoo}
         \newcommand{\erwaehnteWerke}{Werke: Unter dem Doppel-Adler}
               \section[Richard Beer-Hofmann an Arthur Schnitzler, 30. 8. 1896]{ Richard Beer-Hofmann an Arthur Schnitzler, 30. 8. 1896}\nopagebreak\mylabel{v}\rehead{ }\begin{ledgroupsized}[t]{13cm}\normalsize\beginnumbering \toendnotes[C]{\smallbreak\pagebreak[2]} \Standort{CUL, Schnitzler, B 8.}
\physDesc{Bildpostkarte
\newline{}Handschrift: Bleistift, lateinische Kurrent\newline{}Versand: 1) Stempel: »\nobreak{}\oindex{Leipzig@\textbf{Leipzig}|pwk}Leipzig, 30. 8. 96, 10–11 N\nobreak{}«.   2) Stempel: »\nobreak{}\oindex{IX., Alsergrund@\textbf{IX., Alsergrund}|pwk}Wien 9/3, 31. 8. 96, 7.N, Bestellt\nobreak{}«. \newline{}Ordnung: mit Bleistift von unbekannter Hand nummeriert: »81« }\pstart{}{\pb}D\textsuperscript{r}
                  Arthur Schnitzler\pend{}\pstart{}Wien\oindex{Wien@\textbf{Wien}|pw}\pend{}\pstart{}IX. Frankgasse 1\oindex{Frankgasse@\textbf{Frankgasse}|pw}\pend{}{\bigskip}\pstart
           \noindent{}\centering{}\textcolor{gray}{\textbf{{\pb}Gruss aus dem Zoolog. Garten\oindex{Zoo@\textbf{Zoo}|pw} in Leipzig\oindex{Leipzig@\textbf{Leipzig}|pw}, d.}}{ }30/VIII 96\pend
           \pstart
           \noindent{}\centering{}\textcolor{gray}{\textbf{Besitzer E.
                        Pinkert\pwindex{Pinkert, Ernst 5.2.1844 – 28.4.1909@\textsc{Pinkert, Ernst} (5.2.1844 – 28.4.1909), \emph{Gastwirt, Zoodirektor}|pw}}}\pend
           \pstart
           \noindent{}\centering{}\textcolor{gray}{\textbf{Bären-Zwinger.}}\hspace*{1.5em}\textcolor{gray}{\textbf{Raubthierhaus.}}\hspace*{1.5em}\textcolor{gray}{\textbf{Antilopen-Haus.}}\hspace*{2em}\textcolor{gray}{\textbf{Teich m. Büffel u. Kameel-Haus.}}\pend
           \pstart
           Lieber! Da man den »Doppeladler\pwindex{\textcolor{red}{\textsuperscript{XXXX1 indx}}!Unter dem Doppel-Adler1893@\strich\emph{Unter dem Doppel-Adler} {[}1893{]}|pw}«
               spielt \uline{muß} ich doch Ihnen schreiben. – Ich bin
                  Donnerstag in Baden\oindex{Baden bei Wien@\textbf{Baden bei Wien}|pw}.\pend
           \pstart
           Herzlichst{\\[\baselineskip]}Ihr \spacefill\mbox{Richard}\pend
           \leftskip=0em{}
         
         \endnumbering\mylabel{h}\end{ledgroupsized}  \newcommand{\dateiname}{L00581}\newcommand{\titel}{Richard Beer-Hofmann an Arthur Schnitzler, 30. 8. 1896}\newcommand{\editorInnen}{Martin Anton Müller und Gerd-Hermann Susen}%% latex-leseansicht-abspann.tex
%% Abspann für die Leseansicht.
%% Der Schalter \ifkorrekturansicht ist bereits durch den Vorspann gesetzt.

%% latex-abspann.tex
%% Gemeinsamer Abspann für Korrekturansicht und Leseansicht.
%% Setzt den Schalter \ifkorrekturansicht voraus (gesetzt in den
%% einbindenden Dateien latex-korrekturansicht-abspann.tex bzw.
%% latex-leseansicht-abspann.tex).
%% ---------------------------------------------------------------

\normalsize

% Das esempio-Environment wird nur in der Leseansicht benötigt
\ifkorrekturansicht\else
\newenvironment{esempio}[3]%
{
    \vspace{1.5ex}
    \rlap{\underline{#1}}
    \par
    \setlength{\parindent}{0cm}
    \nopagebreak
    \leftskip=#2cm
    \rightskip=#3cm
}
{
    \par
}
\fi

\doendnotes{C}
\bigskip
\vfill

\clearpage

\footnotesize

\ifkorrekturansicht
  \lohead{\textsc{register}}
\fi

% theindex-Environment neu definieren ohne reledmac
\makeatletter
\renewenvironment{theindex}{%
  \ifkorrekturansicht
    \section*{\indexname}%
  \else
    \subsubsection*{Index der erwähnten Entitäten}%
  \fi
  \setlength{\parindent}{0pt}%
  \setlength{\parskip}{0pt plus 0.3pt}%
  \let\item\@idxitem
}{%
  \ifkorrekturansicht\clearpage\fi
}
\makeatother

\IfFileExists{\jobname-pw.ind}{\input{\jobname-pw.ind}}{}

% Quellenangabe nur in der Leseansicht
\ifkorrekturansicht\else
% Fallback-Definitionen, falls die .tex-Datei \titel etc. nicht gesetzt hat
\providecommand{\titel}{}
\providecommand{\editorInnen}{}
\providecommand{\dateiname}{\jobname}

\vspace{3cm}

\vfill

\footnotesize
\textsc{Quelle}: \titel. Herausgegeben von {\editorInnen}. In: \emph{Arthur Schnitzler: Briefwechsel mit Autorinnen und Autoren}.
 Digitale Edition, https://schnitzler-briefe.acdh.oeaw.ac.at/{\dateiname}.html (Stand \today)
\fi

\end{document}


      