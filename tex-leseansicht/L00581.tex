%% latex-leseansicht-vorspann.tex
%% Vorspann für die Leseansicht.
%% Lädt die gemeinsame Datei latex-vorspann.tex mit nicht gesetztem Schalter.

\newif\ifkorrekturansicht
\korrekturansichtfalse

\input{../tex-inputs/latex-vorspann}


\section[Richard Beer-Hofmann an Arthur Schnitzler, 30. 8. 1896]{L00581 Richard Beer-Hofmann an Arthur Schnitzler, 30. 8. 1896}
\nopagebreak\mylabel{L00581v}
\rehead{ }\normalsize\beginnumbering\briefempfaengerindex{Schnitzler, Arthur@\textsc{Schnitzler, Arthur}!zzzBeer-Hofmann, Richard@\emph{von Richard Beer-Hofmann}!1896-08-301@{30. 8. 1896}|(be}
\toendnotes[C]{\smallbreak\pagebreak[2]}
\correspDesc{Versand  durch Richard Beer-Hofmann am 30. 8. 1896 in Leipzig
\newline{}Erhalt  durch Arthur Schnitzler am 31. 8. 1896 in Wien}\toendnotes[C]{\smallbreak}
\Standort{CUL, Schnitzler, B 8.}
\physDesc{Bildpostkarte, 167 Zeichen
\newline{}Handschrift: Bleistift, lateinische Kurrent
\newline{}Versand: 1) Stempel: »\nobreak{}\oindex{Leipzig@\textbf{Leipzig}, \emph{Hauptstadt}|pwk}Leipzig, 30. 8. 96, 10–11 N\nobreak{}«.   2) Stempel: »\nobreak{}\oindex{IX., Alsergrund@\textbf{IX., Alsergrund}, \emph{Verwaltungsgebiet}|pwk}Wien 9/3, 31. 8. 96, 7.N, Bestellt\nobreak{}«. 
\newline{}Ordnung: mit Bleistift von unbekannter Hand nummeriert:
                                    »81« }\pstart{}{\pb}D\textsuperscript{r}
                  Arthur Schnitzler\pend{}\pstart{}Wien\oindex{Wien@\textbf{Wien}, \emph{Verwaltungsgebiet}|pw}\pend{}\pstart{}IX. Frankgasse 1\oindex{Wien@\textbf{Wien}!IX., Alsergrund@\textbf{IX., Alsergrund}!Frankgasse 1@\textbf{Frankgasse 1}, \emph{Wohngebäude}|pw}\pend{}{\bigskip}
\pstart
           \noindent{}\centering{}{\pb}\textcolor{gray}{\textbf{Gruss aus dem Zoolog. Garten\oindex{Leipziger Zoo@\textbf{Leipziger Zoo}, \emph{Zoo}|pw} in Leipzig\oindex{Leipzig@\textbf{Leipzig}, \emph{Hauptstadt}|pw}, d.}}\pend
           
\pstart
           \centering{}\textcolor{gray}{\textbf{Besitzer E. Pinkert\pwindex{Pinkert, Ernst 5.\,2.\,1844 Hirschfelde – 28.\,4.\,1909 Leipzig@\textsc{Pinkert, Ernst} (5.\,2.\,1844 Hirschfelde – 28.\,4.\,1909 Leipzig), \emph{Gastwirt, Zoodirektor}|pw}}}\pend
           
\pstart
           \centering{}\textcolor{gray}{\textbf{Bären-Zwinger.}}\hspace*{1.5em}\textcolor{gray}{\textbf{Raubthierhaus.}}\hspace*{1.5em}\textcolor{gray}{\textbf{Antilopen-Haus.}}\hspace*{2em}\textcolor{gray}{\textbf{Teich m. Büffel u. Kameel-Haus.}}\pend
           \vspace{1em}
\pstart
           \raggedleft{}{\pb}30/VIII 96\pend
           \vspace{0.5em}
\pstart
           Lieber! Da man den »Doppeladler\pwindex{\textcolor{red}{\textsuperscript{XXXX indx1}}!Unter dem Doppel-Adler@\strich\emph{Unter dem Doppel-Adler}|pw}« spielt \uline{muß} ich doch Ihnen
               schreiben. – Ich bin Donnerstag in Baden\oindex{Baden bei Wien@\textbf{Baden bei Wien}, \emph{Hauptstadt}|pw}.\pend
           
\pstart
           Herzlichst{\\[\baselineskip]}Ihr \spacefill\mbox{Richard}\pend
           \leftskip=0em{}\selectlanguage{ngerman}\endnumbering\briefempfaengerindex{Schnitzler, Arthur@\textsc{Schnitzler, Arthur}!zzzBeer-Hofmann, Richard@\emph{von Richard Beer-Hofmann}!1896-08-301@{30. 8. 1896}|)be}\mylabel{L00581h}  \newcommand{\dateiname}{L00581}\newcommand{\titel}{Richard Beer-Hofmann an Arthur Schnitzler, 30. 8. 1896}\newcommand{\editorInnen}{Martin Anton Müller und Gerd-Hermann Susen}%% latex-leseansicht-abspann.tex
%% Abspann für die Leseansicht.
%% Der Schalter \ifkorrekturansicht ist bereits durch den Vorspann gesetzt.

%% latex-abspann.tex
%% Gemeinsamer Abspann für Korrekturansicht und Leseansicht.
%% Setzt den Schalter \ifkorrekturansicht voraus (gesetzt in den
%% einbindenden Dateien latex-korrekturansicht-abspann.tex bzw.
%% latex-leseansicht-abspann.tex).
%% ---------------------------------------------------------------

\normalsize

% Das esempio-Environment wird nur in der Leseansicht benötigt
\ifkorrekturansicht\else
\newenvironment{esempio}[3]%
{
    \vspace{1.5ex}
    \rlap{\underline{#1}}
    \par
    \setlength{\parindent}{0cm}
    \nopagebreak
    \leftskip=#2cm
    \rightskip=#3cm
}
{
    \par
}
\fi

\doendnotes{C}
\bigskip
\vfill

\clearpage

\footnotesize

\ifkorrekturansicht
  \lohead{\textsc{register}}
\fi

% theindex-Environment neu definieren ohne reledmac
\makeatletter
\renewenvironment{theindex}{%
  \ifkorrekturansicht
    \section*{\indexname}%
  \else
    \subsubsection*{Index der erwähnten Entitäten}%
  \fi
  \setlength{\parindent}{0pt}%
  \setlength{\parskip}{0pt plus 0.3pt}%
  \let\item\@idxitem
}{%
  \ifkorrekturansicht\clearpage\fi
}
\makeatother

\IfFileExists{\jobname-pw.ind}{\input{\jobname-pw.ind}}{}

% Quellenangabe nur in der Leseansicht
\ifkorrekturansicht\else
% Fallback-Definitionen, falls die .tex-Datei \titel etc. nicht gesetzt hat
\providecommand{\titel}{}
\providecommand{\editorInnen}{}
\providecommand{\dateiname}{\jobname}

\vspace{3cm}

\vfill

\footnotesize
\textsc{Quelle}: \titel. Herausgegeben von {\editorInnen}. In: \emph{Arthur Schnitzler: Briefwechsel mit Autorinnen und Autoren}.
 Digitale Edition, https://schnitzler-briefe.acdh.oeaw.ac.at/{\dateiname}.html (Stand \today)
\fi

\end{document}


