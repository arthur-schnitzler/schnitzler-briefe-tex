%% latex-korrekturansicht-vorspann.tex
%% Vorspann für die Korrekturansicht.
%% Lädt die gemeinsame Datei latex-vorspann.tex mit gesetztem Schalter.

\newif\ifkorrekturansicht
\korrekturansichttrue

\input{../tex-inputs/latex-vorspann}


\section[Hermann Bahr an Arthur Schnitzler, {[}7. 11. 1893?{]}]{L00281 Hermann Bahr an Arthur Schnitzler, {[}7. 11. 1893?{]}}
\nopagebreak\mylabel{L00281v}
\rehead{ }\normalsize\beginnumbering\briefempfaengerindex{Schnitzler, Arthur@\textsc{Schnitzler, Arthur}!zzzBahr, Hermann@\emph{von Hermann Bahr}!1893-11-072@{{[}7. 11. 1893?{]}}|(be}
\toendnotes[C]{\smallbreak\pagebreak[2]}\Standort{CUL, Schnitzler, B 5b.}
\physDesc{Visitenkarte, 64 Zeichen
\newline{}Handschrift: schwarze Tinte, deutsche Kurrent
\newline{}Schnitzler: 1) mit Bleistift datiert: »92?«  2) mit rotem Buntstift nummeriert: »3« 3) mit Bleistift nummeriert: »3«}
\buchAbdrucke{\weitereDrucke{Hermann Bahr, Arthur Schnitzler: \emph{Briefwechsel, Aufzeichnungen, Dokumente (1891–1931)}. Göttingen: \emph{Wallstein} 2018, S. 47.} }\toendnotes[C]{\smallbreak}
\pstart
           \noindent{}\centering{}\textcolor{gray}{\textbf{{\pb}Hermann Bahr}}\pend
           
\pstart
           {\pb}Herzlichen Dank, lieber Freund, für Ihre große
               Güte.\pend
           
\pstart
           Ihr treuer{\\[\baselineskip]}\spacefill\mbox{\label{K_L00281-1v}\edtext{hr}{\lemma{\textnormal{\emph{hr}}}\Cendnote{\textnormal{undatiert. Gegen das »92?« von Schnitzler{ }spricht, dass 1892 eine andere
                     Visitenkarte im Einsatz war. Hier wird die Karte in die Nähe des anderen Briefes mit
                     vergleichbarer Unterschrift (Hermann Bahr an Arthur Schnitzler, 20. 9. 1893)
                     an die wahrscheinlichste Stelle eingeordnet.}}}\label{K_L00281-1}}\pend
           \leftskip=0em{}\selectlanguage{ngerman}\endnumbering\briefempfaengerindex{Schnitzler, Arthur@\textsc{Schnitzler, Arthur}!zzzBahr, Hermann@\emph{von Hermann Bahr}!1893-11-072@{{[}7. 11. 1893?{]}}|)be}\mylabel{L00281h}  \normalsize

\doendnotes{C}
\bigskip
\vfill

\clearpage

\footnotesize

\lohead{\textsc{register}}

% Definiere theindex-Environment komplett neu ohne reledmac
\makeatletter
\renewenvironment{theindex}{%
  \section*{\indexname}%
  \setlength{\parindent}{0pt}%
  \setlength{\parskip}{0pt plus 0.3pt}%
  \let\item\@idxitem
}{%
  \clearpage
}
\makeatother

\IfFileExists{\jobname-pw.ind}{\input{\jobname-pw.ind}}{}

\end{document}

      