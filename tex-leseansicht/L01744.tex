%% latex-leseansicht-vorspann.tex
%% Vorspann für die Leseansicht.
%% Lädt die gemeinsame Datei latex-vorspann.tex mit nicht gesetztem Schalter.

\newif\ifkorrekturansicht
\korrekturansichtfalse

\input{../tex-inputs/latex-vorspann}


\section[Hermann Bahr an Arthur Schnitzler, 23. 12. 1907]{L01744 Hermann Bahr an Arthur Schnitzler, 23. 12. 1907}
\nopagebreak\mylabel{L01744v}
\rehead{ }\normalsize\beginnumbering\briefempfaengerindex{Schnitzler, Arthur@\textsc{Schnitzler, Arthur}!zzzBahr, Hermann@\emph{von Hermann Bahr}!1907-12-231@{23. 12. 1907}|(be}
\toendnotes[C]{\smallbreak\pagebreak[2]}
\correspDesc{Versand  durch Hermann Bahr am 23. 12. 1907 in Wien
\newline{}Erhalt  durch Arthur Schnitzler im Zeitraum [23. 12. 1907 – 27. 12. 1907?] in Wien}\toendnotes[C]{\smallbreak}
\Standort{CUL, Schnitzler, B 5b.}
\physDesc{Brief, 1 Blatt, 4 Seiten, 1625 Zeichen
\newline{}Handschrift: blaue Tinte, deutsche Kurrent
\newline{}Schnitzler: 1) mit Bleistift beschriftet: »Bahr«  2) mit rotem Buntstift vereinzelte Unterstreichungen
\newline{}Ordnung: mit Bleistift von unbekannter Hand nummeriert:
                                    »153« }
\buchAbdrucke{\weitereDrucke{Hermann Bahr, Arthur Schnitzler: \emph{Briefwechsel, Aufzeichnungen, Dokumente (1891–1931)}. Herausgegeben von Kurt Ifkovits und Martin Anton Müller. Göttingen: \emph{Wallstein} 2018, S. 400.} }\toendnotes[C]{\smallbreak}
\pstart
           \raggedleft{}{\pb}23. 12. 07\pend
           
\pstart\center{}Lieber Arthur!\pend\vspace{0.5em}
\pstart
           Danke{ }ſchön für Deinen Brief. Ich möchte nicht, daß Du falſch deuteſt, was ich über
                  Reinhardts\pwindex{Reinhardt, Max 9.\,9.\,1873 Baden bei Wien – 30.\,10.\,1943 New York City@\textsc{Reinhardt, Max} (9.\,9.\,1873 Baden bei Wien – 30.\,10.\,1943 New York City), \emph{Theaterleiter, Regisseur, Schauspieler}|pw} Verhältnis zu Deinen Werken{ }ſchrieb. Er bemüht{ }ſich{ }ſehr, ihnen gerecht zu{ }ſein, aber ich habe immer das Gefühl,
               daß ihm das innere Verſtehen dafür fehlt; und es ist{ }ſchon{ }ſehr bös, wenn einer{ }ſich
               erſt bemühen muß. Aber am guten Willen fehlts ihm{ }ſicher nicht. Nur daß dieſer dabei
               leider{ }ſchließlich gar nichts nützt. – Der Ritſcher\pwindex{Ritscher, Helene 2.\,6.\,1888 Zalaegerszeg – 27.\,11.\,1964 Hamburg@\textsc{Ritscher, Helene} (2.\,6.\,1888 Zalaegerszeg – 27.\,11.\,1964 Hamburg), \emph{Schauspielerin}|pw} müßte geſagt werden, daß{ }ſie Anfang Mai oder im September hier{ }ſein{ }ſoll. Die Mildenburg\pwindex{Bahr-Mildenburg, Anna 29.\,11.\,1872 Wien – 27.\,1.\,1947 ebd.@\textsc{Bahr-Mildenburg, Anna} (29.\,11.\,1872 Wien – 27.\,1.\,1947 ebd.), \emph{Sängerin}|pw}{ }{\pb}hat eine merkwürdige Macht über{ }ſie,{ }ſodaß{ }ſie
               nicht blos aus ihr heraus holen,{ }ſondern{ }ſogar bis zu einem gewiſſen Grad in{ }ſie
               hinein pumpen kann. Ihr würde ich das Darſtelleriſche ganz überlaſſen, ohne{ }ſelbſt
               dreinzureden; bei zweien kommt nichts heraus. Ich aber würde mit großer Paſſion den
                  Strakoſch\pwindex{Strakosch, Alexander 3.\,12.\,1846 Šarišské Lúky – 16.\,9.\,1909 Berlin@\textsc{Strakosch, Alexander} (3.\,12.\,1846 Šarišské Lúky – 16.\,9.\,1909 Berlin), \emph{Schauspieler, Sprachlehrer, Vortragslehrer}|pw} machen und dem Mädel den Rhythmus
               der Verſe ein\substVorne{}\textsuperscript{t}\substDazwischen{}b\substHinten{}läuen, wovon ich aus Erfahrung weiß, daß ichs kann. Wenn es{ }ſchließlich
               trotzdem{ }ſcheußlich wird, können wir nichts {\pb}dafür.
                  \uline{Garantieren} könnte ich für die Höflich\pwindex{Höflich, Lucie 20.\,2.\,1883 Hannover – 8.\,10.\,1956 Schmargendorf@\textsc{Höflich, Lucie} (20.\,2.\,1883 Hannover – 8.\,10.\,1956 Schmargendorf), \emph{Schauspielerin}|pw} ja auch nicht, die freilich einen vagen Schimmer von
               Seele oder Poeſie oder wie man das nennt für die Rolle hätte, den das Chaotiſche, das
               die Ritſcher\pwindex{Ritscher, Helene 2.\,6.\,1888 Zalaegerszeg – 27.\,11.\,1964 Hamburg@\textsc{Ritscher, Helene} (2.\,6.\,1888 Zalaegerszeg – 27.\,11.\,1964 Hamburg), \emph{Schauspielerin}|pw}{ }ſehr{ }ſtark hat, vielleicht nicht völlig erſetzen
               kann.\pend
           
\pstart
           Ich{ }ſelbſt habe vor Anſteckungen gar keine Furcht, muß aber auf meine \label{K_L01744-1v}\edtext{Frauen\pwindex{Bahr, Rosa 26.\,10.\,1871 Prag – 17.\,2.\,1940 Berlin@\textsc{Bahr, Rosa} (26.\,10.\,1871 Prag – 17.\,2.\,1940 Berlin), \emph{Schauspielerin}|pwuv}\pwindex{Roth, Eugenie von 1877 Levoča – 1971 Perchtoldsdorf@\textsc{Roth, Eugenie von} (1877 Levoča – 1971 Perchtoldsdorf), \emph{Haushälterin, Gesellschafterin}|pwuv}\pwindex{Bahr-Mildenburg, Anna 29.\,11.\,1872 Wien – 27.\,1.\,1947 ebd.@\textsc{Bahr-Mildenburg, Anna} (29.\,11.\,1872 Wien – 27.\,1.\,1947 ebd.), \emph{Sängerin}|pwv}}{\lemma{\textnormal{\emph{Frauen}}}\Cendnote{\textnormal{Gemeint ist in jedem Fall seine
                  Partnerin Anna von Mildenburg\pwindex{Bahr-Mildenburg, Anna 29.\,11.\,1872 Wien – 27.\,1.\,1947 ebd.@\textsc{Bahr-Mildenburg, Anna} (29.\,11.\,1872 Wien – 27.\,1.\,1947 ebd.), \emph{Sängerin}|pwk}, eventuell
                  mit ihrer Gesellschafterin Eugenie Roth\pwindex{Roth, Eugenie von 1877 Levoča – 1971 Perchtoldsdorf@\textsc{Roth, Eugenie von} (1877 Levoča – 1971 Perchtoldsdorf), \emph{Haushälterin, Gesellschafterin}|pwk}.
                  Vielleicht inkludiert er auch seine erste Frau, Rosa\pwindex{Bahr, Rosa 26.\,10.\,1871 Prag – 17.\,2.\,1940 Berlin@\textsc{Bahr, Rosa} (26.\,10.\,1871 Prag – 17.\,2.\,1940 Berlin), \emph{Schauspielerin}|pwk}, mit der er noch verheiratet war.}}}\label{K_L01744-1} Rückſicht nehmen, hoffe
               jedoch, da ich früheſtens erſt am 15. Januar zu Reinhardt\pwindex{Reinhardt, Max 9.\,9.\,1873 Baden bei Wien – 30.\,10.\,1943 New York City@\textsc{Reinhardt, Max} (9.\,9.\,1873 Baden bei Wien – 30.\,10.\,1943 New York City), \emph{Theaterleiter, Regisseur, Schauspieler}|pw} zurückkehre, daß Deine {\pb}liebe Frau\pwindex{Schnitzler, Olga 17.\,1.\,1882 Wien – 13.\,1.\,1970 Lugano@\textsc{Schnitzler, Olga} (17.\,1.\,1882 Wien – 13.\,1.\,1970 Lugano), \emph{Schauspielerin, Sängerin}|pwv}, der ich das Allerbeſte wünſche, \substVorne{}\textsuperscript{\textcolor{gray}{ſ}}\substDazwischen{}n\substHinten{}och vorher{ }ſo weit \strikeout{ſ\textcolor{gray}{e}\textcolor{gray}{×}h\textcolor{gray}{×}}{ }ſein wird, daß ich zu Euch kann, was ich Dich
               bitte, mich gleich wiſſen zu laſſen.\pend
           
\pstart
           Herzlichſt{\\[\baselineskip]}mit den wärmſten Weihnachtswünſchen{\\[\baselineskip]}Dein{\\[\baselineskip]}\spacefill\mbox{H}\pend
           \leftskip=0em{}\selectlanguage{ngerman}\endnumbering\briefempfaengerindex{Schnitzler, Arthur@\textsc{Schnitzler, Arthur}!zzzBahr, Hermann@\emph{von Hermann Bahr}!1907-12-231@{23. 12. 1907}|)be}\mylabel{L01744h}  \newcommand{\dateiname}{L01744}\newcommand{\titel}{Hermann Bahr an Arthur Schnitzler, 23. 12. 1907}\newcommand{\editorInnen}{Herausgegeben von Martin Anton Müller}%% latex-leseansicht-abspann.tex
%% Abspann für die Leseansicht.
%% Der Schalter \ifkorrekturansicht ist bereits durch den Vorspann gesetzt.

%% latex-abspann.tex
%% Gemeinsamer Abspann für Korrekturansicht und Leseansicht.
%% Setzt den Schalter \ifkorrekturansicht voraus (gesetzt in den
%% einbindenden Dateien latex-korrekturansicht-abspann.tex bzw.
%% latex-leseansicht-abspann.tex).
%% ---------------------------------------------------------------

\normalsize

% Das esempio-Environment wird nur in der Leseansicht benötigt
\ifkorrekturansicht\else
\newenvironment{esempio}[3]%
{
    \vspace{1.5ex}
    \rlap{\underline{#1}}
    \par
    \setlength{\parindent}{0cm}
    \nopagebreak
    \leftskip=#2cm
    \rightskip=#3cm
}
{
    \par
}
\fi

\doendnotes{C}
\bigskip
\vfill

\clearpage

\footnotesize

\ifkorrekturansicht
  \lohead{\textsc{register}}
\fi

% theindex-Environment neu definieren ohne reledmac
\makeatletter
\renewenvironment{theindex}{%
  \ifkorrekturansicht
    \section*{\indexname}%
  \else
    \subsubsection*{Index der erwähnten Entitäten}%
  \fi
  \setlength{\parindent}{0pt}%
  \setlength{\parskip}{0pt plus 0.3pt}%
  \let\item\@idxitem
}{%
  \ifkorrekturansicht\clearpage\fi
}
\makeatother

\IfFileExists{\jobname-pw.ind}{\input{\jobname-pw.ind}}{}

% Quellenangabe nur in der Leseansicht
\ifkorrekturansicht\else
% Fallback-Definitionen, falls die .tex-Datei \titel etc. nicht gesetzt hat
\providecommand{\titel}{}
\providecommand{\editorInnen}{}
\providecommand{\dateiname}{\jobname}

\vspace{3cm}

\vfill

\footnotesize
\textsc{Quelle}: \titel. Herausgegeben von {\editorInnen}. In: \emph{Arthur Schnitzler: Briefwechsel mit Autorinnen und Autoren}.
 Digitale Edition, https://schnitzler-briefe.acdh.oeaw.ac.at/{\dateiname}.html (Stand \today)
\fi

\end{document}


