%% latex-leseansicht-vorspann.tex
%% Vorspann für die Leseansicht.
%% Lädt die gemeinsame Datei latex-vorspann.tex mit nicht gesetztem Schalter.

\newif\ifkorrekturansicht
\korrekturansichtfalse

\input{../tex-inputs/latex-vorspann}


         
         \renewcommand{\erwaehntePersonen}{Personen: Rosa Bahr, Anna Bahr-Mildenburg, Lucie Höflich, Max Reinhardt, Helene Ritscher, Eugenie von Roth, Olga Schnitzler, Alexander Strakosch}
         \renewcommand{\erwaehnteOrte}{Orte: Wien}
         \renewcommand{\erwaehnteWerke}{}
               \section[Hermann Bahr an Arthur Schnitzler, 23. 12. 1907]{ Hermann Bahr an Arthur Schnitzler, 23. 12. 1907}\nopagebreak\mylabel{v}\rehead{ }\begin{ledgroupsized}[t]{13cm}\normalsize\beginnumbering \toendnotes[C]{\smallbreak\pagebreak[2]} \Standort{CUL, Schnitzler, B 5b.}
\physDesc{Brief, 1 Blatt, 4 Seiten, 1625 Zeichen
\newline{}Handschrift: blaue Tinte, deutsche Kurrent
\newline{}Schnitzler: 1) mit Bleistift beschriftet: »Bahr«  2) mit rotem Buntstift vereinzelte Unterstreichungen
\newline{}Ordnung: mit Bleistift von unbekannter Hand nummeriert:
                                    »153« }\buchAbdrucke{\weitereDrucke{Hermann Bahr, Arthur Schnitzler: \emph{Briefwechsel, Aufzeichnungen, Dokumente (1891–1931)}. Hg. Kurt Ifkovits und Martin Anton Müller. Göttingen: \emph{Wallstein} 2018, S. 400.} }\toendnotes[C]{\smallbreak}\pstart
           \raggedleft{}{\pb}23. 12. 07\pend
           \pstart\center{}Lieber Arthur!\pend\pstart
           Danke ſchön für Deinen Brief. Ich möchte nicht, daß Du falſch deuteſt, was ich über
                  Reinhardts\pwindex{Reinhardt, Max 09.09.1873 – 30.10.1943@\textsc{Reinhardt, Max} (09.09.1873 – 30.10.1943), \emph{Theaterleiter, Regisseur, Schauspieler}|pw} Verhältnis zu Deinen Werken
               ſchrieb. Er bemüht ſich ſehr, ihnen gerecht zu ſein, aber ich habe immer das Gefühl,
               daß ihm das innere Verſtehen dafür fehlt; und es ist ſchon ſehr bös, wenn einer ſich
               erſt bemühen muß. Aber am guten Willen fehlts ihm ſicher nicht. Nur daß dieſer dabei
               leider ſchließlich gar nichts nützt. – Der Ritſcher\pwindex{Ritscher, Helene 02.06.1888 – 1964-11-27@\textsc{Ritscher, Helene} (02.06.1888 – 1964-11-27), \emph{Schauspielerin}|pw} müßte geſagt werden, daß ſie Anfang Mai oder im September hier ſein
               ſoll. Die Mildenburg\pwindex{Bahr-Mildenburg, Anna 29.11.1872 – 27.01.1947@\textsc{Bahr-Mildenburg, Anna} (29.11.1872 – 27.01.1947), \emph{Sängerin}|pw}{ }{\pb}hat eine merkwürdige Macht über ſie, ſodaß ſie
               nicht blos aus ihr heraus holen, ſondern ſogar bis zu einem gewiſſen Grad in ſie
               hinein pumpen kann. Ihr würde ich das Darſtelleriſche ganz überlaſſen, ohne ſelbſt
               dreinzureden; bei zweien kommt nichts heraus. Ich aber würde mit großer Paſſion den
                  Strakoſch\pwindex{Strakosch, Alexander 03.12.1846 – 16.09.1909@\textsc{Strakosch, Alexander} (03.12.1846 – 16.09.1909), \emph{Schauspieler, Sprachlehrer, Vortragslehrer}|pw} machen und dem Mädel den Rhythmus
               der Verſe ein\substVorne{}\textsuperscript{t}\substDazwischen{}b\substHinten{}läuen, wovon ich aus Erfahrung weiß, daß ichs kann. Wenn es ſchließlich
               trotzdem ſcheußlich wird, können wir nichts {\pb}dafür.
                  \uline{Garantieren} könnte ich für die Höflich\pwindex{Hoeflich, Lucie 20.02.1883 – 08.10.1956@\textsc{Höflich, Lucie} (20.02.1883 – 08.10.1956), \emph{Schauspielerin}|pw} ja auch nicht, die freilich einen vagen Schimmer von
               Seele oder Poeſie oder wie man das nennt für die Rolle hätte, den das Chaotiſche, das
               die Ritſcher\pwindex{Ritscher, Helene 02.06.1888 – 1964-11-27@\textsc{Ritscher, Helene} (02.06.1888 – 1964-11-27), \emph{Schauspielerin}|pw}{ }ſehr ſtark hat, vielleicht nicht völlig erſetzen
               kann.\pend
           \pstart
           Ich ſelbſt habe vor Anſteckungen gar keine Furcht, muß aber auf meine \label{K_L01744-1v}\edtext{Frauen\pwindex{Bahr, Rosa 26.10.1871 – 17.02.1940@\textsc{Bahr, Rosa} (26.10.1871 – 17.02.1940), \emph{Schauspielerin}|pwuv}\pwindex{Roth, Eugenie von 1877 – 1971@\textsc{Roth, Eugenie von} (1877 – 1971), \emph{Haushälterin, Gesellschafterin}|pwuv}\pwindex{Bahr-Mildenburg, Anna 29.11.1872 – 27.01.1947@\textsc{Bahr-Mildenburg, Anna} (29.11.1872 – 27.01.1947), \emph{Sängerin}|pwv}}{\lemma{\textnormal{\emph{Frauen}}}\Cendnote{\textnormal{Gemeint ist in jedem Fall seine
                  Partnerin Anna von Mildenburg\pwindex{Bahr-Mildenburg, Anna 29.11.1872 – 27.01.1947@\textsc{Bahr-Mildenburg, Anna} (29.11.1872 – 27.01.1947), \emph{Sängerin}|pwk}, eventuell
                  mit ihrer Gesellschafterin Eugenie Roth\pwindex{Roth, Eugenie von 1877 – 1971@\textsc{Roth, Eugenie von} (1877 – 1971), \emph{Haushälterin, Gesellschafterin}|pwk}.
                  Vielleicht inkludiert er auch seine erste Frau, Rosa\pwindex{Bahr, Rosa 26.10.1871 – 17.02.1940@\textsc{Bahr, Rosa} (26.10.1871 – 17.02.1940), \emph{Schauspielerin}|pwk}, mit der er noch verheiratet war.}}}\label{K_L01744-1h} Rückſicht nehmen, hoffe
               jedoch, da ich früheſtens erſt am 15. Januar zu Reinhardt\pwindex{Reinhardt, Max 09.09.1873 – 30.10.1943@\textsc{Reinhardt, Max} (09.09.1873 – 30.10.1943), \emph{Theaterleiter, Regisseur, Schauspieler}|pw} zurückkehre, daß Deine {\pb}liebe Frau\pwindex{Schnitzler, Olga 17.01.1882 – 13.01.1970@\textsc{Schnitzler, Olga} (17.01.1882 – 13.01.1970), \emph{Schauspielerin, Sängerin}|pwv}, der ich das Allerbeſte wünſche, \substVorne{}\textsuperscript{\textcolor{gray}{ſ}}\substDazwischen{}n\substHinten{}och vorher ſo weit \strikeout{ſ \textcolor{gray}{e}\textcolor{gray}{×}h\textcolor{gray}{×}}{ }ſein wird, daß ich zu Euch kann, was ich Dich
               bitte, mich gleich wiſſen zu laſſen.\pend
           \pstart
           Herzlichſt{\\[\baselineskip]}mit den wärmſten Weihnachtswünſchen{\\[\baselineskip]}Dein{\\[\baselineskip]}\spacefill\mbox{H}\pend
           \leftskip=0em{}
         
         \endnumbering\mylabel{h}\end{ledgroupsized}  \newcommand{\dateiname}{L01744}\newcommand{\titel}{Hermann Bahr an Arthur Schnitzler, 23. 12. 1907}\newcommand{\editorInnen}{ Kurt Ifkovits,  Martin Anton Müller}%% latex-leseansicht-abspann.tex
%% Abspann für die Leseansicht.
%% Der Schalter \ifkorrekturansicht ist bereits durch den Vorspann gesetzt.

%% latex-abspann.tex
%% Gemeinsamer Abspann für Korrekturansicht und Leseansicht.
%% Setzt den Schalter \ifkorrekturansicht voraus (gesetzt in den
%% einbindenden Dateien latex-korrekturansicht-abspann.tex bzw.
%% latex-leseansicht-abspann.tex).
%% ---------------------------------------------------------------

\normalsize

% Das esempio-Environment wird nur in der Leseansicht benötigt
\ifkorrekturansicht\else
\newenvironment{esempio}[3]%
{
    \vspace{1.5ex}
    \rlap{\underline{#1}}
    \par
    \setlength{\parindent}{0cm}
    \nopagebreak
    \leftskip=#2cm
    \rightskip=#3cm
}
{
    \par
}
\fi

\doendnotes{C}
\bigskip
\vfill

\clearpage

\footnotesize

\ifkorrekturansicht
  \lohead{\textsc{register}}
\fi

% theindex-Environment neu definieren ohne reledmac
\makeatletter
\renewenvironment{theindex}{%
  \ifkorrekturansicht
    \section*{\indexname}%
  \else
    \subsubsection*{Index der erwähnten Entitäten}%
  \fi
  \setlength{\parindent}{0pt}%
  \setlength{\parskip}{0pt plus 0.3pt}%
  \let\item\@idxitem
}{%
  \ifkorrekturansicht\clearpage\fi
}
\makeatother

\IfFileExists{\jobname-pw.ind}{\input{\jobname-pw.ind}}{}

% Quellenangabe nur in der Leseansicht
\ifkorrekturansicht\else
% Fallback-Definitionen, falls die .tex-Datei \titel etc. nicht gesetzt hat
\providecommand{\titel}{}
\providecommand{\editorInnen}{}
\providecommand{\dateiname}{\jobname}

\vspace{3cm}

\vfill

\footnotesize
\textsc{Quelle}: \titel. Herausgegeben von {\editorInnen}. In: \emph{Arthur Schnitzler: Briefwechsel mit Autorinnen und Autoren}.
 Digitale Edition, https://schnitzler-briefe.acdh.oeaw.ac.at/{\dateiname}.html (Stand \today)
\fi

\end{document}


      