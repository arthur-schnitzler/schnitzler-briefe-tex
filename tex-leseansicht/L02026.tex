%% latex-korrekturansicht-vorspann.tex
%% Vorspann für die Korrekturansicht.
%% Lädt die gemeinsame Datei latex-vorspann.tex mit gesetztem Schalter.

\newif\ifkorrekturansicht
\korrekturansichttrue

\input{../tex-inputs/latex-vorspann}


\section[Richard Beer-Hofmann an Arthur Schnitzler, {[}10.? 9. 1911{]}]{L02026 Richard Beer-Hofmann an Arthur Schnitzler, {[}10.? 9. 1911{]}}
\nopagebreak\mylabel{L02026v}
\rehead{ }\normalsize\beginnumbering\briefempfaengerindex{Schnitzler, Arthur@\textsc{Schnitzler, Arthur}!zzzBeer-Hofmann, Richard@\emph{von Richard Beer-Hofmann}!1911-09-101@{{[}10.? 9. 1911{]}}|(be}
\toendnotes[C]{\smallbreak\pagebreak[2]}\Standort{CUL, Schnitzler, B 8.}
\physDesc{Telegramm, 240 Zeichen
\newline{}maschinell
\newline{}Schnitzler: mit Bleistift beschriftet: »BH« 
\newline{}Ordnung: mit Bleistift von unbekannter Hand nummeriert:
                                    »263« }
\buchAbdrucke{\weitereDrucke{Arthur Schnitzler, Richard Beer-Hofmann: \emph{Briefwechsel 1891–1931}. Wien, Zürich: \emph{Europaverlag} 1992, S. 215.} }\toendnotes[C]{\smallbreak}
\pstart
           {\pb}badaussee\oindex{Bad Aussee@\textbf{Bad Aussee}, \emph{P.PPLA3}|pw} 496 44 11/40\pend
           \vspace{0.5em}
\pstart
           bitte uebermitteln sye ihrem bruder\pwindex{Schnitzler, Julius 13.07.1865 – 29.06.1939@\textsc{Schnitzler, Julius} (13.07.1865 – 29.06.1939), \emph{Chirurg/Chirurgin}|pwv} u ihrer schwester\pwindex{Hajek, Gisela 20.12.1867 – 03.02.1953@\textsc{Hajek, Gisela} (20.12.1867 – 03.02.1953)|pwv} mein aufrichtigstes \label{K_L02026-1v}\edtext{bejlejd}{\lemma{\textnormal{\emph{bejlejd}}}\Cendnote{\textnormal{Die Mutter Louise Schnitzler\pwindex{Schnitzler, Louise 1840-07-08 – 1911-09-09@\textsc{Schnitzler, Louise} (1840-07-08 – 1911-09-09)|pwk} war am
                     9. 9. 1911 gestorben.}}}\label{K_L02026-1} sye selbst lieber arthur wiszen denke
               ich dasz ich heute wie immer an allem was sye an gutem und boesem trifft von herzen
               antejl nehme \spacefill\mbox{richard .+}\pend
           \selectlanguage{ngerman}\endnumbering\briefempfaengerindex{Schnitzler, Arthur@\textsc{Schnitzler, Arthur}!zzzBeer-Hofmann, Richard@\emph{von Richard Beer-Hofmann}!1911-09-101@{{[}10.? 9. 1911{]}}|)be}\mylabel{L02026h}  \normalsize

\doendnotes{C}
\bigskip
\vfill

\clearpage

\footnotesize

\lohead{\textsc{register}}

% Definiere theindex-Environment komplett neu ohne reledmac
\makeatletter
\renewenvironment{theindex}{%
  \section*{\indexname}%
  \setlength{\parindent}{0pt}%
  \setlength{\parskip}{0pt plus 0.3pt}%
  \let\item\@idxitem
}{%
  \clearpage
}
\makeatother

\IfFileExists{\jobname-pw.ind}{\input{\jobname-pw.ind}}{}

\end{document}

      