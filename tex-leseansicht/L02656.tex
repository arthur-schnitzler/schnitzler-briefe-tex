%% latex-leseansicht-vorspann.tex
%% Vorspann für die Leseansicht.
%% Lädt die gemeinsame Datei latex-vorspann.tex mit nicht gesetztem Schalter.

\newif\ifkorrekturansicht
\korrekturansichtfalse

\input{../tex-inputs/latex-vorspann}


         
         \renewcommand{\erwaehntePersonen}{Personen: Marie Glümer, Paul Goldmann, Paul Martin Marton}
         \renewcommand{\erwaehnteInstitutionen}{Institutionen: Neues Theater}
         \renewcommand{\erwaehnteOrte}{Orte: Berlin, Wien}
         \renewcommand{\erwaehnteWerke}{Werke: Berliner Lokal-Anzeiger, Das Ewig-Weibliche. Ein heiteres Phantasiespiel in 4 Akten, Die neueste Verlobung in Berliner Theaterkreisen}
               \section[Paul Goldmann an Arthur Schnitzler, 27. 9. 1901]{ Paul Goldmann an Arthur Schnitzler, 27. 9. 1901}\nopagebreak\mylabel{v}\rehead{ }\begin{ledgroupsized}[t]{13cm}\normalsize\beginnumbering\briefempfaengerindex{Schnitzler, Arthur@\textsc{Schnitzler, Arthur}!zzzGoldmann, Paul@\emph{von Paul Goldmann}!1901-09-271@{27. 9. 1901}|(be} \toendnotes[C]{\smallbreak\pagebreak[2]} \Standort{DLA, A:Schnitzler, HS.NZ85.1.3171.}
\physDesc{Telegramm, 144 Zeichen
\newline{}maschinell
\newline{}Ordnung: beschnitten }\toendnotes[C]{\smallbreak}\pstart
           \centering{}{\pb}de berlin\oindex{Berlin@\textbf{Berlin}|pw}
                  44846 22 27 9 15m=\pend
           \pstart
           lokalanzeiger\pwindex{?? Werk@Nicht ermittelte Verfasserinnen und Verfasser!Berliner Lokal-Anzeiger1883@\emph{Berliner Lokal-Anzeiger} {[}1883{]}|pw}{ }\label{K_L02656-1v}\edtext{meldet\pwindex{?? Werk@Nicht ermittelte Verfasserinnen und Verfasser!neueste Verlobung in Berliner Theaterkreisen1901-09-27@\emph{Die neueste Verlobung in Berliner Theaterkreisen} {[}1901-09-27{]}|pwv}}{\lemma{\textnormal{\emph{meldet}}}\Cendnote{\textnormal{»\textbf{Die neueste Verlobung in Berlin\oindex{Berlin@\textbf{Berlin}|pw}er Theaterkreisen}, die sich in aller Stille vollzogen
                        hat, betrifft \so{Paul Martin}\pwindex{Marton, Paul Martin @\textsc{Marton, Paul Martin}, \emph{Schriftsteller, Theaterleiter}|pw}, den Mitdirector des ›Neuen
                           Theater\orgindex{Neues Theater@Neues Theater|pw}s‹, und Fräulein \so{Marie Glümer}\pwindex{Gluemer, Marie 03.07.1867 – 16.11.1925@\textsc{Glümer, Marie} (03.07.1867 – 16.11.1925), \emph{Schauspielerin}|pw}. Die Vermählung soll schon in den nächsten Wochen stattfinden. Demnach
                        hat ›Das Ewig-Weibliche\pwindex{\textcolor{red}{\textsuperscript{XXXX1 indx}}!Ewig-Weibliche. Ein heiteres Phantasiespiel in 4 Akten1900@\strich\emph{Das Ewig-Weibliche. Ein heiteres Phantasiespiel in 4 Akten} {[}1900{]}|pw}‹ bei Director
                           Martin\pwindex{Marton, Paul Martin @\textsc{Marton, Paul Martin}, \emph{Schriftsteller, Theaterleiter}|pw} in dieser Saison bereits
                        einen zweiten Erfolg aufzuweisen.\pwindex{?? Werk@Nicht ermittelte Verfasserinnen und Verfasser!neueste Verlobung in Berliner Theaterkreisen1901-09-27@\emph{Die neueste Verlobung in Berliner Theaterkreisen} {[}1901-09-27{]}|pwv}« \emph{Berliner Lokal-Anzeiger}\pwindex{?? Werk@Nicht ermittelte Verfasserinnen und Verfasser!Berliner Lokal-Anzeiger1883@\emph{Berliner Lokal-Anzeiger} {[}1883{]}|pwk}, Jg. 19, Nr. 453,
                        27. 9. 1901, Morgenblatt, 1. Ausgabe, S. 2.}}}\label{K_L02656-1h}
               verlobung von fraeulein marie gluemer\pwindex{Gluemer, Marie 03.07.1867 – 16.11.1925@\textsc{Glümer, Marie} (03.07.1867 – 16.11.1925), \emph{Schauspielerin}|pw} mit
               direktor martin\pwindex{Marton, Paul Martin @\textsc{Marton, Paul Martin}, \emph{Schriftsteller, Theaterleiter}|pw}. \label{K_L02656-2v}\edtext{weisst du etwas darueber}{\lemma{\textnormal{\emph{weisst du etwas darueber}}}\Cendnote{\textnormal{Siehe A. S.: \emph{Tagebuch}, 27. 9. 1901.
               }}}\label{K_L02656-2h}?\pend
           \pstart gruesse = \spacefill\mbox{goldmann .+}\pend{}
         
         \endnumbering\mylabel{h}\end{ledgroupsized}  \newcommand{\dateiname}{L02656}\newcommand{\titel}{Paul Goldmann an Arthur Schnitzler, 27. 9. 1901}\newcommand{\editorInnen}{Martin Anton Müller und Laura Untner}%% latex-leseansicht-abspann.tex
%% Abspann für die Leseansicht.
%% Der Schalter \ifkorrekturansicht ist bereits durch den Vorspann gesetzt.

%% latex-abspann.tex
%% Gemeinsamer Abspann für Korrekturansicht und Leseansicht.
%% Setzt den Schalter \ifkorrekturansicht voraus (gesetzt in den
%% einbindenden Dateien latex-korrekturansicht-abspann.tex bzw.
%% latex-leseansicht-abspann.tex).
%% ---------------------------------------------------------------

\normalsize

% Das esempio-Environment wird nur in der Leseansicht benötigt
\ifkorrekturansicht\else
\newenvironment{esempio}[3]%
{
    \vspace{1.5ex}
    \rlap{\underline{#1}}
    \par
    \setlength{\parindent}{0cm}
    \nopagebreak
    \leftskip=#2cm
    \rightskip=#3cm
}
{
    \par
}
\fi

\doendnotes{C}
\bigskip
\vfill

\clearpage

\footnotesize

\ifkorrekturansicht
  \lohead{\textsc{register}}
\fi

% theindex-Environment neu definieren ohne reledmac
\makeatletter
\renewenvironment{theindex}{%
  \ifkorrekturansicht
    \section*{\indexname}%
  \else
    \subsubsection*{Index der erwähnten Entitäten}%
  \fi
  \setlength{\parindent}{0pt}%
  \setlength{\parskip}{0pt plus 0.3pt}%
  \let\item\@idxitem
}{%
  \ifkorrekturansicht\clearpage\fi
}
\makeatother

\IfFileExists{\jobname-pw.ind}{\input{\jobname-pw.ind}}{}

% Quellenangabe nur in der Leseansicht
\ifkorrekturansicht\else
% Fallback-Definitionen, falls die .tex-Datei \titel etc. nicht gesetzt hat
\providecommand{\titel}{}
\providecommand{\editorInnen}{}
\providecommand{\dateiname}{\jobname}

\vspace{3cm}

\vfill

\footnotesize
\textsc{Quelle}: \titel. Herausgegeben von {\editorInnen}. In: \emph{Arthur Schnitzler: Briefwechsel mit Autorinnen und Autoren}.
 Digitale Edition, https://schnitzler-briefe.acdh.oeaw.ac.at/{\dateiname}.html (Stand \today)
\fi

\end{document}


      