%% latex-korrekturansicht-vorspann.tex
%% Vorspann für die Korrekturansicht.
%% Lädt die gemeinsame Datei latex-vorspann.tex mit gesetztem Schalter.

\newif\ifkorrekturansicht
\korrekturansichttrue

\input{../tex-inputs/latex-vorspann}


\section[Paul Goldmann an Arthur Schnitzler, 27. 9. 1901]{L02656 Paul Goldmann an Arthur Schnitzler, 27. 9. 1901}
\nopagebreak\mylabel{L02656v}
\rehead{ }\normalsize\beginnumbering\briefempfaengerindex{Schnitzler, Arthur@\textsc{Schnitzler, Arthur}!zzzGoldmann, Paul@\emph{von Paul Goldmann}!1901-09-271@{27. 9. 1901}|(be}
\toendnotes[C]{\smallbreak\pagebreak[2]}\Standort{DLA, A:Schnitzler, HS.NZ85.1.3171.}
\physDesc{Telegramm, 144 Zeichen
\newline{}maschinell
\newline{}Ordnung: beschnitten }\toendnotes[C]{\smallbreak}
\pstart
           \centering{}{\pb}de berlin\oindex{Berlin@\textbf{Berlin}, \emph{P.PPLC}|pw}
                  44846 22 27 9 15m=\pend
           \vspace{0.5em}
\pstart
           lokalanzeiger\pwindex{Berliner Lokal-Anzeiger@\emph{Berliner Lokal-Anzeiger}|pw}{ }\label{K_L02656-1v}\edtext{meldet\pwindex{neueste Verlobung in Berliner Theaterkreisen@\emph{Die neueste Verlobung in Berliner Theaterkreisen}|pwv}}{\lemma{\textnormal{\emph{meldet}}}\Cendnote{\textnormal{»\textbf{Die neueste Verlobung in Berlin\oindex{Berlin@\textbf{Berlin}, \emph{P.PPLC}|pw}er Theaterkreisen}, die sich in aller Stille vollzogen
                        hat, betrifft \so{Paul Martin}\pwindex{Marton, Paul Martin @\textsc{Marton, Paul Martin}, \emph{Schriftsteller/Schriftstellerin, Theaterleiter/Theaterleiterin}|pw}, den Mitdirector des ›Neuen
                           Theater\orgindex{Neues Theater@Neues Theater|pw}s‹, und Fräulein \so{Marie Glümer}\pwindex{Gluemer, Marie 03.07.1867 – 16.11.1925@\textsc{Glümer, Marie} (03.07.1867 – 16.11.1925), \emph{Schauspieler/Schauspielerin}|pw}. Die Vermählung soll schon in den nächsten Wochen stattfinden. Demnach
                        hat ›Das Ewig-Weibliche\pwindex{Ewig-Weibliche. Ein heiteres Phantasiespiel in 4 Akten@\emph{Das Ewig-Weibliche. Ein heiteres Phantasiespiel in 4 Akten}|pw}‹ bei Director
                           Martin\pwindex{Marton, Paul Martin @\textsc{Marton, Paul Martin}, \emph{Schriftsteller/Schriftstellerin, Theaterleiter/Theaterleiterin}|pw} in dieser Saison bereits
                        einen zweiten Erfolg aufzuweisen.\pwindex{neueste Verlobung in Berliner Theaterkreisen@\emph{Die neueste Verlobung in Berliner Theaterkreisen}|pwv}« \emph{Berliner Lokal-Anzeiger}\pwindex{Berliner Lokal-Anzeiger@\emph{Berliner Lokal-Anzeiger}|pwk}, Jg. 19, Nr. 453,
                        27. 9. 1901, Morgenblatt, 1. Ausgabe, S. 2.}}}\label{K_L02656-1}
               verlobung von fraeulein marie gluemer\pwindex{Gluemer, Marie 03.07.1867 – 16.11.1925@\textsc{Glümer, Marie} (03.07.1867 – 16.11.1925), \emph{Schauspieler/Schauspielerin}|pw} mit
               direktor martin\pwindex{Marton, Paul Martin @\textsc{Marton, Paul Martin}, \emph{Schriftsteller/Schriftstellerin, Theaterleiter/Theaterleiterin}|pw}. \label{K_L02656-2v}\edtext{weisst du etwas darueber}{\lemma{\textnormal{\emph{weisst du etwas darueber}}}\Cendnote{\textnormal{Siehe A. S.: \emph{Tagebuch}, 27. 9. 1901.
               }}}\label{K_L02656-2}?\pend
           \pstart gruesse = \spacefill\mbox{goldmann .+}\pend{}\selectlanguage{ngerman}\endnumbering\briefempfaengerindex{Schnitzler, Arthur@\textsc{Schnitzler, Arthur}!zzzGoldmann, Paul@\emph{von Paul Goldmann}!1901-09-271@{27. 9. 1901}|)be}\mylabel{L02656h}  \normalsize

\doendnotes{C}
\bigskip
\vfill

\clearpage

\footnotesize

\lohead{\textsc{register}}

% Definiere theindex-Environment komplett neu ohne reledmac
\makeatletter
\renewenvironment{theindex}{%
  \section*{\indexname}%
  \setlength{\parindent}{0pt}%
  \setlength{\parskip}{0pt plus 0.3pt}%
  \let\item\@idxitem
}{%
  \clearpage
}
\makeatother

\IfFileExists{\jobname-pw.ind}{\input{\jobname-pw.ind}}{}

\end{document}

      