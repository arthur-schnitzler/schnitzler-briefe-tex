%% latex-korrekturansicht-vorspann.tex
%% Vorspann für die Korrekturansicht.
%% Lädt die gemeinsame Datei latex-vorspann.tex mit gesetztem Schalter.

\newif\ifkorrekturansicht
\korrekturansichttrue

\input{../tex-inputs/latex-vorspann}


\section[ Paul Goldmann an Arthur Schnitzler, 24. 11. {[}1900{]}]{L02940 Paul Goldmann an Arthur Schnitzler, 24. 11. {[}1900{]}}
\nopagebreak\mylabel{L02940v}
\rehead{ }\normalsize\beginnumbering\briefempfaengerindex{Schnitzler, Arthur@\textsc{Schnitzler, Arthur}!zzzGoldmann, Paul@\emph{von Paul Goldmann}!1900-11-241@{24. 11. {[}1900{]}}|(be}
\toendnotes[C]{\smallbreak\pagebreak[2]}\Standort{DLA, A:Schnitzler, HS.NZ85.1.3170.}
\physDesc{Brief, 1 Blatt, 2 Seiten, 517 Zeichen
\newline{}Handschrift: blaue Tinte, deutsche Kurrent
\newline{}Schnitzler: mit Bleistift das Jahr »900« vermerkt }\toendnotes[C]{\smallbreak}
\pstart
           \raggedleft{}{\pb}\textcolor{gray}{\textbf{DESSAUERSTRASSE 19}}\oindex{Dessauer Strasse@\textbf{Dessauer Straße}, \emph{Straße (K.STR)}|pw}\pend
           
\pstart
           Berlin\oindex{Berlin@\textbf{Berlin}, \emph{P.PPLC}|pw}, 24. November.\pend
           
\pstart\center{}Mein lieber Freund,\pend\vspace{0.5em}
\pstart
           Ich kann Dich leider nicht begrüßen kommen, denn ich habe den ganzen Nachmittag im
                  \label{K_L02940-1v}\edtext{Reichstage\oindex{Reichstag@\textbf{Reichstag}, \emph{Regierungsgebäude (K.RGB)}|pw}}{\lemma{\textnormal{\emph{Reichstage}}}\Cendnote{\textnormal{Siehe Paul Goldmann an Arthur Schnitzler, 30. 10. [1900].
               }}}\label{K_L02940-1} zu thun. Einſtweilen alſo heiße ich Dich auf dieſem Wege \label{K_L02940-2v}\edtext{herzlichſt willkommen}{\lemma{\textnormal{\emph{herzlichſt willkommen}}}\Cendnote{\textnormal{Schnitzler hielt sich vom 24. 11. 1900 bis zum 28. 11. 1900 in Berlin\oindex{Berlin@\textbf{Berlin}, \emph{P.PPLC}|pwk} auf.}}}\label{K_L02940-2}. Abends
                  zwiſchen 9 und 10 Uhr hoffe ich mit meiner Arbeit fertig zu ſein. Bitte,
               ſende mir alſo eine Nachricht in meine Wohnung, wo ich Dich um dieſe Zeit treffen {\pb}kann? Am Beſten wäre es, Du kämeſt
                  zwiſchen 9 und 10 Uhr ſelbſt \label{K_L02940-3v}\edtext{zu mir}{\lemma{\textnormal{\emph{zu mir}}}\Cendnote{\textnormal{Am 24. 11. 1900 trafen
                  sich Goldmann\pwindex{Goldmann, Paul 31.01.1865 – 25.09.1935@\textsc{Goldmann, Paul} (31.01.1865 – 25.09.1935), \emph{Schriftsteller/Schriftstellerin, Journalist/Journalistin}|pwk} und Schnitzler mit Marie
                     Glümer\pwindex{Gluemer, Marie 03.07.1867 – 16.11.1925@\textsc{Glümer, Marie} (03.07.1867 – 16.11.1925), \emph{Schauspieler/Schauspielerin}|pwk}, Paul Martin Marton\pwindex{Marton, Paul Martin @\textsc{Marton, Paul Martin}, \emph{Schriftsteller/Schriftstellerin, Theaterleiter/Theaterleiterin}|pwk} und Moritz Coschell\pwindex{Coschell, Moritz 1872-09-18 – 1943-07-11@\textsc{Coschell, Moritz} (1872-09-18 – 1943-07-11), \emph{Maler/Malerin}|pwk} im Hotel Kaiserhof\oindex{Hotel Kaiserhof [Berlin]@\textbf{Hotel Kaiserhof [Berlin]}, \emph{Hotel (K.HTL)}|pwk}. Am 25. 11. 1900 war Schnitzler tatsächlich zu Mittag bei Goldmann\pwindex{Goldmann, Paul 31.01.1865 – 25.09.1935@\textsc{Goldmann, Paul} (31.01.1865 – 25.09.1935), \emph{Schriftsteller/Schriftstellerin, Journalist/Journalistin}|pwk} und traf ihn abends noch einmal
                  gemeinsam mit Moritz Coschell\pwindex{Coschell, Moritz 1872-09-18 – 1943-07-11@\textsc{Coschell, Moritz} (1872-09-18 – 1943-07-11), \emph{Maler/Malerin}|pwk} und Alfred Kerr\pwindex{Kerr, Alfred 25.12.1867 – 12.10.1948@\textsc{Kerr, Alfred} (25.12.1867 – 12.10.1948), \emph{Schriftsteller/Schriftstellerin, Kritiker/Kritikerin}|pwk}.}}}\label{K_L02940-3}. Und morgen{ }Mittag biſt Du natürlich bei mir zu Tiſch.\pend
           
\pstart
           Herzlichſt {\\[\baselineskip]}Dein {\\[\baselineskip]}\spacefill\mbox{Paul Goldmann.}\pend
           \leftskip=0em{}\selectlanguage{ngerman}\endnumbering\briefempfaengerindex{Schnitzler, Arthur@\textsc{Schnitzler, Arthur}!zzzGoldmann, Paul@\emph{von Paul Goldmann}!1900-11-241@{24. 11. {[}1900{]}}|)be}\mylabel{L02940h}  \normalsize

\doendnotes{C}
\bigskip
\vfill

\clearpage

\footnotesize

\lohead{\textsc{register}}

% Definiere theindex-Environment komplett neu ohne reledmac
\makeatletter
\renewenvironment{theindex}{%
  \section*{\indexname}%
  \setlength{\parindent}{0pt}%
  \setlength{\parskip}{0pt plus 0.3pt}%
  \let\item\@idxitem
}{%
  \clearpage
}
\makeatother

\IfFileExists{\jobname-pw.ind}{\input{\jobname-pw.ind}}{}

\end{document}

      