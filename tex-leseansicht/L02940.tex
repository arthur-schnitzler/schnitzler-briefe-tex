%% latex-leseansicht-vorspann.tex
%% Vorspann für die Leseansicht.
%% Lädt die gemeinsame Datei latex-vorspann.tex mit nicht gesetztem Schalter.

\newif\ifkorrekturansicht
\korrekturansichtfalse

\input{../tex-inputs/latex-vorspann}


\section[ Paul Goldmann an Arthur Schnitzler, 24. 11. [1900]]{L02940 Paul Goldmann an Arthur Schnitzler,  24. 11. [1900]}
\nopagebreak\mylabel{L02940v}
\rehead{ }\normalsize\beginnumbering\briefempfaengerindex{Schnitzler, Arthur@\textsc{Schnitzler, Arthur}!zzzGoldmann, Paul@\emph{von Paul Goldmann}!1900-11-241@{24. 11. [1900]}|(be}
\toendnotes[C]{\smallbreak\pagebreak[2]}
\correspDesc{Versand  durch Paul Goldmann am 24. 11. [1900] in Berlin
\newline{}Erhalt  durch Arthur Schnitzler am [24. 11. 1900?] in Berlin}\toendnotes[C]{\smallbreak}
\Standort{DLA, A:Schnitzler, HS.NZ85.1.3170.}
\physDesc{Brief, 1 Blatt, 2 Seiten, 517 Zeichen
\newline{}Handschrift: blaue Tinte, deutsche Kurrent
\newline{}Schnitzler: mit Bleistift das Jahr »900« vermerkt }\toendnotes[C]{\smallbreak}
\pstart
           \raggedleft{}{\pb}\textcolor{gray}{\textbf{DESSAUERSTRASSE 19}}\oindex{Dessauer Straße@\textbf{Dessauer Straße}, \emph{Straße}|pw}\pend
           
\pstart
           Berlin\oindex{Berlin@\textbf{Berlin}, \emph{Hauptstadt}|pw}, 24. November.\pend
           
\pstart\center{}Mein lieber Freund,\pend\vspace{0.5em}
\pstart
           Ich kann Dich leider nicht begrüßen kommen, denn ich habe den ganzen Nachmittag im
                  \label{K_L02940-1v}\edtext{Reichstage\oindex{Reichstag@\textbf{Reichstag}, \emph{Regierungsgebäude}|pw}}{\lemma{\textnormal{\emph{Reichstage}}}\Cendnote{\textnormal{Siehe XXXX Auszeichnungsfehler: Dokument L02937 nicht gefunden.
               }}}\label{K_L02940-1} zu thun. Einſtweilen alſo heiße ich Dich auf dieſem Wege \label{K_L02940-2v}\edtext{herzlichſt willkommen}{\lemma{\textnormal{\emph{herzlichst willkommen}}}\Cendnote{\textnormal{Schnitzler hielt sich vom 24. 11. 1900 bis zum 28. 11. 1900 in Berlin\oindex{Berlin@\textbf{Berlin}, \emph{Hauptstadt}|pwk} auf.}}}\label{K_L02940-2}. Abends
                  zwiſchen 9 und 10 Uhr hoffe ich mit meiner Arbeit fertig zu{ }ſein. Bitte,{ }ſende mir alſo eine Nachricht in meine Wohnung, wo ich Dich um dieſe Zeit treffen {\pb}kann? Am Beſten wäre es, Du kämeſt
                  zwiſchen 9 und 10 Uhr{ }ſelbſt \label{K_L02940-3v}\edtext{zu mir}{\lemma{\textnormal{\emph{zu mir}}}\Cendnote{\textnormal{Am 24. 11. 1900 trafen
                  sich Goldmann\pwindex{Goldmann, Paul 31.\,1.\,1865 Breslau – 25.\,9.\,1935 Wien@\textsc{Goldmann, Paul} (31.\,1.\,1865 Breslau – 25.\,9.\,1935 Wien), \emph{Schriftsteller, Journalist}|pwk} und Schnitzler mit Marie
                     Glümer\pwindex{Glümer, Marie 3.\,7.\,1867 Wien – 16.\,11.\,1925 München@\textsc{Glümer, Marie} (3.\,7.\,1867 Wien – 16.\,11.\,1925 München), \emph{Schauspielerin}|pwk}, Paul Martin Marton\pwindex{Marton, Paul Martin @\textsc{Marton, Paul Martin}, \emph{Schriftsteller, Theaterleiter}|pwk} und Moritz Coschell\pwindex{Coschell, Moritz 18.\,9.\,1872 Wien – 11.\,7.\,1943 ebd.@\textsc{Coschell, Moritz} (18.\,9.\,1872 Wien – 11.\,7.\,1943 ebd.), \emph{Maler}|pwk} im Hotel Kaiserhof\oindex{Hotel Kaiserhof [Berlin]@\textbf{Hotel Kaiserhof [Berlin]}, \emph{Hotel}|pwk}. Am 25. 11. 1900 war Schnitzler tatsächlich zu Mittag bei Goldmann\pwindex{Goldmann, Paul 31.\,1.\,1865 Breslau – 25.\,9.\,1935 Wien@\textsc{Goldmann, Paul} (31.\,1.\,1865 Breslau – 25.\,9.\,1935 Wien), \emph{Schriftsteller, Journalist}|pwk} und traf ihn abends noch einmal
                  gemeinsam mit Moritz Coschell\pwindex{Coschell, Moritz 18.\,9.\,1872 Wien – 11.\,7.\,1943 ebd.@\textsc{Coschell, Moritz} (18.\,9.\,1872 Wien – 11.\,7.\,1943 ebd.), \emph{Maler}|pwk} und Alfred Kerr\pwindex{Kerr, Alfred 25.\,12.\,1867 Breslau – 12.\,10.\,1948 Hamburg@\textsc{Kerr, Alfred} (25.\,12.\,1867 Breslau – 12.\,10.\,1948 Hamburg), \emph{Schriftsteller, Kritiker}|pwk}.}}}\label{K_L02940-3}. Und morgen{ }Mittag biſt Du natürlich bei mir zu Tiſch.\pend
           
\pstart
           Herzlichſt {\\[\baselineskip]}Dein {\\[\baselineskip]}\spacefill\mbox{Paul Goldmann.}\pend
           \leftskip=0em{}\selectlanguage{ngerman}\endnumbering\briefempfaengerindex{Schnitzler, Arthur@\textsc{Schnitzler, Arthur}!zzzGoldmann, Paul@\emph{von Paul Goldmann}!1900-11-241@{24. 11. [1900]}|)be}\mylabel{L02940h}  \newcommand{\dateiname}{L02940}\newcommand{\titel}{Paul Goldmann an Arthur Schnitzler, 24. 11. [1900]}\newcommand{\editorInnen}{Martin Anton Müller und Laura Untner}%% latex-leseansicht-abspann.tex
%% Abspann für die Leseansicht.
%% Der Schalter \ifkorrekturansicht ist bereits durch den Vorspann gesetzt.

%% latex-abspann.tex
%% Gemeinsamer Abspann für Korrekturansicht und Leseansicht.
%% Setzt den Schalter \ifkorrekturansicht voraus (gesetzt in den
%% einbindenden Dateien latex-korrekturansicht-abspann.tex bzw.
%% latex-leseansicht-abspann.tex).
%% ---------------------------------------------------------------

\normalsize

% Das esempio-Environment wird nur in der Leseansicht benötigt
\ifkorrekturansicht\else
\newenvironment{esempio}[3]%
{
    \vspace{1.5ex}
    \rlap{\underline{#1}}
    \par
    \setlength{\parindent}{0cm}
    \nopagebreak
    \leftskip=#2cm
    \rightskip=#3cm
}
{
    \par
}
\fi

\doendnotes{C}
\bigskip
\vfill

\clearpage

\footnotesize

\ifkorrekturansicht
  \lohead{\textsc{register}}
\fi

% theindex-Environment neu definieren ohne reledmac
\makeatletter
\renewenvironment{theindex}{%
  \ifkorrekturansicht
    \section*{\indexname}%
  \else
    \subsubsection*{Index der erwähnten Entitäten}%
  \fi
  \setlength{\parindent}{0pt}%
  \setlength{\parskip}{0pt plus 0.3pt}%
  \let\item\@idxitem
}{%
  \ifkorrekturansicht\clearpage\fi
}
\makeatother

\IfFileExists{\jobname-pw.ind}{\input{\jobname-pw.ind}}{}

% Quellenangabe nur in der Leseansicht
\ifkorrekturansicht\else
% Fallback-Definitionen, falls die .tex-Datei \titel etc. nicht gesetzt hat
\providecommand{\titel}{}
\providecommand{\editorInnen}{}
\providecommand{\dateiname}{\jobname}

\vspace{3cm}

\vfill

\footnotesize
\textsc{Quelle}: \titel. Herausgegeben von {\editorInnen}. In: \emph{Arthur Schnitzler: Briefwechsel mit Autorinnen und Autoren}.
 Digitale Edition, https://schnitzler-briefe.acdh.oeaw.ac.at/{\dateiname}.html (Stand \today)
\fi

\end{document}


