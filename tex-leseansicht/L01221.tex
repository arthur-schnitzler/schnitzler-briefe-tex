%% latex-korrekturansicht-vorspann.tex
%% Vorspann für die Korrekturansicht.
%% Lädt die gemeinsame Datei latex-vorspann.tex mit gesetztem Schalter.

\newif\ifkorrekturansicht
\korrekturansichttrue

\input{../tex-inputs/latex-vorspann}


\section[Hermann Bahr an Arthur Schnitzler, 20. 5. 1902]{L01221 Hermann Bahr an Arthur Schnitzler, 20. 5. 1902}
\nopagebreak\mylabel{L01221v}
\rehead{ }\normalsize\beginnumbering\briefempfaengerindex{Schnitzler, Arthur@\textsc{Schnitzler, Arthur}!zzzBahr, Hermann@\emph{von Hermann Bahr}!1902-05-201@{20. 5. 1902}|(be}
\toendnotes[C]{\smallbreak\pagebreak[2]}\Standort{CUL, Schnitzler, B 5b.}
\physDesc{Briefkarte, 145 Zeichen (Briefkarte mit Trauerrand)
\newline{}Druck
\newline{}Handschrift: schwarze Tinte, deutsche Kurrent
\newline{}Ordnung: mit Bleistift von unbekannter Hand nummeriert:
                                    »89« }
\buchAbdrucke{\weitereDrucke{Hermann Bahr, Arthur Schnitzler: \emph{Briefwechsel, Aufzeichnungen, Dokumente (1891–1931)}. Göttingen: \emph{Wallstein} 2018, S. 238.} }\toendnotes[C]{\smallbreak}
\pstart
           \noindent{}{\pb}\textcolor{gray}{\textbf{Für die vielen Beweise herzlicher Teilnahme bei dem
                     Hinscheiden und der Beerdigung unserer lieben, unvergesslichen Mutter,
                     Schwiegermutter, Schwägerin u. Tante}}\pend
           
\pstart
           \textcolor{gray}{\textbf{Frau Mina Bahr geb. von
                        Weidlich\pwindex{Bahr, Wilhelmine 06.06.1835 – 16.05.1902@\textsc{Bahr, Wilhelmine} (06.06.1835 – 16.05.1902)|pw}}}\pend
           
\pstart
           \textcolor{gray}{\textbf{sprechen ihren innigsten Dank aus}}\pend
           
\pstart
           \textcolor{gray}{\textbf{Salzburg\oindex{Salzburg@\textbf{Salzburg}, \emph{A.ADM2}|pw}, 19. Mai
                        1902}}\pend
           
\pstart
           \raggedleft{}\textcolor{gray}{\textbf{Die tieftrauernd Hinterbliebenen.}}\pend
           
\pstart
           {\pb}Wie eine fixe Idee verfolgt mich dieſe ganzen Tage
               der Satz: \label{K_L01221-1v}\edtext{es gibt alſo Fälle, wo Salzburg\oindex{Salzburg@\textbf{Salzburg}, \emph{A.ADM2}|pw} nicht wirkt\pwindex{Lebendige Stunden. Vier Einakter@\emph{Lebendige Stunden. Vier Einakter}|pwv}}{\lemma{\textnormal{\emph{es … wirkt}}}\Cendnote{\textnormal{Vgl. Bahrs\pwindex{Bahr, Hermann 19.07.1863 – 15.01.1934@\textsc{Bahr, Hermann} (19.07.1863 – 15.01.1934), \emph{Schriftsteller/Schriftstellerin, Kritiker/Kritikerin}|pwk} Feuilleton \emph{Lebendige Stunden (Vier Einacter von Arthur Schnitzler:
                        »Lebendige Stunden«, »Die Frau mit dem Dolche«, »Die letzten Masken« und
                        »Literatur«. Zum ersten Male aufgeführt im Carl-Theater am
                           6. Mai 1902. Erste Vorstellung des Berliner Deutschen
                        Theaters)}\pwindex{Lebendige Stunden (Vier Einacter von Arthur Schnitzler: »Lebendige Stunden«, »Die Frau mit dem Dolche«, »Die letzten Masken« und »Literatur«. Zum ersten Male aufgefuehrt im Carl-Theater am 6. Mai 1902. Erste Vorstellung des Berliner Deutschen Theaters)@\emph{Lebendige Stunden (Vier Einacter von Arthur Schnitzler: »Lebendige Stunden«, »Die Frau mit dem Dolche«, »Die letzten Masken« und »Literatur«. Zum ersten Male aufgeführt im Carl-Theater am 6. Mai 1902. Erste Vorstellung des Berliner Deutschen Theaters)}|pwk} und vgl. A. S.: \emph{Tagebuch}, 11. 9. 1911.}}}\label{K_L01221-1}.\pend
           
\pstart
           Es dankt Dir ſehr{\\[\baselineskip]}Dein{\\[\baselineskip]}\spacefill\mbox{Hermann}\pend
           \leftskip=0em{}
\pstart
           Salzburg\oindex{Salzburg@\textbf{Salzburg}, \emph{A.ADM2}|pw}{ }20. 5.\pend
           \selectlanguage{ngerman}\endnumbering\briefempfaengerindex{Schnitzler, Arthur@\textsc{Schnitzler, Arthur}!zzzBahr, Hermann@\emph{von Hermann Bahr}!1902-05-201@{20. 5. 1902}|)be}\mylabel{L01221h}  \normalsize

\doendnotes{C}
\bigskip
\vfill

\clearpage

\footnotesize

\lohead{\textsc{register}}

% Definiere theindex-Environment komplett neu ohne reledmac
\makeatletter
\renewenvironment{theindex}{%
  \section*{\indexname}%
  \setlength{\parindent}{0pt}%
  \setlength{\parskip}{0pt plus 0.3pt}%
  \let\item\@idxitem
}{%
  \clearpage
}
\makeatother

\IfFileExists{\jobname-pw.ind}{\input{\jobname-pw.ind}}{}

\end{document}

      