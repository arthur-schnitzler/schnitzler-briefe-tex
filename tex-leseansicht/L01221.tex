%% latex-leseansicht-vorspann.tex
%% Vorspann für die Leseansicht.
%% Lädt die gemeinsame Datei latex-vorspann.tex mit nicht gesetztem Schalter.

\newif\ifkorrekturansicht
\korrekturansichtfalse

\input{../tex-inputs/latex-vorspann}


         
         \renewcommand{\erwaehntePersonen}{Personen: Hermann Bahr, Wilhelmine Bahr}
         \renewcommand{\erwaehnteOrte}{Orte: Salzburg, Wien}
         \renewcommand{\erwaehnteWerke}{Werke: Lebendige Stunden (Vier Einacter von Arthur Schnitzler: »Lebendige Stunden«, »Die Frau mit dem Dolche«, »Die letzten Masken« und »Literatur«. Zum ersten Male aufgeführt im Carl-Theater am 6. Mai 1902. Erste Vorstellung des Berliner Deutschen Theaters), Lebendige Stunden. Vier Einakter}
               \section[Hermann Bahr an Arthur Schnitzler, 20. 5. 1902]{ Hermann Bahr an Arthur Schnitzler, 20. 5. 1902}\nopagebreak\mylabel{v}\rehead{ }\begin{ledgroupsized}[t]{13cm}\normalsize\beginnumbering\briefempfaengerindex{Schnitzler, Arthur@\textsc{Schnitzler, Arthur}!zzzBahr, Hermann@\emph{von Hermann Bahr}!1902-05-201@{20. 5. 1902}|(be} \toendnotes[C]{\smallbreak\pagebreak[2]} \Standort{CUL, Schnitzler, B 5b.}
\physDesc{Briefkarte, 145 Zeichen (Trauerrand)
\newline{}Druck
\newline{}Handschrift: schwarze Tinte, deutsche Kurrent
\newline{}Ordnung: mit Bleistift von unbekannter Hand nummeriert:
                                    »89« }\buchAbdrucke{\weitereDrucke{Hermann Bahr, Arthur Schnitzler: \emph{Briefwechsel, Aufzeichnungen, Dokumente (1891–1931)}. Hg. Kurt Ifkovits und Martin Anton Müller. Göttingen: \emph{Wallstein} 2018, S. 238.} }\toendnotes[C]{\smallbreak}\pstart
           \noindent{}{\pb}\textcolor{gray}{\textbf{Für die vielen Beweise herzlicher Teilnahme bei dem
                     Hinscheiden und der Beerdigung unserer lieben, unvergesslichen Mutter,
                     Schwiegermutter, Schwägerin u. Tante}}\pend
           \pstart
           \textcolor{gray}{\textbf{Frau Mina Bahr geb. von
                        Weidlich\pwindex{Bahr, Wilhelmine 06.06.1835 – 16.05.1902@\textsc{Bahr, Wilhelmine} (06.06.1835 – 16.05.1902)|pw}}}\pend
           \pstart
           \textcolor{gray}{\textbf{sprechen ihren innigsten Dank aus}}\pend
           \leftskip=3em{}\pstart
           \noindent{}\textcolor{gray}{\textbf{Salzburg\oindex{Salzburg@\textbf{Salzburg}|pw}, 19. Mai
                        1902}}\pend
           \leftskip=0em{}\pstart
           \noindent{}\raggedleft{}\textcolor{gray}{\textbf{Die tieftrauernd Hinterbliebenen.}}\pend
           \pstart
           {\pb}Wie eine fixe Idee verfolgt mich dieſe ganzen Tage
               der Satz: \label{K_L01221-1v}\edtext{es gibt alſo Fälle, wo Salzburg\oindex{Salzburg@\textbf{Salzburg}|pw} nicht wirkt\pwindex{Schnitzler, Arthur 15.05.1862 – 21.10.1931@\textsc{Schnitzler, Arthur} (15.05.1862 – 21.10.1931), \emph{Schriftsteller, Mediziner}!Lebendige Stunden. Vier Einakter1901-12-23@\strich\emph{Lebendige Stunden. Vier Einakter} {[}1901-12-23{]}|pwv}}{\lemma{\textnormal{\emph{es … wirkt}}}\Cendnote{\textnormal{Vgl. Bahrs\pwindex{Bahr, Hermann 19.07.1863 – 15.01.1934@\textsc{Bahr, Hermann} (19.07.1863 – 15.01.1934), \emph{Schriftsteller, Kritiker}|pwk} Feuilleton \emph{Lebendige Stunden (Vier Einacter von Arthur Schnitzler:
                        »Lebendige Stunden«, »Die Frau mit dem Dolche«, »Die letzten Masken« und
                        »Literatur«. Zum ersten Male aufgeführt im Carl-Theater am
                           6. Mai 1902. Erste Vorstellung des Berliner Deutschen
                        Theaters)}\pwindex{Bahr, Hermann 19.07.1863 – 15.01.1934@\textsc{Bahr, Hermann} (19.07.1863 – 15.01.1934), \emph{Schriftsteller, Kritiker}!Lebendige Stunden (Vier Einacter von Arthur Schnitzler: »Lebendige Stunden«, »Die Frau mit dem Dolche«, »Die letzten Masken« und »Literatur«. Zum ersten Male aufgefuehrt im Carl-Theater am 6. Mai 1902. Erste Vorstellung des Berliner Deutschen Theaters)7. 5. 1902@\strich\emph{Lebendige Stunden (Vier Einacter von Arthur Schnitzler: »Lebendige Stunden«, »Die Frau mit dem Dolche«, »Die letzten Masken« und »Literatur«. Zum ersten Male aufgeführt im Carl-Theater am 6. Mai 1902. Erste Vorstellung des Berliner Deutschen Theaters)} {[}7. 5. 1902{]}|pwk} und vgl. A. S.: \emph{Tagebuch}, 11. 9. 1911.}}}\label{K_L01221-1h}.\pend
           \pstart
           Es dankt Dir ſehr{\\[\baselineskip]}Dein{\\[\baselineskip]}\spacefill\mbox{Hermann}\pend
           \leftskip=0em{}\pstart
           Salzburg\oindex{Salzburg@\textbf{Salzburg}|pw}{ }20. 5.\pend
           
         
         \endnumbering\mylabel{h}\end{ledgroupsized}  \newcommand{\dateiname}{L01221}\newcommand{\titel}{Hermann Bahr an Arthur Schnitzler, 20. 5. 1902}\newcommand{\editorInnen}{ Kurt Ifkovits,  Martin Anton Müller}%% latex-leseansicht-abspann.tex
%% Abspann für die Leseansicht.
%% Der Schalter \ifkorrekturansicht ist bereits durch den Vorspann gesetzt.

%% latex-abspann.tex
%% Gemeinsamer Abspann für Korrekturansicht und Leseansicht.
%% Setzt den Schalter \ifkorrekturansicht voraus (gesetzt in den
%% einbindenden Dateien latex-korrekturansicht-abspann.tex bzw.
%% latex-leseansicht-abspann.tex).
%% ---------------------------------------------------------------

\normalsize

% Das esempio-Environment wird nur in der Leseansicht benötigt
\ifkorrekturansicht\else
\newenvironment{esempio}[3]%
{
    \vspace{1.5ex}
    \rlap{\underline{#1}}
    \par
    \setlength{\parindent}{0cm}
    \nopagebreak
    \leftskip=#2cm
    \rightskip=#3cm
}
{
    \par
}
\fi

\doendnotes{C}
\bigskip
\vfill

\clearpage

\footnotesize

\ifkorrekturansicht
  \lohead{\textsc{register}}
\fi

% theindex-Environment neu definieren ohne reledmac
\makeatletter
\renewenvironment{theindex}{%
  \ifkorrekturansicht
    \section*{\indexname}%
  \else
    \subsubsection*{Index der erwähnten Entitäten}%
  \fi
  \setlength{\parindent}{0pt}%
  \setlength{\parskip}{0pt plus 0.3pt}%
  \let\item\@idxitem
}{%
  \ifkorrekturansicht\clearpage\fi
}
\makeatother

\IfFileExists{\jobname-pw.ind}{\input{\jobname-pw.ind}}{}

% Quellenangabe nur in der Leseansicht
\ifkorrekturansicht\else
% Fallback-Definitionen, falls die .tex-Datei \titel etc. nicht gesetzt hat
\providecommand{\titel}{}
\providecommand{\editorInnen}{}
\providecommand{\dateiname}{\jobname}

\vspace{3cm}

\vfill

\footnotesize
\textsc{Quelle}: \titel. Herausgegeben von {\editorInnen}. In: \emph{Arthur Schnitzler: Briefwechsel mit Autorinnen und Autoren}.
 Digitale Edition, https://schnitzler-briefe.acdh.oeaw.ac.at/{\dateiname}.html (Stand \today)
\fi

\end{document}


      