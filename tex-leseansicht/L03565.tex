%% latex-korrekturansicht-vorspann.tex
%% Vorspann für die Korrekturansicht.
%% Lädt die gemeinsame Datei latex-vorspann.tex mit gesetztem Schalter.

\newif\ifkorrekturansicht
\korrekturansichttrue

\input{../tex-inputs/latex-vorspann}


\section[ Felix Salten an Arthur Schnitzler, 10. 8. 1914]{L03565 Felix Salten an Arthur Schnitzler, 10. 8. 1914}
\nopagebreak\mylabel{L03565v}
\rehead{ }\normalsize\beginnumbering\briefempfaengerindex{Schnitzler, Arthur@\textsc{Schnitzler, Arthur}!zzzSalten, Felix@\emph{von Felix Salten}!1914-08-102@{10. 8. 1914}|(be}
\toendnotes[C]{\smallbreak\pagebreak[2]}\Standort{CUL, Schnitzler, B 89, B 2.}
\physDesc{Briefkarte, 890 Zeichen
\newline{}Handschrift: schwarze Tinte, lateinische Kurrent
\newline{}Schnitzler: 1) mit Bleistift Vermerk: »\textsc{Salten}«  2) mit rotem Buntstift eine Unterstreichung
\newline{}Ordnung: mit Bleistift von unbekannter Hand nummeriert: »278« }\toendnotes[C]{\smallbreak}
\pstart
           \raggedleft{}{\pb}Berghof\oindex{Berghof@\textbf{Berghof}, \emph{Wohngebäude (K.WHS)}|pw}, 10. 8. 14\pend
           
\pstart{}Lieber,\pend\vspace{0.5em}
\pstart
           Ihre Karte aus der \label{K_L03565-1v}\edtext{Schweiz\oindex{Schweiz@\textbf{Schweiz}, \emph{A.PCLI}|pw}}{\lemma{\textnormal{\emph{Schweiz}}}\Cendnote{\textnormal{Schnitzler war am 18. 7. 1914 mit seiner
                     Frau\pwindex{Schnitzler, Olga 17.01.1882 – 13.01.1970@\textsc{Schnitzler, Olga} (17.01.1882 – 13.01.1970), \emph{Schauspieler/Schauspielerin, Sänger/Sängerin}|pwkv} und den Kindern\pwindex{Schnitzler, Heinrich 09.08.1902 – 12.07.1982@\textsc{Schnitzler, Heinrich} (09.08.1902 – 12.07.1982), \emph{Regisseur/Regisseurin, Schauspieler/Schauspielerin}|pwkv}\pwindex{Cappellini, Lili 13.09.1909 – 26.07.1928@\textsc{Cappellini, Lili} (13.09.1909 – 26.07.1928)|pwkv} in der
                     Schweiz\oindex{Schweiz@\textbf{Schweiz}, \emph{A.PCLI}|pwk} angekommen. Die Heimreise nach
                  Kriegsausbruch erwies sich als schwierig. Am 15. 8. 1914 reisten sie nach Österreich\oindex{Oesterreich@\textbf{Österreich}, \emph{A.PCLI}|pwk}, zuerst aber nach Bad Ischl\oindex{Bad Ischl@\textbf{Bad Ischl}, \emph{P.PPL}|pwk} (vgl. Arthur und Olga Schnitzler an Richard Beer-Hofmann, 20. 8. 1914). Am 2. 9. 1914 waren sie wieder in Wien\oindex{Wien@\textbf{Wien}, \emph{A.ADM2}|pwk}.}}}\label{K_L03565-1} bekam ich vor zwei Tagen, nehme aber an, dass
               Sie jetzt wieder zu Hause sind. Wann ich nach Wien\oindex{Wien@\textbf{Wien}, \emph{A.ADM2}|pw}
               komme, weiß ich nicht, weiß nicht einmal, ob ich es soll. Hier\oindex{Unterach am Attersee@\textbf{Unterach am Attersee}, \emph{P.PPL}|pwv} ist es so ganz still, ganz einsam und
               das beruhigt einigermaßen. Sonst – wenn man sich’s klar macht, was jetzt geschieht
               und warum es geschieht – könnte man verzweifeln. Wer dran glaubt, \label{K_L03565-2v}\edtext{dies alles sei wegen Serbien\oindex{Serbien@\textbf{Serbien}, \emph{A.PCLI}|pw}}{\lemma{\textnormal{\emph{dies … Serbien}}}\Cendnote{\textnormal{Das Attentat von Sarajevo\oindex{Sarajevo@\textbf{Sarajevo}, \emph{P.PPLC}|pwk} wurde in Zusammenhang mit Bestrebungen Serbiens\oindex{Serbien@\textbf{Serbien}, \emph{A.PCLI}|pwk} gesehen, das eine politische Einigung
                  am Balkan\oindex{Balkanhalbinsel@\textbf{Balkanhalbinsel}, \emph{T.PEN}|pwk} unter seiner Führung
                  anstrebte.}}}\label{K_L03565-2}, ist eigentlich zu beneiden. Denn er hat doch etwas, um sein
               Rechtsgefühl damit zu füttern. Vielleicht ist es gut, dass dieser Krieg eben \label{K_L03565-3v}\edtext{jetzt ausgebrochen}{\lemma{\textnormal{\emph{jetzt ausgebrochen}}}\Cendnote{\textnormal{Österreich\oindex{Oesterreich@\textbf{Österreich}, \emph{A.PCLI}|pwk} hatte Serbien\oindex{Serbien@\textbf{Serbien}, \emph{A.PCLI}|pwk} am 28. 7. 1914 den
                  Krieg erklärt. Damit hatte der Erste Weltkrieg begonnen.}}}\label{K_L03565-3} wird. Gut: für
               unsere Söhne. Das mag hässlich und egoistisch gedacht sein, aber ich denke es eben.
                  \label{K_L03565-4v}\edtext{Beer-Hofmanns\pwindex{Beer-Hofmann, Richard 1866-07-11 – 1945-09-26@\textsc{Beer-Hofmann, Richard} (1866-07-11 – 1945-09-26), \emph{Schriftsteller/Schriftstellerin}|pw}\pwindex{Beer-Hofmann, Paula 25.02.1879 – 30.10.1939@\textsc{Beer-Hofmann, Paula} (25.02.1879 – 30.10.1939)|pw} sind hier in Weißenbach\oindex{Weissenbach am Attersee@\textbf{Weißenbach am Attersee}, \emph{A.ADM3}|pw}}{\lemma{\textnormal{\emph{Beer-Hofmanns … Weißenbach}}}\Cendnote{\textnormal{Vgl. Richard Beer-Hofmann an Arthur Schnitzler, 10. 8. 1914.
               }}}\label{K_L03565-4}. Ich glaube, sie
               sind dort fast die einzigen. Wir sehen uns manchmal. Lassen Sie mich wißen, wie es
               bei Ihnen geht. Viele herzlichste Grüße von uns an Sie Beide\pwindex{Schnitzler, Olga 17.01.1882 – 13.01.1970@\textsc{Schnitzler, Olga} (17.01.1882 – 13.01.1970), \emph{Schauspieler/Schauspielerin, Sänger/Sängerin}|pwv} und die Kinder\pwindex{Schnitzler, Heinrich 09.08.1902 – 12.07.1982@\textsc{Schnitzler, Heinrich} (09.08.1902 – 12.07.1982), \emph{Regisseur/Regisseurin, Schauspieler/Schauspielerin}|pwv}\pwindex{Cappellini, Lili 13.09.1909 – 26.07.1928@\textsc{Cappellini, Lili} (13.09.1909 – 26.07.1928)|pwv}!\pend
           \pstart Ihr \spacefill\mbox{Salten}\pend{}\selectlanguage{ngerman}\endnumbering\briefempfaengerindex{Schnitzler, Arthur@\textsc{Schnitzler, Arthur}!zzzSalten, Felix@\emph{von Felix Salten}!1914-08-102@{10. 8. 1914}|)be}\mylabel{L03565h}  \normalsize

\doendnotes{C}
\bigskip
\vfill

\clearpage

\footnotesize

\lohead{\textsc{register}}

% Definiere theindex-Environment komplett neu ohne reledmac
\makeatletter
\renewenvironment{theindex}{%
  \section*{\indexname}%
  \setlength{\parindent}{0pt}%
  \setlength{\parskip}{0pt plus 0.3pt}%
  \let\item\@idxitem
}{%
  \clearpage
}
\makeatother

\IfFileExists{\jobname-pw.ind}{\input{\jobname-pw.ind}}{}

\end{document}

      