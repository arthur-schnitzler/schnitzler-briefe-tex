%% latex-leseansicht-vorspann.tex
%% Vorspann für die Leseansicht.
%% Lädt die gemeinsame Datei latex-vorspann.tex mit nicht gesetztem Schalter.

\newif\ifkorrekturansicht
\korrekturansichtfalse

\input{../tex-inputs/latex-vorspann}


\section[ Felix Salten an Arthur Schnitzler, 10. 8. 1914]{L03565 Felix Salten an Arthur Schnitzler,  10. 8. 1914}
\nopagebreak\mylabel{L03565v}
\rehead{ }\normalsize\beginnumbering\briefempfaengerindex{Schnitzler, Arthur@\textsc{Schnitzler, Arthur}!zzzSalten, Felix@\emph{von Felix Salten}!1914-08-102@{10. 8. 1914}|(be}
\toendnotes[C]{\smallbreak\pagebreak[2]}
\correspDesc{Versand  durch Felix Salten am 10. 8. 1914 in Unterach am Attersee
\newline{}Erhalt  durch Arthur Schnitzler im Zeitraum [11. 8. 1914
                  – 15. 8. 1914?] in Wien}\toendnotes[C]{\smallbreak}
\Standort{CUL, Schnitzler, B 89, B 2.}
\physDesc{Briefkarte, 890 Zeichen
\newline{}Handschrift: schwarze Tinte, lateinische Kurrent
\newline{}Schnitzler: 1) mit Bleistift Vermerk: »\textsc{Salten}«  2) mit rotem Buntstift eine Unterstreichung
\newline{}Ordnung: mit Bleistift von unbekannter Hand nummeriert: »278« }\toendnotes[C]{\smallbreak}
\pstart
           \raggedleft{}{\pb}Berghof\oindex{Berghof@\textbf{Berghof}, \emph{Wohngebäude}|pw}, 10. 8. 14\pend
           
\pstart{}Lieber,\pend\vspace{0.5em}
\pstart
           Ihre Karte aus der \label{K_L03565-1v}\edtext{Schweiz\oindex{Schweiz@\textbf{Schweiz}|pw}}{\lemma{\textnormal{\emph{Schweiz}}}\Cendnote{\textnormal{Schnitzler war am 18. 7. 1914 mit seiner
                     Frau\pwindex{Schnitzler, Olga 17.\,1.\,1882 Wien – 13.\,1.\,1970 Lugano@\textsc{Schnitzler, Olga} (17.\,1.\,1882 Wien – 13.\,1.\,1970 Lugano), \emph{Schauspielerin, Sängerin}|pwkv} und den Kindern\pwindex{Schnitzler, Heinrich 9.\,8.\,1902 Hinterbrühl – 12.\,7.\,1982 Wien@\textsc{Schnitzler, Heinrich} (9.\,8.\,1902 Hinterbrühl – 12.\,7.\,1982 Wien), \emph{Regisseur, Schauspieler}|pwkv}\pwindex{Cappellini, Lili 13.\,9.\,1909 Wien – 26.\,7.\,1928 Venedig@\textsc{Cappellini, Lili} (13.\,9.\,1909 Wien – 26.\,7.\,1928 Venedig)|pwkv} in der
                     Schweiz\oindex{Schweiz@\textbf{Schweiz}|pwk} angekommen. Die Heimreise nach
                  Kriegsausbruch erwies sich als schwierig. Am 15. 8. 1914 reisten sie nach Österreich\oindex{Österreich@\textbf{Österreich}|pwk}, zuerst aber nach Bad Ischl\oindex{Bad Ischl@\textbf{Bad Ischl}|pwk} (vgl. XXXX Auszeichnungsfehler: Dokument L02192 nicht gefunden). Am 2. 9. 1914 waren sie wieder in Wien\oindex{Wien@\textbf{Wien}, \emph{Verwaltungsgebiet}|pwk}.}}}\label{K_L03565-1} bekam ich vor zwei Tagen, nehme aber an, dass
               Sie jetzt wieder zu Hause sind. Wann ich nach Wien\oindex{Wien@\textbf{Wien}, \emph{Verwaltungsgebiet}|pw}
               komme, weiß ich nicht, weiß nicht einmal, ob ich es soll. Hier\oindex{Unterach am Attersee@\textbf{Unterach am Attersee}|pwv} ist es so ganz still, ganz einsam und
               das beruhigt einigermaßen. Sonst – wenn man sich’s klar macht, was jetzt geschieht
               und warum es geschieht – könnte man verzweifeln. Wer dran glaubt, \label{K_L03565-2v}\edtext{dies alles sei wegen Serbien\oindex{Serbien@\textbf{Serbien}|pw}}{\lemma{\textnormal{\emph{dies … Serbien}}}\Cendnote{\textnormal{Das Attentat von Sarajevo\oindex{Sarajevo@\textbf{Sarajevo}, \emph{Hauptstadt}|pwk} wurde in Zusammenhang mit Bestrebungen Serbiens\oindex{Serbien@\textbf{Serbien}|pwk} gesehen, das eine politische Einigung
                  am Balkan\oindex{Balkanhalbinsel@\textbf{Balkanhalbinsel}|pwk} unter seiner Führung
                  anstrebte.}}}\label{K_L03565-2}, ist eigentlich zu beneiden. Denn er hat doch etwas, um sein
               Rechtsgefühl damit zu füttern. Vielleicht ist es gut, dass dieser Krieg eben \label{K_L03565-3v}\edtext{jetzt ausgebrochen}{\lemma{\textnormal{\emph{jetzt ausgebrochen}}}\Cendnote{\textnormal{Österreich\oindex{Österreich@\textbf{Österreich}|pwk} hatte Serbien\oindex{Serbien@\textbf{Serbien}|pwk} am 28. 7. 1914 den
                  Krieg erklärt. Damit hatte der Erste Weltkrieg begonnen.}}}\label{K_L03565-3} wird. Gut: für
               unsere Söhne. Das mag hässlich und egoistisch gedacht sein, aber ich denke es eben.
                  \label{K_L03565-4v}\edtext{Beer-Hofmanns\pwindex{Beer-Hofmann, Richard 11.\,7.\,1866 Wien – 26.\,9.\,1945 New York City@\textsc{Beer-Hofmann, Richard} (11.\,7.\,1866 Wien – 26.\,9.\,1945 New York City), \emph{Schriftsteller}|pw}\pwindex{Beer-Hofmann, Paula 25.\,2.\,1879 Wien – 30.\,10.\,1939 Zürich@\textsc{Beer-Hofmann, Paula} (25.\,2.\,1879 Wien – 30.\,10.\,1939 Zürich)|pw} sind hier in Weißenbach\oindex{Weißenbach am Attersee@\textbf{Weißenbach am Attersee}, \emph{Verwaltungsgebiet}|pw}}{\lemma{\textnormal{\emph{Beer-Hofmanns … Weißenbach}}}\Cendnote{\textnormal{Vgl. XXXX Auszeichnungsfehler: Dokument L02191 nicht gefunden.
               }}}\label{K_L03565-4}. Ich glaube, sie
               sind dort fast die einzigen. Wir sehen uns manchmal. Lassen Sie mich wißen, wie es
               bei Ihnen geht. Viele herzlichste Grüße von uns an Sie Beide\pwindex{Schnitzler, Olga 17.\,1.\,1882 Wien – 13.\,1.\,1970 Lugano@\textsc{Schnitzler, Olga} (17.\,1.\,1882 Wien – 13.\,1.\,1970 Lugano), \emph{Schauspielerin, Sängerin}|pwv} und die Kinder\pwindex{Schnitzler, Heinrich 9.\,8.\,1902 Hinterbrühl – 12.\,7.\,1982 Wien@\textsc{Schnitzler, Heinrich} (9.\,8.\,1902 Hinterbrühl – 12.\,7.\,1982 Wien), \emph{Regisseur, Schauspieler}|pwv}\pwindex{Cappellini, Lili 13.\,9.\,1909 Wien – 26.\,7.\,1928 Venedig@\textsc{Cappellini, Lili} (13.\,9.\,1909 Wien – 26.\,7.\,1928 Venedig)|pwv}!\pend
           \pstart Ihr \spacefill\mbox{Salten}\pend{}\selectlanguage{ngerman}\endnumbering\briefempfaengerindex{Schnitzler, Arthur@\textsc{Schnitzler, Arthur}!zzzSalten, Felix@\emph{von Felix Salten}!1914-08-102@{10. 8. 1914}|)be}\mylabel{L03565h}  \newcommand{\dateiname}{L03565}\newcommand{\titel}{Felix Salten an Arthur Schnitzler, 10. 8. 1914}\newcommand{\editorInnen}{Martin Anton Müller und Laura Untner}%% latex-leseansicht-abspann.tex
%% Abspann für die Leseansicht.
%% Der Schalter \ifkorrekturansicht ist bereits durch den Vorspann gesetzt.

%% latex-abspann.tex
%% Gemeinsamer Abspann für Korrekturansicht und Leseansicht.
%% Setzt den Schalter \ifkorrekturansicht voraus (gesetzt in den
%% einbindenden Dateien latex-korrekturansicht-abspann.tex bzw.
%% latex-leseansicht-abspann.tex).
%% ---------------------------------------------------------------

\normalsize

% Das esempio-Environment wird nur in der Leseansicht benötigt
\ifkorrekturansicht\else
\newenvironment{esempio}[3]%
{
    \vspace{1.5ex}
    \rlap{\underline{#1}}
    \par
    \setlength{\parindent}{0cm}
    \nopagebreak
    \leftskip=#2cm
    \rightskip=#3cm
}
{
    \par
}
\fi

\doendnotes{C}
\bigskip
\vfill

\clearpage

\footnotesize

\ifkorrekturansicht
  \lohead{\textsc{register}}
\fi

% theindex-Environment neu definieren ohne reledmac
\makeatletter
\renewenvironment{theindex}{%
  \ifkorrekturansicht
    \section*{\indexname}%
  \else
    \subsubsection*{Index der erwähnten Entitäten}%
  \fi
  \setlength{\parindent}{0pt}%
  \setlength{\parskip}{0pt plus 0.3pt}%
  \let\item\@idxitem
}{%
  \ifkorrekturansicht\clearpage\fi
}
\makeatother

\IfFileExists{\jobname-pw.ind}{\input{\jobname-pw.ind}}{}

% Quellenangabe nur in der Leseansicht
\ifkorrekturansicht\else
% Fallback-Definitionen, falls die .tex-Datei \titel etc. nicht gesetzt hat
\providecommand{\titel}{}
\providecommand{\editorInnen}{}
\providecommand{\dateiname}{\jobname}

\vspace{3cm}

\vfill

\footnotesize
\textsc{Quelle}: \titel. Herausgegeben von {\editorInnen}. In: \emph{Arthur Schnitzler: Briefwechsel mit Autorinnen und Autoren}.
 Digitale Edition, https://schnitzler-briefe.acdh.oeaw.ac.at/{\dateiname}.html (Stand \today)
\fi

\end{document}


