%% latex-leseansicht-vorspann.tex
%% Vorspann für die Leseansicht.
%% Lädt die gemeinsame Datei latex-vorspann.tex mit nicht gesetztem Schalter.

\newif\ifkorrekturansicht
\korrekturansichtfalse

\input{../tex-inputs/latex-vorspann}


\section[Peter Altenberg an Arthur Schnitzler, {[}26. 4. 1913{]}]{L02133 Peter Altenberg an Arthur Schnitzler, {[}26. 4. 1913{]}}
\nopagebreak\mylabel{L02133v}
\rehead{ }\normalsize\beginnumbering\briefempfaengerindex{Schnitzler, Arthur@\textsc{Schnitzler, Arthur}!zzzAltenberg, Peter@\emph{von Peter Altenberg}!1913-04-261@{{[}26. 4. 1913{]}}|(be}
\toendnotes[C]{\smallbreak\pagebreak[2]}
\correspDesc{Versand  durch Peter Altenberg am [26. 4. 1913] in Wien
\newline{}Erhalt  durch Arthur Schnitzler im Zeitraum [26. 4. 1913
                  – 30. 4. 1913?] in Wien}\toendnotes[C]{\smallbreak}
\Standort{CUL, Schnitzler, B 2.}
\physDesc{Brief, 1 Blatt, 3 Seiten, 1024 Zeichen
\newline{}Handschrift: schwarze Tinte, deutsche Kurrent
\newline{}Schnitzler: mit Bleistift datiert: »\substVorne{}\textsuperscript{4}\substDazwischen{}6\substHinten{}/4 913« 
\newline{}Ordnung: von unbekannter Hand nummeriert: »13« }\toendnotes[C]{\smallbreak}
\pstart{}{\pb}Lieber beſter \textsc{D\textsuperscript{r}} Arthur Schnitzler,\pend\vspace{0.5em}
\pstart
           bitte, das hätten Sie nicht{ }ſagen{ }ſollen, daſs ich drauſſen wieder \uline{eventuell} zu trinken anfangen könnte! Daran \uline{klammert} man{ }ſich jetzt. Ich habe \uline{5} Monate\strikeout{ll} lang gar nicht eine Sekunde lang
               an Alkohol oder{ }ſelbſt Bier, gedacht, ich entbehre es nicht, war \uline{nie} ein Alkoholiker,{ }ſondern nahm es als Schlafmittel.\pend
           
\pstart
           Jeder Tag länger hier, jede \uline{aus Verzweiflung über das
                  Hierſein},{ }ſchlaflos, in \uuline{\edtext{Seelen-Noth}{\Cendnote{dreifach unterstrichen}}}
               verbrachte Nacht, verhindert künſtlich meine eingetretene {\pb}\textsc{Reconvalescenz}! Das bitte, wiederholen Sie eindringlich,{ }ſchriftlich, dem Herrn \textsc{Primarius}{ }Richter\pwindex{Richter, Karl 9.\,3.\,1862 Bruntál – 25.\,6.\,1937 Wien@\textsc{Richter, Karl} (9.\,3.\,1862 Bruntál – 25.\,6.\,1937 Wien), \emph{Mediziner, Sanatoriumsleiter}|pw}! Dadurch erretten Sie mich vor den
               Martern des Zuwartens! Man will mich heimtückiſcher Weiſe (mein Bruder\pwindex{Engländer, Georg 3.\,4.\,1862 Wien – 10.\,4.\,1927 ebd.@\textsc{Engländer, Georg} (3.\,4.\,1862 Wien – 10.\,4.\,1927 ebd.), \emph{Privatbeamter}|pwv}) durch dieſes Zuwarten in einen
               neuerlichen Zuſtand von Nerven-Erſchöpfung und Überreizung bringen, um dadurch eine
                  {\pb}Gelegenheit zu haben, mich weiter in
               dieſem \uuline{ſchrecklichen}{ }Kerker\oindex{Wien@\textbf{Wien}!XIV., Penzing@\textbf{XIV., Penzing}!Otto-Wagner-Spital@\textbf{Otto-Wagner-Spital}, \emph{Krankenhaus}|pwv} feſtzuhalten!\pend
           
\pstart
           Erretten Sie mich, \uuline{\edtext{befreien}{\Cendnote{dreifach unterstrichen}}} Sie mich, durch Ihre
               Mitteilung an den Primarius Richter\pwindex{Richter, Karl 9.\,3.\,1862 Bruntál – 25.\,6.\,1937 Wien@\textsc{Richter, Karl} (9.\,3.\,1862 Bruntál – 25.\,6.\,1937 Wien), \emph{Mediziner, Sanatoriumsleiter}|pw}, der mich
               fragte, was \uline{Sie} davon hielten?!?\pend
           
\pstart
           Ihr ewig dankbarer{\\[\baselineskip]}\spacefill\mbox{Peter Altenberg}\pend
           \leftskip=0em{}\selectlanguage{ngerman}\endnumbering\briefempfaengerindex{Schnitzler, Arthur@\textsc{Schnitzler, Arthur}!zzzAltenberg, Peter@\emph{von Peter Altenberg}!1913-04-261@{{[}26. 4. 1913{]}}|)be}\mylabel{L02133h}  \newcommand{\dateiname}{L02133}\newcommand{\titel}{Peter Altenberg an Arthur Schnitzler, [26. 4. 1913]}\newcommand{\editorInnen}{Martin Anton Müller und Gerd-Hermann Susen}%% latex-leseansicht-abspann.tex
%% Abspann für die Leseansicht.
%% Der Schalter \ifkorrekturansicht ist bereits durch den Vorspann gesetzt.

%% latex-abspann.tex
%% Gemeinsamer Abspann für Korrekturansicht und Leseansicht.
%% Setzt den Schalter \ifkorrekturansicht voraus (gesetzt in den
%% einbindenden Dateien latex-korrekturansicht-abspann.tex bzw.
%% latex-leseansicht-abspann.tex).
%% ---------------------------------------------------------------

\normalsize

% Das esempio-Environment wird nur in der Leseansicht benötigt
\ifkorrekturansicht\else
\newenvironment{esempio}[3]%
{
    \vspace{1.5ex}
    \rlap{\underline{#1}}
    \par
    \setlength{\parindent}{0cm}
    \nopagebreak
    \leftskip=#2cm
    \rightskip=#3cm
}
{
    \par
}
\fi

\doendnotes{C}
\bigskip
\vfill

\clearpage

\footnotesize

\ifkorrekturansicht
  \lohead{\textsc{register}}
\fi

% theindex-Environment neu definieren ohne reledmac
\makeatletter
\renewenvironment{theindex}{%
  \ifkorrekturansicht
    \section*{\indexname}%
  \else
    \subsubsection*{Index der erwähnten Entitäten}%
  \fi
  \setlength{\parindent}{0pt}%
  \setlength{\parskip}{0pt plus 0.3pt}%
  \let\item\@idxitem
}{%
  \ifkorrekturansicht\clearpage\fi
}
\makeatother

\IfFileExists{\jobname-pw.ind}{\input{\jobname-pw.ind}}{}

% Quellenangabe nur in der Leseansicht
\ifkorrekturansicht\else
% Fallback-Definitionen, falls die .tex-Datei \titel etc. nicht gesetzt hat
\providecommand{\titel}{}
\providecommand{\editorInnen}{}
\providecommand{\dateiname}{\jobname}

\vspace{3cm}

\vfill

\footnotesize
\textsc{Quelle}: \titel. Herausgegeben von {\editorInnen}. In: \emph{Arthur Schnitzler: Briefwechsel mit Autorinnen und Autoren}.
 Digitale Edition, https://schnitzler-briefe.acdh.oeaw.ac.at/{\dateiname}.html (Stand \today)
\fi

\end{document}


