%% latex-korrekturansicht-vorspann.tex
%% Vorspann für die Korrekturansicht.
%% Lädt die gemeinsame Datei latex-vorspann.tex mit gesetztem Schalter.

\newif\ifkorrekturansicht
\korrekturansichttrue

\input{../tex-inputs/latex-vorspann}


\section[Hugo von Hofmannsthal an Arthur Schnitzler, 7. 10. 1902]{L01239 Hugo von Hofmannsthal an Arthur Schnitzler, 7. 10. 1902}
\nopagebreak\mylabel{L01239v}
\rehead{ }\normalsize\beginnumbering\briefempfaengerindex{Schnitzler, Arthur@\textsc{Schnitzler, Arthur}!zzzHofmannsthal, Hugo von@\emph{von Hugo von Hofmannsthal}!1902-10-072@{7. 10. 1902}|(be}
\toendnotes[C]{\smallbreak\pagebreak[2]}\Standort{CUL, Schnitzler, B 43.}
\physDesc{Bildpostkarte, 189 Zeichen
\newline{}Handschrift: Bleistift, deutsche Kurrent
\newline{}Versand: 1) Stempel: »\nobreak{}\oindex{Roma Termini@\textbf{Roma Termini}, \emph{S.RSTN}|pwk}Roma Ferrovia, 7 10 02, 11S\nobreak{}«.   2) Stempel: »\nobreak{}\oindex{IX., Alsergrund@\textbf{IX., Alsergrund}, \emph{A.ADM3}|pwk}Wien 9/3, 9. 10. 02, 11.V, Bestellt\nobreak{}«. 
\newline{}Ordnung: 1) mit Bleistift von unbekannter Hand nummeriert:
                                    »302«  2) mit Bleistift von unbekannter Hand nummeriert:
                                    »186«}
\buchAbdrucke{\weitereDrucke{Hugo von Hofmannsthal, Arthur Schnitzler: \emph{Briefwechsel}. Frankfurt am Main: \emph{S. Fischer} 1964, S. 162.} }\toendnotes[C]{\smallbreak}\pstart{}{\pb}\textsc{Herrn D\textsuperscript{r} Arthur Schnitzler}\pend{}\pstart{}\textsc{Austria\oindex{Oesterreich@\textbf{Österreich}, \emph{A.PCLI}|pw}}\pend{}\pstart{}\textsc{Wien}\oindex{Wien@\textbf{Wien}, \emph{A.ADM2}|pw}\pend{}\pstart{}\textsc{IX Franckgasse 1}\oindex{Frankgasse 1@\textbf{Frankgasse 1}, \emph{Wohngebäude (K.WHS)}|pw}.\pend{}{\bigskip}
\pstart
           \noindent{}\centering{}{\pb}\textcolor{gray}{\textbf{Roma\oindex{Rom@\textbf{Rom}, \emph{P.PPLC}|pw}{ }Monte Pincio\oindex{Monte Pincio@\textbf{Monte Pincio}, \emph{Berg (N.BRG)}|pw} – passeggiata}}\pend
           \vspace{1em}
\pstart
           \raggedleft{}{\pb}7/X.{\\}\textsc{Hôtel Hassler}\oindex{Hôtel Hassler@\textbf{Hôtel Hassler}, \emph{Hotel (K.HTL)}|pw}.\pend
           \vspace{0.5em}
\pstart
           Es fällt mir der ſchöne \label{K_L01239-1v}\edtext{Tag}{\lemma{\textnormal{\emph{Tag}}}\Cendnote{\textnormal{Siehe A. S.: \emph{Tagebuch}, 3. 7. 1902.
               }}}\label{K_L01239-1} ein, wo wir durchs Stubaithal\oindex{Stubaital@\textbf{Stubaital}, \emph{T.VAL}|pw} gegangen
               ſind und ſo viel geredet haben. Wie gehts Ihnen?\pend
           
\pstart
           Ihr{\\[\baselineskip]}\spacefill\mbox{Hugo}\pend
           \leftskip=0em{}\selectlanguage{ngerman}\endnumbering\briefempfaengerindex{Schnitzler, Arthur@\textsc{Schnitzler, Arthur}!zzzHofmannsthal, Hugo von@\emph{von Hugo von Hofmannsthal}!1902-10-072@{7. 10. 1902}|)be}\mylabel{L01239h}  \normalsize

\doendnotes{C}
\bigskip
\vfill

\clearpage

\footnotesize

\lohead{\textsc{register}}

% Definiere theindex-Environment komplett neu ohne reledmac
\makeatletter
\renewenvironment{theindex}{%
  \section*{\indexname}%
  \setlength{\parindent}{0pt}%
  \setlength{\parskip}{0pt plus 0.3pt}%
  \let\item\@idxitem
}{%
  \clearpage
}
\makeatother

\IfFileExists{\jobname-pw.ind}{\input{\jobname-pw.ind}}{}

\end{document}

      