%% latex-korrekturansicht-vorspann.tex
%% Vorspann für die Korrekturansicht.
%% Lädt die gemeinsame Datei latex-vorspann.tex mit gesetztem Schalter.

\newif\ifkorrekturansicht
\korrekturansichttrue

\input{../tex-inputs/latex-vorspann}


\section[Arthur Schnitzler an Hugo von Hofmannsthal, 18. 7. 1899]{L00947 Arthur Schnitzler an Hugo von Hofmannsthal, 18. 7. 1899}
\nopagebreak\mylabel{L00947v}
\rehead{ }\normalsize\beginnumbering\briefempfaengerindex{Hofmannsthal, Hugo von@\textsc{Hofmannsthal, Hugo von}!zzzSchnitzler, Arthur@\emph{von Arthur Schnitzler}!1899-07-182@{18. 7. 1899}|(be}
\toendnotes[C]{\smallbreak\pagebreak[2]}\Standort{FDH, Hs-30885,84.}
\physDesc{Briefkarte, 772 Zeichen
\newline{}Handschrift: Bleistift, deutsche Kurrent
\newline{}Ordnung: mit Bleistift die Jahreszahl ergänzt: »99« wahrscheinlich erst bei der Durchsicht der Briefe
                                    1929 ergänzt }
\buchAbdrucke{\weitereDrucke{Hugo von Hofmannsthal, Arthur Schnitzler: \emph{Briefwechsel}. Frankfurt am Main: \emph{S. Fischer} 1964, S. 126.} }\toendnotes[C]{\smallbreak}
\pstart
           \raggedleft{}{\pb}18. 7.\pend
           \vspace{0.5em}
\pstart
           lieber Hugo, ich bin heut Früh hier angeko{\geminationm}en. \introOben{}Meine\introOben{}{ }Mutter\pwindex{Schnitzler, Louise 1840-07-08 – 1911-09-09@\textsc{Schnitzler, Louise} (1840-07-08 – 1911-09-09)|pwv} und Schweſter\pwindex{Hajek, Gisela 20.12.1867 – 03.02.1953@\textsc{Hajek, Gisela} (20.12.1867 – 03.02.1953)|pwv} wohnen hier. – Habe
                  Nachmitt\textcolor{gray}{ag} mit Schwager\pwindex{Burger, Rudolf *~06.12.1866@\textsc{Burger, Rudolf} (*~06.12.1866), \emph{Versicherungsdirektor/Versicherungsdirektorin}|pwv} u Schweſter\pwindex{Burger, Caroline 11.07.1869 – 15.03.1959@\textsc{Burger, Caroline} (11.07.1869 – 15.03.1959)|pwv} (von \uline{ihr}\pwindex{Reinhard, Marie 1871-03-13 – 1899-03-18@\textsc{Reinhard, Marie} (1871-03-13 – 1899-03-18), \emph{Gesangspädagoge/Gesangspädagogin}|pwv}) am See ein Rendezvous. – Heut iſt der \uline{18}. – –
               Warte auf Nachricht von Richard\pwindex{Beer-Hofmann, Richard 1866-07-11 – 1945-09-26@\textsc{Beer-Hofmann, Richard} (1866-07-11 – 1945-09-26), \emph{Schriftsteller/Schriftstellerin}|pw}, ob er nicht
               arbeitet (eine Karte deutet es an) – bevor ich ihn beſuche. – Bleibe mindeſtens
               8 Tage hier. – Ob ich meine Radtour bis 1. Sept. hinausſchiebe,
               fraglich. – Auch Salten\pwindex{Salten, Felix 06.09.1869 – 08.10.1945@\textsc{Salten, Felix} (06.09.1869 – 08.10.1945), \emph{Schriftsteller/Schriftstellerin, Journalist/Journalistin, Chefredakteur/Chefredakteurin}|pw} wollte ſie mitmachen. –
               Keiner bindet {\pb}den andern. Im Auguſt{ }ſehn wir uns jedenfalls, ko{\geminationm}e ins Salzka{\geminationm}ergut\oindex{Salzkammergut@\textbf{Salzkammergut}, \emph{L.RGN}|pw} – wäre ſchön, we{\geminationn} wir zusa{\geminationm}en wären u jeder
               arbeitete.\pend
           
\pstart
           – Will jetzt gleich, in dieſer Minute, mein Stück\pwindex{Schleier der Beatrice. Schauspiel in fuenf Akten@\emph{Der Schleier der Beatrice. Schauspiel in fünf Akten}|pwv} hervornehmen. – Was iſt das Ihre\pwindex{Bergwerk zu Falun@\emph{Das Bergwerk zu Falun}|pwv}? Historisch? Was neues? Neue Idee? Ich
               freue mich dſs Sie in Sti{\geminationm}ung ſind. Bitte gleich wieder
               eine Zeile.\pend
           \pstart Von Herzen Ihr \spacefill\mbox{Arth}\pend{}
\pstart
           \noindent{}\textsc{Velden, Pension Pundschu}\oindex{Pension Pundschu@\textbf{Pension Pundschu}, \emph{Hotel (K.HTL)}|pw}\pend
           \selectlanguage{ngerman}\endnumbering\briefempfaengerindex{Hofmannsthal, Hugo von@\textsc{Hofmannsthal, Hugo von}!zzzSchnitzler, Arthur@\emph{von Arthur Schnitzler}!1899-07-182@{18. 7. 1899}|)be}\mylabel{L00947h}  \normalsize

\doendnotes{C}
\bigskip
\vfill

\clearpage

\footnotesize

\lohead{\textsc{register}}

% Definiere theindex-Environment komplett neu ohne reledmac
\makeatletter
\renewenvironment{theindex}{%
  \section*{\indexname}%
  \setlength{\parindent}{0pt}%
  \setlength{\parskip}{0pt plus 0.3pt}%
  \let\item\@idxitem
}{%
  \clearpage
}
\makeatother

\IfFileExists{\jobname-pw.ind}{\input{\jobname-pw.ind}}{}

\end{document}

      