%% latex-leseansicht-vorspann.tex
%% Vorspann für die Leseansicht.
%% Lädt die gemeinsame Datei latex-vorspann.tex mit nicht gesetztem Schalter.

\newif\ifkorrekturansicht
\korrekturansichtfalse

\input{../tex-inputs/latex-vorspann}


\section[Arthur Schnitzler an Hugo von Hofmannsthal, 18. 7. 1899]{L00947 Arthur Schnitzler an Hugo von Hofmannsthal, 18. 7. 1899}
\nopagebreak\mylabel{L00947v}
\rehead{ }\normalsize\beginnumbering\briefempfaengerindex{Hofmannsthal, Hugo von@\textsc{Hofmannsthal, Hugo von}!zzzSchnitzler, Arthur@\emph{von Arthur Schnitzler}!1899-07-182@{18. 7. 1899}|(be}
\toendnotes[C]{\smallbreak\pagebreak[2]}
\correspDesc{Versand  durch Arthur Schnitzler am 18. 7. 1899 in Velden am Wörthersee
\newline{}Erhalt  durch Hugo von Hofmannsthal im Zeitraum [19. 7. 1899
                  – 23. 7. 1899?] in Marienbad}\toendnotes[C]{\smallbreak}
\Standort{FDH, Hs-30885,84.}
\physDesc{Briefkarte, 772 Zeichen
\newline{}Handschrift: Bleistift, deutsche Kurrent
\newline{}Ordnung: mit Bleistift die Jahreszahl ergänzt: »99« wahrscheinlich erst bei der Durchsicht der Briefe
                                    1929 ergänzt }
\buchAbdrucke{\weitereDrucke{Hugo von Hofmannsthal, Arthur Schnitzler: \emph{Briefwechsel}. Herausgegeben von Therese Nickl und Heinrich Schnitzler. Frankfurt am Main: \emph{S. Fischer} 1964, S. 126.} }\toendnotes[C]{\smallbreak}
\pstart
           \raggedleft{}{\pb}18. 7.\pend
           \vspace{0.5em}
\pstart
           lieber Hugo, ich bin heut Früh hier angeko{\geminationm}en. \introOben{}Meine\introOben{}{ }Mutter\pwindex{Schnitzler, Louise 8.\,7.\,1840 Kőszeg – 9.\,9.\,1911 Wien@\textsc{Schnitzler, Louise} (8.\,7.\,1840 Kőszeg – 9.\,9.\,1911 Wien)|pwv} und Schweſter\pwindex{Hajek, Gisela 20.\,12.\,1867 Wien – 3.\,2.\,1953 Cambridge@\textsc{Hajek, Gisela} (20.\,12.\,1867 Wien – 3.\,2.\,1953 Cambridge)|pwv} wohnen hier. – Habe
                  Nachmitt\textcolor{gray}{ag} mit Schwager\pwindex{Burger, Rudolf *~6.\,12.\,1866 Wien@\textsc{Burger, Rudolf} (*~6.\,12.\,1866 Wien), \emph{Versicherungsdirektor}|pwv} u Schweſter\pwindex{Burger, Caroline 11.\,7.\,1869 Wien – 15.\,3.\,1959 ebd.@\textsc{Burger, Caroline} (11.\,7.\,1869 Wien – 15.\,3.\,1959 ebd.)|pwv} (von \uline{ihr}\pwindex{Reinhard, Marie 13.\,3.\,1871 Wien – 18.\,3.\,1899 ebd.@\textsc{Reinhard, Marie} (13.\,3.\,1871 Wien – 18.\,3.\,1899 ebd.), \emph{Gesangspädagogin}|pwv}) am See ein Rendezvous. – Heut iſt der \uline{18}. – –
               Warte auf Nachricht von Richard\pwindex{Beer-Hofmann, Richard 11.\,7.\,1866 Wien – 26.\,9.\,1945 New York City@\textsc{Beer-Hofmann, Richard} (11.\,7.\,1866 Wien – 26.\,9.\,1945 New York City), \emph{Schriftsteller}|pw}, ob er nicht
               arbeitet (eine Karte deutet es an) – bevor ich ihn beſuche. – Bleibe mindeſtens
               8 Tage hier. – Ob ich meine Radtour bis 1. Sept. hinausſchiebe,
               fraglich. – Auch Salten\pwindex{Salten, Felix 6.\,9.\,1869 Budapest – 8.\,10.\,1945 Zürich@\textsc{Salten, Felix} (6.\,9.\,1869 Budapest – 8.\,10.\,1945 Zürich), \emph{Schriftsteller, Journalist, Chefredakteur}|pw} wollte{ }ſie mitmachen. –
               Keiner bindet {\pb}den andern. Im Auguſt{ }ſehn wir uns jedenfalls, ko{\geminationm}e ins Salzka{\geminationm}ergut\oindex{Salzkammergut@\textbf{Salzkammergut}, \emph{Region}|pw} – wäre{ }ſchön, we{\geminationn} wir zusa{\geminationm}en wären u jeder
               arbeitete.\pend
           
\pstart
           – Will jetzt gleich, in dieſer Minute, mein Stück\pwindex{Schnitzler, Arthur 15.\,5.\,1862 Wien – 21.\,10.\,1931 ebd.@\textsc{Schnitzler, Arthur} (15.\,5.\,1862 Wien – 21.\,10.\,1931 ebd.), \emph{Schriftsteller, Mediziner}!Schleier der Beatrice. Schauspiel in fünf Akten@\strich\emph{Der Schleier der Beatrice. Schauspiel in fünf Akten}|pwv} hervornehmen. – Was iſt das Ihre\pwindex{Hofmannsthal, Hugo von 1.\,2.\,1874 Wien – 15.\,7.\,1929 Rodaun@\textsc{Hofmannsthal, Hugo von} (1.\,2.\,1874 Wien – 15.\,7.\,1929 Rodaun), \emph{Schriftsteller}!Bergwerk zu Falun@\strich\emph{Das Bergwerk zu Falun}|pwv}? Historisch? Was neues? Neue Idee? Ich
               freue mich dſs Sie in Sti{\geminationm}ung{ }ſind. Bitte gleich wieder
               eine Zeile.\pend
           \pstart Von Herzen Ihr \spacefill\mbox{Arth}\pend{}
\pstart
           \noindent{}\textsc{Velden, Pension Pundschu}\oindex{Pension Pundschu@\textbf{Pension Pundschu}, \emph{Hotel}|pw}\pend
           \selectlanguage{ngerman}\endnumbering\briefempfaengerindex{Hofmannsthal, Hugo von@\textsc{Hofmannsthal, Hugo von}!zzzSchnitzler, Arthur@\emph{von Arthur Schnitzler}!1899-07-182@{18. 7. 1899}|)be}\mylabel{L00947h}  \newcommand{\dateiname}{L00947}\newcommand{\titel}{Arthur Schnitzler an Hugo von Hofmannsthal, 18. 7. 1899}\newcommand{\editorInnen}{Martin Anton Müller und Gerd-Hermann Susen}%% latex-leseansicht-abspann.tex
%% Abspann für die Leseansicht.
%% Der Schalter \ifkorrekturansicht ist bereits durch den Vorspann gesetzt.

%% latex-abspann.tex
%% Gemeinsamer Abspann für Korrekturansicht und Leseansicht.
%% Setzt den Schalter \ifkorrekturansicht voraus (gesetzt in den
%% einbindenden Dateien latex-korrekturansicht-abspann.tex bzw.
%% latex-leseansicht-abspann.tex).
%% ---------------------------------------------------------------

\normalsize

% Das esempio-Environment wird nur in der Leseansicht benötigt
\ifkorrekturansicht\else
\newenvironment{esempio}[3]%
{
    \vspace{1.5ex}
    \rlap{\underline{#1}}
    \par
    \setlength{\parindent}{0cm}
    \nopagebreak
    \leftskip=#2cm
    \rightskip=#3cm
}
{
    \par
}
\fi

\doendnotes{C}
\bigskip
\vfill

\clearpage

\footnotesize

\ifkorrekturansicht
  \lohead{\textsc{register}}
\fi

% theindex-Environment neu definieren ohne reledmac
\makeatletter
\renewenvironment{theindex}{%
  \ifkorrekturansicht
    \section*{\indexname}%
  \else
    \subsubsection*{Index der erwähnten Entitäten}%
  \fi
  \setlength{\parindent}{0pt}%
  \setlength{\parskip}{0pt plus 0.3pt}%
  \let\item\@idxitem
}{%
  \ifkorrekturansicht\clearpage\fi
}
\makeatother

\IfFileExists{\jobname-pw.ind}{\input{\jobname-pw.ind}}{}

% Quellenangabe nur in der Leseansicht
\ifkorrekturansicht\else
% Fallback-Definitionen, falls die .tex-Datei \titel etc. nicht gesetzt hat
\providecommand{\titel}{}
\providecommand{\editorInnen}{}
\providecommand{\dateiname}{\jobname}

\vspace{3cm}

\vfill

\footnotesize
\textsc{Quelle}: \titel. Herausgegeben von {\editorInnen}. In: \emph{Arthur Schnitzler: Briefwechsel mit Autorinnen und Autoren}.
 Digitale Edition, https://schnitzler-briefe.acdh.oeaw.ac.at/{\dateiname}.html (Stand \today)
\fi

\end{document}


