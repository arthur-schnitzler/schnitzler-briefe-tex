%% latex-leseansicht-vorspann.tex
%% Vorspann für die Leseansicht.
%% Lädt die gemeinsame Datei latex-vorspann.tex mit nicht gesetztem Schalter.

\newif\ifkorrekturansicht
\korrekturansichtfalse

\input{../tex-inputs/latex-vorspann}

\begin{center}
            \textcolor{red}{ENTWURF. ENTZIFFERUNG NOCH NICHT KORREKTURGELESEN}
                      \end{center}
            
               \section[Hugo von Hofmannsthal an Arthur Schnitzler, 28. 12. 1906]{ Hugo von Hofmannsthal an Arthur Schnitzler, 28. 12. 1906}\nopagebreak\mylabel{v}\rehead{ }\begin{ledgroupsized}[t]{13cm}\normalsize\beginnumbering\briefempfaengerindex{Schnitzler, Arthur@\textsc{Schnitzler, Arthur}!zzzHofmannsthal, Hugo von@\emph{von Hugo von Hofmannsthal}!1906-12-281@{28. 12. 1906}|(be} \toendnotes[C]{\smallbreak\pagebreak[2]} \Standort{CUL, Schnitzler, B 43.}
\physDesc{Postkarte
\newline{}Handschrift: schwarze Tinte, deutsche Kurrent\newline{}Versand: 1) Rohrpost 2) Stempel: »\nobreak{}\oindex{I., Innere Stadt@\textbf{I., Innere Stadt}|pwk}1/1 Wien, 29 XII 06, 10 20V\nobreak{}«. 3) Stempel: »\nobreak{}\oindex{XVIII., Waehring@\textbf{XVIII., Währing}|pwk}18/1 Wien 110, 229 XII 06, 11–V\nobreak{}«. 4) Stempel: »\nobreak{}\oindex{XVIII., Waehring@\textbf{XVIII., Währing}|pwk}18/\textsubscript{1}
                                       Wien, 29 XII 06, XI\textsuperscript{50}\nobreak{}«. \newline{}Ordnung: 1) mit Bleistift von unbekannter Hand nummeriert: »\strikeout{272}« 2) mit Bleistift von unbekannter Hand nummeriert:
                                    »269«}\buchAbdrucke{\weitereDrucke{Hugo von Hofmannsthal, Arthur Schnitzler: \emph{Briefwechsel}. Hg. Therese Nickl und Heinrich Schnitzler. Frankfurt am Main: \emph{S. Fischer} 1964, S. 225.} }\toendnotes[C]{\smallbreak}\pstart{}{\pb}Herrn D\textsuperscript{r} Arthur Schnitzler\pend{}\pstart{}Wien\oindex{Wien@\textbf{Wien}|pw}\pend{}\pstart{}XVII Spöttelgasse 7\oindex{Edmund-Weiss-Gasse@\textbf{Edmund-Weiß-Gasse}|pw}.\pend{}{\bigskip}\pstart
           \raggedleft{}{\pb}28 XII.\pend
           \pstart
           lieber, ſehr lieb und gut daſs Sie ko{\geminationm}en wollen, aber unter dieſen Umſtänden erwarten wir Sie \uuline{nicht}, denn gerade Allein-herüber-fahren iſt das Langweilige und
               Unerfreuliche, beſonders in der Dunkelheit, das wollen wir nicht, alſo bald ein
               andres Mal Ihr beide.\pend
           \pstart
           Bitte den Abend des \uline{17}\textsuperscript{ten} Jänner freihalten für meinen
               (nicht-öffentlichen) \label{K_L01646_1v}\edtext{Vortrag\pwindex{Hofmannsthal, Hugo von 01.02.1874 – 15.07.1929@\textsc{Hofmannsthal, Hugo von} (01.02.1874 – 15.07.1929), \emph{Schriftsteller}!Dichter und diese Zeit30.11.1906 – 30.11.1906@\strich\emph{Der Dichter und diese Zeit} {[}30.11.1906 – 30.11.1906{]}|pw}}{\lemma{\textnormal{\emph{Vortrag}}}\Cendnote{\textnormal{Am 17. 1. 1907 hielt Hofmannsthal\pwindex{Hofmannsthal, Hugo von 01.02.1874 – 15.07.1929@\textsc{Hofmannsthal, Hugo von} (01.02.1874 – 15.07.1929), \emph{Schriftsteller}|pwk} den Vortrag \emph{Der Dichter und diese Zeit}\pwindex{Hofmannsthal, Hugo von 01.02.1874 – 15.07.1929@\textsc{Hofmannsthal, Hugo von} (01.02.1874 – 15.07.1929), \emph{Schriftsteller}!Dichter und diese Zeit30.11.1906 – 30.11.1906@\strich\emph{Der Dichter und diese Zeit} {[}30.11.1906 – 30.11.1906{]}|pwk} im \emph{Kunstsalon Miethke}\orgindex{Galerie Miethke@Galerie Miethke|pwk} vor geladenen, zehn Kronen zahlenden Gästen.}}}\label{K_L01646_1h}. Ihr
                  beko{\geminationm}t Eure Plätze direct von mir.\pend
           \pstart Ihr \spacefill\mbox{Hugo}\pend{}\pstart
           \noindent{}Wir ko{\geminationm}en baldmöglichſt zu Euch.\pend
           \endnumbering\briefempfaengerindex{Schnitzler, Arthur@\textsc{Schnitzler, Arthur}!zzzHofmannsthal, Hugo von@\emph{von Hugo von Hofmannsthal}!1906-12-281@{28. 12. 1906}|)be}\mylabel{h}\end{ledgroupsized}  \newcommand{\dateiname}{L01646}\newcommand{\titel}{Hugo von Hofmannsthal an Arthur Schnitzler, 28. 12. 1906}\newcommand{\editorInnen}{Martin Anton Müller und Gerd-Hermann Susen}%% latex-leseansicht-abspann.tex
%% Abspann für die Leseansicht.
%% Der Schalter \ifkorrekturansicht ist bereits durch den Vorspann gesetzt.

%% latex-abspann.tex
%% Gemeinsamer Abspann für Korrekturansicht und Leseansicht.
%% Setzt den Schalter \ifkorrekturansicht voraus (gesetzt in den
%% einbindenden Dateien latex-korrekturansicht-abspann.tex bzw.
%% latex-leseansicht-abspann.tex).
%% ---------------------------------------------------------------

\normalsize

% Das esempio-Environment wird nur in der Leseansicht benötigt
\ifkorrekturansicht\else
\newenvironment{esempio}[3]%
{
    \vspace{1.5ex}
    \rlap{\underline{#1}}
    \par
    \setlength{\parindent}{0cm}
    \nopagebreak
    \leftskip=#2cm
    \rightskip=#3cm
}
{
    \par
}
\fi

\doendnotes{C}
\bigskip
\vfill

\clearpage

\footnotesize

\ifkorrekturansicht
  \lohead{\textsc{register}}
\fi

% theindex-Environment neu definieren ohne reledmac
\makeatletter
\renewenvironment{theindex}{%
  \ifkorrekturansicht
    \section*{\indexname}%
  \else
    \subsubsection*{Index der erwähnten Entitäten}%
  \fi
  \setlength{\parindent}{0pt}%
  \setlength{\parskip}{0pt plus 0.3pt}%
  \let\item\@idxitem
}{%
  \ifkorrekturansicht\clearpage\fi
}
\makeatother

\IfFileExists{\jobname-pw.ind}{\input{\jobname-pw.ind}}{}

% Quellenangabe nur in der Leseansicht
\ifkorrekturansicht\else
% Fallback-Definitionen, falls die .tex-Datei \titel etc. nicht gesetzt hat
\providecommand{\titel}{}
\providecommand{\editorInnen}{}
\providecommand{\dateiname}{\jobname}

\vspace{3cm}

\vfill

\footnotesize
\textsc{Quelle}: \titel. Herausgegeben von {\editorInnen}. In: \emph{Arthur Schnitzler: Briefwechsel mit Autorinnen und Autoren}.
 Digitale Edition, https://schnitzler-briefe.acdh.oeaw.ac.at/{\dateiname}.html (Stand \today)
\fi

\end{document}


      