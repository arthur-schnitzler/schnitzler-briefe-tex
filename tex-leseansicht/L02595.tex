%% latex-leseansicht-vorspann.tex
%% Vorspann für die Leseansicht.
%% Lädt die gemeinsame Datei latex-vorspann.tex mit nicht gesetztem Schalter.

\newif\ifkorrekturansicht
\korrekturansichtfalse

\input{../tex-inputs/latex-vorspann}


\section[Marie Herzfeld an Arthur Schnitzler, 10.\,3.\,1931]{L02595 Marie Herzfeld an Arthur Schnitzler, 10.\,3.\,1931}
\nopagebreak\mylabel{L02595v}
\rehead{ }\normalsize\beginnumbering\briefempfaengerindex{Schnitzler, Arthur@\textsc{Schnitzler, Arthur}!zzzHerzfeld, Marie@\emph{von Marie Herzfeld}!1931-03-101@{10.\,3.\,1931}|(be}
\toendnotes[C]{\smallbreak\pagebreak[2]}
\correspDesc{Versand  durch Marie Herzfeld am 10. 3. 1931 in Wien
\newline{}Erhalt  durch Arthur Schnitzler im Zeitraum [10. 3. 1931
                  – 14. 3. 1931?] in Wien}\toendnotes[C]{\smallbreak}
\Standort{DLA, A:Schnitzler, HS.1985.1.3436,6.}
\physDesc{Brief, 1 Blatt, 2 Seiten, 620 Zeichen (Briefpapier mit Trauerrand)
\newline{}Handschrift: schwarze Tinte, lateinische Kurrent
\newline{}Schnitzler: mit rotem Buntstift Vermerk »\textsc{Herzfeld.}« und »\textsc{(\textsc{Hofmsthl\pwindex{Hofmannsthal, Hugo von 1.\,2.\,1874 Wien – 15.\,7.\,1929 Rodaun@\textsc{Hofmannsthal, Hugo von} (1.\,2.\,1874 Wien – 15.\,7.\,1929 Rodaun), \emph{Schriftsteller}|pw}}}« sowie drei Unterstreichungen }\toendnotes[C]{\smallbreak}
\pstart
           \raggedleft{}{\pb}Wien III/\textsubscript{3},
                        Oetzeltg. 1 \textsuperscript{III}/\textsubscript{ii}\oindex{Ölzeltgasse@\textbf{Ölzeltgasse}, \emph{Straße}|pw}{\\}den 10. März 1931\pend
           
\pstart\center{}Sehr geehrter Herr Doktor!\pend\vspace{0.5em}
\pstart
           Trotz des negativen Inhaltes Ihrer \label{K_L02595-1v}\edtext{Zeilen}{\lemma{\textnormal{\emph{Zeilen}}}\Cendnote{\textnormal{Siehe XXXX Auszeichnungsfehler: Dokument L02598 nicht gefunden.
               }}}\label{K_L02595-1}haben sie mich doch sehr erfreut. Mir war es, trotz der Maschinschrift, als
               hörte ich plötzlich Ihre Stimme, nur war sie tiefer und ernster geworden, im Lauf der
               Jahre, in denen man {\pb}allerlei durch- und mitgemacht
               hat.\pend
           
\pstart
           Ich gehe leider gar nicht mehr ins Theater, – ich bin fast taub, – doch ich folge
               Ihrer Produktion für die Bühne, indem ich Ihre Stücke lese: sie verlieren dabei
               nichts. Mit Dank und den wärmsten Grüßen,\pend
           \pstart \spacefill\mbox{Marie Herzfeld}\pend{}
\pstart
           \noindent{}\label{K_L02595-2v}\edtext{NB.}{\lemma{\textnormal{\emph{NB.}}}\Cendnote{\textnormal{Notabene, lateinisch: merke wohl}}}\label{K_L02595-2} Ich schreibe an
                     Prof. Zimmer\pwindex{Zimmer, Heinrich 6.\,12.\,1890 Greifswald – 20.\,3.\,1943 New York City@\textsc{Zimmer, Heinrich} (6.\,12.\,1890 Greifswald – 20.\,3.\,1943 New York City), \emph{Indologe}|pw}, wegen des \label{K_L02595-3v}\edtext{Ren.-Dramas\pwindex{Hofmannsthal, Hugo von 1.\,2.\,1874 Wien – 15.\,7.\,1929 Rodaun@\textsc{Hofmannsthal, Hugo von} (1.\,2.\,1874 Wien – 15.\,7.\,1929 Rodaun), \emph{Schriftsteller}!Ascanio und Gioconda@\strich\emph{Ascanio und Gioconda}|pw}}{\lemma{\textnormal{\emph{Ren.-Dramas}}}\Cendnote{\textnormal{Siehe XXXX Auszeichnungsfehler: Dokument L02589 nicht gefunden, XXXX Auszeichnungsfehler: Dokument L02598 nicht gefunden.
                  }}}\label{K_L02595-3}; der wird mehr wissen!\pend
           \selectlanguage{ngerman}\endnumbering\briefempfaengerindex{Schnitzler, Arthur@\textsc{Schnitzler, Arthur}!zzzHerzfeld, Marie@\emph{von Marie Herzfeld}!1931-03-101@{10.\,3.\,1931}|)be}\mylabel{L02595h}  \newcommand{\dateiname}{L02595}\newcommand{\titel}{Marie Herzfeld an Arthur Schnitzler, 10. 3. 1931}\newcommand{\editorInnen}{Martin Anton Müller und Laura Untner}%% latex-leseansicht-abspann.tex
%% Abspann für die Leseansicht.
%% Der Schalter \ifkorrekturansicht ist bereits durch den Vorspann gesetzt.

%% latex-abspann.tex
%% Gemeinsamer Abspann für Korrekturansicht und Leseansicht.
%% Setzt den Schalter \ifkorrekturansicht voraus (gesetzt in den
%% einbindenden Dateien latex-korrekturansicht-abspann.tex bzw.
%% latex-leseansicht-abspann.tex).
%% ---------------------------------------------------------------

\normalsize

% Das esempio-Environment wird nur in der Leseansicht benötigt
\ifkorrekturansicht\else
\newenvironment{esempio}[3]%
{
    \vspace{1.5ex}
    \rlap{\underline{#1}}
    \par
    \setlength{\parindent}{0cm}
    \nopagebreak
    \leftskip=#2cm
    \rightskip=#3cm
}
{
    \par
}
\fi

\doendnotes{C}
\bigskip
\vfill

\clearpage

\footnotesize

\ifkorrekturansicht
  \lohead{\textsc{register}}
\fi

% theindex-Environment neu definieren ohne reledmac
\makeatletter
\renewenvironment{theindex}{%
  \ifkorrekturansicht
    \section*{\indexname}%
  \else
    \subsubsection*{Index der erwähnten Entitäten}%
  \fi
  \setlength{\parindent}{0pt}%
  \setlength{\parskip}{0pt plus 0.3pt}%
  \let\item\@idxitem
}{%
  \ifkorrekturansicht\clearpage\fi
}
\makeatother

\IfFileExists{\jobname-pw.ind}{\input{\jobname-pw.ind}}{}

% Quellenangabe nur in der Leseansicht
\ifkorrekturansicht\else
% Fallback-Definitionen, falls die .tex-Datei \titel etc. nicht gesetzt hat
\providecommand{\titel}{}
\providecommand{\editorInnen}{}
\providecommand{\dateiname}{\jobname}

\vspace{3cm}

\vfill

\footnotesize
\textsc{Quelle}: \titel. Herausgegeben von {\editorInnen}. In: \emph{Arthur Schnitzler: Briefwechsel mit Autorinnen und Autoren}.
 Digitale Edition, https://schnitzler-briefe.acdh.oeaw.ac.at/{\dateiname}.html (Stand \today)
\fi

\end{document}


