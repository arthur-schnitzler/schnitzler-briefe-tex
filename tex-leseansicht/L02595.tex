%% latex-leseansicht-vorspann.tex
%% Vorspann für die Leseansicht.
%% Lädt die gemeinsame Datei latex-vorspann.tex mit nicht gesetztem Schalter.

\newif\ifkorrekturansicht
\korrekturansichtfalse

\input{../tex-inputs/latex-vorspann}


               \section[Marie Herzfeld an Arthur Schnitzler, 10. 3. 1931]{ Marie Herzfeld an Arthur Schnitzler, 10. 3. 1931}\nopagebreak\mylabel{v}\rehead{ }\begin{ledgroupsized}[t]{13cm}\normalsize\beginnumbering\briefempfaengerindex{Schnitzler, Arthur@\textsc{Schnitzler, Arthur}!zzzHerzfeld, Marie@\emph{von Marie Herzfeld}!1931-03-101@{10. 3. 1931}|(be} \toendnotes[C]{\smallbreak\pagebreak[2]} \Standort{DLA, A:Schnitzler, HS.1985.1.03436,6.}
\physDesc{Brief, 1 Blatt (Briefpapier mit Trauerrand), 2 Seiten
\newline{}Handschrift: schwarze Tinte, lateinische Kurrent
\newline{}Schnitzler: mit rotem Buntstift Vermerk »\textsc{Herzfeld.}«
                                 und »\textsc{(\textsc{Hofmsthl\pwindex{Hofmannsthal, Hugo von 01.02.1874 – 15.07.1929@\textsc{Hofmannsthal, Hugo von} (01.02.1874 – 15.07.1929), \emph{Schriftsteller}|pw}}}« sowie drei Unterstreichungen }\toendnotes[C]{\smallbreak}\pstart
           \raggedleft{}{\pb}Wien III/\textsubscript{3}, Oetzeltg. 1 \textsuperscript{III}/\textsubscript{ii}\oindex{Oelzeltgasse@\textbf{Ölzeltgasse}|pw}{\\}den 10. März 1931\pend
           \pstart\center{}Sehr geehrter Herr Doktor!\pend\pstart
           Trotz des negativen Inhaltes Ihrer \label{K_L02595-1v}\edtext{Zeilen }{\lemma{\textnormal{\emph{Zeilen }}}\Cendnote{\textnormal{siehe Arthur Schnitzler an Marie Herzfeld, 7. 3. 1931}}}\label{K_L02595-1h}haben sie mich doch sehr erfreut. Mir war
               es, trotz der Maschinschrift, als hörte ich plötzlich Ihre Stimme, nur war sie
               tiefer und ernster geworden, im Lauf der Jahre, in denen man {\pb}allerlei durch- und mitgemacht hat.\pend
           \pstart
           Ich gehe leider gar nicht mehr ins Theater, – ich bin fast taub, – doch ich folge
               Ihrer Produktion für die Bühne, indem ich Ihre Stücke lese: sie verlieren dabei
               nichts. Mit Dank und den wärmsten Grüßen,\pend
           \pstart \spacefill\mbox{Marie Herzfeld}\pend{}\pstart
           \noindent{}\label{K_L02595-2v}\edtext{NB.}{\lemma{\textnormal{\emph{NB.}}}\Cendnote{\textnormal{Notabene, lateinisch: merke wohl}}}\label{K_L02595-2h} Ich schreibe an
                     Prof. Zimmer\pwindex{Zimmer, Heinrich 06.12.1890 – 20.03.1943@\textsc{Zimmer, Heinrich} (06.12.1890 – 20.03.1943), \emph{Indologe}|pw}, wegen des \label{K_L02595-3v}\edtext{Ren.-Dramas\pwindex{Hofmannsthal, Hugo von 01.02.1874 – 15.07.1929@\textsc{Hofmannsthal, Hugo von} (01.02.1874 – 15.07.1929), \emph{Schriftsteller}!Ascanio und Gioconda1979@\strich\emph{Ascanio und Gioconda} {[}1979{]}|pw}}{\lemma{\textnormal{\emph{Ren.-Dramas}}}\Cendnote{\textnormal{siehe Marie Herzfeld an Arthur Schnitzler, 5. 3. 1931, Arthur Schnitzler an Marie Herzfeld, 7. 3. 1931}}}\label{K_L02595-3h}; der wird mehr wissen!\pend
           \endnumbering\briefempfaengerindex{Schnitzler, Arthur@\textsc{Schnitzler, Arthur}!zzzHerzfeld, Marie@\emph{von Marie Herzfeld}!1931-03-101@{10. 3. 1931}|)be}\mylabel{h}\end{ledgroupsized}  \newcommand{\dateiname}{L02595}\newcommand{\titel}{Marie Herzfeld an Arthur Schnitzler, 10. 3. 1931}\newcommand{\editorInnen}{Martin Anton Müller und Laura Untner}%% latex-leseansicht-abspann.tex
%% Abspann für die Leseansicht.
%% Der Schalter \ifkorrekturansicht ist bereits durch den Vorspann gesetzt.

%% latex-abspann.tex
%% Gemeinsamer Abspann für Korrekturansicht und Leseansicht.
%% Setzt den Schalter \ifkorrekturansicht voraus (gesetzt in den
%% einbindenden Dateien latex-korrekturansicht-abspann.tex bzw.
%% latex-leseansicht-abspann.tex).
%% ---------------------------------------------------------------

\normalsize

% Das esempio-Environment wird nur in der Leseansicht benötigt
\ifkorrekturansicht\else
\newenvironment{esempio}[3]%
{
    \vspace{1.5ex}
    \rlap{\underline{#1}}
    \par
    \setlength{\parindent}{0cm}
    \nopagebreak
    \leftskip=#2cm
    \rightskip=#3cm
}
{
    \par
}
\fi

\doendnotes{C}
\bigskip
\vfill

\clearpage

\footnotesize

\ifkorrekturansicht
  \lohead{\textsc{register}}
\fi

% theindex-Environment neu definieren ohne reledmac
\makeatletter
\renewenvironment{theindex}{%
  \ifkorrekturansicht
    \section*{\indexname}%
  \else
    \subsubsection*{Index der erwähnten Entitäten}%
  \fi
  \setlength{\parindent}{0pt}%
  \setlength{\parskip}{0pt plus 0.3pt}%
  \let\item\@idxitem
}{%
  \ifkorrekturansicht\clearpage\fi
}
\makeatother

\IfFileExists{\jobname-pw.ind}{\input{\jobname-pw.ind}}{}

% Quellenangabe nur in der Leseansicht
\ifkorrekturansicht\else
% Fallback-Definitionen, falls die .tex-Datei \titel etc. nicht gesetzt hat
\providecommand{\titel}{}
\providecommand{\editorInnen}{}
\providecommand{\dateiname}{\jobname}

\vspace{3cm}

\vfill

\footnotesize
\textsc{Quelle}: \titel. Herausgegeben von {\editorInnen}. In: \emph{Arthur Schnitzler: Briefwechsel mit Autorinnen und Autoren}.
 Digitale Edition, https://schnitzler-briefe.acdh.oeaw.ac.at/{\dateiname}.html (Stand \today)
\fi

\end{document}


      