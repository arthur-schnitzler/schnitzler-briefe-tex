%% latex-korrekturansicht-vorspann.tex
%% Vorspann für die Korrekturansicht.
%% Lädt die gemeinsame Datei latex-vorspann.tex mit gesetztem Schalter.

\newif\ifkorrekturansicht
\korrekturansichttrue

\input{../tex-inputs/latex-vorspann}


\section[Elsa Plessner an Arthur Schnitzler, 15. 9. 1896]{L03702 Elsa Plessner an Arthur Schnitzler, 15. 9. 1896}
\nopagebreak\mylabel{L03702v}
\rehead{ }\normalsize\beginnumbering\briefempfaengerindex{Schnitzler, Arthur@\textsc{Schnitzler, Arthur}!zzzPlessner, Elsa@\emph{von Elsa Plessner}!1896-09-153@{15. 9. 1896}|(be}
\toendnotes[C]{\smallbreak\pagebreak[2]}\Standort{DLA, A:Schnitzler, HS.1985.1.419.}
\physDesc{Brief,  Blätter, 3 Seiten, 1639 Zeichen
\newline{}Handschrift: , lateinische Kurrent
\newline{}Schnitzler: drei Unterstreichungen }\toendnotes[C]{\smallbreak}
\pstart
           {\pb}I. Bäckerstrasse N\textsuperscript{o}
                     1\oindex{Baeckerstrasse 1@\textbf{Bäckerstraße 1}, \emph{Wohngebäude (K.WHS)}|pw}, den 15. 9. 96. \pend
           
\pstart{}Verehrter Meister \label{K_L03702-1v}\edtext{Anatol}{\lemma{\textnormal{\emph{Anatol}}}\Cendnote{\textnormal{Bezugnahme auf Arthur Schnitzlers Einakter-Zyklus \emph{Anatol}\pwindex{Anatol@\emph{Anatol}|pwk} und den gleichnamigen
                  Protagonisten}}}\label{K_L03702-1}!\pend\vspace{0.5em}
\pstart
           Hiemit übersende, Ihrem Wunsch gemäß, den \label{K_L03702-2v}\edtext{Brief}{\lemma{\textnormal{\emph{Brief}}}\Cendnote{\textnormal{nicht
                  überliefert}}}\label{K_L03702-2}, den Sie so gütig waren, an mich zu richten sowie einen andern
               der mir heute Früh zukam.\pend
           
\pstart
           Diese liebenswürdigen \label{K_L03702-3v}\edtext{Zeilen von Frau
                  Janitschek\pwindex{Janitschek, Maria 1859-07-23 – 1927-04-28@\textsc{Janitschek, Maria} (1859-07-23 – 1927-04-28), \emph{Schriftsteller/Schriftstellerin}|pw}}{\lemma{\textnormal{\emph{Zeilen … Janitschek}}}\Cendnote{\textnormal{Nicht überliefert. Plessner\pwindex{Plessner, Elsa 22.08.1875 – 01.05.1932@\textsc{Plessner, Elsa} (22.08.1875 – 01.05.1932), \emph{Schriftsteller/Schriftstellerin}|pwk} hatte Maria
                     Janitscheks\pwindex{Janitschek, Maria 1859-07-23 – 1927-04-28@\textsc{Janitschek, Maria} (1859-07-23 – 1927-04-28), \emph{Schriftsteller/Schriftstellerin}|pwk} Buch \emph{Vom Weibe}\pwindex{Vom Weibe. Charakterzeichnungen@\emph{Vom Weibe. Charakterzeichnungen}|pwk} eine
                  ausführliche Rezension gewidmet: \emph{Vom Weibe}\pwindex{Vom Weibe@\emph{Vom Weibe}|pwk}. In: \emph{Morgen-Presse}\pwindex{Presse@\emph{Die Presse}|pwk}, Jg. 49, Nr. 167,
                        18. 6. 1896, S. [1]–2.}}}\label{K_L03702-3} haben mich aufrichtig erfreut und dürften auch Sie einigermaßen
               interessieren! Nicht wahr? –\pend
           
\pstart
           Sodann bringt dies umfangreiche Paket meinen \label{K_L03702-4v}\edtext{zukünftigen Band\pwindex{glaeserne Kaefig. Skizzen und Novellen@\emph{Der gläserne Käfig. Skizzen und Novellen}|pwv} Skizzen}{\lemma{\textnormal{\emph{zukünftigen Band Skizzen}}}\Cendnote{\textnormal{Elsa Plessners\pwindex{Plessner, Elsa 22.08.1875 – 01.05.1932@\textsc{Plessner, Elsa} (22.08.1875 – 01.05.1932), \emph{Schriftsteller/Schriftstellerin}|pwk} Band \emph{Der gläserne Käfig}\pwindex{glaeserne Kaefig. Skizzen und Novellen@\emph{Der gläserne Käfig. Skizzen und Novellen}|pwk} mit vierzehn Novellen und Skizzen
                  erschien 1901. Welche der Texte daraus sie in welcher Reihenfolge mit
                  diesem Brief schickte, läßt sich nur zum Teil rekonstruieren.}}}\label{K_L03702-4}, von dem ich
               mir das Schlimmste, was Sie mir darüber sagen können selber schon gesagt habe. Allein
               wie bemerkt zwingen mich rein \label{K_L03702-5v}\edtext{äußerliche Gründe}{\lemma{\textnormal{\emph{äußerliche Gründe}}}\Cendnote{\textnormal{Am
                     19. 9. 1895 war ihr Vater Louis
                     Plessner\pwindex{Plessner, Louis 1847-12-03 – 1895-09-19@\textsc{Plessner, Louis} (1847-12-03 – 1895-09-19), \emph{Journalist/Journalistin, Kaufmann/Kauffrau}|pwk} gestorben, woraus finanzielle Schwierigkeiten entstanden sein
                  dürften.}}}\label{K_L03702-5} ein »Buch\pwindex{glaeserne Kaefig. Skizzen und Novellen@\emph{Der gläserne Käfig. Skizzen und Novellen}|pwv}«
               vom Stapel {\pb}zu lassen – beklagen sollt Ihr mich, doch nimmer richten!! –
               Doch bitte ich Sie herzlich \label{K_L03702-6v}\edtext{N\textsuperscript{o}
                  1\pwindex{Warten@\emph{Warten}|pwv}}{\lemma{\textnormal{\emph{N\textsuperscript{o}
                  1}}}\Cendnote{\textnormal{Dass es sich um den Text \emph{Warten}\pwindex{Warten@\emph{Warten}|pwk} (zunächst unter dem Titel »Blätter« geplant)
                  handelt, ergibt sich aus Plessners\pwindex{Plessner, Elsa 22.08.1875 – 01.05.1932@\textsc{Plessner, Elsa} (22.08.1875 – 01.05.1932), \emph{Schriftsteller/Schriftstellerin}|pwk} folgendem
                  Brief vom 21. 9. 1896.}}}\label{K_L03702-6}, das Fragment\pwindex{Warten@\emph{Warten}|pwv}
               oder quasi-\label{K_L03702-7v}\edtext{\begin{otherlanguage}{french}croquis\end{otherlanguage}}{\lemma{\textnormal{\emph{croquis}}}\Cendnote{\textnormal{französisch: Entwurf}}}\label{K_L03702-7} nochmals zu
               lesen und dabei zu vergessen, dass ich je beabsichtigte, es \introOben{}weiter\introOben{} auszuführen. Vielleicht ändern Sie dann ein wenig Ihre Meinung
               umsomehr, als ich ja stark daran gefeilt und geändert habe! – N\textsuperscript{o} 2\pwindex{Leiter der Seele@\emph{Die Leiter der Seele}|pwv} ist
                  \label{K_L03702-8v}\edtext{aus dem Simplicissimus\pwindex{Simplicissimus@\emph{Simplicissimus}|pw}}{\lemma{\textnormal{\emph{aus dem Simplicissimus}}}\Cendnote{\textnormal{Der Text, der im Band \emph{Der gläserne Käfig}\pwindex{glaeserne Kaefig. Skizzen und Novellen@\emph{Der gläserne Käfig. Skizzen und Novellen}|pwk} unter dem Titel \emph{Der Selbstmörder}\pwindex{Leiter der Seele@\emph{Die Leiter der Seele}|pwk} publiziert wurde, erschien im ersten
                  Jahrgang des \emph{Simplicissimus}\pwindex{Simplicissimus@\emph{Simplicissimus}|pwk} unter dem Titel
                     \emph{Die Leiter der Seele}\pwindex{Leiter der Seele@\emph{Die Leiter der Seele}|pwk} (E. Pleßner\pwindex{Plessner, Elsa 22.08.1875 – 01.05.1932@\textsc{Plessner, Elsa} (22.08.1875 – 01.05.1932), \emph{Schriftsteller/Schriftstellerin}|pwk}: \emph{Die Leiter der Seele}\pwindex{Leiter der Seele@\emph{Die Leiter der Seele}|pwk}. In: \emph{Simplicissimus}\pwindex{Simplicissimus@\emph{Simplicissimus}|pwk}, Jg. 1, Nr. 10, 6. 6. 1896,
                     S. 6).}}}\label{K_L03702-8}, sowie 3, 6 u 7 von Langen\pwindex{Langen, Albert 1869-07-08 – 1909-04-30@\textsc{Langen, Albert} (1869-07-08 – 1909-04-30), \emph{Verleger/Verlegerin}|pw} für Simpl.\pwindex{Simplicissimus@\emph{Simplicissimus}|pw} aus 10 Skizzen
                  \label{K_L03702-9v}\edtext{ausgewählt}{\lemma{\textnormal{\emph{ausgewählt}}}\Cendnote{\textnormal{Neben \emph{Die Leiter der
                     Seele}\pwindex{Leiter der Seele@\emph{Die Leiter der Seele}|pwk} lassen sich keine weitere Texte Plessners\pwindex{Plessner, Elsa 22.08.1875 – 01.05.1932@\textsc{Plessner, Elsa} (22.08.1875 – 01.05.1932), \emph{Schriftsteller/Schriftstellerin}|pwk} im \emph{Simplicissimus}\pwindex{Simplicissimus@\emph{Simplicissimus}|pwk}
                  nachweisen.}}}\label{K_L03702-9} wurden. (? – !) 3 und 6 ganz \label{K_L03702-10v}\edtext{\uuline{alte} Arbeiten }{\lemma{\textnormal{\emph{alte Arbeiten }}}\Cendnote{\textnormal{Möglicherweise \emph{Baby}\pwindex{Baby@\emph{Baby}|pwk} und
                     \emph{Begräbnißtag}\pwindex{Begraebnisstag@\emph{Der Begräbnißtag}|pwk}, die Plessner\pwindex{Plessner, Elsa 22.08.1875 – 01.05.1932@\textsc{Plessner, Elsa} (22.08.1875 – 01.05.1932), \emph{Schriftsteller/Schriftstellerin}|pwk} im Brief vom 12. 10. 1900 als ihre frühesten Arbeiten
                  benennt.}}}\label{K_L03702-10}{[}.{]} Als beste von Alle\substVorne{}\textsuperscript{n}\substDazwischen{}m\substHinten{}, wenn man so sagen darf, gilt mir N\textsuperscript{o} 8 – »Im Widerschein\pwindex{Im Widerschein@\emph{Im Widerschein}|pw}«. – Doch wir werden ja sehen!\pend
           
\pstart
           Seien Sie immer so grob, als Sie nur können, und glauben Sie mir, verehrter Herr
               Doctor, dass mich eine solide, ehrliche Grobheit von Ihnen mehr freut, als alle {\pb}Complimente sämmtlicher Esel\strikeout{-} von Wien\oindex{Wien@\textbf{Wien}, \emph{A.ADM2}|pw} zusammengenommen! Die \label{K_L03702-11v}\edtext{Abdrücke\pwindex{Begraebnisstag@\emph{Der Begräbnißtag}|pwv}\pwindex{Im Feuer geprueft@\emph{Im Feuer geprüft}|pwv}\pwindex{Im Widerschein@\emph{Im Widerschein}|pwv}}{\lemma{\textnormal{\emph{Abdrücke}}}\Cendnote{\textnormal{E. Pleßner\pwindex{Plessner, Elsa 22.08.1875 – 01.05.1932@\textsc{Plessner, Elsa} (22.08.1875 – 01.05.1932), \emph{Schriftsteller/Schriftstellerin}|pwk}: \emph{Der Begräbnißtag}\pwindex{Begraebnisstag@\emph{Der Begräbnißtag}|pwk}. In: \emph{Neues Wiener Journal}\pwindex{Neues Wiener Journal@\emph{Neues Wiener Journal}|pwk}, Nr. 951, 17. 6. 1896,
                  S. 1–2. E. Pleßner\pwindex{Plessner, Elsa 22.08.1875 – 01.05.1932@\textsc{Plessner, Elsa} (22.08.1875 – 01.05.1932), \emph{Schriftsteller/Schriftstellerin}|pwk}: \emph{Im Feuer geprüft}\pwindex{Im Feuer geprueft@\emph{Im Feuer geprüft}|pwk}. In: \emph{Neues Wiener Journal}\pwindex{Neues Wiener Journal@\emph{Neues Wiener Journal}|pwk}, Nr. 1008, 14. 8. 1896,
                     S. 1–2. E. P.\pwindex{Plessner, Elsa 22.08.1875 – 01.05.1932@\textsc{Plessner, Elsa} (22.08.1875 – 01.05.1932), \emph{Schriftsteller/Schriftstellerin}|pwk}: \emph{Im Widerschein}\pwindex{Im Widerschein@\emph{Im Widerschein}|pwk}. In: \emph{Neues Wiener
                        Journal}\pwindex{Neues Wiener Journal@\emph{Neues Wiener Journal}|pwk}, Nr. 1028, 4. 9. 1896, S. 1.}}}\label{K_L03702-11} sind –
               verdammen Sie mich nicht – aus dem N. W.-
                  Journal\pwindex{Neues Wiener Journal@\emph{Neues Wiener Journal}|pw}! – – –\pend
           
\pstart
           Und somit überliefre ich mich Ihrer Gnade – ich glaub an sie und hoff' auf sie, wobei
               ich schließlich noch \introOben{}soeben\introOben{} bemerke dass meine Handschrift
               ein wenig der Ihrigen ähnlich ist.\pend
           
\pstart
           Mit Verehrung und Dankbarkeit{\\[\baselineskip]}\spacefill\mbox{Elsa Plessner}\pend
           \leftskip=0em{}
\pstart
           \noindent{}\label{K_L03702-12v}\edtext{\uline{mit 3 Beilagen}}{\lemma{\textnormal{\emph{mit 3 Beilagen}}}\Cendnote{\textnormal{Die Beilagen sind nicht überliefert.
                     Wie aus dem vorliegenden Brief hervorgeht, handelte es sich um ein
                     Korrespondenzstück Schnitzlers, ein Brief
                     der Schriftstellerin Maria Janitschek\pwindex{Janitschek, Maria 1859-07-23 – 1927-04-28@\textsc{Janitschek, Maria} (1859-07-23 – 1927-04-28), \emph{Schriftsteller/Schriftstellerin}|pwk},
                     ein Konvolut mit Novellen und Skizzen, die später Eingang in den Band \emph{Der gläserne Käfig}\pwindex{glaeserne Kaefig. Skizzen und Novellen@\emph{Der gläserne Käfig. Skizzen und Novellen}|pwk} fanden, darunter
                     Abdrucke von Texten Plessners\pwindex{Plessner, Elsa 22.08.1875 – 01.05.1932@\textsc{Plessner, Elsa} (22.08.1875 – 01.05.1932), \emph{Schriftsteller/Schriftstellerin}|pwk} aus dem \emph{Neuen Wiener Journal}\pwindex{Neues Wiener Journal@\emph{Neues Wiener Journal}|pwk}.}}}\label{K_L03702-12}\pend
           \selectlanguage{ngerman}\endnumbering\briefempfaengerindex{Schnitzler, Arthur@\textsc{Schnitzler, Arthur}!zzzPlessner, Elsa@\emph{von Elsa Plessner}!1896-09-153@{15. 9. 1896}|)be}\mylabel{L03702h}
\begin{anhang}
\end{anhang}\normalsize

\doendnotes{C}
\bigskip
\vfill

\clearpage

\footnotesize

\lohead{\textsc{register}}

% Definiere theindex-Environment komplett neu ohne reledmac
\makeatletter
\renewenvironment{theindex}{%
  \section*{\indexname}%
  \setlength{\parindent}{0pt}%
  \setlength{\parskip}{0pt plus 0.3pt}%
  \let\item\@idxitem
}{%
  \clearpage
}
\makeatother

\IfFileExists{\jobname-pw.ind}{\input{\jobname-pw.ind}}{}

\end{document}

      