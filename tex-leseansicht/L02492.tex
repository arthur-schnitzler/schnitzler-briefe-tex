%% latex-leseansicht-vorspann.tex
%% Vorspann für die Leseansicht.
%% Lädt die gemeinsame Datei latex-vorspann.tex mit nicht gesetztem Schalter.

\newif\ifkorrekturansicht
\korrekturansichtfalse

\input{../tex-inputs/latex-vorspann}


               \section[Arthur Schnitzler an Stefan Großmann, 7. 11. 1927]{ Arthur Schnitzler an Stefan Großmann, 7. 11. 1927}\nopagebreak\mylabel{v}\rehead{ }\begin{ledgroupsized}[t]{13cm}\normalsize\beginnumbering\briefempfaengerindex{Grossmann, Stefan@\textsc{Großmann, Stefan}!zzzSchnitzler, Arthur@\emph{von Arthur Schnitzler}!1927-11-071@{7. 11. 1927}|(be} \toendnotes[C]{\smallbreak\pagebreak[2]} \Standort{DLA, A:Schnitzler, HS.NZ85.1.896.}
\physDesc{Brief, 1 Blatt, 1 Seite, maschineller Durchschlag
\newline{}Handschrift: roter Buntstift, lateinische Kurrent (\noindent{}»Grossmann«, »Berlin\oindex{Berlin@\textbf{Berlin}|pw}«, Unterstreichungen)}\toendnotes[C]{\smallbreak}\pstart
           \raggedleft{}{\pb}7. 11. 1927.\pend
           \pstart{}Verehrter Herr Stefan Grossmann.\pend\pstart
           Ende dieses Monats wird mein Aphorismenbuch\pwindex{Schnitzler, Arthur 15.05.1862 – 21.10.1931@\textsc{Schnitzler, Arthur} (15.05.1862 – 21.10.1931), \emph{Schriftsteller, Mediziner}!Buch der Sprueche und Bedenken1927@\strich\emph{Buch der Sprüche und Bedenken} {[}1927{]}|pwv} erscheinen und wenn Sie ihren
                    freundlichen Wunsch von früher her noch aufrecht erhalten, so würde ich Ihnen
                    gerne etliches (noch Ungedrucktes) aus dem Buch\pwindex{Schnitzler, Arthur 15.05.1862 – 21.10.1931@\textsc{Schnitzler, Arthur} (15.05.1862 – 21.10.1931), \emph{Schriftsteller, Mediziner}!Buch der Sprueche und Bedenken1927@\strich\emph{Buch der Sprüche und Bedenken} {[}1927{]}|pwv}{ }\strikeout{zur Verfügung stellen} zum Vorabdruck zur
                    Verfügung stellen.\pend
           \pstart
           Stimmt es, dass in Ihrem »Tagebuch\pwindex{Tage-Buch1920-01-01 – 1933-01-01@\emph{Das Tage-Buch}|pw}« im
                        Sommer dieses Jahres wieder einige meiner Aphorismen (entweder
                    aus der »Neuen Freien Presse\pwindex{Neue Freie Presse1864 – 1939@\emph{Neue Freie Presse}|pw}\pwindex{Schnitzler, Arthur 15.05.1862 – 21.10.1931@\textsc{Schnitzler, Arthur} (15.05.1862 – 21.10.1931), \emph{Schriftsteller, Mediziner}!Bemerkungen. Aus dem noch unveroeffentlichten »Buch der Sprueche und Bedenken«.1927-04-17 – 1927-04-17@\strich\emph{Bemerkungen. Aus dem noch unveröffentlichten »Buch der Sprüche und Bedenken«.} {[}1927-04-17 – 1927-04-17{]}|pwv}« oder einer Dresdner Zeitung\pwindex{Dresdner Neueste Nachrichten1893 – 1943@\emph{Dresdner Neueste Nachrichten}|pwv}\pwindex{Schnitzler, Arthur 15.05.1862 – 21.10.1931@\textsc{Schnitzler, Arthur} (15.05.1862 – 21.10.1931), \emph{Schriftsteller, Mediziner}!Bemerkungen. (Aus dem noch unveroeffentlichten »Buch der Sprueche und Bedenken«)1927-04-17 – 1927-04-17@\strich\emph{Bemerkungen. (Aus dem noch unveröffentlichten »Buch der Sprüche und Bedenken«)} {[}1927-04-17 – 1927-04-17{]}|pwv} \label{K_L02492_1v}\edtext{abgedruckt waren}{\lemma{\textnormal{\emph{abgedruckt waren}}}\Cendnote{\textnormal{Es waren 1927 keine
                        Aphorismen Schnitzler\pwindex{Schnitzler, Arthur 15.05.1862 – 21.10.1931@\textsc{Schnitzler, Arthur} (15.05.1862 – 21.10.1931), \emph{Schriftsteller, Mediziner}|pwk}s abgedruckt. Erst
                        in Folge dieses Briefes erschienen am 19. 11. 1927{ }\emph{Bemerkungen}\pwindex{Schnitzler, Arthur 15.05.1862 – 21.10.1931@\textsc{Schnitzler, Arthur} (15.05.1862 – 21.10.1931), \emph{Schriftsteller, Mediziner}!Bemerkungen1927-11-19@\strich\emph{Bemerkungen} {[}1927-11-19{]}|pwk} (Jg. 8, H. 47,
                            S. 1879–1881).}}}\label{K_L02492_1h}? Dies frage ich nur der Ordnung wegen.\pend
           \pstart
           Mit verbindlichen Grüssen{\\[\baselineskip]}Ihr sehr ergebener\pend
           \leftskip=0em{}{\bigskip}\pstart
           \noindent{}Herrn Stefan Grossmann{\\}Herausgeber des »Tagebuch\orgindex{Tage-Buch@Das Tage-Buch|pw}«,{\\}Berlin SW. 19, Beuthstr. 19\oindex{Beuthstrasse@\textbf{Beuthstrasse}|pw}.\pend
           \endnumbering\briefempfaengerindex{Grossmann, Stefan@\textsc{Großmann, Stefan}!zzzSchnitzler, Arthur@\emph{von Arthur Schnitzler}!1927-11-071@{7. 11. 1927}|)be}\mylabel{h}\end{ledgroupsized}  \newcommand{\dateiname}{L02492}\newcommand{\titel}{Arthur Schnitzler an Stefan Großmann, 7. 11. 1927}\newcommand{\editorInnen}{Martin Anton Müller und Gerd-Hermann Susen}%% latex-leseansicht-abspann.tex
%% Abspann für die Leseansicht.
%% Der Schalter \ifkorrekturansicht ist bereits durch den Vorspann gesetzt.

%% latex-abspann.tex
%% Gemeinsamer Abspann für Korrekturansicht und Leseansicht.
%% Setzt den Schalter \ifkorrekturansicht voraus (gesetzt in den
%% einbindenden Dateien latex-korrekturansicht-abspann.tex bzw.
%% latex-leseansicht-abspann.tex).
%% ---------------------------------------------------------------

\normalsize

% Das esempio-Environment wird nur in der Leseansicht benötigt
\ifkorrekturansicht\else
\newenvironment{esempio}[3]%
{
    \vspace{1.5ex}
    \rlap{\underline{#1}}
    \par
    \setlength{\parindent}{0cm}
    \nopagebreak
    \leftskip=#2cm
    \rightskip=#3cm
}
{
    \par
}
\fi

\doendnotes{C}
\bigskip
\vfill

\clearpage

\footnotesize

\ifkorrekturansicht
  \lohead{\textsc{register}}
\fi

% theindex-Environment neu definieren ohne reledmac
\makeatletter
\renewenvironment{theindex}{%
  \ifkorrekturansicht
    \section*{\indexname}%
  \else
    \subsubsection*{Index der erwähnten Entitäten}%
  \fi
  \setlength{\parindent}{0pt}%
  \setlength{\parskip}{0pt plus 0.3pt}%
  \let\item\@idxitem
}{%
  \ifkorrekturansicht\clearpage\fi
}
\makeatother

\IfFileExists{\jobname-pw.ind}{\input{\jobname-pw.ind}}{}

% Quellenangabe nur in der Leseansicht
\ifkorrekturansicht\else
% Fallback-Definitionen, falls die .tex-Datei \titel etc. nicht gesetzt hat
\providecommand{\titel}{}
\providecommand{\editorInnen}{}
\providecommand{\dateiname}{\jobname}

\vspace{3cm}

\vfill

\footnotesize
\textsc{Quelle}: \titel. Herausgegeben von {\editorInnen}. In: \emph{Arthur Schnitzler: Briefwechsel mit Autorinnen und Autoren}.
 Digitale Edition, https://schnitzler-briefe.acdh.oeaw.ac.at/{\dateiname}.html (Stand \today)
\fi

\end{document}


      