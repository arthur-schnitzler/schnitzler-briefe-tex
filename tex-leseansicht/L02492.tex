%% latex-leseansicht-vorspann.tex
%% Vorspann für die Leseansicht.
%% Lädt die gemeinsame Datei latex-vorspann.tex mit nicht gesetztem Schalter.

\newif\ifkorrekturansicht
\korrekturansichtfalse

\input{../tex-inputs/latex-vorspann}


\section[Arthur Schnitzler an Stefan Großmann, 7. 11. 1927]{L02492 Arthur Schnitzler an Stefan Großmann, 7. 11. 1927}
\nopagebreak\mylabel{L02492v}
\rehead{ }\normalsize\beginnumbering\briefempfaengerindex{Großmann, Stefan@\textsc{Großmann, Stefan}!zzzSchnitzler, Arthur@\emph{von Arthur Schnitzler}!1927-11-071@{7. 11. 1927}|(be}
\toendnotes[C]{\smallbreak\pagebreak[2]}
\correspDesc{Versand  durch Arthur Schnitzler am 7. 11. 1927 in Wien
\newline{}Erhalt  durch Stefan Großmann im Zeitraum [8. 11. 1927
                  – 12. 11. 1927?] in Berlin}\toendnotes[C]{\smallbreak}
\Standort{DLA, A:Schnitzler, HS.NZ85.1.896.}
\physDesc{Brief, Durchschlag, 1 Blatt, 1 Seite, 631 Zeichen
\newline{}Handschrift: roter Buntstift, lateinische Kurrent (\noindent{}»Grossmann«, »Berlin\oindex{Berlin@\textbf{Berlin}, \emph{Hauptstadt}|pw}«, Unterstreichungen)}\toendnotes[C]{\smallbreak}
\pstart
           \raggedleft{}{\pb}7. 11. 1927.\pend
           
\pstart{}Verehrter Herr Stefan Grossmann.\pend\vspace{0.5em}
\pstart
           Ende dieses Monats wird mein Aphorismenbuch\pwindex{Schnitzler, Arthur 15.\,5.\,1862 Wien – 21.\,10.\,1931 ebd.@\textsc{Schnitzler, Arthur} (15.\,5.\,1862 Wien – 21.\,10.\,1931 ebd.), \emph{Schriftsteller, Mediziner}!Buch der Sprüche und Bedenken@\strich\emph{Buch der Sprüche und Bedenken}|pwv} erscheinen und wenn Sie ihren freundlichen
               Wunsch von früher her noch aufrecht erhalten, so würde ich Ihnen gerne etliches (noch
               Ungedrucktes) aus dem Buch\pwindex{Schnitzler, Arthur 15.\,5.\,1862 Wien – 21.\,10.\,1931 ebd.@\textsc{Schnitzler, Arthur} (15.\,5.\,1862 Wien – 21.\,10.\,1931 ebd.), \emph{Schriftsteller, Mediziner}!Buch der Sprüche und Bedenken@\strich\emph{Buch der Sprüche und Bedenken}|pwv}{ }\strikeout{zur Verfügung stellen} zum Vorabdruck zur Verfügung
               stellen.\pend
           
\pstart
           Stimmt es, dass in Ihrem »Tagebuch\pwindex{Tage-Buch@\emph{Das Tage-Buch}|pw}« im
                  Sommer dieses Jahres wieder einige meiner Aphorismen (entweder aus
               der »Neuen Freien Presse\pwindex{Neue Freie Presse@\emph{Neue Freie Presse}|pw}\pwindex{Schnitzler, Arthur 15.\,5.\,1862 Wien – 21.\,10.\,1931 ebd.@\textsc{Schnitzler, Arthur} (15.\,5.\,1862 Wien – 21.\,10.\,1931 ebd.), \emph{Schriftsteller, Mediziner}!Bemerkungen. Aus dem noch unveröffentlichten »Buch der Sprüche und Bedenken«.@\strich\emph{Bemerkungen. Aus dem noch unveröffentlichten »Buch der Sprüche und Bedenken«.}|pwv}« oder einer Dresdner Zeitung\pwindex{Dresdner Neueste Nachrichten@\emph{Dresdner Neueste Nachrichten}|pwv}\pwindex{Schnitzler, Arthur 15.\,5.\,1862 Wien – 21.\,10.\,1931 ebd.@\textsc{Schnitzler, Arthur} (15.\,5.\,1862 Wien – 21.\,10.\,1931 ebd.), \emph{Schriftsteller, Mediziner}!Bemerkungen. (Aus dem noch unveröffentlichten »Buch der Sprüche und Bedenken«)@\strich\emph{Bemerkungen. (Aus dem noch unveröffentlichten »Buch der Sprüche und Bedenken«)}|pwv}\label{K_L02492-1v}\edtext{abgedruckt waren}{\lemma{\textnormal{\emph{abgedruckt waren}}}\Cendnote{\textnormal{Im \emph{Tagebuch}\pwindex{Tage-Buch@\emph{Das Tage-Buch}|pwk} wurden 1927 keine Aphorismen
                     Schnitzlers abgedruckt. Erst in Folge
                  dieses Briefes erschienen am 19. 11. 1927{ }\emph{Bemerkungen}\pwindex{Schnitzler, Arthur 15.\,5.\,1862 Wien – 21.\,10.\,1931 ebd.@\textsc{Schnitzler, Arthur} (15.\,5.\,1862 Wien – 21.\,10.\,1931 ebd.), \emph{Schriftsteller, Mediziner}!Bemerkungen@\strich\emph{Bemerkungen}|pwk} (\emph{Tagebuch}\pwindex{Tage-Buch@\emph{Das Tage-Buch}|pwk}, Jg. 8, H. 47,
                     S. 1879–1881).}}}\label{K_L02492-1}? Dies frage ich nur der Ordnung wegen.\pend
           
\pstart
           Mit verbindlichen Grüssen{\\[\baselineskip]}Ihr sehr ergebener\pend
           \leftskip=0em{}{\vspace{1\baselineskip}}
\pstart
           Herrn Stefan Grossmann{\\}Herausgeber des »Tagebuch\orgindex{Tage-Buch@Das Tage-Buch|pw}«,{\\}Berlin SW. 19, Beuthstr. 19\oindex{Beuthstrasse@\textbf{Beuthstrasse}, \emph{Straße}|pw}.\pend
           \selectlanguage{ngerman}\endnumbering\briefempfaengerindex{Großmann, Stefan@\textsc{Großmann, Stefan}!zzzSchnitzler, Arthur@\emph{von Arthur Schnitzler}!1927-11-071@{7. 11. 1927}|)be}\mylabel{L02492h}  \newcommand{\dateiname}{L02492}\newcommand{\titel}{Arthur Schnitzler an Stefan Großmann, 7. 11. 1927}\newcommand{\editorInnen}{Martin Anton Müller und Gerd-Hermann Susen}%% latex-leseansicht-abspann.tex
%% Abspann für die Leseansicht.
%% Der Schalter \ifkorrekturansicht ist bereits durch den Vorspann gesetzt.

%% latex-abspann.tex
%% Gemeinsamer Abspann für Korrekturansicht und Leseansicht.
%% Setzt den Schalter \ifkorrekturansicht voraus (gesetzt in den
%% einbindenden Dateien latex-korrekturansicht-abspann.tex bzw.
%% latex-leseansicht-abspann.tex).
%% ---------------------------------------------------------------

\normalsize

% Das esempio-Environment wird nur in der Leseansicht benötigt
\ifkorrekturansicht\else
\newenvironment{esempio}[3]%
{
    \vspace{1.5ex}
    \rlap{\underline{#1}}
    \par
    \setlength{\parindent}{0cm}
    \nopagebreak
    \leftskip=#2cm
    \rightskip=#3cm
}
{
    \par
}
\fi

\doendnotes{C}
\bigskip
\vfill

\clearpage

\footnotesize

\ifkorrekturansicht
  \lohead{\textsc{register}}
\fi

% theindex-Environment neu definieren ohne reledmac
\makeatletter
\renewenvironment{theindex}{%
  \ifkorrekturansicht
    \section*{\indexname}%
  \else
    \subsubsection*{Index der erwähnten Entitäten}%
  \fi
  \setlength{\parindent}{0pt}%
  \setlength{\parskip}{0pt plus 0.3pt}%
  \let\item\@idxitem
}{%
  \ifkorrekturansicht\clearpage\fi
}
\makeatother

\IfFileExists{\jobname-pw.ind}{\input{\jobname-pw.ind}}{}

% Quellenangabe nur in der Leseansicht
\ifkorrekturansicht\else
% Fallback-Definitionen, falls die .tex-Datei \titel etc. nicht gesetzt hat
\providecommand{\titel}{}
\providecommand{\editorInnen}{}
\providecommand{\dateiname}{\jobname}

\vspace{3cm}

\vfill

\footnotesize
\textsc{Quelle}: \titel. Herausgegeben von {\editorInnen}. In: \emph{Arthur Schnitzler: Briefwechsel mit Autorinnen und Autoren}.
 Digitale Edition, https://schnitzler-briefe.acdh.oeaw.ac.at/{\dateiname}.html (Stand \today)
\fi

\end{document}


