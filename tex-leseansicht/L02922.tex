%% latex-korrekturansicht-vorspann.tex
%% Vorspann für die Korrekturansicht.
%% Lädt die gemeinsame Datei latex-vorspann.tex mit gesetztem Schalter.

\newif\ifkorrekturansicht
\korrekturansichttrue

\input{../tex-inputs/latex-vorspann}


\section[ Paul Goldmann an Arthur Schnitzler, 27. 6. {[}1900{]}]{L02922 Paul Goldmann an Arthur Schnitzler, 27. 6. {[}1900{]}}
\nopagebreak\mylabel{L02922v}
\rehead{ }\normalsize\beginnumbering\briefempfaengerindex{Schnitzler, Arthur@\textsc{Schnitzler, Arthur}!zzzGoldmann, Paul@\emph{von Paul Goldmann}!1900-06-271@{27. 6. {[}1900{]}}|(be}
\toendnotes[C]{\smallbreak\pagebreak[2]}\Standort{DLA, A:Schnitzler, HS.NZ85.1.3170.}
\physDesc{Brief, 1 Blatt, 3 Seiten, 665 Zeichen
\newline{}Handschrift: blaue Tinte, deutsche Kurrent
\newline{}Schnitzler: 1) mit Bleistift das Jahr »\textcolor{gray}{900}« vermerkt  2) mit rotem Buntstift zwei Unterstreichungen}\toendnotes[C]{\smallbreak}
\pstart
           {\pb}\textcolor{gray}{\textbf{DESSAUERSTRASSE 19}}\oindex{Dessauer Strasse@\textbf{Dessauer Straße}, \emph{Straße (K.STR)}|pw}\hfill Berlin\oindex{Berlin@\textbf{Berlin}, \emph{P.PPLC}|pw}, 27. \substVorne{}\textsuperscript{\textcolor{gray}{×}\-\textcolor{gray}{×}}\substDazwischen{}Ju\substHinten{}ni.\pend
           
\pstart\center{}Mein lieber Freund,\pend\vspace{0.5em}
\pstart
           Für heut nur ein Wort bezüglich der \label{K_L02922-1v}\edtext{Sommerpläne}{\lemma{\textnormal{\emph{Sommerpläne}}}\Cendnote{\textnormal{Siehe Paul Goldmann an Arthur Schnitzler, 16. 6. [1900].
               }}}\label{K_L02922-1}. Ich möchte bald in’s Klare kommen, da ich mich nach verſchiedenen Seiten
               entſcheiden ſoll. Auf \textsc{Richard\pwindex{Beer-Hofmann, Richard 1866-07-11 – 1945-09-26@\textsc{Beer-Hofmann, Richard} (1866-07-11 – 1945-09-26), \emph{Schriftsteller/Schriftstellerin}|pw}} iſt alſo nicht zu rechnen. Überdies habe ich \strikeout{\textcolor{gray}{dir}} ihm auch \label{K_L02922-2v}\edtext{direkt
                  geſchrieben}{\lemma{\textnormal{\emph{direkt
                  geſchrieben}}}\Cendnote{\textnormal{wohl der Brief vom 20. 6. {[}1900{]} (\emph{Houghton Library}\orgindex{Houghton Library@Houghton Library|pwk},
                     Harvard, Signatur 825.978.)}}}\label{K_L02922-2}, und er
               antwortet mir nicht. Alſo nicht! Auf Dich ſcheint auch nicht {\pb}zu rechnen zu ſein. Bitte; gib’ mir eine
               entſcheidende Antwort hierüber. In dieſem Falle würde ich einer Einladung \textsc{Hirschfelds\pwindex{Hirschfeld, Robert 17.09.1857 – 02.04.1914@\textsc{Hirschfeld, Robert} (17.09.1857 – 02.04.1914), \emph{Journalist/Journalistin, Musikkritiker/Musikkritikerin}|pw}} nach \textsc{Seekirn\oindex{Sekirn@\textbf{Sekirn}, \emph{P.PPL}|pw}} folgen und mit dieſem zuſammen \strikeout{\textcolor{gray}{wan}} eine Wanderung nach Südtirol\oindex{Suedtirol@\textbf{Südtirol}, \emph{A.ADM2}|pw} machen, –
               wenn ich überhaupt fortkomme\strikeout{\textcolor{gray}{n}}, was noch immer zweifelhaft iſt.\pend
           
\pstart
           {\pb}In welche\substVorne{}\textsuperscript{\textcolor{gray}{r}}\substDazwischen{}m\substHinten{} Orte wird \label{K_L02922-3v}\edtext{\textsc{Leo\pwindex{Van-Jung, Leo 15.10.1866 – 02.07.1939@\textsc{Van-Jung, Leo} (15.10.1866 – 02.07.1939), \emph{Gesangspädagoge/Gesangspädagogin, Mathematiker/Mathematikerin}|pw}} im Auguſt}{\lemma{\textnormal{\emph{Leo im Auguſt}}}\Cendnote{\textnormal{Vor der gemeinsamen Alpen\oindex{Alpen@\textbf{Alpen}, \emph{kein passender Code gefunden}|pwk}wanderung hielt sich Leo Van-Jung\pwindex{Van-Jung, Leo 15.10.1866 – 02.07.1939@\textsc{Van-Jung, Leo} (15.10.1866 – 02.07.1939), \emph{Gesangspädagoge/Gesangspädagogin, Mathematiker/Mathematikerin}|pwk}, wie Schnitzlers{ }\emph{Tagebuch}\pwindex{Tagebuch@\emph{Tagebuch}|pwk} zu entnehmen ist, in Salzburg\oindex{Salzburg@\textbf{Salzburg}, \emph{A.ADM2}|pwk} auf.}}}\label{K_L02922-3} ſtecken?\pend
           
\pstart
           Viele treue Grüße! {\\[\baselineskip]}Dein {\\[\baselineskip]}\spacefill\mbox{Paul Goldmnn}\pend
           \leftskip=0em{}\selectlanguage{ngerman}\endnumbering\briefempfaengerindex{Schnitzler, Arthur@\textsc{Schnitzler, Arthur}!zzzGoldmann, Paul@\emph{von Paul Goldmann}!1900-06-271@{27. 6. {[}1900{]}}|)be}\mylabel{L02922h}  \normalsize

\doendnotes{C}
\bigskip
\vfill

\clearpage

\footnotesize

\lohead{\textsc{register}}

% Definiere theindex-Environment komplett neu ohne reledmac
\makeatletter
\renewenvironment{theindex}{%
  \section*{\indexname}%
  \setlength{\parindent}{0pt}%
  \setlength{\parskip}{0pt plus 0.3pt}%
  \let\item\@idxitem
}{%
  \clearpage
}
\makeatother

\IfFileExists{\jobname-pw.ind}{\input{\jobname-pw.ind}}{}

\end{document}

      