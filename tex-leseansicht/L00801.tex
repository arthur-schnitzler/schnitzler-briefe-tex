%% latex-leseansicht-vorspann.tex
%% Vorspann für die Leseansicht.
%% Lädt die gemeinsame Datei latex-vorspann.tex mit nicht gesetztem Schalter.

\newif\ifkorrekturansicht
\korrekturansichtfalse

\input{../tex-inputs/latex-vorspann}


         
         \newcommand{\erwaehntePersonen}{Personen: }
         \newcommand{\erwaehnteInstitutionen}{}
         \newcommand{\erwaehnteOrte}{}
         \newcommand{\erwaehnteWerke}{
               \section[Arthur Schnitzler an Richard Beer-Hofmann, 4. 6. 1898]{ Arthur Schnitzler an Richard Beer-Hofmann, 4. 6. 1898}\nopagebreak\mylabel{v}\rehead{ }\begin{ledgroupsized}[t]{13cm}\normalsize\beginnumbering \toendnotes[C]{\smallbreak\pagebreak[2]} \Standort{YCGL, MSS 31.}
\physDesc{Brief, 1 Blatt, 3 Seiten, Umschlag
\newline{}Handschrift: Bleistift, deutsche Kurrent\newline{}Versand: 1) Stempel: »\nobreak{}\oindex{XXXX Ortsangabe fehlt|pwk}Wien 9/1, 4. 6. 98, 7–8 N\nobreak{}«.   2) Stempel: »\nobreak{}\oindex{XXXX Ortsangabe fehlt|pwk}Steindorf am Ossiacher See, 5 6 {[}98{]}\nobreak{}«. }\buchAbdrucke{\weitereDrucke{Arthur Schnitzler, Richard Beer-Hofmann: \emph{Briefwechsel 1891–1931}. Hg. Konstanze Fliedl. Wien, Zürich: \emph{Europaverlag} 1992, S. 117.} }\pstart{}{\pb}Herrn \textsc{Dr. Richard
                     Beer-Hofmann}\pend{}\pstart{}\textsc{Steindorf\oindex{XXXX Ortsangabe fehlt|pw}}\pend{}\pstart{}\textsc{am Ossiacher-See\oindex{XXXX Ortsangabe fehlt|pw}}\pend{}\pstart{}\textsc{Kärnthen}\oindex{XXXX Ortsangabe fehlt|pw}\pend{}{\bigskip}\pstart
           \raggedleft{}{\pb}Samſtag{ }Nachmitg{\\}4. 6. 98.\pend
           \pstart
           Lieber Richard, ich habe heute einen Postcarton an Ihre Adreſſe
               aufgegeben und komme bald nach. Morgen So{\geminationn}tag{ }früh 7.45 fahre ich auf den \textsc{Semmering}\oindex{XXXX Ortsangabe fehlt|pw}; dort ſetz ich mich aufs Rad und will ſehn, wie weit ich komme. Von der {\pb}Reiſe aus verſtändige ich Sie.
                  Dinſtag bin ich wohl in \textsc{Steindorf}\oindex{XXXX Ortsangabe fehlt|pw}. Ob \textsc{Kramer}\pwindex{\textcolor{red}{\textsuperscript{XXXX1 indx}}|pw} mitfährt, iſt ungewiſs. Ich
               glaub nicht. Eben telephonirt er mir, dſs ihm ſein Rad geſtohlen worden iſt; er will
               ſich gleich ein neues kaufen, aber – zum mindeſtens das letztere {\pb}iſt unſahrscheinlich. –\pend
           \pstart
           Herzlichen Gruſs. Ihren Brief hab ich heute früh beko{\geminationm}en; – »bete und
               arbeite« – d. h. ſchreiben Sie und lernen Sie \textsc{Bicycle}fahren.\pend
           \pstart Ihr \spacefill\mbox{Arthur Sch}\pend{}
         
         \endnumbering\mylabel{h}\end{ledgroupsized}  \newcommand{\dateiname}{L00801}\newcommand{\titel}{Arthur Schnitzler an Richard Beer-Hofmann, 4. 6. 1898}\newcommand{\editorInnen}{Martin Anton Müller und Gerd-Hermann Susen}%% latex-leseansicht-abspann.tex
%% Abspann für die Leseansicht.
%% Der Schalter \ifkorrekturansicht ist bereits durch den Vorspann gesetzt.

%% latex-abspann.tex
%% Gemeinsamer Abspann für Korrekturansicht und Leseansicht.
%% Setzt den Schalter \ifkorrekturansicht voraus (gesetzt in den
%% einbindenden Dateien latex-korrekturansicht-abspann.tex bzw.
%% latex-leseansicht-abspann.tex).
%% ---------------------------------------------------------------

\normalsize

% Das esempio-Environment wird nur in der Leseansicht benötigt
\ifkorrekturansicht\else
\newenvironment{esempio}[3]%
{
    \vspace{1.5ex}
    \rlap{\underline{#1}}
    \par
    \setlength{\parindent}{0cm}
    \nopagebreak
    \leftskip=#2cm
    \rightskip=#3cm
}
{
    \par
}
\fi

\doendnotes{C}
\bigskip
\vfill

\clearpage

\footnotesize

\ifkorrekturansicht
  \lohead{\textsc{register}}
\fi

% theindex-Environment neu definieren ohne reledmac
\makeatletter
\renewenvironment{theindex}{%
  \ifkorrekturansicht
    \section*{\indexname}%
  \else
    \subsubsection*{Index der erwähnten Entitäten}%
  \fi
  \setlength{\parindent}{0pt}%
  \setlength{\parskip}{0pt plus 0.3pt}%
  \let\item\@idxitem
}{%
  \ifkorrekturansicht\clearpage\fi
}
\makeatother

\IfFileExists{\jobname-pw.ind}{\input{\jobname-pw.ind}}{}

% Quellenangabe nur in der Leseansicht
\ifkorrekturansicht\else
% Fallback-Definitionen, falls die .tex-Datei \titel etc. nicht gesetzt hat
\providecommand{\titel}{}
\providecommand{\editorInnen}{}
\providecommand{\dateiname}{\jobname}

\vspace{3cm}

\vfill

\footnotesize
\textsc{Quelle}: \titel. Herausgegeben von {\editorInnen}. In: \emph{Arthur Schnitzler: Briefwechsel mit Autorinnen und Autoren}.
 Digitale Edition, https://schnitzler-briefe.acdh.oeaw.ac.at/{\dateiname}.html (Stand \today)
\fi

\end{document}


      