%% latex-korrekturansicht-vorspann.tex
%% Vorspann für die Korrekturansicht.
%% Lädt die gemeinsame Datei latex-vorspann.tex mit gesetztem Schalter.

\newif\ifkorrekturansicht
\korrekturansichttrue

\input{../tex-inputs/latex-vorspann}


\section[Arthur Schnitzler an Richard Beer-Hofmann, 16. 9. 1908]{L01790 Arthur Schnitzler an Richard Beer-Hofmann, 16. 9. 1908}
\nopagebreak\mylabel{L01790v}
\rehead{ }\normalsize\beginnumbering\briefempfaengerindex{Beer-Hofmann, Richard@\textsc{Beer-Hofmann, Richard}!zzzSchnitzler, Arthur@\emph{von Arthur Schnitzler}!1908-09-161@{16. 9. 1908}|(be}
\toendnotes[C]{\smallbreak\pagebreak[2]}\Standort{YCGL, MSS 31.}
\physDesc{Brief, 1 Blatt, 3 Seiten, Umschlag, 1101 Zeichen
\newline{}Handschrift: schwarze Tinte, deutsche Kurrent
\newline{}Versand: Stempel: »\nobreak{}Wien, 16, IX. 08, XII\nobreak{}«.  }
\buchAbdrucke{\weitereDrucke{Arthur Schnitzler, Richard Beer-Hofmann: \emph{Briefwechsel 1891–1931}. Wien, Zürich: \emph{Europaverlag} 1992, S. 190.} }\toendnotes[C]{\smallbreak}\pstart{}{\pb}\textcolor{gray}{\textbf{Dr. Arthur Schnitzler}}\pend{}\pstart{}\textcolor{gray}{\textbf{Wien XVIII. Spoettelgasse 7\oindex{Edmund-Weiss-Gasse 7@\textbf{Edmund-Weiß-Gasse 7}, \emph{Wohngebäude (K.WHS)}|pw}.}}\pend{}{\bigskip}\pstart{}{\pb}\textsc{Dr. Richard Beer-Hofmann,}\pend{}\pstart{}Wien XVIII\oindex{XVIII., Waehring@\textbf{XVIII., Währing}, \emph{A.ADM3}|pw}\pend{}\pstart{}\textsc{Hasenauerstr. 59}\oindex{Hasenauerstrasse 59@\textbf{Hasenauerstraße 59}, \emph{Wohngebäude (K.WHS)}|pw}.\pend{}{\bigskip}\vspace{1em}
\pstart
           {\pb}\textcolor{gray}{\textbf{Dr. Arthur Schnitzler}}\hfill 16. 9. 08\pend
           
\pstart
           \textcolor{gray}{\textbf{Wien XVIII. Spoettelgasse 7\oindex{Edmund-Weiss-Gasse 7@\textbf{Edmund-Weiß-Gasse 7}, \emph{Wohngebäude (K.WHS)}|pw}.}}\pend
           \vspace{0.5em}
\pstart
           lieber Richard, geſtern hab ich auf dem Umweg über Auſſee\oindex{Bad Aussee@\textbf{Bad Aussee}, \emph{P.PPLA3}|pw} – wo es Dr Rudi
                  Kaufmann\pwindex{Kaufmann, Rudolf 03.09.1871 – 20.06.1927@\textsc{Kaufmann, Rudolf} (03.09.1871 – 20.06.1927), \emph{Internist/Internistin}|pw} der Agnes Speyer\pwindex{Ulmann, Agnes 23. 12. 1875 – 1. 4. 1942@\textsc{Ulmann, Agnes} (23. 12. 1875 – 1. 4. 1942), \emph{Maler/Malerin, Bildhauer/Bildhauerin}|pw} erzählt hat,
                  verno{\geminationm}en, daſs man Paula\pwindex{Beer-Hofmann, Paula 25.02.1879 – 30.10.1939@\textsc{Beer-Hofmann, Paula} (25.02.1879 – 30.10.1939)|pw} von der überſtandenen Krankheit überhaupt nichts mehr anſieht – ſo
               darf man alſo hoffen, daſs alle Jammergründe verſchwunden ſind. Ihre Karte, aus \textsc{Seis\oindex{Seis am Schlern@\textbf{Seis am Schlern}, \emph{P.PPL}|pw}} nachgeſchickt, fand ich vorgeſtern Montag früh bei unſrer An{\pb}kunft aus München\oindex{Muenchen@\textbf{München}, \emph{P.PPLA}|pw} vor. Haben Sie unſre \label{K_L01790-1v}\edtext{Karte aus \textsc{Martino}\oindex{San Martino di Castrozza@\textbf{San Martino di Castrozza}, \emph{P.PPL}|pw}}{\lemma{\textnormal{\emph{Karte aus Martino}}}\Cendnote{\textnormal{nicht überliefert}}}\label{K_L01790-1} beko{\geminationm}en? –\pend
           
\pstart
           Wir ſind mit dem Auto – einem Poſtauto, also keinem \label{K_L01790-2v}\edtext{Nachkaſtl}{\lemma{\textnormal{\emph{Nachkaſtl}}}\Cendnote{\textnormal{Vgl. Arthur und Olga Schnitzler an Richard und Paula Beer-Hofmann,
               11. 5. 1908.
               }}}\label{K_L01790-2} von Bozen\oindex{Bozen@\textbf{Bozen}, \emph{P.PPLA2}|pw} hin u wieder zurückgefahren.
               In München\oindex{Muenchen@\textbf{München}, \emph{P.PPLA}|pw} war das intereſſanteſte, was wir
               geſehen haben, die \label{K_L01790-3v}\edtext{\textsc{Faust}\pwindex{Faust. Eine Tragoedie@\emph{Faust. Eine Tragödie}|pw} Inſcenirung}{\lemma{\textnormal{\emph{Faust Inſcenirung}}}\Cendnote{\textnormal{Siehe A. S.: \emph{Tagebuch}, 12. 9. 1908.
               }}}\label{K_L01790-3} von \textsc{Erler}\pwindex{Erler, Fritz 1868-12-15 – 1940-07-11@\textsc{Erler, Fritz} (1868-12-15 – 1940-07-11), \emph{Maler/Malerin, Bühnenbildner/Bühnenbildnerin, Künstler/Künstlerin}|pw} im Künſtleriſchen Theater\orgindex{Muenchner Kuenstlertheater@Münchner Künstlertheater|pw}. Auch das \label{K_L01790-4v}\edtext{Zwiſchenſpiel\pwindex{Zwischenspiel. Komoedie in drei Akten@\emph{Zwischenspiel. Komödie in drei Akten}|pw} hab ich erlebt}{\lemma{\textnormal{\emph{Zwiſchenſpiel … erlebt}}}\Cendnote{\textnormal{Siehe A. S.: \emph{Tagebuch}, 10. 9. 1908.
               }}}\label{K_L01790-4}, im Reſidenztheater\oindex{Residenztheater Muenchen@\textbf{Residenztheater München}, \emph{Theater (K.THE)}|pw}, aber es iſt mir
               ſchon beſſer. Von meinem Roman\pwindex{Weg ins Freie. Roman@\emph{Der Weg ins Freie. Roman}|pwv}{ }{\pb}kommt eben die 14.–20. Auflage. Ich werde trotzdem
               nicht \strikeout{a\textcolor{gray}{us}} irre an ihm{ }{\dots}\pend
           
\pstart
           Angefangen habe ich manches in \textsc{Seis\oindex{Seis am Schlern@\textbf{Seis am Schlern}, \emph{P.PPL}|pw}}; darüber mündlich. Wann kommen Sie – ? Ich ſchicke den Brief an Ihre Wien\oindex{Wien@\textbf{Wien}, \emph{A.ADM2}|pw}er Adreſſe, da Sie ſchon am 15.{ }\textsc{Venedig}\oindex{Venedig@\textbf{Venedig}, \emph{P.PPLA}|pw} verlaſſen.\pend
           
\pstart
           Ich wünſche von Herzen {\dotstwo} ebenſo wie Olga\pwindex{Schnitzler, Olga 17.01.1882 – 13.01.1970@\textsc{Schnitzler, Olga} (17.01.1882 – 13.01.1970), \emph{Schauspieler/Schauspielerin, Sänger/Sängerin}|pw}{ }{\dotstwo}{ }nun Sie wiſſen es Beide\pwindex{Beer-Hofmann, Paula 25.02.1879 – 30.10.1939@\textsc{Beer-Hofmann, Paula} (25.02.1879 – 30.10.1939)|pwv}. Grüßen Sie auch die Kinder\pwindex{Beer-Hofmann, Gabriel 09.01.1901 – 24.03.1971@\textsc{Beer-Hofmann, Gabriel} (09.01.1901 – 24.03.1971), \emph{Schriftsteller/Schriftstellerin, Filmagent/Filmagentin}|pwv}\pwindex{Beer-Hofmann, Naemah 20.12.1898 – 10.11.1971@\textsc{Beer-Hofmann, Naëmah} (20.12.1898 – 10.11.1971)|pwv}\pwindex{Beer-Hofmann, Mirjam 04.09.1897 – 24.12.1984@\textsc{Beer-Hofmann, Mirjam} (04.09.1897 – 24.12.1984)|pwv}.\pend
           
\pstart
           Ihr{\\[\baselineskip]}\spacefill\mbox{Arthur.}\pend
           \leftskip=0em{}\selectlanguage{ngerman}\endnumbering\briefempfaengerindex{Beer-Hofmann, Richard@\textsc{Beer-Hofmann, Richard}!zzzSchnitzler, Arthur@\emph{von Arthur Schnitzler}!1908-09-161@{16. 9. 1908}|)be}\mylabel{L01790h}  \normalsize

\doendnotes{C}
\bigskip
\vfill

\clearpage

\footnotesize

\lohead{\textsc{register}}

% Definiere theindex-Environment komplett neu ohne reledmac
\makeatletter
\renewenvironment{theindex}{%
  \section*{\indexname}%
  \setlength{\parindent}{0pt}%
  \setlength{\parskip}{0pt plus 0.3pt}%
  \let\item\@idxitem
}{%
  \clearpage
}
\makeatother

\IfFileExists{\jobname-pw.ind}{\input{\jobname-pw.ind}}{}

\end{document}

      