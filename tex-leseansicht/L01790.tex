%% latex-leseansicht-vorspann.tex
%% Vorspann für die Leseansicht.
%% Lädt die gemeinsame Datei latex-vorspann.tex mit nicht gesetztem Schalter.

\newif\ifkorrekturansicht
\korrekturansichtfalse

\input{../tex-inputs/latex-vorspann}


         
         \renewcommand{\erwaehntePersonen}{Personen: Richard Beer-Hofmann, Paula Beer-Hofmann, Gabriel Beer-Hofmann, Naëmah Beer-Hofmann, Mirjam Beer-Hofmann, Fritz Erler, Rudolf Kaufmann, Olga Schnitzler, Agnes Ulmann}
         \renewcommand{\erwaehnteInstitutionen}{Institutionen: Münchner Künstlertheater}
         \renewcommand{\erwaehnteOrte}{Orte: Bad Aussee, Bozen, Edmund-Weiß-Gasse, Hasenauerstraße, München, Residenztheater München, San Martino di Castrozza, Seis am Schlern, Venedig, Wien, XVIII., Währing}
         \renewcommand{\erwaehnteWerke}{Werke: Der Weg ins Freie. Roman, Faust. Eine Tragödie, Zwischenspiel. Komödie in drei Akten}
               \section[Arthur Schnitzler an Richard Beer-Hofmann, 16. 9. 1908]{ Arthur Schnitzler an Richard Beer-Hofmann, 16. 9. 1908}\nopagebreak\mylabel{v}\rehead{ }\begin{ledgroupsized}[t]{13cm}\normalsize\beginnumbering\briefempfaengerindex{Beer-Hofmann, Richard@\textsc{Beer-Hofmann, Richard}!zzzSchnitzler, Arthur@\emph{von Arthur Schnitzler}!1908-09-161@{16. 9. 1908}|(be} \toendnotes[C]{\smallbreak\pagebreak[2]} \Standort{YCGL, MSS 31.}
\physDesc{Brief, 1 Blatt, 3 Seiten, Umschlag, 1101 Zeichen (Briefpapier mit Trauerrand)
\newline{}Handschrift: schwarze Tinte, deutsche Kurrent
\newline{}Versand: Stempel: »\nobreak{}Wien, 16, IX. 08, XII\nobreak{}«.  }\buchAbdrucke{\weitereDrucke{Arthur Schnitzler, Richard Beer-Hofmann: \emph{Briefwechsel 1891–1931}. Hg. Konstanze Fliedl. Wien, Zürich: \emph{Europaverlag} 1992, S. 190.} }\toendnotes[C]{\smallbreak}\pstart{}{\pb}\textcolor{gray}{\textbf{Dr. Arthur Schnitzler}}\pend{}\pstart{}\textcolor{gray}{\textbf{Wien XVIII. Spoettelgasse 7\oindex{XXXX Ortsangabe fehlt|pw}.}}\pend{}{\bigskip}\pstart{}{\pb}\textsc{Dr. Richard Beer-Hofmann,}\pend{}\pstart{}Wien XVIII\oindex{XVIII., Waehring@\textbf{XVIII., Währing}|pw}\pend{}\pstart{}\textsc{Hasenauerstr. 59}\oindex{Hasenauerstrasse@\textbf{Hasenauerstraße}|pw}.\pend{}{\bigskip}\pstart
           \noindent{}{\pb}\textcolor{gray}{\textbf{Dr. Arthur Schnitzler}}\hfill 16. 9. 08\pend
           \pstart
           \textcolor{gray}{\textbf{Wien XVIII. Spoettelgasse 7\oindex{XXXX Ortsangabe fehlt|pw}.}}\pend
           \pstart
           lieber Richard, geſtern hab ich auf dem Umweg über Auſſee\oindex{Bad Aussee@\textbf{Bad Aussee}|pw} – wo es Dr Rudi
                  Kaufmann\pwindex{Kaufmann, Rudolf 03.09.1871 – 20.06.1927@\textsc{Kaufmann, Rudolf} (03.09.1871 – 20.06.1927), \emph{Mediziner}|pw} der Agnes Speyer\pwindex{Ulmann, Agnes 23. 12. 1875 – 1. 4. 1942@\textsc{Ulmann, Agnes} (23. 12. 1875 – 1. 4. 1942), \emph{Bildende Künstlerin >> Maler, Bildende Künstlerin >> Bildhauer}|pw} erzählt hat,
                  verno{\geminationm}en, daſs man Paula\pwindex{Beer-Hofmann, Paula 25.02.1879 – 30.10.1939@\textsc{Beer-Hofmann, Paula} (25.02.1879 – 30.10.1939)|pw} von der überſtandenen Krankheit überhaupt nichts mehr anſieht – ſo
               darf man alſo hoffen, daſs alle Jammergründe verſchwunden ſind. Ihre Karte, aus \textsc{Seis\oindex{Seis am Schlern@\textbf{Seis am Schlern}|pw}} nachgeſchickt, fand ich vorgeſtern Montag früh bei unſrer An{\pb}kunft aus München\oindex{Muenchen@\textbf{München}|pw} vor. Haben Sie unſre \label{K_L01790-1v}\edtext{Karte aus \textsc{Martino}\oindex{San Martino di Castrozza@\textbf{San Martino di Castrozza}|pw}}{\lemma{\textnormal{\emph{Karte aus Martino}}}\Cendnote{\textnormal{nicht überliefert}}}\label{K_L01790-1h} beko{\geminationm}en? –\pend
           \pstart
           Wir ſind mit dem Auto – einem Poſtauto, also keinem \label{K_L01790-2v}\edtext{Nachkaſtl}{\lemma{\textnormal{\emph{Nachkaſtl}}}\Cendnote{\textnormal{vgl. Arthur und Olga Schnitzler an Richard und Paula Beer-Hofmann,
               11. 5. 1908}}}\label{K_L01790-2h} von Bozen\oindex{Bozen@\textbf{Bozen}|pw} hin u wieder zurückgefahren.
               In München\oindex{Muenchen@\textbf{München}|pw} war das intereſſanteſte, was wir
               geſehen haben, die \label{K_L01790-3v}\edtext{\textsc{Faust}\pwindex{\textcolor{red}{\textsuperscript{XXXX1 indx}}!Faust. Eine Tragoedie1808@\strich\emph{Faust. Eine Tragödie} {[}1808{]}|pw} Inſcenirung}{\lemma{\textnormal{\emph{Faust Inſcenirung}}}\Cendnote{\textnormal{siehe A. S.: \emph{Tagebuch}, 12. 9. 1908}}}\label{K_L01790-3h} von \textsc{Erler}\pwindex{Erler, Fritz 1868-12-15 – 1940-07-11@\textsc{Erler, Fritz} (1868-12-15 – 1940-07-11), \emph{Bildender Künstler, Bildender Künstler, Künstler}|pw} im Künſtleriſchen Theater\orgindex{Muenchner Kuenstlertheater@Münchner Künstlertheater|pw}. Auch das \label{K_L01790-4v}\edtext{Zwiſchenſpiel\pwindex{Schnitzler, Arthur 15.05.1862 – 21.10.1931@\textsc{Schnitzler, Arthur} (15.05.1862 – 21.10.1931), \emph{Schriftsteller, Mediziner}!Zwischenspiel. Komoedie in drei Akten1905-10-12@\strich\emph{Zwischenspiel. Komödie in drei Akten} {[}1905-10-12{]}|pw} hab ich erlebt}{\lemma{\textnormal{\emph{Zwiſchenſpiel … erlebt}}}\Cendnote{\textnormal{siehe A. S.: \emph{Tagebuch}, 10. 9. 1908}}}\label{K_L01790-4h}, im Reſidenztheater\oindex{Residenztheater Muenchen@\textbf{Residenztheater München}|pw}, aber es iſt mir
               ſchon beſſer. Von meinem Roman\pwindex{Schnitzler, Arthur 15.05.1862 – 21.10.1931@\textsc{Schnitzler, Arthur} (15.05.1862 – 21.10.1931), \emph{Schriftsteller, Mediziner}!Weg ins Freie. Roman1.1.1908 – 1.6.1908@\strich\emph{Der Weg ins Freie. Roman} {[}1.1.1908 – 1.6.1908{]}|pwv}{ }{\pb}kommt eben die 14.–20. Auflage. Ich werde trotzdem
               nicht \strikeout{a\textcolor{gray}{us}} irre an ihm{ }{\dots}\pend
           \pstart
           Angefangen habe ich manches in \textsc{Seis\oindex{Seis am Schlern@\textbf{Seis am Schlern}|pw}}; darüber mündlich. Wann kommen Sie – ? Ich ſchicke den Brief an Ihre Wien\oindex{Wien@\textbf{Wien}|pw}er Adreſſe, da Sie ſchon am 15.{ }\textsc{Venedig}\oindex{Venedig@\textbf{Venedig}|pw} verlaſſen.\pend
           \pstart
           Ich wünſche von Herzen {\dotstwo} ebenſo wie Olga\pwindex{Schnitzler, Olga 17.01.1882 – 13.01.1970@\textsc{Schnitzler, Olga} (17.01.1882 – 13.01.1970), \emph{Schauspielerin, Sängerin}|pw}{ }{\dotstwo}{ }nun Sie wiſſen es Beide\pwindex{Beer-Hofmann, Paula 25.02.1879 – 30.10.1939@\textsc{Beer-Hofmann, Paula} (25.02.1879 – 30.10.1939)|pwv}. Grüßen Sie auch die Kinder\pwindex{Beer-Hofmann, Gabriel 09.01.1901 – 24.03.1971@\textsc{Beer-Hofmann, Gabriel} (09.01.1901 – 24.03.1971), \emph{Schriftsteller, Filmagent}|pwv}\pwindex{Beer-Hofmann, Naemah 20.12.1898 – 10.11.1971@\textsc{Beer-Hofmann, Naëmah} (20.12.1898 – 10.11.1971)|pwv}\pwindex{Beer-Hofmann, Mirjam 04.09.1897 – 24.12.1984@\textsc{Beer-Hofmann, Mirjam} (04.09.1897 – 24.12.1984)|pwv}.\pend
           \pstart
           Ihr{\\[\baselineskip]}\spacefill\mbox{Arthur.}\pend
           \leftskip=0em{}
         
         \endnumbering\mylabel{h}\end{ledgroupsized}  \newcommand{\dateiname}{L01790}\newcommand{\titel}{Arthur Schnitzler an Richard Beer-Hofmann, 16. 9. 1908}\newcommand{\editorInnen}{Martin Anton Müller und Gerd-Hermann Susen}%% latex-leseansicht-abspann.tex
%% Abspann für die Leseansicht.
%% Der Schalter \ifkorrekturansicht ist bereits durch den Vorspann gesetzt.

%% latex-abspann.tex
%% Gemeinsamer Abspann für Korrekturansicht und Leseansicht.
%% Setzt den Schalter \ifkorrekturansicht voraus (gesetzt in den
%% einbindenden Dateien latex-korrekturansicht-abspann.tex bzw.
%% latex-leseansicht-abspann.tex).
%% ---------------------------------------------------------------

\normalsize

% Das esempio-Environment wird nur in der Leseansicht benötigt
\ifkorrekturansicht\else
\newenvironment{esempio}[3]%
{
    \vspace{1.5ex}
    \rlap{\underline{#1}}
    \par
    \setlength{\parindent}{0cm}
    \nopagebreak
    \leftskip=#2cm
    \rightskip=#3cm
}
{
    \par
}
\fi

\doendnotes{C}
\bigskip
\vfill

\clearpage

\footnotesize

\ifkorrekturansicht
  \lohead{\textsc{register}}
\fi

% theindex-Environment neu definieren ohne reledmac
\makeatletter
\renewenvironment{theindex}{%
  \ifkorrekturansicht
    \section*{\indexname}%
  \else
    \subsubsection*{Index der erwähnten Entitäten}%
  \fi
  \setlength{\parindent}{0pt}%
  \setlength{\parskip}{0pt plus 0.3pt}%
  \let\item\@idxitem
}{%
  \ifkorrekturansicht\clearpage\fi
}
\makeatother

\IfFileExists{\jobname-pw.ind}{\input{\jobname-pw.ind}}{}

% Quellenangabe nur in der Leseansicht
\ifkorrekturansicht\else
% Fallback-Definitionen, falls die .tex-Datei \titel etc. nicht gesetzt hat
\providecommand{\titel}{}
\providecommand{\editorInnen}{}
\providecommand{\dateiname}{\jobname}

\vspace{3cm}

\vfill

\footnotesize
\textsc{Quelle}: \titel. Herausgegeben von {\editorInnen}. In: \emph{Arthur Schnitzler: Briefwechsel mit Autorinnen und Autoren}.
 Digitale Edition, https://schnitzler-briefe.acdh.oeaw.ac.at/{\dateiname}.html (Stand \today)
\fi

\end{document}


      