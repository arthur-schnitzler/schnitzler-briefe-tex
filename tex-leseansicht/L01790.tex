%% latex-leseansicht-vorspann.tex
%% Vorspann für die Leseansicht.
%% Lädt die gemeinsame Datei latex-vorspann.tex mit nicht gesetztem Schalter.

\newif\ifkorrekturansicht
\korrekturansichtfalse

\input{../tex-inputs/latex-vorspann}


\section[Arthur Schnitzler an Richard Beer-Hofmann, 16. 9. 1908]{L01790 Arthur Schnitzler an Richard Beer-Hofmann, 16. 9. 1908}
\nopagebreak\mylabel{L01790v}
\rehead{ }\normalsize\beginnumbering\briefempfaengerindex{Beer-Hofmann, Richard@\textsc{Beer-Hofmann, Richard}!zzzSchnitzler, Arthur@\emph{von Arthur Schnitzler}!1908-09-161@{16. 9. 1908}|(be}
\toendnotes[C]{\smallbreak\pagebreak[2]}
\correspDesc{Versand  durch Arthur Schnitzler am 16. 9. 1908 in Wien
\newline{}Erhalt  durch Richard Beer-Hofmann im Zeitraum [16. 9. 1908
                  – 20. 9. 1908?] in Wien}\toendnotes[C]{\smallbreak}
\Standort{YCGL, MSS 31.}
\physDesc{Brief, 1 Blatt, 3 Seiten, Kuvert, 1101 Zeichen
\newline{}Handschrift: schwarze Tinte, deutsche Kurrent
\newline{}Versand: Stempel: »\nobreak{}\oindex{Wien@\textbf{Wien}, \emph{Verwaltungsgebiet}|pwk}Wien, 16, IX. 08, XII\nobreak{}«.  }
\buchAbdrucke{\weitereDrucke{Arthur Schnitzler, Richard Beer-Hofmann: \emph{Briefwechsel 1891–1931}. Herausgegeben von Konstanze Fliedl. Wien, Zürich: \emph{Europaverlag} 1992, S. 190.} }\toendnotes[C]{\smallbreak}\pstart{}{\pb}\textcolor{gray}{\textbf{Dr. Arthur Schnitzler}}\pend{}\pstart{}\textcolor{gray}{\textbf{Wien XVIII. Spoettelgasse 7\oindex{Wien@\textbf{Wien}!XVIII., Währing@\textbf{XVIII., Währing}!Edmund-Weiß-Gasse 7@\textbf{Edmund-Weiß-Gasse 7}, \emph{Wohngebäude}|pw}.}}\pend{}{\bigskip}\pstart{}{\pb}\textsc{Dr. Richard Beer-Hofmann,}\pend{}\pstart{}Wien XVIII\oindex{XVIII., Währing@\textbf{XVIII., Währing}, \emph{Verwaltungsgebiet}|pw}\pend{}\pstart{}\textsc{Hasenauerstr. 59}\oindex{Wien@\textbf{Wien}!XVIII., Währing@\textbf{XVIII., Währing}!Hasenauerstraße 59@\textbf{Hasenauerstraße 59}, \emph{Wohngebäude}|pw}.\pend{}{\bigskip}\vspace{1em}
\pstart
           {\pb}\textcolor{gray}{\textbf{Dr. Arthur Schnitzler}}\hfill 16. 9. 08\pend
           
\pstart
           \textcolor{gray}{\textbf{Wien XVIII. Spoettelgasse 7\oindex{Wien@\textbf{Wien}!XVIII., Währing@\textbf{XVIII., Währing}!Edmund-Weiß-Gasse 7@\textbf{Edmund-Weiß-Gasse 7}, \emph{Wohngebäude}|pw}.}}\pend
           \vspace{0.5em}
\pstart
           lieber Richard, geſtern hab ich auf dem Umweg über Auſſee\oindex{Bad Aussee@\textbf{Bad Aussee}, \emph{Hauptstadt}|pw} – wo es Dr Rudi
                  Kaufmann\pwindex{Kaufmann, Rudolf 3.\,9.\,1871 Wien – 20.\,6.\,1927 ebd.@\textsc{Kaufmann, Rudolf} (3.\,9.\,1871 Wien – 20.\,6.\,1927 ebd.), \emph{Internist}|pw} der Agnes Speyer\pwindex{Ulmann, Agnes 23.\,12.\,1875 Wien – 1.\,4.\,1942 New York City@\textsc{Ulmann, Agnes} (23.\,12.\,1875 Wien – 1.\,4.\,1942 New York City), \emph{Malerin, Bildhauerin}|pw} erzählt hat,
                  verno{\geminationm}en, daſs man Paula\pwindex{Beer-Hofmann, Paula 25.\,2.\,1879 Wien – 30.\,10.\,1939 Zürich@\textsc{Beer-Hofmann, Paula} (25.\,2.\,1879 Wien – 30.\,10.\,1939 Zürich)|pw} von der überſtandenen Krankheit überhaupt nichts mehr anſieht –{ }ſo
               darf man alſo hoffen, daſs alle Jammergründe verſchwunden{ }ſind. Ihre Karte, aus \textsc{Seis\oindex{Seis am Schlern@\textbf{Seis am Schlern}|pw}} nachgeſchickt, fand ich vorgeſtern Montag früh bei unſrer An{\pb}kunft aus München\oindex{München@\textbf{München}|pw} vor. Haben Sie unſre \label{K_L01790-1v}\edtext{Karte aus \textsc{Martino}\oindex{San Martino di Castrozza@\textbf{San Martino di Castrozza}|pw}}{\lemma{\textnormal{\emph{Karte aus Martino}}}\Cendnote{\textnormal{nicht überliefert}}}\label{K_L01790-1} beko{\geminationm}en? –\pend
           
\pstart
           Wir{ }ſind mit dem Auto – einem Poſtauto, also keinem \label{K_L01790-2v}\edtext{Nachkaſtl}{\lemma{\textnormal{\emph{Nachkastl}}}\Cendnote{\textnormal{Vgl. XXXX Auszeichnungsfehler: Dokument L01771 nicht gefunden.
               }}}\label{K_L01790-2} von Bozen\oindex{Bozen@\textbf{Bozen}, \emph{Hauptstadt}|pw} hin u wieder zurückgefahren.
               In München\oindex{München@\textbf{München}|pw} war das intereſſanteſte, was wir
               geſehen haben, die \label{K_L01790-3v}\edtext{\textsc{Faust}\pwindex{\textcolor{red}{\textsuperscript{XXXX indx1}}!Faust. Eine Tragödie@\strich\emph{Faust. Eine Tragödie}|pw} Inſcenirung}{\lemma{\textnormal{\emph{Faust Inscenirung}}}\Cendnote{\textnormal{Siehe A. S.: \emph{Tagebuch}, 12. 9. 1908.
               }}}\label{K_L01790-3} von \textsc{Erler}\pwindex{Erler, Fritz 15.\,12.\,1868 Ząbkowice Śląskie – 11.\,7.\,1940 München@\textsc{Erler, Fritz} (15.\,12.\,1868 Ząbkowice Śląskie – 11.\,7.\,1940 München), \emph{Maler, Bühnenbildner, Künstler}|pw} im Künſtleriſchen Theater\orgindex{Münchner Künstlertheater@Münchner Künstlertheater|pw}. Auch das \label{K_L01790-4v}\edtext{Zwiſchenſpiel\pwindex{Schnitzler, Arthur 15.\,5.\,1862 Wien – 21.\,10.\,1931 ebd.@\textsc{Schnitzler, Arthur} (15.\,5.\,1862 Wien – 21.\,10.\,1931 ebd.), \emph{Schriftsteller, Mediziner}!Zwischenspiel. Komödie in drei Akten@\strich\emph{Zwischenspiel. Komödie in drei Akten}|pw} hab ich erlebt}{\lemma{\textnormal{\emph{Zwischenspiel … erlebt}}}\Cendnote{\textnormal{Siehe A. S.: \emph{Tagebuch}, 10. 9. 1908.
               }}}\label{K_L01790-4}, im Reſidenztheater\oindex{Residenztheater München@\textbf{Residenztheater München}, \emph{Theater}|pw}, aber es iſt mir{ }ſchon beſſer. Von meinem Roman\pwindex{Schnitzler, Arthur 15.\,5.\,1862 Wien – 21.\,10.\,1931 ebd.@\textsc{Schnitzler, Arthur} (15.\,5.\,1862 Wien – 21.\,10.\,1931 ebd.), \emph{Schriftsteller, Mediziner}!Weg ins Freie. Roman@\strich\emph{Der Weg ins Freie. Roman}|pwv}{ }{\pb}kommt eben die 14.–20. Auflage. Ich werde trotzdem
               nicht \strikeout{a\textcolor{gray}{us}} irre an ihm{ }{\dots}\pend
           
\pstart
           Angefangen habe ich manches in \textsc{Seis\oindex{Seis am Schlern@\textbf{Seis am Schlern}|pw}}; darüber mündlich. Wann kommen Sie – ? Ich{ }ſchicke den Brief an Ihre Wien\oindex{Wien@\textbf{Wien}, \emph{Verwaltungsgebiet}|pw}er Adreſſe, da Sie{ }ſchon am 15.{ }\textsc{Venedig}\oindex{Venedig@\textbf{Venedig}|pw} verlaſſen.\pend
           
\pstart
           Ich wünſche von Herzen {\dotstwo} ebenſo wie Olga\pwindex{Schnitzler, Olga 17.\,1.\,1882 Wien – 13.\,1.\,1970 Lugano@\textsc{Schnitzler, Olga} (17.\,1.\,1882 Wien – 13.\,1.\,1970 Lugano), \emph{Schauspielerin, Sängerin}|pw}{ }{\dotstwo}{ }nun Sie wiſſen es Beide\pwindex{Beer-Hofmann, Paula 25.\,2.\,1879 Wien – 30.\,10.\,1939 Zürich@\textsc{Beer-Hofmann, Paula} (25.\,2.\,1879 Wien – 30.\,10.\,1939 Zürich)|pwv}. Grüßen Sie auch die Kinder\pwindex{Beer-Hofmann, Gabriel 9.\,1.\,1901 Wien – 24.\,3.\,1971 St Albans@\textsc{Beer-Hofmann, Gabriel} (9.\,1.\,1901 Wien – 24.\,3.\,1971 St Albans), \emph{Schriftsteller, Filmagent}|pwv}\pwindex{Beer-Hofmann, Naëmah 20.\,12.\,1898 Wien – 10.\,11.\,1971 New York City@\textsc{Beer-Hofmann, Naëmah} (20.\,12.\,1898 Wien – 10.\,11.\,1971 New York City)|pwv}\pwindex{Beer-Hofmann, Mirjam 4.\,9.\,1897 Wien – 24.\,12.\,1984 New York City@\textsc{Beer-Hofmann, Mirjam} (4.\,9.\,1897 Wien – 24.\,12.\,1984 New York City)|pwv}.\pend
           
\pstart
           Ihr{\\[\baselineskip]}\spacefill\mbox{Arthur.}\pend
           \leftskip=0em{}\selectlanguage{ngerman}\endnumbering\briefempfaengerindex{Beer-Hofmann, Richard@\textsc{Beer-Hofmann, Richard}!zzzSchnitzler, Arthur@\emph{von Arthur Schnitzler}!1908-09-161@{16. 9. 1908}|)be}\mylabel{L01790h}  \newcommand{\dateiname}{L01790}\newcommand{\titel}{Arthur Schnitzler an Richard Beer-Hofmann, 16. 9. 1908}\newcommand{\editorInnen}{Martin Anton Müller und Gerd-Hermann Susen}%% latex-leseansicht-abspann.tex
%% Abspann für die Leseansicht.
%% Der Schalter \ifkorrekturansicht ist bereits durch den Vorspann gesetzt.

%% latex-abspann.tex
%% Gemeinsamer Abspann für Korrekturansicht und Leseansicht.
%% Setzt den Schalter \ifkorrekturansicht voraus (gesetzt in den
%% einbindenden Dateien latex-korrekturansicht-abspann.tex bzw.
%% latex-leseansicht-abspann.tex).
%% ---------------------------------------------------------------

\normalsize

% Das esempio-Environment wird nur in der Leseansicht benötigt
\ifkorrekturansicht\else
\newenvironment{esempio}[3]%
{
    \vspace{1.5ex}
    \rlap{\underline{#1}}
    \par
    \setlength{\parindent}{0cm}
    \nopagebreak
    \leftskip=#2cm
    \rightskip=#3cm
}
{
    \par
}
\fi

\doendnotes{C}
\bigskip
\vfill

\clearpage

\footnotesize

\ifkorrekturansicht
  \lohead{\textsc{register}}
\fi

% theindex-Environment neu definieren ohne reledmac
\makeatletter
\renewenvironment{theindex}{%
  \ifkorrekturansicht
    \section*{\indexname}%
  \else
    \subsubsection*{Index der erwähnten Entitäten}%
  \fi
  \setlength{\parindent}{0pt}%
  \setlength{\parskip}{0pt plus 0.3pt}%
  \let\item\@idxitem
}{%
  \ifkorrekturansicht\clearpage\fi
}
\makeatother

\IfFileExists{\jobname-pw.ind}{\input{\jobname-pw.ind}}{}

% Quellenangabe nur in der Leseansicht
\ifkorrekturansicht\else
% Fallback-Definitionen, falls die .tex-Datei \titel etc. nicht gesetzt hat
\providecommand{\titel}{}
\providecommand{\editorInnen}{}
\providecommand{\dateiname}{\jobname}

\vspace{3cm}

\vfill

\footnotesize
\textsc{Quelle}: \titel. Herausgegeben von {\editorInnen}. In: \emph{Arthur Schnitzler: Briefwechsel mit Autorinnen und Autoren}.
 Digitale Edition, https://schnitzler-briefe.acdh.oeaw.ac.at/{\dateiname}.html (Stand \today)
\fi

\end{document}


