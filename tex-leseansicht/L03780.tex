%% latex-leseansicht-vorspann.tex
%% Vorspann für die Leseansicht.
%% Lädt die gemeinsame Datei latex-vorspann.tex mit nicht gesetztem Schalter.

\newif\ifkorrekturansicht
\korrekturansichtfalse

\input{../tex-inputs/latex-vorspann}


\section[Arthur Schnitzler an Stefan Zweig, 2[4?]. 5. 1913]{L03780 Arthur Schnitzler an Stefan Zweig, 2[4?]. 5. 1913}
\nopagebreak\mylabel{L03780v}
\rehead{ }\normalsize\beginnumbering\briefempfaengerindex{Zweig, Stefan@\textsc{Zweig, Stefan}!zzzSchnitzler, Arthur@\emph{von Arthur Schnitzler}!1913-05-241@{2[4?]. 5. 1913}|(be}
\toendnotes[C]{\smallbreak\pagebreak[2]}
\correspDesc{Versand  durch Arthur Schnitzler am 2[4?]. 5. 1913 in Wien
\newline{}Erhalt  durch Stefan Zweig im Zeitraum [24. 5. 1913
                  – 28. 5. 1913?] in Wien}\toendnotes[C]{\smallbreak}
\Standort{Jerusalem, National Library of Israel, ARC. Ms. Var. 305 1 58 Stefan Zweig Collection.}
\physDesc{Briefkarte, 1360 Zeichen
\newline{}Handschrift: schwarze Tinte, lateinische Kurrent}\toendnotes[C]{\smallbreak}
\pstart
           {\pb}\textcolor{gray}{\textbf{Dr. Arthur Schnitzler}}\hfill 2\textcolor{gray}{4}. 5. 913\pend
           
\pstart
           \textcolor{gray}{\textbf{Wien XVIII. Sternwartestrasse 71\oindex{Wien@\textbf{Wien}!XVIII., Währing@\textbf{XVIII., Währing}!Sternwartestraße 71@\textbf{Sternwartestraße 71}, \emph{Wohngebäude}|pw}}}\pend
           \vspace{0.5em}
\pstart
           lieber Herr Doctor, Ihr{ }ſchöner \label{K_L03780-1v}\edtext{Brief}{\lemma{\textnormal{\emph{Brief}}}\Cendnote{\textnormal{XXXX Auszeichnungsfehler: Dokument L03641 nicht gefunden. }}}\label{K_L03780-1} hat mir wahrhaft
               wohlgethan. So sicher ich bei dem Dichter des »Kinderlands\pwindex{Zweig, Stefan 28.\,11.\,1881 Wien – 23.\,2.\,1942 Petrópolis@\textsc{Zweig, Stefan} (28.\,11.\,1881 Wien – 23.\,2.\,1942 Petrópolis), \emph{Schriftsteller}!Erstes Erlebnis. Vier Geschichten aus Kinderland@\strich\emph{Erstes Erlebnis. Vier Geschichten aus Kinderland}|pw}« auf vollko{\geminationm}enes Verständnis
               gefasst sein durfte (Ihre Bedenken hinsichtlich der Schlusses theil ich sogar – seit
               einiger Zeit erst); die warme menschliche Antheilnahme die Sie an meinem Schaffen
               haben und deren ich i{\geminationm}er gewiss war, hat sich selten so
               lebhaft ausgedrückt als in Ihren letzten Worten, für die ich Ihnen freundschaftlichst
               die Hand drücke. –\pend
           
\pstart
           Ich danke auch für die Einladg zur \label{K_L03780-2v}\edtext{Bahr\pwindex{Bahr, Hermann 19.\,7.\,1863 Linz – 15.\,1.\,1934 München@\textsc{Bahr, Hermann} (19.\,7.\,1863 Linz – 15.\,1.\,1934 München), \emph{Schriftsteller, Kritiker}|pw} Feier\eventindex{Elektrotechnisches Institut der Technischen Universität@\textbf{Elektrotechnisches Institut der Technischen Universität}!Hermann-Bahr-Feier, 26.5.1913@Hermann-Bahr-Feier, 26.5.1913|pw}}{\lemma{\textnormal{\emph{Bahr Feier}}}\Cendnote{\textnormal{Siehe XXXX Auszeichnungsfehler: Dokument L03641 nicht gefunden.}}}\label{K_L03780-2} u. bitte
               zugleich um Entschuldigg, daß ich nicht kommen werde. Sie wissen ja, daß ich mich
               (aus Gründen, die nicht ausschließlich \label{K_L03780-3v}\edtext{nervöser Natur}{\lemma{\textnormal{\emph{nervöser Natur}}}\Cendnote{\textnormal{Schnitzler litt an Tinnitus.}}}\label{K_L03780-3} sind) von solchen {\pb}Veranstaltungen wie es nur irgend angeht fern halte (das
                  \label{K_L03780-4v}\edtext{Hauptmann\pwindex{Hauptmann, Gerhart 15.\,11.\,1862 Szczawno-Zdrój – 6.\,6.\,1946 Jagniątków@\textsc{Hauptmann, Gerhart} (15.\,11.\,1862 Szczawno-Zdrój – 6.\,6.\,1946 Jagniątków), \emph{Schriftsteller}|pw} Bankett\eventindex{Österreichischer Ingenieur- und Architektenverein@\textbf{Österreichischer Ingenieur- und Architektenverein}!Hauptmann-Bankett der Concordia, 17.11.1912@Hauptmann-Bankett der Concordia, 17.11.1912|pw}}{\lemma{\textnormal{\emph{Hauptmann Bankett}}}\Cendnote{\textnormal{ Das Bankett zu Ehren von Gerhart Hauptmann\pwindex{Hauptmann, Gerhart 15.\,11.\,1862 Szczawno-Zdrój – 6.\,6.\,1946 Jagniątków@\textsc{Hauptmann, Gerhart} (15.\,11.\,1862 Szczawno-Zdrój – 6.\,6.\,1946 Jagniątków), \emph{Schriftsteller}|pwk}\eventindex{Österreichischer Ingenieur- und Architektenverein@\textbf{Österreichischer Ingenieur- und Architektenverein}!Hauptmann-Bankett der Concordia, 17.11.1912@Hauptmann-Bankett der Concordia, 17.11.1912|pwk} wurde vom Journalisten- und Schriftstellerverein \emph{Concordia}\orgindex{Concordia. Journalisten- und Schriftstellerverein@Concordia. Journalisten- und Schriftstellerverein|pwk} veranstaltet und fand am 17. 11. 1912 im Österreichischer
                     Ingenieur- und Architektenverein\oindex{Wien@\textbf{Wien}!I., Innere Stadt@\textbf{I., Innere Stadt}!Österreichischer Ingenieur- und Architektenverein@\textbf{Österreichischer Ingenieur- und Architektenverein}|pwk} statt. }}}\label{K_L03780-4} war eine Ausnahme, weil
               ich, nach einem Misverständnis zwischen Hauptma{\geminationn}\pwindex{Hauptmann, Gerhart 15.\,11.\,1862 Szczawno-Zdrój – 6.\,6.\,1946 Jagniątków@\textsc{Hauptmann, Gerhart} (15.\,11.\,1862 Szczawno-Zdrój – 6.\,6.\,1946 Jagniątków), \emph{Schriftsteller}|pw} u. mir die Gelegenheit benutzen mußte ihm zu begegnen) – auch Bahr\pwindex{Bahr, Hermann 19.\,7.\,1863 Linz – 15.\,1.\,1934 München@\textsc{Bahr, Hermann} (19.\,7.\,1863 Linz – 15.\,1.\,1934 München), \emph{Schriftsteller, Kritiker}|pw} (der übrigens glaub ich dasselbe thut) kennt diese meine
               Gepflogenheit und wird fern davon sein mir mein Ausbleiben übel zu nehmen. Sie aber,
               lieber Freund, bitt ich um das gleiche – und zugleich um Mittheilg wo Ihre \label{K_L03780-5v}\edtext{Rede\pwindex{Zweig, Stefan 28.\,11.\,1881 Wien – 23.\,2.\,1942 Petrópolis@\textsc{Zweig, Stefan} (28.\,11.\,1881 Wien – 23.\,2.\,1942 Petrópolis), \emph{Schriftsteller}!Hermann Bahr, der Fünfzigjährige. (Eine Rede im Akademischen Verband für Literatur)@\strich\emph{Hermann Bahr, der Fünfzigjährige. (Eine Rede im Akademischen Verband für Literatur)}|pwv} ausführlich in Druck}{\lemma{\textnormal{\emph{Rede … Druck}}}\Cendnote{\textnormal{Stefan Zweig\pwindex{Zweig, Stefan 28.\,11.\,1881 Wien – 23.\,2.\,1942 Petrópolis@\textsc{Zweig, Stefan} (28.\,11.\,1881 Wien – 23.\,2.\,1942 Petrópolis), \emph{Schriftsteller}|pwk}: \emph{Hermann Bahr, der Fünfzigjährige. (Eine Rede im Akademischen
                        Verband für Literatur)}\pwindex{Zweig, Stefan 28.\,11.\,1881 Wien – 23.\,2.\,1942 Petrópolis@\textsc{Zweig, Stefan} (28.\,11.\,1881 Wien – 23.\,2.\,1942 Petrópolis), \emph{Schriftsteller}!Hermann Bahr, der Fünfzigjährige. (Eine Rede im Akademischen Verband für Literatur)@\strich\emph{Hermann Bahr, der Fünfzigjährige. (Eine Rede im Akademischen Verband für Literatur)}|pwk}. In: \emph{Neue Freie
                        Presse}\pwindex{Neue Freie Presse@\emph{Neue Freie Presse}|pwk}, Nr. 17.513, 13. 5. 1913, Morgenblatt,
                     S. 1–3.}}}\label{K_L03780-5} erscheinen wird. Wie sind Ihre So{\geminationm}erpläne? Wir wollen \label{K_L03780-6v}\edtext{Anfang Juni einige Wochen fort
               sein}{\lemma{\textnormal{\emph{Anfang … sein}}}\Cendnote{\textnormal{Vgl. XXXX Auszeichnungsfehler: Dokument L03638 nicht gefunden.}}}\label{K_L03780-6}, und da{\geminationn} bis gegen Ende Juli in Wien\oindex{Wien@\textbf{Wien}, \emph{Verwaltungsgebiet}|pw} verbringen.\pend
           
\pstart
           Ein baldiges Wiedersehen erhoffend und mit herzlichen Grüßen{\\[\baselineskip]}Ihr aufrichtig ergebner{\\[\baselineskip]}\spacefill\mbox{Arthur Schnitzler}\pend
           \leftskip=0em{}\selectlanguage{ngerman}\endnumbering\briefempfaengerindex{Zweig, Stefan@\textsc{Zweig, Stefan}!zzzSchnitzler, Arthur@\emph{von Arthur Schnitzler}!1913-05-241@{2[4?]. 5. 1913}|)be}\mylabel{L03780h}  \newcommand{\dateiname}{L03780}\newcommand{\titel}{Arthur Schnitzler an Stefan Zweig, 2[4?]. 5. 1913}\newcommand{\editorInnen}{Selma Jahnke und Martin Anton Müller}%% latex-leseansicht-abspann.tex
%% Abspann für die Leseansicht.
%% Der Schalter \ifkorrekturansicht ist bereits durch den Vorspann gesetzt.

%% latex-abspann.tex
%% Gemeinsamer Abspann für Korrekturansicht und Leseansicht.
%% Setzt den Schalter \ifkorrekturansicht voraus (gesetzt in den
%% einbindenden Dateien latex-korrekturansicht-abspann.tex bzw.
%% latex-leseansicht-abspann.tex).
%% ---------------------------------------------------------------

\normalsize

% Das esempio-Environment wird nur in der Leseansicht benötigt
\ifkorrekturansicht\else
\newenvironment{esempio}[3]%
{
    \vspace{1.5ex}
    \rlap{\underline{#1}}
    \par
    \setlength{\parindent}{0cm}
    \nopagebreak
    \leftskip=#2cm
    \rightskip=#3cm
}
{
    \par
}
\fi

\doendnotes{C}
\bigskip
\vfill

\clearpage

\footnotesize

\ifkorrekturansicht
  \lohead{\textsc{register}}
\fi

% theindex-Environment neu definieren ohne reledmac
\makeatletter
\renewenvironment{theindex}{%
  \ifkorrekturansicht
    \section*{\indexname}%
  \else
    \subsubsection*{Index der erwähnten Entitäten}%
  \fi
  \setlength{\parindent}{0pt}%
  \setlength{\parskip}{0pt plus 0.3pt}%
  \let\item\@idxitem
}{%
  \ifkorrekturansicht\clearpage\fi
}
\makeatother

\IfFileExists{\jobname-pw.ind}{\input{\jobname-pw.ind}}{}

% Quellenangabe nur in der Leseansicht
\ifkorrekturansicht\else
% Fallback-Definitionen, falls die .tex-Datei \titel etc. nicht gesetzt hat
\providecommand{\titel}{}
\providecommand{\editorInnen}{}
\providecommand{\dateiname}{\jobname}

\vspace{3cm}

\vfill

\footnotesize
\textsc{Quelle}: \titel. Herausgegeben von {\editorInnen}. In: \emph{Arthur Schnitzler: Briefwechsel mit Autorinnen und Autoren}.
 Digitale Edition, https://schnitzler-briefe.acdh.oeaw.ac.at/{\dateiname}.html (Stand \today)
\fi

\end{document}


