%% latex-korrekturansicht-vorspann.tex
%% Vorspann für die Korrekturansicht.
%% Lädt die gemeinsame Datei latex-vorspann.tex mit gesetztem Schalter.

\newif\ifkorrekturansicht
\korrekturansichttrue

\input{../tex-inputs/latex-vorspann}


\section[Arthur Schnitzler an Stefan Zweig, 29. 5. 1913]{L03780 Arthur Schnitzler an Stefan Zweig, 29. 5. 1913}
\nopagebreak\mylabel{L03780v}
\rehead{ }\normalsize\beginnumbering\briefempfaengerindex{Zweig, Stefan@\textsc{Zweig, Stefan}!zzzSchnitzler, Arthur@\emph{von Arthur Schnitzler}!1913-05-291@{29. 5. 1913}|(be}
\toendnotes[C]{\smallbreak\pagebreak[2]}\Standort{Jerusalem, National Library of Israel, ARC. Ms. Var. 305 1 58 Stefan Zweig Collection.}
\physDesc{Briefkarte, 1 Blatt, 2 Seiten, 1364 Zeichen
\newline{}Handschrift: schwarze Tinte, lateinische Kurrent}\toendnotes[C]{\smallbreak}
\pstart
           {\pb}\textcolor{gray}{\textbf{Dr. Arthur Schnitzler}}\hfill 29. 5. 913\pend
           
\pstart
           \textcolor{gray}{\textbf{Wien XVIII. Sternwartestrasse 71\oindex{Sternwartestrasse 71@\textbf{Sternwartestraße 71}, \emph{Wohngebäude (K.WHS)}|pw}}}\pend
           \vspace{0.5em}
\pstart
           lieber Herr Doctor, Ihr ſchöner \label{K_L03780-1v}\edtext{Brief}{\lemma{\textnormal{\emph{Brief}}}\Cendnote{\textnormal{Stefan Zweig an Arthur Schnitzler, 23. 5. 1913.
               }}}\label{K_L03780-1} hat mir wahrhaft wohlgethan.
               So sicher ich bei dem Dichter des »Kinderlands\pwindex{Erstes Erlebnis. Vier Geschichten aus Kinderland@\emph{Erstes Erlebnis. Vier Geschichten aus Kinderland}|pw}«
               auf vollkommenes Verständnis gefasst sein dürfte (ihr Bedenken hinsichtlich der
               Schlusses theil ich sogar – seit einiger Zeit erst); die warme menschliche
               Antheilnahme die Sie an meinem Schaffen haben und deren ich i{\geminationm}er gewiss war, hat
               sich selten so lebhaft ausgedrückt als in Ihren letzten Worten für die ich Ihnen
               freundschhaftlichst die Hand drücke. –\pend
           
\pstart
           Ich danke auch für die Einladg zur \label{K_L03780-2v}\edtext{Bahr\pwindex{Bahr, Hermann 19.07.1863 – 15.01.1934@\textsc{Bahr, Hermann} (19.07.1863 – 15.01.1934), \emph{Schriftsteller/Schriftstellerin, Kritiker/Kritikerin}|pw} Feier}{\lemma{\textnormal{\emph{Bahr Feier}}}\Cendnote{\textnormal{Siehe Stefan Zweig an Arthur Schnitzler, 23. 5. 1913.}}}\label{K_L03780-2} u. bitte zugleich um Entschuldigg, daß ich nicht kommen werde. Sie
               wissen ja, daß ich mich (aus Gründen, die nicht ausschließlich nervöser Natur sind)
               von solchen {\pb}Veranstaltungen wie es mir irgend angeht fern
               halte (das \label{K_L03780-3v}\edtext{Hauptmann\pwindex{Hauptmann, Gerhart 15.11.1862 – 06.06.1946@\textsc{Hauptmann, Gerhart} (15.11.1862 – 06.06.1946), \emph{Schriftsteller/Schriftstellerin}|pw} Bankett}{\lemma{\textnormal{\emph{Hauptmann Bankett}}}\Cendnote{\textnormal{
                  Das \emph{Bankett zu Ehren von Gerhart Hauptmann\pwindex{Hauptmann, Gerhart 15.11.1862 – 06.06.1946@\textsc{Hauptmann, Gerhart} (15.11.1862 – 06.06.1946), \emph{Schriftsteller/Schriftstellerin}|pwk}}\eventindex{Oesterreichischer Ingenieur- und Architektenverein@\textbf{Österreichischer Ingenieur- und Architektenverein}!Hauptmann-Bankett der Concordia, 17.11.1912@Hauptmann-Bankett der Concordia, 17.11.1912|pwk} wurde vom Journalisten- und Schriftstellerverein \emph{Concordia}\orgindex{Concordia. Journalisten- und Schriftstellerverein@Concordia. Journalisten- und Schriftstellerverein|pwk} veranstaltet und
                  fand am 17. 11. 1912 im Österreichischer Ingenieur- und Architektenverein\oindex{Oesterreichischer Ingenieur- und Architektenverein@\textbf{Österreichischer Ingenieur- und Architektenverein}, \emph{Vereinslokal (K.VRN)}|pwk} statt.
               }}}\label{K_L03780-3} war eine Ausnahme, weil ich, nach einem Misverständnis zwischen Hauptma{\geminationn}\pwindex{Hauptmann, Gerhart 15.11.1862 – 06.06.1946@\textsc{Hauptmann, Gerhart} (15.11.1862 – 06.06.1946), \emph{Schriftsteller/Schriftstellerin}|pw} u. mir die Gelegenheit benutzen mußte
               ihm zu begegnen) – auch Bahr\pwindex{Bahr, Hermann 19.07.1863 – 15.01.1934@\textsc{Bahr, Hermann} (19.07.1863 – 15.01.1934), \emph{Schriftsteller/Schriftstellerin, Kritiker/Kritikerin}|pw} (der übrigens
               glaub ich dasselbe thut) kennt diese meine Gepflogenheit und wird fern davon sein mir
               mein Ausbleiben übel zu nehmen. Sie aber, lieber Freund, bitt ich um das gleiche –
               und zugleich um Mittheilg wo Ihre \label{K_L03780-4v}\edtext{Rede\pwindex{Hermann Bahr, der Fuenfzigjaehrige. (Eine Rede im Akademischen Verband fuer Literatur)@\emph{Hermann Bahr, der Fünfzigjährige. (Eine Rede im Akademischen Verband für Literatur)}|pwv} ausführlich in Druck}{\lemma{\textnormal{\emph{Rede … Druck}}}\Cendnote{\textnormal{Stefan Zweig\pwindex{Zweig, Stefan 28.11.1881 – 23.02.1942@\textsc{Zweig, Stefan} (28.11.1881 – 23.02.1942), \emph{Schriftsteller/Schriftstellerin}|pwk}: \emph{Hermann Bahr, der Fünfzigjährige. (Eine Rede im Akademischen
                        Verband für Literatur)}\pwindex{Hermann Bahr, der Fuenfzigjaehrige. (Eine Rede im Akademischen Verband fuer Literatur)@\emph{Hermann Bahr, der Fünfzigjährige. (Eine Rede im Akademischen Verband für Literatur)}|pwk}. In: \emph{Neue Freie
                        Presse}\pwindex{Neue Freie Presse@\emph{Neue Freie Presse}|pwk}, Nr. 17.513, 13. 5. 1913,
                     Morgenblatt, S. 1–3.}}}\label{K_L03780-4} erscheinen wird. Wie sind Ihre
               So{\geminationm}erpläne? Wir wollen Anfang Juni einige Wochen fort sein, und da{\geminationn} bis gegen Ende
               Juli in Wien\oindex{Wien@\textbf{Wien}, \emph{A.ADM2}|pw} verbringen.\pend
           
\pstart
           Ein baldgs Wiedersehen erhoffend und mit herzlichen Grüßen{\\[\baselineskip]}Ihr aufrichtig ergebner{\\[\baselineskip]}\spacefill\mbox{Arthur Schnitzler}\pend
           \leftskip=0em{}\selectlanguage{ngerman}\endnumbering\briefempfaengerindex{Zweig, Stefan@\textsc{Zweig, Stefan}!zzzSchnitzler, Arthur@\emph{von Arthur Schnitzler}!1913-05-291@{29. 5. 1913}|)be}\mylabel{L03780h}
\begin{anhang}
\end{anhang}\normalsize

\doendnotes{C}
\bigskip
\vfill

\clearpage

\footnotesize

\lohead{\textsc{register}}

% Definiere theindex-Environment komplett neu ohne reledmac
\makeatletter
\renewenvironment{theindex}{%
  \section*{\indexname}%
  \setlength{\parindent}{0pt}%
  \setlength{\parskip}{0pt plus 0.3pt}%
  \let\item\@idxitem
}{%
  \clearpage
}
\makeatother

\IfFileExists{\jobname-pw.ind}{\input{\jobname-pw.ind}}{}

\end{document}

      