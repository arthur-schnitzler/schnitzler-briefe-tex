%% latex-leseansicht-vorspann.tex
%% Vorspann für die Leseansicht.
%% Lädt die gemeinsame Datei latex-vorspann.tex mit nicht gesetztem Schalter.

\newif\ifkorrekturansicht
\korrekturansichtfalse

\input{../tex-inputs/latex-vorspann}


\section[Felix Salten an Arthur Schnitzler, {{[}}13. 10. 1892{{]}}]{L03116 Felix Salten an Arthur Schnitzler, {[}13. 10. 1892{]}}
\nopagebreak\mylabel{L03116v}
\rehead{ }\normalsize\beginnumbering\briefempfaengerindex{Schnitzler, Arthur@\textsc{Schnitzler, Arthur}!zzzSalten, Felix@\emph{von Felix Salten}!1892-10-131@{{[}13. 10. 1892{]}}|(be}
\toendnotes[C]{\smallbreak\pagebreak[2]}
\correspDesc{Versand  durch Felix Salten am [13. 10. 1892] in Wien
\newline{}Erhalt  durch Arthur Schnitzler im Zeitraum [13. 10. 1892 – 15. 10. 1892?] in Wien}\toendnotes[C]{\smallbreak}
\Standort{CUL, Schnitzler, B 89, A 1.}
\physDesc{Brief, 1 Blatt, 1 Seite, 230 Zeichen
\newline{}Handschrift: schwarze Tinte, lateinische Kurrent
\newline{}Schnitzler: mit Bleistift datiert: »13. X. 92.« 
\newline{}Ordnung: mit Bleistift von unbekannter Hand nummeriert: »20a« }\toendnotes[C]{\smallbreak}
\pstart
           \noindent{}{\pb}lieber Doctor! Beiliegend \label{K_L03116-1v}\edtext{»Amerika!«}{\lemma{\textnormal{\emph{»Amerika!«}}}\Cendnote{\textnormal{Die Beilage ist nicht erhalten. Vermutlich handelt es sich nicht um Schnitzlers Erzählung \emph{Amerika}\pwindex{Schnitzler, Arthur 15.\,5.\,1862 Wien – 21.\,10.\,1931 ebd.@\textsc{Schnitzler, Arthur} (15.\,5.\,1862 Wien – 21.\,10.\,1931 ebd.), \emph{Schriftsteller, Mediziner}!Amerika@\strich\emph{Amerika}|pwk} (1889), sondern um eine Publikation anlässlich des 400. Jubiläum der
                     ›Entdeckung‹ Amerikas\oindex{Amerika@\textbf{Amerika}|pwk} durch Christoph Columbus\pwindex{Kolumbus, Christoph um 1451 Genua – 30.\,5.\,1506 Valladolid@\textsc{Kolumbus, Christoph} (um 1451 Genua – 30.\,5.\,1506 Valladolid), \emph{Seefahrer}|pwk}.}}}\label{K_L03116-1} Den Leuten, die nach mir fragen sagen Sie, bitte, dass ich zu stark verkühlt
               bin u. zu müde um Abends ausgehen zu können, wenigstens für die nächsten paar
               Tage.\pend
           
\pstart
           Auf baldiges Wiedersehen\pend
           \pstart Ihr \spacefill\mbox{F. Salten}\pend{}\selectlanguage{ngerman}\endnumbering\briefempfaengerindex{Schnitzler, Arthur@\textsc{Schnitzler, Arthur}!zzzSalten, Felix@\emph{von Felix Salten}!1892-10-131@{{[}13. 10. 1892{]}}|)be}\mylabel{L03116h}  \newcommand{\dateiname}{L03116}\newcommand{\titel}{Felix Salten an Arthur Schnitzler, [13. 10. 1892]}\newcommand{\editorInnen}{Martin Anton Müller und Laura Untner}%% latex-leseansicht-abspann.tex
%% Abspann für die Leseansicht.
%% Der Schalter \ifkorrekturansicht ist bereits durch den Vorspann gesetzt.

%% latex-abspann.tex
%% Gemeinsamer Abspann für Korrekturansicht und Leseansicht.
%% Setzt den Schalter \ifkorrekturansicht voraus (gesetzt in den
%% einbindenden Dateien latex-korrekturansicht-abspann.tex bzw.
%% latex-leseansicht-abspann.tex).
%% ---------------------------------------------------------------

\normalsize

% Das esempio-Environment wird nur in der Leseansicht benötigt
\ifkorrekturansicht\else
\newenvironment{esempio}[3]%
{
    \vspace{1.5ex}
    \rlap{\underline{#1}}
    \par
    \setlength{\parindent}{0cm}
    \nopagebreak
    \leftskip=#2cm
    \rightskip=#3cm
}
{
    \par
}
\fi

\doendnotes{C}
\bigskip
\vfill

\clearpage

\footnotesize

\ifkorrekturansicht
  \lohead{\textsc{register}}
\fi

% theindex-Environment neu definieren ohne reledmac
\makeatletter
\renewenvironment{theindex}{%
  \ifkorrekturansicht
    \section*{\indexname}%
  \else
    \subsubsection*{Index der erwähnten Entitäten}%
  \fi
  \setlength{\parindent}{0pt}%
  \setlength{\parskip}{0pt plus 0.3pt}%
  \let\item\@idxitem
}{%
  \ifkorrekturansicht\clearpage\fi
}
\makeatother

\IfFileExists{\jobname-pw.ind}{\input{\jobname-pw.ind}}{}

% Quellenangabe nur in der Leseansicht
\ifkorrekturansicht\else
% Fallback-Definitionen, falls die .tex-Datei \titel etc. nicht gesetzt hat
\providecommand{\titel}{}
\providecommand{\editorInnen}{}
\providecommand{\dateiname}{\jobname}

\vspace{3cm}

\vfill

\footnotesize
\textsc{Quelle}: \titel. Herausgegeben von {\editorInnen}. In: \emph{Arthur Schnitzler: Briefwechsel mit Autorinnen und Autoren}.
 Digitale Edition, https://schnitzler-briefe.acdh.oeaw.ac.at/{\dateiname}.html (Stand \today)
\fi

\end{document}


