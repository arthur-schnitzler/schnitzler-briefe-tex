%% latex-korrekturansicht-vorspann.tex
%% Vorspann für die Korrekturansicht.
%% Lädt die gemeinsame Datei latex-vorspann.tex mit gesetztem Schalter.

\newif\ifkorrekturansicht
\korrekturansichttrue

\input{../tex-inputs/latex-vorspann}


\section[Felix Salten an Arthur Schnitzler, {[}13. 10. 1892{]}]{L03116 Felix Salten an Arthur Schnitzler, {[}13. 10. 1892{]}}
\nopagebreak\mylabel{L03116v}
\rehead{ }\normalsize\beginnumbering\briefempfaengerindex{Schnitzler, Arthur@\textsc{Schnitzler, Arthur}!zzzSalten, Felix@\emph{von Felix Salten}!1892-10-131@{{[}13. 10. 1892{]}}|(be}
\toendnotes[C]{\smallbreak\pagebreak[2]}\Standort{CUL, Schnitzler, B 89, A 1.}
\physDesc{Brief, 1 Blatt, 1 Seite, 230 Zeichen
\newline{}Handschrift: schwarze Tinte, lateinische Kurrent
\newline{}Schnitzler: mit Bleistift datiert: »13. X. 92.« 
\newline{}Ordnung: mit Bleistift von unbekannter Hand nummeriert: »20a« }\toendnotes[C]{\smallbreak}
\pstart
           \noindent{}{\pb}lieber Doctor! Beiliegend \label{K_L03116-1v}\edtext{»Amerika!«}{\lemma{\textnormal{\emph{»Amerika!«}}}\Cendnote{\textnormal{Die Beilage ist nicht erhalten. Vermutlich handelt es sich nicht um Schnitzlers Erzählung \emph{Amerika}\pwindex{Amerika@\emph{Amerika}|pwk} (1889), sondern um eine Publikation anlässlich des 400. Jubiläum der
                     ›Entdeckung‹ Amerikas\oindex{Amerika@\textbf{Amerika}, \emph{kein passender Code gefunden}|pwk} durch Christoph Columbus\pwindex{Kolumbus, Christoph um 1451 – 30.05.1506@\textsc{Kolumbus, Christoph} (um 1451 – 30.05.1506), \emph{Seefahrer/Seefahrerin}|pwk}.}}}\label{K_L03116-1} Den Leuten, die nach mir fragen sagen Sie, bitte, dass ich zu stark verkühlt
               bin u. zu müde um Abends ausgehen zu können, wenigstens für die nächsten paar
               Tage.\pend
           
\pstart
           Auf baldiges Wiedersehen\pend
           \pstart Ihr \spacefill\mbox{F. Salten}\pend{}\selectlanguage{ngerman}\endnumbering\briefempfaengerindex{Schnitzler, Arthur@\textsc{Schnitzler, Arthur}!zzzSalten, Felix@\emph{von Felix Salten}!1892-10-131@{{[}13. 10. 1892{]}}|)be}\mylabel{L03116h}  \normalsize

\doendnotes{C}
\bigskip
\vfill

\clearpage

\footnotesize

\lohead{\textsc{register}}

% Definiere theindex-Environment komplett neu ohne reledmac
\makeatletter
\renewenvironment{theindex}{%
  \section*{\indexname}%
  \setlength{\parindent}{0pt}%
  \setlength{\parskip}{0pt plus 0.3pt}%
  \let\item\@idxitem
}{%
  \clearpage
}
\makeatother

\IfFileExists{\jobname-pw.ind}{\input{\jobname-pw.ind}}{}

\end{document}

      