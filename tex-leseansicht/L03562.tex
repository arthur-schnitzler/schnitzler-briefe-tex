%% latex-leseansicht-vorspann.tex
%% Vorspann für die Leseansicht.
%% Lädt die gemeinsame Datei latex-vorspann.tex mit nicht gesetztem Schalter.

\newif\ifkorrekturansicht
\korrekturansichtfalse

\input{../tex-inputs/latex-vorspann}


\section[ Felix Salten an Arthur und Olga Schnitzler, 7. 9. 1913]{L03562 Felix Salten an Arthur und Olga Schnitzler,  7. 9. 1913}
\nopagebreak\mylabel{L03562v}
\rehead{ }\normalsize\beginnumbering\briefempfaengerindex{Schnitzler, Olga@\textsc{Schnitzler, Olga}!zzzSalten, Felix@\emph{von Felix Salten}!1913-09-071@{7. 9. 1913}|(be}\briefempfaengerindex{Schnitzler, Arthur@\textsc{Schnitzler, Arthur}!zzzSalten, Felix@\emph{von Felix Salten}!1913-09-071@{7. 9. 1913}|(be}
\toendnotes[C]{\smallbreak\pagebreak[2]}
\correspDesc{Versand  durch Felix Salten am 7. 9. 1913 in Berlin
\newline{}Übermittlung  am 8. 9. 1913 in Berlin
\newline{}Erhalt  durch Arthur Schnitzler, Olga Schnitzler im Zeitraum [9. 9. 1913
                  – 13. 9. 1913?] in Wien}\toendnotes[C]{\smallbreak}
\Standort{CUL, Schnitzler, B 89, B 2.}
\physDesc{Bildpostkarte, 307 Zeichen
\newline{}Handschrift: schwarze Tinte, lateinische Kurrent
\newline{}Versand: Stempel: »\nobreak{}\oindex{Charlottenburg@\textbf{Charlottenburg}, \emph{Ehemaliger Ort}|pwk}Charlottenburg 2, 8. \textcolor{gray}{9}. 13, 9–10 V\nobreak{}«.  
\newline{}Ordnung: mit Bleistift von unbekannter Hand nummeriert: »275« }\toendnotes[C]{\smallbreak}\pstart{}{\pb}Herrn u. Frau D\textsuperscript{r} Arthur Schnitzler\pend{}\pstart{}Wien\oindex{Wien@\textbf{Wien}, \emph{Verwaltungsgebiet}|pw}\pend{}\pstart{}XVIII. Sternwartestraße 71\oindex{Wien@\textbf{Wien}!XVIII., Währing@\textbf{XVIII., Währing}!Sternwartestraße 71@\textbf{Sternwartestraße 71}, \emph{Wohngebäude}|pw}\pend{}{\bigskip}
\pstart
           {\pb}\textcolor{gray}{\textbf{ZOOLOGISCHER GARTEN BERLIN\oindex{Zoologischer Garten Berlin@\textbf{Zoologischer Garten Berlin}, \emph{Zoo}|pw}.}}\hfill \textcolor{gray}{\textbf{Löwe.}}\pend
           \vspace{1em}
\pstart
           \noindent{}{\pb}Viele herzliche Grüße!\pend
           
\pstart
           Ich bin nun schon eine Woche hier\oindex{Berlin@\textbf{Berlin}, \emph{Hauptstadt}|pw} und bleibe
               voraussichtlich bis gegen den Zwanzigsten. Hoffentlich
               sind Sie alle gesund und \label{K_L03562-1v}\edtext{Heini\pwindex{Schnitzler, Heinrich 9.\,8.\,1902 Hinterbrühl – 12.\,7.\,1982 Wien@\textsc{Schnitzler, Heinrich} (9.\,8.\,1902 Hinterbrühl – 12.\,7.\,1982 Wien), \emph{Regisseur, Schauspieler}|pw} ganz erholt}{\lemma{\textnormal{\emph{Heini ganz erholt}}}\Cendnote{\textnormal{Vgl. XXXX Auszeichnungsfehler: Dokument L03561 nicht gefunden.
               }}}\label{K_L03562-1}. Auf ein frohes Wiedersehen und alles herzliche Ihnen allen\pend
           \pstart Ihr \spacefill\mbox{Salten}\pend{}
\pstart
           Berlin\oindex{Berlin@\textbf{Berlin}, \emph{Hauptstadt}|pw}, 7. 9. 913\pend
           \selectlanguage{ngerman}\endnumbering\briefempfaengerindex{Schnitzler, Olga@\textsc{Schnitzler, Olga}!zzzSalten, Felix@\emph{von Felix Salten}!1913-09-071@{7. 9. 1913}|)be}\briefempfaengerindex{Schnitzler, Arthur@\textsc{Schnitzler, Arthur}!zzzSalten, Felix@\emph{von Felix Salten}!1913-09-071@{7. 9. 1913}|)be}\mylabel{L03562h}  \newcommand{\dateiname}{L03562}\newcommand{\titel}{Felix Salten an Arthur und Olga Schnitzler, 7. 9. 1913}\newcommand{\editorInnen}{Martin Anton Müller und Laura Untner}%% latex-leseansicht-abspann.tex
%% Abspann für die Leseansicht.
%% Der Schalter \ifkorrekturansicht ist bereits durch den Vorspann gesetzt.

%% latex-abspann.tex
%% Gemeinsamer Abspann für Korrekturansicht und Leseansicht.
%% Setzt den Schalter \ifkorrekturansicht voraus (gesetzt in den
%% einbindenden Dateien latex-korrekturansicht-abspann.tex bzw.
%% latex-leseansicht-abspann.tex).
%% ---------------------------------------------------------------

\normalsize

% Das esempio-Environment wird nur in der Leseansicht benötigt
\ifkorrekturansicht\else
\newenvironment{esempio}[3]%
{
    \vspace{1.5ex}
    \rlap{\underline{#1}}
    \par
    \setlength{\parindent}{0cm}
    \nopagebreak
    \leftskip=#2cm
    \rightskip=#3cm
}
{
    \par
}
\fi

\doendnotes{C}
\bigskip
\vfill

\clearpage

\footnotesize

\ifkorrekturansicht
  \lohead{\textsc{register}}
\fi

% theindex-Environment neu definieren ohne reledmac
\makeatletter
\renewenvironment{theindex}{%
  \ifkorrekturansicht
    \section*{\indexname}%
  \else
    \subsubsection*{Index der erwähnten Entitäten}%
  \fi
  \setlength{\parindent}{0pt}%
  \setlength{\parskip}{0pt plus 0.3pt}%
  \let\item\@idxitem
}{%
  \ifkorrekturansicht\clearpage\fi
}
\makeatother

\IfFileExists{\jobname-pw.ind}{\input{\jobname-pw.ind}}{}

% Quellenangabe nur in der Leseansicht
\ifkorrekturansicht\else
% Fallback-Definitionen, falls die .tex-Datei \titel etc. nicht gesetzt hat
\providecommand{\titel}{}
\providecommand{\editorInnen}{}
\providecommand{\dateiname}{\jobname}

\vspace{3cm}

\vfill

\footnotesize
\textsc{Quelle}: \titel. Herausgegeben von {\editorInnen}. In: \emph{Arthur Schnitzler: Briefwechsel mit Autorinnen und Autoren}.
 Digitale Edition, https://schnitzler-briefe.acdh.oeaw.ac.at/{\dateiname}.html (Stand \today)
\fi

\end{document}


