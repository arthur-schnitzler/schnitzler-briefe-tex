%% latex-korrekturansicht-vorspann.tex
%% Vorspann für die Korrekturansicht.
%% Lädt die gemeinsame Datei latex-vorspann.tex mit gesetztem Schalter.

\newif\ifkorrekturansicht
\korrekturansichttrue

\input{../tex-inputs/latex-vorspann}


\section[Georg Brandes an Arthur Schnitzler, 23. 12. 1914]{L02201 Georg Brandes an Arthur Schnitzler, 23. 12. 1914}
\nopagebreak\mylabel{L02201v}
\rehead{ }\normalsize\beginnumbering\briefempfaengerindex{Schnitzler, Arthur@\textsc{Schnitzler, Arthur}!zzzBrandes, Georg@\emph{von Georg Brandes}!1914-12-231@{23. 12. 1914}|(be}
\toendnotes[C]{\smallbreak\pagebreak[2]}\Standort{CUL, Schnitzler, B 17.}
\physDesc{Postkarte, 937 Zeichen
\newline{}Handschrift: schwarze Tinte, lateinische Kurrent
\newline{}Versand: Stempel: »\nobreak{}\oindex{Kopenhagen@\textbf{Kopenhagen}, \emph{P.PPLC}|pwk}Kjøbenhavn, 23. 12. 14, 2–3E\nobreak{}«.  
\newline{}Schnitzler: mit Bleistift beschriftet: »\textsc{Brandes}« 
\newline{}Ordnung: mit Bleistift von unbekannter Hand nummeriert:
                                    »44« }
\buchAbdrucke{\weitereDrucke{Georg Brandes, Arthur Schnitzler: \emph{Ein Briefwechsel}. Bern: \emph{Francke} 1956, S. 113–114.} }\toendnotes[C]{\smallbreak}\pstart{}{\pb}Herrn Dr. Arthur
                  Schnitzler\pend{}\pstart{}Sternwartestrasse 71\oindex{Sternwartestrasse 71@\textbf{Sternwartestraße 71}, \emph{Wohngebäude (K.WHS)}|pw}\pend{}\pstart{}Wien XVIII\oindex{XVIII., Waehring@\textbf{XVIII., Währing}, \emph{A.ADM3}|pw}\pend{}{\bigskip}\vspace{1em}
\pstart
           \raggedleft{}{\pb}Kopenhagen\oindex{Kopenhagen@\textbf{Kopenhagen}, \emph{P.PPLC}|pw}{ }23 Dec 14\pend
           
\pstart{}Verehrter und lieber Freund\pend\vspace{0.5em}
\pstart
           Es freute mich ein Lebenszeichen von Ihnen zu sehen. Es freut mich noch mehr, dass
               Sie und die Ihrigen in guter und ruhiger Stimmung sind. Meine einzige Tochter\pwindex{Philipp, Edith 17.01.1879 – 1968-02-16@\textsc{Philipp, Edith} (17.01.1879 – 1968-02-16)|pwv} ist in Berlin\oindex{Berlin@\textbf{Berlin}, \emph{P.PPLC}|pw} verheirathet. Ihr junger Mann\pwindex{Philipp, Reinhold 15.08.1883 – 1968@\textsc{Philipp, Reinhold} (15.08.1883 – 1968), \emph{Fabrikant/Fabrikantin}|pwv} ist Fabrikant und Gardelieutenant der
               Artillerie, er wurde schon im September zum Oberlieutenant befördert und
               bekam im November das eiserne Kreuz. Aber er ist in steter Lebensgefahr. Meine Tochter\pwindex{Philipp, Edith 17.01.1879 – 1968-02-16@\textsc{Philipp, Edith} (17.01.1879 – 1968-02-16)|pwv} war mehrere Monate
               hier mit zwei Kleinen\pwindex{Philipp, Gerda 27.11.1907 – 1968@\textsc{Philipp, Gerda} (27.11.1907 – 1968)|pwv}\pwindex{Philipp, Georg 1912-06-21 – 1995-11-08@\textsc{Philipp, Georg} (1912-06-21 – 1995-11-08), \emph{Schauspieler/Schauspielerin}|pwv}, einer Tochter\pwindex{Philipp, Gerda 27.11.1907 – 1968@\textsc{Philipp, Gerda} (27.11.1907 – 1968)|pwv} von 7 Jahren und einem Jungen\pwindex{Philipp, Georg 1912-06-21 – 1995-11-08@\textsc{Philipp, Georg} (1912-06-21 – 1995-11-08), \emph{Schauspieler/Schauspielerin}|pwv} von 2 Jahren, beide sehr hübsch; sie ist jetzt in
                  Berlin\oindex{Berlin@\textbf{Berlin}, \emph{P.PPLC}|pw} und natürlich recht unruhig und
               mitgenommen von der ewigen Spannung. Ich arbeite viel, schreibe im Augenblick ein Buch über Goethe\pwindex{Goethe, Johann Wolfgang von 1749-08-28 – 1832-03-22@\textsc{Goethe, Johann Wolfgang von} (1749-08-28 – 1832-03-22), \emph{Schriftsteller/Schriftstellerin}|pw}\pwindex{Wolfgang Goethe@\emph{Wolfgang Goethe}|pwv}, parallel zu dem, ich einmal über Shspeare\pwindex{Shakespeare, William 23.4.1564? – 03.05.1616@\textsc{Shakespeare, William} (23.4.1564? – 03.05.1616), \emph{Schauspieler/Schauspielerin, Dramatiker/Dramatikerin}|pw}\pwindex{William Shakespeare@\emph{William Shakespeare}|pwv} schrieb. Ausserdem habe ich fast jeden Monat ein grosses Essay
               veröffentlicht.\pend
           \pstart Grüssen Sie Ihre Frau Gemahlin\pwindex{Schnitzler, Olga 17.01.1882 – 13.01.1970@\textsc{Schnitzler, Olga} (17.01.1882 – 13.01.1970), \emph{Schauspieler/Schauspielerin, Sänger/Sängerin}|pwv} und Beer-Hoffmanns\pwindex{Beer-Hofmann, Richard 1866-07-11 – 1945-09-26@\textsc{Beer-Hofmann, Richard} (1866-07-11 – 1945-09-26), \emph{Schriftsteller/Schriftstellerin}|pw}\pwindex{Beer-Hofmann, Paula 25.02.1879 – 30.10.1939@\textsc{Beer-Hofmann, Paula} (25.02.1879 – 30.10.1939)|pw}. Ihr \spacefill\mbox{G. B.}\pend{}\selectlanguage{ngerman}\endnumbering\briefempfaengerindex{Schnitzler, Arthur@\textsc{Schnitzler, Arthur}!zzzBrandes, Georg@\emph{von Georg Brandes}!1914-12-231@{23. 12. 1914}|)be}\mylabel{L02201h}  \normalsize

\doendnotes{C}
\bigskip
\vfill

\clearpage

\footnotesize

\lohead{\textsc{register}}

% Definiere theindex-Environment komplett neu ohne reledmac
\makeatletter
\renewenvironment{theindex}{%
  \section*{\indexname}%
  \setlength{\parindent}{0pt}%
  \setlength{\parskip}{0pt plus 0.3pt}%
  \let\item\@idxitem
}{%
  \clearpage
}
\makeatother

\IfFileExists{\jobname-pw.ind}{\input{\jobname-pw.ind}}{}

\end{document}

      