%% latex-leseansicht-vorspann.tex
%% Vorspann für die Leseansicht.
%% Lädt die gemeinsame Datei latex-vorspann.tex mit nicht gesetztem Schalter.

\newif\ifkorrekturansicht
\korrekturansichtfalse

\input{../tex-inputs/latex-vorspann}


         
         \newcommand{\erwaehntePersonen}{Personen: Richard Beer-Hofmann, Paula Beer-Hofmann, Johann Wolfgang von Goethe, Edith Philipp, Reinhold Philipp, Gerda Philipp, Georg Philipp, Olga Schnitzler, William Shakespeare}
         \newcommand{\erwaehnteInstitutionen}{}
         \newcommand{\erwaehnteOrte}{Orte: Berlin, Kopenhagen, Sternwartestraße, Wien, XVIII., Währing}
         \newcommand{\erwaehnteWerke}{Werke: William Shakespeare, Wolfgang Goethe}
               \section[Georg Brandes an Arthur Schnitzler, 23. 12. 1914]{ Georg Brandes an Arthur Schnitzler, 23. 12. 1914}\nopagebreak\mylabel{v}\rehead{ }\begin{ledgroupsized}[t]{13cm}\normalsize\beginnumbering \toendnotes[C]{\smallbreak\pagebreak[2]} \Standort{CUL, Schnitzler, B 17.}
\physDesc{Postkarte
\newline{}Handschrift: schwarze Tinte, lateinische Kurrent\newline{}Versand: Stempel: »\nobreak{}\oindex{Kopenhagen@\textbf{Kopenhagen}|pwk}Kjøbenhavn, 23. 12. 14, 2–3E\nobreak{}«.  
\newline{}Schnitzler: mit Bleistift beschriftet: »\textsc{Brandes}« \newline{}Ordnung: mit Bleistift von unbekannter Hand nummeriert:
                                        »44« }\buchAbdrucke{\weitereDrucke{Georg Brandes, Arthur Schnitzler: \emph{Ein Briefwechsel}. Hg. Kurt Bergel. Bern: \emph{Francke} 1956, S. 113–114.} }\toendnotes[C]{\smallbreak}\pstart{}{\pb}Herrn Dr. Arthur
                        Schnitzler\pend{}\pstart{}Sternwartestrasse 71\oindex{Sternwartestrasse@\textbf{Sternwartestraße}|pw}\pend{}\pstart{}Wien XVIII\oindex{XVIII., Waehring@\textbf{XVIII., Währing}|pw}\pend{}{\bigskip}\pstart
           \raggedleft{}{\pb}Kopenhagen\oindex{Kopenhagen@\textbf{Kopenhagen}|pw}{ }23 Dec 14\pend
           \pstart{}Verehrter und lieber Freund\pend\pstart
           Es freute mich ein Lebenszeichen von Ihnen zu sehen. Es freut mich noch mehr,
                    dass Sie und die Ihrigen in guter und ruhiger Stimmung sind. Meine einzige Tochter\pwindex{Philipp, Edith 17.01.1879 – 1968-02-16@\textsc{Philipp, Edith} (17.01.1879 – 1968-02-16)|pwv} ist in Berlin\oindex{Berlin@\textbf{Berlin}|pw} verheirathet. Ihr junger Mann\pwindex{Philipp, Reinhold 15.08.1883 – 1968@\textsc{Philipp, Reinhold} (15.08.1883 – 1968), \emph{Fabrikant}|pwv} ist Fabrikant und
                    Gardelieutenant der Artillerie, er wurde schon im September zum
                    Oberlieutenant befördert und bekam im November das eiserne Kreuz. Aber er ist in
                    steter Lebensgefahr. Meine Tochter\pwindex{Philipp, Edith 17.01.1879 – 1968-02-16@\textsc{Philipp, Edith} (17.01.1879 – 1968-02-16)|pwv} war mehrere Monate hier mit zwei Kleinen\pwindex{Philipp, Gerda 27.11.1907 – 1968@\textsc{Philipp, Gerda} (27.11.1907 – 1968)|pwv}\pwindex{Philipp, Georg 1912-06-21 – 1995-11-08@\textsc{Philipp, Georg} (1912-06-21 – 1995-11-08), \emph{Schauspieler}|pwv}, einer Tochter\pwindex{Philipp, Gerda 27.11.1907 – 1968@\textsc{Philipp, Gerda} (27.11.1907 – 1968)|pwv} von 7 Jahren und einem Jungen\pwindex{Philipp, Georg 1912-06-21 – 1995-11-08@\textsc{Philipp, Georg} (1912-06-21 – 1995-11-08), \emph{Schauspieler}|pwv} von 2 Jahren, beide
                    sehr hübsch; sie ist jetzt in Berlin\oindex{Berlin@\textbf{Berlin}|pw} und
                    natürlich recht unruhig und mitgenommen von der ewigen Spannung. Ich arbeite
                    viel, schreibe im Augenblick ein Buch über Goethe\pwindex{Goethe, Johann Wolfgang von 1749-08-28 – 1832-03-22@\textsc{Goethe, Johann Wolfgang von} (1749-08-28 – 1832-03-22), \emph{Schriftsteller}|pw}\pwindex{Brandes, Georg 04.02.1842 – 19.02.1927@\textsc{Brandes, Georg} (04.02.1842 – 19.02.1927)!Wolfgang Goethe1915@\strich\emph{Wolfgang Goethe} {[}1915{]}|pwv}, parallel zu dem, ich einmal über Shspeare\pwindex{Shakespeare, William 23.4.1564? – 03.05.1616@\textsc{Shakespeare, William} (23.4.1564? – 03.05.1616), \emph{Schauspieler, Dramatiker}|pw}\pwindex{Brandes, Georg 04.02.1842 – 19.02.1927@\textsc{Brandes, Georg} (04.02.1842 – 19.02.1927)!William Shakespeare1895 – 1896@\strich\emph{William Shakespeare} {[}1895 – 1896{]}|pwv}
               schrieb. Ausserdem habe ich fast jeden Monat ein grosses Essay
                    veröffentlicht.\pend
           \pstart Grüssen Sie Ihre Frau Gemahlin\pwindex{Schnitzler, Olga 17.01.1882 – 13.01.1970@\textsc{Schnitzler, Olga} (17.01.1882 – 13.01.1970), \emph{Schauspielerin, Sängerin}|pwv} und Beer-Hoffmanns\pwindex{Beer-Hofmann, Richard 1866-07-11 – 1945-09-26@\textsc{Beer-Hofmann, Richard} (1866-07-11 – 1945-09-26), \emph{Schriftsteller}|pw}\pwindex{Beer-Hofmann, Paula 25.02.1879 – 30.10.1939@\textsc{Beer-Hofmann, Paula} (25.02.1879 – 30.10.1939)|pw}. Ihr \spacefill\mbox{G. B.}\pend{}
         
         \endnumbering\mylabel{h}\end{ledgroupsized}  \newcommand{\dateiname}{L02201}\newcommand{\titel}{Georg Brandes an Arthur Schnitzler, 23. 12. 1914}\newcommand{\editorInnen}{Martin Anton Müller und Gerd-Hermann Susen}%% latex-leseansicht-abspann.tex
%% Abspann für die Leseansicht.
%% Der Schalter \ifkorrekturansicht ist bereits durch den Vorspann gesetzt.

%% latex-abspann.tex
%% Gemeinsamer Abspann für Korrekturansicht und Leseansicht.
%% Setzt den Schalter \ifkorrekturansicht voraus (gesetzt in den
%% einbindenden Dateien latex-korrekturansicht-abspann.tex bzw.
%% latex-leseansicht-abspann.tex).
%% ---------------------------------------------------------------

\normalsize

% Das esempio-Environment wird nur in der Leseansicht benötigt
\ifkorrekturansicht\else
\newenvironment{esempio}[3]%
{
    \vspace{1.5ex}
    \rlap{\underline{#1}}
    \par
    \setlength{\parindent}{0cm}
    \nopagebreak
    \leftskip=#2cm
    \rightskip=#3cm
}
{
    \par
}
\fi

\doendnotes{C}
\bigskip
\vfill

\clearpage

\footnotesize

\ifkorrekturansicht
  \lohead{\textsc{register}}
\fi

% theindex-Environment neu definieren ohne reledmac
\makeatletter
\renewenvironment{theindex}{%
  \ifkorrekturansicht
    \section*{\indexname}%
  \else
    \subsubsection*{Index der erwähnten Entitäten}%
  \fi
  \setlength{\parindent}{0pt}%
  \setlength{\parskip}{0pt plus 0.3pt}%
  \let\item\@idxitem
}{%
  \ifkorrekturansicht\clearpage\fi
}
\makeatother

\IfFileExists{\jobname-pw.ind}{\input{\jobname-pw.ind}}{}

% Quellenangabe nur in der Leseansicht
\ifkorrekturansicht\else
% Fallback-Definitionen, falls die .tex-Datei \titel etc. nicht gesetzt hat
\providecommand{\titel}{}
\providecommand{\editorInnen}{}
\providecommand{\dateiname}{\jobname}

\vspace{3cm}

\vfill

\footnotesize
\textsc{Quelle}: \titel. Herausgegeben von {\editorInnen}. In: \emph{Arthur Schnitzler: Briefwechsel mit Autorinnen und Autoren}.
 Digitale Edition, https://schnitzler-briefe.acdh.oeaw.ac.at/{\dateiname}.html (Stand \today)
\fi

\end{document}


      