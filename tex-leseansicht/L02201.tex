%% latex-leseansicht-vorspann.tex
%% Vorspann für die Leseansicht.
%% Lädt die gemeinsame Datei latex-vorspann.tex mit nicht gesetztem Schalter.

\newif\ifkorrekturansicht
\korrekturansichtfalse

\input{../tex-inputs/latex-vorspann}


\section[Georg Brandes an Arthur Schnitzler, 23. 12. 1914]{L02201 Georg Brandes an Arthur Schnitzler, 23. 12. 1914}
\nopagebreak\mylabel{L02201v}
\rehead{ }\normalsize\beginnumbering\briefempfaengerindex{Schnitzler, Arthur@\textsc{Schnitzler, Arthur}!zzzBrandes, Georg@\emph{von Georg Brandes}!1914-12-231@{23. 12. 1914}|(be}
\toendnotes[C]{\smallbreak\pagebreak[2]}
\correspDesc{Versand  durch Georg Brandes am 23. 12. 1914 in Kopenhagen
\newline{}Erhalt  durch Arthur Schnitzler im Zeitraum [24. 12. 1914 – 28. 12. 1914?] in Wien}\toendnotes[C]{\smallbreak}
\Standort{CUL, Schnitzler, B 17.}
\physDesc{Postkarte, 937 Zeichen
\newline{}Handschrift: schwarze Tinte, lateinische Kurrent
\newline{}Versand: Stempel: »\nobreak{}\oindex{Kopenhagen@\textbf{Kopenhagen}, \emph{Hauptstadt}|pwk}Kjøbenhavn, 23.\,12.\,14, 2–3E\nobreak{}«.  
\newline{}Schnitzler: mit Bleistift beschriftet: »\textsc{Brandes}« 
\newline{}Ordnung: mit Bleistift von unbekannter Hand nummeriert:
                                    »44« }
\buchAbdrucke{\weitereDrucke{Georg Brandes, Arthur Schnitzler: \emph{Ein Briefwechsel}. Herausgegeben von Kurt Bergel. Bern: \emph{Francke} 1956, S. 113–114.} }\toendnotes[C]{\smallbreak}\pstart{}{\pb}Herrn Dr. Arthur
                  Schnitzler\pend{}\pstart{}Sternwartestrasse 71\oindex{Wien@\textbf{Wien}!XVIII., Währing@\textbf{XVIII., Währing}!Sternwartestraße 71@\textbf{Sternwartestraße 71}, \emph{Wohngebäude}|pw}\pend{}\pstart{}Wien XVIII\oindex{XVIII., Währing@\textbf{XVIII., Währing}, \emph{Verwaltungsgebiet}|pw}\pend{}{\bigskip}\vspace{1em}
\pstart
           \raggedleft{}{\pb}Kopenhagen\oindex{Kopenhagen@\textbf{Kopenhagen}, \emph{Hauptstadt}|pw}{ }23 Dec 14\pend
           
\pstart{}Verehrter und lieber Freund\pend\vspace{0.5em}
\pstart
           Es freute mich ein Lebenszeichen von Ihnen zu sehen. Es freut mich noch mehr, dass
               Sie und die Ihrigen in guter und ruhiger Stimmung sind. Meine einzige Tochter\pwindex{Philipp, Edith 17.\,1.\,1879 Berlin – 16.\,2.\,1968 Kopenhagen@\textsc{Philipp, Edith} (17.\,1.\,1879 Berlin – 16.\,2.\,1968 Kopenhagen)|pwv} ist in Berlin\oindex{Berlin@\textbf{Berlin}, \emph{Hauptstadt}|pw} verheirathet. Ihr junger Mann\pwindex{Philipp, Reinhold 15.\,8.\,1883 – 1968@\textsc{Philipp, Reinhold} (15.\,8.\,1883 – 1968), \emph{Fabrikant}|pwv} ist Fabrikant und Gardelieutenant der
               Artillerie, er wurde schon im September zum Oberlieutenant befördert und
               bekam im November das eiserne Kreuz. Aber er ist in steter Lebensgefahr. Meine Tochter\pwindex{Philipp, Edith 17.\,1.\,1879 Berlin – 16.\,2.\,1968 Kopenhagen@\textsc{Philipp, Edith} (17.\,1.\,1879 Berlin – 16.\,2.\,1968 Kopenhagen)|pwv} war mehrere Monate
               hier mit zwei Kleinen\pwindex{Philipp, Gerda 27.\,11.\,1907 Berlin – 1968@\textsc{Philipp, Gerda} (27.\,11.\,1907 Berlin – 1968)|pwv}\pwindex{Philipp, Georg 21.\,6.\,1912 Dänemark – 8.\,11.\,1995 ebd.@\textsc{Philipp, Georg} (21.\,6.\,1912 Dänemark – 8.\,11.\,1995 ebd.), \emph{Schauspieler}|pwv}, einer Tochter\pwindex{Philipp, Gerda 27.\,11.\,1907 Berlin – 1968@\textsc{Philipp, Gerda} (27.\,11.\,1907 Berlin – 1968)|pwv} von 7 Jahren und einem Jungen\pwindex{Philipp, Georg 21.\,6.\,1912 Dänemark – 8.\,11.\,1995 ebd.@\textsc{Philipp, Georg} (21.\,6.\,1912 Dänemark – 8.\,11.\,1995 ebd.), \emph{Schauspieler}|pwv} von 2 Jahren, beide sehr hübsch; sie ist jetzt in
                  Berlin\oindex{Berlin@\textbf{Berlin}, \emph{Hauptstadt}|pw} und natürlich recht unruhig und
               mitgenommen von der ewigen Spannung. Ich arbeite viel, schreibe im Augenblick ein Buch über Goethe\pwindex{Goethe, Johann Wolfgang von 28.\,8.\,1749 Frankfurt am Main – 22.\,3.\,1832 Weimar@\textsc{Goethe, Johann Wolfgang von} (28.\,8.\,1749 Frankfurt am Main – 22.\,3.\,1832 Weimar), \emph{Schriftsteller}|pw}\pwindex{Brandes, Georg 4.\,2.\,1842 Kopenhagen – 19.\,2.\,1927 ebd.@\textsc{Brandes, Georg} (4.\,2.\,1842 Kopenhagen – 19.\,2.\,1927 ebd.)!Wolfgang Goethe@\strich\emph{Wolfgang Goethe}|pwv}, parallel zu dem, ich einmal über Shspeare\pwindex{Shakespeare, William 23.\,4.\,1564? Stratford-upon-Avon – 3.\,5.\,1616 ebd.@\textsc{Shakespeare, William} (23.\,4.\,1564? Stratford-upon-Avon – 3.\,5.\,1616 ebd.), \emph{Schauspieler, Dramatiker}|pw}\pwindex{Brandes, Georg 4.\,2.\,1842 Kopenhagen – 19.\,2.\,1927 ebd.@\textsc{Brandes, Georg} (4.\,2.\,1842 Kopenhagen – 19.\,2.\,1927 ebd.)!William Shakespeare@\strich\emph{William Shakespeare}|pwv} schrieb. Ausserdem habe ich fast jeden Monat ein grosses Essay
               veröffentlicht.\pend
           \pstart Grüssen Sie Ihre Frau Gemahlin\pwindex{Schnitzler, Olga 17.\,1.\,1882 Wien – 13.\,1.\,1970 Lugano@\textsc{Schnitzler, Olga} (17.\,1.\,1882 Wien – 13.\,1.\,1970 Lugano), \emph{Schauspielerin, Sängerin}|pwv} und Beer-Hoffmanns\pwindex{Beer-Hofmann, Richard 11.\,7.\,1866 Wien – 26.\,9.\,1945 New York City@\textsc{Beer-Hofmann, Richard} (11.\,7.\,1866 Wien – 26.\,9.\,1945 New York City), \emph{Schriftsteller}|pw}\pwindex{Beer-Hofmann, Paula 25.\,2.\,1879 Wien – 30.\,10.\,1939 Zürich@\textsc{Beer-Hofmann, Paula} (25.\,2.\,1879 Wien – 30.\,10.\,1939 Zürich)|pw}. Ihr \spacefill\mbox{G. B.}\pend{}\selectlanguage{ngerman}\endnumbering\briefempfaengerindex{Schnitzler, Arthur@\textsc{Schnitzler, Arthur}!zzzBrandes, Georg@\emph{von Georg Brandes}!1914-12-231@{23. 12. 1914}|)be}\mylabel{L02201h}  \newcommand{\dateiname}{L02201}\newcommand{\titel}{Georg Brandes an Arthur Schnitzler, 23. 12. 1914}\newcommand{\editorInnen}{Martin Anton Müller und Gerd-Hermann Susen}%% latex-leseansicht-abspann.tex
%% Abspann für die Leseansicht.
%% Der Schalter \ifkorrekturansicht ist bereits durch den Vorspann gesetzt.

%% latex-abspann.tex
%% Gemeinsamer Abspann für Korrekturansicht und Leseansicht.
%% Setzt den Schalter \ifkorrekturansicht voraus (gesetzt in den
%% einbindenden Dateien latex-korrekturansicht-abspann.tex bzw.
%% latex-leseansicht-abspann.tex).
%% ---------------------------------------------------------------

\normalsize

% Das esempio-Environment wird nur in der Leseansicht benötigt
\ifkorrekturansicht\else
\newenvironment{esempio}[3]%
{
    \vspace{1.5ex}
    \rlap{\underline{#1}}
    \par
    \setlength{\parindent}{0cm}
    \nopagebreak
    \leftskip=#2cm
    \rightskip=#3cm
}
{
    \par
}
\fi

\doendnotes{C}
\bigskip
\vfill

\clearpage

\footnotesize

\ifkorrekturansicht
  \lohead{\textsc{register}}
\fi

% theindex-Environment neu definieren ohne reledmac
\makeatletter
\renewenvironment{theindex}{%
  \ifkorrekturansicht
    \section*{\indexname}%
  \else
    \subsubsection*{Index der erwähnten Entitäten}%
  \fi
  \setlength{\parindent}{0pt}%
  \setlength{\parskip}{0pt plus 0.3pt}%
  \let\item\@idxitem
}{%
  \ifkorrekturansicht\clearpage\fi
}
\makeatother

\IfFileExists{\jobname-pw.ind}{\input{\jobname-pw.ind}}{}

% Quellenangabe nur in der Leseansicht
\ifkorrekturansicht\else
% Fallback-Definitionen, falls die .tex-Datei \titel etc. nicht gesetzt hat
\providecommand{\titel}{}
\providecommand{\editorInnen}{}
\providecommand{\dateiname}{\jobname}

\vspace{3cm}

\vfill

\footnotesize
\textsc{Quelle}: \titel. Herausgegeben von {\editorInnen}. In: \emph{Arthur Schnitzler: Briefwechsel mit Autorinnen und Autoren}.
 Digitale Edition, https://schnitzler-briefe.acdh.oeaw.ac.at/{\dateiname}.html (Stand \today)
\fi

\end{document}


