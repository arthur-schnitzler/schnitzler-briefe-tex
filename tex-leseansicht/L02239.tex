%% latex-korrekturansicht-vorspann.tex
%% Vorspann für die Korrekturansicht.
%% Lädt die gemeinsame Datei latex-vorspann.tex mit gesetztem Schalter.

\newif\ifkorrekturansicht
\korrekturansichttrue

\input{../tex-inputs/latex-vorspann}


\section[Arthur Schnitzler an Richard Beer-Hofmann, 23. 8. 1916]{L02239 Arthur Schnitzler an Richard Beer-Hofmann, 23. 8. 1916}
\nopagebreak\mylabel{L02239v}
\rehead{ }\normalsize\beginnumbering\briefempfaengerindex{Beer-Hofmann, Richard@\textsc{Beer-Hofmann, Richard}!zzzSchnitzler, Arthur@\emph{von Arthur Schnitzler}!1916-08-231@{23. 8. 1916}|(be}
\toendnotes[C]{\smallbreak\pagebreak[2]}\Standort{YCGL, MSS 31.}
\physDesc{Kartenbrief, 1010 Zeichen
\newline{}Handschrift: Bleistift, deutsche Kurrent
\newline{}Versand: Stempel: »\nobreak{}\oindex{Altaussee@\textbf{Altaussee}, \emph{A.ADM3}|pwk}Alt Aussee, 23. VIII. 1\textcolor{gray}{6}\nobreak{}«.  
\newline{}Beer-Hofmann: mit blauem Buntstift den Empfang vermerkt:
                                 »E« }
\buchAbdrucke{\weitereDrucke{Arthur Schnitzler, Richard Beer-Hofmann: \emph{Briefwechsel 1891–1931}. Wien, Zürich: \emph{Europaverlag} 1992, S. 222.} }\toendnotes[C]{\smallbreak}\pstart{}{\pb}\textsc{Schnitzler}.\pend{}\pstart{}\textsc{Altaussee\oindex{Altaussee@\textbf{Altaussee}, \emph{A.ADM3}|pw}}\pend{}\pstart{}\textsc{Fischerndorf 79\oindex{Fischerndorf@\textbf{Fischerndorf}, \emph{P.PPL}|pw}}\pend{}{\bigskip}\pstart{}\textsc{Herrn Dr. Richard Beer-Hofmann}\pend{}\pstart{}\textsc{Bad Ischl}\oindex{Bad Ischl@\textbf{Bad Ischl}, \emph{P.PPL}|pw}\pend{}\pstart{}\textsc{Grazerstr 52}\oindex{Grazer Strasse [Bad Ischl]@\textbf{Grazer Straße [Bad Ischl]}, \emph{Straße (K.STR)}|pw}.\pend{}{\bigskip}\vspace{1em}
\pstart
           \raggedleft{}{\pb}Altaussee\oindex{Altaussee@\textbf{Altaussee}, \emph{A.ADM3}|pw},{\\}23. 8. 1916\pend
           \vspace{0.5em}
\pstart
           lieber Richard, vielen Dank für Ihre Bemühungen und das Telegra{\geminationm} – nun ko{\geminationm}en wir doch
               nicht nach Iſchl\oindex{Bad Ischl@\textbf{Bad Ischl}, \emph{P.PPL}|pw} (\uline{dem
                     Kreuz\oindex{Hotel zum goldenen Kreuz [Bad Ischl]@\textbf{Hotel zum goldenen Kreuz [Bad Ischl]}, \emph{Hotel (K.HTL)}|pw} hab ich natürlich ſchon
                  abtelegrafirt}) – nicht ſo ſehr wegen des Wetters, als weil sich \textsc{Steiners}\pwindex{Steiner, Franz 15.09.1873 – 04.11.1954@\textsc{Steiner, Franz} (15.09.1873 – 04.11.1954), \emph{Sänger/Sängerin}|pw}\pwindex{Steiner, Margit *~08.08.1877@\textsc{Steiner, Margit} (*~08.08.1877)|pw} gerade für \label{K_L02239-1v}\edtext{Freitag}{\lemma{\textnormal{\emph{Freitag}}}\Cendnote{\textnormal{Siehe A. S.: \emph{Tagebuch}, 25. 8. 1916.
               }}}\label{K_L02239-1} bei uns angeſagt haben.\pend
           
\pstart
           – Von meiner Schwägerin\pwindex{Steinrueck, Elisabeth 19.11.1885 – 07.04.1920@\textsc{Steinrück, Elisabeth} (19.11.1885 – 07.04.1920)|pwv} ko{\geminationm}en etwas bedenkliche Nachrichten; es iſt ſehr möglich,
               daſs Olga\pwindex{Schnitzler, Olga 17.01.1882 – 13.01.1970@\textsc{Schnitzler, Olga} (17.01.1882 – 13.01.1970), \emph{Schauspieler/Schauspielerin, Sänger/Sängerin}|pw} (wenn ſie das Paſsviſum bekommt) auf
               8–12 Tage nach Partenkirchen\oindex{Partenkirchen@\textbf{Partenkirchen}, \emph{Teil eines besiedelten Ortes (A.BSOX)}|pw} fährt – auch ich
               bemühe mich um ein Viſum, – warte aber jedenfalls, wenn Olga\pwindex{Schnitzler, Olga 17.01.1882 – 13.01.1970@\textsc{Schnitzler, Olga} (17.01.1882 – 13.01.1970), \emph{Schauspieler/Schauspielerin, Sänger/Sängerin}|pw}{ }\strikeout{\textcolor{gray}{×}\-\textcolor{gray}{×}\-\textcolor{gray}{×}\-\textcolor{gray}{×}} reiſt, ein Telegra{\geminationm} von ihr aus Partenk.\oindex{Partenkirchen@\textbf{Partenkirchen}, \emph{Teil eines besiedelten Ortes (A.BSOX)}|pw} ab, ehe auch ich hinführe. So wäre es
               alſo denkbar, daſs wir gegen Ende des Monats in Salzburg\oindex{Salzburg@\textbf{Salzburg}, \emph{A.ADM2}|pw} wären, wohin ich O.\pwindex{Schnitzler, Olga 17.01.1882 – 13.01.1970@\textsc{Schnitzler, Olga} (17.01.1882 – 13.01.1970), \emph{Schauspieler/Schauspielerin, Sänger/Sängerin}|pw}
               jedenfalls begleiten würde; vielleicht haben Sie auch noch einen Salzb.\oindex{Salzburg@\textbf{Salzburg}, \emph{A.ADM2}|pw} Abſtecher vor, und man könnte dort zuſa{\geminationm}en ſein? Nach Iſchl\oindex{Bad Ischl@\textbf{Bad Ischl}, \emph{P.PPL}|pw}
               alſo ko{\geminationm}en wir in den nächſten Tagen kaum. Von allem
               weitern verſtändige ich Sie. Hören Sie was von \textsc{Arthur Kaufma{\geminationn}}\pwindex{Kaufmann, Arthur 04.04.1872 – 25.07.1938@\textsc{Kaufmann, Arthur} (04.04.1872 – 25.07.1938), \emph{Rechtswissenschaftler/Rechtswissenschaftlerin, Privatgelehrte/Privatgelehrte, Privatier/Privatière}|pw}? Ko{\geminationm}t er nach Iſchl\oindex{Bad Ischl@\textbf{Bad Ischl}, \emph{P.PPL}|pw}?\pend
           
\pstart
           Herzlichſt Ihr{\\[\baselineskip]}\spacefill\mbox{Arthur}\pend
           \leftskip=0em{}\selectlanguage{ngerman}\endnumbering\briefempfaengerindex{Beer-Hofmann, Richard@\textsc{Beer-Hofmann, Richard}!zzzSchnitzler, Arthur@\emph{von Arthur Schnitzler}!1916-08-231@{23. 8. 1916}|)be}\mylabel{L02239h}  \normalsize

\doendnotes{C}
\bigskip
\vfill

\clearpage

\footnotesize

\lohead{\textsc{register}}

% Definiere theindex-Environment komplett neu ohne reledmac
\makeatletter
\renewenvironment{theindex}{%
  \section*{\indexname}%
  \setlength{\parindent}{0pt}%
  \setlength{\parskip}{0pt plus 0.3pt}%
  \let\item\@idxitem
}{%
  \clearpage
}
\makeatother

\IfFileExists{\jobname-pw.ind}{\input{\jobname-pw.ind}}{}

\end{document}

      