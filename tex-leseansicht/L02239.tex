%% latex-leseansicht-vorspann.tex
%% Vorspann für die Leseansicht.
%% Lädt die gemeinsame Datei latex-vorspann.tex mit nicht gesetztem Schalter.

\newif\ifkorrekturansicht
\korrekturansichtfalse

\input{../tex-inputs/latex-vorspann}


         
         \newcommand{\erwaehntePersonen}{Personen: Richard Beer-Hofmann, Arthur Kaufmann, Olga Schnitzler, Franz Steiner, Margit Steiner, Elisabeth Steinrück}
         \newcommand{\erwaehnteInstitutionen}{}
         \newcommand{\erwaehnteOrte}{Orte: Altaussee, Bad Ischl, Fischerndorf, Goldenes Kreuz, Grazer Straße, Partenkirchen, Salzburg}
         \newcommand{\erwaehnteWerke}{
               \section[Arthur Schnitzler an Richard Beer-Hofmann, 23. 8. 1916]{ Arthur Schnitzler an Richard Beer-Hofmann, 23. 8. 1916}\nopagebreak\mylabel{v}\rehead{ }\begin{ledgroupsized}[t]{13cm}\normalsize\beginnumbering \toendnotes[C]{\smallbreak\pagebreak[2]} \Standort{YCGL, MSS 31.}
\physDesc{Kartenbrief
\newline{}Handschrift: Bleistift, deutsche Kurrent\newline{}Versand: Stempel: »\nobreak{}\oindex{Altaussee@\textbf{Altaussee}|pwk}Alt Aussee, 23. VIII. 1\textcolor{gray}{6}\nobreak{}«.  
\newline{}Beer-Hofmann: mit blauem Buntstift den Empfang vermerkt:
                                 »E« }\buchAbdrucke{\weitereDrucke{Arthur Schnitzler, Richard Beer-Hofmann: \emph{Briefwechsel 1891–1931}. Hg. Konstanze Fliedl. Wien, Zürich: \emph{Europaverlag} 1992, S. 222.} }\toendnotes[C]{\smallbreak}\pstart{}{\pb}\textsc{Schnitzler}.\pend{}\pstart{}\textsc{Altaussee\oindex{Altaussee@\textbf{Altaussee}|pw}}\pend{}\pstart{}\textsc{Fischerndorf 79\oindex{Fischerndorf@\textbf{Fischerndorf}|pw}}\pend{}{\bigskip}\pstart{}\textsc{Herrn Dr. Richard Beer-Hofmann}\pend{}\pstart{}\textsc{Bad Ischl}\oindex{Bad Ischl@\textbf{Bad Ischl}|pw}\pend{}\pstart{}\textsc{Grazerstr 52}\oindex{Grazer Strasse@\textbf{Grazer Straße}|pw}.\pend{}{\bigskip}\pstart
           \raggedleft{}{\pb}Altaussee\oindex{Altaussee@\textbf{Altaussee}|pw},{\\}23. 8. 1916\pend
           \pstart
           lieber Richard, vielen Dank für Ihre Bemühungen und das Telegra{\geminationm} – nun ko{\geminationm}en wir doch
               nicht nach Iſchl\oindex{Bad Ischl@\textbf{Bad Ischl}|pw} (\uline{dem Kreuz\oindex{Goldenes Kreuz@\textbf{Goldenes Kreuz}|pw} hab ich natürlich ſchon abtelegrafirt}) –
               nicht ſo ſehr wegen des Wetters, als weil sich \textsc{Steiners}\pwindex{Steiner, Franz 15.09.1873 – 04.11.1954@\textsc{Steiner, Franz} (15.09.1873 – 04.11.1954), \emph{Sänger}|pw}\pwindex{Steiner, Margit *~08.08.1877@\textsc{Steiner, Margit} (*~08.08.1877)|pw} gerade für \label{K_L02239-1v}\edtext{Freitag}{\lemma{\textnormal{\emph{Freitag}}}\Cendnote{\textnormal{siehe A. S.: \emph{Tagebuch}, 25. 8. 1916}}}\label{K_L02239-1h} bei uns
               angeſagt haben.\pend
           \pstart
           – Von meiner Schwägerin\pwindex{Steinrueck, Elisabeth 19.11.1885 – 07.04.1920@\textsc{Steinrück, Elisabeth} (19.11.1885 – 07.04.1920)|pwv} ko{\geminationm}en etwas bedenkliche Nachrichten; es iſt ſehr möglich,
               daſs Olga\pwindex{Schnitzler, Olga 17.01.1882 – 13.01.1970@\textsc{Schnitzler, Olga} (17.01.1882 – 13.01.1970), \emph{Schauspielerin, Sängerin}|pw} (wenn ſie das Paſsviſum bekommt) auf
               8–12 Tage nach Partenkirchen\oindex{Partenkirchen@\textbf{Partenkirchen}|pw} fährt – auch ich
               bemühe mich um ein Viſum, – warte aber jedenfalls, wenn Olga\pwindex{Schnitzler, Olga 17.01.1882 – 13.01.1970@\textsc{Schnitzler, Olga} (17.01.1882 – 13.01.1970), \emph{Schauspielerin, Sängerin}|pw}{ }\strikeout{\textcolor{gray}{×}\-\textcolor{gray}{×}\-\textcolor{gray}{×}\-\textcolor{gray}{×}} reiſt, ein Telegra{\geminationm} von ihr aus Partenk.\oindex{Partenkirchen@\textbf{Partenkirchen}|pw} ab, ehe auch ich hinführe. So wäre es alſo denkbar,
               daſs wir gegen Ende des Monats in Salzburg\oindex{Salzburg@\textbf{Salzburg}|pw} wären,
               wohin ich O.\pwindex{Schnitzler, Olga 17.01.1882 – 13.01.1970@\textsc{Schnitzler, Olga} (17.01.1882 – 13.01.1970), \emph{Schauspielerin, Sängerin}|pw} jedenfalls begleiten würde;
               vielleicht haben Sie auch noch einen Salzb.\oindex{Salzburg@\textbf{Salzburg}|pw} Abſtecher
               vor, und man könnte dort zuſa{\geminationm}en ſein? Nach Iſchl\oindex{Bad Ischl@\textbf{Bad Ischl}|pw} alſo ko{\geminationm}en wir in
               den nächſten Tagen kaum. Von allem weitern verſtändige ich Sie. Hören Sie was von \textsc{Arthur Kaufma{\geminationn}}\pwindex{Kaufmann, Arthur 04.04.1872 – 25.07.1938@\textsc{Kaufmann, Arthur} (04.04.1872 – 25.07.1938), \emph{Rechtswissenschaftler, Privatgelehrte, Privatier}|pw}? Ko{\geminationm}t er nach Iſchl\oindex{Bad Ischl@\textbf{Bad Ischl}|pw}?\pend
           \pstart
           Herzlichſt Ihr{\\[\baselineskip]}\spacefill\mbox{Arthur}\pend
           \leftskip=0em{}
         
         \endnumbering\mylabel{h}\end{ledgroupsized}  \newcommand{\dateiname}{L02239}\newcommand{\titel}{Arthur Schnitzler an Richard Beer-Hofmann, 23. 8. 1916}\newcommand{\editorInnen}{Martin Anton Müller und Gerd-Hermann Susen}%% latex-leseansicht-abspann.tex
%% Abspann für die Leseansicht.
%% Der Schalter \ifkorrekturansicht ist bereits durch den Vorspann gesetzt.

%% latex-abspann.tex
%% Gemeinsamer Abspann für Korrekturansicht und Leseansicht.
%% Setzt den Schalter \ifkorrekturansicht voraus (gesetzt in den
%% einbindenden Dateien latex-korrekturansicht-abspann.tex bzw.
%% latex-leseansicht-abspann.tex).
%% ---------------------------------------------------------------

\normalsize

% Das esempio-Environment wird nur in der Leseansicht benötigt
\ifkorrekturansicht\else
\newenvironment{esempio}[3]%
{
    \vspace{1.5ex}
    \rlap{\underline{#1}}
    \par
    \setlength{\parindent}{0cm}
    \nopagebreak
    \leftskip=#2cm
    \rightskip=#3cm
}
{
    \par
}
\fi

\doendnotes{C}
\bigskip
\vfill

\clearpage

\footnotesize

\ifkorrekturansicht
  \lohead{\textsc{register}}
\fi

% theindex-Environment neu definieren ohne reledmac
\makeatletter
\renewenvironment{theindex}{%
  \ifkorrekturansicht
    \section*{\indexname}%
  \else
    \subsubsection*{Index der erwähnten Entitäten}%
  \fi
  \setlength{\parindent}{0pt}%
  \setlength{\parskip}{0pt plus 0.3pt}%
  \let\item\@idxitem
}{%
  \ifkorrekturansicht\clearpage\fi
}
\makeatother

\IfFileExists{\jobname-pw.ind}{\input{\jobname-pw.ind}}{}

% Quellenangabe nur in der Leseansicht
\ifkorrekturansicht\else
% Fallback-Definitionen, falls die .tex-Datei \titel etc. nicht gesetzt hat
\providecommand{\titel}{}
\providecommand{\editorInnen}{}
\providecommand{\dateiname}{\jobname}

\vspace{3cm}

\vfill

\footnotesize
\textsc{Quelle}: \titel. Herausgegeben von {\editorInnen}. In: \emph{Arthur Schnitzler: Briefwechsel mit Autorinnen und Autoren}.
 Digitale Edition, https://schnitzler-briefe.acdh.oeaw.ac.at/{\dateiname}.html (Stand \today)
\fi

\end{document}


      