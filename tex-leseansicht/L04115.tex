%% latex-leseansicht-vorspann.tex
%% Vorspann für die Leseansicht.
%% Lädt die gemeinsame Datei latex-vorspann.tex mit nicht gesetztem Schalter.

\newif\ifkorrekturansicht
\korrekturansichtfalse

\input{../tex-inputs/latex-vorspann}


\section[Arthur Schnitzler an Gustav Schwarzkopf, 24. 6. 1897]{L04115 Arthur Schnitzler an Gustav Schwarzkopf, 24. 6. 1897}
\nopagebreak\mylabel{L04115v}
\rehead{ }\normalsize\beginnumbering\briefempfaengerindex{Schwarzkopf, Gustav@\textsc{Schwarzkopf, Gustav}!zzzSchnitzler, Arthur@\emph{von Arthur Schnitzler}!1897-06-241@{24. 6. 1897}|(be}
\toendnotes[C]{\smallbreak\pagebreak[2]}
\correspDesc{Versand  durch Arthur Schnitzler am 24. 6. 1897 in Wien
\newline{}Übermittlung  am 25. 6. 1897 in Wien
\newline{}Erhalt  durch Gustav Schwarzkopf am 25. 6. 1897 in Wien}\toendnotes[C]{\smallbreak}
\Standort{CUL, Schnitzler, B 96.}
\physDesc{Postkarte, 311 Zeichen
\newline{}Handschrift: Bleistift, deutsche Kurrent
\newline{}Versand: 1) Rohrpost  2) Stempel: »\nobreak{}\oindex{VIII., Josefstadt@\textbf{VIII., Josefstadt}, \emph{Verwaltungsgebiet}|pwk}Wien 8/1, 25 VI 97, 7 10V\nobreak{}«.  3) Stempel: »\nobreak{}\oindex{I., Innere Stadt@\textbf{I., Innere Stadt}, \emph{Verwaltungsgebiet}|pwk}Wien 1/1, 25 VI 96, 7 30V\nobreak{}«. }\toendnotes[C]{\smallbreak}\pstart{}{\pb}Herrn \textsc{Gustav Schwarzkopf}\pend{}\pstart{}Wien\oindex{Wien@\textbf{Wien}, \emph{Verwaltungsgebiet}|pw}\pend{}\pstart{}\textsc{I. Tiefer
                        Graben 23}\oindex{Wien@\textbf{Wien}!I., Innere Stadt@\textbf{I., Innere Stadt}!Tiefer Graben 23@\textbf{Tiefer Graben 23}, \emph{Wohngebäude}|pw}.
               \pend{}{\bigskip}\vspace{1em}
\pstart
           \noindent{}{\pb}Lieber Guſtav. Wenn es Ihnen alſo wirklich nicht unangenehm iſt,
                  \label{K_L04115-1v}\edtext{morgen Freitg Abd}{\lemma{\textnormal{\emph{morgen Freitg Abd}}}\Cendnote{\textnormal{Das erlaubt die Datierung der Postkarte.
                  Durch die fehlende handschriftliche Datierung und zwei unterschiedliche
                  Jahreszahlen bei den Poststempeln besteht eine gewisse Unsicherheit, ob die Karte
                     1896 oder 1897 lief. Da aber Schnitzler nur 1897 am 25. oder 26. Juni am Wiener Westbahnhof\oindex{Wien@\textbf{Wien}!XV., Rudolfsheim-Fünfhaus@\textbf{XV., Rudolfsheim-Fünfhaus}!Westbahnhof@\textbf{Westbahnhof}, \emph{Bahnhof}|pwk} war, lässt
                  sich die Entscheidung treffen. Weil die Karte an einem Donnerstag
                  geschrieben ist, die Poststempel aber einen Freitag bezeichnen, muss die Karte
                  am Donnerstag, dem 24. 6. 1897 verfasst sein, während das \emph{Tagebuch}\pwindex{Schnitzler, Arthur 15. 5. 1862 Wien – 21. 10. 1931 ebd.@\textsc{Schnitzler, Arthur} (15. 5. 1862 Wien – 21. 10. 1931 ebd.), \emph{Schriftsteller, Mediziner}!Tagebuch@\strich\emph{Tagebuch}|pwk} für
                     den Freitag, den 25. 6. 1897 den Besuch am Westbahnhof\oindex{Wien@\textbf{Wien}!XV., Rudolfsheim-Fünfhaus@\textbf{XV., Rudolfsheim-Fünfhaus}!Westbahnhof@\textbf{Westbahnhof}, \emph{Bahnhof}|pwk} dokumentiert – in
                     Begleitung von Schwarzkopf\pwindex{Schwarzkopf, Gustav 7.\,11.\,1853 Wien – 13.\,11.\,1939 ebd.@\textsc{Schwarzkopf, Gustav} (7.\,11.\,1853 Wien – 13.\,11.\,1939 ebd.), \emph{Schriftsteller}|pwk}. }}}\label{K_L04115-1} mit mir auf d Weſtbahn\oindex{Wien@\textbf{Wien}!XV., Rudolfsheim-Fünfhaus@\textbf{XV., Rudolfsheim-Fünfhaus}!Westbahnhof@\textbf{Westbahnhof}, \emph{Bahnhof}|pw} zu fahren,{ }ſo holen Sie mich vor
                  8 bei mir zu Hause ab. Allerdings muſs ich eine Bedingung ſtellen:
               Sie dürfen beim Abſchied nicht weinen. –\pend
           \pstart Herzlich grüßend Ihr \spacefill\mbox{Arth Sch}\pend{}\selectlanguage{ngerman}\endnumbering\briefempfaengerindex{Schwarzkopf, Gustav@\textsc{Schwarzkopf, Gustav}!zzzSchnitzler, Arthur@\emph{von Arthur Schnitzler}!1897-06-241@{24. 6. 1897}|)be}\mylabel{L04115h}
\begin{anhang}
\end{anhang}\newcommand{\dateiname}{L04115}\newcommand{\titel}{Arthur Schnitzler an Gustav Schwarzkopf, 24. 6. 1897}\newcommand{\editorInnen}{Herausgegeben von Jahnke, SelmaMüller, Martin Anton}%% latex-leseansicht-abspann.tex
%% Abspann für die Leseansicht.
%% Der Schalter \ifkorrekturansicht ist bereits durch den Vorspann gesetzt.

%% latex-abspann.tex
%% Gemeinsamer Abspann für Korrekturansicht und Leseansicht.
%% Setzt den Schalter \ifkorrekturansicht voraus (gesetzt in den
%% einbindenden Dateien latex-korrekturansicht-abspann.tex bzw.
%% latex-leseansicht-abspann.tex).
%% ---------------------------------------------------------------

\normalsize

% Das esempio-Environment wird nur in der Leseansicht benötigt
\ifkorrekturansicht\else
\newenvironment{esempio}[3]%
{
    \vspace{1.5ex}
    \rlap{\underline{#1}}
    \par
    \setlength{\parindent}{0cm}
    \nopagebreak
    \leftskip=#2cm
    \rightskip=#3cm
}
{
    \par
}
\fi

\doendnotes{C}
\bigskip
\vfill

\clearpage

\footnotesize

\ifkorrekturansicht
  \lohead{\textsc{register}}
\fi

% theindex-Environment neu definieren ohne reledmac
\makeatletter
\renewenvironment{theindex}{%
  \ifkorrekturansicht
    \section*{\indexname}%
  \else
    \subsubsection*{Index der erwähnten Entitäten}%
  \fi
  \setlength{\parindent}{0pt}%
  \setlength{\parskip}{0pt plus 0.3pt}%
  \let\item\@idxitem
}{%
  \ifkorrekturansicht\clearpage\fi
}
\makeatother

\IfFileExists{\jobname-pw.ind}{\input{\jobname-pw.ind}}{}

% Quellenangabe nur in der Leseansicht
\ifkorrekturansicht\else
% Fallback-Definitionen, falls die .tex-Datei \titel etc. nicht gesetzt hat
\providecommand{\titel}{}
\providecommand{\editorInnen}{}
\providecommand{\dateiname}{\jobname}

\vspace{3cm}

\vfill

\footnotesize
\textsc{Quelle}: \titel. Herausgegeben von {\editorInnen}. In: \emph{Arthur Schnitzler: Briefwechsel mit Autorinnen und Autoren}.
 Digitale Edition, https://schnitzler-briefe.acdh.oeaw.ac.at/{\dateiname}.html (Stand \today)
\fi

\end{document}


