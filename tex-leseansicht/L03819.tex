%% latex-leseansicht-vorspann.tex
%% Vorspann für die Leseansicht.
%% Lädt die gemeinsame Datei latex-vorspann.tex mit nicht gesetztem Schalter.

\newif\ifkorrekturansicht
\korrekturansichtfalse

\input{../tex-inputs/latex-vorspann}


\section[Sigmund Freud an Arthur Schnitzler, 8. 5. 1906]{L03819 Sigmund Freud an Arthur Schnitzler, 8. 5. 1906}
\nopagebreak\mylabel{L03819v}
\rehead{ }\normalsize\beginnumbering\briefempfaengerindex{Schnitzler, Arthur@\textsc{Schnitzler, Arthur}!zzzFreud, Sigmund@\emph{von Sigmund Freud}!1906-05-081@{8. 5. 1906}|(be}
\toendnotes[C]{\smallbreak\pagebreak[2]}
\correspDesc{Versand  durch Sigmund Freud am 8. 5. 1906 in Wien
\newline{}Erhalt  durch Arthur Schnitzler im Zeitraum [8. 5. 1906
                  – 11. 5. 1906?] in Wien}\toendnotes[C]{\smallbreak}
\Standort{CUL, Schnitzler, B 31.}
\physDesc{Brief, 1 Blatt, 2 Seiten, 831 Zeichen
\newline{}Handschrift: schwarze Tinte, deutsche Kurrent}
\buchAbdrucke{\weitereDrucke{1) Sigmund Freud: \emph{Briefe an Arthur Schnitzler.}Herausgegeben von Henry Schnitzler In: \emph{Neue deutsche Rundschau}, Jg. 66 (Januar 1955) Nr. 1, S. 95.} \weitereDrucke{2) Sigmund Freud: \emph{Briefe 1873–1939}. Ausgewählt und herausgegeben von Ernst L. Freud. Frankfurt am Main: \emph{S. Fischer} 1960, S. 249–250.} \weitereDrucke{3) Sigmund Freud: \emph{Sigmund Freud Edition. Digitale historisch-kritische Gesamtausgabe}. Herausgegeben von Christine Diercks, Arkadi Blatow und Elisabeth Skale. (2014–2025) \url{https://www.freudedition.net/briefe/freud-sigmund/schnitzler-arthur/1906/05/08}.} }\toendnotes[C]{\smallbreak}
\pstart
           \raggedleft{}{\pb}8. 5. 06.\pend
           
\pstart
           \textcolor{gray}{\textbf{Prof. Dr. Freud}}\hfill \textcolor{gray}{\textbf{IX., Berggasse 19\oindex{Wien@\textbf{Wien}!IX., Alsergrund@\textbf{IX., Alsergrund}!Berggasse 19@\textbf{Berggasse 19}, \emph{Wohngebäude}|pw}.}}\pend
           
\pstart{}Verehrter Herr Doktor\pend\vspace{0.5em}
\pstart
           Seit vielen Jahren bin ich mir der weit reichenden Übereinstimmung bewußt, die
               zwiſchen Ihren u meinen Auffaſſungen mancher pſychologischer und erotischer Probleme
               beſteht und kürzlich habe ich ja den Mut gefunden eine ſolche \label{K_L03819-1v}\edtext{ausdrücklich hervorzuheben}{\lemma{\textnormal{\emph{ausdrücklich hervorzuheben}}}\Cendnote{\textnormal{In einer Fußnote zu einer Stelle, demnach 
                     Kranke gar nicht immer willens sind, ihre Krankheit aufzugeben: »Ein Dichter, der allerdings auch Arzt
                        ist, \so{Arthur Schnitzler}, hat dieser Erkenntnis in seinem ›Paracelsus\pwindex{Schnitzler, Arthur 15.\,5.\,1862 Wien – 21.\,10.\,1931 ebd.@\textsc{Schnitzler, Arthur} (15.\,5.\,1862 Wien – 21.\,10.\,1931 ebd.), \emph{Schriftsteller, Mediziner}!Paracelsus. Versspiel in einem Akt@\strich\emph{Paracelsus. Versspiel in einem Akt}|pw}‹ sehr richtigen
                        Ausdruck gegeben.« (\emph{Bruchstück einer Hysterie-Analyse}\pwindex{Freud, Sigmund 6.\,5.\,1856 Pribor – 23.\,9.\,1939 London@\textsc{Freud, Sigmund} (6.\,5.\,1856 Pribor – 23.\,9.\,1939 London), \emph{Psychoanalytiker}!Bruchstück einer Hysterie-Analyse@\strich\emph{Bruchstück einer Hysterie-Analyse}|pwk}. In:
                        \emph{Monatsschrift für Psychiatrie und
                        Neurologie}\pwindex{Monatsschrift für Psychiatrie und Neurologie@\emph{Monatsschrift für Psychiatrie und Neurologie}|pwk}. Bd. 18, H. 4, Oktober 1905, S. 285–309 und H. 5, November 1905, S. 408–467,
                     hier S. 411).}}}\label{K_L03819-1} (Bruchſtück einer
                  Hyſterieanalyſe\pwindex{Freud, Sigmund 6.\,5.\,1856 Pribor – 23.\,9.\,1939 London@\textsc{Freud, Sigmund} (6.\,5.\,1856 Pribor – 23.\,9.\,1939 London), \emph{Psychoanalytiker}!Bruchstück einer Hysterie-Analyse@\strich\emph{Bruchstück einer Hysterie-Analyse}|pw}{ } 1905). Ich habe mich oft verwundert gefragt,
               woher Sie dieſe oder jene geheime Ke{\geminationn}tniß nehmen
               könnten, die ich mir durch mühſeliges Erforſchen des Objektes erworben und endlich
               kam ich dazu, den Dichter zu beneiden, den ich ſonſt bewundert.\pend
           
\pstart
           {\pb}Nun mögen Sie erraten, wie{ }ſehr mich die
                  \label{K_L03819-2v}\edtext{Zeilen}{\lemma{\textnormal{\emph{Zeilen}}}\Cendnote{\textnormal{Vgl. XXXX Auszeichnungsfehler: Dokument L03815 nicht gefunden.}}}\label{K_L03819-2} erfreut und erhoben, in
               denen Sie mir ſagen, daß auch Sie aus meinen Schriften Anregung geſchöpft haben. Es
               kränkt mich faſt, daß ich 50 Jahre alt werden mußte, um etwas ſo Ehrenvolles zu
               erfahren.\pend
           
\pstart
           Ihr in Verehrung ergebener{\\[\baselineskip]}\spacefill\mbox{D\textsuperscript{r} Freud}\pend
           \leftskip=0em{}\selectlanguage{ngerman}\endnumbering\briefempfaengerindex{Schnitzler, Arthur@\textsc{Schnitzler, Arthur}!zzzFreud, Sigmund@\emph{von Sigmund Freud}!1906-05-081@{8. 5. 1906}|)be}\mylabel{L03819h}
\begin{anhang}
\end{anhang}\newcommand{\dateiname}{L03819}\newcommand{\titel}{Sigmund Freud an Arthur Schnitzler, 8. 5. 1906}\newcommand{\editorInnen}{Selma Jahnke und Martin Anton Müller}%% latex-leseansicht-abspann.tex
%% Abspann für die Leseansicht.
%% Der Schalter \ifkorrekturansicht ist bereits durch den Vorspann gesetzt.

%% latex-abspann.tex
%% Gemeinsamer Abspann für Korrekturansicht und Leseansicht.
%% Setzt den Schalter \ifkorrekturansicht voraus (gesetzt in den
%% einbindenden Dateien latex-korrekturansicht-abspann.tex bzw.
%% latex-leseansicht-abspann.tex).
%% ---------------------------------------------------------------

\normalsize

% Das esempio-Environment wird nur in der Leseansicht benötigt
\ifkorrekturansicht\else
\newenvironment{esempio}[3]%
{
    \vspace{1.5ex}
    \rlap{\underline{#1}}
    \par
    \setlength{\parindent}{0cm}
    \nopagebreak
    \leftskip=#2cm
    \rightskip=#3cm
}
{
    \par
}
\fi

\doendnotes{C}
\bigskip
\vfill

\clearpage

\footnotesize

\ifkorrekturansicht
  \lohead{\textsc{register}}
\fi

% theindex-Environment neu definieren ohne reledmac
\makeatletter
\renewenvironment{theindex}{%
  \ifkorrekturansicht
    \section*{\indexname}%
  \else
    \subsubsection*{Index der erwähnten Entitäten}%
  \fi
  \setlength{\parindent}{0pt}%
  \setlength{\parskip}{0pt plus 0.3pt}%
  \let\item\@idxitem
}{%
  \ifkorrekturansicht\clearpage\fi
}
\makeatother

\IfFileExists{\jobname-pw.ind}{\input{\jobname-pw.ind}}{}

% Quellenangabe nur in der Leseansicht
\ifkorrekturansicht\else
% Fallback-Definitionen, falls die .tex-Datei \titel etc. nicht gesetzt hat
\providecommand{\titel}{}
\providecommand{\editorInnen}{}
\providecommand{\dateiname}{\jobname}

\vspace{3cm}

\vfill

\footnotesize
\textsc{Quelle}: \titel. Herausgegeben von {\editorInnen}. In: \emph{Arthur Schnitzler: Briefwechsel mit Autorinnen und Autoren}.
 Digitale Edition, https://schnitzler-briefe.acdh.oeaw.ac.at/{\dateiname}.html (Stand \today)
\fi

\end{document}


