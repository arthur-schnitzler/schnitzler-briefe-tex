%% latex-korrekturansicht-vorspann.tex
%% Vorspann für die Korrekturansicht.
%% Lädt die gemeinsame Datei latex-vorspann.tex mit gesetztem Schalter.

\newif\ifkorrekturansicht
\korrekturansichttrue

\input{../tex-inputs/latex-vorspann}


\section[Sigmund Freud an Arthur Schnitzler, 8. 5. 1906]{L03819 Sigmund Freud an Arthur Schnitzler, 8. 5. 1906}
\nopagebreak\mylabel{L03819v}
\rehead{ }\normalsize\beginnumbering\briefempfaengerindex{Schnitzler, Arthur@\textsc{Schnitzler, Arthur}!zzzFreud, Sigmund@\emph{von Sigmund Freud}!1906-05-081@{8. 5. 1906}|(be}
\toendnotes[C]{\smallbreak\pagebreak[2]}\Standort{CUL, Schnitzler, B 31.}
\physDesc{Brief, 1 Blatt, 2 Seiten, 831 Zeichen
\newline{}Handschrift: schwarze Tinte, deutsche Kurrent}\toendnotes[C]{\smallbreak}
\pstart
           \raggedleft{}{\pb}8. 5. 06. \pend
           
\pstart
           \textcolor{gray}{\textbf{Prof. Dr.
                     Freud}}\hfill \textcolor{gray}{\textbf{IX., Berggasse 19\oindex{Berggasse 19@\textbf{Berggasse 19}, \emph{Wohngebäude (K.WHS)}|pw}.}}\pend
           
\pstart{}Verehrter Herr Doktor\pend\vspace{0.5em}
\pstart
           Seit vielen Jahren bin ich mir der weit reichenden Übereinstimmung bewußt, die
               zwiſchen Ihren u meinen Auffaſſungen mancher pſychologischer und erotischer Probleme
               beſteht und kürzlich habe ich ja den Mut gefunden eine ſolche \label{K_L03819-1v}\edtext{ausdrücklich hervorzuheben}{\lemma{\textnormal{\emph{ausdrücklich hervorzuheben}}}\Cendnote{\textnormal{Freud\pwindex{Freud, Sigmund 06.05.1856 – 23.09.1939@\textsc{Freud, Sigmund} (06.05.1856 – 23.09.1939), \emph{Psychoanalytiker/Psychoanalytikerin}|pwk} würdigte Schnitzler in einer
                  Fußnote zum Thema des Widerstands von Kranken, ihr Leiden aufzugeben: »Ein Dichter,
                     der allerdings auch Arzt ist, Arthur Schnitzler, hat dieser Erkenntnis in
                     seinem ›Paracelsus‹ sehr richtigen Ausdruck gegeben.« (\emph{Bruchstück einer Hysterie-Analyse}\pwindex{Bruchstueck einer Hysterie-Analyse@\emph{Bruchstück einer Hysterie-Analyse}|pwk}. In:
                        \emph{Monatsschrift für Psychiatrie und Neurologie}\pwindex{Monatsschrift fuer Psychiatrie und Neurologie@\emph{Monatsschrift für Psychiatrie und Neurologie}|pwk}. Bd. 18, Nr. 4, 1905,
                     S. 285–309 und 408–467, hier S. 411).}}}\label{K_L03819-1} (Bruchſtück einer Hyſterieanalyſe\pwindex{Bruchstueck einer Hysterie-Analyse@\emph{Bruchstück einer Hysterie-Analyse}|pw} 1905). Ich
               habe mich oft verwundert gefragt, woher Sie dieſe oder jene geheime Ke{\geminationn}tniß nehmen
               könnten, die ich mir durch mühſeliges Erforſchen des Objektes erworben und endlich
               kam ich dazu, den Dichter zu beneiden, den ich ſonſt bewundert.\pend
           
\pstart
           {\pb}Nun mögen Sie erraten, wie ſehr mich
                  die \label{K_L03819-2v}\edtext{Zeilen}{\lemma{\textnormal{\emph{Zeilen}}}\Cendnote{\textnormal{Arthur Schnitzler an Sigmund Freud, 6. 5. 1906,
               Briefentwurf.}}}\label{K_L03819-2} erfreut und erhoben, in denen Sie mir ſagen, daß auch Sie aus meinen
                  Schriften Anregung geſchöpft haben. Es kränkt mich faſt, daß ich 50 Jahre alt
                  werden mußte, um etwas ſo Ehrenvolles zu erfahren.\pend
           
\pstart
           Ihr in Verehrung
                     ergebener{\\[\baselineskip]}\spacefill\mbox{D\textsuperscript{r} Freud}\pend
           \leftskip=0em{}\selectlanguage{ngerman}\endnumbering\briefempfaengerindex{Schnitzler, Arthur@\textsc{Schnitzler, Arthur}!zzzFreud, Sigmund@\emph{von Sigmund Freud}!1906-05-081@{8. 5. 1906}|)be}\mylabel{L03819h}
\begin{anhang}
\end{anhang}\normalsize

\doendnotes{C}
\bigskip
\vfill

\clearpage

\footnotesize

\lohead{\textsc{register}}

% Definiere theindex-Environment komplett neu ohne reledmac
\makeatletter
\renewenvironment{theindex}{%
  \section*{\indexname}%
  \setlength{\parindent}{0pt}%
  \setlength{\parskip}{0pt plus 0.3pt}%
  \let\item\@idxitem
}{%
  \clearpage
}
\makeatother

\IfFileExists{\jobname-pw.ind}{\input{\jobname-pw.ind}}{}

\end{document}

      