%% latex-korrekturansicht-vorspann.tex
%% Vorspann für die Korrekturansicht.
%% Lädt die gemeinsame Datei latex-vorspann.tex mit gesetztem Schalter.

\newif\ifkorrekturansicht
\korrekturansichttrue

\input{../tex-inputs/latex-vorspann}


\section[Arthur Schnitzler an Stefan Zweig, 22. 10. 1908]{L03801 Arthur Schnitzler an Stefan Zweig, 22. 10. 1908}
\nopagebreak\mylabel{L03801v}
\rehead{ }\normalsize\beginnumbering\briefempfaengerindex{Zweig, Stefan@\textsc{Zweig, Stefan}!zzzSchnitzler, Arthur@\emph{von Arthur Schnitzler}!1908-10-221@{22. 10. 1908}|(be}
\toendnotes[C]{\smallbreak\pagebreak[2]}\Standort{Jerusalem, National Library of Israel, ARC. Ms. Var. 305 1 58 Stefan Zweig Collection.}
\physDesc{Briefkarte, 1 Blatt, 1 Seite, 154 Zeichen
\newline{}Handschrift: schwarze Tinte, deutsche Kurrent}\toendnotes[C]{\smallbreak}
\pstart
           {\pb}\textcolor{gray}{\textbf{Dr Arthur Schnitzler}}\hfill 22. X. 908\pend
           
\pstart
           \textcolor{gray}{\textbf{Wien XVIII.
                        Spoettelgasse 7\oindex{Edmund-Weiss-Gasse@\textbf{Edmund-Weiß-Gasse}, \emph{R.ST}|pw}.}}\pend
           \vspace{0.5em}
\pstart
           lieber Herr Doktor, hätten Sie vielleicht Zeit, am \label{K_L03801-1v}\edtext{Samſtag Abend}{\lemma{\textnormal{\emph{Samſtag Abend}}}\Cendnote{\textnormal{Vgl. Stefan Zweig an Arthur Schnitzler, [22. 10. 1908?].}}}\label{K_L03801-1} etwa 6, ½ 7 zu mir zu ko{\geminationm}en? Ich
               würde mich ſehr freuen.\pend
           
\pstart
           Herzlichſt grüßend{\\[\baselineskip]}Ihr{\\[\baselineskip]}\spacefill\mbox{A. S.}\pend
           \leftskip=0em{}\selectlanguage{ngerman}\endnumbering\briefempfaengerindex{Zweig, Stefan@\textsc{Zweig, Stefan}!zzzSchnitzler, Arthur@\emph{von Arthur Schnitzler}!1908-10-221@{22. 10. 1908}|)be}\mylabel{L03801h}  \normalsize

\doendnotes{C}
\bigskip
\vfill

\clearpage

\footnotesize

\lohead{\textsc{register}}

% Definiere theindex-Environment komplett neu ohne reledmac
\makeatletter
\renewenvironment{theindex}{%
  \section*{\indexname}%
  \setlength{\parindent}{0pt}%
  \setlength{\parskip}{0pt plus 0.3pt}%
  \let\item\@idxitem
}{%
  \clearpage
}
\makeatother

\IfFileExists{\jobname-pw.ind}{\input{\jobname-pw.ind}}{}

\end{document}

      