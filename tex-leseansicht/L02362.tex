%% latex-leseansicht-vorspann.tex
%% Vorspann für die Leseansicht.
%% Lädt die gemeinsame Datei latex-vorspann.tex mit nicht gesetztem Schalter.

\newif\ifkorrekturansicht
\korrekturansichtfalse

\input{../tex-inputs/latex-vorspann}


               \section[Stefan Großmann an Arthur Schnitzler, 10. 2. 1921]{ Stefan Großmann an Arthur Schnitzler, 10. 2. 1921}\nopagebreak\mylabel{v}\rehead{ }\begin{ledgroupsized}[t]{13cm}\normalsize\beginnumbering\briefempfaengerindex{Schnitzler, Arthur@\textsc{Schnitzler, Arthur}!zzzGrossmann, Stefan@\emph{von Stefan Großmann}!1921-02-101@{10. 2. 1921}|(be} \toendnotes[C]{\smallbreak\pagebreak[2]} \Standort{CUL, Schnitzler, B 34.}
\physDesc{Brief, 1 Blatt, 1 Seite
\newline{}Schreibmaschine
\newline{}Handschrift: blaue Tinte, deutsche Kurrent (\noindent{}Unterschrift)
\newline{}Schnitzler: 1) mit rotem Buntstift zwei Unterstreichungen 2) mit Bleistift auf der Rückseite das Antwortschreiben in
                                 Lateinschrift skizziert: »\noindent{}{\pb}Viel\textcolor{gray}{en} Dank für Ihre freund
                                       Zeilen.{ / }Sich\textcolor{gray}{er} keine Absicht –{ / }Gra mit Herr Hard\textcolor{gray}{en}\pwindex{Harden, Maximilian 20.10.1861 – 30.10.1927@\textsc{Harden, Maximilian} (20.10.1861 – 30.10.1927), \emph{Schriftsteller, Publizist}|pw} {\dotstwo}{ / }Üb hiesige\textcolor{gray}{s}{\dotstwo} haben Sie wohl geles{ / }Ich käme mir nur komisch vor sollt ich und Herr Kunsch\pwindex{Kunschak, Leopold 11.11.1871 – 13.03.1953@\textsc{Kunschak, Leopold} (11.11.1871 – 13.03.1953), \emph{Politiker}|pw} od nur der
                                       Schusterlehrling, polemis, der das Theater stürmt {\dots} in d\textcolor{gray}{em} Rufe
                                       »Man schändet uns Frauen« (u das Stück \strikeout{\textcolor{gray}{imm}} das \textcolor{gray}{er} kannte.{ / }Wobei m\textcolor{gray}{eine} Sympathie noch im mehr bei d
                                       Schusterlehrlg als bei den »Seipel\pwindex{Seipel, Ignaz 19.07.1876 – 02.08.1932@\textsc{Seipel, Ignaz} (19.07.1876 – 02.08.1932), \emph{Politiker, Prälat, Bundeskanzler}|pw} u Kun\pwindex{Kunschak, Leopold 11.11.1871 – 13.03.1953@\textsc{Kunschak, Leopold} (11.11.1871 – 13.03.1953), \emph{Politiker}|pw}{ / }– Aehnlich\textcolor{gray}{es} ist \textcolor{gray}{im
                                          wieder} einem{ / }passirt, Gustl\pwindex{Schnitzler, Arthur 15.05.1862 – 21.10.1931@\textsc{Schnitzler, Arthur} (15.05.1862 – 21.10.1931), \emph{Schriftsteller, Mediziner}!Lieutenant Gustl. Novelle25. 12. 1900@\strich\emph{Lieutenant Gustl. Novelle} {[}25. 12. 1900{]}|pw} – Bernha\pwindex{Schnitzler, Arthur 15.05.1862 – 21.10.1931@\textsc{Schnitzler, Arthur} (15.05.1862 – 21.10.1931), \emph{Schriftsteller, Mediziner}!Professor Bernhardi. Komoedie in fuenf Akten1912@\strich\emph{Professor Bernhardi. Komödie in fünf Akten} {[}1912{]}|pw}.{ / }Die \strikeout{Stücke dank}{ }\textcolor{gray}{von} meine Stüc u d\textcolor{gray}{ie}
                                       Blamage mein\textcolor{gray}{er} Gegne\textcolor{gray}{r}{ / }\textcolor{gray}{Unerhörtes}!{ / }Herzl«\newline{}Ordnung: mit Bleistift von unbekannter Hand nummeriert:
                                    »15« }\toendnotes[C]{\smallbreak}\pstart
           \noindent{}\centering{}{\pb}\textcolor{gray}{\textbf{Das Tage-Buch\orgindex{Tage-Buch@Das Tage-Buch|pw}}}\pend
           \pstart
           \noindent{}\centering{}\textcolor{gray}{\textbf{Erscheint jeden Sonnabend ⋅ Herausgeber: Stefan Großmann}}\pend
           \pstart
           \noindent{}\centering{}\textcolor{gray}{\textbf{Ernst Rowohlt Verlag\orgindex{Ernst Rowohlt Verlag@Ernst Rowohlt Verlag|pw} ⋅ Berlin W 35\oindex{Berlin@\textbf{Berlin}|pw}}}\pend
           \pstart
           \noindent{}\centering{}\textcolor{gray}{\textbf{POTSDAMER STRASSE 123\textsuperscript{B}
                        ⋅ AN DER POTSDAMER BRÜCKE\oindex{Potsdamerstrasse@\textbf{Potsdamerstraße}|pw}}}\pend
           \pstart
           \noindent{}\centering{}\textcolor{gray}{\textbf{TELEGRAMM-ADRESSE: TAGEBUCH
                        BERLIN\orgindex{Tage-Buch@Das Tage-Buch|pw} ⋅ FERNSPRECHER: AMT LÜTZOW\orgindex{Fernsprechamt Lietzow@Fernsprechamt Lietzow|pw}
                     Nr. 4931}}\pend
           \pstart
           \noindent{}\centering{}\textcolor{gray}{\textbf{SPRECHSTUNDE DER REDAKTION: 12–1 UHR}}\pend
           \pstart
           \noindent{}Gr/Sch\pend
           \pstart
           \centering{}10. Februar 1921\pend
           \pstart
           \textcolor{gray}{\textbf{\emph{REDAKTION}}}\pend
           \pstart
           Herrn\pend
           \leftskip=3em{}\pstart
           \noindent{}Dr.med. Arthur \so{Schnitzler}\pend
           \leftskip=0em{}\leftskip=3em{}\pstart
           \so{Wien}\oindex{Wien@\textbf{Wien}|pw}\pend
           \leftskip=0em{}\leftskip=3em{}\pstart
           Sternwartstr. 71\oindex{Sternwartestrasse@\textbf{Sternwartestraße}|pw}\pend
           \leftskip=0em{}\pstart{}Verehrter lieber Herr Dr. Schnitzler!\pend\pstart
           Ich übersende Ihnen heute einige Nummern des »Tage-Buch\orgindex{Tage-Buch@Das Tage-Buch|pw}«, in denen ich die etwas heuchlerische Hetze gegen den »Reigen\pwindex{Schnitzler, Arthur 15.05.1862 – 21.10.1931@\textsc{Schnitzler, Arthur} (15.05.1862 – 21.10.1931), \emph{Schriftsteller, Mediziner}!Reigen. Zehn Dialoge1900@\strich\emph{Reigen. Zehn Dialoge} {[}1900{]}|pw}« satyrisch behandelt\pwindex{Tilla zuernt der Zeit1921-01-15 – 1921-01-15@\emph{Tilla zürnt der Zeit} {[}1921-01-15 – 1921-01-15{]}|pwv}\pwindex{Grossmann, Stefan 19.05.1875 – 03.01.1935@\textsc{Großmann, Stefan} (19.05.1875 – 03.01.1935), \emph{Schriftsteller, Journalist}!Haenischs Reigen. Eine unsittliche Szenenfolge1921-01-15 – 1921-01-15@\strich\emph{Hänischs Reigen. Eine unsittliche Szenenfolge} {[}1921-01-15 – 1921-01-15{]}|pwv} habe. Es ist mir bekannt, dass Sie
               niemals zu Ihrem Schaffen selbst das Wort nehmen wollten. Wenn Sie aber bedenken, in
               wie unangenehmer Form Harden\pwindex{Harden, Maximilian 20.10.1861 – 30.10.1927@\textsc{Harden, Maximilian} (20.10.1861 – 30.10.1927), \emph{Schriftsteller, Publizist}|pw} jetzt gegen die »Reigen\pwindex{Schnitzler, Arthur 15.05.1862 – 21.10.1931@\textsc{Schnitzler, Arthur} (15.05.1862 – 21.10.1931), \emph{Schriftsteller, Mediziner}!Reigen. Zehn Dialoge1900@\strich\emph{Reigen. Zehn Dialoge} {[}1900{]}|pw}«-Aufführung geschrieben\pwindex{Harden, Maximilian 20.10.1861 – 30.10.1927@\textsc{Harden, Maximilian} (20.10.1861 – 30.10.1927), \emph{Schriftsteller, Publizist}!Reigen08. 01. 1921@\strich\emph{Reigen} {[}08. 01. 1921{]}|pw} hat, wäre es vielleicht doch von Wert und Nutzen, wenn Sie sich
               entschliessen könnten, im »Tage-Buch\orgindex{Tage-Buch@Das Tage-Buch|pw}« selbst das
               Wort zu ergreifen und sich zur öffentlichen Aufführung des »\label{T_L02362_1v}\edtext{Reigen\pwindex{Schnitzler, Arthur 15.05.1862 – 21.10.1931@\textsc{Schnitzler, Arthur} (15.05.1862 – 21.10.1931), \emph{Schriftsteller, Mediziner}!Reigen. Zehn Dialoge1900@\strich\emph{Reigen. Zehn Dialoge} {[}1900{]}|pw}}{\lemma{\textnormal{\emph{Reigen}}}\Cendnote{\textnormal{geschrieben Reiegn}}}\label{T_L02362_1h}« zu äussern.
               Jedenfalls bitte ich Sie, über meine Zeitschrift zu verfügen. Das »Tage-Buch\orgindex{Tage-Buch@Das Tage-Buch|pw}« hat sich in den fünfviertel Jahren seines Bestehens in
                  Deutschland\oindex{Deutschland@\textbf{Deutschland}|pw} vollkommen durchgesetzt und Sie
               sprechen durch mein »Tage-Buch\orgindex{Tage-Buch@Das Tage-Buch|pw}« zu dem gebildeten
                  Deutschland\oindex{Deutschland@\textbf{Deutschland}|pw}, das ehedem die »Zukunft\pwindex{Zukunft1892 – 1922@\emph{Die Zukunft}|pw}« gelesen hat. Ich würde mich freuen und das Gefühl haben,
               einer gerechten Sache zu dienen, wenn Sie sich entschliessen wollten, durch das »Tage-Buch\orgindex{Tage-Buch@Das Tage-Buch|pw}« zu sprechen.\pend
           \pstart
           Mit herzlichen Grüssen{\\[\baselineskip]} Ihr sehr ergebener{\\[\baselineskip]}\spacefill\mbox{{[}hs.:{]} Stefan Großmann}\pend
           \leftskip=0em{}          \endnumbering\briefempfaengerindex{Schnitzler, Arthur@\textsc{Schnitzler, Arthur}!zzzGrossmann, Stefan@\emph{von Stefan Großmann}!1921-02-101@{10. 2. 1921}|)be}\mylabel{h}\end{ledgroupsized}  \newcommand{\dateiname}{L02362}\newcommand{\titel}{Stefan Großmann an Arthur Schnitzler, 10. 2. 1921}\newcommand{\editorInnen}{ Martin Anton Müller und Gerd-Hermann Susen}
            \footnotesize
\begin{ledgroupsized}[t]{11.5cm}
\doendnotes{C}
\end{ledgroupsized}
         %% latex-leseansicht-abspann.tex
%% Abspann für die Leseansicht.
%% Der Schalter \ifkorrekturansicht ist bereits durch den Vorspann gesetzt.

%% latex-abspann.tex
%% Gemeinsamer Abspann für Korrekturansicht und Leseansicht.
%% Setzt den Schalter \ifkorrekturansicht voraus (gesetzt in den
%% einbindenden Dateien latex-korrekturansicht-abspann.tex bzw.
%% latex-leseansicht-abspann.tex).
%% ---------------------------------------------------------------

\normalsize

% Das esempio-Environment wird nur in der Leseansicht benötigt
\ifkorrekturansicht\else
\newenvironment{esempio}[3]%
{
    \vspace{1.5ex}
    \rlap{\underline{#1}}
    \par
    \setlength{\parindent}{0cm}
    \nopagebreak
    \leftskip=#2cm
    \rightskip=#3cm
}
{
    \par
}
\fi

\doendnotes{C}
\bigskip
\vfill

\clearpage

\footnotesize

\ifkorrekturansicht
  \lohead{\textsc{register}}
\fi

% theindex-Environment neu definieren ohne reledmac
\makeatletter
\renewenvironment{theindex}{%
  \ifkorrekturansicht
    \section*{\indexname}%
  \else
    \subsubsection*{Index der erwähnten Entitäten}%
  \fi
  \setlength{\parindent}{0pt}%
  \setlength{\parskip}{0pt plus 0.3pt}%
  \let\item\@idxitem
}{%
  \ifkorrekturansicht\clearpage\fi
}
\makeatother

\IfFileExists{\jobname-pw.ind}{\input{\jobname-pw.ind}}{}

% Quellenangabe nur in der Leseansicht
\ifkorrekturansicht\else
% Fallback-Definitionen, falls die .tex-Datei \titel etc. nicht gesetzt hat
\providecommand{\titel}{}
\providecommand{\editorInnen}{}
\providecommand{\dateiname}{\jobname}

\vspace{3cm}

\vfill

\footnotesize
\textsc{Quelle}: \titel. Herausgegeben von {\editorInnen}. In: \emph{Arthur Schnitzler: Briefwechsel mit Autorinnen und Autoren}.
 Digitale Edition, https://schnitzler-briefe.acdh.oeaw.ac.at/{\dateiname}.html (Stand \today)
\fi

\end{document}


      