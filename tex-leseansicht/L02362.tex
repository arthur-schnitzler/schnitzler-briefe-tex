%% latex-korrekturansicht-vorspann.tex
%% Vorspann für die Korrekturansicht.
%% Lädt die gemeinsame Datei latex-vorspann.tex mit gesetztem Schalter.

\newif\ifkorrekturansicht
\korrekturansichttrue

\input{../tex-inputs/latex-vorspann}


\section[Stefan Großmann an Arthur Schnitzler, 10. 2. 1921]{L02362 Stefan Großmann an Arthur Schnitzler, 10. 2. 1921}
\nopagebreak\mylabel{L02362v}
\rehead{ }\normalsize\beginnumbering\briefempfaengerindex{Schnitzler, Arthur@\textsc{Schnitzler, Arthur}!zzzGrossmann, Stefan@\emph{von Stefan Großmann}!1921-02-101@{10. 2. 1921}|(be}
\toendnotes[C]{\smallbreak\pagebreak[2]}\Standort{CUL, Schnitzler, B 34.}
\physDesc{Brief, 1 Blatt, 1 Seite, 1117 Zeichen
\newline{}Schreibmaschine
\newline{}Handschrift: blaue Tinte, deutsche Kurrent (\noindent{}Unterschrift)
\newline{}Schnitzler: 1) mit rotem Buntstift zwei Unterstreichungen  2) mit Bleistift auf der Rückseite das Antwortschreiben in
                                 Lateinschrift skizziert: »\noindent{}{\pb}Viel\textcolor{gray}{en} Dank für Ihre freund
                                       Zeilen.{ / }Sich\textcolor{gray}{er} keine Absicht –{ / }Gra mit Herr Hard\textcolor{gray}{en}\pwindex{Harden, Maximilian 20.10.1861 – 30.10.1927@\textsc{Harden, Maximilian} (20.10.1861 – 30.10.1927), \emph{Schriftsteller/Schriftstellerin, Publizist/Publizistin}|pw}{\dotstwo}{ / }Üb hiesige\textcolor{gray}{s}{\dotstwo} haben Sie wohl geles{ / }Ich käme mir nur komisch vor sollt ich und Herr Kunsch\pwindex{Kunschak, Leopold 11.11.1871 – 13.03.1953@\textsc{Kunschak, Leopold} (11.11.1871 – 13.03.1953), \emph{Politiker/Politikerin}|pw} od nur der
                                       Schusterlehrling, polemis, der das Theater stürmt {\dots} in d\textcolor{gray}{em} Rufe
                                       ›Man schändet uns Frauen‹ (u das Stück \strikeout{\textcolor{gray}{imm}} das \textcolor{gray}{er} kannte.{ / }Wobei m\textcolor{gray}{eine} Sympathie noch im mehr bei d
                                       Schusterlehrlg als bei den ›Seipel\pwindex{Seipel, Ignaz 19.07.1876 – 02.08.1932@\textsc{Seipel, Ignaz} (19.07.1876 – 02.08.1932), \emph{Politiker/Politikerin, Prälat/Prälatin, Bundeskanzler/Bundeskanzlerin}|pw} u Kun\pwindex{Kunschak, Leopold 11.11.1871 – 13.03.1953@\textsc{Kunschak, Leopold} (11.11.1871 – 13.03.1953), \emph{Politiker/Politikerin}|pw}{ / }– Aehnlich\textcolor{gray}{es} ist \textcolor{gray}{im
                                          wieder} einem{ / }passirt, Gustl\pwindex{Lieutenant Gustl. Novelle@\emph{Lieutenant Gustl. Novelle}|pw} – Bernha\pwindex{Professor Bernhardi. Komoedie in fuenf Akten@\emph{Professor Bernhardi. Komödie in fünf Akten}|pw}.{ / }Die \strikeout{Stücke dank}{ }\textcolor{gray}{von} meine Stüc u d\textcolor{gray}{ie}
                                       Blamage mein\textcolor{gray}{er} Gegne\textcolor{gray}{r}{ / }\textcolor{gray}{Unerhörtes}!{ / }Herzl«
\newline{}Ordnung: mit Bleistift von unbekannter Hand nummeriert:
                                    »15« }\toendnotes[C]{\smallbreak}
\pstart
           \centering{}{\pb}\textcolor{gray}{\textbf{Das Tage-Buch\orgindex{Tage-Buch@Das Tage-Buch|pw}}}\pend
           
\pstart
           \centering{}\textcolor{gray}{\textbf{Erscheint jeden Sonnabend ⋅ Herausgeber: Stefan Großmann}}\pend
           
\pstart
           \centering{}\textcolor{gray}{\textbf{Ernst Rowohlt Verlag\orgindex{Ernst Rowohlt Verlag@Ernst Rowohlt Verlag|pw} ⋅ Berlin W 35\oindex{Berlin@\textbf{Berlin}, \emph{P.PPLC}|pw}}}\pend
           
\pstart
           \centering{}\textcolor{gray}{\textbf{POTSDAMER STRASSE 123\textsuperscript{B} ⋅ AN DER POTSDAMER BRÜCKE\oindex{Potsdamer Strasse@\textbf{Potsdamer Straße}, \emph{Straße (K.STR)}|pw}}}\pend
           
\pstart
           \centering{}\textcolor{gray}{\textbf{TELEGRAMM-ADRESSE: TAGEBUCH
                        BERLIN\orgindex{Tage-Buch@Das Tage-Buch|pw} ⋅ FERNSPRECHER: AMT LÜTZOW\orgindex{Fernsprechamt Lietzow@Fernsprechamt Lietzow|pw}
                     Nr. 4931}}\pend
           
\pstart
           \centering{}\textcolor{gray}{\textbf{SPRECHSTUNDE DER REDAKTION: 12–1 UHR}}\pend
           
\pstart
           Gr/Sch\pend
           
\pstart
           \centering{}10. Februar 1921\pend
           
\pstart
           \textcolor{gray}{\textbf{\emph{REDAKTION}}}\pend
           
\pstart
           Herrn\pend
           
\pstart
           Dr.med. Arthur \so{Schnitzler}\pend
           
\pstart
           \so{Wien}\oindex{Wien@\textbf{Wien}, \emph{A.ADM2}|pw}\pend
           
\pstart
           Sternwartstr. 71\oindex{Sternwartestrasse 71@\textbf{Sternwartestraße 71}, \emph{Wohngebäude (K.WHS)}|pw}\pend
           
\pstart{}Verehrter lieber Herr Dr. Schnitzler!\pend\vspace{0.5em}
\pstart
           Ich übersende Ihnen heute einige Nummern des »Tage-Buch\orgindex{Tage-Buch@Das Tage-Buch|pw}«, in denen ich die etwas heuchlerische Hetze gegen den »Reigen\pwindex{Reigen. Zehn Dialoge@\emph{Reigen. Zehn Dialoge}|pw}« satyrisch behandelt\pwindex{Tilla zuernt der Zeit@\emph{Tilla zürnt der Zeit}|pwv}\pwindex{Haenischs Reigen. Eine unsittliche Szenenfolge@\emph{Hänischs Reigen. Eine unsittliche Szenenfolge}|pwv} habe. Es ist mir bekannt, dass Sie
               niemals zu Ihrem Schaffen selbst das Wort nehmen wollten. Wenn Sie aber bedenken, in
               wie unangenehmer Form Harden\pwindex{Harden, Maximilian 20.10.1861 – 30.10.1927@\textsc{Harden, Maximilian} (20.10.1861 – 30.10.1927), \emph{Schriftsteller/Schriftstellerin, Publizist/Publizistin}|pw} jetzt gegen die
                  »Reigen\pwindex{Reigen. Zehn Dialoge@\emph{Reigen. Zehn Dialoge}|pw}«-Aufführung geschrieben\pwindex{Reigen@\emph{Reigen}|pw} hat, wäre es vielleicht doch von Wert und Nutzen,
               wenn Sie sich entschliessen könnten, im »Tage-Buch\orgindex{Tage-Buch@Das Tage-Buch|pw}« selbst das Wort zu ergreifen und sich zur öffentlichen Aufführung
               des »\label{T_L02362-1v}\edtext{Reigen\pwindex{Reigen. Zehn Dialoge@\emph{Reigen. Zehn Dialoge}|pw}}{\lemma{\textnormal{\emph{Reigen}}}\Cendnote{\textnormal{geschrieben Reiegn}}}\label{T_L02362-1}« zu äussern.
               Jedenfalls bitte ich Sie, über meine Zeitschrift zu verfügen. Das »Tage-Buch\orgindex{Tage-Buch@Das Tage-Buch|pw}« hat sich in den fünfviertel Jahren
               seines Bestehens in Deutschland\oindex{Deutschland@\textbf{Deutschland}, \emph{A.PCLI}|pw} vollkommen
               durchgesetzt und Sie sprechen durch mein »Tage-Buch\orgindex{Tage-Buch@Das Tage-Buch|pw}« zu dem gebildeten Deutschland\oindex{Deutschland@\textbf{Deutschland}, \emph{A.PCLI}|pw}, das ehedem die »Zukunft\pwindex{Zukunft@\emph{Die Zukunft}|pw}«
               gelesen hat. Ich würde mich freuen und das Gefühl haben, einer gerechten Sache zu
               dienen, wenn Sie sich entschliessen wollten, durch das »Tage-Buch\orgindex{Tage-Buch@Das Tage-Buch|pw}« zu sprechen.\pend
           
\pstart
           Mit herzlichen Grüssen{\\[\baselineskip]} Ihr sehr ergebener{\\[\baselineskip]}\spacefill\mbox{{[}hs.:{]} Stefan Großmann}\pend
           \leftskip=0em{}\selectlanguage{ngerman}\endnumbering\briefempfaengerindex{Schnitzler, Arthur@\textsc{Schnitzler, Arthur}!zzzGrossmann, Stefan@\emph{von Stefan Großmann}!1921-02-101@{10. 2. 1921}|)be}\mylabel{L02362h}  \normalsize

\doendnotes{C}
\bigskip
\vfill

\clearpage

\footnotesize

\lohead{\textsc{register}}

% Definiere theindex-Environment komplett neu ohne reledmac
\makeatletter
\renewenvironment{theindex}{%
  \section*{\indexname}%
  \setlength{\parindent}{0pt}%
  \setlength{\parskip}{0pt plus 0.3pt}%
  \let\item\@idxitem
}{%
  \clearpage
}
\makeatother

\IfFileExists{\jobname-pw.ind}{\input{\jobname-pw.ind}}{}

\end{document}

      