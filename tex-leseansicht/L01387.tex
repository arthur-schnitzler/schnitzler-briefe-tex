\input{../tex-inputs/latex-pdf-vorspann}
\begin{center}
            \textcolor{red}{ENTWURF. ENTZIFFERUNG NOCH NICHT KORREKTURGELESEN}
                      \end{center}
            
               \section[Hugo von Hofmannsthal an Arthur Schnitzler, 21. 3. 1904]{ Hugo von Hofmannsthal an Arthur Schnitzler, 21. 3. 1904}\nopagebreak\mylabel{v}\rehead{ }\begin{ledgroupsized}[t]{13cm}\normalsize\beginnumbering\briefempfaengerindex{Schnitzler, Arthur@\textsc{Schnitzler, Arthur}!zzzHofmannsthal, Hugo von@\emph{von Hugo von Hofmannsthal}!1904-03-211@{21. 3. 1904}|(be} \toendnotes[C]{\smallbreak\pagebreak[2]} \buchAlsQuelle{Hugo von Hofmannsthal, Arthur Schnitzler: \emph{Briefwechsel}. Hg. Therese Nickl und Heinrich Schnitzler. Frankfurt am Main: \emph{S. Fischer} 1964, S. 185.}\toendnotes[C]{\smallbreak}\pstart
           \noindent{}{\pb}{[}Telegramm{]}\hfill {[}21. März 1904{]}\pend
           \pstart
           Mama\pwindex{Hofmannsthal, Anna von 27.01.1849 – 22.03.1904@\textsc{Hofmannsthal, Anna von} (27.01.1849 – 22.03.1904)|pwv} wird heute Abend
                  6 Uhr{ }Sanatorium Fürth\oindex{Sanatorium Fuerth@\textbf{Sanatorium Fürth}|pw} operiert \pend
           \pstart \spacefill\mbox{Hugo}\pend{}\endnumbering\briefempfaengerindex{Schnitzler, Arthur@\textsc{Schnitzler, Arthur}!zzzHofmannsthal, Hugo von@\emph{von Hugo von Hofmannsthal}!1904-03-211@{21. 3. 1904}|)be}\mylabel{h}\end{ledgroupsized}  \newcommand{\dateiname}{L01387}\newcommand{\titel}{Hugo von Hofmannsthal an Arthur Schnitzler, 21. 3. 1904}\newcommand{\editorInnen}{Martin Anton Müller und Gerd-Hermann Susen}\input{../tex-inputs/latex-pdf-abspann}
      