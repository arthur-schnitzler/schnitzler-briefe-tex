%% latex-korrekturansicht-vorspann.tex
%% Vorspann für die Korrekturansicht.
%% Lädt die gemeinsame Datei latex-vorspann.tex mit gesetztem Schalter.

\newif\ifkorrekturansicht
\korrekturansichttrue

\input{../tex-inputs/latex-vorspann}


\section[Richard Beer-Hofmann an Arthur Schnitzler, 5. 6. 1897]{L00683 Richard Beer-Hofmann an Arthur Schnitzler, 5. 6. 1897}
\nopagebreak\mylabel{L00683v}
\rehead{ }\normalsize\beginnumbering\briefempfaengerindex{Schnitzler, Arthur@\textsc{Schnitzler, Arthur}!zzzBeer-Hofmann, Richard@\emph{von Richard Beer-Hofmann}!1897-06-052@{5. 6. 1897}|(be}
\toendnotes[C]{\smallbreak\pagebreak[2]}\Standort{CUL, Schnitzler, B 8.}
\physDesc{Brief, 1 Blatt, 4 Seiten, 462 Zeichen
\newline{}Handschrift: blauer Buntstift, lateinische Kurrent
\newline{}Ordnung: mit Bleistift von unbekannter Hand nummeriert:
                                    »97« }
\buchAbdrucke{\weitereDrucke{Arthur Schnitzler, Richard Beer-Hofmann: \emph{Briefwechsel 1891–1931}. Wien, Zürich: \emph{Europaverlag} 1992, S. 107.} }\toendnotes[C]{\smallbreak}
\pstart
           \centering{}{\pb}Ischl\oindex{Bad Ischl@\textbf{Bad Ischl}, \emph{P.PPL}|pw}{ }5/VI 97\pend
           \vspace{0.5em}
\pstart
           Lieber Arthur, ich bin seit Donnerstag hier und wünsche
               Sie recht bald auch hier zu haben. Wann ko{\geminationm}en Sie
               voraussichtlich, und wie lange {\pb}bleiben Sie hier? Anderes hoffe ich mündlich von Ihnen zu erfahren.\pend
           
\pstart
           Ko{\geminationm}t Goldmann\pwindex{Goldmann, Paul 31.01.1865 – 25.09.1935@\textsc{Goldmann, Paul} (31.01.1865 – 25.09.1935), \emph{Schriftsteller/Schriftstellerin, Journalist/Journalistin}|pw},
               und wann?\pend
           
\pstart
           Hier ist es wunderschön und da ich nur 4 Lectionen {\pb}im Radfahren erst in Wien\oindex{Wien@\textbf{Wien}, \emph{A.ADM2}|pw} geno{\geminationm}en habe, muß
               ich hier weiterlernen. Ich hoffe aber daß Rad u. Radfahren mich nicht hindern werden
               am {\dots}ten Capitel\pwindex{Tod Georgs@\emph{Der Tod Georgs}|pwv}{ }{\pb}zu arbeiten.\pend
           
\pstart
           Herzlichst{\\[\baselineskip]}Ihr \spacefill\mbox{Richard}\pend
           \leftskip=0em{}\selectlanguage{ngerman}\endnumbering\briefempfaengerindex{Schnitzler, Arthur@\textsc{Schnitzler, Arthur}!zzzBeer-Hofmann, Richard@\emph{von Richard Beer-Hofmann}!1897-06-052@{5. 6. 1897}|)be}\mylabel{L00683h}  \normalsize

\doendnotes{C}
\bigskip
\vfill

\clearpage

\footnotesize

\lohead{\textsc{register}}

% Definiere theindex-Environment komplett neu ohne reledmac
\makeatletter
\renewenvironment{theindex}{%
  \section*{\indexname}%
  \setlength{\parindent}{0pt}%
  \setlength{\parskip}{0pt plus 0.3pt}%
  \let\item\@idxitem
}{%
  \clearpage
}
\makeatother

\IfFileExists{\jobname-pw.ind}{\input{\jobname-pw.ind}}{}

\end{document}

      