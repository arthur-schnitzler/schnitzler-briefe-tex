\input{../tex-inputs/latex-pdf-vorspann}
\begin{center}
            \textcolor{red}{ENTWURF. ENTZIFFERUNG NOCH NICHT KORREKTURGELESEN}
                      \end{center}
            
               \section[Richard Beer-Hofmann an Arthur Schnitzler, 5. 6. 1897]{ Richard Beer-Hofmann an Arthur Schnitzler, 5. 6. 1897}\nopagebreak\mylabel{v}\rehead{ }\begin{ledgroupsized}[t]{13cm}\normalsize\beginnumbering\briefempfaengerindex{Schnitzler, Arthur@\textsc{Schnitzler, Arthur}!zzzBeer-Hofmann, Richard@\emph{von Richard Beer-Hofmann}!1897-06-052@{5. 6. 1897}|(be} \toendnotes[C]{\smallbreak\pagebreak[2]} \Standort{CUL, Schnitzler, B 8.}
\physDesc{Brief, 1 Blatt, 4 Seiten
\newline{}Handschrift: blauer Buntstift, lateinische Kurrent\newline{}Ordnung: mit Bleistift von unbekannter Hand nummeriert: »97« }\buchAbdrucke{\weitereDrucke{Arthur Schnitzler, Richard Beer-Hofmann: \emph{Briefwechsel 1891–1931}. Hg. Konstanze Fliedl. Wien, Zürich: \emph{Europaverlag} 1992, S. 107.} }\toendnotes[C]{\smallbreak}\pstart
           \centering{}{\pb}Ischl\oindex{Bad Ischl@\textbf{Bad Ischl}|pw}{ }5/VI 97\pend
           \pstart
           Lieber Arthur, ich bin seit Donnerstag hier und wünsche
               Sie recht bald auch hier zu haben. Wann ko{\geminationm}en Sie voraussichtlich, und wie lange {\pb}bleiben Sie hier? Anderes hoffe ich
               mündlich von Ihnen zu erfahren.\pend
           \pstart
           Ko{\geminationm}t Goldmann\pwindex{Goldmann, Paul 31.01.1865 – 25.09.1935@\textsc{Goldmann, Paul} (31.01.1865 – 25.09.1935), \emph{Schriftsteller, Journalist}|pw}, und
               wann?\pend
           \pstart
           Hier ist es wunderschön und da ich nur 4 Lectionen {\pb}im Radfahren erst in Wien\oindex{Wien@\textbf{Wien}|pw} geno{\geminationm}en habe, muß ich hier
               weiterlernen. Ich hoffe aber daß Rad u. Radfahren mich nicht hindern werden am {\dots}ten Capitel\pwindex{Beer-Hofmann, Richard 11.07.1866 – 26.09.1945@\textsc{Beer-Hofmann, Richard} (11.07.1866 – 26.09.1945), \emph{Schriftsteller}!Tod Georgs1900@\strich\emph{Der Tod Georgs} {[}1900{]}|pwv}{ }{\pb}zu arbeiten.\pend
           \pstart
           Herzlichst{\\[\baselineskip]}Ihr \spacefill\mbox{Richard}\pend
           \leftskip=0em{}\endnumbering\briefempfaengerindex{Schnitzler, Arthur@\textsc{Schnitzler, Arthur}!zzzBeer-Hofmann, Richard@\emph{von Richard Beer-Hofmann}!1897-06-052@{5. 6. 1897}|)be}\mylabel{h}\end{ledgroupsized}  \newcommand{\dateiname}{L00683}\newcommand{\titel}{Richard Beer-Hofmann an Arthur Schnitzler, 5. 6. 1897}\newcommand{\editorInnen}{Martin Anton Müller und Gerd-Hermann Susen}\input{../tex-inputs/latex-pdf-abspann}
      