%% latex-leseansicht-vorspann.tex
%% Vorspann für die Leseansicht.
%% Lädt die gemeinsame Datei latex-vorspann.tex mit nicht gesetztem Schalter.

\newif\ifkorrekturansicht
\korrekturansichtfalse

\input{../tex-inputs/latex-vorspann}


\section[Paul Goldmann an Arthur Schnitzler, 22. 8. {[}1895{]}]{L02746 Paul Goldmann an Arthur Schnitzler, 22. 8. [1895]}
\nopagebreak\mylabel{L02746v}
\rehead{ }\normalsize\beginnumbering\briefempfaengerindex{Schnitzler, Arthur@\textsc{Schnitzler, Arthur}!zzzGoldmann, Paul@\emph{von Paul Goldmann}!1895-08-221@{22. 8. [1895]}|(be}
\toendnotes[C]{\smallbreak\pagebreak[2]}
\correspDesc{Versand  durch Paul Goldmann am 22. 8. [1895] in Bad Tölz
\newline{}Erhalt  durch Arthur Schnitzler im Zeitraum [23. 8. 1895
                  – 27. 8. 1895?] in Salzburg}\toendnotes[C]{\smallbreak}
\Standort{DLA, A:Schnitzler, HS.NZ85.1.3165.}
\physDesc{Brief, 1 Blatt, 4 Seiten, 1910 Zeichen
\newline{}Handschrift: schwarze Tinte, deutsche Kurrent
\newline{}Schnitzler: 1) mit Bleistift das Jahr »95« vermerkt  2) mit rotem Buntstift zwei Unterstreichungen}\toendnotes[C]{\smallbreak}
\pstart
           {\pb}\textcolor{gray}{\textbf{\textbf{Frankfurter Zeitung\orgindex{Frankfurter Zeitung@Frankfurter Zeitung|pw}}}}\pend
           
\pstart
           \textcolor{gray}{\textbf{(\begin{otherlanguage}{french}Gazette de Francfort\end{otherlanguage}\orgindex{Frankfurter Zeitung@Frankfurter Zeitung|pw}).}}\hfill \textsc{Toelz\oindex{Bad Tölz@\textbf{Bad Tölz}, \emph{Hauptstadt}|pw}}, 22. Auguſt.\pend
           
\pstart
           \textcolor{gray}{\textbf{\textbf{\begin{otherlanguage}{french}Fondateur M. L.
                              Sonnemann\pwindex{Sonnemann, Leopold 29.\,10.\,1831 Höchberg – 30.\,10.\,1909 Frankfurt am Main@\textsc{Sonnemann, Leopold} (29.\,10.\,1831 Höchberg – 30.\,10.\,1909 Frankfurt am Main), \emph{Journalist, Herausgeber}|pw}\end{otherlanguage}.}}}\pend
           
\pstart
           \begin{otherlanguage}{french}\textcolor{gray}{\textbf{Journal politique, financier,}}\end{otherlanguage}\pend
           
\pstart
           \begin{otherlanguage}{french}\textcolor{gray}{\textbf{commercial et littéraire.}}\end{otherlanguage}\pend
           
\pstart
           \begin{otherlanguage}{french}\textcolor{gray}{\textbf{\textbf{Paraissant trois fois par jour.}}}\end{otherlanguage}\pend
           
\pstart
           \begin{otherlanguage}{french}\textcolor{gray}{\textbf{\textbf{Bureau à Paris\oindex{Paris@\textbf{Paris}, \emph{Hauptstadt}|pw}}}}\end{otherlanguage}\pend
           
\pstart
           \begin{otherlanguage}{french}\textcolor{gray}{\textbf{\textbf{24. Rue Feydeau\oindex{rue Feydeau@\textbf{rue Feydeau}, \emph{Straße}|pw}.}}}\end{otherlanguage}\pend
           
\pstart\center{}Mein lieber Freund,\pend\vspace{0.5em}
\pstart
           Telegraphire mir jedenſalls, \strikeout{\textcolor{gray}{ob}} wann Du in Tegernſee\oindex{Tegernsee@\textbf{Tegernsee}|pw} eintriffſt u. ob
               ich Dir hier Nachtquartier beſtellen{ }ſoll? Ich möchte Dir{ }ſchon gern entgegenkommen
               u. es lag auch ohne Deine Anregung in meiner Abſicht. Nun habe ich aber{ }ſeit einigen
               Tagen als Folge der Kur einen{ }ſo{ }ſchrecklichen \label{K_L02746-1v}\edtext{Magen-Katarrh}{\lemma{\textnormal{\emph{Magen-Katarrh}}}\Cendnote{\textnormal{Entzündung der Magenschleimhaut}}}\label{K_L02746-1}, daß ich kaum kriechen kann. Außerdem habe
               ich in Tegernſee\oindex{Tegernsee@\textbf{Tegernsee}|pw} Verwandte,{ }ſo daß mir ein
               anderer Rendezvous-Ort lieber wäre. Wie wäre es denn mit \textsc{Schliersee\oindex{Schliersee@\textbf{Schliersee}|pw}}? Dort {\pb}ſpielt am Sonntag{ }Abend das \label{K_L02746-2v}\edtext{Bauern-Theater\orgindex{Schlierseer Bauerntheater@Schlierseer Bauerntheater|pw}}{\lemma{\textnormal{\emph{Bauern-Theater}}}\Cendnote{\textnormal{Das 1892 gegründete Theater\orgindex{Schlierseer Bauerntheater@Schlierseer Bauerntheater|pwkv} war ein von
                  ehemaligen Handwerkern betriebenes Unternehmen\orgindex{Schlierseer Bauerntheater@Schlierseer Bauerntheater|pwkv}, das durch Tourneen weithin berühmt
                  war.}}}\label{K_L02746-2}, was{ }ſehr intereſſant{ }ſein{ }ſoll. Liegt das nicht auch auf \label{K_L02746-3v}\edtext{Eurer\pwindex{Salten, Felix 6.\,9.\,1869 Budapest – 8.\,10.\,1945 Zürich@\textsc{Salten, Felix} (6.\,9.\,1869 Budapest – 8.\,10.\,1945 Zürich), \emph{Schriftsteller, Journalist, Chefredakteur}|pwv}}{\lemma{\textnormal{\emph{Eurer}}}\Cendnote{\textnormal{Schnitzler wurde von Felix Salten\pwindex{Salten, Felix 6.\,9.\,1869 Budapest – 8.\,10.\,1945 Zürich@\textsc{Salten, Felix} (6.\,9.\,1869 Budapest – 8.\,10.\,1945 Zürich), \emph{Schriftsteller, Journalist, Chefredakteur}|pwk} begleitet.}}}\label{K_L02746-3} Route? Übrigens, wie Du
               willſt. Du beſtimmſt, und wenn ich irgend mich bewegen kann, komme ich hin. Wenn
               nicht, erwarte ich Dich in \textsc{Toelz\oindex{Bad Tölz@\textbf{Bad Tölz}, \emph{Hauptstadt}|pw}}.\pend
           
\pstart
           Auch anderes Ärgerniß gibt es inzwiſchen. Ich fürchte, ich werde nur wenige Tage mit
               Euch zuſammenſein können. Familien-Pflichten! Meinem Onkel\pwindex{Mamroth, Fedor 21.\,2.\,1851 Breslau – 25.\,6.\,1907 Frankfurt am Main@\textsc{Mamroth, Fedor} (21.\,2.\,1851 Breslau – 25.\,6.\,1907 Frankfurt am Main), \emph{Journalist, Kritiker}|pwv} fällt es jetzt plötzlich ein, ich
               müßte \uline{mich} mit ihm in der Schweiz\oindex{Schweiz@\textbf{Schweiz}|pw} treffen. Mein Schwager\pwindex{Rosengart, Josef 8.\,2.\,1860 Laupheim – 4.\,8.\,1927 Frankfurt am Main@\textsc{Rosengart, Josef} (8.\,2.\,1860 Laupheim – 4.\,8.\,1927 Frankfurt am Main), \emph{Arzt}|pwv} will nach \textsc{Muenchen\oindex{München@\textbf{München}|pw}} kommen und mich mit{ }ſich fort nach der {\pb}Schweiz\oindex{Schweiz@\textbf{Schweiz}|pw} nehmen. Es iſt allerlei Wichtiges in
               Familien-Dingen zu erörtern. Ich erkläre Dir das Nähere mündlich. Würdeſt Du
               eventuell auf ein paar Tage \label{K_L02746-4v}\edtext{mit nach
               der Schweiz\oindex{Schweiz@\textbf{Schweiz}|pw}}{\lemma{\textnormal{\emph{mit nach
               der Schweiz}}}\Cendnote{\textnormal{nicht umgesetzt}}}\label{K_L02746-4} kommen?\pend
           
\pstart
           Wirklich, diesmal geht Alles{ }ſchief. Es iſt ekelhaft.\pend
           
\pstart
           Ich erhalte{ }ſoeben die \label{K_L02746-5v}\edtext{»Freie Bühne\pwindex{Freie Bühne für den Entwickelungskampf der Zeit@\emph{Freie Bühne für den Entwickelungskampf der Zeit}|pw}« mit der \strikeout{»E\textcolor{gray}{a}} »kleinen Komödie\pwindex{Schnitzler, Arthur 15.\,5.\,1862 Wien – 21.\,10.\,1931 ebd.@\textsc{Schnitzler, Arthur} (15.\,5.\,1862 Wien – 21.\,10.\,1931 ebd.), \emph{Schriftsteller, Mediziner}!kleine Komödie@\strich\emph{Die kleine Komödie}|pw}«}{\lemma{\textnormal{\emph{»Freie … Komödie«}}}\Cendnote{\textnormal{Arthur Schnitzler: \emph{Die kleine Komödie}\pwindex{Schnitzler, Arthur 15.\,5.\,1862 Wien – 21.\,10.\,1931 ebd.@\textsc{Schnitzler, Arthur} (15.\,5.\,1862 Wien – 21.\,10.\,1931 ebd.), \emph{Schriftsteller, Mediziner}!kleine Komödie@\strich\emph{Die kleine Komödie}|pwk}. In: \emph{Neue Deutsche Rundschau}\pwindex{Neue Deutsche Rundschau@\emph{Neue Deutsche Rundschau}|pwk}, Jg. 6, H. 8, 1. 8. 1895,
                     S. 779–798. (Die \emph{Neue Deutsche
                     Rundschau}\pwindex{Neue Deutsche Rundschau@\emph{Neue Deutsche Rundschau}|pwk} war als \emph{Freie Bühne}\pwindex{Freie Bühne für modernes Leben@\emph{Freie Bühne für modernes Leben}|pwk}
                  gegründet, aber nach vier Jahrgängen umbenannt worden.)}}}\label{K_L02746-5}. Es{ }ſind
               glänzende Sachen darin, und beſonders gelungen{ }ſind die Anfangsbriefe\pwindex{Schnitzler, Arthur 15.\,5.\,1862 Wien – 21.\,10.\,1931 ebd.@\textsc{Schnitzler, Arthur} (15.\,5.\,1862 Wien – 21.\,10.\,1931 ebd.), \emph{Schriftsteller, Mediziner}!kleine Komödie@\strich\emph{Die kleine Komödie}|pwv}, welche die beiderſeitigen
                  \label{K_L02746-6v}\edtext{\begin{otherlanguage}{french}\textsc{états d’âme}\end{otherlanguage}}{\lemma{\textnormal{\emph{états d’âme}}}\Cendnote{\textnormal{französisch: Seelenstände (die deutsche
                  Begriffsprägung stammt von Hermann
                  Bahr\pwindex{Bahr, Hermann 19.\,7.\,1863 Linz – 15.\,1.\,1934 München@\textsc{Bahr, Hermann} (19.\,7.\,1863 Linz – 15.\,1.\,1934 München), \emph{Schriftsteller, Kritiker}|pwk})}}}\label{K_L02746-6} auseinanderſetzen. Aber im Ganzen {\pb}\strikeout{mag ich es} mag ich es nicht{ }ſehr. Es iſt gar zu
               erzwungen und zu gekünſtelt in{ }ſeinen thatſächlichen Vorausſetzungen. Auch fehlt mir
               das einfach und tief Menſchliche, das ich an Deinen{ }ſonſtigen Arbeiten{ }ſo liebe. Aber
               auch bei dieſer weniger gelungenen Arbeit\pwindex{Schnitzler, Arthur 15.\,5.\,1862 Wien – 21.\,10.\,1931 ebd.@\textsc{Schnitzler, Arthur} (15.\,5.\,1862 Wien – 21.\,10.\,1931 ebd.), \emph{Schriftsteller, Mediziner}!kleine Komödie@\strich\emph{Die kleine Komödie}|pwv} iſt Eines zu bemerken: die ungemeine Sicherheit der Schreibweiſe, –{ }ſo, was beim Maler die feſte Hand iſt, welche die künſtleriſche Reife mit{ }ſich
                  bringt{\dotstwo}\textcolor{gray}{{\dotstwo}}\pend
           
\pstart
           Viele treue Grüße an Euch Alle! {\\[\baselineskip]}Dein {\\[\baselineskip]}\spacefill\mbox{Paul Goldmann}\pend
           \leftskip=0em{}\selectlanguage{ngerman}\endnumbering\briefempfaengerindex{Schnitzler, Arthur@\textsc{Schnitzler, Arthur}!zzzGoldmann, Paul@\emph{von Paul Goldmann}!1895-08-221@{22. 8. [1895]}|)be}\mylabel{L02746h}  \newcommand{\dateiname}{L02746}\newcommand{\titel}{Paul Goldmann an Arthur Schnitzler, 22. 8. [1895]}\newcommand{\editorInnen}{Martin Anton Müller und Laura Untner}%% latex-leseansicht-abspann.tex
%% Abspann für die Leseansicht.
%% Der Schalter \ifkorrekturansicht ist bereits durch den Vorspann gesetzt.

%% latex-abspann.tex
%% Gemeinsamer Abspann für Korrekturansicht und Leseansicht.
%% Setzt den Schalter \ifkorrekturansicht voraus (gesetzt in den
%% einbindenden Dateien latex-korrekturansicht-abspann.tex bzw.
%% latex-leseansicht-abspann.tex).
%% ---------------------------------------------------------------

\normalsize

% Das esempio-Environment wird nur in der Leseansicht benötigt
\ifkorrekturansicht\else
\newenvironment{esempio}[3]%
{
    \vspace{1.5ex}
    \rlap{\underline{#1}}
    \par
    \setlength{\parindent}{0cm}
    \nopagebreak
    \leftskip=#2cm
    \rightskip=#3cm
}
{
    \par
}
\fi

\doendnotes{C}
\bigskip
\vfill

\clearpage

\footnotesize

\ifkorrekturansicht
  \lohead{\textsc{register}}
\fi

% theindex-Environment neu definieren ohne reledmac
\makeatletter
\renewenvironment{theindex}{%
  \ifkorrekturansicht
    \section*{\indexname}%
  \else
    \subsubsection*{Index der erwähnten Entitäten}%
  \fi
  \setlength{\parindent}{0pt}%
  \setlength{\parskip}{0pt plus 0.3pt}%
  \let\item\@idxitem
}{%
  \ifkorrekturansicht\clearpage\fi
}
\makeatother

\IfFileExists{\jobname-pw.ind}{\input{\jobname-pw.ind}}{}

% Quellenangabe nur in der Leseansicht
\ifkorrekturansicht\else
% Fallback-Definitionen, falls die .tex-Datei \titel etc. nicht gesetzt hat
\providecommand{\titel}{}
\providecommand{\editorInnen}{}
\providecommand{\dateiname}{\jobname}

\vspace{3cm}

\vfill

\footnotesize
\textsc{Quelle}: \titel. Herausgegeben von {\editorInnen}. In: \emph{Arthur Schnitzler: Briefwechsel mit Autorinnen und Autoren}.
 Digitale Edition, https://schnitzler-briefe.acdh.oeaw.ac.at/{\dateiname}.html (Stand \today)
\fi

\end{document}


