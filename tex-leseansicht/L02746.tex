%% latex-leseansicht-vorspann.tex
%% Vorspann für die Leseansicht.
%% Lädt die gemeinsame Datei latex-vorspann.tex mit nicht gesetztem Schalter.

\newif\ifkorrekturansicht
\korrekturansichtfalse

\input{../tex-inputs/latex-vorspann}


         
         \newcommand{\erwaehntePersonen}{Personen: Hermann Bahr, Fedor Mamroth, Josef Rosengart, Felix Salten, Leopold Sonnemann}
         \newcommand{\erwaehnteInstitutionen}{Institutionen: Frankfurter Zeitung, Schlierseer Bauerntheater}
         \newcommand{\erwaehnteOrte}{Orte: Bad Tölz, München, Paris, Salzburg, Schliersee, Schweiz, Tegernsee, rue Feydeau}
         \newcommand{\erwaehnteWerke}{Werke: Die kleine Komödie, Freie Bühne für den Entwickelungskampf der Zeit, Freie Bühne für modernes Leben, Neue Deutsche Rundschau}
               \section[Paul Goldmann an Arthur Schnitzler, 22. 8. {[}1895{]}]{ Paul Goldmann an Arthur Schnitzler, 22. 8. {[}1895{]}}\nopagebreak\mylabel{v}\rehead{ }\begin{ledgroupsized}[t]{13cm}\normalsize\beginnumbering \toendnotes[C]{\smallbreak\pagebreak[2]} \Standort{DLA, A:Schnitzler, HS.NZ85.1.3165.}
\physDesc{Brief, 1 Blatt, 4 Seiten
\newline{}Handschrift: schwarze Tinte, deutsche Kurrent
\newline{}Schnitzler: 1) mit Bleistift das Jahr »95« vermerkt  2) mit rotem Buntstift zwei Unterstreichungen}\toendnotes[C]{\smallbreak}\pstart
           \noindent{}{\pb}\textcolor{gray}{\textbf{\textbf{Frankfurter Zeitung\orgindex{Frankfurter Zeitung@Frankfurter Zeitung|pw}}}}\pend
           \pstart
           \textcolor{gray}{\textbf{(\begin{otherlanguage}{french}Gazette de Francfort\end{otherlanguage}\orgindex{Frankfurter Zeitung@Frankfurter Zeitung|pw}).}}\hfill \textsc{Toelz\oindex{Bad Toelz@\textbf{Bad Tölz}|pw}}, 22. Auguſt.\pend
           \pstart
           \textcolor{gray}{\textbf{\textbf{\begin{otherlanguage}{french}Fondateur M. L.
                              Sonnemann\pwindex{Sonnemann, Leopold 1831-10-29 – 1909-10-30@\textsc{Sonnemann, Leopold} (1831-10-29 – 1909-10-30), \emph{Journalist, Herausgeber}|pw}\end{otherlanguage}.}}}\pend
           \pstart
           \begin{otherlanguage}{french}\textcolor{gray}{\textbf{Journal politique, financier,}}\end{otherlanguage}\pend
           \pstart
           \begin{otherlanguage}{french}\textcolor{gray}{\textbf{commercial et littéraire.}}\end{otherlanguage}\pend
           \pstart
           \begin{otherlanguage}{french}\textcolor{gray}{\textbf{\textbf{Paraissant trois fois par jour.}}}\end{otherlanguage}\pend
           \pstart
           \begin{otherlanguage}{french}\textcolor{gray}{\textbf{\textbf{Bureau à Paris\oindex{Paris@\textbf{Paris}|pw}}}}\end{otherlanguage}\pend
           \pstart
           \begin{otherlanguage}{french}\textcolor{gray}{\textbf{\textbf{24. Rue Feydeau\oindex{rue Feydeau@\textbf{rue Feydeau}|pw}.}}}\end{otherlanguage}\pend
           \pstart\center{}Mein lieber Freund,\pend\pstart
           Telegraphire mir jedenſalls, \strikeout{\textcolor{gray}{ob}}
               wann Du in Tegernſee\oindex{Tegernsee@\textbf{Tegernsee}|pw} eintriffſt u. ob ich Dir
               hier Nachtquartier beſtellen ſoll? Ich möchte Dir ſchon gern entgegenkommen u. es lag
               auch ohne Deine Anregung in meiner Abſicht. Nun habe ich aber ſeit einigen Tagen als
               Folge der Kur einen ſo ſchrecklichen \label{K_L02746-1v}\edtext{Magen-Katarrh}{\lemma{\textnormal{\emph{Magen-Katarrh}}}\Cendnote{\textnormal{Entzündung der
                  Magenschleimhaut}}}\label{K_L02746-1h}, daß ich kaum kriechen kann. Außerdem habe ich in Tegernſee\oindex{Tegernsee@\textbf{Tegernsee}|pw} Verwandte, ſo daß mir ein anderer
               Rendezvous-Ort lieber wäre. Wie wäre es denn mit \textsc{Schliersee\oindex{Schliersee@\textbf{Schliersee}|pw}}? Dort {\pb}ſpielt am Sonntag{ }Abend das \label{K_L02746-55v}\edtext{Bauern-Theater\orgindex{Schlierseer Bauerntheater@Schlierseer Bauerntheater|pw}}{\lemma{\textnormal{\emph{Bauern-Theater}}}\Cendnote{\textnormal{Das 1892 gegründete Theater\orgindex{Schlierseer Bauerntheater@Schlierseer Bauerntheater|pwkv} war ein von
                  ehemaligen Handwerkern betriebenes Unternehmen\orgindex{Schlierseer Bauerntheater@Schlierseer Bauerntheater|pwkv}, das durch Tourneen weithin berühmt
                  war.}}}\label{K_L02746-55h}, was ſehr intereſſant ſein ſoll. Liegt das nicht auch auf \label{K_L02746-88v}\edtext{Eurer\pwindex{Salten, Felix 06.09.1869 – 08.10.1945@\textsc{Salten, Felix} (06.09.1869 – 08.10.1945), \emph{Schriftsteller, Journalist}|pwv}}{\lemma{\textnormal{\emph{Eurer}}}\Cendnote{\textnormal{Schnitzler\pwindex{Schnitzler, Arthur 15.05.1862 – 21.10.1931@\textsc{Schnitzler, Arthur} (15.05.1862 – 21.10.1931), \emph{Schriftsteller, Mediziner}|pwk} wurde von Felix Salten\pwindex{Salten, Felix 06.09.1869 – 08.10.1945@\textsc{Salten, Felix} (06.09.1869 – 08.10.1945), \emph{Schriftsteller, Journalist}|pwk} begleitet.}}}\label{K_L02746-88h} Route? Übrigens, wie Du
               willſt. Du beſtimmſt, und wenn ich irgend mich bewegen kann, komme ich hin. Wenn
               nicht, erwarte ich Dich in \textsc{Toelz\oindex{Bad Toelz@\textbf{Bad Tölz}|pw}}.\pend
           \pstart
           Auch anderes Ärgerniß gibt es inzwiſchen. Ich fürchte, ich werde nur wenige Tage mit
               Euch zuſammenſein können. Familien-Pflichten! Meinem Onkel\pwindex{Mamroth, Fedor 21.02.1851 – 25.06.1907@\textsc{Mamroth, Fedor} (21.02.1851 – 25.06.1907), \emph{Journalist, Kritiker}|pwv} fällt es jetzt plötzlich ein, ich
               müßte \uline{mich} mit ihm in der Schweiz\oindex{Schweiz@\textbf{Schweiz}|pw} treffen. Mein Schwager\pwindex{Rosengart, Josef 1860-02-08 – 1927-08-04@\textsc{Rosengart, Josef} (1860-02-08 – 1927-08-04), \emph{Arzt}|pwv} will nach \textsc{Muenchen\oindex{Muenchen@\textbf{München}|pw}} kommen und mich mit ſich fort nach der {\pb}Schweiz\oindex{Schweiz@\textbf{Schweiz}|pw} nehmen. Es iſt allerlei Wichtiges in
               Familien-Dingen zu erörtern. Ich erkläre Dir das Nähere mündlich. Würdeſt Du
               eventuell auf ein paar Tage \label{K_L02746-2v}\edtext{mit nach
               der Schweiz\oindex{Schweiz@\textbf{Schweiz}|pw}}{\lemma{\textnormal{\emph{mit nach
               der Schweiz}}}\Cendnote{\textnormal{nicht umgesetzt}}}\label{K_L02746-2h} kommen?\pend
           \pstart
           Wirklich, diesmal geht Alles ſchief. Es iſt ekelhaft.\pend
           \pstart
           Ich erhalte ſoeben die \label{K_L02746-3v}\edtext{»Freie Bühne\pwindex{Freie Buehne fuer den Entwickelungskampf der Zeit1892 – 1893@\emph{Freie Bühne für den Entwickelungskampf der Zeit} {[}1892 – 1893{]}|pw}« mit der \strikeout{»E\textcolor{gray}{a}} »kleinen Komödie\pwindex{Schnitzler, Arthur 15.05.1862 – 21.10.1931@\textsc{Schnitzler, Arthur} (15.05.1862 – 21.10.1931), \emph{Schriftsteller, Mediziner}!kleine Komoedie1895-08-01@\strich\emph{Die kleine Komödie} {[}1895-08-01{]}|pw}«}{\lemma{\textnormal{\emph{»Freie … Komödie«}}}\Cendnote{\textnormal{Arthur Schnitzler\pwindex{Schnitzler, Arthur 15.05.1862 – 21.10.1931@\textsc{Schnitzler, Arthur} (15.05.1862 – 21.10.1931), \emph{Schriftsteller, Mediziner}|pwk}: \emph{Die kleine Komödie}\pwindex{Schnitzler, Arthur 15.05.1862 – 21.10.1931@\textsc{Schnitzler, Arthur} (15.05.1862 – 21.10.1931), \emph{Schriftsteller, Mediziner}!kleine Komoedie1895-08-01@\strich\emph{Die kleine Komödie} {[}1895-08-01{]}|pwk}. In: \emph{Neue Deutsche Rundschau}\pwindex{Neue Deutsche Rundschau1894-01-01 – 1903-12-31@\emph{Neue Deutsche Rundschau} {[}1894-01-01 – 1903-12-31{]}|pwk}, Jg. 6, H. 8, 1. 8. 1895,
                     S. 779–798. (Die \emph{Neue Deutsche
                     Rundschau}\pwindex{Neue Deutsche Rundschau1894-01-01 – 1903-12-31@\emph{Neue Deutsche Rundschau} {[}1894-01-01 – 1903-12-31{]}|pwk} wurde als \emph{Freie Bühne}\pwindex{Freie Buehne fuer modernes Leben1890 – 1891@\emph{Freie Bühne für modernes Leben} {[}1890 – 1891{]}|pwk}
                  gegründet, war aber nach vier Jahrgängen umbenannt worden.)}}}\label{K_L02746-3h}. Es ſind
               glänzende Sachen darin, und beſonders gelungen ſind die Anfangsbriefe\pwindex{Schnitzler, Arthur 15.05.1862 – 21.10.1931@\textsc{Schnitzler, Arthur} (15.05.1862 – 21.10.1931), \emph{Schriftsteller, Mediziner}!kleine Komoedie1895-08-01@\strich\emph{Die kleine Komödie} {[}1895-08-01{]}|pwv}, welche die beiderſeitigen
                  \label{K_L02746-4v}\edtext{\begin{otherlanguage}{french}\textsc{états d’âme}\end{otherlanguage}}{\lemma{\textnormal{\emph{états d’âme}}}\Cendnote{\textnormal{französisch: Seelenstände (die deutsche
                  Begriffsprägung stammt von Hermann
                  Bahr\pwindex{Bahr, Hermann 19.07.1863 – 15.01.1934@\textsc{Bahr, Hermann} (19.07.1863 – 15.01.1934), \emph{Schriftsteller, Kritiker}|pwk})}}}\label{K_L02746-4h} auseinanderſetzen. Aber im Ganzen {\pb}\strikeout{mag ich es} mag ich es nicht ſehr. Es iſt gar zu
               erzwungen und zu gekünſtelt in ſeinen thatſächlichen Vorausſetzungen. Auch fehlt mir
               das einfach und tief Menſchliche, das ich an Deinen ſonſtigen Arbeiten ſo liebe. Aber
               auch bei dieſer weniger gelungenen Arbeit\pwindex{Schnitzler, Arthur 15.05.1862 – 21.10.1931@\textsc{Schnitzler, Arthur} (15.05.1862 – 21.10.1931), \emph{Schriftsteller, Mediziner}!kleine Komoedie1895-08-01@\strich\emph{Die kleine Komödie} {[}1895-08-01{]}|pwv} iſt Eines zu bemerken: die ungemeine Sicherheit der Schreibweiſe, –
               ſo, was beim Maler die feſte Hand iſt, welche die künſtleriſche Reife mit ſich
                  bringt{\dotstwo}\textcolor{gray}{{\dotstwo}}\pend
           \pstart
           Viele treue Grüße an Euch Alle! {\\[\baselineskip]}Dein {\\[\baselineskip]}\spacefill\mbox{Paul Goldmann}\pend
           \leftskip=0em{}
         
         \endnumbering\mylabel{h}\end{ledgroupsized}  \newcommand{\dateiname}{L02746}\newcommand{\titel}{Paul Goldmann an Arthur Schnitzler, 22. 8. [1895]}\newcommand{\editorInnen}{Martin Anton Müller und Laura Untner}%% latex-leseansicht-abspann.tex
%% Abspann für die Leseansicht.
%% Der Schalter \ifkorrekturansicht ist bereits durch den Vorspann gesetzt.

%% latex-abspann.tex
%% Gemeinsamer Abspann für Korrekturansicht und Leseansicht.
%% Setzt den Schalter \ifkorrekturansicht voraus (gesetzt in den
%% einbindenden Dateien latex-korrekturansicht-abspann.tex bzw.
%% latex-leseansicht-abspann.tex).
%% ---------------------------------------------------------------

\normalsize

% Das esempio-Environment wird nur in der Leseansicht benötigt
\ifkorrekturansicht\else
\newenvironment{esempio}[3]%
{
    \vspace{1.5ex}
    \rlap{\underline{#1}}
    \par
    \setlength{\parindent}{0cm}
    \nopagebreak
    \leftskip=#2cm
    \rightskip=#3cm
}
{
    \par
}
\fi

\doendnotes{C}
\bigskip
\vfill

\clearpage

\footnotesize

\ifkorrekturansicht
  \lohead{\textsc{register}}
\fi

% theindex-Environment neu definieren ohne reledmac
\makeatletter
\renewenvironment{theindex}{%
  \ifkorrekturansicht
    \section*{\indexname}%
  \else
    \subsubsection*{Index der erwähnten Entitäten}%
  \fi
  \setlength{\parindent}{0pt}%
  \setlength{\parskip}{0pt plus 0.3pt}%
  \let\item\@idxitem
}{%
  \ifkorrekturansicht\clearpage\fi
}
\makeatother

\IfFileExists{\jobname-pw.ind}{\input{\jobname-pw.ind}}{}

% Quellenangabe nur in der Leseansicht
\ifkorrekturansicht\else
% Fallback-Definitionen, falls die .tex-Datei \titel etc. nicht gesetzt hat
\providecommand{\titel}{}
\providecommand{\editorInnen}{}
\providecommand{\dateiname}{\jobname}

\vspace{3cm}

\vfill

\footnotesize
\textsc{Quelle}: \titel. Herausgegeben von {\editorInnen}. In: \emph{Arthur Schnitzler: Briefwechsel mit Autorinnen und Autoren}.
 Digitale Edition, https://schnitzler-briefe.acdh.oeaw.ac.at/{\dateiname}.html (Stand \today)
\fi

\end{document}


      