%% latex-leseansicht-vorspann.tex
%% Vorspann für die Leseansicht.
%% Lädt die gemeinsame Datei latex-vorspann.tex mit nicht gesetztem Schalter.

\newif\ifkorrekturansicht
\korrekturansichtfalse

\input{../tex-inputs/latex-vorspann}


\section[Berta Zuckerkandl an Arthur Schnitzler, {{[}}9. oder 10. 9. 1911?{{]}}]{L03999 Berta Zuckerkandl an Arthur Schnitzler, {[}9. oder 10. 9. 1911?{]}}
\nopagebreak\mylabel{L03999v}
\rehead{ }\normalsize\beginnumbering\briefempfaengerindex{Schnitzler, Arthur@\textsc{Schnitzler, Arthur}!zzzZuckerkandl, Berta@\emph{von Berta Zuckerkandl}!1911-09-101@{{[}9. oder 10. 9. 1911?{]}}|(be}
\toendnotes[C]{\smallbreak\pagebreak[2]}
\correspDesc{Versand  durch Berta Zuckerkandl im Zeitraum [9. oder
                  10. 9. 1911?] in Wien
\newline{}Erhalt  durch Arthur Schnitzler in Wien}\toendnotes[C]{\smallbreak}
\Standort{CUL, Schnitzler, B 200.}
\physDesc{Brief, 1 Blatt, 2 Seiten, 263 Zeichen
\newline{}Handschrift: schwarze Tinte, lateinische Kurrent
\newline{}Schnitzler: beschriftet: »Zuckerkandl« }\toendnotes[C]{\smallbreak}
\pstart{}{\pb}Hochverehrter Herr Doktor!\pend\vspace{0.5em}
\pstart
           Darf ich Ihnen mein \label{K_L03999-1v}\edtext{innigstes
                  Mitfühlen}{\lemma{\textnormal{\emph{innigstes
                  Mitfühlen}}}\Cendnote{\textnormal{Schnitzlers Mutter Louise Schnitzer geb. Markbreiter\pwindex{Schnitzler, Louise 8.\,7.\,1840 Kőszeg – 9.\,9.\,1911 Wien@\textsc{Schnitzler, Louise} (8.\,7.\,1840 Kőszeg – 9.\,9.\,1911 Wien)|pwk} war am 9. 9. 1911
                  nachmittags gegen 15.30 h gestorben. Die Morgenausgabe der Zeitung\pwindex{Neue Freie Presse@\emph{Neue Freie Presse}|pwkv} vom
                     10. 9. 1911 brachte die Todesanzeige (\emph{Neue Freie Presse}\pwindex{Neue Freie Presse@\emph{Neue Freie Presse}|pwk}, Nr. 16.901,
                        10. 9. 1911, Morgenblatt, S. 30), das Begräbnis fand am
                     11. 9. 1911
                  statt. Der undatierte Kondolenzbrief hat den Charakter einer spontane Beileidsbekundung,
                  nicht den eines ausformulierten Erinnerungsbriefes oder eines Ersatz für die Teilnahme
                  am Begräbnis. Er dürfte daher am Todestag selbst oder am ersten Tag danach
                  verfasst worden sein.}}}\label{K_L03999-1} sagen? Ich habe ja Ihre Frau Mama\pwindex{Schnitzler, Louise 8.\,7.\,1840 Kőszeg – 9.\,9.\,1911 Wien@\textsc{Schnitzler, Louise} (8.\,7.\,1840 Kőszeg – 9.\,9.\,1911 Wien)|pwv} so gut gekannt. Und weiss – wie
               glücklich {\pb}und stolz gerade durch \uline{Ihre} Liebe – sie immer war. – So ein Bewusstsein macht
               Alles noch trauriger.\pend
           
\pstart
           In Verehrung {\\[\baselineskip]}\spacefill\mbox{Berta Zuckerkandl}\pend
           \leftskip=0em{}\selectlanguage{ngerman}\endnumbering\briefempfaengerindex{Schnitzler, Arthur@\textsc{Schnitzler, Arthur}!zzzZuckerkandl, Berta@\emph{von Berta Zuckerkandl}!1911-09-091@{{[}9. oder 10. 9. 1911?{]}}|)be}\mylabel{L03999h}
\begin{anhang}
\end{anhang}\newcommand{\dateiname}{L03999}\newcommand{\titel}{Berta Zuckerkandl an Arthur Schnitzler, [9. oder 10. 9. 1911?]}\newcommand{\editorInnen}{Herausgegeben von Jahnke, SelmaMüller, Martin Anton}%% latex-leseansicht-abspann.tex
%% Abspann für die Leseansicht.
%% Der Schalter \ifkorrekturansicht ist bereits durch den Vorspann gesetzt.

%% latex-abspann.tex
%% Gemeinsamer Abspann für Korrekturansicht und Leseansicht.
%% Setzt den Schalter \ifkorrekturansicht voraus (gesetzt in den
%% einbindenden Dateien latex-korrekturansicht-abspann.tex bzw.
%% latex-leseansicht-abspann.tex).
%% ---------------------------------------------------------------

\normalsize

% Das esempio-Environment wird nur in der Leseansicht benötigt
\ifkorrekturansicht\else
\newenvironment{esempio}[3]%
{
    \vspace{1.5ex}
    \rlap{\underline{#1}}
    \par
    \setlength{\parindent}{0cm}
    \nopagebreak
    \leftskip=#2cm
    \rightskip=#3cm
}
{
    \par
}
\fi

\doendnotes{C}
\bigskip
\vfill

\clearpage

\footnotesize

\ifkorrekturansicht
  \lohead{\textsc{register}}
\fi

% theindex-Environment neu definieren ohne reledmac
\makeatletter
\renewenvironment{theindex}{%
  \ifkorrekturansicht
    \section*{\indexname}%
  \else
    \subsubsection*{Index der erwähnten Entitäten}%
  \fi
  \setlength{\parindent}{0pt}%
  \setlength{\parskip}{0pt plus 0.3pt}%
  \let\item\@idxitem
}{%
  \ifkorrekturansicht\clearpage\fi
}
\makeatother

\IfFileExists{\jobname-pw.ind}{\input{\jobname-pw.ind}}{}

% Quellenangabe nur in der Leseansicht
\ifkorrekturansicht\else
% Fallback-Definitionen, falls die .tex-Datei \titel etc. nicht gesetzt hat
\providecommand{\titel}{}
\providecommand{\editorInnen}{}
\providecommand{\dateiname}{\jobname}

\vspace{3cm}

\vfill

\footnotesize
\textsc{Quelle}: \titel. Herausgegeben von {\editorInnen}. In: \emph{Arthur Schnitzler: Briefwechsel mit Autorinnen und Autoren}.
 Digitale Edition, https://schnitzler-briefe.acdh.oeaw.ac.at/{\dateiname}.html (Stand \today)
\fi

\end{document}


