%% latex-leseansicht-vorspann.tex
%% Vorspann für die Leseansicht.
%% Lädt die gemeinsame Datei latex-vorspann.tex mit nicht gesetztem Schalter.

\newif\ifkorrekturansicht
\korrekturansichtfalse

\input{../tex-inputs/latex-vorspann}


\section[ Felix Salten an Arthur Schnitzler, 30. 4. 1899]{L03289 Felix Salten an Arthur Schnitzler,  30. 4. 1899}
\nopagebreak\mylabel{L03289v}
\rehead{ }\normalsize\beginnumbering\briefempfaengerindex{Schnitzler, Arthur@\textsc{Schnitzler, Arthur}!zzzSalten, Ottilie@\emph{von Ottilie Salten}!1899-04-303@{30. 4. 1899}|(be}\briefempfaengerindex{Schnitzler, Arthur@\textsc{Schnitzler, Arthur}!zzzPetersen, Elly@\emph{von Elly Petersen}!1899-04-303@{30. 4. 1899}|(be}\briefempfaengerindex{Schnitzler, Arthur@\textsc{Schnitzler, Arthur}!zzzGold, Alfred@\emph{von Alfred Gold}!1899-04-303@{30. 4. 1899}|(be}\briefempfaengerindex{Schnitzler, Arthur@\textsc{Schnitzler, Arthur}!zzzAltenberg, Peter@\emph{von Peter Altenberg}!1899-04-303@{30. 4. 1899}|(be}\briefempfaengerindex{Schnitzler, Arthur@\textsc{Schnitzler, Arthur}!zzzWassermann, Jakob@\emph{von Jakob Wassermann}!1899-04-303@{30. 4. 1899}|(be}\briefempfaengerindex{Schnitzler, Arthur@\textsc{Schnitzler, Arthur}!zzzHirschfeld, Georg@\emph{von Georg Hirschfeld}!1899-04-303@{30. 4. 1899}|(be}\briefempfaengerindex{Schnitzler, Arthur@\textsc{Schnitzler, Arthur}!zzzSalten, Felix@\emph{von Felix Salten}!1899-04-303@{30. 4. 1899}|(be}
\toendnotes[C]{\smallbreak\pagebreak[2]}
\correspDesc{Versand  durch Felix Salten, Georg Hirschfeld, Jakob Wassermann, Peter Altenberg, Alfred Gold, Elly Hirschfeld, Ottilie Metzl am 30. 4. 1899 in Wien
\newline{}Erhalt  durch Arthur Schnitzler am 1. 5. 1899 in Berlin}\toendnotes[C]{\smallbreak}
\Standort{CUL, Schnitzler, B 89, A 2.}
\physDesc{Postkarte, 260 Zeichen
\newline{}Handschrift Felix Salten: Bleistift, lateinische Kurrent
\newline{}Handschrift Georg Hirschfeld: Bleistift, deutsche Kurrent
\newline{}Handschrift Jakob Wassermann: Bleistift, lateinische Kurrent
\newline{}Handschrift Peter Altenberg: Bleistift, deutsche Kurrent
\newline{}Handschrift Alfred Gold: Bleistift
\newline{}Handschrift Elly Petersen: Bleistift, lateinische Kurrent
\newline{}Handschrift Ottilie Salten: Bleistift
\newline{}Versand: 1) Stempel: »\nobreak{}\oindex{Wien@\textbf{Wien}, \emph{Verwaltungsgebiet}|pwk}88 {[}Wie{]}n 13/1, 30. 4. 99, 7–8 V\nobreak{}«.   2) Stempel: »\nobreak{}1/5.99, \oindex{Berlin@\textbf{Berlin}, \emph{Hauptstadt}|pwk}{[}Berlin{]}, 7½–8½V., Bestellt vom Postamt\textcolor{gray}{e} 7\nobreak{}«. 
\newline{}Ordnung: mit Bleistift von unbekannter Hand nummeriert: »113« }\pstart{}{\pb}Herrn D\textsuperscript{r} Arthur Schnitzler\pend{}\pstart{}Berlin\oindex{Berlin@\textbf{Berlin}, \emph{Hauptstadt}|pw}\pend{}\pstart{}Hotel Savoy\oindex{Hotel Savoy [Berlin]@\textbf{Hotel Savoy [Berlin]}, \emph{Hotel}|pw}\pend{}{\bigskip}\vspace{1em}
\pstart
           \noindent{}{\pb}lieber, herzliche Grüße und auf baldiges Wiedersehen.
               {\\}Ihr \spacefill\mbox{Salten}\pend
           
\pstart
           {[}hs. Hirschfeld:{]} Herzlichſte Grüße, Ihr \spacefill\mbox{Georg
                  Hirschfeld.}\pend
           
\pstart
           {[}hs. Wassermann:{]} Ihr JakobWassermann grüsst herzlich.\pend
           
\pstart
           {[}hs. Altenberg:{]} Peter Altenberg herzlichſt grüßend!\pend
           
\pstart
           {[}hs. Gold:{]} \spacefill\mbox{Alfred Gold}\pend
           
\pstart
           {[}hs. Petersen:{]} Elly Hirschfeld grüsst.\pend
           
\pstart
           {[}hs. Salten:{]} \spacefill\mbox{Ottilie Metzl}\pend
           \selectlanguage{ngerman}\endnumbering\briefempfaengerindex{Schnitzler, Arthur@\textsc{Schnitzler, Arthur}!zzzSalten, Ottilie@\emph{von Ottilie Salten}!1899-04-303@{30. 4. 1899}|)be}\briefempfaengerindex{Schnitzler, Arthur@\textsc{Schnitzler, Arthur}!zzzPetersen, Elly@\emph{von Elly Petersen}!1899-04-303@{30. 4. 1899}|)be}\briefempfaengerindex{Schnitzler, Arthur@\textsc{Schnitzler, Arthur}!zzzGold, Alfred@\emph{von Alfred Gold}!1899-04-303@{30. 4. 1899}|)be}\briefempfaengerindex{Schnitzler, Arthur@\textsc{Schnitzler, Arthur}!zzzAltenberg, Peter@\emph{von Peter Altenberg}!1899-04-303@{30. 4. 1899}|)be}\briefempfaengerindex{Schnitzler, Arthur@\textsc{Schnitzler, Arthur}!zzzWassermann, Jakob@\emph{von Jakob Wassermann}!1899-04-303@{30. 4. 1899}|)be}\briefempfaengerindex{Schnitzler, Arthur@\textsc{Schnitzler, Arthur}!zzzHirschfeld, Georg@\emph{von Georg Hirschfeld}!1899-04-303@{30. 4. 1899}|)be}\briefempfaengerindex{Schnitzler, Arthur@\textsc{Schnitzler, Arthur}!zzzSalten, Felix@\emph{von Felix Salten}!1899-04-303@{30. 4. 1899}|)be}\mylabel{L03289h}  \newcommand{\dateiname}{L03289}\newcommand{\titel}{Felix Salten an Arthur Schnitzler, 30. 4. 1899}\newcommand{\editorInnen}{Martin Anton Müller und Laura Untner}%% latex-leseansicht-abspann.tex
%% Abspann für die Leseansicht.
%% Der Schalter \ifkorrekturansicht ist bereits durch den Vorspann gesetzt.

%% latex-abspann.tex
%% Gemeinsamer Abspann für Korrekturansicht und Leseansicht.
%% Setzt den Schalter \ifkorrekturansicht voraus (gesetzt in den
%% einbindenden Dateien latex-korrekturansicht-abspann.tex bzw.
%% latex-leseansicht-abspann.tex).
%% ---------------------------------------------------------------

\normalsize

% Das esempio-Environment wird nur in der Leseansicht benötigt
\ifkorrekturansicht\else
\newenvironment{esempio}[3]%
{
    \vspace{1.5ex}
    \rlap{\underline{#1}}
    \par
    \setlength{\parindent}{0cm}
    \nopagebreak
    \leftskip=#2cm
    \rightskip=#3cm
}
{
    \par
}
\fi

\doendnotes{C}
\bigskip
\vfill

\clearpage

\footnotesize

\ifkorrekturansicht
  \lohead{\textsc{register}}
\fi

% theindex-Environment neu definieren ohne reledmac
\makeatletter
\renewenvironment{theindex}{%
  \ifkorrekturansicht
    \section*{\indexname}%
  \else
    \subsubsection*{Index der erwähnten Entitäten}%
  \fi
  \setlength{\parindent}{0pt}%
  \setlength{\parskip}{0pt plus 0.3pt}%
  \let\item\@idxitem
}{%
  \ifkorrekturansicht\clearpage\fi
}
\makeatother

\IfFileExists{\jobname-pw.ind}{\input{\jobname-pw.ind}}{}

% Quellenangabe nur in der Leseansicht
\ifkorrekturansicht\else
% Fallback-Definitionen, falls die .tex-Datei \titel etc. nicht gesetzt hat
\providecommand{\titel}{}
\providecommand{\editorInnen}{}
\providecommand{\dateiname}{\jobname}

\vspace{3cm}

\vfill

\footnotesize
\textsc{Quelle}: \titel. Herausgegeben von {\editorInnen}. In: \emph{Arthur Schnitzler: Briefwechsel mit Autorinnen und Autoren}.
 Digitale Edition, https://schnitzler-briefe.acdh.oeaw.ac.at/{\dateiname}.html (Stand \today)
\fi

\end{document}


