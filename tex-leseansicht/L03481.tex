%% latex-korrekturansicht-vorspann.tex
%% Vorspann für die Korrekturansicht.
%% Lädt die gemeinsame Datei latex-vorspann.tex mit gesetztem Schalter.

\newif\ifkorrekturansicht
\korrekturansichttrue

\input{../tex-inputs/latex-vorspann}


\section[ Paul Goldmann an Arthur Schnitzler, 22. 4. 1927]{L03481 Paul Goldmann an Arthur Schnitzler, 22. 4. 1927}
\nopagebreak\mylabel{L03481v}
\rehead{ }\normalsize\beginnumbering\briefempfaengerindex{Schnitzler, Arthur@\textsc{Schnitzler, Arthur}!zzzGoldmann, Paul@\emph{von Paul Goldmann}!1927-04-221@{22. 4. 1927}|(be}
\toendnotes[C]{\smallbreak\pagebreak[2]}\Standort{DLA, A:Schnitzler, HS.NZ85.1.3176.}
\physDesc{Postkarte, 451 Zeichen
\newline{}Handschrift: 1) schwarze Tinte, deutsche Kurrent\hspace{1em}2) schwarze Tinte, lateinische Kurrent (\noindent{}Adresse)\hspace{1em}
\newline{}Versand: Stempel: »\nobreak{}\textcolor{gray}{Wien} 110\nobreak{}«. Stempel: »\nobreak{}Wien 1, 22{[}.{]} 4. 27, 9–10 N\nobreak{}«.  
\newline{}Schnitzler: mit rotem Buntstift eine Unterstreichung }\toendnotes[C]{\smallbreak}\pstart{}{\pb}Herrn\pend{}\pstart{}Dr. Arthur Schnitzler\pend{}\pstart{}Sternwartestraſse 71\oindex{Sternwartestrasse 71@\textbf{Sternwartestraße 71}, \emph{Wohngebäude (K.WHS)}|pw}\pend{}\pstart{}Wien XVIII.\oindex{XVIII., Waehring@\textbf{XVIII., Währing}, \emph{A.ADM3}|pw}\pend{}{\bigskip}\vspace{1em}
\pstart
           Wien\oindex{Wien@\textbf{Wien}, \emph{A.ADM2}|pw}{ }22. \substVorne{}\textsuperscript{7}\substDazwischen{}4\substHinten{}. 27.\pend
           
\pstart{}Lieber Freund,\pend\vspace{0.5em}
\pstart
           Meine Frau\pwindex{Goldmann, Eva Marie 27.10.1877 – 02.11.1937@\textsc{Goldmann, Eva Marie} (27.10.1877 – 02.11.1937)|pwv} u. ich ſind für
               einige Tage in Wien\oindex{Wien@\textbf{Wien}, \emph{A.ADM2}|pw}. Ich habe heut bei Dir angerufen, um Dich zu fragen, wann wir\pwindex{Goldmann, Eva Marie 27.10.1877 – 02.11.1937@\textsc{Goldmann, Eva Marie} (27.10.1877 – 02.11.1937)|pwv} Dich beſuchen können. Zu
               meinem großen Bedauern erfahre ich, daß Du \label{K_L03481-1v}\edtext{verreiſt}{\lemma{\textnormal{\emph{verreiſt}}}\Cendnote{\textnormal{Schnitzler war seit 1. 4. 1927 und noch
                  bis 2. 5. 1927 in
                     Venedig\oindex{Venedig@\textbf{Venedig}, \emph{P.PPLA}|pwk}.}}}\label{K_L03481-1} biſt. {\pb}Ich ſende Dir alſo auf dieſem Wege meiner Frau\pwindex{Goldmann, Eva Marie 27.10.1877 – 02.11.1937@\textsc{Goldmann, Eva Marie} (27.10.1877 – 02.11.1937)|pwv} u. meine herzlichſten
               Grüße. Wir hoffen auf ein \label{K_L03481-2v}\edtext{Wiederſehen
               in Berlin\oindex{Berlin@\textbf{Berlin}, \emph{P.PPLC}|pw}}{\lemma{\textnormal{\emph{Wiederſehen
               in Berlin}}}\Cendnote{\textnormal{Schnitzler und Goldmann\pwindex{Goldmann, Paul 31.01.1865 – 25.09.1935@\textsc{Goldmann, Paul} (31.01.1865 – 25.09.1935), \emph{Schriftsteller/Schriftstellerin, Journalist/Journalistin}|pwk} sahen sich die nächsten Male am 7. 10. 1927 in Wien\oindex{Wien@\textbf{Wien}, \emph{A.ADM2}|pwk} und am 5. 12. 1927 in Berlin\oindex{Berlin@\textbf{Berlin}, \emph{P.PPLC}|pwk}.}}}\label{K_L03481-2}, da wir ſo bald nicht wieder nach Wien\oindex{Wien@\textbf{Wien}, \emph{A.ADM2}|pw} kommen dürften.\pend
           
\pstart
           Dein {\\[\baselineskip]}\spacefill\mbox{Paul Goldmann.}\pend
           \leftskip=0em{}\selectlanguage{ngerman}\endnumbering\briefempfaengerindex{Schnitzler, Arthur@\textsc{Schnitzler, Arthur}!zzzGoldmann, Paul@\emph{von Paul Goldmann}!1927-04-221@{22. 4. 1927}|)be}\mylabel{L03481h}  \normalsize

\doendnotes{C}
\bigskip
\vfill

\clearpage

\footnotesize

\lohead{\textsc{register}}

% Definiere theindex-Environment komplett neu ohne reledmac
\makeatletter
\renewenvironment{theindex}{%
  \section*{\indexname}%
  \setlength{\parindent}{0pt}%
  \setlength{\parskip}{0pt plus 0.3pt}%
  \let\item\@idxitem
}{%
  \clearpage
}
\makeatother

\IfFileExists{\jobname-pw.ind}{\input{\jobname-pw.ind}}{}

\end{document}

      