%% latex-korrekturansicht-vorspann.tex
%% Vorspann für die Korrekturansicht.
%% Lädt die gemeinsame Datei latex-vorspann.tex mit gesetztem Schalter.

\newif\ifkorrekturansicht
\korrekturansichttrue

\input{../tex-inputs/latex-vorspann}


\section[Stefan Zweig an Arthur Schnitzler, 1. 8. 1923]{L03668 Stefan Zweig an Arthur Schnitzler, 1. 8. 1923}
\nopagebreak\mylabel{L03668v}
\rehead{ }\normalsize\beginnumbering\briefempfaengerindex{Schnitzler, Arthur@\textsc{Schnitzler, Arthur}!zzzZweig, Stefan@\emph{von Stefan Zweig}!1923-08-012@{1. 8. 1923}|(be}
\toendnotes[C]{\smallbreak\pagebreak[2]}\Standort{CUL, Schnitzler, B 118.}
\physDesc{Postkarte, 1 Blatt, 1 Seite, 506 Zeichen
\newline{}Handschrift: lila Tinte, lateinische Kurrent
\newline{}Versand: 1) Aufkleber: »Express«  2) Stempel: »\nobreak{}\oindex{Salzburg@\textbf{Salzburg}, \emph{A.ADM2}|pwk}Salzburg 1, 1. VIII. 23\nobreak{}«.  3) Stempel: »\nobreak{}\oindex{XVIII., Waehring@\textbf{XVIII., Währing}, \emph{A.ADM3}|pwk}Wien 111, \textcolor{gray}{2. 8. 23}, 11\textsuperscript{10}\nobreak{}«. }
\buchAbdrucke{\weitereDrucke{Stefan Zweig: \emph{Briefwechsel mit Hermann Bahr, Sigmund Freud, Rainer Maria
                        Rilke und Arthur Schnitzler}. Frankfurt am Main: \emph{S. Fischer} 1987, S. 418.} }\toendnotes[C]{\smallbreak}\pstart{}{\pb}D\textsuperscript{r}
                  Arthur Schnitzler\pend{}\pstart{}Wien – Cottage\oindex{Waehringer Cottage@\textbf{Währinger Cottage}, \emph{Teil eines besiedelten Ortes (A.BSOX)}|pw}\pend{}\pstart{}\label{K_L03668-1v}\edtext{Sternwartestrasse 71 oder 72}{\lemma{\textnormal{\emph{Sternwartestrasse … 72}}}\Cendnote{\textnormal{Zweig\pwindex{Zweig, Stefan 28.11.1881 – 23.02.1942@\textsc{Zweig, Stefan} (28.11.1881 – 23.02.1942), \emph{Schriftsteller/Schriftstellerin}|pwk} wechselt
                        bei der Adressierung seiner Schreiben an Schnitzler immer wieder zwischen der richtigen Hausnummer
                           »71« und der falschen
                  »72«.}}}\label{K_L03668-1}\oindex{Sternwartestrasse 71@\textbf{Sternwartestraße 71}, \emph{Wohngebäude (K.WHS)}|pw}\pend{}{\bigskip}\vspace{1em}
\pstart
           \noindent{}{\pb}Lieber verehrter Herr
                  Doktor, das Zimmer im Österr. Hof\oindex{Oesterreichischer Hof@\textbf{Österreichischer Hof}, \emph{Hotel (K.HTL)}|pw} ist für
                  Freitag reserviert. In der Annahme, dass Sie um
                  5 Uhr ankommen werden wir um ½ 6 im Österr Hof\oindex{Oesterreichischer Hof@\textbf{Österreichischer Hof}, \emph{Hotel (K.HTL)}|pw} den Thee nehmen \introOben{}und dort
                  auf Sie warten\introOben{}. Ich hätte sie natürlich zu uns gebeten, aber R.\pwindex{Rolland, Romain 29.01.1866 – 30.12.1944@\textsc{Rolland, Romain} (29.01.1866 – 30.12.1944), \emph{Schriftsteller/Schriftstellerin}|pw} will nachher um ¾ 8 zu dem Concert
               der Kammermusik und wir speisen dann gleich unten. Sind Sie aber Samstag
               noch da, so bitten wir Sie, \uline{herzlich} Mittags bei uns
               mit R.\pwindex{Rolland, Romain 29.01.1866 – 30.12.1944@\textsc{Rolland, Romain} (29.01.1866 – 30.12.1944), \emph{Schriftsteller/Schriftstellerin}|pw} zu speisen. In Herzlichkeit ergeben Ihr\pend
           \pstart \spacefill\mbox{Stefan Zweig}\pend{}\selectlanguage{ngerman}\endnumbering\briefempfaengerindex{Schnitzler, Arthur@\textsc{Schnitzler, Arthur}!zzzZweig, Stefan@\emph{von Stefan Zweig}!1923-08-012@{1. 8. 1923}|)be}\mylabel{L03668h}
\begin{anhang}
\end{anhang}\normalsize

\doendnotes{C}
\bigskip
\vfill

\clearpage

\footnotesize

\lohead{\textsc{register}}

% Definiere theindex-Environment komplett neu ohne reledmac
\makeatletter
\renewenvironment{theindex}{%
  \section*{\indexname}%
  \setlength{\parindent}{0pt}%
  \setlength{\parskip}{0pt plus 0.3pt}%
  \let\item\@idxitem
}{%
  \clearpage
}
\makeatother

\IfFileExists{\jobname-pw.ind}{\input{\jobname-pw.ind}}{}

\end{document}

      