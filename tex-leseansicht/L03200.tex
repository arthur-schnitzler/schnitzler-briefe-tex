%% latex-leseansicht-vorspann.tex
%% Vorspann für die Leseansicht.
%% Lädt die gemeinsame Datei latex-vorspann.tex mit nicht gesetztem Schalter.

\newif\ifkorrekturansicht
\korrekturansichtfalse

\input{../tex-inputs/latex-vorspann}


\section[ Paul Goldmann an Arthur Schnitzler, 20. 3. {[}1902{]}]{L03200 Paul Goldmann an Arthur Schnitzler,  20. 3. [1902]}
\nopagebreak\mylabel{L03200v}
\rehead{ }\normalsize\beginnumbering\briefempfaengerindex{Schnitzler, Arthur@\textsc{Schnitzler, Arthur}!zzzGoldmann, Paul@\emph{von Paul Goldmann}!1902-03-201@{20. 3. [1902]}|(be}
\toendnotes[C]{\smallbreak\pagebreak[2]}
\correspDesc{Versand  durch Paul Goldmann am 20. 3. [1902] in Berlin
\newline{}Erhalt  durch Arthur Schnitzler im Zeitraum [21. 3. 1902
                  – 25. 3. 1902?] in Wien}\toendnotes[C]{\smallbreak}
\Standort{DLA, A:Schnitzler, HS.NZ85.1.3172.}
\physDesc{Brief, 1 Blatt, 2 Seiten, 909 Zeichen
\newline{}Handschrift: blaue Tinte, deutsche Kurrent
\newline{}Schnitzler: mit Bleistift das Jahr »902« vermerkt }\toendnotes[C]{\smallbreak}
\pstart
           \raggedleft{}{\pb}\textcolor{gray}{\textbf{DESSAUERSTRASSE 19}}\oindex{Dessauer Straße@\textbf{Dessauer Straße}, \emph{Straße}|pw}\pend
           
\pstart
           Berlin\oindex{Berlin@\textbf{Berlin}, \emph{Hauptstadt}|pw}, 20. März\pend
           
\pstart{}Mein lieber Freund,\pend\vspace{0.5em}
\pstart
           Deinen letzten,{ }ſo{ }ſehr lieben und intereſſanten Brief, der mich wahrhaft erfreut
               hat, beantworte ich demnächſt. Meine Frankfurt\oindex{Frankfurt am Main@\textbf{Frankfurt am Main}, \emph{Hauptstadt}|pw}er
                  Freundin\pwindex{Rottenberg, Theodore 7.\,9.\,1875 – 5.\,4.\,1945 Limburg an der Lahn@\textsc{Rottenberg, Theodore} (7.\,9.\,1875 – 5.\,4.\,1945 Limburg an der Lahn)|pw} iſt in Berlin\oindex{Berlin@\textbf{Berlin}, \emph{Hauptstadt}|pw} und nimmt alle meine freie Zeit in Anſpruch. Wir
               verleben frohe Tage; aber auch hier miſcht{ }ſich mancherlei Bitterkeit ein.\pend
           
\pstart
           Für heut nur Folgendes: Zu Oſtern möchte ich (ohne
               Urlaub) auf zwei, drei Tage fortreiſen. Nach Wien\oindex{Wien@\textbf{Wien}, \emph{Verwaltungsgebiet}|pw}
               kann ich nicht kommen, weil die Reiſe zu weit iſt und weil ich eben ohne Urlaub
               weggehen will. Aber ich würde, wenn Du Luſt hätteſt, Dich auf halbem Wege zwiſchen
                  Berlin\oindex{Berlin@\textbf{Berlin}, \emph{Hauptstadt}|pw}{ }{\pb}und Wien\oindex{Wien@\textbf{Wien}, \emph{Verwaltungsgebiet}|pw} mit mir
               zu treffen,{ }ſehr gern nach \label{K_L03200-1v}\edtext{\textsc{Prag\oindex{Prag@\textbf{Prag}, \emph{Land}|pw}}}{\lemma{\textnormal{\emph{Prag}}}\Cendnote{\textnormal{Goldmann\pwindex{Goldmann, Paul 31.\,1.\,1865 Breslau – 25.\,9.\,1935 Wien@\textsc{Goldmann, Paul} (31.\,1.\,1865 Breslau – 25.\,9.\,1935 Wien), \emph{Schriftsteller, Journalist}|pwk} fuhr von Ende März bis Anfang April 1902 nach Prag\oindex{Prag@\textbf{Prag}, \emph{Land}|pwk}, es kam dabei jedoch zu keinem
                  Zusammentreffen mit Schnitzler.}}}\label{K_L03200-1}
               kommen, das ich noch nicht kenne und das eine intereſſante Stadt{ }ſein{ }ſoll. Ich würde
               mich unendlich freuen, wenn Du es möglich machen könnteſt, die Oſtertage mit mir zu
               verbringen. Bitte, antworte mir umgehend!\pend
           
\pstart
           Viele Grüße an \textsc{Olga\pwindex{Schnitzler, Olga 17.\,1.\,1882 Wien – 13.\,1.\,1970 Lugano@\textsc{Schnitzler, Olga} (17.\,1.\,1882 Wien – 13.\,1.\,1970 Lugano), \emph{Schauspielerin, Sängerin}|pw}} und an Dich! {\\[\baselineskip]}Von Herzen {\\[\baselineskip]}Dein {\\[\baselineskip]}\spacefill\mbox{Paul Goldm}\pend
           \leftskip=0em{}
\pstart
           \noindent{}Auch an \label{K_L03200-2v}\edtext{\textsc{Richard\pwindex{Beer-Hofmann, Richard 11.\,7.\,1866 Wien – 26.\,9.\,1945 New York City@\textsc{Beer-Hofmann, Richard} (11.\,7.\,1866 Wien – 26.\,9.\,1945 New York City), \emph{Schriftsteller}|pw}}}{\lemma{\textnormal{\emph{Richard}}}\Cendnote{\textnormal{Goldmann\pwindex{Goldmann, Paul 31.\,1.\,1865 Breslau – 25.\,9.\,1935 Wien@\textsc{Goldmann, Paul} (31.\,1.\,1865 Breslau – 25.\,9.\,1935 Wien), \emph{Schriftsteller, Journalist}|pwk} schrieb Beer-Hofmann\pwindex{Beer-Hofmann, Richard 11.\,7.\,1866 Wien – 26.\,9.\,1945 New York City@\textsc{Beer-Hofmann, Richard} (11.\,7.\,1866 Wien – 26.\,9.\,1945 New York City), \emph{Schriftsteller}|pwk} noch am selben Tag, vgl. \emph{Houghton Library}\orgindex{Houghton Library@Houghton Library|pwk},
                        Harvard (Signatur 825.978).
                  }}}\label{K_L03200-2}{ }ſchreibe ich.\pend
           \selectlanguage{ngerman}\endnumbering\briefempfaengerindex{Schnitzler, Arthur@\textsc{Schnitzler, Arthur}!zzzGoldmann, Paul@\emph{von Paul Goldmann}!1902-03-201@{20. 3. [1902]}|)be}\mylabel{L03200h}  \newcommand{\dateiname}{L03200}\newcommand{\titel}{Paul Goldmann an Arthur Schnitzler, 20. 3. [1902]}\newcommand{\editorInnen}{Martin Anton Müller und Laura Untner}%% latex-leseansicht-abspann.tex
%% Abspann für die Leseansicht.
%% Der Schalter \ifkorrekturansicht ist bereits durch den Vorspann gesetzt.

%% latex-abspann.tex
%% Gemeinsamer Abspann für Korrekturansicht und Leseansicht.
%% Setzt den Schalter \ifkorrekturansicht voraus (gesetzt in den
%% einbindenden Dateien latex-korrekturansicht-abspann.tex bzw.
%% latex-leseansicht-abspann.tex).
%% ---------------------------------------------------------------

\normalsize

% Das esempio-Environment wird nur in der Leseansicht benötigt
\ifkorrekturansicht\else
\newenvironment{esempio}[3]%
{
    \vspace{1.5ex}
    \rlap{\underline{#1}}
    \par
    \setlength{\parindent}{0cm}
    \nopagebreak
    \leftskip=#2cm
    \rightskip=#3cm
}
{
    \par
}
\fi

\doendnotes{C}
\bigskip
\vfill

\clearpage

\footnotesize

\ifkorrekturansicht
  \lohead{\textsc{register}}
\fi

% theindex-Environment neu definieren ohne reledmac
\makeatletter
\renewenvironment{theindex}{%
  \ifkorrekturansicht
    \section*{\indexname}%
  \else
    \subsubsection*{Index der erwähnten Entitäten}%
  \fi
  \setlength{\parindent}{0pt}%
  \setlength{\parskip}{0pt plus 0.3pt}%
  \let\item\@idxitem
}{%
  \ifkorrekturansicht\clearpage\fi
}
\makeatother

\IfFileExists{\jobname-pw.ind}{\input{\jobname-pw.ind}}{}

% Quellenangabe nur in der Leseansicht
\ifkorrekturansicht\else
% Fallback-Definitionen, falls die .tex-Datei \titel etc. nicht gesetzt hat
\providecommand{\titel}{}
\providecommand{\editorInnen}{}
\providecommand{\dateiname}{\jobname}

\vspace{3cm}

\vfill

\footnotesize
\textsc{Quelle}: \titel. Herausgegeben von {\editorInnen}. In: \emph{Arthur Schnitzler: Briefwechsel mit Autorinnen und Autoren}.
 Digitale Edition, https://schnitzler-briefe.acdh.oeaw.ac.at/{\dateiname}.html (Stand \today)
\fi

\end{document}


