%% latex-korrekturansicht-vorspann.tex
%% Vorspann für die Korrekturansicht.
%% Lädt die gemeinsame Datei latex-vorspann.tex mit gesetztem Schalter.

\newif\ifkorrekturansicht
\korrekturansichttrue

\input{../tex-inputs/latex-vorspann}


\section[Hugo von Hofmannsthal an Arthur Schnitzler, 8. 12. {[}1903{]}]{L01347 Hugo von Hofmannsthal an Arthur Schnitzler, 8. 12. {[}1903{]}}
\nopagebreak\mylabel{L01347v}
\rehead{ }\normalsize\beginnumbering\briefempfaengerindex{Schnitzler, Arthur@\textsc{Schnitzler, Arthur}!zzzHofmannsthal, Hugo von@\emph{von Hugo von Hofmannsthal}!1903-12-081@{8. 12. {[}1903{]}}|(be}
\toendnotes[C]{\smallbreak\pagebreak[2]}\Standort{CUL, Schnitzler, B 43.}
\physDesc{Brief, 1 Blatt, 4 Seiten, 1172 Zeichen
\newline{}Handschrift: schwarze Tinte, deutsche Kurrent
\newline{}Schnitzler: mit Bleistift die Jahreszahl ergänzt: »903.« 
\newline{}Ordnung: 1) mit Bleistift von unbekannter Hand nummeriert: »\strikeout{222}«  2) mit Bleistift von unbekannter Hand nummeriert:
                                    »206«}
\buchAbdrucke{\weitereDrucke{Hugo von Hofmannsthal, Arthur Schnitzler: \emph{Briefwechsel}. Frankfurt am Main: \emph{S. Fischer} 1964, S. 178–179.} }\toendnotes[C]{\smallbreak}
\pstart
           \raggedleft{}{\pb}8. XII.\pend
           
\pstart{}lieber,\pend\vspace{0.5em}
\pstart
           nun ſind es wieder vielleicht 4 Wochen, daſs man ſich nicht geſehen hat!\hspace*{1em}Iſt das nicht ſchad?\hspace*{1.5em}Und ich konnte diesmal abſolut nichts machen als warten, da Sie beim letzten Mal
               beſtimmt geſagt hatten, Sie würden herüberkommen. Wenn Ihnen aber das in der ganzen
                  {\pb}Zeit niemals paſste, warum
               dann kein \textsc{rendez-vous} in Hietzing\oindex{XIII., Hietzing@\textbf{XIII., Hietzing}, \emph{A.ADM3}|pw}? –\pend
           
\pstart
           Dieſe Woche bin ich Mittwoch Samstag Sonntag beſtimmt nicht frei.\hspace*{1.5em}Daſs Sie auch nie eine Zeile ſchreiben! \pend
           
\pstart
           Ich habe in der Zwiſchenzeit »Frau Bertha \textsc{Garlan}\pwindex{Frau Bertha Garlan. Roman@\emph{Frau Bertha Garlan. Roman}|pw}« wieder geleſen, mit noch viel {\pb}intenſiverem Vergnügen als das
               erſte mal, ja mit ungetrübtem Genuſs. Dieſes Buch und das neue Stück\pwindex{einsame Weg. Schauspiel in fuenf Akten@\emph{Der einsame Weg. Schauspiel in fünf Akten}|pwv} ſind wohl Ihre ſchönſten Arbeiten.
               Kaum zu glauben daſs das von einer Hand iſt, mit einem ſo dürren quälenden Buch wie
                  »Sterben\pwindex{Sterben. Novelle@\emph{Sterben. Novelle}|pw}« einem Buch, wie es deren eigentlich
               keine geben dürfte. {\pb}So viel Kraft
               und Wärme, Überſicht, Tact, Weltgefühl und Herzenskenntnis ſteckt in dieſer »Bertha \textsc{Garlan}\pwindex{Frau Bertha Garlan. Roman@\emph{Frau Bertha Garlan. Roman}|pw}«, ſo ſchön zuſammengehalten iſt es und ſo gut und geſcheidt dabei.\pend
           
\pstart
           Wenn Sie einmal ein überflüſſiges Exemplar der »Frau
                  des Weiſen\pwindex{Frau des Weisen. Novelletten@\emph{Die Frau des Weisen. Novelletten}|pw}« haben, meins iſt geſtohlen.\pend
           
\pstart
           Haben Sie nun ſchon die »Elektra\pwindex{Elektra. Tragoedie in einem Aufzug@\emph{Elektra. Tragödie in einem Aufzug}|pw}« oder nicht? –
               bekommen übrigens nächſtens auch noch etwas \label{K_L01347-1v}\edtext{andres\pwindex{gerettete Venedig. Trauerspiel in fuenf Aufzuegen@\emph{Das gerettete Venedig. Trauerspiel in fünf Aufzügen}|pwuv}}{\lemma{\textnormal{\emph{andres}}}\Cendnote{\textnormal{Schnitzler rechnete damit, \emph{Das gerettete Venedig}\pwindex{gerettete Venedig. Trauerspiel in fuenf Aufzuegen@\emph{Das gerettete Venedig. Trauerspiel in fünf Aufzügen}|pwk} zu bekommen; siehe Arthur Schnitzler an Hugo von Hofmannsthal, 10. 12. 1903.
               }}}\label{K_L01347-1}.\pend
           \pstart Von Herzen \spacefill\mbox{Hugo.}\pend{}\selectlanguage{ngerman}\endnumbering\briefempfaengerindex{Schnitzler, Arthur@\textsc{Schnitzler, Arthur}!zzzHofmannsthal, Hugo von@\emph{von Hugo von Hofmannsthal}!1903-12-081@{8. 12. {[}1903{]}}|)be}\mylabel{L01347h}  \normalsize

\doendnotes{C}
\bigskip
\vfill

\clearpage

\footnotesize

\lohead{\textsc{register}}

% Definiere theindex-Environment komplett neu ohne reledmac
\makeatletter
\renewenvironment{theindex}{%
  \section*{\indexname}%
  \setlength{\parindent}{0pt}%
  \setlength{\parskip}{0pt plus 0.3pt}%
  \let\item\@idxitem
}{%
  \clearpage
}
\makeatother

\IfFileExists{\jobname-pw.ind}{\input{\jobname-pw.ind}}{}

\end{document}

      