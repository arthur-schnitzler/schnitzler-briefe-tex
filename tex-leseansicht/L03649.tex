%% latex-leseansicht-vorspann.tex
%% Vorspann für die Leseansicht.
%% Lädt die gemeinsame Datei latex-vorspann.tex mit nicht gesetztem Schalter.

\newif\ifkorrekturansicht
\korrekturansichtfalse

\input{../tex-inputs/latex-vorspann}


\section[Stefan Zweig an Arthur Schnitzler, {{[}}29. 11. 1914{{]}}]{L03649 Stefan Zweig an Arthur Schnitzler, {[}29. 11. 1914{]}}
\nopagebreak\mylabel{L03649v}
\rehead{ }\normalsize\beginnumbering\briefempfaengerindex{Schnitzler, Arthur@\textsc{Schnitzler, Arthur}!zzzZweig, Stefan@\emph{von Stefan Zweig}!1914-11-291@{{[}29. 11. 1914{]}}|(be}
\toendnotes[C]{\smallbreak\pagebreak[2]}
\correspDesc{Versand  durch Stefan Zweig am [29. 11. 1914] in Wien
\newline{}Erhalt  durch Arthur Schnitzler im Zeitraum [29. 11. 1914 – 3. 12. 1914?] in Wien}\toendnotes[C]{\smallbreak}
\Standort{CUL, Schnitzler, B 118.}
\physDesc{Brief, 1 Blatt, 4 Seiten, 2483 Zeichen
\newline{}Handschrift: blaue Tinte, lateinische Kurrent
\newline{}Schnitzler: 1) mit Bleistift datiert: »29/11 914« und beschriftet: »\textsc{Zweig}«  2) mit rotem Buntstift eine Unterstreichung}
\buchAbdrucke{\weitereDrucke{Stefan Zweig: \emph{Briefwechsel mit Hermann Bahr, Sigmund Freud, Rainer Maria
                        Rilke und Arthur Schnitzler}. Herausgegeben von Jeffrey B. Berlin, Hans-Ulrich Lindken und Donald A. Prater. Frankfurt am Main: \emph{S. Fischer} 1987, S. 384–385.} }\toendnotes[C]{\smallbreak}
\pstart
           {\pb}\textcolor{gray}{\textbf{SZ}}\hfill \textcolor{gray}{\textbf{VIII. KOCHGASSE\oindex{Wien@\textbf{Wien}!VIII., Josefstadt@\textbf{VIII., Josefstadt}!Kochgasse 8@\textbf{Kochgasse 8}, \emph{Wohngebäude}|pw}}}\pend
           
\pstart
           \raggedleft{}\textcolor{gray}{\textbf{WIEN\oindex{Wien@\textbf{Wien}, \emph{Verwaltungsgebiet}|pw},}}\pend
           \vspace{0.5em}
\pstart
           Verehrter lieber Herr Doktor, Sie sind so gütig, meine bescheidene
               Meinung in dieser Sache anzufragen und ich sage sie aufrichtigst. Ich glaube nur der
               erste Teil der Berichtigung\pwindex{Schnitzler, Arthur 15.\,5.\,1862 Wien – 21.\,10.\,1931 ebd.@\textsc{Schnitzler, Arthur} (15.\,5.\,1862 Wien – 21.\,10.\,1931 ebd.), \emph{Schriftsteller, Mediziner}!Une protestation d’Arthur Schnitzler@\strich\emph{Une protestation d’Arthur Schnitzler}|pwv}
               ist \uline{notwendig}, \label{K_L03649-1v}\edtext{der zweite}{\lemma{\textnormal{\emph{der zweite}}}\Cendnote{\textnormal{Siehe
                  die beiden letzten Seiten der Beilage von Schnitzlers Brief vom XXXX Auszeichnungsfehler: Dokument L03774 nicht gefunden. Dieser Schlussteil des Textes nahm Bezug auf den Abdruck (A. S.: \emph{»Das Zeitlose ist von kürzester Dauer«}, Artur Schnitzler über den Krieg. Brief an einen Schulfreund in New York, 17. 11. 1914) eines stark veränderten
                  Privatbriefes Schnitzlers an Eugen Deimel\pwindex{Deimel, Eugen März 1860 – 10.\,3.\,1920 New York City@\textsc{Deimel, Eugen} (März 1860 – 10.\,3.\,1920 New York City), \emph{Journalist}|pwk}, zu dieser Angelegenheit vgl.
                     Schnitzlers Brief an Eugen Deimel\pwindex{Deimel, Eugen März 1860 – 10.\,3.\,1920 New York City@\textsc{Deimel, Eugen} (März 1860 – 10.\,3.\,1920 New York City), \emph{Journalist}|pwk} vom 25. 11. 1914 (Heinz P.
                     Adamek: \emph{In die Neue Welt… Arthur Schnitzler – Eugen Deimel
                        Briefwechsel}. Wien: \emph{Holzhausen}{ }2003, S. 210–211). Der dies bezügliche Textteil wurde nicht im Rahmen der Berichtigung\pwindex{Schnitzler, Arthur 15.\,5.\,1862 Wien – 21.\,10.\,1931 ebd.@\textsc{Schnitzler, Arthur} (15.\,5.\,1862 Wien – 21.\,10.\,1931 ebd.), \emph{Schriftsteller, Mediziner}!Une protestation d’Arthur Schnitzler@\strich\emph{Une protestation d’Arthur Schnitzler}|pwkv} publiziert. Schnitzler hatte gegen den Abdruck des verfälschten Briefes zudem bereits am
                     20. 11. 1914 mit einem anderen Schreiben (A. S.: \emph{»Das Zeitlose ist von kürzester Dauer«}, Ein Brief von Artur Schnitzler, 20. 11. 1914) im \emph{Neuen Wiener Journal}\pwindex{Neues Wiener Journal@\emph{Neues Wiener Journal}|pwk} protestiert.}}}\label{K_L03649-1} bloss eben nur
               eine Richtigstellung einer Veränderung, die niemanden beleidigt. Und im ersten Teile
               hätte ich so gerne von einem Manne Ihrer Gerechtigkeit eines gesehen: \label{K_L03649-2v}\edtext{ein Wort des
               Positiven}{\lemma{\textnormal{\emph{ein Wort des
               Positiven}}}\Cendnote{\textnormal{Schnitzler erweiterte den Text\pwindex{Schnitzler, Arthur 15.\,5.\,1862 Wien – 21.\,10.\,1931 ebd.@\textsc{Schnitzler, Arthur} (15.\,5.\,1862 Wien – 21.\,10.\,1931 ebd.), \emph{Schriftsteller, Mediziner}!Une protestation d’Arthur Schnitzler@\strich\emph{Une protestation d’Arthur Schnitzler}|pwkv} gegenüber dem Entwurf um das Doppelte mit dem von Zweig\pwindex{Zweig, Stefan 28.\,11.\,1881 Wien – 23.\,2.\,1942 Petrópolis@\textsc{Zweig, Stefan} (28.\,11.\,1881 Wien – 23.\,2.\,1942 Petrópolis), \emph{Schriftsteller}|pwk} angeregten Bekenntnis zu den Werken von Tolstoi\pwindex{Tolstoi, Lew Nikolajewitsch 9.\,9.\,1828 Yasnaya Polyana – 20.\,11.\,1910 Lev Tolstoy@\textsc{Tolstoi, Lew Nikolajewitsch} (9.\,9.\,1828 Yasnaya Polyana – 20.\,11.\,1910 Lev Tolstoy), \emph{Schriftsteller}|pwk}, Anatole France\pwindex{France, Anatole 16.\,4.\,1844 Paris – 12.\,10.\,1924 Saint-Cyr-sur-Loire@\textsc{France, Anatole} (16.\,4.\,1844 Paris – 12.\,10.\,1924 Saint-Cyr-sur-Loire), \emph{Schriftsteller}|pwk}, Maeterlinck\pwindex{Maeterlinck, Maurice 29.\,8.\,1862 Gent – 6.\,5.\,1949 Nizza@\textsc{Maeterlinck, Maurice} (29.\,8.\,1862 Gent – 6.\,5.\,1949 Nizza), \emph{Schriftsteller}|pwk} und Shakespeare\pwindex{Shakespeare, William 23.\,4.\,1564? Stratford-upon-Avon – 3.\,5.\,1616 ebd.@\textsc{Shakespeare, William} (23.\,4.\,1564? Stratford-upon-Avon – 3.\,5.\,1616 ebd.), \emph{Schauspieler, Dramatiker}|pwk}, siehe A. S.: \emph{»Das Zeitlose ist von kürzester Dauer«}, Une protestation d’Arthur Schnitzler, 21. 12. 1914.}}}\label{K_L03649-2}, der Bejahung. Ich glaube, nie war eine Zeit besser für das Bekennen, nie
               es notwendiger, die Unerschütterlichkeit unserer innern Überzeugungen gegen gewisse
               Versuche aufrechtzuerhalten, den {\pb}politischen Constellationen unsere künstlerischen Empfindungen preiszugeben. Ich
               meine: es wäre schön und vorbildlich gewesen (und zugleich die stärkste, die
               schlagendste Berichtigung jeder Entstellung), \strikeout{\textcolor{gray}{Sie}} wenn Sie an einer Stelle sagten, wie sehr Sie Tolstoi\pwindex{Tolstoi, Lew Nikolajewitsch 9.\,9.\,1828 Yasnaya Polyana – 20.\,11.\,1910 Lev Tolstoy@\textsc{Tolstoi, Lew Nikolajewitsch} (9.\,9.\,1828 Yasnaya Polyana – 20.\,11.\,1910 Lev Tolstoy), \emph{Schriftsteller}|pw} bewundern und auch Ihr Verhältnis zu France\pwindex{France, Anatole 16.\,4.\,1844 Paris – 12.\,10.\,1924 Saint-Cyr-sur-Loire@\textsc{France, Anatole} (16.\,4.\,1844 Paris – 12.\,10.\,1924 Saint-Cyr-sur-Loire), \emph{Schriftsteller}|pw} und Maeterlinck\pwindex{Maeterlinck, Maurice 29.\,8.\,1862 Gent – 6.\,5.\,1949 Nizza@\textsc{Maeterlinck, Maurice} (29.\,8.\,1862 Gent – 6.\,5.\,1949 Nizza), \emph{Schriftsteller}|pw}
               in künstlerischer Bejahung andeuteten. Ich glaube, wir müssen ein Beispiel bei jedem
               Anlass geben, zu zeigen, dass unsere Neigungen nicht ein Tauschgeschäft auf
               Gegenliebe sind, sondern unerschütterlich selbst durch Hass und Anfeindung. Gerade
               weil Einige versuchen, jeden, der gegen Deutschland heute auftritt, zu negieren,
               statt seine Argumente zu befeinden, müssen wir unsere Unabhängigkeit in der eigensten
               engsten Welt unseres Standes und Wirkens {\pb}mit sichtbarem Willen betonen. Nichts ist gemäßer in diesen Tagen als
               Wahrhaftigkeit, die sich nicht einschüchtern lässt durch die Reden am Markt: ich
               glaube, wir sollen heute \substVorne{}\textsuperscript{\textcolor{gray}{je} als \textcolor{gray}{mehr}}\substDazwischen{}unentwegt\substHinten{}{ }Tolstoi\pwindex{Tolstoi, Lew Nikolajewitsch 9.\,9.\,1828 Yasnaya Polyana – 20.\,11.\,1910 Lev Tolstoy@\textsc{Tolstoi, Lew Nikolajewitsch} (9.\,9.\,1828 Yasnaya Polyana – 20.\,11.\,1910 Lev Tolstoy), \emph{Schriftsteller}|pw} einen der wirklichsten Menschen aller
               Zeiten nennen und brauchen nicht zu zögern mit Ehrerbietung vor der Leistung eines
                  Anatole France\pwindex{France, Anatole 16.\,4.\,1844 Paris – 12.\,10.\,1924 Saint-Cyr-sur-Loire@\textsc{France, Anatole} (16.\,4.\,1844 Paris – 12.\,10.\,1924 Saint-Cyr-sur-Loire), \emph{Schriftsteller}|pw}. Ein Vermeiden dieser
               Höflichkeitsbezeugung und dieser freien Zustimmung zu ihren Werken (die längst vor
               diesen Tagen entstanden) könnte leicht darauf deuten, wenn schon nicht eine Äusserung
               so sei doch Ihre Gesinnung jenen feindlich. Und das ist doch nicht Ihre Absicht.\pend
           
\pstart
           Ich wage natürlich nicht, diese meine Empfindung zur Ihren machen zu wollen: es ist
               nur eine Antwort auf Ihre gütige Frage. Gerne expediere ich den {\pb}Brief\pwindex{Schnitzler, Arthur 15.\,5.\,1862 Wien – 21.\,10.\,1931 ebd.@\textsc{Schnitzler, Arthur} (15.\,5.\,1862 Wien – 21.\,10.\,1931 ebd.), \emph{Schriftsteller, Mediziner}!Une protestation d’Arthur Schnitzler@\strich\emph{Une protestation d’Arthur Schnitzler}|pwv} in dieser Fassung wie in
               jeder andern an R. R.\pwindex{Rolland, Romain 29.\,1.\,1866 Clamecy – 30.\,12.\,1944 Vézelay@\textsc{Rolland, Romain} (29.\,1.\,1866 Clamecy – 30.\,12.\,1944 Vézelay), \emph{Schriftsteller}|pw}, es wird ihm eine grosse
               Freude sein, Sie unter den Wenigen zu wissen, die heute, mitten im Kampf, schon an
               die Versöhnung denken.\pend
           
\pstart
           Ich bin \label{K_L03649-3v}\edtext{morgen Montag}{\lemma{\textnormal{\emph{morgen Montag}}}\Cendnote{\textnormal{Dass der folgende Tag ein Montag war, bestätigt die Datierung Schnitzlers auf dem von Zweig\pwindex{Zweig, Stefan 28.\,11.\,1881 Wien – 23.\,2.\,1942 Petrópolis@\textsc{Zweig, Stefan} (28.\,11.\,1881 Wien – 23.\,2.\,1942 Petrópolis), \emph{Schriftsteller}|pwk} nicht datierten Brief auf den 29. 11. 1914, denn der 30. 11. 1914 war der Montag innnerhalb der mit der Korrektur des Textes befassten Tage zwischen 27. 11. 1914 und 2. 12. 1914.}}}\label{K_L03649-3} nach dem Bureau\orgindex{Kriegsarchiv@Kriegsarchiv|pwv} bestimmt zwischen 4–5 zu hause und freute mich sehr Ihres Anrufes.
               Vielen vielen Dank für Ihr Vertrauen und alles Herzliche Ihnen und den Ihren!
               Treulichst\pend
           \pstart \spacefill\mbox{Stefan Zweig}\pend{}\selectlanguage{ngerman}\endnumbering\briefempfaengerindex{Schnitzler, Arthur@\textsc{Schnitzler, Arthur}!zzzZweig, Stefan@\emph{von Stefan Zweig}!1914-11-291@{{[}29. 11. 1914{]}}|)be}\mylabel{L03649h}  \newcommand{\dateiname}{L03649}\newcommand{\titel}{Stefan Zweig an Arthur Schnitzler, [29. 11. 1914]}\newcommand{\editorInnen}{Selma Jahnke und Martin Anton Müller}%% latex-leseansicht-abspann.tex
%% Abspann für die Leseansicht.
%% Der Schalter \ifkorrekturansicht ist bereits durch den Vorspann gesetzt.

%% latex-abspann.tex
%% Gemeinsamer Abspann für Korrekturansicht und Leseansicht.
%% Setzt den Schalter \ifkorrekturansicht voraus (gesetzt in den
%% einbindenden Dateien latex-korrekturansicht-abspann.tex bzw.
%% latex-leseansicht-abspann.tex).
%% ---------------------------------------------------------------

\normalsize

% Das esempio-Environment wird nur in der Leseansicht benötigt
\ifkorrekturansicht\else
\newenvironment{esempio}[3]%
{
    \vspace{1.5ex}
    \rlap{\underline{#1}}
    \par
    \setlength{\parindent}{0cm}
    \nopagebreak
    \leftskip=#2cm
    \rightskip=#3cm
}
{
    \par
}
\fi

\doendnotes{C}
\bigskip
\vfill

\clearpage

\footnotesize

\ifkorrekturansicht
  \lohead{\textsc{register}}
\fi

% theindex-Environment neu definieren ohne reledmac
\makeatletter
\renewenvironment{theindex}{%
  \ifkorrekturansicht
    \section*{\indexname}%
  \else
    \subsubsection*{Index der erwähnten Entitäten}%
  \fi
  \setlength{\parindent}{0pt}%
  \setlength{\parskip}{0pt plus 0.3pt}%
  \let\item\@idxitem
}{%
  \ifkorrekturansicht\clearpage\fi
}
\makeatother

\IfFileExists{\jobname-pw.ind}{\input{\jobname-pw.ind}}{}

% Quellenangabe nur in der Leseansicht
\ifkorrekturansicht\else
% Fallback-Definitionen, falls die .tex-Datei \titel etc. nicht gesetzt hat
\providecommand{\titel}{}
\providecommand{\editorInnen}{}
\providecommand{\dateiname}{\jobname}

\vspace{3cm}

\vfill

\footnotesize
\textsc{Quelle}: \titel. Herausgegeben von {\editorInnen}. In: \emph{Arthur Schnitzler: Briefwechsel mit Autorinnen und Autoren}.
 Digitale Edition, https://schnitzler-briefe.acdh.oeaw.ac.at/{\dateiname}.html (Stand \today)
\fi

\end{document}


