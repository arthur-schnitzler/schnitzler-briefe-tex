%% latex-leseansicht-vorspann.tex
%% Vorspann für die Leseansicht.
%% Lädt die gemeinsame Datei latex-vorspann.tex mit nicht gesetztem Schalter.

\newif\ifkorrekturansicht
\korrekturansichtfalse

\input{../tex-inputs/latex-vorspann}

\begin{center}
            \textcolor{red}{ENTWURF, NICHT FERTIG KORRIGIERT}
                      \end{center}
            
         
         \renewcommand{\erwaehntePersonen}{Personen: Alfred Dreyfus, Fedor Mamroth, Josef Rosengart, Vally Rosengart}
         \renewcommand{\erwaehnteInstitutionen}{Institutionen: Frankfurter Zeitung}
         \renewcommand{\erwaehnteOrte}{Orte: Berlin, Deutsches Theater Berlin, Frankfurt am Main, Wien}
         \renewcommand{\erwaehnteWerke}{Werke: Der grüne Kakadu – Paracelsus – Die Gefährtin. Drei Einakter}
               \section[ Paul Goldmann an Arthur Schnitzler, 29. 4. 1899]{ Paul Goldmann an Arthur Schnitzler, 29. 4. 1899}\nopagebreak\mylabel{v}\rehead{ }\begin{ledgroupsized}[t]{13cm}\normalsize\beginnumbering \toendnotes[C]{\smallbreak\pagebreak[2]} \Standort{DLA, A:Schnitzler, HS.NZ85.1.3169.}
\physDesc{Brief, 1 Blatt, 2 Seiten
\newline{}Handschrift: schwarze Tinte, deutsche Kurrent}\toendnotes[C]{\smallbreak}\pstart
           \noindent{}{\pb}\textcolor{gray}{\textbf{\textsc{Frankfurter Zeitung}}}\orgindex{Frankfurter Zeitung@Frankfurter Zeitung|pw}\hfill \textcolor{gray}{\textbf{Frankfurt a. M.\oindex{Frankfurt am Main@\textbf{Frankfurt am Main}|pw},}}{ }29. April \textcolor{gray}{\textbf{189}}9.\pend
           \pstart
           \textsc{\textcolor{gray}{\textbf{und}}}\pend
           \pstart
           \textcolor{gray}{\textbf{\textsc{Handelsblatt.}}}\pend
           \pstart
           \textcolor{gray}{\textbf{\textsc{Redaktion\orgindex{Frankfurter Zeitung@Frankfurter Zeitung|pwv}.\footnote{\noindent{}\textcolor{gray}{\textbf{\textsc{Für die Redaktion\orgindex{Frankfurter Zeitung@Frankfurter Zeitung|pwv} beſtimmte Briefe und Sendungen
                                    wolle man \so{nicht} an die Perſon eines
                                    Redakteurs, ſondern ſtets \textbf{an die Redaktion der
                                          Frankfurter Zeitung\orgindex{Frankfurter Zeitung@Frankfurter Zeitung|pw}} adreſſiren.}}}}}}}\pend
           \pstart
           \textcolor{gray}{\textbf{\textsc{Telegramm-Adreſſe:}}}\pend
           \pstart
           \textcolor{gray}{\textbf{\textsc{Zeitung\orgindex{Frankfurter Zeitung@Frankfurter Zeitung|pwv}{ }Frankfurt Main\oindex{Frankfurt am Main@\textbf{Frankfurt am Main}|pw}.}}}\pend
           \pstart{}Mein lieber Freund,\pend\pstart
           Dank für Deine Karte, die mich ſehr beruhigt hat. Ich bin recht froh, Dich in \label{K_L02873-1v}\edtext{Berlin\oindex{Berlin@\textbf{Berlin}|pw}}{\lemma{\textnormal{\emph{Berlin}}}\Cendnote{\textnormal{Schnitzler\pwindex{Schnitzler, Arthur 15.05.1862 – 21.10.1931@\textsc{Schnitzler, Arthur} (15.05.1862 – 21.10.1931), \emph{Schriftsteller, Mediziner}|pwk} war von 25. 4. 1899 bis 2. 5. 1899 für die
                  Premiere von Der grüne Kakadu –
                     Paracelsus – Die Gefährtin. Drei Einakter\pwindex{Schnitzler, Arthur 15.05.1862 – 21.10.1931@\textsc{Schnitzler, Arthur} (15.05.1862 – 21.10.1931), \emph{Schriftsteller, Mediziner}!gruene Kakadu – Paracelsus – Die Gefaehrtin. Drei Einakter1898 – 1899@\strich\emph{Der grüne Kakadu – Paracelsus – Die Gefährtin. Drei Einakter} {[}1898 – 1899{]}|pwkv} (am 29. 4. 1899) am Deutschen Theater\oindex{Deutsches Theater Berlin@\textbf{Deutsches Theater Berlin}|pwk} nach Berlin\oindex{Berlin@\textbf{Berlin}|pwk} gereist.}}}\label{K_L02873-1h} zu wiſſen. Mein Brief erreicht Dich
               jedenfalls am Morgen nach einem neuen großen Erfolge\pwindex{Schnitzler, Arthur 15.05.1862 – 21.10.1931@\textsc{Schnitzler, Arthur} (15.05.1862 – 21.10.1931), \emph{Schriftsteller, Mediziner}!gruene Kakadu – Paracelsus – Die Gefaehrtin. Drei Einakter1898 – 1899@\strich\emph{Der grüne Kakadu – Paracelsus – Die Gefährtin. Drei Einakter} {[}1898 – 1899{]}|pwv} und ſoll Dir auch gleich meinen
               Glückwunſch bringen.\pend
           \pstart
           Nochmals, bitte: \label{K_L02873-4v}\edtext{\uuline{komm nach Frankfurt\oindex{Frankfurt am Main@\textbf{Frankfurt am Main}|pw}}}{\lemma{\textnormal{\emph{komm nach Frankfurt}}}\Cendnote{\textnormal{nicht geschehen}}}\label{K_L02873-4h}! Die \label{K_L02873-2v}\edtext{\textsc{Dreyfus\pwindex{Dreyfus, Alfred 1859-10-09 – 1935-07-12@\textsc{Dreyfus, Alfred} (1859-10-09 – 1935-07-12), \emph{Militär}|pw}-\begin{otherlanguage}{french}Enquête\end{otherlanguage}}}{\lemma{\textnormal{\emph{Dreyfus-Enquête}}}\Cendnote{\textnormal{siehe Paul Goldmann an Arthur Schnitzler, 26. 4. 1899}}}\label{K_L02873-2h} geht dieſe Woche zu Ende. Nächſte Woche werde ich ſicherlich mehr Zeit haben.
               Wenn Du da biſt, kann ich mich jeden Nachmittag von 5 Uhr ab freimachen.
               Du brauchſt Dich doch wirklich nicht ſo zu eilen, nach Wien\oindex{Wien@\textbf{Wien}|pw} zurückzukommen. Je länger Du fortbleibſt, umſo beſſer iſt es. Und vor
               ein {\pb}paar Stunden Eiſenbahnfahrt mehr wirſt Du Dich
               doch gewiß nicht fürchten.\pend
           \pstart
           Was mich anlangt, ſo dringe ich deshalb ſo ſehr darauf, Dich jetzt zu ſehen, weil ich
               keine Ahnung habe, ob ich in dieſem Jahr\textcolor{gray}{e} überhaupt Urlaub
               bekomme. Die Redaktion\orgindex{Frankfurter Zeitung@Frankfurter Zeitung|pwv} hat eine
               Reihe von Reiſemiſſionen für mich in Ausſicht, und es iſt nicht unmöglich, daß ſie
               den ganzen Sommer und Herbſt füllen. Laß’ Dich erbitten und komm’ her! Wenn ich nicht
               Zeit habe, wirſt Du bei meinem Schwager\pwindex{Rosengart, Josef 1860-02-08 – 1927-08-04@\textsc{Rosengart, Josef} (1860-02-08 – 1927-08-04), \emph{Arzt}|pwv}, meiner Schweſter\pwindex{Rosengart, Vally *~1866-12-29@\textsc{Rosengart, Vally} (*~1866-12-29)|pwv}, meinem Onkel\pwindex{Mamroth, Fedor 21.02.1851 – 25.06.1907@\textsc{Mamroth, Fedor} (21.02.1851 – 25.06.1907), \emph{Journalist, Kritiker}|pwv} ſein. Allein werden wir Dich ſchon nicht laſſen. Auch ſonſt wirſt du
               hier Den und Jenen kennen lernen, der Dich \strikeout{h\textcolor{gray}{×}\-\textcolor{gray}{×}\-\textcolor{gray}{×}\-\textcolor{gray}{×}} intereſſiren wird. Bitte, bitte, komm’ hierher!\pend
           \pstart
           Viele treue Grüße! {\\[\baselineskip]}Dein \spacefill\mbox{Paul Goldmann.}\pend
           \leftskip=0em{}
         
         \endnumbering\mylabel{h}\end{ledgroupsized}  \newcommand{\dateiname}{L02873}\newcommand{\titel}{Paul Goldmann an Arthur Schnitzler, 29. 4. 1899}\newcommand{\editorInnen}{Martin Anton Müller und Laura Untner}%% latex-leseansicht-abspann.tex
%% Abspann für die Leseansicht.
%% Der Schalter \ifkorrekturansicht ist bereits durch den Vorspann gesetzt.

%% latex-abspann.tex
%% Gemeinsamer Abspann für Korrekturansicht und Leseansicht.
%% Setzt den Schalter \ifkorrekturansicht voraus (gesetzt in den
%% einbindenden Dateien latex-korrekturansicht-abspann.tex bzw.
%% latex-leseansicht-abspann.tex).
%% ---------------------------------------------------------------

\normalsize

% Das esempio-Environment wird nur in der Leseansicht benötigt
\ifkorrekturansicht\else
\newenvironment{esempio}[3]%
{
    \vspace{1.5ex}
    \rlap{\underline{#1}}
    \par
    \setlength{\parindent}{0cm}
    \nopagebreak
    \leftskip=#2cm
    \rightskip=#3cm
}
{
    \par
}
\fi

\doendnotes{C}
\bigskip
\vfill

\clearpage

\footnotesize

\ifkorrekturansicht
  \lohead{\textsc{register}}
\fi

% theindex-Environment neu definieren ohne reledmac
\makeatletter
\renewenvironment{theindex}{%
  \ifkorrekturansicht
    \section*{\indexname}%
  \else
    \subsubsection*{Index der erwähnten Entitäten}%
  \fi
  \setlength{\parindent}{0pt}%
  \setlength{\parskip}{0pt plus 0.3pt}%
  \let\item\@idxitem
}{%
  \ifkorrekturansicht\clearpage\fi
}
\makeatother

\IfFileExists{\jobname-pw.ind}{\input{\jobname-pw.ind}}{}

% Quellenangabe nur in der Leseansicht
\ifkorrekturansicht\else
% Fallback-Definitionen, falls die .tex-Datei \titel etc. nicht gesetzt hat
\providecommand{\titel}{}
\providecommand{\editorInnen}{}
\providecommand{\dateiname}{\jobname}

\vspace{3cm}

\vfill

\footnotesize
\textsc{Quelle}: \titel. Herausgegeben von {\editorInnen}. In: \emph{Arthur Schnitzler: Briefwechsel mit Autorinnen und Autoren}.
 Digitale Edition, https://schnitzler-briefe.acdh.oeaw.ac.at/{\dateiname}.html (Stand \today)
\fi

\end{document}


      