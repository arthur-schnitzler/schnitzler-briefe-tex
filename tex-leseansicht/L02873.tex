%% latex-korrekturansicht-vorspann.tex
%% Vorspann für die Korrekturansicht.
%% Lädt die gemeinsame Datei latex-vorspann.tex mit gesetztem Schalter.

\newif\ifkorrekturansicht
\korrekturansichttrue

\input{../tex-inputs/latex-vorspann}


\section[ Paul Goldmann an Arthur Schnitzler, 29. 4. 1899]{L02873 Paul Goldmann an Arthur Schnitzler, 29. 4. 1899}
\nopagebreak\mylabel{L02873v}
\rehead{ }\normalsize\beginnumbering\briefempfaengerindex{Schnitzler, Arthur@\textsc{Schnitzler, Arthur}!zzzGoldmann, Paul@\emph{von Paul Goldmann}!1899-04-291@{29. 4. 1899}|(be}
\toendnotes[C]{\smallbreak\pagebreak[2]}\Standort{DLA, A:Schnitzler, HS.NZ85.1.3169.}
\physDesc{Brief, 1 Blatt, 2 Seiten, 1265 Zeichen
\newline{}Handschrift: schwarze Tinte, deutsche Kurrent}\toendnotes[C]{\smallbreak}
\pstart
           {\pb}\textcolor{gray}{\textbf{\textbf{Frankfurter Zeitung}}}\orgindex{Frankfurter Zeitung@Frankfurter Zeitung|pw}\hfill \textcolor{gray}{\textbf{\textbf{Frankfurt a. M.\oindex{Frankfurt am Main@\textbf{Frankfurt am Main}, \emph{P.PPLA3}|pw},}}}{ }29. April \textcolor{gray}{\textbf{189}}9.\pend
           
\pstart
           \textcolor{gray}{\textbf{und}}\pend
           
\pstart
           \textcolor{gray}{\textbf{Handelsblatt.}}\pend
           
\pstart
           \textcolor{gray}{\textbf{\textbf{Redaktion\orgindex{Frankfurter Zeitung@Frankfurter Zeitung|pwv}.}\noindent{}\textcolor{gray}{\textbf{Für die Redaktion\orgindex{Frankfurter Zeitung@Frankfurter Zeitung|pwv} beſtimmte Briefe und Sendungen wolle man
                                 \so{nicht} an die Perſon eines Redakteurs,
                              ſondern ſtets \textbf{an die Redaktion der Frankfurter Zeitung\orgindex{Frankfurter Zeitung@Frankfurter Zeitung|pw}} adreſſiren. }}}}\pend
           
\pstart
           \textcolor{gray}{\textbf{Telegramm-Adreſſe:}}\pend
           
\pstart
           \textcolor{gray}{\textbf{\textbf{Zeitung\orgindex{Frankfurter Zeitung@Frankfurter Zeitung|pwv}{ }Frankfurt Main\oindex{Frankfurt am Main@\textbf{Frankfurt am Main}, \emph{P.PPLA3}|pw}.}}}\pend
           
\pstart{}Mein lieber Freund,\pend\vspace{0.5em}
\pstart
           Dank für Deine Karte, die mich ſehr beruhigt hat. Ich bin recht froh, Dich in \label{K_L02873-1v}\edtext{Berlin\oindex{Berlin@\textbf{Berlin}, \emph{P.PPLC}|pw}}{\lemma{\textnormal{\emph{Berlin}}}\Cendnote{\textnormal{Schnitzler war für den Zeitraum vom 25. 4. 1899 bis zum 2. 5. 1899 aus Anlass der
                  Premiere von \emph{Der grüne Kakadu –
                     Paracelsus – Die Gefährtin. Drei Einakter}\pwindex{gruene Kakadu – Paracelsus – Die Gefaehrtin. Drei Einakter@\emph{Der grüne Kakadu – Paracelsus – Die Gefährtin. Drei Einakter}|pwk}  nach Berlin\oindex{Berlin@\textbf{Berlin}, \emph{P.PPLC}|pwk} gereist. Diese fand
                  am 29. 4. 1899 am \emph{Deutschen Theater}\orgindex{Deutsches Theater Berlin@Deutsches Theater Berlin|pwk} statt.}}}\label{K_L02873-1} zu wiſſen. Mein Brief erreicht Dich
               jedenfalls am Morgen nach einem neuen großen Erfolge\pwindex{gruene Kakadu – Paracelsus – Die Gefaehrtin. Drei Einakter@\emph{Der grüne Kakadu – Paracelsus – Die Gefährtin. Drei Einakter}|pwv} und ſoll Dir auch gleich meinen
               Glückwunſch bringen.\pend
           
\pstart
           Nochmals, bitte: \label{K_L02873-2v}\edtext{\uuline{komm’ nach Frankfurt\oindex{Frankfurt am Main@\textbf{Frankfurt am Main}, \emph{P.PPLA3}|pw}}}{\lemma{\textnormal{\emph{komm’ nach Frankfurt}}}\Cendnote{\textnormal{Dazu kam es nicht.}}}\label{K_L02873-2}! Die \label{K_L02873-3v}\edtext{\textsc{Dreyfus\pwindex{Dreyfus, Alfred 1859-10-09 – 1935-07-12@\textsc{Dreyfus, Alfred} (1859-10-09 – 1935-07-12), \emph{Militär/Militärin}|pw}-\begin{otherlanguage}{french}Enquête\end{otherlanguage}}}{\lemma{\textnormal{\emph{Dreyfus-Enquête}}}\Cendnote{\textnormal{Siehe Paul Goldmann an Arthur Schnitzler, 26. 4. 1899.
               }}}\label{K_L02873-3} geht dieſe Woche zu Ende. Nächſte Woche werde ich ſicherlich mehr Zeit haben.
               Wenn Du da biſt, kann ich mich jeden Nachmittag von 5 Uhr ab freimachen.
               Du brauchſt Dich doch wirklich nicht ſo zu eilen, nach Wien\oindex{Wien@\textbf{Wien}, \emph{A.ADM2}|pw} zurückzukommen. Je länger Du fortbleibſt, umſo beſſer iſt es. Und vor
               ein {\pb}paar Stunden Eiſenbahnfahrt mehr wirſt Du Dich
               doch gewiß nicht fürchten.\pend
           
\pstart
           Was mich anlangt, ſo dringe ich deshalb ſo ſehr darauf, Dich jetzt zu ſehen, weil ich
               keine Ahnung habe, ob ich in dieſem Jahr\textcolor{gray}{e} überhaupt Urlaub
               bekomme. Die Redaktion\orgindex{Frankfurter Zeitung@Frankfurter Zeitung|pwv} hat eine
               Reihe von Reiſemiſſionen für mich in Ausſicht, und es iſt nicht unmöglich, daß ſie
               den ganzen Sommer und Herbſt füllen. Laß’ Dich erbitten und komm’ her! Wenn ich nicht
               Zeit habe, wirſt Du bei meinem Schwager\pwindex{Rosengart, Josef 1860-02-08 – 1927-08-04@\textsc{Rosengart, Josef} (1860-02-08 – 1927-08-04), \emph{Arzt/Ärztin}|pwv}, meiner Schweſter\pwindex{Rosengart, Vally 1866-12-29 – nach 1926@\textsc{Rosengart, Vally} (1866-12-29 – nach 1926)|pwv}, meinem Onkel\pwindex{Mamroth, Fedor 21.02.1851 – 25.06.1907@\textsc{Mamroth, Fedor} (21.02.1851 – 25.06.1907), \emph{Journalist/Journalistin, Kritiker/Kritikerin}|pwv} ſein. Allein werden wir Dich ſchon nicht laſſen. Auch ſonſt wirſt du
               hier Den und Jenen kennen lernen, der Dich \strikeout{h\textcolor{gray}{×}\-\textcolor{gray}{×}\-\textcolor{gray}{×}\-\textcolor{gray}{×}} intereſſiren wird. Bitte, bitte, komm’ hierher!\pend
           
\pstart
           Viele treue Grüße! {\\[\baselineskip]}Dein \spacefill\mbox{Paul Goldmann.}\pend
           \leftskip=0em{}\selectlanguage{ngerman}\endnumbering\briefempfaengerindex{Schnitzler, Arthur@\textsc{Schnitzler, Arthur}!zzzGoldmann, Paul@\emph{von Paul Goldmann}!1899-04-291@{29. 4. 1899}|)be}\mylabel{L02873h}  \normalsize

\doendnotes{C}
\bigskip
\vfill

\clearpage

\footnotesize

\lohead{\textsc{register}}

% Definiere theindex-Environment komplett neu ohne reledmac
\makeatletter
\renewenvironment{theindex}{%
  \section*{\indexname}%
  \setlength{\parindent}{0pt}%
  \setlength{\parskip}{0pt plus 0.3pt}%
  \let\item\@idxitem
}{%
  \clearpage
}
\makeatother

\IfFileExists{\jobname-pw.ind}{\input{\jobname-pw.ind}}{}

\end{document}

      