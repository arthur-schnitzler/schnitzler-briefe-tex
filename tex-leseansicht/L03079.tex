%% latex-leseansicht-vorspann.tex
%% Vorspann für die Leseansicht.
%% Lädt die gemeinsame Datei latex-vorspann.tex mit nicht gesetztem Schalter.

\newif\ifkorrekturansicht
\korrekturansichtfalse

\input{../tex-inputs/latex-vorspann}


\section[ Paul Goldmann an Arthur Schnitzler, 15. 8. {[}1901{]}]{L03079 Paul Goldmann an Arthur Schnitzler,  15. 8. [1901]}
\nopagebreak\mylabel{L03079v}
\rehead{ }\normalsize\beginnumbering\briefempfaengerindex{Schnitzler, Arthur@\textsc{Schnitzler, Arthur}!zzzGoldmann, Paul@\emph{von Paul Goldmann}!1901-08-151@{15. 8. [1901]}|(be}
\toendnotes[C]{\smallbreak\pagebreak[2]}
\correspDesc{Versand  durch Paul Goldmann am 15. 8. [1901] in Riva del Garda
\newline{}Übermittlung  am 16. 8. [1901] in Riva del Garda
\newline{}Erhalt  durch Arthur Schnitzler am 17. 8. [19]01 in Welsberg-Taisten}\toendnotes[C]{\smallbreak}
\Standort{DLA, A:Schnitzler, HS.NZ85.1.3171.}
\physDesc{Bildpostkarte, 263 Zeichen
\newline{}Handschrift: schwarze Tinte, deutsche Kurrent
\newline{}Versand: 1) Stempel: »\nobreak{}\oindex{Mori@\textbf{Mori}, \emph{Hauptstadt}|pwk}Mori, \textcolor{gray}{16}. 8\nobreak{}«.   2) Stempel: »\nobreak{}\oindex{Welsberg-Taisten@\textbf{Welsberg-Taisten}, \emph{Verwaltungsgebiet}|pwk}Wel{[}sbe{]}rg, 17. 8. 01\nobreak{}«. 
\newline{}Schnitzler: mit Bleistift das Jahr »901« vermerkt }\toendnotes[C]{\smallbreak}\pstart{}\textsc{{\pb}Herrn}\pend{}\pstart{}\textsc{Dr. Arthur Schnitzler}\pend{}\pstart{}\textsc{Welsberg im Pusterthal\oindex{Welsberg-Taisten@\textbf{Welsberg-Taisten}, \emph{Verwaltungsgebiet}|pw}}\pend{}\pstart{}\textsc{Wildbad-Hôtel\oindex{Wildbad Waldbrunn@\textbf{Wildbad Waldbrunn}, \emph{Spa}|pw}.}\pend{}{\bigskip}
\pstart
           \noindent{}\centering{}{\pb}\textcolor{gray}{\textbf{PALAST HÔTEL LIDO\oindex{Palast Hotel Lido@\textbf{Palast Hotel Lido}, \emph{Hotel}|pw}, RIVA \textsuperscript{A}/GARDASEE\oindex{Riva del Garda@\textbf{Riva del Garda}, \emph{Hauptstadt}|pw}.}}\pend
           \vspace{1em}
\pstart
           {\pb}15. Auguſt.\pend
           \vspace{0.5em}
\pstart
           Es iſt kühl und \textcolor{gray}{kö}ſtlich hier. Aber einſam. Bitte, lieber \textsc{Arthur}, hol’ auch meine \textsc{Poste
                  restante} Briefe von der Poſt u. heb’{ }ſie mir auf, damit ich{ }ſie \label{K_L03079-1v}\edtext{Sonntag}{\lemma{\textnormal{\emph{Sonntag}}}\Cendnote{\textnormal{Am Sonntag, dem 18. 8. 1901 kam Goldmann\pwindex{Goldmann, Paul 31.\,1.\,1865 Breslau – 25.\,9.\,1935 Wien@\textsc{Goldmann, Paul} (31.\,1.\,1865 Breslau – 25.\,9.\,1935 Wien), \emph{Schriftsteller, Journalist}|pwk} in
                     Welsberg\oindex{Welsberg-Taisten@\textbf{Welsberg-Taisten}, \emph{Verwaltungsgebiet}|pwk} an. Schnitzler war bereits seit 15. 8. 1901 dort. Am 26. 8. 1901 reiste Schnitzler weiter nach Villach\oindex{Villach@\textbf{Villach}, \emph{Verwaltungsgebiet}|pwk}, Goldmann\pwindex{Goldmann, Paul 31.\,1.\,1865 Breslau – 25.\,9.\,1935 Wien@\textsc{Goldmann, Paul} (31.\,1.\,1865 Breslau – 25.\,9.\,1935 Wien), \emph{Schriftsteller, Journalist}|pwk}
                  nach Pörtschach\oindex{Pörtschach am Wörthersee@\textbf{Pörtschach am Wörthersee}|pwk}.}}}\label{K_L03079-1} finde. Herzliche
               Grüße an alle! \spacefill\mbox{P. G.}\pend
           \selectlanguage{ngerman}\endnumbering\briefempfaengerindex{Schnitzler, Arthur@\textsc{Schnitzler, Arthur}!zzzGoldmann, Paul@\emph{von Paul Goldmann}!1901-08-151@{15. 8. [1901]}|)be}\mylabel{L03079h}  \newcommand{\dateiname}{L03079}\newcommand{\titel}{Paul Goldmann an Arthur Schnitzler, 15. 8. [1901]}\newcommand{\editorInnen}{Martin Anton Müller und Laura Untner}%% latex-leseansicht-abspann.tex
%% Abspann für die Leseansicht.
%% Der Schalter \ifkorrekturansicht ist bereits durch den Vorspann gesetzt.

%% latex-abspann.tex
%% Gemeinsamer Abspann für Korrekturansicht und Leseansicht.
%% Setzt den Schalter \ifkorrekturansicht voraus (gesetzt in den
%% einbindenden Dateien latex-korrekturansicht-abspann.tex bzw.
%% latex-leseansicht-abspann.tex).
%% ---------------------------------------------------------------

\normalsize

% Das esempio-Environment wird nur in der Leseansicht benötigt
\ifkorrekturansicht\else
\newenvironment{esempio}[3]%
{
    \vspace{1.5ex}
    \rlap{\underline{#1}}
    \par
    \setlength{\parindent}{0cm}
    \nopagebreak
    \leftskip=#2cm
    \rightskip=#3cm
}
{
    \par
}
\fi

\doendnotes{C}
\bigskip
\vfill

\clearpage

\footnotesize

\ifkorrekturansicht
  \lohead{\textsc{register}}
\fi

% theindex-Environment neu definieren ohne reledmac
\makeatletter
\renewenvironment{theindex}{%
  \ifkorrekturansicht
    \section*{\indexname}%
  \else
    \subsubsection*{Index der erwähnten Entitäten}%
  \fi
  \setlength{\parindent}{0pt}%
  \setlength{\parskip}{0pt plus 0.3pt}%
  \let\item\@idxitem
}{%
  \ifkorrekturansicht\clearpage\fi
}
\makeatother

\IfFileExists{\jobname-pw.ind}{\input{\jobname-pw.ind}}{}

% Quellenangabe nur in der Leseansicht
\ifkorrekturansicht\else
% Fallback-Definitionen, falls die .tex-Datei \titel etc. nicht gesetzt hat
\providecommand{\titel}{}
\providecommand{\editorInnen}{}
\providecommand{\dateiname}{\jobname}

\vspace{3cm}

\vfill

\footnotesize
\textsc{Quelle}: \titel. Herausgegeben von {\editorInnen}. In: \emph{Arthur Schnitzler: Briefwechsel mit Autorinnen und Autoren}.
 Digitale Edition, https://schnitzler-briefe.acdh.oeaw.ac.at/{\dateiname}.html (Stand \today)
\fi

\end{document}


