%% latex-korrekturansicht-vorspann.tex
%% Vorspann für die Korrekturansicht.
%% Lädt die gemeinsame Datei latex-vorspann.tex mit gesetztem Schalter.

\newif\ifkorrekturansicht
\korrekturansichttrue

\input{../tex-inputs/latex-vorspann}


\section[ Paul Goldmann an Arthur Schnitzler, 15. 8. {[}1901{]}]{L03079 Paul Goldmann an Arthur Schnitzler, 15. 8. {[}1901{]}}
\nopagebreak\mylabel{L03079v}
\rehead{ }\normalsize\beginnumbering\briefempfaengerindex{Schnitzler, Arthur@\textsc{Schnitzler, Arthur}!zzzGoldmann, Paul@\emph{von Paul Goldmann}!1901-08-151@{15. 8. {[}1901{]}}|(be}
\toendnotes[C]{\smallbreak\pagebreak[2]}\Standort{DLA, A:Schnitzler, HS.NZ85.1.3171.}
\physDesc{Bildpostkarte, 263 Zeichen
\newline{}Handschrift: 1) schwarze Tinte, deutsche Kurrent\hspace{1em}2) schwarze Tinte, lateinische Kurrent (\noindent{}Adresse)\hspace{1em}
\newline{}Versand: 1) Stempel: »\nobreak{}\oindex{Mori@\textbf{Mori}, \emph{P.PPLA3}|pwk}Mori, \textcolor{gray}{16}. 8\nobreak{}«.   2) Stempel: »\nobreak{}\oindex{Welsberg-Taisten@\textbf{Welsberg-Taisten}, \emph{A.ADM3}|pwk}Wel{[}sbe{]}rg, 17. 8. 01\nobreak{}«. 
\newline{}Schnitzler: mit Bleistift das Jahr »901« vermerkt }\toendnotes[C]{\smallbreak}\pstart{}{\pb}Herrn\pend{}\pstart{}Dr. Arthur Schnitzler\pend{}\pstart{}Welsberg im Pusterthal\oindex{Welsberg-Taisten@\textbf{Welsberg-Taisten}, \emph{A.ADM3}|pw}\pend{}\pstart{}Wildbad-Hôtel\oindex{Wildbad Waldbrunn@\textbf{Wildbad Waldbrunn}, \emph{S.SPA}|pw}.\pend{}{\bigskip}
\pstart
           \noindent{}\centering{}{\pb}\textcolor{gray}{\textbf{PALAST HÔTEL LIDO\oindex{Palast Hotel Lido@\textbf{Palast Hotel Lido}, \emph{Hotel (K.HTL)}|pw}, RIVA \textsuperscript{A}/GARDASEE\oindex{Riva del Garda@\textbf{Riva del Garda}, \emph{P.PPLA3}|pw}.}}\pend
           \vspace{1em}
\pstart
           {\pb}15. Auguſt.\pend
           \vspace{0.5em}
\pstart
           Es iſt kühl und \textcolor{gray}{kö}ſtlich hier. Aber einſam. Bitte, lieber \textsc{Arthur}, hol’ auch meine \textsc{Poste
                  restante} Briefe von der Poſt u. heb’ ſie mir auf, damit ich ſie \label{K_L03079-1v}\edtext{Sonntag}{\lemma{\textnormal{\emph{Sonntag}}}\Cendnote{\textnormal{Am Sonntag, dem 18. 8. 1901 kam Goldmann\pwindex{Goldmann, Paul 31.01.1865 – 25.09.1935@\textsc{Goldmann, Paul} (31.01.1865 – 25.09.1935), \emph{Schriftsteller/Schriftstellerin, Journalist/Journalistin}|pwk} in
                     Welsberg\oindex{Welsberg-Taisten@\textbf{Welsberg-Taisten}, \emph{A.ADM3}|pwk} an. Schnitzler war bereits seit 15. 8. 1901 dort. Am 26. 8. 1901 reiste Schnitzler weiter nach Villach\oindex{Villach@\textbf{Villach}, \emph{A.ADM3}|pwk}, Goldmann\pwindex{Goldmann, Paul 31.01.1865 – 25.09.1935@\textsc{Goldmann, Paul} (31.01.1865 – 25.09.1935), \emph{Schriftsteller/Schriftstellerin, Journalist/Journalistin}|pwk}
                  nach Pörtschach\oindex{Poertschach am Woerthersee@\textbf{Pörtschach am Wörthersee}, \emph{P.PPL}|pwk}.}}}\label{K_L03079-1} finde. Herzliche
               Grüße an alle! \spacefill\mbox{P. G.}\pend
           \selectlanguage{ngerman}\endnumbering\briefempfaengerindex{Schnitzler, Arthur@\textsc{Schnitzler, Arthur}!zzzGoldmann, Paul@\emph{von Paul Goldmann}!1901-08-151@{15. 8. {[}1901{]}}|)be}\mylabel{L03079h}  \normalsize

\doendnotes{C}
\bigskip
\vfill

\clearpage

\footnotesize

\lohead{\textsc{register}}

% Definiere theindex-Environment komplett neu ohne reledmac
\makeatletter
\renewenvironment{theindex}{%
  \section*{\indexname}%
  \setlength{\parindent}{0pt}%
  \setlength{\parskip}{0pt plus 0.3pt}%
  \let\item\@idxitem
}{%
  \clearpage
}
\makeatother

\IfFileExists{\jobname-pw.ind}{\input{\jobname-pw.ind}}{}

\end{document}

      