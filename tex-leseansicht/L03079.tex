%% latex-leseansicht-vorspann.tex
%% Vorspann für die Leseansicht.
%% Lädt die gemeinsame Datei latex-vorspann.tex mit nicht gesetztem Schalter.

\newif\ifkorrekturansicht
\korrekturansichtfalse

\input{../tex-inputs/latex-vorspann}


         \renewcommand{\erwaehnteOrte}{Orte: Mori, Palast Hotel Lido, Pörtschach, Riva del Garda, Villach, Welsberg-Taisten, Wildbad Waldbrunn}
         \renewcommand{\erwaehnteWerke}{}
               \section[ Paul Goldmann an Arthur Schnitzler, 15. 8. {[}1901{]}]{ Paul Goldmann an Arthur Schnitzler, 15. 8. {[}1901{]}}\nopagebreak\mylabel{v}\rehead{ }\begin{ledgroupsized}[t]{13cm}\normalsize\beginnumbering \toendnotes[C]{\smallbreak\pagebreak[2]} \Standort{DLA, A:Schnitzler, HS.NZ85.1.3171.}
\physDesc{Bildpostkarte, 263 Zeichen
\newline{}Handschrift: 1) schwarze Tinte, deutsche Kurrent\hspace{1em}2) schwarze Tinte, lateinische Kurrent (\noindent{}Adresse)\hspace{1em}
\newline{}Versand: 1) Stempel: »\nobreak{}\oindex{Mori@\textbf{Mori}|pwk}Mori, \textcolor{gray}{16}. 8\nobreak{}«.   2) Stempel: »\nobreak{}\oindex{Welsberg-Taisten@\textbf{Welsberg-Taisten}|pwk}Wel{[}sbe{]}rg, 17. 8. 01\nobreak{}«. 
\newline{}Schnitzler: mit Bleistift das Jahr »{[}1{]}901« vermerkt }\toendnotes[C]{\smallbreak}\pstart{}{\pb}Herrn\pend{}\pstart{}Dr. Arthur Schnitzler\pend{}\pstart{}Welsberg im Pusterthal\oindex{Welsberg-Taisten@\textbf{Welsberg-Taisten}|pw}\pend{}\pstart{}Wildbad-Hôtel\oindex{Wildbad Waldbrunn@\textbf{Wildbad Waldbrunn}|pw}.\pend{}{\bigskip}\pstart
           \noindent{}\centering{}{\pb}\textcolor{gray}{\textbf{PALAST HÔTEL LIDO\oindex{Palast Hotel Lido@\textbf{Palast Hotel Lido}|pw}, RIVA \textsuperscript{A}/GARDASEE\oindex{Riva del Garda@\textbf{Riva del Garda}|pw}.}}\pend
           \pstart
           15. Auguſt.\pend
           \pstart
           Es iſt kühl und \textcolor{gray}{kö}ſtlich hier. Aber einſam. Bitte, lieber \textsc{Arthur}, hol’ auch meine \textsc{Poste
                  restante} Briefe von der Poſt u. heb’ ſie mir auf, damit ich ſie \label{K_L03079-1v}\edtext{Sonntag}{\lemma{\textnormal{\emph{Sonntag}}}\Cendnote{\textnormal{Am Sonntag, dem 18. 8. 1901, kam Goldmann\pwindex{Goldmann, Paul 31.01.1865 – 25.09.1935@\textsc{Goldmann, Paul} (31.01.1865 – 25.09.1935), \emph{Schriftsteller, Journalist}|pwk} in
                     Welsberg\oindex{Welsberg-Taisten@\textbf{Welsberg-Taisten}|pwk} an. Schnitzler\pwindex{Schnitzler, Arthur 15.05.1862 – 21.10.1931@\textsc{Schnitzler, Arthur} (15.05.1862 – 21.10.1931), \emph{Schriftsteller, Mediziner}|pwk} war bereits seit 15. 8. 1901 dort. Am 26. 8. 1901 reiste Schnitzler\pwindex{Schnitzler, Arthur 15.05.1862 – 21.10.1931@\textsc{Schnitzler, Arthur} (15.05.1862 – 21.10.1931), \emph{Schriftsteller, Mediziner}|pwk} weiter nach Villach\oindex{Villach@\textbf{Villach}|pwk}, Goldmann\pwindex{Goldmann, Paul 31.01.1865 – 25.09.1935@\textsc{Goldmann, Paul} (31.01.1865 – 25.09.1935), \emph{Schriftsteller, Journalist}|pwk}
                  nach Pörtschach\oindex{Poertschach@\textbf{Pörtschach}|pwk}.}}}\label{K_L03079-1h} finde. Herzliche
               Grüße an alle! \spacefill\mbox{P. G.}\pend
           
         
         \endnumbering\mylabel{h}\end{ledgroupsized}  \newcommand{\dateiname}{L03079}\newcommand{\titel}{Paul Goldmann an Arthur Schnitzler, 15. 8. [1901]}\newcommand{\editorInnen}{Martin Anton Müller und Laura Untner}%% latex-leseansicht-abspann.tex
%% Abspann für die Leseansicht.
%% Der Schalter \ifkorrekturansicht ist bereits durch den Vorspann gesetzt.

%% latex-abspann.tex
%% Gemeinsamer Abspann für Korrekturansicht und Leseansicht.
%% Setzt den Schalter \ifkorrekturansicht voraus (gesetzt in den
%% einbindenden Dateien latex-korrekturansicht-abspann.tex bzw.
%% latex-leseansicht-abspann.tex).
%% ---------------------------------------------------------------

\normalsize

% Das esempio-Environment wird nur in der Leseansicht benötigt
\ifkorrekturansicht\else
\newenvironment{esempio}[3]%
{
    \vspace{1.5ex}
    \rlap{\underline{#1}}
    \par
    \setlength{\parindent}{0cm}
    \nopagebreak
    \leftskip=#2cm
    \rightskip=#3cm
}
{
    \par
}
\fi

\doendnotes{C}
\bigskip
\vfill

\clearpage

\footnotesize

\ifkorrekturansicht
  \lohead{\textsc{register}}
\fi

% theindex-Environment neu definieren ohne reledmac
\makeatletter
\renewenvironment{theindex}{%
  \ifkorrekturansicht
    \section*{\indexname}%
  \else
    \subsubsection*{Index der erwähnten Entitäten}%
  \fi
  \setlength{\parindent}{0pt}%
  \setlength{\parskip}{0pt plus 0.3pt}%
  \let\item\@idxitem
}{%
  \ifkorrekturansicht\clearpage\fi
}
\makeatother

\IfFileExists{\jobname-pw.ind}{\input{\jobname-pw.ind}}{}

% Quellenangabe nur in der Leseansicht
\ifkorrekturansicht\else
% Fallback-Definitionen, falls die .tex-Datei \titel etc. nicht gesetzt hat
\providecommand{\titel}{}
\providecommand{\editorInnen}{}
\providecommand{\dateiname}{\jobname}

\vspace{3cm}

\vfill

\footnotesize
\textsc{Quelle}: \titel. Herausgegeben von {\editorInnen}. In: \emph{Arthur Schnitzler: Briefwechsel mit Autorinnen und Autoren}.
 Digitale Edition, https://schnitzler-briefe.acdh.oeaw.ac.at/{\dateiname}.html (Stand \today)
\fi

\end{document}


      