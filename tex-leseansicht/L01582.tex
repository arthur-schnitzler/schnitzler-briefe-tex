%% latex-korrekturansicht-vorspann.tex
%% Vorspann für die Korrekturansicht.
%% Lädt die gemeinsame Datei latex-vorspann.tex mit gesetztem Schalter.

\newif\ifkorrekturansicht
\korrekturansichttrue

\input{../tex-inputs/latex-vorspann}


\section[Arthur Schnitzler: Widmungsexemplar Ruf des Lebens für Hugo von Hofmannsthal, {[}21.?{]} 2. 1906]{L01582 Arthur Schnitzler: Widmungsexemplar Ruf des Lebens für Hugo von
               Hofmannsthal, {[}21.?{]} 2. 1906}
\nopagebreak\mylabel{L01582v}
\rehead{ }\normalsize\beginnumbering\briefempfaengerindex{Hofmannsthal, Hugo von@\textsc{Hofmannsthal, Hugo von}!zzzSchnitzler, Arthur@\emph{von Arthur Schnitzler}!1906-02-211@{{[}21.?{]} 2. 1906}|(be}
\toendnotes[C]{\smallbreak\pagebreak[2]}\Standort{FDH, FDH 3232.}
\physDesc{Widmung am Vorsatzblatt, 46 Zeichen
\newline{}Handschrift: schwarze Tinte, deutsche Kurrent}
\buchAbdrucke{\weitereDrucke{Hugo von Hofmannsthal: \emph{Bibliothek}. Frankfurt am Main: \emph{S. Fischer} 2011, S. 605.} }\toendnotes[C]{\smallbreak}
\pstart
           \noindent{}{\pb}Meinem lieben Hugo\pend
           \pstart \spacefill\mbox{ArthurSch}\pend{}
\pstart
           Berlin\oindex{Berlin@\textbf{Berlin}, \emph{P.PPLC}|pw}{ }\label{K_L01582-1v}\edtext{Feber 906}{\lemma{\textnormal{\emph{Feber 906}}}\Cendnote{\textnormal{\emph{Der Ruf des Lebens}\pwindex{Ruf des Lebens. Schauspiel in drei Akten@\emph{Der Ruf des Lebens. Schauspiel in drei Akten}|pwk} wurde am 13. 2. 1906 vom \emph{Börsenblatt für den deutschen Buchhandel}\pwindex{Boersenblatt fuer den Deutschen Buchhandel@\emph{Börsenblatt für den Deutschen Buchhandel}|pwk} als Neuerscheinung gemeldet.
                     Schnitzler war vom 4. 2. 1906 bis zum
                     7. 2. 1906
                     und vom 18. 2. 1906 bis zum 27. 2. 1906 in Berlin\oindex{Berlin@\textbf{Berlin}, \emph{P.PPLC}|pwk}. Es ist anzunehmen, dass er
                     die Widmungen während der ersten Reise eintrug.}}}\label{K_L01582-1}.\pend
           \selectlanguage{ngerman}\vspace{1em}{\vspace{1\baselineskip}}
\pstart
           \centering{}{\pb}\textcolor{gray}{\textbf{\textbf{Der Ruf des Lebens\pwindex{Ruf des Lebens. Schauspiel in drei Akten@\emph{Der Ruf des Lebens. Schauspiel in drei Akten}|pw}}}}\pend
           
\pstart
           \centering{}\textcolor{gray}{\textbf{\so{Schauſpiel in drei Akten von}}}{\\}\textcolor{gray}{\textbf{\textbf{\so{Arthur Schnitzler}}}}\pend
           {\vspace{1\baselineskip}}
\pstart
           \centering{}\textcolor{gray}{\textbf{S. Fiſcher, Verlag\orgindex{S. Fischer Verlag@S. Fischer Verlag|pw}, Berlin\oindex{Berlin@\textbf{Berlin}, \emph{P.PPLC}|pw}}}\pend
           
\pstart
           \centering{}\textcolor{gray}{\textbf{1906}}\pend
           \selectlanguage{ngerman}\endnumbering\briefempfaengerindex{Hofmannsthal, Hugo von@\textsc{Hofmannsthal, Hugo von}!zzzSchnitzler, Arthur@\emph{von Arthur Schnitzler}!1906-02-211@{{[}21.?{]} 2. 1906}|)be}\mylabel{L01582h}  \normalsize

\doendnotes{C}
\bigskip
\vfill

\clearpage

\footnotesize

\lohead{\textsc{register}}

% Definiere theindex-Environment komplett neu ohne reledmac
\makeatletter
\renewenvironment{theindex}{%
  \section*{\indexname}%
  \setlength{\parindent}{0pt}%
  \setlength{\parskip}{0pt plus 0.3pt}%
  \let\item\@idxitem
}{%
  \clearpage
}
\makeatother

\IfFileExists{\jobname-pw.ind}{\input{\jobname-pw.ind}}{}

\end{document}

      