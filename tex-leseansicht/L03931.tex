%% latex-leseansicht-vorspann.tex
%% Vorspann für die Leseansicht.
%% Lädt die gemeinsame Datei latex-vorspann.tex mit nicht gesetztem Schalter.

\newif\ifkorrekturansicht
\korrekturansichtfalse

\input{../tex-inputs/latex-vorspann}


\section[Arthur Schnitzler an Theodor Herzl, 21. 6. 1895]{L03931 Arthur Schnitzler an Theodor Herzl, 21. 6. 1895}
\nopagebreak\mylabel{L03931v}
\rehead{ }\normalsize\beginnumbering\briefempfaengerindex{Herzl, Theodor@\textsc{Herzl, Theodor}!zzzSchnitzler, Arthur@\emph{von Arthur Schnitzler}!1895-06-211@{21. 6. 1895}|(be}
\toendnotes[C]{\smallbreak\pagebreak[2]}
\correspDesc{Versand  durch Arthur Schnitzler am 21. 6. 1895 in Wien
\newline{}Erhalt  durch Theodor Herzl in Wien}\toendnotes[C]{\smallbreak}
\Standort{Jerusalem, Central Zionist Archives, H1:1925-16.}
\physDesc{,  Blätter,  Seiten
\newline{}Handschrift: , deutsche Kurrent}\toendnotes[C]{\smallbreak}
\pstart{}{\pb}Lieber Freund,\pend\vspace{0.5em}
\pstart
           haben Sie{ }ſich mit Pr.\oindex{Prag@\textbf{Prag}, \emph{Land}|pw}\pwindex{Teweles, Heinrich 13.\,11.\,1856 Prag – 9.\,8.\,1927 Prein an der Rax@\textsc{Teweles, Heinrich} (13.\,11.\,1856 Prag – 9.\,8.\,1927 Prein an der Rax), \emph{Schriftsteller, Journalist, Theaterleiter}|pwv} vielleicht direct in Verbindung
               geſetzt? An Doctor B.\pwindex{Baumgarten, Julius 26.\,5.\,1860 Wien – 28.\,8.\,1934 ebd.@\textsc{Baumgarten, Julius} (26.\,5.\,1860 Wien – 28.\,8.\,1934 ebd.), \emph{Anwalt}|pw} oder an mich iſt bisher keine
               Nachricht gelangt. Liegt etwas neues vor,{ }ſo wiſſen Sie, daſs es mich beträchtlich
               intereſſirt; und {\pb}ich bitte Sie, mir was darüber
               zu ſagen. –\pend
           
\pstart
           Haben Sie{ }ſich den Sommer{ }ſchon eingerichtet? Ich werde Anfangs{ }Juli abreiſen und Sie jedenfalls verſtändigen, wo mich Nachrichten
               treffen;{ }ſollten Sie aber noch vorher Zeit haben, mir mit zwei Worten mitzutheilen,
               wie {\pb}es Ihnen geht und was Sie treiben,{ }ſo wird es mich
               herzlich freuen.\pend
           
\pstart
           Mit vielen Grüßen{\\[\baselineskip]}Ihr{ }ſehr ergebner{\\[\baselineskip]}\spacefill\mbox{ArthSchn}\pend
           \leftskip=0em{}
\pstart
           21. 6. 95.\pend
           
\pstart
           Wien\oindex{Wien@\textbf{Wien}, \emph{Verwaltungsgebiet}|pw}\pend
           \selectlanguage{ngerman}\endnumbering\briefempfaengerindex{Herzl, Theodor@\textsc{Herzl, Theodor}!zzzSchnitzler, Arthur@\emph{von Arthur Schnitzler}!1895-06-211@{21. 6. 1895}|)be}\mylabel{L03931h}
\begin{anhang}
\end{anhang}\newcommand{\dateiname}{L03931}\newcommand{\titel}{Arthur Schnitzler an Theodor Herzl, 21. 6. 1895}\newcommand{\editorInnen}{Herausgegeben von Jahnke, SelmaMüller, Martin Anton}%% latex-leseansicht-abspann.tex
%% Abspann für die Leseansicht.
%% Der Schalter \ifkorrekturansicht ist bereits durch den Vorspann gesetzt.

%% latex-abspann.tex
%% Gemeinsamer Abspann für Korrekturansicht und Leseansicht.
%% Setzt den Schalter \ifkorrekturansicht voraus (gesetzt in den
%% einbindenden Dateien latex-korrekturansicht-abspann.tex bzw.
%% latex-leseansicht-abspann.tex).
%% ---------------------------------------------------------------

\normalsize

% Das esempio-Environment wird nur in der Leseansicht benötigt
\ifkorrekturansicht\else
\newenvironment{esempio}[3]%
{
    \vspace{1.5ex}
    \rlap{\underline{#1}}
    \par
    \setlength{\parindent}{0cm}
    \nopagebreak
    \leftskip=#2cm
    \rightskip=#3cm
}
{
    \par
}
\fi

\doendnotes{C}
\bigskip
\vfill

\clearpage

\footnotesize

\ifkorrekturansicht
  \lohead{\textsc{register}}
\fi

% theindex-Environment neu definieren ohne reledmac
\makeatletter
\renewenvironment{theindex}{%
  \ifkorrekturansicht
    \section*{\indexname}%
  \else
    \subsubsection*{Index der erwähnten Entitäten}%
  \fi
  \setlength{\parindent}{0pt}%
  \setlength{\parskip}{0pt plus 0.3pt}%
  \let\item\@idxitem
}{%
  \ifkorrekturansicht\clearpage\fi
}
\makeatother

\IfFileExists{\jobname-pw.ind}{\input{\jobname-pw.ind}}{}

% Quellenangabe nur in der Leseansicht
\ifkorrekturansicht\else
% Fallback-Definitionen, falls die .tex-Datei \titel etc. nicht gesetzt hat
\providecommand{\titel}{}
\providecommand{\editorInnen}{}
\providecommand{\dateiname}{\jobname}

\vspace{3cm}

\vfill

\footnotesize
\textsc{Quelle}: \titel. Herausgegeben von {\editorInnen}. In: \emph{Arthur Schnitzler: Briefwechsel mit Autorinnen und Autoren}.
 Digitale Edition, https://schnitzler-briefe.acdh.oeaw.ac.at/{\dateiname}.html (Stand \today)
\fi

\end{document}


