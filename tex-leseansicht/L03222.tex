%% latex-korrekturansicht-vorspann.tex
%% Vorspann für die Korrekturansicht.
%% Lädt die gemeinsame Datei latex-vorspann.tex mit gesetztem Schalter.

\newif\ifkorrekturansicht
\korrekturansichttrue

\input{../tex-inputs/latex-vorspann}


\section[ Paul Goldmann an Arthur Schnitzler, 1. 9. {[}1902{]}]{L03222 Paul Goldmann an Arthur Schnitzler, 1. 9. {[}1902{]}}
\nopagebreak\mylabel{L03222v}
\rehead{ }\normalsize\beginnumbering\briefempfaengerindex{Schnitzler, Arthur@\textsc{Schnitzler, Arthur}!zzzGoldmann, Paul@\emph{von Paul Goldmann}!1902-09-011@{1. 9. {[}1902{]}}|(be}
\toendnotes[C]{\smallbreak\pagebreak[2]}\Standort{DLA, A:Schnitzler, HS.NZ85.1.3172.}
\physDesc{Brief, 1 Blatt, 2 Seiten, 800 Zeichen
\newline{}Handschrift: schwarze Tinte, deutsche Kurrent
\newline{}Schnitzler: mit rotem Buntstift eine Unterstreichung }\toendnotes[C]{\smallbreak}
\pstart
           \raggedleft{}{\pb}\textcolor{gray}{\textbf{\textsc{\label{K_L03222-1v}\edtext{G\textsuperscript{ds}{ }Hôtels}{\lemma{\textnormal{\emph{G\textsuperscript{ds} Hôtels}}}\Cendnote{\textnormal{\begin{otherlanguage}{french}Grands Hôtels\end{otherlanguage}}}}\label{K_L03222-1}}}}\pend
           
\pstart
           \raggedleft{}\textcolor{gray}{\textbf{\textbf{MONNEY\oindex{Grand Hôtel Monney@\textbf{Grand Hôtel Monney}, \emph{Hotel (K.HTL)}|pw}}}}\pend
           
\pstart
           \raggedleft{}\textcolor{gray}{\textbf{{\kaufmannsund}}}\pend
           
\pstart
           \textcolor{gray}{\textbf{A. MONNEY\pwindex{Monney, A. @\textsc{Monney, A.}, \emph{Hotelbesitzer/Hotelbesitzerin}|pw}}}\hfill \textcolor{gray}{\textbf{\textbf{BEAU SÉJOUR AU LAC\oindex{Grand Hôtel Beau Sejour au Lac@\textbf{Grand Hôtel Beau Séjour au Lac}, \emph{Hotel (K.HTL)}|pw}}}}\pend
           
\pstart
           \textcolor{gray}{\textbf{\label{K_L03222-2v}\edtext{\begin{otherlanguage}{french}PROPRIET\textsuperscript{RE}\end{otherlanguage}}{\lemma{\textnormal{\emph{Propriet\textsuperscript{RE}}}}\Cendnote{\textnormal{propriétaire, französisch:
                              Eigentümer/Eigentümerin}}}\label{K_L03222-2}}}\hfill \textcolor{gray}{\textbf{MONTREUX\oindex{Montreux@\textbf{Montreux}, \emph{Besiedelter Ort (A.BSO)}|pw} (\begin{otherlanguage}{french}SUISSE\oindex{Schweiz@\textbf{Schweiz}, \emph{A.PCLI}|pw}\end{otherlanguage})}}\pend
           
\pstart
           \centering{}\textcolor{gray}{\textbf{\label{K_L03222-3v}\edtext{\begin{otherlanguage}{french}Ascenseur hydraulique\end{otherlanguage}}{\lemma{\textnormal{\emph{Ascenseur hydraulique}}}\Cendnote{\textnormal{französisch: hydraulischer
                        Aufzug}}}\label{K_L03222-3}}}\pend
           
\pstart
           \textsc{Montreux\oindex{Montreux@\textbf{Montreux}, \emph{Besiedelter Ort (A.BSO)}|pw}}, 1. September.\pend
           
\pstart\center{}Mein lieber Freund,\pend\vspace{0.5em}
\pstart
           Mit jeder Poſt aus Berlin\oindex{Berlin@\textbf{Berlin}, \emph{P.PPLC}|pw} habe ich Deine lieben
               Nachrichten erwartet. Nachdem ſie heut wieder nicht
               eingelangt ſind, bin ich wirklich in Unruhe. Ich werde Donnerſtag in Frankfurt\oindex{Frankfurt am Main@\textbf{Frankfurt am Main}, \emph{P.PPLA3}|pw} ſein und bitte
               Dich ſehr, mir dorthin an die Adreſſe meines Schwagers\pwindex{Rosengart, Josef 1860-02-08 – 1927-08-04@\textsc{Rosengart, Josef} (1860-02-08 – 1927-08-04), \emph{Arzt/Ärztin}|pwv} (\textsc{Dr. Rosengart\pwindex{Rosengart, Josef 1860-02-08 – 1927-08-04@\textsc{Rosengart, Josef} (1860-02-08 – 1927-08-04), \emph{Arzt/Ärztin}|pw}}, \textsc{Reuterweg 59}\oindex{Reuterweg@\textbf{Reuterweg}, \emph{Straße (K.STR)}|pw}) zu ſchreiben, wie es Dir, dem Kinde\pwindex{Schnitzler, Heinrich 09.08.1902 – 12.07.1982@\textsc{Schnitzler, Heinrich} (09.08.1902 – 12.07.1982), \emph{Regisseur/Regisseurin, Schauspieler/Schauspielerin}|pwv} und der Mutter\pwindex{Schnitzler, Olga 17.01.1882 – 13.01.1970@\textsc{Schnitzler, Olga} (17.01.1882 – 13.01.1970), \emph{Schauspieler/Schauspielerin, Sänger/Sängerin}|pwv} geht?\pend
           
\pstart
           {\pb}Nach \textsc{Wien}\oindex{Wien@\textbf{Wien}, \emph{A.ADM2}|pw} komme ich nicht. Die Zeit iſt um, das Geld iſt alle. Ich habe ein ſehr ſchönes
               Stück Welt geſehen. In der Schweiz\oindex{Schweiz@\textbf{Schweiz}, \emph{A.PCLI}|pw} ſind die
               großartigen landſchaftlichen Eindrücke gehäufter, als in Tirol\oindex{Tirol@\textbf{Tirol}, \emph{A.ADM1}|pw}\oindex{Suedtirol@\textbf{Südtirol}, \emph{A.ADM2}|pw}, und leichter zu erreichen. \label{K_L03222-4v}\edtext{Nächſtes Jahr mußt Du hingehen}{\lemma{\textnormal{\emph{Nächſtes … hingehen}}}\Cendnote{\textnormal{Dazu kam es nicht.}}}\label{K_L03222-4}. Ich war \strikeout{die ga} 14 Tage lang mit meinem Onkel\pwindex{Mamroth, Fedor 21.02.1851 – 25.06.1907@\textsc{Mamroth, Fedor} (21.02.1851 – 25.06.1907), \emph{Journalist/Journalistin, Kritiker/Kritikerin}|pwv} zuſammen und habe in ihm einen überaus
               liebenswürdigen und anregenden Reiſekameraden gehabt.\pend
           
\pstart
           Viele treue Grüße! {\\[\baselineskip]}Dein {\\[\baselineskip]}\spacefill\mbox{Paul Goldm}\pend
           \leftskip=0em{}
\pstart
           \noindent{}Herzliche Grüße an \textsc{Olga\pwindex{Schnitzler, Olga 17.01.1882 – 13.01.1970@\textsc{Schnitzler, Olga} (17.01.1882 – 13.01.1970), \emph{Schauspieler/Schauspielerin, Sänger/Sängerin}|pw}}!\pend
           \selectlanguage{ngerman}\endnumbering\briefempfaengerindex{Schnitzler, Arthur@\textsc{Schnitzler, Arthur}!zzzGoldmann, Paul@\emph{von Paul Goldmann}!1902-09-011@{1. 9. {[}1902{]}}|)be}\mylabel{L03222h}  \normalsize

\doendnotes{C}
\bigskip
\vfill

\clearpage

\footnotesize

\lohead{\textsc{register}}

% Definiere theindex-Environment komplett neu ohne reledmac
\makeatletter
\renewenvironment{theindex}{%
  \section*{\indexname}%
  \setlength{\parindent}{0pt}%
  \setlength{\parskip}{0pt plus 0.3pt}%
  \let\item\@idxitem
}{%
  \clearpage
}
\makeatother

\IfFileExists{\jobname-pw.ind}{\input{\jobname-pw.ind}}{}

\end{document}

      