%% latex-leseansicht-vorspann.tex
%% Vorspann für die Leseansicht.
%% Lädt die gemeinsame Datei latex-vorspann.tex mit nicht gesetztem Schalter.

\newif\ifkorrekturansicht
\korrekturansichtfalse

\input{../tex-inputs/latex-vorspann}


\section[ Paul Goldmann an Arthur Schnitzler, 1. 9. {[}1902{]}]{L03222 Paul Goldmann an Arthur Schnitzler,  1. 9. [1902]}
\nopagebreak\mylabel{L03222v}
\rehead{ }\normalsize\beginnumbering\briefempfaengerindex{Schnitzler, Arthur@\textsc{Schnitzler, Arthur}!zzzGoldmann, Paul@\emph{von Paul Goldmann}!1902-09-011@{1. 9. [1902]}|(be}
\toendnotes[C]{\smallbreak\pagebreak[2]}
\correspDesc{Versand  durch Paul Goldmann am 1. 9. [1902] in Montreux
\newline{}Erhalt  durch Arthur Schnitzler im Zeitraum [2. 9. 1902
                  – 6. 9. 1902?] in Wien}\toendnotes[C]{\smallbreak}
\Standort{DLA, A:Schnitzler, HS.NZ85.1.3172.}
\physDesc{Brief, 1 Blatt, 2 Seiten, 800 Zeichen
\newline{}Handschrift: schwarze Tinte, deutsche Kurrent
\newline{}Schnitzler: mit rotem Buntstift eine Unterstreichung }\toendnotes[C]{\smallbreak}
\pstart
           \raggedleft{}{\pb}\textcolor{gray}{\textbf{\textsc{\label{K_L03222-1v}\edtext{G\textsuperscript{ds}{ }Hôtels}{\lemma{\textnormal{\emph{G\textsuperscript{ds} Hôtels}}}\Cendnote{\textnormal{\begin{otherlanguage}{french}Grands Hôtels\end{otherlanguage}}}}\label{K_L03222-1}}}}\pend
           
\pstart
           \raggedleft{}\textcolor{gray}{\textbf{\textbf{MONNEY\oindex{Grand Hôtel Monney@\textbf{Grand Hôtel Monney}, \emph{Hotel}|pw}}}}\pend
           
\pstart
           \raggedleft{}\textcolor{gray}{\textbf{{\kaufmannsund}}}\pend
           
\pstart
           \textcolor{gray}{\textbf{A. MONNEY\pwindex{Monney, A. @\textsc{Monney, A.}, \emph{Hotelbesitzer/Hotelbesitzerin}|pw}}}\hfill \textcolor{gray}{\textbf{\textbf{BEAU SÉJOUR AU LAC\oindex{Grand Hôtel Beau Séjour au Lac@\textbf{Grand Hôtel Beau Séjour au Lac}, \emph{Hotel}|pw}}}}\pend
           
\pstart
           \textcolor{gray}{\textbf{\label{K_L03222-2v}\edtext{\begin{otherlanguage}{french}PROPRIET\textsuperscript{RE}\end{otherlanguage}}{\lemma{\textnormal{\emph{Propriet\textsuperscript{RE}}}}\Cendnote{\textnormal{propriétaire, französisch:
                              Eigentümer/Eigentümerin}}}\label{K_L03222-2}}}\hfill \textcolor{gray}{\textbf{MONTREUX\oindex{Montreux@\textbf{Montreux}|pw} (\begin{otherlanguage}{french}SUISSE\oindex{Schweiz@\textbf{Schweiz}|pw}\end{otherlanguage})}}\pend
           
\pstart
           \centering{}\textcolor{gray}{\textbf{\label{K_L03222-3v}\edtext{\begin{otherlanguage}{french}Ascenseur hydraulique\end{otherlanguage}}{\lemma{\textnormal{\emph{Ascenseur hydraulique}}}\Cendnote{\textnormal{französisch: hydraulischer
                        Aufzug}}}\label{K_L03222-3}}}\pend
           
\pstart
           \textsc{Montreux\oindex{Montreux@\textbf{Montreux}|pw}}, 1. September.\pend
           
\pstart\center{}Mein lieber Freund,\pend\vspace{0.5em}
\pstart
           Mit jeder Poſt aus Berlin\oindex{Berlin@\textbf{Berlin}, \emph{Hauptstadt}|pw} habe ich Deine lieben
               Nachrichten erwartet. Nachdem{ }ſie heut wieder nicht
               eingelangt{ }ſind, bin ich wirklich in Unruhe. Ich werde Donnerſtag in Frankfurt\oindex{Frankfurt am Main@\textbf{Frankfurt am Main}, \emph{Hauptstadt}|pw}{ }ſein und bitte
               Dich{ }ſehr, mir dorthin an die Adreſſe meines Schwagers\pwindex{Rosengart, Josef 8.\,2.\,1860 Laupheim – 4.\,8.\,1927 Frankfurt am Main@\textsc{Rosengart, Josef} (8.\,2.\,1860 Laupheim – 4.\,8.\,1927 Frankfurt am Main), \emph{Arzt}|pwv} (\textsc{Dr. Rosengart\pwindex{Rosengart, Josef 8.\,2.\,1860 Laupheim – 4.\,8.\,1927 Frankfurt am Main@\textsc{Rosengart, Josef} (8.\,2.\,1860 Laupheim – 4.\,8.\,1927 Frankfurt am Main), \emph{Arzt}|pw}}, \textsc{Reuterweg 59}\oindex{Reuterweg@\textbf{Reuterweg}, \emph{Straße}|pw}) zu{ }ſchreiben, wie es Dir, dem Kinde\pwindex{Schnitzler, Heinrich 9.\,8.\,1902 Hinterbrühl – 12.\,7.\,1982 Wien@\textsc{Schnitzler, Heinrich} (9.\,8.\,1902 Hinterbrühl – 12.\,7.\,1982 Wien), \emph{Regisseur, Schauspieler}|pwv} und der Mutter\pwindex{Schnitzler, Olga 17.\,1.\,1882 Wien – 13.\,1.\,1970 Lugano@\textsc{Schnitzler, Olga} (17.\,1.\,1882 Wien – 13.\,1.\,1970 Lugano), \emph{Schauspielerin, Sängerin}|pwv} geht?\pend
           
\pstart
           {\pb}Nach \textsc{Wien}\oindex{Wien@\textbf{Wien}, \emph{Verwaltungsgebiet}|pw} komme ich nicht. Die Zeit iſt um, das Geld iſt alle. Ich habe ein{ }ſehr{ }ſchönes
               Stück Welt geſehen. In der Schweiz\oindex{Schweiz@\textbf{Schweiz}|pw}{ }ſind die
               großartigen landſchaftlichen Eindrücke gehäufter, als in Tirol\oindex{Tirol@\textbf{Tirol}, \emph{Land}|pw}\oindex{Südtirol@\textbf{Südtirol}, \emph{Verwaltungsgebiet}|pw}, und leichter zu erreichen. \label{K_L03222-4v}\edtext{Nächſtes Jahr mußt Du hingehen}{\lemma{\textnormal{\emph{Nächstes … hingehen}}}\Cendnote{\textnormal{Dazu kam es nicht.}}}\label{K_L03222-4}. Ich war \strikeout{die ga} 14 Tage lang mit meinem Onkel\pwindex{Mamroth, Fedor 21.\,2.\,1851 Breslau – 25.\,6.\,1907 Frankfurt am Main@\textsc{Mamroth, Fedor} (21.\,2.\,1851 Breslau – 25.\,6.\,1907 Frankfurt am Main), \emph{Journalist, Kritiker}|pwv} zuſammen und habe in ihm einen überaus
               liebenswürdigen und anregenden Reiſekameraden gehabt.\pend
           
\pstart
           Viele treue Grüße! {\\[\baselineskip]}Dein {\\[\baselineskip]}\spacefill\mbox{Paul Goldm}\pend
           \leftskip=0em{}
\pstart
           \noindent{}Herzliche Grüße an \textsc{Olga\pwindex{Schnitzler, Olga 17.\,1.\,1882 Wien – 13.\,1.\,1970 Lugano@\textsc{Schnitzler, Olga} (17.\,1.\,1882 Wien – 13.\,1.\,1970 Lugano), \emph{Schauspielerin, Sängerin}|pw}}!\pend
           \selectlanguage{ngerman}\endnumbering\briefempfaengerindex{Schnitzler, Arthur@\textsc{Schnitzler, Arthur}!zzzGoldmann, Paul@\emph{von Paul Goldmann}!1902-09-011@{1. 9. [1902]}|)be}\mylabel{L03222h}  \newcommand{\dateiname}{L03222}\newcommand{\titel}{Paul Goldmann an Arthur Schnitzler, 1. 9. [1902]}\newcommand{\editorInnen}{Martin Anton Müller und Laura Untner}%% latex-leseansicht-abspann.tex
%% Abspann für die Leseansicht.
%% Der Schalter \ifkorrekturansicht ist bereits durch den Vorspann gesetzt.

%% latex-abspann.tex
%% Gemeinsamer Abspann für Korrekturansicht und Leseansicht.
%% Setzt den Schalter \ifkorrekturansicht voraus (gesetzt in den
%% einbindenden Dateien latex-korrekturansicht-abspann.tex bzw.
%% latex-leseansicht-abspann.tex).
%% ---------------------------------------------------------------

\normalsize

% Das esempio-Environment wird nur in der Leseansicht benötigt
\ifkorrekturansicht\else
\newenvironment{esempio}[3]%
{
    \vspace{1.5ex}
    \rlap{\underline{#1}}
    \par
    \setlength{\parindent}{0cm}
    \nopagebreak
    \leftskip=#2cm
    \rightskip=#3cm
}
{
    \par
}
\fi

\doendnotes{C}
\bigskip
\vfill

\clearpage

\footnotesize

\ifkorrekturansicht
  \lohead{\textsc{register}}
\fi

% theindex-Environment neu definieren ohne reledmac
\makeatletter
\renewenvironment{theindex}{%
  \ifkorrekturansicht
    \section*{\indexname}%
  \else
    \subsubsection*{Index der erwähnten Entitäten}%
  \fi
  \setlength{\parindent}{0pt}%
  \setlength{\parskip}{0pt plus 0.3pt}%
  \let\item\@idxitem
}{%
  \ifkorrekturansicht\clearpage\fi
}
\makeatother

\IfFileExists{\jobname-pw.ind}{\input{\jobname-pw.ind}}{}

% Quellenangabe nur in der Leseansicht
\ifkorrekturansicht\else
% Fallback-Definitionen, falls die .tex-Datei \titel etc. nicht gesetzt hat
\providecommand{\titel}{}
\providecommand{\editorInnen}{}
\providecommand{\dateiname}{\jobname}

\vspace{3cm}

\vfill

\footnotesize
\textsc{Quelle}: \titel. Herausgegeben von {\editorInnen}. In: \emph{Arthur Schnitzler: Briefwechsel mit Autorinnen und Autoren}.
 Digitale Edition, https://schnitzler-briefe.acdh.oeaw.ac.at/{\dateiname}.html (Stand \today)
\fi

\end{document}


