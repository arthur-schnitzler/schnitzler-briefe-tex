%% latex-leseansicht-vorspann.tex
%% Vorspann für die Leseansicht.
%% Lädt die gemeinsame Datei latex-vorspann.tex mit nicht gesetztem Schalter.

\newif\ifkorrekturansicht
\korrekturansichtfalse

\input{../tex-inputs/latex-vorspann}

\begin{center}
            \textcolor{red}{ENTWURF, NICHT FERTIG KORRIGIERT}
                      \end{center}
            
         
         \renewcommand{\erwaehntePersonen}{Personen: Olga Schnitzler}
         \renewcommand{\erwaehnteOrte}{Orte: Berlin, Montreux, Schweiz, Wien}
         \renewcommand{\erwaehnteWerke}{}
               \section[ Paul Goldmann an Arthur Schnitzler, 1. 9. {[}1902{]}]{ Paul Goldmann an Arthur Schnitzler, 1. 9. {[}1902{]}}\nopagebreak\mylabel{v}\rehead{ }\begin{ledgroupsized}[t]{13cm}\normalsize\beginnumbering \toendnotes[C]{\smallbreak\pagebreak[2]} \Standort{DLA, A:Schnitzler, HS.NZ85.1.3172.}
\physDesc{Brief, 1 Blatt, 2 Seiten
\newline{}Handschrift: schwarze Tinte, deutsche Kurrent
\newline{}Schnitzler: mit rotem Buntstift eine Unterstreichung }\pstart
           \noindent{}{\pb}\textcolor{gray}{\textbf{G\textsuperscript{DS} HÔTELS\textcolor{red}{\textsuperscript{\textbf{KEY}}}{\\}\textbf{MONNEY\textcolor{red}{\textsuperscript{\textbf{KEY}}}{\\}&{\\}\textbf{BEAU SÉJOUR AU LAC\textcolor{red}{\textsuperscript{\textbf{KEY}}}}}{\\}MONTREUX\oindex{Montreux@\textbf{Montreux}|pw} (\begin{otherlanguage}{french}SUISSE\oindex{Schweiz@\textbf{Schweiz}|pw}\end{otherlanguage}){\\}\begin{otherlanguage}{french}Ascenseur hydraulique\end{otherlanguage}{\\}A. MONNEY\textcolor{red}{\textsuperscript{\textbf{KEY}}}{\\}\begin{otherlanguage}{french}PROPRIET\textsuperscript{RE}\end{otherlanguage}}}\pend
           \pstart
           \textsc{Montreux\oindex{Montreux@\textbf{Montreux}|pw}}, 1. September.\pend
           \pstart\center{}Mein lieber Freund,\pend\pstart
           Mit jeder Poſt aus Berlin\oindex{Berlin@\textbf{Berlin}|pw} habe ich Deine
                    lieben Nachrichten erwartet. Nachdem ſie heut wieder nicht
                    eingelangt ſind, bin ich wirklich in Unruhe. Ich werde Donnerſtag
                    in Frankfurt\textcolor{red}{\textsuperscript{\textbf{KEY}}} ſein und bitte Dich ſehr, mir dorthin
                    an die Adreſſe meines Schwager\textcolor{red}{\textsuperscript{\textbf{KEY}}}s (\textsc{Dr. Rosengart\textcolor{red}{\textsuperscript{\textbf{KEY}}}}, \textsc{Reuterweg} 59\textcolor{red}{\textsuperscript{\textbf{KEY}}}) zu ſchreiben, wie es Dir, dem
                        Kinde\textcolor{red}{\textsuperscript{\textbf{KEY}}} und der Mutter\textcolor{red}{\textsuperscript{\textbf{KEY}}} geht? {\pb}\pend
           \pstart
           Nach Wien\oindex{Wien@\textbf{Wien}|pw} komme ich nicht. Die Zeit iſt
                    um, das Geld iſt alle. Ich habe ein ſehr ſchönes Stück Welt geſehen. In der Schweiz\textcolor{red}{\textsuperscript{\textbf{KEY}}} ſind die großartigen landſchaftlichen
                    Eindrücke gehäufter, als in Tirol\textcolor{red}{\textsuperscript{\textbf{KEY}}}, und leichter zu
                    erreiſen. Nächſtes Jahr mußt Du hingehen. Ich war \strikeout{die
                        ga} 14 Tage lang mit meinem Onkel\textcolor{red}{\textsuperscript{\textbf{KEY}}}
                    zuſammen und habe in ihm einen überaus liebenswürdigen und anregenden
                    Reiſekameraden gehabt. {\\[\baselineskip]}Viele treue Grüße!\pend
           \leftskip=0em{}\pstart
           {\\[\baselineskip]}Dein\pend
           \leftskip=0em{}\pstart
           {\\[\baselineskip]}\spacefill\mbox{Paul Goldmnn}\pend
           \leftskip=0em{}\pstart
           Herzliche Grüße an \textsc{Olga\pwindex{Schnitzler, Olga 17.01.1882 – 13.01.1970@\textsc{Schnitzler, Olga} (17.01.1882 – 13.01.1970), \emph{Schauspielerin, Sängerin}|pw}}! \pend
           
         
         \endnumbering\mylabel{h}\end{ledgroupsized}\begin{anhang}\end{anhang}\newcommand{\dateiname}{L03222}\newcommand{\titel}{Paul Goldmann an Arthur Schnitzler, 1. 9. [1902]}\newcommand{\editorInnen}{Martin Anton Müller und Laura Untner}%% latex-leseansicht-abspann.tex
%% Abspann für die Leseansicht.
%% Der Schalter \ifkorrekturansicht ist bereits durch den Vorspann gesetzt.

%% latex-abspann.tex
%% Gemeinsamer Abspann für Korrekturansicht und Leseansicht.
%% Setzt den Schalter \ifkorrekturansicht voraus (gesetzt in den
%% einbindenden Dateien latex-korrekturansicht-abspann.tex bzw.
%% latex-leseansicht-abspann.tex).
%% ---------------------------------------------------------------

\normalsize

% Das esempio-Environment wird nur in der Leseansicht benötigt
\ifkorrekturansicht\else
\newenvironment{esempio}[3]%
{
    \vspace{1.5ex}
    \rlap{\underline{#1}}
    \par
    \setlength{\parindent}{0cm}
    \nopagebreak
    \leftskip=#2cm
    \rightskip=#3cm
}
{
    \par
}
\fi

\doendnotes{C}
\bigskip
\vfill

\clearpage

\footnotesize

\ifkorrekturansicht
  \lohead{\textsc{register}}
\fi

% theindex-Environment neu definieren ohne reledmac
\makeatletter
\renewenvironment{theindex}{%
  \ifkorrekturansicht
    \section*{\indexname}%
  \else
    \subsubsection*{Index der erwähnten Entitäten}%
  \fi
  \setlength{\parindent}{0pt}%
  \setlength{\parskip}{0pt plus 0.3pt}%
  \let\item\@idxitem
}{%
  \ifkorrekturansicht\clearpage\fi
}
\makeatother

\IfFileExists{\jobname-pw.ind}{\input{\jobname-pw.ind}}{}

% Quellenangabe nur in der Leseansicht
\ifkorrekturansicht\else
% Fallback-Definitionen, falls die .tex-Datei \titel etc. nicht gesetzt hat
\providecommand{\titel}{}
\providecommand{\editorInnen}{}
\providecommand{\dateiname}{\jobname}

\vspace{3cm}

\vfill

\footnotesize
\textsc{Quelle}: \titel. Herausgegeben von {\editorInnen}. In: \emph{Arthur Schnitzler: Briefwechsel mit Autorinnen und Autoren}.
 Digitale Edition, https://schnitzler-briefe.acdh.oeaw.ac.at/{\dateiname}.html (Stand \today)
\fi

\end{document}


      