%% latex-korrekturansicht-vorspann.tex
%% Vorspann für die Korrekturansicht.
%% Lädt die gemeinsame Datei latex-vorspann.tex mit gesetztem Schalter.

\newif\ifkorrekturansicht
\korrekturansichttrue

\input{../tex-inputs/latex-vorspann}


\section[Arthur Schnitzler an Georg Brandes, 28. 12. 1921]{L02372 Arthur Schnitzler an Georg Brandes, 28. 12. 1921}
\nopagebreak\mylabel{L02372v}
\rehead{ }\normalsize\beginnumbering\briefempfaengerindex{Brandes, Georg@\textsc{Brandes, Georg}!zzzSchnitzler, Arthur@\emph{von Arthur Schnitzler}!1921-12-281@{28. 12. 1921}|(be}
\toendnotes[C]{\smallbreak\pagebreak[2]}\Standort{Kopenhagen, Det Kongelige Bibliotek, Georg Brandes Arkiv, box 125.}
\physDesc{Brief, 1 Blatt, 2 Seiten, 1301 Zeichen
\newline{}Handschrift: schwarze Tinte, lateinische Kurrent
\newline{}Ordnung: mit Bleistift von unbekannter Hand beschriftet:
                                    »Schnitzler« und nummeriert:
                                 »46.« }
\buchAbdrucke{\weitereDrucke{1) Georg Brandes, Arthur Schnitzler: \emph{Ein Briefwechsel}. Bern: \emph{Francke} 1956, S. 131–132.} \weitereDrucke{2) Arthur Schnitzler: \emph{Briefe 1913–1931}. Frankfurt am Main: \emph{S. Fischer} 1984, S. 262–263.} }\toendnotes[C]{\smallbreak}
\pstart
           \raggedleft{}{\pb}Wien\oindex{Wien@\textbf{Wien}, \emph{A.ADM2}|pw}{ }28. 12. 21\pend
           \vspace{0.5em}
\pstart
           lieber und verehrter Freund, für die Freude, die Sie mir durch die
               Übersendung Ihres Goethe\pwindex{Goethe, Johann Wolfgang von 1749-08-28 – 1832-03-22@\textsc{Goethe, Johann Wolfgang von} (1749-08-28 – 1832-03-22), \emph{Schriftsteller/Schriftstellerin}|pw}-Buchs\pwindex{Wolfgang Goethe@\emph{Wolfgang Goethe}|pwv} bereitet haben, sag ich
               Ihnen innigsten Dank. Ich lese es mit dem größten Genuß – ich habe fast alle andre
               Lectüre unterbrochen, da mein Antheil wie an dem Gegenstand so an dem Verfasser mit
               jedem Absatz aufs neue angeregt und entzündet wird. Welche Klarheit, Einfachheit,
               Lebendigkeit in der Behandlung jedes einzelnen Werkes und dieses ganzen
               unvergleichlich reichen Daseins. Wenn ich nach einem Vergleich suche, ka{\geminationn} ich wieder nur auf ein andres Buch von Ihnen
               zurückgreifen: auf Ihren Shakespeare\pwindex{Shakespeare, William 23.4.1564? – 03.05.1616@\textsc{Shakespeare, William} (23.4.1564? – 03.05.1616), \emph{Schauspieler/Schauspielerin, Dramatiker/Dramatikerin}|pw}\pwindex{William Shakespeare@\emph{William Shakespeare}|pw}. Wie viel Altersreife war schon in dem früheren Buch, – wie viel Jugendfrische
                  \introOben{}ist\introOben{} in diesem neuen. Welchen Glanz breiten Sie über die
               Oberflächen; in welche Tiefen dringen Sie, ohne jemals dunkel zu werden. {\pb}Kritische Betrachtung, historischer Bericht,
               culturgeschichtliches Erfassen ergänzen sich, fließen zusa{\geminationm}en, und das Ganze einer genialen Persönlichkeit steht
               wohlbekannt und doch von einer leichten und starken Hand neu erschaffen, in scharfen
               und hellen Linien da. Überall Goethe\pwindex{Goethe, Johann Wolfgang von 1749-08-28 – 1832-03-22@\textsc{Goethe, Johann Wolfgang von} (1749-08-28 – 1832-03-22), \emph{Schriftsteller/Schriftstellerin}|pw} wie er
               war, – und überall auch Brandes wie er ist – und noch lange bleiben möge! Dies mein
               Gruß, lieber und hochverehrter Freund, und mein Wunsch zum neuen Jahre. – In
               Bewunderung und Treue\pend
           \pstart Ihr \spacefill\mbox{Arthur Schnitzler}\pend{}\selectlanguage{ngerman}\endnumbering\briefempfaengerindex{Brandes, Georg@\textsc{Brandes, Georg}!zzzSchnitzler, Arthur@\emph{von Arthur Schnitzler}!1921-12-281@{28. 12. 1921}|)be}\mylabel{L02372h}  \normalsize

\doendnotes{C}
\bigskip
\vfill

\clearpage

\footnotesize

\lohead{\textsc{register}}

% Definiere theindex-Environment komplett neu ohne reledmac
\makeatletter
\renewenvironment{theindex}{%
  \section*{\indexname}%
  \setlength{\parindent}{0pt}%
  \setlength{\parskip}{0pt plus 0.3pt}%
  \let\item\@idxitem
}{%
  \clearpage
}
\makeatother

\IfFileExists{\jobname-pw.ind}{\input{\jobname-pw.ind}}{}

\end{document}

      