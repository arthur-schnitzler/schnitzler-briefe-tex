%% latex-leseansicht-vorspann.tex
%% Vorspann für die Leseansicht.
%% Lädt die gemeinsame Datei latex-vorspann.tex mit nicht gesetztem Schalter.

\newif\ifkorrekturansicht
\korrekturansichtfalse

\input{../tex-inputs/latex-vorspann}


               \section[Arthur Schnitzler an Arno Holz, 28. 2. 1917]{ Arthur Schnitzler an Arno Holz, 28. 2. 1917}\nopagebreak\mylabel{v}\rehead{ }\begin{ledgroupsized}[t]{13cm}\normalsize\beginnumbering\briefempfaengerindex{Holz, Arno@\textsc{Holz, Arno}!zzzSchnitzler, Arthur@\emph{von Arthur Schnitzler}!1917-02-281@{28. 2. 1917}|(be} \toendnotes[C]{\smallbreak\pagebreak[2]} \Standort{DLA, A:Schnitzler, HS.NZ85.1.1040.}
\physDesc{Brief, 1 Blatt, 1 Seite, maschineller Durchschlag
\newline{}Schreibmaschine
\newline{}Handschrift: Bleistift, lateinische Kurrent (\noindent{}Beschriftung: »Arno Holz« und drei Unterstreichungen)}\toendnotes[C]{\smallbreak}\pstart
           \raggedleft{}{\pb}28. 2. 1917\pend
           \pstart\center{}Sehr verehrter Herr Holz.\pend\pstart
           Möchten Sie mir vielleicht gütigst einige Prospekte oder was dafür gelten könnte
                    über die von Ihnen projektierte Ausgabe der »Blechschmiede\pwindex{Holz, Arno 26.04.1863 – 26.10.1929@\textsc{Holz, Arno} (26.04.1863 – 26.10.1929), \emph{Schriftsteller}!Blechschmiede1902@\strich\emph{Die Blechschmiede} {[}1902{]}|pw}« zukommen lassen? Es wäre wohl denkbar, dass man Ihnen
                    eventuell auch durch meinen Buchhändler\pwindex{Heller, Hugo 08.05.1870 – 29.11.1923@\textsc{Heller, Hugo} (08.05.1870 – 29.11.1923), \emph{Verleger, Buchhändler}|pwv}, der hauptsächlich in Bibliophilenkreisen bekannt ist,
                    eine Anzahl Subscribenten verschaffen könnte. Auf mich muss ich Sie leider
                    bitten diesmal zu verzichten. Meine Einnahmen sind so erheblich gesunken, meine
                    Ausgaben so ungeheuerlich gestiegen, dass \introOben{}ich\introOben{} es mir
                    leider versagen muss, für ein Buch und wäre es das allerschönste hundert Mark zu
                    verausgaben.\pend
           \pstart
           In besonderer Hochachtung{\\[\baselineskip]} Ihr sehr ergebener\pend
           \leftskip=0em{}          \endnumbering\briefempfaengerindex{Holz, Arno@\textsc{Holz, Arno}!zzzSchnitzler, Arthur@\emph{von Arthur Schnitzler}!1917-02-281@{28. 2. 1917}|)be}\mylabel{h}\end{ledgroupsized}  \newcommand{\dateiname}{L02258}\newcommand{\titel}{Arthur Schnitzler an Arno Holz, 28. 2. 1917}\newcommand{\editorInnen}{Martin Anton Müller und Gerd-Hermann Susen}%% latex-leseansicht-abspann.tex
%% Abspann für die Leseansicht.
%% Der Schalter \ifkorrekturansicht ist bereits durch den Vorspann gesetzt.

%% latex-abspann.tex
%% Gemeinsamer Abspann für Korrekturansicht und Leseansicht.
%% Setzt den Schalter \ifkorrekturansicht voraus (gesetzt in den
%% einbindenden Dateien latex-korrekturansicht-abspann.tex bzw.
%% latex-leseansicht-abspann.tex).
%% ---------------------------------------------------------------

\normalsize

% Das esempio-Environment wird nur in der Leseansicht benötigt
\ifkorrekturansicht\else
\newenvironment{esempio}[3]%
{
    \vspace{1.5ex}
    \rlap{\underline{#1}}
    \par
    \setlength{\parindent}{0cm}
    \nopagebreak
    \leftskip=#2cm
    \rightskip=#3cm
}
{
    \par
}
\fi

\doendnotes{C}
\bigskip
\vfill

\clearpage

\footnotesize

\ifkorrekturansicht
  \lohead{\textsc{register}}
\fi

% theindex-Environment neu definieren ohne reledmac
\makeatletter
\renewenvironment{theindex}{%
  \ifkorrekturansicht
    \section*{\indexname}%
  \else
    \subsubsection*{Index der erwähnten Entitäten}%
  \fi
  \setlength{\parindent}{0pt}%
  \setlength{\parskip}{0pt plus 0.3pt}%
  \let\item\@idxitem
}{%
  \ifkorrekturansicht\clearpage\fi
}
\makeatother

\IfFileExists{\jobname-pw.ind}{\input{\jobname-pw.ind}}{}

% Quellenangabe nur in der Leseansicht
\ifkorrekturansicht\else
% Fallback-Definitionen, falls die .tex-Datei \titel etc. nicht gesetzt hat
\providecommand{\titel}{}
\providecommand{\editorInnen}{}
\providecommand{\dateiname}{\jobname}

\vspace{3cm}

\vfill

\footnotesize
\textsc{Quelle}: \titel. Herausgegeben von {\editorInnen}. In: \emph{Arthur Schnitzler: Briefwechsel mit Autorinnen und Autoren}.
 Digitale Edition, https://schnitzler-briefe.acdh.oeaw.ac.at/{\dateiname}.html (Stand \today)
\fi

\end{document}


      