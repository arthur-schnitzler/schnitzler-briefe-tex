%% latex-leseansicht-vorspann.tex
%% Vorspann für die Leseansicht.
%% Lädt die gemeinsame Datei latex-vorspann.tex mit nicht gesetztem Schalter.

\newif\ifkorrekturansicht
\korrekturansichtfalse

\input{../tex-inputs/latex-vorspann}


\section[Arthur Schnitzler an Gustav Schwarzkopf, 7. 10. 1905]{L04019 Arthur Schnitzler an Gustav Schwarzkopf, 7. 10. 1905}
\nopagebreak\mylabel{L04019v}
\rehead{ }\normalsize\beginnumbering\briefempfaengerindex{Schwarzkopf, Gustav@\textsc{Schwarzkopf, Gustav}!zzzSchnitzler, Arthur@\emph{von Arthur Schnitzler}!1905-10-071@{7. 10. 1905}|(be}
\toendnotes[C]{\smallbreak\pagebreak[2]}
\correspDesc{Versand  durch Arthur Schnitzler am 7. 10. 1905 in Wien
\newline{}Erhalt  durch Gustav Schwarzkopf im Zeitraum [7. 10. 1905 – 10. 10. 1905?] in Wien}\toendnotes[C]{\smallbreak}
\Standort{CUL, Schnitzler, B 96.}
\physDesc{Postkarte, 316 Zeichen
\newline{}Handschrift: schwarze Tinte, deutsche Kurrent
\newline{}Versand: 1) Rohrpost  2) Stempel: »\nobreak{}\oindex{XVIII., Währing@\textbf{XVIII., Währing}, \emph{Verwaltungsgebiet}|pwk}Wien 18/\textcolor{gray}{1}
                                       110, 7 X {[}05{]}, 4\textsuperscript{10}N \nobreak{}«.  3) Stempel: »\nobreak{}\oindex{I., Innere Stadt@\textbf{I., Innere Stadt}, \emph{Verwaltungsgebiet}|pwk}Wien 1/1, 7 X 05, 4 30N\nobreak{}«. }\toendnotes[C]{\smallbreak}\pstart{}{\pb}Herrn Guſtav
                  Schwarzkopf\pend{}\pstart{}Wien I\oindex{I., Innere Stadt@\textbf{I., Innere Stadt}, \emph{Verwaltungsgebiet}|pw}\pend{}\pstart{}Tiefer Graben 23\oindex{Wien@\textbf{Wien}!I., Innere Stadt@\textbf{I., Innere Stadt}!Tiefer Graben 23@\textbf{Tiefer Graben 23}, \emph{Wohngebäude}|pw}.\pend{}{\bigskip}\vspace{1em}
\pstart
           \noindent{}{\pb}lieber Guſtav, da Mama\pwindex{Schnitzler, Louise 8.\,7.\,1840 Kőszeg – 9.\,9.\,1911 Wien@\textsc{Schnitzler, Louise} (8.\,7.\,1840 Kőszeg – 9.\,9.\,1911 Wien)|pw} nun in
               meine Loge\eventindex{Burgtheater@\textbf{Burgtheater}!Uraufführung von Zwischenspiel, 12.10.1905@Uraufführung von Zwischenspiel, 12.10.1905|pwv}, und Hajeks\pwindex{Hajek, Markus 25.\,11.\,1861 Vršac – 4.\,4.\,1941 London@\textsc{Hajek, Markus} (25.\,11.\,1861 Vršac – 4.\,4.\,1941 London), \emph{Mediziner, Laryngologe}|pw}\pwindex{Hajek, Gisela 20.\,12.\,1867 Wien – 3.\,2.\,1953 Cambridge@\textsc{Hajek, Gisela} (20.\,12.\,1867 Wien – 3.\,2.\,1953 Cambridge)|pw} u. Schnitzlers\pwindex{Schnitzler, Julius 13.\,7.\,1865 Wien – 29.\,6.\,1939 ebd.@\textsc{Schnitzler, Julius} (13.\,7.\,1865 Wien – 29.\,6.\,1939 ebd.), \emph{Chirurg}|pw}\pwindex{Schnitzler, Helene 16.\,7.\,1871 Budapest – September 1941 Atlantischer Ozean@\textsc{Schnitzler, Helene} (16.\,7.\,1871 Budapest – September 1941 Atlantischer Ozean)|pw} (Julius\pwindex{Schnitzler, Julius 13.\,7.\,1865 Wien – 29.\,6.\,1939 ebd.@\textsc{Schnitzler, Julius} (13.\,7.\,1865 Wien – 29.\,6.\,1939 ebd.), \emph{Chirurg}|pw}) vermut in \uline{eine} andere Loge gehn,
               muſs ich heute verzichten, Sie in der Autorenloge zu{ }ſehn\footnote{\noindent{}zu beherbergen mein ich, –{ }ſeh’n hoffentlich
                  dort.}, und frage an, ob ich Ihnen und Dr. Max\pwindex{Schwarzkopf, Max 12.\,6.\,1857 Wien – 14.\,4.\,1928 ebd.@\textsc{Schwarzkopf, Max} (12.\,6.\,1857 Wien – 14.\,4.\,1928 ebd.), \emph{Rechtsanwalt}|pw} für den 2. Abend\eventindex{Burgtheater@\textbf{Burgtheater}!2. Aufführung von Zwischenspiel, 13.10.1905@2. Aufführung von Zwischenspiel, 13.10.1905|pwv} meine Autorenſitze \label{K_L04019-1v}\edtext{ſchicken}{\lemma{\textnormal{\emph{schicken}}}\Cendnote{\textnormal{Siehe XXXX Auszeichnungsfehler: Dokument L04017 nicht gefunden.}}}\label{K_L04019-1} darf?\pend
           
\pstart
           Herzlich grüßen\textcolor{gray}{d} Ihr{\\[\baselineskip]}\spacefill\mbox{ArthSch}\pend
           \leftskip=0em{}\selectlanguage{ngerman}\endnumbering\briefempfaengerindex{Schwarzkopf, Gustav@\textsc{Schwarzkopf, Gustav}!zzzSchnitzler, Arthur@\emph{von Arthur Schnitzler}!1905-10-071@{7. 10. 1905}|)be}\mylabel{L04019h}
\begin{anhang}
\end{anhang}\newcommand{\dateiname}{L04019}\newcommand{\titel}{Arthur Schnitzler an Gustav Schwarzkopf, 7. 10. 1905}\newcommand{\editorInnen}{Herausgegeben von Jahnke, SelmaMüller, Martin Anton}%% latex-leseansicht-abspann.tex
%% Abspann für die Leseansicht.
%% Der Schalter \ifkorrekturansicht ist bereits durch den Vorspann gesetzt.

%% latex-abspann.tex
%% Gemeinsamer Abspann für Korrekturansicht und Leseansicht.
%% Setzt den Schalter \ifkorrekturansicht voraus (gesetzt in den
%% einbindenden Dateien latex-korrekturansicht-abspann.tex bzw.
%% latex-leseansicht-abspann.tex).
%% ---------------------------------------------------------------

\normalsize

% Das esempio-Environment wird nur in der Leseansicht benötigt
\ifkorrekturansicht\else
\newenvironment{esempio}[3]%
{
    \vspace{1.5ex}
    \rlap{\underline{#1}}
    \par
    \setlength{\parindent}{0cm}
    \nopagebreak
    \leftskip=#2cm
    \rightskip=#3cm
}
{
    \par
}
\fi

\doendnotes{C}
\bigskip
\vfill

\clearpage

\footnotesize

\ifkorrekturansicht
  \lohead{\textsc{register}}
\fi

% theindex-Environment neu definieren ohne reledmac
\makeatletter
\renewenvironment{theindex}{%
  \ifkorrekturansicht
    \section*{\indexname}%
  \else
    \subsubsection*{Index der erwähnten Entitäten}%
  \fi
  \setlength{\parindent}{0pt}%
  \setlength{\parskip}{0pt plus 0.3pt}%
  \let\item\@idxitem
}{%
  \ifkorrekturansicht\clearpage\fi
}
\makeatother

\IfFileExists{\jobname-pw.ind}{\input{\jobname-pw.ind}}{}

% Quellenangabe nur in der Leseansicht
\ifkorrekturansicht\else
% Fallback-Definitionen, falls die .tex-Datei \titel etc. nicht gesetzt hat
\providecommand{\titel}{}
\providecommand{\editorInnen}{}
\providecommand{\dateiname}{\jobname}

\vspace{3cm}

\vfill

\footnotesize
\textsc{Quelle}: \titel. Herausgegeben von {\editorInnen}. In: \emph{Arthur Schnitzler: Briefwechsel mit Autorinnen und Autoren}.
 Digitale Edition, https://schnitzler-briefe.acdh.oeaw.ac.at/{\dateiname}.html (Stand \today)
\fi

\end{document}


