%% latex-leseansicht-vorspann.tex
%% Vorspann für die Leseansicht.
%% Lädt die gemeinsame Datei latex-vorspann.tex mit nicht gesetztem Schalter.

\newif\ifkorrekturansicht
\korrekturansichtfalse

\input{../tex-inputs/latex-vorspann}

\begin{center}
            \textcolor{red}{ENTWURF, NICHT FERTIG KORRIGIERT}
                      \end{center}
            
         \renewcommand{\erwaehnteOrte}{Orte: Genf, Paris, Wien}
         \renewcommand{\erwaehnteWerke}{Werke: Jung-Wien im Auslande, La Semaine Littéraire, Sterben. Novelle, Wiener Allgemeine Zeitung}
               \section[Felix Salten an Arthur Schnitzler, {[}11. 5. 1895{]}]{ Felix Salten an Arthur Schnitzler, {[}11. 5. 1895{]}}\nopagebreak\mylabel{v}\rehead{ }\begin{ledgroupsized}[t]{13cm}\normalsize\beginnumbering \toendnotes[C]{\smallbreak\pagebreak[2]} \Standort{CUL, Schnitzler, B 89, A 1.}
\physDesc{Brief, 1 Blatt, 1 Seite
\newline{}Handschrift: Bleistift, lateinische Kurrent
\newline{}Schnitzler: mit Bleistift datiert: »11/5 94« }\toendnotes[C]{\smallbreak}\pstart
           \noindent{}{\pb}L. F. herzlichen Dank mit der Bitte, zu entschuldigen, dass es nicht
               früher möglich war. – Die \label{K_L03153-1v}\edtext{Notiz\pwindex{Jung-Wien im Auslande1895-05-12@\emph{Jung-Wien im Auslande} {[}1895-05-12{]}|pwv}}{\lemma{\textnormal{\emph{Notiz}}}\Cendnote{\textnormal{[O. V. = Salten\pwindex{Salten, Felix 06.09.1869 – 08.10.1945@\textsc{Salten, Felix} (06.09.1869 – 08.10.1945), \emph{Schriftsteller, Journalist}|pwk}]: \emph{Jung-Wien im Auslande}\pwindex{Jung-Wien im Auslande1895-05-12@\emph{Jung-Wien im Auslande} {[}1895-05-12{]}|pwk}. In: \emph{Wiener Allgemeine Zeitung}\pwindex{?? Werk@Nicht ermittelte Verfasserinnen und Verfasser!Wiener Allgemeine Zeitung1.3.1880 – 11.2.1934@\emph{Wiener Allgemeine Zeitung} {[}1.3.1880 – 11.2.1934{]}|pwk}, Nr. 5.156,
                        12. 5. 1895, S. 4: »Der erst kürzlich erschienene
                     Roman ›\so{Sterben}\pwindex{Schnitzler, Arthur 15.05.1862 – 21.10.1931@\textsc{Schnitzler, Arthur} (15.05.1862 – 21.10.1931), \emph{Schriftsteller, Mediziner}!Sterben. Novelle1894-10-01 – 1894-12-01@\strich\emph{Sterben. Novelle} {[}1894-10-01 – 1894-12-01{]}|pw}‹ des Wien\oindex{Wien@\textbf{Wien}|pw}er Dichters \so{Arthur Schnitzler}\pwindex{Schnitzler, Arthur 15.05.1862 – 21.10.1931@\textsc{Schnitzler, Arthur} (15.05.1862 – 21.10.1931), \emph{Schriftsteller, Mediziner}|pw} ist bereits in’s Französische übersetzt worden. Die bekannte französische
                     Wochenschrift in Genf\oindex{Genf@\textbf{Genf}|pw} ›\so{La Semaine Littéraire}\pwindex{?? Werk@Nicht ermittelte Verfasserinnen und Verfasser!Semaine Litteraire1893 – 1927@\emph{La Semaine Littéraire} {[}1893 – 1927{]}|pw}‹ beginnt in ihrer letzten Nummer mit der Veröffentlichung dieses Romanes,
                     welcher demnächst auch in Paris\oindex{Paris@\textbf{Paris}|pw} in Buchform
                     erscheinen wird.«}}}\label{K_L03153-1h} über Semaine
                  littéraire\pwindex{?? Werk@Nicht ermittelte Verfasserinnen und Verfasser!Semaine Litteraire1893 – 1927@\emph{La Semaine Littéraire} {[}1893 – 1927{]}|pw} habe ich \label{K_L03153-11v}\edtext{heute
               erst, – weil Sonntagsblatt – gegeben}{\lemma{\textnormal{\emph{heute … gegeben}}}\Cendnote{\textnormal{Zwei am Seitenende angebrachte Zeichen fordern zum Umblättern auf, verweisen
                  möglicherweise auf die nicht erhaltene Beilage der erwähnten
               Zeitungsnotiz.}}}\label{K_L03153-11h}. \pend
           \pstart
           Ihr {\\[\baselineskip]}\spacefill\mbox{Salten}\pend
           \leftskip=0em{}
         
         \endnumbering\mylabel{h}\end{ledgroupsized}\begin{anhang}\end{anhang}\newcommand{\dateiname}{L03153}\newcommand{\titel}{Felix Salten an Arthur Schnitzler, [11. 5. 1895]}\newcommand{\editorInnen}{Martin Anton Müller und Laura Untner}%% latex-leseansicht-abspann.tex
%% Abspann für die Leseansicht.
%% Der Schalter \ifkorrekturansicht ist bereits durch den Vorspann gesetzt.

%% latex-abspann.tex
%% Gemeinsamer Abspann für Korrekturansicht und Leseansicht.
%% Setzt den Schalter \ifkorrekturansicht voraus (gesetzt in den
%% einbindenden Dateien latex-korrekturansicht-abspann.tex bzw.
%% latex-leseansicht-abspann.tex).
%% ---------------------------------------------------------------

\normalsize

% Das esempio-Environment wird nur in der Leseansicht benötigt
\ifkorrekturansicht\else
\newenvironment{esempio}[3]%
{
    \vspace{1.5ex}
    \rlap{\underline{#1}}
    \par
    \setlength{\parindent}{0cm}
    \nopagebreak
    \leftskip=#2cm
    \rightskip=#3cm
}
{
    \par
}
\fi

\doendnotes{C}
\bigskip
\vfill

\clearpage

\footnotesize

\ifkorrekturansicht
  \lohead{\textsc{register}}
\fi

% theindex-Environment neu definieren ohne reledmac
\makeatletter
\renewenvironment{theindex}{%
  \ifkorrekturansicht
    \section*{\indexname}%
  \else
    \subsubsection*{Index der erwähnten Entitäten}%
  \fi
  \setlength{\parindent}{0pt}%
  \setlength{\parskip}{0pt plus 0.3pt}%
  \let\item\@idxitem
}{%
  \ifkorrekturansicht\clearpage\fi
}
\makeatother

\IfFileExists{\jobname-pw.ind}{\input{\jobname-pw.ind}}{}

% Quellenangabe nur in der Leseansicht
\ifkorrekturansicht\else
% Fallback-Definitionen, falls die .tex-Datei \titel etc. nicht gesetzt hat
\providecommand{\titel}{}
\providecommand{\editorInnen}{}
\providecommand{\dateiname}{\jobname}

\vspace{3cm}

\vfill

\footnotesize
\textsc{Quelle}: \titel. Herausgegeben von {\editorInnen}. In: \emph{Arthur Schnitzler: Briefwechsel mit Autorinnen und Autoren}.
 Digitale Edition, https://schnitzler-briefe.acdh.oeaw.ac.at/{\dateiname}.html (Stand \today)
\fi

\end{document}


      