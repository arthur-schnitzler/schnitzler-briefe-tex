%% latex-leseansicht-vorspann.tex
%% Vorspann für die Leseansicht.
%% Lädt die gemeinsame Datei latex-vorspann.tex mit nicht gesetztem Schalter.

\newif\ifkorrekturansicht
\korrekturansichtfalse

\input{../tex-inputs/latex-vorspann}


\section[ Felix Salten an Arthur Schnitzler, [11. 5. 1895]]{L03153 Felix Salten an Arthur Schnitzler,  [11. 5. 1895]}
\nopagebreak\mylabel{L03153v}
\rehead{ }\normalsize\beginnumbering\briefempfaengerindex{Schnitzler, Arthur@\textsc{Schnitzler, Arthur}!zzzSalten, Felix@\emph{von Felix Salten}!1895-05-111@{{[}11. 5. 1895{]}}|(be}
\toendnotes[C]{\smallbreak\pagebreak[2]}
\correspDesc{Versand  durch Felix Salten am [11. 5. 1895] in Wien
\newline{}Erhalt  durch Arthur Schnitzler am [11. 5. 1895] in Wien}\toendnotes[C]{\smallbreak}
\Standort{CUL, Schnitzler, B 89, A 1.}
\physDesc{Brief, 1 Blatt, 1 Seite, 177 Zeichen
\newline{}Handschrift: Bleistift, lateinische Kurrent
\newline{}Schnitzler: mit Bleistift datiert: »11/5 9\textcolor{gray}{5}« 
\newline{}Ordnung: mit Bleistift von unbekannter Hand nummeriert: »54a?« }\toendnotes[C]{\smallbreak}
\pstart
           \noindent{}{\pb}\label{K_L03153-1v}\edtext{L. F.}{\lemma{\textnormal{\emph{L. F.}}}\Cendnote{\textnormal{Lieber Freund}}}\label{K_L03153-1} herzlichen Dank mit
               der Bitte, zu entschuldigen, dass es nicht früher möglich war. – Die \label{K_L03153-2v}\edtext{Notiz\pwindex{Jung-Wien im Auslande@\emph{Jung-Wien im Auslande}|pwv}}{\lemma{\textnormal{\emph{Notiz}}}\Cendnote{\textnormal{[Felix Salten\pwindex{Salten, Felix 6.\,9.\,1869 Budapest – 8.\,10.\,1945 Zürich@\textsc{Salten, Felix} (6.\,9.\,1869 Budapest – 8.\,10.\,1945 Zürich), \emph{Schriftsteller, Journalist, Chefredakteur}|pwk}]: \emph{Jung-Wien im Auslande}\pwindex{Jung-Wien im Auslande@\emph{Jung-Wien im Auslande}|pwk}. In: \emph{Wiener Allgemeine Zeitung}\pwindex{Wiener Allgemeine Zeitung@\emph{Wiener Allgemeine Zeitung}|pwk}, Nr. 5156, 12. 5. 1895, S. 4: »Der erst kürzlich
                     erschienene Roman ›\so{Sterben}\pwindex{Schnitzler, Arthur 15.\,5.\,1862 Wien – 21.\,10.\,1931 ebd.@\textsc{Schnitzler, Arthur} (15.\,5.\,1862 Wien – 21.\,10.\,1931 ebd.), \emph{Schriftsteller, Mediziner}!Sterben. Novelle@\strich\emph{Sterben. Novelle}|pw}‹ des Wien\oindex{Wien@\textbf{Wien}, \emph{Verwaltungsgebiet}|pw}er Dichters \so{Arthur Schnitzler} ist bereits in’s Französische übersetzt worden. Die bekannte französische
                     Wochenschrift in Genf\oindex{Genf@\textbf{Genf}|pw} ›\so{La Semaine Littéraire}\pwindex{Semaine Littéraire@\emph{La Semaine Littéraire}|pw}‹ beginnt in ihrer letzten Nummer mit der Veröffentlichung dieses Roman\pwindex{Schnitzler, Arthur 15.\,5.\,1862 Wien – 21.\,10.\,1931 ebd.@\textsc{Schnitzler, Arthur} (15.\,5.\,1862 Wien – 21.\,10.\,1931 ebd.), \emph{Schriftsteller, Mediziner}!Sterben. Novelle@\strich\emph{Sterben. Novelle}|pwv}es, welcher
                     demnächst auch in Paris\oindex{Paris@\textbf{Paris}, \emph{Hauptstadt}|pw} in Buchform
                     erscheinen wird.«}}}\label{K_L03153-2} über Semaine
                  littéraire\pwindex{Semaine Littéraire@\emph{La Semaine Littéraire}|pw} habe ich \label{K_L03153-3v}\edtext{heute erst, – weil Sonntagsblatt\pwindex{Semaine Littéraire@\emph{La Semaine Littéraire}|pwv} – gegeben}{\lemma{\textnormal{\emph{heute … gegeben}}}\Cendnote{\textnormal{Zwei am Seitenende angebrachte Zeichen fordern zum Umblättern
                  auf und verweisen möglicherweise auf die nicht erhaltene Beilage der erwähnten Zeitungsnotiz\pwindex{Jung-Wien im Auslande@\emph{Jung-Wien im Auslande}|pwkv}.}}}\label{K_L03153-3}.\pend
           
\pstart
           Ihr {\\[\baselineskip]}\spacefill\mbox{Salten}\pend
           \leftskip=0em{}\selectlanguage{ngerman}\endnumbering\briefempfaengerindex{Schnitzler, Arthur@\textsc{Schnitzler, Arthur}!zzzSalten, Felix@\emph{von Felix Salten}!1895-05-111@{{[}11. 5. 1895{]}}|)be}\mylabel{L03153h}  \newcommand{\dateiname}{L03153}\newcommand{\titel}{Felix Salten an Arthur Schnitzler, [11. 5. 1895]}\newcommand{\editorInnen}{Martin Anton Müller und Laura Untner}%% latex-leseansicht-abspann.tex
%% Abspann für die Leseansicht.
%% Der Schalter \ifkorrekturansicht ist bereits durch den Vorspann gesetzt.

%% latex-abspann.tex
%% Gemeinsamer Abspann für Korrekturansicht und Leseansicht.
%% Setzt den Schalter \ifkorrekturansicht voraus (gesetzt in den
%% einbindenden Dateien latex-korrekturansicht-abspann.tex bzw.
%% latex-leseansicht-abspann.tex).
%% ---------------------------------------------------------------

\normalsize

% Das esempio-Environment wird nur in der Leseansicht benötigt
\ifkorrekturansicht\else
\newenvironment{esempio}[3]%
{
    \vspace{1.5ex}
    \rlap{\underline{#1}}
    \par
    \setlength{\parindent}{0cm}
    \nopagebreak
    \leftskip=#2cm
    \rightskip=#3cm
}
{
    \par
}
\fi

\doendnotes{C}
\bigskip
\vfill

\clearpage

\footnotesize

\ifkorrekturansicht
  \lohead{\textsc{register}}
\fi

% theindex-Environment neu definieren ohne reledmac
\makeatletter
\renewenvironment{theindex}{%
  \ifkorrekturansicht
    \section*{\indexname}%
  \else
    \subsubsection*{Index der erwähnten Entitäten}%
  \fi
  \setlength{\parindent}{0pt}%
  \setlength{\parskip}{0pt plus 0.3pt}%
  \let\item\@idxitem
}{%
  \ifkorrekturansicht\clearpage\fi
}
\makeatother

\IfFileExists{\jobname-pw.ind}{\input{\jobname-pw.ind}}{}

% Quellenangabe nur in der Leseansicht
\ifkorrekturansicht\else
% Fallback-Definitionen, falls die .tex-Datei \titel etc. nicht gesetzt hat
\providecommand{\titel}{}
\providecommand{\editorInnen}{}
\providecommand{\dateiname}{\jobname}

\vspace{3cm}

\vfill

\footnotesize
\textsc{Quelle}: \titel. Herausgegeben von {\editorInnen}. In: \emph{Arthur Schnitzler: Briefwechsel mit Autorinnen und Autoren}.
 Digitale Edition, https://schnitzler-briefe.acdh.oeaw.ac.at/{\dateiname}.html (Stand \today)
\fi

\end{document}


