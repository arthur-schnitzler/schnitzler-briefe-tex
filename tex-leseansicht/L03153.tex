%% latex-korrekturansicht-vorspann.tex
%% Vorspann für die Korrekturansicht.
%% Lädt die gemeinsame Datei latex-vorspann.tex mit gesetztem Schalter.

\newif\ifkorrekturansicht
\korrekturansichttrue

\input{../tex-inputs/latex-vorspann}


\section[ Felix Salten an Arthur Schnitzler, {[}11. 5. 1895{]}]{L03153 Felix Salten an Arthur Schnitzler, {[}11. 5. 1895{]}}
\nopagebreak\mylabel{L03153v}
\rehead{ }\normalsize\beginnumbering\briefempfaengerindex{Schnitzler, Arthur@\textsc{Schnitzler, Arthur}!zzzSalten, Felix@\emph{von Felix Salten}!1895-05-111@{{[}11. 5. 1895{]}}|(be}
\toendnotes[C]{\smallbreak\pagebreak[2]}\Standort{CUL, Schnitzler, B 89, A 1.}
\physDesc{Brief, 1 Blatt, 1 Seite, 177 Zeichen
\newline{}Handschrift: Bleistift, lateinische Kurrent
\newline{}Schnitzler: mit Bleistift datiert: »11/5 9\textcolor{gray}{5}« 
\newline{}Ordnung: mit Bleistift von unbekannter Hand nummeriert: »54a?« }\toendnotes[C]{\smallbreak}
\pstart
           \noindent{}{\pb}\label{K_L03153-1v}\edtext{L. F.}{\lemma{\textnormal{\emph{L. F.}}}\Cendnote{\textnormal{Lieber Freund}}}\label{K_L03153-1} herzlichen Dank mit
               der Bitte, zu entschuldigen, dass es nicht früher möglich war. – Die \label{K_L03153-2v}\edtext{Notiz\pwindex{Jung-Wien im Auslande@\emph{Jung-Wien im Auslande}|pwv}}{\lemma{\textnormal{\emph{Notiz}}}\Cendnote{\textnormal{[Felix Salten\pwindex{Salten, Felix 06.09.1869 – 08.10.1945@\textsc{Salten, Felix} (06.09.1869 – 08.10.1945), \emph{Schriftsteller/Schriftstellerin, Journalist/Journalistin, Chefredakteur/Chefredakteurin}|pwk}]: \emph{Jung-Wien im Auslande}\pwindex{Jung-Wien im Auslande@\emph{Jung-Wien im Auslande}|pwk}. In: \emph{Wiener Allgemeine Zeitung}\pwindex{Wiener Allgemeine Zeitung@\emph{Wiener Allgemeine Zeitung}|pwk}, Nr. 5156, 12. 5. 1895, S. 4: »Der erst kürzlich
                     erschienene Roman ›\so{Sterben}\pwindex{Sterben. Novelle@\emph{Sterben. Novelle}|pw}‹ des Wien\oindex{Wien@\textbf{Wien}, \emph{A.ADM2}|pw}er Dichters \so{Arthur Schnitzler} ist bereits in’s Französische übersetzt worden. Die bekannte französische
                     Wochenschrift in Genf\oindex{Genf@\textbf{Genf}, \emph{P.PPLA}|pw} ›\so{La Semaine Littéraire}\pwindex{Semaine Litteraire@\emph{La Semaine Littéraire}|pw}‹ beginnt in ihrer letzten Nummer mit der Veröffentlichung dieses Roman\pwindex{Sterben. Novelle@\emph{Sterben. Novelle}|pwv}es, welcher
                     demnächst auch in Paris\oindex{Paris@\textbf{Paris}, \emph{P.PPLC}|pw} in Buchform
                     erscheinen wird.«}}}\label{K_L03153-2} über Semaine
                  littéraire\pwindex{Semaine Litteraire@\emph{La Semaine Littéraire}|pw} habe ich \label{K_L03153-3v}\edtext{heute erst, – weil Sonntagsblatt\pwindex{Semaine Litteraire@\emph{La Semaine Littéraire}|pwv} – gegeben}{\lemma{\textnormal{\emph{heute … gegeben}}}\Cendnote{\textnormal{Zwei am Seitenende angebrachte Zeichen fordern zum Umblättern
                  auf und verweisen möglicherweise auf die nicht erhaltene Beilage der erwähnten Zeitungsnotiz\pwindex{Jung-Wien im Auslande@\emph{Jung-Wien im Auslande}|pwkv}.}}}\label{K_L03153-3}.\pend
           
\pstart
           Ihr {\\[\baselineskip]}\spacefill\mbox{Salten}\pend
           \leftskip=0em{}\selectlanguage{ngerman}\endnumbering\briefempfaengerindex{Schnitzler, Arthur@\textsc{Schnitzler, Arthur}!zzzSalten, Felix@\emph{von Felix Salten}!1895-05-111@{{[}11. 5. 1895{]}}|)be}\mylabel{L03153h}  \normalsize

\doendnotes{C}
\bigskip
\vfill

\clearpage

\footnotesize

\lohead{\textsc{register}}

% Definiere theindex-Environment komplett neu ohne reledmac
\makeatletter
\renewenvironment{theindex}{%
  \section*{\indexname}%
  \setlength{\parindent}{0pt}%
  \setlength{\parskip}{0pt plus 0.3pt}%
  \let\item\@idxitem
}{%
  \clearpage
}
\makeatother

\IfFileExists{\jobname-pw.ind}{\input{\jobname-pw.ind}}{}

\end{document}

      