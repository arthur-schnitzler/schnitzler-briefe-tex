\input{../tex-inputs/latex-pdf-vorspann}
\begin{center}
            \textcolor{red}{ENTWURF. ENTZIFFERUNG NOCH NICHT KORREKTURGELESEN}
                      \end{center}
            
               \section[Richard Beer-Hofmann an Arthur Schnitzler, {[}17. 9. 1904{]}]{ Richard Beer-Hofmann an Arthur Schnitzler,
               {[}17. 9. 1904{]}}\nopagebreak\mylabel{v}\rehead{ }\begin{ledgroupsized}[t]{13cm}\normalsize\beginnumbering\briefempfaengerindex{Schnitzler, Arthur@\textsc{Schnitzler, Arthur}!zzzBeer-Hofmann, Richard@\emph{von Richard Beer-Hofmann}!1904-09-171@{17. 9. 1904}|(be} \toendnotes[C]{\smallbreak\pagebreak[2]} \Standort{CUL, Schnitzler, B 8.}
\physDesc{Brief, 1 Blatt, 1 Seite
\newline{}Handschrift: Bleistift, lateinische Kurrent
\newline{}Schnitzler: mit Bleistift beschriftet: »Salzburg\oindex{Salzburg@\textbf{Salzburg}|pw}{ }17/9 904« \newline{}Ordnung: mit Bleistift von unbekannter Hand nummeriert: »191« }\toendnotes[C]{\smallbreak}\pstart
           \noindent{}{\pb}Gehe zuerst Buchhandlung Höllriegl\orgindex{Buchhandlung und Verlag Eduard Hoellrigel@Buchhandlung und Verlag Eduard Höllrigel|pw} (Kerber\orgindex{Buchhandlung und Verlag Eduard Hoellrigel@Buchhandlung und Verlag Eduard Höllrigel|pw})
               beim Durchhaus\oindex{Ritzerhaus@\textbf{Ritzerhaus}|pwv} auf den Marktplatz\oindex{Alter Markt@\textbf{Alter Markt}|pw}, dann zu \label{K_L01447_1v}\edtext{Svatek\orgindex{Wenzel Swatek@Wenzel Swatek|pw}}{\lemma{\textnormal{\emph{Svatek}}}\Cendnote{\textnormal{Sowohl bei \emph{Swatek}\orgindex{Wenzel Swatek@Wenzel Swatek|pwk} wie \emph{Schwarz}\orgindex{Karl Schwarz Antiquitaeten@Karl Schwarz Antiquitäten|pwk} handelt es sich um Antiquitätenhändler.}}}\label{K_L01447_1h}
               (Parterre oder I Stock) dann zu Schwarz\orgindex{Karl Schwarz Antiquitaeten@Karl Schwarz Antiquitäten|pw}{ }Kaigasse\oindex{Kaigasse@\textbf{Kaigasse}|pw}.\pend
           \pstart \spacefill\mbox{Richard}\pend{}\endnumbering\briefempfaengerindex{Schnitzler, Arthur@\textsc{Schnitzler, Arthur}!zzzBeer-Hofmann, Richard@\emph{von Richard Beer-Hofmann}!1904-09-171@{17. 9. 1904}|)be}\mylabel{h}\end{ledgroupsized}  \newcommand{\dateiname}{L01447}\newcommand{\titel}{Richard Beer-Hofmann an Arthur Schnitzler, [17. 9. 1904]}\newcommand{\editorInnen}{Martin Anton Müller und Gerd-Hermann Susen}\input{../tex-inputs/latex-pdf-abspann}
      