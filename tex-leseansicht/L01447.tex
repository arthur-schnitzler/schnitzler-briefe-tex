%% latex-leseansicht-vorspann.tex
%% Vorspann für die Leseansicht.
%% Lädt die gemeinsame Datei latex-vorspann.tex mit nicht gesetztem Schalter.

\newif\ifkorrekturansicht
\korrekturansichtfalse

\input{../tex-inputs/latex-vorspann}


\section[Richard Beer-Hofmann an Arthur Schnitzler, {{[}}17. 9. 1904{{]}}]{L01447 Richard Beer-Hofmann an Arthur Schnitzler, {[}17. 9. 1904{]}}
\nopagebreak\mylabel{L01447v}
\rehead{ }\normalsize\beginnumbering\briefempfaengerindex{Schnitzler, Arthur@\textsc{Schnitzler, Arthur}!zzzBeer-Hofmann, Richard@\emph{von Richard Beer-Hofmann}!1904-09-171@{17. 9. 1904}|(be}
\toendnotes[C]{\smallbreak\pagebreak[2]}
\correspDesc{Versand  durch Richard Beer-Hofmann am 17. 9. 1904 in Salzburg
\newline{}Erhalt  durch Arthur Schnitzler am 17. 9. 1904 in Salzburg}\toendnotes[C]{\smallbreak}
\Standort{CUL, Schnitzler, B 8.}
\physDesc{Brief, 1 Blatt, 1 Seite, 142 Zeichen
\newline{}Handschrift: Bleistift, lateinische Kurrent
\newline{}Schnitzler: mit Bleistift beschriftet: »Salzburg\oindex{Salzburg@\textbf{Salzburg}, \emph{Verwaltungsgebiet}|pw}{ }17/9 904« 
\newline{}Ordnung: mit Bleistift von unbekannter Hand nummeriert:
                                    »191« }\toendnotes[C]{\smallbreak}
\pstart
           \noindent{}{\pb}Gehe zuerst Buchhandlung Höllriegl\orgindex{Buchhandlung und Verlag Eduard Höllrigel@Buchhandlung und Verlag Eduard Höllrigel|pw} (Kerber\orgindex{Buchhandlung und Verlag Eduard Höllrigel@Buchhandlung und Verlag Eduard Höllrigel|pw}) beim Durchhaus\oindex{Ritzerhaus@\textbf{Ritzerhaus}, \emph{Gebäude}|pwv} auf den Marktplatz\oindex{Alter Markt@\textbf{Alter Markt}, \emph{Platz}|pw}, dann zu
                  \label{K_L01447-1v}\edtext{Svatek\orgindex{Wenzel Swatek@Wenzel Swatek|pw}}{\lemma{\textnormal{\emph{Svatek}}}\Cendnote{\textnormal{Sowohl bei \emph{Swatek}\orgindex{Wenzel Swatek@Wenzel Swatek|pwk} wie \emph{Schwarz}\orgindex{Karl Schwarz Antiquitäten@Karl Schwarz Antiquitäten|pwk} handelt
                  es sich um Antiquitätenhändler.}}}\label{K_L01447-1} (Parterre oder I Stock) dann zu Schwarz\orgindex{Karl Schwarz Antiquitäten@Karl Schwarz Antiquitäten|pw}{ }Kaigasse\oindex{Kaigasse@\textbf{Kaigasse}, \emph{Straße}|pw}.\pend
           \pstart \spacefill\mbox{Richard}\pend{}\selectlanguage{ngerman}\endnumbering\briefempfaengerindex{Schnitzler, Arthur@\textsc{Schnitzler, Arthur}!zzzBeer-Hofmann, Richard@\emph{von Richard Beer-Hofmann}!1904-09-171@{17. 9. 1904}|)be}\mylabel{L01447h}  \newcommand{\dateiname}{L01447}\newcommand{\titel}{Richard Beer-Hofmann an Arthur Schnitzler, [17. 9. 1904]}\newcommand{\editorInnen}{Martin Anton Müller und Gerd-Hermann Susen}%% latex-leseansicht-abspann.tex
%% Abspann für die Leseansicht.
%% Der Schalter \ifkorrekturansicht ist bereits durch den Vorspann gesetzt.

%% latex-abspann.tex
%% Gemeinsamer Abspann für Korrekturansicht und Leseansicht.
%% Setzt den Schalter \ifkorrekturansicht voraus (gesetzt in den
%% einbindenden Dateien latex-korrekturansicht-abspann.tex bzw.
%% latex-leseansicht-abspann.tex).
%% ---------------------------------------------------------------

\normalsize

% Das esempio-Environment wird nur in der Leseansicht benötigt
\ifkorrekturansicht\else
\newenvironment{esempio}[3]%
{
    \vspace{1.5ex}
    \rlap{\underline{#1}}
    \par
    \setlength{\parindent}{0cm}
    \nopagebreak
    \leftskip=#2cm
    \rightskip=#3cm
}
{
    \par
}
\fi

\doendnotes{C}
\bigskip
\vfill

\clearpage

\footnotesize

\ifkorrekturansicht
  \lohead{\textsc{register}}
\fi

% theindex-Environment neu definieren ohne reledmac
\makeatletter
\renewenvironment{theindex}{%
  \ifkorrekturansicht
    \section*{\indexname}%
  \else
    \subsubsection*{Index der erwähnten Entitäten}%
  \fi
  \setlength{\parindent}{0pt}%
  \setlength{\parskip}{0pt plus 0.3pt}%
  \let\item\@idxitem
}{%
  \ifkorrekturansicht\clearpage\fi
}
\makeatother

\IfFileExists{\jobname-pw.ind}{\input{\jobname-pw.ind}}{}

% Quellenangabe nur in der Leseansicht
\ifkorrekturansicht\else
% Fallback-Definitionen, falls die .tex-Datei \titel etc. nicht gesetzt hat
\providecommand{\titel}{}
\providecommand{\editorInnen}{}
\providecommand{\dateiname}{\jobname}

\vspace{3cm}

\vfill

\footnotesize
\textsc{Quelle}: \titel. Herausgegeben von {\editorInnen}. In: \emph{Arthur Schnitzler: Briefwechsel mit Autorinnen und Autoren}.
 Digitale Edition, https://schnitzler-briefe.acdh.oeaw.ac.at/{\dateiname}.html (Stand \today)
\fi

\end{document}


