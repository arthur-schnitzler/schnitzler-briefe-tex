%% latex-korrekturansicht-vorspann.tex
%% Vorspann für die Korrekturansicht.
%% Lädt die gemeinsame Datei latex-vorspann.tex mit gesetztem Schalter.

\newif\ifkorrekturansicht
\korrekturansichttrue

\input{../tex-inputs/latex-vorspann}


\section[ Paul Goldmann an Arthur Schnitzler, 26. 12. 1910]{L03471 Paul Goldmann an Arthur Schnitzler, 26. 12. 1910}
\nopagebreak\mylabel{L03471v}
\rehead{ }\normalsize\beginnumbering\briefempfaengerindex{Schnitzler, Arthur@\textsc{Schnitzler, Arthur}!zzzGoldmann, Paul@\emph{von Paul Goldmann}!1910-12-261@{26. 12. 1910}|(be}
\toendnotes[C]{\smallbreak\pagebreak[2]}\Standort{DLA, A:Schnitzler, HS.NZ85.1.3175.}
\physDesc{Postkarte, 567 Zeichen
\newline{}Handschrift: 1) schwarze Tinte, deutsche Kurrent\hspace{1em}2) schwarze Tinte, lateinische Kurrent (\noindent{}Adresse)\hspace{1em}
\newline{}Versand: Stempel: »\nobreak{}\oindex{I., Innere Stadt@\textbf{I., Innere Stadt}, \emph{A.ADM3}|pwk}Wien 1/1, 26 XII 10, 1 40N\nobreak{}«.  
\newline{}Ordnung: mit Bleistift »G\pwindex{Goldmann, Paul 31.01.1865 – 25.09.1935@\textsc{Goldmann, Paul} (31.01.1865 – 25.09.1935), \emph{Schriftsteller/Schriftstellerin, Journalist/Journalistin}|pw}« vermerkt }\toendnotes[C]{\smallbreak}\pstart{}{\pb}Herrn\pend{}\pstart{}Dr. Arthur Schnitzler\pend{}\pstart{}Wien\oindex{Wien@\textbf{Wien}, \emph{A.ADM2}|pw}\pend{}\pstart{}Sternwartstraſse 71\oindex{Sternwartestrasse 71@\textbf{Sternwartestraße 71}, \emph{Wohngebäude (K.WHS)}|pw}.\pend{}{\bigskip}\vspace{1em}
\pstart
           Montag 26. 12. 10.\pend
           
\pstart{}Lieber Freund,\pend\vspace{0.5em}
\pstart
           Darf ich heut{ }Abend mit \textsc{Schwarzkopf\pwindex{Schwarzkopf, Gustav 07.11.1853 – 13.11.1939@\textsc{Schwarzkopf, Gustav} (07.11.1853 – 13.11.1939), \emph{Schriftsteller/Schriftstellerin}|pw}} ſo um 8 herum{ }\label{K_L03471-1v}\edtext{zu Dir kommen}{\lemma{\textnormal{\emph{zu Dir kommen}}}\Cendnote{\textnormal{Bei dem Besuch Goldmanns\pwindex{Goldmann, Paul 31.01.1865 – 25.09.1935@\textsc{Goldmann, Paul} (31.01.1865 – 25.09.1935), \emph{Schriftsteller/Schriftstellerin, Journalist/Journalistin}|pwk} – in Begleitung von Gustav
                     Schwarzkopf\pwindex{Schwarzkopf, Gustav 07.11.1853 – 13.11.1939@\textsc{Schwarzkopf, Gustav} (07.11.1853 – 13.11.1939), \emph{Schriftsteller/Schriftstellerin}|pwk} – kam es zu einem Streit, der sich zwei Tage später noch
                  intensivierte (vgl. A. S.: \emph{Tagebuch}, 26. 12. 1910
                  und 28. 12. 1910).
                  Die Themen, die die bereits angeschlagene Beziehung zwischen Goldmann\pwindex{Goldmann, Paul 31.01.1865 – 25.09.1935@\textsc{Goldmann, Paul} (31.01.1865 – 25.09.1935), \emph{Schriftsteller/Schriftstellerin, Journalist/Journalistin}|pwk} und Schnitzler nun endgültig ins Wanken brachten, schlagen sich in den
                  folgenden Briefen nieder.}}}\label{K_L03471-1}? Wenn ja, ſo erbitte ich mir {\pb}Antwort durch Rohrpoſtkarte ins \textsc{Hotel Sacher\oindex{Hotel Sacher@\textbf{Hotel Sacher}, \emph{Hotel (K.HTL)}|pw}}, wo ich ſie mir gegen 7 Uhr Abends abholen werde. Haſt Du aber
               über den Abend bereits verfügt (was ſehr wahrſcheinlich iſt), ſo
               brauchſt Du gar nicht zu antworten, u. ich verſuche dann in einigen Tagen (morgen muß ich zu \textsc{Benedikt\pwindex{Benedikt, Moriz 27.05.1849 – 18.03.1920@\textsc{Benedikt, Moriz} (27.05.1849 – 18.03.1920), \emph{Journalist/Journalistin, Herausgeber/Herausgeberin}|pw}} auf den \textsc{Semmering\oindex{Semmering@\textbf{Semmering}, \emph{A.ADM3}|pw}}) von Neuem, Dich zu erreichen. Herzliche Grüße Dir u. Deiner Frau\pwindex{Schnitzler, Olga 17.01.1882 – 13.01.1970@\textsc{Schnitzler, Olga} (17.01.1882 – 13.01.1970), \emph{Schauspieler/Schauspielerin, Sänger/Sängerin}|pwv} von Deinem \spacefill\mbox{Paul
                  Goldmann.}\pend
           
\pstart
           \noindent{}\label{T_L03471-1v}\edtext{\textsc{Schwarzkopf\pwindex{Schwarzkopf, Gustav 07.11.1853 – 13.11.1939@\textsc{Schwarzkopf, Gustav} (07.11.1853 – 13.11.1939), \emph{Schriftsteller/Schriftstellerin}|pw}} verſtändige \uline{ich}.}{\lemma{\textnormal{\emph{Schwarzkopf … ich.}}}\Cendnote{\textnormal{seitlich am rechten Rand, verkehrt zum Text}}}\label{T_L03471-1}\pend
           \selectlanguage{ngerman}\endnumbering\briefempfaengerindex{Schnitzler, Arthur@\textsc{Schnitzler, Arthur}!zzzGoldmann, Paul@\emph{von Paul Goldmann}!1910-12-261@{26. 12. 1910}|)be}\mylabel{L03471h}  \normalsize

\doendnotes{C}
\bigskip
\vfill

\clearpage

\footnotesize

\lohead{\textsc{register}}

% Definiere theindex-Environment komplett neu ohne reledmac
\makeatletter
\renewenvironment{theindex}{%
  \section*{\indexname}%
  \setlength{\parindent}{0pt}%
  \setlength{\parskip}{0pt plus 0.3pt}%
  \let\item\@idxitem
}{%
  \clearpage
}
\makeatother

\IfFileExists{\jobname-pw.ind}{\input{\jobname-pw.ind}}{}

\end{document}

      