%% latex-leseansicht-vorspann.tex
%% Vorspann für die Leseansicht.
%% Lädt die gemeinsame Datei latex-vorspann.tex mit nicht gesetztem Schalter.

\newif\ifkorrekturansicht
\korrekturansichtfalse

\input{../tex-inputs/latex-vorspann}


         \renewcommand{\erwaehnteInstitutionen}{Institutionen: An der schönen blauen Donau, Josef Eberle  Stein-, Buch und Musikaliendruckerei}
         \renewcommand{\erwaehnteOrte}{Orte: Berggasse, Seidengasse, Wien}
         \renewcommand{\erwaehnteWerke}{Werke: Mein Freund Ypsilon}
               \section[Fedor Mamroth und Paul Goldmann an Arthur Schnitzler, 9. 12. 1888]{ Fedor Mamroth und Paul Goldmann an Arthur Schnitzler,
               9. 12. 1888}\nopagebreak\mylabel{v}\rehead{ }\begin{ledgroupsized}[t]{13cm}\normalsize\beginnumbering \toendnotes[C]{\smallbreak\pagebreak[2]} \Standort{DLA, A:Schnitzler, HS.NZ85.1.3162.}
\physDesc{Brief, 1 Blatt, 4 Seiten
\newline{}Handschrift Paul Goldmann: blaue Tinte, deutsche Kurrent}\toendnotes[C]{\smallbreak}\pstart
           \noindent{}\centering{}{\pb}\textcolor{gray}{\textbf{\textbf{Adminiſtration: VII.
                           Seidengaſſe 7\oindex{Seidengasse@\textbf{Seidengasse}|pw}} (Jos. Eberle {\kaufmannsund} Co.\orgindex{Josef Eberle Stein-, Buch und Musikaliendruckerei@Josef Eberle Stein-, Buch und Musikaliendruckerei|pw})}}\pend
           \pstart
           \noindent{}\centering{}\textcolor{gray}{\textbf{An der Schönen Blauen Donau\orgindex{der schoenen blauen Donau@An der schönen blauen Donau|pw}}}\pend
           \pstart
           \noindent{}\centering{}\textcolor{gray}{\textbf{Chef-Redacteur: Dr. F. Mamroth – Redaction: IX., Berggaſſe 31\oindex{Berggasse@\textbf{Berggasse}|pw}.}}\pend
           \pstart
           \raggedleft{}\textcolor{gray}{\textbf{Wien\oindex{Wien@\textbf{Wien}|pw}, den}}{ }9. Dezember \textcolor{gray}{\textbf{18}}88.\pend
           \pstart\center{}Hochgeehrter Herr!\pend\pstart
           Wir haben die \label{K_L02551-2v}\edtext{Erzählung\pwindex{Schnitzler, Arthur 15.05.1862 – 21.10.1931@\textsc{Schnitzler, Arthur} (15.05.1862 – 21.10.1931), \emph{Schriftsteller, Mediziner}!Mein Freund Ypsilon15. 01. 1889@\strich\emph{Mein Freund Ypsilon} {[}15. 01. 1889{]}|pwv}}{\lemma{\textnormal{\emph{Erzählung}}}\Cendnote{\textnormal{vgl. A. S.: \emph{Tagebuch}, 10. 12. 1888}}}\label{K_L02551-2h}, die Sie uns freundlichſt eingeſandt, mit dem lebhafteſten Intereſſe
               geleſen. Wir finden die Idee Ihrer Arbeit\pwindex{Schnitzler, Arthur 15.05.1862 – 21.10.1931@\textsc{Schnitzler, Arthur} (15.05.1862 – 21.10.1931), \emph{Schriftsteller, Mediziner}!Mein Freund Ypsilon15. 01. 1889@\strich\emph{Mein Freund Ypsilon} {[}15. 01. 1889{]}|pwv} originell und feſſelnd, die Durchführung recht gewandt; überhaupt
               ſcheint ſie uns zu einem neuen Genre zu gehören, das verdient kultiviert zu
               werden.\pend
           \pstart
           Wir ſind freilich auch mit einigem in Ihrer Arbeit\pwindex{Schnitzler, Arthur 15.05.1862 – 21.10.1931@\textsc{Schnitzler, Arthur} (15.05.1862 – 21.10.1931), \emph{Schriftsteller, Mediziner}!Mein Freund Ypsilon15. 01. 1889@\strich\emph{Mein Freund Ypsilon} {[}15. 01. 1889{]}|pwv} nicht {\pb}einverſtanden.
               Wir meinen, es dürfe nicht, wie das geſchieht, der Leſer bis zum Schluſſe im Unklaren
               gelaſſen werden, ob er einen Wahnſinnigen oder einen Phantaſten vor ſich hat. Wir
               glauben, es würde der Erzählung\pwindex{Schnitzler, Arthur 15.05.1862 – 21.10.1931@\textsc{Schnitzler, Arthur} (15.05.1862 – 21.10.1931), \emph{Schriftsteller, Mediziner}!Mein Freund Ypsilon15. 01. 1889@\strich\emph{Mein Freund Ypsilon} {[}15. 01. 1889{]}|pwv} entſchieden zum Vortheil gereichen, wenn das erzählende »Ich« als
               Mediziner hingeſtellt würde, der ſich über das Benehmen ſeines Freundes im Verlaufe
               der Entwicklung ziemlich entſchieden vom mediziniſchen Standpunkt ausſpräche; er
               braucht ihn ja nicht geradezu als irrſinnig zu erklären, aber er kann doch hier und
               da auf die flüſſige Grenze zwiſchen Wahnſinn und dichteriſchem Talent hinweiſen und
               ausdrücken, daß {\pb}der Fall ſeines Freundes in dieſes
               Grenzgebiet gehöre. Mit einem Worte: die Erzählung ſoll einen Stich ins
                  Mediziniſ\textcolor{gray}{c}he bekommen.\pend
           \pstart
           Wenn Sie, hochgeehrter Herr, ſich freundlichſt bereit finden, eine Änderung Ihrer Arbeit\pwindex{Schnitzler, Arthur 15.05.1862 – 21.10.1931@\textsc{Schnitzler, Arthur} (15.05.1862 – 21.10.1931), \emph{Schriftsteller, Mediziner}!Mein Freund Ypsilon15. 01. 1889@\strich\emph{Mein Freund Ypsilon} {[}15. 01. 1889{]}|pwv} in dieſem Sinne
               vorzunehmen, ſo ſind wir mit vielem Vergnügen bereit, dieſelbe in unſerem Blatte\orgindex{der schoenen blauen Donau@An der schönen blauen Donau|pwv} zu veröffentlichen.\pend
           \pstart
           Wir bitten Sie, uns baldgefälligſt antworten zu wollen, und empfehlen {\pb}uns Ihnen\pend
           \pstart
           Hochachtungsvoll{\\[\baselineskip]}\textcolor{gray}{\textbf{\textit{Die Redaction}}}{\\[\baselineskip]}\textcolor{gray}{\textbf{\textit{der}}}{\\[\baselineskip]}\textcolor{gray}{\textbf{\textit{»Schönen blauen Donau\orgindex{der schoenen blauen Donau@An der schönen blauen Donau|pw}«}}}{\\[\baselineskip]}\spacefill\mbox{\label{K_L02551-1v}\edtext{p.}{\lemma{\textnormal{\emph{p.}}}\Cendnote{\textnormal{für »per«, vgl. Fedor Mamroth an Arthur Schnitzler, 4. 4. 1894}}}\label{K_L02551-1h} Dr. F. Mamroth.}\pend
           \leftskip=0em{}
         
         \endnumbering\mylabel{h}\end{ledgroupsized}  \newcommand{\dateiname}{L02551}\newcommand{\titel}{Fedor Mamroth und Paul Goldmann an Arthur Schnitzler, 9. 12. 1888}\newcommand{\editorInnen}{Martin Anton Müller und Gerd-Hermann Susen}%% latex-leseansicht-abspann.tex
%% Abspann für die Leseansicht.
%% Der Schalter \ifkorrekturansicht ist bereits durch den Vorspann gesetzt.

%% latex-abspann.tex
%% Gemeinsamer Abspann für Korrekturansicht und Leseansicht.
%% Setzt den Schalter \ifkorrekturansicht voraus (gesetzt in den
%% einbindenden Dateien latex-korrekturansicht-abspann.tex bzw.
%% latex-leseansicht-abspann.tex).
%% ---------------------------------------------------------------

\normalsize

% Das esempio-Environment wird nur in der Leseansicht benötigt
\ifkorrekturansicht\else
\newenvironment{esempio}[3]%
{
    \vspace{1.5ex}
    \rlap{\underline{#1}}
    \par
    \setlength{\parindent}{0cm}
    \nopagebreak
    \leftskip=#2cm
    \rightskip=#3cm
}
{
    \par
}
\fi

\doendnotes{C}
\bigskip
\vfill

\clearpage

\footnotesize

\ifkorrekturansicht
  \lohead{\textsc{register}}
\fi

% theindex-Environment neu definieren ohne reledmac
\makeatletter
\renewenvironment{theindex}{%
  \ifkorrekturansicht
    \section*{\indexname}%
  \else
    \subsubsection*{Index der erwähnten Entitäten}%
  \fi
  \setlength{\parindent}{0pt}%
  \setlength{\parskip}{0pt plus 0.3pt}%
  \let\item\@idxitem
}{%
  \ifkorrekturansicht\clearpage\fi
}
\makeatother

\IfFileExists{\jobname-pw.ind}{\input{\jobname-pw.ind}}{}

% Quellenangabe nur in der Leseansicht
\ifkorrekturansicht\else
% Fallback-Definitionen, falls die .tex-Datei \titel etc. nicht gesetzt hat
\providecommand{\titel}{}
\providecommand{\editorInnen}{}
\providecommand{\dateiname}{\jobname}

\vspace{3cm}

\vfill

\footnotesize
\textsc{Quelle}: \titel. Herausgegeben von {\editorInnen}. In: \emph{Arthur Schnitzler: Briefwechsel mit Autorinnen und Autoren}.
 Digitale Edition, https://schnitzler-briefe.acdh.oeaw.ac.at/{\dateiname}.html (Stand \today)
\fi

\end{document}


      