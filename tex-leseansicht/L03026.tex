%% latex-korrekturansicht-vorspann.tex
%% Vorspann für die Korrekturansicht.
%% Lädt die gemeinsame Datei latex-vorspann.tex mit gesetztem Schalter.

\newif\ifkorrekturansicht
\korrekturansichttrue

\input{../tex-inputs/latex-vorspann}


\section[ Arthur Schnitzler an Felix Salten, 30. 5. 1931]{L03026 Arthur Schnitzler an Felix Salten, 30. 5. 1931}
\nopagebreak\mylabel{L03026v}
\rehead{ }\normalsize\beginnumbering\briefempfaengerindex{Salten, Felix@\textsc{Salten, Felix}!zzzSchnitzler, Arthur@\emph{von Arthur Schnitzler}!1931-05-301@{30. 5. 1931}|(be}
\toendnotes[C]{\smallbreak\pagebreak[2]}\Standort{Wienbibliothek im Rathaus, ZPH 1681, 2.1.516.}
\physDesc{Brief, 1 Blatt, 2 Seiten, 533 Zeichen
\newline{}Handschrift: schwarze Tinte, lateinische Kurrent
\newline{}Ordnung: mit Bleistift von unbekannter Hand nummeriert: »1« }
\buchAbdrucke{\weitereDrucke{Arthur Schnitzler: \emph{Briefe 1913–1931}. Frankfurt am Main: \emph{S. Fischer} 1984, S. 792.} }\toendnotes[C]{\smallbreak}
\pstart
           \raggedleft{}{\pb}Wien\oindex{Wien@\textbf{Wien}, \emph{A.ADM2}|pw}, 30. 5. 931\pend
           \vspace{0.5em}
\pstart
           lieber, ich danke Ihnen sehr herzlich für die freundliche Uebersendg
               Ihres \label{K_L03026-1v}\edtext{Amerika\oindex{Amerika@\textbf{Amerika}, \emph{kein passender Code gefunden}|pw} Buchs\pwindex{Fuenf Minuten Amerika@\emph{Fünf Minuten Amerika}|pwv} und der persönlichen
                  Widmung}{\lemma{\textnormal{\emph{Amerika … Widmung}}}\Cendnote{\textnormal{Siehe Felix Salten: Widmungsexemplar Fünf Minuten Amerika für Arthur
               Schnitzler, [zwischen 1. und 29.?] 5. 1931.
               }}}\label{K_L03026-1}{[}.{]} Daſs ich im übrigen so wenig von mir sehen und hören
                  lasse\textcolor{gray}{,} bitte ich Sie damit zu entschuldig\textcolor{gray}{en},
               daſs ich mich, sowohl seelisch als körperlich, aber sagen wir der Einfachheit halber
               mit den »Nerven« nicht übermäßg wohl und insbesondre höchst ungesellig befinde. Ich
               nehme an dſs wieder {\pb}eine bessere Periode
                  ko{\geminationm}en wird und dann meld ich mich.\pend
           
\pstart
           Sein Sie bis dahin vielmals und freundschaftlich gegrüßt {\\[\baselineskip]}Ihr {\\[\baselineskip]}\spacefill\mbox{Arth}\pend
           \leftskip=0em{}\selectlanguage{ngerman}\endnumbering\briefempfaengerindex{Salten, Felix@\textsc{Salten, Felix}!zzzSchnitzler, Arthur@\emph{von Arthur Schnitzler}!1931-05-301@{30. 5. 1931}|)be}\mylabel{L03026h}  \normalsize

\doendnotes{C}
\bigskip
\vfill

\clearpage

\footnotesize

\lohead{\textsc{register}}

% Definiere theindex-Environment komplett neu ohne reledmac
\makeatletter
\renewenvironment{theindex}{%
  \section*{\indexname}%
  \setlength{\parindent}{0pt}%
  \setlength{\parskip}{0pt plus 0.3pt}%
  \let\item\@idxitem
}{%
  \clearpage
}
\makeatother

\IfFileExists{\jobname-pw.ind}{\input{\jobname-pw.ind}}{}

\end{document}

      