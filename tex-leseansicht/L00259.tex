%% latex-leseansicht-vorspann.tex
%% Vorspann für die Leseansicht.
%% Lädt die gemeinsame Datei latex-vorspann.tex mit nicht gesetztem Schalter.

\newif\ifkorrekturansicht
\korrekturansichtfalse

\input{../tex-inputs/latex-vorspann}


         
         \renewcommand{\erwaehntePersonen}{Personen: Hugo von Hofmannsthal, Felix Salten,  Tizian}
         \renewcommand{\erwaehnteOrte}{Orte: Cortina d'Ampezzo, Pieve di Cadore, Strobl, Toblach, Valle di Cadore, Valle d’Ampezzo, Wien}
         \renewcommand{\erwaehnteWerke}{Werke: Der Tod des Tizian. Ein Bruchstück}
               \section[Arthur Schnitzler und Felix Salten an Hugo von Hofmannsthal, 24. 8. 1893]{ Arthur Schnitzler und Felix Salten an Hugo von Hofmannsthal,
               24. 8. 1893}\nopagebreak\mylabel{v}\rehead{ }\begin{ledgroupsized}[t]{13cm}\normalsize\beginnumbering\briefempfaengerindex{Hofmannsthal, Hugo von@\textsc{Hofmannsthal, Hugo von}!zzzSalten, Felix@\emph{von Felix Salten}!1893-08-241@{24. 8. 1893}|(be}\briefempfaengerindex{Hofmannsthal, Hugo von@\textsc{Hofmannsthal, Hugo von}!zzzSchnitzler, Arthur@\emph{von Arthur Schnitzler}!1893-08-241@{24. 8. 1893}|(be} \toendnotes[C]{\smallbreak\pagebreak[2]} \Standort{FDH, Hs-30885,11.}
\physDesc{Brief, 1 Blatt, 4 Seiten, 1391 Zeichen (Briefpapier mit Trauerrand)
\newline{}Handschrift Arthur Schnitzler: Bleistift, deutsche Kurrent\newline{}Handschrift Felix Salten: Bleistift, deutsche Kurrent
\newline{}Hofmannsthal: mit Bleistift Vermerk: »\uline{Launiger Brief}« und Ergänzung: »›Des Meiſters von Cadore\pwindex{Tizian zwischen 1488 und 1490 – 27.08.1576@\textsc{Tizian} (zwischen 1488 und 1490 – 27.08.1576), \emph{Maler}|pwv}
                                    reiche Farben‹ – Th. Morren. –« 
\newline{}Ordnung: mit Bleistift von Schnitzler mutmaßlich bei der Durchsicht der Briefe
                                    1929 auf der ersten Seite datiert: »24/8 93« }\buchAbdrucke{\weitereDrucke{Hugo von Hofmannsthal, Arthur Schnitzler: \emph{Briefwechsel}. Hg. Therese Nickl und Heinrich Schnitzler. Frankfurt am Main: \emph{S. Fischer} 1964, S. 44–45.} }\toendnotes[C]{\smallbreak}\pstart
           \noindent{}{\pb}{[}hs. Salten:{]} \introOben{}\uuline{Launiger Brief}\introOben{}\pend
           \pstart
           {[}hs. Schnitzler:{]} Mein lieber Hugo, Sie haben allerdings Tizians Tod\pwindex{Hofmannsthal, Hugo von 1874-02-01 – 1929-07-15@\textsc{Hofmannsthal, Hugo von} (1874-02-01 – 1929-07-15), \emph{Schriftsteller}!Tod des Tizian. Ein BruchstueckOktober 1892@\strich\emph{Der Tod des Tizian. Ein Bruchstück} {[}Oktober 1892{]}|pw} geſchrieben, wir aber haben ſoeben das Zi{\geminationm}er betreten, in welchem Tizian\pwindex{Tizian zwischen 1488 und 1490 – 27.08.1576@\textsc{Tizian} (zwischen 1488 und 1490 – 27.08.1576), \emph{Maler}|pw} geboren ward. Wir ſind nemlich in \textsc{Pieve di Cadore}\oindex{Pieve di Cadore@\textbf{Pieve di Cadore}|pw}; heute früh von \textsc{Toblach}\oindex{Toblach@\textbf{Toblach}|pw} mit unſeren Rädern abgefahren, und über \textsc{Cortina}\oindex{Cortina d'Ampezzo@\textbf{Cortina d'Ampezzo}|pw} hieher – manchmal {\pb}unter Hagel und Regen, und
               keineswegs ohne daſs uns die Zollbehörden anhielten.  – Hier haben wir in den paar
               Stunden unſres Aufenthaltes viel Schönheit und Leben geſehen: blonde Kinder\footnote{\noindent{}Schönheit}, die auf ſteinernen Löwen\footnote{\noindent{}Leben}{ }ſpielten, andre wieder, die »Muſikbande« ſpielten
               und wo der Kapellmeiſter ſeine ſämtlichen auf {\pb}Holzſtäben
               und Löffeln muſicirenden Untergebenen jä{\geminationm}erlich
                  prügelte.\footnote{\noindent{}Schönheit} Ein altes Weib,\footnote{\noindent{}Leben} das von Haus zu Haus ging und die kleinen Kinder küſſte, ein Kerl, der zum
               Fenſter hinausſchaute und dem Strümpfe\footnote{\noindent{}Schönheit} zum Mund heraushingen, mit welchen ich, wie \textsc{Salten}\pwindex{Salten, Felix 06.09.1869 – 08.10.1945@\textsc{Salten, Felix} (06.09.1869 – 08.10.1945), \emph{Schriftsteller, Journalist}|pw} meint, verbleiben ſoll\pend
           \pstart Ihr hoch- u rad-fahrender \spacefill\mbox{ArthSch.}\pend{}\pstart
           \noindent{}{\pb}{[}hs. Salten:{]} lieber Freund! Die Fahrt durch die Pracht des Ampezzo\oindex{Valle DAmpezzo@\textbf{Valle d’Ampezzo}|pw} u Cadore Thales\oindex{XXXX Ortsangabe fehlt|pw}
               und der Aufenthalt hier haben gelehrt: Es genügt nicht, dass der Mensch den Tod des Tizian\pwindex{Hofmannsthal, Hugo von 1874-02-01 – 1929-07-15@\textsc{Hofmannsthal, Hugo von} (1874-02-01 – 1929-07-15), \emph{Schriftsteller}!Tod des Tizian. Ein BruchstueckOktober 1892@\strich\emph{Der Tod des Tizian. Ein Bruchstück} {[}Oktober 1892{]}|pw} schreibe, er muss auch Bicycle
               fahren können. Ersteres haben Sie gethan, das Zweite bleibt Ihnen noch. Wir
               allerdings haben beim zweiten angefangen, und das Schwierigere steht uns noch bevor,
               was wir, wie Arthur meint, heute ’mal versuchen wollen.\pend
           \pstart
           Herzlichst{\\}Ihr{\\}\spacefill\mbox{Salten}\pend
           \pstart
           \noindent{}{[}hs. Schnitzler:{]} \label{T_L00259-1v}\edtext{\textsc{Pieve di Cadore}\oindex{Pieve di Cadore@\textbf{Pieve di Cadore}|pw}}{\lemma{\textnormal{\emph{Pieve di Cadore}}}\Cendnote{\textnormal{dies und das Folgende am unteren
                  Blattrand auf dem Kopf. Möglicherweise handelt es sich um den ursprünglichen
                  Briefkopf?}}}\label{T_L00259-1h}\pend
           \pstart
           {[}hs. Salten:{]} den 24. August 93\pend
           \pstart
           Ein \label{K_L00259-1v}\edtext{Jahr}{\lemma{\textnormal{\emph{Jahr}}}\Cendnote{\textnormal{siehe A. S.: \emph{Tagebuch}, 31. 8. 1892}}}\label{K_L00259-1h}, nach dem Loris in Strobl\oindex{Strobl@\textbf{Strobl}|pw} seinen Freunden
                  »Tizians Tod\pwindex{Hofmannsthal, Hugo von 1874-02-01 – 1929-07-15@\textsc{Hofmannsthal, Hugo von} (1874-02-01 – 1929-07-15), \emph{Schriftsteller}!Tod des Tizian. Ein BruchstueckOktober 1892@\strich\emph{Der Tod des Tizian. Ein Bruchstück} {[}Oktober 1892{]}|pw}« las.\pend
           
         
         \endnumbering\mylabel{h}\end{ledgroupsized}  \newcommand{\dateiname}{L00259}\newcommand{\titel}{Arthur Schnitzler und Felix Salten an Hugo von Hofmannsthal, 24. 8. 1893}\newcommand{\editorInnen}{Martin Anton Müller und Gerd-Hermann Susen}%% latex-leseansicht-abspann.tex
%% Abspann für die Leseansicht.
%% Der Schalter \ifkorrekturansicht ist bereits durch den Vorspann gesetzt.

%% latex-abspann.tex
%% Gemeinsamer Abspann für Korrekturansicht und Leseansicht.
%% Setzt den Schalter \ifkorrekturansicht voraus (gesetzt in den
%% einbindenden Dateien latex-korrekturansicht-abspann.tex bzw.
%% latex-leseansicht-abspann.tex).
%% ---------------------------------------------------------------

\normalsize

% Das esempio-Environment wird nur in der Leseansicht benötigt
\ifkorrekturansicht\else
\newenvironment{esempio}[3]%
{
    \vspace{1.5ex}
    \rlap{\underline{#1}}
    \par
    \setlength{\parindent}{0cm}
    \nopagebreak
    \leftskip=#2cm
    \rightskip=#3cm
}
{
    \par
}
\fi

\doendnotes{C}
\bigskip
\vfill

\clearpage

\footnotesize

\ifkorrekturansicht
  \lohead{\textsc{register}}
\fi

% theindex-Environment neu definieren ohne reledmac
\makeatletter
\renewenvironment{theindex}{%
  \ifkorrekturansicht
    \section*{\indexname}%
  \else
    \subsubsection*{Index der erwähnten Entitäten}%
  \fi
  \setlength{\parindent}{0pt}%
  \setlength{\parskip}{0pt plus 0.3pt}%
  \let\item\@idxitem
}{%
  \ifkorrekturansicht\clearpage\fi
}
\makeatother

\IfFileExists{\jobname-pw.ind}{\input{\jobname-pw.ind}}{}

% Quellenangabe nur in der Leseansicht
\ifkorrekturansicht\else
% Fallback-Definitionen, falls die .tex-Datei \titel etc. nicht gesetzt hat
\providecommand{\titel}{}
\providecommand{\editorInnen}{}
\providecommand{\dateiname}{\jobname}

\vspace{3cm}

\vfill

\footnotesize
\textsc{Quelle}: \titel. Herausgegeben von {\editorInnen}. In: \emph{Arthur Schnitzler: Briefwechsel mit Autorinnen und Autoren}.
 Digitale Edition, https://schnitzler-briefe.acdh.oeaw.ac.at/{\dateiname}.html (Stand \today)
\fi

\end{document}


      