%% latex-korrekturansicht-vorspann.tex
%% Vorspann für die Korrekturansicht.
%% Lädt die gemeinsame Datei latex-vorspann.tex mit gesetztem Schalter.

\newif\ifkorrekturansicht
\korrekturansichttrue

\input{../tex-inputs/latex-vorspann}


\section[Arthur Schnitzler und Felix Salten an Hugo von Hofmannsthal, 24. 8. 1893]{L00259 Arthur Schnitzler und Felix Salten an Hugo von Hofmannsthal,
               24. 8. 1893}
\nopagebreak\mylabel{L00259v}
\rehead{ }\normalsize\beginnumbering\briefempfaengerindex{Hofmannsthal, Hugo von@\textsc{Hofmannsthal, Hugo von}!zzzSalten, Felix@\emph{von Felix Salten}!1893-08-241@{24. 8. 1893}|(be}\briefempfaengerindex{Hofmannsthal, Hugo von@\textsc{Hofmannsthal, Hugo von}!zzzSchnitzler, Arthur@\emph{von Arthur Schnitzler}!1893-08-241@{24. 8. 1893}|(be}
\toendnotes[C]{\smallbreak\pagebreak[2]}\Standort{FDH, Hs-30885,11.}
\physDesc{Brief, 1 Blatt, 4 Seiten, 1391 Zeichen (Briefpapier mit Trauerrand)
\newline{}Handschrift Arthur Schnitzler: Bleistift, deutsche Kurrent
\newline{}Handschrift Felix Salten: Bleistift, deutsche Kurrent
\newline{}Hofmannsthal: mit Bleistift Vermerk: »\uline{Launiger Brief}« und Ergänzung: »›Des Meiſters von Cadore\pwindex{Tizian zwischen 1488 und 1490 – 27.08.1576@\textsc{Tizian} (zwischen 1488 und 1490 – 27.08.1576), \emph{Maler/Malerin}|pwv}
                                    reiche Farben‹ – Th. Morren. –« 
\newline{}Ordnung: mit Bleistift von Schnitzler mutmaßlich bei der Durchsicht der Briefe
                                    1929 auf der ersten Seite datiert: »24/8 93« }
\buchAbdrucke{\weitereDrucke{Hugo von Hofmannsthal, Arthur Schnitzler: \emph{Briefwechsel}. Frankfurt am Main: \emph{S. Fischer} 1964, S. 44–45.} }\toendnotes[C]{\smallbreak}
\pstart
           {\pb}{[}hs. :{]} \introOben{}\uuline{Launiger Brief}\introOben{}\pend
           \vspace{0.5em}
\pstart
           {[}hs. :{]} Mein lieber Hugo, Sie haben allerdings Tizians Tod\pwindex{Tod des Tizian. Ein Bruchstueck@\emph{Der Tod des Tizian. Ein Bruchstück}|pw} geſchrieben, wir aber haben ſoeben das Zi{\geminationm}er betreten, in welchem Tizian\pwindex{Tizian zwischen 1488 und 1490 – 27.08.1576@\textsc{Tizian} (zwischen 1488 und 1490 – 27.08.1576), \emph{Maler/Malerin}|pw} geboren ward. Wir ſind nemlich in \textsc{Pieve di Cadore}\oindex{Pieve di Cadore@\textbf{Pieve di Cadore}, \emph{A.ADM3}|pw}; heute früh von \textsc{Toblach}\oindex{Toblach@\textbf{Toblach}, \emph{A.ADM3}|pw} mit unſeren Rädern abgefahren, und über \textsc{Cortina}\oindex{Cortina DAmpezzo@\textbf{Cortina d’Ampezzo}, \emph{P.PPLA3}|pw} hieher – manchmal {\pb}unter Hagel und Regen, und
               keineswegs ohne daſs uns die Zollbehörden anhielten.  – Hier haben wir in den paar
               Stunden unſres Aufenthaltes viel Schönheit und Leben geſehen: blonde Kinder\noindent{}Schönheit, die auf ſteinernen Löwen\noindent{}Leben{ }ſpielten, andre wieder, die »Muſikbande« ſpielten
               und wo der Kapellmeiſter ſeine ſämtlichen auf {\pb}Holzſtäben
               und Löffeln muſicirenden Untergebenen jä{\geminationm}erlich
                  prügelte.\noindent{}Schönheit Ein altes Weib,\noindent{}Leben das von Haus zu Haus ging und die kleinen Kinder küſſte, ein Kerl, der zum
               Fenſter hinausſchaute und dem Strümpfe\noindent{}Schönheit zum Mund heraushingen, mit welchen ich, wie \textsc{Salten}\pwindex{Salten, Felix 06.09.1869 – 08.10.1945@\textsc{Salten, Felix} (06.09.1869 – 08.10.1945), \emph{Schriftsteller/Schriftstellerin, Journalist/Journalistin, Chefredakteur/Chefredakteurin}|pw} meint, verbleiben ſoll\pend
           \pstart Ihr hoch- u rad-fahrender \spacefill\mbox{ArthSch.}\pend{}\selectlanguage{ngerman}\vspace{1em}
\pstart
           \noindent{}{\pb}{[}hs. :{]} lieber Freund! Die Fahrt durch die Pracht des Ampezzo\oindex{Valle DAmpezzo@\textbf{Valle d’Ampezzo}, \emph{T.VAL}|pw} u Cadore Thales\oindex{Valle di Cadore@\textbf{Valle di Cadore}, \emph{A.ADM3}|pw}
               und der Aufenthalt hier haben gelehrt: Es genügt nicht, dass der Mensch den Tod des Tizian\pwindex{Tod des Tizian. Ein Bruchstueck@\emph{Der Tod des Tizian. Ein Bruchstück}|pw} schreibe, er muss auch Bicycle
               fahren können. Ersteres haben Sie gethan, das Zweite bleibt Ihnen noch. Wir
               allerdings haben beim zweiten angefangen, und das Schwierigere steht uns noch bevor,
               was wir, wie Arthur meint, heute ’mal versuchen wollen.\pend
           
\pstart
           Herzlichst{\\}Ihr{\\}\spacefill\mbox{Salten}\pend
           \selectlanguage{ngerman}\vspace{1em}
\pstart
           \noindent{}{[}hs. :{]} \label{T_L00259-1v}\edtext{\textsc{Pieve di Cadore}\oindex{Pieve di Cadore@\textbf{Pieve di Cadore}, \emph{A.ADM3}|pw}}{\lemma{\textnormal{\emph{Pieve di Cadore}}}\Cendnote{\textnormal{Dies und das Folgende steht am unteren
                  Blattrand auf dem Kopf. Möglicherweise handelt es sich um den ursprünglichen
                  Briefkopf.}}}\label{T_L00259-1}\pend
           
\pstart
           {[}hs. :{]} den 24. August 93\pend
           
\pstart
           Ein \label{K_L00259-1v}\edtext{Jahr}{\lemma{\textnormal{\emph{Jahr}}}\Cendnote{\textnormal{Siehe A. S.: \emph{Tagebuch}, 31. 8. 1892.
               }}}\label{K_L00259-1}, nach dem Loris in Strobl\oindex{Strobl@\textbf{Strobl}, \emph{A.ADM3}|pw} seinen Freunden
                  »Tizians Tod\pwindex{Tod des Tizian. Ein Bruchstueck@\emph{Der Tod des Tizian. Ein Bruchstück}|pw}« las.\pend
           \selectlanguage{ngerman}\endnumbering\briefempfaengerindex{Hofmannsthal, Hugo von@\textsc{Hofmannsthal, Hugo von}!zzzSalten, Felix@\emph{von Felix Salten}!1893-08-241@{24. 8. 1893}|)be}\briefempfaengerindex{Hofmannsthal, Hugo von@\textsc{Hofmannsthal, Hugo von}!zzzSchnitzler, Arthur@\emph{von Arthur Schnitzler}!1893-08-241@{24. 8. 1893}|)be}\mylabel{L00259h}  \normalsize

\doendnotes{C}
\bigskip
\vfill

\clearpage

\footnotesize

\lohead{\textsc{register}}

% Definiere theindex-Environment komplett neu ohne reledmac
\makeatletter
\renewenvironment{theindex}{%
  \section*{\indexname}%
  \setlength{\parindent}{0pt}%
  \setlength{\parskip}{0pt plus 0.3pt}%
  \let\item\@idxitem
}{%
  \clearpage
}
\makeatother

\IfFileExists{\jobname-pw.ind}{\input{\jobname-pw.ind}}{}

\end{document}

      