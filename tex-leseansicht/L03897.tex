%% latex-leseansicht-vorspann.tex
%% Vorspann für die Leseansicht.
%% Lädt die gemeinsame Datei latex-vorspann.tex mit nicht gesetztem Schalter.

\newif\ifkorrekturansicht
\korrekturansichtfalse

\input{../tex-inputs/latex-vorspann}


\section[Sigmund Freud: Widmungsexemplar von Psychoanalytische Studien an Werken der Dichtung und Kunst, [erste Hälfte] November 1924]{L03897 Sigmund Freud: Widmungsexemplar von Psychoanalytische Studien an Werken
               der Dichtung und Kunst, [erste Hälfte] November 1924}
\nopagebreak\mylabel{L03897v}
\rehead{ }\normalsize\beginnumbering\briefempfaengerindex{Schnitzler, Arthur@\textsc{Schnitzler, Arthur}!zzzFreud, Sigmund@\emph{von Sigmund Freud}!1924-11-151@{[erste Hälfte] November 1924}|(be}
\toendnotes[C]{\smallbreak\pagebreak[2]}
\correspDesc{Versand  durch Sigmund Freud im Zeitraum [erste Hälfte] November
                  1924 in Wien
\newline{}Erhalt  durch Arthur Schnitzler im Zeitraum [erste Hälfte] November
                  1924 in Wien}\toendnotes[C]{\smallbreak}
\Standort{Wien, Österreichische Nationalbibliothek, 231446-B Rara 774.}
\physDesc{Widmung am Vorsatzblatt, 69 Zeichen
\newline{}Handschrift: schwarze Tinte, deutsche Kurrent
\newline{}Zusatz: Auf die Existenz des Exemplars hat erstmals Luigi Reitani
                                 aufmerksam gemacht (Arthur
                                       Schnitzler: \emph{Sulla
                                    Psicoanalisia}. Milano:
                                       \emph{Studio Editoriale}{ }1987, S. 106.) }\toendnotes[C]{\smallbreak}
\pstart
           \noindent{}\centering{}{\pb}Arthur Schnitzler\pend
           
\pstart
           \centering{}in geziemender Schüchternheit\pend
           \pstart \spacefill\mbox{der Verfasser}\pend{}
\pstart
           \raggedleft{}\label{K_L03892-1v}\edtext{Nov. 1924}{\lemma{\textnormal{\emph{Nov.
                     1924}}}\Cendnote{\textnormal{Die Datierung lässt
                     sich mit Hilfe des \emph{Tagebuch}\pwindex{Schnitzler, Arthur 15.\,5.\,1862 Wien – 21.\,10.\,1931 ebd.@\textsc{Schnitzler, Arthur} (15.\,5.\,1862 Wien – 21.\,10.\,1931 ebd.), \emph{Schriftsteller, Mediziner}!Tagebuch@\strich\emph{Tagebuch}|pwk}-Eintrags zum
                        19. 11. 1924
                     zeitlich in der ersten Hälfte des Novembers verorten: »– Die besondre
                        Lebhaftigkeit des Traums könnte auch durch meine Absicht Freud\pwindex{Freud, Sigmund 6.\,5.\,1856 Pribor – 23.\,9.\,1939 London@\textsc{Freud, Sigmund} (6.\,5.\,1856 Pribor – 23.\,9.\,1939 London), \emph{Psychoanalytiker}|pw} zu besuchen bedingt sein (sowie ich
                        ungewöhnlich viel träumte, als ich 1900 seine Traumdeutung\pwindex{Freud, Sigmund 6.\,5.\,1856 Pribor – 23.\,9.\,1939 London@\textsc{Freud, Sigmund} (6.\,5.\,1856 Pribor – 23.\,9.\,1939 London), \emph{Psychoanalytiker}!Traumdeutung@\strich\emph{Die Traumdeutung}|pw} las). – Er sandte mir neulich ein paar
                        Aufsätze ›mit geziemender Schüchternheit‹. –« }}}\label{K_L03892-1}\pend
           \selectlanguage{ngerman}\vspace{1em}
\pstart
           \noindent{}\centering{}{\pb}\textcolor{gray}{\textbf{Psychoanalytische Studien\pwindex{Freud, Sigmund 6.\,5.\,1856 Pribor – 23.\,9.\,1939 London@\textsc{Freud, Sigmund} (6.\,5.\,1856 Pribor – 23.\,9.\,1939 London), \emph{Psychoanalytiker}!Psychoanalytische Studien an Werken der Dichtung und Kunst@\strich\emph{Psychoanalytische Studien an Werken der Dichtung und Kunst}|pw}}}{\\}\textcolor{gray}{\textbf{an Werken der\pwindex{Freud, Sigmund 6.\,5.\,1856 Pribor – 23.\,9.\,1939 London@\textsc{Freud, Sigmund} (6.\,5.\,1856 Pribor – 23.\,9.\,1939 London), \emph{Psychoanalytiker}!Psychoanalytische Studien an Werken der Dichtung und Kunst@\strich\emph{Psychoanalytische Studien an Werken der Dichtung und Kunst}|pw}}}{\\}\textcolor{gray}{\textbf{Dichtung und Kunst\pwindex{Freud, Sigmund 6.\,5.\,1856 Pribor – 23.\,9.\,1939 London@\textsc{Freud, Sigmund} (6.\,5.\,1856 Pribor – 23.\,9.\,1939 London), \emph{Psychoanalytiker}!Psychoanalytische Studien an Werken der Dichtung und Kunst@\strich\emph{Psychoanalytische Studien an Werken der Dichtung und Kunst}|pw}}}\pend
           
\pstart
           \centering{}\textcolor{gray}{\textbf{Von}}\pend
           
\pstart
           \centering{}\textcolor{gray}{\textbf{Sigm. Freud}}\pend
           {\vspace{1\baselineskip}}
\pstart
           \centering{}\textcolor{gray}{\textbf{1924}}\pend
           
\pstart
           \centering{}\textcolor{gray}{\textbf{\so{Internationaler Psychoanalytischer Verlag}\orgindex{Internationaler Psychoanalytischer Verlag@Internationaler Psychoanalytischer Verlag|pw}}}\pend
           
\pstart
           \centering{}\textcolor{gray}{\textbf{Leipzig\oindex{Leipzig@\textbf{Leipzig}, \emph{Hauptstadt}|pw} / Wien\oindex{Wien@\textbf{Wien}, \emph{Verwaltungsgebiet}|pw} / Zürich\oindex{Zürich@\textbf{Zürich}|pw}}}\pend
           \selectlanguage{ngerman}\endnumbering\briefempfaengerindex{Schnitzler, Arthur@\textsc{Schnitzler, Arthur}!zzzFreud, Sigmund@\emph{von Sigmund Freud}!1924-11-011@{[erste Hälfte] November 1924}|)be}\mylabel{L03897h}
\begin{anhang}
\end{anhang}\newcommand{\dateiname}{L03897}\newcommand{\titel}{Sigmund Freud: Widmungsexemplar von Psychoanalytische Studien an Werken der Dichtung und Kunst, [erste Hälfte] November 1924}\newcommand{\editorInnen}{Selma Jahnke und Martin Anton Müller}%% latex-leseansicht-abspann.tex
%% Abspann für die Leseansicht.
%% Der Schalter \ifkorrekturansicht ist bereits durch den Vorspann gesetzt.

%% latex-abspann.tex
%% Gemeinsamer Abspann für Korrekturansicht und Leseansicht.
%% Setzt den Schalter \ifkorrekturansicht voraus (gesetzt in den
%% einbindenden Dateien latex-korrekturansicht-abspann.tex bzw.
%% latex-leseansicht-abspann.tex).
%% ---------------------------------------------------------------

\normalsize

% Das esempio-Environment wird nur in der Leseansicht benötigt
\ifkorrekturansicht\else
\newenvironment{esempio}[3]%
{
    \vspace{1.5ex}
    \rlap{\underline{#1}}
    \par
    \setlength{\parindent}{0cm}
    \nopagebreak
    \leftskip=#2cm
    \rightskip=#3cm
}
{
    \par
}
\fi

\doendnotes{C}
\bigskip
\vfill

\clearpage

\footnotesize

\ifkorrekturansicht
  \lohead{\textsc{register}}
\fi

% theindex-Environment neu definieren ohne reledmac
\makeatletter
\renewenvironment{theindex}{%
  \ifkorrekturansicht
    \section*{\indexname}%
  \else
    \subsubsection*{Index der erwähnten Entitäten}%
  \fi
  \setlength{\parindent}{0pt}%
  \setlength{\parskip}{0pt plus 0.3pt}%
  \let\item\@idxitem
}{%
  \ifkorrekturansicht\clearpage\fi
}
\makeatother

\IfFileExists{\jobname-pw.ind}{\input{\jobname-pw.ind}}{}

% Quellenangabe nur in der Leseansicht
\ifkorrekturansicht\else
% Fallback-Definitionen, falls die .tex-Datei \titel etc. nicht gesetzt hat
\providecommand{\titel}{}
\providecommand{\editorInnen}{}
\providecommand{\dateiname}{\jobname}

\vspace{3cm}

\vfill

\footnotesize
\textsc{Quelle}: \titel. Herausgegeben von {\editorInnen}. In: \emph{Arthur Schnitzler: Briefwechsel mit Autorinnen und Autoren}.
 Digitale Edition, https://schnitzler-briefe.acdh.oeaw.ac.at/{\dateiname}.html (Stand \today)
\fi

\end{document}


