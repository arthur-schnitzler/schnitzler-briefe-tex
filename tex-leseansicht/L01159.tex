%% latex-leseansicht-vorspann.tex
%% Vorspann für die Leseansicht.
%% Lädt die gemeinsame Datei latex-vorspann.tex mit nicht gesetztem Schalter.

\newif\ifkorrekturansicht
\korrekturansichtfalse

\input{../tex-inputs/latex-vorspann}

\begin{center}
            \textcolor{red}{ENTWURF. ENTZIFFERUNG NOCH NICHT KORREKTURGELESEN}
                      \end{center}
            
               \section[Arthur Schnitzler an Hugo von Hofmannsthal, 10. 8. 1901]{ Arthur Schnitzler an Hugo von Hofmannsthal,
                    10. 8. 1901}\nopagebreak\mylabel{v}\rehead{ }\begin{ledgroupsized}[t]{13cm}\normalsize\beginnumbering\briefempfaengerindex{Hofmannsthal, Hugo von@\textsc{Hofmannsthal, Hugo von}!zzzSchnitzler, Arthur@\emph{von Arthur Schnitzler}!1901-08-102@{10. 8. 1901}|(be} \toendnotes[C]{\smallbreak\pagebreak[2]} \Standort{FDH, Hs-30885,96.}
\physDesc{Brief, 1 Blatt, 4 Seiten
\newline{}Handschrift: schwarze Tinte, deutsche Kurrent}\buchAbdrucke{\weitereDrucke{1) Hugo von Hofmannsthal, Arthur Schnitzler: \emph{Briefwechsel}. Hg. Therese Nickl und Heinrich Schnitzler. Frankfurt am Main: \emph{S. Fischer} 1964, S. 150–151.} \weitereDrucke{2) Hermann Bahr, Arthur Schnitzler: \emph{Briefwechsel, Aufzeichnungen, Dokumente
                                (1891–1931)}. Hg. Kurt Ifkovits und Martin Anton Müller. Göttingen: \emph{Wallstein} 2018, S. 215.} }\toendnotes[C]{\smallbreak}\pstart
           \raggedleft{}{\pb}\textsc{Vahrn}\oindex{Vahrn@\textbf{Vahrn}|pw}, 10. 8. 901\pend
           \pstart
           mein lieber Hugo, ſeit vier Wochen bin ich hier, und habe mich,
                    in angenehmer Gesellſchaft, mit Neigung zu Arbeit u\textcolor{gray}{.} einigem
                    Fleiſs und gelegentlichem Talent, in einer wunderbaren Luft, mit Sonne und Wald,
                    recht behaglich gefühlt. Montag reiſ\damage{en} wir nach Bozen\oindex{Bozen@\textbf{Bozen}|pw}, wo man Goldma{\geminationn}\pwindex{Goldmann, Paul 31.01.1865 – 25.09.1935@\textsc{Goldmann, Paul} (31.01.1865 – 25.09.1935), \emph{Schriftsteller, Journalist}|pw}
                    trifft, dann nach Trient\oindex{Trient@\textbf{Trient}|pw}, und endlich etwa
                        16. 8. gehts nach \uline{\textsc{Welsberg}} im Puſthertal\oindex{Welsberg-Taisten@\textbf{Welsberg-Taisten}|pw}, \uline{\textsc{Bad Waldbrunn}\oindex{Wildbad Waldbrunn@\textbf{Wildbad Waldbrunn}|pw}}, das ich neulich entdeckt habe u von dem ich mich nur wundre {\pb}daſs es kaum bekannt iſt. Ende
                        Auguſt möchte ich in Wien\oindex{Wien@\textbf{Wien}|pw}{ }ſein,
                    vor allem 2 neue
                        Einakter\pwindex{Schnitzler, Arthur 15.05.1862 – 21.10.1931@\textsc{Schnitzler, Arthur} (15.05.1862 – 21.10.1931), \emph{Schriftsteller, Mediziner}!Lebendige Stunden01. 12. 1901@\strich\emph{Lebendige Stunden} {[}01. 12. 1901{]}|pwv}\pwindex{Schnitzler, Arthur 15.05.1862 – 21.10.1931@\textsc{Schnitzler, Arthur} (15.05.1862 – 21.10.1931), \emph{Schriftsteller, Mediziner}!Frau mit dem Dolche1901@\strich\emph{Die Frau mit dem Dolche} {[}1901{]}|pwv} dictiren, die der »Literatur\pwindex{Schnitzler, Arthur 15.05.1862 – 21.10.1931@\textsc{Schnitzler, Arthur} (15.05.1862 – 21.10.1931), \emph{Schriftsteller, Mediziner}!Literatur1901@\strich\emph{Literatur} {[}1901{]}|pw}« vorangehen ſollen. Die drei Stückchen ſind nur durch einen
                    Grundgedanken verbunden, und eines mag immer das andre beleuchten. Auch das
                    dreiaktige Stück\pwindex{Schnitzler, Arthur 15.05.1862 – 21.10.1931@\textsc{Schnitzler, Arthur} (15.05.1862 – 21.10.1931), \emph{Schriftsteller, Mediziner}!einsame Weg. Schauspiel in fuenf Akten1904@\strich\emph{Der einsame Weg. Schauspiel in fünf Akten} {[}1904{]}|pwv} kann
                    bald beendet sein.\pend
           \pstart
           Ich freue mich auf einen ſchönen Septemberabend, wo wir einander allerlei
                    erzählen und vorleſen{\pb} können. Um den verlornen
                        Innsbruck\oindex{Innsbruck@\textbf{Innsbruck}|pw}er Abend thut es mir ſehr leid.
                    Anonymität wäre übrigens gar nicht vonnöthen geweſen, jeder Grund fehlt,
                    beſonders Ihnen und Ihrer Frau\pwindex{Hofmannsthal, Gertrude von 16.03.1880 – 09.11.1959@\textsc{Hofmannsthal, Gertrude von} (16.03.1880 – 09.11.1959)|pwv} gegenüber. Wir\pwindex{Schnitzler, Olga 17.01.1882 – 13.01.1970@\textsc{Schnitzler, Olga} (17.01.1882 – 13.01.1970), \emph{Schauspielerin, Sängerin}|pwv} waren damals an der Bahn, – der andre einzige Ort, wo man \strikeout{\textcolor{gray}{nie}} im Freien speiſen kann, nachdem mir der dritte einzige Ort, in der Nähe
                    der \textsc{Weierburg}\oindex{Schloss Weiherburg@\textbf{Schloss Weiherburg}|pw}, nicht zuſagte. –\pend
           \pstart
           Viel Freude habe ich heuer wieder vom Radfahren gehabt und mich mehr{\pb} als einmal an unsre Fahrt am Genfer See erinnert, die nun drei Jahre hinter uns
                    liegt.\pend
           \pstart
           Ich höre hoffentlich noch von Ihnen, ehe wir uns wiederſehn\pend
           \pstart
           Herzliche Grüße{\\[\baselineskip]}Ihr{\\[\baselineskip]}\spacefill\mbox{Arthur.}\pend
           \leftskip=0em{}\pstart
           \noindent{}Wenn Poldi\pwindex{Andrian-Werburg, Leopold von 09.05.1875 – 19.11.1951@\textsc{Andrian-Werburg, Leopold von} (09.05.1875 – 19.11.1951), \emph{Schriftsteller, Diplomat}|pw} bei Ihnen iſt, grüßen Sie
                        ihn vielmals. Michel\pwindex{Michel, Robert 24.02.1876 – 12.02.1957@\textsc{Michel, Robert} (24.02.1876 – 12.02.1957), \emph{Schriftsteller}|pw} hat mir einen ſo
                        netten Brief geſchrieben. Auch Bahr\pwindex{Bahr, Hermann 19.07.1863 – 15.01.1934@\textsc{Bahr, Hermann} (19.07.1863 – 15.01.1934), \emph{Schriftsteller, Kritiker}|pw},
                        den Sie ja öfters ſehn, grüßen Sie herzlich. Und empfehlen mich Ihrer Frau\pwindex{Hofmannsthal, Gertrude von 16.03.1880 – 09.11.1959@\textsc{Hofmannsthal, Gertrude von} (16.03.1880 – 09.11.1959)|pwv}.{\\}Ihr
                            \spacefill\mbox{A.}\pend
           \endnumbering\briefempfaengerindex{Hofmannsthal, Hugo von@\textsc{Hofmannsthal, Hugo von}!zzzSchnitzler, Arthur@\emph{von Arthur Schnitzler}!1901-08-102@{10. 8. 1901}|)be}\mylabel{h}\end{ledgroupsized}  \newcommand{\dateiname}{L01159}\newcommand{\titel}{Arthur Schnitzler an Hugo von Hofmannsthal, 10. 8. 1901}\newcommand{\editorInnen}{ Martin Anton Müller und Gerd-Hermann Susen}%% latex-leseansicht-abspann.tex
%% Abspann für die Leseansicht.
%% Der Schalter \ifkorrekturansicht ist bereits durch den Vorspann gesetzt.

%% latex-abspann.tex
%% Gemeinsamer Abspann für Korrekturansicht und Leseansicht.
%% Setzt den Schalter \ifkorrekturansicht voraus (gesetzt in den
%% einbindenden Dateien latex-korrekturansicht-abspann.tex bzw.
%% latex-leseansicht-abspann.tex).
%% ---------------------------------------------------------------

\normalsize

% Das esempio-Environment wird nur in der Leseansicht benötigt
\ifkorrekturansicht\else
\newenvironment{esempio}[3]%
{
    \vspace{1.5ex}
    \rlap{\underline{#1}}
    \par
    \setlength{\parindent}{0cm}
    \nopagebreak
    \leftskip=#2cm
    \rightskip=#3cm
}
{
    \par
}
\fi

\doendnotes{C}
\bigskip
\vfill

\clearpage

\footnotesize

\ifkorrekturansicht
  \lohead{\textsc{register}}
\fi

% theindex-Environment neu definieren ohne reledmac
\makeatletter
\renewenvironment{theindex}{%
  \ifkorrekturansicht
    \section*{\indexname}%
  \else
    \subsubsection*{Index der erwähnten Entitäten}%
  \fi
  \setlength{\parindent}{0pt}%
  \setlength{\parskip}{0pt plus 0.3pt}%
  \let\item\@idxitem
}{%
  \ifkorrekturansicht\clearpage\fi
}
\makeatother

\IfFileExists{\jobname-pw.ind}{\input{\jobname-pw.ind}}{}

% Quellenangabe nur in der Leseansicht
\ifkorrekturansicht\else
% Fallback-Definitionen, falls die .tex-Datei \titel etc. nicht gesetzt hat
\providecommand{\titel}{}
\providecommand{\editorInnen}{}
\providecommand{\dateiname}{\jobname}

\vspace{3cm}

\vfill

\footnotesize
\textsc{Quelle}: \titel. Herausgegeben von {\editorInnen}. In: \emph{Arthur Schnitzler: Briefwechsel mit Autorinnen und Autoren}.
 Digitale Edition, https://schnitzler-briefe.acdh.oeaw.ac.at/{\dateiname}.html (Stand \today)
\fi

\end{document}


      