%% latex-korrekturansicht-vorspann.tex
%% Vorspann für die Korrekturansicht.
%% Lädt die gemeinsame Datei latex-vorspann.tex mit gesetztem Schalter.

\newif\ifkorrekturansicht
\korrekturansichttrue

\input{../tex-inputs/latex-vorspann}


\section[Arthur Schnitzler an Hugo von Hofmannsthal, 10. 8. 1901]{L01159 Arthur Schnitzler an Hugo von Hofmannsthal, 10. 8. 1901}
\nopagebreak\mylabel{L01159v}
\rehead{ }\normalsize\beginnumbering\briefempfaengerindex{Hofmannsthal, Hugo von@\textsc{Hofmannsthal, Hugo von}!zzzSchnitzler, Arthur@\emph{von Arthur Schnitzler}!1901-08-102@{10. 8. 1901}|(be}
\toendnotes[C]{\smallbreak\pagebreak[2]}\Standort{FDH, Hs-30885,96.}
\physDesc{Brief, 1 Blatt, 4 Seiten, 1585 Zeichen
\newline{}Handschrift: schwarze Tinte, deutsche Kurrent}
\buchAbdrucke{\weitereDrucke{1) Hugo von Hofmannsthal, Arthur Schnitzler: \emph{Briefwechsel}. Frankfurt am Main: \emph{S. Fischer} 1964, S. 150–151.} \weitereDrucke{2) Hermann Bahr, Arthur Schnitzler: \emph{Briefwechsel, Aufzeichnungen, Dokumente (1891–1931)}. Göttingen: \emph{Wallstein} 2018, S. 215.} }\toendnotes[C]{\smallbreak}
\pstart
           \raggedleft{}{\pb}\textsc{Vahrn}\oindex{Vahrn@\textbf{Vahrn}, \emph{P.PPLA3}|pw}, 10. 8. 901\pend
           \vspace{0.5em}
\pstart
           mein lieber Hugo, ſeit vier Wochen bin ich hier, und habe mich, in
               angenehmer Gesellſchaft, mit Neigung zu Arbeit u\textcolor{gray}{.} einigem Fleiſs
               und gelegentlichem Talent, in einer wunderbaren Luft, mit Sonne und Wald, recht
               behaglich gefühlt. Montag reiſ\damage{en} wir nach Bozen\oindex{Bozen@\textbf{Bozen}, \emph{P.PPLA2}|pw}, wo man Goldma{\geminationn}\pwindex{Goldmann, Paul 31.01.1865 – 25.09.1935@\textsc{Goldmann, Paul} (31.01.1865 – 25.09.1935), \emph{Schriftsteller/Schriftstellerin, Journalist/Journalistin}|pw} trifft, dann nach Trient\oindex{Trient@\textbf{Trient}, \emph{P.PPLA}|pw}, und endlich etwa
                  16. 8. gehts nach \uline{\textsc{Welsberg}} im Puſthertal\oindex{Welsberg-Taisten@\textbf{Welsberg-Taisten}, \emph{A.ADM3}|pw}, \uline{\textsc{Bad Waldbrunn}\oindex{Wildbad Waldbrunn@\textbf{Wildbad Waldbrunn}, \emph{S.SPA}|pw}}, das ich neulich entdeckt habe u von dem ich mich nur wundre {\pb}daſs es kaum bekannt iſt. Ende Auguſt möchte
               ich in Wien\oindex{Wien@\textbf{Wien}, \emph{A.ADM2}|pw}{ }ſein, vor allem 2 neue Einakter\pwindex{Lebendige Stunden@\emph{Lebendige Stunden}|pwv}\pwindex{Frau mit dem Dolche@\emph{Die Frau mit dem Dolche}|pwv} dictiren, die der
                  »Literatur\pwindex{Literatur@\emph{Literatur}|pw}« vorangehen ſollen. Die drei
               Stückchen ſind nur durch einen Grundgedanken verbunden, und eines mag immer das andre
               beleuchten. Auch das dreiaktige Stück\pwindex{einsame Weg. Schauspiel in fuenf Akten@\emph{Der einsame Weg. Schauspiel in fünf Akten}|pwv} kann bald beendet sein.\pend
           
\pstart
           Ich freue mich auf einen ſchönen Septemberabend, wo wir einander allerlei erzählen
               und vorleſen {\pb}können. Um den verlornen Innsbruck\oindex{Innsbruck@\textbf{Innsbruck}, \emph{A.ADM2}|pw}er Abend thut es mir ſehr leid. Anonymität wäre
               übrigens gar nicht vonnöthen geweſen, jeder Grund fehlt, beſonders Ihnen und Ihrer
                  Frau\pwindex{Hofmannsthal, Gertrude von 16.03.1880 – 09.11.1959@\textsc{Hofmannsthal, Gertrude von} (16.03.1880 – 09.11.1959)|pwv} gegenüber. Wir\pwindex{Schnitzler, Olga 17.01.1882 – 13.01.1970@\textsc{Schnitzler, Olga} (17.01.1882 – 13.01.1970), \emph{Schauspieler/Schauspielerin, Sänger/Sängerin}|pwv} waren damals an der Bahn,
               – der andre einzige Ort, wo man \strikeout{\textcolor{gray}{nie}} im Freien speiſen kann, nachdem mir der dritte einzige Ort, in der Nähe der
                  \textsc{Weierburg}\oindex{Schloss Weiherburg@\textbf{Schloss Weiherburg}, \emph{Schloss (K.SLS)}|pw}, nicht zuſagte. –\pend
           
\pstart
           Viel Freude habe ich heuer wieder vom Radfahren gehabt und mich mehr {\pb}als einmal an unsre Fahrt am Genfer
                  See\oindex{Genfer See@\textbf{Genfer See}, \emph{H.LK}|pw} erinnert, die nun drei Jahre hinter uns liegt.\pend
           
\pstart
           Ich höre hoffentlich noch von Ihnen, ehe wir uns wiederſehn\pend
           
\pstart
           Herzliche Grüße{\\[\baselineskip]}Ihr{\\[\baselineskip]}\spacefill\mbox{Arthur.}\pend
           \leftskip=0em{}
\pstart
           \noindent{}Wenn Poldi\pwindex{Andrian-Werburg, Leopold von 09.05.1875 – 19.11.1951@\textsc{Andrian-Werburg, Leopold von} (09.05.1875 – 19.11.1951), \emph{Schriftsteller/Schriftstellerin, Diplomat/Diplomatin}|pw} bei Ihnen iſt, grüßen Sie ihn
                  vielmals. Michel\pwindex{Michel, Robert 24.02.1876 – 12.02.1957@\textsc{Michel, Robert} (24.02.1876 – 12.02.1957), \emph{Schriftsteller/Schriftstellerin, Offizier/Offizierin, Krimiautor/Krimiautorin}|pw} hat mir einen ſo netten
                  Brief geſchrieben. Auch Bahr\pwindex{Bahr, Hermann 19.07.1863 – 15.01.1934@\textsc{Bahr, Hermann} (19.07.1863 – 15.01.1934), \emph{Schriftsteller/Schriftstellerin, Kritiker/Kritikerin}|pw}, den Sie ja
                  öfters ſehn, grüßen Sie herzlich. Und empfehlen mich Ihrer Frau\pwindex{Hofmannsthal, Gertrude von 16.03.1880 – 09.11.1959@\textsc{Hofmannsthal, Gertrude von} (16.03.1880 – 09.11.1959)|pwv}.{\\}Ihr \spacefill\mbox{A.}\pend
           \selectlanguage{ngerman}\endnumbering\briefempfaengerindex{Hofmannsthal, Hugo von@\textsc{Hofmannsthal, Hugo von}!zzzSchnitzler, Arthur@\emph{von Arthur Schnitzler}!1901-08-102@{10. 8. 1901}|)be}\mylabel{L01159h}  \normalsize

\doendnotes{C}
\bigskip
\vfill

\clearpage

\footnotesize

\lohead{\textsc{register}}

% Definiere theindex-Environment komplett neu ohne reledmac
\makeatletter
\renewenvironment{theindex}{%
  \section*{\indexname}%
  \setlength{\parindent}{0pt}%
  \setlength{\parskip}{0pt plus 0.3pt}%
  \let\item\@idxitem
}{%
  \clearpage
}
\makeatother

\IfFileExists{\jobname-pw.ind}{\input{\jobname-pw.ind}}{}

\end{document}

      