\input{../tex-inputs/latex-pdf-vorspann}
\begin{center}
            \textcolor{red}{ENTWURF. ENTZIFFERUNG NOCH NICHT KORREKTURGELESEN}
                      \end{center}
            
               \section[Ferdinand von Saar an Arthur Schnitzler, 11. 10. 1895]{ Ferdinand von Saar an Arthur Schnitzler,
                    11. 10. 1895}\nopagebreak\mylabel{v}\rehead{ }\begin{ledgroupsized}[t]{13cm}\normalsize\beginnumbering\briefempfaengerindex{Schnitzler, Arthur@\textsc{Schnitzler, Arthur}!zzzSaar, Ferdinand von@\emph{von Ferdinand von Saar}!1895-10-111@{11. 10. 1895}|(be} \toendnotes[C]{\smallbreak\pagebreak[2]} \Standort{CUL, Schnitzler, B 88.}
\physDesc{Visitenkarte
\newline{}Handschrift: schwarze Tinte, deutsche Kurrent
\newline{}Schnitzler: mit Bleistift nummeriert: »4« }\pstart
           \noindent{}\centering{}{\pb}\textcolor{gray}{\textbf{\textsc{Ferdinand von Saar}}}\pend
           \pstart
           gratuliert herzlich zum Erfolg\pwindex{Schnitzler, Arthur 15.05.1862 – 21.10.1931@\textsc{Schnitzler, Arthur} (15.05.1862 – 21.10.1931), \emph{Schriftsteller, Mediziner}!Liebelei. Schauspiel in drei Akten9. 10. 1895@\strich\emph{Liebelei. Schauspiel in drei Akten} {[}9. 10. 1895{]}|pw}!\pend
           \pstart
           \textsc{Wien-Döbling}\oindex{XIX., Doebling@\textbf{XIX., Döbling}|pw}, 11\textsuperscript{ter} Octbr 1895.\pend
           \endnumbering\briefempfaengerindex{Schnitzler, Arthur@\textsc{Schnitzler, Arthur}!zzzSaar, Ferdinand von@\emph{von Ferdinand von Saar}!1895-10-111@{11. 10. 1895}|)be}\mylabel{h}\end{ledgroupsized}  \newcommand{\dateiname}{L00504}\newcommand{\titel}{Ferdinand von Saar an Arthur Schnitzler, 11. 10. 1895}\newcommand{\editorInnen}{Martin Anton Müller und Gerd-Hermann Susen}\input{../tex-inputs/latex-pdf-abspann}
      