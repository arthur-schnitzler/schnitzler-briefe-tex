%% latex-leseansicht-vorspann.tex
%% Vorspann für die Leseansicht.
%% Lädt die gemeinsame Datei latex-vorspann.tex mit nicht gesetztem Schalter.

\newif\ifkorrekturansicht
\korrekturansichtfalse

\input{../tex-inputs/latex-vorspann}


\section[Arthur Schnitzler an Richard Beer-Hofmann, 29. 7. 1896]{L00571 Arthur Schnitzler an Richard Beer-Hofmann, 29. 7. 1896}
\nopagebreak\mylabel{L00571v}
\rehead{ }\normalsize\beginnumbering\briefempfaengerindex{Beer-Hofmann, Richard@\textsc{Beer-Hofmann, Richard}!zzzSchnitzler, Arthur@\emph{von Arthur Schnitzler}!1896-07-291@{29. 7. 1896}|(be}
\toendnotes[C]{\smallbreak\pagebreak[2]}
\correspDesc{Versand  durch Arthur Schnitzler am 29. 7. 1896 in Stockholm
\newline{}Erhalt  durch Richard Beer-Hofmann am 30. 7. 1896 in Kopenhagen}\toendnotes[C]{\smallbreak}
\Standort{YCGL, MSS 31.}
\physDesc{Brief, 1 Blatt, 2 Seiten, Kuvert, 1039 Zeichen
\newline{}Handschrift: Bleistift, deutsche Kurrent
\newline{}Versand: 1) Stempel: »\nobreak{}\oindex{Stockholm@\textbf{Stockholm}, \emph{Hauptstadt}|pwk}Stockholm, 29 7 96\nobreak{}«.   2) Stempel: »\nobreak{}\oindex{Kopenhagen@\textbf{Kopenhagen}, \emph{Hauptstadt}|pwk}Kjøbenhavn, 30. 7. 96, 20 MB\nobreak{}«. }
\buchAbdrucke{\weitereDrucke{Arthur Schnitzler, Richard Beer-Hofmann: \emph{Briefwechsel 1891–1931}. Herausgegeben von Konstanze Fliedl. Wien, Zürich: \emph{Europaverlag} 1992, S. 94.} }\pstart{}{\pb}Herrn \textsc{Dr. Richard
                     Beer-Hofmann}\pend{}\pstart{}\textsc{Kopenhagen}\oindex{Kopenhagen@\textbf{Kopenhagen}, \emph{Hauptstadt}|pw}\pend{}\pstart{}\textsc{Hotel König von Dänemark}\oindex{Hotel Kongen af Danmark@\textbf{Hotel Kongen af Danmark}, \emph{Hotel}|pw}\pend{}{\bigskip}\vspace{1em}
\pstart
           \raggedleft{}{\pb}Stockholm\oindex{Stockholm@\textbf{Stockholm}, \emph{Hauptstadt}|pw}{ }29/7 96. 6 Uhr Nm\pend
           \vspace{0.5em}
\pstart
           Lieber Richard, finde eben Ihren Brief. Ich bleibe hier bis
                  Freitag{ }Abend, 31., fahre am Abend nach Gothenburg\oindex{Göteborg@\textbf{Göteborg}|pw}, bin dort Samſtag{ }\strikeout{(\introOben{}am\introOben{} nächſt} fahre So{\geminationn}tag früh nach \textsc{Kopenhagen}\oindex{Kopenhagen@\textbf{Kopenhagen}, \emph{Hauptstadt}|pw}, bin Abends in \textsc{Kopenhagen}\oindex{Kopenhagen@\textbf{Kopenhagen}, \emph{Hauptstadt}|pw}. Gibts was neues,{ }ſo kann ich Nachricht von Ihnen, wohl Telegramm{ }ſpäteſtens
                  Freitag{ }\introOben{}Nach-\introOben{}Mittag hieher ins \textsc{Grand Hotel}\oindex{Grand Hotel Stockholm@\textbf{Grand Hotel Stockholm}, \emph{Hotel}|pw} empfangen. Erfahre ich nichts weitres,{ }ſo nehme ich an, dſs Sie mich in Ihrem
               Hotel in K.\oindex{Kopenhagen@\textbf{Kopenhagen}, \emph{Hauptstadt}|pw}{ }So{\geminationn}tag Abend wiſſen laſſen, wo Sie zu
               finden (Wahrſcheinlich{ }ſteig ich {\pb}auch dort
               ab.) Vielleicht geht doch \textsc{Skotsborg}\oindex{Skodsborg@\textbf{Skodsborg}|pw}, wäre mir{ }ſympathiſcher – im übrigen wie Sie wollen. Muſs jedenfalls noch
               8 Tage{ }ſehr fleißig arbeiten. Dem Paul\pwindex{Goldmann, Paul 31.\,1.\,1865 Breslau – 25.\,9.\,1935 Wien@\textsc{Goldmann, Paul} (31.\,1.\,1865 Breslau – 25.\,9.\,1935 Wien), \emph{Schriftsteller, Journalist}|pw} hab ich
               auch nur{ }ſchreiben können, \textsc{Kopenhagen}\oindex{Kopenhagen@\textbf{Kopenhagen}, \emph{Hauptstadt}|pw} u dann wahrſcheinlich \textsc{Skottsborg}\oindex{Skodsborg@\textbf{Skodsborg}|pw} – wir werden einander wohl nicht verfehlen. Vergeſſen Sie Vornamen auf Telegr.
               nicht – es läuft hier noch ein Schnitzler\pwindex{Schnitzler @\textsc{Schnitzler}|pw} mit
               einer Frau A. Schnitzler\pwindex{Schnitzler, A. @\textsc{Schnitzler, A.}|pw} herum, der
               wahrſcheinlich die meiſten meiner Briefe bekommt. Freue mich{ }ſehr auf Wiederſehen\pend
           \pstart Herzlich Ihr \spacefill\mbox{Arthur}\pend{}\selectlanguage{ngerman}\endnumbering\briefempfaengerindex{Beer-Hofmann, Richard@\textsc{Beer-Hofmann, Richard}!zzzSchnitzler, Arthur@\emph{von Arthur Schnitzler}!1896-07-291@{29. 7. 1896}|)be}\mylabel{L00571h}  \newcommand{\dateiname}{L00571}\newcommand{\titel}{Arthur Schnitzler an Richard Beer-Hofmann, 29. 7. 1896}\newcommand{\editorInnen}{Martin Anton Müller und Gerd-Hermann Susen}%% latex-leseansicht-abspann.tex
%% Abspann für die Leseansicht.
%% Der Schalter \ifkorrekturansicht ist bereits durch den Vorspann gesetzt.

%% latex-abspann.tex
%% Gemeinsamer Abspann für Korrekturansicht und Leseansicht.
%% Setzt den Schalter \ifkorrekturansicht voraus (gesetzt in den
%% einbindenden Dateien latex-korrekturansicht-abspann.tex bzw.
%% latex-leseansicht-abspann.tex).
%% ---------------------------------------------------------------

\normalsize

% Das esempio-Environment wird nur in der Leseansicht benötigt
\ifkorrekturansicht\else
\newenvironment{esempio}[3]%
{
    \vspace{1.5ex}
    \rlap{\underline{#1}}
    \par
    \setlength{\parindent}{0cm}
    \nopagebreak
    \leftskip=#2cm
    \rightskip=#3cm
}
{
    \par
}
\fi

\doendnotes{C}
\bigskip
\vfill

\clearpage

\footnotesize

\ifkorrekturansicht
  \lohead{\textsc{register}}
\fi

% theindex-Environment neu definieren ohne reledmac
\makeatletter
\renewenvironment{theindex}{%
  \ifkorrekturansicht
    \section*{\indexname}%
  \else
    \subsubsection*{Index der erwähnten Entitäten}%
  \fi
  \setlength{\parindent}{0pt}%
  \setlength{\parskip}{0pt plus 0.3pt}%
  \let\item\@idxitem
}{%
  \ifkorrekturansicht\clearpage\fi
}
\makeatother

\IfFileExists{\jobname-pw.ind}{\input{\jobname-pw.ind}}{}

% Quellenangabe nur in der Leseansicht
\ifkorrekturansicht\else
% Fallback-Definitionen, falls die .tex-Datei \titel etc. nicht gesetzt hat
\providecommand{\titel}{}
\providecommand{\editorInnen}{}
\providecommand{\dateiname}{\jobname}

\vspace{3cm}

\vfill

\footnotesize
\textsc{Quelle}: \titel. Herausgegeben von {\editorInnen}. In: \emph{Arthur Schnitzler: Briefwechsel mit Autorinnen und Autoren}.
 Digitale Edition, https://schnitzler-briefe.acdh.oeaw.ac.at/{\dateiname}.html (Stand \today)
\fi

\end{document}


