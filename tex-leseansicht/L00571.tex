%% latex-korrekturansicht-vorspann.tex
%% Vorspann für die Korrekturansicht.
%% Lädt die gemeinsame Datei latex-vorspann.tex mit gesetztem Schalter.

\newif\ifkorrekturansicht
\korrekturansichttrue

\input{../tex-inputs/latex-vorspann}


\section[Arthur Schnitzler an Richard Beer-Hofmann, 29. 7. 1896]{L00571 Arthur Schnitzler an Richard Beer-Hofmann, 29. 7. 1896}
\nopagebreak\mylabel{L00571v}
\rehead{ }\normalsize\beginnumbering\briefempfaengerindex{Beer-Hofmann, Richard@\textsc{Beer-Hofmann, Richard}!zzzSchnitzler, Arthur@\emph{von Arthur Schnitzler}!1896-07-291@{29. 7. 1896}|(be}
\toendnotes[C]{\smallbreak\pagebreak[2]}\Standort{YCGL, MSS 31.}
\physDesc{Brief, 1 Blatt, 2 Seiten, Umschlag, 1039 Zeichen
\newline{}Handschrift: Bleistift, deutsche Kurrent
\newline{}Versand: 1) Stempel: »\nobreak{}\oindex{Stockholm@\textbf{Stockholm}, \emph{P.PPLC}|pwk}Stockholm, 29 7 96\nobreak{}«.   2) Stempel: »\nobreak{}\oindex{Kopenhagen@\textbf{Kopenhagen}, \emph{P.PPLC}|pwk}Kjøbenhavn, 30. 7. 96, 20 MB\nobreak{}«. }
\buchAbdrucke{\weitereDrucke{Arthur Schnitzler, Richard Beer-Hofmann: \emph{Briefwechsel 1891–1931}. Wien, Zürich: \emph{Europaverlag} 1992, S. 94.} }\pstart{}{\pb}Herrn \textsc{Dr. Richard
                     Beer-Hofmann}\pend{}\pstart{}\textsc{Kopenhagen}\oindex{Kopenhagen@\textbf{Kopenhagen}, \emph{P.PPLC}|pw}\pend{}\pstart{}\textsc{Hotel König von Dänemark}\oindex{Hotel Kongen af Danmark@\textbf{Hotel Kongen af Danmark}, \emph{Hotel (K.HTL)}|pw}\pend{}{\bigskip}\vspace{1em}
\pstart
           \raggedleft{}{\pb}Stockholm\oindex{Stockholm@\textbf{Stockholm}, \emph{P.PPLC}|pw}{ }29/7 96. 6 Uhr Nm\pend
           \vspace{0.5em}
\pstart
           Lieber Richard, finde eben Ihren Brief. Ich bleibe hier bis
                  Freitag{ }Abend, 31., fahre am Abend nach Gothenburg\oindex{Goeteborg@\textbf{Göteborg}, \emph{P.PPLA}|pw}, bin dort Samſtag{ }\strikeout{(\introOben{}am\introOben{} nächſt} fahre So{\geminationn}tag früh nach \textsc{Kopenhagen}\oindex{Kopenhagen@\textbf{Kopenhagen}, \emph{P.PPLC}|pw}, bin Abends in \textsc{Kopenhagen}\oindex{Kopenhagen@\textbf{Kopenhagen}, \emph{P.PPLC}|pw}. Gibts was neues, ſo kann ich Nachricht von Ihnen, wohl Telegramm ſpäteſtens
                  Freitag{ }\introOben{}Nach-\introOben{}Mittag hieher ins \textsc{Grand Hotel}\oindex{Grand Hotel Stockholm@\textbf{Grand Hotel Stockholm}, \emph{Hotel (K.HTL)}|pw} empfangen. Erfahre ich nichts weitres, ſo nehme ich an, dſs Sie mich in Ihrem
               Hotel in K.\oindex{Kopenhagen@\textbf{Kopenhagen}, \emph{P.PPLC}|pw}{ }So{\geminationn}tag Abend wiſſen laſſen, wo Sie zu
               finden (Wahrſcheinlich ſteig ich {\pb}auch dort
               ab.) Vielleicht geht doch \textsc{Skotsborg}\oindex{Skodsborg@\textbf{Skodsborg}, \emph{P.PPL}|pw}, wäre mir ſympathiſcher – im übrigen wie Sie wollen. Muſs jedenfalls noch
               8 Tage ſehr fleißig arbeiten. Dem Paul\pwindex{Goldmann, Paul 31.01.1865 – 25.09.1935@\textsc{Goldmann, Paul} (31.01.1865 – 25.09.1935), \emph{Schriftsteller/Schriftstellerin, Journalist/Journalistin}|pw} hab ich
               auch nur ſchreiben können, \textsc{Kopenhagen}\oindex{Kopenhagen@\textbf{Kopenhagen}, \emph{P.PPLC}|pw} u dann wahrſcheinlich \textsc{Skottsborg}\oindex{Skodsborg@\textbf{Skodsborg}, \emph{P.PPL}|pw} – wir werden einander wohl nicht verfehlen. Vergeſſen Sie Vornamen auf Telegr.
               nicht – es läuft hier noch ein Schnitzler\pwindex{Schnitzler @\textsc{Schnitzler}|pw} mit
               einer Frau A. Schnitzler\pwindex{Schnitzler, A. @\textsc{Schnitzler, A.}|pw} herum, der
               wahrſcheinlich die meiſten meiner Briefe bekommt. Freue mich ſehr auf Wiederſehen\pend
           \pstart Herzlich Ihr \spacefill\mbox{Arthur}\pend{}\selectlanguage{ngerman}\endnumbering\briefempfaengerindex{Beer-Hofmann, Richard@\textsc{Beer-Hofmann, Richard}!zzzSchnitzler, Arthur@\emph{von Arthur Schnitzler}!1896-07-291@{29. 7. 1896}|)be}\mylabel{L00571h}  \normalsize

\doendnotes{C}
\bigskip
\vfill

\clearpage

\footnotesize

\lohead{\textsc{register}}

% Definiere theindex-Environment komplett neu ohne reledmac
\makeatletter
\renewenvironment{theindex}{%
  \section*{\indexname}%
  \setlength{\parindent}{0pt}%
  \setlength{\parskip}{0pt plus 0.3pt}%
  \let\item\@idxitem
}{%
  \clearpage
}
\makeatother

\IfFileExists{\jobname-pw.ind}{\input{\jobname-pw.ind}}{}

\end{document}

      