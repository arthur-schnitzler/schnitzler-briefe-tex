\input{../tex-inputs/latex-pdf-vorspann}
\begin{center}
            \textcolor{red}{ENTWURF. ENTZIFFERUNG NOCH NICHT KORREKTURGELESEN}
                      \end{center}
            
               \section[Arthur Schnitzler an Georg Brandes, 4. 1. 1899]{ Arthur Schnitzler an Georg Brandes, 4. 1. 1899}\nopagebreak\mylabel{v}\rehead{ }\begin{ledgroupsized}[t]{13cm}\normalsize\beginnumbering\briefempfaengerindex{Brandes, Georg@\textsc{Brandes, Georg}!zzzSchnitzler, Arthur@\emph{von Arthur Schnitzler}!1899-01-041@{4. 1. 1899}|(be} \toendnotes[C]{\smallbreak\pagebreak[2]} \Standort{Kopenhagen, Det Kongelige Bibliotek, Georg Brandes Arkiv, box 125.}
\physDesc{Briefkarte
\newline{}Handschrift: schwarze Tinte, deutsche Kurrent\newline{}Ordnung: von unbekannter Hand nummeriert:
                                        »12.« }\buchAbdrucke{\weitereDrucke{Georg Brandes, Arthur Schnitzler: \emph{Ein Briefwechsel}. Hg. Kurt Bergel. Bern: \emph{Francke} 1956, S. 69.} }\toendnotes[C]{\smallbreak}\pstart
           \noindent{}{\pb}Verehrteſter Herr Brandes, aus der \label{K_L00874_1v}\edtext{Zeitung}{\lemma{\textnormal{\emph{Zeitung}}}\Cendnote{\textnormal{In der
                            \emph{Neuen Freien Presse}\orgindex{Neue Freie Presse@Neue Freie Presse|pwk} findet sich die
                        Meldung am 3. 1. 1899 ([O. V.:] \emph{Ein Nachruf für Frau Brandes\pwindex{Brandes, Emilie 22.03.1818 – 27.12.1898@\textsc{Brandes, Emilie} (22.03.1818 – 27.12.1898)|pwk}}\pwindex{?? Werk@Nicht ermittelte Verfasserinnen und Verfasser!Nachruf fuer Frau Brandes03. 01. 1899@\emph{Ein Nachruf für Frau Brandes} {[}03. 01. 1899{]}|pwk}, Nr. 12344, S. 5–6).}}}\label{K_L00874_1h} erfahre ich, dſs Ihre Mutter\pwindex{Brandes, Emilie 22.03.1818 – 27.12.1898@\textsc{Brandes, Emilie} (22.03.1818 – 27.12.1898)|pwv} geſtorben iſt. In herzlicher
                    Theilnahme drücke ich Ihnen die Hand.\pend
           \pstart
           Ihr Ihnen wahrhaft ergebener{\\[\baselineskip]}\spacefill\mbox{Arthur Schnitzler}\pend
           \leftskip=0em{}\pstart
           Wien\oindex{Wien@\textbf{Wien}|pw}{\\}4. 1. 99.\pend
           \endnumbering\briefempfaengerindex{Brandes, Georg@\textsc{Brandes, Georg}!zzzSchnitzler, Arthur@\emph{von Arthur Schnitzler}!1899-01-041@{4. 1. 1899}|)be}\mylabel{h}\end{ledgroupsized}  \newcommand{\dateiname}{L00874}\newcommand{\titel}{Arthur Schnitzler an Georg Brandes, 4. 1. 1899}\newcommand{\editorInnen}{Martin Anton Müller und Gerd-Hermann Susen}\input{../tex-inputs/latex-pdf-abspann}
      