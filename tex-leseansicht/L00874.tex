%% latex-korrekturansicht-vorspann.tex
%% Vorspann für die Korrekturansicht.
%% Lädt die gemeinsame Datei latex-vorspann.tex mit gesetztem Schalter.

\newif\ifkorrekturansicht
\korrekturansichttrue

\input{../tex-inputs/latex-vorspann}


\section[Arthur Schnitzler an Georg Brandes, 4. 1. 1899]{L00874 Arthur Schnitzler an Georg Brandes, 4. 1. 1899}
\nopagebreak\mylabel{L00874v}
\rehead{ }\normalsize\beginnumbering\briefempfaengerindex{Brandes, Georg@\textsc{Brandes, Georg}!zzzSchnitzler, Arthur@\emph{von Arthur Schnitzler}!1899-01-041@{4. 1. 1899}|(be}
\toendnotes[C]{\smallbreak\pagebreak[2]}\Standort{Kopenhagen, Det Kongelige Bibliotek, Georg Brandes Arkiv, box 125.}
\physDesc{Briefkarte, 191 Zeichen
\newline{}Handschrift: schwarze Tinte, deutsche Kurrent
\newline{}Ordnung: von unbekannter Hand nummeriert: »12.« }
\buchAbdrucke{\weitereDrucke{Georg Brandes, Arthur Schnitzler: \emph{Ein Briefwechsel}. Bern: \emph{Francke} 1956, S. 69.} }\toendnotes[C]{\smallbreak}
\pstart
           \noindent{}{\pb}Verehrteſter Herr Brandes, aus der \label{K_L00874-1v}\edtext{Zeitung}{\lemma{\textnormal{\emph{Zeitung}}}\Cendnote{\textnormal{In der \emph{Neuen Freien Presse}\orgindex{Neue Freie Presse@Neue Freie Presse|pwk} findet sich die Meldung am
                     3. 1. 1899 ([O. V.]: \emph{Ein
                        Nachruf für Frau Brandes}\pwindex{Nachruf fuer Frau Brandes@\emph{Ein Nachruf für Frau Brandes}|pwk}, Nr. 12.344, S. 5–6).}}}\label{K_L00874-1} erfahre ich, dſs Ihre Mutter\pwindex{Brandes, Emilie 22.03.1818 – 27.12.1898@\textsc{Brandes, Emilie} (22.03.1818 – 27.12.1898)|pwv} geſtorben iſt. In
               herzlicher Theilnahme drücke ich Ihnen die Hand.\pend
           
\pstart
           Ihr Ihnen wahrhaft ergebener{\\[\baselineskip]}\spacefill\mbox{Arthur Schnitzler}\pend
           \leftskip=0em{}
\pstart
           Wien\oindex{Wien@\textbf{Wien}, \emph{A.ADM2}|pw}{\\}4. 1. 99.\pend
           \selectlanguage{ngerman}\endnumbering\briefempfaengerindex{Brandes, Georg@\textsc{Brandes, Georg}!zzzSchnitzler, Arthur@\emph{von Arthur Schnitzler}!1899-01-041@{4. 1. 1899}|)be}\mylabel{L00874h}  \normalsize

\doendnotes{C}
\bigskip
\vfill

\clearpage

\footnotesize

\lohead{\textsc{register}}

% Definiere theindex-Environment komplett neu ohne reledmac
\makeatletter
\renewenvironment{theindex}{%
  \section*{\indexname}%
  \setlength{\parindent}{0pt}%
  \setlength{\parskip}{0pt plus 0.3pt}%
  \let\item\@idxitem
}{%
  \clearpage
}
\makeatother

\IfFileExists{\jobname-pw.ind}{\input{\jobname-pw.ind}}{}

\end{document}

      