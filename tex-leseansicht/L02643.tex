%% latex-korrekturansicht-vorspann.tex
%% Vorspann für die Korrekturansicht.
%% Lädt die gemeinsame Datei latex-vorspann.tex mit gesetztem Schalter.

\newif\ifkorrekturansicht
\korrekturansichttrue

\input{../tex-inputs/latex-vorspann}


\section[Paul Goldmann an Arthur Schnitzler, 6. 8. 1889]{L02643 Paul Goldmann an Arthur Schnitzler, 6. 8. 1889}
\nopagebreak\mylabel{L02643v}
\rehead{ }\normalsize\beginnumbering\briefempfaengerindex{Schnitzler, Arthur@\textsc{Schnitzler, Arthur}!zzzGoldmann, Paul@\emph{von Paul Goldmann}!1889-08-061@{6. 8. 1889}|(be}
\toendnotes[C]{\smallbreak\pagebreak[2]}\Standort{DLA, A:Schnitzler, HS.NZ85.1.3162.}
\physDesc{Brief, 1 Blatt, 2 Seiten, 526 Zeichen
\newline{}Handschrift: schwarze Tinte, deutsche Kurrent}\toendnotes[C]{\smallbreak}
\pstart
           \centering{}{\pb}\textcolor{gray}{\textbf{\textbf{Adminiſtration: VII.
                           Seidengaſſe 7\oindex{Seidengasse@\textbf{Seidengasse}, \emph{Straße (K.STR)}|pw}} (Jos. Eberle {\kaufmannsund} Co.\orgindex{Josef Eberle Stein-, Buch und Musikaliendruckerei@Josef Eberle Stein-, Buch und Musikaliendruckerei|pw})}}\pend
           
\pstart
           \centering{}\textcolor{gray}{\textbf{An der Schönen Blauen Donau\orgindex{der schoenen blauen Donau@An der schönen blauen Donau|pw}}}\pend
           
\pstart
           \centering{}\textcolor{gray}{\textbf{Chef-Redacteur: Dr. F.
                        Mamroth\pwindex{Mamroth, Fedor 21.02.1851 – 25.06.1907@\textsc{Mamroth, Fedor} (21.02.1851 – 25.06.1907), \emph{Journalist/Journalistin, Kritiker/Kritikerin}|pw}. – Redaction: IX.,
                        Berggaſſe 31\oindex{Berggasse@\textbf{Berggasse}, \emph{Straße (K.STR)}|pw}.}}\pend
           
\pstart
           \raggedleft{}\textcolor{gray}{\textbf{Wien\oindex{Wien@\textbf{Wien}, \emph{A.ADM2}|pw}, den}}{ }6. Auguſt \textcolor{gray}{\textbf{18}}89.\pend
           
\pstart\center{}Verehrter Herr Doctor!\pend\vspace{0.5em}
\pstart
           Herzlichſten Dank ſür Ihre ausführlichen Mittheilungen. Ich hoffe, Freitag{ }früh in \label{K_L02643-1v}\edtext{\textsc{Ischl\oindex{Bad Ischl@\textbf{Bad Ischl}, \emph{P.PPL}|pw}}}{\lemma{\textnormal{\emph{Ischl}}}\Cendnote{\textnormal{Am 9. 8. 1889 reisten Goldmann\pwindex{Goldmann, Paul 31.01.1865 – 25.09.1935@\textsc{Goldmann, Paul} (31.01.1865 – 25.09.1935), \emph{Schriftsteller/Schriftstellerin, Journalist/Journalistin}|pwk}, Schnitzler
                  und dessen Bruder Julius Schnitzler\pwindex{Schnitzler, Julius 13.07.1865 – 29.06.1939@\textsc{Schnitzler, Julius} (13.07.1865 – 29.06.1939), \emph{Chirurg/Chirurgin}|pwk} nach
                     Traunkirchen\oindex{Traunkirchen@\textbf{Traunkirchen}, \emph{P.PPL}|pwk}. Auf dem Weg dorthin,
                  möglicherweise bereits in Ischl\oindex{Bad Ischl@\textbf{Bad Ischl}, \emph{P.PPL}|pwk}, trafen sie
                  aufeinander.}}}\label{K_L02643-1} ſein zu können. Freilich kann mir leicht etwas dazwiſchen
               kommen. Jedenfalls erhalten Sie Donnerſtag ein
               telegraphiſches \label{K_L02643-2v}\edtext{Aviſo}{\lemma{\textnormal{\emph{Aviſo}}}\Cendnote{\textnormal{nicht überliefert}}}\label{K_L02643-2}.\pend
           
\pstart
           Die \label{K_L02643-3v}\edtext{Ausrüſtung}{\lemma{\textnormal{\emph{Ausrüſtung}}}\Cendnote{\textnormal{für die bevorstehende Wanderung}}}\label{K_L02643-3} beſorge ich mir,
               ſoweit es in der kurzen {\pb}Zeit noch möglich iſt. Ein
               Punkt dürfte auf Schwierigkeiten ſtoßen: \label{K_L02643-4v}\edtext{Sacktücher}{\lemma{\textnormal{\emph{Sacktücher}}}\Cendnote{\textnormal{Taschentücher}}}\label{K_L02643-4}! Wo ſoll man die in Wien\oindex{Wien@\textbf{Wien}, \emph{A.ADM2}|pw}
               herbekommen! {\dots}\pend
           
\pstart
           Herzlichen Gruß dem \label{K_L02643-5v}\edtext{Dr. \textsc{Spitzer\pwindex{Spitzer, Sigmund 1813-04-01 – 1894-12-26@\textsc{Spitzer, Sigmund} (1813-04-01 – 1894-12-26), \emph{Mediziner/Medizinerin, Diplomat/Diplomatin}|pwu}}, \label{K_L02643-6v}\edtext{dafern}{\lemma{\textnormal{\emph{dafern}}}\Cendnote{\textnormal{veraltet: sofern}}}\label{K_L02643-6} er noch in \textsc{Ischl\oindex{Bad Ischl@\textbf{Bad Ischl}, \emph{P.PPL}|pw}}}{\lemma{\textnormal{\emph{Dr. … Ischl}}}\Cendnote{\textnormal{Über die Ischler\oindex{Bad Ischl@\textbf{Bad Ischl}, \emph{P.PPL}|pwk} Kurlisten
               lässt sich nur ein einziger Dr. Spitzer ermitteln, der bereits sehr alte Anatom und Diplomat Sigmund Spitzer\pwindex{Spitzer, Sigmund 1813-04-01 – 1894-12-26@\textsc{Spitzer, Sigmund} (1813-04-01 – 1894-12-26), \emph{Mediziner/Medizinerin, Diplomat/Diplomatin}|pwk}. Näheres zum Grad der Beziehung 
               ist nicht bekannt.}}}\label{K_L02643-5} iſt.\pend
           
\pstart
           Herzlichen Gruß auch Ihnen! {\\[\baselineskip]}Ihr ergeben\textcolor{gray}{er}{\\[\baselineskip]}\spacefill\mbox{Dr. Paul Goldmann.}\pend
           \leftskip=0em{}\selectlanguage{ngerman}\endnumbering\briefempfaengerindex{Schnitzler, Arthur@\textsc{Schnitzler, Arthur}!zzzGoldmann, Paul@\emph{von Paul Goldmann}!1889-08-061@{6. 8. 1889}|)be}\mylabel{L02643h}  \normalsize

\doendnotes{C}
\bigskip
\vfill

\clearpage

\footnotesize

\lohead{\textsc{register}}

% Definiere theindex-Environment komplett neu ohne reledmac
\makeatletter
\renewenvironment{theindex}{%
  \section*{\indexname}%
  \setlength{\parindent}{0pt}%
  \setlength{\parskip}{0pt plus 0.3pt}%
  \let\item\@idxitem
}{%
  \clearpage
}
\makeatother

\IfFileExists{\jobname-pw.ind}{\input{\jobname-pw.ind}}{}

\end{document}

      