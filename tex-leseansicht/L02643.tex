%% latex-leseansicht-vorspann.tex
%% Vorspann für die Leseansicht.
%% Lädt die gemeinsame Datei latex-vorspann.tex mit nicht gesetztem Schalter.

\newif\ifkorrekturansicht
\korrekturansichtfalse

\input{../tex-inputs/latex-vorspann}


\section[Paul Goldmann an Arthur Schnitzler, 6. 8. 1889]{L02643 Paul Goldmann an Arthur Schnitzler, 6. 8. 1889}
\nopagebreak\mylabel{L02643v}
\rehead{ }\normalsize\beginnumbering\briefempfaengerindex{Schnitzler, Arthur@\textsc{Schnitzler, Arthur}!zzzGoldmann, Paul@\emph{von Paul Goldmann}!1889-08-061@{6. 8. 1889}|(be}
\toendnotes[C]{\smallbreak\pagebreak[2]}
\correspDesc{Versand  durch Paul Goldmann am 6. 8. 1889 in Wien
\newline{}Erhalt  durch Arthur Schnitzler im Zeitraum [6. 8. 1889
                  – 10. 8. 1889?] in Wien}\toendnotes[C]{\smallbreak}
\Standort{DLA, A:Schnitzler, HS.NZ85.1.3162.}
\physDesc{Brief, 1 Blatt, 2 Seiten, 526 Zeichen
\newline{}Handschrift: schwarze Tinte, deutsche Kurrent}\toendnotes[C]{\smallbreak}
\pstart
           \centering{}{\pb}\textcolor{gray}{\textbf{\textbf{Adminiſtration: VII.
                           Seidengaſſe 7\oindex{Wien@\textbf{Wien}!VII., Neubau@\textbf{VII., Neubau}!Seidengasse@\textbf{Seidengasse}, \emph{Straße}|pw}} (Jos. Eberle {\kaufmannsund} Co.\orgindex{Josef Eberle Stein-, Buch und Musikaliendruckerei@Josef Eberle Stein-, Buch und Musikaliendruckerei|pw})}}\pend
           
\pstart
           \centering{}\textcolor{gray}{\textbf{An der Schönen Blauen Donau\orgindex{der schönen blauen Donau@An der schönen blauen Donau|pw}}}\pend
           
\pstart
           \centering{}\textcolor{gray}{\textbf{Chef-Redacteur: Dr. F.
                        Mamroth\pwindex{Mamroth, Fedor 21.\,2.\,1851 Breslau – 25.\,6.\,1907 Frankfurt am Main@\textsc{Mamroth, Fedor} (21.\,2.\,1851 Breslau – 25.\,6.\,1907 Frankfurt am Main), \emph{Journalist, Kritiker}|pw}. – Redaction: IX.,
                        Berggaſſe 31\oindex{Wien@\textbf{Wien}!IX., Alsergrund@\textbf{IX., Alsergrund}!Berggasse@\textbf{Berggasse}, \emph{Straße}|pw}.}}\pend
           
\pstart
           \raggedleft{}\textcolor{gray}{\textbf{Wien\oindex{Wien@\textbf{Wien}, \emph{Verwaltungsgebiet}|pw}, den}}{ }6. Auguſt \textcolor{gray}{\textbf{18}}89.\pend
           
\pstart\center{}Verehrter Herr Doctor!\pend\vspace{0.5em}
\pstart
           Herzlichſten Dank{ }ſür Ihre ausführlichen Mittheilungen. Ich hoffe, Freitag{ }früh in \label{K_L02643-1v}\edtext{\textsc{Ischl\oindex{Bad Ischl@\textbf{Bad Ischl}|pw}}}{\lemma{\textnormal{\emph{Ischl}}}\Cendnote{\textnormal{Am 9. 8. 1889 reisten Goldmann\pwindex{Goldmann, Paul 31.\,1.\,1865 Breslau – 25.\,9.\,1935 Wien@\textsc{Goldmann, Paul} (31.\,1.\,1865 Breslau – 25.\,9.\,1935 Wien), \emph{Schriftsteller, Journalist}|pwk}, Schnitzler
                  und dessen Bruder Julius Schnitzler\pwindex{Schnitzler, Julius 13.\,7.\,1865 Wien – 29.\,6.\,1939 ebd.@\textsc{Schnitzler, Julius} (13.\,7.\,1865 Wien – 29.\,6.\,1939 ebd.), \emph{Chirurg}|pwk} nach
                     Traunkirchen\oindex{Traunkirchen@\textbf{Traunkirchen}|pwk}. Auf dem Weg dorthin,
                  möglicherweise bereits in Ischl\oindex{Bad Ischl@\textbf{Bad Ischl}|pwk}, trafen sie
                  aufeinander.}}}\label{K_L02643-1}{ }ſein zu können. Freilich kann mir leicht etwas dazwiſchen
               kommen. Jedenfalls erhalten Sie Donnerſtag ein
               telegraphiſches \label{K_L02643-2v}\edtext{Aviſo}{\lemma{\textnormal{\emph{Aviso}}}\Cendnote{\textnormal{nicht überliefert}}}\label{K_L02643-2}.\pend
           
\pstart
           Die \label{K_L02643-3v}\edtext{Ausrüſtung}{\lemma{\textnormal{\emph{Ausrüstung}}}\Cendnote{\textnormal{für die bevorstehende Wanderung}}}\label{K_L02643-3} beſorge ich mir,{ }ſoweit es in der kurzen {\pb}Zeit noch möglich iſt. Ein
               Punkt dürfte auf Schwierigkeiten{ }ſtoßen: \label{K_L02643-4v}\edtext{Sacktücher}{\lemma{\textnormal{\emph{Sacktücher}}}\Cendnote{\textnormal{Taschentücher}}}\label{K_L02643-4}! Wo{ }ſoll man die in Wien\oindex{Wien@\textbf{Wien}, \emph{Verwaltungsgebiet}|pw}
               herbekommen! {\dots}\pend
           
\pstart
           Herzlichen Gruß dem \label{K_L02643-5v}\edtext{Dr. \textsc{Spitzer\pwindex{Spitzer, Sigmund 1.\,4.\,1813 Mikulov – 26.\,12.\,1894 Wien@\textsc{Spitzer, Sigmund} (1.\,4.\,1813 Mikulov – 26.\,12.\,1894 Wien), \emph{Mediziner, Diplomat}|pwu}}, \label{K_L02643-6v}\edtext{dafern}{\lemma{\textnormal{\emph{dafern}}}\Cendnote{\textnormal{veraltet: sofern}}}\label{K_L02643-6} er noch in \textsc{Ischl\oindex{Bad Ischl@\textbf{Bad Ischl}|pw}}}{\lemma{\textnormal{\emph{Dr. … Ischl}}}\Cendnote{\textnormal{Über die Ischler\oindex{Bad Ischl@\textbf{Bad Ischl}|pwk} Kurlisten
               lässt sich nur ein einziger Dr. Spitzer ermitteln, der bereits sehr alte Anatom und Diplomat Sigmund Spitzer\pwindex{Spitzer, Sigmund 1.\,4.\,1813 Mikulov – 26.\,12.\,1894 Wien@\textsc{Spitzer, Sigmund} (1.\,4.\,1813 Mikulov – 26.\,12.\,1894 Wien), \emph{Mediziner, Diplomat}|pwk}. Näheres zum Grad der Beziehung 
               ist nicht bekannt.}}}\label{K_L02643-5} iſt.\pend
           
\pstart
           Herzlichen Gruß auch Ihnen! {\\[\baselineskip]}Ihr ergeben\textcolor{gray}{er}{\\[\baselineskip]}\spacefill\mbox{Dr. Paul Goldmann.}\pend
           \leftskip=0em{}\selectlanguage{ngerman}\endnumbering\briefempfaengerindex{Schnitzler, Arthur@\textsc{Schnitzler, Arthur}!zzzGoldmann, Paul@\emph{von Paul Goldmann}!1889-08-061@{6. 8. 1889}|)be}\mylabel{L02643h}  \newcommand{\dateiname}{L02643}\newcommand{\titel}{Paul Goldmann an Arthur Schnitzler, 6. 8. 1889}\newcommand{\editorInnen}{Martin Anton Müller und Laura Untner}%% latex-leseansicht-abspann.tex
%% Abspann für die Leseansicht.
%% Der Schalter \ifkorrekturansicht ist bereits durch den Vorspann gesetzt.

%% latex-abspann.tex
%% Gemeinsamer Abspann für Korrekturansicht und Leseansicht.
%% Setzt den Schalter \ifkorrekturansicht voraus (gesetzt in den
%% einbindenden Dateien latex-korrekturansicht-abspann.tex bzw.
%% latex-leseansicht-abspann.tex).
%% ---------------------------------------------------------------

\normalsize

% Das esempio-Environment wird nur in der Leseansicht benötigt
\ifkorrekturansicht\else
\newenvironment{esempio}[3]%
{
    \vspace{1.5ex}
    \rlap{\underline{#1}}
    \par
    \setlength{\parindent}{0cm}
    \nopagebreak
    \leftskip=#2cm
    \rightskip=#3cm
}
{
    \par
}
\fi

\doendnotes{C}
\bigskip
\vfill

\clearpage

\footnotesize

\ifkorrekturansicht
  \lohead{\textsc{register}}
\fi

% theindex-Environment neu definieren ohne reledmac
\makeatletter
\renewenvironment{theindex}{%
  \ifkorrekturansicht
    \section*{\indexname}%
  \else
    \subsubsection*{Index der erwähnten Entitäten}%
  \fi
  \setlength{\parindent}{0pt}%
  \setlength{\parskip}{0pt plus 0.3pt}%
  \let\item\@idxitem
}{%
  \ifkorrekturansicht\clearpage\fi
}
\makeatother

\IfFileExists{\jobname-pw.ind}{\input{\jobname-pw.ind}}{}

% Quellenangabe nur in der Leseansicht
\ifkorrekturansicht\else
% Fallback-Definitionen, falls die .tex-Datei \titel etc. nicht gesetzt hat
\providecommand{\titel}{}
\providecommand{\editorInnen}{}
\providecommand{\dateiname}{\jobname}

\vspace{3cm}

\vfill

\footnotesize
\textsc{Quelle}: \titel. Herausgegeben von {\editorInnen}. In: \emph{Arthur Schnitzler: Briefwechsel mit Autorinnen und Autoren}.
 Digitale Edition, https://schnitzler-briefe.acdh.oeaw.ac.at/{\dateiname}.html (Stand \today)
\fi

\end{document}


