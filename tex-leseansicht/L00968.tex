%% latex-leseansicht-vorspann.tex
%% Vorspann für die Leseansicht.
%% Lädt die gemeinsame Datei latex-vorspann.tex mit nicht gesetztem Schalter.

\newif\ifkorrekturansicht
\korrekturansichtfalse

\input{../tex-inputs/latex-vorspann}


\section[Richard Beer-Hofmann an Arthur Schnitzler, 7. 9. 1899]{L00968 Richard Beer-Hofmann an Arthur Schnitzler, 7. 9. 1899}
\nopagebreak\mylabel{L00968v}
\rehead{ }\normalsize\beginnumbering\briefempfaengerindex{Schnitzler, Arthur@\textsc{Schnitzler, Arthur}!zzzBeer-Hofmann, Richard@\emph{von Richard Beer-Hofmann}!1899-09-071@{7. 9. 1899}|(be}
\toendnotes[C]{\smallbreak\pagebreak[2]}
\correspDesc{Versand  durch Richard Beer-Hofmann am 7. 9. 1899 in Brixen
\newline{}Erhalt  durch Arthur Schnitzler am 9. 9. 1899 in Bad Ischl}\toendnotes[C]{\smallbreak}
\Standort{CUL, Schnitzler, B 8.}
\physDesc{Bildpostkarte, 175 Zeichen
\newline{}Handschrift: Bleistift, lateinische Kurrent
\newline{}Versand: 1) Stempel: »\nobreak{}8. 9. 99\nobreak{}«.   2) Stempel: »\nobreak{}\oindex{Bad Ischl@\textbf{Bad Ischl}|pwk}Ischl, 9. 9. 99, 12–1N\nobreak{}«. 
\newline{}Ordnung: mit Bleistift von unbekannter Hand nummeriert: »140« }\pstart{}{\pb}D\textsuperscript{r}
                  Arthur Schnitzler\pend{}\pstart{}Ischl\oindex{Bad Ischl@\textbf{Bad Ischl}|pw}\pend{}\pstart{}Pension Petter\oindex{Hotel und Pension Rudolfshöhe (Leopold Petter)@\textbf{Hotel und Pension Rudolfshöhe (Leopold Petter)}, \emph{Hotel}|pw}\pend{}{\bigskip}
\pstart
           \noindent{}\centering{}{\pb}\textcolor{gray}{\textbf{Adlergasse\oindex{Adlerbrückengasse@\textbf{Adlerbrückengasse}, \emph{Straße}|pw}. Gruss aus Brixen\oindex{Brixen@\textbf{Brixen}, \emph{Hauptstadt}|pw}.}}\pend
           \vspace{1em}
\pstart
           \raggedleft{}{\pb}7/IX 99\pend
           \vspace{0.5em}
\pstart
           Lieber Arthur! Ich übersiedle, hoffentlich auf einige Wochen, nach
                  \uline{Vahrn\oindex{Vahrn@\textbf{Vahrn}, \emph{Hauptstadt}|pw} bei Brixen\oindex{Brixen@\textbf{Brixen}, \emph{Hauptstadt}|pw}}. Nähere Adresse nicht notwendig. Herzlichst \spacefill\mbox{Richard}\pend
           \selectlanguage{ngerman}\endnumbering\briefempfaengerindex{Schnitzler, Arthur@\textsc{Schnitzler, Arthur}!zzzBeer-Hofmann, Richard@\emph{von Richard Beer-Hofmann}!1899-09-071@{7. 9. 1899}|)be}\mylabel{L00968h}  \newcommand{\dateiname}{L00968}\newcommand{\titel}{Richard Beer-Hofmann an Arthur Schnitzler, 7. 9. 1899}\newcommand{\editorInnen}{Martin Anton Müller und Gerd-Hermann Susen}%% latex-leseansicht-abspann.tex
%% Abspann für die Leseansicht.
%% Der Schalter \ifkorrekturansicht ist bereits durch den Vorspann gesetzt.

%% latex-abspann.tex
%% Gemeinsamer Abspann für Korrekturansicht und Leseansicht.
%% Setzt den Schalter \ifkorrekturansicht voraus (gesetzt in den
%% einbindenden Dateien latex-korrekturansicht-abspann.tex bzw.
%% latex-leseansicht-abspann.tex).
%% ---------------------------------------------------------------

\normalsize

% Das esempio-Environment wird nur in der Leseansicht benötigt
\ifkorrekturansicht\else
\newenvironment{esempio}[3]%
{
    \vspace{1.5ex}
    \rlap{\underline{#1}}
    \par
    \setlength{\parindent}{0cm}
    \nopagebreak
    \leftskip=#2cm
    \rightskip=#3cm
}
{
    \par
}
\fi

\doendnotes{C}
\bigskip
\vfill

\clearpage

\footnotesize

\ifkorrekturansicht
  \lohead{\textsc{register}}
\fi

% theindex-Environment neu definieren ohne reledmac
\makeatletter
\renewenvironment{theindex}{%
  \ifkorrekturansicht
    \section*{\indexname}%
  \else
    \subsubsection*{Index der erwähnten Entitäten}%
  \fi
  \setlength{\parindent}{0pt}%
  \setlength{\parskip}{0pt plus 0.3pt}%
  \let\item\@idxitem
}{%
  \ifkorrekturansicht\clearpage\fi
}
\makeatother

\IfFileExists{\jobname-pw.ind}{\input{\jobname-pw.ind}}{}

% Quellenangabe nur in der Leseansicht
\ifkorrekturansicht\else
% Fallback-Definitionen, falls die .tex-Datei \titel etc. nicht gesetzt hat
\providecommand{\titel}{}
\providecommand{\editorInnen}{}
\providecommand{\dateiname}{\jobname}

\vspace{3cm}

\vfill

\footnotesize
\textsc{Quelle}: \titel. Herausgegeben von {\editorInnen}. In: \emph{Arthur Schnitzler: Briefwechsel mit Autorinnen und Autoren}.
 Digitale Edition, https://schnitzler-briefe.acdh.oeaw.ac.at/{\dateiname}.html (Stand \today)
\fi

\end{document}


