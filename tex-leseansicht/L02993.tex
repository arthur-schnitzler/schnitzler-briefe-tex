%% latex-korrekturansicht-vorspann.tex
%% Vorspann für die Korrekturansicht.
%% Lädt die gemeinsame Datei latex-vorspann.tex mit gesetztem Schalter.

\newif\ifkorrekturansicht
\korrekturansichttrue

\input{../tex-inputs/latex-vorspann}


\section[ Arthur Schnitzler an Felix Salten, 13. 12. 1904]{L02993 Arthur Schnitzler an Felix Salten, 13. 12. 1904}
\nopagebreak\mylabel{L02993v}
\rehead{ }\normalsize\beginnumbering\briefempfaengerindex{Salten, Felix@\textsc{Salten, Felix}!zzzSchnitzler, Arthur@\emph{von Arthur Schnitzler}!1904-12-131@{13. 12. 1904}|(be}
\toendnotes[C]{\smallbreak\pagebreak[2]}\Standort{Wienbibliothek im Rathaus, ZPH 1681, 2.1.516.}
\physDesc{Kartenbrief, 411 Zeichen
\newline{}Handschrift: schwarze Tinte, deutsche Kurrent
\newline{}Versand: Stempel: »\nobreak{}\oindex{VIII., Josefstadt@\textbf{VIII., Josefstadt}, \emph{A.ADM3}|pwk}18/1 Wien 110, 14. X\textcolor{gray}{II}. 04, X\nobreak{}«.  
\newline{}Ordnung: mit Bleistift von unbekannter Hand nummeriert: »31« }\toendnotes[C]{\smallbreak}\pstart{}{\pb}Herrn Felix Salten\pend{}\pstart{}Wien IX\oindex{IX., Alsergrund@\textbf{IX., Alsergrund}, \emph{A.ADM3}|pw}\pend{}\pstart{}\textsc{Porzellangasse 45\oindex{Porzellangasse@\textbf{Porzellangasse}, \emph{Straße (K.STR)}|pw}.}\pend{}{\bigskip}\vspace{1em}
\pstart
           \raggedleft{}{\pb}13. 12. 904\pend
           \vspace{0.5em}
\pstart
           lieber, könnten Sie am Samſtag (we{\geminationn} Ihre Frau\pwindex{Salten, Ottilie 07.03.1868 – 22.06.1942@\textsc{Salten, Ottilie} (07.03.1868 – 22.06.1942), \emph{Schauspieler/Schauspielerin}|pwv} ſchon da iſt, natürlich Sie beide) bei uns
               nachtmahlen? Beſtimmen Sie ſelbſt die Stunde.\pend
           
\pstart
           Herzlichſt der Ihrige {\\[\baselineskip]}\spacefill\mbox{Arthur.}\pend
           \leftskip=0em{}
\pstart
           \noindent{}Über Ihren \label{K_L02993-1v}\edtext{Artikel\pwindex{Artur Schnitzler-Abend@\emph{Artur Schnitzler-Abend}|pwv}}{\lemma{\textnormal{\emph{Artikel}}}\Cendnote{\textnormal{Am 12. 12. 1904
                     hatte ein »Arthur-Schnitzler-Abend« im Carl-Theater\oindex{Carl-Theater@\textbf{Carl-Theater}, \emph{Theater (K.THE)}|pwk} stattgefunden. Dieser wurde für das seit 1787 bestehende \emph{Erste öffentliche Kinderkrankeninstitut}\orgindex{Erstes oeffentliches Kinderkrankeninstitut@Erstes öffentliches Kinderkrankeninstitut|pwk}
                     abgehalten, dessen Leitung Carl Hochsinger\pwindex{Hochsinger, Carl 12.07.1860 – 28.10.1942@\textsc{Hochsinger, Carl} (12.07.1860 – 28.10.1942), \emph{Pädiater/Pädiaterin}|pwk}
                     innehatte. Salten\pwindex{Salten, Felix 06.09.1869 – 08.10.1945@\textsc{Salten, Felix} (06.09.1869 – 08.10.1945), \emph{Schriftsteller/Schriftstellerin, Journalist/Journalistin, Chefredakteur/Chefredakteurin}|pwk} rezensierte ihn in: Felix Salten\pwindex{Salten, Felix 06.09.1869 – 08.10.1945@\textsc{Salten, Felix} (06.09.1869 – 08.10.1945), \emph{Schriftsteller/Schriftstellerin, Journalist/Journalistin, Chefredakteur/Chefredakteurin}|pwk}: \emph{Artur Schnitzler-Abend}\pwindex{Artur Schnitzler-Abend@\emph{Artur Schnitzler-Abend}|pwk}. In: \emph{Die Zeit}\pwindex{Zeit@\emph{Die Zeit}|pwk}, Jg. 3, Nr. 796, Morgenblatt, 13. 12. 1904 , S. 3.}}}\label{K_L02993-1} hab ich mich
                  wie Sie ſich denken können ſehr gefreut. Im allgemeinen hab ich allerdings \label{K_L02993-11v}\edtext{diesmal
                  die Empfindung, als we{\geminationn} man mich in Schulden}{\lemma{\textnormal{\emph{diesmal … Schulden}}}\Cendnote{\textnormal{Hier handelt es sich um eine implizite Anspielung
                  auf die letzte Rezension einer Arbeit Schnitzlers durch Salten\pwindex{Salten, Felix 06.09.1869 – 08.10.1945@\textsc{Salten, Felix} (06.09.1869 – 08.10.1945), \emph{Schriftsteller/Schriftstellerin, Journalist/Journalistin, Chefredakteur/Chefredakteurin}|pwk}, 
                     vgl. Arthur Schnitzler an Felix Salten, 7. 11. 1903.}}}\label{K_L02993-11} geſtürzt
                  hätte, die ich nicht bezahlen kann.\pend
           \selectlanguage{ngerman}\endnumbering\briefempfaengerindex{Salten, Felix@\textsc{Salten, Felix}!zzzSchnitzler, Arthur@\emph{von Arthur Schnitzler}!1904-12-131@{13. 12. 1904}|)be}\mylabel{L02993h}  \normalsize

\doendnotes{C}
\bigskip
\vfill

\clearpage

\footnotesize

\lohead{\textsc{register}}

% Definiere theindex-Environment komplett neu ohne reledmac
\makeatletter
\renewenvironment{theindex}{%
  \section*{\indexname}%
  \setlength{\parindent}{0pt}%
  \setlength{\parskip}{0pt plus 0.3pt}%
  \let\item\@idxitem
}{%
  \clearpage
}
\makeatother

\IfFileExists{\jobname-pw.ind}{\input{\jobname-pw.ind}}{}

\end{document}

      