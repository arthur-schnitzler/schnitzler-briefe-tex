%% latex-leseansicht-vorspann.tex
%% Vorspann für die Leseansicht.
%% Lädt die gemeinsame Datei latex-vorspann.tex mit nicht gesetztem Schalter.

\newif\ifkorrekturansicht
\korrekturansichtfalse

\input{../tex-inputs/latex-vorspann}

\begin{center}
            \textcolor{red}{ENTWURF, NICHT FERTIG KORRIGIERT}
                      \end{center}
            
         
         \renewcommand{\erwaehntePersonen}{Personen: Carl Hochsinger, Felix Salten, Ottilie Salten}
         \renewcommand{\erwaehnteInstitutionen}{Institutionen: Erstes öffentliches Kinderkrankeninstitut}
         \renewcommand{\erwaehnteOrte}{Orte: Carl-Theater, IX., Alsergrund, Porzellangasse, VIII., Josefstadt, Wien}
         \renewcommand{\erwaehnteWerke}{Werke: Artur Schnitzler-Abend, Die Zeit}
               \section[ Arthur Schnitzler an Felix Salten, 13. 12. 1904]{ Arthur Schnitzler an Felix Salten, 13. 12. 1904}\nopagebreak\mylabel{v}\rehead{ }\begin{ledgroupsized}[t]{13cm}\normalsize\beginnumbering \toendnotes[C]{\smallbreak\pagebreak[2]} \Standort{Wienbibliothek im Rathaus, ZPH 1681, 2.1.516.}
\physDesc{Kartenbrief, 411 Zeichen
\newline{}Handschrift: schwarze Tinte, deutsche Kurrent
\newline{}Versand: Stempel: »\nobreak{}\oindex{VIII., Josefstadt@\textbf{VIII., Josefstadt}|pwk}18/1 Wien 110, 14. X\textcolor{gray}{II}. 04, X\nobreak{}«.  
\newline{}Ordnung: mit Bleistift von unbekannter Hand nummeriert: »31« }\toendnotes[C]{\smallbreak}\pstart{}{\pb}Herrn Felix Salten\pend{}\pstart{}Wien IX\oindex{IX., Alsergrund@\textbf{IX., Alsergrund}|pw}\pend{}\pstart{}\textsc{Porzellangasse 45\oindex{Porzellangasse@\textbf{Porzellangasse}|pw}.}\pend{}{\bigskip}\pstart
           \raggedleft{}{\pb}13. 12. 904\pend
           \pstart
           lieber, könnten Sie am Samſtag (we{\geminationn} Ihre Frau\pwindex{Salten, Ottilie 07.03.1868 – 22.06.1942@\textsc{Salten, Ottilie} (07.03.1868 – 22.06.1942), \emph{Schauspielerin}|pwv} ſchon da iſt, natürlich Sie beide) bei uns
               nachtmahlen? Beſtimmen Sie ſelbſt die Stunde.\pend
           \pstart
           Herzlichſt der Ihrige {\\[\baselineskip]}\spacefill\mbox{Arthur.}\pend
           \leftskip=0em{}\pstart
           \noindent{}Über Ihren \label{K_L02993-1v}\edtext{Artikel\pwindex{Salten, Felix 06.09.1869 – 08.10.1945@\textsc{Salten, Felix} (06.09.1869 – 08.10.1945), \emph{Schriftsteller, Journalist}!Artur Schnitzler-Abend13. 12. 1904@\strich\emph{Artur Schnitzler-Abend} {[}13. 12. 1904{]}|pwv}}{\lemma{\textnormal{\emph{Artikel}}}\Cendnote{\textnormal{Am 12. 12. 1904
                     hatte ein »Arthur-Schnitzler-Abend« im Carl-Theater\oindex{Carl-Theater@\textbf{Carl-Theater}|pwk} stattgefunden. Dieser wurde für das seit 1787 bestehende \emph{Erste öffentliche Kinderkrankeninstitut}\orgindex{Erstes oeffentliches Kinderkrankeninstitut@Erstes öffentliches Kinderkrankeninstitut|pwk}
                     abgehalten, dessen Leitung Carl Hochsinger\pwindex{Hochsinger, Carl 12.07.1860 – 28.10.1942@\textsc{Hochsinger, Carl} (12.07.1860 – 28.10.1942), \emph{Pädiater}|pwk}
                     inne hatte. Salten\pwindex{Salten, Felix 06.09.1869 – 08.10.1945@\textsc{Salten, Felix} (06.09.1869 – 08.10.1945), \emph{Schriftsteller, Journalist}|pwk} rezensierte ihn in: Felix Salten\pwindex{Salten, Felix 06.09.1869 – 08.10.1945@\textsc{Salten, Felix} (06.09.1869 – 08.10.1945), \emph{Schriftsteller, Journalist}|pwk}: \emph{Artur Schnitzler-Abend}\pwindex{Salten, Felix 06.09.1869 – 08.10.1945@\textsc{Salten, Felix} (06.09.1869 – 08.10.1945), \emph{Schriftsteller, Journalist}!Artur Schnitzler-Abend13. 12. 1904@\strich\emph{Artur Schnitzler-Abend} {[}13. 12. 1904{]}|pwk}. In: \emph{Die Zeit}\pwindex{Zeit1902-09-27 – 1919@\emph{Die Zeit} {[}1902-09-27 – 1919{]}|pwk}, Jg. 3, Nr. 796, Morgenblatt, 13. 12. 1904 , S. 3.}}}\label{K_L02993-1h} hab ich mich
                  wie Sie ſich denken können ſehr gefreut. Im allgemeinen hab ich allerdings \label{K_L02993-11v}\edtext{diesmal
                  die Empfindung, als we{\geminationn} man mich in Schulden}{\lemma{\textnormal{\emph{diesmal … Schulden}}}\Cendnote{\textnormal{Hier handelt es sich um eine implizite Anspielung
                  auf die letzte Rezension einer Arbeit Schnitzler\pwindex{Schnitzler, Arthur 15.05.1862 – 21.10.1931@\textsc{Schnitzler, Arthur} (15.05.1862 – 21.10.1931), \emph{Schriftsteller, Mediziner}|pwk}s durch Salten\pwindex{Salten, Felix 06.09.1869 – 08.10.1945@\textsc{Salten, Felix} (06.09.1869 – 08.10.1945), \emph{Schriftsteller, Journalist}|pwk}, 
                     vgl. Arthur Schnitzler an Felix Salten, 7. 11. 1903.}}}\label{K_L02993-11h} geſtürzt
                  hätte, die ich nicht bezahlen kann.\pend
           
         
         \endnumbering\mylabel{h}\end{ledgroupsized}  \newcommand{\dateiname}{L02993}\newcommand{\titel}{Arthur Schnitzler an Felix Salten, 13. 12. 1904}\newcommand{\editorInnen}{Martin Anton Müller und Laura Untner}%% latex-leseansicht-abspann.tex
%% Abspann für die Leseansicht.
%% Der Schalter \ifkorrekturansicht ist bereits durch den Vorspann gesetzt.

%% latex-abspann.tex
%% Gemeinsamer Abspann für Korrekturansicht und Leseansicht.
%% Setzt den Schalter \ifkorrekturansicht voraus (gesetzt in den
%% einbindenden Dateien latex-korrekturansicht-abspann.tex bzw.
%% latex-leseansicht-abspann.tex).
%% ---------------------------------------------------------------

\normalsize

% Das esempio-Environment wird nur in der Leseansicht benötigt
\ifkorrekturansicht\else
\newenvironment{esempio}[3]%
{
    \vspace{1.5ex}
    \rlap{\underline{#1}}
    \par
    \setlength{\parindent}{0cm}
    \nopagebreak
    \leftskip=#2cm
    \rightskip=#3cm
}
{
    \par
}
\fi

\doendnotes{C}
\bigskip
\vfill

\clearpage

\footnotesize

\ifkorrekturansicht
  \lohead{\textsc{register}}
\fi

% theindex-Environment neu definieren ohne reledmac
\makeatletter
\renewenvironment{theindex}{%
  \ifkorrekturansicht
    \section*{\indexname}%
  \else
    \subsubsection*{Index der erwähnten Entitäten}%
  \fi
  \setlength{\parindent}{0pt}%
  \setlength{\parskip}{0pt plus 0.3pt}%
  \let\item\@idxitem
}{%
  \ifkorrekturansicht\clearpage\fi
}
\makeatother

\IfFileExists{\jobname-pw.ind}{\input{\jobname-pw.ind}}{}

% Quellenangabe nur in der Leseansicht
\ifkorrekturansicht\else
% Fallback-Definitionen, falls die .tex-Datei \titel etc. nicht gesetzt hat
\providecommand{\titel}{}
\providecommand{\editorInnen}{}
\providecommand{\dateiname}{\jobname}

\vspace{3cm}

\vfill

\footnotesize
\textsc{Quelle}: \titel. Herausgegeben von {\editorInnen}. In: \emph{Arthur Schnitzler: Briefwechsel mit Autorinnen und Autoren}.
 Digitale Edition, https://schnitzler-briefe.acdh.oeaw.ac.at/{\dateiname}.html (Stand \today)
\fi

\end{document}


      