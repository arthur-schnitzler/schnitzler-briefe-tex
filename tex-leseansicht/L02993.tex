%% latex-leseansicht-vorspann.tex
%% Vorspann für die Leseansicht.
%% Lädt die gemeinsame Datei latex-vorspann.tex mit nicht gesetztem Schalter.

\newif\ifkorrekturansicht
\korrekturansichtfalse

\input{../tex-inputs/latex-vorspann}


\section[ Arthur Schnitzler an Felix Salten, 13. 12. 1904]{L02993 Arthur Schnitzler an Felix Salten,  13. 12. 1904}
\nopagebreak\mylabel{L02993v}
\rehead{ }\normalsize\beginnumbering\briefempfaengerindex{Salten, Felix@\textsc{Salten, Felix}!zzzSchnitzler, Arthur@\emph{von Arthur Schnitzler}!1904-12-131@{13. 12. 1904}|(be}
\toendnotes[C]{\smallbreak\pagebreak[2]}
\correspDesc{Versand  durch Arthur Schnitzler am 13. 12. 1904 in Wien
\newline{}Übermittlung  am 14. 12. 1904 in Wien
\newline{}Erhalt  durch Felix Salten im Zeitraum [14. 12. 1904 – 16. 12. 1904] in Wien}\toendnotes[C]{\smallbreak}
\Standort{Wienbibliothek im Rathaus, ZPH 1681, 2.1.516.}
\physDesc{Kartenbrief, 411 Zeichen
\newline{}Handschrift: schwarze Tinte, deutsche Kurrent
\newline{}Versand: Stempel: »\nobreak{}\oindex{VIII., Josefstadt@\textbf{VIII., Josefstadt}, \emph{Verwaltungsgebiet}|pwk}18/1 Wien 110, 14. X\textcolor{gray}{II}. 04, X\nobreak{}«.  
\newline{}Ordnung: mit Bleistift von unbekannter Hand nummeriert:
                                            »31« }\toendnotes[C]{\smallbreak}\pstart{}{\pb}Herrn Felix Salten\pend{}\pstart{}Wien IX\oindex{IX., Alsergrund@\textbf{IX., Alsergrund}, \emph{Verwaltungsgebiet}|pw}\pend{}\pstart{}\textsc{Porzellangasse 45\oindex{Wien@\textbf{Wien}!IX., Alsergrund@\textbf{IX., Alsergrund}!Porzellangasse@\textbf{Porzellangasse}, \emph{Straße}|pw}.}\pend{}{\bigskip}\vspace{1em}
\pstart
           \raggedleft{}{\pb}13. 12. 904\pend
           \vspace{0.5em}
\pstart
           lieber, könnten Sie am Samſtag
                        (we{\geminationn} Ihre Frau\pwindex{Salten, Ottilie 7.\,3.\,1868 Prag – 22.\,6.\,1942 Zürich@\textsc{Salten, Ottilie} (7.\,3.\,1868 Prag – 22.\,6.\,1942 Zürich), \emph{Schauspielerin}|pwv}{ }ſchon da iſt, natürlich Sie beide) bei uns
                    nachtmahlen? Beſtimmen Sie{ }ſelbſt die Stunde.\pend
           
\pstart
           Herzlichſt der Ihrige {\\[\baselineskip]}\spacefill\mbox{Arthur.}\pend
           \leftskip=0em{}
\pstart
           \noindent{}Über Ihren \label{K_L02993-1v}\edtext{Artikel\pwindex{Salten, Felix 6.\,9.\,1869 Budapest – 8.\,10.\,1945 Zürich@\textsc{Salten, Felix} (6.\,9.\,1869 Budapest – 8.\,10.\,1945 Zürich), \emph{Schriftsteller, Journalist, Chefredakteur}!Artur Schnitzler-Abend@\strich\emph{Artur Schnitzler-Abend}|pwv}}{\lemma{\textnormal{\emph{Artikel}}}\Cendnote{\textnormal{Am 12. 12. 1904
                            hatte ein »Arthur-Schnitzler-Abend« im Carl-Theater\oindex{Wien@\textbf{Wien}!II., Leopoldstadt@\textbf{II., Leopoldstadt}!Carl-Theater@\textbf{Carl-Theater}, \emph{Theater}|pwk} stattgefunden. Dieser wurde für das seit
                                1787 bestehende \emph{Erste
                                öffentliche Kinderkrankeninstitut}\orgindex{Erstes öffentliches Kinderkrankeninstitut@Erstes öffentliches Kinderkrankeninstitut|pwk} abgehalten, dessen Leitung
                                Carl Hochsinger\pwindex{Hochsinger, Carl 12.\,7.\,1860 Wien – 28.\,10.\,1942 Konzentrationslager Theresienstadt@\textsc{Hochsinger, Carl} (12.\,7.\,1860 Wien – 28.\,10.\,1942 Konzentrationslager Theresienstadt), \emph{Pädiater}|pwk} innehatte.
                                Salten\pwindex{Salten, Felix 6.\,9.\,1869 Budapest – 8.\,10.\,1945 Zürich@\textsc{Salten, Felix} (6.\,9.\,1869 Budapest – 8.\,10.\,1945 Zürich), \emph{Schriftsteller, Journalist, Chefredakteur}|pwk} rezensierte ihn in: Felix Salten\pwindex{Salten, Felix 6.\,9.\,1869 Budapest – 8.\,10.\,1945 Zürich@\textsc{Salten, Felix} (6.\,9.\,1869 Budapest – 8.\,10.\,1945 Zürich), \emph{Schriftsteller, Journalist, Chefredakteur}|pwk}: \emph{Artur Schnitzler-Abend}\pwindex{Salten, Felix 6.\,9.\,1869 Budapest – 8.\,10.\,1945 Zürich@\textsc{Salten, Felix} (6.\,9.\,1869 Budapest – 8.\,10.\,1945 Zürich), \emph{Schriftsteller, Journalist, Chefredakteur}!Artur Schnitzler-Abend@\strich\emph{Artur Schnitzler-Abend}|pwk}. In: \emph{Die Zeit}\pwindex{Zeit@\emph{Die Zeit}|pwk}, Jg. 3, Nr. 796, Morgenblatt,
                                    13. 12. 1904, S. 3.}}}\label{K_L02993-1}
                        hab ich mich wie Sie{ }ſich denken können{ }ſehr gefreut. Im allgemeinen hab ich
                        allerdings \label{K_L02993-11v}\edtext{diesmal die
                        Empfindung, als we{\geminationn} man mich in Schulden}{\lemma{\textnormal{\emph{diesmal … Schulden}}}\Cendnote{\textnormal{Hier handelt es sich um eine
                            implizite Anspielung auf die letzte Rezension einer Arbeit Schnitzlers durch Salten\pwindex{Salten, Felix 6.\,9.\,1869 Budapest – 8.\,10.\,1945 Zürich@\textsc{Salten, Felix} (6.\,9.\,1869 Budapest – 8.\,10.\,1945 Zürich), \emph{Schriftsteller, Journalist, Chefredakteur}|pwk}, vgl. XXXX Auszeichnungsfehler: Dokument L02988 nicht gefunden.}}}\label{K_L02993-11}
                        geſtürzt hätte, die ich nicht bezahlen kann.\pend
           \selectlanguage{ngerman}\endnumbering\briefempfaengerindex{Salten, Felix@\textsc{Salten, Felix}!zzzSchnitzler, Arthur@\emph{von Arthur Schnitzler}!1904-12-131@{13. 12. 1904}|)be}\mylabel{L02993h}  \newcommand{\dateiname}{L02993}\newcommand{\titel}{Arthur Schnitzler an Felix Salten, 13. 12. 1904}\newcommand{\editorInnen}{Martin Anton Müller und Laura Untner}%% latex-leseansicht-abspann.tex
%% Abspann für die Leseansicht.
%% Der Schalter \ifkorrekturansicht ist bereits durch den Vorspann gesetzt.

%% latex-abspann.tex
%% Gemeinsamer Abspann für Korrekturansicht und Leseansicht.
%% Setzt den Schalter \ifkorrekturansicht voraus (gesetzt in den
%% einbindenden Dateien latex-korrekturansicht-abspann.tex bzw.
%% latex-leseansicht-abspann.tex).
%% ---------------------------------------------------------------

\normalsize

% Das esempio-Environment wird nur in der Leseansicht benötigt
\ifkorrekturansicht\else
\newenvironment{esempio}[3]%
{
    \vspace{1.5ex}
    \rlap{\underline{#1}}
    \par
    \setlength{\parindent}{0cm}
    \nopagebreak
    \leftskip=#2cm
    \rightskip=#3cm
}
{
    \par
}
\fi

\doendnotes{C}
\bigskip
\vfill

\clearpage

\footnotesize

\ifkorrekturansicht
  \lohead{\textsc{register}}
\fi

% theindex-Environment neu definieren ohne reledmac
\makeatletter
\renewenvironment{theindex}{%
  \ifkorrekturansicht
    \section*{\indexname}%
  \else
    \subsubsection*{Index der erwähnten Entitäten}%
  \fi
  \setlength{\parindent}{0pt}%
  \setlength{\parskip}{0pt plus 0.3pt}%
  \let\item\@idxitem
}{%
  \ifkorrekturansicht\clearpage\fi
}
\makeatother

\IfFileExists{\jobname-pw.ind}{\input{\jobname-pw.ind}}{}

% Quellenangabe nur in der Leseansicht
\ifkorrekturansicht\else
% Fallback-Definitionen, falls die .tex-Datei \titel etc. nicht gesetzt hat
\providecommand{\titel}{}
\providecommand{\editorInnen}{}
\providecommand{\dateiname}{\jobname}

\vspace{3cm}

\vfill

\footnotesize
\textsc{Quelle}: \titel. Herausgegeben von {\editorInnen}. In: \emph{Arthur Schnitzler: Briefwechsel mit Autorinnen und Autoren}.
 Digitale Edition, https://schnitzler-briefe.acdh.oeaw.ac.at/{\dateiname}.html (Stand \today)
\fi

\end{document}


