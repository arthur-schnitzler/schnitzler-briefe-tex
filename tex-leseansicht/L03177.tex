%% latex-leseansicht-vorspann.tex
%% Vorspann für die Leseansicht.
%% Lädt die gemeinsame Datei latex-vorspann.tex mit nicht gesetztem Schalter.

\newif\ifkorrekturansicht
\korrekturansichtfalse

\input{../tex-inputs/latex-vorspann}


\section[ Felix Salten u. a. an Arthur Schnitzler, 6. 8. 1896]{L03177 Felix Salten u. a. an Arthur Schnitzler,  6. 8. 1896}
\nopagebreak\mylabel{L03177v}
\rehead{ }\normalsize\beginnumbering\briefempfaengerindex{Schnitzler, Arthur@\textsc{Schnitzler, Arthur}!zzzSchlesinger, Alfred@\emph{von Alfred Schlesinger}!1896-08-061@{6. 8. 1896}|(be}\briefempfaengerindex{Schnitzler, Arthur@\textsc{Schnitzler, Arthur}!zzzBerger, Else@\emph{von Else Berger}!1896-08-061@{6. 8. 1896}|(be}\briefempfaengerindex{Schnitzler, Arthur@\textsc{Schnitzler, Arthur}!zzzSchlesinger, Franziska@\emph{von Franziska Schlesinger}!1896-08-061@{6. 8. 1896}|(be}\briefempfaengerindex{Schnitzler, Arthur@\textsc{Schnitzler, Arthur}!zzzSchlesinger, Margherita@\emph{von Margherita Schlesinger}!1896-08-061@{6. 8. 1896}|(be}\briefempfaengerindex{Schnitzler, Arthur@\textsc{Schnitzler, Arthur}!zzzLaurent, M.@\emph{von M. Laurent}!1896-08-061@{6. 8. 1896}|(be}\briefempfaengerindex{Schnitzler, Arthur@\textsc{Schnitzler, Arthur}!zzzEisler, Richard@\emph{von Richard Eisler}!1896-08-061@{6. 8. 1896}|(be}\briefempfaengerindex{Schnitzler, Arthur@\textsc{Schnitzler, Arthur}!zzzFanto, Leonhard@\emph{von Leonhard Fanto}!1896-08-061@{6. 8. 1896}|(be}\briefempfaengerindex{Schnitzler, Arthur@\textsc{Schnitzler, Arthur}!zzzSchlesinger, Julius@\emph{von Julius Schlesinger}!1896-08-061@{6. 8. 1896}|(be}\briefempfaengerindex{Schnitzler, Arthur@\textsc{Schnitzler, Arthur}!zzzSchlesinger, Therese@\emph{von Therese Schlesinger}!1896-08-061@{6. 8. 1896}|(be}\briefempfaengerindex{Schnitzler, Arthur@\textsc{Schnitzler, Arthur}!zzzSalten, Felix@\emph{von Felix Salten}!1896-08-061@{6. 8. 1896}|(be}
\toendnotes[C]{\smallbreak\pagebreak[2]}
\correspDesc{Versand  durch Felix Salten, Therese Schlesinger, Julius Schlesinger, Leonhard Fanto, Richard Eisler, M. Laurent, Gretl Schlesinger, Franziska Schlesinger, Else Berger, Alfred Schlesinger am 6. 8. 1896 in Bad Ischl
\newline{}Übermittlung  am 7. 8. 1896 in Bad Ischl
\newline{}Erhalt  durch Arthur Schnitzler im Zeitraum [8. 8. 1896
                  – 13. 8. 1896?] in Skodsborg}\toendnotes[C]{\smallbreak}
\Standort{CUL, Schnitzler, B 89, A 1.}
\physDesc{Postkarte, 858 Zeichen
\newline{}Handschrift Felix Salten: schwarze Tinte, lateinische Kurrent
\newline{}Handschrift Margherita Schlesinger: schwarze Tinte
\newline{}Handschrift Leonhard Fanto: schwarze Tinte
\newline{}Handschrift Richard Eisler: schwarze Tinte
\newline{}Handschrift Franziska Schlesinger: schwarze Tinte
\newline{}Handschrift Alfred Schlesinger: schwarze Tinte, lateinische Kurrent
\newline{}Handschrift Else Berger: schwarze Tinte, lateinische Kurrent
\newline{}Handschrift Julius Schlesinger: schwarze Tinte
\newline{}Handschrift M. Laurent: schwarze Tinte
\newline{}Versand: Stempel: »\nobreak{}\oindex{Bad Ischl@\textbf{Bad Ischl}|pwk}Ischl, 7. 8. 96, 10–11 V\textcolor{gray}{.}\nobreak{}«.  
\newline{}Ordnung: mit Bleistift von unbekannter Hand nummeriert: »76« }\toendnotes[C]{\smallbreak}\pstart{}{\pb}Herrn D\textsuperscript{r} Arthur Schnitzler\pend{}\pstart{}Skodsborg\oindex{Skodsborg@\textbf{Skodsborg}|pw}\pend{}\pstart{}Dänemark\oindex{Dänemark@\textbf{Dänemark}|pw}\pend{}\pstart{}Badehôtel\oindex{Badehotellet@\textbf{Badehotellet}, \emph{Hotel}|pw}\pend{}{\bigskip}\vspace{1em}
\pstart
           \raggedleft{}{\pb}Ischl\oindex{Bad Ischl@\textbf{Bad Ischl}|pw}, \substVorne{}\textsuperscript{2}\substDazwischen{}6\substHinten{}. August 96.\pend
           \vspace{0.5em}
\pstart
           Man soupirt nämlich heute{ }Abend bei Schlesinger\pwindex{Schlesinger, Franziska 17.\,8.\,1851 Wien – 11.\,8.\,1932 ebd.@\textsc{Schlesinger, Franziska} (17.\,8.\,1851 Wien – 11.\,8.\,1932 ebd.)|pw}\pwindex{Schlesinger, Emil 10.\,5.\,1844 Wien – 31.\,5.\,1899 ebd.@\textsc{Schlesinger, Emil} (10.\,5.\,1844 Wien – 31.\,5.\,1899 ebd.), \emph{Bankdirektor}|pw}.
               Es war Kalbsbraten da, und über den Weg »Schnitzl« kam ein Toast auf Sie zu stande.
               Die Consequenz dieses lobenden Gefühlsausbruches ist »vorliegende« Karte, welche
               Ihnen Grüße von nachstehenden \label{K_L03177-1v}\edtext{Persönlichkeiten}{\lemma{\textnormal{\emph{Persönlichkeiten}}}\Cendnote{\textnormal{Siehe dazu auch
                     XXXX Auszeichnungsfehler: Dokument L03178 nicht gefunden.
               }}}\label{K_L03177-1} übermittelt:\pend
           
\pstart
           \spacefill\mbox{{[}hs. Schlesinger:{]} Therese Schlesinger}{ }\spacefill\mbox{{[}hs. Schlesinger:{]} Julius Schlesinger}{\\}\spacefill\mbox{{[}hs. Fanto:{]} Fanto}{ }\spacefill\mbox{{[}hs. Eisler:{]} D\textsuperscript{r} REisler}{ }\spacefill\mbox{{[}hs. Laurent:{]} M. Laurent}{\\}\spacefill\mbox{{[}hs. Schlesinger:{]} Gretl Schlesinger,}{ }\spacefill\mbox{{[}hs. Schlesinger:{]} Fanny Schlesinger}\pend
           
\pstart
           {[}hs. Berger:{]} Trotzdem Herr Salten mir absolut nicht erlauben \introOben{}will\introOben{} mehr als meinen Namen zu schreiben, benutze ich die gute
               Gelegenheit Ihnen viele herzliche Grüße zu senden. Herzlich und freundschaftlich,
               Ihre \spacefill\mbox{Else.}\pend
           
\pstart
           {[}hs. Schlesinger:{]} Med. Dr. Alfred Schlesinger \introOben{}in
                  spe\introOben{} grüßt den zukünftigen Herrn Collegen bestens nachdem er seine Matura
               glücklich überstanden\pend
           
\pstart
           \noindent{}\label{T_L03177-1v}\edtext{{[}hs. Berger:{]} Fanny fragt warum »nachstehend« nicht unter
                  Anführungszeichen steht. Bitte, erklären \uuline{Sie} ihr
                  das!! \spacefill\mbox{\textcolor{gray}{E}}}{\lemma{\textnormal{\emph{Fanny … E}}}\Cendnote{\textnormal{am linken Rand, quer zum Text}}}\label{T_L03177-1}\pend
           \selectlanguage{ngerman}\endnumbering\briefempfaengerindex{Schnitzler, Arthur@\textsc{Schnitzler, Arthur}!zzzSchlesinger, Alfred@\emph{von Alfred Schlesinger}!1896-08-061@{6. 8. 1896}|)be}\briefempfaengerindex{Schnitzler, Arthur@\textsc{Schnitzler, Arthur}!zzzBerger, Else@\emph{von Else Berger}!1896-08-061@{6. 8. 1896}|)be}\briefempfaengerindex{Schnitzler, Arthur@\textsc{Schnitzler, Arthur}!zzzSchlesinger, Franziska@\emph{von Franziska Schlesinger}!1896-08-061@{6. 8. 1896}|)be}\briefempfaengerindex{Schnitzler, Arthur@\textsc{Schnitzler, Arthur}!zzzSchlesinger, Margherita@\emph{von Margherita Schlesinger}!1896-08-061@{6. 8. 1896}|)be}\briefempfaengerindex{Schnitzler, Arthur@\textsc{Schnitzler, Arthur}!zzzLaurent, M.@\emph{von M. Laurent}!1896-08-061@{6. 8. 1896}|)be}\briefempfaengerindex{Schnitzler, Arthur@\textsc{Schnitzler, Arthur}!zzzEisler, Richard@\emph{von Richard Eisler}!1896-08-061@{6. 8. 1896}|)be}\briefempfaengerindex{Schnitzler, Arthur@\textsc{Schnitzler, Arthur}!zzzFanto, Leonhard@\emph{von Leonhard Fanto}!1896-08-061@{6. 8. 1896}|)be}\briefempfaengerindex{Schnitzler, Arthur@\textsc{Schnitzler, Arthur}!zzzSchlesinger, Julius@\emph{von Julius Schlesinger}!1896-08-061@{6. 8. 1896}|)be}\briefempfaengerindex{Schnitzler, Arthur@\textsc{Schnitzler, Arthur}!zzzSchlesinger, Therese@\emph{von Therese Schlesinger}!1896-08-061@{6. 8. 1896}|)be}\briefempfaengerindex{Schnitzler, Arthur@\textsc{Schnitzler, Arthur}!zzzSalten, Felix@\emph{von Felix Salten}!1896-08-061@{6. 8. 1896}|)be}\mylabel{L03177h}  \newcommand{\dateiname}{L03177}\newcommand{\titel}{Felix Salten u. a. an Arthur Schnitzler, 6. 8. 1896}\newcommand{\editorInnen}{Martin Anton Müller und Laura Untner}%% latex-leseansicht-abspann.tex
%% Abspann für die Leseansicht.
%% Der Schalter \ifkorrekturansicht ist bereits durch den Vorspann gesetzt.

%% latex-abspann.tex
%% Gemeinsamer Abspann für Korrekturansicht und Leseansicht.
%% Setzt den Schalter \ifkorrekturansicht voraus (gesetzt in den
%% einbindenden Dateien latex-korrekturansicht-abspann.tex bzw.
%% latex-leseansicht-abspann.tex).
%% ---------------------------------------------------------------

\normalsize

% Das esempio-Environment wird nur in der Leseansicht benötigt
\ifkorrekturansicht\else
\newenvironment{esempio}[3]%
{
    \vspace{1.5ex}
    \rlap{\underline{#1}}
    \par
    \setlength{\parindent}{0cm}
    \nopagebreak
    \leftskip=#2cm
    \rightskip=#3cm
}
{
    \par
}
\fi

\doendnotes{C}
\bigskip
\vfill

\clearpage

\footnotesize

\ifkorrekturansicht
  \lohead{\textsc{register}}
\fi

% theindex-Environment neu definieren ohne reledmac
\makeatletter
\renewenvironment{theindex}{%
  \ifkorrekturansicht
    \section*{\indexname}%
  \else
    \subsubsection*{Index der erwähnten Entitäten}%
  \fi
  \setlength{\parindent}{0pt}%
  \setlength{\parskip}{0pt plus 0.3pt}%
  \let\item\@idxitem
}{%
  \ifkorrekturansicht\clearpage\fi
}
\makeatother

\IfFileExists{\jobname-pw.ind}{\input{\jobname-pw.ind}}{}

% Quellenangabe nur in der Leseansicht
\ifkorrekturansicht\else
% Fallback-Definitionen, falls die .tex-Datei \titel etc. nicht gesetzt hat
\providecommand{\titel}{}
\providecommand{\editorInnen}{}
\providecommand{\dateiname}{\jobname}

\vspace{3cm}

\vfill

\footnotesize
\textsc{Quelle}: \titel. Herausgegeben von {\editorInnen}. In: \emph{Arthur Schnitzler: Briefwechsel mit Autorinnen und Autoren}.
 Digitale Edition, https://schnitzler-briefe.acdh.oeaw.ac.at/{\dateiname}.html (Stand \today)
\fi

\end{document}


